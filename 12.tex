\documentclass[10pt,a4paper,onecolumn,titlepage,twoside,openany]{book}
% \usepackage[utf8]{vietnam}
%\usepackage{fouriernc}
\usepackage{tasks}
%%%%%%%%%%%% KIỂU MÀU VÀ ĐỘ RỘNG NOTE
\def\kieumau{Y} %Y: Màu; N: đen-trắng
% \def\kieumau{N} %Y: Màu; N: đen-trắng
\usepackage{xcolor} 
\def\leftnote{5} %Độ rộng cột Note
%%%%%%%%%%%%% ĐN CƠ BẢN
\input{cautrucDT/color\kieumau} %MÀU
%=====================================
% Khai báo nhóm Tex (cơ bản)
%=====================================
\usepackage{amsmath,amssymb,mathrsfs,maybemath,xlop,polynom,slashbox}
\usepackage{yhmath} %\let\widering\relax %cần khi sd với font fouriernc

\usepackage{enumerate}
\usepackage{tikz} 
\usepackage{tkz-euclide}
%\usepackage{ex_tkz-euclide}
%\usetkzobj{all}
\usepackage{tikz-3dplot}
\usepackage{tkz-tab}
\usepackage{pifont} %kí hiệu đặc biệt
% \usepackage{xcolor}
%\usepackage{bbding}
%\usepackage{array}
\usepackage{tasks}
% \usepackage{casiovn}
%==========
\usetikzlibrary{math,through,calc,intersections,angles,quotes,shapes,shapes.geometric,arrows,patterns,snakes,matrix,chains,arrows.meta,decorations.shapes,decorations.fractals,decorations.markings,shadows}
\usetikzlibrary{positioning,decorations.text,decorations.pathmorphing}% Để uốn cong văn bản 
\usetikzlibrary{shadings,fadings} %ĐỔ BÓNG
\usepackage{pgfplots}
\usepackage{pgfornament}
\usepgfplotslibrary{fillbetween}
\pgfplotsset{compat=1.9}
\usepackage[hidelinks,unicode]{hyperref}
\usepackage{currfile}
\usepackage[outline]{contour} %viền
\usepackage{fontawesome} % Gói kí hiệu
\usepackage{lipsum} %Lấy text
\usepackage{tabularx}
%%---------
%\usepackage{setspace}
%\usepackage{scrextend}
\usepackage{varwidth}
%===========Bảng
\usepackage{longtable,multirow,makecell}
\usepackage{diagbox}
\renewcommand{\tabcolsep}{3mm}
\newcolumntype{C}[1]{>{\centering\arraybackslash}p{#1}}
\newcolumntype{L}[1]{>{\raggedright\arraybackslash}p{#1}}
%-----------Trang vb


%%%%%%%%%%%%% Các thông số trang tài liệu
\def\tren{1.5}\def\duoi{1.5}\def\trai{1.25}\def\phai{0.75} %cách lề
\def\topset{0.75} %kc giữa đáy header và vùng vb
\def\botset{0.75} %kc giữa đỉnh footer và vùng vb
%\usepackage{ifoddpage}
\pgfmathsetmacro{\mepphai}{\phai+\leftnote} 
%\usepackage[top=\tren cm, bottom=\duoi cm, left=\trai cm, right=\mepphai cm] {geometry}
%%%%%%%%%%%%%
%---------------------------------Các thông số trang tài liệu
\pgfmathsetmacro{\so}{\leftnote - 0.5} 
\usepackage[top=\tren cm, bottom=\duoi cm, left=\trai cm, right=\mepphai cm,
marginparwidth=\so cm, marginparsep=5mm,
%,headsep=6mm
%,footskip=10mm
] {geometry}
%-------------------------------------
\usepackage{marginnote}
\setlength{\marginparwidth}{\so cm}
\renewcommand*{\marginfont}{\small}
%--------------Gói trắc nghiệm EX-TEST
% \usepackage[dethi]{ex_test}
\usepackage[loigiai]{ex_test} 
% \usepackage[solcolor]{ex_test}
%----Lời giải, Hiền thị tên EX; Dấu kết thúc
\font\damEX=ugqb8v at 11pt
\def\loigiaiEX{\color{\mauLG}\damEX\strut\faCommenting\ Lời giải.}
%lời giải EXS
\def\loigiaiEXS{\loigiaiEX{\fontsize{8}{16}\selectfont\color{\maucham}\dotfill}}
%--
\renewcommand{\nameex}{\damEX\color{\mauEX} CÂU}
\newtheorem{EX}{\nameex} %MÔI TRƯỜNG PHỤ CHO TÁCH CÂU
\def\mauVuong{cyan}
\def\qedEX{\color{\mauVuong}\ensuremath{\square}}
%--------------Cài đặt lại dòng kẻ \dotline
\renewcommand{\dotlineEX}[1]{
	\def\numlinedot{#1}
	\par
	\foreach \dotline in{1,...,\numlinedot}
	{
		\noindent
		\fontsize{8}{16}\selectfont
		\color{\maucham}\dotfill
		\par
	}
}
% sd cho \dongcham
\newcommand{\dotlineEXS}[1]{
	\def\numlinedot{#1}
	\foreach \dotline in{1,...,\numlinedot}
	{
		\noindent
		\fontsize{8}{16}\selectfont
		\color{\maucham}\dotfill
		\par
	}
}
%---------- Khai báo viết tắt, in đáp án
\newcommand{\hoac}[1]{ %hệ hoặc
	\left[\begin{aligned}#1\end{aligned}\right.}
\newcommand{\heva}[1]{ %hệ và
	\left\{\begin{aligned}#1\end{aligned}\right.}
%--In đáp án
\newcommand{\indapan}[2]{
	\addcontentsline{toc}{subsection}{\sf Bảng đáp án} % đưa MT vào mục lục
	\begin{center}
		\begin{tikzpicture}%
			\node[thick,scale=1,fill=\mauEX!2,draw=\maufoot,minimum width=3.5cm,minimum height=0.1cm,rounded corners=2mm]{\fontfamily{qag}\fontsize{11}{11}\selectfont\bfseries\color{\mauEX} BẢNG ĐÁP ÁN};
		\end{tikzpicture}%
	\end{center}
	\inputansbox{#1}{#2}
}
%----------
\usepackage{esvect}
\def\vec{\vv} %vecto
\def\overrightarrow{\vv}
%Lệnh song song
\DeclareSymbolFont{symbolsC}{U}{txsyc}{m}{n}
\DeclareMathSymbol{\varparallel}{\mathrel}{symbolsC}{9}
\DeclareMathSymbol{\parallel}{\mathrel}{symbolsC}{9}
%--------------------------
% HEADER AND FOOTER STYLING
%--------------------------
%--------------------------
\newcommand{\myfancyhead}{% trên và chấm trái
		\boldmath
\begin{tikzpicture}[remember picture,overlay,>=stealth]
		\path ([yshift=-\tren cm+0.5*\topset cm]current page.north west) coordinate (AA)
		++(\paperwidth,0)coordinate (BB); 
\checkoddpage\ifoddpage %nếu trang lẻ
		%-----đường kẻ
		\draw[\maufoot, line width=2pt] 
		([xshift=\trai cm]AA) --([xshift=-\phai cm]BB);
		%-----bên phải
		\node[text=\maufoot, anchor=south east,inner sep=0pt] at ([xshift=-\phai cm,yshift=4pt]BB){
			\fontfamily{qag}\fontsize{8.5pt}{12pt}\selectfont 
			{\color{\mauSO}\faMapMarker}\,\, \diachi\,\,{\color{\mauSO}\faMapMarker}
		};
		%-----bên trái
		\node[text=\maufoot, anchor=south west,inner sep=0pt] at ([xshift=\trai cm,yshift=4pt]AA){
			\fontfamily{qag}\fontsize{10pt}{10pt}\selectfont\bfseries\faEdit\, \tenchuyende
		};
		%----- Kẻ đứng
		\draw[\maufoot] ([xshift=-\mepphai cm+2.5mm]BB)--([yshift=\duoi cm-0.75*\botset cm,xshift=-\mepphai cm+2.5mm]current page.south east);
		%--		
		\path ([yshift=-\tren cm+0.5*\topset cm-0.5cm,xshift=-\phai cm-0.5*\leftnote cm+2.5mm]current page.north east) coordinate (DDD); 
		\begin{scope}
			\clip ([yshift=-\tren cm+0.5*\topset cm-1pt,xshift=-\phai cm]current page.north east) rectangle ([yshift=\duoi cm-0.5*\botset cm,xshift=-\mepphai cm+5mm]current page.south east);% cắt chấm
			\node[inner sep =0pt,scale=1,anchor=north] at ([yshift=0cm,xshift=0pt]DDD) {
				\parbox{\leftnote cm}{\centering
					\def\maucham{\maufoot}\dotlineEX{60}
				}
			};
			%--note dưới
			\node[inner sep =6pt, text=white,scale=1,anchor=north,fill=\maufoot] (noteduoi) at ([yshift=2.5mm]DDD) {
				\parbox{\leftnote cm-5mm-12pt}{ \fontsize{11}{1}\fontfamily{qag}\selectfont\bfseries\centering
					QUICK NOTE
				}
			};
			\draw[\maufoot, line width=0.4pt] ([yshift=-2pt]noteduoi.south west)--([yshift=-2pt]noteduoi.south east);
		\end{scope}
\else %chẵn
		%-----đường kẻ
		\draw[\maufoot, line width=2pt] 
		([xshift=\phai cm]AA) --([xshift=-\trai cm]BB);
		%-----bên trái
		\node[text=\maufoot, anchor=south west,inner sep=0pt] at ([xshift=\phai cm,yshift=4pt]AA){
			\fontfamily{qag}\fontsize{8.5pt}{12pt}\selectfont 
			{\color{\mauSO}\faMapMarker}\,\, \diachi\,\,{\color{\mauSO}\faMapMarker}
		};
		%-----bên phải
		\node[text=\maufoot, anchor=south east,inner sep=0pt] at ([xshift=-\trai cm,yshift=4pt]BB){
			\fontfamily{qag}\fontsize{10pt}{10pt}\selectfont\bfseries\faEdit\, \tenchuyende
		};
		%----- Kẻ đứng
		\draw[\maufoot] ([xshift=\mepphai cm-2.5mm]AA)--([yshift=\duoi cm-0.75*\botset cm,xshift=\mepphai cm-2.5mm]current page.south west);
		%--		
		\path ([yshift=-\tren cm+0.5*\topset cm-0.5cm,xshift=\phai cm+0.5*\leftnote cm-2.5mm]current page.north west) coordinate (DDD); 
		\begin{scope}
			\clip ([yshift=-\tren cm+0.5*\topset cm-1pt,xshift=\phai cm]current page.north west) rectangle ([yshift=\duoi cm-0.5*\botset cm,xshift=\mepphai cm-5mm]current page.south west);% cắt chấm
			\node[inner sep =0pt,scale=1,anchor=north] at ([yshift=0cm,xshift=0pt]DDD) {
				\parbox{\leftnote cm}{\centering
					\def\maucham{\maufoot}\dotlineEX{60}
				}
			};
			%--note dưới
			\node[inner sep =6pt, text=white,scale=1,anchor=north,fill=\maufoot] (noteduoi) at ([yshift=2.5mm]DDD) {
				\parbox{\leftnote cm-5mm-12pt}{ \fontsize{11}{1}\fontfamily{qag}\selectfont\bfseries\centering
					QUICK NOTE
				}
			};
			\draw[\maufoot, line width=0.4pt] ([yshift=-2pt]noteduoi.south west)--([yshift=-2pt]noteduoi.south east);
		\end{scope}
\fi
\end{tikzpicture}%
}
% trên mục lục
\newcommand{\headmucluc}{%
	\boldmath
\begin{tikzpicture}[remember picture,overlay,>=stealth]
	\path ([yshift=-\tren cm+0.5*\topset cm]current page.north west) coordinate (AA)
	++(\paperwidth,0)coordinate (BB); 
\checkoddpage\ifoddpage %nếu trang lẻ
	%-----đường kẻ
	\draw[\maufoot, line width=2pt] 
	([xshift=\trai cm]AA) --([xshift=-\phai cm]BB);
	%-----bên phải
	\node[text=\maufoot, anchor=south east,inner sep=0pt] at ([xshift=-\phai cm,yshift=4pt]BB){
		\fontfamily{qag}\fontsize{8.5pt}{12pt}\selectfont 
		{\color{\mauSO}\faMapMarker}\,\, \diachi\,\,{\color{\mauSO}\faMapMarker}
	};
	%-----bên trái
	\node[text=\maufoot, anchor=south west,inner sep=0pt] at ([xshift=\trai cm,yshift=4pt]AA){
		\fontfamily{qag}\fontsize{12pt}{12pt}\selectfont\bfseries\faEdit\, \tenchuyende
	};
\else %chẵn
	%-----đường kẻ
	\draw[\maufoot, line width=2pt] 
	([xshift=\phai cm]AA) --([xshift=-\trai cm]BB);
	%-----bên trái
	\node[text=\maufoot, anchor=south west,inner sep=0pt] at ([xshift=\phai cm,yshift=4pt]AA){
		\fontfamily{qag}\fontsize{8.5pt}{12pt}\selectfont 
		{\color{\mauSO}\faMapMarker}\,\, \diachi\,\,{\color{\mauSO}\faMapMarker}
	};
	%-----bên phải
	\node[text=\maufoot, anchor=south east,inner sep=0pt] at ([xshift=-\trai cm,yshift=4pt]BB){
		\fontfamily{qag}\fontsize{12pt}{12pt}\selectfont\bfseries\faEdit\, \tenchuyende
	};
\fi
\end{tikzpicture}%
}
%===========================
\newcommand{\myfancyfoot}{% dưới
	\begin{tikzpicture}[remember picture,overlay]
	\path ([yshift=\duoi cm-0.75*\botset cm]current page.south west) coordinate (AA)
	++(\paperwidth,0)coordinate (BB); 
	\checkoddpage\ifoddpage %nếu trang lẻ
		%---kẻ
		\draw[\maufoot, line width=2pt] ([xshift=2*\trai cm+4pt]AA)--([xshift=-\phai cm+3pt]BB);
		%-----bên trái
		\fill[fill=\maufoot, rounded corners=2mm] ([xshift=2*\trai cm,yshift=0.25 cm]AA) rectangle +(-3*\trai cm,-0.5cm);
		%-----trang
		\node[anchor=west,text=white,inner sep=0pt,xshift=-0.75cm] at ([xshift=2*\trai cm]AA) {\fontfamily{put}\bfseries\thepage};
		%-----tên tg
		\node[anchor=west,text=\maufoot,inner sep=0pt,fill=white] at ([xshift=2*\trai cm]AA){\fontfamily{qag}\fontsize{9pt}{1pt}\selectfont\bfseries \,\,\, \tentacgia \,\,\, };
	\else %chẵn
		%---kẻ
		\draw[\maufoot, line width=2pt] ([xshift=-2*\trai cm+4pt]BB)--([xshift=\phai cm-3pt]AA);
		%-----bên trái
		\fill[fill=\maufoot, rounded corners=2mm] ([xshift=-2*\trai cm,yshift=0.25 cm]BB) rectangle +(3*\trai cm,-0.5cm);
		%-----trang
		\node[anchor=east,text=white,inner sep=0pt,xshift=0.75cm] at ([xshift=-2*\trai cm]BB) {\fontfamily{put}\bfseries\thepage};
		%-----tên tg
		\node[anchor=east,text=\maufoot,inner sep=0pt,fill=white] at ([xshift=-2*\trai cm]BB){\fontfamily{qag}\fontsize{9pt}{1pt}\selectfont\bfseries \,\,\, \tentacgia \,\,\,};
	\fi
	\end{tikzpicture}%
}
%======================Head chapter theo note, fullwidth
%----------------------
\usepackage{changepage}
\strictpagecheck
\usepackage{lastpage}
\usepackage{fancyhdr,lastpage}
\pagestyle{fancy}
\fancyhf{}
\fancypagestyle{plain}{
	\fancyhead[LO,RE]{\headmucluc}
	\fancyfoot[LO,RE]{\myfancyfoot}
}
\fancyhead[LO,RE]{\myfancyhead}
\fancyfoot[LO,RE]{\myfancyfoot}
\renewcommand{\footrulewidth}{0pt}
\renewcommand{\headrulewidth}{0pt}
%--------------4.2
\usepackage[most]{tcolorbox}
\colorlet{tcbcol@back}{tcbcolback}
\colorlet{tcbcol@frame}{tcbcolframe}
%---------------------------------------------------------------
% ĐỊNH NGHĨA SECTION. SUBSECTION, SUBSUBSECTION ... THEO Ý RIÊNG
%---------------------------------------------------------------
\usepackage[explicit]{titlesec} % để gọi #1
\usepackage{titledot} % gói lệnh chứa cả titlesec và titletoc
%=====================================
\setcounter{secnumdepth}{4} %độ sâu
\renewcommand{\thechapter}{\Roman{chapter}}
\renewcommand{\thesection}{\arabic{section}}
\renewcommand{\thesubsection}{\Alph{subsection}}
\renewcommand{\thesubsubsection}{\arabic{subsubsection}}
%--------------Tròn
\newcommand{\tron}[1]{
	\begin{tikzpicture}[baseline=(A.base)]%
		\node[circle,draw=\mauSUBSEC,line width=0.5pt,fill=white,inner sep=2pt,outer sep=1pt] (A) {\color{white} #1};
		\node[circle,draw=none,fill=\mauSUBSEC,inner sep=1pt,outer sep=1pt] (A) {\color{white} #1};
	\end{tikzpicture}%
}
%================= Đn chương
\font\fontchap=ugqb8v at 21pt
\titlespacing{\chapter}{0cm}{0cm}{0.5cm}[0cm] %1: , 2: Trên, 3: dưới
\titleformat{\chapter}[display]
{\fontsize{20pt}{20pt}\fontfamily{qag}\selectfont\bfseries\color{\mauCHUONG}} %định dạng chung
{\fontsize{16pt}{20pt}\selectfont\chaptername\, \thechapter.} %đánh số
{1mm}
{\fontchap\centering\MakeUppercase{#1}}
[\vspace{0cm}]
%============================Mục lục - Chapter*
\titleformat{name=\chapter,numberless}[display]
{\fontsize{14pt}{16pt}\fontfamily{qag}\selectfont\bfseries\color{\mauCHUONG}} %định dạng chung
{}
{-1em}
{%
	\begin{tikzpicture}
		%-----Nội dung
		\node[inner sep=0pt,right] (ndchuong) at (0,0){\fontchap \MakeUppercase{#1}};
%		%-----Đường kẻ ngang
%		\begin{scope}
%			\clip (0,-0.75) rectangle +(\textwidth,1.5);
%			\draw[\mauchuong,line width=2pt] (ndchuong.south east)++(10pt,8pt) --++(\linewidth,0);
%		\end{scope}
	\end{tikzpicture}
}
[
\vspace{-3mm}
%\thispagestyle{empty}
]
%--------Đn Section---------------------------
%\titlespacing*{\section}{0cm}{0cm}{0cm}[0cm]
\titleformat
{\section}
{\color{\mauSEC}\fontfamily{qag}\fontsize{16pt}{1pt}\selectfont\bfseries\centering}
{Bài\,\thesection.}
{3mm}
{\MakeUppercase{#1}}
[]
%-------Đn subsection---------------------------
\titlespacing{\subsection}{0cm}{0cm}{0cm}[0cm]
\titleformat{\subsection}
{\normalfont\fontsize{15pt}{20pt}\fontfamily{put}\selectfont\bfseries\color{\mausubsec}}
{\thesubsection.}
{3mm}
{\MakeUppercase{#1}}
[]
%----------ĐN subsubsection-----------------------
\titlespacing{\subsubsection}{0pt}{0mm}{0mm}[0cm]
\titleformat{\subsubsection}
{\fontsize{13pt}{18pt}\fontfamily{put}\selectfont\bfseries\color{\mausubsubsec}}
{\thesubsubsection.}
{3mm}
{#1}
[]
%----------ĐN paragraph-----------------------
\titlespacing{\paragraph}{0pt}{0mm}{0mm}[0cm]
\titleformat{\paragraph}
{\fontsize{11.5pt}{17pt}\fontfamily{put}\selectfont\bfseries\color{\mausubsubsec}}
{\theparagraph.}
{3mm}
{#1}
[]
%============================
\def\itemKN{\color{\mauitemKN}\faCheckSquareO}
\def\itemCI{\color{\mauitemCI}\faCheckCircleO}
%%======= Thiết lập labelitem, labelenumerate
%\renewcommand{\labelitemi}{\color{red}\faCheckSquareO}
\renewcommand{\labelitemi}{\color{\mauitem}\faCheckCircleO}
\renewcommand{\labelitemii}{\color{\mauitem}\bf ---}
\renewcommand{\labelitemiii}{\color{\mauitem}\bf +}
\renewcommand{\labelenumi}{\alph{enumi})}
%\renewcommand{\labelenumii}{\color{blue}\bf\arabic{enumi}.\arabic{enumii}}
%============================
%============================
% Canh chỉnh mục lục chính
\setcounter{secnumdepth}{4} %Độ sâu đánh số
\setcounter{tocdepth}{2} %Độ sâu mục lục
\contentsmargin{0cm}
%~~~~~~~~~~~~~~~~~~~~~
\renewcommand*\l@part[2]{%
	\ifnum \c@tocdepth >-2\relax
	\addpenalty{-\@highpenalty}%
	\addvspace{10pt \@plus\p@}%
	\setlength\@tempdima{3em}%
	\begingroup
	\hypersetup{linkcolor=violet}
	\tikz[remember picture, overlay]{
		\fill[\mauPHAN] (0,0) rectangle +(\textwidth,1);
		\draw (0,0.5) node[right=5pt]
		{\color{white}\fontsize{16pt}{1pt}\fontfamily{qag}\selectfont\bfseries  {\scshape Phần} #1};
	}
	\par\smallskip
	\penalty\@highpenalty
	\endgroup
	\fi
}
%------------------------
%\titlecontents{part}[0pc]
%{\addvspace{10pt}%
%	\color{red!70!black}\fontsize{18pt}{1pt}\fontfamily{qag}\selectfont\bfseries 
%}%
%{}
%{}
%{}%
%~~~~~~~~~~~~~~~~~~~~~
\titlecontents{chapter}[6.5pc] % nd cách trái
{\addvspace{5pt}%
	\color{\mauCHUONG}\fontsize{13pt}{16pt}\fontfamily{put}\selectfont\bfseries
}%
{\contentslabel[\chaptertitlename\,\thecontentslabel.]{6.5pc}} %nhãn
{}
{\hfill\bfseries\thecontentspage
}%
[\vspace*{5pt}]
%~~~~~~~~~~~~~~~~~~~~~
\titlecontents{section}[10pc]
{\addvspace{0pt}\bfseries\color{\mauSEC}}
{\fontsize{12.5pt}{15pt}\selectfont\sffamily\contentslabel[{Bài\,\thecontentslabel.}]{3.5pc}}
{}
{\hfill
	\thecontentspage
}
[]
%~~~~~~~~~~~~~~~~~~~~~
\titlecontents{subsection}[10pc]
{\addvspace{0pt}\color{\mauSUBSEC}}
{\fontsize{12pt}{15pt}\selectfont\sffamily\contentslabel[\tron{\thecontentslabel}]{2.8pc}}
{}
{{\tiny\dotfill}\thecontentspage}
[]
%~~~~~~~~~~~~~~~~~~~~~
%--------------------------
% ĐỊNH NGHĨA CÁC MÔI TRƯỜNG 
%--------------------------
\listenumerate{dn,dl,tc,nx,ex}%xuống dòng khi liệt kê
\theoremstyle{plain} %
\theoremheaderfont{\scshape} %đầu
\theorembodyfont{\normalfont} % thân
\theoremseparator {.} % Ngăn cách
\newtheorem{dn}{\color{\maudn}\faBolt\, Định nghĩa}[section]
%===================================ĐNghĩa
\theoremstyle{plain} %
\theoremheaderfont{\fontfamily{put}\bfseries} %đầu
\theorembodyfont{\normalfont} % thân
\theoremseparator {.} % Ngăn cách
\newtheorem{vd}{\color{\mauVD}\damEX
	%\faToggleOn\ 
	%\faUnlink\ 
	VÍ DỤ}%[section]
\newtheorem{bt}{\color{\mauBT}\damEX BÀI}
%===================================
\theoremstyle{plain} %
\theoremheaderfont{\scshape} %đầu
\theorembodyfont{\slshape} % thân 
\theoremseparator {.} % Ngăn cách 
\newtheorem{dl}{\color{\maudl}\faBolt\, Định lí}[section]
\newtheorem{tc}{\color{\mauhq!70!black}\faBolt\, Tính chất}[section]
\newtheorem{hq}{\color{\mauhq!70!black}\faBolt\, Hệ quả}[section]
%====================================
\theoremstyle{nonumberplain} %ko đánh số, ko xuống dòng
\theoremheaderfont{\scshape} %đầu
\theorembodyfont{\normalfont} %phần thân
\theoremseparator {.} %ngăn cách
\newtheorem{nx}{\color{\mauhq!70!black}\faBolt\, Nhận xét}
\newtheorem{tomtat}{\!\!\!\!\!\!\!\!}
%====================================Hộp
%--------------------Chú ý
\newenvironment{note}
{\begin{tcolorbox}
		[enhanced jigsaw,breakable,pad at break*=1mm,
		opacityback=0,boxrule=0pt,frame hidden,
		left=8mm, right=0pt, bottom=0pt, top=0pt,
		before skip=1mm,
		after skip=1mm,
		underlay unbroken and first={
			\draw ([xshift=0.3cm,yshift=-0.32cm]interior.north west) node[\mauly]{\large\bfseries \faExclamationTriangle};
		},
		fontupper=\it,
		]}
	{\end{tcolorbox}}
%\let\mynote\note
%\renewcommand{\note}{\mynote{\bfseries\color{\mauly}Lưu ý:}} 
%---------------Dạng toán
\newcounter{dang}\setcounter{dang}{0}
\renewcommand{\thedang}{\arabic{dang}}
%---Dạng 1
\newtcolorbox{dang}[1]{
	fonttitle=\fontfamily{qag}\bfseries,%fontupper=\itshape,
	colframe=\maudang,colback=yellow!20,coltitle=white,
	sharp corners, breakable, halign title=center,%adjusted title=center, %canh giữa DẠNG
	before skip=2mm,after skip=3mm,
	left=2mm,right=2mm,top=2mm,bottom=2mm,
	boxrule=1pt,
	title={\faFolderOpen\ Dạng~\stepcounter{dang}\thedang.\ #1}
		\addcontentsline{toc}{subsection}{\it\sffamily \faFolderOpen\ Dạng~\thedang.#1}
		\setcounter{subsubsection}{0}
		\setcounter{vd}{0}
		\setcounter{ex}{0}
		\setcounter{bt}{0}
}
%====================
\setlength{\parindent}{0pt} %không thụt đầu dòng
%--Name
\newcounter{deso}
\font\dam=ugqb8v at 13pt
\font\damTT=ugqb8v at 18pt
%================đn notenam
\def\notename{
		\begin{tikzpicture}[remember picture,overlay,>=stealth]
		\checkoddpage\ifoddpage %nếu trang lẻ
		%--tiêu đề phải
		\path ([yshift=-\tren cm+0.5*\topset cm-0.5cm,xshift=-\phai cm-0.5*\leftnote cm+2.5mm]current page.north east) coordinate (DD); 
		%--
		\fill[white] ([yshift=-\tren cm+0.5*\topset cm-5pt,xshift=\trai cm-2pt]current page.north east) rectangle ([yshift=\duoi cm-0.5*\botset cm-12cm,xshift=-\mepphai cm+3mm]current page.north east);
		\node[inner sep =0pt,anchor=north] (thanhta) at ([yshift=-2mm]DD) {
			\includegraphics[width=4.5cm]{logo/logo.jpg}		
		};	
		\else
		%--tiêu đề phải
		\path ([yshift=-\tren cm+0.5*\topset cm-0.5cm,xshift=\phai cm+0.5*\leftnote cm-2.5mm]current page.north west) coordinate (DD); 
		%--
		\fill[white] ([yshift=-\tren cm+0.5*\topset cm-5pt,xshift=\phai cm-2pt]current page.north west) rectangle ([yshift=\duoi cm-0.5*\botset cm-12cm,xshift=\mepphai cm-3mm]current page.north west);
		\node[inner sep =0pt,anchor=north] (thanhta) at ([yshift=-2mm]DD) {
			\includegraphics[width=4.5cm]{logo/logo.jpg}		
		};
		\fi
		%\draw (thanhta.south) node[below=0pt,xscale=0.8]{\small\normalfont\color{\mauname} Sưu tầm \& Biên tập};
		%---note
		\node[inner sep =6pt, text=black,scale=1,anchor=north,fill=\maufoot!3,draw=\maufoot] (bon) at ([yshift=-1cm]thanhta.south) {
			\parbox{\leftnote cm-5mm-12pt}{ \fontsize{10}{15}\selectfont\normalfont
				\vspace*{2pt}
				\chamngon%
			}
		};
		\draw[\maufoot, line width=5pt] (bon.north west)--(bon.north east);
		\draw[\maufoot!50] ([yshift=9pt,line width=0.4pt]bon.north east)--([yshift=9pt]bon.north west)
		node[fill=white,inner sep=2pt,anchor=south west,yshift=-2pt,xshift=-2pt]{\bfseries\color{\maufoot}ĐIỂM:}
		;
		%--note dưới
		\node[inner sep =6pt, text=white,scale=1,anchor=north,fill=\maufoot] (noteduoi) at ([yshift=-0.25cm]bon.south) {
			\parbox{\leftnote cm-5mm-12pt}{ \fontsize{11}{1}\selectfont\bfseries\centering
				QUICK NOTE
			}
		};
		\draw[\maufoot, line width=0.4pt] ([yshift=-2pt]noteduoi.south west)--([yshift=-2pt]noteduoi.south east);
	\end{tikzpicture}
}
%===================note và nonote
%FULL WIDTH
\def\FULLWIDTH{
	\newpage
	\fancyhead[LO,RE]{\headmucluc}
	\def\notename{}
	\newgeometry{top=\tren cm, bottom=\duoi cm, left=\trai cm, right=\phai cm}
}
\def\NOTE{
	\newpage
	\fancyhead[LO,RE]{\myfancyhead}
	\def\notename{
		\begin{tikzpicture}[remember picture,overlay,>=stealth]
			\checkoddpage\ifoddpage %nếu trang lẻ
			%--tiêu đề phải
			\path ([yshift=-\tren cm+0.5*\topset cm-0.5cm,xshift=-\phai cm-0.5*\leftnote cm+2.5mm]current page.north east) coordinate (DD); 
			%--
			\fill[white] ([yshift=-\tren cm+0.5*\topset cm-5pt,xshift=\trai cm-2pt]current page.north east) rectangle ([yshift=\duoi cm-0.5*\botset cm-12cm,xshift=-\mepphai cm+3mm]current page.north east);
			\node[inner sep =0pt,anchor=north] (thanhta) at ([yshift=-2mm]DD) {
				\includegraphics[width=4.5cm]{logo/logo.jpg}		
			};	
			\else
			%--tiêu đề phải
			\path ([yshift=-\tren cm+0.5*\topset cm-0.5cm,xshift=\phai cm+0.5*\leftnote cm-2.5mm]current page.north west) coordinate (DD); 
			%--
			\fill[white] ([yshift=-\tren cm+0.5*\topset cm-5pt,xshift=\phai cm-2pt]current page.north west) rectangle ([yshift=\duoi cm-0.5*\botset cm-12cm,xshift=\mepphai cm-3mm]current page.north west);
			\node[inner sep =0pt,anchor=north] (thanhta) at ([yshift=-2mm]DD) {
				\includegraphics[width=4.5cm]{logo/logo.jpg}		
			};
			\fi
			%\draw (thanhta.south) node[below=0pt,xscale=0.8]{\small\normalfont\color{\mauname} Sưu tầm \& Biên tập};
			%---note
			\node[inner sep =6pt, text=black,scale=1,anchor=north,fill=\maufoot!3,draw=\maufoot] (bon) at ([yshift=-1cm]thanhta.south) {
				\parbox{\leftnote cm-5mm-12pt}{ \fontsize{10}{15}\selectfont\normalfont
					\vspace*{2pt}
					\chamngon%
				}
			};
			\draw[\maufoot, line width=5pt] (bon.north west)--(bon.north east);
			\draw[\maufoot!50] ([yshift=9pt,line width=0.4pt]bon.north east)--([yshift=9pt]bon.north west)
			node[fill=white,inner sep=2pt,anchor=south west,yshift=-2pt,xshift=-2pt]{\bfseries\color{\maufoot}ĐIỂM:}
			;
			%--note dưới
			\node[inner sep =6pt, text=white,scale=1,anchor=north,fill=\maufoot] (noteduoi) at ([yshift=-0.25cm]bon.south) {
				\parbox{\leftnote cm-5mm-12pt}{ \fontsize{11}{1}\selectfont\bfseries\centering
					QUICK NOTE
				}
			};
			\draw[\maufoot, line width=0.4pt] ([yshift=-2pt]noteduoi.south west)--([yshift=-2pt]noteduoi.south east);
		\end{tikzpicture}
	}
	\newgeometry{top=\tren cm, bottom=\duoi cm, left=\trai cm, right=\mepphai cm}
}
%===================đn name
\newcommand{\name}[4]{
%	\NOTE
%	\newpage
	\setcounter{ex}{0}\setcounter{bt}{0}%\setcounter{EX}{0}
	\boldmath\fontfamily{qag}\selectfont\color{\mauname}
\hoten \dotfill {\fontsize{10}{11}\selectfont \ngaylamde}
	\begin{tcolorbox}[boxrule=0.7pt,arc=0mm,breakable,colframe=\mauSO,colback=\mauname!2,before skip=2mm,after skip=2mm]\color{\mauname}
	\begin{center}
		%%---
		{\damTT \MakeUppercase{#1}}\\[1pt]
		{\dam \MakeUppercase{#2 --- Đề} \stepcounter{deso}\thedeso}\\[1pt]
		% {\dam \MakeUppercase{#2}}\\[1pt]
		{\dam\color{\mauSO} \MakeUppercase{#3}}\\[1pt]
		{\fontsize{10}{10}\selectfont \textit{#4}}%\\[-1mm]
	\end{center}
	\end{tcolorbox}
	%%--- Phần note đầu đề
	\notename
\vspace*{0.5cm}
	\addcontentsline{toc}{section}{\hspace*{-4.2cm}\sf Đề \thedeso: #2 --- #3} % đưa MT vào mục lục
}
%--Sang trang 
\BeforeBeginEnvironment{name}{
	\ifnum\the\value{deso}>0
	\newpage
	\fi
}
%%---Đánh số trang
%\AtEndEnvironment{name}{
%	\ifnum\the\value{deso}=1
%	\pagenumbering{arabic}%đánh số trang dạng 1,2,...
%	\fi
%}
%---------
\def\chap#1{
	\begin{center}
		\fontchap\color{\mauCHUONG} #1
	\end{center}
	\addcontentsline{toc}{chapter}{\hspace*{-2.75cm}#1}
}
%---Hiện bảng ĐA
\newcommand{\hienDA}{
	\renewcommand{\indapan}[2]{
		\addcontentsline{toc}{subsection}{\hspace*{-4.2cm}\sf Bảng đáp án} % đưa MT vào mục lục
		%		\begin{center}
		\par\vspace*{5mm}
		\begin{tikzpicture}%
			\draw (0,0)++(0.5*\textwidth,0) node[thick,scale=1,fill=\mauEX!2,draw=\maufoot,minimum width=3.5cm,minimum height=0.1cm,rounded corners=2mm] {\damEX\color{\mauname} BẢNG ĐÁP ÁN};
		\end{tikzpicture}%
		%		\end{center}
		\vspace*{-2mm}
		\inputansbox{##1}{##2}
	}
}
%---Ẩn bảng ĐA
\newcommand{\anDA}{
	\renewcommand{\indapan}[2]{}
}
%---Dòng chấm từng câu
\newcommand{\dongchamEX}[1]{
%	\hideansEX{ex}
	\anLG
	\AfterEndEnvironment{ex}{%
		\foreach \cauEX/\dongEX in {#1}{
			\ifnum\dongEX=0
			\else
			\ifnum\the\value{ex}=\cauEX
			\par\noindent\loigiaiEXS\par
			\dotlineEX{\dongEX}
			\fi
			\fi
		}
	}
}
%---Dòng chấm nhiều câu
\newcommand{\dongchamEXS}[2]{
%	\hideansEX{ex}
	\anLG
	\AfterEndEnvironment{ex}{%
		\foreach \cauEX in {#1}{
			\ifnum#2=0
			\else
			\ifnum\the\value{ex}=\cauEX
			\par\noindent\loigiaiEXS\par
			\dotlineEXS{#2}
			\fi
			\fi
		}
	}
}
%---Dòng chấm từng câu theo đề
\newcommand{\DEdongchamEX}[2]{
%	\hideansEX{ex}
	\anLG
	\AfterEndEnvironment{ex}{%
		\foreach \cauEX/\dongEX in {#2}{
			\ifnum\dongEX=0
			\else
			\ifnum\the\value{deso}=#1
			\ifnum\the\value{ex}=\cauEX
			\par\noindent\loigiaiEXS\par
			\dotlineEX{\dongEX}
			\fi
			\fi
			\fi
		}
	}
}
%---Dòng chấm nhiều câu theo đề
\newcommand{\DEdongchamEXS}[3]{
%	\hideansEX{ex}
	\anLG
	\AfterEndEnvironment{ex}{%
		\foreach \cauEX in {#2}{
			\ifnum#3=0
			\else
			\ifnum\the\value{deso}=#1
			\ifnum\the\value{ex}=\cauEX
			\par\noindent\loigiaiEXS\par
			\dotlineEX{#3}
			\fi
			\fi
			\fi
		}
	}
}
%---Ẩn LG
\newcommand{\anLG}{
	\renewcommand{\loigiai}[1]{	}%
	% \chooseNSA
	\renewcommand{\TrueTF}{\FalseTF}
	\renewcommand{\TrueEX}{\FalseEX}
	\renewcommand{\writekeyTFone}{\gdef\TrueX{}\gdef\FalseX{}}
	\renewcommand{\writekeyTF}{&&}
}
%---Hiện LG
\newcommand{\hienLG}{
	%Xuất hiện chữ Lời giải trong môi trường onlysolution
	\renewcommand{\loigiai}[1]{%
		\begin{onlysolution}%
			##1
		\end{onlysolution}%
	}%
	%---
	\def\loigiaiEXS{}
	%\choiceTF
	\renewcommand{\writekeyTFone}{\gdef\TrueX{}\gdef\FalseX{\tickF}}
	\renewcommand{\writekeyTF}{%
		&\centering\leavevmode\TrueX%
		&\parbox[t]{\linewidth}{\centering\leavevmode\FalseX}%
			\gdef\TrueX{}\gdef\FalseX{\tickF}%
	}
	\def\kindSA{ShowSAKeyColor}
	\showanswers
	% \SAOPTN{kindSA=oly}
	\renewcommand{\dotlineEXS}[1]{}	
}
%=======Đn các phương án
\def\khoanhtrondapan{
	\renewcommand*\circled[1]{\tikz[baseline=(char.base)]{
			\node[shape=circle,draw=\mauDA,inner sep=1pt] (char) {##1};}}
	\renewcommand{\TrueEX}{\stepcounter{dapan}
		{\squareEX{\textbf{\damEX\color{\mauDA}\Alph{dapan}}}} \ignorespaces}
	\renewcommand{\FalseEX}{\stepcounter{dapan}
		{\circled{\textbf{\damEX\color{\mauDA}\Alph{dapan}}}} \ignorespaces}
	%---Chọn đáp án
	\renewcommand{\circEX}[2][fill=\mauTrue!3,draw=\mauTrue]{%
	\tikz[baseline=(char.base)]{\node[shape=circle,inner sep=1pt,##1] (char) {\color{red}##2};}}
}
%----
\def\khongkhoanhtrondapan{
	\renewcommand{\TrueEX}{\stepcounter{dapan}
		{\squareEX{\textbf{\damEX\color{\mauDA}\Alph{dapan}}}} \ignorespaces}
	\renewcommand{\FalseEX}{\stepcounter{dapan}
		{\textbf{\damEX\color{\mauDA}\Alph{dapan}.}} \ignorespaces}
	%---Chọn đáp án
	\renewcommand{\circEX}[2][fill=\mauTrue!3,draw=\mauTrue]{%
	\tikz[baseline=(char.base)]{\node[shape=circle,inner sep=1pt,##1] (char) {\color{red}##2};}}
}
%=============ĐN HIỆN CÂU EX CẦN THIẾT if
\newcommand{\hienEXS}[2]{
	%\foreach \bdem in {#1,...,#2}{%11-30
	\def\biendau{#1}\def\biencuoi{#2}%
	\pgfmathsetmacro{\sodau}{\fpeval{round(\biendau-1,0)}}
	\pgfmathsetmacro{\socuoi}{\fpeval{round(\biencuoi+1,0)}}
	\setcounter{EX}{#1-1}
	\RenewEnviron{ex}{
		\stepcounter{ex}%
		\ifnum\value{ex}<\socuoi
		\ifnum\value{ex}>\sodau
		\par%
		\begin{EX}
			\BODY% 
		\end{EX}
		\fi\fi
	}
	%
	\AtEndEnvironment{name}{\setcounter{EX}{#1-1}}
	%
	\AtEndEnvironment{EX}{
		\ifnum\the\value{numTrue}=1
		\scantokens{\begin{EXsol}A\end{EXsol}}
		\fi
		\ifnum\the\value{numTrue}=2
		\scantokens{\begin{EXsol}B\end{EXsol}}
		\fi
		\ifnum\the\value{numTrue}=3
		\scantokens{\begin{EXsol}C\end{EXsol}}
		\fi
		\ifnum\the\value{numTrue}=4
		\scantokens{\begin{EXsol}D\end{EXsol}}
		\fi
		\setcounter{numTrue}{0}
	}
}
%%%%%%%%%%%%%%%%%%%%%%%

%============================ Khung
\newenvironment{khung}
{\begin{tcolorbox}[
		enhanced,breakable,
		colback=yellow!10,
		colframe=blue,
		boxrule=0.5pt,
		%		drop fuzzy shadow=gray,
		left=5pt,right=5pt,top=5pt,bottom=5pt,
		arc=0mm
		]}
	{\end{tcolorbox}}
%-----------------------------Mục con = subsub
\newcounter{muccon}
\newcommand{\muccon}[1]{%
	\stepcounter{muccon}
	{%\setcounter{bt}{0}\setcounter{vd}{0}\setcounter{ex}{0}
		%\fontsize{13pt}{15pt}\selectfont
		%		\color{violet!70!black}\sffamily
		\bfseries\sffamily\bfseries\hspace*{0mm}\themuccon.\  
		#1}
}
%----------------------------------------------------

% Hộp định nghĩa
\newenvironment{boxdn}
{\begin{tcolorbox}
		[enhanced jigsaw,breakable,pad at break*=1mm,
		colback=cyan!2,
		%standard jigsaw, 
		opacityback=0, %ko nền
		boxrule=0pt,frame hidden, left=0.7cm, right=0pt, bottom=2pt, top=0pt,
		borderline west={1mm}{0.5cm}{cyan},
		overlay={
			\fill[fill=cyan!20,draw=none] ([xshift=0.6cm]interior.north west) rectangle (interior.south east)
			;
		}
		\setcounter{muccon}{0}
		]}%0mm lề trái
	{\end{tcolorbox}}
%===============================================
\theoremstyle{nonumberbreak} % ko đánh số
\theoremheaderfont{\sffamily\bfseries} %tên
\theorembodyfont{\normalfont} %thân
\theoremsymbol{\ensuremath{_\blacksquare}} %Dấu kết thúc là ô vuông đen.
\theoremseparator {:} % Dấu ngăn cách
\newtheorem{myphantich}{\color{violet}%\faServer\ 
	\faFileText\ PHÂN TÍCH}
%===============================================
\newenvironment{phantich}{\begin{boxdn}\begin{myphantich}}{\end{myphantich}\end{boxdn}}
%-------------- Khung (Trong main này ko sd)
\newtcolorbox[auto counter]{khung4}[1]{enhanced, breakable,
	before skip=1mm,after skip=1mm,
	left=1mm,right=1mm,top=2mm,bottom=1mm,
	colframe=myblue,colback=cyan!0,colbacktitle=cyan!6,coltitle=myblue,colupper=black,sharp corners,
	,boxrule=0.4mm,
	coltext=mauE,
	attach boxed title to top center=
	{yshift=-0.1mm-\tcboxedtitleheight/2,yshifttext=2mm-\tcboxedtitleheight/2},
	varwidth boxed title*=-3cm,
	boxed title style={boxrule=0.3mm,
		frame code={ \path[tcb fill frame] ([xshift=-4mm]frame.west)
			-- (frame.north west) -- (frame.north east) -- ([xshift=4mm]frame.east)
			-- (frame.south east) -- (frame.south west) -- cycle; },
		interior code={ \path[tcb fill interior] ([xshift=-2mm]interior.west)
			-- (interior.north west) -- (interior.north east)
			-- ([xshift=2mm]interior.east) -- (interior.south east) -- (interior.south west)
			-- cycle;} 
	},
	fonttitle=\fontsize{10}{0}
	\bfseries,
	fontupper=\fontsize{10}{0},
	title={#1}
}

\newcommand{\boxmini}[1]{
	\vspace*{-2mm}
	\begin{center}
		\begin{tikzpicture}[outline/.style={draw=##1,thick,fill=##1!3},outline/.default=myblue]
			\node [outline,
			sharp corners] at (0,0) {\fontfamily{qag} \selectfont\bfseries\color{\mauEX} #1};
		\end{tikzpicture}
	\end{center}
	\vspace*{0mm}
}

%-----------------------
\newcommand{\inden}[1]{
	{\fontsize{11pt}{9pt}\sffamily \selectfont\bfseries\color{\maudn} #1}
}
\newcommand{\indam}[1]{
	{\fontsize{11.5pt}{9pt}\sffamily \selectfont\bfseries\color{\maudn} #1}
}
\newcommand{\indamm}[1]{
	{\fontsize{11.5pt}{9pt}\sffamily \selectfont\bfseries\color{\maudl} #1}
}
\newcommand{\ind}[1]{
	{\fontsize{11.5pt}{9pt}\sffamily \selectfont\bfseries\color{\maucham} #1}
}
%%%===Các biểu tượng===
\def\iconGN{{\color{magenta}\faPencilSquareO}}
\def\iconNS{{\color{gray}\faStar}}
\def\iconQS{{\color{magenta}\faFolderOpen}}
\def\iconMT{{\color{magenta!80!black}\faSunO}}
\def\iconX{{\color{red}\faClose}}
\def\iconCH{{\color{myblue}\faCheckCircle}}
\def\iconVD{\faCubes}
\def\iconCV{{\color{myblue}\faCubes}}

\newcommand{\dongcham}[1]{
	\def\sod{#1}
	\pgfmathsetmacro{\sodong}{2*\sod -1} 
	\columnsep=10pt
	\vspace*{-3.5mm}
	\begin{multicols}{2}
		\foreach \dotline in{1,...,\sodong}
		{\noindent\color{gray}{\dotfill}\\[1mm]
		}\noindent\color{gray}{\dotfill}\\[-4mm]
	\end{multicols}
}


\def\TNTF{
    {\bfseries Phần II. Trong mỗi ý a), b), c) và d) ở mỗi câu, học sinh chọn đúng hoặc sai.}
}
\def\TN{
    {\bfseries Phần I. Mỗi câu hỏi học sinh chọn một trong bốn phương án A, B, C, D.}
}
\def\TNSA{
    {\bfseries Phần III. Học sinh điền kết quả vào ô trống.}
}
\def\BTTL{
    \begin{center}
        \fcolorbox{black}{white}{{\bfseries BÀI TẬP TỰ LUẬN TRẢ LỜI NGẮN}}
    \end{center}
}
\def\BTTF{
    \begin{center}
        \fcolorbox{black}{white}{{\bfseries BÀI TẬP TRẮC NGHIỆM ĐÚNG SAI}}
    \end{center}
%    \TNTF
}
\def\BTTN{
    \begin{center}
        \fcolorbox{black}{white}{{\bfseries BÀI TẬP TRẮC NGHIỆM 4 PHƯƠNG ÁN}}
    \end{center}
}
\def\TL{
    {\bfseries Phần II. Câu hỏi tự luận.}
}
 %Khai báo cơ bản
\usepackage{tkz-euclide,circuitikz,casio580x}
%%%%%%%%%%%%% ĐIỀU KHIỂN LỜI GIẢI ,DÒNG CHẤM, ĐÁP SỐ
%------------Dòng chấm bằng chiều dài LG (bật)
% \dotlinefull{ex}
% \dotlinefull{vd}
%------------Thay Loi giải bằng n dòng kẻ (bật)
% \dotlineans{2}{ex}
%------------Ẩn lời giải
%\hideansEX{ex}
%------------Dòng chấm tùy ý (ko cần \loigiai{})
%---Nhiều câu cùng dòng chấm (tách dụng lên mọi đề)
% \dongchamEXS{13,...,19}{3}
% \dongchamEXS{20,21,22}{8}
%---Nhiều câu cùng dòng chấm (tách dụng lên MỘT ĐỀ đc chọn)
%\DEdongchamEXS{3}{1,...,20}{2} %{3} là đề thứ 3
%---Dòng chấm từng câu, tác dụng lên mọi đề
%\dongchamEX{1/3,2/5,3/7} % câu / số dòng chấm của câu đó
%---Dòng chấm từng câu, tác dụng lên 1 đê
%\DEdongchamEX{3}{1/3,2/5,3/7} % câu / số dòng chấm của câu đó, {3} là đề số 3 
\renewcommand{\dongcham}[1]{}
\renewcommand{\cham}[1]{}
%------------Ẩn đáp số (bật), đáp án
\exitdapso %ẩn đs
%\renewcommand{\indapan}[2]{} %ẩn đáp án
%%%%%%%%%%%%% khung NAME
\def\hoten{Gọi tôi là:}
\def\ngaylamde{Ngày làm đề: ...../...../........} %để {} nếu ko muốn
\def\tenchude{GTLN-GTNN CỦA HÀM SỐ}
\def\tendethi{ÔN TẬP THPTQG}
\def\tentruong{LỚP TOÁN THẦY PHÁT}
\def\thoigian{Thời gian làm bài: 90 phút}
%%%%%%%%%%%%% Nội dung head & foot
% \def\diachi{ }
\def\diachi{VNPmath - 0962940819}
\def\tenchuyende{\tenchude}
\def\tentacgia{GV.VŨ NGỌC PHÁT}
\def\chamngon{\lq\lq  It's not how much time you have, it's how you use it.\rq\rq
}
%%%%%%%%%%%%% Đn lại A.B.C.D
\khoanhtrondapan
% \khongkhoanhtrondapan
%%%%%%%%%%%%%
\renewcommand{\arraystretch}{1}
%TF Option
\OPTN{kindTF=t,dapanTF=a,boldTF=1,phatbieu=Mệnh đề,viettat=1}
% \OPTN{kindSA=oly,ketquaSA=KQ:,widthSA=4,heightSA=0.9,dapanSA=a}
% 
%%===================================================
%=================BẮT ĐẦU TÀI LIỆU===================
\begin{document}
\renewcommand{\chaptername}{Chương}
\pagenumbering{arabic}%đánh số trang dạng 1,2,...
%====================================================
%==================BẮT ĐẦU TÀI LIỆU==================
%\hienEXS{41}{50} %chỉ hiện câu từ 41 đến 50 của đề
%--------Đề bài
% \setcounter{deso}{10}
% \FULLWIDTH \anLG \anDA
\NOTE \anLG \anDA
% --
% \notename
% Chương I. Hàm số
%%Bài 1. Đơn điệu, Cực trị
% \section{TÍNH ĐƠN ĐIỆU VÀ CỰC TRỊ CỦA HÀM SỐ}
\subsection{LÝ THUYẾT CẦN NHỚ}
\subsubsection{Tính đơn điệu của hàm số}
\begin{enumerate}[\iconMT]
	\item \indam{Định nghĩa:} Cho hàm số $y=f(x)$ xác định trên $K$ ($K$ là khoảng, đoạn hoặc nửa khoảng). \\
\begin{minipage}[b]{6cm}
\begin{khung4}{Ghi nhớ 1}
	Hàm số đồng biến trên $K$ nếu
	$\forall x_1,\,x_2 \in K$, $$ x_1<x_2 \Rightarrow f(x_1)<f(x_2)$$
	\centerline{\begin{tikzpicture}[>=stealth,scale=0.6]
		\draw[->] (-1,0)--(0,0)%
		node[below left]{$O$}--(5,0) node[below]{$x$};
		\draw[->] (0,-1) --(0,3) node[right]{$y$};
		\draw [black,thick, domain=0.2:4, samples=100] %
		plot (\x, {0.1*(\x)^2+1});
		\draw [dashed] (1,0)node[below]{\footnotesize$x_1$} --(1,1.1)--(0,1.1)node[left]{\footnotesize$f(x_1)$};
		\draw [dashed] (3,0)node[below]{\footnotesize$x_2$} --(3,1.9)--(0,1.9)node[left]{\footnotesize$f(x_2)$};
		\draw[fill=blue] (1,1.1) circle(2pt);
		\draw[fill=blue] (3,1.9) circle(2pt);
	\end{tikzpicture}}\\
Trên $K$, đồ thị là một "\textbf{đường đi lên}" khi xét từ trái sang phải.
\end{khung4}
\end{minipage}\hspace{0.5cm}
\begin{minipage}[b]{6cm}
\begin{khung4}{Ghi nhớ 2}
		Hàm số nghịch biến trên $K$ nếu
		$\forall x_1,\,x_2 \in K$, $$ x_1<x_2 \Rightarrow f(x_1)>f(x_2)$$
		\centerline{\begin{tikzpicture}[>=stealth,scale=0.6]
			\draw[->] (-1,0)--(0,0)%
			node[below left]{$O$}--(5,0) node[below]{$x$};
			\draw[->] (0,-1) --(0,3) node[right]{$y$};
			\draw [thick, domain=0.2:4, samples=100] %
			plot (\x, {-0.1*(\x)^2+2.5});
			\draw [dashed] (1,0)node[below]{\footnotesize$x_1$} --(1,2.4)--(0,2.4)node[left]{\footnotesize$f(x_1)$};
			\draw [dashed] (3,0)node[below]{\footnotesize$x_2$} --(3,1.6)--(0,1.6)node[left]{\footnotesize$f(x_2)$};
			\draw[fill=blue] (1,2.4) circle(2pt);
			\draw[fill=blue] (3,1.6) circle(2pt);
		\end{tikzpicture}}\\
	Trên $K$, đồ thị là một "\textbf{đường đi xuống}" khi xét từ trái sang phải.
\end{khung4}
\end{minipage}
	\item \indam{Liên hệ giữa đạo hàm và tính đơn điệu:}
	Cho hàm số $y=f(x)$ có đạo hàm trên khoảng $(a;b)$.
	\begin{boxdn}
	\begin{listEX}[1]
		\item [$\bullet$] Nếu $y'\ge 0$, $\forall x \in (a;b)$ và dấu bằng chỉ xảy ra tại hữu hạn điểm thì hàm số $y=f(x)$ đồng biến trên $(a;b)$.
		\item [$\bullet$] Nếu $y'\le 0$, $\forall x \in (a;b)$ và dấu bằng chỉ xảy ra tại hữu hạn điểm thì hàm số  $y=f(x)$ nghịch biến trên $(a;b)$.
	\end{listEX}
	\end{boxdn}
\end{enumerate}
\subsubsection{Cực trị của hàm số}
\begin{enumerate}[\iconMT]
	\item \indam{Định nghĩa:} Cho hàm số $y=f(x)$ xác định và liên tục trên khoảng $(a ; b)$ ( $a$ có thể là $-\infty, b$ có thể là $+\infty)$ và điểm $x_0 \in(a ; b)$.
	\begin{boxdn}
	\begin{itemize}
		\item [$\bullet$] Nếu tồn tại số $h>0$ sao cho $f(x)<f\left(x_0\right)$ với mọi $x \in\left(x_0-h ; x_0+h\right) \subset(a ; b)$ và $x \neq x_0$ thì ta nói hàm số $f(x)$ đạt cực đại tại $x_0$.
		\item [$\bullet$] Nếu tồn tại số $h>0$ sao cho $f(x)>f\left(x_0\right)$ với mọi $x \in\left(x_0-h ; x_0+h\right) \subset(a ; b)$ và $x \neq x_0$ thì ta nói hàm số $f(x)$ đạt cực tiểu tại $x_0$.
	\end{itemize}
	\end{boxdn}
	\item \indam{Định lý:} Giả sử hàm số $y=f(x)$ liên tục trên khoảng $(a ; b)$ chứa điểm $x_0$ và có đạo hàm trên các khoảng $\left(a ; x_0\right)$ và $\left(x_0 ; b\right)$. Khi đó:
	\begin{boxdn}
	\begin{itemize}
		\item [$\bullet$] Nếu $f^{\prime}(x)<0$ với mọi $x \in\left(a ; x_0\right)$ và $f^{\prime}(x)>0$ với mọi $x \in\left(x_0 ; b\right)$ thì $x_0$ là một điểm cực tiểu của hàm số $f(x)$.
		\item [$\bullet$] Nếu $f^{\prime}(x)>0$ với mọi $x \in\left(a ; x_0\right)$ và $f^{\prime}(x)<0$ với mọi $x \in\left(x_0 ; b\right)$ thì $x_0$ là một điểm cực đại của hàm số $f(x)$.
	\end{itemize}
	\end{boxdn}
	\item \indam{Các tên gọi:}\\
		\begin{tikzpicture}[smooth,samples=300,scale=1.15,>=stealth]
			\draw[->,>=stealth] (-2.5,0)--(2.7,0) node[below]{$x$};
			\draw[->,>=stealth] (0,-1.5)--(0,4) node[right]{$y$};
			\draw (0,0) node[above left]{$O$};
			\draw[blue,domain=-2:2,line width = 1.2pt] plot(\x,{(\x)^3-3*(\x)+1})node[right]{$y=f(x)$};
			\draw[fill=black] (1,0) circle(1pt) (1,-1) circle(2pt) (0,-1) circle(1pt) (-1,0) circle(1pt) (-1,3) circle(2pt) (0,3) circle(1pt);
			\draw[dashed] (1,0)node[above]{\small$x_2$}--(1,-1)--(0,-1)node[left]{\small$y_2$} (-1,0)node[below]{\small$x_1$}--(-1,3)--(0,3)node[right]{\small$y_1$};
			
			\draw[-,dotted] (-0.5,3.7)--(4,3.7)node[right]{$(x_1;y_1)$ là điểm cực đại của đồ thị hàm số;}; 
			\draw[->,dotted] (-0.5,3.7)--(-1,3.15);
			\node[right] at (4.5,3.1) {$\bullet$ $x_1$ là điểm cực đại của hàm số;};
			\node[right] at (4.5,2.5) {$\bullet$ $y_1$ là giá trị cực đại của hàm số.};
			
			\draw[-,dotted] (2,-1)--(2,1)--(4,1)node[right]{$(x_2;y_2)$ là điểm cực tiểu của đồ thị hàm số;}; \draw[->,dotted] (2,-1)--(1.15,-1);
			\node[right] at (4.5,0.4) {$\bullet$ $x_2$ là điểm cực tiểu của hàm số;};
			\node[right] at (4.5,-0.2) {$\bullet$ $y_2$ là giá trị cực tiểu của hàm số.};
		\end{tikzpicture}
\end{enumerate}
\subsection{PHÂN LOẠI VÀ PHƯƠNG PHÁP GIẢI TOÁN}
\begin{dang}{Bài toán tìm khoảng đơn điệu và cực trị của hàm số cho trước}
	\begin{listEX}[1]
		\item [\ding{172}] Tìm tập xác định $\mathscr{D}$ của hàm số $y=f(x)$ .
		\item [\ding{173}] Tính đạo hàm $f'(x)$. Tìm các điểm $x_i \,(i = 1, 2, ..., n)$ thuộc $\mathscr{D}$ mà tại đó đạo hàm bằng $0$ hoặc không xác định.
		\item [\ding{174}] Sắp xếp các điểm $x_i$ theo thứ tự tăng dần, xét dấu $y'$ và lập bảng biến thiên. Từ đây, nêu các khoảng đồng biến, nghịch biến và các điểm cực trị.
	\end{listEX}
\end{dang}
\indamm{Ghi nhớ cách xét dấu:}
\begin{note}
\begin{enumerate}[\iconCH]
		% \item Nếu $$f'(x)=(x-a)(x-b)^2(x-c)^{2n}(x-d)^{2n+1},\,\forall n \in \mathbb{N}*$$
		% thì phương trình $f'(x)=0$ có
		% \begin{itemize}
		% 	\item 	$x=a$ là nghiệm đơn;
		% 	\item  $x=b$ là nghiệm kép;
		% 	\item  $x=c$ là nghiệm bội chẵn;
		% 	\item  $x=d$ là nghiệm bội lẻ.
		% \end{itemize}
		\item Khi xét dấu $f'(x)$ thì $f'(x)$ sẽ không đổi dấu khi qua nghiệm kép (nghiệm bội chẵn) và đổi dấu khi qua nghiệm đơn (nghiệm bội lẻ).
	\end{enumerate}
	% \begin{tikzpicture}[smooth,samples=300,scale=0.8,>=stealth,font=\footnotesize]
	% 	\draw[->] (-3.5,0)--(6,0) node[below]{$x$};
	% 	\draw[->] (0,-1.5)--(0,4) node[left]{$y$};
	% 	\draw (0,0) node[above left]{$O$};
	% 	\draw[blue,line width=0.7pt,domain=-2.15:1.5] plot(\x,{(\x+2)*(\x-1)^2});
	% 	\draw[blue,line width=0.7pt,domain=1.5:4.7] plot(\x,{-1*(\x-1.64)*(\x-4)^2})node[below]{$y=f'(x)$};
	% 	\draw[fill=red] (-2,0)node[above left]{$x_1$} circle(1.5pt) (1,0)node[below]{$x_2$} circle(1.5pt) (4,0)node[above right]{$x_4$} circle(1.5pt) (1.64,0)node[above right]{$x_3$} circle(1.5pt);
	% 	\draw[dashed,<-] (-1.8,-0.2)--(0.5,-2.3)node[below]{\fbox{\scriptsize\text{Nghiệm bội lẻ}}};
	% 	\draw[dashed,->](0.5,-2.3)--(1.58,-0.2);
	% 	\draw[dashed,<-] (1,0.2)--(2,3)node[above]{\fbox{\scriptsize\text{Nghiệm bội chẵn}}};
	% 	\draw[dashed,->](2,3)--(3.9,0.1);
	% 	\end{tikzpicture}
\end{note}
\boxmini{BÀI TẬP TỰ LUẬN}

\begin{vd}
	Tìm các khoảng đơn điệu và các điểm cực trị của hàm số sau
	\begin{tasks}(3)
		\task $ y=-x^3+3x^2-4$;
		\task $ y=x^3-3x^2+1$;
		\task $y=x^3+3x^2+3x+2$;
		\task $y=-2x^4+4x^2$;
		\task $y=x^4+4x^3-1$;
		\task $y=-16x^4+x-1$.
	\end{tasks}
	\loigiai{
	\begin{enumEX}[a)]{1}
		\item Tập xác định: $\mathscr{D}=\mathbb{R}$. \\
		Đạo hàm: $y'=-3x^2+6x$.\\
		Xét $y'=0 \Leftrightarrow -3x^2+6x=0 \Leftrightarrow
		\hoac{
			& x=0 \\
			& x=2 }$
		Bảng biến thiên:\begin{center}
			\begin{tikzpicture}
				\tkzTabInit[nocadre=false,lgt=0.7,espcl=2.1,deltacl=0.6]
				{$x$ /0.6,$y'$ /0.6,$y$ /2}
				{$-\infty$,$0$,$2$,$+\infty$}
				\tkzTabLine{,-,$0$,+,$0$,-,}
				\tkzTabVar{+/$+\infty$, -/$-4$,+/$0$,-/$-\infty$}
			\end{tikzpicture}
		\end{center}
		\item Ta có: $ y'=3x^2-6x\Rightarrow y'=0\Leftrightarrow \hoac{&x=0\\&x=2.} $\\
		Từ bảng biến thiên suy ra hàm số đồng biến trên khoảng $ (-\infty;0) $ và $ (2;+\infty). $
		\begin{center}
			\begin{tikzpicture}
				\tkzTabInit[nocadre=false,lgt=1,espcl=3]
				{$x$ /1,$y'$ /1,$y$ /2}
				{$-\infty$,$0$, $2$,$+\infty$}
				\tkzTabLine{,+,$0$,-,$0$,+, }
				\tkzTabVar{-/ $-\infty$,+/$1 $ ,-/$-3$,+/$+\infty$}
			\end{tikzpicture}
		\end{center}
		\item Hàm số đã cho xác định trên $\mathscr{D}=\mathbb{R}$.\\
		Ta có $y'=3x^2+6x+3$. Cho $y'=0 \Leftrightarrow 3x^2+6x+3=0 \Leftrightarrow x=-1$.\\
		Bảng biến thiên
		\begin{center}
			\begin{tikzpicture}
				\tkzTabInit[lgt=1,espcl=3]
				{$x$/0.7,$y'$/0.7,$y$/2}
				{$-\infty$,$-1$,$+\infty$}
				\tkzTabLine{,+,0,+,}
				\tkzTabVar{-/$-\infty$,R/,+/$+\infty$}
				
			\end{tikzpicture}
		\end{center}
		Vậy hàm số đồng biến trên $\mathbb{R}$.
		\item Tập xác định của hàm số là $ \mathscr{D}=\mathbb{R}$.\\
		Ta có $y'=-8x^3+8x$.
		Cho $y'=0 \Leftrightarrow -8x^3+8x=0 \Leftrightarrow 8x(-x^2+1)=0$\\
		\centerline{$ \Leftrightarrow \left[\begin{aligned}
				&8x=0 \\
				&-x^2+1=0
			\end{aligned}\right. \Leftrightarrow \left[\begin{aligned}
				&x=0 \\
				&x^2=1
			\end{aligned}\right. \Leftrightarrow \left[\begin{aligned}
				&x=0 \\
				&x=\pm 1.
			\end{aligned}\right. $}
		Bảng biến thiên
		\begin{center}
			\begin{tikzpicture}
				\tkzTabInit[lgt=1,espcl=3]
				{$x$/0.7,$y'$/0.7,$y$/2}
				{$-\infty$,$-1$,$0$,$1$,$+\infty$}
				\tkzTabLine{,+,0,-,0,+,0,-,}
				\tkzTabVar{-/$-\infty$,+/ $2$/,-/$0$,+/$2$,-/$-\infty$}
			\end{tikzpicture}
		\end{center}
		Vậy hàm số đồng biến trên mỗi khoảng $(-\infty;-1)$ và $(0;1)$,\\
		\indent{ } hàm số nghịch biến trên mỗi khoảng $(-1;0)$ và $(1;+\infty)$.
		\item Hàm số đã cho xác định trên $\mathscr{D}=\mathbb{R}$.\\
		Ta có $y'=4x^3+12x^2=0=4x^2(x+3)$.\\
		Cho $y'=0 \Leftrightarrow 4x^2(x+3)=0 \Leftrightarrow \left[\begin{aligned}
			&x=0 \\
			&x=-3.
		\end{aligned}\right.$\\
		Bảng biến thiên
		\begin{center}
			\begin{tikzpicture}
				\tkzTabInit[lgt=1,espcl=3]
				{$x$/0.7,$y'$/0.7,$y$/2}
				{$-\infty$,$-3$,$0$,$+\infty$}
				\tkzTabLine{,-,0,+,0,+,}
				\tkzTabVar{+/$+\infty$,-/$-28$ /,R,+/$+\infty$}
			\end{tikzpicture}
		\end{center}
		Vậy hàm số nghịch biến trên khoảng $(-\infty;-3)$ và đồng biến trên khoảng $(-3;+\infty)$.
		\item Ta có $y'=-64x^3+1<0\Leftrightarrow x>\dfrac{1}{4}$ nên hàm số nghịch biến trên khoảng $\left(\dfrac{1}{4};+\infty\right)$.
\end{enumEX}}
\end{vd}

\begin{vd}
	Tìm các khoảng đơn điệu và cực trị của các hàm số sau:
	\begin{tasks}(3)
		\task $y=\dfrac{2x+1}{x+1}$;
		\task $y=\dfrac{3x+1}{x-1}$;
		\task $y=\dfrac{x^2+2x+2}{x+1}$;
		\task $y=x+\dfrac{4}{x}$;
		\task $y=\sqrt{x^2-2x}$;
		\task $y=x-3\sqrt[3]{x^2}$ .
	\end{tasks}
	\loigiai{
		\begin{enumEX}[a)]{1}
			\item Ta có $y'=\dfrac{1}{(x+1)^2} > 0, \forall x \in \mathbb{R} \backslash \{-1\}$.\\
			Vậy hàm số đồng biến trên $(-\infty ;-1)$ và $(-1 ;+\infty)$.\\
			Hàm số không có cực trị.
			\item Ta có $y'=\dfrac{-4}{(x-1)^2}>0,\,\forall x\in\mathbb{R}\setminus\{1\}$.\\
			Do vậy hàm số nghịch biến trên các khoảng  $(-\infty;1)$; $(1;+\infty)$.\\
			Hàm số không có cực trị.
			\item \begin{itemize}
				\item TXĐ: $\mathscr{D}=\mathbb{R}\setminus \left\{-1\right\}$.
				\item $y'=\dfrac{x^2+2x}{(x+1)^2}$, $y'=0\Leftrightarrow \hoac{& x=-2 \\ & x=0.}$\\
				Ta có bảng biến thiên
				\begin{center}
					\begin{center}
						\begin{tikzpicture}
							\tkzTabInit[nocadre=True,lgt=1,espcl=2]
							{$x$ /0.7,$y'$ /0.7,$y$ /2}
							{$-\infty$,$-2$,$-1$,$0$,$+\infty$}
							\tkzTabLine{,+,$0$,-,d,-,$0$,+,}
							\tkzTabVar{-/$-\infty$,+/$-2$,-D+/$-\infty$/$+\infty$,-/$2$,+/$+\infty$}
						\end{tikzpicture}
					\end{center}
				\end{center}
				Hàm số đồng biến trên khoảng $\left( -\infty;-2\right)$ và $\left( 0;+\infty\right)$;  nghịch biến trên $(-2;-1)$ và $(-1;0)$.\\
				Hàm số đạt cực đại tại $x=-2$, giá trị cực đại $y=-2$\\
				Hàm số đạt cực tiểu tại $x=0$, giá trị cực tiểu $y=2$.\\
			\end{itemize}
			\item Tập xác định $\mathscr{D}=\mathbb{R}\setminus\{0\}$.\\
			Ta có $y'=1-\dfrac{4}{x^2}=\dfrac{x^2-4}{x^2}$, $y'=0\Leftrightarrow x=\pm 2$.\\
			Bảng biến thiên
			\begin{center}
				\begin{tikzpicture}[yscale=.8,xscale=1.5,]
					\begin{scope}[shift={(-.5,.5)}]
						\draw
						(0,0) rectangle +(8,-5)
						(0,-1)--+(0:8) (0,-2)--+(0:8) (1,0)--+(-90:5);
					\end{scope}
					\path
					(0,0) node{$x$}          % <<< dòng 1
					++(0:1) node{$-\infty$}
					++(0:2) node{$-2$}
					++(0:1) node{$0$}
					++(0:1) node{$2$}
					++(0:2) node{$+\infty$}
					(0,-1)   node{$y'$}         % <<< dòng 2
					++(0:2) node{$+$}
					++(0:1) node{$0$}
					++(0:.5) node{$-$}
					++(0:1) node{$-$}
					++(0:.5) node{$0$}
					++(0:1) node{$+$}
					(0,-3)   node{$y$}       % <<< dòng 3
					++(0:1) ++(-90:1)  node (A) {$-\infty$}
					++(0:2) ++(90:2) node (B) {$-4$}
					++(0:1) ++(-90:2) node (C)[left]
					{$-\infty$}
					++(90:2) node (D)[right]{$+\infty$}
					++(0:1) ++(-90:2) node (E) {$4$}
					++(0:2) ++(90:2) node (F) {$+\infty$};
					\draw[-stealth] (A)--(B);
					\draw[-stealth] (B)--(C);
					\draw[-stealth] (D)--(E);
					\draw[-stealth] (E)--(F);
					\draw[double] (4,-.5)--(4,-4.5);
				\end{tikzpicture}
			\end{center}
			Hàm số đồng biến trên khoảng $\left( -\infty;-4\right)$ và $\left( 2;+\infty\right)$; nghịch biến trên các khoảng $(-2;0)$ và $(0;2)$.\\
			Hàm số đạt cực đại tại $x=-2$, giá trị cực đại $y=-4$\\
			Hàm số đạt cực tiểu tại $x=2$, giá trị cực tiểu $y=4$\\
			\item Tập xác định: $\mathscr{D}=(-\infty;0]\cup [2;+\infty)$.\\
			Ta có $y'=\dfrac{x-1}{\sqrt{x^2-2x}},\forall x\in (-\infty;0)\cup (2;+\infty)$.\\
			$y'=0 \Leftrightarrow \dfrac{x-1}{\sqrt{x^2-2x}}=0 \Rightarrow x-1=0 \Leftrightarrow x=1 \notin \mathscr{D}$.\\
			Bảng biến thiên:
			\begin{center}
				\begin{tikzpicture}
					\tkzTabInit[lgt=1,espcl=3]
					{$x$/0.7,$y'$/0.7,$y$/2}
					{$-\infty$,$0$,$2$,$+\infty$}
					\tkzTabLine{,-,d,h,d,+,}
					\tkzTabVar{+/$+\infty$,-H/$0$/,-/$0$,+/$+\infty$}
				\end{tikzpicture}
			\end{center}
			Vậy hàm số nghịch biến trên khoảng $(-\infty;0)$ và đồng biến trên khoảng $(2;+\infty)$.\\
			Hàm số không có cực trị.
			\item Tập xác định: $\mathscr{D}=\mathbb{R}$.\\
			Đạo hàm $y'=1-\dfrac{2}{\sqrt[3]{x}}$, xác định với mọi $x\neq 0$.\\
			$y'=0\Leftrightarrow \sqrt[3]{x}=2\Leftrightarrow x=8$.\\
			Đạo hàm không xác định tại $x=0$.\\
			Bảng biến thiên
			\begin{center}
				\begin{tikzpicture}
					\tkzTabInit[nocadre,lgt=1,espcl=2]{$x$/0.7,$y'$/0.7,$y$/2}{$-\infty$,$0$,$8$,$+\infty$}%
					\tkzTabLine{,+,d,-,z,+,}
					\tkzTabVar{-/$-\infty$ , +/$0$,-/$-4$, +/$+\infty$}%
				\end{tikzpicture}
			\end{center}
	\end{enumEX}}
\end{vd}

\begin{vd}
	Thể tích $V$ (đơn vị: centimét khối) của $1 \mathrm{~kg}$ nước tại nhiệt độ $T\,\left(0^{\circ} \mathrm{C} \leq T \leq 30^{\circ} \mathrm{C}\right)$ được tính bởi công thức	$$	V(T)=999,87-0,06426 T+0,0085043 T^2-0,0000679 T^3$$
	 Hỏi thể tích $V(T), \,0^{\circ} \mathrm{C} \leq T \leq 30^{\circ} \mathrm{C}$, giảm trong khoảng nhiệt độ nào?
	\loigiai{
		Xét hàm số  $V(T)=999{,}87-0{,}06426T+0{,}0085043T^2-0{,}0000679T^3$, với $T\in [0;30]$.\\
	Ta có $V'(T)=-0{,}0002037T^2+0{,}0170086T-0{,}06426$.\\
	$V'(T)=0\Leftrightarrow T=3{,}966514624=T_1$ hoặc $T=79{,}53176716\not\in [0;30]$.\\
	Bảng biến thiên của hàm số $V(T)$ như sau
	\begin{center}
		\begin{tikzpicture}[font=\footnotesize,thick,>=stealth]
			\tikzset{double style/.append style={double distance=1.5pt}}
			\tkzTabInit[nocadre=false,lgt=1.2,espcl=2.5,deltacl=0.6,lw=.75pt,color,colorL=green!50,colorV=green!50]
			{$T$ /0.7, $V'(T)$ /0.8, $V(T)$ /2}
			{$0$,$T_1$,$30$}
			\tkzTabLine{ ,-,$0$,+, }
			\tkzTabVar{+/$V(0)$,-/$V(T_1)$,+/$V(30)$}
		\end{tikzpicture}
	\end{center}
	Từ bảng biến thiên suy ra, thể tích $V(T), 0^{\circ}\mathrm{C}\leq T \leq 30^{\circ}\mathrm{C}$, giảm trong khoảng nhiệt độ từ $0^\circ$C đến $3{,}966514624^\circ$C.}
\end{vd}

\boxmini{BÀI TẬP TRẮC NGHIỆM}
\ind{PHẦN I.} \inden{Câu trắc nghiệm nhiều phương án lựa chọn. Mỗi câu hỏi học sinh chỉ chọn một phương án.}\\
\setcounter{ex}{0}
\Opensolutionfile{ans}[ans/2D1-B1-d1-1]

\begin{ex}%[KSCL, Sở GD \& ĐT Hà Nam, 2018]%[Lê Quốc Hiệp, dự án 12EX10-18]%[2D1B1-2]%
	\immini
	{Cho hàm số $y=f(x)$ có đồ thị như hình vẽ bên. Hàm số $y=f(x)$ nghịch biến trên khoảng nào dưới đây?
		\haicot
		{$(\sqrt{2};+\infty)$}
		{$(-2;2)$}
		{$(-\infty;0)$}
		{\True $(0;\sqrt{2})$}
	}
	{\begin{tikzpicture}[line cap=round,line join=round,x=1.0cm,y=1.0cm,>=stealth,scale=0.7]
			\draw[->,color=black,smooth,samples=100] (-2.5,0.) -- (2.5,0.) node[below] {\footnotesize $x$};
			\draw[->,color=black,smooth,samples=100] (0.,-2.5) -- (0.,3) node[left] {\footnotesize $y$};
			\draw plot[smooth,tension=.7] coordinates {(-2,3) (-1.41,-2)  (0,2) (1.41,-2) (2,3)};
			\draw[fill=black] (0,0) circle [radius=1pt] node[above left] {\footnotesize $O$};
			\fill (-1.41,0) node[shift={(90:2ex)}]{\footnotesize $-\sqrt{2}$} circle(1pt);
			\fill (1.41,0) node[shift={(90:2ex)}]{\footnotesize $\sqrt{2}$} circle(1pt);
			\fill (0,-2) node[shift={(-45:1.5ex)}]{\footnotesize $-2$} circle(1pt);
			\fill (0,2) node[shift={(45:1.5ex)}]{\footnotesize $2$} circle(1pt);
			\draw[dashed] (-1.41,0)|-(0,-2)-|(1.41,0);
	\end{tikzpicture}}
	\loigiai
	{
		Dựa vào đồ thị, ta thấy trên khoảng $(0;\sqrt{2})$ đồ thị đi xuống nên hàm số $y=f(x)$ nghịch biến trên khoảng đó.
	}
\end{ex}

\begin{ex}
	\immini{Cho hàm số $y=f(x)$ có đồ thị như hình vẽ bên. Mệnh đề nào sau đây là mệnh đề \textbf{sai}?
		\choice
		{Hàm số đạt cực đại tại $x=0$}
		{Hàm số có giá trị cực tiểu bằng $-2$}
		{\True Hàm số đồng biến trên $(-\infty; 2)$}
		{Hàm số nghịch biến trên $(0; 2)$}
	}
	{
		
		\begin{tikzpicture}[smooth,samples=300,scale=0.7,>=stealth]
			\draw[->] (-2,0)--(3.7,0) node[below]{$x$};
			\draw[->] (0,-2.5)--(0,2.5) node[right]{$y$};
			\draw (0,0) node[below right]{$O$};
			\draw[smooth,samples=100,domain=-1:3]
			plot(\x,{(\x)^3-3*(\x)^2+2});
			\draw[fill=black] (2,0) circle(1.5pt) (0,2) circle(1.5pt) (0,-2) circle(1.5pt);
			\draw[dashed] (2,0)node[above]{\small$2$}--(2,-2)--(0,-2)node[left]{\small$-2$} (0,2.1)node[left]{\small$2$};
		\end{tikzpicture}
	}
	
	\loigiai{
	}
	
\end{ex}

\begin{ex}
	\immini{
		Hàm số $y=f(x)$ có đồ thị là đường cong trong hình vẽ bên. Hàm số $y=f(x)$ đạt cực tiểu tại điểm nào dưới đây?
		\haicot
		{$x=2$}
		{\True $x=0$}
		{$x=-2$}
		{$x=4$}
	}{
		\begin{tikzpicture}[xscale=.7,yscale=.6, font=\footnotesize, line join=round, line cap=round, >=stealth]
			\draw[->] (-2.7,0)--(3,0) node[below]{$x$};
			\draw[->] (0,-1.25)--(0,5) node[left]{$y$};
			\draw[dashed] (-2^.5,0)--(-2^.5,4)--(2^.5,4)--(2^.5,0);
			\draw[domain=-2.05:2.05] plot(\x,{-(\x)^2*((\x)^2-4)});
			\path
			(0,0) node[below right]{$O$}
			(2,0) node[above right]{$2$}
			(-2,0) node[above left]{$-2$}
			(0,4) node[below right]{$4$}
			(-2^.5,0) node[below]{$-\sqrt{2}$}
			(2^.5,0) node[below]{$\sqrt{2}$};
		\end{tikzpicture}
	}
	\loigiai{
		Dựa vào đồ thị hàm số ta thấy hàm số đạt cực tiểu tại $x=0$.}
\end{ex}

\begin{ex}
	\immini{Cho hàm số $y=f(x)$ có bảng biến thiên như hình bên. Mệnh đề nào sau đây là mệnh đề đúng?
	\choice
	{Hàm số đồng biến trên khoảng $(-\infty;3)$}
	{Hàm số nghịch biến trên khoảng $(-2;+\infty)$}
	{Hàm số đạt cực đại tại $x=3$}
	{\True Hàm số đạt cực tiểu tại $x=2$}}{
	\begin{tikzpicture}
	\tkzTabInit[lgt=1.2,espcl=1.8,nocadre=True]
	{$x$/0.6,$f'(x)$/0.6,$f(x)$/2}{$-\infty$,$-2$,$2$,$+\infty$}
	\tkzTabLine{,+,0,-,0,+,}
	\tkzTabVar{-/$-\infty$,+/$3$,-/$0$,+/$+\infty$}
\end{tikzpicture}}
	\loigiai{
	}
	
\end{ex}

\begin{ex}
	Cho hàm số $y=f(x)$ có bảng biến thiên bên dưới
	\begin{center}
		\begin{tikzpicture}
			\tikzset{double style/.append style = {draw=\tkzTabDefaultWritingColor,double=\tkzTabDefaultBackgroundColor,double distance=2pt}}
			\tkzTabInit[lgt=1.2,espcl=2,nocadre=True]
			{$x$ /.7, $f’(x)$ /.7,$f(x)$ /2}
			{$-\infty$ , $-2$, $0$ ,$2$ ,$+\infty$}
			\tkzTabLine{ ,+,$0$,-,d,-,$0$,+, }
			\tkzTabVar{ -/ $-\infty$,+/ $-4$/,-D+/$- \infty$ /$+\infty$,-/ $4$,+ /$+\infty$}
		\end{tikzpicture}
	\end{center}
Khẳng định nào sau đây là khẳng định \textbf{sai}?
	\choice
	{Hàm số có hai điểm cực trị}
	{Tọa độ điểm cực đại của đồ thị hàm số là $(-2;-4)$}
	{\True Hàm số nghịch biến trên khoảng $(-2;2)$}
	{Hàm số đồng biến trên khoảng $(3;+\infty)$}
	\loigiai{
	}
\end{ex}


\begin{ex}
	Cho hàm số $y= - \dfrac{1}{3} x^3 - x -3 $. Mệnh đề nào dưới đây đúng?
	\choice
	{Hàm số đồng biến trên $(-\infty; 1)$ và trên $(1; +\infty)$}
	{\True Hàm số nghịch biến trên $\mathbb{R}$}
	{Hàm số đồng biến trên $(-1;1)$}
	{Hàm số đồng biến trên $\mathbb{R}$}
	\loigiai{
		Tập xác định $\mathscr D = \mathbb{R}$.\\
		$y'=-x^2 -1<0 $ với mọi $x$.\\
		Suy ra  hàm số đã cho nghịch biến trên $\mathbb{R}$.}
\end{ex} 


\begin{ex}
	Gọi $x_1$ là điểm cực đại $x_2$ là điểm cực tiểu của hàm số $y=-x^3+3x+2$. Tính $x_1+2x_2$.
	\choice{$2$}
	{$1$}
	{\True $-1$}
	{$0$}
	\loigiai{
		Ta có $y'=-3x^2+3$, $y'=0\Leftrightarrow x=\pm 1$.\\
		Vì $y'$ đổi dấu từ âm sang dương khi qua $x=-1$ và đổi dấu từ dương sang âm khi qua $x=1$ nên $x_2=-1$ là điểm cực tiểu và $x_1=1$ là điểm cực đại của hàm số. Do đó $x_1+2x_2=1-2=-1$.
	}
\end{ex} 


\begin{ex}
	Khoảng cách giữa hai điểm cực trị của đồ thị hàm số $y=x^3-3x^2+4$ bằng
	\choice
	{\True $2\sqrt{5}$}
	{$2\sqrt{2}$}
	{$2$}
	{$ 4 $}
	\loigiai{
		Ta có $y'=3x^2-6x$, $ y'=0\Rightarrow \hoac{&x=0\Rightarrow y=4\\&x=2\Rightarrow y=0.} $\\
		Suy ra hai điểm cực trị của đồ thị hàm số là $A(0;4),B(2;0)$.\\
		Do đó $AB=\sqrt{2^2+(-4)^2}=2\sqrt{5}$.
	}
\end{ex} 

\begin{ex}%[2D1B1]
	Hàm số $y=x^4-2x^2+1$ đồng biến trên khoảng nào dưới đây?
	\choice
	{\True $(-1;0)$}
	{$(-1;+ \infty)$}
	{$(-3;8)$}
	{$(- \infty ; -1)$}
	\loigiai
	{
		$y'= 4x^3-4x$ $\Rightarrow y'=0 \Leftrightarrow 4x^3-4x=0$ $\Leftrightarrow \hoac{x&= -1 \\ x&=0 \\ x &= 1}$\\
		Bảng xét dấu
		\begin{center}
			\begin{tikzpicture}
				\tkzTabInit[nocadre=false, lgt=1, espcl=2.5]{$x$ /1,$y$ /1}{$-\infty$,$-1$,$0$,$1$,$+\infty$}
				\tkzTabLine{,-,$0$,+,$0$,-,$0$,+}
			\end{tikzpicture}
		\end{center}
	}
	
\end{ex} 

\begin{ex}%[2HK1-13-ChuyenLeQuyDon-QuangTri]%[2D1B2-1]%
	Cho hàm số $ y = - \dfrac{1}{4}x^4 + \dfrac{1}{2}x^2 - 3 $. Khẳng định nào sau đây là khẳng định đúng?
	\choice
	{Hàm số đạt cực tiểu tại $ x = -3 $}
	{ \True Hàm số đạt cực tiểu tại $ x = 0 $}
	{Hàm số đạt cực đại tại $ x = 0 $}
	{Hàm số đạt cực tiểu tại $ x = -1 $}
	\loigiai{
		Ta có $ y' = - x^3 + x = - x (x^2 - 1) $.
		Ta có bảng biến thiên như hình bên
		\begin{center}
			\begin{tikzpicture}[scale=1]
				\tkzTabInit[lgt=1.5,espcl=2.5]{$x$  /1,$y'$  /1,$y$ /2}
				{$-\infty$,$ -1 $,$ 0 $,$ 1 $,$+\infty$}%
				\tkzTabLine{,+,z,-,z,+,z,-,}
				\tkzTabVar{-/$ -\infty $,+/   $\dfrac{-11}{4}$ /,-/ $-3$,+/$ \dfrac{-11}{4} $,-/$ -\infty $}
				%\tkzTabIma{1}{3}{2}{$ 0 $}
			\end{tikzpicture}
		\end{center}
	}
\end{ex} 

\begin{ex}
	Cho hàm số $y=\dfrac{3x-1}{x-2}$. Mệnh đề nào dưới đây là đúng?
	\choice
	{Hàm số nghịch biến trên $\mathbb{R}$}
	{Hàm số đồng biến trên các khoảng $(-\infty;2)$ và $(2;+\infty)$}
	{\True Hàm số nghịch biến trên các khoảng $(-\infty;2)$ và $(2;+\infty)$}
	{Hàm số đồng biến trên $\mathbb{R}\setminus\{2\}$}
	\loigiai{Tập xác định là $\mathscr{D}=\mathbb{R}\setminus\{2\}$.\\
		Có $y'=\dfrac{-5}{(x-2)^2}<0$, $\forall x\in\mathscr{D}$ nên hàm số nghịch biến trên các khoảng $(-\infty;2)$ và $(2;+\infty)$.}
	
\end{ex} 

\begin{ex}
	Cho hàm số $y=\dfrac{x-2}{x+3}$. Mệnh đề nào dưới đây đúng?
	\choice
	{Hàm số nghịch biến trên khoảng $(-\infty;-3)\cup (-3;+\infty) $}
	{\True Hàm số đồng biến trên khoảng $(-\infty;-3) $ và $(-3;+\infty)$}
	{Hàm số nghịch biến trên khoảng $(-\infty;-3)$ và $(-3;+\infty)$}
	{Hàm số đồng biến trên khoảng $(-\infty;-3)\cup (-3;+\infty) $}
	\loigiai{
		Tập xác định $\mathscr{D}=\mathbb{R}\setminus \{-3\}$. Ta có $y'=\dfrac{5}{(x+3)^2}>0$, $\forall x\in\mathscr{D}$.\\ Suy ra hàm số đồng biến trên khoảng $(-\infty;-3)$ và $(-3;+\infty)$.
	}
\end{ex} 

\begin{ex}
	Gọi $y_{\text{CĐ}},\,y_{\text{CT}}$ lần lượt là giá trị cực đại và giá trị cực tiểu của hàm số $y=\dfrac{x^2+3x+3}{x+2}$. Giá trị của biểu thức $y_{\text{CĐ}}^2-2y_{\text{CT}}^2$ bằng
	\choice
	{$8$}
	{\True $7$}
	{$9$}
	{$6$}
	\loigiai{
		Ta có $y'=\dfrac{x^2+4x+3}{(x+2)^2}$; $y'=0 \Leftrightarrow \left[\begin{aligned}
			&x=-1 \\
			&x=-3
		\end{aligned}\right. $. \\
		Bảng biến thiên
		\begin{center}
			\begin{tikzpicture}
				\tkzTab
				[lgt=1,espcl=2] % tùy chọn
				{$x$/0.7, $y'$/0.7, $y$/2} % cột đầu tiên
				{$-\infty$, $-3$, $-2$, $-1$, $+\infty$} % hàng 1 cột 2
				{,+,0,-,d,-,0,+,} % hàng 2 cột 2
				{-/ $-\infty$, +/ $-3$, -D+/ $-\infty$ / $+\infty$, -/ $1$, +/ $+\infty$} % hàng 3 cột 2
			\end{tikzpicture}
		\end{center}
		Từ bảng biến thiên ta tìm được $y_{\text{CĐ}}=-3;\,y_{\text{CT}}=1$ $ \Rightarrow $ $y_{\text{CĐ}}^2-2y_{\text{CT}}^2$ $=9-2=7$.}
\end{ex} 

\begin{ex}
	Tìm điểm cực tiểu của hàm số $f(x)=(x-3)\mathrm{e}^x$.
	\choice
	{$x=3$}
	{$x=0$}
	{\True $x=2$}
	{$x=1$}
	\loigiai{
		\begin{itemize}
			\item Ta có $f'(x)=\mathrm{e}^x(x-2)$, $f''(x)=\mathrm{e}^x(x-1)$.
			\item $f'(x)=0\Rightarrow x=2$ và $f''(2)=\mathrm{e}^2>0$.
		\end{itemize}
		Vậy hàm số đã cho đạt cực tiểu tại $x=2$.}
\end{ex} 

\begin{ex}
	Cho hàm số $y=x^2+4\ln(3-x)$. Tìm giá trị cực đai $y_\text{CĐ}$ của hàm số đã cho.
	\choice
	{$y_\text{CĐ}=2$}
	{\True $y_\text{CĐ}=4$}
	{$y_\text{CĐ}=1+4\ln2$}
	{$y_\text{CĐ}=1$}
	\loigiai{
		Tập xác định $\mathscr{D}=(-\infty;3)$.\\
		Đạo hàm $y'=2x-\dfrac{4}{3-x}=\dfrac{-2x^2+6x-4}{3-x}$.\\
		$y'=0\Leftrightarrow -2x^2+6x-4=0\Leftrightarrow \hoac{&x=1\\&x=2}$.\\
		Bảng biến thiên
		\begin{center}
			\begin{tikzpicture}[>=stealth]
				\tkzTabInit[nocadre=false,lgt=1,espcl=2,deltacl=0.5]{$x$/.7,$y'$/.7,$y$/2}
				{$-\infty$,$1$,$2$,$3$}
				\tkzTabLine{,-,0,+,0,-,d}
				\tkzTabVar{+/$+\infty$,-/$1+4\ln 2$,+/$4$,-D/$-\infty$}
			\end{tikzpicture}
		\end{center}
		Hàm số đạt cực đại tại $x=2$, $y_\text{CĐ}=4$.
	}
\end{ex} 


\begin{ex}%[2D1K2]
	Cho hàm số $y = f(x)$ xác định trên $\mathbb{R}$ và có đạo hàm $y' = f'(x) = 3x^3 - 3x^2$. Mệnh đề nào sau đây \textbf{sai}?
	\choice
	{Trên khoảng $(1;+\infty)$ hàm số đồng biến}
	{Trên khoảng $(-1;1)$ hàm số nghịch biến}
	{\True Đồ thị hàm số có hai điểm cực trị}
	{Đồ thị hàm số có một điểm cực tiểu}
	\loigiai
	{
		Ta có: $y' = 0 \Leftrightarrow 3x^3 - 3x^2 = 0 \Leftrightarrow \hoac{& x = 0 \\& x = 1.}$\\
		Bảng biến thiên:
		\begin{center}
			\begin{tikzpicture}[>=stealth]
				\tkzTabInit[nocadre, lgt=1, espcl=2.5]
				{$x$ /0.7,$y'$ /0.7,$y$ /1.7}
				{$-\infty$,$0$,$1$,$+\infty$}
				\tkzTabLine{,-,$0$,-,$0$,+,}
				\tkzTabVar{+/ $+\infty$, R, -/{\text{CT}}, +/ $+\infty$}
			\end{tikzpicture}
		\end{center}
		Hàm số đồng biến trên khoảng $(1;+\infty)$.\\
		Hàm số nghịch biến trên khoảng $(-\infty;1)$.\\
		Hàm số đạt cực tiểu tại $x = 1$.
	}
\end{ex} 

\begin{ex}%[2D1B2]
	Cho hàm số $ y=f(x) $ liên tục trên $ \mathbb{R} $ và có đạo hàm $ f'(x)=x(x-1)^2(x-2)^3 $. Số điểm cực trị của hàm số $ y=f(x) $ là
	\choice{1}{\True 2}{0}{3}
	\loigiai{	Ta có bảng xét dấu của $ f'(x) $:
		\begin{center}
			\begin{tikzpicture}
				\tkzTabInit[lgt=2,espcl=1.5]%
				{$x$ /1,$f'(x)$ /1}
				{$-\infty$ , $0$ , $1$ , $2$ ,$+\infty$}
				\tkzTabLine{ ,+,0,-,0,-,0,+,}
			\end{tikzpicture}
		\end{center}
		Dựa vào bảng xét dấu ta thấy $ f(x) $ có 2 điểm cực trị.
}\end{ex} 



\begin{ex}%[2D1K2-2]%
	\immini{Cho hàm số bậc bốn $ y=f(x) $. Biết $f'(x) $ có đồ thị như hình bên. Khẳng định nào sau đây là khẳng định đúng?
		\choice
		{Hàm số $f(x)$ đồng biến trên khoảng $(-\infty;0)$}
		{Hàm số $f(x)$ nghịch biến trên khoảng $(-1;1)$}
		{Hàm số $f(x)$ có đúng một điểm cực tiểu}
		{\True Hàm số $f(x)$ có đúng một điểm cực đại}
	}{
		\begin{tikzpicture}[>=stealth,line join=round,line cap=round,font=\footnotesize,scale=0.7,smooth]
			\draw[->] (-3,0)--(7,0)node[below]{$x$};
			\foreach \x in {-2,-1,1,2,3,4}\draw[shift={(\x,0)}] (0,2pt)--(0,-2pt) node[below]{\scriptsize $\x$};
			\draw[->] (0,-2)--(0,3)node[right]{$y$};
			\draw[] plot[smooth,tension=.65] coordinates{(-1.7,-2) (-1,0) (0,.7) (1,0)(2.7,-1.2)(4,0) (5,2.5)}node[right]{$y=f'(x)$};
		\end{tikzpicture}
	}
	\loigiai{
		\immini{Dựa vào đồ thị, ta có bảng biến thiên như hình vẽ. \\
		}{% Cần khai báo \usepackage{tkz-tab}
			\begin{tikzpicture}[scale=.8, font=\footnotesize, line join=round, line cap=round, >=stealth]
				\tkzTabInit[nocadre=false,lgt=1,espcl=2,deltacl=0.5]{$x$/.7 ,$y'$/.7,$y$/2}
				{$-\infty$ , $-1$ , $1$, $ 4 $, $+\infty$}
				\tkzTabLine{ , - , $0$ ,+, $ 0 $, -, $0$ , + , }
				\tkzTabVar{+/$+\infty$ , -/$f(-1)$ ,+/$f(-1)$ ,-/$ f(4) $, +/$+\infty$}
		\end{tikzpicture}}
	}
\end{ex} 

\begin{ex}
	\immini{
		Cho hàm số $y=f(x)$ xác định và liên tục trên $\mathbb{R}$. Biết rằng hàm số $f(x)$ có đạo hàm $f'(x)$ và hàm số $y=f'(x)$ có đồ thị như hình vẽ. Khi đó nhận xét nào sau đây đúng?
		\choice
		{\True Hàm số $f(x)$ không có cực trị}
		{Đồ thị hàm số $f(x)$ có đúng $2$ điểm cực tiểu}
		{Đồ thị hàm số $f(x)$ có đúng một cực đại}
		{Hàm số $f(x)$ có $3$ cực trị}
	}{
		\begin{tikzpicture}[scale=.8,font=\footnotesize, line join=round,line cap=round,>=stealth]
			\draw[->] (-2.5,0)--(2.5,0)node[below]{$x$};
			\draw[->] (0,-1)--(0,3.5)node[left]{$y$};
			\draw[samples=100,domain=-1.7:1.7] plot(\x,{(\x)^4-2*(\x)^2+1});
			\draw[dashed] (-1,0)node[below]{$-1$}circle(1pt) (1,0)node[below]{$1$}circle(1pt) (0,1)node[above right]{$1$}circle(1pt);
		\end{tikzpicture}
	}
	\loigiai{
		Dựa vào đồ thị ta thấy $f'(x)\geq 0$, với mọi $x\in\mathbb{R}$.\\
		Suy ra, hàm số $f(x)$ không có cực trị.
	}
\end{ex} 


\Closesolutionfile{ans}

\ind{PHẦN II.} \inden{Câu trắc nghiệm đúng sai. Trong mỗi ý a), b), c), d) ở mỗi câu, học sinh chọn đúng hoặc sai.}\\
\Opensolutionfile{ans}[ans/2D1-B1-d1-2]

\begin{ex}
	Cho hàm số $y=f(x)$ liên tục trên $\mathbb{R}$ và có bảng xét dấu đạo hàm như hình bên.
	\begin{center}
		\begin{tikzpicture}
			\tikzset{double style/.append style = {draw=\tkzTabDefaultWritingColor,double=\tkzTabDefaultBackgroundColor,double distance=2pt}}
			\tkzTabInit[nocadre=false, lgt=1, espcl=1.2]{$x$ /0.7,$y'$ /1}{$-\infty$,$0$,$1$,$2$,$+\infty$}
			\tkzTabLine{,+,$0$,-,d,+,$0$,+,}
		\end{tikzpicture}
	\end{center}
	% \immini{
		\choiceTF
		{Hàm số đồng biến trên khoảng $(-\infty;1)$}
		{\True Hàm số đồng biến trên khoảng $(1;+\infty)$}
		{Hàm số đạt cực đại tại $x=2$}
		{Hàm số có một điểm cực đại và hai điểm cực tiểu}
	% }{\vspace{0.1cm}
		%}
	\loigiai{
		Ta có bảng biến thiên như sau:
		\begin{center}
			\begin{tikzpicture}
				\tikzset{double style/.append style = {draw=\tkzTabDefaultWritingColor,double=\tkzTabDefaultBackgroundColor,double distance=2pt}}
				\tkzTabInit[lgt=1.1,espcl=2,nocadre=True]
				{$x$ /.7, $y'$ /.7,$y$ /2}
				{$-\infty$ , $0$, $1$ ,$2$ ,$+\infty$}
				\tkzTabLine{ ,+,$0$,-,d,+,$0$,+, }
				\tkzTabVar{ -/,+/ /,-/,R,+/$+\infty$}
			\end{tikzpicture}
		\end{center}
	Từ đây, suy ra:
		\begin{enumerate}[a)]
			\item Hàm số đồng biến trên khoảng $(-\infty;1)$ là khẳng định sai.
			\item Hàm số đồng biến trên khoảng $(1;+\infty)$ là khẳng định đúng.
			\item Hàm số đạt cực đại tại $x=2$ là khẳng định sai.
			\item Hàm số có một điểm cực đại và hai điểm cực tiểu là khẳng định sai.
		\end{enumerate}
	}
	
\end{ex} 

\begin{ex}
	Cho hàm số $y=x^3-3x^2+4$ có đồ thị $(C)$. Gọi $A$, $B$ là hai điểm cực trị của $(C)$.
	\choiceTF
	{\True Tập xác định của hàm số là $\mathbb{R}$}
	{Hàm số đồng biến trên khoảng $(0;2)$}
	{\True PTĐT qua hai điểm cực trị của đồ thị hàm số là $2x+y-4=0$}
	{\True Diện tích của tam giác $OAB$ bằng $4$, với $O$ là gốc tọa độ}
	\loigiai{
		\begin{enumerate}[a)]
			\item Hàm số đa thức nên có tập xác định là $D=\mathbb{R}$.
			\item Ta có 
			\begin{itemize}
				\item [$\bullet$] $y'=3x^2-6x$ và $y'=0 \Leftrightarrow x=0$ hoặc $x=2$.
			\end{itemize}
			Bảng biến thiên:
			\begin{center}
				\begin{tikzpicture}
					\tkzTabInit[lgt=1,espcl=3]
					{$x$ /0.7, $y'$ /0.7, $y$ /2.5}
					{$-\infty$,$0$,$2$,$+\infty$}
					\tkzTabLine{,+,$0$,-,$0$,+,}
					\tkzTabVar{-/$-\infty$,+/$4$,-/$0$,+/$+\infty$}
				\end{tikzpicture}
			\end{center}
		Suy ra hàm nghịch biến trên $(0;2)$.
			\item Tọa độ $A(0;4)$, $B(2;0)$. PTĐT $AB$ là
			$$\dfrac{x-0}{2-0}=\dfrac{y-4}{0-4} \Leftrightarrow 2x+y-4=0$$
			\item Diện tích tam giác vuông $OAB$ là $S_{OAB}=\dfrac{1}{2}OA \cdot OB=4$.
		\end{enumerate}

	}
\end{ex} 

\begin{ex}
	Cho hàm số $y=\dfrac{x^2+2x+2}{x+1}$ có đồ thị $(C)$. Gọi $A$, $B$ lần lượt là điểm cực tiểu và điểm cực đại của $(C)$.
	\choiceTF
	{Tập xác định của hàm số là $\mathbb{R}$}
	{Hàm số nghịch biến trên khoảng $(-2;0)$}
	{Tọa độ điểm $A(-2;-2)$, $B(0;2)$}
	{Khoảng cách giữa hai điểm cực trị là $AB=2\sqrt{5}$}
	\loigiai{
		\begin{enumerate}[a)]
			\item Đặt điều kiện mẫu số khác 0, ta được $x+1 \ne 0 \Leftrightarrow x \ne -1$. Suy ra $\mathscr{D}=\mathbb{R}\setminus \left\{-1\right\}$.
			\item $y'=\dfrac{x^2+2x}{(x+1)^2}\Rightarrow y'=0\Leftrightarrow \hoac{& x=-2 \\ & x=0.}$\\
			Ta có bảng xét dấu của hàm $f'(x)$ như sau
			\begin{center}
					\begin{tikzpicture}
					\tkzTabInit[nocadre=false,lgt=1,espcl=3]
					{$x$ /0.7,$y'$ /0.7,$y$ /2}
					{$-\infty$,$-2$,$-1$,$0$,$+\infty$}
					\tkzTabLine{,+,$0$,-,d,-,$0$,+,}
					\tkzTabVar{-/$-\infty$,+/$-2$,-D+/$-\infty$/$+\infty$,-/$2$,+/$+\infty$}
				\end{tikzpicture}
			\end{center}
			Dựa vào bảng xét dấu ta thấy rằng hàm số $y=f'(x)$ nghịch biến trên $(-2;-1)$ và $(-1;0)$.
			\item Tọa độ điểm $A(0;2)$, $B(-2;-2)$
			\item Độ dài $AB=\sqrt{(-2-0)^2+(-2-2)^2}=2\sqrt{5}$.
		\end{enumerate}

	}
\end{ex} 


\begin{ex}
	Xét một chất điểm chuyển động dọc theo trục $Ox$. Toạ độ của chất điểm tại thời điểm $t$ được xác định bởi hàm số $x(t)=t^3-6t^2+9t$ với $t\geq 0$. Khi đó $x'(t)$ là vận tốc của chất điểm tại thời điểm $t$, kí hiệu $v(t)$; $v'(t)$ là gia tốc chuyển động của chất điểm tại thời điểm $t$, kí hiệu $a(t)$.
	\choiceTF
	{Phương trình hàm vận tốc là $v(t)=3t^2-6t+9$}
	{\True Phương trình hàm gia tốc là $a(t)=6t-12$}
	{Vận tốc của chất điểm tăng khi $t\in (0;1)$ hoặc  $t \in (3;+\infty)$}
	{Vận tốc của chất điểm giảm khi $t\in (1;3)$}
	\loigiai{
		\begin{enumerate}
			\item $v(t)=x'(t)=3t^2-12t+9$
			\item $a(t)=v'(t)=6t-12$.
			\item Xét $v'(t)=6t-12$, $v'(t)=0\Leftrightarrow t=2$\\
			Bảng xét dấu
			\begin{center}
				\begin{tikzpicture}
					\tkzTabInit[nocadre=false,lgt=2,espcl=2.1]
					{$t$ /0.6,$v'(t)$ /0.6}
					{$0$,$2$,$+\infty$}
					\tkzTabLine{,-,$0$,+,}
				\end{tikzpicture}
			\end{center}
			Suy ra vận tốc của chất điểm tăng khi $t\in (2;+\infty) $, giảm khi $t\in (0;2)$.
		\end{enumerate}
	}
\end{ex} 

\Closesolutionfile{ans}
% \begin{dang}{Bài toán tìm m để hàm số đồng biến (nghịch biến) trên khoảng cho trước}
\begin{enumerate}[\iconCV]
\item Xét hàm số bậc ba $y=ax^3+bx^2+cx+d$ có $y'=3ax^2+2bx+c$.
	\begin{listEX}[1]
		\item [\ding{172}] Hàm số đồng biến trên  $\mathbb{R}$ khi và chỉ khi $$y' \ge 0,\,\forall x \in \mathbb{R} \Leftrightarrow \heva{&a>0\\&\Delta_{y'}\le 0}.$$
		\item [\ding{173}] Hàm số nghịch biến trên  $\mathbb{R}$ khi và chỉ khi $$y' \le 0, \,\forall x \in \mathbb{R} \Leftrightarrow \heva{&a<0\\&\Delta_{y'}\le 0}.$$
	\end{listEX}
\textit{Trường hợp hệ số $a$ có chứa tham số, ta kiểm tra thêm trường hợp $a=0$.}
\item Xét hàm phân thức $y=\displaystyle\frac{ax+b}{cx+d}$ có $y'=\dfrac{ad-cb}{(cx+d)^2}$, với $ad-cb \ne 0$ và $c \ne 0$.
\begin{itemize}
	\item [\ding{172}] Hàm số đồng biến trên từng khoảng xác định của nó khi và chỉ khi
	$$y'>0,\, \forall x \ne -\dfrac{d}{c}\Leftrightarrow ad-cb>0.$$
	\item [\ding{173}]  Hàm số nghịch biến trên từng khoảng xác định của nó khi và chỉ khi
	$$y'<0,\, \forall x \ne -\dfrac{d}{c}\Leftrightarrow ad-cb<0.$$
\end{itemize}
\item Xét hàm phân thức $y=\displaystyle\frac{ax^2+bx+c}{dx+e}$ có $y'=\dfrac{adx^2+2aex+be-dc}{(dx+e)^2}$, với $ad \ne 0$.
\begin{itemize}
	\item [\ding{172}] Hàm số đồng biến trên từng khoảng xác định của nó khi và chỉ khi
	$$y'\ge 0,\, \forall x \ne -\dfrac{e}{d}\Leftrightarrow adx^2+2aex+be-dc\ge 0,\, \forall x \ne -\dfrac{e}{d}.$$
	\item [\ding{173}]  Hàm số nghịch biến trên từng khoảng xác định của nó khi và chỉ khi
	$$y'\le 0,\, \forall x \ne -\dfrac{e}{d}\Leftrightarrow adx^2+2aex+be-dc\le 0,\, \forall x \ne -\dfrac{e}{d}.$$
\end{itemize}
\end{enumerate}
\end{dang}
\boxmini{BÀI TẬP TỰ LUẬN}
\setcounter{vd}{0}

\begin{vd}
	Tìm tất cả giá trị của tham số $m$ để hàm số
	\begin{tasks}
		\task $y=x^3+mx^2+2mx+2$ đồng biến trên $(-\infty;+\infty)$.
		\task $y=-\dfrac{1}{3}x^3-mx^2+\left(2m-3\right)x-m+2$ nghịch biến trên $\mathbb{R}$.
		\task $ y=\dfrac{1}{3}x^3-mx^2-(2m+1)x+1$ nghịch biến trên khoảng $(0;5)$.
		\task $y=x^3-3x^2+(5-m)x$ đồng biến trên khoảng $(2;+\infty)$.
	\end{tasks}
\loigiai{
\begin{enumerate}[a)]
	\item Hàm số đã cho có tập xác định $\mathscr{D}=\mathbb{R}$ và $y'=3x^2+2mx+2m$.\\
	Hàm số đã cho đồng biến trên $\mathbb{R}$ khi và chỉ khi
	\[y'\ge0,~\forall x\in\mathbb{R}\Leftrightarrow m^2-6m\le0\Leftrightarrow 0\le m\le6.\]
	\item Tập xác định: $D=\mathbb{R}$. Ta có $y'=-x^2-2mx+2m-3$.\\
	Để hàm số nghịch biến trên $\mathbb{R}$ thì:\\
	$y'\le 0,\forall x\in\mathbb{R} \Leftrightarrow\left\{
	\begin{aligned}
		&a_{y'}<0\\
		&\Delta'\le 0
	\end{aligned}
	\right.
	\Leftrightarrow \left\{
	\begin{aligned}
		&-1<0\\
		&m^2+2m-3\le0
	\end{aligned}
	\right.
	\Leftrightarrow -3\le m\le 1$.
	\item Tập xác định $\mathscr{D}=\mathbb{R}$.\\
	Ta có $y'=x^2-2mx-(2m+1)$, $ y'=0\Leftrightarrow\hoac{&x=-1\\&x=2m+1.}$\\
	Nếu $2m+1\leq-1\Leftrightarrow m\leq-1$ thì $y'\leq 0\Leftrightarrow x\in\left[2m+1;-1\right]$.\\
	Suy ra hàm số không nghịch biến trên khoảng $(0;5)$. \\
	$\Rightarrow m\leq-1$ không thỏa mãn.\\
	Nếu $2m+1>-1\Leftrightarrow m>-1$ thì $y'\leq 0\Leftrightarrow x\in\left[-1;2m+1\right]$.\\
	Để hàm số nghịch biến trên khoảng $(0;5)$ thì ta có $2m+1\geq 5\Leftrightarrow m\geq 2$.
	\item \textbf{\underline{Cách 1:}} Tập xác định $\mathscr{D}=\mathbb{R}$.\\
	Ta có $y'=3x^2-6x+5-m$.\\
	Hàm số $y=x^3-3x^2+(5-m)x$ đồng biến trên khoảng $(2;+\infty)$ khi và chỉ khi
	\allowdisplaybreaks
	\begin{eqnarray*}
		&&y'\ge 0,\,\forall x\in (2;+\infty)\\
		&\Leftrightarrow& 3x^2-6x+5-m\ge 0,\,\forall x\in (2;+\infty)\\
		&\Leftrightarrow& m\le 3x^2-6x+5, \,\forall x\in (2;+\infty)
	\end{eqnarray*}
	Xét hàm $g(x)=3x^2-6x+5$ trên $(2;+\infty)$ có $g'(x)=6x-6$ và $g'(x)=0\Leftrightarrow x=1$.\\
	Bảng biến thiên của $g(x)$
	\begin{center}
		\begin{tikzpicture}
			\tkzTabInit[nocadre=false,lgt=1.5,espcl=2,deltacl=0.5]
			{$x$/0.6,$g'(x)$/0.6,$g(x)$/1.5}
			{$2$,$+\infty$}
			\tkzTabLine{,+,}
			\tkzTabVar{-/$5$,+/$+\infty$}
		\end{tikzpicture}
	\end{center}
	Dựa vào bảng biến thiên của $g(x)$, ta được
	$$m\le 3x^2-6x+5, \,\forall x\in (2;+\infty) \Leftrightarrow m\le 5.$$
	\textbf{\underline{Cách 2:}} Tập xác định $\mathscr{D}=\mathbb{R}$.\\
	Ta có $y'=3x^2-6x+5-m$.\\
	Hàm số $y=x^3-3x^2+(5-m)x$ đồng biến trên khoảng $(2;+\infty)$ khi và chỉ khi
	$$y'\ge 0,\,\forall x\in (2;+\infty) 
	\Leftrightarrow \heva{& y'(2)\ge 0 \\ & -\dfrac{b}{2a} \le 2} 
	\Leftrightarrow \heva{& 5-m\ge 0 \\ & 1 \le 2}
	\Leftrightarrow m \le 5. $$
\end{enumerate}}
\end{vd}

\begin{vd}
	Tìm tất cả giá trị của tham số $m$ để hàm số
	\begin{tasks}
		\task $y=\dfrac{mx+2}{x+1}$ đồng biến trên từng khoảng xác định.
		\task $y=\dfrac{mx-2}{x+m-3}$ nghịch biến trên các khoảng xác định
		\task $y = \dfrac{mx-8}{x-2m}$ đồng biến trên $(3;+\infty )$.
		\task $y=\dfrac{mx+9}{4x+m}$ nghịch biến trên khoảng $(0;4)$.
	\end{tasks}
\loigiai{
\begin{enumerate}[a)]
	\item Từ yêu cầu bài toán, $\forall x \neq -1$ ta xét $y'>0$ $\Leftrightarrow m-2>0 \Leftrightarrow m>2$.
	\item Tập xác định $\mathbb{R}\setminus\{3-m\}$.\\
	$y' = \dfrac{m(m - 3) + 2}{\left( x + m - 3\right)^2} = \dfrac{m^2 - 3m + 2}{\left(x + m - 3\right)^2}$. \\
	Điều kiện để hàm số nghịch biến trên các khoảng xác định của nó là $y' < 0,\,\forall x \ne 3 - m$ hay $m^2 - 3m + 2 < 0 \Leftrightarrow m \in (1;2)$.
	\item Tập xác định: $\mathscr{D} = \mathbb{R} \setminus \{2m\}$.\\
	$y' = \dfrac{-2m^2+8}{(x-2m)^2}$.\\
	Hàm số luôn đơn điệu trên từng khoảng xác định $(-\infty; 2m)$ và $(2m; +\infty)$ khi $-2m^2 + 8 \ne 0$.\\
	Vậy hàm số đồng biến trên $(3;+\infty)$ khi và chỉ khi $-2m^2+8 > 0$ và $(3;+\infty) \subset (2m ;+\infty)$. \\
	Điều này tương đương $\heva{&-2<m<2\\&2m \le 3}$, hay $-2 < m \le \dfrac{3}{2}$.
	\item Tập xác định $\mathscr{D}=\mathbb{R}\setminus\left\{-\dfrac{m}{4}\right\}$.\\
	Ta có $y=\dfrac{mx+9}{4x+m}\Rightarrow y'=\dfrac{m^2-36}{(4x+m)^2}$.\\
	Để hàm số nghịch biến trên khoảng $(0;4)$ thì
	$$\heva{& y'<0 ,\forall x\in(0;4)\\ & -\dfrac{m}{4}\notin (0;4)}\Leftrightarrow\heva{& m^2-36<0 \\ &\hoac{&-\dfrac{m}{4}\geq4\\&-\dfrac{m}{4}\leq 0}}\Leftrightarrow\heva{& -6<m<6 \\ &\hoac{&m\leq-16\\&m\geq 0}}\Leftrightarrow 0\leq m<6.$$
\end{enumerate}}
\end{vd}

\begin{vd}
	Tìm tất cả giá trị của tham số $m$ để hàm số
	\begin{tasks}
		\task $ y = \dfrac{2x^2+3x+m+1}{x+1} $ đồng biến trên các khoảng xác định.
		\task $y=\dfrac{x^2+(m+1)x-1}{2-x}$ ($m$ là tham số) nghịch biến trên mỗi khoảng xác định.
	\end{tasks}
	\loigiai{
		\begin{enumerate}[a)]
			\item Tập xác định: $\mathbb{R}\setminus\{-1\}$.\\
			Ta có $y'=\dfrac{2x^2+4x+2-m}{(x+1)^2}$. Hàm số đồng biến trên các khoảng xác định khi 
			$$2x^2+4x+2-m\ge 0, \forall x\in \mathbb{R} \Leftrightarrow m\le \min\limits{\mathbb{R}\setminus \{-1\} } (2x^2+4x+2) = 0.$$
			\item Tập xác định $\mathscr{D}=\mathbb{R}\backslash\{2\}$.\\
			Đạo hàm: $y'=\dfrac{-x^2+4x+2m+1}{(2-x)^2}=\dfrac{g(x)}{(2-x)^2}$.\\
			Hàm số nghịch biến trên mỗi khoảng xác định của nó khi và chỉ khi $y'\le 0,\forall x\in \mathscr{D}$ (Dấu \lq\lq $=$\rq\rq~ chỉ xảy ra tại hữu hạn điểm thuộc $\mathscr{D}$).\\
			$\Leftrightarrow g(x)=-x^2+4x+2m+1\le 0,$  $\forall x\in \mathbb{R}$\\
			Điều kiện: ${\Delta}'\le 0$ (vì $a=-1<0$) $\Leftrightarrow 4-(-1)\cdot(2m+1)\le 0\Leftrightarrow 2m+5\le 0\Leftrightarrow m\le -\dfrac{5}{2}$.
	\end{enumerate}}
\end{vd}

\boxmini{BÀI TẬP TRẮC NGHIỆM}
\ind{PHẦN I.} \inden{Câu trắc nghiệm nhiều phương án lựa chọn. Học sinh trả lời từ câu 1 đến câu 17. Mỗi câu hỏi học sinh chỉ chọn một phương án.}\\
\setcounter{ex}{0}
\Opensolutionfile{ans}[ans/2D1-B1-d2-1]

\begin{ex}%[Nguyễn Trung Kiên, dự án 12-EX-7-2020]%[2D1B1-3]%
	Tất cả giá trị của $m$ để hàm số $y=\dfrac{x+m}{x-2}$ nghịch biến trên từng khoảng xác định là
	\choice
	{\True $m>-2$}
	{$m<-2$}
	{$m\leq -2$}
	{$m\geq -2$}
	\loigiai
	{Tập xác định $\mathscr{D}=\mathbb{R}\setminus \{2\}$ và $y'=\dfrac{-2-m}{(x-2)^2}$.\\
		Hàm số nghịch biến trên các khoảng $(-\infty;2)$ và $(2;+\infty)$ khi và chỉ khi
		\[y'<0,\, \forall x\neq 2\Leftrightarrow -2-m<0 \Leftrightarrow m>-2.\]}
\end{ex} 

\begin{ex}
	Cho hàm số $y=\dfrac{mx-2}{x+1-m}$. Tìm tất cả giá trị của tham số $m$ để hàm số đồng biến trên từng khoảng xác định.
	\choice
	{$\hoac{& m> 2\\& m< -1}$}
	{\True $-1<m<2$}
	{$-1\le m\le 2$}
	{$\hoac{& m\ge 2\\ &m\le -1}$}
	\loigiai{
		Yêu cầu bài toán $\Leftrightarrow ad-bc>0 \Leftrightarrow m(1-m)+2>0 \Leftrightarrow -1<m<2$.
	}
\end{ex} 

\begin{ex}
	Cho hàm số $ y=\dfrac{x+m}{x+2} $. Tập hợp tất cả các giá trị của $ m $ để hàm số đồng biến trên khoảng $ \left(0;+\infty\right)  $ là
	\choice
	{$ \left[2;+\infty\right) $}
	{$ \left(2;+\infty\right)  $}
	{$ \left(-\infty;2\right ]  $}
	{\True $\left(-\infty;2\right)   $}
	\loigiai{
		Hàm số xác định khi $ x\ne -2. $\\
		Có $ y'=\dfrac{2-m}{\left(x+2\right)^2 }, x\ne -2 $.\\
		Hàm số đồng biến trên $ (0;+\infty) $ khi và chỉ khi $ 2-m>0\Leftrightarrow m<2. $
	}
\end{ex} 

\begin{ex}
	Cho hàm số $f(x)=\dfrac{mx-4}{x-m}$ ( $m$ là tham số thực). Có bao nhiêu giá trị nguyên của $m$ để hàm số đồng biến trên khoảng $\left( 0;+\infty  \right)$?  
	\choice
	{$5$}
	{$4$}
	{$3$}
	{\True  $2$}
	\loigiai{
		Ta có $f'(x)=\dfrac{-m^2+4}{{{\left( x-m \right)}^{2}}}$\\
		Hàm số đồng biến trên khoảng $\left( 0;+\infty  \right)$ $\Leftrightarrow$ $\dfrac{-m^2+4}{\left( x-m \right)^2}>0,\,\, \forall x\in \left( 0;+\infty  \right)$\\
		$\Rightarrow \heva{
			& -m^2+4>0 \\ 
			& x\ne m\ \ \forall x\in \left( 0;+\infty  \right) \\ 
		}\Leftrightarrow \heva{
			& m\in \left( -2;2 \right) \\ 
			& m\in \left( -\infty ;0 \right] \\ 
		}\Leftrightarrow m\in \left( -2;0 \right]$\\
		Vậy có hai giá trị nguyên của $m$ là $-1$ và $0$.      
	}
\end{ex} 

\begin{ex}
	Tìm tất cả các giá trị của $m$ để hàm số $y=\dfrac{mx+4}{x+m}$ nghịch biến trên $(-\infty;1)$.
	\choice
	{$-2<m<2$}
	{$-2<m <-1$}
	{$-2\leq m <-1$}
	{\True $-2<m\leq-1$}
	\loigiai{
		ĐKXĐ: $x\neq-m$.\\
		Hàm số $y=\dfrac{mx+4}{x+m}$ nghịch biến trên $(-\infty;1)$\\$\Leftrightarrow y'=\dfrac{m^2-4}{(x+m)^2}<0$, $\forall x\in(-\infty;1)$
		$ \Leftrightarrow\heva{&m^2-4<0\\&-m\geq 1}\Leftrightarrow\heva{&-2<m<2\\&m\leq-1}\Leftrightarrow-2<m\leq-1 $.}
\end{ex} 

\begin{ex}%[THPT Tĩnh Gia - Thanh Hóa, 2020]%[Bùi Mạnh Tiến, 12EX7]%[2D1B1-3]%
	Số giá trị nguyên của tham số $m$ để hàm số $y=\dfrac{mx+10}{2x+m}$ nghịch biến trên khoảng $(0;2)$ là
	\choice
	{\True $6$}
	{$5$}
	{$4$}
	{$9$}
	\loigiai
	{
		Ta có $y'=\dfrac{m^2-20}{(2x+m)^2}$.\\
		Do đó hàm số $y=\dfrac{mx+10}{2x+m}$ nghịch biến trên $(0;2)$ khi và chỉ khi
		\begin{align*}
			\heva{& m^2-20<0 \\ & -\dfrac{m}{2}\notin (0;2)}\Leftrightarrow \heva{& -2\sqrt{5}<m<2\sqrt{5} \\ & \hoac{& -\dfrac{m}{2}\le 0 \\ & -\dfrac{m}{2}\ge 2}}\Leftrightarrow \hoac{& 0\le m<2\sqrt{5} \\ & -2\sqrt{5}<m\le -4.}
		\end{align*}
		Vì $m\in \mathbb{Z}$ nên $m\in \left\{-4;0;1;2;3;4\right\}$.\\
		Vậy có tất cả $6$ giá trị nguyên của $m$ thỏa mãn yêu cầu bài toán.
	}
\end{ex} 

\begin{ex}
	Có bao nhiêu giá trị nguyên của tham số $m$ để hàm số $y=x^3-2mx^2+\left(m^2+3\right)x$ đồng biến trên $\mathbb{R}$?
	\choice
	{$8$}
	{$6$}
	{\True $7$}
	{$0$}
	\loigiai{
		Hàm số $y=x^3-2mx^2+\left(m^2+3\right)x$ đồng biến trên $\mathbb{R}$
		\begin{eqnarray*}
			&\Leftrightarrow &y'=3x^2-4mx+m^2+3\ge 0, \, \forall x\in \mathbb{R}\\
			&\Leftrightarrow & \Delta'=4m^2-3\left(m^2+3\right)\le 0\\
			&\Leftrightarrow & m^2-9\le 0\Leftrightarrow-3\le m\le 3.
		\end{eqnarray*}
		Do $m$ là số nguyên nên $m\in \left\lbrace -3;-2;-1;0;1;2;3\right\rbrace $.\\
		Vậy có $7$ giá trị nguyên của tham số $m$.
	}
\end{ex} 

\begin{ex}
	Cho hàm số $y=-x^3-mx^2+(4m+9)x+5$. Có bao nhiêu giá trị nguyên của $m$ để hàm số nghịch biến trên $\mathbb{R}$?
	\choice
	{\True $7$}
	{$4$}
	{$5$}
	{$6$}
	\loigiai{
		Ta có $y'=-3x^2-2mx+(4m+9)$. Hàm số đã cho nghịch biến trên $\mathbb{R}$ khi và chỉ khi
		\[ \Delta'\le 0 \Leftrightarrow m^2+12m+27\le 0 \Leftrightarrow -9\le m\le -3. \]
		Vậy có tất cả $7$ giá trị nguyên của $m$ thỏa mãn bài toán.
	}
\end{ex} 

\begin{ex}
	Cho hàm số $y=(m-1)x^3 + (m-1)x^2 -2x+5$ với $m$ là tham số. Có bao nhiêu giá trị nguyên của $m$ để hàm số nghịch biến trên khoảng $(-\infty;+\infty)$?
	\choice
	{$5$}
	{\True $7$}
	{$8$}
	{$6$}
	\loigiai{
		\textbf{Trường hợp 1:} $m-1=0 \Leftrightarrow m=1$ khi đó $y=-2x+5$ nghịch biến trên $\mathbb{R}$. Do đó nhận $m=1$.\\
		\textbf{Trường hợp 2:} $m-1\ne 0 \Leftrightarrow m\ne 1$.\\
		Ta có $y'=3(m-1)x^2+2(m-1)x-2$. \\
		Hàm số nghịch biến trên $(-\infty;+\infty) $ $\Leftrightarrow y' \le 0 $, $\forall x\in (-\infty;+\infty)$
		$$\Leftrightarrow \heva{& 3(m-1)<0 \\ & (m-1)^2-3(m-1)\cdot (-2) \le 0} \Leftrightarrow \heva{& m<1 \\ & -5 \le m \le 1} \Leftrightarrow -5 \le m <1.$$.\\
		Do $m \in \mathbb{Z} \Rightarrow m\in \{-5;-4;-3;-2;-1;0\}$.\\
		Vậy cả $2$ trường hợp thì ta có tất cả $7$ giá trị $m$ thỏa yêu cầu bài toán là $\{-5;-4;-3;-2;-1;0;1\}$.
	}
\end{ex} 

\begin{ex}
	Tìm tất cả các giá trị thực của tham số $m$ để hàm số $y=x^3-3mx^2-9m^2x$ nghịch biến trên khoảng $(0;1)$.
	\choice
	{$-1<m<\dfrac{1}{3}$}
	{$m<-1$}
	{$m>\dfrac{1}{3}$}
	{\True $m\ge \dfrac{1}{3}$ hoặc $m\le -1$}
	\loigiai{
		Đặt $f(x)=y'=3x^2-6mx -9m^2$.\\
		Vì $y'$ là hàm số bậc hai với hệ số $a=3>0$ nên để hàm số nghịch biến trên $(0;1)$ thì phương trình $y'=0$ có hai nghiệm phân biệt $x_1, x_2$ thỏa mãn $x_1\le 0<1 \le x_2$ $$\Leftrightarrow \heva{&af(0)\le 0\\&af(1) \le0} \Leftrightarrow \heva{&-9m^2\le 0\\&3-6m-9m^2 \le 0} \Leftrightarrow \hoac{&x\le -1\\&x\ge \dfrac{1}{3}.}$$
	}
\end{ex} 

\begin{ex}
	Có bao nhiêu giá trị nguyên của tham số $ m$ thuộc khoảng $( -2019;2020 )$ để hàm số $ y=2x^3-3( 2m+1 )x^2+6m(m+1)x+2019$ đồng biến trên khoảng $(2;+\infty )$?
	\choice
	{\True $2020$}
	{$2018$}
	{$2021$}
	{$2019$}
	\loigiai{
		Ta có $y'=6x^2-6(2m+1)x+6m^2+6m$.\\
		Xét $y'=0$ $\Leftrightarrow x^2-(2m+1)x+m^2+m=0$, có $\Delta =(2m+1)^2-4\left( m^2+m \right)$ $=1>0$, $\forall m\in \mathbb{R}$. Suy ra phương trình $y'=0$ luôn có hai nghiệm phân biệt: $x_1=m$; $x_2=m+1$. Dễ thấy $x_1<x_2$.\\
		Bảng biến thiên
		\begin{center}
			\begin{tikzpicture}
				\tkzTabInit[nocadre=true,lgt=0.7,espcl=2.1]
				{$x$ /0.6,$y'$ /0.6,$y$ /2}
				{$-\infty$,$m$,$m+1$,$+\infty$}
				\tkzTabLine{,+,$0$,-,$0$,+,}
				\tkzTabVar{-/$-\infty$, +/$y(m)$,-/$y(m+1)$,+/$+\infty$}
			\end{tikzpicture}
		\end{center}
		Dựa vào bảng biến thiên ta thấy hàm số đồng biến trên mỗi khoảng $( -\infty ;m )$; $( m+1;+\infty )$. Vì thế, hàm số đồng biến trên $( 2:+\infty )$ khi $ m+1\le 2\Leftrightarrow m\le 1$.\\
		Suy ra có $2020$ giá trị nguyên của $ m$ thỏa mãn yêu cầu đề bài. }
\end{ex} 

\begin{ex}
	Tập hợp các giá trị thực của tham số $m$ để hàm số $y = - x^3 - 6x^2 + \left(4m - 9\right)x + 4$ nghịch biến trên khoảng $\left(- \infty; - 1\right)$ là
	\choice
	{$\left(- \infty; 0\right]$}
	{$\left[-\dfrac{3}{4}; +\infty\right)$}
	{\True $\left(- \infty; -\dfrac{3}{4}\right]$}
	{$\left[0; +\infty \right)$}
	\loigiai{ 
		Ta có $y'=-3x^2-12x+4m-9$. \\
		Hàm số đã cho nghịch biến trên khoảng $(-\infty;-1)$ khi và chỉ khi $y'\le 0$, $\forall x\in (-\infty;-1)$
		\begin{center}
			$\Leftrightarrow -3x^2-12x+4m-9\le 0\Leftrightarrow 4m\le 3x^2+12x+9$, $\forall x\in (-\infty;-1)$.
		\end{center}
		Đặt $g(x)=3x^2+12x+9\Rightarrow g'(x)=6x+12$. Giải $g'(x)=0\Leftrightarrow x=-2$.\\
		Bảng biến thiên của hàm số $g(x)$ trên $(-\infty;-1)$.
		\begin{center}
			\begin{tikzpicture}
				\tkzTabInit[nocadre=false,lgt=2,espcl=3.5,deltacl=0.6] %phần bắt buộc
				{$x$ /0.6,$g'(x)$ /0.6,$g(x)$ /2}%phần bắt buộc
				{$-\infty$,$-2$,$-1$}
				\tkzTabLine{,-,$0$,+,}
				\tkzTabVar{+/$+\infty$, -/$-3$,+/$0$}
			\end{tikzpicture}
		\end{center}
		Dựa vào bảng biến thiên suy  ra $4m\le g(x)$, $\forall x\in (-\infty;-1)\Leftrightarrow 4m\le -3\Leftrightarrow m\le -\dfrac{3}{4}$.
	}
\end{ex} 

\begin{ex}
	Tìm tất cả các giá trị thực của tham số $m$ sao cho hàm số $y=x^3-6x^2+mx+1$ đồng biến trên khoảng $\left(0;+\infty\right)$.
	\choice
	{$m\leq 12$}
	{\True $m\geq 12$}
	{$m\leq 0$}
	{$m\geq 0$}
	\loigiai{
		Tập xác định $\mathscr{D} =\mathbb{R}$.\\
		$y'=3x^2-12x+m$.\\
		Hàm số đồng biến trên khoảng $\left(0;+\infty\right)$ khi và chỉ khi
		{\allowdisplaybreaks
			\begin{eqnarray*}
				& & f'(x)\geq 0 , \forall x\in \left(0;+\infty\right) \\
				& \Leftrightarrow & 3x^2-12x+m \geq 0 , \forall x\in \left(0;+\infty\right) \\
				& \Leftrightarrow & m \geq -3x^2+12x , \forall x\in \left(0;+\infty\right).
		\end{eqnarray*}}
		Xét hàm số $g(x)= -3x^2+12x$ trên $\left(0;+\infty\right)$.
		Ta có $g'(x)=-6x+12 \Leftrightarrow x=2$.\\
		Bảng biến thiên của hàm số $g(x)$
		\begin{center}
			\begin{tikzpicture}
				\tkzTabInit[lgt=1.2,espcl=3]{$x$ /1, $y'$ /1,$y$ /2}{
					$0$,$2$,$+\infty$}
				\tkzTabLine{,+,0 ,-, }
				\tkzTabVar{-/$0$, +/$12$ ,-/$-\infty$ }
			\end{tikzpicture}
		\end{center}
		Suy ra hàm số đồng biến trên khoảng $\left(0;+\infty\right)$ khi $m \geq 12$.
	}
\end{ex} 

\begin{ex}
	Tìm tất cả các giá trị $m$ để hàm số $y=\dfrac{x^2-8x}{x+m}$ đồng biến trên mỗi khoảng xác định.
	\choice
	{$(-8;0)$}
	{$(0;8)$}
	{$[0;8]$}
	{\True $[-8;0]$}
	\loigiai{
		Ta có $y'=\dfrac{x^2+2mx-8m}{(x+m)^2}$. Khi đó
		\allowdisplaybreaks
		\begin{eqnarray*}
			\text{YCBT} &\Leftrightarrow & x^2+2mx-8m\ge 0, \forall x \Leftrightarrow \Delta' \le 0\\
			&\Leftrightarrow & m^2+8m\le 0\Leftrightarrow -8\le m\le 0.
		\end{eqnarray*}
	}
\end{ex} 

\begin{ex}
	Tập hợp các giá trị thực của tham số $m$ để hàm số $y=x+1+\dfrac{m}{x-2}$ đồng biến trên mỗi khoảng xác định của nó là
	\choice
	{$\left(-\infty;0\right)$}
	{$\left[0;1\right)$}
	{$\left[0;+\infty \right)\backslash \left\{1\right\}$}
	{\True $\left(-\infty;0\right]$}
	\loigiai{
		Tập xác định $\mathscr{D}=\mathbb{R}\backslash \left\{2\right\}$.
		Ta có $y'=1-\dfrac{m}{\left(x-2\right)^2}$.\\
		Hàm số đồng biến trên mỗi khoảng các định của nó khi và chỉ khi
		\begin{eqnarray*}
			&&y'\geq 0,\;\forall x\in \mathbb{R}\backslash \left\{2\right\}\Leftrightarrow 1-\dfrac{m}{\left(x-2\right)^2}\geq 0,\;\forall x\in \mathbb{R}\backslash \left\{2\right\}\\
			&\Leftrightarrow &m\le {\left(x-2\right)}^2,\;\forall x\in \mathbb{R}\backslash \left\{2\right\}\Leftrightarrow m\leq 0.
		\end{eqnarray*}
	}
\end{ex} 

\begin{ex}%[2D1K1-3]%
	Tìm tất cả các giá trị thực của tham số $ m $ để hàm số $ f(x)=2^{x^3-x^2+mx+1}$ đồng biến trên khoảng $(1; 2)$.
	\choice
	{$m\leq-8$}
	{$m>-8$}
	{\True $m\geq-1$}
	{$m<-1$}
	\loigiai{
		Ta có $ f'(x)=(3x^2-2x+m)\cdot 2^{x^3-x^2+mx+1}\cdot\ln 2 $.\\
		Ta thấy\allowdisplaybreaks{
			\begin{eqnarray*}
				&& f(x) \textrm{ đồng biến trên } (1; 2)\\
				\Leftrightarrow && (3x^2-2x+m)\cdot 2^{x^3-x^2+mx+1}\cdot\ln 2\geq 0,\forall x\in (1; 2)\\
				\Leftrightarrow && (3x^2-2x+m)\geq 0,\forall x\in (1; 2)\\
				\Leftrightarrow && m\geq (-3x^2+2x),\forall x\in (1; 2)\\
				\Leftrightarrow && m\geq\max\limits_{[1; 2]} (-3x^2+2x)\\
				\Leftrightarrow && m\geq-1.
			\end{eqnarray*}
		}
	}
\end{ex} 

\begin{ex}
	Có bao nhiêu giá trị nguyên dương của tham số $m$ để hàm số $f(x)=(x+1)\ln x+(2-m)x$ đồng biến trên khoảng $(0;\mathrm{e}^2)$?
	\choice
	{0}
	{3}
	{5}
	{\True 4}
	\loigiai
	{Hàm số đã cho xác định khi $x>0$ hay $D=\big(0;+\infty\big)$\\
			Với $x>0$, ta có $f'(x)=\ln x+\dfrac{x+1}{x}+2-m$.\\
			Hàm số đã cho đồng biến trên khoảng $(0;\mathrm{e}^2)$ khi
			\allowdisplaybreaks
			\begin{align*}
				f'(x) \geq 0, \forall x \in (0;\mathrm{e}^2) &\Leftrightarrow \ln x+\dfrac{x+1}{x}+2-m \geq 0, \forall x \in (0;\mathrm{e}^2)\\
				&\Leftrightarrow m \leq \ln x+\dfrac{x+1}{x}+2, \forall x \in (0;\mathrm{e}^2). \tag{$*$}
			\end{align*}
			Xét hàm số $g(x)=\ln x+\dfrac{x+1}{x}+2, \forall x \in (0;\mathrm{e}^2)$.\\
			Ta có $g'(x)=\dfrac{1}{x}-\dfrac{1}{x^2}=\dfrac{x-1}{x^2}$. Khi đó $g'(x)=0$ có nghiệm $x=1 \in (0;\mathrm{e}^2)$.\\
			Bảng biến thiên của hàm số $g$
			\begin{center}
				\begin{tikzpicture}
					\tkzTabInit[nocadre=false,lgt=1.5,espcl=3.5,deltacl=0.6] %phần bắt buộc
					{$x$/0.6, $g'(x)$/0.6, $g(x)$/2} %phần bắt buộc
					{$0$, $1$, $\mathrm{e}^2$} % hàng 1 cột 2
					\tkzTabLine{,-,z,+,}
					\tkzTabVar{+/$+\infty$,-/$4$,+/$g(\mathrm{e}^2)$}
				\end{tikzpicture}
			\end{center}
			Từ bảng biến thiên trên, bất phương trình $(*)$ thỏa mãn khi $m \leq 4$.
	}
\end{ex} 


\Closesolutionfile{ans}

\ind{PHẦN II.} \inden{Câu trắc nghiệm đúng sai. Học sinh trả lời từ câu 18 đến câu 20. Trong mỗi ý a), b), c), d) ở mỗi câu, học sinh chọn đúng hoặc sai.}\\
	
\Opensolutionfile{ans}[ans/2D1-B1-d2-2]

\begin{ex}
	Cho hàm số $ y=mx^3+mx^2-(m+1)x+1 $, với $m$ là tham số.
	\choiceTF
	{\True Hàm số là hàm số bậc ba khi $m \ne 0$}
	{\True Tập xác định của hàm số là $\mathbb{R}$}
	{Hàm số đồng biến trên $\mathbb{R}$ khi và chỉ khi $m<-\dfrac{3}{4}$ hoặc $m \ge 0$}
	{Hàm số nghịch biến trên $\mathbb{R}$ khi và chỉ khi $-\dfrac{3}{4}\leq m<0$}
	\loigiai{
		\begin{enumerate}[a)]
			\item Với $m \ne 0$ thì hàm số đã cho là một hàm số bậc ba.
			\item Hàm số là hàm đa thức nên có tập xác định là $\mathbb{R}$.
			\item Ta có $ y'=3mx^2+2mx-(m+1)$.
			\begin{itemize}
				\item [$\bullet$] Với $m=0$ thì $y'=-1<0$ (không thỏa)
				\item [$\bullet$] Với $m \ne 0$, yêu cầu bài toán tương đương với
				$\heva{&m>0\\&\Delta \le 0} \Leftrightarrow \heva{&m>0\\&4m^2+3m \le 0}$ (không tồn tại $m$)
			\end{itemize}
			\item 
			\begin{itemize}
				\item [$\bullet$] Với $m=0$ thì $y'=-1<0$ (thỏa)
				\item [$\bullet$] Với $m \ne 0$, yêu cầu bài toán tương đương với
				$$\heva{&m<0\\&\Delta \le 0} \Leftrightarrow \heva{&m<0\\&4m^2+3m \le 0} \Leftrightarrow -\dfrac{3}{4}\leq m<0$$
			\end{itemize}
		Suy ra $-\dfrac{3}{4}\leq m \leq 0$.
		\end{enumerate}
	}
\end{ex} 

\begin{ex}
	Cho hàm số $y=\dfrac{1}{3}x^3 + (m + 1)x^2 + \left(m^2 + 2m\right)x - 3$, với $m$ là tham số.
	\choiceTF
	{Tập xác định của hàm số là $\mathbb{R}$}
	{\True Phương trình $y'=0$ có hai nghiệm phân biệt $x_1=-m$ và $x_2=-m-2$}
	{\True Không tồn tại giá trị của tham số $m$ để hàm số đồng biến trên $\mathbb{R}$}
	{Hàm số nghịch biến trên khoảng $(- 1; 1)$ khi và chỉ khi $m \ge -1$}
	\loigiai{
		\begin{enumerate}[a)]
			\item Hàm số là hàm đa thức nên có tập xác định là $\mathbb{R}$
			\item Ta có $y'=x^2+2(m+1)x+m^2+2m$. Do $\Delta'=b'^2-ac=(m+1)^2-(m^2+2m)=1>0$ nên phương trình có hai nghiệm phân biệt
			$x_1=\dfrac{-b'+\sqrt{\Delta'}}{a}=-m$ và $x_2=\dfrac{-b'-\sqrt{\Delta'}}{a}=-m-2$.
			\item Bảng biến thiên
				\begin{center}
					\begin{tikzpicture}
						\tkzTabInit[lgt=1,espcl=3,nocadre=True]
						{$x$ /0.7, $y'$ /0.7, $y$ /2.5}
						{$-\infty$,$-m-2$,$-m$,$+\infty$}
						\tkzTabLine{,+,$0$,-,$0$,+,}
						\tkzTabVar{-/$-\infty$,+/$y(-m-2)$,-/$y(-m)$,+/$+\infty$}
					\end{tikzpicture}
				\end{center}
			Từ bảng biến thiên, suy ra không tồn tại giá trị của tham số $m$ để hàm số đồng biến trên $\mathbb{R}$
			\item Bảng biến thiên
			\begin{center}
				\begin{tikzpicture}
					\tkzTabInit[lgt=1,espcl=3,nocadre=True]
					{$x$ /0.7, $y'$ /0.7, $y$ /2.5}
					{$-\infty$,$-m-2$,$-m$,$+\infty$}
					\tkzTabLine{,+,$0$,-,$0$,+,}
					\tkzTabVar{-/$-\infty$,+/$y(-m-2)$,-/$y(-m)$,+/$+\infty$}
				\end{tikzpicture}
			\end{center}
			Từ bảng biến thiên, suy ra hàm số nghịch biến trên khoảng $(- 1; 1)$ khi và chỉ khi 
			$$\heva{&-m-2 \le -1\\& -m \ge 1} \Leftrightarrow m = -1.$$
		\end{enumerate}
		
	}
\end{ex} 

\begin{ex}
	Cho hàm số $ y=\dfrac{x+5}{x+m}$, với $m$ là tham số.
	\choiceTF
	{Tập xác định của hàm số là $\mathbb{R}$}
	{Hàm số đồng biến trên từng khoảng xác định khi và chỉ khi $m \ge 5$}
	{\True Hàm số nghịch biến trên từng khoảng xác định khi và chỉ khi $m < 5$}
	{Hàm số đồng biến trên khoảng $\left(-\infty ;\, -8\right)$ khi và chỉ khi $\left(5;\, 8\right)$}
	\loigiai{
		\begin{enumerate}[a)]
			\item Điều kiện $x+m \ne 0 \Leftrightarrow x \ne -m$. Tập xác định là $D=\mathbb{R} \backslash\{-m\}$.
			\item Ta có $y'=\dfrac{m-5}{\left( x+m \right)^2},\forall x\in \mathbb{R}\backslash \left\{ -m \right\}.$\\
			Hàm số đồng biến trên từng khoảng xác định $\Leftrightarrow m-5>0 \Leftrightarrow m>5$.
			\item Ta có $y'=\dfrac{m-5}{\left( x+m \right)^2},\forall x\in \mathbb{R}\backslash \left\{ -m \right\}.$\\
			Hàm số nghịch biến trên từng khoảng xác định $\Leftrightarrow m-5<0 \Leftrightarrow m<5$.
			\item 	Hàm số $ y=\dfrac{x+5}{x+m}$ đồng biến trên khoảng $\left(-\infty ;\, -8\right)$ khi và chỉ khi 
			$$\heva{
				&\dfrac{m-5}{\left(x+m\right)^2}> 0\\
				&-m\notin\left(-\infty ;\, -8\right)
			}\Leftrightarrow \heva{
				&m > 5\\
				&-m\ge-8
			} \Leftrightarrow 5 < m\le 8.$$
		\end{enumerate}
	}
\end{ex} 

\Closesolutionfile{ans}


% \begin{dang}{Bài toán tìm m để hàm số có cực trị hoặc đạt cực trị tại điểm cho trước}
	\begin{enumerate}[\iconCV]
		\item Tìm $m$ để hàm số $y=f(x)$ đạt cực trị tại điểm $x_0$ cho trước ($f(x)$ có đạo hàm tại $x_0$):
		\begin{listEX}[1]
			\item [\ding{172}] Giải điều kiện $y'(x_0)=0$, tìm $m$.
			\item [\ding{173}] Lập bảng biến thiên với $m $ vừa tìm được và chọn giá trị $m$ nào thỏa yêu cầu.
				\end{listEX}
	\item Biện luận cực trị hàm số $y=ax^3+bx^2+cx+d$.\\
	Tính $y'=3ax^2+2bx+c$ với $\Delta_{y'}=b^2-3ac$
	\begin{itemize}
		\item[\ding{172}] $\heva{&\Delta_{y'} >0\\&a \ne 0}$: Hàm số có hai điểm cực trị
		\item[\ding{173}]  $\Delta_{y'} \le 0$ hoặc suy biến $\heva{&a=0\\&b=0}$: Hàm số không có cực trị.
	\end{itemize}
	% \begin{note}
		\begin{enumerate}[\iconMT]
				\item Gọi $x_1$, $x_2$ là hai nghiệm phân biệt của $y'=0$ thì $x_1+x_2=-\dfrac{2b}{3a}$ và $x_1\cdot x_2 =\dfrac{c}{3a}$.
			\begin{itemize}
				\item [$\bullet$] $x_1^2+x_2^2=(x_1+x_2)^2-2x_1 x_2$
				\item [$\bullet$] $(x_1-x_2)^2=(x_1+x_2)^2-4x_1 x_2$
				\item [$\bullet$] $x_1^3+x_2^3=(x_1+x_2)^3-3x_1x_2(x_1+x_2)$.
			\end{itemize}
			\item Các công thức tính toán thường gặp:
			\begin{itemize}
				\item [$\bullet$] Độ dài $MN=\sqrt{(x_N-x_M)^2+(y_N-y_M)^2}$
				\item [$\bullet$]  Khoảng cách từ $M$ đến $\Delta$: $d(M,\Delta)=\dfrac{|Ax_M+By_M+C|}{\sqrt{A^2+B^2}}$, với $\Delta \colon Ax+By+C=0$.
				\item [$\bullet$] Tam giác $ABC$ vuông tại $A \Leftrightarrow \overrightarrow{AB} \cdot \overrightarrow{AC}=0 \Leftrightarrow \text{hoành}\cdot\text{hoành}+\text{tung}\cdot\text{tung}=0$.
				\item [$\bullet$] Diện tích tam giác $ABC$ là  $S=\dfrac{1}{2}|a_1b_2-a_2b_1|$, với $\overrightarrow{AB}=(a_1;b_1)$, $\overrightarrow{AC}=(a_2;b_2)$.
			\end{itemize}
			\item PTĐT qua hai điểm cực trị là $y=-\dfrac{2}{9a}(b^2-3ac)x+d-\dfrac{bc}{9a}$.
		\end{enumerate}
	% \end{note}
	\end{enumerate}
\end{dang}
\boxmini{BÀI TẬP TỰ LUẬN}
\setcounter{vd}{0}
\begin{vd}
	Tìm $m$ để hàm số
	\begin{tasks}
		\task  $y=\dfrac{x^3}{3}-mx^2+(m^2-m+1)x+1$ đạt cực tiểu tại $x=3$.
		\task  $y=x^3-3mx^2+3(m^2-1)x$ đạt cực đại tại $x_0=1$.
	\end{tasks}
\loigiai{
\begin{enumerate}[a)]
	\item Ta có $y'=x^2-2mx+m^2-m+1$. Hàm số đạt cực tiểu tại $x=3$ thì
	$$y'(3)=0 \Leftrightarrow 9-6m+m^2-m+1=0 \Leftrightarrow \hoac{&m=2\\&m=5}.$$
	Lập bảng biến thiên của hàm số với lần lượt hai giá trị $m$ vừa tìm được, ta thấy $m=2$ thỏa.\\
	Vậy $m=2$.
	\item Ta có $y'=3x^2-6mx+3(m^2-1)$\\
	Điều kiện cần và đủ để thỏa điều kiện bài toán là
	\begin{eqnarray*}
		\heva{&y'(1)=0 \\&y''(1)<0}
		\Leftrightarrow \heva{&3m^2-6m=0 \\&6-6m<0}
		\Leftrightarrow \heva{&m=0 \vee m=2 \\&m>1}
		\Leftrightarrow m=2.
	\end{eqnarray*}
	Vậy $m=2$ thì thỏa bài toán.
\end{enumerate}}
\end{vd}

\begin{vd}
	Tìm tất cả giá trị của tham số $m$ để hàm số (đồ thị hàm số)
	\begin{tasks}
		\task $ y=x^3-3x^2+2mx+m+2024$ có hai điểm cực trị.
		\task $ y=\dfrac{1}{3}x^3-mx^2+\left(m+2\right)x+2019$ không có cực trị.
		\task $y=x^3-3(m+1)x^2+12mx+2019$ có hai điểm cực trị $x_1,\ x_2$ thỏa mãn $x_1+x_2+2x_1x_2=-8$.
		\task $y=-x^3-3mx^2+m-2$ với $m$ là tham số có hai điểm cực trị $A,B$ sao cho $AB=2$.
	\end{tasks}
\loigiai{
\begin{enumerate}[a)]
	\item Ta có $y’=3x^2-6x+2m$.\\
	Hàm số có cực đại, cực tiểu khi và chỉ khi phương trình $y’=0$ có hai nghiệm phân biệt $\Leftrightarrow {\Delta }’_{y’}>0$ $\Leftrightarrow 9-6m>0$ $\Leftrightarrow m<\dfrac{3}{2}$.
	\item Ta có $y’=x^2-2mx+m+2$\\
	Hàm số đã cho không có cực trị $\Leftrightarrow$ phương trình $y’=0$ vô nghiệm hoặc có nghiệm kép hay ${\Delta }’_{y’} \le 0$ $\Leftrightarrow m^2-\left( m+2 \right)\le 0$ $\Leftrightarrow -1\le m\le 2$.
	\item Ta có $y'=3x^2-6(m+1)x+12m,\ y'=0\Leftrightarrow 3x^2-6(m+1)x+12m=0$. \\
	Hàm số có hai điểm cực trị $\Leftrightarrow \Delta '=9m^2-18m+9>0\Leftrightarrow m\ne 1$.\tagEX{1}
	Giả sử $x_1,\ x_2$ là hai nghiệm của phương trình $y'=0$, theo định lí Vi-ét ta có
	$$\heva{&x_1+x_2=-\dfrac{b}{a}=2(m+1)\\&x_1\cdot x_2=\dfrac{c}{a}=4m.}$$
	Do đó $x_1+x_2+2x_1\cdot x_2=-8\Leftrightarrow 2(m+1)+8m=-8\Leftrightarrow 10m=-10\Leftrightarrow m=-1$ thỏa mãn $(1)$.\\
	Vậy $m=-1$ là giá trị cần tìm của $m$.
	\item Ta có $y'=-3x^2-6mx$; $y'=0\Leftrightarrow
	\hoac{&x=0\\&x=-2m.\\}$\\
	Hàm số có hai điểm cực trị khi và chỉ khi $m\ne 0$.\\
	Gọi hai điểm cực trị của đồ thị hàm số là $A$, $B$.\\
	Ta có $A\left(0;m-2\right)$, $ B\left(-2m;-4{m}^{3}+m-2\right)$.\\
	Do đó
	{\allowdisplaybreaks
		\begin{align*}
			AB^2=4m^2+16m^6=4&\Leftrightarrow 4m^6+m^2-1=0\\
			&\Leftrightarrow m^2=\dfrac{1}{2}\Leftrightarrow m=\pm \dfrac{1}{\sqrt{2}}.
	\end{align*}}
\end{enumerate}}
\end{vd}

\boxmini{BÀI TẬP TRẮC NGHIỆM}
\ind{PHẦN I.} \inden{Câu trắc nghiệm nhiều phương án lựa chọn. Mỗi câu hỏi học sinh chỉ chọn một phương án.}\\
\setcounter{ex}{0}
\Opensolutionfile{ans}[ans/2D1-B1-d3-1]

\begin{ex}
	Tìm tất cả giá trị của tham số $m$ để hàm số $y=\dfrac{1}{3}x^3+(m+1)x^2+(1-3m)x+2$ có cực đại và cực tiểu.
	\choice
	{$m\leq-5;m\geq 0$}
	{\True $m <-5$; $m>0$}
	{$-5<m<0$}
	{$-5\leq m\leq 0$}
	\loigiai{
		Tập xác định $\mathscr{D}=\mathbb{R}$.\\
		Ta có $y’=x^2+2(m+1)x+1-3m$.\\
		Hàm số có cực đại và cực tiểu khi phương trình $y’=0$ có hai nghiệm phân biệt và đổi dấu qua các nghiệm đó.\\
		Khi đó $\Delta’_{y’}=(m+1)^2-(1-3m)>0\Leftrightarrow m^2+5m>0\Leftrightarrow \hoac{&m<-5\\&m>0.}$}
\end{ex} 

\begin{ex}
	Tìm tất cả các giá trị của tham số $ m $ để hàm số $ y=-x^3-3x^2+mx+2 $ có cực đại và cực tiểu.
	\choice
	{\True $m>-3$}
	{$m\geq 3$}
	{$m\geq-3$}
	{$m>3$}
	\loigiai{
		Ta có $ y'=-3x^2-6x+m $. Hàm số đã cho có cực đại và cực tiểu khi và chỉ khi phương trình $ y'=0 $ có $ 2 $ nghiệm phân biệt $\Leftrightarrow\Delta'>0\Leftrightarrow 9+3m>0\Leftrightarrow m>-3 $.
	}
\end{ex} 

\begin{ex}
	Cho hàm số $y=x^3-3(m+1)x^2+3(7m-3)x$. Số giá trị nguyên của tham số $m$ để hàm số không có cực trị là
	\choice
	{$2$}
	{$1$}
	{\True $4$}
	{$3$}
	\loigiai{
		Hàm số bậc $3$ không có cực trị khi và chỉ khi phương trình $y'=0 \Leftrightarrow 3x^2-6(m+1)x+3(7m-3)=0$ có nghiệm kép hoặc vô nghiệm hay
		$$\Delta' \le 0 \Leftrightarrow 9(m+1)^2-9(7m-3)\le 0 \Leftrightarrow m^2-5m+4 \le 0 \Leftrightarrow 1 \le m \le 4.$$
		Mà $m \in \mathbb{Z}$ nên $ m \in \{1;2;3;4\}$.\\
		Vậy có $4$ giá trị nguyên của $m$ thỏa mãn yêu cầu bài toán.
	}
\end{ex} 

\begin{ex}
	Cho hàm số $y=x^3-3(m+1)x^2+3(7m-3)x$. Gọi $S$ là tập hợp tất cả các giá trị nguyên của tham số $m$ để hàm số không có cực trị. Số phần tử của $S$ là
	\choice
	{$2$}
	{\True $4$}
	{$0$}
	{Vô số}
	\loigiai{
		Tập xác định là $\mathscr{D}=\mathbb{R}$.\\
		$y'=3x^2-6(m+1)x+3(7m-3)$.\\
		Hàm số không có cực trị khi và chỉ khi $\Delta'=9(m+1)^2-9(7m-3)\le 0\Leftrightarrow m^2-5m+4\le 0\Leftrightarrow 1\le m \le 4.$\\
		Vậy có $m\in \{1;2;3;4\}$.
	}
\end{ex} 

\begin{ex}
	Giả sử hàm số $ y=\dfrac{1}{3}x^3-x^2-\dfrac{1}{3}mx$ có hai điểm cực trị $x_1, x_2$ thỏa mãn $x_1+ x_2+2x_1x_2=0$. Giá trị của $ m$ là
	\choice
	{$ m=\dfrac{4}{3}$}
	{$ m=-3$}
	{\True $ m=3$}
	{$ m=2$}
	\loigiai{
		Ta có $y’=x^2-2x-\dfrac{1}{3}m$.\\
		$y’=0\Leftrightarrow  3x^2-6x-m=0$.\\
		Hàm số có hai cực trị $\Leftrightarrow y'=0$ có hai nghiệm phân biệt $\Leftrightarrow 9+3m>0 \Leftrightarrow m>-3$.\\
		Khi đó $x_1+ x_2 + 2x_1x_2=0 \Leftrightarrow 2-\dfrac{2m}{3}=0 \Leftrightarrow m=3$ (TM).}
\end{ex} 

\begin{ex}
	Cho hàm số $ f\left( x \right)=x^3-3x^2+mx-1$. Tìm giá trị của tham số $m$ để hàm số có hai cực trị $x_1, x_2$ thỏa $x_1^2+x_2^2=3$.
	\choice
	{$ m=\dfrac{1}{2}$}
	{$ m=-2$}
	{$ m=1$}
	{\True $ m=\dfrac{3}{2}$}
	\loigiai{
		TXĐ $D=\mathbb{R}$.\\
		${f}’\left( x \right)=3x^2-6x+m$.\\
		Hàm số có hai điểm cực trị $x_1, x_2 \Leftrightarrow {f}’\left( x \right)=0$ có hai nghiệm phân biệt $\Leftrightarrow 9-3m>0  \Leftrightarrow m<3$.\\
		Theo hệ thức Vi-et: $x_1+ x_2=2$; $x_1.x_2=\dfrac{m}{3}$.\\
		Khi đó: $x_1^2+x_2^2=3  \Leftrightarrow  \left( {x_1+ x_2} \right)^2 - 2x_1x_2=3 \Leftrightarrow 2^2-\dfrac{2m}{3}=3 \Leftrightarrow m=\dfrac{3}{2}$.}
\end{ex} 

\begin{ex}
	Tìm tất cả các giá trị của tham số $m$ để đồ thị hàm số $y=x^3-12x+m+2$ có hai cực trị và hai điểm cực
	trị này nằm về hai phía trục hoành?
	\choice
	{$m=-2$}
	{\True $-18<m<14$}
	{$\forall m\in \mathbb{R}$}
	{$m\neq 1$}
	\loigiai{
		Ta có $y'=3x^2-12$. Suy ra $y'=0\Leftrightarrow \hoac{& x=2\Rightarrow y=m-14 \\ & x=-2\Rightarrow y=m+18.}$\\
		Đồ thị hàm số có hai điểm cực trị nằm về hai phía trục hoành khi và chỉ khi
		$$(m-14)(m+18)<0\Leftrightarrow -18<m<14.$$
	}
\end{ex} 

\begin{ex}
	Tập hợp các giá trị của $m$ để đồ thị hàm số $y=x^3+mx^2-\left(m^2-4\right)x+1$ có hai điểm cực trị nằm ở hai phía của trục $Oy$ là
	\choice
	{$(-\infty;2)$}
	{\True $\mathbb{R}\setminus[-2;2]$}
	{$(-2;2)$}
	{$(2;+\infty)$}
	\loigiai{
		Ta có $y'=3x^2+2mx+4-m^2$.\\
		Đồ thị hàm số có hai cực trị nằm hai phía đối với trục $Oy$ khi và chỉ khi $y'=0$ có hai nghiệm trái dấu $\Leftrightarrow P=\dfrac{4-m^2}{3}<0\Leftrightarrow\hoac{&m>2\\&m<-2.}$}
\end{ex} 

\begin{ex}
	Cho hàm số $y=x^3+3mx^2+3(m^2-1)x+m^3.$ Tìm $m$ để hàm số đạt cực tiểu tại điểm $x=0.$
	\choice
	{$m=-1$}
	{\True $m=1$}
	{$m=0$}
	{$m=2$}
	\loigiai{
		Ta có $y'=3x^2+6mx+3(m^2-1)$ và $y''=6x+6m\Rightarrow y''(0)=6m.$\\
		Hàm số đạt cực tiểu tại $x=0\Rightarrow y'(0)=0\Leftrightarrow 3(m^2-1)=0\Leftrightarrow m=\pm 1.$\\
		Với $m=1\Rightarrow y''(0)=6>0\Rightarrow$  hàm số đạt cực tiểu tại $x=0.$\\
		Với $m=-1\Rightarrow y''(0)=-6<0\Rightarrow$  hàm số đạt cực đại tại $x=0.$\\
		Vậy $m=1$ thỏa mãn bài.
	}
\end{ex} 

\begin{ex}
	Hàm số $ y=x^3-2mx^2+m^2x-2 $ đạt cực tiểu tại $ x=1 $ khi
	\choice
	{$ m=3 $}
	{$ m=-3 $}
	{\True $ m=1 $}
	{$ m=-1 $}
	\loigiai{
		Ta có: $ y'=3x^2-4mx+m^2 ,
		y''=6x-4m. $\\
		Hàm số đạt cực tiểu tại $ x=1 $, suy ra $y'(1)=0\Leftrightarrow m^2-4m+3=0 \Leftrightarrow \hoac{&m=1\\&m=3.}$
		\begin{itemize}
			\item Với $m=1$ ta có $y'(1)=0, y''(1)=2>0$ nên hàm số đạt cực tiểu tại $x=1$.
			\item Với $m=3$ ta có $y'(1)=0, y''(1)=-6<0$ nên hàm số đạt cực đại tại $x=1$.
	\end{itemize}}
\end{ex} 

\begin{ex}
	Tìm giá trị thực của tham số $m$ để hàm số $y=\dfrac{1}{3}x^3-mx^2+(m^2-4)x+3$ đạt cực tiểu tại $x=3$.
	\choice
	{$m=-1$}
	{\True $m=1$}
	{$m=-7$}
	{$m=5$}
	\loigiai{
		Ta có $y'=x^2-2mx+m^2-4$ và $y''=2x-2m$.\\
		Hàm số đạt cực tiểu tại $x=3$ nên $y'(3)=0 \Leftrightarrow 9-6m+m^2-4=0 \Leftrightarrow \hoac{&m=5 \\ &m=1.}$\\
		Với $m=5$ thì $y''(3)=-4<0$, loại.\\
		Với $m=1$ thì $y''(3)=4>0$, thỏa mãn.
	}
\end{ex} 

\begin{ex}
	Đồ thị hàm số $y=x^3-3x^2+2ax+b$ (với $a, b \in \mathbb{R}$) có điểm cực tiểu $A(2;-2)$. Khi đó $a+b$ bằng
	\choice
	{$-4$}
	{$4$}
	{\True $2$}
	{$-2$}
	\loigiai{
		Ta có: $y’=3x^2-6x+2a; y''=6x-6$.\\
		Đồ thị hàm số có điểm cực tiểu $A(2;-2)$ nên ta có:
		$\heva{&y’(2)=0\\&y(2)=-2} \Leftrightarrow \heva{&2a=0\\&4a+b=2} \Leftrightarrow \heva{&a=0\\&b=2.}$\\
		Với $a=2,b=0$ ta thấy $y''(2)=6.2-6=6>0$ nên hàm số đạt cực tiểu tại $x=2$, thỏa yêu cầu bài toán.\\
		Suy ra $a+b=2$.
	}
\end{ex} 

\begin{ex}
	Gọi $m_1, m_2$ là các giá trị của tham số $m$ để đồ thị hàm số $y=2x^{3}-3x^{2}+m-1$ có hai điểm cực trị $B, C$ sao cho tam giác $OBC$ có diện tích bằng 2, với $O$ là gốc tọa độ. Tích $m_{1} \cdot m_{2}$ bằng
	\choice
	{$12$}
	{$6$}
	{\True $-15$}
	{$-20$}
	\loigiai{
		Tập xác định: $\mathscr{D}=\mathbb{R}$.\\
		Ta có \begin{eqnarray*}
			y'=6 x^{2}-6 x=0 &\Leftrightarrow&
			\hoac{x=0 \Rightarrow y=m-1 \Rightarrow B(0 ; m-1) \\ x=1 \Rightarrow y=m-2 \Rightarrow C(1 ; m-2)}\\
			&\Rightarrow& S_{\triangle OBC}=\dfrac{1}{2} d(C ; O B) \cdot O B=\dfrac{1}{2} \cdot 1 \cdot |m-1|=2\\
			&\Leftrightarrow& |m-1|=4
			\Leftrightarrow \hoac{&m_1=5 \\ &m_2=-3.}
		\end{eqnarray*}
		Vậy $m_1 \cdot m_2 = -15$.
	}
\end{ex} 

\begin{ex}
	Cho hàm số $y=x^3-3mx^2+3m^3$. Biết rằng có hai giá trị của tham số $m$ để đồ thị hàm số có hai điểm cực trị $A,B$ và tam giác $OAB$ có diện tích bằng $48$. Khi đó tổng các giá trị của $m$ là
	\choice
	{\True $0$}
	{$2$}
	{$\sqrt{2}$}
	{$-2$}
	\loigiai{
		Tập xác định $\mathscr{D}=\mathbb{R}$.\\
		Đạo hàm $y'=3x^2-6mx$, xác định với mọi $x\in\mathbb{R}$.\\
		$y'=0\Leftrightarrow\hoac{&x=0\\&x=2m.}$ \\
		Do đó hàm số có hai cực trị khi và chỉ khi $m\neq 0$.\\
		Khi đó $A\left(0;3m^3\right)$, $B\left(2m;-m^3\right)$.\\
		Suy ra $\overrightarrow{OA}=\left(0;3m^3\right)$, $\overrightarrow{OB}=\left(2m;-m^3\right)$.\\
		$S_{\triangle OAB}=48\Leftrightarrow \dfrac{1}{2}\left|\left[\overrightarrow{OA},\overrightarrow{OB}\right]\right|=48\Leftrightarrow \left|-6m^4\right|=96\Leftrightarrow m=\pm 2$.\\
		Vậy tổng các giá trị của $m$ là $0$.
	}
\end{ex} 
\Closesolutionfile{ans}

\ind{PHẦN II.} \inden{Câu trắc nghiệm đúng sai. Trong mỗi ý a), b), c), d) ở mỗi câu, học sinh chọn đúng hoặc sai.}\\
\Opensolutionfile{ans}[ans/2D1-B1-d3-2]

\begin{ex}
	Cho hàm số $ y=\dfrac{m}{3}x^3+2x^2+mx+1$, với $m$ là tham số.
	\choiceTF
	{Hàm số có hai điểm cực trị khi $-2<m<2$}
	{Hàm số có đúng một điểm cực trị khi $m=0$ hoặc $m=2$}
	{\True Hàm số không có cực trị khi $m \le -2$ hoặc $m \ge 2$}
	{\True Hàm số có $2$ điểm cực trị thỏa mãn $x_\text{CĐ}<x_{CT}$ khi $0<m<2$}
	\loigiai{
		\begin{enumerate}[a)]
			\item Ta có $y’=mx^2+4x+m$.\\
			Hàm số có $2$ điểm cực trị $\Leftrightarrow y’=0$ có $2$ nghiệm phân biệt $\Leftrightarrow \left\{ \begin{aligned}
				& m\ne 0 \\
				& 4-m^2>0 \\
			\end{aligned} \right.\Leftrightarrow \left\{ \begin{aligned}
				& m\ne 0 \\
				& -2<m<2 \\
			\end{aligned} \right.\quad(1)$.
			\item Hàm số có đúng 1 cực trị khi hàm số này bị suy biến về hàm bậc hai, nghĩa là $\dfrac{m}{3}=0 \Leftrightarrow m=0$.
			\item Với $m=0$ thì hàm số trở thành $y=2x^2+1$. Hàm số này có 1 điểm cực tiểu. Điều này không thỏa yêu cầu bài toán\\
			Với $m \ne 0$: Hàm số không có cực trị $\Leftrightarrow y’=0$ có vô nghiệm hoặc nghiệm kép. $\Leftrightarrow \left\{ \begin{aligned}
				& m\ne 0 \\
				& 4-m^2 \le 0 \\
			\end{aligned} \right.\Leftrightarrow \left\{ \begin{aligned}
				& m\ne 0 \\
				&  m \le -2,\,m \ge 2\\
			\end{aligned} \right.$.
			\item Dựa vào dạng đồ thị hàm số bậc $3$, hàm số có $2$ điểm cực trị thỏa mãn $x_\text{CĐ}<x_{CT}$ khi $ m>0$ $(2)$\\
			Từ $\left(1\right)$ và $\left(2\right)$ suy ra giá trị $ m$ cần tìm là $0<m<2$.
		\end{enumerate}
}
\end{ex} 

\begin{ex}
	Cho hàm số $y=x^3-3mx^2+3\left(m^2-1\right)x-m^3$ với $m$ là tham số.
	\choiceTF
	{\True Hàm số luôn có hai điểm cực trị với mọi $m$}
	{\True Hàm số đạt cực tiểu tại $x=3$ khi $m=2$}
	{\True Khi đồ thị hàm số có hai điểm cực trị thì khoảng cách giữa hai điểm cực trị bằng $2\sqrt{5}$}
	{\True Điểm cực tiểu của đồ thị hàm số luôn thuộc đường thẳng cố định với hệ số góc $k=-3$}
	\loigiai
	{
		\begin{enumerate}[a)]
			\item Ta có $y'=3x^2-6mx+3\left(m^2-1\right). y'=0\Leftrightarrow \hoac{&x_1=m-1\\&x_2=m+1}$.\\
			Do $x_1 \ne x_2, \,\forall m$ nên hàm số luôn có hai điểm cực trị.
			\item Dễ thấy $x=m+1$ là điểm cực tiểu. Suy ra, hàm số đạt cực tiểu tại $x=3$ khi $m+1=3 \Leftrightarrow m=2$.
			\item Với mọi $m$, tọa độ hai điểm cực trị là $A(m+1;-3m-2)$ và $B(m-1;-3m+2)$.\\
			Khoảng cách giữa hai điểm cực trị là $AB=\sqrt{(x_B-x_A)^2+(y_B-y_A)^2}=2\sqrt{5}$.
			\item Ta có $y'=3x^2-6mx+3\left(m^2-1\right). y'=0\Leftrightarrow \hoac{&x=m-1\\&x=m+1}$\\
			Vì là hàm số bậc ba với hệ số $a=1>0$ nên điểm cực tiểu của hàm số là $A\left(m+1;-3m-2\right)$. \\
			Lại có $-3m-2=-3\left(m+1\right)+1$ nên điểm cực tiểu của hàm số luôn thuộc đường thẳng $d:y=-3x+1$, hệ số góc $k=-3$.
		\end{enumerate}
	}
\end{ex} 

\begin{ex}
	Cho hàm số $y=\dfrac{x^2-2mx +m +2}{x-m}$, với $m$ là tham số.
	\choiceTF
	{\True Tập xác định của hàm số là $\mathbb{R}\backslash\{m\}$}
	{\True Có hai giá trị nguyên của tham số $m$ để hàm số có hai điểm cực trị}
	{\True Hàm số đạt cực đại tại $x=-1$ khi $m=\dfrac{1}{2}$}
	{Khi đồ thị hàm số có hai điểm cực trị thì đường thẳng qua hai điểm cực trị của đồ thị có phương trình là $y=2x-2m$}
	\loigiai{
		\begin{enumerate}[a)]
			\item Hàm số xác định khi $x-m \ne 0 \Leftrightarrow x \ne m$. Suy ra $\mathscr{D}=\mathbb{R}\backslash\{m\}$.
			\item $y'=\dfrac{x^2-2mx+2m^2-m-2}{(x-m)^2}$.\\
			Để hàm số có hai điểm cực trị thì $y'=0$ có hai nghiệm phân biệt khác $m$ hay $g(x)=x^2-2mx+2m^2-m-2$ có hai nghiệm phân biệt khác $m$.
			$$\Leftrightarrow \heva{&\Delta'>0\\&g(m) \ne 0} \Leftrightarrow \heva{&-m^2+m+2>0\\&m^2-m-2 \ne 0}  \Leftrightarrow m \in (-1;2).$$
			Vì $m$ nguyên nên $m \in \{0;1\}$.
			\item Hàm số đạt cực trị tại $x=-1$ thì $y'(-1)=0 \Leftrightarrow 2m^2+m-1 =0 \Leftrightarrow m=-1$ hoặc $m=\dfrac{1}{2}$.\\
			Thử lại với $m=\dfrac{1}{2}$, ta có $y'=\dfrac{x^2-x-2}{x-\dfrac{1}{2}}$.\\
				Bảng biến thiên
				\begin{center}
					\begin{tikzpicture}
						\tkzTabInit[nocadre=false,lgt=1,espcl=3]
						{$x$ /0.7,$y'$ /0.7,$y$ /2}
						{$-\infty$,$-1$,$0.5$,$2$,$+\infty$}
						\tkzTabLine{,+,$0$,-,d,-,$0$,+,}
						\tkzTabVar{-/$-\infty$,+/$y_1$,-D+/$-\infty$/$+\infty$,-/$y_2$,+/$+\infty$}
					\end{tikzpicture}
				\end{center}
			Suy ra $m=\dfrac{1}{2}$ thỏa yêu cầu bài toán.
			\item Cho hàm số $y=\dfrac{u(x)}{v(x)}$. Nếu đồ thị hàm số có hai điểm cực trị thì đường thẳng qua hai điểm cực trị có dạng $y=\dfrac{u'(x)}{v'(x)}$.\\
			Áp dụng, ta được $y=\dfrac{(x^2-2mx+m+2)'}{(x-m)'}=2x-2m$
		\end{enumerate}
	}
\end{ex} 

\Closesolutionfile{ans}

% \input{data/12/2D1-B1-4.tex}
% \begin{dang}{Cực trị hàm hợp, hàm chứa trị tuyệt đối}
    \begin{itemize}
        \item Các phép biến đổi đồ thị
        \begin{itemize}
            \item Đồ thị hàm $y=f(x+a)$ vẽ bằng cách dời đồ thị $y=f(x)$ sang trái $a$ đơn vị.
            \item Đồ thị hàm $y=f(x)+b$ vẽ bằng cách dời đồ thị $y=f(x)$ lên trên $b$ đơn vị.
            \item Đồ thị hàm $y=f(|x|)$ vẽ bằng cách "lật qua trái".
            \item Đồ thị hàm $y=|f(x)|$ vẽ bằng cách "lật lên".
            \item Đồ thị hàm $y=|f(|x|)|$ vẽ bằng cách "lật lên rồi lật qua trái".
        \end{itemize}
        \begin{note} Hàm $y=f(x)$ có $m$ điểm cực trị, $n$ nghiệm bội lẻ, $p$ điểm cực trị dương. Khi đó
            \begin{itemize}
                \item[-]Hàm $y=f(ax+b)+c$ cũng có $m$ điểm cực trị.
                \item[-]	Hàm $y=|f(x)|$ có $m+n$  điểm cực trị.
                \item[-] 	Hàm $y=f(|x|)$ có $2p+1$  điểm cực trị.
            \end{itemize}
        \end{note}
        \item Hàm $y=f(u)$.
        \begin{itemize}
            \item \textbf{Bước 1: } Tính đạo hàm $y'=u'f'(u)$.
            \item \textbf{Bước 2: } Lập bảng xét dấu của $y'$ hoặc đếm số nghiệm bội lẻ của $y'=0$.
            \item \textbf{Bước 3: } Kết luận.
        \end{itemize}
        \item Hàm $y=f(u)+g(x)$.
        \begin{itemize}
            \item \textbf{Bước 1: } Tính đạo hàm $y'=u'f'(u)+g'$.
            \item \textbf{Bước 2: } Lập bảng xét dấu của $y'$ hoặc đếm số nghiệm bội lẻ của $y'=0$ (dựa vào tương giao giữa hai đồ thị).
            \item \textbf{Bước 3: } Kết luận.
        \end{itemize}
    \end{itemize}
\end{dang}
\begin{vd}
    \immini{Cho hàm số $y=f(x)$ có bảng biến thiên như hình vẽ. Tìm các điểm cực trị, các cực trị của hàm số sau
        \begin{listEX}[1]
            \item $y=f(x+2)$
            \item $y=f(x)-3$
            \item $y=f(2x-3)+1$
            \item $y=f(1-2x)+2025$
    \end{listEX}}{\begin{tikzpicture}[>=stealth]
            \tkzTabInit[nocadre=false,lgt=1,espcl=2,deltacl=0.5]{$x$/.6 ,$y'$/.6,$y$/1.8}
            {$-\infty$ , $0$ , $2$ , $+\infty$}
            \tkzTabLine{ , - , $0$ , + , $0$ , - , }
            \tkzTabVar{+/$+\infty$ , -/$1$ , +/$5$ , -/$-\infty$}
    \end{tikzpicture}}
    \loigiai{}
\end{vd}
\begin{vd}
    \immini{Cho hàm số $y=f(x)$ có bảng biến thiên như hình vẽ.Tìm các điểm cực trị của hàm số sau
        \begin{listEX}[1]
            \item $y=f(x^2)$
            \item $y=f(3x^2-2x)$
            \item $y=f(\sqrt{x^2+2x+2})$
    \end{listEX}}{
        \begin{tikzpicture}[>=stealth]
            \tkzTabInit[nocadre=false,lgt=1,espcl=2,deltacl=0.5]{$x$/.6,$y'$/.6,$y$/1.8}
            {$-\infty$ , $0$ , $2$ , $+\infty$}
            \tkzTabLine{ , - , $0$ , + , $0$ , - , }
            \tkzTabVar{+/$+\infty$ , -/$1$ , +/$5$ , -/$-\infty$}
        \end{tikzpicture}
    }
    \loigiai{}
\end{vd}
\begin{vd}%[2D1G5-5]
    \immini{Cho hàm số $y=f(x)$ có đồ thị $y=f'(x)$ như hình vẽ. Tìm số điểm cực trị của các hàm số sau
        \begin{listEX}[2]
            \item $y=f(x)$
            \item $y=2f(x)-x$
            \item $y=f(3x)+2x$
            \item $y=f(x)+\dfrac{x^2}{2}-x$
            \item $y=3f(x)-2x^3$
            \item $y=f(2x+1)-4x$
    \end{listEX}}{\begin{tikzpicture}[smooth, >=latex, line cap =round, line join =round,font=\scriptsize,x=1.4cm]
            \begin{scope}[scale=.5]
                \draw[->] (-3,0)--(3,0) node[below]{$x$};
                \draw[->] (0,-2.5) -- (0,3) node[left] {$y$};
                \draw[ name path=dcong] (-2,-2)..controls +(90:0.3) and +(180:0.3)..(-0.7,2.7)..controls +(0:0.2) and +(180:0.3)..(0,0.5)..controls +(30:0.2) and +(180:0.3)..(1,2)..controls +(0:0.3) and +(90:0.1).. (2,-2);
                \draw[thick,dashed] (-1,0) node[below] {$-1$} --(-1,2) --(1,2) -- (1,0) node[below] {$1$} (0,2) node[above right] {$2$};
            \end{scope}
    \end{tikzpicture}}
    \loigiai{}
\end{vd}
\begin{vd}
    \immini{Cho hàm số $y=f(x)$ có đồ thị như hình vẽ. Tìm số điểm cực trị của hàm số
        \begin{listEX}[2]
            \item $y=f(|x|)$
            \item $y=|f(x)|$
            \item $y=|f(|x|)|$
            \item $y=f(|x|-a)$
            \item $y=f(|x+b|)$
            \item $y=|f(x+2025)|$
    \end{listEX}}{
        \begin{tikzpicture}[>=stealth,line join=round, line cap=round, font=\scriptsize]
            \begin{scope}[scale=.8]
                \draw[-stealth](-4,0)--(0,0)node[below right]{$O$}--(4,0)node[below left]{$x$};
                \draw[-stealth](0,-2)--(0,3)node[below left]{$y$};
                \draw[dashed]
                (-3,0)node[above]{$a$}--(-3,-2)
                (3,0)node[below]{$b$}--(3,3)
                ;
                \draw[smooth]
                (-3,-2)..controls+(85:3) and+(180:.5)..(-2,2)
                ..controls+(0:.5) and+(180:.5)..(-1,1)
                ..controls+(0:.5)and+(180:.5)..(0.5,2)
                ..controls+(0:.5)and+(180:.75)..(1.5,-1.5)
                ..controls+(0:.75)and+(-95:.3)..(3,3)
                ;
            \end{scope}
    \end{tikzpicture}}
    \loigiai{}
\end{vd}
\begin{vd}
    Tìm $m$ để
    \begin{listEX}
        \item  Hàm số $y=|f(x)|$ có $5$ điểm cực trị, với  $f(x)= 3x^3+3x^2+mx+m$
        \item Hàm số $y=f\left(\vert x\vert\right)$ có $5$ điểm cực trị, với $f(x)=x^3-(2m-1)x^2+(2-m)x+2$.
    \end{listEX}
    \loigiai{
        \begin{listEX}
            \item Đặt $f(x)=3x^3+3x^2+mx+m=3x^2(x+1)+m(x+1)=(x+1)(3x^2+m)$.\\
            Suy ra $f'(x)=9x^2+6x+m$.\\
            Phương trình $f'(x)=0$ có $2$ nghiệm phân biệt $x_1$, $x_2$ khi và chỉ khi $\Delta'=9-9m>0\Leftrightarrow m<1$. Khi đó ta có $x_1+x_2=-\dfrac{2}{3}$, $x_1x_2=\dfrac{m}{9}$.\\
            Hàm số $y=|f(x)|$ có $5$ điểm cực trị khi và chỉ khi $\heva{&\Delta'>0\\&y(x_1)\cdot y(x_2)<0.}$\\
            Thực hiện biến đổi
            \allowdisplaybreaks
            \begin{eqnarray*}
                y(x_1)\cdot y(x_2) &=&\ (x_1+1)(3x_1^2+m)\cdot(x_2+1)(3x_2^2+m)\\
                &=&\ \left[9(x_1x_2)^2+3m(x_1^2+x_2^2)+m^2\right]\left(x_1x_2+x_1+x_2+1\right)\\
                &=&\ \left[\dfrac{m^2}{9}+3m\left[\left(-\dfrac{2}{3}\right)^2-\dfrac{2m}{9}\right]+m^2\right]\left(\dfrac{m}{9}-\dfrac{2}{3}+1\right)\\
                &=&\ \dfrac{1}{9}(4m^2+12m)(m+3).
            \end{eqnarray*}
            Suy ra $y(x_1)\cdot y(x_2)<0\Leftrightarrow (4m^2+12m)(m+3)<0\Leftrightarrow -3\neq m<0$.\\
            Kết hợp với điều kiện $m$ là số nguyên thỏa $|m|<10$ ta được $m\in\{-1;-2;-4;-5;-6;-7;-8;-9\}$.\\
            Vậy có $8$ giá trị nguyên của tham số $m$.
            \item Tập xác định $\mathscr{D}=\mathbb{R}$.\\
            Ta có $f\left(|-x|\right)=f\left(|x|\right)$, $\forall x\in\mathbb{R}$ nên $y=f\left(|x|\right)$ là hàm số chẵn. \\
            Do đó, đồ thị hàm số $y=f\left(|x|\right)$ đối xứng qua trục tung.\\
            Suy ra hàm số $y=f\left(|x|\right)$ luôn có một điểm cực trị là $x=0$.\\
            Do đó, $y=f\left(|x|\right)$ có $5$ điểm cực trị $\Leftrightarrow$ hàm số $y=f(x)$ có $2$ điểm cực trị dương.\\
            \phantom{Do đó, số $y=f\left(|x|\right)$ có $5$ điểm cực trị} $\Leftrightarrow$ $f'(x)=0$ có hai nghiệm dương phân biệt.\\
            Ta có $f'(x)=3x^2-2(m-1)x+2-m$.\\
            Yêu cầu bài toán $\Leftrightarrow\heva{&\Delta'>0 \\ &S>0 \\ &P>0}\Leftrightarrow\heva{&4m^2-m-5>0 \\ &2m-1>0 \\ &2-m>0}\Leftrightarrow\heva{&m<-1\;\text{hoặc}\;m>\dfrac{5}{4} \\ &m>\dfrac{1}{2} \\ &m<3}\Leftrightarrow \dfrac{5}{4}<m<2$.
        \end{listEX}
    }
\end{vd}
\boxmini{BÀI TẬP TRẮC NGHIỆM}
\Opensolutionfile{ans}[ans/2D1-2-DANG-3]
\begin{ex}%[2D1K2-6]
    \immini
    {Cho hàm số $f(x)$ có đồ thị $f'(x)$ có đồ thị như hình vẽ bên dưới.\\ Hàm số $y=f(1-2x)$ có bao nhiêu cực trị ?
        \choice[2]
        {$4$}
        {$7$}
        {\True $3$}
        {$9$}
    }
    {
        \begin{tikzpicture}[>=stealth,font=\scriptsize]
            \begin{scope}[scale=0.55]
                \draw[->] (0,-1)--(0,3.5)node[right]{\scriptsize $y$};
                \draw[->] (-2,0)--(5,0)node[below]{\scriptsize $x$};
                \fill (0,0) node[below left]{\scriptsize  $O$} circle(1.5pt);
                \draw (-0.8,0) node[below left]{ $-1$} (0.9,0) node[below left]{ $1$} (2,0) node[below]{ $2$} (4,0) node[below]{ $4$};
                \clip (-2,-1) rectangle (5,3.5);
                \draw[] plot[smooth,tension=.65] coordinates{(-1.05,-0.9) (-0.3,2.5) (1.2,-0.5) (2.7,0.7) (4.2,0.2) (4.8,3.5)};
            \end{scope}
        \end{tikzpicture}
    }
    \loigiai{
        Đặt $g(x)=f(1-2x)$\\
        Dựa vào đồ thị, ta thấy $f'(x)=0$ có nghiệm $x_1=-1,x_2=1,x_3=2$ và $x_4=4$ nên $f'(x)$ có dạng $$f'(x)=k(x+1)(x-1)(x-2)(x-4)$$
        Khi đó $g'(x)=-2f'(1-2x)=-2k(2-2x)(-2x)(-1-2x)(-3-2x)^2$
        $$g'(x)=0 \Leftrightarrow \hoac{&x=1\\&x=0\\&x=-\dfrac{1}{2}\\&x=-\dfrac{3}{2} \text{ (kép)}}$$
        Bảng xét dấu $g'(x)$
        \begin{center}
            \begin{tikzpicture}[every node/.style={circle,fill=white,inner sep=0pt},arrow/.style={>=stealth,->,shorten <= 0.3cm,shorten >= 0.3cm},font=\footnotesize,xscale=1,yscale=1]
                \def\mnumline{1} %Số dòng
                \def\mnumcol{11} %Số cột
                \foreach \j in {0,...,\mnumline}
                \foreach \i in {0,...,\mnumcol}{
                    \coordinate (\j\i) at (\i,-\j);
                }
                \pgfmathsetmacro\yline{\mnumline/2-1}
                \path node at (00){$x$} node at (10){$g'(x)$};
                \foreach \x/\mnamex in {01/$-\infty$,03/$-\dfrac{3}{2}$,05/$-\dfrac{1}{2}$,07/$0$,09/$1$,0\mnumcol/$+\infty$} \path node at (\x) {\mnamex};
                \foreach \dy/\mnamedy in {12/$-$,13/$0$,14/$-$,15/$0$,16/$+$,17/$0$,18/$-$,19/$0$,110/$+$} \path node at (\dy) {\mnamedy};
                \draw[thick] (-.5,.5)rectangle([xshift=0.5cm,yshift=-0.5cm]\mnumline\mnumcol) ([xshift=-0.5cm,yshift=-0.5cm]00)--([xshift=0.5cm,yshift=-0.5cm]0\mnumcol)  ([xshift=0.5cm,yshift=0.5cm]00)--([xshift=0.5cm,yshift=-0.5cm]\mnumline0);
            \end{tikzpicture}
        \end{center}
        Dựa vào bảng xét dấu, ta thấy $g'(x)$ đổi dấu 3 lần nên $y=f(1-2x)$ có 3 cực trị.
    }
\end{ex}
\begin{ex}%[2D1K2-2]
    \immini{Cho hàm số $ f(x) $ có đạo hàm là $ f'(x) $. Đồ thị của hàm số $ y=f'(x) $ như hình vẽ bên. Khi đó hàm số $ y=f(x^2) $ có bao nhiêu điểm cực trị?
        \choice[2]
        {$2$}
        {$4$}
        {\True $3$}
        {$5$}}
    {
        \begin{tikzpicture}[line join=round, line cap=round,>=stealth,font=\scriptsize]
            \begin{scope}[scale=0.35]
                \tikzset{label style/.style={font=\footnotesize}}
                \def \xmin{-1.5}
                \def \xmax{6.5}
                \def \ymin{-2}
                \def \ymax{5.5}
                \def \hamso{-0.11*(\x)^3+1.09*(\x)^2-1.73*(\x)}
                \draw[->] (\xmin,0)--(\xmax,0) node[below left] {$x$};
                \draw[->] (0,\ymin)--(0,\ymax) node[below left] {$y$};
                \draw (0,0) node [below left] {$O$};
                \begin{scope}
                    \clip (\xmin+0.01,\ymin+0.01) rectangle (\xmax-0.01,\ymax-0.01);
                    \draw[samples=350,domain=-1.2:5.5,smooth,variable=\x] plot (\x,{\hamso});
                \end{scope}
                \draw [dashed] (5,4.6)--(5,0) node[below]{$5$} (2,0) node[below]{$2$};
            \end{scope}
        \end{tikzpicture}
    }
    \loigiai{$ y'=2xf'(x^2) $. Cho $ y'=0 \Leftrightarrow \hoac{&x=0\\&f'(x^2)=0} \Leftrightarrow \hoac{&x=0\\&x^2=0\\&x^2=2} \Leftrightarrow \hoac{&x=0\\&x=0 \text{ (nghiệm kép)}\\&x=\pm \sqrt{2}} $.\\
        $ y'=0 $ có 3 nghiệm bội bậc lẻ nên hàm số có 3 điểm cực trị.
    }
\end{ex}
\begin{ex}%[2D1K2-6]
    \immini{	Cho hàm số $y=f(x)$ xác định trên $\mathbb{R}$ và hàm số $y=f'(x)$ có đồ thị như hình vẽ. Hàm số $y=f(1-x^2)$ đạt cực đại tại điểm nào sau đây?
        \choice[2]
        {$x=-1$}
        {\True $x=\pm \sqrt{2}$}
        {$x=3$}
        {$x=0$}}{
        \begin{tikzpicture}[>=stealth, font=\scriptsize, line join=round, line cap=round,y=0.7cm]
            \begin{scope}[scale=.5]
                \def\a{1} \def\b{-2} \def\c{-2.5} % Hệ số
                \def\xmin{-2} \def\xmax{4}
                \def\ymin{-4} \def\ymax{1.5}
                %\draw[color=gray!50,dashed] (\xmin,\ymin) grid (\xmax,\ymax);
                \draw[->] (\xmin,0)--(\xmax,0);
                \draw[->] (0,\ymin)--(0,\ymax);
                \node at (0,0) [below right]{$O$};
                \node at (-1,0) [below left]{$-1$};
                \node at (3,0) [below right]{$3$};
                \clip (\xmin+0.1,\ymin+0.1) rectangle (\xmax-0.5,\ymax-0.1);
                \draw[smooth,samples=300,domain=-1.3:3.3] plot(\x,{\a*(\x)^2+\b*(\x)+\c});
            \end{scope}
    \end{tikzpicture}}
    \loigiai{Đặt $g(x)=f(1-x^2)$\\
        Khi đó $g'(x)=-2x\cdot f'(1-x^2)$\\
        Cho $g'(x)=0 \Leftrightarrow -2x \cdot f'(1-x^2) =0$
        $$ \Leftrightarrow \hoac{&x=0\\&f'(1-x^2)=0 \Leftrightarrow \hoac{&1-x^2=-1\Leftrightarrow x^2=2 \Leftrightarrow x=\pm \sqrt{2}\\&1-x^2=3}}$$
        Bảng xét dấu
        \begin{center}
            \begin{tikzpicture}[every node/.style={circle,fill=white,inner sep=0pt},arrow/.style={>=stealth,->,shorten <= 0.3cm,shorten >= 0.3cm},font=\footnotesize,xscale=1.4,yscale=.8]
                \def\mnumline{3} %Số dòng
                \def\mnumcol{9} %Số cột
                \foreach \j in {0,...,\mnumline}
                \foreach \i in {0,...,\mnumcol}{
                    \coordinate (\j\i) at (\i,-\j);
                    %	\draw[gray!30] ([xshift=-0.5cm,yshift=0.5cm]\j\i)--([xshift=0.5cm,yshift=0.5cm]\j\i)--([xshift=0.5cm,yshift=-0.5cm]\j\i)--([xshift=-0.5cm,yshift=-0.5cm]\j\i)--cycle (\j\i)node[]{\j\i}; %Ẩn lệnh này sau khi hoàn thành BBT
                }
                \pgfmathsetmacro\yline{\mnumline/2-1}
                \path node at (00){$x$} node at (10){$-x$} node at (20){\scriptsize $f'(1-x^2)$} node at (30){$g'(x)$};
                \foreach \x/\mnamex in {01/$-\infty$,03/$-\sqrt{2}$,05/$0$,07/$\sqrt{2}$,0\mnumcol/$+\infty$} \path node at (\x) {\mnamex};
                \foreach \dy/\mnamedy in {12/$-$,13/$0$,14/$+$,16/$+$} \path node at (\dy) {\mnamedy};
                \path node at ($(12)$){$+$} node at ($(13)$){$|$} node at ($(14)$){$+$} node at ($(15)$){$0$} node at ($(16)$){$-$} node at ($(17)$){$|$} node at ($(18)$){$-$} node at ($(22)$){$+$} node at ($(23)$){$0$} node at ($(24)$){$-$} node at ($(25)$){$|$} node at ($(26)$){$-$} node at ($(27)$){$0$} node at ($(28)$){$+$} node at ($(32)$){$+$} node at ($(33)$){$0$} node at ($(34)$){$-$} node at ($(35)$){$0$} node at ($(36)$){$+$} node at ($(37)$){$0$} node at ($(38)$){$-$};
                \draw[thick] (-.5,.5)rectangle([xshift=0.5cm,yshift=-0.5cm]\mnumline\mnumcol) ([xshift=-0.5cm,yshift=-0.5cm]00)--([xshift=0.5cm,yshift=-0.5cm]0\mnumcol) ([xshift=-0.5cm,yshift=-0.5cm]10)--([xshift=0.5cm,yshift=-0.5cm]1\mnumcol)

                ([xshift=-0.5cm,yshift=-0.5cm]20)--([xshift=0.5cm,yshift=-0.5cm]2\mnumcol)

                ([xshift=0.5cm,yshift=0.5cm]00)--([xshift=0.5cm,yshift=-0.5cm]\mnumline0); %Lệnh tự động kẻ bảng
            \end{tikzpicture}
        \end{center}
        Dựa vào bảng xét dấu ta xác định được hàm số đạt cực đại tại $x=\pm \sqrt{2}$.}
\end{ex}
\begin{ex}%[2D1K2-6]
    \immini{Cho hàm số $y=f(x)$ có đồ thị hàm $f'(x)=ax^2+bx+c$ như hình bên dưới. Hỏi hàm số $y=f(x-x^2)$ có bao nhiêu cực trị?
        \choice[2]
        {$0$}
        {\True $1$}
        {$2$}
        {$3$}}{
        \begin{tikzpicture}[>=stealth,x=1.2cm,y=0.7cm,font=\scriptsize]
            \begin{scope}[scale=0.35]
                \clip (-2,-2) rectangle (5,5.5);
                \def\a{1}
                \def\b{-3}
                \def\c{2}
                \draw[->] (-2,0) -- (4,0) node[below] { $x$};
                \draw[->] (0,-1) -- (0,5) node[left] {$y$};
                \draw (0,0)node[below left]{ $O$} circle(1.5pt);
                \draw (1,0) node[below]{$1$} (2,0) node[below]{  $2$} (0,2) node[left]{$2$};
                \pgfmathsetmacro\xdinh{-(\b)/2*(\a)}
                \pgfmathsetmacro\ydinh{(4*(\a)*(\c)-(\b)^2)/(4*(\a))}
                \draw[samples=150,smooth,domain=-5:5] plot(\x,{\a*(\x)^2+(\b)*\x+(\c)});
            \end{scope}
        \end{tikzpicture}
    }
    \loigiai{
        Đặt $g(x)=f\left(x-x^2\right)$\\
        Dựa vào đồ thị ta thấy $f'(x)=0$ có hai nghiệm $x_1=1,x_2=2$ nên $f'(x)$ có dạng $$f'(x)=k(x-1)(x-2)$$
        Khi đó $g'(x)=(1-2x)f'\left(x-x^2\right)=0$
        $$ \Leftrightarrow \hoac{&1-2x=0\\&f'\left(x-x^2\right)=0} \Leftrightarrow \hoac{&x=\dfrac{1}{2}\\&x-x^2=1\\&x-x^2=2} \Leftrightarrow \hoac{&x=\dfrac{1}{2}\\& \text{ vô nghiệm}\\&\text{ vô nghiệm.}}$$
        Bảng xét dấu
        \begin{center}
            \begin{tikzpicture}[every node/.style={circle,fill=white,inner sep=0pt},arrow/.style={>=stealth,->,shorten <= 0.3cm,shorten >= 0.3cm},font=\footnotesize,xscale=1,yscale=1]
                \def\mnumline{1} %Số dòng
                \def\mnumcol{5} %Số cột
                \foreach \j in {0,...,\mnumline}
                \foreach \i in {0,...,\mnumcol}{
                    \coordinate (\j\i) at (\i,-\j);
                }
                \pgfmathsetmacro\yline{\mnumline/2-1}
                \path node at (00){$x$} node at (10){$g'(x)$};
                \foreach \x/\mnamex in {01/$-\infty$,03/$\dfrac{1}{2}$,0\mnumcol/$+\infty$} \path node at (\x) {\mnamex};
                \foreach \dy/\mnamedy in {12/$+$,13/$0$,14/$-$} \path node at (\dy) {\mnamedy};
                \draw[thick] (-.5,.5)rectangle([xshift=0.5cm,yshift=-0.5cm]\mnumline\mnumcol) ([xshift=-0.5cm,yshift=-0.5cm]00)--([xshift=0.5cm,yshift=-0.5cm]0\mnumcol)  ([xshift=0.5cm,yshift=0.5cm]00)--([xshift=0.5cm,yshift=-0.5cm]\mnumline0);
            \end{tikzpicture}
        \end{center}
        Dựa vào bảng xét dấu, ta thấy $g(x)$ có 1 cực đại.
    }
\end{ex}
\begin{ex}%[2D1K2-2]
    \immini{Cho hàm số bậc bốn $y=f(x)$. Hàm số $y=f'(x)$
        có đồ thị như hình bên. Số điểm cực trị của hàm số $y=f\left(\sqrt{x^{2}+2 x+2}\right)$ là
        \choice[2]
        {$1$}
        {$2$}
        {$4$}
        {\True $3$}}
    {
        \begin{tikzpicture}[line join=round, line cap=round,>=stealth,font=\scriptsize]
            \begin{scope}[scale=0.5]
                \tikzset{label style/.style={font=\footnotesize}}
                \def \xmin{-2}
                \def \xmax{4.5}
                \def \ymin{-2}
                \def \ymax{3.5}
                \def \hamso{0.55*(\x)^3-1.76*(\x)^2-0.31*(\x)+2}
                \draw[->] (\xmin,0)--(\xmax,0) node[below left] {$x$};
                \draw[->] (0,\ymin)--(0,\ymax) node[below left] {$y$};
                \draw (0,0) node [below left] {$O$};
                \begin{scope}
                    \clip (\xmin+0.01,\ymin+0.01) rectangle (\xmax-0.01,\ymax-0.01);
                    \draw[samples=350,domain=-1.3:3.3,smooth,variable=\x] plot (\x,{\hamso});
                \end{scope}
                \draw (-1,0) node[below left]{$-1$} (1,0) node[below]{$1$} (3,0) node[below right]{$3$} (0,2) node[above left]{$2$};
            \end{scope}
        \end{tikzpicture}
    }
    \loigiai{
        $ y'=\dfrac{x+1}{\sqrt{x^2+2x+2}}f'(\sqrt{x^2+2x+2}) $.\\$ y'=0 \Leftrightarrow \hoac{&x=-1\\&f'(\sqrt{x^2+2x+2})=0} \Leftrightarrow \hoac{&x=-1\\&\sqrt{x^2+2x+2}=-1\\&\sqrt{x^2+2x+2}=1\\&\sqrt{x^2+2x+2}=3} \Leftrightarrow \hoac{&x=-1\\&x^2+2x+1=0\\&x^2+2x-7=0}\Leftrightarrow \hoac{&x=-1\\&x=-1 \text{ (nghiệm kép)}\\&x=-1\pm 2\sqrt{2}} $\\
        $ y'=0 $ có 3 nghiệm bội bậc lẻ nên hàm số có 3 điểm cực trị.
    }
\end{ex}
\begin{ex}%[2D1K2-2]
    \immini{Cho hàm số $ y=f(x) $ liên tục trên $ (a,b) $ và có đồ thị như hình bên. Số điểm cực trị của hàm số $ y=\left[f(x)\right]^2 $ trên $ (a;b) $ là
        \choice[2]
        {$4$}
        {$6$}
        {$2$}
        {\True $5$}}
    {
        \begin{tikzpicture}[line join=round, line cap=round,>=stealth,font=\scriptsize]
            \begin{scope}[scale=.35]
                \def \xmin{-3.5}
                \def \xmax{4.5}
                \def \ymin{-4}
                \def \ymax{3.5}
                \def \hamso{-0.37*(\x)^3+0.15*(\x)^2+2.41*(\x)-1}
                \draw[->] (\xmin,0)--(\xmax,0) node[below] {$x$};
                \draw[->] (0,\ymin)--(0,\ymax) node[left] {$y$};
                \draw (0,0) node [below left] {$O$};
                \clip (\xmin+0.01,\ymin+0.01) rectangle (\xmax-0.01,\ymax-0.01);
                \draw[samples=350,domain=-3:4,smooth,variable=\x] plot (\x,{\hamso});
                \draw[dashed] (-3,3.11)--(-3,0) node[below]{$a$} (3,-2.41)--(3,0) node[above]{$b$};
            \end{scope}
        \end{tikzpicture}
    }
    \loigiai{\immini{$ y=\left(f(x)\right)^2 $ nên $ y'=2f(x)f'(x) $.\\$ y'=0 \Leftrightarrow \hoac{&f(x)=0\\&f'(x)=0} \Leftrightarrow \hoac{&x=x_1,\ x=x_2,\ x=x_3\\&x=c,\ x=d}$.\\
            $ y'=0 $ có 5 nghiệm bội bậc lẻ thuộc $ (a,b) $ nên Số điểm cực trị của hàm số $ y=\left(f(x)\right)^2 $ trên $ (a;b) $ là 5.}
        {
            \begin{tikzpicture}[line join=round, line cap=round,>=stealth,thick,scale=0.8]
                \tikzset{label style/.style={font=\footnotesize}}
                \def \xmin{-3.5}
                \def \xmax{4.5}
                \def \ymin{-4}
                \def \ymax{3.5}
                \def \hamso{-0.37*(\x)^3+0.15*(\x)^2+2.41*(\x)-1}
                \draw[->] (\xmin,0)--(\xmax,0) node[below left] {$x$};
                \draw[->] (0,\ymin)--(0,\ymax) node[below left] {$y$};
                \draw (0,0) node [below left] {$O$};
                \begin{scope}
                    \clip (\xmin+0.01,\ymin+0.01) rectangle (\xmax-0.01,\ymax-0.01);
                    \draw[samples=350,domain=-3:4,smooth,variable=\x] plot (\x,{\hamso});
                \end{scope}
                \draw[dashed] (-3,3.11)--(-3,0) node[below]{\footnotesize $a$} (3,-2.41)--(3,0) node[above]{\footnotesize $b$} (2.54,0) node[below left]{\footnotesize $x_3$} (0.41,0) node[below right]{\footnotesize $x_2$} (-2.55,0) node[above right]{\footnotesize $x_1$} (-1.34,-3.07) -- (-1.34,0) node[above]{\footnotesize $c$} (1.61,1.72)--(1.61,0) node[below]{\footnotesize $d$};
            \end{tikzpicture}
        }
    }
\end{ex}
\begin{ex}%[2D1G2-1]
    \immini{Cho hàm số $y=f(x)$ có đạo hàm trên $\mathbb{R}$ và có bảng xét dấu $f'(x)$ như hình bên. Hàm số $y=f\left(x^{2}-2 x\right)$ có bao nhiêu điểm cực tiểu?
        \choice
        {\True $1$}
        {$2$}
        {$3$}
        {$4$}}{\begin{tikzpicture}
            \tkzTabInit[lgt=1,espcl=1.2]
            {$x$ /.7, $y'$ /.7}
            {$-\infty$,$-2$,$1$,$3$,$+\infty$}
            \tkzTabLine{ ,-,0,+,0,+,0,-, }
    \end{tikzpicture}}
    \loigiai{$ y'=(2x-2)f'(x^2-2x) $.
        \begin{eqnarray*}
            y'=0 	&\Leftrightarrow& \hoac{&x=1\\&f'(x^2-2x)=0}\\
            &\Leftrightarrow& \hoac{&x=1\\&x^2-2x=-2 \text{ (vô nghiệm)}\\&x^2-2x=1 \text{ (nghiệm bội bậc chẵn)}\\&x^2-2x=3} \\
            &\Leftrightarrow& \hoac{&x=1\\&x=1-\sqrt{2} \text{ (nghiệm bội bậc chẵn)}\\&x=1+\sqrt{2} \text{ (nghiệm bội bậc chẵn)}\\&x=3, \ x=-1.}
        \end{eqnarray*}
        $ y'=0 $ có 3 nghiệm bội bậc lẻ, khi đó $ y' $ đổi dấu qua các nghiệm này.\\
        $ y'=0 $ có 2 nghiệm bội bậc chẵn và $ y' $ sẽ không đổi dấu qua các nghiệm này.\\
        Tại $ x=4 $ thì $ y'(4)=(2\cdot 4 -2)f'(4^2-2\cdot 4)=6f'(8)<0 $.\\
        Bảng xét dấu
        \begin{center}
            \begin{tikzpicture}
                \tkzTabInit[lgt=1,espcl=1.2]
                {$x$ /1, $y'$ /1}
                {$-\infty$,$-1$,$1-\sqrt{2}$,$1+\sqrt{2}$,$3$,$+\infty$}
                \tkzTabLine{ ,-,0,+,0,+,0,+,0,-, }
            \end{tikzpicture}
        \end{center}
        Vậy hàm số có 1 điểm cực tiểu.
    }
\end{ex}
\begin{ex}%[2D1K2-6]
    \immini{Cho hàm số $f(x)$ có bảng biến thiên bên dưới. Trên khoảng $(-\sqrt{5};\sqrt{5})$ thì hàm số $y=f(x^2)$ đạt cực đại tại điểm nào sau đây?\choice
        {$x=\sqrt{2}$}
        {$x=-\sqrt{2}$}
        {\True $x=0$}
        {$x=2$}}{\begin{tikzpicture}
            \tkzTabInit[nocadre=false,lgt=1,espcl=1.6,deltacl=0.5]{$x$/.7 ,$f$/.7}
            {$-\infty$ , $0$ , $2$ , $+\infty$}
            \tkzTabLine{  , + , 0, - , 0 , +  }
    \end{tikzpicture}}
    \loigiai{Đặt $g(x)=f(x^2)$.\\
        Khi đó $g'(x)=2x \cdot f'(x^2)$.\\
        Cho $g'(x)=0 \Leftrightarrow 2x \cdot f'(x^2) =0 \Leftrightarrow
        \hoac{&x=0\\&f'(x^2)=0 \Leftrightarrow \hoac{x^2=0\\x^2=2} \Leftrightarrow \hoac{x=0\\x=\pm \sqrt{2}}}$\\
        Bảng xét dấu
        \begin{center}
            \begin{tikzpicture}[every node/.style={circle,fill=white,inner sep=0pt},arrow/.style={>=stealth,->,shorten <= 0.3cm,shorten >= 0.3cm},font=\footnotesize,xscale=1,yscale=.7]
                \def\mnumline{3} %Số dòng
                \def\mnumcol{9} %Số cột
                \foreach \j in {0,...,\mnumline}
                \foreach \i in {0,...,\mnumcol}{
                    \coordinate (\j\i) at (\i,-\j);
                }
                \pgfmathsetmacro\yline{\mnumline/2-1}
                \path node at (00){$x$} node at (10){$x$} node at (20){$f'(x^2)$} node at (30){$g'(x)$};
                \foreach \x/\mnamex in {01/$-\sqrt{5}$,03/$-\sqrt{2}$,05/$0$,07/$\sqrt{2}$,0\mnumcol/$\sqrt{5}$} \path node at (\x) {\mnamex};
                \foreach \dy/\mnamedy in {12/$-$,13/$0$,14/$+$,16/$+$} \path node at (\dy) {\mnamedy};
                \path node at ($(12)$){$-$} node at ($(13)$){$|$} node at ($(14)$){$-$} node at ($(15)$){$0$} node at ($(16)$){$+$} node at ($(17)$){$|$} node at ($(18)$){$+$} node at ($(22)$){$+$} node at ($(23)$){$0$} node at ($(24)$){$-$} node at ($(25)$){$0$} node at ($(26)$){$-$} node at ($(27)$){$0$} node at ($(28)$){$+$} node at ($(32)$){$-$} node at ($(33)$){$0$} node at ($(34)$){$+$} node at ($(35)$){$0$} node at ($(36)$){$-$} node at ($(37)$){$0$} node at ($(38)$){$+$};
                \draw[thick] (-.5,.5)rectangle([xshift=0.5cm,yshift=-0.5cm]\mnumline\mnumcol) ([xshift=-0.5cm,yshift=-0.5cm]00)--([xshift=0.5cm,yshift=-0.5cm]0\mnumcol) ([xshift=-0.5cm,yshift=-0.5cm]10)--([xshift=0.5cm,yshift=-0.5cm]1\mnumcol)

                ([xshift=-0.5cm,yshift=-0.5cm]20)--([xshift=0.5cm,yshift=-0.5cm]2\mnumcol)

                ([xshift=0.5cm,yshift=0.5cm]00)--([xshift=0.5cm,yshift=-0.5cm]\mnumline0); %Lệnh tự động kẻ bảng
            \end{tikzpicture}
        \end{center}
        Dựa vào bảng xét dấu ta xác định được hàm số đạt cực đại tại $x=0$.
    }
\end{ex}
\begin{ex}%[2D1K2-6]
    \immini{Cho hàm số $f(x)$ có bảng biến thiên bên dưới. Hàm số $y=f(x^2-2)$ đạt cực đại tại điểm nào sau đây?
        \choice
        {$x=-2$}
        {$x=-1$}
        {\True $x=0$}
        {$x=2$}}{\begin{tikzpicture}
            \tkzTabInit[nocadre=false,lgt=1,espcl=1.6,deltacl=0.5]{$x$/.7 ,$f$/.7}
            {$-\infty$ , $-1$ , $2$ , $+\infty$}
            \tkzTabLine{  , - , 0, - , 0 , +  }
    \end{tikzpicture}}
    \loigiai{Đặt $g(x)=f(x^2-2)$\\
        Khi đó $g'(x)=2x \cdot f'(x^2-2)$\\
        Cho $g'(x)=0 \Leftrightarrow 2x \cdot f'(x^2-2) =0$
        $$ \Leftrightarrow \hoac{&x=0\\&f'(x^2-2)=0 \Leftrightarrow \hoac{&x^2-2=-1\\&x^2-2=2} \Leftrightarrow \hoac{x^2=1\\x^2=4} \Leftrightarrow \hoac{x=\pm 1\\x=\pm 2}}$$
        Bảng xét dấu
        \begin{center}
            \begin{tikzpicture}[every node/.style={circle,fill=white,inner sep=0pt},arrow/.style={>=stealth,->,shorten <= 0.3cm,shorten >= 0.3cm},font=\footnotesize,xscale=1,yscale=1]
                \def\mnumline{3} %Số dòng
                \def\mnumcol{14} %Số cột
                \foreach \j in {0,...,\mnumline}
                \foreach \i in {0,...,\mnumcol}{
                    \coordinate (\j\i) at (\i,-\j);
                }
                \pgfmathsetmacro\yline{\mnumline/2-1}
                \path node at ([xshift=0.5cm]00){$x$} node at ([xshift=0.5cm]10){$x$}  node at ([xshift=0.5cm]20){$f'\left(x^2-2\right)$} node at ([xshift=0.5cm]\mnumline0){$g'(x)$};
                \foreach \x/\mnamex in {02/$-\infty$,04/$-2$,06/$-1$,08/$0$,010/$1$,012/$2$,0\mnumcol/$+\infty$} \path node at (\x) {\mnamex};
                \foreach \dy/\mnamedy in {13/$-$,14/$0$,15/$+$,16/$+$} \path node at (\dy) {\mnamedy};
                \path node at ($(13)$){$-$} node at ($(14)$){$|$} node at ($(15)$){$-$} node at ($(16)$){$|$} node at ($(17)$){$-$} node at ($(18)$){$0$} node at ($(19)$){$+$} node at ($(110)$){$|$} node at ($(111)$){$+$} node at ($(112)$){$|$} node at ($(113)$){$+$}
                node at ($(23)$){$+$} node at ($(24)$){$0$} node at ($(25)$){$-$} node at ($(26)$){$0$} node at ($(27)$){$-$} node at ($(28)$){$|$} node at ($(29)$){$-$} node at ($(210)$){$0$} node at ($(211)$){$-$} node at ($(212)$){$0$} node at ($(213)$){$+$}
                node at ($(33)$){$-$} node at ($(34)$){$0$} node at ($(35)$){$+$} node at ($(36)$){$0$} node at ($(37)$){$+$} node at ($(38)$){$0$} node at ($(39)$){$-$} node at ($(310)$){$0$} node at ($(311)$){$-$} node at ($(312)$){$0$} node at ($(313)$){$+$};
                \draw[thick] (-.5,.5)rectangle([xshift=0.5cm,yshift=-0.5cm]\mnumline\mnumcol) ([xshift=-0.5cm,yshift=-0.5cm]00)--([xshift=0.5cm,yshift=-0.5cm]0\mnumcol)
                ([xshift=-0.5cm,yshift=-0.5cm]20)--([xshift=0.5cm,yshift=-0.5cm]2\mnumcol)
                ([xshift=-0.5cm,yshift=-0.5cm]10)--([xshift=0.5cm,yshift=-0.5cm]1\mnumcol) ([xshift=0.5cm,yshift=0.5cm]01)--([xshift=0.5cm,yshift=-0.5cm]\mnumline1); %Lệnh tự động kẻ bảng
            \end{tikzpicture}
        \end{center}
        Dựa vào bảng xét dấu ta xác định được hàm số đạt cực đại tại $x=0$.
    }
\end{ex}
\begin{ex}%[2D1K2-1]
    Cho hàm số $ y=f(x) $ có đạo hàm $ f'(x)=x^2(x-1)(x-4)^2 $. Khi đó hàm số $ y=f(x^2) $ có bao nhiêu điểm cực trị?
    \choice
    {$4$}
    {\True $3$}
    {$5$}
    {$2$}
    \loigiai{$ f'(x)=0 \Leftrightarrow x=1 $ (nghiệm đơn), $ x=0 $ (nghiệm kép), $ x=4 $ (nghiệm kép).\\
        $ y=f(x^2) $ thì $ y'=2xf'(x^2) $.\\$y'=0 \Leftrightarrow \hoac{&x=0\\&x^2=1\\&x^2=0 \text{ (nghiệm kép)}\\&x^2=4 \text{ (nghiệm kép)}} \Leftrightarrow \hoac{&x=0\\&x=\pm 1\\&x=0 \text{ (nghiệm bội chẵn)}\\&x=\pm 2 \text{ (nghiệm bội chẵn).}} $\\
        Vậy hàm số có 3 điểm cực trị.
    }
\end{ex}
\begin{ex}%[2D1K2-6]
    Cho hàm $f(x)$ có đạo hàm $f'(x)=x^2-2x,\forall x\in \mathbb{R}$. Hàm số $y=f\left(1-\dfrac{1}{2}x\right)+4x$ có bao nhiêu điểm cực trị?
    \choice
    {0}
    {1}
    {\True 2}
    {3}
    \loigiai{Ta có $y'=-\dfrac{1}{2}f'\left(1-\dfrac{1}{2}x\right)+4$\\
        $y'=0 \Leftrightarrow
        f'\left(1-\dfrac{1}{2}x\right)=8\Leftrightarrow \left(1-\dfrac{1}{2}x\right)^2-2\left(1-\dfrac{1}{2}x\right)=8 \Leftrightarrow \dfrac{1}{4}x^2-9=0
        \Leftrightarrow \hoac{&x=-6\\&x=6}$\\
        Bảng xét dấu
        \begin{center}
            \begin{tikzpicture}[every node/.style={circle,fill=white,inner sep=0pt},arrow/.style={>=stealth,->,shorten <= 0.3cm,shorten >= 0.3cm},font=\footnotesize,xscale=1,yscale=1]
                \def\mnumline{1} %Số dòng
                \def\mnumcol{7} %Số cột
                \foreach \j in {0,...,\mnumline}
                \foreach \i in {0,...,\mnumcol}{
                }
                \pgfmathsetmacro\yline{\mnumline/2-1}
                \path node at (00){$x$} node at (10){$y'$};
                \foreach \x/\mnamex in {01/$-\infty$,03/$-6$,05/$6$,0\mnumcol/$+\infty$} \path node at (\x) {\mnamex};
                \foreach \dy/\mnamedy in {12/$+$,13/$0$,14/$-$,15/$0$,16/$+$} \path node at (\dy) {\mnamedy};
                \draw[thick] (-.5,.5)rectangle([xshift=0.5cm,yshift=-0.5cm]\mnumline\mnumcol) ([xshift=-0.5cm,yshift=-0.5cm]00)--([xshift=0.5cm,yshift=-0.5cm]0\mnumcol)  ([xshift=0.5cm,yshift=0.5cm]00)--([xshift=0.5cm,yshift=-0.5cm]\mnumline0); %Lệnh tự động kẻ bảng
            \end{tikzpicture}
        \end{center}
        Vậy hàm số $y=f\left(1-\dfrac{1}{2}x\right)+4x$ có 2 điểm cực trị.}
\end{ex}
\begin{ex}%[2D1G2-1]
    Cho hàm số $ y=f(x) $ có đạo hàm $ f'(x)=(x-1)^2(x^2-2x) $, với mọi $ x \in \mathbb{R} $. Có bao nhiêu giá trị nguyên dương của tham số $m$ để hàm số $ y=f(x^2-8x+m) $ có 5 điểm cực trị?
    \choice
    {\True $15$}
    {$16$}
    {$17$}
    {$18$}
    \loigiai{$ f'(x)=0 \Leftrightarrow x=1 $ (nghiệm kép), $ x=0 $ (nghiệm đơn), $ x=2 $ (nghiệm đơn).\\
        $ y=f(x^2-8x+m) $ thì $ y'=(2x-8)f'(x^2-8x+m) $.\\$y'=0 \Leftrightarrow \hoac{&x=4\\&x^2-8x+m=1 \text{ (nghiệm kép)}\\&x^2-8x+m=0 \quad (1)\\&x^2-8x+m=2 \quad (2)} $.\\
        Hàm số có 5 điểm cực trị $ \Leftrightarrow (1) $ có 2 nghiệm phân biệt khác 4 và $ (2) $ có 2 nghiệm phân biệt khác 4.\\$(1) $ có 2 nghiệm phân biệt khác 4 $ \Leftrightarrow \heva{&16-32+m \ne 0\\&\Delta'=16-m>0} \Leftrightarrow \heva{&m \ne 16\\&m<16}\Leftrightarrow m<16$.\\
        $(2) $ có 2 nghiệm phân biệt khác 4 $ \Leftrightarrow \heva{&16-32+m \ne 2\\&\Delta'=16-m+2>0} \Leftrightarrow \heva{&m \ne 18\\&m<18} \Leftrightarrow m<18$.\\
        Vậy ta có $ m<16 $ mà $ m $ nguyên dương nên $ m \in \{1,2,\cdots,15\} $ (15 số $ m $ thỏa mãn).
    }
\end{ex}
\begin{ex}%[2D1Y2-2]
    \immini
    {
        Cho hàm số $y=f(x)$ có đạo hàm liên tục trên $\mathbb{R}$. Đồ thị hàm số $y=f'(x)$ như hình vẽ bên. Số điểm cực trị của hàm số $y=f(x)-5x$ là
        \choice
        {$2$}
        {$3$}
        {$4$}
        {\True $1$}
    }
    {
        \begin{tikzpicture}[font=\scriptsize, line join=round, line cap=round, >=stealth,y=.8cm]
            \begin{scope}[scale=.6]
                \draw[->,>=latex](-3,0)--(3,0)node[above]{$x$};
                \draw[->,>=latex](0,-1)--(0,5)node[right]{$y$};
                \node[above left] at (0,0){$O$};
                \draw plot [samples=100,domain=-2.1:2.1] (\x,{(\x)^3-3*(\x)+2});
                \foreach\i in{-1,1}{\node[below] at (\i,0){$\i$};}
                \foreach\i in{4,2}{\node[right] at (0,\i){$\i$};}
                \draw[dashed](-1,0)--(-1,4)--(0,4);
            \end{scope}
        \end{tikzpicture}
    }
    \loigiai
    {
        Gọi $g(x)=f(x)-5x$. Ta có đạo hàm $g'(x)=f'(x)-5$. Bảng biến thiên của $g'(x)$ như hình dưới.
        \begin{center}
            \begin{tikzpicture}
                \tkzTabInit[nocadre=false,lgt=1.2,espcl=2.5,deltacl=0.6]
                {$x$/1, $f'(x)$/2, $g'(x)$/2}
                {$-\infty$, $-1$, $1$, $+\infty$}
                \tkzTabVar{-/ $-\infty$, +/$4$, -/$0$, +/$+\infty$}
                \tkzTabVar{-/ $-\infty$, +/$-1$, -/$-5$, +/$+\infty$}
            \end{tikzpicture}
        \end{center}
        Ta thấy $g'(x)$ chỉ đổi dấu một lần từ âm sang dương.\\
        Vì vậy hàm số $y=f(x)-5x$ có một điểm cực trị.
    }
\end{ex}
\begin{ex}%[2D1Y2-2]
    \immini
    {
        Cho hàm số $y=f(x)$ có đạo hàm trên $\mathbb{R}$. Biết hàm số $y=f'(x)$ có đồ thị như hình vẽ. Khẳng định nào sau đây đúng về cực trị của hàm số $g(x)=f(x)+x$?
        \choice
        {Hàm số có một điểm cực đại và một điểm cực tiểu}
        {Hàm số không có điểm cực đại và một điểm cực tiểu}
        {Hàm số có một điểm cực đại và hai điểm cực tiểu}
        {\True Hàm số có hai điểm cực đại và một điểm cực tiểu}
    }
    {
        \begin{tikzpicture}[ font=\scriptsize, line join=round, line cap=round, >=stealth]
            \begin{scope}[scale=.5]
                \foreach\x in{-1,0,...,3}{\draw[color=gray!30](\x,-3)--(\x,3.3);}
                \foreach\y in{-2,-1,...,3}{\draw[color=gray!30](-2,\y)--(4,\y);}
                \draw[->,>=latex](-2,0)--(4,0)node[above]{$x$};
                \draw[->,>=latex](0,-3)--(0,3.3)node[right]{$y$};
                \node[above left] at (0,0){$O$};
                \draw plot [samples=100,domain=-1.12:3.1] (\x,{-(\x)^3+3*(\x)^2-2});
            \end{scope}
        \end{tikzpicture}
    }
    \loigiai
    {
        Ta có $g'(x)=f'(x)+1$. Bảng biến thiên của $g'(x)$ như hình dưới.
        \begin{center}
            \begin{tikzpicture}
                \tkzTabInit[nocadre=false,lgt=1.2,espcl=2.5,deltacl=0.6]
                {$x$/1, $f'(x)$/2, $g'(x)$/2}
                {$-\infty$, $0$, $2$, $+\infty$}
                \tkzTabVar{+/ $+\infty$, -/$-2$, +/$2$, -/$-\infty$}
                \tkzTabVar{+/ $+\infty$, -/$-1$, +/$3$, -/$-\infty$}
            \end{tikzpicture}
        \end{center}
        Dựa vào bảng biến thiên của $g'(x)$, ta thấy đạo hàm đổi dấu từ dương sang âm hai lần, từ âm sang dương một lần.\\
        Do đó hàm số $g(x)$ có hai điểm cực đại và một điểm cực tiểu.
    }
\end{ex}
\begin{ex}%[2D1K2-6]
    \immini{	Cho hàm số $y=f(x)$ có đạo hàm trên $\mathbb{R}$ và có đồ thị hàm số $f'(x)$ như hình vẽ. Hàm số $y=2f(x)+x^2$ đạt cực đại tại điểm nào sau đây ?
        \choice[2]
        {\True $x=-1$}
        {$x=0$}
        {$x=1$}
        {$x=2$}}{\begin{tikzpicture}[>=stealth,font=\scriptsize,x=1.3cm]
            \begin{scope}[scale=.7]
                \draw[->] (-2,0) -- (3,0) node[below] {\scriptsize $x$};
                \draw[->] (0,-3) -- (0,2.5) node[left] { $y$};
                \draw (0,0)node[below left]{$O$} (-1.2,0) node[below]{ $-1$} (0,-2) node[below right]{ $-2$} (0,1) node[above left]{$1$} (0,-1) node[below right]{  $-1$} (1,0) node[above]{$1$} (2,0) node[above]{ $2$};
                \draw plot[smooth,tension=.65] coordinates{(-1.05,1.7) (-0.5,-2.6) (0.17,0.5) (0.9,-0.9) (1.5,-1.1) (2.1,-1.9) (2.3,2)};
                \draw[dashed] (-1,0) -- (-1,1) -- (0,1) (1,0)--(1,-1)--(0,-1) (2,0)--(2,-2)--(0,-2);
            \end{scope}
    \end{tikzpicture}}
    \loigiai{
        Đặt $g(x)=2f(x)+x^2$\\
        Khi đó $g'(x)=2f'(x)+2x=0 \Leftrightarrow 2\left(f'(x)+x\right)=0 \Leftrightarrow f'(x)=-x \quad (*)$\\
        Số nghiệm của phương trình $(*)$ là số giao điểm của đồ thị hàm số $y=f'(x)$ và $y=-x$\\
        Dựa vào hình bên ta thấy có $4$ giao điểm lần lượt có tọa độ là $(-1;1),(0;0),(1;-1)$ và $(2;-2)$ \\ $ \Rightarrow (*)  \Leftrightarrow \hoac{&x=-1 \quad \text{(đơn)}\\&x=0 \quad \text{(đơn)}\\&x=1 \quad \text{(kép)}\\&x=2 \quad \text{(kép)}.}$\\
        Bảng xét dấu
        \begin{center}
            \begin{tikzpicture}[every node/.style={circle,fill=white,inner sep=0pt},arrow/.style={>=stealth,->,shorten <= 0.3cm,shorten >= 0.3cm},font=\footnotesize,xscale=1,yscale=1]
                \def\mnumline{1} %Số dòng
                \def\mnumcol{11} %Số cột
                \foreach \j in {0,...,\mnumline}
                \foreach \i in {0,...,\mnumcol}{
                    \coordinate (\j\i) at (\i,-\j);
                }
                \pgfmathsetmacro\yline{\mnumline/2-1}
                \path node at (00){$x$} node at (10){$g'(x)$};
                \foreach \x/\mnamex in {01/$-\infty$,03/$-1$,05/$0$,07/$1$,09/$2$,0\mnumcol/$+\infty$} \path node at (\x) {\mnamex};
                \foreach \dy/\mnamedy in {12/$+$,13/$0$,14/$-$,15/$0$,16/$+$,17/$0$,18/$+$,19/$0$,110/$+$} \path node at (\dy) {\mnamedy};
                \draw[thick] (-.5,.5)rectangle([xshift=0.5cm,yshift=-0.5cm]\mnumline\mnumcol) ([xshift=-0.5cm,yshift=-0.5cm]00)--([xshift=0.5cm,yshift=-0.5cm]0\mnumcol)  ([xshift=0.5cm,yshift=0.5cm]00)--([xshift=0.5cm,yshift=-0.5cm]\mnumline0);
            \end{tikzpicture}
        \end{center}
        Dựa vào bảng xét dấu, ta thấy $g(x)$ đạt cực đại tại $x=-1$.
    }
\end{ex}
\begin{ex}%[2D1K2-6]
    \immini{Hàm số $y=f(x)$ liên tục trên $\mathbb{R}$ và có đồ thị hàm số $f'(x)$ như hình vẽ bên dưới. Hàm số $y=f(x)-\dfrac{1}{3}x^3+x^2-x+2$ đạt cực đại tại điểm nào sau đây ?
        \choice[2]
        {\True $x=1$}
        {$x=-1$}
        {$x=0$}
        {$x=2$}}{\begin{tikzpicture}[>=stealth,font=\scriptsize,y=.7cm]
            \begin{scope}[scale=.8]
                \draw[->] (-2,0) -- (3,0) node[below] {\scriptsize $x$};
                \draw[->] (0,-3) -- (0,2.5) node[left] {\scriptsize $y$};
                \draw (0,0)node[below left]{ $O$}  (-1.2,0) node[below left]{ $-1$} (0,-2) node[right]{ $-2$} (0,1) node[above left]{  $1$} (1,0) node[below]{ $1$} (2,0) node[below]{ $2$};
                \draw plot [samples=100,domain=-1.1:2.2] (\x,{(\x)^3-2*(\x)^2+1});
                \draw[dashed] (-1,0) -- (-1,-2) -- (0,-2) (0,1)--(2,1)--(2,0);
            \end{scope}
    \end{tikzpicture}}
    \loigiai{
        Đặt $g(x)=f(x)-\dfrac{1}{3}x^3+x^2-x+2$	\\
        Khi đó $g'(x)=f'(x)-x^2+2x-1$.\\
        $g'(x)=0 \Leftrightarrow f'(x)=x^2-2x+1 \quad (*)$
        \immini{Số nghiệm của $(*)$ cũng chính là số giao điểm của đồ thị hàm số $y=f'(x)$ với $y=x^2-2x+1$\\
            Dựa vào hình vẽ bên, ta thấy có $3$ giao điểm lần lượt có tọa độ là $(1;0),(2;1),(0;1)$. Khi đó,
            $(*) \Leftrightarrow \hoac{&x=1\\&x=0\\&x=2.}$
        }
        {\begin{tikzpicture}[>=stealth,x=1.0cm,y=1.0cm,scale=0.6]
                \draw[->] (-2,0) -- (3,0) node[below] {\scriptsize $x$};
                \draw[->] (0,-3) -- (0,2.5) node[left] {\scriptsize $y$};
                \draw (0,0)node[below right]{\scriptsize $O$} circle(1.5pt) (-1.2,0) node[below]{\scriptsize $-1$} (0,-2) node[right]{\scriptsize  $-2$} (0,1) node[left]{\scriptsize  $1$} (1,0) node[below]{\scriptsize  $1$} (2,0) node[below]{\scriptsize  $2$};
                \def\a{1}
                \def\b{-2}
                \def\c{1}
                \pgfmathsetmacro\xdinh{-(\b)/2*(\a)}
                \pgfmathsetmacro\ydinh{(4*(\a)*(\c)-(\b)^2)/(4*(\a))}
                \fill[dashed] (\xdinh,\ydinh)circle(2pt) edge (\xdinh,0) edge (0,\ydinh);
                \clip (-2,-3)rectangle(3,3);
                \draw[thick,samples=150,smooth,domain=-5:5] plot(\x,{\a*(\x)^2+(\b)*\x+(\c)});
                \draw[thick] plot[smooth,tension=.65] coordinates{(-1.1,-2.2) (-0.1,1) (1.4,-0.2) (2.5,2.8)};
                \draw[dashed] (-1,0) -- (-1,-2) -- (0,-2) (0,1)--(2,1)--(2,0);
        \end{tikzpicture}}
        \noindent
        Bảng xét dấu
        \begin{center}
            \begin{tikzpicture}[every node/.style={circle,fill=white,inner sep=0pt},arrow/.style={>=stealth,->,shorten <= 0.3cm,shorten >= 0.3cm},font=\footnotesize,xscale=1,yscale=1]
                \def\mnumline{1} %Số dòng
                \def\mnumcol{9} %Số cột
                \foreach \j in {0,...,\mnumline}
                \foreach \i in {0,...,\mnumcol}{
                    \coordinate (\j\i) at (\i,-\j);
                }
                \pgfmathsetmacro\yline{\mnumline/2-1}
                \path node at (00){$x$} node at (10){$g'(x)$};
                \foreach \x/\mnamex in {01/$-\infty$,03/$0$,05/$1$,07/$2$,0\mnumcol/$+\infty$} \path node at (\x) {\mnamex};
                \foreach \dy/\mnamedy in {12/$-$,13/$0$,14/$+$,15/$0$,16/$-$,17/$0$,18/$+$} \path node at (\dy) {\mnamedy};
                \draw[thick] (-.5,.5)rectangle([xshift=0.5cm,yshift=-0.5cm]\mnumline\mnumcol) ([xshift=-0.5cm,yshift=-0.5cm]00)--([xshift=0.5cm,yshift=-0.5cm]0\mnumcol)  ([xshift=0.5cm,yshift=0.5cm]00)--([xshift=0.5cm,yshift=-0.5cm]\mnumline0);
            \end{tikzpicture}
        \end{center}
        Hàm số đạt cực đại tại $x=1$.
    }
\end{ex}
\begin{ex}%[2D1G2-6]
    \immini{	Cho hàm số $f(x)$ có đạo hàm liên tục trên $\mathbb{R}$ và đồ thị $y=f'(x)$ như hình vẽ dưới đây. Xét trên khoảng $(-\pi;2\pi)$, số điểm cực trị của hàm số $g(x)=f(2\cos x)+2\cos2x$ là
        \choice[2]
        {$13$}
        {$10$}
        {\True $11$}
        {$9$}}{\begin{tikzpicture}[>=stealth,font=\scriptsize,x=1.3cm]
            \begin{scope}[scale=.5]
                \draw[->] (-2.5,0) -- (2.5,0) node[below] { $x$};
                \draw[->] (0,-2.5) -- (0,2.5) node[left] {$y$};
                \draw (0,0)node[below left]{ $O$};
                \draw (0,-2)node[below right]{$-2$} (0,2)node[above right]{\scriptsize $2$} (1,0) node[above]{ $1$} (-2,0) node[above]{ $-2$} (-1,0) node[below]{ $-1$} (2,0) node[below]{$2$};
                \draw[dashed] (-2,0)--(-2,-2)--(1,-2)--(1,0) (-1,0)--(-1,2)--(2,2)--(2,0);
                \clip (-2.5,-2.5)rectangle(2.5,3);
                \draw[samples=150,smooth,domain=-2.1:2.1] plot(\x,{(\x)^3-3*\x});

                \fill[black] (-2,0) circle(1.5pt) (-1,0) circle(1.5pt) (1,0) circle(1.5pt) (2,0) circle(1.5pt)(-2,-2) circle(1.5pt)(0,-2) circle(1.5pt)(1,-2) circle(1.5pt)(-1,2) circle(1.5pt)(0,2) circle(1.5pt)(2,2) circle(1.5pt);
            \end{scope}
    \end{tikzpicture}}
    \loigiai{
        Ta có $g'(x)=f'(2\cos x)\cdot(-2\sin x)-2\sin{2x}\cdot2=-2\sin{x}\left[f'(2\cos x)+4\cos x\right]$.\\
        Suy ra $g'(x)=0 \Leftrightarrow \hoac{&\sin x=0\\&f'(2\cos x)=-4\cos x.}$\\
        \begin{itemize}
            \item $\sin x=0 \Leftrightarrow x\in\{0;\pi\}$ vì $x\in(-\pi;2\pi)$.
            \item $f'(2\cos x)=-4\cos x$.\\
            Đặt $t=2\cos x$, vì $x\in(-\pi;2\pi)$ nên $t\in(-1;1)$.\\
            Phương trình trở thành $f'(t)=-2t$. Nghiệm của phương trình này là hoành độ giao điểm của đồ thị hàm số $y=f'(t)$ và đường thẳng $y=-2t$.\\
            \begin{center}
                \begin{tikzpicture}[>=stealth,x=1cm,y=1cm,scale=1]
                    \draw[->] (-2.5,0) -- (2.5,0) node[below] {\scriptsize $t$};
                    \draw[->] (0,-2.5) -- (0,2.5) node[left] {\scriptsize $y$};
                    \draw (0,0)node[below left]{\scriptsize $O$};
                    \draw (0,-2)node[below right]{\scriptsize $-2$} (0,2)node[above right]{\scriptsize $2$} (1,0) node[above]{\scriptsize $1$} (-2,0) node[above]{\scriptsize $-2$} (-1,0) node[below]{\scriptsize $-1$} (2,0) node[below]{\scriptsize $2$};
                    \draw[dashed] (-2,0)--(-2,-2)--(1,-2)--(1,0) (-1,0)--(-1,2)--(2,2)--(2,0);
                    \clip (-2.5,-2.5)rectangle(2.5,2.5);
                    \draw[thick,samples=150,smooth,domain=-2.1:2.1] plot(\x,{(\x)^3-3*\x}) node[right]{$(l)$};
                    \node[above left] at (2,2){\scriptsize $y=f'(t)$};
                    \fill[black] (-2,0) circle(1.5pt) (-1,0) circle(1.5pt) (1,0) circle(1.5pt) (2,0) circle(1.5pt)(-2,-2) circle(1.5pt)(0,-2) circle(1.5pt)(1,-2) circle(1.5pt)(-1,2) circle(1.5pt)(0,2) circle(1.5pt)(2,2) circle(1.5pt);
                    \draw[thick,samples=150,smooth,domain=-2.1:2.1] plot(\x,{-2*(\x)});
                \end{tikzpicture}
            \end{center}
            Suy ra $f'(t)=-2t \Leftrightarrow \hoac{&t=-1\\&t=0\\&t=1.}$

            \begin{itemize}
                \item Với $t=-1 \Rightarrow 2\cos x=-1 \Leftrightarrow \cos x=-\dfrac{1}{2} \Leftrightarrow x\in\left\{-\dfrac{2\pi}{3};\dfrac{2\pi}{3};\dfrac{4\pi}{3}
                \right\}$ vì $x\in(-\pi;2\pi)$.
                \item Với $t=0 \Rightarrow \cos x=0 \Leftrightarrow x\in\left\{-\dfrac{\pi}{2};\dfrac{\pi}{2};\dfrac{3\pi}{2}
                \right\}$ vì $x\in(-\pi;2\pi)$.
                \item Với $t=1 \Rightarrow 2\cos x=1 \Leftrightarrow \cos x=\dfrac{1}{2} \Leftrightarrow x\in\left\{-\dfrac{\pi}{3};\dfrac{\pi}{3};\dfrac{5\pi}{3}
                \right\}$ vì $x\in(-\pi;2\pi)$.
            \end{itemize}
        \end{itemize}
        Và
        \begin{itemize}
            \item $f'(t)+2t>0\Leftrightarrow f'(t)>-2t\Leftrightarrow \hoac{&-1<t<0\\&t>1}\\
            \Rightarrow \hoac{&-\dfrac{1}{2}<\cos x<0\\&\cos x>\dfrac{1}{2}}\Leftrightarrow \hoac{&-\dfrac{2\pi}{3}<x<-\dfrac{\pi}{3}\\&\dfrac{4\pi}{3}<x<\dfrac{5\pi}{3}\\&\dfrac{\pi}{3}<x<\dfrac{2\pi}{3}}$ (vì $x\in(-\pi;2\pi)$).
            \item $f'(t)+2t<0\Leftrightarrow f'(t)<-2t\Leftrightarrow \hoac{&t<-1\\&0<t<1}\\
            \Rightarrow \hoac{&\cos x<-\dfrac{1}{2}\\&0<\cos x<\dfrac{1}{2}} \Leftrightarrow \hoac{&-\pi<x<-\dfrac{2\pi}{3}\\&-\dfrac{\pi}{3}<x<\dfrac{\pi}{3}\\&\dfrac{2\pi}{3}<x<\dfrac{4\pi}{3}}$ (vì $x\in(-\pi;2\pi)$).
        \end{itemize}
        Bảng biến thiên hàm số $y=g(x)$
        \begin{center}
            \begin{tikzpicture}
                \tkzTabInit[nocadre=false,lgt=4,espcl=1]
                {$x$ /1.1,$-2\sin x$ /0.7,$f'(2\cos x)+4\cos x$ /0.7,$g'(x)$ /0.7,$g(x)$ /2}
                {$-\pi$,$-\dfrac{2\pi}{3}$,$-\dfrac{\pi}{2}$,$-\dfrac{\pi}{3}$,$0$,$\dfrac{\pi}{3}$,$\dfrac{\pi}{2}$,$\dfrac{2\pi}{3}$,$\pi$,$\dfrac{4\pi}{3}$,$\dfrac{3\pi}{2}$,$\dfrac{5\pi}{3}$,$2\pi$}
                \tkzTabLine{,+,|,+,|,+,|,+,$0$,-,|,-,|,-,|,-,$0$,+,|,+,|,+,|,+,}
                \tkzTabLine{,-,$0$,+,$0$,-,$0$,+,|,+,$0$,-,$0$,+,$0$,-,|,-,$0$,+,$0$,-,$0$,+,}
                \tkzTabLine{,-,$0$,+,$0$,-,$0$,+,|,-,$0$,+,$0$,-,$0$,+,|,-,$0$,+,$0$,-,$0$,+,}
                \tkzTabVar{+/,-/,+/,-/,+/,-/,+/,-/,+/,-/,+/,-/,+/,}
            \end{tikzpicture}
        \end{center}
        Từ bảng biến thiên ta suy ra hàm số $y=g(x)$ có $11$ điểm cực trị trên khoảng $(-\pi;2\pi)$.
    }
\end{ex}
\begin{ex}%[2D1G2-6]
    \immini{	Cho hàm số $y=f(x)$ có đồ thị của $y=f'(x)$ có đồ thị như hình vẽ bên dưới. Hàm số $g(x)=f(x^3-3x)-x^3+3x$ có bao nhiêu điểm cực tiểu? \choice[2]
        {$2$}
        {$4$}
        {$3$}
        {\True $5$}}{\begin{tikzpicture}[>=stealth,font=\scriptsize]
            \begin{scope}[scale=.5]
                \draw[->,line width = 1pt] (-2,0)--(0,0) node[below left]{$O$}--(5,0) node[below]{$x$};
                \draw[->,line width = 1pt] (0,-2) --(0,3) node[right]{$y$};
                \draw (-1,0) node[below left]{$-1$} circle (1pt);
                \draw (0,2) node[above right]{$2$} circle (1pt);
                \draw (2,0) node[below left]{$2$} circle (1pt);
                \draw (4,0) node[below right]{$4$} circle (1pt);
                \draw [ domain=-1.3:4.6, samples=100] %
                plot (\x, {0.25*(\x)^3-1.25*(\x)^2+0.5*(\x)+2});
            \end{scope}
    \end{tikzpicture}}
    \loigiai{
        $g'(x)=f'(x^3-3x)\cdot (3x^2-3)-3x^2+3=3(x^2-1)\left[f'(x^3-3x)-1\right].\\
        \Rightarrow g'(x)=0\Leftrightarrow \hoac{&x^2=1\\&f'(x^2-3x)=1} \Leftrightarrow \hoac{&x=\pm1\\&x^3-3x=a\quad (-1<a<0)\\&x^3-3x=b\quad (0<b<2)\\&x^3-3x=c\quad (c>4).}$\\
        \begin{center}
            \begin{tikzpicture}[>=stealth]
                \draw[->,line width = 1pt] (-2,0)--(0,0) node[below left]{$O$}--(5,0) node[below]{$x$};
                \draw[->,line width = 1pt] (0,-2) --(0,3) node[right]{$y$};
                \draw (-1,0) node[below left]{$-1$} circle (1pt);
                \draw (0,2) node[above right]{$2$} circle (1pt);
                \draw (2,0) node[below left]{$2$} circle (1pt);
                \draw (4,0) node[below right]{$4$} circle (1pt);
                \draw [thick, domain=-1.3:4.6, samples=100] %
                plot (\x, {0.25*(\x)^3-1.25*(\x)^2+0.5*(\x)+2});
                \draw [thick, domain=-2:5, samples=100] %
                plot (\x, {0*(\x)+1});
                \draw (-1,1) node[above left]{$y=1$};
                \draw (1,1.8) node[right]{$y=f'(x)$};
                \draw (4,0) node[below right]{$4$} circle(1pt);
                \draw (4.323404276086477,1) node[below right] {$c$} circle(1pt);
                \draw (1.3579263675184994,1) node[below left] {$b$} circle(1pt);
                \draw (-0.6813306436049771,1) node[below right] {$a$} circle(1pt);
            \end{tikzpicture}
        \end{center}
        \begin{itemize}
            \item Phương trình $x^3-3x=a$ có $3$ nghiệm $x_1$, $x_2$, $x_3$ với $x_1<x_2<x_3$.
            \item Phương trình $x^3-3x=b$ có $3$ nghiệm $x_4$, $x_5$, $x_6$ với $x_4<x_5<x_6$.
            \item Phương trình $x^3-3x=c$ có $1$ nghiệm $x_7\quad(x_7>x_6)$.
        \end{itemize}
        \begin{center}
            \begin{tikzpicture}[>=stealth]
                \draw[->,line width = 1pt] (-3,0)--(0,0) node[below left]{$O$}--(3,0) node[below]{$x$};
                \draw[->,line width = 1pt] (0,-3) --(0,6) node[right]{$y$};
                \draw (-1,0) node[below left]{$-1$} circle (1pt);
                \draw (0,2) node[above right]{$2$} circle (1pt);
                \draw (1,0) node[above left]{$1$} circle (1pt);
                \draw (2,0) node[below right]{$2$} circle (1pt);
                \draw (0,-2) node[below left]{$-2$} circle (1pt);
                \draw [thick,color=red, domain=-2.1:2.3, samples=100] %
                plot (\x, {(\x)^3-3*(\x)});
                \draw [thick, domain=-3:3, samples=100] plot (\x, {0*(\x)+4.32});
                \draw [thick, domain=-3:3, samples=100] plot (\x, {0*(\x)+1.36});
                \draw [thick, domain=-3:3, samples=100] plot (\x, {0*(\x)-0.68});
                \draw (2.3,5) node[above right]{$y=x^3-3x$};
                \draw (-2,-0.7) node[below left]{$y=a$};
                \draw (-2,1.4) node[below left]{$y=b$};
                \draw (-2,4) node[left]{$y=c$};
                \draw[dashed](-1,0)--(-1,2)--(0,2);
                \draw[dashed](0,-2)--(1,-2)--(1,0);
                \draw (-1.84,-0.68) node[below right] {$x_1$};
                \draw (0.23,-0.68) node[below right] {$x_2$};
                \draw (1.61,-0.68) node[below right] {$x_3$};
                \draw (-1.43,1.36) node[above left] {$x_4$};
                \draw (-0.56,1.36) node[above right] {$x_5$};
                \draw (1.93,1.36) node[below right] {$x_6$};
                \draw (2.22,4.32) node[below right] {$x_7$};
            \end{tikzpicture}
        \end{center}
        Bảng xét dấu $g'(x)$
        \begin{center}
            \begin{tikzpicture}
                \tkzTabInit[nocadre=false,lgt=2.3,espcl=1.3]
                {$x$ /1.1,$x^2-1$ /0.7,$f'(x^3-3x)$ /0.7,$g'(x)$ /0.7,$g(x)$ /2}
                {$-\infty$,$x_1$,$x_4$,$-1$,$x_5$,$x_2$,$1$,$x_3$,$x_6$,$x_7$,$+\infty$}
                \tkzTabLine{,+,|,+,|,+,$0$,-,|,-,|,-,$0$,+,|,+,|,+,|,+,}
                \tkzTabLine{,-,$0$,+,$0$,-,|,-,$0$,+,$0$,-,|,-,$0$,+,$0$,-,$0$,+,}
                \tkzTabLine{,-,$0$,+,$0$,-,$0$,+,$0$,-,$0$,+,$0$,-,$0$,+,$0$,-,$0$,+,}
                \tkzTabVar{+/,-/,+/,-/,+/,-/,+/,-/,+/,-/,+/}
            \end{tikzpicture}
        \end{center}
        Dựa vào bảng xét dấu ta kết luận hàm số $y=g(x)$ có $5$ điểm cực tiểu.
    }
\end{ex}
\begin{ex}%[2D1G2-2]
    \immini{Cho hàm số $y=f(x)$ có đạo hàm và liên tục trên $\mathbb{R}$ và có đồ thị $y=f'(x)$ như hình vẽ. Hàm $y=f(x^2-2)-\dfrac{1}{2}x^4+\dfrac{3}{2}x^2$ có bao nhiêu điểm cực tiểu?
        \choice[2]
        {$4$}
        {$1$}
        {$2$}
        {\True$3$}}
    {\begin{tikzpicture}[>=stealth,font=\scriptsize,x=1.2cm]
            \begin{scope}[scale=.7]
                \def\mx{-1} \def\max{3}
                \def\my{-1} \def\may{3.5}
                \def\hamso(#1,#2){plot [samples=200,smooth,domain=#1:#2](\x,{
                        2*(\x)^4-5*(\x)^3+1.5*(\x)^2+2.25*(\x)
                    })}
                \draw[fill=black]
                (-0.5,0)circle (.7pt)node[shift={(-110:.5)}]{$-\dfrac{1}{2}$}
                (0.5,0)circle (.7pt)node[shift={(-90:.5)}]{$\dfrac{1}{2}$}
                (2,0)circle (.7pt)node[shift={(-90:.5)}]{$2$}
                (1.5,0)circle (.7pt)node[shift={(-90:.5)}]{$\dfrac{3}{2}$}
                (0,1)circle (.7pt)node[shift={(180:.3)}]{$1$}
                (0,2.5)circle (.7pt)node[shift={(180:.3)}]{$\dfrac{5}{2}$}
                (0.5,1)circle (.7pt)
                (2,2.5)circle (.7pt)
                ;
                \draw[dashed,thin] (0.5,0)|-(0,1)(2,0)|-(0,2.5);
                %===========================================
                \draw[->] (\mx,0)--(0,0) node [below right] {$O$}--(\max,0) node[below] {$x$};
                \draw[->] (0,\my)--(0,\may) node[left] {$y$};
                \clip (\mx,\my) rectangle (\max,\may);
                \draw \hamso(\mx,\max);
            \end{scope}
    \end{tikzpicture}}
    \loigiai{
        \immini{
            Ta có $y'=2xf'(x^2-2)-2x^3+3x=2x\left(f'(x^2-2)-x^2+\dfrac{3}{2}\right)$.\\
            $y'=0\Leftrightarrow \hoac{&x=0\\&f'(x^2-2)-x^2+\dfrac{3}{2}=0.\quad(*)}$\\
            Đặt $t=x^2-2$ ta có $(*)\Leftrightarrow f'(t)-t-\dfrac{1}{2}=0\Leftrightarrow f'(t)=t+\dfrac{1}{2}$.\\
            Dựa vào đồ thị hàm số bên ta có
            $$f'(t)=t+\dfrac{1}{2}\Leftrightarrow \hoac{&t=-\dfrac{1}{2}\\&t=\dfrac{1}{2}\\&t=2.}$$
        }{
            \begin{tikzpicture}[scale=1.2,>=stealth,font=\footnotesize,y=.8cm]
                \def\mx{-1} \def\max{3}
                \def\my{-1} \def\may{3.5}
                \def\hamso(#1,#2){plot [samples=200,smooth,domain=#1:#2](\x,{
                        (\x)+0.5
                    })}
                \def\ham(#1,#2){plot [samples=200,smooth,domain=#1:#2](\x,{
                        2*(\x)^4-5*(\x)^3+1.5*(\x)^2+2.25*(\x)
                    })}
                \draw[fill=black]
                (-0.5,0)circle (.7pt)node[shift={(-110:.5)}]{$-\dfrac{1}{2}$}
                (0.5,0)circle (.7pt)node[shift={(-90:.5)}]{$\dfrac{1}{2}$}
                (2,0)circle (.7pt)node[shift={(-90:.5)}]{$2$}
                (1.5,0)circle (.7pt)node[shift={(-90:.5)}]{$\dfrac{3}{2}$}
                (0,1)circle (.7pt)node[shift={(180:.3)}]{$1$}
                (0,2.5)circle (.7pt)node[shift={(180:.3)}]{$\dfrac{5}{2}$}
                (0.5,1)circle (.7pt)
                (2,2.5)circle (.7pt)
                ;
                \draw[dashed,thin] (0.5,0)|-(0,1)(2,0)|-(0,2.5);
                %===========================================
                \draw[->] (\mx,0)--(0,0) node [below right] {$O$}--(\max,0) node[below] {$t$};
                \draw[->] (0,\my)--(0,\may) node[left] {$y$};
                \clip (\mx,\my) rectangle (\max,\may);
                \draw \hamso(\mx,\max)\ham(\mx,\max);
        \end{tikzpicture}}
        \noindent
        Suy ra $\hoac{&x^2-2=-\dfrac{1}{2}\\&x^2-2=\dfrac{1}{2}\\&x^2-2=2}\Leftrightarrow\hoac{&x=\pm\dfrac{\sqrt{6}}{2}\\&x=\pm\dfrac{\sqrt{10}}{2} &\text{ (nghiệm kép)}\\&x=\pm 2.}$\\
        Bảng xét dấu
        \begin{center}
            \begin{tikzpicture}
                \tkzTabInit[nocadre=false, lgt=0.7,espcl=1.6,deltacl=0.5]
                {$x$/1.2, $y'$ /0.6}
                {$-\infty$ , $-2$ , $-\dfrac{\sqrt{10}}{2}$ ,$-\dfrac{\sqrt{6}}{2}$,$0$,$\dfrac{\sqrt{6}}{2}$,$\dfrac{\sqrt{10}}{2}$ ,$2$, $+\infty$}
                \tkzTabLine{ ,-,0,+, 0 ,+, 0 ,-,0,+,0,-,0,-,0,+ }
            \end{tikzpicture}
        \end{center}
        Suy ra hàm số có $3$ điểm cực tiểu.
    }
\end{ex}
\begin{ex}%[2D1G2-6]
    \immini{Cho hàm số $y=f(x)$ có bảng biến thiên bên dưới. Số điểm cực đại và số điểm cực tiểu của hàm số $y=f^2(2x)-2f(2x)+1$ lần lượt là
        \choice
        {\True $2$ và $3$}
        {$3$ và $2$}
        {$1$ và $1$}
        {$2$ và $2$}}{\begin{tikzpicture}
            \tkzTabInit[nocadre=false,lgt=1.2,espcl=2,deltacl=0.6]
            {$x$ /0.6,$f'(x)$ /0.6,$f(x)$ /2}
            {$-\infty$,$-1$,$2$,$+\infty$}
            \tkzTabLine{,-,$0$,+,$0$,-,}
            \tkzTabVar{+/$+\infty$,-/$0$,+/$3$,-/$-\infty$}
    \end{tikzpicture}}
    \loigiai{
        Đặt $g(x)=	f^2(2x)-2f(2x)+1=\left[f(2x)-1\right]^2$.\\
        $\Rightarrow g'(x)=2\cdot\left[f(2x)-1\right]\cdot f'(2x)$.\\
        $\Rightarrow g'(x)=0\Leftrightarrow \hoac{&f(2x)=1\\&f'(2x)=0.}$\\
        \begin{itemize}
            \item $f(2x)=1\Leftrightarrow \hoac{&2x=a\quad(a<-1)\\&2x=b\quad(-1<b<2)\\&2x=c\quad(2<c)}\Leftrightarrow \hoac{&x=\dfrac{a}{2}\quad(\dfrac{a}{2}<-\dfrac{1}{2})\\&x=\dfrac{b}{2}\quad(-\dfrac{1}{2}<\dfrac{b}{2}<1)\\&x=\dfrac{c}{2}\quad(1<\dfrac{c}{2}).}$
            \item $f'(2x)=0\Leftrightarrow\hoac{&2x=-1\\&2x=2}\Leftrightarrow\hoac{&x=-\dfrac{1}{2}\\&x=1.}$
        \end{itemize}
        Bảng biến thiên hàm số $y=g(x)$
        \begin{center}
            \begin{tikzpicture}
                \tkzTabInit[nocadre=false,lgt=2.1,espcl=2,deltacl=0.6]
                {$x$ /1.1,$f'(2x)$ /0.7,$f(2x)-1$ /0.7,$g'(x)$ /0.7,$g(x)$ /2}
                {$-\infty$,$\dfrac{a}{2}$,$-\dfrac{1}{2}$,$\dfrac{b}{2}$,$1$,$\dfrac{c}{2}$,$+\infty$}
                \tkzTabLine{,-,|,-,$0$,+,|,+,$0$,-,|,-,}
                \tkzTabLine{,+,$0$,-,|,-,$0$,+,|,+,$0$,-,}
                \tkzTabLine{,-,$0$,+,$0$,-,$0$,+,$0$,-,$0$,+,}
                \tkzTabVar{+/,-/,+/,-/,+/,-/,+/}
            \end{tikzpicture}
        \end{center}
        Dựa vào bảng biến thiên ta thấy hàm số $y=g(x)$ có $2$ điểm cực đại và $3$ điểm cực tiểu.
    }
\end{ex}

\begin{ex}%[2D1G2-6]
    Cho hàm số bậc ba $y=f(x)$ có đồ thị như hình bên. Có bao nhiêu giá trị nguyên của tham số $m$ để hàm số $y=\vert f^2(x)+2f(x)+m\vert$ có $9$ điểm cực trị?
    \choice
    {\True $24$}
    {Vô số}
    {$25$}
    {$23$}
    \loigiai{
        Đặt $y=g(x)=f^2(x)+2f(x)+m=\left[f(x)+1\right]^2+m-1.\\
        \Rightarrow g'(x)=2\left[f(x)+1\right]\cdot f'(x).\\
        \Rightarrow g'(x)=0\Leftrightarrow \hoac{&f'(x)=0\\&f(x)=-1}\Leftrightarrow \hoac{&x=1\\&x=3\\&x=a\quad(0<a<1)\\&x=b\quad(1<b<3)\\&x=c\quad(3<c).}$\\
        Từ đồ thị ta suy ra
        \begin{itemize}
            \item $f'(x)+1>0\Leftrightarrow f'(x)>-1\Leftrightarrow a<x<b \text{ hoặc } x>c$.
            \item $f'(x)+1<0\Leftrightarrow f'(x)<-1\Leftrightarrow x<a \text{ hoặc } b<x<c$.
        \end{itemize}
        Bảng biến thiên hàm số $y=g(x)$
        \begin{center}
            \begin{tikzpicture}
                \tkzTabInit[nocadre=false,lgt=2.3,espcl=1.8]
                {$x$ /1.1,$f'(x)$ /0.7,$f(x)+1$ /0.7,$g'(x)$ /0.7,$g(x)$ /2}
                {$-\infty$,$0$,$a$,$1$,$b$,$3$,$c$,$+\infty$}
                \tkzTabLine{,+,t,+,|,+,$0$,-,|,-,$0$,+,|,+,}
                \tkzTabLine{,-,t,-,$0$,+,|,+,$0$,-,|,-,$0$,+,}
                \tkzTabLine{,-,t,-,$0$,+,$0$,-,$0$,+,$0$,-,$0$,+,}
                \tkzTabVar{+/,R,-/$m-1$,+/$m+24$,-/$m-1$,+/$m$,-/$m-1$,+/}
            \end{tikzpicture}
        \end{center}
        Đồ thị hàm số $y=|g(x)|$ gồm có $2$ phần như sau:
        \begin{itemize}
            \item Phần 1: Trùng với đồ thị hàm số $y=g(x)$ với $g(x)\ge0$.
            \item Phần 2: Là phần đối xứng với phần đồ thị của hàm số $y=g(x)$ với $g(x)<0$ qua trục $\text{Ox}$.
        \end{itemize}
        Kết hợp với bảng biến thiên hàm số $y=g(x)$ ta suy ra hàm số $y=|g(x)|$ có $9$ điểm cực trị khi và chỉ khi $m\le 0<m+24 \Leftrightarrow -24<m\le0$. Mà $m$ là số nguyên nên ta được $24$ giá trị của $m$.
    }
\end{ex}
\begin{ex}%[2D1K2-6]
    Có bao giá trị nguyên của tham số $m$ thoả mãn $\vert m\vert<10$ sao cho hàm số $y=\vert x^3-(m-2)x^2-mx-m^2\vert$ có $3$ điểm cực tiểu?
    \choice
    {$9$}
    {$10$}
    {\True $8$}
    {$16$}
    \loigiai{
        Đặt hàm số $y=f(x)=x^3-(m-2)x^2-mx-m^2=(x-m)(x^2+2x+m)=(x-m)\left[x(x+2)+m\right]$.
        Suy ra $f'(x)=3x^2-2(m-2)x-m=0$ có $\Delta'=(m-2)^2+3m=m^2-m+4>0$ với mọi $m$.\\
        Theo định lí Vi-ét ta có $\heva{&x_1+x_2=\dfrac{2(m-2)}{3}\\&x_1x_2=-\dfrac{m}{3}}$.\\
        Hàm số $y=|f(x)|$ có $3$ điểm cực tiểu khi và chỉ khi $y(x_1)\cdot y(x_2)<0$.\\
        Thực hiện biến đổi\\
        $y(x_1)\cdot y(x_2)=  (x_1-m)(x_2-m)\left[x_1(x_1+2)+m\right]\left[x_2(x_2+2)+m\right]\\
        = (x_1-m)(x_2-m)\left[x_1x_2(x_1+2)(x_2+2)+m(x_1^2+x_2^2)+2m(x_1+x_2)+m^2\right]\\
        = \left[x_1x_2-m(x_1+x_2)+m^2\right]\left[x_1x_2\left(x_1x_2+2(x_1+x_2)+4\right)+m(x_1^2+x_2^2)+2m(x_1+x_2)+m^2\right]\\
        = \left(\dfrac{m^2}{3}+m\right)\left[-\dfrac{m}{3}\left(m+\dfrac{4}{3}\right)+m\left(\dfrac{4m^2}{9}-\dfrac{10m}{9}+\dfrac{16}{9}\right)+\dfrac{4m^2}{3}-\dfrac{8m}{3}+m^2\right]\\
        = \dfrac{2}{27}m^2(m+3)(2m^2+4m-5)$.\\
        Suy ra $y(x_1)\cdot y(x_2)<0\Leftrightarrow m^2(m+3)(2m^2+4m-5)<0\Leftrightarrow \hoac{&m<-3\\&\dfrac{-2-\sqrt{14}}{2}<m<0\\&0<m<\dfrac{-2+\sqrt{14}}{2}}$.\\
        Kết hợp với điều kiện $m$ là số nguyên thỏa $|m|<10$ ta được $m\in\{-9;-8;-7;-6;-5;-4;-2;-1\}$.\\
        Vậy có $8$ giá trị nguyên của tham số $m$.
    }
\end{ex}

\begin{ex}%[2D1G2-6]
    \immini{	Cho hàm số $f(x)=ax^4+bx^3+cx^2+dx+e, (ae<0)$. Đồ thì hàm số $y=f'(x)$ như hình bên dưới. Hàm số $y=\left|4f(x)-x^2\right|$ có bao nhiêu điểm cực tiểu?
        \choice[2]
        {$4$}
        {$5$}
        {\True $3$}
        {$2$}
    }{\begin{tikzpicture}[>=stealth,font=\scriptsize,x=1.3cm,y=1.5cm]
            \begin{scope}[scale=0.5]
                \draw[->] (-2,0) -- (3,0) node[below] { $x$};
                \draw[->] (0,-1) -- (0,3) node[left] {\ $y$};
                \draw (0,0) node[above left] { $O$} circle (1pt);
                \draw[smooth](0,0) parabola bend (-0.6,-1)(-1.7,1.7);
                \draw(0,0) parabola bend (1.3,2.5)(2,1);
                \draw [dashed] (0,1)--(2,1)--(2,0);
                \draw [dashed] (-1.05,0)--(-1.05,-0.5)--(0,-0.5);
                \draw (-1,0) node[above] { $-1$};
                \draw (0,-0.5) node[right] { $-\dfrac{1}{2}$};
                \draw (0,1) node[left] { $1$};
                \draw (2,0) node[below] { $2$};
            \end{scope}
    \end{tikzpicture}}
    \loigiai{
        Ta có $f'(x)=4ax^3+3bx^2+2cx+d$. Từ đồ thị hàm số $f'(x)$ suy ra $a<0$, do đó $e>0$.\\
        Đặt $y=g(x)=4f(x)-x^2\Rightarrow g'(x)	=4f'(x)-2x=4\left[f'(x)-\dfrac{x}{2}\right]$.\\
        \begin{center}
            \begin{tikzpicture}[>=stealth,x=1.0cm,y=1.0cm,thick, scale=1.0]
                \draw[->] (-2,0) -- (3,0) node[below] {\footnotesize $x$};
                \draw[->] (0,-1) -- (0,3) node[left] {\footnotesize $y$};
                \draw (0,0) node[below left] {\footnotesize $O$} circle (1pt);
                \draw[smooth](0,0) parabola bend (-0.6,-1)(-1.7,1.7);
                \draw(0,0) parabola bend (1.3,2.5)(2,1);
                \draw [dashed] (0,1)--(2,1)--(2,0);
                \draw [dashed] (-1.05,0)--(-1.05,-0.5)--(0,-0.5);
                \draw (-1,0) node[above] {\footnotesize $-1$};
                \draw (0,-0.5) node[right] {\footnotesize $-\dfrac{1}{2}$};
                \draw (0,1) node[left] {\footnotesize $1$};
                \draw (2,0) node[below] {\footnotesize $2$};
                \draw [thick, domain=-2.1:2.5, samples=100] %
                plot (\x, {0.5*(\x)});
            \end{tikzpicture}
        \end{center}
        Suy ra $g'(x)=0\Leftrightarrow f'(x)-\dfrac{x}{2}=0\Leftrightarrow f'(x)=\dfrac{x}{2}\Leftrightarrow \hoac{&x=-1\\&x=0\\&x=2}$.\\
        Bảng biến thiên
        \begin{center}
            \begin{tikzpicture}
                \tkzTabInit[nocadre=false,lgt=1.2,espcl=2.3]
                {$x$ /0.6,$g'(x)$ /0.6,$g(x)$ /2}
                {$-\infty$,$-1$,$0$,$2$,$+\infty$}
                \tkzTabLine{,+,$0$,-,$0$,+,$0$,-,}
                \tkzTabVar{-/,+/,-/$4e$,+/,-/}
            \end{tikzpicture}
        \end{center}
        Vì $4e>0$ nên từ bảng biến thiên hàm số $g(x)$ ta suy ra hàm số $y=\left|g(x)\right|$ có $3$ điểm cực tiểu.
    }
\end{ex}

\begin{ex}%[2D1G2-6]
    \immini{	Cho hàm số bậc bốn $f(x)$ có $f(0)=-1$. Hàm số $y=f'(x)$ có đồ thị là hình bên. Số điểm cực trị của hàm số $y=\vert 4f(x+1)+x^2+2x\vert$ là
        \choice[2]
        {$3$}
        {\True $5$}
        {$4$}
        {$6$}}{\begin{tikzpicture}[>=stealth,font=\scriptsize,y=.8cm]
            \begin{scope}[scale=.5]
                \draw[->] (-3.5,0) -- (5,0) node[below] {$x$};
                \draw[->] (0,-4) -- (0,2) node[left] { $y$};
                \draw (0,0) node[below left] {$O$} circle (1pt);
                \draw (-3.6,-3.6) ..controls +(60:0.2) and +(-180:1.3).. (-1.5,1.2) ..controls +(0:1.6) and +(-180:1.6) .. (2.8,-3.4)..controls +(0:0.5) and +(-100:4.5) .. (4.95,1.8);
                \draw [dashed] (-2,0)--(-2,1)--(0,1);
                \draw [dashed] (0,-2)--(4,-2)--(4,0);
                \draw (-2,0) node[below] { $-2$};
                \draw (0,-2) node[left] { $-2$};
                \draw (0,1) node[right] { $1$};
                \draw (4,0) node[above] {$4$};
            \end{scope}
    \end{tikzpicture}}
    \loigiai{
        Đặt $y=g(x)=4f(x+1)+x^2+2x\Rightarrow g'(x)=4f'(x+1)+2x+2=4\left[f'(x+1)+\dfrac{x+1}{2}\right]$.\\
        Suy ra $g'(x)=0\Leftrightarrow f'(x+1)=-\dfrac{x+1}{2}$.\\
        Đặt $t=x+1$ thì phương trình trở thành $f'(t)=-\dfrac{t}{2}$. Nghiệm của phương trình này là hoành độ giao điểm của đồ thị hàm số $y=f'(t)$ và $y=-\dfrac{t}{2}$.
        \begin{center}
            \begin{tikzpicture}[>=stealth,x=1.0cm,y=1.0cm,thick, scale=0.8]
                \draw[->] (-3.5,0) -- (5,0) node[below] {\footnotesize $t$};
                \draw[->] (0,-4) -- (0,2) node[left] {\footnotesize $y$};
                \draw (0,0) node[below left] {\footnotesize $O$} circle (1pt);
                \draw[very thick] (-3.6,-3.6) ..controls +(60:0.2) and +(-180:1.3).. (-1.5,1.2) ..controls +(0:1.6) and +(-180:1.6) .. (2.8,-3.4)..controls +(0:0.5) and +(-100:4.5) .. (4.95,1.8);
                \draw [thick, domain=-3.5:5, samples=100] %
                plot (\x, {-0.5*(\x)});
                \draw [dashed] (-2,0)--(-2,1)--(0,1);
                \draw [dashed] (0,-2)--(4,-2)--(4,0);
                \draw (-2,0) node[below] {\footnotesize $-2$};
                \draw (0,-2) node[left] {\footnotesize $-2$};
                \draw (0,1) node[right] {\footnotesize $1$};
                \draw (4,0) node[above] {\footnotesize $4$};
            \end{tikzpicture}
        \end{center}
        Do đó\\
        $$f'(t)=-\dfrac{t}{2}\Leftrightarrow\hoac{&t=-2\\&t=0\\&t=4}\Rightarrow \hoac{&x+1=-2\\&x+1=0\\&x+1=4}\Leftrightarrow \hoac{&x=-3\\&x=-1\\&x=3.}$$
        Bảng biến thiên
        \begin{center}
            \begin{tikzpicture}
                \tkzTabInit[nocadre=false,lgt=1.2,espcl=2.3]
                {$x$ /0.6,$g'(x)$ /0.6,$g(x)$ /2}
                {$-\infty$,$-3$,$-1$,$3$,$+\infty$}
                \tkzTabLine{,-,$0$,+,$0$,-,$0$,+,}
                \tkzTabVar{+/,-/,+/$-5$,-/,+/}
            \end{tikzpicture}
        \end{center}
        Từ bảng biến thiên suy ra hàm số $y=g(x)$ có $3$ cực trị âm, do đó hàm số $y=\left|g(x)\right|$ có $5$ điểm cực trị.
    }
\end{ex}
\begin{ex}%[2D1Y2-6]
    \immini{	Cho hàm số $y=f(x)$ có bảng biến thiên như hình vẽ. Hàm số $y=f\left(|x|\right)$ đạt cực đại tại.
        \choice[2]
        {$x=-1$}
        {\True $x=0$}
        {$x=2$}
        {$x=-2$}}{\begin{tikzpicture}[>=stealth,scale=0.9]
            \tkzTabInit[nocadre=false,lgt=1,espcl=2.6,deltacl=0.5]{$x$/.7 ,$y'$/.7,$y$/2}
            {$-\infty$ , $-1$ , $2$ , $+\infty$}
            \tkzTabLine{ , + , $0$ , - , $0$ , + , }
            \tkzTabVar{-/$-\infty$ , +/$3$ , -/$1$ , +/$+\infty$}
    \end{tikzpicture}}
    \loigiai{Từ bảng biến thiên của hàm số $y=f(x)$ ta có bảng biến thiên của hàm số $y=f\left(|x|\right)$ như sau
        \begin{center}
            \begin{tikzpicture}[scale=0.9]
                \foreach \x/\texn in {0/x,2/-\infty,4/-2,6/0,8/2,10/+\infty} \path (\x,3.5)node{$\texn$};
                \foreach \x/\texn in
                {0/y',3/-,4/0,5/+,6/0,7/-,8/0,9/+} \path (\x,2.5)node{$\texn$};
                \foreach \x/\y/\texn in {0/1/y,
                    2/1.5/+\infty,4/0/1,6/1.0/f(0),8/0/1,10/1.5/+\infty}
                \path (\x,\y) node(\x){$\texn$};
                \foreach \x/\y/\texn in {2/4,4/6,6/8,8/10}
                \draw[-stealth] (\x)--(\y);
                \draw
                (-.5,3)--(10.5,3) (1,4)--(1,0)
                (-.5,2)--(10.5,2);
            \end{tikzpicture}
        \end{center}
        Từ bảng biến thiên ta thấy hàm số $y=f\left(|x|\right)$ đạt cực đại tại $x=0$.
    }
\end{ex}
%%=====Câu 70
\begin{ex}%[2D1Y2-6]
    \immini{	Cho hàm số $y=f(x)$ có bảng biến thiên như hình vẽ. Tổng các giá trị cực đại của hàm số $y=\left|f(x)\right|$ là
        \choice[2]
        {\True $9$}
        {$-3$}
        {$3$}
        {$7$}}{\begin{tikzpicture}[>=stealth]
            \tkzTabInit[nocadre=false,lgt=1,espcl=1.7,deltacl=0.5]{$x$/.7 ,$y'$/.7,$y$/2}
            {$-\infty$ , $-1$ , $0$ , $1$ , $+\infty$}
            \tkzTabLine{ , - , $0$ , + , $0$ , - , $0$ , + , }
            \tkzTabVar{+/$+\infty$ , -/$-2$, +/$3$ , -/$-4$ , +/$+\infty$}
    \end{tikzpicture}}
    \loigiai{Từ bảng biến thiên của hàm số $y=f(x)$ ta có bảng biến thiên của hàm số $y=\left|f(x)\right|$ như sau
        \begin{center}
            \begin{tikzpicture}[scale=0.8]
                \foreach \x/\texn in {0/x,2/-\infty,4/x_1,6/-1,8/x_2,10/0,12/x_3,14/1,16/x_4,18/+\infty} \path (\x,3.5)node{$\texn$};
                \foreach \x/\texn in
                {0/y',3/-,4/||,5/+,6/0,7/-,8/||,9/+,10/0,11/-,12/||,13/+,14/0,15/-,16/||,17/+} \path (\x,2.5)node{$\texn$};
                \foreach \x/\y/\texn in {0/1/y,
                    2/1.5/+\infty,4/0/0,6/0.5/2,8/0/0,10/0.7/3,12/0/0,14/1.0/4,16/0/0,18/1.5/+\infty}
                \path (\x,\y) node(\x){$\texn$};
                \foreach \x/\y/\texn in {2/4,4/6,6/8,8/10,10/12,12/14,14/16,16/18}
                \draw[-stealth] (\x)--(\y);
                \foreach \x in {4,8,12}\draw[dashed,red] (\x)--+(3.7,0);
                \draw[dashed,red] (1.2,0)--+(2.7,0) (16)--+(2.0,0) node[below=-.1]{\small $y=0$};
                \draw
                (-.5,3)--(18.5,3) (1,4)--(1,0)
                (-.5,2)--(18.5,2);
            \end{tikzpicture}
        \end{center}
        Từ bảng biến thiên ta thấy hàm số $y=\left|f(x)\right|$ có $3$ giá trị cực đại lần lượt là $2$, $3$, $4$.\\
        Tổng các giá trị cực đại là $9$.
    }
\end{ex}
\begin{ex}%[2D1Y2-6]
    Cho hàm số $y=f(x)$ có đạo hàm $y=f'(x)=(x-1)(x-2)^4(x^2-4)$. Số điểm cực trị của hàm số $y=f(|x|)$ là
    \choice
    {$3$}
    {$2$}
    {$4$}
    {\True $5$}
\end{ex}
\begin{ex}%[2D1Y2-6]
    Cho hàm số $y=f(x)$ có đạo hàm $y=f'(x)=(x^3-2x^2)(x^3-2x)$ trên $\mathbb{R}$. Hàm số $y=|f(4-2021x)|$ có nhiều nhất bao nhiêu điểm cực trị?
    \choice
    {\True$9$}
    {$11$}
    {$2021$}
    { $5$}
\end{ex}
\begin{ex}%[2D1B2-6]
    Có bao nhiêu giá trị nguyên của tham số $m$ để hàm số $y=|3x^4-4x^3-12x^2+m|$ có $7$ điểm cực trị?
    \choice
    {$3$}
    {$5$}
    {$6$}
    {\True $4$}
    \loigiai{
        Đặt $f(x)=3x^4-4x^3-12x^2+m$ $\Rightarrow f'(x)=12x^3-12x^2-24x=0 \Rightarrow x=0; x=-1; x=2$.\\
        Qua BBT của $y=f(x)$ ta suy ra $y=|f(x)|$ có $7$ điểm cực trị $\Rightarrow \heva{&m>0\\&m-5<0} \Rightarrow 0<m<5$. Vậy có $4$ giá trị nguyên $m$ thỏa yêu cầu bài toán.
    }
\end{ex}
\begin{ex}%[2D1B2-6]
    Tìm các giá trị của $m$ để hàm số $f(x)=|x^3+3x^2+m-3|$ có ba điểm cực trị.
    \choice
    {$m=3; m=-1$}
    {$m\ge 1; m \le-3$}
    {$1\le m \le 3$}
    {\True $m\ge 3; m \le -1$}
\end{ex}
\begin{ex}%[2D1B2-6]
    Cho hàm số $y=f(x)=x^3-3mx^2+3(m^2-4)x+1$, có bao nhiêu số nguyên $m \in (-10;10)$  để hàm số $y=f(|x|)$ có đúng $5$ điểm cực trị.
    \choice
    {$3$}
    {$6$}
    {$8$}
    {\True $7$}
    \loigiai{
        $y=f(|x|)$ có đúng $5$ điểm cực trị $\Rightarrow y=f(x)$ có hai điểm cực trị dương.\\
        $f'(x)=3x^2-6mx+3(m^2-4)=0 \Rightarrow x=m-2; x=m+2$ có hai nghiệm dương $\Leftrightarrow m-2>0 \Leftrightarrow m >2$.\\ Vậy có $7$ giá trị $m$ thỏa yêu cầu bài toán.
    }
\end{ex}
\begin{ex}%[2D1G2-6]
    Cho hàm số $f(x)=\dfrac{1}{3}x^3-(2m-1)x^2+(8-m)x+2020$ với $m$ là tham số. Tập hợp tất cả các giá trị của tham số $m$ để hàm số $y=f\left(\vert x\vert\right)$ có điểm $5$ cực trị là khoảng $(a;b)$. Tích $a\cdot b$ bằng
    \choice
    {$12$}
    {$16$}
    {$10$}
    {\True $14$}
    \loigiai{
        Tập xác định $\mathscr{D}=\mathbb{R}$.\\
        Ta có $f\left(|-x|\right)=f\left(|x|\right)$, $\forall x\in\mathbb{R}$ nên $y=f\left(|x|\right)$ là hàm số chẵn. \\
        Do đó, đồ thị hàm số $y=f\left(|x|\right)$ đối xứng qua trục tung.\\
        Suy ra hàm số $y=f\left(|x|\right)$ luôn có một điểm cực trị là $x=0$.\\
        Do đó, $y=f\left(|x|\right)$ có $5$ điểm cực trị $\Leftrightarrow$ hàm số $y=f(x)$ có $2$ điểm cực trị dương.\\
        \phantom{Do đó, $y=f\left(|x|\right)$ có $5$ điểm cực trị} $\Leftrightarrow$  $f'(x)=0$ có hai nghiệm dương phân biệt.\\
        Ta có $f'(x)=x^2-2(m-1)x+8-m$.\\
        Yêu cầu bài toán $\Leftrightarrow\heva{&\Delta'>0 \\ &S>0 \\ &P>0}\Leftrightarrow\heva{&4m^2-3m-7>0 \\ &2m-1>0 \\ &8-m>0}\Leftrightarrow\heva{&m<-1\;\text{hoặc}\;m>\dfrac{7}{4} \\ &m>\dfrac{1}{2} \\ &m<8}\Leftrightarrow \dfrac{7}{4}<m<8$.
        Suy ra $a\cdot b=14$.
    }
\end{ex}
\begin{ex}%[2D1G2-6]
    \immini{	Cho hàm số $f(x)$ có đạo hàm liên tục trên $\mathbb{R}$ và đồ thị hàm số $f'(x)$ như hình vẽ. Hàm số $y=f\left(x^2-2\vert x\vert\right)$ có bao nhiêu điểm cực tiểu?
        \choice[2]
        {$1$}
        {\True $2$}
        {$5$}
        {$3$}}{\begin{tikzpicture}[>=stealth,font=\scriptsize]
            \draw[->] (-2,0) -- (2,0) node[below] {$x$};
            \draw[->] (0,-1) -- (0,2) node[left] {$y$};
            \draw (0,0) node[below right] {$O$} circle(1pt) (-1,0) node[above left]{ $-1$} (1,0) node[below]{ $1$} (0,1) node[right]{ $1$} (-1/3,0) node[below]{\ $-\dfrac{1}{3}$};
            \draw [dashed] (-1/3,0)--(-1/3,1.2);
            \draw plot[smooth,tension=.65] coordinates{(-1.2,-0.5) (-1/3,1.2) (1,0) (1.8,1.5)};
    \end{tikzpicture}}
    \loigiai{
        Đặt $g(x)=f(x^2-2x)\Rightarrow g'(x)=2(x-1)f'(x^2-2x).\\
        g'(x)=0\Leftrightarrow \hoac{&x=1\\&f'(x^2-2x)=0}\Leftrightarrow \hoac{&x=1\\&x^2-2x=-1\\&x^2-2x=1}\Leftrightarrow \hoac{&x=1\text{ (bội 3)}\\&x=1-\sqrt{2}\\&x=1+\sqrt{2}.}$\\
        Ta có
        \begin{itemize}
            \item $f'(x)>0\Leftrightarrow \heva{&x>-1\\&x\neq1}$ nên $f'(x^2-2x)>0\Leftrightarrow \heva{&x^2-2x>-1\\&x^2-2x\neq -1}\Leftrightarrow \heva{&x\neq 1\\&x=1\pm\sqrt{2}.}$
            \item $f'(x)<0\Leftrightarrow x<-1$ nên $f'(x^2-2x)<0\Leftrightarrow x^2-2x<-1 \text{ (Vô nghiệm)}.$
        \end{itemize}
        Bảng biến thiên hàm số $y=g(x)$
        \begin{center}
            \begin{tikzpicture}
                \tkzTabInit[nocadre=false,lgt=2.5,espcl=2.1,deltacl=0.6]
                {$x$ /0.7,$x-1$ /0.7,$f'(x^2-2x)$ /0.7,$g'(x)$ /0.7,$g(x)$ /2}
                {$-\infty$,$1-\sqrt{2}$,$0$,$1$,$1+\sqrt{2}$,$+\infty$}
                \tkzTabLine{,-,|,-,t,-,$0$,+,|,+,}
                \tkzTabLine{,+,$0$,+,t,+,$0$,+,$0$,+,}
                \tkzTabLine{,-,$0$,-,t,-,$0$,+,$0$,+,}
                \tkzTabVar{+/,R,R,-/,R,+/}
            \end{tikzpicture}
        \end{center}
        Do hàm số $y=f(x^2-2\vert x\vert)$ là hàm số chẵn nên từ bảng biến thiên trên ta suy ra đồ thị hàm số $y=f(x^2-2\vert x\vert)$ gồm hai nhánh như sau
        \begin{itemize}
            \item Nhánh thứ nhất là phần đồ thị hàm số $y=g(x)$ với $x\ge 0$.
            \item Nhánh thứ hai là phần đối xứng với nhánh thức nhất qua trục $Oy$
        \end{itemize}
        Do đó hàm số $y=f(x^2-2\vert x\vert)$ có $2$ điểm cực tiểu.

    }
\end{ex}
\begin{ex}%[2D1G2-6]
    \immini{	Cho hàm bậc bốn $y=f(x)$ có đồ thị như hình vẽ dưới đây. Số điểm cực trị của hàm số $g(x)=f\left(\vert x\vert^3-3\vert x\vert\right)$ là
        \choice[2]
        {$5$}
        {$3$}
        {\True $7$}
        {$11$}}{\begin{tikzpicture}[>=stealth,font=\scriptsize,y=.6cm,x=.7cm]
            \begin{scope}[scale=.5]
                \draw[->] (-5,0) -- (5,0) node[below] {\footnotesize $x$};
                \draw[->] (0,-4.5) -- (0,4) node[left] {\footnotesize $y$};
                \draw (0,0) node[below left] {\footnotesize $O$} circle (1pt);
                \draw[smooth](-1.1,0) parabola bend (-2,-2)(-4,4);
                \draw(-1.1,0) parabola bend (0,2)(1.5,-2.1);
                \draw[smooth](1.5,-2.1) parabola bend (2.6,-4)(4,4);
                \draw (1.5,0) node[above] {\footnotesize $2$};
                \draw (-1.1,0) node[above left] {\footnotesize $-2$};
            \end{scope}
    \end{tikzpicture}}
    \loigiai{
        Đặt $g(x)=f(x^3-3x)\Rightarrow g'(x)=3(x^2-1)f'(x^3-3x)$.\\
        Suy ra $g'(x)=0\Leftrightarrow \hoac{&x^2=1\\&f'(x^3-3x)=0}\Leftrightarrow \hoac{&x=\pm1\\&x^3-3x=2\\&x^3-3x=-2\\&x^3-3x=a\quad(a<-2)\\&x^3-3x=b\quad(b>2).}$\\
        Ta có
        \begin{itemize}
            \item $x^3-3x=2\Leftrightarrow \hoac{&x=2\\&x=-1.}$
            \item $x^3-3x=-2\Leftrightarrow \hoac{&x=-2\\&x=1.}$
            \item $x^3-3x=a\Leftrightarrow x=m \text{ (với $m<-2$)}$.
            \item $x^3-3x=b\Leftrightarrow x=n \text{ (với $n>2$)}$.
        \end{itemize}
        Từ đồ thị hàm số $f'(x)$ ta có $f'(x)>0\Leftrightarrow \hoac{&x<a\\&-2<x<2\\&x>b.}$\\
        Suy ra $h'(x)>0\Leftrightarrow \hoac{&x^3-3x<a\\&-2<x^3-3x<2\\&x^3-3x>b}\Leftrightarrow \hoac{&x<m\\&-2<x<-1\\&-1<x<2\\&x>n.}$\\
        Bảng biến thiên hàm số $y=g(x)$
        \begin{center}
            \begin{tikzpicture}
                \tkzTabInit[nocadre=false,lgt=2.5,espcl=1.7,deltacl=0.5]
                {$x$ /0.7,$x^2-1$ /0.7,$f'(x^3-3x)$ /0.7,$g'(x)$ /0.7,$g(x)$ /2}
                {$-\infty$,$m$,$-2$,$-1$,$0$,$1$,$2$,$n$,$+\infty$}
                \tkzTabLine{,+,|,+,|,+,$0$,-,t,-,$0$,+,|,+,|,+,}
                \tkzTabLine{,+,|,-,|,+,$0$,+,t,+,$0$,+,|,-,|,+,}
                \tkzTabLine{,+,|,-,|,+,$0$,-,t,-,$0$,+,|,-,|,+,}
                \tkzTabVar{-/,+/,-/,+/,R,-/,+/,-/,+/}
            \end{tikzpicture}
        \end{center}
        Từ bảng biến thiên suy ra hàm số $y=g(x)$ có $3$ điểm cực trị ứng với $x>0$ nên hàm số $y=f(|x|^3-3|x|)$ có $7$ điểm cực trị.

    }
\end{ex}
\begin{ex}%[2D1K2-2]
    \immini{
        Hình vẽ dưới đây là đồ thị của hàm số $y=f(x)$.
        Có bao nhiêu giá trị nguyên dương của tham số $m$ để hàm số $y=\left|f(x+1)+m\right|$ có $5$ cực trị?
        \choice
        {$0$}
        {\True $3$}
        {$2$}
        {$1$}
    }{
        \begin{tikzpicture}[>=stealth, font=\footnotesize, line join=round, line cap=round,y=.7cm]
            \begin{scope}[scale=.5]
                \def\xmin{-4} \def\xmax{3}
                \def\ymin{-5.5} \def\ymax{4}
                %\draw[color=gray!50,dashed] (\xmin,\ymin) grid (\xmax,\ymax);
                \draw[->] (\xmin,0)--(\xmax,0) node [below]{$x$};
                \draw[->] (0,\ymin)--(0,\ymax) node [left]{$y$};
                \node at (0,0) [below right]{$O$};
                \clip (\xmin+0.1,\ymin+0.1) rectangle (\xmax-0.5,\ymax-0.1);
                \draw[smooth,samples=300,domain=-3.5:0] plot(\x,{-1.24*(\x)^3-5.74*(\x)^2-5.78*(\x)});
                \draw[smooth,samples=300,domain=0:2.3] plot(\x,{1.78*(\x)^3-1.61*(\x)^2-4.57*(\x)-0.04});
                \draw[dashed](-2.5,0)|-(0,-2.07)  (-0.7,0)|-(0,1.7) (1.3,0)|-(0,-4.8);
                \draw[fill=black](0,-2.07)node[below left]{$-3$}circle(1pt)
                (0,-4.8)node[left]{$-6$}circle(1pt)
                (0,1.7)node[right]{$2$}circle(1pt);
            \end{scope}
        \end{tikzpicture}
    }
    \loigiai{
        Nhận xét
        \begin{itemize}
            \item  Hàm số $y=\left|f(x)-\alpha\right|$ có số điểm cực trị bằng số cực trị của hàm $y=f(x)$ và số giao điểm của đồ thị hàm $y=f(x)$ với đường thẳng $y=\alpha$ (không tính giao điểm là các điểm cực trị).
            \item  Số điểm cực trị của hàm $y=f(x)$ bằng số điểm cực trị của hàm $y=f(x+a)$.
        \end{itemize}
        Từ nhận xét trên ta có: Hàm số $y=f(x+1)$ có $3$ cực trị.\\
        Vậy ta cần đường thẳng $y=-m$ cắt đồ thị hàm số $y=f(x+1)$ tại 2 điểm khác cực trị.\\
        Từ đồ thị ta suy ra: $\hoac{&-6 <-m\leq-3\\&-m\geq 2}\Leftrightarrow\hoac{&3\leq m<6\\&m\leq-2.}$ \\
        Do $m\in\mathbb{N}^*$ nên $m\in\{3,4,5\}$.
    }
\end{ex}
\Closesolutionfile{ans}
%% \indapan{10}{ans/2D1-2-DANG-3}
%%Bài 2. Max-min
% \setcounter{section}{1}
\section{GIÁ TRỊ LỚN NHẤT - NHỎ NHẤT CỦA HÀM SỐ}

\subsection{LÝ THUYẾT CẦN NHỚ}
Cho hàm số $y=f(x)$ xác định trên tập $\mathscr{D}$. Ta có
\immini{\begin{itemize}
		\item[\ding{172}] $M$ là giá trị lớn nhất của hàm số nếu $\heva{&f(x) \le M,\forall x \in \mathscr{D}\\& \exists x_0 \in \mathscr{D}: f(x_0)=M.}$\\
		Kí hiệu \fbox{$\displaystyle\max_{x \in \mathscr{D}}f(x)=M$}
		\vskip 0.5cm
		\item[\ding{173}] $n$ là giá trị nhỏ nhất của hàm số nếu $\heva{&f(x) \ge n,\forall x \in \mathscr{D}\\& \exists x_0 \in \mathscr{D}: f(x_0)=n.}$\\
		Kí hiệu \fbox{$\displaystyle\min_{x \in \mathscr{D}}f(x)=n$}
\end{itemize}
}{
\begin{tikzpicture}[smooth,samples=300,scale=0.7,>=stealth]
	\draw[->] (-1.5,0)--(4.8,0) node[below]{$x$};
	\draw[->] (0,-2)--(0,4) node[right]{$y$};
	\draw (0,0) node[above left]{$O$};
	\draw[line width = 1.2pt,domain=-1:4,blue] plot(\x,{0.5*((\x)^2-4*(\x)+1)});
	\draw[fill=black] (-1,0) circle(1.5pt) (-1,3) circle(2pt) (0,3) circle(1.5pt) (0,-1.5) circle(1.5pt) (2,0) circle(1.5pt) (2,-1.5) circle(2pt) (4,0) circle(1.5pt) (4,0.5) circle(1.5pt);
	\draw[dashed] (-1,0)node[below]{\small$a$}--(-1,3)--(0,3)node[right]{\small$f(a)$} (2,0)node[above]{\small$x_0$}--(2,-1.5)--(0,-1.5)node[left]{\small$f(x_0)$}
	(4,0)node[below]{\small$b$}--(4,0.5);
	\node[above] at (-1,3) {\small $y_{\max}$};
	\node[below] at (2,-1.5) {\small $y_{\min}$};
\end{tikzpicture}}
\begin{note}
	\begin{listEX}[1]
		\item [\ding{172}] Khi yêu cầu tìm max min của hàm số mà không nói rõ xét trên tập nào, thì ta hiểu là tìm max min trên miền xác định của hàm số đó.
		\item [\ding{173}] Để tìm max min của hàm số $y=f(x)$ trên miền $\mathscr{D}$, ta thường lập bảng biến thiên của hàm số $y=f(x)$ trên $\mathscr{D}$. Từ bảng biến thiên, ta kết luận:
		\begin{itemize}
			\item [$\bullet$] Điểm ở vị trí cao nhất $\longrightarrow$ Kết luận max;
			\item [$\bullet$] Điểm ở vị trí thấp nhất $\longrightarrow$ Kết luận min.
		\end{itemize}
		\item [\ding{174}] Để tìm max min của hàm số $y=f(x)$ trên đoạn $[a;b]$ (\textit{$f(x)$ liên tục trên đoạn $[a ; b]$ và có đạo hàm trên $(a ; b)$ (có thể trừ một số hữu hạn các điểm) và $f^{\prime}(x)=0$ chỉ tại một số hữu hạn các điểm trong $(a ; b)$}), thì ta có thể giải như sau:
		\begin{itemize}
			\item [$\bullet$] Giải $f'(x)=0$ tìm các nghiệm $x_0 \in (a;b)$; 
			\item [$\bullet$] Tìm các điểm $x_i\in (a;b)$ mà tại đó đạo hàm không xác định (nếu có).
			\item [$\bullet$] Tính toán $f(a)$, $f(x_0)$, $f(x_i)$, $f(b)$ \quad ($\star$)
			\item [$\bullet$]  Gọi $M$, $n$ lần lượt là số lớn nhất và số nhỏ nhất của các kết quả tính toán ở bước ($\star$) thì
			$$M=\displaystyle\max_{[a;b]}f(x); \quad n=\displaystyle\min_{[a;b]}f(x)$$
		\end{itemize}
	\item [\ding{175}] Ta có thể dùng các bất đẳng thức có sẵn để đánh giá biểu thức cần tìm max, min. 
	% \begin{itemize}
	% 	\item [$\bullet$] Bất đẳng thức Cauchy cho hai số không âm $a$, $b$:
	% 	$$a+b \ge 2\sqrt{ab}$$
	% 	Dấu "=" xảy ra khi $a=b$.
	% 	\item [$\bullet$]  Bất đẳng thức Cauchy cho ba số không âm $a$, $b$, $c$:
	% 	$$a+b +c\ge 3\sqrt[3]{abc}$$
	% 	Dấu "=" xảy ra khi $a=b=c$.
	% 	\item [$\bullet$]  Bất đẳng thức Cauchy cho $n$ số không âm $a_1$, $a_2$,..., $a_n$:
	% 	$$a_1+a_2 +...+a_n \ge n\sqrt[n]{a_1a_2...a_n}$$
	% 	Dấu "=" xảy ra khi $a_1=a_2=...=a_n$.
	% \end{itemize}
	\end{listEX}
\end{note}
% \newpage
\subsection{PHÂN LOẠI VÀ PHƯƠNG PHÁP GIẢI TOÁN}
\begin{dang} {Bài toán tìm max, min của hàm số $y=f(x)$ trên miền $\mathscr{D}$}
	\begin{enumerate}[\iconMT]
		\item \indam{Phương pháp giải:} 
		\begin{listEX}[1]
			\item [\ding{172}] Tính $y'$. Giải phương trình $y'=0$ tìm các nghiệm $x_i \in \mathscr{D}$ và tìm các điểm $x_j \in \mathscr{D}$ mà tại đó $y'$ không xác định.
			\item [\ding{173}] Lập bảng biến thiên của hàm số trên $\mathscr{D}$.
			\item [\ding{174}] Từ bảng biến thiên, kết luận:
			\begin{itemize}
				\item [$\bullet$] Điểm ở vị trí cao nhất $\longrightarrow$ Kết luận max;
				\item [$\bullet$] Điểm ở vị trí thấp nhất $\longrightarrow$ Kết luận min.
			\end{itemize}
		\end{listEX}
		\item \indam{Lưu ý:} Nếu $\mathscr{D}$ là đoạn $\left[a;b\right]$ và hàm số $y=f(x)$ liên tục trên đoạn $\left[a;b\right]$ thì ta có thể làm như sau:
		\begin{listEX}[1]
			\item [\ding{172}] Giải $f'(x)=0$ tìm các nghiệm $x_0 \in (a;b)$;
			\item [\ding{173}] Tìm các điểm $x_i\in (a;b)$ mà tại đó đạo hàm không xác định (nếu có).
			\item [\ding{174}] Tính toán $f(a)$, $f(x_0)$, $f(x_i)$, $f(b)$ \quad ($\star$)
			\item [\ding{175}] Gọi $M$, $n$ lần lượt là số lớn nhất và số nhỏ nhất của các kết quả tính toán ở bước ($\star$) thì
			$$M=\displaystyle\max_{[a;b]}f(x); \quad n=\displaystyle\min_{[a;b]}f(x)$$
		\end{listEX}
		\begin{note}
			\begin{itemize}
				\item[\iconCH] Nếu hàm số $y=f(x)$ đồng biến trên đoạn $\left[a;b\right]$ thì $\min\limits_{[a;b]} f(x)=f(a)$ và $\max\limits_{[a;b]} f(x)=f(b)$.
				\item[\iconCH]  Nếu hàm số $y=f(x)$ nghịch biến trên đoạn $\left[a;b\right]$ thì $\min\limits_{[a;b]} f(x)=f(b)$ và $\max\limits_{[a;b]} f(x)=f(a)$.
			\end{itemize}
		\end{note}
	\end{enumerate}
\end{dang}

\boxmini{BÀI TẬP TỰ LUẬN}
\begin{vd}
	Tìm giá trị lớn nhất và nhỏ nhất (nếu có) của hàm số sau trên đoạn đã chỉ ra.
	\begin{tasks}(2)
		\task $f(x)=-x^3+3x^2+10$ trên đoạn $[-3;1]$.
		\task $f(x)=\dfrac{x^3}{3}-2x^2+3x+1$ trên đoạn $[-3;2]$.
		\task $f(x) = - 2x^4 + 4x^2 + 3$ trên đoạn $\left[0;2\right]$
		\task $f(x)=\dfrac{2x+3}{x+1}$ trên đoạn $[0;4]$.
		\task $f(x)=x+\dfrac{4}{x}$ trên khoảng $(0;+\infty)$;
		\task $f(x)=3x+\dfrac{4}{x^2}$ trên $(0;+\infty)$.
		\task $f(x)=\dfrac{2x^2+4x+5}{x^2+1}$ trên $\mathbb{R}$.
		\task $f(x)=\sqrt{-x^2+2x}$ trên miền xác định.
	\end{tasks}
\loigiai{
\begin{enumerate}[a)]
	\item Hàm số liên tục trên $[-3;1]$. Ta có $f'(x)=-3x^2+6x$; $f'(x)=0 \,\Leftrightarrow \hoac{&x=0 \in [-3;1]\\&x=2 \notin [-3;1]}$.\\
	Ta có $f(-3)=64$, $f(0)=10$, $f(1)=12$. Suy ra, $\max\limits_{[-3;1]} f(x)=f(-3)=64$; $\min\limits_{[-3;1]} f(x)=f(0)=10$.
	
	\item Hàm số liên tục trên $[-3;2]$
	Ta có $f^{\prime}(x)=x^2-4x+3$; $f^{\prime}(x)=0\Leftrightarrow \heva{&x=1\\x=3\notin[-3;2]}$.\\
	$f(1)=\dfrac{7}{3}$, $f(3)=-35$, $f(2)=\dfrac{5}{3}$.\\
	Vậy 
	$\max\limits_{[-3;2]}f(x)=\dfrac{7}{3}$ và
	$\min\limits_{[-3;2]}f(x)=-35$.	
	\item Ta có $f'(x)=- 8x^3 + 8x =- 8x(x^2 - 1) =- 8x(x - 1)(x + 1)$.\\
	Xét $f(0) = 3$, $f(1) = 5$ và $f(2) =- 13$.\\
	Vậy 
	$\max\limits_{[0;2]}f(x)=5$ và
	$\min\limits_{[0;2]}f(x)=-13$.	
	\item Hàm số đã cho liên tục trên đoạn $[0;4]$.\\
	Ta có $y'=-\dfrac{1}{(x+1)^2} < 0$, $\forall x \in [0;4]$. Suy ra hàm số đã cho nghịch biến trên đoạn $[0;4]$.\\
	Vậy $\max\limits_{[0;4]} y = y(0) = 3$ và $\min\limits_{[0;4]} y = y(4) = \dfrac{11}{5}$.
	
	\item Xét hàm số $f(x)=x+\dfrac{4}{x}$ trên khoảng $(0;+\infty)$.\\
	Đạo hàm $f'(x)=1-\dfrac{4}{x^2}=\dfrac{x^2-4}{x^2}$.\\
	Cho $f'(x)=0 \Leftrightarrow x^2-4=0 \Leftrightarrow \hoac{&x=2\in(0;+\infty)\\&x=-2\notin(0;+\infty).}$\\
	Bảng biến thiên
	\begin{center}
		\begin{tikzpicture}[font=\footnotesize,thick,>=stealth]
			\tikzset{double style/.append style={double distance=1.75pt}}
			\tkzTabInit[nocadre=false,lgt=1.2,espcl=2.5,deltacl=0.6,lw=.5pt,color,colorL=green!50,colorV=green!50]
			{$x$ /0.6,$f'(x)$ /0.6,$f(x)$ /2}
			{$-\infty$,$-2$,$0$,$2$,$+\infty$}
			\tkzTabLine{,+,$0$,-,d,-,$0$,+,}
			\tkzTabVar{-/$-\infty$,+/$-4$,-D+/$-\infty$/$+\infty$,-/$4$,+/$+\infty$}
			%\draw[pattern={Lines[angle=60,distance=1.25mm]},pattern color=blue,thin] (N11)--(N31)--(N33)--(N13);
		\end{tikzpicture}
	\end{center}
	Căn cứ vào bảng biến thiên, ta có $\min\limits_{(0;+\infty)}f(x)=4$.
	
	\item Áp dụng bất đẳng thức Cauchy cho $3$ số dương, ta có
	$$y=\dfrac{3x}{2}+\dfrac{3x}{2}+\dfrac{4}{x^2} \geq 3\sqrt[3]{\dfrac{3x}{2}\cdot \dfrac{3x}{2}\cdot \dfrac{4}{x^2}}=3\sqrt[3]{9}.$$
	Đẳng thức xảy ra khi $\dfrac{3x}{2}=\dfrac{4}{x^2} \Leftrightarrow x=\dfrac{2}{\sqrt[3]{2}}=2\sqrt[3]{2}$.
	
	\item Tập xác định $\mathscr{D}= \mathbb{R}$.\\
	Ta có $y'= \dfrac{-4x^2-6x+4}{(x^2+1)^2}$, \; $y'=0 \Leftrightarrow -4x^2-6x+4=0\Leftrightarrow \hoac{&x=-2\\&x=\dfrac{1}{2}.}$
	\begin{center}
		\begin{tikzpicture}\tkzTabInit[nocadre=false,lgt=1.2,espcl=2.5,deltacl=0.6]
			{$x$ /1, $y'$ /0.6, $y$ /2.5}
			{$-\infty$,$-2$,$\dfrac{1}{2}$,$+\infty$}
			\tkzTabLine{,-,$0$,+,$0$,-,}
			\tkzTabVar{+/$2$,-/$1$,+/$6$,-/$2$}
		\end{tikzpicture}
	\end{center}
	Suy ra $M=6$ và $m=1$.
	
	\item Hàm số $f(x)=\sqrt{-x^2+2x}$ liên tục trên $[0;2]$.\\
	$f'(x)=\dfrac{1-x}{\sqrt{-x^2+2x}}$, $f'(x)=0 \Leftrightarrow x=1$.\\
	Ta có $f(0)=0$, $f(2)=0$, $f(1)=1$.\\
	Vậy $\displaystyle\max_{x\in [0;2]}f(x)=1$ và $\displaystyle\min_{x\in [0;2]}f(x)=0$.
\end{enumerate}}
\end{vd}
\dongcham{54}
\begin{vd}
	Tìm giá trị lớn nhất và nhỏ nhất của hàm số sau trên miền đã chỉ ra.
	\begin{tasks}(2)
		\task $y=x-\sin 2x$ trên đoạn $\left[-\dfrac{\pi}{2};\pi\right]$
		\task $y = \mathrm{e}^{x^3 - 3x + 3}$ trên đoạn $[0; 2]$
		\task $y=\mathrm{e}^{x}(x^{2}-3)$ trên đoạn $[-2;2]$
		\task $y=\dfrac{\ln^2x}{x}$ trên đoạn $\left[1;\mathrm{e}^5\right]$
	\end{tasks}
	\loigiai{
		\begin{enumerate}[a)]
			\item Ta có
			\begin{itemize}
				\item $y'=1-2\cos 2x$.
				\item $\heva{&x\in \left(-\dfrac{\pi}{2};\pi\right)\\& y'=0}\Leftrightarrow \heva{&x\in \left(-\dfrac{\pi}{2};\pi\right)\\&\cos 2x=\dfrac{1}{2}}\Leftrightarrow \heva{&x\in \left(-\dfrac{\pi}{2};\pi\right)\\&x=\pm \dfrac{\pi}{6}+k\pi} \Leftrightarrow \hoac {&x=\pm\dfrac{\pi}{6}\\& x=\dfrac{5\pi}{6}.}$
				\item $f\left(-\dfrac{\pi}{2}\right)=-\dfrac{\pi}{2}$,  $f(\pi)=\pi$,
				$f\left(-\dfrac{\pi}{6}\right)=-\dfrac{\pi}{6}+\dfrac{\sqrt{3}}{2}$,  $f\left(\dfrac{\pi}{6}\right)=\dfrac{\pi}{6}-\dfrac{\sqrt{3}}{2}$,  $f\left(\dfrac{5\pi}{2}\right)=\dfrac{5\pi}{6}+\dfrac{\sqrt{3}}{2}$.
			\end{itemize}
			Vậy giá trị lớn nhất và giá trị nhỏ nhất của hàm số $y=x-\sin 2x$ trên đoạn $\left[-\dfrac{\pi}{2};\pi\right]$ lần lượt là $\dfrac{5\pi+3\sqrt{3}}{6}$ và $-\dfrac{\pi}{2}$.
			\item Ta có $y' = (3x^2 - 3)\cdot \mathrm{e}^{x^3 - 3x + 3}$.\\
			$y' = 0 \Leftrightarrow 3x^2 - 3 = 0 \Leftrightarrow x = 1$ do $x \in [0;2]$.\\
			Khi đó $y(0) = \mathrm{e}^3$; $y(2) = \mathrm{e}^5$; $y(1) = \mathrm{e}$.
			Vậy $ \max \limits_{[0; 2]} y = \mathrm{e}^5 $ khi $x = 2$.
			\item Ta có $y'=\mathrm{e}^x(x^2+2x-3)=0\Leftrightarrow \hoac{&x=-3\\ &x=1}$.
			Xét các giá trị: $f(-2)=\mathrm{e}^{-2}$; $f(1)=-2\mathrm{e}$; $f(2)=\mathrm{e}^2$, từ đó suy ra $y_{\min}=-2\mathrm{e}$.
			\item $y'=\dfrac{2\ln x-\ln^2x}{x^2}$, $y'=0\Leftrightarrow \hoac{
				& \ln x=0 \\
				& \ln x=2 \\} \Leftrightarrow \hoac{
				& x=1 \\
				& x=\mathrm{e}^2. \\}$\\
			Tính $y(1)=0$, $y(\mathrm{e}^2)=\dfrac{4}{\mathrm{e}^2}\approx 0{,}54$, $y(\mathrm{e}^5)=\dfrac{9}{\mathrm{e}^5}\approx 0{,}16$.\\
			Vậy $\max\limits_{x \in \left[1;\mathrm{e}^5\right]}y=\dfrac{4}{\mathrm{e}^2}$
	\end{enumerate}}
\end{vd}
\dongcham{40}
\begin{vd}
	Tìm giá trị lớn nhất và nhỏ nhất (nếu có) của hàm số sau trên miền đã chỉ ra.
	\begin{tasks}(2)
		\task $f(x)=\dfrac{5\sin x+1}{\sin x+2}$ trên đoạn $\left[0;\dfrac{\pi}{6}\right]$.
		\task $ y=\cos^3x +2\sin^2x+\cos x$ trên miền xác định.
	\end{tasks}
	\loigiai{
		\begin{enumerate}[a)]
			\item Đặt $t=\sin x,\; x\in \left[0;\dfrac{\pi}{6}\right]\Rightarrow t \in \left[0;\dfrac{1}{2}\right]$.\\
			Ta được hàm số $y=g(t)=\dfrac{5t+1}{t+2}$.\\
			$g'(t)=\dfrac{9}{(t+2)^2}>0,\forall t \in \left[0;\dfrac{1}{2}\right]$.\\
			Vì $g(0)=\dfrac{1}{2}$, $g\left(\dfrac{1}{2}\right)=\dfrac{7}{5}$ nên $\min\limits_{\left[0;\tfrac{1}{2}\right]}g(t)=g(0)=\dfrac{1}{2}.$\\
			Vậy $\min\limits_{\left[0;\tfrac{\pi}{6}\right]}f(x)=\min\limits_{\left[0;\tfrac{1}{2}\right]}g(t)=\dfrac{1}{2}$
			\item Ta có $ y=\cos^3x +2\sin^2x+\cos x =\cos^3x +2(1-\cos^2x)+\cos x =\cos^3x-2\cos^2x+\cos x+2$.\\
			Đặt $ t=\cos x,\, t\in [-1;1] $. Ta được $ f(t)=t^3-2t^2 +t+2$.\\
			Ta có $ f'(t)=3t^2-4t+1;\,y'=0\Leftrightarrow \hoac{&t=1\in[-1;1]\\&t=\dfrac{1}{3}\in[-1;1].} $\\
			Mà $f\left(-1\right)=-2$, $ f\left(\dfrac{1}{3}\right)=\dfrac{58}{27} $, $f(1)=2$ nên $\max\limits_{x\in \mathbb{R}}y=\max\limits_{\left[-1;1\right]}f(t)=\dfrac{58}{27}$
			\item
			\item
	\end{enumerate}}
\end{vd}
\dongcham{43}
\boxmini{BÀI TẬP TRẮC NGHIỆM}
\ind{PHẦN I.} \inden{Câu trắc nghiệm nhiều phương án lựa chọn. Mỗi câu hỏi học sinh chỉ chọn một phương án.}\\
\setcounter{ex}{0}
\Opensolutionfile{ans}[ans/2D1-B2-d1-1]

\begin{ex}
	\immini
	{Hàm số $y=f(x)$ liên tục trên đoạn $[-1;3]$ và có bảng biến thiên như sau.\\
		Gọi $M$ là giá trị lớn nhất của hàm số $y=f(x)$ trên đoạn $[-1;3]$. Khẳng định nào sau đây là khẳng định đúng?
		\choice
		{\True $M=f(0)$}
		{$M=f(-1)$}
		{$M=f(3)$}
		{$M=f(2)$}
	}
	{\begin{tikzpicture}
			\tkzTabInit[nocadre=True,lgt=1.2,espcl=2]
			{$x$ /0.7,$y'$ /0.7,$y$ /2.1}
			{$-1$,$0$,$2$,$3$}
			\tkzTabLine{,+,$0$,-,$0$,+,}
			\tkzTabVar{-/$0$, +/$5$,-/$1$,+/$4$}
	\end{tikzpicture}}
	\loigiai
	{Dựa vào bảng biến thiên ta có $M=f(0)=5$.}
\end{ex} \dongcham{1}

\begin{ex}
	\immini{Cho hàm số $f(x)$ liên tục trên đoạn $[-1;5]$ và có đồ thị như hình vẽ bên. Gọi $M$ và $m$ lần lượt là giá trị lớn nhất và nhỏ nhất của hàm số đã cho trên $[-1;5]$. Giá trị của $M+m$ bằng
		\choice
		{$5$}
		{$6$}
		{$3$}
		{\True $1$}
	}{
		\begin{tikzpicture}[scale=0.65, font=\footnotesize, line join=round, line cap=round, >=stealth]
			%%Nhập giới hạn đồ thị và hàm số cần vẽ
			\def \xmin{-1.5}
			\def \xmax{6.3}
			\def \ymin{-2.8}
			\def \ymax{4}
			%%Tự động
			\draw[->] (\xmin,0)--(\xmax,0) node[below left] {$x$};
			\draw[->] (0,\ymin)--(0,\ymax) node[below left] {$y$};
			\draw[fill=black] (0,0) circle(1pt) node [below right] {$O$};
			%%Vẽ các điểm trên 2 hệ trục
			\foreach \x in {3,4,5}
			\draw[fill=black] (\x,0) circle(1pt) node [below] {$\x$};
			\foreach \x in {-1,2}
			\draw[fill=black] (\x,0) circle(1pt) node [above] {$\x$};
			\foreach \y in {-2,1,3}
			\draw[fill=black] (0,\y) circle(1pt) node [above right] {$\y$};
			\draw[dashed](-1,0)--(-1,-2)--(0,-2)--(2,-2)--(2,0) (5,0)--(5,1)--(0,1) (3,0)--(3,1) (4,0)--(4,3)--(0,3);
			%%Tự động
			\draw
			(-1.1,-2.7) to[out=80, in=-100] (-1,-2)
			..controls +(80:1.2) and +(180:.5)..(0,1)
			..controls +(0:.6) and +(180:0.7)..(2,-2)
			..controls +(0:0.4) and +(-100:1.2)..(2.8,0)
			to[in=80, out=-100] (3,1)
			..controls +(75:1.5) and +(180:0.3)..(4,3)
			..controls +(0:0.5) and +(-75:1)..(5,1)
			to[in=105, out=-75] (6,-2.7);
			\fill[black]
			(-1,-2) circle(1pt)
			(2,-2) circle(1pt)
			(3,1) circle(1pt)
			(4,3) circle(1pt)
			(5,1) circle(1pt)
			;
		\end{tikzpicture}
	}
	\loigiai{
		Dựa vào đồ thị, suy ra $m=\min\limits_{[-1;5]} f(x)=f(-1)=-2$, $M=\max\limits_{[-1;5]} f(x)=f(4)=3$.\\
		Do đó $M+m=3-2=1$.
	}
\end{ex} \dongcham{1}

\begin{ex}
	\immini{Cho hàm số $y=f(x)$ có đồ thị là đường cong ở hình bên. Tìm giá trị nhỏ nhất $m$ của hàm số $y=f(x)$ trên đoạn $[-1;1] $.
		\haicot
		{$m=2 $}
		{\True $m=-2 $}
		{$m=1 $}
		{$m=-1 $}}{\vspace{-0.5cm}
		\begin{tikzpicture}[smooth,samples=300,scale=0.68,>=stealth]
			\draw[->] (-2.3,0)--(2.3,0) node[below]{$x$};
			\draw[->] (0,-2.5)--(0,2.5) node[right]{$y$};
			\draw (0,0) node[above right]{$O$};
			\draw[line width = 1pt,domain=-2:2] plot(\x,{(\x)^(3)-3*(\x)});
			\draw[fill=black] (-1,2) circle(1.5pt) (1,-2) circle(1.5pt);
			\draw[dashed] (-1,0)node[below]{\small$-1$}--(-1,2)--(0,2)node[right]{\small$2$} (1,0)node[above]{\small$1$}--(1,-2)--(0,-2)node[left]{\small$-2$};
	\end{tikzpicture}}
	\loigiai{
		Dựa vào đồ thị ta có giá trị nhỏ nhất của hàm số trên đoạn $[-1;1] $ bằng $-2$.	
		
	}
	
\end{ex} \dongcham{1}

\begin{ex}
	Cho hàm số $y=f(x)$ có bảng biến thiên trên đoạn $[-2;3]$ như hình bên dưới.
	\begin{center}
		\begin{tikzpicture}[>=stealth,scale=1]
			\tkzTabInit[nocadre=false,lgt=1.2,espcl=2,deltacl=0.6]
			{$x$/0.6,$f’(x)$/0.6,$f(x)$/2}
			{$-\infty$,$-2$,$-1$,$1$,$3$,$+\infty$}
			\tkzTabLine{,h,,+,z,-,d,+,,h}
			\tkzTabVar{+H/,-/$0$,+/$1$,-/$-2$,+H/$5$}
		\end{tikzpicture}
	\end{center}
	Gọi $M$ và $m$ lần lượt là giá trị lớn nhất và giá trị nhỏ nhất của hàm số đã cho trên đoạn $[-1;3]$. Giá trị của biểu thức $M-m$ là
	\choice
	{$5$}
	{\True $7$}
	{$-1$}
	{$3$}
	\loigiai{
		Dựa vào bảng biến thiên ta thấy giá trị lớn nhất của hàm số là $M=5$ và giá trị nhỏ nhất của hàm số là $m=-2$ nên $M-m=7$.}
\end{ex}
% \newpage
\begin{ex}
	Giá trị lớn nhất và nhỏ nhất của hàm số $y=x^3-12x+1$ trên đoạn $[-2;3]$ lần lượt là
	\choice
	{\True $17$, $-15$}
	{$10$, $-26$}
	{$-15$, $17$}
	{$6$, $-26$}
	\loigiai{
		Ta có $y'=3x^2-12$, do đó $y'=0\Leftrightarrow 3x^2-12=0\Leftrightarrow x=\pm 2\in [-2;3]$.\\
		Mặt khác $f(-2)=17$, $f(2)=-15$, $f(3)=-8$.\\
		Vậy giá trị lớn nhất và nhỏ nhất cần tìm lần lượt là $17$, $-15$.
	}
\end{ex} \dongcham{12}

\begin{ex}
	Gọi $ M, m $ lần lượt là giá trị lớn nhất và giá trị nhỏ nhất của hàm số $ y = x^3 + 3x^2 - 9x + 1 $ trên $ [-4;4] $. Tính tổng $ M + m. $
	\choice 
	{$ 12 $}
	{$ 98 $}
	{$ 17 $}
	{\True $ 73 $}
	\loigiai
	{
		Ta có $ y' = 3x^2 + 6x - 9 = 0 \Leftrightarrow \hoac{&x = 1\\ &x = -3.} $\\
		Khi đó: $ y(-4) = 21 $,\, $ y(-3) = 28, $
		\, $ y(1) = -4, $
		\, $ y(4) = 77. $\\
		Do đó $ M + m = 77 + (-4) = 73. $
	}
\end{ex} \dongcham{12}

\begin{ex}
	Giá trị lớn nhất của hàm số $f(x)=-x^4+12x^2+1$ trên đoạn $\left[ -1;2\right] $ bằng
	\choice
	{\True $33$}
	{$37$}
	{$12$}
	{$1$}
	\loigiai{
		Hàm số $f(x)=-x^4+12x^2+1$ liên tục trên đoạn $\left[ -1;2\right] $.\\
		Ta có $f'(x)=-4x^3+24x=-4x(x^2-6)$.\\
		$f'(x)=0 \Leftrightarrow \hoac{& x=-\sqrt{6} \not \in \left[ -1;2\right] \\ &x=0 \in \left[ -1;2\right] \\&x=\sqrt{6} \not \in \left[ -1;2\right]. }$\\
		Ta có $f(-1)=12; f(0)=1; f(2)=33$.\\
		Vậy $\max\limits_{\left[ -1;2\right] } f(x)=33$.
	}
\end{ex} \dongcham{12}
\begin{ex}
	Giá trị lớn nhất của hàm số $y=x^4-3x^2+2$ trên đoạn $\left[ 0;3\right] $ bằng
	\choice
	{ $ 57 $}
	{\True $ 56 $}
	{$ 54$}
	{$ 55 $}
	\loigiai{
		Hàm số $y$ liên tục trên đoạn $\left[ 0;3\right] $ và có đạo hàm $y'=4x^3-6x$.\\
		Ta có $y'=0 \Leftrightarrow 4x^3-6x=0 \Leftrightarrow \hoac{& x=0 \in \left[ 0;3\right]  \\&x=\sqrt{\dfrac{3}{2}} \in \left[ 0;3\right]\\ & x=- \sqrt{\dfrac{3}{2}}\notin \left[ 0;3\right].}$\\
		Ta có $y(0)=2$, $y(3)=56$, $y\left(\sqrt{\dfrac{3}{2}}\right) =-\dfrac{1}{4} $.\\
		Do đó giá trị lớn nhất của hàm số $y=x^4-3x^2+2$ trên đoạn $\left[ 0;3\right] $ bằng $56$.
	}
\end{ex} \dongcham{7}

\begin{ex}%[2D1Y3-1]
	Giá trị nhỏ nhất của hàm số $y=\dfrac{x-1}{x+1}$ trên đoạn $[0;3]$ là
	\choice
	{$\min\limits_{[0;3]}y=\dfrac{1}{2}$}
	{$\min\limits_{[0;3]}y=-3$}
	{$\min\limits_{[0;3]}y=1$}
	{\True $\min\limits_{[0;3]}y=-1$}	
	\loigiai{
		Trên đoạn $[0;3]$ hàm số luôn xác định.\\
		Ta có $y'=\dfrac{2}{(x+1)^2}>0$, $\forall x \in [0;3]$ nên hàm số đã cho đồng biến trên đoạn $[0;3]$.\\
		Do đó $\min\limits_{[0;3]}y=y(0)=-1$.
	}	
\end{ex} \dongcham{7}

\begin{ex}%
	Giá trị nhỏ nhất của hàm số $y=\dfrac{2x+3}{x+1}$ trên đoạn $[0;4]$ là
	\choice
	{$2$}
	{$\dfrac{7}{5}$}
	{$3$}
	{\True $\dfrac{11}{5}$}
	\loigiai
	{
		Ta có $y'=\dfrac{-1}{(x+1)^2}<0$ nên $\min\limits_{[0;4]} y=y(4)=\dfrac{11}{5}$.
	}
\end{ex} \dongcham{7}

\begin{ex}%[2D1B3]
	Giá trị lớn nhất của hàm số $y=\dfrac{x^2-3x+3}{x-1}$ trên đoạn $\left[-2;\dfrac{1}{2}\right]$ bằng
	\choice
	{$4$}
	{\True $-3$}
	{$-\dfrac{7}{2}$}
	{$-\dfrac{13}{3}$}
	\loigiai{
		Ta có $y'=\dfrac{x^2-2x}{(x-1)^2}$. Xét $y'=0\Leftrightarrow  x^2-2x=0\Leftrightarrow \hoac{&x=0\in \left[-2;\dfrac{1}{2}\right]\\&x=2\notin\left[-2;\dfrac{1}{2}\right]}$.\\
		Ta có $y(0)=-3$, $y(-2)=\dfrac{-13}{3}$, $y\left(\dfrac{1}{2}\right)=\dfrac{-7}{2}$.\\
		Suy ra $\underset{x\in \left[-2;\dfrac{1}{2}\right]}{\max y}=-3$
	}
\end{ex} \dongcham{10}

\begin{ex}
	Giá trị lớn nhất của hàm số $y=\sqrt{4-x^2}$ là
	\choice
	{$M=-2$}
	{\True $M=2$ }
	{$M=4$}
	{$M=0$}
	\loigiai
	{
		Tập xác định: $\mathscr{D}=\left[-2;2\right]$.\\
		Đạo hàm $y'=\dfrac{-x}{\sqrt{4-x^{2}}}$; $y'=0 \Leftrightarrow x=0 \in \left[-2;2\right]$.\\
		Ta có $y(2)=y(-2)=0$; $y(0)=2$.\\
		Vậy giá trị lớn nhất của hàm số đã cho bằng $2$.
	}
\end{ex} \dongcham{8}

\begin{ex}%[2D1B3-1]
	Tìm giá trị lớn nhất $M$ của hàm số $y=\sqrt{7+6x-x^2}$.
	\choice
	{\True $M=4$}
	{$M=\sqrt{7}$}
	{$M=7$}
	{$M=3$}
	\loigiai{
		Tập xác định $\mathscr{D}=[-1;7]$.\\
		$y'=\dfrac{-x+3}{\sqrt{7+6x-x^2}}$.\\
		Cho $y'=0\Leftrightarrow x= 3\in \mathscr{D}$.\\
		Có $y(3)=4, y(-1)=0, y(7)=0$. Vậy $M=4$.	
	}
\end{ex} \dongcham{8}

\begin{ex}%[Nguyễn Quang Hiệp - Phát triển đề minh họa 2021]%[2D2B4-4]%
	Tính giá trị lớn nhất của hàm số $y=x-\ln x$ trên $\left[\dfrac{1}{2};\mathrm{e}\right]$.\\
	\choice
	{$\max\limits_{x \in \left[\frac{1}{2};\mathrm{e}\right]}y=1$}
	{\True $\max\limits_{x \in \left[\frac{1}{2};\mathrm{e}\right]}y=\mathrm{e}-1$}
	{$\max\limits_{x \in \left[\frac{1}{2};\mathrm{e}\right]}y=\mathrm{e}$}
	{$\max\limits_{x \in \left[\frac{1}{2};\mathrm{e}\right]}y=\dfrac{1}{2}+\ln 2$}
	\loigiai{
		Hàm số $y=x-\ln x$liên tục trên đoạn $\left[\dfrac{1}{2};\mathrm{e}\right]$.\\
		Ta có $y'=1-\dfrac{1}{x}\Rightarrow y'=0\Leftrightarrow x=1\in \left[\dfrac{1}{2};\mathrm{e}\right]$.\\
		Do $y\left(\dfrac{1}{2}\right)=\dfrac{1}{2}+\ln 2$; $y(\mathrm{e})=\mathrm{e}-1$; $y(1)=1$ nên $\max\limits_{x \in \left[\frac{1}{2};\mathrm{e}\right]}y=\mathrm{e}-1$.}
\end{ex} \dongcham{8}

\begin{ex}
	Gọi $M, N$ lần lượt là giá trị lớn nhất và nhỏ nhất của hàm số $y = x^2 - 4\ln (1 - x)$ trên đoạn $[-2;0]$. Tính $M - N$.
	\choice
	{$M - N = 4\ln 2$}
	{$M - N = -1$}
	{\True $M - N = 4\ln 2 -1$}
	{$M - N = 4\ln 3 -4$}
	\loigiai{
		Tập xác định: $\mathscr{D} = (-\infty;1)$.\\
		Ta có $y' = 2x + \dfrac{4}{1 - x} = \dfrac{-2x^2 + 2x + 4}{1 - x}$.\\
		Khi đó $y' = 0 \Leftrightarrow -2x^2 + 2x + 4 = 0 \Leftrightarrow \hoac{&x = -1 \quad \mbox{(nhận)} \\&x = 2 \quad \mbox{(loại)}. }$\\
		Khi đó $\heva{& y(-2) = 4 - 4\ln 3 \approx -0{,}4 \\& y(-1) = 1 - 4\ln 2  \approx -1{,}7\\& y(0) = 0.} \Rightarrow M = 0, N = 1 -4\ln 2$\\
		Vậy $M - N = 4\ln 2 -1$.
	}
\end{ex} \dongcham{8}

\begin{ex}
	Cho hàm số $f(x)$ nghịch biến trên $\mathbb{R}$. Giá trị nhỏ nhất của hàm số $g(x)=\mathrm{e}^{3x^2-2x^3}-f(x)$ trên đoạn $[0;1]$ bằng
	\choice
	{$\mathrm{e}-f(1)$}
	{$f(1)$}
	{$f(0)$}
	{\True $1-f(0)$}
	\loigiai{
		Ta có $g'(x)=(6x-6x^2)\mathrm{e}^{3x^2-2x^3}-f'(x)$.\\
		Trên đoạn $[0;1]$ thì $6x-6x^2\ge 0$, $f'(x)\le 0$ nên $g'(x)\ge 0$, suy ra hàm số $g(x)$ đồng biến, suy ra giá trị nhỏ nhất là $g(0)=1-f(0)$.
	}
\end{ex} \dongcham{8}

\begin{ex}
	\immini{Cho hàm số $y=f(x)$ xác định và liên tục trên đoạn $\left[0;\dfrac{7}{2}\right]$, có
		đồ thị của hàm số $y=f'(x)$ như hình vẽ. Hỏi hàm số $y=f(x)$ đạt giá trị nhỏ nhất trên đoạn $\left[0;\dfrac{7}{2}\right]$ tại điểm $x_0$ nào dưới đây?
		\choice
		{\True $x_0=3$}
		{$x_0=2$}
		{$x_0=1$}
		{$x_0=0$}
	}
	{\begin{tikzpicture}[scale=1,font=\footnotesize, line join=round,line cap=round,>=stealth]
			\draw [->] (-1,0)--(0,0)
			node[below left]{$O$}--(4.5,0)node[below]{$x$}; % Hệ trục tọa độ
			\draw[->] (0,-1.5) --(0,2) node[left]{$y$};
			\draw[dashed](3.5,0)node[below]{$\tfrac{7}{2}$}--(3.5,25/16);
			\draw(1,0)node[above]{$1$}(3,0)node[above left]{$3$};
			\draw [domain=0:3.5,samples=100] plot (\x, {(\x-1)^2*(\x-3)/2});
	\end{tikzpicture}}
	\loigiai{
		Từ đồ thị hàm số ta có $f'(x)=0 \Leftrightarrow \hoac{&x=1\\&x=3.}$\\
		Bảng biến thiên của hàm số $y=f(x)$ trên đoạn $\left[0;\dfrac{7}{2}\right]$
		\begin{center}
			\begin{tikzpicture}
				\tkzTabInit[nocadre=false,lgt=1.2,espcl=2.5,deltacl=0.7]{$x$ / 1.1 , $f’(x)$ /0.7, $f(x)$ / 2}
				{$0$,$1$, $3$ , $\dfrac{7}{2}$}%
				\tkzTabLine{,-,0,-,0,+,}%
				\tkzTabVar{+ /$f(0)$,R/,-/$f(3)$,+ / $f\left(\dfrac{7}{2}\right)$}%
				\tkzTabIma{1}{3}{2}{$f(0)$}
			\end{tikzpicture}
		\end{center}
		Từ bảng biến thiên ta có hàm số $y=f(x)$ đạt giá trị nhỏ nhất trên đoạn $\left[0;\dfrac{7}{2}\right]$ tại điểm $x_0=3.$
	}
\end{ex} \dongcham{10}

\begin{ex}%[2D1K3-1]
	\immini{Cho hàm số $y=f(x)$, biết hàm số $y=f'(x)$ có đồ thị như hình vẽ dưới đây. Hàm số $y=f(x)$ đạt giá trị nhỏ nhất trên đoạn $\left[\dfrac{1}{2};\dfrac{3}{2} \right]$ tại điểm nào sau đây?
		\choicew{0,25 \textwidth}
		\choice
		{$x=\dfrac{3}{2}$}
		{$x=\dfrac{1}{2}$}
		{\True $x=1$}
		{$x=0$}}{\vspace{-0.5cm}\begin{tikzpicture}[>=stealth,scale=1.5]
			\draw[->] (-0.5,0)--(2.5,0) node[below]{\footnotesize $x$};
			\draw[->] (0,-0.5)--(0,1.5) node[right]{\footnotesize $y$};
			\draw (0,0) node[below left]{\footnotesize $O$};
			\draw[line width = 1pt,smooth,domain=-0.4:1.7] plot({\x},{(\x)^2-\x});
			\draw[fill=black] (1.5,0.75) circle(1pt);
			\draw [dashed] (1.5,0)
			node[below]{\footnotesize$\dfrac{3}{2}$}--(1.5,0.75)--(0,0.75)(1,0)node[below]{$1$};
	\end{tikzpicture}}
	\loigiai{
		Dựa vào đồ thị hàm số $y=f'(x)$. Ta có bảng biến thiên
		\begin{center}\begin{tikzpicture}
				\tkzTabInit[nocadre=false,lgt=1,espcl=2.1]
				{$x$ /1,$y'$ /0.6,$y$ /2}
				{$\dfrac{1}{2}$,$1$,$\dfrac{3}{2}$}
				\tkzTabLine{,-,$0$,+,$0$,}
				\tkzTabVar{+/, -/,+/}
			\end{tikzpicture}
		\end{center}
		Suy ra hàm số đạt giá trị nhỏ nhất trên $\left[\dfrac{1}{2};\dfrac{3}{2} \right]$ tại $x=1$.}
\end{ex} \dongcham{10}

\begin{ex}
	\immini{ Cho hàm số $f(x)$ có đồ thị của hàm số $y=f'(x)$ như hình vẽ. Biết $f(0)+f(1)-2f(2)=f(4)-f(3)$. Giá trị nhỏ nhất $m$, giá trị lớn nhất $M$ của hàm số $f(x)$ trên đoạn $[0;4]$ là
		\choice
		{$m=f(4)$, $M=f(1)$}
		{\True $m=f(4)$, $M=f(2)$}
		{$m=f(1)$, $M=f(2)$}
		{$m=f(0)$, $M=f(2)$}
	}{
		\begin{tikzpicture}[scale=0.79, >=stealth]
			\draw[->] (-0.6,0.) -- (5.3,0.);
			\draw[->] (0.,-1.7) -- (0.,1.6);
			\draw[dashed] (4,0) -- (4,-1.2);
			\clip(-0.6,-1.7) rectangle (5.3,1.7);
			\draw[smooth,samples=100,domain=0:2] plot(\x,{-0.8*((\x)^2-2*(\x))});
			\draw[smooth,samples=100,domain=2:4.5] plot(\x,{0.2*(((\x)-2)*((\x)-7)});
			\draw (-0.3,-0.3) node {$O$} (5.2,0.3) node {$x$} (0.35,1.5) node {$y$} (1.9,-0.3) node {$2$} (4.0,0.3) node {$4$} (2.0,1.15) node {$y=f'(x)$};
			\fill (0,0) circle(1pt) (2,0) circle(1pt) (4,0) circle(1pt); 
		\end{tikzpicture}
	}
	\loigiai{
		Từ đồ thị hàm số $y=f'(x)$ ta suy ra $f'(x)=0 \Leftrightarrow \hoac{&x=0\\&x=2.}$\\
		Ta có bảng biến thiên: 
		\begin{center}\begin{tikzpicture}[>=stealth,scale=1]
				\tkzTabInit[lgt=1.2,espcl=2.5]
				{$x$/1,$f'(x)$/1,$f(x)$/2.5}
				{$0$,$2$,$4$}
				\tkzTabLine{$0$,+,$0$,- }
				\tkzTabVar{-/$f(0)$,+/$f(2)$,-/$f(4)$}
		\end{tikzpicture}\end{center}
		Từ bảng biến thiên ta thấy $M=f(2)$.\\
		Mặt khác, từ bảng biến thiên ta có $\heva{&f(1)<f(2)\\&f(3)<f(2)}\Rightarrow f(1)+f(3)<2f(2)$.\\
		Do đó $f(4)=f(0)+f(1)+f(3)-2f(2)<f(0)+f(2)+f(2)-2f(2)=f(0) \Rightarrow m=f(4)$.	
	}
\end{ex} \dongcham{10}


\begin{ex}
	Giá trị lớn nhất, giá trị nhỏ nhất của hàm số $y={\sin}^3x-3{\sin}^2x+2$ lần lượt là $M$, $m$. Tổng $M+m$ bằng
	\choice
	{\True $0$}
	{$4$}
	{$1$}
	{$3$}
	\loigiai{
		Đặt $t=\sin x \, (-1\le t\le 1)$. Ta có $y=f(t)=t^3-3t^2+2 \, (-1\le t\le 1)$.\\$y'=3t^2-6t=0\Leftrightarrow\hoac{&t=0\in \left[-1;1\right]\\&t=2\notin \left[-1;1\right].}$\\
		Ta có $f(-1)=-2,\,f(1)=0, \,f(0)=2$. Vậy $M=2$ và $m=-2\Rightarrow M+m=0$.}
\end{ex} \dongcham{11}

\begin{ex}
	Giá trị nhỏ nhất của hàm số $f(x)=(x+1)(x+2)(x+3)(x+4)+2019$ là
	\choice
	{$2017$}
	{$2020$}
	{\True $2018$}
	{$2019$}
	\loigiai{
		Tập xác định $\mathscr{D}=\mathbb{R}$.\\
		Biến đổi $f(x)=(x+1)(x+2)(x+3)(x+4)+2019=(x^2+5x+4)(x^2+5x+6)+2019$.\\
		Đặt $t=x^2+5x+4\Rightarrow t=\left( x+\dfrac{5}{2} \right)^2-\dfrac{9}{4}\Rightarrow t\ge-\dfrac{9}{4}\,\forall x\,\in \,\mathbb{R}$.\\
		Hàm số đã cho trở thành $f(x)=t^2+2t+2019=(t+1)^2+2018\ge 2018 \,\forall t\ge -\dfrac{9}{4}$.\\
		Vậy giá trị nhỏ nhất của hàm số đã cho bằng $2018$ tại $t=-1\in \left[-\dfrac{9}{4};+\infty \right)$.
	}
\end{ex} \dongcham{11}

\Closesolutionfile{ans}

\ind{PHẦN II.} \inden{Câu trắc nghiệm đúng sai. Trong mỗi ý a), b), c), d) ở mỗi câu, học sinh chọn đúng hoặc sai.}\\
\Opensolutionfile{ans}[ans/2D1-B2-d1-2]

\begin{ex}%[2D1Y3]
		Cho hàm số $y=f(x)$ là hàm số liên tục trên $\mathbb{R}$ và có bảng biến thiên như hình vẽ dưới đây. 
		\begin{center}
			\begin{tikzpicture}
				\tkzTabInit[nocadre=false, lgt=1.2, espcl=1.5]{$x$ /0.6,$f'(x)$ /0.6,$f(x)$ /1.7}{$-\infty$,$-1$,$0$,$1$,$+\infty$}
				\tkzTabLine{,+,$0$,-,$0$,+,$0$,-,}
				\tkzTabVar{-/ $-\infty$ ,+/$4$,-/$3$,+/$4$,-/$-\infty$}
			\end{tikzpicture}
		\end{center}
	Xét tính đúng, sai của các khẳng định sau:
		\choiceTF
		{\True Cực đại của hàm số là $4$}
		{\True Cực tiểu của hàm số là $3$}
		{\True $\max\limits_{\mathbb{R}}{y}=4$}
		{$\min\limits_{\mathbb{R}}{y}=3$}
	\loigiai{
		Tử bảng biến thiên ta thấy $\lim\limits_{x\to+\infty}f(x)=-\infty$ nên hàm số không có giá trị nhỏ nhất trên $\mathbb{R}.$}
\end{ex} \dongcham{8} 

\begin{ex}
	Hình bên cho biết sự thay đổi của nhiệt độ ở một thành phố trong một ngày. Xét tính đúng, sai của các khẳng định sau:
	\begin{center}
		\begin{tikzpicture}[>=stealth,x=0.25cm,y=0.15cm]
			\draw[->] (-2,0)--(0,0) node[below left]{$O$}--(28,0) node[below right]{$x$ (giờ)};
			\draw[->] (0,-4)--(0,40) node[left]{$t$ ($^\circ C$)};
			\foreach \x/\g in {4/-90,8/-90,12/-90,16/-90,20/-90,24/-90}
			\draw[thin] (\x,2pt)--(\x,-2pt) + (\g:3mm) node {$\x$};
			%Vẽ các điểm trên trục Oy
			\foreach \y/\g in {25/180}
			\draw[thin] (2pt,\y)--(-2pt,\y) + (\g:3mm) node {$\y$};
			\path
			(0,25) coordinate (25)
			(4,20) coordinate (20)
			(8,31) coordinate (31)
			(12,28) coordinate (28)
			(16,34) coordinate (34)
			(20,27) coordinate (27)
			(24,24) coordinate (24);
			\draw [dashed] (4,0)--(4,20) (8,0)--(8,31) (12,0)--(12,28) (16,0)--(16,34) (20,0)--(20,27) (24,0)--(24,24); 
			\draw[smooth, thick, red]
			(25) .. controls +(-10:1) and +(-180:1) .. (20)
			(20) .. controls +(0:1) and +(-180:1) .. (31)
			(31) .. controls +(0:1) and +(160:1) .. (28)
			(28) .. controls +(0:1) and +(-180:2) .. (34)
			(34) .. controls +(0:1.5) and +(130:1.5) .. (27)
			(27) .. controls +(-60:1.5) and +(-180:2) .. (24);
			\foreach \x in {20,31,28,34,27,24}
			\fill (\x) +(90:3mm) node {$\x$};
		\end{tikzpicture}
	\end{center}
		\choiceTF
		{Nhiệt độ cao nhất trong ngày là $28^{\circ} \mathrm{C}$}
		{\True Nhiệt độ thấp nhất trong ngày là $20^{\circ} \mathrm{C}$}
		{\True Thời điểm có nhiệt độ cao nhất trong ngày là lúc $16$ giờ}
		{\True Thời điểm có nhiệt độ thấp nhất trong ngày là lúc $4$ giờ}
	\loigiai{}
\end{ex} \dongcham{8}

\begin{ex}
	Cho hàm số $y=f\left(x\right)$ có đạo hàm $y=f'\left(x\right)$ liên tục trên $\mathbb{R}$ và đồ thị của hàm số $f'\left(x\right)$ trên đoạn $\left[-2;6\right]$ như hình vẽ bên. 	Xét tính đúng, sai của các khẳng định sau:
	\begin{center}
		\begin{tikzpicture}[line join=round, line cap=round,>=stealth,scale=.7]
			\def\xmin{-3}\def\xmax{6.5}\def\ymin{-1}\def\ymax{3.5}
			\draw[->] (\xmin-0.2,0)--(\xmax+0.2,0) node[below] {\small $x$};
			\draw[->] (0,\ymin-0.2)--(0,\ymax+0.2) node[right] {\small $y$};
			\draw (0,0) node [below left] {\footnotesize $O$};
			\foreach \x in {-2,-1,2,6}\draw (\x,0.05)--(\x,-0.05) node [below] {\scriptsize $\x$};
			\foreach \y in {-1,1,2,3}\draw (0.05,\y)--(-0.05,\y) node [left] {\scriptsize $\y$};
			\clip (\xmin,\ymin) rectangle (\xmax,\ymax);
			\draw[thick,smooth,samples=200,domain=-2:6] plot (\x,{13/3360*(\x)^4-61/672*(\x)^3+173/336*(\x)^2-11/42*(\x)-61/70});
			\draw[dashed](-2,0)|-(0,2.5)(6,0)|-(0,1.5);
		\end{tikzpicture}
	\end{center}
		\choiceTF
		{$\max\limits_{\left[-2;6\right]}f\left(x\right)=f\left(-1\right)$}
		{$\max\limits_{\left[-2;6\right]}f\left(x\right)=f\left(6\right)$}
		{$\max\limits_{\left[-2;6\right]}f\left(x\right)=f\left(-2\right)$}
		{\True $\max\limits_{\left[-2;6\right]}f\left(x\right)=\max\left\{ f\left(-1\right),f\left(6\right)\right\}$}
	
	\loigiai{
		\begin{center}
			\begin{tikzpicture}
				\tkzTabInit[nocadre,,lgt=1.2,espcl=2.5,deltacl=0.6]
				{$x$/0.6,$y'$/0.6,$y$/2}
				{$-2$,$-1$,$2$,$6$}
				\tkzTabLine{,+,0,-,0,+,}
				\tkzTabVar{-/$f(-2)$,+/$f(-1)$,-/$f(2)$,+/$f(6)$}
			\end{tikzpicture}
		\end{center}
		Dựa vào bảng biến thiên, ta thấy
		\begin{itemize}
			\item Hàm số đồng biến trên $\left( { - 2; - 1} \right)$ và $\left( {2;6} \right)$ do $f'\left( x \right) > 0$, suy ra
			\begin{center}
				$f\left( { - 1} \right) > f\left( { - 2} \right)$ và $f\left( 6 \right) > f\left( 2 \right)$ (1).
			\end{center}
			\item Hàm số nghịch biến trên $\left( { - 1;2} \right)$ do $f'\left( x \right) < 0$, suy ra
			\begin{center}
				$f\left( { - 1} \right) > f\left( 2 \right)$  $ (2) $.
			\end{center}
		\end{itemize}
		Từ $ (1) $, $ (2) $ suy ra $\mathop {\max }\limits_{\left[ { - 2;6} \right]} f\left( x \right) = \max \left\{ {f\left( { - 2} \right),f\left( { - 1} \right),f\left( 2 \right),f\left( 6 \right)} \right\} = \max \left\{ {f\left( { - 1} \right),f\left( 6 \right)} \right\}$.
	}
\end{ex} \dongcham{13}

\begin{ex}
	Cho hàm số $f(x)$ có đạo hàm là $f'(x)$. Đồ thị $y=f'(x)$ được cho như hình vẽ. Biết rằng $f(0)+f(3)=f(2)+f(5)$. Xét tính đúng, sai của các khẳng định sau:
	\begin{center}
		\begin{tikzpicture}[scale=1, font=\footnotesize, line join=round, line cap=round, >=stealth]
			\draw[->](-1,0)--(5.5,0) node[right] {$x$};
			\draw[->](0,-1)--(0,2.5) node[right] {$y$};
			\node (0,0) [below left]{$O$};
			\foreach \x in {1,...,5}
			\draw[shift={(\x,0)},color=black] (0pt,2pt) -- (0pt,-2pt);
			\foreach \y in {1,...,2}
			\draw[shift={(0,\y)},color=black] (2pt,0pt) -- (-2pt,0pt);
			\draw (-0.3,1.2) .. controls (0.1,-1.8) and (1.5,-0.5) .. (2,0) .. controls (3,1.) and (4,1.2) .. (5,1.3) .. controls (5.1,1.3) and (5.3,1.3) .. (5.5,1.3);
			\clip (-1,-1) rectangle (5.5,2.5);
			\draw[dashed](5,0)--(5,1.3);
			\fill (0,0) circle(1pt) (2,0) circle(1pt) node[below right]{$2$} (5,0) circle(1pt) node[below]{$5$};
		\end{tikzpicture}
	\end{center}
	\choiceTF
	{Hàm số nghịch biến trên khoảng $(-\infty;0)$}
	{\True Hàm số nghịch biến trên khoảng $(0;2)$}
	{$\min\limits_{[0;5]}f(x)=f(0)$ và $\max\limits_{[0;5]}f(x)=f(5)$}
	{\True $\min\limits_{[0;5]}f(x)=f(2)$ và $\max\limits_{[0;5]}f(x)=f(5)$}
	
	\loigiai
	{
		Bảng biến thiên của hàm số trên đoạn $[0;5]$
		\begin{center}
			\begin{tikzpicture}
				\tkzTabInit[lgt=1.5,espcl=3,deltacl=0.6]
				{$x$/0.6, $f'(x)$/0.6, $f(x)$/2}
				{$0$, $2$, $3$, $5$}
				\tkzTabLine{,-,z,+, ,+,}
				\tkzTabVar{+/$f(0)$, -/$f(2)$, R, +/$f(5)$}
				\tkzTabVal[draw]{2}{4}{0.5}{}{$f(3)$}
			\end{tikzpicture}
		\end{center}
		Từ bảng biến thiên suy ra $\min\limits_{[0;5]}f(x)=f(2)$ và $\max\limits_{[0;5]}f(x)=\max\{f(0);f(5)\}$.\\
		Theo bảng biến thiên thì $f(3)>f(2)$ nên $f(3)-f(2)>0$.\\
		Theo giả thiết, ta có
		\[f(0)+f(3)=f(2)+f(5) \Leftrightarrow f(5)=f(0)+\left[f(3)-f(2)\right]>f(0).\]
		Suy ra $\max\limits_{[0;5]}f(x)=f(5)$.\\
		Vậy $\min\limits_{[0;5]}f(x)=f(2)$ và $\max\limits_{[0;5]}f(x)=f(5)$.
	}
\end{ex} \dongcham{13}

\Closesolutionfile{ans}
\begin{dang}{Bài toán max, min có chứa tham số $m$}
\end{dang}
\boxmini{BÀI TẬP TỰ LUẬN}
\begin{vd}
	Tìm tất cả giá trị của tham số $m$ để 
	\begin{tasks}
		\task giá trị lớn nhất của hàm số $f(x)= - x^3 -3x^2 +m$ trên $[-1;1]$ bằng $0$.
		\task giá trị nhỏ nhất của hàm số $ f(x)=\dfrac{x+5m}{x-3} $ trên $[1;2]$ bằng $4$.
	\end{tasks}
	\loigiai{
		\begin{enumerate}[a)]
			\item Hàm số liên tục và xác định trên đoạn $[-1;1]$.\\
			Ta có $f'(x)= -3x^2 -6x$.\\
			Cho $f'(x)=0 \Leftrightarrow \hoac{& x=0 \in [-1;1]\\& x= -2 \notin [-1;1].}$\\
			Xét $f(-1)= -2 + m $; $f(1)= -4 + m$.\\
			Suy ra $\displaystyle \max_{[-1;1]} f(x) = -2 + m$.\\
			Theo đề bài, $-2+ m=0 \Leftrightarrow m=2.$
			\item Ta có $ y'=\dfrac{-3-5m}{(x-3)^2} $.
			\begin{itemize}
				\item Trường hợp $ -3-5m>0\Leftrightarrow m<-\dfrac{3}{5}$\\
				$\Rightarrow y'>0 $ thì $ \displaystyle\min_{[1;2]}y=y(1)\Leftrightarrow -\dfrac{1}{2}(1+5m)=4\Leftrightarrow m=-\dfrac{9}{5}$ (nhận vì $ -\dfrac{9}{5}<-\dfrac{3}{5} $).
				\item Trường hợp $ -3-5m<0\Leftrightarrow m>-\dfrac{3}{5}$\\
				$ \Rightarrow y'<0 $ thì $ \displaystyle\min_{[1;4]}y=y(2)\Leftrightarrow -(2+5m)=4\Leftrightarrow m=-\dfrac{6}{5} $ (loại vì $ -\dfrac{6}{5}<-\dfrac{3}{5} $).
			\end{itemize}
			Vậy  $m=-\dfrac{9}{5}$.
		\end{enumerate}
		}
\end{vd}
\dongcham{20}
\boxmini{BÀI TẬP TRẮC NGHIỆM}

\setcounter{ex}{0}
\Opensolutionfile{ans}[ans/2D1-B2-d2-1]

\begin{ex}
	Cho hàm số $f(x) = 2x^3 -3x^2 + m$ thoả mãn $\displaystyle \min_{[0;5]} f(x) = 5$. Khi đó giá trị của $m$ bằng
	\choice
	{$10 $}
	{$ 5$}
	{\True $ 6$}
	{$ 7$}
	\loigiai{
		Ta có $f'(x)= 6x^2 -6x$.\\
		Cho $f'(x)=0 \Leftrightarrow \hoac{&x=0 \in [0;5] \\& x=1 \in [0;5].}$\\
		Xét $f(0)= m$; $f(1)= -1+ m$; $f(5)= 175 +m$.\\
		Suy ra $\displaystyle \min_{[0;5]} f(x)= -1+m$.\\
		Theo giả thiết $-1+ m= 5 \Leftrightarrow m=6$.}
\end{ex} \dongcham{10}

\begin{ex}
	Tìm $m$ để giá trị nhỏ nhất của hàm số $f(x) = 3x^3 - 4x^2 + 2(m - 10)$ trên đoạn $[1; 3]$ bằng $-5$.
	\choice
	{$m = \dfrac{15}{2}$}
	{$m = - 15$}
	{\True $m = 8$}
	{$m = -8$}
	\loigiai{
		$\bullet$ $f'(x) = 9x^2-8x$. Ta có $f'(x) = 0 \Leftrightarrow \hoac{&x = 0\\&x = \dfrac{8}{9}.}$\\
		$\bullet$ Ta có bảng biến thiên
		\begin{center}
			\begin{tikzpicture}
				\tkzTabInit[espcl=4,lgt=2,deltacl=1]{$x$/1,$f'(x)$/1,$f(x)$/2}
				{$1$,$3$}
				\tkzTabLine{,+,}
				\tkzTabVar{-/$2m-21$,+/$2m+25$}
			\end{tikzpicture}
		\end{center}
		$\bullet$ Giá trị nhỏ nhất của $f(x)$ trên đoạn $[1;3]$ bằng $-5 \Leftrightarrow 2m - 21 = -5 \Leftrightarrow m= 8$.
	}
\end{ex} \dongcham{12}

\begin{ex}
	Tìm $m$ để giá trị nhỏ nhất của hàm số $f(x)=\dfrac{x-m^2+m}{x+1}$ trên đoạn $[0;1]$ bằng $-2$.
	\choice
	{$\hoac{&m=1\\&m=-2}$}
	{$\hoac{&m=1\\&m=2}$}
	{$m=\dfrac{1\pm\sqrt{21}}{2}$}
	{\True $\hoac{&m=-1\\&m=2}$}
	\loigiai{
		$\mathscr{D}=\mathbb{R}\setminus\{-1\}$.\\
		Ta có $f'(x)=\dfrac{m^2-m+1}{(x+1)^2}>0$, $\forall x\in\mathscr{D}$.\\
		Khi đó $\min\limits_{x\in[0;1]}f(x)=f(0)\Leftrightarrow -2=-m^2+m\Leftrightarrow \hoac{&m=-1\\&m=2}$.
	}
\end{ex} \dongcham{12}

\begin{ex}
	Hàm số $y=\dfrac{x-m}{x+2}$ thỏa mãn $\min \limits_{x\in[0;3]}y+\max \limits_{x\in[0;3]}y=\dfrac{7}{6}$. Hỏi giá trị $m$ thuộc khoảng nào trong các khoảng dưới đây?
	\choice
	{$(2;+\infty)$}
	{$(0;2)$}
	{$(-\infty;-1)$}
	{\True $(-1;0)$}
	\loigiai{
		Do hàm số $y=\dfrac{x-m}{x+2}$ luôn đơn điệu trên đoạn $[0;3]$.\\
		Do đó $\min \limits_{x\in[0;3]}y+\max \limits_{x\in[0;3]}y=y(0)+y(3)=\dfrac{-m}{2}+\dfrac{3-m}{5}=\dfrac{7}{6}\Leftrightarrow\dfrac{-7m}{10}=\dfrac{17}{30}\Leftrightarrow m=\dfrac{-17}{21}$.}
\end{ex} \dongcham{11}

\begin{ex}
	Cho hàm số $y=\dfrac{x+m}{x+1}$ ($m$ là tham số thực) thỏa mãn $\min\limits_{[1;2]} y+\max\limits_{[1;2]} y=\dfrac{16}{3}$. Mệnh đề nào dưới đây đúng?
	\choice
	{\True $m>4$}
	{$m\le 0$}
	{$0<m\le 2$}
	{$2<m\le 4$}
	\loigiai{
		Tập xác định $\mathscr{D}=\mathbb{R}$.\\
		Ta có $y'=\dfrac{1-m}{(x+1)^2}$.
		\begin{itemize}
			\item Với $m=1$ thì $y=1$ nên $\min\limits_{[1;2]} y+\max\limits_{[1;2]} y=2$ (không thỏa mãn).
			\item Với $m\neq 1$ thì hàm số đơn điệu trên $[1;2]$ nên
			\begin{eqnarray*}
				&& \min\limits_{[1;2]} y+\max\limits_{[1;2]} y=\dfrac{16}{3}\\
				& \Leftrightarrow & y(1)+y(2)=\dfrac{16}{3}\\
				& \Leftrightarrow & \dfrac{m+1}{2}+\dfrac{m+2}{3}=\dfrac{16}{3}\\
				& \Leftrightarrow & m=5>4.
			\end{eqnarray*}
		\end{itemize}
	}
\end{ex} \dongcham{11}

\begin{ex}
	Cho hàm số $ f(x)=\dfrac{x+m}{x-1} $ ($ m $ là tham số thực) thỏa mãn $ \min\limits_{[2 ; 4]} f(x)=3 $. Mệnh đề nào dưới đây đúng ?
	\choice
	{$1\leq m<3$}
	{$m < -1$}
	{$3<m\leq 4$}
	{\True$m>4$}
	\loigiai{
		Tập xác định $ \mathscr{D} = \mathbb{R} \setminus\{1\}$.\\
		Ta có $ f'(x)=\dfrac{-1-m}{(x-1)^{2}} $.\\
		\underline{\textbf{TH1}}: $ -1-m<0 \Leftrightarrow m >-1 $.\\
		Ta có $ \min\limits_{[2 ; 4]}y=y(4)=\dfrac{4+m}{4-1}=3\Leftrightarrow m=5$ (thỏa mãn).\\
		\underline{\textbf{TH2}}: $  -1-m>0 \Leftrightarrow m <-1 $.\\
		Ta có $ \min\limits_{[2 ; 4]}y=y(2)=\dfrac{2+m}{2-1}=3\Leftrightarrow m=1$ (loại).\\
		Vậy $ m=5>4 $.
	}
\end{ex} \dongcham{11}

\begin{ex}
	Gọi $S$ là tổng giá trị của $m$ để hàm số $f(x) = \dfrac{x - m^2 - m}{x+1}$ có giá trị nhỏ nhất trên $[0;1]$ bằng $-2$. Mệnh đề nào sau đây đúng?
	\choice
	{\True $S=-1 $}
	{$S=1 $}
	{$S=-2 $}
	{$ S=-3$}
	\loigiai{
		Ta có $f'(x)= \dfrac{m^2 + m -1 }{(x+1)^2}$.
		\begin{itemize}
			\item Trường hợp $1$: $y'<0 \Leftrightarrow m^2 + m -1 <0$.\\
			Khi đó hàm số nghịch biến trên $[0;1]$.\\
			Suy ra $\displaystyle \min_{[0;1]} f(x) = f(1)= \dfrac{-m^2 -m +1}{2}$.\\
			Theo giả thiết $\dfrac{-m^2 -m +1}{2} = -2 \Leftrightarrow m^2 + m =5$ (không thoả điều kiện $m^2 +m <1$).
			\item Trường hợp $2$: $y'>0 \Leftrightarrow m^2 + m -1>0$.\\
			Khi đó $\displaystyle \min_{[0;1]} f(x) = f(0)=-m^2 -m$.\\
			Theo giả thiết $-m^2 -m =-2  \Leftrightarrow \hoac{&m= 1 \text{ (nhận) }\\& m=-2 \text{ (nhận).}}$
		\end{itemize}
		Vậy tổng các giá trị của $m$ là $-2+1 =-1.$
	}
\end{ex} \dongcham{11}

\begin{ex}
	Cho hàm số $f(x)=x^3+m x^2-m^2x+2$ với tham số $m>0$. Biết $\min\limits_{[-m ; m]}f(x)=\dfrac{14}{ 27}$. Mệnh đề nào dưới đây đúng
	\choice
	{$m\in (-\infty;-3)$}
	{$m\in (3;+\infty)$}
	{\True $m\in (1;3)$}
	{$m\in (-3;-1)$}
	\loigiai{
		Ta có $f'(x)=3x^2+2mx-m^2=(x+m)(3x-m)$.\\
		$f'(x)=0\Leftrightarrow \hoac{& x=-m \\ & x=\dfrac{m}{3}}$. Suy ra $\heva{& f(-m)=m^3 +2\\ & f(m)=m^3+2\\ &f\left(\dfrac{m}{3}\right)=-\dfrac{5m^3}{27}+2.}$\\
		Vì $m>0$ nên $f(m)=f(-m)>f\left(\dfrac{m}{3}\right)$, suy ra $\min\limits_{  [-m;m]} f(x)=f\left(\dfrac{m}{3}\right)=\dfrac{14}{27}$.\\
		Do đó $m=2$, vậy $m\in(1;3)$.
	}
\end{ex} \dongcham{11}

\begin{ex}%[2D1K3-1]
	Có tất cả bao nhiêu giá trị nguyên của tham số $m$ để giá trị nhỏ nhất của hàm số $y=x^3+\left(m^2-m+1\right)x+m^3-4m^2+m+2025$ trên đoạn $[0;2]$ bằng $2019$?
	\choice
	{$0$}
	{$1$}
	{$2$}
	{\True $3$}
	\loigiai{
		Ta có $y'=f'(x)=3x^2+\left(m^2-m+1\right)$ trên đoạn $[0;2]$.\\
		Ta có $y'=3x^2+\left(m-\dfrac{1}{2}\right)^2+\dfrac{3}{4}>0,\forall x\in\mathbb{R}$.\\
		Do đó hàm số đồng biến trên $\mathbb{R}\Rightarrow$ ta có $\min\limits_{[0;2]}y=f(0)=m^3-4m^2+m+2025$.\\
		Ta có $f(0)=2019\Leftrightarrow m^3-4m^2+m+2025=2019\Leftrightarrow m^3-4m^2+m+6=0\Leftrightarrow\hoac{&m=-1\\&m=2\\&m=3.}$\\
		Vậy tập các giá trị $m$ thỏa mãn là $\{-1;2;3\}$. Hay có tất cả $3$ giá trị $m$ thỏa mãn.}
\end{ex} \dongcham{11}

\begin{ex}
	Gọi $S$ là tập tất cả các giá trị của $m$ sao cho giá trị nhỏ nhất của hàm số $y=\left(x^3-3x+m \right)^2 $ trên
	đoạn $[-1;1]$ bằng $4$. Tính tổng các phần tử của $S$.
	\choice
	{\True  $ 0 $}
	{$ 6 $}
	{$ -5 $}
	{$ 3 $}
	\loigiai{
		\immini{Ta có  $\displaystyle\min\limits_{[-1;1]}\left(x^3-3x+m \right)^2=4  \Leftrightarrow \displaystyle\min\limits_{[-1;1]}\left|x^3-3x+m \right|=2$.\\Xét hàm số $y=f(x)=x^3-3x+m$ trên $[-1;1]$.\\
			Ta có $y'=3x^2-3=3(x^2-1)$, $y'=0\Leftrightarrow x=\pm1$.\\
			Bảng biến thiên hàm số như hình bên.
		}{\begin{tikzpicture}[scale=.8,line join=round, line cap=round,font=\footnotesize,>=stealth]
				\def\a{6}
				\def\b{3.7}
				\draw[shift={(-.5,.5)},blue!50!black]
				(0,0) rectangle +(\a,-\b)
				(0,-1)--+(0:\a)
				(0,-2)--+(0:\a)
				(1,0)--+(-90:\b)
				;
				\path
				(0,0) node{$x$}
				(0,-1) node{$y'$}
				(0,-2.3) node{$y$}
				(1,0) node{$-1$}
				(5,0) node{$1$}
				(1,-1) node{$0$}
				(3,-1) node{$-$}
				(5,-1) node{$0$}
				(1.2,-1.8) node (A) {$m+2$}
				(4.8,-3) node (C){$m-2$}
				;
				\draw[->] (A)--(C);
		\end{tikzpicture}}
		\noindent Từ bảng biến thiên của hàm số $y=f(x)$, ta có $\displaystyle\min\limits_{[-1;1]}\left|x^3-3x+m \right|=2$ khi và chỉ khi
		\begin{enumerate}[TH1.]
			\item $\heva{&m+2<0\\&m+2=-2}\Leftrightarrow m=-4$.
			\item $\heva{&m-2>0\\&m-2=2}\Leftrightarrow m=4$.
		\end{enumerate}
		Vậy $S=\{-4,4\}\Rightarrow $ Tổng các phần tử của $S$ bằng $0$.
	}
\end{ex} \dongcham{12}

\Closesolutionfile{ans}



% 
\begin{dang}{Bài toán vận dụng, thực tiễn có liên quan đến max min}
	\begin{enumerate}[\iconMT]
		\item \indam{Bài toán chuyển động:}
		\begin{itemize}
			\item [$\bullet$] Gọi $s(t)$ là hàm quãng đường; $v(t)$ là hàm vận tốc; $a(t)$ là hàm giá tốc;
			\item [$\bullet$] Khi đó $s'(t)=v(t)$; $v'(t)=a(t)$.
		\end{itemize}
		\item \indam{Bài toán thực tế -- tối ưu:}
		\begin{itemize}
			\item[$\bullet$] Biểu diễn dữ kiện cần đạt max -- min qua một hàm $f(t)$. 
			\item[$\bullet$] Khảo sát hàm $f(t)$ trên miền điều kiện của hàm và suy ra kết quả.
		\end{itemize}
	\end{enumerate}
\end{dang}
\boxmini{BÀI TẬP TỰ LUẬN}
\begin{vd}%[2D1B3-6]
	Một chất điểm chuyển động có vận tốc tức thời $v(t)$ phụ thuộc vào thời gian $t$ theo hàm số $v(t)=-t^4+24t^2+500$ (m/s). Trong khoảng thời gian từ $t=0$ (s) đến $t=5$ (s) chất điểm đạt vận tốc lớn nhất tại thời điểm nào?
	\loigiai{Ta có $v'(t)=-4t^3+48t=-4t(t^2-12)$\\
		$v'(t)=0\Leftrightarrow \hoac{&t=0\\&t=\pm 2\sqrt{3}}$.\\
		Bài toán trở thành tìm giá trị lớn nhất của hàm số $v(t)$ trên đoạn $[0;10]$, ta có:\\
		$v(0)=500$, $v(2\sqrt{3})=664$, $v(5)=475$.\\
		Vậy vận tốc lớn nhất khi $t=2\sqrt{3}\approx 4$ (s).
	}
\end{vd}
\dongcham{8}
\begin{vd}
	\immini{
		Sự phân huỷ của rác thải hữu cơ có trong nước sẽ làm tiêu hao oxygen hoà tan trong nước. Nồng độ oxygen (mg/l) trong một hồ nước sau $t$ giờ $(t \geq 0)$ khi một lượng rác thải hữu cơ bị xả vào hồ được xấp xỉ bởi hàm số (có đồ thị như đường màu đỏ ở hình bên)
		$$
		y(t)=5-\frac{15 t}{9 t^2+1}.
		$$
	}{
		\begin{tikzpicture}[>=stealth,x=1cm,y=0.3cm,scale=1.5,font=\footnotesize]
			\draw[->] (-0.5,0) -- (4,0) node[below] {$t$};
			\draw[->] (0,-1) -- (0,6) node[left] {$y$};
			\filldraw (0,0) circle (1pt)node[below left]{$O$};
			\draw[domain=0:4,samples=200,red] plot (\x,{5-(15*(\x))/(9*(\x)^2+1)});
			\draw[dashed] (0,5) node [left] {$5$}--(4,5);
			\foreach \x/\g in {1/-90,2/-90,3/-90}
			\draw[thin] (\x,2pt)--(\x,-2pt) + (\g:3mm) node {$\x$};
		\end{tikzpicture}
	}
	\noindent
	Vào các thời điểm nào nồng độ oxygen trong nước cao nhất và thấp nhất?\
	\loigiai{
		Xét hàm số $y(t)=5-\dfrac{15t}{9t^2+1}$ xác định và liên tục trên khoảng $[0;+\infty)$ .\\
		Ta có $y'(t)=\dfrac{135t^2-15}{(9t^2+1)^2}=0\Leftrightarrow t=\dfrac{1}{3}$ (giờ).\\
		Mặt khác $\lim\limits_{t\to+\infty}y(t)=\lim\limits_{t\to+\infty}\left[5-\dfrac{15t}{9t^2+1}\right]=5$ và $\lim\limits_{t\to 0}y(t)=\lim\limits_{t\to 0}\left[5-\dfrac{15t}{9t^2+1}\right]=5$.\\
		Bảng biến thiên
		\begin{center}
			\begin{tikzpicture}
				\tkzTabInit[espcl=3,lgt=1.5]
				{$t$/0.6,$y'(t)$/0.6,$y(t)$/1.5}
				{$0$,$\frac{1}{3}$,$+\infty$}
				\tkzTabLine{,-,0,+,}
				\tkzTabVar{+/$5$,-/$0$,+/$5$}
			\end{tikzpicture}
		\end{center}
		Từ bảng biến thiên, ta thấy $\min\limits_{[0;+\infty)}y(x)=0$ và $\mathop{\rm{max}}\limits_{[0;+\infty)}y(x)=5$.
	}
\end{vd}
\dongcham{10}
\begin{vd}%[2D1T3-6]
	\immini[0.02]{
		Tính diện tích lớn nhất $S_{\max}$ của một hình chữ nhật nội tiếp trong nửa đường tròn bán kính $R=6$ cm nếu một cạnh của hình chữ nhật nằm dọc theo đường kính của hình tròn mà hình chữ nhật đó nội tiếp.
	}{
		\begin{tikzpicture}[line join = round, line cap = round,>=stealth,font=\footnotesize,scale=1] 
			\def\R{2}
			\coordinate[label = below:$O$] (O) at (0,0); 
			\coordinate (A) at (-\R,0); 
			\coordinate (B) at ($(A)!2!(O)$);
			\coordinate[label = above right:$C$] (C) at (50:\R); 
			\coordinate[label = above left:$D$] (D) at (130:\R);
			\coordinate[label = below:$A$] (AA) at ($(A)!(D)!(B)$); 
			\coordinate[label = below:$B$] (BB) at ($(A)!(C)!(B)$); 
			\draw (A) arc(180:0:\R)--cycle;
			\draw[fill=cyan!20] (BB)--(C)--(D)--(AA)--cycle;
			\foreach \x in {AA,O,BB} \fill[black] (\x) circle (1.5pt); 
		\end{tikzpicture}
	}
	\loigiai{
		\immini{
			{\bf Cách 1.}\\
			Gọi chiều dài $AD=2x$ ($0<x<6$)\\
			$\Rightarrow AB=\sqrt{36-x^{2}}$.\\
			Diện tích hình chữ nhật là $S=2x\sqrt{36-x^{2}}$.\\
			Xét $f(x)=x\sqrt{36-x^{2}}$ trên $(0;6)$, ta có $$f'(x)=\sqrt{36-x^{2}}-\dfrac{x^{2}}{\sqrt{36-x^{2}}}=0\Leftrightarrow x=\pm 3\sqrt{2}.$$
		}{
			\begin{tikzpicture}
				\tikzset{on double/.style = {fill = \tkzTabDefaultBackgroundColor}} 
				\tikzset{h style/.style = {pattern=north west lines}} 
				\tkzTabInit[lgt=1.2,espcl=2]
				{$x$ /.6,$f'(x)$ /.6, $f(x)$ /1.5}
				{$0$,$3\sqrt{2}$,$6$}
				\tkzTabLine{d,+,0,-,d}
				\tkzTabVar{-/$0$,+/$36$,-/$0$}
			\end{tikzpicture}
		}
		Bảng biến thiên hàm số $f(x)$ trên $(0,6)$ ở hình bên\\
		Vậy giá trị lớn nhất của diện tích hình chữ nhật $ABCD$ là $36$ cm$^2$.\\
		{\bf Cách 2.}\\
		Đặt $AB=CD=2x$ ($0<x<6$). Khi đó $AD=\sqrt{DO^2-AO^2}=\sqrt{36-x^2}$. Suy ra
		\begin{align*}
			S_{ABCD}=2x\sqrt{36-x^2}\le 2\cdot \dfrac{x^2+36-x^2}{2}=36.
		\end{align*}
		Dấu bằng xảy ra khi $x=\sqrt{36-x^2}$ hay $x=3\sqrt{2}$.\\
		Vậy giá trị lớn nhất của diện tích hình chữ nhật $ABCD$ là $36$ cm$^2$.
	}
\end{vd}
\dongcham{14}
\begin{vd}%[2D1K3-6]
	\immini{Một người muốn xây một cái bể chứa nước, dạng một khối hộp chữ nhật không nắp có thể tích
	bằng $288$ dm$^3$. Đáy bể là hình chữ nhật có chiều dài gấp đôi chiều rộng, giá thuê nhân công để xây bể là
	$500000$ đồng/ m$^2$. Nếu người đó biết xác định các kích thước của bể hợp lí thì chi phí thuê nhân công sẽ
	thấp nhất. Hỏi người đó trả chi phí thấp nhất để thuê nhân công xây dựng bể đó là bao nhiêu?}{\hspace{1cm}
	\begin{tikzpicture}[scale=0.8, line join=round, line cap=round]
		\tkzDefPoints{0/0/A,-1.3/-1.1/B,2/-1.1/C}
		\coordinate (D) at ($(A)+(C)-(B)$);
		\coordinate (A') at ($(A)+(0,2.5)$);
		\tkzDefPointsBy[translation=from A to A'](B,C,D){B'}{C'}{D'}
		\tkzDrawPolygon(A',B',B,C,D,D')
		\tkzDrawSegments(B',C' C',D' C,C')
		\tkzDrawSegments[dashed](A,B A,D A,A')
\end{tikzpicture}}
	\loigiai{
		Gọi $x(x>0)$ là chiều rộng của đáy bể. Khi đó, chiều dài của bể là $2x$ và chiều cao của bể là $\dfrac{0,144}{x^2}$.\\
		Diện tích cần xây $2x^2+\dfrac{0,864}{x}$\\
		Xét $f(x) = 2x^2 + \dfrac{0,864}{x}$, có
		$f'(x) = 4x - \dfrac{0,864}{x^2}$\\
		$f'(x) = 0 \Leftrightarrow 4x - \dfrac{0,864}{x^2} \Leftrightarrow x=0,6.$\\
		Bảng biến thiên
		\begin{center}
			\begin{tikzpicture}
				\tkzTabInit[nocadre=false, lgt=1.2, espcl=3]
				{$x$ /0.6,$f'(x)$ /0.6,$f(x)$ /1.5} 	
				{$0$, $0{,}6$, $+\infty$}
				\tkzTabLine{,-,$0$,+}
				\tkzTabVar{+/ $+\infty$ ,-/$2{,}16$,+/$+\infty$}
			\end{tikzpicture}
		\end{center}
		Từ bảng biến thiên ta có $\min f(x)= 2,16.$\\
		Vậy chi phí thấp nhất để thuê nhân công xây bể là $2,16 \times 500000 = 1080000$ đồng.
	}
\end{vd}
\dongcham{18}
\begin{vd}%[2D1T3-2]
	\immini{Một nhà sản xuất cần làm ra những chiếc bình có dạng hình trụ với dung tích $1000\mathrm{~cm}^3$. Mặt trên và mặt dưới của bình được làm bằng vật liệu có giá 1,2 nghìn đồng$/\mathrm{cm}^2$, trong khi mặt bên của bình được làm bằng vật liệu có giá $0{,}75$ nghìn đồng$/\mathrm{cm}^2$. Tìm các kích thước của bình để chi phí vật liệu sản xuất mỗi chiếc bình là nhỏ nhất.}{\hspace{1cm}
	\begin{tikzpicture}[line join=round,line cap=round,line width=.6pt,font=\footnotesize,scale=0.46,>=stealth]
		\coordinate[label=right:$A$] (A) at (3,0);
		\coordinate[label=left:$O$] (O) at (0,0);
		\coordinate[label=right:$A'$] (A1) at ($(A)+(90:6)$);
		\coordinate[label=left:$O'$] (O1) at ($(O)+(90:6)$);
		\draw (A) arc (0:-180:3 and 3/4)--($(A1)!2!(O1)$) arc (180:0:3 and 3/4) arc (0:-180:3 and 3/4) (A)--(A1)--(O1);
		\draw[dashed] (O1)--(O)--(A) arc (0:180:3 and 3/4);
		\fill (O)circle(1.5pt) (O1)circle(1.5pt) (A)circle(1.5pt) (A1)circle(1.5pt);
\end{tikzpicture}}
	\loigiai{
			Gọi bán kính đáy của bình là $x$ (cm), ($x > 0$).\\
			Chiều cao của bình là $\dfrac{1000}{\pi \cdot x^2}$ (cm).\\
			Chi phí để sản xuất một chiếc bình là 
			\[
			T(x)=2\cdot1{,}2\cdot\pi \cdot x^2+0{,}75\cdot \dfrac{2000}{x}=2{,}4\pi \cdot x^2+\dfrac{1500}{x}~\text{(nghìn đồng)}.
			\]
			Để chi phí sản xuất mỗi chiếc bình là thấp nhất thì $T(x)$ là nhỏ nhất.\\
			$T^{\prime}(x)=4,8\pi x-\dfrac{1500}{x^2}, T^{\prime}(x)=0\Leftrightarrow x=\sqrt[3]{\dfrac{625}{2\pi}}$ (thỏa mãn).\\
			Bảng biến thiên:
			\begin{center}
				\begin{tikzpicture}[scale=1, font=\footnotesize]
					\tkzTabInit[nocadre=false, lgt=1.2, espcl=2, deltacl=0.6]
					{$x$/0.8,$T'(x)$/0.6,$T(x)$/2}
					{$0$,$\sqrt[3]{\frac{625}{2\pi}}$,$12$};
					\tkzTabLine{,-,$0$,+,};
					\tkzTabVar{+/$+\infty$,-/$T\left(\sqrt[3]{\frac{625}{2\pi}}\right)$,+/$T(12)$};
				\end{tikzpicture}
			\end{center}
			Để chi phí sản xuất mỗi chiếc bình là nhỏ nhất thì bán kính đáy của bình là $\sqrt[3]{\dfrac{625}{2\pi}}$ cm và chiều cao của bình là $\dfrac{1000}{\pi \cdot\left(\sqrt[3]{\dfrac{625}{2\pi}}\right)^2}$ cm.
	}
\end{vd}
\dongcham{20}
\boxmini{BÀI TẬP TRẮC NGHIỆM}
\ind{PHẦN I.} \inden{Câu trắc nghiệm nhiều phương án lựa chọn. Mỗi câu hỏi học sinh chỉ chọn một phương án.}\\
\setcounter{ex}{0}
\Opensolutionfile{ans}[ans/2D1-B2-d3-1]
\begin{ex}%[2D1K3]
	Một chất điểm chuyển động với quãng đường $s(t)$ cho bởi công thức $s(t)=6t^2-t^3$, $t$ (giây) là thời gian. Hỏi trong khoảng thời gian từ $0$ đến $4$ giây, vận tốc tức thời của chất điểm đạt giá trị lớn nhất tại thời điểm  $t$ (giây) bằng bao nhiêu?
	\choice
	{$t=3$ s}
	{$t=4$ s}
	{\True $t=2$ s}
	{$t=6$ s}
	\loigiai{Ta có $v(t)=s'(t)=12t-3t^2$.\\
		$v'(t)=12-6t$, $v'(t)=0\Leftrightarrow t=2$. \\
		Lập bảng biến thiên ta thấy $v(t)$ đạt giá trị lớn nhất tại $t=2$.
	}
\end{ex} \dongcham{7}

\begin{ex}
	Trong $3$ giây đầu tiên, một chất điểm chuyển động theo phương trình $s(t)=-t^3+6t^2+t+5,$ trong đó $t$ tính bằng giây và $s$ tính bằng mét. Chất điểm có vận tốc tức thời lớn nhất bằng bao nhiêu trong $3$ giây đầu tiên đó?
	\choice
	{\True 13 m/s}
	{10 m/s}
	{9 m/s}
	{12 m/s}
	\loigiai{
		Ta có $v(t)=s'(t)=-3t^2+12t+1.$ Xét hàm số $v(t)=-3t^2+12t+1$ trên đoạn $[0;5]$.\\
		$v'(t)=-6t+12$; $v'(t)=0 \Leftrightarrow t=2$.\\
		Tính các giá trị $v(0)=1$, $v\left(2\right)=13$, $v(3)=10$.\\
		So sánh các giá trị, ta có $\max\limits_{[0;3]}v(t)=13$.
	}
\end{ex}
\dongcham{7}
\begin{ex}
	Độ giảm huyết áp của một bệnh nhân được cho bởi công thức $G(x)=0{,}025x^2(30-x)$, trong đó $x$ là liều lượng thuốc được tiêm cho bệnh nhân ($x$ được tính bằng miligam). Liều lượng thuốc cần tiêm cho bệnh nhân là bao nhiêu để huyết áp được giảm nhanh nhất?
	\choice
	{$24$ mg}
	{\True $20$ mg}
	{$15$ mg}
	{$10$ mg}
	\loigiai
	{ 
		Bài toán trở thành: Tìm $x\in[0;30]$ để hàm số $G(x)=0{,}025x^2(30-x)$ đạt giá trị lớn nhất. \\
		Ta có $G(x)=0{,}025\left(30x^2-x^3\right) \Rightarrow G'(x)=0{,}025\left(60x-3x^2\right)$. \\
		Xét $G'(x)=0 \Leftrightarrow \hoac{ & x=0 \\ & x=20.}$ \\
		Bảng biến thiên hàm số $G(x)$
		\begin{center}
			\begin{tikzpicture}[scale=1]
				\tkzTabInit[nocadre=false, lgt=1.2, espcl=3.5, deltacl=0.6]{$x$/0.6, $G'(x)$/0.6, $G(x)$/2}{$0$, $20$, $30$}
				\tkzTabLine{0,+,0,-,}
				\tkzTabVar{-/$0$, +/$100$, -/$0$}
			\end{tikzpicture}
		\end{center}
		Từ bảng biến thiên ta có $\max\limits_{[0;30]} G(x)=G(20)=100$. \\
		Vậy liều lượng thuốc cần tiêm cho bệnh nhân để huyết áp giảm nhanh nhất là $20$ mg.
	}
\end{ex}
\dongcham{7}
\begin{ex}
	Trong thí nghiệm y học, người ta cấy $1\,000$ vi khuẩn vào môi trường dinh dưỡng. Bằng thực nghiệm, người ta xác định số lượng vi khuẩn thay đổi theo thời gian bởi công thức \[N(t)=1\,000+\dfrac{100t}{100+t^2}\,\text(con).\]
	trong đó $t$ là thời gian tính bằng giây. Tính số lượng vi khuẩn lớn nhất kể từ khi thực hiện cấy vi khuẩn vào môi trường dinh dưỡng.
	\choice
	{$1\,008$ con}
	{$1\,012$ con}
	{\True $1\,005$ con}
	{$1\,020$ con}
	\loigiai{
		Xét hàm số $N(t)=1\,000+\dfrac{100t}{100+t^2}$ ($t>0$).\\
		Ta có $N'(t)=\dfrac{100\cdot (100+t^2)-100t\cdot 2t}{\left(100+t^2\right)^2}=\dfrac{100\cdot (100-t^2)}{\left(100+t^2\right)^2}$.\\
		Khi đó, với $t>0$, $N'(t)=0\Leftrightarrow 100-t^2=0\Leftrightarrow t^2=100\Leftrightarrow t=10$.\\
		Bảng biến thiên của hàm số $N(t)$ như sau
		\begin{center}
			\begin{tikzpicture}[>=stealth]
				\tkzTabInit[nocadre=false,lgt=1.5,espcl=3,deltacl=0.6]{$t$/.6 ,$N'(t)$/.6,$N(t)$/1.5}
				{$0$ , $10$ , $+\infty$}
				\tkzTabLine{ ,+ , $0$ , - , }
				\tkzTabVar{-/$1\,000$ , +/$1\,005$ , -/$1\,000$}
			\end{tikzpicture}
		\end{center}
		Căn cứ vào bảng biến thiên, ta thấy trên khoảng $(0;+\infty)$, hàm số $N(t)$ đạt giá trị lớn nhất bằng $1\,005$ tại $t=10$.\\
		Vậy số lượng vi khuẩn lớn nhất kể từ khi thực hiện nuôi cấy vi khuẩn vào môi trường dinh dưỡng là $1\,005$ con.
	}
\end{ex}
\dongcham{14}
\begin{ex}
	Tam giác vuông có cạnh huyền bằng $5 \mathrm{~cm}$ có thể có diện tích lớn nhất bằng bao nhiêu?
	\choice
	{25 $\text{cm}^2$}
	{$\dfrac{125}{4}\,\text{cm}^2$}
	{$\dfrac{625}{4}\,\text{cm}^2$}
	{$125 \text{cm}^2$}
	\loigiai{Gọi một cạnh góc vuông là $x$ ($0<x<5$) thì cạnh góc vuông còn lại là $\sqrt{25-x^2}$.\\ Như vậy, diện tích tam giác là $S=\dfrac{x\cdot\sqrt{25-x^2}}{2}$.
		Đặt $f(x)=25x^2-x^4$. 
		\\Ta có $f'(x)=50x-4x^3$. Khi đó
		$f'(x)=0 \Leftrightarrow x=\dfrac{5\sqrt{2}}{2}$.\\
		Vì vậy $\displaystyle\max _{(0;5)} f(x)=f\left( \dfrac{5\sqrt{2}}{2}\right) =\dfrac{625}{4}$.\\
		Vậy tam giác vuông có cạnh huyền bằng $5 \mathrm{~cm}$ có thể có diện tích lớn nhất bằng $\dfrac{625}{4}$.}
\end{ex}
\dongcham{18}
\begin{ex}
	\immini{
		Từ một tấm tôn có hình dạng là nửa hình tròn bán kính $R=3$, người ta muốn cắt ra một hình chữ nhật (hình vẽ bên). Diện tích lớn nhất có thể của tấm tôn hình chữ nhật là
		\choice
		{$\dfrac{9}{2}$}
		{$6\sqrt2$}
		{\True $9$}
		{$9\sqrt2$}
	}
	{
		\begin{tikzpicture}[thick,scale=0.57]
			\draw [-] (-4,0)--(4,0);
			\draw [-] (-3,0)--(-2.99,2.65)--(2.99,2.65)--(3,0);
			\draw[smooth,samples=200,variable=\t,domain=0:180] plot({(4)*cos (\t)},{(4)*sin(\t)});
			\draw (0,0) [fill=black] circle (1pt) node[below]{$O$};
			\draw (-3,0) [fill=black] circle (1pt) node[below]{$Q$};
			\draw (3,0) [fill=black] circle (1pt) node[below]{$P$};
			\draw (-2.99,2.65) [fill=black] circle (1pt) node[left]{$M$};
			\draw (2.99,2.65) [fill=black] circle (1pt) node[right]{$N$};
			\draw[pattern=north east lines,pattern color=black!50!] (-3,0)--(-2.99,2.65)--(2.99,2.65)--(3,0);
		\end{tikzpicture}
	}
	\loigiai{
		Đặt $OQ=x,\ (0<x<3) \Rightarrow MQ=\sqrt{MO^2-OQ^2}=\sqrt{9-x^2}$.\\
		Ta có  $S_{MNPQ}=PQ\cdot MQ=2x\cdot\sqrt{9-x^2}\le 2\cdot\dfrac{x^2+9-x^2}{2}=9.$\\
		Dấu $=$ xảy ra khi $x=\dfrac{3\sqrt2}{2}.$
	}
\end{ex} \dongcham{18}

\begin{ex}%[2D1K3-6]
	Cho một tấm tôn hình chữ nhật có kích thước $10$ cm $\times$ $16$ cm. Người ta cắt bỏ $4$ góc của tấm tôn $4$ miếng hình vuông bằng nhau rồi gò lại thành một hình hộp chữ nhật không có nắp. Để thể tích của hình hộp đó lớn nhất thì độ dài cạnh hình vuông của các miếng tôn bị cắt bỏ bằng
	\choice
	{$3$ m}
	{$4$ m}
	{$5$ m}
	{\True $2$ m}
	\loigiai{
		\immini
		{Giả sử độ dài cạnh hình vuông của các miếng tôn bị cắt bỏ bằng $x$ $(0<2x<10\Leftrightarrow 0<x<5)$. Khi đó hình hộp chữ nhật có chiều cao bằng $x$, chiều rộng bằng $10-2x$ và chiều dài bằng $16-2x$. Suy ra hình hộp chữ nhật có thể tích $V=x(10-2x)(16-2x)=4x^3-52x^2+160x$.}
		{
			\begin{tikzpicture}[scale=0.7]
				\tkzInit[xmin=-5,xmax=6,ymin=-3,ymax=6]
				\tkzDefPoints{0/0/A, 8/0/D, 8/6/C, 0/6/B, 0/1/E, 0/5/F, 1/6/G, 7/6/H, 8/5/I, 8/1/J, 1/0/M, 7/0/N}
				\tkzDrawPoints(A,B,C,D,M,N,E,F,G,H,I,J)
				\tkzLabelSegments[above](B,G H,C){$x$}
				\tkzLabelSegments[right](C,I J,D){$x$}
				\tkzLabelSegment[left](A,B){$10$}
				\tkzLabelSegment[below](A,D){$16$}
				\tkzDrawSegments[thin](A,B A,D B,C C,D F,I E,J G,M H,N)
			\end{tikzpicture}
		}
		Xét hàm $f(x)=4x^3-52x^2+160x$ trên $(0; 5)$. Tập xác định $\mathscr{D}=\mathbb{R}$,\\ $f'(x)=12x^2-104x+160=0\Leftrightarrow\hoac{&x=2\\&x=\dfrac{20}{3}.}$
		Bảng biến thiên hàm số $f(x)$ trên $(0; 5)$:
		\begin{center}
			\begin{tikzpicture}
				\tkzTabInit%
				{$x$/1,%
					$f’(x)$ /1,%
					$f(x)$ /2}%
				{$0$ ,$2$ , $5$}%
				\tkzTabLine{ ,+, 0 ,-,}
				\tkzTabVar %
				{
					-/,+/ ,-/
				}
			\end{tikzpicture}
		\end{center}
		Dựa vào bảng biến thiên ta có hàm số đạt giá trị lớn nhất trên $(0; 5)$ tại $x=2$ hay hình hộp chữ nhật có thể tích lớn nhất khi độ dài cạnh hình vuông của miếng tôn bị cắt bỏ bằng $2$ m.
	}
\end{ex}
\dongcham{18}

\begin{ex}%[2H1K3-6]
	Ông Bình dự định sử dụng hết $5,5\,\mathrm{m^2}$ kính để làm một bể cá bằng kính có dạng hình hộp chữ nhật không nắp, chiều dài gấp đôi chiều rộng (các mối ghép có kích thước không đáng kể). Bể cá có dung tích lớn nhất bằng bao nhiêu (làm tròn đến hàng phần trăm)?
	\choice
	{ $1{,}01\,\mathrm{m^3}$}
	{\True $1{,}17\,\mathrm{m^3}$}
	{ $1{,}51\,\mathrm{m^3}$}
	{ $1{,}40\,\mathrm{m^3}$}
	\loigiai{
		\immini{
			Gọi $x,2x,y$ với $x,y>0$  lần lượt là chiều rộng, chiều dài, chiều cao của bể cá.
			Theo giả thiết ta có: $$2\cdot 2xy+2\cdot xy+2x^2=5{,}5\Leftrightarrow 6xy+2x^2=5{,}5\Rightarrow y=\dfrac{5{,}5-2x^2}{6x}.$$
			Do $y>0$ nên $5,5 - 2x^2 >0 \Rightarrow 0<x<\dfrac{\sqrt{11}}{2}$.\\
			Thể tích bể cá là $$V(x)=2x^2y=2x^2\cdot \dfrac{5{,}5-2x^2}{6x}=-\dfrac{2}{3}{x^3}+\dfrac{11}{6}x.$$
			Khảo sát hàm số $V(x)=-\dfrac{2}{3}{x^3}+\dfrac{11}{6}x$ trên khoảng $\left( 0;\dfrac{\sqrt{11}}{2} \right) $
			\begin{itemize}
				\item [$\bullet$] $V'(x)=-2x^2+\dfrac{11}{6}$; $V'(x)=0\Leftrightarrow x=\sqrt{\dfrac{11}{12}}$.
				\item [$\bullet$] Bảng biến thiên:
				\begin{center}
					\begin{tikzpicture}
						\tkzTabInit[nocadre=True,lgt=1,espcl=3]
						{$x$ /1,$V'$ /0.6,$V$ /2}
						{$0$,$\sqrt{\frac{11}{12}}$,$\frac{\sqrt{11}}{2}$}
						\tkzTabLine{,+,$0$,-,}
						\tkzTabVar{-/, +/$y_0$,-/}
					\end{tikzpicture}
				\end{center}
			\end{itemize}
			Thể tích lớn nhất của bể cá là $V\left( \sqrt{\dfrac{11}{12}} \right)=1{,}17\,\mathrm{m^3}$.}{
			\begin{tikzpicture}
				\def\tls{.4}
				\path
				(0,0) coordinate (A)
				++ (0:4)coordinate (B)
				++ (30:2.3)coordinate (C)
				($(A)+(C)-(B)$)coordinate (D)
				\foreach \x in {A,B,C,D}{(\x)++(90:2.5) coordinate (\x_1)}
				;
				\draw[dashed]
				(A)--(D) node[pos=.5,sloped,above]{$x$}
				(D)--(C) node[pos=.4,sloped,above]{$2x$}
				(D)--(D_1) node[pos=.4, right]{$y$}
				;
				\draw
				(A)--(B)--(C)
				(A_1)--(B_1)--(C_1)--(D_1)--cycle
				(A)--(A_1) (B)--(B_1) (C)--(C_1)
				;
		\end{tikzpicture}}
	}
\end{ex} \dongcham{20}

\begin{ex}%[2D1T3-6]
	Người ta muốn xây một chiếc bể nước có hình dạng là	một khối hộp chữ nhật không nắp có thể tích bằng $\dfrac {500}{3}$ m$^3$. Biết đáy bể là một hình chữ nhật có chiều dài gấp đôi chiều rộng và giá thuê thợ xây là $700.000$ đồng/m$^2$. Để chi phí thuê nhân công ít nhất thì chi phí thuê nhân công là
	\choice
	{$120$ triệu đồng}	
	{\True $105$ triệu đồng}
	{$115$ triệu đồng}	
	{$110$ triệu đồng}
	\loigiai{
		Gọi $x,y$ lần lượt là chiều rộng và chiều cao của bể cá (điều kiện $x,y>0$ ).
		\immini{Với giả thiết của bài toán, thể tích bể cá là $$V=2x^2y=\dfrac {500}{3}\Rightarrow y=\dfrac {250}{3x^2}.$$
			Để chi phí thuê nhân công ít nhất thì tổng diện tích các mặt của bể cá phải nhỏ nhất. Tổng diện tích các mặt của bể cá} 
		{\begin{tikzpicture}[scale=0.8, font=\footnotesize, line join=round, line cap=round, >=stealth]
				\tkzDefPoints{0/0/A,-1.3/-1.1/B,2/-1.1/C}
				\coordinate (D) at ($(A)+(C)-(B)$);
				\coordinate (A') at ($(A)+(0,2.5)$);
				\tkzDefPointsBy[translation=from A to A'](B,C,D){B'}{C'}{D'}
				\tkzDrawPolygon(A',B',B,C,D,D')
				\tkzDrawSegments(B',C' C',D' C,C')
				\tkzDrawSegments[dashed](A,B A,D A,A')
				\tkzDrawPoints[fill=black](A,B,D,C,A',B',C',D')
				\tkzLabelSegment[left](B',B){$y$}
				\tkzLabelSegment[below](B,C){$2x$}
				\tkzLabelSegment[right](A,B){$x$}
		\end{tikzpicture}}
		$S=2xy+2\cdot 2xy+2x^2=6xy+2x^2=\dfrac {500}{x}+2x^2$.\\
		Xét hàm số $S(x)=\dfrac {500}{x}+2x^2$ trên khoảng $(0;+\infty)$.\\
		$\Rightarrow S'(x)=-\dfrac {500}{x^2}+4x$.\\
		$S'(x)=0\Leftrightarrow -500+4x^3=0\Leftrightarrow x=5$.\\
		Bảng biến thiên
		\begin{center}
			\begin{tikzpicture}[scale=1, font=\footnotesize, line join=round, line cap=round, >=stealth]
				\tkzTabInit[nocadre=false,lgt=1.2,espcl=2, deltacl=0.5]
				{$x$/0.6,$S’(x)$/0.6,$S(x)$/1.5}
				{$0$,$5$,$+\infty$}
				\tkzTabLine{,-,z,+,}
				\tkzTabVar{+/$+\infty$,-/$150$,+/$+\infty$}
			\end{tikzpicture}
		\end{center}
		Do đó $\min S=150$ tại $x=5$. \\
		Khi đó, chi phí thuê nhân công là $150\cdot 700000=105$ triệu đồng.\\Vậy chi phí thuê nhân công ít nhất là $105$ triệu đồng.}
\end{ex}
\dongcham{13}
\begin{ex}%[2D1V3-6]
	Từ một tấm bìa hình chữ nhật có chiều rộng $30 \mathrm{~cm}$ và chiều dài $80 \mathrm{~cm}$ (Hình a), người ta cắt ở bốn góc bốn hình vuông có cạnh $x(\mathrm{~cm})$ với $5 \leq x \leq 10$ và gấp lại để tạo thành chiếc hộp có dạng hình hộp chữ nhật không nắp như Hình b. Tìm $x$ để thể tích chiếc hộp là lớn nhất (kết quả làm tròn đến hàng phần trăm).
	\begin{center}
		\begin{tikzpicture}[line join=round, line cap=round,scale=0.9]
			\coordinate (A) at (0,3);
			\coordinate (B) at (5,3);
			\coordinate (D) at (0,0);
			\coordinate (C) at ($(B)+(D)-(A)$);
			\draw(A)--(B)--(C)--(D)--cycle;
			\draw (0,0) rectangle (1,1) (A) rectangle (1,2) (B) rectangle (4,2) (4,1) rectangle (C);
			\draw[dashed] (1,1) rectangle (4,2);
			%	\foreach \i/\g in {A/90,B/90,C/-90,D/-90}{\draw[fill=black](\i) circle (1pt) ($(\i)+(\g:3mm)$) node[scale=1]{$\i$};}
			\draw (0,.5) node [left] {$x$};
			\draw (.5,0) node [below] {$x$};
			\draw (0,2.5) node [left] {$x$};
			\draw (0.5,3) node [above] {$x$};
			%%%%%%%%%
			\draw (4.5,0) node [below] {$x$};
			\draw (5,0.5) node [right] {$x$};
			\draw (5,2.5) node [right] {$x$};
			\draw (4.5,3) node [above] {$x$};
			%%%%%%%%
			\draw[<->] (-1,0)--(-1,3) node[above,midway,sloped] {$30$cm};
			\draw[<->] (0,-1)--(5,-1) node[above,midway] {$80$cm};
			\path (current bounding box.south) node[below, black]{a)}; %dưới
		\end{tikzpicture}
		\hspace*{1cm}
		\begin{tikzpicture}[scale=0.9, font=\footnotesize, line join=round, line cap=round, >=stealth]
			\def\bc{3} % cạnh BC
			\def\ba{1} % cạnh BA
			\def\h{1.5} % đường cao
			\def\gocnghieng{90} % góc nghiêng
			\def\gocB{35} % góc B của đáy
			\coordinate (B) at (0,0);
			\coordinate (A) at (\gocB:\ba);
			\coordinate (C) at (\bc,0);
			\coordinate (D) at ($(C)-(B)+(A)$);
			\coordinate (A') at ($(A)+(\gocnghieng:\h)$);
			\coordinate (B') at ($(B)-(A)+(A')$);
			\coordinate (C') at ($(C)-(A)+(A')$);
			\coordinate (D') at ($(D)-(A)+(A')$);
			\draw (B')--(B)--(C)--(D)--(D')--(A')--(B')--(C')--(D') (C)--(C');
			\draw[dashed] (A')--(A)--(D) (A)--(B);
			\path (current bounding box.south) node[below, black]{b)}; %dưới
		\end{tikzpicture}
	\end{center}
	\choice
	{\True $x=\dfrac{20}{3} \mathrm{~cm}$}
	{$x=\dfrac{20}{7} \mathrm{~cm}$}
	{$x=\dfrac{25}{3} \mathrm{~cm}$}
	{$x=\dfrac{25}{7} \mathrm{~cm}$}
	\loigiai{
		Thể tích chiếc hộp là $V(x)=x(30-2 x)(80-2 x)=2400 x-220 x^2+4 x^3$ với $5 \leq x \leq 10$.\\
		Ta có: $V'(x)=12 x^2-440 x+2400$;\\
		$V'(x)=0 \Leftrightarrow x=\dfrac{20}{3}$ hoặc $x=30$ (loại vì không thuộc $[5 ; 10]$);
		\begin{center}
			$V(5)=7000 ; V\left(\dfrac{20}{3}\right)=\dfrac{200000}{27} ; V(10)=6000$.
		\end{center}
		Do đó $\max \limits_{[5 ; 10]} V(x)=\dfrac{200000}{27}$ khi $x=\dfrac{20}{3}$.
		Vậy để thể tích chiếc hộp là lớn nhất thì $x=\dfrac{20}{3} \mathrm{~cm}$.}
\end{ex}
\dongcham{13}
\begin{ex}%[2D1K3]
	Một sợi dây có chiều dài là $6$ m, được chia thành $2$ phần. Phần thứ nhất được uốn thành hình tam giác đều, phần thứ hai uốn thành hình vuông. Hỏi độ dài của cạnh hình tam giác đều bằng bao nhiêu để tổng diện tích $2$ hình thu được là nhỏ nhất?
	\begin{center}
		\begin{tikzpicture}[scale=0.8,>=stealth]
			\draw(0,0)--(7,0);
			\draw (0,0)circle (1pt)(7,0) circle (1pt)(3,0) circle (1pt);
			\draw[->](1.5,-0.3)--(1.5,-0.7);
			\draw[->](5,-0.3)--(5,-0.7);
			\draw(4.5,-0.9)--(5.5,-0.9)--(5.5,-1.9)--(4.5,-1.9)--(4.5,-0.9);
			\draw(1.5,-0.9)--(2,-1.9)--(1,-1.9)--(1.5,-0.9);
		\end{tikzpicture}
	\end{center}
	\choice
	{$\dfrac{12}{4+\sqrt{3}}$ m}
	{$\dfrac{18\sqrt{3}}{4+\sqrt{3}}$ m}
	{$\dfrac{36\sqrt{3}}{4+\sqrt{3}}$ m}
	{\True $\dfrac{18}{9+4\sqrt{3}}$ m}
	\loigiai{
		Gọi độ dài cạnh hình tam giác đều là $x$ (m). Khi đó độ dài cạnh hình vuông là $\dfrac{6-3x}{4}$.\\
		Tổng diện tích khi đó là $S =\dfrac{\sqrt{3}}{4}x^2 + \left(\dfrac{{6 - 3x}}{4}\right)^2 =\dfrac{1}{16}\left[\left(9+4\sqrt{3}\right)x^2 - 36x + 36 \right)]$.\\
		Xét hàm số $f(x)=\left(9+4\sqrt{3}\right)x^2-36x+36, x\in(0;6)$.\\
		Ta có $f(x)$ là tam thức bậc $2$ có $-\dfrac{b}{2a}=\dfrac{18}{9+4\sqrt{3}} \in (0;6)$ và $a>0$.\\
		Suy ra $f(x)$ đạt giá trị nhỏ nhất tại
		$x=-\dfrac{b}{2a}\dfrac{18}{9+4\sqrt{3}}$.\\
		Vậy diện tích nhỏ nhất khi $x=\dfrac{18}{9+4\sqrt{3}}$ m.
	}
\end{ex}
\dongcham{14}
\begin{ex}
	Một doanh nghiệp tư nhân $A$ chuyên kinh doanh xe gắn máy các loại. Hiện nay doanh nghiệp đang tập trung vào chiến lược kinh doanh xe $X$ với chi phí mua vào một chiếc là 27 triệu đồng và bán ra với giá 31 triệu đồng. Với giá bán này, số lượng xe mà khách hàng đã mua trong một năm là 600 chiếc. Nhằm mục tiêu đẩy mạnh hơn nữa lượng tiêu thụ dòng xe đang bán chạy này, doanh nghiệp dự định giảm giá bán. Bộ phận nghiên cứu thị trường ước tính rằng nếu giảm 1 triệu đồng mỗi chiếc xe thì số lượng xe bán ra trong một năm sẽ tăng thêm 200 chiếc. Hỏi theo đó, giá bán mới là bao nhiêu thì lợi nhuận thu được cao nhất?
	\choice
	{$30$ triệu đồng}
	{\True $30,5$ triệu đồng}
	{$29,5$ triệu đồng}
	{$32$ triệu đồng}
	\loigiai{
		Gọi giá bán mới là $x$ (triệu đồng) với $x \in [27;31]$.\\
		Khi đó số xe bán ra là $600+(31-x) \cdot 200$.\\
		Lợi nhuận thu được là 
		\begin{eqnarray*}
			f(x) &=& [600+(31-x) \cdot 200](x-27)\\
			&=& (-200x+6800)(x-27)\\
			&=& -200x^2+12200x-183600\\
			&=& -200\left(x-\dfrac{61}{2}\right)^2+2450\\
			&\leq&2450.
		\end{eqnarray*}
		Vậy giá bán mới là $30,5$ triệu đồng thì lợi nhuận thu được là lớn nhất là $2\,450$ (triệu đông).
	}
\end{ex}
\dongcham{14}
\Closesolutionfile{ans}
\ind{PHẦN II.} \inden{Câu trắc nghiệm đúng sai. Trong mỗi ý a), b), c), d) ở mỗi câu, học sinh chọn đúng hoặc sai.}\\
\Opensolutionfile{ans}[ans/2D1-B2-d3-2]

\begin{ex}
	Người ta bơm xăng vào bình xăng của một xe ô tô. Biết rằng thể tích $V$ (lít) của lượng xăng trong bình xăng tính theo thời gian bơm xăng $t$ (phút) được cho bởi công thức $$V(t)=300(t^2-t^3)+4 \text{ với } 0\le t\le 0{,}5.$$
Gọi $V'(t)$ là tốc độ tăng thể tích tại thời điểm $t$ với $0\le t\le 0{,}5$.
\choiceTF
{Lượng xăng trong bình ban đầu là $1$ lít}
{\True Lượng xăng lớn nhất bơm vào bình xăng là $41{,}5$ lít}
{$V'(t)=300(2t-3t^2)+4$, với $0\le t\le 0{,}5$}
{\True Xăng chảy vào bình xăng vào thời điểm ở giây thứ $30$ có tốc độ tăng thể tích là lớn nhất}
	\loigiai{
		\begin{enumerate}[a)]
			\item Số xăng trong bình ban đầu là $V(0)=4$ lít.
			\item Lượng xăng lớn nhất bơm vào bình xăng là $V=V\left(\dfrac{1}{2}\right)=41{,}5$ lít.
			\item Xét hàm số $V(t)=300(t^2-t^3)+4 \text{ với } 0\le t\le 0{,}5.$\\
			Đạo hàm $V'(t)=300t(2-3t)$.\\
			\item Cho $V'(t)=0 \Leftrightarrow 300t(t-3t)=0 \Leftrightarrow \hoac{&t=0\in[0;0{,}5]\\&t=\dfrac{2}{3}\notin[0;0{,5}].}$\\
			Các giá trị $V(0)=4$, $V\left(\dfrac{1}{2}\right)=41{,}5$.\\
			Xăng chảy vào bình xăng vào thời điểm ở giây thứ $30$ có tốc độ tăng thể tích là lớn nhất.
		\end{enumerate}
	}
\end{ex}
\dongcham{20}
\begin{ex}
	Tại một xí nghiệp chuyên sản xuất vật liệu xây dựng, nếu trong một ngày xí nghiệp sản xuất $x$ (m$^3$) sản phẩm thì phải bỏ ra các khoản chi phí bao gồm: $4$ triệu đồng chi phí cố định; $0{,}2$ triệu đồng chi phí cho mỗi mét khối sản phẩm và $0{,}001 x^2$ triệu đồng chi phí bảo dưỡng máy móc. Biết rằng, mỗi ngày xí nghiệp sản xuất được tối đa $100$ m$^3$ sản phẩm. Goi $C(x)$ là tổng chi phí để xí nghiệp sản xuất $x$ (m$^3$) sản phẩm trong một ngày và $\overline{C}$ là chi phí trung bình  trên mỗi mét khối sản phẩm.
	\choiceTF
	{$C=0{,}2 x+0{,}001 x^2 \quad \text { với } 0 \leq x \leq 100$}
	{\True Tổng chi phí khi sản xuất 100 m$^3$ sản phẩm là 34 triệu đồng}
	{\True $\overline{C}=0{,}001 x+\dfrac{4}{x}+0{,}2 \quad\text { với } 0<x \leq 100$}
	{\True $\overline{C}$ có giá trị thấp nhất bằng 0,326 triệu đồng (\textit{kết quả làm tròn 3 chữ số thập phân})}
	\loigiai{
		\begin{enumerate}
			\item Tổng chi phí (triệu đồng) để xí nghiệp sản xuất $x$ (m$^3$) sản phẩm trong một ngày là
			$$
			C=C(x)=4+0{,}2 x+0{,}001 x^2 \text { với } 0 \leq x \leq 100.
			$$
			\item Thay $x=100$ vào hàm $C(x)$, ta được kết quả 34 (triệu đồng).
			\item Chi phí trung bình (triệu đồng) trên mỗi mét khối sản phẩm là
			$$
			\overline{C}=\overline{C}(x)=\dfrac{C(x)}{x}=\dfrac{4+0{,}2 x+0{,}001 x^2}{x}=0{,}001 x+\dfrac{4}{x}+0{,}2 \text { với } 0<x \leq 100.
			$$
			\item Ta có $\bar{C}'(x)=0{,}001-\dfrac{4}{x^2}$;
			$$
			\overline{C}'(x)=0 \Leftrightarrow 0{,}001-\dfrac{4}{x^2}=0 \Leftrightarrow x^2=4\,000 \Leftrightarrow x=20 \sqrt{10} \in(0 ; 100].
			$$
			
			Ta có $\overline{C}(20 \sqrt{10})=\dfrac{\sqrt{10}}{25}+\dfrac{1}{5} \approx 0,326$.\\
			Bảng biến thiên
			\begin{center}
				\begin{tikzpicture}
					\tikzset{double style/.append style={double distance=2pt}}
					\tkzTabInit[lgt=1.2, espcl=2]
					{$x$/0.6,$\overline{C'}(x)$/0.6,$\overline{C}(x)$/2.5}{$0$,$20\sqrt{10}$,$100$}
					\tkzTabLine{,-,0,+,}
					\tkzTabVar{+/$+\infty$,-/$\dfrac{\sqrt{10}}{25}+\dfrac{1}{5}$,+/$0{,}34$}
				\end{tikzpicture}
			\end{center}
			Từ bảng biến thiên, ta thấy chi phí trung bình thấp nhất là $\bar{C}(20 \sqrt{10}) \approx 0{,}326$ (triệu đồng/m$^3$ sản phẩm), đạt được khi $x=20 \sqrt{10} \approx 63$ (m$^3$).
		\end{enumerate}
	}
\end{ex}
\dongcham{20}
\begin{ex}
	Nhà máy $A$ chuyên sản xuất một loại sản phẩm cung cấp cho nhà máy $B$. Hai nhà máy thoả thuận rằng, hằng tháng $A$ cung cấp cho $B$ số lượng sản phẩm theo đơn đặt hàng của $B$ (tối đa $100$ tấn sản phẩm). Nếu số lượng đặt hàng là $x$ tấn sản phẩm thì giá bán cho mỗi tấn sản phẩm là $P(x)=45-0{,}001 x^2$ (triệu đồng). Chi phí để $A$ sản xuất $x$ tấn sản phẩm trong một tháng là $C(x)=100+30 x$ (triệu đồng) (gồm $100$ triệu đồng chi phí cố định và $30$ triệu đồng cho mỗi tấn sản phẩm).
	\choiceTF
	{\True Chi phí để  A sản xuất 10 tấn sảm phẩm trong một tháng là 400 triệu đồng}
	{Số tiền  A thu được khi bán 10 tấn sản phẩm cho B là 600 triệu đồng}
	{\True Lợi nhuận mà A thu được khi bán $x$ tấn sản phẩm ($0\le x \le 100)$ cho  B là $-0{,}001 x^3+15 x-100$}
	{\True A bán cho $B$ khoảng 70,7 tấn sản phẩm mỗi tháng thì thu được lợi nhuận lớn nhất}
	\loigiai{
		\begin{enumerate}[a)]
			\item Chi phí để  A sản xuất 10 tấn sảm phẩm trong một tháng là $C(10)=100+30\cdot 10=400$ (triệu)
			\item Số tiền mà $A$ thu được (gọi là doanh thu) từ việc bán $x$ tấn sản phẩm $(0 \leq x \leq 100)$ cho $B$ là
			$$
			R(x)=x \cdot P(x)=x\left(45-0{,}001 x^2\right)=45 x-0{,}001 x^3 \text { (triệu đồng). }
			$$
			Thay $x=10$, ta được $R(10)=449$ (triệu đồng).
			\item Lợi nhuận (triệu đồng) mà $A$ thu được là
			$$
			P(x)=R(x)-C(x)=x\left(45-0{,}001 x^2\right)-(100+30 x)=-0{,}001 x^3+15 x-100.
			$$
			\item Xét hàm số $P(x)=-0{,}001 x^3+15 x-100$ với $0 \leq x \leq 100$, ta có
			$$
			\begin{aligned}
				& P'(x)=-0{,}003 x^2+15; \\
				& P'(x)=0 \Leftrightarrow-0{,}003 x^2+15=0 \Leftrightarrow x^2=5\,000 \Leftrightarrow x=50 \sqrt{2} \in[0 ; 100].
			\end{aligned}
			$$
			
			Ta có $P(0)=-100$; $P(50 \sqrt{2})=500 \sqrt{2}-100 \approx 607$; $P(100)=400$.\\
			Bảng biến thiên
			\begin{center}
				\begin{tikzpicture}
					\tkzTabInit[lgt=1, espcl=4]
					{$x$/1,$y'$/0.6,$y$/3}{$0$,$50\sqrt{2}$,$100$}
					\tkzTabLine{,+,0,-,}
					\tkzTabVar{-/$100$,+/$500\sqrt{2}-100$,-/$400$}
				\end{tikzpicture}
			\end{center}
			
			Từ bảng biến thiên, ta có $\max \limits_{[0 ; 100]} P=P(50 \sqrt{2})=500 \sqrt{2}-100 \approx 607$.\\
			Vậy $A$ thu được lợi nhuận lớn nhất khi bán $50 \sqrt{2} \approx 70{,}7$ tấn sản phẩm cho $B$ mỗi tháng và lợi nhuận lớn nhất thu được khoảng $607$ triệu đồng.
		\end{enumerate}
	}
\end{ex}
\dongcham{20}
\Closesolutionfile{ans}
%%Bài 3. Tiệm cận
% \setcounter{section}{2}
\section{ĐƯỜNG TIỆM CẬN CỦA ĐỒ THỊ HÀM SỐ}
\subsection{LÝ THUYẾT CẦN NHỚ}
\subsubsection{Đường tiệm cận ngang (TCN):}
\begin{enumerate}[\iconMT]
	\item \indam{Định nghĩa:} Đường thẳng $y=m$ được gọi là một \inden{đường tiệm cận ngang} (hay \inden{tiệm cận ngang}) của đồ thị hàm số $y=f(x)$ nếu 
	$$\lim\limits_{x \rightarrow-\infty} f(x)=m \text{ hoặc }\lim\limits_{x \rightarrow+\infty} f(x)=m.$$
Đường thẳng $y=m$ là tiệm cận ngang của đồ thị hàm số $y=f(x)$ được minh hoạ như hình bên dưới\\
	\begin{tikzpicture}[scale=1,>=stealth, font=\footnotesize, line join=round, line cap=round]
		\def\xmin{-4} \def\xmax{2}
		\def\ymin{-0.5} \def\ymax{3}
		%\draw[color=gray!50,dashed] (\xmin,\ymin) grid (\xmax,\ymax);
		\draw[->] (\xmin,0)--(\xmax,0) node [below]{$x$};
		\draw[->] (0,\ymin)--(0,\ymax) node [left]{$y$};
		\fill (0,0) circle (1pt) node[shift={(-135:2.5mm)}]{$O$};
		\node at (current bounding box.south) [below=-2pt] {a) $\lim\limits_{x \rightarrow-\infty} f(x)=m$};
		\clip (\xmin+0.1,\ymin+0.1) rectangle (\xmax-0.1,\ymax-0.1);
		\draw[red,thick,smooth,samples=300,domain=\xmin:\xmax]
		(-4,0.9)..controls +(0:2) and +(180:0.5)
		..(-1.5,0.5)..controls +(0:0.5) and +(180:0.5)
		..(-0.3,1.4)..controls +(0:0.5) and +(135:1)
		..(1.8,0.3);
		\draw [blue](\xmin,1)--(\xmax,1);
		\path[blue] (-3,1)node[above]{$y=m$};
		\path[red] (0,1.3)node[above left]{$y=f(x)$};
		\fill (0,1) circle (1pt) node[shift={(-135:3mm)}]{$m$};
	\end{tikzpicture}\hspace*{.5cm}
	\begin{tikzpicture}[scale=1,>=stealth, font=\footnotesize, line join=round, line cap=round]
		\def\xmin{-1.5} \def\xmax{4}
		\def\ymin{-0.5} \def\ymax{3}
		%\draw[color=gray!50,dashed] (\xmin,\ymin) grid (\xmax,\ymax);
		\draw[->] (\xmin,0)--(\xmax,0) node [below]{$x$};
		\draw[->] (0,\ymin)--(0,\ymax) node [left]{$y$};
		\fill (0,0) circle (1pt) node[shift={(-135:2.5mm)}]{$O$};
		\node at (current bounding box.south) [below=-2pt] {b) $\lim\limits_{x \rightarrow+\infty} f(x)=m$};
		\clip (\xmin+0.1,\ymin+0.1) rectangle (\xmax-0.1,\ymax-0.1);
		\draw[red,thick,smooth,samples=300,domain=\xmin:\xmax]
		(-1,3)..controls +(-80:1) and +(170:1)
		..(0.5,1.1)..controls +(170:-1) and +(180:-0.5)
		..(3.9,0.8);
		\draw [blue](\xmin,0.7)--(\xmax,0.7);
		\path[blue] (4,0.7)node[below left]{$y=m$};
		\path[red] (0.5,1)node[above right]{$y=f(x)$};
		\fill (0,0.7) circle (1pt) node[shift={(-135:3mm)}]{$m$};
	\end{tikzpicture}
	\item \indam{Các bước tìm TCN:}
	\begin{boxdn}
		\begin{listEX}[1]
			\item [\ding{172}] Tính $\lim \limits_{x \to +\infty} f(x)$ và $\lim \limits_{x \to -\infty} f(x)$.
			\item [\ding{173}] Xem ở "vị trí" nào ra kết quả hữu hạn thì ta kết luận có tiệm cận ngang ở "vị trí" đó.
		\end{listEX}
	\end{boxdn}
\end{enumerate}
\subsubsection{Đường tiệm cận đứng (TCĐ)}
\begin{enumerate}[\iconMT]
	\item \indam{Định nghĩa:}	Đường thẳng $x=a$ được gọi là một \inden{đường tiệm cận đứng} (hay \inden{tiệm cận đứng}) của đồ thị hàm số $y=f(x)$ nếu ít nhất một trong các điều kiện sau thoả mãn:		
	$$
	\lim\limits_{x \rightarrow a^{-}} f(x)=+\infty,\,\, \lim\limits_{x \rightarrow a^{+}} f(x)=+\infty,\,\, \lim\limits_{x \rightarrow a^{-}} f(x)=-\infty,\,\, \lim\limits_{x \rightarrow a^{+}} f(x)=-\infty \text {. }
	$$
		Đường thẳng $x=a$ là tiệm cận đứng của đồ thị hàm số $y=f(x)$ được minh hoạ như hình bên dưới.\\
		\begin{center}
		\begin{tikzpicture}[scale=.7,>=stealth, font=\footnotesize, line join=round, line cap=round]
			%Hình a
			\def\xmin{-2.2} \def\xmax{3.5}
			\def\ymin{-2} \def\ymax{2} 
			%\draw[color=gray!50,dashed] (\xmin,\ymin) grid (\xmax,\ymax); 
			\draw[->] (\xmin,0)--(\xmax,0) node [below]{$x$};
			\draw[->] (0,\ymin)--(0,\ymax) node [left]{$y$};
			\fill (0,0) circle (1pt) node[shift={(-45:2.5mm)}]{$O$};
			\draw (2.1,\ymin)--(2.1,\ymax)node[below right]{$x=a$};
			\fill (2.1,0) circle (1pt) node[shift={(-45:3mm)}]{$a$};
			%\clip (\xmin+0.1,\ymin+0.1) rectangle (\xmax-0.1,\ymax-0.1);
			\draw[red] (-2,-1)..controls +(80:0.5) and +(0:-.5)..(-1,0.5)node[above]{$y=f(x)$}
			..controls +(0:0.5) and +(180:0.5)..(0.5,-1.5)
			..controls +(0:0.5) and +(87:-0.2)..(1.6,0)
			..controls +(87:-.2) and +(90:-0.2)
			..(2,1.85);
			\node at (current bounding box.south) [below=-2pt] {a) $\lim\limits_{x \rightarrow a^{-}} f(x)=+\infty$};
		\end{tikzpicture}
		\begin{tikzpicture}[scale=.7,>=stealth, font=\footnotesize, line join=round, line cap=round]
			%Hình b
			\def\xmin{-1.2} \def\xmax{4}
			\def\ymin{-2} \def\ymax{2} 
			%\draw[color=gray!50,dashed] (\xmin,\ymin) grid (\xmax,\ymax); 
			\draw[->] (\xmin,0)--(\xmax,0) node [below]{$x$};
			\draw[->] (0,\ymin)--(0,\ymax) node [left]{$y$};
			\fill (0,0) circle (1pt) node[shift={(-45:2.5mm)}]{$O$};
			\draw (1,\ymin)node[above right]{$x=a$}--(1,\ymax);
			\fill (1,0) circle (1pt) node[shift={(-135:3mm)}]{$a$};
			\path[red] (1.25,1)node[above right]{$y=f(x)$};
			%\clip (\xmin+0.1,\ymin+0.1) rectangle (\xmax-0.1,\ymax-0.1);
			\draw[red] (1.2,2)..controls +(80:0) and +(0:-1.4)..(2.5,-0.8)
			..controls +(0:0.1) and +(-80:-0.6)
			..(3.5,-1.5);
			\node at (current bounding box.south) [below=-2pt] {b) $\lim\limits_{x \rightarrow a^{+}} f(x)=+\infty$};		
		\end{tikzpicture}\\
		\begin{tikzpicture}[scale=.7,>=stealth, font=\footnotesize, line join=round, line cap=round]
			%Hình c
			\def\xmin{-2.2} \def\xmax{3.5}
			\def\ymin{-2} \def\ymax{2} 
			%\draw[color=gray!50,dashed] (\xmin,\ymin) grid (\xmax,\ymax); 
			\draw[->] (\xmin,0)--(\xmax,0) node [below]{$x$};
			\draw[->] (0,\ymin)--(0,\ymax) node [left]{$y$};
			\fill (0,0) circle (1pt) node[shift={(-45:2.5mm)}]{$O$};
			\draw (2,\ymin)--(2,\ymax)node[below right]{$x=a$};
			\fill (2,0) circle (1pt) node[shift={(-45:3mm)}]{$a$};
			\path[red] (-2.25,1.2)node[below right]{$y=f(x)$};
			%\clip (\xmin+0.1,\ymin+0.1) rectangle (\xmax-0.1,\ymax-0.1);
			\draw[red] (-2,1.4)..controls +(-10:-0.2) and +(-55:-.7)
			..(1.3,0.65)..controls +(-50:0.4) and +(-90:0)
			..(1.8,-2)
			;
			\node at (current bounding box.south) [below=-2pt] {c) $\lim\limits_{x \rightarrow a^{-}} f(x)=-\infty$};		
		\end{tikzpicture}
		\begin{tikzpicture}[scale=.7,>=stealth, font=\footnotesize, line join=round, line cap=round]
			%Hình d
			\def\xmin{-2.2} \def\xmax{3.5}
			\def\ymin{-2} \def\ymax{2} 
			%\draw[color=gray!50,dashed] (\xmin,\ymin) grid (\xmax,\ymax); 
			\draw[->] (\xmin,0)--(\xmax,0) node [below]{$x$};
			\draw[->] (0,\ymin)--(0,\ymax) node [left]{$y$};
			\fill (0,0) circle (1pt) node[shift={(-135:2.5mm)}]{$O$};
			\draw (.6,\ymin)--(.6,\ymax)node[below right]{$x=a$};
			\fill (.6,0) circle (1pt) node[shift={(-135:3mm)}]{$a$};
			%\clip (\xmin+0.1,\ymin+0.1) rectangle (\xmax-0.1,\ymax-0.1);
			\draw[red] (0.7,-2)..controls +(85:0.2) and +(180:0.2)
			..(1.2,-0.3)..controls +(0:0.2) and +(180:0.2)
			..(1.7,-0.6)..controls +(0:0.4) and +(90:0)
			..(2.5,2)
			;
			\node at (current bounding box.south) [below=-2pt] {d) $\lim\limits_{x \rightarrow a^{+}} f(x)=-\infty$};		
		\end{tikzpicture}
	\end{center}
	\item \indam{Các bước tìm TCĐ:}
	\begin{boxdn}
		\begin{listEX}[1]
			\item [\ding{172}] Tìm nghiệm của mẫu, giả sử nghiệm đó là $x=x_0$.
			\item [\ding{173}] Tính giới hạn một bên tại $x_0$. Nếu xảy ra $\lim \limits_{x \to x_0^{-}} f(x) =\infty \text{ hoặc} \lim \limits_{x \to x_0^{+}} f(x) =\infty$
			thì ta kết luận $x=x_0$ là đường tiệm cận đứng.
		\end{listEX}
	\end{boxdn}
\end{enumerate}
\subsubsection{Đường tiệm cận xiên}
\begin{enumerate}[\iconMT]
	\item \indam{Định nghĩa:} Đường thẳng $y=ax+b$, $a \neq 0$, được gọi là \inden{đường tiệm cận xiên} (hay \inden{tiệm cận xiên}) của đồ thị hàm số $y=f(x)$ nếu 
	$$\lim\limits_{x \rightarrow-\infty}[f(x)-(ax+b)]=0 \text{ hoặc }\lim\limits_{x \rightarrow+\infty}[f(x)-(ax+b)]=0.$$
	Đường thẳng $y=ax+b$ là tiệm cận xiên của đồ thị hàm số $y=f(x)$ được minh hoạ như hình bên dưới:\\	
		\begin{tikzpicture}[scale=1,>=stealth, font=\footnotesize, line join=round, line cap=round]
			\def\xmin{-4} \def\xmax{2.5}
			\def\ymin{-0.5} \def\ymax{3}
			%\draw[color=gray!50,dashed] (\xmin,\ymin) grid (\xmax,\ymax);
			\draw[->] (\xmin,0)--(\xmax,0) node [below]{$x$};
			\draw[->] (0,\ymin)--(0,\ymax) node [left]{$y$};
			\fill (0,0) circle (1pt) node[shift={(-135:2.5mm)}]{$O$};
			\node at (current bounding box.south) [below=-2pt] {a) $\lim\limits_{x \rightarrow-\infty}\left[f(x)-(ax+b)\right]=0$};
			\clip (\xmin+0.1,\ymin+0.1) rectangle (\xmax-0.1,\ymax-0.1);
			\draw[red,thick,smooth,samples=300,domain=\xmin:\xmax]
			(-3.8,-0.6)..controls +(34:0.5) and +(180:.75)
			..(-0.2,1.2)..controls +(0:0.75) and +(180:.75)
			..(1,0.3)..controls +(0:0.5) and +(80:0)
			..(2.2,1);
			\draw[blue,smooth,samples=300,domain=\xmin:\xmax] plot(\x,{2/3*(\x)+2});
			\path[blue] (-3,0)--(0,2)node[below,sloped,pos=1.3]{$y=ax+b$};
			\path[red] (0.5,1)node[above right]{$y=f(x)$};
		\end{tikzpicture}\hspace{.5cm}
		\begin{tikzpicture}[scale=1,>=stealth, font=\footnotesize, line join=round, line cap=round]
			\def\xmin{-3.5} \def\xmax{3}
			\def\ymin{-0.5} \def\ymax{3}
			%\draw[color=gray!50,dashed] (\xmin,\ymin) grid (\xmax,\ymax);
			\draw[->] (\xmin,0)--(\xmax,0) node [below]{$x$};
			\draw[->] (0,\ymin)--(0,\ymax) node [left]{$y$};
			\fill (0,0) circle (1pt) node[shift={(-135:2.5mm)}]{$O$};
			\node at (current bounding box.south) [below=-2pt] {a) $\lim\limits_{x \rightarrow+\infty}\left[f(x)-(ax+b)\right]=0$};
			\clip (\xmin+0.1,\ymin+0.1) rectangle (\xmax-0.1,\ymax-0.1);
			\draw[red,thick,smooth,samples=300,domain=\xmin:\xmax]
			(-3,0.8)..controls +(60:0.5) and +(180:.75)
			..(-1.5,2)..controls +(0:.5) and +(180:.75)
			..(0.5,1.3)..controls +(0:.75) and +(-160:.5)
			..(2.8,1.8);
			\draw[blue,smooth,samples=300,domain=\xmin:\xmax] plot(\x,{1/3*(\x)+0.75});
			\path[blue] (-3,-0.25)--(0,0.75)node[below,sloped,pos=1.6]{$y=ax+b$};
			\path[red] (-2.5,2)node[above right]{$y=f(x)$};
		\end{tikzpicture}
	\item \indam{Các bước tìm TCX y = ax + b:}
	Ta xác định hệ số của $a$ và $b$ trong 2 trường hợp sau:
	\begin{boxdn}
		\begin{listEX}[1]
			\item [\ding{172}] Tính $a=\lim\limits_{x \rightarrow+\infty} \dfrac{f(x)}{x}$, $b=\lim\limits_{x \rightarrow+\infty}[f(x)-ax]$.
			\item [\ding{173}] Tính $a=\lim\limits_{x \rightarrow-\infty} \dfrac{f(x)}{x}$, $b=\lim\limits_{x \rightarrow-\infty}[f(x)-ax]$.
		\end{listEX}
	\end{boxdn}
\end{enumerate}
\subsection{PHÂN LOẠI VÀ PHƯƠNG PHÁP GIẢI TOÁN}
\begin{dang}{Bài toán tìm tiệm cận đứng và tiệm cận ngang của đồ thị hàm số}
	Cho hàm số $y=f(x)$. Để tìm tiệm cận đứng và tiệm cận ngang, ta làm như sau:
	\begin{enumerate}[\iconCH]
		\item \indamm{Các bước tìm tiệm cận đứng:}
		\begin{listEX}[1]
			\item [\ding{172}] Tìm nghiệm của mẫu, giả sử nghiệm đó là $x=x_0$.
			\item [\ding{173}] Tính giới hạn một bên tại $x_0$. Nếu xảy ra $\lim \limits_{x \to x_0^{-}} f(x) =\infty \text{ hoặc} \lim \limits_{x \to x_0^{+}} f(x) =\infty$
			thì ta kết luận $x=x_0$ là đường tiệm cận đứng.
		\end{listEX}
		\item \indamm{Các bước tìm tiệm cận ngang:}
		\begin{listEX}[1]
			\item [\ding{172}] Tính $\lim \limits_{x \to +\infty} f(x)$ và $\lim \limits_{x \to -\infty} f(x)$.
			\item [\ding{173}] Xem ở "vị trí" nào ra kết quả hữu hạn thì ta kết luận có tiệm cận ngang ở "vị trí" đó.
		\end{listEX}
		\item \indamm{Lưu ý:} Đồ thị hàm số $y=\dfrac{ax+b}{cx+d}$ luôn có TCĐ $x=-\dfrac{d}{c}$ và TCN: $y=\dfrac{a}{c}$.
	\end{enumerate}
\end{dang}
\boxmini{BÀI TẬP TỰ LUẬN}
\begin{vd}
	Xác định tiệm cận đứng và tiệm cận ngang của đồ thị hàm số cho bởi công thức sau:
	\begin{enumEX}[a)]{4}
		\item $y=\dfrac{2x-1}{x+1}$;
		\item $y=\dfrac{2 x-3}{1-2 x}$;
		\item $y=\dfrac{x^2-5x+4}{x^2-1}$;
		\item $y=\dfrac{2x-1}{x^2-3x+2}$.
	\end{enumEX}
\loigiai{
\begin{enumerate}[a)]
	\item Xét $\lim\limits_{x \to -1^+} \dfrac{2x-1}{x+1}=-\infty$ (hoặc $\lim\limits_{x \to -1^-} \dfrac{2x-1}{x+1}=+\infty$) nên đường thẳng $x=-1$ là tiệm cận đứng.\\
	Xét $\lim\limits_{x \to \pm \infty } \dfrac{2x-1}{x+1}=2$ nên đường thẳng $y=2$ là tiệm cận ngang.
	\item Ta có
	\begin{itemize}
		\item $\lim\limits_{x \to \pm\infty} y=\lim\limits_{x \to \pm\infty} \dfrac{2x-3}{1-2x}=-1$ suy ra $y=-1$ là tiệm cận ngang.
		\item $\heva{& \lim\limits_{x \to \tfrac{1}{2}^+} \dfrac{2x-3}{1-2x}=+\infty \\ & \lim\limits_{x \to \tfrac{1}{2}^-} \dfrac{2x-3}{1-2x}=-\infty}$ suy ra $x=\dfrac{1}{2}$ là tiệm cận đứng.
	\end{itemize}
	\item Điều kiện xác định: $\heva{&x\neq-1\\ &x\neq1.}$
	\begin{itemize}
		\item $\lim\limits_{x\to\pm\infty}\dfrac{x^2-5x+4}{x^2-1}=1$
		\item $\lim\limits_{x\to(-1)^-}\dfrac{x^2-5x+4}{x^2-1}=+\infty$
		\item $\lim\limits_{x\to1}\dfrac{x^2-5x+4}{x^2-1}=-\dfrac{3}{2}$
	\end{itemize}
	Vậy đồ thị hàm số có một tiệm cận ngang $y=1$ và một tiệm cận đứng $x=-1$.
	\item Tập xác định $\mathscr{D}=\mathbb{R}\setminus\{1; 2\}$.\\
	Ta có\begin{itemize}
		\item $\lim\limits_{x\to\pm\infty}y=\lim\limits_{x\to\pm\infty}\dfrac{2x-1}{x^2-3x+2}=0$ nên $y=0$ là đường tiệm cận ngang.
		\item $\lim\limits_{x\to 1^-}y=\lim\limits_{x\to 1^-}\dfrac{2x-1}{x^2-3x+2}=\lim\limits_{x\to 1^-}\dfrac{2x-1}{(x-1)(x-2)}=+\infty$ và $\lim\limits_{x\to 1^+}y=-\infty$ nên $x=1$ là đường tiệm cận đứng.
		\item $\lim\limits_{x\to 2^-}y=\lim\limits_{x\to 2^-}\dfrac{2x-1}{x^2-3x+2}=\lim\limits_{x\to 2^-}\dfrac{2x-1}{(x-1)(x-2)}=-\infty$ và $\lim\limits_{x\to 2^+}y=2=+\infty$ nên $x=2$ là đường tiệm cận đứng.
	\end{itemize}
\end{enumerate}}
\end{vd}

\boxmini{BÀI TẬP TRẮC NGHIỆM}
\ind{PHẦN I.} \inden{Câu trắc nghiệm nhiều phương án lựa chọn. Mỗi câu hỏi học sinh chỉ chọn một phương án.}\\
\setcounter{ex}{0}
\Opensolutionfile{ans}[ans/2D1-B3-d1-1]
\begin{ex}
	Đường tiệm cận ngang của đồ thị hàm số $y=\dfrac{2x-4}{x+2}$ là
	\choice
	{\True $y=2$}
	{  $x=2$}
	{ $x=-2$}
	{$y=-2$}
	
	\loigiai{$\underset{x\to -\infty }{\mathop{\lim \limits_{n \to +\infty}}}\,\dfrac{2x-4}{x+2}=2$ và $\underset{x\to +\infty }{\mathop{\lim \limits_{n \to +\infty}}}\,\dfrac{2x-4}{x+2}=2$ nên hàm số có tiệm cận ngang là $y=2$.
	}
\end{ex}

\begin{ex}
	Tìm tiệm cận ngang của đồ thị hàm số $ y = \dfrac{2x + 1}{ x +1} $.
	\choice
	{ \True $  y = -2 $}
	{$  x = -2 $}
	{ $  y = 2 $}
	{$ x = 1 $}
	\loigiai{
		Ta có $ \displaystyle \lim_{ x \rightarrow \pm \infty } \dfrac{2x + 1}{-x + 1} = -2  $.	
	}		
\end{ex}

\begin{ex}
	Đường thẳng $y=3$ là tiệm cận ngang của đồ thị hàm số nào sau đây?
	\choice
	{$y=\dfrac{1-3x}{2+x}$}
	{$y=\dfrac{x^2+3x+2}{x-2}$}
	{\True $y=\dfrac{1+3x}{1+x}$}
	{$y=\dfrac{3x^2+2}{2-x}$}
	\loigiai{
		Ta có $\lim\limits_{x\to \pm \infty}\dfrac{1+3x}{1+x}=3$ nên $y=3$ là tiệm cận ngang của đồ thị hàm số $y=\dfrac{1+3x}{1+x}$.}
\end{ex}

\begin{ex}
	Hàm số nào có đồ thị nhận đường thẳng $x = 2$ làm đường tiệm cận đứng?
	\choice
	{$y=x-2+\dfrac{1}{x+1}$}
	{$y=\dfrac{1}{x+1}$}
	{$y=\dfrac{2}{x+2}$}
	{\True $y=\dfrac{5x}{2-x}$}
	\loigiai{ Xét hàm số $y=\dfrac{5x}{2-x}$\\
		Ta có $\lim\limits_{x\to 2^+}5x=10>0$; $\lim\limits_{x\to 2^+}(2-x)$ và $x-2<0$ khi $x>2$ suy ra $\lim\limits_{x\to 2^+}\dfrac{5x}{2-x}=-\infty$.\\
		Vậy đồ thị hàm số $y=\dfrac{5x}{2-x}$ nhận đường thẳng $x=2$ làm tiệm cận đứng.
	}
\end{ex}

\begin{ex}
	Đường tiệm cận đứng của đồ thị hàm số $y=\dfrac{3x+1}{x-2}$ là đường thẳng
	\choice
	{$x=-2$}
	{\True $x=2$}
	{$y=3$}
	{$y=-\dfrac{1}{2}$}
	\loigiai{Ta có: $\lim \limits_{x\to 2^+}{\dfrac{3x+1}{x-2}}=+\infty$.
	}
\end{ex}

\begin{ex}
	Đường tiệm cận đứng của đồ thị hàm số $y=\dfrac{x+1}{x^2+4x-5}$ có phương trình là
	\choice
	{$x=-1$}
	{$y=1;y=-5$}
	{\True $x=1;x=-5$}
	{$x=\pm 5$}
	\loigiai{
		Ta có $\mathop{\lim}\limits_{x\rightarrow 1^+}y=+\infty$, $\mathop{\lim}\limits_{x\rightarrow 1^-}y=-\infty$, $\mathop{\lim}\limits_{x\rightarrow 5^+}y=+\infty$, $\mathop{\lim}\limits_{x\rightarrow 5^-}y=-\infty$.\\
		Vậy đồ thị hàm số có hai đường tiệm cận đứng là $x=1$ và $x=-5$.}
\end{ex}

\begin{ex}
	Tìm số đường tiệm cận của đồ thị hàm số $ y = \dfrac{x^2 - 3x + 2}{x^2 - 4}. $
	\choice
	{$1$}
	{$ 0$}
	{\True $2$}
	{$3$}
	\loigiai
	{
		Tập xác định: $ \mathscr D = \mathbb{R} \backslash \{\pm2 \} $.\\
		Ta có $ \lim \limits_{x \to \pm  \infty} y = 1 \Rightarrow  $ đồ thị hàm số có 1 tiệm cận ngang là $ y = 1. $\\
		Ta lại có $\lim \limits_{x \to 2} y =  \lim \limits_{x \to 2} \dfrac{x-1}{x+2} = \dfrac{1}{4} $ và $\lim \limits_{x \to -2^+} y =  \lim \limits_{x \to -2^+} \dfrac{x-1}{x+2} = -\infty$ nên đồ thị hàm số có 1 tiệm cận đứng là $ x = -2. $\\
		Vậy đồ thị hàm số đã cho có 2 đường tiệm cận.
	}
\end{ex}

\begin{ex}
	Số đường tiệm cận của đồ thị hàm số $y=\dfrac{3}{x-2}$ là
	\choice
	{$1$}
	{\True $2$}
	{$0$}
	{$3$}
	\loigiai{
		Tiệm cận đứng $x=2$.\\
		Tiệm cận ngang $y=0$.
	}
\end{ex}

\begin{ex}
	Cho hàm số $y=f(x)$ có đồ thị là đường cong $(C)$ và các giới hạn $\lim\limits_{x\to 2^{+}}f(x)=1$, $\lim\limits_{x\to 2^{-}}f(x)=1$, $\lim\limits_{x\to +\infty}f(x)=2$, $\lim\limits_{x\to -\infty}f(x)=2$. Hỏi mệnh đề nào sau đây đúng?
	\choice
	{\True Đường thẳng $y=2$ là tiệm cận ngang của $(C)$}
	{Đường thẳng $y=1$ là tiệm cận ngang của $(C)$}
	{Đường thẳng $x=2$ là tiệm cận ngang của $(C)$}
	{Đường thẳng $x=2$ là tiệm cận đứng của $(C)$}
	\loigiai{
		Ta có $\lim\limits_{x\to +\infty}f(x)=2$, $\lim\limits_{x\to -\infty}f(x)=2\Rightarrow y=2$ là tiệm cận ngang của $(C)$.
	}
\end{ex}

\begin{ex}
	Số tiệm cận đứng của đồ thị hàm số $y=\dfrac{\sqrt{x+9}-3}{x^2+x}$ là
	\choice
	{$3$}
	{$2$}
	{$0$}
	{\True $1$}
	\loigiai{
		Tập xác định $\mathscr{D}=[-9;+\infty)\setminus \{-1;0\}$. \\
		Ta có $\left\{\begin{aligned}
			&\lim\limits_{x\to -1^+} \dfrac{\sqrt{x+9}-3}{x^2+x}=+\infty \\
			&\lim\limits_{x\to -1^-} \dfrac{\sqrt{x+9}-3}{x^2+x}=-\infty
		\end{aligned}\right. \Rightarrow x=-1$ là tiệm cận đứng. \\
		Ngoài ra $\lim\limits_{x\to 0} \dfrac{\sqrt{x+9}-3}{x^2+x}=\dfrac{1}{6}$ nên $x=0$ không thể là một tiệm cận được.}
\end{ex} 

\begin{ex}%[2D1B4]
	\immini{Cho hàm số $y=f(x)$ xác định trên $\mathbb{R}\setminus\left\{\pm1\right\}$ liên tục trên mỗi khoảng xác định và có bảng biến thiên như hình vẽ. Số đường tiệm cận của đồ thị hàm số là
	\choice
	{$1$}
	{$2$}
	{\True $3$}
	{$4$}}{\hspace{0.5cm}
\begin{tikzpicture}
	\tikzset{double style/.append style = {draw=\tkzTabDefaultWritingColor,double=\tkzTabDefaultBackgroundColor,double distance=2pt}}
	\tikzset{double style/.append style = {double distance=0.5pt}} 
	\tkzTabInit[nocadre=false,lgt=1,espcl=1.7]
	{$x$/.7,$y'$ /.7, $y$ /2.3}
	{$-\infty$ ,$-1$,$0$,$1$,$+\infty$}
	\tkzTabLine{,-,d,-,0,+,d,+,}
	\tkzTabVar {+/$-2$,-D+/$-\infty$/$+\infty$,-/$1$,+D-/$+\infty$/$-\infty$,+/$-2$}
\end{tikzpicture}}
	\loigiai{
		Dựa vào bảng biến thiên ta có:\\
		$\lim\limits_{x\to -1^\pm}f(x)=\pm\infty$. 
		$\lim\limits_{x\to 1^\pm}f(x)=\mp\infty$.\\
		Do đó $x=1$ và $x=-1$ là các đường tiệm cận đứng của đồ thị hàm số.\\
		Lại có $\lim\limits_{x\to \pm\infty}f(x)=-2$. Do đó $y=-2$ là tiệm cận ngang của đồ thị hàm số.\\
		Vậy đồ thị hàm số có $3$ đường tiệm cận.
	}
\end{ex}

\begin{ex}
	\immini{Cho hàm số $y=f(x)$ xác định trên $\mathbb{R}\backslash \left\{0\right\},$ liên tục trên mỗi khoảng xác định và có bảng biến thiên như hình bên. Chọn khẳng định đúng.
	\choice
	{Đồ thị hàm số có đúng một tiệm cận ngang}
	{Đồ thị hàm số có hai tiệm cận ngang}
	{\True Đồ thị hàm số có đúng một tiệm cận đứng}
	{Đồ thị hàm số không có tiệm đứng và tiệm cận ngang}}{
	\begin{tikzpicture}[>=stealth]
		\tikzset{double style/.append style = {draw=\tkzTabDefaultWritingColor,double=\tkzTabDefaultBackgroundColor,double distance=2pt}}
		\tkzTabInit[nocadre=false,lgt=1,espcl=2]{$x$/.6,$y'$/.7,$y$/2}{$-\infty$,$0$,$1$,$+\infty$}
		\tkzTabLine{,-, d ,+,z,-,} 
		\tkzTabVar{+/$+\infty$ / , -D- / $-1$ /$-\infty$,+/$2$,-/$-\infty$}
\end{tikzpicture}}
	\loigiai{
		Do $\lim\limits_{x \to +\infty} y=-\infty$ và $\lim\limits_{x \to -\infty} y=+\infty$  nên đồ thị hàm số không có tiệm cận ngang.\\
		Do $\lim\limits_{x \to 0^+} y=+\infty$ suy ra $x=0$ là tiệm cận đứng của đồ thị hàm số.
	} 
\end{ex}

\begin{ex}
	\immini{Cho hàm số $ y=f(x) $ có bảng biến thiên như hình bên. Hỏi đồ thị hàm số đã cho có bao nhiêu đường tiệm cận?
	\choice
	{\True $ 2 $}
	{$ 3 $}
	{$ 4 $}
	{$ 1 $}}{
\begin{tikzpicture}[yscale=.8,xscale=1.15,
	kxd/.pic={\draw[double] (90:.4)--(-90:.4);}]
	\begin{scope}[shift={(-.5,.5)}]
		\fill[pattern=north east lines,pattern color=black]
		(1,-1) rectangle +(1.45,-4);
		\draw
		(0,0) rectangle +(7,-5)
		(0,-1)--+(0:7) (0,-2)--+(0:7) (1,0)--+(-90:5);
	\end{scope}
	\path
	(0,0) node{$ x $}
	++ (0:1) node{$ -\infty $}
	++(0:1)node{$ -2 $}
	++(0:2)node{$ 0 $}
	++(0:2)node{$ +\infty $}
	(0,-1)node{$ y' $}
	++(0:2)pic{kxd}
	++(0:1)node{$ + $}
	++(0:1)pic{kxd}
	++(0:1)node{$ - $}
	(0,-3)node{$ y $}
	++(0:2)pic[yscale=3]{kxd}
	+(-90:1)node[below right](A){$ -\infty $}
	++(0:2) pic[yscale=3]{kxd}
	node[above right](C){$ 1 $}
	+(90:1)node[left](B){$ 2 $}
	++(0:2)node[below](D){$ 0 $};
	\draw[-stealth,black](A)--(B)
	;
	\draw[-stealth,black] (C)--(D);
\end{tikzpicture}}
	\loigiai{
		Dựa vào bảng biến thiên của hàm số, suy ra
		\begin{itemize}
			\item  $ \lim\limits_{x \to +\infty} f(x)=0 $, đồ thị hàm số có tiệm cận ngang là $ y=0 $.
			\item $ \lim\limits_{x \to (-2)^+} f(x)=-\infty $, đồ thị hàm số có tiệm cận đứng là $ x=-2 $.
			Vậy đồ thị hàm số đã cho có $ 2 $ đường tiệm cận.
		\end{itemize}
	}
\end{ex}

\Closesolutionfile{ans}

\ind{PHẦN II.} \inden{Câu trắc nghiệm đúng sai. Trong mỗi ý a), b), c), d) ở mỗi câu, học sinh chọn đúng hoặc sai.}\\
\Opensolutionfile{ans}[ans/2D1-B3-d1-2]
\begin{ex}
	Cho hàm số $y=f(x)$ có bảng biến thiên như hình bên. Xét tính đúng, sai của các khẳng định sau:
	\begin{center}
		\begin{tikzpicture}
			\tikzset{double style/.append style = {draw=\tkzTabDefaultWritingColor,double=\tkzTabDefaultBackgroundColor,double distance=2pt}}
			\tkzTab[nocadre=false,lgt=1.2,espcl=1.7,deltacl=0.6]
			{$x$/0.6, $y'$/0.6, $y$/2}
			{$-\infty$, $0$, $2$, $+\infty$}
			{,-,d,-,$0$,+,}
			{+/ $2$, -D+/ $-\infty$ / $+\infty$, -/ $2$,+/$+\infty$}
		\end{tikzpicture}
	\end{center}
	\choiceTF
	{\True $f(-5)<f(4)$}
	{Hàm số có giá trị nhỏ nhất bằng $2$}
	{\True Đồ thị hàm số có tiệm cận đứng $x=0$}
	{Đồ thị hàm số không có tiệm cận ngang}
	\loigiai{
		\begin{enumerate}[a)]
			\item Từ bảng biến thiên ta thấy $f(-5)<2$ và $f(4)>2$ nên $f(-5)<f(4)$.
			\item Do $\lim \limits_{x\to 0^-}y=-\infty$ nên hàm số không có giá trị nhỏ nhất.
			\item Do $\lim \limits_{x\to 0^-}y=-\infty$ nên đồ thị hàm số có tiệm cận đứng $x=0$.
			\item Do $\lim \limits_{x\to -\infty}y=2y$ nên đồ thị hàm số có tiệm cận ngang $y=2$.
		\end{enumerate}
}
\end{ex}


\begin{ex}
	Cho hàm số hàm số $y=\dfrac{-4x+5}{2x+3}$ có đồ thị $(C)$.
	Xét tính đúng sai của các khẳng định sau:
	\choiceTF
	{\True Hàm số không có cực trị}
	{Đồ thị hàm số có tiệm cận đứng $x=-3$}
	{Đồ thị hàm số có tiệm cận ngang $y=-2$}
	{\True Các đường tiệm cận của đồ thị tạo với hai trục toạ độ một hình chữ nhật có diện tích bằng $3$}
	\loigiai{
		Tập xác định $\mathscr D=\mathbb{R}\setminus \left \{-\dfrac{3}{2}\right \}$\\
		$\lim \limits_{x\to \left (-\frac{3}{2}\right )^+}y=+\infty; \ \lim \limits_{x\to \left (-\frac{3}{2}\right )^-}y=-\infty$ nên đồ thị hàm số có tiệm cận đứng $x=-\dfrac{3}{2}$\\
		$\lim \limits_{x\to -\infty}y=-2, \ \lim \limits_{x\to +\infty}y=-2$ nên đồ thị hàm số có một tiệm cận ngang là $y=-2$\\
		Diện tích hình chữ nhật cần tìm là $S=\left |-\dfrac{3}{2}\right |\cdot \left |-2\right |=3$
	}
\end{ex}

\Closesolutionfile{ans}

\begin{dang}{Bài toán tìm tiệm cận đứng và tiệm cận xiên của đồ thị hàm số}
	\begin{enumerate}[\iconCH]
		\item \indamm{Các bước tìm TCX y = ax + b:}
		Ta xác định hệ số của $a$ và $b$ trong 2 trường hợp sau:
			\begin{listEX}[1]
				\item [\ding{172}] Tính $a=\lim\limits_{x \rightarrow+\infty} \dfrac{f(x)}{x}$, $b=\lim\limits_{x \rightarrow+\infty}[f(x)-ax]$.
				\item [\ding{173}] Tính $a=\lim\limits_{x \rightarrow-\infty} \dfrac{f(x)}{x}$, $b=\lim\limits_{x \rightarrow-\infty}[f(x)-ax]$.
			\end{listEX}
		\item \indamm{Lưu ý:} 
		\begin{listEX}[1]
			\item [\ding{172}] Nếu $a=0$ thì tiệm cận xiên chính là tiệm cận ngang.
			\item [\ding{173}] Đối với hàm số phân thức $f(x)=\dfrac{ax^2+bx+c}{mx+n}$, ta có thể chia đa thức, biến đổi về dạng
			$$f(x)=a'x+b'+\dfrac{e}{mx+n}, \, \text{ với } e \ne0$$
			Suy ra $y=a'x+b'$ là đường tiệm cận xiên của đồ thị hàm số.
		\end{listEX}
	\end{enumerate}
	
\end{dang}
\boxmini{BÀI TẬP TỰ LUẬN}

\begin{vd}
	Tìm các tiệm cận đứng và tiệm cận xiên của đồ thị hàm số sau:
	\begin{listEX}[3]
		\item $y=\dfrac{x^{2}+2}{2x-4}$;
		\item $y=\dfrac{2x^{2}-3x-6}{x+2}$;
		\item $y=\dfrac{2x^{2}+9x+11}{2x+5}$.
	\end{listEX}
	\loigiai{
		\begin{listEX}
			\item Hàm số $y=f(x)=\dfrac{x^{2}+2}{2x-4}$ có tập xác định $\mathscr{D}=\mathbb{R} \setminus \left\lbrace 2\right\rbrace$.
			\begin{itemize}
				\item Ta có $\lim\limits_{x \rightarrow 2^{-}} \dfrac{x^{2}+2}{2x-4}=-\infty$; $\lim\limits_{x \rightarrow 2^{+}} \dfrac{x^{2}+2}{2x-4}=+\infty$.\\
				Suy ra đường thẳng $x=2$ là một tiệm cận đứng của đồ thị hàm số.
				\item Ta có $\begin{aligned}[t]
					a&=\lim\limits_{x \rightarrow+\infty} \dfrac{f(x)}{x}=\lim\limits_{x \rightarrow+\infty} \dfrac{x+\dfrac{2}{x}}{2x-4}=\dfrac{1}{2};\\
					b&=\lim\limits_{x \rightarrow+\infty}[f(x)-ax]=\lim\limits_{x \rightarrow+\infty}\left(\dfrac{x^{2}+2}{2x-4}-\dfrac{1}{2}x\right)=\lim\limits_{x \rightarrow+\infty} \dfrac{2x+2}{2x-4}=1.
				\end{aligned}$\\
				Ta cũng có $\lim\limits_{x \rightarrow-\infty} \dfrac{f(x)}{x}=\dfrac{1}{2}$; $\lim\limits_{x \rightarrow-\infty}[f(x)-\dfrac{1}{2}x]=1$.
				\\
				Do đó, đồ thị hàm số có tiệm cận xiên là đường thẳng $y=\dfrac{1}{2}x+1$.
			\end{itemize}	
			\item Hàm số $y=f(x)=\dfrac{2x^{2}-3x-6}{x+2}$ có tập xác định $\mathscr{D}=\mathbb{R} \setminus \left\lbrace -2\right\rbrace$.
			\begin{itemize}
				\item Ta có $\lim\limits_{x \rightarrow \left(-2\right)^{-}} \dfrac{2x^{2}-3x-6}{x+2}=-\infty$; $\lim\limits_{x \rightarrow \left(-2\right)^{+}} \dfrac{2x^{2}-3x-6}{x+2}=+\infty$.\\
				Suy ra đường thẳng $x=-2$ là một tiệm cận đứng của đồ thị hàm số.
				\item Ta có $\begin{aligned}[t]
					a&=\lim\limits_{x \rightarrow+\infty} \dfrac{f(x)}{x}=\lim\limits_{x \rightarrow+\infty} \dfrac{2x-3-\dfrac{6}{x}}{x+2}=2;\\
					b&=\lim\limits_{x \rightarrow+\infty}[f(x)-ax]=\lim\limits_{x \rightarrow+\infty}\left(\dfrac{2x^{2}-3x-6}{x+2}-2x\right)=\lim\limits_{x \rightarrow+\infty} \dfrac{-7x-6}{x+2}=-7.
				\end{aligned}$\\
				Ta cũng có $\lim\limits_{x \rightarrow-\infty} \dfrac{f(x)}{x}=2$; $\lim\limits_{x \rightarrow-\infty}[f(x)-2x]=-7$.\\
				Do đó, đồ thị hàm số có tiệm cận xiên là đường thẳng $y=2x-7$.
			\end{itemize}
			\item Hàm số $y=f(x)=\dfrac{2x^{2}+9x+11}{2x+5}$ có tập xác định $\mathscr{D}=\mathbb{R} \setminus \left\lbrace -\dfrac{5}{2}\right\rbrace$.
			\begin{itemize}
				\item 
				Ta có $\lim\limits_{x \rightarrow \left(-\tfrac{5}{2}\right)^{-}} \dfrac{2x^{2}+9x+11}{2x+5}=-\infty$; $\lim\limits_{x \rightarrow \left(-\tfrac{5}{2}\right)^{+}} \dfrac{2x^{2}+9x+11}{2x+5}=+\infty$.\\
				Suy ra đường thẳng $x=-\dfrac{5}{2}$ là một tiệm cận đứng của đồ thị hàm số.
				\item Ta có $\begin{aligned}[t]
					a&=\lim\limits_{x \rightarrow+\infty} \dfrac{f(x)}{x}=\lim\limits_{x \rightarrow+\infty} \dfrac{2x+9+\dfrac{11}{x}}{2x+5}=1;\\
					b&=\lim\limits_{x \rightarrow+\infty}[f(x)-ax]=\lim\limits_{x \rightarrow+\infty}\left(\dfrac{2x^{2}+9x+11}{2x+5}-x\right)=\lim\limits_{x \rightarrow+\infty} \dfrac{4x+11}{2x+5}=2.
				\end{aligned}$\\
				Ta cũng có $\lim\limits_{x \rightarrow-\infty} \dfrac{f(x)}{x}=1$; $\lim\limits_{x \rightarrow-\infty}[f(x)-x]=2$.\\
				Do đó, đồ thị hàm số có tiệm cận xiên là đường thẳng $y=x+2$.
			\end{itemize}
		\end{listEX}	
	}
\end{vd}

\boxmini{BÀI TẬP TRẮC NGHIỆM}
\ind{PHẦN I.} \inden{Câu trắc nghiệm nhiều phương án lựa chọn. Mỗi câu hỏi học sinh chỉ chọn một phương án.}\\
\setcounter{ex}{0}
\Opensolutionfile{ans}[ans/2D1-B3-d2-1]
\begin{ex}
	Đường tiệm cận xiên của đồ thị hàm số $y=f(x)=2x-1-\dfrac{1}{x+1}$ có phương trình là
	\choice
	{$y=x+1$}
	{\True $y=2x-1$}
	{$y=x-1$}
	{$y=2x+1$}
	\loigiai{
		Do $\lim\limits_{x\to +\infty}[f(x)-(2x-1)]=\lim\limits_{x\to +\infty}\dfrac{-1}{x+1}=0$ nên đường thẳng $y=2x-1$
		là tiệm cận xiên của đồ thị hàm số đã cho.}
\end{ex}

\begin{ex}
	Đường tiệm cận xiên của đồ thị hàm số $y=f(x)=x+3+\dfrac{1}{2x+1}$ có phương trình là
	\choice
	{$y=2x+1$}
	{$y=x-3$}
	{\True $y=x+3$}
	{$y=2x-1$}
	\loigiai{
		Do $\lim\limits_{x\to \pm\infty}[f(x)-(x+3)]=\lim\limits_{x\to \pm\infty}\dfrac{1}{2x+1}=0$ nên đường thẳng $y=x+3$
		là tiệm cận xiên của đồ thị hàm số đã cho.}
\end{ex}

\begin{ex}
	Tìm tiệm cận xiên của đồ thị hàm số $y=f(x)=\dfrac{x^2+3x}{x-2}$.
	\choice
	{$y=2x-5$}
	{$y=x-2$}
	{\True $y=x+5$}
	{$y=x-5$}
	\loigiai{
	Ta có
	\begin{itemize}
		\item $a=\lim\limits_{x\to +\infty}\dfrac{f(x)}{x}=\lim\limits_{x\to +\infty}\dfrac{x^2+3x}{x(x-2)}=1$
		\item và $b=\lim\limits_{x\to +\infty}[f(x)-x]=\lim\limits_{x\to +\infty}\dfrac{5x}{x-2}=5$.
	\end{itemize}
	Vậy đường thẳng $y=x+5$ là tiệm cận xiên của đồ thị hàm số đã cho (khi $x \to +\infty$).\\
	Tương tự, do $\lim\limits_{x\to -\infty}\dfrac{f(x)}{x}=1$ và $\lim\limits_{x\to -\infty}[f(x)-x]=5$ nên đường thẳng $y=x+5$ cũng là tiệm cận xiên của đồ thị hàm số đã cho (khi $x \to -\infty$).}
\end{ex}

\begin{ex}%[2D1H4-1]
	Tiệm cận xiên của đồ thị hàm số $y=\dfrac{x^2+2x-2}{x+2}$ là
	\choice
	{$y=-2$}
	{$y=1$}
	{$y=x+2$}
	{\True $y=x$}
	\loigiai{
		Ta có $y=\dfrac{x^2+2x-2}{x+2}=\dfrac{x(x+2)-2}{x+2}=x-\dfrac{2}{x+2}$.\\
		$\underset{x\to +\infty}{\mathop{\lim}} [ y-x ] =\underset{x\to +\infty}{\mathop{\lim}}\dfrac{-2}{x+2}=0$ và $\underset{x\to -\infty}{\mathop{\lim}} [ y-x ] =\underset{x\to -\infty}{\mathop{\lim}}\dfrac{-2}{x+2}=0$. \\ 
		Vậy đồ thị hàm số có tiệm cận xiên là đường thẳng $y=x$. 
	}
\end{ex}

\begin{ex}
	Tìm tiệm cận xiên của đồ thị hàm số $f(x)=\dfrac{x^{2}-3 x+1}{x-2}$.
	\choice
	{$y=x+1$}
	{$y=-3x+1$}
	{$y=x-2$}
	{\True $y=x-1$}
	\loigiai{
		Tập xác định: $\mathscr{D}=\mathbb{R} \setminus\{2\}$.
		\\
		Ta có $\begin{aligned}[t]
			a&=\lim\limits_{x \rightarrow+\infty} \dfrac{f(x)}{x}=\lim\limits_{x \rightarrow+\infty} \dfrac{x^{2}-3 x+1}{x^{2}-2 x}=1;\\
			b&=\lim\limits_{x \rightarrow+\infty}[f(x)-a x]=\lim\limits_{x \rightarrow+\infty}\left(\dfrac{x^{2}-3 x+1}{x-2}-x\right)=\lim\limits_{x \rightarrow+\infty} \dfrac{-x+1}{x-2}=-1.
		\end{aligned}$\\
		Ta cũng có $\lim\limits_{x \rightarrow-\infty} \dfrac{f(x)}{x}=1$; $\lim\limits_{x \rightarrow-\infty}[f(x)-x]=-1$.
		\\
		Do đó, đồ thị hàm số có tiệm cận xiên là đường thẳng $y=x-1$.
	}
\end{ex}

\begin{ex}
	Đường tiệm cận xiên của đồ thị hàm số $y=\dfrac{2x^{2}-3x}{x+5}$ đi qua điểm nào sau đây?
	\choice
	{$(5;3)$}
	{$(-4;-5)$}
	{\True $(6;-1)$}
	{$(2;-10)$}
	\loigiai{
		Tập xác định: $\mathscr{D}=\mathbb{R} \setminus\{-5\}$.
		\\
		Ta có $\begin{aligned}[t]
			a&=\lim\limits_{x \rightarrow+\infty} \dfrac{f(x)}{x}=\lim\limits_{x \rightarrow+\infty} \dfrac{2x^{2}-3x}{x^2+5x}=2;\\
			b&=\lim\limits_{x \rightarrow+\infty}[f(x)-ax]=\lim\limits_{x \rightarrow+\infty}\left(\dfrac{2x^{2}-3x}{x+5}-2x\right)=\lim\limits_{x \rightarrow+\infty} \dfrac{-13x}{x+5}=-13.
		\end{aligned}$\\
		Ta cũng có $\lim\limits_{x \rightarrow-\infty} \dfrac{f(x)}{x}=2$; $\lim\limits_{x \rightarrow-\infty}[f(x)-x]=-13$.
		\\
		Do đó, đồ thị hàm số có tiệm cận xiên là đường thẳng $y=2x-13$.	Đường thẳng này qua $(6;-1)$.
	}
\end{ex}

\begin{ex}
	Giao điểm của đường tiệm cận đứng và đường tiệm cận xiên của đồ thị hàm số $y=\dfrac{2x^2-3x+2}{x-1}$ là
	\choice
	{$(1;2)$}
	{\True $(1;1)$}
	{$(1;-1)$}
	{$(1;0)$}
	\loigiai{
	Ta viết lại $y=\dfrac{2x^2-3x+2}{x-1}=2x-1+\dfrac{1}{x-1}$. Suy ra
	\begin{itemize}
		\item [$\bullet$] Tiệm cận đứng $x=1$;
		\item [$\bullet$] Tiệm cận ngang $y=2x-1$.
	\end{itemize}
Xét hệ $\heva{&x=1\\&y=2x-1} \Leftrightarrow \heva{&x=1\\&y=1}$}
\end{ex}

\Closesolutionfile{ans}

\ind{PHẦN II.} \inden{Câu trắc nghiệm đúng sai. Trong mỗi ý a), b), c), d) ở mỗi câu, học sinh chọn đúng hoặc sai.}\\
\Opensolutionfile{ans}[ans/2D1-B3-d2-2]

\begin{ex}
	\immini{Cho hàm số $y=f(x)=\dfrac{ax^2+bx+c}{dx+e}$ có đồ thị như hình bên. 
		\choiceTF
		{Tập xác định của hàm số là $\mathbb{R}$}
		{\True Hàm số có hai điểm cực trị}
		{Đồ thị hàm số có đường tiệm cận đứng là $x=0$}
		{Đồ thị hàm số có đường tiệm cận xiên là $y=x+1$}
	}{
		\begin{tikzpicture}[scale=.4, font=\footnotesize, line join=round, line cap=round, >=stealth]
			\draw[->] (-6,0)--(0,0) node[below left]{$O$}--(6,0) node[below]{$x$};
			\draw[->] (0,-8) --(0,6) node[right]{$y$};
			\clip (-6,-8) rectangle (6,6);
			\draw[violet] [domain=-0.8:6, samples=100,thick] %
			plot (\x, {\x-1+ (2)/((\x)+1)});
			\draw[violet] [domain=-6:-1.3, samples=100,thick] %
			plot (\x, {\x-1+ (2)/((\x)+1)});
			\draw[fill] (0,0) circle (1pt) (-1,0) circle (1pt) (-1,-2) circle (1pt) (1,0) circle (1pt)node[above] {$1$} (0,-1) circle (1pt)node[right] {$-1$};
			\draw[domain=-8:7, samples=100] %
			plot (\x, {\x-1});
			\draw (-1,-8)--(-1,0)node[above left] {$-1$}--(-1,6);
	\end{tikzpicture}}
\end{ex}

\begin{ex}
	\immini{Cho đồ thị của hàm số $y=f(x)=\dfrac{2 x^2}{x^2-1}$. Xét tính đúng sai của các khẳng định sau:
	\choiceTF
	{Đồ thị hàm số có 3 điểm cực trị}
	{$\lim \limits_{x \rightarrow-\infty} f(x)=2$ ; $\lim \limits_{x \rightarrow 1^{-}} f(x)=-\infty$}
	{Đồ thị hàm số có 3 đường tiệm cận đứng $x=-1$, $x=0$, $x=1$} 
	{Đồ thị hàm số có hai đường tiệm cận ngang $y=2$ và $y=0$} 
	}{
\begin{tikzpicture}[scale=.5,>=stealth, font=\footnotesize, line join=round, line cap=round]
	\def\xmin{-6} \def\xmax{6}
	\def\ymin{-5} \def\ymax{7}
	%\draw[color=gray!50,dashed] (\xmin,\ymin) grid (\xmax,\ymax);
	\draw[->] (\xmin,0)--(\xmax,0) node [below]{$x$};
	\draw[->] (0,\ymin)--(0,\ymax) node [left]{$y$};
	\fill (0,0) circle (1pt) node[shift={(135:2.5mm)}]{$O$};
	\clip (\xmin+0.1,\ymin+0.1) rectangle (\xmax-0.1,\ymax-0.1);
	\draw[thick,smooth,violet,samples=300,domain=(\xmin:-1.01)] plot(\x,{(2*(\x)^2)/((\x)^2-1)});		
	\draw[thick,smooth,violet,samples=300,domain=(-0.9:0.9)] plot(\x,{(2*(\x)^2)/((\x)^2-1)});
	\draw[thick,smooth,violet,samples=300,domain=(1.1:\xmax)] plot(\x,{(2*(\x)^2)/((\x)^2-1)});
	\draw[blue] (\xmin,2)--(\xmax,2);	
	\draw[blue] (-1,\ymin)--(-1,\ymax);	
	\draw[blue] (1,\ymin)--(1,\ymax);		
	\foreach \x in {\xmin,...,\xmax}
	\draw (\x,-0.1)--(\x,0.1);
	\foreach \y in {\ymin,...,\ymax}
	\draw (-0.1,\y)--(0.1,\y);
	\node at (-5,2)[below]{$y=2$};
	\node at (-1.2,-4)[left]{$x=-1$};
	\node at (1.2,-4)[right]{$x=1$};
	%\node at (-1,0)[shift={(-135:2.5mm)}]{$-1$};
	%\node at (.5,0)[shift={(-75:2.5mm)}]{$\dfrac{1}{2}$};
	%\node at (0,-1)[left]{$-1$};
	%\node at (0,2)[shift={(135:2.5mm)}]{$2$};		
\end{tikzpicture}}
	\loigiai{
	\begin{enumerate}[a)]
		\item Đồ thị hàm số có một điểm cực trị $(0;0)$.
		\item Theo hình vẽ thì $\lim \limits_{x \rightarrow-\infty} f(x)=2$; $\lim \limits_{x \rightarrow 1^{-}} f(x)=-\infty$.
		\item Đồ thị hàm số có 2 đường tiệm cận đứng $x= \pm 1$.
		\item Đồ thị hàm số có 1 đường tiệm cận ngang $y= 2$.
\end{enumerate}}
\end{ex}

\Closesolutionfile{ans}
\begin{dang}{Bài toán về đường tiệm cận có chứa tham số}
\end{dang}
\boxmini{BÀI TẬP TỰ LUẬN}
\begin{vd}%[2D1Y4-2]
	Tìm tham số $m$ để đồ thị hàm số 
	\begin{tasks}
		\task $y=\dfrac{3x-1}{x-m}$ có đường tiệm cận đứng là $x=5$.
		\task $y=\dfrac{(m+1)x-5m}{2x-m}$ có tiệm cận ngang là đường thẳng $y=1$.
	\end{tasks}
	\loigiai{
		\begin{enumerate}[a)]
			\item Điều kiện để đồ thị hàm số có tiệm cận đứng là $-3m+1\neq 0\Leftrightarrow m\neq \dfrac{1}{3}$.\\
			Đồ thị hàm số có tiệm cận đứng $x=m$.\\
			Theo đề bài ta có $m=5$ (thoả mãn).
			\item Điều kiện để đồ thị hàm số có tiệm cận ngang là $-m(m+1)+10m\neq 0$.\\
			Tiệm cận ngang là $y=\dfrac{a}{c}=\dfrac{m+1}{2}.$\\
			Theo đề bài ta có $\dfrac{m+1}{2}=1\Leftrightarrow m+1=2\Leftrightarrow m=1$ (thoả mãn).
		\end{enumerate}
	}
\end{vd}

\begin{vd}%[2D1K4-2]
	Tìm $m$ để đồ thị hàm số 
	\begin{tasks}
		\task $y=\dfrac{x-2}{x^2-mx+1}$ có hai đường tiệm cận đứng.
		\task $y=\dfrac{2x^2-3x+m}{x-m}$ có đường tiệm cận xiên.
	\end{tasks}
	\loigiai{
		\begin{enumerate}[a)]
			\item Đồ thị hàm số có hai tiệm cận đứng $\Leftrightarrow$ phương trình $g(x)=x^2-mx+1=0$ có hai nghiệm phân biệt khác $2$.
			$$\Leftrightarrow\heva{&a=1\neq 0 \, (\textrm{LĐ})\\ & \Delta =m^2-4>0\\&g(2)=2^2-2m+1\neq 0} \Leftrightarrow \heva{&\hoac{&m<-2\\&m>2}\\& m\neq \dfrac{5}{2}}.$$
			Vậy $m\in\left(-\infty; -2\right) \cup \left(2; +\infty\right) \setminus \left\{\dfrac{5}{2}\right\}$.
			\item 	Đồ thị hàm số có đường tiệm cận xiên khi và chỉ khi phương trình $g(x)=2x^2-3x+m=0$ không có nghiệm $x=m$. Tức là:
			$$g(m)\neq 0 \Leftrightarrow 2m^2-2m\neq 0 \Leftrightarrow \heva{&m\neq 0\\ &n\neq 1}.$$
			Vậy $m\in\mathbb{R}\setminus\left\{0; 1\right\}$ là các giá trị cần tìm.
		\end{enumerate}
	}	
\end{vd}


\boxmini{BÀI TẬP TRẮC NGHIỆM}
\ind{PHẦN I.} \inden{Câu trắc nghiệm nhiều phương án lựa chọn. Mỗi câu hỏi học sinh chỉ chọn một phương án.}\\
\setcounter{ex}{0}
\Opensolutionfile{ans}[ans/2D1-B3-d3-1]

\begin{ex}
	Tìm tất cả các giá trị của $m$ để đồ thị hàm số $y=\dfrac{mx+2}{x-5}$ có đường tiệm cận ngang đi qua điểm $A(1; 3)$.
	\choice
	{$m=-3$}
	{$m=1$}
	{$m=-1$}
	{\True $m=3$}
	\loigiai{
		Tiệm cận ngang $y=m$ đi qua điểm $A(1; 3)$ nên $m=3$.
	}
\end{ex} 

\begin{ex}
	Tìm tham số thực $m$ để đồ thị hàm số $y=\dfrac{mx+3}{x-m}$ có tiệm cận đứng là đường $x=1$, tiệm cận ngang là đường $y=1$.
	\choice
	{\True $m=1$}
	{$m=2$}
	{$m=-1$}
	{$m=3$}
	\loigiai{
		\begin{itemize}
			\item Điều kiện để đồ thị hàm số có tiệm cận là $-m^2-3\ne 0 \ \forall m$
			\item Phương trình đường tiệm cận đứng là $x=m$ nên có $m=1$
			\item Phương trình đường tiệm cận ngang là $y=m$ nên có $m=1$\\
			Vậy $m=1$.
		\end{itemize}
	}
\end{ex}

\begin{ex}
	Biết rằng hai đường tiệm cận của đồ thị hàm số $y=\dfrac{2x+1}{x-m}$ (với $m$ là tham số) tạo với hai trục tọa độ một hình chữ nhật có diện tích bằng $2$. Giá trị của $m$ là
	\choice
	{$m=\pm 2$}
	{$m=-1$}
	{$m=2$}
	{\True $m=\pm 1$}
	\loigiai{
		Điều kiện $ m\neq -\dfrac{1}{2} $.\\
		Ta có $\lim\limits_{x\to+\infty}\dfrac{2x+1}{x-m}=2$ và $\lim\limits_{x\to-\infty}\dfrac{2x+1}{x-m}=2\Rightarrow y=2$ là tiệm cận ngang của đồ thị hàm số.\\
		\begin{itemize}
			\item Xét $ m>-\dfrac{1}{2} $, ta có $\lim\limits_{x\to m^{+}}\dfrac{2x+1}{x-m}=+\infty$, $\lim\limits_{x\to m^{-}}\dfrac{2x+1}{x-m}=-\infty\Rightarrow x=m$ là tiệm cận đứng của đồ thị hàm số.
			\item Xét $ m<-\dfrac{1}{2} $, ta có $\lim\limits_{x\to m^{+}}\dfrac{2x+1}{x-m}=-\infty$, $\lim\limits_{x\to m^{-}}\dfrac{2x+1}{x-m}=+\infty\Rightarrow x=m$ là tiệm cận đứng của đồ thị hàm số.
		\end{itemize}
		Diện tích hình chữ nhật là $|2m|=2\Rightarrow m=\pm 1$ (thỏa mãn).
	}
\end{ex} 


\begin{ex}
	Tìm giá trị của $m$ để đồ thị hàm số $y=\dfrac{2x^2-5x+m}{x-m}$ có tiệm cận đứng.
	\choice
	{$\hoac{&m=0\\&m=2}$}
	{$m\ne 0$}
	{$m\ne 2$}
	{\True $\heva{&m\ne 0\\&m\ne 2}$}
	\loigiai{
		Ta có $x-m=0\Leftrightarrow x=m$. \\
		Để đồ thị hàm số có tiệm cận đứng thì $2(m)^2-5(m)+m\ne 0\Leftrightarrow 2m^2-4m\ne 0\Leftrightarrow \heva{&m\ne 0\\&m\ne 2}$.
	}
\end{ex} 

\begin{ex}%[2D1Y4-1]
	Tìm tất cả các giá trị thực của tham số $m$ để đồ thị hàm số $y=\dfrac{x-4}{x^2-mx+4}$ có hai đường tiệm cận đứng?
	\choice
	{$m \in \left (-\infty;-4\right] \cup \left [4;+\infty \right )$}
	{$m \ne 5$}
	{\True $m \in \left (-\infty;-4\right) \cup \left (4;+\infty \right ) \setminus \left \{5\right \}$}
	{$m \in \left (-\infty;-4\right) \cup \left (4;+\infty \right )$}
	\loigiai{
		Đồ thị hàm số có hai tiệm cận đứng khi phương trình $x^2-mx+4=0$ có hai nghiệm phân biệt khác $4\Leftrightarrow \heva{&m^2-16>0\\&16-4m+4\ne 0}\Leftrightarrow m \in \left (-\infty;-4\right) \cup \left (4;+\infty \right ) \setminus \left \{5\right \}$ 
	}
\end{ex}

\begin{ex}%[2D1B4-2]
	Cho hàm số $ y = \dfrac{2x^2-3x+m}{x-m} $ có đồ thị $ (C) $. Tìm tất cả các giá trị của tham số $ m $ để $ (C) $ không có tiệm cận đứng.
	\choice
	{\True $ m = 0 $ hoặc $ m = 1 $}
	{$ m = 2 $}
	{$ m = 1 $}
	{$ m = 0 $}
	\loigiai{
		Đồ thị $ (C) $ không có tiệm cận đứng khi $ m $ là nghiệm của $ 2x^2-3x+m $
		\begin{align*}
			\Leftrightarrow 2m^2 - 3m + m = 0 \Leftrightarrow \hoac{& m = 0 \\& m = 1.}
		\end{align*}
	}
\end{ex}

\begin{ex}
	Tìm tất cả các giá trị của tham số thực $m$ để đồ thị hàm số $y=\dfrac{x-2}{x^2-mx+1}$ có đúng $3$ đường tiệm cận.
	\choice
	{\True $\left[\begin{aligned}
			&\left\{\begin{aligned}
				&m>2 \\
				&m\ne \dfrac{5}{2}
			\end{aligned}\right. \\
			&m<-2
		\end{aligned}\right. $}
	{$\left[\begin{aligned}
			&m>2 \\
			&\left\{\begin{aligned}
				&m<-2 \\
				&m\ne -\dfrac{5}{2}
			\end{aligned}\right.
		\end{aligned}\right. $}
	{$\left[\begin{aligned}
			&m>2 \\
			&m<-2
		\end{aligned}\right. $}
	{$-2<m<2$}
	\loigiai{
		ĐKXĐ : $x^2-mx+1\ne 0$ \\
		Ta có $\displaystyle\lim \limits_{x\to \pm \infty}y=\displaystyle\lim \limits_{x\to \pm \infty}\dfrac{x-2}{x^2-mx+1}=0$ $ \Rightarrow y=0$ là tiệm cận ngang. \\
		Do đó đồ thị hàm số $y=\dfrac{x-2}{x^2-mx+1}$ có đúng $3$ đường tiệm cận khi và chỉ khi phương trình $x^2-mx+1=0$ có hai nghiệm phân biệt khác $2$. \\
		$ \Leftrightarrow \left\{\begin{aligned}
			& \Delta =m^2-4>0 \\
			&2^2-2m+1\ne 0
		\end{aligned}\right. \Leftrightarrow \left\{\begin{aligned}
			&\left[\begin{aligned}
				&m>2 \\
				&m<-2
			\end{aligned}\right. \\
			&m\ne \dfrac{5}{2}
		\end{aligned}\right. $. }
\end{ex} 

\begin{ex}
	Cho hàm số $y=\dfrac{ax+1}{bx-2}$, xác định $a$ và $b$ để đồ thị của hàm số trên nhận đường thẳng $x=1$ làm tiệm cận đứng và đường thẳng $y=\dfrac{1}{2}$ làm tiệm cận ngang.
	\choice
	{$ \heva{&a=-1\\&b=-2} $}
	{\True $ \heva{&a=1\\&b=2} $}
	{$ \heva{&a=2\\&b=2} $}
	{$ \heva{&a=2\\&b=-2} $}
	\loigiai{Yêu cầu bài toán $\Leftrightarrow\heva{&\dfrac{a}{b}=\dfrac{1}{2}\\&\dfrac{2}{b}=1}\Leftrightarrow\heva{&b=2\\&a=1}$.}
\end{ex} 


\begin{ex}%[2D1Y4-1]
	Cho hàm số $y=\dfrac{mx+1}{x+3n+1}$. Đồ thị hàm số nhận trục hoành và trục tung làm tiệm cận ngang và tiệm cận đứng. Tính $m+n$.
	\choice
	{\True $m+n=-\dfrac{1}{3}$}
	{$m+n=\dfrac{1}{3}$}
	{$m+n=\dfrac{2}{3}$}
	{$m+n=0$}
	\loigiai{
		\begin{itemize}
			\item Điều kiện để đồ thị hàm số có tiệm cận là $m\left (3n+1\right )\ne 0$
			\item Phương trình đường tiệm cận đứng là $x=-3n-1$ nên có $n=-\dfrac{1}{3}$
			\item Phương trình đường tiệm cận ngang là $y=m$ nên có $m=0$\\
			Vậy $m+n=-\dfrac{1}{3}$.
		\end{itemize}
	}
\end{ex}

\begin{ex}%[2D1K4-2]
	Đồ thị hàm số $y=\dfrac{(4a-b)x^2+ax+1}{x^2+ax+b-12}$ nhận trục hoành và trục tung làm hai tiệm cận. Tính giá trị của $a+b$.
	\choice
	{$a+b=10$}
	{$a+b=12$}
	{$a+b=18$}
	{\True $a+b=15$}
	\loigiai{
		Tiệm cận đứng $x=0 \Rightarrow 0^2+a.0+b-12=0\Leftrightarrow b=12.$\\
		Tiệm cận ngang $y=0 \Rightarrow 4a-b=0\Leftrightarrow 4a-12=0 \Leftrightarrow a=3.$\\
		\textbf{Kết luận:} $a+b=15.$
	}
\end{ex}

\Closesolutionfile{ans}

\ind{PHẦN II.} \inden{Câu trắc nghiệm đúng sai. Trong mỗi ý a), b), c), d) ở mỗi câu, học sinh chọn đúng hoặc sai.}\\
\Opensolutionfile{ans}[ans/2D1-B3-d3-2]
\begin{ex}%[2D1B4-2]
	Cho hàm số $y=\dfrac{mx^2+6x-2}{x+2}$, với $m$ là tham số.
	\choiceTF
	{\True Tập xác định của hàm số là $\mathbb{R}\backslash\{-2\}$}
	{Đồ thị hàm số có tiệm cận ngang khi $m>0$}
	{Đồ thị hàm số có tiệm cận đứng khi $m\ne 0$}
	{\True Tập hợp tất cả giá trị của $m$ đề đồ thị có hai đường tiệm cận là $\mathbb{R}\setminus\left\{\dfrac{7}{2}\right\}$}
	\loigiai
	{
		\begin{enumerate}[a)]
			\item Điều kiện $x+2 \ne 0 \Leftrightarrow x \ne -2$. Vậy Tập xác định là $\mathbb{R}\backslash\{-2\}$
			\item Đồ thị hàm số có tiệm cận ngang khi hệ số của $x^2$ trên tử số phải bằng 0. Suy ra $m=0$.
			\item Đồ thị hàm số có tiệm cận đứng khi $x=-2$ không là nghiệm của tam thức $g(x)=mx^2+6x-2$. Suy ra
			$$g(-2)\ne 0 \Leftrightarrow m \ne \dfrac{7}{2}$$
			\item Đồ thị hàm số chắc chắn có 1 tiệm cận xiên (hoặc ngang). Suy ra, để đồ thị có hai đường tiệm cận thì nó phải có 1 tiệm cận đứng. Điều này tương đương với $m \ne \dfrac{7}{2}$.
		\end{enumerate}
	}
\end{ex}

\Closesolutionfile{ans}

% \section[TIỆM CẬN]{ĐƯỜNG TIỆM CẬN CỦA ĐỒ THỊ HÀM SỐ}
\subsection{TÓM TẮT LÝ THUYẾT}
\subsubsection{Đường tiệm cận ngang}%[Lý Văn Hoàng, Dự án TeX hóa Lý Thuyết]
\begin{dn}
    Đường thẳng $y=m$ là đường tiệm cận ngang (hay tiệm cận ngang)
    của đồ thị hàm số $y=f(x)$ nếu ít nhất một trong các điều kiện sau được thỏa mãn:\\
    \centerline{$\lim\limits_{x\to+\infty}f(x)=m, \quad\lim\limits_{x\to-\infty}f(x)=m $.}
\end{dn}
\begin{center}
    \begin{tikzpicture}[scale=1,>=stealth, font=\footnotesize, line join=round, line cap=round]
        \def\xmin{-4} \def\xmax{2}
        \def\ymin{-0.5} \def\ymax{3}
        %\draw[color=gray!50,dashed] (\xmin,\ymin) grid (\xmax,\ymax);
        \draw[->] (\xmin,0)--(\xmax,0) node [below]{$x$};
        \draw[->] (0,\ymin)--(0,\ymax) node [left]{$y$};
        \fill (0,0) circle (1pt) node[shift={(-135:2.5mm)}]{$O$};
        \node at (current bounding box.south) [below=-2pt] {a) $\lim\limits_{x \rightarrow-\infty} f(x)=m$};
        \clip (\xmin+0.1,\ymin+0.1) rectangle (\xmax-0.1,\ymax-0.1);
        \draw[red,thick,smooth,samples=300,domain=\xmin:\xmax]
        (-4,0.9)..controls +(0:2) and +(180:0.5)
        ..(-1.5,0.5)..controls +(0:0.5) and +(180:0.5)
        ..(-0.3,1.4)..controls +(0:0.5) and +(135:1)
        ..(1.8,0.3);
        \draw [blue](\xmin,1)--(\xmax,1);
        \path[blue] (-3,1)node[above]{$y=m$};
        \path[red] (0,1.3)node[above left]{$y=f(x)$};
        \fill (0,1) circle (1pt) node[shift={(-135:3mm)}]{$m$};
    \end{tikzpicture}\hspace*{1cm}
    \begin{tikzpicture}[scale=1,>=stealth, font=\footnotesize, line join=round, line cap=round]
        \def\xmin{-1.5} \def\xmax{4}
        \def\ymin{-0.5} \def\ymax{3}
        %\draw[color=gray!50,dashed] (\xmin,\ymin) grid (\xmax,\ymax);
        \draw[->] (\xmin,0)--(\xmax,0) node [below]{$x$};
        \draw[->] (0,\ymin)--(0,\ymax) node [left]{$y$};
        \fill (0,0) circle (1pt) node[shift={(-135:2.5mm)}]{$O$};
        \node at (current bounding box.south) [below=-2pt] {b) $\lim\limits_{x \rightarrow+\infty} f(x)=m$};
        \clip (\xmin+0.1,\ymin+0.1) rectangle (\xmax-0.1,\ymax-0.1);
        \draw[red,thick,smooth,samples=300,domain=\xmin:\xmax]
        (-1,3)..controls +(-80:1) and +(170:1)
        ..(0.5,1.1)..controls +(170:-1) and +(180:-0.5)
        ..(3.9,0.8);
        \draw [blue](\xmin,0.7)--(\xmax,0.7);
        \path[blue] (4,0.7)node[below left]{$y=m$};
        \path[red] (0.5,1)node[above right]{$y=f(x)$};
        \fill (0,0.7) circle (1pt) node[shift={(-135:3mm)}]{$m$};
    \end{tikzpicture}
\end{center}
\begin{nx} \quad
    \begin{itemize}
        \item Để tìm tiệm cận ngang của đồ thị hàm số ta cần tính giới hạn của hàm số tại vô cực $(\pm \infty)$.
        \item Tìm giới hạn ở vô cực của hàm $y=\dfrac{P(x)}{Q(x)}$ với $P(x)$, $Q(x)$ là các đa thức không căn.
        \begin{enumerate}[i)]
            \item Bậc của $P(x)$ nhỏ hơn bậc của $Q(x) \Rightarrow \lim\limits_{x\to \pm\infty} y =0 \Rightarrow$ Tiệm cận ngang $Ox \colon y=0$.
            \item Bậc của $P(x)$ bằng bậc của $Q(x) \Rightarrow \lim\limits_{x\to \pm\infty} y = \dfrac{\text{Hệ số x bậc cao của P(x) }}{\text{Hệ số x bậc cao của Q(x)}} = \alpha$ (một số cụ thể) $\Rightarrow y= \alpha$ là tiệm cận ngang.
            \item Bậc của $P(x)$ lớn hơn bậc của $Q(x) \Rightarrow \lim\limits_{x\to \pm\infty} y = \pm \infty \Rightarrow$ Không có tiệm cận ngang.
        \end{enumerate}
    \end{itemize}
\end{nx}
\subsubsection{Đường tiệm cận đứng}%[Lý Văn Hoàng, Dự án TeX hóa Lý Thuyết]
\begin{dn}
    Đường thẳng $x=a$ được gọi là đường tiệm cận đứng (hay tiệm cận đứng) của đồ thị hàm số $y=f(x)$ nếu ít nhất một trong các điều kiện sau được thỏa mãn:
    $$ \lim\limits_{x \to a^{+} } f(x)= + \infty; \lim\limits_{x \to a^{+} } f(x)= - \infty ;$$ $$ \lim\limits_{x \to a^{-} } f(x)= + \infty; \lim\limits_{x \to a^{-}} f(x)= - \infty.$$
\end{dn}
\begin{center}
    \begin{tikzpicture}[scale=.7,>=stealth, font=\footnotesize, line join=round, line cap=round]
        %Hình a
        \def\xmin{-2.2} \def\xmax{3.5}
        \def\ymin{-2} \def\ymax{2}
        %\draw[color=gray!50,dashed] (\xmin,\ymin) grid (\xmax,\ymax);
        \draw[->] (\xmin,0)--(\xmax,0) node [below]{$x$};
        \draw[->] (0,\ymin)--(0,\ymax) node [left]{$y$};
        \fill (0,0) circle (1pt) node[shift={(-45:2.5mm)}]{$O$};
        \draw (2.1,\ymin)--(2.1,\ymax)node[below right]{$x=a$};
        \fill (2.1,0) circle (1pt) node[shift={(-45:3mm)}]{$a$};
        %\clip (\xmin+0.1,\ymin+0.1) rectangle (\xmax-0.1,\ymax-0.1);
        \draw[red] (-2,-1)..controls +(80:0.5) and +(0:-.5)..(-1,0.5)node[above]{$y=f(x)$}
        ..controls +(0:0.5) and +(180:0.5)..(0.5,-1.5)
        ..controls +(0:0.5) and +(87:-0.2)..(1.6,0)
        ..controls +(87:-.2) and +(90:-0.2)
        ..(2,1.85);
        \node at (current bounding box.south) [below=-2pt] {a) $\lim\limits_{x \rightarrow a^{-}} f(x)=+\infty$};
    \end{tikzpicture}
    \begin{tikzpicture}[scale=.7,>=stealth, font=\footnotesize, line join=round, line cap=round]
        %Hình b
        \def\xmin{-1.2} \def\xmax{4}
        \def\ymin{-2} \def\ymax{2}
        %\draw[color=gray!50,dashed] (\xmin,\ymin) grid (\xmax,\ymax);
        \draw[->] (\xmin,0)--(\xmax,0) node [below]{$x$};
        \draw[->] (0,\ymin)--(0,\ymax) node [left]{$y$};
        \fill (0,0) circle (1pt) node[shift={(-45:2.5mm)}]{$O$};
        \draw (1,\ymin)node[above right]{$x=a$}--(1,\ymax);
        \fill (1,0) circle (1pt) node[shift={(-135:3mm)}]{$a$};
        \path[red] (1.25,1)node[above right]{$y=f(x)$};
        %\clip (\xmin+0.1,\ymin+0.1) rectangle (\xmax-0.1,\ymax-0.1);
        \draw[red] (1.2,2)..controls +(80:0) and +(0:-1.4)..(2.5,-0.8)
        ..controls +(0:0.1) and +(-80:-0.6)
        ..(3.5,-1.5);
        \node at (current bounding box.south) [below=-2pt] {b) $\lim\limits_{x \rightarrow a^{+}} f(x)=+\infty$};
    \end{tikzpicture}
    \begin{tikzpicture}[scale=.7,>=stealth, font=\footnotesize, line join=round, line cap=round]
        %Hình c
        \def\xmin{-2.2} \def\xmax{3.5}
        \def\ymin{-2} \def\ymax{2}
        %\draw[color=gray!50,dashed] (\xmin,\ymin) grid (\xmax,\ymax);
        \draw[->] (\xmin,0)--(\xmax,0) node [below]{$x$};
        \draw[->] (0,\ymin)--(0,\ymax) node [left]{$y$};
        \fill (0,0) circle (1pt) node[shift={(-45:2.5mm)}]{$O$};
        \draw (2,\ymin)--(2,\ymax)node[below right]{$x=a$};
        \fill (2,0) circle (1pt) node[shift={(-45:3mm)}]{$a$};
        \path[red] (-2.25,1.2)node[below right]{$y=f(x)$};
        %\clip (\xmin+0.1,\ymin+0.1) rectangle (\xmax-0.1,\ymax-0.1);
        \draw[red] (-2,1.4)..controls +(-10:-0.2) and +(-55:-.7)
        ..(1.3,0.65)..controls +(-50:0.4) and +(-90:0)
        ..(1.8,-2)
        ;
        \node at (current bounding box.south) [below=-2pt] {c) $\lim\limits_{x \rightarrow a^{-}} f(x)=-\infty$};
    \end{tikzpicture}
    \begin{tikzpicture}[scale=.7,>=stealth, font=\footnotesize, line join=round, line cap=round]
        %Hình d
        \def\xmin{-2.2} \def\xmax{3.5}
        \def\ymin{-2} \def\ymax{2}
        %\draw[color=gray!50,dashed] (\xmin,\ymin) grid (\xmax,\ymax);
        \draw[->] (\xmin,0)--(\xmax,0) node [below]{$x$};
        \draw[->] (0,\ymin)--(0,\ymax) node [left]{$y$};
        \fill (0,0) circle (1pt) node[shift={(-135:2.5mm)}]{$O$};
        \draw (.6,\ymin)--(.6,\ymax)node[below right]{$x=a$};
        \fill (.6,0) circle (1pt) node[shift={(-135:3mm)}]{$a$};
        %\clip (\xmin+0.1,\ymin+0.1) rectangle (\xmax-0.1,\ymax-0.1);
        \draw[red] (0.7,-2)..controls +(85:0.2) and +(180:0.2)
        ..(1.2,-0.3)..controls +(0:0.2) and +(180:0.2)
        ..(1.7,-0.6)..controls +(0:0.4) and +(90:0)
        ..(2.5,2)
        ;
        \node at (current bounding box.south) [below=-2pt] {d) $\lim\limits_{x \rightarrow a^{+}} f(x)=-\infty$};
    \end{tikzpicture}
\end{center}
\immini{\textbf{Đặc biệt} Đối với hàm số $y= \dfrac{ax+b}{cx+d}$ có tiệm cận ngang $y=\dfrac{a}{c}$ và tiệm cận đứng $x= -\dfrac{d}{c}$. Tâm đối xứng là giao điểm của hai đường tiệm cận.
}{
    \begin{tikzpicture}[>=stealth, line join=round, line cap=round, font=\scriptsize,x=.8cm,y=.7cm]
        \begin{scope}[scale=.7]
            \def\a{1}
            \def\b{1}
            \def\c{1}
            \def\d{-2}
            \def\mau{red}
            \draw[->] (-5,0) -- (8,0) node[below] {$x$};
            \draw[->] (0,-5) -- (0,5) node[left] {$y$};
            \draw (0,0)node[below left]{$O$};
            \draw[dashed,blue] ({-\d/\c},-5)--({-\d/\c},5) (-5,{\a/\c})--(8,{\a/\c}); % Vẽ TCĐ và TCN
            \clip (-5,-5)rectangle(8,5);
            \draw ({-\d/\c},0)node[below right]{$2$};
            \draw (0,{\a/\c})node[above left]{$1$};
            \draw (7,-4)node[above left]{{\normalsize }$y=\dfrac{x+1}{x-2}$};
            \pgfmathsetmacro{\can}{-(\d)/(\c)}
            \draw[\mau,samples=150,smooth,domain=-5:{\can-.1}] plot(\x,{(\a*\x+(\b))/(\c*\x+(\d))}); % Vẽ nhánh bên trái TCĐ
            \draw[\mau,samples=150,smooth,domain={\can+.1}:8] plot(\x,{(\a*\x+(\b))/(\c*\x+(\d))}); % Vẽ nhánh bên phải TCĐ
        \end{scope}
\end{tikzpicture}}
\begin{nx} \quad
    \begin{itemize}
        \item Để tìm tiệm cận đứng của đồ thị hàm số, ta cần tính giới hạn một bên của $x_0$, với $x_0$ thường là điều kiện của hàm số (hay tại $x_0$ thì hàm số không xác định).
        \item Kỹ năng sử dụng máy tính (tham khảo):
        \begin{enumerate}[i)]
            \item Tính $\lim\limits_{x \to x_0^+} f(x)$ thì nhập $f(x)$ và CALC $x= x_0 + 10^{-9}$.
            \item Tính $\lim\limits_{x \to x_0^-} f(x)$ thì nhập $f(x)$ và CALC $x= x_0 - 10^{-9}$.
        \end{enumerate}
    \end{itemize}
\end{nx}
\subsubsection{Đường tiệm cận xiên}
\begin{dn}
    Đường thẳng $y=ax+b$ được gọi là đường tiệm cận xiên của đồ thị $(C):y=f(x)$ nếu \[\lim \limits_{x \to -\infty} \left[f(x)-(ax+b)\right]=0 \text{ hoặc }\lim \limits_{x \to +\infty} \left[f(x)-(ax+b)\right]=0\]
\end{dn}
\begin{center}
    \begin{tikzpicture}[scale=0.8,>=stealth, font=\footnotesize, line join=round, line cap=round]
        \def\xmin{-4} \def\xmax{2.5}
        \def\ymin{-0.5} \def\ymax{3}
        %\draw[color=gray!50,dashed] (\xmin,\ymin) grid (\xmax,\ymax);
        \draw[->] (\xmin,0)--(\xmax,0) node [below]{$x$};
        \draw[->] (0,\ymin)--(0,\ymax) node [left]{$y$};
        \fill (0,0) circle (1pt) node[shift={(-135:2.5mm)}]{$O$};
        \node at (current bounding box.south) [below=-2pt] {a) $\lim\limits_{x \rightarrow-\infty}\left[f(x)-(ax+b)\right]=0$};
        \clip (\xmin+0.1,\ymin+0.1) rectangle (\xmax-0.1,\ymax-0.1);
        \draw[red,thick,smooth,samples=300,domain=\xmin:\xmax]
        (-3.8,-0.6)..controls +(34:0.5) and +(180:.75)
        ..(-0.2,1.2)..controls +(0:0.75) and +(180:.75)
        ..(1,0.3)..controls +(0:0.5) and +(80:0)
        ..(2.2,1);
        \draw[blue,smooth,samples=300,domain=\xmin:\xmax] plot(\x,{2/3*(\x)+2});
        \path[blue] (-3,0)--(0,2)node[below,sloped,pos=1.3]{$y=ax+b$};
        \path[red] (0.5,1)node[above right]{$y=f(x)$};
    \end{tikzpicture}\hspace{1cm}
    \begin{tikzpicture}[scale=0.8,>=stealth, font=\footnotesize, line join=round, line cap=round]
        \def\xmin{-3.5} \def\xmax{3}
        \def\ymin{-0.5} \def\ymax{3}
        %\draw[color=gray!50,dashed] (\xmin,\ymin) grid (\xmax,\ymax);
        \draw[->] (\xmin,0)--(\xmax,0) node [below]{$x$};
        \draw[->] (0,\ymin)--(0,\ymax) node [left]{$y$};
        \fill (0,0) circle (1pt) node[shift={(-135:2.5mm)}]{$O$};
        \node at (current bounding box.south) [below=-2pt] {a) $\lim\limits_{x \rightarrow+\infty}\left[f(x)-(ax+b)\right]=0$};
        \clip (\xmin+0.1,\ymin+0.1) rectangle (\xmax-0.1,\ymax-0.1);
        \draw[red,thick,smooth,samples=300,domain=\xmin:\xmax]
        (-3,0.8)..controls +(60:0.5) and +(180:.75)
        ..(-1.5,2)..controls +(0:.5) and +(180:.75)
        ..(0.5,1.3)..controls +(0:.75) and +(-160:.5)
        ..(2.8,1.8);
        \draw[blue,smooth,samples=300,domain=\xmin:\xmax] plot(\x,{1/3*(\x)+0.75});
        \path[blue] (-3,-0.25)--(0,0.75)node[below,sloped,pos=1.6]{$y=ax+b$};
        \path[red] (-2.5,2)node[above right]{$y=f(x)$};
    \end{tikzpicture}
\end{center}
\begin{nx}\quad
    \begin{itemize}
        \item Để tìm TCX của đồ thị hàm số $y=f(x)$ ta giải hệ phương trình: $\heva{& \lim \limits_{x \to +\infty} \dfrac{f(x)}{x}=a \ne 0 \\ & \lim \limits_{x \to +\infty} \left[f(x)-ax\right]=b}$ hoặc $\heva{& \lim \limits_{x \to -\infty} \dfrac{f(x)}{x}=a \ne 0 \\ & \lim \limits_{x \to -\infty} \left[f(x)-ax\right]=b}$, khi đó tiệm cận xiên của đồ thị hàm số $y=f(x)$ là đường thẳng $y=ax+b$.
        \item Đồ thị hàm số $y=\dfrac{mx^2+nx+p}{cx+d}=ax+b+\dfrac{r}{cx+d}$ có đường tiệm cận xiên là đường thẳng $y=ax+b$.
        \item Hàm phân thức có bậc tử bé hơn hoặc bằng bậc mẫu, bậc tử lớn hơn bậc mẫu 2 bậc thì không có tiệm cận xiên.
    \end{itemize}
\end{nx}
%\subsection{CÁC DẠNG TOÁN}
\begin{dang}{Tìm các đường tiệm cận qua biểu thức hàm số, bảng biến thiên}
\end{dang}
\begin{vd} Tìm các đường tiệm cận đứng, ngang, xiên (nếu có) của đồ thị hàm số sau
    \begin{listEX}[3]
        \item $y=\dfrac{2x+1}{x+1}$.
        \item $y=\dfrac{x}{2x-1}$.
        \item $y=\dfrac{3-x}{x+1}$.
        \item $y=2x+1+\dfrac{1}{x-3}$
        \item $y=\dfrac{4x^2-3x+10}{x-1}$.
        \item $y=\dfrac{x^2-4x+3}{x^2-1}$.
        \item $y=\dfrac{2x+4}{x^2+x-2}$.
        \item $y=\dfrac{\sqrt{9-x^2}}{x-1}$.
        \item $y=x+\sqrt{x^2-1}$
        %	\item $y=\dfrac{x}{\sqrt{x^2+1}}$.
        %	\item $y=\dfrac{\sqrt{x+25}-5}{x^2+x}$.
    \end{listEX}
    \loigiai{}
\end{vd}
\begin{vd}
    Tìm các đường tiệm cận của đồ thị hàm số $y=f(x)$, biết
    \begin{listEX}[2]
        \item \begin{tikzpicture}[scale=.7,>=stealth, font=\footnotesize, line join=round, line cap=round]
            \def\xmin{-2} \def\xmax{4}
            \def\ymin{-3} \def\ymax{3}
            %\draw[color=gray!50,dashed] (\xmin,\ymin) grid (\xmax,\ymax);
            \draw[->] (\xmin,0)--(\xmax,0) node [below]{$x$};
            \draw[->] (0,\ymin)--(0,\ymax) node [left]{$y$};
            \fill (0,0) circle (1pt) node[shift={(135:2.5mm)}]{$O$};
            %\node at (current bounding box.south) [below=-2pt] {a) $y=\dfrac{2x-3}{5x^{2}-15x+10}$};
            \clip (\xmin+0.1,\ymin+0.1) rectangle (\xmax-0.1,\ymax-0.1);
            \draw[thick,smooth,samples=300,domain=\xmin:0.99] plot(\x,{(2*(\x)-3)/(5*(\x)^2-15*(\x)+10)});
            \draw[thick,smooth,samples=300,domain=1.01:1.99] plot(\x,{(2*(\x)-3)/(5*(\x)^2-15*(\x)+10)});
            \draw[thick,smooth,samples=300,domain=2.01:\xmax] plot(\x,{(2*(\x)-3)/(5*(\x)^2-15*(\x)+10)});
            \draw[dashed] (1,\ymin)--(1,\ymax);
            \draw[dashed] (2,\ymin)--(2,\ymax);
            \foreach \s/\t in {2/-45,1/-45}
            \fill (\s,0) circle (1pt) node[shift={(\t:3mm)}]{$\s$};
        \end{tikzpicture}
        \item \begin{tikzpicture}[scale=.5,>=stealth, font=\footnotesize, line join=round, line cap=round]
            \def\xmin{-4} \def\xmax{4}
            \def\ymin{-3} \def\ymax{5}
            %\draw[color=gray!50,dashed] (\xmin,\ymin) grid (\xmax,\ymax);
            \draw[->] (\xmin,0)--(\xmax,0) node [below]{$x$};
            \draw[->] (0,\ymin)--(0,\ymax) node [right]{$y$};
            \fill (0,0) circle (1pt) node[shift={(-135:2.5mm)}]{$O$};
            %\node at (current bounding box.south) [below=-2pt] {c) $y=\dfrac{16x^{2}-8x}{16x^{2}+1}$};
            \clip (\xmin+0.1,\ymin+0.1) rectangle (\xmax-0.1,\ymax-0.1);
            \draw[thick,smooth,samples=300,domain=\xmin:\xmax] plot(\x,{(16*(\x)^2-8*(\x))/(16*(\x)^2+1)});
            \draw[dashed](\xmin,1)--(\xmax,1);
            \foreach \p/\r in {1/45}
            \fill (0,\p) circle (1pt) node[shift={(\r:3mm)}]{$\p$};
        \end{tikzpicture}
        \item 	\begin{tikzpicture}[scale=.7,>=stealth, font=\footnotesize, y=.7cm]
            \def\xmin{-.5} \def\xmax{6}
            \def\ymin{-.5} \def\ymax{5}
            \draw[->] (\xmin,0)--(\xmax,0) node [below]{$x$};
            \draw[->] (0,\ymin)--(0,\ymax) node [left]{$y$};
            \fill (0,0) circle (1pt) node[shift={(-135:2.5mm)}]{$O$};
            \node at (1,-.5)[right]{$x=1$};
            \clip (\xmin+0.1,\ymin+0.1) rectangle (\xmax-0.1,\ymax-0.1);
            \draw[smooth,thick,samples=300,domain=(1.01:\xmax)] plot(\x,{2/sqrt(\x-1)});
            \draw[blue,dashed] (1,\ymin)--(1,\ymax);
            \draw[blue,dashed] (-1,.8)--(6,.8)node[below left]{$y=0.5$};
            \foreach \x in {\xmin,...,\xmax}
            \draw (\x,-0.1)--(\x,0.1);
            \foreach \y in {\ymin,...,\ymax}
            \draw (-0.1,\y)--(0.1,\y);
        \end{tikzpicture}
        \item \begin{tikzpicture}[scale=0.5, font=\footnotesize, line join=round, line cap=round, >=stealth]
            \clip(-3,-2) rectangle (5.1,4.1);
            \draw[->] (-3,0) -- (5,0);\draw (4.9,0) node[below] { $x$};
            \draw[->] (0,-2) -- (0,4);\draw (0,3.9) node[right] { $y$};
            \draw[fill=black] (0,0) node[below right]{$O$} circle (1pt);
            \draw (1,0) node[below right]{$2$};
            \draw (0,1) node[above left]{$1$};
            \draw[thick] plot[domain=-3:0.5,samples=100] (\x, {(1 + \x)/(\x - 1)});
            \draw[thick] plot[domain= 1.5:5,samples=100] (\x, {(1 + \x)/(\x - 1)});
            \draw [-,dashed] (-3,1)--(5,1); %TCN
            \draw [-,dashed] (1,-2)--(1,4); %TCĐ
            \draw[fill=black] (0,0) circle(1pt);
        \end{tikzpicture}
        \item \begin{tikzpicture}[scale=.7,>=stealth, font=\footnotesize, line join=round, line cap=round]
            \def\xmin{-4} \def\xmax{4}
            \def\ymin{-3} \def\ymax{5}
            %\draw[color=gray!50,dashed] (\xmin,\ymin) grid (\xmax,\ymax);
            \draw[->] (\xmin,0)--(\xmax,0) node [below]{$x$};
            \draw[->] (0,\ymin)--(0,\ymax) node [right]{$y$};
            \fill (0,0) circle (1pt) node[shift={(-135:2.5mm)}]{$O$};
            %\node at (current bounding box.south) [below=-2pt] {b) $y=\dfrac{x^{2}+x-1}{x}$};
            \clip (\xmin+0.1,\ymin+0.1) rectangle (\xmax-0.1,\ymax-0.1);
            \draw[thick,smooth,samples=300,domain=\xmin:-0.01] plot(\x,{((\x)^2+(\x)-1)/(\x)});
            \draw[thick,smooth,samples=300,domain=0.01:\xmax] plot(\x,{((\x)^2+(\x)-1)/(\x)});
            \draw[dashed,smooth,samples=300,domain=\xmin:\xmax] plot(\x,{(\x)+1});
            \foreach \s/\t in {-1/-90}
            \fill (\s,0) circle (1pt) node[shift={(\t:3mm)}]{$\s$};
            \foreach \p/\r in {1/-20}
            \fill (0,\p) circle (1pt) node[shift={(\r:3mm)}]{$\p$};
        \end{tikzpicture}
        \item \begin{tikzpicture}[scale=.7,>=stealth, font=\footnotesize,x=.7cm,y=.7cm]
            \def\xmin{-6} \def\xmax{6}
            \def\ymin{-5} \def\ymax{7}
            %\draw[color=gray!50,dashed] (\xmin,\ymin) grid (\xmax,\ymax);
            \draw[->] (\xmin,0)--(\xmax,0) node [below]{$x$};
            \draw[->] (0,\ymin)--(0,\ymax) node [left]{$y$};
            \fill (0,0) circle (1pt) node[shift={(135:2.5mm)}]{$O$};
            \clip (\xmin+0.1,\ymin+0.1) rectangle (\xmax-0.1,\ymax-0.1);
            \draw[smooth,thick,samples=300,domain=(\xmin:-1.01)] plot(\x,{(2*(\x)^2)/((\x)^2-1)});
            \draw[smooth,thick,samples=300,domain=(-0.9:0.9)] plot(\x,{(2*(\x)^2)/((\x)^2-1)});
            \draw[smooth,thick,samples=300,domain=(1.1:\xmax)] plot(\x,{(2*(\x)^2)/((\x)^2-1)});
            \draw[dashed] (\xmin,2)--(\xmax,2);
            \draw[dashed] (-1,\ymin)--(-1,\ymax);
            \draw[dashed] (1,\ymin)--(1,\ymax);
            \foreach \x in {\xmin,...,\xmax}
            \draw (\x,-0.1)--(\x,0.1);
            \foreach \y in {\ymin,...,\ymax}
            \draw (-0.1,\y)--(0.1,\y);
            \node at (-5,2)[below]{$y=2$};
            \node at (-1.2,-4)[left]{$x=-1$};
            \node at (1.2,-4)[right]{$x=1$};
        \end{tikzpicture}
        \item
        \begin{tikzpicture}[>=stealth]
            \tkzTabInit[nocadre=false,lgt=1,espcl=2,deltacl=0.5]{$x$/.7 ,$y'$/.7,$y$/2}
            {$-\infty$ , $1$ , $+\infty$}
            \tkzTabLine{ , - , d , - , }
            \tkzTabVar{+/$2$ ,-D+/$-\infty$/$+\infty$ , -/$2$}
        \end{tikzpicture}
        \item
        \begin{tikzpicture}[>=stealth]
            \tkzTabInit[nocadre=false,lgt=1,espcl=1.5,deltacl=0.5]{$x$/.7 ,$y'$/.7,$y$/2}
            {$-\infty$ , $0$,$1$ , $+\infty$}
            \tkzTabLine{ , + , 0,-, d , + , }
            \tkzTabVar{-/$0$, +/$2$ ,-D-/$-\infty$/$3$ , +/$5$}
        \end{tikzpicture}
        % \item
        % \begin{tikzpicture}[>=stealth]
        %     \tkzTabInit[nocadre=false,lgt=1,espcl=1.8,deltacl=0.5]{$x$/.7 ,$y'$/.7,$y$/2}
        %     {$-\infty$ , $-1$,$1$ , $+\infty$}
        %     \tkzTabLine{ , - , d,-, 0 , + , }
        %     \tkzTabVar{+/$2$ ,-D+/$-5$/$3$, -/$-1$ , +/$+\infty$}
        % \end{tikzpicture}
        % \item
        % \begin{tikzpicture}[>=stealth]
        %     \tkzTabInit[nocadre=false,lgt=1,espcl=1.4,deltacl=0.5]{$x$/.7 ,$y'$/.7,$y$/2}
        %     {$-\infty$ , $-2$, $0$,$1$ , $+\infty$}
        %     \tkzTabLine{ , - , d,-, 0 , + ,d,-, }
        %     \tkzTabVar{+/$-1$ ,-D+/$-\infty$/$2$, -/$-4$, +/$3$ , -/$0$}
        % \end{tikzpicture}
    \end{listEX}
    \loigiai{}
\end{vd}
\begin{vd}
    Một bể bơi chứa $5\,000$ lít nước tinh khiết. Người ta bơm vào bể đó nước muối có nồng đồ $30$ gam muối cho mỗi lít nước với tốc độ $25$ lít/phút.
    \begin{listEX}
        \item Lập hàm số biểu diễn nồng độ muối trong bể sau $t$ phút.
        \item Tìm tiệm cận ngang của hàm số vừa tìm được.
        \item Nêu nhận xét về nồng độ muối trong bể khi thời gian $t$ ngày càng lớn.
    \end{listEX}
    \loigiai{
        \begin{enumerate}[a)]
            \immini{\item Sau $t$ phút, ta có: khối lượng muối trong bể là $25\cdot 30\cdot t=750t$ (gam); thể tích của lượng nước trong bể là $5\,000+25t$ (lít). Vậy nồng độ muối sau $t$ phút là
                $$f(t)=\dfrac{750t}{5\,000+25t}=\dfrac{30t}{200+t}\,\text{(gam/lít)}.$$
                \item Ta có\\
                $\lim\limits_{t\to +\infty}f(t)=\lim\limits_{t \to +\infty}\dfrac{30t}{200+t}=\lim\limits_{t\to +\infty}\left(30-\dfrac{6\,000}{200+t}\right)=30$.\\
                Vậy đường thẳng $y=30$ là tiệm cận ngang của đồ thị hàm số $f(t)$ \texttt{(Hình 17).}}{\begin{tikzpicture}[scale=.1,xscale=0.1, font=\footnotesize, line join=round, line cap=round, >=stealth]
                    \draw[->] (-.5,0)--(0,0) node[below right]{$O$}--(500,0) node[below]{$x$};
                    \draw[->] (0,-1) --(0,34) node[right]{$y$};
                    \draw[blue] [domain=0:500, samples=100] %
                    plot (\x, {(30*(\x))/((\x)+200)});
                    \draw[fill] (0,0) circle (1pt);
                    \foreach \y/\g in {30/180}
                    \draw[fill] (0,\y) circle(1pt)node [shift={(\g:.3)}] {$\y$};
                    \draw[thick] (-.1,30)--(500,30);
                    \draw (250,-5) node{Hình 17};
            \end{tikzpicture}}
            \item Ta có đồ thị hàm số $y=f(t)$ nhận đường thẳng $y=30$ làm đường tiệm cận ngang, tức là khi $t$ càng lớn thì nồng độ muối trong bể sẽ tiến gần đến mức $30$ (gam/lít). Lúc đó, nồng độ muối trong bể sẽ gần như bằng nồng độ nước muối bơm vào bể.
        \end{enumerate}
    }
\end{vd}
\begin{vd}
    Một mô hình kinh tế mô tả lượng cung cầu theo giá cả được cho bởi hàm:
    \[
    Q(p) = \frac{k}{p - p_0}
    \]
    trong đó \( Q(p) \) là lượng cung cầu, \( p \) là giá cả, \( p_0 \) là mức giá tối thiểu, và \( k \) là hằng số tỷ lệ. Xác định tiệm cận đứng của hàm số này và nêu ý nghĩa của nó.
    % \shortans{$Q=p$, khi giá giảm về mức tối thiểu thì nhu cầu tăng lên vô hạn}
    \loigiai{
        Để tìm tiệm cận đứng, ta xem xét các giá trị của \( p \) làm cho mẫu số của phương trình bằng 0:
        \[
        p - p_0 = 0 \Rightarrow p = p_0
        \]
        Vậy đường thẳng \( p = p_0 \) là tiệm cận đứng của đồ thị hàm số.
        \textbf{Ý nghĩa:} Từ đó ta suy ra khi giá cả \( p \) càng sát với \( p_0 \), lượng cung cầu \( Q(p) \) sẽ tăng lên vô hạn. Điều này có nghĩa là nếu giá cả của sản phẩm giảm gần bằng mức giá tối thiểu \( p_0 \), thì nhu cầu đối với sản phẩm đó sẽ tăng lên vô hạn.}
\end{vd}
\BTTN
\Opensolutionfile{ans}[ans/2D1-4-DANG-1]
\begin{ex}%[Nguyễn Văn Sang, dự án Tex hoá đề cương trường Marie Curie - Lần 6]%[2D1Y4-1]
    Đường thẳng nào dưới đây là tiệm cận ngang của đồ thị hàm số $y=\dfrac{x-1}{x+1}$?
    \choice
    {$y=-1$}
    {$x=-1$}
    {\True $y=1$}
    {$x=1$}
    \loigiai{
        Tập xác định $\mathscr{D}=\mathbb{R}\setminus\left\lbrace -1\right\rbrace$.
        \begin{itemize}
            \item $\lim\limits_{x \to \pm\infty} y=\lim\limits_{x \to \pm\infty} \dfrac{x-1}{x+1}=1$ suy ra $y=1$ là tiệm cận ngang.
            \item $\heva{& \lim\limits_{x \to -1^+} \dfrac{x-1}{x+1}=-\infty \\ & \lim\limits_{x \to -1^-} \dfrac{x-1}{x+1}=+\infty}$ suy ra $x=-1$ là tiệm cận đứng.
        \end{itemize}
    }
\end{ex}
%%=====Câu 13
\begin{ex}%[Nguyễn Văn Sang, dự án Tex hoá đề cương trường Marie Curie - Lần 6]%[2D1Y4-1]
    Đồ thị hàm số $y=\dfrac{2 x-3}{1-2 x}$ có tiệm cận đứng là đường thẳng
    \choice
    {$x=3$}
    {$x=2$}
    {\True $x=\dfrac{1}{2}$}
    {$x=\dfrac{3}{2}$}
    \loigiai{
        Tập xác định $\mathscr{D}=\mathbb{R}\setminus\left\lbrace \dfrac{1}{2}\right\rbrace$.
        \begin{itemize}
            \item $\lim\limits_{x \to \pm\infty} y=\lim\limits_{x \to \pm\infty} \dfrac{2x-3}{1-2x}=-1$ suy ra $y=-1$ là tiệm cận ngang.
            \item $\heva{& \lim\limits_{x \to \tfrac{1}{2}^+} \dfrac{2x-3}{1-2x}=+\infty \\ & \lim\limits_{x \to \tfrac{1}{2}^-} \dfrac{2x-3}{1-2x}=-\infty}$ suy ra $x=\dfrac{1}{2}$ là tiệm cận đứng.
        \end{itemize}
    }
\end{ex}
\begin{ex}%[Nguyễn Văn Sang, dự án Tex hoá đề cương trường Marie Curie - Lần 6]%[2D1Y4-1]
    Đồ thị hàm số $y=\dfrac{2-3 x}{2 x-3}$ có tiệm cận đứng và ngang lần lượt là
    \choice
    {\True $x=\dfrac{3}{2}$ và $y=-\dfrac{3}{2}$}
    {$x=\dfrac{3}{2}$ và $y=1$}
    {$x=\dfrac{2}{3}$ và $y=-\dfrac{3}{2}$}
    {$x=\dfrac{2}{3}$ và $y=1$}
    \loigiai{
        Tập xác định $\mathscr{D}=\mathbb{R}\setminus\left\lbrace \dfrac{3}{2}\right\rbrace$.
        \begin{itemize}
            \item $\lim\limits_{x \to \pm\infty} y=\lim\limits_{x \to \pm\infty} \dfrac{2-3 x}{2 x-3}=\dfrac{-3}{2}$ suy ra $y=-\dfrac{3}{2}$ là tiệm cận ngang.
            \item $\heva{& \lim\limits_{x \to \tfrac{3}{2}^+} \dfrac{2-3 x}{2 x-3}=-\infty \\ & \lim\limits_{x \to \tfrac{3}{2}^-} \dfrac{2-3 x}{2 x-3}=+\infty}$ suy ra $x=\dfrac{3}{2}$ là tiệm cận đứng.
        \end{itemize}
    }
\end{ex}
\begin{ex}%[BGD-THPT-2020-104-L2]%[2D1Y4-1]
    Tiệm cận đứng của đồ thị hàm số $y=\dfrac{x+1}{x+3}$ có phương trình là
    \choice
    {$x=-1$}
    {$x=1$}
    {\True $x=-3$}
    {$x=3$}
    \loigiai{
        Tập xác định của hàm số đã cho $\mathscr{D}=\mathbb{R}\setminus\{-3\}$.\\
        Ta có $\lim\limits_{x\rightarrow-3^-}y=\lim\limits_{x\rightarrow-3^-}\dfrac{x+1}{x+3}=+\infty$ và $\lim\limits_{x\rightarrow-3^+}y=\lim\limits_{x\rightarrow-3^+}\dfrac{x+1}{x+3}=-\infty$.\\
        Khi đó đường tiệm cận đứng của đồ thị hàm số đã cho là $x=-3$.
    }
\end{ex}
%%==========Câu 11
\begin{ex}%[BGD-Minh Họa-2020-L2]%[2D1Y4-1]
    Tiệm cận ngang của đồ thị hàm số $y=\dfrac{x-2}{x+1}$ có phương trình là
    \choice
    {$y=-2$}
    {\True $y=1$}
    {$x=-1$}
    {$x=2$}
    \loigiai
    {
        Tập xác định: $\mathscr{D}=\mathbb{R}\setminus \{-1\}$.\\
        Ta có $\lim \limits_{x \to +\infty} y=\lim \limits_{x \to +\infty} \dfrac{x-2}{x+1}=\lim \limits_{x \to +\infty} \dfrac{1-\dfrac{2}{x}}{1+\dfrac{1}{x}}=1$ và $\lim \limits_{x \to -\infty} y=\lim \limits_{x \to -\infty} \dfrac{x-2}{x+1}=\lim \limits_{x \to -\infty} \dfrac{1-\dfrac{2}{x}}{1+\dfrac{1}{x}}=1$ nên đường thẳng $y=1$ là đường tiệm cận ngang của đồ thị.
    }
\end{ex}
%%==========Câu 12
\begin{ex}%[BGD-THPT-2021-101-L1]%[2D1Y4-1]
    Tiệm cận đứng của đồ thị hàm số $ y=\dfrac{2x-1}{x-1}$ là đường thẳng có phương trình là
    \choice
    {\True $x=1$}
    {$x=-1$}
    {$x=2$}
    {$x=\dfrac{1}{2}$}
    \loigiai{
        Vì $\lim\limits_{x\to 1^+}\dfrac{2x-1}{x-1}=+\infty $ và $\lim\limits_{x\to 1^-}\dfrac{2x-1}{x-1}=-\infty $ nên đồ thị hàm số $ y=\dfrac{2x-1}{x-1}$ có một tiệm cận đứng là đường thẳng $ x=1 $.
    }
\end{ex}
%%==========Câu 13
\begin{ex}%[BGD-THPT-2021-102-L1]%[2D1Y4-1]
    Tiệm cận đứng của đồ thị hàm số $ y=\dfrac{x+1}{x-2}$ là đường thẳng có phương trình là
    \choice
    {$x=-1$}
    {$x=-2$}
    {\True $x=2$}
    {$x=1$}
    \loigiai{
        Ta có $\displaystyle\lim\limits_{x\to 2^+}\dfrac{x+1}{x-2}=+\infty $; $\displaystyle\lim\limits_{x\to 2^-}\dfrac{x+1}{x-2}=-\infty $.\\
        Vậy đồ thị hàm số $ y=\dfrac{x+1}{x-2}$ có tiệm cận đứng là đường thẳng $ x=2 $.
    }
\end{ex}
%%==========Câu 14
\begin{ex}%[BGD-THPT-2021-103-L1]%[2D1B4-1]
    Tiệm cận đứng của đồ thị hàm số $ y=\dfrac{2x+1}{x-1}$ là đường thẳng có phương trình là
    \choice
    {$x=2$}
    {\True $x=1$}
    {$x=-\dfrac{1}{2}$}
    {$x=-1$}
    \loigiai{
        Ta có $\lim\limits_{x\to 1^+}y=\lim\limits_{x\to 1^+}\dfrac{2x+1}{x-1}=+\infty $ nên tiệm cận đứng của đồ thị hàm số là đường thẳng $ x=1 $.
    }
\end{ex}
%%==========Câu 15
\begin{ex}%[BGD-THPT-2021-104-L1]%[2D1Y4-1]
    Tiệm cận đứng của đồ thị hàm số $ y=\dfrac{x-1}{x+2}$ là đường thẳng có phương trình là
    \choice
    {$x=2$}
    {$x=-1$}
    {\True $x=-2$}
    {$x=1$}
    \loigiai{
        Ta có $\lim\limits_{x\to (-2)^{+}}\dfrac{x-1}{x+2}=-\infty $, $\lim\limits_{x\to (-2)^{-}}\dfrac{x-1}{x+2}=+\infty $.\\
        Đồ thị hàm số có tiệm cận đứng là đường thẳng có phương trình $ x=-2 $.
    }
\end{ex}
\begin{ex}%[2D1Y4-1]
    Giao điểm của tiệm cận đứng và tiệm cận ngang của đồ thị hàm số $y=\dfrac{-2}{3x-1}$ là điểm
    \choice
    {$Q\left(\dfrac{1}{3};-2\right)$}
    {$M\left(\dfrac{1}{3};-\dfrac{2}{3}\right)$}
    {$N\left(\dfrac{1}{3};2\right)$}
    {\True $P\left(\dfrac{1}{3};0\right)$}
    \loigiai{
        Tiệm cận đứng, tiệm cận ngang của đồ thị hàm số lần lượt là $x=\dfrac{1}{3}$ và $y=0$. Giao điểm của $2$ tiệm cận là $P\left(\dfrac{1}{3};0\right)$.
    }
\end{ex}
\begin{ex}%[2D1Y4-1]
    Đồ thị hàm số $y=\dfrac{3-4x}{x-5}$ có tâm đối xứng là điểm
    \choice
    {$M\left(5;-\dfrac{3}{5}\right)$}
    {$P\left(5;\dfrac{4}{5}\right)$}
    {$Q(5;3)$}
    {\True $N(5;-4)$}
    \loigiai{
        Tiệm cận đứng, tiệm cận ngang của đồ thị hàm số lần lượt là $x=5$ và $y=-4$. Tâm đối xứng là điểm $N(5;-4)$.
    }
\end{ex}
\begin{ex}%[2D1B4-1]
    Đồ thị hàm số nào dưới đây có tiệm cận đứng?
    \choice
    {$y=\dfrac{x^2-3x+2}{x-1}$}
    {$y=\dfrac{x^2}{x^2+1}$}
    {$y=\sqrt{x^2-1}$}
    {\True $y=\dfrac{x}{x+1}$}
    \loigiai{
    }
\end{ex}
\begin{ex}%[2D1B4-1]
    Cho hàm số $y=f(x)$ có bảng biến thiên như hình bên. Tổng số tiệm cận đứng và tiệm cận ngang của đồ thị hàm số đã cho là
    \begin{center}
        \begin{tikzpicture}
            \tkzTabInit[nocadre=false,lgt=1.5,espcl=3,deltacl=0.6]
            {$x$ /0.6,$y’$ /0.6,$y$ /2}
            {$-\infty$ ,$0$, $1$, $+\infty$}
            \tkzTabLine{,-,d,+,0,-,}
            \tkzTabVar{+/$+\infty$,-D-/$-\infty$/$-1$,+/$2$,-/$-3$}
        \end{tikzpicture}
    \end{center}
    \choice
    {$1$}
    {$3$}
    {\True $2$}
    {$4$}
    \loigiai{Dựa vào bảng biến thiên ta thấy đồ thị hàm số có tiệm cận đứng $x=0$ và tiệm cận ngang $y=-3$.}
\end{ex}
\begin{ex}%[2D1B4-1]
    Cho hàm số $y=f(x)$ có bảng biến thiên như hình bên. Tổng số tiệm cận đứng và tiệm cận ngang của đồ thị hàm số đã cho là
    \begin{center}
        \begin{tikzpicture}[scale=0.8]
            \tkzTabInit[nocadre=false,lgt=1.5,espcl=3,deltacl=0.6]
            {$x$ /0.6,$y’$ /0.6,$y$ /2}
            {$-\infty$ , $0$,$2$, $+\infty$}
            \tkzTabLine{,-,0,+,d,-,}
            \tkzTabVar{+/$8$,-/$1$,+/$4$,-/$2$}
        \end{tikzpicture}
    \end{center}
    \choice
    {$1$}
    {$3$}
    {\True $2$}
    {$4$}
    \loigiai{
        Dựa vào bảng biến thiên ta thấy đồ thị hàm số có tiệm cận ngang $y=8$ và $y=2$.
    }
\end{ex}
\begin{ex}%[2D1B4-1]
    Cho hàm số $y=f(x)$ có bảng biến thiên như hình bên. Tổng số tiệm cận đứng và tiệm cận ngang của đồ thị hàm số đã cho là
    \begin{center}
        \begin{tikzpicture}[scale=0.8]
            \tkzTabInit[nocadre=false,lgt=1.5,espcl=3,deltacl=0.6]
            {$x$ /0.6,$y’$ /0.6,$y$ /2}
            {$-\infty$ ,$1$, $2$, $+\infty$}
            \tkzTabLine{,+,d,-,d,+,}
            \tkzTabVar{-/$-4$,+/$3$,-/$-5$,+/$+\infty$}
        \end{tikzpicture}
    \end{center}
    \choice
    {\True $1$}
    {$3$}
    {$2$}
    {$0$}
    \loigiai{
        Dựa vào bảng biến thiên ta thấy đồ thị hàm số có một tiệm cận ngang $y=-4$.
    }
\end{ex}
\begin{ex}%[2D1B4-1]
    Cho hàm số $y=f(x)$ có bảng biến thiên như hình bên. Đồ thị hàm số đã cho có tiệm cận đứng là đường thẳng
    \begin{center}
        \begin{tikzpicture}[scale=0.8, font=\footnotesize, line join=round, line
            cap=round, >=stealth]
            \tkzTabInit[espcl=2.5,lgt=1,nocadre=false]
            {$x$/0.7,$f(x)$/2.1}
            {$-\infty$,$0$,$1$,$2$,$+\infty$}
            \tkzTabVar{-/$-\infty$,+/$2$,-D+/$-\infty$/$+\infty$,-/$4$,+/$+\infty$}
        \end{tikzpicture}
    \end{center}
    \choice
    {$x=0$}
    {\True $x=1$}
    {$x=2$}
    {$x=4$}
    \loigiai{Dựa vào bảng biến thiên ta thấy đồ thị hàm có tiệm cận đứng $x=1$.}
\end{ex}
%%==========Câu 16
\begin{ex}%[BGD-THPT-2019-103]%[2D1B4-1]
    Cho hàm số $y=f(x)$ có bảng biến thiên như sau
    \begin{center}
        \begin{tikzpicture}[scale=1, font=\footnotesize,line join=round, >=stealth]
            \tkzTabInit[nocadre=false,lgt=1.5,espcl=3]{$x$/.7,$y'$/.7,$y$/2.5}{$-\infty$,$0$,$3$,$+\infty$}%
            \tkzTabLine{,-,d,+,0,-,}%
            \tkzTabVar{+/$1$ , -D+/$-\infty$/$2$,-/$-3$, +/$3$}%
        \end{tikzpicture}
    \end{center}
    Tổng số tiệm cận đứng và tiệm cận ngang của đồ thị hàm số đã cho là
    \choice
    {1}
    {2}
    {\True 3}
    {4}
    \loigiai{
        Nhìn bảng biến thiên ta thấy\\
        $\lim\limits_{x \to 0^-} f(x)=-\infty \Rightarrow x=0$ là TCĐ của đồ thị hàm số.\\
        $\lim\limits_{x \to +\infty} f(x)=3 \Rightarrow y=3$ là TCN của đồ thị hàm số.\\
        $\lim\limits_{x \to -\infty} f(x)=1 \Rightarrow y=1$ là TCN của đồ thị hàm số.\\
        Vậy hàm số có 3 tiệm cận.}
\end{ex}
%%==========Câu 17
\begin{ex}%[BGD-THPT-2019-102]%[2D1B4-1]
    Cho hàm số $f(x)$ có bảng biến thiên như sau
    \begin{center}
        \begin{tikzpicture}[scale=1, font=\footnotesize,line join=round, >=stealth]
            \tkzTabInit[lgt=1.2,espcl=3]
            {$x$/0.8,$f’(x)$/0.8,$f(x)$/2}
            {$-\infty$,$0$,$1$,$+\infty$}
            \tkzTabLine{ ,-,d,-,0,+,}
            \tkzTabVar{+/$0$,-D+/$-\infty$/$2$,-/$-2$,+/$+\infty$}
        \end{tikzpicture}
    \end{center}
    Tổng số tiệm cận đứng và tiệm cận ngang của đồ thị hàm số đã cho là
    \choice
    {$3$}
    {$1$}
    {\True $2$}
    {$4$}
    \loigiai{
        Từ bảng biến thiên đã cho ta có\\
        $\lim\limits_{x \to -\infty} f(x)=0$ nên đường thẳng $y=0$ là một tiệm cận ngang của đồ thị hàm số.\\
        $\lim\limits_{x \to 0^-} f(x)=-\infty$ nên đường thẳng $x=0$ là một tiệm cận đứng của đồ thị hàm số.\\
        Vậy đồ thị hàm số đã cho có hai đường tiệm cận.}
\end{ex}
\begin{ex}%[2D1B4-1]
    Cho hàm số $y=f(x)$ có bảng biến thiên như hình bên. Tổng số tiệm cận đứng và tiệm cận ngang của đồ thị hàm số đã cho là
    \begin{center}
        \begin{tikzpicture}[scale=0.8]
            \tkzTabInit[nocadre=false,lgt=1.5,espcl=3,deltacl=0.6]
            {$x$ /0.6,$y’$ /0.6,$y$ /2}
            {$-\infty$ ,$0$, $1$, $+\infty$}
            \tkzTabLine{,+,0,-,d,-,}
            \tkzTabVar{-/$4$,+/$2$,-D+/$-\infty$/$5$,-/$-3$}
        \end{tikzpicture}
    \end{center}
    \choice
    {$1$}
    {\True $3$}
    {$2$}
    {$4$}
    \loigiai{
        Dựa vào bảng biến thiên ta thấy đồ thị hàm số có tiệm cận đứng $x=1$, tiệm cận ngang $y=4$ và $y=-3$.
    }
\end{ex}
\begin{ex}%[2D1B4-1]
    Cho hàm số $y=f\left(x\right)$ có bảng biến thiên như sau
    \begin{center}
        \begin{tikzpicture}[scale=1,line join=round,>=stealth]\tikzset{double style/.append style={double distance=2pt}}
            \tkzTabInit[nocadre=false,lgt=1.2,espcl=2.2,deltacl=0.6]
            {$x$ /.6,$y'$ /.6,$y$ /2.2}
            {$ -\infty $,$-2$,$0$,$+\infty$}
            \tkzTabLine{,-,d,+,d,-}
            \tkzTabVar{+/$+\infty$,-D-/$1$/$-\infty$,+D+/$+\infty$/$1$,-/$0$,}
        \end{tikzpicture}
    \end{center}
    Tổng số đường tiệm cận đứng và tiệm cận ngang của đồ thị hàm số đã cho bằng
    \choice
    {$2$}
    {$1$}
    {$0$}
    {\True $3$}
    \loigiai{
        Ta có
        \begin{itemize}
            \item $\lim\limits_{x \to -2^{+}} y=-\infty \Rightarrow x=-2$ là tiệm cận đứng.
            \item $\lim\limits_{x \to 0^{-}} y=+\infty \Rightarrow x=0$ là tiệm cận đứng.
            \item $\lim\limits_{x \to +\infty} y=0 \Rightarrow y=0$ là tiệm cận ngang.
        \end{itemize}
        Vậy đồ thị hàm số đã cho có tổng đường tiệm cận đứng và tiệm cận ngang là $3$.}
\end{ex}
\begin{ex}%[2D1B4-1]
    Cho hàm số $y=f\left(x\right)$ liên tục trên $\mathbb{R} \backslash\{1\}$ có bảng biến thiên như bảng sau:
    \begin{center}
        \begin{tikzpicture}[scale=1,line join=round,>=stealth]
            \tikzset{double style/.append style={double distance=2pt}}
            \tkzTabInit[nocadre=false,lgt=1.2,espcl=2.8,deltacl=0.6]
            {$x$ /0.6,$y'$ /0.6,$y$ /2.2}
            {$ -\infty $,$-1$,$1$,$+\infty$}
            \tkzTabLine{,-,0,+,d,+}
            \tkzTabVar{+/$1$,-/$-\sqrt 2$,+D-/$+\infty$/$-\infty$,+/$-1$,}
        \end{tikzpicture}
    \end{center}
    Tổng số đường tiệm cận đứng và đường tiệm cận ngang của đồ thị hàm số $y=f\left(x\right)$ là
    \choice
    {$1$}
    {$4$}
    {$2$}
    {\True $3$}
    \loigiai{
        Do $\lim\limits_{x \to 1^{+}} y=-\infty \Rightarrow$ Tiệm cận đứng $x=1$.\\
        Lại có $\lim\limits_{x \to +\infty} y=-1 ; \lim\limits_{x \to -\infty} y=1 \Rightarrow$ Đồ thị có $2$ tiệm cận ngang là $y=\pm 1$.\\
        Vậy, đồ thị hàm số đã cho có tổng số tiệm cận là $3$.}
\end{ex}
\begin{ex}%[2D1B4-1]
    Cho hàm số $y=f\left(x\right)$ có bảng biến như sau:
    \begin{center}
        \begin{tikzpicture}[scale=1,line join=round,>=stealth]
            \tikzset{double style/.append style={double distance=2pt}}
            \tkzTabInit[nocadre=false,lgt=1.2,espcl=2.5,deltacl=0.6]{$x$ /.6,$y'$ /.6,$y$ /2}
            {$ -\infty $,$-3$,$3$,$+\infty$}
            \tkzTabLine{,+,d,+,d,+}
            \tkzTabVar{-/$0$,+D-/$+\infty$/$-\infty$,+D-/$+\infty$/$-\infty$,+/$0$,}
        \end{tikzpicture}
    \end{center}
    Số đường tiệm cận của đồ thị hàm số là
    \choice
    {\True $3$}
    {$1$}
    {$4$}
    {$2$}
    \loigiai{
        Từ bảng biến thiên của hàm số ta có
        \begin{itemize}
            \item $\lim\limits_{x \to -\infty} y=0 ; \lim\limits_{x \to +\infty} y=0 \Rightarrow$ Đường thẳng $y=0$ là tiệm cận ngang.
            \item $\lim\limits_{x \to (-3)^{-}} y=+\infty \Rightarrow$ Đường thẳng $x=-3$ là tiệm cận đứng.
            \item $+\lim\limits_{x \to 3^{-}} y=+\infty \Rightarrow$ Đường thẳng $x=3$ là tiệm cận đứng.
        \end{itemize}
        Vậy số đường tiệm cận của đồ thị hàm số là $3$.}
\end{ex}
\begin{ex}%[2D1B4-1]
    Cho hàm số $y=f\left(x\right)$ có bảng biến thiên như sau
    \begin{center}
        \begin{tikzpicture}[scale=1,line join=round,>=stealth]
            \tikzset{double style/.append style={double distance=2pt}}
            \tkzTabInit[nocadre=false,lgt=1.2,espcl=2.5,deltacl=0.6]
            {$x$ /.6,$y'$ /.6,$y$ /2.2}
            {$ -\infty $,$-2$,$2$,$+\infty$}
            \tkzTabLine{,-,d,-,d,-}
            \tkzTabVar{+/$0$,-D+/$-\infty$/$+\infty$,-D+/$-\infty$/$+\infty$,-/$-\infty$,}
        \end{tikzpicture}
    \end{center}
    Tổng số tiệm cận đứng và tiệm cận ngang của đồ thị hàm số đã cho là
    \choice
    {$4$}
    {$2$}
    {\True $3$}
    {$1$}
    \loigiai{
        Dựa vào bảng biến thiên, ta có:
        \begin{itemize}
            \item $\lim\limits_{x \to -\infty} f(x)=0$ nên đường thẳng $y=0$ là đường tiệm cận ngang.
            \item $\lim\limits_{x \to -2^{+}} f(x)=+\infty $ nên đường thẳng $x=-2$ là đường tiệm cận đứng.
            \item $\lim\limits_{x \to 2^{+}} f(x)=+\infty$ nên đường thẳng $x=2$ là đường tiệm cận đứng.
        \end{itemize}
        Vậy, tổng số tiệm cận đứng và tiệm cận ngang của đồ thị hàm số đã cho là $3$.}
\end{ex}
%%==========Câu 20
\begin{ex}%[THPT Yên Định - Thanh Hóa 2019]%[2D1B4-1]
    Cho hàm số $ y=f(x) $ xác định và có đạo hàm trên $ \mathbb{R}\setminus\{\pm 1\} $. Hàm số có bảng biến thiên như hình vẽ dưới đây.
    \begin{center}
        \begin{tikzpicture}[scale=1, font=\footnotesize,line join=round, >=stealth]
            \tkzTabInit[nocadre=false,lgt=1.2,espcl=2.5,deltacl=0.6]{$x$/.6 ,$y'$/.6,$y$/2.5} {$-\infty$ , $-1$ , $0$ , $1$ , $+\infty$}
            \tkzTabLine{ , + , d , - , d , + , d , + , }
            \tkzTabVar{-/$-4$ , +D-/$+\infty$/$-\infty$ , +/$2$,-D-/$-\infty$/$-\infty$,+/$-1$}
        \end{tikzpicture}
    \end{center}
    Tổng số đường tiệm cận đứng và tiệm cận ngang của đồ thị hàm số đã cho là
    \choice
    { $ 1 $ }
    { $ 2 $ }
    { $ 3 $ }
    {\True $ 4 $ }
    \loigiai{
        Dựa vào bảng biến thiên, suy ra:\\
        $ \lim \limits_{x \to - \infty} y=-4 $, $ \lim \limits_{x \to + \infty} y=-1$. Đồ thị có hai tiệm cận ngang là $ y=-4 $ và $ y=-1 $.\\
        Lại có $ \lim \limits_{x \to (-1)^+} y=+\infty $ và $ \lim \limits_{x \to 1^-} y=+\infty $, $ \lim \limits_{x \to 1^-} y=-\infty $. Đồ thị hàm số có hai đường tiệm cận đứng là $ x=1 $ và $ x=-1 $.
    }
\end{ex}
\begin{ex}%[2D1B4-1]
    Cho hàm số $y=f\left(x\right)$ có bảng biến thiên như hình vẽ dưới đây.
    \begin{center}
        \begin{tikzpicture}[line cap=round,line join=round,>=triangle 45,x=1.0cm,y=1.0cm]
            \clip(-1.58,-2.4) rectangle (12.58,2.);
            \fill[line width=1.2pt,dash pattern=on 15 pt off 5pt,color=white,fill=black,pattern=north east lines,pattern color=black] (0.,1.) -- (2.84,1.) -- (2.84,-1.96) -- (0.,-1.96) -- cycle;
            \draw (-1.,1.)-- (12.,1.);
            \draw (-1.,0.)-- (12.,0.);
            \draw (0.,1.62)-- (0.,-1.96);
            \draw (0.08,1.5) node[anchor=north west] {$-\infty$};
            \draw (2.46,1.5) node[anchor=north west] {$-2$};
            \draw (6.85,1.5) node[anchor=north west] {$0$};
            \draw (11.14,1.5) node[anchor=north west] {$+\infty$};
            \draw (2.84,1.)-- (2.84,-1.96);
            \draw (3.,1.)-- (3.,-1.96);
            \draw (7.,1.)-- (7.,-1.96);
            \draw (7.14,1.)-- (7.14,-1.96);
            \draw (-0.7,1.5) node[anchor=north west] {$x$};
            \draw (-0.72,0.78) node[anchor=north west] {$y'$};
            \draw (-0.64,-0.8) node[anchor=north west] {$y$};
            \draw [->] (3.96,-1.54) -- (6.24,-0.64);
            \draw [->] (7.64,-0.58) -- (11.12,-1.7);
            \draw (11.34,-1.5) node[anchor=north west] {$0$};
            \draw (8.98,0.7) node[anchor=north west] {$-$};
            \draw (4.68,0.7) node[anchor=north west] {$+$};
            \draw (6.0,-0.2) node[anchor=north west] {$+\infty$};
            \draw (7.22,-0.2) node[anchor=north west] {$1$};
            \draw (3.08,-1.5) node[anchor=north west] {$-\infty$};
        \end{tikzpicture}
    \end{center}
    Hỏi đồ thị của hàm số đã cho có bao nhiêu đường tiệm cận?
    \choice
    {\True $3$}
    {$2$}
    {$4$}
    {$1$}
    \loigiai{
        Dựa vào bảng biến thiên ta có:\\
        $\lim\limits_{x \to -2^{+}} f(x)=-\infty$, suy ra đường thẳng $x=-2$ là tiệm cận đứng của đồ thị hàm số.\\
        $\lim\limits_{x \to 0^{-}} f(x)=+\infty$, suy ra đường thẳng $x=0$ là tiệm cận đứng của đồ thị hàm số.\\
        $\lim\limits_{x \to +\infty} f(x)=0$, suy ra đường thẳng $y=0$ là tiệm cận ngang của đồ thị hàm số.\\
        Vậy đồ thị hàm số có $3$ đường tiệm cận.}
\end{ex}
%%=====Câu 5
\begin{ex}%[Nguyễn Văn Sang, dự án Tex hoá đề cương trường Marie Curie - Lần 6]%[2D1Y4-1]
    Cho hàm số $y=f(x)$ có $\lim\limits_{x \rightarrow 3} f(x)=+\infty$, $\lim\limits_{x \rightarrow+\infty} f(x)=-\infty$, $\lim\limits_{x \rightarrow-\infty} f(x)=8$ và $\lim\limits_{x \rightarrow 7} f(x)=5 $. Tổng số tiệm cận ngang và tiệm cận đứng của đồ thị hàm số đã cho là
    \choice
    {$4$}
    {\True $2$}
    {$1$}
    {$3$}
    \loigiai{
        Ta có
        \begin{itemize}
            \item $\lim\limits_{x \rightarrow-\infty} f(x)=8$, suy ra $y=8$ là tiệm cận ngang.
            \item $\lim\limits_{x \rightarrow 3} f(x)=+\infty$, suy ra $x=3$ là tiệm cận đứng.
            \item $\lim\limits_{x \rightarrow 7} f(x)=5 $, suy ra $x=7$ không là tiệm cận đứng.
        \end{itemize}
        Vậy đồ thị hàm số có $1$ tiệm cận đứng và $1$ tiệm cận ngang.
    }
\end{ex}
%%=====Câu 7
\begin{ex}%[Nguyễn Văn Sang, dự án Tex hoá đề cương trường Marie Curie - Lần 6]%[2D1Y4-1]
    Cho hàm số $y=f(x)$ có $\lim\limits_{x \rightarrow 1^{+}} f(x)=+\infty$ và $\lim\limits_{x \rightarrow 1^{-}} f(x)=2$. Mệnh đề nào sau đây đúng?
    \choice
    {Đồ thị hàm số không có tiệm cận}
    {\True Đồ thị hàm số có tiệm cận đứng $x=1$}
    {Đồ thị hàm số có hai tiệm cận}
    {Đồ thị hàm số tiệm cận ngang $y=2$}
    \loigiai{
        Ta có $\lim\limits_{x \rightarrow 1^{-}} f(x)=2$, suy ra $x=1$ là tiệm cận đứng.
    }
\end{ex}
\begin{ex}%[2D1B4-1]
    Cho hàm số $y=\dfrac{\sqrt{x+1}}{\sqrt{x^2-4}}$ mệnh đề nào sau đây đúng?
    \choice
    {\True Đồ thị hàm số có một tiệm cận đứng và một tiệm cận ngang}
    {Đồ thị hàm số có một tiệm cận đứng và hai tiệm cận ngang}
    {Đồ thị hàm số có hai tiệm cận đứng và hai tiệm cận ngang}
    {Đồ thị hàm số có hai tiệm cận đứng và một tiệm cận ngang}
    \loigiai{
        Tập xác định $\mathscr{D}=[-1;+\infty) \setminus \{2\}$. \\
        Đồ thị hàm số có một tiệm cận đứng $x=2$, tiệm cận ngang là $y=0$.
    }
\end{ex}
\begin{ex}%[2-HK1-49-THPT-NKKN-TPHCM, 12EX5]%[Nhật Thiện, ID6]%[2D1K4-2]%
    Với giá trị nào của $m$ thì đồ thị hàm số $y=\dfrac{mx-1}{2x+m}$ có tiệm cận đứng là đường thẳng $x=-1$?
    \choice
    {$m=2$}
    {\True $m=-2$}
    {$m=\dfrac{1}{2}$}
    {$m=0$}
    \loigiai{
        Đồ thị hàm số $y=\dfrac{mx-1}{2x+m}$ có tiệm cận đứng là đường thẳng $x=-1$ khi và chỉ khi $$\heva{&m(-1)-1\ne 0\\&2(-1)+m=0}\Leftrightarrow \heva{&m\ne -1\\&m=-2(n).}$$
    }
\end{ex}
\begin{ex}%[2D1K4-1]
    Đồ thị hàm số $y=\dfrac{2x-1-\sqrt{x^2+x+3}}{x^2-5x+6}$ có tất cả đường tiệm cận đứng là đường thẳng
    \choice
    {$x=-3$ và $x=-2$}
    {$x=-3$}
    {$x=3$ và $x=-2$}
    {\True $x=3$}
    \loigiai{
        Điều kiện xác định $x \ne 3$, $x \ne 2$.\\
        Với điều kiện xác định trên, ta có
        {\allowdisplaybreaks
            \begin{eqnarray*}
                y&=&\dfrac{2x-1-\sqrt{x^2+x+3}}{x^2-5x+6}=\dfrac{(3x+1)(x-2)}{(x-2)(x-3)\left(2x-1+\sqrt{x^2+x+3}\right)}\\
                &=&\dfrac{3x+1}{(x-3)\left(2x-1+\sqrt{x^2+x+3}\right)}.
        \end{eqnarray*} }
        Tiệm cận đứng của đồ thị hàm số là $x=3$.
    }
\end{ex}
\begin{ex}%[2D1K4-1]
    Số tiệm cận đứng của đồ thị hàm số $y=\dfrac{\sqrt{x+9}-3}{x^2+x}$ là
    \choice
    {$3$}
    {$2$}
    {$0$}
    {\True $1$}
    \loigiai{
        Tập xác định $\mathscr{D}=[-9;+\infty)\setminus \{-1;0\}$. \\
        Ta có $\left\{\begin{aligned}
            &\lim\limits_{x\to -1^+} y=\lim\limits_{x\to -1^+} \dfrac{\sqrt{x+9}-3}{x^2+x}=+\infty \\
            &\lim\limits_{x\to -1^-} y =\lim\limits_{x\to -1^-} \dfrac{\sqrt{x+9}-3}{x^2+x}=-\infty
        \end{aligned}\right. \Rightarrow x=-1$ là tiệm cận đứng. \\
        Ngoài ra $\lim\limits_{x\to 0} y =\lim\limits_{x\to 0} \dfrac{\sqrt{x+9}-3}{x^2+x}=\dfrac{1}{6}$ nên $x=0$ không phải là một tiệm cận đứng.}
\end{ex}
\BTTF
\begin{ex}%[EX-TF-2024, Lê Đạt]%[2D1N4-1]
    Cho hàm số $y=\dfrac{2x-3}{x-1}$. Xét tính đúng sai các khẳng định dưới đây
    \choiceTF
    {\True Đường tiệm cận đứng của đồ thị hàm số là $ x=1 $}
    {Đường tiệm cận đứng của đồ thị hàm số là $ y=2 $}
    {Đường tiệm cận ngang của đồ thị hàm số là $ x=1 $}
    {\True Đường tiệm cận ngang của đồ thj hàm số là $ y=2 $}
    \loigiai{
        Ta có $\lim\limits_{x\to -\infty}y=\lim\limits_{x\to +\infty}y=2$ nên đồ thị hàm số đã cho có tiệm cận ngang là $y=2$.\\
        Ta có $\lim\limits_{x\to 1^+}y=-\infty$ nên đồ thị hàm số đã cho có tiệm cận ngang là $ x=1 $.
        \begin{itemchoice}
            \itemch Đường tiệm cận đứng của đồ thị hàm số là $ x=1 $.
            \itemch Đường tiệm cận đứng của đồ thị hàm số là $ x=1 $.
            \itemch Đường tiệm cận ngang của đồ thj hàm số là $ y=2 $.
            \itemch Đường tiệm cận ngang của đồ thj hàm số là $ y=2 $.
        \end{itemchoice}
    }
\end{ex}
\begin{ex}%[EX-TF-2024, Lê Đạt]%[2D1N4-1]
    Cho hàm số $y=f(x)$ có bảng biến thiên như sau
    \begin{center}
        \begin{tikzpicture}[>=stealth]
            \tkzTabInit[nocadre=false,lgt=1,espcl=3,deltacl=0.6]
            {$x$/.7 ,$y'$/.7,$y$/2}
            {$-\infty$ , $-2$ , $0$, $+\infty$}
            \tkzTabLine{ , - , d , + , d , -, }
            \tkzTabVar{+/$+\infty$ , -D-/$1$/$-\infty$ , +D+/$+\infty$ /$1$, -/$0$}
        \end{tikzpicture}
    \end{center}
    Xét tính đúng sai của các khẳng định sau
    \choiceTF
    {\True $ x=0 $ là tiệm cận đứng của đồ thị hàm số $ y=f(x) $}
    {\True $ x=-2 $ là tiệm cận đứng của đồ thị hàm số $ y=f(x) $}
    {$ x=1 $ là tiệm cận đứng của đồ thị hàm số $ y=f(x) $}
    {\True $ y=0 $ là tiệm cận ngang của đồ thị hàm số $ y=f(x) $}
    \loigiai{
        \begin{itemchoice}
            \itemch $\lim \limits_{x \to 0^-} f(x)=+\infty\Rightarrow x=0$ là đường tiệm cận đứng của đồ thị hàm số $f(x)$.
            \itemch $\lim \limits_{x \to (-2)^+} f(x)=-\infty\Rightarrow x=-2$ là đường tiệm cận đứng của đồ thị hàm số $f(x)$.
            \itemch Đồ thị hàm số chỉ có hai tiệm cận đứng là $ x=0 $ và $ x=-2 $.
            \itemch $\lim \limits_{x \to +\infty} f(x)=0\Rightarrow y=0$ là đường tiệm cận ngang của đồ thị hàm số $f(x)$.
        \end{itemchoice}
    }
\end{ex}
%===== DẠNG 2
\begin{ex}%[EX-TF-2024, Lê Đạt]%[2D1H4-2]
    Cho hàm số $ y=\dfrac{m^2x+1}{x-1} $. Xét tính đúng sai của các khẳng định sau
    \choiceTF
    {\True Đồ thị hàm số luôn có tiệm cận ngang}
    {\True Đồ thị hàm số luôn có tiệm cận đứng}
    {\True Khi $ m=1$ đồ thị hàm số có $ 2 $ đường tiệm cận}
    {Khi $ m=0 $ đồ thị hàm số có $ 1 $ đường tiệm cận}
    \loigiai{
        \begin{itemchoice}
            \itemch $\lim\limits_{x\to -\infty}y=\lim\limits_{x\to +\infty}y=m^2$ suy ra hàm số luôn có tiệm cận ngang.
            \itemch $\lim\limits_{x\to 1^+}y=+\infty$ nên đồ thị hàm số đã cho có tiệm cận ngang là $ x=1 $.
            \itemch Khi $ m=1 $ ta được hàm số $ y=\dfrac{x+1}{x-1} $ suy ra đồ thì hàm số có $ x=1 $ là tiệm cận đứng và $ y=1 $ là tiệm cận ngang nên đồ thị hàm số có $ 2 $ tiệm cận.
            \itemch Khi $ m=0 $ ta được hàm số $ y=\dfrac{1}{x-1} $ suy ra đồ thì hàm số có $ x=1 $ là tiệm cận đứng và $ y=0 $ là tiệm cận ngang nên đồ thị hàm số có $ 2 $ tiệm cận.
        \end{itemchoice}
    }
\end{ex}
\begin{ex}%[EX-TF-2024, Lê Đạt]%[2D1H4-2]
    Cho hàm số $y=\dfrac{m x^{2}+6 x-2}{x+2}$. Xét tính đúng sai của các khẳng định sau
    \choiceTF
    {Đồ thị hàm số luôn có tiệm cận đứng với mọi $ m $}
    {Đồ thị hàm số không có tiệm cận ngang với mọi $ m $}
    {\True Khi $ m=1 $ đồ thị hàm số có một tiệm cận xiên là $ y=x+4 $ }
    {Đồ thị hàm số luôn có tiệm cận xiên}
    \loigiai{
        \begin{itemchoice}
            \itemch Khi $ m=\dfrac{7}{2} $ hàm số trở thành $y=\dfrac{\dfrac{7}{2} x^{2}+6 x-2}{x+2}=\dfrac{7}{2}\left(x-\dfrac{2}{7} \right) $ suy ra đồ thị hàm số không có tiệm cận đứng.
            \itemch Khi $ m=0 $ hàm số trở thành $ y=\dfrac{6x-2}{x+2} $ từ đó suy ra đồ thị hàm số có $ y=6 $ là tiệm cận ngang.
            \itemch Khi $ m=1 $ hàm số trở thành $ y=\dfrac{x^2+6x-2}{x+2}=x+4-\dfrac{10}{x+2} $ từ đó suy ra $ y=x+4 $ là một tiệm cận ngang.
            \itemch Khi $ m=0 $ hàm số trở thành $ y=\dfrac{6x-2}{x+2} $ từ đó suy ra đồ thị hàm số có $ y=6 $ là tiệm cận ngang, $ x=-2 $ là tiệm cận đứng và không có tiệm cận xiên.
        \end{itemchoice}
    }
\end{ex}
\begin{ex}
    Cho hàm số $y=\dfrac{x-1}{x^2-8 x+m}$, $m$ là tham số. Các mệnh đề sau đúng hay sai?
    \choiceTF
    {\True Đồ thị hàm số có 1 đường tiệm cận ngang}
    {Khi $m<16$ thì đồ thị hàm số có 3 đường tiệm cận}
    {Khi $m=16$ thì đồ thị hàm số có 2 đường tiệm cận đứng}
    {\True Có 14 giá trị nguyên dương của $m$ để đồ thị hàm số có 3 đường tiệm cận}
    \loigiai{
        Ta có $\lim \limits{n \to +\infty}_{x \rightarrow-\infty} \frac{x-1}{x^2-8 x+m}=\lim \limits{n \to +\infty}_{x \rightarrow+\infty} \frac{x-1}{x^2-8 x+m}=0$ nên hàm số có một tiện cận ngang $y=0$.
        Hàm số có 3 đường tiệm cận khi và chỉ khi hàm số có hai đường tiệm cận đứng $\Leftrightarrow$ phương trình $x^2-8 x+m=0$ có hai nghiệm phân biệt khác $1 \Leftrightarrow\left\{\begin{array}{l}\Delta^{\prime}=16-m>0 \\ m-7 \neq 0\end{array} \Leftrightarrow\left\{\begin{array}{l}m<16 \\ m \neq 7\end{array}\right.\right.$.
        Kết hợp với điều kiện $m$ nguyên dương ta có $\quad m \in\{1 ; 2 ; 3 ; \ldots ; 6 ; 8 ; \ldots ; 15\}$. Vậy có 14 giá trị của $m$ thỏa mãn đề bài.}
\end{ex}
\begin{ex}
    Cho hàm số $y=\dfrac{x^2+m x-1}{x-1}\left(C_m\right)$ ( $m$ là tham số). Các mệnh đề sau đúng hay sai?
    \choiceTF
    {\True Để đồ thị $\left(C_m\right)$ của hàm số có tiệm cận xiên thì $m \neq 0$.}
    {\True Để tiệm cận xiên của $\left(C_m\right)$ đi qua $M(2,-5)$ thì $m=-8$}
    { Để tiệm cận xiên của $\left(C_m\right)$ tạo với hai trục toạ độ một tam giác có diện tích bằng 8 thì tổng tất cả các giá trị $m$ tìm được bằng 2}
    { Với $m=3$ thì giao điểm của hai đường tiệm cận của $\left(C_m\right)$ nằm trên Parapol $y=x^2+3$}
    \loigiai{
        Hàm số xác định trên $\mathbb{R} \backslash\{1\}$.
        \begin{listEX}
            \item Ta có $y=x+m+1+\frac{m}{x-1}$
            Để đồ thị $\left(C_m\right)$ của hàm số có tiệm cận xiên thì $m \neq 0$.
            - Với $m \neq 0,\left(C_m\right)$ có tiệm cận xiên
            $y=x+m+1\left(\Delta_m\right)$ vì $\lim \limits{n \to +\infty}_{x \rightarrow \infty}[y-(x+m+1)]=\lim \limits{n \to +\infty}_{x \rightarrow \infty} \frac{m}{x-1}=0$.
            \item Để $\left(\Delta_m\right)$ qua $M(2,-5)$ thì $-5=2+m+1 \Leftrightarrow m=-8$. (thỏa mãn $m \neq 0$ ).
            \item Gọi $A$ là giao điểm của $\Delta_m$ với $O x$. Khi đó $A(-m-1 ; 0)$
            Gọi $B$ là giao điểm của $\Delta_m$ với $O y$. Khi đó $B(0 ; m+1)$.
            Suy ra $S_{\triangle O A B}=\frac{1}{2} O A \cdot O B=\frac{1}{2}|-m-1||m+1|=\frac{1}{2}(m+1)^2$
            Để $S_{\triangle O A B}=8 \Leftrightarrow \frac{1}{2}(m+1)^2=8 \Leftrightarrow\left[\begin{array}{l}m=-5 \\ m=3\end{array}\right.$ (thỏa mãn $m \neq 0$ ).
            \item Ta có với $m \neq 0, x=1$ là tiệm cận đứng vì $\lim \limits{n \to +\infty}_{x \rightarrow 1} y=\infty$ nên $y=x+m+1$ là tiệm cận xiên.
            Khi đó giao điểm của 2 tiệm cận là $I(1, m+2)$.
            Để $I$ nằm trên Parabol $y=x^2+3$ thì $m+2=1+3 \Leftrightarrow m=2(\mathrm{t} / \mathrm{m} m \neq 0)$.
        \end{listEX}
    }
\end{ex}
%===== DẠNG 3
\begin{ex}%[EX-TF-2024, Lê Đạt]%[2D1N4-3]
    \immini{Cho hàm số $y=f(x)$ có đồ thị như hình bên. Xét tính đúng sai của các khẳng định sau
        \choiceTF
        {$ x=2 $ là đường tiệm cận ngang của đồ thị hàm số}
        {\True $ x=-1 $ là đường tiệm cận đứng của đồ thị hàm số}
        {\True Đồ thị hàm số có hai đường tiệm cận}
        {\True Đồ thị hàm số không có tiệm cận xiên}
    }{
        \begin{tikzpicture}[scale=0.5, font=\footnotesize, line join=round, line cap=round, >=stealth]
            \draw[->](-5,0)--(5,0)node[below]{ $x$};
            \draw[->](0,-4)--(0,5)node[right]{ $y$};
            \draw [fill=black,draw=black] (0,0) circle (1pt)node[above left] { $O$};
            \foreach \x in {-1}\draw[shift={(\x,0)}](0pt,-2pt)--(0pt,2pt) node[below left]{ $\x$};
            \foreach \y in {2}\draw[shift={(0,\y)}](-2pt,0pt)--(2pt,0pt)node[above right]{ $\y$};
            \clip(-5,-4) rectangle (5,5);
            \draw[smooth,samples=100,domain=-5:-1.1] plot(\x,{(2*(\x)-1)/((\x)+1)});
            \draw[smooth,samples=100,domain=-0.9:5] plot(\x,{(2*(\x)-1)/((\x)+1)});
            \draw[dashed](-5,2)--(5,2) (-1,-4)--(-1,5);
        \end{tikzpicture}
    }
    \loigiai{
        \begin{itemchoice}
            \itemch $ y=2 $ là đường tiệm cận ngang của đồ thị hàm số.
            \itemch $ x=-1 $ là đường tiệm cận đứng của đồ thị hàm số.
            \itemch $ x=-1 $ là đường tiệm cận đứng và $ y=2 $ là đường tiệm cận ngang của đồ thị hàm số suy ra đồ thị hàm số có hai đường tiệm cận.
            \itemch Đồ thị hàm số không có tiệm cận xiên.
        \end{itemchoice}
    }
\end{ex}
\begin{ex}%[EX-TF-2024, Lê Đạt]%[2D1H4-3]
    \immini{Cho hàm số $y=f(x)$ có đồ thị như hình bên. Xét tính đúng sai của các khẳng định sau
        \choiceTF
        {\True $ x=0 $ là một đường tiệm cận đứng của đồ thị hàm số}
        {$ y=-x $ là một đường tiệm cận xiên của đồ thị hàm số}
        {\True $ y=x $ là một đường tiệm cận xiên của đồ thị hàm số}
        {Đồ thị hàm số có ba đường tiệm cận}
    }{
        \begin{tikzpicture}[scale=.9, font=\footnotesize, line join=round, line cap=round,>=stealth]
            \def\a{0} \def\b{1} \def\c{1} \def\d{-1} % Hệ số
            \def\xmin{-3} \def\xmax{3.5}
            \def\ymin{-2.8} \def\ymax{3.3}
            \draw[color=gray!50,dashed] (\xmin,\ymin) grid (\xmax,\ymax);
            \draw[->] (\xmin,0)--(\xmax,0) node [below]{$x$};
            \draw[->] (0,\ymin)--(0,\ymax) node [left]{$y$};
            \fill (0,0) circle(1pt) node[shift=(-45:0.25)]{$O$};
            \clip (\xmin+0.1,\ymin+0.1) rectangle (\xmax-0.1,\ymax-0.1);
            \draw[smooth,samples=300,domain=-3:3] plot(\x,{\x+1/(7*\x)});
            \draw[dashed,smooth,samples=300,domain=-3:3] plot(\x,{\x});
            %	\fill (-1,0) circle (1.0pt) node[below]{$-1$} (1,0) circle (1.0pt) node[below right]{$1$};
    \end{tikzpicture}}
    \loigiai{
        \begin{itemchoice}
            \itemch $ x=0 $ là một đường tiệm cận đứng của đồ thị hàm số.
            \itemch	$ y=x $ là một đường tiệm cận xiên của đồ thị hàm số.
            \itemch $ y=x $ là một đường tiệm cận xiên của đồ thị hàm số.
            \itemch Đồ thị hàm số có $ x=0 $ là tiệm cận đứng và $ y=x $ là tiệm cận xiên nên có hai tiệm cận.
        \end{itemchoice}
    }
\end{ex}
\BTTL
\begin{ex}%[2D1K4-2]
    Nếu đồ thị hàm số $y=\dfrac{(m+1)x+2}{x-n+1}$ lần lượt nhận trục hoành và trục tung làm đường đường tiệm cận ngang và tiệm cận đứng thì $m+n$ bằng bao nhiêu?
    \shortans{$0$}
    \loigiai{
        Theo đề bài, ta có $\heva{&m+1=0\\&n-1=0} \Leftrightarrow \heva{&m=-1\\&n=1.}$\\
        Suy ra $m+n=0$.
    }
\end{ex}
\begin{ex}%[2D1K4-2]
    Tìm giá trị của $m$ để đồ thị hàm số $y=\dfrac{(2m+1)x+3}{x+1}$ có đường đường tiệm cận đi qua điểm $A(-2;7)$.
    \shortans{m=3}
    \loigiai{
        Từ đề bài, suy ra $2m+1=7 \Leftrightarrow m=3$.\\
        Suy ra $m+n=0$.
    }
\end{ex}
\begin{ex}%[2D1K4-2]
    Cho hàm số $y=\dfrac{-3+mx}{x+n}$. Tìm giá trị của $m$ và $n$ để đồ thị hàm số đã cho có tiệm cận đứng $x=2$ và tiệm cận ngang $y=2$.
    \shortans{$m=2, n=-2$}
    \loigiai{
        Từ yêu cầu đề bài, suy ra $\heva{&m=2\\&-n=2} \Leftrightarrow \heva{&m=2\\&n=-2.}$
    }
\end{ex}
\begin{ex}%[2D1K4-2]
    Để đường tiệm cận đứng và tiệm cận ngang của đồ thị hàm số $y=\dfrac{mx+1}{2m+1-x}$ cùng với hai trục tọa độ tạo thành một hình chữ nhật có diện tích bằng $3$ thì giá trị của $m$ bằng bao nhiêu?
    \shortans{$1$ hay $-\dfrac{3}{2}$}
    \loigiai{
        Từ yêu cầu đề bài, suy ra $|-m| \cdot |2m+1|=3 \Leftrightarrow \hoac{&m=1\\&m=-\dfrac{3}{2}.}$
    }
\end{ex}
\begin{ex}%[2D1K4-2]%[Thầy Hải Toán]%Câu 2.
    Đường tiệm cận đứng và đường tiệm cận ngang của đồ thị hàm số $y=\dfrac{mx+1}{2m+1-x}$ cùng với hai trục tọa độ tạo thành một hình chữ nhật có diện tích bằng $3$. Tính giá trị của $m$.
    \shortans{$m=1$; $m=-\dfrac{3}{2}$}
    \loigiai{
        Ta có $\lim\limits_{x\to+\infty}\dfrac{mx+1}{2m+1-x}=-m$; $\lim\limits_{x\to(2m+1)^+}\dfrac{mx+1}{2m+1-x} =\lim\limits_{x\to(2m+1)^+}\dfrac{m(2m+1)+1}{2m+1-x} =\lim\limits_{x\to(2m+1)^+}\dfrac{2m^2+m+1}{2m+1-x}$
        $\lim\limits_{x\to(2m+1)^+}\left(2m^2+m+1\right)=2m^2+m+1>0$; $\lim\limits_{x\to(2m+1)^+}(2m+1-x)=0$ và $2m+1-x<0\forall x>2m+1$ \\
        $ \Rightarrow\lim\limits_{x\to(2m+1)^+}\dfrac{mx+1}{2m+1-x}=-\infty $.\\
        Vậy đồ thị hàm số có hai đường tiệm cận $x=2m+1$ và $y=-m$.\\
        Hai đường tiệm cận tạo với hai trục tọa độ một hình chữ nhật có diện tích bằng $3$ suy ra $|2m+1|\cdot|m|=3\Leftrightarrow\hoac{&2m^2+m=3\\&2m^2+m=-3(PTVN)}\Leftrightarrow 2m^2+m-3=0\Leftrightarrow\hoac{&m=1\\&m=\dfrac{-3}{2}}$.}
\end{ex}
\begin{ex}%[KSCL L1, THPT Nhã Nam - Bắc Giang, 2019]%[Phạm An Bình, 12EX3]%[2D1K4-2]%
    Biết rằng đồ thị của hàm số $y=\dfrac{(n-3)x+n-2017}{x+m+3}$ ($m$, $n$ là tham số) nhận trục hoành làm tiệm cận ngang và trục tung làm tiệm cận đứng. Tính tổng $m-2n$.
    \shortans{$-9$}
    \loigiai{
        $\bullet$ $\lim\limits_{x\to +\infty}y=\lim\limits_{x\to +\infty}\dfrac{n-3+\dfrac{n-2017}{x}}{1+\dfrac{m+3}{x}} =n-3$.\\
        Vì đồ thị nhận trục hoành làm tiệm cận ngang nên $n-3=0\Leftrightarrow n=3$.\\
        $\bullet$ Vì đồ thị hàm số nhận trục tung làm tiệm cận đứng nên $\heva{&n-2017\ne 0\\&m+3=0}\Leftrightarrow \heva{&n\ne 2017\\&m= -3.}$\\
        Vậy $m-2n=-9$.
    }
\end{ex}
\begin{ex}%[TT Nguyễn Đăng Đạo, Bắc Ninh, lần 3, đề 152, 2018]%[2D1K4-2]%[Nguyễn Vân Trường, 12EX-8]%
    Tìm $m$ để tiệm cận đứng của đồ thị hàm số $y = \dfrac{m^2x-4m}{2x-m^2}$ đi qua điểm $A(2;1)$.
    \shortans{$m=-2$}
    \loigiai{
        Để hàm số có tiệm cận đứng thì \\
        $\hoac{& m \ne 0 \\ & m^2\cdot \dfrac{m^2}{2} - 4m \ne 0} \Leftrightarrow \hoac{& m \ne 0 \\ & m(m^3-8) \ne 0} \Leftrightarrow \hoac{& m \ne 0 \\ & m \ne 2}.$\\
        Khi đó tiệm cận đứng của hàm số là $x = \dfrac{m^2}{2}.$ Theo giả thiết ta có $ \dfrac{m^2}{2} = 2 \Leftrightarrow \hoac{& m =2 \text{ (loại)} \\ & m=-2 \text{ (thỏa mãn).}}$ Vậy $m=-2$.
    }
\end{ex}
\begin{ex}%[TT, Chuyên Lê Quý Đôn, Lai Châu, 2018]%[2D1K4-2]%[Nguyễn Tiến Thùy, 12EX-8]%
    Tìm $m$ để đồ thị hàm số $y=\dfrac{(m+1)x-5m}{2x-m}$ có tiệm cận ngang là đường thẳng $y=1$.
    \shortans{$m=1$}
    \loigiai{
        Ta có $\lim\limits_{x\rightarrow \pm\infty}f(x)=\lim\limits_{x\rightarrow \pm\infty}\dfrac{(m+1)x-5m}{2x-m}=\dfrac{m+1}{2}$, suy ra $y=\dfrac{m+1}{2}$ là tiệm cận ngang.\\
        Theo bài ra ta có $y=\dfrac{m+1}{2}=1\Leftrightarrow m=1$.
    }
\end{ex}
\begin{ex}%[2D1K4-1]
    Tìm tất cả các đường tiệm cận ngang của đồ thị hàm số $y=\dfrac{\sqrt{4x^2-x+1}}{2x+1}$.
    \shortans{$y=1$ và $y=-1$}
    \loigiai{
        Điều kiện xác định $x \ne \dfrac{-1}{2}$.\\
        Ta có $\lim\limits_{x \to +\infty} \dfrac{\sqrt{4x^2-x+1}}{2x+1}=\lim\limits_{x \to +\infty} \dfrac{|2x|\sqrt{1-\dfrac{1}{4x}+\dfrac{1}{4x^2}}}{2x\left(1+\dfrac{1}{2x}\right)}=1$.\\
        $\lim\limits_{x \to -\infty} \dfrac{\sqrt{4x^2-x+1}}{2x+1}=\lim\limits_{x \to -\infty} \dfrac{|2x|\sqrt{1-\dfrac{1}{4x}+\dfrac{1}{4x^2}}}{2x\left(1+\dfrac{1}{2x}\right)}=-1$.\\
        Tiệm cận ngang của đồ thị hàm số là $y= \pm 1$.
    }
\end{ex}
\begin{ex}%[2D1K4-1]
    Đồ thị hàm số $y=\dfrac{1-\sqrt{x^2+x+1}}{x^3+1}$ có tất cả bao nhiêu tiệm cận đứng và ngang?
    \shortans{$1$}
    \loigiai{
        Tập xác định $\mathscr{D}=\mathbb{R} \setminus \{-1\}$.
        \begin{itemize}
            \item
            {\allowdisplaybreaks
                \begin{eqnarray*}
                    \lim\limits_{x\to -1} \dfrac{1-\sqrt{x^2+x+1}}{x^3+1}&=&\lim\limits_{x\to -1} \dfrac{-x(x+1)}{(x+1)\left(x^2-x+1\right)\left(1+\sqrt{x^2+x+1}\right)}\\
                    &=&\lim\limits_{x\to -1} \dfrac{-x}{\left(x^2-x+1\right)\left(1+\sqrt{x^2+x+1}\right)}\\
                    &=&\dfrac{1}{6}.
            \end{eqnarray*} }
            \item $\lim\limits_{x\to +\infty}\dfrac{1-\sqrt{x^2+x+1}}{x^3+1}=0$.
        \end{itemize}
        Đồ thị hàm số không có tiệm cận đứng, tiệm cận ngang là $y=0$.
    }
\end{ex}
\begin{ex}%[2D1K4-1]
    Đồ thị hàm số $y=\dfrac{|x|}{\sqrt{x^2-1}}$ có tất cả bao nhiêu tiệm cận đứng và ngang?
    \shortans{$3$}
    \loigiai{
        Tập xác định $\mathscr{D}=(-\infty;-1) \cup (1;+\infty)$.
        \begin{itemize}
            \item $\lim\limits_{x\to-1^-}\dfrac{|x|}{\sqrt{x^2-1}}=+\infty$.
            \item $\lim\limits_{x\to 1^+}\dfrac{|x|}{\sqrt{x^2-1}}=+\infty$.
            \item $\lim\limits_{x\to +\infty}\dfrac{|x|}{\sqrt{x^2-1}}=1$.
            \item $\lim\limits_{x\to -\infty}\dfrac{|x|}{\sqrt{x^2-1}}=1$.
        \end{itemize}
        Đồ thị hàm số có $2$ tiệm cận đứng là $x=\pm 1$, tiệm cận ngang là $y=1$.
    }
\end{ex}
\begin{ex}%[2D1K4-1]
    Đồ thị hàm số $y=\dfrac{x}{\sqrt{x^2+1}}$ có tất cả bao nhiêu tiệm cận đứng và ngang?
    \shortans{$2$}
    \loigiai{
        Tập xác định $\mathscr{D}=\mathbb{R}$.
        \begin{itemize}
            \item $\lim\limits_{x\to +\infty}\dfrac{x}{\sqrt{x^2+1}}=1$.
            \item $\lim\limits_{x\to -\infty}\dfrac{x}{\sqrt{x^2+1}}=-1$.
        \end{itemize}
        Đồ thị hàm số không có tiệm cận đứng, tiệm cận ngang là $y=\pm 1$.
    }
\end{ex}
\begin{ex}%[2D1K4-1]
    Đồ thị hàm số $y=\dfrac{\sqrt{x^2-4}}{x^2-5x+6}$ có tất cả bao nhiêu tiệm cận đứng và ngang?
    \shortans{$3$}
    \loigiai{
        Tập xác định $\mathscr{D}=(-\infty;-2] \cup (2;+\infty) \setminus \{3\}$.
        \begin{itemize}
            \item $\lim\limits_{x\to 2^+}\dfrac{\sqrt{x^2-4}}{x^2-5x+6}=-\infty$.
            \item $\lim\limits_{x\to -2^-}\dfrac{\sqrt{x^2-4}}{x^2-5x+6}=-\infty$.
            \item $\lim\limits_{x\to +\infty}\dfrac{\sqrt{x^2-4}}{x^2-5x+6}=0$.
            \item $\lim\limits_{x\to -\infty}\dfrac{\sqrt{x^2-4}}{x^2-5x+6}=0$.
        \end{itemize}
        Đồ thị hàm số có $2$ tiệm cận đứng là $x=\pm 2$, tiệm cận ngang là $y=0$.
    }
\end{ex}
\begin{ex}
    Nồng độ thuốc trong máu $C(t)$ sau $t$ giờ khi uống một liều thuốc có thể được mô tả bởi hàm $C(t) = \dfrac{3}{1 + 2t}$. Tìm đường tiệm cận của nồng độ thuốc khi thời gian tăng lên rất lớn.
    \shortans{$0$}
\end{ex}
\begin{ex}
    Tốc độ (km/h) của một chiếc xe hơi tăng theo thời gian được mô tả bởi hàm $ v(t) = \dfrac{120t}{3+ t}$. Tìm đường tiệm cận của tốc độ khi thời gian tăng lên rất lớn.
    \shortans{$120$}
\end{ex}
\begin{ex}%[TeX hóa SGK CTST 12]%[Phạm Phương]%[2D1V4-4]
    Nồng độ oxygen trong hồ theo thời gian $t$ cho bởi công thức $y(t)=5-\dfrac{15t}{9t^{2}+1}$, với $y$ được tính theo mg/l và $t$ được tính theo giờ, $t \geq 0$. Tìm các đường tiệm cận của đồ thị hàm số $y(t)$. Từ đó, có nhận xét gì về nồng độ oxygen trong hồ khi thời gian $t$ trở nên rất lớn?
    \shortans{$y=5$, nồng độ tiến về $5$ mg/l}
    \loigiai{
        Hàm số $y(t)=5-\dfrac{15t}{9t^{2}+1}$ có tập xác định $\mathscr{D}=\mathbb{R}$.\\
        Ta có $\lim\limits_{x \rightarrow+\infty} \left(5-\dfrac{15t}{9t^{2}+1}\right)=5$.\\
        Vậy đồ thị hàm số có tiệm cận ngang là đường thẳng $y=5$.\\
        Khi thời gian $t$ trở nên rất lớn thì nồng độ oxygen trong hồ sẽ tiến dần về $5$ mg/l.
    }
\end{ex}
\begin{ex}
    Mô hình phát triển số lượng lợi khuẩn $P(t)$ theo thời gian có thể được mô tả bởi hàm $P(t) = \dfrac{100}{1 + 5e^{-2t}}$. Tính số lượng lợi khuẩn khi thời gian tăng lên rất lớn.\\
    \shortans{$100$}
\end{ex}
\begin{ex}
    Đáp ứng xung của một hệ thống điện tử the thời gian $t$ được mô tả bởi hàm \( h(t) = 120 e^{-\sqrt{3}t} \sin(2 t + \pi) \). Tìm và nêu ý nghĩa của đường tiệm cận của đáp ứng xung khi thời gian tăng.
    \shortans{$0$}
\end{ex}
\begin{ex}
    Điện áp của pin sạc theo thời gian được mô tả bởi hàm \( V(t) = 220 \left(1 - e^{-\dfrac{t}{\tau}}\right) \), trong đó \( \tau \) là hằng số thời gian. Tìm và nêu ý nghĩa của đường tiệm cận của điện áp khi thời gian tăng.
    \shortans{$220$}
\end{ex}
\begin{ex}%[0D1K1-4]
    Số lượng sản phẩm bán được của một công ty trong $x$ (tháng) được tính theo công thức $S(x)=200\left(5-\dfrac{9}{2+x}\right)$, trong đó $x\ge 1$ \emph{(Nguồn: R.Larson and B.Edwards, Calculus 10e, Cengage 2014).}
    \begin{enumerate}[a)]
        \item Xem $y=S(x)$ là một hàm số xác định trên nửa khoảng $[1;+\infty)$, hãy tìm tiệm cận ngang của đồ thị hàm số đó.
        \item Nêu nhận xét về số lượng sản phẩm bán được của công ty trong $x$ (tháng) khi $x$ đủ lớn.
    \end{enumerate}
    \shortans{$y=1$, sản phẩm gần $1\,000$}
    \loigiai{
        \begin{enumerate}[a)]
            \item Ta có $\lim\limits_{x\to +\infty}\left[200\left(5+\dfrac{9}{2-x}\right)\right]=200\cdot 5=1000$.\\
            Vậy $y=1\,000$ là tiệm cận ngang của đồ thị hàm số $y=S(x)$.
            \item Từ phần trên ta có thể rút ra nhận xét: khi số tháng đủ lớn thì công ty có thể bán được số sản phẩm gần bằng $1\,000$.
        \end{enumerate}
    }
\end{ex}
\begin{ex}
    Công ty cung cấp dịch vụ internet tính $75\$$ phí lắp đặt thiết bị ban đầu và phí sử dụng internet $40\$$ mỗi tháng
    \begin{listEX}
        \item Lập hàm số thể hiện chi phí sử dụng trung bình mỗi tháng sau $x$ tháng sử dụng
        \item Chi phí sử dụng trung bình thay đổi thế nào khi số tháng sử dụng tăng lên rất nhiều.
    \end{listEX}
    \shortans{$y=\dfrac{40x+75}{40}$, chi phí tiến về $40\$$}
\end{ex}
\begin{ex}
    Nhà trường dự định tổ chức tiệc liên hoan chào mừng lớp 12, tiền thuê hội trường là $1$ tỷ. Cứ mỗi người tham gia sẽ tính thêm phí phục vụ là $2$ triệu mỗi người. Gọi $x$ là số người tham gia bữa tiệc
    \begin{listEX}
        \item Lập hàm số thể hiện tổng chi phí của bữa tiệc
        \item Lập hàm số thể hiện chi phí trung bình của mỗi người bỏ ra cho bữa tiệc
        \item Chi phí trung bình của mỗi người thay đổi thế nào khi số người tham gia tăng lên rất lớn.
    \end{listEX}
    \shortans{$y=\dfrac{0,02x+1}{x}$, tiến về $2$ triệu}
\end{ex}
\begin{ex}
    Số lượng vi khuẩn trong một môi trường dinh dưỡng có thể được mô tả bởi hàm:
    \[
    N(t) = \dfrac{N_0}{1 - \dfrac{t}{T}}
    \]
    trong đó \( N(t) \) là số lượng vi khuẩn tại thời gian \( t \), \( N_0 \) là số lượng vi khuẩn ban đầu, và \( T \) là thời gian mà môi trường dinh dưỡng không còn đủ để hỗ trợ sự tăng trưởng của vi khuẩn. Xác định tiệm cận đứng của hàm số này và nêu ý nghĩa của nó.
    \shortans{$t=T$, khi $t$ tiến về $T$ thì số lượng vi khuẩn tăng lên vô hạn}
    \loigiai{
        Để tìm tiệm cận đứng, ta xem xét các giá trị của \( t \) làm cho mẫu số của phương trình bằng 0:
        \[
        1 - \frac{t}{T} = 0 \Rightarrow t = T
        \]
        Vậy đường thẳng \( t = T \) là tiệm cận đứng của đồ thị hàm số.
        \textbf{Ý nghĩa:} Từ đó ta suy ra khi thời gian \( t \) càng sát với \( T \), số lượng vi khuẩn \( N(t) \) sẽ tăng lên vô hạn. Điều này có nghĩa là khi thời gian tiếp cận \( T \) thì số lượng vi khuẩn sẽ tăng lên nhanh chóng đến mức vô hạn.}
\end{ex}
\begin{ex}
    Trong vật lý, vận tốc tối đa \(V\) của một vật rơi qua một chất lỏng được mô tả bằng phương trình:
    \[
    V(t) = \frac{mg}{b} \left(1 - e^{-\dfrac{bt}{m}}\right)
    \]
    trong đó \(m\) là khối lượng của vật, \(g\) là gia tốc trọng trường, \(b\) là hệ số ma sát, và \(t\) là thời gian. Xác định tiệm cận đứng của hàm số này và nêu ý nghĩa của nó.
    \shortans{Không có TCĐ}
    \loigiai{Để tìm tiệm cận đứng, ta xem xét các giá trị của \(t\) làm cho mẫu số của phương trình bằng 0. Tuy nhiên, trong trường hợp này, hàm số không có tiệm cận đứng vì biểu thức mũ đảm bảo hàm số được xác định cho tất cả các số thực.
        \textbf{Ý nghĩa:} Điều này ngụ ý rằng không có giới hạn về thời gian để vật đạt đến vận tốc tối đa. Khi thời gian tăng lên, vận tốc của vật sẽ tiệm cận đến vận tốc tối đa, nhưng vật không bao giờ thực sự đạt được nó.}
\end{ex}
\begin{ex}
    Trong sinh học, sự tăng trưởng của dân số \(P\) theo thời gian \(t\) có thể được mô hình bằng hàm số:
    \[
    P(t) = \frac{P_0}{1 - kP_0t}
    \]
    trong đó \(P_0\) là kích thước dân số ban đầu và \(k\) là hằng số tốc độ tăng trưởng. Xác định tiệm cận đứng của hàm số này và nêu ý nghĩa của nó.
    \shortans{$t = \frac{1}{kP_0}$}
    \loigiai{Để tìm tiệm cận đứng, ta xem xét các giá trị của \(t\) làm cho mẫu số của phương trình bằng 0:
        \[
        1 - kP_0t = 0 \Rightarrow t = \frac{1}{kP_0}
        \]
        Vậy tiệm cận đứng là \(t = \frac{1}{kP_0}\).
        \textbf{Ý nghĩa:} Điều này ngụ ý rằng hàm số tăng trưởng dân số có một tiệm cận đứng tại thời điểm \(t\) bằng nghịch đảo của tích của hằng số tốc độ tăng trưởng \(k\) và kích thước dân số ban đầu \(P_0\). Điều này chỉ ra một giới hạn cho tốc độ tăng trưởng dân số theo thời gian.}
\end{ex}
\begin{ex}
    Trong khoa học máy tính, độ phức tạp thời gian \(T(n)\) của một thuật toán với kích thước đầu vào \(n\) có thể được biểu diễn bằng hàm số:
    \[
    T(n) = \frac{an^2 + bn + c}{n}
    \]
    trong đó \(a\), \(b\), và \(c\) là các hằng số. Xác định tiệm cận đứng của hàm số này và nêu ý nghĩa của nó.
    \shortans{$T=0$}
    \loigiai{Để tìm tiệm cận đứng, ta xem xét các giá trị của \(n\) làm cho mẫu số của phương trình bằng 0:
        \[
        n = 0
        \]
        Vậy tiệm cận đứng là \(n = 0\).
        \textbf{Ý nghĩa:} Điều này ngụ ý rằng hàm số độ phức tạp thời gian không có tiệm cận đứng. Trong phân tích tính toán, một tiệm cận đứng tại \(n = 0\) sẽ ngụ ý rằng thuật toán có độ phức tạp thời gian vô hạn cho các đầu vào có kích thước bằng 0, điều này không có ý nghĩa trong hầu hết các trường hợp.}
\end{ex}
\Closesolutionfile{ans}
\begin{dang}{Đường tiệm cận liên quan tham số $m$}
\end{dang}
\begin{vd}
    Tìm $m$ để đồ thị hàm số
    \begin{listEX}[2]
        \item $y=\dfrac{x-2}{x^2-mx+1}$ có hai đường tiệm cận đứng.
        \item $y=\dfrac{x-1}{x^2-mx+1}$ có đúng ba đường tiệm cận.
        \item $y=\dfrac{\sqrt{x-3}}{x^2+x-m}$ có đúng hai đường tiệm cận.
        \item $y=\dfrac{\sqrt{1-x}}{x^2+4x+m}$ có ba đường tiệm cận.
        %	\item* $y=\dfrac{x}{x^2-2(m+1)x+m^2}$ có đúng hai đường tiệm cận.
    \end{listEX}
    \loigiai{}
\end{vd}
\BTTN
\Opensolutionfile{ans}[ans/2D1-4-DANG-2]
\begin{ex}%[Phạm Văn Long]%[Latex-HK2-TT-2020-2021]%[2D1K4-2]%
    Tìm $m$ để đồ thị hàm số $y=\dfrac{2x^2-3x+4}{x^2+mx+1}$ có duy nhất một đường tiệm cận?
    \choice
    {\True $m\in (-2;2)$}
    {$m\in [-2;2]$}
    {$m\in \{-2;2\}$}
    {$m\in (2;+\infty)$}
    \loigiai{
        Ta thấy đồ thị hàm số đã cho luôn có một tiệm cận ngang là đường $y=2$.\\
        Do đó, để đồ thị hàm số đã cho có duy nhất một đường tiệm cận thì đồ thị hàm số đã cho không có tiệm cận đứng.\\
        $\Rightarrow$ Phương trình $x^2+mx+1=0$ vô nghiệm $\Leftrightarrow \Delta <0 \Leftrightarrow m^2-4<0\Leftrightarrow m\in (-2;2)$.
    }
\end{ex}
\begin{ex}%[2D1K4-2]%[Đề GHK1, THPT Trần Nhân Tông, Hà Nội 2018]%[WTT2D1-128]%
    Tìm giá trị thực của tham số $m$ để đồ thị hàm số $y=\dfrac{x-4}{m-x^2}$ có đường tiệm cận đứng.
    \choice
    {$m\ge0;\,m\ne16$}
    {\True $m\ge0$}
    {$m>0$}
    {$m>0;\,m\ne16$}
    \loigiai{
        Điều kiện xác định: $m-x^2\neq0$.\\
        Để đồ thị hàm số có đường tiệm cận đứng thì phương trình $m-x^2=0$ có nghiệm, tức là $m\ge0$.\\
        Với $m=16$ thì $y=\dfrac{-1}{4+x}$ có một tiệm cận đứng là $x=-4$. Vậy giá trị $m$ cần tìm là $m\ge0$.
    }
\end{ex}
\begin{ex}%[2D1K4-2]%
    Có bao nhiêu giá trị của tham số $m$ thoả mãn đồ thị hàm số $y=\dfrac{x+3}{x^2-x-m}$ có đúng hai đường tiệm cận?
    \choice
    {$1$}
    {$4$}
    {\True $2$}
    {$3$}
    \loigiai{
        Đồ thị hàm số có đúng hai đường tiệm cận khi phương trình $x^2-x-m=0$ có nghiệm kép hoặc có hai nghiệm phân biệt với một nghiệm bằng $-3$. Khi đó
        \[\hoac{&\Delta=0\\&\heva{&\Delta>0\\&g(-3)=0}}\Leftrightarrow\hoac{&4m+1=0\\&\heva{&4m+1>0\\&m=12}}\hoac{&m=-\dfrac{1}{4}\\&m=12.}\]
        Vậy có hai giá trị của m.}
\end{ex}
\begin{ex}%[2D1K4-2]%[Đề kiểm tra giữa học kì I, 2017 - 2018 trường THPT Chu Văn An, Hà Nội]%[WTT2D1-156]%
    Tìm tất cả các giá trị thưc của tham số $m$ để đồ thị hàm số $y=\dfrac{x^2+m}{x^2-3x+2}$ có đúng hai tiệm cận.
    \choice
    {$m=-1$}
    {$m\in\left\{1;4\right\}$}
    {\True $m\in\left\{-1;-4\right\}$}
    {$m=4$}
    \loigiai{
        Vì $\lim\limits_{x\to\pm\infty}\dfrac{x^2+m}{x^2-3x+2}=1,\,\forall m$ nên đồ thị hàm số luôn có một tiêm cận ngang là $y=1$.\\
        Để đồ thị hàm số có đúng hai tiệm cận thì đồ thị hàm số có thêm một tiệm cận đứng là $x=1$ hoặc là $x=2$.
        \begin{itemize}
            \item Đồ thị hàm số có một tiệm cận đứng $x=1$, suy ra pt $x^2+m=0$ và phương trình $x^2-3x+2=0$ có nghiệm chung là $x=1\Rightarrow m=-1$.
            \item Đồ thị hàm số có một tiệm cận đứng $x=2$, suy ra pt $x^2+m=0$ và phương trình $x^2-3x+2=0$ có nghiệm chung là $x=2\Rightarrow m=-4$.
        \end{itemize}
        Vậy $m\in\left\{-1;4\right\}$ thỏa yêu cầu bài toán.
    }
\end{ex}
\begin{ex}%[Thi thử, THPT Lục Ngạn - Bắc Giang, 2019]%[Trần Như Ngọc, 12EX3-2019]%[2D1K4-2]%
    Có bao nhiêu giá trị nguyên dương của tham số $m$ để đồ thị hàm số
    $y=\dfrac{\sqrt{9-x}}{x^2-2(m+1)x+m^2+2m}$
    có đúng hai đường tiệm cận.
    \choice
    {\True $2$}
    {$1$}
    {$4$}
    {$3$}
    \loigiai{
        Ta có $ x^2-2(m+1)x+m^2+2m = 0 \Leftrightarrow \hoac{& x=m \\ & x=m+2}$
        $( \Delta ' = 1 )$. \\
        Hàm số xác định khi $ \heva{& x \le 9 \\ & x \ne m \\ & x \ne m+2.} $\\
        Ta có $\lim \limits_{x\to -\infty}y = 0$ nên đồ thị hàm số có một tiệm cận ngang là $ y = 0 $.\\
        Đồ thị hàm số có đúng hai tiệm cận khi và chỉ khi nó có đúng một tiệm cận đứng \\
        $\Leftrightarrow$ phương trình trên có một nghiệm nhỏ hơn hoặc bằng $ 9 $.\\
        $\Leftrightarrow m \le 9 < m+2 \Leftrightarrow 7 < m \le 9 $.\\
        Vậy có $ 2 $ giá trị $ m $ nguyên dương thỏa mãn điều kiện bài toán.
    }
\end{ex}
\begin{ex}%[Thi học kỳ I, Trường THPT Chuyên Lê Quý Đôn - Khánh Hòa, 2021]%[Lê Hồng Phi, 12EX5]%[2D1K4-2]%
    Cho hàm số $y=\dfrac{2x-3}{\sqrt{x^2+2(m-2)x+m^2}}$ với $m$ là tham số thực và $m>1$. Hỏi đồ thị hàm số có bao nhiêu đường tiệm cận (tiệm cận ngang và tiệm cận đứng)?
    \choice
    {$1$}
    {\True $2$}
    {$3$}
    {$4$}
    \loigiai
    {Phương trình $x^2+2(m-2)x+m^2=0$ có $\Delta'=(m-2)^2-m^2=-2(2m-2)=-4(m-1)<0,\ \forall m>1$ nên vô nghiệm.\\
        Do đó tập xác định của hàm số là $\mathscr{D}=\mathbb{R}$.\\
        Như thế đồ thị hàm số không có đường tiệm cận đứng.\\
        Ta tính được
        \begin{itemize}
            \item $\lim\limits_{x\to +\infty}y=\lim\limits_{x\to +\infty}\dfrac{2-\dfrac{3}{x}}{\sqrt{1+\dfrac{2(m-2)}{x}+\dfrac{m^2}{x}}}=2$ nên $y=2$ là đường tiệm cận ngang.
            \item $\lim\limits_{x\to -\infty}y=\lim\limits_{x\to -\infty}\dfrac{2-\dfrac{3}{x}}{-\sqrt{1+\dfrac{2(m-2)}{x}+\dfrac{m^2}{x}}}=-2$ nên $y=-2$ là đường tiệm cận ngang.
        \end{itemize}
        Vậy đồ thị hàm số đã cho có $2$ đường tiệm cận.
    }
\end{ex}
\begin{ex}%[Đề Khảo sát lần 1 THPT Quang Hà - Vĩnh Phúc, 2021]%[Trần Nhân Kiệt, 12EX4-2021]%[2D1K4-2]%
    Có bao nhiêu giá trị nguyên của tham số $m$ để đồ thị tham số $y=\dfrac{1+\sqrt{x+1}}{\sqrt{x^2-(1-m)x+2m}}$ có hai tiệm cận đứng?
    \choice
    {$2$}
    {\True $3$}
    {$1$}
    {$0$}
    \loigiai{
        Điều kiện $\heva{& x\ge -1 \\ & x^2-(1-m)x+2m>0.}$\\
        Đồ thị hàm số có hai tiệm cận đứng khi và chỉ khi phương trình $x^2-(1-m)x+2m=0$ có hai nghiệm phân biệt lớn hơn hoặc bằng $-1$.\\
        Ta có $x^2-(1-m)x+2m=0\Leftrightarrow x^2-x+m(x+2)=0\Leftrightarrow m=\dfrac{-x^2+x}{x+2}$.\\
        Đặt $f(x)=\dfrac{-x^2+x}{x+2}$, $x\ge -1$.\\
        Ta có $f'(x)=\dfrac{-x^2-4x+2}{(x+2)^2}$, suy ra $f'(x)=0\Leftrightarrow -x^2-4x+2=0\Leftrightarrow x=-2\pm \sqrt{6}$.
        \begin{center}
            \begin{tikzpicture}[>=stealth]
                \tkzTabInit[nocadre=false,lgt=1.2,espcl=3,deltacl=0.5]
                {$x$/.7 ,$f'(x)$/.7,$f(x)$/2}
                {$-1$ , $-2+\sqrt{6}$ , $+\infty$}
                \tkzTabLine{ , - , $0$ , + , }
                \tkzTabVar{-/$-2$ , +/$5-2\sqrt{6}$ , -/$-\infty$}
            \end{tikzpicture}
        \end{center}
        Từ bảng biến thiên suy ra $m\in [-2;5-2\sqrt{6})$.\\
        Vì $m$ nguyên nên $m\in \{-2;-1;0\}$.\\
        Vậy có $3$ giá trị nguyên của $m$ thỏa mãn bài.
    }
\end{ex}
\BTTL
\begin{ex}%[2D1K4-2]%
    Cho hàm số $y=\dfrac{2x^2-3x+m}{x-m}$ có đồ thị $(C)$. Với tất cả các giá trị thực nào của tham số $m$ thì đồ thị $(C)$ không có tiệm cận đứng?
    \shortans{$m=0$ hoặc $m=1$}
    %	\choice
    %	{$m=2$}
    %	{$m=0$}
    %	{$m=1$}
    %	{\True $m=0$ hoặc $m=1$}
    \loigiai{
        Đồ thị không có tiệm cận đứng khi $x=m$ là nghiệm của phương trình $2x^2-3x+m=0$, suy ra $2m^2-3m+m=0 \Leftrightarrow \hoac{&m=0\\&m=1}$.
    }
\end{ex}
\begin{ex}%[2D1K4-2]%
    Với tất cả các giá trị thực nào của tham số $m$ thì đồ thị hàm số $y=\dfrac{x^2+x-2}{x^2+x+m}$ có ba đường tiệm cận?
    \shortans{$m<\dfrac{1}{4}$ và $m\ne -2$}
    %	\choice
    %	{$m>\dfrac{1}{4}$ và $m\ne 2$}
    %	{$m>\dfrac{1}{4}$}
    %	{$m<\dfrac{1}{4}$}
    %	{\True $m<\dfrac{1}{4}$ và $m\ne -2$}
    \loigiai{
        Đồ thị hàm số chỉ có $1$ tiệm cận ngang là $y=1$.\\
        Ta có $x^2+x-2 \Leftrightarrow \hoac{&x=1\\&x=-2.}$\\
        Đồ thị hàm số có ba đường tiệm cận khi và chỉ khi có $2$ tiệm cận đứng. Điều này tương đương với phương trình $x^2+x+m=0$ có $2$ nghiệm phân biệt khác $1$ và $-2$, nghĩa là\\
        $\heva{&1-4m>0\\&1^2+1+m \ne 0\\& (-2)^2-2+m\ne 0} \Leftrightarrow \heva{&m<\dfrac{1}{4}\\&m\ne -2.}$
    }
\end{ex}
\begin{ex}%[đề thi thử THPT Quốc gia, đề số 3, nguyễn hoàng thanh]%[2D1K4-2]%
    Tìm số giá trị nguyên thuộc đoạn $ [-2025;2025] $ của tham số $ m $ để đồ thị hàm số $ y=\dfrac{\sqrt{x-3}}{x^2+x-m} $ có đúng hai đường tiệm cận.
    \shortans{$2014$}
    %	\choice
    %	{$ 2007 $}
    %	{$ 2010 $}
    %	{$ 2009 $}
    %	{\True $ 2008 $}
    \loigiai{
        Điều kiện xác định của hàm số $ \heva{& x\ge 3\\& x^2+x-m\ne 0.} $\\
        Vì $ \lim\limits_{x\to +\infty}\dfrac{\sqrt{x-3}}{x^2+x-m}=\lim\limits_{x\to +\infty}\dfrac{\sqrt{\frac{1}{x}-\frac{3}{x^2 }}}{1+\frac{1}{x}-\frac{m}{x^2}}=0 $, suy ra $ y=0 $ là tiệm cận ngang.\\
        Để đồ thị hàm số có đúng hai tiệm cận thì đồ thị hàm số chỉ có thêm một tiệm cận đứng, tương đương $ f(x)=x^2+x-m $ có đúng một nghiệm lớn hơn $ 3 $. Xét các trường hợp xảy ra như sau
        \begin{enumerate}
            \item $ f(x)=0 $ có nghiệm kép $ x_{1}=x_2=-\dfrac{1}{2}<3 $ (không thỏa mãn).
            \item $ f(x)=0 $ có hai nghiệm thỏa $ x_1<3\le x_2\Leftrightarrow a\cdot f(3)\le 0\Leftrightarrow 12-m\le 0\Leftrightarrow m\ge 12 $.
        \end{enumerate}
        Kết hợp với yêu cầu bài toán ta suy ra $ \heva{&m\in \mathbb{Z}\\ &m\in[12;2025]} $, suy ra có $ 2025-12+1=2014 $ giá trị nguyên của $ m $ thỏa mãn bài toán.
    }
\end{ex}
\begin{ex}%[Đề tham khảo THPT Quốc gia 2021 - Đề 5]%[Đoàn Minh Tân]%[2D1K4-2]%
    Tìm tất cả giá trị thực của tham số $m$ để đồ thị hàm số $y=\dfrac{3x+2018}{\sqrt{mx^2+5x+6}}$ có hai đường tiệm cận ngang.
    \shortans{$m>0$}
    %	\choice
    %	{$m\in \varnothing$}
    %	{$m<0$}
    %	{$m=0$}
    %	{\True $m>0$}
    \loigiai{
        Ta có $\displaystyle \lim \limits_{x\to +\infty} y=\displaystyle\lim\limits_{x\to +\infty}\dfrac{3x+2018}{\sqrt{mx^2+5x+6}}=\displaystyle\lim\limits_{x\to +\infty}\dfrac{3+\dfrac{2018}{x}}{\sqrt{m+\dfrac{5}{x}+\dfrac{6}{x^2}}}=\dfrac{3}{\sqrt{m}}$ tồn tại khi $m>0$.\\
        $\displaystyle\lim\limits_{x\to -\infty}=\displaystyle\lim\limits_{x\to -\infty}\dfrac{3x+2018}{\sqrt{mx^2+5x+6}}=\displaystyle\lim\limits_{x\to -\infty}\dfrac{3+\dfrac{2018}{x}}{-\sqrt{m+\dfrac{5}{x}+\dfrac{6}{x^2}}}=-\dfrac{3}{\sqrt{m}}$ tồn tại khi $m>0$.\\
        Hiên nhiên $\displaystyle\lim\limits_{x\to +\infty}y\ne \displaystyle \lim \limits_{x\to -\infty}y$.\\
        Vậy đồ thị hàm số đã cho có hai tiệm cận ngang khi và chỉ khi $m>0$.
    }
\end{ex}
\begin{ex}%[2D1K4-2]%
    Có bao nhiêu giá trị nguyên của tham số thực $m$ thuộc đoạn $[-20; 10]$ để đồ thị hàm số $y=\dfrac{x+2}{\sqrt{x^2-4x+m}}$ có hai đường tiệm cận đứng?
    \shortans{$23$}
    %	\choice
    %	{$20$}
    %	{$21$}
    %	{$22$}
    %	{\True $23$}
    \loigiai{
        Đồ thị hàm số có hai đường tiệm cận đứng $\Leftrightarrow$ phương trình $x^2-4x+m=0$ có hai nghiệm phân biệt khác $-2$ \\
        $ \Leftrightarrow\heva{&2^2-m>0\\&(-2)^2-4\cdot (-2)+m\neq 0}\Leftrightarrow\heva{&m<4\\&m\neq-12.} $ \\
        Do $m$ nguyên và $m\in[-20; 10]$ nên $m\in\left\{-20;-19;\ldots;-13;-11;\ldots; 2; 3\right\}$, gồm $23$ giá trị thỏa mãn.}
\end{ex}
\Closesolutionfile{ans}
\begin{dang}{Tìm các đường tiệm cận đồ thị hàm ẩn}
\end{dang}
\begin{vd}
    Cho hàm số $y=f(x)$ có bảng biến thiên như hình vẽ sau
    \begin{center}
        \begin{tikzpicture}[>=stealth]
            \tkzTabInit[nocadre=false,lgt=1,espcl=1.5,deltacl=0.5]{$x$/.7 ,$y'$/.7,$y$/2}
            {$-\infty$ , $-1$ , $2$ , $+\infty$}
            \tkzTabLine{ , + , $0$ , - , d , + , }
            \tkzTabVar{-/$1$ , +/$4$ , -/$-5$ , +/$+\infty$}
        \end{tikzpicture}
    \end{center}
    Tìm TCĐ, TCN của đồ thị hàm số
    \begin{listEX}[3]
        \item $y=\dfrac{2}{f(x)-3}$
        \item $y=\dfrac{-3}{f(x)+2}$
        \item $y=\dfrac{x-2}{f(x)+5}$
        \item $y=\dfrac{x+1}{f(x)-4}$
        \item $y=\dfrac{2}{f(x^2)+3}$
        \item $y=\dfrac{4f(x)-5}{3f(x)+1}$
    \end{listEX}
    \loigiai{}
\end{vd}
\begin{vd}\immini{Cho hàm bậc ba $y=f(x)$ có đồ thị như hình vẽ. Tìm số tiệm cận đứng của đồ thị hàm số
        \begin{listEX}[2]
            \item $y=\dfrac{\sqrt{x+3}}{(x-1)f(x)}$
            \item $g(x)=\dfrac{(x^2+4x+3)\sqrt{x^2+x}}{x\left[f^2(x)-2f(x)\right]}$ .
    \end{listEX}}{\begin{tikzpicture}[line cap=round,line join=round, >=stealth,font=\footnotesize]
            \begin{scope}[scale=.5]
                \def\a{-1} % Hệ số a phải khác 0
                \def\b{-13/2}
                \def\c{-12}
                \def\d{-9/2}
                \draw[->] (-5,0) -- (2,0)node[below]{$x$};
                \draw[->] (0,-3) -- (0,4) node[left] {$y$};
                \draw (0,0)node[below right]{$O$} (-3,0)node[below]{$-3$};
                \draw[dashed] (-1,0)node[below]{$-1$}|-(0,2)node[right]{$2$};
                \draw[samples=150,smooth,domain=-4:.-.2] plot(\x,{\a*(\x)^3+(\b)*(\x)^2+(\c)*\x+(\d)});
            \end{scope}
    \end{tikzpicture}}
    \loigiai{
        \begin{center}
            \begin{tikzpicture}[line cap=round,line join=round, >=stealth,font=\footnotesize,scale=1]
                \def\a{-1} % Hệ số a phải khác 0
                \def\b{-13/2}
                \def\c{-12}
                \def\d{-9/2}
                \draw[->] (-5,0) -- (2,0)node[below]{$x$};
                \draw[->] (0,-3) -- (0,4) node[left] {$y$};
                \draw (0,0)node[below right]{$O$} (-3,0)node[below]{$-3$} (-.3,0)node[above]{$a$};
                \draw[dashed] (-3.78,0)node[below]{$c$}|-(0,2)|-(-1.71,0)node[below]{$b$}|-(0,2) (-1,0)node[below]{$-1$}|-(0,2)node[right]{$2$};
                \draw[samples=150,smooth,domain=-4:.-.2] plot(\x,{\a*(\x)^3+(\b)*(\x)^2+(\c)*\x+(\d)});
            \end{tikzpicture}
        \end{center}
        $g(x)=\dfrac{(x^2+4x+3)\sqrt{x^2+x}}{x\left[f^2(x)-2f(x)\right]}=\dfrac{(x+1)(x+3)\sqrt{x(x+1)}}{x\left[f^2(x)-2f(x)\right]}$.\\
        Điều kiện của căn là $x\le -1; x\ge 0$.\\
        Dựa vào đồ thị ta có \[x\left[f^2(x)-2f(x)\right]=0 \Leftrightarrow \hoac{&x=0\\&f(x)=0\\& f(x)=2} \Leftrightarrow \hoac{&x=0\text{ (nhận)}\\&x=-3\text{ (nhận)};\ x=a \text{ (loại)} \\&x=-1\text{ (nhận)};\ x=b\text{ (nhận)};\ x=c\text{ (nhận)}}\]\\
        Số TCĐ lúc này chính là số nghiệm không bị rút gọn của mẫu, vậy có bốn TCĐ là $x=0; x=-3; x=b; x=c$.
    }
\end{vd}
\BTTN
\Opensolutionfile{ans}[ans/2D1-4-DANG-3]
\begin{ex}%[2D1K4-1]
    Cho hàm số $y=f(x)$ có bảng biến thiên như hình bên. Đồ thị hàm số $y=\dfrac{-5}{f(x)+4}$ có bao nhiêu tiệm cận đứng?
    \begin{center}
        \begin{tikzpicture}[scale=0.8]
            \tkzTabInit[nocadre=false,lgt=1.5,espcl=3,deltacl=0.6]
            {$x$ /0.6,$y’$ /0.6,$y$ /2}
            {$-\infty$ ,$1$, $2$, $+\infty$}
            \tkzTabLine{,+,d,-,d,+,}
            \tkzTabVar{-/$-4$,+/$3$,-/$-5$,+/$+\infty$}
        \end{tikzpicture}
    \end{center}
    \choice
    {$1$}
    {$3$}
    {\True $2$}
    {$4$}
    \loigiai{
        Dựa vào bảng biến thiên suy ra
        $f(x)+4=0 \Leftrightarrow f(x) =-4$, phương trình này có $2$ nghiệm phân biệt nên đồ thị hàm số $y=\dfrac{-5}{f(x)+4}$ có $2$ tiệm cận đứng.
    }
\end{ex}
\begin{ex}%[2D1K4-1]
    Cho hàm số $y=f(x)$ có bảng biến thiên như hình bên. Đồ thị hàm số $y=\dfrac{x+2}{2f(x)-1}$ có bao nhiêu tiệm cận đứng?
    \begin{center}
        \begin{tikzpicture}[scale=0.8]
            \tkzTabInit[nocadre=false,lgt=1.5,espcl=3,deltacl=0.6]
            {$x$ /0.6,$y’$ /0.6,$y$ /2}
            {$-\infty$ ,$-1$, $0$, $1$, $+\infty$}
            \tkzTabLine{,+,0,-,0,+,0,-,}
            \tkzTabVar{-/$-\infty$,+/$0$,-/$-\dfrac{5}{3}$,+/$0$,-/$-\infty$}
        \end{tikzpicture}
    \end{center}
    \choice
    {$1$}
    {$3$}
    {$2$}
    {\True $0$}
    \loigiai{
        Dựa vào bảng biến thiên suy ra
        $2f(x)-1=0 \Leftrightarrow f(x) =\dfrac{1}{2}$, phương trình này có $0$ nghiệm nên đồ thị hàm số $y=\dfrac{x+2}{2f(x)-1}$ không có tiệm cận đứng.
    }
\end{ex}
%69
\begin{ex}%[2D1K4-1]
    Cho hàm số $y=f(x)$ có bảng biến thiên như hình bên. Đồ thị hàm số $y=\dfrac{1}{2f(x)-3}$ có bao nhiêu tiệm cận đứng?
    \begin{center}
        \begin{tikzpicture}[scale=0.8]
            \tkzTabInit[nocadre=false,lgt=1.5,espcl=3,deltacl=0.6]
            {$x$ /0.6,$y’$ /0.6,$y$ /2}
            {$-\infty$ ,$0$, $1$, $+\infty$}
            \tkzTabLine{,+,0,-,0,+,}
            \tkzTabVar{-/$-\infty$,+/$5$,-/$-1$,+/$+\infty$}
        \end{tikzpicture}
    \end{center}
    \choice
    {$1$}
    {\True $3$}
    {$2$}
    {$0$}
    \loigiai{
        Dựa vào bảng biến thiên suy ra
        $2f(x)-3=0 \Leftrightarrow f(x) =-\dfrac{3}{2}$, phương trình này có $3$ nghiệm phân biệt nên đồ thị hàm số $y=\dfrac{1}{2f(x)-3}$ có ba tiệm cận đứng.
    }
\end{ex}
%70
%71
%72
\begin{ex}%[2D1K4-1]
    Cho hàm số $y=f(x)$ có bảng biến thiên như hình bên. Đồ thị hàm số $y=\dfrac{x}{f(x)-3}$ có bao nhiêu tiệm cận đứng?
    \begin{center}
        \begin{tikzpicture}[scale=0.8]
            \tkzTabInit[nocadre=false,lgt=1.5,espcl=3,deltacl=0.6]
            {$x$ /0.6,$y’$ /0.6,$y$ /2}
            {$-\infty$ ,$-1$, $0$, $1$, $+\infty$}
            \tkzTabLine{,-,0,+,0,-,0,+,}
            \tkzTabVar{+/$+\infty$,-/$0$,+/$3$,-/$0$,+/$+\infty$}
        \end{tikzpicture}
    \end{center}
    \choice
    {$1$}
    {\True $3$}
    {$2$}
    {$4$}
    \loigiai{
        Dựa vào bảng biến thiên suy ra
        $f(x)-3=0 \Leftrightarrow f(x) =3$, phương trình này có $2$ nghiệm phân biệt khác $0$ và một nghiệm bội chẵn $x=0$ nên đồ thị hàm số $y=\dfrac{x}{f(x)-3}$ có ba tiệm cận đứng.
    }
\end{ex}
\begin{ex}%[2D1K4-1]
    Cho hàm số $y=f(x)$ có bảng biến thiên như hình bên. Đồ thị hàm số $y=\dfrac{4}{f(x)+1}$ có tiệm cận ngang là đường thẳng
    \begin{center}
        \begin{tikzpicture}[scale=0.8]
            \tkzTabInit[nocadre=false,lgt=1.5,espcl=3,deltacl=0.6]
            {$x$ /0.6,$y’$ /0.6,$y$ /2}
            {$-\infty$ ,$-1$, $2$, $+\infty$}
            \tkzTabLine{,+,0,-,0,+,}
            \tkzTabVar{-/$1$,+/$4$,-/$-5$,+/$1$}
        \end{tikzpicture}
    \end{center}
    \choice
    {$y=1$}
    {$y=-5$}
    {\True $y=2$}
    {$y=4$}
    \loigiai{
        Dựa vào bảng biến thiên suy ra
        $\lim \limits_{x \to \pm \infty} f(x)=1 \Leftrightarrow \lim \limits_{x \to \pm \infty} \dfrac{4}{f(x)+1} =2$ nên đồ thị hàm số đã cho có tiệm cận ngang là $y=2$.
    }
\end{ex}
\begin{ex}%[2D1K4-1]
    Cho hàm số $y=f(x)$ có bảng biến thiên như hình bên. Đồ thị hàm số $y=\dfrac{2-f(x)}{f(x)+3}$ có tiệm cận ngang là đường thẳng
    \begin{center}
        \begin{tikzpicture}[scale=0.8]
            \tkzTabInit[nocadre=false,lgt=1.5,espcl=3,deltacl=0.6]
            {$x$ /0.6,$y’$ /0.6,$y$ /2}
            {$-\infty$ ,$0$, $2$, $+\infty$}
            \tkzTabLine{,-,0,+,0,-,}
            \tkzTabVar{+/$+\infty$,-/$1$,+/$5$,-/$-\infty$}
        \end{tikzpicture}
    \end{center}
    \choice
    {$y=1$}
    {$y=-3$}
    {$y=2$}
    {\True $y=-1$}
    \loigiai{
        Dựa vào bảng biến thiên suy ra
        $\lim \limits_{x \to \pm \infty} f(x)=\pm \infty \Leftrightarrow \lim \limits_{x \to \pm \infty} \dfrac{2-f(x)}{f(x)+3} =-1$ nên đồ thị hàm số $y=\dfrac{2-f(x)}{f(x)+3}$ có tiệm cận ngang là $y=-1$.
    }
\end{ex}
\begin{ex}%[2D1K4-1]
    Cho hàm số $y=f(x)$ có bảng biến thiên như hình bên. Đồ thị hàm số $y=\dfrac{1}{f^2(x)-4f(x)+4}$ có bao nhiêu tiệm cận đứng?
    \begin{center}
        \begin{tikzpicture}[scale=0.8]
            \tkzTabInit[nocadre=false,lgt=1.5,espcl=3,deltacl=0.6]
            {$x$ /0.6,$y’$ /0.6,$y$ /2}
            {$-\infty$, $2$, $+\infty$}
            \tkzTabLine{,-,0,+,}
            \tkzTabVar{+/$1$,-/$-3$,+/$1$}
        \end{tikzpicture}
    \end{center}
    \choice
    {$1$}
    {$3$}
    {$2$}
    {$0$}
    \loigiai{
        Dựa vào bảng biến thiên suy ra $f^2(x)-4f(x)+4=0 \Leftrightarrow f(x)=2$, phương trình $f(x)=2$ vô nghiệm nên đồ thị hàm số đã cho không có tiệm cận đứng.
    }
\end{ex}
%83
\begin{ex}%[2D1K4-1]
    Cho hàm số $y=f(x)$ có bảng biến thiên như hình bên. Đồ thị hàm số $y=\dfrac{1}{f(3-x)-2}$ có bao nhiêu tiệm cận đứng?
    \begin{center}
        \begin{tikzpicture}[scale=0.8]
            \tkzTabInit[nocadre=false,lgt=1.5,espcl=3,deltacl=0.6]
            {$x$ /0.6,$y’$ /0.6,$y$ /2}
            {$-\infty$ ,$-2$, $2$, $+\infty$}
            \tkzTabLine{,+,0,-,0,+,}
            \tkzTabVar{-/$-\infty$,+/$3$,-/$0$,+/$+\infty$}
        \end{tikzpicture}
    \end{center}
    \choice
    {$1$}
    {\True $3$}
    {$2$}
    {$0$}
    \loigiai{
        Dựa vào bảng biến thiên suy ra $f(3-x)-2=0 \Leftrightarrow f(3-x)=2$, phương trình này có $3$ nghiệm phân biệt nên đồ thị hàm số đã cho có $3$ tiệm cận đứng.
    }
\end{ex}
\begin{ex}%[2D1G4-1]
    Cho hàm số $y=f(x)$ có bảng biến thiên như hình bên. Đồ thị hàm số $y=\dfrac{4}{f(x^2)-2}$ có bao nhiêu tiệm cận đứng?
    \begin{center}
        \begin{tikzpicture}[scale=0.8]
            \tkzTabInit[nocadre=false,lgt=1.5,espcl=3,deltacl=0.6]
            {$x$ /0.6,$y’$ /0.6,$y$ /2}
            {$-\infty$ ,$0$, $3$, $+\infty$}
            \tkzTabLine{,-,0,+,d,-,}
            \tkzTabVar{+/$8$,-/$1$,+/$4$,-/$2$}
        \end{tikzpicture}
    \end{center}
    \choice
    {$5$}
    {$3$}
    {\True $2$}
    {$4$}
    \loigiai{
        Dựa vào bảng biến thiên suy ra
        $f(x^2)-2=0 \Leftrightarrow f(x^2) =2$. Kẻ đường thẳng $y=2$ ta thấy đường thẳng cắt đồ thị hàm số tại hai điểm phân biệt. Suy ra
        $$\hoac{&x^2=a \; (a<0)\\&x^2=b \; (b >0)} \Rightarrow x=\pm \sqrt{b}.$$
        Do đó đồ thị hàm số đã cho có $2$ tiệm cận đứng.
    }
\end{ex}%89
\begin{ex}%[2D1G4-1]
    Cho hàm số $y=f(x)$ có bảng biến thiên như hình bên. Đồ thị hàm số $y=\dfrac{2}{f(|x|)-3}$ có bao nhiêu tiệm cận ngang?
    \begin{center}
        \begin{tikzpicture}[scale=0.8]
            \tkzTabInit[nocadre=false,lgt=1.5,espcl=3,deltacl=0.6]
            {$x$ /0.6,$y’$ /0.6,$y$ /2}
            {$-\infty$ ,$0$, $2$, $+\infty$}
            \tkzTabLine{,+,0,-,0,+,}
            \tkzTabVar{-/$-\infty$,+/$3$,-/$-1$,+/$+\infty$}
        \end{tikzpicture}
    \end{center}
    \choice
    {$4$}
    {\True $3$}
    {$5$}
    {$6$}
    \loigiai{
        Dựa vào bảng biến thiên suy ra
        $f(|x|)-3=0 \Leftrightarrow f(|x|) =3$.\\
        Bảng biến thiên hàm số $y=f(|x|)$ như sau
        \begin{center}
            \begin{tikzpicture}[scale=0.8]
                \tkzTabInit[nocadre=false,lgt=1.5,espcl=3,deltacl=0.6]
                {$x$ /0.6,$y’$ /0.6,$y$ /2}
                {$-\infty$ ,$-2$, $0$, $2$, $+\infty$}
                \tkzTabLine{,-,0,+,0,-,0,+,}
                \tkzTabVar{+/$+\infty$,-/$-1$,+/$3$,-/$-1$,+/$+\infty$}
            \end{tikzpicture}
        \end{center}
        Dựa vào bảng biến thiên hàm số $y=f(|x|)$, phương trình $f(|x|) =3$ có ba nghiệm phân biệt, do đó đồ thị hàm số $y=\dfrac{2}{f(|x|)-3}$ có $3$ tiệm cận đứng.
    }
\end{ex}
\begin{ex}
    \immini{ %Câu 90
        Cho hàm số bậc ba $f(x)= ax^3 +bx^2 +cx +d$ có đồ thị như hình vẽ bên. Đồ thị hàm số $g(x) = \dfrac{\sqrt{x+1}}{(x-3)\cdot f(x)}$ có bao nhiêu đường tiệm cận đứng?
        \choice
        {5}
        {2}
        {4}
        {\True 3}}{\begin{tikzpicture}[scale=.5, font=\footnotesize, line join=round, line cap=round, >=stealth]
            \def\xmin{-3}\def\xmax{3}\def\ymin{-5}\def\ymax{1}
            \draw[->] (\xmin-0.2,0)--(\xmax+0.2,0) node[below] {\footnotesize $x$};
            \draw[->] (0,\ymin-0.2)--(0,\ymax+0.2) node[right] {\footnotesize $y$};
            \draw (0,0) node [below left] {\footnotesize $O$};
            \foreach \x in {-1}\draw (\x,-0.1)--(\x,0.1) node [above] {\footnotesize $\x$};
            \foreach \x in {2}\draw (\x,-0.1)--(\x,0.1) node [above right] {\footnotesize $\x$};
            \foreach \y in {}\draw (-0.1,\y)--(0.1,\y) node [right] {\footnotesize $\y$};
            \clip (\xmin,\ymin) rectangle (\xmax,\ymax);
            \draw[smooth,samples=200,domain=\xmin:\xmax] plot (\x,{1*((\x)^3)+0*((\x)^2)+-3*(\x)+-2});
        \end{tikzpicture}
    }
    \loigiai{
        * Điều kiện: $\heva{&x \ne 3\\&f(x) \ne 0\\&x \ge -1.}$\\
        Nhìn hình vẽ ta thấy
        $f(x)=0\Leftrightarrow \hoac{&x=-1&(\text{nghiệm kép}) \\&x=2&(\text{nghiệm đơn}).}$\\
        Vậy $g(x) = \dfrac{\sqrt{x+1}}{(x-3)\cdot a(x+1)^2 (x-2)}.$ \\
        Đồ thị hàm số $g(x)$ có 3 đường tiệm cận đứng.}
\end{ex}
\begin{ex}
    \immini{ %Câu 92.
        Đường cong ở hình bên là đồ thị của hàm số $y = ax^3 +bx^2 +cx+d$. Đồ thị hàm số $y =\dfrac{(2x+1)\sqrt{x-1}}{x\cdot f(x-2)}$ có tất cả bao nhiêu tiệm cận đứng?
        \choice
        {1}
        {3}
        {4}
        {\True 2}}{\begin{tikzpicture}[scale=.6, font=\footnotesize, line join=round, line cap=round, >=stealth]
            \def\xmin{-3}\def\xmax{3}\def\ymin{-3}\def\ymax{3}
            \draw[->] (\xmin-0.2,0)--(\xmax+0.2,0) node[below] {\footnotesize $x$};
            \draw[->] (0,\ymin-0.2)--(0,\ymax+0.2) node[right] {\footnotesize $y$};
            \draw (0,0) node [below left] {\footnotesize $O$};
            \foreach \x in {-2}\draw (\x,-0.1)--(\x,0.1) node [above left] {\footnotesize $\x$};
            \foreach \x in {2}\draw (\x,-0.1)--(\x,0.1) node [above right] {\footnotesize $\x$};
            \foreach \y in {}\draw (-0.1,\y)--(0.1,\y) node [right] {\footnotesize $\y$};
            \clip (\xmin,\ymin) rectangle (\xmax,\ymax);
            \draw[smooth,samples=200,domain=\xmin:\xmax] plot (\x,{(2/3)*((\x)^3)+0*((\x)^2)+-(8/3)*(\x)});
    \end{tikzpicture}}
    \loigiai{
        * Điều kiện: $\heva{&x \ne 0\\&f(x-2) \ne 0\\&x \ge 1.}$\\
        Nhìn hình vẽ ta thấy
        $f(x-2)=0\Leftrightarrow \hoac{&x-2=-2\\&x-2=0\\&x-2=2}\Leftrightarrow \hoac{&x=0&(\text{không thỏa mãn})\\&x=2&(\text{nghiệm đơn})\\&x=4&(\text{nghiệm đơn}).}$\\
        Vậy $g(x) =\dfrac{(2x+1)\sqrt{x-1}}{x\cdot f(x-2)}=\dfrac{(x-1)\sqrt{x+2}}{x\cdot ax(x-2)(x-4)}.$ \\
        Đồ thị hàm số $g(x)$ có 2 đường tiệm cận đứng.}
\end{ex}
\begin{ex}
    \immini{ %Câu 93.
        Cho hàm số $y= f(x)$ có đồ thị cắt trục hoành tại đúng 3 điểm như hình bên. Đồ thị hàm số $y =\dfrac{(x+2)\sqrt{3-x}}{f(|x|)}$
        có tất cả bao nhiêu tiệm cận đứng?
        \choice
        {1}
        {3}
        {4}
        {\True 2}}{\begin{tikzpicture}[scale=.5, font=\footnotesize, line join=round, line cap=round, >=stealth]
            \def\xmin{-2}\def\xmax{5}\def\ymin{-3}\def\ymax{5}
            \draw[->] (\xmin-0.2,0)--(\xmax+0.2,0) node[below] {\footnotesize $x$};
            \draw[->] (0,\ymin-0.2)--(0,\ymax+0.2) node[right] {\footnotesize $y$};
            \draw (0,0) node [below left] {\footnotesize $O$};
            \foreach \x in {-1,2,4}\draw (\x,-0.1)--(\x,0.1) node [above left] {\footnotesize $\x$};
            \foreach \y in {}\draw (-0.1,\y)--(0.1,\y) node [right] {\footnotesize $\y$};
            \clip (\xmin,\ymin) rectangle (\xmax,\ymax);
            \draw[smooth,samples=200,domain=-1.2:0] plot(\x,{0-8.48*(\x)^(2.0)-5.48*(\x)+3.0});
            \draw[smooth,samples=200,domain=0:2]
            plot(\x,{0-2.7989489689153735*(\x)^(3.0)+8.326740175055514*(\x)^(2.0)-6.957684474449535*(\x)+3.0});
            \draw[smooth,samples=200,domain=2:5]
            plot(\x,{2.395330112721417*(\x)^(2.0)-14.371980676328501*(\x)+19.162640901771336});
    \end{tikzpicture}}
    \loigiai{
        * Điều kiện: $\heva{&f(|x|) \ne 0\\&x \le 3.}$\\
        Nhìn hình vẽ ta thấy
        $f(|x|)=0\Leftrightarrow \hoac{&|x|=-1\\&|x|=2\\&|x|=4}\Leftrightarrow \hoac{&x=\pm 2&(\text{nghiệm đơn})\\&x=- 4&(\text{nghiệm đơn})\\&x=4&(\text{không thỏa mãn}).}$\\
        Vậy $y =\dfrac{(x+2)\sqrt{3-x}}{a(x-2)(x+2)(x+4)(x-4)}$ \\
        Đồ thị hàm số có 2 đường tiệm cận đứng.}
\end{ex}
\begin{ex}
    \immini{ %Câu 94.
        Đường cong ở hình bên là đồ thị của hàm số $y = ax^3 +bx^2 +cx+d$. Đồ thị hàm số $y =\dfrac{(2x+1)\sqrt{1-x}}{f(|x|)}$ có tất cả bao nhiều tiệm cận đứng?
        \choice
        { 1}
        {3}
        {4}
        {\True 2}}{\begin{tikzpicture}[scale=.8, font=\footnotesize, line join=round, line cap=round, >=stealth]
            \def\xmin{-1}\def\xmax{2}\def\ymin{-1.5}\def\ymax{1.5}
            \draw[->] (\xmin-0.2,0)--(\xmax+0.2,0) node[below] {\footnotesize $x$};
            \draw[->] (0,\ymin-0.2)--(0,\ymax+0.2) node[right] {\footnotesize $y$};
            \draw (0.15,0) node [below left] {\footnotesize $O$};
            \foreach \x in {}\draw (\x,0.1)--(\x,-0.1) node [below] {\footnotesize $\x$};
            \foreach \y in {-1,1}\draw (0.1,\y)--(-0.1,\y) node [left] {\footnotesize $\y$};
            \clip (\xmin,\ymin) rectangle (\xmax,\ymax);
            \draw[smooth,samples=200,domain=\xmin:\xmax] plot (\x,{4*((\x)^3)+-6*((\x)^2)+0*(\x)+1});
            \draw[dashed] (0.5,0)--(0.5,0.0)--(0,0.0);
            \draw (0.5,-1pt)--(0.5,1pt) node [above] {\footnotesize $\frac{1}{2}$};
            \draw (-0.7,-1pt)--(-0.7,1pt) node [above] {\footnotesize $-\frac{1}{2}$};
            \draw (1,-1pt)--(1,1pt) node [above] {\footnotesize $1$};
            \draw[dashed] (0.0,0)--(0.0,1.0)--(0,1.0);
            \draw[dashed] (1.0,0)--(1.0,-1.0)--(0,-1.0);
    \end{tikzpicture}}
    \loigiai{
        * Điều kiện: $\heva{&f(|x|) \ne 0\\&x \le 1.}$\\
        Nhìn hình vẽ ta thấy
        $f(|x|)=0\Leftrightarrow \hoac{&|x|=-\dfrac{1}{2}\\&|x|=\dfrac{1}{2}\\&|x|=x_1>1}\Leftrightarrow \hoac{&x=\pm \dfrac{1}{2}&(\text{hai nghiệm đơn})\\&x=- x_1&(\text{nghiệm đơn})\\&x=x_1&(\text{không thỏa mãn}).}$\\
        Vậy $y =\dfrac{(2x+1)\sqrt{1-x}}{f(|x|)}=\dfrac{(2x+1)\sqrt{1-x}}{a\left(x-\dfrac{1}{2}\right)\left(x+\dfrac{1}{2}\right)(x+x_1)(x-x_1)}$ \\
        Đồ thị hàm số có 2 đường tiệm cận đứng.}
\end{ex}
\begin{ex}
    \immini{ %Câu 96.
        Cho đồ thị hàm số $y =f(x)$ và trục hoành có đúng 2 điểm chung như hình bên. Đồ thị hàm số $y =\dfrac{(x-1)\sqrt{3-x}}{f(x^2)}$ có tất cả bao nhiêu tiệm cận đứng?
        \choice
        {1}
        {3}
        {4}
        {\True 2}}{\begin{tikzpicture}[scale=.8, font=\footnotesize, line join=round, line cap=round, >=stealth]
            \def\xmin{-1.5}\def\xmax{2}\def\ymin{-1}\def\ymax{4.5}
            \draw[->] (\xmin-0.2,0)--(\xmax+0.2,0) node[below] {\footnotesize $x$};
            \draw[->] (0,\ymin-0.2)--(0,\ymax+0.2) node[right] {\footnotesize $y$};
            \draw (0,0) node [below left] {\footnotesize $O$};
            \foreach \x in {1}\draw (\x,0.1)--(\x,-0.1) node [below] {\footnotesize $\x$};
            \foreach \x in {-1}\draw (\x,0.1)--(\x,-0.1) node [below left] {\footnotesize $\x$};
            \clip (\xmin,\ymin) rectangle (\xmax,\ymax);
            \draw[smooth,samples=200,domain=-1.1:0] plot(\x,{21.044670464836045*(\x)^(3.0)+24.701786337609526*(\x)^(2.0)+5.65711587277348*(\x)+2.0});
            \draw[smooth,samples=200,domain=0:\xmax] plot(\x,{10.591704641658401*(\x)^(3.0)-19.26315454354621*(\x)^(2.0)+6.6714499018878115*(\x)+2.0});
    \end{tikzpicture}}
    \loigiai{
        * Điều kiện: $\heva{&f(x^2) \ne 0\\&x \le 3.}$\\
        Nhìn hình vẽ ta thấy
        $f(x^2)=0\Leftrightarrow \hoac{&x^2=-1\\&x^2=1}\Leftrightarrow x=\pm 1\,(\text{nghiệm kép}).$\\
        Vậy $y=\dfrac{(x-1)\sqrt{3-x}}{f(x^2)}=\dfrac{(x-1)\sqrt{3-x}}{(x-1)^2(x+1)^2}$ \\
        Đồ thị hàm số có 2 đường tiệm cận đứng.}
\end{ex}
\begin{ex}%[2D1G4-3]%Câu 52
    Cho hàm số $y=ax^3+bx^2+cx+d$ có đồ thị như hình vẽ. Đồ thị của hàm số $g(x)=\dfrac{x^2-x}{f^2(x)-2f(x)}$ có bao nhiêu đường tiệm cận đứng?
    \choice
    {$2$}
    {$3$}
    {\True $4$}
    {$5$}
    \begin{center}
        \begin{tikzpicture}[thick,>=stealth,x=1cm,y=1cm,scale=.7]
            \draw[thin,color=gray!50] (-3.3,-1.3) grid (3.9,5.9);
            \draw[->] (-3.2,0) -- (4.2,0) node[right] {$x$};
            \draw[->] (0,-1.2) -- (0,5.2) node[above] {$y$};
            \draw[color=blue, domain=-2.15:2.15,samples=300] plot (\x,{(\x)^3-3*(\x)+2}) node[right] {$y=f(x)$};
            \draw (-2,0) circle (1.5pt) node[below left]{$-2$};
            \draw (-1,0) circle (1.5pt) node[below]{$-1$};
            \draw (0,0) circle (1.5pt) node[above left]{$O$};
            \draw (1,0) circle (1.5pt) node[below]{$1$};
            \draw (0,4) circle (1.5pt) node[right]{$4$};
            \draw (-1,4) circle (1.5pt);
            \draw[dashed] (-1,0)--(-1,4)--(0,4);
            \draw[red] (-3,2)--(3.2,2);
            \draw[red] (3.5,2) node[right]{$f(x)=2$};
        \end{tikzpicture}
    \end{center}
    \loigiai{
        Xét phương trình $f^2(x)-2f(x)=0 \Leftrightarrow \hoac{&f(x)=0\\&f(x)=2}\Leftrightarrow \hoac{&x=1 \, (\textrm{nghiệm kép trùng nghiệm đơn ở tử số})\\&x=-2\, (\textrm{nghiệm đơn khác nghiệm của tử})\\&x=a\in(-2; -1)\\&x=0\, (\textrm{nghiệm đơn trùng nghiệm ở tử})\\&x=b\in(1; 2)}$\\
        \textbf{Kết luận:} Đồ thị hàm số có $4$ đường tiệm cận đứng.
    }
\end{ex}
\begin{ex}%[Thi thử L3, Lương Thế Vinh, Hà Nội, 2018]%[Phạm Toàn, Dự án (12EX-10)]%[2D1G4-3]%
    \immini{Cho hàm số $y=f(x)$ có đạo hàm liên tục trên $\mathbb{R}$. Đồ thị hàm $f(x)$ như hình vẽ. Số đường tiệm cận đứng của đồ thị hàm số $y=\dfrac{x^2-1}{f^2(x)-4f(x)}$ bằng
        \choice
        {$3$}
        {$1$}
        {$2$}
        {\True $4$}
    }{\begin{tikzpicture}[>=stealth,x=1cm,y=0.75cm,scale=0.7]
            \draw[->] (-2.5,0)--(0,0)%
            node[below right]{$O$}--(2.5,0) node[below]{$x$};
            \draw[->] (0,-2) --(0,5) node[right]{$y$};
            \foreach \x in {-1,1}{
                \draw (\x,0) node[below]{\footnotesize $\x$} circle (1pt);%Ox
            }
            \foreach \y in {2,4}{
                \draw (0,\y) node[right]{\footnotesize $\y$} circle (1pt);%Oy
            }
            \draw[samples=100,domain=-2.05:2] plot (\x,{(\x -1)^2*(\x+2)});
            \draw [dashed] (-1,0)--(-1,4)--(0,4);
            \draw(-1,4) circle (1pt);
    \end{tikzpicture}}
    \loigiai{Xét $f^2(x)-4f(x)=0\Leftrightarrow \hoac{& f(x)=0\\ &f(x)=4.}$\\
        Xét $f(x)=0$ có hai nghiệm, nghiệm $x_1\ne \pm 1$ và nghiệm $x_2=1$ là nghiệm bội (do đồ thị tiếp xúc với trục hoành tại $x=1$. Trường hợp này có $2$ tiệm cận đứng.\\
        Xét $f(x)=4$ có hai nghiệm, nghiệm $x_3\ne \pm 1$ và nghiệm $x_4=-1$ là nghiệm bội (do đồ thị tiếp xúc với đường thẳng $y=4$ tại $x=-1$. Trường hợp này có $2$ tiệm cận đứng.\\
        Vậy đồ thị có $4$ tiệm cận đứng.}
\end{ex}
\begin{ex}%[Thi thử, Trường THPT Lý Thái Tổ - Bắc Ninh, 2019]%[Duong Xuan Loi, 12EX3]%[2D1G4-3]%
    \immini{
        Cho hàm số $f(x)$ có đồ thị như hình bên. Số đường tiệm cận đứng của đồ thị hàm số
        $y=\dfrac{(x^2-4)(x^2+2x)}{[f(x)]^2+2f(x)-3}$ là
        \choice
        {\True $4$}
        {$5$}
        {$3$}
        {$2$}
    }{
        \begin{tikzpicture}[scale=0.5, font=\footnotesize, line join=round, line cap=round, >=stealth]
            \def\a{1} \def\b{-8} \def\c{1} % Hệ số
            \def\xt{-3.7} \def\xp{4} \def\yt{2} \def\yd{-3.7} % x_trái, x_phải, y_trên, y_dưới (giới hạn)
            \draw[->] (\xt,0)--(\xp,0) node [below]{$x$};
            \draw[->] (0,\yd)--(0,\yt) node [left]{$y$};
            \node at (0,0) [below left]{$O$};
            \clip (\xt-0.1,\yd+0.1) rectangle (\xp-0.1,\yt-0.1);
            \draw[smooth,samples=300] plot(\x,{1/4*(\a*(\x)^4+\b*(\x)^2)+\c});
            \draw[dashed] (-2,0)node[above]{$-2$}--(-2,-3)--(2,-3)--(2,0)node[above]{$2$};
            \node at (0,-3)[above left]{$-3$};
            \node at (-3,0)[above left]{$-3$};
            \node at (0,1)[above right]{$1$};
            \node at (3,0)[above right]{$3$};
            \fill (0,0) circle (1pt) (0,-3) circle (1pt) (2,0) circle (1pt) (-2,0) circle (1pt) (-3,0) circle (1pt) (0,1) circle (1pt) (3,0) circle (1pt);
        \end{tikzpicture}
    }
    \loigiai{
        Ta có $y=\dfrac{(x^2-4)(x^2+2x)}{[f(x)]^2+2f(x)-3}$ có các nghiệm ở tử là $x=0$ (bội $1$), $x=2$ (bội $1$), $x=-2$ (bội $2$).\\
        Mặt khác, từ đồ thị $f(x)$ ta thấy hàm số $y=\dfrac{(x^2-4)(x^2+2x)}{[f(x)]^2+2f(x)-3}$ có các nghiệm ở mẫu là
        $f^2(x)+2f(x)-3=0\Leftrightarrow \hoac{& f(x)=1 \\ & f(x)=-3}
        \Leftrightarrow \hoac{& x=0,x=x_1,x=x_2 \\ & x=-2,x=2.}$\\
        Trong đó nghiệm $x=0$, $x=-2$, $x=2$ đều có bội $2$ và $x_1$, $x_2$ khác các nghiệm của tử.\\
        So sánh bội nghiệm ở mẫu và bội nghiệm ở tử thì thấy đồ thị có các tiệm cận đứng là $x=0$, $x=2$; $x=x_1$; $x=x_2$.
    }
\end{ex}
\begin{ex}%[Thi thử, THPT Sơn Tây, Hà Nội, 2019]%[Huỳnh Xuân Tín, 12EX3]%[2D1G4-3]%
    \immini{Cho hàm số $ f(x)=(x+3)(x+1)^2(x-1)(x-3)$ có đồ thị như hình vẽ. Đồ thị hàm số $ g(x)=\dfrac{\sqrt{x-1}}{f^2(x)-9f(x)}$ có bao nhiêu tiệm cận đứng và tiệm cận ngang?
        \choice
        {$3$}
        {\True$ 4$}
        {$ 9$}
        { $8$}
    }{\begin{tikzpicture}[scale=0.3, font=\footnotesize, line join=round, line cap=round, >=stealth]
            %\draw[dashed, line width=0.1pt, gray] (-3.2,-5.5) grid (5.2,4.5);
            \draw[->] (-3.5,0)--(0,0) node[below right]{$O$}--(3.6,0) node[below]{$x$};
            \draw[fill=black] (0,0) circle (1pt);
            \draw[->] (0,-7.7) --(0,6.5) node[right]{$y$};
            \foreach \x in {-3,-1,3}{
                \draw[fill=black] (\x,0) node[below left]{$\x$} circle (1pt);}
            \draw[fill=black] (1,0) node[below right]{$1$} circle (1pt);
            \draw[fill=black] (0,1.35) node[above left]{$9$} circle (1pt);
            \draw [black, domain=-3.2:3.18, samples=100] %
            plot(\x,{0.15*(\x+3)*(\x+1)^2*(\x-1)*(\x-3)});
    \end{tikzpicture}}
    \loigiai{Điều kiện xác định của hàm số $g(x)$ là $\heva{&x\ge1\\ &f^2(x)-9f(x)\not=0.}$\\
        Từ $f^2(x)-9f(x)=0\Leftrightarrow \hoac{&f(x)=0\\&f(x)=9.}$\\
        Với $f(x)=0$ có nghiệm là $x=\pm 1, x=\pm 3$.\\
        Dựa vào đồ thị ta thấy nghiệm của phương trình $f(x)=9$ là hoành độ giao điểm của đường thẳng $y=9$ với đồ thị hàm số $y=f(x)$ nên có nghiệm là $-3<x_3<x_2<-1<0<x_1<1<3<x_0$.\\
        Do đó tập xác định của hàm số $y=g(x)$ là $\mathscr{D}=\left[1;+\infty \right)\setminus\left\lbrace1;3;x_0 \right\rbrace $.\\
        Khi đó ta có \begin{itemize}
            \item $\lim\limits_{x\rightarrow1^+ } g(x)=\lim\limits_{x\rightarrow1^+ }\dfrac{\sqrt{x-1}}{f(x)\left(f(x)-9 \right)}=+\infty$ (vì $x$ tiến gần bên phải $1$ thì $f(x)<0, f(x)-9<0$), suy ra đường thẳng $x=1$ là tiệm cận đứng.
            \item $\lim\limits_{x\rightarrow3^+ } g(x)=\lim\limits_{x\rightarrow3^+ }\dfrac{\sqrt{x-1}}{f(x)\left(f(x)-9 \right)}=-\infty$ (vì $x$ tiến gần bên phải $3$ thì $f(x)>0, f(x)-9<0$), suy ra đường thẳng $x=3$ là tiệm cận đứng.
            \item $\lim\limits_{x\rightarrow x_0^+} g(x)=\lim\limits_{x\rightarrow x_0^+ }\dfrac{\sqrt{x-1}}{f(x)\left(f(x)-9 \right)}=+\infty$ (vì $x$ tiến gần bên phải $x_0$ thì $f(x)>0, f(x)-9>0$), suy ra đường thẳng $x=x_0$ là tiệm cận đứng.
        \end{itemize}
        Và $\lim\limits_{x\rightarrow +\infty} g(x)=\lim\limits_{x\rightarrow +\infty }\dfrac{\sqrt{x-1}}{f(x)\left(f(x)-9 \right)}=0$ (vì bậc ở mẫu của $y=g(x)$ là $10$ và bậc tử của nó là $\dfrac{1}{2}$). Do vậy đồ thị hàm số $y=g(x)$ có một tiệm cận ngang là đường thẳng $y=0$.\\
        Vậy đồ thị hàm số $y=g(x)$ có bốn tiệm cận ngang và đứng. }
\end{ex}
\begin{ex}%[Thi thử, Chuyên Quang Trung-Bình Phước, 2021,lần 1]%[Trần Hòa, 12EX6]%[2D1G4-3]%
    \immini{Cho hàm số $y=f(x)=ax^3+bx^2+cx+d$, có đồ thị như hình vẽ. Số đường tiệm cận đứng của đồ thị hàm số $y=\dfrac{x^2+x-2}{f^2(x)-f(x)}$ là
        \choice
        {$3$}
        {$2$}
        {\True $4$}
        {$5$}}
    {\begin{tikzpicture}[scale=.5, font=\footnotesize, line join=round, line cap=round, >=stealth]
            \draw[->] (-2.5,0)--(0,0) node[below right]{$O$}--(2,0) node[below]{$x$};
            \draw[->] (0,-.5) --(0,4.5) node[right]{$y$};
            \draw [domain=-2.05:2.05, samples=100] %
            plot (\x, {(\x+2)*(\x-1)^2});
            \draw[fill] (0,0) circle (1pt);
            \foreach \x/\g in {-2/140,-1/-90,1/-90}
            \draw[fill] (\x,0) circle(.5pt)node [shift={(\g:.3)}] {$\x$};
            \foreach \y/\g in {2/0,4/0}
            \draw[fill] (0,\y) circle(.5pt)node [shift={(\g:.3)}] {$\y$};
            \draw[dashed] (-1,0)--(-1,4)--(0,4);
    \end{tikzpicture}}
    \loigiai{
        \begin{itemize}
            \item $x^2+x-2=(x-1)(x+2)$.\\
            \item Dựa vào đồ thị hàm số $y=f(x)$ ta có $f^2(x)-f(x)=0\Leftrightarrow\hoac{&f(x)=0\\&f(x)=1.}$\\
            $f(x)=0\Leftrightarrow x=-2$, $x=1$ (nghiệm kép).\\
            $f(x)=1\Leftrightarrow\hoac{&x=x_1,(x_1\in (-2;-1))\\&x=x_2,(x_2\in (0;1))\\&x=x_3,(x_3>1). }$
            \item Do đó $y=\dfrac{(x-1)(x+2)}{a^2(x+2)(x-1)^2(x-x_1)(x-x_2)(x-x_3)}$.
        \end{itemize}
        Suy ra đồ thị có các đườn tiệm cận đứng $x=1$, $x=x_1$, $x=x_2$, $x=x_3$.
    }
\end{ex}
\begin{ex}%[Đề thi hết học kì 2, Bình Minh, Ninh Bình 2018]%[Nguyễn Tuấn Anh, dự án EX9]%[2D1G4-3]%
    \immini{Cho hàm số bậc ba $f(x)=ax^3+bx^2+cx+d$ có đồ thị như hình vẽ bên dưới. Hỏi đồ thị hàm số $g(x)=\dfrac{(x^2-3x+2)\sqrt{x-1}}{x[f^2(x)-f(x)]}$ có bao nhiêu tiệm cận đứng?
        \choice
        {$5$}
        {$6$}
        {\True $3$}
        {$4$}
    }{
        \begin{tikzpicture}[line width=1.0pt,line join=round,>=stealth,x=1cm,y=1cm,scale=1.0]
            \draw[->,line width = 1pt] (-1,0)--(0,0) node[below right]{$O$}--(4,0) node[below]{$x$};
            \draw[->,line width = 1pt] (0,-1.5) --(0,2.5) node[right]{$y$};
            \foreach \x in {1,2}{
                \draw (\x,0) node[below]{$\x$} circle (1pt);
            }
            \foreach \y in {1}{
                \draw (0,\y) node[left]{$\y$} circle (1pt);
            }
            \clip(-0.8,-1) rectangle (3.8,2.3);
            \draw [line width=1.0pt, thick, domain=-0.5:3.5, samples=100]%,domain=-1.5:3] %
            plot (\x, {(5*(\x)-4)*((\x)-2)^2});
            \draw [dash pattern=on 4pt off 4pt] (1.,0.)-- (1.,1.)-- (0.,1.);
            \draw (1,1) circle (1pt);
        \end{tikzpicture}
    }
    \loigiai{
        Điều kiện $\heva{&x\geq 1\\ &x\ne 0\\ &f^2(x)-f(x)\ne 0}\Leftrightarrow \heva{&x\geq 1\\ &f(x)\ne 0\\ & f(x)\ne 1.}$\\
        Dựa vào đồ thị hàm số $y=f(x)$, ta thấy $f(x)=0$ có hai nghiệm, một nghiệm $x_1<1$ và một nghiệm kép bằng $2$. Do đó ta biểu diễn được $f(x)$ dưới dạng
        $$ f(x)=a(x-x_1)(x-2)^2. $$
        Dựa vào đồ thị hàm số $y=f(x)$, ta thấy phương trình $f(x)=1$ có ba nghiệm $1,x_2, x_3$, với $1<x_2<2<x_3$. Do đó ta biểu diễn được $f(x)-1$ dưới dạng
        $$ f(x)-1=a(x-1)(x-x_2)(x-x_3). $$
        Lúc này điều kiện được viết lại như sau $\heva{&x>1\\ &x\ne x_2, x\ne 2, x\ne x_3.}$\\
        Với điều kiện đó thì $g(x)$ được viết lại là
        $$ g(x)=\dfrac{\sqrt{x-1}}{a^2x(x-x_1)(x-x_2)(x-2)(x-x_3)}. $$
        Ta có
        \begin{align*}
            &\lim\limits_{x\to 1^+}g(x)=\lim\limits_{x\to 1^+}\dfrac{\sqrt{x-1}}{a^2x(x-x_1)(x-x_2)(x-2)(x-x_3)}=0,\\
            & (x=1\mbox{ \textbf{không} là tiệm cận đứng}) \\
            &\lim\limits_{x\to x_2^+}g(x)=\lim\limits_{x\to x_2^+}\dfrac{\sqrt{x-1}}{a^2x(x-x_1)(x-x_2)(x-2)(x-x_3)}=+\infty,\\
            & (x=x_2\mbox{ là tiệm cận đứng}) \\
            &\lim\limits_{x\to 2^+}g(x)=\lim\limits_{x\to 2^+}\dfrac{\sqrt{x-1}}{a^2x(x-x_1)(x-x_2)(x-2)(x-x_3)}=-\infty,\\
            & (x=2\mbox{ là tiệm cận đứng}) \\
            &\lim\limits_{x\to x_3^+}g(x)=\lim\limits_{x\to x_3^+}\dfrac{\sqrt{x-1}}{a^2x(x-x_1)(x-x_2)(x-2)(x-x_3)}=+\infty,\\
            & (x=x_3\mbox{ là tiệm cận đứng}) \\
        \end{align*}
        Vậy đồ thị hàm số $g(x)$ có tất cả $3$ tiệm cận đứng.
    }
\end{ex}
\begin{ex}%[VDC5-Đỗ Đường Hiếu]%[2D1G4-3]%
    \immini{Cho hàm số $f(x)=(x+3)(x+1)^2(x-1)(x-3)$ có đồ thị như hình vẽ. Đồ thị hàm số $g(x)=\dfrac{\sqrt{x-1}}{f^2(x)-9f(x)}$ có bao nhiêu tiệm cận đứng và tiệm cận ngang?
        \choice
        {$3$}
        {\True $4$}
        {$9$}
        {$8$}}
    {\begin{tikzpicture}[xscale=0.8,yscale=0.05, line join=round, line cap=round,font=\footnotesize,>=stealth]
            \draw[->] (-4,0)--(4,0) node[below]{$x$};
            \draw[->] (0,-56)--(0,30) node[left]{$y$};
            \coordinate[label=below left:$O$] (O) at (0,0);
            \draw (-1,0) node[below] { $-1$}(1,0) node[below] { $1$};
            \draw (-3,0) node[below left] { $-3$};
            \draw (3,0) node[below right] { $3$};
            \clip (-3.3,-60) rectangle (3.5,26);
            \draw[smooth,samples=300,domain=-3.5:3.5] plot(\x,{(\x+3)*(\x+1)^2*(\x-1)*(\x-3)});
            \foreach \x in {-3,-1,1,3}
            \draw[shift={(\x,0)},color=black] (0pt,20pt) -- (0pt,-20pt);
            \draw[shift={(0,9)},color=black] (2pt,0pt) -- (-2pt,0pt) node[left] {$9$};
        \end{tikzpicture}
    }
    \loigiai{%GV tổng quát hóa bài toán:
        Cho hàm số đa thức $y=f(x)$ có đồ thị $(C)$. Tìm số đường tiệm cận của đồ thị hàm số $g(x)=\dfrac{\sqrt{ax+b}}{P\left(f(x) \right) }$, trong đó $P\left(f(x) \right)$ là một đa thức của $f(x)$.
        Nếu $a>0$ thì $\lim\limits_{x\to +\infty}g(x)=0$.\\
        Nếu $a<0$ thì $\lim\limits_{x\to -\infty}g(x)=0$.\\
        Do đó đồ thị hàm số $y=g(x)$ luôn có duy nhất một đường tiệm cận ngang là $y=0$.\\
        Gọi $x=x_0$ là một nghiệm của phương trình $P\left(f(x) \right) =0$ thỏa mãn điều kiện $ax+b\ge 0$. Rõ ràng khi đó $\lim\limits_{x\to x_0^+}g(x)=+\infty$ hoặc $\lim\limits_{x\to x_0^+}g(x)=-\infty$.\\
        Bởi vậy, số đường tiệm cận đứng của đồ thị hàm số $y=g(x)$ chính là số nghiệm của phương trình $P\left(f(x) \right) =0$ thỏa mãn điều kiện $ax+b\ge 0$.
        \immini{Ta có $f^2(x)-9f(x)=0\Leftrightarrow \hoac{&f(x)=0\\&f(x)=9.}$\\
            \begin{itemize}
                \item $f(x)=0$ có các nghiệm thuộc $\left[1;+\infty\right)$ là $x=1$ và $x=3$.
                \item Đường thẳng $y=9$ cắt đồ thị hàm số $y=f(x)$ tại duy nhất một điểm có hoành độ thuộc $\left[1;+\infty\right)$ là $x=a>3$.
            \end{itemize}
        }
        {\begin{tikzpicture}[xscale=0.8,yscale=0.05, line join=round, line cap=round,font=\footnotesize,>=stealth]
                \draw[->] (-4,0)--(4,0) node[below]{$x$};
                \draw[->] (0,-56)--(0,30) node[left]{$y$};
                \coordinate[label=below left:$O$] (O) at (0,0);
                \draw (-4,9)--(4,9);
                \draw (-1,0) node[below] { $-1$}(1,0) node[below] { $1$};
                \draw (-3,0) node[below left] { $-3$};
                \draw (3,0) node[below right] { $3$};
                \clip (-3.3,-60) rectangle (3.5,26);
                \draw[smooth,samples=300,domain=-3.5:3.5] plot(\x,{(\x+3)*(\x+1)^2*(\x-1)*(\x-3)});
                \foreach \x in {-3,-1,1,3}
                \draw[shift={(\x,0)},color=black] (0pt,20pt) -- (0pt,-20pt);
                \draw[shift={(0,9)},color=black] (2pt,0pt) -- (-2pt,0pt) node[above left] {$9$};
        \end{tikzpicture}}
        \noindent
        Bởi vậy, hàm số $g(x)=\dfrac{\sqrt{x-1}}{f^2(x)-9f(x)}$ có tập xác định là $\mathscr D=\left[1;3\right) \cup \left(3;a\right) \cup\left( a;+\infty\right)$.\\
        Khi đó ta có
        \begin{itemize}
            \item $\lim\limits_{x\to+\infty}g(x)=0$ nên đồ thị hàm số $y=g(x)$ có một đường tiệm cận ngang là đường thẳng $y=0$.
            \item $\lim\limits_{x\to 1^+}g(x)=\lim\limits_{x\to 1^+}\dfrac{\sqrt{x-1}}{f(x)\left[f(x)-9\right] }=+\infty$;\\
            $\lim\limits_{x\to 3^+}g(x)=\lim\limits_{x\to 3^+}\dfrac{\sqrt{x-1}}{f(x)\left[f(x)-9\right] }=-\infty$;\\
            $\lim\limits_{x\to a^+}g(x)=\lim\limits_{x\to a^+}\dfrac{\sqrt{x-1}}{f(x)\left[f(x)-9\right] }=+\infty$.\\
            Do đó nên đồ thị hàm số $y=g(x)$ có $3$ đường tiệm cận đứng là các đường thẳng $x=1$, $x=3$ và $x=a$.
        \end{itemize}
        Như vậy, đồ thị hàm số $y=g(x)$ có $4$ đường tiệm cận, trong đó có $1$ đường tiệm cận ngang và $3$ đường tiệm cận đứng.
    }
\end{ex}
\begin{ex}%[VDC5-Đỗ Đường Hiếu]%[2D1G4-3]%
    \immini{Cho hàm số bậc ba $y=f(x)$ có đồ thị như hình vẽ bên. Đồ thị hàm số $g(x)=\dfrac{x\sqrt{x+1}}{f(x)\left[f^2(x)-16 \right] }$ có bao nhiêu tiệm cận đứng?
        \choice
        {\True $4$}
        {$5$}
        {$6$}
        {$7$}}
    {\begin{tikzpicture}[scale=0.6,line join=round, line cap=round,font=\footnotesize,>=stealth]
            \draw[->] (-2.5,0)--(4,0) node[below]{$x$};
            \draw[->] (0,-5)--(0,2.5) node[left]{$y$};
            \coordinate[label=below left:$O$] (O) at (0,0);
            \draw[dashed] (-1,0)--(-1,-4)--(0,-4);
            \clip (-2.3,-5) rectangle (3.5,2.5);
            \draw[smooth,samples=300,domain=-3.5:3.5] plot(\x,{-0.5*(\x+2)*(\x-1)*(\x-3)});
            \foreach \x in {-2,-1,1,3}
            \draw[shift={(\x,0)},color=black] (0pt,2pt) -- (0pt,-2pt) node[above] { $\x$};
            \foreach \y in {-4,-3,1}
            \draw[shift={(0,\y)},color=black] (2pt,0pt) -- (-2pt,0pt) node[right] {$\y$};
        \end{tikzpicture}
    }
    \loigiai{
        Xét phương trình $f(x)\left[f^2(x)-16 \right]=0$ \, $(*)$, với điều kiện $x\in\left[-1;+\infty \right) $.\\
        Ta có $f(x)\left[f^2(x)-16 \right]=0\Leftrightarrow \hoac{&f(x)=0\\&f(x)=4\\&f(x)=-4.}$\\
        \begin{itemize}
            \item Phương trình $f(x)=0$ có hai nghiệm $x\in\left[-1;+\infty \right) $ là $x=1$ và $x=3$.
            \item Phương trình $f(x)=4$ có không có nghiệm $x\in\left[-1;+\infty \right) $.
            \item Phương trình $f(x)=-4$ có hai nghiệm $x\in\left[-1;+\infty \right) $ là $-1<x_1<0$ và $x_2>3$.
        \end{itemize}
        Rõ ràng $\lim\limits_{x\to x_0^+}g(x)=+\infty$ hoặc $\lim\limits_{x\to x_0^+}g(x)=-\infty$, trong đó $x=x_0$ là nghiệm thuộc $\left[-1;+\infty \right) $ của phương trình $(*)$. Do đó đường thẳng $x=x_0$ là tiệm cận đứng của đồ thị hàm số $y=g(x)$.\\
        Từ đó suy ra đồ thị hàm số $g(x)=\dfrac{x\sqrt{x+1}}{f(x)\left[f^2(x)-16 \right] }$ có $4$ tiệm cận đứng.
    }
\end{ex}
\begin{ex}%[VDC5-Đỗ Đường Hiếu]%[2D1G4-3]%
    \immini{Cho $y=f(x)$ là hàm số đa thức có đồ thị như hình vẽ bên. Đặt $g(x)=\dfrac{\sqrt{x-1}}{\left[f(x)\right]^2-2f(x)}$ có bao nhiêu đường tiệm cận đứng?
        \choice
        {$5$}
        {$3$}
        {$4$}
        {\True $2$}}
    {\begin{tikzpicture}[scale=0.6,line join=round, line cap=round,font=\footnotesize,>=stealth]
            \draw[->] (-3,0)--(2.5,0) node[below]{$x$};
            \draw[->] (0,-1)--(0,5) node[left]{$y$};
            \coordinate[label=above left:$O$] (O) at (0,0);
            \draw[dashed] (-1,0)--(-1,4)--(0,4);
            \clip (-2.3,-1) rectangle (2.5,4.5);
            \draw[smooth,samples=300,domain=-3.5:3.5] plot(\x,{(\x)^3-3*(\x)+2});
            \foreach \x in {-2,-1,1}
            \draw[shift={(\x,0)},color=black] (0pt,2pt) -- (0pt,-2pt) node[below] { $\x$};
            \foreach \y in {2,4}
            \draw[shift={(0,\y)},color=black] (2pt,0pt) -- (-2pt,0pt) node[right] {$\y$};
        \end{tikzpicture}
    }
    \loigiai{
        Xét phương trình $\left[f(x)\right]^2-2f(x)=0$ \, $(*)$, với điều kiện $x\in\left[1;+\infty \right) $.\\
        Ta có $\left[f(x)\right]^2-2f(x)=0\Leftrightarrow \hoac{&f(x)=0\\&f(x)=2.}$\\
        \begin{itemize}
            \item Phương trình $f(x)=0$ có một nghiệm $x\in\left[1;+\infty \right) $ là $x=1$.
            \item Phương trình $f(x)=2$ có một nghiệm $x\in\left[1;+\infty \right) $ là $x=x_1>1$.
        \end{itemize}
        Rõ ràng $\lim\limits_{x\to x_0^+}g(x)=+\infty$ hoặc $\lim\limits_{x\to x_0^+}g(x)=-\infty$, trong đó $x=x_0$ là nghiệm thuộc $\left[1;+\infty \right) $ của phương trình $(*)$. Do đó đường thẳng $x=x_0$ là tiệm cận đứng của đồ thị hàm số $y=g(x)$.\\
        Từ đó suy ra đồ thị hàm số $g(x)=\dfrac{\sqrt{x-1}}{\left[f(x)\right]^2-2f(x)}$ có $2$ tiệm cận đứng.
    }
\end{ex}
\begin{ex}%[VDC5-NgocDungHo]%[2D1G4-3]%
    \immini
    {
        Cho hàm số $f(x)$ có đồ thị như hình bên. Số đường tiệm cận đứng của đồ thị hàm số $y=\dfrac{(x^2-4)(x^2+2x)}{[f(x)]^2-4f(x)+3}$ là
        \choice
        {$4$}
        {\True $5$}
        {$3$}
        {$2$}
    }
    {\begin{tikzpicture}[>=stealth,scale=0.5, line join=round, line cap=round]
            \def\f[#1]{-0.25*((#1)^4-8*(#1)^2+4)}
            \draw[->] (-4.1,0)--(4,0) node [below]{$x$};
            \draw[->] (0,-2)--(0,4) node [left]{$y$};
            \node at (0,0) [above left]{$O$};
            % \clip;
            \draw[domain=-2.9:2.9,samples=300,thick] plot (\x,{\f[\x]});
            \foreach \x in {-2,2} \filldraw (\x,0) node[below]{\x} circle (2pt);
            %\foreach \x in {-3,3} \filldraw (\x,0) node[below left]{\x} circle (2pt);
            \filldraw (-3,0) node[below left]{$-3$} circle (2pt);
            \filldraw (3,0) node[below right]{$3$} circle (2pt);
            \filldraw (0,1) node[left]{$1$} circle (2pt);
            \filldraw (0,3) node[above left]{$3$} circle (2pt);
            \draw[dashed](-2,0)--(-2,3)--(2,3)--(2,0);
            \draw (3,-1.75) node[right]{$y=f(x)$};
        \end{tikzpicture}
    }
    \loigiai{
        Xét hàm số $y=g(x)=\dfrac{(x^2-4 )(x^2+2x)}{[f(x)]^2-4f(x)+3}$.
        \immini
        {
            Giải phương trình $(x^2-4)(x^2+2x)=0 $\\
            $\Leftrightarrow \hoac{& x^2-4=0 \\ & x^2+2x=0}\Leftrightarrow \hoac{& x=\pm 2 \\ & x=0.}$\\
            Giải phương trình $[f(x)]^2-4f(x)+3=0$\\
            $ \Leftrightarrow \hoac{& f(x)=1 \\ & f(x)=3} \Leftrightarrow \hoac{& x = \pm 2 \\ & x=a\\&x=b\\&x=c\\&x=d.}$\\ với $-3<a<-2<b<c<2<d<3$.\\
        }
        {\begin{tikzpicture}[>=stealth,scale=0.8, line join=round, line cap=round]
                \def\f[#1]{-0.25*((#1)^4-8*(#1)^2+4)}
                \def\g[#1]{1}
                \def\h[#1]{3}
                \draw[->] (-4.1,0)--(4,0) node [below]{$x$};
                \draw[->] (0,-2)--(0,4) node [left]{$y$};
                \node at (0,0) [above left]{$O$};
                % \clip;
                \draw[domain=-2.9:2.9,samples=300,thick] plot (\x,{\f[\x]});
                \draw[domain=-4:4,samples=300,thick] plot (\x,{\g[\x]});
                \draw[domain=-4:4,samples=300,thick] plot (\x,{\h[\x]});
                \foreach \x in {-3,-2,2,3} \filldraw (\x,0) node[below]{\x} circle (2pt);
                % \filldraw (-3,0) node[above left]{$-3$} circle (2pt);
                % \filldraw (3,0) node[above ]{$3$} circle (2pt);
                \filldraw (0,1) node[below left]{$1$} circle (2pt);
                \filldraw (0,-1) node[below left]{$-1$} circle (2pt);
                \filldraw (0,3) node[above left]{$3$} circle (2pt);
                \draw[dashed](-2,0)--(-2,3) (2,3)--(2,0) (2.61,0)node[below]{$d$}--(2.61,1) (-2.61,0)node[below]{$a$}--(-2.61,1) (1.08,0)node[below]{$c$}--(1.08,1)(-1.08,0)node[below]{$b$}--(-1.08,1);
                \draw (3,2.75) node[right]{$y=f(x)$};
            \end{tikzpicture}
        }
        Trong điều kiện xác định của hàm số $y=g(x)$ ta có thể viết $$y=g(x)=\dfrac{x(x-2)(x+2)^2}{(x-a)(x-b)(x-c)(x-d) (x-2)^2(x+2)^2}=\dfrac{x}{(x-a)(x-b)(x-c)(x-d)(x-2)}$$
        Vậy số tiệm cận đứng của đồ thị hàm số $y=g(x)$ bằng $5$.
    }
\end{ex}
\Closesolutionfile{ans}
%\subsection{ĐỀ ÔN LUYỆN}
%\boxde
\BTTN
\begin{ex}%[2D1N3-1]Câu 2
 Đường thẳng $x = a$ là một đường tiệm cận đứng của
 đồ thị hàm số $ y = f (x)$ nếu điều kiện sau thoả mãn
 \choice
 {$\displaystyle\lim_{x\to +\infty }f(x)=a$}
 {\True $\displaystyle\lim_{x\to a^-}f(x)=+\infty $}
 {$\displaystyle\lim_{x\to -\infty }f(x)=a$}
 {$\displaystyle\lim_{x\to a^-}f(x)=a $}
 \loigiai{ Đường thẳng $x = a$ được gọi là một đường tiệm cận đứng (hay tiệm cận đứng) của đồ thị hàm số $ y = f (x)$ nếu ít nhất một trong các điều kiện sau thoả mãn: \\$\displaystyle\lim_{x\to a^+}f(x)=+\infty $, $\displaystyle\lim_{x\to a^+}f(x)=-\infty $, $\displaystyle\lim_{x\to a^-}f(x)=-\infty $, $\displaystyle\lim_{x\to a^-}f(x)=+\infty $.}
\end{ex}
\begin{ex}%[2D1N3-1]Câu 4
 Đường thẳng $y = ax + b$ ($a \neq 0$) được gọi là đường tiệm cận xiên của đồ thị hàm số $y = f(x)$ nếu
 \choice
 {\True $\displaystyle\lim_{x\to -\infty }\big(f(x)-ax-b\big)=0$ hoặc $\displaystyle\lim_{x\to +\infty }\big(f(x)-ax-b\big)=0$}
 {$\displaystyle\lim_{x\to -\infty }\big(f(x)-ax+b\big)=0$ hoặc $\displaystyle\lim_{x\to +\infty }\big(f(x)-ax+b\big)=0$}
 {$\displaystyle\lim_{x\to 0 }\big(f(x)-ax+b\big)=+\infty$ hoặc $\displaystyle\lim_{x\to 0 }\big(f(x)-ax+b\big)=+\infty$}
 {$\displaystyle\lim_{x\to 0 }\big(f(x)-ax-b\big)=-\infty$ hoặc $\displaystyle\lim_{x\to 0 }\big(f(x)-ax-b\big)=-\infty$}
 \loigiai{Đường thẳng $y = ax + b, a \neq 0$, được gọi là đường tiệm cận xiên (hay tiệm cận xiên) của đồ thị hàm số $y = f(x)$ nếu\\ $\displaystyle\lim_{x\to -\infty }[f(x)-(ax+b)]=\displaystyle\lim_{x\to -\infty }(f(x)-ax-b)=0$ hoặc\\ $\displaystyle\lim_{x\to +\infty }[f(x)-(ax+b)]=\displaystyle\lim_{x\to +\infty }(f(x)-ax-b)=0$.
 }
\end{ex}
\begin{ex}
 Tiệm cận ngang của đồ thị hàm số $ y=\dfrac{2x-1}{x+1} $ là đường thẳng
 \choice
 {$y=-1$}
 {$ x=-1 $}
 {\True $ y=2 $}
 {$ x=2 $}
 \loigiai
 {
 Ta có $ \lim\limits_{x\to \pm\infty}y=2$ suy ra đường thẳng $ y=2 $ là tiệm cận ngang của đồ thị hàm số $ y=\dfrac{2x-1}{x+1} $.
 }
\end{ex}
\begin{ex}
 Tiệm cận ngang của đồ thị hàm số $y=\dfrac{1}{2x-3}$ là đường thẳng
 \choice
 {$y=\dfrac{3}{2}$}
 {$x=\dfrac{3}{2}$}
 {\True $y=0$}
 {$y=\dfrac{1}{2}$}
 \loigiai{
 Vì $\lim\limits_{x\to -\infty} \dfrac{1}{2x-3}=\lim\limits_{x\to +\infty} \dfrac{1}{2x-3}=0$ nên đồ thị hàm số có tiệm cận ngang $y=0$.
 }
\end{ex}
\begin{ex}
 Đồ thị hàm số $f(x)=\dfrac{2x-3}{x+1}$ có đường tiệm cận đứng là
 \choice
 {$y=2$}
 {\True $x=-1$}
 {$y=-1$}
 {$x=2$}
 \loigiai{
 Ta có $\displaystyle \lim_{x \to (-1)^-}f(x)=\displaystyle \lim_{x \to (-1)^-}\dfrac{2x-3}{x+1}=+\infty $; $\displaystyle \lim_{x \to (-1)^+}f(x)=\displaystyle \lim_{x \to (-1)^+}\dfrac{2x-3}{x+1}=-\infty$ nên đường thẳng $x=-1$ là đường tiệm cận đứng của đồ thị hàm số.}
\end{ex}
\begin{ex}
 Hàm số nào sau đây có đồ thị nhận đường thẳng $x=2$ là đường tiệm cận đứng?
 \choice
 {$y=\dfrac{2}{x+2}$}
 {\True $y=\dfrac{5x}{2-x}$}
 {$y=\dfrac{1}{x+1}$}
 {$y=x-2+\dfrac{1}{x+1}$}
 \loigiai{
 Ta có $\lim\limits_{x\to 2^+} \dfrac{5x}{2-x}=-\infty $ và $\lim\limits_{x\to 2^-} \dfrac{5x}{2-x}=+\infty $ nên đồ thị hàm số $y=\dfrac{5x}{2-x}$ nhận $x=2$ làm tiệm cận đứng.}
\end{ex}
\begin{ex}%[2D1V3-1]Câu 12
 Đồ thị của hàm số nào sau đây có giao điểm của hai đường tiệm cận thuộc đường thẳng $y=x$?
 \choice
 {$y=\dfrac{2x-1}{x+3}$}
 {\True$y=\dfrac{x+4}{x-1}$}
 {$y=\dfrac{2x+1}{x+2}$}
 {$\dfrac{1}{x+3}$}
 \loigiai{
 Đáp án $y=\dfrac{2x-1}{x+3}$ có giao hai đường tiệm tiệm cận là $(-3;2)\notin d$\\
 Đáp án $y=\dfrac{x+4}{x-1}$ có giao hai đường tiệm cận là $(1;1)\in d$\\
 Đáp án $y=\dfrac{2x+1}{x+2}$ có giao hai đường tiệm cận là $(-2;2)\notin d$\\
 Đáp án $\dfrac{1}{x+3}$ có giao hai đường tiệm cận là $(-3;0)\notin d$\\
 }
\end{ex}
\begin{ex}%[2D1N3-1]Câu 6
 Đồ thị hàm số $y=\dfrac{x-2}{x^{2}-4}$ có mấy đường tiệm cận?
 \choice
 {$3$}
 {$1$}
 {\True$2$}
 {$0$}
 \loigiai{ Hàm số $y=\dfrac{x-2}{x^{2}-4}=\dfrac{x-2}{(x-2)(x+2)}=\dfrac{1}{x+2}$.\\
 $\heva{&\displaystyle\lim_{x\to +\infty }\dfrac{1}{x+2}=0\\&
 \displaystyle\lim_{x\to -\infty }\dfrac{1}{x+2}=0.}$\\
 Nên $y=0$ là đường tiệm cận ngang của hàm số, hàm số có tiệm cận ngang thì không có tiệm cận xiên.\\
 $\heva{&\displaystyle\lim_{x\to -2^- }\dfrac{1}{x+2}= - \infty \\&
 \displaystyle\lim_{x\to -2^+ }\dfrac{1}{x+2}= + \infty.}$\\
 Nên $x=-2$ là đường tiệm cận đứng của hàm số.\\
 Vậy hàm số có hai đường tiệm cận.
 }
\end{ex}
\begin{ex}
 Tiệm cận xiên của đồ thị hàm số $y=\dfrac{x^2+x-1}{x}$ có phương trình là
 \choice
 {$y=x-1$}
 {$y=x-2$}
 {$y=x-3$}
 {\True$y=x+1$}
 \loigiai{
 Ta có $y=\dfrac{x^2+x-1}{x}=x+1-\dfrac{1}{x}$.\\
 Xét $$\displaystyle\lim_{x\to \pm \infty }\big(y-(x+1)\big)=\displaystyle\lim_{x\to \pm \infty }\dfrac{-1}{x}=0$$
 Vậy đường tiệm cận xiên cần tìm của hàm số $f(x)$ có phương trình $y=x+1$.}
\end{ex}
\begin{ex}
 Tiệm cận xiên của đồ thị hàm số $y=\dfrac{2x^2-3x+4}{x-1}$ có phương trình là
 \choice
 {$y=x-1$}
 {\True $y=2x-1$}
 {$y=2x+1$}
 {$y=x+1$}
 \loigiai{
 Ta có $y=\dfrac{2x^2-3x+4}{x-1}=2x-1+\dfrac{3}{x-1}$. Suy ra $y=2x-1$ là đường tiệm cận xiên của đồ thị hàm số.
 }
\end{ex}
\begin{ex}
 Cho hàm số $y=f(x)$ xác định $ \mathbb{R} \setminus \left\lbrace 0\right\rbrace $, liên tục trên mỗi khoảng xác định và có bảng biến thiên như sau.\\
 \begin{center}
 \begin{tikzpicture}[>=stealth,font=\footnotesize,scale=1]
 \tikzset{double style/.append style = {draw=\tkzTabDefaultWritingColor,double=\tkzTabDefaultBackgroundColor,double distance=2pt}}
 \tkzTabInit[nocadre=false,lgt=1.2,espcl=2.5,deltacl=0.6]
 {$x$ /0.6,$y'$ /0.6,$y$ /2}
 {$-\infty$,$0$,$ 1 $,$+\infty$}
 \tkzTabLine{,-,d,+,$ 0 $,- }
 \tkzTabVar{+/ $+\infty$,-D- /$-1$/$-\infty$,+/$2$,-/$ -\infty $}
 \end{tikzpicture}
 \end{center}
 Chọn khẳng định đúng
 \choice
 {Đồ thị hàm số có hai tiệm cận ngang}
 {\True Đồ thị hàm số có đúng một tiệm cận đứng}
 {Đồ thị hàm số không có tiệm cận đứng và tiệm cận ngang}
 {Đồ thị hàm số có đúng một tiệm cận ngang}
 \loigiai
 {
 Dựa vào bảng biến thiên ta thấy
 $ \lim\limits_{x \to + \infty} f(x)=+\infty$; $ \lim\limits_{x \to -\infty}f(x)=-\infty$; $ \lim\limits_{x \to 0^+}f(x)=-\infty$.\\
 Suy ra đồ thị hàm số có đúng một tiệm cận đứng.
 }
\end{ex}
\begin{ex}%[2D1N3-1]Câu 5
 \immini{Cho hàm số $y=f(x)$ có đồ thị như hình bên dưới. Khẳng định nào sau đây là khẳng định đúng?
 \choice
 {Đồ thị hàm số chỉ có 2 đường tiệm cận đứng $x=-1$ và $x=1$}
 {\True Đồ thị hàm số có 3 đường tiệm cận}
 {Đồ thị hàm số có 4 đường tiệm cận}
 {Đồ thị hàm số có 2 đường tiệm cận đứng và 1 đường tiệm cận xiên}}{
 \begin{tikzpicture}[>=stealth]
 \draw[->] (-4,0) --(4,0);
 \draw[->](0,-4)--(0,4);
 \draw (0,0) node[below left]{$O$};
 \draw (4,0) node[below]{$x$};
 \draw (0,4) node[left]{$y$};
 \draw (1,0) node[above left]{$1$};
 \draw (-1,0) node[above left]{$-1$};
 \clip (-4,-4) rectangle(4,4);
 \draw[thick,samples=100] plot[domain=-4:4]
 (\x,{(\x)/((\x)^(2)-1)});
 \draw (-1.7,1.5) node
 {$x=-1$};
 \draw (1.5,-1.5) node
 {$x=1$};
 \end{tikzpicture}}
 \loigiai{ Đồ thị hàm số có 2 đường tiệm cận đứng $x=-1$ và $x=1$ và một đường tiệm cận ngang $y=0$, hàm số không có đường tiệm cận xiên.}
\end{ex}
\begin{ex}
 Biết rằng đồ thị hàm số $ y=\dfrac{ax+1}{bx-2}$ có tiệm cận đứng là $x=2$ và tiệm cận ngang là $y=3$. Giá trị của $a+b$ bằng
 \choice
 {$0$}
 {\True $4$}
 {$5$}
 {$1$}
 \loigiai{
 Điều kiện để đồ thị hàm số $ y=\dfrac{ax+1}{bx-2}$ có tiệm cận đứng và tiệm cận ngang là $-2a-b\ne 0$. \quad$(*)$\\
 $b\ne 0$ vì nếu $ b=0$, đồ thị hàm số $ y=\dfrac{ax+1}{-2}$ không có tiệm cận.\\
 Tập xác định của hàm số $y=\dfrac{ax+1}{bx-2}$ là $\mathscr{D}=\left(-\infty;\dfrac{2}{b}\right)\cup\left(\dfrac{2}{b};+\infty\right)$.\\
 $\lim\limits_{x\to\pm\infty}\dfrac{ax+1}{bx-2}=\dfrac{a}{b}\Rightarrow y=\dfrac{a}{b}$ là đường tiệm cận ngang của đồ thị hàm số.\\
 Theo giả thiết ta có $\dfrac{a}{b}=3\Leftrightarrow a=3b$.\\
 Đồ thị hàm số $y=\dfrac{ax+1}{bx-2}$ có $ x=\dfrac{2}{b}$ là đường tiệm cận đứng.\\
 Theo giả thiết ta có $\dfrac{2}{b}=2\Leftrightarrow b=1\Rightarrow a=3$ (thỏa mãn điều kiện $(*)$).\\
 Vậy $a+b=4$.
 }
\end{ex}
\begin{ex}
 Tìm tất cả giá trị của tham số $m$ để đường tiệm cận xiên của đồ thị hàm số $y=2mx+3-\dfrac{4}{x+1}$ đi qua điểm $M(1;7)$.
 \choice
 {$m=1$}
 {$m=3$}
 {\True $m=2$}
 {$m=-2$}
 \loigiai{
 Xét $\displaystyle\lim_{x\to \pm \infty }\left( y-\left( 2mx+3\right) \right) =\displaystyle\lim_{x\to \pm \infty }\dfrac{-4}{x+1}=0$.\\
 Vậy đường tiệm cận xiên có phương trình $y=2mx+3$.\\
 Đường thẳng này qua điểm $M(1;7)$, suy ra $2m \cdot 1 ++3=7 \Leftrightarrow m=2$.
 }
\end{ex}
\begin{ex}
 Tại một công ty sản xuất đồ chơi A, công ty phải chi 50000 USD để thiết lập dây chuyền sản xuất ban đầu. Sau đó, cứ sản xuất được một sản phẩm đồ chơi A, công ty phải trả 5 USD cho nguyên liệu thô và nhân công. Gọi $x\,(x \geq 1)$ là số đồ chơi A mà công ty đã sản xuất và $T(x)$ (đơn vị USD) là tổng số tiền bao gồm cả chi phí ban đầu mà công ty phải chi trả khi sản xuất $x$ đồ chơi A. Người ta xác định chi phí trung bình cho mỗi sản phẩm đồ chơi A là $M(x)=\dfrac{T(x)}{x}$. Khi $x$ đủ lớn ($x\to +\infty$) thì chi phí trung bình (USD) cho mỗi sản phẩm đồ chơi $A$ gần nhất với kết quả nào sau đây?
 \choice
 {$50\,000$}
 {$50\,005$}
 {10}
 {\True $5$}
 \loigiai{
 Gọi $T(x)$ (đơn vị USD) là tổng số tiền bao gồm cả chi phí ban đầu mà công ty phải chi trả khi sản xuất $x$ đồ chơi A thì $T(x)=50\,000 + 5x$.\\
 Ta có $$\displaystyle\lim_{x\to + \infty }\dfrac{T(x)}{x} =\displaystyle\lim_{x\to + \infty }\left(\dfrac{50\,000}{x}+5\right) =5.$$
 }
\end{ex}
\BTTF
\begin{ex}
 Cho hàm số $y=f(x)$ có $\displaystyle\lim_{x\rightarrow 3^{-}}f(x)=1$, $\displaystyle\lim\limits_{x\rightarrow 3^{+}}f(x)=+\infty$ và $\displaystyle\lim_{x\rightarrow -\infty}f(x)=1$, $\displaystyle\lim\limits_{x\rightarrow +\infty}f(x)=+\infty$. Xét tính đúng sai của các khẳng định sau:
 \choiceTF
 {\True Đồ thị của hàm số $y=f(x)$ có tiệm cận ngang là đường thẳng $y=1$}
 {\True Đồ thị của hàm số $y=f(x)$ có tiệm cận đứng là đường thẳng $x=3$}
 {Đồ thị của hàm số $y=f(x)$ không có tiệm cận ngang}
 {Đồ thị của hàm số $y=f(x)$ không có tiệm cận đứng}
 \loigiai{
 \begin{itemchoice}
 \itemch Do $\displaystyle\lim_{x\rightarrow -\infty}f(x)=1$ nên $y=1$ là đường tiệm cận ngang của đồ thị hàm số. (1)
 \itemch Do $\displaystyle\lim\limits_{x\rightarrow 3^{+}}f(x)=+\infty$ nên $x=3$ là đường tiệm cận đứng của đồ thị hàm số. (2)
 \itemch Từ (1) suy ra khẳng định này sai.
 \itemch Từ (2) suy ra khẳng định này sai.
 \end{itemchoice}
 }
\end{ex}
\begin{ex}
 Cho hàm số $y=f(x)$ xác định trên $\mathbb{R}\backslash\{\pm 2\}$ và có bảng biến thiên như hình vẽ bên dưới.
 \begin{center}
 \begin{tikzpicture}[scale=0.8,>=stealth]
 \tikzset{double style/.append style = {draw=\tkzTabDefaultWritingColor,double=\tkzTabDefaultBackgroundColor,double distance=2pt}}
 \tkzTabInit[nocadre=false, lgt=1, espcl=4,deltacl=1pt]{$x$ /1,$y'$ /1,$y$ /2.2}{$-\infty$,$-2$,$2$,$+\infty$}
 \tkzTabLine{,-,d,-,d,-,}
 \tkzTabVar{+/ $0$ ,-D+/ $-10$/$+\infty$ , -D+/ $-\infty$/$+\infty$,-/$0$}
 \end{tikzpicture}
 \end{center}
 Xét tính đúng sai của các khẳng định sau:
 \choiceTF
 {\True Hàm số không có điểm cực trị}
 {$\lim\limits_{x\to -2^{-}}f(x)=+\infty$}
 {\True Đồ thị hàm số có đúng 1 tiệm cận ngang}
 {Đồ thị hàm số có đúng $1$ tiệm cận đứng}
 \loigiai{
 Dựa vào bảng biến thiên ta thấy
 \begin{itemchoice}
 \itemch Hàm số không có điểm cực trị;
 \itemch $\lim\limits_{x\to -2^{-}}f(x)=-10$;
 \itemch $\lim\limits_{x\to \pm \infty}f(x)=0$. Suy ra đồ thị có đúng 1 đường tiệm cận ngang là $y=0$.
 \itemch $\lim\limits_{x\to -2^{+}}f(x)=+\infty$ và $\lim\limits_{x\to 2^{+}}f(x)=+\infty$ nên đồ thị hàm số có đúng 2 đường tiệm cận đứng $x = \pm 2$.
 \end{itemchoice}
 }
\end{ex}
\begin{ex}
 Cho hàm số $y=\dfrac{\sqrt{x^2-x+2}}{x-1}$. Xét tính đúng sai của các khẳng định sau:
 \choiceTF
 {\True Tập xác định của hàm số là $\mathbb{R} \backslash\{1\}$}
 {\True Đồ thị hàm số có các đường tiệm cận ngang là $y=1,\,y=-1$}
 {Đồ thị hàm số đã cho có tất cả 2 đường tiệm cận}
 {Các đường tiệm cận của đồ thị cùng với trục $O y$ tạo thành 1 đa giác có diện tích bằng 1}
 \loigiai{
 \begin{itemchoice}
 \itemch Điều kiện xác định $\heva{&x^2-x+2>0\text{ luôn đúng}\\& x-1 \ne 0} \Leftrightarrow x \ne 1$. Vậy tập xác định của hàm số là $\mathbb{R} \backslash\{1\}$
 \itemch Ta có
 \begin{itemize}
 \item [$\bullet$] $\displaystyle\lim_{x\rightarrow -\infty}f(x)=-1$ nên $y=-1$ là đường tiệm cận ngang;
 \item [$\bullet$] $\displaystyle\lim_{x\rightarrow +\infty}f(x)=1$ nên $y=1$ là đường tiệm cận ngang;
 \end{itemize}
 \itemch Do $\displaystyle\lim_{x\rightarrow 1^+}f(x)=+\infty$ nên $x=1$ là đường tiệm cận đứng. Vậy đồ thị hàm số có tất cả 3 đường tiệm cận (2 TCN và 1 TCĐ).
 \itemch Minh họa miền giới hạn của các đường tiệm cận và trục $Oy$ như sau:
 \begin{center}
 \begin{tikzpicture}[smooth,samples=300,scale=0.8,>=stealth]
 \draw[->] (-3,0)--(6,0) node[below]{$x$};
 \draw[->] (0,-3)--(0,3) node[right]{$y$};
 \draw (0,0) node[below left]{$O$};
 \draw[pattern = north west lines] (0,-1)--(1,-1)--(1,1)--(0,1);
 \draw
 (-3,-1)--(4,-1)node[below]{\scriptsize TCN $y=-1$}
 (-3,1)--(4,1)node[above]{\scriptsize TCN $y=1$}
 (1,-3)--(1,3)node[above right]{\scriptsize TCĐ $x=1$};
 \end{tikzpicture}
 \end{center}
 Miền giới hạn là hình chữ nhật có diện tích là $S=2 \cdot 1 =2$.
 \end{itemchoice}
 }
\end{ex}
\begin{ex}
 Cho hàm số $y=f(x)=\dfrac{2 x^2+2 x+5}{2 x+1}$. Xét tính đúng sai của các khẳng định sau:
 \choiceTF
 {\True Đạo hàm của hàm số đã cho là $y'=\dfrac{4\left(x^2+x-2\right)}{(2 x+1)^2}$}
 {\True Các điểm cực trị của đồ thị hàm số có toạ độ là $(-2 ;-3)$ và $(1 ; 3)$}
 {\True Đường tiệm cận đứng của đồ thị hàm số có phương trình là $x=-\dfrac{1}{2}$}
 {\True Đường tiệm cận xiên của đồ thị hàm số có phương trình là $y=x+\dfrac{1}{2}$}
 \loigiai{
 \begin{itemchoice}
 \itemch Ta có $y'=\dfrac{(2 x^2+2 x+5)'(2 x+1)-(2 x+1)'(2 x^2+2 x+5)}{(2x+1)^2}=\dfrac{4\left(x^2+x-2\right)}{(2 x+1)^2}$.
 \itemch $y'=0 \Leftrightarrow x^2+x-2 =0 \Leftrightarrow \hoac{&x=1\\&x=-2}$.\\
 Thay vào hàm số, ta tính được toạ độ các điểm cực trị là $(-2 ;-3)$ và $(1 ; 3)$.
 \itemch Điều kiện xác định $x \ne -\dfrac{1}{2}$.\\
 $\displaystyle\lim_{x\rightarrow -\frac{1}{2}^+}f(x)=+\infty$ nên $x=-\dfrac{1}{2}$ là đường tiệm cận đứng;
 \itemch $y=\dfrac{2 x^2+2 x+5}{2 x+1}=x+\dfrac{1}{2}+\dfrac{9}{2(2x+1)}$. \\
 Suy ra đồ thị có đường tiệm cận xiên là $y=x+\dfrac{1}{2}$.
 \end{itemchoice}
 }
\end{ex}
\BTTL
\begin{ex}%[2D1B4-1]%
 Các đường tiệm cận của đồ thị hàm số $ y=\dfrac{2x+3}{x-1}$ tạo với hai trục tọa độ một hình chữ nhật có
 diện tích bằng bao nhiêu?\\
 \shortans[3]{$2$}
 \loigiai{
 Tập xác định $\mathscr{D}=\mathbb{R}\setminus\{1\}$.
 \begin{itemize}
 \item $\lim\limits_{x\to 1} y=\lim\limits_{x\to 1^+}\dfrac{2x+3}{x-1}=+\infty\Rightarrow x=1 $ là tiệm cận đứng của đồ thị hàm số.
 \item $\lim\limits_{x\to+\infty} y=\lim\limits_{x\to+\infty}\dfrac{2x+1}{x-1}=2\Rightarrow y=2 $ là tiệm cận ngang của đồ thị hàm số.
 \end{itemize}
 Hai đường tiệm cận của đồ thị hàm số tạo với hai trục tọa độ một hình chữ nhật có diện tích $ S=1\cdot 2=2 $.
 }
\end{ex}
\begin{ex}%[2D1B4-1]%
 Cho hàm số $y=\dfrac{x+1}{x-3}$ có đồ thị $(C)$ và đường thẳng $\Delta: y=mx+m-3.$ Biết đường thẳng $\Delta$ đi qua giao điểm hai đường tiệm cận của
 $(C).$ Khi đó giá trị của $m$ bằng bao nhiêu?\\
 \shortans[3]{$1$}
 \loigiai{
 Đồ thị (C) có TCĐ là $x=3$ và TCN là $y=1$, suy ra $I(3 ; 1)$ là giao điểm hai tiệm cận của $(C)$.\\
 Do $I \in \Delta \Rightarrow 1=3m+m-3 \Leftrightarrow 4m-4=0 \Leftrightarrow m=1$.
 }
\end{ex}
\begin{ex}
 Cho hàm số $y=\dfrac{3x^2+2x}{4x+4}$. Khoảng cách từ điểm $M(3;-2)$ đến đường tiệm cận xiên của đồ thị hàm số này bằng bao nhiêu?\\
 \shortans[3]{$3{,}2$}
 \loigiai{
 $y=\dfrac{3x^2+2x}{4x+4}=\dfrac{3}{4}x-\dfrac{1}{4}+\dfrac{1}{4x+4}$.\\
 Xét $\displaystyle\lim_{x\to \pm \infty }\left( y-\left( \dfrac{3}{4}x-\dfrac{1}{4}\right) \right) =\displaystyle\lim_{x\to \pm \infty }\dfrac{1}{4x+4}=0$.\\
 Vậy đường tiệm cận xiên có phương trình $y=\dfrac{3}{4}x-\dfrac{1}{4} \Leftrightarrow 3x-4y-1=0$.\\
 Khoảng cách từ điểm $M$ đến đường tiệm cận xiên là
 $$d=\dfrac{\big|3 \cdot 3 -4 \cdot (-2)-1\big|}{\sqrt{3^2+(-4)^2}}=\dfrac{16}{5}=3,2$$
 }
\end{ex}
\begin{ex}
 Nồng độ oxygen trong hồ theo thời gian $t$ cho bởi công thức $y(t)=5-\dfrac{15 t}{9 t^2+1}$, với $y$ được tính theo $\mathrm{mg} / l$ và $t$ được tính theo giờ, $t \geq 0$. Đường tiệm cận ngang của đồ thị hàm số $y=y(t)$ khi $t \to +\infty$ có dạng $y=a$. Giá trị của $a$ bằng bao nhiêu?\\
 \shortans[3]{$5$}
 \loigiai{
 $\displaystyle\lim_{t\rightarrow +\infty}y(t)=\lim_{t\rightarrow +\infty}\left( 5-\dfrac{15 t}{9 t^2+1}\right) =5$ nên $y=5$ là đường tiệm cận ngang.
 }
\end{ex}
\begin{ex}
 Số lượng sản phẩm bán được của một công ty trong $x$ (tháng) được tính theo công thức $S(x)=200\left(5-\dfrac{9}{2+x}\right)$, trong đó $x \geq 1$. Xem $y=S(x)$ là một hàm số xác định trên nửa khoảng $[1 ;+\infty)$. Biết $y=a$ là tiệm cận ngang của đồ thị hàm số đó. Giá trị của $a$ bằng bao nhiêu?\\
 \shortans[3]{$1000$}
 \loigiai{
 Ta có
 $S(x)=200\left(5-\dfrac{9}{2+x}\right)=1000-\dfrac{1800}{2+x}.$\\
 Vì $\displaystyle\lim \limits_{n \to +\infty}_{x \rightarrow\pm\infty} S(x)=\lim \limits_{n \to +\infty}_{x \rightarrow\pm\infty} \left(1000-\dfrac{1800}{2+x}\right)=1000$
 nên đường thẳng $y=1000$ là tiệm cận ngang của đồ thị hàm số đã cho.
 }
\end{ex}
\begin{ex}%
 \immini{Cho hàm đa thức bậc ba $y=f(x)$ có đồ thị như hình vẽ.	Đồ thị hàm số $y=\dfrac{(x+1)(x^2-1)}{f(x)}$ có bao nhiêu đường tiệm cận (đứng và ngang)?\\
 \shortans[3]{$3$}}{
 \begin{tikzpicture}[>=stealth]
 \draw[->] (-3,0) --(3,0)node[below]{$x$};
 \draw[->](0,-3)--(0,2.5)node[left]{$y$};
 \draw (0,0.5) node[below left]{$O$};
 \draw (2,0) node[above left]{$2$};
 \draw (-1,0) node[above left]{$-1$};
 \draw[dashed] (0,-2)
 node[left]{$-2$} -- (1,-2) --
 (1,0) node[above]{$1$};
 \draw[thick,samples=100] plot[domain=-2.2:2.3]
 (\x,{(1/2)*(\x)^3-(3/2)*\x-1})node[above]{$y=f(x)$};
 \end{tikzpicture}}
 \loigiai{ Hàm số có dạng $f(x)=ax^3+bx^2+cx-1$ (vì là hàm bậc ba cắt trục tung tại điểm có tung độ $-1$)\\
 Đồ thị hàm số đã cho đi qua các điểm có tọa độ là $(-1;0)$, $(1;-2)$, $(2;0)$ \\
 $\to \heva{&8a+4b+2c=1\\& -a+b-c=1 \\&a+b+c=-1}\Leftrightarrow \heva{&a=\dfrac{1}{2}\\&b=0 \\&c=\dfrac{-3}{2}.}$\\
 $\to f(x)=\dfrac{1}{2}x^3-\dfrac{3}{2}x-1=\dfrac{1}{2}(x-1)^2(x-2)$.\\
 Khi đó $y=\dfrac{(x+1)(x^2-1)}{f(x)}=\dfrac{(x+1)(x^2-1)}{\dfrac{1}{2}(x-1)^2(x-2)}=\dfrac{2(x+1)^2}{(x-1)(x-2)}$.\\
 Đồ thị hàm số trên có tiệm cận ngang $y=2$ và tiệm cận đứng là $x=1,x=2$.\\
 Vậy đồ thị hàm số $y=\dfrac{(x+1)(x^2-1)}{f(x)}$ có 3 đường tiệm cận.
 }
\end{ex}
%\boxde
\BTTN
\begin{ex}%[2D1B4-1]
    Phương trình đường tiệm cận ngang của đồ thị hàm số $y=\dfrac{x-3}{x-1}$ là
    \choice
    {$y=5$}
    {$y=0$}
    {$x=1$}
    {\True $y=1$}
    \loigiai
    {
        Ta có $\lim\limits_{x \to \pm \infty}\dfrac{x-3}{x-1} = \lim\limits_{x \to \pm \infty}\dfrac{1-\dfrac{3}{x}}{1-\dfrac{1}{x}}=1$, nên đường thẳng $y=1$ là tiệm cận ngang của đồ thị hàm số đã cho.
    }
\end{ex}


\begin{ex}
    Đường thẳng nào dưới đây là tiệm cận đứng của đồ thị hàm số $y=\dfrac{2x}{x+2}$?
    \choice
    {$x=2$}
    {$x=0$}
    {\True $x=-2$}
    {$x=1$}
    \loigiai{
        Tập xác định $\mathscr{D}=\mathbb{R}\setminus\{-2\}$.
        \begin{itemize}
            \item $\lim\limits_{x\to -2^+}\dfrac{2x}{x+2}=-\infty$.
            \item $\lim\limits_{x\to -2^-}\dfrac{2x}{x+2}=+\infty$.
        \end{itemize}
        Vậy $x=2$ là đường tiệm cận đứng của đồ thị hàm số.
    }
\end{ex}

\begin{ex}%[2D1Y4-1]
    Cho hàm số $y=\dfrac{x+1}{2x-2}$. Khẳng định nào sau đây đúng?
    \choice
    {Đồ thị hàm số có tiệm cận đứng là $x=\dfrac{1}{2}$}
    {Đồ thị hàm số có tiệm cận ngang là $y=-\dfrac{1}{2}$}
    {\True Đồ thị hàm số có tiệm cận ngang là $y=\dfrac{1}{2}$}
    {Đồ thị hàm số có tiệm cận đứng là $x=2$}
    \loigiai{
        Đồ thị hàm số $y=\dfrac{x+1}{2x-2}$ có tiệm cận đứng $x=1$ và tiệm cận ngang $y=\dfrac{1}{2}$.
    }
\end{ex}


\begin{ex}%[2D1Y4-1]
    Cho hàm số $y=f(x)$ có $\lim\limits_{x\to -\infty} f(x)= -2$ và $\lim\limits_{x\to +\infty} f(x)= 2$. Khẳng định nào sau đây đúng?
    \choice
    {Đồ thị hàm số đã cho có đúng một tiệm cận ngang}
    {Đồ thị hàm số đã cho không có tiệm cận ngang}
    {Đồ thị hàm số đã cho có hai tiệm cận ngang là hai đường thẳng $x=-2$ và $x=2$}
    {\True Đồ thị hàm số đã cho có hai tiệm cận ngang là hai đường thẳng $y=-2$ và $y=2$}
    \loigiai{
        $\lim\limits_{x\to -\infty} f(x)= -2$ nên $y=-2$ là tiệm cận ngang.\\
        $\lim\limits_{x\to +\infty} f(x)= 2$ nên $y=2$ là tiệm cận ngang.
    }
\end{ex}

\begin{ex}%[2D1Y4-1]
    Cho hàm số $y= \dfrac{2017}{x-2}$ có đồ thị $(H)$. Số đường tiệm cận của $(H)$ là
    \choice
    {$0$}
    {\True $2$}
    {$3$}
    {$4$}
    \loigiai{
        Đồ thị $(H)$ có tiệm cận đứng là $x=2$, tiệm cận ngang là $y=0$.\\
        Vậy số đường tiệm cận của $(H)$ là $2$.
    }
\end{ex}

\begin{ex}
    Tìm số đường tiệm cận của đồ thị hàm số $ y = \dfrac{x^2 - 3x + 2}{x^2 - 4}. $
    \choice
    {$1$}
    {$ 0$}
    {\True $2$}
    {$3$}
    \loigiai
    {
        Tập xác định: $ \mathscr D = \mathbb{R} \backslash \{\pm2 \} $.\\
        Ta có $ \lim \limits_{x \to \pm  \infty} y = 1 \Rightarrow  $ đồ thị hàm số có 1 tiệm cận ngang là $ y = 1. $\\
        Ta lại có $\lim \limits_{x \to 2} y =  \lim \limits_{x \to 2} \dfrac{x-1}{x+2} = \dfrac{1}{4} $ và $\lim \limits_{x \to -2^+} y =  \lim \limits_{x \to -2^+} \dfrac{x-1}{x+2} = -\infty$ nên đồ thị hàm số có 1 tiệm cận đứng là $ x = -2. $\\
        Vậy đồ thị hàm số đã cho có 2 đường tiệm cận.
    }
\end{ex}


\begin{ex}%[Đề minh họa BGD 2018-2019]%[2D1B4-1]
    \immini[thm]{Cho hàm số $y=f(x)$ có bảng biến thiên như sau. Tổng số tiệm cận ngang và tiệm cận đứng của đồ thị hàm số đã cho là
        \haicot
        {$4$}
        {$1$}
        {\True $3$}
        {$2$}
    }{
        \begin{tikzpicture}
            \tikzset{double style/.append style = {draw=\tkzTabDefaultWritingColor,double=\tkzTabDefaultBackgroundColor,double distance=2pt}}
            \tkzTabInit[nocadre=false, lgt=1.2, espcl=2.5,deltacl=0.6
            ]{$x$ /0.6,$y'$ /0.6,$y$ /1.6}{$-\infty$,$1$,$+\infty$}
            \tkzTabLine{,+, d ,+,}
            \tkzTabVar{-/ $2$ / , +D-/ $+\infty$ / $3$ , +/ $5$ /}
    \end{tikzpicture}}
    \loigiai{
        Từ bảng biến thiên ta có
        \begin{itemize}
            \item $\lim\limits_{x \to -\infty} y =2$ suy ra $y=2$ là tiệm cận ngang.
            \item $\lim\limits_{x \to +\infty} y =5$ suy ra $y=5$ là tiệm cận ngang.
            \item $\lim\limits_{x \to 1^-} y = +\infty$ suy ra $x=1$ là tiệm cận đứng.
        \end{itemize}
        Vậy đồ thị hàm số tổng cộng có $3$ đường tiệm cận ngang và tiệm cận đứng.

    }
\end{ex}

\begin{ex}%[Đề tập huấn Sở Ninh Bình, 2019]%[Nguyễn Văn Hải, dự án(12EX-5-2019)]%[2D1B4-1]
    \immini[thm]{Cho hàm số $y=f(x)$ có bảng biến thiên như hình bên. Hỏi đồ thị hàm số $y=f(x)$ có tổng số bao nhiêu tiệm cận (tiệm cận đứng và tiệm cận ngang)?
        \haicot
        {$0$}
        {\True $2$}
        {$3$}
        {$1$}
    }{
        \begin{tikzpicture}
            \tikzset{double style/.append style = {draw=\tkzTabDefaultWritingColor,double=\tkzTabDefaultBackgroundColor,double distance=2pt}}
            \tkzTabInit[nocadre=false,lgt=1,espcl=2,deltacl=0.6]
            {$x$ /0.6,$y’$ /0.6,$y$ /2.2}
            {$-\infty$ , $1$ , $3$ , $+\infty$}
            \tkzTabLine{,+,d,+,0,-,}
            \tkzTabVar{-/$-1$ ,+D- / $+\infty$ /  $-\infty$,+/ $2$, -/$-\infty$}
    \end{tikzpicture}}
    \loigiai{
        Ta có $\lim\limits_{x\to -\infty}f(x)=-1$, $\lim\limits_{x\to +\infty}f(x)=-\infty$ nên $y=-1$ là tiệm cận ngang.\\
        Ta có $\lim\limits_{x\to 1^+}f(x)=-\infty$ nên $x=1$ là tiệm cận đứng.\\
        Vậy đồ thị hàm số có $2$ đường tiệm cận.
    }
\end{ex}


\begin{ex}%[GHK1, THCS - THPT Nguyễn Khuyến, HCM, 2019]%[Vinh Vo, 12Ex3-2019]%[2D1B4-1]
    \immini[thm]{Cho hàm số $ y = f(x) $ xác định trên $ (-2;0) \cup (0;+\infty) $ và có bảng biến thiên như hình vẽ. Số đường tiệm cận của đồ thị hàm số $ f(x) $ là
        \haicot
        {$ 4 $}
        {$ 2 $}
        {$ 1 $}
        {\True $ 3 $}
    }{
        \begin{tikzpicture}
            \tikzset{double style/.append style = {double distance=2pt}}
            \tkzTabInit[lgt=1.2,espcl=3.5,nocadre=false]
            {$x$ /0.6, $f’(x)$ /0.7,$f(x)$ /1.7}
            { $-2$ , $0$ , $+\infty$}
            \tkzTabLine{d,+,d,-, }
            \tkzTabVar{D-/ /$ -\infty $, +D+/$ +\infty $/ $ 1 $, -/ $ 0 $}
            \draw[pattern = north west lines] ($(N13)-(0.2ex,0)$) rectangle (T11);
    \end{tikzpicture}}
    \loigiai{
        Từ bảng biến thiên, ta thấy $ \heva{& \lim \limits_{ x \to -2^{+} } f(x) = - \infty \\ & \lim \limits_{x \to 0^{-} } f(x) = + \infty \\ & \lim \limits_{x \to + \infty} f(x) = 0  } $, suy ra đồ thị hàm số $ f(x) $ có $ 3 $ tiệm cận trong đó có $ 2 $ tiệm cận đứng và $ 1 $ tiệm cận ngang.
    }
\end{ex}

\begin{ex}%[Đề thi khảo sát chất lượng trường THCS-THPT Lômônôxốp, Hà Nội 2018 ,Nhật Thiện 12EX1-2019]%[2D1Y4-1]
    \immini[thm]{Cho hàm số $y=f(x)$ xác định trên $\mathbb{R}\backslash \left\{{0}\right\}$, liên tục trên mỗi khoảng xác định và có bảng biến thiên như hình bên. Hỏi đồ thị hàm số có bao nhiêu đường tiệm cận?
        \haicot
        {$1$}
        {\True $2$}
        {$3$}
        {$4$}}{\begin{tikzpicture}
            \tikzset{double style/.append style = {draw=\tkzTabDefaultWritingColor,double=\tkzTabDefaultBackgroundColor,double distance=2pt}}
            \tkzTabInit[nocadre=false,lgt=1,espcl=2.3,deltacl=0.6]{$x$ /0.6,$y'$ /0.6,$y$ /1.8}{$-\infty$,$0$,$1$,$+\infty$}
            \tkzTabLine{,+,d,-,0,+}
            \tkzTabVar{+/ $2$ ,-D-/ $-\infty$/$-\infty$, +/ $1$ ,-/ $-\infty$ /}
    \end{tikzpicture}}
    \loigiai{
        Dựa vào bảng biến thiên, ta có $$\lim\limits_{x\to -\infty}y=2;\qquad \lim\limits_{x\to 0^{\pm}}y=-\infty$$
        Vậy hàm số có một tiệm cận ngang $y=2$, một tiệm cận đứng $x=0$.
    }
\end{ex}

\begin{ex}
    Số tiệm cận đứng của đồ thị hàm số $y=\dfrac{\sqrt{x+9}-3}{x^2+x}$ là
    \choice
    {$3$}
    {$2$}
    {$0$}
    {\True $1$}
    \loigiai{
        Tập xác định $\mathscr{D}=[-9;+\infty)\setminus \{-1;0\}$. \\
        Ta có $\left\{\begin{aligned}
            &\lim\limits_{x\to -1^+} \dfrac{\sqrt{x+9}-3}{x^2+x}=+\infty \\
            &\lim\limits_{x\to -1^-} \dfrac{\sqrt{x+9}-3}{x^2+x}=-\infty
        \end{aligned}\right. \Rightarrow x=-1$ là tiệm cận đứng. \\
        Ngoài ra $\lim\limits_{x\to 0} \dfrac{\sqrt{x+9}-3}{x^2+x}=\dfrac{1}{6}$ nên $x=0$ không là tiệm cận.}
\end{ex}

\begin{ex}
    Phương trình đường tiệm cận xiên của đồ thị hàm số $y=\dfrac{2x^2-x+1}{x-1}$ là
    \choice
    {$y=x-1$}
    {\True $y=2x+1$}
    {$y=2x+3$}
    {$y=x+1$}
    \loigiai{
        Sau khi chia đa thức, ta viết lại hàm số $y=2x+1+\dfrac{2}{x-1}$.\\
        Do $\lim\limits_{x\to \pm \infty}\left[y-(2x+1)\right]=\lim\limits_{x\to \pm \infty}\dfrac{2}{x-1}=0$ nên $y=2x+1$ là đường tiệm cận xiên.}
\end{ex}

\begin{ex}
    Giao điểm của đường tiệm cận đứng và đường tiệm cận xiên của đồ thị hàm số $y=\dfrac{x^2-3x+5}{x-2}$ có tọa độ là
    \choice
    {$(2;3)$}
    {$(-2;1)$}
    {\True $(2;1)$}
    {$(-2;3)$}
    \loigiai{
        Sau khi chia đa thức, ta viết lại hàm số $y=x-1+\dfrac{3}{x-2}$.
        \begin{itemize}
            \item [$\bullet$] Đồ thị hàm số có tiệm cận đứng là $x=2$
            \item [$\bullet$] Do $\lim\limits_{x\to \pm \infty}\left[y-(x-1)\right]=\lim\limits_{x\to \pm \infty}\dfrac{3}{x-2}=0$ nên $y=x-1$ là đường tiệm cận xiên.
        \end{itemize}
        Giải hệ $\heva{&x=2\\&y=x-1} \Leftrightarrow \heva{&x=2\\&y=1}$. Suy ra, giao hai đường tiệm cận có tọa độ $(2;1)$.
    }
\end{ex}

\begin{ex}%[Đề thi giữa HK1, THPT Bình Sơn Đồng Nai, 2019]%[Phan Minh Tâm, dự án EX3]%[2D1B4-2]
    Tiệm cận đứng của đồ thị hàm số $ y=\dfrac{2x+1}{x-m} $ đi qua điểm $ M(2;5) $ khi $ m $ bằng bao nhiêu?
    \choice
    {$ m=-2 $}
    {$ m=-5 $}
    {$ m=5 $}
    {\True $ m=2 $}
    \loigiai{
        Với $m \ne -\dfrac{1}{2}$ đồ thị có tiệm cận đứng là đường thẳng $ x=m $. Tiệm cận đứng $ x=m $ đi qua $ M(2;5) $ khi chỉ khi $ m=2 $.
    }
\end{ex}

\begin{ex}%[GHK1, THPT Quế Võ 2-Bắc Ninh, 2019]%[TranTony,12EX2]%[2D1B4-2]
    Cho hàm số $ y = \dfrac{2x^2-3x+m}{x-m} $ có đồ thị $ (C) $. Tìm tất cả các giá trị của tham số $ m $ để $ (C) $ không có tiệm cận đứng.
    \choice
    {\True $ m = 0 $ hoặc $ m = 1 $}
    {$ m = 2 $}
    {$ m = 1 $}
    {$ m = 0 $}
    \loigiai{
        Đồ thị $ (C) $ không có tiệm cận đứng khi $ m $ là nghiệm của $ 2x^2-3x+m $
        \begin{align*}
            \Leftrightarrow 2m^2 - 3m + m = 0 \Leftrightarrow \hoac{& m = 0 \\& m = 1.}
        \end{align*}
    }
\end{ex}

\BTTF

\begin{ex}
    Cho hàm số $y=f(x)=\dfrac{3-2x}{x+1}$. Xét tính đúng sai của các khẳng định sau:
    \choiceTF
    {\True Tập xác định của hàm số là $\mathbb{R}\backslash\{-1\}$}
    {\True  Đồ thị hàm số có đường tiệm cận đứng là $x=-1$}
    {Đồ thị hàm số có đường tiệm cận ngang là $y=3$}
    {Hai đường tiệm cận (đứng và ngang) của đồ thị tạo với hai trục tọa độ một hình phẳng có diện tích bằng $3$}
    \loigiai{
        \begin{itemchoice}
            \itemch Điều kiện xác định $x+1 \ne 0 \Leftrightarrow x \ne -1$. Suy ra $D=\mathbb{R}\backslash\{-1\}$.
            \itemch Đồ thị hàm số có tiệm cận đứng là $x=-1$.
            \itemch Đồ thị hàm số có tiệm cận ngang là $y=\dfrac{-2}{1}=-2$.
            \itemch Hai đường tiệm cận (đứng và ngang) của đồ thị tạo với hai trục tọa độ một hình chữ nhật như hình vẽ
            \begin{center}
                \begin{tikzpicture}[smooth,samples=300,scale=0.8,>=stealth]
                    \draw[->] (-3,0)--(2,0) node[below]{$x$};
                    \draw[->] (0,-3)--(0,1) node[right]{$y$};
                    \draw (0,0) node[below right]{$O$};
                    \draw[pattern = north west lines] (0,0)--(0,-2)--(-1,-2)--(-1,0);
                    \draw (-3,-2)--(2,-2)node[above]{\scriptsize TCN $y=-2$} (-1,-3)--(-1,1)node[above]{\scriptsize TCĐ $x=-1$};
                    \draw[fill=black] (-1,0) circle(1.5pt) (-1,-2) circle(1pt) (0,-2) circle(1.5pt);
                    \node[right] at (0,-2.3) {$A$};
                    \node[left] at (-1,0.3) {$B$};
                \end{tikzpicture}
            \end{center}
            Diện tích hình chữ nhật này là
            $$S=OA \cdot OB=2 \cdot 1=2.$$
    \end{itemchoice}}
\end{ex}

\begin{ex}%[2D1K4]
    Cho hàm số $y=f(x)$ xác định trên $(-\infty;2) \backslash\{-2\}$ và có bảng biến thiên như hình vẽ dưới đây.
    \begin{center}
        \begin{tikzpicture}
            \tikzset{double style/.append style = {draw=\tkzTabDefaultWritingColor,double=\tkzTabDefaultBackgroundColor,double distance=2pt}}
            \tkzTabInit[nocadre=false,lgt=1,espcl=3]
            {$x$ /0.7,$y'$ /0.7,$y$ /2.1}
            {$-\infty$,$-2$,$0$,$2$,$+\infty$}
            \tkzTabLine{,+,d,-,0,+,d,}
            \tkzTabVar{-/$-2$/,+D+/ $3$ / $+\infty$,-/$-2$/,+D/ $+\infty$ / }
            \draw[pattern = north west lines] ($(N43)+(0.1ex,0)$) rectangle (T21);
        \end{tikzpicture}
    \end{center}
    Xét tính đúng sai của các khẳng định sau:
    \choiceTF
    {\True Hàm số có giá trị nhỏ nhất bằng $-2$}
    {Hàm số có giá trị lớn nhất bằng $3$}
    {\True Đồ thị hàm số có hai đường tiệm cận đứng là $x=-2$ và $x=2$}
    {Đồ thị hàm số có hai đường tiệm cận ngang là $y=-2$ và $y=3$}
    \loigiai
    {
        Căn cứ vào bảng biến thiên của hàm số, ta có
        \begin{itemchoice}
            \itemch Hàm số đạt giá trị nhỏ nhất bằng $-2$ khi $x=0$.
            \itemch Do $\lim\limits_{x\to 2^{-}}f(x)=+\infty$ nên hàm số không có giá trị lớn nhất.
            \itemch Do $\lim\limits_{x\to -2^{+}}f(x)=+\infty$ và $\lim\limits_{x\to 2^{-}}f(x)=+\infty$ nên đồ thị hàm số có hai đường tiệm cận đứng là $x=-2$ và $x=2$.
            \itemch Do $\lim\limits_{x\to -\infty}f(x)=-2$ nên đồ thị hàm số có hai đường tiệm cận ngang là $y=-2$.
    \end{itemchoice}}
\end{ex}

\begin{ex}
    Cho hàm số $y=f(x)=\dfrac{5 x^2+9 x+9}{x-4}$. Xét tính đúng sai của các khẳng định sau:
    \choiceTF
    {Tập xác định của hàm số là $\mathbb{R}\backslash\{-4\}$}
    {\True Đường tiệm cận đứng của đồ thị hàm số có phương trình là $x=4$}
    {\True Đường tiệm cận xiên của đồ thị hàm số có phương trình là $y=5x+29$}
    {Giao điểm hai đường tiệm cận của đồ thị hàm số có toạ độ là $(4;29)$}
    \loigiai{
        Hàm số được viết thành $y=5x+29+\dfrac{125}{x-4}$.
        \begin{itemchoice}
            \itemch Điều kiện $x-4 \ne 0 \Leftrightarrow x \ne 4$. Suy ra $D=\mathbb{R}\backslash\{4\}$.
            \itemch Đường tiệm cận đứng của đồ thị hàm số có phương trình là $x=4$
            \itemch Đường tiệm cận xiên của đồ thị hàm số có phương trình là $y=5x+29$
            \itemch Giải hệ $\heva{&x=4\\&y=5x+29}\Leftrightarrow \heva{&x=4\\&y=49}$. Giao điểm hai đường tiệm cận của đồ thị hàm số có toạ độ là $(4;49)$.
    \end{itemchoice}}
\end{ex}

\begin{ex}
    Cho hàm số $y=f(x)=\dfrac{x^2-4 x+7}{x-1}$. Xét tính đúng sai của các khẳng định sau:
    \choiceTF
    {\True Đường tiệm cận đứng của đồ thị hàm số có phương trình là $x=1$}
    {\True Đường tiệm cận xiên của đồ thị hàm số có phương trình là $y=x-3$}
    {\True Giao điểm hai đường tiệm cận của đồ thị hàm số có toạ độ là $(1 ;-2)$}
    {Diện tích tam giác tạo bởi đường tiệm cận xiên của đồ thị hàm số và hai trục toạ độ là $\dfrac{9}{4}$}
    \loigiai{
        Hàm số được viết thành $y=x-3+\dfrac{4}{x-1}$.
        \begin{itemchoice}
            \itemch Đường tiệm cận đứng của đồ thị hàm số có phương trình là $x=1$
            \itemch Đường tiệm cận xiên của đồ thị hàm số có phương trình là $y=x-3$
            \itemch Giải hệ $\heva{&x=1\\&y=x-3}\Leftrightarrow \heva{&x=1\\&y=-2}$. Giao điểm hai đường tiệm cận của đồ thị hàm số có toạ độ là $(1;-2)$.
            \itemch Giao của đường thẳng $d \colon y=x-3$ với các trục tọa độ lần lượt tại $A(0;-3)$ và $B(3;0)$.\\
            Diện tích tam giác $OAB$ là $S=\dfrac{1}{2} OA \cdot OB=\dfrac{9}{2}$.
    \end{itemchoice}}
\end{ex}

\BTTL

\begin{ex}%[2D1N4-1]Câu 1
    Cho hàm số $y=\dfrac{2x-1}{x+3}$. Gọi $x=m$ và $y=n$ lần lượt là đường tiệm cận đứng và tiệm cận ngang của đồ thị hàm số. Tính giá trị của biểu thức $P=\dfrac{2m-1}{n+3}$.\\
    \shortans[3]{$-1{,}4$}
    \loigiai{Ta có:\\
        $\bullet \underset{x \to -3}{\lim}\,y=\infty \Rightarrow x=-3$ là đường tiệm cận đứng.\\
        $\bullet \underset{x \to +\infty}{\lim}\,y=2$ và $\underset{x \to -\infty}{\lim}\,y=2 \Rightarrow y=2$ là đường tiệm cận ngang.\\
        Vậy $m=-3; n=2 \Rightarrow P=\dfrac{2\cdot(-3)-1}{2+3}=\dfrac{-7}{5}=-1{,}4$.}
\end{ex}

\begin{ex}%[2D1B4-3]
    Cho đồ thị $(C)\colon y=\dfrac{x-3}{x+2}$ có hai đường tiệm cận cắt nhau tại $I$. Với $O$ là gốc tọa độ, hãy tính độ dài đoạn thẳng $OI$ (làm tròn đến hàng phần trăm).\\
    \shortans[3]{$2{,}24$}
    \loigiai{
        Ta có tiệm cận đứng của đồ thị $(C)$ là $x=-2$ và tiệm cận ngang là $y=1$. Do đó $I(-2;1)$ là giao điểm của hai đường tiệm cận của đồ thị $(C)$.\\
        Ta có $OI=\sqrt{ (-2-0)^2+(1-0)^2}=\sqrt{5} \approx 2{,}24$.
    }
\end{ex}

\begin{ex}
    Nếu trong một ngày, một xưởng sản xuất được $x$ kilôgam sản phẩm thì chi phí trung bình (tính bằng nghìn đồng) cho một sản phẩm được cho bởi công thức:
    $$
    y=C(x)=\dfrac{50x+2000}{x}
    $$
    Đồ thị hàm số $C(x)$ có một đường tiệm cận ngang (khi $x \to +\infty$) là $y=y_0$. Giá trị $y_0$ bằng bao nhiêu?\\
    \shortans[3]{$50$}
    \loigiai{
        Ta có $\lim\limits_{x \rightarrow+\infty} \dfrac{50x+2000}{x}=\lim\limits_{x \rightarrow+\infty} \left(50+\dfrac{2000}{x}\right)=50$.\\
        Vậy đường thẳng $y=50$ là tiệm cận ngang của đồ thị hàm số.
    }
\end{ex}

\begin{ex}%[2D1C4-3]Câu 6
    Cho hàm số $y=\dfrac{x-2}{x^2-3mx+m}$ tìm $m$ để đồ thị hàm số có đúng một tiệm cận đứng. Biết tổng các giá trị của tham số $m$ có dạng phân số $\dfrac{a}{b}$, tính tổng $S=a+b$.\\
    \shortans[3]{$101$}
    \loigiai{Dễ thấy tử thức có một nghiệm là $x=2$ do đó để đồ thị hàm số có đúng một tiệm cận đứng thì phương trình $x^2-3mx+m=0$ có nghiệm kép hoặc có hai nghiệm phân biệt trong đó có một nghiệm bằng $2$.\\
        $\Rightarrow \hoac{&\Delta=0\\&\heva{&\Delta>0\\&4-3\cdot2m+m=0}} \Leftrightarrow \hoac{&9m^2-4m=0\\&\heva{&9m^2-4m>0\\&4-3\cdot2m+m=0}} \Leftrightarrow \hoac{&m=0\\&m=\dfrac{4}{9}\\&\heva{&\hoac{&m<0\\&m>\dfrac{4}{9}}\\&m=\dfrac{4}{5}}} \Leftrightarrow \hoac{&m=0\\&m=\dfrac{4}{9}\\&m=\dfrac{4}{5}}$\\
        Vậy tổng các giá trị của tham số $m$ bằng $\dfrac{56}{45} \Rightarrow S=101$.
    }
\end{ex}

\begin{ex}%[2D1H4-1]Câu 4
    Cho hàm số $y=\dfrac{x^2+2x-3}{x-2}$, đồ thị hàm số có đường tiệm cận xiên có dạng $(C) \colon y=ax+b$. Tính giá trị của biểu thức $P=\dfrac{a}{b}$.\\
    \shortans[3]{$0{,}25$}
    \loigiai{Ta xét $y=\dfrac{x^2+2x-3}{x-2}=x+4+\dfrac{5}{x-2} \Rightarrow (C) \colon y=x+4$ là đường tiệm cận xiên của đồ thị hàm số. Vậy $P=\dfrac{1}{4}$.}
\end{ex}

\begin{ex}
    Gọi $d$ là đường tiệm cận xiên của đồ thị hàm số $y=mx+4-3m+\dfrac{3}{x+2}$, $m$ là tham số. Đường thẳng $d$ luôn qua điểm cố định $M$. Tính độ dài đoạn $OM$, với $O$ là gốc tọa độ.\\
    \shortans[3]{$5$}
    \loigiai{
        Đường tiệm cần xiên của đồ thị là $y=mx+4-3m \Leftrightarrow (x-3)m+4-y=0$.\\
        Ta có $(x-3)m+4-y=0,\,\forall m \Leftrightarrow \heva{&x-3=0\\&4-y=0} \Leftrightarrow \heva{&x=3\\&y=4}$.\\
        Đường thẳng này luôn qua điểm cố định $M(3;4)$. Khi đó $OM=\sqrt{3^2+4^2}=5$.
    }
\end{ex}
%\boxde
\BTTN
\Opensolutionfile{ans}[ans/2D1-4-DEON-1]
\begin{ex}%[2D1B4-1]
    Cho hàm số $y=f(x)$ có bảng biến thiên như hình bên. Đồ thị hàm số đã cho có tiệm cận ngang là đường thẳng
    \begin{center}
        \begin{tikzpicture}[scale=0.8, font=\footnotesize, line join=round, line
            cap=round, >=stealth]
            \tkzTabInit[espcl=2.5,lgt=1,nocadre=false]
            {$x$/0.7,$f(x)$/2.1}
            {$-\infty$,$0$,$1$,$2$,$+\infty$}
            \tkzTabVar{-/$-\infty$,+/$2$,-D+/$-\infty$/$+\infty$,-/$4$,+/$6$}
        \end{tikzpicture}
    \end{center}
    \choice
    {$y=2$}
    {$y=1$}
    {\True $y=6$}
    {$y=4$}
    \loigiai{Dựa vào bảng biến thiên ta thấy đồ thị hàm có tiệm cận ngang $y=6$.
    }
\end{ex}
%56
\begin{ex}%[2D1B4-1]
    Cho hàm số $y=f(x)$ có bảng biến thiên như hình bên. Tổng số tiệm cận đứng và tiệm cận ngang của đồ thị hàm số đã cho là
    \begin{center}
        \begin{tikzpicture}[scale=0.8]
            \tkzTabInit[nocadre=false,lgt=1.5,espcl=3,deltacl=0.6]
            {$x$ /0.6,$y’$ /0.6,$y$ /2}
            {$-\infty$ , $1$, $+\infty$}
            \tkzTabLine{,+,d,+,}
            \tkzTabVar{-/$2$,+D-/$+\infty$/$3$,+/$5$}
        \end{tikzpicture}
    \end{center}
    \choice
    {$1$}
    {\True $3$}
    {$2$}
    {$4$}
    \loigiai{Dựa vào bảng biến thiên ta thấy đồ thị hàm số có tiệm cận đứng $x=1$ và tiệm cận ngang $y=2$ và $y=5$.}
\end{ex}
\begin{ex}%[2D1B4-1]
    Cho hàm số $y=f(x)$ có bảng biến thiên như hình bên. Tổng số tiệm cận đứng và tiệm cận ngang của đồ thị hàm số đã cho là
    \begin{center}
        \begin{tikzpicture}[scale=0.8]
            \tkzTabInit[nocadre=false,lgt=1.5,espcl=3,deltacl=0.6]
            {$x$ /0.6,$y’$ /0.6,$y$ /2}
            {$-\infty$ ,$0$, $1$, $+\infty$}
            \tkzTabLine{,+,0,-,d,-,}
            \tkzTabVar{-/$4$,+/$2$,-D+/$-1$/$+\infty$,-/$-3$}
        \end{tikzpicture}
    \end{center}
    \choice
    {$1$}
    {\True $3$}
    {$2$}
    {$4$}
    \loigiai{
        Dựa vào bảng biến thiên ta thấy đồ thị hàm số có tiệm cận đứng $x=1$, tiệm cận ngang $y=4$ và $y=-3$.
    }
\end{ex}
%61
\begin{ex}%[2D1B4-1]
    Cho hàm số $y=f(x)$ có bảng biến thiên như hình bên. Tổng số tiệm cận đứng và tiệm cận ngang của đồ thị hàm số đã cho là
    \begin{center}
        \begin{tikzpicture}[scale=0.8]
            \tkzTabInit[nocadre=false,lgt=1.5,espcl=3,deltacl=0.6]
            {$x$ /0.6,$y’$ /0.6,$y$ /2}
            {$-\infty$ ,$0$, $1$, $+\infty$}
            \tkzTabLine{,-,0,+,d,+,}
            \tkzTabVar{+/$5$,-/$-4$,+D-/$+\infty$/$-\infty$,+/$2$}
        \end{tikzpicture}
    \end{center}
    \choice
    {$1$}
    {\True $3$}
    {$2$}
    {$4$}
    \loigiai{Dựa vào bảng biến thiên ta thấy đồ thị hàm số có một tiệm cận đứng $x=1$, hai tiệm cận ngang $y=5$ và $y=2$.}
\end{ex}
\begin{ex}%[2D1B4-1]
    Đồ thị hàm số nào trong các hàm số dưới đây có tiệm cận đứng?
    \choice
    {\True $y=\dfrac{1}{\sqrt{x}}$}
    {$y=\dfrac{1}{x^2+x+1}$}
    {$y=\dfrac{1}{x^4+1}$}
    {$y=\dfrac{1}{x^2+1}$}
    \loigiai{
    }
\end{ex}
\begin{ex}%[2D1K4-1]
    Số tiệm cận đứng của đồ thị hàm số $y=\dfrac{\sqrt{x+4}-2}{x^2+x}$ là
    \choice
    {$3$}
    {$0$}
    {$2$}
    {\True $1$}
    \loigiai{
        Tập xác định hàm số $ \mathscr{D}=[-4;+\infty)\setminus\lbrace -1;0\rbrace
        $.\\
        Ta có $ \lim\limits_{x\to -1^{+}}y=+\infty $, $ \lim\limits_{x\to 0^{+}}y=1
        $ và $ \lim\limits_{x\to 0^{-}}y=1 $.\\
        Suy ra đồ thị hàm số chỉ có $ 1 $ tiệm cận đứng là $ x=-1 $.
    }
\end{ex}
\begin{ex}%[Nguyễn Văn Sang, dự án Tex hoá đề cương trường Marie Curie - Lần 6]%[2D1Y4-1]
    Đường thẳng nào dưới đây là tiệm cận ngang của đồ thị hàm số $y=\dfrac{3+2 x}{x+1}$?
    \choice
    {$y=3$}
    {$x=-1$}
    {\True $y=2$}
    {$x=2$}
    \loigiai{
        Tập xác định $\mathscr{D}=\mathbb{R}\setminus\left\lbrace -1\right\rbrace$.
        \begin{itemize}
            \item $\lim\limits_{x \to \pm\infty} y=\lim\limits_{x \to \pm\infty} \dfrac{3+2 x}{x+1}=2$ suy ra $y=2$ là tiệm cận ngang.
            \item $\heva{& \lim\limits_{x \to -1^+} \dfrac{3+2 x}{x+1}=+\infty \\ & \lim\limits_{x \to -1^-} \dfrac{3+2 x}{x+1}=-\infty}$ suy ra $x=-1$ là tiệm cận đứng.
        \end{itemize}
    }
\end{ex}
%%=====Câu 15
\begin{ex}%[2D1Y4-1]
    Giao điểm của tiệm cận đứng và tiệm cận ngang của đồ thị hàm số $y=\dfrac{3x-2}{1-x}$ là điểm
    \choice
    {$M(1;3)$}
    {$P(-3;1)$}
    {\True $Q(1;-3)$}
    {$N\left(\dfrac{2}{3};3\right)$}
    \loigiai{
        Tiệm cận đứng, tiệm cận ngang của đồ thị hàm số lần lượt là $x=1$ và $y=-3$. Giao điểm của $2$ tiệm cận là $Q(1;-3)$.
    }
\end{ex}
\begin{ex}%[2D1K4-2]%
    Nếu đồ thị hàm số $y=\dfrac{(m+1)x+2}{x-n+1}$ lần lượt nhận trục hoành và trục tung làm đường đường tiệm cận ngang và tiệm cận đứng thì $m+n$ bằng bao nhiêu?
    \choice
    {\True $m+n=0$}
    {$m+n=2$}
    {$m+n=-1$}
    {$m+n=1$}
    \loigiai{
        Theo đề bài, ta có $\heva{&m+1=0\\&n-1=0} \Leftrightarrow \heva{&m=-1\\&n=1.}$\\
        Suy ra $m+n=0$.
    }
\end{ex}
\begin{ex}%[2D1K4-1]
    Cho hàm số $y=f(x)$ có bảng biến thiên như hình bên. Đồ thị hàm số $y=\dfrac{x-2}{f(x)-1}$ có bao nhiêu tiệm cận đứng?
    \begin{center}
        \begin{tikzpicture}
            \tkzTabInit[espcl=3]{$x$ / 1 , $f’(x)$ / 1, $f(x)$ / 2}
            {$-\infty$, $-1$ , $5$, $+\infty$}%
            \tkzTabLine{,-,0,+,0,-,}%
            \tkzTabVar{+/ $+\infty$, - / $-1$, + / $3$,-/$-2$}%
            \tkzTabVal[draw]{2}{3}{0.4}{$2$}{$1$}
        \end{tikzpicture}
    \end{center}
    \choice
    {$1$}
    {$3$}
    {\True $2$}
    {$4$}
    \loigiai{
        Dựa vào bảng biến thiên suy ra
        $f(x)-1=0 \Leftrightarrow f(x) =1$, phương trình này có $2$ nghiệm phân biệt khác $2$ và một nghiệm $x=2$ nên đồ thị hàm số $y=\dfrac{x-2}{f(x)-1}$ có hai tiệm cận đứng.
    }
\end{ex}
%68
\begin{ex}%[2D1K4-1]
    Cho hàm số $y=f(x)$ có bảng biến thiên như hình bên. Đồ thị hàm số $y=\dfrac{1}{2f(x)+1}$ có bao nhiêu tiệm cận đứng?
    \begin{center}
        \begin{tikzpicture}[scale=0.8]
            \tkzTabInit[nocadre=false,lgt=1.5,espcl=3,deltacl=0.6]
            {$x$ /0.6,$y’$ /0.6,$y$ /2}
            {$-\infty$ ,$-2$, $2$, $+\infty$}
            \tkzTabLine{,+,0,-,0,+,}
            \tkzTabVar{-/$-\infty$,+/$3$,-/$0$,+/$+\infty$}
        \end{tikzpicture}
    \end{center}
    \choice
    {\True $1$}
    {$3$}
    {$2$}
    {$0$}
    \loigiai{
        Dựa vào bảng biến thiên suy ra
        $2f(x)+1=0 \Leftrightarrow f(x) =-\dfrac{1}{2}$, phương trình này có $1$ nghiệm nên đồ thị hàm số $y=\dfrac{1}{2f(x)+1}$ có một tiệm cận đứng.
    }
\end{ex}
\begin{ex}%[2D1K4-1]
    Cho hàm số $y=f(x)$ có bảng biến thiên như hình bên. Đồ thị hàm số $y=\dfrac{1}{2f(x)-1}$ có bao nhiêu tiệm cận ngang?
    \begin{center}
        \begin{tikzpicture}[scale=0.8]
            \tkzTabInit[nocadre=false,lgt=1.5,espcl=3,deltacl=0.6]
            {$x$ /0.6,$y’$ /0.6,$y$ /2}
            {$-\infty$, $2$, $+\infty$}
            \tkzTabLine{,-,0,+,}
            \tkzTabVar{+/$1$,-/$-3$,+/$1$}
        \end{tikzpicture}
    \end{center}
    \choice
    {$1$}
    {\True $3$}
    {$2$}
    {$0$}
    \loigiai{
        Dựa vào bảng biến thiên suy ra
        \begin{itemize}
            \item 	$\lim \limits_{x \to \pm \infty} f(x)=1 \Leftrightarrow \lim \limits_{x \to \pm \infty}\dfrac{1}{2f(x)-1} =1$ nên đồ thị hàm số đã cho có tiệm cận ngang là $y=1$.
            \item $2f(x)-1=0 \Leftrightarrow f(x)=\dfrac{1}{2}$, phương trình này có $2$ nghiệm phân biệt nên đồ thị hàm số đã cho có hai tiệm cận đứng.
        \end{itemize}
    }
\end{ex}
%80
\begin{ex}%[2D1K4-1]
    Cho hàm số $y=f(x)$ có bảng biến thiên như hình bên. Đồ thị hàm số $y=\dfrac{1}{f^2(x)+f(x)}$ có bao nhiêu tiệm cận đứng?
    \begin{center}
        \begin{tikzpicture}[scale=0.8]
            \tkzTabInit[nocadre=false,lgt=1.5,espcl=3,deltacl=0.6]
            {$x$ /0.6,$y’$ /0.6,$y$ /2}
            {$-\infty$ ,$-4$, $6$, $+\infty$}
            \tkzTabLine{,-,0,+,0,-,}
            \tkzTabVar{+/$+\infty$,-/$-2$,+/$5$,-/$-\infty$}
        \end{tikzpicture}
    \end{center}
    \choice
    {$4$}
    {$3$}
    {$2$}
    {\True $6$}
    \loigiai{
        Dựa vào bảng biến thiên suy ra $f^2(x)+f(x)=0 \Leftrightarrow \hoac{&f(x)=0\\&f(x)=-1}$, mỗi phương trình này có $3$ nghiệm phân biệt nên đồ thị hàm số đã cho có $6$ tiệm cận đứng.
    }
\end{ex}
\begin{ex}%[2D1K4-1]
    Cho hàm số $y=f(x)$ có bảng biến thiên như hình bên. Đồ thị hàm số $y=\dfrac{3}{f(x^2)+1}$ có bao nhiêu tiệm cận đứng?
    \begin{center}
        \begin{tikzpicture}[scale=0.8]
            \tkzTabInit[nocadre=false,lgt=1.5,espcl=3,deltacl=0.6]
            {$x$ /0.6,$y’$ /0.6,$y$ /2}
            {$-\infty$ ,$0$, $2$, $+\infty$}
            \tkzTabLine{,+,d,-,0,+,}
            \tkzTabVar{-/$-\infty$,+/$1$,-/$-2$,+/$+\infty$}
        \end{tikzpicture}
    \end{center}
    \choice
    {\True $4$}
    {$3$}
    {$6$}
    {$0$}
    \loigiai{
        Dựa vào bảng biến thiên suy ra
        $f(x^2)+1=0 \Leftrightarrow f(x^2) =-1$. Kẻ đường thẳng $y=-1$ ta thấy đường thẳng cắt đồ thị hàm số tại 3 điểm phân biệt. Suy ra
        $$\hoac{&x^2=a \; (a<0)\\&x^2=b \; (b \in (0;2)\\&x^2=c \; (c>2)} \Rightarrow \hoac{&x=\pm \sqrt{b}\\&x=\pm \sqrt{c}.}$$
        Do đó đồ thị hàm số $y=\dfrac{2}{f(x^2)+1}$ có $4$ tiệm cận đứng.
    }
\end{ex}
\BTTF
\begin{ex}%[EX-TF-2024, Lê Đạt]%[2D1N4-1]
    Cho hàm số $y=\dfrac{2x-3}{x-1}$. Xét tính đúng sai các khẳng định dưới đây
    \choiceTF
    {\True Đường tiệm cận đứng của đồ thị hàm số là $ x=1 $}
    {Đường tiệm cận đứng của đồ thị hàm số là $ y=2 $}
    {Đường tiệm cận ngang của đồ thị hàm số là $ x=1 $}
    {\True Đường tiệm cận ngang của đồ thj hàm số là $ y=2 $}
    \loigiai{
        Ta có $\lim\limits_{x\to -\infty}y=\lim\limits_{x\to +\infty}y=2$ nên đồ thị hàm số đã cho có tiệm cận ngang là $y=2$.\\
        Ta có $\lim\limits_{x\to 1^+}y=-\infty$ nên đồ thị hàm số đã cho có tiệm cận ngang là $ x=1 $.
        \begin{itemchoice}
            \itemch Đường tiệm cận đứng của đồ thị hàm số là $ x=1 $.
            \itemch Đường tiệm cận đứng của đồ thị hàm số là $ x=1 $.
            \itemch Đường tiệm cận ngang của đồ thj hàm số là $ y=2 $.
            \itemch Đường tiệm cận ngang của đồ thj hàm số là $ y=2 $.
        \end{itemchoice}
    }
\end{ex}
%===== DẠNG 2
\begin{ex}%[EX-TF-2024, Lê Đạt]%[2D1H4-2]
    Cho hàm số $ y=\dfrac{m^2x+1}{x-1} $. Xét tính đúng sai của các khẳng định sau
    \choiceTF
    {\True Đồ thị hàm số luôn có tiệm cận ngang}
    {\True Đồ thị hàm số luôn có tiệm cận đứng}
    {\True Khi $ m=1$ đồ thị hàm số có $ 2 $ đường tiệm cận}
    {Khi $ m=0 $ đồ thị hàm số có $ 1 $ đường tiệm cận}
    \loigiai{
        \begin{itemchoice}
            \itemch $\lim\limits_{x\to -\infty}y=\lim\limits_{x\to +\infty}y=m^2$ suy ra hàm số luôn có tiệm cận ngang.
            \itemch $\lim\limits_{x\to 1^+}y=+\infty$ nên đồ thị hàm số đã cho có tiệm cận ngang là $ x=1 $.
            \itemch Khi $ m=1 $ ta được hàm số $ y=\dfrac{x+1}{x-1} $ suy ra đồ thì hàm số có $ x=1 $ là tiệm cận đứng và $ y=1 $ là tiệm cận ngang nên đồ thị hàm số có $ 2 $ tiệm cận.
            \itemch Khi $ m=0 $ ta được hàm số $ y=\dfrac{1}{x-1} $ suy ra đồ thì hàm số có $ x=1 $ là tiệm cận đứng và $ y=0 $ là tiệm cận ngang nên đồ thị hàm số có $ 2 $ tiệm cận.
        \end{itemchoice}
    }
\end{ex}
%===== DẠNG 3
\begin{ex}%[EX-TF-2024, Lê Đạt]%[2D1N4-3]
    \immini{Cho hàm số $y=f(x)$ có đồ thị như hình bên. Xét tính đúng sai của các khẳng định sau
        \choiceTF
        {$ x=2 $ là đường tiệm cận ngang của đồ thị hàm số}
        {\True $ x=-1 $ là đường tiệm cận đứng của đồ thị hàm số}
        {\True Đồ thị hàm số có hai đường tiệm cận}
        {\True Đồ thị hàm số không có tiệm cận xiên}
    }{
        \begin{tikzpicture}[scale=0.5, font=\footnotesize, line join=round, line cap=round, >=stealth]
            \draw[->](-5,0)--(5,0)node[below]{ $x$};
            \draw[->](0,-4)--(0,5)node[right]{ $y$};
            \draw [fill=black,draw=black] (0,0) circle (1pt)node[above left] { $O$};
            \foreach \x in {-1}\draw[shift={(\x,0)}](0pt,-2pt)--(0pt,2pt) node[below left]{ $\x$};
            \foreach \y in {2}\draw[shift={(0,\y)}](-2pt,0pt)--(2pt,0pt)node[above right]{ $\y$};
            \clip(-5,-4) rectangle (5,5);
            \draw[smooth,samples=100,domain=-5:-1.1] plot(\x,{(2*(\x)-1)/((\x)+1)});
            \draw[smooth,samples=100,domain=-0.9:5] plot(\x,{(2*(\x)-1)/((\x)+1)});
            \draw[dashed](-5,2)--(5,2) (-1,-4)--(-1,5);
        \end{tikzpicture}
    }
    \loigiai{
        \begin{itemchoice}
            \itemch $ y=2 $ là đường tiệm cận ngang của đồ thị hàm số.
            \itemch $ x=-1 $ là đường tiệm cận đứng của đồ thị hàm số.
            \itemch $ x=-1 $ là đường tiệm cận đứng và $ y=2 $ là đường tiệm cận ngang của đồ thị hàm số suy ra đồ thị hàm số có hai đường tiệm cận.
            \itemch Đồ thị hàm số không có tiệm cận xiên.
        \end{itemchoice}
    }
\end{ex}
\BTTL
\begin{ex}%[2D1K4-2]%
    Đường tiệm cận đứng và tiệm cận ngang của đồ thị hàm số $y=\dfrac{mx+1}{2m+1-x}$ cùng với hai trục tọa độ tạo thành một hình chữ nhật có diện tích bằng $3$. Khi đó $m$ bằng
    \shortans{$1$ hay $-\dfrac{3}{2}$}
    % \choice
    % {$1$ hay $\dfrac{3}{2}$}
    % {$-1$ hay $-\dfrac{3}{2}$}
    % {\True $1$ hay $-\dfrac{3}{2}$}
    % {$-1$ hay $3$}
    \loigiai{
        Từ yêu cầu đề bài, suy ra $|-m| \cdot |2m+1|=3 \Leftrightarrow \hoac{&m=1\\&m=-\dfrac{3}{2}.}$
    }
\end{ex}
\begin{ex}%[2D1K4-2]%
    Tìm tất cả các giá trị thực $m$ sao cho đồ thị hàm số $y=\dfrac{5x-3}{x^2-2mx+1}$ không có tiệm cận đứng.
    \shortans{$-1<m<1$}
    % \choice
    % {\True $-1<m<1$}
    % {$m=1$}
    % {$m=-1$}
    % {$m <-1$ hoặc $m>1$}
    \loigiai{
        Xét $f(x)=5x-3$, có $f(x)=0\Leftrightarrow x=\dfrac{3}{5}$; $g(x)=x^2-2mx+1$ có $\Delta’=m^2-1$.\\
        Đồ thị hàm số không có tiệm cận đứng khi phương trình $g(x)=0$ vô nghiệm $\Leftrightarrow m^2-1<0\Leftrightarrow-1<m<1$.\\
        Vậy với $-1<m<1$ thì đồ thị hàm số đã cho không có tiệm cận đứng.}
\end{ex}
\begin{ex}%[Nguyễn Văn Sang, dự án VDC-Hàm số 2020 - Lần 2]%[2D1K4-2]%
    Cho hàm số $y=\dfrac{1+\sqrt{x+1}}{\sqrt{x^2-mx-3m}}$ với $m$ là tham số. Tìm tập hợp các giá trị của tham số $m$ để đồ thị hàm số có hai tiệm cận đứng.
    \shortans{$\left(0;\dfrac{1}{2}\right)$}
    % \choice
    % {\True $\left(0;\dfrac{1}{2}\right)$}
    % {$\left(\left. 0;\dfrac{1}{2}\right]\right.$}
    % {$\left(0;+\infty \right)$}
    % {$\left(-\infty;-12\right)\cup \left(0;+\infty \right)$}
    \loigiai{
        Ta có $\sqrt{x+1}$ xác định khi $x\ge-1.$\\
        Yêu cầu bài toán $\Leftrightarrow $ phương trình $x^2-mx-3m=0$ có hai nghiệm phân biệt $x_1$, $x_2$ thỏa mãn $$-1<x_1<x_2\Leftrightarrow \heva{
            & \Delta >0 \\
            & x_1+x_2>-2 \\
            & a\cdot f\left(-1\right)>0 \\}\Leftrightarrow \heva{
            & m^2+12m>0 \\
            & m>-2 \\
            & 1\cdot \left(1-2m\right)>0 \\}\Leftrightarrow 0<m<\dfrac{1}{2}.$$
    }
\end{ex}
\begin{ex}%[VDC5-NgocDungHo]%[2D1G4-3]%
    \immini{Cho hàm số $f(x)$ có đồ thị như hình bên. Số đường tiệm cận đứng của đồ thị hàm số $y=\dfrac{(x^2-4)(x^2+2x)}{[f(x)]^2+2f(x)-3}$ là bao nhiêu?
        \shortans{$4$}
        % \choice
        % {\True $4$}
        % {$5$}
        % {$3$}
        % {$2$}
    }{\begin{tikzpicture}[>=stealth,scale=0.5, line join=round, line cap=round]
            \def\f[#1]{0.25*((#1)^4-8*(#1)^2+4)}
            \draw[->] (-4.1,0)--(4,0) node [below]{$x$};
            \draw[->] (0,-3.5)--(0,4) node [left]{$y$};
            \node at (0,0) [below left]{$O$};
            % \clip;
            \draw[domain=-3:3,samples=300,thick] plot (\x,{\f[\x]});
            \foreach \x in {-2,2} \filldraw (\x,0) node[above]{\x} circle (2pt);
            \foreach \x in {-3,3} \filldraw (\x,0) node[below]{\x} circle (2pt);
            \filldraw (0,1) node[above left]{$1$} circle (2pt);
            \filldraw (0,-3) node[below left]{$-3$} circle (2pt);
            \draw[dashed](-2,0)--(-2,-3)--(2,-3)--(2,0);
            \draw (3,2.75) node[right]{$y=f(x)$};
    \end{tikzpicture}}
    \loigiai{%GV tổng quát hóa bài toán:
        Cho hàm số $f(x)$ có đồ thị $(C)$ cho trước. Xác định số đường tiệm cận đứng của đồ thị hàm số $y=\dfrac{u(x)}{v[f(x)]}$.
        \begin{enumerate}
            \item Tìm tập xác định của hàm số $y=\dfrac{u(x)}{v[f(x)]}$.\\
            \item Tìm nghiệm của phương trình $u(x)=0\quad (1)$.\\
            \item Tìm nghiệm của phương trình $v[f(x)]=0\quad (2)$. Giả sử $f(x)=m_1$, $f(x)=m_2,\ldots$.
        \end{enumerate}
        Dựa vào đồ thị $(C)$, xác định hoành độ giao điểm của $(C)$ với các đường thẳng $d_1\colon f(x)=m_1$, $d_2\colon f(x)=m_2,\ldots$.\\
        Số đường tiệm cận đứng của đồ thị hàm số $y=\dfrac{u(x)}{v[f(x)]}$ chính là tổng của:
        \begin{itemize}
            \item Số nghiệm riêng của phương trình $(2)$.
            \item Số nghiệm chung $x=x_0$ của $(1) $ và $(2)$ mà bậc của $(x-x_0)$ ở mẫu lớn hơn bậc của $(x-x_0)$ ở tử.
        \end{itemize}
        \noindent
        Xét hàm số $y=g(x)=\dfrac{(x^2-4)(x^2+2x)}{[f(x)]^2+2f(x)-3}$.
        \immini
        {
            Giải phương trình $(x^2-4)(x^2+2x)=0\,(1)$\\$ \Leftrightarrow \hoac{& x^2-4=0 \\ & x^2+2x=0}\Leftrightarrow \hoac{& x=\pm 2 \\ & x=0.}$\\
            Giải phương trình $[f(x)]^2+2f(x)-3=0\,(2)$\\
            $ \Leftrightarrow \hoac{& f(x)=1 \\ & f(x)=-3.}$\\
        }
        {\begin{tikzpicture}[>=stealth,scale=0.7, line join=round, line cap=round]
                \def\f[#1]{0.25*((#1)^4-8*(#1)^2+4)}
                \def\g[#1]{1}
                \def\h[#1]{-3}
                \draw[->] (-4.1,0)--(4,0) node [below]{$x$};
                \draw[->] (0,-3.5)--(0,4) node [left]{$y$};
                \node at (0,0) [below left]{$O$};
                % \clip;
                \draw[domain=-3:3,samples=300,thick] plot (\x,{\f[\x]});
                \draw[domain=-4:4,samples=300,thick] plot (\x,{\g[\x]});
                \draw[domain=-4:4,samples=300,thick] plot (\x,{\h[\x]});
                \foreach \x in {-2,2} \filldraw (\x,0) node[above]{\x} circle (2pt);
                \filldraw (-3,0) node[above left]{$-3$} circle (2pt);
                \filldraw (3,0) node[above right]{$3$} circle (2pt);
                \filldraw (-2.85,0) node[below]{$a$} circle (2pt);
                \filldraw (2.85,0) node[below]{$b$} circle (2pt);
                \filldraw (0,1) node[above left]{$1$} circle (2pt);
                \filldraw (0,-3) node[below left]{$-3$} circle (2pt);
                \draw[dashed](-2,0)--(-2,-3)--(2,-3)--(2,0) (-2.85,0)--(-2.85,1) (2.85,0)--(2.85,1);
                \draw (3,2.75) node[right]{$(C):y=f(x)$};
                \draw (4.2,1) node[above]{$d_1:y=1$};
                \draw (4,-3) node[below]{$d_2:y=-3$};
            \end{tikzpicture}
        }
        Dựa vào đồ thị đã cho $(2)\Leftrightarrow \hoac{& x = \pm 2 \\ & x=0\\&x=a\\&x=b.}$
        với $-3<a<-2<2<b<3$.\\
        Trong điều kiện xác định của hàm số $y=g(x)$ ta có thể viết $$y=g(x)=\dfrac{x(x-2)(x+2)^2}{x^2(x-a)(x-b)(x-2)^2(x+2)^2}=\dfrac{1}{x(x-a)(x-b)(x-2)}$$
        Vậy số tiệm cận đứng của đồ thị hàm số $y=g(x)$ bằng $4$.
    }
\end{ex}
\begin{ex}
    \immini{%Câu 97.
        Đường cong ở hình bên là đồ thị của hàm số $y = ax^3 +bx^2 +cx+d$. Đồ thị hàm số $y =\dfrac{(x+1)\sqrt{1-x}}{f(x^2)}$ có tất cả bao nhiêu tiệm cận đứng?
        \shortans{$2$}
        % \choice
        % {1}
        % {6}
        % {4}
        % {\True 2}
    }{\begin{tikzpicture}[scale=.6, font=\footnotesize, line join=round, line cap=round, >=stealth]
            \def\xmin{-2}\def\xmax{4}\def\ymin{-3}\def\ymax{3}
            \draw[->] (\xmin-0.2,0)--(\xmax+0.2,0) node[below] {\footnotesize $x$};
            \draw[->] (0,\ymin-0.2)--(0,\ymax+0.2) node[right] {\footnotesize $y$};
            \draw (0,0) node [below left] {\footnotesize $O$};
            \foreach \x in {1,2}\draw (\x,-0.1)--(\x,0.1) node [above ] {\footnotesize $\x$};
            \foreach \x in {-1,3}\draw (\x,-0.1)--(\x,0.1) node [above left] {\footnotesize $\x$};
            \foreach \y in {-2}\draw (0.1,\y)--(-0.1,\y) node [left] {\footnotesize $\y$};
            \foreach \y in {2}\draw (-0.1,\y)--(0.1,\y) node [right] {\footnotesize $\y$};
            \clip (\xmin,\ymin) rectangle (\xmax,\ymax);
            \draw[smooth,samples=200,domain=\xmin:\xmax] plot (\x,{0.6666666666666666*((\x)^3)+-2*((\x)^2)+-0.6666666666666666*(\x)+2});
            \draw[dashed] (1.0,0)--(1.0,0.0)--(0,0.0);\fill (1.0,0.0) circle (1pt);
            \draw[dashed] (2,0)--(2,-2)--(0,-2);
    \end{tikzpicture}}
    \loigiai{
        * Điều kiện: $\heva{&f(x^2) \ne 0\\&x \le 1.}$\\
        Nhìn hình vẽ ta thấy
        $f(x^2)=0\Leftrightarrow \hoac{&x^2=-1\\&x^2=1\\&x^2=3}\Leftrightarrow \hoac{&x=\pm 1\,(\text{nghiệm đơn})\\&x=- \sqrt{3}\,(\text{nghiệm đơn})\\&x= \sqrt{3}\,(\text{không thỏa mãn})}.$\\
        Vậy $y=\dfrac{(x+1)\sqrt{1-x}}{f(x^2)}=\dfrac{(x+1)\sqrt{1-x}}{(x - 1)(x + 1)(x + \sqrt{3})}$ \\
        Đồ thị hàm số có 2 đường tiệm cận đứng.}
\end{ex}
\begin{ex}
    \immini{ %Câu 95.
        Đường cong ở hình bên là đồ thị của hàm số $y = ax^3 +bx^2 +cx+d$. Đồ thị hàm số $y =\dfrac{(2x+1)\sqrt{x-1}}{f(|x|)}$ có tất cả bao nhiêu tiệm cận đứng?
        \shortans{$1$}
        % \choice
        % {\True 1}
        % {3}
        % {4}
        % {2}
    }{\begin{tikzpicture}[scale=.5, font=\footnotesize, line join=round, line cap=round, >=stealth]
            \def\xmin{-3}\def\xmax{3}\def\ymin{-5}\def\ymax{5}
            \draw[->] (\xmin-0.2,0)--(\xmax+0.2,0) node[below] {\footnotesize $x$};
            \draw[->] (0,\ymin-0.2)--(0,\ymax+0.2) node[right] {\footnotesize $y$};
            \draw (0,0) node [below left] {\footnotesize $O$};
            \foreach \x in {-1,2}\draw (\x,0.1)--(\x,-0.1) node [below] {\footnotesize $\x$};
            \foreach \x in {-2,1}\draw (\x,-0.1)--(\x,0.1) node [above] {\footnotesize $\x$};
            \foreach \y in {-4,2}\draw (-0.1,\y)--(0.1,\y) node [right] {\footnotesize $\y$};
            \foreach \y in {-2,4}\draw (0.1,\y)--(-0.1,\y) node [left] {\footnotesize $\y$};
            \clip (\xmin,\ymin) rectangle (\xmax,\ymax);
            \draw[smooth,samples=200,domain=\xmin:\xmax] plot (\x,{1.3333333333333333*((\x)^3)+0*((\x)^2)+-3.3333333333333335*(\x)+0});
            \draw[dashed] (-2,0)--(-2,-4)--(0,-4);
            \draw[dashed] (2,0)--(2,4)--(0,4);
            \draw[dashed] (1,0)--(1,-2)--(0,-2);
            \draw[dashed] (-1,0)--(-1,2)--(0,2);
    \end{tikzpicture}}
    \loigiai{
        * Điều kiện: $\heva{&f(|x|) \ne 0\\&x \ge 1.}$\\
        Nhìn hình vẽ ta thấy
        $f(|x|)=0\Leftrightarrow \hoac{&|x|=x_1\,(-2<x_1<-1)\\&|x|=0\\&|x|=x_2\,(1<x_2<2)}\Leftrightarrow \hoac{&x=0&(\text{không thỏa mãn})\\&x=- x_2&(\text{không thỏa mãn})\\&x=x_2&(\text{nghiệm đơn}).}$\\
        Vậy $y =\dfrac{(2x+1)\sqrt{x-1}}{f(|x|)}=\dfrac{(2x+1)\sqrt{x-1}}{ax(x+x_2)(x-x_2)}.$ \\
        Đồ thị hàm số có 1 đường tiệm cận đứng.}
\end{ex}
\begin{ex}
    Đáp ứng tần số của một hệ thống điều khiển có thể được mô tả bởi hàm truyền \( H(s) = \dfrac{\omega_n^2}{s^2 + 2\zeta\omega_ns + \omega_n^2} \), trong đó \( \omega_n \) là tần số tự nhiên và \( \zeta \) là hệ số tắt dần. Tìm đường tiệm cận ngang của đáp ứng tần số khi tần số góc \( s \) tăng và nêu ý nghĩa của nó.
    \shortans{$y=0$}
    \loigiai{
        Khi \( s \) tăng vô hạn, các thành phần bậc cao trong mẫu số chiếm ưu thế:
        \[
        H(s) \approx \frac{\omega_n^2}{s^2}
        \]
        Đường tiệm cận ngang của \( H(s) \) khi \( s \to \infty \) là:
        \[
        |H(s)| \approx \frac{\omega_n^2}{s^2} \to 0
        \]}
\end{ex}
\begin{ex}
    Trong thuyết tương đối của Einstein, khối lượng của vật chuyển động với vận tốc $v$ được cho bởi công thức:
    $$m(v)=\dfrac{m_0}{\sqrt{1-\dfrac{v^2}{c^2}}},$$
    trong đó $m_0$ là khối lượng của vật khi nó đứng yên, $c$ là vận tốc ánh sáng.\\
    (nguồn: https://www.britannica.com/science/relativity/Relativistic-mass)\\
    Xem $m$ là hàm số theo vận tốc $v$, tìm đường tiệm cận đứng của đồ thị hàm số. Từ đó nhận xét khối lượng của vật khi vận tốc của nó càng gần với vận tốc ánh sáng.
    \shortans{$v=c$, khối lượng tăng lên vô hạn}
    \loigiai{
        Điều kiện xác định: $\heva{&1-\dfrac{v^2}{c^2}>0\\
            &v>0}\Leftrightarrow\heva{& -c<v<c\\
            &v>0}\Leftrightarrow 0<v<c$.\\
        Ta có $\lim\limits_{v\to c^{-}} m(v)=\lim\limits_{v\to c^{-}}\dfrac{m_0}{\sqrt{1-\dfrac{v^2}{c^2}}}=+\infty$ nên đường thẳng $v=c$ là tiệm cận đứng của đồ thị hàm số.\\
        Từ đó ta suy ra khi vận tốc của vật càng sát với vận tốc ánh sáng thì khối lượng của vật tăng lên vô hạn.
    }
\end{ex}
\Closesolutionfile{ans}
%\boxde
\BTTN
\Opensolutionfile{ans}[ans/2D1-4-DEON-2]
\begin{ex}%[2D1B4-1]
    Cho hàm số $y=f(x)$ có bảng biến thiên như hình bên. Tổng số tiệm cận đứng và tiệm cận ngang của đồ thị hàm số đã cho là
    \begin{center}
        \begin{tikzpicture}
            \tkzTabInit[nocadre=false,lgt=1.5,espcl=3,deltacl=0.6]
            {$x$ /0.6,$y’$ /0.6,$y$ /2}
            {$-\infty$ ,$0$, $1$, $+\infty$}
            \tkzTabLine{,-,d,+,0,-,}
            \tkzTabVar{+/$+\infty$,-D-/$-1$/$-\infty$,+/$2$,-/$-\infty$}
        \end{tikzpicture}
    \end{center}
    \choice
    {\True $1$}
    {$3$}
    {$2$}
    {$4$}
    \loigiai{
        Dựa vào bảng biến thiên ta thấy đồ thị hàm số có một tiệm cận đứng $x=0$.
    }
\end{ex}
%62
%63
\begin{ex}%[2D1B4-1]
    Cho hàm số $y=f(x)$ xác định, liên tục trên $\mathbb{R} \backslash \{0;1\}$ và có bảng biến thiên như hình bên. Đồ thị hàm số $y=f(x)$ có
    \begin{center}
        \begin{tikzpicture}
            \tkzTabInit[nocadre=false,lgt=1.5,espcl=3,deltacl=0.6]
            {$x$ /0.6,$y’$ /0.6,$y$ /2}
            {$-\infty$ ,$0$, $1$, $+\infty$}
            \tkzTabLine{,+,d,+,d,+,}
            \tkzTabVar{-/$-5$,+D-/$+\infty$/$-\infty$,+D-/$3$/$-\infty$,+/$+\infty$}
        \end{tikzpicture}
    \end{center}
    \choice
    {\True $2$ tiệm cận đứng và $1$ tiệm cận ngang}
    {$2$ tiệm cận đứng và $2$ tiệm cận ngang}
    {$1$ tiệm cận đứng và $1$ tiệm cận ngang}
    {$1$ tiệm cận đứng và $2$ tiệm cận ngang}
    \loigiai{
        Dựa vào bảng biến thiên ta thấy đồ thị hàm số có hai tiệm cận đứng $x=0$ và $x=1$; một tiệm cận ngang $y=-5$.
    }
\end{ex}
%64
\begin{ex}%[2D1B4-1]
    Cho hàm số $y=f(x)$ có bảng biến thiên như hình bên. Tổng số tiệm cận đứng và tiệm cận ngang của đồ thị hàm số đã cho là
    \begin{center}
        \begin{tikzpicture}[scale=0.8]
            \tkzTabInit[nocadre=false,lgt=1.5,espcl=3,deltacl=0.6]
            {$x$ /0.6,$y’$ /0.6,$y$ /2}
            {$-\infty$ ,$-1$, $1$, $+\infty$}
            \tkzTabLine{,+,d,+,0,-,}
            \tkzTabVar{-/$2$,+D-/$4$/$-\infty$,+/$3$,-/$-1$}
        \end{tikzpicture}
    \end{center}
    \choice
    {$1$}
    {\True $3$}
    {$2$}
    {$4$}
    \loigiai{Dựa vào bảng biến thiên ta thấy đồ thị hàm số có một tiệm cận đứng $x=-1$; hai tiệm cận ngang $y=-1$ và $y=2$.
    }
\end{ex}
%65
\begin{ex}%[2D1B4-1]
    Cho hàm số $y=f(x)$ có bảng biến thiên như hình bên. Tổng số tiệm cận đứng và tiệm cận ngang của đồ thị hàm số đã cho là
    \begin{center}
        \begin{tikzpicture}[scale=0.8]
            \tkzTabInit[nocadre=false,lgt=1.5,espcl=3,deltacl=0.6]
            {$x$ /0.6,$y’$ /0.6,$y$ /2}
            {$-\infty$ ,$0$, $3$, $+\infty$}
            \tkzTabLine{,-,0,+,d,-,}
            \tkzTabVar{+/$8$,-/$1$,+/$4$,-/$2$}
        \end{tikzpicture}
    \end{center}
    \choice
    {$1$}
    {$3$}
    {\True $2$}
    {$4$}
    \loigiai{
        Dựa vào bảng biến thiên ta thấy đồ thị hàm số có hai tiệm cận ngang $y=2$ và $y=8$.
    }
\end{ex}
\begin{ex}%[Nguyễn Văn Sang, dự án Tex hoá đề cương trường Marie Curie - Lần 6]%[2D1Y4-1]
    Đường thẳng nào dưới đây là tiệm cận đứng của đồ thị hàm số $y=\dfrac{2 x+1}{x+1}$?
    \choice
    {$x=1$}
    {$y=-1$}
    {$y=2$}
    {\True $x=-1$}
    \loigiai{
        Tập xác định $\mathscr{D}=\mathbb{R}\setminus\left\lbrace -1\right\rbrace$.
        \begin{itemize}
            \item $\lim\limits_{x \to \pm\infty} y=\lim\limits_{x \to \pm\infty} \dfrac{2x+1}{x+1}=2$ suy ra $y=2$ là tiệm cận ngang.
            \item $\heva{& \lim\limits_{x \to -1^+} \dfrac{2x+1}{x+1}=-\infty \\ & \lim\limits_{x \to -1^-} \dfrac{2x+1}{x+1}=+\infty}$ suy ra $x=-1$ là tiệm cận đứng.
        \end{itemize}
    }
\end{ex}
\begin{ex}%[2D1Y4-1]
    Đồ thị hàm số $y=\dfrac{2x-3}{2x+1}$ có tâm đối xứng là điểm
    \choice
    {\True $M\left(-\dfrac{1}{2};1\right)$}
    {$P\left(-\dfrac{1}{2};2\right)$}
    {$Q\left(-\dfrac{1}{2};-3\right)$}
    {$N\left(1;-\dfrac{1}{2}\right)$}
    \loigiai{
        Tiệm cận đứng, tiệm cận ngang của đồ thị hàm số lần lượt là $x=-\dfrac{1}{2}$ và $y=3$. Tâm đối xứng là điểm $M\left(-\dfrac{1}{2};1\right)$.
    }
\end{ex}
\begin{ex}%[2D1K4-1]
    Đồ thị hàm số $y=\dfrac{\sqrt{x}}{x+1}-\dfrac{1}{x}$ có tất cả bao nhiêu tiệm cận đứng và ngang?
    \choice
    {$0$}
    {$3$}
    {\True $2$}
    {$1$}
    \loigiai{
        Tập xác định $\mathscr{D}=(0;+\infty)$.
        \begin{itemize}
            \item $\lim\limits_{x\to 0^+} \left(\dfrac{\sqrt{x}}{x+1}-\dfrac{1}{x}\right)=-\infty$.
            \item $\lim\limits_{x\to +\infty}\left(\dfrac{\sqrt{x}}{x+1}-\dfrac{1}{x}\right)=0$.
        \end{itemize}
        Suy ra đồ thị hàm số có tiệm cận đứng $x=0$, tiệm cận ngang $y=0$.
    }
\end{ex}
\begin{ex}%[2D1K4-1]
    Số tiệm cận đứng của đồ thị hàm số $y=\dfrac{x^2-3x-4}{x^2-16}$ là
    \choice
    {$2$}
    {$3$}
    {\True $1$}
    {$0$}
    \loigiai{
        Điều kiện xác định $x \ne \pm 4$.\\
        Với điều kiện xác định trên, ta có $y=\dfrac{x^2-3x-4}{x^2-16}=\dfrac{(x+1)(x-4)}{(x-4)(x+4)}=\dfrac{x+1}{x+4}$.\\
        Tiệm cận đứng của đồ thị hàm số là $x=-4$.
    }
\end{ex}
\begin{ex}%[2D1K4-1]
    Số đường tiệm cận đứng và ngang của đồ thị hàm số $y=\dfrac{x-1}{x^2-x-2}$ là
    \choice
    {\True $3$}
    {$1$}
    {$0$}
    {$2$}
    \loigiai{
        Điều kiện xác định $x \ne -1$, $x \ne 2$.\\
        Với điều kiện xác định trên, ta có $y=\dfrac{x-1}{x^2-x-2}=\dfrac{x-1}{(x+1)(x-2)}$.\\
        Tiệm cận đứng của đồ thị hàm số là $x=-1$, $x=2$, tiệm cận ngang của đồ thị hàm số là $y=0$.
    }
\end{ex}
%81
\begin{ex}%[2D1K4-1]
    Cho hàm số $y=f(x)$ có bảng biến thiên như hình bên. Đồ thị hàm số $y=\dfrac{1}{f^2(x)-2f(x)}$ có bao nhiêu tiệm cận đứng?
    \begin{center}
        \begin{tikzpicture}[scale=0.8]
            \tkzTabInit[nocadre=false,lgt=1.5,espcl=3,deltacl=0.6]
            {$x$ /0.6,$y’$ /0.6,$y$ /2}
            {$-\infty$ ,$-1$, $2$, $+\infty$}
            \tkzTabLine{,+,d,-,0,+,}
            \tkzTabVar{-/$-\infty$,+/$1$,-/$-2$,+/$+\infty$}
        \end{tikzpicture}
    \end{center}
    \choice
    {\True $4$}
    {$3$}
    {$2$}
    {$6$}
    \loigiai{
        Dựa vào bảng biến thiên suy ra $f^2(x)-2f(x)=0 \Leftrightarrow \heva{&f(x)=0\\&f(x)=2}$, phương trình $f(x)=0$ có $3$ nghiệm phân biệt và phương trình $f(x)=2$ có $1$ nghiệm nên đồ thị hàm số đã cho có $4$ tiệm cận đứng.
    }
\end{ex}
%79
\begin{ex}%[2D1K4-1]
    Cho hàm số $y=f(x)$ có bảng biến thiên như hình bên. Tổng số tiệm cận ngang và tiệm cận đứng của đồ thị hàm số $y=\dfrac{2}{f(x)+3}$ là
    \begin{center}
        \begin{tikzpicture}[scale=0.8]
            \tkzTabInit[nocadre=false,lgt=1.5,espcl=3,deltacl=0.6]
            {$x$ /0.6,$y’$ /0.6,$y$ /2}
            {$-\infty$ ,$-4$, $6$, $+\infty$}
            \tkzTabLine{,-,0,+,0,-,}
            \tkzTabVar{+/$+\infty$,-/$-2$,+/$5$,-/$-\infty$}
        \end{tikzpicture}
    \end{center}
    \choice
    {$4$}
    {$3$}
    {\True $2$}
    {$1$}
    \loigiai{
        Dựa vào bảng biến thiên suy ra
        \begin{itemize}
            \item 	$\lim \limits_{x \to \pm \infty} f(x)=\pm \infty \Leftrightarrow \lim \limits_{x \to \pm \infty}\dfrac{2}{f(x)+3} =0$ nên đồ thị hàm số đã cho có tiệm cận ngang là $y=0$.
            \item $f(x)+3=0 \Leftrightarrow f(x) =-3$, phương trình này có $1$ nghiệm $x=a>6$ nên đồ thị hàm số đã cho có một tiệm cận đứng.
        \end{itemize}
    }
\end{ex}
\begin{ex}%[2D1K4-1]
    Cho hàm số $y=f(x)$ có bảng biến thiên như hình bên. Đồ thị hàm số $y=\dfrac{x+1}{f(x)-4}$ có bao nhiêu tiệm cận đứng?
    \begin{center}
        \begin{tikzpicture}[scale=0.8]
            \tkzTabInit[nocadre=false,lgt=1.5,espcl=3,deltacl=0.6]
            {$x$ /0.6,$y’$ /0.6,$y$ /2}
            {$-\infty$ ,$-1$, $2$, $+\infty$}
            \tkzTabLine{,+,0,-,0,+,}
            \tkzTabVar{-/$1$,+/$4$,-/$-5$,+/$+\infty$}
        \end{tikzpicture}
    \end{center}
    \choice
    {$1$}
    {$3$}
    {\True $2$}
    {$4$}
    \loigiai{
        Dựa vào bảng biến thiên suy ra
        $f(x)-4=0 \Leftrightarrow f(x) =4$, phương trình này có $1$ nghiệm khác $-1$ và một nghiệm bội chẵn $x=-1$ nên đồ thị hàm số $y=\dfrac{x+1}{f(x)-4}$ có hai tiệm cận đứng.
    }
\end{ex}
\begin{ex}%[2D1K4-1]
    Cho hàm số $y=f(x)$ có bảng biến thiên như hình bên. Đồ thị hàm số $y=\dfrac{x-5}{f(x)-1}$ có bao nhiêu tiệm cận đứng?
    \begin{center}
        \begin{tikzpicture}
            \tkzTabInit[espcl=3]{$x$ / 1 , $f’(x)$ / 1, $f(x)$ / 2}
            {$-\infty$, $-1$ , $2$, $+\infty$}%
            \tkzTabLine{,-,0,+,d,-,}%
            \tkzTabVar{+/ $+\infty$, - / $-1$, + / $3$,-/$-\infty$}%
            \tkzTabVal[draw]{3}{4}{0.4}{$5$}{$1$}%
            %\tkzTabVal[draw]{2}{3}{0.4}{$e^2$}{$1$}%
        \end{tikzpicture}
    \end{center}
    \choice
    {$1$}
    {$3$}
    {\True $2$}
    {$4$}
    \loigiai{
        Dựa vào bảng biến thiên suy ra
        $f(x)-1=0 \Leftrightarrow f(x) =1$, phương trình này có $2$ nghiệm phân biệt khác $5$ và một nghiệm $x=5$ nên đồ thị hàm số $y=\dfrac{x-5}{f(x)-1}$ có hai tiệm cận đứng.
    }
\end{ex}
\begin{ex}%[VDC5-NgocDungHo]%[2D1G4-3]%
    \immini
    {
        Cho hàm số $f(x)$ có đồ thị như hình bên. Số đường tiệm cận đứng của đồ thị hàm số $y=\dfrac{(x^2-1)(x^2+x)}{[f(x)]^2-2f(x)-3}$ là
        \choice
        {$4$}
        {$5$}
        {\True $3$}
        {$2$}
    }
    {
        \begin{tikzpicture}[>=stealth,scale=0.7, line join=round, line cap=round]
            \def\f[#1]{(#1)^3-3*(#1)+1)}
            \draw[->] (-2.2,0)--(2.4,0) node [below]{$x$};
            \draw[->] (0,-1.5)--(0,3.5) node [left]{$y$};
            \node at (0,0) [below left]{$O$};
            % \clip;
            \draw[domain=-2.1:2.1,samples=300,thick] plot (\x,{\f[\x]});
            \filldraw (-1,0) node[below]{$-1$} circle (2pt);
            \filldraw (1,0) node[above]{$1$} circle (2pt);
            \filldraw (0,-1) node[ left]{$1$} circle (2pt);
            \filldraw (0,3) node[ right]{$3$} circle (2pt);
            \draw[dashed](-1,0)--(-1,3)--(0,3) (1,0)--(1,-1)--(0,-1);
            \draw (2,3) node[right]{$y=f(x)$};
        \end{tikzpicture}
    }
    \loigiai{
        Xét hàm số $y=g(x)=\dfrac{(x^2-1)(x^2+x)}{[f(x)]^2-2f(x)-3}$.
        \immini
        {
            Giải phương trình $(x^2-1)(x^2+x)=0 \Leftrightarrow \hoac{& x^2-1=0 \\ & x^2+x=0}\Leftrightarrow \hoac{& x=\pm 1 \\ & x=0.}$\\
            Giải phương trình $[f(x)]^2-2f(x)-3=0$\\$ \Leftrightarrow \hoac{& f(x)=-1 \\ & f(x)=3} \Leftrightarrow \hoac{& x = \pm 1 \\ & x=a\\&x=b\;(a<-1<1<b).}$
        }
        {
            \begin{tikzpicture}[>=stealth,scale=0.7, line join=round, line cap=round]
                \def\f[#1]{(#1)^3-3*(#1)+1)}
                \def\g[#1]{3}
                \def\h[#1]{-1}
                \draw[->] (-2.5,0)--(4,0) node [below]{$x$};
                \draw[->] (0,-1.5)--(0,3.5) node [left]{$y$};
                \node at (0,0) [below left]{$O$};
                % \clip;
                \draw[domain=-2.5:4,samples=300,thick] plot (\x,{\g[\x]});
                \draw[domain=-2.5:4,samples=300,thick] plot (\x,{\h[\x]});
                \draw[domain=-2.1:2.1,samples=300,thick] plot (\x,{\f[\x]});
                \filldraw (-1,0) node[below]{$-1$} circle (2pt);
                \filldraw (1,0) node[above]{$1$} circle (2pt);
                \filldraw (0,-1) node[ left]{$1$} circle (2pt);
                \filldraw (0,3) node[ right]{$3$} circle (2pt);
                \draw[dashed](-1,0)--(-1,3)--(0,3) (1,0)--(1,-1) (2,0)node[below]{$b$}--(2,3) (-2,0)node[above]{$a$}--(-2,-1);
                \draw (2,2) node[right]{$y=f(x)$};
                \draw (3.3,3) node[above]{$d_1:y=3$};
                \draw (3,-1) node[below]{$d_2:y=-1$};
            \end{tikzpicture}
        }
        Trong điều kiện xác định của hàm số $y=g(x)$ ta có thể viết
        $$y=g(x)=\dfrac{x(x-1)(x+1)^2}{(x-a)(x-b)(x-1)^2(x+1)^2}=\dfrac{x}{(x-a)(x-b)(x-1)}$$
        Vậy số tiệm cận đứng của đồ thị hàm số $y=g(x)$ bằng $3$.
    }
\end{ex}
\begin{ex}
    \immini{ %Câu 91.
        Đường cong ở hình bên là đồ thị của hàm số $y = ax^4 + bx^2 +c$. Đồ thị hàm số $g(x) =\dfrac{(x^2-x)\sqrt{x+2}}{(x-2)\cdot f(x+1)}$
        có bao nhiêu đường tiệm cận đứng?
        \choice
        {1}
        {3}
        {4}
        {2}}{
        \begin{tikzpicture}[scale=.8, font=\footnotesize, line join=round, line cap=round, >=stealth]
            \def\xmin{-2}\def\xmax{2}\def\ymin{-3}\def\ymax{1}
            \draw[->] (\xmin-0.2,0)--(\xmax+0.2,0) node[below] {\footnotesize $x$};
            \draw[->] (0,\ymin-0.2)--(0,\ymax+0.2) node[right] {\footnotesize $y$};
            \draw (0,0) node [below left] {\footnotesize $O$};
            \foreach \x in {-1,1}\draw (\x,-0.1)--(\x,0.1) node [above left] {\footnotesize $\x$};
            \foreach \y in {-2}\draw (0.1,\y)--(-0.1,\y) node [ below left] {\footnotesize $\y$};
            \clip (\xmin,\ymin) rectangle (\xmax,\ymax);
            \draw[smooth,samples=200,domain=\xmin:\xmax] plot (\x,{((\x)^4)+((\x)^2)+-2});
        \end{tikzpicture}
    }
    \loigiai{
        * Điều kiện: $\heva{&x \ne 2\\&f(x+1) \ne 0\\&x \ge -2.}$\\
        Nhìn hình vẽ ta thấy
        $f(x+1)=0\Leftrightarrow \hoac{&x+1=-1\\&x+1=1}\Leftrightarrow \hoac{&x=-2&(\text{nghiệm đơn})\\&x=0&(\text{nghiệm đơn}).}$\\
        Vậy $g(x) = \dfrac{(x^2-x)\sqrt{x+2}}{(x-2)\cdot ax^2(x^2+2) }=\dfrac{(x-1)\sqrt{x+2}}{(x-2)\cdot ax(x^2+2)}.$ \\
        Đồ thị hàm số $g(x)$ có 2 đường tiệm cận đứng.}
\end{ex}
\begin{ex}%[Thi thử THPT Yên Phong 1 - Bắc Ninh, 2021]%[Duong Xuan Loi,12EX 6- 2021]%[2D1G4-3]%
    \immini{
        Cho hàm số $y=f(x)$ có đồ thị như hình vẽ. Biết $f'(x)<0$, $\forall x <-1$ và $f'(x)>0$, $\forall x>1$. Khi đó, tổng số tiệm cận của đồ thị hàm số $y=\dfrac{2024}{\sqrt{xf(x+1)}[xf(x+1)+1]-2}$ là
        \choice
        {$1$}
        {$3$}
        {$4$}
        {\True $2$}
    }{
        \begin{tikzpicture}[scale=0.7, font=\footnotesize, line join=round, line cap=round,>=stealth]
            \def\xmin{-2} \def\xmax{2}
            \def\ymin{-2} \def\ymax{3.3}
            \draw[color=gray!50,dashed] (\xmin,\ymin) grid (\xmax,\ymax);
            \draw[->] (\xmin,0)--(\xmax,0) node [below]{$x$};
            \draw[->] (0,\ymin)--(0,\ymax) node [left]{$y$};
            \node at (0,0) [above right]{$O$};
            \clip (\xmin+0.1,\ymin+0.1) rectangle (\xmax-0.1,\ymax-0.1);
            \draw[smooth,samples=300,domain=-1.8:1.4] plot(\x,{(\x+1)*(\x+1)*(\x)*(\x-1)});
            \fill (-1,0) circle (1.0pt) node[below]{$-1$} (1,0) circle (1.0pt) node[below right]{$1$};
        \end{tikzpicture}
    }
    \loigiai{
        Xét phương trình $\sqrt{xf(x+1)}[xf(x+1)+1]-2=0.\quad(1)$\\
        Đặt $t=\sqrt{xf(x+1)}(t\geq 0)$, ta được phương trình $t\left(t^2+1\right)=2\Leftrightarrow t^3+t-2=0\Leftrightarrow t=1$.\\
        Với $t=1\Rightarrow\sqrt{xf(x+1)}=1\Leftrightarrow xf(x+1)-1=0$.\\
        Đặt $u=x+1\Rightarrow x=u-1$, ta được phương trình $(u-1)f(u)-1=0\Leftrightarrow f(u)=\dfrac{1}{u-1}.\quad(2)$
        \begin{center}
            \begin{tikzpicture}[scale=1, font=\footnotesize, line join=round, line cap=round,>=stealth]
                \def\a{0} \def\b{1} \def\c{1} \def\d{-1} % Hệ số
                \def\xmin{-3} \def\xmax{3.5}
                \def\ymin{-2.8} \def\ymax{3.3}
                \draw[color=gray!50,dashed] (\xmin,\ymin) grid (\xmax,\ymax);
                \draw[->] (\xmin,0)--(\xmax,0) node [below]{$u$};
                \draw[->] (0,\ymin)--(0,\ymax) node [left]{$y$};
                \node at (0,0) [above right]{$O$};
                \clip (\xmin+0.1,\ymin+0.1) rectangle (\xmax-0.1,\ymax-0.1);
                \draw[smooth,samples=300,domain=-1.8:1.4] plot(\x,{(\x+1)*(\x+1)*(\x)*(\x-1)});
                \draw[smooth,samples=300,domain=\xmin:(-\d/\c-0.1)] plot(\x,{(\a*(\x)+\b)/(\c*(\x)+\d)});
                \draw[smooth,samples=300,domain=(-\d/\c+0.1:\xmax)] plot(\x,{(\a*(\x)+\b)/(\c*(\x)+\d)});
                \fill (-1,0) circle (1.0pt) node[below]{$-1$} (1,0) circle (1.0pt) node[below right]{$1$};
            \end{tikzpicture}
        \end{center}
        Nhận thấy đồ thị của các hàm số $y=f(u)$, $y=\dfrac{1}{u-1}$ chỉ cắt nhau tại $1$ điểm do đó phương trình $(2)$ có nghiệm duy nhất $\Rightarrow(1)$ có nghiệm duy nhất, suy ra đồ thị có $1$ tiệm cận đứng.\\
        Mặt khác: $\lim\limits_{x\to+\infty} f(x+1)=+\infty\Rightarrow\lim\limits_{x\to+\infty}\dfrac{2021}{\sqrt{xf(x+1)}[xf(x+1)+1]-2}=a>0$.\\
        $\lim\limits_{x\to-\infty} xf(x+1)=-\infty\Rightarrow\lim\limits_{x\to+\infty}\dfrac{2021}{\sqrt{xf(x+1)}[xf(x+1)+1]-2}$ không tồn tại.\\
        Do đó đường thẳng $y=a$ là tiệm cận ngang.}
\end{ex}
\BTTF
\begin{ex}%[EX-TF-2024, Lê Đạt]%[2D1N4-1]
    Cho hàm số $y=f(x)$ có bảng biến thiên như sau
    \begin{center}
        \begin{tikzpicture}[>=stealth]
            \tkzTabInit[nocadre=false,lgt=1,espcl=3,deltacl=0.6]
            {$x$/.7 ,$y'$/.7,$y$/2}
            {$-\infty$ , $-2$ , $0$, $+\infty$}
            \tkzTabLine{ , - , d , + , d , -, }
            \tkzTabVar{+/$+\infty$ , -D-/$1$/$-\infty$ , +D+/$+\infty$ /$1$, -/$0$}
        \end{tikzpicture}
    \end{center}
    Xét tính đúng sai của các khẳng định sau
    \choiceTF
    {\True $ x=0 $ là tiệm cận đứng của đồ thị hàm số $ y=f(x) $}
    {\True $ x=-2 $ là tiệm cận đứng của đồ thị hàm số $ y=f(x) $}
    {$ x=1 $ là tiệm cận đứng của đồ thị hàm số $ y=f(x) $}
    {\True $ y=0 $ là tiệm cận ngang của đồ thị hàm số $ y=f(x) $}
    \loigiai{
        \begin{itemchoice}
            \itemch $\lim \limits_{x \to 0^-} f(x)=+\infty\Rightarrow x=0$ là đường tiệm cận đứng của đồ thị hàm số $f(x)$.
            \itemch $\lim \limits_{x \to (-2)^+} f(x)=-\infty\Rightarrow x=-2$ là đường tiệm cận đứng của đồ thị hàm số $f(x)$.
            \itemch Đồ thị hàm số chỉ có hai tiệm cận đứng là $ x=0 $ và $ x=-2 $.
            \itemch $\lim \limits_{x \to +\infty} f(x)=0\Rightarrow y=0$ là đường tiệm cận ngang của đồ thị hàm số $f(x)$.
        \end{itemchoice}
    }
\end{ex}
\begin{ex}%[EX-TF-2024, Lê Đạt]%[2D1H4-2]
    Cho hàm số $y=\dfrac{m x^{2}+6 x-2}{x+2}$. Xét tính đúng sai của các khẳng định sau
    \choiceTF
    {Đồ thị hàm số luôn có tiệm cận đứng với mọi $ m $}
    {Đồ thị hàm số không có tiệm cận ngang với mọi $ m $}
    {\True Khi $ m=1 $ đồ thị hàm số có một tiệm cận xiên là $ y=x+4 $ }
    {Đồ thị hàm số luôn có tiệm cận xiên}
    \loigiai{
        \begin{itemchoice}
            \itemch Khi $ m=\dfrac{7}{2} $ hàm số trở thành $y=\dfrac{\dfrac{7}{2} x^{2}+6 x-2}{x+2}=\dfrac{7}{2}\left(x-\dfrac{2}{7} \right) $ suy ra đồ thị hàm số không có tiệm cận đứng.
            \itemch Khi $ m=0 $ hàm số trở thành $ y=\dfrac{6x-2}{x+2} $ từ đó suy ra đồ thị hàm số có $ y=6 $ là tiệm cận ngang.
            \itemch Khi $ m=1 $ hàm số trở thành $ y=\dfrac{x^2+6x-2}{x+2}=x+4-\dfrac{10}{x+2} $ từ đó suy ra $ y=x+4 $ là một tiệm cận ngang.
            \itemch Khi $ m=0 $ hàm số trở thành $ y=\dfrac{6x-2}{x+2} $ từ đó suy ra đồ thị hàm số có $ y=6 $ là tiệm cận ngang, $ x=-2 $ là tiệm cận đứng và không có tiệm cận xiên.
        \end{itemchoice}
    }
\end{ex}
\begin{ex}%[EX-TF-2024, Lê Đạt]%[2D1H4-3]
    \immini{Cho hàm số $y=f(x)$ có đồ thị như hình bên. Xét tính đúng sai của các khẳng định sau
        \choiceTF
        {\True $ x=0 $ là một đường tiệm cận đứng của đồ thị hàm số}
        {$ y=-x $ là một đường tiệm cận xiên của đồ thị hàm số}
        {\True $ y=x $ là một đường tiệm cận xiên của đồ thị hàm số}
        {Đồ thị hàm số có ba đường tiệm cận}
    }{
        \begin{tikzpicture}[scale=.9, font=\footnotesize, line join=round, line cap=round,>=stealth]
            \def\a{0} \def\b{1} \def\c{1} \def\d{-1} % Hệ số
            \def\xmin{-3} \def\xmax{3.5}
            \def\ymin{-2.8} \def\ymax{3.3}
            \draw[color=gray!50,dashed] (\xmin,\ymin) grid (\xmax,\ymax);
            \draw[->] (\xmin,0)--(\xmax,0) node [below]{$x$};
            \draw[->] (0,\ymin)--(0,\ymax) node [left]{$y$};
            \fill (0,0) circle(1pt) node[shift=(-45:0.25)]{$O$};
            \clip (\xmin+0.1,\ymin+0.1) rectangle (\xmax-0.1,\ymax-0.1);
            \draw[smooth,samples=300,domain=-3:3] plot(\x,{\x+1/(7*\x)});
            \draw[dashed,smooth,samples=300,domain=-3:3] plot(\x,{\x});
            %	\fill (-1,0) circle (1.0pt) node[below]{$-1$} (1,0) circle (1.0pt) node[below right]{$1$};
    \end{tikzpicture}}
    \loigiai{
        \begin{itemchoice}
            \itemch $ x=0 $ là một đường tiệm cận đứng của đồ thị hàm số.
            \itemch	$ y=x $ là một đường tiệm cận xiên của đồ thị hàm số.
            \itemch $ y=x $ là một đường tiệm cận xiên của đồ thị hàm số.
            \itemch Đồ thị hàm số có $ x=0 $ là tiệm cận đứng và $ y=x $ là tiệm cận xiên nên có hai tiệm cận.
        \end{itemchoice}
    }
\end{ex}
\begin{ex}
    \immini
    {
        Cho hàm số $y=f(x)$ có đạo hàm liên tục trên $R$. Hàm số $y=f^{\prime}(x)$ có đồ thị như hình bên. Xác định tính đúng, sai của các mệnh đề sau
        \choiceTF
        {Hàm số $y=f(x)$ có hai cực trị}
        {Hàm số $y=f(x)$ đồng biến trên khoảng $(1 ;+\infty)$}
        {\True $f(1)>f(2)>f(4)$.}
        {\True Trên đoạn $[-1 ; 4]$, giá trị lớn nhất của hàm số $y=f(x)$ là $f(1)$.}
    }
    {
        \begin{tikzpicture}[line join=round, line cap=round,>=stealth,font=\scriptsize]
            \begin{scope}[scale=0.5]
                \tikzset{label style/.style={font=\footnotesize}}
                \def \xmin{-2}
                \def \xmax{4.5}
                \def \ymin{-2}
                \def \ymax{3.5}
                \def \hamso{0.55*(\x)^3-1.76*(\x)^2-0.31*(\x)+2}
                \draw[->] (\xmin,0)--(\xmax,0) node[below left] {$x$};
                \draw[->] (0,\ymin)--(0,\ymax) node[below left] {$y$};
                \draw (0,0) node [below left] {$O$};
                \begin{scope}
                    \clip (\xmin+0.01,\ymin+0.01) rectangle (\xmax-0.01,\ymax-0.01);
                    \draw[samples=350,domain=-1.3:3.3,smooth,variable=\x] plot (\x,{\hamso});
                \end{scope}
                \draw (-1,0) node[below left]{$-1$} (1,0) node[below]{$1$} (3,0) node[below right]{$4$} (0,2) node[above left]{$2$};
            \end{scope}
        \end{tikzpicture}
    }
    \loigiai{}
\end{ex}
\begin{ex}
    \immini{Cho hàm số $y=f(x)$ liên tục trên đoạn $\left[0 ; \frac{7}{2}\right]$ có đồ thị hàm số $y=f^{\prime}(x)$ như hình vẽ.
        \choiceTF
        {\True Hàm số $y=f(x)$ đồng biến trên khoảng $\left(3 ; \frac{7}{2}\right)$}
        {\True $f(0)>f(3)$}
        {$f(3)>f\left(\frac{7}{2}\right)$}
        {Hàm số $y=f(x)$ đạt giá trị nhỏ nhất trên đoạn $\left[0 ; \frac{7}{2}\right]$ tại điểm $x_0=\frac{7}{2}$}
    }{\begin{tikzpicture}[>=stealth, samples=100,smooth,y=.7cm,font=\scriptsize]
            \begin{scope}[scale=.7]
                \draw[->] (-1,0)--(4.5,0) node[below] {$x$};
                \draw[->] (0,-2)--(0,4) node[right] {$y$};
                \draw (0,0) node [below left] {$O$};
                \draw[dashed] (3.6,0)--(3.6,4);
                \draw[samples=200,domain=0.2:3.6,smooth,variable=\x]
                plot (\x,{1.06*(\x)^3-5.3*(\x)^2+7.23*(\x)-3});
                \path
                (3.6,0)node[below]{$\dfrac{7}{2}$}
                (3,0)node[above left]{$3$}
                (1,0)node[above]{$1$}
                ;
            \end{scope}
    \end{tikzpicture}}
    \loigiai{}
\end{ex}
\BTTL
\begin{ex}%[2D1K4-2]%
    Đồ thị hàm số $y=\dfrac{(2m+1)x+3}{x+1}$ có đường đường tiệm cận đi qua điểm $A(-2;7)$ khi và chỉ khi
    \shortans{$m=3$}
    %	\choice
    %	{\True $m=3$}
    %	{$m=1$}
    %	{$m=-1$}
    %	{$m=-3$}
    \loigiai{
        Từ đề bài, suy ra $2m+1=7 \Leftrightarrow m=3$.\\
        Suy ra $m+n=0$.
    }
\end{ex}
\begin{ex}%[Học kì 1, THPT Nguyễn Thi Minh Khai - Hà Nội, 2020-2021]%[Bùi Mạnh Tiến, 12EX5]%[2D1K4-2]%
    Cho hàm số $ y=\dfrac{2mx+m}{x-1}$. Với giá trị nào của tham số $m$ thì đường tiệm cận đứng, tiệm cận ngang của đồ thị hàm số cùng hai trục tọa độ tạo thành một hình chữ nhật có diện tích bằng $8$?
    \shortans{$m=\pm 4$}
    %	\choice
    %	{$ m=2$}
    %	{ $m=\pm 2$}
    %	{\True $m=\pm 4$}
    %	{$ m=\pm\dfrac{1}{2}$}
    \loigiai{
        Hàm số $y=\dfrac{2mx+m}{x-1}$ có $a=2m$, $b=m$, $c=1$, $d=-1$.\\
        Tiệm cận ngang $y=\dfrac{a}{c}=2m$.\\
        Tiệm cận đứng $x=-\dfrac{d}{c}=1$.\\
        Diện tích hình chữ nhật tạo thành bởi hai đường tiệm cận và hai trục tọa độ có diện tích
        \begin{align*}
            |2m|\cdot 1=8\Leftrightarrow m=\pm 4.
        \end{align*}
    }
\end{ex}
\begin{ex}%[KSCL lần 1, Liễn Sơn - Vĩnh Phúc, 2021]%[Phạm Doãn Lê Bình, 12EX4-2021]%[2D1K4-2]%
    Cho hàm số $y=\dfrac{x-\sqrt{x^2+2x}}{x^2+mx-m-3}$ có đồ thị $(C)$. Giá trị của $m$ để $(C)$ có đúng hai tiệm cận thuộc tập nào sau đây?
    \shortans{$(-5;2)$}
    %	\choice
    %	{$(-2;1)$}
    %	{$(1;5)$}
    %	{$(5;8)$}
    %	{\True $(-5;2)$}
    \loigiai{
        Điều kiện xác định của hàm số đã cho $\heva{& \hoac{ & x\ge 0 \\ & x\le -2}\\ & x^2+mx-m-3\ne 0.}$\\
        Ta có $\lim \limits_{x\to +\infty} y = \lim \limits_{x\to -\infty} y = 0$ nên $(C)$ có một tiệm cận ngang $y=0$.\\
        Xét phương trình $x^2+mx-m-3=0$.\hfill $(1)$\\
        Ta có
        \begin{itemize}
            \item $\Delta = m^2+4m+12>0$, $\forall m \in \mathbb{R}$.\\ Vậy phương trình $(1)$ luôn có hai nghiệm phân biệt $x_1,x_2$ ($x_1<x_2$).
            \item $x-\sqrt{x^2+2x}=0 \Leftrightarrow \heva{& x\ge 0 \\ & x^2=x^2+2x} \Leftrightarrow x=0$.
            \item Phương trình $(1)$ có nghiệm $x=0 \Leftrightarrow m=-3$. Với $m=-3$ ta có
            $ y =\dfrac{x-\sqrt{x^2+2x}}{x^2-3x}.$
            Khi đó
            \begin{eqnarray*}
                & \lim \limits_{x\to 0^+} y & =\lim \limits_{x\to 0^+} \dfrac{x-\sqrt{x^2+2x}}{x^2-3x}\\
                & & =\lim \limits_{x\to 0^+} \dfrac{-2x}{(x^2-3x)\left( x+\sqrt{x^2+2x}\right)}\\
                & & = \lim \limits_{x\to 0^+} \dfrac{-2}{(x-3)\left( x+\sqrt{x^2+2x}\right)}=+\infty
            \end{eqnarray*}
            và $\lim \limits_{x\to 3^+} y =-\infty$
            nên $(C)$ có thêm hai tiệm cận đứng $x=0$ và $x=3$ (không thỏa yêu cầu bài toán).
            \item Với $m\ne -3$ thì $(C)$ có đúng hai tiệm cận khi và chỉ khi $\hoac{& x_1<-2<x_2<0 &(2)\\ & -2<x_1<0<x_2. & (3)}$
            \item Đặt $f(x)=x^2+mx-m-3$. Ta có
            $(2)\Leftrightarrow \heva{& f(-2)< 0 \\ & f(0) >0 \\ & 0>-m} \Leftrightarrow \heva{& m>\dfrac{1}{3}\\ & m<-3 \\ & m>0}\Leftrightarrow m \in \varnothing.$
            \item $(3)\Leftrightarrow \heva{& f(-2)> 0 \\ & f(0) <0 \\ & -2<-m} \Leftrightarrow \heva{& m<\dfrac{1}{3}\\ & m>-3 \\ & m<2}\Leftrightarrow -3<m<\dfrac{1}{3}.$
        \end{itemize}
        Vậy $m\in \left(-3;\dfrac{1}{3}\right)$.
    }
\end{ex}
\begin{ex}%[kiểm tra GHK1, Sở GD và ĐT - Vĩnh Phúc, 2021]%[Huỳnh Xuân Tín, 12EX4]%[2D1K4-2]%
    Gọi $S$ là tập tất cả các giá trị của tham số $m$ để đồ thị hàm số	$y=\dfrac{x-3}{x^2-2x-m}$ có đúng một đường
    tiệm cận đứng. Tính tổng các phần tử của tập $S$.
    \shortans{$2$}
    %	\choice
    %	{$-1$}
    %	{\True $2$}
    %	{$-6$}
    %	{$1$}
    \loigiai{
        Để đồ thị hàm số	$y=\dfrac{x-3}{x^2-2x-m}$ có đúng một đường
        tiệm cận đứng, ta có hai trường hợp sau
        \begin{enumerate}[TH 1.]
            \item $x^2-2x-m=0$ có nghiệm kép $\Leftrightarrow \Delta'=1+m=0\Leftrightarrow m=-1$.
            \item $x^2-2x-m=0$ có hai nghiệm phân biệt trong đó có một nghiệm bằng $3$
            \[\Leftrightarrow \heva{&\Delta'=1+m>0\\& 3^2-6-m=0}\Leftrightarrow\heva{&m>-1\\&m=3}\Leftrightarrow m=3.\]
        \end{enumerate}
        Khi đó $S=\{-1;3\}$ và có tổng là $2$.
    }
\end{ex}
\begin{ex}
    Tốc độ phản ứng của enzyme theo nồng độ cơ chất \( S \) được mô tả bởi phương trình Michaelis-Menten: $v(S) = \dfrac{V_{\text{max}} S}{K_m + S}$,
    trong đó \( v(S) \) là tốc độ phản ứng, \( S \) là nồng độ cơ chất, \( V_{\text{max}} \) là tốc độ tối đa, và \( K_m \) là hằng số Michaelis. Xác định và nêu ý nghĩa của đường tiệm cận đứng của hàm số này.
    \shortans{không có TCĐ, tốc độ phản ứng không thể tới vô hạn}
    \loigiai{
        Để tìm tiệm cận đứng, ta xét các giá trị của \( S \) làm cho mẫu số của phương trình bằng 0:
        \[
        K_m + S = 0 \Rightarrow S = -K_m
        \]
        Vì nồng độ cơ chất \( S \) không thể âm, không có tiệm cận đứng trong trường hợp này.
        \textbf{Ý nghĩa:} Điều này có nghĩa là tốc độ phản ứng enzyme không có giá trị nào dẫn đến tốc độ phản ứng tiến đến vô hạn trong phạm vi các giá trị hợp lý của \( S \).}
\end{ex}
\begin{ex}
    \immini{Một ống khói của nhà máy điện hạt nhân có mặt cắt là một hypebol $(H)$ có phương trình chính tắc là $\dfrac{x^2}{27^2}-\dfrac{y^2}{40^2}=1$ (Hình $1.25$). Xét hai nhánh bên trên $Ox$ của $(H)$ là đồ thị $(C)$ của hàm số $y=\dfrac{40}{27}\sqrt{x^2-27^2}$ (phần nét liền đậm). Tìm tất cả các đường tiệm cận xiên của $(C)$.}{\begin{tikzpicture}[>=latex,line join=round, line cap=round, scale=.04, font=\footnotesize]
            \draw[->] (-90,0)--(90,0) node[above]{$x$};
            \draw[->] (0,-130)--(0,80) node[left]{$y$};
            \foreach \x in {-80,-60,-40,-20,20,40,60,80}
            \draw[fill=black] (\x,0) circle (15pt) node[below, fill=white]{$\x$};
            \foreach \y in {-120,-100,-80,-60,-40,-20,20,40,60}
            \draw[fill=black] (0,\y) circle (15pt) node[left]{$\y$};
            \clip (-90,-130) rectangle (90,80);
            \draw[samples=200,smooth,blue,line width=1] plot[domain=-90:-27] (\x,{40*sqrt((\x)^2-27^2)/27});
            \draw[samples=200,smooth,blue,line width=1] plot[domain=27:90] (\x,{40*sqrt((\x)^2-27^2)/27});
            \draw[samples=200,smooth,blue,line width=1, dashed] plot[domain=-90:-27] (\x,{-40*sqrt((\x)^2-27^2)/27});
            \draw[samples=200,smooth,blue,line width=1, dashed] plot[domain=27:90] (\x,{-40*sqrt((\x)^2-27^2)/27});
            \draw (0,0) node[above right]{$O$};
    \end{tikzpicture}}
    \shortans{$y=\pm \dfrac{40}{27}$}
    \loigiai{
        Ta có
        \allowdisplaybreaks
        \begin{eqnarray*}
            a&=&\lim\limits_{x\to+\infty}\dfrac{f(x)}{x}	 =\lim\limits_{x\to+\infty}\dfrac{\dfrac{40}{27}\sqrt{x^2-27^2}}{x}\\
            &=&\lim\limits_{x\to+\infty}\dfrac{\dfrac{40}{27}x\sqrt{1-\dfrac{27^2}{x^2}}}{x}
            =\lim\limits_{x\to+\infty}\dfrac{40}{27}\sqrt{1-\dfrac{27^2}{x^2}}
            =\dfrac{40}{27}.\\
            b&=&\lim\limits_{x\to+\infty}\left[f(x)-ax\right]
            =\lim\limits_{x\to+\infty}\left[\dfrac{40}{27}\sqrt{x^2-27^2}-\dfrac{40}{27}x\right]\\
            &=&\lim\limits_{x\to+\infty}\dfrac{40}{27}\left(\sqrt{x^2-27^2}-x\right)
            =\lim\limits_{x\to+\infty}\dfrac{40}{27}\cdot\dfrac{x^2-27^2-x^2}{\sqrt{x^2-27^2}+x}\\
            &=&\lim\limits_{x\to+\infty}\dfrac{40}{27}\cdot\dfrac{-27^2}{x\left(\sqrt{1-\dfrac{27}{x^2}}+1\right)}
            =0.
        \end{eqnarray*}
        Vậy đường thẳng $y=\dfrac{40}{27}x$ là một tiệm cận xiên của đồ thị.\\
        Tương tự, $\lim\limits_{x\to-\infty}\dfrac{f(x)}{x}=-\dfrac{40}{27}\Rightarrow a=-\dfrac{40}{27}$; $\lim\limits_{x\to-\infty} \left[f(x)-ax\right]=0\Rightarrow b=0$.\\
        Vậy đường thẳng $y=-\dfrac{40}{27}x$ là tiệm cận xiên của đồ thị.
    }
\end{ex}

\Closesolutionfile{ans}
% \begin{dang}{Tìm các đường tiệm cận đồ thị hàm ẩn}
\end{dang}
\begin{vd}
    Cho hàm số $y=f(x)$ có bảng biến thiên như hình vẽ sau
    \begin{center}
        \begin{tikzpicture}[>=stealth]
            \tkzTabInit[nocadre=false,lgt=1,espcl=1.5,deltacl=0.5]{$x$/.7 ,$y'$/.7,$y$/2}
            {$-\infty$ , $-1$ , $2$ , $+\infty$}
            \tkzTabLine{ , + , $0$ , - , d , + , }
            \tkzTabVar{-/$1$ , +/$4$ , -/$-5$ , +/$+\infty$}
        \end{tikzpicture}
    \end{center}
    Tìm TCĐ, TCN của đồ thị hàm số
    \begin{listEX}[3]
        \item $y=\dfrac{2}{f(x)-3}$
        \item $y=\dfrac{-3}{f(x)+2}$
        \item $y=\dfrac{x-2}{f(x)+5}$
        \item $y=\dfrac{x+1}{f(x)-4}$
        \item $y=\dfrac{2}{f(x^2)+3}$
        \item $y=\dfrac{4f(x)-5}{3f(x)+1}$
    \end{listEX}
    \loigiai{}
\end{vd}
\begin{vd}\immini{Cho hàm bậc ba $y=f(x)$ có đồ thị như hình vẽ. Tìm số tiệm cận đứng của đồ thị hàm số
        \begin{listEX}[2]
            \item $y=\dfrac{\sqrt{x+3}}{(x-1)f(x)}$
            \item $g(x)=\dfrac{(x^2+4x+3)\sqrt{x^2+x}}{x\left[f^2(x)-2f(x)\right]}$ .
    \end{listEX}}{\begin{tikzpicture}[line cap=round,line join=round, >=stealth,font=\footnotesize]
            \begin{scope}[scale=.5]
                \def\a{-1} % Hệ số a phải khác 0
                \def\b{-13/2}
                \def\c{-12}
                \def\d{-9/2}
                \draw[->] (-5,0) -- (2,0)node[below]{$x$};
                \draw[->] (0,-3) -- (0,4) node[left] {$y$};
                \draw (0,0)node[below right]{$O$} (-3,0)node[below]{$-3$};
                \draw[dashed] (-1,0)node[below]{$-1$}|-(0,2)node[right]{$2$};
                \draw[samples=150,smooth,domain=-4:.-.2] plot(\x,{\a*(\x)^3+(\b)*(\x)^2+(\c)*\x+(\d)});
            \end{scope}
    \end{tikzpicture}}
    \loigiai{
        \begin{center}
            \begin{tikzpicture}[line cap=round,line join=round, >=stealth,font=\footnotesize,scale=1]
                \def\a{-1} % Hệ số a phải khác 0
                \def\b{-13/2}
                \def\c{-12}
                \def\d{-9/2}
                \draw[->] (-5,0) -- (2,0)node[below]{$x$};
                \draw[->] (0,-3) -- (0,4) node[left] {$y$};
                \draw (0,0)node[below right]{$O$} (-3,0)node[below]{$-3$} (-.3,0)node[above]{$a$};
                \draw[dashed] (-3.78,0)node[below]{$c$}|-(0,2)|-(-1.71,0)node[below]{$b$}|-(0,2) (-1,0)node[below]{$-1$}|-(0,2)node[right]{$2$};
                \draw[samples=150,smooth,domain=-4:.-.2] plot(\x,{\a*(\x)^3+(\b)*(\x)^2+(\c)*\x+(\d)});
            \end{tikzpicture}
        \end{center}
        $g(x)=\dfrac{(x^2+4x+3)\sqrt{x^2+x}}{x\left[f^2(x)-2f(x)\right]}=\dfrac{(x+1)(x+3)\sqrt{x(x+1)}}{x\left[f^2(x)-2f(x)\right]}$.\\
        Điều kiện của căn là $x\le -1; x\ge 0$.\\
        Dựa vào đồ thị ta có \[x\left[f^2(x)-2f(x)\right]=0 \Leftrightarrow \hoac{&x=0\\&f(x)=0\\& f(x)=2} \Leftrightarrow \hoac{&x=0\text{ (nhận)}\\&x=-3\text{ (nhận)};\ x=a \text{ (loại)} \\&x=-1\text{ (nhận)};\ x=b\text{ (nhận)};\ x=c\text{ (nhận)}}\]\\
        Số TCĐ lúc này chính là số nghiệm không bị rút gọn của mẫu, vậy có bốn TCĐ là $x=0; x=-3; x=b; x=c$.
    }
\end{vd}
\BTTN
\Opensolutionfile{ans}[ans/2D1-4-DANG-3]
\begin{ex}%[2D1K4-1]
    Cho hàm số $y=f(x)$ có bảng biến thiên như hình bên. Đồ thị hàm số $y=\dfrac{-5}{f(x)+4}$ có bao nhiêu tiệm cận đứng?
    \begin{center}
        \begin{tikzpicture}[scale=0.8]
            \tkzTabInit[nocadre=false,lgt=1.5,espcl=3,deltacl=0.6]
            {$x$ /0.6,$y’$ /0.6,$y$ /2}
            {$-\infty$ ,$1$, $2$, $+\infty$}
            \tkzTabLine{,+,d,-,d,+,}
            \tkzTabVar{-/$-4$,+/$3$,-/$-5$,+/$+\infty$}
        \end{tikzpicture}
    \end{center}
    \choice
    {$1$}
    {$3$}
    {\True $2$}
    {$4$}
    \loigiai{
        Dựa vào bảng biến thiên suy ra
        $f(x)+4=0 \Leftrightarrow f(x) =-4$, phương trình này có $2$ nghiệm phân biệt nên đồ thị hàm số $y=\dfrac{-5}{f(x)+4}$ có $2$ tiệm cận đứng.
    }
\end{ex}
\begin{ex}%[2D1K4-1]
    Cho hàm số $y=f(x)$ có bảng biến thiên như hình bên. Đồ thị hàm số $y=\dfrac{x+2}{2f(x)-1}$ có bao nhiêu tiệm cận đứng?
    \begin{center}
        \begin{tikzpicture}[scale=0.8]
            \tkzTabInit[nocadre=false,lgt=1.5,espcl=3,deltacl=0.6]
            {$x$ /0.6,$y’$ /0.6,$y$ /2}
            {$-\infty$ ,$-1$, $0$, $1$, $+\infty$}
            \tkzTabLine{,+,0,-,0,+,0,-,}
            \tkzTabVar{-/$-\infty$,+/$0$,-/$-\dfrac{5}{3}$,+/$0$,-/$-\infty$}
        \end{tikzpicture}
    \end{center}
    \choice
    {$1$}
    {$3$}
    {$2$}
    {\True $0$}
    \loigiai{
        Dựa vào bảng biến thiên suy ra
        $2f(x)-1=0 \Leftrightarrow f(x) =\dfrac{1}{2}$, phương trình này có $0$ nghiệm nên đồ thị hàm số $y=\dfrac{x+2}{2f(x)-1}$ không có tiệm cận đứng.
    }
\end{ex}
%69
\begin{ex}%[2D1K4-1]
    Cho hàm số $y=f(x)$ có bảng biến thiên như hình bên. Đồ thị hàm số $y=\dfrac{1}{2f(x)-3}$ có bao nhiêu tiệm cận đứng?
    \begin{center}
        \begin{tikzpicture}[scale=0.8]
            \tkzTabInit[nocadre=false,lgt=1.5,espcl=3,deltacl=0.6]
            {$x$ /0.6,$y’$ /0.6,$y$ /2}
            {$-\infty$ ,$0$, $1$, $+\infty$}
            \tkzTabLine{,+,0,-,0,+,}
            \tkzTabVar{-/$-\infty$,+/$5$,-/$-1$,+/$+\infty$}
        \end{tikzpicture}
    \end{center}
    \choice
    {$1$}
    {\True $3$}
    {$2$}
    {$0$}
    \loigiai{
        Dựa vào bảng biến thiên suy ra
        $2f(x)-3=0 \Leftrightarrow f(x) =-\dfrac{3}{2}$, phương trình này có $3$ nghiệm phân biệt nên đồ thị hàm số $y=\dfrac{1}{2f(x)-3}$ có ba tiệm cận đứng.
    }
\end{ex}
%70
%71
%72
\begin{ex}%[2D1K4-1]
    Cho hàm số $y=f(x)$ có bảng biến thiên như hình bên. Đồ thị hàm số $y=\dfrac{x}{f(x)-3}$ có bao nhiêu tiệm cận đứng?
    \begin{center}
        \begin{tikzpicture}[scale=0.8]
            \tkzTabInit[nocadre=false,lgt=1.5,espcl=3,deltacl=0.6]
            {$x$ /0.6,$y’$ /0.6,$y$ /2}
            {$-\infty$ ,$-1$, $0$, $1$, $+\infty$}
            \tkzTabLine{,-,0,+,0,-,0,+,}
            \tkzTabVar{+/$+\infty$,-/$0$,+/$3$,-/$0$,+/$+\infty$}
        \end{tikzpicture}
    \end{center}
    \choice
    {$1$}
    {\True $3$}
    {$2$}
    {$4$}
    \loigiai{
        Dựa vào bảng biến thiên suy ra
        $f(x)-3=0 \Leftrightarrow f(x) =3$, phương trình này có $2$ nghiệm phân biệt khác $0$ và một nghiệm bội chẵn $x=0$ nên đồ thị hàm số $y=\dfrac{x}{f(x)-3}$ có ba tiệm cận đứng.
    }
\end{ex}
\begin{ex}%[2D1K4-1]
    Cho hàm số $y=f(x)$ có bảng biến thiên như hình bên. Đồ thị hàm số $y=\dfrac{4}{f(x)+1}$ có tiệm cận ngang là đường thẳng
    \begin{center}
        \begin{tikzpicture}[scale=0.8]
            \tkzTabInit[nocadre=false,lgt=1.5,espcl=3,deltacl=0.6]
            {$x$ /0.6,$y’$ /0.6,$y$ /2}
            {$-\infty$ ,$-1$, $2$, $+\infty$}
            \tkzTabLine{,+,0,-,0,+,}
            \tkzTabVar{-/$1$,+/$4$,-/$-5$,+/$1$}
        \end{tikzpicture}
    \end{center}
    \choice
    {$y=1$}
    {$y=-5$}
    {\True $y=2$}
    {$y=4$}
    \loigiai{
        Dựa vào bảng biến thiên suy ra
        $\lim \limits_{x \to \pm \infty} f(x)=1 \Leftrightarrow \lim \limits_{x \to \pm \infty} \dfrac{4}{f(x)+1} =2$ nên đồ thị hàm số đã cho có tiệm cận ngang là $y=2$.
    }
\end{ex}
\begin{ex}%[2D1K4-1]
    Cho hàm số $y=f(x)$ có bảng biến thiên như hình bên. Đồ thị hàm số $y=\dfrac{2-f(x)}{f(x)+3}$ có tiệm cận ngang là đường thẳng
    \begin{center}
        \begin{tikzpicture}[scale=0.8]
            \tkzTabInit[nocadre=false,lgt=1.5,espcl=3,deltacl=0.6]
            {$x$ /0.6,$y’$ /0.6,$y$ /2}
            {$-\infty$ ,$0$, $2$, $+\infty$}
            \tkzTabLine{,-,0,+,0,-,}
            \tkzTabVar{+/$+\infty$,-/$1$,+/$5$,-/$-\infty$}
        \end{tikzpicture}
    \end{center}
    \choice
    {$y=1$}
    {$y=-3$}
    {$y=2$}
    {\True $y=-1$}
    \loigiai{
        Dựa vào bảng biến thiên suy ra
        $\lim \limits_{x \to \pm \infty} f(x)=\pm \infty \Leftrightarrow \lim \limits_{x \to \pm \infty} \dfrac{2-f(x)}{f(x)+3} =-1$ nên đồ thị hàm số $y=\dfrac{2-f(x)}{f(x)+3}$ có tiệm cận ngang là $y=-1$.
    }
\end{ex}
\begin{ex}%[2D1K4-1]
    Cho hàm số $y=f(x)$ có bảng biến thiên như hình bên. Đồ thị hàm số $y=\dfrac{1}{f^2(x)-4f(x)+4}$ có bao nhiêu tiệm cận đứng?
    \begin{center}
        \begin{tikzpicture}[scale=0.8]
            \tkzTabInit[nocadre=false,lgt=1.5,espcl=3,deltacl=0.6]
            {$x$ /0.6,$y’$ /0.6,$y$ /2}
            {$-\infty$, $2$, $+\infty$}
            \tkzTabLine{,-,0,+,}
            \tkzTabVar{+/$1$,-/$-3$,+/$1$}
        \end{tikzpicture}
    \end{center}
    \choice
    {$1$}
    {$3$}
    {$2$}
    {$0$}
    \loigiai{
        Dựa vào bảng biến thiên suy ra $f^2(x)-4f(x)+4=0 \Leftrightarrow f(x)=2$, phương trình $f(x)=2$ vô nghiệm nên đồ thị hàm số đã cho không có tiệm cận đứng.
    }
\end{ex}
%83
\begin{ex}%[2D1K4-1]
    Cho hàm số $y=f(x)$ có bảng biến thiên như hình bên. Đồ thị hàm số $y=\dfrac{1}{f(3-x)-2}$ có bao nhiêu tiệm cận đứng?
    \begin{center}
        \begin{tikzpicture}[scale=0.8]
            \tkzTabInit[nocadre=false,lgt=1.5,espcl=3,deltacl=0.6]
            {$x$ /0.6,$y’$ /0.6,$y$ /2}
            {$-\infty$ ,$-2$, $2$, $+\infty$}
            \tkzTabLine{,+,0,-,0,+,}
            \tkzTabVar{-/$-\infty$,+/$3$,-/$0$,+/$+\infty$}
        \end{tikzpicture}
    \end{center}
    \choice
    {$1$}
    {\True $3$}
    {$2$}
    {$0$}
    \loigiai{
        Dựa vào bảng biến thiên suy ra $f(3-x)-2=0 \Leftrightarrow f(3-x)=2$, phương trình này có $3$ nghiệm phân biệt nên đồ thị hàm số đã cho có $3$ tiệm cận đứng.
    }
\end{ex}
\begin{ex}%[2D1G4-1]
    Cho hàm số $y=f(x)$ có bảng biến thiên như hình bên. Đồ thị hàm số $y=\dfrac{4}{f(x^2)-2}$ có bao nhiêu tiệm cận đứng?
    \begin{center}
        \begin{tikzpicture}[scale=0.8]
            \tkzTabInit[nocadre=false,lgt=1.5,espcl=3,deltacl=0.6]
            {$x$ /0.6,$y’$ /0.6,$y$ /2}
            {$-\infty$ ,$0$, $3$, $+\infty$}
            \tkzTabLine{,-,0,+,d,-,}
            \tkzTabVar{+/$8$,-/$1$,+/$4$,-/$2$}
        \end{tikzpicture}
    \end{center}
    \choice
    {$5$}
    {$3$}
    {\True $2$}
    {$4$}
    \loigiai{
        Dựa vào bảng biến thiên suy ra
        $f(x^2)-2=0 \Leftrightarrow f(x^2) =2$. Kẻ đường thẳng $y=2$ ta thấy đường thẳng cắt đồ thị hàm số tại hai điểm phân biệt. Suy ra
        $$\hoac{&x^2=a \; (a<0)\\&x^2=b \; (b >0)} \Rightarrow x=\pm \sqrt{b}.$$
        Do đó đồ thị hàm số đã cho có $2$ tiệm cận đứng.
    }
\end{ex}%89
\begin{ex}%[2D1G4-1]
    Cho hàm số $y=f(x)$ có bảng biến thiên như hình bên. Đồ thị hàm số $y=\dfrac{2}{f(|x|)-3}$ có bao nhiêu tiệm cận ngang?
    \begin{center}
        \begin{tikzpicture}[scale=0.8]
            \tkzTabInit[nocadre=false,lgt=1.5,espcl=3,deltacl=0.6]
            {$x$ /0.6,$y’$ /0.6,$y$ /2}
            {$-\infty$ ,$0$, $2$, $+\infty$}
            \tkzTabLine{,+,0,-,0,+,}
            \tkzTabVar{-/$-\infty$,+/$3$,-/$-1$,+/$+\infty$}
        \end{tikzpicture}
    \end{center}
    \choice
    {$4$}
    {\True $3$}
    {$5$}
    {$6$}
    \loigiai{
        Dựa vào bảng biến thiên suy ra
        $f(|x|)-3=0 \Leftrightarrow f(|x|) =3$.\\
        Bảng biến thiên hàm số $y=f(|x|)$ như sau
        \begin{center}
            \begin{tikzpicture}[scale=0.8]
                \tkzTabInit[nocadre=false,lgt=1.5,espcl=3,deltacl=0.6]
                {$x$ /0.6,$y’$ /0.6,$y$ /2}
                {$-\infty$ ,$-2$, $0$, $2$, $+\infty$}
                \tkzTabLine{,-,0,+,0,-,0,+,}
                \tkzTabVar{+/$+\infty$,-/$-1$,+/$3$,-/$-1$,+/$+\infty$}
            \end{tikzpicture}
        \end{center}
        Dựa vào bảng biến thiên hàm số $y=f(|x|)$, phương trình $f(|x|) =3$ có ba nghiệm phân biệt, do đó đồ thị hàm số $y=\dfrac{2}{f(|x|)-3}$ có $3$ tiệm cận đứng.
    }
\end{ex}
\begin{ex}
    \immini{ %Câu 90
        Cho hàm số bậc ba $f(x)= ax^3 +bx^2 +cx +d$ có đồ thị như hình vẽ bên. Đồ thị hàm số $g(x) = \dfrac{\sqrt{x+1}}{(x-3)\cdot f(x)}$ có bao nhiêu đường tiệm cận đứng?
        \choice
        {5}
        {2}
        {4}
        {\True 3}}{\begin{tikzpicture}[scale=.5, font=\footnotesize, line join=round, line cap=round, >=stealth]
            \def\xmin{-3}\def\xmax{3}\def\ymin{-5}\def\ymax{1}
            \draw[->] (\xmin-0.2,0)--(\xmax+0.2,0) node[below] {\footnotesize $x$};
            \draw[->] (0,\ymin-0.2)--(0,\ymax+0.2) node[right] {\footnotesize $y$};
            \draw (0,0) node [below left] {\footnotesize $O$};
            \foreach \x in {-1}\draw (\x,-0.1)--(\x,0.1) node [above] {\footnotesize $\x$};
            \foreach \x in {2}\draw (\x,-0.1)--(\x,0.1) node [above right] {\footnotesize $\x$};
            \foreach \y in {}\draw (-0.1,\y)--(0.1,\y) node [right] {\footnotesize $\y$};
            \clip (\xmin,\ymin) rectangle (\xmax,\ymax);
            \draw[smooth,samples=200,domain=\xmin:\xmax] plot (\x,{1*((\x)^3)+0*((\x)^2)+-3*(\x)+-2});
        \end{tikzpicture}
    }
    \loigiai{
        * Điều kiện: $\heva{&x \ne 3\\&f(x) \ne 0\\&x \ge -1.}$\\
        Nhìn hình vẽ ta thấy
        $f(x)=0\Leftrightarrow \hoac{&x=-1&(\text{nghiệm kép}) \\&x=2&(\text{nghiệm đơn}).}$\\
        Vậy $g(x) = \dfrac{\sqrt{x+1}}{(x-3)\cdot a(x+1)^2 (x-2)}.$ \\
        Đồ thị hàm số $g(x)$ có 3 đường tiệm cận đứng.}
\end{ex}
\begin{ex}
    \immini{ %Câu 92.
        Đường cong ở hình bên là đồ thị của hàm số $y = ax^3 +bx^2 +cx+d$. Đồ thị hàm số $y =\dfrac{(2x+1)\sqrt{x-1}}{x\cdot f(x-2)}$ có tất cả bao nhiêu tiệm cận đứng?
        \choice
        {1}
        {3}
        {4}
        {\True 2}}{\begin{tikzpicture}[scale=.6, font=\footnotesize, line join=round, line cap=round, >=stealth]
            \def\xmin{-3}\def\xmax{3}\def\ymin{-3}\def\ymax{3}
            \draw[->] (\xmin-0.2,0)--(\xmax+0.2,0) node[below] {\footnotesize $x$};
            \draw[->] (0,\ymin-0.2)--(0,\ymax+0.2) node[right] {\footnotesize $y$};
            \draw (0,0) node [below left] {\footnotesize $O$};
            \foreach \x in {-2}\draw (\x,-0.1)--(\x,0.1) node [above left] {\footnotesize $\x$};
            \foreach \x in {2}\draw (\x,-0.1)--(\x,0.1) node [above right] {\footnotesize $\x$};
            \foreach \y in {}\draw (-0.1,\y)--(0.1,\y) node [right] {\footnotesize $\y$};
            \clip (\xmin,\ymin) rectangle (\xmax,\ymax);
            \draw[smooth,samples=200,domain=\xmin:\xmax] plot (\x,{(2/3)*((\x)^3)+0*((\x)^2)+-(8/3)*(\x)});
    \end{tikzpicture}}
    \loigiai{
        * Điều kiện: $\heva{&x \ne 0\\&f(x-2) \ne 0\\&x \ge 1.}$\\
        Nhìn hình vẽ ta thấy
        $f(x-2)=0\Leftrightarrow \hoac{&x-2=-2\\&x-2=0\\&x-2=2}\Leftrightarrow \hoac{&x=0&(\text{không thỏa mãn})\\&x=2&(\text{nghiệm đơn})\\&x=4&(\text{nghiệm đơn}).}$\\
        Vậy $g(x) =\dfrac{(2x+1)\sqrt{x-1}}{x\cdot f(x-2)}=\dfrac{(x-1)\sqrt{x+2}}{x\cdot ax(x-2)(x-4)}.$ \\
        Đồ thị hàm số $g(x)$ có 2 đường tiệm cận đứng.}
\end{ex}
\begin{ex}
    \immini{ %Câu 93.
        Cho hàm số $y= f(x)$ có đồ thị cắt trục hoành tại đúng 3 điểm như hình bên. Đồ thị hàm số $y =\dfrac{(x+2)\sqrt{3-x}}{f(|x|)}$
        có tất cả bao nhiêu tiệm cận đứng?
        \choice
        {1}
        {3}
        {4}
        {\True 2}}{\begin{tikzpicture}[scale=.5, font=\footnotesize, line join=round, line cap=round, >=stealth]
            \def\xmin{-2}\def\xmax{5}\def\ymin{-3}\def\ymax{5}
            \draw[->] (\xmin-0.2,0)--(\xmax+0.2,0) node[below] {\footnotesize $x$};
            \draw[->] (0,\ymin-0.2)--(0,\ymax+0.2) node[right] {\footnotesize $y$};
            \draw (0,0) node [below left] {\footnotesize $O$};
            \foreach \x in {-1,2,4}\draw (\x,-0.1)--(\x,0.1) node [above left] {\footnotesize $\x$};
            \foreach \y in {}\draw (-0.1,\y)--(0.1,\y) node [right] {\footnotesize $\y$};
            \clip (\xmin,\ymin) rectangle (\xmax,\ymax);
            \draw[smooth,samples=200,domain=-1.2:0] plot(\x,{0-8.48*(\x)^(2.0)-5.48*(\x)+3.0});
            \draw[smooth,samples=200,domain=0:2]
            plot(\x,{0-2.7989489689153735*(\x)^(3.0)+8.326740175055514*(\x)^(2.0)-6.957684474449535*(\x)+3.0});
            \draw[smooth,samples=200,domain=2:5]
            plot(\x,{2.395330112721417*(\x)^(2.0)-14.371980676328501*(\x)+19.162640901771336});
    \end{tikzpicture}}
    \loigiai{
        * Điều kiện: $\heva{&f(|x|) \ne 0\\&x \le 3.}$\\
        Nhìn hình vẽ ta thấy
        $f(|x|)=0\Leftrightarrow \hoac{&|x|=-1\\&|x|=2\\&|x|=4}\Leftrightarrow \hoac{&x=\pm 2&(\text{nghiệm đơn})\\&x=- 4&(\text{nghiệm đơn})\\&x=4&(\text{không thỏa mãn}).}$\\
        Vậy $y =\dfrac{(x+2)\sqrt{3-x}}{a(x-2)(x+2)(x+4)(x-4)}$ \\
        Đồ thị hàm số có 2 đường tiệm cận đứng.}
\end{ex}
\begin{ex}
    \immini{ %Câu 94.
        Đường cong ở hình bên là đồ thị của hàm số $y = ax^3 +bx^2 +cx+d$. Đồ thị hàm số $y =\dfrac{(2x+1)\sqrt{1-x}}{f(|x|)}$ có tất cả bao nhiều tiệm cận đứng?
        \choice
        { 1}
        {3}
        {4}
        {\True 2}}{\begin{tikzpicture}[scale=.8, font=\footnotesize, line join=round, line cap=round, >=stealth]
            \def\xmin{-1}\def\xmax{2}\def\ymin{-1.5}\def\ymax{1.5}
            \draw[->] (\xmin-0.2,0)--(\xmax+0.2,0) node[below] {\footnotesize $x$};
            \draw[->] (0,\ymin-0.2)--(0,\ymax+0.2) node[right] {\footnotesize $y$};
            \draw (0.15,0) node [below left] {\footnotesize $O$};
            \foreach \x in {}\draw (\x,0.1)--(\x,-0.1) node [below] {\footnotesize $\x$};
            \foreach \y in {-1,1}\draw (0.1,\y)--(-0.1,\y) node [left] {\footnotesize $\y$};
            \clip (\xmin,\ymin) rectangle (\xmax,\ymax);
            \draw[smooth,samples=200,domain=\xmin:\xmax] plot (\x,{4*((\x)^3)+-6*((\x)^2)+0*(\x)+1});
            \draw[dashed] (0.5,0)--(0.5,0.0)--(0,0.0);
            \draw (0.5,-1pt)--(0.5,1pt) node [above] {\footnotesize $\frac{1}{2}$};
            \draw (-0.7,-1pt)--(-0.7,1pt) node [above] {\footnotesize $-\frac{1}{2}$};
            \draw (1,-1pt)--(1,1pt) node [above] {\footnotesize $1$};
            \draw[dashed] (0.0,0)--(0.0,1.0)--(0,1.0);
            \draw[dashed] (1.0,0)--(1.0,-1.0)--(0,-1.0);
    \end{tikzpicture}}
    \loigiai{
        * Điều kiện: $\heva{&f(|x|) \ne 0\\&x \le 1.}$\\
        Nhìn hình vẽ ta thấy
        $f(|x|)=0\Leftrightarrow \hoac{&|x|=-\dfrac{1}{2}\\&|x|=\dfrac{1}{2}\\&|x|=x_1>1}\Leftrightarrow \hoac{&x=\pm \dfrac{1}{2}&(\text{hai nghiệm đơn})\\&x=- x_1&(\text{nghiệm đơn})\\&x=x_1&(\text{không thỏa mãn}).}$\\
        Vậy $y =\dfrac{(2x+1)\sqrt{1-x}}{f(|x|)}=\dfrac{(2x+1)\sqrt{1-x}}{a\left(x-\dfrac{1}{2}\right)\left(x+\dfrac{1}{2}\right)(x+x_1)(x-x_1)}$ \\
        Đồ thị hàm số có 2 đường tiệm cận đứng.}
\end{ex}
\begin{ex}
    \immini{ %Câu 96.
        Cho đồ thị hàm số $y =f(x)$ và trục hoành có đúng 2 điểm chung như hình bên. Đồ thị hàm số $y =\dfrac{(x-1)\sqrt{3-x}}{f(x^2)}$ có tất cả bao nhiêu tiệm cận đứng?
        \choice
        {1}
        {3}
        {4}
        {\True 2}}{\begin{tikzpicture}[scale=.8, font=\footnotesize, line join=round, line cap=round, >=stealth]
            \def\xmin{-1.5}\def\xmax{2}\def\ymin{-1}\def\ymax{4.5}
            \draw[->] (\xmin-0.2,0)--(\xmax+0.2,0) node[below] {\footnotesize $x$};
            \draw[->] (0,\ymin-0.2)--(0,\ymax+0.2) node[right] {\footnotesize $y$};
            \draw (0,0) node [below left] {\footnotesize $O$};
            \foreach \x in {1}\draw (\x,0.1)--(\x,-0.1) node [below] {\footnotesize $\x$};
            \foreach \x in {-1}\draw (\x,0.1)--(\x,-0.1) node [below left] {\footnotesize $\x$};
            \clip (\xmin,\ymin) rectangle (\xmax,\ymax);
            \draw[smooth,samples=200,domain=-1.1:0] plot(\x,{21.044670464836045*(\x)^(3.0)+24.701786337609526*(\x)^(2.0)+5.65711587277348*(\x)+2.0});
            \draw[smooth,samples=200,domain=0:\xmax] plot(\x,{10.591704641658401*(\x)^(3.0)-19.26315454354621*(\x)^(2.0)+6.6714499018878115*(\x)+2.0});
    \end{tikzpicture}}
    \loigiai{
        * Điều kiện: $\heva{&f(x^2) \ne 0\\&x \le 3.}$\\
        Nhìn hình vẽ ta thấy
        $f(x^2)=0\Leftrightarrow \hoac{&x^2=-1\\&x^2=1}\Leftrightarrow x=\pm 1\,(\text{nghiệm kép}).$\\
        Vậy $y=\dfrac{(x-1)\sqrt{3-x}}{f(x^2)}=\dfrac{(x-1)\sqrt{3-x}}{(x-1)^2(x+1)^2}$ \\
        Đồ thị hàm số có 2 đường tiệm cận đứng.}
\end{ex}
\begin{ex}%[2D1G4-3]%Câu 52
    Cho hàm số $y=ax^3+bx^2+cx+d$ có đồ thị như hình vẽ. Đồ thị của hàm số $g(x)=\dfrac{x^2-x}{f^2(x)-2f(x)}$ có bao nhiêu đường tiệm cận đứng?
    \choice
    {$2$}
    {$3$}
    {\True $4$}
    {$5$}
    \begin{center}
        \begin{tikzpicture}[thick,>=stealth,x=1cm,y=1cm,scale=.7]
            \draw[thin,color=gray!50] (-3.3,-1.3) grid (3.9,5.9);
            \draw[->] (-3.2,0) -- (4.2,0) node[right] {$x$};
            \draw[->] (0,-1.2) -- (0,5.2) node[above] {$y$};
            \draw[color=blue, domain=-2.15:2.15,samples=300] plot (\x,{(\x)^3-3*(\x)+2}) node[right] {$y=f(x)$};
            \draw (-2,0) circle (1.5pt) node[below left]{$-2$};
            \draw (-1,0) circle (1.5pt) node[below]{$-1$};
            \draw (0,0) circle (1.5pt) node[above left]{$O$};
            \draw (1,0) circle (1.5pt) node[below]{$1$};
            \draw (0,4) circle (1.5pt) node[right]{$4$};
            \draw (-1,4) circle (1.5pt);
            \draw[dashed] (-1,0)--(-1,4)--(0,4);
            \draw[red] (-3,2)--(3.2,2);
            \draw[red] (3.5,2) node[right]{$f(x)=2$};
        \end{tikzpicture}
    \end{center}
    \loigiai{
        Xét phương trình $f^2(x)-2f(x)=0 \Leftrightarrow \hoac{&f(x)=0\\&f(x)=2}\Leftrightarrow \hoac{&x=1 \, (\textrm{nghiệm kép trùng nghiệm đơn ở tử số})\\&x=-2\, (\textrm{nghiệm đơn khác nghiệm của tử})\\&x=a\in(-2; -1)\\&x=0\, (\textrm{nghiệm đơn trùng nghiệm ở tử})\\&x=b\in(1; 2)}$\\
        \textbf{Kết luận:} Đồ thị hàm số có $4$ đường tiệm cận đứng.
    }
\end{ex}
\begin{ex}%[Thi thử L3, Lương Thế Vinh, Hà Nội, 2018]%[Phạm Toàn, Dự án (12EX-10)]%[2D1G4-3]%
    \immini{Cho hàm số $y=f(x)$ có đạo hàm liên tục trên $\mathbb{R}$. Đồ thị hàm $f(x)$ như hình vẽ. Số đường tiệm cận đứng của đồ thị hàm số $y=\dfrac{x^2-1}{f^2(x)-4f(x)}$ bằng
        \choice
        {$3$}
        {$1$}
        {$2$}
        {\True $4$}
    }{\begin{tikzpicture}[>=stealth,x=1cm,y=0.75cm,scale=0.7]
            \draw[->] (-2.5,0)--(0,0)%
            node[below right]{$O$}--(2.5,0) node[below]{$x$};
            \draw[->] (0,-2) --(0,5) node[right]{$y$};
            \foreach \x in {-1,1}{
                \draw (\x,0) node[below]{\footnotesize $\x$} circle (1pt);%Ox
            }
            \foreach \y in {2,4}{
                \draw (0,\y) node[right]{\footnotesize $\y$} circle (1pt);%Oy
            }
            \draw[samples=100,domain=-2.05:2] plot (\x,{(\x -1)^2*(\x+2)});
            \draw [dashed] (-1,0)--(-1,4)--(0,4);
            \draw(-1,4) circle (1pt);
    \end{tikzpicture}}
    \loigiai{Xét $f^2(x)-4f(x)=0\Leftrightarrow \hoac{& f(x)=0\\ &f(x)=4.}$\\
        Xét $f(x)=0$ có hai nghiệm, nghiệm $x_1\ne \pm 1$ và nghiệm $x_2=1$ là nghiệm bội (do đồ thị tiếp xúc với trục hoành tại $x=1$. Trường hợp này có $2$ tiệm cận đứng.\\
        Xét $f(x)=4$ có hai nghiệm, nghiệm $x_3\ne \pm 1$ và nghiệm $x_4=-1$ là nghiệm bội (do đồ thị tiếp xúc với đường thẳng $y=4$ tại $x=-1$. Trường hợp này có $2$ tiệm cận đứng.\\
        Vậy đồ thị có $4$ tiệm cận đứng.}
\end{ex}
\begin{ex}%[Thi thử, Trường THPT Lý Thái Tổ - Bắc Ninh, 2019]%[Duong Xuan Loi, 12EX3]%[2D1G4-3]%
    \immini{
        Cho hàm số $f(x)$ có đồ thị như hình bên. Số đường tiệm cận đứng của đồ thị hàm số
        $y=\dfrac{(x^2-4)(x^2+2x)}{[f(x)]^2+2f(x)-3}$ là
        \choice
        {\True $4$}
        {$5$}
        {$3$}
        {$2$}
    }{
        \begin{tikzpicture}[scale=0.5, font=\footnotesize, line join=round, line cap=round, >=stealth]
            \def\a{1} \def\b{-8} \def\c{1} % Hệ số
            \def\xt{-3.7} \def\xp{4} \def\yt{2} \def\yd{-3.7} % x_trái, x_phải, y_trên, y_dưới (giới hạn)
            \draw[->] (\xt,0)--(\xp,0) node [below]{$x$};
            \draw[->] (0,\yd)--(0,\yt) node [left]{$y$};
            \node at (0,0) [below left]{$O$};
            \clip (\xt-0.1,\yd+0.1) rectangle (\xp-0.1,\yt-0.1);
            \draw[smooth,samples=300] plot(\x,{1/4*(\a*(\x)^4+\b*(\x)^2)+\c});
            \draw[dashed] (-2,0)node[above]{$-2$}--(-2,-3)--(2,-3)--(2,0)node[above]{$2$};
            \node at (0,-3)[above left]{$-3$};
            \node at (-3,0)[above left]{$-3$};
            \node at (0,1)[above right]{$1$};
            \node at (3,0)[above right]{$3$};
            \fill (0,0) circle (1pt) (0,-3) circle (1pt) (2,0) circle (1pt) (-2,0) circle (1pt) (-3,0) circle (1pt) (0,1) circle (1pt) (3,0) circle (1pt);
        \end{tikzpicture}
    }
    \loigiai{
        Ta có $y=\dfrac{(x^2-4)(x^2+2x)}{[f(x)]^2+2f(x)-3}$ có các nghiệm ở tử là $x=0$ (bội $1$), $x=2$ (bội $1$), $x=-2$ (bội $2$).\\
        Mặt khác, từ đồ thị $f(x)$ ta thấy hàm số $y=\dfrac{(x^2-4)(x^2+2x)}{[f(x)]^2+2f(x)-3}$ có các nghiệm ở mẫu là
        $f^2(x)+2f(x)-3=0\Leftrightarrow \hoac{& f(x)=1 \\ & f(x)=-3}
        \Leftrightarrow \hoac{& x=0,x=x_1,x=x_2 \\ & x=-2,x=2.}$\\
        Trong đó nghiệm $x=0$, $x=-2$, $x=2$ đều có bội $2$ và $x_1$, $x_2$ khác các nghiệm của tử.\\
        So sánh bội nghiệm ở mẫu và bội nghiệm ở tử thì thấy đồ thị có các tiệm cận đứng là $x=0$, $x=2$; $x=x_1$; $x=x_2$.
    }
\end{ex}
\begin{ex}%[Thi thử, THPT Sơn Tây, Hà Nội, 2019]%[Huỳnh Xuân Tín, 12EX3]%[2D1G4-3]%
    \immini{Cho hàm số $ f(x)=(x+3)(x+1)^2(x-1)(x-3)$ có đồ thị như hình vẽ. Đồ thị hàm số $ g(x)=\dfrac{\sqrt{x-1}}{f^2(x)-9f(x)}$ có bao nhiêu tiệm cận đứng và tiệm cận ngang?
        \choice
        {$3$}
        {\True$ 4$}
        {$ 9$}
        { $8$}
    }{\begin{tikzpicture}[scale=0.3, font=\footnotesize, line join=round, line cap=round, >=stealth]
            %\draw[dashed, line width=0.1pt, gray] (-3.2,-5.5) grid (5.2,4.5);
            \draw[->] (-3.5,0)--(0,0) node[below right]{$O$}--(3.6,0) node[below]{$x$};
            \draw[fill=black] (0,0) circle (1pt);
            \draw[->] (0,-7.7) --(0,6.5) node[right]{$y$};
            \foreach \x in {-3,-1,3}{
                \draw[fill=black] (\x,0) node[below left]{$\x$} circle (1pt);}
            \draw[fill=black] (1,0) node[below right]{$1$} circle (1pt);
            \draw[fill=black] (0,1.35) node[above left]{$9$} circle (1pt);
            \draw [black, domain=-3.2:3.18, samples=100] %
            plot(\x,{0.15*(\x+3)*(\x+1)^2*(\x-1)*(\x-3)});
    \end{tikzpicture}}
    \loigiai{Điều kiện xác định của hàm số $g(x)$ là $\heva{&x\ge1\\ &f^2(x)-9f(x)\not=0.}$\\
        Từ $f^2(x)-9f(x)=0\Leftrightarrow \hoac{&f(x)=0\\&f(x)=9.}$\\
        Với $f(x)=0$ có nghiệm là $x=\pm 1, x=\pm 3$.\\
        Dựa vào đồ thị ta thấy nghiệm của phương trình $f(x)=9$ là hoành độ giao điểm của đường thẳng $y=9$ với đồ thị hàm số $y=f(x)$ nên có nghiệm là $-3<x_3<x_2<-1<0<x_1<1<3<x_0$.\\
        Do đó tập xác định của hàm số $y=g(x)$ là $\mathscr{D}=\left[1;+\infty \right)\setminus\left\lbrace1;3;x_0 \right\rbrace $.\\
        Khi đó ta có \begin{itemize}
            \item $\lim\limits_{x\rightarrow1^+ } g(x)=\lim\limits_{x\rightarrow1^+ }\dfrac{\sqrt{x-1}}{f(x)\left(f(x)-9 \right)}=+\infty$ (vì $x$ tiến gần bên phải $1$ thì $f(x)<0, f(x)-9<0$), suy ra đường thẳng $x=1$ là tiệm cận đứng.
            \item $\lim\limits_{x\rightarrow3^+ } g(x)=\lim\limits_{x\rightarrow3^+ }\dfrac{\sqrt{x-1}}{f(x)\left(f(x)-9 \right)}=-\infty$ (vì $x$ tiến gần bên phải $3$ thì $f(x)>0, f(x)-9<0$), suy ra đường thẳng $x=3$ là tiệm cận đứng.
            \item $\lim\limits_{x\rightarrow x_0^+} g(x)=\lim\limits_{x\rightarrow x_0^+ }\dfrac{\sqrt{x-1}}{f(x)\left(f(x)-9 \right)}=+\infty$ (vì $x$ tiến gần bên phải $x_0$ thì $f(x)>0, f(x)-9>0$), suy ra đường thẳng $x=x_0$ là tiệm cận đứng.
        \end{itemize}
        Và $\lim\limits_{x\rightarrow +\infty} g(x)=\lim\limits_{x\rightarrow +\infty }\dfrac{\sqrt{x-1}}{f(x)\left(f(x)-9 \right)}=0$ (vì bậc ở mẫu của $y=g(x)$ là $10$ và bậc tử của nó là $\dfrac{1}{2}$). Do vậy đồ thị hàm số $y=g(x)$ có một tiệm cận ngang là đường thẳng $y=0$.\\
        Vậy đồ thị hàm số $y=g(x)$ có bốn tiệm cận ngang và đứng. }
\end{ex}
\begin{ex}%[Thi thử, Chuyên Quang Trung-Bình Phước, 2021,lần 1]%[Trần Hòa, 12EX6]%[2D1G4-3]%
    \immini{Cho hàm số $y=f(x)=ax^3+bx^2+cx+d$, có đồ thị như hình vẽ. Số đường tiệm cận đứng của đồ thị hàm số $y=\dfrac{x^2+x-2}{f^2(x)-f(x)}$ là
        \choice
        {$3$}
        {$2$}
        {\True $4$}
        {$5$}}
    {\begin{tikzpicture}[scale=.5, font=\footnotesize, line join=round, line cap=round, >=stealth]
            \draw[->] (-2.5,0)--(0,0) node[below right]{$O$}--(2,0) node[below]{$x$};
            \draw[->] (0,-.5) --(0,4.5) node[right]{$y$};
            \draw [domain=-2.05:2.05, samples=100] %
            plot (\x, {(\x+2)*(\x-1)^2});
            \draw[fill] (0,0) circle (1pt);
            \foreach \x/\g in {-2/140,-1/-90,1/-90}
            \draw[fill] (\x,0) circle(.5pt)node [shift={(\g:.3)}] {$\x$};
            \foreach \y/\g in {2/0,4/0}
            \draw[fill] (0,\y) circle(.5pt)node [shift={(\g:.3)}] {$\y$};
            \draw[dashed] (-1,0)--(-1,4)--(0,4);
    \end{tikzpicture}}
    \loigiai{
        \begin{itemize}
            \item $x^2+x-2=(x-1)(x+2)$.\\
            \item Dựa vào đồ thị hàm số $y=f(x)$ ta có $f^2(x)-f(x)=0\Leftrightarrow\hoac{&f(x)=0\\&f(x)=1.}$\\
            $f(x)=0\Leftrightarrow x=-2$, $x=1$ (nghiệm kép).\\
            $f(x)=1\Leftrightarrow\hoac{&x=x_1,(x_1\in (-2;-1))\\&x=x_2,(x_2\in (0;1))\\&x=x_3,(x_3>1). }$
            \item Do đó $y=\dfrac{(x-1)(x+2)}{a^2(x+2)(x-1)^2(x-x_1)(x-x_2)(x-x_3)}$.
        \end{itemize}
        Suy ra đồ thị có các đườn tiệm cận đứng $x=1$, $x=x_1$, $x=x_2$, $x=x_3$.
    }
\end{ex}
\begin{ex}%[Đề thi hết học kì 2, Bình Minh, Ninh Bình 2018]%[Nguyễn Tuấn Anh, dự án EX9]%[2D1G4-3]%
    \immini{Cho hàm số bậc ba $f(x)=ax^3+bx^2+cx+d$ có đồ thị như hình vẽ bên dưới. Hỏi đồ thị hàm số $g(x)=\dfrac{(x^2-3x+2)\sqrt{x-1}}{x[f^2(x)-f(x)]}$ có bao nhiêu tiệm cận đứng?
        \choice
        {$5$}
        {$6$}
        {\True $3$}
        {$4$}
    }{
        \begin{tikzpicture}[line width=1.0pt,line join=round,>=stealth,x=1cm,y=1cm,scale=1.0]
            \draw[->,line width = 1pt] (-1,0)--(0,0) node[below right]{$O$}--(4,0) node[below]{$x$};
            \draw[->,line width = 1pt] (0,-1.5) --(0,2.5) node[right]{$y$};
            \foreach \x in {1,2}{
                \draw (\x,0) node[below]{$\x$} circle (1pt);
            }
            \foreach \y in {1}{
                \draw (0,\y) node[left]{$\y$} circle (1pt);
            }
            \clip(-0.8,-1) rectangle (3.8,2.3);
            \draw [line width=1.0pt, thick, domain=-0.5:3.5, samples=100]%,domain=-1.5:3] %
            plot (\x, {(5*(\x)-4)*((\x)-2)^2});
            \draw [dash pattern=on 4pt off 4pt] (1.,0.)-- (1.,1.)-- (0.,1.);
            \draw (1,1) circle (1pt);
        \end{tikzpicture}
    }
    \loigiai{
        Điều kiện $\heva{&x\geq 1\\ &x\ne 0\\ &f^2(x)-f(x)\ne 0}\Leftrightarrow \heva{&x\geq 1\\ &f(x)\ne 0\\ & f(x)\ne 1.}$\\
        Dựa vào đồ thị hàm số $y=f(x)$, ta thấy $f(x)=0$ có hai nghiệm, một nghiệm $x_1<1$ và một nghiệm kép bằng $2$. Do đó ta biểu diễn được $f(x)$ dưới dạng
        $$ f(x)=a(x-x_1)(x-2)^2. $$
        Dựa vào đồ thị hàm số $y=f(x)$, ta thấy phương trình $f(x)=1$ có ba nghiệm $1,x_2, x_3$, với $1<x_2<2<x_3$. Do đó ta biểu diễn được $f(x)-1$ dưới dạng
        $$ f(x)-1=a(x-1)(x-x_2)(x-x_3). $$
        Lúc này điều kiện được viết lại như sau $\heva{&x>1\\ &x\ne x_2, x\ne 2, x\ne x_3.}$\\
        Với điều kiện đó thì $g(x)$ được viết lại là
        $$ g(x)=\dfrac{\sqrt{x-1}}{a^2x(x-x_1)(x-x_2)(x-2)(x-x_3)}. $$
        Ta có
        \begin{align*}
            &\lim\limits_{x\to 1^+}g(x)=\lim\limits_{x\to 1^+}\dfrac{\sqrt{x-1}}{a^2x(x-x_1)(x-x_2)(x-2)(x-x_3)}=0,\\
            & (x=1\mbox{ \textbf{không} là tiệm cận đứng}) \\
            &\lim\limits_{x\to x_2^+}g(x)=\lim\limits_{x\to x_2^+}\dfrac{\sqrt{x-1}}{a^2x(x-x_1)(x-x_2)(x-2)(x-x_3)}=+\infty,\\
            & (x=x_2\mbox{ là tiệm cận đứng}) \\
            &\lim\limits_{x\to 2^+}g(x)=\lim\limits_{x\to 2^+}\dfrac{\sqrt{x-1}}{a^2x(x-x_1)(x-x_2)(x-2)(x-x_3)}=-\infty,\\
            & (x=2\mbox{ là tiệm cận đứng}) \\
            &\lim\limits_{x\to x_3^+}g(x)=\lim\limits_{x\to x_3^+}\dfrac{\sqrt{x-1}}{a^2x(x-x_1)(x-x_2)(x-2)(x-x_3)}=+\infty,\\
            & (x=x_3\mbox{ là tiệm cận đứng}) \\
        \end{align*}
        Vậy đồ thị hàm số $g(x)$ có tất cả $3$ tiệm cận đứng.
    }
\end{ex}
\begin{ex}%[VDC5-Đỗ Đường Hiếu]%[2D1G4-3]%
    \immini{Cho hàm số $f(x)=(x+3)(x+1)^2(x-1)(x-3)$ có đồ thị như hình vẽ. Đồ thị hàm số $g(x)=\dfrac{\sqrt{x-1}}{f^2(x)-9f(x)}$ có bao nhiêu tiệm cận đứng và tiệm cận ngang?
        \choice
        {$3$}
        {\True $4$}
        {$9$}
        {$8$}}
    {\begin{tikzpicture}[xscale=0.8,yscale=0.05, line join=round, line cap=round,font=\footnotesize,>=stealth]
            \draw[->] (-4,0)--(4,0) node[below]{$x$};
            \draw[->] (0,-56)--(0,30) node[left]{$y$};
            \coordinate[label=below left:$O$] (O) at (0,0);
            \draw (-1,0) node[below] { $-1$}(1,0) node[below] { $1$};
            \draw (-3,0) node[below left] { $-3$};
            \draw (3,0) node[below right] { $3$};
            \clip (-3.3,-60) rectangle (3.5,26);
            \draw[smooth,samples=300,domain=-3.5:3.5] plot(\x,{(\x+3)*(\x+1)^2*(\x-1)*(\x-3)});
            \foreach \x in {-3,-1,1,3}
            \draw[shift={(\x,0)},color=black] (0pt,20pt) -- (0pt,-20pt);
            \draw[shift={(0,9)},color=black] (2pt,0pt) -- (-2pt,0pt) node[left] {$9$};
        \end{tikzpicture}
    }
    \loigiai{%GV tổng quát hóa bài toán:
        Cho hàm số đa thức $y=f(x)$ có đồ thị $(C)$. Tìm số đường tiệm cận của đồ thị hàm số $g(x)=\dfrac{\sqrt{ax+b}}{P\left(f(x) \right) }$, trong đó $P\left(f(x) \right)$ là một đa thức của $f(x)$.
        Nếu $a>0$ thì $\lim\limits_{x\to +\infty}g(x)=0$.\\
        Nếu $a<0$ thì $\lim\limits_{x\to -\infty}g(x)=0$.\\
        Do đó đồ thị hàm số $y=g(x)$ luôn có duy nhất một đường tiệm cận ngang là $y=0$.\\
        Gọi $x=x_0$ là một nghiệm của phương trình $P\left(f(x) \right) =0$ thỏa mãn điều kiện $ax+b\ge 0$. Rõ ràng khi đó $\lim\limits_{x\to x_0^+}g(x)=+\infty$ hoặc $\lim\limits_{x\to x_0^+}g(x)=-\infty$.\\
        Bởi vậy, số đường tiệm cận đứng của đồ thị hàm số $y=g(x)$ chính là số nghiệm của phương trình $P\left(f(x) \right) =0$ thỏa mãn điều kiện $ax+b\ge 0$.
        \immini{Ta có $f^2(x)-9f(x)=0\Leftrightarrow \hoac{&f(x)=0\\&f(x)=9.}$\\
            \begin{itemize}
                \item $f(x)=0$ có các nghiệm thuộc $\left[1;+\infty\right)$ là $x=1$ và $x=3$.
                \item Đường thẳng $y=9$ cắt đồ thị hàm số $y=f(x)$ tại duy nhất một điểm có hoành độ thuộc $\left[1;+\infty\right)$ là $x=a>3$.
            \end{itemize}
        }
        {\begin{tikzpicture}[xscale=0.8,yscale=0.05, line join=round, line cap=round,font=\footnotesize,>=stealth]
                \draw[->] (-4,0)--(4,0) node[below]{$x$};
                \draw[->] (0,-56)--(0,30) node[left]{$y$};
                \coordinate[label=below left:$O$] (O) at (0,0);
                \draw (-4,9)--(4,9);
                \draw (-1,0) node[below] { $-1$}(1,0) node[below] { $1$};
                \draw (-3,0) node[below left] { $-3$};
                \draw (3,0) node[below right] { $3$};
                \clip (-3.3,-60) rectangle (3.5,26);
                \draw[smooth,samples=300,domain=-3.5:3.5] plot(\x,{(\x+3)*(\x+1)^2*(\x-1)*(\x-3)});
                \foreach \x in {-3,-1,1,3}
                \draw[shift={(\x,0)},color=black] (0pt,20pt) -- (0pt,-20pt);
                \draw[shift={(0,9)},color=black] (2pt,0pt) -- (-2pt,0pt) node[above left] {$9$};
        \end{tikzpicture}}
        \noindent
        Bởi vậy, hàm số $g(x)=\dfrac{\sqrt{x-1}}{f^2(x)-9f(x)}$ có tập xác định là $\mathscr D=\left[1;3\right) \cup \left(3;a\right) \cup\left( a;+\infty\right)$.\\
        Khi đó ta có
        \begin{itemize}
            \item $\lim\limits_{x\to+\infty}g(x)=0$ nên đồ thị hàm số $y=g(x)$ có một đường tiệm cận ngang là đường thẳng $y=0$.
            \item $\lim\limits_{x\to 1^+}g(x)=\lim\limits_{x\to 1^+}\dfrac{\sqrt{x-1}}{f(x)\left[f(x)-9\right] }=+\infty$;\\
            $\lim\limits_{x\to 3^+}g(x)=\lim\limits_{x\to 3^+}\dfrac{\sqrt{x-1}}{f(x)\left[f(x)-9\right] }=-\infty$;\\
            $\lim\limits_{x\to a^+}g(x)=\lim\limits_{x\to a^+}\dfrac{\sqrt{x-1}}{f(x)\left[f(x)-9\right] }=+\infty$.\\
            Do đó nên đồ thị hàm số $y=g(x)$ có $3$ đường tiệm cận đứng là các đường thẳng $x=1$, $x=3$ và $x=a$.
        \end{itemize}
        Như vậy, đồ thị hàm số $y=g(x)$ có $4$ đường tiệm cận, trong đó có $1$ đường tiệm cận ngang và $3$ đường tiệm cận đứng.
    }
\end{ex}
\begin{ex}%[VDC5-Đỗ Đường Hiếu]%[2D1G4-3]%
    \immini{Cho hàm số bậc ba $y=f(x)$ có đồ thị như hình vẽ bên. Đồ thị hàm số $g(x)=\dfrac{x\sqrt{x+1}}{f(x)\left[f^2(x)-16 \right] }$ có bao nhiêu tiệm cận đứng?
        \choice
        {\True $4$}
        {$5$}
        {$6$}
        {$7$}}
    {\begin{tikzpicture}[scale=0.6,line join=round, line cap=round,font=\footnotesize,>=stealth]
            \draw[->] (-2.5,0)--(4,0) node[below]{$x$};
            \draw[->] (0,-5)--(0,2.5) node[left]{$y$};
            \coordinate[label=below left:$O$] (O) at (0,0);
            \draw[dashed] (-1,0)--(-1,-4)--(0,-4);
            \clip (-2.3,-5) rectangle (3.5,2.5);
            \draw[smooth,samples=300,domain=-3.5:3.5] plot(\x,{-0.5*(\x+2)*(\x-1)*(\x-3)});
            \foreach \x in {-2,-1,1,3}
            \draw[shift={(\x,0)},color=black] (0pt,2pt) -- (0pt,-2pt) node[above] { $\x$};
            \foreach \y in {-4,-3,1}
            \draw[shift={(0,\y)},color=black] (2pt,0pt) -- (-2pt,0pt) node[right] {$\y$};
        \end{tikzpicture}
    }
    \loigiai{
        Xét phương trình $f(x)\left[f^2(x)-16 \right]=0$ \, $(*)$, với điều kiện $x\in\left[-1;+\infty \right) $.\\
        Ta có $f(x)\left[f^2(x)-16 \right]=0\Leftrightarrow \hoac{&f(x)=0\\&f(x)=4\\&f(x)=-4.}$\\
        \begin{itemize}
            \item Phương trình $f(x)=0$ có hai nghiệm $x\in\left[-1;+\infty \right) $ là $x=1$ và $x=3$.
            \item Phương trình $f(x)=4$ có không có nghiệm $x\in\left[-1;+\infty \right) $.
            \item Phương trình $f(x)=-4$ có hai nghiệm $x\in\left[-1;+\infty \right) $ là $-1<x_1<0$ và $x_2>3$.
        \end{itemize}
        Rõ ràng $\lim\limits_{x\to x_0^+}g(x)=+\infty$ hoặc $\lim\limits_{x\to x_0^+}g(x)=-\infty$, trong đó $x=x_0$ là nghiệm thuộc $\left[-1;+\infty \right) $ của phương trình $(*)$. Do đó đường thẳng $x=x_0$ là tiệm cận đứng của đồ thị hàm số $y=g(x)$.\\
        Từ đó suy ra đồ thị hàm số $g(x)=\dfrac{x\sqrt{x+1}}{f(x)\left[f^2(x)-16 \right] }$ có $4$ tiệm cận đứng.
    }
\end{ex}
\begin{ex}%[VDC5-Đỗ Đường Hiếu]%[2D1G4-3]%
    \immini{Cho $y=f(x)$ là hàm số đa thức có đồ thị như hình vẽ bên. Đặt $g(x)=\dfrac{\sqrt{x-1}}{\left[f(x)\right]^2-2f(x)}$ có bao nhiêu đường tiệm cận đứng?
        \choice
        {$5$}
        {$3$}
        {$4$}
        {\True $2$}}
    {\begin{tikzpicture}[scale=0.6,line join=round, line cap=round,font=\footnotesize,>=stealth]
            \draw[->] (-3,0)--(2.5,0) node[below]{$x$};
            \draw[->] (0,-1)--(0,5) node[left]{$y$};
            \coordinate[label=above left:$O$] (O) at (0,0);
            \draw[dashed] (-1,0)--(-1,4)--(0,4);
            \clip (-2.3,-1) rectangle (2.5,4.5);
            \draw[smooth,samples=300,domain=-3.5:3.5] plot(\x,{(\x)^3-3*(\x)+2});
            \foreach \x in {-2,-1,1}
            \draw[shift={(\x,0)},color=black] (0pt,2pt) -- (0pt,-2pt) node[below] { $\x$};
            \foreach \y in {2,4}
            \draw[shift={(0,\y)},color=black] (2pt,0pt) -- (-2pt,0pt) node[right] {$\y$};
        \end{tikzpicture}
    }
    \loigiai{
        Xét phương trình $\left[f(x)\right]^2-2f(x)=0$ \, $(*)$, với điều kiện $x\in\left[1;+\infty \right) $.\\
        Ta có $\left[f(x)\right]^2-2f(x)=0\Leftrightarrow \hoac{&f(x)=0\\&f(x)=2.}$\\
        \begin{itemize}
            \item Phương trình $f(x)=0$ có một nghiệm $x\in\left[1;+\infty \right) $ là $x=1$.
            \item Phương trình $f(x)=2$ có một nghiệm $x\in\left[1;+\infty \right) $ là $x=x_1>1$.
        \end{itemize}
        Rõ ràng $\lim\limits_{x\to x_0^+}g(x)=+\infty$ hoặc $\lim\limits_{x\to x_0^+}g(x)=-\infty$, trong đó $x=x_0$ là nghiệm thuộc $\left[1;+\infty \right) $ của phương trình $(*)$. Do đó đường thẳng $x=x_0$ là tiệm cận đứng của đồ thị hàm số $y=g(x)$.\\
        Từ đó suy ra đồ thị hàm số $g(x)=\dfrac{\sqrt{x-1}}{\left[f(x)\right]^2-2f(x)}$ có $2$ tiệm cận đứng.
    }
\end{ex}
\begin{ex}%[VDC5-NgocDungHo]%[2D1G4-3]%
    \immini
    {
        Cho hàm số $f(x)$ có đồ thị như hình bên. Số đường tiệm cận đứng của đồ thị hàm số $y=\dfrac{(x^2-4)(x^2+2x)}{[f(x)]^2-4f(x)+3}$ là
        \choice
        {$4$}
        {\True $5$}
        {$3$}
        {$2$}
    }
    {\begin{tikzpicture}[>=stealth,scale=0.5, line join=round, line cap=round]
            \def\f[#1]{-0.25*((#1)^4-8*(#1)^2+4)}
            \draw[->] (-4.1,0)--(4,0) node [below]{$x$};
            \draw[->] (0,-2)--(0,4) node [left]{$y$};
            \node at (0,0) [above left]{$O$};
            % \clip;
            \draw[domain=-2.9:2.9,samples=300,thick] plot (\x,{\f[\x]});
            \foreach \x in {-2,2} \filldraw (\x,0) node[below]{\x} circle (2pt);
            %\foreach \x in {-3,3} \filldraw (\x,0) node[below left]{\x} circle (2pt);
            \filldraw (-3,0) node[below left]{$-3$} circle (2pt);
            \filldraw (3,0) node[below right]{$3$} circle (2pt);
            \filldraw (0,1) node[left]{$1$} circle (2pt);
            \filldraw (0,3) node[above left]{$3$} circle (2pt);
            \draw[dashed](-2,0)--(-2,3)--(2,3)--(2,0);
            \draw (3,-1.75) node[right]{$y=f(x)$};
        \end{tikzpicture}
    }
    \loigiai{
        Xét hàm số $y=g(x)=\dfrac{(x^2-4 )(x^2+2x)}{[f(x)]^2-4f(x)+3}$.
        \immini
        {
            Giải phương trình $(x^2-4)(x^2+2x)=0 $\\
            $\Leftrightarrow \hoac{& x^2-4=0 \\ & x^2+2x=0}\Leftrightarrow \hoac{& x=\pm 2 \\ & x=0.}$\\
            Giải phương trình $[f(x)]^2-4f(x)+3=0$\\
            $ \Leftrightarrow \hoac{& f(x)=1 \\ & f(x)=3} \Leftrightarrow \hoac{& x = \pm 2 \\ & x=a\\&x=b\\&x=c\\&x=d.}$\\ với $-3<a<-2<b<c<2<d<3$.\\
        }
        {\begin{tikzpicture}[>=stealth,scale=0.8, line join=round, line cap=round]
                \def\f[#1]{-0.25*((#1)^4-8*(#1)^2+4)}
                \def\g[#1]{1}
                \def\h[#1]{3}
                \draw[->] (-4.1,0)--(4,0) node [below]{$x$};
                \draw[->] (0,-2)--(0,4) node [left]{$y$};
                \node at (0,0) [above left]{$O$};
                % \clip;
                \draw[domain=-2.9:2.9,samples=300,thick] plot (\x,{\f[\x]});
                \draw[domain=-4:4,samples=300,thick] plot (\x,{\g[\x]});
                \draw[domain=-4:4,samples=300,thick] plot (\x,{\h[\x]});
                \foreach \x in {-3,-2,2,3} \filldraw (\x,0) node[below]{\x} circle (2pt);
                % \filldraw (-3,0) node[above left]{$-3$} circle (2pt);
                % \filldraw (3,0) node[above ]{$3$} circle (2pt);
                \filldraw (0,1) node[below left]{$1$} circle (2pt);
                \filldraw (0,-1) node[below left]{$-1$} circle (2pt);
                \filldraw (0,3) node[above left]{$3$} circle (2pt);
                \draw[dashed](-2,0)--(-2,3) (2,3)--(2,0) (2.61,0)node[below]{$d$}--(2.61,1) (-2.61,0)node[below]{$a$}--(-2.61,1) (1.08,0)node[below]{$c$}--(1.08,1)(-1.08,0)node[below]{$b$}--(-1.08,1);
                \draw (3,2.75) node[right]{$y=f(x)$};
            \end{tikzpicture}
        }
        Trong điều kiện xác định của hàm số $y=g(x)$ ta có thể viết $$y=g(x)=\dfrac{x(x-2)(x+2)^2}{(x-a)(x-b)(x-c)(x-d) (x-2)^2(x+2)^2}=\dfrac{x}{(x-a)(x-b)(x-c)(x-d)(x-2)}$$
        Vậy số tiệm cận đứng của đồ thị hàm số $y=g(x)$ bằng $5$.
    }
\end{ex}
\Closesolutionfile{ans}
%%Bài 4. Đồ thị
% \section{KHẢO SÁT SỰ BIẾN THIÊN VÀ VẼ ĐỒ THỊ HÀM SỐ}
\subsection{LÝ THUYẾT CẦN NHỚ}
\subsubsection{Sơ đồ khảo sát hàm số y= f(x)}
\begin{tcolorbox}[colframe=cyan,colback=red!3!white,boxrule=0.5mm]
		\begin{itemize}
		\item[\iconCH] \indamm{Bước 1.} Tìm tập xác định của hàm số.
		\item [\iconCH] \indamm{Bước 2.} Khảo sát sự biến thiên của hàm số
		\begin{itemize}
			\item Tính đạo hàm $y'$. Tìm các điểm mà tại đó $y'$ bằng $0$ hoặc đạo hàm không tồn tại.
			\item Tìm các giới hạn tại vô cực, giới hạn vô cực và tìm tiệm cận của đồ thị hàm số.
			\item Lập bảng biến thiên; xác định chiều biến thiên và cực trị của hàm số.
		\end{itemize}
		\item [\iconCH] \indamm{Bước 3.} Cho thêm điểm và vẽ đồ thị của hàm số dựa vào bảng biến thiên.
	\end{itemize}
\end{tcolorbox}
\subsubsection{Hàm số bậc ba $\mathbf{y=ax^3+bx^2+cx+d}$}
\	\begin{minipage}[b]{10cm}
		\begin{enumerate}[\iconCH]
			\item \indamm{TH1.} $y'=0$ có hai nghiệm phân biệt $x_1$ và $x_2$. Khi đó, hàm số có hai điểm cực trị $x=x_1$ và $x=x_2$.\\
			\begin{tikzpicture}[smooth,samples=300,line width=0.6pt,scale=0.8,>=stealth,font=\footnotesize]
				\draw[->] (-2.5,0)--(2.5,0) node[below]{$x$};
				\draw[->] (0,-1)--(0,2) node[right]{$y$};
				\draw (0,0) node[below left]{$O$};
				\draw[blue,line width=1pt,domain=-2.1:2.1] plot(\x,{0.4*((\x)^3-3*(\x)+1)});
				\draw[fill=black] (0,0.4) circle(2pt) (-1,1.2) circle(2pt) (1,-0.4) circle(2pt);
				\draw[dashed] (1,0)node[above]{\footnotesize$x_2$}--(1,-0.4)--(0,-0.4) (-1,0)node[below]{\footnotesize$x_1$}--(-1,1.2)--(0,1.2);
				\node[right] at (0,0.6) {\footnotesize $I$};
				\node[right] at (-2,2) {\tiny\fbox{$a>0$}};
			\end{tikzpicture}
			\hspace{0.3cm}
			\begin{tikzpicture}[smooth,samples=300,line width=0.6pt,scale=0.8,>=stealth,font=\footnotesize]
				\draw[->] (-2.5,0)--(2.5,0) node[below]{$x$};
				\draw[->] (0,-1)--(0,2) node[right]{$y$};
				\draw (0,0) node[below right]{$O$};
				\draw[blue,line width=1pt,domain=-2.1:2.1] plot(\x,{0.4*(-(\x)^3+3*(\x)+1)});
				\draw[fill=black] (0,0.4) circle(2pt) (1,1.2) circle(2pt) (-1,-0.4) circle(2pt);
				\draw[dashed] (-1,0)node[above]{\footnotesize$x_1$}--(-1,-0.4)--(0,-0.4) (1,0)node[below]{\footnotesize$x_2$}--(1,1.2)--(0,1.2);
				\node[left] at (0,0.6) {\footnotesize$I$};
				\node[right] at (-2,2) {\tiny\fbox{$a<0$}};
			\end{tikzpicture}
			\item \indamm{TH2.} $y'=0$ có nghiệm kép $x_0$. Khi đó, hàm số không có cực trị.\\
			\begin{tikzpicture}[smooth,samples=300,line width=0.6pt,scale=0.8,>=stealth,font=\footnotesize]
				\draw[->] (-2,0)--(2.5,0) node[below]{$x$};
				\draw[->] (0,-1)--(0,2) node[right]{$y$};
				\draw (0,0) node[below right]{$O$};
				\draw[blue,line width=1pt,domain=-0.7:1.6] plot(\x,{(\x-0.5)^3+0.7});
				\draw[fill=black] (0.5,0.7) circle(2pt);
				\node[above] at (0.5,0.7) {\footnotesize$I$};
				\node[right] at (-2,2) {\tiny\fbox{$a>0$}};
			\end{tikzpicture}
			\hspace{0.5cm}
			\begin{tikzpicture}[smooth,samples=300,line width=0.6pt,scale=0.8,>=stealth,font=\footnotesize]
				\draw[->] (-2,0)--(2.5,0) node[below]{$x$};
				\draw[->] (0,-1)--(0,2) node[right]{$y$};
				\draw (0,0) node[below right]{$O$};
				\draw[blue,line width=1pt,domain=-0.6:1.6] plot(\x,{-((\x-0.5)^3-0.5)});
				\draw[fill=black] (0.5,0.5) circle(2pt);
				\node[above] at (0.5,0.5) {\footnotesize$I$};
				\node[right] at (-2,2) {\tiny\fbox{$a<0$}};
			\end{tikzpicture}
			\item \indamm{TH3.} $y'=0$ vô nghiệm. Khi đó, hàm số không có cực trị.\\
			\begin{tikzpicture}[smooth,samples=300,line width=0.6pt,scale=0.8,>=stealth,font=\footnotesize]
				\draw[->] (-2,0)--(2.5,0) node[below]{$x$};
				\draw[->] (0,-1)--(0,2.5) node[right]{$y$};
				\draw (0,0) node[below right]{$O$};
				\draw[blue,line width=1pt,domain=-0.5:1.5] plot(\x,{((\x-0.6)^3+0.7*(\x)+0.7)});
				\draw[fill=black] (0.6,1.12) circle(2pt);
				\node[below right] at (0.5,1.12) {\footnotesize$I$};
				\node[right] at (-2,2) {\tiny\fbox{$a>0$}};
			\end{tikzpicture}
			\hspace{0.5cm}
			\begin{tikzpicture}[smooth,samples=300,line width=0.6pt,scale=0.8,>=stealth,font=\footnotesize]
				\draw[->] (-2,0)--(2.5,0) node[below]{$x$};
				\draw[->] (0,-1)--(0,2.5) node[right]{$y$};
				\draw (0,0) node[below left]{$O$};
				\draw[blue,line width=1pt,domain=-0.5:1.5] plot(\x,{-(\x-0.6)^3-0.7*(\x)+0.7)});
				\draw[fill=black] (0.6,0.28) circle(1.5pt);
				\node[right] at (-2,2) {\tiny\fbox{$a<0$}};
				\node[above] at (0.6,0.28) {\footnotesize$I$};
			\end{tikzpicture}
		\end{enumerate}
		\vspace{0.4cm}
	\end{minipage}\hspace{0.5cm}
	\begin{minipage}[b]{6.5cm}
		\begin{khung4}{GHI NHỚ}
			\ding{172} Hàm số không có điểm cực trị
			$$b^2-3ac\le 0 \text{ hoặc } \heva{&a=0 \\&b=0.}$$
			\ding{173} Hàm số có hai điểm cực trị
			$$\heva{&a \ne 0\\&b^2-3ac >0.}$$
			\ding{174} Liên hệ tổng tích hai nghiệm
			$$\heva{&x_1+x_2=-\dfrac{2b}{3a}\\&x_1x_2=\dfrac{c}{3a}}$$
			\ding{175} Tọa độ tâm đối xứng của đồ thị, nó chính là trung điểm của đoạn nối 2 điểm cực trị. Hoành độ tâm đối xứng là nghiệm phương trình $y''=0 \Leftrightarrow x=-\dfrac{b}{3a}$.
			\ding{176} Tiếp tuyến tại tâm đối xứng sẽ có hệ số góc nhỏ nhất nếu $a>0$ và lớn nhất nếu $a<0$.
		\end{khung4}
		\vspace{0.1cm}
	\end{minipage}
\newpage
\subsubsection{Hàm số $\mathbf{y = \dfrac{{ax + b}}{{cx + d}}\left( {c \ne 0,ad - bc \ne 0} \right)}$}
\begin{minipage}[b]{10cm}
	\begin{enumerate}[\iconCH]
		\item Tập xác định $D=\mathbb{R}\backslash \left\{-\dfrac{d}{c}\right\}$; Đạo hàm $y'=\dfrac{ad-cb}{(cx+d)^2}$.
		\item Đồ thị nhận giao điểm của hai đường tiệm cận làm tâm đối xứng.
		\item Hình dạng đồ thị:\\
		\begin{tikzpicture}[smooth,samples=300,line width=0.6pt,>=stealth, scale=0.45]
			\draw[->] (-5,0)--(3.5,0) node[below]{$x$};
			\draw[->] (0,-1.6)--(0,5.7) node[right]{$y$};
			\draw (0,0) node[below right]{$O$};
			\node at (-3,5.3) {\tiny\fbox{$y'>0$}};
			\clip (-5,-1.5) rectangle (3,5.5);
			\draw[dashed] (-1,-2)--(-1,5.5) (-5,2)--(3,2);
			\draw[blue,line width=1pt,domain=-5:-1.1] plot(\x,{(2*(\x)+1)/((\x)+1)});
			\draw[blue,line width=1pt,domain=-0.9:3] plot(\x,{(2*(\x)+1)/((\x)+1)});
			\draw[fill=black] (-1,2) circle(1.5pt) circle(1.5pt) (-1,0) circle(1pt) (0,2) circle(1pt);
			\node[left] at (-1,1.5) {\footnotesize $I$};
			\node[below left] at (-1,0) {\tiny $-\dfrac{d}{c}$};
			\node[above right] at (0,2) {\tiny $\dfrac{a}{c}$};
		\end{tikzpicture}
		\hspace{0.5cm}
		\begin{tikzpicture}[smooth,samples=300,line width=0.6pt,>=stealth, scale=0.45]
			\draw[->] (-3,0)--(5.5,0) node[below]{$x$};
			\draw[->] (0,-1.6)--(0,5.7) node[left]{$y$};
			\draw (0,0) node[below left]{$O$};
			\node at (3,5.3) {\tiny \fbox{$y'<0$}};
			\clip (-3,-1.5) rectangle (5,5.5);
			\draw[dashed] (1,-2)--(1,5.5) (-3,2)--(5,2);
			\draw[blue,line width=1pt,domain=-3:0.9] plot(\x,{(2*(\x)-1)/((\x)-1)});
			\draw[blue,line width=1pt,domain=1.1:5] plot(\x,{(2*(\x)-1)/((\x)-1)});
			\draw[fill=black] (1,2) circle(1.5pt) (1,0) circle(1pt) (0,2) circle(1pt);
			\node[right] at (1,1.5) {\footnotesize $I$};
			\node[below right] at (1,0) {\tiny $-\dfrac{d}{c}$};
			\node[above left] at (0,2) {\tiny $\dfrac{a}{c}$};
		\end{tikzpicture}
	\end{enumerate}
	%	\vspace{1.5cm}
\end{minipage}\hspace{0.5cm}
\begin{minipage}[b]{6.5cm}
	\begin{khung4}{GHI NHỚ}
		\ding{172} Tiệm cận đứng $x=-\dfrac{d}{c}$.\\
		\ding{173} Tiệm cận ngang $y=\dfrac{a}{c}$.\\
		\ding{174} Giao với $Ox$: $y=0 \Rightarrow x=-\dfrac{b}{a}$.\\
		\ding{175} Giao với $Oy$: $x=0 \Rightarrow y=\dfrac{b}{d}$.\\
	\end{khung4}
\end{minipage}
\subsubsection{Hàm số $\mathbf{y = \dfrac{{a{x^2} + bx + c}}{{mx + n}}\left( {a \ne 0,m \ne 0} \right)}$ (đa thức tử không chia hết cho đa thức mẫu)}
\begin{enumerate}[\iconCH]
	\item Tập xác định $D=\mathbb{R}\backslash \left\{-\dfrac{n}{m}\right\}$; Đạo hàm $y'=\dfrac{am\cdot x^2+2an\cdot x + b.n - m.c}{(mx+n)^2}$.
	\item Hàm số $2$ điểm cực trị khi $y'=0$ có $2$ nghiệm phân biệt; Hàm số không có cực trị khi $y'=0$ vô nghiệm.
	\item Đồ thị nhận giao điểm của tiệm cận đứng và tiệm cận xiên làm tâm đối xứng.
	\item Hình dạng đồ thị:\\
	\begin{tikzpicture}[line cap=butt,line join=miter,>=stealth,scale=0.4,font=\footnotesize]
		\tikzset{declare function={xmin=-5.5;xmax=3.5;ymin=-4.6;ymax=4.6;},
			smooth,samples=450}
		\draw[->] (xmin,-0.5)--(xmax,-0.5) node[above]{$ x $};
		\draw[->] (0,ymin)--(0,ymax) node[right]{$ y $};
		\fill (0,-0.5) node[above right]{$ O $};
		\path (current bounding box.south) node[below, black]{\tiny\fbox{$a>0$, $y'=0$ có $2$ nghiệm phân biệt}};
		\clip (xmin,ymin) rectangle (xmax,ymax);
		\def\f(#1){((#1)^2+2*(#1)+2)/((#1)+1)} % Hàm số
		\def\q(#1){((#1)+1)} % Tiệm cận xiên	
		\draw[blue,thick,samples=250] plot[domain=xmin:-1.1] (\x,{\f(\x)});	
		\draw[blue,thick,samples=250] plot[domain=-0.9:xmax] (\x,{\f(\x)});
		%--------- Tiệm cận
		\draw[dashed] plot [domain=xmin:xmax] (\x,{\q(\x)}) ;
		\draw[dashed] (-1,ymin)--(-1,ymax);
	\end{tikzpicture}	
	\hspace{.25cm}
	\begin{tikzpicture}[line cap=butt,line join=miter,>=stealth,scale=0.4,font=\footnotesize]
		\tikzset{declare function={xmin=-5.5;xmax=3.5;ymin=-4.6;ymax=4.6;},
			smooth,samples=450}
		\draw[->] (xmin,-0.5)--(xmax,-0.5) node[above]{$ x $};
		\draw[->] (0,ymin)--(0,ymax) node[right]{$ y $};
		\fill (0,-0.5) node[above right]{$ O $};
		\path (current bounding box.south) node[below, black]{\tiny\fbox{$a<0$, $y'=0$ có $2$ nghiệm phân biệt}};
		\clip (xmin,ymin) rectangle (xmax,ymax);
		\def\f(#1){(-(#1)^2-2*(#1)-2)/((#1)+1)} % Hàm số
		\def\q(#1){(-(#1)-1)} % Tiệm cận xiên	
		\draw[blue,thick,samples=250] plot[domain=xmin:-1.1] (\x,{\f(\x)});	
		\draw[blue,thick,samples=250] plot[domain=-0.9:xmax] (\x,{\f(\x)});
		%--------- Tiệm cận
		\draw[dashed] plot [domain=xmin:xmax] (\x,{\q(\x)}) ;
		\draw[dashed] (-1,ymin)--(-1,ymax);
	\end{tikzpicture}	
	\hspace{.25cm}
	\begin{tikzpicture}[line cap=butt,line join=miter,>=stealth,scale=0.4,font=\footnotesize]
		\tikzset{declare function={xmin=-5.5;xmax=3.5;ymin=-4.6;ymax=4.6;},
			smooth,samples=450}
		\draw[->] (xmin,-0.5)--(xmax,-0.5) node[above]{$ x $};
		\draw[->] (0,ymin)--(0,ymax) node[right]{$ y $};
		\fill (0,-0.5) node[above right]{$ O $};
		\path (current bounding box.south) node[below, black]{\tiny\fbox{$a>0$, $y'=0$ vô nghiệm}};
		\clip (xmin,ymin) rectangle (xmax,ymax);
		\def\f(#1){((#1)^2+2*(#1))/((#1)+1)} % Hàm số
		\def\q(#1){((#1)+1)} % Tiệm cận xiên	
		\draw[blue,thick,samples=250] plot[domain=xmin:-1.1] (\x,{\f(\x)});	
		\draw[blue,thick,samples=250] plot[domain=-0.9:xmax] (\x,{\f(\x)});
		%--------- Tiệm cận
		\draw[dashed] plot [domain=xmin:xmax] (\x,{\q(\x)}) ;
		\draw[dashed] (-1,ymin)--(-1,ymax);
	\end{tikzpicture}	
	\hspace{.25cm}
	\begin{tikzpicture}[line cap=butt,line join=miter,>=stealth,scale=0.4,font=\footnotesize]
		\tikzset{declare function={xmin=-5.5;xmax=3.5;ymin=-4.6;ymax=4.6;},
			smooth,samples=450}
		\draw[->] (xmin,-0.5)--(xmax,-0.5) node[above]{$ x $};
		\draw[->] (0,ymin)--(0,ymax) node[right]{$ y $};
		\fill (0,-0.5) node[above right]{$ O $};
		\path (current bounding box.south) node[below, black]{\tiny\fbox{$a<0$, $y'=0$ vô nghiệm}};
		\clip (xmin,ymin) rectangle (xmax,ymax);
		\def\f(#1){(-(#1)^2-2*(#1))/((#1)+1)} % Hàm số
		\def\q(#1){(-(#1)-1)} % Tiệm cận xiên	
		\draw[blue,thick,samples=250] plot[domain=xmin:-1.1] (\x,{\f(\x)});	
		\draw[blue,thick,samples=250] plot[domain=-0.9:xmax] (\x,{\f(\x)});
		%--------- Tiệm cận
		\draw[dashed] plot [domain=xmin:xmax] (\x,{\q(\x)}) ;
		\draw[dashed] (-1,ymin)--(-1,ymax);
	\end{tikzpicture}
\end{enumerate}
\subsection{PHÂN LOẠI VÀ PHƯƠNG PHÁP GIẢI TOÁN}
\begin{dang}{Khảo sát và vẽ đồ thị hàm số bậc ba}
	Ta khảo sát theo sơ đồ đã nhắc đến ở phần lý thuyết.
\end{dang}
\boxmini{BÀI TẬP TỰ LUẬN}
\begin{vd}
	Khảo sát sự biến thiên và vẽ đồ thị các hàm số sau:
	\begin{tasks}(2)
		\task $y=x^3-3x^2+1$;
		\task $y =-2{x^3}-3{x^2}+1$;
		\task $y = {x^3}+3{x^2}+3x+2$;
		\task $y=x^3-3x^2+4x-2$.
	\end{tasks}
\loigiai{
\begin{enumerate}[a)]
	\item Tập xác định $\mathbb{R}$.\\
	Sự biến thiên:
	\begin{itemize}
		\item [$\bullet$] $y'=3x^2-6x$; $y'=0\Leftrightarrow \hoac{&x=0\\&x=2.}$.
		\item [$\bullet$]  Giới hạn: $\lim\limits_{x\to -\infty}y=-\infty$; $\lim\limits_{x\to +\infty}y=+\infty$.
		\item [$\bullet$] \immini{Bảng biến thiên như hình bên:\\
		Suy ra hàm số đồng biến trên các khoảng $(-\infty;0)$ và $(2;+\infty)$; nghịch biến trên $(0;2)$.\\
	Hàm số đạt cực đại tại $x=0; y_{\text{CĐ}}=1$; hàm số đạt cực tiểu tại $x=2; y_{\text{CT}}=-3$.}
	{\hspace{1cm}
		\begin{tikzpicture}
			\tkzTabInit[lgt=1.1,espcl=2,nocadre=True]{$x$/0.6,$y'$/0.6,$y$/2}{$-\infty$,$0$,$2$,$+\infty$}
			\tkzTabLine{,+,z,-,z,+,}
			\tkzTabVar{-/$-\infty$ , +/$1$,-/$-3$, +/$+\infty$}%
		\end{tikzpicture}}
			\end{itemize}
		Đồ thị:
			\immini{
				\begin{itemize}
					\item [$\bullet$] Đồ thị đi qua các điểm $(2;-3)$, $(-1;-3)$, $(3;1)$
					\item [$\bullet$] Đồ thị nhận điểm $I(1;-1)$ làm tâm đối xứng.
				\end{itemize}
				}
			{
				\begin{tikzpicture}[smooth,samples=300,scale=0.8,>=stealth]
					\draw[->] (-2,0)--(4,0) node[below]{$x$};
					\draw[->] (0,-3.9)--(0,2) node[right]{$y$};
					\draw (0,0) node[below left]{$O$};
					\draw[domain=-1.1:3.1] plot(\x,{(\x)^3-3*(\x)^2+1});
					\draw[fill=black] (-1,-3) circle(1.5pt) (0,1) circle(1pt) (2,-3) circle(1pt) (3,1) circle(1pt);
					\draw[dashed] (-1,0)--(-1,-3)--(0,-3)node[below left]{$-3$}--(2,-3)--(2,0)node[above]{$2$}
					(1,0)node[above]{$1$}--(1,-1)--(0,-1)node[left]{$-1$}
					(3,0)node[below]{$3$}--(3,1)--(0,1)node[left]{$1$}
					;
				\end{tikzpicture}
			}
	\item Tập xác định: $\mathbb{R}$.\\
	Sự biến thiên:
	\begin{itemize}
		\item [$\bullet$] $y' = - 6{x^2} - 6x;\,\,y' = 0 \Leftrightarrow x = 0$ hoặc $x = - 1$.
		\item [$\bullet$]  Giới hạn: $\lim\limits_{x\to -\infty}y=+\infty$; $\lim\limits_{x\to +\infty}y=-\infty$.
		\item [$\bullet$] \immini{Bảng biến thiên như hình bên:\\
			Suy ra hàm số nghịch biến trên các khoảng $(-\infty;-1)$ và $(0;+\infty)$; đồng biến trên $(-1;0)$.\\
			Hàm số đạt cực đại tại $x=0; y_{\text{CĐ}}=1$; hàm số đạt cực tiểu tại $x=-1; y_{\text{CT}}=0$.}
		{\hspace{1cm}
			\begin{tikzpicture}[scale=1, font=\footnotesize, line join=round, line cap=round, >=stealth]
				\tkzTabInit[nocadre=false,lgt=1.2,espcl=1.6,deltacl=0.6]
				{$x$ /0.6,$f'(x)$ /0.6,$f(x)$ /1.5}
				{$-\infty$,$-1$,$0$,$+\infty$}
				\tkzTabLine{,-,0,+,0,-,}
				\tkzTabVar{+/$+\infty$,-/$0$,+/$1$,-/$-\infty$}
		\end{tikzpicture}}
	\end{itemize}
Đồ thị:
	\immini{
		\begin{itemize}
			\item [$\bullet$] Đồ thị qua các điểm $(1;-4)$, $(-2;5)$.
			\item [$\bullet$] Đồ thị của hàm số có tâm đối xứng là điểm $I\left({- \dfrac{1}{2};\dfrac{1}{2}} \right)$
		\end{itemize}
		}{	
		\begin{tikzpicture}[scale=1.0,>=stealth, font=\footnotesize, line join=round, line cap=round]
			\def\xmin{-2} \def\xmax{2}
			\def\ymin{-2} \def\ymax{2}
			\draw[->] (\xmin,0)--(\xmax,0) node [below]{$x$};
			\draw[->] (0,\ymin)--(0,\ymax) node [left]{$y$};
			\fill (0,0) circle (1pt) node[shift={(-135:2.5mm)}]{$O$};
			\clip (\xmin+0.1,\ymin+0.1) rectangle (\xmax-0.1,\ymax-0.1);
			\draw[smooth,red,samples=300,domain=(\xmin:3.01)] plot(\x,{-2*(\x)^3-3*(\x)^2+1});	
			\foreach \x in {\xmin,...,\xmax}
			\draw (\x,-0.05)--(\x,0.05);
			\foreach \y in {\ymin,...,\ymax}
			\draw (-0.05,\y)--(0.05,\y);
			\node at (-1,0)[below]{$ -1 $};
			\node at (1,0)[below]{$ 1 $};	
			\node at (0,1)[shift={(135:1.5mm)}]{$ 1 $};	
	\end{tikzpicture}}
	\item Tập xác định $\mathbb{R}$.\\
	Sự biến thiên:
	\begin{itemize}
		\item [$\bullet$] $y' = 3{x^2} + 6x + 3;\,\,y' = 0 \Leftrightarrow x = - 1$.
		\item [$\bullet$] Giới hạn: $\lim\limits_{x\to -\infty}y=-\infty$; $\lim\limits_{x\to +\infty}y=+\infty$.
		\item [$\bullet$] \immini{Bảng biến thiên như hình bên:\\
			Suy ra hàm số đồng biến trên $\mathbb{R}$.\\
			Hàm số không có cực trị.}
		{\hspace{1cm}
			\begin{tikzpicture}[>=stealth]
				\tkzTabInit[nocadre=false,lgt=1,espcl=2.5,deltacl=0.5]{$x$/.6 ,$y'$/.6,$y$/1.5}
				{$-\infty$ , $-1$ , $+\infty$}
				\tkzTabLine{ , + , $0$ , + , }
				\tkzTabVar{-/$-\infty$ , R , +/$+\infty$}
				\tkzTabIma{1}{3}{2}{$1$}
		\end{tikzpicture}}
	\end{itemize}
	Đồ thị:
		\immini{	
	Đồ thị của hàm số có tâm đối xứng là điểm $I\left( {- 1;1} \right)$
	}{	
		\begin{tikzpicture}[scale=0.7,>=stealth, font=\footnotesize, line join=round, line cap=round]
			\def\xmin{-3} \def\xmax{2}
			\def\ymin{-2} \def\ymax{4}
			\draw[->] (\xmin,0)--(\xmax,0) node [below]{$x$};
			\draw[->] (0,\ymin)--(0,\ymax) node [left]{$y$};
			\fill (0,0) circle (1pt) node[shift={(-135:2.5mm)}]{$O$};
			\clip (\xmin+0.1,\ymin+0.1) rectangle (\xmax-0.1,\ymax-0.1);
			\draw[smooth,red,samples=300,domain=(\xmin:3.01)] plot(\x,{(\x)^3+3*(\x)^2+3*(\x)+2});	
			\foreach \x in {\xmin,...,\xmax}
			\draw (\x,-0.05)--(\x,0.05);
			\foreach \y in {\ymin,...,\ymax}
			\draw (-0.05,\y)--(0.05,\y);	
			\node at (1,0)[below]{$ 1 $};
			\node at (-1,1)[shift={(70:2mm)}]{$ I $};
			\draw[dashed]
			(-1,0)node[below]{$ -1 $}|-(0,1)node[right]{$1$}	
			;	
		\end{tikzpicture}}				
	\item Tập xác định: $\mathbb{R}$.\\
	Sự biến thiên
	\begin{itemize}
		\item [$\bullet$] $y'=3x^2-6x+4>0$ với $\forall x\in\mathbb{R}$.
		\item [$\bullet$] Giới hạn: $\lim\limits_{x\to -\infty}y=-\infty$; $\lim\limits_{x\to +\infty}y=+\infty$
		\item [$\bullet$] \immini{Bảng biến thiên như hình bên:\\
			Suy ra hàm số đồng biến trên $\mathbb{R}$.\\
			Hàm số không có cực trị.}
		{\hspace{1cm}
			\begin{tikzpicture}
				\tkzTabInit[lgt=1.1,espcl=4]{$x$/0.6,$y'$/0.6,$y$/2}{$-\infty$,$+\infty$}
				\tkzTabLine{,+,}
				\tkzTabVar{-/$-\infty$ ,+/$+\infty$}
		\end{tikzpicture}}
	\end{itemize}
Đồ thị\\
\immini{
	\begin{itemize}
		\item [$\bullet$] Đồ thị đi qua $(2;2)$, $(0;-2)$, $(1;0)$.
		\item [$\bullet$] Đồ thị nhận $I(1;0)$ làm tâm đối xứng.
	\end{itemize}
}
{
	\begin{tikzpicture}[line cap=round,line join=round,x=1cm,y=1cm]
		\draw[->](-3.08,0)--(4.06,0);
		\foreach \x in {-1,2,3}
		\draw[shift={(\x,0)},color=black] (0pt,2pt)--(0pt,-2pt) node[below]{$\x$};
		\draw[->,color=black](0,-4.06)--(0,2.98);
		\foreach \y in {-2,-1,1,2}
		\draw[shift={(0,\y)},color=black](2pt,0pt)--(-2pt,0pt) node[left]{\normalsize $\y$};
		\draw[color=black](3.8,.2)node[right]{$x$};
		\draw[color=black](.2,3)node[right]{$y$};
		\draw[color=black](0pt,-8pt)node[right]{\normalsize $O$};
		\clip(-3.08,-4.06) rectangle (4.06,2.98);
		%Vẽ đồ thị
		\draw[smooth,samples=100,domain=-4:4]plot(\x,{(\x)^3-3*(\x)^2+4*(\x)-2});
		%Vẽ râu ria
		\draw[dashed](2,0)--(2,2)--(0,2);
		\node[below right] at (1,0){$I$};
		\node[above] at (1,0) {$1$};
	\end{tikzpicture}
}
\end{enumerate}
}
\end{vd}
\dongcham{45}
\boxmini{BÀI TẬP TRẮC NGHIỆM}
\ind{PHẦN I.} \inden{Câu trắc nghiệm nhiều phương án lựa chọn. Mỗi câu hỏi học sinh chỉ chọn một phương án.}\\
\setcounter{ex}{0}
\Opensolutionfile{ans}[ans/2D1-B4-d1-1]
\begin{ex}
	\immini[thm]{Bảng biến thiên ở hình bên là của một trong bốn hàm số sau đây. Hỏi đó là hàm số nào?
		\choice
		{$y=-x^3-2x^2+5$}
		{\True $y=x^3-3x^2+5$}
		{$y=-x^3-3x+5$}
		{$y=x^3+3x^2+5$}}{
		\begin{tikzpicture}
			\tkzTabInit[nocadre=false, lgt=1.2, espcl=1.6]{$x$ /0.6,$f'(x)$ /0.6,$f(x)$ /1.5}{$-\infty$,$0$,$2$,$+\infty$}
			\tkzTabLine{,+,$0$,-,$0$,+,}
			\tkzTabVar{-/ $-\infty$/, +/$5$ , -/$1$  , +/$+\infty$/}
	\end{tikzpicture}}
\end{ex} \dongcham{1}

\begin{ex}
	\immini[thm]{Bảng biến thiên ở hình bên là của một trong bốn hàm số sau đây. Hỏi đó là hàm số nào?
	\choice
	{$ y=-x^3+3x^2 $}
	{$ y=x^3-3x^2-1$}
	{$ y=x^4+2x^2+1 $}
	{\True$ y=-x^3+3x^2+1 $}}{
\begin{tikzpicture}
	\tkzTabInit[nocadre=false,lgt=1.2,espcl=1.6,deltacl=0.6]
	{$x$/0.6, $y'$/0.6, $y$/1.5}
	{$-\infty$,$0$,$2$,$+\infty$}
	\tkzTabLine{,-,z,+,z,-,}
	\tkzTabVar{+/$+\infty$ ,-/ $1$ ,+/$5$, -/$-\infty$}
\end{tikzpicture}}
	\loigiai{
		Ta thấy đây là hàm số bậc ba và $\displaystyle\lim\limits_{x\rightarrow-\infty}=-\infty$ nên $a<0$.\\
		Ta có $f(0)=1$ nên hàm số cần tìm là $y=-x^3+3x^2+1$.
	}
\end{ex} \dongcham{1}

\begin{ex}
	\immini[thm]{Bảng biến thiên ở hình bên là của một trong bốn hàm số sau đây. Hỏi đó là hàm số nào?
		\haicot
		{$y=x^3-3x^2+x+3$}
		{$y=x^3-3x+4$}
		{\True $y=x^3-3x^2+3x+1$}
		{$y=x^3+3x^2+5$}}{
		\begin{tikzpicture}
			\tkzTabInit[lgt=1,espcl=2.5]
			{$x$/0.6,$y'$/0.6,$y$/1.5}
			{$-\infty$,$1$,$+\infty$}
			\tkzTabLine{,+,$0$,+,}
			\tkzTabVar{-/$-\infty$,R,+/$+\infty$}
			\tkzTabIma[draw]{1}{3}{2}{$2$}
	\end{tikzpicture}}
\end{ex} \dongcham{1}

\begin{ex}%[2D1B5-1]
	\immini[thm]{Đường cong bên là đồ thị của một trong bốn hàm số đã cho sau đây. Hỏi đó là hàm số nào?
		\choice
		{$y=-x^3+x^2-2$}
		{\True $y=x^3+3x^2-2$}
		{$y=x^3-3x+2$}
		{$y=x^2-3x-2$}
	}{
		\begin{tikzpicture}[scale=0.55, font=\footnotesize,line join=round, line cap=round,>=stealth]
			\draw[->] (-3.7,0.) -- (2.5,0.) node[below]{$x$};
			\draw[->] (0,-2.5) -- (0,2.5) node[right]{$y$};
			\fill (0,0) node[above left]{$O$};
			\fill (0,-2) circle(2pt) node[below left]{$-2$};
			\draw[line width=1pt,smooth,samples=300,domain=-2.9:1] plot(\x,{(\x+2)^3-3*(\x+2)^2+2});
		\end{tikzpicture}
	}
	\loigiai{
		Dựa vào hình dáng đồ thị, ta thấy đây là đồ thị của hàm số bậc ba $y=ax^3+bx^2+cx+d$ với $a>0$ nên loại các hàm $y=x^4+x^2-2$, $y=-x^2-3x-2$. Mặt khác, đồ thị đi qua điểm $(0;-2)$ nên loại hàm $y=x^3-3x+2$.\\
		(Ngoài ra, ta có thể đánh giá dấu của các hệ số $a,~b,~c$ thông qua hoành độ $2$ điểm cực trị và hoành độ trung điểm của hai điểm cực trị. Trong đồ thị này ta còn thấy hàm số có điểm cực tiểu $x=0$ nên $c=0$)
	}
\end{ex} \dongcham{1}

\begin{ex}%
	\immini[thm]{Đường cong bên là đồ thị của một trong bốn hàm số đã cho sau đây. Hỏi đó là hàm số nào?
		\choice
		{$y=x^3+3x-2  $}
		{$ y=x^3-3x+2$}
		{\True $y=-x^3+3x+2$}
		{$y=-x^3-3x-2$}
	}{
		\begin{tikzpicture}[scale=0.6, font=\footnotesize, line join=round, line cap=round, >=stealth]
			\clip(-2.5,-1.2) rectangle (5,5);
			\draw[->] (-2.5,0) -- (3,0) node[below]{ $x$};
			\draw[->] (0,-1.5) -- (0,4.7) node[left]{ $y$};
			\draw[line width=1pt,smooth,samples=100,domain=-2.05:2.05] plot(\x,{-(\x)^3+3*(\x)+2});
			\draw [fill=black] (0,0) circle (1pt)node[below left]{\footnotesize $O$}(-1,1);
			\draw[dashed](-2,0)node[below]{\scriptsize $-2$}--(-2,4)--(0,4)node[below left]{\scriptsize $4$}--(1,4)--(1,0)node[below]{\scriptsize $1$};
			\draw(2,0)node[below right]{\scriptsize $2$};
	\end{tikzpicture}}
	
	\loigiai{
		Quan sát đồ thị, ta thấy nhánh cuối của đồ thị hướng xuống dưới nên $\lim\limits_{x\rightarrow +\infty}y=-\infty$, suy ra hệ số $a<0$. Như vậy hai hàm số 	$y=x^3+3x-2; y=x^3-3x+2$ không thỏa mãn.
		\\Mặt khác hàm số có hai điểm cực trị nên hàm số $y=-x^3-3x-2$ có $y'=-3x^2-3<0$ $\forall x\in \mathbb{R}$ không thỏa mãn.
	}
\end{ex} \dongcham{1}


\begin{ex}
	\immini[thm]{
		Đường cong bên là đồ thị của một trong bốn hàm số đã cho sau đây. Hỏi đó là hàm số nào?
		\choice
		{$y= - x^3 + 3x^2 + 1$}
		{$y= - x^2 - 3x - 1$}
		{$y=x^4 + 2x^2 - 1$}
		{\True $y=x^3 - 3x + 1$}
	}{
		\begin{tikzpicture}[scale=0.6, font=\footnotesize, line join=round, line cap=round, >=stealth]
			\draw[->] (-2.7,0)--(0,0) node[below left]{$O$}--(2.5,0) node[below]{$x$};
			\draw[->] (0,-1.5) --(0,3.8) node[right]{$y$};
			\tkzDefPoints{0/0/O}
			\draw(-1.2,0) node[below]{$-1$};
			\draw(1,0) node[above]{$1$};
			\draw(0,-1) node[left]{$-1$};
			\draw(0,3) node[right]{$3$};
			\draw [domain=-2.02:2.02, samples=100] %
			plot (\x, {(\x)^3-3*(\x)+1}) ;
			\draw [dashed] (0,3)--(-1,3)--(-1,0);
			\draw [dashed] (1,0)--(1,-1)--(0,-1);
			\tkzDrawPoints[fill=black](O)
		\end{tikzpicture}
	}
	\loigiai{
		Đường cong trong hình là đồ thị của hàm số bậc ba có hệ số $a<0$. Trong các hàm số đã cho, chỉ có duy nhất hàm số $y=x^3 - 3x + 1$ thỏa mãn.
	}
\end{ex} \dongcham{1}

\begin{ex}
	\immini[thm]{Đường cong bên là đồ thị của một trong bốn hàm số đã cho sau đây. Hỏi đó là hàm số nào?
		\choice
		{$y=x^3-3x^2-4$}
		{$y=-x^3-4$}
		{$y=-x^3+3x^2-2$}
		{\True $y=-x^3+3x^2-4$}
	}{\begin{tikzpicture}[scale=0.6,>=stealth, font=\footnotesize, line join=round, line cap=round]
			\def\a{-1} \def\b{3} \def\c{0} \def\d{-4} % Hệ số
			\def\xmin{-1.5} \def\xmax{3.8}
			\def\ymin{-4.5} \def\ymax{1.5}
			%\draw[color=gray!50,dashed] (\xmin,\ymin) grid (\xmax,\ymax);
			\foreach \x in {-1,2}
			\draw[thin] (\x,1pt)--(\x,-1pt) node [above] {$\x$};
			\foreach \y in {-4}
			\draw[thin] (1pt,\y)--(-1pt,\y) node [left] {$\y$};
			\draw[->] (\xmin,0)--(\xmax,0) node [below]{$x$};
			\draw[->] (0,\ymin)--(0,\ymax) node [left]{$y$};
			\node at (0,0) [below left]{$O$};
			\clip (\xmin+0.1,\ymin+0.1) rectangle (\xmax-0.5,\ymax-0.1);
			\draw[smooth,samples=300] plot(\x,{\a*(\x)^3+\b*(\x)^2+\c*(\x)+\d});
	\end{tikzpicture}}
	\loigiai{
		\begin{itemize}
			\item Đồ thị hàm số có dạng chữ N ngược nên đây là đồ thị hàm số $y=ax^3+bx^2+cx+d$ với $a<0$. Loại phương án $y=x^3-3x^2-4$.
			\item Đồ thị hàm số giao $Oy$ tại điểm có tung độ bằng $-4$ nên $d=-4$, loại phương án $y=-x^3+3x^2-2$.
			\item Hàm số có hai điểm cực trị $x=0, x=2$ nên loại phương án $y=-x^3-4$ (vì phương án này có $y'=-3x^2$, hàm số không có điểm cực trị).
	\end{itemize}}
\end{ex} \dongcham{1}

\begin{ex}
	\immini[thm]{Đường cong bên là đồ thị của một trong bốn hàm số đã cho sau đây. Hỏi đó là hàm số nào?
		\haicot
		{$y=x^3-1$}
		{$y=(x+1)^3$}
		{\True $y=(x-1)^3$}
		{$y=x^3+1$}}
	{
		\begin{tikzpicture}[scale=0.8,>=stealth]
			\draw[->] (-1,0)--(0,0)node[above left]{$O$}--(2.2,0)node[below]{$x$};
			\draw[->] (0,-2)--(0,1.7)node[left]{$y$};
			\draw[line width=1pt,smooth,samples=100,domain=-0.25:2.2] plot(\x,{(\x-1)^3});
			\draw [fill=black] (1,0) circle (1.5pt);
			\draw (1,0)node[above]{$1$} (0,-1)node[left]{$-1$};
		\end{tikzpicture}
	}
	\loigiai{
		$(C)$ tiếp xúc với $Ox$ tại điểm uốn, suy ra $f(x)$ có nghiệm bội ba $x=1$ nên hàm số có dạng $y=a(x-1)^3$. Mà $(0;-1)\in (C)$ nên $a=1$.
	}
\end{ex} \dongcham{1}

\begin{ex}
	\immini[thm]{Cho hàm số $y = ax^3 + bx^2 + cx + d$ có đồ thị như hình vẽ bên. Khẳng định nào sau đây là đúng?
		
		\choice
		{$a > 0$, $b > 0$, $c > 0$, $d > 0$}
		{$a < 0$, $b < 0$, $c > 0$, $d > 0$}
		{$a > 0$, $b < 0$, $c < 0$, $d > 0$}
		{\True $a > 0$, $b < 0$, $c > 0$, $d > 0$}}
	{\begin{tikzpicture}[scale=0.9, font= \footnotesize, line join=round, line cap=round, >=stealth]
			\draw[->] (-2,0) -- (4,0) node[above] {$x$};
			\draw[->] (0,-1.3) -- (0,2) node[right] {$y$};
			\draw[fill=black] (1,0) circle (1.5pt);
			\draw[fill=black] (0,0) circle (1.5pt);
			\draw[line width=1pt,smooth,samples=100,domain=-0.4:3.1] plot(\x,{(\x)^3-4*(\x)^2 + 3*(\x) + 1});
			\node[below right] at (0,0) {$O$};
			\node[above] at (1,0) {$1$};
	\end{tikzpicture}}
	
	\loigiai{
		Nhìn vào đồ thị, ta thấy đồ thị hàm số đi từ $-\infty$ lên $+\infty$ nên $a > 0$. \\
		Giao điểm với trục tung nằm trên trục hoành, do đó $d > 0$.\\
		Hàm số có hai điểm cực trị, và hai điểm cực trị đều dương. Suy ra tổng hai điểm cực trị và tích hai điểm cực trị đều dương.\\ 	Ta có $f'(x) = 3ax^2 + 2bx + c$ nên tổng hai điểm cực trị là $\dfrac{-2b}{3a}$. Suy ra $\dfrac{-2b}{3a} > 0$, hay $b < 0$.\\ Còn tích hai điểm cực trị là $\dfrac{c}{3a}$. Suy ra $\dfrac{c}{3a} > 0$ hay $c > 0$.}
\end{ex} \dongcham{1}

\begin{ex}%[2D1B5-1]
	\immini[thm]{Cho hàm số $ y=ax^3+bx^2+cx+d $ có đồ thị như hình vẽ bên. Mệnh đề nào sau đây đúng?
		\choice
		{$ a<0 $, $ b<0 $, $ c<0 $, $ d>0 $}
		{$ a<0 $, $ b>0 $, $ c<0 $, $ d>0 $}
		{\True $ a<0 $, $ b>0 $, $ c>0 $, $ d<0 $}
		{$ a<0 $, $ b<0 $, $ c>0 $, $ d<0 $}}{
		\begin{tikzpicture}[smooth,samples=300,scale=0.7,>=stealth]
			\draw[->] (-2,0)--(3.7,0) node[below]{$x$};
			\draw[->] (0,-1.7)--(0,2.8) node[right]{$y$};
			\draw (0,0) node[above left]{$O$};
			\draw[line width=1pt,domain=-1.5:2.5] plot(\x,{-(\x-0.5)^3+3*(\x-0.5)+0.5});
		\end{tikzpicture}
	}
	\loigiai{
		Dựa vào hình dáng đồ thị suy ra $ a<0 $.\\
		Dựa vào vị trí điểm cực đại và điểm cực tiểu, suy ra $ x_{\text{CT}}+x_{\text{CĐ}}>0 \Rightarrow -\dfrac{b}{a}>0\Rightarrow b>0$.\\
		Hai điểm cực trị có hoành độ trái dấu nên $ x_{\text{CT}}\cdot x_{\text{CĐ}}<0\Rightarrow \dfrac{c}{a}<0\Rightarrow c>0 $.\\
		Đồ thị hàm số cắt trục tung tại điểm có tung độ dương nên $ d>0 $.\\
		Vậy $ a<0 $, $ b>0 $, $ c>0 $ và $ d>0 $.
	}
\end{ex} \dongcham{1}

\begin{ex}%[2D1K5-1]
	\immini[thm]{Cho hàm số $y=ax^3+bx^2+cx+d$ có đồ thị như hình vẽ bên. Mệnh đề nào dưới đây đúng?
		\choice
		{$a<0$, $b>0$, $c>0$, $d>0$}
		{$a<0$, $b<0$, $c=0$, $d>0$}
		{\True $a<0$, $b>0$, $c=0$, $d>0$}
		{$a>0$, $b<0$, $c>0$, $d>0$}}{
		\begin{tikzpicture}[smooth,samples=300,scale=0.7,>=stealth]
			\draw[->] (-2,0)--(3.7,0) node[below]{$x$};
			\draw[->] (0,-1)--(0,4.5) node[right]{$y$};
			\draw (0,0) node[below left]{$O$};
			\draw[line width=1pt,domain=-1:3.1] plot(\x,{-(\x)^3+3*(\x)^2+0.3});
			%\draw[fill=black] (2,-1) circle(1.5pt) (2,0) circle(1pt) (0,-1) circle(1pt);
			%\draw[dashed] (2,-1.5)--(2,2.5) (2,-1)--(0,-1)node[left]{\small$-\dfrac{\Delta}{4a}$};
			%\node[right] at (2,2.4) {\small $x=-\tfrac{b}{2a}$};
			%\node[right] at (0.5,-2) {\fbox{$a>0$}};
		\end{tikzpicture}
	}
	\loigiai{
		Dựa vào đồ thị ta có thể thấy $a<0$, đồ thị cắt trục tung tại điểm có tung độ dương nên $d>0$.\\
		Hàm số có hai cực trị thỏa $\heva{&S>0\\&P=0}\Leftrightarrow\heva{&-\dfrac{b}{a}>0\\&\dfrac{c}{a}=0}\Leftrightarrow\heva{&b>0\\&c=0.}$
	}
\end{ex} \dongcham{4}

\begin{ex}
	\immini[thm]{Cho hàm số $y=ax^3+bx^2+cx+d$ có bảng biến
	thiên như hình bên. Trong các hệ số $a$, $b$, $c$ và $d$ có bao nhiêu số âm?
	\choice
	{$2$}
	{\True $1$}
	{$4$}
	{$3$}}{
\begin{tikzpicture}[>=stealth,scale=1]
	\tkzTabInit[lgt=1.2,espcl=2]
	{$x$ /0.6, $f’(x)$ /0.6, $f(x)$ /2}
	{$-\infty$,$-1$,$2$,$+\infty$}
	\tkzTabLine{ ,-,z,+,z,-, }
	\tkzTabVar{+/,-/$0$,+/,-/}
\end{tikzpicture}}
	\loigiai
	{
		Từ bảng biến thiên ta thấy hàm số có $2$ điểm cực trị nên bậc của đa thức phải lớn hơn $2\Rightarrow a\ne 0$. Mà $\lim \limits_{x \to +\infty} y=-\infty\Rightarrow a<0$.\\
		Từ bảng biến thiên ta có $d=y(0)>y(-1)=0$.\\
		Ta có $y'=3ax^2+2bx+c$ có hai nghiệm là $-1$ và $2$ nên $\heva{& -\dfrac{2b}{3a}=-1+2=1>0 \\ & \dfrac{c}{3a}=(-1)\cdot 2=-2<0}\Rightarrow \heva{& b>0 \\ & c>0.}$
	}
\end{ex} \dongcham{4}

\Closesolutionfile{ans}

\ind{PHẦN II.} \inden{Câu trắc nghiệm đúng sai. Trong mỗi ý a), b), c), d) ở mỗi câu, học sinh chọn đúng hoặc sai.}\\
\Opensolutionfile{ans}[ans/2D1-B4-d1-2]

\begin{ex}
	\immini[thm]{Cho hàm số $y=f(x)=ax^3+bx^2+cx+d$ có đồ thị như hình vẽ.
		\choiceTF
		{Hàm số đạt cực tiểu tại $x=1$}
		{\True Đồ thị hàm số cắt trục $Oy$ tại điểm $(0;1)$}
		{Hàm số đồng biến trên khoảng $(-\infty;-1)$}
		{$2a+3b+c=9$}
	}{
		\begin{tikzpicture}
			[scale=1,line join=round, line cap=round, >=stealth]
			\draw[->] (-3,0)--(0,0) node[below left]{$O$}--(2,0) node[below]{$x$};
			\draw[->] (0,-1) --(0,3) node[right]{$y$};
			\draw [domain=-2.3:.7, samples=100] %
			plot (\x, {(\x)^3+2*(\x)^2+1});
			\draw [dashed] (-2,0)node[below]{$-2$}--(-2,1) --(0,1)node[below right]{$1$}
			(-1,0)node[below]{$-1$}--(-1,2)--(0,2)node[right]{$2$};
			\draw[fill] (0,1) circle (1pt) (-2,1) circle (1pt) (-1,2) circle (1pt);
		\end{tikzpicture}}
	\loigiai{
		Theo hình vẽ thì:
		\begin{enumerate}[a)]
			\item Hàm số đạt cực tiểu tại $x=0$, giá trị cực tiểu $y=1$;
			\item Đồ thị hàm số cắt trục $Oy$ tại điểm $(0;1)$;
			\item Hàm số đồng biến trên khoảng $(-\infty;x_0)$, với $-2<x_0<-1$;
			\item Đồ thị qua 3 điểm $(-2;1)$, $(-1;2)$, $(0;1)$ và đạt cực trị tại $x=1$ nên ta được hệ
			$$\heva{&-8a+4b-2c+d=1\\&-a+b-c+d=2 \\& d=1\\&c=0} \Leftrightarrow a=1;\,b=2,\,c=0,\,d=1$$
			nên $2a+3b+c=8$.
		\end{enumerate}
	}
\end{ex} \dongcham{10}

\begin{ex}
	\immini[thm]{Cho hàm số bậc ba $ f(x)=ax^3+bx^2+cx+d $ có đồ thị như hình vẽ.\\
		Tính tổng $ T=$.
		\choiceTF
		{\True Đồ thị hàm số cắt trục tung tại điểm $(0;1)$}
		{\True Đường thẳng đi qua điểm $(0;1)$ luôn cắt đồ thị tại ba điểm phân biệt có hoành độ lập thành 1 cấp số cộng}
		{\True $a-b+c+d =-1$}
		{Đồ thị hàm số đi qua điểm $(3;18)$}
	}{\begin{tikzpicture}[>=stealth,line join=round,line cap=round,scale=.8]
			\draw[->] (-2.3,0)--(2.5,0)node[below]{$x$};
			\draw[->] (0,-1.5)--(0,3.5)node[right]{$y$};
			\draw[domain=-2:2, samples=100] plot (\x,{(\x)^3-3*(\x)+1});
			\draw[fill] (-1,3) circle (1pt) (0,1) circle (1pt) (1,-1) circle (1pt);
			\draw[dashed] (-1,0)node[below]{$-1$}|-(0,3)node[right]{$3$} (0,-1)node[left]{$-1$}-|(1,0)node[above]{$1$}
			;
	\end{tikzpicture}}
	\loigiai{
		\begin{enumerate}[a)]
			\item Đồ thị hàm số có hai điểm cực trị $(-1;3)$ và $(1;-1)$. Suy ra tọa độ tâm đối xứng là $(0;1)$. Suy ra đồ thị hàm số cắt trục tung tại điểm $(0;1)$
			\item Do $I(0;1)$ là tâm đối xứng của đồ thị, nên đường thẳng qua nó sẽ cắt đồ thị tại ba điểm phân biệt $I$, $A$, $B$ với $I$ là trung điểm của $AB$. Suy ra $x_A+x_B=2x_I$. Vậy ba điểm này có hoành độ lập thành 1 cấp số cộng.
			\item Ta có $ f'(x)=3ax^2+2bx+c $. Từ hình vẽ, ta có
			$$\heva{&f(-1)=3\\&f(1)=-1\\&f'(-1)=0\\&f'(1)=0} \Leftrightarrow \heva{&-a+b-c+d=3\\&a+b+c+d=-1\\&3a-2b+c=0\\&3a+2b+c=0}$$
			Giải hệ, ta được $a=1$, $b=0$, $c=-3$,$d=1$.
			Vậy $ T=a-b+c+d=-1 $.
			\item Ta có $ f'(x)=3ax^2+2bx+c $. Từ hình vẽ, ta có
			$$\heva{&f(-1)=3\\&f(1)=-1\\&f'(-1)=0\\&f'(1)=0} \Leftrightarrow \heva{&-a+b-c+d=3\\&a+b+c+d=-1\\&3a-2b+c=0\\&3a+2b+c=0}$$
			Giải hệ, ta được $a=1$, $b=0$, $c=-3$,$d=1$. Suy ra $y=x^2-3x+1$.\\
			Thay tọa độ $(3;18)$ vào phương trình, không thỏa mãn. Vậy đồ thị hàm số không đi qua điểm $(3;18)$.
		\end{enumerate}
		}
\end{ex} \dongcham{14}

\begin{ex}
	\immini[thm]{Cho hàm số $ y=f(x)=ax^3+bx^2+cx+d$ có bảng biến thiên như hình bên.
		\choiceTF
		{Hàm số đạt giá trị lớn nhất là $ 4 $}
		{\True Đường thẳng $ y=2$ cắt đồ thị hàm số $ y=f(x)$ tại $ 3 $ điểm phân biệt}
		{\True Trong bốn hệ số $a$, $b$, $c$, $d$ có đúng hai số âm}
		{\True Đồ thị hàm số đi qua điểm $(-4;20)$}
	}{
		\begin{tikzpicture}
			\tkzTabInit[nocadre=false,lgt=1.2,espcl=1.6,deltacl=0.6]
			{$x$ /0.6, $y'$ /0.6, $y$ /2.3}
			{$-\infty$,$-2$,$0$,$+\infty$}
			\tkzTabLine{,-,0,+,0,-,}
			\tkzTabVar{+/$+\infty$,-/$0$,+/$4$,-/$-\infty$}
	\end{tikzpicture}}
	\loigiai{
		Dựa vào bảng biến thiên ta thấy:
		\begin{enumerate}[a)]
			\item Hàm số $ y=f(x)$ không có giá trị lớn nhất trên $\mathbb{R}$.
			\item Vẽ đường thẳng $y=2$ qua điểm $(0;2)$ và song song với $Ox$, rõ ràng đường thẳng này cắt đồ thị tại ba điểm phân biệt.
			\item Từ các thông số trên hình, ta có thể giải ra chính xác giá trị $a$, $b$, $c$, $d$ bởi hệ
			$$\heva{&f(-2)=0\\&f(0)=4\\&f'(-2)=0\\&f'(0)=0} \Leftrightarrow a=-1,\,b=-3,\,c=0,\,d=4.$$
			Vậy trong 4 hệ số, có đúng 2 số âm.
			\item Từ các thông số trên hình, ta có thể giải ra chính xác giá trị $a$, $b$, $c$, $d$ bởi hệ
			$$\heva{&f(-2)=0\\&f(0)=4\\&f'(-2)=0\\&f'(0)=0} \Leftrightarrow a=-1,\,b=-3,\,c=0,\,d=4.$$
			Suy ra $y=-x^3-3x^2+4$. Thay tọa độ $(-4;20)$ vào phương trình, thỏa mãn. Suy ra Đồ thị hàm số đi qua điểm $(-4;20)$.
		\end{enumerate}
		
	}
\end{ex} \dongcham{14}

\Closesolutionfile{ans}





% \begin{dang}{Khảo sát và vẽ đồ thị hàm số phân thức hữu tỉ bậc I/I}
	Ta khảo sát theo sơ đồ
	\begin{itemize}
		\item[\iconCH] \indamm{Bước 1.} Tìm tập xác định $D=\mathbb{R}\backslash \left\{-\dfrac{d}{c}\right\}$.
		\item [\iconCH] \indamm{Bước 2.} Khảo sát sự biến thiên của hàm số
		      \begin{itemize}
			      \item Tính đạo hàm $y'=\dfrac{ad-cb}{(cx+d)^2}$.
			      \item Tìm các giới hạn tại vô cực, giới hạn vô cực và tìm tiệm cận của đồ thị hàm số.
			      \item Lập bảng biến thiên; xác định chiều biến thiên và cực trị của hàm số.
		      \end{itemize}
		\item [\iconCH] \indamm{Bước 3.} Cho thêm điểm và vẽ đồ thị của hàm số dựa vào bảng biến thiên.
	\end{itemize}
\end{dang}
\boxmini{BÀI TẬP TỰ LUẬN}
\begin{vd}
	Khảo sát sự biến thiên và vẽ đồ thị các hàm số sau:
	\begin{tasks}(3)
		\task $y=\dfrac{x-1}{x+1}$;
		\task $y=\dfrac{2 x+1}{x-1}$;
		\task $y = \dfrac{5 + x}{2 - x}$.
	\end{tasks}
	\loigiai{
		\begin{enumerate}[a)]
			\item Tập xác định: $\mathbb{R} \backslash\{-1\}$.\\
			      Sự biến thiên:
			      \begin{itemize}
				      \item [$\bullet$] Đạo hàm $y^{\prime}=\dfrac{2}{(x+1)^{2}}>0$ với mọi $x \neq -1$.
				      \item [$\bullet$] Giới hạn và tiệm cận:\\
				            $\displaystyle\lim _{x \rightarrow -1^{-}} y= +\infty, \displaystyle\lim _{x \rightarrow -1^{+}} y= -\infty$. Do đó, đường thẳng $x=-1$ là tiệm cận đứng của đồ thị hàm số.\\
				            $\displaystyle\lim _{x \rightarrow-\infty} y=1, \displaystyle\lim _{x \rightarrow +\infty} y=1$. Do đó, đường thẳng $y=1$ là tiệm cận ngang của đồ thị hàm số.
				      \item Bảng biến thiên:
				            \begin{center}
					            \begin{tikzpicture}[font=\normalsize,t style/.style={style=solid},scale=.8]
						            %dòng khai báo
						            \tkzTabInit[lgt=1.2,espcl=4,deltacl=0.9]
						            {$x$ /0.75, $y^{\prime}$/0.75, $y$/2.5}
						            {$ -\infty $,$ -1 $,$ +\infty $}
						            %dòng xét dấu
						            \tkzTabLine{ , + ,d , - , } % z, t, d;
						            %dòng biến thiên
						            \path ($(N12)!0.5!(N13)$) node (A1){$ 1 $}
						            ($(N22)!0.1!(N23)+(-17pt,-0)$) node (A2){$ +\infty $}
						            ($(N22)!0.9!(N23)+(12pt,0)$) node (A3){$ -\infty $}
						            ($(N32)!0.5!(N33)$) node (A4){$ 1 $};
						            \draw[double] (N22)--(N23);
						            \foreach \x/\y in {A1/A2,A3/A4}{
								            \draw[-stealth] (\x)--(\y);
							            }
					            \end{tikzpicture}
				            \end{center}
				            Hàm số đồng biến trên mỗi khoảng $(-\infty ; -1)$ và $(-1 ;+\infty)$.\\
				            Hàm số không có cực trị.
			      \end{itemize}
			      Đồ thị:\\
			      \immini{	\begin{itemize}
					      \item Giao điểm của đồ thị với trục tung: $(0 ;-1)$.
					      \item Giao điểm của đồ thị với trục hoành: $\left(1 ; 0\right)$.
					      \item Đồ thị hàm số đi qua các điểm $(0 ;-1)$, $\left(1 ; 0\right)$,  $(-3 ;2)$, $(-2 ;3)$.
				      \end{itemize}
			      }
			      {		\begin{tikzpicture}[line cap=butt,line join=miter,>=stealth,scale=.7,font=\footnotesize]
					      \tikzset{declare function={xmin=-6.1;xmax=4.1;ymin=-4.1;ymax=6.1;},
						      smooth,samples=450}
					      \draw[->] (xmin,0)--(xmax,0) node[shift={(0:7pt)}]{$ x $};
					      \draw[->] (0,ymin)--(0,ymax) node[shift={(90:7pt)}]{$ y $};
					      \fill (0,0) node[shift={(130:8pt)}]{$ O $};
					      \clip (-6,-4.6) rectangle (4,6);
					      \foreach \i in {-3,-2,-1,1}{
							      \draw(\i,1.5pt)--(\i,-1.5pt)node[below]{$\i$};}
					      \foreach \j in {-1,2,3}{
							      \draw(-1.5pt,\j)--(1.5pt,\j) node[right]{$\j$};}
					      \draw(-1.5pt,1)--(1.5pt,1)node[shift={(6pt,3pt)}]{$1$};
					      \def\f(#1){((#1)-1)/((#1)+1)}
					      \def\a{-2}
					      \def\b{-3}
					      \def\c{1}
					      \def\d{0}
					      \pgfmathsetmacro\fa{\f(\a)}
					      \pgfmathsetmacro\fb{\f(\b)}
					      \pgfmathsetmacro\fc{\f(\c)}
					      \pgfmathsetmacro\fd{\f(\d)}
					      \draw[samples=100] plot[domain=-6:-1.1] (\x,{\f(\x)});
					      \draw[samples=100] plot[domain=-0.9:4] (\x,{\f(\x)});
					      \draw[] (-1,-4)--(-1,6);
					      \draw[] (-6,1)--(4,1);
					      \foreach \x/\y in {\a/\fa,\b/\fb,\c/\fc,\d/\fd}{
							      \draw[dashed] (\x,0)|-(0,\y);
							      %\draw[dashed] (-2,3)--(0,-1) (-3,2)--(1,0);
							      \fill[white,draw=black] (\x,\y) circle (1pt);}
					      \node at (-1,1) [ shift = (45:7pt)] {I};
				      \end{tikzpicture}	}
			\item Tập xác định $\mathbb{R} \backslash\{1\}$.\\
			      Sự biến thiên:
			      \begin{itemize}
				      \item [$\bullet$] Đạo hàm: $y^{\prime}=\dfrac{-3}{(x-1)^{2}}<0$ với mọi $x \neq 1$.
				      \item [$\bullet$] Giới hạn và các đường tiệm cận:\\
				            $\displaystyle\lim _{x \rightarrow 1^{-}} y=-\infty, \displaystyle\lim _{x \rightarrow 1^{+}} y=+\infty$. Do đó, đường thẳng $x=1$ là tiệm cận đứng của đồ thị hàm số.\\
				            $\displaystyle\lim _{x \rightarrow+\infty} y=2, \displaystyle\lim _{x \rightarrow-\infty} y=2$. Do đó, đường thẳng $y=2$ là tiệm cận ngang của đồ thị hàm số.
				      \item [$\bullet$] Bảng biến thiên:
				            \begin{center}
					            \begin{tikzpicture}[font=\normalsize,t style/.style={style=solid},scale=.8]
						            %dòng khai báo
						            \tkzTabInit[lgt=1.2,espcl=4,deltacl=0.75]
						            {$x$ /0.75, $y^{\prime}$/0.75, $y$/2.5}
						            {$ -\infty $,$ 1 $,$ +\infty $}
						            %dòng xét dấu
						            \tkzTabLine{ , -,d , -, } % z, t, d;
						            %dòng biến thiên
						            \path ($(N12)!0.5!(N13)$) node (A1){$ 2 $}
						            ($(N22)!0.9!(N23)+(-17pt,0)$) node (A2){$ -\infty $}
						            ($(N22)!0.1!(N23)+(12pt,0)$) node (A3){$ +\infty $}
						            ($(N32)!0.5!(N33)$) node (A4){$ 2 $};
						            \draw[double] (N22)--(N23);
						            \foreach \x/\y in {A1/A2,A3/A4}{
								            \draw[-stealth] (\x)--(\y);
							            }
					            \end{tikzpicture}
				            \end{center}
				            Hàm số nghịch biến trên mỗi khoảng $(-\infty ; 1)$ và $(1 ;+\infty)$.\\
				            Hàm số không có cực trị.
			      \end{itemize}
			      Đồ thị:\\
			      \immini{
				      \begin{itemize}
					      \item Giao điểm của đồ thị với trục tung: $(0 ;-1)$.
					      \item Giao điểm của đồ thị với trục hoành: $\left(-\dfrac{1}{2} ; 0\right)$.
					      \item Đồ thị hàm số đi qua các điểm $(0 ;-1),\left(-\dfrac{1}{2} ; 0\right)$, $(-2 ; 1),(2 ; 5),\left(\dfrac{5}{2} ; 4\right)$ và $(4 ; 3)$.
				      \end{itemize}
			      }
			      {		\begin{tikzpicture}[line cap=butt,line join=miter,>=stealth,scale=0.7,font=\tiny]
					      \tikzset{declare function={xmin=-3.1;xmax=5.1;ymin=-2.1;ymax=6.1;},
						      smooth,samples=450}
					      \draw[->] (xmin-.1,0)--(xmax+.1,0) node[shift={(0:7pt)}]{$ x $};
					      \draw[->] (0,ymin-.1)--(0,ymax+.1) node[shift={(90:7pt)}]{$ y $};
					      \fill (0,0) node[shift={(55:6pt)}]{$ O $};
					      \clip (xmin,ymin-.5) rectangle (xmax,ymax);
					      \foreach \i in {-2,-1,2,3,4}{
							      \draw(\i,1.5pt)--(\i,-1.5pt)node[below]{$\i$};}
					      \foreach \j in {-1,3,4,5}{
							      \draw(-1.5pt,\j)--(1.5pt,\j) node[left]{$\j$};}
					      \draw(-1.5pt,1)--(1.5pt,1)node[shift={(0:3pt)}]{};
					      \draw(-1.5pt,2)--(1.5pt,2)node[shift={(-135:7.5pt)}]{$2$};
					      \draw(1,-1.5pt)--(1,1.5pt)node[shift={(3pt,-7.2pt)}]{$1$};
					      \def\f(#1){(2*(#1)+1)/((#1)-1)} % Hàm số: ( 2x+1 )/( x-1 )
					      \def\a{-2}
					      \def\b{-1}
					      \def\c{-0.5}
					      \def\d{0}
					      \def\e{2}
					      \def\g{2.5}
					      \def\h{4}
					      \pgfmathsetmacro\fa{\f(\a)}
					      \pgfmathsetmacro\fb{\f(\b)}
					      \pgfmathsetmacro\fc{\f(\c)}
					      \pgfmathsetmacro\fd{\f(\d)}
					      \pgfmathsetmacro\fe{\f(\e)}
					      \pgfmathsetmacro\fg{\f(\g)}
					      \pgfmathsetmacro\fh{\f(\h)}
					      \draw[samples=100] plot[domain=-4.8:0.7] (\x,{\f(\x)});
					      \draw[samples=100] plot[domain=1.05:5] (\x,{\f(\x)});
					      \draw[] (1,ymin)--(1,ymax);
					      \draw[] (xmin,2)--(xmax,2);
					      \foreach \x/\y in {\a/\fa,\b/\fb,\c/\fc,\d/\fd,\e/\fe,\g/\fg,\h/\fh}{
							      %\draw[dashed] (0,-1)--(2,5)  (-.5,0)--(2.5,4) ;
							      \draw[dashed] (\x,0)|-(0,\y);
							      \fill[black] (\x,\y) circle (1pt);}
					      \node at (1,2) [shift = (135:5pt)] {I};
				      \end{tikzpicture}	}
			\item Tập xác định: $D = \mathbb{R} \setminus \left\{ 2\right\}$.\\
			      Sự biến thiên:
			      \begin{itemize}
				      \item [$\bullet$] Đạo hàm $y' = \dfrac{ 7}{(-x + 2)^2}>0$, với mọi $x \neq 2$
				      \item [$\bullet$] Giới hạn và tiệm cận:\\
				            $\displaystyle\lim _{x \rightarrow 2^{-}} y=+\infty, \displaystyle\lim _{x \rightarrow 2^{+}} y=-\infty$. Do đó, đường thẳng $x=2$ là tiệm cận đứng của đồ thị hàm số.\\
				            $\displaystyle\lim _{x \rightarrow+\infty} y=-1, \displaystyle\lim _{x \rightarrow-\infty} y=-1$. Do đó, đường thẳng $y=-1$ là tiệm cận ngang của đồ thị hàm số.
				      \item [$\bullet$] Bảng biến thiên:
				            \begin{center}
					            \begin{tikzpicture}[scale=1, font=\footnotesize, line join=round, line cap=round, >=stealth]
						            \tkzTabInit[nocadre=false,lgt=1.2,espcl=2.6,deltacl=0.6]
						            {$x$ /0.6,$y'$ /0.6,$y$ /1.6}
						            {$-\infty$,$2$,$+\infty$}
						            \tkzTabLine{,+,d,+,}
						            \tkzTabVar{-/$-1$,+D-/$+\infty$/$-\infty$,+/$-1$}
					            \end{tikzpicture}
				            \end{center}
				            Hàm số đồng biến trên khoảng $(-\infty;2)$ và $(2;+\infty)$.\\
				            Hàm số không có cực trị.
			      \end{itemize}
			      Đồ thị:
			      \begin{center}
				      \begin{tikzpicture}[scale=0.7,>=stealth, font=\footnotesize, line join=round, line cap=round]
					      \def\xmin{-5} \def\xmax{8}
					      \def\ymin{-8} \def\ymax{8}
					      %\draw[color=gray!50,dashed] (\xmin,\ymin) grid (\xmax,\ymax);
					      \draw[->] (\xmin,0)--(\xmax,0) node [below]{$x$};
					      \draw[->] (0,\ymin)--(0,\ymax) node [left]{$y$};
					      \fill (0,0) circle (1pt) node[shift={(-135:2.5mm)}]{$O$};
					      \node at (current bounding box.south) [below=-2pt] {c)};
					      \clip (\xmin+0.1,\ymin+0.1) rectangle (\xmax-0.1,\ymax-0.1);
					      \draw[smooth,red,samples=300,domain=(\xmin:1.8)] plot(\x,{((\x)+5)/(-(\x)+2)});
					      \draw[smooth,red,samples=300,domain=(2.2:\xmax)] plot(\x,{((\x)+5)/(-(\x)+2)});
					      \draw[blue] (\xmin,-1)--(\xmax,-1);
					      \draw[blue] (2,\ymin)--(2,\ymax);
					      \foreach \x in {\xmin,...,\xmax}
					      \draw (\x,-0.1)--(\x,0.1);
					      \foreach \y in {\ymin,...,\ymax}
					      \draw (-0.1,\y)--(0.1,\y);
					      \node at (0,2.5)[right]{$\frac{5}{2}$};
					      \node at (7,-1)[below]{$y=-1$};
					      \node at (2,7)[right]{$x=2$};
					      \node at (2,-1)[shift={(45:2.5mm)}]{$I$};
				      \end{tikzpicture}\hspace*{2cm}
			      \end{center}
		\end{enumerate}}
\end{vd}
\dongcham{48}
\boxmini{BÀI TẬP TRẮC NGHIỆM}
\ind{PHẦN I.} \inden{Câu trắc nghiệm nhiều phương án lựa chọn. Mỗi câu hỏi học sinh chỉ chọn một phương án.}\\
\setcounter{ex}{0}
\Opensolutionfile{ans}[ans/2D1-B4-d2-1]
\begin{ex}%[2D1B5-1]
	\immini{Hàm số nào trong bốn hàm số dưới đây có bảng biến thiên như hình bên?
		\choice
		{$ y=\dfrac{2x-1}{x+3} $}
		{$ y=\dfrac{4x-6}{x-2} $}
		{$ y=\dfrac{3-x}{2-x}$}
		{\True $ y=\dfrac{x+5}{x-2} $}}
	{\begin{tikzpicture}
			% \tikzset{double style/.append style = {draw=\tkzTabDefaultWritingColor,double=\tkzTabDefaultBackgroundColor,double distance=2pt}}
		\tkzTabInit[nocadre=false,lgt=1,espcl=2.5,deltacl=0.6]
		{$x$/0.6,$y'$/0.6,$y$/1.5}
		{$-\infty$,$2$,$+\infty$}
		\tkzTabLine{,-,d,-,}
		\tkzTabVar{+/$1$,-D+/$-\infty$/$+\infty$,-/$1$}
	\end{tikzpicture}
	}
	\loigiai{
		Xét hàm số $ y=\dfrac{x+5}{x-2} $ có
		$$\heva{&y'=\dfrac{-7}{(x-2)^2}<0, \forall x \in \mathbb{R} \setminus \{2\} \\ & \lim\limits_{x \to \pm \infty} y=1.}$$
	}
\end{ex} \dongcham{1}

\begin{ex}%[2D1B5-1]
	\immini{Hàm số nào trong bốn hàm số dưới đây có bảng biến thiên như hình bên?
		\choice
		{$y=\dfrac{x-1}{x-3}$}
		{$y=\dfrac{x-1}{-x-3}$}
		{\True $y=\dfrac{x+5}{-x+3}$}
		{$y=\dfrac{1}{x-3}$}
	}{
		\begin{tikzpicture}
			% \tikzset{double style/.append style = {draw=\tkzTabDefaultWritingColor,double=\tkzTabDefaultBackgroundColor,double distance=2pt}}
			\tkzTabInit[lgt=1,espcl=2.6]
			{$x$/0.6,$y'$/0.6,$y$/1.5}{$-\infty$,$3$,$+\infty$}
			\tkzTabLine{,+,d,+,}
			\tkzTabVar{-/$-1$,+D-/$+\infty$/$-\infty$,+/$-1$}
		\end{tikzpicture}
	}
	\loigiai{Dựa vào bảng biến thiên, ta suy ra
		\begin{itemize}
			\item Hàm số nghịch biến trên từng khoảng xác định.
			\item Đồ thị hàm số nhận đường thẳng $x=2$ và đường thẳng $y=1$ làm tiệm cận đứng và tiệm cận ngang.
		\end{itemize}
		Vậy ta nhận hàm số $y=\dfrac{x+5}{x-2}$.}
\end{ex} \dongcham{1}

\begin{ex}
	\immini
	{Đường cong trong hình vẽ bên là đồ thị của một trong bốn hàm số sau. Hỏi đó là hàm số nào?
		\haicot
		{\True $y=\dfrac{2x-1}{x+1}$}
		{$y=\dfrac{1-2x}{x+1}$}
		{$y=\dfrac{2x+1}{x-1}$}
		{$y=\dfrac{2x+1}{x+1}$}
	}
	{
		\begin{tikzpicture}[smooth,samples=300,scale=0.45,>=stealth]
			\draw[->] (-5,0)--(3,0) node[below]{$x$};
			\draw[->] (0,-2.5)--(0,4.5) node[right]{$y$};
			\draw (0,0) node[above left]{$O$};
			\draw[line width=1pt,domain=-0.3:3] plot(\x,{(2*\x-1)/(\x+1)});
			\draw[line width=1pt,domain=-5:-2.2] plot(\x,{(2*\x-1)/(\x+1)});
			\draw[fill=black] (0,2) circle(1.5pt) (0,-1) circle(1.5pt) (-1,0) circle(1.5pt);
			\draw (-5,2)--(3,2) (-1,-2.5)--(-1,4.5);
			\draw (0,-1) node[right]{$-1$};
			\draw (-1,0) node[below left]{$-1$};
			\draw (0,2) node[above right]{$2$};
		\end{tikzpicture}
	}
	\loigiai
	{
		Đồ thị hàm số có tiệm cận đứng là $x=-1$ nên loại đáp án $ y=\dfrac{2x+1}{x-1}$.\\
		Đồ thị hàm số đi qua điểm $A(0;-1)$ nên loại đáp án $y=\dfrac{1-2x}{x+1}$ và $ y=\dfrac{2x+1}{x+1}$.
	}
\end{ex} \dongcham{1}

\begin{ex}
	\immini{Đường cong trong hình vẽ bên là đồ thị của một trong bốn hàm số sau. Hỏi đó là hàm số nào?
		\choice
		{$y=\dfrac{x-1}{x-2}$}
		{$y=x+2$}
		{$y=x^4-3x^2+1$}
		{\True $y=\dfrac{2x+1}{x-1}$}
	}{\begin{tikzpicture}[scale=0.5, line join=round, line cap=round,font=\footnotesize,>=stealth,x=0.7cm,y=0.7cm]
			\draw[fill,->] (-5,0)--(0,0) node[below left]{$O$}circle(0.05)--(6,0) node [below] {$x$};
			\draw[->] (0,-4)--(0,6) node [left] {$y$};
			\draw[black,domain=1.75:6, samples=100]plot(\x,{(2*(\x)+1)/((\x)-1)});
			\draw[black,domain=-4.9:0.5, samples=100]plot(\x,{(2*(\x)+1)/((\x)-1)});
			\draw[black,domain=-5:6, samples=100]plot(\x,{2});
			\draw[black,domain=-4:6, samples=100, variable=\t]plot(1,\t);
			\foreach \x in {1}
			\draw (\x,0.05)--(\x,-0.05) node [below right] {\x};
			\foreach \y in {2}
			\draw (0.05,\y)--(-0.05,\y) node [below left] {\y};
		\end{tikzpicture}}
	\loigiai{
		Đồ thị hàm số như hình vẽ nhận đường thẳng $x=1$ là tiệm cận đứng.\\
		Do đó, hàm số cần tìm là $y=\dfrac{2x+1}{x-1}$.
	}
\end{ex} \dongcham{1}

\begin{ex}
	\immini{Đường cong trong hình vẽ bên là đồ thị của một trong bốn hàm số sau. Hỏi đó là đồ thị của hàm số nào?
		\haicot
		{$y=\dfrac{x-2}{x+1}$}
		{$y=\dfrac{x+2}{x-2}$}
		{\True $y=\dfrac{x-2}{x-1}$}
		{$y=\dfrac{x+2}{x-1}$}
	}
	{
		\begin{tikzpicture}[>=stealth,x=1cm,y=1cm,scale=0.5]
			\draw[->] (-3,0)--(0,0) node[below left]{$O$}--(5,0) node[above]{$x$};
			\draw[->] (0,-3) --(0,5) node[left]{$y$};
			\foreach \x in {1,2}{\draw[-] (\x,-0.1)--(\x,0.1);}
			\foreach \y in {1,2}{\draw[-] (-0.1,\y)--(0.1,\y);}
			\draw [domain=-3:0.75, samples=100] plot (\x, {(\x-2)/(\x-1)});
			\draw [domain=1.25:5, samples=100] plot (\x, {(\x-2)/(\x-1)});
			\draw [dashed](-3,1)--(5,1) (1,-3)--(1,5);
			\draw (0,1) node[below left]{$1$};
			\draw (0,2) node[above left]{$2$};
			\draw (1,0) node[below left]{$1$};
			\draw (2,0) node[below right]{$2$};
		\end{tikzpicture}
	}
	\loigiai{
		Từ đồ thị ta thấy
		\begin{itemize}
			\item Tiệm cận ngang là $y=1$, tiệm cận đứng là $x=1$ nên các hàm số $y=\dfrac{x+2}{x-2}$, $y=\dfrac{x-2}{x+1}$ không thỏa mãn.
			\item Giao điểm của đồ thị với trục tung là $(0;2)$ nên hàm số $y=\dfrac{x+2}{x-1}$ không thỏa mãn, hàm số $y=\dfrac{x-2}{x-1}$ thỏa mãn.
		\end{itemize}
	}
\end{ex} \dongcham{1}


\begin{ex}
	\immini
	{Cho hàm số $y=\dfrac{ax-b}{x+c}$ ($a,b,c\in \mathbb{R}$) có đồ thị như hình vẽ bên. Giá trị của biểu thức $2a+b-3c$ bằng
		\haicot
		{$-3$}
		{$4$}
		{\True $7$}
		{$-5$}
	}
	{\begin{tikzpicture}[scale=0.7, font=\footnotesize, line join=round, line cap=round, >=stealth]
			\def\xt{-2.5} \def\xp{4.5} \def\yt{4.5} \def\yd{-2.5}
			\draw[->] (\xt,0)--(\xp,0) node [below]{$x$};
			\draw[->] (0,\yd)--(0,\yt) node [left]{$y$};
			\node at (0,0) [below left]{$O$};
			\clip (\xt,\yd) rectangle (\xp,\yt);
			\draw[smooth,samples=200,domain=\xt:0.99] plot(\x,{(\x-2)/(\x-1)});
			\draw[smooth,samples=300,domain=1.01:\xp] plot(\x,{(\x-2)/(\x-1)});
			\draw[dashed] (1,\yd)--(1,\yt);
			\draw[dashed] (\xt,1)--(\xp,1);
			\fill (1,0)node[shift={(-120:0.3)}]{$1$} circle(1pt);
			\fill (2,0)node[shift={(-60:0.3)}]{$2$} circle(1pt);
			\fill (0,1)node[shift={(230:0.3)}]{$1$} circle(1pt);
			\fill (0,2)node[shift={(150:0.3)}]{$2$} circle(1pt);
		\end{tikzpicture}}
	\loigiai
	{Từ đồ thị hàm số ta có:\\
		Đường tiệm cận đứng là $x=1$ nên $-c=1 \Leftrightarrow c=-1$.\\
		Đường tiệm cận ngang là $y=1$ nên $a=1$.\\
		Đồ thị hàm số đi qua điểm $(0;2)$ nên $\dfrac{-b}{c}=2 \Leftrightarrow b=2$.\\
		Vậy $2a+b-3c = 2+2+3=7$.}
\end{ex} \dongcham{1}

\begin{ex}
	\immini{Cho hàm số $ y=\dfrac{ax+1}{bx-2} $ có đồ thị như hình vẽ. Tính $T=a+b$
		\haicot
		{\True $ T=2 $}
		{$ T=0 $}
		{$ T=-1 $}
		{$ T=3 $}}{
		\begin{tikzpicture}[scale=0.8, font=\footnotesize, line join=round, line cap=round, >=stealth,x=0.7cm,y=0.7cm]
			\def\xmin{-1.3}\def\xmax{6}\def\ymin{-2}\def\ymax{4}
			\draw[->] (\xmin-0.2,0)--(\xmax+0.4,0) node[below] {\footnotesize $x$};
			\draw[->] (0,\ymin-0.2)--(0,\ymax+0.4) node[right] {\footnotesize $y$};
			\draw (0,1) node [above left] {\footnotesize $1$};
			\draw (2,0) node [below right] {\footnotesize $2$};
			\draw (0,0) node [above left] {\footnotesize $O$};
			\foreach \x in {-1,1,3,4,5,6}\draw (\x,0.1)--(\x,-0.1) node [below] {\footnotesize $\x$};
			\foreach \y in {-2,-1,2,3,4}\draw (0.1,\y)--(-0.1,\y) node [left] {\footnotesize $\y$};
			\clip (\xmin,\ymin) rectangle (\xmax,\ymax);
			\draw[dashed] (\xmin,1.0)--(\xmax,1.0);
			\draw[dashed] (2.0,\ymin)--(2.0,\ymax);
			\draw[line width=1pt,smooth,samples=200,domain=\xmin:1.5] plot (\x,{(1*(\x)+1)/(1*(\x)+-2)});
			\draw[line width=1pt,smooth,samples=200,domain=2.3:\xmax] plot (\x,{(1*(\x)+1)/(1*(\x)+-2)});
		\end{tikzpicture}
	}
	\loigiai{
		Từ biểu thức của hàm số, suy ra tiệm cận đứng là $ x=\dfrac{2}{b} $, tiệm cận ngang là $ y=\dfrac{a}{b} $.\\
		Dựa vào hình vẽ, suy ra tiệm cận đứng $ x=2 $, tiệm cận ngang $ y=1 $.\\
		Từ hai điều trên suy ra $ a=1 $, $ b=1 $. Vậy $ T=1+1=2 $.
	}
\end{ex} \dongcham{1}

\begin{ex}
	\immini{
		Cho hàm số $y=\dfrac{ax-b}{cx+2}$ ($a$, $b$, $c\in\mathbb{R}$; $c\neq 0$) có đồ thị như hình vẽ bên. Giá trị của biểu thức $a+b+c$ bằng
		\choice
		{$-3$}
		{$5$}
		{$-4$}
		{\True $3$}
	}{
		\begin{tikzpicture}[scale=0.7, font=\footnotesize, line join=round, line cap=round, >=stealth]
			\def\a{1} \def\b{-3} \def\c{-1} \def\d{2} % Hệ số
			\def\xt{-2} \def\xp{6} \def\yt{2} \def\yd{-4} % x_trái, x_phải, y_trên, y_dưới (giới hạn)
			\draw[->] (\xt,0)--(\xp,0) node [below]{$x$};
			\draw[->] (0,\yd)--(0,\yt) node [left]{$y$};
			\fill (0,0) circle (1.5pt) node[above left]{$O$} (1,0) circle (1.5pt) node[below]{$1$} (2,0) circle (1.5pt) node[below left]{$2$} (3,0) circle (1.5pt) node[below]{$3$} (0,-1) circle (1.5pt) node[above left]{$-1$} (0,-1.5) circle (1.5pt) node[below left]{$-\dfrac{3}{2}$};
			\clip (\xt+0.1,\yd+0.1) rectangle (\xp-0.1,\yt-0.1);
			\draw[smooth,samples=300,domain=\xt:(-\d/\c-0.1)] plot(\x,{(\a*(\x)+\b)/(\c*(\x)+\d)});
			\draw[smooth,samples=300,domain=(-\d/\c+0.1:\xp)] plot(\x,{(\a*(\x)+\b)/(\c*(\x)+\d)});
			\draw[dashed] (-\d/\c,\yd)--(-\d/\c,\yt);
			\draw[dashed] (\xt,\a/\c)--(\xp,\a/\c);
		\end{tikzpicture}
	}
	\loigiai{
		Từ hình vẽ, ta thấy đồ thị hàm số có
		\begin{itemize}
			\item Đường tiệm cận đứng $x=2$, suy ra $-\dfrac{2}{c}=2 \Leftrightarrow c=-1$.
			\item Đường tiệm cận ngang $y=-1$, suy ra $\dfrac{a}{c}=-1 \Leftrightarrow a=-c=1$.
			\item Giao điểm với trục $Oy$ tại điểm $\left(0;-\dfrac{3}{2}\right)$, suy ra $-\dfrac{b}{2}=-\dfrac{3}{2} \Leftrightarrow b=3$.
		\end{itemize}
		Vậy $a+b+c=1+3-1=3$.
	}
\end{ex} \dongcham{1}

\begin{ex}
	\immini{Hãy xác định $a$, $b$ để hàm số $y = \dfrac{2 - ax}{x + b}$ có đồ thị như hình vẽ?
		\choice
		{$a = 1$; $b = - 2$}
		{$a = b = 2$}
		{\True $a = - 1$; $b = -2$}
		{$a = b = -2$}}{

		\begin{tikzpicture}[smooth,samples=300,scale=0.5,>=stealth]
			\draw[->] (-3.5,0)--(6.5,0) node[below]{$x$};
			\draw[->] (0,-2.5)--(0,5) node[right]{$y$};
			\draw (0,0) node[above left]{$O$};
			\draw[line width=1pt,domain=-3.5:0.8] plot(\x,{(\x+2)/(\x-2)});
			\draw[line width=1pt,domain=3:6.5] plot(\x,{(\x+2)/(\x-2)});
			\draw[fill=black] (0,1) circle(1.5pt) (-2,0) circle(1.5pt) (2,0) circle(1.5pt) (0,-1) circle(1.5pt);
			\draw [dashed](-3.5,1)--(6.5,1) (2,-2.5)--(2,5);
			\draw (0,-1) node[right]{$-1$};
			\draw (2,0) node[below right]{$2$};
			\draw (-2,0) node[below left]{$-2$};
			\draw (0,1) node[above right]{$1$};
		\end{tikzpicture}
	}
	\loigiai{
		Đồ thị hàm số có đường tiệm cận đứng là $x = 2$ nên $b + 2 = 0 \Leftrightarrow b = -2$.\\
		Đồ thị hàm số cắt trục hoành tại điểm $\left(-2; 0\right)$ nên $2 + 2a = 0 \Rightarrow a = -1$.
	}
\end{ex} \dongcham{1}


\begin{ex}
	\immini{Cho đồ thị hàm số $y=\dfrac{ax-b}{x-1}$ như hình vẽ. Tìm khẳng định đúng?
		\haicot
		{$a<0$, $b<0$}
		{$0<b<a$}
		{\True $b<0<a$}
		{$a<b<0$}}{
		\begin{tikzpicture}[>=stealth,scale=0.4, line join=round, line
				cap=round,font=\footnotesize]
			\draw[->] (-4,0)--(6,0) node [below]{$x$};
			\draw[->] (0,-4)--(0,6) node [right]{$y$};
			\draw[fill=black] (0,0) circle (2pt) node[below left]{$O$};
			\draw[smooth,samples=300,domain=-4:0.4] plot(\x,{(\x+2)/(\x-1)});
			\draw[smooth,samples=300,domain=1.6:6] plot(\x,{(\x+2)/(\x-1)});
			\draw[dashed] (-4,1)--(6,1) (1,-4)--(1,6);
			\draw[fill=black] (1,0) circle (2pt) node[below left]{$1$};
			\draw[fill=black] (0,1) circle (2pt) node[below left]{$1$};
			\draw[fill=black] (-2,0) circle (2pt) node[below left]{$-2$};
			\draw[fill=black] (0,-2) circle (2pt) node[below left]{$-2$};
		\end{tikzpicture}}
	\loigiai{
		Hàm số có dạng $y=\dfrac{ax-b}{x-1}$.
		\begin{itemize}
			\item Tiệm cận ngang $y=1 \Rightarrow a=1$.
			\item Đồ thị đi qua $(-2;0) \Rightarrow -2a-b=0$
		\end{itemize}
		Suy ra $b<0<a$.
	}
\end{ex} \dongcham{1}

\begin{ex}
	\immini{Cho hàm số $y=\dfrac{ax+4}{bx+c}\ (a,\ b,\ c\in \mathbb{R})$ có bảng biến thiên như sau. Trong các số $a,\ b,\ c$ có bao nhiêu số dương?
		\choice
		{$0$}
		{\True $1$}
		{$2$}
		{$3$}}{
		\begin{tikzpicture}
			\tikzset{double style/.append style = {draw=\tkzTabDefaultWritingColor,double=\tkzTabDefaultBackgroundColor,double distance=2pt}}
			\tkzTabInit[nocadre=false,lgt=1.2,espcl=2.5,deltacl=0.6]
			{$x$ /0.6, $f'(x)$ /0.6, $f(x)$ /1.5}
			{$-\infty$,$1$,$+\infty$}
			\tkzTabLine{ ,+,d,+, }
			\tkzTabVar{-/$3$,+D-/$+\infty$/$-\infty$,+/$3$}
		\end{tikzpicture}}
	\loigiai
	{
		Dựa vào bảng biến thiên, ta có $y(0)>3\Rightarrow\dfrac{4}{c}>0\Rightarrow c>0$.\\
		Đồ thị có tiệm cận đứng $x=1$ và tiệm cận ngang $y=3$ nên $\heva{& -\dfrac{c}{b}>0\\& \dfrac{a}{b}>0}\Rightarrow\heva{&b<0\\&a<0.}$\\
		Vậy $c>0$, $a<0$, $b<0$.
	}
\end{ex} \dongcham{4}

\begin{ex}%[2D1K5-1]%
	\immini
	{
		Cho hàm số $y=\dfrac{ax+b}{cx+d}$ với $a>0$ có đồ thị như hình vẽ bên. Mệnh đề nào sau đây đúng?
		\choice
		{$b<0$, $c<0$, $d<0$}
		{$b>0$, $c<0$, $d<0$}
		{$b<0$, $c>0$, $d<0$}
		{\True $b>0$, $c>0$, $d<0$}
	}
	{\begin{tikzpicture}[scale=0.7, font=\footnotesize, line join=round, line cap=round,>=stealth,x=0.4cm,y=0.4cm]
			\def \xmin{-5.0};
			\def \xmax{6.3};
			\def \ymin{-4.0};
			\def \ymax{5.5};
			\draw[->] (\xmin, 0.) -- (\xmax,0.) node[anchor=north] {$x$};
			\draw[->] (0.,\ymin) -- (0.,\ymax) node[anchor=west] {$y$};
			\clip(\xmin,\ymin) rectangle (\xmax,\ymax);
			\draw[smooth,samples=100,domain=\xmin-0.1:1-0.1] plot(\x,{((\x)+2)/((\x)-1)});
			\draw[smooth,samples=100,domain=1+0.1:\xmax-0.1] plot(\x,{((\x)+2)/((\x)-1)});
			\draw[dashed] (\xmin,1)--(\xmax,1) (1,\ymin)--(1,\ymax);
			\draw[fill=black] (0,0) circle (1pt) node[above left] {$O$};
		\end{tikzpicture}
	}
	\loigiai{
		Đồ thị hàm số có đường tiệm cận ngang $y=\dfrac{a}{c}$ nằm trên trục $Ox$ nên $\dfrac{a}{c}>0\overset{a>0}{\Rightarrow} c>0$.\\
		Đồ thị hàm số có đường tiệm cận đứng $x=-\dfrac{d}{c}$ nằm bên phải trục $Oy$ nên $-\dfrac{d}{c}>0\overset{c>0}{\Rightarrow}d<0$.\\
		Vậy mệnh đề đúng là \lq\lq $b>0$, $c>0$, $d<0$\rq\rq.
	}
\end{ex} \dongcham{4}

\begin{ex}
	\immini{Hình vẽ bên là đồ thị của hàm số $y=\dfrac{ax+b}{cx+d}$. Mệnh đề nào sau đây là đúng?
		\choice
		{$ab>0,bd<0$}
		{$ab<0,ad>0$}
		{\True $ab<0,ad<0$}
		{$bd>0,ad>0$}
	}{\begin{tikzpicture}[smooth,samples=300,line width=0.6pt,>=stealth, scale=0.5]
			\draw[->] (-4,0)--(4.5,0) node[below]{$x$};
			\draw[->] (0,-2)--(0,4.5) node[right]{$y$};
			\draw (0,0) node[below right]{$O$};
			\draw[dashed] (-0.5,-2)--(-0.5,4.5) (-4,1)--(4.5,1);
			\draw[line width=1pt,domain=-4:-0.8] plot(\x,{(2*(\x)-1)/(2*(\x)+1)});
			\draw[line width=1pt,domain=-0.15:4.5] plot(\x,{(2*(\x)-1)/(2*(\x)+1)});
		\end{tikzpicture}
	}
	\loigiai{
		Ta có
		\begin{itemize}
			\item [$\bullet$] Đường tiệm cận đứng $x=-\dfrac{d}{c}$. Theo hình vẽ thì $-\dfrac{d}{c}<0 \Rightarrow cd >0$ \quad (1).
			\item [$\bullet$] Đường tiệm cận ngang $y=\dfrac{a}{c}$. Theo hình vẽ thì $\dfrac{a}{c}<0 \Rightarrow ac <0$ \quad (2).
			\item [$\bullet$] Giao điểm với trục tung tại điểm có tung độ $y=\dfrac{b}{d}$. Theo hình vẽ thì $\dfrac{b}{d}>0 \Rightarrow bd >0$ \quad (3).
			\item [$\bullet$] Giao điểm với trục hoành tại điểm có hoành độ $x=-\dfrac{b}{a}$. Theo hình vẽ thì $-\dfrac{b}{a}>0 \Rightarrow ab <0$ \quad (4).
		\end{itemize}
		Lấy (3) nhân với (4), ta được $ad \cdot b^2 <0$. Suy ra $ad<0$.\\
		Mặt khác theo (4) thì $ab<0$.
	}
\end{ex} \dongcham{4}


\begin{ex}
	\immini{Hình vẽ dưới đây là đồ thị hàm số $y=\dfrac{ax+b}{cx+d}$ $ac\ne0$, $ad-cb\ne0$. Mệnh đề nào sau đây đúng?
		\choice
		{\True $ad>0$ và $ab<0$}
		{$bd<0$ và $ab>0$}
		{$ad<0$ và $ab<0$}
		{$ad>0$ và $bd>0$}
	}
	{\begin{tikzpicture}[>=stealth,font=\footnotesize,scale=0.6]
			\draw[->](-4,0)--(3,0)node[below]{$x$};
			\draw[->](0,-3.5)--(0,3)node[right]{$y$};
			\draw[smooth,samples=100,domain=-4:-1.4]plot(\x,{(\x-1)/(2*\x+2)});
			\draw[smooth,samples=100,domain=-0.75:3]plot(\x,{(\x-1)/(2*\x+2)});
			\draw(-4,0.5)--(3,0.5) (-1,-3.5)--(-1,3);
			\fill (0,0)node[below left]{$O$}circle (1.2pt);
		\end{tikzpicture}}
	\loigiai{
		\begin{itemize}
			\item Đồ thị hàm số cắt trục $Oy$ tại điểm có tung độ âm $\Rightarrow\dfrac{b}{d}< 0\Rightarrow bd<0$.
			\item Đồ thị hàm số cắt trục $Ox$ tại điểm có hoành độ dương $\Rightarrow-\dfrac{b}{a}> 0\Rightarrow ab<0$.
			\item Đồ thị hàm số có tiệm cận ngang $y=\dfrac{a}{c}>0\Rightarrow ac>0.\quad(1)$
			\item Đồ thị hàm số có tiệm cận đứng $x=-\dfrac{d}{c}<0\Rightarrow cd>0.\quad(2)$
		\end{itemize}
		Từ $(1)$ và $(2)\Rightarrow ad>0$.}
\end{ex} \dongcham{4}
\Closesolutionfile{ans}

\ind{PHẦN II.} \inden{Câu trắc nghiệm đúng sai. Trong mỗi ý a), b), c), d) ở mỗi câu, học sinh chọn đúng hoặc sai.}\\
\Opensolutionfile{ans}[ans/2D1-B4-d2-2]

\begin{ex}
	\immini{Cho hàm số $y = \dfrac{x + a}{b x +c}$, $\left( a, b, c \in \mathbb{Z}\right) $.
		\choiceTF
		{\True Đồ thị hàm số có tiệm cận đứng $x=1$}
		{Đồ thị hàm số có tiệm cận ngang $y=0$}
		{Hàm số đồng biến trên $\mathbb{R}$}
		{\True $a - 3b - 2c=-3$}
	}{
		\begin{tikzpicture}[font=\footnotesize,line join=round, line cap=round,>=stealth,scale=0.7]
			\tikzset{label style/.style={font=\footnotesize}}
			\def \xmin{-2.7}
			\def \xmax{4}
			\def \ymin{-2.2}
			\def \ymax{4}
			\draw[->] (\xmin,0)--(\xmax,0) node[below left] {$x$};
			\draw[->] (0,\ymin)--(0,\ymax) node[below left] {$y$};
			\draw (0,0) node [below left] {$O$};
			\draw (1,0) node [below left] {$1$} circle (1.2pt);
			\draw (2,0) node [below right] {$2$} circle (1.2pt);
			\draw (0,1) node [above left] {$1$} circle (1.2pt);
			\draw (0,2) node [above left] {$2$} circle (1.2pt);
			\begin{scope}
				\clip (\xmin+0.01,\ymin+0.01) rectangle (\xmax-0.01,\ymax-0.01);
				\draw[samples=350,domain=\xmin+0.01:\xmax-0.01,smooth,variable=\x] plot (\x,{(\x-2)/(\x-1)});
				\draw[samples=200,domain=\xmin+0.01:\xmax-0.01,smooth,variable=\x] plot (\x,{1});
			\end{scope}
		\end{tikzpicture}
	}
	\loigiai{
		Căn cứ vào đồ thị, ta có
		\begin{enumerate}[a)]
			\item Đồ thị hàm số có tiệm cận đứng $x=1$.
			\item Đồ thị hàm số có tiệm cận ngang $y=1$
			\item Hàm số đồng biến trên các khoảng $(-\infty,1)$ và $(1;+\infty)$
			\item Đồ thị hàm số có tiệm cận ngang $y = 1$ nên $\dfrac{1}{b} = 1 \Rightarrow b = 1$.\\
			      Đồ thị hàm số có tiệm cận đứng $x = 1$ nên $-\dfrac{c}{b} = 1$ mà $b = 1$ $\Rightarrow c = -1$.\\
			      Đồ thị hàm số cắt trục tung tại điểm $(0; 2)$ nên $\dfrac{a}{c} = 2$ mà $c = -1$ nên $a = -2$.\\
			      Vậy $T = a - 3b - 2c = -2 - 3 \cdot 1 -2 \cdot (-1) =-3 $.
		\end{enumerate}

	}
\end{ex} \dongcham{4}

\begin{ex}%[2D1K5-1]%
	Cho hàm số $ f(x)=\dfrac{a x-1}{b x+c}\ (a, b, c\in\mathbb{R})$ có bảng biến thiên như sau.
	\begin{center}
		\begin{tikzpicture}
			\tikzset{double style/.append style = {draw=\tkzTabDefaultWritingColor,double=\tkzTabDefaultBackgroundColor,double distance=2pt}}
			\tkzTabInit[espcl=2.5,lgt=1.2,nocadre=false]
			{$x $ /0.7, $ f'(x)$ /0.7, $ f(x)$ /2.1}
			{$-\infty $, $ 3 $, $+\infty$}
			\tkzTabLine{,-,d,-,}
			\tkzTabVar{+/ $\dfrac{1}{2}$,-D+/ $-\infty $ / $+\infty $,-/ $\dfrac{1}{2}$}
		\end{tikzpicture}
	\end{center}
		\choiceTF
		{\True Hàm số nghịch biến trên khoảng $\left( -\infty,\dfrac{1}{2}\right)$}
		{Đồ thị hàm số có tiệm cận đứng $x=\dfrac{1}{2}$}
		{\True Đồ thị giao với trục hoành tại điểm có hoành độ nhỏ hơn $3$}
		{\True $\hoac{&b>\dfrac{2}{3}\\ &b<0}$}
	\loigiai{
		\begin{enumerate}[a)]
			\item Hàm số đồng biến trên các khoảng $(-\infty,3)$ nên nghịch biến trên khoảng $\left( -\infty,\dfrac{1}{2}\right)$.
			\item Đồ thị hàm số có tiệm cận đứng $x=3$.
			\item Đồ thị giao với trục hoành tại điểm thuộc nhánh trái của đồ thị, suy ra hoành độ giao điểm này nhỏ hơn $3$.
			\item Từ bảng biến thiên suy ra
			      \[
				      \heva{&\dfrac{a}{b}=\dfrac{1}{2}\\&-\dfrac{c}{b}=3.}\quad\quad (1)
			      \]
			      Ta có $ y'=\dfrac{ac+b}{(bx+c)^2}<0 $, $\forall x\ne-\dfrac{c}{b}\Leftrightarrow ac+b<0 $.\quad\quad (2)\\
			      Từ (1) và (2) suy ra $\dfrac{b}{2}\cdot (-3b)+b<0\Leftrightarrow\hoac{&b>\dfrac{2}{3}\\ &b<0.}$
		\end{enumerate}
	}
\end{ex} \dongcham{4}

\begin{ex}
	\immini{Cho hàm số $f(x)=\dfrac{ax+b}{cx+d}$ với $a$, $b$, $c$, $d \in \mathbb{R}$ có đồ thị hàm số $y=f'(x)$ nhận $x=-1$ làm tiệm cận đứng như hình vẽ bên. Biết rằng giá trị lớn nhất của hàm số $y=f(x)$ trên đoạn $[-3;-2]$ bằng $8$.
		\choiceTF
		{\True $f'(0)=3$}
		{Hàm số $f(x)$ nghịch biến trên khoảng $(-1;+\infty)$}
		{Giá trị của $f(-3)$ bằng $8$}
		{\True  Giá trị của $f(2)$ bằng $4$}
	}
	{\begin{tikzpicture}[>=stealth,scale=0.6, line join=round, line cap=round]
			\def\a{3} \def\b{0} \def\c{1} \def\d{1} % Hệ số
			\def\xt{-4.5} \def\xp{4.5} \def\yt{5.5} \def\yd{-1}
			\draw[->] (\xt,0)--(\xp,0) node [below]{$x$};
			\draw[->] (0,\yd)--(0,\yt) node [left]{$y$};
			\node at (0,0) [below left]{$O$};
			\clip (\xt-0.1,\yd+0.1) rectangle (\xp-0.1,\yt-0.1);
			\draw[smooth,samples=300,domain=\xt:(-\d/\c-0.1)] plot(\x,{(\a)/(\c*(\x)+\d)^2});
			\draw[smooth,samples=300,domain=(-\d/\c+0.1:\xp)] plot(\x,{(\a)/(\c*(\x)+\d)^2});
			\draw (-\d/\c,\yd)--(-\d/\c,\yt);
			\draw (-1,0) node [below left] {$-1$} circle (1.2pt)
			(0,3) node [right] {$3$} circle (1.2pt);
		\end{tikzpicture}}
	\loigiai{
		\begin{enumerate}[a)]
			\item Theo hình vẽ, đồ thị $f'(x)$ qua điểm $(0;3)$ nên $f'(0)=3$.
			\item Do $f'(x)>0$, $\forall x \ne -1$ nên hàm số $f(x)$ đồng biến trên các khoảng $(-\infty;-1)$ và $(-1;+\infty)$.
			\item Vì $f'(x)>0$, $\forall x \ne -1 \Rightarrow \max \limits_{[-3;-2]} f(x)=f(-2)=8$. Suy ra $f(-3) \ne 8$.
			\item Ta có $f'(x)=\dfrac{ad-bc}{(cx+d)^2}$.\\
			      Đồ thị hàm số đi qua điểm $(0;3)$ nên $f'(0)=3 \Leftrightarrow \dfrac{ad-bc}{d^2}=3$.\\
			      Mặt khác, đồ thị hàm số $y=f'(x)$ có tiệm cận đứng $x=-1$ nên $-c+d=0$.\\
			      Vì $f'(x)>0$, $\forall x \ne -1 \Rightarrow \max \limits_{[-3;-2]} f(x)=f(-2)=8 \Leftrightarrow \dfrac{-2a+b}{-2c+d}=8$.\\
			      Vậy ta có hệ phương trình $\heva{&ad-bc=3d^2\\&-c+d=0\\&b-2a=8(d-2c)}\Leftrightarrow\heva{&c=d\\&a-b=3d\\&b-2a=-8d}\Leftrightarrow\heva{&a=5d\\&b=2d\\&c=d.}$\\
			      Từ đó suy ra $f(x)=\dfrac{5\mathrm{\,d}x+2d}{\mathrm{\,d}x+d}=\dfrac{5x+2}{x+1} \Rightarrow f(2)=4$.
		\end{enumerate}
	}
\end{ex} \dongcham{5}
\Closesolutionfile{ans}
% \begin{dang}{Khảo sát và vẽ đồ thị hàm số phân thức hữu tỉ bậc II/I}
	\begin{itemize}
		\item[\iconCH] \indamm{Bước 1.} Tập xác định $D=\mathbb{R}\backslash \left\{-\dfrac{n}{m}\right\}$.
		\item [\iconCH] \indamm{Bước 2.} Khảo sát sự biến thiên của hàm số
		\begin{itemize}
			\item Tính đạo hàm $y'=\dfrac{am\cdot x^2+2an\cdot x + b.n - m.c}{(mx+n)^2}$. Giải $y'=0 \Leftrightarrow am\cdot x^2+2an\cdot x + b.n - m.c=0$, tìm nghiệm.
			\item Tìm các giới hạn tại vô cực, giới hạn vô cực và tìm tiệm cận của đồ thị hàm số.
			\item Lập bảng biến thiên; xác định chiều biến thiên và cực trị của hàm số.
		\end{itemize}
		\item [\iconCH] \indamm{Bước 3.} Cho thêm điểm và vẽ đồ thị của hàm số dựa vào bảng biến thiên.
	\end{itemize}
\end{dang}
\boxmini{BÀI TẬP TỰ LUẬN}
\begin{vd}
	Khảo sát sự biến thiên và vẽ đồ thị các hàm số sau:
	\begin{tasks}(3)
		\task $ y = \dfrac{x^2+ 2x - 2}{x - 1}$;
		\task $y=-x+2-\dfrac{1}{x+1}$;
		\task $y=\dfrac{-x^2-3x+4}{x+2}$.
	\end{tasks}
\loigiai{
\begin{enumerate}[a)]
	\item Ta viết lại hàm số $ y = \dfrac{x^2+ 2x - 2}{x - 1}=x+3+\dfrac{1}{x-1}$.\\
	Tập xác định: $D = \mathbb{R} \setminus \left\{ 1 \right\}$.\\
	Sự biến thiên:
	\begin{itemize}
		\item [$\bullet$] Đạo hàm $y'= \dfrac{x^2 - 2x}{(x - 1)^2}$; $y' = 0 \Leftrightarrow x = 0$  hoặc $x = 2$.
		\item [$\bullet$] Giới hạn và tiệm cận:\\
		$\displaystyle\lim \limits{n \to +\infty}_{x \rightarrow-\infty} y=-\infty,\, \displaystyle\lim \limits{n \to +\infty}_{x \rightarrow +\infty} y=+\infty$.\\
		$\displaystyle\lim \limits{n \to +\infty}_{x \rightarrow 1^{-}} y= -\infty,\, \displaystyle\lim \limits{n \to +\infty}_{x \rightarrow 1^{+}} y=+\infty$. Suy ra $x=1$ là tiệm cận đứng.\\
		$\displaystyle\lim \limits{n \to +\infty}_{x \rightarrow -\infty} \left(y-(x+3) \right) = 0,\, \displaystyle\lim \limits{n \to +\infty}_{x \rightarrow +\infty} \left(y-(x+3) \right) = 0$. Suy ra $y=x+3$ là tiệm cận xiên.
		\item [$\bullet$] Bảng biến thiên:
		\begin{center}
			\begin{tikzpicture}[scale=1, font=\footnotesize, line join=round, line cap=round, >=stealth]
				\tikzset{double style/.append style = {draw=\tkzTabDefaultWritingColor,double=\tkzTabDefaultBackgroundColor,double distance=2pt}}
				\tkzTabInit[nocadre=false,lgt=1.2,espcl=2.2,deltacl=0.6]
				{$x$ /0.6,$y'$ /0.6,$y$ /1.6}
				{$-\infty$,$0$,$1$,$2$,$+\infty$}
				\tkzTabLine{,+,0,-,d,-,0,+,}
				\tkzTabVar{-/$-\infty$,+/$2$,-D+/$-\infty$/$+\infty$,-/$6$,+/$+\infty$}
			\end{tikzpicture}
		\end{center}
	Hàm số đồng biến trên khoảng $(-\infty;0)$ và $(2;+\infty)$; nghịch biến trên khoảng $(0;1)$ và $(1;2)$.\\
	Hàm số đạt cực tiểu tại $x = 2$  và ${y_{CT}} = 6$ .\\
	Hàm số đạt cực đại tại $x = 0$ và ${y_{CĐ}} = 2$ .\\
	\end{itemize}
	Đồ thị:\\
	\immini{
		\begin{itemize}
			\item [$\bullet$] Đồ thị hàm số giao với trục $Ox$ tại điểm $(-1+\sqrt{3}; 0)$ và điểm $(-1-\sqrt{3}; 0)$.
			\item [$\bullet$] Đồ thị nhận $I(1;4)$ làm tâm đối xứng.
		\end{itemize}
	}{
	\begin{tikzpicture}[line join=round, line cap=round,>=stealth,thick,x=0.8cm,y=0.8cm]
		\tikzset{every node/.style={scale=0.9}}
		\draw[->] (-3.8,0)--(5.6,0) node[below] {$x$};
		\draw[->] (0,-1.1)--(0,7.6) node[below left] {$y$};
		\draw (0,0) node [below left] {$O$};
		\draw (1,4) circle (1pt) node [below right] {$I$};
		\draw (1,-1.5) node [right] {Hình 5};
		\draw[dashed,thin] (1.01,-1)--(1.01,7.5) node [pos=0.4,sloped,black,below] {$x=1$} ;
		\begin{scope}
			\clip (-4,-1) rectangle (6.5,7.5);
			\draw[samples=200,domain=-3.5:0.99,smooth,variable=\x] plot (\x,{(1*((\x)^2)+2*(\x)+-2)/(1*(\x)+-1)});
			\draw[samples=200,domain=1.01:6,smooth,variable=\x] plot (\x,{(1*((\x)^2)+2*(\x)+-2)/(1*(\x)+-1)});
			\draw[dashed,thin] (-3.6,-0.6)--(4.1,7.1) node [pos=0.8,sloped,black,below] {$y=x+3$};
		\end{scope}
		\foreach \x/\g in {-3/-90,-2/-90,-1/-90,1/-60,2/-90,3/-90,4/-90,5/-90}
		\draw[thin] (\x,2pt)--(\x,-2pt) + (\g:3mm) node [scale=0.8] {$\x$};
		%Vẽ các điểm trên trục Oy
		\foreach \y/\g in {1/180,2/140,3/180,6/180,4/180,5/180}
		\draw[thin] (2pt,\y)--(-2pt,\y) + (\g:3mm) node [scale=0.8] {$\y$};
\end{tikzpicture}}

	\item Tập xác định: $\mathscr{D}=\mathbb{R} \backslash\{-1\}$.\\
	Sự biến thiên:
	\begin{itemize}
		\item [$\bullet$] Đạo hàm $y'=-1+\dfrac{1}{(x+1)^2}$, $y'=0\Leftrightarrow x=-2$ hoặc $x=0$.
		\item [$\bullet$] Giới hạn và tiệm cận:\\
		\begin{itemize}
			\item $\lim\limits_{x \rightarrow +\infty} y= -\infty, \lim\limits_{x \rightarrow -\infty} y= +\infty$.
			\item $\lim\limits_{x \rightarrow (-1)^{-}} y= +\infty, \lim\limits_{x \rightarrow (-1)^{+}} y= -\infty$.
		\end{itemize}
		Do đó, đường thẳng $x=-1$ là tiệm cận đứng của đồ thị hàm số.
		\begin{itemize}
			\item $\lim\limits_{x \rightarrow+\infty}[y - (-x+2)]=\lim\limits_{x \rightarrow +\infty} \dfrac{-1}{x+1}=0$,
			\item $ \lim\limits_{x \rightarrow-\infty}[y - (-x+2)]=\lim\limits_{x \rightarrow +\infty} \dfrac{-1}{x+1}=0$.
		\end{itemize}
		Do đó, đường thẳng $y= -x+2 $ là tiệm cận xiên của đồ thị hàm số.
		\item [$\bullet$] Bảng biến thiên:
		\begin{center}
			\begin{tikzpicture}
				\tikzset{double style/.append style = {draw=\tkzTabDefaultWritingColor,double=\tkzTabDefaultBackgroundColor,double distance=2pt}}
				\tkzTabInit[lgt=1.2, espcl=2.5, deltacl=0.6]
				{$x$/0.6, $y'$/0.6, $y$/2}
				{$-\infty$, $-2$, $-1$, $0$, $+\infty$}
				\tkzTabLine{, -, 0, +, d, +, 0, -, }
				\tkzTabVar{+/$+\infty$, -/$5$, +D-/$+\infty$/$-\infty$, +/$1$, -/$-\infty$}
			\end{tikzpicture}
		\end{center}
		Hàm số đồng biến trên các khoảng $(-2;-1)$, $(-1;0)$ và nghịch biến trên các khoảng $(-\infty;-2)$, $(0;+\infty)$.\\
		Hàm số đạt cực tiểu tại $x=-2$, $y_{_\text{CT}}=5$; đạt cực đại tại $x=0$, $y_{_\text{CĐ}}=1$.\\
	\end{itemize}
	Đồ thị:\\
	\immini{
		\begin{itemize}
			\item [$\bullet$] Đồ thị hàm số qua các điểm $\left(-3;-\dfrac{11}{2} \right)$, $\left(3;-\dfrac{5}{4} \right)$.
			\item [$\bullet$] Đồ thị nhận $I(-1;3)$ làm tâm đối xứng.
		\end{itemize}
	}{
		\begin{tikzpicture}[>=stealth, scale=0.6, font=\footnotesize,x=1cm,y=1cm]
			\draw[->] (-5,0)--(5,0) node[below] {$x$};
			\draw[->] (0,-4)--(0,8.5) node[left] {$y$};
			\draw[domain=-0.85:3.8, smooth] plot (\x, {-(\x)^2+(\x)+1)/(\x+1)});
			\draw[domain=-4:-1.2, smooth] plot (\x, {-(\x)^2+(\x)+1)/(\x+1)});
			\draw[domain=-4.5:4, smooth] plot (\x, {-\x+2});
			\draw (-1,-4)--(-1,8.2);
			\draw[fill=black] (0,0) node[below left=-0.1] {$O$} circle (1.2pt);
			\draw[fill=black] (1,0) node[below] {$1$} circle (1.2pt);
			\draw[fill=black] (2,0) node[above] {$2$} circle (1.2pt);
			\draw[fill=black] (-1,0) node[above] {$-1$} circle (1.2pt);
			\draw[fill=black] (-2,0) node[below ] {$-2$} circle (1.2pt);
			\draw[fill=black] (3,0) node[above ] {$3$} circle (1.2pt);
			%	\draw[fill=black] (-4,0) node[below] {$-4$} circle (1.2pt);
			\draw[fill=black] (0,5) node[right] {$5$} circle (1.2pt);
			%		\draw[fill=black] (0,17) node[right] {$17$} circle (1.2pt);
			\draw[fill=black] (0,-1.25) node[ left] {$-\dfrac{5}{4}$} circle (1.2pt);
			\draw[fill=black] (0,1) node[above right] {$1$} circle (1.2pt);
			\draw[dashed] (-2,0)--(-2,5)--(0,5) (3,0)--(3,-1.25)--(0,-1.25) (1,0)--(1,0.5)--(0,0.5) ;
	\end{tikzpicture}}
	
	\item Ta viết lại hàm số $ y = \dfrac{x^2+ 2x - 2}{x - 1}=x+3+\dfrac{1}{x-1}$.\\
	Tập xác định: $D=\mathbb{R} \setminus\{-2\}$.\\
	Sự biến thiên:
	\begin{itemize}
		\item [$\bullet$] Đạo hàm Đạo hàm $y'=\dfrac{-x^2-4x-10}{(x+2)^2}<0$, với mọi $x \ne -2$.
		\item [$\bullet$] Giới hạn và tiệm cận:\\
		$$
		\lim\limits_{x \to-\infty} y=\lim\limits_{x \to-\infty} \dfrac{-x^2-3 x+4}{x+2}=+\infty; \lim\limits_{x \to+\infty} y=\lim\limits_{x \to+\infty} \dfrac{-x^2-3 x+4}{x+2}=-\infty.
		$$		
		Ta có 
		\begin{itemize}
			\item $a=\lim\limits_{x \to+\infty} \dfrac{-x^2-3x+4}{x^2+2x}=-1$.
			\item $b=\lim\limits_{x \to+\infty}\left[\dfrac{-x^2-3x+4}{x+2}-(-1) x\right]=\lim\limits_{x \to+\infty}\left(\dfrac{-x+4}{x+2}\right)=-1$.
		\end{itemize}	
		Suy ra đường thẳng $y=-x-1$ là tiệm cận xiên của đồ thị hàm số.\\		
		Ta có $\lim\limits_{x \to-2^{-}} y=\lim\limits_{x \to-2^{-}} \dfrac{-x^2-3x+4}{x+2}=-\infty; \lim\limits_{x \to-2^{+}} y=\lim\limits_{x \to-2^{+}} \dfrac{-x^2-3x+4}{x+2}=+\infty$. Suy ra đường thẳng $x=-2$ là tiệm cận đứng của đồ thị hàm số.
		\item [$\bullet$] Bảng biến thiên:
		\begin{center}
			\begin{tikzpicture}
				\tikzset{double style/.append style = {draw=\tkzTabDefaultWritingColor,double=\tkzTabDefaultBackgroundColor,double distance=2pt}}
				\tkzTabInit[nocadre=false,espcl=3,lgt=1.5]
				{$x$/0.7,$y'$/0.7,$y$/2.1}
				{$-\infty$,$-2$,$+\infty$}
				\tkzTabLine{,-,d,-,}
				\tkzTabVar{+/$+\infty$,-D+/$-\infty$/$+\infty$,-/$-\infty$}
			\end{tikzpicture}
		\end{center}
		Hàm số nghịch biến trên khoảng $(-\infty;-2)$ và $(-2;+\infty)$.\\
		Hàm số không có cực trị.
	\end{itemize}
	Đồ thị:\\
	\immini{
		\begin{itemize}
			\item [$\bullet$] Đồ thị hàm số giao với trục $Ox$ tại điểm $(-4; 0)$ và điểm $(1; 0)$.
			\item [$\bullet$] Đồ thị nhận $I(-2;1)$ làm tâm đối xứng.
		\end{itemize}
	}{
		\begin{tikzpicture}[>=stealth, scale=0.6, font=\footnotesize]
			\draw[->] (-7,0)--(5.5,0) node[below] {$x$};
			\draw[->] (0,-7)--(0,7) node[left] {$y$};
			\draw[domain=-1.1:5, smooth] plot (\x, {-(\x)^2-3*(\x)+4)/(\x+2)});
			\draw[domain=-6.4:-2.7, smooth] plot (\x, {-(\x)^2-3*(\x)+4)/(\x+2)});
			\draw[domain=-6:5, smooth] plot (\x, {-\x-1});
			\draw (-2,-7)--(-2,7);
			\draw[fill=black] (0,0) node[below left=-0.1] {$O$} circle (1.2pt);
			\draw[fill=black] (1,0) node[below] {$1$} circle (1.2pt);
			\draw[fill=black] (-1,0) node[below left] {$-1$} circle (1.2pt);
			\draw[fill=black] (2,0) node[below right=0 and -0.1] {$2$} circle (1.2pt);
			\draw[fill=black] (4,0) node[above] {$4$} circle (1.2pt);
			\draw[fill=black] (0,6) node[right] {$6$} circle (1.2pt);
			\draw[fill=black] (0,2) node[below left] {$2$} circle (1.2pt);
			\draw[fill=black] (0,-2.8) node[left] {$-\dfrac{14}{5}$} circle (1.2pt);
			\draw[fill=black] (0,-4) node[left] {$-4$} circle (1.2pt);
			\draw[dashed] (3,0)--(3,-2.8)--(0,-2.8) (4,0)--(4,-4)--(0,-4) (-1,0)--(-1,6)--(0,6);
			\node [above=-1mm, fill=white,font=\footnotesize] at (1.5,-7) {\it Hình $6$};
	\end{tikzpicture}}

\end{enumerate}}
\end{vd}

\boxmini{BÀI TẬP TRẮC NGHIỆM}
\ind{PHẦN I.} \inden{Câu trắc nghiệm nhiều phương án lựa chọn. Mỗi câu hỏi học sinh chỉ chọn một phương án.}\\
\setcounter{ex}{0}
\Opensolutionfile{ans}[ans/2D1-B4-d3-1]
\begin{ex}
	\immini{Bảng biến thiên sau là của một trong bốn hàm số sau. Hỏi đó là hàm số nào?
	\choice
	{$y=\dfrac{x^2-3x+4}{-x-4}$}
	{\True $y=\dfrac{x^2-4x+4}{-x-4}$}
	{$y=\dfrac{x^2-5x+4}{x+4}$}
	{$y=\dfrac{x^2-4x+4}{x+4}$}}{
	\begin{tikzpicture}
		\tikzset{double style/.append style = {draw=\tkzTabDefaultWritingColor,double=\tkzTabDefaultBackgroundColor,double distance=2pt}}
		\tkzTabInit[nocadre=false,lgt=1,espcl=1.6]
		{$x$ /0.6,$y'$ /0.6,$y$ /1.5}
		{$-\infty$,$-10$,$-4$,$2$,$+\infty$}
		\tkzTabLine{,-,$0$,+,d,+,$0$,-,}
		\tkzTabVar{+/$+\infty$,-/$24$,+D-/$+\infty$/$-\infty$,+/$0$,-/$-\infty$}
	\end{tikzpicture}
}
\loigiai{
}
\end{ex}

\begin{ex}
	\immini{Bảng biến thiên sau là của một trong bốn hàm số sau. Hỏi đó là hàm số nào?
		\choice
		{$y=\dfrac{x^2-4x+3}{x-3}$}
		{$y=\dfrac{-x^2-x+2}{x-3}$}
		{\True $y=\dfrac{-x^2+x+2}{x-3}$}
		{$y=\dfrac{x^2-4x+4}{-x+3}$}}{
		\begin{tikzpicture}
			\tikzset{double style/.append style = {draw=\tkzTabDefaultWritingColor,double=\tkzTabDefaultBackgroundColor,double distance=2pt}}
			\tkzTabInit[nocadre=false,lgt=1,espcl=1.6]
			{$x$ /0.6,$y'$ /0.6,$y$ /1.5}
			{$-\infty$,$1$,$3$,$5$,$+\infty$}
			\tkzTabLine{,-,$0$,+,d,+,$0$,-,}
			\tkzTabVar{+/$+\infty$,-/$-1$,+D-/$+\infty$/$-\infty$,+/$-9$,-/$-\infty$}
		\end{tikzpicture}
	}
\loigiai{
}
\end{ex}

\begin{ex}
	\immini{Bảng biến thiên sau là của một trong bốn hàm số sau. Hỏi đó là hàm số nào?
		\choice
		{\True $y=\dfrac{x^2-2x+1}{x+4}$}
		{$y=\dfrac{x^2-4x+2}{x+4}$}
		{$y=\dfrac{x^2-x+2}{-x-4}$}
		{$y=\dfrac{x^2-3x+4}{-x-4}$}}{
		\begin{tikzpicture}
			\tikzset{double style/.append style = {draw=\tkzTabDefaultWritingColor,double=\tkzTabDefaultBackgroundColor,double distance=2pt}}
			\tkzTabInit[nocadre=false,lgt=1,espcl=1.6]
			{$x$ /0.7,$y'$ /0.7,$y$ /2}
			{$-\infty$,$-9$,$-4$,$1$,$+\infty$}
			\tkzTabLine{,+,$0$,-,d,-,$0$,+,}
			\tkzTabVar{-/$-\infty$,+/$-20$,-D+/$-\infty$/$+\infty$,-/$0$,+/$+\infty$}
		\end{tikzpicture}
	}
\loigiai{
}
\end{ex}

\begin{ex}
	\immini{Bảng biến thiên sau là của một trong bốn hàm số sau. Hỏi đó là hàm số nào?
		\choice
		{$y=\dfrac{x^2-3}{x-2}$}
		{\True $y=\dfrac{x^2-4x+2}{x-2}$}
		{$y=\dfrac{x^2-x}{x-2}$}
		{$y=\dfrac{x^2-4x+5}{x-2}$}}{
			\begin{tikzpicture}
				\tkzTabInit[nocadre=false,lgt=1,espcl=3]
				{$x$ /0.6,$y'$ /0.6,$y$ /2}
				{$-\infty$,$2$,$+\infty$}
				\tkzTabLine{,+,d,+,}
				\tkzTabVar{-/$-\infty$,+D-/$+\infty$/$-\infty$,+/$+\infty$}
			\end{tikzpicture}
	}
\loigiai{
}
\end{ex}

\begin{ex}
	\immini{Đồ thị hình bên là của một trong bốn hàm số sau. Hỏi đó là hàm số nào?
		\choice
		{$y=\dfrac{x^2+x-1}{x-1}$}
		{\True $y=\dfrac{x^{2}-x+1}{x-1}$}
		{$y=\dfrac{x^2-4x-1}{-x+1}$}
		{$y=\dfrac{x^2-3x-1}{-x+1}$}}{
		\begin{tikzpicture}[line cap=butt,line join=miter,>=stealth,scale=0.4,font=\footnotesize]
			\tikzset{declare function={xmin=-3.5;xmax=4.7;ymin=-3.5;ymax=6;},
				smooth,samples=450}
			\draw[->] (xmin,0)--(xmax,0) node[shift={(0:7pt)}]{$ x $};
			\draw[->] (0,ymin-.2)--(0,ymax) node[shift={(90:7pt)}]{$ y $};
			\fill (0,0) node[shift={(140:6pt)}]{$ O $};
			\clip (xmin,ymin) rectangle (xmax,ymax);
			\foreach \i in {-3,-2,2,3,4}{
				\draw(\i,1.5pt)--(\i,-1.5pt)node[below]{$\i$};}	
			\foreach \j in {-2,1,2,3,4,5}{
				\draw(-1.5pt,\j)--(1.5pt,\j) node[left]{$\j$};}
			\draw(-1.5pt,-1)--(1.5pt,-1)node[shift={(160:6.5pt)}]{$-1$};
			\draw(1,-1.5pt)--(1,1.5pt)node[shift={(-75:7pt)}]{$1$};
			\draw(-1,-1.5pt)--(-1,1.5pt)node[shift={(100:5pt)}]{$-1$};
			\def\f(#1){((#1)^2-(#1)+1)/((#1)-1)}
			\def\a{-1}
			\def\b{0}
			\def\c{0.5}
			\def\d{1.5}	
			\def\e{2}
			\def\g{3}	
			\pgfmathsetmacro\fa{\f(\a)}
			\pgfmathsetmacro\fb{\f(\b)}
			\pgfmathsetmacro\fc{\f(\c)}
			\pgfmathsetmacro\fd{\f(\d)}	
			\pgfmathsetmacro\fe{\f(\e)}
			\pgfmathsetmacro\fg{\f(\g)}	
			\draw[samples=100] plot[domain=-5.3:0.9] (\x,{\f(\x)});	
			\draw[samples=100] plot[domain=1.05:5.2] (\x,{\f(\x)});
			\draw[] (1,ymin)--(1,ymax) node [pos=0.95,sloped, above]{$x=1$};
			\draw[] (xmin,ymin)--(6,ymax) node [pos=0.08,sloped, above]{$y=x$};
	\end{tikzpicture}
	}
\loigiai{
}
\end{ex}

\begin{ex}
	\immini{Đồ thị hình bên là của một trong bốn hàm số sau. Hỏi đó là hàm số nào?
		\choice
		{$y=\dfrac{x^2-x}{x+1}$}
		{$y=\dfrac{x^2-3x}{x+1}$}
		{$y=\dfrac{x^2+1x+2}{x+1}$}
		{\True $y=\dfrac{-x^{2}}{x+1}$}}{
		\begin{tikzpicture}[line cap=butt,line join=miter,>=stealth,scale=0.35,font=\footnotesize]
			\tikzset{declare function={xmin=-6.2;xmax=4.8;ymin=-4.6;ymax=7.8;},
				smooth,samples=450}
			\draw[->] (xmin,0)--(xmax,0) node[shift={(0:7pt)}]{$ x $};
			\draw[->] (0,ymin)--(0,ymax) node[shift={(90:7pt)}]{$ y $};
			\fill (0,0) node[shift={(140:5pt)}]{$ O $};
			\clip (xmin,ymin-.7) rectangle (xmax,ymax);
			\foreach \i in {-2,2}{
				\draw(\i,1.5pt)--(\i,-1.5pt)node[below]{$\i$};}
			\foreach \j in {-2,2,4}{
				\draw(-1.5pt,\j)--(1.5pt,\j) node[right]{$\j$};}	
			\def\f(#1){(-(#1)^2)/((#1)+1)} % Hàm số
			\def\q(#1){(-(#1)+1)} % Tiệm cận xiên
			\def\a{0}
			\def\b{-2}	
			\pgfmathsetmacro\fa{\f(\a)}
			\pgfmathsetmacro\fb{\f(\b)}	
			\draw[samples=250] plot[domain=-7.4:-1.1] (\x,{\f(\x)});	
			\draw[samples=250] plot[domain=-0.9:15] (\x,{\f(\x)});
			\draw[] plot [domain=-7.4:7] (\x,{\q(\x)}) ;
			\draw[] (-1,ymin)--(-1,ymax) node[sloped,pos=0.9,below] {$x=-1$};
			\foreach \x/\y in {\a/\fa,\b/\fb}{	
				\draw[dashed] (\x,0)|-(0,\y);}
			\foreach \x/\y in {\a/\fa,\b/\fb}{
				\fill[white,draw=black] (\x,\y) circle (1pt);}
			\fill[white,draw=black] (-1,2) circle (1pt) node[text=black,shift = {(14pt,5pt)}] {$I $};
		\end{tikzpicture}
	}
\loigiai{
}
\end{ex}

\begin{ex}
	\immini{Đồ thị hình bên là của một trong bốn hàm số sau. Hỏi đó là hàm số nào?
		\choice
		{$y=\dfrac{x^2-x+4}{x+1}$}
		{$y=\dfrac{x^2-2x+3}{x+1}$}
		{\True $y=\dfrac{-x^2-x+2}{x+1}$}
		{$y=\dfrac{x^2+x-1}{x+1}$}}{
		\begin{tikzpicture}[>=stealth, scale=0.35, font=\footnotesize]
			\draw[->] (-5,0)--(4.4,0) node[below] {$x$};
			\draw[->] (0,-5)--(0,6) node[left] {$y$};
			\draw[domain=-0.6:4, smooth] plot (\x, {-(\x)^2-(\x)+2)/(\x+1)});
			\draw[domain=-5:-1.35, smooth] plot (\x, {-(\x)^2-(\x)+2)/(\x+1)});
			\draw[domain=-5:4, smooth] plot (\x, {-\x});
			\draw (-1,-5)--(-1,6);
			\draw[fill=black] (0,0) node[below left=-0.1] {$O$} circle (1.2pt);
			\draw[fill=black] (1,0) node[below] {$1$} circle (1.2pt);
			\draw[fill=black] (-1,0) node[below left] {$-1$} circle (1.2pt);
			\draw[fill=black] (3,0) node[above] {$3$} circle (1.2pt);
			\draw[fill=black] (0,2) node[below left] {$2$} circle (1.2pt);
			\draw[fill=black] (0,-2.5) node[left] {$-\dfrac{5}{2}$} circle (1.2pt);
			\draw[dashed] (3,0)--(3,-2.5)--(0,-2.5) ;
			\end{tikzpicture}
	}
\loigiai{
}
\end{ex}

\begin{ex}
	\immini{Đồ thị hình bên là của một trong bốn hàm số sau. Hỏi đó là hàm số nào?
		\choice
		{$y=\dfrac{x^2+3}{x-1}$}
		{\True 	$y=\dfrac{x^{2}+x-3}{x-1}$}
		{$y=\dfrac{x^2-2x+3}{-x+1}$}
		{$y=\dfrac{x^2+3}{-x+1}$}}{
		\begin{tikzpicture}[line cap=butt,line join=miter,>=stealth,scale=0.35,font=\footnotesize]
			\tikzset{declare function={xmin=-3.8;xmax=4.8;ymin=-3.6;ymax=7.8;},
				smooth,samples=450}
			\draw[->] (xmin,0)--(xmax,0) node[shift={(0:7pt)}]{$ x $};
			\draw[->] (0,ymin)--(0,ymax) node[shift={(90:7pt)}]{$ y $};
			\fill (0,0) node[shift={(140:5pt)}]{$ O $};
			\clip (xmin,ymin-.7) rectangle (xmax,ymax);
			\foreach \i in {-2,2}{
				\draw(\i,1.5pt)--(\i,-1.5pt)node[below]{$\i$};}
			\foreach \j in {-2,4}{
				\draw(-1.5pt,\j)--(1.5pt,\j) node[left]{$\j$};}	
			\draw(-1.5pt,2)--(1.5pt,2) node[right]{$2$};
			\def\f(#1){((#1)^2+(#1)-3)/((#1)-1)} % Hàm số
			\def\q(#1){((#1)+2)} % Tiệm cận xiên
			\def\a{0}	
			\pgfmathsetmacro\fa{\f(\a)}	
			\draw[samples=250] plot[domain=-7.4:0.9] (\x,{\f(\x)});	
			\draw[samples=250] plot[domain=1.1:15] (\x,{\f(\x)});
			\draw[] plot [domain=-7.4:7] (\x,{\q(\x)});
			\draw[] (1,ymin)--(1,ymax) node[rotate=180 ,pos=0.9,sloped,above] {$x=1$};
			\foreach \x/\y in {\a/\fa}{	
				\draw[dashed] (\x,0)|-(0,\y);}
			\foreach \x/\y in {\a/\fa}{
				\fill[white,draw=black] (\x,\y) circle (1pt);}
			;
			\node at (2.6,5.4) [rotate=45,right,fill=white]{$y=x+2$};
		\end{tikzpicture}
	}
\loigiai{
}
\end{ex}

\Closesolutionfile{ans}

\ind{PHẦN II.} \inden{Câu trắc nghiệm đúng sai. Trong mỗi ý a), b), c), d) ở mỗi câu, học sinh chọn đúng hoặc sai.}\\
\Opensolutionfile{ans}[ans/2D1-B4-d3-2]
\begin{ex}
	\immini{Cho hàm số $y=\dfrac{ax^2+bx+c}{mx+n}$ có đồ thị như hình bên.
	\choiceTF
	{Tập xác định của hàm số là $\mathbb{R}\backslash\{1\}$}
	{\True Hàm số nghịch biến trên khoảng $(-\infty;2)$ và $(2;+\infty)$}
	{\True Điểm $I(2;1)$ là tâm đối xứng của đồ thị}
	{\True Hệ số $a$ và $m$ trái dấu}}{
\begin{tikzpicture}[line join=round, line cap=round,>=stealth,x=0.5cm, y=0.5cm]
	\tikzset{every node/.style={scale=0.9}}
	\draw[->] (-4.1,0)--(6.1,0) node[below left] {$x$};
	\draw[->] (0,-6.1)--(0,6.1) node[below left] {$y$};
	\draw (0,0) node [below left] {\scriptsize$O$};
	\foreach \x/\nx in {-4/-4,-2/-2,1/ ,2/ ,4/4}
	\draw[thin] (\x,1pt)--(\x,-1pt) node [below] {$\nx$};
	\draw (2,0) node[above left]{\scriptsize$2$};
	\foreach \y/\ny in {-4/-4,-2/-2,-1/-1,2/2,4/4}
	\draw[thin] (1pt,\y)--(-1pt,\y) node [left] {$\ny$};
	\draw[dashed,thin](2,0)--(2,-1)--(0,-1);
	\draw[dashed,thin] (2,-6)--(2,6);
	\begin{scope}
		\clip (-4,-6) rectangle (6,6);
		\draw[samples=200,domain=-4:1.99,smooth,variable=\x] plot (\x,{(-1*((\x)^2)+3*(\x)+-1)/(1*(\x)+-2)});
		\draw[samples=200,domain=2.01:6,smooth,variable=\x] plot (\x,{(-1*((\x)^2)+3*(\x)+-1)/(1*(\x)+-2)});
		\draw[dashed,thin] (-6.1,7.1)--(6.1,-5.1);
	\end{scope}
\end{tikzpicture}}
\loigiai{
\begin{enumerate}[a)]
	\item 
	\item
	\item
	\item
\end{enumerate}}
\end{ex}

\begin{ex}
	\immini{Cho hàm số $y=\dfrac{ax^2+bx+c}{x+n}$ có đồ thị như hình bên.
		\choiceTF
		{\True Tập xác định của hàm số là $\mathbb{R}\backslash\{1\}$}
		{Điểm $I(1;2)$ là tâm đối xứng của đồ thị}
		{$a+2b=4$}
		{\True Đồ thị qua điểm $(2;10)$ khi $c=4$}}{
		\begin{tikzpicture}[line cap=round, line join=round,font=\footnotesize,>=stealth, scale=1,x=0.5cm, y=0.25cm]
			\tikzset{label style/.style={font=\footnotesize}}
			\draw[->] (-4,0)--(6,0) node[below] {$x$};
			\draw[->] (0,-8)--(0,15) node[left] {$y$};
			\draw[smooth, samples=100] plot[domain=-4:0.5] (\x, {  (2*(\x)^2-(\x)+4)/(\x-1) });
			\draw[smooth, samples=100] plot[domain=1.5:6] (\x, {  (2*(\x)^2-(\x)+4)/(\x-1) });
			\draw[dashed] (1,-8)node [right]{$x=1$}--(1,15) 
			plot[domain=-4:6](\x, {2*(\x)+1}) node[rotate=45,below]{$y=2x+1$};
	\end{tikzpicture}}
\loigiai{
	\begin{enumerate}[a)]
		\item
		\item
		\item
		\item
\end{enumerate}}
\end{ex}

\Closesolutionfile{ans}
% \begin{dang}{Sự tương giao của hai đồ thị}
	\begin{enumerate}[\iconCH]
		\item \indamm{Xác định tọa độ giao điểm của hai đồ thị $y=f(x)$ và $y=g(x)$:}
		\begin{listEX}[1]
			\item [\ding{172}] Giải phương trình hoành độ giao điểm $f(x)=g(x)$, tìm các nghiệm $x_0 \in \mathscr{D}_f \cap \mathscr{D}_g$.
			\item [\ding{173}] Với $x_0$ vừa tìm, thay vào một trong hai hàm số ban đầu để tìm $y_0$.
			\item [\ding{174}] Kết luận giao điểm $(x_0;y_0)$.
		\end{listEX}
		\item \indamm{Ứng dụng đồ thị để biện luận nghiệm phương trình:}
		\immini{
		\begin{enumerate}[]
			\item Xét phương trình $f(x)=m$, với $m$ là tham số. Nghiệm của phương trình này có thể coi là hoành độ giao điểm của đồ thị $y=f(x)$ (cố định) với đường thẳng $y=m$ (nằm ngang).
			\item Từ đó, để biện luận nghiệm phương trình $f(x)=m$, ta có thể thực hiện các bước như sau:
				\begin{itemize}
					\item [$\bullet$] Lập bảng biến thiên của hàm số $y=f(x)$ trên miền xác định mà đề bài yêu cầu.
					\item [$\bullet$] Tịnh tiến đường thẳng $y=m$ theo hướng "\textit{lên, xuống}". Quan sát số giao điểm để quy ra số nghiệm tương ứng.
				\end{itemize}
		\end{enumerate}}{
	\begin{tikzpicture}[scale=0.7, font=\footnotesize, line join=round, line cap=round, >=stealth]
		\draw[->] (-2.5,0) -- (3,0) node[below]{ $x$};
		\draw[->] (0,-1.5) -- (0,4) node[right]{ $y$};
		\draw[blue,line width=1pt,smooth,samples=100,domain=-2.09:2.1] plot(\x,{-(\x)^3+3*(\x)+1})node[right]{\footnotesize $y=f(x)$};
		\draw[dashed](-2,3)--(0,3)node[above left]{\footnotesize $3$}--(1,3);
		\draw[dashed](-1,-1)--(0,-1)node[right]{\footnotesize $-1$};
		\draw[fill=black] (-2,3) circle(1.5pt) (-1,-1) circle(1.5pt) (1,3) circle(1.5pt);
		\draw[fill=red] (0.3473,2) circle(2.5pt) (1.5321,2) circle(2.5pt) (-1.8794,2) circle(2.5pt);
		\draw[line width=1pt,red](-2.5,2)--(3,2)node[above]{\footnotesize $y=m$};
\end{tikzpicture}}
	\end{enumerate}
\end{dang}
\boxmini{BÀI TẬP TỰ LUẬN}
\begin{vd}
	Xác định tọa độ giao điểm của hai đồ thị hàm số sau:
	\begin{tasks}(2)
		\task $y=x^3-2x^2+x-1$ và $y=1-2x$;
		\task $y=\dfrac{x+8}{x-2}$ và $y=x+2$.
	\end{tasks}
\loigiai{
\begin{enumerate}[a)]
	\item Xét phương trình hoành độ giao điểm\\
	\centerline{$x^3-2x^2+x-1=1-2x\Leftrightarrow x^3-2x^2+3x-2=0\Leftrightarrow(x-1)\left(x^2-x+2\right)=0\Leftrightarrow x=1$.}\\
	Do đó $2$ đồ thị  hàm số có giao điểm là $(1;-1)$.
	\item Với điều kiện $x\ne 2$ ta có\\
	Phương trình hoành độ giao điểm $x+2=\dfrac{x+8}{x-2}\Leftrightarrow x^2-4=x+8 \Leftrightarrow x^2 -x -12 =0 \Leftrightarrow \hoac{&x=3\\&x=-4.}$\\
	Từ đó được $A(3;5)$ và $B(-4;-2)$.
\end{enumerate}}
\end{vd}

\begin{vd}
	Tìm tập hợp các giá trị thực của tham số $m$  để đồ thị hàm số $y=(x-2)(x^2+mx+m^2-3)$ cắt trục hoành tại ba điểm phân biệt.
	\loigiai{
		Đồ thị hàm số đã cho cắt trục hoành tại ba điểm phân biệt khi phương trình $$(x-2)(x^2+mx+m^2-3)=0$$ có $3$ nghiệm phân biệt hay phương trình $x^2+mx+m^2-3=0$ có $2$ nghiệm phân biệt khác $2$ 
		$$\Leftrightarrow \heva{\Delta& =-3m^2+12>0\\m&^2+2m+1\ne 0}\Leftrightarrow \heva{-&2<m<2\\m&\ne -1}.$$
	}
\end{vd}

\begin{vd}
	Tìm tham số m để phương trình $x^3 - 3x + 2-m=0$ có ba nghiệm phân biệt.
	\loigiai{
		Phương trình tương đương với  $x^3 - 3x + 2=m$.
		\immini {
			\begin{itemize}
				\item [$\bullet$] Số nghiệm của phương trình bằng số giao điểm của đồ thị $y=x^3 - 3x + 2$ với đường thẳng $y=m$ (nằm ngang).
				\item [$\bullet$] Đồ thị hàm số $ y = x^3 - 3x +2 $ như hình bên. Để đường thẳng $ y = m $ cắt đồ thị tại 3 điểm phân biệt khi và chỉ khi $ 0 < m < 4. $
			\end{itemize}
		Vậy $ 0 < m < 4. $
			}
		{\begin{tikzpicture}[>=stealth,scale=0.6,every node/.style={scale=0.8}]
				\draw[->,black] (-2.5,0) -- (3,0)node[above left] {$x$};
				từ tọa độ (-1.5,0) đến tạo độ (3.5,0) ghi tên x ở trên bên trái
				\draw[->,black] (0,-1) -- (0,4.5)node[right] {$y$};
				\foreach \x in {-1,1,2}
				\draw[shift={(\x,0)}] (0pt,1pt) -- (0pt,-1pt) node[below] {\footnotesize $\x$};
				\foreach \y in {2,3,4}
				\draw[shift={(0,\y)}] (1pt,0pt) -- (-1pt,0pt) node[right] {\footnotesize $\y$};
				\node at (0,0) [below right] {\footnotesize $O$};
				\draw[smooth,samples=100,domain=-2.1:2.02] plot(\x,{(\x)^3-3*(\x)+2});
				\draw [dashed] (-1,0)--(-1,4)--(0,4);
				\draw [blue] (-2.4,2.5)--(2.9,2.5);
				\node at (2,2.5) [below right] {\footnotesize $y=m$};
			\end{tikzpicture}
		}
	}
\end{vd}

\boxmini{BÀI TẬP TRẮC NGHIỆM}
\setcounter{ex}{0}
\Opensolutionfile{ans}[ans/2D1-B4-d4-1]
\begin{ex}%[2D1B5-4]
	Đường thẳng $y=x-1$ cắt đồ thị hàm số $y=x^3-x^2+x-1$ tại hai điểm. Tìm tổng tung độ các giao điểm đó.
	\choice
	{$-3 $}
	{$2 $}
	{$0 $}
	{\True $-1 $}
	\loigiai{Phương trình hoành độ giao điểm
		$$x^3-x^2+x-1=x-1\Leftrightarrow \left[ \begin{aligned} &x=1 \Rightarrow y=0\\ &x=0 \Rightarrow y=-1.\end{aligned} \right.$$
		Tổng tung độ các giao điểm là $0+(-1)=-1$.
	}
\end{ex}

\begin{ex}%[2D1Y5-4]
	Số giao điểm của đồ thị hàm số $y=(x-1)(x^2-3x+2)$ và trục hoành là
	\choice
	{$0$}
	{$1$}
	{\True $2$}
	{$3$}
	\loigiai{
		Phương trình $y=0$ có hai nghiệm là $x=1$ và $x=2$.
	}
\end{ex}

\begin{ex}
	Đồ thị hàm số $y=x^3-3x^2+2x-1$ cắt đồ thị hàm số $y=x^2-3x+1$ tại hai điểm phân biệt $A,B$. Tính độ dài $AB$.
	\choice
	{$AB=3$}
	{$AB=2\sqrt2$}
	{$AB=2$}
	{\True $AB=1$}
	\loigiai{
		Phương trình hoành độ giao điểm $$x^3-3x^2+2x-1=x^2-3x+1\Leftrightarrow x^3-4x^2+5x-2=0\Leftrightarrow \hoac{& x=1\\& x=2}\Rightarrow \hoac{& y=-1\\& y=-1}.$$
		Không mất tính tổng quát, ta giả sử $A(1;-1),B(2;-1)$. Suy ra $\vec{AB}=(1;0)\Rightarrow AB=1$.
	}
\end{ex}

\begin{ex}
	Đồ thị của hàm số $ y = \dfrac{x - 1}{x+1} $ cắt hai trục $ Ox $ và $ Oy $ tại $ A $ và $ B $. Khi đó diện tích của tam giác $ OAB $ (với $ O $ là gốc tọa độ) bằng
	\choice
	{$ 1 $}
	{$ \dfrac{1}{4} $}
	{$ 2 $}
	{\True $ \dfrac{1}{2} $}
	\loigiai{
		Ta có $ A(1;0), B(0;-1) $. Diện tích $ S_{\triangle OAB} = \dfrac{OA\cdot OB}{2} = \dfrac{1}{2} $.
	}	
\end{ex}

\begin{ex}
	Biết đường thẳng $y=x-2$ cắt đồ thị hàm số $ y=\dfrac{x}{x-1} $ tại $ 2 $ điểm phân biệt $ A, $ $ B. $ Tìm hoành độ trọng tâm tam giác $OAB$ với $O$ là gốc tọa độ.
	\choice
	{$ \dfrac{2}{3} $}
	{$ 2 $}
	{\True $ \dfrac{4}{3} $}
	{$ 4 $}
	\loigiai{
		
		Xét phương trình hoành độ giao điểm $ x-2=\dfrac{x}{x-1} $ (Điều kiện $ x\neq 1 $).
		
		$ \Rightarrow (x-2)(x-1)=x\Leftrightarrow x^2-4x+2=0 \,  (1).$
		
		Khi đó $ A(x_1;x_1-2), $ $ B(x_2;x_2-2) $  với $ x_1, x_2 $ là $ 2 $ nghiệm của phương trình $ (1) $ thỏa mãn 
		
		$ \heva{&x_1+x_2=4\\&x_1.x_2=2}. $ Gọi $ G\left(x_G;y_G\right) $ là trọng tâm tam giác $ OAB. $
		
		$ \Rightarrow x_G=\dfrac{0+x_1+x_2}{3}=\dfrac{4}{3}. $
		
	}
\end{ex}

\begin{ex}%[2D1B5]
	Gọi $ M, N $ là giao điểm của đường thẳng $ y = x + 1 $ và đường cong $ y = \dfrac{2x+4}{x-1} $. Tìm hoành độ trung điểm của đoạn thẳng $ MN. $
	\choice
	{$ x = -1 $}
	{\True $ x = 1 $}
	{$ x = -2 $}
	{ $ x = 2$}
	\loigiai
	{Xét phương trình hoành độ giao điểm $ x+ 1 = \dfrac{2x +4}{x-1} \Leftrightarrow \heva{&x \ne 1\\ &x^2 - 2x- 5 = 0}$\\
		$ \Rightarrow x_M + x_N = 2 \Rightarrow x_I = \dfrac{x_M+x_N}{2} = 1.$}
\end{ex}

\begin{ex}
	Cho hàm số $y=\dfrac{2x}{x+1}$ có đồ thị $(C)$. Gọi $A,B$ là giao điểm của đường thẳng $d:y=x$ với đồ thị $(C)$. Tính độ dài đoạn $AB$.
	\choice
	{\True $AB=\sqrt{2}$}
	{ $AB=\dfrac{\sqrt{2}}{2}$}
	{$AB=1$}
	{ $AB=2$}
	\loigiai{
		Phương trình hoành độ giao điểm\\
		$\dfrac{2x}{x+1}=x,\left({x\ne -1}\right)\Rightarrow x^2-x=0\Rightarrow \left[{\begin{aligned}&{x=0\Rightarrow y=0\Rightarrow A\left({0;0}\right)} \\ &{x=1\Rightarrow y=1\Rightarrow B\left({1;1}\right)} \\ \end{aligned}}\right.$\\
		Vậy $AB=\sqrt{2}$.
	}
\end{ex}

\begin{ex}%[2D1B5-3]
	\immini
	{
		Cho hàm số $y=f(x)$ có đồ thị như hình vẽ. Số nghiệm của phương trình $2f(x)-3=0$ là
		\haicot
		{$2$}
		{$1$}
		{$0$}
		{\True $3$}
	}
	{\begin{tikzpicture}[smooth,samples=300,scale=0.5,>=stealth]
			\draw[->] (-2.3,0)--(3,0) node[below]{$x$};
			\draw[->] (0,-1.5)--(0,4) node[right]{$y$};
			\draw (0,0) node[above left]{$O$};
			\draw[thick,domain=-2.05:2.05] plot(\x,{1*((\x)^3)-3*(\x)+1});
			\draw[fill=black] (0,3) circle(1pt) (-1,3) circle(1.5pt) (0,-1) circle(1pt) (1,-1) circle(1.5pt);
			\draw[dashed] (1,-1)--(0,-1)node[left]{$-1$} (-1,3)--(0,3)node[right]{$3$};
		\end{tikzpicture}
	}
	
	\loigiai{
		Ta có $2f(x)-3=0\Leftrightarrow f(x)=\dfrac{3}{2}$.\\
		Từ đồ thị suy ra phương trình có $3$ nghiệm phân biệt.
	}
\end{ex}


\begin{ex}%[2D1B5-3]
	\immini
	{Cho hàm số $f(x)=ax^3 +bx^2 +cx +d$ $(d\ne 0)$ có đồ thị như hình vẽ bên. Số nghiệm của phương trình $3f(x) -1 =0$ bằng
		\haicot
		{$0$}
		{\True $1$}
		{$2$}
		{$3$}
	}
	{\hspace{1cm}\begin{tikzpicture}[line join=round, line cap=round,>=stealth,scale=0.5]
			\tikzset{label style/.style={font=\footnotesize}}
			\draw[->] (-1.1,0)--(3.1,0) node[above right] {$x$};
			\draw[->] (0,-2.1)--(0,5.1) node[right] {$y$};
			\draw (0,0) node [above left] {$O$};
			\foreach \x in {1,2}
			\draw[thin] (\x,1pt)--(\x,-1pt) node [above] {$\x$};
			\foreach \y in {-1,4}
			\draw[thin] (1pt,\y)--(-1pt,\y) node [left] {$\y$};
			%\draw[dashed,thin](-1,0)--(-1,3)--(0,3);
			\draw[dashed,thin](1,0)--(1,-1)--(0,-1);
			\begin{scope}
				\clip (-1,-2) rectangle (3,5);
				\draw[samples=200,domain=-1:3,smooth,variable=\x] plot (\x,{-2*((\x)^3)+9*((\x)^2)+-12*(\x)+4});
			\end{scope}
	\end{tikzpicture}}
	\loigiai{
		Ta có $3f(x)-1=0 \Leftrightarrow f(x) = \dfrac{1}{3}$.\\
		Khi đó số giao điểm của đồ thị $y=f(x)$ và đường thẳng $y=\dfrac{1}{3}$ chính là số nghiệm của phương trình $3f(x) -1=0$. Dựa vào đồ thị ta có số nghiệm của phương trình là 1.}
\end{ex}

\begin{ex}%[2D1B5-3]
	\immini{Cho hàm số $y = f(x)$ có bảng biến thiên như sau. Số giao điểm của đồ thị hàm số $y = f(x)$ với trục hoành là
		\haicot
		{$ 1$}
		{$ 0$}
		{$ 2  $}
		{\True $ 3 $}}{
		\begin{tikzpicture}
			\tkzTabInit[lgt=1,espcl=1.8]
			{$x$/0.6, $y’$/0.6, $y$/1.6}
			{$-\infty$,$0$,$1$,$+\infty$}
			\tkzTabLine{ ,-,$0$,+,$0$,-, }
			\tkzTabVar{+/$+\infty$,-/$-1$,+/$3$,-/$-\infty$}
	\end{tikzpicture}}
	\loigiai{
		Dựa vào bảng biến thiên thì đồ thị hàm số $y = f(x)$ và trục hoành có $3$ điểm chung.	
	}
\end{ex}

\begin{ex}%[2D1B5-3]
	\immini{Cho hàm số $y=f(x)$ liên tục trên $(-\infty;+\infty)$ và có bảng biến thiên như hình bên. Số nghiệm thực của phương trình $2\big|f(x)\big|=7$ bằng
		\choice
		{$3$}
		{\True $2$}
		{$4$}
		{$2$}	
		
		
	}{\begin{tikzpicture}[>=stealth,line join=round,line cap=round,font=\footnotesize,scale=.8]
			\begin{scope}[xscale=1.15,yscale=0.8]
				\begin{scope}[shift={(-0.5,0.5)}]
					\def\a{8}
					\def\b{4}
					\draw (0,0)rectangle +(\a,-\b)
					(1,0)--+(-90:\b)
					(0,-1)--+(0:\a)
					(0,-2)--+(0:\a)
					;
				\end{scope}
				\draw
				(0,0)node{$x$}++(0:1)node{$-\infty$}++(0:2)node{$1$}++(0:2)node{$2$}
				++(0:2)node{$+\infty$}
				(0,-1)node{$y'$}	++(0:2)node{$+$}++(0:1)node{$0$}++(0:1)node{$-$}++(0:1)node{$0$}++(0:1)node{$+$}
				(0,-2.5)node{$y$}
				(1,-3.2) node (A)  {$-\infty$} 
				(3,-2)node (B) {$5$} 
				(5,-2.8) node (C){$4$} 
				(7,-1.9)node (D){$+\infty$} 
				;
				\draw[->] (A)--(B);
				\draw[->] (B)--(C);
				\draw[->] (C)--(D);
			\end{scope}	
	\end{tikzpicture}}
	
	\loigiai{
		
	}
\end{ex}

\begin{ex}%[2D1K5-3]
	\immini{Cho hàm số $y=f(x)$ liên tục trên $\mathbb{R}\setminus\{0\}$ và có bảng biến thiên như hình bên. Hỏi phương trình $3|f(x)|-10=0$ có bao nhiêu nghiệm?
		\choice
		{$2$ nghiệm}
		{$4$ nghiệm}
		{\True $3$ nghiệm}
		{$1$ nghiệm}
	}{
		\begin{tikzpicture}
			\tikzset{double style/.append style = {draw=\tkzTabDefaultWritingColor,double=\tkzTabDefaultBackgroundColor,double distance=2pt}}
			\tkzTabInit[nocadre=false,lgt=1.2,espcl=1.7,deltacl=0.6]
			{$x$ /.6,$f'(x)$ /.6,$f(x)$ /1.7}{$-\infty$,$0$,$1$,$+\infty$}
			\tkzTabLine{,-,d,-,0,+,}
			\tkzTabVar{+/$2$,-D+/$-\infty$/+$\infty$,-/$3$,+/$+\infty$}
	\end{tikzpicture}}
	\loigiai
	{
		Từ bảng biến thiên đề bài, ta có bảng biến thiên của hàm số $y=|f(x)|$ như sau
		\begin{center}
			\begin{tikzpicture}
				\tkzTabInit[nocadre=false,lgt=1.3,espcl=2.5,deltacl=0.6]
				{$x$ /.6,$f'(x)$ /.6,$|f(x)|$ /2}{$-\infty$,,$0$,$1$,$+\infty$}
				\tkzTabLine{,,-,,d,-,0,+,}
				\tkzTabVar{+/$2$,-/$0$,+D+/$+\infty$/+$\infty$,-/$3$,+/$+\infty$}
			\end{tikzpicture}
		\end{center}
		Ta có $3|f(x)|-10=0\Leftrightarrow |f(x)|=\dfrac{10}{3}.\qquad(1)$\\
		Số nghiệm của phương trình (1) bằng số giao điểm của đồ thị $y=|f(x)|$ và đường thẳng $y=3$.\\
		Dựa vào bảng biến thiên trên, suy ra phương trình (1) có $3$ nghiệm. 
	}
\end{ex}

\begin{ex}%[2D1K5-3]
	\immini{Cho hàm số $y = f(x)$ xác định và liên tục trên $\mathbb{R}$, có bảng biến thiên như sau. Số nghiệm của phương trình $2[f(x)]^2- 3 f(x)+ 1 = 0$ là
		\haicot
		{$2$}
		{\True $3$}
		{$6$}
		{$0$}}
	{\begin{tikzpicture}[scale=0.8]
			\tkzTabInit[espcl=2.3,lgt=1.2,deltacl=0.6]
			{$x$/0.6,$y'$/0.6,$y$/2}
			{$-\infty$,$-1$,$1$,$+\infty$}
			\tkzTabLine{,+,0,-,0,+,}
			\tkzTabVar{-/$1$,+/$3$,-/$\dfrac{1}{3}$,+/$1$}
	\end{tikzpicture}}
	\loigiai{
		Ta có $ 2[f(x)]^2- 3 f(x)+ 1 = 0\Leftrightarrow \left[\begin{array}{l}{f(x)= 1}\\{f(x)= \dfrac{1}{2}.}\end{array}\right.$\\
		Phương trình $f(x)= 1$ có duy nhất nghiệm $ x_0 $.\\
		Phương trình $f(x)= \dfrac{1}{2}$ có $2$ nghiệm phân biệt khác $x_{0}$.  Vậy phương trình có ba nghiệm.
	}
\end{ex}

\begin{ex}
	\immini{Cho hàm số $f(x)$ có bảng biến thiên như hình bên. Tìm tất cả các giá trị thực của tham số $m$ để phương trình $f(x)=m+1$ có ba nghiệm thực phân biệt.
		\choice
		{$-3\le m \le 3$}
		{$-2\le m \le 4$}
		{$-2<m<4$}
		{\True $-3<m<3$}
	}{
		\begin{tikzpicture}
			\tkzTabInit[nocadre=false,lgt=1,espcl=1.9,deltacl=0.6]
			{$x$ /0.6, $y'$ /0.6, $y$ /1.6}
			{$-\infty$,$-1$,$3$,$+\infty$}
			\tkzTabLine{,+,$0$,-,$0$,+,}
			\tkzTabVar{-/$-\infty$,+/$4$,-/$-2$,+/$+\infty$}
	\end{tikzpicture}}
	\loigiai{
		Dựa vào bảng biến thiên phương trình $f(x)=m+1$ có ba nghiệm thực phân biệt khi
		\begin{center}
			$-2<m+1<4 \Leftrightarrow -3<m<3$.
		\end{center}
	}
\end{ex}

\begin{ex}
	\immini{Cho hàm số $y=f(x)$ có bảng biến thiên như hình bên. Phương trình $f(4x-x^2)-2=0$ có bao nhiêu nghiệm thực?
		\choice 
		{$2$}
		{$6$}
		{$0$}
		{\True $4$}
	}{
		\begin{tikzpicture}
			\tkzTabInit[nocadre=false,lgt=1.2,espcl=2.2,deltacl=0.6]
			{$x$ /0.6,$y’$ /0.6,$y$ /1.6}
			{$-\infty$ ,$0$ , $4$, $+\infty$}
			\tkzTabLine{,-,0,+,0,-}
			\tkzTabVar{+/ $+\infty $ / , -/ $-1$ /,+/ $3$/ , -/ $-\infty$ /}  
	\end{tikzpicture} }
	\loigiai{ 
		Đặt $t=4x-x^2$. Khi đó $t=-(x-2)^2+4 \leq 4$.\\
		Từ mỗi giá trị $t<4$ ta tìm được hai giá trị $x$. Với $t=4$ ta tìm được $x=2$.\\
		Từ bảng biến thiên, ta thấy phương trình $f(t)=2 \Leftrightarrow \left [ \begin{aligned} &t=\alpha \in (-\infty;0)\\ 
			&t=\beta \in (0;4)  \\
			&t=\gamma \in  (4;+\infty)  \end{aligned} \right.$\\
		Vậy phương trình $f(4x-x^2)-2=0$ có $4$ nghiệm. 
	}   
\end{ex}

\Closesolutionfile{ans}


%%Bài 5. Ứng dụng TT
% \setcounter{section}{4}
\section{ỨNG DỤNG ĐẠO HÀM VÀ KHẢO SÁT HÀM SỐ ĐỂ GIẢI QUYẾT MỘT SỐ BÀI TOÁN THỰC TIỄN}
\subsection{LÝ THUYẾT CẦN NHỚ}
\subsubsection{Tốc độ thay đổi của một đại lượng}
Ta có đạo hàm $f'(a)$ là tốc độ thay đổi tức thời của đại lượng $y=f(x)$ đối với $x$ tại điểm $x=a$. Dưới đây, chúng ta xem xét một số ứng dụng của ý tưởng này đối với vật lí, hoá học, sinh học và kinh tế: 
\begin{itemize}
	\item Nếu $s=s(t)$ là hàm vị trí của một vật chuyển động trên một đường thẳng thì $v=s'(t)$ biểu thị vận tốc tức thời của vật (tốc độ thay đổi của độ dịch chuyển theo thời gian). Tốc độ thay đổi tức thời của vận tốc theo thời gian là gia tốc tức thời của vật:
	$$
	a(t)=v'(t)=s''(t).
	$$
	\item Nếu $C=C(t)$ là nồng độ của một chất tham gia phản ứng hoá học tại thời điểm $t$, thì $C'(t)$ là tốc độ phản ứng tức thời (tức là độ thay đổi nồng độ) của chất đó tại thời điểm $t$.
	\item Nếu $P=P(t)$ là số lượng cá thể trong một quần thể động vật hoặc thực vật tại thời điểm $t$, thì $P'(t)$ biểu thị tốc độ tăng trưởng tức thời của quần thể tại thời điểm $t$.
	\item  Nếu $C=C(x)$ là hàm chi phí, tức là tổng chi phí khi sản xuất $x$ đơn vị hàng hoá, thì tốc độ thay đổi tức thời $C'(x)$ của chi phí đối với số lượng đơn vị hàng được sản xuất được gọi là chi phí biên.
	\item Về ý nghĩa kinh tế, chi phí biên $C'(x)$ xấp xỉ với chi phí để sản xuất thêm một đơn vị hàng hoá tiếp theo, tức là đơn vị hàng hoá thứ $x+1$ (xem SGK Toán 11 tập hai, trang 87, bộ sách Kết nối tri thức với cuộc sống). 
\end{itemize}
\subsubsection{Bài toán tối ưu hóa}
Một trong những ứng dụng phổ biến nhất của đạo hàm là cung cấp một phương pháp tổng quát, hiệu quả để giải những bài toán tối ưu hoá. Trong mục này, chúng ta sẽ giải quyết những vấn đề thường gặp như tối đa hoá diện tích, khối lượng, lợi nhuận, cũng như tối thiểu hoá khoảng cách, thời gian, chi phí.\\
Khi giải những bài toán như vậy, khó khăn lớn nhất thường là việc chuyển đổi bài toán thực tế cho bằng lời thành bài toán tối ưu hoá toán học bằng cách thiết lập một hàm số phù hợp mà ta cần tìm giá trị lớn nhất hoặc giá trị nhỏ nhất của nó, trên miền biến thiên phù hợp của biến số.\\
Quy trình giải một số bài toán tối ưu hoá  đơn giản:
\begin{itemize}
	\item[\iconCH]\indamm{Bước 1.} Xác định đại lượng Q mà ta cần làm cho giá trị của đại lượng ấy lớn nhất hoặc nhỏ nhất và biểu diễn nó qua các đại lượng khác trong bài toán.
	
	\item[\iconCH]\indamm{Bước 2.}  Chọn một đại lượng thích hợp nào đó, kí hiệu là $x$, và biểu diễn các đại lượng khác ở \indamm{Bước 1} theo $x$. Khi đó, đại lượng $Q$ sẽ là hàm số của một biến $x$. Tìm tập xác định của hàm số $Q=Q(x)$.
	
	\item[\iconCH]\indamm{Bước 3.}  Tìm giá trị lớn nhât hoặc giá trị nhỏ nhất của hàm số $Q=Q(x)$ bằng các phương pháp đã biết và kết luận.
\end{itemize}

\subsection{PHÂN LOẠI VÀ PHƯƠNG PHÁP GIẢI TOÁN}
\begin{dang}{Bài toán về tốc độ thay đổi của một đại lượng}
\end{dang}
\begin{vd}
	Khi bỏ qua sức cản của không khí, độ cao (mét) của một vật được phóng thẳng đứng lên trên từ điểm cách mặt đất $2$ m với vận tốc ban đầu $24{,}5$ m/s là $h(t)=2+24{,}5t-4{,}9t^2$ (theo Vật lí đại cương, NXB Giáo dục Việt Nam, $2016$).
	\begin{enumerate}
		\item Tìm vận tốc của vật sau $2$ giây.
		\item Khi nào vật đạt độ cao lớn nhất và độ cao lớn nhất đó là bao nhiêu?
		\item Khi nào thì vật chạm đất và vận tốc của vật lúc chạm đất là bao nhiêu?
	\end{enumerate}
	\loigiai{
		\begin{enumerate}
			\item Theo ý nghĩa cơ học của đạo hàm, vận tốc của vật là $v=h'(t)=24{,}5-9{,}8t$ m/s.\\				
			Do đó, vận tốc của vật sau $2$ giây là $v(2)=24{,}5-9{,}8\cdot 2=4{,}9$ m/s.
			\item Vì $h(t)$ là hàm số bậc hai có hệ số $a=-4{,}9< 0$ nên $h(t)$ đạt giá trị lớn nhất tại $t=-\dfrac{b}{2a}=\dfrac{24{,}5}{2\cdot 4{,}9}=2{,}5$ (giây). Khi đó, độ cao lớn nhất của vật là $h(2{,}5)=32{,}625$ m.
			\item Vật chạm đất khi độ cao bằng 0, tức là $h=2+24{,}5t-4{,}9t^2=0$, hay $t \approx 5{,}08$ (giây).\\
			Vận tốc của vật lúc chạm đất là $v(5{,}08)=24{,}5-9{,}8\cdot 5{,}08=-25{,}284$ m/s.\\
			Vận tốc âm chứng tỏ chiều chuyển động của vật là ngược chiều dương (hướng lên trên) của trục đã chọn (khi lập phương trình chuyển động của vật).
		\end{enumerate}
	}
\end{vd}

\begin{vd}
	Xét phản ứng hóa học tạo ra chất $C$ từ hai chất $A$ và $B$: $A+B\longrightarrow C$. Giả sử nồng độ của hai chất $A$ và $B$ bằng nhau $[A]=[B]=a$ (mol/l). Khi đó, nồng độ của chất $C$ theo thời gian $t$ ($t>0$) được cho bởi công thức: $[C]=\dfrac{a^2Kt}{aKt+1}$ (mol/l), trong đó $K$ là hằng số dương.
	\begin{enumerate}
		\item Tìm tốc độ phản ứng ở thời điểm $t>0$.
		\item Chứng minh nếu $x=[C]$ thì $x'(t)=K(a-x)^2$.
		\item Nêu hiện tượng xảy ra với nồng độ các chất khi $t\longrightarrow +\infty$.
		\item Nêu hiện tượng xảy ra với tốc độ phản ứng khi $t\longrightarrow +\infty$.
	\end{enumerate}
	\loigiai{
		\begin{enumerate}
			\item Tìm tốc độ phản ứng ở thời điểm $t>0$.\\
			Tốc độ của phản ứng là đạo hàm của $[C]=\dfrac{a^2Kt}{aKt+1}$ theo biến $t$. Do đó 
			\[[C]^\prime =\left(\dfrac{a^2Kt}{aKt+1}\right)^\prime=\dfrac{a^2K\left(aKt+1\right)-a^2Kt\cdot aK}{\left(aKt+1\right)^2}=\dfrac{a^2K}{\left(aKt+1\right)^2}.\]
			\item Chứng minh nếu $x=[C]$ thì $x'(t)=K(a-x)^2$.\\
			Theo câu trên, nếu nếu $x=[C]$ thì $x^\prime(t)=\dfrac{a^2K}{\left(aKt+1\right)^2}$.\\
			Ta lại có 
			\[K(a-x)^2=K \left(a-\dfrac{a^2Kt}{aKt+1}\right)^2=\dfrac{a^2K}{\left(aKt+1\right)^2}.\]
			Vậy $x'(t)=K(a-x)^2$.
			\item Nêu hiện tượng xảy ra với nồng độ các chất khi $t\longrightarrow +\infty$.\\
			Ta có $\lim\limits_{t\to +\infty}[C]=\lim\limits_{t\to +\infty}\dfrac{a^2Kt}{aKt+1}=a\  (mol/l)$.\\
			Vậy nồng độ của chất $C$ dần đến $a\  (mol/l)$.
			\item Nêu hiện tượng xảy ra với tốc độ phản ứng khi $t\longrightarrow +\infty$.\\
			Ta có $\lim\limits_{t\to +\infty}x^\prime(t)=\lim\limits_{t\to +\infty}\dfrac{a^2K}{\left(aKt+1\right)^2}=0$.\\
			Vậy tốc độ  của phản ứng  dần đến $0$.\\
	\end{enumerate}}
\end{vd}

\begin{vd}
	Giả sử số lượng của một quần thể nấm men tại môi trường nuôi cấy trong phòng thí nghiệm được mô hình hoá bằng hàm số $P(t)=\dfrac{a}{b+\mathrm{e}^{-0{,}75t}}$, trong đó thời gian $t$ được tính bằng giờ. Tại thời điểm ban đầu $t=0$, quần thể có 20 tế bào và tăng với tốc độ $12$ tế bào/giờ. Tìm các giá trị của $a$ và $b$. Theo mô hình này, điều gì xảy ra với quần thể nấm men về lâu dài?
	\loigiai{
		Ta có $P'(t)=\dfrac{0,75a \mathrm{e}^{-0,75t}}{\left(b+\mathrm{e}^{-0{,}75t}\right)^2}, t \geq 0$.\\
		Theo đề bài, ta có $P(0)=20$ và $P'(0)=12$. Do đó, ta có hệ phương trình:
		$$
		\heva{&\dfrac{a}{b+1}=20 \\& \dfrac{0,75a}{(b+1)^2=12}} \Leftrightarrow \heva{&a=20(b+1)\\&\dfrac{15}{b+1}=12}
		$$
		Giải hệ phương trình này, ta được $a=25$ và $b=\dfrac{1}{4}$.\\
		Khi đó, $P'(t)=\dfrac{18{,}75\mathrm{e}^{-0{,}75t}}{\left(\dfrac{1}{4}+\mathrm{e}^{-0{,}75 t}\right)^2} > 0, \forall t \geq 0$, tức là số lượng quần thể nấm men luôn tăng.\\
		Tuy nhiên, do $\lim\limits_{t \rightarrow+\infty} P(t)=\lim\limits_{t \rightarrow+\infty} \dfrac{25}{\dfrac{1}{4}+\mathrm{e}^{-0{,}75t}}=100$ nên số lượng quần thể nấm men tăng nhưng không vượt quá $100$ tế bào. 
	}
\end{vd}

\begin{vd}
	Giả sử chi phí $C(x)$ (nghìn đồng) để sản xuất $x$ đơn vị của một loại hàng hoá nào đó được cho bởi hàm số $C(x)=30\,000+300x-2{,}5x^2+0{,}125x^3$.
	\begin{enumerate}
		\item Tìm hàm chi phí biên.
		\item Tìm $C'(200)$ và giải thích ý nghĩa.
		\item So sánh $C'(200)$ với chi phí sản xuất đơn vị hàng hoá thứ 201.
	\end{enumerate}
	\loigiai{
		\begin{enumerate}
			\item Hàm chi phí biên là $C'(x)=300-5x+0{,}375x^2$.
			\item Ta có $C'(200)=300-5\cdot 200+0,375\cdot 200^2=14300$.\\				
			Chi phí biên tại $x=200$ là $14\,300$ nghìn đồng, nghĩa là chi phí để sản xuất thêm một đơn vị hàng hoá tiếp theo (đơn vị hàng hoá thứ 201) là khoảng $14\,300$ nghìn đồng.
			\item Chi phí sản xuất đơn vị hàng hoá thứ $201$ là
			$$
			C(201)-C(200)=1\,004\,372{,}625- 990\,000=14\,372{,}625 \text { (nghìn đồng).}
			$$				
			Giá trị này xấp xỉ với chi phí biên $C'(200)$ đã tính ở câu b.
		\end{enumerate}
		
	}
\end{vd}

\begin{dang}{Bài toán tối ưu hoá đơn giản}
\end{dang}

\begin{vd}
	\immini{Một nhà sản xuất cần làm những hộp đựng hình trụ có thể tích $ 1 $ lít. Tìm các kích thước của hộp đựng để chi phi vật liệu dùng để sản xuất là nhỏ nhất (kết quả được tính theo centimét và làm tròn đến chứ số thập phân thứ hai).}{
	\begin{tikzpicture}[line join=round,line cap=round,line width=.6pt,font=\footnotesize,scale=0.45,>=stealth]
		\coordinate[label=right:$A$] (A) at (3,0);
		\coordinate[label=left:$O$] (O) at (0,0);
		\coordinate[label=right:$A'$] (A1) at ($(A)+(90:6)$);
		\coordinate[label=left:$O'$] (O1) at ($(O)+(90:6)$);
		\draw (A) arc (0:-180:3 and 3/4)--($(A1)!2!(O1)$) arc (180:0:3 and 3/4) arc (0:-180:3 and 3/4) (A)--(A1)--(O1);
		\draw[dashed] (O1)--(O)--(A) arc (0:180:3 and 3/4);
		\fill (O)circle(1.5pt) (O1)circle(1.5pt) (A)circle(1.5pt) (A1)circle(1.5pt);
\end{tikzpicture}}
	\loigiai{
		Đổi $1 \text{ lít} =1000 \text{ cm}^3$.
		\\
		Gọi $r( cm )$ là bán kính đáy của hình trụ, $h( cm )$ là chiều cao của hình trụ.
		\\
		Diện tích toàn phần của hinh trụ là $S=2 \pi r^2+2 \pi r h$.
		\\
		Do thể tích của hình trụ là $1000 \text{ cm}^3$ nên ta có: $1000=V=\pi r^2 h$, hay $h=\dfrac{1000}{\pi r^2}$.
		\\
		Do đó, diện tích toàn phần của hình trụ là $S=2 \pi r^2+\dfrac{2000}{r},\, r>0$.
		\\
		Ta cần tìm $r$ sao cho $S$ đạt giá trị nhỏ nhất. Ta có
		\begin{align*}
			&S'=4 \pi r-\dfrac{2000}{r^2}=\dfrac{4 \pi r^3-2000}{r^2};
			\\
			&S'=0 \Leftrightarrow \pi r^3=500 \Leftrightarrow r=\sqrt[3]{\dfrac{500}{\pi}}
		\end{align*}
		Bảng biến thiên
		\begin{center}
			\begin{tikzpicture}[font=\footnotesize,thick,>=stealth]
				\tikzset{double style/.append style={double distance=1.5pt}}\tkzTabInit[nocadre=false,lgt=1.2,espcl=3.5,deltacl=0.6,lw=.75pt,color,colorL=green!50,colorV=green!50]
				{$r$ /1.2, $S'(r)$ /1, $S(r)$ /2.5}
				{$0$,$\sqrt[3]{\dfrac{500}{\pi}}$,$+\infty$}
				\tkzTabLine{ ,-,$0$,+, }
				\tkzTabVar{+/$+\infty$,-/$S\left( \sqrt[3]{\dfrac{500}{\pi}} \right)$,+/$+\infty$}
			\end{tikzpicture}
		\end{center}
		Khi đó
		$$
		h=\dfrac{1000}{\pi r^2}=\dfrac{1000}{\pi \sqrt[3]{\frac{250000}{\pi^2}}}=\dfrac{100}{\sqrt[3]{250 \pi}}.
		$$
		Vậy cần sản xuất các hộp đựng hình trụ có bán kinh đáy $r=\sqrt[3]{\dfrac{500}{\pi}} \approx 5,42 \text{ (cm)}$ và chiều cao $h=\dfrac{100}{\sqrt[3]{250 \pi}} \approx 10,84\text{ (cm)}$.
	}
\end{vd}

\begin{vd}
	Một bác nông dân có ba tấm lưới B40, mỗi tấm dài $a \text{ (m)}$ và muốn rào một mảnh vườn dọc bờ sông có dạng hình thang cân $ABCD$ như {\it Hình 36} (bờ sông là đường thẳng $CD$ không phải rào). Hỏi bác đó có thể rào được mảnh vườn có diện tích lớn nhất là bao nhiêu mét vuông?
	\begin{center}
		\begin{tikzpicture}[scale=.6]
			\path 
			(-3,0) coordinate (D)
			(3,0) coordinate (C)
			($(D)+(65:3)$) coordinate (A)
			($(C)+(115:3)$) coordinate (B)
			;
			\fill[cyan!50] (-4.5,-1) rectangle (4,0);
			\draw[thick] (A)--node[above]{$a \text{(m)}$}(B)--node[right]{$a \text{(m)}$}(C)--(D)--node[left]{$a \text{(m)}$} cycle;
			\node at (0,-1.5) {\it Hình 36};
			\foreach \x/\g in {A/120,B/60,C/-60,D/-120}		\fill[black] 	(\x) circle (1pt)
			($(\g:3mm)+(\x)$) node {$\x$};
		\end{tikzpicture}
		
	\end{center}
	
	\loigiai{
		\begin{center}
			\begin{tikzpicture}
				\path 
				(-3,0) coordinate (D)
				(3,0) coordinate (C)
				($(D)+(65:3)$) coordinate (A)
				($(C)+(115:3)$) coordinate (B)
				($(C)!(A)!(D)$) coordinate (M)
				($(C)!(B)!(D)$) coordinate (N)
				;
				\draw[thick] (A)--(B)--(C)--(D)-- cycle;
				\draw[dashed] (A)--(M) (B)--(N);
				\foreach \x/\g in {A/120,B/60,C/-60,D/-120,M/-90,N/-90}		\fill[black] 	(\x) circle (1pt)
				($(\g:3mm)+(\x)$) node {$\x$};
			\end{tikzpicture}
		\end{center}
		Gọi $M$, $N$ lần lượt là hình chiếu vuông góc của $A$, $B$ trên $CD$.\\
		Đặt $x=MD$, $\left( 0<x<a\right)$. Suy ra $AM=\sqrt{AD^2-MD^2}=\sqrt{a^2-x^2}$.\\
		Diện tích của mảnh vườn hình thang cân là $S(x)=\dfrac{(AB+CD)AM}{2}=(a+x)\sqrt{a^2-x^2}$.\\
		Xét hàm số $f(x)= (a+x)\sqrt{a^2-x^2}$ trên khoảng $\left( 0<x<a\right)$.\\
		$f^\prime (x)=\dfrac{-2x^2-ax+a^2}{\sqrt{a^2-x^2}}$, $f^\prime (x)=0\Leftrightarrow \dfrac{-2x^2-ax+a^2}{\sqrt{a^2-x^2}}=0\Leftrightarrow \hoac{&x=-a \notin \left( 0<x<a\right) \\&x=\dfrac{a}{2}\in \left( 0<x<a\right) }$.\\
		Bảng biến thiên hàm số $f(x)$ trên khoảng $\left( 0;a\right)$.
		\begin{center}
			\begin{tikzpicture}
				\tkzTabInit[lgt=1.2,espcl=4.5,deltacl=0.6]
				{$x$/1,$f'(x)$/1,$f(x)$/3} {$0$,$\dfrac{a}{2}$,$a$}
				\tkzTabLine{,+,0,-,}
				\tkzTabVar{-/$a^2$,+/$\dfrac{3\sqrt{3}a^2}{4}$,-/$0$}
			\end{tikzpicture}
		\end{center}
		Từ bảng biến thiên suy ra $\max\limits_{(0;a)} f(x)=f\left(\dfrac{a}{2}\right)=\dfrac{3\sqrt{3}a^2}{4}$.\\
		Vậy bác nông dân có thể rào được mảnh vườn có diện tích lớn nhất $\dfrac{3\sqrt{3}a^2}{4} \text{ m}^2$.
	}
\end{vd}

\begin{vd}
	Có hai xã $A$, $B$ cùng ở một bên bờ sông Lam, khoảng cách từ hai xã đó đến bờ sông lần lượt là $AA'=500 \text{ m}$, $BB'=600 \text{ m}$ và người ta đo được $A'B'=2\,200 \text{ m}$ {\it Hình 37}. Các kĩ sư muốn xây một trạm cung cấp nước sạch nằm bên bờ sông Lam cho dân hai xã. Để tiết kiệm chi phí, các kĩ sư cần phải chọn vị trí $M$ của trạm cung cấp nước sạch đó trên đoạn $A'B'$ sao cho tổng khoảng cách từ hai xã đến vị trí $M$ là nhỏ nhất. Hãy tìm giá trị nhỏ nhất của tổng khoảng cách đó.
	\begin{center}
		\begin{tikzpicture}[scale=.6]
			\path 
			(0:0) coordinate (A')
			(0:6) coordinate (B')
			(0:2) coordinate (M)
			($(A')+(90:2.5)$) coordinate (A)
			($(B')+(90:3)$) coordinate (B)
			;
			\fill[cyan!50] (-1.5,-1) rectangle (7.5,0);
			\draw[thick] (A')--node[left]{$500 \text{(m)}$}(A)--(M)--(B)--node[right]{$600 \text{(m)}$}(B');
			
			\foreach \i/\j in{A'/-100,B'/-80,A/100,B/80,M/-90}{\fill [black](\i) circle (1pt) ($(\i)+(\j:3mm)$) node {$\i$};}
			
			\draw [dashed,<->]	(0,.6)--(6,.6) node[pos=0.75,sloped,above]{$2\,200\text{(m)}$}; %Tùy chọn sloped,above,below
			\node at (3,-1.5){\it Hình 37};
		\end{tikzpicture}
	\end{center}
	\loigiai{
		\begin{center}
			\begin{tikzpicture}
				\path 
				(0:0) coordinate (A')
				(0:6) coordinate (B')
				(0:2) coordinate (M)
				($(A')+(90:2.5)$) coordinate (A)
				($(B')+(90:3)$) coordinate (B)
				;
				\draw[thick] (A')--node[left]{$500 \text{(m)}$}(A)--(M)--(B)--node[right]{$600 \text{(m)}$}(B') (A')--(B');
				
				\foreach \i/\j in{A'/-100,B'/-80,A/100,B/80,M/-90}{\fill [black](\i) circle (1pt) ($(\i)+(\j:3mm)$) node {$\i$};}
				
				\draw [dashed,<->]	(0,.6)--(6,.6) node[pos=0.75,sloped,above]{$2\,200\text{(m)}$}; %Tùy chọn sloped,above,below
				\node at (3,-1.5){\it Hình 37};
			\end{tikzpicture}
		\end{center}
		Đặt $A'M=x$, $(0<x<2200)$,  $B'M=2200-x$.\\
		Ta có: $AM=\sqrt{x^2+500^2}$, $BM=\sqrt{(2200-x)^2+600^2}$.\\
		Khi đó tổng khoảng cách từ hai xã đến vị trí $M$ là $AM+BM= \sqrt{x^2+500^2}+\sqrt{(2200-x)^2+600^2} $.\\
		Xét hàm số $f(x)= \sqrt{x^2+500^2}+\sqrt{(2200-x)^2+600^2}$ trên khoảng $(0<x<2200)$.\\
		%$f(x)=\sqrt{x^2+500}+\sqrt{x^2-4400x+4840600}$ .\\
		$f^\prime (x)=\dfrac{x}{\sqrt{x^2+500^2}}-\dfrac{2200-x}{\sqrt{(2200-x)^2+600^2}}$, $f^\prime (x)=0\Leftrightarrow \dfrac{x}{\sqrt{x^2+500^2}}=\dfrac{2200-x}{\sqrt{(2200-x)^2+600^2}}$\\
		$\Leftrightarrow \dfrac{x^2}{x^2+500^2}=\dfrac{(2200-x)^2}{(2200-x)^2+600^2}$\\
		$\Leftrightarrow \dfrac{x^2+500^2}{x^2}=\dfrac{(2200-x)^2+600^2}{(2200-x)^2}$\\
		$\Leftrightarrow 1+\dfrac{500^2}{x^2}=1+\dfrac{600^2}{(2200-x)^2}$\\
		$\Leftrightarrow \dfrac{25}{x^2}=\dfrac{36}{(2200-x)^2}$\\
		$\Leftrightarrow \dfrac{5}{x}=\dfrac{6}{2200-x}\Leftrightarrow x=1000$, vì $ x>0$.\\ 
		Bảng biến thiên hàm số $f(x)$ trên khoảng $\left( 0;2200\right)$.
		\begin{center}
			\begin{tikzpicture}
				\tkzTabInit[lgt=1.2,espcl=4.5,deltacl=0.6]
				{$x$/1,$f'(x)$/1,$f(x)$/3} {$0$,$1000$,$2200$}
				\tkzTabLine{,-,0,+,}
				\tkzTabVar{+/$2780$,-/$2460$,+/$2856$}
			\end{tikzpicture}
		\end{center}
		Vậy giá trị nhỏ nhất của tổng khoảng cách từ hai xã đó đến bờ sông  là khoảng $2460 \text{ m}$, tại vị trí $M$ cách điểm $A'$  là $1000 \text{ m}$.
	}
\end{vd}

\subsection{BÀI TẬP TỰ LUYỆN}
\begin{bt}
	Một tàu đổ bộ tiếp cận Mặt Trăng theo cách tiếp cận thẳng đứng và đốt cháy các tên lửa hãm ở độ cao $250$ km so với bề mặt của Mặt Trăng.\\
	Trong khoảng $50$ giây đầu tiên kể từ khi đốt cháy các tên lửa hãm, độ cao $h$ của con tàu so với bề mặt của Mặt Trăng được tính (gần đúng) bởi hàm $h(t)=-0{,}01t^3+1{,}1t^2-30t+250$, trong đó $t$ là thời gian tính bằng giây và $h$ là độ cao tính bằng kilômét.\\
	\textit{(Nguồn: A. Bigalke et al., Mathematik, Grundkurs ma-1, Cornelsen 2016).}
	\begin{enumerate}
		\item Vẽ đồ thị của hàm số $y=h(t)$ với $0\leq t\leq 50$ (đơn vị trên trục hoành là $10$ giây, đơn vị trên trục tung là $10$ km).
		\item Gọi $v(t)$ là vận tốc tức thời của con tàu ở thời điểm $t$ (giây) kể từ khi đốt cháy các tên lửa hãm với $(0\leq t\leq 50$). Xác định hàm số $v(t)$.
		\item Vận tốc tức thời của con tàu lúc bắt đầu hãm phanh là bao nhiêu? Tại thời điểm $t=25$ (giây) là bao nhiêu?
		\item Tại thời điểm $t=25$ (giây), vận tốc tức thời của con tàu vẫn giảm hay đang tăng trở lại?
		\item Tìm thời điểm $t$ ($0\leq t\leq 50$) sao cho con tàu đạt khoảng cách nhỏ nhất so với bề mặt của Mặt Trăng. Khoảng cách nhỏ nhất này là bao nhiêu?
	\end{enumerate}
	\loigiai{
		\begin{enumerate}
			\item Vẽ đồ thị của hàm số $h(t)=-0{,}01t^3+1{,}1t^2-30t+250$.
			\begin{itemize}
				\item Miền khảo sát: $[0;50]$.
				\item Đạo hàm: $h'(t)=-0{,}03t^2+2{,}2t-30$.
				\[h'(t)=0\Leftrightarrow -0{,}03t^2+2{,}2t-30=0\Leftrightarrow \hoac{&t\approx 18\\ &t\approx 55.}\]
				\item Bảng biến thiên:
				\begin{center}
					\begin{tikzpicture}
						\tkzTabInit[lgt=1.2, espcl=3, deltacl=0.6]
						{$t$/0.6, $h'(t)$/0.6, $h(t)$/2}
						{$0$, $18$, $50$}
						\tkzTabLine{, -, 0, +, }
						\tkzTabVar{+/ $250$, -/$8{,}08$, +/$250$}
					\end{tikzpicture}
				\end{center}
				\begin{itemize}
					\item Hàm số nghịch biến trên các khoảng $(0;18)$ và đồng biến trên khoảng $(18;50)$.
					\item Hàm số đạt cực tiểu tại $t=18$, $y_{_\text{CT}}=h(18)=8{,}08$.
				\end{itemize}
				\item Bảng giá trị:
				\begin{center}
					\begin{tikzpicture}
						\tkzTabInit[lgt=1.2, espcl=2.5, deltacl=1]
						{$x$/0.7, $y$/0.7}
						{$0$, $18$, $50$}
						\tkzTabLine{250, , 8.08, , 250}
					\end{tikzpicture}
				\end{center}
				\item Đồ thị:
				\begin{center}
					\begin{tikzpicture}[>=stealth, scale=1, font=\footnotesize]
						\draw[->] (-1,0)--(4.5,0) node[below] {$t$};
						\draw[->] (0,-1)--(0,8) node[left] {$h(t)$};
						\draw[fill=black] (0,0) node[below left=-0.1] {$O$} circle (1.2pt);
						\draw[fill=black] (0.8,0) node[below] {$18$} circle (1.2pt);
						\draw[fill=black] (2.34,0) node[below] {$50$} circle (1.2pt);
						\draw[fill=black] (0,0.3) node[above left = -0.1 and 0] {$8{,}08$} circle (1.2pt);
						\draw[fill=black] (0,7) node[left] {$250$} circle (1.2pt);
						\draw[dashed] (0.8,0)--(0.8,0.3)--(0,0.3) (2.34,0)--(2.34,7)--(0,7);
						\clip (0,0) rectangle (3,7);
						\draw (0,7) parabola bend (0.8,0.3) (1.5,3) parabola bend (3,8) (3,8);
					\end{tikzpicture}
				\end{center}
			\end{itemize}
			\item Xác định $v(t)$.\\
			Ta có $v(t)=h'(t)=-0{,}03t^2+2{,}2t-30$.
			\item Tính vận tốc tức thời lúc bắt đầu hãm phanh và lúc $t=25$ (giây).
			\begin{itemize}
				\item Vận tốc tức thời lúc bắt đầu hãm phanh là: $v(0)=-30$ (km/s).
				\item Vận tốc tức thời lúc $t=25$ (giây) là: $v(25)=6{,}25$ (km/s).
			\end{itemize}
			\item Tại thời điểm $t=25$ (giây), vận tốc tức thời của con tàu vẫn giảm hay tăng trở lại?
			\begin{itemize}
				\item Ta có phương trình gia tốc: $a(t)=v'(t)=-0{,}06t+2{,}2t$.
				\item Vì $a(25)=53{,}5>0$ nên tại thời điểm $t=25$ (giây), vận tốc tức thời của con tàu đang tăng trở lại.
			\end{itemize}
			\item Tìm thời điểm mà khoảng cách giữa con tàu và Mặt Trăng nhỏ nhất.\\
			Dựa vào đồ thị ta thấy tại thời điểm $t=18$ (giây) thì khoảng cách giữa con tàu và Mặt Trăng nhỏ nhất, khoảng cách này bằng $8{,}08$ km.
		\end{enumerate}
	}
\end{bt}

\begin{bt}
	Để loại bỏ $x\%$ chất gây ô nhiễm không khí từ khí thải của một nhà máy, người ta ước tính chi phí cần bỏ ra là
	$$
	C(x)=\dfrac{300 x}{100-x} \text { (triệu đồng), } 0 \leq x < 100.
	$$		
	Khảo sát sự biến thiên và vẽ đồ thị của hàm số $y=C(x)$. Từ đó, hãy cho biết:
	\begin{enumerate}
		\item Chi phí cần bỏ ra sẽ thay đổi như thế nào khi $x$ tăng?
		\item Có thể loại bỏ được $100 \%$ chất gây ô nhiễm không khí không? Vì sao?
	\end{enumerate}
	\loigiai{
		Xét hàm số $y=C(x)=\dfrac{300x}{100-x}, 0\leq x < 100$.\\
		Ta có 
		\begin{itemize}
			\item  $y'=\dfrac{30\,000}{(100-x)^2} > 0$, với mọi $x \in[0; 100)$.\\
			Do đó hàm số luôn đồng biến trên nửa khoảng $[0; 100)$.
			\item  $\lim\limits_{x \to 100^{-}} C(x)=\lim\limits_{x \to 100^{-}} \dfrac{300x}{100-x}=+\infty$, nên đồ thị hàm số có tiệm cận đứng là $x=100$.
		\end{itemize}
		Bảng biến thiên:
		\begin{center}
			\begin{tikzpicture}
				\tkzTabInit%[nocadre,lgt=1.2,espcl=2]
				{$x$/0.7,$C'(x)$/0.7,$C(x)$/2.5}{$0$,$+\infty$}  
				\tkzTabLine{ ,$+$, }
				\tkzTabVar{-/$0$,+/$\infty$}
			\end{tikzpicture}
		\end{center}
		Đồ thị hàm số như Hình $1.34$.
		\begin{enumerate}
			\item  Chi phí cần bỏ ra $C(x)$ sẽ luôn tăng khi $x$ tăng.
			\item  Vì $\lim\limits_{x \rightarrow 100^{-}} C(x)=+\infty$ (hàm số $C(x)$ không xác định khi $x=100$) nên nhà máy không thể loại bỏ $100\%$ chất gây ô nhiễm không khí (dù bỏ ra chi phí là bao nhiêu đi chăng nữa).
		\end{enumerate}
		\begin{tikzpicture}[xscale=1/50,yscale=1/50,>=stealth, font=\footnotesize, line join=round, line cap=round]
			\def\xmin{-100} \def\xmax{200}
			\def\ymin{-100} \def\ymax{450}
			%\draw[color=gray!50,dashed] (\xmin,\ymin) grid (\xmax,\ymax);
			\draw[->] (\xmin,0)--(\xmax,0) node [below]{$x$};
			\draw[->] (0,\ymin)--(0,\ymax) node [left]{$y$};
			\fill (0,0) circle (1pt) node[shift={(-45:2.5mm)}]{$O$};	
			\clip (\xmin+0.1,\ymin+0.1) rectangle (\xmax-0.1,\ymax-0.1);
			\draw[red,smooth,samples=50,domain=0:48] plot(\x,{(300*\x)/(100-1.5*\x)});
			\foreach \x in {-50,0,50,100,150}
			\draw (\x,-0.1)--(\x,0.1);	
			\foreach \x/\r in {-50/-100, 50/100, 100/200, 150/300}
			\node at (\x,0) [below,scale=0.8] {\r};
			\foreach \y in {\ymin,...,\ymax}
			\draw (-0.1,\y)--(0.1,\y);
			\foreach \y/\r in {-50/-100, 50/100, 100/200, 150/300, 200/400,250/500,300/600,350/700,400/800}
			\node at (0,\y) [left,scale=0.8] {\r};
			\draw (50,-50)--(50,450)	;
		\end{tikzpicture}
	}
\end{bt}

\begin{bt}
	Khi máu di chuyển từ tim qua các động mạch chính rồi đến các mao mạch và quay trở lại qua các tĩnh mạch, huyết áp tâm thu (tức là áp lực của máu lên động mạch khi tim co bóp) liên tục giảm xuống. Giả sử một người có huyết áp tâm thu $P$ (tính bằng mmHg) được cho bởi hàm số
	$$
	P(t)=\dfrac{25 t^2+125}{t^2+1}, 0 \leq t \leq 10,
	$$
	trong đó thời gian $t$ được tính bằng giây. Tính tốc độ thay đổi của huyết áp sau $5$ giây kể từ khi máu rời tim.
	\loigiai{
		Ta có tốc độ thay đổi của huyết áp là $P'(t)=\dfrac{-100t}{(t^2+1)^2}$.\\
		Do đó tốc độ thay đổi huyết áp sau $5$ s là $P'(5)=-\dfrac{125}{169}$.
	}
\end{bt}

\begin{bt}
	Bạn Việt muốn dùng tấm bìa hình vuông cạnh $6$ dm làm một chiếc hộp không nắp, có đáy là hình vuông bằng cách cắt bỏ đi $4$ hình vuông nhỏ ở bốn góc của tấm bìa (Hình bên dưới).
	\begin{center}
		\begin{tikzpicture}[>=stealth,line join=round,line cap=round,font=\footnotesize,scale=1]
			\fill[blue!20] (0,0)--(3,0)--(3,3)--(0,3);
			\fill[white] (0,0)--(0,0.4)--(0.4,0.4)--(0.4,0) (2.6,0)--(3,0)--(3,0.4)--(2.6,0.4)
			(2.6,2.6)--(2.6,3)--(3,3)--(3,2.6) (0,2.6)--(0.4,2.6)--(0.4,3)--(0,3);
			\draw (0,0)--(3,0)--(3,3)--(0,3)--cycle;
			\draw[line width=0.2pt] (0.4,0.4)--(2.6,0.4)--(2.6,2.6)--(0.4,2.6)--cycle;
			\draw [line width=0.05pt,<->] (3.05,2.6)--(3.05,3);
			\path (3,2.6)--(3,3)node[pos=0.5,sloped,black,below]{$x$};					
		\end{tikzpicture}~\begin{tikzpicture}[>=stealth,line join=round,line cap=round,font=\footnotesize,scale=1]
			\fill[blue!20] (0,0)--(1.5,-1)--(3,0)--(1.5,1);		
			\fill[black!20] (0,0)--(-0.1,0.4)--(1.4,1.4)--(1.5,1);
			\draw (0,0)--(-0.1,0.4)--(1.4,1.4)--(1.5,1)--cycle;
			\fill[black!20] (1.5,1)--(1.6,1.4)--(3.1,0.4)--(3,0);
			\draw (1.5,1)--(1.6,1.4)--(3.1,0.4)--(3,0)--cycle;
			\fill[black!20] (1.5,-1)--(1.5,-0.6)--(3.1,0.4)--(3,0);
			\draw (1.5,-1)--(1.5,-0.6)--(3.1,0.4)--(3,0)--cycle;
			\fill[black!20] (1.5,-1)--(1.3,-0.7)--(-0.3,0.3)--(0,0);
			\draw (1.5,-1)--(1.3,-0.7)--(-0.3,0.3)--(0,0)--cycle;
		\end{tikzpicture}
	\end{center}
	
	Bạn Việt muốn tìm độ dài cạnh hình vuông cần cắt bỏ để chiếc hộp đạt thể tích lớn nhất.
	\begin{enumerate}
		\item Hãy thiết lập hàm số biểu thị thể tích hộp theo $x$ với $x$ là độ dài cạnh hình vuông cần cắt đi.
		\item Khảo sát và vẽ đồ thị hàm số tìm được.\\		
		Từ đó, hãy tư vấn cho bạn Việt cách giải quyết vấn đề và giải thích vì sao cần chọn giá trị này. (Làm tròn kết quả đến hàng phần mười.)
	\end{enumerate}
	\loigiai{
		\begin{enumerate}
			\item Hãy thiết lập hàm số biểu thị thể tích hộp theo $x$ với $x$ là độ dài cạnh hình vuông cần cắt đi.
			Mặt đáy của hộp là hình vuông có cạnh bằng $6-2x$ (cm), với $0<x<3$. Vậy diện tích của đáy hộp là $S=(6-2x)^2$.\\
			Khối hộp có chiều cao $h=x$ (cm).\\
			Vậy thể tích hộp là $V=S\cdot h=(6-2x)^2 \cdot x=4x^3-24x^2+36x$ (cm$^3$).
			\item Khảo sát và vẽ đồ thị hàm số tìm được.\\
			Xét hàm $f(x)=4x^3-24x^2+36x,\,\,0<x<3$.
			\begin{enumerate}
				\item Tập xác định: $\mathscr{D}=(0;3)$.
				\item Sự biến thiên.
				\begin{itemize}
					\item Giới hạn tại vô cực: $\displaystyle\displaystyle\lim \limits{n \to +\infty}_{x \rightarrow+\infty} f(x)=+\infty, \displaystyle\displaystyle\lim \limits{n \to +\infty}_{x \rightarrow-\infty} f(x)=-\infty$.
					\item Ta có $f'(x)=12x^2-48x+36\Rightarrow f'(x)=0\Leftrightarrow x^2-4x+3=0\Leftrightarrow \hoac{& x=1\\ & x=3.}$\\
					Ta có bảng biến thiên:
					\begin{center}
						\begin{tikzpicture}[>=stealth]
							\tkzTabInit[nocadre=false,lgt=1,espcl=2,deltacl=0.5]{$x$/.7 ,$y'$/.7,$y$/2}
							{$0$, $1$ , $3$ }
							\tkzTabLine{ , + , $0$ , - , $0$ }
							\tkzTabVar{-/$0$ , +/$16$ , -/$0$ }
						\end{tikzpicture}
					\end{center}
					Hàm số đồng biến trên $(0;1)$ và nghịch biến trên khoảng $(1;3)$.\\
					Hàm số không có cực trị.
					\item Đồ thị hàm số đi qua các điểm $(0 ; 0),(1 ; 16),(3 ; 0)$.
					\begin{center}
						\begin{tikzpicture}[line cap=butt,line join=miter,>=stealth,scale=0.8,font=\footnotesize,y=0.5cm]
							\tikzset{declare function={xmin=-1;xmax=4;ymin=-1;ymax=17;},
								smooth,samples=450}
							\draw[->] (xmin,0)--(xmax,0) node[shift={(0:7pt)},]{$ x $};
							\draw[->] (0,ymin)--(0,ymax) node[shift={(90:7pt)}]{$ y $};
							\fill (0,0) node[shift={(-150:7pt)}]{$ O $};
							\clip (xmin,ymin) rectangle (xmax,ymax);
							\foreach \i in {1}{
								\draw(\i,1.5pt)--(\i,-1.5pt)node[below]{$\i$};}
							\foreach \j in {16}{
								\draw(-1.5pt,\j)--(1.5pt,\j) node[left]{$\j$};}
							%	\draw(-1.5pt,-1)--(1.5pt,-1)node[shift={(7pt,0pt)}]{$-1$};	
							\def\f(#1){4*(#1)^3-24*(#1)^2+36*(#1)} % Đồ thị hàm số y=x^3+3x^2+3x+1
							\def\c{-1}
							\def\d{0}
							\def\e{-2}	
							\pgfmathsetmacro\fc{\f(\c)}
							\pgfmathsetmacro\fd{\f(\d)}
							\pgfmathsetmacro\fe{\f(\e)}
							\draw[samples=100] plot[domain=0:3] (\x,{\f(\x)});
							\foreach \x/\y in {\c/\fc,\d/\fd,\e/\fe}{
								\draw[dashed] (1,0)--(1,16)--(0,16);
								\fill[white,draw=black] (\x,\y) circle (1pt);}	
							%\node at (-2,-3.2) [right,fill=white,font=\footnotesize]{\it Hình LT $2b$};
						\end{tikzpicture}
					\end{center}
				\end{itemize}
			\end{enumerate}
			Vậy hình vuông mà bạn Việt cần cắt bỏ pải có độ dài cạnh $x=1$ dm thì chiếc hộp đạt thể tích lớn nhất.
		\end{enumerate}
	}
\end{bt}

%%Tổng hợp vdc
% \setcounter{ex}{0}
\section*{TỔNG HỢP VDC - CHƯƠNG I}
\Opensolutionfile{ans}[ans/ansBTchoice]
\TN
\begin{ex}%[2D1C5-7]
Tìm được trên đồ thị $(C)$ của hàm số $y=\dfrac{x^2+4x+5}{x+2}$ hai điểm $M(a;b)$ và $N(c;d)$ có khoảng cách đến đường thẳng $\Delta\colon 3x+y+6=0$ nhỏ nhất. Khi đó $a+b+c+d$ bằng
\choice
{$4$}
{$9$}
{$-9$}
{\True $-4$}
\loigiai
{Tập xác định $\mathscr{D}=\mathbb{R}\setminus\{-2\}$.
Gọi $M\left(x_0;y_0\right) \in (C)$. Ta có $x_0\neq -2$ và $y_0=\dfrac{x_0^2+4x_0+5}{x_0+2}$.\\
Khoảng cách từ $M$ đến đường thẳng $\Delta\colon 3x+y+6=0$ là
$$\mathrm{d}\left(M,\Delta\right)=\dfrac{1}{\sqrt{10}}\left|\dfrac{4x_0^2+16x_0+17}{x_0+2}\right|=\dfrac{1}{\sqrt{10}}\left|4\left(x_0+2\right)+\dfrac{1}{x_0+2}\right|.$$
Hàm số $f(t)=4t+\dfrac{1}{t}$ với $t\neq 0$ có $f'(t)=\dfrac{4t^2-1}{t^2}$ và $f'(t)=0\Leftrightarrow 4t^2-1=0\Leftrightarrow t=\pm \dfrac{1}{2}$.\\
Bảng biến thiên \begin{center}
\begin{tikzpicture}
\tkzTabInit[nocadre=false, lgt=1.2, espcl=2.5, deltacl=0.6]{$t$/1.2,$f'(t)$/0.6,$f(t)$/2}
{$-\infty$, $-\dfrac{1}{2}$, $0$, $\dfrac{1}{2}$, $+\infty$}
\tkzTabLine {,+,0,-,d,-,0,+,}
\tkzTabVar{-/$-\infty$, +/$-4$, -D+/$-\infty$/$+\infty$,-/$4$, +/$+\infty$}
\end{tikzpicture}
\end{center}
Suy ra $\min\limits_{t\neq 0} \left|f(t)\right|=4$ khi $t=\pm\dfrac{1}{2}$.\\
Do đó $\min\mathrm{d}(M,\Delta)=\dfrac{4}{\sqrt{10}}$ khi
$x_0+2=\pm\dfrac{1}{2} \Leftrightarrow \hoac{& x_0=-\dfrac{3}{2} \Rightarrow y_0=\dfrac{5}{2}\\ &x_0=-\dfrac{5}{2} \Rightarrow y_0=-\dfrac{5}{2}.}$\\
Như thế có hai điểm thoả yêu cầu bài toán là $M\left(-\dfrac{3}{2};\dfrac{5}{2}\right)$ và $N\left(-\dfrac{5}{2};-\dfrac{5}{2}\right)$.\\
Vậy $a+b+c+d=-\dfrac{3}{2}+\dfrac{5}{2}-\dfrac{5}{2}-\dfrac{5}{2}=-4$.
}
\end{ex}

\begin{ex}%[Dự Án Giảng 12 4 in 1, Lê Văn Toàn]%[2D1C5-6]
Trên đồ thị của hàm số $y=\dfrac{3x}{x-2}$ có điểm $M\left(x_0;y_0\right)$ $\left( \text{ với }x_0<0\right)$ sao cho tiếp tuyến tại điểm đó cùng với các trục tọa độ tạo thành một tam giác có diện tích bằng $\dfrac{3}{4}$. Khi đó $x_0+2y_0$ bằng
\choice
{$\dfrac{1}{2}$}
{$-1$}
{$-\dfrac{1}{2}$}
{\True $1$}
\loigiai{
Gọi $(C)$ là đồ thị của hàm số $y=\dfrac{3x}{x-2}$, $M\left(x_0;y_0\right)\in (C)$, suy ra $y_0=\dfrac{3x_0}{x_0-2}$ và $y'\left(x_0\right)=\dfrac{-6}{\left(x_0-2\right)^2}$.\\
Phương trình tiếp tuyến của $(C)$ tại $M\left(x_0;y_0\right)$ là $\Delta\colon y=\dfrac{-6}{\left(x_0-2\right)^2}\left(x-x_0\right)+\dfrac{3x_0}{x_0-2}$.\\
Gọi $A=\Delta \cap Ox\Rightarrow -6x+3x^2_0=0\Rightarrow x=\dfrac{x^2_0}{2}\Rightarrow A\left(\dfrac{x^2_0}{2};0\right)$.\\
Gọi $B=\Delta\cap Oy\Rightarrow y=\dfrac{6x_0}{\left(x_0-2\right)^2}+\dfrac{3x_0}{x_0-2}=\dfrac{3x_0^2}{\left(x_0-2\right)^2}\Rightarrow B\left(0;\dfrac{3x_0^2}{\left(x_0-2\right)^2}\right).$\\
Ta có $$S_{OAB}=\dfrac{1}{2}OA\cdot OB=\dfrac{1}{2}\cdot \dfrac{x^2_0}{2}\cdot \dfrac{3x^2_0}{\left(x_0-2\right)^2}=\dfrac{3}{4}\Leftrightarrow x^4_0=\left(x_0-2\right)^2\Leftrightarrow \hoac{&x^2_0=x_0-2\\&x^2_0=-x_0+2}\Leftrightarrow \hoac{&x_0=1\\&x_0=2.}$$
Do $x_0<0$ nên  nhận $x_0=-2\Rightarrow y_0=\dfrac{3}{2}$.\\
Vậy $x_0+2y_0=1$.
}
\end{ex}

\begin{ex}%[Mức độ C]%[2D1C5-3]
Tìm tất cả các giá trị thực của $m$ để phương trình $|x^4-2x^2-3|=2m-1$ có đúng $6$ nghiệm thực phân biệt.
\choice{$1<m<\dfrac{1}{3}$}{$4<m<5$}{$3<m<4$}{\True $2<m<\dfrac{5}{2}$}
\loigiai{Xét $g(x)=x^4-2x^2-3$; $g'(x)=4x^3-4x$.\\
$g'(x)=0\Leftrightarrow 4x^3-4x=0 \Leftrightarrow \left[\begin{array}{l}
x=0\\x=1\\x=-1.
\end{array}\right.$ \\
Bảng biến thiên của hàm số $g(x)$
\begin{center}
\begin{tikzpicture}
\tkzTabInit[nocadre=false, lgt=1.5,espcl=3.5]
{$x$/1,$g'(x)$/1,$g(x)$/2}
{$-\infty$,$-1$,$0$,$1$,$+\infty$}
\tkzTabLine{,-,0,+,0,-,0,+, }
\tkzTabVar{+/$+\infty$,-/$-4$,+/$-3$,-/$-4$,+/$+\infty$/}
\end{tikzpicture}
\end{center}
Bảng biến thiên của hàm số $f(x)=|x^4-2x^2-3|$ là
\begin{center}
\begin{tikzpicture}[scale=0.75]
\tkzTabInit[nocadre=false, lgt=1.5,espcl=3.5]
{$x$/1,$f(x)$/2}
{$-\infty$,$-\sqrt{3}$,$-1$,$0$,$1$,$\sqrt{3}$,$+\infty$}
\tkzTabVar{+/$+\infty$,-/$0$,+/$4$,-/$3$,+/$4$,-/$0$,+/$+\infty$/}
\end{tikzpicture}
\end{center}
Để phương trình $|x^4-2x^2-3|=2m-1$ có đúng $6$ nghiệm thực phân biệt khi và chỉ khi $$ 3<2m-1<4 \Leftrightarrow 2<m<\dfrac{5}{2}.$$.
}
\end{ex}

\begin{ex}%[Mức độ C]%[2D1C5-3]
Cho hàm số $y=f(x)$ liên tục trên $\mathbb{R}$ và có đồ thị như hình vẽ dưới đây. Tập hợp tất cả các giá trị thực của tham số $m$ để phương trình $f(x^2+2x-2)=3m+1$ có nghiệm thuộc khoảng $[0;1]$.
\begin{center}
\begin{tikzpicture}[>=stealth]
\draw [->] (-3.5,0)--(1.5,0);
\draw [->] (0,-1)--(0,5);
\draw (0,0) node[below right]{$O$};
\draw (1.5,0) node[below]{$x$};
\draw (0,5) node[below left]{$y$};
\draw (1,0) node[below]{$1$};
\draw (0,4) node[below left]{$4$};
\draw (-2,0) node[below]{$-2$};
\clip (-4,-1) rectangle (5,5);
\draw [thick,samples=100] plot[domain=-5:5](\x,{(\x)^3+3*(\x)^2});
\draw[dashed] (0,4) -- (1,4) --(1,0);
\fill[black] (1,4) circle(2pt);
\fill[black] (-2,4) circle(2pt);
\fill[black] (0,0) circle(2pt);
\draw[dashed] (0,4) -- (-2,4) --(-2,0);
\draw (1.1,5) node[below right]{$f(x)$};
\end{tikzpicture}
\end{center}
\choice
{$[0;4]$}
{$[-1;0]$}
{$[0;1]$}
{\True $\bigg[-\dfrac{1}{3};1\bigg]$}
\loigiai{Đặt $t=x^2+2x-2$, Với $x \in [0;1] \Rightarrow t \in [-2;1].$\\
Để phương trình $f(x^2+2x-2)=3m+1$ có nghiệm thuộc đoạn $[0;1]$ khi và chỉ khi phương trình $f(t)=3m+1$ có nghiệm thuộc $[-2;1].$\\ Do đó $ 0\leq m \leq 4 \Leftrightarrow -\dfrac{1}{3}\leq m \leq 1$.\\
Vậy $m \in \bigg[-\dfrac{1}{3};1\bigg]$}
\end{ex}

\begin{ex}%[Dự án TL12New-4in1-NCT]%[2D1C4-2]
Tìm tất cả các giá trị thực của tham số $m$ sao cho đồ thị hàm số $y=\dfrac{x+2}{\sqrt{mx^2+1}+\sqrt{(1-m)x^2+1}}$ có hai tiệm cận ngang.
\choice
{$m>0$}
{$m<1$}
{\True $0\leq m\leq 1$}
{$0<m<1$}
\loigiai{
Xét các trường hợp
\begin{itemize}
\item $m<0$ hoặc $m>1$, khi đó $\lim\limits_{x\to \infty}y$ không tồn tại nên đồ thị hàm số không thể có hai tiệm cận ngang.
\item Với $m=0$ hoặc $m=1$ thì hàm số trở thành $y=\dfrac{x+2}{1+\sqrt{x^2+1}}$. Đồ thị hàm số có đúng hai đường tiệm cận ngang $y=1$ và $y=-1$. Do đó $m=0$ và $m=1$ thỏa mãn yêu cầu bài toán.
\item Với $0<m<1$ ta có:
\begin{enumerate}[*]
\item $\lim\limits_{x\to +\infty}y=\lim\limits_{x\to +\infty}\dfrac{1+\dfrac{2}{x}}{\sqrt{m+\dfrac{1}{x^2}}+\sqrt{1-m+\dfrac{1}{x^2}}}=\dfrac{1}{\sqrt{m}+\sqrt{1-m}}$
\item $\lim\limits_{x\to -\infty}y=\lim\limits_{x\to -\infty}\dfrac{1+\dfrac{2}{x}}{-\sqrt{m+\dfrac{1}{x^2}}-\sqrt{1-m+\dfrac{1}{x^2}}}=\dfrac{-1}{\sqrt{m}+\sqrt{1-m}}$.
\end{enumerate}
Do đó đồ thị có hai tiệm cận ngang khi và chỉ khi\\ $\dfrac{1}{\sqrt{m}+\sqrt{1-m}}\neq \dfrac{-1}{\sqrt{m}+\sqrt{1-m}} \Leftrightarrow \dfrac{1}{\sqrt{m}+\sqrt{1-m}}\neq 0$ (hiển nhiên).
\end{itemize}
Tóm lại, $0\leq m\leq 1$ là các giá trị của $m$ thỏa mãn yêu cầu bài toán.
}
\end{ex}

\begin{ex}%[BG-12NEW-4in1, Nguyen Huynh]%[2D1C4-1]
Đồ thị hàm số $y=\log\dfrac{x^2-4x-5}{x^2-4}$ có tất cả bao nhiêu đường tiệm cận?
\choice
{$1$}
{$3$}
{\True $5$}
{$2$}
\loigiai{
Tập xác định của hàm số $\mathscr{D}=(-\infty;-2)\cup(-1;2)\cup(5;+\infty)$.\\
Mà $\lim\limits_{x\to \pm\infty}\log\dfrac{x^2-4x-5}{x^2-4}=\log 1=0$, suy ra $y=0$ là tiệm cận ngang của đồ thị hàm số.\\
Mà $\lim\limits_{x\to -2^-}\log\dfrac{x^2-4x-5}{x^2-4}=\lim\limits_{x\to +\infty}\log (x)=+\infty$, suy ra $x=-2$ là tiệm cận đứng của đồ thị hàm số.\\
Mà $\lim\limits_{x\to 2^-}\log\dfrac{x^2-4x-5}{x^2-4}=\lim\limits_{x\to +\infty}\log (x)=+\infty$, suy ra $x=2$ là tiệm cận đứng của đồ thị hàm số.\\
Mà $\lim\limits_{x\to -1^+}\log\dfrac{x^2-4x-5}{x^2-4}=\lim\limits_{x\to 0^+}\log (x)=-\infty$, suy ra $x=-1$ là tiệm cận đứng của đồ thị hàm số.\\
Mà $\lim\limits_{x\to 5^+}\log\dfrac{x^2-4x-5}{x^2-4}=\lim\limits_{x\to 0^+}\log (x)=-\infty$, suy ra $x=5$ là tiệm cận đứng của đồ thị hàm số.\\
Vậy đồ thị hàm số có $5$ đường tiệm cận.
}
\end{ex}

\begin{ex}%[THPTGQ 2018, mã 103, MĐ4]%[2D1C2-6]
Có bao nhiêu giá trị nguyên của tham số $m$ để hàm số $y=x^8+(m-4)x^5-(m^2-16)x^4+1$ đạt cực tiểu tại $x=0$.
\choice
{\True $8$}
{Vô số}
{$7$}
{$9$}
\loigiai{
Ta có $y'=8x^7+5(m-4)x^4-4(m^2-16)x^3$. \\
Đặt $g(x)=8x^4+5(m-4)x-4(m^2-16)$. Có $2$ trường hợp cần xét liên quan $(m^2-16)$:
\begin{itemize}
\item Trường hợp 1: $m^2-16=0 \Leftrightarrow m=\pm 4$.
\begin{itemize}
\item[+] Khi $m=4$ ta có $y'=8x^7 \Rightarrow x=0$ là điểm cực tiểu.
\item[+] Khi $m=-4$ ta có $y'=x^4(8x^4-40) \Rightarrow x=0$ không là điểm cực tiểu.
\end{itemize}
\item Trường hợp 2: $m^2-16\ne 0 \Leftrightarrow m\ne \pm 4$. Khi đó $x=0$ không là nghiệm của $g(x)$.\\
Ta có $x^3$ đổi dấu từ $-$ sang $+$ khi qua $x_0=0$, do đó\\
$y'=x^3\cdot g(x)$ đổi dấu từ $-$ sang $+$ khi qua $x_0=0 \Leftrightarrow \lim\limits_{x \to 0} g(x)>0 \Leftrightarrow m^2-16<0$.
\end{itemize}
Kết hợp các trường hợp giải được ta nhận $m \in \{-3;-2;-1;0;1;2;3;4\}$.}
\end{ex}

\begin{ex}%[THPTGQ 2018, mã 102, MĐ4]%[2D1C2-6]
Có bao nhiêu giá trị nguyên của tham số $m$ để hàm số
\begin{eqnarray*}
y =x^8+ (m - 1)x^5- (m^2- 1)x^4+ 1
\end{eqnarray*}
đạt cực tiểu tại $x = 0$?
\choice
{$3$}
{\True $2$}
{Vô số}
{$1$}
\loigiai{
Ta có $ y'=8x^7+5(m-1)x^4-4(m^2-1)x^3+1=x^3\left[ 8x^4+5(m-1)x-4(m^2-1) \right]  $,
$$ y'=0\Leftrightarrow \hoac{&x=0\\&8x^4+5(m-1)x-4(m^2-1)=0.\ \ (*)} $$
\begin{itemize}
\item Nếu $ m=1 $ thì $ y'=8x^7 $, suy ra hàm số đạt cực tiểu tại $ x=0 $.
\item Nếu $ m=-1 $ thì $ $$y'=0\Leftrightarrow\hoac{&x=0\\&8x^4-10x=0}\Leftrightarrow\hoac{&x=0\mbox{ (nghiệm kép)}\\&x=\sqrt[3]{\dfrac{5}{4}.}}$
Do đó $x=0$ không phải là điểm cực trị.
\item Nếu $ m\ne\pm1 $ thì $ x=0 $ là nghiệm đơn.\\
Đặt $ g(x)=8x^4+5(m-1)x-4(m^2-1) $. Hàm số đã cho đạt cực tiểu tại $ x=0 $ khi chỉ khi $$ \lim_{x\to 0^{-}}g(x)>0\Leftrightarrow -4(m^2-1)>0\Leftrightarrow m^2-1<0\Leftrightarrow -1<m<1. $$
Vì $ m\in\mathbb{Z} $ nên $ m=0 $.
\end{itemize}
Vậy giá trị $ m $ thỏa mãn yêu cầu bài toán là $ m=0 $, $ m=1 $.
}
\end{ex}

\begin{ex}%[MĐ4]%[2D1C2-6]
Có bao nhiêu giá trị nguyên dương của tham số $ m $ không vượt quá $ 2019 $ để hàm số $ f(x) = \dfrac{x^2}{8} + \sqrt{x + m + 2} $ không có điểm cực trị?
\choice
{ $ 0 $}
{\True$ 1 $}
{$ 2018 $}
{$ 2019 $}
\loigiai{
Tập xác định  $ \mathscr{D} = [ - m - 2;+ \infty)$.\\
Ta thấy \allowdisplaybreaks{
\begin{eqnarray*}
&& f'(x) =  \dfrac{x}{4}  + \dfrac{1}{ 2\sqrt{x + m + 2} }, x \neq - m - 2 \\
& \Leftrightarrow & 4 f'(x) = x + \dfrac{2}{ \sqrt{x + m  + 2} } \\
& \Leftrightarrow & 4 f'(x) =  (x + m + 2) + \dfrac{1}{ \sqrt{x + m  + 2}} + \dfrac{1}{ \sqrt{x + m  + 2}} - (m + 2) \\
& \Leftrightarrow & 4f'(x) \geq 3 - (m + 2)\\
& \Leftrightarrow & f'(x) \geq \dfrac{1 - m}{4}. \quad \quad (1)
\end{eqnarray*}
}%
Đẳng thức $ (1) $ xảy ra $ \Leftrightarrow x + m + 2 = \dfrac{1}{ \sqrt{x + m + 2} } \Leftrightarrow x = - m - 1 \in \mathscr{D} \setminus \{ - m - 2 \} $.\\
Vì $ \lim \limits_{x \to + \infty} f'(x) = + \infty $ nên hàm số $ f(x) $ không có cực trị khi và chỉ khi $ 1 - m \geq 0 \Leftrightarrow m \leq 1 $.\\
Vì $ m $ nguyên dương và không vượt quá $ 2019 $ nên $ m = 1 $.\\
Vậy có đúng $ 1 $ giá trị $ m $ thỏa mãn đề bài.
}
\end{ex}

\begin{ex}%[Mức độ 4]%[Dự án giảng new 4in1, Trần Quang Thạnh]%[2D1C1-4]
Có bao nhiêu cặp số nguyên dương $(x;y)$ thoả mãn $y\leq 1000$ và $$\log\dfrac{x+1}{3y+1}\leq 9y^2-x^2+6y-2x?$$
\choice
{$1501100$}
{$1501300$}
{$1501400$}
{\True $1501500$}
\loigiai
{Ta có $$\log\dfrac{x+1}{3y+1}\leq 9y^2-x^2+6y-2x \Leftrightarrow \log(x+1)+(x+1)^2\leq \log(3y+1)+ (3y+1)^2.$$
Xét hàm $f(t)=\log t+t^2$ trên $(0;+\infty)$.\\
Ta có $f'(t)=\dfrac{1}{t \ln 10}+2 t>0$, $\forall t \in(0;+\infty)$.\\
Suy ra $f(t)$ là hàm đồng biến trên $t\in(0;+\infty)$.\\
Khi đó $(*)\Leftrightarrow f(x+1) \leq f\left(3y+1\right) \Leftrightarrow x+1 \leq 3y+1 \Leftrightarrow x \leq 3y$.\\
Vì $y \leq 1000$ nên ta có các trường hợp sau
\begin{itemize}
\item $y=1 \Rightarrow x \in\{1;2;3\}$.
\item $y=2 \Rightarrow x \in\{1;2;3;4;5;6\}$.\\
$\ldots$
\item $y=1000 \Rightarrow x \in\{1;2;\cdots;3000\}$.
\end{itemize}
Vậy số cặp nghiệm thoả mãn điều kiện đề bài là $3+6+9+\ldots+3000=1501500$.
}
\end{ex}

\begin{ex}%[Mức độ 3]%[Dự án bài giảng new 4in1, Trần Quang Thạnh]%[2D1C1-3]
Có bao nhiêu giá trị nguyên của tham số $m$ để hàm số $y=\dfrac{16-m^2}{(x+1)^2}$ đồng biến trên $(0;+\infty)$?
\choice
{\True $7$}
{$9$}
{Vô số}
{$11$}
\loigiai{
Ta có $y'=-\dfrac{2(16-m^2)}{(x+1)^3}$.\\
Nhận thấy $y'=0 \Leftrightarrow m=\pm 4$ và khi đó hàm số đã cho là hàm hằng.\\
Do đó, hàm số đã cho đồng biến trên đồng biến trên $(0;+\infty)$ khi và chỉ khi $y'>0$, với mọi $x>0$, tức là $16-m^<0$ hay $-4<m<4$.\\
Vậy có $7$ giá trị nguyên của tham số $m$ để hàm số đã cho đồng biến trên $(0;+\infty)$ là $-3; -2; -1; 0; 1; 2; 3$.
}
\end{ex}


\Closesolutionfile{ans}

\Opensolutionfile{ans}[ans/ansBTchoiceTF]

\TNTF
\begin{ex}%[2D1C5-7]
Cho hàm số $y=\dfrac{-x^2+2(m+1)x-5}{x-1}$. Xét tính đúng sai của các mệnh đề sau.
\choiceTF[t]
{\True Khi $m=0$ thì đồ thị hàm số có tiệm cận xiên là $y=-x+1$}
{\True Khi $m=0$ thì đồ thị hàm số không cắt $Ox$}
{Để hàm số có cực đại cực tiểu thì $m>2$}
{\True Khi $m=0$ thì hàm số có đồ thị là $(C)$. Biết rằng tồn tại điểm $M$ thuộc đồ thị $(C)$ sao cho $x_M>1$  và $IM$ ngắn nhất ($I$ là tâm đối xứng của $(C)$), khi đó $y_M<-4$}
\loigiai
{\begin{itemchoice}
\itemch Đúng. Khi $m=0$ thì $y=\dfrac{-x^2+2x-5}{x-1}=-x+1-\dfrac{4}{x-1}$ nên đồ thị có tiệm cận xiên $y=-x+1$.
\itemch Đúng. Khi $m=0$ thì $y=\dfrac{-x^2+2x-5}{x-1}$ và $y=0\Leftrightarrow -x^2+2x-5=0$ vô nghiệm nên đồ thị hàm số không cắt $Ox$.
\itemch Sai. Ta có $y'=\dfrac{-x^2+2x-2m+3}{(x-1)^2}$.\\
Hàm số có cực đại, cực tiểu khi phương trình $-x^2+2x-2m+3=0$ có $2$ nghiệm phân biệt khác $1$. Điều kiện tương đương là
$$\heva{& \Delta'=(-1)^2-2m+3>0 \\ & -1^2+2\cdot 1-2m+3\neq 0}\Leftrightarrow\heva{& 2m<4 \\ & 2m\neq 4}\Leftrightarrow m<2.$$
\itemch Đúng. Khi $m=0$ thì đồ thị $(C)$ của hàm số $y=\dfrac{-x^2+2x-5}{x-1}=-x+1-\dfrac{4}{x-1}$ có tiệm cận đứng là $x=1$ và tiệm cận xiên là $y=-x+1$. Suy ra giao điểm của hai tiệm cận là $I(1;0)$.\\
Gọi $M\left(x_M;y_M\right)$ là điểm thuộc $(C)$ có $x_M>1$.\\
Ta có $y_M=-x_M+1-\dfrac{4}{x_M-1}$ và \begin{eqnarray*}
IM^2&=&\left(x_M-1\right)^2+\left(-x_M+1\right)^2+\dfrac{16}{\left(x_M-1\right)^2}+8\\
&=&2\left(x_M-1\right)^2+\dfrac{16}{\left(x_M-1\right)^2}+8\\
&\geq & 8\sqrt{2}+8.
\end{eqnarray*}
Dấu ``$=$'' xảy ra khi $2\left(x_M-1\right)^2=\dfrac{16}{\left(x_M-1\right)^2}\Leftrightarrow\left(x_M-1\right)^2=8\Leftrightarrow x_M=1+\sqrt[4]{8}$ (do $x_M>1$).\\
Suy ra $IM$ ngắn nhất bằng $\sqrt{8\sqrt{2}+8}$ khi $x_M=1+\sqrt[4]{8}$.\\
Khi đó $y_M=-\sqrt[4]{8}-\dfrac{4}{\sqrt[4]{8}}<-4$.
\end{itemchoice}
}
\end{ex}

\begin{ex}%[2D1C5-6]
Cho hàm số $y=\dfrac{x^2+3 x+3}{x+2}$ có đồ thị là $(C)$. Xét tính đúng sai của các mệnh đề sau.
\choiceTF[t]
{Biết hàm số có $2$ điểm cực trị khi đó tổng của giá trị cực đại và giá trị cực tiểu bằng $-4$}
{\True Đường tiệm cận xiên của đồ thị hàm số đi qua điểm $A(0;1)$}
{\True Gọi $\Delta$ là tiếp tuyến của $(C)$ và vuông góc với đường thẳng $x-3 y-6=0$. Khi đó $\Delta$ đi qua điểm $B\left(-\dfrac{3}{2};\dfrac{3}{2}\right)$}
{Để phương trình $x^2+3x+3=m|x+2|$ có $4$ nghiệm phân biệt thì $m>2$}
\loigiai
{\begin{itemchoice}
\itemch Sai. Tập xác định $\mathscr{D}=\mathbb{R}\setminus\{-2\}$.\\
Ta có $y'=\dfrac{x^2+4x+3}{(x+2)^2}$ và $y'=0\Leftrightarrow x^2+4x+3=0\Leftrightarrow\hoac{& x=-1 \\ & x=-3}$.\\
Bảng biến thiên
\begin{center}
\begin{tikzpicture}
\tkzTabInit[nocadre=false, lgt=1.2, espcl=2.5, deltacl=0.6]{$x$/0.6,$y'$/0.6,$y$/2}
{$-\infty$, $-3$, $-2$, $-1$, $+\infty$}
\tkzTabLine {,+,0,-,d,-,0,+,}
\tkzTabVar{-/$-\infty$, +/3, -D+/$-\infty$/$+\infty$, -/$-1$, +/$+\infty$}
\end{tikzpicture}
\end{center}
Vậy tổng của giá trị cực đại và giá trị cực tiểu là $3+(-1)=2$.
\itemch Đúng. Ta có $y=x+1+\dfrac{1}{x+2}$ nên đồ thị $(C)$ có tiệm cận xiên là $y=x+1$. Tiệm cận xiên này đi qua $A(0;1)$.
\itemch Đúng. Đường thẳng $x-3y-6=0$ có hệ số góc bằng $\dfrac{1}{3}$ nên tiếp tuyến $\Delta$ có hệ số góc bằng $-3$.\\
Gọi $M\left(x_0;y_0\right)$ là tiếp điểm. Ta có hệ số góc của $\Delta$ là $y'(x_0)=\dfrac{x_0^2+4x_0+3}{\left(x_0+2\right)^2}$. Khi đó
\allowdisplaybreaks\begin{eqnarray*}
y'(x_0)=-3&\Leftrightarrow & \dfrac{x_0^2+4x_0+3}{\left(x_0+2\right)^2}=-3\\
&\Leftrightarrow &\heva{& x_0\neq -2 \\ & x_0^2+4x_0+3=-3\left(x_0^2+4x_0+4\right)}\\
&\Leftrightarrow &\heva{& x_0\neq -2 \\ & 4x_0^2+16x_0+15=0}\\
&\Leftrightarrow &\heva{& x_0\neq -2 \\ & \hoac{& x_0=-\dfrac{3}{2} \\ & x_0=-\dfrac{5}{2}}}\\
&\Leftrightarrow &\hoac{& x_0=-\dfrac{3}{2}\Rightarrow y_0=\dfrac{3}{2} \\ & x_0=-\dfrac{5}{2}\Rightarrow y_0=-\dfrac{7}{2}.}
\end{eqnarray*}
Suy ra có tiếp tuyến $\Delta$ đi qua điểm $B\left(-\dfrac{3}{2};\dfrac{3}{2}\right)$.
\itemch Sai. Nhận thấy $x=-2$ không là nghiệm của phương trình $x^2+3x+3=m|x+2|$ nên ta viết lại $\dfrac{x^2+3x+3}{|x+2|}=m$. Đây là phương trình hoành độ giao điểm giữa đồ thị hàm số $y=\dfrac{x^2+3x+3}{|x+2|}$ và đường thẳng $y=m$.\\
Gọi $(C)$ là đồ thị của hàm số $y=\dfrac{x^2+3x+3}{x+2}$.\\
Ta có $y=\dfrac{x^2+3x+3}{|x+2|}=\heva{& \dfrac{x^2+3x+3}{x+2} &&\text{nếu } x\geq -2 \\ & -\dfrac{x^2+3x+3}{x+2}&& \text{nếu } x<-2.}$\\
Do đó, đồ thị $(C')$ của hàm số $y=\dfrac{x^2+3x+3}{|x+2|}$ gồm phần đồ thị $(C_1)$ trùng với $(C)$ khi $x\geq -2$ và $(C_2)$ đối xứng với $(C)$ qua trục $Ox$ khi $x<-2$.
\begin{center}
\begin{tikzpicture}[scale=1, font=\footnotesize, line join=round, line cap=round,x=0.5cm,y=0.5cm,>=stealth]
\def \xmin{-8.0};
\def \xmax{6.1};
\def \ymin{-7.0};
\def \ymax{6.5};
\def\f(#1){0.03*(#1)^4-0.07*(#1)^3-0.44*(#1)^2+0.9*(#1)+1};
\def\g(#1){0.03*(#1-2)^4-0.07*(#1-2)^3-0.44*(#1-2)^2+0.9*(#1-2)+1};
\draw[->] (\xmin, 0.) -- (\xmax,0.) node[anchor=north] {$x$};
\draw[->] (0.,\ymin) -- (0.,\ymax) node[anchor=west] {$y$};
\clip(\xmin,\ymin) rectangle (\xmax,\ymax);
\begin{scope}
\clip (\xmin,\ymin) rectangle (6,0);
\draw[smooth,dashed,samples=100] plot[domain=\xmin-0.1:-2.01] (\x,{((\x)^2+3*(\x)+3)/((\x)+2)});
\end{scope}
\begin{scope}[yscale=-1]
\clip (\xmin,\ymin) rectangle (6,0);
\draw[smooth,samples=100] plot[domain=\xmin-0.1:-2.01] (\x,{((\x)^2+3*(\x)+3)/((\x)+2)});
\path[font=\tiny,postaction={decorate,decoration={text along path,text align=right, raise=1mm,text={|\tiny|{{$(C_2)$}}}}}] plot[domain=-6.5:-6.0] (\x,{((\x)^2+3*(\x)+3)/((\x)+2)});
\end{scope}
\begin{scope}
\clip (\xmin,0) rectangle (\xmax,\ymax);
\draw[smooth,samples=100] plot[domain=-1.9:\xmax] (\x,{((\x)^2+3*(\x)+3)/((\x)+2)});
\path[font=\tiny,postaction={decorate,decoration={text along path,text align=right, raise=1mm,text={|\tiny|{{$(C_1)$}}}}}] plot[domain=4.5:5] (\x,{((\x)^2+3*(\x)+3)/((\x)+2)});
\end{scope}
\draw (-2,\ymin)--(-2,\ymax);
\draw[dashed] (-3,0)node[below left]{$-3$}|-(0,3)node[right]{$3$} (-3,0)|-(0,-3)node[right]{$-3$} (-1,0)node[below]{$-1$}|-(0,1)node[right]{$1$};
\draw[smooth,samples=100] plot[domain=\xmin:\xmax] (\x,{(\x)+1});
\draw[fill=black] (0,0) circle (1pt) node[below right] {$O$} (-3,3) circle (1pt) (-3,-3) circle (1pt) (0,-3) circle (1pt) (0,3) circle (1pt) (0,1) circle (1pt) (-1,0) circle (1pt) (-1,1) circle (1pt) (-3,0) circle (1pt) (-2,0) circle (1pt);
\end{tikzpicture}
\end{center}
Dựa vào đồ thị, phương trình đã cho có $4$ nghiệm phân biệt khi và chỉ khi $m>3$.
\end{itemchoice}
}
\end{ex}

\begin{ex}%[2D1C5-6]
Cho hàm số $y=x-\dfrac{1}{x+1}$ có đồ thị là $(C)$. Xét tính đúng sai của các mệnh đề sau.
\choiceTF[t]
{Đồ thị của hàm số có tiệm cận đứng là $x=1$}
{\True Đồ thị hàm số cắt trục $Oy$ tại $M$. Phương trình tiếp tuyến của $(C)$ tại $M$ là $y=2x-1$}
{Tồn tại hai tiếp tuyến của đồ thị vuông góc với nhau}
{\True Để đường thẳng $y=k$ cắt $(C)$ tại hai điểm phân biệt $A$ và $B$ sao cho $OA\perp OB$ thì $k$ là nghiệm của phương trình $k^2-k-1=0$}
\loigiai
{\begin{itemchoice}
\itemch Sai. Đồ thị  $(C)$ có tiệm cận đứng là $x=-1$.
\itemch Đúng. Đồ thị $(C)$ cắt trục $Oy$ tại $M(0;-1)$.\\
Ta có $y'=1+\dfrac{1}{(x+1)^2}\Rightarrow y'(0)=2$.\\
Phương trình tiếp tuyến của $(C)$ tại $M$ là $y=2x-1$.
\itemch Sai. Tiếp tuyến của đồ thị $(C)$ tại tiếp điểm $M_1(x_1;y_1)$ có hệ số góc $k_1=y'\left(x_1\right)=1+\dfrac{1}{(x_1+1)^2}>0$.\\
Tiếp tuyến của đồ thị $(C)$ tại tiếp điểm $M_2(x_2;y_2)$ có hệ số góc $k_2=y'\left(x_2\right)=1+\dfrac{1}{(x_2+1)^2}>0$.\\
Khi đó $k_1k_2>0$ nên không tồn tại hai tiếp tuyến của đồ thị vuông góc với nhau.
\itemch Đúng. Phương trình hoành độ giao điểm giữa đồ thị $(C)$ và đường thẳng $y=k$ là $$x-\dfrac{1}{x+1}=k\Leftrightarrow\heva{& x\neq -1 \\ & x^2+x-1=k(x+1).\quad (1)}\quad (I)$$
Nhận thấy $x=-1$ không thỏa mãn $(1)$ nên $$(I)\Leftrightarrow x^2+(1-k)x-1-k=0.\quad (2)$$
Phương trình $(2)$ có $\Delta=(1-k)^2+4(1+k)=k^2+2k+5=(k+1)^2+4>0,\ \forall k$.\\
Do đó, đường thẳng $y=k$ luôn cắt đồ thị $(C)$ tại hai điểm phân biệt $A(x_A;k)$, $B(x_B;k)$ với $x_A$, $x_B$ là nghiệm của phương trình $(2)$.\\
Theo Vi-et thì $x_A x_B=-1-k$.\\
Ta có $OA\perp OB\Leftrightarrow\overrightarrow{OA}\cdot\overrightarrow{OB}=0\Leftrightarrow x_Ax_B+k^2=0\Leftrightarrow -1-k+k^2=0$.\\
Vậy $OA\perp OB$ thì $k$ là nghiệm của phương trình $k^2-k-1=0$.
\end{itemchoice}
}
\end{ex}

\begin{ex}%[2D1C5-2]
Cho hàm số $y= \log_3 \left(\dfrac{1}{x} \right)$ có đồ thị $(C_1)$ và hàm số $y=f(x)$ có đồ thị $(C_2)$ đối xứng với $(C_1)$ qua gốc tọa độ. Xét tính đúng sai của các mệnh đề sau.
\choiceTF[t]
{Hàm số $y=f(x)$ có tập xác định $\mathscr{D}=(0;+\infty)$}
{\True Đồ thị hàm số $y=f(x)$ đi qua điểm $M(-3;1)$}
{Đồ thị hàm số $y=f(x)$ có tiệm cận ngang là trục hoành}
{\True Hàm số $y=\left|f(x)\right|$ nghịch biến trên $(-\infty;-1)$}
\loigiai{
\begin{itemchoice}
\itemch Sai. Hàm số $y=\log_3\left(\dfrac{1}{x}\right)=-\log_3 x$ có tập xác định là $(0;+\infty)$ nên hàm số $y=f(x)$ có tập xác định là $(-\infty;0)$.
\itemch Đúng. Hàm số $y=\log_3\left(\dfrac{1}{x}\right)$ đi qua điểm $N(3;-1)$.\\
Ta có $M(-3;1)$ đối xứng với $N(3;-1)$ qua gốc tọa độ $O$ nên $M$ thuộc đồ thị hàm số $y=f(x)$.
\itemch Sai. Đồ thị hàm số $y=\log_3\left(\dfrac{1}{x}\right)=-\log_3 x$ chỉ có tiệm cận đứng là trục $Oy$ nên đồ thị $(C_2)$ cũng chỉ có tiệm cận đứng là $Oy$.
\itemch Gọi $M(x_0;y_0)$ là điểm thuộc $(C_2)$, $x_0<0$. Khi đó $y_0=f(x_0)$\\
Điểm $N$ đối xứng với $M$ qua gốc tọa độ $O$ có tọa độ là $N(-x_0;-y_0)$.\\
Ta có $N$ thuộc $(C_1)$ nên $-y_0=\log_3\left(\dfrac{1}{-x_0}\right)$ hay $y_0=\log_3(-x_0)$.\\
Do đó $f(x_0)=\log_3(-x_0)$ với $x_0<0$.\\
Suy ra $f(x)=\log_3(-x)$ với $x<0$.\\
Khi đó $y=|f(x)|=|\log_3(-x)|=\heva{&\log_3(-x),&&x \leq -1\\& -\log_3(-x),&&-1<x<0.}$\\
Suy ra $y'=\heva{&\dfrac{1}{x\ln 3}, &&x<-1\\ &-\dfrac{1}{x\ln 3},&&-1<x<0.}$\\
Như thế $y'<0$ khi $x<-1$. Vậy hàm số $y=\left|f(x)\right|$ nghịch biến trên $(-\infty;-1)$.
\end{itemchoice}
}
\end{ex}

\begin{ex}%[2D1C5-2]
Cho hàm số $y=f(x)$ có đồ thị đối xứng với đồ thị hàm số $y=2^x+x$ qua đường thẳng $y=x$. Xét tính đúng sai của các mệnh đề sau.
\choiceTF[t]
{\True Hàm số $y=f(x)$ có tập xác định $\mathscr{D}=\mathbb{R}$}
{Đồ thị hàm số $y=f(x)$ không có đường tiệm cận xiên}
{\True Đồ thị hàm số $y=f(x)$ nên bên dưới đường thẳng $y=x$}
{\True Đồ thị hàm số $y=f(x)$ là một đường đi lên từ trái sang phải}

\loigiai
{\begin{itemchoice}
\itemch Đúng. Hàm số $y=2^x+x$  xác định và liên tục với mọi $x$.\\
Ta có $\lim\limits_{x\to -\infty}y=-\infty$ và $\lim\limits_{x\to +\infty}y=+\infty$ nên nó có tập giá trị là $(-\infty;+\infty)$.\\
Vì đồ thị hàm số $y=f(x)$ đối xứng với đồ thị hàm số $y=2^x+x$ qua đường thẳng $y=x$ nên tập giá trị của hàm số $y=2^x+x$ là tập xác định của hàm số $y=f(x)$.\\
Vậy hàm số $y=f(x)$ có tập xác định $\mathscr{D}=\mathbb{R}$.
\itemch Sai. Hàm số $y=2^x+x$ có tập xác định $\mathscr{D}=\mathbb{R}$ và $\lim\limits_{x\to -\infty}(y-x)=\lim\limits_{x\to -\infty} 2^x=0$ nên đồ thị có tiệm cận xiên $y=x$.\\
Do  đồ thị hàm số $y=f(x)$ đối xứng với đồ thị hàm số $y=2^x+x$ qua đường thẳng $y=x$ nên đồ thị hàm số $y=f(x)$ cũng có tiệm cận xiên.
\itemch Đúng. Ta có $2^x+x>x$, $\forall x\in\mathbb{R}$ nên đồ thị hàm số $y=2^x+x$ nằm phía trên đường thẳng $y=x$.\\
Vì đồ thị hàm số $y=f(x)$ đối xứng với đồ thị hàm số $y=2^x+x$ qua đường thẳng $y=x$ nên đồ thị của $y=f(x)$ nằm bên dưới đường thẳng $y=x$.
\itemch Gọi $M(x_0;y_0)$ là điểm tùy ý thuộc đồ thị hàm số $y=f(x)$. Khi đó, $y_0=f(x_0)$.\\
Ta có điểm đối xứng với $M$ qua đường thẳng $y=x$ là $N(y_0;x_0)$.\\
Hàm số $y=2^x+x$ có $y'=2^x\ln x+1>0$, $\forall x\in\mathbb{R}$ nên đồng biến trên $\mathbb{R}$.\\
Lấy hai điểm $M_1(x_1,y_1)$ và $M_2(x_2;y_2)$ thuộc đồ thị hàm số $y=f(x)$ sao cho $x_1<x_2$.\hfill $(1)$\\
Gọi $N_1$, $N_2$ lần lượt là điểm đối xứng của $M_1$, $M_2$ qua đường thẳng $y=x$.\\
Khi đó $N_1(y_1;x_1)$, $N_2(y_2;x_2)$ thuộc đồ thị hàm số $y=2^x+x$ và do hàm số này đồng biến nên từ $x_1<x_2$ suy ra $y_1<y_2$ hay $f(x_1)<f(x_2)$.\hfill $(2)$\\
Từ $(1)$ và $(2)$ suy ra hàm số $y=f(x)$ là hàm số đồng biến.\\
Vậy đồ thị hàm số $y=f(x)$ là đường đi lên từ trái sang phải.
\end{itemchoice}
}
\end{ex}

\begin{ex}%[Dự án TL12New-4in1-NCT]%[2D1C4-1]
\immini
{
Cho hàm số bậc ba $f(x)$ có đồ thị như hình vẽ.
Xét tính đúng sai của các khẳng định sau.
\choiceTF
{\True Đồ thị hàm số $ g_1(x)=\dfrac{1}{f(x)} $ có $3$ tiệm cận đứng}
{Đồ thị hàm số $ g_2(x)=\dfrac{1}{f(x)-2} $ có $3$ tiệm cận đứng}
{\True Đồ thị hàm số $ g_3(x)=\dfrac{x^2-x}{f(x)} $ có $2$ tiệm cận đứng}
{\True Đồ thị hàm số $ g_4(x)=\dfrac{x^2-x}{\left[f(x)\right]^2-2f(x)} $ có $4$ tiệm cận đứng và $1$ tiệm cận ngang}
}{
\begin{tikzpicture}[scale=.6,>=stealth]
\draw[->](-1.9,0)--(3.5,0)node[below]{$x$};
\draw[->](0,-2.9)--(0,2.9)node[left]{$y$};
\draw[dashed](2,0)--(2,-2)circle(1.5 pt)--(0,-2);
\node at (1,0) [above ] {\footnotesize $1$};
\node at (2,0) [below left] {\footnotesize $2$};
\node at (0,2) [right] {\footnotesize $2$};
\node at (0,-2) [left] {\footnotesize $-2$};
\draw [fill] (0,0) circle (1.5 pt)node[below right] {\footnotesize $O$};
\draw[smooth,samples=100,domain=-1.1:3.1] plot(\x,{(\x)^3-3*(\x)^2+2});
\end{tikzpicture}
}
\loigiai{
\begin{center}
\begin{tikzpicture}[scale=.6,>=stealth]
\draw[->](-1.9,0)--(3.5,0)node[below]{$x$};
\draw[->](0,-2.9)--(0,2.9)node[left]{$y$};
\draw[dashed](2,0)--(2,-2)circle(1.5 pt)--(0,-2) (-1.9,2)--(3.5,2) (3,2)--(3,0);
\node at (1,0) [above ] {\footnotesize $1$};
\node at (2,0) [below left] {\footnotesize $2$};
\node at (0,2) [right] {\footnotesize $2$};
\node at (0,-2) [left] {\footnotesize $-2$};
\node at (3,0) [below] {\footnotesize $3$};
\draw [fill] (0,0) circle (1.5 pt)node[above right] {\footnotesize $O$};
\draw [fill] (-0.75,0) circle (1.5 pt)node[below right] {\footnotesize $a$};
\draw [fill] (2.7,0) circle (1.5 pt)node[above left] {\footnotesize $b$};
\draw[smooth,samples=100,domain=-1.1:3.1] plot(\x,{(\x)^3-3*(\x)^2+2});
\end{tikzpicture}
\end{center}
\begin{itemchoice}
\itemch Dựa vào đồ thị ta thấy $f(x)=0\Leftrightarrow\hoac{&x=a<0\\&x=1\\&x=b>2}$.\\
Từ đó suy ra đồ thị hàm số $g_1(x)$ có $3$  tiệm cận đứng là $ x=a,  x=1, x=b$.
\itemch Dựa vào đồ thị ta thấy $f(x)=2\Leftrightarrow\hoac{&x=0 \text{ nghiệm bội 2}\\&x=b>2}$.\\
Từ đó suy ra đồ thị hàm số $g_2(x)$ có $2$  tiệm cận đứng là $ x=0, x=b$.
\itemch Dựa vào đồ thị ta thấy $f(x)=0\Leftrightarrow\hoac{&x=a<0\\&x=1\\&x=b>2}$.\\
Từ đó suy ra đồ thị hàm số $g_3(x)$ có $2$  tiệm cận đứng là $ x=a, x=b$.
\itemch Dựa vào đồ thị ta thấy $f^2(x)-2f(x)=0\Leftrightarrow\hoac{&f(x)=0\\&f(x)=2}\Leftrightarrow \hoac{&x=a \quad (a<0)\\&x=1\\&x=b\quad  (b>2)\\&x=0 \\&x=3.}$\\
Do đó ta viết $f^2(x)-2f(x)=k (x-a)(x-1)(x-b)x^2(x-3)$.\\
Xét hàm số $  g(x)\dfrac{x^2-x}{\left[f(x)\right]^2-2f(x)} =\dfrac{x(x-1)}{k (x-a)(x-1)(x-b)x^2(x-3)} $.\\
Tập xác định $ \mathscr{D}=\mathbb{R}\backslash\{a;1;b;0;3\} $.\\
Từ đó suy ra đồ thị hàm số $g(x)$ có $4$  tiệm cận đứng là $ x=a,  x=0, x=b, x=3 $ và $1$ tiệm cận ngang là $y=0$.
\end{itemchoice}
}
\end{ex}

\begin{ex}%[Dự án TL12New-4in1-NCT]%[2D1C4-1]
\immini{Cho hàm số $y=f(x)=ax^4+bx^2+c (a\ne 0)$ có đồ thị như hình vẽ.
Xét tính đúng sai của các khẳng định sau.
\choiceTF
{\True Đồ thị hàm số $g(x)=\dfrac{2025(x-2)^3\sqrt{x^2+2026}}{f(x)}$ có $1$ tiệm cận đứng}
{Đồ thị hàm số $g(x)=\dfrac{2025(x+2)^3\sqrt{x^2+2026}}{f(x)}$ có $2$ tiệm cận đứng}
{Đồ thị hàm số $g(x)=\dfrac{2025(x+2)^3\sqrt{x^2+2026}}{f(x)}$ có $1$ tiệm cận ngang}
{\True Đồ thị hàm số $g(x)=\dfrac{2025(x-2)^3\sqrt{x^2+2026}}{f(x)}$ có $2$ tiệm cận ngang}
}{
\begin{tikzpicture}[scale=0.5,>=stealth]
\path
(1,1) coordinate (A);
\draw[->](-3.5,0)--(3.5,0)node[below]{$x$};
\draw[->](0,-1)--(0,5)node[left]{$y$};
\draw[smooth,samples=200,domain=-3:3]plot(\x,{3/16*(\x)^4-3/2*(\x)^2+3});
\draw(0,0)node[below left]{$O$} (-2,0)node[below]{$-2$} (2,0)node[below]{$2$} (0,3)node[ left]{$3$};
\end{tikzpicture}}
\loigiai{
Ta có $f(x)=0\Leftrightarrow\hoac{&x=-2\\&x=2.}$\\
Do đó ta viết $f(x)=a(x+2)^2(x-2)^2$.
\begin{itemchoice}
\itemch Xét hàm số  $g(x)=\dfrac{2025(x-2)^3\sqrt{x^2+2026}}{f(x)}=\dfrac{2025(x-2)^3\sqrt{x^2+2026}}{a(x+2)^2(x-2)^2}$.\\
Hàm số $g(x)$ có tập xác định $\mathscr{D}=\mathbb{R}\backslash\{-2;2\}$.\\
Từ đó suy ra đồ thị hàm số $g(x)$ có một tiệm cận đứng là $x=-2$.
\itemch Xét hàm số  $g(x)=\dfrac{2025(x+2)^3\sqrt{x^2+2026}}{f(x)}=\dfrac{2025(x+2)^3\sqrt{x^2+2026}}{a(x+2)^2(x-2)^2}$.\\
Hàm số $g(x)$ có tập xác định $\mathscr{D}=\mathbb{R}\backslash\{-2;2\}$.\\
Từ đó suy ra đồ thị hàm số $g(x)$ có một tiệm cận đứng là $x=2$.
\itemch Xét hàm số  $g(x)=\dfrac{2025(x+2)^3\sqrt{x^2+2026}}{f(x)}=\dfrac{2025(x+2)^3\sqrt{x^2+2026}}{a(x+2)^2(x-2)^2}$.\\
Hàm số $g(x)$ có tập xác định $\mathscr{D}=\mathbb{R}\backslash\{-2;2\}$.\\
Từ đó suy ra đồ thị hàm số $g(x)$ có hai tiệm cận ngang là $y=-\dfrac{2025}{a}$, $y=\dfrac{2025}{a}$.
\itemch Xét hàm số  $g(x)=\dfrac{2025(x-2)^3\sqrt{x^2+2026}}{f(x)}=\dfrac{2025(x-2)^3\sqrt{x^2+2026}}{a(x+2)^2(x-2)^2}$.\\
Hàm số $g(x)$ có tập xác định $\mathscr{D}=\mathbb{R}\backslash\{-2;2\}$.\\
Từ đó suy ra đồ thị hàm số $g(x)$ có hai tiệm cận ngang là $y=-\dfrac{2025}{a}$, $y=\dfrac{2025}{a}$.
\end{itemchoice}
}
\end{ex}

\begin{ex}%[BG-12NEW-4in1, Nguyen Huynh]%[2D1C4-1]
Cho hàm số $y=\dfrac{x-3}{x+1}$ có đồ thị $(C)$. Xét tính đúng sai của các mệnh đề sau
\choiceTF[t]
{ Hàm số đã cho đồng biến trên $\mathbb R\setminus\{1\}$}
{Đồ thị của hàm số chỉ có tiệm cận ngang là $y=3$}
{\True Hai đường tiệm cận của đồ thị hàm số giao nhau tại điểm $I\left(-1;1 \right)$}
{\True Có hai điểm $M$ trên $(C)$ sao cho tiếp tuyến tại $M$ của $(C)$ tạo với hai đường tiệm cận của $(C)$ một tam giác có bán kính đường tròn nội tiếp lớn nhất}
\loigiai{
\begin{itemchoice}
\itemch Ta có $y=1-\dfrac{4}{x+1} \Rightarrow y'=\dfrac{4}{(x+1)^2}>0$ với $x \neq-1$.
\\Do đó hàm số đã cho đồng biến trên từng khoảng xác định.
\itemch Ta có $\lim\limits_{x\to +\infty}f(x)=\lim\limits_{x\to+\infty}\dfrac{x-3}{x+1}=1$ và $\lim\limits_{x\to -\infty}f(x)=\lim\limits_{x\to-\infty}\dfrac{x-3}{x+1}=1$.
Suy ra đường thẳng $y=1$ là tiệm cận ngang của đồ thị $(C)$.
\itemch	Do $\lim\limits_{x\to (-1)^+}f(x)=\lim\limits_{x\to(-1)^+}\dfrac{x-3}{x+1}=-\infty$ nên đường thẳng $x=-1$ là tiệm cận đứng của đồ thị hàm số $y=f(x)$.
\\Vậy $I(-1;1)$ là giao điểm của hai tiệm cận của đồ thị $(C)$.
\itemch  Gọi $M\left(a ; 1-\dfrac{4}{a+1}\right), a \neq-1$.\\
Phương trình tiếp tuyến của $(C)$ tại $M$ là $$d\colon y=\dfrac{4}{(a+1)^2}(x-a)+1-\dfrac{4}{a+1}.$$
Gọi $A$ và $B$ lần lượt là giao điểm của tiếp tuyến $d$ với đường tiệm cận đứng và tiệm cận ngang.\\
Giao điểm của $d$ và tiệm cận đứng là $A\left(-1 ; 1-\dfrac{8}{a+1}\right)$.\\
Giao điểm của $d$ và tiệm cận ngang là $B(2 a+1 ; 1)$.\\
Suy ra $I A=\dfrac{8}{|a+1|}, I B=2|a+1|, A B=\sqrt{4(a+1)^2+\dfrac{64}{(a+1)^2}}$.\\
Vì $\triangle I A B$ vuông tại $I$ nên $S_{\triangle I A B}=\dfrac{1}{2} I A \cdot I B=8$.\\
Nửa chu vi của $\triangle I A B$ là $p=\dfrac{I A+I B+A B}{2}=\dfrac{4}{|a+1|}+|a+1|+\sqrt{(a+1)^2+\dfrac{16}{(a+1)^2}}$.\\
Bán kính đường tròn nội tiếp $\Delta I A B$ là $r=\dfrac{S_{\Delta I A B}}{p}$ nên $r$ lớn nhất khi $p$ nhỏ nhất.\\
Áp dụng bất đắng AM-GM ta có
$$
\dfrac{4}{|a+1|}+|a+1|+\sqrt{(a+1)^2+\dfrac{16}{(a+1)^2}} \geq 2 \sqrt{\dfrac{4}{|a+1|} \cdot|a+1|}+\sqrt{2 \cdot \sqrt{(a+1)^2 \cdot \dfrac{16}{(a+1)^2}}}=4+2 \sqrt{2}.
$$
Suy ra $p \geq 4+2 \sqrt{2}$.\\
$p$ đạt giá trị nhỏ nhất bằng $4+2 \sqrt{2}$ khi $\dfrac{4}{|a+1|}=|a+1| \Leftrightarrow(a+1)^2=4 \Leftrightarrow\hoac{&a=1 \\& a=-3.}$\\
Vậy có hai điểm thỏa mãn yêu cầu bài toán là $M(1 ;-1), M(-3 ; 3)$.
\end{itemchoice}
}
\end{ex}


\Closesolutionfile{ans}

\Opensolutionfile{ans}[ans/ansBTshortans]

\TNSA
\begin{ex}%[2D1C5-7]
Trong mặt phẳng $Oxy$, xét tứ giác tứ giác $ABCD$ có các đỉnh có hoành độ là các số nguyên liên tiếp và nằm trên đồ thị của hàm số $y=\ln x$. Biết diện tích tứ giác $ABCD$ bằng $\ln \dfrac{91}{90}$, tính tổng các chữ số của hoành độ đỉnh xa gốc tọa độ nhất.
\shortans{$6$}
\loigiai{
\immini
{
Giả sử hoành độ của các đỉnh của tứ giác lần lượt là $a$, $a+1$, $a+2$, $a+3$ ($a\in\mathbb{N}^*$) tương ứng với các đỉnh $A$, $B$, $C$, $D$.\\
Khi đó $A(a;\ln a)$, $B(a+1;\ln (a+1))$, $C(a+2;\ln (a+2))$, $D(a+3;\ln (a+3))$.\\
Xét các điểm $M(a;0)$, $N(a+1;0)$, $P(a+2;0)$, $Q(a+3;0)$ thì các tứ giác $ABNM$, $BCPN$, $CDQP$ và $ADQM$ là các hình thang vuông. Khi đó
}
{
\begin{tikzpicture}[line join=round, line cap = round, >=stealth, scale=.8,font=\footnotesize,transform shape]
\pgfmathsetmacro{\a}{ln(2)/ln(1.5)};
\pgfmathsetmacro{\b}{ln(3)/ln(1.5)};
\pgfmathsetmacro{\c}{ln(4)/ln(1.5)};
\pgfmathsetmacro{\d}{ln(5)/ln(1.5)}
\foreach \x/\y/\z/\g in
{
2/\a/A/135,3/\b/B/90,4/\c/C/90,5/\d/D/90,
2/0/M/-90,3/0/N/-90,4/0/P/-90,5/0/Q/-90,0/0/O/-135
}
\draw[fill=black] (\x,\y) circle(1pt) coordinate (\z) ($(\z)+(\g:3mm)$) node{$\z$};
\draw[->] (-.5,0)--(6,0) node[anchor=north]{$x$};
\draw[->] (0,-.5)--(0,4.3) node[anchor=east]{$y$};
\draw[samples=100,domain=.9:5.5] plot(\x,{ln(\x)/ln(1.5)});
\draw (M)--(A) (N)--(B) (P)--(C) (Q)--(D) (A)--(B)--(C)--(D)--(A);
\end{tikzpicture}
}
\begin{eqnarray*}
2S_{ABCD} &=& 2S_{ABNM} + 2S_{BCPN} + 2S_{CDQP} - 2S_{ADQM}\\
&= & \left[ \ln a + \ln (a+1) \right] + \left[ \ln (a+1) + \ln (a+2) \right] + \left[ \ln (a+2) + \ln (a+3) \right] - 3\left[\ln a + \ln (a+3)\right]\\
&= &2\ln \dfrac{(a+1)(a+2)}{a(a+3)}.
\end{eqnarray*}
Kết hợp với $S_{ABCD}=\ln \dfrac{91}{90}$ ta có
$$
\dfrac{(a+1)(a+2)}{a(a+3)} = \dfrac{91}{90} \Leftrightarrow 91a(a+3) = 90(a+1)(a+2)
\Leftrightarrow a^2 + 3a - 180 = 0 \Leftrightarrow \hoac{&a=12\text{ (thỏa mãn)}\\ &a=-15\text{ (loại)}.}
$$
Suy ra đỉnh xa gốc tọa độ nhất là $D(15;\ln 15)$.\\
Vậy $1+5=6$.
}
\end{ex}

\begin{ex}%[2D1C5-7]
Cho hàm số $y=\dfrac{x^2+mx+m^2-2m-4}{x-2}$ có đồ thị $(C)$.
Tìm $m$ để đồ thị $(C)$ có hai điểm cực trị và hai điểm cực trị cách đều đường thẳng $\Delta\colon 2 x+y+1=0$.
\shortans{$-9$}
\loigiai{
Tập xác định $\mathscr{D}=\mathbb{R}\setminus\{2\}$.\\
Ta có  $y'=\dfrac{x^2-4x+4-m^2}{(x-2)^2}$.\\
Dấu của $y'$ là dấu của $g(x)=x^2-4x+4-m^2$.\\
Hàm số có hai điểm cực trị khi và chỉ khi phương trình $g(x)=0$ có hai nghiệm phân biệt khác $2$. Điều kiện tương đương là
$$\heva{& \Delta'=4-4+m^2=m^2>0\\ &
4-8+4-m^2 \neq 0} \Leftrightarrow m \neq 0.$$
Nghiệm của $g(x)=0$ là $x_1=2-m$, $x_2=2+m$, suy ra hai điểm cực trị của đồ thị hàm số là $A(2-m;4-m)$, $B(2+m;4+3m)$.\\
Khoảng cách từ $A$, $B$ đến đường thẳng $\Delta$ lần lượt là $\mathrm{d}(A,\Delta)=\dfrac{|9-3m|}{\sqrt{5}}$ và $\mathrm{d}(B,\Delta)=\dfrac{|9+5m|}{\sqrt{5}}$.\\
Khi đó $$\mathrm{d}(A,\Delta)=\mathrm{d}(B,\Delta)\Leftrightarrow |9-3m|=|9+5m|\Leftrightarrow\hoac{& 9-3m=9+5m\\ &9-3m=-9-5m}\Leftrightarrow\hoac{& m=0 \\ & m=-9.}$$
So với điều kiện $m \neq 0$ ta nhận $m=-9$.\\
Vậy giá trị $m$ cần tìm là $m=-9$.
}
\end{ex}

\begin{ex}%[Dự Án Giảng 12 4 in 1, Lê Văn Toàn]%[2D1C5-7]
Cho hàm số $y=\dfrac{x+2}{x-1}$ có đồ thị $(C)$. Gọi $I$ là giao điểm hai đường tiệm cận của $(C)$. Biết tọa độ điểm $M(a; b)$ có hoành độ dương thuộc đồ thị $(C)$ sao cho $M I$ ngắn nhất. Tính giá trị của $ab-2\sqrt{3}$.
\shortans{$4$}
\loigiai{
Giả sử $M\left(x_0; \dfrac{x_0+2}{x_0-1}\right) \in(C)\left(x_0>0; x_0 \neq 1\right)$.\\
Giao điểm của hai đường tiệm cận của $(C)$ là $I(1; 1)$.\\
Khi đó $M I=\sqrt{\left(1-x_0\right)^2+\left(1-\dfrac{x_0+2}{x_0-1}\right)^2}=\sqrt{\left(1-x_0\right)^2+\dfrac{9}{\left(x_0-1\right)^2}} \geq \sqrt{6}$.\\
Dấu bằng xảy ra khi $$\left(1-x_0\right)^2=\dfrac{9}{\left(x_0-1\right)^2} \Leftrightarrow\left[\begin{array}{l}x_0=1+\sqrt{3} \Rightarrow y_0=1+\sqrt{3} \\ x_0=1-\sqrt{3} \text { (loại).}\end{array}\right.$$
Suy ra $M(1+\sqrt{3}; 1+\sqrt{3}) \Rightarrow(1+\sqrt{3})^2=4+2 \sqrt{3}$.
}
\end{ex}

\begin{ex}%[Dự Án Giảng 12 4 in 1, Lê Văn Toàn]%[2D1C5-6]
Cho hàm số $y=\dfrac{1}{4}x^4-\dfrac{7}{2}x^2$ có đồ thị $(C)$. Tiếp tuyến tại điểm $A$ thuộc $(C)$ cắt $(C)$ tại hai điểm phân biệt $M\left(x_1;y_1\right)$, $N\left(x_2;y_2\right)$ ($M$, $N$ khác $A)$ thỏa mãn $y_1-y_2=6\left(x_1-x_2\right)$. Các điểm $A$ thỏa mãn có tổng các hoành độ là
\shortans{$-3$}
\loigiai{
Gọi $A\left(x_0;y_0\right)\in\,(C)$ là tọa độ tiếp điểm của phương trình tiếp tuyến.\\
Ta có hệ số góc $k=y'\left(x_0\right)=x_0^3-7x_0$.\\
Phương trình tiếp tuyến $y=k\left(x-x_0\right)+y_0=\left(x_0^3-7x_0\right)\left(x-x_0\right)+y_0$.\\
Ta có
\begin{eqnarray*}
&&y_1-y_2=6\left(x_1-x_2\right)\\
&\Leftrightarrow& k\left(x_1-x_0\right)+y_0-\left[k\left(x_2-x_0\right)+y_0\right]=6\left(x_1-x_2\right)\\
&\Leftrightarrow& k\left(x_1-x_2\right)=6\left(x_1-x_2\right)\\
&\Leftrightarrow& k=6\\
&\Leftrightarrow& x_0^3-7x_0=6\\
&\Leftrightarrow& x_0^3-7x_0-6=0\\
&\Leftrightarrow& \hoac{&x_0=3\Rightarrow y_0=-\dfrac{45}{4}\\&x_0=-1\Rightarrow y_0=-\dfrac{13}{4}\\&x_0=-2\Rightarrow y_0=-10.}
\end{eqnarray*}
Khi đó các phương trình tiếp tuyến tương ứng là
$$\hoac{&d_1\colon y=6(x-3)-\dfrac{45}{4}=6x-\dfrac{117}{4}\\&d_2\colon y=6(x+1)-\dfrac{13}{4}=6x+\dfrac{11}{4}\\&d_3\colon y=6(x+2)-10=6x+2.}$$
Phương trình hoành độ giao điểm của $(C)$ với các tiếp tuyến là
$$\hoac{&\dfrac{1}{4}x^4-\dfrac{7}{2}x^2-6x+\dfrac{117}{4}=0\text{ (có 1 nghiệm nên không thỏa)}\\&\dfrac{1}{4}x^4-\dfrac{7}{2}x^2-6x-\dfrac{11}{4}=0\text{ (có 3 nghiệm nên thỏa mãn)}\\&\dfrac{1}{4}x^4-\dfrac{7}{2}x^2-6x-2=0\text{ (có 3 nghiệm nên thỏa mãn).}}$$
Do đó tổng các hoành độ điểm các tiếp điểm là $-1-2=-3$.
}
\end{ex}

\begin{ex}%[2D1C5-5]
\immini
{
Cho hàm số $ f(x)=\dfrac{ax+b}{cx+d} $ (với $ a,\, b,\, c,\, d $ là các số thực) có đồ thị hàm số $ f'(x) $ như hình vẽ. Biết rằng giá trị lớn nhất của hàm số $ y=f(x) $ trên đoạn $ [-3;-2] $ bằng $ 7 $. Giá trị $ f(2) $ bằng
}
{
\begin{tikzpicture}[>=stealth,line join=round,line cap=round,scale=.7]
\def\f(#1){-3/((#1)+1)}
\draw[->] (0,-1)--(0,8)node[right]{$y$};
\draw[->] (-5,0)--(5,0)node[below]{$x$};
\clip (-5,-1) rectangle (5-0.1,8-0.1);
\draw[thick,samples=150,smooth,domain=5:-5] plot(\x,{abs(\f(\x))});
\draw (-1,-1)--(-1,8);
\fill (-1,0)node[below left]{$ -1 $};
\fill (0,0)node[below left]{$O$}circle(1pt);
\end{tikzpicture}
}
\shortans{$3$}
\loigiai{
Ta có $ f'(x)=\dfrac{ad-bc}{(cx+d)^2} $.\\
Từ đồ thị ta có $\heva{&-c+d=0\\ &a d-bc=3d^2}\Leftrightarrow\heva{&c=d\\ &a d-b d=3d^2}\Leftrightarrow\heva{&c=d\\ &a-b=3d.}$ \\
Từ đồ thị $ f'(x)>0 $ nên hàm số $ f(x)=\dfrac{ax+b}{cx+d} $ đồng biến trên $ (-\infty;-1)  $ và $ (-1;+\infty) $.\\
$\Rightarrow\max\limits_{[-3 ; -2]} f(x)=f(-2)=7 \Rightarrow \dfrac{-2a+b}{2c+d} = 7 \Leftrightarrow \dfrac{-2(3d+b)+b}{-2d+d}=7 \Leftrightarrow -6d-b=-7d \Leftrightarrow b=d$.\\
Vậy $f(2)=\dfrac{2a+b}{2c+d}=\dfrac{9d}{3d}=3$.
}
\end{ex}

\begin{ex}%[2D1C5-4]
Cho đường thẳng $d: y=mx+m+2$ ($m$ là tham số) và đường cong $(C): y=\dfrac{2x-1}{x+1}$. Biết rằng khi $m=m_0$ thì $(C)$ cắt $d$ tại hai điểm $A, B$ thỏa mãn độ dài $AB$ ngắn nhất. Tìm $m_0$.
\shortans{$-1$}
\loigiai{Tập xác đinh $\mathscr{D}=\mathbb{R}\setminus \{-1\}$.\\
Phương trình hoành độ giao điểm $mx+m+2=\dfrac{2x-1}{x+1}\Leftrightarrow \dfrac{mx^2+2mx+m+3}{x+1}=0$.\\
Điều kiện cần và đủ để $d$ cắt $C$ tại hai điểm phân biệt là phương trình $mx^2+2mx+m+3=0$ có hai nghiệm phân biệt khác $-1.$ \\
Điều này tương đương
$\heva{&m\ne 0\\&\Delta'=-3m>0\\&3\ne 0}\Leftrightarrow m<0.$\\
Với $m<0$ thì $d$ cắt $C$ tại hai điểm $A(x_1;mx_1+m+2)$ và $B(x_2;mx_2+m+2)$.\\
Theo Vi-et $x_1+x_2=-2$, $x_1x_2=1+\dfrac{3}{m}$. Ta có
\begin{eqnarray*}
AB^2&=& (x_1-x_2)^2+(mx_1-mx_2)^2\\
&=& (m^2+1)((x_1+x_2)^2-4x_1x_2)\\
&=& -\dfrac{12}{m}(m^2+1)\\
&=& -12m-\dfrac{12}{m}\ge 2\sqrt{\dfrac{-12}{m}(-12m)}=24.
\end{eqnarray*}
Dấu bằng xảy ra khi và chỉ khi $m^2=1\Leftrightarrow \heva{&m=1\\&m=-1.}$\\
Kết hợp với $m<0$ ta có $m=-1$ thỏa yêu cầu bài toán.
}
\end{ex}

\begin{ex}%[2D1C5-4]
%[Thi thử L1, chuyên Hùng Vương, Gia Lai 2018]%[2D1G5-4]%[Nguyễn Tài Chung, 12EX-7]
Cho hàm số đa thức bậc ba $y=f(x)$ có đồ thị đi qua các điểm $A(2;4)$, $B(3;9)$, $C(4;16)$. Các đường thẳng $AB, AC, BC$ lại cắt đồ thị tại lần lượt tại các điểm $D$, $E$, $F$ ($D$ khác $A$ và $B$; $E$ khác $A$ và $C$; $F$ khác $B$ và $C$). Biết rằng tổng các hoành độ của $D$, $E$, $F$ bằng $24$. Tính $f(0)$.
\shortans{$6{,}25$}
\loigiai{
Giải sử $f(x)=a(x-2)(x-3)(x-4)+x^2$ ($a\ne 0$). Ta có
$$AB\colon y=5x-6;AC\colon y=6x-8;BC\colon y=7x-12.$$
Hoành độ điểm $D$ là nghiệm của phương trình
\begin{eqnarray*}
& & a\left({x-2}\right)\left({x-3}\right)\left({x-4}\right)=-x^2+5x-6\\
&\Leftrightarrow & a\left({x-2}\right)\left({x-3}\right)\left({x-4}\right)=-\left({x-2}\right)\left({x-3}\right)\\
&\Leftrightarrow & a(x-4)=-1\Rightarrow x=-\dfrac{1}{a}+4.
\end{eqnarray*}
Hoành độ điểm $E$ là nghiệm của phương trình
\begin{eqnarray*}
& & a\left({x-2}\right)\left({x-3}\right)\left({x-4}\right)=-x^2+6x-8\\
&\Leftrightarrow & a\left({x-2}\right)\left({x-3}\right)\left({x-4}\right)=-\left({x-2}\right)\left({x-4}\right)\\
&\Leftrightarrow & a(x-3)=-1\Rightarrow x=-\dfrac{1}{a}+3.
\end{eqnarray*}
Hoành độ điểm $F$ là nghiệm của phương trình
\begin{eqnarray*}
& & a\left({x-2}\right)\left({x-3}\right)\left({x-4}\right)=-x^2+7x-12\\
&\Leftrightarrow & a\left({x-2}\right)\left({x-3}\right)\left({x-4}\right)=-\left({x-3}\right)\left({x-4}\right)\\
&\Leftrightarrow & a(x-2)=-1\Rightarrow x=-\dfrac{1}{a}+2.
\end{eqnarray*}
Theo giả thiết ta có $$-\dfrac{1}{a}+2-\dfrac{1}{a}+3+-\dfrac{1}{a}+4=24\Leftrightarrow -\dfrac{3}{a}=15\Leftrightarrow a=-\dfrac{1}{5}.$$
Do đó $f(0)=a\left({-2}\right)\left({-3}\right)\left({-4}\right)=\dfrac{24}{5}=6{,}25$.}
\end{ex}

\begin{ex}%[2D1C5-4]
Tập hợp tất cả các giá trị thực của tham số $m$ để đồ thị của hàm số $y=x^3-3x^2+2m+1$ cắt trục hoành tại ba điểm phân biệt cách đều nhau là
\shortans{$0{,}5$}
\loigiai{
Phương trình hoành độ giao điểm $x^3-3x^2+2m+1=0\quad (*)$.\\
Giả sử $x_1$; $x_2$; $x_3$ là ba nghiệm của $(*)$.\\
Để $x_1$; $x_2$; $x_3$ cách đều nhau $\Leftrightarrow$ $2x_2=x_1+x_3$ \quad $(1)$\\
Mặt khác\\ $x^3-3x^2+2m+1=(x-x_1)(x-x_2)(x-x_3)$
$=x^3-(x_1+x_2+x_3)x^2+(x_1x_2+x_2x_3+x_3x_1)x-x_1x_2x_3$\\
Nên $x_1+x_2+x_3=3$ $(2)$\\
Từ $(1)$ và $(2)$ suy ra $x_2=1$ $(3)$\\
Thế $(3)$ vào $(*)$ ta được $1-3+2m+1=0 \Leftrightarrow m=\dfrac{1}{2}$.\\
Với $m=\dfrac{1}{2}$ thế vào $(*)$, ta được $x^3-3x^2+2=0 \Leftrightarrow \hoac{&x=1-\sqrt{3}\\&x=1\\&x=1+\sqrt{3}.}$\\
Rõ ràng $3$ nghiệm này cách đều nhau.\\ Vậy $m=\dfrac{1}{2}=0{,}5$ là giá trị cần tìm.}
\end{ex}

\begin{ex}%[2D1C5-4]
Số giao điểm của hai đồ thị hàm số $f(x)=2(m+1)x^3+2mx^2-2(m+1)x-2m$, $\left( m \text{ là tham số khác} -\dfrac{3}{4}\right)$  và $g(x)=-x^4+x^2$ là
\shortans{$4$}
\loigiai{
Phương trình hoành độ giao điểm của hai đồ thị hàm số là
\begin{eqnarray*}
&&-x^4+x^2=2(m+1)x^3+2mx^2-2(m+1)x-2m\\
&\Leftrightarrow & -x^2(x^2-1)=2m(x^3+x^2-x-1)+2x^3-2x\\
&\Leftrightarrow & -x^2(x^2-1)=2m(x^2-1)(x+1)+2x(x^2-1)\\
&\Leftrightarrow &  (x^2-1)\left[x^2+2(m+1)x+2m\right]=0\\
&\Leftrightarrow & \hoac{&x^2-1=0\\&h(x)=x^2+2(m+1)x+2m=0}\\
&\Leftrightarrow & \hoac{&x=\pm 1\\&h(x)=x^2+2(m+1)x+2m=0 \quad(1)}
\end{eqnarray*}
Xét $(1)$ có $\heva{&\Delta =m^2+1>0,\forall m\\&h(-1)=-1\ne 0, \forall m\\&h(1)=4m+3 \ne 0, \forall m \ne -\dfrac{3}{4}.}$\\
$\Rightarrow$ Phương trình $(1)$ luôn có $2$ nghiệm phân biệt khác $\pm 1$.\\
Vậy phương trình đã cho có $4$ nghiệm phân biệt.
}
\end{ex}

\begin{ex}%[2D1C5-4]
\immini{
Cho hàm số bậc ba $y=f(x)$ có đồ thị như hình vẽ bên. Tìm số điểm cực trị của hàm số $g(x)=f\left(\mathrm{e}^x-x\right)$.

}{
\begin{tikzpicture}[scale=0.6, font=\footnotesize, line join=round, line cap=round,>=stealth]
\def\xmin{-1} \def\xmax{5}
\def\ymin{-2} \def\ymax{3}
\draw[->] (\xmin,0)--(\xmax,0) node [below]{$x$};
\draw[->] (0,\ymin)--(0,\ymax) node [right]{$y$};
\node at (0,0) [above left]{$O$};
\draw[color=black] (-0.7,-2) parabola bend (1,1.7) (2,0) parabola bend (3,-1.6) (4.6,3);
\draw[dashed] (1,0) -- (1,1.7);
\foreach\i/\j/\goc/\diem in{1/0/-90/1,2/0/-100/2}
\fill(\i,\j) circle(1.0pt) node[shift={(\goc:10pt)}]{$\diem$};
\end{tikzpicture}
}
\shortans{$3$}
\loigiai{
Từ đồ thị của hàm số $y=f(x)$ ta có $f'(x)=0\Leftrightarrow\hoac{&x=1\\&x=a>2.}$ \\
Ta có $g'(x)=f'\left(\mathrm{e}^x-x\right)\cdot \left(\mathrm{e}^x-1\right)$ và $g'(x)=0\Leftrightarrow\hoac{&\mathrm{e}^x-1=0\\&\mathrm{e}^x-x=1\\&\mathrm{e}^x-x=a>2}\Leftrightarrow\hoac{&x=0 &&\\&\mathrm{e}^x-x=1&&(1)\\&\mathrm{e}^x-x=a>2.&&(2)}$ \\
Xét hàm số $h(x)=\mathrm{e}^x-x$ trên $\mathbb{R}$, ta có $h'(x)=\mathrm{e}^x-1=0\Leftrightarrow x=0$.\\
Bảng biến thiên của hàm số $y=h(x)$
\begin{center}
\begin{tikzpicture}
\tkzTabInit[nocadre=false,lgt=1.2,espcl=2.5,deltacl=0.6]
{$x$ /0.8, $h'(x)$ /0.8, $h(x)$ /2.5}
{$-\infty$,$0$,$+\infty$}
\tkzTabLine{,-,$0$,+,}
\tkzTabVar{+/$+\infty$, -/$1$,+/$+\infty$}
\end{tikzpicture}
\end{center}
Phương trình $(2)$ có $2$ nghiệm phân biệt khác $0$, phương trình $(1)$ có nghiệm kép $x=0$, do đó phương trình $g'(x)=0$ có $3$ nghiệm trong đó $x=0$ là nghiệm bội $3$.\\
Vậy hàm số $g(x)=f\left(\mathrm{e}^x-x\right)$ có $3$ điểm cực trị.
}
\end{ex}

\begin{ex}%[Mức độ C]%[2D1C5-3]
Cho hàm số $y=f(x)$ có đạo hàm liên tục trên $\mathbb{R}$ và có đồ thị $y=f'(x)$ như hình vẽ. Đặt $g(x)=f(x-m)-\dfrac{1}{2}(x-m-1)^2+2024$, với $m$ là tham số thực. Gọi $S$ là tập hợp các giá trị nguyên dương của $m$ để hàm số $y=g(x)$ đồng biến trên khoảng $(5;6)$. Tổng tất cả các phần tử trong $S$ bằng bao nhiêu?
\begin{center}
\begin{tikzpicture}[>=stealth]
\draw [->] (-2,0)--(4,0);
\draw [->] (0,-3)--(0,3);
\draw (0,0) node[below left]{$O$};
\draw (4,0) node[below]{$x$};
\draw (0,3) node[below left]{$y$};
\draw (3,0) node[below]{$3$};
\draw (0,2) node[above left]{$2$};
\foreach \x in {-1,1,2,}{\draw (\x,-.1)--(\x,.1) node[below left,black]{$\x$};}
\foreach \y in {-2}{\draw [-] (-.1,\y)--(.1,\y) node[below left,black]{$\y$};}
\clip (-2,-3) rectangle (4,3);
\draw [thick,samples=100] plot[domain=-4:4](\x,{(\x)^3-3*(\x)^2+2});
\draw (3.1,2.9) node[below left]{$(C)$};
\draw[dashed] (2,0)--(2,-2)--(0,-2);
\draw[dashed] (-1,0)--(-1,-2)--(0,-2);
\draw[dashed] (3,0)--(3,2)--(0,2);
\end{tikzpicture}
\end{center}
\shortans{$4$}
\loigiai{Xét hàm số $g(x)=f(x-m)-\dfrac{1}{2}(x-m-1)^2+2019\text{; }g'(x)=f'(x-m)-(x-m-1)$.\\
Cho $g'(x)=0 (1)$.\\
Đặt $x-m=t$, phương trình $(1)$ trở thành $f'(t)-(t-1)=0 \Leftrightarrow f'(t)=t-1 \text{  (2)}$.\\
Nghiệm của phương trình  $(2)$ là hoành độ giao điểm của hai đồ thị hàm số $y=f'(t)$ và $y=t-1$.\\
Đồ thị hai hàm số $y=f'(t)$ và $y=t-1$.
\begin{center}
\begin{tikzpicture}[>=stealth]
\draw [->] (-2,0)--(4,0);
\draw [->] (0,-3)--(0,3);
\draw (0,0) node[below left]{$O$};
\draw (4,0) node[below]{$x$};
\draw (0,3) node[below left]{$y$};
\draw (3,0) node[below]{$3$};
\draw (0,2) node[above right]{$2$};
\foreach \x in {-1,1,2,}{\draw (\x,-.1)--(\x,.1) node[above,black]{$\x$};}
\foreach \y in {-2}{\draw [-] (-.1,\y)--(.1,\y) node[below left,black]{$\y$};}
\clip (-2,-3) rectangle (4,3);
\draw [thick,samples=100] plot[domain=-4:4](\x,{(\x)^3-3*(\x)^2+2});
\draw (3.1,2.9) node[below left]{$(C)$};
\draw[dashed] (2,0)--(2,-2)--(0,-2);
\draw[dashed] (-1,0)--(-1,-2)--(0,-2);
\draw[dashed] (3,0)--(3,2)--(0,2);
\draw [thick,samples=100] plot[domain=-4:4](\x,{(\x)-1});
\fill[black] (-1,-2) circle(2pt);
\fill[black] (1,0) circle(2pt);
\fill[black] (3,2) circle(2pt);
\end{tikzpicture}
\end{center}
Từ đồ thị ta có phương trình $(2)$ có nghiệm là $\left[\begin{array}{l}
t=-1\\t=1\\t=3
\end{array}\right. \Rightarrow \left[\begin{array}{l}
x=m-1\\x=m+1\\x=m+3.
\end{array}\right.$\\
Bảng biến thiên
\begin{center}
\begin{tikzpicture}
\tkzTabInit[nocadre=false, lgt=1.5,espcl=3.5]
{$x$/1,$y'$/1,$y$/2}
{$-\infty$,$m-1$,$m+1$,$m+3$,$+\infty$}
\tkzTabLine{,-,0,+,0,-,0,+, }
\tkzTabVar{+/$+\infty$,-/$ $,+/$ $,-/$ $,+/$+\infty$/}
\end{tikzpicture}
\end{center}
Để hàm số $y=g(x)$ đồng biến trên khoảng $(5;6)$ cần $\left[\begin{array}{l}
\begin{cases}
m-1 \leq 5\\m+1 \geq 6
\end{cases}\\
m+3 \leq 5
\end{array}\right. \Leftrightarrow \left[\begin{array}{l}
5 \leq m \leq 6\\
m \leq 2.
\end{array}\right.$\\
Vì $m \in \mathbb{N^{*}}$ mên $m = \{1;2;5;6\}$.\\
Vậy $S=1+2+5+6=14$.
}


\end{ex}

\begin{ex}%[2D1C5-2]
Cho hàm số $f(x)=\dfrac{x^2+5x+2}{2x+1}$. Có tất cả bao nhiêu giá trị nguyên dương của tham số $m$ để bất phương trình $2021f\left(\sqrt{3x^2-18x+28}\right)-m\sqrt{3x^2-18x+28} \geq m+4042$ nghiệm đúng với mọi $x$ thuộc đoạn $[2;4]$?
\shortans{$673$}
\loigiai{
Đặt $u=\sqrt{3 x^2-18 x+28}=\sqrt{3(x-3)^2+1}=\sqrt{3(x-2)(x-4)+4}$.\\
Hàm số $t=3x^2-18x+28$ có $t'=6x-18$ và $t'=0\Leftrightarrow t=3$.\\
Bảng biến thiên của $t$ trên đoạn $[2;4]$ như sau \begin{center}
\begin{tikzpicture}
\tkzTabInit[nocadre=false, lgt=1.2, espcl=2.5, deltacl=0.6]{$x$/0.6,$t'$/0.6,$t$/2}
{$2$, $3$, $4$}
\tkzTabLine {,-,0,+,}
\tkzTabVar{+/$4$, -/$1$, +/$4$}
\end{tikzpicture}
\end{center}
Suy ra $u\in [1;2]$ khi $x\in [2;4]$.\\
Bất phương trình đã cho được viết lại $2021f(u)-mu \geq m+4042 \Leftrightarrow 2021[f(u)-2] \geq m(u+1)$.\\
Ta có $f(x)=\dfrac{x^2+5x+2}{2x+1}$ nên $f(u)-2=\dfrac{u^2+5u+2}{2u+1}-2=\dfrac{u^2+u}{2u+1}$.\\ Do vậy bất phương trình được viết lại thành $\dfrac{2021\left(u^2+u\right)}{2u+1} \geq m(u+1) \Leftrightarrow m \leq \dfrac{2021u}{2u+1}$.\\
Lúc này yêu cầu bài toán tương đương $m \leq \dfrac{2021u}{2u+1},\forall u \in [1;2] \Leftrightarrow m \leq \min\limits_{u \in 1;2]}g(u)$.\\
Xét hàm số $g(u)=\dfrac{2021u}{2u+1}$, $u \in [1;2]$ ta có $g'(u)=\dfrac{2021}{(2u+1)^2}>0$, $\forall u \in [1;2]$.\\
Do vậy hàm số $g(u)$ tăng trên đoạn $[1;2]$.\\
Vì vậy $\min\limits_{u \in [1;2]}g(u)=\dfrac{2021u}{2u+1}=g(1)=\dfrac{2021}{3}$.\\
Kết hợp với $m$ là các số nguyên dương ta được $m \in\{1 ; 2 ; 3 ; \ldots ; 673\}$.\\
Vậy tìm được $673$ số nguyên dương thỏa mãn yêu cầu bài toán.
}
\end{ex}

\begin{ex}%[Dự án Giảng 12 Nhóm Toán & LaTex, Lê Minh Thiện Anh]%[2D1C5-1]
Cho hàm số $f(x)=\dfrac{2-ax}{bx-c}\,(a, b, c \in \mathbb{R}, b \neq 0)$ có bảng biến thiên như sau
\begin{center}
\begin{tikzpicture}
\tkzTabInit[nocadre=false,lgt=1.2,espcl=3]
{$x$/.6,$f'(x)$/.6,$f(x)$/2}
{$-\infty$,$1$,$+\infty$}
\tkzTabLine{ ,+,d,+,}
\tkzTabVar{-/$3$,+D-/$+\infty$/$-\infty$,+/$3$}
\end{tikzpicture}
\end{center}
Tổng $(a+b+c)^2$ thuộc khoảng $\left(0;\dfrac{4}{n}\right)$. Tìm $n$.
\shortans{$9$}
\loigiai{
Ta có $\lim\limits_{x \rightarrow \infty}\dfrac{2-a x}{b x-c}=\dfrac{-a}{b}$, theo giả thiết suy ra $\dfrac{-a}{b}=3 \Leftrightarrow a=-3b$.\\
Hàm số không xác định tại $x=1 \Rightarrow b-c=0 \Leftrightarrow b=c$.\\
Hàm số đồng biến trên từng khoảng xác định nên $f'(x)=\dfrac{ac-2b}{(bx-c)^2}>0$, $\forall x\neq 1$.\\
Suy ra $ac-2b>0 \Leftrightarrow-3b^2-2b>0 \Leftrightarrow-\dfrac{2}{3}<b<0 \Leftrightarrow 0<-b<\dfrac{2}{3}$.\\
Lại có $a+b+c=-3b+b+b=-b$. Suy ra $(a+b+c)^2=b^2 \in\left(0 ; \dfrac{4}{9}\right)$.\\
Vậy tổng $a+b+c$ thuộc khoảng $\left(0;\dfrac{4}{9}\right)$. Vậy $n=9$.
}
\end{ex}

\begin{ex}%[Dự án Giảng 12 Nhóm Toán & LaTex, Lê Minh Thiện Anh]%[2D1C5-1]
Biết hàm số $f(x)=x^3+ax^2+bx+c$ đạt cực đại tại điểm $x=-3$, $f(-3)=28$ và đồ thị của hàm số cắt trục tung tai điểm có tung độ bằng $1$. Tính $S=a^2+b^2-c^2$.
\shortans{$89$}
\loigiai{
Ta có $f'(x)=3 x^2+2 a x+b$; $f''(x)=6x+2a$.\\
Hàm số $f(x)$ đạt cực đại tại điểm $x=-3$ khi và chi khi $\heva{& f'(-3)=0 \\& f''(-3)<0} \Leftrightarrow\heva{& -6a+b=-27 \\ & a<9}$ (1).\\
Mà $f(-3)=28 \Rightarrow 9 a-3 b+c=55(2)$.\\
Ngoài ra, đồ thị của hàm số $f(x)$ cắt trục tung tại điểm có tung độ bằng 1 nên $c=1$ (3).\\
Tù (1), (2), (3) suy ra $\heva{& -6 a+b=-27 \\& 9a-3b+c=55 \\& c=1 \\& a<9} \Leftrightarrow\heva{& a=3 \\& b=-9 \\& c=1 \\ & a<9.}$\\
Do đó $S=3^2+(-9)^2-1^2=89$.
}
\end{ex}

\begin{ex}%[Mức độ 4]giảng 12, Phạm Tiến Long]%[2D1C4-3]
Cho hàm số $f(x)=\dfrac{x^2-2}{x-4}$ có đồ thị $(C)$. Biết đường thẳng $\Delta\colon y=-x+m$ cắt tiệm cận đứng và tiệm cận xiên của $(C)$ lần lượt tại hai điểm $B$, $C$ sao cho tam giác $OBC$ có diện tích bằng $\dfrac{11}{4}$ (với $O$ là gốc tọa độ). Biết $m$ là số nguyên và lớn hơn $1$. Tính giá trị $m^2-1$.
\shortans{$120$}
\loigiai{
Hàm số đã cho có tập xác định là $\mathbb{R}\backslash\{4\}$.\\
Ta có $\lim\limits_{x\to 4^+}f(x)=+\infty$  và 	$\lim\limits_{x\to 4^-}f(x)=-\infty$.\\
Suy ra tiệm cận đứng của $(C)$ là đường thẳng $d\colon x=4$.\\
Mặt khác,	ta có $\begin{aligned}[t]
a&=\lim\limits_{x \rightarrow+\infty} \dfrac{f(x)}{x}=\lim\limits_{x \rightarrow+\infty} \dfrac{x^2-2}{x^2-4x}=1;\\
b&=\lim\limits_{x \rightarrow+\infty}[f(x)-x]=\lim\limits_{x \rightarrow+\infty}\left(\dfrac{x^2-2}{x-4}-x\right)=\lim\limits_{x \rightarrow+\infty} \dfrac{4x-2}{x-4}=4.
\end{aligned}$\\
Ta cũng có $\lim\limits_{x \rightarrow-\infty} \dfrac{f(x)}{x}=1$; $\lim\limits_{x \rightarrow-\infty}[f(x)-x]=4$.
\\
Do đó, đồ thị hàm số có tiệm cận xiên là đường thẳng $d'\colon y=x+4\Leftrightarrow x-y+4=0$.\\
Đường thẳng $\Delta\colon y=-x+m$ cắt hai đường thẳng $d$ và $d'$ lần lượt tại hai điểm $B(4;m-4)$ và $C\left(\dfrac{m-4}{2};\dfrac{m+4}{2}\right)$.\\
Ta có
\begin{itemize}
\item $BC=\sqrt{\left(\dfrac{m-4}{2}-4\right)^2+\left(\dfrac{m+4}{2}-m+4\right)^2}=\sqrt{\dfrac{1}{2}m^2-12m+72}$.
\item $\mathrm{d}(O,\Delta)=\dfrac{|m|\sqrt{2}}{2}$.
\end{itemize} .\\
Theo giả thiết ta có
\begin{eqnarray*}
& & S_{\triangle OBC}=\dfrac{11}{4}\\
&\Leftrightarrow & \dfrac{1}{2}\cdot BC \cdot \mathrm{d}(O,\Delta)=\dfrac{11}{4}\\
&\Leftrightarrow & \dfrac{1}{2}\sqrt{\dfrac{1}{2}m^2-12m+72} \cdot \dfrac{|m|\sqrt{2}}{2}=\dfrac{11}{4}\\
&\Leftrightarrow & \dfrac{1}{4}\cdot \left(\dfrac{1}{2}m^2-12m+72\right)\cdot \dfrac{m^2}{2} =\dfrac{121}{16} \\
&\Leftrightarrow & \dfrac{1}{16}m^4-\dfrac{3}{2}m^3+9m^2-\dfrac{121}{16}=0\\
&\Leftrightarrow & \hoac{&m=1\\&m=11\\&m=6-\sqrt{47}\\&m=6+\sqrt{47}.}
\end{eqnarray*}
Vì $m$ là số nguyên và $m>1$ nên $m=11\Rightarrow m^2-1=120$.
}
\end{ex}

\begin{ex}%[Mức độ 4]giảng 12, Phạm Tiến Long]%[2D1C4-3]
Cho hàm số $f(x)=\dfrac{x^2-4x+5}{x+2}$ có đồ thị $(C)$. Gọi $I$ là giao điểm của tiệm cận đứng và tiệm cận xiên của $(C)$. Đường thẳng $y=m$ (với $m\ne 0$) cắt tiệm cận đứng và tiệm cận xiên của $(C)$ tại hai điểm $A$, $B$ sao cho tam giác $IAB$ có diện tích bằng $32$. Tìm $m$.
\shortans{$-16$}
\loigiai{
Hàm số đã cho có tập xác định là $\mathbb{R}\backslash\{-2\}$.\\
Ta có $\lim\limits_{x\to -2^+}f(x)=+\infty$  và 	$\lim\limits_{x\to -2^-}f(x)=-\infty$.\\
Suy ra tiệm cận đứng của $(C)$ là đường thẳng $d\colon x=-2$.\\
Mặt khác,	ta có $\begin{aligned}[t]
a&=\lim\limits_{x \rightarrow+\infty} \dfrac{f(x)}{x}=\lim\limits_{x \rightarrow+\infty} \dfrac{x^2-4x+5}{x^2+2x}=1;\\
b&=\lim\limits_{x \rightarrow+\infty}[f(x)-x]=\lim\limits_{x \rightarrow+\infty}\left(\dfrac{x^2-4x+5}{x+2}-x\right)=\lim\limits_{x \rightarrow+\infty} \dfrac{-6x+5}{x+2}=-6.
\end{aligned}$\\
Ta cũng có $\lim\limits_{x \rightarrow-\infty} \dfrac{f(x)}{x}=1$; $\lim\limits_{x \rightarrow-\infty}[f(x)-x]=-6$.
\\
Do đó, đồ thị hàm số có tiệm cận xiên là đường thẳng $d'\colon y=x-6$.\\
$I$ là giao điểm của $d$ và $d' \Rightarrow I(-2;-8)$.\\
Đường thẳng $y=m$ cắt hai đường thẳng $d$ và $d'$ lần lượt tại hai điểm $A(-2;m)$ và $B(m+6;m)$.\\
Ta có $\heva{&IA=\sqrt{(m+8)^2}=|m+8|\\&AB=\sqrt{(m+8)^2}=|m+8|}\Rightarrow IA=AB$\quad(1)\\
Dễ thấy đường thẳng $y=m$ vuông góc với $d$ tại $A$.\quad(2)\\
Từ (1) và (2) suy ra tam giác $IAB$ vuông cân tại $A$.\\
Theo giả thiết ta có
\begin{eqnarray*}
& & S_{\triangle IAB}=32\\
&\Leftrightarrow & \dfrac{1}{2}\cdot AB^2=32\\
&\Leftrightarrow & (m+8)^2=64\\
&\Leftrightarrow & m^2+16m=0\\
&\Leftrightarrow & \hoac{&m=0\\&m=-16.}
\end{eqnarray*}
Vì $m\ne 0$  nên suy ra $m=-16$.
}
\end{ex}

\begin{ex}%[Dự án TL12New-4in1-NCT]%[2D1C4-2]
Cho hàm số $y=\dfrac{2x+1}{x-3}$ có đồ thị là $(C)$. Gọi $M$ là điểm bất kì trên đồ thị $(C)$, tìm giá trị nhỏ nhất của tổng khoảng cách từ $M$ đến hai tiệm cận của đồ thị (làm tròn đến $1$ chữ số thập phân).
\shortans{$5{,}3$}
\loigiai{Gọi $M\left(x_M;\dfrac{2x_M+1}{x_M-3}\right),x_M\ne 3$. Các đường tiệm cận ngang, tiệm cận đứng của đồ thị có phương trình lần lượt là $y=2,x=3$.\\
Tổng khoảng cách từ điểm $M$ đến hai tiệm cận là\newline $d=|x_M-3|+\left|\dfrac{2x_M+1}{x_M-3}-2\right|=|x_M-3|+\dfrac{7}{\left|x_M-3\right|}\ge 2\sqrt{7}$.\\
Đẳng thức xảy ra khi và chỉ khi $|x_M-3|=\dfrac{7}{\left|x_M-3\right|}\Leftrightarrow \left[\begin{aligned}&x_M=3+\sqrt{7}\\&x_M=3-\sqrt{7}\end{aligned}\right.$\\
Vậy giá trị giá trị nhỏ nhất của tổng khoảng cách từ $M$ đến hai tiệm cận của đồ thị là $2\sqrt 7\approx5{,}3$.}
\end{ex}

\begin{ex}%[Dự án TL12New-4in1-NCT]%[2D1C4-1]
\immini{
Cho hàm số bậc ba $f(x)$ có đồ thị như hình vẽ. Xác định tổng số các đường tiệm cận đứng và tiệm cận ngang của đồ thị hàm số $ g(x)=\dfrac{\left(x^2-2x-3\right)\sqrt{x+2}}{(x^2-x)\left[f^2(x)+f(x)\right]} $.
\shortans{$7$}
}{
\begin{tikzpicture}[scale=.7,>=stealth]
\draw[->](-2.5,0)--(4,0)node[below]{$x$};
\draw[->](0,-2.5)--(0,3)node[left]{$y$};
\draw[dashed](-1,0)--(-1,-1)circle(1.5 pt)--(0,-1);
\node at (-1,0) [above] {\footnotesize $-1$};
\node at (2,0) [above] {\footnotesize $2$};
\node at (0,-1) [ right] {\footnotesize $-1$};
\draw [fill] (0,0) circle (1.5 pt)node[above right] {\footnotesize $O$};
\draw[smooth,samples=100,domain=-2.1:3.5] plot(\x,{-20/81*((\x)+1.45)*((\x)-2)^2});
\end{tikzpicture}
}
\loigiai{
\begin{center}
\begin{tikzpicture}[scale=.7,>=stealth]
\draw[->](-2.5,0)--(4,0)node[below]{$x$};
\draw[->](0,-2.5)--(0,3)node[left]{$y$};
\draw[dashed](-1,0)--(-1,-1)circle(1.5 pt)--(0,-1);
\node at (-1,0) [above] {\footnotesize $-1$};
\node at (2,0) [above] {\footnotesize $2$};
\node at (0,-1) [above left] {\footnotesize \footnotesize $-1$};
\draw [fill] (0,0) circle (1.5 pt)node[above left] {\footnotesize $O$};
\draw [fill] (-1.45,0) circle (1.5 pt)node[above left] {\footnotesize $a$};
\draw [fill] (2.9,0) circle (1.5 pt)node[above] {\footnotesize $c$};
\draw [dashed] (-2,-1)--(3.5,-1) (0.55,0)--(0.55,-1)(2.9,0)--(2.9,-1);
\draw [fill] (0.55,0) circle (1.5 pt)node[above ] {\footnotesize $b$};
\draw[smooth,samples=100,domain=-2.1:3.5] plot(\x,{-20/81*((\x)+1.45)*((\x)-2)^2});
\end{tikzpicture}
\end{center}
Dựa vào đồ thị ta thấy $f^2(x)+f(x)=0\Leftrightarrow\hoac{&f(x)=0\\&f(x)=-1}\Leftrightarrow \hoac{&x=a \quad (-2<a<-1)\\&x=2\\&x=-1\\&x=b\quad  (0<b<1)\\&x=c \quad (c>2).}$\\
Do đó ta viết $f^2(x)+f(x)=k x(x-1)(x-a)(x-2)^2(x+1)(x-b)(x-c)$.\\
Xét hàm số $  g(x)=\dfrac{\left(x^2-2x-3\right)\sqrt{x+2}}{(x^2-x)\left[f^2(x)+f(x)\right]}=\dfrac{(x+1)(x-3)\sqrt{x+2}}{k x(x-1)(x-a)(x-2)^2(x+1)(x-b)(x-c)} $.\\
Tập xác định $ \mathscr{D}=[-2;+\infty)\backslash\{0;1;a;2;-1;b;c\} $.\\
Từ đó suy ra đồ thị hàm số $g(x)$ có $6$  tiệm cận đứng là $ x=0, x=1, x=a, x=2, x=b, x=c $ và $1$ tiệm cận ngang là $y=0$.
}
\end{ex}

\begin{ex}%[Dự án TL12New-4in1-NCT]%[2D1C4-1]
\immini{
Cho hàm số $y=ax^4+bx^2+c$ có đồ thị như hình vẽ. Đồ thị hàm số $y=\dfrac{(x^2-4)(x^2+2x)}{\left[f(x)\right]^2+2f(x)-3}$ có bao nhiêu đường tiệm cận đứng?\shortans{$4$}}	{\hspace{0.5cm}
\begin{tikzpicture}[scale=0.5,>=stealth]
\path
(1,1) coordinate (A);
\draw[->](-3.5,0)--(3.5,0)node[below]{$x$};
\draw[->](0,-4)--(0,2.5)node[left]{$y$};
\draw[smooth,samples=200,domain=-2.9:2.9]plot(\x,{1/4*(\x)^4-2*(\x)^2+1});
\draw[dashed](-2,0)--(-2,-3)-- (2,-3)--(2,0);
\draw(0,0)node[below left]{$O$} (-2,0)node[above]{$-2$} (2,0)node[above]{$2$} (0,1)node[ left]{$1$} (0,-3)node[below left]{$-3$};
\end{tikzpicture}}
\loigiai{
\begin{center}
\begin{tikzpicture}[scale=0.7,>=stealth]
\path
(1,1) coordinate (A);
\draw[->](-3.5,0)--(4,0)node[below]{$x$};
\draw[->](0,-4)--(0,2.5)node[left]{$y$};
\draw[smooth,samples=200,domain=-2.9:2.9]plot(\x,{1/4*(\x)^4-2*(\x)^2+1});
\draw[dashed] (-3.3,1)--(3.3,1);
\draw[dashed] (-2.85,1)--(-2.85,0)node[below ]{$m$};
\draw[dashed] (2.85,1)--(2.85,0)node[below ]{$n$};
\draw[dashed] (-3.3,-3)--(3.3,-3);
\draw[dashed](-2,0)--(-2,-3)-- (2,-3)--(2,0);
\draw(0,0)node[below left]{$O$} (-2,0)node[above]{$-2$} (2,0)node[above]{$2$} (0,1)node[above left]{$1$} (0,-3)node[below left]{$-3$};
\end{tikzpicture}
\end{center}
Ta có $\left[f(x)\right]^2+2f(x)-3=0\Leftrightarrow\hoac{&f(x)=1\\&f(x)=-3}\Leftrightarrow\hoac{&x=m \quad (m<-2)\\&x=0\\&x=n\quad (n>2)\\&x=2\\&x=-2.}$\\
Do đó ta viết $\left[f(x)\right]^2+2f(x)-3=kx^2(x+2)^2(x-2)^2(x-m)(x-n)$.\\
Xét hàm số $y=\dfrac{(x^2-4)(x^2+2x)}{\left[f(x)\right]^2+2f(x)-3}=\dfrac{x(x+2)^2(x-2)}{kx^2(x+2)^2(x-2)^2(x-m)(x-n)}$.\\
Hàm số có tập xác định $\mathscr{D}=\mathbb{R}\backslash\{m;-2;0;2;n\}$.\\
Từ đó suy ra đồ thị hàm số đã cho có bốn tiệm cận đứng là $x=0$, $x=2$, $x=m, x=n$.
}

\end{ex}

\begin{ex}%[Dự án TL12New-4in1-NCT]%[2D1C4-1]
\immini
{
Cho hàm số bậc ba $f(x)$  có đồ thị như hình vẽ. Hỏi đồ thị hàm số \break $ g(x)=\dfrac{\left(x^2+4x+3\right)\sqrt{x^2+x}}{x\left[f^2(x)-2f(x)\right]} $ có bao nhiêu đường tiệm cận đứng? 	\shortans{$4$}
}{
\begin{tikzpicture}[scale=.6,>=stealth]
\draw[->](-4.5,0)--(2,0)node[below]{$x$};
\draw[->](0,-2.5)--(0,4)node[left]{$y$};
\draw[dashed](-1,0)--(-1,2)circle(1.5 pt)--(0,2);
\node at (-3,0) [below] {\footnotesize $-3$};
\node at (-1,0) [below] {\footnotesize $-1$};
\node at (0,2) [right] {\footnotesize $2$};
\draw [fill] (0,0) circle (1.5 pt)node[below right] {\footnotesize $O$};
\draw[smooth,samples=100,domain=-4:-0.3] plot(\x,{-(\x)^3-6.5*(\x)^2-12*(\x)-4.5});
\end{tikzpicture}
}
\loigiai{
\begin{center}
\begin{tikzpicture}[scale=.8,>=stealth]
\draw[->](-4.5,0)--(2,0)node[below]{$x$};
\draw[->](0,-2.5)--(0,4)node[left]{$y$};
\draw[dashed](-1,0)--(-1,2)circle(1.5 pt)--(0,2);
\node at (-3,0) [below] {\footnotesize $-3$};
\node at (-1,0) [below] {\footnotesize $-1$};
\node at (0,2) [right] {\footnotesize $2$};
\draw [fill] (0,0) circle (1.5 pt)node[below right] {\footnotesize $O$};
\draw [dashed] (-4.5,2)--(1,2);
\draw [fill](-0.5,0)circle (1.5 pt)node[below right]{$x_3$};
\draw [dashed][fill](-3.8,2)--(-3.8,0)circle (1.5 pt)node[below ]{$x_1$};
\draw [dashed][fill](-1.7,2)--(-1.7,0)circle (1.5 pt)node[below ]{$x_2$};
\draw[smooth,samples=100,domain=-4:-0.3] plot(\x,{-(\x)^3-6.5*(\x)^2-12*(\x)-4.5});
\end{tikzpicture}
\end{center}
Dựa vào đồ thị ta thấy $ f(x)=0\Leftrightarrow \hoac{&x=-3\\&x=x_3\in (-1;0).} $\\
Do đó, ta viết $ f(x)=a(x+3)^2(x-x_3) $.\\
Đồng thời, $ f(x)=2\Leftrightarrow \hoac{&x=x_1\in (-\infty;-3)\\&x=x_2\in (-3;-1)\\&x=-1} $. Do đó, ta viết $ f(x)-2=a(x-x_1)(x-x_2)(x+1) $.\\
Xét hàm số $ g(x)=\dfrac{\left(x^2+4x+3\right)\sqrt{x^2+x}}{x\left[f^2(x)-2f(x)\right]}=\dfrac{(x+1)(x+3)\sqrt{x^2+x}}{a^2x(x+3)^2(x+1)(x-x_1)(x-x_2)(x-x_3)} $.\\
Tập xác định $ \mathscr{D}=(-\infty;x_1)\cup(x_1;-3)\cup(-3;x_2)\cup(x_2;-1)\cup(0;+\infty) $.\\
Từ đó suy ra đồ thị hàm số đã cho có bốn tiệm cận đứng là $ x=0,x=3,x=x_1,x=x_2$.
}
\end{ex}

\begin{ex}%[BG-12NEW-4in1, Nguyen Huynh]%[2D1C4-1]
Gọi $S$ là tập các giá trị nguyên của tham số $m$ sao cho đồ thị hàm số $y = \log (mx^{2} - 2(m+1)x + m+1)$ có hai tiệm cận đứng mà khoảng cách giữa chúng lớn hơn 1. Tích của các phần tử của $S$ bằng bao nhiêu?
\shortans{$-24$}
\loigiai{
Yêu cầu bài toán tương đương với\\ Phương trình $mx^{2} - 2(m+1)x + m+1=0$ có 2 nghiệm phân biệt $x_{1}, x_{2}$ sao cho $|x_{1}-x_{2}|>1$.\\
$\Leftrightarrow \heva{& m \neq 0 \\ & \Delta' > 0 \\ & (x_{1}+x_{2})^{2}-4x_{1}x_{2}>1} \heva{& m\neq 0 \\ & m > -1 \\ & -m^{2}+4m+4 > 0.}$\\
Do $m \in \mathbb{Z}$ nên $S= \{ -1; 1; 2; 3; 4 \}$. Suy ra tích cần tìm bằng $-24$.
}
\end{ex}

\begin{ex}%[BG-12NEW-4in1, Nguyen Huynh]%[2D1C4-1]
Đồ thị hàm số $y=\log\dfrac{x^2-4x+3}{x(x-2)}$ có tất cả bao nhiêu đường tiệm cận?
\shortans{$5$}
\loigiai{
Tập xác định của hàm số là $\mathscr{D}=(-\infty;0)\cup(1;2)\cup(3;+\infty)$.\\
Ta có $\lim\limits_{x\to \pm\infty}\log\dfrac{x^2-4x+3}{x(x-2)}=\log(1)=0$, suy ra đồ thị có tiệm cận ngang là $y=0$.\\
Ta có $\lim\limits_{x\to 0^-}\log f(x)=\lim\limits_{x\to +\infty}\log x=+\infty$, suy ra đồ thị có tiệm cận đứng là $x=0$.\\
Ta có $\lim\limits_{x\to 1^+}\log f(x)=\lim\limits_{x\to 0^+}\log x=-\infty$, suy ra đồ thị có tiệm cận đứng là $x=1$.\\
Ta có $\lim\limits_{x\to 2^-}\log f(x)=\lim\limits_{x\to +\infty}\log x=+\infty$, suy ra đồ thị có tiệm cận đứng là $x=2$.\\
Ta có $\lim\limits_{x\to 3^+}\log f(x)=\lim\limits_{x\to 0^+}\log x=-\infty$, suy ra đồ thị có tiệm cận đứng là $x=3$.
}
\end{ex}

\begin{ex}%[BG-12NEW-4in1, Nguyen Huynh]%[2D1C4-1]
Đồ thị hàm số $y=\log\dfrac{x-2}{x+1}$ có tất cả bao nhiêu đường tiệm cận?
\shortans{$3$}
\loigiai{
Tập xác định của hàm số $\mathscr{D}=(-\infty;-1)\cup(2;+\infty)$.\\
Mà $\lim\limits_{x\to \pm\infty}\log(\dfrac{x-2}{x+1})=\log 1=0$, suy ra $y=0$ là tiệm cận ngang.\\
Mà $\lim\limits_{x\to -1^-}\log\left( \dfrac{x-2}{x+1}\right) =\lim\limits_{x\to -1^-}\log(+\infty)=+\infty$, suy ra $x=-1$ là tiệm cận đứng.\\
Mà $\lim\limits_{x\to 2^+}\log\left( \dfrac{x-2}{x+1}\right) =\lim\limits_{x\to 0^+}\log x=-\infty$, suy ra $x=2$ là tiệm cận đứng.\\
Vậy đồ thị hàm số đã cho có $3$ tiệm cận.
}
\end{ex}

\begin{ex}%[BG-12NEW-4in1, Nguyen Huynh]%[2D1C4-1]
Có tất cả bao nhiêu điểm trên đồ thị hàm số $y=\dfrac{x+1}{x-2}$ sao cho tổng khoảng cách từ điểm đó đến hai đường tiệm cận là nhỏ nhất?
\shortans{$2$}
\loigiai{
Xét $M_0\left(x_0;\dfrac{x_0+1}{x_0-2}\right)$ thuộc đồ thị hàm số.\\
Hai đường tiệm cận của đồ thị hàm số là $x=2$ (tiệm cận đứng) và $y=1$ (tiệm cận ngang).\\
Tổng khoảng cách từ $M_0$ đến hai đường tiệm cận là $$\left|x_0-2\right|+\left|\dfrac{x_0+1}{x_0-2}-1\right|=\left|x_0-2\right|+\dfrac{3}{\left|x_0-2\right|}\geq 2\sqrt{3}.$$
Đẳng thức xảy ra khi và chỉ khi $$\left|x_0-2\right|=\dfrac{3}{\left|x_0-2\right|} \Leftrightarrow |x_0-2|=\sqrt{3} \Leftrightarrow \hoac{x_0=2+\sqrt{3}\\ x_0=2-\sqrt{3}} \Rightarrow \hoac{y_0=1+\sqrt{3}\\ y_0=1-\sqrt{3}.}$$
}
\end{ex}

\begin{ex}%[Mức độ C]%[2D1C3-6]
Ông $A$ muốn xây một cái bể chứa nước lớn dạng một khối hộp chữ nhật không nắp có thể tích bằng $288$cm$^2$. Đáy bể là hình chữ nhật có chiều dài gấp đôi chiều rộng. Hỏi tổng diện tích bể bằng bao nhiêu để chi phí thuê nhân công xây dựng là thấp nhất?

\shortans{$216$}
\loigiai{Theo bài toán, để chi phí thuê nhân công thấp nhất thì ta phải xây dựng bể sao cho tổng diện tích xung quanh và diện tích đáy là nhỏ nhất.\\
Gọi các kích thước của bể lần lượt là $a$(m), $2a$(m), $c$(m).\\
%	\begin{center}
%		\begin{tikzpicture}\def\a{3}\def\b{1}\def\g{30}\def\h{2}
%			\path
%			(0:0) coordinate (A)--++(\g:\b) coordinate (B)--++(0:\a) coordinate (C)--++(\g-180:\b) coordinate (D)
%			\foreach \x in {A,B,C,D}{
%				($(\x)+(90:\h)$) coordinate (\x
%				’)};
%			\draw[dashed] (B’)--(B)--(A)
%			(B)--(C);
%			\draw
%			(A)--(D)--(D’)--(A’)--cycle(A’)--(B’)--(C’)--(D’)(D)--(C)--(C’)			; \end{tikzpicture}
%	\end{center}
Ta có tổng diện tích các mặt cần xây là $S=2a^2+4ac+2ac=2a^2+6ac$.\\
Thể tích bể $V=a \cdot 2a\cdot c=2a^2\cdot c=288 \Rightarrow c=\dfrac{144}{a^2}$.\\
Suy ra $S=2a^2+6a\dfrac{144}{a^2}=2a^2+\dfrac{864}{a}=2a^2+\dfrac{432}{a}+\dfrac{432}{a} \geq 3.\sqrt{2a^2\cdot \dfrac{432}{a}\cdot \dfrac{432}{a}}=216.$\\
Do đó diện tích bể nhỏ nhất là $S=216$.\\
Vậy diện tích bể $S=216$m$^2$ thì chi phí thuê nhân công xây dựng là thấp nhất.}
\end{ex}

\begin{ex}%[Mức độ 4]%[BG12-4IN1, Nguyễn Khánh Trọng]%[2D1C3-6]
\immini[thm]{
Cho một tấm gỗ hình vuông cạnh $200$ cm. Người ta cắt một tấm gỗ có hình một tam giác vuông $ABC$ từ tấm gỗ hình vuông đã cho như hình vẽ bên. Biết $AB=x$ ($0<x<60$ cm) là một cạnh góc vuông của tam giác $ABC$ và tổng độ dài cạnh góc vuông $AB$ với cạnh huyền $BC$ bằng $120$ cm. Tìm $x$ để tam giác $ABC$ có diện tích lớn nhất.
\shortans{$40$}
}{
\begin{tikzpicture}[scale=0.72, font=\footnotesize, line join=round, line cap=round, >=stealth]
\draw[dashed] (0,0)--(4,0)--(0,1)--(0,0);
\draw (4,0)--(5,0)--(5,5)--(0,5)--(0,1);
\node at (0,0.5)[below left] {$x$}; \node at (2,0.5)[above,rotate=-13] {$120-x$}; \node at (2.5,5)[above] {$200$};
\fill (0,0) circle (1.5pt) node[below left]{$A$} (4,0) circle (1.5pt) node[below]{$C$} (0,1) circle (1.5pt) node[left]{$B$};
\end{tikzpicture}
}

\loigiai{
Độ dài cạnh huyền $BC$ là $120-x$.\\
Khi đó độ dài cạnh $AC=\sqrt{BC^2-AB^2}=\sqrt{(120-x)^2-x^2}=\sqrt{14400-240x}$.\\
Diện tích tam giác $ABC$ là $S=\dfrac{1}{2}AB\cdot AC=\dfrac{1}{2}x\sqrt{14400-240x}$.\\
Xét hàm số $f(x)=x\sqrt{14400-240x}$ với $0<x<60$.\\
Ta có $f'(x)=\sqrt{14400-240x}-\dfrac{120x}{\sqrt{14400-240x}}=\dfrac{14400-360x}{\sqrt{14400-240x}}$;\\
$f'(x)=0\Leftrightarrow x=40\in(0;60)$.\\
Bảng biến thiên
\begin{center}
\begin{tikzpicture}
\tkzTabInit[nocadre=false,lgt=1.2,espcl=2.5,deltacl=0.6]
{$x$ /0.6,$f'(x)$ /0.6,$f(x)$ /2}
{$0$,$40$,$60$}
\tkzTabLine{,+,$0$,-,}
\tkzTabVar{-/, +/,-/}
\end{tikzpicture}
\end{center}
Vậy tam giác $ABC$ có diện tích lớn nhất khi $AB=40$ cm.
}
\end{ex}

\begin{ex}%[SGK 12 - Cùng Khám Phá, Mức độ 4]%[BG12-4IN1, Nguyễn Khánh Trọng]%[2D1C3-6]
\immini{Một thùng chứa nhiên liệu gồm phần ở giữa là một hình trụ có chiều dài $h$ mét $(h>0)$ và hai đầu là các nửa hình cầu bán kính $r$ $(r>0)$ (\textit{Hình 1.11}). Biết rằng thể tích của thùng chứa là $144\,000 \pi$ m$^3$. Để sơn mặt ngoài của phần hình cầu cần $20\,000$ đồng cho $1$ m$^2$, còn sơn mặt ngoài cho phần hình trụ cần $10\,000$ đồng cho $1$ m$^2$. Xác định $r$ để chi phí cho việc sơn diện tích mặt ngoài thùng chứa (bao gồm diện tích xung quanh hình trụ và diện tích hai nửa hình cầu) là nhỏ nhất, biết rằng bán kính $r$ không được vượt quá $50$ m.
\shortans{$30$}
}{

\begin{tikzpicture}[scale=.7]
\draw[white,fill=blue!10] (0,0) rectangle (5,3);
\draw(0,0)--(5,0);
\draw(0,3)--(5,3);
\draw[red](0,3)--(0,4);
\draw[red](5,3)--(5,4);
\draw[fill=blue!10] (5,3) arc(90:-90:1.5);
\draw[fill=blue!10] (0,0) arc(-90:-270:1.5);
\draw[white] (0,3) arc (90:270:0.75 and 1.5);
\draw (0,0) arc (-90:90:0.75 and 1.5);
\draw[white] (5,3) arc (90:270:0.75 and 1.5);
\draw (5,0) arc (-90:90:0.75 and 1.5);
\draw[red,<->] (0,3.5)--(5,3.5) node[midway,above]{$h$};
\draw[->] (0,1.5)--(0,3) node[midway,left]{$r$};
\draw (2,0) node[below right]{\textit{Hình 1.11}};
\end{tikzpicture}
}
\loigiai{
Ta có thể tích của thùng chứa nhiên liệu là $V=\pi \cdot r^2 \cdot h + \dfrac{4}{3} \pi \cdot r^3 = 144\, 000 \pi$ \\
Suy ra $h=\dfrac{(144\,000-\dfrac{4}{3} \cdot r^3)}{r^2}$ \\
Khi đó chi phí sơn diện tích mặt ngoài thùng chứa là
$$2 \pi \cdot r \cdot \dfrac{(144\,000-\dfrac{4}{3} \cdot r^3)}{r^2} \cdot 10^4 + 4 \pi \cdot r^2 \cdot 2 \cdot 10^4 =2 \pi \cdot 10^4 \left( \dfrac{144\,000}{r} + \dfrac{8}{3} r^2 \right). $$
Xét hàm số $f(r)=\dfrac{144\,000}{r} + \dfrac{8}{3} r^2$ với $r \in (0;50]$.\\
Ta có $f'(r)=-\dfrac{144\,000}{r^2} + \dfrac{16}{3} r=\dfrac{16r^3-432\,000}{3r^2}$ và $f'(r)=0 \Leftrightarrow r=30$ m.\\
Bảng biến thiên\\
\begin{center}
\begin{tikzpicture}
\tkzTabInit[nocadre=false,lgt=1.2,espcl=3.5,deltacl=0.6]
{$r$/1,$f'(r)$/1,$f(r)$/3}
{$0$,	$30$,	$50$}
\tkzTabLine{,	-,	$0$,	+}
\tkzTabVar{+/$+\infty$,	-/$7200$,	+/$9546{,}7$}
\end{tikzpicture}
\end{center}
Vậy với $r=30$ m thì chi phí cho việc sơn diện tích mặt ngoài của thùng chứa là nhỏ nhất.
}
\end{ex}

\begin{ex}%[Mức độ 4]%[Dự án giảng 12 - Nguyễn Sĩ Đạt]%[2D1C3-4]
Gọi $S$ là tập hợp các giá trị nguyên của tham số $m\in \left[ 0;2024 \right]$ để bất phương trình ${{x}^{2}}-m+\sqrt{{{\left( 1-{{x}^{2}} \right)}^{3}}}\le 0$ nghiệm đúng với mọi $x\in \left[ -1;1 \right]$. Tập $S$ có bao nhiêu phần tử?
\shortans{$2024$}
\loigiai{
Đặt $t=\sqrt{1-{{x}^{2}}}$, với $x\in \left[ -1;1 \right]\Rightarrow t\in \left[ 0;1 \right]$.\\
Bất phương trình đã cho trở thành ${{t}^{3}}-{{t}^{2}}+1-m\le 0\Leftrightarrow m\ge {{t}^{3}}-{{t}^{2}}+1$.(1)\\
Yêu cầu của bài toán tương đương với bất phương trình (1) nghiệm đúng với mọi $t\in \left[ 0;1 \right]$.\\
Xét hàm số $f\left( t \right)={{t}^{3}}-{{t}^{2}}+1\Rightarrow {f}'\left( t \right)=3{{t}^{2}}-2t$.\\
${f}'\left( t \right)=0\Leftrightarrow \hoac{&t=0\notin \left( 0;1 \right)  \\&t=\frac{2}{3}\in \left( 0;1 \right) .}$\\
Vì $f\left( 0 \right)=f\left( 1 \right)=1$, $f\left( \dfrac{2}{3} \right)=\dfrac{23}{27}$ nên $\underset{\left[ 0;1 \right]}{\mathop{\max }}\,f\left( t \right)=1$.\\
Do đó bất phương trình (1) nghiệm đúng với mọi $t\in \left[ 0;1 \right]$ khi và chỉ khi $m\ge 1$.\\
Mặt khác $m$ là số nguyên thuộc $\left[ 0;2024 \right]$ nên $m\in \left\{ 1;2;3;\ldots;2024 \right\}$.\\
Vậy có $2024$ giá trị của $m$ thỏa mãn bài toán.
}
\end{ex}

\begin{ex}%[Mức độ 4]%[BG12-4IN1, Nguyễn Khánh Trọng]%[2D1C3-4]
Cho hàm số $y=f(x)$ có bảng biến thiên như sau
\begin{center}
\begin{tikzpicture}
\tkzTabInit[nocadre=false,lgt=1.2,espcl=2.5,deltacl=0.6]
{$x$ /0.6,$f'(x)$ /0.6,$f(x)$ /2}
{$0$,  $1$, $3$}
\tkzTabLine{,+,$0$,-}
\tkzTabVar{-/ $8$ ,+/$9$,-/$5$}
\end{tikzpicture}
\end{center}
Gọi $S$ là tập hợp các số nguyên dương $m$ để bất phương trình $f(x) \ge mx^2\left(x^2-2\right)+2m$ có nghiệm thuộc đoạn $[0;3]$. Tìm số phần tử của tập $S$.
\shortans{$9$}
\loigiai{
Bất phương trình đã cho tương đương với $\dfrac{f(x)}{x^4-2x^2+2} \ge m$.\\
Từ bảng biến thiên ta thấy $5 \le f(x) \le 9$ với mọi $x\in [0;3]$.\\
Xét hàm số $g(x)=x^4-2x^2+2$ với $x\in [0;3]$ ta có $g'(x)=4x^3-4x$, $g'(x)=0 \Leftrightarrow \hoac{&x=0\\&x=1.}$\\
Ta lại có $g(0)=2$, $g(1)=1$, $g(3)=65$. Từ đó suy ra $1 \le g(x) \le 65$ với mọi $x\in [0;3]$.\\
Xét hàm số $h(x)=\dfrac{f(x)}{g(x)}$, $x\in [0;3]$. Từ đó ta có đánh giá $\dfrac{5}{65} \le h(x) \le 9$ với mọi $x\in [0;3]$.\\
Từ đó suy ra $\min\limits_{x\in [0;3]} h(x)=\dfrac{5}{65}$ khi $x=3$; $\max\limits_{x\in [0;3]} h(x)=9$ khi $x=1$.\\
Vậy bất phương trình đã cho có nghiệm thuộc đoạn $[0;3]$ khi và chỉ khi $ m \le 9$.\\
Vì $m$ nguyên dương nên có tất cả $9$ giá trị thỏa đề bài.
}
\end{ex}

\begin{ex}%[Mức độ 4]%[BG12-4IN1, Nguyễn Khánh Trọng]%[2D1C3-4]
Cho hàm số $ y=f(x) $ liên tục trên $ \mathbb{R} $ và có bảng biến thiên sau
\begin{center}
\begin{tikzpicture}
\tkzTabInit[lgt=1.5,espcl=2]
{$x$/1,$f’(x)$/.7,$f(x)$/2}
{$-\infty$,$-1$,$1$,$\dfrac{21}{4}$,$7$,$10$,$+\infty$}
\tkzTabLine{ ,+,z,-,z,+,z,-,d,+,z,- }
\tkzTabVar{-/$-\infty$,+/$4$,-/$2$,+/$5$,-/$0$,+/$8$,-/$-\infty$}
\end{tikzpicture}
\end{center}
Gọi $S$ là tập hợp các số nguyên của tham số $m\in[-5;15]$ để bất phương trình $f(x^2-2x)-m\ge 0$ có nghiệm
trên khoảng $\left(-\dfrac{3}{2};\dfrac{7}{2}\right)$. Tìm số phần tử của tập $S$.
\shortans{$10$}
\loigiai{
Đặt $t=x^2-2x$. Với $x\in \left[-\dfrac{3}{2};\dfrac{7}{2}\right] \Leftrightarrow -1 \le (x-1)^2-1 \le \dfrac{21}{4}$ nên $t\in \left[-1;\dfrac{21}{4}\right]$.\\
Xét hàm số $ y=f(t)$, với $t \in \left[-1;\dfrac{21}{4}\right] $.\\
Từ BBT, ta có $\max \limits_{t \in \left[-1;\tfrac{21}{4}\right]}f(t)= f\left(\dfrac{21}{4}\right)=5$.\\
Bất phương trình $f(x^2-2x)-m\ge 0$ có nghiệm
trên khoảng $\left(-\dfrac{3}{2};\dfrac{7}{2}\right)$ khi và chỉ khi
$$\max \limits_{t \in \left[-1;\tfrac{21}{4}\right]}f(t)>m\Leftrightarrow m<5.$$
Vì $m$ nguyên thuộc đoạn $[-5;15]$ nên $m\in\left\{-5;-4;-3;\ldots;3;4\right\}$, suy ra ta có $10$ giá trị thỏa đề bài.
}
\end{ex}

\begin{ex}giảng 12-4in1, Nhật Thiện]%[2D1C3-1]
Giá trị lớn nhất của hàm số $y=\dfrac{x^3+x^2-m}{x+1}$ trên $[0;2]$ bằng $5$. Tham số $m$ nhận giá trị là
\shortans{$-3$}
\loigiai{
Đặt $f(x)=\dfrac{x^3+x^2-m}{x+1}$.\\
Giá trị lớn nhất của $y=f(x)$ trên $[0; 2]$ bằng $5\Leftrightarrow \heva{& f(x)\leq 5,  \forall x\in [0;  2] \\ & \exists x_0\in [0;2]\colon f(x_0)=5.}$\\
\begin{itemize}
\item $f(x)\leq 5$, $\forall x\in [0;2]\Leftrightarrow \dfrac{x^3+x^2-m}{x+1}\leq 5$, $\forall x\in [0;2]$\\
\phantom{$f(x)\leq 5$, $\forall x\in [0;2]$} $\Leftrightarrow m\geq x^3+x^2-5x-5$, $\forall x\in [0;2]$\\
\phantom{$f(x)\leq 5$, $\forall x\in [0;2]$} $\Leftrightarrow m\geq \max\limits_{[0;2]} h(x)$, với $h(x)=x^3+x^2-5x-5$.\\
Ta có $h'(x)=3x^2+2x-5$, $h'(x)=0\Leftrightarrow 3x^2+2x-5=0\Leftrightarrow \hoac{& x=1 \\ & x=-\dfrac{5}{3}\;\text{(loại)}.}$\\
Ta có $h(0)=-5$, $h(2)=-3$, $h(1)=-8$.\\
Suy ra $\max\limits_{[0;2]} h(x)=-3$, $\min\limits_{[0;2]} h(x)=-8$.\\
Vậy $m\geq-3$. \hfill $(1)$
\item $\exists x_0\in [0;2]\colon f(x_0)=5\Leftrightarrow \dfrac{x^3+x^2-m}{x+1}=5$ có nghiệm trên $[0;2]$.\\
\phantom{$\exists x_0\in [0;2]\colon f(x_0)=5$} $\Leftrightarrow m=x^3+x^2-5x-5$ có nghiệm trên $[0;2]$.\\
Theo phần trên, ta suy ra $-8\leq m\leq-3$. \hfill $(2)$
\end{itemize}
Từ $(1)$ và $(2)$ suy ra $m=-3$.
}
\end{ex}

\begin{ex}giảng 12-4in1, Nhật Thiện]%[2D1C2-7]
Gia đình An xây bể hình trụ có thể tích $150$\text{m}$^3$. Đáy bể làm bằng bê tông giá $100000$\text{đ/m}$^2$. Phần thân làm bằng vật liệu chống thấm giá $90000$\text{đ/m}$^2$, nắp bằng nhôm giá $120000$\text{đ/m}$^2$. Hỏi tỷ số giữa chiều cao bể và bán kính đáy là bao nhiêu để chi phí sản xuất bể đạt cực đại? (làm tròn đến hai chữ số thập phân)
\shortans{$2{,}44$}
\loigiai{
Ta có $\pi r^2\cdot h=150 \Rightarrow h=\dfrac{150}{\pi r^2}$.\\
$S_{xq}+S_{đáy}+S_{nắp}=2\pi r\cdot h+\pi r^2+\pi r^2=\dfrac{300}{r}+2\pi r^2$.\\
Chi phí sản xuất bể là $S=\dfrac{300}{r}\cdot 90000+\pi r^2\cdot 220000$.\\
Ta có $S'=-\dfrac{27000000}{r^2}+440000\pi\cdot r$; $S'=0 \Leftrightarrow r=\sqrt[3]{\dfrac{675}{11\pi}}$.\\
Bảng biến thiên
\begin{center}
\begin{tikzpicture}
\tkzTabInit[nocadre=false,lgt=1.2,espcl=2.5,deltacl=0.6]
{$r$ /0.96,$S'$ /0.6,$S$ /3}
{$0$,$\sqrt[3]{\dfrac{675}{11\pi}}$,$+\infty$}
\tkzTabLine{,-,$0$,+,}
\tkzTabVar{+/$+\infty$, -/$S\left(\sqrt[3]{\dfrac{675}{11\pi}}\right)$,+/$+\infty$}
\end{tikzpicture}
\end{center}
Suy ra chi phí sản xuất bể đạt cực trị khi $r=\sqrt[3]{\dfrac{675}{11\pi}}\approx 2{,}44$.
}
\end{ex}

\begin{ex}%[MĐ4]%[2D1C2-6]
Cho hàm số $f(x)=\dfrac{x^2-m(m+1)x+m^3+1}{x-m}$ có đồ thị là $(C_m)$. Điểm $A(a;b)$ vừa là điểm cực đại của $(C_{m_1})$ vừa là điểm cực tiểu của $(C_{m_2})$. Tính $a-b$. \shortans{$1{,}25$}
\loigiai{
Ta có $f(x)=\dfrac{x^2-2mx+m^2-1}{(x-m)^2}$. Suy ra $f'(x)= 0 \Leftrightarrow x=m\pm 1$. Do đó, ta có bảng biến thiên
\begin{center}
\begin{tikzpicture}[font=\footnotesize,>=stealth, scale=1]
\tkzTabInit[nocadre=false,lgt=1.2,espcl=3.5,deltacl=0.6]
{$x$ /0.6,$f'(x)$ /0.6,$f$ /2}
{$-\infty$, $m-1$, $m$, $m+1$, $+\infty$}
\tkzTabLine{,+,0,-,d,-,0,+,}
\tkzTabVar{-/$-\infty$, +/$-m^2+m-2$, -D+/$-\infty$/$+\infty$, -/$-m^2+m+2$,+/$+\infty$}
\end{tikzpicture}
\end{center}
Từ giả thiết, ta có
$$ \heva{&m_1-1=m_2+1\\ &-m_1^2+m_1-2=-m_2^2+m_2+2} \Leftrightarrow \heva{&m_1=m_2+2\\ &4m_2=-6} \Leftrightarrow \heva{&m_1=\frac{1}{2}\\ &m_2=-\frac{3}{2}.} $$
Khi đó $A\left(-\dfrac{1}{2};-\dfrac{7}{4}\right)$, vậy $a-b=-\dfrac{1}{2}+\dfrac{7}{4}=1{,}25$.
}
\end{ex}

\begin{ex}%[MĐ4]%[2D1C2-6]
Cho hàm số $f(x)=\dfrac{x^2+m\left(m^2-1\right)x-m^4+1}{x-m}$, với $m$ là tham số, có đồ thị $(C_m)$. Biết rằng tồn tại duy nhất một điểm vừa điểm cực đại của $(C_{m_1})$ và là cực tiểu của $(C_{m_2})$, tính giá trị của $m_1m_2$.
\shortans{$-1$}
\loigiai{
Ta có $f(x)=x+m^3+\dfrac{1}{x-m}$, $f'(x)=1-\dfrac{1}{(x-m)^2}$. Suy ra $f'(x)=0 \Leftrightarrow x=m\pm 1$. Từ đó, ta có bảng biến thiên
\begin{center}
\begin{tikzpicture}[font=\footnotesize,>=stealth, scale=1]
\tkzTabInit[nocadre=false,lgt=1.2,espcl=2.5,deltacl=0.6]
{$x$ /0.6,$f'(x)$ /0.6,$f(x)$ /2}
{$-\infty$, $m-1$, $m$, $m+1$, $+\infty$}
\tkzTabLine{,+,0,-,d,-,0,+,}
\tkzTabVar{-/$-\infty$, +/$y_1$, -D+/$-\infty$/$+\infty$, -/$y_2$, +/$+\infty$}
\end{tikzpicture}
\end{center}
Ta có đường thẳng đi qua hai điểm cực trị có phương trình
$$ y=\dfrac{\left(x^2+m\left(m^2-1\right)x-m^4+1\right)'}{(x-m)'} \text{ hay } y=2x+m\left(m^2-1\right). $$
Suy ra $y_1=m^3+m-2$ và $y_2=m^3+m+2$. Theo giả thiết, tồn tại một điểm vừa là điểm cực đại của $(C_{m_1})$ và vừa là điểm cực tiểu của $(C_{m_2})$ nên
$$ \heva{&m_1-1=m_2+1\\ &m_1^3+m_1-2=m_2^3+m_2+2} \Leftrightarrow \heva{&m_1=m_2+2\\ &m_2^2+2m_2+1=0} \Leftrightarrow \heva{&m_1=1\\ &m_2=-1.} $$
Vậy $m_1m_2=-1$.
}
\end{ex}

\begin{ex}%[Mức độ C]%[Dự án giảng 12 - Trung Anh]%[2D1C2-5]
Với giá trị nào của tham số $m$ thì hàm số $y=x^4+2mx^2+m^2+m$ có ba điểm cực trị lập thành một tam giác có một góc bằng $120^\circ$? (lấy giá trị xấp xỉ đến hàng phần trăm)
\shortans{$-0{,}69$}
\loigiai
{
Tập xác định của hàm số là $\mathscr{D}=\mathbb{R}$.\\
Ta có $y'=4x^3+4mx = 4x(x^2+m)$.
$$y'=0 \Leftrightarrow 4x(x^2+m)=0 \Leftrightarrow \left[\begin{aligned}&x=0 \\&x^2=-m.\end{aligned}\right.$$
Đồ thị hàm số đã cho có ba điểm cực trị khi phương trình $x^2=-m$ có hai nghiệm phân biệt $x\neq 0$, suy ra $-m >0$ hay $m < 0$.\\
Như vậy, với $m<0$ đồ thị hàm số đã cho có điểm cực trị là $A(0;m^2+m)$, $B\left(\sqrt{-m};m\right)$, $C\left(-\sqrt{-m};m\right)$.\\
Dễ thấy tam giác $ABC$ cân tại $A$. Khi đó $\overrightarrow{AB}=\left(\sqrt{-m};-m^2\right)$, $\overrightarrow{AC} = \left(-\sqrt{-m}; -m^2\right)$.\\
Tam giác $ABC$ có một góc bằng $120^\circ$ nên $\widehat{A} = \left(\overrightarrow{AB},\overrightarrow{AC}\right) = 120^\circ$.\\
Suy ra
\begin{eqnarray*}
\cos\left(\overrightarrow{AB},\overrightarrow{AC}\right) = -\dfrac{1}{2} \Leftrightarrow \dfrac{m^4+m}{m^4-m} = -\dfrac{1}{2} \Leftrightarrow 3m^4+m=0 \Leftrightarrow m(3m^3+1)=0 \Leftrightarrow \left[\begin{aligned}&m=0 \\&m=-\dfrac{1}{\sqrt[3]{3}}.\end{aligned}\right.
\end{eqnarray*}
Kết hợp điều kiện $m<0$ ta được $m=-\dfrac{1}{\sqrt[3]{3}}\approx -0{,}69$ là giá trị thỏa mãn yêu cầu bài toán.
}
\end{ex}

\begin{ex}%[Sách tham khảo, Mức độ C]%[Dự án giảng 12 - Trung Anh]%[2D1C2-5]
Cho hàm số $y=x^4-2m^2x^2+m^2$ có đồ thị $(C)$. Tích các giá trị của $m$ để đồ thị $(C)$ có ba điểm cực trị $A$, $B$, $C$ sao cho bốn điểm $A$, $B$, $C$, $O$ là bốn đỉnh của hình thoi ($O$ là gốc tọa độ).
\shortans{$-0{,}5$}
\loigiai{
Ta có $y'=4x^3-4m^2x$; $y'=0\Leftrightarrow \left[ \begin{aligned}
x=0 \\
x=m^2 \\
\end{aligned} \right. $.\\
Điều kiện để hàm số có ba cực trị là $y'=0$ có ba nghiệm phân biệt $\Leftrightarrow m\ne 0$.\\
Khi đó: $y'=0\Leftrightarrow \left[ \begin{aligned}
x=0 \\
x=\pm m \\
\end{aligned} \right. $.\\
Tọa độ các điểm cực trị là $A(0;m^2)$, $B(m;-m^4+m^2)$, $C(m;-m^4+m^2)$.\\
Ta có $OA\bot BC$, nên bốn điểm $A$, $B$, $C$, $O$ là bốn đỉnh của hình thoi điều kiện cần và đủ là $OA$ và $BC$ cắt nhau tại trung điểm mỗi đoạn\\
$\Leftrightarrow \left\{ \begin{aligned}
{{x}_A}+{{x}_O}={{x}_B}+{{x}_C} \\
{{y}_A}+{{y}_O}={{y}_B}+{{y}_C} \\
\end{aligned} \right. \Leftrightarrow \left\{ \begin{aligned}
0=0 \\
m^2+0=(-m^4+m^2)+(-m^4+m^2) \\
\end{aligned} \right. $\\
$\Leftrightarrow 2m^4-m^2=0 \Leftrightarrow m^2=\dfrac{1}{2} \Leftrightarrow m=\pm \dfrac{\sqrt{2}}{2}$.\\
Vậy $m=\pm \dfrac{\sqrt{2}}{2}$.
}
\end{ex}

\begin{ex}%[Mức độ C]%[Dự án giảng 12 - Trung Anh]%[2D1C2-4]
Hàm số $f(x)=\dfrac{1}{3}x^3-x^2+(m^2-3)x+2018$ có hai điểm cực trị $x_1, x_2$. Tìm giá trị lớn nhất của biểu thức $P=|x_1(x_2-2)-2(x_2+1)|$.
\shortans{$9$}
\loigiai{
Tập xác định $\mathbb{R}$. Đạo hàm $y'=x^2-2x+m^2-3$.\\
Hàm số có hai điểm cực trị $\Leftrightarrow $ phương trình $y'=0$ có hai nghiệm phân biệt $\Leftrightarrow \Delta '>0 \Leftrightarrow 4-m^2>0\Leftrightarrow m \in (-2;2)$.\\
Áp dụng định lí Vi-et ta có $x_1+x_2=2;x_1x_2=m^2-3$.\\
Ta có $P=|x_1x_2-2(x_1+x_2)-2|=|m^2-9|$.\\
Xét hàm số $f(m)=m^2-9, m \in (-2;2)$. Ta có $f'(m)=2m=0\Leftrightarrow m=0$.\\
Bảng biến thiên:
\begin{center}
\begin{tikzpicture}
\tkzTabInit[nocadre=false,lgt=1.5,espcl=2.5,deltacl=0.6]
{$x$ /0.6,$f'(m)$ /0.6,$f(m)$ /2}
{$-2$,$0$,$2$}
\tkzTabLine{,-,0,+,}
\tkzTabVar{+/ $-5$ / , -/ $-9$ /, +/ $-5$ /}
\end{tikzpicture}
\end{center}
Vậy $P_{\max}=9$ đạt tại $m=0$.
}
\end{ex}

\begin{ex}%[Sách tham khảo, Mức độ C]%[Dự án giảng 12 - Trung Anh]%[2D1C2-4]
Gọi $S$ là tập hợp giá trị $m$ là số nguyên để hàm số $y = \dfrac{1}{3}x^3 - \left(m + 1\right)x^2 + \left(m - 2\right)x + 2m - 3$ đạt cực trị tại hai điểm $x_{1}$, $x_{2}$ thỏa mãn $x^2_{1} + x^2_{2} = 18$. Tính tổng các phần tử nguyên thuộc tập $S$.
\shortans{$1$}
\loigiai{Tập xác định $\mathscr{D} = \mathbb{R}$.\\
Ta có $y' = x^2 - 2\left(m + 1\right)x + m - 2$.\\
Xét $y' = 0$ suy ra $x^2 - 2\left(m + 1\right)x + m - 2 = 0\quad (*)$.\\
Để hàm số có hai điểm cực trị khi và chỉ khi phương trình $(*)$ có hai nghiệm phân biệt.
$$\Leftrightarrow \Delta' > 0\Leftrightarrow \left(m + 1\right)^2 - \left(m - 2\right) > 0\Leftrightarrow m^2 + m + 3 > 0\Leftrightarrow \left(m + \dfrac{1}{2}\right)^2 + \dfrac{11}{4} > 0.$$
Dễ thấy $(*)$ luôn có $2$ nghiệm phân biệt với mọi $m$.\\
Khi đó $x_{1}$, $x_{2}$ là nghiệm của phương trình $(*)$.\\
Theo định lý Vi-ét ta có $\heva{&x_{1} + x_{2}  = 2m  +2\\ &x_{1}\cdot x_{2} = m - 2}$\\
Để thỏa mãn bài toán $x_{1}^2 + x_{2}^2  = 18\Leftrightarrow \left(x_{1} + x_{2}\right)^2 - 2x_{1}x_{2} - 18 = 0\quad (**)$.\\
Áp dụng định lý Vi-ét  $(**)$ trở thành
\begin{eqnarray*}
&{ }&\left(2m + 2\right)^2- 2\left(m - 2\right) - 18 = 0\\
&\Leftrightarrow& 4m^2 + 6m - 10 = 0\Leftrightarrow\hoac{& m = 1 \\ &m = - \dfrac{5}{2}.}
\end{eqnarray*}
Do giả thiết suy ra $m = 1$ nên $P = 1$.
}
\end{ex}

\begin{ex}%[BG12, Tran Tony]%[2D1C2-3]
Có bao nhiêu giá trị nguyên của tham số $m$ để hàm số $y=x^6+(m+4)x^5+(16-m^2)x^4+2$ đạt cực tiểu tại $x=0$?
\shortans{$8$}
\loigiai{
Tập xác định $\mathscr D=\mathbb{R}$.\\
Ta có $y'=x^3\cdot\left[6x^2+5(m+4)x+64-4m^2\right]$; $y'=0 \Leftrightarrow\hoac{& x=0\,(\text{bội } 3)\\& 6x^2+5(m+4)x+64-4m^2=0. \quad (*)}$\\
Để hàm số đã cho đạt cực tiểu tại $x=0$ thì $\hoac{& (*) \text{ vô nghiệm}\\& (*) \text{ có nghiệm kép } x=0\\& (*) \text{ có hai nghiệm phân biệt cùng dấu (do } a=6>0).}$
\begin{itemize}
\item \textbf{Trường hợp 1.} $(*)$ vô nghiệm $\Leftrightarrow \Delta<0 \Leftrightarrow 25\cdot (m+4)^2-24\cdot (64-4m^2)<0$.\\
$$\Leftrightarrow 121m^2+200m-1136<0 \Leftrightarrow -4<m<\dfrac{284}{121}.$$
\item \textbf{Trường hợp 2.} $(*)$ có nghiệm kép $x=0 \Leftrightarrow \heva{& \Delta=0\\& x=0.}$\\
$$\Leftrightarrow \heva{& 121m^2+200m-1136=0\\& 64-4m^2=0} \Leftrightarrow \heva{& m=-4 \vee m=\dfrac{284}{121}\\& m=\pm 4} \Leftrightarrow m=-4.$$
\item \textbf{Trường hợp 3.} $(*)$ có hai nghiệm phân biệt khác cùng dấu $\Leftrightarrow \heva{& \Delta>0\\& 6\cdot (64-4m^2)>0.}$\\
$$\Leftrightarrow \heva{& 121m^2+200m-1136>0\\& 64-4m^2>0} \Leftrightarrow \heva{& m\in (-\infty; -4) \cup \left(\dfrac{284}{121}; +\infty\right)\\& m\in (-4;4)} \Leftrightarrow m\in \left(\dfrac{284}{121}; 4\right).$$
\end{itemize}
Do đó $m\in \left[-4; \dfrac{284}{121}\right)\cup \left(\dfrac{284}{121};4\right)$.\\
Lại có $m\in\mathbb{Z}$ nên $m\in \{-4;-3;-2;-1;0;1;2;3\}$.\\
Vậy có $8$ giá trị nguyên của $m$ thỏa mãn yêu cầu bài toán.
}
\end{ex}

\begin{ex}%[BG12, Tran Tony]%[2D1C2-3]
Cho hàm số $y=x^2-2mx-2\ln \left(x^2-2mx+m^2+1\right)$, với $m$ là tham số. Gọi $S$ là tập hợp các giá trị của $m$ để hàm số đã cho đạt cực tiểu tại điểm $x=2$. Tính tổng các phần tử của $S$.
\shortans{$4$}
\loigiai{
Hàm số xác định khi $x^2-2mx+m^2+1>0\Leftrightarrow (x-m)^2+1>0,\,\forall x\in\mathbb{R}$.\\
Do đó hàm số có tập xác định $\mathscr D=\mathbb{R}$.\\
Ta có
\allowdisplaybreaks
\begin{eqnarray*}
y'&=&2x-2m-\dfrac{2(2x-2m)}{x^2-2mx+m^2+1}\\
&=&2(x-m)\left[1-\dfrac{2}{x^2-2mx+m^2+1}\right]\\
&=&\dfrac{2(x-m)(x^2-2mx+m^2-1)}{x^2-2mx+m^2+1}\\
&=&\dfrac{2(x-m)(x-m-1)(x-m+1)}{x^2-2mx+m^2+1}.
\end{eqnarray*}
Bảng xét dấu $y'$
\begin{center}
\begin{tikzpicture}
\tkzTabInit[lgt=1.2,espcl=3]
{$x$ /0.6, $y'$ /0.6}
{$-\infty$,$m-1$,$m$,$m+1$,$+\infty$}
\tkzTabLine{ ,-,$0$,+,$0$,-,$0$, +, }
\end{tikzpicture}
\end{center}
Từ bảng xét dấu $y'$, suy ra hàm số đạt cực tiểu tại các điểm $x=m-1$ và $x=m+1$.\\
Do đó, để hàm số đạt cực tiểu tại điểm $x=2$ thì $\hoac{&m-1=2\\&m+1=2}\Leftrightarrow\hoac{&m=3\\&m=1.}$\\
Suy ra $S=\{1;3\}$.\\
Vậy tổng các phần tử của $S$ là $4$.
}
\end{ex}

\begin{ex}%[BG12, Tran Tony]%[2D1C2-2]
\immini{Cho hàm số $f(x)$ có đạo hàm liên tục trên $\mathbb{R}$. Đồ thị của hàm số $y=f(5-2x)$ như hình vẽ bên. Có bao nhiêu giá trị thực của tham số $m$ thuộc khoảng $(-9;9)$ thoả mãn $2m\in \mathbb{Z}$ và hàm số $y=\left|2f\left(4x^3+1\right)+m-\dfrac{1}{2}\right|$ có $5$ điểm cực trị?
\shortans{$26$}
}
{
\begin{tikzpicture}[>=stealth,line cap=round,line join=round,scale=0.5,font=\footnotesize]
\draw[->] (-1.5,0) -- (5,0) node[below] {\scriptsize $x$};
\draw[->] (0,-4.5) -- (0,4) node[left] {\scriptsize $y$};
\draw (0,0)node[below left]{\scriptsize $O$};
\clip (-1.5,-4.5) rectangle(5,4);
\draw[samples=150,smooth,domain=-1.5:3] plot(\x,{-9/16*(\x)^2*(\x-3)});
\draw[samples=150,smooth,domain=3:5] plot(\x,{2.37*(\x)^3-22.21*(\x)^2+63.88*(\x)-55.67});
\draw[dashed] (2,0)|-(0,2.25) (4,0)|-(0,-3.8);
\draw[fill] (0,-3.8) circle(1pt) node[left]{$-4$} (2,0) circle(1pt) node[below]{$2$} (4,0) circle(1pt) node[above]{$4$} (0,2.25) circle(1pt) node[left]{$\dfrac{9}{4}$} (2,2.25) circle(1pt) (4,-3.8) circle(1pt) (0,0) circle(1pt);
\end{tikzpicture}
}
\loigiai{
Dựa vào đồ thị, ta thấy hàm số 	$y=f(5-2x)$ có ba điểm cực trị là $0$, $2$, $4$.\\
Suy ra phương trình $y'=-2f'(5-2x)=0$ có ba nghiệm phân biệt $0$, $2$, $4$ $\Rightarrow \hoac{& 5-2x=0\\& 5-2x=2\\ & 5-2x=4}\Leftrightarrow \hoac{& x=\dfrac{5}{2}\\ & x=\dfrac{3}{2}\\ & x=\dfrac{1}{2}.}$\\
Suy ra hàm số $f(x)$ có ba điểm cực trị là $\dfrac{1}{2}$, $\dfrac{3}{2}$, $\dfrac{5}{2}$.\\
Ta có bảng biến thiên của hàm số $y=f(x)$ như sau
\begin{center}
\begin{tikzpicture}
\tkzTabInit[espcl=2.5,lgt=1.5]
{$x$/1,$f'(x)$/0.7,$f(x)$/1.8}
{$-\infty$,$\dfrac{1}{2}$, $\dfrac{3}{2}$,$\dfrac{5}{2}$,$+\infty$}
\tkzTabLine{,-,0,,+,0,-,0,+}
\tkzTabVar{+/$+\infty$,-/$-4$,+/$\dfrac{9}{4}$,-/$0$,+/$+\infty$}
\end{tikzpicture}
\end{center}
Đặt $t=4x^3+1$, dễ thấy hàm số $u$ đồng biến trên $\mathbb{R}$ và ứng với mỗi giá trị $t$, ta tìm được duy nhất giá trị $x$.\\
Ta cũng có bảng biến thiên của hàm số $y=f\left(4x^3+1\right)$ như sau
\begin{center}
\begin{tikzpicture}
\tkzTabInit[espcl=2.5,lgt=2.5]
{$x$/0.7,$f\left(4x^3+1\right)$/2}
{$-\infty$,$a$, $b$, $c$,$+\infty$}
\tkzTabVar{+/$+\infty$,-/$-4$,+/$\dfrac{9}{4}$,-/$0$,+/$+\infty$}
\end{tikzpicture}
\end{center}
Từ đó suy ra hàm số $f\left(4x^3+1\right)$ có ba điểm cực trị.\\
Suy ra hàm số $y=2f\left(4x^3+1\right)+m-\dfrac{1}{2}$ có ba điểm cực trị.\\
Do đó, hàm số $y=\left|2f\left(4x^3+1\right)+m-\dfrac{1}{2}\right|$ có $5$ điểm cực trị khi và chỉ khi phương trình $$2f\left(4x^3+1\right)+m-\dfrac{1}{2}=0$$ có hai nghiệm bội lẻ phân biệt.\\
Xét phương trình $2f\left(4x^3+1\right)+m-\dfrac{1}{2}=0\Leftrightarrow -\dfrac{m}{2}+\dfrac{1}{4}=f\left(4x^3+1\right)$.\quad $(1)$\\
Dựa vào bảng biến thiên của $f\left(4x^3+1\right)$, phương trình $(1)$ có hai nghiệm bội lẻ khi và chỉ khi
$$\hoac{&-\dfrac{m}{2}+\dfrac{1}{4}\ge \dfrac{9}{4}\\ & -4<-\dfrac{m}{2}+\dfrac{1}{4}\le 0}\Leftrightarrow \hoac{& m\le -4\\ & \dfrac{1}{2}\le m<\dfrac{17}{2}}\Leftrightarrow \hoac{& 2m\le -8\\ & 1\le 2m<17.}$$
Do $m\in (-9;9)$ nên $2m\in (-18;18)$ và $2m\in \mathbb{Z}$ nên $2m\in \{-17;-16;\ldots;-8;1;2;\ldots;16\}$.\\
Vậy có $26$ giá trị $m$ cần tìm.
}
\end{ex}

\begin{ex}%[BG12, Tran Tony]%[2D1C2-2]
\immini{
Cho hàm số bậc bốn $y=f(x)$ có đồ thị hàm số $y=f'(x)$ như hình vẽ. Gọi $m$, $n$ lần lượt là số điểm cực đại và số điểm cực tiểu của hàm số $$h(x)=2f\left(\left|3-x\right|\right)+1.$$ Tính $T=2m+3n$
\shortans{$13$}
}
{
\begin{tikzpicture}[line join = round, line cap = round,>=stealth,font=\footnotesize,scale=0.8]
\def \xmin{-1.5};
\def \xmax{3};
\def \ymin{-1.3};
\def \ymax{2.3};
\draw[->] (\xmin,0) -- (\xmax,0) node[below] {$x$};
\draw[->] (0,\ymin) -- (0,0) node[below left] {$O$} -- (0,\ymax) node[left] {$y$};
\clip (\xmin+0.1,\ymin+0.1) rectangle (\xmax+0.1,\ymax-0.1);
\draw[smooth, samples=100] plot[domain=-1.5:1.45] (\x,{2*(\x+1)*(\x-0.5)*(\x-1)}) node[below right] {$y=f'(x)$};
\end{tikzpicture}
}
\loigiai{
Dựa vào đồ thị ta thấy $f'(x)=0\Leftrightarrow \hoac{& x=a<0 \\ & x=b>0\\& x=c>b.}$\\
Ta có $h'(x)=2\dfrac{(x-3)}{|x-3|}f'\left(|3-x|\right)$, $h'(x)=0\Leftrightarrow \hoac{& |3-x|=b \\ & |3-x|=c}\Leftrightarrow \hoac{& x=3+b \\ & x=3-b\\& x=3+c\\& x=3-c.}$\\
Bảng xét dấu đạo hàm
\begin{center}
\begin{tikzpicture}[line join = round, line cap = round,>=stealth,font=\footnotesize,scale=1]
\tkzTabInit[nocadre=false,lgt=1.2,espcl=2.5,deltacl=0.6]
{$x$ /0.6, $h'(x)$ /0.6}
{$-\infty$,$3-c$,$3-b$,$3$,$3+b$,$3+c$,$+\infty$}
\tkzTabLine{ ,-,z,+,z,-,d,+,z,-,z,+, }
\end{tikzpicture}
\end{center}
Dựa vào bảng xét dấu của $h'(x)$ ta thấy hàm số $y=h(x)$ có $2$ điểm cực đại và $3$ điểm cực tiểu.\\
Vậy $T=2m+3n=13$.
}
\end{ex}

\begin{ex}%[Mức độ 4]%[2D1C2-1]
Giả sử $A$, $B$ là hai điểm cực trị của đồ thị hàm số $y=x^3+a x^2+b x+c$ và đường thẳng $(AB)$ đi qua gốc tọa độ. Giá trị nhỏ nhất $\mathrm{P}_{\min}$ của $P=a b c+a b+c$ bằng $-\dfrac{m}{n}$ (với $\dfrac{m}{n}$ là phân số tối giản, $m;n$ nguyên dương). Tính $m+n$.
\shortans{$34$}
\loigiai
{
Tập xác định $\mathscr{D}=\mathbb{R}$.\\
$f'(x)=3x^2+2ax+b$.\\
Điều kiện để hàm số có hai điểm cực trị là $f'(x)=0$ có hai nghiệm phân biệt $\Rightarrow a^2-3b>0$.\\
Lấy $f(x)$ chia cho $f'(x)$, ta có $f(x)=f'(x)\left(\dfrac{1}{3}x+\dfrac{1}{9}a\right)+\left(\dfrac{2}{3}b-\dfrac{2}{9}\right)x+c-\dfrac{1}{9}ab$.\\
Suy ra, đường thẳng qua hai cực trị là $(AB): y=\left(\dfrac{2}{3} b-\dfrac{2a^2}{9}\right) x+c-\dfrac{a b}{9}$.\\
Do $(AB)$ qua gốc $O$ nên $c-\dfrac{a b}{9}=0\Leftrightarrow a b=9c$.\\
Khi đó $P=a b c+a b+c=9c^2+10c=\left(3c+\dfrac{5}{3}\right)^2-\dfrac{25}{9}\ge-\dfrac{25}{9},\forall c\in\mathbb{R}$.\\
Vậy $\mathrm{P}_{\text{min}}=-\dfrac{25}{9}$ khi $\heva{&c=-\dfrac{5}{9}\\&a b=-5.}$\\
Suy ra $m+n=34$.
}
\end{ex}

\begin{ex}%[Mức độ 4]%[2D1C2-1]
Cho hàm số $y=f(x)$ có đúng ba điểm cực trị là $-2;-1;0$ và có đạo hàm liên tục trên $\mathbb{R}$. Khi đó hàm số $y=f\left(x^2-2x\right)$ có bao nhiêu điểm cực trị?
\shortans{$7$}
\loigiai
{
Do hàm số $y=f(x)$ có đúng ba điểm cực trị là $-2;-1;0$ và có đạo hàm liên tục trên $\mathbb{R}$ nên $f'(x)=0$ có ba nghiệm là $x=-2;x=-1;x=0$.\\
Đặt $g(x)=f\left(x^2-2x\right)\Rightarrow g'(x)=(2x-2)\cdot f'\left(x^2-2x\right)$. \\
Vì $f'(x)$ liên tục trên $\mathbb{R}$ nên $g'(x)$ cũng liên tục trên $\mathbb{R}$.\\
Do đó những điểm $g'(x)$ có thể đổi dấu khi đi qua các điểm thỏa mãn
$$\hoac{&{2 x-2=0}\\&
{x^2-2 x=-2}\\&
{x^2-2 x=-1}\\&
{x^2-2 x=0}}
\Leftrightarrow\hoac{&x=1\\&x=0\\&x=2.}$$
Vậy hàm số $g(x)$ có ba điểm cực trị.
}
\end{ex}

\begin{ex}%[CD12-CTST, Mức độ 4]%[2D1C1-5]
\immini{Mặt cắt ngang của một máng dẫn nước là một hình thang cân có độ dài đáy bé bằng độ dài cạnh bên và bằng $a$ (cm) không đổi (Hình vẽ). Gọi $\alpha$ là một góc của hình thang cân tạo bởi đáy bé và cạnh bên $\left(\dfrac{\pi}{2} \leq \alpha<\pi\right)$. Tìm $\alpha$ để diện tích mặt cắt ngang của máng lớn nhất.}
{\begin{tikzpicture}[>=stealth,line join=round,line cap=round,font=\footnotesize,scale=0.6]
\path
(0,0) coordinate (D)
(4,0) coordinate (C)
($(C)!1!-120:(D)$) coordinate (B)
($(D)!1!120:(C)$) coordinate (A)
;
\draw(A)--(D)node[pos=0.5,left]{$a$};
\draw(D)--(C)node[pos=0.5,below]{$a$};
\draw(C)--(B)node[pos=0.5,right]{$a$};
\draw[fill=cyan!80!blue] (A)--(B)--(C)--(D)--(A);
\draw (A)--(D)--(C)--(B);
\draw pic[draw, angle radius=3mm, angle eccentricity=1.5]{ angle = B--C--D};
\node at ($(C)+(100:0.75)$){\small $\alpha$};
%			\draw (1.8,-0.7)node[below]{$\text{Hình 5}$ };
\end{tikzpicture}}
Hàm số $S(\alpha)$ mô tả diện tích mặt cắt ngang theo góc $\alpha$ có bảng biến thiên như sau
\begin{center}
\begin{tikzpicture}
\tkzTabInit[lgt=1.5, espcl=4]
{$x$/.6,$S'(\alpha)$/0.6,$S(\alpha)$/3}{$0$,$a$,$\pi$}
\tkzTabLine{,+,0,-,}
\tkzTabVar{-/$0$,+/$b$,-/$1$}
\end{tikzpicture}
\end{center}
Tính $a\cdot b$. (làm tròn đến hàng phần trăm)
\shortans{$2{,}72$}
\loigiai{
\immini{Vì $ABCD$ là hình thang cân nên $\heva{&\widehat{A}=\widehat{B}\\&\widehat{C}=\widehat{D}=\alpha.}$\\
$\widehat{A}+\widehat{B}+\widehat{C}+\widehat{D}=2\pi$.\\
$2\widehat{A}+2\alpha=2\pi$ hay $\widehat{A}=\pi-\alpha$.\\
$DH=AD\sin A=a\cdot \sin \left(\pi-a\right)=a\sin \alpha$.\\
$\begin{aligned}[t]AB&=DC+2AH\\
&=a+2a\cos \left(\pi-a\right)\\
&=a-2a\cos \alpha\\
&=a\left(1-2\cos \alpha\right).
\end{aligned}$}
{\begin{tikzpicture}[>=stealth,line join=round,line cap=round,font=\footnotesize,scale=0.6]
\path
(0,0) coordinate (D)
(4,0) coordinate (C)
($(C)!1!-120:(D)$) coordinate (B)
($(D)!1!120:(C)$) coordinate (A)
($(A)!(D)!(B)$) coordinate (H)
;
\draw(A)--(D)node[pos=0.5,left]{$a$};
\draw(D)--(C)node[pos=0.5,below]{$a$};
\draw(C)--(B)node[pos=0.5,right]{$a$};
\draw[fill=cyan!80!blue] (A)--(B)--(C)--(D)--(A);
\draw(A)--(D)--(C)--(B) (D)--(H);
\foreach \x/\y in {A/90,B/90,C/-90,D/-90,H/90}
\draw[fill=black] (\x) circle (1.1pt) + (\y:0.5cm) node{$\x$};
\end{tikzpicture}}
\noindent Diện tích mặt cắt ngang
\begin{eqnarray*}
S&=&\dfrac{1}{2}\cdot \left(AB+CD\right)\cdot DH\\
&=&\dfrac{1}{2}\left[a\left(1-2\cos \alpha\right)+a\right]a\sin \alpha\\
&=&a^2\left(1-\cos \alpha\right)\sin \alpha.
\end{eqnarray*}
$S'(\alpha)=2a^2\sin \dfrac{3\alpha}{2}\sin \dfrac{\alpha}{2}$.\\
$S'(\alpha)=0\Leftrightarrow \hoac{&\alpha=\dfrac{2k\pi}{3}\\&\alpha=2k\pi.}\, (k\in \mathbb{Z})$.\\
Vì $\dfrac{\pi}{2}\leq \alpha<\pi$ nên  $x=\dfrac{2\pi}{3}$.\\
$S\left(\dfrac{2\pi}{3}\right)=\dfrac{3\sqrt{3}}{4}$.\\
$S\left(\dfrac{\pi}{2}\right)=1$.\\
Bảng biến thiên
\begin{center}
\begin{tikzpicture}
\tkzTabInit[lgt=1.5, espcl=4]
{$x$/1,$S'(\alpha)$/0.6,$S(\alpha)$/3}{$0$,$\dfrac{2\pi}{3}$,$\pi$}
\tkzTabLine{,+,0,-,}
\tkzTabVar{-/$0$,+/$\dfrac{3\sqrt{3}}{4}$,-/$1$}
\end{tikzpicture}
\end{center}
Vậy $a\cdot b =\dfrac{2\pi}{3}\cdot \dfrac{3\sqrt{3}}{4}\approx2{,}72$.
}
\end{ex}

\begin{ex}%[CD12-CTST, Mức độ 4]%[2D1C1-5]
\immini{

Người ta muốn thiết kế một lồng nuôi cá có bề mặt hình chữ nhật bao gồm phần mặt nước có diện tích bằng $54$ m$^2$ và phần đường đi xung quanh với kích thước (đơn vị: m) như Hình vẽ.
Khi kích thước $a$ thay đổi trong khoảng $(3;+\infty)$ thì giá trị hàm số mô tả diện tích lối đi theo kích thước $a$ sẽ giảm đến giá trị $S_0$ rồi tăng lên. Xác định giá trị $S_0$.

}
{\begin{tikzpicture}[>=stealth,line join=round,line cap=round,font=\footnotesize,scale=0.75]
\path
(0,0) coordinate (A)
(5,0) coordinate (B)
(0,4) coordinate (D)
($(D)+(B)-(A)$) coordinate (C)
($(B)+(-0.5,0)$) coordinate (M)
($(B)+(0,0.5)$) coordinate (N)
($(A)+(1,0)$) coordinate (H)
($(A)+(0,0.5)$) coordinate (T)
($(C)+(-0.5,0)$) coordinate (Q)
($(C)+(0,-0.5)$) coordinate (P)
($(D)+(1,0)$) coordinate (R)
($(D)+(0,-0.5)$) coordinate (S)
($(T)+(H)-(A)$) coordinate (A')
($(M)+(N)-(B)$) coordinate (B')
($(Q)+(P)-(C)$) coordinate (C')
($(R)+(S)-(D)$) coordinate (D')
($(D)+(0,0.5)$) coordinate (x)
($(C)+(0,0.5)$) coordinate (y)
($(D)+(-0.5,0)$) coordinate (u)
($(A)+(-0.5,0)$) coordinate (v)
($(A)+(0,-0.5)$) coordinate (x')
($(H)+(0,-0.5)$) coordinate (y')
($(M)+(0,-0.5)$) coordinate (u')
($(B)+(0,-0.5)$) coordinate (v')
($(C)+(0.5,0)$) coordinate (x'')
($(P)+(0.5,0)$) coordinate (y'')
($(N)+(0.5,0)$) coordinate (u'')
($(B)+(0.5,0)$) coordinate (v'')
;
\draw[fill=cyan!20!brown](A)--(B)--(C)--(D)--(A);
\draw[fill=cyan!90!blue](A')--(B')--(C')--(D')--(A');
\draw(D)--(x) (C)--(y)(D)--(u)(A)--(v)(A)--(x')(H)--(y')(M)--(u')(B)--(v')(C)--(x'') (P)--(y'') (N)--(u'') (B)--(v'');
\draw[<->] (x)--(y)node[pos=0.5,above]{$a$};
\draw[<->] (u)--(v)node[pos=0.5,left]{$b$};
\draw[<->] (x')--(y')node[pos=0.5,below]{$2$};
\draw[<->] (u')--(v')node[pos=0.5,below]{$1$};
\draw[<->] (u'')--(v'')node[pos=0.5,right]{$1$};
\draw[<->] (x'')--(y'')node[pos=0.5,right]{$1$};
%			\draw (2.2,-0.5)node[below]{$\text{Hình 8}$ };
\end{tikzpicture}}
\shortans{$42$}
\loigiai{
Gọi $x$, $y$ lần lượt là độ dài, rộng của mặt nước. Điều kiện $x$, $y>0$.\\
Phần mặt nước có diện tích bằng $54$ m$^2$ nên ta có $$x\cdot y=54.\, \hfill(*)$$
Theo đề bài ta có $x=a-3$, $y=b-2$.\\
Từ $(*)$ suy ra $$(a-3)(b-2)=54\Rightarrow b=\dfrac{54}{a-3}+2=\dfrac{2a+48}{a-3}.$$
Diện tích lối đi là \begin{eqnarray*}
S(a)&=&a\cdot b-x\cdot y\\
&=&ab-54\\
&=&a\cdot \dfrac{2a+48}{a-3}-54\\
&=&\dfrac{2a^2+48a}{a-3}-54.
\end{eqnarray*}
$S'(a)=\dfrac{2a^2-12a-144}{\left(a-3\right)^2}$;\\
$S'(a)=0\Leftrightarrow \hoac{&a=-6\\&a=12.}$\\
Bảng biến thiên
\begin{center}
\begin{tikzpicture}
\tkzTabInit[nocadre=false, lgt=1.2, espcl=2.4]{$a$ /0.7,$S'(a)$ /0.7,$S(a)$ /2.5}{$0$,$3$,$12$,$+\infty$}
\tkzTabLine{,-,d,-,$0$,+,}
\tkzTabVar{+/$-54$  ,-D+/$-\infty$/$+\infty$,-/$42$,+/$+\infty$}
\end{tikzpicture}
\end{center}
Vậy $S_0=42$.
}
\end{ex}

\begin{ex}%[Mức độ 3]%[Dự án giảng new 4in1, Trần Quang Thạnh]%[2D1C1-4]
Tính tổng tất cả các nghiệm của phương trình $\log_{3}\dfrac{x^{2}+x+3}{2x^{2}+4x+5}=x^{2}+3x+2 $.
\shortans{$1$}
\loigiai{Xét phương trình $\log_{3}\dfrac{x^{2}+x+3}{2x^{2}+4x+5}=x^{2}+3x+2$. $\qquad(*)$\\
Điều kiện xác định $ \heva{&\dfrac{x^{2}+x+3}{2x^{2}+4x+5}>0\\ &2x^{2}+4x+5\neq 0}\Leftrightarrow \forall x\in\mathbb{R} $.\\
Ta có
\begin{eqnarray*}
(*)&\Leftrightarrow& \log_{3} (x^{2}+x+3)-\log_{3}(2x^{2}+4x+5)=(2x^{2}+4x+5)-(x^{2}+x+3)\\
&\Leftrightarrow& (x^{2}+x+3)+\log_{3} (x^{2}+x+3)=\log_{3}(2x^{2}+4x+5)+(2x^{2}+4x+5).
\end{eqnarray*}
Xét hàm số $ f(t)=t+\log_{3}t $ trên khoảng $ (0;+\infty ) $.\\
Ta có cơ số $ 3>1 $ và hàm số $ y=t $ đồng biến nên $ f(t) $ đồng biến trên khoảng $ (0;+\infty) $.\\
Do đó $ f(x^{2}+x+3)=f(2x^{2}+4x+5)\Leftrightarrow x^{2}+x+3=2x^{2}+4x+5\Leftrightarrow x^{2}+3x+2=0\Leftrightarrow\hoac{&x=-1\\ &x=2.} $\\
Do đó $ x=-1 $, $ x=2 $ là các nghiệm của phương trình.\\
Vậy tổng các nghiệm của phương trình là $1$.
}
\end{ex}

\begin{ex}%[Mức độ 4]%[Dự án giảng new 4in1, Trần Quang Thạnh]%[2D1C1-4]
Có bao nhiêu số nguyên dương $y>4$ sao cho tồn tại số thực $x\in(1;6)$ thỏa mãn $4(x-1)\mathrm{e}^x=y(\mathrm{e}^x+xy-2x^2-3)$?
\shortans{$14$}
\loigiai{
\allowdisplaybreaks
\begin{eqnarray*}
&&	4(x-1)\mathrm{e}^x=y(\mathrm{e}^x+xy-2x^2-3)\\
&\Leftrightarrow& 4(x-1)\mathrm{e}^x-y(\mathrm{e}^x+xy-2x^2-3)=0.
\end{eqnarray*}
Xét hàm số $y=f(x)=4(x-1)\mathrm{e}^x-y(\mathrm{e}^x+xy-2x^2-3)$ liên tục trên $[1;6]$ có
\allowdisplaybreaks
\begin{eqnarray*}
f'(x)&=&4\mathrm{e}^x+4(x-1)\mathrm{e}^x-y(\mathrm{e}^x+y-4x)\\
&=&(\mathrm{e}^x+y)(4x-y).
\end{eqnarray*}
\noindent
Cho $f'(x)=0\Leftrightarrow x=\dfrac{y}{4}$.\\
Do $x\in(1;6)$ nên hàm số $y=f(x)$ sẽ tồn tại điểm cực trị $x=\dfrac{y}{4}$ khi $ y\in (4;24)$.\\
Từ đó ta có cơ sở chia các trường hợp như sau
\begin{itemize}
\item Trường hợp 1: $y\ge 24$.
\begin{center}
\begin{tikzpicture}
\tkzTabInit[nocadre=false,lgt=1.2,espcl=2.5,deltacl=0.6]
{$x$/0.7,$f'(x)$/0.7,$f(x)$/2}
{$1$,$6$}
\tkzTabLine{,-,}
\tkzTabVar{+/$f(1)$,-/$f(6)$}
\end{tikzpicture}
\end{center}
Ta có $\heva{&f(1)=-y(\mathrm{e}+y-5)\\&f(6)=20\mathrm{e}^6-y(\mathrm{e}^6+6y-75).}$\\
Điều kiện cần và đủ để tồn tại $x$ là
$$\heva{&f(6)<0\\&f(1)\cdot f(6)<0}\Rightarrow f(1)>0.$$
Mặt khác ta thấy $-y(\mathrm{e}+y-5)<0,\;\forall y\ge 24$ (vô lí) nên loại.
\item Trường hợp 2: $4<y<24$.
\begin{center}
\begin{tikzpicture}
\tkzTabInit[nocadre=false,lgt=1.2,espcl=2.5,deltacl=0.6]
{$x$/0.7,$f'(x)$/0.7,$f(x)$/2}
{$1$,$\tfrac{y}{4}$,$6$}
\tkzTabLine{,-,z,+,}
\tkzTabVar{+/$f(1)$,-/$f\left(\tfrac{y}{4}\right)$,+/$f(6)$}
\end{tikzpicture}
\end{center}
Do $f(1)<0$ nên để tồn tại nghiệm $x\in(1;6)$ thì $f(6)>0$
\allowdisplaybreaks
\begin{eqnarray*}
&\Leftrightarrow&20\mathrm{e}^6-y(\mathrm{e}^6+6y-75>0\\
&\Leftrightarrow&\heva{&-6y^2-(\mathrm{e}^6-75)y+20\mathrm{e}^6>0\\&y\in\mathbb{N}^*;\;y\in(4;24)}\\
&\Leftrightarrow&y\in\{5;6;\ldots;18\}.
\end{eqnarray*}
\end{itemize}
Vậy có tất cả $14$ giá trị nguyên dương $y$ thỏa đề bài.
}
\end{ex}

\begin{ex}%[Đề Tham khảo 2021, Mức độ 4]%[Dự án giảng new 4in1, Trần Quang Thạnh]%[2D1C1-4]
Có bao nhiêu số nguyên $a ~(a\ge2)$ sao cho tồn tại số thực $x$ thỏa mãn $\left(a^{\log x}+2\right)^{\log a}=x-2$?
\shortans{$8$}
\loigiai{
Với $x>0$ đặt $y=a^{\log x}+2>0$ ta được $y^{\log a}=x-2\Leftrightarrow x=a^{\log y}+2$. \\
Từ đó ta có $y=a^{\log x}+2$ và $x=a^{\log y}+2$.\\
Do $a\ge2$ nên $f(t)=a^t+2$ đồng biến trên $\mathbb R$.\\ Giả sử $x\ge y$ thì $f(y)\ge f(x)$, suy ra $y\ge x$ tức là $x=y$. Tương tự $x\le y$ cũng có $x=y$.\\
Vì thế chỉ xét phương trình $x=a^{\log x}+2$ với $x>0$ hay $x-x^{\log a}=2$.\\
Ta phải có $x>2$ và $x>x^{\log a}\Leftrightarrow 1>\log a\Leftrightarrow a<10$.\\
Ngược lại $a<10$ thì xét hàm số liên tục $g(x)=x-x^{\log a}-2=x^{\log a}\left(x^{1-\log a}-1\right)-2$ có $\lim\limits_{x\to+\infty}g(x)=+\infty$ và $g(2)<0$ nên $g(x)$ sẽ có nghiệm trên $(2;+\infty)$. \\
Do đó các số $a\in \{2,3,\ldots,9\}$ đều thỏa mãn.
}
\end{ex}

\begin{ex}%[Mức độ 4]%[Dự án giảng new 4in1, Trần Quang Thạnh]%[2D1C1-3]
Có bao nhiêu giá trị nguyên của tham số $m$ trên khoảng $(-100;100)$ sao cho hàm số $y=\dfrac{-\mathrm{e}^x+3}{\mathrm{e}^x+m}$ nghịch biến trên khoảng $(0;+\infty)$?
\shortans{$101$}
\loigiai{
Điều kiện: $\mathrm{e}^x\neq-m$.\\
Ta có $y'=\mathrm{e}^x\cdot\dfrac{-m-3}{\left(\mathrm{e}^x+m\right)^2}$.\\
Ta có $\heva{&\mathrm{e}^x>0\\&\mathrm{e}^x\in(1;+\infty)}$, $\forall x\in(0;+\infty)$ và khi $x\in(0;+\infty)$ thì $\mathrm{e}^x \in (1;+\infty)$.\\
Suy ra hàm số $y$ nghịch biến trên khoảng $(0;+\infty)$ khi và chỉ khi
$$\heva{&-m-3<0\\&-m\notin(1;+\infty)}\Leftrightarrow\heva{&m >-3\\&m\geq-1}\Leftrightarrow m\geq-1.$$
Vì $\heva{&m\in\mathbb{Z}\\&m\in(-100;100)}$ nên $m\in\left\{-1;0;1;\ldots;99\right\}$.\\
Suy ra có $101$ giá trị của $m$ thỏa mãn.
}
\end{ex}

\begin{ex}%[Mức độ 4]%[Dự án giảng new 4in1, Trần Quang Thạnh]%[2D1C1-3]
Có bao nhiêu giá trị nguyên của tham số $a$ trên đoạn $[-100; 100]$ để hàm số $f(x)=\dfrac{(a+1)\ln x-6}{\ln x-3a}$ nghịch biến trên khoảng $(1; \mathrm{e})$?
\shortans{$198$}
\loigiai{
Ta có $f'(x)=(\ln x)' \cdot \dfrac{-3a^2-3a+6}{(\ln x-3a)^2}=\dfrac{1}{x}\cdot \dfrac{-3a^2-3a+6}{(\ln x-3a)^2}.$\\
Với
$1<x<\mathrm{e}$ thì  $0<\ln x<1$.\\
Do đó hàm số $f(x)$ nghịch biến trên khoảng $(1; \mathrm{e})$ khi và chỉ khi
$$\heva{&-3a^2-3a+6<0\\&3a\notin(0; 1)}\Leftrightarrow\heva{&\hoac{&a <-2\\&a>1}\\&\hoac{&a\leq 0\\&a\geq\dfrac{1}{3}}}\Leftrightarrow\hoac{&a <-2\\&a>1.}$$
Vì $m\in [-100;100]$ nên $m\in \{-100;-99;\ldots;-3;1;\ldots;100\}$.\\
Vậy có $198$ giá trị nguyên của tham số $a$ để hàm số đã cho nghịch biến trên khoảng $(1; \mathrm{e})$.
}
\end{ex}

\begin{ex}%[BG12new-4in1, Trần Hoà]%[2D1C1-2]
\immini{
Cho hàm số $f(x)$ có đồ thị như hình vẽ bên. Xét $x\in \left(0;\dfrac{\pi}{2}\right)$, biết hàm số $f(\sin x)$ nghịch biến trên khoảng $(a;b)$. Khi đó giá trị lớn nhất của $|a-b|$ bằng bao  nhiêu? (kết quả làm tròn đến hàng phần phần trăm).
\shortans[]{$0{,}52$}
}{
\begin{tikzpicture}[>=stealth,font=\footnotesize,yscale=1.2, xscale=1.5]
\draw[->] (-0.5,0) -- (1.3,0) node[below] {$x$};
\draw[->] (0,-1.2) -- (0,1) node[left] {$y$};
\filldraw (0,0) node[above left=-0.1] {$O$} circle (1pt);
\draw[smooth,samples=100,domain=-0.4:0.9] plot(\x,{(16/3)*(\x)^3-4*(\x)^2}) node[right]{$f(x)$};
\filldraw (0.5,0) node[above] {$\frac{1}{2}$} circle (1pt);
\draw[dashed] (0.5,0) -- (0.5,-0.33);
\end{tikzpicture}
}
\loigiai{
Từ đồ thị của hàm số $f(x)$, suy ra
$$f'(x)<0\Leftrightarrow 0<x<\dfrac{1}{2};\qquad f'(x)>0\Leftrightarrow\hoac{&x<0\\ &x>\dfrac{1}{2}.}$$
Đặt $g(x)=f(\sin x)$, ta có $g'(x)=\cos x\cdot f'(\sin x)$. Xét trên khoảng $(0;\pi)$:
$$g'(x)<0\Leftrightarrow\hoac{&\cos x>0\ \text{và}\ f'(\sin x)<0\\ &\cos x<0\ \text{và}\ f'(\sin x)>0}\Leftrightarrow \hoac{&0<x< \dfrac{\pi}{2}\ \text{và}\ f'(\sin x)<0\qquad \qquad\ (1)\\ &\dfrac{\pi}{2}<x<\pi\ \text{và}\ f'(\sin x)>0.\qquad \qquad (2)}$$
Ta có
{\allowdisplaybreaks
\begin{align*}
&(1)\Leftrightarrow \heva{&0<x<\dfrac{\pi}{2}\\ &0<\sin x<\dfrac{1}{2}}\Leftrightarrow 0<x<\dfrac{\pi}{6}.\\
&(2)\Leftrightarrow \heva{&\dfrac{\pi}{2}<x<\pi\\ &\hoac{&\sin x<0\\ &\sin x>\dfrac{1}{2}}}\Leftrightarrow \dfrac{\pi}{2}<x<\dfrac{5\pi}{6}.
\end{align*}}
Vậy hàm số nghịch biến trên khoảng $\left(0;\dfrac{\pi}{6}\right)$. Suy ra $a=0$, $b=\dfrac{\pi}{6}$. Vậy $a+b\approx 0{,}52$.
}
\end{ex}

\begin{ex}%[2D1C1-1]
Cho hàm số $y= f(x)$ có đạo hàm $f'(x)=(x-1)(x-2)$.
Biết hàm số $y = f(x-x^2)$ nghịch biến trên khoảng có dạng $\left(\dfrac{a}{b};+\infty\right)$ với $\dfrac{a}{b}$ là tối giản và $b>0$. Giá trị của biểu thức $a^2+b^2$ bằng bao nhiêu?
\shortans[]{$5$}
\loigiai{
Đặt $y = g(x) = f(x-x^2) \Rightarrow g'(x) = f'(x-x^2) \cdot (x-x^2)' = (1-2x)f'(x-x^2)$.\\
Khi đó
\allowdisplaybreaks
\begin{eqnarray*}
g'(x) = 0
&\Leftrightarrow& \hoac{&1-2x=0\\&f'(x-x^2) =0 }\\
&\Leftrightarrow&  \hoac{& 1-2x=0\\ & x-x^2 =1\\& x-x^2 =2}\\
&\Leftrightarrow&  x = \dfrac{1}{2}.
\end{eqnarray*}
Với $x< \dfrac{1}{2}$ thì $\heva{&1-2x>0\\& f'\left[ - \left( x- \dfrac{1}{2}\right)^2 + \dfrac{1}{4}  \right] >0  } $ nên $g'(x)>0$.\\
Với $x> \dfrac{1}{2}$ thì $\heva{&1-2x<0\\& f'\left[ - \left( x- \dfrac{1}{2}\right)^2 + \dfrac{1}{4}  \right] >0  } $ nên $g'(x)<0$.\\
Hay hàm số $g(x) = f(x-x^2)$ nghịch biến trên khoảng $\left( \dfrac{1}{2}; + \infty \right) $.\\
Suy ra $a=1$, $b=2$ nên $a^2+b^2=5$.
}
\end{ex}

\begin{ex}%[2D1C5-5]
\immini{Cho hàm số $y=f(x)$ có đạo hàm trên $\mathbb{R}$. Biết rằng hàm số $y=f'(x)$ có đồ thị như hình vẽ bên. Hỏi đồ thị hàm số $y=f(2x-3)$ cắt đường thẳng $y=-3x+2$ tại nhiều nhất bao nhiêu điểm?
}
{
\begin{tikzpicture}[>=stealth,line join=round,line cap=round,font=\scriptsize,scale=0.7]
\draw [->] (-2.5,0)--(2.5,0)node[below]{$ x $};
\draw [->] (0,-3.5)--(0,2)node[left]{$ y $};
\draw [fill=black] (0,0)node[below left]{$ O $}circle(1pt) (-1,0)node[below left]{$ -1 $}circle(1pt) (1,0)node[below]{$ 1 $}circle(1pt) (0,1)node[left]{$ 1 $}circle(1pt) (0,-1)node[right]{$ -1 $}circle(1pt) (0,-3)node[right]{$-3$} circle (1pt) ;
\draw [smooth,domain=-2.1:2.1] plot(\x,{-(\x)^3+3*(\x)-1});
\draw[dashed] (-1,0)|-(0,-3) (1,0)|-(0,1);
\clip (-2.5,-3.5) rectangle (2.5,2.2);
\end{tikzpicture}
}
\shortans{$4$}
\loigiai{
Phương trình hoành độ giao điểm của hai đồ thị hàm số $y=f(2x-3)$ và $y=-3x+2$
$$f(2x-3)=-3x+2\Leftrightarrow f(2x-3)+3x-2=0.$$
Đặt $g(x)=f(2x-3)+3x-2$, ta có $g'(x)=2f'(2x-3)+3=0\Leftrightarrow f'(2x-3)=-\dfrac{3}{2}$.\\
Dựa vào đồ thị, đường thẳng $y=-\dfrac{3}{2}$ cắt đồ thị $f'(x)$ tại ba điểm phân biệt nên phương trình $f'(2x-3)=-\dfrac{3}{2}$ cũng có ba nghiệm phân biệt, giả sử ba nghiệm đó lần lượt là $a$, $b$, $c$ với $a<b<c$.\\
Ta có bảng biến thiên
\begin{center}
\begin{tikzpicture}
\tkzTabInit[nocadre=false,lgt=1.2,espcl=2.5,deltacl=.6]
{$x$/1,$g'(x)$/0.6,$g(x)$/2}
{$-\infty$, $a$, $b$, $c$, $+\infty$}
\tkzTabLine{,+,0,-,0,+,0,-,}
\tkzTabVar{-/,+/,-/,+/,-/}
\end{tikzpicture}
\end{center}
Dựa vào bảng biến thiên, suy ra phương trình $g(x)=0$ có tối đa $4$ nghiệm hay đồ thị hàm số $y=f(2x-3)$ cắt đường thẳng $y=-3x+2$ tại nhiều nhất $4$ điểm.
}
\end{ex}


\Closesolutionfile{ans}

\newpage
\begin{center}
    \bfseries\faGg~\faGg~\faGg~BẢNG ĐÁP ÁN TRẮC NGHIỆM~\faGg~\faGg~\faGg
\end{center}
\inputansbox{8}{ans/ansBTchoice}
\inputansbox[3]{2}{ans/ansBTchoiceTF}
\inputansbox{6}{ans/ansBTshortans}
% \newpage

%Chương II. Vector trong KG
%%Bài 1. Vector trong khong gian
% \section{VECTƠ TRONG KHÔNG GIAN}
\subsection{LÝ THUYẾT CẦN NHỚ}
\subsubsection{Tổng của hai véc tơ}
\begin{enumerate}[\iconMT]
	\item \textbf{Định nghĩa:}
	 \immini{Trong không gian, cho hai véctơ $\vec{a}$ và $\vec{b}$. Lấy ba điểm $O$, $A$, $B$ sao cho $\vec{OA}=\vec{a}$, $\vec{AB}=\vec{b}$. Ta gọi $\vec{OB}$ là \textbf{tổng của hai véctơ} $\vec{a}$ và $\vec{b}$, ký hiệu $\vec{a}+\vec{b}$.\\
	 Phép lấy tổng của hai véctơ $\vec{a}$ và $\vec{b}$ được gọi là \textbf{phép cộng véctơ}.}
	 {
	 \begin{tikzpicture}[>=stealth,scale=.5,font=\footnotesize]
	 \foreach \x\y\t in {3/0.4/A1,4./4.4/A2,9.5/4.3/B1,14.5/1.3/B2,5/0/O}
	 \coordinate (\t) at (\x,\y);
	 \coordinate (A) at ($(A2)-(A1)+(O)$);
	 \coordinate (B) at ($(B2)-(B1)+(A)$);
	 \foreach \a/\b in {A1/A2,B1/B2,O/A,A/B,O/B}
	 {\draw[-{Stealth[length=2.5mm]}](\a)--(\b);}
	 \node at ($(A1)!1/2!(A2)$)[left=-2pt]{$\vec{a}$};
	 \node at ($(B1)!1/2!(B2)$)[above right=-2pt]{$\vec{b}$};
	 \node at ($(O)!1/2!(A)$)[left=-2pt]{$\vec{a}$};
	 \node at ($(A)!1/2!(B)$)[above right=-2pt]{$\vec{b}$};
	 \node at ($(O)!1/2!(B)$)[rotate=7,above=-2pt]{$\vec{a}+\vec{b}$};
	 \foreach \t/\g in {O/-140,A/90,B/0}
	 \draw[fill=black] (\t)circle(1.2pt) +(\g:12pt)node{$\t$};
	 \end{tikzpicture}}
	\item \textbf{Các quy tắc cần nhớ:}
	 \begin{listEX}[1]
	 \immini{\item [\ding{172}] Quy tắc ba điểm: Với ba điểm $A$, $B$, $C$, ta có
	 \fbox{$\vec{AB} + \vec{BC} = \vec{AC}$}
	 \item [\ding{173}] Quy tắc hình bình hành: Cho $ABCD$ là hình bình hành, ta có
	 \fbox{$\vec{AB} + \vec{AD} = \vec{AC}$}}{
	 \begin{tikzpicture}[scale=0.6, font=\footnotesize, line join=round, line cap=round]
	 \begin{scope}
	 \foreach \x\y\t in {0/0/A, -2/-2/B, 2.5/-1.5/C}
	 \coordinate (\t) at (\x,\y);
	 \foreach \a\b in {A/B, B/C,A/C}
	 \draw[-{Stealth[length=2mm]}] (\a)--(\b);
	 \foreach \t\g in {A/90, B/-100, C/-80}
	 \draw[fill=black] (\t)circle(0.6pt) +(\g:8pt)node{$\t$};
	 \end{scope}
	 \begin{scope}[xshift=5cm]
	 \foreach \x\y\t in {0/0/A, -1.5/-2/B, 2.5/-2/C,4/0/D}
	 \coordinate (\t) at (\x,\y);
	 \foreach \a\b in {A/B, A/D,A/C}
	 \draw[-{Stealth[length=2mm]}] (\a)--(\b);
	 \draw[dashed] (B)--(C)--(D);
	 \foreach \t\g in {A/90, B/-90, C/-45,D/50}
	 \draw[fill=black] (\t)circle(0.6pt) +(\g:8pt)node{$\t$};
	 \end{scope}
	 \end{tikzpicture}}
	 \immini{\item [\ding{174}] Quy tắc hình hộp:
	 Cho hình hộp $ABCD.A'B'C'D'$. Ta có
	 \fbox{$\vec{AB} + \vec{AD} + \vec{AA'} = \vec{AC'}$}
	 \begin{note}
	 Hệ thức tương tự: \quad $\vec{BA} + \vec{BC} + \vec{BB'} = \vec{BD'}$.
	 \end{note}
	 }{
	 \begin{tikzpicture}[scale=0.6, font=\footnotesize, line join=round, line cap=round]
	 \def\h{4}
	 \foreach \x\y\t in {0/0/A',-1/-1.1/B',2.6/-1.1/C'}
	 \coordinate (\t) at (\x,\y);
	 \coordinate (D') at ($(A')+(C')-(B')$);
	 \coordinate (A) at ($(A')+(0,2.5)$);
	 \coordinate (B) at ($(B')+(0,2.5)$);
	 \coordinate (C) at ($(C')+(0,2.5)$);
	 \coordinate (D) at ($(D')+(0,2.5)$);
	 \foreach \a\b in {A/B, A/D,A/C}
	 \draw[-{Stealth[length=2mm]}] (\a)--(\b);
	 \foreach \a\b in {A/A', A/C'}
	 \draw[-{Stealth[length=2mm]},dashed] (\a)--(\b);
	 \draw (D)--(C)--(B)--(B')--(C')--(D')--(D) (C')--(C);
	 \draw[dashed](B')--(A')--(D');
	 \foreach \t/\g in {A'/170,B'/-150,C'/-70,D'/0,A/100,B/170,C/-20,D/50}
	 \draw[fill=black] (\t) circle(1pt)
	 node[shift={(\g:7pt)}]{$\t$};
	 \end{tikzpicture}
	 }
	 \end{listEX}
	\item \textbf{Tính chất:}
	 \begin{itemize}
	 \item[\ding{172}] Tính chất giao hoán: $\vec{a}+\vec{b}=\vec{b}+\vec{a}$;
	 \item[\ding{173}] Tính chất kết hợp: $\left(\vec{a}+\vec{b}\right)+\vec{c}=\vec{a}+\left(\vec{b}+\vec{c}\right)$;
	 \item[\ding{174}] Với mọi véctơ $\vec{a}$, ta luôn có: $\vec{a}+\vec{0}=\vec{0}+\vec{a}=\vec{a}$.
	 \item[\ding{175}] Tổng của ba véctơ $\vec{a}$, $\vec{b}$, $\vec{c}$:\quad $\vec{a}+\vec{b}+\vec{c}=\left(\vec{a}+\vec{b}\right)+\vec{c}.$
	 \end{itemize}
\end{enumerate}
\subsubsection{Hiệu của hai véc tơ}
\begin{enumerate}[\iconMT]
	\item \textbf{Véctơ đối:}
	 \begin{listEX}[1]
	 \item [\ding{172}] Vectơ đối của $\vec{a}$ kí hiệu là $-\vec{a}$.
	 \item [\ding{173}] Vectơ đối của $\vec{AB}$ là $\vec{BA}$: $-\vec{AB}=\vec{BA}$.
	 \item [\ding{174}] Vectơ $\vec{0}$ được coi là vectơ đối của chính nó.
	 \end{listEX}
	 \immini{
	\item \textbf{Định nghĩa hiệu của hai véctơ:} Trong không gian, cho hai véctơ $\vec{a}$, $\vec{b}$. Ta gọi $\vec{a}+\left(-\vec{b}\right)$ là \textbf{hiệu của hai véctơ} $\vec{a}$ và $\vec{b}$, ký hiệu $\vec{a}-\vec{b}$.\\
	 Phép lấy hiệu của hai véctơ được gọi là \textbf{phép trừ véctơ}.
	\item \textbf{Các quy tắc cần nhớ:}
	 \begin{listEX}[1]
	 \item [\ding{172}] Với ba điểm $A$, $B$, $C$ ta có $\vec{AB}-\vec{AC}=\vec{CB}$.
	 \item [\ding{173}] Hai véc tơ $\vec{a}$ và $\vec{b}$ đối nhau thì $\vec{a}+\vec{b}=\vec{0}$.
	 \end{listEX}
	 }{
	 \begin{tikzpicture}[scale=1, font=\footnotesize, line join=round, line cap=round]
	 \foreach \x\y\t in {0/0/O,1.5/1/A,3.1/-0.4/B,-0.3/1/a1,1.1/1.6/b1}
	 \coordinate (\t) at (\x,\y);
	 \coordinate (a2) at ($(a1)+(A)$);
	 \coordinate (b2) at ($(b1)+(B)$);
	 \foreach \a\b in {a1/a2, b1/b2,O/A,O/B,B/A}
	 \draw[-{Stealth[length=2mm]}] (\a)--(\b);
	 \path (a1)--(a2)node[pos=0.5,above left]{$\vec{a}$}
	 (b1)--(b2)node[pos=0.5,above]{$\vec{b}$}
	 (O)--(A)node[pos=0.5,above left]{$\vec{a}$}
	 (O)--(B)node[pos=0.5,below]{$\vec{b}$}
	 (B)--(A)node[pos=0.7,right=3pt]{$\vec{a}-\vec{b}$};
	 \foreach \t\g in {A/90, O/180, B/0}
	 \draw[fill=black] (\t)circle(0.2pt) +(\g:5pt)node{$\t$};
	 \end{tikzpicture}}
\end{enumerate}
\subsubsection{Tích của một số với một véc-tơ}
\begin{enumerate}[\iconMT]
	\item \textbf{Định nghĩa:} Cho số thực $k\ne 0$ và vectơ $\vec{a} \ne \vec{0}$. Tích của một số $k$ với vectơ $\vec{a}$ là một vectơ, kí hiệu là $k\vec{a}$, được xác định như sau:
	 \begin{itemize}
	 \item Cùng hướng với vectơ $\vec{a}$ nếu $k>0$, ngược hướng với vectơ $\vec{a}$ nếu $k<0$.
	 \item Có độ dài bằng $|k| \cdot |\vec{a}|$.
	 \end{itemize}
	 \begin{note}
	 $0\cdot \vec{a}=\vec{0}$ và $k\cdot \vec{0}=\vec{0}$.
	 \end{note}
	 % \immini{\textbf{Ví dụ:} Theo hình vẽ bên, thì $\vec{b}=3\vec{a}$; $\vec{c}=-2\vec{a}$; $\vec{c}=-\dfrac{2}{3}\vec{b}$.
	 % }{
	 % \begin{tikzpicture}[>=stealth,scale=0.5, line join=round, line cap=round]
	 % 	 \draw[line width=0.05pt,gray,dashed] (-0.7,-0.7) grid (8.7,3.7);
	 % 	 \draw[->,thick](1,2)--(2,3)node[above left]{$\vec{a}$};
	 % 	 \draw[->,thick](1,0)--(4,3)node[above right]{$\vec{b}$};
	 % 	 \draw[->,thick](7,3)--(5,1)node[below right]{$\vec{c}$};
	 % \end{tikzpicture}}
	\item \textbf{Hệ thức trung điểm, trọng tâm:}
	 \immini{
	 \begin{itemize}
	 \item [\ding{172}] $I$ là trung điểm của đoạn thẳng $AB$ thì
	 \begin{itemize}
	 \item [$\bullet$] $\vec{IA} + \vec{IB} = \vec 0$;
	 \item [$\bullet$] $\vec{IA}=-\vec{IB}$; $\vec{AI}=\dfrac{1}{2}\vec{AB}$;...
	 \end{itemize}
	 \item [\ding{173}] $G$ là trọng tâm của tam giác $ABC$ thì
	 \begin{listEX}[1]
	 \item [$\bullet$] $\vec{GA}+\vec{GB}+\vec{GC}=\vec{0}$;
	 \item [$\bullet$] $\vec{GA}=-\dfrac{2}{3}\vec{AK}$; $\vec{GA}=-2\vec{GK}$;...
	 \end{listEX}
	 \end{itemize}}{
	 \begin{tikzpicture}[scale=0.8, font=\footnotesize, line join=round, line cap=round]
	 \begin{scope}
	 \foreach \x\y\t in {-2/-2/A, 0/0/B}
	 \coordinate (\t) at (\x,\y);
	 \coordinate (I) at ($(A)!0.5!(B)$);
	 \foreach \a\b in {A/B}
	 \draw[] (\a)--(\b);
	 \foreach \t\g in {A/-90, B/40,I/1200}
	 \draw[fill=black] (\t)circle(0.6pt) +(\g:8pt)node{$\t$};
	 \end{scope}
	 \begin{scope}[xshift=4cm]
	 \foreach \x\y\t in {0/0/A, -2/-2/B, 2.5/-2/C}
	 \coordinate (\t) at (\x,\y);
	 \coordinate (M) at ($(A)!0.5!(B)$);
	 \coordinate (N) at ($(A)!0.5!(C)$);
	 \coordinate (K) at ($(C)!0.5!(B)$);
	 \coordinate (G) at ($(A)!2/3!(K)$);
	 \foreach \a\b in {A/B, B/C, A/C, A/K, M/C, B/N}
	 \draw[] (\a)--(\b);
	 \foreach \t\g in {A/90, B/-100, C/-80, M/120, N/40, K/-90,G/60}
	 \draw[fill=black] (\t)circle(0.8pt) +(\g:10pt)node{$\t$};
	 \end{scope}
	 \end{tikzpicture}}
	\item \textbf{Nhận xét:}
	 \begin{itemize}
	 \item[\ding{172}] Với hai véctơ $\vec{a}$ và $\vec{b}$ bất kỳ, với mọi số $h$ và $k$, ta luôn có
	 \begin{enumEX}[$\bullet$]{3}
	 \item $k\left(\vec{a}+\vec{b}\right)=k\vec{a}+k\vec{b}$;
	 \item $\left(h+k\right)\vec{a}=h\vec{a}+k\vec{a}$;
	 \item $h\left(k\vec{a}\right)=\left(hk\right)\vec{a}$;
	 \item $1\cdot \vec{a}=\vec{a}$;
	 \item $\left(-1\right)\cdot\vec{a}=-\vec{a}$;
	 \item $k\vec{a}=\vec{0} \Leftrightarrow \hoac{&\vec{a}=\vec{0}\\& k=0}$.
	 \end{enumEX}
	 \item[\ding{173}] Hai véctơ $\vec{a}$ và $\vec{b}$ ($\vec{b}$ khác $\vec{0}$) cùng phương khi và chỉ khi có số $k$ sao cho $\vec{a}=k\vec{b}$.
	 \item[\ding{174}] Ba điểm phân biệt $A$, $B$, $C$ thẳng hàng khi và chỉ khi có số $k \neq 0$ để $\vec{AB}=k\vec{AC}$.
	 \end{itemize}
\end{enumerate}
\subsubsection{Tích vô hướng của hai véc-tơ}
\begin{enumerate}[\iconMT]
	\item \textbf{Góc giữa hai véctơ:}
	 \immini{
	 Trong không gian, cho $\vec{u}$ và $\vec{v}$ là hai véctơ khác $\vec{0}$. Lấy một điểm $A$ bất kỳ, gọi $B$ và $C$ là hai điểm sao cho $\vec{AB}=\vec{u}$, $\vec{AC}=\vec{v}$. Khi đó, ta gọi $\widehat{BAC}$ là góc giữa hai véctơ $\vec{u}$ và $\vec{v}$, ký hiệu $\left(\vec{u}, \vec{v}\right)$.
	 \begin{note}
	 $0^{\circ} \leq \left(\vec{u},\vec{v}\right) \leq 180^{\circ}$.
	 \end{note}
	 \begin{note}
	 \begin{itemize}
	 \item [$\bullet$] Nếu $\vec{u}$ cùng hướng với $\vec{v}$ thì $\left(\vec{u}, \vec{v}\right)=0^\circ$;
	 \item [$\bullet$] Nếu $\vec{u}$ ngược hướng với $\vec{v}$ thì $\left(\vec{u}, \vec{v}\right)=180^\circ$;
	 \item [$\bullet$] Nếu $\vec{u}$ vuông góc với $\vec{v}$ thì $\left(\vec{u}, \vec{v}\right)=90^\circ$.
	 \end{itemize}
	 \end{note}
	 }{\vspace{-0.5cm}
	 \begin{tikzpicture}[scale=0.8, font=\footnotesize, line join=round, line cap=round]
	 \foreach \x\y\t in {0/0/A,2/0.8/B,3.2/-1./C,-1/1/u1,-0.5/-1.5/v1}
	 \coordinate (\t) at (\x,\y);
	 \coordinate (u2) at ($(u1)+(B)$);
	 \coordinate (v2) at ($(v1)+(C)$);
	 \draw (-1.5,-1.2)--(3.5,-1.2)--(4.5,1)--(-0.5,1)--cycle;
	 \draw[dashed] (A)--(u1) (B)--(u2) (A)--(v1) (C)--(v2);
	 \foreach \a\b in {A/B, A/C,u1/u2,v1/v2}
	 \draw[-{Stealth[length=2mm]}] (\a)--(\b);
	 \foreach \t\g in {A/-170, B/0, C/50}
	 \draw[fill=black] (\t)circle(0.6pt) +(\g:8pt)node{$\t$};
	 \path (A) pic[draw,angle radius=9]{angle=C--A--B};
	 \path
	 (u1)--(u2)node[pos=0.5,above]{$\vec{u}$}
	 (v1)--(v2)node[pos=0.5,above]{$\vec{v}$};
	 \end{tikzpicture}}
	\item \textbf{Định nghĩa tích vô hướng của hai véc tơ:}
	 Trong không gian, cho hai véctơ $\vec{u}$ và $\vec{v}$ khác $\vec{0}$.\\
	 Tích vô hướng của hai véctơ $\vec{u}$ và $\vec{v}$ là một số, kí hiệu $\vec{u} \cdot \vec{v}$, được xác định bởi công thức
	 \fbox{$\vec{u} \cdot \vec{v}=|\vec{u}| \cdot |\vec{v}| \cdot \cos (\vec{u}, \vec{v})$}
	 \vspace{-0.6cm}
	 \begin{note}
	 \begin{itemize}
	 \item[\ding{172}] Trong trường hợp $\vec{u}=0$ hoặc $\vec{v}=0$, ta quy ước $\vec{u} \cdot \vec{v}=0$.
	 \item[\ding{173}] $\vec{u} \cdot \vec{u}=\vec{u}^2=|\vec{u}|^2$; \quad $\vec{u}^2 \geqslant 0$. $ \vec{u}^2 = 0 \Leftrightarrow \vec{u}=\vec{0}$.
	 \item[\ding{174}] Với hai véctơ $\vec{u}$, $\vec{v}$ khác $\vec{0}$, ta có $\cos (\vec{u},\vec{v}) = \dfrac{\vec{u} \cdot \vec{v}}{|\vec{u}| \cdot |\vec{v}|}$
	 \item[\ding{175}] Với hai véctơ $\vec{u}$, $\vec{v}$ khác $\vec{0}$, ta có $\vec{u} \perp \vec{v} \Leftrightarrow \vec{u} \cdot \vec{v}= \vec{0}$.
	 \end{itemize}
	 \end{note}
	\item \textbf{TÍnh chất:} Với ba véctơ $\vec{a}$, $\vec{b}$, $\vec{c}$ và số thực $k$, ta có:
	 \begin{enumEX}[$\bullet$]{3}
	 \item $\vec{a} \cdot \vec{b}= \vec{b} \cdot \vec{a}$;
	 \item $\vec{a} \cdot \left( {\vec{b} + \vec{c}} \right) = \vec{a} \cdot \vec{b} + \vec{a} \cdot \vec{c}$;
	 \item $(k\vec{a}) \cdot \vec{b}= k(\vec{a} \cdot \vec{b}) = \vec{a} \cdot (k\vec{b})$.
	 \end{enumEX}
\end{enumerate}
\subsection{PHÂN LOẠI VÀ PHƯƠNG PHÁP GIẢI TOÁN}
\begin{dang}{Xác định véc-tơ, chứng minh đẳng thức véc tơ,độ dài véc tơ}
\end{dang}
\boxmini{BÀI TẬP TỰ LUẬN}
\setcounter{vd}{0}
\begin{vd}
	\immini{Cho hình hộp $ABCD.A'B'C'D'$. Hãy xác định các véc-tơ (khác $\vec{0}$) có điểm đầu, điểm cuối là các đỉnh của hình hộp $ABCD.A'B'C'D'$ thỏa
	\begin{tasks}(2)
	\task cùng phương với $\vec{AB}$;
	\task cùng phương $\vec{AA'}$;
	\task bằng với $\vec{AD}$;
	\task bằng với $\vec{A'B}$;
	\task đối với $\vec{CD'}$;
	\task đối với $\vec{B'C}$.
	\end{tasks}}{
	\begin{tikzpicture}[scale=0.7, font=\footnotesize,>=stealth]
	%Gán số liệu.
	\def\canhAD{3};\def\canhBA{2};\def\gocBAD{-130};\def\h{3};\def\xdinhA'{-0.5};
	%Gán tọa độ.
	\coordinate (A) at (0,0);
	\coordinate (B) at ($(A)+(\gocBAD:\canhBA)$);
	\coordinate (C) at ($(B)+(0:\canhAD)$);
	\coordinate (D) at ($(A)+(0:\canhAD)$);
	\coordinate (A') at ($(A)+(\xdinhA',\h)$);
	\coordinate (B') at ($(B)+(\xdinhA',\h)$);
	\coordinate (C') at ($(C)+(\xdinhA',\h)$);
	\coordinate (D') at ($(D)+(\xdinhA',\h)$);
	%Vẽ khối lẳng trụ ABCD.A'B'C'D'.
	\draw (A')--(B')--(B)--(C)--(C')--(D')--cycle (B')--(C') (D')--(D)--(C);
	\draw[dashed] (A)--(D) ;
	\draw[->,dashed] (A)--(C');
	\draw[->,dashed] (A)--(A');
	\draw[->, dashed] (A)--(D);
	\draw[->, dashed] (A)--(B);
	\draw[->, dashed] (A)--(C);
	%Gán nhãn.
	\foreach \x/\y in {A/180, B/180, C/0, D/0, A'/180, B'/180, C'/0, D'/0}{\fill (\x) circle(1pt) ($(\x)+(\y:0.3cm)$) node{$\x$};}
	\end{tikzpicture}}
\end{vd}
\dongcham{3}
\begin{vd}
	Cho hình chóp $S . A B C D$ có đáy $A B C D$ là hình bình hành. Gọi $M$, $N$, $O$ lần lượt là trung điểm của $A B, C D$ và $AC$. Chứng minh rằng
	\begin{listEX}[3]
	\item $\vec{B N}$ và $\vec{D M}$ đối nhau;
	\item $\vec{SA}+\vec{SB}+\vec{SC}+\vec{SD}=4\vec{SO}$;
	\item $\vec{S D}-\vec{B N}-\vec{C M}=\vec{S C}$.
	\end{listEX}
	\loigiai{
	\immini{\vspace*{-3mm}
	\begin{enumerate}
	\item Tứ giác $A B C D$ là hình bình hành nên $A B=C D$ và $A B\parallel C D$, suy ra $B M=DN$ và $B M \parallel D N$.\\
	 Do đó $BMDN$ là hình bình hành.\\
	 Hai véc-tơ $\vec{BN}$ và $\vec{DN}$ có cùng độ dài và ngược hướng nên chúng là hai véc-tơ đối nhau.
	\item Ta có $\vec{SA}+\vec{SC}=2\vec{SO}$; $\vec{SB}+\vec{SD}=2\vec{SO}$. Suy ra
	 $$\vec{SA}+\vec{SB}+\vec{SC}+\vec{SD}=4\vec{SO}.$$
	\item Từ câu a, ta có $\vec{BN}=-\vec{DM}$.\\
	 Suy ra $\vec{S D}-\vec{B N}-\vec{C M}=\vec{S D}+\vec{DM}-\vec{CM}=\vec{SM}+\vec{MC}=\vec{S C}$.
	\end{enumerate}
	}{
	\begin{tikzpicture}[line join=round, line cap = round, >=stealth, scale=.9,font=\footnotesize]
	\def\a{4}
	\path 	(0:0) coordinate (A)
	++(0:\a) coordinate (D)
	++(-130:\a/2) coordinate (C)
	($(A)+(C)-(D)$) coordinate (B)
	($(A)+(80:0.7*\a)$) coordinate (S)
	(intersection of A--C and B--D) coordinate (O)
	($(A)!0.5!(B)$) coordinate (M)
	($(C)!0.5!(D)$) coordinate (N)
	($(A)!0.5!(C)$) coordinate (O)
	;%giao điểm O
	\draw[dashed] 	(B)--(A)--(D)	(A)--(S) (A)--(C) (B)--(D);
	\draw 	(B)-- (C)--(D)
	(B)--(S)	(C)--(S)	(D)--(S);
	\foreach \x/\g in {A/135,B/-135,C/-45,D/45,S/90,M/-50,N/-30,O/-90}
	\fill[black] 	(\x) circle (1pt)
	($(\g:3mm)+(\x)$) node {$\x$};
	\draw [dashed] (D)--(M) (B)--(N);
	\end{tikzpicture}}
	}
\end{vd}
\dongcham{16}
\begin{vd}
	Cho hình lập phương $A B C D . A' B' C' D'$ cạnh bằng $a$. Gọi $G$ là trọng tâm tam giác $AB'D'$.
	\begin{tasks}(2)
	\task Tìm vectơ: $\vec{C C'}+\vec{B A}$; \quad $\vec{C C'}+\vec{B A}+\vec{D' A'}$.
	\task Chứng minh: $\vec{B C}+\vec{D C}+\vec{A A'}=\vec{A C'}$.
	\task Chứng minh: $\vec{B'B} + \vec{AD} + \vec{CD} = \vec{B'D}$.
	\task Chứng minh: $\vec{BB'} - \vec{C'B'} - \vec{D'C'} = \vec{BD'}$.
	\task Chứng minh: $\vec{A'C} = 3\vec{A'G}$.
	\task Tính độ dài véc tơ $\vec{u}= \vec{AB}+\vec{A'D'}+\vec{AA'}$.
	\end{tasks}
	\loigiai{
	\begin{enumerate}[a)]
	\immini{
	\item Vì $A B C D . A' B' C' D'$ là hình hộp nên $\vec{B A}=\vec{C D}$ và $\vec{D' A'}=\vec{C B}$.\\
	 Suy ra $\vec{C C'}+\vec{B A}+\vec{D' A'}=\vec{C C'}+\vec{C D}+\vec{C B}=\vec{C A'}$.
	\item Vì tứ giác $A B C D$ là hình bình hành nên $\vec{B C}=\vec{A D}$ và $\vec{D C}=\vec{A B}$. Áp dụng quy tắc hình hộp suy ra $$\vec{B C}+\vec{D C}+\vec{A A'}=\vec{A D}+\vec{A B}+\vec{A A'}=\vec{A C'}$$
	\item Ta có $\vec{AD} = \vec{B'C'}$, $\vec{CD} = \vec{B'A'}$. Do đó
	 $$\vec{B'B} + \vec{AD} + \vec{CD} = \vec{B'B} + \vec{B'C'} + \vec{B'A'} = \vec{B'D}.$$
	\item Ta có \begin{eqnarray*}
	 \vec{BB'} - \vec{C'B'} - \vec{D'C'} &=& \vec{BB'} - \left( \vec{D'C'} + \vec{C'B'} \right)
	 = \vec{BB'} - \vec{D'B'} \\
	 &=& \vec{BB'} + \left( - \vec{D'B'}\right)
	 = \vec{BB'} + \vec{B'D'}= \vec{BD'}.
	 \end{eqnarray*}
	 }{
	 \begin{tikzpicture}[scale=0.8, font=\footnotesize, line join=round, line cap=round]
	 \def\h{4}
	 \foreach \x\y\t in {0/0/A,-1/-1.1/B,2.6/-1.1/C}
	 \coordinate (\t) at (\x,\y);
	 \coordinate (D) at ($(A)+(C)-(B)$);
	 \coordinate (A') at ($(A)+(0,3.2)$);
	 \coordinate (B') at ($(B)+(0,3.2)$);
	 \coordinate (C') at ($(C)+(0,3.2)$);
	 \coordinate (D') at ($(D)+(0,3.2)$);
	 \draw (B')--(A')--(D')--(C')--(B')--(B)--(C)--(D)--(D') (C')--(C);
	 \draw[dashed](B)--(A)--(D) (A)--(A');
	 \foreach \t/\g in {A/170,B/-150,C/-70,D/0,A'/100,B'/170,C'/-20,D'/50}
	 \draw[fill=black] (\t) circle(1pt)
	 node[shift={(\g:7pt)}]{$\t$};
	 \end{tikzpicture}
	 }
	\item Do $G$ là trọng tâm tam giác $AB'D'$ nên $\vec{GA} + \vec{GB'} + \vec{GD'} = \vec{0}$. Khi đó, theo quy tắc hình hộp ta có
	 \begin{eqnarray*}
	 & \vec{A'C} & = \vec{A'A} + \vec{A'B'} + \vec{A'D'}\\
	 & & = \vec{A'G} + \vec{GA} + \vec{A'G} + \vec{GB'} + \vec{A'G} + \vec{GD'}\\
	 & & = 3\vec{A'G}.
	 \end{eqnarray*}
	\item Ta có $\vec{u}= \vec{AB}+\vec{A'D'}+\vec{AA'}=\vec{AB}+\vec{AD}+\vec{AA'}=\vec{AC'}$. Suy ra
	 $\big|\vec{u}\big|=AC'=a\sqrt{3}.$
	\end{enumerate}
	}
\end{vd}
\dongcham{34}
\begin{vd}%[2H2H1-4]
	\immini{
	Ba lực $\vec{F_1}$, $\vec{F_2}$, $\vec{F_3}$ cùng tác động vào một vật có phương đôi một vuông góc nhau và có độ lớn lần lượt là $2 \mathrm{\,N}$, $3 \mathrm{\,N}$, $4 \mathrm{\,N}$.
	\begin{tasks}
	\task Tính độ lớn hợp lực của $\vec{F_2}$, $\vec{F_3}$.
	\task Tính độ lớn hợp lực của ba lực đã cho.
	\end{tasks}}
	{
	\begin{tikzpicture}[scale=1.3, font=\footnotesize, line join=round, line cap=round]
	\foreach \x\y\t in {0/0/O,0/1/a,1.3/0/b,-1.2/-1/c}
	\coordinate (\t) at (\x,\y);
	\foreach \a\b in {O/a,O/b,O/c}
	\draw[-{Stealth[length=2mm]}] (\a)--(\b);
	\path (O)--(a) node[pos=0.5,left]{$\vec{F_1}$}
	(O)--(b) node[pos=0.5,above]{$\vec{F_2}$}
	(O)--(c) node[pos=0.5,above left]{$\vec{F_3}$};
	\path
	pic[draw,angle radius=4]{right angle=a--O--b}
	pic[draw,angle radius=4]{right angle=c--O--b}
	pic[draw,angle radius=4]{right angle=c--O--a};
	\end{tikzpicture}}
	\loigiai{
	\immini{
	\begin{enumerate}[a)]
	\item Gọi $O$ là vị trí trên vật mà ba lực cùng tác động vào. Gọi$A$, $B$, $C$ là các điểm sao cho $\vec{F_1}=\vec{OA}$, $\vec{F_2}=\vec{OB}$, $\vec{F_3}=\vec{OC}$. Khi đó
	 $$\left|\vec{F_2}+\vec{F_3}\right|=OE=\sqrt{3^2+4^2}=5 \text{N}.$$
	\item Dựng các hình chữ nhật $OBEC$ và $OEFA$ thì ta có
	 $$\heva{&\vec{OB}+\vec{OC}=\vec{OE}\\&\vec{OA}+\vec{OE}=\vec{OF}.}$$
	 Do đó $\vec{F_1}+\vec{F_2}+\vec{F_3}=\vec{OA}+\vec{OB}+\vec{OC}=\vec{OA}+\vec{OE}=\vec{OF}.$\\
	 Vậy độ lớn hợp lực của $F_1$, $\vec{F_2}$ và $\vec{F_3}$ là
	 $$\begin{aligned}
	 \left|\vec{F_1}+\vec{F_2}+\vec{F_3}\right|=OF
	 & =\sqrt{OA^2+OE^2} \\
	 & =\sqrt{OA^2+OB^2+OC^2} \\
	 & =\sqrt{2^2+3^2+4^2}=\sqrt{29} \mathrm{\,N}.
	 \end{aligned}$$
	\end{enumerate}
	}
	{
	\begin{tikzpicture}[scale=1.8, font=\footnotesize, line join=round, line cap=round]
	\foreach \x\y\t in {0/0/O,0/1/A,1.3/0/B,-0.9/-1.2/C}
	\coordinate (\t) at (\x,\y);
	\coordinate (E) at ($(B)+(C)$);
	\coordinate (F) at ($(A)+(E)$);
	\foreach \a\b in {O/A,O/B,O/C,O/F,O/E}
	\draw[-{Stealth[length=2mm]}] (\a)--(\b);
	\path
	(O)--(A) node[pos=0.5,left]{$\vec{F_1}$}
	(O)--(B) node[pos=0.5,above]{$\vec{F_2}$}
	(O)--(C) node[pos=0.5,above left]{$\vec{F_3}$};
	\path
	pic[draw,angle radius=4]{right angle=A--O--B}
	pic[draw,angle radius=4]{right angle=C--O--B}
	pic[draw,angle radius=4]{right angle=C--O--A};
	\foreach \t\g in {A/90, B/0, C/180,E/-80,F/0,O/180}
	\draw[fill=black] (\t)circle(0.1pt) +(\g:4pt)node{$\t$};
	\draw[dashed] (C)--(E)--(B) (A)--(F)--(E);
	\end{tikzpicture}}
	}
\end{vd}
\dongcham{23}
\boxmini{BÀI TẬP TRẮC NGHIỆM}
\textbf{PHẦN I.} \textit{Câu trắc nghiệm nhiều phương án lựa chọn. Mỗi câu hỏi học sinh chỉ chọn một phương án.}\\
\setcounter{ex}{0}
\Opensolutionfile{ans}[ans/2H2-B1-d1-1]
%%==========Câu 1
\begin{ex}%[1H3B1-1]
	\immini{Cho hình hộp $ABCD.EFGH$. Các véc-tơ có điểm đầu và điểm cuối là các đỉnh của hình hộp và bằng véc-tơ $\vec{AB}$ là các véc-tơ nào sau đây?
	\choice
	{$\vec{CD}$, $\vec{HG}$, $\vec{EF}$}
	{\True $\vec{DC}$, $\vec{HG}$, $\vec{EF}$}
	{$\vec{DC}$, $\vec{HG}$, $\vec{FE}$}
	{$\vec{DC}$, $\vec{GH}$, $\vec{EF}$}}{
	\begin{tikzpicture}[scale=0.75, font=\footnotesize, line join=round, line cap=round]
	\foreach \x\y\t in {0/0/A,-0.8/-1.1/B,2.8/-1.1/C}
	\coordinate (\t) at (\x,\y);
	\coordinate (D) at ($(A)+(C)-(B)$);
	\coordinate (E) at ($(A)+(-0.5,2.5)$);
	\coordinate (F) at ($(B)+(E)-(A)$);
	\coordinate (G) at ($(C)+(E)-(A)$);
	\coordinate (H) at ($(D)+(E)-(A)$);
	\foreach \a\b in {A/B, A/D,A/E}
	\draw[dashed] (\a)--(\b);
	\foreach \a\b in {C/D,C/B,C/G}
	\draw[] (\a)--(\b);
	\draw (F)--(E)--(H)--(G)--(F)--(B) (D)--(H);
	\foreach \t/\g in {A/170,B/-130,C/-60,D/0,E/90,F/180,G/-20,H/70}
	\draw[fill=black] (\t) circle(1pt)
	node[shift={(\g:7pt)}]{$\t$};
	\end{tikzpicture}}
	\loigiai{
	Các véc-tơ bằng với véc-tơ $\vec{AB}$ là $\vec{DC}$, $\vec{HG}$, $\vec{EF}$
	}
\end{ex} \dongcham{2}
%%==========Câu 2
\begin{ex}%[1H3B1-2]
	\immini{Cho hình hộp $ABCD.A'B'C'D'.$ Trong các khẳng định sau, khẳng định nào \textbf{sai}?
	\choice
	{$\vec{AB}+\vec{B'D'}=\vec{AD}$}
	{$\vec{AB}+\vec{CD}=\vec{0}$}
	{$\vec{AC'}+\vec{A'C}=2\vec{AC}$}
	{\True $\vec{AC}-\vec{D'D}=\vec{0}$}}{
	\begin{tikzpicture}[scale=0.55, font=\footnotesize, line join=round, line cap=round]
	\foreach \x\y\t in {0/0/A,-2/-1.5/B,3.9/0/D,-0.5/3.5/A'}
	\coordinate (\t) at (\x,\y);
	\coordinate (C) at ($(B)+(D)-(A)$);
	\coordinate (B') at ($(A')+(B)-(A)$);
	\coordinate (C') at ($(B')+(C)-(B)$);
	\coordinate (D') at ($(C')+(D)-(C)$);
	\draw (A')--(B')--(B)--(C)--(C');
	\draw (A')--(D')--(D);
	\draw (D')--(C') (C)--(D);
	\draw (B')--(C') (D)--(D');
	\draw[dashed] (A)--(B) (A')--(A)--(D);
	\foreach \t/\g in {A/180,B/180,C/0,D/0,A'/180,B'/180,C'/0,D'/0} \draw (\t) node[shift={(\g:10pt)}]{$\t$};
	\end{tikzpicture}}
	\loigiai{
	\immini{
	\begin{itemize}
	\item [$\bullet$] $\vec{AB}+\vec{B'D'}=\vec{AB}+\vec{BD}=\vec{AD}$'
	\item [$\bullet$] $\vec{AB}$ và $\vec{CD}$ đối nhau nên $\vec{AB}+\vec{CD}=\vec{0}$.
	\item [$\bullet$] Theo quy tắc hình bình hành ta có\\ $\vec{AC'}+\vec{A'C}=\vec{AC}+\vec{AA'}+\vec{A'A}+\vec{A'C'}=2\cdot\vec{AC}.$
	\item [$\bullet$] $\vec{AC}-\vec{D'D}=\vec{AC}+\vec{CC'}=\vec{AC'}$
	\end{itemize}
	}
	{\begin{tikzpicture}[scale=0.55, font=\footnotesize, line join=round, line cap=round]
	\foreach \x\y\t in {0/0/A,-2/-1.5/B,3.9/0/D,-0.5/3.5/A'}
	\coordinate (\t) at (\x,\y);
	\coordinate (C) at ($(B)+(D)-(A)$);
	\coordinate (B') at ($(A')+(B)-(A)$);
	\coordinate (C') at ($(B')+(C)-(B)$);
	\coordinate (D') at ($(C')+(D)-(C)$);
	\draw (A')--(B')--(B)--(C)--(C');
	\draw (A')--(D')--(D);
	\draw (D')--(C') (C)--(D);
	\draw (B')--(C') (D)--(D');
	\draw[dashed] (A)--(B) (A')--(A)--(D) (A)--(C') (A')--(C);
	\foreach \t/\g in {A/180,B/180,C/0,D/0,A'/180,B'/180,C'/0,D'/0} \draw (\t) node[shift={(\g:10pt)}]{$\t$};
	\end{tikzpicture}}
	}
\end{ex} \dongcham{8}
%%==========Câu 3
\begin{ex}
	\immini{Cho hình lập phương $ ABCD.A'B'C'D'$ cạnh $ a$. Khẳng định nào sau đây là khẳng định \textbf{sai}?
	\choice
	{$\big|\vec{AC}\big|=a\sqrt{2}$}
	{$\big|\vec{AC'}\big|=a\sqrt{3}$}
	{$\vec{BD}+\vec{D'B'}=\vec{0}$}
	{\True $\vec{BA}+\vec{BC}+\vec{BB'}=\vec{BC'}$}
	}{
	\begin{tikzpicture}[scale=0.55, font=\footnotesize, line join=round, line cap=round]
	\foreach \x\y\t in {0/0/A,-2/-1.5/B,3.9/0/D,0/3.5/A'}
	\coordinate (\t) at (\x,\y);
	\coordinate (C) at ($(B)+(D)-(A)$);
	\coordinate (B') at ($(A')+(B)-(A)$);
	\coordinate (C') at ($(B')+(C)-(B)$);
	\coordinate (D') at ($(C')+(D)-(C)$);
	\draw (A')--(B')--(B)--(C)--(C');
	\draw (A')--(D')--(D);
	\draw (D')--(C') (C)--(D);
	\draw (B')--(C') (D)--(D');
	\draw[dashed] (A)--(B) (A')--(A)--(D) (C)--(A)--(C');
	\foreach \t/\g in {A/180,B/180,C/0,D/0,A'/180,B'/180,C'/0,D'/0} \draw (\t) node[shift={(\g:10pt)}]{$\t$};
	\end{tikzpicture}
	}
	\loigiai{
	}
\end{ex} \dongcham{8}
%%==========Câu 4
\begin{ex}%[1H3B1-3]
	\immini{Cho hình lập phương $ABCD.A'B'C'D'$. Gọi $O$ là tâm của hình lập phương. Khẳng định nào dưới đây là đúng?
	\choice
	{$\vec{AO}=\dfrac{1}{3}\left(\vec{AB}+\vec{AD}+\vec{AA'}\right)$}
	{\True $\vec{AO}=\dfrac{1}{2}\left(\vec{AB}+\vec{AD}+\vec{AA'}\right)$}
	{$\vec{AO}=\dfrac{1}{4}\left(\vec{AB}+\vec{AD}+\vec{AA'}\right)$}
	{$\vec{AO}=\dfrac{2}{3}\left(\vec{AB}+\vec{AD}+\vec{AA'}\right)$}}{
	\begin{tikzpicture}[scale=0.55, font=\footnotesize, line join=round, line cap=round]
	\foreach \x\y\t in {0/0/A,-2/-1.5/B,3.9/0/D,0/3.5/A'}
	\coordinate (\t) at (\x,\y);
	\coordinate (C) at ($(B)+(D)-(A)$);
	\coordinate (B') at ($(A')+(B)-(A)$);
	\coordinate (C') at ($(B')+(C)-(B)$);
	\coordinate (D') at ($(C')+(D)-(C)$);
	\coordinate (O) at ($(A)!0.5!(C')$);
	\draw (A')--(B')--(B)--(C)--(C');
	\draw (A')--(D')--(D);
	\draw (D')--(C') (C)--(D);
	\draw (B')--(C') (D)--(D');
	\draw[dashed] (A)--(B) (A')--(A)--(D) (A)--(C') (A')--(C);
	\foreach \t/\g in {A/180,B/180,C/0,D/0,A'/180,B'/180,C'/0,D'/0,O/-100} \draw (\t) node[shift={(\g:10pt)}]{$\t$};
	\end{tikzpicture}}
	\loigiai{\vspace{-0.5cm}
	\immini{
	Theo quy tắc hình hộp, ta có $\vec{AC'}=\vec{AB}+\vec{AD}+\vec{AA'}$. \\
	Mà $O$ là trung điểm của $AC'$\\
	nên $\vec{AO}=\dfrac{1}{2}\vec{AC'}=\dfrac{1}{2}\left(\vec{AB}+\vec{AD}+\vec{AA'}\right)$.}
	{\vspace{-0.5cm}
	\begin{tikzpicture}[scale=0.55, font=\footnotesize, line join=round, line cap=round]
	\foreach \x\y\t in {0/0/A,-2/-1.5/B,3.9/0/D,0/3.5/A'}
	\coordinate (\t) at (\x,\y);
	\coordinate (C) at ($(B)+(D)-(A)$);
	\coordinate (B') at ($(A')+(B)-(A)$);
	\coordinate (C') at ($(B')+(C)-(B)$);
	\coordinate (D') at ($(C')+(D)-(C)$);
	\coordinate (O) at ($(A)!0.5!(C')$);
	\draw (A')--(B')--(B)--(C)--(C');
	\draw (A')--(D')--(D);
	\draw (D')--(C') (C)--(D);
	\draw (B')--(C') (D)--(D');
	\draw[dashed] (A)--(B) (A')--(A)--(D) (A)--(C') (A)--(C);
	\foreach \t/\g in {A/180,B/180,C/0,D/0,A'/180,B'/180,C'/0,D'/0,O/-90} \draw (\t) node[shift={(\g:10pt)}]{$\t$};
	\end{tikzpicture}}}
\end{ex} \dongcham{8}
%%==========Câu 5
\begin{ex}%[1H3B1-2]%
	Cho hình lập phương $ ABCD.A'B'C'D'$ cạnh $ a$. Tính độ dài vectơ $\vec x=\vec{AB'}+\vec{AD'}$ theo $ a$.
	\choice
	{$\left|\vec x\right|=a\sqrt 2 $}
	{$\left|\vec x\right|=2a\sqrt 2 $}
	{$\left|\vec x\right|=2a\sqrt 6 $}
	{\True $\left|\vec x\right|=a\sqrt 6 $}
	\loigiai{
	\immini{Ta có $\vec x=\vec{AB'}+\vec{AD'}=2\vec{AI}$, với $ I$ là trung điểm của $ B'D'$. Khi đó $\left|\vec x\right|=2AI$.\\
	Do tam giác $ AB'D'$ đều cạnh $ a\sqrt 2 $ nên $ AI=\dfrac{a\sqrt 6}{2}$. \\
	Vậy $\left|\vec x\right|=a\sqrt 6 $.}
	{\begin{tikzpicture}[scale=1, font=\footnotesize, line join=round, line cap=round, >=stealth]
	\def\bc{4} % cạnh BC
	\def\ba{2} % cạnh BA
	\def\gocB{35} % góc B của đáy
	\coordinate[label=below left:$B$] (B) at (0,0);
	\coordinate[label=above left:$A$] (A) at (\gocB:\ba);
	\coordinate[label=below:$C$] (C) at (\bc,0);
	\coordinate[label=right:$D$] (D) at ($(C)-(B)+(A)$);
	\coordinate[label=above left:$A'$] (A') at ($(A)+(90:\bc)$);
	\coordinate[label=left:$B'$] (B') at ($(B)-(A)+(A')$);
	\coordinate[label=below right:$C'$] (C') at ($(C)-(A)+(A')$);
	\coordinate[label=right:$D'$] (D') at ($(D)-(A)+(A')$);
	\tkzDefMidPoint(B',D') \tkzGetPoint{I}
	\tkzLabelPoints[above](I);
	\draw (B')--(B)--(C)--(D)--(D')--(A')--(B')--(C')--(D') (C)--(C') (B')--(D');
	\draw[dashed] (A')--(A)--(D) (A)--(B) (A)--(B') (A)--(D') (A)--(I);
	\foreach \diem in {A,B,C,D,A',B',C',D',I}\fill (\diem)circle(1.5pt);
	\end{tikzpicture}}
	}
\end{ex} \dongcham{8}
%%==========Câu 6
\begin{ex}%[1H3B1-3]
	\immini{Hình lập phương $ABCD.A'B'C'D'$ cạnh $a$. Tính độ dài véctơ $\vec{x}=\vec{AA'}+\vec{AC'}$ theo~$a$.
	\haicot
	{$a\sqrt{2}$}
	{$\left(1+\sqrt{3}\right)a$}
	{\True $a\sqrt{6}$}
	{$\dfrac{a\sqrt{6}}{2}$}}{\hspace{1cm}
	\begin{tikzpicture}[scale=0.7, line join=round, line cap=round]
	\tikzset{label style/.style={font=\footnotesize}}
	\tkzDefPoints{0/0/A,-1.3/-1.1/B,2/-1.1/C}
	\coordinate (D) at ($(A)+(C)-(B)$);
	\coordinate (A') at ($(A)+(0,2.5)$);
	\tkzDefPointsBy[translation=from A to A'](B,C,D){B'}{C'}{D'}
	\tkzDrawPolygon(A',B',B,C,D,D')
	\tkzDrawSegments(B',C' C',D' C,C')
	\tkzDrawSegments[dashed](A,B A,D A,A')
	\tkzDrawPoints[fill=black,size=4](A,B,D,C,A',B',C',D')
	\tkzLabelPoints[above](A',D')
	\tkzLabelPoints[below](A,B,C)
	\tkzLabelPoints[left](B')
	\tkzLabelPoints[right](C',D)
	\end{tikzpicture}}
	\loigiai{
	\immini{
	Gọi $O'$ là tâm $A'B'C'D'\Rightarrow A'O'=\dfrac{a\sqrt{2}}{2}$.\\
	Ta có $\vec{AA'}+\vec{AC'}=2\vec{AO'}\Rightarrow \vert \vec{x} \vert =2\left| \vec{AO'} \right| =2AO'$.\\
	$\triangle AA'O'$ vuông tại $A'\Rightarrow AO'=\sqrt{AA'^2+A'O'^2}=\dfrac{a\sqrt{6}}{2}$.\\
	Vậy $\vert \vec{x} \vert =2AO'=a\sqrt{6}$.
	}{
	\begin{tikzpicture}[scale=.5, line join=round, line cap=round,>=stealth]
	\tkzDefPoints{0/0/B,3/2/A,10/2/D,7/0/C,0/7/B',3/9/A',7/7/C',10/9/D'}
	\tkzInterLL(A',C')(B',D')\tkzGetPoint{O'}
	\tkzDrawSegments[](A',B' B',C' C',D' D',A' B,C C,D B,B' C,C' D,D' A',C' B',D')
	\tkzDrawSegments[dashed](A,A' A,B A,D A,C' A,O')
	\tkzLabelPoints[above](A',D',O')
	\tkzLabelPoints[above left](A,B')
	\tkzLabelPoints[below left](B)
	\tkzLabelPoints[below right](C,C')
	\tkzLabelPoints[right](D)
	\tkzDrawPoints[fill=black](A,B,C,D,A',B',C',D',O')
	\end{tikzpicture}
	}
	}
\end{ex} \dongcham{8}
%%==========Câu 7
\begin{ex}%[Trần Toàn]%[1H3Y1-2]%
	\immini{Cho tứ diện $ABCD$. Mệnh đề nào dưới đây là mệnh đề đúng?
	\choice
	{$\vec {AB}-\vec {AD}=\vec {CD}+\vec {BC}$}
	{$\vec {AC}-\vec {AD}=\vec {BD}-\vec {BC}$}
	{$\vec {BC}+\vec {AB}=\vec {DA}-\vec {DC}$}
	{\True $\vec {AB}-\vec {AC}=\vec {DB}-\vec {DC}$}}{
	\begin{tikzpicture}[scale=0.8, font=\footnotesize,>=stealth]
	\path
	(0,0) coordinate (A)
	(5,0) coordinate (C)
	(1.2,-1.5) coordinate (B)
	($(B)!0.5!(C)$)coordinate (M)
	($(A)!2/3!(M)$)coordinate (G)
	($(G)+(0,3)$)coordinate (D)
	;
	\draw (D)--(A)--(B)--(D)--(C)--(B);
	\draw[dashed] (A)--(C);
	\foreach \x/\g in {A/180,B/-90,C/0,D/90}\draw[fill=black] (\x) circle (.04) +(\g:.4)node{\footnotesize$\x$};
	\end{tikzpicture}}
	\loigiai{
	Ta có $\vec{AB}-\vec{AC}=\vec{CB}=\vec{DB}-\vec{DC}$.
	}
\end{ex} \dongcham{6}
%%==========Câu 8
\begin{ex}%[1H3B1-2]%
	\immini{Cho tứ diện $ABCD$. Gọi $ G$ là trọng tâm tam giác $ABC$. Tìm $ k$ thỏa đẳng thức vectơ $\vec{DA}+\vec{DB}+\vec{DC}=k\cdot\vec{DG}$.
	\haicot
	{$ k=1$}
	{$ k=3$}
	{$ k=2$}
	{\True $ k=3$}}{
	\begin{tikzpicture}[scale=0.8, font=\footnotesize,>=stealth]
	\path
	(0,0) coordinate (A)
	(5,0) coordinate (C)
	(1.2,-1.5) coordinate (B)
	($(B)!0.5!(C)$)coordinate (M)
	($(A)!2/3!(M)$)coordinate (G)
	($(G)+(0,3)$)coordinate (D)
	;
	\draw (D)--(A)--(B)--(D)--(C)--(B);
	\draw[dashed] (A)--(C) (D)--(G);
	\foreach \x/\g in {A/180,B/-90,C/0,D/90,G/0}\draw[fill=black] (\x) circle (.04) +(\g:.4)node{\footnotesize$\x$};
	\end{tikzpicture}}
	\loigiai{
	\immini{
	$\vec{DA}+\vec{DB}+\vec{DC}=\vec{DG}+\vec{GA}+\vec{DG}+\vec{GB}+\vec{DG}+\vec{GC}=3\vec{DG}$.}
	{\begin{tikzpicture}[scale=1, font=\footnotesize,>=stealth]
	\path
	(0,0) coordinate (A)
	(4,0) coordinate (C)
	(1.5,-1.5) coordinate (B)
	($(B)!0.5!(C)$)coordinate (M)
	($(A)!2/3!(M)$)coordinate (G)
	($(G)+(0,3)$)coordinate (D)
	;
	\draw (D)--(A)--(B)--(D)--(C)--(B);
	\draw[dashed] (A)--(C) (D)--(G) (B)--(G)--(A) (G)--(C);
	\foreach \x/\g in {A/180,B/-90,C/0,D/90,G/0}\draw[fill=black] (\x) circle (.04) +(\g:.4)node{\footnotesize$\x$};
	\end{tikzpicture}
	}
	}
\end{ex} \dongcham{8}
%%==========Câu 9
\begin{ex}%[1H3B1-3]
	\immini{Cho hình lăng trụ $ABC.A'B'C'$. Gọi $G'$ là trọng tâm của tam giác $A'B'C'$. Đặt $\vec{a}=\vec{AA'}, \vec{b}=\vec{AB}, \vec{c}=\vec{AC}$. Véc-tơ $\vec{AG'}$ bằng
	\choice
	{$\dfrac{1}{3}\left(\vec{a}+3\vec{b}+\vec{c}\right)$}
	{\True $\dfrac{1}{3}\left(3\vec{a}+\vec{b}+\vec{c}\right)$}
	{$\dfrac{1}{3}\left(\vec{a}+\vec{b}+3\vec{c}\right)$}
	{$\dfrac{1}{3}\left(\vec{a}+\vec{b}+\vec{c}\right)$}}{
	\begin{tikzpicture}[scale=0.8, font=\footnotesize,>=stealth]
	\path
	(0,0) coordinate (A)
	(4,0) coordinate (C)
	(1.5,-1.5) coordinate (B)
	($(A)+(0.4,3)$)coordinate (A')
	($(B)+(0.4,3)$)coordinate (B')
	($(C)+(0.4,3)$)coordinate (C')
	($(B')!0.5!(C')$)coordinate (I)
	($(A')!2/3!(I)$)coordinate (G')
	;
	\draw (B)--(C)--(C')--(B')--(B)--(A)--(A')--(B') (I)--(A')--(C');
	\draw[dashed] (A)--(C);
	\foreach \x/\g in {A/180,B/-45,C/0,A'/180,B'/190,C'/0,G'/-90}\draw[fill=black] (\x) circle (.04) +(\g:.4)node{\footnotesize$\x$};
	\end{tikzpicture}}
	\loigiai{\vspace{-0.5cm}
	\immini{
	Gọi $I$ là trung điểm của $B'C'$. \\
	Vì $G'$ là trọng tâm của tam giác $A'B'C' \Rightarrow \vec{A'G'}=\dfrac{2}{3}\vec{A'I}$. \\
	$\begin{aligned}
	\text{Ta có} \vec{AG'} & =\vec{AA'}+\vec{A'G'}=\vec{AA'}+\dfrac{2}{3}\vec{A'I} \\
	 & =\vec{AA'}+\dfrac{1}{3}\left(\vec{A'B'}+\vec{A'C'}\right) \\
	 & =\vec{AA'}+\dfrac{1}{3}\left(\vec{AB}+\vec{AC}\right) \\
	 & =\dfrac{1}{3}\left(3\vec{AA'}+\vec{AB}+\vec{AC}\right)=\dfrac{1}{3}\left(3\vec{a}+\vec{b}+\vec{c}\right).
	\end{aligned}$}
	{\begin{tikzpicture}[scale=0.7, font=\footnotesize,>=stealth]
	\path
	(0,0) coordinate (A)
	(4,0) coordinate (C)
	(1.5,-1.5) coordinate (B)
	($(A)+(0.4,3)$)coordinate (A')
	($(B)+(0.4,3)$)coordinate (B')
	($(C)+(0.4,3)$)coordinate (C')
	($(B')!0.5!(C')$)coordinate (I)
	($(A')!2/3!(I)$)coordinate (G')
	;
	\draw (B)--(C)--(C')--(B')--(B)--(A)--(A')--(B') (I)--(A')--(C');
	\draw[dashed] (A)--(C);
	\foreach \x/\g in {A/180,B/-45,C/0,A'/180,B'/190,C'/0,G'/-90,I/-90}\draw[fill=black] (\x) circle (.04) +(\g:.4)node{\footnotesize$\x$};
	\end{tikzpicture}}}
\end{ex} \dongcham{12}
%%==========Câu 10
\begin{ex}%[1H3B1-2]
	\immini{Cho hình chóp $S.ABCD$ có đáy $ABCD$ là hình bình hành. Đặt $\vec{SA}=\vec{a}$, $\vec{SB}=\vec{b}$, $\vec{SC}=\vec{c}$, $\vec{SD}=\vec{d}$. Khẳng định nào dưới đây là đúng?
	\choice
	{\True $\vec{a}+\vec{c}=\vec{b}+\vec{d}$}
	{$\vec{a}+\vec{b}+\vec{c}+\vec{d}=\vec{0}$}
	{$\vec{a}+\vec{d}=\vec{b}+\vec{c}$}
	{$\vec{a}+\vec{b}=\vec{c}+\vec{d}$}}{
	\begin{tikzpicture}[scale=0.8, font=\footnotesize,>=stealth]
	\path
	(0,0) coordinate (A)
	(-1.6,-1.5) coordinate (B)
	(5,0) coordinate (D)
	($(B)+(D)-(A)$)coordinate (C)
	($(A)!1/2!(C)$)coordinate (O)
	($(A)+(0.4,3)$)coordinate (S)
	;
	\draw (C)--(D)--(S)--(C)--(B)--(S);
	\draw[dashed] (D)--(A)--(B) (S)--(A);
	\foreach \x/\g in {A/160,B/-90,C/-90,D/0,S/90}\draw[fill=black] (\x) circle (.04) +(\g:.4)node{\footnotesize$\x$};
	\end{tikzpicture}}
	\loigiai{
	\immini{
	Gọi $O$ là tâm hình bình hành $ABCD$. \\
	Vì $O$ là trung điểm của $AC$\\ \indent\qquad
	nên $\vec{SA}+\vec{SC}=2\vec{SO} \Leftrightarrow 2\vec{SO}=\vec{a}+\vec{c}$. \hfill (1) \\
	Và $O$ là trung điểm của $BD$\\\indent\qquad
	nên $\vec{SB}+\vec{SD}=2\vec{SO} \Leftrightarrow 2\vec{SO}=\vec{b}+\vec{d}$.\hfill (2)\\
	Từ $(1)$ và $(2)$, suy ra $\vec{a}+\vec{c}=\vec{b}+\vec{d}$.}
	{\begin{tikzpicture}[scale=0.5, font=\footnotesize,>=stealth]
	\path
	(0,0) coordinate (A)
	(-1.6,-1.5) coordinate (B)
	(5,0) coordinate (D)
	($(B)+(D)-(A)$)coordinate (C)
	($(A)!1/2!(C)$)coordinate (O)
	($(A)+(0.4,3)$)coordinate (S)
	;
	\draw (C)--(D)--(S)--(C)--(B)--(S);
	\draw[dashed] (D)--(A)--(B)--(D) (S)--(A)--(C) (O)--(S);
	\foreach \x/\g in {A/160,B/-90,C/-90,D/0,S/90}\draw[fill=black] (\x) circle (.04) +(\g:.4)node{\footnotesize$\x$};
	\end{tikzpicture}}}
\end{ex} \dongcham{15}
%%==========Câu 11
\begin{ex}
	\immini{
	Cho tứ diện $ABCD$. Các vectơ có điểm đầu là $A$ và điểm cuối là các đỉnh còn lại của hình tứ diện là
	\choice
	{$\vec{AB},\vec{CA},\vec{AD}$}
	{$\vec{BA},\vec{AC},\vec{AD}$}
	{$\vec{AB},\vec{AC},\vec{DA}$}
	{\True $\vec{AB},\vec{AC},\vec{AD}$}
	}{\begin{tikzpicture}[line join = round, line cap = round, thick, font = \small, scale = .6]
	\path
	(0:0) coordinate (B)
	+(0:5) coordinate (C)
	+(-70:3) coordinate (D)
	++(90:4) coordinate (A)
	;
	\draw[dashed]
	(B)--(C)
	;
	\draw
	(A)--(B)--(D)--(C)--cycle
	(A)--(D)
	;
	\foreach \x/\g in {B/180,C/0,D/-90,A/90}
	\fill (\x) circle (1.5pt)
	+(\g:3mm) node {$\x$};
	\end{tikzpicture}
	}
	\loigiai{
	}
\end{ex}
%%==========Câu 12
\begin{ex}
	\immini{
	Cho hình lăng trụ tam giác $ABC.A'B'C'$.Gọi $M$, $N$ lần lượt là trung điểm của $AB$, $AC$. Trong 4 vectơ $\vec{AB}$, $\vec{CB}$, $\vec{B'C'}$, $\vec{A'C'}$ vectơ nào cùng hướng với vectơ $\vec{MN}$
	\choice
	{$\vec{AB}$}
	{$\vec{CB}$}
	{\True $\vec{B'C'}$}
	{$\vec{A'C'}$}
	}{\begin{tikzpicture}[line join = round, line cap = round, thick, font = \small, scale = .7]
	\path
	(0:0) coordinate (A)
	+(0:4) coordinate (C)
	+(-50:2) coordinate (B)
	+(75:3.5) coordinate (A')
	($(A')+(B)-(A)$) coordinate (B')
	($(A')+(C)-(A)$) coordinate (C')
	($(A)!.5!(B)$) coordinate (M)
	($(A)!.5!(C)$) coordinate (N)
	;
	\draw[dashed]
	(A)--(C) (M)--(N)
	;
	\draw
	(A)--(B)--(C)--(C')--(A')--cycle
	(B')--(A') (B')--(B) (B')--(C')
	;
	\foreach \x/\g in {A/180,B/-90,C/0,A'/-180,B'/70,C'/0,M/-135,N/-45}
	\fill (\x) circle (1.5pt)
	+(\g:3mm) node {$\x$};
	\end{tikzpicture}
	}
	\loigiai{
	Vì $MN$ là đường trung bình của tam giác $ABC$ nên $MN$ song song với $BC$. Mà tứ giác $BCC'B'$ là hình bình hành. Do đó $MN$ song song với $B'C'$. Vậy hai vectơ $\vec{MN}$ và $\vec{B'C'}$ cùng hướng.
	}
\end{ex}
%%==========Câu 13
\begin{ex}
	\immini{
	Cho hình hộp $ABCD.A'B'C'D'$.Số các vectơ có điểm đầu, điểm cuối là các đỉnh của hình hộp và bằng vectơ $\vec{AB}$ là
	\choice
	{$1$}
	{$2$}
	{\True $3$}
	{$4$}
	}{\begin{tikzpicture}[line join = round, line cap = round, thick, font = \small, scale = .7]
	\path
	(0:0) coordinate (D')
	+(75:3.5) coordinate (D)
	+(0:3) coordinate (C')
	+(40:2) coordinate (A')
	($(C')+(D)-(D')$) coordinate (C)
	($(D)+(A')-(D')$) coordinate (A)
	($(C')+(A')-(D')$) coordinate (B')
	($(C)+(A)-(D)$) coordinate (B)
	;
	\draw[dashed]
	(A')--(A) (A')--(B') (A')--(D')
	;
	\draw
	(A)--(B)--(B')--(C')--(D')--(D)--cycle
	(C)--(B) (C)--(D) (C)--(C')
	;
	\foreach \x/\g in {D'/-90,C'/-90,D/180,A'/135,C/-45,A/90,B'/0,B/90}
	\fill (\x) circle (1.5pt)
	+(\g:3mm) node {$\x$};
	\end{tikzpicture}
	}
	\loigiai{
	$\vec{AB}=\vec{DC}=\vec{D'C'}=\vec{A'B'}$
	}
\end{ex}
%%==========Câu 14
\begin{ex}
	Cho hình hộp $ABCD.A'B'C'D'$. Trong các khẳng định dưới đây, đâu là khẳng định đúng?
	\choice
	{$\vec{AB}+\vec{AC}+\vec{AD}=\vec{AC'}$}
	{\True $\vec{AB}+\vec{AA'}+\vec{AD}=\vec{AC'}$}
	{$\vec{AB}+\vec{AA'}+\vec{AD}=\vec{AC}$}
	{$\vec{AB}+\vec{AA'}+\vec{AD}=\vec{0}$}
	\loigiai{
	Xét hình hộp $ABCD.A'B'C'D'$ ta có $\vec{AB}+\vec{AA'}+\vec{AD}=\vec{AC'}$
	}
\end{ex}
%%==========Câu 15
\begin{ex}
	Trong không gian cho tam giác $ABC$ có $G$ là trọng tâm và điểm $M$ nằm ngoài mặt phẳng $(ABC)$. Khẳng định nào sau đây là đúng?
	\choice
	{$\vec{MA}+\vec{MB}+\vec{MC}=\vec{0}$}
	{$\vec{GA}+\vec{GB}+\vec{GC}=0$}
	{$\vec{MA}+\vec{MB}+\vec{MC}=\vec{MG}$}
	{\True $\vec{MA}+\vec{MB}+\vec{MC}=3\vec{MG}$}
	\loigiai{
	Vì $G$ là trọng tâm tam giác $ABC$ nên $\vec{MA}+\vec{MB}+\vec{MC}=3\vec{MG}$
	}
\end{ex}
%%==========Câu 16
\begin{ex}
	Cho hình chóp đều $S.ABCD$ tất cả các cạnh bằng $2\sqrt{3}$. Tính độ dài vectơ $\vec{u}=\vec{SA}-\vec{SC}$.
	\choice
	{$\sqrt{3}$}
	{$\sqrt{2}$}
	{\True $2\sqrt{6}$}
	{$2\sqrt{2}$}
	\loigiai{
	Ta có: $|\vec{u}| = |\vec{SA}-\vec{SC}| = |\vec{CA}| = AB\sqrt{2} =2\sqrt{6}$.
	}
\end{ex}
%%==========Câu 17
\begin{ex}
	Cho tứ diện $ABCD$. Mệnh đề nào dưới đây là mệnh đề đúng?
	\choice
	{$\vec{BC}-\vec{BA}=\vec{DA}-\vec{DC}$}
	{$\vec{AC}-\vec{AD}=\vec{BD}-\vec{BC}$}
	{\True $\vec{AB}-\vec{AC}=\vec{DB}-\vec{DC}$}
	{$\vec{AB}-\vec{AD}=\vec{CD}-\vec{CB}$}
	\loigiai{
	Ta có: $\heva{& \vec{AB}-\vec{AC}=\vec{CB} \\& \vec{DB}-\vec{DC}=\vec{CB}}\Rightarrow \vec{AB}-\vec{AC}=\vec{DB}-\vec{DC}$.
	}
\end{ex}
%%==========Câu 18
\begin{ex}
	Cho hình lăng trụ $ABC.A'B'C'$, $M$ là trung điểm của $BB'$. Đặt $\vec{CA}=\vec{a}$, $\vec{CB}=\vec{b}$, $\vec{AA'}=\vec{c}$. Khẳng định nào sau đây đúng?
	\choice
	{$\vec{AM}=\vec{b}+\vec{c}-\dfrac{1}{2}\vec{a}$}
	{$\vec{AM}=\vec{a}-\vec{c}+\dfrac{1}{2}\vec{b}$}
	{$\vec{AM}=\vec{a}+\vec{c}-\dfrac{1}{2}\vec{b}$}
	{\True $\vec{AM}=\vec{b}-\vec{a}+\dfrac{1}{2}\vec{c}$}
	\loigiai{
	\immini{
	Ta có: $\vec{AM}=\vec{AB}+\vec{BM}=\vec{CB}-\vec{CA}+\dfrac{1}{2}\vec{BB'}=\vec{CB}-\vec{CA}+\dfrac{1}{2}\vec{AA'}=\vec{b}-\vec{a}+\dfrac{1}{2}\vec{c}$
	}{\begin{tikzpicture}[line join = round, line cap = round, thick, font = \small, scale = .6]
	\path
	(0:0) coordinate (A)
	+(0:4) coordinate (C)
	+(-50:2) coordinate (B)
	+(75:4) coordinate (A')
	($(A')+(B)-(A)$) coordinate (B')
	($(A')+(C)-(A)$) coordinate (C')
	($(B)!.5!(B')$) coordinate (M)
	;
	\draw[dashed]
	(A)--(C)
	;
	\draw
	(A)--(B)--(C)--(C')--(A')--cycle
	(B')--(A') (B')--(B) (B')--(C') (A)--(M)
	;
	\foreach \x/\g in {A/180,B/-90,C/0,A'/-180,B'/70,C'/0,M/135}
	\fill (\x) circle (1.5pt)
	+(\g:3mm) node {$\x$};
	\end{tikzpicture}
	}
	}
\end{ex}
%%==========Câu 19
\begin{ex}
	Cho hình lập phương $ABCD.A'B'C'D'$ cạnh $a$. Tính độ dài véctơ $\vec{x}=\vec{A'C'}-\vec{A'A}$ theo $a$?
	\choice
	{$a\sqrt{2}$}
	{$\dfrac{a\sqrt{3}}{2}$}
	{$a\sqrt{6}$}
	{\True $a\sqrt{3}$}
	\loigiai{
	Ta có $\vec{x}=\vec{A'C'}-\vec{A'A}=\vec{AC'}=a\sqrt{3}$.
	}
\end{ex}
%%==========Câu 20
\begin{ex}
	\immini{Cho tứ diện $S.ABC$ có $M$, $N$, $P$ là trung điểm của $SA$, $SB$, $SC$. Tìm khẳng định đúng?}
	{\begin{tikzpicture}[line join = round, line cap = round, thick, font = \small, scale = .7]
	\path
	(0:0) coordinate (A)
	+(0:5) coordinate (C)
	+(-70:2) coordinate (B)
	+(65:4) coordinate (S)
	($(S)!.5!(A)$) coordinate (M)
	($(S)!.5!(B)$) coordinate (N)
	($(S)!.5!(C)$) coordinate (P)
	;
	\draw[dashed]
	(A)--(C) (M)--(P)
	;
	\draw
	(A)--(B)--(C)--(S)--cycle
	(S)--(B) (M)--(N)--(P)
	;
	\foreach \x/\g in {A/180,C/0,B/-90,S/45,M/135,N/-45,P/45}
	\fill (\x) circle (1.5pt)
	+(\g:3mm) node {$\x$};
	\end{tikzpicture}}
	\choice
	{$\vec{AB}=\dfrac{1}{2}\left(\vec{PN}-\vec{PM}\right)$}
	{$\vec{AB}=\vec{PN}-\vec{PM}$}
	{$\vec{AB}=2\left(\vec{PM}-\vec{PN}\right)$}
	{\True $\vec{AB}=2\left(\vec{PN}-\vec{PM}\right)$}
	\loigiai{
	Ta có: $\vec{AB}=2\vec{MN}=2\left(\vec{PN}-\vec{PM}\right)$.
	}
\end{ex}
%%==========Câu 21
\begin{ex}
	\immini{Cho tứ diện $S.ABC$ có đáy là tam giác đều cạnh $a$, $SB$ vuông góc với đáy và $SB=\sqrt{3}a$. Góc giữa hai vectơ $(\vec{AB},\vec{AS})$ là}
	{\begin{tikzpicture}[line join = round, line cap = round, thick, font = \small, scale = .7]
	\path
	(0:0) coordinate (B)
	+(0:5) coordinate (C)
	+(-50:3) coordinate (A)
	+(90:4) coordinate (S)
	;
	\draw[dashed]
	(B)--(C)
	;
	\draw
	(S)--(B)--(A)--(C)--cycle
	(S)--(A)
	\foreach \x/\y/\z in {S/B/C,S/B/A}{
	pic[draw, angle radius = 8pt]{right angle = \x--\y--\z}
	}
	;
	\foreach \x/\g in {B/180,C/0,A/-90,S/90}
	\fill (\x) circle (1.5pt)
	+(\g:3mm) node {$\x$};
	\end{tikzpicture}}
	\choice
	{\True $60^\circ$}
	{$30^\circ$}
	{$45^\circ$}
	{$90^\circ$}
	\loigiai{
	Ta có: $\left(\vec{AB},\vec{AS}\right)=\widehat{SAB}$.\\
	Xét $\triangle SBA$ vuông tại $B$ ta có: $\tan \left(\widehat{SAB}\right)=\dfrac{SB}{AB}=\sqrt{3}$. Suy ra: $\left(\vec{AB},\vec{AS}\right)=60^\circ$
	}
\end{ex}
%%==========Câu 22
\begin{ex}
	Cho hình chóp $S.ABC$ có $AB=4$, $\widehat{BAC}=60^\circ$, $\vec{AB} \cdot \vec{AC}=6$. Khi đó độ dài $\vec{AC}$ là
	\choice
	{\True $3$}
	{$6$}
	{$4$}
	{$12$}
	\loigiai{
	Ta có: $\vec{AB} \cdot \vec{AC}=AB \cdot AC \cdot \cos \widehat{BAC}\Leftrightarrow 6=4 \cdot AC \cdot \cos 60^\circ \Leftrightarrow AC=3$.
	}
\end{ex}
%%==========Câu 23
\begin{ex}
	Trong không gian cho vectơ $\vec{AB}$. Khi đó:
	\choice
	{Giá của vectơ $\vec{AB}$ là $\vec{AB}$}
	{Giá của vectơ $\vec{AB}$ là $\left| \vec{AB} \right|$}
	{\True Giá của vectơ $\vec{AB}$ là đường thẳng $AB$}
	{Giá của vectơ $\vec{AB}$ là đoạn thẳng $AB$}
	\loigiai{
	Giá của vectơ $\vec{AB}$ là đường thẳng $AB$.
	}
\end{ex}
%%==========Câu 24
\begin{ex}
	Cho hình hộp chữ nhật $ABCD.A'B'C'D'$. Trong các vectơ dưới đây, vectơ nào cùng phương với vectơ $\vec{AB}$?
	\choice
	{Vectơ$\vec{AD}$}
	{Vectơ$\vec{CC'}$}
	{Vectơ$\vec{BD}$}
	{\True Vectơ$\vec{CD}$}
	\loigiai{
	$AB \parallel CD$ nên $\vec{AB}$ và $\vec{CD}$ cùng phương.
	}
\end{ex}
%%==========Câu 25
\begin{ex}
	Cho hình hộp $ABCD.A'B'C'D'$. Vectơ $\vec{u}=\vec{A'A}+\vec{A'B'}+\vec{A'D'}$ bằng vectơ nào dưới đây?
	\choice
	{\True $\vec{A'C}$}
	{$\vec{CA'}$}
	{$\vec{AC'}$}
	{$\vec{C'A}$}
	\loigiai{
	Do $A'B'BA$ là hình bình hành nên $\vec{A'A}+\vec{A'B'}=\vec{A'B}$. Lại có, $A'BCD'$ cũng là hình bình hành nên $\vec{A'B}+\vec{A'D'}=\vec{A'C}$. Vậy $\vec{A'A}+\vec{A'B'}+\vec{A'D'}=\vec{A'C}$
	}
\end{ex}
%%==========Câu 26
\begin{ex}
	Cho hình lăng trụ tam giác $ABC.A'B'C'$. Đặt $\vec{AA'}=\vec{a}$, $\vec{AB}=\vec{b}$, $\vec{AC}=\vec{c}$, $\vec{BC}=\vec{d}$. Trong các biểu thức vec tơ sau đây, biểu thức nào là đúng?
	\choice
	{$\vec{a}=\vec{b}+\vec{c}$}
	{$\vec{a}+\vec{b}+\vec{c}+\vec{d}=\vec{0}$}
	{\True $\vec{b}-\vec{c}+\vec{d}=\vec{0}$}
	{$\vec{a}+\vec{b}+\vec{c}=\vec{d}$}
	\loigiai{
	Ta có: $\vec{b}-\vec{c}+\vec{d}=\vec{AB}-\vec{AC}+\vec{BC}=\vec{CB}+\vec{BC}=\vec{0}$.
	}
\end{ex}
%%==========Câu 27
\begin{ex}
	Cho lập phương $ABCD.A'B'C'D'$ có độ dài mỗi cạnh bằng $1$. Tính độ dài của vectơ $\vec{AC}+\vec{C'D'}$.
	\choice
	{$\sqrt{3}$}
	{$\sqrt{2}$}
	{\True $1$}
	{$2\sqrt{2}$}
	\loigiai{
	Ta có: $A'C'CA$ là hình chữ nhật nên $\vec{A'C'}=\vec{AC}$.\\
	Khi đó, $\vec{AC}+\vec{C'D'}=\vec{A'C'}+\vec{C'D'}=\vec{A'D'}$. Vậy $\left| \vec{AC}+\vec{C'D'} \right| =\left| \vec{A'D'} \right| =A'D'=1$
	}
\end{ex}
%%==========Câu 28
\begin{ex}
	Cho $O$ là tâm hình bình hành $ABCD$. Hỏi vectơ $\left(\vec{AO}-\vec{DO}\right)$ bằng vectơ nào?
	\choice
	{$\vec{BA}$}
	{\True $\vec{AD}$}
	{$\vec{DC}$}
	{$\vec{AC}$}
	\loigiai{
	Ta có: $\vec{AO}-\vec{DO}=\vec{AO}+\vec{OD}=\vec{AD}$.
	}
\end{ex}
%%==========Câu 29
\begin{ex}
	Cho ba điểm phân biệt $A$, $B$, $C$. Nếu $\vec{AB}=-3\vec{AC}$ thì đẳng thức nào dưới đây đúng?
	\choice
	{$\vec{BC}=-4\vec{AC}$}
	{$\vec{BC}=-2\vec{AC}$}
	{$\vec{BC}=2\vec{AC}$}
	{\True $\vec{BC}=4\vec{AC}$}
	\loigiai{
	Ta có: $\vec{AB}=-3\vec{AC}\Leftrightarrow \vec{CB}-\vec{CA}=-3\vec{AC}\Leftrightarrow \vec{AC}+3\vec{AC}=-\vec{CB}\Leftrightarrow \vec{BC}=4\vec{AC}$.
	}
\end{ex}
%%==========Câu 30
\begin{ex}
	Cho tam giác $ABC$ có điểm $O$ thỏa mãn: $\left| \vec{OA}+\vec{OB}-2\vec{OC} \right| = \left| \vec{OA}-\vec{OB} \right|$. Khẳng định nào sau đây là đúng?
	\choice
	{Tam giác $ABC$ đều}
	{Tam giác $ABC$ cân tại $C$}
	{\True Tam giác $ABC$ vuông tại $C$}
	{Tam giác $ABC$ cân tại $B$}
	\loigiai{
	Gọi $M$ là trung điểm $AB$, ta có $\vec{OA}+\vec{OB}=2\vec{OM}$.\\
	Do đó, $\left| \vec{OA}+\vec{OB}-2\vec{OC} \right| =\left| \vec{OA}-\vec{OB} \right|\Leftrightarrow \left| 2\vec{OM}-2\vec{OC} \right| =\left| \vec{BA} \right|\Leftrightarrow 2\left| \vec{CM} \right| =BA\Leftrightarrow CM=\dfrac{1}{2}BA$ \hfill $(1)$\\
	Vì $M$ là trung điểm $AB$ nên $CM$ là đường trung tuyến của $\triangle ABC$, Từ $(1)$ suy ra, tam giác $\triangle ABC$ vuông tại $C$.
	}
\end{ex}
%%==========Câu 31
\begin{ex}
	Cho hình hộp $ABCD.A'B'C'D'$. Đẳng thức nào dưới đây là đúng?
	\choice
	{$\vec{AC'}=\vec{AB}+\vec{AD}+\vec{AC}$}
	{$\vec{AC'}=\vec{AA'}+\vec{AD}+\vec{AC}$}
	{\True $\vec{AC'}=\vec{AB'}+\vec{AD}$}
	{$\vec{AC'}=\vec{AC}+\vec{AB}+\vec{AA'}$}
	\loigiai{
	Do $AB'C'D$ là hình bình hành nên $\vec{AC'}=\vec{A'B'}+\vec{AD}$.
	}
\end{ex}
%%==========Câu 32
\begin{ex}
	Cho hình lập phương $ABCD.A'B'C'D'$ có độ dài cạnh bằng $a$. Tính độ dài của vectơ $\vec{AD'}+\vec{BA'}$.
	\choice
	{$\sqrt{3}a$}
	{$\sqrt{2}a$}
	{\True $\sqrt{6}a$}
	{$2\sqrt{3}a$}
	\loigiai{
	Gọi $O'$ là tâm của hình vuông $A'B'C'D'$.\\
	Ta có $ABC'D'$ là hình bình hành nên $\vec{AD'}=\vec{BC'}$, do đó $\vec{BA'}+\vec{AD'}=\vec{BA'}+\vec{BC'}=2\vec{BO'}$.\\
	Tam giác $BA'C'$ là tam giác đều cạnh $a\sqrt{2}$ nên $BO'=\dfrac{\sqrt{3}}{2}a\sqrt{2}=\dfrac{\sqrt{6}}{2}a$.\\
	Từ đó độ dài của vectơ $\vec{AD'}+\vec{BA'}$ bằng $\sqrt{6}a$.
	}
\end{ex}
%%==========Câu 33
\begin{ex}%[2H2H1-4]
	\immini{
	Trong điện trường đều, lực tĩnh điện $\vec{F}$ (đơn vị: N) tác dụng lên điện tích điểm có điện tích $q$ (đơn vị: C) được tính theo công thức $\vec{F}=q \cdot \vec{E}$, trong đó $\vec{E}$ là cường độ điện trường (đơn vị: N/C). Tính độ lớn của lực tĩnh điện tác dụng lên điện tích điểm khi $q=10^{-9}$ C và độ lớn điện trường $E=10^5$ N/C.
	\choice
	{$10^{-3}$ N}
	{$10^{4}$ N}
	{$10^{-14}$ N}
	{\True $10^{-4}$ N}
	}{\hspace{1cm}
	\begin{tikzpicture}[scale=.7]
	\def\drong{3} % khoảng cách giữa 2 thanh
	\def\drog{0.3} % một nữa độ rộng của thanh
	\def\num{8}
	\filldraw [green!5, draw=green!80!black] ($(0,0)+(-\drog,\drog)$) rectangle ($(0,0)+(\num,0)+(\drog,-\drog)$)
	($(0,-\drong)+(-\drog,\drog)$) rectangle ($(0,-\drong)+(\num,0)+(\drog,-\drog)$);
	\foreach \a in {0,1,...,\num}{
	\draw[->,>=Latex] ($(0,0)+(\a,0)$)node[red]{$+$} ($(0,0)+(\a,0)+(0,-\drog)$)-- ($(0,-0.5*\drong)+(\a,0)$);
	\draw ($(0,-0.5*\drong)+(\a,0)$) -- ($(0,-\drong)+(\a,0)+(0,\drog)$) ($(0,-\drong)+(\a,0)$)node[red]{$-$} ;}
	\draw[->,>=Latex] (3.5,-0.4*\drong)--++(-90:1)node[below]{$\vec{F}$};
	\filldraw [green!5, draw=green!80!black] (3.5,-0.4*\drong)node[green!90!black]{$+$} circle (0.2cm) ;
	\draw (4.5,-0.5*\drong)node{$M$} (8.5,-0.5*\drong)node[red]{$\vec{E}$};
	\end{tikzpicture}
	}
	\loigiai{
	Từ công thức $\vec{F}=q \cdot \vec{E}$ suy ra $\begin{aligned}[t]
	|\vec{F}| & =q|\vec{E}| \\
	 & =10^{-9} \cdot 10^5 \\
	 & =10^{-4} \text{N}.
	\end{aligned}$\\
	Vậy độ lớn của lực tĩnh điện tác dụng lên điện tích điểm là $10^{-4}$ N.
	}
\end{ex} \dongcham{4}
\Closesolutionfile{ans}
\textbf{PHẦN II.} \textit{Câu trắc nghiệm đúng sai. Trong mỗi ý a), b), c), d) ở mỗi câu, học sinh chọn đúng hoặc sai.}\\
\Opensolutionfile{ans}[ans/2H2-B1-d1-2]
%%==========Câu 34
\begin{ex}
	\immini{Cho hình hộp chữ nhật $ABCD.A'B'C'D'$ có cạnh $AB=a$; $AD=a\sqrt{3}$; $AA'=2a$. Xét tính đúng, sai của các khẳng định sau:
	\choiceTF
	{$\vec{AB'}+\vec{CD'}=\vec{0}$}
	{\True $\vec{A'D}+\vec{CB'}=\vec{0}$}
	{$\big|\vec{AB}+\vec{AD}\big|=a\sqrt{5}$}
	{\True $\big|\vec{AB}+\vec{A'D'}+\vec{CC'}\big|=2\sqrt{2}a$}}{
	\begin{tikzpicture}[scale=0.6, font=\footnotesize, line join=round, line cap=round]
	\def\h{4}
	\foreach \x\y\t in {0/0/A,-1/-1.1/B,4.6/-1.1/C}
	\coordinate (\t) at (\x,\y);
	\coordinate (D) at ($(A)+(C)-(B)$);
	\coordinate (A') at ($(A)+(0,3.2)$);
	\coordinate (B') at ($(B)+(0,3.2)$);
	\coordinate (C') at ($(C)+(0,3.2)$);
	\coordinate (D') at ($(D)+(0,3.2)$);
	\draw (B')--(A')--(D')--(C')--(B')--(B)--(C)--(D)--(D') (C')--(C);
	\draw[dashed](B)--(A)--(D) (A)--(A');
	\foreach \t/\g in {A/170,B/-150,C/-70,D/0,A'/100,B'/170,C'/-20,D'/50}
	\draw[fill=black] (\t) circle(1pt)
	node[shift={(\g:7pt)}]{$\t$};
	\end{tikzpicture}}
	\loigiai{\begin{enumerate}[a)]
	\item $\vec{AB'}$ và $\vec{CD'}$ không đối nhau nên $\vec{AB'}+\vec{CD'} \ne \vec{0}$
	\item $\vec{A'D}$ và $\vec{CB'}$ đối nhau nên $\vec{AB'}+\vec{CD'} = \vec{0}$
	\item $\big|\vec{AB}+\vec{AD}\big|=\big|\vec{AC}\big|=AC=\sqrt{AB^2+AD^2}=2a$
	\item $\big|\vec{AB}+\vec{A'D'}+\vec{CC'}\big|=\big|\vec{AB}+\vec{AD}+\vec{AA'}\big|=AC'=\sqrt{AB^2+AD^2+AA^2}=2\sqrt{2}a$
	\end{enumerate}}
\end{ex} \dongcham{8}
%%==========Câu 35
\begin{ex}%[2H2H1-2]
	\immini{Cho hình lập phương $ABCD.A'B'C'D'$ có cạnh bằng $a$. Xét tính đúng, sai của các khẳng định sau:
	\choiceTF
	{\True $\vec{B'B} - \vec{DB} = \vec{B'D}$}
	{$\vec{BA}+\vec{BC}+\vec{BB'}=\vec{BD}$}
	{$\big|\vec{BA}+\vec{BC}+\vec{BB'}\big|=a\sqrt{2}$}
	{\True $\big|\vec{BC}-\vec{BA}+\vec{C'A}\big|=a$}
	}{
	\begin{tikzpicture}[scale=0.6, font=\footnotesize, line join=round, line cap=round]
	\def\h{4}
	\foreach \x\y\t in {0/0/A,-1/-1.1/B,2.6/-1.1/C}
	\coordinate (\t) at (\x,\y);
	\coordinate (D) at ($(A)+(C)-(B)$);
	\coordinate (A') at ($(A)+(0,3.2)$);
	\coordinate (B') at ($(B)+(0,3.2)$);
	\coordinate (C') at ($(C)+(0,3.2)$);
	\coordinate (D') at ($(D)+(0,3.2)$);
	\draw (B')--(A')--(D')--(C')--(B')--(B)--(C)--(D)--(D') (C')--(C);
	\draw[dashed](B)--(A)--(D) (A)--(A');
	\foreach \t/\g in {A/170,B/-150,C/-70,D/0,A'/100,B'/170,C'/-20,D'/50}
	\draw[fill=black] (\t) circle(1pt)
	node[shift={(\g:7pt)}]{$\t$};
	\end{tikzpicture}}
	\loigiai{
	\immini{\vspace*{-3mm}
	\begin{listEX}
	\item Ta có \begin{eqnarray*}
	\vec{B'B} - \vec{DB} &=& \vec{B'B} + \left( - \vec{DB} \right) \\
	&=& \vec{B'B} + \vec{BD} \\
	&=& \vec{B'D}.
	\end{eqnarray*}
	\item Áp dụng quy tắc hình hộp ta có $\vec{BA}+\vec{BC}+\vec{BB'}=\vec{BD'}$.\\
	\item $\big|\vec{BA}+\vec{BC}+\vec{BB'}\big|=\big|\vec{BD'}\big|=BD'=a\sqrt{3}$
	\item Ta có $\vec{BC}-\vec{BA}+\vec{C'A}=\vec{AC}+\vec{C'A}=\vec{C'C}$.\\
	Do đó $\big|\vec{BC}-\vec{BA}+\vec{C'A}\big|=C'C=a$
	\end{listEX}}
	{
	\begin{tikzpicture}[scale=0.6, font=\footnotesize, line join=round, line cap=round]
	\def\h{4}
	\foreach \x\y\t in {0/0/A,-1/-1.1/B,2.6/-1.1/C}
	\coordinate (\t) at (\x,\y);
	\coordinate (D) at ($(A)+(C)-(B)$);
	\coordinate (A') at ($(A)+(0,3.2)$);
	\coordinate (B') at ($(B)+(0,3.2)$);
	\coordinate (C') at ($(C)+(0,3.2)$);
	\coordinate (D') at ($(D)+(0,3.2)$);
	\draw (B')--(A')--(D')--(C')--(B')--(B)--(C)--(D)--(D') (C')--(C);
	\draw[dashed](B)--(A)--(D) (A)--(A');
	\foreach \t/\g in {A/170,B/-150,C/-70,D/0,A'/100,B'/170,C'/-20,D'/50}
	\draw[fill=black] (\t) circle(1pt)
	node[shift={(\g:7pt)}]{$\t$};
	\end{tikzpicture}}
	}
\end{ex} \dongcham{14}
%%==========Câu 36
\begin{ex}%[2H2N1-2]
	\immini{Cho hình lăng trụ tam giác $A B C.A' B' C'$ có $\vec{A A'}=\vec{a}$, $\vec{A B}=\vec{b}$ và $\vec{A C}=\vec{c}$. Gọi $M$ là trung điểm của $BC$. Xét tính đúng, sai của các khẳng định sau:
	\choiceTF
	{\True $\vec{B'C}=-\vec{a}-\vec{b}+\vec{c}$}
	{\True $\vec{BC'}=\vec{a}-\vec{b}+\vec{c}$}
	{$\vec{AM}=\vec{b}+\vec{c}$}
	{\True $\vec{A'M}=-\vec{a}+\dfrac{1}{2}\vec{b}+\dfrac{1}{2}\vec{c}$}
	}{
	\begin{tikzpicture}[scale=0.8, font=\footnotesize,>=stealth]
	\path
	(0,0) coordinate (A)
	(4,0) coordinate (C)
	(1.5,-1.5) coordinate (B)
	($(A)+(0.4,3)$)coordinate (A')
	($(B)+(0.4,3)$)coordinate (B')
	($(C)+(0.4,3)$)coordinate (C')
	($(B)!1/2!(C)$)coordinate (M)
	;
	\draw (B)--(C)--(C')--(B')--(B)--(A)--(A')--(B') (A')--(C');
	\draw[dashed] (C)--(A)--(M)--(A');
	\foreach \x/\g in {A/180,B/-45,C/0,A'/180,B'/-30,C'/0,M/-90}\draw[fill=black] (\x) circle (.04) +(\g:.4)node{\footnotesize$\x$};
	\end{tikzpicture}}
	\loigiai{
	\immini{\begin{enumerate}[a)]
	\item $\vec{B'C}=\vec{B'A'}+\vec{A'C'}+\vec{C'C}=-\vec{AB}+\vec{AC}-\vec{AA'}$ hay $\vec{B'C}=-\vec{a}-\vec{b}+\vec{c}$;
	\item $\vec{BC'}=\vec{BB'}+\vec{B'A'}+\vec{A'C'}=\vec{AA'}-\vec{AB}+\vec{AC}$ hay $\vec{BC'}=\vec{a}-\vec{b}+\vec{c}$;
	\item Ta có $\vec{AB}+\vec{AC}=2\vec{AM}$, suy ra $\vec{AM}=\dfrac{1}{2}\vec{AB}+\dfrac{1}{2}\vec{AC}=\dfrac{1}{2}\vec{b}+\dfrac{1}{2}\vec{c}$
	\item $\vec{A'M}=\vec{A'A}+\vec{AM}=\vec{A'A}+\dfrac{1}{2}\vec{AB}+\dfrac{1}{2}\vec{AC}=-\vec{a}+\dfrac{1}{2}\vec{b}+\dfrac{1}{2}\vec{c}$
	\end{enumerate}}{\begin{tikzpicture}[scale=0.6, font=\footnotesize,>=stealth]
	\path
	(0,0) coordinate (A)
	(4,0) coordinate (C)
	(1.5,-1.5) coordinate (B)
	($(A)+(0.4,3)$)coordinate (A')
	($(B)+(0.4,3)$)coordinate (B')
	($(C)+(0.4,3)$)coordinate (C')
	($(B)!1/2!(C)$)coordinate (M)
	;
	\draw (B)--(C)--(C')--(B')--(B)--(A)--(A')--(B') (A')--(C');
	\draw[dashed] (C)--(A)--(M)--(A');
	\foreach \x/\g in {A/180,B/-45,C/0,A'/180,B'/-30,C'/0,M/-90}\draw[fill=black] (\x) circle (.04) +(\g:.4)node{\footnotesize$\x$};
	\end{tikzpicture}}
	}
\end{ex} \dongcham{14}
%%==========Câu 37
\begin{ex}
	\immini{
	Cho tứ diện $ABCD$. Gọi $M$, $N$ lần lượt là trung điểm của các cạnh $AD$ và $BC$, $I$ là trung điểm $MN$. Xét tính đúng, sai của các khẳng định sau:
	\choiceTF
	{$\vv{A B}-\vv{C D}=\vv{A C}-\vv{B D}$}
	{\True $\vec{AB} + \vec{CD} = \vec{AD} + \vec{CB}$}
	{\True $\vec{AB} + \vec{DC}=2\vec{MN}$}
	{\True $\vec{IA} + \vec{IB} + \vec{IC} + \vec{ID} = \vec{0}$}
	}{
	\vspace*{-3mm}
	\begin{tikzpicture}[scale=0.5, font=\footnotesize, line join=round, line cap=round]
	\foreach \x\y\t in {0/0/B,6/0/D,1.5/-2/C,1.5/5/A}
	\coordinate (\t) at (\x,\y);
	\coordinate (M) at ($(A)!1/2!(D)$);
	\coordinate (N) at ($(B)!1/2!(C)$);
	\coordinate (I) at ($(M)!1/2!(N)$);
	\draw (A)--(B)--(C)--(D)--(A)--(C);
	\draw[dashed] (D)--(I)--(A) (B)--(I)--(C) (M)--(N) (B)--(D);
	\foreach \t/\g in {A/90,B/180,C/-90,D/0,M/0,N/180,I/20} \draw (\t) node[shift={(\g:10pt)}]{$\t$};
	\end{tikzpicture}}
	\loigiai{
	\begin{enumerate}
	\item Sử dụng quy tắc ba điểm và quy tắc hiệu, ta có
	 \begin{align*}
	 \vv{A B}-\vv{C D} & \ =\left(\vv{A C}+\vv{C B}\right)-\vv{C D} \\
	 & \ =\vv{A C}+\left(\vv{C B}-\vv{C D}\right) \\
	 & \ =\vv{A C}+\vv{D B} \\
	 & \ =\vv{A C}-\vv{B D}.
	 \end{align*}
	\item Theo quy tắc ba điểm, ta có $\vec{AB} = \vec{AD} + \vec{DB}$. Do đó
	 \begin{eqnarray*}
	 \vec{AB} + \vec{CD} &=& \vec{AD} + \vec{DB} + \vec{CD} \\
	 &=&\vec{AD}+ \left( \vec{CD} + \vec{DB} \right) \\
	 &=& \vec{AD} + \vec{CB}.
	 \end{eqnarray*}
	\item Ta có
	\item
	\end{enumerate}
	}
\end{ex} \dongcham{8}
%%==========Câu 38
\begin{ex}
	\immini
	{
	Một chiếc ô tô được đặt trên mặt đáy dưới của một khung sắt có dạng hình hộp chữ nhật với đáy trên là hình chữ nhật $ABCD$, mặt phẳng $(ABCD)$ song song với mặt phẳng nằm ngang. Khung sắt đó được buộc vào móc $E$ của chiếc cần cẩu sao cho các đoạn dây cáp $EA$, $EB$, $EC$, $ED$ có độ dài bằng nhau và cùng tạo với mặt phẳng $(ABCD)$ một góc bằng $60^\circ$. Chiếc cần cẩu kéo khung sắt lên theo phương thẳng đứng. Biết rằng các lực căng $\vec{F_1}$, $\vec{F_2}$, $\vec{F_3}$, $\vec{F_4}$ đều có cường độ là $4700$ N và trọng lượng của khung sắt là $3000$ N.
	\choiceTF
	{$\vec{F_1}+\vec{F_2}=\vec{F_3}+\vec{F_4}$}
	{\True $\vec{F_1}+\vec{F_3}=\vec{F_2}+\vec{F_4}$}
	{\True $\big|\vec{F_1}+\vec{F_3}\big|=8141$ N (\textit{làm tròn đến hàng đơn vị})}
	{Trọng lượng của chiếc xe ô tô là $16282$ N (\textit{làm tròn đến hàng đơn vị})}
	}
	{\hspace{1cm}
	\includegraphics[scale=.09]{images/xe-1.jpg}
	}
	\loigiai{
	Lấy các điểm $M$, $N$, $P$, $Q$ lần lượt trên các tia $EA$, $EB$, $EC$, $ED$ sao cho
	\[
	\vec{EM} = \vec{F_1},\ \vec{EN} = \vec{F_2},\ \vec{EP} = \vec{F_3},\ \vec{EQ} = \vec{F_4}.
	\]
	Do các lực căng $\vec{F_1}$, $\vec{F_2}$, $\vec{F_3}$, $\vec{F_4}$ đều có cường độ là $4700$ N nên $EM = EN = EP = EQ = 4700$.
	\begin{center}
	\begin{tikzpicture}[line join=round, line cap = round, >=stealth, scale=.8,font=\footnotesize,transform shape]
	\foreach \x/\y/\z/\g in
	{
	-3/0/A/180, -1/-1/B/-90, 3/0/C/0, 1/1/D/45, 0/4/E/90
	}
	\draw[fill=black] (\x,\y) circle(1pt) coordinate (\z) ($(\z)+(\g:3.5mm)$) node{$\z$};
	\path
	($(E)!.75!(A)$) coordinate (M)
	($(E)!.75!(B)$) coordinate (N)
	($(E)!.75!(C)$) coordinate (P)
	($(E)!.75!(D)$) coordinate (Q)
	($(M)!.5!(P)$) coordinate (O)
	;
	\draw (E)--(A)--(B)--(E)--(C)--(B) (M)--(N)--(P);
	\draw[dashed] (M)--(P)--(Q)--(M) (N)--(Q) (O)--(E)--(D)--(C) (A)--(D);
	\foreach \x/\g in {M/135, N/-45,P/45,Q/45,O/135}
	\draw[fill = white] (\x) circle(1pt) ($(\x)+(\g:3mm)$) node{$\x$};
	\end{tikzpicture}
	\end{center}
	\begin{enumerate}[a)]
	\item Ta có
	 \begin{itemize}
	 \item [$\bullet$] $\vec{F_1}+\vec{F_2}=\vec{EM}+\vec{EN}=2\vec{EH}$, với $H$ là trung điểm của $MN$.
	 \item [$\bullet$] $\vec{F_3}+\vec{F_4}=\vec{EP}+\vec{EQ}=2\vec{EK}$, với $K$ là trung điểm của $PQ$.
	 \end{itemize}
	 Suy ra $\vec{F_1}+\vec{F_2}\ne \vec{F_3}+\vec{F_4}$
	\item Ta có
	 \begin{itemize}
	 \item [$\bullet$] $\vec{F_1}+\vec{F_3}=\vec{EM}+\vec{EP}=2\vec{EO}$, với $O$ là trung điểm của $MP$.
	 \item [$\bullet$] $\vec{F_2}+\vec{F_4}=\vec{EN}+\vec{EQ}=2\vec{EO}$, với $O$ là trung điểm của $MP$.
	 \end{itemize}
	 Suy ra $\vec{F_1}+\vec{F_3}=\vec{F_2}+\vec{F_4}$.
	\item $\big|\vec{F_1}+\vec{F_3}\big|=\big|2\vec{EO}\big|=2EO$.\\
	 Theo giả thiết, góc giữa $EA$ với $(ABCD)$ bằng $60^\circ$, suy ra góc giữa $EM$ với $(MNPQ)$ cũng bằng $60^\circ$ hay $\widehat{SMO}=60^\circ$.\\
	 Xét $\triangle EMO$ có $EM=4700$, $\widehat{SMO}=60^\circ$. Suy ra $EO = EM \sin 60^\circ = 2350\sqrt{3}$.\\
	 Từ đây, ta tính được $\big|\vec{F_1}+\vec{F_3}\big|=2EO=8141$ N.
	\item Gọi $\vec{P}$ là trọng lực tác dụng lên cả hệ, do $O$ là trung điểm $MP$, $NQ$ nên ta có:
	 \begin{eqnarray*}
	 & \vec{P} & = \vec{F_1}+\vec{F_2}+\vec{F_3}+\vec{F_4}\\
	 & & = \vec{EM} + \vec{EN} + \vec{EP} + \vec{EQ}\\
	 & & = \vec{EO} + \vec{OM} + \vec{EO} + \vec{ON} + \vec{EO} + \vec{OP} + \vec{EO} + \vec{OQ}\\
	 & & = 4\vec{EO} + \left(\vec{OM} + \vec{OP}\right) + \left(\vec{ON} + \vec{OQ}\right)\\
	 & & = 4\vec{EO}.
	 \end{eqnarray*}
	 Suy ra trọng lượng của toàn bộ hệ là $\left| \vec{P} \right| = 4\left| \vec{EO}\right| = 4EO = 9400\sqrt{3}$ N.\\
	 Do trọng trượng khung sắt là $3000$ N nên trọng lượng của xe ô tô là $9400\sqrt{3} - 3000 \approx 13281$ N.
	\end{enumerate}
	}
\end{ex} \dongcham{14}
%%==========Câu 39
\begin{ex}
	\immini{Cho tứ diện $ABCD$ có $AB=AC=AD=a$ và $\widehat{BAC}=\widehat{BAD}=60^\circ ,\widehat{CAD}=90^\circ $. Gọi $I$ là điểm trên cạnh $AB$ sao cho $AI=3IB$ và $J$ là trung điểm của $CD$. Gọi $\alpha $ là góc giữa hai vectơ $\vec{AB}$ và $\vec{IJ}$.
	\choiceTF
	{\True Tam giác $BCD$ vuông cân}
	{$\vec{IJ}=\dfrac{1}{2}\vec{AC}+\dfrac{1}{2}\vec{AD}+\dfrac{3}{2}\vec{AB}$}
	{$\vec{AB} \cdot \vec{AC}+\vec{AC} \cdot \vec{AD}+\vec{AD} \cdot \vec{AB}=\dfrac{a^2}{2}$}
	{\True $\cos \alpha =-\dfrac{\sqrt{5}}{5}$}
	}{\begin{tikzpicture}[line join = round, line cap = round, thick, font = \small, scale = .7]
	\path
	(0:0) coordinate (B)
	+(0:5) coordinate (C)
	+(-70:2) coordinate (D)
	+(75:4) coordinate (A)
	($(B)!1/4!(A)$) coordinate (I)
	($(C)!.5!(D)$) coordinate (J)
	;
	\draw[dashed]
	(B)--(C) (I)--(J)
	;
	\draw
	(A)--(B)--(D)--(C)--cycle
	(A)--(D)
	;
	\foreach \x/\g in {B/180,C/0,D/-90,A/90,I/135,J/-45}
	\fill (\x) circle (1.5pt)
	+(\g:3mm) node {$\x$};
	\end{tikzpicture}}
	\loigiai{
	\begin{enumerate}[a)]
	\item Tam giác $ABC$, $ABD$ đều cạnh bằng $a$, tam giác $ACD$ vuông cân đỉnh $A\Rightarrow CD=a\sqrt{2}$. Vậy tam giác $BCD$ có $BC=BD=a,CD=a\sqrt{2}$ nên tam giác $BCD$ vuông cân.
	\item $\vec{IJ}=\vec{IA}+\vec{AJ}=-\dfrac{3}{4}\vec{AB}+\dfrac{1}{2}\left(\vec{AC}+\vec{AD}\right)=\dfrac{1}{2}\vec{AC}+\dfrac{1}{2}\vec{AD}-\dfrac{3}{4>}\vec{AB}$.
	\item Ta có: $\vec{AC} \cdot \vec{AD}=0$, $\vec{AB} \cdot \vec{AD}=AB \cdot AD \cdot \cos 60^\circ =\dfrac{a^2}{2}$, $\vec{AC} \cdot \vec{AB}=\dfrac{a^2}{2}$. Suy ra $\vec{AB} \cdot \vec{AC}+\vec{AC} \cdot \vec{AD}+\vec{AD} \cdot \vec{AB}=a^2$.\\
	\item $IJ^2=\vec{IJ}^2=\dfrac{1}{4}{{\left(\vec{AC}+\vec{AD}-\dfrac{3}{2}\vec{AB}\right)}^2}
	 =\dfrac{1}{4}\left(\dfrac{17}{4}a^2+2\vec{AC} \cdot \vec{AD}-3\vec{AC} \cdot \vec{AB}-3\vec{AB} \cdot \vec{AD}\right)
	 =\dfrac{5a^2}{16}\Rightarrow IJ=\dfrac{a\sqrt{5}}{4}$.\\
	 $\vec{IJ} \cdot \vec{AB}=\dfrac{1}{2}\left(\vec{AC}+\vec{AD}-\dfrac{3}{2}\vec{AB}\right) \cdot \vec{AB}= \dfrac{1}{2}\left(\vec{AC} \cdot \vec{AB}+\vec{AD} \cdot \vec{AB}-\dfrac{3}{2}{{\vec{AB}}^2}\right)=-\dfrac{a^2}{4}$.\\
	 $\cos \left(\vec{IJ},\vec{AB}\right)=\dfrac{\vec{IJ} \cdot \vec{AB}}{IJ \cdot AB}=\dfrac{-\dfrac{a^2}{4}}{\dfrac{a\sqrt{5}}{4} \cdot a}=-\dfrac{\sqrt{5}}{5}$.
	\end{enumerate}
	}
\end{ex}
%%==========Câu 40
\begin{ex}
	\immini{Cho tứ diện $ABCD$. Gọi $M$, $N$, $P$, $Q$, $R$, $S$, $G$ lần lượt là trung điểm các đoạn thẳng $AB$, $CD$, $AC$, $BD$, $AD$, $BC$, $MN$.
	\choiceTF
	{\True $\vec{MR}=\vec{SN}$}
	{\True $\vec{GA}+\vec{GB}+\vec{GC}+\vec{GD}=\vec{0}$}
	{$2\vec{PQ}=\vec{AB}+\vec{AC}+\vec{AD}$}
	{\True $|\vec{IA}+\vec{IB}+\vec{IC}+\vec{ID}|$ nhỏ nhất khi và chỉ khi điểm $I$ trùng với điểm $G$}
	}{\begin{tikzpicture}[line join = round, line cap = round, thick, font = \small, scale = .7]
	\path
	(0:0) coordinate (B)
	+(0:5) coordinate (C)
	+(-70:2) coordinate (D)
	+(75:4) coordinate (A)
	($(A)!.5!(B)$) coordinate (M)
	($(C)!.5!(D)$) coordinate (N)
	($(A)!.5!(C)$) coordinate (P)
	($(B)!.5!(D)$) coordinate (Q)
	($(A)!.5!(D)$) coordinate (R)
	($(B)!.5!(C)$) coordinate (S)
	($(M)!.5!(N)$) coordinate (G)
	;
	\draw[dashed]
	(B)--(C) (M)--(N)
	;
	\draw
	(A)--(B)--(D)--(C)--cycle
	(A)--(D)
	;
	\foreach \x/\g in {B/180,C/0,D/-90,A/90,M/135,N/-45,P/45,Q/-135,R/180,S/45,G/45}
	\fill (\x) circle (1.5pt)
	+(\g:3mm) node {$\x$};
	\end{tikzpicture}}
	\loigiai{
	\begin{center}
	\begin{tikzpicture}[line join = round, line cap = round, thick, font = \small, scale = .7]
	\path
	(0:0) coordinate (B)
	+(0:5) coordinate (C)
	+(-70:2) coordinate (D)
	+(75:4) coordinate (A)
	($(A)!.5!(B)$) coordinate (M)
	($(C)!.5!(D)$) coordinate (N)
	($(A)!.5!(C)$) coordinate (P)
	($(B)!.5!(D)$) coordinate (Q)
	($(A)!.5!(D)$) coordinate (R)
	($(B)!.5!(C)$) coordinate (S)
	($(M)!.5!(N)$) coordinate (G)
	;
	\draw[dashed]
	(B)--(C) (M)--(N) (P)--(Q) (R)--(S)
	;
	\draw
	(A)--(B)--(D)--(C)--cycle
	(A)--(D)
	;
	\foreach \x/\g in {B/180,C/0,D/-90,A/90,M/135,N/-45,P/45,Q/-135,R/180,S/45,G/45}
	\fill (\x) circle (1.5pt)
	+(\g:3mm) node {$\x$};
	\end{tikzpicture}
	\end{center}
	\begin{enumerate}[a)]
	\item $\vec{MR}=\vec{SN}=\dfrac12 \vec{BD}$.
	\item Vì $M$ là trung điểm của $AB$ nên $\vec{GA}+\vec{GB}=2\vec{GM}$\\
	 Vì $N$ là trung điểm của $CD$ nên $\vec{GC}+\vec{GD}=2\vec{GN}$\\
	 Vì $G$ là trung điểm của $MN$ nên $\vec{GM}+\vec{GN}=\vec{0}$\\
	 Do đó: $\vec{GA}+\vec{GB}+\vec{GC}+\vec{GD}=2\left(\vec{GM}+\vec{GN}\right)=2 \cdot \vec{0}=\vec{0}$.
	\item $\vec{PQ}=\vec{AQ}-\vec{AP}=\dfrac{1}{2}\left(\vec{AB}+\vec{AD}\right)-\dfrac{1}{2}\vec{AC}\Leftrightarrow 2\vec{PQ}=\vec{AB}-\vec{AC}+\vec{AD}$
	\item $\vec{IA}+\vec{IB}+\vec{IC}+\vec{ID}=4\vec{IG}+\left(\vec{GA}+\vec{GB}+\vec{GC}+\vec{GD}\right)=4\vec{IG}$.\\
	 $\Rightarrow | \vec{IA}+\vec{IB}+\vec{IC}+\vec{ID}|=| 4\vec{IG}|=4IG$\\
	 Do đó: $| \vec{IA}+\vec{IB}+\vec{IC}+\vec{ID}|$ nhỏ nhất khi $IG=0\Leftrightarrow I\equiv G$
	\end{enumerate}
	}
\end{ex}
%%==========Câu 41
\begin{ex}
	Cho tứ diện đều $SABC$ có cạnh $a$. Gọi $M$, $N$ lần lượt là trung điểm $SA$, $BC$. Các mệnh đề sau đúng hay sai?
	\begin{center}
	\begin{tikzpicture}[scale=1, font=\footnotesize, line join=round, line cap=round, >=stealth]
	\def\ac{4} % cạnh AC
	\def\ab{2} % cạnh AB
	\def\as{4} % cạnh AS
	\def\gocA{50} % góc A của đáy
	\path
	(0,0) coordinate (A)
	(\ac,0) coordinate (C)
	(-\gocA:\ab) coordinate (B)
	(70:\as) coordinate (S)
	($(S)!.5!(A)$) coordinate (M)
	($(B)!.5!(C)$) coordinate (N)
	;
	\draw (A)--(B)--(C)--(S)--cycle (S)--(B);
	\draw (S)--(A)node[midway,above left]{$a$};
	\draw[dashed] (A)--(C) (M)--(N);
	\foreach \x/\g in {A/180,B/-90,C/0,S/90}\fill (\x) circle (1pt)+(\g:3mm) node[black]{$\x$};
	\end{tikzpicture}
	\end{center}
	\choiceTF
	{\True Độ dài của vectơ $\vec{SA}$ bằng $a$.}
	{\True $\vec{SA} \cdot \vec{SB}=\dfrac{a^2\sqrt{3}}{2}$}
	{$\vec{SB}+\vec{AB}+\vec{SC}+\vec{AC}=4\vec{MN}$}
	{Gọi $I$ là trọng tâm của tứ diện. Khoảng cách từ $I$ đến $(ABC)$ bằng $\dfrac{3a\sqrt{6}}{4}$}
	\loigiai{
	\begin{center}
	\begin{tikzpicture}[scale=1, font=\footnotesize, line join=round, line cap=round, >=stealth]
	\def\ac{4} % cạnh AC
	\def\ab{2} % cạnh AB
	\def\as{4} % cạnh AS
	\def\gocA{50} % góc A của đáy
	\path
	(0,0) coordinate (A)
	(\ac,0) coordinate (C)
	(-\gocA:\ab) coordinate (B)
	(70:\as) coordinate (S)
	($(S)!.5!(A)$) coordinate (M)
	($(B)!.5!(C)$) coordinate (N)
	($(M)!.5!(N)$) coordinate (I)
	($(A)!2/3!(N)$) coordinate (G)
	;
	\draw (A)--(B)--(C)--(S)--cycle (S)--(B) (S)--(N) (M)--(B);
	\draw (S)--(A)node[midway,above left]{$a$};
	\draw[dashed] (A)--(C)--(M)--(N)--(A) (S)--(G) ;
	\foreach \x/\g in {A/180,B/-90,C/0,S/90}\fill (\x) circle (1pt)+(\g:3mm) node[black]{$\x$};
	\end{tikzpicture}
	\end{center}
	\begin{enumerate}[a)]
	\item $|\vec{SA}|=SA=a$.
	\item $\vec{SA} \cdot \vec{SB}=\left| \vec{SA} \right| \cdot \left| \vec{SB} \right| \cdot \sin \widehat{ASB}=a \cdot a \cdot \sin 60^\circ=\dfrac{a^2\sqrt{3}}{2}$.
	\item Do $N$ là trung điểm của $BC$ nên $\vec{SB}+\vec{SC}=2\vec{SN}$ và $\vec{AB}+\vec{AC}=2\vec{MB}$.\\
	Suy ra $\vec{SB}+\vec{SC}+\vec{AB}+\vec{AC}=2\left(\vec{SN}+\vec{AN}\right)$\\
	Do $M$ là trung điểm của $SA$ nên $\vec{NA}+\vec{NS}=2\vec{NM}\Leftrightarrow \vec{AN}+\vec{SN}=2\vec{MN}$.\\
	Do đó $\vec{SB}+\vec{SC}+\vec{AB}+\vec{AC}=2 \cdot 2 \cdot \vec{MN}=4\vec{MN}$.
	\item Gọi $G$ là trọng tâm tam giác $ABC$.\\
	Do tứ diện $SABC$ là tứ diện đều và $I$ là trọng tâm tứ diện nên $d\left(I,(ABC)\right)=IG$\\
	Tam giác $ABC$ đều cạnh $a$, $N$ là trung điểm của $BC$, suy ra $AN=\dfrac{a\sqrt{3}}{2}$.\\
	Do $G$ là trọng tâm tam giác$ABC$ nên $AG=\dfrac{2}{3}AN=\dfrac{a\sqrt{3}}{3}$.\\
	Do tứ diện $SABC$ là tứ diện đều nên $SG\bot (ABC)\Rightarrow SG\bot AG$.\\
	Tam giác $SAG$ vuông tại $G$ nên $SG=\sqrt{SA^2-AG^2}=\sqrt{a^2-\dfrac{a^2}{3}}=\dfrac{a\sqrt{6}}{3}$.\\
	Do $I$ là trọng tâm tứ diện$SABC$ nên $IG=\dfrac{1}{4}SG=\dfrac{1}{4} \cdot \dfrac{a\sqrt{6}}{3}=\dfrac{a\sqrt{6}}{12}$.\\
	Vậy $d\left(I,(ABC)\right)=\dfrac{a\sqrt{6}}{12}$.
	\end{enumerate}
	}
\end{ex}
%%==========Câu 42
\begin{ex}
	\immini{Cho hình hộp chữ nhật $ABCD \cdot EFGH$ có $AB=AE=2$, $AD=3$ và đặt $\vec{a}=\vec{AB},\vec{b}=\vec{AD},\vec{c}=\vec{AE}$. Lấy điểm $M$ thỏa $\vec{AM}=\dfrac{1}{5}\vec{AD}$ và điểm $N$ thỏa $\vec{EN}=\dfrac{2}{5}\vec{EC}$. (tham khảo hình vẽ).
	\choiceTF
	{\True $\vec{MA}=-\dfrac{1}{5}\vec{b}$}
	{\True $\vec{EN}=\dfrac{2}{5}\left(\vec{a}-\vec{b}+\vec{c}\right)$}
	{${{\left(m \cdot \vec{a}+n \cdot \vec{b}+n \cdot \vec{c}\right)}^2}=m^2 \cdot {{\vec{a}}^2}+n^2 \cdot {{\vec{b}}^2}+p^2 \cdot {{\vec{c}}^2}$ với $m,n,p$ là các số thực}
	{\True $MN=\dfrac{\sqrt{61}}{5}$}
	}{\begin{tikzpicture}[line join = round, line cap = round, thick, font = \small, scale = .7]
	\path
	(0:0) coordinate (H)
	+(75:3.5) coordinate (D)
	+(0:3) coordinate (G)
	+(40:2) coordinate (E)
	($(G)+(D)-(H)$) coordinate (C)
	($(D)+(E)-(H)$) coordinate (A)
	($(G)+(E)-(H)$) coordinate (F)
	($(C)+(A)-(D)$) coordinate (B)
	;
	\draw[dashed]
	(E)--(A) (E)--(F) (E)--(H)
	;
	\draw
	(A)--(B)--(F)--(G)--(H)--(D)--cycle
	(C)--(B) (C)--(D) (C)--(G)
	;
	\foreach \x/\g in {H/-90,G/-90,D/180,E/135,C/-45,A/90,F/0,B/90}
	\fill (\x) circle (1.5pt)
	+(\g:3mm) node {$\x$};
	\end{tikzpicture}}
	\loigiai{
	\begin{enumerate}[a)]
	\item $\vec{MA}=-\vec{AM}=-\dfrac{1}{5}\vec{AD}=-\dfrac{1}{5}\vec{b}$.
	\item $\vec{EN}=\dfrac{2}{5}\vec{EC}=\dfrac{2}{5}\left(\vec{EF}+\vec{EH}+\vec{EA}\right)=\dfrac{2}{5}\left(\vec{a}+\vec{b}-\vec{c}\right)$.
	\item ${{\left(m \cdot \vec{a}+n \cdot \vec{b}+p \cdot \vec{c}\right)}^2}=m^2 \cdot {{\vec{a}}^2}+n^2 \cdot {{\vec{b}}^2}+p^2 \cdot {{\vec{c}}^2}+2mn \cdot \vec{a} \cdot \vec{b}+2np \cdot \vec{b} \cdot \vec{c}+2mp \cdot \vec{a} \cdot \vec{c}$\\
	 $=m^2 \cdot {{\vec{a}}^2}+n^2 \cdot {{\vec{b}}^2}+p^2 \cdot {{\vec{c}}^2}$. (vì $\vec{a},\vec{b},\vec{c}$ đôi một vuông góc nên $\vec{a} \cdot \vec{b}=\vec{b} \cdot \vec{c}=\vec{a} \cdot \vec{c}=0$).
	\item $\vec{MN}=\vec{MA}+\vec{AE}+\vec{EN}=-\dfrac{1}{5}\vec{b}+\vec{c}+\dfrac{2}{5}\left(\vec{a}+\vec{b}-\vec{c}\right)=\dfrac{2}{5}\vec{a}+\dfrac{1}{5}\vec{b}+\dfrac{3}{5}\vec{c}$.\\
	 $MN^2={{\vec{MN}}^2}={{\left(\dfrac{2}{5}\vec{a}+\dfrac{1}{5}\vec{b}+\dfrac{3}{5}\vec{c}\right)}^2}=\dfrac{4}{25}{{\vec{a}}^2}+\dfrac{1}{25}{{\vec{b}}^2}+\dfrac{9}{25}{{\vec{c}}^2}=\dfrac{4}{25} \cdot 4+\dfrac{1}{25} \cdot 9+\dfrac{9}{25} \cdot 4=\dfrac{61}{25}$.\\
	 Suy ra $MN=\dfrac{\sqrt{61}}{5}$.
	\end{enumerate}
	}
\end{ex}
%%==========Câu 43
\begin{ex}
	\immini{Cho hình lăng trụ tam giác đều $ABC.A'B'C'$ có cạnh đáy bằng $x$ và chiều cao bằng $y$. (tham khảo hình vẽ)
	\choiceTF
	{\True $\vec{AB} \cdot \vec{AC}=\dfrac{1}{2}x^2$}
	{\True $\vec{AC'}=\vec{AC}+\vec{AA'}$}
	{$\vec{CB'}=\vec{AB}-\vec{CA}+\vec{AA'}$}
	{Góc $\left(AC',CB'\right)>60^\circ $ khi $\dfrac{y}{x}<\sqrt{2}$}
	}{\begin{tikzpicture}[line join = round, line cap = round, thick, font = \small, scale = .6]
	\path
	(0:0) coordinate (A)
	+(0:4) coordinate (C)
	+(-50:2) coordinate (B)
	+(90:4) coordinate (A')
	($(A')+(B)-(A)$) coordinate (B')
	($(A')+(C)-(A)$) coordinate (C')
	;
	\draw[dashed]
	(A)--(C)
	;
	\draw
	(A)--(B)--(C)--(C')--(A')--cycle
	(B')--(A') (B')--(B) (B')--(C')
	;
	\foreach \x/\g in {A/180,B/-90,C/0,A'/-180,B'/70,C'/0}
	\fill (\x) circle (1.5pt)
	+(\g:3mm) node {$\x$};
	\end{tikzpicture}}
	\loigiai{
	\begin{enumerate}[a)]
	\item $\vec{AB} \cdot \vec{AC}=AB \cdot AC \cdot \cos 60^\circ =\dfrac{1}{2}x^2$.
	\item $\vec{AC'}=\vec{AC}+\vec{AA'}$ (vì $ACC'A'$ là hình chữ nhật).
	\item $\vec{CB'}=\vec{CB}+\vec{CC'}=\vec{AB}-\vec{AC}+\vec{AA'}$.
	\item Ta có $\vec{AC'} \cdot \vec{CB'}=\left(\vec{AC}+\vec{AA'}\right) \cdot \left(\vec{AB}-\vec{AC}+\vec{AA'}\right)=y^2-\dfrac{1}{2}x^2$ và $AC'=CB'=\sqrt{x^2+y^2}$.\\
	 Khi đó $\cos \left(AC',CB'\right)=\left| \cos \left(\vec{AC'},\vec{CB'}\right) \right|=\dfrac{\left| \vec{AC'} \cdot \vec{CB'} \right|}{AC' \cdot CB'}=\dfrac{\left| y^2-\dfrac{1}{2}x^2\right|}{x^2+y^2}$.\\
	 Theo đề $\left(AC',CB'\right)>60^\circ $, suy ra $\dfrac{\left| y^2-\dfrac{1}{2}x^2\right|}{x^2+y^2}<\dfrac{1}{2}\Leftrightarrow 3y^4-6x^2y^2<0\Leftrightarrow \dfrac{y}{x}<\sqrt{2}$.
	\end{enumerate}
	}
\end{ex}
\textbf{PHẦN III.} \textit{Câu trắc nghiệm trả lời ngắn.}\\
%%==========Câu 44
\begin{ex}
	\immini{Cho hình lăng trụ $ABC.A'B'C'$. Đặt $\vec{AB}=\vec{a},\vec{AA'}=\vec{b},\vec{AC}=\vec{c}$. Ta biểu diễn $\vec{B'C}=m\vec{a}+n\vec{b}+p\vec{c}$, khi đó $m+n+p$ bằng bao nhiêu?}
	{\begin{tikzpicture}[line join = round, line cap = round, thick, font = \small, scale = .6]
	\path
	(0:0) coordinate (A)
	+(0:4) coordinate (C)
	+(-50:2) coordinate (B)
	+(75:4) coordinate (A')
	($(A')+(B)-(A)$) coordinate (B')
	($(A')+(C)-(A)$) coordinate (C')
	;
	\draw[dashed]
	(A)--(C)
	;
	\draw
	(A)--(B)--(C)--(C')--(A')--cycle
	(B')--(A') (B')--(B) (B')--(C')
	;
	\foreach \x/\g in {A/180,B/-90,C/0,A'/-180,B'/70,C'/0}
	\fill (\x) circle (1.5pt)
	+(\g:3mm) node {$\x$};
	\end{tikzpicture}}
	\loigiai{
	\SA{-1}
	$\vec{B'C}=\vec{B'B}+\vec{BC}=-\vec{BB'}+\vec{BA}+\vec{AC}=-\vec{BB'}-\vec{AB}+\vec{AC}=-\vec{b}-\vec{a}+\vec{c}$\\
	$\Rightarrow \vec{B'C}=-\vec{a}-\vec{b}+\vec{c}$.\\
	Suy ra $m=-1$, $n=-1$, $p=1$. Do đó $m+n+p=-1$.
	}
\end{ex}
%%==========Câu 45
\begin{ex}
	Cho tứ diện $ABCD$, gọi $I$, $J$ lần lượt là trung điểm của $AB$ và $CD$. Biết $\vec{IJ}=\dfrac{a}{b}\vec{AC}+\dfrac{c}{d}\vec{BD}$. Giá trị biểu thức $P=ab+cd$ bằng
	\loigiai{
	\SA{4}
	$\vec{AC}+\vec{BD}=\vec{AI}+\vec{IJ}+\vec{JC}+\vec{BI}+\vec{IJ}+\vec{JD}=2\vec{IJ}\Rightarrow \vec{IJ}=\dfrac{1}{2}\left(\vec{AC}+\vec{BD}\right)$.
	}
\end{ex}
%%==========Câu 46
\begin{ex}
	Cho tứ diện đều $ABCD$ có cạnh bằng $15$. Biết độ dài của $\vec{AB}+\vec{AC}+\vec{AD}$ bằng $a\sqrt{6}$, khi đó giá trị của $a$ là?
	\loigiai{
	\SA{15}
	\immini{
	Gọi $G$ là trọng tâm tâm giác $BCD$, $M$ là trung điểm $CD$.\\
	Ta có $\vec{GB}+\vec{GC}+\vec{GD}=\vec{0}\Leftrightarrow \left(\vec{GA}+\vec{AB}\right)+\left(\vec{GA}+\vec{AC}\right)+\left(\vec{GA}+\vec{AD}\right)=\vec{0}\Leftrightarrow 3\vec{GA}+\left(\vec{AB}+\vec{AC}+\vec{AD}\right)=\vec{0}$\\
	$\Leftrightarrow \vec{AB}+\vec{AC}+\vec{AD}=-3\vec{GA}=3\vec{AG}\Rightarrow | \vec{AB}+\vec{AC}+\vec{AD}|=| 3\vec{AG}|=3AG$.\\
	Xét tam giác đều $BCD$ có $BM=BC \cdot \dfrac{\sqrt{3}}{2}=\dfrac{15\sqrt{3}}{2}\Rightarrow BG=\dfrac{2}{3}BM=5\sqrt{3}$.\\
	Vì tứ diện $ABCD$ đều nên $AG\bot (BCD)\Rightarrow \widehat{AGB}=90^\circ $.\\
	Xét tam giác $ABG$ có $AG=\sqrt{AB^2-BG^2}=\sqrt{{{15}^2}-{{\left(5\sqrt{3}\right)}^2}}=5\sqrt{6}$.\\
	Do đó $| \vec{AB}+\vec{AC}+\vec{AD}|=3AG=15\sqrt{6}\Rightarrow a=15$.\\
	Vậy giá trị của $a=15$.
	}{\begin{tikzpicture}[line join = round, line cap = round, thick, font = \small, scale = .7]
	\path
	(0:0) coordinate (B)
	+(0:5) coordinate (C)
	+(-70:3) coordinate (D)
	(barycentric cs:B=1,C=1,D=1) coordinate (G)
	++(90:4) coordinate (A)
	($(C)!.5!(D)$) coordinate (M)
	;
	\draw[dashed]
	(M)--(B)--(C) (A)--(G)
	;
	\draw
	(A)--(B)--(D)--(C)--cycle
	(A)--(D)
	;
	\foreach \x/\g in {B/180,C/0,D/-90,G/45,A/90,M/-45}
	\fill (\x) circle (1.5pt)
	+(\g:3mm) node {$\x$};
	\end{tikzpicture}}
	}
\end{ex}
%%==========Câu 47
\begin{ex}
	Một chiếc cân đòn tay đang cân một vật có khối lượng $m=3\,\text{kg}$ được thiết kế với đĩa cân được giữ bởi bốn đoạn xích $SA$, $SB$, $SC$, $SD$ sao cho $S.ABCD$ là hình chóp tứ giác đều có $\widehat{ASC}=90^\circ $. Biết độ lớn của lực căng cho mỗi sợi xích có dạng $\dfrac{a\sqrt{2}}{4}$. Lấy $g=10 \mathrm{m}/\mathrm{s}^2$, khi đó giá trị của $a$ bằng bao nhiêu?
	\begin{center}
	\includegraphics{images/candon.png}
	\end{center}
	\loigiai{
	\SA{30}
	\immini{
	Gọi $O$ là tâm của hình vuông $ABCD$.\\
	Ta có $\vec{OA}+\vec{OB}+\vec{OC}+\vec{OD}=\vec{0}\Leftrightarrow \vec{OS}+\vec{SA}+\vec{OS}+\vec{SB}+\vec{OS}+\vec{SC}+\vec{OS}+\vec{SD}=\vec{0}$\\
	$\Leftrightarrow \vec{SA}+\vec{SB}+\vec{SC}+\vec{SD}=-4\vec{OS}=4\vec{SO}\Rightarrow | \vec{SA}+\vec{SB}+\vec{SC}+\vec{SD}| =| 4\vec{SO}|=4SO$.\\
	Trọng lượng của vật nặng là $P=mg=3 \cdot 10=30$ (N). Suy ra $4| \vec{SO}|=P=30$ (N) $\Rightarrow SO=\dfrac{15}{2}$.\\
	Lại có tam giác $ASC$ vuông cân tại $S$ nên\\
	$SO=SA \cdot \sin \widehat{SAC}\Rightarrow SA=\dfrac{SO}{\sin \widehat{SAC}}=\dfrac{\dfrac{15}{2}}{\sin 45^\circ}=\dfrac{15\sqrt{2}}{2}=\dfrac{30\sqrt{2}}{4}\Rightarrow a=30$.\\
	Vậy $a=30$.
	}{\begin{tikzpicture}[line join = round, line cap = round, thick, font = \small, scale = .7]
	\path
	(0:0) coordinate (A)
	+(0:5) coordinate (B)
	+(-140:2.5) coordinate (D)
	($(B)+(D)-(A)$) coordinate (C)
	(intersection of A--C and B--D) coordinate (O)
	++(90:5) coordinate (S)
	;
	\draw[dashed]
	(A)--(B) (A)--(D) (A)--(S) (A)--(C) (B)--(D) (S)--(O)
	;
	\draw
	(D)--(C)--(B)
	(S)--(B) (S)--(C) (S)--(D)
	;
	\foreach \x/\g in {A/135,B/0,C/-45,D/-135,O/-90,S/90}
	\fill (\x) circle (1.5pt)
	+(\g:3mm) node {$\x$};
	\end{tikzpicture}}
	}
\end{ex}
%%==========Câu 48
\begin{ex}
	Cho tứ diện $ABCD$. Trên các cạnh $AD$ và $BC$ lần lượt lấy $M$, $N$ sao cho $AM=3MD$, $BN=3NC$. Gọi $P$, $Q$ lần lượt là trung điểm của $AD$ và $BC$. Phân tích vectơ $\vec{MN}$ theo hai vectơ $\vec{PQ}$ và $\vec{DC}$ ta được $\vec{MN}=a\vec{PQ}+b\vec{DC}$. Tính $a+2b$.
	\loigiai{
	\SA{1,5}
	\begin{center}
	\begin{tikzpicture}[line join = round, line cap = round, thick, font = \small, scale = .7]
	\path
	(0:0) coordinate (B)
	+(0:5) coordinate (D)
	+(-30:4) coordinate (C)
	+(70:4) coordinate (A)
	($(A)!1/2!(D)$) coordinate (P)
	($(P)!1/2!(D)$) coordinate (M)
	($(B)!1/2!(C)$) coordinate (Q)
	($(Q)!1/2!(C)$) coordinate (N)
	;
	\draw[dashed]
	(B)--(D) (P)--(Q) (M)--(N)
	;
	\draw
	(A)--(B)--(C)--(D)--cycle
	(A)--(C)
	;
	\foreach \x/\g in {B/180,D/0,C/-90,A/90,M/45,P/45,N/-135,Q/-135}
	\fill (\x) circle (1.5pt)
	+(\g:3mm) node {$\x$};
	\end{tikzpicture}
	\end{center}
	Do $AM=3MD$, $BN=3NC$ và $P$, $Q$ lần lượt là trung điểm của $AD$ và $BC$ nên $M$, $N$ lần lượt là trung điểm của $PD$ và $QC$. \\
	Ta có $\heva{& \vec{MN}=\vec{MP}+\vec{PQ}+\vec{QN} \\& \vec{MN}=\vec{MD}+\vec{DC}+\vec{CN}}\Rightarrow 2\vec{MN}=\vec{PQ}+\vec{DC}\Rightarrow \vec{MN}=\dfrac{1}{2}\left(\vec{PQ}+\vec{DC}\right)$\\
	$\Rightarrow a=\dfrac{1}{2};\ b=\dfrac{1}{2}\Rightarrow a+2b=\dfrac{3}{2}=1,5$.
	}
\end{ex}
%%==========Câu 49
\begin{ex}
	Cho hình chóp $S.ABCD$ có đáy $ABCD$ là hình bình hành. Một mặt phẳng $(\alpha)$ cắt các cạnh $SA$, $SB$, $SC$, $SD$ lần lượt tại $A',B',C',D'$. Giá trị của biểu thức $P=\dfrac{SA}{SA'}+\dfrac{SC}{SC'}-\dfrac{SB}{SB'}-\dfrac{SD}{SD'}$.
	\loigiai{
	\SA{0}
	\begin{center}
	\begin{tikzpicture}[line join = round, line cap = round, thick, font = \small, scale = 1]
	\path
	(0:0) coordinate (A)
	+(0:5) coordinate (B)
	+(-140:2.5) coordinate (D)
	($(B)+(D)-(A)$) coordinate (C)
	($(A)!.5!(C)$) coordinate (O)
	++(100:5) coordinate (S)
	($(S)!6/13!(A)$) coordinate (A')
	($(S)!.6!(B)$) coordinate (B')
	($(S)!.5!(C)$) coordinate (C')
	($(S)!.4!(D)$) coordinate (D')
	;
	\draw[dashed]
	(A)--(B) (A)--(D) (A)--(S) (A)--(C) (B)--(D) (S)--(O) (B')--(A')--(D')--cycle (A')--(C')
	;
	\draw
	(D)--(C)--(B)
	(S)--(B) (S)--(C) (S)--(D) (B')--(C')--(D')
	;
	\foreach \x/\g in {A/135,B/0,C/-45,D/-135,O/-90,S/90,A'/135,B'/45,C'/-30,D'/135}
	\fill (\x) circle (1.5pt)
	+(\g:3mm) node {$\x$};
	\end{tikzpicture}
	\end{center}
	Gọi $O$ là tâm của hình bình hành $ABCD$ thì $\vec{SA}+\vec{SC}=\vec{SB}+\vec{SD}=2\vec{SO}$\\
	$\Leftrightarrow \dfrac{SA}{SA'}\vec{SA'}+\dfrac{SC}{SC'}\vec{SC'}=\dfrac{SB}{SB'}\vec{SB'}+\dfrac{SD}{SD'}\vec{SD'}$\\
	Do $A',B',C',D'$ đồng phẳng nên $\Rightarrow \dfrac{SA}{SA'}+\dfrac{SC}{SC'}=\dfrac{SB}{SB'}+\dfrac{SD}{SD'}\Rightarrow P=\dfrac{SA}{SA'}+\dfrac{SC}{SC'}-\dfrac{SB}{SB'}-\dfrac{SD}{SD'}=0$.
	}
\end{ex}
%%==========Câu 50
\begin{ex}
	Cho hình lập phương $B'C$ có đường chéo $A'C=\dfrac{3}{16}$. Gọi $O$ là tâm hình vuông $ABCD$ và điểm $20$ thỏa mãn: $\vec{OS}=\vec{OA}+\vec{OB}+\vec{OC}+\vec{OD}+\vec{OA'}+\vec{OB'}+\vec{OC'}+\vec{OD'}$. Khi đó độ dài của đoạn $OS$ bằng $\dfrac{a\sqrt{3}}{b}$ với $a,b\in \mathbb{N}$ và $\dfrac{a}{b}$ là phân số tối giản. Tính giá trị của biểu thức $P=a^2+b^2$.
	\loigiai{
	\SA{17}
	\begin{center}
	\begin{tikzpicture}[line join = round, line cap = round, thick, font = \small, scale = .7]
	\def \canh{4}
	\path
	(0:0) coordinate (D')
	+(90:\canh) coordinate (D)
	+(0:\canh) coordinate (C')
	+(40:.6*\canh) coordinate (A')
	($(C')+(D)-(D')$) coordinate (C)
	($(D)+(A')-(D')$) coordinate (A)
	($(C')+(A')-(D')$) coordinate (B')
	($(C)+(A)-(D)$) coordinate (B)
	($(A)!.5!(C)$) coordinate (O)
	($(A')!.5!(C')$) coordinate (O')
	;
	\draw[dashed]
	(A')--(A) (A')--(B') (C')--(A')--(D')--(B') (O)--(O')
	;
	\draw
	(A)--(B)--(B')--(C')--(D')--(D)--cycle
	(A)--(C)--(B)--(D)--(C) (C)--(C')
	;
	\foreach \x/\g in {D'/-90,C'/-90,D/180,A'/135,C/-45,A/90,B'/0,B/90,O/90,O'/-90}
	\fill (\x) circle (1.5pt)
	+(\g:3mm) node {$\x$};
	\end{tikzpicture}
	\end{center}
	Ta có: $A'C^2=A'A^2+AC^2=3A'A^2\Rightarrow A'A=\dfrac{A'C}{\sqrt{3}}=\dfrac{\sqrt{3}}{16}$.\\
	Gọi $O'$ là tâm của hình vuông $A'B'C'D'$.\\
	Lại có :
	$\begin{aligned}[t]
	\vec{OS}
	 & =\vec{OA}+\vec{OB}+\vec{OC}+\vec{OD}+\vec{OA'}+\vec{OB'}+\vec{OC'}+\vec{OD'} \\
	 & =\left(\vec{OA}+\vec{OC}\right)+\left(\vec{OB}+\vec{OD}\right)+\left(\vec{OA'}+\vec{OC'}\right)+\left(\vec{OB'}+\vec{OD'}\right) \\
	 & =2\vec{OO'}+2\vec{OO'}=4\vec{OO'}
	\end{aligned}$\\
	Suy ra $OS=\left| \vec{OS} \right| =\left| 4\vec{OO'} \right| =4OO'=4 \cdot \dfrac{\sqrt{3}}{16}=\dfrac{\sqrt{3}}{4}$.\\
	Khi đó $a=1,b=4\Rightarrow P=a^2+b^2=17$.
	}
\end{ex}
%%==========Câu 51
\begin{ex}
	Khi chuyển động trong không gian, máy bay luôn chịu tác động của 4 lực chính: lực đẩy của động cơ, lực cản của không khí, trọng lực và lực nâng khí động học (hình ảnh 2.20).
	\begin{center}
	\includegraphics*{images/h2.20.png}
	\end{center}
	Lực cản của không khí ngược hướng với lực đẩy của động cơ và có độ lớn tỉ lệ thuận với bình phương vận tốc máy bay. Một chiếc máy bay tăng vận tốc từ 900(km/h) lên 920(km/h), trong quá trình tăng tốc máy bay giữ nguyên hướng bay. Lực cản của không khí khi máy bay đạt vận tốc 900(km/h) và 920(km/h) lần lượt biểu diễn bởi hai véc tơ $\vec{F_1}$ và $\vec{F_2}$ với $\vec{F_1}=k\vec{F_2}(k\in \mathbb{R};k>0)$. Tính giá trị của $k$ (làm tròn kết quả đến chữ số thập phân thứ hai).
	\loigiai{
	\SA{0,96}
	Vì trong quá trình máy bay tăng vận tốc từ 900(km/h) lên 900(km/h), máy bay giữ nguyên hướng bay nên hai véc tơ $\vec{F_1}$ và $\vec{F_2}$ có cùng hướng và $\vec{F_1}=k\vec{F_2}(k>0)$.\\
	Gọi $v_1,v_2$ lần lượt là vận tốc của chiếc máy bay khi đạt 900(km/h) và 920(km/h).\\
	Suy ra $v_1=900$(km/h), $v_2=920$(km/h).\\
	Vì lực cản của không khí ngược hướng với lực đẩy của động cơ và có độ lớn tỉ lệ thuận với bình phương vận tốc máy bay nên $\left| \dfrac{\vec{F_1}}{\vec{F_2}} \right| =\dfrac{v_1^2}{v_2^2}=\dfrac{900^2}{920^2}=\dfrac{2025}{2116}\Rightarrow \left| \vec{F_1} \right| =\dfrac{2025}{2116}\left| \vec{F_2} \right|\Rightarrow \vec{F_1}=\dfrac{2025}{2116}\vec{F_2}$.\\
	Từ đó suy ra: $k=\dfrac{2025}{2116}\approx 0{,}96$.
	}
\end{ex}
%%==========Câu 52
\begin{ex}
	\immini{
	Một chiếc đèn tròn được treo song song với mặt phẳng nằm ngang bởi ba sợi dây không dãn xuất phát từ điểm $O$ trên trần nhà và lần lượt buộc vào ba điểm $A$, $B$, $C$ trên đèn tròn sao cho các lực căng $\vec{F_1}$, $\vec{F_2}$, $\vec{F_3}$ lần lượt trên mỗi dây $OA$, $OB$, $OC$ đôi một vuông góc với nhau và $\left| \vec{F_1} \right| = \left| \vec{F_2} \right| = \left| \vec{F_3} \right|$ = $15$ (N). Tính trọng lượng của chiếc đèn tròn đó (làm tròn đến hàng phần chục).
	}{\hspace{1cm}
	\begin{tikzpicture}[scale=0.55, font=\footnotesize, line
	join=round, line cap=round]
	%Toa do cac diem
	\coordinate (O) at (0,5);
	\coordinate (A) at (-2.2163, -0.4626);
	\coordinate (C) at (2.5, 0.1);
	\coordinate (B) at (-1.245, 0.867);
	\coordinate (A_1) at ($(O)!1/2!(A)$);
	\coordinate (B_1) at ($(O)!3/4!(B)$);
	\coordinate (C_1) at ($(O)!1/2!(C)$);
	%Ve hai day
	\draw[fill=blue!30] (0,-0.44) ellipse ({2.5} and {1});
	\draw[fill=blue!30] (0,0) ellipse ({2.5} and {1});
	%Ve cac doan thang
	\draw[line width=3] (-3,5)--(3,5);
	\draw(O)--(A);
	\draw(O)--(B);
	\draw(O)--(C);
	\draw[dashed] (0,0)--(C);
	\draw[dashed] (0,0)--(0,-2.5/2);
	%Ve cac diem
	\draw(A) node[above right]{$A$};
	\draw(B) node[above right]{$B$};
	\draw(C) node[above right]{$C$};
	\draw(C) node[above right]{$C$};
	\draw (0.2,5) node [below right]{$O$};
	\draw [fill=red] (0,0) circle (3.2pt);
	%Ve cac vecto
	\draw[-stealth, very thick, blue] (O)--(A_1) node [pos=0.5, left]{$\vec{F_1}$};
	\draw[-stealth, very thick, blue] (O)--(B_1) node [pos=0.5, right]{$\vec{F_2}$};
	\draw[-stealth, very thick, blue] (O)--(C_1) node [pos=0.5, right]{$\vec{F_3}$};
	\draw[-stealth,very thick](0,-2.5/2)--(0,-3.5) node [pos=0.5, right]{$\vec{P}$};
	\end{tikzpicture}}
	\loigiai{
	\SA{26,0}
	\immini{
	Gọi $A_1$, $B_1$, $C_1$ lần lượt là các điểm sao cho $\vec{OA_1} = \vec{F_1}$, $\vec{OB_1} = \vec{F_2}$, $\vec{OC_1} = \vec{F_3}$. Lấy các điểm $D_1$, $A_1'$, $B_1'$, $D_1'$ sao cho $OA_1D_1B_1.C_1A_1'D_1'B_1'$ là hình hộp (như hình bên). Khi đó, áp dụng quy tắc hình hộp ta có
	$$\vec{OA_1}+\vec{OB_1}+\vec{OC_1}=\vec{OD_1'}.$$
	Mặt khác, do các lực căng $\vec{F_1}$, $\vec{F_2}$, $\vec{F_3}$ đôi một vuông góc và $\left| \vec{F_1} \right| = \left| \vec{F_2} \right| = \left| \vec{F_3} \right|$ = $15$ (N) nên hình hộp $OA_1D_1B_1.C_1A_1'D_1'B_1'$ có ba cạnh $OA_1$, $OB_1$, $OC_1$ đôi một vuông góc và bằng nhau. Vì thế hình hộp đó là hình lập phương có độ dài cạnh bằng $15$. Suy ra độ dài đường chéo $OD_1'$ của hình lập phương đó bằng $15 \sqrt{3}$.\\
	Do chiếc đèn ở vị trí cân bằng nên $\vec{F_1}+\vec{F_2}+\vec{F_3}=\vec{P}$, ở đó $\vec{P}$ là trọng lực tác dụng lên chiếc đèn. Suy ra trọng lượng của chiếc đèn là $\left| \vec{P} \right| = \left| \vec{OD_1'} \right| =15\sqrt{3}$ (N).
	}{\begin{tikzpicture}[scale=0.5, font=\footnotesize, line join=round, line cap=round]
	\foreach \x\y\t in {0/0/B_1,-3.5/-3.5/D_1,2.5/-2.5/B_1',-0.6/3/O}
	\coordinate (\t) at (\x,\y);
	\coordinate (D_1') at ($(B_1')+(D_1)-(B_1)$);
	\coordinate (A_1) at ($(O)+(D_1)-(B_1)$);
	\coordinate (A_1') at ($(A_1)+(D_1')-(D_1)$);
	\coordinate (C_1) at ($(O)+(B_1')-(B_1)$);
	\draw (D_1)--(D_1')--(B_1');
	\draw (D_1)--(A_1)--(A_1')--(D_1');
	\draw (A_1')--(C_1)--(B_1');
	\draw[dashed] (D_1)--(B_1)--(B_1');
	\draw[-stealth] (O)--(A_1)node [pos=0.5, above left]{$\vec{F_1}$};
	\draw[-stealth] (O)--(C_1) node [pos=0.5, above right]{$\vec{F_3}$};
	\draw[dashed, -stealth] (O)--(B_1) node [pos=0.5, right]{$\vec{F_2}$};
	\draw[dashed, -stealth] (O)--(D_1');
	\foreach \t/\g in {B_1/-80,D_1/180,D_1'/-30,B_1'/0,O/100,A_1/180,A_1'/-160,C_1/0} \draw (\t) node[shift={(\g:10pt)}]{$\t$};
	\end{tikzpicture}}
	}
\end{ex}
%%==========Câu 53
\begin{ex}
	Cho hình hộp $ABCD.A'B'C'D'$. Xét các điểm $M$, $N$ lần lượt thuộc các đường thẳng $A'C$, $C'D$ sao cho đường thẳng $MN$ song song với đường thẳng $BD'$. Khi đó tỉ số $\dfrac{MN}{BD'}$ bằng
	\loigiai{
	\shortans{0,25}
	\begin{center}
	\begin{tikzpicture}[line join = round, line cap = round, thick, font = \small, scale = 1]
	\path 
	(0:0) coordinate (D)
	+(100:3) coordinate (D')
	+(0:3) coordinate (C)
	+(35:2) coordinate (A)
	($(C)+(D')-(D)$) coordinate (C')
	($(D')+(A)-(D)$) coordinate (A')
	($(C)+(A)-(D)$) coordinate (B)
	($(C')+(A')-(D')$) coordinate (B')
	($(C)!1/4!(A')$) coordinate (M)
	($(C')!.5!(D)$) coordinate (N)
	;
	\draw[dashed] 
	(A')--(A) (A)--(B) (A)--(D) (B)--(D') (A')--(C) (M)--(N)
	;
	\draw 
	(A')--(B')--(B)--(C)--(D)--(D')--cycle
	(C')--(B') (C')--(D') (D)--(C')--(C)
	;
	\foreach \x/\g in {D/-90,C/-90,D'/180,A/-60,C'/-45,A'/90,B/0,B'/90,M/-135,N/135}
	\fill (\x) circle (1.5pt)
	+(\g:3mm) node {$\x$};
	\end{tikzpicture}
	\end{center}
	Đặt $\vec{BA}=\vec{x}$, $\vec{BB'}=\vec{y}$, $\vec{BC}=\vec{z}$.\\
	Do $\vec{CM}$, $\vec{CA'}$ là hai vectơ cùng phương $\Rightarrow \exists \,k\in \mathbb{R}\colon \,\vec{CM}=k \cdot \vec{CA'}$.\\
	Và $\vec{C'N}$, $\vec{C'D}$ là hai vectơ cùng phương $\Rightarrow \exists \,h\in \mathbb{R}\colon \,\vec{C'N}=h \cdot \vec{C'D}$.\\
	Ta có: $\vec{BD'}=\vec{BA}+\vec{BC}+\vec{BB'}=\vec{x}+\vec{y}+\vec{z}$, \hfill (1)\\
	$\begin{aligned}[t]
	\vec{MN} & =\vec{CN}-\vec{CM}=\vec{CC'}+\vec{C'N}-\vec{CM}=\vec{CC'}+h \cdot \vec{C'D}-k \cdot \vec{CA'} \\
	& =\vec{y}+h \cdot (-\vec{y}+\vec{x})-k \cdot \left(\vec{y}-\vec{z}+\vec{x}\right)=(h-k) \cdot \vec{x}+(1-h-k) \cdot \vec{y}+k \cdot \vec{z}
	\end{aligned}$ \hfill (2)\\
	Do $MN\parallel B'D$ nên tồn tại $t\in \mathbb{R} \colon \vec{MN}=t \cdot \vec{BD'}$.\\
	Từ (1) và (2) ta có$\heva{& h-k=t \\& 1-h-k=t \\& k=t}\Leftrightarrow \heva{& k=t \\& h=2t \\& 1-3t=t}\Rightarrow t=\dfrac{1}{4}\Rightarrow \vec{MN}=\dfrac{1}{4}\vec{BD'}$.\\
	Vậy $\dfrac{MN}{BD'}=\dfrac{1}{4}=0,25$.
	}
\end{ex}
\Closesolutionfile{ans}
% \begin{dang}{Xác định góc và tính tích vô hướng của hai véctơ}
\end{dang}
\boxmini{BÀI TẬP TỰ LUẬN}
\setcounter{vd}{0}
\begin{vd}
	Cho hình lập phương $ ABCD.A'B'C'D' $ có cạnh bằng 5.
	\begin{tasks}
	\task Tìm góc giữa các cặp véc-tơ sau: $\vec{AC}$ và $\vec{AB}$; $\vec{AC}$ và $\vec{B'D'}$; $\vec{AC}$ và $\vec{CD}$; $\vec{AD'}$ và $\vec{BD}$.
	\task Tính các tích vô hướng $\vec{AC}\cdot \vec{AB}$; $\vec{AC}\cdot \vec{B'D'}$; $\vec{AD'}\cdot\vec{BD}$;
	\task Chứng minh $\vec{AC'}$ vuông góc với $\vec{BD}$.
	\end{tasks}
	\loigiai{
	\immini{\begin{enumerate}[a)]
	\item Ta có :
	 \begin{itemize}
	 \item [$\bullet$] $\left( \vec{AC},\vec{AB}\right)=\widehat{CAB}=45^\circ$
	 \item [$\bullet$] $\left(\vec{AC},\vec{B'D'}\right)=\left(\vec{AC},\vec{BD}\right)=90^\circ$.
	 \item [$\bullet$] $\left(\vec{AC},\vec{CD}\right)=\left(\vec{CE},\vec{CD}\right)=180^\circ-45^\circ=135^\circ$ (E là điểm đối xứng của $A$ qua $C$).
	 \item [$\bullet$] $ \vec{AD'}=\vec{BC'} \Rightarrow \left(\vec{AD'},\vec{BD}\right)=\left(\vec{BC'},\vec{BD}\right)=\widehat{C'BD}$.
	 Lại có, tam giác $ C'BD $ là tam giác đều nên $ \widehat{C'BD}=60^\circ\Rightarrow \left(\vec{AD'},\vec{BD}\right)=60^\circ $.
	 \end{itemize}
	\item Ta có $AC=BD=B'D'=5\sqrt{2}$. Suy ra
	 \begin{itemize}
	 \item [$\bullet$] $\vec{AC}\cdot \vec{AB}=AC.AB.\cos45^\circ =25$.
	 \item [$\bullet$] Do $AC$ vuông góc $B'D'$ nên $\vec{AC}\cdot \vec{B'D'}=0$.
	 \item [$\bullet$] $\vec{AD'}\cdot \vec{BD}=AD'.BD.\cos60^\circ =5\sqrt{2}.5\sqrt{2}.\dfrac{1}{2}=25$.
	 \end{itemize}
	\item Ta cần chứng minh $\vec{AC'}\cdot \vec{BD}=0$.\\
	 Ta có: $\vec{AC'}=\vec{AB}+\vec{AD}+\vec{AA'}$ và $\vec{BD}=\vec{AD}-\vec{AB}$ nên
	 \begin{eqnarray*}
	 \vec{AC'}\cdot \vec{BD}
	 &=&\left( \vec{AB}+\vec{AD}+\vec{AA'}\right)\cdot \left(\vec{AD}-\vec{AB} \right) \\
	 &=&\vec{AB}.\vec{AD}-\vec{AB}^2+\vec{AD}^2-\vec{AD}.\vec{AB}+\vec{AA'}.\vec{AD}-\vec{AA'}.\vec{AB}=5^2-5^2=0
	 \end{eqnarray*}
	 Suy ra $\vec{AC'}$ vuông góc với $\vec{BD}$.
	\end{enumerate}}{
	\begin{tikzpicture}[scale=0.6, font=\footnotesize, line join=round, line cap=round]
	\def\h{4}
	\foreach \x\y\t in {0/0/A,-1/-1.1/B,2.6/-1.1/C}
	\coordinate (\t) at (\x,\y);
	\coordinate (D) at ($(A)+(C)-(B)$);
	\coordinate (A') at ($(A)+(0,3.2)$);
	\coordinate (B') at ($(B)+(0,3.2)$);
	\coordinate (C') at ($(C)+(0,3.2)$);
	\coordinate (D') at ($(D)+(0,3.2)$);
	\coordinate (E) at ($2*(C)-(A)$);
	\draw (B')--(A')--(D')--(C')--(B')--(B)--(C)--(D)--(D') (C')--(C)--(E);
	\draw[dashed](B)--(A)--(D) (A)--(A') (A)--(C);
	\foreach \t/\g in {A/170,B/-150,C/-100,D/0,A'/100,B'/170,C'/-20,D'/50,E/0}
	\draw[fill=black] (\t) circle(1pt)
	node[shift={(\g:7pt)}]{$\t$};
	\end{tikzpicture}
	}
	}
\end{vd}

\begin{vd}%[2H2H1-3]
	Cho tứ diện đều $ABCD$ có cạnh bằng $a$ và $M$ là trung điểm của $CD$.
	\begin{listEX}[2]
	\item Tính các tích vô hướng $\vec{AB} \cdot \vec{AC}$, $\vec{AB} \cdot \vec{AM}$.
	\item Tính góc $(\vec{AB}, \vec{CD})$.
	\end{listEX}
	\loigiai{
	\begin{enumerate}
	\item Ta có $\begin{aligned}[t]
	 \vec{AC} \cdot \vec{AC} & = |\vec{AB}| \cdot |\vec{AC}| \cdot \cos (\vec{AB},\vec{AC}) \\
	 & = AB \cdot AC \cdot \cos \widehat{BAC} \\
	 & = a \cdot a \cdot \cos {60^0} \\
	 & = \frac{a^2}{2}.
	 \end{aligned}$\\
	 Tương tự ta cũng có $\vec{AB} \cdot \vec{AD} = \dfrac{a^2}{2}$.
	 \immini{
	 Ta lại có $\vec{AM} = \dfrac{1}{2}(\vec{AC} + \vec{AD})$, suy ra
	 \[ \vec{AB} \cdot \vec{AM} = \vec{AB} \cdot \frac{1}{2}(\vec{AC} +\vec{AD}) = \frac{1}{2}(\vec{AB} \cdot \vec{AC} + \vec{AB} \cdot \vec{AD}) = \frac{1}{2}\left(\frac{a^2}{2} + \frac{a^2}{2}\right) = \frac{a^2}{2}.\]
	\item Ta có $\vec{AB} \cdot \vec{CD}=(\vec{AM}+\vec{MB}) \cdot \vec{CD}=\vec{AM} \cdot \vec{CD} +\vec{MB} \cdot \vec{CD}$.\\
	 Mà $AM$, $BM$ là trung tuyến của các tam giác đều $ACD$, $BCD$ nên $\vec{AM} \perp \vec{CD}, \vec{MB} \perp \vec{CD}$.\\
	 Suy ra $\vec{AM} \cdot \vec{CD} = \vec{MB} \cdot \vec{CD} = 0$.\\
	 Từ các kết quả trên ta có $\vec{AM} \cdot \vec{CD}=0$.
	 Suy ra $(\vec{AB}, \vec{CD})=90^\circ$.
	 }{
	 \begin{tikzpicture}[scale=1, font=\footnotesize, line join=round, line cap=round, >=Stealth]
	 \def\a{3}
	 \path
	 (0:0) coordinate (B)
	 (5:\a) coordinate (D)
	 ($(D)+(-135:\a/2)$) coordinate (C)
	 ($(B)+(65:\a)$) coordinate (A)
	 ($(C)!.5!(D)$) coordinate (M)
	 ;
	 \draw[->] (A)--(D);
	 \draw[->] (A)--(C);
	 \draw[->] (A)--(B);
	 \draw[->] (A)--(M);
	 \draw[dashed] (B)--(D);
	 \draw (B)--(C)--(D);
	 \foreach \x/\g in {A/90,B/180,C/-90,D/0,M/0}
	 \draw[fill=black] 	(\x) circle (.5pt)
	 ($(\g:.3)+(\x)$) node {$\x$};
	 \end{tikzpicture}
	 }
	\end{enumerate}
	}
\end{vd}

\begin{vd}
	\immini{Cho biết công $A$ (đơn vị: $J$) sinh bởi lực $\vec{F}$ tác dụng lên một vật được tính bằng công thức $A = \vec{F}\cdot\vec{d}$, trong đó $\vec{d}$ là vectơ biểu thị độ dịch chuyển của vật (đơn vị của $\left|\vec{d}\right|$ là m) khi chịu tác dụng của lực $\vec{F}$.
	}{
	\begin{tikzpicture}[scale=0.68,font=\footnotesize, line join=round, line cap=round, >=stealth]
	\def\xe{orange!60}
	\def\den{red}
	\draw (0,3)--(10,0) (0,0)--(10,0);
	\fill[black!30] (2.48,2.5)--(2.57,2.75)--(5.8,1.7)--(5.73,1.47)--cycle ;
	\fill[black!70] (2.48,2.5)--(2.57,2.75)--(5.2,2)--(5.18,1.6)--cycle ;
	\fill[black!20] (2.57,2.75)--(3,3.9)--(5.2,3.2)--(4.78,2)--cycle ;
	\fill[\xe] (4.78,2)--(5.13,3)--(5.7,2.7)--(5.9,2.2)--(5.75,1.72)--cycle ;
	\fill[green] (5.2,2.8)--(5.65,2.56)--(5.8,2.2)--(5.72,1.94)--(5,2.2)--cycle ;
	\fill[black!70] (3.2,2.36) circle (0.3) (5,1.81) circle (0.3) ;
	\fill[black!20] (3.2,2.36) circle (0.2) (5,1.81) circle (0.2) ;
	\fill[blue](3.2,2.36) circle (3pt) (5,1.81) circle (3pt) (4,2.7) circle (2pt) ;
	\draw[dashed] (4,2.7)--(3.3,0.3) ;
	\draw[->] (4,2.7)--(4,0.3) node[right] {$\vec{P}$};
	\draw[->] (4,2.7)--(10,0.8) node[above] {$\vec{d}$};
	\end{tikzpicture} }
	Một chiếc xe có khối lượng $1{,}5$ tấn đang đi xuống trên một đoạn đường dốc có góc nghiêng $5^\circ$ so với phương ngang. Tính công sinh bởi trọng lực $\vec{P}$ khi xe đi hết đoạn đường dốc dài $30$ m (làm tròn kết quả đến hàng đơn vị), biết rằng trọng lực $\vec{P}$ được xác định bởi công thức $\vec{P} = m\vec{g}$, với $m$ (đơn vị: kg) là khối lượng của vật và $\vec{g}$ là gia tốc rơi tự do có độ lớn $g = 9{,}8$ m/s$^2$.
	\loigiai{
	\begin{center}
	\begin{tikzpicture}[scale=0.8,font=\footnotesize, line join=round, line cap=round, >=stealth]
	\def\xe{orange!60}
	\def\den{red}
	\draw (0,3)--(10,0) (0,0)--(10,0);
	\fill[black!30] (2.48,2.5)--(2.57,2.75)--(5.8,1.7)--(5.73,1.47)--cycle ;
	\fill[black!70] (2.48,2.5)--(2.57,2.75)--(5.2,2)--(5.18,1.6)--cycle ;
	\fill[black!20] (2.57,2.75)--(3,3.9)--(5.2,3.2)--(4.78,2)--cycle ;
	\fill[\xe] (4.78,2)--(5.13,3)--(5.7,2.7)--(5.9,2.2)--(5.75,1.72)--cycle ;
	\fill[green] (5.2,2.8)--(5.65,2.56)--(5.8,2.2)--(5.72,1.94)--(5,2.2)--cycle ;
	\fill[black!70] (3.2,2.36) circle (0.3) (5,1.81) circle (0.3) ;
	\fill[black!20] (3.2,2.36) circle (0.2) (5,1.81) circle (0.2) ;
	\fill[blue](3.2,2.36) circle (3pt) (5,1.81) circle (3pt) (4,2.7) circle (2pt) ;
	\draw[dashed] (4,2.7)--(3.3,0.3) ;
	\draw[->] (4,2.7)--(4,0.3) node[right] {$\vec{P}$};
	\draw[->] (4,2.7)--(10,0.8) node[above] {$\vec{d}$};
	\end{tikzpicture}
	%\includegraphics{hinhanh/H1.png} 
	\end{center}
	\noindent
	Ta có $1{,}5$ tấn = $1~500$ kg.\\
	Độ lớn của trọng lực tác dụng lên chiếc xe là $\left|\vec{P}\right| = m \left|\vec{g}\right| = 1~500\cdot 9{,}8 = 14~700$ (N).\\
	Vectơ d biểu thị độ dịch chuyển của xe có độ dài là $\left|\vec{d}\right| = 30$ (m) và $ \left(\vec{P},\vec{d}\right)= 90^\circ - 5^\circ = 85^\circ$.\\
	Công sinh ra bởi trọng lực $\vec{P}$ khi xe đi hết đoạn đường dốc dài $30$ m là
	$$A=\vec{P}\cdot\vec{d}=\left|\vec{P}\right|\cdot\left|\vec{d}\right|\cdot\cos\left(\vec{P},\vec{d}\right) = 14~700\cdot 30\cdot \cos 85^\circ \approx 38~436~(J).$$
	}
\end{vd}

\begin{vd}
	\immini{
	Một chất điểm $A$ nằm trên mặt phẳng nằm ngang $\left(\alpha\right)$, chịu tác động bởi ba lực $\vec{F}_1$, $\vec{F}_2$, $\vec{F}_3$. Các lực $\vec{F}_1$, $\vec{F}_2$ có giá nằm trong $\left(\alpha\right)$ và $\left(\vec{F}_1, \vec{F}_2\right)=135^\circ$, còn lực $\vec{F}_3$ có giá vuông góc với $\left(\alpha\right)$ và hướng lên trên. Xác định cường độ hợp lực của các lực $\vec{F}_1$, $\vec{F}_2$, $\vec{F}_3$ biết rằng độ lớn của ba lực đó lần lượt là $20$ N, $15$ N và $10$ N.
	}{
	\begin{tikzpicture}[line join=round, line cap = round, >=stealth, scale=0.75,font=\footnotesize]
	\path
	(0,0) coordinate (A)
	(-2,-1.5) coordinate (B)
	(3,0) coordinate (C)
	(0,3) coordinate (D)
	;
	\draw[->] (A)--(B) node[below] {$\vec{F}_1$};
	\draw[->] (A)--(C) node[below] {$\vec{F}_2$};
	\draw[->] (A)--(D) node[right] {$\vec{F}_3$};
	\draw pic[draw,angle eccentricity=1.2,angle radius=0.2cm]{right angle= D--A--C};
	\draw pic[draw,angle eccentricity=1.2,angle radius=0.2cm]{right angle= D--A--B};
	\draw pic[draw,angle eccentricity=1.2,angle radius=0.3cm]{angle= B--A--C};
	\draw ($(A)-(-0.3,0.6)$) node {$135^\circ$};
	\foreach \x/\g in {A/140} \fill[black](\x) circle (1pt) ($(\x)+(\g:4mm)$)node{$\x$};
	\end{tikzpicture}
	}
	\loigiai{
	Gọi $\vec{F}$ là hợp lực của các lực $\vec{F}_1$, $\vec{F}_2$, $\vec{F}_3$, tức là $\vec{F}= \vec{F}_1+ \vec{F}_2+ \vec{F}_3$, ta có
	\allowdisplaybreaks
	\begin{eqnarray*}
	\left|\vec{F}\right|^2 &=& \left(\vec{F}_1+ \vec{F}_2+ \vec{F}_3\right)^2\\
	&=& \vec{F}_1^2+ \vec{F}_2^2+ \vec{F}_3^2+ 2\vec{F}_1\cdot\vec{F}_2+ 2\vec{F}_2\cdot\vec{F}_3+ 2\vec{F}_3\cdot\vec{F}_1\\
	&=& 20^2+ 15^2+ 10^2+ 2\cdot 20\cdot 15\cdot\cos 135^\circ\\
	&=& 725-300\sqrt2.
	\end{eqnarray*}
	Vậy $\left|\vec{F}\right|= \sqrt{725-300\sqrt2} \approx 17{,}34$ (N).
	}
\end{vd}

\boxmini{BÀI TẬP TRẮC NGHIỆM}
\textbf{PHẦN I.} \textit{Câu trắc nghiệm nhiều phương án lựa chọn. Mỗi câu hỏi học sinh chỉ chọn một phương án.}\\
\setcounter{ex}{0}
\Opensolutionfile{ans}[ans/2H2-B1-d2-1]
%%==========Câu 1
\begin{ex}
	\immini{Cho hình lập phương $ABCD.A'B'C'D'$. Khẳng định nào sau đây là khẳng định \textbf{sai}?
	\choice
	{ $\left(\vec{A'C'},\vec{AD}\right)=45^\circ$}
	{$\left(\vec{A'C'},\vec{B'B}\right)=90^\circ$}
	{\True $\left(\vec{A'A}, \vec{CB'}\right)=45^\circ$}
	{$\left(\vec{AB},\vec{CD}\right)=180^\circ$}
	}{
	\begin{tikzpicture}[scale=0.65, font=\footnotesize,>=stealth]
	%Gán số liệu.
	\def\canhAD{3};\def\canhBA{2};\def\gocBAD{-130};\def\h{3};\def\xdinhA'{0};
	%Gán tọa độ.
	\coordinate (A) at (0,0);
	\coordinate (B) at ($(A)+(\gocBAD:\canhBA)$);
	\coordinate (C) at ($(B)+(0:\canhAD)$);
	\coordinate (D) at ($(A)+(0:\canhAD)$);
	\coordinate (A') at ($(A)+(\xdinhA',\h)$);
	\coordinate (B') at ($(B)+(\xdinhA',\h)$);
	\coordinate (C') at ($(C)+(\xdinhA',\h)$);
	\coordinate (D') at ($(D)+(\xdinhA',\h)$);
	%Vẽ khối lẳng trụ ABCD.A'B'C'D'.
	\draw (A')--(B')--(B)--(C)--(C')--(D')--cycle (B')--(C') (D')--(D)--(C) (A')--(C');
	\draw[dashed] (A)--(D) (A')--(A)--(B) (A)--(C);
	%Gán nhãn.
	\foreach \x/\y in {A/180, B/180, C/0, D/0, A'/180, B'/180, C'/0, D'/0}{\fill (\x) circle(1pt) ($(\x)+(\y:0.3cm)$) node{$\x$};}
	\end{tikzpicture}}
	\loigiai{
	\begin{itemize}
	\item [$\bullet$] Ta có $\left(\vec{A'C'},\vec{AD}\right)=\left(\vec{A'C'},\vec{A'D'}\right)=\widehat{C'A'D'}=45^\circ$.
	\item [$\bullet$] $\left(\vec{A'C'},\vec{B'B}\right)=\left(\vec{A'C'},\vec{A'A}\right)=\widehat{AA'C'}=90^\circ$.
	\item [$\bullet$] Ta có $\vec{B'B}=\vec{A'A}$, suy ra\\
	 $\left(\vec{A'A},\vec{CB'}\right)=\left(\vec{B'B},\vec{CB'}\right)=180^{\circ}-\widehat{BB'C}=180^{\circ}-45^{\circ}=135^{\circ}$
	\item [$\bullet$] $\vec{AB}$ ngược hướng với $\vec{CD}$ nên $\left(\vec{AB},\vec{CD}\right)=180^\circ$.
	\end{itemize}
	}
\end{ex} 
%%==========Câu 2
\begin{ex}
	\immini{Cho tứ diện đều $ABCD$, Gọi $M$, $N$ lần lượt là trung điểm các cạnh $AB$, $AC$. Hãy tính góc giữa hai vectơ $\vec{MN}$ và $\vec{BD}$.
	\choice
	{$ \left(\vec{MN}, \vec{BD} \right) = 150^\circ$}
	{$ \left(\vec{MN}, \vec{BD} \right) = 120^\circ$}
	{$ \left(\vec{MN}, \vec{BD} \right) = 30^\circ$}
	{\True $ \left(\vec{MN}, \vec{BD} \right) = 60^\circ$}}{
	\begin{tikzpicture}[scale=0.55, font=\footnotesize, line join=round, line cap=round]
	\foreach \x\y\t in {0/0/B,6/0/D,1.5/-2/C,1.5/5/A}
	\coordinate (\t) at (\x,\y);
	\coordinate (M) at ($(A)!1/2!(B)$);
	\coordinate (N) at ($(A)!1/2!(C)$);
	\draw (A)--(B)--(C)--(D)--(A)--(C) (M)--(N);
	\draw[dashed] (B)--(D);
	\foreach \t/\g in {A/90,B/180,C/-90,D/0,M/180,N/0} \draw (\t) node[shift={(\g:10pt)}]{$\t$};
	\end{tikzpicture}}
	\loigiai{
	\immini{Xét tam giác $ABC$ có $M$, $N$ là trung điểm của $AB$, $AC$ nên $MN$ là đường trung bình của tam giác $ABC$. Do đó $MN \parallel BC$.\\
	Ta có $ \left(\vec{MN}, \vec{BD} \right)= \left(\vec{BC}, \vec{BD} \right) = \widehat{CBD}$. \\
	Vì $ABCD$ là tứ diện đều nên $BC=CD=DB$. Do đó tam giác $BCD$ đều suy ra $\widehat{CBD} = 60^\circ$.\\
	Vậy $ \left(\vec{MN}, \vec{BD} \right) = 60^\circ$.}
	{\begin{tikzpicture}[scale=0.55, font=\footnotesize, line join=round, line cap=round]
	\foreach \x\y\t in {0/0/B,6/0/D,1.5/-2/C,1.5/5/A}
	\coordinate (\t) at (\x,\y);
	\coordinate (M) at ($(A)!1/2!(B)$);
	\coordinate (N) at ($(A)!1/2!(C)$);
	\draw (A)--(B)--(C)--(D)--(A)--(C);
	\draw[dashed,-stealth,blue,very thick] (B)--(D);
	\draw[-stealth,blue,very thick](M)--(N);
	\foreach \t/\g in {A/90,B/180,C/-90,D/0,M/180,N/0} \draw (\t) node[shift={(\g:10pt)}]{$\t$};
	\end{tikzpicture}}
	}
\end{ex} 
%%==========Câu 3
\begin{ex}
	\immini{Cho hình chóp $S. A B C D$ có đáy $A B C D$ là hình bình hành và mặt bên $S A B$ là tam giác đều. Tính góc giữa hai vectơ $\vec{D C}$ và $\vec{B S}$.
	\haicot
	{\True $\left(\vec{D C}, \vec{B S}\right)=120^{\circ}$}
	{$\left(\vec{D C}, \vec{B S}\right)=60^{\circ}$}
	{$\left(\vec{D C}, \vec{B S}\right)=90^{\circ}$}
	{$\left(\vec{D C}, \vec{B S}\right)=150^{\circ}$}
	}{
	\begin{tikzpicture}[scale=0.8, font=\footnotesize,>=stealth]
	%Gán số liệu.
	\def\canhAD{4};\def\canhBA{2};\def\gocBAD{-130};\def\h{2};\def\xdinhS{-1};
	%Gán tọa độ.
	\coordinate (A) at (0,0);
	\coordinate (B) at ($(A)+(\gocBAD:\canhBA)$);
	\coordinate (C) at ($(B)+(0:\canhAD)$);
	\coordinate (D) at ($(A)+(0:\canhAD)$);
	\coordinate (S) at ($(A)+(\xdinhS,\h)$);
	\draw (B)--(S)--(C)--cycle (S)--(D)--(C);
	\draw[dashed] (A)--(D) (S)--(A)--(B);
	\foreach \x/\y in {A/180,B/-90,C/-90,D/0,S/90}{\fill (\x) circle(1pt) ($(\x)+(\y:0.3cm)$) node{$\x$};}
	\end{tikzpicture}
	}
	\loigiai{
	\immini{Vì $A B C D$ là hình bình hành nên $A B \parallel D C$.\\
	Trên tia $A B$ lấy điểm $E$ sao cho $\vec{B E}=\vec{D C}$ (Hình $2.20$). Ta có
	$$
	\left(\vec{D C}, \vec{B S}\right)=\left(\vec{B E}, \vec{B S}\right)=\widehat{E B S}=180^{\circ}-60^{\circ}=120^{\circ}.
	$$
	Vậy $\left(\vec{D C}, \vec{B S}\right)=120^{\circ}$.
	}{
	\begin{tikzpicture}[scale=0.8, font=\footnotesize,>=stealth]
	%Gán số liệu.
	\def\canhAD{4};\def\canhBA{2};\def\gocBAD{-130};\def\h{2};\def\xdinhS{-1};
	%Gán tọa độ.
	\coordinate (A) at (0,0);
	\coordinate (B) at ($(A)+(\gocBAD:\canhBA)$);
	\coordinate (C) at ($(B)+(0:\canhAD)$);
	\coordinate (D) at ($(A)+(0:\canhAD)$);
	\coordinate (S) at ($(A)+(\xdinhS,\h)$);
	\coordinate (E) at ($(B)!-1!(A)$);
	%Vẽ khối chóp S.ABCD.
	\draw (B)--(S)--(C)--cycle (S)--(D)--(C);
	\draw[dashed] (A)--(D) (S)--(A)--(B);
	\draw[->] (B)--(E);
	\draw[->] (D)--(C);
	%	%Gán nhãn.
	\draw pic["$60^\circ$",draw,angle eccentricity=1.6,angle radius=0.5cm]{angle=A--B--S};
	\draw pic["$120^\circ$",draw,double,angle eccentricity=1.7,angle radius=0.4cm]{angle=S--B--E};
	\foreach \x/\y in {A/180,B/-90,C/-90,D/0,S/90, E/90}{\fill (\x) circle(1pt) ($(\x)+(\y:0.3cm)$) node{$\x$};}
	\end{tikzpicture}
	}
	}
\end{ex} 
%%==========Câu 4
\begin{ex}
	\immini{Cho hình chóp $S.A B C D$ có đáy $A B C D$ là hình bình hành. Mặt bên $A S B$ là tam giác vuông cân tại $S$ và có cạnh $A B=a$. Tính $\vec{D C} \cdot \vec{A S}$.
	\haicot
	{$\dfrac{a^2}{4}$}
	{$-\dfrac{a^2}{4}$}
	{$-\dfrac{a^2}{2}$}
	{\True $\dfrac{a^2}{2}$}}{
	\begin{tikzpicture}[scale=0.79, font=\footnotesize,>=stealth]
	\def\canhAD{4};\def\canhBA{2};\def\gocBAD{-130};\def\h{2};\def\xdinhS{-1};
	%Gán tọa độ.
	\coordinate (A) at (0,0);
	\coordinate (B) at ($(A)+(\gocBAD:\canhBA)$);
	\coordinate (C) at ($(B)+(0:\canhAD)$);
	\coordinate (D) at ($(A)+(0:\canhAD)$);
	\coordinate (S) at ($(A)+(\xdinhS,\h)$);
	\draw (B)--(S)--(C)--cycle (S)--(D)--(C);
	\draw[dashed] (A)--(D) (S)--(A)--(B);
	\draw[->] (D)--(C);
	%\draw pic["$45^\circ$",draw,angle eccentricity=1.6,angle radius=0.3cm]{angle=S--A--B};
	\foreach \x/\y in {A/60,B/-90,C/-90,D/0,S/90}{\fill (\x) circle(1pt) ($(\x)+(\y:0.3cm)$) node{$\x$};}
	\end{tikzpicture}}
	\loigiai{
	\immini{$\vec{D C} \cdot \vec{A S}=\vec{A B} \cdot \vec{A S}=\left|\vec{A B}\right| \cdot\left|\vec{A S}\right| \cdot \cos \left(\vec{A B}, \vec{A S}\right)=a \cdot \dfrac{a \sqrt{2}}{2} \cdot \cos 45^{\circ}=\dfrac{a^2}{2}$.
	}{
	\begin{tikzpicture}[scale=0.9, font=\footnotesize,>=stealth]
	\def\canhAD{4};\def\canhBA{2};\def\gocBAD{-130};\def\h{2};\def\xdinhS{-1};
	%Gán tọa độ.
	\coordinate (A) at (0,0);
	\coordinate (B) at ($(A)+(\gocBAD:\canhBA)$);
	\coordinate (C) at ($(B)+(0:\canhAD)$);
	\coordinate (D) at ($(A)+(0:\canhAD)$);
	\coordinate (S) at ($(A)+(\xdinhS,\h)$);
	\draw (B)--(S)--(C)--cycle (S)--(D)--(C);
	\draw[dashed] (A)--(D) (S)--(A)--(B);
	\draw[->] (D)--(C);
	\draw pic["$45^\circ$",draw,angle eccentricity=1.6,angle radius=0.3cm]{angle=S--A--B};
	\foreach \x/\y in {A/60,B/-90,C/-90,D/0,S/90}{\fill (\x) circle(1pt) ($(\x)+(\y:0.3cm)$) node{$\x$};}
	\end{tikzpicture}}
	}
\end{ex} 
%%==========Câu 5
\begin{ex}
	\immini{Cho hình lập phương $ABCD.EFGH$ có các cạnh bằng $ a $. Tính $\vec{AB}\cdot\vec{EG}$.
	\haicot
	{$a^2\sqrt{2}$}
	{\True $a^2$}
	{$\dfrac{a^2\sqrt{2}}{2}$}
	{$a^2\sqrt{3}$}}{
	\begin{tikzpicture}[scale=0.5, font=\footnotesize, line join=round, line cap=round, >=stealth]
	\coordinate (A) at (0,0);
	\coordinate (B) at (-1.5,-1.5);
	\coordinate (D) at (3,0);
	\coordinate (A') at (0,3);
	\coordinate (C) at ($(D)+(B)-(A)$);
	\coordinate (B') at ($ (A')-(A)+(B) $);
	\coordinate (C') at ($ (A')-(A)+(C) $);
	\coordinate (D') at ($ (A')-(A)+(D) $);
	\draw (B)--(C)--(C')--(B')--cycle (A')--(B')--(C')--(D')--cycle (C)--(D)--(D') (C)--(D);
	\draw[dashed] (B)--(A)--(D) (A)--(A');
	\draw[dashed] (A)--(C);
	\draw[](A')--(C');
	\foreach \p/\t/\q in {A/A/150,B/B/-90,C/C/-90,D/D/30, A'/E/90, B'/F/180, C'/G/-30, D'/H/90} \draw[black,fill=white] (\p) circle(0.8pt)node[shift={(\q:6pt)}]{\color{black}$\t$};
	\end{tikzpicture}}
	\loigiai{
	\immini
	{
	Ta có $\vec{AB}\cdot\vec{EG}=\vec{AB}\cdot\vec{AC}=AB\cdot AC\cdot\cos45^\circ=a\cdot a\sqrt{2}\cdot\dfrac{\sqrt2}{2}=a^2$.
	}
	{
	\begin{tikzpicture}[scale=0.5, font=\footnotesize, line join=round, line cap=round, >=stealth]
	\coordinate (A) at (0,0);
	\coordinate (B) at (-1.5,-1.5);
	\coordinate (D) at (3,0);
	\coordinate (A') at (0,3);
	\coordinate (C) at ($(D)+(B)-(A)$);
	\coordinate (B') at ($ (A')-(A)+(B) $);
	\coordinate (C') at ($ (A')-(A)+(C) $);
	\coordinate (D') at ($ (A')-(A)+(D) $);
	\draw (B)--(C)--(C')--(B')--cycle (A')--(B')--(C')--(D')--cycle (C)--(D)--(D') (C)--(D);
	\draw[dashed] (B)--(A)--(D) (A)--(A');
	\draw[->,dashed] (A)--(C);
	\draw[->](A')--(C');
	\foreach \p/\t/\q in {A/A/150,B/B/-90,C/C/-90,D/D/30, A'/E/90, B'/F/180, C'/G/-30, D'/H/90} \draw[black,fill=white] (\p) circle(0.8pt)node[shift={(\q:6pt)}]{\color{black}$\t$};
	\end{tikzpicture}
	}
	}
\end{ex} 
%%==========Câu 6
\begin{ex}
	\immini{Cho hình lập phương $ABCD.A'B'C'D'$ có cạnh bằng $a$. Tính $ \vec{A B'} \cdot \vec{A' C'}$.
	\haicot
	{$\dfrac{a^2}{2}$}
	{$-a^2$}
	{\True $a^2$}
	{$-\dfrac{a^2}{2}$}
	}{\hspace{1.5cm}
	\begin{tikzpicture}[scale=0.7, font=\footnotesize,>=stealth]
	%Gán số liệu.
	\def\canhAD{3};\def\canhBA{2};\def\gocBAD{-130};\def\h{3};\def\xdinhA'{0};
	%Gán tọa độ.
	\coordinate (A) at (0,0);
	\coordinate (B) at ($(A)+(\gocBAD:\canhBA)$);
	\coordinate (C) at ($(B)+(0:\canhAD)$);
	\coordinate (D) at ($(A)+(0:\canhAD)$);
	\coordinate (A') at ($(A)+(\xdinhA',\h)$);
	\coordinate (B') at ($(B)+(\xdinhA',\h)$);
	\coordinate (C') at ($(C)+(\xdinhA',\h)$);
	\coordinate (D') at ($(D)+(\xdinhA',\h)$);
	%\coordinate (F) at ($(A)!-1!(E)$);
	%\draw pic[draw,angle radius=0.3cm]{right angle=B'--H--E};
	%Vẽ khối lẳng trụ ABCD.A'B'C'D'.
	\draw (A')--(B')--(B)--(C)--(C')--(D')--cycle (B')--(C') (D')--(D)--(C) (A')--(C') ;
	\draw[dashed] (A)--(D) (A')--(A)--(B) (A)--(C) (A)--(B') (B)--(D) ;
	%Gán nhãn.
	\foreach \x/\y in {A/180, B/-90, C/0, D/0, A'/180, B'/180, C'/0, D'/0}{\fill (\x) circle(1pt) ($(\x)+(\y:0.3cm)$) node{$\x$};}
	\end{tikzpicture}}
	\loigiai{
	Ta có $A'C'=AC$.\\
	Vì $AB'=AC=B'C=a\sqrt{2}$ nên tam giác $AB'C$ đều. Suy ra $\widehat{B'AC}=60^\circ$.\\
	Ta có $\begin{aligned}[t]
	\vec{A B'} \cdot \vec{A' C'} & \ =\left|\vec{AB'}\right|\cdot\left|\vec{A'C'}\right|\cdot\cos \left(\vec{AB'},\vec{A'C'}\right) \\
	 & \ = AB'\cdot A'C' \cdot \cos \left(\vec{AB'}, \vec{AC}\right) \\
	 & \ = AB'\cdot A'C' \cdot \cos \widehat{B'AC} \\
	 & \ = a\sqrt{2}\cdot a\sqrt{2}\cdot \cos 60^\circ= a^2.
	\end{aligned}$
	}
\end{ex} 
%%==========Câu 7
\begin{ex}
	\immini{Cho hình lập phương $ABCD.A'B'C'D'$ có cạnh bằng $a$. Tính $\vec{A B'} \cdot \vec{B D} $.
	\haicot
	{$\dfrac{a^2}{2}$}
	{$-a^2$}
	{\True $a^2$}
	{$-\dfrac{a^2}{2}$}
	}{\hspace{1.5cm}
	\begin{tikzpicture}[scale=0.7, font=\footnotesize,>=stealth]
	%Gán số liệu.
	\def\canhAD{3};\def\canhBA{2};\def\gocBAD{-130};\def\h{3};\def\xdinhA'{0};
	%Gán tọa độ.
	\coordinate (A) at (0,0);
	\coordinate (B) at ($(A)+(\gocBAD:\canhBA)$);
	\coordinate (C) at ($(B)+(0:\canhAD)$);
	\coordinate (D) at ($(A)+(0:\canhAD)$);
	\coordinate (A') at ($(A)+(\xdinhA',\h)$);
	\coordinate (B') at ($(B)+(\xdinhA',\h)$);
	\coordinate (C') at ($(C)+(\xdinhA',\h)$);
	\coordinate (D') at ($(D)+(\xdinhA',\h)$);
	%\coordinate (F) at ($(A)!-1!(E)$);
	%\draw pic[draw,angle radius=0.3cm]{right angle=B'--H--E};
	%Vẽ khối lẳng trụ ABCD.A'B'C'D'.
	\draw (A')--(B')--(B)--(C)--(C')--(D')--cycle (B')--(C') (D')--(D)--(C) (A')--(C') ;
	\draw[dashed] (A)--(D) (A')--(A)--(B) (A)--(C) (A)--(B') (B)--(D) ;
	%Gán nhãn.
	\foreach \x/\y in {A/180, B/-90, C/0, D/0, A'/180, B'/180, C'/0, D'/0}{\fill (\x) circle(1pt) ($(\x)+(\y:0.3cm)$) node{$\x$};}
	\end{tikzpicture}}
	\loigiai{
	Ta có $ABCD.A'B'C'D$ là hình lập phương nên $\heva{&AA'\perp AB\\ &AB\perp BC\\ &\vec{AD}=\vec{BC}\\
	&\vec{AB'}=\vec{AA'}+\vec{AB}\\ &\vec{BD}=\vec{BA}+\vec{BC}.}$\\
	Khi đó $\begin{aligned}[t]
	\vec{A B'} \cdot \vec{B D} & \ =\left(\vec{AA'}+\vec{AB}\right)\cdot \left(\vec{BA}+\vec{BC}\right) \\
	 & \ =\vec{AA'}\cdot \vec{BA} +\vec{AA'}\cdot\vec{BC}+\vec{AB}\cdot\vec{BA}+\vec{AB}\cdot \vec{BC}. \\
	 & \ = 0 + 0 - AB^2 + 0 =-a^2.
	\end{aligned}$
	}
\end{ex} 
%%==========Câu 8
\begin{ex}
	\immini{Cho hình chóp tứ giác đều $S . A B C D$ có độ dài tất cả các cạnh bằng $a$. Tính $\vec{A S} \cdot \vec{B C}$.
	\haicot
	{$-\dfrac{a^2}{4}$}
	{\True $\dfrac{a^2}{2}$}
	{$-\dfrac{a^2}{2}$}
	{$\dfrac{a^2}{4}$}
	}{
	\begin{tikzpicture}[line join=round, line cap = round, >=stealth, scale=.6,font=\footnotesize]
	\def\a{4}
	\def\h{3}
	\path 	(0:0) coordinate (A)
	++(0:\a) coordinate (D)
	++(-150:\a/2) coordinate (C)
	($(A)+(C)-(D)$) coordinate (B)
	(intersection of A--C and B--D) coordinate (O)
	($(O)+(90:\h)$) coordinate (S);
	\draw[dashed] 	(A)--(D)
	(B)--(D)	(S)--(O)	;
	\draw	(C)--(D)
	(B)--(S)	(C)--(S)	(D)--(S);
	\foreach \x/\g in {A/135,B/-135,C/-45,D/45,S/90,O/-90}
	\fill[black] 	(\x) circle (1.5pt)
	($(\g:3mm)+(\x)$) node {$\x$};
	\draw[] (B)--(C);
	\draw[dashed] (A)--(S) (A)--(C) (A)--(B);
	\end{tikzpicture}	}
	\loigiai{
	Tam giác $S A D$ có ba cạnh bằng nhau nên là tam giác đều, suy ra $\widehat{S A D}=60^{\circ}$.\\
	Tứ giác $A B C D$ là hình vuông nên $\vec{A D}=\vec{B C}$, suy ra $(\vec{A S}, \vec{B C})=(\vec{A S}, \vec{A D})=\widehat{S A D}=60^{\circ}$.\\
	Do đó $\vec{A S} \cdot \vec{B C}=|\vec{A S}| \cdot|\vec{B C}| \cdot \cos 60^{\circ}=a \cdot a \cdot \dfrac{1}{2}=\dfrac{a^2}{2}$.
	}
\end{ex} 
%%==========Câu 9
\begin{ex}
	\immini{Cho hình chóp tứ giác đều $S . A B C D$ có độ dài tất cả các cạnh bằng $a$. Tính $\vec{A S} \cdot \vec{A C}$.
	\haicot
	{$-a^2$}
	{$\dfrac{a^2}{2}$}
	{$-\dfrac{a^2}{2}$}
	{\True $a^2$}
	}{
	\begin{tikzpicture}[line join=round, line cap = round, >=stealth, scale=.6,font=\footnotesize]
	\def\a{4}
	\def\h{3}
	\path 	(0:0) coordinate (A)
	++(0:\a) coordinate (D)
	++(-150:\a/2) coordinate (C)
	($(A)+(C)-(D)$) coordinate (B)
	(intersection of A--C and B--D) coordinate (O)
	($(O)+(90:\h)$) coordinate (S);
	\draw[dashed] 	(A)--(D)
	(B)--(D)	(S)--(O)	;
	\draw	(C)--(D)
	(B)--(S)	(C)--(S)	(D)--(S);
	\foreach \x/\g in {A/135,B/-135,C/-45,D/45,S/90,O/-90}
	\fill[black] 	(\x) circle (1.5pt)
	($(\g:3mm)+(\x)$) node {$\x$};
	\draw[] (B)--(C);
	\draw[dashed] (A)--(S) (A)--(C) (A)--(B);
	\end{tikzpicture}	}
	\loigiai{
	Tứ giác $A B C D$ là hình vuông có độ dài mỗi cạnh là a nên độ dài đường chéo $A C$ là $\sqrt{2} a$.\\
	Tam giác $S A C$ có $S A=S C=a$ và $A C=\sqrt{2} a$ nên tam giác $S A C$ vuông cân tại $S$, suy ra $\widehat{S A C}=45^{\circ}$.\\
	Do đó $\vec{A S} \cdot \vec{A C}=|\vec{A S}| \cdot|\vec{A C}| \cdot \cos \widehat{S A C}=a \cdot \sqrt{2} a \cdot \dfrac{\sqrt{2}}{2}=a^2$.
	}
\end{ex} 
%%==========Câu 10
\begin{ex}
	\immini{Cho tứ diện $ABCD$ biết $AB=AD=BD=a$, $AC=2a$ và $\widehat{CAD}=120^{\circ}$. Tính $\vec{BC}\cdot \vec{AD}$.
	\haicot
	{\True $-\dfrac{3}{2}a^2$}
	{$\dfrac{3}{2} a^2$}
	{$\dfrac{1}{2} a^2$}
	{$-\dfrac{1}{2} a^2$}}{
	\begin{tikzpicture}[scale=0.65, font=\footnotesize,>=stealth]
	\path
	(0,0) coordinate (B)
	(5,0) coordinate (C)
	(1.5,-1.5) coordinate (D)
	(1,3) coordinate (A)
	;
	\draw (B)--(A)node[midway,sloped,scale=0.7]{$||$}--(D)node[midway,sloped,scale=0.7]{$||$}--(C)--(A) (B)--(D)node[midway,sloped,scale=0.7]{$||$};
	\draw[dashed](B)--(C);
	\foreach \x/\g in {B/180,A/90,C/0,D/-90}\draw[fill=black] (\x) circle (.05) +(\g:.5)node{\footnotesize$\x$};
	\draw pic["$120^\circ$",draw,angle eccentricity=1.6,angle radius=0.5cm]{angle=D--A--C};
	\end{tikzpicture}}
	\loigiai{
	Theo giả thiết tam giác $ABD$ là tam giác đều. Ta có
	\begin{eqnarray*}
	\vec{BC}\cdot\vec{AD}&=&\left(\vec{AC}-\vec{AB}\right)\cdot \vec{AD}\\&=&\vec{AC}\cdot\vec{AD}-\vec{AB}\cdot\vec{AD}\\
	&=&AC \cdot AD \cdot \cos 120^{\circ}-AB\cdot AD\cdot \cos 60^{\circ}\\
	&=&\dfrac{-3}{2}a^2.
	\end{eqnarray*}
	}
\end{ex} 
%%==========Câu 11
\begin{ex}
	\immini{ Cho hình chóp $S.A B C$ có $S A=S B=S C=A B=A C=a$ và $B C=a \sqrt{2}$. Tính góc giữa các vectơ $\vec{S C}$ và $\vec{A B}$.
	\haicot
	{$60^{\circ}$}
	{$90^{\circ}$}
	{\True $120^{\circ}$}
	{$150^{\circ}$}}{
	\begin{tikzpicture}[scale=0.6, font=\footnotesize,>=stealth]
	\path
	(0,0) coordinate (A)
	(5,0) coordinate (C)
	(1.5,-1.5) coordinate (B)
	(1,3) coordinate (S)
	;
	\draw (S)--(A)node[midway,sloped,scale=0.7]{$||$}--(B)node[midway,sloped,scale=0.7]{$||$}--(C)--(S)node[midway,sloped,scale=0.7]{$||$} (S)--(B)node[midway,sloped,scale=0.7]{$||$};
	\draw[dashed](A)--(C)node[midway,sloped,scale=0.7]{$||$};
	\foreach \x/\g in {A/180,S/90,C/0,B/-90}\draw[fill=black] (\x) circle (.05) +(\g:.5)node{\footnotesize$\x$};
	\end{tikzpicture}}
	\loigiai{
	\immini{Ta có
	\begin{align*}
	\cos \left(\vec{S C}, \vec{A B}\right) & \ =\dfrac{\vec{S C} \cdot \vec{A B}}{\left|\vec{S C}\right| \cdot\left|\vec{A B}\right|}=\dfrac{\left(\vec{S A}+\vec{A C}\right) \cdot \vec{A B}}{a^2} \\
	 & \ =\dfrac{\vec{S A} \cdot \vec{A B}+\vec{A C} \cdot \vec{A B}}{a^2}.
	\end{align*}
	Từ giả thiết suy ra $S A B$ là tam giác đều và $A B C$ là tam giác vuông cân tại $A$. Từ đó ta tính được
	$\vec{S A} \cdot \vec{A B}=a\cdot a \cdot \cos 120^{\circ}=-\dfrac{a^2}{2}$ và $\vec{A C} \cdot \vec{A B}=0$.\\
	Suy ra $\cos \left(\vec{S C}, \vec{A B}\right)=-\dfrac{1}{2}$.\\
	Vậy $\cos \left(\vec{S C}, \vec{A B}\right)=120^{\circ}$.
	}{
	\begin{tikzpicture}[scale=0.7, font=\footnotesize,>=stealth]
	\path
	(0,0) coordinate (A)
	(5,0) coordinate (C)
	(1.5,-1.5) coordinate (B)
	(1,3) coordinate (S)
	;
	\draw (S)--(A)node[midway,sloped,scale=0.7]{$||$}--(B)node[midway,sloped,scale=0.7]{$||$}--(C)--(S)node[midway,sloped,scale=0.7]{$||$} (S)--(B)node[midway,sloped,scale=0.7]{$||$};
	\draw[dashed](A)--(C)node[midway,sloped,scale=0.7]{$||$};
	\foreach \x/\g in {A/180,S/90,C/0,B/-90}\draw[fill=black] (\x) circle (.05) +(\g:.5)node{\footnotesize$\x$};
	\draw pic[draw,angle radius=0.3cm]{right angle=B--S--C};
	\end{tikzpicture}}
	}
\end{ex} 
%%==========Câu 12
\begin{ex}
	\immini{Cho tứ diện $OABC$ có các cạnh $OA$, $OB$, $OC$ đôi một vuông góc và $OA=OB=OC=1$. Gọi $M$ là trung điểm của cạnh $AB$. Tính góc giữa hai vectơ $\vec{OM}$ và $\vec{AC}$.
	\haicot
	{$90^\circ$}
	{\True $120^\circ$}
	{$60^\circ$}
	{$30^\circ$}}{
	\begin{tikzpicture}[scale=0.4, font=\footnotesize, line join=round, line cap=round]
	\foreach \x\y\t in {0/0/O,3/-3.6/A,7/2/B,0.3/5/C} \coordinate (\t) at (\x,\y);
	\draw (O)--(C)--(B)--(A)--(O);
	\coordinate (M) at ($(A)!1/2!(B)$);
	\draw[-stealth,thick] (A)--(C);
	\draw[-stealth,dashed,thick] (O)--(M);
	\draw[dashed] (O)--(B);
	\foreach \t/\g in {A/-90,B/0,C/90,O/180,M/0} \draw (\t) node[shift={(\g:10pt)}]{$\t$};
	\end{tikzpicture}}
	\loigiai{
	\immini{Đặt $\vec{OA}=\vec{a}$, $\vec{OB}=\vec{b}$, $\vec{OC}=\vec{c}$. \\
	Khi đó, $\left| \vec{a} \right| = \left| \vec{b} \right| = \left| \vec{c} \right| = 1$ và $\vec{a} \cdot \vec{b} = \vec{a} \cdot \vec{c} = \vec{b} \cdot \vec{c} = 0$.\\
	Ta có $\cos \left(\vec{OM},\vec{AC} \right) = \dfrac{\vec{OM} \cdot \vec{AC}}{\left| \vec{OM} \right| \cdot \left| \vec{AC} \right|}$.\\
	Mặt khác do $\vec{OM} = \dfrac{1}{2} \left(\vec{OA}+\vec{OB} \right) = \dfrac{1}{2} \left(\vec{a}+\vec{b} \right)$\\
	và $\vec{AC} = \vec{OC} - \vec{OA} = \vec{c} - \vec{a}$\\
	nên $\begin{aligned}[t]
	\vec{OM} \cdot \vec{AC} & =\dfrac{1}{2} \left(\vec{a}+\vec{b} \right) \cdot \left(\vec{c}-\vec{a} \right) \\
	 & =\dfrac{1}{2} \left(\vec{a} \cdot \vec{c}-\vec{a}^2+ \vec{b} \cdot \vec{c} - \vec{b} \cdot \vec{a} \right) = -\dfrac{1}{2}. \\
	\end{aligned}$
	}
	{\begin{tikzpicture}[scale=0.45, font=\footnotesize, line join=round, line cap=round]
	\foreach \x\y\t in {0/0/O,3/-3.6/A,7/2/B,0.3/5/C} \coordinate (\t) at (\x,\y);
	\draw (O)--(C)--(B)--(A)--(O);
	\coordinate (M) at ($(A)!1/2!(B)$);
	\draw[-stealth,red,very thick] (A)--(C);
	\draw[-stealth,dashed,red,very thick] (O)--(M);
	\draw[dashed] (O)--(B);
	\foreach \t/\g in {A/-90,B/0,C/90,O/180,M/0} \draw (\t) node[shift={(\g:10pt)}]{$\t$};
	\end{tikzpicture}}
	Ta lại có $\left| \vec{OM} \right| = OM =\dfrac{\sqrt{2}}{2}$, $\left| \vec{AC} \right| = AC = \sqrt{2}$. \\
	Do đó $\cos \left(\vec{OM},\vec{AC} \right) = \dfrac{\vec{OM} \cdot \vec{AC}}{\left| \vec{OM} \right| \cdot \left| \vec{AC} \right|} = \dfrac{\dfrac{-1}{2}}{\dfrac{\sqrt{2}}{2} \cdot \sqrt{2}} = \dfrac{-1}{2}$. \\
	Vậy $\left( \vec{OM}, \vec{AC}\right) = 120^\circ$.
	}
\end{ex} 
%%==========Câu 13
\begin{ex}
	Cho hình lập phương $ABCD.A'B'C'D'$ cạnh bằng $a$. Tích vô hướng của hai vectơ $\vec{DD'}$ và $\vec{A'C'}$ bằng
	\choice
	{$\sqrt{2}a^2$}
	{$a^2$}
	{$-\sqrt{2}a^2$}
	{\True $0$}
	\loigiai{
	Ta có: $\vec{A'C'}=\vec{A'D'}+\vec{D'C'}$, mà tứ giác $ADD'A'$ và $DCC'D'$ là hình vuông nên $\vec{DD'} \cdot \vec{A'D'}=\vec{DD'} \cdot \vec{D'C'}=0$. Do đó $\vec{DD'} \cdot \left(\vec{A'D'}+\vec{D'C'}\right)=0$.
	}
\end{ex}
\Closesolutionfile{ans}
\textbf{PHẦN II.} \textit{Câu trắc nghiệm đúng sai. Trong mỗi ý a), b), c), d) ở mỗi câu, học sinh chọn đúng hoặc sai.}\\
\Opensolutionfile{ans}[ans/2H2-B1-d2-2]
%%==========Câu 14
\begin{ex}%[2H2H1-3]
	Trong không gian, cho hai véc-tơ $\vec{a}$ và $\vec{b}$ cùng có độ dài bằng $1$. Biết rằng góc giữa hai véc-tơ đó là $45^{\circ}$.
	\choiceTF
	{\True $\vec{a}\cdot \vec{b}=\dfrac{\sqrt{2}}{2}$}
	{\True $\left( \vec{a}+3 \vec{b}\right) \cdot\left( \vec{a}-2 \vec{b}\right)=-5+\dfrac{\sqrt{2}}{2}$}
	{$\left| \vec{a}+ \vec{b}\right|=2+\sqrt{2} $}
	{$\left| \vec{a}-\sqrt{2}\vec{b}\right|=0$}
	\loigiai{
	\begin{enumerate}[a)]
	\item $\vec{a}\cdot \vec{b}=\left| \vec{a}\right|\cdot \left| \vec{b}\right|\cos \left(\vec{a},\vec{b} \right)=\dfrac{\sqrt{2}}{2}$.
	\item $\left( \vec{a}+3 \vec{b}\right) \cdot\left( \vec{a}-2 \vec{b}\right)=\left| \vec{a}\right|^2+\cdot\vec{a}\cdot \vec{b}-6\left| \vec{b}\right|^2 =1+\cdot\dfrac{\sqrt{2}}{2}-6=-5+\dfrac{\sqrt{2}}{2} $.
	\item $\left( \vec{a}+ \vec{b}\right)^2= \vec{a}^2+2\vec{a}\cdot \vec{b}+\vec{b}^2=1+2\cdot\dfrac{\sqrt{2}}{2}+1=2+\sqrt{2}$. Suy ra $\left| \vec{a}+ \vec{b}\right|=\sqrt{2+\sqrt{2}}$.
	\item $\left( \vec{a}-\sqrt{2} \vec{b}\right)^2= \vec{a}^2+2\sqrt{2}\vec{a}\cdot \vec{b}+2\vec{b}^2=1+2\sqrt{2}\cdot\dfrac{\sqrt{2}}{2}+2=2$. Suy ra $\left| \vec{a}- \sqrt{2}\vec{b}\right|=\sqrt{2}$.
	\end{enumerate}
	}
\end{ex} 
%%==========Câu 15
\begin{ex}%[2H2H1-3]
	\immini{Cho tứ diện đều $ABCD$ có cạnh bằng $a$ và $M$ là trung điểm của $CD$.
	\choiceTF
	{\True $\vec{AM} \cdot \vec{CD}=0$}
	{\True $\vec{AB} \cdot \vec{AC}=\dfrac{a^2}{2}$}
	{\True $\vec{AB}\cdot\vec{CD}=0$}
	{$\vec{AM}\cdot\vec{AB} =-\dfrac{a^2}{2}$}
	}{
	\begin{tikzpicture}[scale=1, font=\footnotesize, line join=round, line cap=round, >=Stealth]
	\def\a{3}
	\path
	(0:0) coordinate (B)
	(5:\a) coordinate (D)
	($(D)+(-135:\a/2)$) coordinate (C)
	($(B)+(65:\a)$) coordinate (A)
	($(C)!.5!(D)$) coordinate (M)
	;
	\draw[->] (A)--(D);
	\draw[->] (A)--(C);
	\draw[->] (A)--(B);
	\draw[->] (A)--(M);
	\draw[dashed] (B)--(D);
	\draw (B)--(C)--(D);
	\foreach \x/\g in {A/90,B/180,C/-90,D/0,M/0}
	\draw[fill=black] 	(\x) circle (.5pt)
	($(\g:.3)+(\x)$) node {$\x$};
	\end{tikzpicture}}
	\loigiai{
	\begin{enumerate}[a)]
	\item Tam giác $ACD$ đều, suy ra $AM$ vuông góc với $CD$ nên $\vec{AM}\cdot \vec{CD}=0$.
	\item Ta có $\begin{aligned}[t]
	 \vec{AB} \cdot \vec{AC} & = |\vec{AB}| \cdot |\vec{AC}| \cdot \cos (\vec{AB},\vec{AC}) \\
	 & = AB \cdot AC \cdot \cos \widehat{BAC} \\
	 & = a \cdot a \cdot \cos {60^0} \\
	 & = \dfrac{a^2}{2}.
	 \end{aligned}$\\
	\item Ta có $\vec{AB} \cdot \vec{CD}=(\vec{AM}+\vec{MB}) \cdot \vec{CD}=\vec{AM} \cdot \vec{CD} +\vec{MB} \cdot \vec{CD}$.\\
	 Mà $AM$, $BM$ là trung tuyến của các tam giác đều $ACD$, $BCD$ nên $\vec{AM} \perp \vec{CD}, \vec{MB} \perp \vec{CD}$.\\
	 Suy ra $\vec{AM} \cdot \vec{CD} = \vec{MB} \cdot \vec{CD} = 0$.\\
	 Từ các kết quả trên ta có $\vec{AM} \cdot \vec{CD}=0$.
	 Suy ra $(\vec{AB}, \vec{CD})=90^\circ$.
	\item Ta có $\vec{AM} = \dfrac{1}{2}(\vec{AC} + \vec{AD})$, suy ra
	 \[ \vec{AB} \cdot \vec{AM} = \vec{AB} \cdot \frac{1}{2}(\vec{AC} +\vec{AD}) = \frac{1}{2}(\vec{AB} \cdot \vec{AC} + \vec{AB} \cdot \vec{AD}) = \frac{1}{2}\left(\frac{a^2}{2} + \frac{a^2}{2}\right) = \dfrac{a^2}{2}.\]
	\end{enumerate}
	}
\end{ex} 
%%==========Câu 16
\begin{ex}
	\immini{
	Một chất điểm ở vị trí đỉnh $A$ của hình lập phương $ABCD.A'B'C'D'$. Chất điểm chịu tác động bởi ba lực $\vec{a}$, $\vec{b}$, $\vec{c}$ lần lượt cùng hướng với $\vec{AD}$, $\vec{AB}$ và $\vec{AC'}$ như hình vẽ. Độ lớn của các lực $\vec{a}$, $\vec{b}$ và $\vec{c}$ tương ứng là $10$ N, $10$ N và $20$ N.
	\choiceTF
	{$\vec{a}+\vec{b}=\vec{c}$}
	{$\big|\vec{a}+\vec{b}\big|=20$ (N)}
	{\True $\big|\vec{a}+\vec{c}\big|=\big|\vec{b}+\vec{c}\big|$}
	{\True $\big|\vec{a}+\vec{b}+\vec{c}\big|=32{,}59$ (N) (\textit{làm tròn kết quả đến hàng phần mười})}
	}{\hspace{0.5cm}
	\begin{tikzpicture}[line join=round, line cap = round, >=stealth, scale=0.65,font=\footnotesize]
	\path
	(0,0) coordinate (A')
	(-1.5,-1.5) coordinate (D')
	(2,-1.5) coordinate (C')
	(3.5,0) coordinate (B')
	(0,3.5) coordinate (A)
	($(A)+(B')-(A')$) coordinate (B)
	($(A)+(C')-(A')$) coordinate (C)
	($(A)+(D')-(A')$) coordinate (D)
	($(A)!1/2!(D)$) coordinate (M)
	($(A)!1/2!(B)$) coordinate (N)
	($(A)!1/2!(C')$) coordinate (P)
	;
	\draw[->,thick](A)--node [left]{$\vec{a}$}(M);
	\draw[->,thick](A)--node [above]{$\vec{b}$}(N);
	\draw[->,thick](A)--node[right]{$\vec{c}$}(P);
	\draw[dashed] (D')--(A')--(B') (A')--(A)--(C');
	\draw (D)--(C)--(B)--(A)--(D)--(D')--(C')--(C) (C')--(B')--(B);
	\draw pic[draw,angle eccentricity=1.2,angle radius=0.25cm]{right angle= D--A--B};
	\foreach \x/\g in {A/90,B/80,C/-40,D/110,A'/160,B'/-65,C'/-90,D'/-100} \fill[black](\x) circle (1pt) ($(\x)+(\g:3mm)$)node{$\x$};
	\end{tikzpicture}
	}
	\loigiai{
	Từ giả thiết, ta có $\vec{a} \perp \vec{b};\cos\left(\vec{a},\vec{c}\right)= \cos \widehat{DAC'}= \dfrac{1}{\sqrt3}; \cos \left(\vec{b},\vec{c}\right)= \cos \widehat{BAC'}= \dfrac{1}{\sqrt3}$.\\
	\begin{enumerate}[a)]
	\item Giả sử $\vec{a}+\vec{b}=\vec{d}$. Theo quy tắc hình bình hành thì $\vec{d}$ cùng hướng với $\vec{AC}$. Suy ra $\vec{a}+\vec{b}\ne \vec{c}$
	\item $\big|\vec{a}+\vec{b}\big|=10\sqrt{2}$ (đường chéo hình vuông cạnh bằng 10).
	\item Ta có
	 \begin{itemize}
	 \item [$\bullet$] $\big(\vec{a}+\vec{c}\big)^2=|\vec{a}|^2+2\vec{a} \cdot \vec{c}+|\vec{c}|^2=10^2+2.10.20.\dfrac{1}{\sqrt{3}}+20^2=500+\dfrac{400\sqrt{3}}{3}$.\\
	 Suy ra $\big|\vec{a}+\vec{c}\big|=\sqrt{500+\dfrac{400\sqrt{3}}{3}}$.
	 \item [$\bullet$] $\big(\vec{b}+\vec{c}\big)^2=|\vec{b}|^2+2\vec{b} \cdot \vec{c}+|\vec{c}|^2=10^2+2.10.20.\dfrac{1}{\sqrt{3}}+20^2=500+\dfrac{400\sqrt{3}}{3}$.\\
	 Suy ra $\big|\vec{b}+\vec{c}\big|=\sqrt{500+\dfrac{400\sqrt{3}}{3}}$.
	 \end{itemize}
	 Vậy $\big|\vec{a}+\vec{c}\big|=\big|\vec{b}+\vec{c}\big|$.
	\item Giả sử lực tổng hợp là $\vec{m}$, tức là $\vec{m}=\vec{a}+\vec{b}+\vec{c}$.\\
	 Do đó
	 \allowdisplaybreaks
	 \begin{eqnarray*}
	 \vec{m}=\vec{a}+\vec{b}+\vec{c} &\Leftrightarrow& \left|\vec{m}\right|^2= \left(\vec{a}+\vec{b}+\vec{c}\right)^2\\
	 &\Leftrightarrow& \left|\vec{m}\right|^2= \vec{a}^2+\vec{b}^2+\vec{c}^2+2\vec{a}\cdot\vec{b}+2\vec{b}\cdot\vec{c}+2\vec{c}\cdot\vec{a}\\
	 &\Leftrightarrow& \left|\vec{m}\right|^2= 10^2+10^2+20^2+0+2\cdot 10\cdot 20 \cdot \dfrac{1}{\sqrt3} +2\cdot 10\cdot 20 \cdot \dfrac{1}{\sqrt3}\\
	 &\Leftrightarrow& \left|\vec{m}\right|^2= 10^2+10^2+20^2+0+2\cdot 10\cdot 20 \cdot \dfrac{1}{\sqrt3} +2\cdot 10\cdot 20 \cdot \dfrac{1}{\sqrt3}\\
	 &\Leftrightarrow& \left|\vec{m}\right| \approx 32{,}59.
	 \end{eqnarray*}
	 Vậy cường độ hợp lực của $\vec{a}$, $\vec{b}$ và $\vec{c}$ là $\approx 32{,}59$ (N).
	\end{enumerate}
	}
\end{ex} 
%%==========Câu 17
\begin{ex}
	Cho hình chóp $S.ABCD$ có đáy $ABCD$ là hình chữ nhật. Biết rằng cạnh $AB=a$, $AD=2a$, cạnh bên $SA=2a$ và vuông góc với mặt đáy. Gọi $M$, $N$ lần lượt là trung điểm của các cạnh $SB$, $SD$. Các mệnh đề sau đúng hay sai ?
	\choiceTF
	{Hai vectơ $\vec{AB}$, $\vec{CD}$ là hai vectơ cùng phương, cùng hướng}
	{Góc giữa hai vectơ $\vec{SC}$ và $\vec{AC}$ bằng $60^\circ $}
	{\True Tích vô hướng $\vec{AM} \cdot \vec{AB}=\dfrac{a^2}{2}$}
	{Độ dài của vectơ $\vec{AM}-\vec{AN}$ là $\dfrac{a\sqrt{3}}{2}$}
	\loigiai{
	\begin{center}
	\begin{tikzpicture}[line join = round, line cap = round, thick, font = \small, scale = .7]
	\path
	(0:0) coordinate (A)
	+(0:5) coordinate (B)
	+(-150:2.5) coordinate (D)
	+(90:5) coordinate (S)
	($(B)+(D)-(A)$) coordinate (C)
	($(S)!.5!(B)$) coordinate (M)
	($(S)!.5!(D)$) coordinate (N)
	;
	\draw[dashed]
	(D)--(A)--(B) (C)--(A)--(S) (M)--(A)--(N)
	;
	\draw
	(D)--(C)--(B)
	(S)--(B) (S)--(C) (S)--(D)
	\foreach \x/\y/\z in {S/A/B,S/A/D}{
	pic[draw, angle radius = 6pt]{right angle = \x--\y--\z}
	}
	;
	\foreach \x/\g in {A/135,B/0,C/-45,D/-135,S/90,M/45,N/135}
	\fill (\x) circle (1.5pt)
	+(\g:3.5mm) node {$\x$};
	\end{tikzpicture}
	\end{center}
	\begin{enumerate}[a)]
	\item $\vec{AB}=-\vec{CD}$. Suy ra hai vectơ $\vec{AB}$, $\vec{CD}$ là hai vectơ ngược hướng.
	\item Ta có: $ABCD$ là hình chữ nhật nên: $AC=\sqrt{AB^2+AD^2}=a\sqrt{5}$.\\
	 Hình chóp $S.ABCD$ có $SA$ vuông góc với mặt đáy nên tam giác $SAC$ là tam giác vuông tại $A$. Suy ra: $\tan \widehat{SCA}=\dfrac{SA}{AC}=\dfrac{2a}{a\sqrt{5}}\Rightarrow \widehat{SCA}\approx 41^\circ 48'$.\\
	 Ta có: $\left(\vec{SC},\vec{AC}\right)=\left(\vec{CS},\vec{CA}\right)=\widehat{SCA}\approx 41^\circ 48'$.
	\item Hình chóp $S.ABCD$ có $SA$ vuông góc với mặt đáy nên tam giác $SAB$ là tam giác vuông tại $A$.\\
	 Suy ra: $SB=\sqrt{SA^2+AB^2}=a\sqrt{5}$.\\
	 Trong tam giác $SAB$ vuông tại $A$ có $AM$ là đường trung tuyến nên: \\
	 $AM=\dfrac{1}{2}SB=\dfrac{a\sqrt{5}}{2}$.\\
	 Lại có: $M$ là trung điểm của $SB$ nên $MB=\dfrac{1}{2}SB=\dfrac{a\sqrt{5}}{2}$. \\
	 Ta tính được: $\cos \widehat{MAB}=\dfrac{MA^2+AB^2-MB^2}{2MA \cdot AB}=\dfrac{\sqrt{5}}{5}$.\\
	 Mà: $\left(\vec{AM},\vec{AB}\right)=\widehat{MAB}$, suy ra: \\
	 $\vec{AM}\cdot \vec{AB}=\left| \vec{AM} \right| \cdot \left| \vec{AB} \right| \cdot \cos \left(\vec{AM},\vec{AB}\right)=\dfrac{a\sqrt{5}}{2} \cdot a \cdot \dfrac{\sqrt{5}}{5}=\dfrac{a^2}{2}$.
	\item Ta có: $M$, $N$ lần lượt là trung điểm của các cạnh $SB$, $SD$ nên $MN$ là đường trung bình của tam giác $SBD$.
	 Do đó: $MN=\dfrac{1}{2}BD=\sqrt{AB^2+AD^2}=\dfrac{a\sqrt{5}}{2}$.\\
	 Suy ra: $\left| \vec{AM}-\vec{AN} \right| =\left| \vec{MN} \right| =\dfrac{a\sqrt{5}}{2}$.
	\end{enumerate}
	}
\end{ex}
%%==========Câu 18
\begin{ex}
	Cho hình lập phương $ABCD.A'B'C'D'$ có cạnh bằng $a$. Trên các cạnh $AA'$, $CC'$ lần lượt lấy các điểm $M$, $N$ sao cho $AM=\dfrac{2}{3}AA'$, $CN=NC'$. Các mệnh đề sau đúng hay sai?
	\choiceTF
	{Góc giữa hai vectơ $\vec{AN}$ và $\vec{AC}$ bằng $60^\circ $}
	{\True Độ dài của vectơ $\vec{MN}+\vec{AM}$ là $\dfrac{3a}{2}$}
	{Tích vô hướng $\vec{AN}\cdot \vec{AC}=a^2$}
	{\True Tích vô hướng $\vec{MN}\cdot \vec{A'C'}=2a^2$}
	\loigiai{
	\begin{center}
	\begin{tikzpicture}[line join = round, line cap = round, thick, font = \small, scale = .7]
	\def \canh{4}
	\path
	(0:0) coordinate (D')
	+(90:\canh) coordinate (D)
	+(0:\canh) coordinate (C')
	+(40:.4*\canh) coordinate (A')
	($(C')+(D)-(D')$) coordinate (C)
	($(D)+(A')-(D')$) coordinate (A)
	($(C')+(A')-(D')$) coordinate (B')
	($(C)+(A)-(D)$) coordinate (B)
	($(A)!2/3!(A')$) coordinate (M)
	($(C)!.5!(C')$) coordinate (N)
	($(C)!2/3!(C')$) coordinate (M')
	;
	\draw[dashed]
	(A')--(A) (A')--(B') (A')--(D') (M')--(M)--(N)
	;
	\draw
	(A)--(B)--(B')--(C')--(D')--(D)--cycle
	(C)--(B) (C)--(D) (C)--(C')
	;
	\foreach \x/\g in {D'/-90,C'/-90,D/180,A'/135,C/-45,A/90,B'/0,B/90,M/180,N/0,M'/0}
	\fill (\x) circle (1.5pt)
	+(\g:3mm) node {$\x$};
	\end{tikzpicture}
	\end{center}
	\begin{enumerate}[a)]
	\item Ta có: $AC=\sqrt{AB^2+AC^2}=a\sqrt{2}$.\\
	 Lại có: $CN=NC'$ nên $CN=NC'=\dfrac{a}{2}$.\\
	 $ABCD.A'B'C'D'$ là hình lập phương nên tam giác $NAC$ là tam giác vuông tại $C$.\\
	 Suy ra: $\tan NAC=\dfrac{CN}{AC}=\dfrac{\sqrt{2}}{4}\Rightarrow \widehat{NAC}\approx 19^\circ 28'$\\
	 Ta có: $\left(\vec{AN},\vec{AC}\right)=\widehat{NAC}\approx 19^\circ 28'$.
	\item Trong tam giác $NAC$ vuông tại $C$ có: $AN=\sqrt{AC^2+CN^2}=\dfrac{3a}{2}$.\\
	 Ta có: $\left| \vec{MN}+\vec{AM} \right| =\left| \vec{AN} \right| =\dfrac{3a}{2}$.
	\item Ta có: $\tan \widehat{NAC}=\dfrac{\sqrt{2}}{4}\Rightarrow \cos \widehat{NAC}=\dfrac{2\sqrt{2}}{3}$ (Do $\widehat{NAC}<90^\circ $).\\
	 Do đó: $\vec{AN}\cdot \vec{AC}=\left| \vec{AN} \right| \cdot \left| \vec{AC} \right| \cdot \cos \left(\vec{AN},\vec{AC}\right)=\dfrac{3a}{2} \cdot a\sqrt{2} \cdot \dfrac{2\sqrt{2}}{3}=2a^2$.
	\item Trên cạnh $CC'$ lấy điểm $M'$ sao cho: $\dfrac{CM'}{CC'}=\dfrac{2}{3}$.\\
	 Suy ra: $\heva{& NM'=NC'-M'C'=\dfrac{a}{6} \\ & MM'\parallel AC \\ & MM'=AC=a\sqrt{2}} $.\\
	 Ta có: $\cos \widehat{NMM'}=\dfrac{NM^2+M'M^2-M'N^2}{2 \cdot NM \cdot M'M}=\dfrac{6\sqrt{146}}{73}$.\\
	 Mặt khác: $\left(\vec{MN},\vec{A'C'}\right)=\left(\vec{MN},\vec{MM'}\right)=\widehat{NMM'}$.\\
	 Tam giác $MNM'$ vuông tại $M'$ có: $MN=\sqrt{M'N^2+M'M^2}=\dfrac{a\sqrt{73}}{6}$.\\
	 Do đó: $\vec{MN}\cdot \vec{A'C'}=\left| \vec{MN} \right| \cdot \left| \vec{A'C'} \right| \cdot \cos \left(\vec{MN},\vec{A'C'}\right)=2a^2$.
	\end{enumerate}
	}
\end{ex}
%%==========Câu 19
\begin{ex}
	Cho hình lăng trụ đứng $ABC.A'B'C'$ đáy là tam giác đều cạnh $2a,AA'=a\sqrt{3}$. $H$, $K$ lần lượt là trung điểm $BC$, $B'C'$. Các mệnh đề sau đúng hay sai?
	\choiceTF
	{Hai vectơ $\vec{AH}$, $\vec{KA'}$ là hai vectơ cùng phương, cùng hướng}
	{Góc giữa hai vectơ $\vec{A'H}$ và $\vec{AH}$ bằng $60^\circ $}
	{Tích vô hướng $\vec{AK}\cdot \vec{AB'}=\dfrac{5a^2}{2}$}
	{\True Độ dài của vectơ $\vec{AK}+\vec{AH}$ là $\dfrac{a\sqrt{3}}{2}$}
	\loigiai{
	\begin{center}
	\begin{tikzpicture}[line join = round, line cap = round, thick, font = \small, scale = .7]
	\path
	(0:0) coordinate (A)
	+(0:4) coordinate (C)
	+(-50:2) coordinate (B)
	+(90:4) coordinate (A')
	($(A')+(B)-(A)$) coordinate (B')
	($(A')+(C)-(A)$) coordinate (C')
	($(B)!.5!(C)$) coordinate (H)
	($(B')!.5!(C')$) coordinate (K)
	($(H)!.5!(K)$) coordinate (I)
	;
	\draw[dashed]
	(A)--(C) (H)--(A)--(K) (A)--(I)
	;
	\draw
	(A)--(B)--(C)--(C')--(A')--cycle
	(B')--(A') (B')--(B) (B')--(C') (A')--(K)--(H)
	;
	\foreach \x/\g in {A/180,B/-90,C/0,A'/-180,B'/-150,C'/0,H/-45,K/-30,I/0}
	\fill (\x) circle (1.5pt)
	+(\g:3mm) node {$\x$};
	\end{tikzpicture}
	\end{center}
	\begin{enumerate}[a)]
	\item Ta có tam giác $\triangle ABC,\triangle A'B'C'$ đều cạnh $2a$ suy ra $A'K=AH=a\sqrt{3}$\\
	 Xét tứ giác $AA'KH$ có $AA'=KH=AH=A'K=a\sqrt{3}$, $AA'\perp AH$ suy ra tứ giác $AA'KH$ là hình vuông , từ đó dễ thấy hai vectơ $\vec{AH}$, $\vec{KA'}$ là hai vecto cùng phương ngược hướng.
	\item Ta có: $AA'KH$ là hình vuông suy ra $\widehat{A'HA}=45^\circ $\\
	 Có $A'A\perp AH\Rightarrow \triangle A'AH$ vuông tại $A\Rightarrow \left(\vec{A'H},\vec{AH}\right)=\widehat{A'HA}=45^\circ $.
	\item Ta có $\triangle AB'C'$ cân tại $A$, suy ra $AK\perp B'C'$, $AK=a\sqrt{6},B'K=a$\\
	 $AB'=\sqrt{AB^2+BB'^2}=\sqrt{4a^2+3a^2}=a\sqrt{7}$\\
	 Xét $\triangle AKB'$ có $\cos \widehat{KAB'}=\dfrac{AK}{AB'}=\dfrac{a\sqrt{6}}{a\sqrt{7}}=\sqrt{\dfrac{6}{7}}$.\\
	 $\vec{AK} \cdot \vec{AB'}=AK \cdot AB'\cdot \cos \widehat{KAB'}=a\sqrt{6} \cdot a\sqrt{7} \cdot \sqrt{\dfrac{6}{7}}=6a^2$.
	\item Gọi $I$ là trung điểm $HK\Rightarrow IH=\dfrac{a\sqrt{3}}{2}$, $AI=\sqrt{IH^2+AH^2}=\sqrt{\dfrac{3a^2}{4}+3a^2}=\dfrac{a\sqrt{15}}{2}$.\\
	 Ta có $\left| \vec{AK}+\vec{AH} \right| =\left| 2 \cdot \vec{AI} \right| =2AI=a\sqrt{15}$.
	\end{enumerate}
	}
\end{ex}
%%==========Câu 20
\begin{ex}
	Cho tứ diện đều $ABCD$ cạnh $a$. $E$ là điểm trên đoạn $CD$ sao cho $ED=2CE$. Các mệnh đề sau đúng hay sai?
	\choiceTF
	{Có $6$ vectơ (khác vectơ $\vec{0}$) có điểm đầu và điểm cuối được tạo thành từ các đỉnh của tứ diện}
	{Góc giữa hai vectơ $\vec{AB}$ và $\vec{BC}$ bằng $60^\circ $}
	{Nếu $\vec{BE}=m\vec{BA}+n\vec{BC}+p\vec{BD}$ thì $m+n+p=\dfrac{2}{3}$}
	{\True Tích vô hướng $\vec{AD} \cdot \vec{BE}=\dfrac{a^2}{6}$}
	\loigiai{
	\begin{center}
	\begin{tikzpicture}[line join = round, line cap = round, thick, font = \small, scale = .7]
	\path
	(0:0) coordinate (B)
	+(0:5) coordinate (C)
	+(-70:2) coordinate (D)
	+(70:4) coordinate (A)
	($(C)!1/3!(D)$) coordinate (E)
	;
	\draw[dashed]
	(B)--(C) (B)--(E)
	;
	\draw
	(A)--(B)--(D)--(C)--cycle
	(E)--(A)--(D)
	;
	\foreach \x/\g in {B/180,C/0,D/-90,A/90,E/-45}
	\fill (\x) circle (1.5pt)
	+(\g:3mm) node {$\x$};
	\end{tikzpicture}
	\end{center}
	\begin{enumerate}[a)]
	\item Số vectơ (khác $\vec{0}$) có điểm đầu và điểm cuối được tạo thành từ các đỉnh của tứ diện là $A_4^2=12$.
	\item $(\vec{AB},\vec{BC})=180^\circ-(\vec{BA},\vec{BC})={{180}^\circ}-\widehat{ABC}=120^\circ$.
	\item $\vec{BE}=\vec{BC}+\vec{CE}=\vec{BC}+\dfrac{1}{3}\vec{CD}=\vec{BC}+\dfrac{1}{3}\left(\vec{BD}-\vec{BC}\right)=\dfrac{2}{3}\vec{BC}+\dfrac{1}{3}\vec{BD}$.\\
	 Do đó $m=0$,$n=\dfrac{2}{3}$,$p=\dfrac{1}{3}$. Suy ra $m+n+p=1$.
	\item Ta có: $\vec{BE}=\vec{AE}-\vec{AB}=\left(\vec{AC}+\vec{CE}\right)-\vec{AB}=\vec{AC}+\dfrac{1}{3}\vec{CD}-\vec{AB}$\\
	 $=\vec{AC}+\dfrac{1}{3}\left(\vec{AD}-\vec{AC}\right)-\vec{AB}=\dfrac{2}{3}\vec{AC}+\dfrac{1}{3}\vec{AD}-\vec{AB}$\\
	 Suy ra: $\vec{AD} \cdot \vec{BE}=\vec{AD} \cdot \left(\dfrac{2}{3}\vec{AC}+\dfrac{1}{3}\vec{AD}-\vec{AB}\right)=\dfrac{2}{3} \cdot \vec{AD} \cdot \vec{AC}+\dfrac{1}{3} \cdot {{\vec{AD}}^2}-\vec{AD} \cdot \vec{AB}$\\
	 $=\dfrac{2}{3} \cdot a \cdot a \cdot \cos 60^\circ +\dfrac{1}{3}a^2-a \cdot a \cdot \cos 60^\circ =\dfrac{a^2}{6}$.
	\end{enumerate}
	}
\end{ex}
%%==========Câu 21
\begin{ex}
	Cho tứ diện $ABCD$ có cạnh $a$. Gọi $M$, $N$ lần lượt là trung điểm của $AB$, $CD$. Các mệnh đề sau đúng hay sai?
	\choiceTF
	{$\vec{AB}$ và $\vec{CD}$ cùng hướng}
	{\True $\vec{EA}+\vec{EB}+\vec{EC}+\vec{ED}=\vec{0}$ với $E$ là trung điểm $MN$}
	{\True $\vec{AB} \cdot \vec{CD}+\vec{AC} \cdot \vec{DB}+\vec{AD} \cdot \vec{BC}=\vec{0}$}
	{\True Điểm $I$ xác định bởi $P=3\vec{IA}^2+\vec{IB}^2+\vec{IC}^2+\vec{ID}^2$ có giá trị nhỏ nhất. Khi đó giá trị nhỏ nhất của $P$ là $2a^2$}
	\loigiai{
	\begin{center}
	\begin{tikzpicture}[line join = round, line cap = round, thick, font = \small, scale = 1]
	\path 
	(0:0) coordinate (B)
	+(0:5) coordinate (C)
	+(-45:2) coordinate (D)
	+(70:4) coordinate (A)
	($(A)!.5!(B)$) coordinate (M)
	($(C)!.5!(D)$) coordinate (N)
	($(M)!.5!(N)$) coordinate (E)
	($(B)!2/3!(N)$) coordinate (G)
	($(A)!.5!(G)$) coordinate (O)
	;
	\draw[dashed] 
	(B)--(C) (A)--(G) (M)--(N)
	;
	\draw 
	(A)--(B)--(D)--(C)--cycle
	(A)--(D)
	;
	\foreach \x/\g in {B/180,C/0,D/-90,G/0,A/90,M/135,N/-45,E/0,O/0}
	\fill (\x) circle (1.5pt)
	+(\g:3mm) node {$\x$};
	\end{tikzpicture}
	\end{center}
	\begin{enumerate}[a)]
	\item $\vec{AB}$ và $\vec{CD}$ ngược hướng.
	\item Vì $M$ là trung điểm $AB$ nên $\vec{EA}+\vec{EB}=2\vec{EM}$, $N$ là trung điểm $CD$ nên $\vec{EC}+\vec{ED}=2\vec{EN}$.\\
	Ta có $\vec{EA}+\vec{EB}+\vec{EC}+\vec{ED}=2\left(\vec{EM}+\vec{EN}\right)=\vec{0}$.
	\item $\begin{aligned}[t]
	&\vec{AB} \cdot \vec{CD}+\vec{AC} \cdot \vec{DB}+\vec{AD} \cdot \vec{BC}=\left(\vec{AC}+\vec{CB}\right) \cdot \vec{CD}+\vec{AC} \cdot \vec{DB}+\vec{AD} \cdot \vec{BC}\\
	 = & \vec{AC} \cdot \left(\vec{CD}+\vec{DB}\right)+\vec{AD} \cdot \vec{BC}+\vec{CB \cdot }\vec{CD}=\vec{AC} \cdot \vec{CB}+\vec{AD} \cdot \vec{BC}+\vec{CB \cdot }\vec{CD} \\
	 = &\vec{CB}\left(\vec{AC}-\vec{AD}\right)+\vec{CB \cdot }\vec{CD}=\vec{0} 
	\end{aligned}$
	\item Gọi $O$ là điểm thoả mãn hệ thức $3\vec{OA}+\vec{OB}+\vec{OC}+\vec{OD}=\vec{0}$ suy ra $O$ cố định vì $A$, $B,C$, $D$ cố định. Ta có
	\begin{align*}
	P& =3\vec{IA}^2+\vec{IB}^2+\vec{IC}^2+\vec{ID}^2 \\
	& =3\left(\vec{IO}+\vec{OA}\right)^2+\left(\vec{IO}+\vec{OB}\right)^2+\left(\vec{IO}+\vec{OC}\right)^2+\left(\vec{IO}+\vec{OD}\right)^2 \\
	& =6IO^2+3OA^2+OB^2+OC^2+OD^2+2\vec{IO}\left(3\vec{OA}+\vec{OB}+\vec{OC}+\vec{OD}\right) \\
	& =6IO^2+3OA^2+OB^2+OC^2+OD^2.
	\end{align*}
	Do đó để $P$ nhỏ nhất thì $I$ trùng với $O$. Gọi $G$ là trọng tâm tam giác $BCD$.\\
	Vì $3\vec{OA}+\vec{OB}+\vec{OC}+\vec{OD}=3\vec{OA}+\left(\vec{OB}+\vec{OC}+\vec{OD}\right) =3\vec{OA}+3\vec{OG}$ nên $\vec{OA}+\vec{OG}=\vec{0}$.\\	
	Suy ra $O$ là trung điểm của $AG$.\\
	Ta có $BG=\dfrac{2}{3} \cdot \dfrac{a\sqrt{3}}{2}=\dfrac{a}{\sqrt{3}}\Rightarrow AG=\sqrt{AB^2-BG^2}=\sqrt{a^2-{{\left(\dfrac{a}{\sqrt{3}}\right)}^2}}=\dfrac{a\sqrt{2}}{\sqrt{3}}$\\
	$\Rightarrow OA=\dfrac{1}{2}AG=\dfrac{a}{\sqrt{6}}\Rightarrow OA^2=\dfrac{a^2}{6}$.\\
	Lại có $OD^2=OC^2=OB^2=OG^2+BG^2=\dfrac{a^2}{6}+\dfrac{a^2}{3}=\dfrac{a^2}{2}$.\\
	Vậy giá trị nhỏ nhất là $P=3 \cdot \dfrac{a^2}{6}+3 \cdot \dfrac{a^2}{2}=2a^2$ khi $I$ trùng với $O$.
	\end{enumerate}
	}
\end{ex}
\textbf{PHẦN III.} \textit{Câu trắc nghiệm trả lời ngắn.}\\
%%==========Câu 22
\begin{ex}
	Cho tứ diện đều $ABCD$ có cạnh bằng $4$. Giá trị tích vô hướng $\vec{AB}\left(\vec{AB}-\vec{CA}\right)$ bằng
	\loigiai{
	\SA{24}
	$\begin{aligned}[t]
	\vec{AB}\left(\vec{AB}-\vec{CA}\right)
	 & =\vec{AB} \cdot \vec{AB}+\vec{AB} \cdot \vec{AC}={{\vec{AB}}^2}+| \vec{AB}| \cdot | \vec{AC}| \cdot \cos \left(\vec{AB},\vec{AC}\right) \\
	 & =AB^2+AB \cdot AC \cdot \cos \left(\widehat{BAC}\right)=4^2+4 \cdot 4 \cdot \cos 60^\circ=4^2+\dfrac{4^2}{2}=\dfrac{{{3 \cdot 4}^2}}{2}=24.
	\end{aligned}$
	}
\end{ex}
%%==========Câu 23
\begin{ex}
	Trong không gian, cho hai vectơ $\vec{a}$ và $\vec{b}$ có cùng độ dài bằng $6$. Biết độ dài của vectơ $\vec{a}+2\vec{b}$ bằng $6\sqrt{3}$. Biết số đo góc giữa hai vectơ $\vec{a}$ và $\vec{b}$ là $x$ độ. Giá trị của $x$ là bao nhiêu?
	\loigiai{
	\SA{120}
	$\vec{a} \cdot \vec{b} = \dfrac{1}{4} \left[\left(\vec{a}+2 \vec{b}\right)^2 - \vec{a}^2 - 4\vec{b}^2\right]
	= \dfrac{1}{4} \left[\left|\vec{a}+2 \vec{b}\right|^2 - |\vec{a}|^2 - 4|\vec{b}|^2\right]
	= \dfrac{1}{4} \left[\left(6\sqrt{3}\right)^2 - 6^2 - 4\cdot 6^2\right] = -18$.\\
	Lại có $\vec{a} \cdot \vec{b}=| \vec{a}| \cdot | \vec{b}| \cdot \cos \left(\vec{a}\,,\vec{b}\right)\Leftrightarrow \cos \left(\vec{a}\,,\vec{b}\right)=\dfrac{\vec{a} \cdot \vec{b}}{| \vec{a}| \cdot | \vec{b}|}=\dfrac{-18}{6 \cdot 6}=\dfrac{-1}{2}\Leftrightarrow \left(\vec{a}\,,\vec{b}\right)=120^\circ $.\\
	Khi đó góc giữa hai vectơ $\vec{a}$ và $\vec{b}$ là $120^\circ $.
	}
\end{ex}
%%==========Câu 24
\begin{ex}
	Cho hình lập phương $ABCD.A'B'C'D'$ có cạnh bằng $2$. Tính $\vec{AB} \cdot \vec{A'C'}$.
	\loigiai{
	\SA{4}
	Ta có: $\left(\vec{AB},\vec{A'C'}\right)=\left(\vec{AB},\vec{AC}\right)=45^\circ $.\\
	Khi đó: $\vec{AB} \cdot \vec{A'C'}=AB \cdot A'C'\cdot \cos \left(\vec{AB},\vec{A'C'}\right)=2 \cdot 2\sqrt{2} \cdot \cos 45^\circ =4$.
	}
\end{ex}
%%==========Câu 25
\begin{ex}
	Cho tứ diện $ABCD$, gọi $M$, $N$ lần lượt là trung điểm của $BC$ và $AD$, biết $AB=a$, $CD=a$, $MN=\dfrac{a\sqrt{3}}{2}$. Tìm số đo (đơn vị độ) góc giữa hai đường thẳng $AB$ và $CD$.
	\loigiai{
	\SA{60}
	\begin{center}
	\begin{tikzpicture}[line join = round, line cap = round, thick, font = \small, scale = .7]
	\path
	(0:0) coordinate (B)
	+(0:5) coordinate (D)
	+(-30:4) coordinate (C)
	+(70:4) coordinate (A)
	($(B)!1/2!(C)$) coordinate (M)
	($(A)!1/2!(D)$) coordinate (N)
	($(A)!1/2!(C)$) coordinate (I)
	;
	\draw[dashed]
	(B)--(D) (M)--(N)
	;
	\draw
	(A)--(B)--(C)--(D)--cycle
	(A)--(C) (M)--(I)--(N)
	;
	\foreach \x/\g in {B/180,D/0,C/-90,A/90,M/-135,I/180,N/45}
	\fill (\x) circle (1.5pt)
	+(\g:3mm) node {$\x$};
	\end{tikzpicture}
	\end{center}
	Gọi $I$ là trung điểm của $AC$.\\
	Ta có $\heva{& IM \parallel AB \\& IN \parallel CD}\Rightarrow \widehat{\left(AB,CD\right)}=\widehat{\left(IM,IN\right)}$.\\
	Đặt $\widehat{MIN}=\alpha $. Xét tam giác $IMN$, có: $IM=\dfrac{AB}{2}=\dfrac{a}{2}$, $IN=\dfrac{CD}{2}=\dfrac{a}{2}$, $MN=\dfrac{a\sqrt{3}}{2}$.\\
	Theo định lý cosin, có $\cos \alpha =\dfrac{IM^2+IN^2-MN^2}{2 \cdot IM \cdot IN}=-\dfrac{1}{2}<0$.\\
	$\Rightarrow \widehat{MIN}=120^\circ \Rightarrow \widehat{\left(AB,CD\right)}=60^\circ $.
	}
\end{ex}
%%==========Câu 26
\begin{ex}
	Cho hình lập phương $ABCD.A'B'C'D'$. Góc giữa hai vectơ $\vec{A'B}$ và $\vec{AC'}$ bằng
	\loigiai{
	\SA{90}
	$\vec{A'B}=\vec{A'A}+\vec{AB}=\vec{AB}-\vec{AA'}$.\\
	$\vec{AC'}=\vec{AB}+\vec{AD}+\vec{AA'}$.\\
	$\Rightarrow \vec{A'B} \cdot \vec{AC'} = \left(\vec{AB}-\vec{AA'}\right) \cdot \left(\vec{AB}+\vec{AD}+\vec{AA'}\right)={{\vec{AB}}^2}-{{\vec{AA'}}^2}=0$.\\
	$\Rightarrow$ Góc giữa hai vectơ $\vec{A'B}$ và $\vec{AC'}$ bằng $90^\circ$.
	}
\end{ex}
%%==========Câu 27
\begin{ex}
	Cho hình chóp $S.ABC$ có $SA$, $SB$, $SC$ đôi một vuông góc nhau và $SA=SB=SC=a$. Gọi $M$ là trung điểm của $AB$. Góc giữa hai vectơ $\vec{SM}$ và $\vec{BC}$ bằng
	\loigiai{
	\shortans{120}	
	Ta có $\cos \left(\vec{SM},\vec{BC}\right)=\dfrac{\vec{SM} \cdot \vec{BC}}{|\vec{SM}| \cdot |\vec{BC}|}=\dfrac{\vec{SM} \cdot \vec{BC}}{SM \cdot BC}$.\\
	\begin{align*}
	\vec{SM} \cdot \vec{BC} & =\dfrac{1}{2}\left(\vec{SA}+\vec{SB}\right) \cdot \left(\vec{SC}-\vec{SB}\right)\\
	& =\dfrac{1}{2}\left(\vec{SA} \cdot \vec{SC}-\vec{SA} \cdot \vec{SB}+\vec{SB} \cdot \vec{SC}-\vec{SB} \cdot \vec{SB}\right) \\
	& =-\dfrac{1}{2}\vec{SB} \cdot \vec{SB}=-\dfrac{1}{2}SB^2=-\dfrac{a^2}{2}.
	\end{align*}
	Tam giác $SAB$ và $SBC$ vuông cân tại $S$ nên $AB=BC=a\sqrt{2}$.\\
	$\Rightarrow SM=\dfrac{AB}{2}=\dfrac{a\sqrt{2}}{2}$.\\
	Do đó $\cos \left(\vec{SM},\vec{BC}\right)=\dfrac{-\dfrac{a^2}{2}}{\dfrac{a\sqrt{2}}{2} \cdot a\sqrt{2}}=-\dfrac{1}{2}$. Suy ra $\left(\vec{SM},\vec{BC}\right)={120}^\circ$.
	}
\end{ex}
\Closesolutionfile{ans}
%%Bài 2. Tọa độ vector trong không gian
% \setcounter{section}{1}
\section{TỌA ĐỘ CỦA VÉC TƠ TRONG KHÔNG GIAN}
\subsection{LÝ THUYẾT CẦN NHỚ}
\subsubsection{Hệ tọa độ trong không gian}
Trong không gian, ba trục $O x$, $O y$, $O z$ đôi một vuông góc với nhau tại gốc $O$ của mỗi trục. Gọi $\vec{i}$, $\vec{j}$, $\vec{k}$ lần lượt là các véc-tơ đơn vị trên các trục $O x$, $O y$, $O z$.
\immini{
	\begin{itemize}
		\item  Hệ ba trục như vậy được gọi là hệ trục toạ độ Descartes vuông góc $Oxyz$, hay đơn giản là hệ toạ độ $Oxyz$. Điểm $O$ được gọi là gốc toạ độ.
		\item  Các mặt phẳng $(O x y)$, $(O y z)$, $(O z x)$ đôi một vuông góc với nhau được gọi là các mặt phẳng toạ độ.
		\item  ${\vec{i}^2} = {\vec{j}^2} = {\vec{k}^2} = 1$ \\
		và $\vec{i} \cdot \vec{j} = \vec{j} \cdot \vec{k} = \vec{k} \cdot \vec{i}  = 0$
	\end{itemize}
}{\hspace{1cm}
	\begin{tikzpicture}[>=stealth,line join=round,line cap=round,scale=1]
		\def\a{3.0}
		\path
		(0,0) coordinate (A1)
		(\a,0) coordinate (A2)
		(\a,\a) coordinate (A3)
		(0,\a) coordinate (A4);
		\foreach \i in {1,...,4}
		\path (A\i)+(45:.75) coordinate (B\i);
		\draw (B1)--(A1) (B1)--(B2) (B1)--(B4);
		%	\draw(A4)--(B4)--(B3)--(B2)--(A2) (A3)--(B3)
		%	(A1)--(A2)--(A3)--(A4)--cycle;
		\draw[-stealth] (B1)--(B2)node[right]{$y$};
		\draw[-stealth] (B1)--(B4)node[above]{$z$};
		\draw[dashed](B1)--+(45:0.85)[dashed](B1)--+(180:0.85)(B1)--+(270:0.85);
		\draw[-stealth] (A1)--+(-135:.95)node[below]{$x$};
		\draw[-stealth,blue] (B1)--+(0:.95)node[above]{$\vec{j}$};
		\draw[-stealth,blue] (B1)--+(90:.85)node[above left]{$\vec{k}$};
		\draw[-stealth,blue] (B1)--+(-135:0.95)node[right]{$\vec{i}$};
		\fill(B1)circle(1pt) node[below right]{$O$};
	\end{tikzpicture}}
Không gian với hệ toạ độ $Oxyz$ còn được gọi là không gian $Oxyz$.
\subsubsection{Tọa độ của điểm}
Trong KG $Oxyz$, cho điểm $M$. Tọa độ điểm $M$ được xác định như sau:
\immini{
	\begin{itemize}
		\item Xác định hình chiếu $M_1$ của điểm $M$ trên mặt phẳng $Oxy$. Trong mặt phẳng tọa độ $Oxy$, tìm hoành độ $a$, tung độ $b$ của điểm $M_1$.
		\item Xác định hình chiếu $P$ của điểm $M$ trên trục cao $Oz$, điểm $P$ ứng với số $c$ trên trục $Oz$. Số $c$ là cao độ của điểm $M$.
	\end{itemize}
	Bộ số $(a;b;c)$ là toạ độ của điểm $M$ trong không gian với hệ toạ độ $Oxyz$, kí hiệu là $M(a;b;c)$.
}{
	\begin{tikzpicture}[scale=0.6, font=\small,>=stealth]
		\path
		(0,0) coordinate (O)
		(-2,-2) coordinate (H)
		(3,-2) coordinate (M_1)
		(5,0) coordinate (K)
		(3,1) coordinate (M)
		(0,3) coordinate (P)
		;
		\draw[->] (0,0)--(6.7,0) node[below]{$y$};
		\draw[->] (0,0)--(-3,-3) node[below]{$x$};
		\draw[->] (0,0)--(0,4.3) node[left]{$z$};
		\draw[dashed] (P)node[left]{$c$}--(M)--(M_1)--(H)node[left]{$a$} (O)--(M_1)--(K)node[above]{$b$} (O)--(M);
		\foreach \x/\g in {O/160,M_1/-90,M/30,H/-80,K/-70,P/30}\draw[fill=black] (\x) circle (.05) +(\g:.5)node{\small$\x$};
		\foreach \x/\y/\z in {M_1/H/O,M_1/K/O,M/P/O}{\path pic[draw,angle radius=5pt]{right angle= \x--\y--\z};}
	\end{tikzpicture}
}
\subsubsection{Tọa độ của vectơ}
Trong KG $Oxyz$:
\immini{
	\begin{itemize}
		\item Toạ độ của điểm $M$ cũng là toạ độ của vectơ $\overrightarrow{OM}$.
		\item Cho $\vec{u}$. Dựng điểm $M(a;b;c)$ thỏa $\vec{OM}=\vec{u}$ thì tọa độ của điểm $M$ là tọa độ của $\vec{u}$. Theo hình vẽ thì
		      $$\vec{u}=\vec{OM}=\vec{OH}+\vec{OK}+\vec{OP}=a\vec{i}+b\vec{j}+c\vec{k}.$$
		      Suy ra
		      $$\vec{u}=\left(a;b;c \right)\Leftrightarrow \vec{u}=a\vec{i}+b\vec{j}+c\vec{k}. $$
	\end{itemize}
}{
	\begin{tikzpicture}[scale=0.6, font=\small,>=stealth]
		\path
		(0,0) coordinate (O)
		(-2,-2) coordinate (H)
		(3,-2) coordinate (M_1)
		(5,0) coordinate (K)
		(3,1) coordinate (M)
		(0,3) coordinate (P)
		;
		\draw[->] (0,0)--(6.7,0) node[below]{$y$};
		\draw[->] (0,0)--(-3,-3) node[below]{$x$};
		\draw[->] (0,0)--(0,4.3) node[left]{$z$};
		\draw[-stealth,blue,thick] (O)--(-1,-1)node[above]{$\vec{i}$};
		\draw[-stealth,blue,thick](O)--(1,0)node[below right]{$\vec{j}$};
		\draw[-stealth,blue,thick] (O)--(0,1)node[above left]{$\vec{k}$};
		\draw[dashed] (P)node[left]{$c$}--(M)--(M_1)--(H)node[left]{$a$} (O)--(M_1)--(K)node[above]{$b$};
		\draw[thick,->](O)--(M)node[midway,sloped,above,scale=1]{$\vec{u}$};
		\foreach \x/\g in {O/160,M_1/-90,M/30,H/-80,K/-70,P/30}\draw[fill=black] (\x) circle (.05) +(\g:.5)node{\small$\x$};
		\foreach \x/\y/\z in {M_1/H/O,M_1/K/O,M/P/O}{\path pic[draw,angle radius=5pt]{right angle= \x--\y--\z};}
	\end{tikzpicture}}
\begin{note}
	Tọa độ các véc tơ đơn vị lần lượt là: $\vec{i}=(1;0;0)$,\quad $\vec{j}=(0;1;0)$,\quad $\vec{k}=(0;0;1)$.
\end{note}
\subsection{PHÂN LOẠI VÀ PHƯƠNG PHÁP GIẢI TOÁN}

\begin{dang}{Tọa độ điểm, tọa độ vec tơ}
	\indamm{Khi xác định tọa độ điểm, tọa độ véc tơ ta chú ý các kết quả sau:}
	\begin{enumerate}
		\item $\vec{u}=a\vec{i}+b\vec{j}+c\vec{k} \Leftrightarrow \vec{u}=\big(a;b;c\big)$.
		\item $\vec{u}\big(u_1;u_2;u_3\big)=\vec{v}\big(v_1;v_2;v_3\big) \Leftrightarrow \heva{&u_1=v_1\\&u_2=v_2\\&u_3=v_3}$
		\item $\vec{OM}=(a;b;c)$ thì $M\big(a;b;c\big)$.
		\item $\vec{AB}=\big(x_B-x_A;y_B-y_A;z_B-z_A \big).$
		\item Chiếu điểm $M(a;b;c)$ lên mặt phẳng tọa độ (hoặc trục tọa độ) thì "thành phần bị khuyết" bằng $0$. Chẳng hạn: $M(1;2;3)$ chiếu lên $(Oxy)$ thì $z=0$. Suy ra hình chiếu là $M_1(1;2;0)$.
		\item Tứ giác $ABCD$ là hình bình hành khi và chỉ khi $$\vec{AD}=\vec{BC}$$
	\end{enumerate}
\end{dang}
\BTTL
\begin{vd}
	Trong KG $Oxyz$, cho $A(3 ;-2 ;-1)$. Gọi $ A_1, A_2, A_3$ lần lượt là hình chiếu của điểm $A$ trên các mặt phẳng toạ độ $(Oxy),(Oyz),(Oxz)$. Tìm toạ độ của các điểm $ A_1, A_2, A_3$.
	\loigiai{
		Toạ độ của các điểm $ A_1=(3 ;-2 ;0)$.\\
		Toạ độ của các điểm $ A_2=(3 ;0 ;-1)$.\\
		Toạ độ của các điểm $ A_3=(0 ;-2 ;-1)$
	}
\end{vd}

\begin{vd}
	Trong KG $Oxyz$, cho $A(-2;3;4)$. Gọi $H, K, P$ lần lượt là hình chiếu của điểm $A$ trên các trục $Ox, Oy, Oz$. Tìm tọa độ của các điểm $H,K,P$.
	\loigiai{
		Tìm tọa độ của các điểm $H=(-2;0;0)$.\\
		Tìm tọa độ của các điểm $K=(0;3;0)$.\\
		Tìm tọa độ của các điểm $P=(0;0;4)$.
	}
\end{vd}

\begin{vd} Trong KG $Oxyz$, cho $A(1; 1;-2)$, $B(4; 3; 1)$ và $C(-1;-2; 2)$.
	\begin{tasks}
		\task Tìm tọa độ của véctơ $\overrightarrow{A B}$.
		\task Tìm toạ độ của điểm $D$ sao cho $ABCD$ là hình bình hành.
	\end{tasks}
	\loigiai{
		\begin{enumerate}
			\item Ta có $
				      \overrightarrow{AB}=(4-1; 3-1; 1-(-2))=(3; 2; 3) .
			      $
			\item Gọi tọa độ của điểm $D$ là $\left(x_D; y_D; z_D\right)$, ta có
			      $
				      \overrightarrow{DC}=\left(-1-x_D;-2-y_D; 2-z_D\right) .
			      $\\
			      Tứ giác $A B C D$ là hình bình hành khi và chỉ khi
			      $$
				      \overrightarrow{DC}=\overrightarrow{A B} \Leftrightarrow\heva{&
					      - 1 - x _ { D } = 3 \\&
					      - 2 - y _ { D } = 2 \\&
					      2 - z _ { D } = 3.}
				      \Leftrightarrow \heva{&
					      x_D=-4 \\&
					      y_D=-4 \\&
					      z_D=-1.}$$
			      Vậy $D(-4;-4;-1)$.
		\end{enumerate}}
\end{vd}

\begin{vd}
	Trong KG $Oxyz$, cho hình hộp $ABCD \cdot A'B'C'D'$ có $A(4;6;-5)$, $B(5;7;-4)$, $C(5;6;-4)$, $D'(2;0;2)$. Tìm tọa độ các đỉnh còn lại của hình hộp $ABCD\cdot A'B'C'D'$.
	\loigiai{
		\begin{center}
			\begin{tikzpicture}[scale=0.7, font=\small, line join=round, line cap=round, >=stealth]
				\def\bc{4} % cạnh BC
				\def\ba{3} % cạnh BA
				\def\gocB{35} % góc B của đáy
				\coordinate[label=below left:$B(5;7;-4)$] (B) at (0,0);
				\coordinate[label=above left:$A(4;6;-5)$] (A) at (\gocB:\ba);
				\coordinate[label=below:$C(5;6;-4)$] (C) at (\bc,0);
				\coordinate[label=right:$D$] (D) at ($(C)-(B)+(A)$);
				\coordinate[label=above left:$A'$] (A') at ($(A)+(90:\bc)$);
				\coordinate[label=left:$B'$] (B') at ($(B)-(A)+(A')$);
				\coordinate[label=below right:$C'$] (C') at ($(C)-(A)+(A')$);
				\coordinate[label=right:$D'(2;0;2)$] (D') at ($(D)-(A)+(A')$);
				\draw (B')--(B)--(C)--(D)--(D')--(A')--(B')--(C')--(D') (C)--(C');
				\draw[dashed] (A')--(A)--(D) (A)--(B);
				\foreach \diem in {A,B,C,D,A',B',C',D'}	\fill (\diem)circle(1.5pt);
			\end{tikzpicture}
		\end{center}
		Ta có  $\overrightarrow{AD}=\overrightarrow{BC}\Leftrightarrow \heva{x_D&=x_A-x_B+x_C\\y_D&=y_A-y_B+y_C\\z_D&=z_A-z_B+z_C}\Leftrightarrow \heva{x_D&=4\\y_D&=5\\z_D&=-5}$. Suy ra $D(4;5;-5)$.\\
		Do đó $\overrightarrow{DD'}=(2-4;0-5;2-(-5)) =(-2;-5;7)$.\\
		Theo tính chất của hình hộp ta có $\overrightarrow{AA'}=\overrightarrow{BB'}=\overrightarrow{CC'}=\overrightarrow{DD'}=(-2;-5;7)$. Suy ra tọa độ đỉnh còn lại của hình hộp là $A'=(2;1;2)$, $B'(3;2;3)$, $C'(3;1;3)$.
	}
\end{vd}

\BTTN
\setcounter{ex}{0}
\Opensolutionfile{ans}[ans/2H2-B2-d1-1]

\begin{ex}
	Trong KG $Oxyz$, cho $\overrightarrow{a}=-2\overrightarrow{i}+3\overrightarrow{j}+5\overrightarrow{k}$. Toạ độ của véc-tơ $\overrightarrow{a}$ là
	\choice
	{$(2;-3;-5)$}
	{$(2;3;-5)$}
	{\True $(-2;3;5)$}
	{$(2;3;5)$}
	\loigiai{
		Toạ độ của véc-tơ $\overrightarrow{a}$ là $(-2;3;5)$.}
\end{ex} 

\begin{ex}
	Trong KG $Oxyz$, cho véc-tơ $\overrightarrow{u}=3\overrightarrow{i}+4\overrightarrow{k}-\overrightarrow{j}$. Tọa độ của véc-tơ $\overrightarrow{u}$ là
	\choice
	{\True $(3;-1;4)$}
	{$(3;4;-1)$}
	{$(4;-1;3)$}
	{$(4;3;-1)$}
	\loigiai
	{
		Tọa độ của véc-tơ $\overrightarrow{u}$ là $(3;-1;4)$.
	}
\end{ex} 

\begin{ex}
	Trong KG $Oxyz$, điểm nào sau đây thuộc trục $Oz$?
	\choice
	{$M(1;0;0)$}
	{$M(1;0;2)$}
	{$M(1;2;0)$}
	{\True $M(0;0;-2)$}
	\loigiai{
		Ta có $M(0;0;-2) \in Oz$.
	}
\end{ex} 

\begin{ex}%[An Do - Dự án 2H3-LVD]%[2H3Y1-1]%
	Trong KG $Oxyz$, cho điểm $M$ thỏa $\vec{OM} = 2\vec{i} + \vec{j}$. Tọa độ điểm $M$ là
	\choice
	{$M(0;2;1)$}
	{$M(1;2;0)$}
	{$M(2;0;1)$}
	{\True$M(2;1;0)$}
	\loigiai{
		Tọa độ $\vec{OM} = 2\vec{i} + \vec{j} = (2;0;0) + (0;1;0) = (2;1;0)$.\\
		Vậy $M (2;1;0)$.
	}
\end{ex} 

\begin{ex}%[2H3Y1-1]%[Đoàn Mạnh Hùng]%
	Trong KG $Oxyz$, cho vectơ $\overrightarrow{OA}=\overrightarrow{j}-2\overrightarrow{k}$. Tọa độ điểm $A$ là
	\choice
	{$(1;0;-2)$}
	{\True $(0;1;-2)$}
	{$(0;-1;2)$}
	{$(1;-2;0)$}
	\loigiai{
		Ta có $\overrightarrow{OA}=\vec{j}-2\vec{k}\Leftrightarrow A(0;1;-2)$.
	}
\end{ex} 

\begin{ex}
	Trong không gian $O x y z$, xác định toạ độ của điểm $A$ biết $A$ nằm trên tia $O x$ và $O A=2$.
	\choice
	{$A(0;0;2)$}
	{$A(2;2;0)$}
	{$A(0;2;0)$}
	{\True $A(2;0;0)$}
	\loigiai{$A$ nằm trên tia $O x$ và $O A=2$ nên $A(2;0;0)$.
	}
\end{ex} 

\begin{ex}
	Trong không gian $O x y z$, xác định toạ độ của điểm $A$ biết $A$ nằm trên tia đối của tia $O y$ và $O A=3$.
	\choice
	{$A(0;3;0)$}
	{\True $A(0;-3;0)$}
	{$A(0;-9;0)$}
	{$A(3;-3;0)$}
	\loigiai{
		$A$ nằm trên tia đối của tia $O y$ và $O A=3$ nên $A(0;-3;0)$.}
\end{ex} 

\begin{ex}
	Trong KG $Oxyz$, cho hai điểm $A(1;-1;2)$ và $B(2;1;-4)$. Véc-tơ $\vec{AB}$ có tọa độ là
	\choice
	{$(-1;-2;6)$}
	{$(3;0;-2)$}
	{$(1;0;-6)$}
	{\True $(1;2;-6)$}
	\loigiai{
		Ta có $\vec{AB} = (1;2;-6)$.
	}
\end{ex} 

\begin{ex}
	Trong không gian $ Oxyz $, cho hai điểm $ A(1;3;-2) $, $ B(3;-2;4) $. Véc-tơ $ \overrightarrow{AB} $ có tọa độ là
	\choice
	{\True $ (2;5;6) $}
	{$ (4;1;2) $}
	{\True $ (2;-5;6) $}
	{$ (-2;5;6) $}
	\loigiai{
		Véc-tơ $ \overrightarrow{AB} $ có tọa độ là $ (2;-5;6) $.}
\end{ex} 

\begin{ex}
	Cho hai điểm $A$, $B$ thỏa mãn $\vec{OA} = (2;-1; 3)$ và  $\vec{OB}= (5;2;-1)$. Tìm tọa độ véc-tơ $\vec{AB}$.
	\choice
	{$\vec{AB} =(2;-1;3)$}
	{\True  $\vec{AB} =(3;3;-4)$}
	{$\vec{AB} = (7;1;2)$}
	{$\vec{AB} =(3;-3;4)$}
	\loigiai{
		$\vec{AB} = \vec{OB} - \vec{OA} = (5-2;2+1;-1-3)=(3;3;-4)$.}
\end{ex} 

\begin{ex}
	Trong KG $Oxyz$, cho hai điểm $M$ và $N$ biết $M(2;1;-1)$ và $\vv{MN}=(-1;2-3)$. Tọa độ $N$ là
	\choice
	{$N(1;-3;-4)$}
	{\True $N(1;3;-4)$}
	{$N(-1;3;-4)$}
	{$N(1;3;4)$}
	\loigiai
	{
		Gọi $N(x,y,z)$, khi đó ta có $\heva{&x-2=-1\\&y-1=2\\&z+1=-3}\Leftrightarrow \heva{&x=1\\&y=3\\&z=-4}\Rightarrow N(1;3;-4)$.\\
	}
\end{ex} 

\begin{ex}
	Hình chiếu vuông góc của điểm $A(3;-4;5)$ trên mặt phẳng $(Oxz)$ là điểm
	\choice
	{$M(3;0;0)$}
	{$M(0;-4;5)$}
	{$M(0;0;5)$}
	{\True $M(3;0;5)$}
	\loigiai{
		Hình chiếu vuông góc của điểm $A(3;-4;5)$ trên mặt phẳng $(Oxz)$ là điểm $M(3;0;5)$.}
\end{ex} 

\begin{ex}
	Hình chiếu vuông góc của điểm $A(1;2;3)$ trên mặt phẳng $(Oxy)$ là điểm
	\choice
	{$M(0;0;3)$}
	{\True $N(1;2;0)$}
	{$Q(0;2;0)$}
	{$P(1;0;0)$}
	\loigiai
	{
		Hình chiếu vuông góc của điểm $A(1;2;3)$ trên mặt phẳng $(Oxy)$ là điểm $N(1;2;0)$.
	}
\end{ex} 

\begin{ex}
	Hình chiếu vuông góc của điểm $M(2;1;-3)$ lên mặt phẳng $(Oyz)$ có tọa độ là
	\choice
	{$(2;0;0)$}
	{$(2;1;0)$}
	{\True $(0;1;-3)$}
	{$(2;0;-3)$}
	\loigiai{
		Điểm thuộc $(Oyz)$ có tọa độ $(0;y;z)$ nên hình chiếu của $M$ lên $(Oyz)$ có tọa độ là $(0;-1;3)$.
	}
\end{ex} 

\begin{ex}
	Hình chiếu vuông góc của điểm $A(3;2;1)$ trên trục $Ox$ có tọa độ là
	\choice
	{$(0;2;1)$}
	{$(0;2;0)$}
	{\True $(3;0;0)$}
	{$(0;0;1)$}
	\loigiai{
		Hình chiếu vuông góc của điểm $A(3;2;1)$ lên trục $Ox$ là $A'(3;0;0)$.
	}
\end{ex} 

\begin{ex}
	Hình chiếu của điểm $M(2;3;-2)$ trên trục $Oy$ có tọa độ là
	\choice
	{$ (2;0;0) $}
	{\True $ (0;3;0) $}
	{$ (0;0;-2) $}
	{$ (2;0;-2) $}
	\loigiai{
		Hình chiếu của điểm $M(2;3;-2)$ trên trục $Oy$ có tọa độ là $(0;3;0)$.
	}
\end{ex} 

\begin{ex}
	\immini{Trong KG $Oxyz$, cho hình bình hành $ABCD$ với $A(-2;3;1)$, $B(3;0;-1)$, $C(6;5;0)$. Tọa độ đỉnh $D$ là
		\choice
		{$D(11;2;2)$}
		{\True $D(1;8;2)$}
		{$D(11;2;-2)$}
		{$D(1;8;-2)$}}{
		\begin{tikzpicture}[scale=0.7, font=\small,>=stealth]
			\path
			%	Vẽ mp
			(0,0) coordinate (A)
			(1,1.5) coordinate (B)
			(4,0) coordinate (D)
			($(D)+(B)-(A)$)coordinate (C)
			;
			\draw (A)--(B)--(C)--(D)--(A);
			\foreach \x/\g in {A/-90,B/90,C/0,D/-90}\draw[fill=black] (\x) circle (.05) +(\g:.5)node{\small$\x$};
		\end{tikzpicture}
	}
	\loigiai{
		Ta có $\heva{& x_D = x_A+x_C-x_B = 1 \\ & y_D = y_A +y_C -y_B = 8 \\ & z_D = z_A + z_C - z_B = 2}\Rightarrow D(1;8;2)$.
	}
\end{ex} 

\begin{ex}
	\immini{Trong KG $Oxyz$, cho các điểm $A(1;0;3)$, $B(2;3;-4)$,$C(-3;1;2)$. Tìm tọa độ điểm $D$ sao cho tứ giác $ABCD$ là hình bình hành.
		\choice
		{$D(4;2;9) $}
		{$D(-2;4;-5) $}
		{$D(6;2;-3) $}
		{\True $(-4;-2;9) $}}{
		\begin{tikzpicture}[scale=0.7, font=\small,>=stealth]
			\path
			%	Vẽ mp
			(0,0) coordinate (A)
			(1,1.5) coordinate (B)
			(4,0) coordinate (D)
			($(D)+(B)-(A)$)coordinate (C)
			;
			\draw (A)--(B)--(C)--(D)--(A);
			\foreach \x/\g in {A/-90,B/90,C/0,D/-90}\draw[fill=black] (\x) circle (.05) +(\g:.5)node{\small$\x$};
		\end{tikzpicture}}
	\loigiai{
		Gọi $D(x;y;z) \Rightarrow \vec{CD}=(x+3;y-1;z-2)$ và $\vec{BA}=(-1;-3;7)$.\\
		Để tứ giác $ABCD$ là hình bình hành ta có $\vec{BA}=\vec{CD}$ $\Rightarrow \heva{&x+3=-1\\&y-1=-3\\&z-2=7} \Rightarrow D(-4;-2;9)$.
	}
\end{ex} 

\begin{ex}
	\immini{Cho hình hộp $A B C D . A' B' C' D'$ có $A(1 ; 0 ; 1)$, $B(2 ; 1 ; 2)$, $D(1 ;-1 ; 1), C'(4 ; 5 ;-5)$. Tìm tọa độ đỉnh $C$ của hình hộp.
		\haicot
		{$C(2;0;2)$}
		{$C(2;0;2)$}
		{$C(2;0;2)$}
		{$C(2;0;2)$}}{
		\begin{tikzpicture}[scale=0.65, font=\small, line join=round, line cap=round, >=stealth]
			\def\bc{4} % cạnh BC
			\def\ba{2} % cạnh BA
			\def\gocB{35} % góc B của đáy
			\coordinate[label=below left:$B$] (B) at (0,0);
			\coordinate[label=above left:$A$] (A) at (\gocB:\ba);
			\coordinate[label=below:$C$] (C) at (\bc,0);
			\coordinate[label=right:$D$] (D) at ($(C)-(B)+(A)$);
			\coordinate[label=above left:$A'$] (A') at ($(A)+(100:\bc)$);
			\coordinate[label=left:$B'$] (B') at ($(B)-(A)+(A')$);
			\coordinate[label=below right:$C'$] (C') at ($(C)-(A)+(A')$);
			\coordinate[label=right:$D'$] (D') at ($(D)-(A)+(A')$);
			\draw (B')--(B)--(C)--(D)--(D')--(A')--(B')--(C')--(D') (C)--(C');
			\draw[dashed] (A')--(A)--(D) (A)--(B);
			\foreach \diem in {A,B,C,D,A',B',C',D'}	\fill (\diem)circle(1.5pt);
		\end{tikzpicture}}
	\loigiai{
		Ta có $\overrightarrow{AB}=\overrightarrow{DC}\Leftrightarrow \heva{& 2-1=x_C-1\\ & 1-0=y_C-(-1) \\ & 2-1=z_C-1} \Leftrightarrow \heva{&x_C = 2 \\ & y_C = 0 \\ & z_C=2}\Rightarrow C(2;0;2)$.
	}
\end{ex} 


\begin{ex} %[2H2H2-2]
	Cho hình hộp $A B C D . A' B' C' D'$ có $A(1 ; 0 ; 1)$, $B(2 ; 1 ; 2)$, $D(1 ;-1 ; 1), C'(4 ; 5 ;-5)$. Tìm tọa độ đỉnh $A'$ của hình hộp.
	\choice
	{$A'(-1;-5;8)$}
	{$A'(-1;-5;8)$}
	{$A'(-1;-5;8)$}
	{$A'(-1;-5;8)$}
	\loigiai{
		Ta có
		\begin{itemize}
			\item $\overrightarrow{AB}=\overrightarrow{DC}\Leftrightarrow \heva{& 2-1=x_C-1\\ & 1-0=y_C-(-1) \\ & 2-1=z_C-1} \Leftrightarrow \heva{&x_C = 2 \\ & y_C = 0 \\ & z_C=2}\Rightarrow C(2;0;2)$;
			\item $\overrightarrow{AA'}=\overrightarrow{CC'}\Leftrightarrow \heva{& x_{A'}-1=2-4\\ & y_{A'}-0=0-5 \\ & z_{A'}-1=2-(-5)} \Leftrightarrow \heva{& x_{A'} = -1 \\ & y_{A'} = -5 \\ & z_{A'}=8}\Rightarrow A'(-1;-5;8)$;
		\end{itemize}
	}
\end{ex} 


\begin{ex}%[2H2H2-2]
	\immini{Cho hình hộp $A B C D . A' B' C' D'$ có $A(1 ; 0 ; 1)$, $B(2 ; 1 ; 2)$, $D(1 ;-1 ; 1), C'(4 ; 5 ;-5)$. Tìm tọa độ đỉnh $D'$ của hình hộp.
		\haicot
		{$D'(-1;-6;8)$}
		{$D'(-1;-6;8)$}
		{$D'(-1;-6;8)$}
		{$D'(-1;-6;8)$}}{
		\begin{tikzpicture}[scale=0.65, font=\small, line join=round, line cap=round, >=stealth]
			\def\bc{4} % cạnh BC
			\def\ba{2} % cạnh BA
			\def\gocB{35} % góc B của đáy
			\coordinate[label=below left:$B$] (B) at (0,0);
			\coordinate[label=above left:$A$] (A) at (\gocB:\ba);
			\coordinate[label=below:$C$] (C) at (\bc,0);
			\coordinate[label=right:$D$] (D) at ($(C)-(B)+(A)$);
			\coordinate[label=above left:$A'$] (A') at ($(A)+(100:\bc)$);
			\coordinate[label=left:$B'$] (B') at ($(B)-(A)+(A')$);
			\coordinate[label=below right:$C'$] (C') at ($(C)-(A)+(A')$);
			\coordinate[label=right:$D'$] (D') at ($(D)-(A)+(A')$);
			\draw (B')--(B)--(C)--(D)--(D')--(A')--(B')--(C')--(D') (C)--(C');
			\draw[dashed] (A')--(A)--(D) (A)--(B);
			\foreach \diem in {A,B,C,D,A',B',C',D'}	\fill (\diem)circle(1.5pt);
		\end{tikzpicture}}
	\loigiai{
		Ta có
		\begin{itemize}
			\item $\overrightarrow{AB}=\overrightarrow{DC}\Leftrightarrow \heva{& 2-1=x_C-1\\ & 1-0=y_C-(-1) \\ & 2-1=z_C-1} \Leftrightarrow \heva{&x_C = 2 \\ & y_C = 0 \\ & z_C=2}\Rightarrow C(2;0;2)$;
			\item $\overrightarrow{DD'}=\overrightarrow{CC'}\Leftrightarrow \heva{& x_{D'}-1=2-4\\ & y_{D'}-(-1)=0-5 \\ & z_{D'}-1=2-(-5)} \Leftrightarrow \heva{& x_{D'} = -1 \\ & y_{D'} = -6 \\ & z_{D'}=8}\Rightarrow D'(-1;-6;8)$.
		\end{itemize}
	}
\end{ex} 

\Closesolutionfile{ans}
\BTTF
\Opensolutionfile{ans}[ans/2H2-B2-d1-2]
\begin{ex}
	Trong KG $Oxyz$, cho $\vec{a}=\vec{i}+3\vec{k}-4\vec{j}$ và $\vec{b}=\big(m-n;4m-6n;n^2-3m+2\big)$, với $m$, $n$ là tham số.
	\choiceTF
	{Tọa độ $\vec{a}=\big(1;3;-4\big)$}
	{\True Dựng điểm $A$ thỏa $\vec{OA}=\vec{a}$ thì $A(1;-4;3)$}
	{Tồn tại giá trị của $m$ và $n$ để $\vec{b}=\vec{0}$}
	{\True Nếu $\vec{a}=\vec{b}$ thì $m+n=9$}
	\loigiai{
		\begin{enumerate}[a)]
			\item Tọa độ $\vec{a}=\big(1;-4;3\big)$.
			\item Khi $\vec{OA}=\vec{a}$ thì tọa độ $\vec{a}$ cũng là tọa độ điểm $A$. Suy ra $A(1;-4;3)$.
			\item $\vec{b}=\vec{0} \Leftrightarrow \heva{&m-n=0\\&4m-6n=0\\&n^2-3m+2=0} \Leftrightarrow \heva{&m=0\\&n=0\\&n^2-3m+2=0}$ (vô nghiệm).\\
			      Vậy, không tồn tại $m$, $n$ để $\vec{b}=\vec{0}$.
			\item $\vec{a}=\vec{b} \Leftrightarrow \heva{&m-n=1\\&4m-6n=-4\\&n^2-3m+2=3} \Leftrightarrow \heva{&m=5\\&n=4}$.\\
			      Suy ra $m+n=9$.
		\end{enumerate}}
\end{ex} 
\begin{ex}
	\immini{Trong KG $Oxyz$, cho $\vec{a}=(2;2;0)$, $\vec{b}=2\vec{j}+2\vec{k}$. Dựng $\vec{OA}=\vec{a}$ và $\vec{OB}=\vec{b}$.
		\choiceTF
		{$\vec{a}=2\vec{i}+2\vec{k}$}
		{\True Toạ độ $\vec{b}=(0;2;2)$}
		{\True Toạ độ $\vec{AB}=(-2;2;0)$}
		{Góc $\widehat{AOB}=45^\circ$}}{
		\begin{tikzpicture}[scale=0.5, font=\small,>=stealth]
			\path
			(0,0) coordinate (O)
			(-2,-2) coordinate (H)
			(4,0) coordinate (K)
			(0,3.5) coordinate (P)
			($(P)+(H)-(O)$)coordinate (A)
			($(P)+(K)-(O)$)coordinate (B)
			;
			\draw[->] (0,0)--(6.7,0) node[below]{$y$};
			\draw[->] (0,0)--(-3,-3) node[below]{$x$};
			\draw[->] (0,0)--(0,5) node[left]{$z$};
			\draw[-stealth,blue,thick] (O)--(-1,-1)node[above]{$\vec{i}$};
			\draw[-stealth,blue,thick](O)--(1,0)node[below right]{$\vec{j}$};
			\draw[-stealth,blue,thick] (O)--(0,1)node[above right]{$\vec{k}$};
			\draw[dashed] (H)--(A)--(P)--(B)--(K);
			\draw[thick,->](O)--(A)node[midway,sloped,above,scale=1]{$\vec{a}$};
			\draw[thick,->](O)--(B)node[midway,sloped,below,scale=1]{$\vec{b}$};
			\foreach \x/\g in {O/-90,A/180,B/10}\draw[fill=black] (\x) circle (.05) +(\g:.5)node{\small$\x$};
		\end{tikzpicture}}
	\loigiai{
		\immini{
			\begin{enumerate}[a)]
				\item Ta có $\vec{a}=(2;0;2)\Rightarrow \vec{a}=2\vec{i}+2\vec{k} $.
				\item Ta có $\vec{b}=2\vec{j}+2\vec{k} \Rightarrow \vec{b}=(0;2;2)$.
				\item Ta có $\vec{OA}=\vec{a}$ thì toạ độ véc tơ $\vec{a}$ cũng chính là toạ độ $A$. Suy ra $A(2;0;2)$. Tương tự $B(0;2;2)$. Từ đây, ta tính được
				      $$\vec{AB}=(-2;2;0).$$
				\item Nhận xét $OHMK.PANB$ là hình lập phương. Suy ra $\triangle OAB$ đều. Vậy $\widehat{AOB}=60^\circ$.
			\end{enumerate}}{
			\begin{tikzpicture}[scale=0.8, font=\small,>=stealth]
				\path
				(0,0) coordinate (O)
				(-2,-2) coordinate (H)
				(4,0) coordinate (K)
				(0,3.5) coordinate (P)
				($(P)+(H)-(O)$)coordinate (A)
				($(P)+(K)-(O)$)coordinate (B)
				($(H)+(K)-(O)$)coordinate (M)
				($(A)+(B)-(P)$)coordinate (N)
				;
				\draw[->] (0,0)--(6.7,0) node[below]{$y$};
				\draw[->] (0,0)--(-3,-3) node[below]{$x$};
				\draw[->] (0,0)--(0,4.3) node[left]{$z$};
				\draw[-stealth,blue,thick] (O)--(-1,-1)node[above]{$\vec{i}$};
				\draw[-stealth,blue,thick](O)--(1,0)node[below right]{$\vec{j}$};
				\draw[-stealth,blue,thick] (O)--(0,1)node[above right]{$\vec{k}$};
				\draw[dashed] (H)--(A)--(P)--(B)--(K)--(M)--(H) (A)--(N)--(B) (M)--(N);
				\draw[thick,->](O)--(A)node[midway,sloped,above,scale=1]{$\vec{a}$};
				\draw[thick,->](O)--(B)node[midway,sloped,above,scale=1]{$\vec{b}$};
				\foreach \x/\g in {O/-90,A/180,B/10,H/-90,M/-90,K/20,P/30,N/0}\draw[fill=black] (\x) circle (.05) +(\g:.5)node{\small$\x$};
			\end{tikzpicture}
		}
	}
\end{ex} 

\begin{ex}
	\immini{Trong không gian $O x y z$, cho hình hộp $O A B C . O' A' B' C'$ có $A(1 ; 1 ;-1)$, $B(0 ; 3 ; 0)$, $\vec{BC'}=(2 ;-6 ; 6)$. Gọi $H$, $K$ lần lượt là trọng tâm của tam giác $OA'O'$ và $CB'C'$.
	\choiceTF
		{\True Tọa độ điểm $C'$ là $(2;-3;6)$}
		{\True Tọa độ điểm $O'$ là $(3;-5;5)$}
		{Tọa độ véc tơ $\vec{AB'}=(-2;3;-6)$}
		{Tọa độ véc tơ $\vec{HK}=(-1;2;-1)$}}
		{\begin{tikzpicture}[scale=0.5, font=\small,>=stealth]
			\path
			%	Vẽ mp
			(0,0) coordinate (O)
			(-1.5,-1) coordinate (A)
			(5,0) coordinate (C)
			($(A)+(C)-(O)$)coordinate (B)
			($(O)+(-.5,4)$)coordinate (O')
			($(A)+(C)-(O)$)coordinate (B)
			($(A)+(O')-(O)$)coordinate (A')
			($(A')+(B)-(A)$)coordinate (B')
			($(B')+(C)-(B)$)coordinate (C')
			;
			\draw (B)--(A)--(A')--(B')--(B)--(C)--(C')--(O')--(A') (B')--(C');
			\draw[thick,->] (B)--(C');
			\draw[dashed] (C)--(O)--(O') (O)--(A);
			\foreach \x/\g in {O/170,A/-90,B/-90,C/0,O'/90,A'/180,B'/-5,C'/10}\draw[fill=black] (\x) circle (.05) +(\g:.4)node{\small$\x$};
		\end{tikzpicture}}
	\loigiai{
		\begin{enumerate}[a)]
			\item Gọi $C'(x;y;z)$. Ta có $$\vec{BC'}=(2 ;-6 ; 6) \Rightarrow \heva{&x-0=2\\&y-3=-6\\&z-0=6} \Leftrightarrow \heva{&x=2\\&y=-3\\&z=6}$$
			      Vậy $C(2;-3;6)$.
			\item Gọi $O'(x;y;z)$. Theo hình vẽ thì
			      $$\vec{AO'}=\vec{BC'} \Leftrightarrow \heva{&x-1=2\\&y-1=-6\\&z+1=6} \Leftrightarrow \heva{&x=3\\&y=-5\\&z=5}$$
			      Vậy $O'(3;-5;5)$.
			\item Theo hình vẽ thì $\vec{AB'}=\vec{OC'}=(2;-3;6)$.
			\item Ta có $\vec{HK}=\vec{AB}=(-1;2;1)$.
		\end{enumerate}
	}
\end{ex} 
\Closesolutionfile{ans}

\begin{dang}{Tọa độ hóa một số hình không gian}
	\begin{listEX}[1]
		\item [\ding{172}] Chọn một điểm mà từ đó có ba đường đôi một vuông góc nhau làm gốc tọa độ.
		\item [\ding{173}] Xây dựng tọa độ các điểm trên hình đã cho tương ứng với hệ trục vừa chọn.
		\item [\ding{173}] Tọa độ các điểm đặc biệt:
		\begin{listEX}[3]
			\item [$\bullet$] $M \in Ox \Rightarrow M(x;0;0)$.
			\item [$\bullet$] $M \in Oy \Rightarrow M(0;y;0)$.
			\item [$\bullet$] $M \in Oz \Rightarrow M(0;0;z)$.
			\item [$\bullet$] $M \in (Oxy) \Rightarrow M(x;y;0)$.
			\item [$\bullet$] $M \in (Oxz) \Rightarrow M(x;0;z)$.
			\item [$\bullet$] $M \in (Oyz) \Rightarrow M(0;y;z)$.
		\end{listEX}
	\end{listEX}
\end{dang}
\BTTL
\begin{vd}
	\immini{Cho hình hộp chữ nhật $ABCD.A'B'C'D'$ có cạnh $AB=AA'=2$, $AD=4$. Gọi $E$ là tâm của hình chữ nhật $ABCD$, $F$ là trung điểm $AC'$. Với hệ toạ độ $Oxyz$ được thiết lập như hình bên (gốc tọa độ $O$ trùng với $A$), hãy xác định tọa độ các đỉnh của hình hộp chữ nhật và tọa độ hai điểm $E$, $F$.
	}{
		\begin{tikzpicture}[scale=0.7, font=\small,>=stealth]
			\path
			(0,0) coordinate (A)
			(-2,-2) coordinate (B)
			(5,0) coordinate (D)
			(0,3) coordinate (A')
			($(B)+(D)-(A)$)coordinate (C)
			($(A')+(B)-(A)$)coordinate (B')
			($(B')+(C)-(B)$)coordinate (C')
			($(A')+(D)-(A)$)coordinate (D')
			($(A)!0.5!(C)$)coordinate (E)
			($(A)!0.5!(C')$)coordinate (F)
			;
			\draw[->] (D)--(6.7,0) node[below]{$y$};
			\draw[->] (B)--(-3,-3) node[below]{$x$};
			\draw[->] (A')--(0,4.5) node[left]{$z$};
			\draw (B')--(B)--(C)--(D)--(D')--(A')--(B')--(C')--(D') (C)--(C');
			\draw[dashed] (A')--(A)--(B)--(D)--(A)--(C)--(A') (A)--(C');
			\draw[-stealth,blue,thick] (O)--(-0.6,-0.6)node[above]{$\vec{i}$};
			\draw[-stealth,blue,thick](O)--(1,0)node[below right]{$\vec{j}$};
			\draw[-stealth,blue,thick] (O)--(0,0.7)node[left]{$\vec{k}$};
			\foreach \x/\g in {A/-90,B/180,C/-70,D/10,A'/40,B'/180,C'/10,D'/0,E/-90,F/-10}\draw[fill=black] (\x) circle (.04) +(\g:.5)node{\small$\x$};
		\end{tikzpicture}}
	\loigiai{}
\end{vd}

\begin{vd}%[2H2V2-2]
	\immini{
		Một máy bay $M$ đang cất cánh từ phi trường. Với hệ toạ độ $Oxyz$ được thiết lập như Hình bên, cho biết $M$ là vị trí của máy bay với $OM=14$, $\widehat{NOB}=32^\circ$, $\widehat{MOC}=65^\circ$. Tính toạ độ điểm $M$.
	}{
		\begin{tikzpicture}[scale=0.6, font=\small,>=stealth]
			\path
			(0,0) coordinate (O)
			(-2,-2) coordinate (A)
			(3,-2) coordinate (N)
			(5,0) coordinate (B)
			(3,1) coordinate (M)
			(0,3) coordinate (C)
			;
			\draw[->] (0,0)--(6,0) node[below]{$y$};
			\draw[->] (0,0)--(-3,-3) node[below]{$x$};
			\draw[->] (0,0)--(0,4) node[left]{$z$};
			\draw[dashed] (C)--(M)--(N)--(A) (O)--(N)--(B);
			\draw[fill=blue] (M)circle (0.15)--(O)node[midway,sloped,scale=1,above]{$14$};
			\foreach \x/\g in {O/160,N/-90,M/30,A/-80,B/-70,C/30}\draw[fill=black] (\x) circle (.05) +(\g:.5)node{\small$\x$};
			\foreach \x/\y/\z in {N/A/O,N/B/O,M/C/O}{\path pic[draw,angle radius=5pt]{right angle= \x--\y--\z};}
			\draw pic["$65^\circ$",draw,angle eccentricity=1.9,angle radius=0.3cm]{angle=M--O--C};
			\draw pic["$32^\circ$",draw,angle eccentricity=1.9,angle radius=0.4cm]{angle=N--O--B};
		\end{tikzpicture}
	}
	\loigiai{
	\immini{
	Ta có:\\
	$OC=OM\cos 65^\circ\approx 5{,}9$.\\
	$ON=CN=OM\sin 65^\circ\approx 12{,}7$.\\
	$OB=ON\cos 32^\circ\approx 10{,}8$.\\
	$OA=BN=ON\sin 32^\circ\approx 6{,}7$.\\
	Vì $OANB$ là hình chữ nhật nên $\vec{ON}=\vec{OA}+\vec{OB}$.\\
	Vì $OCMN$ là hình chữ nhật nên $$\vec{OM}=\vec{OC}+\vec{ON}=\vec{OA}+\vec{OB}+\vec{OC}=6{,}7\vec{i}+10{,}8\vec{j}+5{,}9\vec{k}.$$
	Do đó $M(6{,}7; 10{,}8; 5{,}9)$.
	}{
	\begin{tikzpicture}[scale=1, font=\small, line join=round, line cap=round, >=stealth]
		\def\x{2.5}
		\def\y{4}
		\def\z{3}
		\def\gocXY{-150} % góc B của đáy
		\coordinate[label=above left:$O$] (O) at (0,0);
		\coordinate[label=below:$x$] (x) at (\gocXY:\x);
		\coordinate[label=below:$y$] (y) at (\y,0);
		\coordinate[label=right:$z$] (z) at (0,\z);
		\def\vtdv{1}
		\coordinate (i) at (\gocXY:\vtdv);
		\coordinate (j) at (\vtdv,0);
		\coordinate (k) at (0,\vtdv);
		\coordinate[label=above left:$A$] (A) at (\gocXY:0.7*\x);
		\coordinate[label=above:$B$] (B) at (0.8*\y,0);
		\coordinate[label=left:$C$] (C) at (0,0.8*\z);
		\coordinate[label=below:$N$] (N) at ($(A)+(B)$);
		\coordinate[label=right:$M$] (M) at ($(N)+(C)$);
		\draw[->] (O)--(x);
		\draw[->] (O)--(y);
		\draw[->] (O)--(z);
		\draw[->] (O)--(M);
		\draw[->, red] (O)--(i) node[left]{$\vec{i}$};
		\draw[->, red] (O)--(j) node[above]{$\vec{j}$};
		\draw[->, red] (O)--(k) node[left]{$\vec{k}$};
		\draw[dashed] (A)--(N)--(B) (O)--(N) (C)--(M)--(N);
		\foreach \diem in {A,B,C,O,N,M}	\fill (\diem)circle(1.5pt);
		\foreach \A/\B/\C in {O/C/M,N/B/y,N/A/O}
		\draw pic[draw=black,angle radius=6pt] {right angle = \A--\B--\C};
		\draw pic[draw,% double,% nét đôi
				blue,angle radius=5mm,angle eccentricity=2.5,"$32^\circ$"] {angle = N--O--y};
		\draw pic[draw,% double,% nét đôi
				blue,angle radius=3mm,angle eccentricity=2.5,"$65^\circ$"] {angle = M--O--C};
	\end{tikzpicture}
	}
	}
\end{vd}

\BTTN
\Opensolutionfile{ans}[ans/2H2-B2-d2-1]

\begin{ex}
	\immini{Hình bên mô tả một sân cầu lông với kích thước theo tiêu chuẩn quốc tế. Với hệ toạ độ $Oxyz$ được thiết lập như hình bên (đơn vị trên mỗi trục là mét), giả sử $AB$ là một trụ cầu lông để căng lưới, hãy xác định tọa độ của $B$.
		\choice
		{$\big(6,1;6,7;1,55\big)$}
		{\True $\big(6,7;6,1;1,55\big)$}
		{$\big(6,1;0;1,55\big)$}
		{$\big(0;6,7;1,55\big)$}
	}{
		\begin{tikzpicture}[scale=0.55, font=\small,>=stealth]
			\path
			%	Vẽ mp
			(0,0) coordinate (O)
			(8,0) coordinate (M)
			(10,2) coordinate (N)
			(2,2) coordinate (K)
			(4,1) coordinate (F)
			(4,0) coordinate (E)
			(6,3) coordinate (B)
			(6,2) coordinate (A)
			(2.3,-0.7) coordinate (I)
			($(A)!0.5!(B)$)coordinate (C)
			($(E)!0.5!(F)$)coordinate (D)
			;
			\draw[->] (M)--(10.5,0) node[below]{$x$};
			\draw[->] (K)--(3,3) node[above]{$y$};
			\draw[->] (O)--(0,4.5) node[left]{$z$};
			\draw[fill=green!20] (O)--(M)--(N)--(K)--cycle;
			\draw[pattern=north west lines] (C)--(B)--(F)--(D)--cycle;
			\draw (O)--(M)--(N)--(K)--(O) (E)--(F) (A)--(B);
			\draw[<->,dashed] (0,-0.3)--(8,-0.3)node[midway,sloped,below]{\scriptsize$13,40$ m};
			\draw[<->,dashed] (8.3,0)--(10.3,2)node[midway,right]{\scriptsize$6,10$ m};
			\draw[<->,dashed] (6.3,2)--(6.3,3)node[midway,right]{\scriptsize$1,55$ m};
			\foreach \x/\g in {O/180,A/-80,B/90}\draw[fill=black] (\x) circle (.05) +(\g:.5)node{\small$\x$};
		\end{tikzpicture}}
	\loigiai{
		\begin{itemize}
			\item Gọi toạ độ điểm $A$ là $\left(x_A;y_A;z_A\right)$. Vì chiều rộng của sân là $6,1 \mathrm{~m}$ nên $x_A=6,1$. Do một nửa chiều dài của sân là $6,7 \mathrm{~m}$ nên $y_A=6,7$. Điểm $A$ thuộc mặt phẳng $(Oxy)$ nên $z_A=0$. Vì vậy, điểm $A$ có tọa độ là $(6,1;6,7;0)$.
			\item Độ dài đoạn thẳng $AB$ là $1,55 \mathrm{~m}$ nên điểm $B$ có toạ độ là $(6,1;6,7;1,55)$.
		\end{itemize}
		Vậy ta có: $\overrightarrow{AB}=(6,1-6,1;6,7-6,7;1,55-0)$, tức là $\overrightarrow{AB}=(0;0;1,55)$.
	}
	\loigiai{
	}
\end{ex} 

\begin{ex}
	\immini{Cho hình lập phương $ABCD.A'B'C'D'$ có cạnh bằng 2. Với hệ toạ độ $Oxyz$ được thiết lập như hình bên (gốc tọa độ $O$ trùng với điểm $A$), tọa độ điểm $B'$ là
		\haicot
		{$B(0;2;0)$}
		{$B(2;2;2)$}
		{$B(2;2;0)$}
		{\True $B(2;0;2)$}
	}{
		\begin{tikzpicture}[scale=0.5, font=\small,>=stealth]
			\path
			(0,0) coordinate (A)
			(-2,-2) coordinate (B)
			(5,0) coordinate (D)
			(0,3) coordinate (A')
			($(B)+(D)-(A)$)coordinate (C)
			($(A')+(B)-(A)$)coordinate (B')
			($(B')+(C)-(B)$)coordinate (C')
			($(A')+(D)-(A)$)coordinate (D')
			;
			\draw[->] (D)--(6.7,0) node[below]{$y$};
			\draw[->] (B)--(-3,-3) node[below]{$x$};
			\draw[->] (A')--(0,4.5) node[left]{$z$};
			\draw (B')--(B)--(C)--(D)--(D')--(A')--(B')--(C')--(D') (C)--(C');
			\draw[dashed] (A')--(A)--(B) (A)--(D);
			\foreach \x/\g in {A/-90,B/180,C/-70,D/40,A'/40,B'/180,C'/10,D'/20}\draw[fill=black] (\x) circle (.04) +(\g:.6)node{\small$\x$};
		\end{tikzpicture}
	}
	\loigiai{
	}
\end{ex} 
\begin{ex}
	\immini{Cho hình lập phương $ABCD.A'B'C'D'$ có cạnh bằng 2. Với hệ toạ độ $Oxyz$ được thiết lập như hình bên (gốc tọa độ $O$ trùng với điểm $A$), tọa độ điểm $C'$ là
		\haicot
		{$C'(2;2;0)$}
		{\True $C'(2;2;2)$}
		{$C'(2;2;0)$}
		{$C'(2;0;2)$}
	}{
		\begin{tikzpicture}[scale=0.65, font=\small,>=stealth]
			\path
			(0,0) coordinate (A)
			(-2,-2) coordinate (B)
			(5,0) coordinate (D)
			(0,3) coordinate (A')
			($(B)+(D)-(A)$)coordinate (C)
			($(A')+(B)-(A)$)coordinate (B')
			($(B')+(C)-(B)$)coordinate (C')
			($(A')+(D)-(A)$)coordinate (D')
			;
			\draw[->] (D)--(6.7,0) node[below]{$y$};
			\draw[->] (B)--(-3,-3) node[below]{$x$};
			\draw[->] (A')--(0,4.5) node[left]{$z$};
			\draw (B')--(B)--(C)--(D)--(D')--(A')--(B')--(C')--(D') (C)--(C');
			\draw[dashed] (A')--(A)--(B) (A)--(D);
			\foreach \x/\g in {A/-90,B/180,C/-70,D/40,A'/40,B'/180,C'/10,D'/20}\draw[fill=black] (\x) circle (.04) +(\g:.6)node{\small$\x$};
		\end{tikzpicture}
	}
	\loigiai{
	}
\end{ex} 


\begin{ex}
	\immini{Cho hình chóp tứ giác đều $S.ABCD$ có cạnh đáy bằng $a\sqrt{2}$, cạnh bên bằng $a\sqrt{5}$. Gọi $O$ là tâm của hình vuông $ABCD$. Với hệ toạ độ $Oxyz$ được thiết lập như hình bên (gốc tọa độ $O$ trùng với tâm hình vuông $ABCD$), tọa độ $\vec{SC}$ là
		\choice
		{$\vec{SC}=(2a;0;-2a)$}
		{$\vec{SC}=(2a;-a;-2a)$}
		{\True $\vec{SC}=(a;0;-2a)$}
		{$\vec{SC}=(a;0;2a)$}}{
		\begin{tikzpicture}[scale=0.65, font=\small,>=stealth]
			\path
			(0,0) coordinate (A)
			(-3,-2) coordinate (B)
			(5,0) coordinate (D)
			($(B)+(D)-(A)$)coordinate (C)
			($(A)!0.5!(C)$)coordinate (O)
			($(O)+(0,4)$)coordinate (S)
			;
			\draw[->] (D)--(7,0.5) node[below]{$y$};
			\draw[->] (C)--(3,-3) node[below]{$x$};
			\draw[->] (S)--(1,4.5) node[left]{$z$};
			\draw (C)--(D)--(S)--(C)--(B)--(S);
			\draw[dashed] (S)--(A)--(D)--(B)--(A)--(C) (S)--(O);
			\pic[draw,thin,angle radius=2mm] {right angle = C--O--D};
			\foreach \x/\g in {A/180,B/-90,C/-100,D/-30,S/10,O/-90}\draw[fill=black] (\x) circle (.04) +(\g:.6)node{\small$\x$};
		\end{tikzpicture}}
	\loigiai{
	}
\end{ex} 

\begin{ex}%[2H2H2-2]
	\immini{
		Cho tứ diện $SABC$ có $ABC$ là tam giác vuông tại $B$, $BC=3$, $BA=2$, $SA$ vuông góc với mặt phẳng $(ABC)$ và có độ dài bằng $2$. Với hệ toạ độ $Oxyz$ được thiết lập như hình bên (gốc tọa độ $O$ trùng với điểm $B$), tìm khẳng định \textbf{sai}.
		\haicot
		{$A(0; 2; 0)$}
		{$B(0; 0; 0)$}
		{$C(0; 0; 3)$}
		{\True $S(-2; 2; 2)$}
	}{
		\begin{tikzpicture}[scale=1, font=\small, line join=round, line cap=round, >=Stealth]
			\path
			(0:0) coordinate (B)
			(20:4) coordinate (x)
			(90:3) coordinate (z)
			(130:2) coordinate (y)
			($(B)!.7!(x)$) coordinate (C)
			($(B)!4/6!(y)$) coordinate (A)
			($(B)!3/5!(z)$) coordinate (H)
			($(A)+(H)-(B)$) coordinate (S)
			;
			\draw[->] (B)--(x);
			\draw[->] (B)--(y);
			\draw[->] (B)--(z);
			\draw[dashed] 	(A)--(C)
			;
			\draw (A)--(S) (B)--(S)--(C);
			\pic[draw,angle radius=2mm]{right angle=C--B--A}
			pic[draw,angle radius=2mm]{right angle=C--B--H}
			pic[draw,angle radius=2mm]{right angle=H--B--A}
			;
			\foreach \x/\g in {B/-90,x/90,y/180,z/0,C/-90,A/210,H/0,S/90}
			\draw[fill=black] 	(\x)
			($(\g:.2)+(\x)$) node {$\x$};
		\end{tikzpicture}
	}
	\loigiai{
	}
\end{ex} 

\begin{ex}%[2H2H2-2]
	\immini{Cho hình chóp $S.ABC$ có đáy $ABC$ là tam giác đều cạnh bằng $2$, $SA$ vuông góc với đáy và $SA =1$. Với hệ toạ độ $Oxyz$ được thiết lập như hình bên (gốc tọa độ $O$ trùng với trung điểm của đoạn $BC$), hãy tìm toạ độ điểm $S$.
		\haicot
		{$S(0;\sqrt{3};1)$}
		{$S(0;\sqrt{3};1)$}
		{$S(0;\sqrt{3};1)$}
		{$S(0;\sqrt{3};1)$}
	}{
		\begin{tikzpicture}[ font = \small, scale =1,>=stealth]
			\path
			(0:0) coordinate (A)
			++(0:4) coordinate (C)
			++(-160:3)coordinate(B)
			(A)++(90:2) coordinate (S)
			($(B)!1/2!(C)$) coordinate (O)
			($(S)+(O)-(A)$) coordinate (H)
			($(O)!1.4!(A)$) coordinate (y) node[above]{$y$}
			($(O)!1.7!(C)$) coordinate (x) node[above]{$x$}
			($(O)!1.4!(H)$) coordinate (z) node[right]{$z$}
			(intersection of S--C and O--H) coordinate (t)
			;
			\draw (S)--(A)--(B)--(C) (S)--(B) (O)--(z) (S)--(H) (C)--(t)
			;
			\draw[dashed] (C)--(A)--(O) (S)--(t)
			;
			\draw[->] (C)--(x);
			\draw[->] (A)--(y);
			\draw[->] (H)--(z);
			\pic[draw,thin,angle radius=2mm] {right angle = A--O--B}
			pic[draw,thin,angle radius=2mm] {right angle = O--H--S}
			pic[draw,thin,angle radius=2mm] {right angle = H--O--C}
			;
			\foreach \x/\g in {A/-100,C/-50,B/-90,O/-70,H/0,S/90}
			\fill (\x) circle (1pt)
			+(\g:3mm) node{$\x$};
		\end{tikzpicture}
	}
	\loigiai{
	}
\end{ex} 
\begin{ex}%[2H2V2-6]
	\immini{Ở một sân bay, vị trí của máy bay được xác định bởi điểm $M$ Trong KG $Oxyz$ như hình bên. Gọi $H$ là hình chiếu vuông góc của $M$ xuống mặt phẳng $(Oxy)$. Cho biết $OM = 50$, $\left(\overrightarrow{i},\overrightarrow{OH}\right) = 64^\circ$, $\left(\overrightarrow{OH},\overrightarrow{OM}\right) = 48^\circ$. Tìm toạ độ của điểm $M$.
		\choice
		{$M(14{,}7; 30{,}1; 37{,}2)$}
		{$M(14{,}7; 30{,}1; 37{,}2)$}
		{$M(14{,}7; 30{,}1; 37{,}2)$}
		{$M(14{,}7; 30{,}1; 37{,}2)$}
	}{
		\begin{tikzpicture}[scale=0.85, font=\small,>=stealth]
			\path
			(0,0) coordinate (O)
			(-2,-2) coordinate (A)
			(3,-2) coordinate (H)
			(5,0) coordinate (B)
			(3,1) coordinate (M)
			(0,3) coordinate (C)
			;
			\draw[->] (0,0)--(6,0) node[below]{$y$};
			\draw[->] (0,0)--(-3,-3) node[below]{$x$};
			\draw[->] (0,0)--(0,4) node[left]{$z$};
			\draw[dashed] (C)--(M)--(H)--(A) (O)--(H)--(B);
			\draw[fill=blue] (M)circle (0.15)--(O)node[midway,sloped,scale=1,above]{$50$};
			\foreach \x/\g in {O/160,H/-90,M/30,A/-80,B/-70,C/30}\draw[fill=black] (\x) circle (.05) +(\g:.5)node{\small$\x$};
			\foreach \x/\y/\z in {H/A/O,H/B/O,M/C/O}{\path pic[draw,angle radius=5pt]{right angle= \x--\y--\z};}
			\draw pic["\scriptsize$48^\circ$",draw,angle eccentricity=1.9,angle radius=0.45cm]{angle=H--O--M};
			\draw pic["\scriptsize$64^\circ$",draw,angle eccentricity=1.9,angle radius=0.3cm]{angle=A--O--H};
		\end{tikzpicture}
	}
	\loigiai{
		\immini{
			Tam giác $OMH$ vuông tại $H$, $OM = 50$; $\widehat{MOH} = 48^\circ$ nên ta có
			\begin{itemize}
				\item [$\bullet$] $OH = OM\cdot \cos 48 \approx 33{,}5$
				\item [$\bullet$] $OC = MH = OM \cdot \sin 48 \approx 37{,}2$.
			\end{itemize}
			Tam giác $OAH$ vuông tại $A$, $OH = 33{,}5$; $\widehat{AOH} = 64^\circ$ nên ta có
			\begin{itemize}
				\item [$\bullet$] $OA = OH\cdot \cos 64 \approx 14{,}7$,
				\item [$\bullet$] $OB = AH = OH\cdot \sin 64 \approx 30{,}1$.
			\end{itemize}
			Suy ra
			\begin{eqnarray*}
				\overrightarrow{OM} & = & \overrightarrow{OC} + \overrightarrow{OH} = \overrightarrow{OC} + \overrightarrow{OA}+\overrightarrow{OB} \\
				& = & 14{,}7\overrightarrow{i}+30{,}1\overrightarrow{j}+37{,}2\overrightarrow{k}.
			\end{eqnarray*}
			Vậy $M(14{,}7; 30{,}1; 37{,}2)$.
		}{
			\begin{tikzpicture}[scale=0.85, font=\small,>=stealth]
				\path
				(0,0) coordinate (O)
				(-2,-2) coordinate (A)
				(3,-2) coordinate (H)
				(5,0) coordinate (B)
				(3,1) coordinate (M)
				(0,3) coordinate (C)
				;
				\draw[->] (0,0)--(6,0) node[below]{$y$};
				\draw[->] (0,0)--(-3,-3) node[below]{$x$};
				\draw[->] (0,0)--(0,4) node[left]{$z$};
				\draw[dashed] (C)--(M)--(H)--(A) (O)--(H)--(B);
				\draw[fill=blue] (M)circle (0.15)--(O)node[midway,sloped,scale=1,above]{$50$};
				\foreach \x/\g in {O/160,H/-90,M/30,A/-80,B/-70,C/30}\draw[fill=black] (\x) circle (.05) +(\g:.5)node{\small$\x$};
				\foreach \x/\y/\z in {H/A/O,H/B/O,M/C/O}{\path pic[draw,angle radius=5pt]{right angle= \x--\y--\z};}
				\draw pic["\scriptsize$48^\circ$",draw,angle eccentricity=1.9,angle radius=0.45cm]{angle=H--O--M};
				\draw pic["\scriptsize$64^\circ$",draw,angle eccentricity=1.9,angle radius=0.3cm]{angle=A--O--H};
			\end{tikzpicture}
		}
	}
\end{ex} 

\Closesolutionfile{ans}
\BTTF
\Opensolutionfile{ans}[ans/2H2-B2-d2-2]
\begin{ex}
	\immini{Cho hình chóp $S.ABCD$ có đáy $ABCD$ là hình chữ nhật, $AB=1$, $AD=2$, $SA$ vuông góc với mặt đáy và $SA=3$. Với hệ toạ độ $Oxyz$ được thiết lập như sau: Gốc tọa độ $O$ trùng với điểm $A$, các véc tơ $\vec{AB}$, $\vec{AD}$, $\vec{AS}$ lần lượt cùng hướng với $\vec{i}$, $\vec{j}$ và $\vec{k}$. Xét tính đúng sai của các khẳng định sau
		\choiceTF
		{\True Tọa độ $D(0;2;0)$}
		{Tọa độ $C(1;2;3)$}
		{\True Tọa độ $S(2;0;0)$}
		{Tọa độ $I(1;1;0)$}
	}{
		\begin{tikzpicture}[scale=0.6, font=\small,>=stealth]
			\path
			(0,0) coordinate (A)
			(-2,-2) coordinate (B)
			(5,0) coordinate (D)
			($(B)+(D)-(A)$)coordinate (C)
			($(A)!0.5!(C)$)coordinate (I)
			($(A)+(0,3)$)coordinate (S)
			;
			\draw (C)--(D)--(S)--(C)--(B)--(S);
			\draw[dashed] (S)--(A)--(D)--(B)--(A)--(C);
			\pic[draw,thin,angle radius=2mm] {right angle = B--A--D};
			\foreach \x/\g in {A/180,B/-90,C/-100,D/-80,S/90,I/-90}\draw[fill=black] (\x) circle (.04) +(\g:.5)node{\small$\x$};
		\end{tikzpicture}	}
	\loigiai{
		\immini{
			Với hệ trục đã chọn như hình vẽ thì
			\begin{enumerate}[a)]
				\item Điểm $D \in Oy$ và $AD=2$ nên $D(0;2;0)$.
				\item Điểm $C \in (Oxy)$ và có hình chiếu lên $Ox$, $Oy$ lần lượt là điểm $B$ và $D$.\\
				      Do $AB=1$ và $AD=2$ nên $C(2;2;0)$.
				\item Điểm $S \in Oz$ và $AS=3$ nên $S(0;0;3)$.
				\item Điểm $I \in (Oxy)$ và và có hình chiếu lên $Ox$, $Oy$ lần lượt là trung điểm của $AB$ và $AD$ nên $I(0,5;1;0)$.
			\end{enumerate}}{
			\begin{tikzpicture}[scale=0.6, font=\small,>=stealth]
				\path
				(0,0) coordinate (A)
				(-2,-2) coordinate (B)
				(5,0) coordinate (D)
				($(B)+(D)-(A)$)coordinate (C)
				($(A)!0.5!(C)$)coordinate (I)
				($(A)+(0,3)$)coordinate (S)
				;
				\draw[->] (D)--(7,0) node[below]{$y$};
				\draw[->] (B)--(-3,-3) node[below]{$x$};
				\draw[->] (S)--(0,4) node[left]{$z$};
				\draw (C)--(D)--(S)--(C)--(B)--(S);
				\draw[dashed] (S)--(A)--(D)--(B)--(A)--(C);
				\pic[draw,thin,angle radius=2mm] {right angle = B--A--D};
				\foreach \x/\g in {A/180,B/-90,C/-100,D/-80,S/170,I/-90}\draw[fill=black] (\x) circle (.04) +(\g:.5)node{\small$\x$};
			\end{tikzpicture}}
	}
\end{ex} 


\begin{ex}
	\immini{Cho hình lập phương $ABCD.A'B'C'D'$ có cạnh bằng $2$. Với hệ toạ độ $Oxyz$ được thiết lập như hình bên (gốc tọa độ $O$ trùng với tâm hình vuông $ABCD$), hãy xét tính đúng sai của các khẳng định sau:
		\choiceTF
		{Tọa độ $A(-1;0;0)$}
		{\True $\vec{AC'}=(2\sqrt{2};0;2)$}
		{\True Tọa độ $D'(0;\sqrt{2};2)$}
		{$\vec{BD'}=(0;0;2)$}
	}{
		\begin{tikzpicture}[scale=0.45, font=\small,>=stealth]
			\path
			(0,0) coordinate (A)
			(-2,-2) coordinate (B)
			(6,0) coordinate (D)
			(0,4) coordinate (A')
			($(B)+(D)-(A)$)coordinate (C)
			($(A')+(B)-(A)$)coordinate (B')
			($(B')+(C)-(B)$)coordinate (C')
			($(A')+(D)-(A)$)coordinate (D')
			($(A)!0.5!(C)$)coordinate (O)
			($(A')!0.5!(C')$)coordinate (O')
			;
			\draw[->] (D)--(8,0.5) node[below]{$y$};
			\draw[->] (C)--(6,-3) node[below]{$x$};
			\draw[->] (O')--(2,5.5) node[left]{$z$};
			\draw (B')--(B)--(C)--(D)--(D')--(A')--(B')--(C')--(D') (C)--(C')--(A') (B')--(D');
			\draw[dashed] (A')--(A)--(B)--(D)--(A)--(C) (O)--(O');
			\pic[draw,thin,angle radius=2mm] {right angle = C--O--D};
			\foreach \x/\g in {A/-90,B/180,C/-70,D/-40,A'/40,B'/180,C'/10,D'/20,O/-90}\draw[fill=black] (\x) circle (.04) +(\g:.65)node{\small$\x$};
		\end{tikzpicture}
	}
	\loigiai{
		Độ dài $AC=2\sqrt{2}$. Với hệ trục $Oxyz$ đã chọn như hình vẽ thì
		\begin{enumerate}[a)]
			\item Điểm $A \in Ox$, nằm ngược chiều dương và $OA=\sqrt{2}$ nên $A(-\sqrt{2};0;0)$.
			\item Tọa độ $C'(\sqrt{2};0;2)$. Suy ra $\vec{AC'}=(2\sqrt{2};0;2)$.
			\item Điểm $D'$ có hình chiếu vuông góc xuống $(Oxy)$ là điểm $D(0;\sqrt{2};0)$ và $DD'=2$ nên $D'(0;\sqrt{2};2)$.
			\item Tọa độ $B(0;-\sqrt{2};0)$, $D'(0;\sqrt{2};2)$. Suy ra $\vec{BD'}=(0;2\sqrt{2};2)$.
		\end{enumerate}
	}
\end{ex} 

\begin{ex}
	\immini{Cho hình lăng trụ $ABC.A'B'C'$ có đáy $ABC$ là tam giác đều cạnh bằng $2$ như hình vẽ. Hình chiếu vuông góc của $A'$ lên $(ABC)$ trùng với trung điểm cạnh $AB$, góc $\widehat{A'AO}=60^\circ$. Với hệ toạ độ $Oxyz$ được thiết lập như hình bên (gốc tọa độ $O$ trùng với trung điểm của đoạn $BC$), hãy xét tính đúng sai của các khẳng định sau:
		\choiceTF
		{\True Tọa độ điểm $A(-1;0;0)$}
		{\True Tọa độ điểm $C(0;\sqrt{3};0)$}
		{Tọa độ điểm $A'(0;-1;\sqrt{3})$}
		{\True Tọa độ điểm $C'\big(1;\sqrt{3};\sqrt{3}\big)$}
	}{
		\begin{tikzpicture}[scale=0.7, font=\small,>=stealth]
			\path
			(0,0) coordinate (A)
			(2,-2) coordinate (B)
			(5,0) coordinate (C)
			(1,4) coordinate (A')
			($(A)!0.5!(B)$)coordinate (O)
			($(A')+(B)-(A)$)coordinate (B')
			($(A')+(C)-(A)$)coordinate (C')
			;
			\draw[->] (B)--(3,-3) node[below]{$x$};
			\draw[->] (C)--(7,0.5) node[below]{$y$};
			\draw[->] (A')--(1,5) node[left]{$z$};
			\draw (B)--(A)--(A')--(B')--(B)--(C)--(C')--(B') (A')--(C') (O)--(A');
			\draw[dashed] (A)--(C)--(O);
			\pic[draw,thin,angle radius=2mm] {right angle = C--O--B};
			\pic[draw,thin,angle radius=2mm] {right angle = A'--O--B};
			\pic[draw,thin,angle radius=2mm] {right angle = A'--O--C};
			\foreach \x/\g in {A/180,B/-90,C/-90,A'/180,B'/-20,C'/0,O/-100}\draw[fill=black] (\x) circle (.05) +(\g:.5)node{\small$\x$};
		\end{tikzpicture}}
	\loigiai{
		Độ dài $OC=2.\dfrac{\sqrt{3}}{2}=\sqrt{3}$. $OA'=OA.\tan60^\circ=\sqrt{3}$. Với hệ trục $Oxyz$ đã chọn như hình vẽ trên thì
		\begin{enumerate}[a)]
			\item Điểm $A \in Ox$, nằm ngược chiều dương và $OA=1$ nên $A(-1;0;0)$.
			\item Điểm $A' \in Oy$, nằm cùng chiều dương và $OC=\sqrt{3}$ nên $C(0;\sqrt{3};0)$.
			\item $A' \in Oz$, nằm cùng chiều dương và $OA'=\sqrt{3}$ nên $A'(0;0;\sqrt{3})$.
			\item Gọi $C'(x;y;z)$. Ta có
			      $$\vec{A'C'}=\vec{AC} \Leftrightarrow\heva{&x-0=1\\&y-0=\sqrt{3}\\&z-\sqrt{3}=0}\Leftrightarrow\heva{&x=1\\&y=\sqrt{3}\\&z=\sqrt{3}}.$$
		\end{enumerate}
	}
\end{ex} 


\Closesolutionfile{ans}

%%Bài 3. Biểu thức tọa độ
% \setcounter{section}{2}
\setcounter{dang}{0}
\section{BIỂU THỨC TỌA ĐỘ CỦA CÁC PHÉP TOÁN VECTƠ}
\subsection{LÝ THUYẾT CẦN NHỚ}
\subsubsection{Biểu thức tọa độ của phép toán cộng, trừ, nhân một số thực với một vectơ}
Trong không gian $Oxyz$, cho hai véc-tơ $\vec{a} = (a_1;a_2;a_3)$, $\vec{b} = (b_1; b_2; b_3)$ và số $k$. Khi đó
\begin{listEX}[1]
	\item [\ding{172}] $\vec{a}+\vec{b}=(a_1+b_1;a_2+b_2;a_3+b_3)$;
	\item [\ding{173}] $\vec{a}-\vec{b}=(a_1-b_1;a_2-b_2;a_3-b_3)$;
	\item [\ding{174}] $k\vec{a} = (ka_1; ka_2; ka_3)$.
\end{listEX}
\begin{note}
	Cho hai véc-tơ $\vec{a}=(a_1;a_2;a_3)$, $\vec{b}=(b_1;b_2;b_3)$, $\vec{b}\ne \vec{0}$. Hai véc-tơ $\vec{a}$, $\vec{b}$ cùng phương khi và chỉ khi tồn tại một số thực $k$ sao cho $\heva{&a_1=k b_1\\& a_2= k b_2\\& a_3= k b_3.}$
\end{note}
\subsubsection{Biểu thức tọa độ của tích vô hướng hai vectơ}
Trong không gian $Oxyz$, tích vô hướng của hai véc-tơ $\vec{a} = (a_1;a_2;a_3)$ và $\vec{b} = (b_1; b_2; b_3)$ được xác định bởi công thức
\[\vec{a} \cdot \vec{b} = a_1b_1 + a_2b_2 + a_3b_3. \]
\begin{note}
	\begin{itemize}
		\item[\ding{172}] $\vec{a} \perp \vec{b} \Leftrightarrow a_1b_1 + a_2b_2 + a_3b_3 = 0$;
		\item[\ding{173}] $\left| \vec{a} \right| = \sqrt{a_1^2 + a_2^2 +a_3^2}$; \quad $AB=\sqrt{(x_B-x_A)^2+(y_B-y_A)^2+(z_B-z_A)^2}$.
		\item[\ding{174}] $\cos \left(\vec{a}; \vec{b}\right) = \dfrac{\vec{a}\cdot \vec{b}}{\left|\vec{a}\right| \cdot \left|\vec{b}\right|} = \dfrac{a_1b_1 + a_2b_2 + a_3b_3}{\sqrt{a_1^2 + a_2^2 +a_3^2} \cdot \sqrt{b_1^2 + b_2^2 +b_3^2}}$ (với $\vec{a},\vec{b} \ne \vec{0}$).
	\end{itemize}
\end{note}

\subsubsection{Biểu thức tọa độ của tích có hướng hai vectơ}
Cho hai véc-tơ $\vec{a}=(a_1;a_2;a_3)$ và $\vec{b}=(b_1;b_2;b_3)$ không cùng phương. Khi đó vec tơ $$\vec{w}=\bigg(a_2b_3-b_2a_3\,;\,a_3b_1-b_3a_1\,;\,a_1b_2-b_1a_2 \bigg)$$ vuông góc với cả hai véc tơ $\vec{a}$ và $\vec{b}$.
\begin{note}
	\begin{itemize}
		\item [\ding{172}] Véc tơ $\vec{w}$ xác định như trên còn gọi là \textbf{tích có hướng} của hai vectơ $\vec{a}$, $\vec{b}$, kí hiệu  $\vec{w}=\left[\vec{a},\vec{a}\right]$.
		\item [\ding{173}] Quy ước $\left|\begin{array}{l}
				      {a_1}\quad{a_2} \\
				      {b_1}\quad{b_2}
			      \end{array}\right|=a_1b_2-a_2b_1$ thì
		      $$\left[\vec a ,\vec b\right]=\left(\left|\begin{array}{l}
					      {a_2}\quad{a_3} \\
					      {b_2}\quad{b_3}
				      \end{array}\right|;\left|\begin{array}{l}
					      {a_3}\quad {a_1} \\
					      {b_3}\quad{b_1}
				      \end{array}\right|;\left|\begin{array}{l}
					      {a_1}\quad{a_2} \\
					      {b_1}\quad{b_2}
				      \end{array}\right|\right)$$
		\item [\ding{174}] $\vec{a}$ không cùng phương với $\vec{b}$ $\Leftrightarrow \left[\vec a ,\vec b\right] \ne \vec{0}$.
	\end{itemize}
\end{note}
\subsubsection{Ứng dụng của tích có hướng của hai véc-tơ}
	\begin{enumerate}
		\item Xét sự đồng phẳng của ba véc-tơ:
		\begin{itemize}
			\item Ba vectơ $\vec{a}$; $\vec{b}$; $\vec{c}$ đồng phẳng $\Leftrightarrow \left[ \vec{a},\vec{b} \right]\cdot \vec{c}=0$.
			\item Bốn điểm $A$, $B$, $C$, $D$ tạo thành tứ diện $\Leftrightarrow \left[ \vec{AB},\vec{AC} \right]\cdot \vec{AD}\ne 0$.
		\end{itemize}
		\item Diện tích hình bình hành: $S_{ABCD}=\left| \left[ \vec{AB},\vec{AD} \right] \right|$.
		\item Tính diện tích tam giác: $S_{\triangle ABC}=\dfrac{1}{2}\left| \left[ \vec{AB},\vec{AC} \right] \right|$.
		\item Tính thể tích hình hộp: $V_{ABCD.A'B'C'D'}=\left| \left[ \vec{AB},\vec{AC} \right]\cdot\vec{AA'} \right|$.
		\item Tính thể tích tứ diện: $V_{ABCD}=\dfrac{1}{6}\left| \left[ \vec{AB},\vec{AC} \right]\cdot \vec{AD} \right|$.
	\end{enumerate} 
\subsubsection{Biểu thức tọa độ trung điểm đoạn thẳng, trọng tâm tam giác}
\immini{Trong không gian $Oxyz$, tọa độ trung điểm và trong tâm được xác định như sau:
	\begin{itemize}
		\item [\ding{172}] Tọa độ trung điểm $M$ của đoạn thẳng $AB$ là
		      \[ M\left(\dfrac{x_A + x_B}{2}; \dfrac{y_A + y_B}{2}; \dfrac{z_A + z_B}{2} \right).\]
		\item [\ding{173}] Tọa độ trọng tâm $G$ của tam giác $ABC$ là
		      \[ G\left(\dfrac{x_A + x_B +x_C}{3}; \dfrac{y_A + y_B +y_C}{3}; \dfrac{z_A + z_B + z_C}{3} \right).\]
	\end{itemize}}
	{\begin{tikzpicture}[scale=0.8, font=\footnotesize, line join=round, line cap=round]
		\begin{scope}
			\foreach \x\y\t in {-2/-2/A, 0/0/B}
			\coordinate (\t) at (\x,\y);
			\coordinate (M) at ($(A)!0.5!(B)$);
			\foreach \a\b in {A/B}
			\draw[] (\a)--(\b);
			\foreach \t\g in {A/-90, B/40,M/1200}
			\draw[fill=black] (\t)circle(0.6pt) +(\g:8pt)node{$\t$};
		\end{scope}
		\begin{scope}[xshift=3cm]
			\foreach \x\y\t in {0/0/A, -2/-2/B, 2.5/-2/C}
			\coordinate (\t) at (\x,\y);
			\coordinate (M) at ($(A)!0.5!(B)$);
			\coordinate (N) at ($(A)!0.5!(C)$);
			\coordinate (K) at ($(C)!0.5!(B)$);
			\coordinate (G) at ($(A)!2/3!(K)$);
			\foreach \a\b in {A/B, B/C, A/C, A/K, M/C, B/N}
			\draw[] (\a)--(\b);
			\foreach \t\g in {A/90, B/-100, C/-80, M/120, N/40, K/-90,G/60}
			\draw[fill=black] (\t)circle(0.8pt) +(\g:10pt)node{$\t$};
		\end{scope}
	\end{tikzpicture}}
\subsection{PHÂN LOẠI VÀ PHƯƠNG PHÁP GIẢI TOÁN}
\begin{dang}{Tọa độ của các phép toán vec tơ, tọa độ điểm, độ dài đoạn thẳng}
\end{dang}
\BTTL
\begin{vd}
	Cho $\vec{a}=(-2 ; 3 ; 2), \vec{b}=(2 ; 1 ;-1), \vec{c}=(1 ; 2 ; 3)$. Tính tọa độ của mỗi vectơ sau:
	\begin{listEX}[3]
		\item $3 \vec{a}$;
		\item $2 \vec{a}-\vec{b}$;
		\item $\vec{a}+2 \vec{b}-\dfrac{3}{2} \vec{c}$.
	\end{listEX}
	\loigiai{
		Ta có
		\begin{listEX}
			\item $3 \vec{a}=(3 \cdot(-2) ; 3 \cdot 3 ; 3 \cdot 2)$. Vậy $3 \vec{a}=(-6 ; 9 ; 6)$.
			\item Ta có $2 \vec{a}=(-4 ; 6 ; 4)$ và $\vec{b}=(2 ; 1 ;-1)$.\\ Do đó, $2 \vec{a}-\vec{b}=(-4-2 ; 6-1 ; 4-(-1))$.\\
			Vậy $2 \vec{a}-\vec{b}=(-6 ; 5 ; 5)$.
			\item Do $\vec{a}=(-2 ; 3 ; 2)$ và $2 \vec{b}=(4 ; 2 ;-2)$ nên
			\[\vec{a}+2 \vec{b}=(2 ; 5 ; 0).\]
			Ngoài ra, vì $-\dfrac{3}{2} \vec{c}=\left(-\dfrac{3}{2} ;-3 ;-\dfrac{9}{2}\right)$ nên $\vec{a}+2 \vec{b}-\dfrac{3}{2} \vec{c}=\left(\dfrac{1}{2} ; 2 ;-\dfrac{9}{2}\right)$.
		\end{listEX}}
\end{vd}
\dongcham{8}
\begin{vd}
	Trong không gian $Oxyz$, cho các véc-tơ $\vec{u}=3\vec{i}-2\vec{j}+\vec{k}$, $\vec{v}=-\dfrac{3}{2}\vec{i}+\vec{j}-\dfrac{1}{2}\vec{k}$, $\vec{w}=6\vec{i}+m\vec{j}-n\vec{k}$.
	\begin{enumerate}
		\item Chứng minh $\vec{u}$ và $\vec{v}$ cùng phương.
		\item Tìm giá trị của $m$ và $n$ để véc-tơ $\vec{u}$ và $\vec{w}$ cùng phương.
	\end{enumerate}
	\loigiai{
		Ta có $\vec{u}=(3;-2;1)$, $\vec{v}=\left(-\dfrac{3}{2}; 1; -\dfrac{1}{2}\right)$, $\vec{w}=\left(6; m; -n\right)$.
		\begin{enumerate}
			\item Hai véc-tơ $\vec{u}$ và $\vec{v}$ cùng phương khi và chỉ khi
			      $$\vec{v}=k\vec{u}\Leftrightarrow{ \left\{\begin{aligned}& -\dfrac{3}{2}=3k\\&1=-2k\\&-\dfrac{1}{2}=k\end{aligned}\right.}\Leftrightarrow k=-\dfrac{1}{2}$$
			      Như vậy $ \vec{v}=-\dfrac{1}{2}\vec{u} $ nên hai véc-tơ $\vec{u}$ và $\vec{v}$ cùng phương.
			\item Hai véc-tơ $\vec{u}$ và $\vec{w}$ cùng phương khi và chỉ khi
			      $$\vec{w}=k\vec{u}\Leftrightarrow{ \left\{\begin{aligned}&6=3k\\&m=-2k\\&-n=k\end{aligned}\right.}\Leftrightarrow { \left\{\begin{aligned}&k=2\\&m=-4\\&n=-2\end{aligned}\right.}$$
			      Như vậy $ m=-4$ và $ n=-2 $ thì hai véc-tơ $\vec{u}$ và $\vec{w}$ cùng phương. Khi đó $\vec{w}=\left(6; -4; 2\right)$.
		\end{enumerate}
	}
\end{vd}
\dongcham{8}
\begin{vd}
	Trong không gian với hệ tọa độ $Oxyz$, cho ba điểm $A(3;-1;2)$, $B(1;2;3)$, $C(4;-2;1)$.
	\begin{tasks}
		\task Chứng minh ba điểm $A, B, C$ không thẳng hàng. Xác định tọa độ trọng tâm tam giác $ABC$.
		\task Tìm tọa độ điểm $D$ biết $ABCD$ là hình bình hành.
		\task Tìm tọa độ giao điểm $E$ của đường thẳng $BC$ với mặt phẳng tọa độ $\left(Oxz\right)$.
	\end{tasks}
	\loigiai{
		\begin{enumerate}
			\item Ta có $\vec{AB}=(-2;3;1)$, $\vec{AC}=(1;-1;-1)$.
			      Vì $\dfrac{-2}{1}\neq \dfrac{-3}{-1}$ nên hai véc-tơ $\vec{AB}$, $\vec{AC}$ không cùng phương.\\
			      Hay ba điểm $A$, $B$, $C$ không thẳng hàng. Suy ra, tọa độ trọng tâm là $G\left(\dfrac{8}{3};-\dfrac{1}{3};2 \right)$.
			\item
			      \immini{Tứ giác $ABCD$ là hình bình hành khi và chỉ khi
				      $$ \vec{DC}=\vec{AB}\Leftrightarrow{ \left\{\begin{aligned}&4-x_D=-2\\&-2-y_D=3\\&1-z_D=1\end{aligned}\right.}\Leftrightarrow {\left\{\begin{aligned}&x_D=6\\&y_D=-5\\&z_D=0\end{aligned}\right.}$$
				      Vậy $D(6;-5;0)$.
			      }{
				      \begin{tikzpicture}[scale=.8]
					      \tkzDefPoints{0/4/A,-2/0/B,3/0/C}
					      \coordinate (D) at ($(A)+(C)-(B)$);
					      \tkzDrawSegments(A,B B,C C,D D,A)
					      \tkzLabelPoints[left](A,B)
					      \tkzLabelPoints[right](C,D)
					      \tkzDrawPoints(A,B,C,D)
				      \end{tikzpicture}}
			\item Vì $E$ thuộc mặt phẳng $Oxz$ nên $E=(x;0;z)$.\\
			      Ta có $\vec{AE}=(x-3;1;z-2)$.\\
			      Mặt khác $A, B, E$ thẳng hàng nên hai véc-tơ $\vec{AB}$, $\vec{AE}$ cùng phương, do đó:
			      $$ \vec{AE}=k\vec{AB}\Leftrightarrow{ \left\{\begin{aligned}&x-3=-2k\\&1=3k\\&z-2=k\end{aligned}\right.}\Leftrightarrow {\left\{\begin{aligned}&x=\dfrac{7}{3}\\&k=\dfrac{1}{3}\\&z=\dfrac{7}{3}\end{aligned}\right.}$$
			      Vậy $E=\left(\dfrac{7}{3}; 0; \dfrac{7}{3}\right)$.
		\end{enumerate}
	}
\end{vd}
\dongcham{18}
\begin{vd}
	Trong không gian $Oxyz$, cho ba điểm $A(5;-3;0)$, $B(2;1;-1)$, $C(4;1;2)$.
	\begin{enumerate}
		\item Tìm tọa độ của vectơ $\vec{u}=2\vec{AB}+\vec{AC}-5\vec{BC}$.
		\item Tìm tọa độ điểm $N$ sao cho $2\vec{NA}=-\vec{NB}$.
	\end{enumerate}
	\loigiai{
		\begin{enumerate}
			\item Ta có $\heva{&A(5;-3;0)\\ &B(2;1;-1)\\&C(4;1;2)}\Rightarrow\heva{&\vec{AB}=(-3;4;-1)\\&\vec{AC}=(-1;4;2)\\&\vec{BC}=(2;0;3)}\Rightarrow\heva{&2\vec{AB}=(-6;8;-2)\\&\vec{AC}=(-1;4;2)\\&-5\vec{BC}=(-10;0;-15)}\Rightarrow \vec{u}=(-17;12;-15)$.
			\item Gọi $N(x;y;z)$, khi đó $\heva{&\vec{NA}=(5-x;-3-y;-z)\\&\vec{NB}=(2-x;1-y;-1-z)}$\\
			      $$2\vec{NA}=-\vec{NB}\Leftrightarrow \heva{&2(5-x)=-2+x\\&2(-3-y)=-1+y\\&-2z=1+z}\Leftrightarrow \heva{&x=4\\&y=-\dfrac{5}{3}\\&z=-\dfrac{1}{3}}\Rightarrow N\left(4;-\dfrac{5}{3};-\dfrac{1}{3}\right).$$
		\end{enumerate}
	}
\end{vd}
\dongcham{14}
\begin{vd}%[2H2H2-6]
	\immini{Một phòng học có thiết kế dạng hình hộp chữ nhật với chiều dài là $8$ m, chiều rộng là $6$ m và chiều cao là $3$ m. Một chiếc đèn được treo tại chính giữa trần nhà của phòng học. Xét hệ trục toạ độ $Oxyz$ có gốc $O$ trùng với một góc phòng và mặt phẳng $(Oxy)$ trùng với mặt sàn, đơn vị đo được lấy theo mét (\textit{Hình minh họa bên}). Hãy tìm toạ độ của điểm treo đèn.}{
		\begin{tikzpicture}[line cap=round,line join=round, >=stealth,scale=0.6]
			\path (0,0)coordinate[label=left:$O$](O) (-1,-1)coordinate(B) (4,0)coordinate(D) (3,-1)coordinate(C) (0,2)coordinate(O') (-1,1)coordinate(B') (3,1)coordinate(C') (4,2)coordinate(D') (0,3)coordinate[label=right:$z$](E) (5,0)coordinate[label=above:$y$](F) (-2,-2)coordinate[label=left:$x$](G) (-1,0)coordinate[label=left:$3$ m](H) (1,-1)coordinate[label=below:$8$ m](I)
			(3.5,-0.5)coordinate[label=right:$6$ m](K);
			\draw (B)--(B')--(C')--(C)--cycle (O')--(B') (O')--(D') (C')--(D') (D')--(D) (D)--(C);
			\draw[dashed] (O)--(B) (O)--(D) (O)--(O');
			\draw[->] (B)--(G);
			\draw[->] (O')--(E);
			\draw[->] (D)--(F);
		\end{tikzpicture}
	}
	\loigiai{
		\immini{Gọi các điểm $B(3;0;0)$, $C(3;6;0)$, $D(0;6;0)$ như hình vẽ.\\
			$N$ là trung điểm $OC$, $N'$ là hình chiếu của $N$ lên mặt phẳng trần nhà.\\
			Suy ra $N'$ là điểm treo đèn.\\
			Ta có $N$ có tọa độ là $\left(\dfrac{0+3}{2};\dfrac{0+6}{2};\dfrac{0+0}{2}\right)$, suy ra $N\left(\dfrac{3}{2};3;0\right)$.\\
			Suy ra $N'\left(\dfrac{3}{2};3;3\right)$.\\
			Vậy tọa độ của điểm treo đèn là $\left(\dfrac{3}{2};3;3\right)$.
		}
		{\begin{tikzpicture}[line cap=round,line join=round, >=stealth,scale=0.6]
				\path (0,0)coordinate[label=left:$O$](O) (-1,-1)coordinate[label=left:$B$](B) (4,0)coordinate[label=above right:$D$](D) (3,-1)coordinate[label=below:$C$](C) (0,2)coordinate(O') (-1,1)coordinate(B') (3,1)coordinate(C') (4,2)coordinate(D') (0,3)coordinate[label=right:$z$](E) (6,0)coordinate[label=above:$y$](F) (-2,-2)coordinate[label=left:$x$](G) (-1,0)coordinate[label=left:$3$ m](H) (1,-1)coordinate[label=below:$8$ m](I)
				(3.5,-0.5)coordinate[label=right:$6$ m](K);
				\coordinate[label=left:$N$] (N) at (intersection cs:first line={(O)--(C)}, second line={(B)--(D)});
				\coordinate[label=left:$N'$] (N') at (intersection cs:first line={(O')--(C')}, second line={(B')--(D')});
				\draw (B)--(B')--(C')--(C)--cycle (O')--(B') (O')--(D') (C')--(D') (D')--(D) (D)--(C);
				\draw[dashed] (O)--(B) (O)--(D) (O)--(O') (O)--(C) (B)--(D) (N)--(N');
				\draw[->] (B)--(G);
				\draw[->] (O')--(E);
				\draw[->] (D)--(F);
			\end{tikzpicture}}
	}
\end{vd}
\dongcham{12}
\BTTN
\Opensolutionfile{ans}[ans/2H2-B3-d1-1]
Các câu hỏi sau đều xét trong không gian $Oxyz$.
\begin{ex}
	Cho $\vec{a}=(1;2;-3),\vec{b}=(-2;-4;6)$. Khẳng định nào sau đây đúng?
	\choice
	{$\vec{a}=2\vec{b}$}
	{$\vec{b}=2\vec{a}$}
	{\True $\vec{b}=-2\vec{a}$}
	{$\vec{a}=-2\vec{b}$}
	\loigiai{
		Ta có: $-2\vec{a}=\left(-2;-4;6\right)=\vec{b}$.
	}
\end{ex} \dongcham{3}

\begin{ex}
	Cho hai véc-tơ $\vec{x}=(2;1;-3),\vec{y}=(1;0;-1)$. Tìm tọa độ của véc-tơ $\vec{a}=\vec{x}+2\vec{y}$.
	\choice
	{\True $\vec{a}(4;1;-5)$}
	{$\vec{a}(4;1;-1)$}
	{$\vec{a}(3;1;-4)$}
	{$\vec{a}(0;1;-1)$}
	\loigiai{
		Ta có $\vec{a}=(2;1;-3)+2\cdot (1;0;-1)=(4;1;-5)$.}
\end{ex} \dongcham{5}

\begin{ex}
	Cho $\vec{a}=(1;-1;3)$, $\vec{b}=(2;0;-1)$. Tìm tọa độ véc-tơ $\vec{u}=2\vec{a}-3\vec{b}$.
	\choice
	{\True $\vec{u}=(-4;-2;9)$}
	{$\vec{u}=(4;2;-9)$}
	{$\vec{u}=(-4;-5;9)$}
	{$\vec{u}=(1;3;-11)$}
	\loigiai{
		$\vec{u}=2\vec{a}-3\vec{b}=(-4;-2;9)$.
	}
\end{ex} \dongcham{5}

\begin{ex}
	Cho hai véc-tơ  $\vec{a}=(3;0;1)$,  $\vec{c}=(1;1;0)$. Tìm tọa độ của véc-tơ $\vec{b}$ thỏa mãn biểu thức  $\vec{b}-\vec{a}+2\vec{c}=\vec{0}$.
	\choice
	{$\vec{b}=(-2;1;-1)$}
	{$\vec{b}=(-1;2;-1)$}
	{$\vec{b}=(5;2;1)$}
	{\True $\vec{b}=(1;-2;1)$}
	\loigiai
	{Gọi $\vec{b}=\left(x; y; z\right)$. Ta có
		$$\vec{b}-\vec{a}+2\vec{c}=\vec{0}\Leftrightarrow\heva{& x-3+2\cdot 1=0 \\ & y-0+2\cdot 1=0\\ & z-1+2\cdot 0=0}\Leftrightarrow\heva{& x=1 \\ & y=-2\\ & z=1.}$$
		Vậy $\vec{b}=(1;-2;1)$.
	}
\end{ex} \dongcham{5}

\begin{ex}
	Cho vectơ $\vec{a}=(1;-3;4)$. Vectơ  nào sau đây cùng phương với $\vec{a}$?
	\choice
	{$\vec{b}=(-2;-6;8)$}
	{\True $\vec{c}=(-2;6;-8)$}
	{$\vec{d}=(-2;6;8)$}
	{$\vec{m}=(2;-6;-8)$}
	\loigiai
	{
		$$\vec{b}=(-2;6;-8)=-2\vec{a}.$$
	}
\end{ex} \dongcham{4}

\begin{ex}
	Hai véc-tơ $\vec{a}= (m; 2; 3)$ và $\vec{b}= (1; n; 2)$ cùng phương khi
	\choice
	{$\heva{&m=\dfrac{1}{2}\\ &n= \dfrac{4}{3}.}$}
	{\True $\heva{&m=\dfrac{3}{2}\\ &n= \dfrac{4}{3}.}$}
	{$\heva{&m=\dfrac{3}{2}\\ &n= \dfrac{2}{3}.}$}
	{$\heva{&m=\dfrac{2}{3}\\ &n= \dfrac{4}{3}.}$}
	\loigiai
	{
		YCBT $\Leftrightarrow \exists k\in {{\mathbb{R}}^{*}}:\vec{a}=k.\vec{b}\Leftrightarrow \heva{
				& m=k.1 \\
				& 2=k.n \\
				& 3=k.2 \\
			}\Rightarrow \heva{
				& m=\dfrac{3}{2} \\
				& n=\dfrac{4}{3}. \\
			}$
	}
\end{ex} \dongcham{4}

\begin{ex}
	Cho hai điểm $A(2;3;1)$ và  $B(3;1;5)$. Tính độ dài đoạn thẳng $AB$.
	\choice
	{\True  $AB= \sqrt{21}$}
	{$AB= 2\sqrt{3}$}
	{$AB= 2\sqrt{5}$}
	{$AB= \sqrt{13}$}
	\loigiai{
		$AB = \sqrt{(3-2)^2 + (1-3)^2 +(5-1)^2} = \sqrt{21}$.}
\end{ex} \dongcham{4}

\begin{ex}
	Cho hai điểm $M(3;-2;1)$ và $N(0;1;-1)$. Tính độ dài đoạn thẳng $MN$.
	\choice
	{$MN=\sqrt{17}$}
	{$MN=22$}
	{\True $MN=\sqrt{22}$}
	{$MN=\sqrt{19}$}
	\loigiai
	{
		Ta có $\vec{MN}=(-3;3;-2)\Rightarrow MN=\sqrt{9+9+4}=\sqrt{22}$.
	}
\end{ex} \dongcham{4}

\begin{ex}
	Cho hai điểm $A(-1;1;2)$ và $B(3;-5;0)$. Tọa độ trung diểm của đoạn thẳng $AB$ là
	\choice
	{\True $(1;-2;1)$}
	{$(4;-6;2)$}
	{$(2;-3;-1)$}
	{$(2;-4;2)$}
	\loigiai{
		Gọi $M$ là trung điểm $AB$, khi đó tọa độ của $M$ được tính bởi
		\[ \heva{&x_M=\dfrac{x_A+y_A}{2}=1\\ &y_M=\dfrac{y_A+y_B}{2}=-2\\ &z_M=\dfrac{z_A+z_B}{2}=1.} \]
	}
\end{ex} \dongcham{4}

\begin{ex}
	Cho hai điểm $A(1;1;0)$, $B(3;-1;2)$. Tọa độ điểm $C$ sao cho $B$ là trung điểm của đoạn $AC$ là
	\choice
	{\True $C(5;-3;4)$}
	{$C(4;-3;5)$}
	{$C(-1;3;-2)$}
	{$C(2;0;1)$}
	\loigiai{
		Ta có $\heva{&x_B=\dfrac{x_A+x_C}{2}\\&y_B=\dfrac{y_A+y_C}{2}\\&z_B=\dfrac{z_A+z_C}{2}}\Rightarrow \heva{&x_C=2x_B-x_A=5\\&y_C=2y_B-y_A=-3\\&z_C=2z_B-z_A=4.}$
	}
\end{ex} \dongcham{5}

\begin{ex}
	Cho tam giác $ABC$ với $A(0;-1;3)$, $B(2;1;1)$, $C(1;0;-1)$. Tọa độ trọng tâm của tam giác $ABC$ là
	\choice
	{\True$(1;0;1)$}
	{$(-1;0;1)$}
	{$(0;1;1)$}
	{$(1;1;0)$}
	\loigiai{
		Gọi $G$ là trọng tâm của tam giác $ABC$. Khi đó $\heva{&x_G=\dfrac{x_A+x_B+x_C}{3}\\&y_G=\dfrac{y_A+y_B+y_C}{3}\\&z_G=\dfrac{z_A+z_B+z_C}{3}}$ $\Leftrightarrow \heva{&x_G=1\\&y_G=0\\&z_G=1.}$\\
		Vậy tọa độ trọng tâm tam giác $ABC$ là $(1;0;1)$.
	}
\end{ex} \dongcham{5}

\begin{ex}
	Cho $\vec{OA}=\vec{i}-2\vec{j}+3\vec{k}$, điểm $B(3;-4;1)$ và $C(2;0;-1)$. Tọa độ trọng tâm của tam giác $ABC$ là
	\choice
	{$(1;-2;3)$}
	{$(-1;2;-3)$}
	{\True $(2;-2;1)$}
	{$(-2;2;-1)$}
	\loigiai{
		Từ giả thiết: $\vec{OA}=\vec{i}-2\vec{j}+3\vec{k} \Rightarrow A(1;-2;3)$. \\
		Gọi $G$ là trọng tâm tam giác $ABC$, ta có: $\left\{\begin{aligned}
				 & x_G=\dfrac{x_A+x_B+x_C}{3}=2  \\
				 & y_G=\dfrac{y_A+y_B+y_C}{3}=-2 \\
				 & z_G=\dfrac{z_A+z_B+z_C}{3}=1  \\
			\end{aligned}\right. \Rightarrow G(2;-2;1)$. \\
		Vậy trọng tâm của tam giác $ABC$ là điểm $G(2;-2;1)$.}
\end{ex} \dongcham{5}

\begin{ex}
	Cho tam giác $ABC$ trọng tâm $G$. Biết $A(0;2;1)$, $B(1;-1;2)$, $G(1;1;1)$. Khi đó điểm $C$ có tọa độ là
	\choice
	{$(2;2;4)$}
	{$(-2;0;2)$}
	{$(-2;-3;-2)$}
	{\True $(2;2;0)$}
	\loigiai{
		\begin{itemize}
			\item [$\bullet$] Giả sử tọa độ $C$ là $C(a;b;c)$ khi đó $\heva{&\dfrac{0+1+a}{3}=1 \\ &\dfrac{2-1+b}{3}=1 \\ &\dfrac{1+2+c}{3}=1} \Leftrightarrow \heva{&a=2 \\ &b=2 \\ &c=0.}$
			\item [$\bullet$] Vậy điểm $C$ có tọa độ là $(2;2;0)$.
		\end{itemize}
	}
\end{ex} \dongcham{5}

\begin{ex}
	Cho bốn điểm $A(1;0;3)$, $B(2;-1;1)$, $C(-1;3;-4)$, $D(2;6;0)$ tạo thành một hình tứ diện. Gọi $M$, $N$ lần lượt là trung điểm các đoạn thẳng $AB$, $CD$. Tìm tọa độ trung điểm $G$ của đoạn $MN$.
	\choice
	{$G\left(\dfrac{4}{3};\dfrac{8}{3};0\right)$}
	{$G(2;4;0)$}
	{\True $G(1;2;0)$}
	{$G(4;8;0)$}
	\loigiai{
		Gọi $M$ là trung điểm đoạn thẳng $AB\Rightarrow M\left(\dfrac{3}{2};-\dfrac{1}{2};2\right)$.\\
		Gọi $N$ là trung điểm đoạn thẳng $CD\Rightarrow N\left(\dfrac{1}{2};\dfrac{9}{2};-2\right))$.\\
		Gọi $G$ là trung điểm đoạn thẳng $MN\Rightarrow G(1;2;0)$.
	}
\end{ex} \dongcham{5}


\begin{ex}
	Cho hai điểm $B(1;2;-3)$, $C(7;4;-2)$. Nếu $E$ là điểm thỏa mãn đẳng thức $\vec{CE}=2\vec{EB}$ thì tọa độ điểm $E$ là
	\choice
	{$\left(3;\dfrac{8}{3};\dfrac{8}{3}\right)$}
	{$\left(1;2;\dfrac{1}{3}\right)$}
	{$\left(3;3;-\dfrac{8}{3}\right)$}
	{\True $\left(\dfrac{8}{3};3;-\dfrac{8}{3}\right)$}
	\loigiai{
		$E(x;y;z)$, từ $\vec{CE}=2\vec{EB}\Rightarrow\heva{&x=\dfrac{8}{3}\\&y=3\\&z=-\dfrac{8}{3}.}$}
\end{ex} \dongcham{5}

\begin{ex}
	Cho các điểm $A(1;-1;0)$, $B(0;2;0)$, $C(2;1;3)$ và $M$ là điểm thỏa mãn hệ thức $\vec{MA}-\vec{MB}+\vec{MC}=\vec{0}$. Khi đó điểm $M$ có tọa độ là
	\choice
	{$(3;2;3)$}
	{$(3;-2;-3)$}
	{\True$(3;-2;3)$}
	{$(3;2;-3)$}
	\loigiai
	{
		Gọi $M(x;y;z)$, ta có $\heva{&1-x-(0-x)+(2-x)&=0\\&-1-y-(2-y)+1-y&=0\\&0-z-(0-z)+3-z&=0}\Leftrightarrow \heva{&x=3\\&y=-2\\&z=3}\Rightarrow M(3;-2;3).$
	}
\end{ex} \dongcham{5}

\begin{ex}
	Cho tọa độ các điểm $A(-1;3);B(2;-2)$ và $C(m;1)$. Tìm $m$ để $3$ điểm $A,B,C$ thẳng hàng.
	\choice
	{$m=\dfrac{2}{5}$}
	{$m=\dfrac{1}{5}$}
	{$m=-\dfrac{1}{3}$}
	{\True $m=-\dfrac{1}{5}$}
	\loigiai{
		Ta có $\vec{AB}=(3;-5);\vec{AC}=(m+1;-2)$.\\
		$A,B,C$ thẳng hàng $\Leftrightarrow$ $\vec{AB}$ cùng phương với $\vec{AC} \Leftrightarrow \dfrac{3}{m+1}=\dfrac{-5}{-2} \Leftrightarrow m=-\dfrac{1}{5}$.
	}
\end{ex} \dongcham{5}

\begin{ex}
	Cho ba điểm $A\left(-1;1;2\right),\ B(0;1;-1),\ C(x+2;y;-2)$ thẳng hàng. Tổng $x+y$ bằng
	\choice
	{$\dfrac{7}{3}$}
	{$-\dfrac{8}{3}$}
	{\True $-\dfrac{2}{3}$}
	{$-\dfrac{1}{3}$}
	\loigiai{
		\begin{itemize}
			\item [$\bullet$] Ta có $\vec{AB}=(1;0-3),\ \vec{AC}=(x+3;y-1;-4)$.
			\item [$\bullet$] Các điểm $A,\ B,\ C$ thẳng hàng $\Leftrightarrow$ có số thực $t$ thỏa mãn $\vec{AC}=t\vec{AB}$.\tagEX{1}
			      Ta có $(1)\Leftrightarrow\heva{&x+3=t\\&y-1=0\\&-4=-3t}\Leftrightarrow\heva{&x=-\dfrac{5}{3}\\&y=1\\&t=\dfrac{4}{3}}\Rightarrow x+y=-\dfrac{2}{3}$.
			\item [$\bullet$] Vậy tổng $x+y=-\dfrac{2}{3}$.
		\end{itemize}
	}
\end{ex} \dongcham{5}

\begin{ex}
	Tứ giác $ABCD$ là hình bình hành, biết $A(1; 0; 1)$, $B(2; 1; 2)$, $D(1; -1; 1)$. Tìm tọa độ điểm $C$.
	\choice
	{$(0; -2; 0)$}
	{$(2; 2; 2)$}
	{\True $(2; 0; 2)$}
	{$(2; -2; 2)$}
	\loigiai{
		\begin{itemize}
			\item [$\bullet$] Tứ giác $ABCD$ là hình bình hành khi
			      $$\vec{AB}=\vec{DC} \Leftrightarrow \heva{&x_C -1=2-1\\&y_C+1=1-0\\&z_C-1=2-1}\Leftrightarrow \heva{&x_C =2\\&y_C=0\\&z_C=2}.$$
			\item [$\bullet$] Tọa độ điểm $C(2; 0; 2)$.
		\end{itemize}
	}
\end{ex} \dongcham{5}

\begin{ex}
	\immini[thm]{Cho hình hộp $ABCD.A'B'C'D'$ có $A(0;0;0)$, $B(a;0;0)$, $D(0;2a;0)$, $A'(0;0;2a),a\ne 0$. Tính độ dài đoạn thẳng $AC'$.
		\haicot
		{$|a|$}
		{$2|a|$}
		{\True $3|a|$}
		{$\dfrac{3|a|}{2}$}}{
		\begin{tikzpicture}[scale=0.65, font=\footnotesize, line join=round, line cap=round, >=stealth]
			\def\bc{4} % cạnh BC
			\def\ba{2} % cạnh BA
			\def\gocB{35} % góc B của đáy
			\coordinate[label=below left:$B$] (B) at (0,0);
			\coordinate[label=above left:$A$] (A) at (\gocB:\ba);
			\coordinate[label=below:$C$] (C) at (\bc,0);
			\coordinate[label=right:$D$] (D) at ($(C)-(B)+(A)$);
			\coordinate[label=above left:$A'$] (A') at ($(A)+(99:\bc)$);
			\coordinate[label=left:$B'$] (B') at ($(B)-(A)+(A')$);
			\coordinate[label=below right:$C'$] (C') at ($(C)-(A)+(A')$);
			\coordinate[label=right:$D'$] (D') at ($(D)-(A)+(A')$);
			\draw (B')--(B)--(C)--(D)--(D')--(A')--(B')--(C')--(D') (C)--(C');
			\draw[dashed] (A')--(A)--(D) (A)--(B);
			\foreach \diem in {A,B,C,D,A',B',C',D'}	\fill (\diem)circle(1.5pt);
		\end{tikzpicture}}
	\loigiai{
		\immini{
			Ta có: $\vec{AB}=(a;0;0); \vec{AD}=(0;2a;0); \vec{AA'}=(0;0;2a)$.
			$$\vec{AC'}=\vec{AB}+\vec{AD}+\vec{AA'}\Rightarrow \vec{AC'}=(a;2a;2a).$$
			Suy ra $AC'=\sqrt{a^2+4a^2+4a^2}=3|a|$.
		}
		{
			\begin{tikzpicture}[scale=0.65, font=\footnotesize, line join=round, line cap=round, >=stealth]
				\def\bc{4} % cạnh BC
				\def\ba{2} % cạnh BA
				\def\gocB{35} % góc B của đáy
				\coordinate[label=below left:$B$] (B) at (0,0);
				\coordinate[label=above left:$A$] (A) at (\gocB:\ba);
				\coordinate[label=below:$C$] (C) at (\bc,0);
				\coordinate[label=right:$D$] (D) at ($(C)-(B)+(A)$);
				\coordinate[label=above left:$A'$] (A') at ($(A)+(99:\bc)$);
				\coordinate[label=left:$B'$] (B') at ($(B)-(A)+(A')$);
				\coordinate[label=below right:$C'$] (C') at ($(C)-(A)+(A')$);
				\coordinate[label=right:$D'$] (D') at ($(D)-(A)+(A')$);
				\draw (B')--(B)--(C)--(D)--(D')--(A')--(B')--(C')--(D') (C)--(C');
				\draw[dashed] (A')--(A)--(D) (A)--(B);
				\foreach \diem in {A,B,C,D,A',B',C',D'}	\fill (\diem)circle(1.5pt);
			\end{tikzpicture}

		}
	}
\end{ex} \dongcham{5}

\begin{ex}
	\immini[thm]{Cho hình hộp $ABCD.A'B'C'D'$ có $A(0;0;1)$, $B'(1;0;0)$, $C'(1;1;0)$. Tìm tọa độ của điểm $D$.
		\haicot
		{$D(0;-1;1)$}
		{\True $D(0;1;1)$}
		{$D(0;1;0)$}
		{$D(1;1;1)$}}{
		\begin{tikzpicture}[scale=0.65, font=\footnotesize, line join=round, line cap=round, >=stealth]
			\def\bc{4} % cạnh BC
			\def\ba{2} % cạnh BA
			\def\gocB{35} % góc B của đáy
			\coordinate[label=below left:$B$] (B) at (0,0);
			\coordinate[label=above left:$A$] (A) at (\gocB:\ba);
			\coordinate[label=below:$C$] (C) at (\bc,0);
			\coordinate[label=right:$D$] (D) at ($(C)-(B)+(A)$);
			\coordinate[label=above left:$A'$] (A') at ($(A)+(99:\bc)$);
			\coordinate[label=left:$B'$] (B') at ($(B)-(A)+(A')$);
			\coordinate[label=below right:$C'$] (C') at ($(C)-(A)+(A')$);
			\coordinate[label=right:$D'$] (D') at ($(D)-(A)+(A')$);
			\draw (B')--(B)--(C)--(D)--(D')--(A')--(B')--(C')--(D') (C)--(C');
			\draw[dashed] (A')--(A)--(D) (A)--(B);
			\foreach \diem in {A,B,C,D,A',B',C',D'}	\fill (\diem)circle(1.5pt);
		\end{tikzpicture}}
	\loigiai
	{
		\immini
		{
			Gọi $D(x_D;y_D;z_D)$.\\
			Ta có $\vec{B'C'}=(0;1;0)$, $\vec{AD}=\left(x_D;y_D;z_D-1\right)$. Vì $B'C'DA$ là hình bình hành nên
			\begin{align*}
				\vec{B'C'}=\vec{AD}\Leftrightarrow \heva{ & x_D=0 \\ & y_D=1\\& z_D-1=0}\Leftrightarrow \heva{& x_D=0 \\ & y_D=1\\& z_D=1.}
			\end{align*}
			Vậy $D(0;1;1)$.
		}
		{
			\begin{tikzpicture}[scale=0.65, font=\footnotesize, line join=round, line cap=round, >=stealth]
				\def\bc{4} % cạnh BC
				\def\ba{2} % cạnh BA
				\def\gocB{35} % góc B của đáy
				\coordinate[label=below left:$B$] (B) at (0,0);
				\coordinate[label=above left:$A$] (A) at (\gocB:\ba);
				\coordinate[label=below:$C$] (C) at (\bc,0);
				\coordinate[label=right:$D$] (D) at ($(C)-(B)+(A)$);
				\coordinate[label=above left:$A'$] (A') at ($(A)+(99:\bc)$);
				\coordinate[label=left:$B'$] (B') at ($(B)-(A)+(A')$);
				\coordinate[label=below right:$C'$] (C') at ($(C)-(A)+(A')$);
				\coordinate[label=right:$D'$] (D') at ($(D)-(A)+(A')$);
				\draw (B')--(B)--(C)--(D)--(D')--(A')--(B')--(C')--(D') (C)--(C');
				\draw[dashed] (A')--(A)--(D) (A)--(B);
				\foreach \diem in {A,B,C,D,A',B',C',D'}	\fill (\diem)circle(1.5pt);
			\end{tikzpicture}
		}
	}
\end{ex} \dongcham{5}


\Closesolutionfile{ans}
\BTTF
\Opensolutionfile{ans}[ans/2H2-B3-d1-2]
\begin{ex}
	Cho các điểm $A(1 ;-2 ; 3), B(-2 ; 1 ; 2), C(3 ;-1 ; 2)$.
	\choiceTF
	{\True $\vec{A B}=(-3 ; 3 ;-1)$}
	{$\vec{A C}=(-2 ;-1 ; 1)$}
	{$\vec{A B}=3 \vec{A C}$}
	{\True  Ba điểm $A, B, C$ không thẳng hàng}
	\loigiai{
		\begin{enumerate}[a)]
			\item $\vec{A B}=\big(x_B-x_A;y_B-y_A;z_B-z_A\big)=(-3 ; 3 ;-1)$.
			\item $\vec{A C}=\big(x_C-x_A;y_C-y_A;z_C-z_A\big)=(2 ; 1 ;-1)$
			\item $\vec{A B}=(-3 ; 3 ;-1)$, $\vec{A C}=(2 ; 1 ;-1)$. Hai vec tơ này không cùng phương nên không tồn tại số thực $k$ để $\vec{A B}=k \vec{A C}$.
			\item Hai vec tơ $\vec{A B}$ và $\vec{A C}$ không cùng phương nên ba điểm $A, B, C$ không thẳng hàng.
		\end{enumerate}
	}
\end{ex} \dongcham{5}

\begin{ex}
	\immini[thm]{Cho ba điểm $ A(3;3;-6) $, $ B(1;3;2) $ và $ C(-1;-3;1)$. Gọi $M$, $N$, $K$ lần lượt là trung điểm của $AB$, $BC$ và $CA$.
		\choiceTF
		{Tọa độ $M\left(2;3;2 \right)$}
		{Với $G$ là trọng tâm tam giác $ABC$ thì $GC=2\sqrt{5}$}
		{\True Trọng tâm tam giác $MNK$ là $E(1;1;-1)$}
		{\True Với $D(-3;-3;9)$ thì tứ giác $ABDC$ là hình bình hành}}{
		\begin{tikzpicture}[scale=1, font=\footnotesize,>=stealth]
			\path
			%	Vẽ mp
			(0,0) coordinate (B)
			(5,0) coordinate (C)
			(2,3) coordinate (A)
			($(A)!0.5!(B)$)coordinate (M)
			($(B)!0.5!(C)$)coordinate (N)
			($(A)!0.5!(C)$)coordinate (K)
			($(A)!2/3!(N)$)coordinate (G)
			;
			\draw (B)--(A)--(C)--(B) (M)--(N)--(K)--(M);
			\foreach \x/\g in {A/90,B/180,C/0,M/160,N/-90,K/10}\draw[fill=black] (\x) circle (.05) +(\g:.5)node{\footnotesize$\x$};
		\end{tikzpicture}}
	\loigiai{
		\begin{enumerate}[a)]
			\item $M$ là trung điểm của $AB$, suy ra $M\left(\dfrac{x_A+x_B}{2}; \dfrac{y_A+y_B}{2};\dfrac{z_A+z_B}{2}\right)$ hay $M(2;3;-2)$.
			\item Ta có $G(1;1;-1)$. Suy ra $GC=\sqrt{(-1-1)^2+(-3-1)^2+(1+1)^2}=2\sqrt{6}$.
			\item Hai tam giác $ABC$ và $MNK$ có cùng trọng tâm. Suy ra $E$ trùng với $G(1;1;-1)$.
			\item Ta có $\vec{AC}=(-4;-6;7)$, $\vec{BD}=(-4;-6;7)$, suy ra $\vec{AC}=\vec{BD}$. Vậy $ABDC$ là hình bình hành.
		\end{enumerate}
	}
\end{ex} \dongcham{5}

\begin{ex}
	\immini[thm]{Cho hình hộp $ ABCD.A'B'C'D' $, biết điểm $ A(0; 0; 0)$, $ B(1; 0; 0)$, $ C(1; 2; 0)$, $ D'(-1; 3; 5)$. Gọi $M$, $N$ là tâm của các hình bình hành $ABB'A'$, $ADD'A'$.
		\choiceTF
		{\True Tọa độ $D(0; 2; 0)$}
		{\True Tọa độ $A'(-1; 1; 5)$}
		{Tọa độ $\vec{MN}=(-1;1;0)$}
		{$\big|\vec{AB}+\vec{AD}+\vec{CC'}\big|=\sqrt{29}$}}{
		\begin{tikzpicture}[scale=0.7, font=\footnotesize, line join=round, line cap=round,>=stealth]
			\tkzDefPoints{0/0/A,-2/-2/B, 4/0/D,-0.5/3.5/A'}
			\coordinate (C) at ($(B)+(D)$);
			\coordinate (B') at ($(B)+(A')$);
			\coordinate (C') at ($(C)+(A')$);
			\coordinate (D') at ($(D)+(A')$);
			\coordinate (x) at ($(A)!1.5!(B)$);
			\coordinate (y) at ($(A)!1.5!(D)$);
			\coordinate (z) at ($(A)!1.5!(A')$);
			\tkzDrawPoints[fill=black](A,B,C,D,A',B',C',D')
			\draw[dashed] (A)--(B) (A)--(D) (A)--(A');
			\draw (A')--(B')--(C')--(D')--(A') (B)--(B') (C)--(C') (D)--(D') (B)--(C)--(D);
			\tkzLabelPoints[left](A,B,B',A')
			\tkzLabelPoints[below right=-0.1](C,D)
			\tkzLabelPoints[right](C',D')
		\end{tikzpicture}}
	\loigiai{
		\immini{
			\begin{enumerate}[a)]
				\item Theo qui tắc hình bình hành, ta có
				      \[\vec{AD}=\vec{AC}-\vec{AB}=(0; 2; 0)\Rightarrow D(0; 2; 0).\]
				\item Ta có
				      \[\vec{AA'}=\vec{DD'}=(-1; 1; 5)\Rightarrow A'(-1; 1; 5).\]
				\item Theo hình vẽ $\vec{MN}=\vec{BC}=(0;2;0)$.
				\item Ta có $\vec{AC'}=\vec{AB}+\vec{AD}+\vec{AA'}=(0;3;5)$.\\ Xét
				      \begin{eqnarray*}
					      \big|\vec{AB}+\vec{AD}+\vec{CC'}\big|
					      &=&\big|\vec{AB}+\vec{AD}+\vec{AA'}\big|=\big|\vec{AC'}\big|\\
					      &=& \sqrt{0^2+3^2+5^2}=\sqrt{34}.
				      \end{eqnarray*}
			\end{enumerate}
		}{
			\begin{tikzpicture}[scale=0.7, font=\footnotesize, line join=round, line cap=round,>=stealth]
				\tkzDefPoints{0/0/A,-2/-2/B, 4/0/D,-0.5/3.5/A'}
				\coordinate (C) at ($(B)+(D)$);
				\coordinate (B') at ($(B)+(A')$);
				\coordinate (C') at ($(C)+(A')$);
				\coordinate (D') at ($(D)+(A')$);
				\coordinate (x) at ($(A)!1.5!(B)$);
				\coordinate (y) at ($(A)!1.5!(D)$);
				\coordinate (z) at ($(A)!1.5!(A')$);
				\coordinate (M) at ($(A')!0.5!(B)$);
				\coordinate (N) at ($(C)!0.5!(D')$);
				\tkzDrawPoints[fill=black](A,B,C,D,A',B',C',D')
				\draw[dashed] (A)--(B) (A)--(D) (A)--(A') (M)--(N);
				\draw (A')--(B')--(C')--(D')--(A') (B)--(B') (C)--(C') (D)--(D') (B)--(C)--(D) (A')--(B) (C)--(D');
				\draw[->] (B)--(x) node[above] {$x$};
				\draw[->] (D)--(y) node[above] {$y$};
				%\draw[->] (A')--(z) node[left] {$z$};
				\tkzLabelPoints[left](A,B,B',A',M)
				\tkzLabelPoints[below right=-0.1](C,D)
				\tkzLabelPoints[right](C',D',N)
			\end{tikzpicture}
		}
	}
\end{ex} \dongcham{5}

\begin{ex}
	\immini[thm]{Hai chiếc khinh khí cầu bay lên từ cùng một địa điểm. Chiếc thứ nhất cách điểm xuất phát $2$ km về phía nam và $1$ km về phía đông, đồng thời cách mặt đất $0{,}5$ km. Chiếc thứ hai nằm cách điểm xuất phát $1$ km về phía bắc và $1{,}5$ km về phía tây, đồng thời cách mặt đất $0{,}8$ km.\\
		Chọn hệ trục $Oxyz$ với gốc $O$ đặt tại điểm xuất phát của hai khinh khí cầu, mặt phẳng $(Oxy)$ trùng với mặt đất với trục $Ox$ hướng về phía nam, trục $Oy$ hướng về phía đông và trục $Oz$ hướng thẳng đứng lên trời (Hình bên dưới), đơn vị đo lấy theo kilomet.
	}{
		\begin{tikzpicture}[smooth,samples=300,scale=0.8,>=stealth,font=\footnotesize]
			\draw[->] (-4,0)node[above right]{Bắc}--(6,0) node[below]{$x$} node[above left]{Nam};
			\draw[->] (3,3)node[above right]{Tây}--(-2.3,-2.3) node[left]{$y$}node[below right]{Đông};
			\draw[->] (0,0)--(0,4.5) node[right]{$z$};
			\draw (0,0) node[below]{$O$};
			\fill [scale=.1,black,yshift=12 cm,xshift=25 cm]
			(-1,0) rectangle (1,1)
			(-1,2).. controls +(135:1) and +(180:4) .. (0,7)
			.. controls +(180:2) and +(135:2).. (-.5,2)
			.. controls +(120:2) and +(180:1) .. (0,6.99)
			.. controls +(0:1) and +(60:2) .. (.5,2)
			.. controls +(45:2) and +(0:2) .. (0,7)
			.. controls +(0:4) and +(45:1).. (1,2)
			;
			\draw[scale=.1,yshift=12 cm,xshift=25 cm] (-1,0) rectangle (1,2) (0,1)--(0,2);
			;
			\fill [scale=.1,blue,yshift=37 cm,xshift=-20 cm]
			(-1,0) rectangle (1,1)
			(-1,2).. controls +(135:1) and +(180:4) .. (0,7)
			.. controls +(180:2) and +(135:2).. (-.5,2)
			.. controls +(120:2) and +(180:1) .. (0,6.99)
			.. controls +(0:1) and +(60:2) .. (.5,2)
			.. controls +(45:2) and +(0:2) .. (0,7)
			.. controls +(0:4) and +(45:1).. (1,2)
			;
			\draw[blue,scale=.1,yshift=37 cm,xshift=-20 cm] (-1,0) rectangle (1,2) (0,1)--(0,2);
			;
			\draw[dashed] (4,0)--(2.5,-1.5)--(-1.5,-1.5)
			(2.5,1)--(0,0)--(2.5,-1.5)--(2.5,1);
			\draw[dashed] (-3,0)--(-2,1)--(1,1) (-2,1)--(0,0)--(-2,3.5)--(-2,1)
			;
			\draw[fill=black] (2.5,1) circle(2pt) (-2,3.5) circle(2pt);
		\end{tikzpicture}}
	\choiceTF
	{\True Với hệ tọa độ đã chọn, toạ độ khinh khí cầu thứ nhất là $(2;1;0{,}5)$}
	{Với hệ tọa độ đã chọn, toạ độ khinh khí cầu thứ hai  là $(-1{,}5;-1;0{,}8)$}
	{Khoảng cách từ điểm xuất phát đến khinh khí cầu thứ nhất bằng $\sqrt{21}$ km}
	{\True Khoảng cách hai chiếc khinh khí cầu là $3{,}92\text{ km}$ (\textit{Kết quả làm tròn đến hàng phần trăm})}
	\loigiai{
		\begin{enumerate}
			\item Chiếc khinh khí cầu thứ nhất có tọa độ là $(2;1;0{,}5)$.
			\item Chiếc khinh khí cầu thứ hai có tọa độ là $(-1;-1{,}5;0{,}8)$.
			\item Khoảng cách từ điểm xuất phát đến khinh khí cầu thứ nhất bằng $\sqrt{2^2+1^2+0,5^2}=\dfrac{\sqrt{21}}{2}$ (km)
			\item Khoảng cách hai chiếc khinh khí cầu là
			      $\sqrt{(-1-2)^2+(1{,}5-1)^2+(0{,}8-0{,}5)^2}=\sqrt{15{,}34}\approx3{,}92\text{ (km)}.$
		\end{enumerate}
	}
\end{ex}

\Closesolutionfile{ans}

\begin{dang}{Tích vô hướng, tích có hướng hai vec tơ và ứng dụng}
\end{dang}
\BTTL
\begin{vd}
	Cho ba véc-tơ $\vec{a} = (3; 0; 1)$, $\vec{b} = (1; -1; -2)$, $\vec{c} = (2; 1; -1)$, $\vec{d} = (1; 7; -3)$.
	\begin{tasks}(3)
		\task Tính $\vec{a}\cdot \vec{b}$, $\vec{b}\cdot \vec{c}$.
		\task Tính $\left| \vec{a} \right|$, $\left| \vec{b} \right|$, $\cos \left(\vec{a}, \vec{b}\right)$.
		\task Chứng minh $\vec{d} \perp \vec{a}$.
	\end{tasks}
	\loigiai{
		\begin{enumerate}
			\item Ta có $\vec{a} \cdot \vec{b} = 3\cdot 1 + 0\cdot (-1) + 1\cdot (-2) = 1$ và $\vec{b} \cdot \vec{c} = 1\cdot 2 + (-1)\cdot 1 + (-2)\cdot (-1) = 3$.
			\item Ta có $\left| \vec{a} \right| = \sqrt{3^2 + 0^2 + 1^2} = \sqrt{10}$, $\left| \vec{b} \right| \sqrt{1^2 + (-1)^2 + (-2)^2} = \sqrt{6}$.\\
			      $\cos \left(\vec{a}, \vec{b}\right) = \dfrac{\vec{a}\cdot \vec{b}}{\left|\vec{a}\right| \cdot \left|\vec{b}\right|} = \dfrac{1}{\sqrt{10}\cdot \sqrt{6}} = \dfrac{\sqrt{15}}{60}$.
			\item Ta có $\vec{d}\cdot \vec{a} = 1\cdot 3 + 7 \cdot 0 + (-3)\cdot 1 = 0 \Rightarrow \vec{d} \perp \vec{a}$.
		\end{enumerate}
	}
\end{vd}
\dongcham{16}
\begin{vd}
	Trong không gian $Oxyz$, cho $\vec{a}=(1;0;1)$, $\vec{b}=(1;1;0)$ và $\vec{c}=(-4;3;m)$.
	\begin{listEX}
		\item Tính góc giữa hai vectơ $\vec{a}$ và $\vec{b}$.
		\item Tìm $m$ để vectơ $\vec{d}=2\vec{a}+3\vec{b}$ vuông góc với $\vec{c}$.
	\end{listEX}
	\loigiai{
		\begin{listEX}
			\item Ta có $\heva{&\vec{a}=(1;0;1)\\ &\vec{b}=(1;1;0)}\Rightarrow \cos(\vec{a};\vec{b})=\dfrac{\vec{a}\cdot\vec{b}}{|\vec{a}|\cdot |\vec{b}|}=\dfrac{1}{2}$.
			\item Ta có $\vec{d}=2\vec{a}+3\vec{b}=(5;3;2)$.\\
			Ta có $\vec{d}\perp \vec{c}\Leftrightarrow \vec{d}\cdot\vec{c}=0\Leftrightarrow -20+9+2m=0\Leftrightarrow m=\dfrac{11}{2}$.
		\end{listEX}
	}
\end{vd}
\dongcham{16}
\begin{vd}
	Trong không gian $Oxyz$, cho tam giác $ABC$ có $A(-1; 0; 2)$, $B(0; 4; 3)$ và $C(-2; 1; 2)$.
	\begin{tasks}
		\task Chỉ ra tọa độ một véc tơ (khác $\vec{0}$) vuông góc với hai véc tơ $\vec{AB}$, $\vec{AC}$.
		\task Tính chu vi tam giác $ABC$.
		\task Tính $\cos \widehat{BAC}$.
		\task Tìm độ dài đường phân giác trong $AD$ của tam giác $ABC$.
	\end{tasks}
	\loigiai{
		\begin{enumerate}[a)]
			\item
			\item Ta có $AB = \sqrt{1 + 16 + 1} = 3\sqrt{2}$ và $AC = \sqrt{1 + 1 + 0}$.
			\item
			\item Theo tính chất đường phân giác trong của tam giác, ta có $\dfrac{DB}{DC} = \dfrac{AB}{AC} = 3$.\\
			      Suy ra $\overrightarrow{DB} = -3\overrightarrow{DC} \Leftrightarrow \heva{&x_D = \dfrac{x_B + 3x_C}{4} = -\dfrac{3}{2} \\&y_D = \dfrac{y_B + 3y_C}{4} = \dfrac{7}{4} \\&z_D = \dfrac{z_B + 3z_C}{4} = \dfrac{9}{4}\cdot}$\\
			      $\Rightarrow D\left(-\dfrac{3}{2}; \dfrac{7}{4}; \dfrac{9}{4}\right)$.\\
			      Vậy $AD = \sqrt{\dfrac{1}{4} + \dfrac{49}{16} + \dfrac{1}{16}} = \dfrac{3\sqrt{6}}{4}\cdot$
		\end{enumerate}
	}
\end{vd}
\dongcham{20}
\begin{vd}
	Trong không gian $Oxyz$, cho 3 điểm $A\left(0;1;-2\right);B\left(3;0;0\right)$ và điểm $C$ thuộc trục $Oz$. Biết $ABC$ là tam giác cân tại $C$. Tìm toạ độ điểm $C$.
	\loigiai{
		Gọi $C\left(0;0;z\right)$ là điểm thuộc trục $Oz$.\\
		Tam giác $ABC$ cân tại $C$ nên $CA=CB$. \\
		Suy ra $CA^2=CB^2 \Rightarrow 1+(z+2)^2=9+z^2\Rightarrow z=1\Rightarrow C(0;0;1)$.}

\end{vd}
\dongcham{14}
\begin{vd}
	Trong không gian $Oxyz$, cho ba điểm $M\left(2;3;-1\right)$, $N\left(-1;1;1\right)$, $P\left(1;m-1;2\right)$. Với những giá trị nào của $m$ thì tam giác $MNP$ vuông tại $N$?
	\loigiai{Ta có $\vec{NM}=\left(3;2;-2\right)$ và $\vec{NP}=\left(2;m-2;1\right)$.\\
		Vì tam giác $MNP$ vuông tại $N$ nên ta có $\vec{NM} \perp \vec{NP} \Leftrightarrow \vec{NM}.\vec{NP} =0 \Leftrightarrow 2m=0 \Leftrightarrow m=0$.\\
		Vậy $m=0$ thỏa yêu cầu bài toán.}
\end{vd}
\dongcham{10}
\begin{vd}
	Cho hai điểm $A\left({2,-1,1}\right);B\left({3,-2,-1}\right)$. Tìm điểm $N$ trên trục  $Ox$ cách đều $A$ và $B$.
	\loigiai{$N$ nằm trên trục $Ox$ nên $N\left(x;0;0\right).$\\
		Khi đó, ta có $\overrightarrow {AN}  = \left({x-2;1;-1}\right)$;\quad $\overrightarrow {BN}  = \left({x-3;2;1}\right)$.\\
		Vì $N$ cách đều $A$ và $B$ nên $AN=BN\Leftrightarrow \sqrt {{(x-2)^2}+1+1}  = \sqrt {{(x-3)^2}+4+1} \Leftrightarrow x = 4.$\\
		Suy ra $N(4;0;0)$.}

\end{vd}
\dongcham{10}
\begin{vd}
	\immini{Trong Hóa học, cấu tạo của phân tử ammoniac ($\mathrm{NH}_3$) có dạng hình chóp tam giác đều mà đỉnh là nguyên tử nitrogen ($\mathrm{N}$) và đáy là tam giác $H_1H_2H_3$ với $H_1$, $H_2$, $H_3$ là vị trí của ba nguyên tử hydrogen ($\mathrm{H}$). Góc tạo bởi liên kết $\mathrm{H}-\mathrm{N}-\mathrm{H}$, có hai cạnh là hai đoạn thẳng nối $N$ với hai trong ba điểm $H_1$, $H_2$, $H_3$ (chẳng hạn $\widehat{H_1NH_2}$), gọi là góc liên kết của phân tử $\mathrm{NH}_3$. Góc này xấp xỉ $107^{\circ}$.\\
		Trong không gian $Oxyz$, cho một phân tử $\mathrm{NH}_3$ được biểu diễn bởi hình chóp tam giác đều $N.H_1H_2H_3$ với $O$ là tâm của đáy. Nguyên tử nitrogen được biểu diễn bởi điểm $N$ thuộc trục $Oz$, ba nguyên tử hydrogen ở các vị trí $H_1$, $H_2$, $H_3$ trong đó $H_1(0;-2;0)$ và $H_2H_3$ song song với trục $Ox$ (Hình bên).}{
		\begin{tikzpicture}[>=stealth,line join=round,line cap=round,scale=3]
			\draw (0,0)coordinate(H1)--(-35:0.8)coordinate(H2)--(1,0)coordinate(H3);
			\draw[dashed,black] ($(H2)!.5!(H3)$)coordinate(M1)--($(H1)!2/3!(M1)$)coordinate(H)--(H1)--(H3) (H)--($(H)+(90:1)$)coordinate(N);
			\draw[black,scale=2.5] (H1)node[left]{$H_1$}--(N)node[right]{$N$}--(H3)node[right]{$H_3$} (N)--(H2)node[below]{$H_2$};
			\draw[->] (H)node[below]{$O$}--($(H)+(90:1.2)$)node[right]{$z$};
			\draw[->] (H)--($4*(M1)-4*(H)$)node[above]{$y$};
			\draw[->] (H)--($(H2)-(H3)+(H)$)node[below]{$x$};
			\foreach \diem in {N,H1,H2,H3}\fill[blue] (\diem)circle(0.7pt);
		\end{tikzpicture}}
	\begin{listEX}
		\item Tính khoảng cách giữa hai nguyên tử hydrogen.
		\item Tính khoảng cách giữa hai nguyên tử nitrogen với mỗi nguyên tử hydrogen.
	\end{listEX}
	\loigiai{
		\begin{listEX}
			\item Gọi $x=H_1H_2$, khi đó độ dài $OH_1=x\dfrac{\sqrt{3}}{3}\Leftrightarrow 2=x\dfrac{\sqrt{3}}{3}\Leftrightarrow x=2\sqrt{3}$.
			\item Gọi $y$ là khoảng cách giữa hai nguyên tử nitrogen với mỗi nguyên tử hydrogen; khi đó $NH_2=y$.\\
			Áp dụng định lí cosin ta có $$ H_1H_2^2=NH_1^2+NH_2^2-2\cdot NH_1\cdot NH_2\cos \widehat{H_1NH_2}\Leftrightarrow 2y^2-2y^2\cos 107^\circ=12$$ $$\Leftrightarrow y^2=\dfrac{12}{2-2\cos 107}\Leftrightarrow y=2{,}155 $$
		\end{listEX}
	}
\end{vd}
\dongcham{18}
\begin{vd}%[2H2V2-6]
	\immini{
		Một chậu cây được đặt trên một giá đỡ có bốn chân với điểm đặt $S(0; 0; 20)$ và các điểm chạm mặt đất của bốn chân lần lượt là $A(20; 0; 0)$, $B(0; 20; 0)$, $C(-20; 0; 0)$, $D(0; -20; 0)$ (đơn vị cm). Cho biết trọng lực tác dụng lên chậu cây có độ lớn $40$(N) và được phân bố thành bốn lực $\overrightarrow{F_1}$, $\overrightarrow{F_2}$, $\overrightarrow{F_3}$, $\overrightarrow{F_4}$ có độ lớn bằng nhau như Hình 4.  Tìm toạ độ của các lực
		nói trên (mỗi centimét biểu diễn 1 N).
	}{
		\includegraphics[scale=0.7]{images/luc-1.png}
	}
	\loigiai{
		Tứ giác $ABCD$ có hai đường chéo bằng nhau và vuông góc với nhau tại trung điểm của mỗi đường nên là hình vuông.
		\begin{center}
			\begin{tikzpicture}[scale=0.7, font=\footnotesize,>=stealth]
				\path
				%	Vẽ mp
				(0,0) coordinate (O)
				(-2,-1) coordinate (A)
				(5,-1) coordinate (B)
				(2,1) coordinate (C)
				(-5,1) coordinate (D)
				($(O)+(0,6)$)coordinate (S)
				($(A)!0.5!(S)$)coordinate (A')
				($(B)!0.5!(S)$)coordinate (B')
				($(C)!0.5!(S)$)coordinate (C')
				($(D)!0.5!(S)$)coordinate (D')
				($(B')!0.5!(D')$)coordinate (O')
				;
				\draw (A)--(D)--(S)--(A)--(B)--(S) (D')--(A')--(B');
				\draw[dashed] (A)--(C)--(B)--(D)--(C)--(S)--(O) (C')--(B')--(D')--(C')--(A');
				\foreach \x/\g in {A/-90,B/0,C/0,D/180,S/90,O/30,O'/-30,A'/200,B'/0,C'/20,D'/180}\draw[fill=black] (\x) circle (.05) +(\g:.5)node{\footnotesize$\x$};
			\end{tikzpicture}
		\end{center}
		Ta có $\overrightarrow{SA} = (20; 0; -20)$, $\overrightarrow{SB} = (0; 20; -20)$, $\overrightarrow{SC} = (-20; 0; -20)$ , $\overrightarrow{SD} = (0; -20; -20)$.\\
		Suy ra $SA = SB = SC = SD = 20\sqrt{2}$. Do đó $S.ABCD$ là hình chóp tứ giác đều.
		Các vectơ $\overrightarrow{F_1}$, $\overrightarrow{F_2}$, $\overrightarrow{F_3}$, $\overrightarrow{F_4}$ có điểm đầu tại $S$ và điểm cuối lần lượt là $A'$, $B'$, $C'$, $D'$.\\
		Ta có $SA' = SB' = SC' = SD'$ nên $S.A'B'C'D'$ cũng là hình chóp tứ giác đều.\\
		Gọi $\overrightarrow{F} $ là trọng lực tác dụng lên chậu cây và $O'$ là tâm của hình vuông $A'B'C'D'$. Ta có
		\[ \overrightarrow{F} =\overrightarrow{F_1} + \overrightarrow{F_2}+ \overrightarrow{F_3}+ \overrightarrow{F_4} = \overrightarrow{SA'} + \overrightarrow{SB'}+ \overrightarrow{SC'}+ \overrightarrow{SD'} = 4\overrightarrow{SO'}.\]
		Ta có $\left| \overrightarrow{F} \right| =40$, suy ra $\left| \overrightarrow{SO'} \right| = SO' = 10$.
		Do tam giác $SO'A'$ vuông cân nên $SA' = \sqrt{2}SO' = 10\sqrt{2}$.\\
		Suy ra $\overrightarrow{F_1} = \overrightarrow{SA'} = \dfrac{1}{2}\overrightarrow{SA}  = (10; 0; -10)$.
		Chứng minh tương tự ta cũng có\\
		\[ \overrightarrow{F_2} = \dfrac{1}{2}\overrightarrow{SB}  = (0; 10; -10), \overrightarrow{F_3} = \dfrac{1}{2}\overrightarrow{SC}  = (-10; 0; -10), \overrightarrow{F_4} = \dfrac{1}{2}\overrightarrow{SD}  = (0; -10; -10). \]
	}
\end{vd}
\dongcham{24}
\BTTN
\Opensolutionfile{ans}[ans/2H2-B3-d2-1]

\begin{ex}
	Tích vô hướng của hai vectơ $\vec{u}=(3;0;1)$ và $\vec{v}=(2;1;0)$ là
	\choice
	{$0$}
	{\True$6$}
	{$8$}
	{$-6$}
	\loigiai
	{
		$$\vec{u}\cdot\vec{v}=6+0+0=6.$$
	}
\end{ex} \dongcham{4}

\begin{ex}
	Tích vô hướng của hai vectơ $\vec{u} = \vec{i} + 2 \vec{j} - \vec{k}$ và $ \vec{v} = (0;1; -2)$ bằng
	\choice
	{$ -4 $}
	{$ 0 $}
	{\True $ 4 $}
	{$ -2 $}
	\loigiai{
		Ta có $ \vec{u}=(1;2;-1) $.\\
		Suy ra $ \vec{u}\cdot \vec{v}=1\cdot0+2\cdot1+(-1)\cdot(-2)=4 $.
	}
\end{ex} \dongcham{4}


\begin{ex}
	Cho các véc-tơ $\vec{a}=(1;2;1)$ và $\vec{b}=(2;2;1)$. Tính tích vô hướng $\vec{a} \cdot \left(\vec{a}-\vec{b}\right)$.
	\choice
	{\True $-1$}
	{$-2$}
	{$2$}
	{$1$}
	\loigiai{
		Ta có: $\left(\vec{a}-\vec{b}\right)=(-1;0;0) \Rightarrow \vec{a} \cdot \left(\vec{a}-\vec{b}\right)=1 \cdot (-1)+2 \cdot 0+1 \cdot 0=-1$.}
\end{ex} \dongcham{5}

\begin{ex}
	Một thiết bị thăm dò đáy biển được đẩy bởi một lực $\overrightarrow{f} = (5; 4; -2)$ (đơn vị: N) giúp thiết bị thực hiện độ dời $\overrightarrow{a} = (70; 20; -40)$ (đơn vị: m). Tính công sinh bởi lực $\overrightarrow{f}$.
	\choice
	{$480\,(\text{J})$}
	{$530\,(\text{J})$}
	{\True $510\,(\text{J})$}
	{$500\,(\text{J})$}
	\loigiai{
		Công sinh bởi lực $\overrightarrow{f}$ là
		\[ A = \left| \overrightarrow{f} \right| \cdot \left| \overrightarrow{a} \right|\cdot \cos \left(\overrightarrow{f}, \overrightarrow{a}\right) = \overrightarrow{f} \cdot \overrightarrow{a} = 5\cdot 70 + 4\cdot 20 + (-2)\cdot (-40) = 510(\text{J}).\]
	}
\end{ex} \dongcham{5}

\begin{ex}
	Góc giữa hai véc-tơ $ \vec{i} $ và $ \vec{u}=(-\sqrt{3};0,;1) $ bằng
	\choice
	{$ 60^\circ $}
	{$ 120^\circ $}
	{\True $ 150^\circ $}
	{$ 30^\circ $}
	\loigiai{
		$\cos \left(\vec{i},\vec{u}\right)=\dfrac{\vec{i}\cdot \vec{u}}{\vert \vec{i} \vert \cdot \vert \vec{u} \vert}=\dfrac{1\cdot (-\sqrt3)}{1\cdot\sqrt{3+1}}=-\dfrac{\sqrt{3}}{2}$.\\
		Vậy góc của hai véc-tơ đã cho bằng $ 150^\circ $.
	}
\end{ex} \dongcham{5}

\begin{ex}
	Cho hai véc-tơ $ \vec{u}=(-1;1;0) $ và $ \vec{v}=(0;-1;0) $. Góc hợp bởi hai véc-tơ $ \vec{u} $ và $ \vec{v} $ bằng
	\choice
	{$ 60^\circ $}
	{$ 45^\circ $}
	{\True $ 135^\circ $}
	{$ 120^\circ $}
	\loigiai{
		$\cos \left(\vec{u},\vec{v}\right)=\dfrac{\vec{u}\cdot \vec{v}}{\vert \vec{u} \vert \cdot \vert \vec{v} \vert}=\dfrac{(-1)\cdot 0+1\cdot (-1)+0 \cdot 0}{\sqrt{(-1)^2+1^2+0^2}\sqrt{0^2+(-1)^2+0^2}}=-\dfrac{1}{\sqrt{2}}$.\\
		Vậy góc của hai véc-tơ đã cho bằng $ 135^\circ $.}
\end{ex} \dongcham{6}

\begin{ex}
	Cho hai véc-tơ $ \vec{a}(-2;-3;1) $ và $ \vec{b}(1;0;1) $. Tính $ \cos(\vec{a},\vec{b}) $.
	\choice
	{\True $ \cos(\vec{a},\vec{b})=-\dfrac{1}{2\sqrt{7}} $}
	{$ \cos(\vec{a},\vec{b})=-\dfrac{3}{2\sqrt{7}} $}
	{$ \cos(\vec{a},\vec{b})=\dfrac{1}{2\sqrt{7}} $}
	{$ \cos(\vec{a},\vec{b})=\dfrac{3}{2\sqrt{7}} $}
	\loigiai{
		Ta có $ \cos(\vec{a},\vec{b})=\dfrac{(-2)\cdot 1+(-3)\cdot 0+1\cdot 1}{\sqrt{14}\cdot \sqrt{2}}=-\dfrac{1}{2\sqrt{7}} $.
	}
\end{ex} \dongcham{6}

\begin{ex}
	Cho $\vec{a}=(3;2;1)$, $\vec{b}=(-2;2;-4)$. Giá trị của $\left| \vec{a}-\vec{b} \right|$ bằng
	\choice
	{\True$5\sqrt{2}$}
	{$50$}
	{$2\sqrt{5}$}
	{$3$}
	\loigiai
	{
		Gọi $\vec{c}=\vec{a}-\vec{b}=(5;0-5)\Rightarrow \left| \vec{c} \right|=\sqrt{5^2+(-5)^2}=5\sqrt{2}$.}
\end{ex} \dongcham{6}

\begin{ex}
	Cho hai véc-tơ $\vec{u}=(-1;0;2)$ và $\vec{v}=(x;-2;1)$. Biết rằng $\vec{u}\cdot \vec{v}=4$. Khi đó $|\vec{v}|$ bằng
	\choice
	{$\sqrt{21}$}
	{$2$}
	{\True $3$}
	{$5$}
	\loigiai{
		Ta có $\vec{u}\cdot \vec{v}=-x+2=4\Leftrightarrow x=-2$.\\
		Vậy $|\vec{v}|=3$.
	}
\end{ex} \dongcham{6}

\begin{ex}%[2H3Y1-2]%
	Tìm số thực $a$ để vec-tơ $\vec{u}=(a;0;1)$ vuông góc với vec-tơ $\vec{v}=(2;-1;4)$.
	\choice
	{\True $a=-2$}
	{$a=-4$}
	{$a=4$}
	{$a=2$}
	\loigiai{
		Ta có $\vec{u}\perp\vec{v}\Leftrightarrow \vec{u}\cdot\vec{v}=0\Leftrightarrow 2a+0(-1)+4=0\Leftrightarrow a=-2.$
	}
\end{ex} \dongcham{6}

\begin{ex}
	Tìm $x$ để hai véc-tơ $\vec{a}=(x;x-2;2)$ và $\vec{b}=(x;1;-2)$ vuông góc với nhau.
	\choice
	{$x=3$}
	{$x=1$}
	{\True $\hoac{&x=2\\ &x=-3}$}
	{$\hoac{&x=-2\\&x=3}$}
	\loigiai{
		Hai véc-tơ đã cho vuông góc khi $0=\vec{a}\cdot \vec{b}=x^2+x-2-4$ hay $x=2\ \text{hoặc}\ x=-3$.
	}
\end{ex} \dongcham{6}

\begin{ex}
	Cho hai véc-tơ $\vec{u}=(1;-2;1)$ và $\vec{v}=(2;1;-1)$. Véc-tơ nào dưới đây vuông góc với cả hai véc-tơ $\vec{u}$ và $\vec{v}$?
	\choice
	{\True $\vec{w_2}=(1;3;5)$}
	{$\vec{w_3}=(1;-4;7)$}
	{$\vec{w_4}=(1;4;7)$}
	{$\vec{w_1}=(1;-3;5)$}
	\loigiai{
		Ta có $\vec{u}\cdot\vec{w_2}=0$, $\vec{v}\cdot\vec{w_2}=0$. Do đó $\vec{w_2}$ thỏa mãn đề bài.
	}
\end{ex} \dongcham{6}

\begin{ex}
	Tích có hướng của hai véc-tơ $\vec{a}=(-1;2;0)$ và $\vec{b}=(0;4;-3)$ có tọa độ là
	\choice
	{$(-6;3;-4)$}
	{$(6;-3;4)$}
	{$(6;3;4)$}
	{\True $(-6;-3;-4)$}
	\loigiai{
		Ta có
		$\left[\vec{a},\vec{b}\right]=\left(\left|\begin{array}{lr}
					2 & 0 \\ 4 & -3
				\end{array}\right|;\left|\begin{array}{lr}
					0 & -1 \\ -3 & 0
				\end{array}\right|; \left|\begin{array}{lr}
					-1 & 2 \\ 0 & 4
				\end{array}\right|\right)=(-6;-3;-4)$.
	}
\end{ex} \dongcham{6}

\begin{ex}
	Cho $ A(2;1;4), B(-2;2;-6), C(6;0;-1) $. Tính tích vô hướng $ \vec{AB}\cdot \vec{AC} $.
	\choice
	{$\vec{AB}\cdot \vec{AC}=67$}
	{$\vec{AB}\cdot \vec{AC}=-67$}
	{\True$\vec{AB}\cdot \vec{AC}=33$}
	{$\vec{AB}\cdot \vec{AC}=65$}
	\loigiai{
		Ta có: $ \heva{& \vec{AB}=(-4;1;-10) \\ & \vec{AC}=(4;-1;-5).} $\\
		$ \vec{AB}\cdot \vec{AC}=(-4)\cdot6+1\cdot (-1)+(-10)\cdot (-5)=33 $.
	}
\end{ex} \dongcham{6}

\begin{ex}
	Cho $ A(1;-2; 3)$, $ B(2;-4; 1)$, $ C(2; 0; 2)$, khi đó tích vô hướng $\vec{AB}\cdot\vec{AC}$ bằng
	\choice
	{$4$}
	{\True $-1$}
	{$7$}
	{$-5$}
	\loigiai
	{
		Ta có $\vec{AB}=(1;-2;-2)$ và $\vec{AC}=(1; 2;-1)$.\\
		Vì vậy $\vec{AB}\cdot\vec{AC}=1\cdot 1+(-2)\cdot 2+(-2)\cdot (-1)=-1 $.
	}
\end{ex} \dongcham{5}

\begin{ex}%
	Cho tam giác $ABC$ với $A(8; 9; 2)$, $B(3; 5; 1)$, $C(11; 10; 4)$. Số đo góc $A$ của tam giác $ABC$ là
	\choice
	{$60^\circ$}
	{\True $150^\circ$}
	{$30^\circ$}
	{$120^\circ$}
	\loigiai{
		Ta có $\widehat{BAC} = \left( \vec{AB}; \vec{AC} \right)$, $\vec{AB} = (-5; -4; -1)$, $\vec{AC} = (3;1;2)$. Ta có
		$$\cos \left( \vec{AB}; \vec{AC} \right) = \dfrac{\vec{AB} \cdot \vec{AC}}{\left| \vec{AB} \right| \cdot \left| \vec{AC} \right|} = \dfrac{-21}{\sqrt{42} \cdot \sqrt{14}} = - \dfrac{\sqrt{3}}{2} \Rightarrow \widehat{BAC} = \left( \vec{AB}; \vec{AC} \right) = 150^\circ.$$
	}
\end{ex} \dongcham{6}

\begin{ex}
	Cho điểm $A(3;-1;5)$, $B(m;2;7)$. Tìm tất cả các giá trị của $m$ để độ dài đoạn $AB=7$.
	\choice
	{$m=3$ hoặc $m=-3$}
	{\True $m=9$ hoặc $m=-3$}
	{$m=-3$ hoặc $m=-9$}
	{$m=9$ hoặc $m=3$}
	\loigiai{
		$$AB=7\Leftrightarrow \sqrt{\left(m-3\right)^2+3^2+2^2}=7\Leftrightarrow (m-3)^2=36\Leftrightarrow \hoac{&m-3=6\\&m-3=-6}\Leftrightarrow \hoac{&m=9\\&m=-3.}$$
	}
\end{ex} \dongcham{5}



\begin{ex}
	Cho ba điểm $A(3;2;8)$, $B(0;1;3)$ và $C(2;m;4)$. Tìm $m$ để tam giác $ABC$ vuông tại $B$.
	\choice
	{$m=4$}
	{\True $m=-10$}
	{$m=25$}
	{$m=-1$}
	\loigiai{
		Tam giác $ABC$ vuông tại $B$ tương đương với $\vec{BA}\cdot\vec{BC}=\vec{0}$.\\
		Ta có $\vec{BA}=(3;1;5)$, $\vec{BC}=(2;m-1;1)$.\\
		Nên $\vec{BA}\cdot\vec{BC}=0\Leftrightarrow 3\cdot 2+(m-1)+5\cdot 1=0\Leftrightarrow m=-10$.}
\end{ex} \dongcham{5}

\begin{ex}
	Cho ba điểm $M(2;3;-1)$, $N(-1;1;1)$ và $P(1;m-1;2)$. Tìm $m$ để tam giác $MNP$ vuông tại $N$.
	\choice
	{\True $ m=0 $  }
	{ $ m=-4 $}
	{$ m=2 $ }
	{ $ m=-6 $}
	\loigiai{
		$\vec{MN}(-3;-2;2)$; $\vec{NP}(2;m-2;1)$.\\
		Tam giác $MNP$ vuông tại $N \Leftrightarrow \vec{MN} \cdot \vec{NP}=0 \Leftrightarrow -6-2(m-2)+2=0 \Leftrightarrow m-2=-2 \Leftrightarrow m=0$.}
\end{ex} \dongcham{5}

\begin{ex}%[2H2V2-4]%[2H2H2-4]
	Cho tam giác $ABC$ có $A(7; 3; 3)$, $B(1; 2; 4)$, $C(2; 3; 5)$. Tìm toạ độ điểm $H$ là chân đường cao kẻ từ $A$ của tam giác $ABC$.
	\choice
	{\True $H(3; 4; 6)$}
	{$H(-3; 4; 7)$}
	{$H(2; 4; 1)$}
	{$H(2; -4; 3)$}
	\loigiai{
		Ta có $\overrightarrow{BC} = (1; 1; 1)$.\\
		Gọi $H(x; y; z)$ là chân đường cao của tam giác $ABC$ kẻ từ $A$.\\
		Suy ra $\overrightarrow{BH} = (x-1; y-2; z-4)$.\\
		$\overrightarrow{BH}$ cùng phương với $\overrightarrow{BC}$, do đó $x-1 = t$; $y-2 = t$; $z-4=t$. Suy ra $H(1+t; 2+t; 4+t)$.\\
		Ta có $\overrightarrow{AH} = (x_H-x_A; y_H-y_A; z_H-z_A) = (t-6; t-1; t+1)$.\\
		$\overrightarrow{AH} \perp \overrightarrow{BC} \Leftrightarrow \overrightarrow{AH}\cdot \overrightarrow{BC} = 0 \Leftrightarrow t-6 + t-1 + t+ 1 =0\Leftrightarrow 3t =6 \Leftrightarrow t =2$.\\
		Suy ra $H(3; 4; 6)$.
	}
\end{ex} \dongcham{7}

\begin{ex}
	Cho hai điểm $A(1;1;0)$, $B(2;-1;2)$. Gọi $M(0;0;z)$ là điểm thuộc trục $Oz$ sao cho $MA^2+MB^2$ nhỏ nhất. Khẳng định nào sau đây là đúng?
	\choice
	{\True $z \in (0;1]$}
	{$z \in (1;2]$}
	{$z \in (-1;0]$}
	{$z \in (-2;-1]$}
	\loigiai{Gọi $M(0;0; z)$.Khi đó $MA^2+MB^2=2z^2-4z+11=2(z-1)^2+9 \geq 9$.\\
		Dấu $"="$ xảy ra khi và chỉ khi $z=1$. Do đó, $ M(0;0;1)$.}

\end{ex} \dongcham{11}

\Closesolutionfile{ans}

\BTTF
\Opensolutionfile{ans}[ans/2H2-B3-d2-2]

\begin{ex}
	Cho ba vec-tơ $ \overrightarrow a=(-1;1;0)$, $ \overrightarrow b=(1;1;0)$ và $ \overrightarrow c=(1;1;1)$.
	\choiceTF
	{$\left| {\overrightarrow a } \right| = 2$}
	{\True $\left| {\overrightarrow c } \right| = \sqrt 3 $}
	{$\cos\left(\vec{a},\vec{c} \right)=\dfrac{2}{\sqrt{5}}$}
	{ $\overrightarrow b \perp \overrightarrow c$}
	\loigiai{
		\begin{enumerate}[a)]
			\item $\left| \overrightarrow {a} \right| = \sqrt{(-1)^2+1^2}=\sqrt{2}$.
			\item $\left|\overrightarrow {c} \right| = \sqrt{1^2+1^2+1^2}=\sqrt{3}$
			\item $\cos\left(\vec{a},\vec{c} \right)=\dfrac{\vec{a}\cdot \vec{c}}{|\vec{a}|.|\vec{c}|}=0$
			\item $\vec{b} \cdot \vec{c}=2$, suy ra $\vec{b}$ không vuông $\vec{c}$.
		\end{enumerate}
	}
\end{ex} \dongcham{10}

\begin{ex}
	Cho hai vectơ $\vec{u}=(0;2;3)$ và $\vec{v}=(m-1;2m;3)$.
	\choiceTF
	{\True $\big|\vec{u}\big|=\sqrt{13}$}
	{$\big|\vec{u}\big|=\big|\vec{v}\big| \Leftrightarrow m=-\dfrac{3}{5}$}
	{\True $\vec{u}=\vec{v} \Leftrightarrow m=1$}
	{$\vec{u}\perp\vec{v} \Leftrightarrow m=\dfrac{9}{4}$}
	\loigiai{
		\begin{enumerate}[a)]
			\item $\big|\vec{u}\big|=\sqrt{0^2+2^2+3^2}=\sqrt{13}$
			\item $\big|\vec{u}\big|=\big|\vec{v}\big|\Leftrightarrow \sqrt{13}=\sqrt{(m-1)^2+4m^2+9} \Leftrightarrow 5m^2-2m-3=0 \Leftrightarrow m=1$ hoặc $m=-\dfrac{3}{5}$.
			\item khi $m=1$ thì $\vec{v}=(0;2;3)$. Suy ra $\vec{u}=\vec{v}$.
			\item $\vec{u} \perp \vec{u} \Leftrightarrow 4m+9=0 \Leftrightarrow m=-\dfrac{9}{4}$.
		\end{enumerate}
	}
\end{ex} \dongcham{10}
%Câu 1
\begin{ex}
	Trong không gian với hệ trục tọa độ $Oxyz$, cho ba vectơ $\vec{a}(1;2;3)$, $\vec{b}(2;2;-1)$, $\vec{c}(4;0;-4)$.
	\choiceTF
	{\True Tọa độ của vectơ $\vec{x}=\vec{a}+\vec{b}$ là $\vec{x}=(3;4;2)$}
	{Tọa độ của vectơ $\vec{y}=\vec{a}+\vec{c}$ là $\vec{y}=(5;2;1)$}
	{Tọa độ của vectơ $\vec{z}=\vec{b}+\vec{c}$ là $\vec{z}=(6;-2;-5)$}
	{\True Vectơ $\vec{k}=(7;4;-2)$ thỏa mãn đẳng thức $\vec{k}=\vec{a}+\vec{b}+\vec{c}$}
	\loigiai{
		\begin{enumerate}[a)]
			\item $\vec{x}=\vec{a}+\vec{b}=(3;4;2)$.
			\item $\vec{y}=\vec{a}+\vec{c}=(5;2;-1)$.
			\item $\vec{z}=\vec{b}+\vec{c}=(6;2;-5)$.
			\item $\vec{k}=\vec{a}+\vec{b}+\vec{c}=(7;4;-2)$.
		\end{enumerate}
	}
\end{ex}
%Câu 2
\begin{ex}
	Trong không gian $Oxyz$, cho hai vectơ $\vec{a}(1;-1;5)$, $\vec{b}(3;2;-1)$.
	\choiceTF
	{\True $\vec{a}+\vec{b}\ne \vec{0}$}
	{$\vec{a}-\vec{b}=(-2;-3;4)$}
	{$\vec{v}=\vec{b}-\vec{a}$ có tung độ âm}
	{\True Xét $\vec{x}$ thỏa $\vec{a}-\vec{x}=\vec{b}$. Hoành độ của vectơ $\vec{x}$ thuộc khoảng $(-3;1)$}
	\loigiai{
		\begin{enumerate}
			\item $\vec{a}+\vec{b}=(4;1;4)$.
			\item $\vec{a}-\vec{b}=(-2;-3;6)$.
			\item $\vec{b}-\vec{a}=\left(2;3;-4\right)$.
			\item $\vec{a}-\vec{x}=\vec{b}\Leftrightarrow \vec{x}=\vec{a}-\vec{b}=(-2;-3;6)$. Suy ra hoành độ của vectơ $\vec{x}$ là $-2\in (-3;1)$.
		\end{enumerate}
	}
\end{ex}
%Câu 4
\begin{ex}
	Trong không gian $Oxyz$, cho điểm $D(4;-1;3)$ và các điểm $M$, $N$, $P$ lần lượt thuộc các trục
	$Ox$, $Oy$, $Oz$ sao cho $DM$, $DN$, $DP$ đôi một vuông góc với nhau
	\choiceTF
	{Tung độ của điểm $N$ bằng $13$}
	{Cao độ của điểm $P$ bằng $\dfrac{13}{4}$}
	{\True $V_{DMNP}>29$}
	{Gọi $\vec{x}$ là vectơ thỏa $\vec{x} \cdot \vec{DM}=1$; $\vec{x} \cdot \vec{DN}=2$; $\vec{x} \cdot \vec{DP}=-3$ thì tổng hoành độ, tung độ và cao độ của vectơ $\vec{x}$ thuộc khoảng $(3;7)$}
	\loigiai{
		\begin{itemize}
			\item Gọi $M(a;0;0)$, $N(0;b;0)$, $P(0;0;c)$.\\
			      $\vec{DM}=(a-4;1;-3)$, $\vec{DN}=(-4;b+1;-3)$, $\vec{DP}=(-4;1;c-3)$\\
			      Ta có $DM$, $DN$, $DP$ đôi một vuông góc với nhau nên \\
			      $\heva{& \vec{DM} \cdot \vec{DN}=0 \\& \vec{DM} \cdot \vec{DP}=0 \\& \vec{DN} \cdot \vec{DP}=0}\Leftrightarrow \heva{& -4(a-4)+b+1+9=0 \\& -4(a-4)+1-3(c-3)=0 \\& 16+b+1-3(c-3)=0}\Leftrightarrow \heva{& -4a+b=-26 \\& -4a-3c=-26 \\& b-3c=-26}\Leftrightarrow \heva{& a=\dfrac{13}{4} \\& b=-13 \\& c=\dfrac{13}{3}}$.
			\item ${{V}_{DMNP}}=\dfrac{1}{6}DM \cdot DN \cdot DP=\dfrac{1}{6} \cdot \dfrac{13}{4} \cdot 13 \cdot \dfrac{13}{3}=\dfrac{2197}{72}>29$.
			\item Gọi $\vec{x}=\left(m;n;p\right)$\\
			      $\vec{DM}=\left(-\dfrac{3}{4};1;-3\right);\vec{DN}=(-4;-12;-3);\vec{DP}=\left(-4;1;\dfrac{4}{3}\right)$\\
			      $\heva{& \vec{x} \cdot \vec{DM}=1 \\& \vec{x} \cdot \vec{DN}=2 \\& \vec{x} \cdot \vec{DP}=-3}\Leftrightarrow \heva{& -\dfrac{3}{4}m+n-3p=1 \\& -4m-12n-3p=2 \\& -4m+n+\dfrac{4}{3}p=-3}\Leftrightarrow \heva{& m=\dfrac{88}{169} \\& n=-\dfrac{35}{169} \\& p=-\dfrac{90}{169}}$\\
			      $m+n+p=\dfrac{-37}{169}$.
		\end{itemize}
	}
\end{ex}

\begin{ex}
	Cho tam giác $ABC$ có $ A(1;2;0) $, $ B(0;1;1) $, $ C(2;1;0) $.
	\choiceTF
	{\True  Tam giác $ABC$ vuông tại $A$}
	{Chu vi tam giác là $ \sqrt{7}+\sqrt{3}+\sqrt{2} $}
	{Diện tích tam giác $ ABC $ là $ \sqrt{6} $}
	{\True Tâm đường tròn ngoại tiếp tam giác $ ABC $ là $ I\left(1;1;\dfrac{1}{2}\right) $}
	\loigiai{
		Ta có $ \overrightarrow{AB}=(-1;-1;1) \Rightarrow AB=\sqrt{3}$, $ \overrightarrow{AC}=(1;-1;0) \Rightarrow AC=\sqrt{2}$,\\ $\overrightarrow{BC}=(2;0;-1) \Rightarrow BC=\sqrt{5}$.
		\begin{enumerate}[a)]
			\item $ \overrightarrow{AB}\cdot \overrightarrow{AC}=0 $ do đó $ AB\perp AC $, tam giác $ ABC $ vuông tại $ A $.
			\item Chu vi của tam giác là $ AB+AC+BC=\sqrt{3}+\sqrt{2}+\sqrt{5} $.
			\item Diện tích là\\ $ S=\dfrac{1}{2}\cdot AB\cdot AC=\dfrac{\sqrt{6}}{2} $
			\item Tâm đường tròn ngoại tiếp là trung điểm của $ BC $ có tọa độ $ I\left(1;1;\dfrac{1}{2}\right)$.
		\end{enumerate}
	}
\end{ex} \dongcham{25}

\begin{ex}
	Hình minh họa sơ đồ một ngôi nhà trong hệ trục tọa độ $Oxyz$, trong đó nền nhà, bốn bức tường và hai mái nhà đều là hình chữ nhật.
		\choiceTF
		{Tọa độ của các điểm $A(5;0;0)$}
		{Tọa độ của các điểm $H(0;5;3)$}
		{Góc nhị diện có cạnh là đường thẳng $FG$, hai mặt lần lượt là $(FGQP)$ và $(FGHE)$ gọi là góc dốc của mái nhà. Số đo của góc dốc của mái nhà bằng $26{,}6^\circ$ (làm tròn kết quả đến hàng phần mười của độ)}
		{Chiều cao của ngôi nhà là 4}
	\begin{center}
		\begin{tikzpicture}[scale=1, font=\footnotesize, line join=round, line cap=round, >=stealth]
			\path
			(0,0) coordinate (O) node[below]{$O(0;0;0)$}
			(4,0) coordinate (A) node[below]{$A$}
			(0,3) coordinate (E) node[left]{$E(0;0;3)$}
			($(A)+(E)-(O)$) coordinate (F)node[right]{$F$}
			(A)+(1,2) coordinate (B) node[right]{$B(4;5;0)$}
			($(O)+(B)-(A)$) coordinate (C) node[left]{$C$}
			($(C)+(E)-(O)$) coordinate (H) node[above left]{$H$}
			($(H)+(B)-(C)$) coordinate (G) node [right]{$G(4;5;3)$}
			($(O)!.6!(A)$) coordinate (x)
			(x)+(0,4.5) coordinate (P) node[right]{$P(2;0;4)$}
			($(P)+(H)-(E)$) coordinate (Q) node[above]{$Q(2;5;4)$}
			;
			\draw[->] (O)--(E)--($(E)+(90:1)$)node[above]{$z$};
			\draw[->] (O)--(A)--($(A)+(0:1)$) node[below]{$x$};
			\draw[dashed,->] (O)--(C)--($(O)!1.3!(C)$) node[above]{$y$};
			\draw (A)--(B)--(G)--(Q)--(H)--(E)--(F)--cycle (E)--(P)--(F) (P)--(Q) (F)--(G);
			\draw[dashed] (C)--(B) (C)--(H)--(G);
		\end{tikzpicture}
	\end{center}
	\loigiai{
		\begin{enumerate}
			\item Vì nền nhà là hình chữ nhật nên tứ giác $OABC$ là hình chữ nhật, suy ra $x_A=x_B=4$, $y_C=y_B=5$. Do $A$ nằm trên trục $Ox$ nên tọa độ điểm $A$ là $(4;0;0)$.
			\item Tường nhà là hình chữ nhật, suy ra $y_H=y_C=5$, $z_H=z_E=3$. Do $H$ nằm trên mặt phẳng $(Oyz)$ nên tọa độ điểm $H$ là $(0;5;3)$.\\
			\item Để tính góc dốc của mái nhà, ta đi tính số đo góc nhị diện có cạnh là đường thẳng $FG$, hai mặt phẳng lần lượt là $(FGQP)$ và $(FGHE)$. Do mặt phẳng $(Ozx)$ vuông góc với hai mặt phẳng $(FGQP)$ và $(FGHE)$ nên góc $PFE$ là góc phẳng nhị diện ứng với góc nhị diện đó. \\
			      Ta có $\vec{FP}=(-2;0;1)$, $\vec{FE}=(-4;0;0)$.\\
			      Suy ra
			      \begin{eqnarray*}
				      \cos \widehat{PFE}&=&\cos \left(\vec{FP},\vec{FE}\right)=\dfrac{\vec{FP}\cdot \vec{FE}}{\left|\vec{FP}\right|\cdot \left|\vec{FE}\right|}\\
				      &=&\dfrac{(-2)\cdot (-4)+0\cdot 0+1\cdot 0}{\sqrt{(-2)^2+0^2+1^2}\cdot \sqrt{(-4)^2+0^2+0^2}}=\dfrac{2\sqrt{5}}{5}.
			      \end{eqnarray*}
			      Do đó, $\widehat{PFE}\approx 26{,}^\circ$. Vậy góc dốc của mái nhà khoảng $26{,}6^\circ$.
			\item Chiều cao bằng cao độ của điểm $P$. Suy ra $h=4$.
		\end{enumerate}
	}
\end{ex} \dongcham{45}
\BTTL
\begin{ex}
	Trong không gian $Oxyz$, cho hai vectơ $\vec{a}=(1;2;-3);\vec{b}=(-1;-2;z)$. Tìm giá trị $z$ sao cho	$\vec{a}+\vec{b}=\vec{0}$
	\loigiai{
		\SA{3}
		Ta có: $\vec{a}+\vec{b}=\left(0;0;z-3\right)$.\\
		$\vec{a}+\vec{b}=\vec{0}\Leftrightarrow z-3=0\Leftrightarrow z=3$.\\
		Vậy $z=3$.
	}
\end{ex}
%Câu 2
\begin{ex}
	Trong không gian $Oxyz$, cho hai vectơ $\vec{a}=2\vec{i}-3\vec{j}+6\vec{k}$ và $\vec{b}=6\vec{j}+\vec{k}$. Khi đó độ dài của
	$\vec{a}-2\vec{b}$ (làm tròn đến hàng phần mười)
	\loigiai{
		\SA{15,7}
		Ta có: $\vec{a}=2\vec{i}-3\vec{j}+6\vec{k}\Rightarrow \vec{a}=(2;-3;6)$\\
		$\vec{b}=6\vec{j}+\vec{k}\Rightarrow \vec{b}=(0;6;1)$\\
		Khi đó: $\vec{a}-2\vec{b}=(2;-15;4)\Rightarrow \left| \vec{a}-2\vec{b} \right|=7\sqrt{5}\approx 15{,}7$
	}
\end{ex}
%Câu 4
\begin{ex}
	Trong không gian $Oxyz$, cho các vectơ $\vec{a}=(1;0;-2),\text{ }\vec{b}=(-2;1;3)$,$\vec{c}=(3;2;-1)$, $\vec{d}=(9;0;-11)$ và $3$ số thực $m,n,p$ thỏa $m \cdot \vec{a}+n \cdot \vec{b}+p\vec{c}=\vec{d}$. Tính giá trị biểu thức $T=m+n+p$.
	\loigiai{
		\SA{1}
		Ta có: $m \cdot \vec{a}+n \cdot \vec{b}+p\vec{c}=\left(m-2n+3p;n+2p;-2m+3n-p\right)$, $\vec{d}=\left(9;0;-11\right)$.\\
		$m \cdot \vec{a}+n \cdot \vec{b}+p\vec{c}=\vec{d}\Leftrightarrow \heva{& m-2n+3p=9 \\& n+2p=0 \\& -2m+3n-p=-11} \Leftrightarrow \heva{& m=2 \\& n=-2 \\& p=1.} $\\
		Vậy $T=m+n+p=1$.
	}
\end{ex}
\Closesolutionfile{ans}

%Chương III. Mẫu số liệu ghép nhóm
%%Bài 1
% \setcounter{section}{0}
\section{KHOẢNG BIẾN THIÊN, KHOẢNG TỨ PHÂN VỊ CỦA MSL GHÉP NHÓM}
\subsection{LÝ THUYẾT CẦN NHỚ}
\subsubsection{Khoảng biến thiên}
\begin{enumerate}[\iconMT] 
	\item \indam{Định nghĩa:} Xét mẫu số liệu ghép nhóm được cho ở bảng sau:
	\begin{center}
		\begin{tikzpicture}
			\matrix[matrix of nodes,nodes in empty cells,
			row sep=-\pgflinewidth,column sep=-\pgflinewidth,
			nodes={minimum height=7mm,minimum width=20mm,draw=black,anchor=center},
			column 1/.style={nodes={minimum width=24mm,color=black}},
			row 1/.style={nodes={fill=cyan!10}},
			row 2/.style={nodes={minimum height=7mm}},
			]{
				Nhóm &$[u_1;u_2)$&$[u_1;u_2)$&\dots&$[u_k;u_{k+1})$\\ 
				\node[align=center]{Tần số}; &$n_1$&$n_2$&\dots&$n_k$\\
			};
		\end{tikzpicture}
	\end{center}
	Nếu $n_1$ và $n_k$ cùng khác $0$ thì khoảng biến thiên của mẫu số liệu ghép nhóm được tính theo công thức
		\boxmini{$R=u_{k+1}-u_1$}
	% \item \indam{Ý nghĩa:}
	% \begin{listEX}[1]
	% 	\item [\iconCH] Khoảng biến thiên của mẫu số liệu ghép nhóm là giá trị xấp xỉ khoảng biến thiên của mẫu số liệu gốc và có thể dùng để đo mức độ phân tán của mẫu số liệu. Khoảng biến thiên càng lớn thì mẫu số liệu càng phân tán.
	% 	\item [\iconCH] Trong các đại lượng đo mức độ phân tán của mẫu số liệu ghép nhóm, khoảng biến thiên là đại lượng dễ hiểu, dễ tính toán. Tuy nhiên, do khoảng biến thiên chỉ sử dụng hai giá trị $u_1$ và $u_{m+1}$ của mẫu số liệu nên đại lượng đó dễ bị ảnh hưởng bởi các giá trị bất thuờng.
	% \end{listEX}
\end{enumerate}

\subsubsection{Khoảng tứ phân vị}
\begin{enumerate}[\iconMT] 
	\item \indam{Định nghĩa:}
	Khoảng tứ phân vị của mẫu số liệu ghép nhóm, kí hiệu $\Delta_Q$, là hiệu giữa tứ phân vị thứ ba $Q_3$ và tứ phân vị thứ nhất $Q_1$ của mẫu số liệu ghép nhóm đó, tức là \boxmini{$\Delta_Q=Q_3-Q_1$}
	\item \indam{Ý nghĩa:}
	\begin{listEX}[1]
		\item [\iconCH] Khoảng tứ phân vị của mẫu số liệu ghép nhóm là giá trị xấp xỉ cho khoảng tứ phân vị của mẫu số liệu gốc và có thể dùng để đo mức độ phân tán của nửa giữa của mẫu số liệu (tập hợp gồm $50 \%$ số liệu nằm chính giữa mẫu số liệu).
		% \item [\iconCH] Khoảng tứ phân vị của mẫu số liệu ghép nhóm càng nhỏ thì dữ liệu càng tập trung xung quanh trung vị.
		\item [\iconCH] Khoảng tứ phân vị được dùng để xác định giá trị bất thường trong mẫu số liệu. Giá trị $x$ trong mẫu số liệu là giá trị bất thường nếu $x>Q_3+1,5 \Delta_Q$ hoặc $x<Q_1-1,5 \Delta_Q$.
		% \item [\iconCH] Khoảng tứ phân vị của mẫu số liệu ghép nhóm không bị ảnh hưởng nhiều bởi các giá trị bất thường trong mẫu số liệu.
	\end{listEX}
	% \begin{note}
	% 	$
	% 	Q_1=a_p+\dfrac{\frac{n}{4}-\left(m_1+\ldots+m_{p-1}\right)}{m_p}\cdot\left(a_{p+1}-a_p\right),
	% 	$\\
	% 	$
	% 	Q_3=a_p+\dfrac{\frac{3 n}{4}-\left(m_1+\ldots+m_{p-1}\right)}{m_p}\cdot\left(a_{p+1}-a_p\right) .
	% 	$
	% \end{note}
\end{enumerate}

\subsection{PHÂN LOẠI VÀ PHƯƠNG PHÁP GIẢI TOÁN}
\begin{dang}{Tìm khoảng biến thiên của mẫu số liệu ghép nhóm}
% \begin{listEX}[1]
	% \item [\ding{172}] Xác định $ u_1 $ là giá trị đầu mút trái của nhóm đầu tiên và $ u_{k+1} $ là giá trị đầu mút phải của nhóm cuối cùng có chứa dữ liệu (tần số khác $0$).
	% \item [\ding{173}] Khoảng biến thiên $ R=u_{k+1}-u_1 $.
% \end{listEX}
\end{dang}
% \boxmini{BÀI TẬP TỰ LUẬN}
\viduminhhoa
\begin{vd}%[1T5B1-1]
	Cân nặng của $28$ học sinh nam lớp $11$ được cho như sau:
	\begin{center}
		\begin{tabular}{lllllll}
			$55{,}4$ & $62{,}6$ & $54{,}2$ & $56{,}8$ & $58{,}8$ & $59{,}4$ & $60{,}7$ \\
			$58$ & $59{,}5$ & $63{,}6$ & $61{,}8$ & $52{,}3$ & $63{,}4$ & $57{,}9$\\
			$49{,}7$ & $45{,}1$ & $56{,}2$ & $63{,}2$ & $46{,}1$ & $49{,}6$ & $59{,}1$\\
			$55{,}3$ & $55{,}8$ & $45{,}5$ & $46{,}8$ & $54$ & $49{,}2$ & $52{,}6$
		\end{tabular}
	\end{center}
\begin{tasks}
	\task Hãy chuyển mẫu số liệu trên sang mẫu số liệu ghép nhóm gồm $5$ nhóm có độ dài bằng nhau với nhóm đầu tiên là $[45; 49)$.
	\task Tìm khoảng biến thiên của mẫu số liệu gốc và bảng biến thiên của mẫu số liệu ghép nhóm tương ứng.
\end{tasks}
\loigiai{
	\begin{enumEX}[a)]{1}
		\item Các nhóm $[45; 49)$, $[49; 53)$, $[53; 57)$, $[57; 61)$, $[61; 65)$. Khi đó ta có bảng tần số ghép nhóm sau:
		\begin{center}
			\begin{tabular}{|c|c|c|c|c|c|}
				\hline Cân nặng &{$[45; 49)$} &{$[49; 53)$} &{$[53; 57)$} &{$[57; 61)$} &{$[61; 65)$} \\
				\hline Số học sinh & 4 & 5 & 7 & 7 & 5 \\
				\hline
			\end{tabular}
		\end{center}
		\item Khoảng biến thiên của mẫu số liệu gốc là $63{,}6-45{,}1=18{,}5$.\\
		Khoảng biến thiên của mẫu số liệu ghép nhóm $65-45=20$.
	\end{enumEX}=
	}
\end{vd}

\begin{vd}
	Bảng sau thống kê thời gian tập thể dục buổi sáng mỗi ngày trong tháng 9/2022 của bác Bình và bác An.
	\begin{center}
		\begin{tabular}{|c|c|c|c|c|c|}
			\hline \begin{tabular}{c} 
				Thời gian \\
				(phút)
			\end{tabular} &{$[15; 20)$} &{$[20; 25)$} &{$[25; 30)$} &{$[30; 35)$} &{$[35; 40)$} \\
			\hline \begin{tabular}{c} 
				Số ngày tập
				của bác Bình
			\end{tabular} & $ 5 $ & $ 12 $ & $ 8 $ & $ 3 $ & $ 2 $ \\
			\hline \begin{tabular}{c} 
				Số ngày tập
				của bác An
			\end{tabular} & $ 0 $ & $ 25 $ & $ 5 $ & $ 0 $ & $ 0 $ \\
			\hline
		\end{tabular}
	\end{center}
	\begin{enumEX}{1}
		\item Hãy tìm khoảng biến thiên của mẫu số liệu ghép nhóm về thời gian tập thể dục buổi sáng mỗi ngày của bác Bình và bác An.
		\item Sử dụng khoảng biến thiên, hãy cho biết bác nào có thời gian tập phân tán hơn.
	\end{enumEX}
	\loigiai{
		\begin{enumEX}{1}
			\item Khoảng biến thiên của mẫu số liệu ghép nhóm về thời gian tập thể dục buổi sáng của bác Bình là $40-15=25$ (phút).\\
			Trong mẫu số liệu ghép nhóm về thời gian tập thể dục buổi sáng của bác An, khoảng đầu tiên chứa dữ liệu là $[20; 25)$ và khoảng cuối cùng chứa dữ liệu là $[25; 30)$.\\
			Do đó khoảng biến thiên của mẫu số liệu ghép nhóm về thời gian tập thể dục buổi sáng của bác An là $30-20=10$ (phút).
			\item Nếu căn cứ theo khoảng biến thiên thì bác Bình có thời gian tập phân tán hơn bác An.
		\end{enumEX}	
	}
\end{vd}

\begin{vd}
	Thống kê thời gian sử dụng mạng xã hội trong ngày của các bạn Tổ 1, Tổ 2 lớp 12A, được kết quả như bảng sau:
	\begin{center}
		\begin{tabular}{|l|c|c|c|c|}
			\hline Thời gian sử dụng (phút) &{$[0; 10)$} &{$[10; 30)$} &{$[30; 60)$} &{$[60; 90)$} \\
			\hline Số học sinh Tổ 1 & $ 2 $ & $ 4 $ & $ 3 $ & $ 1 $ \\
			\hline Số học sinh Tổ 2 & $ 5 $ & $ 1 $ & $ 3 $ & $ 0 $ \\
			\hline
		\end{tabular}
	\end{center}
	Tìm khoảng biến thiên cho thời gian sử dụng mạng xã hội của học sinh mỗi tổ và giải thích ý nghĩa.
	\loigiai{
		Gọi $R_1, R_2$ tương ứng là khoảng biến thiên của mẫu số liệu ghép nhóm về thời gian sử dụng mạng xã hội trong ngày của các bạn Tổ 1 và Tổ 2.\\
		Ta có: $R_1=90-0=90$ và $R_2=60-0=60$.\\
		Do $R_1>R_2$ nên nếu dựa vào khoảng biến thiên, ta kết luận rằng thời gian sử dụng mạng xã hội trong ngày của các bạn Tổ 1 phân tán hơn thời gian sử dụng mạng xã hội của các bạn Tổ 2.	
	}
\end{vd}
\baitaptn
% \boxmini{BÀI TẬP TRẮC NGHIỆM}
\Opensolutionfile{ans}[ans/2D3-B1-d1]
\begin{ex}
	Khảo sát thời gian tập thể dục của một số học sinh khối $11$ thu được mẫu số liệu ghép nhóm sau:
	\vspace*{-10pt}
	\begin{center}
		\begin{tabular}{|c|c|c|c|c|c|}
			\hline Thời gian & {$[0 ; 20)$} & {$[20 ; 40)$} & {$[40 ; 60)$} & {$[60 ; 80)$} &  {$[80 ;100)$} \\
			\hline Số học sinh & $5$ & $9$ & $12$ & $10$ & $6$  \\
			\hline
		\end{tabular}
	\end{center}
	Tìm khoảng biến thiên của mẫu số liệu ghép nhóm trên.
	\choice
	{$80$}
	{$60$}
	{\True $100$}
	{$12$}
	\loigiai{
		Xác định $ u_1=0 $ là giá trị đầu mút trái của nhóm đầu tiên và $ u_{k+1}=100 $ là giá trị đầu mút phải của nhóm cuối cùng có chứa dữ liệu. Suy ra $R=u_{k+1}-u_{1}=100-0=100$.
}
\end{ex}

\begin{ex}
	Mức thưởng tết (triệu đồng) cho các nhân viên của một công ty được thống kê trong bảng sau:
	\vspace*{-10pt}
	\begin{center}
		\begin{tabular}{|c|c|c|c|c|c|}
			\hline Mức thưởng tết & {$[5 ; 10)$} & {$[10 ; 15)$} & {$[15 ; 20)$} & {$[20 ; 25)$} &  {$[25 ;30)$} \\
			\hline Số nhân viên & $13$ & $35$ & $47$ & $25$ & $10$  \\
			\hline
		\end{tabular}
	\end{center}
	Tìm khoảng biến thiên của mẫu số liệu ghép nhóm trên.
	\choice
	{$20$}
	{\True $25$}
	{$47$}
	{$23$}
	\loigiai{
			Xác định $ u_1=5 $ là giá trị đầu mút trái của nhóm đầu tiên và $ u_{k+1}=30 $ là giá trị đầu mút phải của nhóm cuối cùng có chứa dữ liệu. Suy ra $R=u_{k+1}-u_{1}=30-5=25$.
}
\end{ex}

\begin{ex}
	Cho bảng phân bố tần số ghép lớp sau
	\vspace*{-10pt}
	\begin{center}
		Chiều cao của $40$ học sinh nam ở một trường THPT\\
		\begin{tabular}{|c|c|c|c|c|c|}
			\hline
			Lớp chiều cao (cm) & [160; 163,5) & [164; 167,5) & [168; 171,5) & [172; 175,5) & Cộng\\
			\hline
			Tần số & 9 & 20 & 7 & 4 & 40\\
			\hline
		\end{tabular}
	\end{center}
	Tìm khoảng biến thiên của mẫu số liệu ghép nhóm trên.
	\choice
	{$31$}
	{\True $15,5$}
	{$175,5$}
	{$12$}
	\loigiai{
			Xác định $ u_1=160 $ là giá trị đầu mút trái của nhóm đầu tiên và $ u_{k+1}=175,5 $ là giá trị đầu mút phải của nhóm cuối cùng có chứa dữ liệu. Suy ra $R=u_{k+1}-u_{1}=175,5-160=15,5$.
	}
\end{ex}


\begin{ex}
	Thời gian truy cập Internet mỗi buổi tối của một số học sinh được cho trong bảng sau:
%	\vspace*{-10pt}
	\begin{center}
		\begin{tabular}{|c|c|c|c|c|c|}
			\hline Thời gian (phút) & {$[9{,}5 ; 12{,}5)$} & {$[12{,}5 ; 15{,}5)$} & {$[15{,}5 ; 18{,}5)$} & {$[18{,}5 ; 21{,}5)$} & {$[21{,}5 ; 24{,}5)$} \\
			\hline Số học sinh & $0$ & $12$ & $15$ & $24$ & $26$ \\
			\hline
		\end{tabular}
	\end{center}
	Tìm khoảng biến thiên của mẫu số liệu ghép nhóm trên.
	\choice
	{$26$}
	{$14$}
	{$20$}
	{\True $12$}
	\loigiai{
		Xác định $ u_1=12,5 $ là giá trị đầu mút trái của nhóm đầu tiên và $ u_{k+1}=24,5 $ là giá trị đầu mút phải của nhóm cuối cùng có chứa dữ liệu. Suy ra $R=u_{k+1}-u_{1}=24,5-12,5=12$.
	}
\end{ex}

\begin{ex}%[2D3H1-2]
	Thời gian hoàn thành bài kiểm tra môn Toán của các bạn trong lớp $12$C được cho trong bảng sau:
	%\vspace*{-10pt}
	\begin{center}
		\begin{tabular}{|l|c|c|c|c|}
			\hline
			Thời gian (phút) & $[25;30 )$ & $[30;35)$ &$[35;40 )$ & $[40;45)$ \\
			\hline
			Số học sinh & $8$ & $16$ & $12$ & $2$ \\
			\hline
		\end{tabular}
	\end{center}
	Tìm khoảng biến thiên của mẫu số liệu ghép nhóm trên.
	\choice
	{$24$}
	{$15$}
	{$2$}
	{\True $20$}
	\loigiai{
		Xác định $ u_1=25 $ là giá trị đầu mút trái của nhóm đầu tiên và $ u_{k+1}=45 $ là giá trị đầu mút phải của nhóm cuối cùng có chứa dữ liệu. Suy ra $R=u_{k+1}-u_{1}=45-25=20$.
	}
\end{ex}

\begin{dang}{Tìm tứ phân vị của mẫu số liệu ghép nhóm}
	\indamm{Với mẫu số liệu ghép nhóm}
	\begin{center}
		\begin{tabular}{|l|c|c|c|c|c|}
			\hline Nhóm &{$\left[a_1; a_2\right)$} & $\ldots$ &{$\left[a_i; a_{i+1}\right)$} & $\ldots$ &{$\left[a_k; a_{k+1}\right)$} \\
			\hline Tần số & $m_1$ & $\ldots$ & $m_i$ & $\ldots$ & $m_k$ \\
			\hline
		\end{tabular}
	\end{center}
	\indamm{Các bước thực hiện:}
	\begin{listEX}[1]
		\item [\ding{172}] Tìm tứ phân vị $ Q_1$ và $Q_3 $ theo công thức:
		$$Q_r=a_p+\dfrac{\dfrac{r \cdot n}{4}-\left(m_1+\cdots+m_{p-1}\right)}{m_p} \cdot\left(a_{p+1}-a_p\right), $$
		trong đó $\left[a_p; a_{p+1}\right)$ là nhóm chứa tứ phân vị thứ $r$ với $r=1$, $3$; \quad $n$ là cỡ mẫu.
		\item [\ding{173}] Khoảng tứ phân vị của mẫu số liệu ghép nhóm là $\Delta_Q=Q_3-Q_1$.
	\end{listEX}
\end{dang}
\viduminhhoa
% \boxmini{BÀI TẬP TỰ LUẬN}
\setcounter{vd}{0}
\begin{vd}
	Bảng sau thống kê cân nặng của $ 50 $ quả xoài được lựa chọn ngẫu nhiên sau khi thu hoạch ở một nông trường.
	\begin{center}
		\begin{tabular}{|c|c|c|c|c|c|}
			\hline Cân nặng $(\mathrm{g})$ &{$[250; 290)$} &{$[290; 330)$} &{$[330; 370)$} &{$[370; 410)$} &{$[410; 450)$} \\
			\hline Số quả xoài & $ 3 $ & $ 13 $ & $ 18 $ & $ 11 $ & $ 5 $ \\
			\hline
		\end{tabular}
	\end{center}
	Hãy tìm khoảng tứ phân vị của mẫu số liệu ghép nhóm đã cho.
	\loigiai{
		Cỡ mẫu $n=50$.\\
		Gọi $x_1; x_2; \ldots; x_{50}$ là mẫu số liệu gốc gồm cân nặng của $ 50 $ quả xoài được xếp theo thứ tự không giảm.\\
		Ta có 
		\begin{listEX}[3]
			\item [] $x_1, x_2, x_3 \in[250; 290)$
			\item [] $x_4, \ldots, x_{16} \in[290; 330)$
			\item [] $x_{17}, \ldots, x_{34} \in[330; 370)$
			\item [] $x_{35}, \ldots, x_{45} \in[370; 410)$
			\item [] $x_{46}, \ldots, x_{50} \in[410; 450).$
		\end{listEX}
		Tứ phân vị thứ nhất của mẫu số liệu gốc là $x_{13} \in[290; 330)$.\\
		Do đó, tứ phân vị thứ nhất của mẫu số liệu ghép nhóm là
		$$Q_1=290+\dfrac{\dfrac{50}{4}-3}{13} \cdot(330-290)=\dfrac{4150}{13} .$$	
		Tứ phân vị thứ ba của mẫu số liệu gốc là $x_{38} \in[370; 410)$. Do đó, tứ phân vị thứ ba của mẫu số liệu ghép nhóm là 
		$$Q_3=370+\dfrac{\dfrac{3 \cdot 50}{4}-(3+13+18)}{11} \cdot(410-370)=\dfrac{4210}{11}.$$
		Vậy khoảng tứ phân vị của mẫu số liệu ghép nhóm là
		$$\Delta_Q=\dfrac{4210}{11}-\dfrac{4150}{13}=\dfrac{9080}{143} \approx 63,5.$$
	}
\end{vd}

\begin{vd}%[2D3H1-4]
	Bảng sau đây cho biết chiều cao của các học sinh lớp 12A và 12B.
	\begin{center}
		\begin{tabular}{|c|c|c|c|c|c|c|}
			\hline
			Chiều cao (cm) & $[145;150 )$ & $[150;155)$ &$[155;160 )$ & $[160;165)$ & $[165;170)$ & $[170;175)$ \\
			\hline
			Số học sinh của lớp 12A & $1$ & $0$ & $15$ & $12$ & $10$ & $5$\\
			\hline
			Số học sinh của lớp 12B & $0$ & $0$ & $17$ & $10$ & $9$ & $6$\\
			\hline
		\end{tabular}
	\end{center}
	\begin{enumerate}
		\item Tính khoảng biến thiên, khoảng tứ phần vị cho các mẫu số liệu ghép nhóm của học sinh lớp 12A, 12B.
		\item Để so sánh độ phân tán về chiều cao của học sinh hai lớp này ta nên dùng khoảng biến thiên hay khoảng tứ phân vị? Vì sao?	
	\end{enumerate}
	\loigiai{
		\begin{enumerate}
			\item Ta có
			\begin{center}
				\begin{tabular}{|c|c|c|c|c|c|c|}
					\hline
					Chiều cao (cm) & $[145;150 )$ & $[150;155)$ &$[155;160 )$ & $[160;165)$ & $[165;170)$ & $[170;175)$ \\
					\hline
					Số học sinh \\
					của lớp 12A & $1$ & $0$ & $15$ & $12$ & $10$ & $5$\\
					\hline
					Số học sinh \\
					của lớp 12B & $0$ & $0$ & $17$ & $10$ & $9$ & $6$\\
					\hline
				\end{tabular}
			\end{center}
			Khoảng biến thiên là $175-145 = 30$ (cm).\\
			Xét lớp 12A,\\
			\[Q_1 = 155 + \dfrac{\dfrac{43}{4}-1}{15}\cdot 5 = 158{,}25.\]
			\[Q_3 = 165 + \dfrac{\dfrac{43\cdot 3}{4}-28}{10}\cdot 5 = 167{,}125.\]
			\[\triangle Q = Q_3 -Q_1 = 8{,}875.\]
			Xét lớp 12B,\\
			\[Q_1 = 155 + \dfrac{\dfrac{42}{4}-0}{17}\cdot 5 = 158{,}5\]
			\[Q_3 = 165 + \dfrac{\dfrac{42\cdot 3}{4}-27}{9}\cdot 5 = 167{,}5\]
			\[\triangle Q = Q_3 -Q_1 = 9{,}4.\]
			\item Để so sánh độ phân tán về chiều cao của học sinh hai lớp này ta nên dùng khoảng tứ phân vị, vì khoảng biến thiên của $2$ lớp này là bằng nhau.
		\end{enumerate}
	}
\end{vd}

\begin{vd}
	Hằng ngày ông Thắng đều đi xe buýt từ nhà đến cơ quan. Dưới đây là bảng thống kê thời gian của $ 100 $ lần ông Thắng đi xe buýt từ nhà đến cơ quan.
	\begin{center}
		\begin{tabular}{|c|c|c|c|c|c|c|}
			\hline Thời gian(phút) &{$[15; 18)$} &{$[18; 21)$} &{$[21; 24)$} &{$[24; 27)$} &{$[27; 30)$} &{$[30; 33)$} \\
			\hline Số lần & $ 22 $ & $ 38 $ & $ 27 $ & $ 8 $ & $ 4 $ & $ 1 $ \\
			\hline
		\end{tabular}
	\end{center}
	\begin{enumEX}{1}
		\item Hãy tìm khoảng tứ phân vị của mẫu số liệu ghép nhóm trên. (Làm tròn kết quả đến hàng phần trăm.)
		\item Biết rằng trong $ 100 $ lần đi trên, chỉ có đúng một lần ông Thắng đi hết $ 32 $ phút. Thời gian của lần đi đó có phải là giá trị ngoại lệ không?
	\end{enumEX}
	\loigiai{
		\begin{enumEX}{1}
			\item Cỡ mẫu $n=100$.\\
			Gọi $x_1; x_2; \ldots; x_{100}$ là mẫu số liệu gốc gồm thời gian 100 lần đi xe buýt của ông Thắng.\\
			Ta có: 
			\begin{listEX}[3]
				\item [] $x_1, \ldots, x_{22} \in[15; 18)$
				\item [] $x_{23}, \ldots, x_{60} \in[18; 21)$
				\item [] $x_{61}, \ldots, x_{87} \in[21; 24)$
				\item [] $x_{88}, \ldots, x_{95} \in[24; 27)$
				\item [] $x_{96}, \ldots, x_{99} \in[27; 30)$
				\item [] $x_{100} \in[30; 33)$.
			\end{listEX}
			Tứ phân vị thứ nhất của mẫu số liệu gốc là $\dfrac{1}{2}\left(x_{25}+x_{26}\right) \in[18; 21)$.\\
			Do đó, tứ phân vị thứ nhất của mẫu số liệu ghép nhóm là
			$$Q_1=18+\dfrac{\dfrac{100}{4}-22}{38} \cdot(21-18)=\dfrac{693}{38}.$$
			Tứ phân vị thứ ba của mẫu số liệu gốc là $\dfrac{1}{2}\left(x_{75}+x_{76}\right) \in[21; 24)$.\\
			Do đó, tứ phân vị thứ ba của mẫu số liệu ghép nhóm là
			$$Q_3=21+\dfrac{\dfrac{3 \cdot 100}{4}-(22+38)}{27} \cdot(24-21)=\dfrac{68}{3}.$$
			Vậy khoảng tứ phân vị của mẫu số liệu ghép nhóm là
			$$\Delta_Q=\dfrac{68}{3}-\dfrac{693}{38}=\dfrac{505}{114} \approx 4,43.$$
			\item Trong lần duy nhất ông Thắng đi hết $ 32 $ phút, thời gian đi của ông thuộc nhóm $[30; 33)$.\\
			Vì $Q_3+1,5 \Delta_Q=\dfrac{6683}{228} \approx 29,31<30$ nên thời gian của lần ông Thắng đi hết $ 32 $ phút là giá trị ngoại lệ của mẫu số liệu ghép nhóm.
		\end{enumEX}	
	}
\end{vd}

\begin{vd}
	\immini
	{
		Bảng bên biểu diễn mẫu số liệu ghép nhóm về chiều cao của $ 42 $ mẫu cây ở một vườn thực vật (đơn vị: centimét). Tính khoảng tứ phân vị của mẫu số liệu ghép nhóm đó (làm tròn kết quả đến hàng phần mười nếu cần).
	}
	{
		\begin{tabular}{|c|c|c|}
			\hline Nhóm & Tần số & Tần số tích luỹ\\
			\hline$[40; 45)$ & $ 5 $ & $ 5 $ \\
			{$[45; 50)$} & $ 10 $ & $ 15 $ \\
			{$[50; 55)$} & $ 7 $ & $ 22 $ \\
			{$[55; 60)$} & $ 9 $ & $ 31 $ \\
			{$[60; 65)$} & $ 7 $ & $ 38 $ \\
			{$[65; 70)$} & $ 4 $ & $ 42 $ \\
			\hline & $n=42$ & \\
			\hline
		\end{tabular}
	}
	\loigiai{
		Cỡ mẫu là $n=42$.
		\begin{itemize}
			\item Ta có: $\dfrac{n}{4}=\dfrac{42}{4}=10,5$ mà $5<10,5<15$.\\
			Suy ra nhóm $ 2 $ là nhóm đầu tiên có tần số tích luỹ lớn hơn hoặc bằng $ 10,5 $ nên nhóm $ 2 $ ( nhóm $[45; 50$) ) là chứa tứ phân vị thứ nhất. Áp dụng công thức, ta có tứ phân vị thứ nhất là
			$$Q_1=45+\left(\dfrac{10,5-5}{10}\right) \cdot 5=47,75.$$	
			\item Ta có: $\dfrac{3 n}{4}=\dfrac{3 \cdot 42}{4}=31,5$ mà $31<31,5<38$.\\
			Suy ra nhóm $ 5 $ là nhóm đầu tiên có tần số tích luỹ lớn hơn hoặc bằng $ 31,5 $ nên nhóm $ 5 $ ( nhóm $[60; 65)$)  là nhóm chứa tứ phân vị thứ ba. Áp dụng công thức, ta có tứ phân vị thứ ba là
			$$Q_3=60+\left(\dfrac{31,5-31}{7}\right) \cdot 5 \approx 60,4.$$
		\end{itemize}
		Vậy khoảng tứ phân vị của mẫu số liệu ghép nhóm đã cho là
		$$\Delta_Q=Q_3-Q_1 \approx 60,4-47,75=12,65.$$
	}
\end{vd}
\baitaptn
% \boxmini{BÀI TẬP TRẮC NGHIỆM}
% \ind{PHẦN I.} \inden{Câu trắc nghiệm nhiều phương án lựa chọn. Mỗi câu hỏi học sinh chỉ chọn một phương án.}\\
\setcounter{ex}{0}
\Opensolutionfile{ans}[ans/2D3-B1-d2-1]



\begin{ex}%[1D1B2-2]
	Khảo sát về cân nặng của các học sinh lớp 11D3 người ta được một mẫu dữ liệu ghép nhóm như sau
	\begin{center}
		\begin{tabular}{|c|c|c|c|c|c|c|}
			\hline Cân nặng & {$[30 ; 40)$} & {$[40 ; 50)$} & {$[50 ; 60)$} & {$[60 ; 70)$} & {$[70 ; 80)$} & {$[80 ; 90)$} \\
			\hline Số học sinh & $2$ & $10$ & $16$ & $8$ & $2$ & $2$ \\
			\hline
		\end{tabular}
	\end{center}
	Khoảng tứ phân vị của bảng số liệu ghép nhóm trên là
	\choice
	{$17$}
	{\True $14.5$}
	{$14$}
	{$17.5$}
	\loigiai{
		Ta có $n=40\Rightarrow\dfrac{n}{4}=10$. \\
		Gọi $x_1, \ldots, x_{40}$ là mẫu số liệu gốc về cân nặng của 40 học sinh lớp 11D3 và giả sử rằng dãy số liệu gốc này đã được sắp xếp theo thứ tự tăng dần.\\
		Tứ phân vị thứ nhất của mẫu số liệu gốc là $\dfrac{1}{2}\left( x_{10}+x_{11}\right) $ nên nhóm chứa tứ phân vị thứ nhất là nhóm $\left[40\,;\,50\right)$. Do đó tứ phân vị thứ nhất của mẫu số liệu trên là
		$$Q_1=40+\dfrac{10-2}{10}\cdot10=48.$$
		Ta có $\dfrac{3 n}{4}=30$.\\
		Tứ phân vị thứ ba của mẫu số liệu gốc là $\dfrac{1}{2}\left( x_{30}+x_{31}\right) $ nên nhóm chứa tứ phân vị thứ ba là nhóm $[60 ; 70)$. Do đó tứ phân vị thứ ba của mẫu số liệu trên là
		$$Q_3=60+\dfrac{30-28}{8} \cdot 10=62{,}5.$$
		Khoảng tứ phân vị $\Delta _Q=Q_3-Q_1=62{,}5-48=14,5$.
	}
\end{ex}

\begin{ex}%[1D1B2-2]
	Doanh thu bán hàng trong $20$ ngày được lựa chọn ngẫu nhiên của một của hàng được ghi lại ở bảng sau (đơn vị: triệu đồng)
	\begin{center}
		\begin{tabular}{|c|c|c|c|c|c|}
			\hline Doanh thu & {$[5 ; 7)$} & {$[7 ; 9)$} & {$[9 ; 11)$} & {$[11 ; 13)$} & {$[13 ; 15)$} \\
			\hline Số ngày & $2$ & $7$ & $7$ & $3$ & $1$ \\
			\hline
		\end{tabular}
	\end{center}
	Khoảng tứ phân vị của mẫu số liệu ghép nhóm này là
	\choice
	{$\dfrac{25}{7}$}
	{$\dfrac{13}{7}$}
	{ \True $\dfrac{20}{7}$}
	{$\dfrac{55}{7}$}
	\loigiai{
		Ta có $n=20$. Gọi $x_1$, $x_2$, $\ldots$, $x_{20}$ là doanh thu bán hàng trong 20 ngày xếp theo thứ tự không giảm.\\
		Khi đó 
		\begin{listEX}[3]
			\item [] $x_1$, $x_2 \in[5 ; 7)$
			\item [] $x_3, \ldots, x_9 \in[7 ; 9)$
			\item [] $x_9$, $\ldots$, $x_{16} \in[9 ; 11)$
			\item [] $x_{17}$, $\ldots$, $x_{19} \in[11 ; 13)$
			\item [] $x_{20} \in[13 ; 15)$.
		\end{listEX}
		Tứ phân vị thứ nhất của mẫu số liệu gốc là $\dfrac{1}{2}\left( x_{5}+x_{6}\right)$ nên tứ phân vị thứ nhất của mẫu số liệu thuộc nhóm $[7 ; 9)$.\\
		Tứ phân vị thứ nhất của mẫu số liệu là
				$$Q_1=7+\dfrac{\dfrac{1.20}{4}-2}{7}(9-7) =\dfrac{55}{7}.$$
		Tứ phân vị thứ ba của mẫu số liệu gốc là $\dfrac{1}{2}\left( x_{15}+x_{16}\right)$ nên tứ phân vị thứ ba của mẫu số liệu thuộc nhóm $[9 ; 11)$.\\
		Tứ phân vị thứ ba của mẫu số liệu là
		$$
		Q_3=9+\dfrac{\dfrac{3\cdot20}{4}-9}{7}(11-9) =\dfrac{75}{7}.
		$$
		Khoảng tứ phân vị $\Delta _Q=Q_3-Q_1=\dfrac{20}{7}$.
	}
\end{ex}

\begin{ex}%[1D1B2-2]
	Trung tâm ngoại ngữ thống kê bảng điểm môn Tiếng Anh của một khóa học trong bảng bên dưới
	\begin{center}
		\begin{tabular}{|l|c|c|c|c|c|}
			\hline
			Điểm     & [0;2) & [2;4) & [4;6) & [6;8) & [8;10) \\ \hline
			Học viên & 10   & 30   & 55   & 42   & 9     \\ \hline
		\end{tabular}
	\end{center}
	Khoảng tứ phân vị của mẫu số liệu ghép nhóm này là (làm tròn đến hàng phần trăm)
	\choice
	{\True $2{,}92$}
	{$2{,}93$}
	{$3{,}93$}
	{$3,92$}
	\loigiai{
		Ta có $n=146$. Gọi $x_{1}, x_{2}, ..., x_{146}$ là số liệu được sắp xếp theo thứ tự không giảm. \\
		Tứ phân vị thứ nhất của của dãy số liệu gốc là $x_{37}\in [2;4)$. Do đó, tứ phân vị thứ nhất của mẫu số liệu ghép nhóm trên là 
		$$Q_{1}=2+\dfrac{\dfrac{1.146}{4}-10}{30}.(4-2)=\dfrac{113}{30}.$$
		Tứ phân vị thứ ba của của dãy số liệu gốc là $x_{110}\in [6;8)$. Do đó, tứ phân vị thứ ba của mẫu số liệu ghép nhóm trên là \\
		$$Q_{3}=6+\dfrac{\dfrac{3.146}{4}-(10+30+55)}{42}.(8-6)=\dfrac{281}{42}$$
		Khoảng tứ phân vị $Q_3-Q_1=\dfrac{307}{105}\approx 2{,}92$.
	}
\end{ex}

\begin{ex}%[1T5K2-2]
	Thời gian luyện tập trong một ngày (tính theo giờ) của một số vận động viên được ghi lại ở bảng sau:
	\begin{center}
		\begin{tabular}{|c|c|c|c|c|c|}
			\hline 
			Thời gian luyện tập (giờ)	& $ \left[ 0 ; 2\right) $ & $ \left[ 2 ; 4\right) $ & $ \left[ 4 ; 6\right) $ & $ \left[ 6 ; 8\right) $ & $ \left[8 ; 10 \right) $ \\ 
			\hline 
			Số vận động viên	& $ 3 $ & $ 8 $ & $ 12 $ & $ 12 $ & $ 4 $ \\ 
			\hline 
		\end{tabular} 
	\end{center}
	Hãy xác định khoảng tứ phân vị của mẫu số liệu đã cho (làm tròn đến hàng phần trăm).
	\choice
	{$4{,}52$}
	{\True $3{,}35$}
	{$2{,}85$}
	{$3{,}36$}
	\loigiai{
	Số vận động viên được khảo sát là $ n=3+8+12+12+4=39$.\\
	Gọi $ x_1 $; $ x_2 $; \ldots ;$ x_{39} $ là thời gian luyện tập của $ 39 $ vận động viên được xếp theo thứ tự không giảm.	Ta có 
	\begin{enumEX}[]{3}
		\item $ x_1, x_2, x_3 \in \left[ 0 ; 2\right) $;
		\item $ x_4, \ldots, x_{11}\in \left[2 ; 4 \right) $;
		\item $ x_{12}, \ldots, x_{23}\in \left[ 4 ; 6\right) $;
		\item $ x_{24}, \ldots, x_{35}\in \left[6;8\right) $;
		\item $ x_{36},\ldots , x_{39}\in \left[ 8 ; 10\right) $.
	\end{enumEX}
	\begin{itemize}
		\item Tứ phân vị thứ nhất là $ x_{10} $ thuộc nhóm $\left[ 2 ; 4\right)$;
		\item Tứ phân vị thứ ba là $ x_{30} $ thuộc nhóm $ \left[ 6 ; 8\right) $.
	\end{itemize}
	Tứ phân vị thứ nhất của mẫu số liệu ghép nhóm là $$Q_1=2+\dfrac{\dfrac{1\cdot 39}{4}-3}{8}\cdot(4-2)=\dfrac{59}{16}.$$\\
	Tứ phân vị thứ ba của mẫu số liệu ghép nhóm là $$Q_3=6+\dfrac{\dfrac{3\cdot 39}{4}-(3+8+12)}{12}\cdot(8-6)=\dfrac{169}{24}.$$ 
	Khoảng tứ phân vị $Q_3-Q_1=\dfrac{161}{48}\approx 3{,}35$.
	}
\end{ex}

\begin{ex}
	Ở một phòng điều trị nội trú của bệnh viện, dữ liệu thống kê thời gian ngủ hằng đêm của một bệnh nhân trong suốt một tháng được tổng hợp bởi bảng dưới đây
	\begin{center}
		\begin{tabular}{|c|c|c|}
			\hline Thời gian (phút) & Tần số & \begin{tabular}{c} 
				Tần số \\
				tích luỹ
			\end{tabular} \\
			\hline$[180 ; 240)$ & $2$ & $2$ \\
			\hline$[240 ; 300)$ & $9$ & $11$ \\
			\hline$[300 ; 360)$ & $12$ & $23$\\
			\hline$[360 ; 420)$ & $5$ & $28$ \\
			\hline$[420 ; 480)$ & $2$ & $30$ \\
			\hline
		\end{tabular}
	\end{center}
\choice
{$75{,}53$}
{$84{,}83$}
{\True $80{,}83$}
{$72{,}53$}
\loigiai{
	Kích thước mẫu $n=30$. Ta có $\dfrac{n}{4}=\dfrac{15}{2}=7{,}5 ;\, \dfrac{3 n}{4}=\dfrac{45}{2}=22{,}5$.	\\
\begin{itemize}
	\item [$\bullet$] Nhóm chứa $Q_1$ là $[240 ; 300)$. Suy ra
	$$Q_1=240+\dfrac{7{,}5 -2}{9} \cdot 60 =\dfrac{830}{3}.$$
	\item [$\bullet$] Nhóm chứa $Q_3$ là $[300 ; 360)$.Suy ra
	$$Q_3=300+\dfrac{22{,}5 -11}{12} \cdot 60=357{,}5$$
\end{itemize}
	Vậy $\Delta_Q=357{,}5-\dfrac{830}{3}\approx 80{,}83$.
	}
\end{ex}

\begin{ex}%[2D3H1-3]
	Biểu đồ dưới đây biểu diễn số lượt khách hàng đặt bàn qua hình thức trực tuyến mỗi ngày trong quý III năm 2022 của một nhà hàng. Cột thứ nhất biểu diễn số ngày có từ $1$ đến dưới $6$ lượt đặt bàn; cột thứ hai biểu diễn số ngày có từ $6$ đến dưới $11$ lượt đặt bàn;\ldots.
	\begin{center}
		\begin{tikzpicture}[font=\small, line join=round, line cap=round, >=stealth,x=0.25cm,y=0.6cm]
			\draw[->](0,0)--(0,8)node[left]{\textbf{Số ngày}};
			\draw[->](0,0)--(35,0)node[below]{\textbf{Số lượt đặt bàn}};
			\foreach \i in{1,...,7} \pgfmathsetmacro{\gti}{int(5*(\i))}
			\draw [dotted](0,\i) circle(1pt)node[left]{$\gti$} -- (30,\i);
			\foreach \i/\a/\b in {1/15/20,2/20/25,3/25/30,4/30/35,5/35/40}
			\foreach \i/\j in {1/14,6/30,11/25,16/18,21/5}
			{
				\draw[fill=cyan!50](\i,0)rectangle(\i+5,\j/5);
				\draw (\i,0) node [below] {$\i$};
				\draw (\i+2.5,\j/5) node [above] {$\j$};
			}
			\draw (26,0) node [below] {$26$};
		\end{tikzpicture}
	\end{center}
	Hãy tìm khoảng tứ phân vị của mẫu số liệu ghép nhóm cho bởi biểu đồ trên.
	\choice
	{$9{,}5$}
	{\True $8{,}5$}
	{$10{,}5$}
	{$7{,}5$}
	\loigiai{Dựa vào biểu đồ, ta lập được bảng ghép nhóm như bên dưới.
		\begin{center}
			\begin{tabular}{|c|c|c|c|c|c|}
				\hline
				Lượt đặt bàn & $[1;6)$ & $[6;11)$ & $[11;16)$ & $[16;21)$ & $[21;26)$ \\
				\hline
				Số ngày & $14$ & $30$ & $25$ & $18$ & $5$ \\
				\hline
			\end{tabular}
		\end{center}
		Ta có cỡ mẫu $n=92$.\\
		Gọi $x_1$; $x_2$; \ldots; $x_{92}$ là mẫu số liệu đã cho.\\
		Ta có: 
		\begin{enumEX}[]{3}
			\item $x_1$, \ldots, $x_{14}\in[1;6)$; 
			\item $x_{15}$, \ldots, $x_{44}\in[6;11)$; 
			\item $x_{45}$, \ldots, $x_{69}\in[11;16)$;
			\item $x_{70}$, \ldots, $x_{87}\in[16;21)$;
			\item $x_{88}$, \ldots, $x_{92}\in[21;26)$.
		\end{enumEX} 
		Tứ phân vị thứ nhất của mẫu số liệu là $\dfrac{x_{23}+x_{24}}{2}\in[6;11)$. Do đó, tứ phân vị thứ nhất của mẫu số liệu là
		$$Q_1=6+\dfrac{\dfrac{92}{4}-14}{30}\cdot(11-6)=7{,}5.$$
		Tứ phân vị thứ ba của mẫu số liệu là $\dfrac{x_{69}+x_{70}}{2}$ với $x_{69}\in[11;16)$ và $x_{70}\in[16;21)$. Do đó, tứ phân vị thứ ba của mẫu số liệu là $Q_3=16$.\\
		%Đoạn này sách giáo khoa 12 không đề cập, tôi lấy từ kiến thức của sách CTST lớp 11.
		Vậy khoảng tứ phân vị của mẫu số liệu là $\Delta_Q=Q_3-Q_1=8{,}5$.}
\end{ex}

\Closesolutionfile{ans}

% \ind{PHẦN II.} \inden{Câu trắc nghiệm đúng sai. Trong mỗi ý a), b), c), d) ở mỗi câu, học sinh chọn đúng hoặc sai.}\\
\Opensolutionfile{ans}[ans/2D3-B1-d2-2]

\begin{ex}%[2D3H1-3]
	Kết quả đo chiều cao của $100$ cây keo 3 năm tuổi tại một nông trường được cho ở bảng sau
	\begin{center}
		\begin{tabular}{|c|c|c|c|c|c|}
			\hline
			Chiều cao (m) & $[8{,}4;8{,}6)$ & $[8{,}6;8{,}8)$ & $[8{,}8;9{,}0)$ & $[9{,}0;9{,}2)$ & $[9{,}2;9{,}4)$ \\
			\hline
			Số cây & $5$ & $12$ & $25$ & $44$ & $14$ \\
			\hline
		\end{tabular}
	\end{center}
	\choiceTF
	{\True Khoảng biến thiên của mẫu số liệu này là $R=1$}
	{Tứ phân vị thứ nhất của mẫu số liệu là $Q_1=8$}
	{\True Khoảng tứ phân vị của mẫu số liệu là $\Delta Q=0{,}286$}
	{\True Biết rằng trong $100$ cây keo trên có $1$ cây cao $8{,}4$ m. Chiều cao của cây keo này là giá trị ngoại lệ}
	\loigiai{\begin{enumerate}
			\item Khoảng biến thiên của mẫu số liệu là $R=9{,}4-8{,}4=1$.
			\item
			Ta có cỡ mẫu $n=100$.\\
			Gọi $x_1$; $x_2$; \ldots; $x_{100}$ là mẫu số liệu gồm chiều cao của $100$ cây keo.\\
			Ta có: 
			\begin{enumEX}[]{3}
				\item $x_1$, \ldots, $x_5\in[8{,}4;8{,}6)$; 
				\item $x_6$, \ldots, $x_{17}\in[8{,}6;8{,}8)$;
				\item $x_{18}$, \ldots, $x_{42}\in[8{,}8;9{,}0)$; 
				\item $x_{43}$, \ldots, $x_{86}\in[9{,}0;9{,}2)$; 
				\item $x_{87}$, \ldots, $x_{100}\in[9{,}2;9{,}4)$.
			\end{enumEX}
			Tứ phân vị thứ nhất của mẫu số liệu là $\dfrac{x_{25}+x_{26}}{2}\in[8{,}8;9{,}0)$. Do đó, tứ phân vị thứ nhất của mẫu số liệu ghép nhóm là
			$$Q_1=8{,}8+\dfrac{\dfrac{100}{4}-(5+12)}{25}\cdot(9{,}0-8{,}8)=8{,}864.$$
			\item	Tứ phân vị thứ ba của mẫu số liệu là $\dfrac{x_{75}+x_{76}}{2}\in[9{,}0;9{,}2)$. Do đó, tứ phân vị thứ ba của mẫu số liệu ghép nhóm là
			$$Q_3=9{,}0+\dfrac{\dfrac{3\cdot100}{4}-(5+12+25)}{44}\cdot(9{,}2-9{,}0)=9{,}15.$$
			Vậy khoảng tứ phân vị của mẫu số liệu ghép nhóm là $\Delta_Q=Q_3-Q_1=0{,}286$.
			\item Vì $Q_1-1{,}5\Delta_Q=8{,}435$ và $Q_3+1{,}5\Delta_Q=9{,}579$ nên cây keo có chiều cao $8{,}4$ m là giá trị ngoại lệ của mẫu số liệu ghép nhóm.
	\end{enumerate}}
\end{ex}

\begin{ex}
	\immini{Bảng bên biểu diễn mẫu số liệu ghép nhóm thống kê mức lương của một công ty (đơn vị: triệu đồng).
		\choiceTF
		{Khoảng biến thiên của mẫu số liệu này là $R=25$}
		{\True Tứ phân vị thứ nhất của mẫu số liệu là $Q_1=15$}
		{Tứ phân vị thứ ba của mẫu số liệu là $Q_3=27$}
		{Khoảng tứ phân vị của mẫu số liệu là $\Delta Q=12$}
	}{\begin{tabular}{|c|c|}
			\hline Nhóm & Tần số \\
			\hline$[10 ; 15)$ & $15$ \\
			{$[15 ; 20)$} & $18$ \\
			{$[20 ; 25)$} & $10$ \\
			{$[25 ; 30)$} & $10$ \\
			{$[30 ; 35)$} & $5$ \\
			{$[35 ; 40)$} & $2$ \\
			\hline & $n=60$ \\
			\hline
	\end{tabular}}
	\loigiai{
		\begin{enumerate}
			\item Trong mẫu số liệu ghép nhóm ở bảng, ta có đầu mút trái của nhóm $1$ là $a_1=10$, đầu mút phải của nhóm $6$ là $a_7=40$.\\Vậy khoảng biến thiên của mẫu số liệu ghép nhóm đó là $R=a_7-a_1=40-10=30.$
			\item Ta có bảng sau
			\begin{center}
				\begin{tabular}{|c|c|c|}
					\hline Nhóm & Tần số & Tần số tích luỹ\\
					\hline$[10 ; 15)$ & $15$ & $15$\\
					{$[15 ; 20)$} & $18$ & $33$\\
					{$[20 ; 25)$} & $10$ & $43$\\
					{$[25 ; 30)$} & $10$ & $53$\\
					{$[30 ; 35)$} & $5$ & $58$\\
					{$[35 ; 40)$} & $2$ & $60$\\
					\hline & $n=60$ & \\
					\hline
				\end{tabular}
			\end{center}
			Số phần tử của mẫu là $n=60$. \\
			Nhóm $[15;20)$ là nhóm chứa tứ phân vị thứ nhất. 
			Áp dụng công thức, ta có tứ phân vị thứ nhất là $$Q_1=15+\left(\dfrac{15-15}{18}\right)\cdot 5=15 ~\text{(triệu đồng)}.$$
			\item Nhóm $[25;30)$ là nhóm chứa tứ phân vị thứ 3. Áp dụng công thức, ta có tứ phân vị thứ ba là
			$$Q_3=25+\left(\dfrac{45-43}{10}\right)\cdot5=26 ~\text{(triệu đồng)}.$$
			\item  Khoảng tứ phân vị của mẫu số liệu ghép nhóm đã cho là 
			$$\Delta _Q=Q_3-Q_1=26-15=11 ~\text{(triệu đồng)}.$$
		\end{enumerate}
	}
\end{ex}

\begin{ex}
	Điều tra một số hộ gia đình thu nhập ở mức trung bình sinh sống trên hai địa bàn $A$, $B$, người ta thấy diện tích nhà ở của họ đều nhỏ hơn $100$ m$^2$. Hai biểu đồ dưới biểu diễn kết quả thống kê. 
	\begin{center}
		\begin{tikzpicture}[>=stealth,scale=1]
			%========================
			\draw[opacity=.25,thin,step=.2,cyan](0,0) grid(7,3);
			\draw[opacity=.5,cyan](0,0) grid (7,3);
			\draw[stealth-stealth](0,3) node[left]{Tần số}|-(7,0)node[below]{m$^2$};
			\foreach\x/\dientich[count=\i from 1] in {0/50,.4/60,1/70,2.5/80,.9/90,.2/100}{
				\draw[fill=gray](\i-1,0) rectangle +(1,\x);
				\draw (\i,0) node[below]{\dientich};
			}
			\foreach \y [count=\i from 1] in {10,20,30,40,50}{
				\draw (-.1,\i/2)--(.1,\i/2)(0,\i/2) node[left]{$\y$};}
			\foreach \z [count=\i from 1] in {8,20,50,18,4}{
				\draw (\i+0.5,\z/20) node[above] {$\z$};}
		\end{tikzpicture}\\	
		\textit{Hình a. Diện tích nhà ở của cư dân địa bàn $A$}
		%========================
	\end{center}
	\begin{center}
		\begin{tikzpicture}[>=stealth,scale=1]
			%========================
			\draw[opacity=.25,thin,step=.2,cyan](0,0) grid(7,3);
			\draw[opacity=.5,cyan](0,0) grid (7,3);
			\draw[stealth-stealth](0,3) node[left]{Tần số}|-(7,0)node[below]{m$^2$};
			\foreach\x/\dientich[count=\i from 1] in {0/50,.75/60,1/70,1.5/80,1/90,.75/100}{
				\draw[fill=gray](\i-1,0) rectangle +(1,\x);
				\draw (\i,0) node[below]{\dientich};
			}
			\foreach \y [count=\i from 1] in {10,20,30,40,50}{
				\draw (-.1,\i/2)--(.1,\i/2)(0,\i/2) node[left]{$\y$};}
			\foreach \z [count=\i from 1] in {15,20,30,20,15}{
				\draw (\i+0.5,\z/20-0.05) node[above] {$\z$};}
			%========================
		\end{tikzpicture}\\
		\textit{Hình b. Diện tích nhà ở của cư dân địa bàn $B$}
	\end{center}
	\choiceTF
	{\True Khoảng biến thiên của hai mẫu số liệu này bằng nhau}
	{\True Khoảng tứ phân vị ghép nhóm diện tích căn hộ của địa phương A là $10{,}9$}
	{Khoảng tứ phân vị ghép nhóm diện tích căn hộ của địa phương B là $8{,}5$.}
	{Số liệu về diện tích nhà ở của cư dân thuộc địa bàn A phân tán hơn địa bàn B}
	\loigiai{
		Ta có bảng tần số tích luỹ như sau:
		\begin{center}
			\begin{tabular}{|c|c|c|c|c|c|}
				\hline \begin{tabular}{c}
					Diện tích nhà ở \\
					Địa bàn $A$ (m$^2$) 
				\end{tabular} & Tần số  & \begin{tabular}{c}
					Tần số \\
					tích luỹ 
				\end{tabular}  & \begin{tabular}{c}
					Diện tích nhà ở \\
					Địa bàn $B$ (m$^2$) 
				\end{tabular} & Tần số & \begin{tabular}{c}
					Tần số  \\
					tích luỹ 
				\end{tabular}   \\
				\hline$[50 ; 60)$ & $8$ & $8$& $[50 ; 60)$ & $15$& $15$ \\
				\hline$[60 ; 70)$ & $20$ &$28$&  $[60 ; 70)$ & $20$& $35$ \\
				\hline$[70 ; 80)$ & $50$ &$78$&  $[70 ; 80)$ & $30$& $65$ \\
				\hline$[80 ; 90)$ & $18$ &$96$&  $[80 ; 90)$ & $20$& $85$ \\
				\hline$[90 ; 100)$ & $4$ &$100$&  $[90 ; 100)$ & $15$& $100$ \\
				\hline
			\end{tabular}
		\end{center}
		\begin{enumerate}[a)]
			\item Khoảng biến thiên của hai mẫu số liệu này bằng nhau và bằng $100=50=50$.
			\item Xét bảng số liệu $A$, ta có $N=100; \dfrac{N}{4}=25; \dfrac{N}{2}=50; \dfrac{3N}{4}=75$.
			\begin{itemize}
				\item [$\bullet$] Nhóm chứa $Q_1^A$ là $[60 ; 70)$. Suy ra
						$$Q_1^A=60+\dfrac{25-8}{20} \cdot 10 = 68,5 $$
				\item [$\bullet$] Nhóm chứa $Q_3^A$ là $[70;80)$. Suy ra
						$$Q_3^A=70+\dfrac{75 -28}{50} \cdot 10=79{,}4$$
			\end{itemize}
		Vậy khoảng tứ phân vị ghép nhóm diện tích căn hộ của địa phương A là\\ $\Delta_{Q_A} =79{,}4-68{,}5=10{,}9$. 
			\item  Xét bảng số liệu $B$, ta có $N=100; \dfrac{N}{4}=25; \dfrac{N}{2}=50; \dfrac{3N}{4}=75$.
			\begin{itemize}
				\item [$\bullet$] Nhóm chứa $Q_1^B$ là $[60 ; 70)$. Suy ra
						$$Q_1^B=60+\dfrac{25 -15}{20} \cdot 10=65.$$
				\item [$\bullet$] Nhóm chứa $Q_3^B$ là $[80;90)$.Suy ra
					$$Q_3^B=80+\dfrac{75 -65}{20} \cdot 10= 85.$$
			\end{itemize}
			Vậy khoảng tứ phân vị  ghép nhóm diện tích căn hộ của địa phương B là là $\Delta_{Q_B} =85-65=20$. 
			\item $\Delta_{Q_B}>\Delta_{Q_A}$ nên dựa vào khoảng tứ phân vị về diện tích căn hộ người dân hai địa phương, ta thấy địa phương B phân tán hơn.
		\end{enumerate}
	}
\end{ex}

\begin{ex}%[2D3H1-3]
	Bảng tần số ghép nhóm dưới đây thể hiện kết quả điều tra về tuổi thọ trung bình của nam giới và nữ giới ở $50$ quốc gia.
	\begin{center}
		\begin{tabular}{|c|c|c|}
			\hline
			\diagbox{Nhóm (Tuổi thọ)}{Giới tính} & Nam & Nữ \\
			\hline
			$[50;55)$ & $4$ & $3$ \\
			\hline
			$[55;60)$ & $7$ & $4$ \\
			\hline
			$[60;65)$ & $4$ & $5$ \\
			\hline
			$[65;70)$ & $6$ & $3$ \\
			\hline
			$[70;75)$ & $15$ & $7$ \\
			\hline
			$[75;80)$ & $12$ & $14$ \\
			\hline
			$[80;85)$ & $2$ & $13$ \\
			\hline
			$[85;90)$ & $0$ & $1$ \\
			\hline	
		\end{tabular}
	\end{center}
	\choiceTF
	{Khoảng biến thiên của mẫu số liệu về độ tuổi trung bình của nam giới là $50$}
	{Khoảng tứ phân vị của mẫu số liệu về độ tuổi trung bình của nam giới là $14{,}75$}
	{Khoảng tứ phân vị của mẫu số liệu về độ tuổi trung bình của nữ giới là $15$}
	{\True Dựa vào khoảng tứ phân vị thì tuổi thọ trung bình của nam giới đều hơn tuổi thọ trung bình của nữ giới}
	\loigiai{
		\begin{enumerate}
			\item Khoảng biến thiên của mẫu số liệu về độ tuổi trung bình của nam giới là $90-50=40$.
			\item Xét ở nam giới, ta có cỡ mẫu $n=50$.\\
			Gọi $x_1$; $x_2$; \ldots; $x_{50}$ là mẫu số liệu gồm tuổi thọ của $50$ nam giới.\\
			Ta có: $x_1$, \ldots, $x_4\in[50;55)$; $x_5$, \ldots, $x_{11}\in[55;60)$; $x_{12}$, \ldots, $x_{15}\in[60;65)$; $x_{16}$, \ldots, $x_{21}\in[65;70)$; $x_{22}$, \ldots, $x_{36}\in[70;75)$; $x_{37}$, \ldots, $x_{48}\in[75;80)$; $x_{49}$, $x_{50}\in[80;85)$.\\
			Tứ phân vị thứ nhất của mẫu số liệu là $x_{13}\in[60;65)$. Do đó, tứ phân vị thứ nhất của mẫu số liệu nam giới là
			$$Q_1=60+\dfrac{\dfrac{50}{4}-(4+7)}{4}\cdot(65-60)=\dfrac{495}{8}.$$
			Tứ phân vị thứ ba của mẫu số liệu là $x_{38}\in[75;80)$. Do đó, tứ phân vị thứ ba của mẫu số liệu nam giới là
			$$Q_3=75+\dfrac{\dfrac{3\cdot50}{4}-(4+7+4+6+15)}{12}\cdot(80-75)=\dfrac{605}{8}.$$
			Vậy khoảng tứ phân vị của mẫu số liệu nam giới là $\Delta_Q=Q_3-Q_1=\dfrac{55}{4}=13{,}75$.
			\item Xét ở nữ giới, ta có cỡ mẫu $n=50$.\\
			Gọi $x_1$; $x_2$; \ldots; $x_{50}$ là mẫu số liệu gồm tuổi thọ của $50$ nữ giới.\\
			Ta có: $x_1$, $x_2$, $x_3\in[50;55)$; $x_4$, \ldots, $x_7\in[55;60)$; $x_8$, \ldots, $x_{12}\in[60;65)$; $x_{13}$, $x_{14}$, $x_{15}\in[65;70)$; $x_{16}$, \ldots, $x_{22}\in[70;75)$; $x_{23}$, \ldots, $x_{36}\in[75;80)$; $x_{37}$, \ldots, $x_{49}\in[80;85)$; $x_{50}\in[85;90)$.\\
			Tứ phân vị thứ nhất của mẫu số liệu là $x_{13}\in[65;70)$. Do đó, tứ phân vị thứ nhất của mẫu số liệu nữ giới là
			$$Q_1=65+\dfrac{\dfrac{50}{4}-(3+4+5)}{3}\cdot(70-65)=\dfrac{395}{6}.$$
			Tứ phân vị thứ ba của mẫu số liệu là $x_{38}\in[80;85)$. Do đó, tứ phân vị thứ ba của mẫu số liệu nữ giới là
			$$Q_3=80+\dfrac{\dfrac{3\cdot50}{4}-(3+4+5+3+7+14)}{13}\cdot(85-80)=\dfrac{2095}{26}.$$
			Vậy khoảng tứ phân vị của mẫu số liệu nữ giới là $\Delta_Q=Q_3-Q_1=\dfrac{575}{39}\approx14{,}74$.
			\item Do khoảng tứ phân vị của mẫu số liệu của nam giới nhỏ hơn mẫu số liệu của nữ giới nên tuổi thọ của nam giới đều hơn tuổi thọ của nữ giới.
	\end{enumerate}}
\end{ex}


\Closesolutionfile{ans}

%%Bài 2
% \setcounter{section}{1}
\setcounter{dang}{0}
\section{PHƯƠNG SAI VÀ ĐỘ LỆCH CHUẨN CỦA MSL GHÉP NHÓM}
\subsection{LÝ THUYẾT CẦN NHỚ}
Xét mẫu số liệu ghép nhóm cho bởi bảng sau:
\begin{center}
	\begin{tabular}{|c|c|c|c|c|}
		\hline Nhóm             & {$\left[u_1; u_2\right)$} & {$\left[u_2; u_3\right)$} & $\ldots$ & {$\left[u_k; u_{k+1}\right)$} \\
		\hline Giá trị đại diện & $c_1$                     & $c_2$                     & $\ldots$ & $c_k$                         \\
		\hline Tần số           & $n_1$                     & $n_2$                     & $\ldots$ & $n_k$                         \\
		\hline
	\end{tabular}
\end{center}
\begin{enumerate}[\iconMT]
	\item \indam{Phương sai:} Phuơng sai của mẫu số liệu ghép nhóm, kí hiệu $S^2$, được tính bởi công thức
	      \begin{align*}
			S^2&=\dfrac{1}{n}\left[n_1\left(c_1-\bar{x}\right)^2+n_2\left(c_2-\bar{x}\right)^2+\cdots+n_k\left(c_k-\bar{x}\right)^2\right]\\
		  &=\dfrac{1}{n}\left(n_1 c_1^2+n_2 c_2^2+\cdots+n_k c_k^2\right)-\overline{x}^2
		  \end{align*}
		  
	      trong đó: $n=n_1+n_2+\cdots+n_k$ là cỡ mẫu; $\bar{x}=\dfrac{1}{n}\left(n_1 c_1+n_2 c_2+\cdots+n_k c_k\right)$ là số trung bình.
	\item \indam{Độ lệch chuẩn:} Độ lệch chuẩn của mẫu số liệu ghép nhóm, kí hiệu $S$, là căn bậc hai số học của phương sai, nghĩa là $S=\sqrt{S^2}$.
	% \item \indam{Ý nghĩa:}
	%       \begin{listEX}[1]
	% 	    %   \item [\iconCH] Phương sai (độ lệch chuẩn) của mẫu số liệu ghép nhóm là giá trị xấp xỉ cho phương sai (độ lệch chuẩn) của mẫu số liệu gốc. Chúng được dùng để đo mức độ phân tán của mẫu số liệu ghép nhóm xung quanh số trung bình của mẫu số liệu. Phương sai và độ lệch chuẩn càng lớn thì dữ liệu càng phân tán.
	% 	      \item [\iconCH] Độ lệch chuẩn có cùng đơn vị với đơn vị của mẫu số liệu.
	%       \end{listEX}
\end{enumerate}

\subsection{PHÂN LOẠI VÀ PHƯƠNG PHÁP GIẢI TOÁN}
% \begin{dang}{Tính trung bình cộng của mẫu số liệu ghép nhóm}
% 	Xét mẫu số liệu ghép nhóm cho bởi bảng sau:
% 	\begin{center}
% 		\begin{tabular}{|c|c|c|c|c|}
% 			\hline Nhóm             & {$\left[u_1; u_2\right)$} & {$\left[u_2; u_3\right)$} & $\ldots$ & {$\left[u_k; u_{k+1}\right)$} \\
% 			\hline Giá trị đại diện & $c_1$                     & $c_2$                     & $\ldots$ & $c_k$                         \\
% 			\hline Tần số           & $n_1$                     & $n_2$                     & $\ldots$ & $n_k$                         \\
% 			\hline
% 		\end{tabular}
% 	\end{center}
% 	Số trung bình cộng của mẫu số liệu ghép nhóm trên được tính bằng công thức
% 	\boxmini{$\bar{x}=\dfrac{1}{n}\left(n_1 c_1+n_2 c_2+\cdots+n_k c_k\right)$}
% \end{dang}
% \boxmini{BÀI TẬP TỰ LUẬN}
% \begin{vd}%[1K3B9-1] 
% 	Tìm cân nặng trung bình của học sinh lớp $11D$ cho trong bảng sau:
% 	\begin{center}
% 		\begin{tabular}{|c|c|c|c|c|c|c|}
% 			\hline
% 			Cân nặng    & $\left[40{,}5;45{,}5 \right)$ & $\left[45{,}5;50{,}5 \right)$ & $\left[50{,}5;55{,}5 \right)$ & $\left[55{,}5;60{,}5 \right)$ & $\left[60{,}5;65{,}5 \right)$ & $\left[65{,}5;70{,}5 \right)$ \\
% 			\hline
% 			Số học sinh & $10$                          & $7$                           & $16$                          & $4$                           & $2$                           & $3$                           \\
% 			\hline
% 		\end{tabular}
% 	\end{center}
% 	\loigiai{
% 		Trong mỗi khoảng cân nặng, giá trị đại diện là trung bình cộng của hai giá trị đầu mút nên ta có bảng sau:
% 		\begin{center}
% 			\begin{tabular}{|c|c|c|c|c|c|c|}
% 				\hline
% 				Cân nặng (kg) & $43$ & $48$ & $53$ & $58$ & $63$ & $68$ \\
% 				\hline
% 				Số học sinh   & $10$ & $7$  & $16$ & $4$  & $2$  & $3$  \\
% 				\hline
% 			\end{tabular}
% 		\end{center}
% 		Tổng số học sinh là $n=42$. Cân nặng trung bình của học sinh lớp $11D$ là $$\overline{x}=\dfrac{10\cdot 43+7\cdot 48+16\cdot 53+4\cdot 58+2\cdot 63+3\cdot 68}{42}\approx51{,}81\,\mathrm{(kg)}.$$
% 	}
% \end{vd}

% \begin{vd}%[1T5B1-2]
% 	Kết quả khảo sát cân nặng của $25$ quả cam ở mỗi lô hàng $A$ và $B$ được cho ở bảng sau:
% 	\begin{center}
% 		\begin{tabular}{|c|c|c|c|c|c|}
% 			\hline \multicolumn{1}{|c|}{Cân nặng $(\mathrm{g})$} & {$[150; 155)$} & {$[155; 160)$} & {$[160; 165)$} & {$[165; 170)$} & {$[170; 175)$} \\
% 			\hline Số quả cam ở lô hàng $A$                      & 2              & 6              & 12             & 4              & 1              \\
% 			\hline Số quả cam ở lô hàng $B$                      & 1              & 3              & 7              & 10             & 4              \\
% 			\hline
% 		\end{tabular}
% 	\end{center}
% 	\begin{enumerate}
% 		\item Hãy ước lượng cân nặng trung bình của mỗi quả cam ở lô hàng $A$ và lô hàng $B$.
% 		\item Nếu so sánh theo số trung bình thì cam ở lô hàng nào nặng hơn?
% 	\end{enumerate}
% 	\loigiai{
% 		Ta có bảng thống kê số lượng cam theo giá trị đại diện:
% 		\begin{center}
% 			\begin{tabular}{|c|c|c|c|c|c|}
% 				\hline \multicolumn{1}{|c|}{Cân nặng $(\mathrm{g})$} & {$152{,}5$} & {$157{,}5$} & {$162{,}5$} & {$167{,}5$} & $172{,}5$ \\
% 				\hline Số quả cam ở lô hàng $A$                      & 2           & 6           & 12          & 4           & 1         \\
% 				\hline Số quả cam ở lô hàng $B$                      & 1           & 3           & 7           & 10          & 4         \\
% 				\hline
% 			\end{tabular}
% 		\end{center}
% 		\begin{enumerate}
% 			\item Cân nặng trung bình của mỗi quả cam ở lô hàng $A$ xấp xỉ bằng
% 			      \[(2\cdot 152{,}5+6\cdot 157{,}5+12\cdot 162{,}5+4\cdot 167{,}5+1\cdot 172{,}5): 25=161{,}7\ (\mathrm{g}). \]
% 			      Cân nặng trung bình của mỗi quả cam ở lô hàng $B$ xấp xỉ bằng
% 			      \[(1\cdot 152{,}5+3\cdot 157{,}5+7\cdot 162{,}5+10\cdot 167{,}5+4\cdot 172{,}5): 25=165{,}1\ (\mathrm{g}). \]
% 			\item Nếu so sánh theo số trung bình thì cam ở lô hàng $B$ nặng hơn cam ở lô hàng $A$.
% 		\end{enumerate}
% 	}
% \end{vd}

% \boxmini{BÀI TẬP TRẮC NGHIỆM}
% \Opensolutionfile{ans}[ans/2D3-B2-d1]
% \begin{ex}%%[1D1Y1-2]
% 	Cho mẫu số liệu với cỡ mẫu $n$ được cho dưới bảng tần số ghép nhóm
% 	\begin{center}
% 		\begin{tabular}{|c|c|c|c|c|}
% 			\hline Nhóm             & {$\left[u_1 ; u_2\right)$} & {$\left[u_2 ; u_3\right)$} & $\ldots$ & {$\left[u_k ; u_{k+1}\right)$} \\
% 			\hline Giá trị đại diện & $c_1$                      & $c_2$                      & $\ldots$ & $c_k$                          \\
% 			\hline Tần số           & $n_1$                      & $n_2$                      & $\ldots$ & $n_k$                          \\
% 			\hline
% 		\end{tabular}
% 	\end{center}
% 	Số trung bình $\overline x $ của mẫu số liệu trên được tính bằng công thức nào sau đây
% 	\choice
% 	{$\overline x=\dfrac{u_1+u_2+\ldots+u_k}{n}$}
% 	{$\overline x=\dfrac{c_1+c_2+\ldots+c_k}{n}$}
% 	{$\overline x=\dfrac{n_1u_1+n_2u_2+\ldots+n_k{u_k}}{n}$}
% 	{\True $\overline x=\dfrac{n_1c_1+n_2c_2+\ldots+n_k{c_k}}{n}$}
% 	\loigiai{}
% \end{ex}

% \begin{ex}%[1D1B1-2]
% 	Khảo sát về cân nặng của các học sinh lớp $11D3$ người ta được một mẫu dữ liệu ghép nhóm như sau:
% 	\begin{center}
% 		\begin{tabular}{|c|c|c|c|c|c|c|}
% 			\hline Cân nặng    & {$[30 ; 40)$} & {$[40 ; 50)$} & {$[50 ; 60)$} & {$[60 ; 70)$} & {$[70 ; 80)$} & {$[80 ; 90)$} \\
% 			\hline Số học sinh & $2$           & $10$          & $16$          & $8$           & $2$           & $2$           \\
% 			\hline
% 		\end{tabular}
% 	\end{center}
% 	Số trung bình của mẫu số liệu trên là
% 	\choice
% 	{\True $56$}
% 	{$50$}
% 	{$60$}
% 	{$55$}
% 	\loigiai{
% 		Ta có: Số phần tử của mẫu là $n=40$ và
% 		\begin{center}
% 			\begin{tabular}{|c|c|c|c|c|c|c|}
% 				\hline Cân nặng         & {$[30 ; 40)$} & {$[40 ; 50)$} & {$[50 ; 60)$} & {$[60 ; 70)$} & {$[70 ; 80)$} & {$[80 ; 90)$} \\
% 				\hline Giá trị đại diện & $35$          & $45$          & $55$          & $65$          & $75$          & $85$          \\
% 				\hline Số học sinh      & $2$           & $10$          & $16$          & $8$           & $2$           & $2$           \\
% 				\hline
% 			\end{tabular}
% 		\end{center}
% 		Do đó giá trị trung bình của mẫu số liệu trên là\\
% 		$\overline x=\dfrac{35\cdot2+45\cdot10+55\cdot16+65\cdot8+75\cdot2+85\cdot2}{40}=56$.}
% \end{ex}

% \begin{ex}%[1D1B1-2]
% 	Thống kê về thời lượng mỗi trận đấu bi-a trong vòng tứ kết giải đấu European Open người ta được mẫu số liệu ghép nhóm như sau
% 	\begin{center}
% 		\begin{tabular}{|c|c|c|c|c|c|}
% 			\hline Thời gian & {$[9{,}5 ; 12{,}5)$} & {$[12{,}5 ; 15{,}5)$} & {$[15{,}5 ; 18{,}5)$} & {$[18{,}5 ; 21{,}5)$} & {$[21{,}5 ; 24{,}5)$} \\
% 			\hline Số trận   & $3$                  & $12$                  & $15$                  & $24$                  & $2$                   \\
% 			\hline
% 		\end{tabular}
% 	\end{center}
% 	Số trung bình của mẫu số liệu trên gần nhất với giá trị nào sau đây
% 	\choice{$17$}
% 	{\True $17{,}5$}
% 	{$18$}
% 	{$18{,}5$}
% 	\loigiai{
% 		Ta có số phần tử của mẫu là $n=56$ và
% 		\begin{center}
% 			\begin{tabular}{|c|c|c|c|c|c|}
% 				\hline Thời gian        & {$[9{,}5 ; 12{,}5)$} & {$[12{,}5 ; 15{,}5)$} & {$[15{,}5 ; 18{,}5)$} & {$[18{,}5 ; 21{,}5)$} & {$[21{,}5 ; 24{,}5)$} \\
% 				\hline Giá trị đại diện & $11$                 & $14$                  & $17$                  & $20$                  & $2$3                  \\
% 				\hline Số trận          & $3$                  & $12$                  & $15$                  & $24$                  & $2$                   \\
% 				\hline
% 			\end{tabular}
% 		\end{center}
% 		Do đó giá trị trung bình của mẫu số liệu trên là
% 		$$
% 			\overline{x}=\dfrac{11\cdot3+14\cdot12+17\cdot15+20\cdot24+23\cdot2}{56}=\dfrac{491}{28} \approx 17{,}54.$$
% 	}
% \end{ex}

% \begin{ex}%[1D1B1-2]
% 	Doanh thu bán hàng trong $20$ ngày được lựa chọn ngẫu nhiên của một của hàng được ghi lại ở bảng sau (đơn vị: triệu đồng)
% 	\begin{center}
% 		\begin{tabular}{|c|c|c|c|c|c|}
% 			\hline Doanh thu & {$[5 ; 7)$} & {$[7 ; 9)$} & {$[9 ; 11)$} & {$[11 ; 13)$} & {$[13 ; 15)$} \\
% 			\hline Số ngày   & $2$         & $7$         & $7$          & $3$           & $1$           \\
% 			\hline
% 		\end{tabular}
% 	\end{center}
% 	Số trung bình của mẫu số liệu trên thuộc khoảng nào trong các khoảng dưới đây?
% 	\choice{ $[7 ; 9)$}
% 	{ \True $[9 ; 11)$}
% 	{ $[11 ; 13)$}
% 	{ $[13 ; 15)$}
% 	\loigiai{
% 	Bảng tần số ghép nhóm theo giá trị đại diện là
% 	\begin{center}
% 		\begin{tabular}{|c|c|c|c|c|c|}
% 			\hline Doanh thu        & {$[5 ; 7)$} & {$[7 ; 9)$} & {$[9 ; 11)$} & {$[11 ; 13)$} & {$[13 ; 15)$} \\
% 			\hline Giá trị đại diện & $6$         & $8$         & $10$         & $12$          & $14$          \\
% 			\hline Số ngày          & $2$         & $7$         & $7$          & $3$           & $1$           \\
% 			\hline
% 		\end{tabular}
% 	\end{center}
% 	Số trung bình $\overline{x}=\dfrac{2\cdot 6+7\cdot8+7\cdot10+3\cdot12+1\cdot 14}{20}=9{,}4$.
% 	}
% \end{ex}

% \begin{ex}%[1D1B1-2]
% 	Trung tâm ngoại ngữ thống kê bảng điểm môn Tiếng Anh của một khóa học trong bảng bên dưới
% 	\begin{center}
% 		\begin{tabular}{|l|c|c|c|c|c|}
% 			\hline
% 			Điểm     & [0;2) & [2;4) & [4;6) & [6;8) & [8;10) \\ \hline
% 			Học viên & 10    & 30    & 55    & 42    & 9      \\ \hline
% 		\end{tabular}
% 	\end{center}
% 	Số trung bình của mẫu số liệu thuộc khoảng nào trong các khoảng dưới đây?
% 	\choice
% 	{$[8;10)$}
% 	{\True $[4;6)$}
% 	{$[2;4)$}
% 	{$[6;8)$}
% 	\loigiai{
% 		Ta có bảng thống kê theo giá trị đại diện như sau:
% 		\begin{center}
% 			\begin{tabular}{|l|c|c|c|c|c|}
% 				\hline
% 				Giá trị đại diện & 1  & 3  & 5  & 7  & 9 \\ \hline
% 				Tần số           & 10 & 30 & 55 & 42 & 9 \\ \hline
% 			\end{tabular}
% 		\end{center}
% 		Khi đó ta có số trung bình của mẫu số liệu trên được tính như sau: \\
% 		$$\bar{x}=\dfrac{1.10+3.30+5.55+7.42+9.9}{10+30+55+42+9}\approx 5,14.$$
% 	}
% \end{ex}

% \Closesolutionfile{ans}
\begin{dang}{Tính phương sai và độ lệch chuẩn của mẫu số liệu ghép nhóm}
	\begin{listEX}[1]
		\item [\ding{172}] Xác định cỡ của mẫu số liệu;
		\item [\ding{173}] Tính số trung bình của mẫu số liệu;
		\item [\ding{174}] Áp dụng công thức tính phương sai và độ lệch chuẩn.
	\end{listEX}
\end{dang}
\setcounter{vd}{0}
\setcounter{ex}{0}
% \boxmini{BÀI TẬP TỰ LUẬN}
\viduminhhoa
\begin{vd}
	Cân nặng của một số quả mít trong một khu vườn được thống kê ở bảng sau:
	\begin{center}
		\begin{tabular}{|c|c|c|c|c|c|}
			\hline Cân nặng (kg) & {$[4; 6)$} & {$[6; 8)$} & {$[8; 10)$} & {$[10; 12)$} & {$[12; 14)$} \\
			\hline Số quả mít    & $ 6 $      & $ 12 $     & $ 19 $      & $ 9 $        & $ 4 $        \\
			\hline
		\end{tabular}
	\end{center}
	Hãy tính phương sai và độ lệch chuẩn của mẫu số liệu ghép nhóm trên. (Kết quả các phép tính làm tròn đến hàng phần trăm.)
	\loigiai{
		Ta có bảng thống kê cân nặng của các quả mít theo giá trị đại diện:
		\begin{center}
			\begin{tabular}{|c|c|c|c|c|c|}
				\hline Cân nặng đại diện $(\mathrm{kg})$ & $ 5 $ & $ 7 $  & $ 9 $  & $ 11 $ & $ 13 $ \\
				\hline Tần số                            & $ 6 $ & $ 12 $ & $ 19 $ & $ 9 $  & $ 4 $  \\
				\hline
			\end{tabular}
		\end{center}
		Cỡ mẫu $n=6+12+19+9+4=50$.\\
		Số trung bình của mẫu số liệu ghép nhóm là
		$$\bar{x}=\dfrac{6\cdot 5+12\cdot 7+19\cdot 9+9\cdot 11+4\cdot 13}{50}=8,72.$$
		Phương sai của mẫu số liệu ghép nhóm là
		$$S^2=\dfrac{1}{50}\left(6 \cdot 5^2+12 \cdot 7^2+19 \cdot 9^2+9 \cdot 11^2+4 \cdot 13^2\right)-8,72^2 \approx 4,80.$$
		Độ lệch chuẩn của mẫu số liệu ghép nhóm là
		$$S \approx \sqrt{4,80} \approx 2,19.$$
	}
\end{vd}

\begin{vd}
	Thống kê tổng số giờ nắng trong tháng 9 tại một trạm quan trắc đặt ở Cà Mau trong các năm từ 2002 đến 2021 được thống kê như sau:
	\begin{center}
		\begin{tabular}{cccccccccc}
			$ 111,6 $ & $ 134,9 $ & $ 130,3 $ & $ 134,2 $ & $ 140,9 $ & $ 109,3 $ & $ 154,4 $ & $ 156,3 $ & $ 116,1 $ & $ 96,7 $ \\
			$ 105,2 $ & $ 80,8 $  & $ 80,8 $  & $ 110 $   & $ 109 $   & $ 139 $   & $ 145 $   & $ 161 $   & $ 126 $   & $ 114 $
		\end{tabular}
	\end{center}
	\begin{flushright}
		(Nguồn: Tổng cục Thống kê)
	\end{flushright}
	\begin{enumEX}{1}
		\item Hãy tính phương sai và độ lệch chuẩn của mẫu số liệu trên.
		\item Hãy lập bảng tần số ghép nhóm với nhóm đầu tiên là $[80; 98)$ và độ dài mỗi nhóm bằng $ 18 $. Tính phương sai, độ lệch chuẩn của mẫu số liệu ghép nhóm.
		\item Hãy tính sai số tương đối của độ lệch chuẩn của mẫu số liệu ghép nhóm so với độ lệch chuẩn của mẫu số liệu gốc.
	\end{enumEX}
	\loigiai{
		\begin{enumEX}{1}
			\item Cỡ mẫu là $n=20$.\\
			Số trung bình của mẫu số liệu trên là
			$$\bar{x}_1=\dfrac{111,6+134,9+\cdots+114}{20}=122,755.$$
			Phương sai của mẫu số liệu trên là
			$$S_1^2=\dfrac{1}{20}\left(111,6^2+134,9^2+\cdots+114^2\right)-122,755^2 \approx 515,453.$$
			Độ lệch chuẩn của mẫu số liệu trên là
			$$S_1 \approx \sqrt{515,453} \approx 22,704.$$
			\item Ta có bảng sau:
			\begin{center}
				\begin{tabular}{|c|c|c|c|c|c|}
					\hline Số giờ nắng      & {$[80; 98)$} & {$[98; 116)$} & {$[116; 134)$} & {$[134; 152)$} & {$[152; 170)$} \\
					\hline Giá trị đại diện & 89           & 107           & 125            & 143            & 161            \\
					\hline Số năm           & 3            & 6             & 3              & 5              & 3              \\
					\hline
				\end{tabular}
			\end{center}
			Số trung bình của mẫu số liệu ghép nhóm là
			$$\bar{x}_2=\dfrac{3\cdot 89+6\cdot 107+3\cdot 125+5\cdot 143+3\cdot 161}{20}=124,1.$$
			Phương sai của mẫu số liệu ghép nhóm là
			$$S_2^2=\dfrac{1}{20}\left(3\cdot 89^2+6\cdot 107^2+3\cdot 125^2+5\cdot 143^2+3\cdot 161^2\right)-124,1^2=566,19.$$
			Độ lệch chuẩn của mẫu số liệu ghép nhóm là
			$$S_2=\sqrt{566,19} \approx 23,795.$$
			\item Sai số tương đối của độ lệch chuẩn của mẫu số liệu ghép nhóm so với độ lệch chuẩn của mẫu số liệu gốc là
			$$\dfrac{\left|S_2-S_1\right|}{S_1}=\dfrac{|23,795-22,704|}{22,704} \cdot 100 \% \approx 4,805 \%.$$
		\end{enumEX}
	}
\end{vd}

\begin{vd}
	Thầy Tuấn thống kê lại điểm trung bình cuối năm của các học sinh lớp $11 \mathrm{A}$ và $ 11\mathrm{B} $ ở bảng sau:
	\begin{center}
		\begin{tabular}{|c|c|c|c|c|c|}
			\hline Điểm trung bình                  & {$[5; 6)$} & {$[6; 7)$} & {$[7; 8)$} & {$[8; 9)$} & {$[9; 10)$} \\
			\hline Số học sinh lớp $ 11\mathrm{A} $ & $ 1 $      & $ 0 $      & $ 11 $     & $ 22 $     & $ 6 $       \\
			\hline Số học sinh lớp $ 11\mathrm{B} $ & $ 0 $      & $ 6 $      & $ 8 $      & $ 14 $     & $ 12 $      \\
			\hline
		\end{tabular}
	\end{center}
	\begin{enumEX}{1}
		\item Nếu so sánh theo khoảng biến thiên thì học sinh lớp nào có điểm trung bình ít phân tán hơn?
		\item Nếu so sánh theo độ lệch chuẩn thì học sinh lớp nào có điểm trung bình ít phân tán hơn?
	\end{enumEX}
	\loigiai{
		\begin{enumEX}{1}
			\item Khoảng biến thiên của điểm trung bình của học sinh lớp $11 \mathrm{A}$ là: $10-5=5$.\\
			Khoảng biến thiên của điểm trung bình của học sinh lớp $ 11\mathrm{B} $ là: $10-6=4$.\\
			Nếu so sánh theo khoảng biến thiên thì điểm trung bình của các học sinh lớp $ 11\mathrm{B} $ ít phân tán hơn điểm trung bình của các học sinh lớp $ 11\mathrm{A} $.
			\item Ta có bảng thống kê điểm trung bình theo giá trị đại diện:
			\begin{center}
				\begin{tabular}{|c|c|c|c|c|c|}
					\hline Giá trị đại diện                 & $ 5,5 $ & $ 6,5 $ & $ 7,5 $ & $ 8,5 $ & $ 9,5 $ \\
					\hline Số học sinh lớp $ 11\mathrm{A} $ & $ 1 $   & $ 0 $   & $ 11 $  & $ 22 $  & $ 6 $   \\
					\hline Số học sinh lớp $ 11\mathrm{B} $ & $ 0 $   & $ 6 $   & $ 8 $   & $ 14 $  & $ 12 $  \\
					\hline
				\end{tabular}
			\end{center}
			\begin{itemize}
				\item Xét mẫu số liệu của lớp $ 11\mathrm{A} $:
				      \begin{itemize}
					      \item Cỡ mẫu là $n_1=1+11+22+6=40$.
					      \item Số trung bình của mẫu số liệu ghép nhóm là
					            $$\bar{x}_1=\dfrac{1 \cdot 5,5+11 \cdot 7,5+22 \cdot 8,5+6 \cdot 9,5}{40}=8,3.$$
					      \item Phương sai của mẫu số liệu ghép nhóm là
					            $$S_1^2=\dfrac{1}{40}\left(1 \cdot 5,5^2+11 \cdot 7,5^2+22 \cdot 8,5^2+6 \cdot 9,5^2\right)-8,3^2=0,61.$$
					      \item Độ lệch chuẩn của mẫu số liệu ghép nhóm là $S_1=\sqrt{0,61}$.
				      \end{itemize}
				\item Xét mẫu số liệu của lớp $ 11\mathrm{B} $:
				      \begin{itemize}
					      \item Cỡ mẫu là $n_2=6+8+14+12=40$.
					      \item Số trung bình của mẫu số liệu ghép nhóm là
					            $$
						            \bar{x}_2=\dfrac{6 \cdot 6,5+8 \cdot 7,5+14 \cdot 8,5+12 \cdot 9,5}{40}=8,3.
					            $$
					      \item Phương sai của mẫu số liệu ghép nhóm là
					            $$
						            S_2^2=\dfrac{1}{40}\left(6 \cdot 6,5^2+8 \cdot 7,5^2+14 \cdot 8,5^2+12 \cdot 9,5^2\right)-8,3^2=1,06.
					            $$
					      \item Độ lệch chuẩn của mẫu số liệu ghép nhóm là $S_2=\sqrt{1,06}$.
				      \end{itemize}
			\end{itemize}
			Do $S_1<S_2$ nên nếu so sánh theo độ lệch chuẩn thì học sinh lớp $11 \mathrm{A}$ có điểm trung bình ít phân tán hơn học sinh lớp $ 11\mathrm{B} $.
		\end{enumEX}
	}
\end{vd}

\begin{vd}
	Giá đóng cửa của một cổ phiếu là giá của cổ phiếu đó cuối một phiên giao dịch. Bảng sau thống kê giá đóng cửa (đơn vị: nghìn đồng) của hai mã cổ phiếu $A$ và $B$ trong $ 50 $ ngày giao dịch liên tiếp.
	\begin{center}
		\begin{tabular}{|c|c|c|c|c|c|}
			\hline Giá đóng cửa       & {$[120; 122)$} & {$[122; 124)$} & {$[124; 126)$} & {$[126; 128)$} & {$[128; 130)$} \\
			\hline \begin{tabular}{c}
				       Số ngày giao dịch \\
				       của cổ phiếu $A$
			       \end{tabular} & $ 8 $          & $ 9 $          & $ 12 $         & $ 10 $         & $ 11 $              \\
			\hline \begin{tabular}{c}
				       Số ngày giao dịch \\
				       của cổ phiếu $B$
			       \end{tabular} & $ 16 $         & $ 4 $          & $ 3 $          & $ 6 $          & $ 21 $              \\
			\hline
		\end{tabular}
	\end{center}
	Người ta có thể dùng phương sai và độ lệch chuẩn để so sánh mức độ rủi ro của các loại cổ phiếu có giá trị trung bình gần bằng nhau. Cổ phiếu nào có phương sai, độ lệch chuẩn cao hơn thì được coi là có độ rủi ro lớn hơn.\\
	Theo quan điểm trên, hãy so sánh độ rủi ro của cổ phiếu $A$ và cổ phiếu $B$.
	\loigiai{
		Ta có bảng thống kê giá đóng cửa theo giá trị đại diện:
		\begin{center}
			\begin{tabular}{|c|c|c|c|c|c|}
				\hline Giá đóng cửa       & $ 121 $ & $ 123 $ & $ 125 $ & $ 127 $ & $ 129 $ \\
				\hline \begin{tabular}{c}
					       Số ngày giao dịch \\
					       của cổ phiếu $A$
				       \end{tabular} & $ 8 $   & $ 9 $   & $ 12 $  & $ 10 $  & $ 11 $       \\
				\hline \begin{tabular}{c}
					       Số ngày giao dịch \\
					       của cổ phiếu $B$
				       \end{tabular} & $ 16 $  & $ 4 $   & $ 3 $   & $ 6 $   & $ 21 $       \\
				\hline
			\end{tabular}
		\end{center}
		\begin{itemize}
			\item Xét mẫu số liệu của cổ phiếu $A$:
			      \begin{itemize}
				      \item Số trung bình của mẫu số liệu ghép nhóm là
				            $$
					            \bar{x}_1=\dfrac{8 \cdot 121+9\cdot 123+12 \cdot 125+10\cdot 127+11\cdot 129}{50}=125,28.
				            $$
				      \item Phương sai của mẫu số liệu ghép nhóm là
				            $$
					            S_1^2=\dfrac{1}{50}\left(8 \cdot 121^2+9 \cdot 123^2+12 \cdot 125^2+10 \cdot 127^2+11 \cdot 129^2\right)-(125,28)^2=7,5216.
				            $$
				      \item Độ lệch chuẩn của mẫu số liệu ghép nhóm là $S_1=\sqrt{S_1^2}=\sqrt{7,5216}$.
			      \end{itemize}
			\item Xét mẫu số liệu của cổ phiếu $B$:
			      \begin{itemize}
				      \item Số trung bình của mẫu số liệu ghép nhóm là
				            $$
					            \bar{x}_2=\dfrac{16\cdot 121+4\cdot 123+3\cdot 125+6\cdot 127+21\cdot 129}{50}=125,28.
				            $$
				      \item Phương sai của mẫu số liệu ghép nhóm là
				            $$
					            S_2^2=\dfrac{1}{50}\left(16\cdot 121^2+4\cdot 123^2+3 \cdot 125^2+6 \cdot 127^2+21\cdot 129^2\right)-(125,48)^2=12,4096.$$
				      \item Độ lệch chuẩn của mẫu số liệu ghép nhóm là $S_2=\sqrt{S_2^2}=\sqrt{12,4096}$.
			      \end{itemize}
		\end{itemize}
		Vậy nếu đánh giá độ rủi ro theo phương sai và độ lệch chuẩn thì cổ phiếu $A$ có độ rủi ro thấp hơn cổ phiếu $B$.
	}
\end{vd}
\baitaptn
% \boxmini{BÀI TẬP TRẮC NGHIỆM}
% \ind{PHẦN I.} \inden{Câu trắc nghiệm nhiều phương án lựa chọn. Mỗi câu hỏi học sinh chỉ chọn một phương án.}\\
\setcounter{ex}{0}
\Opensolutionfile{ans}[ans/2D3-B2-d2-1]
\begin{ex}
	Trong các khẳng định sau, khẳng định nào sai?
	\choice
	{Phương sai luôn luôn là số không âm}
	{Phương sai là bình phương của độ lệch chuẩn}
	{Phương sai càng lớn thì độ phân tán của các giá trị quanh số trung bình càng lớn}
	{\True Phương sai luôn luôn lớn hơn độ lệch chuẩn}
	\loigiai{
		Ta có khi $s \in (0;1)$ thì $s^2 < s$. Do đó khẳng định phương sai luôn lớn hơn độ lệch chuẩn là sai.}
\end{ex}

\begin{ex}
	Số đặc trưng nào không sử dụng thông tin của nhóm số liệu đầu tiên và nhóm số liệu cuối cùng?
	\choice
	{Khoảng biến thiên}
	{\True Khoảng tứ phân vị}
	{Phương sai}
	{Độ lệch chuẩn}
	\loigiai{
		Số đặc trưng tứ phân vị không sử dụng thông tin của nhóm số liệu đầu tiên và nhóm số liệu cuối cùng
	}
\end{ex}

\begin{ex}%[2D4H2-2]
	Mỗi ngày bác Hương đều đi bộ để rèn luyện sức khỏe. Quãng đường đi bộ mỗi ngày (đơn vị km) của bác Hương trong $20$ ngày được thống kê lại ở bảng sau
	\begin{center}
		\begin{tabular}{|c|c|c|c|c|c|}
			\hline
			Quãng đường (km) & $[2{,}7;3{,}0)$ & $[3{,}0;3{,}3)$ & $[3{,}3;3{,}6)$ & $[3{,}6;3{,}9)$ & $[3{,}9;4{,}2)$ \\
			\hline
			Số ngày          & $3$             & $6$             & $5$             & $4$             & $2$             \\
			\hline
		\end{tabular}
	\end{center}
	Phương sai của mẫu số liệu ghép nhóm là
	\choice
	{$3{,}39$}
	{$11{,}62$}
	{\True $0{,}1314$}
	{$0{,}36$}
	\loigiai
	{
	Xét mẫu số liệu ghép nhóm cho bởi bảng sau
	\begin{center}
		\begin{tabular}{|c|c|c|c|c|c|}
			\hline
			Nhóm             & $[2{,}7;3{,}0)$ & $[3{,}0;3{,}3)$ & $[3{,}3;3{,}6)$ & $[3{,}6;3{,}9)$ & $[3{,}9;4{,}2)$ \\
			\hline
			Giá trị đại diện & $2{,}85$        & $3{,}15$        & $3{,}45$        & $3{,}75$        & $4{,}05$        \\
			\hline
			Tần số           & $3$             & $6$             & $5$             & $4$             & $2$             \\
			\hline
		\end{tabular}
	\end{center}
	Số trung bình của mẫu số liệu là
	$$\overline{x}=\dfrac{1}{20}\cdot (2{,}85\cdot 3+3{,}15\cdot 6+3{,}45\cdot 5+3{,}75\cdot 4+4{,}05\cdot 2)=3{,}39.$$
	Phương sai của mẫu số liệu ghép nhóm là
	$$S^2=\dfrac{1}{20}\left(3\cdot 2{,}85^2+6\cdot 3{,}15^2+5\cdot 3{,}45^2+4\cdot 3{,}45^2+2\cdot 4{,}05^2\right)-3{,}39^2=0{,}1314.$$
	}
\end{ex}


\begin{ex}%[2D4H2-2]
	Bạn Chi rất thích nhảy hiện đại. Thời gian tập nhảy mỗi ngày trong thời gian gần đây của bạn Chi được thống kê lại ở bảng sau
	\begin{center}
		\begin{tabular}{|c|c|c|c|c|c|}
			\hline
			Thời gian (phút) & $[20;25)$ & $[25;30)$ & $[30;35)$ & $[35;40)$ & $[40;45)$ \\
			\hline
			Số ngày          & $6$       & $6$       & $4$       & $1$       & $1$       \\
			\hline
		\end{tabular}
	\end{center}
	Phương sai của mẫu số liệu ghép nhóm có giá trị gần nhất với giá trị nào dưới đây?
	\choice
	{$31{,}77$}
	{$32$}
	{$31$}
	{\True $31{,}44$}
	\loigiai
	{
	Xét mẫu số liệu ghép nhóm cho bởi bảng sau
	\begin{center}
		\begin{tabular}{|c|c|c|c|c|c|}
			\hline
			Nhóm             & $[20;25)$ & $[25;30)$ & $[30;35)$ & $[35;40)$ & $[40;45)$ \\
			\hline
			Giá trị đại diện & $22{,}5$  & $27{,}5$  & $32{,}5$  & $37{,}5$  & $42{,}5$  \\
			\hline
			Tần số           & $6$       & $6$       & $4$       & $1$       & $1$       \\
			\hline
		\end{tabular}
	\end{center}
	Số trung bình của mẫu số liệu là
	$$\overline{x}=\dfrac{1}{18}\cdot (22{,}5\cdot 6+27{,}5\cdot 6+32{,}5\cdot 4+37{,}5\cdot 1+42{,}5\cdot 1)=\dfrac{85}{3}.$$
	Phương sai của mẫu số liệu ghép nhóm là
	$$S^2=\dfrac{1}{18}\left(6\cdot 22{,}5^2+6\cdot 27{,}5^2+4\cdot 32{,}5^2+1\cdot 37{,}5^2+1\cdot 42{,}5^2\right)-\left(\dfrac{85}{3}\right)^2=31{,}25.$$
	Vậy phương sai của mẫu số liệu ghép nhóm gần nhất với $31{,}44$.
	}
\end{ex}


\begin{ex}%[2D4H2-2]
	Mỗi ngày bác Hương đều đi bộ để rèn luyện sức khỏe. Quãng đường đi bộ mỗi ngày (đơn vị km) của bác Hương trong $20$ ngày được thống kê lại ở bảng sau
	\begin{center}
		\begin{tabular}{|c|c|c|c|c|c|}
			\hline
			Quãng đường (km) & $[2{,}7;3{,}0)$ & $[3{,}0;3{,}3)$ & $[3{,}3;3{,}6)$ & $[3{,}6;3{,}9)$ & $[3{,}9;4{,}2)$ \\
			\hline
			Số ngày          & $3$             & $6$             & $5$             & $4$             & $2$             \\
			\hline
		\end{tabular}
	\end{center}
	Độ lệch chuẩn của mẫu số liệu ghép nhóm có giá trị gần nhất với giá trị nào dưới đây?
	\choice
	{$3{,}41$}
	{$11{,}62$}
	{$0{,}017$}
	{\True $0{,}36$}
	\loigiai
	{
	Xét mẫu số liệu ghép nhóm cho bởi bảng sau
	\begin{center}
		\begin{tabular}{|c|c|c|c|c|c|}
			\hline
			Nhóm             & $[2{,}7;3{,}0)$ & $[3{,}0;3{,}3)$ & $[3{,}3;3{,}6)$ & $[3{,}6;3{,}9)$ & $[3{,}9;4{,}2)$ \\
			\hline
			Giá trị đại diện & $2{,}85$        & $3{,}15$        & $3{,}45$        & $3{,}75$        & $4{,}05$        \\
			\hline
			Tần số           & $3$             & $6$             & $5$             & $4$             & $2$             \\
			\hline
		\end{tabular}
	\end{center}
	Số trung bình của mẫu số liệu là
	$$\overline{x}=\dfrac{1}{20}\cdot (2{,}85\cdot 3+3{,}15\cdot 6+3{,}45\cdot 5+3{,}75\cdot 4+4{,}05\cdot 2)=3{,}39.$$
	Phương sai của mẫu số liệu ghép nhóm là
	$$S^2=\dfrac{1}{20}\left(3\cdot 2{,}85^2+6\cdot 3{,}15^2+5\cdot 3{,}45^2+4\cdot 3{,}45^2+2\cdot 4{,}05^2\right)-3{,}39^2=0{,}1314.$$
	Độ lệch chuẩn của mẫu số liệu ghép nhóm là $S=\sqrt{0{,}1314}\approx 0{,}36$.
	}
\end{ex}


\begin{ex}
	Dũng là học sinh rất giỏi chơi rubik, bạn có thể giải nhiều loại khối rubik khác nhau. Trong một lần tập luyện giải khối rubik $3\times 3$, bạn Dũng đã tự thống kê lại thời gian giải	rubik trong $25$ lần giải liên tiếp ở bảng sau
	\begin{center}
		\begin{tabular}{|c|c|c|c|c|c|}
			\hline
			Thời gian giải rubik (giây) & $[8;10)$ & $[10;12)$ & $[12;14)$ & $[14;16)$ & $[16;18)$ \\
			\hline
			Số ngày                     & $4$      & $6$       & $8$       & $4$       & $3$       \\
			\hline
		\end{tabular}
	\end{center}
	Độ lệch chuẩn của mẫu số liệu ghép nhóm có giá trị gần nhất với giá trị nào dưới đây?
	\choice
	{$5{,}98$}
	{$6$}
	{\True $2{,}44$}
	{$2{,}5$}
	\loigiai
	{
	Xét mẫu số liệu ghép nhóm cho bởi bảng sau
	\begin{center}
		\begin{tabular}{|c|c|c|c|c|c|}
			\hline
			Nhóm             & $[8;10)$ & $[10;12)$ & $[12;14)$ & $[14;16)$ & $[16;18)$ \\
			\hline
			Giá trị đại diện & $9$      & $11$      & $13$      & $15$      & $17$      \\
			\hline
			Tần số           & $4$      & $6$       & $8$       & $4$       & $3$       \\
			\hline
		\end{tabular}
	\end{center}
	Số trung bình của mẫu số liệu là
	$$\overline{x}=\dfrac{1}{25}\cdot (9\cdot 4+11\cdot 6+13\cdot 8+15\cdot 4+17\cdot 3)=12{,}68.$$
	Phương sai của mẫu số liệu ghép nhóm là
	$$S^2=\dfrac{1}{25}\left(4\cdot 9^2+6\cdot 11^2+8\cdot 13^2+4\cdot 15^2+3\cdot 17^2\right)-12{,}68^2=5{,}9776.$$
	Độ lệch chuẩn của mẫu số liệu là
	$$S=\sqrt{5{,}9776}=\approx 2{,}445.$$
	Vậy độ lệch chuẩn của mẫu số liệu ghép nhóm gần nhất với $2{,}44$.
	}
\end{ex}

\begin{ex}
	Để đánh giá chất lượng một lọa pin điện thoại mới, người ta ghi lại thời gian nghe nhạc liên tục của điện thoại được sạc đầy pin cho đến khi hết pin cho kết quả sau
	\begin{center}
		\begin{tabular}{|p{5cm}|c|c|c|c|c|}
			\hline
			Thời gian (giờ)              & $ [5;5{,}5) $ & $ [5{,}5;6) $ & $ [6;6{,}5) $ & $ [6{,}5;7) $ & $ [7;7{,}5) $ \\
			\hline
			Số chiếc điện thoại (tần số) & $ 2 $         & $ 8 $         & $ 15 $        & $ 10 $        & $ 5 $         \\
			\hline
		\end{tabular}
	\end{center}
	Tính độ lệch chuẩn của mẫu số liệu ghép nhóm trên (làm tròn đến 4 chữ số thập phân).
	\choice
	{$0{,}4252$}
	{$0{,}5314$}
	{$0{,}6214$}
	{\True $0{,}5268$}
	\loigiai{\begin{center}
		\begin{tabular}{|p{5cm}|c|c|c|c|c|}
			\hline
			Thời gian (giờ)              & $ [5;5{,}5) $ & $ [5{,}5;6) $ & $ [6;6{,}5) $ & $ [6{,}5;7) $ & $ [7;7{,}5) $ \\
			\hline
			Giá trị đại diện             & $ 5{,}25 $    & $ 5{,}75 $    & $ 6{,}25$     & $ 6{,}75$     & $ 7{,}25 $    \\
			\hline
			Số chiếc điện thoại (tần số) & $ 2 $         & $ 8 $         & $ 15 $        & $ 10 $        & $ 5 $         \\
			\hline
		\end{tabular}
	\end{center}
	Số trung bình của mẫu số liệu\\
	$ \overline{x}=\dfrac{m_{1}\cdot x_{1}+\dots+m_{k}\cdot x_{k}}{n}=\dfrac{2\cdot5{,}25+8\cdot 5{,}75+15\cdot 6{,}25+10\cdot 6{,}75+5\cdot7{,}25}{40}=6{,}35 $.\\
		Phương sai của mẫu số liệu ghép nhóm
		\begin{center}
			$ s^{2}=\dfrac{1}{40}\cdot\left(2\cdot 5{,}25^{2}+8\cdot 5{,}75^{2}+15\cdot 6{,}25^{2}+10\cdot 6{,}75^{2}+5\cdot 7{,}25^{2}\right)-6{,}35^{2}=0{,}2775 $.
		\end{center}
		Độ lệch chuẩn của mẫu số liệu ghép nhóm
		\begin{center}
			$ s=\sqrt{s^{2}}=\sqrt{0{,}2775}\approx 0{,}5268 $.
		\end{center}
	}
\end{ex}

\Closesolutionfile{ans}

% \ind{PHẦN II.} \inden{Câu trắc nghiệm đúng sai. Trong mỗi ý a), b), c), d) ở mỗi câu, học sinh chọn đúng hoặc sai.}\\
\Opensolutionfile{ans}[ans/2D3-B2-d2-2]

\begin{ex}
	Một trang trại phân $1 \, 000$ quả trứng thành $5$ loại, tuỳ theo khối lượng (đã được làm tròn) của chúng	được thống kê bởi bảng dưới đây:
	\begin{center}
		\begin{tabular}{|l|c|c|c|c|c|}
			\hline
			Khối lượng (gam) & $[30; 36)$ & $ [36; 42)$ & $ [42; 48)$ & $ [48; 54)$ & $ [54; 60)$ \\
			\hline
			Số trứng         & $45$       & $190$       & $500$       & $250$       & $15$        \\
			\hline
		\end{tabular}
	\end{center}
	\choiceTF
	{\True Khoảng biến thiên của mẫu số liệu là $30$}
	{\True Khoảng tứ phân vị của mẫu số liệu là $6{,} 48$}
	{\True Khối lượng trung bình của 100 quả trứng là 45 gam}
	{\True Độ lệch chuẩn của mẫu số liệu là $\dfrac{6\sqrt{17}}{5}$}
	\loigiai{
		\begin{enumerate}[a)]
			\item Khoảng biến thiên là $60-30=30$.
			\item Nhóm chứa $Q_1$ là nhóm $[42; 48)$.\\
			      Suy ra $Q_1= 42 + \dfrac{250- 235}{500} \cdot 16=42{,} 48$.\\
			      $\dfrac{3N}{4}= 750$.\\
			      Nhóm chứa $Q_3$ là nhóm $[48; 54)$.\\
			      Khi đó $Q_3 =48 +\dfrac{750- 735 }{250} \cdot 16 = 48{,} 96$.\\
			      Suy ra khoảng tứ phân vị $\Delta_Q = Q_3 - Q_1= 6{,} 48$.
			\item Ta có bảng sau:
			      \begin{center}
				      \begin{tabular}{|l|c|c|c|c|c|}
					      \hline \hline
					      \textbf{Khối lượng (gam)} & $[30; 36)$ & $ [36; 42)$ & $ [42; 48)$ & $ [48; 54)$ & $ [54; 60)$ \\
					      \hline
					      \textbf{Giá trị đại diện} & $33$       & $39$        & $45$        & $51$        & $57$        \\ \hline
					      \textbf{Số trứng }        & $45$       & $190$       & $500$       & $250$       & $15$        \\
					      \hline \hline
				      \end{tabular}
			      \end{center}
			      Khối lượng trung bình $$\overline{x}= \dfrac{33 \cdot 45 + 39 \cdot 190 + 45 \cdot 500 + 51 \cdot 250 + 57 \cdot 15}{1\, 000}= 45\text{ gam}.$$
			\item Phương sai: $\dfrac{33^2 \cdot 45 + 39^2 \cdot 190 + 45^2 \cdot 500 + 51^2 \cdot 250 + 57^2 \cdot 15}{1\, 000} - 45^2=24{,}48$
			      Độ lệch chuẩn $$s= \sqrt{\dfrac{33^2 \cdot 45 + 39^2 \cdot 190 + 45^2 \cdot 500 + 51^2 \cdot 250 + 57^2 \cdot 15}{1\, 000} - 45^2} =\dfrac{6\sqrt{17}}{5} \text{ gam}.$$
		\end{enumerate}
	}
\end{ex}

\begin{ex}
	Kết quả $ 40 $ lần nhảy xa của hai vận động viên nam Dũng và Huy được lần lượt thống kê trong Bảng ở bên (đơn vị: mét).
	\begin{center}
		% \begin{tabular}{|c|c|c|}
		% 	\hline Nhóm          & Dũng   & Huy    \\
		% 	\hline$[6,22; 6,46)$ & $ 3 $  & $ 2 $  \\
		% 	{$[6,46; 6,70)$}     & $ 7 $  & $ 5 $  \\
		% 	{$[6,70; 6,94)$}     & $ 5 $  & $ 8 $  \\
		% 	{$[6,94; 7,18)$}     & $ 20 $ & $ 19 $ \\
		% 	{$[7,18; 7,42)$}     & $ 5 $  & $ 6 $  \\
		% 	\hline               & $n=40$ & $n=40$ \\
		% 	\hline
		% \end{tabular}
		\begin{tabular}{|c|c|c|c|c|c|c|}
			\hline
			Nhóm & $[6,22; 6,46)$ & $[6,46; 6,70)$ & $[6,70; 6,94)$ & $[6,94; 7,18)$ & $[7,18; 7,42)$ & $n$ \\
			\hline
			Dũng & 3              & 7              & 5              & 20             & 5              & 40  \\
			\hline
			Huy  & 2              & 5              & 8              & 19             & 6              & 40  \\
			\hline
		\end{tabular}
	\end{center}
	\choiceTF
	{\True Số trung bình cộng của mẫu số liệu ghép nhóm biểu diễn kết quả $ 40 $ lần nhảy xa của vận động viên Dũng (làm tròn kết quả đến hàng phần trăm) là $6,92\,(\mathrm{m})$}
	{Số trung bình cộng của mẫu số liệu ghép nhóm biểu diễn kết quả $ 40 $ lần nhảy xa của vận động viên Huy (làm tròn kết quả đến hàng phần trăm) là $6,85\,(\mathrm{m})$}
	{\True Độ lệch chuẩn của mẫu số liệu ghép nhóm biểu diễn kết quả $ 40 $ lần nhảy xa của vận động viên Huy (làm tròn kết quả đến hàng phần trăm) là $0,24\,(\mathrm{m})$}
	{\True Dựa vào độ lệch chuẩn thì kết quả nhảy xa của vận động viên Huy đồng đều hơn kết quả nhảy xa của vận động viên Dũng}

	\loigiai{
		Ta có bảng thống kê sau:
		\begin{center}
			\begin{tabular}{|c|c|c|c|}
				\hline Nhóm          & Giá trị đại diện & Dũng   & Huy    \\
				\hline$[6,22; 6,46)$ & $ 6,34 $         & $ 3 $  & $ 2 $  \\
				{$[6,46; 6,70)$}     & $ 6,58 $         & $ 7 $  & $ 5 $  \\
				{$[6,70; 6,94)$}     & $ 6,82 $         & $ 5 $  & $ 8 $  \\
				{$[6,94; 7,18)$}     & $ 7,06 $         & $ 20 $ & $ 19 $ \\
				{$[7,18; 7,42)$}     & $ 7,30 $         & $ 5 $  & $ 6 $  \\
				\hline               &                  & $n=40$ & $n=40$ \\
				\hline
			\end{tabular}
		\end{center}
		\begin{enumEX}{1}
			\item Số trung bình cộng của mẫu số liệu ghép nhóm biểu diễn kết quả $ 40 $ lần nhảy xa của vận động viên Dũng là:
			$$\bar{x}_D=\dfrac{3 \cdot 6,34+7 \cdot 6,58+5 \cdot 6,82+20 \cdot 7,06+5 \cdot 7,30}{40}=\dfrac{276,88}{40} \approx 6,92\,(\mathrm{m}).$$
			\item Số trung bình cộng của mẫu số liệu ghép nhóm biểu diễn kết quả $ 40 $ lần nhảy xa của vận động viên Huy là:
			$$\bar{x}_H=\dfrac{2 \cdot 6,34+5 \cdot 6,58+8 \cdot 6,82+19 \cdot 7,06+6 \cdot 7,30}{40}=\dfrac{278,08}{40} \approx 6,95\,(\mathrm{m}).$$
			\item Phương sai của mẫu số liệu ghép nhóm biểu diễn kết quả $ 40 $ lần nhảy xa của vận động viên Huy (làm tròn kết quả đến hàng phần trăm) là:
			$s_H^2 =\dfrac{1}{40}[2 \cdot(6,34-6,95)^2+5 \cdot(6,58-6,95)^2+8\cdot(6,82-6,95)^2+19 \cdot(7,06-6,95)^2+6 \cdot(7,30-6,95)^2]=\dfrac{2,5288}{40} \approx 0,06.$\\
			Độ lệch chuẩn của mẫu số liệu ghép nhóm trên là:
			$$s_H \approx \sqrt{0,06} \approx 0,24\,(\mathrm{m}).$$
			\item Phương sai của mẫu số liệu ghép nhóm biểu diễn kết quả $ 40 $ lần nhảy xa của vận động viên Dũng (làm tròn kết quả đến hàng phần trăm) là:
			$s_D^2=\dfrac{1}{40}[3 \cdot(6,34-6,92)^2+7 \cdot(6,58-6,92)^2+5 \cdot(6,82-6,92)^2+20 \cdot(7,06-6,92)^2+5 \cdot(7,30-6,92)^2]=\dfrac{2,9824}{40} \approx 0,07.$\\
			Độ lệch chuẩn của mẫu số liệu ghép nhóm trên là: $s_D \approx \sqrt{0,07} \approx 0,26\,(\mathrm{m})$.\\
			Do $s_H \approx 0,24<s_D \approx 0,26$ nên kết quả nhảy xa của vận động viên Huy đồng đều hơn kết quả nhảy xa của vận động viên Dũng.
		\end{enumEX}
	}
\end{ex}

\begin{ex}
	Một công ty giống cây trồng đã thử nghiệm hai phương pháp chăm sóc khác nhau cho cây hướng dương. Sau hai tuần, người ta thấy cây được chăm sóc theo cả hai phương pháp đều thấp hơn $50$ cm.\\
	\begin{tikzpicture}[scale=1]
		\def\y {{1.2, 1.6, 2.4,1.6, 1.2} }
		\foreach \i in {0,...,4} \draw[fill=blue!50] ( 1* \i,0) rectangle ++(1, \y[\i]);
		\foreach \i in {1,...,5}  \draw(\i,0) circle (1pt) node[ below]{$\i 0$};
		\draw(0,1) circle (1pt) node[left]{$5$};  \draw(0,2) circle (1pt) node[left]{$10$};

		\def\d{1.5}
		\def\l{1}
		\draw[gray!,step=0.2,line width=0.05pt](0,0)grid(6,3);
		\draw[red!50,thin,opacity=.5]
		(0,0) grid (6,3);
		\draw[->] (0,0)node[below right]{$O$}--(6,0)node[below]{cm};
		\draw[->] (0,0)--(0,3)node[above]{Tần số};
		\node[right] at (1,-1) {\text{Chiều cao của cây chăm sóc}};
		\node[right] at (1.5,-1.5) {\text{theo phương pháp A}};
	\end{tikzpicture}
	\begin{tikzpicture}[scale=1]
		\def\y {{2.6, 1.2, 0.4,1.2, 2.6} }
		\foreach \i in {0,...,4} \draw[fill=blue!50] ( 1* \i,0) rectangle ++(1, \y[\i]);
		\foreach \i in {1,...,5}  \draw(\i,0) circle (1pt) node[ below]{$\i 0$};
		\draw(0,1) circle (1pt) node[left]{$5$};  \draw(0,2) circle (1pt) node[left]{$10$};

		\def\d{1.5}
		\def\l{1}
		\draw[gray!,step=0.2,line width=0.05pt](0,0)grid(6,3);
		\draw[red!50,thin,opacity=.5]
		(0,0) grid (6,3);
		\draw[->] (0,0)node[below right]{$O$}--(6,0)node[below]{cm};
		\draw[->] (0,0)--(0,3)node[above]{Tần số};
		\node[right] at (1,-1) {\text{Chiều cao của cây chăm sóc}};
		\node[right] at (1.5,-1.5) {\text{theo phương pháp B}};
	\end{tikzpicture}
	\choiceTF
	{\True Khoảng biến thiên của chiều cao các cây được chăm sóc theo mỗi phương pháp A và B bằng nhau}
	{\True Trung bình của chiều cao các cây được chăm sóc theo mỗi phương pháp A và B bằng nhau}
	{\True Độ lệch chuẩn của chiều cao các cây được chăm sóc theo phương án $A$ là 12{,} 65 (cm)}
	{Dựa vào độ lệch chuẩn thì chiều cao của các loại cây được chăm sóc theo phương án $B$ ít bị chênh lệch hơn so với phương án $A$.}
	\loigiai{
		\begin{enumEX}{1}
			\item Khoảng biến thiên của chiều cao các cây được chăm sóc theo mỗi phương pháp A và B bằng nhau và cùng bằng 50.
			\item	Ước tính số trung bình và độ lệch chuẩn của chiều cao các cây được chăm sóc theo mỗi phương pháp.\\
			Cỡ mẫu của hai mẫu số liệu thống kê là $N= 40$.\\
			Ta có bảng tần số ghép nhóm về chiều cao của cây được chăm sóc theo phương pháp A như sau:
			\begin{center}
				\begin{tabular}{|l|c|c|c|c|c|}
					\hline \hline
					\textbf{Chiều cao (cm)}   & $[0; 10)$ & $ [10; 20)$ & $ [20; 30)$ & $ [30; 40)$ & $ [40; 50)$ \\
					\hline
					\textbf{Giá trị đại diện} & $5$       & $15$        & $25$        & $35$        & $45$        \\ \hline
					\textbf{Tần số }          & $ 6$      & $8$         & $12$        & $8$         & $6$         \\
					\hline \hline
				\end{tabular}
			\end{center}
			Chiều cao trung bình của các cây được chăm sóc theo phương án $A$ là $$\overline{x}_A= \dfrac{5 \cdot 6 + 15 \cdot 8 + 25 \cdot 12 + 35 \cdot 8 + 45 \cdot 6}{40}=25.$$
			Ta có bảng tần số ghép nhóm về chiều cao của cây được chăm sóc theo phương pháp B như sau:
			\begin{center}
				\begin{tabular}{|l|c|c|c|c|c|}
					\hline \hline
					\textbf{Chiều cao (cm)}   & $[0; 10)$ & $ [10; 20)$ & $ [20; 30)$ & $ [30; 40)$ & $ [40; 50)$ \\
					\hline
					\textbf{Giá trị đại diện} & $5$       & $15$        & $25$        & $35$        & $45$        \\ \hline
					\textbf{Tần số }          & $ 13 $    & $6$         & $2$         & $6$         & $13$        \\
					\hline \hline
				\end{tabular}
			\end{center}
			Chiều cao trung bình của các cây được chăm sóc theo phương án $B$ là $$\overline{x}_B= \dfrac{5 \cdot 13 + 15 \cdot 6 + 25 \cdot 2 + 35 \cdot 6 + 45 \cdot 13}{40}=25 \text{ cm}.$$
			\item 	Độ lệch chuẩn của chiều cao các cây được chăm sóc theo phương án $A$ là $$s_A =\sqrt{\dfrac{5^2 \cdot 6 + 15^2 \cdot 8 + 25^2 \cdot 12 + 35^2 \cdot 8 + 45 ^2 \cdot 6}{40} - 25^2}\approx 12{,} 65. $$
			\item Độ lệch chuẩn của chiều cao các cây được chăm sóc theo phương án $B$ là $$s_B =\sqrt{\dfrac{5^2 \cdot 13 + 15^2 \cdot 6 + 25^2 \cdot 2 + 35^2 \cdot 6+ 45 ^2 \cdot 13}{40} - 25^2}\approx 17{,} 03 \text{ cm}. $$
			Do $s_A< s_B$ nên chiều cao của các loại cây được chăm sóc theo phương án $A$ ít bị chênh lệch hơn so với phương án $B$.
		\end{enumEX}}
\end{ex}

\Closesolutionfile{ans}

%Chương IV. Nguyên hàm. Tích phân.
% %%Bài 1. Nguyên hàm
% \chap{NGUYÊN HÀM VÀ TÍCH PHÂN}
\section{NGUYÊN HÀM}
\subsection{Tóm tắt lý thuyết}
\subsection{Kiến thức cần nắm}
% \subsubsection{ĐỊNH NGHĨA VÀ TÍNH CHẤT}
\subsubsection{Định nghĩa nguyên hàm}
Cho hàm số $f(x)$ xác định trên khoảng $K$. Hàm số $F(x)$ được gọi là nguyên hàm của hàm số $f(x)$ nếu $F'(x)=f(x)$ với mọi $x\in K$.\\
\textbf{Nhận xét:} Nếu $F(x)$ là một nguyên hàm của $f(x)$ thì $F(x)+C$, $(C\in\mathbb{R})$ cũng là nguyên hàm của $f(x)$.\\
Ký hiệu $\displaystyle\int f(x)\mathrm{\,d}x=F(x)+C$.\\
\subsubsection{Một số tính chất của nguyên hàm}
\begin{itemize}
	\item $\left(\displaystyle\int f(x)\mathrm{\,d}x\right)'=f(x)$.
	\item $\displaystyle\int a\cdot f(x)\mathrm{\,d}x=a\cdot\displaystyle\int f(x)\mathrm{\,d}x\quad\left(a\in\mathbb{R}, a\neq 0\right)$.
	\item $\displaystyle\int\left[f(x)\pm g(x)\right]\mathrm{\,d}x=\displaystyle\int f(x)\mathrm{\,d}x\pm\displaystyle\int g(x)\mathrm{\,d}x$.
\end{itemize}
\subsubsection{Một số nguyên hàm cơ bản}
\begin{longtable}{|c|c|}
	\hline
	 Nguyên hàm của hàm số cơ bản & Nguyên hàm mở rộng \\
	\hline
	$\displaystyle\int a\cdot\mathrm{\,d}x=ax+C, a\in\mathbb{R}$ & \\
	\hline
	$\displaystyle\int x^{\alpha}\mathrm{\,d}x=\dfrac{x^{\alpha+1}}{\alpha+1}+C,\alpha\neq-1$ & $\displaystyle\int(ax+b)^{\alpha}\mathrm{\,d}x=\dfrac{1}{a}\cdot\dfrac{(ax+b)^{\alpha+1}}{\alpha+1}+C$ \\
	\hline
	$\displaystyle\int\dfrac{\mathrm{\,d}x}{x}=\ln |x|+C, x\neq 0$ & $\displaystyle\int\dfrac{\mathrm{\,d}x}{ax+b}=\dfrac{1}{a}\cdot\ln |ax+b|+C$ \\
	\hline
	$\displaystyle\int\dfrac{\mathrm{\,d}x}{\sqrt{x}}=2\sqrt{x}+C, x>0$ & $\displaystyle\int\dfrac{\mathrm{\,d}x}{\sqrt{ax+b}}=\dfrac2a\sqrt{ax
	+b}+C, x>0$ \\
	\hline
	$\displaystyle\int\dfrac{\mathrm{\,d}x}{x^2}=-\dfrac{1}{x}+C, x\neq 0$ & $\displaystyle\int\dfrac{\mathrm{\,d}x}{(ax+b)^2}=-\dfrac{1}{a}\cdot \dfrac{1}{ax+b}+C$ \\
	\hline
	$\displaystyle\int\dfrac{\mathrm{\,d}x}{x^{\alpha}}=-\dfrac{1}{(\alpha-1)x^{\alpha-1}}+C$ & $\displaystyle\int\dfrac{\mathrm{\,d}x}{(ax+b)^{\alpha}}=-\dfrac{1}{a}\cdot \dfrac{1}{(\alpha-1)}\cdot (ax+b)^{\alpha-1}+C$ \\
	\hline
	$\displaystyle\int\mathrm{e}^x\mathrm{\,d}x=\mathrm{e}^x+C$ & $\displaystyle\int\mathrm{e}^{ax+b}\mathrm{\,d}x=\dfrac{1}{a}\cdot\mathrm{e}^{ax+b}+C$ \\
	\hline
	$\displaystyle\int a^x\mathrm{\,d}x=\dfrac{a^x}{\ln a}+C$ & $\displaystyle\int a^{\alpha x+\beta}\mathrm{\,d}x=\dfrac{1}{\alpha}\cdot\dfrac{a^{\alpha x+\beta}}{\ln a}+C$ \\
	\hline
	$\displaystyle\int\cos x\mathrm{\,d}x=\sin x+C$ & $\displaystyle\int\cos (ax+b)\mathrm{\,d}x=\dfrac{1}{a}\cdot\sin (ax+b)+C$ \\
	\hline
	$\displaystyle\int\sin x\mathrm{\,d}x=-\cos x+C$ & $\displaystyle\int\sin (ax+b)\mathrm{\,d}x=-\dfrac{1}{a}\cdot\cos (ax+b)+C$ \\
	\hline
	$\displaystyle\int\dfrac{1}{\cos^2x}\mathrm{\,d}x=\tan x+C$ & $\displaystyle\int\dfrac{1}{\cos^2(ax+b)}\mathrm{\,d}x=\dfrac{1}{a}\cdot \tan (ax+b)+C$ \\
	\hline
	$\displaystyle\int\dfrac{1}{\sin^2x}\mathrm{\,d}x=-\cot x+C$ & $\displaystyle\int\dfrac{1}{\sin^2(ax+b)}\mathrm{\,d}x=-\dfrac{1}{a}\cdot \cot(ax+b)+C$ \\
	\hline
\end{longtable}
\textit{\textbf{Nhận xét:} $[F(ax+b)]'=af(ax+b) \Rightarrow \int f(ax+b) \mathrm{\,d}x = \dfrac{1}{a} F(ax+b)+C$}.
\subsection{Phân loại và phương pháp giải bài tập}
\begin{dang}{Sử dụng định nghĩa nguyên hàm và bảng nguyên hàm}
\end{dang}
\subsubsection{Các ví dụ}
\begin{vd}%Câu 1  %[2D3Y1-1]
	Tìm họ nguyên hàm của các hàm số sau
    \begin{listEX}[2]
        \item $f(x)=4x^3+x+5$.
        \item $f(x)=3x^2-2x$.
        \item $f(x)=\dfrac{1}{x^5}+x^2$.
        \item $f(x)=\dfrac{1}{x^3}+x^2-1$.
    \end{listEX}
	\loigiai{
        \begin{listEX}[1]
            \item Ta có $F(x)=\displaystyle\int f(x)\mathrm{\,d}x =\displaystyle\int{(4x^3+x+5)\textrm{ d}x=x^4+\dfrac{x^2}{2}+5x+C}$.
            \item Ta có $F(x)=\displaystyle\int f(x)\mathrm{\,d}x =\displaystyle\int{(3x^2-2x)\textrm{ d}x=x^3-x^2+C}$.
            \item Ta có $F(x)=\displaystyle\int f(x)\mathrm{\,d}x=\displaystyle\int ({x^{-5}}+x^2)\mathrm{\,d}x =-\dfrac{{x^{-4}}}{4}+\dfrac{x^3}{3}+C$.
            \item Ta có $F(x)=\displaystyle\int{f(x)\mathrm{\,d}x}=\displaystyle\int{\left( {x^{-3}}+x^2-1 \right)\mathrm{\,d}x}=-\dfrac{{x^{-2}}}{2}+\dfrac{x^3}{3}-x$.
        \end{listEX}
		}
    \end{vd}
\begin{vd}%Câu 5 %[2D3Y1-1]
	Tính
    \begin{listEX}[3]
        \item $I=\displaystyle\int{(x^2-3x)(x+1)\mathrm{\,d}x}$.
        \item $I=\displaystyle\int{(x-1)(x^2+2)\mathrm{\,d}x}$.
        \item $I=\displaystyle\int{{{(2x+1)}^5}\mathrm{\,d}x}$
        \item $I=\displaystyle\int{{{(2x-10)}^{2020}}\mathrm{\,d}x}$.
        \item $I=\displaystyle\int{\left( 3x^2+\dfrac{1}{x}-2 \right)\mathrm{\,d}x}$.
        \item $I=\displaystyle\int{\left( 3x^2-\dfrac{2}{x}-\dfrac{1}{x^2} \right)\mathrm{\,d}x}$.
        \item $I=\displaystyle\int{\dfrac{x^2-3x+1}{x}\mathrm{\,d}x}$.
        \item $I=\displaystyle\int{\dfrac{2x^2-6x+3}{x}\mathrm{\,d}x}$.
        \item $I=\displaystyle\int{\dfrac{1}{2x-1}\mathrm{\,d}x}$.
        \item $I=\displaystyle\int{\dfrac{2}{3-4x}\mathrm{\,d}x}$.
        \item $I=\displaystyle\int{\dfrac{1}{{{\left( 2x-1 \right)}^2}}\mathrm{\,d}x}$.
        \item $I=\displaystyle\int{\left[ \dfrac{12}{{{\left( x-1 \right)}^2}}+\dfrac{2}{2x-3} \right]\mathrm{\,d}x}$.
        \item $I=\displaystyle\int{\dfrac{3}{4x^2+4x+1}\textrm{ d}x}$.
        \item $I=\displaystyle\int{\dfrac{4}{x^2+6x+9}\textrm{ d}x}$.
            \item (*) $I=\displaystyle\int{\dfrac{2x-1}{{{\left( x+1 \right)}^2}}\textrm{ d}x}$.
    \end{listEX}
	\loigiai{
        \begin{listEX}[1]
            \item Phân phối được: $I=\displaystyle\int{(x^3-2x^2-3x)\mathrm{\,d}x} =\dfrac{x^4}{4}-\dfrac{2}{3}x^3-\dfrac{3}{2}x^2+C$.
            \item Phân phối được: $I=\displaystyle\int{(x^3-x^2+2x-2)\mathrm{\,d}x} =\dfrac{x^4}{4}-\dfrac{x^3}{3}+x^2-2x+C$.
            \item $I=\displaystyle\int{{{(2x+1)}^5}\mathrm{\,d}x}=\dfrac{1}{2}\dfrac{{{(2x+1)}^6}}{6}+C$.
            \item $I=\displaystyle\int{{{(2x-10)}^{2020}}\mathrm{\,d}x}=\dfrac{1}{2}\dfrac{{{(2x-10)}^{2021}}}{2021}+C$.
            \item Ta có $I=\displaystyle\int{\left( 3x^2+\dfrac{1}{x}-2 \right)\mathrm{\,d}x}=x^3+\ln \left| x \right|-2x+C$.
            \item Ta có $I=\displaystyle\int{\left( 3x^2-\dfrac{2}{x}-\dfrac{1}{x^2} \right)\mathrm{\,d}x}=x^3-2\ln \left| x \right|+\dfrac{1}{x}+C$.
            \item Ta có $I=\displaystyle\int{\dfrac{x^2-3x+1}{x}\mathrm{\,d}x}=\displaystyle\int{\left( x-3+\dfrac{1}{x} \right)\mathrm{\,d}x}=x^2-3x+\ln \left| x \right|+C$.
            \item Ta có $I=\displaystyle\int{\dfrac{2x^2-6x+3}{x}\mathrm{\,d}x}=\displaystyle\int{\left( 2x-6+\dfrac{3}{x} \right)\mathrm{\,d}x}=x^2-6x+3\ln \left| x \right|+C$.
            \item Ta có $I=\displaystyle\int{\dfrac{1}{2x-1}\mathrm{\,d}x}=\dfrac{1}{2}\ln \left| 2x-1 \right|+C$.
            \item Ta có $I=\displaystyle\int{\dfrac{2}{3-4x}\mathrm{\,d}x}=2.\dfrac{1}{-4}.\ln \left| 3-4x \right|+C=-\dfrac{1}{2}\ln \left| 3-4x \right|+C$.
            \item $I=\displaystyle\int{\dfrac{1}{{{\left( 2x-1 \right)}^2}}\mathrm{\,d}x=-\dfrac{1}{2}}.\dfrac{1}{2x-1}+C=\dfrac{-1}{4x-2}+C$.
            \item $I=\displaystyle\int{\left[ \dfrac{12}{{{\left( x-1 \right)}^2}}+\dfrac{2}{2x-3} \right]\mathrm{\,d}x=-\dfrac{12}{1}}.\dfrac{1}{x-1}+\dfrac{2}{2}\ln \left| 2x-3 \right|+C=\dfrac{-12}{x-1}+\ln \left| 2x-3 \right|+C$.
            \item $I=\displaystyle\int{\dfrac{1}{4x^2+4x+1}\textrm{ d}x=}\displaystyle\int{\dfrac{1}{{{\left( 2x+1 \right)}^2}}\textrm{ d}x=-\dfrac{1}{2}}.\dfrac{1}{2x+1}+C=\dfrac{-1}{4x+2}+C$.
            \item $I=\displaystyle\int{\dfrac{4}{x^2+6x+9}\textrm{ d}x=}\displaystyle\int{\dfrac{4}{{{\left( x+3 \right)}^2}}\textrm{ d}x=-\dfrac{4}{1}}.\dfrac{1}{x+3}+C=\dfrac{-4}{x+3}+C$.
            \item $I=\displaystyle\int{\dfrac{2x+2-3}{{{\left( x+1 \right)}^2}}\textrm{ d}x=\displaystyle\int{\left[ \dfrac{2(x+1)}{{{\left( x+1 \right)}^2}}-\dfrac{3}{{{\left( x+1 \right)}^2}} \right]}}\textrm{ d}x=\displaystyle\int{\dfrac{2}{x+1}\textrm{ d}x-\displaystyle\int{\dfrac{3}{{{\left( x+1 \right)}^2}}\textrm{ d}x}}$.\\
            $I=2\ln \left| x+1 \right|-\dfrac{-3}{x+1}+C=2\ln \left| x+1 \right|+\dfrac{3}{x+1}+C$.
            \item $I=\displaystyle\int{\dfrac{2x-2}{{{\left( 2x+1 \right)}^2}}\textrm{ d}x=\displaystyle\int{\left[ \dfrac{2x+1}{{{\left( 2x+1 \right)}^2}}-\dfrac{3}{{{\left( 2x+1 \right)}^2}} \right]}}\textrm{ d}x = \displaystyle\int{\dfrac{1}{2x+1}\textrm{ d}x-\displaystyle\int{\dfrac{3}{{{\left( 2x+1 \right)}^2}}\textrm{ d}x}}$.\\
            $I=\dfrac{1}{2}\ln \left| 2x+1 \right|-\dfrac{-3}{2\left( 2x+1 \right)}+C \Rightarrow I=\dfrac{1}{2}\ln \left| 2x+1 \right|+\dfrac{3}{2\left( 2x+1 \right)}+C$.
        \end{listEX}
		}
\end{vd}
\begin{vd}%[2D3B1-1]%BT3.
    Tìm họ nguyên hàm của các hàm số sau
    \begin{listEX}[3]
        \item $I=\displaystyle\int(\sin x-\cos x) \mathrm{\,d}x$.
        \item $I=\displaystyle\int (3 \cos x-2 \sin x) \mathrm{\,d}x$.
        \item $I=\displaystyle\int (2 \sin 2x-3 \cos 6x) \mathrm{\,d}x$.
        \item $I=\displaystyle\int \sin x \cos x \mathrm{\,d}x$.
        \item $I=\displaystyle\int \cos \left(\dfrac{x}{2}+\dfrac{\pi}{6}\right)\mathrm{\,d}x$.
        \item $I=\displaystyle\int \sin \left(\dfrac{\pi}{3}-\dfrac{x}{3}\right)\mathrm{\,d}x$.
        \item $I=\displaystyle\int (\sin x-\cos x)^2 \mathrm{\,d}x$.
        \item $I=\displaystyle\int (\cos x+\sin x)^2 \mathrm{\,d}x$.
        %    \item $I=\displaystyle\int \left(\cos ^2x-\sin ^2x\right) \mathrm{\,d}x$.
        %    \item $I=\displaystyle\int \left(\cos ^{4}x-\sin ^{4}x\right) \mathrm{\,d}x$.
    \end{listEX}
    \loigiai{
        \begin{listEX}[1]
            \item $I=\displaystyle\int(\sin x-\cos x) \mathrm{\,d}x=-\cos x-\sin x +C$.
            \item $I=\displaystyle\int (3 \cos x-2 \sin x) \mathrm{\,d}x=3\sin x + 2\cos x+C $.
            \item $I=\displaystyle\int (2 \sin 2x-3 \cos 6x) \mathrm{\,d}x=-\cos 2x -\dfrac{1}{2} \sin 6x+C$.
            \item $I=\dfrac{1}{2}\displaystyle\int \sin 2x \mathrm{\,d}x=-\dfrac{1}{4}\cos 2x+C$.
            \item $I=\displaystyle\int \cos \left(\dfrac{x}{2}+\dfrac{\pi}{6}\right)\mathrm{\,d}x=\displaystyle\int \left(\dfrac{\sqrt{3}}{2}\cos\dfrac{x}{2} -\dfrac{1}{2}\sin \dfrac{x}{2}\right) \mathrm{\,d}x = \sqrt{3}\sin \dfrac{x}{2}+\cos \dfrac{x}{2}+C$.
            \item $I=\displaystyle\int \sin \left(\dfrac{\pi}{3}-\dfrac{x}{3}\right)\mathrm{\,d}x= \displaystyle\int  \left(\dfrac{\sqrt{3}}{2}\cos \dfrac{x}{3}-\dfrac{1}{2}\sin \dfrac{x}{3} \right)\mathrm{\,d}x =\dfrac{3\sqrt{3}}{2}\sin \dfrac{x}{3}+\dfrac{3}{2}\cos\dfrac{x}{3}+C$.
            \item $I=\displaystyle\int (\sin x-\cos x)^2 \mathrm{\,d}x=\displaystyle\int (1-\sin 2x)\mathrm{\,d}x=x+\dfrac{1}{2}\cos 2x+C$.
            \item $I=\displaystyle\int (\cos x+\sin x)^2 \mathrm{\,d}x=\displaystyle\int(1+\sin 2x)\mathrm{\,d}x=x-\dfrac{1}{2}\cos 2x+C$.
            \item $I=\displaystyle\int \left(\cos ^2x-\sin ^2x\right) \mathrm{\,d}x= \displaystyle\int \cos 2x \mathrm{\,d}x=\dfrac{1}{2}\sin 2x+C$.
            \item $I=\displaystyle\int \left(\cos ^{4}x-\sin ^{4}x\right) \mathrm{\,d}x=\displaystyle\int \left(\cos ^2x-\sin ^2x\right) \mathrm{\,d}x= \displaystyle\int \cos 2x \mathrm{\,d}x=\dfrac{1}{2}\sin 2x+C$.
        \end{listEX}
    }
\end{vd}

\begin{vd} %[2D3B1-1]
    Tìm họ nguyên hàm của các hàm số sau
    \begin{listEX}[3]
        \item $I=\displaystyle\int \dfrac{1}{\sin ^2x} \mathrm{\,d}x$.
        \item $I=\displaystyle\int \dfrac{6}{\cos ^2 3x} \mathrm{\,d}x$.
        \item $I=\displaystyle\int (\tan x+\cot x)^2 \mathrm{\,d}x$.
        \item $I=\displaystyle\int \sin ^2x \mathrm{\,d}x$.
        \item $I=\displaystyle\int \cos ^2 2x \mathrm{\,d}x$.
        \item $I=\displaystyle\int \sin 4x \cos x \mathrm{\,d}x$.
        \item $I=\displaystyle\int \dfrac{1}{\sin x \cos x} \mathrm{\,d}x$.
    \end{listEX}

    \loigiai{
        \begin{listEX}[1]
            \item $I=\displaystyle\int\left( \dfrac{1}{\cos ^2x}-\dfrac{1}{\sin ^2x}\right) \mathrm{\,d}x=\tan x+\cot x +C$.
            \item $I=\displaystyle\int \dfrac{6}{\cos ^2 3x} \mathrm{\,d}x=2\tan 3x+C$.
            \item $I=\displaystyle\int (\tan x+\cot x)^2 \mathrm{\,d}x=\displaystyle\int (\tan^2 x+\cot^2x+2) \mathrm{\,d}x            =\displaystyle\int (\tan^2 x+1+\cot^2x+1) \mathrm{\,d}x=\tan x-\cot x+C$.
			\item $I=\displaystyle\int \sin ^2x \mathrm{\,d}x = \displaystyle\int \dfrac{1-\cos 2x}{2} \mathrm{\,d}x = \dfrac{1}{2}x-\dfrac{1}{4}\sin 2x+C$.
			\item $I=\displaystyle\int \cos ^2 2x \mathrm{\,d}x = \displaystyle\int \dfrac{1+\cos 4x}{2} \mathrm{\,d}x = \dfrac{1}{2}x+\dfrac{1}{8}\sin 4x+C$.
        \end{listEX}
    }
\end{vd}
\begin{vd} %[2D3B1-1]
    Tìm họ nguyên hàm của các hàm số sau
    \begin{listEX}[3]
        \item $I=\displaystyle\int \mathrm{e} ^{2x} \mathrm{\,d}x$.
        \item $I=\displaystyle\int \mathrm{e}^{1-2x} \mathrm{\,d}x$.
        \item $I=\displaystyle\int \left(2x-\mathrm{e}^{-x}\right) \mathrm{\,d}x$.
        \item $I=\displaystyle\int \mathrm{e}^x\left(1-3 \mathrm{e}^{-2x}\right) \mathrm{\,d}x$.
        \item $I=\displaystyle\int \left(3-\mathrm{e}^x\right)^2 \mathrm{\,d}x$.
        \item $I=\displaystyle\int \left(2+\mathrm{e}^{3x}\right)^2 \mathrm{\,d}x$.
        \item $I=\displaystyle\int 2^{2x+1} \mathrm{\,d}x$.
        \item $I=\displaystyle\int 4^{1-2x} \mathrm{\,d}x$.
        \item $I=\displaystyle\int 3^x \cdot 5^x \mathrm{\,d}x$.
        \item $I=\displaystyle\int 4^x \cdot 3^{x-1} \mathrm{\,d}x$.
        \item $I=\displaystyle\int \dfrac{\mathrm{\,d}x}{\mathrm{e}^{2-5x}}$.
        \item $I=\displaystyle\int \dfrac{\mathrm{\,d}x}{2^{3-2x}}$.
        \item $I=\displaystyle\int \dfrac{4^{x+1} \cdot 3^{x-1}}{2^x} \mathrm{\,d}x$.
        \item $I=\displaystyle\int \dfrac{4^{2x-1} \cdot 6^{x-1}}{3^x} \mathrm{\,d}x$.
    \end{listEX}
    \loigiai{
        \begin{listEX}[1]
            \item Ta có $I=\displaystyle\int \mathrm{e} ^{2x} \mathrm{\,d}x=\dfrac{1}{2} \mathrm{e}^{2x}+C$.
            \item Ta có $I=\displaystyle\int \mathrm{e}^{1-2x} \mathrm{\,d}x=-\dfrac{1}{2}\mathrm{e}^{1-2x}+C$.
            \item $I=\displaystyle\int \left(2x-\mathrm{e}^{-x}\right) \mathrm{\,d}x=x^2+\mathrm{e}^{-x}+C$.
            \item Ta có $I=\displaystyle\int \mathrm{e}^x\left(1-3 \mathrm{e}^{-2x}\right) \mathrm{\,d}x=\displaystyle\int \left(e^x-3e^{-x}\right) \mathrm{\,d}x=e^x+3e^{-x}+C$.
            \item $I=\displaystyle\int \left(3-\mathrm{e}^x\right)^2 \mathrm{\,d}x=\displaystyle\int\left( 9-6\mathrm{e}^x+\mathrm{e}^{2x}\right) \mathrm{\,d}x=9x-6\mathrm{e}^x+\dfrac{1}{2}\mathrm{e}^{2x}+C$.
            \item Ta có $I=\displaystyle\int \left(2+\mathrm{e}^{3x}\right)^2 \mathrm{\,d}x= \displaystyle\int \left(4+4\mathrm{e}^{3x}+ \mathrm{e}^{6x}\right) \mathrm{\,d}x=4x+\dfrac{4}{3}\mathrm{e}^{3x}+\dfrac{1}{6}\mathrm{e}^{6x}+C$.
            \item Ta có $I=\displaystyle\int 2^{2x+1} \mathrm{\,d}x=\dfrac{2^{2x+1}}{2\ln 2}+C$.
            \item Ta có $I=\displaystyle\int 4^{1-2x} \mathrm{\,d}x=-\dfrac{4^{1-2x}}{2\ln 4}+C$.
            \item Ta có $I=\displaystyle\int 15^x \mathrm{\,d}x = \dfrac{15^x}{\ln 15}+C$.
            \item Ta có $I=\dfrac{1}{3}\displaystyle\int 12^x \mathrm{\,d}x=\dfrac{12^x}{3\ln 12}+C$.
            \item $I=\displaystyle\int \mathrm{e}^{5x-2}\mathrm{\,d}x=\dfrac{\mathrm{e}^{5x-2}}{5}+C$.
            \item Ta có $I=\displaystyle\int 2^{2x-3} \mathrm{\,d}x =\dfrac{ 2^{2x-3}}{2\ln 2}+C$.
            \item Ta có $I=\displaystyle\int \dfrac{4^{x+1} \cdot 3^{x-1}}{2^x} \mathrm{\,d}x=\dfrac{4}{3}\displaystyle\int 6^x\mathrm{\,d}x= \dfrac{4\cdot 6^x}{3\cdot \ln 6}+C$.
            \item Ta có $I=\displaystyle\int \dfrac{4^{2x-1} \cdot 6^{x-1}}{3^x} \mathrm{\,d}x=\dfrac{1}{24}\displaystyle\int 32^x \mathrm{\,d}x=\dfrac{32^x}{24\ln 32}+C=\dfrac{2^{5x}}{120\ln 2}+C$.
        \end{listEX}
    }
\end{vd}
\subsubsection{Câu hỏi trắc nghiệm}
% \TN
\Opensolutionfile{ans}[ans/ans-2-B1-D2-LC]
\begin{ex}%[2D4N1-1]
	Cho hàm số $F(x)$ là một nguyên hàm của hàm số $f(x)$ trên $K$. Các mệnh đề sau, mệnh đề nào \textbf{sai}.
	\choice
	{$\displaystyle\int{f(x)\mathrm{\,d}x=}F(x)+C$}
	{$\displaystyle{\left(\displaystyle\int{f(x)\mathrm{\,d}x}\right)'}=f(x)$}
	{\True $\displaystyle{\left(\displaystyle\int{f(x)\mathrm{\,d}x}\right)'}=f'(x)$}
	{$\displaystyle{\left(\displaystyle\int{f(x)\mathrm{\,d}x}\right)'}=F'(x)$}
	\loigiai{
		Ta có $\displaystyle\int{f(x)\mathrm{\,d}x=}F(x)+C\Leftrightarrow F'(x)=f(x)$ nên phương án $\left(\displaystyle\int{f(x)\mathrm{\,d}x}\right)'=f'(x)$ sai.}
\end{ex}

\begin{ex}%[2D4N1-2]
	Họ tất cả các nguyên hàm của hàm số $f(x)=2x+6$ là
	\choice
	{$x^2+C$}
	{\True $x^2+6x+C$}
	{$2x^2+C$}
	{$2x^2+6x+C$}
	\loigiai{
		$\displaystyle\int{(2x+6)\mathrm{\,d}x=x^2+6x+C}$.}
\end{ex}

\begin{ex}%[2D4N1-2]
	$\displaystyle\int{x^2\mathrm{\,d}x}$ bằng
	\choice
	{$2x+C$}
	{\True $\dfrac{1}{3}x^3+C$}
	{$x^3+C$}
	{$3x^3+C$}
	\loigiai{
		Ta có $\displaystyle\int{x^2\mathrm{\,d}x}=\dfrac{1}{3}x^3+C$.}
\end{ex}

\begin{ex}%[2D4N1-2]
	Họ nguyên hàm của hàm số $f(x)=3x^2+1$ là
	\choice
	{$x^3+C$}
	{$\dfrac{x^3}{3}+x+C$}
	{$6x+C$}
	{\True $x^3+x+C$}
	\loigiai{
		$\displaystyle\int{(3x^2+1)\mathrm{\,d}x=x^3+x+C}$.}
\end{ex}

\begin{ex}%[2D4N1-2]
	Nguyên hàm của hàm số $f(x)=x^3+x$ là
	\choice
	{\True $\dfrac{1}{4}x^4+\dfrac{1}{2}x^2+C$}
	{$3x^2+1+C$}
	{$x^3+x+C$}
	{$x^4+x^2+C$}
	\loigiai{
		$\displaystyle\int{(x^3+x^2)\mathrm{\,d}x}=\dfrac{1}{4}x^4+\dfrac{1}{2}x^2+C$.}
\end{ex}

\begin{ex}%[2D4N1-2]
	Nguyên hàm của hàm số $f(x)=x^4+x^2$ là
	\choice
	{\True $\dfrac{1}{5}x^5+\dfrac{1}{3}x^3+C$}
	{$x^4+x^2+C$}
	{$x^5+x^3+C$}
	{$4x^3+2x+C$}
	\loigiai{
		$\displaystyle\int{f(x)\mathrm{\,d}x}=\displaystyle\int{(x^4+x^2)\mathrm{\,d}x}$ $=\dfrac{1}{5}x^5+\dfrac{1}{3}x^3+C$.}
\end{ex}

\begin{ex}%[2D4H1-2]
	Hàm số nào trong các hàm số sau đây không là nguyên hàm của hàm số $y=x^{2022}$?
	\choice
	{$\dfrac{x^{2023}}{2023}+1$}
	{$\dfrac{x^{2023}}{2023}$}
	{\True $y=2022x^{2021}$}
	{$\dfrac{x^{2023}}{2023}-1$}
	\loigiai{
		Ta có $\displaystyle\int{x^{2022}\mathrm{\,d}}x=\dfrac{x^{2023}}{2023}+C$, $C$ là hằng số nên $y=2022x^{2021}$ không là nguyên hàm của hàm số $y=x^{2022}$.}
\end{ex}

\begin{ex}%[2D4H1-2]
	Nguyên hàm của hàm số $f(x)=$ $\dfrac{1}{3}x^3-2x^2+x-2024$ là
	\choice
	{$\dfrac{1}{12}x^4-\dfrac{2}{3}x^3+\dfrac{x^2}{2}+C$}
	{$\dfrac{1}{9}x^4-\dfrac{2}{3}x^3+\dfrac{x^2}{2}-2024x+C$}
	{\True $\dfrac{1}{12}x^4-\dfrac{2}{3}x^3+\dfrac{x^2}{2}-2024x+C$}
	{$\dfrac{1}{9}x^4+\dfrac{2}{3}x^3-\dfrac{x^2}{2}-2024x+C$}
	\loigiai{
		Sử dụng công thức $\displaystyle\int{x^n\mathrm{\,d}x=\dfrac{x^{n+1}}{n+1}+C}$ ta được
		\begin{eqnarray*}
		\displaystyle\int\left(\dfrac{1}{3}x^3-2x^2+x-2024\right)\mathrm{\,d}x&=&\dfrac{1}{3}\cdot \dfrac{x^4}{4}-2\cdot \dfrac{x^3}{3}+\dfrac{x^2}{2}-2024x+C\\&=&\dfrac{1}{12}x^4-\dfrac{2}{3}x^3+\dfrac{1}{2}x^2-2024x+C.
		\end{eqnarray*}
		}
\end{ex}

\begin{ex}%[2D4H1-2]
	Tìm nguyên $F(x)$ của hàm số $f(x)=(x+1)(x+2)(x+3)?$
	\choice
	{$F(x)=\dfrac{x^4}{4}-6x^3+\dfrac{11}{2}x^2-6x+C$}
	{$F(x)=x^4+6x^3+11x^2+6x+C$}
	{\True $F(x)=\dfrac{x^4}{4}+2x^3+\dfrac{11}{2}x^2+6x+C$}
	{$F(x)=x^3+6x^2+11x^2+6x+C$}
	\loigiai{
		Ta có $f(x)=(x+1)(x+2)(x+3)=x^3+6x^2+11x+6$ nên\\
		$\displaystyle F(x)=\displaystyle\int{(x^3+6x^2+11x+6)}\mathrm{\,d}x=\dfrac{x^4}{4}+2x^3+\dfrac{11}{2}x^2+6x+C$.}
\end{ex}

\begin{ex}%[2D4H1-2]
	Tìm nguyên hàm của hàm số $f(x)=(5x+3)^5$.
	\choice
	{$(5x+3)^6+C$}
	{$(5x+3)^4+C$}
	{\True $\dfrac{(5x+3)^6}{30}+C$}
	{$\dfrac{(5x+3)^4}{30}+C$}
	\loigiai{
		$f(x)=(5x+3)^5$ $\displaystyle \Rightarrow \displaystyle\int{f(x)\mathrm{\,d}x=}\displaystyle\int{(5x+ 3)^5\mathrm{\,d}x=}\dfrac{1}{5}\cdot \dfrac{(5x+3)^6}{6}+C=\dfrac{(5x+3)^6}{30}+C$.}
\end{ex}

\begin{ex}%[2D4H1-2]
	Tìm nguyên hàm của hàm số $f(x)=x^2+\dfrac{2}{x^2}$.
	\choice
	{\True $\displaystyle\int{f(x)\mathrm{\,d}x}=\dfrac{x^3}{3}+\dfrac{1}{x}+C$}
	{$\displaystyle\int{f(x)\mathrm{\,d}x}=\dfrac{x^3}{3}-\dfrac{2}{x}+C$}
	{$\displaystyle\int{f(x)\mathrm{\,d}x}=\dfrac{x^3}{3}-\dfrac{1}{x}+C$}
	{$\displaystyle\int{f(x)\mathrm{\,d}x}=\dfrac{x^3}{3}+\dfrac{2}{x}+C$}
	\loigiai{
		Ta có $\displaystyle\int{\left(x^2+\dfrac{2}{x^2}\right)\mathrm{\,d}x}=\dfrac{x^3}{3}-\dfrac{2}{x}+C$.}
\end{ex}

\begin{ex}%[2D4H1-4]
	Tính $\displaystyle\int{\sqrt{x\sqrt{x\sqrt{x}}}\mathrm{\,d}x}$.
	\choice
	{$\dfrac{4}{15}x\sqrt[8]x^7+C$}
	{\True $\dfrac{8}{15}x\sqrt[8]x^7+C$}
	{$\dfrac{8}{15}x\sqrt[8]x+C$}
	{$\dfrac{4}{15}x\sqrt[8]x+C$}
	\loigiai{
		\begin{eqnarray*}
		\displaystyle\int{\sqrt{x\sqrt{x\sqrt{x}}}\mathrm{\,d}x}&=&\displaystyle\int{\sqrt{x\sqrt{x\cdot{x^{\frac{1}{2}}}}}\mathrm{\,d}x}=\displaystyle\int{\sqrt{x\cdot{x^{\frac{3}{4}}}}\mathrm{\,d}x}=\displaystyle\int{x^{\frac{7}{8}}\mathrm{\,d}x}\\&=&\dfrac{x^{\frac{7}{8}+1}}{\dfrac{7}{8}+1}+C=\dfrac{8}{15}x\sqrt[8]x^7+C.	
		\end{eqnarray*}
		}
\end{ex}

\begin{ex}%[2D4H1-4]
	Tính $\displaystyle\int{\dfrac{\sqrt{x}-2\sqrt[3]x^2+1}{\sqrt[4]x}\mathrm{\,d}x}$.
	\choice
	{$x\sqrt[5]x-2x\sqrt[12]x^5+\sqrt[4]x^3+C$}
	{\True $\dfrac{4}{5}x\sqrt[4]x-\dfrac{24}{17}x\sqrt[12]x^5+\dfrac{4}{3}\sqrt[4]x^3+C$}
	{$x\sqrt[5]x-\dfrac{24}{17}x\sqrt[12]x^5+\sqrt[4]x^3+C$}
	{$\dfrac{4}{5}x\sqrt[5]x-2x\sqrt[12]x^5+\dfrac{4}{3}\sqrt[4]x^3+C$}
	\loigiai{
		\begin{eqnarray*}
			\displaystyle\int{\dfrac{\sqrt{x}-2\sqrt[3]x^2+1}{\sqrt[4]x}\mathrm{\,d}x}&=&\displaystyle\int{\dfrac{x^{\frac{1}{2}}-2x^{\frac{2}{3}}+1}{x^{\frac{1}{4}}}\mathrm{\,d}x=}\displaystyle\int{\left(\dfrac{x^{\frac{1}{2}}}{x^{\frac{1}{4}}}-2\dfrac{x^{\frac{2}{3}}}{x^{\frac{1}{4}}}+\dfrac{1}{x^{\frac{1}{4}}}\right)\mathrm{\,d}x}\\
			&=&\displaystyle\int\left(x^{\frac{1}{4}}-2x^{\frac{5}{12}}+x^{-\frac{1}{4}}\right)\mathrm{\,d}x
			=\dfrac{4}{5}x\sqrt[4]x-\dfrac{24}{17}x\sqrt[12]x^5+\dfrac{4}{3}\sqrt[4]x^3+C.
		\end{eqnarray*}
	}
\end{ex}

\begin{ex}%[2D4N1-2]
	Cho hàm số $f(x)=x^2+4$. Mệnh đề nào sau đây đúng?
	
	\choice
	{$\displaystyle{\displaystyle\int f(x)\mathrm{\,d}x=2 x+C}$}
	{$\displaystyle{\displaystyle\int f(x)\mathrm{\,d}x=x^2+4 x+C}$}
	{\True $\displaystyle{\displaystyle\int f(x)\mathrm{\,d}x=\dfrac{x^3}{3}+4 x+C}$}
	{$\displaystyle{\displaystyle\int f(x)\mathrm{\,d}x=x^3+4 x+C}$}
	\loigiai{
		Ta có $f(x)=x^2+4 $ nên $ \displaystyle\int f(x)\mathrm{\,d}x=\dfrac{x^3}{3}+4 x+C$.}
\end{ex}
\begin{ex}%[2D4N1-4]
	Trên khoảng $(0;+\infty)$, cho hàm số $f(x)=x^{\frac{3}{2}}$. Mệnh đề nào sau đây đúng?
	\choice
	{$\displaystyle\int{f(x)}\mathrm{\,d}x=\dfrac{3}{2}x^{\frac{1}{2}}+C$}
	{$\displaystyle\int{f(x)}\mathrm{\,d}x=\displaystyle\int{\sqrt{x^3}}\mathrm{\,d}x$}
	{\True $\displaystyle\int{f(x)}\mathrm{\,d}x=\dfrac{2}{5}x^{\frac{5}{2}}+C$}
	{$\displaystyle\int{f(x)}\mathrm{\,d}x=\dfrac{2}{3}x^{\frac{1}{2}}+C$}
	\loigiai{
		Ta có $\displaystyle\int{f(x)}\mathrm{\,d}x=\displaystyle\int{x^{\frac{3}{2}}}\mathrm{\,d}x=\dfrac{2}{5}x^{\frac{5}{2}}+C$.}
\end{ex}

\begin{ex}%[2D4H1-2]
	Cho hàm số $f(x)=\dfrac{x^4+2}{x^2}$. Mệnh đề nào sau đây đúng?
	\choice
	{$\displaystyle\int{f(x)\mathrm{\,d}x=}\dfrac{x^3}{3}-\dfrac{1}{x}+C$}
	{$\displaystyle\int{f(x)\mathrm{\,d}x=}\dfrac{x^3}{3}+\dfrac{2}{x}+C$}
	{$\displaystyle\int{f(x)\mathrm{\,d}x=}\displaystyle\int{\left(x^2+\dfrac{2}{x^2}\right)}\mathrm{\,d}x$}
	{\True $\displaystyle\int{f(x)\mathrm{\,d}x=}\dfrac{x^3}{3}-\dfrac{2}{x}+C$}
	\loigiai{
		Ta có $\displaystyle\int{f(x)\mathrm{\,d}x=}\displaystyle\int{\dfrac{x^4+2}{x^2}}\mathrm{\,d}x=\displaystyle\int{\left(x^2+\dfrac{2}{x^2}\right)}\mathrm{\,d}x=\dfrac{x^3}{3}-\dfrac{2}{x}+C$.}
\end{ex}

\Closesolutionfile{ans}
\indapan{10}{ans/ans-2-B1-D2-LC}
% \TNTF
\Opensolutionfile{ans}[ans/ans-2-B1-D2-DS]
\begin{ex}%[2D4H1-4]
	Các mệnh đề sau đây đúng hay sai
	\choiceTF
	{\True $\displaystyle\int{(\sqrt[3]x^2+x-2)\mathrm{\,d}x}=\dfrac{3}{5}\sqrt[3]x^5+\dfrac{1}{2}x^2-2x+C$}
	{\True $\displaystyle\int{\dfrac{1}{2023x^{2024}}\mathrm{\,d}x}=\dfrac{1}{2023^2x^{2023}}+C$}
	{$\displaystyle\int{(2x-2024)^2\mathrm{\,d}x}=x-1012+C$}
	{\True $\displaystyle\int{\left(\dfrac{1}{4}x^4+4x^3\right)\mathrm{\,d}x}=\dfrac{1}{20}x^5+\dfrac{4}{3}x^4+C$}
	\loigiai{
	$\displaystyle\int{(\sqrt[3]x^2+x-2)\mathrm{\,d}x}=\dfrac{3}{5}\sqrt[3]x^5+\dfrac{1}{2}x^2-2x+C$.\\
	$\displaystyle\int{\dfrac{1}{2023x^{2024}}\mathrm{\,d}x}=\dfrac{1}{2023}\displaystyle\int{x^{-2024}\mathrm{\,d}x}=\dfrac{1}{2023^2x^{2023}}+C$.\\
	$\displaystyle\int{(2x-2024)^2\mathrm{\,d}x}=\dfrac{(2x-2024)^3}{3}+C$.\\
	$\displaystyle\int{\left(\dfrac{1}{4}x^4+4x^3\right)\mathrm{\,d}x}=\dfrac{1}{20}x^5+\dfrac{4}{3}x^4+C$.}
\end{ex}
\begin{ex}%[2D4H1-2][Lê Công Trường]
	Cho	các mệnh đề sau đây 
	\choiceTF
	{\True $F(x)=\dfrac{x^4}{4}-\dfrac{3}{2}{x^2}+\ln \left| x\right|+C$ là nguyên hàm của hàm số $f(x)=x^3-3x+\dfrac{1}{x}$}
	{$F(x)=\dfrac{(5x+3)^6}{6}+C$ là nguyên hàm của hàm số $f(x)=\left(5x+3\right)^5$}
	{$F(x)=\dfrac{3}{2}x\sqrt x+\dfrac{4}{3}x\sqrt[3]{x}+\dfrac{5}{4}x\sqrt[4]{x}+C$ là nguyên hàm của hàm số $f(x)=\sqrt x+\sqrt[3]{x}+\sqrt[4]{x}$}
	{\True $F(x)=\dfrac{1}{3}{x^3}-2024x+C$ là nguyên hàm của hàm số $f(x)=\dfrac{x^3-2024x}{x}$}
	\loigiai{
		\begin{itemchoice}
			\itemch {\bf Đúng}. Vì $f(x)=x^3-3x+\dfrac{1}{x}$\\
			$\Rightarrow F(x)=\displaystyle\int f(x)dx=\displaystyle\int{(x^3-3x{\rm}+\dfrac{1}{x})dx}$\\
			$=\displaystyle\int{x^3dx}-3\displaystyle\int{xdx}+\displaystyle\int{\dfrac{1}{x}dx}=\dfrac{x^4}{4}-\dfrac{3}{2}{x^2}+\ln \left| x\right|+C$.
			\itemch {\bf Sai.} Vì $f(x)=\left(5x+3\right)^5$ \\
			$\Rightarrow F(x)=\displaystyle\int{f(x)dx=}\displaystyle\int(5x+3)^{5}dx$\\
			$=\displaystyle\int{\rm{(5x+3)}^{\rm{5}}\dfrac{d(5x+3)}{5}=\dfrac{(5x+3)^6}{30}+C}$.
			\itemch {\bf Sai.} Vì $f(x)=\sqrt x+\sqrt[3]{x}+\sqrt[4]{x}$\\
			$\Rightarrow F(x)=\displaystyle\int{\left(\sqrt x+\sqrt[3]{x}+\sqrt[4]{x}\right)}dx=\displaystyle\int{\left(x^{\frac{1}{2}}+x^{\frac{1}{3}}+x^{\frac{1}{4}}\right)}dx$\\
			$=\dfrac{2}{3}{x^{\frac{3}{2}}}+\dfrac{3}{4}{x^{\frac{4}{3}}}+\dfrac{4}{5}{x^{\frac{5}{4}}}+C=\dfrac{2}{3}x\sqrt x+\dfrac{3}{4}x\sqrt[3]{x}+\dfrac{4}{5}x\sqrt[4]{x}+C$.
			\itemch {\bf Đúng.} $f(x)=\dfrac{x^3-2024x}{x}\Rightarrow F(x)=\displaystyle\int{\dfrac{x^3-2024x}{x}dx}=\displaystyle\int\left(x^2-2024\right)dx$\\
			$=\dfrac{1}{3}{x^3}-2024x+C$.
		\end{itemchoice}
	}
\end{ex}
\Closesolutionfile{ans}
\indapan{3}{ans/ans-2-B1-D2-DS}
\Opensolutionfile{ans}[ans/ans-2-B1-D1-KQ]
% \TN
\begin{ex}%[2D4H1-2][Lê Công Trường]
	Hệ số của $x^2$ trong nguyên hàm $F(x)$ của hàm số $f(x)=\dfrac{2}{\sqrt{x}}+3^x+3x-2$ là
	\shortans{$1{,}5$}
	\loigiai{
		$F(x)=\displaystyle\int{\left(\dfrac{2}{\sqrt{x}}+3^x+3x-2\right)\mathrm{\,d}x}=4\sqrt{x}+\dfrac{3^x}{\ln 3}+\dfrac{3}{2}{x^2}-2x+C$.
	}
\end{ex}

\begin{ex}%[2D4H1-2][Lê Công Trường]
	Hệ số của $x^3$ trong nguyên hàm $F(x)$ của hàm số $f(x)=m{x^3}-3x^2+\dfrac{4m}{x^3}+\dfrac{5}{2x}-7m$ ($m$ là tham số) là
	\shortans{$-1$}
	\loigiai{
		$F(x)=\displaystyle\int{\left(m{x^3}-3x^2+\dfrac{4m}{x^3}+\dfrac{5}{2x}-7m\right)\mathrm{\,d}x}=\dfrac{m}{4}{x^4}-x^3-\dfrac{2m}{x^2}-\dfrac{5}{2}\ln {|x|}-7mx+C$
	}
\end{ex}

\begin{ex}% [2D4H1-2][Lê Công Trường]
	Tìm nguyên hàm $F(x)$ của hàm số $f(x)=\dfrac{1}{\sqrt{x}}-\dfrac{2}{\sqrt[3]{x}}$. Tổng hệ số của biến $x$ là
	\shortans{$-1$}
	\loigiai{
		$F(x)=\displaystyle\int f(x)\mathrm{\,d}x=\displaystyle\int\left(\dfrac{1}{\sqrt{x}}-\dfrac{2}{\sqrt[3]{x}}\right)\mathrm{\,d}x=\displaystyle\int\dfrac{1}{\sqrt{x}}\mathrm{\,d}x-\displaystyle\int\dfrac{2}{\sqrt[3]{x}}=\displaystyle\int{x^{\frac{-1}{2}}}\mathrm{\,d}x-\displaystyle\int{2x^{\frac{-1}{3}}}\mathrm{\,d}x$\\
		$=\dfrac{x^{\frac{1}{2}}}{\dfrac{1}{2}}-2.\dfrac{x^{\frac{2}{3}}}{\dfrac{2}{3}}+C=2{x^{\frac{1}{2}}}-3x^{\frac{2}{3}}+C=2\sqrt{x}-3\sqrt[3]{x^2}+C$.
	}
\end{ex}

\begin{ex}%%[2D4H1-2][Lê Công Trường]
	Tìm nguyên hàm $F(x)$ của hàm số $f(x)=\dfrac{(x^2-1)^2}{x^2}$. Tổng hệ số của bậc $3$ và bậc $1$ là (làm tròn đến hàng phần chục).
	\shortans{$-1{,}6$}
	\loigiai{
		$\displaystyle\int  f(x)\mathrm{\,d}x=\displaystyle\int\dfrac{(x^2-1)^2}{x^2}\mathrm{\,d}x=\displaystyle\int\dfrac{x^4-2x^2+1}{x^2}\mathrm{\,d}x=\displaystyle\int\left(x^2-2+\dfrac{1}{x^2}\right)\mathrm{\,d}x$\\
		$=\dfrac{x^3}{3}-2x-\dfrac{1}{x}+C$.
	}
\end{ex}

\begin{ex}%%[2D4H1-2][Lê Công Trường]
	Tính $\displaystyle\int{\left(\dfrac{\left(1-x\right)^3}{\sqrt[3]{x}}\right)\mathrm{\,d}x}$. Giá trị tổng hệ số chứa biến là (làm tròn đến hàng phần trăm).
	\shortans{$0{,}55$}
	\loigiai{$\displaystyle\int\left(\dfrac{\left(1-x\right)^3}{\sqrt[3]{x}}\right)\mathrm{\,d}x=\displaystyle\int\dfrac{1-3x+3x^2-x^3}{x^{\frac{1}{3}}}\mathrm{\,d}x=\displaystyle\int\left(x^{\frac{-1}{3}}-3x^{\frac{2}{3}}+3x^{\frac{5}{3}}-x^{\frac{8}{3}}\right)\mathrm{\,d}x$\\
		$=\dfrac{x^{\frac{2}{3}}}{\dfrac{2}{3}}-3\dfrac{x^{\frac{5}{3}}}{\dfrac{5}{3}}+3\dfrac{x^{\frac{8}{3}}}{\dfrac{8}{3}}-\dfrac{x^{\frac{11}{3}}}{\dfrac{11}{3}}+C=\dfrac{3}{2}{x^{\frac{2}{3}}}-\dfrac{9}{5}{x^{\frac{5}{3}}}+\dfrac{9}{8}{x^{\frac{8}{3}}}-\dfrac{3}{11}{x^{\frac{11}{3}}}+C$.
		
	}
\end{ex}

\begin{ex}%[2D4H1-2][Lê Công Trường]
	Tính $\displaystyle\int{\left(\sqrt[3]{x^2}-\sqrt[4]{x^3}+\sqrt[5]{x^4}\right)\mathrm{\,d}x}$. Giá trị tổng hệ số chứa biến là (làm tròn đến hàng phần trăm).
	\shortans{$0{,}58$}
	\loigiai{
		$\displaystyle\int\left(\sqrt[3]{x^2}-\sqrt[4]{x^3}+\sqrt[5]{x^4}\right)\mathrm{\,d}x=\displaystyle\int\left(x^{\frac{2}{3}}-x^{\frac{3}{4}}+x^{\frac{4}{5}}\right)\mathrm{\,d}x=\dfrac{x^{\frac{5}{3}}}{\dfrac{5}{3}}-\dfrac{x^{\frac{7}{4}}}{\dfrac{7}{4}}+\dfrac{x^{\frac{9}{5}}}{\dfrac{9}{5}}+C$\\
		$=\dfrac{3}{5}{x^{\frac{5}{3}}}-\dfrac{4}{7}{x^{\frac{7}{4}}}+\dfrac{5}{9}{x^{\frac{9}{5}}}+C$.
	}
\end{ex}

\begin{ex}%%[2D4H1-2][Lê Công Trường]
	Tính $\displaystyle\int\left(\sqrt{x}+1\right)\left(x-\sqrt{x}+1\right)\mathrm{\,d}x$. Giá trị tổng hệ số chứa biến là (làm tròn đến hàng phần chục).
	\shortans{$1{,}4 $}
	\loigiai{
		$\left(\sqrt{x}+1\right)\left(x-\sqrt{x}+1\right)=\left(\sqrt{x}+1\right)\left[x-\left(\sqrt{x}-1\right)\right]=x\left(\sqrt{x}+1\right)-\left(\sqrt{x}+1\right)\left(\sqrt{x}-1\right)$\\
		$=x\sqrt{x}+x-\left(x-1\right)=x\sqrt{x}+1$.\\
		Do đó $\displaystyle\int\left(\sqrt{x}+1\right)\left(x-\sqrt{x}+1\right)\mathrm{\,d}x=\displaystyle\int\left(x\sqrt{x}+1\right)\mathrm{\,d}x=\displaystyle\int\left(x^{\frac{3}{2}}+1\right)\mathrm{\,d}x$\\
		$=\dfrac{2}{5}x^{\frac{5}{2}}+x+C$.
	}
\end{ex}

\begin{ex}%%[2D4H1-2][Lê Công Trường]
	Tính $\displaystyle\int{\left(2\sqrt{x}-\dfrac{3}{\sqrt[3]{x}}\right)\mathrm{\,d}x}$. Giá trị tổng hệ số chứa biến là (làm tròn đến hàng phần chục).
	\shortans{$-3{,}1$}
	\loigiai{$\displaystyle\int{\left(2\sqrt {x}-\dfrac{3}{\sqrt[3]{x}}\right)\mathrm{\,d}x=\displaystyle\int{\left(2x^{\frac{1}{2}}-3x^{\frac{-1}{3}}\right)}}\mathrm{\,d}x=\dfrac{4}{3}{x^{\frac{3}{2}}}-\dfrac{9}{2}{x^{\frac{2}{3}}}+C=\dfrac{4}{3}\sqrt[2]{x^3}-\dfrac{9}{2}\sqrt[3]{x^2}+C$.\\
	}
\end{ex}	

\begin{ex}%[2D4H1-2][Lê Công Trường]
	Tính $\displaystyle\int{\dfrac{1}{\sqrt{2x}+\sqrt{3x}}\mathrm{\,d}x}=a\left(\sqrt {b}-\sqrt {c}\right)\sqrt {x}$. Giá trị của tổng $a+b+c$ là
	\shortans{$7$}
	\loigiai{
		Ta có: $\dfrac{1}{\sqrt{2x}+\sqrt{3x}}=\dfrac{\sqrt{3x}-\sqrt{2x}}{\left(\sqrt{3x}-\sqrt{2x}\right)\left(\sqrt{3x}+\sqrt{2x}\right)}=\dfrac{\sqrt{3x}-\sqrt{2x}}{x}=\dfrac{\sqrt {x}}{x}\left(\sqrt 3-\sqrt 2\right)$\\
		$=\left(\sqrt 3-\sqrt 2\right){x^{\frac{-1}{2}}}.$\\
		$\displaystyle\int{\dfrac{1}{\sqrt{2x}+\sqrt{3x}}\mathrm{\,d}x}=\displaystyle\int{\left(\sqrt 3-\sqrt 2\right){x^{\frac{-1}{2}}}\mathrm{\,d}x=}\left(\sqrt 3-\sqrt 2\right)\dfrac{x^{\frac{1}{2}}}{\dfrac{1}{2}}=2\left(\sqrt 3-\sqrt 2\right)\sqrt {x}$.}
\end{ex}

\begin{ex}%%[2D4H1-2][Lê Công Trường]
	Tính $\displaystyle\int{\dfrac{1}{\sqrt{5x}-\sqrt{3x}}\mathrm{\,d}x=\left(\sqrt{a}+\sqrt{b}\right)\sqrt {x}+C}$. Giá trị $a+b$ bằng\\
	\shortans{$8$}
	\loigiai{
		$\dfrac{1}{\sqrt{5x}-\sqrt{3x}}=\dfrac{\sqrt{5x}+\sqrt{3x}}{\left(\sqrt{5x}-\sqrt{3x}\right)\left(\sqrt{5x}+\sqrt{3x}\right)}=\dfrac{\sqrt{5x}+\sqrt{3x}}{2x}=\dfrac{\sqrt {x}}{2x}\left(\sqrt 5+\sqrt 3\right).$\\
		$\displaystyle\int{\dfrac{1}{\sqrt{5x}-\sqrt{3x}}\mathrm{\,d}x}=\displaystyle\int{\dfrac{\sqrt {x}}{2x}\left(\sqrt 5+\sqrt 3\right)\mathrm{\,d}x}=\dfrac{\left(\sqrt 5+\sqrt 3\right)}{2}\displaystyle\int{x^{\frac{-1}{2}}}\mathrm{\,d}x=\dfrac{\left(\sqrt 5+\sqrt 3\right)}{2}\cdot\dfrac{x^{\frac{1}{2}}}{\dfrac{1}{2}}$\\
		$=\left(\sqrt 5+\sqrt 3\right)\sqrt {x}+C$.}
\end{ex}

\begin{ex}%%[2D4H1-2][Lê Công Trường]
	Tính $\displaystyle\int{\left(x^2-1\right)^3\mathrm{\,d}x}$. Giá trị tổng hệ số chứa biến là (làm tròn đến hàng phần chục).
	\shortans{$-0{,}5$}
	\loigiai{
		$\displaystyle\int\left(x^2-1\right)^3\mathrm{\,d}x=\displaystyle\int\left(x^6-3x^4+3x^2-1\right)\mathrm{\,d}x=\dfrac{x^7}{7}-3\dfrac{x^5}{5}+x^3-x+C$.}
\end{ex}

\begin{ex}%%[2D4H1-2][Lê Công Trường]
	Tính $\displaystyle\int{\left(2-x^2\right)^4\mathrm{\,d}x}$. Giá trị tổng hệ số chứa biến là (làm tròn đến hàng phần chục).
	\shortans{$9{,}1 $}
	\loigiai{
		Sử dụng khai triển theo nhị thức Newton, ta có:\\
		$\left(2-x^2\right)^4=x^8-8x^6+24x^4-32x^2+16$.\\
		Do đó\\
		$\displaystyle\int{\left(2-x^2\right)^4\mathrm{\,d}x}=\displaystyle\int{\left(x^8-8x^6+24x^4-32x^2+16\right)}\mathrm{\,d}x$\\
		$=\dfrac{x^8}{9}-\dfrac{8}{7}{x^7}+\dfrac{24}{5}{x^5}-\dfrac{32}{3}{x^3}+16x+C$.}
\end{ex}

\begin{ex}%%[2D4H1-2][Lê Công Trường]
	Tính $\displaystyle\int{\left(x-\sqrt[3]{x}\right)^2\mathrm{\,d}x}$. Giá trị tổng hệ số chứa biến là (làm tròn đến hàng phần chục).
	\shortans{$-1{,}1 $}
	\loigiai{
		$\left(x-\sqrt[3]{x}\right)^2=x^2-2x\sqrt[3]{x}-\sqrt[3]{x^2}=x^2-2x^{\frac{4}{3}}-x^{\frac{2}{3}}$.\\
		$\displaystyle\int{\left(x-\sqrt[3]{x}\right)^2\mathrm{\,d}x}=\displaystyle\int{\left(x^2-2x^{\frac{4}{3}}-x^{\frac{2}{3}}\right)\mathrm{\,d}x=}\dfrac{x^3}{3}-\dfrac{6}{7}{x^{\frac{7}{3}}}-\dfrac{3}{5}{x^{\frac{5}{3}}}+C.$}
\end{ex}

\begin{ex}%%[2D4H1-2][Lê Công Trường]
	Tính $\displaystyle\int{\left(\dfrac{x^2+2\sqrt[3]{x}}{x}\right)^2\mathrm{\,d}x}$.  Giá trị tổng hệ số chứa biến là (làm tròn đến hàng phần chục).
	\shortans{$-8{,}7$}
	\loigiai{
		Ta có: $\left(\dfrac{x^2+2\sqrt[3]{x}}{x}\right)^2=\dfrac{x^4+4x^2\sqrt[3]{x}+4\sqrt[3]{x^2}}{x^2}=x^2+4x^{\frac{1}{3}}+4x^{\frac{-4}{3}}$.\\
		$\displaystyle\int{\left(\dfrac{x^2+2\sqrt[3]{x}}{x}\right)^2\mathrm{\,d}x}=\displaystyle\int{\left(x^2+4x^{\frac{1}{3}}+4x^{\frac{-4}{3}}\right)\mathrm{\,d}x=\dfrac{x^3}{3}+4\dfrac{x^{\frac{4}{3}}}{\dfrac{4}{3}}}+4\dfrac{x^{\frac{-1}{3}}}{\dfrac{-1}{3}}+C$\\
		$=\dfrac{1}{3}{x^3}+3x^{\frac{4}{3}}-12x^{\frac{-1}{3}}+C$.}
\end{ex}

\begin{ex}%[2D4H1-2][Lê Công Trường]
	Tìm $m$ để $F(x)=m{x^3}+(3m+2){x^2}-4x+3$ là một nguyên hàm của hàm số $f(x)=3x^2+10x-4$.
	\shortans{$1$}
	\loigiai{
		$\displaystyle\int{f(x)\mathrm{\,d}x}=\displaystyle\int{\left(3x^2+10x-4\right)}\mathrm{\,d}x=x^3+5x^2-4x+C.$ Suy ra $ m=1$.
	}
\end{ex}

\begin{ex}%%[2D4V1-2][Lê Công Trường]
	Tìm $a,b,c$ để $F(x)=(a{x^2}+bx+c)\sqrt{x^2-4x}$ là một nguyên hàm của hàm số $f(x)=(x-2)\sqrt{x^2-4x}$. Giá trị biểu thức $a+b+c$ bằng.
	\shortans{$-1$}
	\loigiai{
		Đặt $ t=\sqrt{x^2-4x}\Rightarrow{t^2}=x^2-4x\Rightarrow 2t\mathrm{\,d}t=\left(2x-4\right)\mathrm{\,d}x=2\left(x-2\right)\mathrm{\,d}x$.\\
		$\Rightarrow \mathrm{\,d}x=\dfrac{2t\mathrm{\,d}t}{2\left(x-2\right)}=\dfrac{t\mathrm{\,d}t}{x-2}$.\\
		$\displaystyle\int{(x-2)\sqrt{x^2-4x}}\mathrm{\,d}x=\displaystyle\int{t.t.\mathrm{\,d}t=\displaystyle\int{t^2\mathrm{\,d}t}=\dfrac{1}{3}}{t^3}+C=\dfrac{1}{3}\sqrt{\left(x^2-4x\right)^3}+C$\\$=\dfrac{1}{3}\left(x^2-4x\right)\sqrt{x^2-4x}+C=\left(\dfrac{1}{3}{x^2}-\dfrac{4}{3}x\right)\sqrt{x^2-4x}+C$.\\
		Vậy $ a=\dfrac{1}{3};\,\,b=-\dfrac{4}{3};\,\,c=0$.}
\end{ex}

\begin{ex}%%[2D4V1-2][Lê Công Trường]
	Tìm $a,b,c$ để $F(x)=(a{x^2}+bx+c)\sqrt{2x-3}$ là một nguyên hàm của hàm số $f(x)=\dfrac{20x^2-30x+7}{\sqrt{2x-3}}$.  Giá trị biểu thức $a+b+c$ bằng\\
	\shortans{$3$}
	\loigiai{
		Theo định nghĩa nguyên hàm thì $ F'(x)=f(x)$.\\
		Ta có 
		\begin{eqnarray*}
			F'(x) & =& \left(2ax+b\right)\sqrt{2x-3}+(a{x^2}+bx+c)\dfrac{2}{2\sqrt{2x-3}}\\
			&= & \dfrac{\left(2ax+b\right)\left(2x-3\right)+a{x^2}+bx+c}{\sqrt{2x-3}}\\
			&= & \dfrac{5a{x^2}+\left(-6a+3b\right)x-3b+c}{\sqrt{2x-3}}.
		\end{eqnarray*}
		Từ đó ta có $\dfrac{5a{x^2}+\left(-6a+3b\right)x-3b+c}{\sqrt{2x-3}}=\dfrac{20x^2-30x+7}{\sqrt{2x-3}}$.\\
		Sử dụng phương pháp đồng nhất hệ số, ta được\\
		$\heva{
			&5a=20\\
			&-6a+3b=-30\\
			&-3b+c=7.
		}\Leftrightarrow
		\heva{
			&a=4\\
			&b=-2\\
			&c=1.
		}$
	}
\end{ex}
\Closesolutionfile{ans}
\indapan{6}{ans/ans-2-B1-D1-KQ}

\begin{ex}%[2D4H1-3][Lê Công Trường]
	Hàm số $F(x)=\cot x$ là một nguyên hàm của hàm số nào dưới đây trên khoảng $\left(0;\dfrac{\pi}{2}\right)$
	\choice
	{$f_2(x)=\dfrac{1}{\sin^2x}$}
	{$f_1(x)=-\dfrac{1}{\cos^2x}$}
	{$f_4(x)=\dfrac{1}{\cos^2x}$}
	{\True $f_3(x)=-\dfrac{1}{\sin^2x}$}
	\loigiai{
		Có $\displaystyle\int{\dfrac{1}{\sin^2x}\mathrm{\,d}x}=-\cot x+C$ suy ra $F(x)=\cot x$ trên khoảng $\left(0;\dfrac{\pi}{2}\right)$ là một nguyên hàm của hàm số $f_3(x)=-\dfrac{1}{\sin^2x}$.}
\end{ex}

\begin{ex}%[2D4H1-3][Lê Công Trường]
	Cho hàm số $f(x)=1+\sin x$. Khẳng định nào dưới đây đúng?
	\choice
	{\True $\displaystyle\int{f(x){\rm{d}}x}=x-\cos x+C$}
	{$\displaystyle\int{f(x){\rm{d}}x}=x+\sin x+C$}
	{$\displaystyle\int{f(x){\rm{d}}x}=x+\cos x+C$}
	{$\displaystyle\int{f(x){\rm{d}}x}=\cos x+C$}
	\loigiai
	{Ta có $\displaystyle\int{f(x){\rm{d}}x=\displaystyle\int{\left(1+\sin x\right){\rm{d}}x}=\displaystyle\int{1\rm{d}x}+\displaystyle\int{\sin x{\rm{d}}x}=x-\cos x+C}$.}
\end{ex}

\begin{ex}%[2D4H1-3][Lê Công Trường]
	Tìm nguyên hàm $F(x)$ của hàm số $f(x)=\cos ^2\dfrac{x}{2}$
	\choice
	{$F(x)=2\cos\dfrac{x}{2}+C$}
	{\True $F(x)=\dfrac{1}{2}\left(1+\sin x\right)+C$}
	{$F(x)=2\sin\dfrac{x}{2}+C$}
	{$F(x)=\dfrac{1}{2}\left(1-\sin x\right)+C$}
	\loigiai{
		Ta có:$f(x)=\cos ^2\dfrac{x}{2}\Rightarrow F(x)=\displaystyle\int{\cos^2\dfrac{x}{2}\mathrm{\,d}x}=\displaystyle\int{\dfrac{1+\cos x}{2}\mathrm{\,d}x}=\dfrac{1}{2}\displaystyle\int{\left(1+\cos x\right)\mathrm{\,d}x}$\\
		$=\dfrac{1}{2}\left(1+\sin x\right)+C$.}
\end{ex}

\begin{ex}%[2D4H1-3][Lê Công Trường]
	Cho hàm số $f(x)=1-\dfrac{1}{\cos^2x}$. Khẳng định nào dưới đây đúng?
	\choice
	{$\displaystyle\int{f(x){\rm{d}}x}=x+\tan x+C$}
	{$\displaystyle\int{f(x){\rm{d}}x}=x+\cot x+C$}
	{\True $\displaystyle\int{f(x){\rm{d}}x}=x-\tan x+C$}
	{$\displaystyle\int{f(x){\rm{d}}x}=x-\cot x+C$}
	\loigiai
	{
		$\displaystyle\int{f(x){\rm{d}}x}=\displaystyle\int{\left(1-\dfrac{1}{\cos^2x}\right){\rm{d}}x}=x-\tan x+C$.}
\end{ex}

\begin{ex}%%[2D4H1-3][Lê Công Trường]
	Họ nguyên hàm của hàm số $f(x)=\cos x+6x$ là
	\choice
	{\True $\sin x+3x^2+C$}
	{$-\sin x+3x^2+C$}
	{$\sin x+6x^2+C$}
	{$-\sin x+C$}
	\loigiai
	{
		Ta có $\displaystyle\int{f(x){\rm{d}}x=\displaystyle\int{\left(\cos x+6x\right){\rm{d}}x=\sin x+3x^2+C}}$.}
\end{ex}

\begin{ex}%%[2D4H1-3][Lê Công Trường]
	Tìm nguyên hàm của hàm số $f(x)=2\sin x+3x$.
	\choice
	{\True $\displaystyle\int{\left(2\sin x+3x\right)\mathrm{\,d}x=-2\cos x+\dfrac{3}{2}{x^2}+C}$}
	{$\displaystyle\int{\left(2\sin x+3x\right)\mathrm{\,d}x=2\cos x+3x^2+C}$}
	{$\displaystyle\int{\left(2\sin x+3x\right)\mathrm{\,d}x=\sin^2x+\dfrac{3}{2}x+C}$}
	{$\displaystyle\int{\left(2\sin x+3x\right)\mathrm{\,d}x=\sin 2x+\dfrac{3}{2}{x^2}+C}$}
	\loigiai
	{
		$\displaystyle\int{\left(2\sin x+3x\right)\mathrm{\,d}x}=-2\cos x+\dfrac{3}{2}{x^2}+C$}
\end{ex}

\begin{ex}%%[2D4H1-3][Lê Công Trường]
	Tính$\displaystyle\int{\left(x-\sin x\right)}{\rm{d}}x$.
	\choice
	{$\dfrac{x^2}{2}+\sin x+C$}
	{$\dfrac{x^2}{2}-\cos x+C$}
	{$\dfrac{x^2}{2}-\sin x+C$}
	{\True $\dfrac{x^2}{2}+\cos x+C$}
	\loigiai
	{
		Ta có $\displaystyle\int{\left(x-\sin x\right){\rm{d}}x\,\rm{=}\,}\dfrac{x^2}{2}+\cos x+C$.}
\end{ex}
\begin{ex}%[2D4H1-3][Lê Công Trường]
	Họ nguyên hàm của hàm số $f(x)=3x^2+\sin x$ là
	\choice
	{$x^3+\cos x+C$}
	{$6x+\cos x+C$}
	{\True $x^3-\cos x+C$}
	{$6x-\cos x+C$}
	\loigiai
	{
		Ta có $\displaystyle\int\left(3x^2+\sin x\right){\rm{d}}x=x^3-\cos x+C$.}
\end{ex}
\begin{ex}%[2D4H1-3][Lê Công Trường]
	Họ nguyên hàm của hàm số $ f(x)=\dfrac{1}{x}+\sin x$ là
	\choice
	{$\ln x-\cos x+C$}
	{$-\dfrac{1}{x^2}-\cos x+C$}
	{$\ln \left| x\right|+\cos x+C$}
	{\True $\ln \left| x\right|-\cos x+C$}
	\loigiai
	{
		Ta có $\displaystyle\int{f(x){\rm{d}}x}=\displaystyle\int{\left(\dfrac{1}{x}+\sin x\right){\rm{d}}x}=\displaystyle\int{\dfrac{1}{x}{\rm{d}}x}+\displaystyle\int{\sin x{\rm{d}}x}=\ln \left| x\right|-\cos x+C$.}
\end{ex}
\begin{ex}%%[2D4H1-3][Lê Công Trường]
	Cho $\displaystyle\int{f(x)}\,\rm{d}x=-\cos x+C$. Khẳng định nào dưới đây đúng?
	\choice
	{$ f(x)=-\sin x$}
	{$ f(x)=-\cos x$}
	{\True $ f(x)=\sin x$}
	{$ f(x)=\cos x$}
	\loigiai{
		Áp dụng công thức $\smallint{\rm{sin}}x{\rm{\;d}}x=-\rm{cos}x+C$. Suy ra $ f(x)=\rm{sin}x$.}
\end{ex}
\begin{ex}%[2D4H1-3][Lê Công Trường]
	Cho hàm số $ f(x)=\displaystyle\int{\cos\dfrac{x}{2}\sin\dfrac{x}{2}}$. Khẳng định nào dưới đây đúng?
	\choice
	{$\displaystyle\int{\cos\dfrac{x}{2}\sin\dfrac{x}{2}}=\dfrac{1}{2}\sin+C$}
	{$\displaystyle\int{\cos\dfrac{x}{2}\sin\dfrac{x}{2}}=\dfrac{1}{2}\cos x+C$}
	{$\displaystyle\int{\cos\dfrac{x}{2}\sin\dfrac{x}{2}}=-\dfrac{1}{2}\sin x+C$}
	{\True $\displaystyle\int{\cos\dfrac{x}{2}\sin\dfrac{x}{2}}=-\dfrac{1}{2}\cos x+C$}
	\loigiai{
		$\displaystyle\int{\cos\dfrac{x}{2}\sin\dfrac{x}{2}}=\dfrac{1}{2}\displaystyle\int{\sin x}\mathrm{\,d}x=-\dfrac{1}{2}\cos x+C$.}
\end{ex}
\Closesolutionfile{ans}
\indapan{10}{ans/ans-2-B1-D2-TN}
\Opensolutionfile{ans}[ans/ans-2-B1-D2-DS]
% \TNTF
\begin{ex}%%[2D4H1-3][Lê Công Trường]
	Các mệnh đề sau đây đúng hay sai?
	\choiceTF
	{\True $\displaystyle\int{\left(2+\cot^2x\right)\mathrm{\,d}x}=x-\cot x+C$}
	{ $\displaystyle\int{\left(1-\cos^2\dfrac{x}{2}\right)\mathrm{\,d}x}=\dfrac{1}{2}\left(x+\sin x\right)+C$}
	{$\displaystyle\int{\left(\sin\dfrac{x}{2}+\cos\dfrac{x}{2}\right)^2}\mathrm{\,d}x=x+\cos x+C$}
	{ $\displaystyle\int{\left(\sin\dfrac{x}{2}-\cos\dfrac{x}{2}\right)^2}\mathrm{\,d}x=x-\cos x+C$}
	\loigiai{
		\begin{itemchoice}
			\itemch {\bf Đúng}. Vì
			$\displaystyle\int{\left(2+\cot^2x\right)\mathrm{\,d}x}$\\
			$=\displaystyle\int{\left(1+1+\cot^2x\right)\mathrm{\,d}x}=\displaystyle\int{\left(1+\dfrac{1}{\sin^2x}\right)\mathrm{\,d}x}=x-\cot x+C$.
			\itemch {\bf Sai}. Vì $\displaystyle\int{\left(1-\cos^2\dfrac{x}{2}\right)\mathrm{\,d}x}=\displaystyle\int{\sin^2\dfrac{x}{2}\mathrm{\,d}x}=\displaystyle\int{\dfrac{1-\cos x}{2}\mathrm{\,d}x}=\dfrac{1}{2}\left(x-\sin x\right)+C$.
			\itemch {\bf Sai}. Vì $\displaystyle\int{\left(\sin\dfrac{x}{2}+\cos\dfrac{x}{2}\right)^2}\mathrm{\,d}x=\displaystyle\int{\left(1+\sin x\right)}\mathrm{\,d}x=x-\cos x+C$.
			\itemch {\bf Sai}. Vì $\displaystyle\int{\left(\sin\dfrac{x}{2}-\cos\dfrac{x}{2}\right)^2}\mathrm{\,d}x=\displaystyle\int{\left(1-\sin x\right)}\mathrm{\,d}x=x+\cos x+C$.
		\end{itemchoice}
	}
\end{ex}
\Closesolutionfile{ans}
\indapan{2}{ans/ans-2-B1-D2-DS}
\Opensolutionfile{ans}[ans/ans-2-B1-D2-KQ]
% \TN
\begin{ex}%%[2D4H1-3][Lê Công Trường]
	Tìm nguyên hàm $ F(x)$của hàm số $f(x)=2024-2\sin ^2\dfrac{x}{2}$. Hệ số của biến $x$ là
	\shortans{$2023$}
	\loigiai{
		$\Rightarrow F(x)=\displaystyle\int{\left(2024-2\sin^2\dfrac{x}{2}\right)}\mathrm{\,d}x=\displaystyle\int{\left(2023+\cos x\right)}\mathrm{\,d}x=2023x-\sin x+C$.}
\end{ex}
\begin{ex}%%[2D4H1-3][Lê Công Trường]
	Tìm nguyên hàm $ F(x)$của hàm số $f(x)=\dfrac{1}{\sin^2\dfrac{x}{2}\cdot\cos^2\dfrac{x}{2}}==a\cot x+C$. Giá trị $a$ là
	\shortans{$-4$}
	\loigiai{
		Ta có $\dfrac{1}{\sin^2\dfrac{x}{2}\cdot\cos^2\dfrac{x}{2}}=\dfrac{1}{\left(\sin\dfrac{x}{2}\cdot\cos\dfrac{x}{2}\right)^2}=\dfrac{1}{\left(\dfrac{\sin x}{2}\right)^2}=\dfrac{4}{\sin^2x}\cdot$\\
		$ F(x)=\displaystyle\int{f(x)\mathrm{\,d}x=\displaystyle\int{\dfrac{1}{\sin^2\dfrac{x}{2}\cdot\cos^2\dfrac{x}{2}}}}\mathrm{\,d}x=\displaystyle\int{\dfrac{4}{\sin^2x}=-4\cot x+C}$.}
\end{ex}
\begin{ex}%%[2D4H1-3][Lê Công Trường]
	Tìm nguyên hàm $ F(x)$ của hàm số $f(x)=\dfrac{1}{3}{x^2}-2x+\dfrac{1}{2}{\tan ^2}x=\dfrac{x^3}{a}+bx^2+\dfrac{1}{c}x+d\tan x+C$. Giá trị của $a+b+c+d$ là
	\shortans{$6{,}5$}
	\loigiai{$F(x)=\displaystyle\int{f(x)\mathrm{\,d}x}$\\
		$=\displaystyle\int{\left(\dfrac{1}{3}{x^2}-2x+\dfrac{1}{2}{\tan^2}x\right)}\mathrm{\,d}x=\displaystyle\int{\left(\dfrac{1}{3}{x^2}-2x+\dfrac{1}{2}\dfrac{\sin^2x}{\cos^2x}\right)}\mathrm{\,d}x\\
		=\displaystyle\int{\left[\dfrac{1}{3}{x^2}-2x+\dfrac{1}{2}\left(\dfrac{1-\cos^2x}{\cos^2x}\right)\right]}\mathrm{\,d}x=\displaystyle\int{\left[\dfrac{1}{3}{x^2}-2x+\dfrac{1}{2}\left(\dfrac{1}{\cos^2x}-1\right)\right]}\mathrm{\,d}x\\
		=\dfrac{x^3}{9}-x^2+\dfrac{1}{2}\left(\tan x-x\right)+C=\dfrac{x^3}{9}-x^2-\dfrac{1}{2}x+\dfrac{1}{2}\tan x+C$.
	}
\end{ex}
% \begin{ex}%%[2D4V1-3][Lê Công Trường]
% 	Tính $I=\displaystyle\int{x\left(1-\dfrac{\sin^2\dfrac{x}{2}}{2}\right)\mathrm{\,d}x}$. Hệ số của hạng tử $\cos {x}$ của $I$ là
% 	\shortans{$-1$} 
% 	\loigiai{
% 		Đáp án: Ta có $x\left(1-\dfrac{\sin^2\dfrac{x}{2}}{2}\right)=x\left(1-\dfrac{1-cox}{4}\right)=\dfrac{3}{4}x+\dfrac{1}{4}x\cos x$.\\
% 		$\displaystyle\int x\left(1-\dfrac{\sin^2\dfrac{x}{2}}{2}\right)\mathrm{\,d}x=\displaystyle\int\left(\dfrac{3}{4}x+\dfrac{1}{4}x\cos x\right)\mathrm{\,d}x=\displaystyle\int\dfrac{3}{4}x\mathrm{\,d}x+\displaystyle\int\dfrac{1}{4}x\cos x\mathrm{\,d}x$\\
% 		$=\dfrac{3}{8}{x^2}+C_1+\dfrac{1}{4}\displaystyle\int{x\cos x\mathrm{\,d}x.}$\\
% 		Đặt $\heva{
% 			&u=x\Rightarrow \mathrm{\,d}u=\mathrm{\,d}x\\
% 			&dv=\cos x\mathrm{\,d}x\Rightarrow v=\sin x.
% 		}$\\
% 		Sử dụng phương pháp tích phân từng phần, ta có\\
% 		$\displaystyle\int{x\cos x\mathrm{\,d}x}=x\sin x+\displaystyle\int{\sin x\mathrm{\,d}x=x\sin x-\cos x+C_2}$.\\
% 		Vậy $\displaystyle\int{x\left(1-\dfrac{\sin^2\dfrac{x}{2}}{2}\right)\mathrm{\,d}x}=\dfrac{3}{8}{x^2}+x\sin x-\cos x+C.$}
% \end{ex}	
\begin{ex}%%[2D4H1-3][Lê Công Trường]
	Tính $\displaystyle\int{x^2\left(1+\dfrac{1}{x}-\dfrac{\tan^2x}{x^2}\right)\mathrm{\,d}x}=\dfrac{x^m}{n}+\dfrac{x^p}{q}+x+r\tan x+C$. Giá trị biểu thức $P=\dfrac{m}{n}+\dfrac{p}{q}+2r$ là	
	\shortans{$0$} 
	\loigiai{
		$\displaystyle\int{x^2\left(1+\dfrac{1}{x}-\dfrac{\tan^2x}{x^2}\right)\mathrm{\,d}x}=\displaystyle\int{\left(x^2+x-\tan^2x\right)\mathrm{\,d}x}=\dfrac{x^3}{3}+\dfrac{x^2}{2}-(\tan x-x)+C$\\
		$=\dfrac{x^3}{3}+\dfrac{x^2}{2}+x-\tan x+C$.}
\end{ex}
\begin{ex}%%[2D4V1-3][Lê Công Trường]
	Tính $T=\displaystyle\int{x\left(2024-\dfrac{1}{x^3}+\dfrac{\sin x}{x}\right)\mathrm{\,d}x}$. Hệ số của hạng tử $\cos {x}$ của $T$ là
	\shortans{$-1$} 
	\loigiai{
		$\displaystyle\int{x\left(2024-\dfrac{1}{x^3}+\dfrac{\sin x}{x}\right)\mathrm{\,d}x}=\displaystyle\int{\left(2024x-\dfrac{1}{x^2}+\sin x\right)}\mathrm{\,d}x=1012x^2+\dfrac{1}{x}-\cos x+C.$}
\end{ex}
\begin{ex}%Câu 27%[2D4H1-5]
	Tính $R=\displaystyle\int{x^3\left[\dfrac{\left(\sin\dfrac{x}{2}+\cos\dfrac{x}{2}\right)^2}{x^3}-2x+\dfrac{1}{x^{2024}}\right]}\mathrm{\,d}x= ax+b\cos x+c{x^5}-\dfrac{1}{d\cdot x^{2020}}+C$. Giá trị $a+b+c+d+7$ là (làm tròn đến hàng đơn vị)
	\shortans{$2025$} 
	\loigiai{
		Ta có
		\begin{eqnarray*}
			{x^3}\left[\dfrac{\left(\sin\dfrac{x}{2}+\cos\dfrac{x}{2}\right)^2}{x^3}-2x+\dfrac{1}{x^{2024}}\right] &=& \left(\sin\dfrac{x}{2}+\cos\dfrac{x}{2}\right)^2-2x^4+x^{-2021}\\
			&=& \sin ^2\dfrac{x}{2}+\cos^2\dfrac{x}{2}+2\sin\dfrac{x}{2}\cos\dfrac{x}{2}-2x^4+x^{-2021}\\
			&=&1+2\sin x-2x^4+x^{-2021}.
		\end{eqnarray*}
		Khi đó\\
		\begin{eqnarray*}
			\displaystyle\int{x^3\left[\dfrac{\left(\sin\dfrac{x}{2}+\cos\dfrac{x}{2}\right)^2}{x^3}-2x+\dfrac{1}{x^{2024}}\right]}\mathrm{\,d}x&=& \displaystyle\int{\left(1+2\sin x-2x^4+x^{-2021}\right)\mathrm{\,d}x}\\
			&= & x-2\cos x-\dfrac{2}{5}{x^5}-\dfrac{1}{2020x^{2020}}+C.
		\end{eqnarray*}
	}
\end{ex}	
\begin{ex}%[2D4V1-3][Lê Công Trường]
	Tính $\displaystyle\int{x^2\left[\dfrac{1}{x^2\sin^2\dfrac{x}{2}\cdot\cos^2\dfrac{x}{2}}+\dfrac{3}{x^3}-\dfrac{4}{x^4}\right]}\mathrm{\,d}x=a\cot{x}+b\ln \left| x\right|+\dfrac{c}{x}+C$. Giá trị $a+b+c$ là
	\shortans{$3$} 
	\loigiai{
		Ta có\\
		$\dfrac{1}{\sin^2\dfrac{x}{2}\cdot\cos^2\dfrac{x}{2}}=\dfrac{1}{\left(\sin\dfrac{x}{2}\cdot\,\cos\dfrac{x}{2}\right)^2}=\dfrac{1}{\left(\dfrac{\sin x}{2}\right)^2}=\dfrac{4}{\sin^2x}.$\\
		$x^2\left[\dfrac{1}{x^2\sin^2\dfrac{x}{2}\cdot\cos^2\dfrac{x}{2}}+\dfrac{3}{x^3}-\dfrac{4}{x^4}\right]=\dfrac{1}{\sin^2\dfrac{x}{2}\cdot\cos^2\dfrac{x}{2}}+\dfrac{3}{x}-\dfrac{4}{x^2}=\dfrac{4}{\sin^2x}+\dfrac{3}{x}-\dfrac{4}{x^2}$.\\
		Khi đó
		\begin{eqnarray*}
			\displaystyle\int{x^2\left[\dfrac{1}{x^2\sin^2\dfrac{x}{2}\cdot\cos^2\dfrac{x}{2}}+\dfrac{3}{x^3}-\dfrac{4}{x^4}\right]}\mathrm{\,d}x	&= & \displaystyle\int{\left(\dfrac{4}{\sin^2x}+\dfrac{3}{x}-\dfrac{4}{x^2}\right)}\mathrm{\,d}x\\
			&= & -4\cot x+3\ln \left| x\right|+\dfrac{4}{x}+C.
		\end{eqnarray*}
	}
\end{ex}
\Closesolutionfile{ans}
\indapan{6}{ans/ans-2-B1-D2-KQ}

\Opensolutionfile{ans}[ans/ans-C4B1CD1-LC]
% \TN
\begin{ex}%[2D4N2-4]
	Họ nguyên hàm của hàm số $f(x)=e^{3x}$ là hàm số nào sau đây?
	\choice
	{$3e^x+C$}
	{\True $\dfrac{1}{3}e^{3x}+C$}
	{$\dfrac{1}{3}e^{x}+C$}
	{$3e^{3x}+C$}
	\loigiai{
		\textbf{Cách 1:} $\displaystyle\int e^{3x} \mathrm{\,d}x=\displaystyle\int (e^{3})^x \mathrm{\,d}x=\dfrac{(e^3)^x}{\ln e^3}+C=\dfrac{e^{3x}}{3}+C$.\\
		\textbf{Cách 2 (Trắc nghiệm): } $\displaystyle\int e^{3x} \mathrm{\,d}x=\dfrac{1}{3}e^{3x}+C$, với $C$ là hằng số bất kì.
	}
\end{ex}

\begin{ex}%[2D4N2-4]
	Nguyên hàm của hàm số $y=e^{2x-1}$  là
	\choice
	{$2e^{2x-1}+C$}
	{$e^{2x-1}+C$}
	{\True $\dfrac{1}{2}e^{2x-1}+C$}
	{$\dfrac{1}{2}e^{x}+C$}
	\loigiai{
		\textbf{Cách 1:} $\displaystyle\int e^{2x-1} \mathrm{\,d}x=\displaystyle\int e^{-1}(e^{2})^x \mathrm{\,d}x=e^{-1}\dfrac{(e^2)^x}{\ln e^2}+C=\dfrac{e^{2x-1}}{2}+C$.\\
		\textbf{Cách 2:} $\displaystyle\int e^{2x-1} \mathrm{\,d}x=\dfrac{1}{2}\displaystyle\int e^{2x-1} \mathrm{\,d}(2x-1)=\dfrac{1}{2}e^{2x-1}+C$.
	}
\end{ex}

\begin{ex}%[2D4N2-4]
	Cho hàm số $f(x)=e^x+2$. Khẳng định nào dưới đây là \textbf{đúng}?
	\choice
	{$\displaystyle\int f(x) \mathrm{\,d}x=e^{x-2}+C$}
	{\True $\displaystyle\int f(x) \mathrm{\,d}x=e^{x}+2x+C$}
	{$\displaystyle\int f(x) \mathrm{\,d}x=e^{x}+C$}
	{$\displaystyle\int f(x) \mathrm{\,d}x=e^{x}-2x+C$}
	\loigiai{
		Ta có $\displaystyle\int f(x) \mathrm{\,d}x=\displaystyle\int (e^x+2) \mathrm{\,d}x=e^x+2x+C$.
	}
\end{ex}

\begin{ex}%[2D4N2-4]
	Cho hàm số $f(x)=e^x+2x$. Khẳng định nào dưới đây \textbf{đúng}?
	\choice
	{\True $\displaystyle\int f(x) \mathrm{\,d}x=e^{x}+x^2+C$}
	{$\displaystyle\int f(x) \mathrm{\,d}x=e^{x}+C$}
	{$\displaystyle\int f(x) \mathrm{\,d}x=e^{x}-x^2+C$}
	{$\displaystyle\int f(x) \mathrm{\,d}x=e^{x}+2x^2+C$}
	\loigiai{
		Ta có $\displaystyle\int f(x) \mathrm{\,d}x=\displaystyle\int (e^x+2x) \mathrm{\,d}x=e^x+x^2+C$.
	}
\end{ex}

\begin{ex}%[2D4N2-4]
	Tìm nguyên hàm của hàm số  $f(x)=7^x$.
	\choice
	{\True $\displaystyle\int 7^x \mathrm{\,d}x=\dfrac{7^x}{\ln 7}+C$}
	{$\displaystyle\int 7^x \mathrm{\,d}x=7^{x+1}+C$}
	{$\displaystyle\int 7^x \mathrm{\,d}x=\dfrac{7^{x+1}}{x+1}+C$}
	{$\displaystyle\int 7^x \mathrm{\,d}x=7^x\ln 7+C$}
	\loigiai{
		Ta có $\displaystyle\int 7^x \mathrm{\,d}x=\dfrac{7^x}{\ln 7}+C$.
	}
\end{ex}

\begin{ex}%[2D4N2-4]
	Nguyên hàm của hàm số  $f(x)=2^x$ là
	\choice
	{$\displaystyle\int 2^x \mathrm{\,d}x=\ln 2\cdot 2^x+C$}
	{$\displaystyle\int 2^x \mathrm{\,d}x=2^x+C$}
	{\True $\displaystyle\int 2^x \mathrm{\,d}x=\dfrac{2^{x}}{\ln 2}+C$}
	{$\displaystyle\int 2^x \mathrm{\,d}x=\dfrac{2^x}{x+1}\ln 7+C$}
	\loigiai{
		Ta có $\displaystyle\int 2^x \mathrm{\,d}x=\dfrac{2^x}{\ln 2}+C$.
	}
\end{ex}

\begin{ex}%[2D4N2-4]
	Tất cả các nguyên hàm của hàm số  $f(x)=3^{-x}$ là
	\choice
	{\True $-\dfrac{3^{-x}}{\ln 3}+C$}
	{$-3^{-x}+C$}
	{$-3^{-x}\ln 3+C$}
	{$\dfrac{3^{-x}}{\ln 3}+C$}
	\loigiai{
		Ta có $\displaystyle\int 3^{-x} \mathrm{\,d}x=\displaystyle\int (3^{-1})^{x} \mathrm{\,d}x=-\dfrac{3^{-x}}{\ln 3}+C$.
	}
\end{ex}

\begin{ex}%[2D4N2-4]
	Tìm nguyên hàm của hàm số $f(x)=3^x+2x$.
	\choice
	{\True $\displaystyle\int (3^x+2x) \mathrm{\,d}x=\dfrac{3^x}{\ln 3}+x^2+C$}
	{$\displaystyle\int (3^x+2x) \mathrm{\,d}x=3^x\ln 3+x^2+C$}
	{$\displaystyle\int (3^x+2x) \mathrm{\,d}x=\dfrac{3^x}{\ln 3}+x+C$}
	{$\displaystyle\int (3^x+2x) \mathrm{\,d}x=3^x\ln 3+x+C$}
	\loigiai{
		Ta có $\displaystyle\int (3^x+2x) \mathrm{\,d}x=\dfrac{3^x}{\ln 3}+x^2+C$.
	}
\end{ex}

\begin{ex}%[2D4N2-4]
	Họ nguyên hàm của hàm số $f(x)=e^x-2x$ là
	\choice
	{$e^x+x^2+C$}
	{\True $e^x-x^2+C$}
	{$\dfrac{1}{x+1}e^x-x^2+C$}
	{$e^x-2+C$}
	\loigiai{
		Ta có $\displaystyle\int (e^x-2x) \mathrm{\,d}x=e^x-x^2+C$.
	}
\end{ex}

\begin{ex}%[2D4H2-4]
	Tìm nguyên hàm của hàm số $f(x)=e^x\left(2017-\dfrac{2018e^{-x}}{x^5}\right) $.
	\choice
	{$\displaystyle\int f(x) \mathrm{\,d}x=2017e^x-\dfrac{2018}{x^4}+C$}
	{$\displaystyle\int f(x) \mathrm{\,d}x=2017e^x+\dfrac{2018}{x^4}+C$}
	{\True $\displaystyle\int f(x) \mathrm{\,d}x=2017e^x+\dfrac{504{,}5}{x^4}+C$}
	{$\displaystyle\int f(x) \mathrm{\,d}x=2017e^x-\dfrac{504{,}5}{x^4}+C$}
	\loigiai{
		\begin{eqnarray*}
			\displaystyle\int f(x) \mathrm{\,d}x
			&=&\displaystyle\int e^x\left(2017-\dfrac{2018e^{-x}}{x^5}\right)\mathrm{\,d}x\\
			&=&\displaystyle\int \left(2017e^x-\dfrac{2018}{x^5}\right)\mathrm{\,d}x\\
			&=&2017e^x+\dfrac{504{,}5}{x^4}+C
		\end{eqnarray*}
	}
\end{ex}

\begin{ex}%[2D4H2-4]
	Họ nguyên hàm của hàm số $y=e^x\left(2+\dfrac{e^{-x}}{\cos^2x}\right) $ là
	\choice
	{\True $2e^x+\tan x+C$}
	{$2e^x-\tan x+C$}
	{$2e^x-\dfrac{1}{\cos x}+C$}
	{$2e^x+\dfrac{1}{\cos x}+C$}
	\loigiai{
		Ta có $\displaystyle\int y \mathrm{\,d}x=\displaystyle\int e^x\left(2+\dfrac{e^{-x}}{\cos^2x}\right)\mathrm{\,d}x=\displaystyle\int \left(2e^x+\dfrac{1}{\cos^2x}\right)\mathrm{\,d}x=2e^x+\tan x+C$.
	}
\end{ex}

\begin{ex}%[2D4N2-4]
	Tìm họ nguyên hàm của hàm số $y=x^2-3^x+\dfrac{1}{x}$.
	\choice
	{$\dfrac{x^3}{3}-\dfrac{3^x}{\ln 3}-\dfrac{1}{x^2}+C,\,C\in \mathbb{R}$}
	{$\dfrac{x^3}{3}-3^x+\dfrac{1}{x^2}+C,\,C\in \mathbb{R}$}
	{\True $\dfrac{x^3}{3}-\dfrac{3^x}{\ln 3}+\ln \left|x\right|+C,\,C\in \mathbb{R}$}
	{$\dfrac{x^3}{3}-\dfrac{3^x}{\ln 3}-\ln \left|x\right|+C,\,C\in \mathbb{R}$}
	\loigiai{
		Ta có $\displaystyle\int \left( x^2-3^x+\dfrac{1}{x}\right)  \mathrm{\,d}x=\dfrac{x^3}{3}-\dfrac{3^x}{\ln 3}+\ln \left|x\right|+C,\,C\in \mathbb{R}$.
	}
\end{ex}

\begin{ex}%[2D4N2-4]
	Khẳng định nào dưới đây \textbf{đúng}?
	\choice
	{$\displaystyle\int e^x \mathrm{\,d}x=xe^x+C$}
	{$\displaystyle\int e^x \mathrm{\,d}x=e^{x+1}+C$}
	{$\displaystyle\int e^x \mathrm{\,d}x=-e^{x+1}+C$}
	{\True $\displaystyle\int e^x \mathrm{\,d}x=e^x+C$}
	\loigiai{
		Ta có $\displaystyle\int e^x \mathrm{\,d}x=e^x+C$.
	}
\end{ex}

\begin{ex}%[2D4N2-4]
	Cho hàm số $f(x)=1+e^{2x}$. Khẳng định nào dưới đây \textbf{đúng}?
	\choice
	{$\displaystyle\int f(x) \mathrm{\,d}x=x+\dfrac{1}{2}e^x+C$}
	{$\displaystyle\int f(x) \mathrm{\,d}x=x+2e^{2x}+C$}
	{\True $\displaystyle\int f(x) \mathrm{\,d}x=x+\dfrac{1}{2}e^{2x}+C$}
	{$\displaystyle\int f(x) \mathrm{\,d}x=x+e^{2x}+C$}
	\loigiai{
		Ta có $\displaystyle\int (1+e^{2x}) \mathrm{\,d}x=x+\dfrac{1}{2}e^{2x}+C$.
	}
\end{ex}
\Closesolutionfile{ans}
\indapan{6}{ans/ans-C4B1CD1-LC}
% \TNTF
\Opensolutionfile{ans}[ans/ans-C4B1CD1-DS]
\begin{ex}%[2D4N2-4]
	Các mệnh đề sau đây \textbf{đúng} hay \textbf{sai}?
	\choiceTF
	{$\displaystyle\int \dfrac{1}{x} \mathrm{\,d}x=\ln x+C$}
	{\True $\displaystyle\int \dfrac{1}{\cos^2x} \mathrm{\,d}x=\tan x+C$}
	{\True $\displaystyle\int \sin x \mathrm{\,d}x=-\cos x+C$}
	{\True $\displaystyle\int e^x \mathrm{\,d}x=e^x+C$}
	\loigiai{
		\begin{itemchoice}
			\itemch Ta có $\displaystyle\int \dfrac{1}{x} \mathrm{\,d}x=\ln \left|x\right|+C$.
			\itemch Ta có $\displaystyle\int \dfrac{1}{\cos^2x} \mathrm{\,d}x=\tan x+C$
			\itemch Ta có $\displaystyle\int \sin x \mathrm{\,d}x=-\cos x+C$.
			\itemch Ta có $\displaystyle\int e^x \mathrm{\,d}x=e^x+C$.
		\end{itemchoice}
	}
\end{ex}

\begin{ex}%[2D4N2-4]
	Các mệnh đề sau đây \textbf{đúng} hay \textbf{sai}?
	\choiceTF
	{\True $\displaystyle\int \cos x \mathrm{\,d}x=\sin x+C$}
	{\True $\displaystyle\int x^e \mathrm{\,d}x=\dfrac{x^{e+1}}{e+1}+C$}
	{\True $\displaystyle\int \dfrac{1}{x} \mathrm{\,d}x=\ln \left|x\right|+C$}
	{$\displaystyle\int e^x \mathrm{\,d}x=\dfrac{e^{x+1}}{x+1}+C$}
	\loigiai{
		\begin{itemchoice}
			\itemch Ta có $\displaystyle\int \cos x \mathrm{\,d}x=\sin x+C$.
			\itemch Ta có $\displaystyle\int x^e \mathrm{\,d}x=\dfrac{x^{e+1}}{e+1}+C$
			\itemch Ta có $\displaystyle\int \dfrac{1}{x} \mathrm{\,d}x=\ln \left|x\right|+C$.
			\itemch Ta có $\displaystyle\int e^x \mathrm{\,d}x=e^x+C$.
		\end{itemchoice}
	}
\end{ex}

\begin{ex}%[2D4H2-4]
	Các mệnh đề sau đây \textbf{đúng} hay \textbf{sai}?
	\choiceTF
	{$\displaystyle\int 2^x \mathrm{\,d}x=2^x\ln 2+C$}
	{\True $\displaystyle\int e^{2x} \mathrm{\,d}x=\dfrac{e^{2x}}{2}+C$}
	{$\displaystyle\int e^x(e^x-1) \mathrm{\,d}x=\dfrac{1}{2}e^{2x}+e^x+C$}
	{\True $\displaystyle\int e^{3x}\cdot 3^x \mathrm{\,d}x=\dfrac{(3e^{3})^x}{3+\ln 3}+C$}
	\loigiai{
		\begin{itemchoice}
			\itemch Ta có $\displaystyle\int 2^x \mathrm{\,d}x=\dfrac{2^x}{\ln 2}+C$.
			\itemch Ta có $\displaystyle\int e^{2x} \mathrm{\,d}x=\dfrac{e^{2x}}{2}+C$
			\itemch Ta có $\displaystyle\int e^x(e^x-1) \mathrm{\,d}x=\displaystyle\int (e^{2x}-e^x) \mathrm{\,d}x=\dfrac{1}{2}e^{2x}-e^x+C$.
			\itemch Ta có $\displaystyle\int e^{3x}\cdot 3^x \mathrm{\,d}x=\displaystyle\int (3e^{3})^x \mathrm{\,d}x=\dfrac{(3e^{3})^x}{\ln (3e^3)}+C=\dfrac{(3e^{3})^x}{3+\ln (3)}+C$.
		\end{itemchoice}
	}
\end{ex}
\Closesolutionfile{ans}
\indapan{3}{ans/ans-C4B1CD1-DS}
% \TNSA
\Opensolutionfile{ans}[ans/ans-C4B1CD1-KQ]
\begin{ex}%[2D4H2-4]
	Biết rằng $\displaystyle\int (2^x+3^x) \mathrm{\,d}x=\dfrac{2^x}{\ln a}+\dfrac{3^x}{\ln b}+C,\,a,b\in \mathbb{Z}$. Tính $P=a+b$.
	\shortans[4]{$5$}
	\loigiai{
		Ta có $\displaystyle\int (2^x+3^x) \mathrm{\,d}x=\dfrac{2^x}{\ln 2}+\dfrac{3^x}{\ln 3}+C$.\\
		Do đó $a=2$, $b=3\Rightarrow P=a+b=2+3=5$.
	}
\end{ex}

\begin{ex}%[2D4H2-4]
	Cho $\displaystyle\int e^{3x+2024} \mathrm{\,d}x=\dfrac{a}{b}e^{cx+d}+C$ với $a,b,c,d\in \mathbb{Z}$ và $\dfrac{a}{b}$ là phân số tối giãn . Tính giá trị của biểu thức $P=a+b-c+d$.
	\shortans[4]{$2025$}
	\loigiai{
		Ta có $\displaystyle\int e^{3x+2024} \mathrm{\,d}x=\dfrac{1}{3}e^{3x+2024}+C$.\\
		Do đó $a=1$, $b=3$, $c=3$, $d=2024\Rightarrow P=a+b-c+d=1+3-3+2024=2025$.
	}
\end{ex}

\begin{ex}%[2D4H2-4]
	Biết rằng $\displaystyle\int 3^{x+2}\cdot 2^{2x+1} \mathrm{\,d}x=\dfrac{a\cdot 12^x}{b\ln 2+c\ln 3}+C$ với $a,b,c\in \mathbb{Z}$. Tính giá trị của biểu thức $P=\dfrac{a}{b+c}$.
	\shortans[4]{$6$}
	\loigiai{
		Ta có $\displaystyle\int 3^{x+2}\cdot 2^{2x+1} \mathrm{\,d}x=\displaystyle\int 3^2\cdot 3^x\cdot 2\cdot4^x \mathrm{\,d}x=\displaystyle\int 18\cdot 12^x \mathrm{\,d}x=18\cdot \dfrac{12^x}{\ln 12}+C=\dfrac{18\cdot 12^x}{2\ln 2+\ln 3}+C$.\\
		Do đó $a=18$, $b=2$, $c=1\Rightarrow P=\dfrac{a}{b+c}=\dfrac{18}{2+1}=6$.
	}
\end{ex}

\begin{ex}%[2D4H2-4]
	Biết rằng $\displaystyle\int (3^{x}+5^{x})^2\mathrm{\,d}x=\dfrac{9^x}{a\ln 3}+\dfrac{30^x}{b\ln 5+c\ln 2+d\ln 3}+\dfrac{25^x}{e\ln 5}+C$. Tính giá trị của biểu thức $P=a+b+c+d+e$.
	\shortans[4]{$7$}
	\loigiai{
		\begin{eqnarray*}
			\displaystyle\int (3^{x}+5^{x})\mathrm{\,d}x&=&\displaystyle\int (9^{x}+30^{x}+25^{x})\mathrm{\,d}x\\
			&=&
			\dfrac{9^x}{\ln 9}+\dfrac{30^x}{\ln 30+\ln 25}+C\\
			&=&\dfrac{9^x}{2\ln 3}+\dfrac{30^x}{\ln 5+\ln 2+\ln 3}+\dfrac{25^x}{2\ln 5}+C.
		\end{eqnarray*}
		Do đó $a=2$, $b=c=d=1$, $e=2\Rightarrow P=a+b+c+d+e=7$.
	}
\end{ex}

\begin{ex}%[2D4H2-4]
	Cho $\displaystyle\int \dfrac{e^{3x}+1}{e^x+1}\mathrm{\,d}x=\dfrac{a}{b}e^{2x}+ce^x+dx+C$ với $a,b,c,d\in \mathbb{Z}$ và $\dfrac{a}{b}$ là phân số tối giãn. Tính giá trị của biểu thức $P=a^2+b^2+c^2+d^2$.
	\shortans[4]{$7$}
	\loigiai{
		Ta có $\displaystyle\int \dfrac{e^{3x}+1}{e^x+1}\mathrm{\,d}x=\displaystyle\int \dfrac{(e^{x}+1)(e^{2x}-e^x+1)}{e^x+1}\mathrm{\,d}x=\displaystyle\int (e^{2x}-e^x+1)\mathrm{\,d}x=\dfrac{1}{2}e^{2x}-e^x+x+C$.\\
		Do đó $a=d=1$, $b=2$, $c=-1\Rightarrow P=a^2+b^2+c^2+d^2=7$.
	}
\end{ex}

\begin{ex}%[2D4H2-4]
	Biết rằng $\displaystyle\int (e^x+e^{-x})^2\mathrm{\,d}x=\dfrac{1}{m}e^{2x}+\dfrac{1}{n}e^{-2x}+px+C$ với $m,m,p\in \mathbb{Z}$. Tính giá trị của biểu thức $P=m+n+p$.
	\shortans[4]{$2$}
	\loigiai{
		Ta có $\displaystyle\int (e^x+e^{-x})^2\mathrm{\,d}x=\displaystyle\int (e^{2x}+e^{-2x}+2)\mathrm{\,d}x=\dfrac{1}{2}e^{2x}-\dfrac{1}{2}e^{2x}+2x+C$.\\
		Do đó $m=p=2$, $n=-2\Rightarrow P=m+n+p=2$.
	}
\end{ex}

\begin{ex}%[2D4H2-4]
	Biết rằng $\displaystyle\int \dfrac{e^{2x}-1}{1-e^{-x}}\mathrm{\,d}x=\dfrac{1}{m}e^{nx}+pe^x+C$ với $m,m,p\in \mathbb{Z}$. Tính giá trị của biểu thức $P=m+n-p$.
	\shortans[4]{$5$}
	\loigiai{
		Ta có $\displaystyle\int \dfrac{e^{2x}-1}{1-e^{-x}}\mathrm{\,d}x=\displaystyle\int \dfrac{e^x(e^x-1)(e^x+1)}{e^x-1}\mathrm{\,d}x=\displaystyle\int e^x(e^x-1) \mathrm{\,d}x$\\$=\displaystyle\int (e^{2x}-e^x) \mathrm{\,d}x=\dfrac{1}{2}e^{2x}-e^x+C$.\\
		Do đó $m=n=2$, $p=-1\Rightarrow P=m+n-p=5$.
	}
\end{ex}

\begin{ex}%[2D4H2-4]
	Biết rằng $F(x)=(ax+b)\cdot e^x$ là một nguyên hàm của hàm số $f(x)=(4x-1)\cdot e^x$. Tính giá trị biểu thức $P=a+b$.
	\shortans[4]{$-1$}
	\loigiai{
		Ta có $F'(x)=a\cdot e^x+(ax+b)\cdot e^x=e^x(ax+a+b)$.\\
		Mà $F'(x)=f(x)\Rightarrow \heva{&a=4\\&a+b=-1}\Rightarrow \heva{&a=4\\&b=-5.}$\\
		Vậy $P=a+b=-1$.
	}
\end{ex}

\begin{ex}%[2D4H2-4]
	Biết rằng $F(x)=8e^x+\dfrac{na^x}{\ln a}+p\cos x$ (với $m,n,p\in \mathbb{Z}$) là một nguyên hàm của hàm số $f(x)=me^x+2a^x-2\sin x$. Tính giá trị của biểu thức $P=m+n+p$.
	\shortans[4]{$12$}
	\loigiai{
		Ta có $F'(x)=8e^x+\dfrac{na^x}{\ln a}\cdot \ln a-p\sin x=8e^x+na^x-p\sin x$.\\
		Mà $F'(x)=f(x)\Rightarrow m=8, n=2, p=2$.\\
		Vậy $P=m+n+p=12$.
	}
\end{ex}

\begin{ex}%[2D4H2-4]
	Biết rằng  $F(x)=(ax^2+bx+c)e^{-2x}$ (với $a,b,c\in \mathbb{R}$) là một nguyên hàm của hàm số $f(x)=(-2x^2+8x-7)e^{-2x}$. Tính giá trị biểu thức $P=a+b+c$.
	\shortans[4]{$-7$}
	\loigiai{
		Ta có $F'(x)=(2ax+b)e^{-2x}-2(ax^2+bx+c)e^{-2x}=\left[-2ax^2+2(a-b)x+b-2c \right]e^{-2x}$.\\
		Mà $F'(x)=f(x)\Rightarrow \heva{&-2a=-2\\&2(a-b)=8\\&b-2c=7}\Rightarrow \heva{&a=1\\&b=-3\\&c=-5.}$\\
		Vậy $P=a+b+c=1-3-5=-7$.
	}
\end{ex}
\Closesolutionfile{ans}
\indapan{6}{ans/ans-C4B1CD1-KQ}

% % \subsection{NGUYÊN HÀM CÓ ĐIỀU KIỆN}
\begin{dang}{Tìm nguyên hàm khi biết giá trị nguyên hàm}
	Phương pháp: Tìm $F(x)=\int f(x)\mathrm{\,d}x$. Sau đó dựa vào $F(x_0)=a$ để suy ra $C$.
\end{dang}
\Opensolutionfile{ans}[ans/ans-C4B1CD2-LC]
% \TN
\begin{ex}%[2D4H2-2]
Hàm số $F(x)$ là một nguyên hàm của hàm số $f(x)=\dfrac{1}{x}$ trên $(-\infty;0)$ thỏa mãn $F(-2)=0$. Khẳng định nào sau đây \textbf{đúng}?
\choice
{\True $F(x)=\ln \left(-\dfrac{x}{2} \right),\,\forall x\in (-\infty;0)$}
{$F(x)=\ln \left|x\right|+C,\,\forall x\in (-\infty;0)$ với $C$ là một số thực bất kì}
{$F(x)=\ln \left|x\right|+\ln 2,\,\forall x\in (-\infty;0)$}
{$F(x)=\ln \left(-x\right)+C,\,\forall x\in (-\infty;0)$ với $C$ là một số thực bất kì}
\loigiai{
Ta có $F(x)=\displaystyle\int \dfrac{1}{x}\mathrm{\,d}x=\ln \left|x\right|+C=\ln (-x)+C,\,\forall x\in (-\infty;0)$.\\
Lại có $F(-2)=0\Rightarrow \ln 2+C=0\Rightarrow C=-\ln 2$.\\
Do đó $F(x)=\ln (-x)-\ln 2=\ln \left(-\dfrac{x}{2}\right)$.
}
\end{ex}

\begin{ex}%[2D4H2-4]
Biết $F(x)$ là một nguyên hàm của hàm số $f(x)=e^{2x}$ và $F(0)=0$. Giá trị của $F(\ln 3)$ bằng
\choice
{$2$}
{$6$}
{$8$}
{\True $4$}
\loigiai{
Ta có $F(x)=\displaystyle\int e^{2x}\mathrm{\,d}x=\dfrac{1}{2}e^{2x}+C$.\\
Lại có $F(0)=0\Rightarrow \dfrac{1}{2}+C=0\Rightarrow C=-\dfrac{1}{2}$.\\
Do đó $F(\ln 3)=\dfrac{1}{2}e^{2\ln 3}-\dfrac{1}{2}=4$.
}
\end{ex}

\begin{ex}%[2D4H2-4]
Cho $F(x)$ là một nguyên hàm của $f(x)=2^x+x+1$. Biết $F(0)=1$. Giá trị của $F(-1)$ bằng
\choice
{$F(-1)=\dfrac{1}{2\ln 2}$}
{\True $F(-1)=\dfrac{1}{2}-\dfrac{1}{2\ln 2}$}
{$F(-1)=1+\dfrac{1}{2\ln 2}$}
{$F(-1)=\dfrac{1}{2}-\dfrac{1}{\ln 2}$}
\loigiai{
Ta có $F(x)=\displaystyle\int (2^x+x+1)\mathrm{\,d}x=\dfrac{2^x}{\ln 2}+\dfrac{x^2}{2}+x+C$.\\
Lại có $F(0)=1\Rightarrow \dfrac{1}{\ln 2}+C=1\Rightarrow C=1-\dfrac{1}{\ln 2}$.\\
Do đó $F(-1)=\dfrac{1}{2\ln 2}+\dfrac{1}{2}-1+1-\dfrac{1}{\ln 2}=\dfrac{1}{2}-\dfrac{1}{2\ln 2}$.
}
\end{ex}

\begin{ex}%[2D4H2-3]
Tìm nguyên hàm $F(x)$ của hàm số $f(x)=\sin x+\cos x$ thoả mãn $F\left(\dfrac{\pi}{2}\right)=2$.
\choice
{$F(x)=-\cos x+\sin x+3$}
{$F(x)=-\cos x+\sin x-1$}
{\True $F(x)=-\cos x+\sin x+1$}
{$F(x)=\cos x-\sin x+3$}
\loigiai{
Ta có $F(x)=\displaystyle\int (\sin x+\cos x)\mathrm{\,d}x=-\cos x+\sin x+C$.\\
Lại có $F\left(\dfrac{\pi}{2}\right)=2\Rightarrow -\cos\dfrac{\pi}{2}+\sin\dfrac{\pi}{2}+C=2\Rightarrow C=1$.\\
Do đó $F(x)=-\cos x+\sin x+1$.
}
\end{ex}

\begin{ex}%[2D4H2-4]
Cho $F(x)$ là một nguyên hàm của hàm số $f(x)=e^x+2x$ thỏa mãn $F(0)=\dfrac{3}{2}$. Tìm $F(x)$.
\choice
{\True $F(x)=e^x+x^2+\dfrac{1}{2}$}
{$F(x)=e^x+x^2+\dfrac{5}{2}$}
{$F(x)=e^x+x^2+\dfrac{3}{2}$}
{$F(x)=e^x+x^2-\dfrac{1}{2}$}
\loigiai{
Ta có $F(x)=\displaystyle\int (e^x+2x)\mathrm{\,d}x=e^x+x^2+C$.\\
Lại có $F(0)=\dfrac{3}{2}\Rightarrow 1+C=\dfrac{3}{2}\Rightarrow C=\dfrac{1}{2}$.\\
Do đó $F(x)=e^x+x^2+\dfrac{1}{2}$.
}
\end{ex}

\begin{ex}%[2D4H2-2]
Cho hàm số $f(x)=\heva{&2x-1&\text{khi}\quad&x\ge 1\\&3x^2-2&\text{khi}\quad&x<1}$, giả sử $F$ là nguyên hàm của  $f$ trên $\mathbb{R}$ thỏa mãn $F(0)=2$. Giá trị của $F(-1)+2F(2)$ bằng
\choice
{\True $9$}
{$15$}
{$11$}
{$6$}
\loigiai{
Ta có $\displaystyle\int (2x-1)\mathrm{\,d}x=x^2-x+C_1$ và $\displaystyle\int (3x^2-2)\mathrm{\,d}x=x^3-2x+C_2$.\\
Suy ra $F(x)=\displaystyle\int f(x)\mathrm{\,d}x=\heva{&x^2-x+C_1&\text{khi}\quad&x\ge 1\\&x^3-2x+C_2&\text{khi}\quad&x<1.}$ 
Lại có $F(0)=2\Rightarrow C_2=2$.\\
Mặt khác hàm số $F$ là nguyên hàm của $f$ trên $\mathbb{R}$ nên $y=F(x)$ liên tục tại $x=1$.\\
Suy ra  $\lim\limits_{ x\to 1^{+}} F(x)=\lim\limits_{ x\to 1^{-}} F(x)\Rightarrow C_1=1$.\\
Khi đó ta có $F(x)=\heva{&x^2-x+1&\text{khi}\quad&x\ge 1\\&x^3-2x+2&\text{khi}\quad&x<1}\Rightarrow \heva{&F(-1)=3\\&F(2)=3.}$  \\
Vậy $F(-1)+2F(2)=9$.  
}
\end{ex}

\Opensolutionfile{ans}[ans/ans-C4B1CD2-CAU7_8-LC]
\setcounter{ex}{6}
\begin{ex}%[2D4H1-2]
	Cho hàm số $f(x)=\heva{&2x+3 &\text{khi } &x\ge 1\\ &3x^2+2 &\text{khi } &x<1.}$ Giả sử $F$ là nguyên hàm của hàm số $f$ trên $\mathbb{R}$ thỏa mãn $F(0)=2$. Giá trị của $F(-1)+2F(2)$ bằng
	\choice
	{$23$}
	{$11$}
	{$10$}	 
	{\True $21$}
	\loigiai{
		Khi $x\ge 1$ thì $F(x)=\displaystyle\int f(x)\mathrm{\,d}x=\displaystyle\int (2x+3)\mathrm{\,d}x=x^2+3x+\mathrm{C}_1$.\\
		Khi $x<1$ thì $F(x)=\displaystyle\int f(x)\mathrm{\,d}x=\displaystyle\int (3x^2+2)\mathrm{\,d}x=x^3+2x+C_2$.\\
		Theo giả thiết $F(0)=2 \Rightarrow C_2=2$.\\
		Ta có $\lim\limits_{x \to 1^+} f(x)=\lim\limits_{x \to 1^-} f(x)=f(1)=5$ nên hàm số $f(x)$ liên tục tại $x=1$.\\
		Suy ra hàm số $f(x)$ liên tục trên $\mathbb{R}$.\\
		Do đó hàm số $F(x)$ liên tục trên $\mathbb{R} \Rightarrow \lim\limits_{x \to 1^+} F(x)=\lim\limits_{x \to 1^-} F(x) \Rightarrow C_1+4=C_2+3 \Rightarrow C_1=1$.\\
		Vậy $F(-1)+2F(2)=-3+C_2+2(10+C_1)=21$.
	}
\end{ex}
\begin{ex}%[2D4H1-2]
	Cho hàm số $f(x)=\heva{&2x+2 &\text{khi } &x\ge 1\\ &3x^2+1 &\text{khi } &x<1.}$ Giả sử $F$ là nguyên hàm của hàm số $f$ trên $\mathbb{R}$ thỏa mãn $F(0)=2$. Giá trị của $F(-1)+2F(2)$ bằng
	\choice
	{\True $18$}
	{$20$}
	{$9$}	 
	{$24$}
	\loigiai{
		$F$ là nguyên hàm của $f$ trên $\mathbb{R}$ nên $F(x)=\heva{&x^2+2x+C_1 &\text{khi } &x\ge 1\\ &x^3+x+C_2 &\text{khi } &x<1.}$\\
		Ta có $F(0)=2 \Rightarrow C_2=2$. \quad $(1)$\\
		Do $F$ liên tục tại $x=1$ nên $\lim\limits_{x \to 1^+} F(x)=\lim\limits_{x \to 1^-} F(x)=F(1)$.\\
		$\Leftrightarrow C_1+3=C_2+2 \mathop  \Leftrightarrow \limits^{(1)} C_1+3=4 \Leftrightarrow C_1=1$.\\
		Do đó $F(x)=\heva{&x^2+2x+1 &\text{khi } &x\ge 1\\ &x^3+x+2 &\text{khi } &x<1.}$\\
		Suy ra $F(-1)+2F(2)=18$.
	}
\end{ex}

\begin{ex}%[2D4H1-2]
	Cho hàm số $y=f(x)$ có đạo hàm là $f'(x)=12x^2+2, \forall x\in \mathbb{R}$ và $f(1)=3$. Biết $F(x)$ là nguyên hàm của $f(x)$ thỏa mãn $F(0)=2$, khi đó $F(1)$ bằng
	\choice
	{$-3$}
	{\True $1$}
	{$2$}
	{$7$}
	\loigiai{
		Ta có $f'(x)=12x^2+2, \forall x\in \mathbb{R} \Rightarrow f(x)=4x^3+2x+C_1$.\\
		Mà $f(1)=3\Rightarrow 3=6+C_1\Rightarrow C_1=-3\Rightarrow f(x)=4x^3+2x-3\Rightarrow F(x)=x^4+x^2-3x+C_2$.\\
		Lại có $F(0)=2\Rightarrow C_2=2\Rightarrow F(x)=x^4+x^2-3x+2$.\\
		Do đó $F(1)=1$.\\
		\textbf{Cách khác:}\\
		Ta có $F(1)=\displaystyle \int\limits_0^1 {f(x)\mathrm{\,d}x}+F(0)=\displaystyle \int\limits_0^1{(4x^3+2x-3)\mathrm{\,d}x}+2=-1+2=1$.}
\end{ex}
\begin{ex}%[2D4H1-3]
	Cho hàm số $f(x)$ thỏa mãn $f'(x)=3-5\sin x$ và $f(0)=10$. Mệnh đề nào dưới đây \textbf{đúng}?
	\choice
	{$f(x)=3x-5\cos x+15$}
	{$f(x)=3x-5\cos x+2$}
	{\True $f(x)=3x+5\cos x+5$}
	{$f(x)=3x+5\cos x+2$}
	\loigiai{
		Ta có $f(x)=\displaystyle \int (3-5\sin x)\mathrm{\,d}x=3x+5\cos x+C$.\\
		Theo giả thiết $f(0)=10$ nên $5+C=10\Rightarrow C=5$.\\
		Vậy $f(x)=3x+5\cos x+5$.
	}
\end{ex}
\begin{ex}%[2D4H1-4]
	Hàm số $f(x)$ có đạo hàm liên tục trên $\mathbb{R}$ và $f'(x)=2\mathrm{e}^{2x}+1, \forall x; f(0)=2$. Hàm $f(x)$ là
	\choice
	{$y=2\mathrm{e}^x+2x$}
	{$y=2\mathrm{e}^x+2$}
	{$y=\mathrm{e}^{2x}+x+2$}
	{\True $y=\mathrm{e}^{2x}+x+1$}
	\loigiai{
		Ta có $\displaystyle \int f'(x)\mathrm{\,d}x=\displaystyle \int(2\mathrm{e}^{2x}+1)\mathrm{\,d}x=\mathrm{e}^{2x}+x+C$.\\
		Suy ra $f(x)=\mathrm{e}^{2x}+x+C$.\\
		Theo bài ra ta có $f(0)=2\Rightarrow 1+C=2\Leftrightarrow C=1$.\\
		Vậy $f(x)=\mathrm{e}^{2x}+x+1$.
	}
\end{ex}
\begin{ex}%[2D4H1-3]
	Cho hàm số $f(x)$ thỏa mãn $f'(x)=2-5\sin x$ và $ f(0)=10$. Mệnh đề nào dưới đây \textbf{đúng}?
	\choice
	{$f(x)=2x+5\cos x+3$}
	{$f(x)=2x-5\cos x+15$}
	{\True $f(x)=2x+5\cos x+5$}
	{$f(x)=2x-5\cos x+10$}
	\loigiai{
		Ta có $f(x)=\displaystyle \int f'(x)\mathrm{\,d}x=\displaystyle\int (2-5\sin x)\mathrm{\,d}x=2x+5\cos x+C$.\\
		Mà $f(0)=10$ nên $5+C=10\Rightarrow C=5$.\\
		Vậy $f(x)=2x+5\cos x+5$.}
\end{ex}
\begin{ex}%[2D4V1-2]
	Cho hàm số $f(x)$ thỏa mãn $f'(x)=ax^2+\dfrac{b}{x^3}$, $f'(1)=3$, $f(1)=2$, $f\left(\dfrac{1}{2}\right)=-\dfrac{1}{12}$. Khi đó $2a+b$ bằng
	\choice
	{$-\dfrac{3}{2}$}
	{$0$}
	{\True $5$}
	{$\dfrac{3}{2}$}
	\loigiai{
		Ta có $f'(1)=3\Rightarrow a+b=3. \quad (1)$\\
		Hàm số có đạo hàm liên tục trên khoảng $(0;+\infty)$, các điểm $x=1$, $x=\dfrac{1}{2}$ đều thuộc $(0;+\infty)$ nên\\
		$f(x)=\displaystyle \int f'(x)\mathrm{\,d}x=\displaystyle \int (ax^2+\dfrac{b}{x^3})\mathrm{\,d}x=\dfrac{ax^3}{3}-\dfrac{b}{2x^2}+C$.\\
		\begin{itemize}
			\item $f(1)=2\Rightarrow \dfrac{a}{3}-\dfrac{b}{2}+C=2. \quad (2)$
			\item $f\left(\dfrac{1}{2}\right)=-\dfrac{1}{12}\Rightarrow \dfrac{a}{24}-2b+C=-\dfrac{1}{12}. \quad (3)$
		\end{itemize}
		Từ $(1)$, $(2)$ và $(3)$ ta được hệ phương trình $\heva{ &a+b=3\\ &\dfrac{a}{3}-\dfrac{b}{2}+C=2\\ &\dfrac{a}{24}-2b+C=-\dfrac{1}{12}}\Leftrightarrow \heva{&a=2\\ &b=1\\ &C=\dfrac{11}{6}.}$\\
		Vậy $2a+b=2\cdot 2+1=5$.
	}
\end{ex}
\begin{ex}%[2D4V1-2]
	Tìm một nguyên hàm $F(x)$ của hàm số $f(x)=ax+\dfrac{b}{x^2} \quad (x\ne  0)$, biết rằng $F(-1)=1$, $F(1)=4$, $f(1)=0$.
	\choice
	{$F(x)=\dfrac{3}{2}x^2+\dfrac{3}{4x}-\dfrac{7}{4}$}
	{$F(x)=\dfrac{3}{4}x^2-\dfrac{3}{2x}-\dfrac{7}{4}$}
	{\True $F(x)=\dfrac{3}{4}x^2+\dfrac{3}{2x}+\dfrac{7}{4}$}
	{$F(x)=\dfrac{3}{2}x^2-\dfrac{3}{2x}-\dfrac{1}{2}$}
	\loigiai{
		Ta có $F(x)=\displaystyle \int f(x)\mathrm{\,d}x=\displaystyle \int \left(ax+\dfrac{b}{x^2}\right)\mathrm{\,d}x=\dfrac{1}{2}ax^2-\dfrac{b}{x}+C$.\\
		Theo bài ra $\heva{&F(-1)=1\\ &F(1)=4\\ &f(1)=0}\Leftrightarrow \heva{&\dfrac{1}{2}a+b+C=1\\ &\dfrac{1}{2}a-b+C=4\\ &a+b=0} \Leftrightarrow \heva{&b=-\dfrac{3}{2}\\ &a=\dfrac{3}{2}\\ &C=\dfrac{7}{4}.} $\\
		Vậy $F(x)=\dfrac{3}{4}{x^2}+\dfrac{3}{2x}+\dfrac{7}{4}$.
	}
\end{ex}
\begin{ex}%[2D4V1-2]
	Cho hàm số $f(x)$ xác định trên $\mathbb{R}\setminus \{0\}$ thỏa mãn $f'(x)=\dfrac{x+1}{x^2}$, $f(-2)=\dfrac{3}{2}$ và $f(2)=2\ln 2-\dfrac{3}{2}$. Giá trị của biểu thức $f(-1)+f(4)$ bằng
	\choice
	{$\dfrac{6\ln 2-3}{4}$}
	{$\dfrac{6\ln 2+3}{4}$}
	{\True $\dfrac{8\ln 2+3}{4}$}
	{$\dfrac{8\ln 2-3}{4}$}
	\loigiai{
		Có $f(x)=\displaystyle \int f'(x)\mathrm{\,d}x=\displaystyle \int \dfrac{x+1}{x^2}\mathrm{\,d}x=\ln x-\dfrac{1}{x}+C$.\\
		Tìm được $f(x)=\heva{&\ln |x|-\dfrac{1}{x}+C_1 &\text{khi } &x<0\\ &\ln |x|-\dfrac{1}{x}+C_2 &\text{khi } &x>0.}$\\
		Do $f(-2)=\dfrac{3}{2} \Rightarrow \ln 2+\dfrac{1}{2}+C_1=\dfrac{3}{2} \Rightarrow C_1=1-\ln 2$.\\
		Do $f(2)=2\ln 2-\dfrac{3}{2} \Rightarrow \ln 2-\dfrac{1}{2}+C_2=2\ln 2-\dfrac{3}{2} \Rightarrow C_2=\ln 2-1$.\\
		Suy ra $f(x)=\heva{&\ln |x|-\dfrac{1}{x}+1-\ln 2 &\text{khi } &x<0\\ &\ln |x|-\dfrac{1}{x}+\ln 2-1 &\text{khi } &x>0.}$\\
		Vậy $f(-1)+f(4)=(2-\ln 2)+\left(\ln 4-\dfrac{1}{4}+\ln 2-1\right)=\dfrac{8\ln 2+3}{4}$.
	}
\end{ex}

\begin{ex}%[2D4V1-2]
	\immini{Cho hàm số $y=f(x)$. Đồ thị của hàm số \break $y=f'(x)$ trên $[-5 ; 3]$ như hình vẽ (phần cong của đồ thị là một phần của parabol \break $y=a x^2+b x+c$). Biết $f(0)=0$, giá trị của $2 f(-5)+3 f(2)$ bằng
		\choice
		{$33$}
		{$\dfrac{109}{3}$}
		{\True $\dfrac{35}{3}$}
		{$11$}
	}{
		\begin{tikzpicture}[scale=0.7, font=\footnotesize, line join=round, line cap=round,>=stealth]
			%Gán số liệu.
			\def\xmin{-6};\def\ymin{-2};\def\xmax{4};\def\ymax{5};
			%Gán tọa độ.
			\coordinate (O) at (0,0);
			%Trục Oxy.
			\draw[->] (\xmin,0)--(\xmax,0) node[below]{$x$};
			\draw[->] (0,\ymin)--(0,\ymax) node[left]{$y$};
			\fill (O) node[below left]{$O$} circle(1pt);
			%Giới hạn đồ thị.
			\clip ({\xmin-0.1},{\ymin-0.1}) rectangle ({\xmax+0.1},{\ymax+0.1});
			\foreach \x in {-5,-4,-1,1,2,3}{
					\fill (\x,0) node[below]{$\x$} circle(1pt);
				}
			\foreach \y in {-1,2,3,4}{
					\fill (0,\y) node[left]{$\y$} circle(1pt);
				}
			\draw (-5,-1)--(-4,2)--(-1,0);
			\draw[thick,samples=100] plot[domain=-1:3.5](\x,{-(\x)^2+2*\x+3});
			\draw[dashed] (-5,0)|-(0,-1) (-4,0)|-(0,2) (1,0)|-(0,4) (2,0)|-(0,3);
		\end{tikzpicture}
	}
	\loigiai{
		Parabol $y=a x^2+b x+c$ qua các điểm $(2 ; 3)$, $(1 ; 4)$, $(0 ; 3)$, $(-1 ; 0)$, $(3 ; 0)$ nên xác định được $y=-x^2+2 x+3$, $\forall x \geq-1$ suy ra $f(x)=-\dfrac{x^3}{3}+x^2+3 x+C_1$.\\
		Mà $f(0)=0 \Rightarrow C_1=0$, $f(x)=-\dfrac{x^3}{3}+x^2+3 x$.\\
		Có $f(-1)=-\dfrac{5}{3}$, $ f(2)=\dfrac{22}{3}$.\quad $(1)$\\
		Đồ thị $f'(x)$ trên đoạn $[-4 ;-1]$ qua các điểm $(-4 ; 2)$, $(-1 ; 0)$.\\
		Nên $f'(x)=-\dfrac{2}{3}(x+1) \Rightarrow f(x)=-\dfrac{2}{3}\left(\dfrac{x^2}{2}+x\right)+C_2$.\\
		Mà $f(-1)=-\dfrac{5}{3} \Leftrightarrow C_2=-\dfrac{5}{3}+\dfrac{2}{3}\left(-\dfrac{1}{2}\right)=-2 \Rightarrow f(x)=-\dfrac{2}{3}\left(\dfrac{x^2}{2}+x\right)-2$, hay $f(-4)=-\dfrac{14}{3}$.\\
		Đồ thị $f'(x)$ trên đoạn $[-5 ;-4]$ qua các điểm $(-4 ; 2)$, $(-5 ;-1)$.\\
		Nên $f'(x)=3 x+14 \Rightarrow f(x)=\dfrac{3 x^2}{2}+14 x+C_3$.\\
		Mà $f(-4)=-\dfrac{14}{3} \Leftrightarrow \dfrac{3 \cdot(-4)^2}{2}+14 \cdot(-4)+C_3=-\dfrac{14}{3}$ suy ra $C_3=\dfrac{82}{3}$.\\
		Ta có $f(x)=\dfrac{3 x^2}{2}+14 x+\dfrac{82}{3} \Rightarrow f(-5)=-\dfrac{31}{6}$.\quad $(2)$\\
		Từ $(1)$ và $(2)$ ta được $2 f(-5)+3 f(2)=-\dfrac{31}{3}+22=\dfrac{35}{3}$.
	}
\end{ex}
\begin{ex}%[2D4H1-4]
	Cho hàm số $f(x)=2x+\mathrm{e}^x$. Một nguyên hàm $F(x)$ của hàm số $f(x)$ thỏa mãn $F(0)=2024$. Biết $F(x)=ax^2+b\mathrm{e}^x+c$, giá trị của $a+b+c$ là
	\shortans{$2025$}
	\loigiai{
		Ta có $\displaystyle\int f(x)\mathrm{\,d}x=\displaystyle\int (2x+\mathrm{e}^x)\mathrm{\,d}x=x^2+\mathrm{e}^x+C$.\\
		Có $F(x)$ là một nguyên hàm của $f(x)$ và $F(0)=2024$.\\
		Tìm được $\heva{&F(x)=x^2+\mathrm{e}^x+C\\ &F(0)=2024} \Rightarrow 1+C=2024 \Leftrightarrow C=2023$.\\
		Suy ra $F(x)=x^2+\mathrm{e}^x+2023$.\\
		Vậy $a+b+c=2025$.
	}
\end{ex}
% \begin{ex}%[2D4H1-3]
% 	Cho $F(x)$ là một nguyên hàm của hàm số $f(x)=\sin x+1$ biết $F\left( \dfrac{\pi}{6}\right) =0$. Tính giá trị của $F(\pi)$. (Làm tròn đến chữ số thập phân thứ hai)
% 	\shortans{$4{,}48$}
% 	\loigiai{
% 		$F(x)=\displaystyle \int(\sin x+1)\mathrm{\,d}x=x-\cos x+C$.\\
% 		Do $F\left( \dfrac{\pi}{6}\right) =0 \Rightarrow \dfrac{\pi}{6}-\cos \left( \dfrac{\pi}{6}\right) +C=0\Leftrightarrow C=\dfrac{\sqrt{3}}{2}-\dfrac{\pi}{6}$.\\
% 		Suy ra $F(x)=x-\cos x+\dfrac{\sqrt{3}}{2}-\dfrac{\pi}{6}$.\\
% 		Vậy $F(\pi)=4{,}48$. 
% 	}
% \end{ex}
% \begin{ex}%[2D4H1-2]
% 	Cho $F(x)$ là một nguyên hàm của $f(x)=(5x+3)^5$. Biết $F(1)=0$. Tính giá trị của $\sqrt{|F(0)|}$. (Làm tròn đến chữ số thập phân thứ nhất)
% 	\shortans{$93{,}3$}
% 	\loigiai{
% 		Ta có $F(x)=\displaystyle \int f(x)\mathrm{\,d}x=\displaystyle \int (5x+3)^5\mathrm{\,d}x=\dfrac{(5x+3)^6}{30}+C$.\\
% 		Do $F(1)=0\Rightarrow 0=\dfrac{(5\cdot 1+3)^6}{30}+C\Rightarrow C=-\dfrac{131072}{15}$.\\
% 		Suy ra $F(x)=\dfrac{(5x+3)^6}{30}-\dfrac{131072}{15}$.\\
% 		Do đó $F(0)=\dfrac{(5\cdot 0+3)^6}{30}-\dfrac{131072}{15}=-\dfrac{52283}{6}$.\\
% 		Vậy $\sqrt{|F(0)|}=93{,}3$.
% 	}
% \end{ex}
% \begin{ex}%[2D4H1-2]
% 	Cho $F(x)$ là một nguyên hàm của $f(x)=x^3-4x+5$. Biết $F(1)=3$. Tính $|F(0)|$.
% 	\shortans{$0{,}25$}
% 	\loigiai{
% 		Ta có $F(x)=\displaystyle \int f(x)\mathrm{\,d}x=\displaystyle \int(x^3-4x+5)\mathrm{\,d}x=\dfrac{x^4}{4}-2x^2+5x+C$.\\
% 		Do $F(1)=3\Rightarrow 3=\dfrac{1^4}{4}-2\cdot 1^2+5\cdot 1+C\Rightarrow C=-\dfrac{1}{4}$.\\
% 		Suy ra $F(x)=\dfrac{x^4}{4}-2x^2+5x-\dfrac{1}{4}$.\\
% 		Vậy $|F(0)|=0{,}25$.
% 	}
% \end{ex}
% \begin{ex}%[2D4H1-3]
% 	Cho $F(x)$ là một nguyên hàm của $f(x)=3-5\cos x$. Biết $F(\pi )=2$. Tính $F\left(\dfrac{\pi}{2}\right)$. (Làm tròn đến chữ số thập phân thứ nhất)
% 	\shortans{$-7{,}7$}
% 	\loigiai{
% 		Ta có $F(x)=\displaystyle \int f(x)\mathrm{\,d}x=\displaystyle \int(3-5\cos x)\mathrm{\,d}x=3x-5\sin x+C$.\\
% 		Do $F(\pi )=2\Rightarrow 2=3\pi -5\sin \pi+C\Rightarrow C=-3\pi +2$.\\
% 		Suy ra $F(x)=3x-5\sin x-3\pi +2$.\\
% 		Vậy $F\left(\dfrac{\pi}{2}\right)=-7{,}7$.
% 	}
% \end{ex}
% \begin{ex}%[2D4H1-2]
% 	Cho $F(x)$ là một nguyên hàm của $f(x)=\dfrac{3-5x^2}{x}$. Biết $F(\mathrm{e})=1$. Tính $F(2)$. (Làm tròn đến chữ số thập phân thứ hai)
% 	\shortans{$8{,}55$}
% 	\loigiai{
% 		Hàm số $f(x)=\dfrac{3-5x^2}{x}=\dfrac{3}{x}-5x$.\\
% 		Có $F(x)=\displaystyle \int f(x)\mathrm{\,d}x=\displaystyle \int\left(\dfrac{3}{x}-5x\right)\mathrm{\,d}x=3\ln |x|-\dfrac{5}{2} x^2 +C$.\\
% 		Do $F(\mathrm{e})=1\Rightarrow 1=3\ln |\mathrm{e}|-\dfrac{5}{2} \mathrm{e}^2 +C \Rightarrow C=\dfrac{5}{2} \mathrm{e}^2 -2$.\\
% 		Suy ra $F(x)=3\ln |x|-\dfrac{5}{2} x^2 +\dfrac{5}{2} \mathrm{e}^2 -2$.\\
% 		Vậy $F(2)=8{,}55$.
% 	}
% \end{ex}
% \begin{ex}%[2D4H1-2]
% 	Cho $F(x)$ là một nguyên hàm của $f(x)=\dfrac{x^2+1}{x}$. Biết $F(1)=\dfrac{3}{2}$. Tính $F(-1)$.
% 	\shortans{$1{,}5$}
% 	\loigiai{
% 		Hàm số $f(x)=\dfrac{x^2+1}{x}=x+\dfrac{1}{x}$.\\
% 		Có $F(x)=\displaystyle \int f(x)\mathrm{\,d}x=\displaystyle \int\left(x+\dfrac{1}{x}\right)\mathrm{\,d}x=\dfrac{x^2}{2}+\ln |x|+C$.\\
% 		Do $F(1)=\dfrac{3}{2}\Rightarrow \dfrac{3}{2}=\dfrac{1^2}{2}+\ln |1|+C \Rightarrow C=1$.\\
% 		Suy ra $F(x)=\dfrac{x^2}{2}+\ln |x|+1$.\\
% 		Vậy $F(-1)=\dfrac{(-1)^2}{2}+\ln |-1|+1=\dfrac{3}{2}=1{,}5$.
% 	}
% \end{ex}
% \begin{ex}%[2D4H1-2]
% 	Cho $F(x)$ là một nguyên hàm của hàm số $f(x)=\dfrac{x^3-1}{x^2}$. Biết $F(-2)=0$. Tính giá trị của $F(2)$.
% 	\shortans{$1$}
% 	\loigiai{
% 		Hàm số $f(x)=\dfrac{x^3-1}{x^2}=x-\dfrac{1}{x^2}$.\\
% 		Có $F(x)=\displaystyle \int f(x)\mathrm{\,d}x=\displaystyle \int\left(x-\dfrac{1}{x^2}\right)\mathrm{\,d}x=\dfrac{x^2}{2}+\dfrac{1}{x}+C$.\\
% 		Do $F(-2)=0\Rightarrow 0=\dfrac{(-2)^2}{2}+\dfrac{1}{(-2)}+C\Rightarrow C=-\dfrac{3}{2}$.\\
% 		Suy ra $F(x)=\dfrac{x^2}{2}+\dfrac{1}{x}-\dfrac{3}{2}$.\\
% 		Vậy $F(2)=1$.
% 	}
% \end{ex}
% \begin{ex}%[2D4H1-4]
% 	Cho $F(x)$ là một nguyên hàm của hàm số $f(x)=x\sqrt{x}+\dfrac{1}{\sqrt{x}}$. Biết $F(1)=-2$. Tính $F(0)$.
% 	\shortans{$-4{,}4$}
% 	\loigiai{
% 		Hàm số $f(x)=x\sqrt{x}+\dfrac{1}{\sqrt{x}}=x^{\tfrac{3}{2}}+x^{-\tfrac{1}{2}}$.\\
% 		Có $F(x)=\displaystyle \int f(x)\mathrm{\,d}x=\displaystyle \int\left(x^{\tfrac{3}{2}}+x^{-\tfrac{1}{2}}\right)\mathrm{\,d}x=\dfrac{2}{5}x^{\tfrac{5}{2}}+2\sqrt{x}+C$.\\
% 		Do $F(1)=-2\Rightarrow -2=\dfrac{2}{5}\cdot 1^{\tfrac{5}{2}}+2\sqrt{1} +C \Rightarrow C=-\dfrac{22}{5}$.\\
% 		Suy ra $F(x)=\dfrac{2}{5}x^{\tfrac{5}{2}}+2\sqrt{x} -\dfrac{22}{5}$.\\
% 		Vậy $F(0)=-4{,}4$.
% 	}
% \end{ex}
% \begin{ex}%[2D4H1-3]
% 	Cho $F(x)$ là một nguyên hàm của hàm số $f(x)=\sin x +1$. Biết $F\left(\dfrac{\pi}{6}\right)=0$. Tính $F(-1)$. (Làm tròn đến chữ số thập phân thứ nhất)
% 	\shortans{$-1{,}2$}
% 	\loigiai{
% 		Ta có $F(x)=\displaystyle \int f(x)\mathrm{\,d}x=\displaystyle \int(\sin x +1)\mathrm{\,d}x=-\cos x +x +C$.\\
% 		Do $F\left(\dfrac{\pi}{6}\right)=0\Rightarrow 0=-\cos \dfrac{\pi}{6} +\dfrac{\pi}{6} +C\Rightarrow C=-\dfrac{\pi}{6} +\dfrac{\sqrt{3}}{2}$.\\
% 		Suy ra $F(x)=-\cos x +x -\dfrac{\pi}{6} +\dfrac{\sqrt{3}}{2}$.\\
% 		Vậy $F(-1)=-1{,}2$.
% 	}
% \end{ex}
% \begin{ex}%[2D4V1-3]
% 	Cho $F(x)$ là một nguyên hàm của $f(x)=2024-\sin^2 \dfrac{x}{2}$. Biết $F\left(\dfrac{\pi}{2}\right)=2025$. Tính $\sqrt{|F(0)|}$. (Làm tròn đến chữ số thập phân thứ nhất)
% 	\shortans{$34$}
% 	\loigiai{
% 		Hàm số $f(x)=2024-\sin^2 \dfrac{x}{2}=2024-\dfrac{1-\cos x}{2}=\dfrac{4047+\cos x}{2}$.\\
% 		Có $F(x)=\displaystyle \int f(x)\mathrm{\,d}x=\displaystyle \int\left(\dfrac{4047+\cos x}{2}\right)\mathrm{\,d}x=\dfrac{1}{2}(4047x+\sin x)+C$.\\
% 		Do $F\left(\dfrac{\pi}{2}\right)=2025\Rightarrow 2025=\dfrac{1}{2}(4047\cdot \dfrac{\pi}{2} +\sin \dfrac{\pi}{2})+C\Rightarrow C=-\dfrac{4047}{4}\pi +\dfrac{4049}{2}$.\\
% 		Suy ra $F(x)=\dfrac{1}{2}(4047x+\sin x)-\dfrac{4047}{4}\pi +\dfrac{4049}{2}$.\\
% 		Vậy $\sqrt{|F(0)|}=34$.
% 	}
% \end{ex}
% \begin{ex}%[2D4V1-3]
% 	Cho $F(x)$ là một nguyên hàm của $f(x)=\sin^2 \dfrac{x}{4} \cdot \cos^2 \dfrac{x}{4}$. Biết $F\left(\dfrac{\pi}{3}\right)=0$. Tính giá trị của $F(\pi)$. (Làm tròn đến chữ số thập phân thứ hai)
% 	\shortans{$0{,}37$}
% 	\loigiai{
% 		Hàm số $f(x)=\sin^2 \dfrac{x}{4} \cdot \cos^2 \dfrac{x}{4}=\dfrac{1}{8}(1-\cos x)$.\\
% 		Có $F(x)=\displaystyle \int f(x)\mathrm{\,d}x=\displaystyle \int \dfrac{1}{8}(1-\cos x)\mathrm{\,d}x=\dfrac{1}{8}(x-\sin x)+C$.\\
% 		Do $F\left(\dfrac{\pi}{3}\right)=0\Rightarrow 0=\dfrac{1}{8}\left(\dfrac{\pi}{3}-\sin \dfrac{\pi}{3}\right)+C\Rightarrow C=-\dfrac{\pi}{24}+\dfrac{\sqrt{3}}{16}$.\\
% 		Suy ra $F(x)=\dfrac{1}{8}(x-\sin x)-\dfrac{\pi}{24}+\dfrac{\sqrt{3}}{16}$.\\
% 		Vậy $F(\pi)=0{,}37$.
% 	}
% \end{ex}
% \begin{ex}%[2D4H1-2]
% 	Cho hàm số $f(x)=\heva{&2x+5 &\text{khi } &x\ge 1\\ &3x^2+4 &\text{khi } &x<1.}$ Giả sử $F$ là nguyên hàm của $f$ trên $\mathbb{R}$ thỏa mãn $F(0)=2$. Giá trị của $F(-1)+2F(2)$.
% 	\shortans{$27$}
% 	\loigiai{
% 		Ta có $f(x)=\heva{&2x+5 &\text{khi } &x\ge 1\\ &3x^2+4 &\text{khi } &x<1}\Rightarrow \heva{&F(x)=x^2+5x+C_1 &\text{khi } &x\ge 1\\ &F(x)=x^3+4x+C_2 &\text{khi } &x<1.}$\\
% 		Vì $F$ là nguyên hàm của $f$ trên $\mathbb{R}$ thỏa mãn $F(0)=2$ nên $C_2=2\Rightarrow F(x)=x^3+4x+2$.\\
% 		Vì $F(x)$ liên tục trên $\mathbb{R}$ nên $F(x)$ liên tục tại $x=1$ nên:\\
% 		$\lim\limits_{x\to1^+} F(x)=\lim\limits_{x\to1^-} F(x)=F(1)\Rightarrow 6+C_1=7\Rightarrow C_1=1$.\\
% 		Vậy ta có $\heva{&F(x)=x^2+5x+1 &\text{khi } &x\ge 1\\ &F(x)=x^3+4x+2 &\text{khi } &x<1}\Rightarrow F(-1)+2F(2)=27$.
% 	}
% \end{ex}
\begin{ex}%[2D4V1-4]
	Gọi $F(x)$ là một nguyên hàm của hàm số $f(x)=2^x$, thỏa mãn $F(0)=\dfrac{1}{\ln 2}$. Giá trị biểu thức $T=F(0)+F(1)+\cdots +F(2018)+F(2019)$ có dạng $\dfrac{{2^{2020}}+a}{\ln b}$. Giá trị của $\dfrac{a}{b}$ là
	\shortans{$-0{,}5$}
	\loigiai{
		Ta có $\displaystyle \int f(x) \mathrm{\,d}x=\displaystyle \int 2^x \mathrm{\,d}x=\dfrac{2^x}{\ln 2}+C$.\\
		$F(x)$ là một nguyên hàm của hàm số $f(x)=2^x$, ta có $F(x)=\dfrac{2^x}{\ln 2}+C$ mà $F(0)=\dfrac{1}{\ln 2}$.\\
		$\Rightarrow C=0\Rightarrow F(x)=\dfrac{2^x}{\ln 2}$.\\
		\begin{eqnarray*}
			T&=&F(0)+F(1)+\cdots +F(2018)+F(2019)\\
			&=&\dfrac{1}{\ln 2}(1+2+2^2+\cdots +2^{2018}+2^{2019})\\
			&=&\dfrac{1}{\ln 2}\cdot \dfrac{{2^{2020}}-1}{2-1}\\
			&=&\dfrac{{2^{2020}}-1}{\ln 2}.\\
		\end{eqnarray*}
		Vậy $\dfrac{a}{b}=-\dfrac{1}{2}=-0{,}5$
	}
\end{ex}
\begin{ex}%[2D4V1-3]
	Cho $F(x)$ là một nguyên hàm của hàm số $f(x)=\dfrac{1}{\cos^2 x}$. Biết $F\left(\dfrac{\pi}{4}+k\pi \right)=k$ với mọi $k\in \mathbb{Z}$. Tính giá trị của biểu thức $T=F(0)+F(\pi )+F(2\pi )+\cdots +F(10\pi )$.
	\shortans{$44$}
	\loigiai{
		Ta có $\displaystyle \int f(x) \mathrm{\,d}x=\displaystyle \int \dfrac{\mathrm{\,d}x}{\cos^2 x}=\tan x+C$.\\
		Suy ra 
		$F(x)=\heva{&\tan x+C_0, \quad x\in \left(-\dfrac{\pi}{2};\dfrac{\pi}{2}\right)\\ &\tan x+C_1, \quad x\in \left(\dfrac{\pi}{2};\dfrac{3\pi}{2}\right)\\ &\tan x+C_2, \quad x\in \left(\dfrac{3\pi}{2};\dfrac{5\pi}{2}\right)\\ &\cdots\\ &\tan x+C_9, \quad x\in \left(\dfrac{17\pi}{2};\dfrac{19\pi}{2}\right)\\ &\tan x+C_{10}, \quad x\in \left(\dfrac{19\pi}{2};\dfrac{21\pi}{2}\right)}\Rightarrow \heva{&F\left(\dfrac{\pi}{4}+0\pi\right)=1+C_0=0\Rightarrow C_0=-1\\ &F\left(\dfrac{\pi}{4}+\pi\right)=1+C_1=1\Rightarrow C_1=0\\ &F\left(\dfrac{\pi}{4}+2\pi\right)=1+C_2=2\Rightarrow C_2=1\\ &\cdots \\ &F\left(\dfrac{\pi}{4}+9\pi\right)=1+C_9=9\Rightarrow C_9=8\\ &F\left(\dfrac{\pi}{4}+10\pi\right)=1+C_{10}=0\Rightarrow C_{10}=9.}$\\
		Vậy \begin{eqnarray*}
			T&=&F(0)+F(\pi )+F(2\pi )+\cdots +F(10\pi )\\ &=&\tan 0-1+\tan \pi +\tan 2\pi +1+\cdots +\tan 10\pi +9\\ &=&44.
		\end{eqnarray*}
	}
\end{ex}
% \begin{ex}%[2D4V1-3]
% 	Hàm số $f(x)$ có đạo hàm liên tục trên $\mathbb{R}$ và $f'(x)=2024-2\sin^2 \dfrac{x}{2}$, $\forall x$;\hfill \break $f\left(\dfrac{\pi}{2}\right)=\dfrac{2023\pi}{2}$. Tính giá trị của $f(0)$.
% 	\shortans{$-1$}
% 	\loigiai{
% 		$f(x)=\displaystyle \int \left(2024-2\sin^2 \dfrac{x}{2}\right)\mathrm{\,d}x=\displaystyle \int (2023+\cos x)\mathrm{\,d}x=2023x+\sin x+C$.\\
% 		Tìm được $f(x)=2023x+\sin x+C$.\\
% 		Do $f\left(\dfrac{\pi}{2}\right)=\dfrac{2023\pi}{2} \Leftrightarrow \dfrac{2023\pi}{2}=2023\cdot \dfrac{\pi}{2}+\sin \dfrac{\pi}{2}+C\Leftrightarrow C=-1$.\\
% 		Vậy $f(x)=2023x+\sin x-1$.\\
% 		Do đó $f(0)=-1$.
% 	}
% \end{ex}
% \begin{ex}%[2D4H1-4]
% 	Hàm số $f(x)$ có đạo hàm liên tục trên $\mathbb{R}$ và $f'(x)=1+\mathrm{e}^{2x}$, $\forall x$; $f(0)=2$. Tính giá trị của $f(2)$. (Làm tròn đến số thập phân thứ nhất)
% 	\shortans{$30{,}8$}
% 	\loigiai{
% 		Hàm số $f(x)=\displaystyle \int (1+\mathrm{e}^{2x})\mathrm{\,d}x=x+\dfrac{1}{2}\mathrm{e}^{2x}+C$.\\
% 		Do $f(0)=2\Leftrightarrow 2=\dfrac{1}{2}+C\Leftrightarrow C=\dfrac{3}{2}$.\\
% 		Suy ra $f(x)=x+\dfrac{1}{2}\mathrm{e}^{2x}+\dfrac{3}{2}$.\\
% 		Vậy $f(2)=30{,}8$.
% 	}
% \end{ex}
% \begin{ex}%[2D4H1-4]
% 	Hàm số $f(x)$ có đạo hàm liên tục trên $\mathbb{R}$ và $f'(x)=2^x+3^x$, $\forall x$; $f(0)=\dfrac{1}{\ln 3}$. Tính giá trị của $f(1)$. (Làm tròn đến số thập phân thứ hai)
% 	\shortans{$4{,}17$}
% 	\loigiai{
% 		Hàm số $f(x)=\displaystyle \int (2^x+3^x)\mathrm{\,d}x= \displaystyle \int 2^x \mathrm{\,d}x+\displaystyle \int 3^x \mathrm{\,d}x=\dfrac{2^x}{\ln 2}+\dfrac{3^x}{\ln 3}+C$.\\
% 		$f(x)=\dfrac{2^x}{\ln 2}+\dfrac{3^x}{\ln 3}+C$.\\
% 		Do $f(0)=\dfrac{1}{\ln 3} \Leftrightarrow \dfrac{1}{\ln 3}=\dfrac{1}{\ln 2}+\dfrac{1}{\ln 3}+C\Leftrightarrow C=-\dfrac{1}{\ln 2}$\\
% 		Suy ra $f(x)=\dfrac{2^x}{\ln 2}+\dfrac{3^x}{\ln 3}-\dfrac{1}{\ln 2}$.\\
% 		Vậy $f(1)=4{,}17$.
% 	}
% \end{ex}
%%%%%----------Câu 34
\begin{ex}%[2D4H1-4]
	Hàm số $f(x)$ có đạo hàm liên tục trên $\mathbb{R}$ và $f'(x)=\mathrm{e}^{3x+2024}$, $\forall x $ thoả mã $f(-675)=1$. Giá trị của $f(-674)$ bằng
	\shortans[3]{$3{,}34$}
	\loigiai{
		Hàm số $f(x)$ có đạo hàm $f'(x)=\mathrm{e}^{3x+2024}$.\\
		Ta có $f(x)=\displaystyle\int\!\!\mathrm{e}^{3x+2024}\mathrm{d}x =\dfrac{1}{3}\mathrm{e}^{3x+2024}+C$.\\ 
		Suy ra $f(x)=\dfrac{1}{3}\mathrm{e}^{3x+2024}+C$.\\
		Với $f(-675)=1 \Rightarrow 1 =\dfrac{1}{3}\mathrm{e}^{3\cdot(-675)+2024}+C \Rightarrow C=1-\dfrac{1}{3\mathrm{e}}$.\\
		Vậy $f(x)=\dfrac{1}{3}\mathrm{e}^{3x+2024}+1-\dfrac{1}{3\mathrm{e}}$.\\
		Giá trị $f(-274)=\dfrac{1}{3}\mathrm{e}^2+1-\dfrac{1}{3\mathrm{e}}=3{,}34$.
	}
\end{ex}
%%%%%----------Câu 35
\begin{ex}%[2D4H1-4]
	Hàm số $f(x)$ có đạo hàm liên tục trên $\mathbb{R}$ và $f'(x)=3^{x+2}\cdot2^{2x+1}$, $\forall x$ thoả mãn $f(0)~=~\dfrac{1}{2\ln 2}$. Giá trị của $f(1)$ bằng
	\shortans[3]{$80{,}4$}
	\loigiai{
		Hàm số $f(x)$ có đạo hàm $f'(x)=3^{x+2}\cdot2^{2x+1}$.\\
		Ta có $f(x)=\displaystyle\int3^{x+2}\cdot2^{2x+1}\mathrm{d}x =\int3^2\cdot3^x\cdot2\cdot4^x\mathrm{d}x=18\int12^x\mathrm{d}x =18\cdot\dfrac{12^x}{\ln 12}+C$.\\
		Suy ra $f(x)=18\cdot\dfrac{12^x}{\ln 12}+C$.\\
		Với $f(0)=\dfrac{1}{2\ln 2}$
		$\Rightarrow \dfrac{1}{2\ln 2}=18\dfrac{1}{\ln 12}+C \Rightarrow C=\dfrac{1}{2\ln 2}-\dfrac{18}{2\ln 2+\ln 3}$.\\
		Vậy $f(x)=18\cdot\dfrac{12^x}{\ln 12}+\dfrac{1}{2\ln 2}-\dfrac{18}{2\ln 2+\ln 3}$.\\
		Giá trị $f(1)=18\cdot\dfrac{12}{\ln 12}+\dfrac{1}{2\ln 2}-\dfrac{18}{2\ln 2+\ln 3}=\dfrac{216}{\ln 12}+\dfrac{1}{\ln 4}-\dfrac{18}{\ln 4+\ln 3}=80{,}4$.
	}
\end{ex}
%%%%%----------Câu 36
\begin{ex}%[2D4H1-4]
	Hàm số $f(x)$ có đạo hàm liên tục trên $\mathbb{R}$ và $f'(x)=\left( 3^x+5^x \right)^2$, $\forall x$ thoả mãn\break $f(0)=\dfrac{1}{\ln 5+\ln 3+\ln 2}$. Giá trị của $f(1)$ bằng
	\shortans[3]{$19{,}9$}
	\loigiai{
		Hàm số $f(x)$ có đạo hàm $f'(x)=\left( 3^x+5^x \right)^2$.\\
		Ta có 
		$\begin{aligned}[t]
			f(x)&=\displaystyle\int(3^x+5^x)^2\mathrm{d}x\\
			&=\int(9^x+30^x+25^x)\mathrm{d}x\\
			&=\dfrac{9^x}{\ln 9}+\dfrac{30^x}{\ln 30}+\frac{25^x}{\ln 25}+C\\
			&=\dfrac{9^x}{2\ln 3}+\dfrac{30^x}{\ln 5+\ln 3+\ln 2}+\dfrac{25^x}{2\ln 5}+C.
		\end{aligned}$\\
		Suy ra $f(x)=\dfrac{9^x}{2\ln 3}+\dfrac{30^x}{\ln 5+\ln 3+\ln 2}+\dfrac{25^x}{2\ln 5}+C$.\\
		Với 
		$\begin{aligned}[t]
			f(0)=&\ \dfrac{1}{\ln 5+\ln 3+\ln 2}\\
			\Rightarrow &\ \dfrac{1}{\ln 5+\ln 3+\ln 2} =\dfrac{1}{2\ln 3}+\dfrac{1}{\ln 5+\ln 3+\ln 2}+\dfrac{1}{2\ln 5}+C\\
			\Leftrightarrow &\ C=-\dfrac{1}{2\ln 3}-\dfrac{1}{2\ln 5}.
		\end{aligned}$\\
		Vậy
		$\begin{aligned}[t]
			f(x)&=\dfrac{9^x}{2\ln 3}+\dfrac{30^x}{\ln 5+\ln 3+\ln 2}+\dfrac{25^x}{2\ln 5}-\dfrac{1}{2\ln 3}-\dfrac{1}{2\ln 5}\\
			&=\dfrac{9^x}{\ln 9}+\dfrac{30^x}{\ln 30}+\dfrac{25^x}{\ln 25}-\dfrac{1}{\ln 9}-\dfrac{1}{\ln 25}.
		\end{aligned}$\\
		Giá trị của
		$\begin{aligned}[t]
			f(1)&=\dfrac{9}{\ln 9}+\dfrac{30}{\ln 30}+\dfrac{25}{\ln 25}-\dfrac{1}{\ln 9}-\dfrac{1}{\ln 25}\\
			&=\dfrac{8}{\ln 9}+\dfrac{30}{\ln 30}+\dfrac{24}{\ln 25}\\
			&=19{,}9.
		\end{aligned}$
	}
\end{ex}
\Closesolutionfile{ans}
% \indapan{6}{ans/ans-C4B1CD2-CAU31_33-KQ}
% % \subsection{ỨNG DỤNG NGUYÊN HÀM TRONG THỰC TIỄN}
\begin{dang}{Ứng dụng trong bài toán thực tiễn}
	Giả sử $v(t)$ là vận tốc của vật $ {M}$ tại thời điểm $t$ và $s(t)$ là quãng đường vật đi được sau khoảng thời gian $t$ tính từ lúc bắt đầu chuyển động. Ta có mối liên hệ giữa $s(t)$ và $v(t)$ như sau.
	\begin{itemize}
		\item  Đạo hàm của quãng đường là vận tốc $s'(t)=v(t)$.
		\item  Nguyên hàm của vận tốc là quãng đường $s(t)=\displaystyle\int v(t)  \mathrm{\,d} t$.
	\end{itemize}
	Nếu gọi $a(t)$ là gia tốc của vật M thì ta có mối liên hệ giữa $v(t)$ và $a(t)$ như sau.
	\begin{itemize}
		\item Đạo hàm của vận tốc là gia tốc $v'(t)=a(t)$.
		\item Nguyên hàm của gia tốc là vận tốc $v(t)=\displaystyle\int\limits a(t)  \mathrm{\,d} t$.
	\end{itemize}
\end{dang}
% \TN
\Opensolutionfile{ans}[ans/ans2-C4B1CD4-D1]
\begin{ex}%[2D4V1-6]
	Một ô tô đang chạy với vận tốc $20$ m/s thì người lái đạp phanh. Sau khi đạp phanh, ô tô chuyển động chậm dần đều với vận tốc $v(t)=-40t+20$ m/s, trong đó $t$ là khoảng thời gian tính bằng giây kể từ lúc bắt đầu đạp phanh. Gọi  $s(t)$ là quãng đường xe ô tô đi được trong thời gian $t$  (giây) kể từ lúc đạp phanh. Hỏi từ lúc đạp phanh đến khi dừng hẳn, ô tô còn di chuyển bao nhiêu mét?
	\choice{$5$ cm}{$7{,}5$ m}{$\dfrac{5}{2}$ m}{\True $5$ m}
	\loigiai{
		Ta có $v(t)=-40t+20$.\\
		Suy ra $s(t)=\displaystyle\int v(t)\mathrm{\,d}t=\displaystyle\int (-40t+20)\mathrm{\,d}t=-20t^2+20t+C$.\\
		Chọn $t=0$ suy ra $s(0)=0\Rightarrow C=0$.\\
		Khi đó $s(t)=-20t^2+20t$.\\
		Khi xe dừng hẳn $v(t)=0\Leftrightarrow -40t+20=0\Leftrightarrow t=0{,}5$.\\
		Từ lúc đạp phanh đến khi dừng hẳn, ô tô còn di chuyển được\\ $s(0{,}5)=-20\cdot (0{,}5)^2+20\cdot 0{,}5=5$ m.
	}
\end{ex}
\Opensolutionfile{ans}[ans/ansMyLT]
%\begin{ex}%[2D4H1-6]
%Một ô tô đang chạy với vận tốc $20$ m/s thì người người đạp phanh. Sau khi đạp phanh, ô tô chuyển động chậm dần đều với vận tốc $v(t)=-40t+20$ m/s, trong đó $t$ là khoảng thời gian tính bằng giây kể từ lúc bắt đầu đạp phanh. Gọi $s(t)$ là quãng đường xe ô tô đi được trong thời gian $t$ (giây) kể từ lúc đạp phanh. Hỏi từ lúc đạp phanh đến khi dừng hẳn, ô tô còn di chuyển bao nhiêu mét?
%\choice
%{$5$ cm}
%{$7,5$ m}
%{$\dfrac{5}{2}$ m}
%{\True $5$ m}
%\begin{center}
%	\color{red}HÌNH Ở ĐÂY
%\end{center}
%\loigiai{
	%Ta có $v(t)=-40t+20$\\
	%$\Rightarrow s(t)=\displaystyle\int{v(t)}dt=\displaystyle\int{\left(-40t+20\right)}dt=-20t^2+20t+C$\\
	%$\Rightarrow s(t)=-20t^2+20t+C$\\
	%Chọn $t=0\Rightarrow s(0)=0$ $\Rightarrow C=0$\\
	%$\Rightarrow s(t)=-20t^2+20t$\\
	%Khi xe dừng hẳn thì $v(t)=0\Leftrightarrow-40t+20=0\Rightarrow t=0{,}5$.\\
	%từ lúc đạp phanh đến khi dừng hẳn, ô tô còn di chuyển được:\\ $s\left(0{,}5\right)=-20\left(0{,}5\right)^2+20\left(0{,}5\right)=5$ m.}
%\end{ex}

\begin{ex}%[2D4H1-6]
	Bạn Minh Hiền ngồi trên máy bay đi du lịch thế giới với vận tốc chuyển động của máy báy là $v(t)=3t^2+5$ (m/s). Quãng đường máy bay bay từ giây thứ $4$ đến giây thứ $10$ là
	\begin{center}
		\begin{tikzpicture}
			\clip (-4.5,-1) rectangle (4.5,1);
			%\path (0,0) node[opacity=.5,scale=.3] {\includegraphics{6}};
			%\draw[gray!50] (-4,-2) grid (4,2);
			\begin{pgfinterruptboundingbox}
				%Đuôi máy bay
				\def\D{ (-2,-.45)--(-2.36,.24)--(-2.2,.3)--(-1.35,-.25)--cycle
					;}
				\draw \D;
				\fill[brown!40!] \D;
				%Quạt máy bay
				\def\Q{ (.15,-.1)
					..controls +(-140:0.1) and +(140:0.1) .. (.2,-.35)--(.8,-.17)--(.7,.07)--cycle;}
				\draw \Q;
				\draw[xshift=.5cm] \Q;
				\fill[brown!40!,xshift=.5cm] \Q;
				%---------elip2
				\draw[rotate=-70,xshift=-.23cm,yshift=1.27cm] (.7,-.1) ellipse (.12cm and .07cm);
				%Thân máy bay
				\def\T{ (-2.1,-.5)
					..controls +(40:0) and +(-170:0.7) .. (2,.74)
					..controls +(-40:0) and +(170:0.3) .. (2.55,.7)
					..controls +(-150:0) and +(30:.65) .. (2.1,.3)
					..controls +(-150:0) and +(-18:1.25) .. (-2.1,-.6)
					..controls +(90:0) and +(-90:0) .. (-2.1,-.51)
					-- (-2.5,-.52)--cycle;
				}
				\draw \T;
				\fill[brown!70!] \T;
				%Cánh máy bay
				\def\C{ (.5,.05)
					..controls +(170:0) and +(-10:0) .. (-1.48,.12)
					..controls +(-80:0) and +(170:0.02) .. (-1.65,.3)
					..controls +(180:0) and +(0:0) .. (-1.77,.3)
					..controls +(-80:0) and +(110:0) .. (-1.64,.12)
					..controls +(-35:0) and +(145:0) .. (-.2,-.17)
					--cycle;}
				\draw \C;
				\fill[brown!40!] \C;
				%Quạt sau
				\fill[brown!40!] \Q;
				
				%----elip 1
				\draw[rotate=-70,xshift=-.4cm,yshift=.8cm] (.7,-.1) ellipse (.12cm and .07cm);
				%Ô cửa
				\draw (2.18,0.73)--(2.5,0.7)--(2.14,0.58)--cycle;
				\fill[brown!40!](2.18,0.73)--(2.51,0.71)--(2.14,0.57)--cycle;
			\end{pgfinterruptboundingbox}
		\end{tikzpicture}
	\end{center}
	\choice
	{$36$ m}
	{$252$ m}
	{$1134$ m}
	{\True $966$ m}
	
	\loigiai{
		Ta có $v(t)=3t^2+5$.\\
		$\Rightarrow s(t)=\displaystyle\int{v(t)\mathrm{\,d}t}=\displaystyle\int{\left(3t^2+5\right)} \mathrm{\,d}t=t^3+5t+C$.\\
		$\Rightarrow s(t)=t^3+5t+C$.\\
		Chọn $t=0\Rightarrow s(0)=0 \Rightarrow C=0$.\\
		$\Rightarrow s(t)=t^3+5t$.\\
		Quãng đường máy bay bay từ giây thứ $4$ là $s(4)=4^3+5\cdot4=84$ (m).\\
		Quãng đường máy bay bay từ giây thứ $10$ là $s\left(10\right)=10^3+5\cdot 10=1050$ (m).\\
		Quãng đường máy bay bay từ giây thứ $4$ đến giây thứ $10$ là $s(10)-s(4)=966$ (m).}
\end{ex}

\begin{ex}%[2D4H1-6]
	Một ô tô đang chạy với vận tốc $12$ m/s thì người lái đạp phanh; từ thời điểm đó, ô tô chuyển động chậm dần đều với vận tốc $v(t)=-6t+12$ (m/s), trong đó $t$ là khoảng thời gian tính bằng giây kể từ lúc đạp phanh. Hỏi từ lúc đạp phanh đến khi ô tô dừng hẳn, ô tô còn di chuyển được bao nhiêu mét?
	\choice
	{$24$ m}
	{\True $12$ m}
	{$6$ m}
	{$0{,}4$ m}
	\loigiai{
		Ta có 
		\begin{eqnarray*}
			& & v(t)=-6t+12\\
			&\Rightarrow & s(t)=\displaystyle\int{v(t) \mathrm{\,d}t}\\
			&\Leftrightarrow &  s(t)=\displaystyle\int (-6t+12)\mathrm{\,d}t\\
			&\Leftrightarrow &  s(t)=-3t^2+12t+C.
		\end{eqnarray*}
		Chọn $t=0\Rightarrow s(0)=0 \Rightarrow C=0\Rightarrow s(t)=-3t^2+12t$.\\
		Khi xe dừng hẳn thì $v(t)=0\Leftrightarrow-6t+12=0\Rightarrow t=2$.\\
		Từ lúc đạp phanh đến khi ô tô dừng hẳn thì ô tô còn di chuyển được quãng đường là
		$$S=s(2)-s(0)=s(2)=-3\cdot 2^2 +12\cdot 2 =12\text{ (m).}$$
	}
\end{ex}

\begin{ex}%[2D4H1-6]
	Một ô tô đang chạy với vận tốc $36$ km/h thì tăng tốc chuyển động nhanh dần đều với gia tốc $a(t)=1+\dfrac{t}{3}$ (m/s$^2$) tính quãng đường ô tô đi được sau $6$ giây kể từ khi ô tô bắt đầu tăng tốc.
	\choice
	{\True $S=90$ m}
	{$S=246$ m}
	{$S=58$ m}
	{$S=100$ m}
	\loigiai{
		Đổi $36$ km/h $= 36\cdot \dfrac{1000}{3600}=10$ m/s.\\
		Ta có $a(t)=1+\dfrac{t}{3}$.\\
		$\Rightarrow v(t)=\displaystyle\int{a(t) \mathrm{\,d}t}=\displaystyle\int \left(1+\dfrac{t}{3}\right) \mathrm{\,d}t=t+\dfrac{1}{6}t^2+C$.\\
		Từ lúc bắt đầu tăng tốc thì vận tốc của xe là $10$ m/s nên ta có 
		\begin{eqnarray*}
			& & v(0)=10\\
			&\Rightarrow & C=10\\
			&\Rightarrow & v(t)=t+\dfrac{1}{6}t^2+10\\
			&\Rightarrow & s(t)=\displaystyle\int{v(t) \mathrm{\,d}t}\\
			&\Rightarrow & s(t)=\displaystyle\int (t+\dfrac{1}{6}t^2+10)\\
			&\Rightarrow & s(t)=\dfrac{t^2}{2}+\dfrac{t^3}{18}+10t + C_1.
		\end{eqnarray*}
		Quãng đường tính từ lúc xe bắt đầu tăng tốc nên $s(0)=0 \Rightarrow C_1=0$.\\
		Vậy $s(6)=\dfrac{6^2}{2}+\dfrac{6^3}{18}+10\cdot 6 =90$ (m).
	}
\end{ex}

\begin{ex}%[2D4H1-6]
	Một ca nô đang chạy trên hồ Tây với vận tốc $20$ m/s thì hết xăng; từ thời điểm đó, ca nô chuyển động chậm dần đều với vận tốc $v(t)=-5t+20$ (m/s), trong đó $t$ là khoảng thời gian tính bằng giây, kể từ lúc hết xăng. Hỏi từ lúc hết xăng đến lúc ca nô dừng hẳn thì ca nô đi được bao nhiêu mét?
	\choice
	{$10$ m}
	{$20$ m}
	{$30$ m}
	{\True $40$ m}
	\loigiai{
		Ta có $v(t)=-5t+20$.\\
		$\Rightarrow s(t)=\displaystyle\int{v(t)} \mathrm{\,d}t=\displaystyle\int{\left(-5t+20\right)} \mathrm{\,d}t=-\dfrac{5}{2}t^2+20t+C$.\\
		Chọn $t=0\Rightarrow s(0)=0 \Rightarrow C=0$. Suy ra $s(t)=-\dfrac{5}{2}t^2+20t$.\\
		Khi xe dừng hẳn thì $v(t)=0\Leftrightarrow-5t+20=0\Rightarrow t=4$ (s).\\
		Từ lúc đạp phanh đến khi dừng hẳn, ô tô còn di chuyển được $s = -\dfrac{5}{2}\cdot 4^2+20\cdot 4=40$ (m).
	}
\end{ex}

\begin{ex}%[2D4H1-6]
	Một vật chuyển động với vận tốc $10$ m/s thì tăng tốc với gia tốc được tính theo thời gian $t$ là $a(t)=3t+t^2$ (m$^2$/s). Tính quãng đường vật đi được trong $10$s kể từ khi bắt đầu tăng tốc.
	\choice
	{$\dfrac{130}{3}$ m}
	{$\dfrac{310}{3}$ m}
	{$\dfrac{3400}{3}$ m}
	{\True $\dfrac{4300}{3}$ m}
	\loigiai{
		Ta có $a(t)=3t+t^2$\\
		$\Rightarrow v(t)=\displaystyle\int{a(t) \mathrm{\,d}t}=\displaystyle\int (3t+t^2) \mathrm{\,d}t=\dfrac{3}{2}t^2+\dfrac{1}{3}t^3+C$.\\
		Từ lúc bắt đầu tăng tốc thì vận tốc của xe là $10$ m/s nên ta có \\ 
		$v(0)=10 \Rightarrow C=10$.\\
		$\Rightarrow v(t)=\dfrac{3}{2}t^2+\dfrac{1}{3}t^3+10$.\\
		$\Rightarrow s(t)=\displaystyle\int{v(t) \mathrm{\,d}t}=\displaystyle\int (\dfrac{3}{2}t^2+\dfrac{1}{3}t^3+10) \mathrm{\,d}t=\dfrac{1}{2}t^3+\dfrac{1}{12}t^4+10t+C_1$.\\
		Quãng đường tính từ lúc xe bắt đầu tăng tốc nên $s(0)=0 \Rightarrow C_1=0$.\\
		Vậy $s(10)=\dfrac{1}{2}\cdot 10^3+\dfrac{1}{12}\cdot 10^4+10\cdot 10=\dfrac{4300}{3}$ m.
	}
\end{ex}

\begin{ex}%[2D4H1-6]
	Tại một nơi không có gió, một chiếc khí cầu đang đứng yên ở độ cao $162$ m so với mặt đất đã được phi công cài đặt cho nó chế độ chuyển động đi xuống. Biết rằng, khí cầu đã chuyển động theo phương thẳng đứng với vận tốc tuân theo quy luật $v(t)=10t-t^2$, trong đó $t$ (phút) là thời gian tính từ lúc bắt đầu chuyển động, $v(t)$ được tính theo đơn vị mét/phút (m/p). Nếu như vậy thì khi bắt đầu tiếp đất vận tốc $v$ của khí cầu là
	\choice
	{$5$ m/p}
	{$7$ m/p}
	{\True $9$ m/p}
	{$3$ m/p}
	\loigiai{
		Ta có $v(t)=10t-t^2$.\\
		$\Rightarrow s(t)=\displaystyle\int{v(t) \mathrm{\,d}t}=\displaystyle\int (10t-t^2) \mathrm{\,d}t=5t^2-\dfrac{1}{3}t^3+C$.\\
		Từ lúc bắt đầu giảm độ cao thì khinh khí cầu ở độ cao $162$ m nên ta có \\ 
		$s(0)=0 \Rightarrow C=0\Rightarrow s(t)=5t^2-\dfrac{1}{3}t^3$.\\
		mà $s(t)= 162 \Rightarrow 5t^2-\dfrac{1}{3}t^3=162 \Leftrightarrow t=9$ (s).\\
		Suy ra vận tốc khi chạm đất của khinh khí cầu là
		$$v(9)= 10\cdot 9 -9^2=9 \text{ (m/p)}.$$
	}
\end{ex}

\begin{ex}%[2D4H1-6]
	Một viên đạn được bắn lên theo phương thẳng đứng với vận tốc ban đầu là $25$ m/s, gia tốc trọng trường là $9{,}8$ m/s$^2$. Quãng đường viên đạn đi được từ lúc bắn cho đến khi chạm đất gần bằng kết quả nào nhất trong các kết quả sau?
	\choice
	{$30{,}78$ m}
	{\True $31{,}89$ m}
	{$32{,}43$ m}
	{$33{,}88$ m}
	\loigiai{
		Ta có $a(t)=-9{,}8$ (m/s$^2$).\\
		$\Rightarrow v(t)=\displaystyle\int{a(t) \mathrm{\,d}t}=\displaystyle\int (-9{,}8) \mathrm{\,d}t=-9{,}8t+C$.\\
		Từ lúc bắt đầu bắn viên đạn thì viên đạn có vận tốc $25$ m/s nên ta có\\
		$v(0)=25 \Rightarrow C=25 \Rightarrow v(t)=-9{,}8t+25$.\\ 
		$\Rightarrow s(t)=\displaystyle\int{v(t) \mathrm{\,d}t}=\displaystyle\int (-9{,}8t+25) \mathrm{\,d}t=-4{,}9t^2+25t+C_1$.\\
		Ta có $s(0)=0 \Rightarrow C_1=0$.\\
		$s(t)=-4{,}9t^2+25t$.\\
		Tới khi vận tốc viên đạn bằng không thì ta có $v(t)=0 \Rightarrow -9{,}8t+25=0 \Leftrightarrow t \approx 2{,}55$ s.
		Suy ra quãng đường viên đạn đi được từ lúc bắn cho đến khi chạm đất là\\
		$S= 2\cdot (-4{,}9\cdot (2{,}55)^2+25\cdot 2{,}55) \approx 31{,}89$ m.
	}
\end{ex}

\begin{ex}%[2D4H1-6]
	Trong một đợt xả lũ, nhà máy thủy điện đã xả lũ trong $40$ phút với tốc độ lưu lượng nước tại thời điểm $t$ giây là $h'(t)=10t+500$ (m$^3$/s). Hỏi sau thời gian xả lũ trên thì hồ thoát nước của nhà máy đã thoát đi một lượng nước là bao nhiêu?
	\choice
	{$5\cdot 10^4$ m$^3$}
	{$4\cdot 10^6$ m$^3$}
	{\True $3\cdot 10^7$ m$^3$}
	{$6\cdot 10^6$ m$^3$}
	\loigiai{
		Ta có $h'(t)=10t+500$.\\
		$\Rightarrow h(t)=\displaystyle\int{\left(10t+500\right)}\mathrm{\,dx}=5t^2+500t+C$.\\
		$\Rightarrow h(t)=5t^2+500t+C$.\\
		Chọn $t=0\Rightarrow h(0)=0\Rightarrow C=0$.\\
		$\Rightarrow h(t)=5t^2+500t$.\\
		Thủy điện đã xả lũ trong $40$ phút=$2400$ giây thì thoát đi một lượng nước là
		$$h\left(2400\right)=5\cdot 2400^2+500\cdot 2400=3\cdot 10^7\text{ (m$^3$)}.$$
	}
\end{ex}

\begin{ex}%[2D4H1-6]
	Một bác thợ xây bơm nước vào bể chứa nước. Gọi $h(t)$ là thể tích nước bơm được sau $t$ giây. Cho $h'(t)=3a{t^2}+bt$ (m$^3$/s) và ban đầu bể không có nước. Sau $5$ giây thì thể tích nước trong bể là $150$ m$^3$. Sau $10$ giây thì thể tích nước trong bể là $1100$ m$^3$. Hỏi thể tích nước trong bể sau khi bơm được $20$ giây là bao nhiêu.
	\choice
	{\True $8400$ m$^3$}
	{$7400$ m$^3$}
	{$6000$ m$^3$}
	{$4200$ m$^3$}
	\loigiai{
		Ta có $h'(t)=3a{t^2}+bt$.\\
		$\Rightarrow h(t)=\displaystyle\int{\left(3a{t^2}+bt\right)} \mathrm{\,d}t=a{t^3}+\dfrac{1}{2}b{t^2}+C$.\\
		$\Rightarrow h(t)=a{t^3}+\dfrac{1}{2}b{t^2}+C$.\\
		Chọn $t=0\Rightarrow h(0)=0\Rightarrow C=0$.\\
		$\Rightarrow h(t)=a{t^3}+\dfrac{1}{2}b{t^2}$.\\
		Sau $5$ giây thì thể tích nước trong bể là $150$ m$^3$ là  $h(5)=150\Leftrightarrow 125a+\dfrac{25}{2}b=150$.\hfill(1)\\
		Sau $10$ giây thì thể tích nước trong bể là $1100$ m$^3$ là  $h(10)=1100\Leftrightarrow 1000a+50b=1100$.\hfill(2)\\
		Từ (1) và (2) ta có hệ  $\heva{
			&125a+\dfrac{25}{2}b=150\\
			&1000a+50b=1100
		}\Leftrightarrow \heva{&a=1\\&b=2.}$\\
		$\Rightarrow h(t)=t^3+t^2$.\\
		Thể tích nước trong bể sau khi bơm được $20$ giây là $h\left(20\right)=20^3+20^2=8400$ (m$^3$).}
\end{ex}

\begin{ex}%[2D4H1-6]
	Gọi $h(t)$ (m) là mực nước ở bồn chứa sau khi bơm nước được $t$ giây. Biết rằng $h'(t)=\dfrac{1}{5}\sqrt[3]{t}$ (m/s) và lúc đầu bồn không có nước. Tìm mức nước ở bồn sau khi bơm nước được $6$ giây (\textit{làm tròn kết quả đến hàng phần trăm}).
	\choice
	{$2{,}64$ m}
	{$1{,}22$ m}
	{$2{,}22$ m}
	{\True $1{,}64$ m}
	\loigiai{
		Ta có $h'(t)=\dfrac{1}{5}\sqrt[3]{t}$.\\
		$\Rightarrow h(t)=\displaystyle\int{\dfrac{1}{5}\sqrt[3]{t}}\mathrm{\,dx}=\dfrac{1}{5}\displaystyle\int{t^{\frac{1}{3}}}\mathrm{\,dx}=\dfrac{1}{5}\dfrac{t^{\frac{1}{3}+1}}{\frac{1}{3}+1}+C=\dfrac{3}{20}t\sqrt[3]{t}+C$.\\
		$\Rightarrow h(t)=\dfrac{3}{20}t\sqrt[3]{t}+C$\\
		Chọn $t=0\Rightarrow h(0)=0\Rightarrow C=0$.\\
		$\Rightarrow h(t)=\dfrac{3}{20}t\sqrt[3]{t}$.\\
		Mức nước ở bồn sau khi bơm nước được $6$ giây $h(6)=\dfrac{3}{20}\cdot 6\sqrt[3]{6}\approx 1{,}64$ (m).}
\end{ex}

\begin{ex}%[2D4H1-6]
	Sự sản sinh vi rút Zika ngày thứ $t$ có số lượng là $N(t)$ con, biết $N'(t)=\dfrac{1000}{t}$ và lúc đầu đám vi rút có số lượng $250{,}000$ con. Tính số lượng vi rút sau $10$ ngày.
	\choice
	{$272304$ con}
	{$212302$ con}
	{$242102$ con}
	{\True $252302$ con}
	\loigiai{
		Ta có $N'(t)=\dfrac{1000}{t}$.\\
		$\Rightarrow N(t)=\displaystyle\int{\dfrac{1000}{t} \mathrm{\,d}t=1000\ln \left| t\right|}+C$.\\
		$\Rightarrow N(t)=1000\ln \left| t\right|+C$.\\
		Chọn $t=1\Rightarrow N(1)=250000\Rightarrow C=250000$.\\
		$\Rightarrow N(t)=1000\ln \left| t\right|+250000$.\\
		Số lượng vi rút sau $10$ ngày là $N\left(10\right)=1000\ln 10+250000\approx 252302$ (con).
	}
\end{ex}
% \TNSA
\begin{ex}%[2D4H1-6]
	Một chiếc ô tô đang chạy với vận tốc $15$ m/s thì nhìn thấy chướng ngại vật trên đường cách đó $50$ m, người lái xe hãm phanh khẩn cấp. Sau khi hãm phanh, ô tô chuyển động chậm dần đều với vận tốc $v(t)=-3t+15$ (m/s), trong đó $t$ (giây). Gọi $s(t)$ là quãng đường xe ô tô đi được trong thời gian $t$ (giây) kể từ lúc đạp phanh. Hỏi từ lúc hãm phanh đến khi dừng hẳn, ô tô di chuyển được bao nhiêu mét?
	\shortans[]{$37{,}5$}
	\loigiai{
		Quãng đường xe ô tô đi được trong thời gian $t$ (giây) là một nguyên hàm của $v(t)$ nên\\
		$s(t)=\displaystyle\int{v(t)}\mathrm{\,dx}=\displaystyle\int{(-3t+15)}\mathrm{\,dx}=-\dfrac{3t^2}{2}+15t+C$.\\
		$\Rightarrow s(t)=-\dfrac{3t^2}{2}+15t+C$.\\
		Chọn $t=0\Rightarrow s(0)=0$\\
		$\Rightarrow C=0$\\
		$\Rightarrow s(t)=-\dfrac{3t^2}{2}+15t$\\
		Khi xe dừng hẳn thì $v(t)=0\Leftrightarrow-3t+15=0\Rightarrow t=5$ (s).\\
		Thời gian kể từ lúc đạp phanh đến khi dừng hẳn là $5$ giây\\
		Sau khi đạp phanh đến khi dừng hẳn, xe đi được quãng đường
		$$s(5)=-\dfrac{3.5^2}{2}+15\cdot 5=37{,}5 \text{ (m).}$$
	}
\end{ex}

\begin{ex}%[2D4H1-6]
	Một chiếc ô tô đang chạy với vận tốc $72$ km/h thì nhìn thấy chướng ngại vật trên đường cách đó $40$ m, người lái xe hãm phanh khẩn cấp. Sau khi hãm phanh, ô tô chuyển động chậm dần đều với vận tốc $v(t)=-10t+20$ (m/s), trong đó $t$ tính bằng giây. Gọi $s(t)$ là quãng đường xe ô tô đi được trong thời gian $t$ (giây) kể từ lúc đạp phanh.
	Hỏi từ lúc hãm phanh đến khi dừng hẳn, ô tô di chuyển được bao nhiêu mét?
	\shortans[]{$20$}
	\loigiai{
		Quãng đường xe ô tô đi được trong thời gian $t$ (giây) là một nguyên hàm của $v(t)$ nên\\
		$s(t)=\displaystyle\int{v(t)}\mathrm{\,dx}=\displaystyle\int{(-10t+20)}\mathrm{\,dx}=-5t^2+20t+C$.\\
		$\Rightarrow s(t)=-5t^2+20t+C$.\\
		Chọn $t=0\Rightarrow s(0)=0$.\\
		$\Rightarrow C=0$.\\
		$\Rightarrow s(t)=-5t^2+20t$.\\
		Khi xe dừng hẳn thì $v(t)=0\Leftrightarrow-10t+20=0\Rightarrow t=2$ (s).\\
		Thời gian kể từ lúc đạp phanh đến khi dừng hẳn là $2$ giây.\\
		Sau khi đạp phanh đến khi dừng hẳn, xe đi được quãng đường
		$$s(2)=-5\cdot 2^2+20\cdot 2=20 \text{ (m).}$$
	}
\end{ex}

\begin{ex}%[2D4H1-6]
	Một viên đạn được bắn lên theo phương thẳng đứng từ mặt đất. Tại thời điểm $t$ giây vận tốc của nó được cho bởi công thức $v(t)=24{,}5-9{,}8t$ (m/s). Tính quãng đường viên đạn đi từ lúc bắn lên cho tới khi rơi xuống đất (\textit{làm tròn tới hàng đơn vị}).
	\shortans[]{$61$}
	\loigiai{
		Quãng đường viên đạn đi được là\\ $s(t)=\displaystyle\int{\left(24{,}5-9{,}8t\right)}\mathrm{\,dx}=24{,}5t-4{,}9t^2+C$.\\
		$\Rightarrow s(t)=24{,}5t-4{,}9t^2+C$.\\
		Chọn $t=0\Rightarrow s(0)=0$.\\
		$\Rightarrow C=0$.\\
		$\Rightarrow s(t)=24{,}5t-4{,}9t^2$.\\
		Khi viên đạt đạt độ cao lớn nhất thì $v(t)=0\Leftrightarrow 24,5-9,8t=0\Leftrightarrow t=2{,}5$ (s).\\
		Quãng đường viên đạn đi từ lúc bắn lên cho tới khi rơi xuống đất là
		$$2\cdot s\left(2{,}5\right)=2\left(24{,}5\cdot2{,}5-4{,}9\cdot2{,}5^2\right)=61{,}25 \approx 61 \text{ (m).}$$
	}
\end{ex}
\begin{ex}%[2D4V1-6]
	Mực nước trong hồ chứa của nhà máy điện thủy triều thay đổi trong suốt một ngày do nước chảy ra khi thủy triều xuống và nước chảy vào khi thủy triều lên (như hình vẽ). Tốc độ thay đổi của mực nước được xác định bởi hàm số $h'(t)=\dfrac{1}{90}\left(t^2-17t+60\right)$, trong đó $t$ tính bằng giờ $\left(0\le t\le 24\right)$, $h'(t)$ tính bằng mét/giờ. Tại thời điểm $t=0$, mực nước trong hồ chứa cao $8$ m. Mực nước trong hồ cao nhất là bao nhiêu?
	\shortans[]{$20{,}8$}
	\loigiai{
		Ta có $h'(t)=\dfrac{1}{90}\left(t^2-17t+60\right)$.\\
		$\Rightarrow h(t)=\dfrac{1}{90}\displaystyle\int{\left(t^2-17t+60\right) \mathrm{\,d}t=}\dfrac{1}{90}\left(\dfrac{1}{3}{t^3}-\dfrac{17}{2}{t^2}+60t\right)+C$.\\
		$\Rightarrow h(t)=\dfrac{1}{90}\left(\dfrac{1}{3}{t^3}-\dfrac{17}{2}{t^2}+60t\right)+C$.\\
		Tại thời điểm $t=0$, mực nước trong hồ chứa cao $8$ nên $h(0)=8\Rightarrow C=8$.\\
		$\Rightarrow h(t)=\dfrac{1}{90}\left(\dfrac{1}{3}{t^3}-\dfrac{17}{2}{t^2}+60t\right)+8\rm\left(0\le t\le 24\right)$.\\
		Ta có $h'(t)=0\Leftrightarrow{t^2}-17t+60=0\Leftrightarrow \hoac{
			&t=5\\
			&t=12.
		}$\\
		Lập bảng biến thiên:
		\begin{center}
			\begin{tikzpicture}
				\tkzTabInit[nocadre=false,lgt=1.5,espcl=2.5,deltacl=0.6]
				{$x$/0.7,$h'(x)$/0.7,$h(x)$/2.5}{$0$,$5$,$12$,$24$}
				\tkzTabLine{,+,0,-,0,+,}  
				\tkzTabVar{-/$8$,+/$\dfrac{1019}{108}$,-/$\dfrac{44}{5}$,+/$\dfrac{104}{5}$}
			\end{tikzpicture}
		\end{center}
		Mực nước trong hồ cao nhất: $\dfrac{104}{5}=20{,}8$ m\\
	}
\end{ex}

\begin{ex}
	Gọi $h(t)$ là chiều cao của cây keo (tính theo mét) sau khi trồng $t$ năm. Biết rằng năm đầu tiên cây cao $1{,}5$ m, trong những năm tiếp theo, cây phát triển với tốc độ $h'(t)=\dfrac{1}{\sqrt[4]{t}}$ (mét/năm). Sau bao nhiêu năm cây cao được $3$ m (\textit{kết quả làm tròn tới hàng phần trăm}).
	\shortans[]{$2{,}73$}
	\loigiai{
		Ta có $h'(t)=\dfrac{1}{\sqrt[4]{t}}$.\\
		$\Rightarrow h(t)=\displaystyle\int{\dfrac{1}{\sqrt[4]{t}}} \mathrm{\,d}t=\displaystyle\int{t^{-\frac{1}{4}}} \mathrm{\,d}t=\dfrac{t^{-\frac{1}{4}+1}}{^{-\frac{1}{4}+1}}+C=\dfrac{4}{3}\sqrt[4]{t^3}+C$.\\
		$\Rightarrow h(t)=\dfrac{4}{3}\sqrt[4]{t^3}+C$.\\
		Năm đầu tiên cây cao $1{,}5$ m nên $h(1)=1{,}5\Leftrightarrow 1{,}5=\dfrac{4}{3}\sqrt[4]{1}+C\Rightarrow C=\dfrac{1}{6}$.\\
		$\Rightarrow h(t)=\dfrac{4}{3}\sqrt[4]{t^3}+\dfrac{1}{6}$.\\
		Cây cao được $3$ m nên $h(t)=3\Leftrightarrow\dfrac{4}{3}\sqrt[4]{t^3}+\dfrac{1}{6}=3\Leftrightarrow\sqrt[4]{t^3}=\dfrac{17}{8}\Rightarrow t\approx 2{,}73$ (m).
	}
\end{ex}

\begin{ex}%[2D4H1-6]
	Người ta bơm nước vào một bồn chứa, lúc đầu bồn không chứa nước, mức nước ở bồn chứa sau khi bơm phụ thuộc vào thời gian bơm nước theo một hàm số $h=h(t)$ trong đó $h$ tính bằng cm, $t$ tính bằng giây. Biết rằng $h'(t)=\sqrt[3]{2t}$ (cm/s). Mức nước ở bồn sau khi bơm được $13$ giây là bao nhiêu? (\textit{kết quả làm tròn tới hàng đơn vị}).
	\shortans[]{$23$}
	\loigiai{
		Ta có $h'(t)=\sqrt[3]{2t}$ (cm/s).\\
		$\Rightarrow h(t)=\displaystyle\int{\sqrt[3]{2t}} \mathrm{\,d}t=\sqrt[3]{2}\displaystyle\int{t^{\frac{1}{3}}} \mathrm{\,d}t=\sqrt[3]{2}\dfrac{t^{\frac{1}{3}+1}}{^{\frac{1}{3}+1}}+C=\dfrac{3\sqrt[3]{2}}{4}\sqrt[3]{t^4}+C$.\\
		$\Rightarrow h(t)=\dfrac{3}{4}\sqrt[3]{t^4}+C$.\\
		Lúc đầu bồn không chứa nước nên $h(0)=0\Rightarrow C=0$.\\
		$\Rightarrow h(t)=\dfrac{3}{4}\sqrt[3]{t^4}$ (cm).\\
		Mức nước ở bồn sau khi bơm được $13$ giây là $h(t)=\dfrac{3}{4}\sqrt[3]{13^4}\approx 23$ cm.
	}
\end{ex}

\begin{ex}%[2D4H1-6]
	Khi quan sát một đám vi khuẩn trong phòng thí nghiệm người ta thấy tại ngày thứ $t$ có số lượng là $N(t)$. Biết rằng $N'(t)=\dfrac{1500}{t}$ và tại ngày thứ nhất số lượng vi khuẩn là $5000$ con. Tính số lượng vi khuẩn tại ngày thứ $12$ (\textit{làm tròn đến hàng đơn vị}).
	\shortans[]{$8727$}
	\loigiai{
		Ta có $N'(t)=\dfrac{1500}{t}$.\\
		$\Rightarrow N(t)=\displaystyle\int{\dfrac{1500}{t}} \mathrm{\,d}t=1500 \cdot \ln t +C$.\\
		$\Rightarrow N(t)=1500\ln t +C$.\\
		Tại ngày thứ nhất số lượng vi khuẩn là $5000$ con nên\\
		$N(1)=5000 \Rightarrow C= 5000$
		$\Rightarrow N(t)=1500\ln t + 5000$ (con).\\
		Số lượng vi khuẩn tại ngày thứ $12$ là
		$N(12)=1500\ln 12 + 5000 \approx 8727$ con.
	}
\end{ex}

\begin{ex}%[2D4H1-6]
	Vi khuẩn HP (Helicobacter pylori) gây đau dạ dày, tại ngày thứ $t$ với số lượng là $F(t)$. Biết $F'(t)=\dfrac{600}{t}$ và ban đầu bệnh nhân có $2000$ con vi khuẩn. Sau $15$ ngày bệnh nhân phát hiện ra bị bệnh. Hỏi khi đó có bao nhiêu con vi khuẩn trong dạ dày (lấy xấp xỉ tới hàng đơn vị)? Biết rằng nếu phát hiện sớm khi số lượng không vượt quá $4000$ con thì bệnh nhân sẽ được cứu chữa.
	\shortans[]{$3625$}
	\loigiai{
		Ta có 
		\begin{eqnarray*}
			& & F'(t)=\dfrac{600}{t}\\
			&\Leftrightarrow & F(t)=\displaystyle\int{\dfrac{600}{t}} \mathrm{\,d}t\\
			&\Leftrightarrow & F(t)=600 \cdot \ln t +C.
		\end{eqnarray*}
		Tại ngày thứ nhất số lượng vi khuẩn là $2000$ con nên\\
		$ F(1)=2000 \Rightarrow C= 2000$.\\
		$\Rightarrow F(t)=600\ln t + 2000$\\
		Số lượng vi khuẩn sau $15$ ngày là	$F(15)=600\ln 15 + 2000 \approx 3625$ (con).
	}
\end{ex}
\Closesolutionfile{ans}
% \indapan{6}{ans/ans2-C4B1CD4-D2-KQ}
% \subsection{NGUYÊN HÀM HÀM ẨN}

\subsubsection*{Cần nhớ các công thức đạo hàm của hàm hợp}
\begin{itemize}[\color{blue}\faPencilSquare]
	\item $\int{f'(x)\mathrm{d}x}=f(x)+C$
	\item $f'(x)\cdot g(x)+f(x)\cdot g'(x)=\left[f(x)\cdot g(x)\right]'$
	\item $\dfrac{f'(x)\cdot g(x)-f(x)\cdot g'(x)}{g^2(x)} =\left[\dfrac{f(x)}{g(x)}\right]'$
	\item $\dfrac{f'(x)}{f(x)}=\left[\ln f(x) \right]'$
	\item $-\dfrac{f'(x)}{f^2(x)}=\left[ \dfrac{1}{f(x)} \right]'$
	\item $-\dfrac{f'(x)}{f^n(x)}=\left[ \dfrac{1}{(n-1)\left[ f(x) \right]^{n-1}} \right]'$
	\item $n\cdot f'(x)\cdot f^{n-1}(x)=\left[ f^n(x) \right]'$
	\item $\dfrac{f'(x)}{\sqrt{f(x)}}=\left[ 2\sqrt{f(x)} \right]'$
\end{itemize}

\begin{dang}{~}
	\subsubsection{Điều kiện hàm ẩn có dạng}
	$$\left[ \begin{aligned}
			 & f'(x)=g(x)\cdot h\left[ f(x) \right]  \\
			 & f'(x)\cdot h\left[ f(x) \right]=g(x).
		\end{aligned} \right.$$
	\textbf{Phương pháp giải}
	\begin{itemize}[\color{blue}\faPencilSquareO]
		\item $\dfrac{f'(x)}{h[f(x)]}=g(x) \Leftrightarrow \displaystyle\int \dfrac{f'(x)}{h[f(x)]}\mathrm{d}x =\int  {g(x)}\mathrm{d}x \Leftrightarrow \int\dfrac{\mathrm{d}\left[ f(x) \right]}{h\left[ f(x) \right]} =\int  {g(x)\mathrm{d}x}$.
		\item $f'(x)h[f(x)]=g(x)
			      \Leftrightarrow
			      \displaystyle\int f'(x)h[f(x)]\mathrm{d}x=\int g(x)\mathrm{d}x
			      \Leftrightarrow
			      \int h[f(x)]\mathrm{d}\left[ f'(x) \right]=\int g(x)\mathrm{d}x$.
	\end{itemize}
	Chú ý: Ngoài việc nguyên hàm hai vế, ta có thể lấy tích phân hai vế (tùy câu hỏi của bài toán)
	\subsubsection{Điều kiện hàm ẩn có dạng}
	$$u(x)f'(x)+u'(x)f(x)=h(x)$$
	\textbf{Phương pháp giải}
	Dễ dàng thấy rằng $u(x)f'(x)+u'(x)f(x)=[u(x)f(x)]'$.\\
	Do dó $u(x)f'(x)+u'(x)f(x)=h(x) \Leftrightarrow [u(x)f(x)]'=h(x)$.\\
	Suy ra $u(x)f(x)=\displaystyle\int   h(x)\mathrm{d}x$.\\
	Từ đây ta dễ dàng tính được $f(x)$.
\end{dang}

%PHẦN I. Câu trắc nghiệm nhiều phương án lựa chọn. Mỗi câu hỏi thí sinh chỉ chọn một phương án.
% \TN
\Opensolutionfile{ans}[ans/ans-LC-2-C4B1CD3_1-8]

%%%==============EX_1============%%%
\begin{ex}%[2D4V1-3]
	Cho hàm số $f(x)$ thỏa mãn $f\left(\dfrac{\pi}{4} \right)=0$ và $f'(x)\sin^2\dfrac{x}{2}\cos^2\dfrac{x}{2}=1$. Tính $f\left(\dfrac{\pi}{2} \right)$.
	\choice
	{$f\left(\dfrac{\pi}{2} \right)=1$}
	{$f\left(\dfrac{\pi}{2} \right)=-1$}
	{$f\left(\dfrac{\pi}{2} \right)=2$}
	{\True $f\left(\dfrac{\pi}{2} \right)=4$}
	\loigiai{
		Ta có
		$\begin{aligned}[t]
				            & \quad f'(x)\sin^2\dfrac{x}{2}\cos^2\dfrac{x}{2}=1             \\
				\Rightarrow & \quad f'(x)=\dfrac{1}{\sin^2\dfrac{x}{2}\cos^2\dfrac{x}{2}}   \\
				\Rightarrow & \quad f'(x)=\dfrac{1}{\tfrac{1}{4}\sin^2}x                    \\
				\Rightarrow & \quad f(x)=4\displaystyle\int  \dfrac{1}{\sin^2x}\mathrm{d}x.
			\end{aligned}$\\
		Tìm được $f(x)=-4\cot x+C$.\\
		Với $f\left(\dfrac{\pi}{4} \right)=0$ thì $C=4$.\\
		Suy ra $f(x)=-4\cot x+4$.\\
		Vậy $f\left(\dfrac{\pi}{2}\right) =-4\cot\dfrac{\pi}{2}+4=4$.
	}
\end{ex}
%%%==============EX_2============%%%
\begin{ex}%[2D4V1-2]
	Cho hàm số $y=f(x)$ thỏa mãn $f'(x)\cdot f(x)=x^4+x^2$. Biết $f(0)=2$. Tính $f^2(2)$.
	\choice
	{$f^2(2)=\dfrac{313}{15}$}
	{\True $f^2(2)=\dfrac{332}{15}$}
	{$f^2(2)=\dfrac{324}{15}$}
	{$f^2(2)=\dfrac{323}{15}$}
	\loigiai{
		Ta có
		$\begin{aligned}[t]
				f'(x)\cdot f(x)=x^4+x^2
				 & \Leftrightarrow f(x)\mathrm{d}f(x)=(x^4+x^3)\mathrm{d}x                                    \\
				 & \Leftrightarrow \displaystyle\int f'(x)\cdot f(x)\mathrm{d}x =\int{(x^4+x^2)\mathrm{d}x}+C \\
				 & \Rightarrow \dfrac{f^2(x)}{2} =\dfrac{x^5}{5}+\dfrac{x^3}{3}+C.
			\end{aligned}$\\
		Do
		$\begin{aligned}[t]
				f(0)=2 & \Rightarrow \dfrac{f^2(0)}{2}=\dfrac{0^5}{5}+\dfrac{0^3}{3}+C
				       & \Rightarrow C=2.
			\end{aligned}$\\
		Vậy $f^2(2)=2\left(\dfrac{32}{5}+\dfrac{8}{3}+2 \right) =\dfrac{332}{15}$.
	}
\end{ex}
%%%==============EX_3============%%%
\begin{ex}%[2D4V1-2]
	Cho hàm số $y=f(x)$ có đạo hàm liên tục trên đoạn $[-2;1]$ thỏa mãn $f(0)=3$ và $\left(f(x)\right)^2\cdot f'(x)=3x^2+4x+2$. Giá trị $f(1)$ là
	\choice
	{$2\sqrt[3]{42}$}
	{$2\sqrt[3]{15}$}
	{\True $\sqrt[3]{42}$}
	{$\sqrt[3]{15}$}
	\loigiai{
		Ta có $\left(f(x)\right)^2\cdot f'(x)=3x^2+4x+2$ (*).\\
		Lấy nguyên hàm 2 vế của phương trình trên ta được
		\begin{eqnarray*}
			\displaystyle\int  f(x)^2\cdot f'(x)\mathrm{d}x =\int\left(3x^2+4x+2\right)\mathrm{d}x
			& \Leftrightarrow & \int \left(f(x)\right)^2\mathrm{d}f(x) =x^3+2x^2+2x+C\\
			& \Leftrightarrow & \dfrac{\left(f(x)\right)^3}{3} =x^3+2x^2+2x+C\\
			& \Leftrightarrow & \left(f(x)\right)^3 =3(x^3+2x^2+2x+C) \quad(1).
		\end{eqnarray*}
		Theo đề bài
		$\begin{aligned}[t]
				f(0)=3 & \overset{(1)}{\Rightarrow} \left(f(0)\right)^3 =3(0^3+2.0^2+2\cdot0+C) \\
				       & \Leftrightarrow 27=3C                                                  \\
				       & \Leftrightarrow C=9.
			\end{aligned}$\\
		Suy ra
		$\begin{aligned}[t]
				            & \left(f(x)\right)^3=3(x^3+2x^2+2x+9)
				\Rightarrow & \ f(x)=\sqrt[3]{3(x^3+2x^2+2x+9)}.
			\end{aligned}$\\
		Vậy $f(1)=\sqrt[3]{42}$.
	}
\end{ex}
%%%==============EX_4============%%%
\begin{ex}%[2D4C1-2]
	Cho hàm số $f(x)$ thỏa mãn $f(2)=-\dfrac{1}{3}$ và $f'(x)=x\left[f(x)\right]^2$ với mọi $x\in \mathbb{R}$. Giá trị của $f(1)$ bằng
	\choice
	{\True $f(1)=-\dfrac{2}{3}$}
	{$f(1)=-\dfrac{2}{9}$}
	{$f(1)=-\dfrac{7}{6}$}
	{$f(1)=-\dfrac{11}{6}$}
	\loigiai{
		Từ hệ thức đề cho: $f'(x)=x\left[f(x)\right]^2$ (1), suy ra $f'(x)\ge 0$ với mọi $x\in [1;2]$ \\
		Do đó $f(x)$ là hàm không giảm trên đoạn $[1;2]$, ta có $f(x)\le f(2) < 0$ với mọi $x\in [1;2]$
		\begin{enumerate}[\color{blue}\bf Cách 1.]
			\item Lấy nguyên hàm\\
			      Ta có
			      $\begin{aligned}[t]
					      f'(x)=x\left[f(x)\right]^2
					       & \Rightarrow \dfrac{f'(x)}{\left[f(x)\right]^2}=x    \\
					       & \Rightarrow \left(-\dfrac{1}{f(x)} \right)'=x       \\
					       & \Rightarrow \left(\dfrac{1}{f(x)} \right)'=-x       \\
					       & \Rightarrow \dfrac{1}{f(x)}=\displaystyle\int(-x)dx \\
					       & \Rightarrow \dfrac{1}{f(x)}=-\dfrac{x^2}{2}+C.
				      \end{aligned}$\\
			      Mà
			      $\begin{aligned}[t]
					      f(2)=-\dfrac{1}{3}
					       & \Rightarrow \dfrac{1}{f(2)}=-2+C              \\
					       & \Leftrightarrow \dfrac{1}{-\tfrac{1}{3}}=-2+C \\
					       & \Rightarrow C=-1.
				      \end{aligned}$\\
			      Tìm được $\dfrac{1}{f(x)}=-\dfrac{x^2}{2}-1$.\\
			      Cho nên $\dfrac{1}{f(1)}=-\dfrac{1}{2}-1 \Leftrightarrow f(1)=-\dfrac{2}{3}$.
			\item Chia 2 vế hệ thức $(1)$ cho $\left[f(x)\right]^2$, ta được $ \dfrac{f'(x)}{\left[f(x)\right]^2}=x,\forall x\in[1;2]$. \\
			      Lấy tích phân 2 vế trên đoạn $[1;2]$ hệ thức vừa tìm được, ta được:
			      \begin{align*}
				      \displaystyle\int\limits_1^2\dfrac{f'(x)}{\left[f(x)\right]^2}\mathrm{d}x =\displaystyle\int\limits_1^2x\mathrm{d}x
				       & \Rightarrow \int\limits_1^2\dfrac{1}{\left[f(x)\right]^2}\mathrm{d}f(x)=\dfrac{3}{2} \\
				       & \Leftrightarrow \dfrac{-1}{f(x)} \bigg|_1^2=\dfrac{3}{2}                             \\
				       & \Leftrightarrow \dfrac{1}{f(1)}-\dfrac{1}{f(2)}=\dfrac{3}{2}                         \\
				       & \Leftrightarrow \dfrac{1}{f(1)}=\dfrac{1}{f(2)}+\dfrac{3}{2}                         \\
				       & \Leftrightarrow f(1)=\dfrac{2f(2)}{2+3f(2)}.
			      \end{align*}
			      Với $f(2)=-\dfrac{1}{3}$ thì $f(1)=\dfrac{2\cdot\left(-\frac{1}{3}\right)}{2+3\cdot\left(-\frac{1}{3}\right)}=-\dfrac{2}{3}$.
		\end{enumerate}
	}
\end{ex}
%%%==============EX_5============%%%
\begin{ex}%[2D4V1-2]
	Cho hàm số $f(x)$ thỏa mãn $f(2)=-\dfrac{1}{25}$ và $f'(x)=4x^3\left[f(x)\right]^2$ với mọi $x\in\mathbb{R}$. Giá trị của $f(1)$ bằng
	\choice
	{$-\dfrac{391}{400}$}
	{$-\dfrac{1}{40}$}
	{$-\dfrac{41}{400}$}
	{\True $-\dfrac{1}{10}$}
	\loigiai{
		Ta có
		$\begin{aligned}[t]
				f'(x)=4x^3\left[f(x)\right]^2
				 & \Rightarrow-\dfrac{f'(x)}{\left[f(x)\right]^2}=-4x^3 \\
				 & \Rightarrow \left[\dfrac{1}{f(x)}\right]'=-4x^3      \\
				 & \Rightarrow \dfrac{1}{f(x)}=-x^4+C.
			\end{aligned}$\\
		Với $f(2)=-\dfrac{1}{25}$ thì $\dfrac{1}{f(2)}=-2^4+C \Leftrightarrow -25=-16+C \Leftrightarrow C=-9$. \\
		Suy ra $f(x)=-\dfrac{1}{x^4+9}$.\\
		Vậy $f(1)=-\dfrac{1}{10}$.
	}
\end{ex}
%%%==============EX_6============%%%
\begin{ex}%[2D4V1-2]
	Cho hàm số $f(x)$ thỏa mãn $f(2)=-\dfrac{1}{5}$ và $f'(x)=x^3\left[f(x)\right]^2$ với mọi $x\in\mathbb{R}$. Giá trị của $f(1)$ bằng
	\choice
	{$-\dfrac{4}{35}$}
	{$-\dfrac{71}{20}$}
	{$-\dfrac{79}{20}$}
	{\True $-\dfrac{4}{5}$}
	\loigiai{
		Ta có
		$\begin{aligned}[t]
				f'(x)=x^3\left[f(x)\right]^2
				 & \Rightarrow \dfrac{f'(x)}{f^2(x)}=x^3                                                                                                   \\
				 & \Rightarrow \displaystyle\int\limits_1^2\dfrac{f'(x)}{f^2(x)}\mathrm{d}x =\int\limits_1^2x^3\mathrm{d}x                                 \\
				 & \Leftrightarrow -\dfrac{1}{f(x)} \bigg|_1^2 =\dfrac{1}{4}\cdot(2^4-1^4)                                                                 \\
				 & \Leftrightarrow -\dfrac{1}{f(2)}+\dfrac{1}{f(1)} =\dfrac{15}{4}                                                                         \\
				 & \Leftrightarrow \dfrac{1}{f(1)}=\dfrac{4+15f(2)}{4f(2)}                                                                                 \\
				 & \Leftrightarrow f(1)=\dfrac{4f(2)}{4+15f(2)}=\dfrac{4\cdot\left(-\frac{1}{5}\right)}{4+15\cdot\left(-\frac{1}{5}\right)}=-\dfrac{4}{5}.
			\end{aligned}$\\
		Vậy $f(1)=-\dfrac{4}{5}$.
	}
\end{ex}
%%%==============EX_7============%%%
\begin{ex}%[2D4V1-2]
	Cho hàm số $y=f(x)$ thỏa mãn $f(2)=-\dfrac{4}{19}$ và $f'(x)=x^3f^2(x)\, \forall x\in \mathbb{R}$. Giá trị của $f(1)$ bằng
	\choice
	{$-\dfrac{2}{3}$}
	{$-\dfrac{1}{2}$}
	{\True $-1$}
	{$-\dfrac{3}{4}$}
	\loigiai{
		Ta có
		$\begin{aligned}[t]
				f'(x)=x^3f^2(x) & \Leftrightarrow \dfrac{f'(x)}{f^2(x)}=x^3                                          \\
				                & \Rightarrow \displaystyle\int\dfrac{f'(x)}{f^2(x)}\mathrm{d}x =\int x^3\mathrm{d}x \\
				                & \Leftrightarrow -\dfrac{1}{f(x)} =\dfrac{x^4}{4}+C.
			\end{aligned}$\\
		Với $f(2)=-\dfrac{4}{9}$ thì $\dfrac{19}{4} =\dfrac{16}{4}+C \Rightarrow C=\dfrac{3}{4}$.\\
		Tìm được $f(x)=-\dfrac{4}{x^4+3}$.\\
		Vậy $f(1)=-1$.
	}
\end{ex}
%%%==============EX_8============%%%
\begin{ex}%[2D4V1-4]
	Cho hàm số $f(x)>0$ xác định và liên tục trên $\mathbb{R}$ đồng thời thỏa mãn $f(0)=\dfrac{1}{2}$, $f'(x)=-\mathrm{e}^xf^2(x),\, \forall x\in \mathbb{R}$. Tính giá trị của $f(\ln 2)$.
	\choice
	{$f(\ln 2)=\dfrac{1}{4}$}
	{\True $f(\ln 2)=\dfrac{1}{3}$}
	{$f(\ln 2)=\ln 2+\dfrac{1}{2}$}
	{$f(\ln 2)=\ln ^22+\dfrac{1}{2}$}
	\loigiai{
		Ta có
		$\begin{aligned}[t]
				f'(x)=-\mathrm{e}^xf^2(x)
				 & \Leftrightarrow \dfrac{f'(x)}{f^2(x)}=-\mathrm{e}^x\ (\text{do } f(x)>0)                          \\
				 & \Leftrightarrow \displaystyle\int\dfrac{f'(x)}{f^2(x)}\mathrm{d}x =\int(-\mathrm{e}^x)\mathrm{d}x \\
				 & \Rightarrow -\dfrac{1}{f(x)}=-\mathrm{e}^x+C                                                      \\
				 & \Rightarrow f(x)=\dfrac{1}{\mathrm{e}^x-C}.
			\end{aligned}$\\
		Với $f(0)=\dfrac{1}{2}$ thì $\dfrac{1}{\mathrm{e}^0-C} =\dfrac{1}{2} \Rightarrow C=-1$.\\
		Suy ra $f(x)=\dfrac{1}{\mathrm{e}^x+1}$.\\
		Vậy $f(\ln 2) =\dfrac{1}{\mathrm{e}^{\ln 2}+1}=\dfrac{1}{3}$.
	}
\end{ex}
%%%==============EX_9============%%%
\begin{ex}%[2D4V1-2]
	Cho hàm số $f(x)\ne0$ thỏa mãn điều kiện $f'(x)=(2x+3)f^2(x)$ và $f(0)=-\dfrac{1}{2}$. Biết rằng tổng $f(1)+f(2)+f(3)+\cdots+f(2024)+f(2025) =\dfrac{a}{b}$ với $\left(a\in \mathbb{Z}, b\in \mathbb{N}^{*} \right)$ và $\dfrac{a}{b}$ là phân số tối giản. Mệnh đề nào sau đây đúng?
	\choice
	{$\dfrac{a}{b} <-1$}
	{$\dfrac{a}{b} > 1$}
	{$a+b=1010$}
	{\True $b-a=1519$}
	\loigiai{
		Ta có
		$\begin{aligned}[t]
				f'(x)=(2x+3)f^2(x)
				 & \Leftrightarrow \dfrac{f'(x)}{f^2(x)}=2x+3                                             \\
				 & \Leftrightarrow \displaystyle\int\dfrac{f'(x)}{f(x)}\mathrm{d}x =\int(2x+3)\mathrm{d}x \\
				 & \Leftrightarrow -\dfrac{1}{f(x)}=x^2+3x+C.
			\end{aligned}$\\
		Với $f(0)=-\dfrac{1}{2}$, thì $-\dfrac{1}{-\frac{1}{2}}=0^2+3\cdot0+C \Rightarrow C=2$\\
		Suy ra $f(x)=-\dfrac{1}{(x+1)(x+2)}=\dfrac{1}{x+2}-\dfrac{1}{x+1}$\\
		Ta có: $\left\{\begin{aligned}
				 & f(1)=\dfrac{1}{3}-\dfrac{1}{2}          \\
				 & f(2)=\dfrac{1}{4}-\dfrac{1}{3}          \\
				 & f(3)=\dfrac{1}{5}-\dfrac{1}{4}          \\
				 & \vdots                                  \\
				 & f(2025)=\dfrac{1}{2026}-\dfrac{1}{2025} \\
			\end{aligned} \right.$ \\
		Tổng $f(1)+f(2)+f(3)+\cdots+f(2024)+f(2025) =-\dfrac{1}{2}+\dfrac{1}{2026}=-\dfrac{506}{1013}$\\
		Do $a\in \mathbb{Z}, b\in \mathbb{N}^{*}
			\Rightarrow a=-506, b=1013\Rightarrow b-a=1519$.
	}
\end{ex}
%%%==============EX_10============%%%
\begin{ex}%[2D4V1-2]
	Cho hàm số $y=f(x)$ đồng biến trên $(0;+\infty)$; $y=f(x)$ liên tục, nhận giá trị dương trên $(0;+\infty)$ và thỏa mãn $f(3)=\dfrac{4}{9}$ và $\left[f'(x)\right]^2=xf(x)$. Tính $f(8)$.
	\choice
	{\True $f(8)=\dfrac{43-24\sqrt{3}}{9}$}
	{$f(8)=\dfrac{43+24\sqrt{3}}{9}$}
	{$f(8)=\dfrac{43-\sqrt{3}}{3}$}
	{$f(8)=\dfrac{43+\sqrt{3}}{3}$}
	\loigiai{
		Với $\forall x\in (0;+\infty)$ thì $y=f(x) > 0$; $x+1 > 0$\\
		Hàm số $y=f(x)$ đồng biến trên $(0;+\infty)$ nên $f'(x)\ge0, \forall x\in (0;+\infty)$\\
		Ta có
		$\begin{aligned}[t]
				\left[f'(x)\right]^2=xf(x)
				 & \Rightarrow f'(x)=\sqrt{xf(x)}                                  \\
				 & \Rightarrow \dfrac{f'(x)}{\sqrt{f(x)}}=\sqrt{x}                 \\
				 & \Rightarrow 2\left(\sqrt{f(x)}\right)'=\sqrt{x}                 \\
				 & \Rightarrow \left(\sqrt{f(x)}\right)'=\dfrac{1}{2}\sqrt{x}      \\
				 & \Rightarrow \sqrt{f(x)}=\dfrac{1}{2}\displaystyle\int\sqrt{x}dx \\
				 & \Rightarrow \sqrt{f(x)}=\dfrac{1}{3}\sqrt{x^3}+C.
			\end{aligned}$\\
		Với $f(3)=\dfrac{4}{9}$, thì $\sqrt{f(3)}=\dfrac{1}{3}\cdot\sqrt{3^3}+C \Leftrightarrow \dfrac{2}{3}=\sqrt{3}+C \Leftrightarrow C=\dfrac{2-\sqrt{3}}{3}$.\\
		Tìm được $\sqrt{f(x)}=\dfrac{1}{3}\sqrt{x^3}+\dfrac{2-3\sqrt{3}}{3}$
		$\Rightarrow f(x)=\left(\dfrac{1}{3}\sqrt{x^3}+\dfrac{2-3\sqrt{3}}{3}\right)^2$.\\
		Vậy $f(8)=\dfrac{43-24\sqrt{3}}{9}$.
	}
\end{ex}
%%%==============EX_11============%%%
\begin{ex}%[2D4V1-4]%[2D4V1-4]%[2D4H1-4]%[2D4H1-4]
	Cho hàm số $f(x)>0$ với mọi $x\in\mathbb{R}$, $f(0)=1$ và $f(x)=\sqrt{x}\cdot f'(x)$ với mọi $x\in\mathbb{R}$. Mệnh đề nào dưới đây đúng?
	\choice
	{$f(3)<2$}
	{$2<f(3)<4$}
	{\True $f(3)>6$}
	{$4<f(3)<6$}
	\loigiai{
	Ta có
	$\begin{aligned}[t]
			f(x)=\sqrt{x}\cdot f'(x)
			 & \Rightarrow \dfrac{f'(x)}{f(x)}=\dfrac{1}{\sqrt{x}}                  \\
			 & \Rightarrow \ln f(x)=\displaystyle\int\dfrac{1}{\sqrt{x}}\mathrm{d}x \\
			 & \Leftrightarrow \ln f(x)=2\sqrt{x}+C                                 \\
			 & \Leftrightarrow f(x)=\mathrm{e}^{2\sqrt{x}+C}.
		\end{aligned}$\\
	Với $f(0)=1$ thì $\ln f(0)=2\sqrt{0}+C \Leftrightarrow C=0$.\\
	Suy ra $f(x)=\mathrm{e}^{2\sqrt{x}}$.\\
	Vậy $f(3)=\mathrm{e}^{2\sqrt{3}}>6$.
	}
\end{ex}
\begin{ex}%[2D4V1-2]
	Cho hàm số $ f(x)$ có đạo hàm cấp hai trên đoạn $\left[0;1\right]$ đồng thời thỏa mãn các điều kiện $f'(0)=-1$, $f'(x)<0$, $\left[f'(x)\right]^2=f''(x)$, $\forall x\in\left[0;1\right]$. Giá trị $ f'(2)$ thuộc khoảng
	\choice
	{$(2;3)$}
	{\True $(-2;0)$}
	{$(0;2)$}
	{$(-3;-2)$}
	\loigiai
	{
		Ta có
		\begin{align*}
			\left[f'(x)\right]^2=f''(x) \Leftrightarrow\dfrac{f''(x)}{\left[f'(x)\right]^2}=1\Leftrightarrow-\left(\dfrac{1}{f'(x)}\right)'=1\Rightarrow\dfrac{1}{f'(x)}=-\displaystyle\int{\mathrm{d} x} \Leftrightarrow\dfrac{1}{f'(x)}=-x+C.
		\end{align*}
		Mà $f'(0)=-1\Rightarrow C=-1$ suy ra
		$$\dfrac{1}{f'(x)}=-x-1\Rightarrow f'(x)=-\dfrac{1}{x+1} \Rightarrow{f}'(2)=-\dfrac{1}{3}.$$
	}
\end{ex}

\begin{ex}%[2D4V2-2]
	Cho hàm số $ f(x)$ đồng biến có đạo hàm đến cấp hai trên đoạn $\left[0;2\right]$ và thỏa mãn $\left[f(x)\right]^2-f(x)\cdot f''(x)+\left[f'(x)\right]^2=0$. Biết $ f(0)=1$, $ f(2)={\mathrm{e}}^6$. Khi đó $ f(1)$ bằng
	\choice
	{$\mathrm{e}^{\tfrac{3}{2}}$}
	{$\mathrm{e}^3$}
	{\True $\mathrm{e}^{\tfrac{5}{2}}$}
	{$\mathrm{e}^2$}
	\loigiai
	{
	Theo đề bài, ta có
	\begin{align*}
		\left[f(x)\right]^2-f(x)\cdot f''(x)+\left[f'(x)\right]^2=0
		 & \Rightarrow\dfrac{f(x)\cdot f''(x)-\left[f'(x)\right]^2}{\left[f(x)\right]^2}=1 \\
		 & \Rightarrow{\left[\dfrac{f'(x)}{f(x)}\right]'}=1                                \\& \Rightarrow\dfrac{f'(x)}{f(x)}=x+C\\
		 & \Rightarrow\ln f(x)=\dfrac{x^2}{2}+Cx+D.
	\end{align*}
	Mà $\heva{
			& f(0)=1\\
			& f(2)=\mathrm{e}^6} \Leftrightarrow\heva{
			& C=2\\
			& D=0.}$\\
	Suy ra $f(x)=\mathrm{e}^{\tfrac{x^2}{2}+2x}\Rightarrow f(1)=\mathrm{e}^{\tfrac{5}{2}}$.}
\end{ex}
\begin{ex}%[2D4V1-2]
	Cho hàm số $ f(x)$ thỏa mãn $(f'(x))^2+f(x)\cdot f''(x)=x^3-2x$, $\forall x\in\mathbb{R}$ và \break $ f(0)=f'(0)=1$. Giá trị của $ T=f^2(2)$ bằng
	\choice
	{$\dfrac{43}{30}$}
	{$\dfrac{16}{15}$}
	{\True $\dfrac{43}{15}$}
	{$\dfrac{26}{15}$}
	\loigiai
	{
		Ta có
		\begin{align*}
			\left( f'(x)\right)^2+f(x)\cdot f''(x)=x^3-2x & \Leftrightarrow \left( f(x)\cdot f'(x)\right)'=x^3-2x \\& \Rightarrow f(x)\cdot f'(x)=\displaystyle\int{(x^3-2x})\mathrm{\,d} x\\&\Rightarrow f(x)\cdot f'(x)=\dfrac{1}{4}{x^4}-x^2+C.
		\end{align*}
		Từ $ f(0)=f'(0)=1$ suy ra $C=1$. Do đó $ f(x)\cdot f'(x)=\dfrac{1}{4}{x^4}-x^2+1$.\\
		Lại có
		\begin{align*}
			2f(x)\cdot f'(x)=\dfrac{1}{2}{x^4}-2x^2+2 & \Leftrightarrow \left( f^2(x)\right) '=\dfrac{1}{2}{x^4}-2x^2+2 \\& \Rightarrow{f^2}(x)=\displaystyle\int{\left(\dfrac{1}{2}{x^4}-2x^2+2\right)}\mathrm{\,d} x\\& \Rightarrow{f^2}(x)=\dfrac{1}{10}{x^5}-\dfrac{2}{3}{x^3}+2x+C.
		\end{align*}
		Vì $ f(0)=1$ nên $C=1$. Do đó $f^2(x)=\dfrac{1}{10}{x^5}-\dfrac{2}{3}{x^3}+2x+1$.\\
		Vậy $T=\dfrac{43}{15}$.}
\end{ex}

\begin{ex}%[2D4V1-2]
	Cho hàm số $ f(x)$ thỏa mãn $\left[f'(x)\right]^2+f(x)\cdot f''(x)=2x^2-x+1$, $\forall x\in\mathbb{R}$ và $ f(0)=f'(0)=3$. Giá trị của $\left[f(1)\right]^2$ bằng
	\choice
	{\True $ 28$}
	{$ 22$}
	{$\dfrac{19}{2}$}
	{$ 10$}
	\loigiai
	{
		Ta có $\left[f(x){f}'(x)\right]'=\left[f'(x)\right]^2+f(x)\cdot f''(x)$.\\
		Do đó theo giả thiết ta được $\left[f(x){f}'(x)\right]'=2x^2-x+1$.\\
		Suy ra $f(x){f}'(x)=\dfrac{2}{3}{x^3}-\dfrac{x^2}{2}+x+C$.\\
		Hơn nữa $ f(0)=f'(0)=3$ suy ra $ C=9$.\\
		Tương tự vì $\left[f^2(x)\right]'=2f(x){f}'(x)$ nên $\left[f^2(x)\right]'=2\left(\dfrac{2}{3}{x^3}-\dfrac{x^2}{2}+x+9\right)$.\\
		Suy ra $f^2(x)=\displaystyle\int{2\left(\dfrac{2}{3}{x^3}-\dfrac{x^2}{2}+x+9\right)\mathrm{\,d}x}=\dfrac{1}{3}{x^4}-\dfrac{x^3}{3}+x^2+18x+C$.\\
		Vì $ f(0)=3$ nên $C=9$ suy ra $f^2(x)=\dfrac{1}{3}{x^4}-\dfrac{x^3}{3}+x^2+18x+9$.\\
		Do đó $\left[f(1)\right]^2=28$.
	}
\end{ex}

\Closesolutionfile{ans}
% \indapan{10}{ans/ans-LC-2-C4B1CD3_1-8}
% \TNSA
\Opensolutionfile{ans}[ans/ans-KQ-2-C4B1CD3]
\begin{ex}%[2D4H1-2]
	Cho hàm số $ y=f(x)$ thỏa mãn $y'=x{y^2}$ và $ f\left(-1\right)=1$. Tính giá trị $f(2)$. (\textit{Kết quả làm tròn đến hàng phần mười}).
	\shortans{$20{,}1$}
	\loigiai{
	Ta có $y'=x{y^2} \Rightarrow\dfrac{y'}{y}=x^2\Rightarrow\displaystyle\int{\dfrac{y'}{y}\mathrm{\,d}x}=\displaystyle\int{x^2\mathrm{\,d}x}\Leftrightarrow\ln y=\dfrac{x^3}{3}+C\Leftrightarrow y=\mathrm{e}^{\tfrac{x^3}{3}+C}$.\\
	Theo giả thiết $ f(-1)=1$ nên $\mathrm{e}^{-\tfrac{1}{3}+C}=1\Leftrightarrow C=\dfrac{1}{3}$.\\
	Do đó	 $ y=f(x)= \mathrm{e}^{\tfrac{x^3}{3}+\tfrac{1}{3}}$.\\
	Vậy $f(2)=\mathrm{e}^3\approx 20{,}1$.}
\end{ex}

\begin{ex}%[2D4V2-2]
	Cho hàm số $ f(x)\ne 0$, liên tục trên đoạn $\left[1;2\right]$ và thỏa mãn $ f(1)=3$, \break $x^2\cdot f'(x)=f^2(x)$ với $\forall x\in\left[1;2\right]$. Tính $f(2)$.
	\shortans{$-6$}
	\loigiai{
		Ta có
		\begin{align*}
			x^2\cdot f'(x)=f^2(x) & \Rightarrow\dfrac{f'(x)}{f^2(x)}=\dfrac{1}{x^2} \Rightarrow{\left(-\dfrac{1}{f(x)}\right)'}=\dfrac{1}{x^2} \\& \Rightarrow\displaystyle\int\limits_1^2\left(-\dfrac{1}{f(x)}\right)'\mathrm{\,d} x=\displaystyle\int\limits_1^2\dfrac{1}{x^2}\mathrm{\,d} x\\&\Rightarrow\left.\left(-\dfrac{1}{f(x)}\right)\right|_1^2=-\left.\dfrac{1}{x}\right|_1^2\\& \Rightarrow-\dfrac{1}{f(2)}+\dfrac{1}{f(1)}=-\dfrac{1}{2}+1\\& \Rightarrow-\dfrac{1}{f(2)}+\dfrac{1}{f(1)}=\dfrac{1}{2}.
		\end{align*}
		Vì $f(1)=3\Rightarrow-\dfrac{1}{f(2)}+\dfrac{1}{3}=\dfrac{1}{2}\Rightarrow f(2)=-6$.}
\end{ex}
\begin{ex}%Câu 7%[2D4C1-2]
	Cho hàm số $ y=f(x)$ thỏa mãn $ f(x)<0$, $\forall x>0$ và có đạo hàm $f'(x)$ liên tục trên khoảng $\left( 0;+\infty\right) $ thỏa mãn $f'(x)=(2x+1){f^2}(x)$, $\forall x>0$ và $ f(1)=-\dfrac{1}{2}$. Tính giá trị của biểu thức $ T=f(1)+f(2)+\ldots +f\left(2023\right)+f\left(2024\right)$. (\textit{Kết quả làm tròn đến hàng đơn vị}).
	\shortans{$-1$}
	\loigiai{
		Ta có
		\begin{align*}
			f'(x)=(2x+1){f^2}(x)
			 & \Rightarrow \dfrac{f'(x)}{f^2(x)}=2x+1                                                              \\
			 & \Rightarrow\displaystyle\int\dfrac{f'(x)}{f^2(x)}\mathrm{\,d}x=\displaystyle\int(2x+1)\mathrm{\,d}x \\&\Rightarrow-\dfrac{1}{f(x)}=x^2+x+C.
		\end{align*}
		Mà $ f(1)=-\dfrac{1}{2}$ $\Rightarrow C=0$ $\Rightarrow f(x)=\dfrac{-1}{x^2+x}$ $=\dfrac{1}{x+1}-\dfrac{1}{x}$.\\
		Ta có $\heva{
				& f(1)=\dfrac{1}{2}-1\\
				& f(2)=\dfrac{1}{3}-\dfrac{1}{2}\\
				& f(3)=\dfrac{1}{4}-\dfrac{1}{3}\\
				&\ldots\\
				& f\left(2024\right)=\dfrac{1}{2023}-\dfrac{1}{2024}.}$\\
		$
			\Rightarrow T=f(1)+f(2)+\ldots+f\left(2024\right)=-1+\dfrac{1}{2025}=-\dfrac{2024}{2025}\approx -1$.}
\end{ex}


\begin{ex}%[2D4V1-4]
	Cho hàm số $f(x)$ thỏa mãn $ f(0)=1-\ln 2$ và $\mathrm{e}^ xf'(x)=2^x\left[f(x)\right]^2$ với mọi $x\in\mathbb{R}$. Giá trị của $f(1)$ bằng bao nhiêu? (\textit{Kết quả làm tròn đến hàng phần trăm}).
	\shortans{$0{,}42$}
	\loigiai{
		Từ giả thiết ta có $f'(x)=\dfrac{2^x}{\mathrm{e}^ x}{\left[f(x)\right]^2}$ với mọi $ x\in\left(1;2\right]$.\\
		Do đó $ f(x)\ge f(1)=1>0$ với mọi $ x\in\left[1;2\right]$.\\
		Xét với mọi $ x\in [1 ; 2]$ ta có
		\begin{align*}
			\mathrm{e}^ x{f}'(x)=2^x{\left[f(x)\right]^2} & \Rightarrow{f}'(x)=\dfrac{2^x}{\mathrm{e}^ x}{\left[f(x)\right]^2} \\&\Rightarrow\dfrac{f'(x)}{\left[f(x)\right]^2}=\left(\dfrac{2}{\mathrm{e}}\right)^x\\&\Rightarrow-\left(\dfrac{1}{f(x)}\right)'=\left(\dfrac{2}{\mathrm{e}}\right)^x \\&\Rightarrow{\left(\dfrac{1}{f(x)}\right)'}=-\left(\dfrac{2}{\mathrm{e}}\right)^x\\&\Rightarrow\dfrac{1}{f(x)}=-\displaystyle\int\left(\dfrac{2}{\mathrm{e}}\right)^x\mathrm{\,d} x\\&\Rightarrow\dfrac{1}{f(x)}=-\dfrac{\left(\dfrac{2}{\mathrm{e}}\right)^x}{\ln \dfrac{2}{\mathrm{e}}}+C\\&\Rightarrow\dfrac{1}{f(x)}=\dfrac{\left(\dfrac{2}{\mathrm{e}}\right)^x}{1-\ln 2}+C.
		\end{align*}
		Mà $ f(0)=1-\ln 2\Rightarrow C=0$. \\Do đó
		$\dfrac{1}{f(x)}=\dfrac{\left(\dfrac{2}{\mathrm{e}}\right)^x}{1-\ln 2}$
		$\Rightarrow f(x)=\dfrac{1-\ln 2}{\left(\dfrac{2}{\mathrm{e}}\right)^x}=\dfrac{(1-\ln 2)\mathrm{e}^x}{2^x}$.\\
		Vậy $ f(1)=\dfrac{\mathrm{e}-\mathrm{e}\ln 2}{2}\approx 0{,}42$.}
\end{ex}
\begin{ex}%[2D4V1-4]
	Cho hàm số $ y=f(x)$ đồng biến và có đạo hàm liên tục trên $\mathbb{R}$ thỏa mãn $\left(f'(x)\right)^2=f(x)\cdot\mathrm{e}^x$, $\forall x\in\mathbb{R}$ và $f(0)=2$. Tính $ f(2)$. (Kết quả làm tròn đến hàng phần trăm).
	\shortans{$9{,}81$}
	\loigiai{
	Vì hàm số $ y=f(x)$ đồng biến và có đạo hàm liên tục trên $\mathbb{R}$ đồng thời $ f(0)=2$ nên $f'(x)\ge 0$ và $ f(x)>0$ với mọi $ x\in\left[0;+\infty\right)$.\\
	Từ giả thiết $\left(f'(x)\right)^2=f(x)\cdot \mathrm{e}^x$, $\forall x\in\mathbb{R}$ suy ra $f'(x)=\sqrt{f(x)}\cdot\mathrm{e}^{\tfrac{x}{2}}$, $\forall x\in\left[0;+\infty\right).$\\
	Do đó $\dfrac{f'(x)}{2\sqrt{f(x)}}=\dfrac{1}{2}{\mathrm{e}^{\tfrac{x}{2}}}$, $\forall x\in\left[0;+\infty\right).$\\
	Lấy nguyên hàm hai vế, ta được $\sqrt{f(x)}=e^{\tfrac{x}{2}}+C$, $\forall x\in\left[0;+\infty\right)$ với $C$ là hằng số.\\
	Kết hợp với $ f(0)=2$, ta được $C=\sqrt{2}-1$.\\
	Suy ra $ f(2)=\left(\mathrm{e}+\sqrt{2}-1\right)^2\approx 9{,}81$.}
\end{ex}
\begin{ex}%[2D4H1-2]
	Giả sử hàm số $ y=f(x)$ liên tục, nhận giá trị dương trên $\left(0;+\infty\right)$ và thỏa mãn $ f(1)=1$, $ f(x)=f'(x)\cdot \sqrt{3x}$, với mọi $x>0$. Tính $f(5)$ \textit{(kết quả làm tròn đến hàng phần trăm}).
	\shortans{$4{,}17$}
	\loigiai{
	Ta có
	\begin{align*}
		f(x)=f'(x)\cdot\sqrt{3x} & \Rightarrow\dfrac{f'(x)}{f(x)}=\dfrac{1}{\sqrt{3x}}             \\&
		\Rightarrow\ln f(x)=\dfrac{1}{\sqrt{3}}\displaystyle\int \dfrac{1}{\sqrt{x}}\mathrm{\,d} x \\ &\Rightarrow\ln f(x)=\dfrac{2}{\sqrt{3}}\sqrt{x}+C\\&\Rightarrow f(x)=e^{\tfrac{2}{\sqrt{3}}\sqrt{x}+C}.
	\end{align*}
	Mà $ f(1)=1$ nên $1=e^{\tfrac{2}{\sqrt{3}}+C}\Rightarrow C=-\dfrac{2}{\sqrt{3}}$
	$\Rightarrow f(x)=e^{\tfrac{2}{\sqrt{3}}\sqrt{x}-\tfrac{2}{\sqrt{3}}}$.\\
	Suy ra $ f(5)=e^{\tfrac{2}{\sqrt{3}}\sqrt{5}-\tfrac{2}{\sqrt{3}}}=e^{\tfrac{2\sqrt{5}-2}{\sqrt{3}}}\approx 4{,}17$.}
\end{ex}

\begin{ex}%[2D4V1-2]
	Cho hàm số $ f(x)$ có đạo hàm trên $\mathbb{R}$ thỏa mãn $\mathrm{e}^{f(x)}-\dfrac{x}{f'(x)}=0$, $\forall x\in\mathbb{R}$. Biết $f(1)=1$, tính $f\left(\mathrm{e}^2\right)$ (\textit{kết quả làm tròn đến hàng phần trăm}).
	\shortans{$ 3{,}38$}
	\loigiai{
	Ta có
	\begin{align*}
		\mathrm{e}^{f(x)}-\dfrac{x}{f'(x)}=0
		 & \Rightarrow f'(x)\mathrm{e}^{f(x)}=x                                \\
		 & \Leftrightarrow \left(\mathrm{e}^{f(x)} \right)'=x                  \\
		 & \Leftrightarrow \mathrm{e}^{f(x)}=\displaystyle\int x\mathrm{\,d} x \\
		 & \Leftrightarrow \mathrm{e}^{f(x)}=\dfrac{x^2}{2}+C.
	\end{align*}
	Mà 	$f(1)=1$ nên $\mathrm{e}=\dfrac{1}{2}+C\Rightarrow C=\mathrm{e}-\dfrac{1}{2}$.\\
	Do đó $\mathrm{e}^{f(x)}=\dfrac{x^2}{2}+ \mathrm{e}-\dfrac{1}{2} \Rightarrow \mathrm{e}^{f\left(\mathrm{e}^2\right)}= \dfrac{\mathrm{e}^4}{2}+ \mathrm{e}-\dfrac{1}{2}\Rightarrow f\left(\mathrm{e}^2\right)=\ln \left(\dfrac{\mathrm{e}^4}{2}+ \mathrm{e}-\dfrac{1}{2}\right) \approx 3{,}38$.}
\end{ex}

\begin{ex}%[2D4V1-4]
	Cho hàm số $ f(x)$ nhận giá trị dương và thỏa mãn $ f(0)=1$, $\left(f'(x)\right)^3=\mathrm{\mathrm{e}}^ x{\left(f(x)\right)^2}$, $\forall x\in\mathbb{R}$. Tính $ f(3)$ (\textit{kết quả làm tròn đến hàng phần mười}).
	\shortans{$20{,}1$}
	\loigiai{
	Ta có

	\begin{align*}
		\left(f'(x)\right)^3=\mathrm{e}^x{\left(f(x)\right)^2},\,\forall x\in\mathbb{R}
		 & \Leftrightarrow{f}'(x)=\sqrt[3]{\mathrm{e}^x}\cdot \sqrt[3]{\left(f(x)\right)^2}\Leftrightarrow\dfrac{f'(x)}{\sqrt[3]{\left(f(x)\right)^2}}=\sqrt[3]{\mathrm{e}^x}     \\
		 & \Leftrightarrow\dfrac{f'(x)}{\sqrt[3]{\left(f(x)\right)^2}}=\sqrt[3]{\mathrm{e}^x}\Leftrightarrow{f}'(x)\cdot \left(f(x)\right)^{-\tfrac{2}{3}}=\sqrt[3]{\mathrm{e}^x} \\&\Leftrightarrow 3\left[\left(f(x)\right)^{\tfrac{1}{3}}\right]'=\sqrt[3]{\mathrm{e}^x}\Leftrightarrow{\left[\left(f(x)\right)^{\tfrac{1}{3}}\right]'}=\dfrac{1}{3}\sqrt[3]{\mathrm{e}^x}\\&\Leftrightarrow{\left(f(x)\right)^{\tfrac{1}{3}}}=\dfrac{1}{3}\displaystyle\int{\sqrt[3]{\mathrm{e}^x}}\mathrm{\,d} x \Leftrightarrow{\left(f(x)\right)^{\tfrac{1}{3}}}=e^{\tfrac{x}{3}}+C.
	\end{align*}
	Vì	$f(0)=1$ nên $1=1+C\Rightarrow C=0\Rightarrow{\left(f(x)\right)^{\tfrac{1}{3}}}=e^{\tfrac{x}{3}}\Rightarrow f(x)=\mathrm{e}^x$.\\
	Vậy	$f(3)=e^3\approx 20{,}1$.
	}
\end{ex}

\begin{ex}%Câu 13%[2D4V1-2]
	Cho hàm số $ y=f(x)$ có đạo hàm liên tục trên $\mathbb{R}$ và thỏa mãn điều kiện $x^6\left( f'(x)\right) ^3+27\left[f(x)-1\right]^4=0$, $\forall x\in\mathbb{R}$ và $ f(1)=0$. Tính giá trị của $f(2)$.
	\shortans{$-7$}
	\loigiai{
		Ta có
		\begin{align*}
			x^6\left( f'(x)\right)^3+27\left[f(x)-1\right]^4=0
			 & \Leftrightarrow{x^6}{\left( f'(x)\right)^3}=-27\left( f(x)-1\right)^4                                       \\&\Leftrightarrow\dfrac{\left( f'(x)\right) ^3}{\left( f(x)-1\right)^4}=-\dfrac{27}{x^6}\\
			 & \Leftrightarrow\dfrac{\left( f'(x)\right) ^3}{\left( f(x)-1\right) ^3\left( f(x)-1\right)}=-\dfrac{27}{x^6} \\&\Leftrightarrow\dfrac{f'(x)}{\left(f(x)-1\right)\sqrt[3]{f(x)-1}}=-\dfrac{3}{x^2}\\&\Leftrightarrow\dfrac{f'(x)}{-3\left(f(x)-1\right)\sqrt[3]{f(x)-1}}=\dfrac{1}{x^2}\\&\Leftrightarrow{\left[\dfrac{1}{\sqrt[3]{f(x)-1}}\right]'}=\dfrac{1}{x^2}
		\end{align*}
		Do đó $\displaystyle\int{\left( \dfrac{1}{\sqrt[3]{f(x)-1}}\right)'}\mathrm{\,d}x=\displaystyle\int{\dfrac{1}{x^2}\mathrm{\,d}x}=-\dfrac{1}{x}+C.$
		\\
		Suy ra $\dfrac{1}{\sqrt[3]{f(x)-1}}=-\dfrac{1}{x}+C$.\\
		Ta có $ f(1)=0\Rightarrow C=0 \Rightarrow f(x)=1-x^3$.\\
		Khi đó $ f(2)=-7$.}
\end{ex}
\begin{ex}%[2D4V1-2]
	Cho hàm số $f(x)$ thỏa mãn $\left[x{f}'(x)\right]^2+1=x^2\left[1-f(x).f''(x)\right]$ với mọi $x$ dương. Biết $f(1)=f'(1)=1$. Tính giá trị $f^2(2)$ (\textit{kết quả làm tròn đến hàng phần trăm}).
	\shortans{$3{,}39$}
	\loigiai{
		Với mọi $x$ dương, ta có
		\begin{align*}
			\left[x{f}'(x)\right]^2+1=x^2\left[1-f(x)\cdot f''(x)\right]; x>0 & \Leftrightarrow{x^2}\cdot\left[f'(x)\right]^2+1=x^2\left[1-f(x)\cdot f''(x)\right] \\
			                                                                  & \Leftrightarrow{\left[f'(x)\right]^2}+\dfrac{1}{x^2}=1-f(x)\cdot f''(x)            \\
			                                                                  & \Leftrightarrow{\left[f'(x)\right]^2}+f(x)\cdot f''(x)=1-\dfrac{1}{x^2}            \\
			                                                                  & \Leftrightarrow\left[f(x)\cdot f'(x)\right]'=1-\dfrac{1}{x^2}.
		\end{align*}
		Do đó $\displaystyle\int\left[f(x)\cdot f'(x)\right]'\mathrm{\, d}x=\displaystyle\int\left(1-\dfrac{1}{x^2}\right)\mathrm{\, d}x\Rightarrow f(x)\cdot f'(x)=x+\dfrac{1}{x}+C.$\\
		Vì $ f(1)=f'(1)=1\Rightarrow 1=2+C\Leftrightarrow C=-1.$\\
		Nên $\displaystyle\int f(x)\cdot f'(x)\mathrm{\, d}x=\displaystyle\int\left(x+\dfrac{1}{x}-1\right) \mathrm{\, d}x$ $\Leftrightarrow\displaystyle\int f(x)\mathrm{\, d}\left(f(x)\right)=\displaystyle\int{\left(x+\dfrac{1}{x}-1\right)}\mathrm{\, d}x$.\\
		Suy ra				$\dfrac{f^2(x)}{2}=\dfrac{x^2}{2}+\ln x-x+C.$\\
		Vì $ f(1)=1\Rightarrow\dfrac{1}{2}=\dfrac{1}{2}-1+C\Leftrightarrow C=1.$\\
		Vậy $\dfrac{f^2(x)}{2}=\dfrac{x^2}{2}+\ln x-x+1\Rightarrow{f^2}(2)=2\ln 2+2\approx 3{,}39$.
	}
\end{ex}
\Closesolutionfile{ans}
% \indapan{6}{ans/ans-KQ-2-C4B1CD3}
\begin{dang}{~}
	\subsubsection{Điều kiện hàm ẩn có dạng} $$A(x)f(x)+B(x)f'(x)=h(x)\quad (1)$$
	\textbf{Phương pháp giải}
	\begin{itemize}
		\item Ta cần nhân thêm một lượng $u(x)$ vào  $(1)$ để tạo thành \break $u'(x) f(x)+u(x) f'(x)=u(x) \cdot h(x)$ và lúc này.
		      \begin{align*}
			                      & \, u'(x) f(x)+u(x) f'(x)=u(x) \cdot h(x)                                                        
			      \Leftrightarrow \,\left[u(x) f(x)\right]'=u(x) \cdot h(x)                                                       \\
			      \Rightarrow     & \, \int\left[u(x) f(x)\right] \mathrm{\,d} x=\int u(x) \cdot h(x) d x 
			      \Rightarrow   \, u(x) f(x)=\int u(x) \cdot h(x) \mathrm{\,d} x                                   \\
			      \Rightarrow     & \, f(x)=\dfrac{\int u(x) \cdot h(x) \mathrm{\,d} x}{u(x)}
		      \end{align*}
		\item Cách tìm $u(x)$.\\
		      $u(x)$ được chọn sao cho  $\heva{&u'(x)=A(x) \\ &u(x)=B(x).}$\\
		      Suy ra
		      \begin{align*}
			                  & \,\dfrac{u'(x)}{u(x)}=\dfrac{A(x)}{B(x)}                                                                    
			      \Rightarrow \, \int \dfrac{u'(x)}{u(x)} \mathrm{\,d} x=\int \dfrac{A(x)}{B(x)} \mathrm{\,d} x \\
			      \Rightarrow & \, \ln |u(x)|=\int \dfrac{A(x)}{B(x)} \mathrm{\,d} x                                           
			      \Rightarrow \, u(x)=\mathrm{e}^{\int\tfrac{A(x)}{B(x)} \mathrm{\,d} x}
		      \end{align*}
	\end{itemize}
	Tóm lại phương pháp giải $A(x)f(x)+B(x)f'(x)=h(x)$\quad $(1)$ như sau.
	\begin{itemize}
		\item \textbf{Bước 1.} Tìm $u(x)$. $u(x)=\mathrm{e}^{\int\tfrac{A(x)}{B(x)} \mathrm{\,d} x}$.
		\item \textbf{Bước 2.} Nhân $u(x)$ vào $(1)$ suy ra $f(x)=\dfrac{\int\limits u(x) \cdot h(x) \mathrm{\,d}x}{u(x)}$.
	\end{itemize}
	\subsubsection*{Một số dạng đặc biệt của $(1)$.}
	\begin{enumerate}
		\item Điều kiện hàm ẩn có dạng $\hoac{&f'(x)+f(x)=h(x)\\ &f'(x)-f(x)=h(x).}$\\
		      Phương pháp giải.
		      \begin{itemize}
			      \item $f'(x)+f(x)=h(x)$.\\
			            Nhân hai vế với $\mathrm{e}^x$ ta được $$\mathrm{e}^x \cdot f'(x)+\mathrm{e}^x \cdot f(x)=\mathrm{e}^x \cdot h(x) \Leftrightarrow\left[\mathrm{e}^x \cdot f(x)\right]'=\mathrm{e}^x \cdot h(x).$$
			            Suy ra $\mathrm{e}^x \cdot f(x)=\int \mathrm{e}^x \cdot h(x)  \mathrm{\,d} x$.\\
			            Từ đây ta dễ dàng tính được $f(x)$.
			      \item $f'(x)-f(x)=h(x)$.\\
			            Nhân hai vế với $\mathrm{e}^{-x}$ ta được $$\mathrm{e}^{-x} \cdot f'(x)-\mathrm{e}^{-x} \cdot f(x)=\mathrm{e}^{-x} \cdot h(x) \Leftrightarrow\left[\mathrm{e}^{-x} \cdot f(x)\right]'=\mathrm{e}^{-x} \cdot h(x).$$
			            Suy ra $\mathrm{e}^{-x} \cdot f(x)=\int \mathrm{e}^{-x} \cdot h(x)  \mathrm{\,d} x$.\\
			            Từ đây ta dễ dàng tính được $f(x)$.
		      \end{itemize}
		\item Điều kiện hàm ẩn có dạng $f'(x)+p(x)\cdot f(x)=h(x)$.\\
		      \textbf{Phương pháp giải.}\\
		      Nhân hai vế với $\mathrm{e}^{\int\limits p(x) \mathrm{\,d} x}$ ta được
		      \begin{align*}
			                      & \,f'(x) \cdot \mathrm{e}^{\int\limits p(x) \mathrm{\,d} x}+p(x) \cdot \mathrm{e}^{\int\limits p(x) \mathrm{\,d} x} \cdot f(x)=h(x) \cdot \mathrm{e}^{\int\limits p(x) \mathrm{\,d}x} \\
			      \Leftrightarrow & \, \left[f(x) \cdot \mathrm{e}^{\int\limits p(x) \mathrm{\,d} x}\right]'=h(x) \cdot \mathrm{e}^{\int\limits p(x) \mathrm{\,d} x}.
		      \end{align*}
		      Suy ra $f(x) \cdot e^{\int p(x)\mathrm{\,d} x}=\int \mathrm{e}^{\int\limits p(x) \mathrm{e} x} h(x)  \mathrm{\,d} x$.\\
		      Từ đây ta dễ dàng tính được $f(x)$.
	\end{enumerate}
\end{dang}
% \TN
\Opensolutionfile{ans}[ans/ans-LC-2-C4B1CD3.1]
\begin{ex}%[2D4V1-4]
	Cho hàm số $f(x)$ thỏa mãn $f(x)+f'(x)= \mathrm{e}^{-x}$, $\forall x \in \mathbb{R}$ và $f(0)=2$. Tất cả các nguyên hàm của $f(x)\mathrm{e}^x$ là
	\choice
	{$x^2+x+C$}
	{$2 x^2+2 x+C$}
	{$2 x^2+x+C$}
	{\True $\dfrac{1}{2} x^2+2 x+C$}
	\loigiai{
		Ta có \begin{align*}
			f(x)+f'(x)= \mathrm{e}^{-x}
			 & \, \Leftrightarrow f(x)  \mathrm{e}^x+f'(x)  \mathrm{e}^x=1          \\
			 & \, \Leftrightarrow\left(f(x)  \mathrm{e}^x\right)'=1                 \\
			 & \, \Rightarrow f(x)  \mathrm{e}^x=\displaystyle\int x \mathrm{\,d} x \\
			 & \, \Leftrightarrow f(x)  \mathrm{e}^x=x+C.
		\end{align*}
		Vì $f(0)=2$ nên $ C=2$.\\
		Suy ra $f(x)  \mathrm{e}^x=x+2
			\Rightarrow \displaystyle\int f(x)  \mathrm{e}^x d x=\displaystyle\int(x+2) \mathrm{\,d} x=\dfrac{1}{2} x^2+2 x+C$.
	}
\end{ex}

\begin{ex}%[2D4V1-4]
	Cho hàm số $y=f(x)$ liên tục trên $\mathbb{R}$ thỏa mãn $f'(x)+2 x \cdot f(x)= \mathrm{e}^{-x^2}$, $\forall x \in \mathbb{R}$ và $f(0)=0$. Tính $f(1)$.
	\choice
	{$f(1)= \mathrm{e}^2$}
	{$f(1)=-\dfrac{1}{ \mathrm{e}}$}
	{$f(1)=\dfrac{1}{ \mathrm{e}^2}$}
	{\True $f(1)=\dfrac{1}{ \mathrm{e}}$}
	\loigiai{
		Ta có
		\begin{align*}
			                & \, f'(x)+2 x \cdot f(x)= \mathrm{e}^{-x^2}                                                                   \\
			\Leftrightarrow & \,  \mathrm{e}^{x^2} f'(x)+2 x \cdot  \mathrm{e}^{x^2} \cdot f(x)=1                                          \\
			\Leftrightarrow & \,\left( \mathrm{e}^{x^2} \cdot f(x)\right)'=1                                                               \\
			\Rightarrow     & \,\displaystyle\int\left(\mathrm{e}^{x^2} \cdot f(x)\right)' \mathrm{\,d} x=\displaystyle\int \mathrm{\,d} x \\
			\Rightarrow     & \,  \mathrm{e}^{x^2} \cdot f(x)=x+C                                                                          \\
			\Rightarrow     & \, f(x)=\dfrac{x+C}{\mathrm{e}^{x^2}}.
		\end{align*}
		Vì $f(0)=0 \Rightarrow C=0$.\\
		Do đó $f(x)=\dfrac{x}{ \mathrm{e}^{x^2}}$.\\
		Vậy $f(1)=\dfrac{1}{ \mathrm{e}}$.
	}
\end{ex}

\begin{ex}%[2D4V1-2]
	Cho hàm số $y=f(x)$ liên tục trên $\mathbb{R} \setminus \{-1 ; 0\}$ thỏa mãn điều kiện $f(1)=-2 \ln 2$ và $x \cdot(x+1) \cdot f'(x)+f(x)=x^2+x$. Biết $f(2)=a+b \cdot \ln 3$  ($a$, $b \in \mathbb{Q}$). Giá trị $2\left(a^2+b^2\right)$ là
	\choice
	{$\dfrac{27}{4}$}
	{\True  $9$}
	{$\dfrac{3}{4}$}
	{$\dfrac{9}{2}$}
	\loigiai{
		Chia cả hai vế của biểu thức $x \cdot(x+1) \cdot f'(x)+f(x)=x^2+x$ cho $(x+1)^2$ ta có
		$$ \dfrac{x}{x+1} \cdot f'(x)+\dfrac{1}{(x+1)^2} f(x)=\dfrac{x}{x+1} \\
			\Leftrightarrow\left[\dfrac{x}{x+1} \cdot f(x)\right]'=\dfrac{x}{x+1}.$$
		Do đó $$\dfrac{x}{x+1} \cdot f(x)=\displaystyle\int\limits\left[\dfrac{x}{x+1} \cdot f(x)\right]'  \mathrm{\,d} x=\displaystyle\int\limits \dfrac{x}{x+1} \mathrm{\,d} x=\displaystyle\int\limits\left(1-\dfrac{1}{x+1}\right)  \mathrm{\,d} x=x-\ln |x+1|+C.$$
		Do $f(1)=-2 \ln 2$ nên ta có $\dfrac{1}{2} \cdot f(1)=1-\ln 2+C \Leftrightarrow-\ln 2=1-\ln 2+C \Leftrightarrow C=-1$.\\
		Khi đó $f(x)=\dfrac{x+1}{x}(x-\ln |x+1|-1)$.\\
		Vậy ta có $f(2)=\dfrac{3}{2}(2-\ln 3-1)=\dfrac{3}{2}(1-\ln 3)=\dfrac{3}{2}-\dfrac{3}{2} \ln 3 \Rightarrow a=\dfrac{3}{2}$, $b=-\dfrac{3}{2}$.\\
		Suy ra $2\left(a^2+b^2\right)=2\left[\left(\dfrac{3}{2}\right)^2+\left(-\dfrac{3}{2}\right)^2\right]=9$.
	}
\end{ex}

\begin{ex}%[2D4V1-2]
	Cho hàm số $y=f(x)$ liên tục trên $\mathbb{R} \setminus \{-1 ; 0\}$ thỏa mãn $f(1)=2 \ln 2+1$, $x(x+1) f'(x)+(x+2) f(x)=x(x+1)$, $\forall x \in \mathbb{R} \backslash\{-1 ; 0\}$. Biết $f(2)=a+b \ln 3$, với $a$, $b$ là hai số hữu tỉ. Tính $T=a^2-b$.
	\choice
	{\True $T=-\dfrac{3}{16}$}
	{$T=\dfrac{21}{16}$}
	{$T=\dfrac{3}{2}$}
	{$T=0$}
	\loigiai{
		Ta có \begin{align*}
			x(x+1) f'(x)+(x+2) f(x)=x(x+1) & \Leftrightarrow f'(x)+\dfrac{x+2}{x(x+1)} f(x)=1                                     \\
			                               & \Leftrightarrow \dfrac{x^2}{x+1} f'(x)+\dfrac{x(x+2)}{(x+1)^2} f(x)=\dfrac{x^2}{x+1} \\
			                               & \Leftrightarrow\left[\dfrac{x^2}{x+1} f(x)\right]'=\dfrac{x^2}{x+1}                  \\
			                               & \Leftrightarrow \dfrac{x^2}{x+1} f(x)=\displaystyle\int\limits \dfrac{x^2}{x+1} d x  \\
			                               & \Leftrightarrow \dfrac{x^2}{x+1} f(x)=\dfrac{x^2}{2}-x+\ln |x+1|+c                   \\
			                               & \Leftrightarrow f(x)=\dfrac{x+1}{x^2}\left(\dfrac{x^2}{2}-x+\ln |x+1|+c\right).
		\end{align*}
		Từ $f(1)=2 \ln 2+1 \Leftrightarrow c=1$.\\
		Từ đó $f(x)=\dfrac{x+1}{x^2}\left(\dfrac{x^2}{2}-x+\ln |x+1|+1\right)$.\\
		$\Rightarrow f(2)=\dfrac{3}{4}+\dfrac{3}{4} \ln 3$.\\
		Nên $\heva{&a=\dfrac{3}{4} \\ &b=\dfrac{3}{4}.}$\\
		Vậy $T=a^2-b=-\dfrac{3}{16}$.
	}
\end{ex}

\begin{ex}%[2D4V1-2]
	Cho hàm số $y=f(x)$ có đạo hàm liên tục trên $(0 ;+\infty)$ thỏa mãn \break $f'(x)+\dfrac{f(x)}{x}=4 x^2+3 x$ và $f(1)=2$. Phương trình tiếp tuyến của đồ thị hàm số $y=f(x)$ tại điểm có hoành độ $x=2$ là
	\choice
	{$y=-16 x-20$}
	{\True $y=16 x-20$}
	{$y=16 x+20$}
	{$y=-16 x+20$}
	\loigiai{
		$$
			f'(x)+\dfrac{f(x)}{x}=4 x^2+3 x \Leftrightarrow x f'(x)+f(x)=4 x^3+3 x^2 \Leftrightarrow \left(x.f(x)\right)'=4x^3+3x^2.
		$$
		Lấy nguyên hàm hai vế ta được $x f(x)=\displaystyle\int\limits\left(4 x^3+3 x^2\right)  \mathrm{\,d} x=x^4+x^3+C$.\\
		Với $x=1$ ta có $f(1)=2+C$.\\
		Theo đề bài ta có: $f(1)=2 \Leftrightarrow 2+C=2 \Leftrightarrow C=0$.\\
		Vậy $x f(x)=x^4+x^3 \Leftrightarrow f(x)=x^3+x^2$.\\
		Ta có $f'(x)=3 x^2+2 x$, $f'(2)=16$, $ f(2)=12$.\\
		Phương trình tiếp tuyến của đồ thị hàm số $y=f(x)$ tại điểm có hoành độ $x=2$ là
		$$
			y=16(x-2)+12 \Leftrightarrow y=16 x-20.
		$$
	}
\end{ex}

\begin{ex}%[2D4V1-2]
	Cho hàm số $y=f(x)$ liên tục trên $(0 ;+\infty)$ thỏa mãn $2 x f'(x)+f(x)=3 x^2 \sqrt{x}$. Biết $f(1)=\dfrac{1}{2}$. Tính $f(4)$.
	\choice
	{$24$}
	{$14$}
	{$4$}
	{\True  $16$}
	\loigiai{
		Trên khoảng $(0 ;+\infty)$ ta có
		\begin{align*}
			2 x f'(x)+f(x)=3 x^2 \sqrt{x}
			 & \Leftrightarrow \sqrt{x} f'(x)+\dfrac{1}{2 \sqrt{x}}.f(x)=\dfrac{3}{2} x^2                                   \\
			 & \Rightarrow(\sqrt{x} \cdot f(x))'=\dfrac{3}{2} x^2                                                           \\
			 & \Rightarrow \displaystyle\int\limits(\sqrt{x} \cdot f(x))' d x=\displaystyle\int\limits \dfrac{3}{2} x^2 d x \\
			 & \Rightarrow \sqrt{x} \cdot f(x)=\dfrac{1}{2} x^3+C.\quad(\ast)
		\end{align*}
		Mà $f(1)=\dfrac{1}{2}$ nên từ $(\ast)$ có $$\sqrt{1} \cdot f(1)=\dfrac{1}{2} \cdot 1^3+C \Leftrightarrow \dfrac{1}{2}=\dfrac{1}{2}+C \Leftrightarrow C=0 \Rightarrow f(x)=\dfrac{x^2 \sqrt{x}}{2}.$$
		Vậy $f(4)=\dfrac{4^2\cdot  \sqrt{4}}{2}=16$.
	}
\end{ex}

\begin{ex}%[2D4V1-2]
	Cho hàm số $f(x)$ thỏa mãn $f(1)=4$ và $f(x)=x f'(x)-2 x^3-3 x^2$ với mọi $x>0$. Giá trị của $f(2)$ bằng
	\choice
	{$5$}
	{$10$}
	{\True $20$}
	{$15$}
	\loigiai{
		Ta có
		\begin{align*}
			f(x)-x f'(x)=-2 x^3-3 x^2
			 & \Leftrightarrow \dfrac{1 \cdot f(x)-x \cdot f'(x)}{x^2}=\dfrac{-2 x^3-3 x^2}{x^2} \\
			 & \Leftrightarrow\left[\dfrac{f(x)}{x}\right]'=2 x+3.
		\end{align*}
		Suy ra $\dfrac{f(x)}{x}$ là một nguyên hàm của hàm số $ {g}(x)=2 x+3$.\\
		Ta có $\displaystyle\int(2 x+3) \mathrm{\,d} x=x^2+3 x+C$, $C \in \mathbb{R}$.\\
		Do đó $\dfrac{f(x)}{x}=x^2+3 x+\mathrm{C}_1$\quad $(1)$ với $\mathrm{C}_1 \in \mathbb{R}$.\\
		Vì $f(1)=4$ theo giả thiết, nên thay $x=1$ vào hai vế của $(1)$ ta thu được $\mathrm{C}_1=0$, từ đó $f(x)=x^3+3 x^2$.\\ Vậy $f(2)=20$.
	}
\end{ex}

\begin{ex}%[2D4V1-2]
	Cho hàm số $y=f(x)$ liên tục trên $(0 ;+\infty)$ thỏa mãn \break $3 x \cdot f(x)-x^2 \cdot f'(x)=2 f^2(x)$, với $f(x) \neq 0$,  $\forall x \in(0 ;+\infty)$ và $f(1)=\dfrac{1}{3}$. Gọi $M$,  $m$ lần lượt là giá trị lớn nhất, giá trị nhỏ nhất của hàm số $y=f(x)$ trên đoạn $[1 ; 2]$. Tính $M+m$.
	\choice
	{$\dfrac{9}{10}$}
	{$\dfrac{21}{10}$}
	{\True  $\dfrac{5}{3}$}
	{$\dfrac{7}{3}$}
	\loigiai{
		Ta có
		\begin{align*}
			3x\cdot f(x)-x^2 \cdot f'(x)=2 f^2(x)
			 & \Rightarrow 3 x^2 \cdot f(x)-x^3 \cdot f'(x)=2 x \cdot f^2(x)                                             \\
			 & \Rightarrow \dfrac{3 x^2 \cdot f(x)-x^3 \cdot f'(x)}{f^2(x)}=2 x,  f(x) \neq 0, \forall x \in(0 ;+\infty) \\
			 & \Rightarrow\left(\dfrac{x^3}{f(x)}\right)'=2 x                                                            \\ &\Rightarrow \dfrac{x^3}{f(x)}=\displaystyle\int 2 x  \mathrm{\,d} x=x^2+C . \\
			 &
		\end{align*}
		Mà $f(1)=\dfrac{1}{3} \Rightarrow C=2 \Rightarrow f(x)=\dfrac{x^3}{x^2+2}$.\\
		Ta có $f(x)=\dfrac{x^3}{x^2+2} \Rightarrow f'(x)=\dfrac{x^4+6 x^2}{\left(x^2+2\right)^2}>0$, $\forall x \in(0 ;+\infty)$.\\
		Vậy, hàm số $f(x)=\dfrac{x^3}{x^2+2}$ đồng biến trên khoảng $(0 ;+\infty)$.\\
		Mà $[1 ; 2] \subset(0 ;+\infty)$ nên hàm số $f(x)=\dfrac{x^3}{x^2+2}$ đồng biến trên đoạn $[1 ; 2]$.\\
		Suy ra $M=f(2)=\dfrac{4}{3}$, $ m=f(1)=\dfrac{1}{3}$.\\
		Vậy $ M+m=\dfrac{5}{3}$.
	}
\end{ex}

\begin{ex}%[2D4V1-4]
	Cho $F(x)$ là một nguyên hàm của hàm số $f(x)=e^{x^2}\left(x^3-4 x\right)$. Hàm số $F\left(x^2+x\right)$ có bao nhiêu điểm cực trị?
	\choice
	{$6$}
	{\True $5$}
	{$3$}
	{$4$}
	\loigiai{
		Ta có $F'(x)=f(x)$. Khi đó
		\begin{align*}
			F'\left(x^2+x\right) & \,=f\left(x^2+x\right) \cdot\left(x^2+x\right)'                                                   \\
			                     & \,=(2 x+1)\left(x^2+x\right) \mathrm{e}^{\left(x^2+x\right)^2}\left[\left(x^2+x\right)^2-4\right] \\
			                     & \, =(2 x+1) x(x+1) \mathrm{e}^{\left(x^2+x\right)^2}\left(x^2+x-2\right)\left(x^2+x+2\right)      \\
			                     & \, =(2 x+1) x(x+1)(x+2)(x-1)\left(x^2+x+2\right) \mathrm{e}^{\left(x^2+x\right)^2}.
		\end{align*}
		$F'(x)=0\Leftrightarrow \hoac{&x=-2\\ &x=\dfrac{-1}{2}\\ &x=1\\ &x=-1\\& x=0.}$\\
		$F'\left(x^2+x\right)=0$ có $5$ nghiệm đơn nên $F\left(x^2+x\right)$ có $5$ điểm cực trị.
	}
\end{ex}

\begin{ex}%[2D4V1-2]
	\immini{Cho hàm số $y=f(x)$. Đồ thị của hàm số \break $y=f'(x)$ trên $[-5 ; 3]$ như hình vẽ (phần cong của đồ thị là một phần của parabol \break $y=a x^2+b x+c$). Biết $f(0)=0$, giá trị của $2 f(-5)+3 f(2)$ bằng
		\choice
		{$33$}
		{$\dfrac{109}{3}$}
		{\True $\dfrac{35}{3}$}
		{$11$}
	}{
		\begin{tikzpicture}[scale=0.7, font=\footnotesize, line join=round, line cap=round,>=stealth]
			%Gán số liệu.
			\def\xmin{-6};\def\ymin{-2};\def\xmax{4};\def\ymax{5};
			%Gán tọa độ.
			\coordinate (O) at (0,0);
			%Trục Oxy.
			\draw[->] (\xmin,0)--(\xmax,0) node[below]{$x$};
			\draw[->] (0,\ymin)--(0,\ymax) node[left]{$y$};
			\fill (O) node[below left]{$O$} circle(1pt);
			%Giới hạn đồ thị.
			\clip ({\xmin-0.1},{\ymin-0.1}) rectangle ({\xmax+0.1},{\ymax+0.1});
			\foreach \x in {-5,-4,-1,1,2,3}{
					\fill (\x,0) node[below]{$\x$} circle(1pt);
				}
			\foreach \y in {-1,2,3,4}{
					\fill (0,\y) node[left]{$\y$} circle(1pt);
				}
			\draw (-5,-1)--(-4,2)--(-1,0);
			\draw[thick,samples=100] plot[domain=-1:3.5](\x,{-(\x)^2+2*\x+3});
			\draw[dashed] (-5,0)|-(0,-1) (-4,0)|-(0,2) (1,0)|-(0,4) (2,0)|-(0,3);
		\end{tikzpicture}
	}
	\loigiai{
		Parabol $y=a x^2+b x+c$ qua các điểm $(2 ; 3)$, $(1 ; 4)$, $(0 ; 3)$, $(-1 ; 0)$, $(3 ; 0)$ nên xác định được $y=-x^2+2 x+3$, $\forall x \geq-1$ suy ra $f(x)=-\dfrac{x^3}{3}+x^2+3 x+C_1$.\\
		Mà $f(0)=0 \Rightarrow C_1=0$, $f(x)=-\dfrac{x^3}{3}+x^2+3 x$.\\
		Có $f(-1)=-\dfrac{5}{3}$, $ f(2)=\dfrac{22}{3}$.\quad $(1)$\\
		Đồ thị $f'(x)$ trên đoạn $[-4 ;-1]$ qua các điểm $(-4 ; 2)$, $(-1 ; 0)$.\\
		Nên $f'(x)=-\dfrac{2}{3}(x+1) \Rightarrow f(x)=-\dfrac{2}{3}\left(\dfrac{x^2}{2}+x\right)+C_2$.\\
		Mà $f(-1)=-\dfrac{5}{3} \Leftrightarrow C_2=-\dfrac{5}{3}+\dfrac{2}{3}\left(-\dfrac{1}{2}\right)=-2 \Rightarrow f(x)=-\dfrac{2}{3}\left(\dfrac{x^2}{2}+x\right)-2$, hay $f(-4)=-\dfrac{14}{3}$.\\
		Đồ thị $f'(x)$ trên đoạn $[-5 ;-4]$ qua các điểm $(-4 ; 2)$, $(-5 ;-1)$.\\
		Nên $f'(x)=3 x+14 \Rightarrow f(x)=\dfrac{3 x^2}{2}+14 x+C_3$.\\
		Mà $f(-4)=-\dfrac{14}{3} \Leftrightarrow \dfrac{3 \cdot(-4)^2}{2}+14 \cdot(-4)+C_3=-\dfrac{14}{3}$ suy ra $C_3=\dfrac{82}{3}$.\\
		Ta có $f(x)=\dfrac{3 x^2}{2}+14 x+\dfrac{82}{3} \Rightarrow f(-5)=-\dfrac{31}{6}$.\quad $(2)$\\
		Từ $(1)$ và $(2)$ ta được $2 f(-5)+3 f(2)=-\dfrac{31}{3}+22=\dfrac{35}{3}$.
	}
\end{ex}
\Closesolutionfile{ans}
% \indapan{10}{ans/ans-LC-2-C4B1CD3.1}
% %%Bài 2. Tích phân
% \setcounter{section}{1}
\section{Tích Phân}
\subsection{Lý thuyết cần nhớ}
\subsubsection{Diện tích hình thang cong}
\begin{center}
	\begin{tikzpicture}[>=stealth]
		%		\tkzInit[xmin=-0.5,ymin=-2.5,xmax=6.5,ymax=2.5] \tkzClip
		\draw[->] (-0.5,0)--(5.3,0) node[below] {$x$} ;
		\draw[->] (0,-.5)--(0,2.3) node[right] {$y$} ;
		\draw (0,0) node[below left] {$O$};
		%		\draw (1,0) ellipse (0.16 and 1);
		%		\draw (4,0) ellipse (0.25 and 1.73);
		%Nhánh trên
		\draw[domain=1:4] 
		plot(\x,{0.31*(\x)^3-2.28*(\x)^2+5.14*(\x)-2.17}) ;
		%Nhánh dưới
		%	\draw[domain=1:4]
		%	plot(\x,{-0.31*(\x)^3+2.28*(\x)^2-5.14*(\x)+2.17}) ;
		%Tô màu
		\draw[pattern = north east lines,opacity=.3, line width = 1.2pt,draw=none] (1,1) plot[domain=1:4] (\x,{0.31*(\x)^3-2.28*(\x)^2+5.14*(\x)-2.17})--(4,0)--(1,0)--cycle;
		
		%Các yếu tố khác
		\draw (1,0) node[below] {$a$};
		\draw (4,0) node[below] {$b$};
		\draw[dashed] (1,0)--(1,1);
		\draw[dashed] (4,0)--(4,1.72);
		\draw (2.5,1.7) node {$y=f(x)$} ;
		\draw (2.5,0.7) node {$S$} ;
		%	\draw[->] (5,0.25) arc (90:270:0.3);
	\end{tikzpicture}
\end{center}
Nếu hàm số $f(x)$ liên tục và không âm trên đoạn $\left[a;b\right]$ thì diện tích $S$ của hình thang cong giới hạn bởi đồ thị $y=f(x)$, trục hoành và hai đường thẳng $x=a$, $x=b$ được tính bởi:
$S=F(b)-F(a)$
trong đó $F(x)$ là một nguyên hàm của $f(x)$ trên đoạn $\left[a;b\right]$.
\subsubsection{Khái niệm tích phân}
Cho hàm số $f(x)$ liên tục trên đoạn $\left[a;b\right]$. Nếu $F(x)$ là nguyên hàm của hàm số $f(x)$ trên đoạn $\left[a;b\right]$ thì hiệu số $F(b)-F(a)$ được gọi là tích phân từ $a$ đến $b$ của hàm số $f(x)$, kí hiệu $\displaystyle\int\limits_a^bf(x)\mathrm{d}x$.\\
\begin{note}Chú ý:
	\begin{itemize}
		\item Hiệu số $F(b)-F(a)$ còn được kí hiệu là $ F(x)\big|_a^b$.\\
		Vậy $\displaystyle\int\limits_a^bf(x)\mathrm{d}x= F(x)\big|_a^b=F(b)-F(a)$.
		\item Ta gọi $\displaystyle\int\limits_a^b{}$ là dấu tích phân, $a$ là cận dưới, $b$ là cận trên, $f(x)\mathrm{d}x$ là biểu thức dưới dấu tích phân và $f(x)$ là hàm số dưới dấu tích phân.
		\item Quy ước: $\displaystyle\int\limits_a^af(x)\mathrm{d}x=0$; $\displaystyle\int\limits_a^bf(x)\mathrm{d}x=-\displaystyle\int\limits_b^af(x)\mathrm{d}x$.
		\item Tích phân của hàm số $f$ từ $a$ đến $b$ chỉ phụ thuộc vào $f$ và các cận $a$, $b$ mà không phụ thuộc vào biến $x$ hay $t$, nghĩa là $\displaystyle\int\limits_a^bf(x)\mathrm{d}x=\displaystyle\int\limits_a^bf(t)\mathrm{d}t$.
		\item Ý nghĩa hình học của tích phân.\\
		\immini{
			Nếu hàm số $f(x)$ liên tục và không âm trên đoạn $\left[a;b\right]$ thì $\displaystyle\int\limits_a^bf(x)\mathrm{d}x$ là diện tích $S$ của hình thang cong giới hạn bởi đồ thị $y=f(x)$, trục hoành và hai đường thẳng $x=a$, $x=b$.
			$$S=\displaystyle\int\limits_a^bf(x)\mathrm{\,d}x.$$}{\begin{tikzpicture}[>=stealth,scale=0.8]
				%		\tkzInit[xmin=-0.5,ymin=-2.5,xmax=6.5,ymax=2.5] \tkzClip
				\draw[->] (-0.5,0)--(5.3,0) node[below] {$x$} ;
				\draw[->] (0,-.5)--(0,2.3) node[right] {$y$} ;
				\draw (0,0) node[below left] {$O$};
				%		\draw (1,0) ellipse (0.16 and 1);
				%		\draw (4,0) ellipse (0.25 and 1.73);
				%Nhánh trên
				\draw[domain=1:4] 
				plot(\x,{0.31*(\x)^3-2.28*(\x)^2+5.14*(\x)-2.17}) ;
				%Nhánh dưới
				%	\draw[domain=1:4]
				%	plot(\x,{-0.31*(\x)^3+2.28*(\x)^2-5.14*(\x)+2.17}) ;
				%Tô màu
				\draw[pattern = north east lines,opacity=.3, line width = 1.2pt,draw=none] (1,1) plot[domain=1:4] (\x,{0.31*(\x)^3-2.28*(\x)^2+5.14*(\x)-2.17})--(4,0)--(1,0)--cycle;
				
				%Các yếu tố khác
				\draw (1,0) node[below] {$a$};
				\draw (4,0) node[below] {$b$};
				\draw[dashed] (1,0)--(1,1);
				\draw[dashed] (4,0)--(4,1.72);
				\draw (2.5,1.7) node {$y=f(x)$} ;
				\draw (2.5,0.7) node {$S$} ;
				%	\draw[->] (5,0.25) arc (90:270:0.3);
		\end{tikzpicture}}
	\end{itemize}
\end{note}
\begin{nx}
	\begin{itemize}
		\item Nếu hàm số $f(x)$ có đạo hàm $f'(x)$ và $f'(x)$ liên tục trên đoạn $\left[a;b\right]$ thì\\
		$f(b)-f(a)=\displaystyle\int\limits_a^bf'(x)\mathrm{d}x$.
		\item Cho hàm số $f(x)$ liên tục trên đoạn $\left[a;b\right]$. Khi đó $\dfrac{1}{b-a}\displaystyle\int\limits_a^bf(x)\mathrm{d}x$ được gọi là giá trị trung bình của hàm số $f(x)$ trên đoạn $\left[a;b\right]$.
		\item Đạo hàm của quãng đường di chuyển của vật theo thời gian bằng tốc độ của chuyển động tại mọi thời điểm $v(t)=s'(t)$. Do đó, nếu biết tốc độ $v(t)$ tại mọi thời điểm $t\in\left[a;b\right]$ thì tính được quãng đường di chuyển trong khoảng thời gian từ $a$ đến $b$ theo công thức: $s=s(b)-s(a)=\displaystyle\int\limits_a^bv(t)\mathrm{d}t$.
	\end{itemize}
\end{nx}
\subsubsection{Tính chất của tích phân}
Cho hai hàm số $f(x)$, $g(x)$ liên tục trên đoạn $\left[a;b\right]$. Khi đó:
\begin{enumerate}
	\item $\displaystyle\int\limits_a^bkf(x)\mathrm{d}x=k\displaystyle\int\limits_a^bf(x)\mathrm{d}x$, với $k$ là hằng số.
	\item $\displaystyle\int\limits_a^b\left[f(x)\pm g(x)\right]\mathrm{\,d}x=\displaystyle\int\limits_a^b{f(x)\mathrm{\,d}x}\pm\displaystyle\int\limits_a^bg(x)\mathrm{\,d}x$.
	\item $\displaystyle\int\limits_a^bf(x)\mathrm{\,d}x=\displaystyle\int\limits_a^cf(x)\mathrm{\,d}x+\displaystyle\int\limits_c^bf(x)\mathrm{\,d}x$ với $c\in\left(a;b\right)$.
\end{enumerate}
\subsection{Phân loại và phương pháp giải bài tập}
\begin{dang}{Tính chất của tích phân}

\end{dang}
\setcounter{ex}{0}
\TN
\Opensolutionfile{ans}[ans/ans-2C4B2CD3-LC]
\begin{ex}%[Câu 1]%[2D4N2-1]
	Nếu $\displaystyle\int\limits_0^3f(x)\mathrm{\,d}x=6$ thì $\displaystyle\int\limits_0^3\left[\dfrac{1}{3}f(x)+2\right]\mathrm{\,d}x$ bằng
	\choice
	{\True $8$}
	{$5$}
	{$9$}
	{$6$}
	\loigiai{
		Ta có $\displaystyle\int\limits_0^3\left[\dfrac{1}{3}f(x)+2\right]\mathrm{\,d}x=\dfrac{1}{3}\displaystyle\int\limits_0^3f(x)\mathrm{\,d}x+\displaystyle\int\limits_0^32\mathrm{\,d}x=\dfrac{1}{3}\cdot 6+6=8$.}
\end{ex}
\begin{ex}%[Câu 2]%[2D4N2-1]
	Nếu $\displaystyle\int_1^4 f(x) \mathrm{\,d}x=3$ và $\displaystyle\int_1^4 g(x) \mathrm{\,d}x=-2$ thì $\displaystyle\int_1^4\left(f(x)-g(x)\right)\mathrm{\,d}x$ bằng
	\choice
	{$-1$}
	{$-5$}
	{\True $5$}
	{$1$}
	\loigiai{
		Ta có $\displaystyle\int _1^4\left[f(x)-g(x)\right]\mathrm{\,d}x=\displaystyle\int _1^4f(x)\mathrm{\,d}x-\displaystyle\int _1^4g(x)\mathrm{\,d}x=3-(-2)=5$.}
\end{ex}
\begin{ex}%[Câu 3]%[2D4N2-1]
	Nếu $\displaystyle\int\limits_1^4f(x)\mathrm{\,d}x=5$ và $\displaystyle\int\limits_1^4g(x)\mathrm{\,d}x=-4$ thì $\displaystyle\int\limits_1^4\left[f(x)-g(x)\right]\mathrm{\,d}x$ bằng
	\choice
	{$-1$}
	{$-9$}
	{$1$}
	{\True $9$}
	\loigiai{
		Ta có $\displaystyle\int\limits_1^4\left[f(x)-g(x)\right]\mathrm{\,d}x=\displaystyle\int\limits_1^4f(x)\mathrm{\,d}x-\displaystyle\int\limits_1^4g(x)\mathrm{\,d}x=5-(-4)=9$.}
\end{ex}
\begin{ex}%[Câu 4]%[2D4N2-1]
	Biết $\displaystyle\int\limits_1^{2024}f(x)\mathrm{\,d}x=-3$ và $\displaystyle\int\limits_{2024}^1g(x)\mathrm{\,d}x=2$. Khi đó $\displaystyle\int\limits_1^{2024}\left[f(x)-g(x)\right]\mathrm{\,d}x$ bằng
	\choice
	{$6$}
	{$-5$}
	{$5$}
	{\True $-1$}
	\loigiai{
		Ta có $\displaystyle\int\limits_{2024}^1g(x)\mathrm{\,d}x=2\Leftrightarrow \displaystyle\int\limits_1^{2024}g(x)\mathrm{\,d}x=-2$.\\
		Do đó $\displaystyle\int\limits_1^{2024}\left[f(x)-g(x)\right]\mathrm{\,d}x=\displaystyle\int\limits_1^{2024}f(x)\mathrm{\,d}x-\displaystyle\int\limits_1^{2024}g(x)\mathrm{\,d}x=-3-(-2)=-1$.}
\end{ex}
\begin{ex}%[Câu 5]%[2D4N2-1]
	Nếu $\displaystyle\int\limits_0^3f(x)\mathrm{\,d}x=3$ thì $\displaystyle\int\limits_0^34f(x)\mathrm{\,d}x$ bằng
	\choice
	{$3$}
	{\True $12$}
	{$36$}
	{$4$}
	\loigiai{
		Ta có $\displaystyle\int\limits_0^34f(x)\mathrm{\,d}x=4\displaystyle\int\limits_0^3f(x)\mathrm{\,d}x=4\cdot 3=12$.}
\end{ex}
\begin{ex}%[Câu 6]%[2D4N2-1]
	Cho $\displaystyle\int\limits_0^2f(x)\mathrm{\,d}x=\dfrac{1}{2024}$. Tính $I=\displaystyle\int\limits_0^2 2024f(x)\mathrm{\,d}x$.
	\choice
	{$I=5$}
	{$I=\dfrac{1}{2024}$}
	{\True $I=1$}
	{$I=2024$}
	\loigiai{
		Ta có $I=\displaystyle\int\limits_0^2 2024f(x)\mathrm{\,d}x=2024\displaystyle\int\limits_0^2f(x)\mathrm{\,d}x=2024\cdot \dfrac{1}{2024}=1$.}
\end{ex}
\begin{ex}%[Câu 7]%[2D4N2-1]
	Nếu $\displaystyle\int\limits_0^5f(x)\mathrm{\,d}x=5$ thì $\displaystyle\int\limits_5^05f(x)\mathrm{\,d}x$ bằng
	\choice
	{$1$}
	{$-1$}
	{$25$}
	{\True $-25$}
	\loigiai{
		Ta có $\displaystyle\int\limits_5^05f(x)\mathrm{\,d}x=5\displaystyle\int\limits_5^0f(x)\mathrm{\,d}x=-5\cdot\displaystyle\int\limits_0^5f(x)\mathrm{\,d}x=(-5)\cdot 5=-25$.
		
	}
\end{ex}
\begin{ex}%[Câu 8]%[2D4N2-1]
	Nếu $\displaystyle\int\limits_0^2f(x)\mathrm{\,d}x=5$ thì $\displaystyle\int\limits_0^2\left[2f(x)-1\right]\mathrm{\,d}x$ bằng
	\choice
	{\True $8$}
	{$9$}
	{$10$}
	{$12$}
	\loigiai{
		Ta có $\displaystyle\int _0^2\left[2f(x)-1\right]\mathrm{\,d}x=2\displaystyle\int _0^2f(x)\mathrm{\,d}x-\displaystyle\int _0^21\mathrm{\,d}x=2\cdot 5-2=8$.}
\end{ex}
\begin{ex}%[Câu 9]%[2D4N2-1]
	Nếu $\displaystyle\int_0^2 f(x) d x=3$ thì $\displaystyle\int_0^2\left[2f(x)-1\right]\mathrm{\,d}x$ bằng
	\choice
	{$6$}
	{\True $4$}
	{$8$}
	{$5$}
	\loigiai{
		Ta có $\displaystyle\int_0^2\left[2f(x)-1\right]\mathrm{\,d}x=2\displaystyle\int_0^2f(x)\mathrm{\,d}x-\displaystyle\int_0^2\mathrm{\,d}x=2\cdot 3-2=4$.}
\end{ex}
\begin{ex}%[Câu 10]%[2D4N2-1]
	Cho $\displaystyle\int\limits_0^1f(x)\mathrm{\,d}x=2$ và $\displaystyle\int\limits_0^1g(x)\mathrm{\,d}x=5$, khi $\displaystyle\int\limits_0^1\left[f(x)-2g(x)\right]\mathrm{\,d}x$ bằng
	\choice
	{\True $-8$}
	{$1$}
	{$-3$}
	{$12$}
	\loigiai{
		Ta có $\displaystyle\int\limits_0^1\left[f(x)-2g(x)\right]\mathrm{\,d}x=\displaystyle\int\limits_0^1f(x)\mathrm{\,d}x-2\displaystyle\int\limits_0^1g(x)\mathrm{\,d}x=2-2\cdot 5=-8$.}
\end{ex}
\begin{ex}%[Câu 11]%[2D4H2-1]
	Cho $\displaystyle\int\limits_0^{\frac{\pi}{2}}f(x)\mathrm{\,d}x=5$. Tính $I=\displaystyle\int\limits_0^{\frac{\pi}{2}}\left[f(x)+2\sin x\right]\mathrm{\,d}x$.
	\choice
	{\True $I=7$}
	{$I=5+\dfrac{\pi}{2}$}
	{$I=3$}
	{$I=5+\pi $}
	\loigiai{
		Ta có
		\begin{eqnarray*}
			&I&=\displaystyle\int\limits_0^{\frac{\pi}{2}}\left[f(x)+2\sin x\right]\mathrm{\,d}x\\
			&&=\displaystyle\int\limits_0^{\frac{\pi}{2}}f(x)\mathrm{\,d}x\text{+2}\displaystyle\int\limits_0^{\tfrac{\pi}{2}}\sin x\mathrm{\,d}x\\
			&&=\displaystyle\int\limits_0^{\frac{\pi}{2}}f(x)\mathrm{\,d}x-2\cos x\bigg|_0^{\frac{\pi}{2}}\\
			&&=5-2(0-1)=7.
		\end{eqnarray*}
	}
\end{ex}
\begin{ex}%[Câu 12]%[2D4H2-1]
	Cho $\displaystyle\int\limits_1^2\left[4f(x)-2x\right]\mathrm{\,d}x=1$. Khi đó $\displaystyle\int\limits_1^2f(x)\mathrm{\,d}x$ bằng
	\choice
	{\True $1$}
	{$-3$}
	{$3$}
	{$-1$}
	\loigiai{
		Ta có \begin{eqnarray*}
			&&\displaystyle\int\limits_1^2\left[4f(x)-2x\right]\mathrm{\,d}x=1\\
			&\Leftrightarrow&4\displaystyle\int\limits_1^2f(x)\mathrm{\,d}x-2\displaystyle\int\limits_1^2x\mathrm{\,d}x=1\\
			&\Leftrightarrow&4\displaystyle\int\limits_1^2f(x)\mathrm{\,d}x-2\cdot  \dfrac{x^2}{2}\bigg|_1^2=1\\
			&\Leftrightarrow&4\displaystyle\int\limits_1^2f(x)\mathrm{\,d}x=4\\
			&\Leftrightarrow&\displaystyle\int\limits_1^2f(x)\mathrm{\,d}x=1.
		\end{eqnarray*}
	}
\end{ex}
% \begin{ex}%[Câu 13]%[2D4H2-1]
% 	Cho $\displaystyle\int\limits_0^1f(x)\mathrm{\,d}x=1$, tích phân $\displaystyle\int\limits_0^1\left(2f(x)-3x^2\right)\mathrm{\,d}x$ bằng
% 	\choice
% 	{\True $1$}
% 	{$0$}
% 	{$3$}
% 	{$-1$}
% 	\loigiai{Ta có 
% 		$\displaystyle\int\limits_0^1(2f(x)-3x^2)\mathrm{\,d}x=2\displaystyle\int\limits_0^1f(x)\mathrm{\,d}x-3\displaystyle\int\limits_0^1x^2\mathrm{\,d}x=2-1=1$.}
% \end{ex}
% \begin{ex}%[Câu 14]%[2D4H2-1]
% 	Cho $\displaystyle\int\limits_{-1}^2f(x)\mathrm{\,d}x=2$ và $\displaystyle\int\limits_{-1}^2g(x)\mathrm{\,d}x=-1$. Tính $I=\displaystyle\int\limits_{-1}^2\left[x+2f(x)-3g(x)\right]\mathrm{\,d}x$.
% 	\choice
% 	{\True $I=\dfrac{17}{2}$}
% 	{$I=\dfrac{5}{2}$}
% 	{$I=\dfrac{7}{2}$}
% 	{$I=\dfrac{11}{2}$}
% 	\loigiai{
% 		Ta có 
% 		\begin{eqnarray*}
% 			&I&=\displaystyle\int\limits_{-1}^2\left[x+2f(x)-3g(x)\right]\mathrm{\,d}x\\
% 			&&= \dfrac{x^2}{2}\bigg|_{-1}^2+2\displaystyle\int\limits_{-1}^2f(x)\mathrm{\,d}x-3\displaystyle\int\limits_{-1}^2g(x)\mathrm{\,d}x\\
% 			&&=\dfrac{3}{2}+2\cdot 2-3(-1)=\dfrac{17}{2}.
% 		\end{eqnarray*}
% 	}
% \end{ex}
% \begin{ex}%[Câu 15]%[2D4H2-1]
% 	Cho $\displaystyle\int\limits_0^2f(x)\mathrm{\,d}x=3$,$\displaystyle\int\limits_0^2g(x)\mathrm{\,d}x=-1$ thì $\displaystyle\int\limits_0^2\left[f(x)-5g(x)+x\right]\mathrm{\,d}x$ bằng
% 	\choice
% 	{$12$}
% 	{$0$}
% 	{$8$}
% 	{\True $10$}
% 	\loigiai{Ta có 
% 		$\displaystyle\int\limits_0^2\left[f(x)-5g(x)+x\right]\mathrm{\,d}x=\displaystyle\int\limits_0^2f(x)\mathrm{\,d}x-5\displaystyle\int\limits_0^2\mathrm{g}(x)\mathrm{\,d}x+\displaystyle\int\limits_0^2x\mathrm{\,d}x=3+5+2=10$.}
% \end{ex}
% \begin{ex}%[Câu 16]%[2D4H2-1]
% 	Cho $\displaystyle\int\limits_0^5f(x)\mathrm{\,d}x=-2$. Tích phân $\displaystyle\int\limits_0^5\left[4f(x)-3x^2\right]\mathrm{\,d}x$ bằng
% 	\choice
% 	{$-140$}
% 	{$-130$}
% 	{$-120$}
% 	{\True $-133$}
% 	\loigiai{Ta có
% 		$\displaystyle\int\limits_0^5\left[4f(x)-3x^2\right]\mathrm{\,d}x=4\displaystyle\int\limits_0^5f(x)\mathrm{\,d}x-\displaystyle\int\limits_0^53x^2\mathrm{\,d}x=-8-x^3\bigg|_0^5=-8-125=-133$.}
% \end{ex}
% \begin{ex}%[Câu 17]%[2D4H2-1]
% 	Cho $\displaystyle\int\limits_1^2\left[4f(x)-2x\right]\mathrm{\,d}x=1$. Khi đó $\displaystyle\int\limits_1^2f(x)\mathrm{\,d}x$ bằng:
% 	\choice
% 	{\True $1$}
% 	{$-3$}
% 	{$3$}
% 	{$-1$}
% 	\loigiai{Ta có
% 		\begin{eqnarray*}
% 			&&\displaystyle\int\limits_1^2\left[4f(x)-2x\right]\mathrm{\,d}x=1\\
% 			&\Leftrightarrow&4\displaystyle\int\limits_1^2f(x)\mathrm{\,d}x-2\displaystyle\int\limits_1^2x\mathrm{\,d}x=1\\
% 			&\Leftrightarrow&4\displaystyle\int\limits_1^2f(x)\mathrm{\,d}x-2\cdot  \dfrac{x^2}{2}\bigg|_1^2=1\\
% 			&\Leftrightarrow&4\displaystyle\int\limits_1^2f(x)\mathrm{\,d}x=4\\
% 			&\Leftrightarrow&\displaystyle\int\limits_1^2f(x)\mathrm{\,d}x=1.
% 		\end{eqnarray*}
% 	}
% \end{ex}
% \begin{ex}%[Câu 18]%[2D4H2-1]
% 	Cho $\displaystyle\int\limits_{-2}^2f(x)\mathrm{\,d}x=1$, $\displaystyle\int\limits_{-2}^4f(t)\mathrm{\,d}t=-4$. Tính $\displaystyle\int\limits_2^4f(y)\mathrm{\,d}y$.
% 	\choice
% 	{$I=5$}
% 	{$I=-3$}
% 	{$I=3$}
% 	{\True $I=-5$}
% 	\loigiai{	
% 		Ta có $\displaystyle\int\limits_{-2}^4f(t)\mathrm{\,d}t=\displaystyle\int\limits_{-2}^4f(x)\mathrm{\,d}x$, $\displaystyle\int\limits_2^4f(y)\mathrm{\,d}y=\displaystyle\int\limits_2^4f(x)\mathrm{\,d}x$.\\
% 		Khi đó $\displaystyle\int\limits_{-2}^2f(x)\mathrm{\,d}x+\displaystyle\int\limits_2^4f(x)\mathrm{\,d}x=\displaystyle\int\limits_{-2}^4f(x)\mathrm{\,d}x$. Do đó
% 		$$ \displaystyle\int\limits_2^4f(x)\mathrm{\,d}x=\displaystyle\int\limits_{-2}^4f(x)\mathrm{\,d}x-\displaystyle\int\limits_{-2}^2f(x)\mathrm{\,d}x=-4-1=-5.$$
% 		Vậy $\displaystyle\int\limits_2^4f(y)\mathrm{\,d}y=-5$.}
% \end{ex}
% \begin{ex}%[Câu 19]%[2D4H2-1]
% 	Cho hàm số $f(x)$ liên tục trên $\mathbb{R}$ và có $\displaystyle\int\limits_0^2f(x)\mathrm{\,d}x=9;\displaystyle\int\limits_2^4f(x)\mathrm{\,d}x=4$. Tính $I=\displaystyle\int\limits_0^4f(x)\mathrm{\,d}x$.
% 	\choice
% 	{$I=5$}
% 	{$I=36$}
% 	{$I=\dfrac{9}{4}$}
% 	{\True $I=13$}
% 	\loigiai{
% 		Ta có $I=\displaystyle\int\limits_0^4f(x)\mathrm{\,d}x=\displaystyle\int\limits_0^2f(x)\mathrm{\,d}x+\displaystyle\int\limits_2^4f(x)\mathrm{\,d}x=9+4=13$.}
% \end{ex}
% \begin{ex}%[Câu 20]%[2D4H2-1]
% 	Cho hàm số $f(x)$ liên tục trên $\mathbb{R}$ và $\displaystyle\int\limits_0^4f(x)\mathrm{\,d}x=10$, $\displaystyle\int\limits_3^4f(x)\mathrm{\,d}x=4$. Tích phân $\displaystyle\int\limits_0^3f(x)\mathrm{\,d}x$ bằng
% 	\choice
% 	{$4$}
% 	{$7$}
% 	{$3$}
% 	{\True $6$}
% 	\loigiai{
% 		Theo tính chất của tích phân, ta có $\displaystyle\int\limits_0^3f(x)\mathrm{\,d}x+\displaystyle\int\limits_3^4f(x)\mathrm{\,d}x=\displaystyle\int\limits_0^4f(x)\mathrm{\,d}x$.\\
% 		Suy ra  $\displaystyle\int\limits_0^3f(x)\mathrm{\,d}x=\displaystyle\int\limits_0^4f(x)\mathrm{\,d}x-\displaystyle\int\limits_3^4f(x)\mathrm{\,d}x=10-4=6$.\\
% 		Vậy $\displaystyle\int\limits_0^3f(x)\mathrm{\,d}x=6$.}
% \end{ex}
% \begin{ex}%[Câu 21]%[2D4H2-1]
% 	Cho hàm số $f(x)$ liên tục trên đoạn $[0;10]$ và $\displaystyle\int\limits_0^{10}f(x)\mathrm{\,d}x=7$; $\displaystyle\int\limits_2^6f(x)\mathrm{\,d}x=3$.\\
% 	Tính $P=\displaystyle\int\limits_0^2f(x)\mathrm{\,d}x+\displaystyle\int\limits_6^{10}f(x)\mathrm{\,d}x$.
% 	\choice
% 	{\True $P=4$}
% 	{$P=10$}
% 	{$P=7$}
% 	{$P=-4$}
% 	\loigiai{
% 		Ta có $\displaystyle\int\limits_0^{10}f(x)\mathrm{\,d}x=\displaystyle\int\limits_0^2f(x)\mathrm{\,d}x+\displaystyle\int\limits_2^6f(x)\mathrm{\,d}x+\displaystyle\int\limits_6^{10}f(x)\mathrm{\,d}x$ hay $7=P+3\Leftrightarrow P=4$.}
% \end{ex}
% \begin{ex}%[Câu 22]%[2D4H2-1]
% 	Cho hàm số $f(x)$ liên tục trên đoạn $[0; 6]$ thỏa mãn $\displaystyle\int\limits_0^6f(x)\mathrm{\,d}x=10$ và $\displaystyle\int\limits_2^4f(x)\mathrm{\,d}x=6$. 	Tính giá trị của biểu thức $P=\displaystyle\int\limits_0^2f(x)\mathrm{\,d}x+\displaystyle\int\limits_4^6f(x)\mathrm{\,d}x$.
% 	\choice
% 	{\True$P=4$}
% 	{$P=16$}
% 	{$P=8$}
% 	{$P=10$}
% 	\loigiai{
% 		Ta có $\displaystyle\int\limits_0^{6}f(x)\mathrm{\,d}x=\displaystyle\int\limits_0^2f(x)\mathrm{\,d}x+\displaystyle\int\limits_2^4f(x)\mathrm{\,d}x+\displaystyle\int\limits_4^{6}f(x)\mathrm{\,d}x$ hay $7=P+3\Leftrightarrow P=4$.	
% 	}
% \end{ex}
\Closesolutionfile{ans}
% \indapan{10}{ans/ans-2C4B2CD3-LC}
\TNTF
\Opensolutionfile{ans}[ans/ans-2C4B2CD3-DS]
\begin{ex}%[Câu 23]%[2D4H2-1]
	Cho hai hàm $f$, $g$ liên tục trên $K$ và $a$, $b$ là các số bất kỳ thuộc $K$.
	\choiceTF
	{\True $\displaystyle\int\limits_a^b\left[f(x)+2g(x)\right]\mathrm{\,d}x=\displaystyle\int\limits_a^bf(x)\mathrm{\,d}x\text{+2}\displaystyle\int\limits_a^bg(x)\mathrm{\,d}x$}
	{$\displaystyle\int\limits_a^b\dfrac{f(x)}{g(x)}\mathrm{\,d}x=\dfrac{\displaystyle\int\limits_a^bf(x)\mathrm{\,d}x}{\displaystyle\int\limits_a^bg(x)\mathrm{\,d}x}$}
	{$\displaystyle\int\limits_a^b\left[f(x)\cdot g(x)\right]\mathrm{\,d}x=\displaystyle\int\limits_a^bf(x)\mathrm{\,d}x \displaystyle\int\limits_a^bg(x)\mathrm{\,d}x$}
	{$\displaystyle\int\limits_a^bf^2(x)\mathrm{\,d}x=\left[\displaystyle\int\limits_a^bf(x)\mathrm{\,d}x\right]^2$}
	\loigiai{
		\begin{itemchoice}
			\itemch Đúng. Theo tính chất tích phân ta có
			$\displaystyle\int\limits_a^b\left[f(x)+g(x)\right]\mathrm{\,d}x=\displaystyle\int\limits_a^bf(x)\mathrm{\,d}x+\displaystyle\int\limits_a^bg(x)\mathrm{\,d}x;\displaystyle\int\limits_a^bkf(x)\mathrm{\,d}x=k\displaystyle\int\limits_a^bf(x)\mathrm{\,d}x$, với $k\in \mathbb{R}$.
			\itemch Sai. Cho $a=1,b=2$ và $f(x)=x+1, g(x)=x$. Khi đó
			$$VT=\displaystyle\int\limits_{1}^2\dfrac{x+1}{x}\mathrm{\,d}x==\displaystyle\int\limits_{1}^2\left(1+\dfrac{1}{x}\right)\mathrm{\,d}x=\left(x+\ln x\right)\bigg|_1^2=1+\ln 2.$$
			và $$VP=\dfrac{\displaystyle\int\limits_1^2(x+1)\mathrm{\,d}x}{\displaystyle\int\limits_1^2x\mathrm{\,d}x}=\dfrac{\left(\dfrac{x^2}{2}+x\right)\bigg|_1^2}{\dfrac{x^2}{2}\bigg|_1^2}=\dfrac{1}{3}.$$
			Do đó $VT\neq VP$.
			\itemch Sai. Cho $a=1, b=2$ và $f(x)=x, g(x)=\dfrac{1}{x}$. Khi đó
			$$VT=\displaystyle\int\limits_1^2\left[x\cdot \dfrac{1}{x}\right]\mathrm{\,d}x=x\bigg|_1^2=1.$$
			và $$VP=\displaystyle\int\limits_1^2x\mathrm{\,d}x\cdot \displaystyle\int\limits_1^2\dfrac{1}{x}\mathrm{\,d}x=\left(\dfrac{x^2}{2}\right)\bigg|_1^2\cdot \ln x\bigg|_1^2=\dfrac{3}{2}\ln 2.$$
			Do đó $VT\neq VP$.
			\itemch Sai. Cho $a=1,b=2$ và $f(x)=x$. Khi đó
			$$VT=\displaystyle\int\limits_1^2x^2\mathrm{\,d}x=\left(\dfrac{x^3}{3}\right)\bigg|_1^2=\dfrac{7}{3}.$$
			và $$VP=\left(\displaystyle\int\limits_1^2x\mathrm{\,d}x\right)^2=\left(\dfrac{x^2}{2}\bigg|_1^2\right)^2=\dfrac{9}{4}.$$
			Do đó $VT\neq VP$.
		\end{itemchoice}
	}
\end{ex}
\begin{ex}%[Câu 24]%[2D4H2-1]
	Cho hàm số $f(x),g(x)$ liên tục trên $\mathbb{R}$.
	\choiceTF
	{\True Nếu $\displaystyle\int\limits_0^2f(x)\mathrm{\,d}x=4$ thì $\displaystyle\int\limits_0^2\left[\dfrac{1}{2}f(x)+2\right]\mathrm{\,d}x=6$}
	{\True Nếu $\displaystyle\int\limits_2^5f(x)\mathrm{\,d}x=3$ và $\displaystyle\int\limits_2^5g(x)\mathrm{\,d}x=-2$ thì $\displaystyle\int\limits_2^5\left[f(x)+g(x)\right]\mathrm{\,d}x=1$}
	{Nếu $\displaystyle\int\limits_1^4f(x)\mathrm{\,d}x=6$ và $\displaystyle\int\limits_1^4g(x)\mathrm{\,d}x=-5$ thì $\displaystyle\int\limits_1^4\left[f(x)-g(x)\right]\mathrm{\,d}x=1$}
	{\True Nếu $\displaystyle\int\limits_2^3f(x)\mathrm{\,d}x=4$ và$\displaystyle\int\limits_2^3g(x)\mathrm{\,d}x=1$ thì $\displaystyle\int\limits_2^3\left[f(x)-g(x)\right]\mathrm{\,d}x=3$}
	\loigiai{
		\begin{itemchoice}
			\itemch Đúng. Ta có $\displaystyle\int\limits_0^2\left[\dfrac{1}{2}f(x)+2\right]\mathrm{\,d}x=\dfrac{1}{2}\displaystyle\int\limits_0^2f(x)\mathrm{\,d}x+\displaystyle\int\limits_0^22\mathrm{\,d}x=\dfrac{1}{2}\cdot 4+4=6$.
			\itemch Đúng. Ta có $\displaystyle\int\limits_2^5\left[f(x)+g(x)\right]\mathrm{\,d}x=\displaystyle\int\limits_2^5f(x)\mathrm{\,d}x+\displaystyle\int\limits_2^5g(x)\mathrm{\,d}x=3+(-2)=1$.
			\itemch Sai. Ta có $\displaystyle\int\limits_1^4\left[f(x)-g(x)\right]\mathrm{\,d}x=\displaystyle\int\limits_1^4f(x)\mathrm{\,d}x-\displaystyle\int\limits_1^4g(x)\mathrm{\,d}x=6-(-5)=11$.
			\itemch Đúng. Ta có $\displaystyle\int\limits_2^3\left[f(x)-g(x)\right]\mathrm{\,d}x=\displaystyle\int\limits_2^3f(x)\mathrm{\,d}x-\displaystyle\int\limits_2^3g(x)\mathrm{\,d}x=4-1=3$.
		\end{itemchoice}
	}
\end{ex}
\begin{ex}%[Câu 25]%[2D4H2-1]
	Cho hàm số $f(x),g(x)$ liên tục trên $\mathbb{R}$.
	\choiceTF
	{Biết $\displaystyle\int\limits_2^3f(x)\mathrm{\,d}x=3$ và $\displaystyle\int\limits_3^2g(x)\mathrm{\,d}x=1$. Khi đó $\displaystyle\int\limits_2^3\left[f(x)+g(x)\right]\mathrm{\,d}x=4$}
	{\True Biết $\displaystyle\int\limits_1^3f(x)\mathrm{\,d}x=2022$ và $\displaystyle\int\limits_3^1g(x)\mathrm{\,d}x=1$. Khi đó $\displaystyle\int\limits_1^3\left[f(x)+g(x)\right]\mathrm{\,d}x=2021$}
	{\True Biết $\displaystyle\int\limits_1^2f(x)\mathrm{\,d}x=3$ và $\displaystyle\int\limits_1^2g(x)\mathrm{\,d}x=2$. Khi đó $\displaystyle\int\limits_1^2\left[f(x)-g(x)\right]\mathrm{\,d}x=1$}
	{Biết $\displaystyle\int\limits_2^5f(x)\mathrm{\,d}x=2$. Khi đó $\displaystyle\int\limits_2^53f(x)\mathrm{\,d}x=2$}
	\loigiai{
		\begin{itemchoice}
			\itemch Sai. Ta có
			$\displaystyle\int\limits_2^3\left[f(x)+g(x)\right]\mathrm{\,d}x=\displaystyle\int\limits_2^3f(x)\mathrm{\,d}x+\displaystyle\int\limits_2^3g(x)\mathrm{\,d}x=\displaystyle\int\limits_2^3f(x)\mathrm{\,d}x-\displaystyle\int\limits_3^2g(x)\mathrm{\,d}x=2$.
			\itemch Đúng. Ta có $\displaystyle\int\limits_3^1g(x)\mathrm{\,d}x=1\Leftrightarrow \displaystyle\int\limits_1^3g(x)\mathrm{\,d}x=-1$. Do đó 
			$$\displaystyle\int\limits_1^3\left[f(x)+g(x)\right]\mathrm{\,d}x=\displaystyle\int\limits_1^3f(x)\mathrm{\,d}x+\displaystyle\int\limits_1^3g(x)\mathrm{\,d}x=2022+(-1)=2021.$$
			\itemch Đúng. Ta có $\displaystyle\int\limits_1^2\left[f(x)-g(x)\right]\mathrm{\,d}x=\displaystyle\int\limits_1^2f(x)\mathrm{\,d}x-\displaystyle\int\limits_1^2g(x)\mathrm{\,d}x=3-2=1$.
			\itemch Sai. Ta có $\displaystyle\int\limits_2^53f(x)\mathrm{\,d}x=3\displaystyle\int\limits_2^5f(x)\mathrm{\,d}x=3\cdot 2=6$.
		\end{itemchoice}
	}
\end{ex}
\begin{ex}%[Câu 26]%[2D4H2-1]
	Cho hàm số $f(x)$ liên tục trên $\mathbb{R}$.
	\choiceTF
	{\True Nếu $\displaystyle\int\limits_0^3f(x)\mathrm{\,d}x=3$ thì $\displaystyle\int\limits_0^32f(x)\mathrm{\,d}x=6$}
	{\True Nếu $\displaystyle\int\limits_1^4f(x)\mathrm{\,d}x=2024$ thì $\displaystyle\int\limits_4^1f(x)\mathrm{\,d}x=-2024$}
	{Nếu $\displaystyle\int\limits_6^0f(x)\mathrm{\,d}x=12$ thì $\displaystyle\int\limits_0^62022f(x)\mathrm{\,d}x=24264$}
	{\True Nếu $\displaystyle\int\limits_0^1f(x)\mathrm{\,d}x=4$ thì $\displaystyle\int\limits_0^12f(x)\mathrm{\,d}x=8$}
	\loigiai{
		\begin{itemchoice}
			\itemch Đúng. Ta có $\displaystyle\int\limits_0^32f(x)\mathrm{\,d}x=2\displaystyle\int\limits_0^3f(x)\mathrm{\,d}x=2\cdot 3=6$.
			\itemch Đúng. Ta có $\displaystyle\int\limits_4^1f(x)\mathrm{\,d}x=-\displaystyle\int\limits_1^4f(x)\mathrm{\,d}x=-2024$.
			\itemch Sai. Ta có $\displaystyle\int\limits_0^62022f(x)\mathrm{\,d}x=2022\displaystyle\int\limits_0^6f(x)\mathrm{\,d}x=2022\cdot (-12)=-24264$.
			\itemch Đúng. Ta có $\displaystyle\int\limits_0^12f(x)\mathrm{\,d}x=2\displaystyle\int\limits_0^1f(x)\mathrm{\,d}x=2\cdot 4=8$.
		\end{itemchoice}
	}
\end{ex}
% \begin{ex}%[Câu 27]%[2D4H2-1]
% 	Cho hàm số $f(x),g(x)$ liên tục trên $\mathbb{R}$.
% 	\choiceTF
% 	{Nếu $\displaystyle\int_0^2 f(x)d x=6$ thì $\displaystyle\int_0^2\left[2f(x)-1\right]\mathrm{\,d}x=-10$}
% 	{\True Nếu $\displaystyle\int\limits_0^2f(x)\mathrm{\,d}x=4$ thì $\displaystyle\int\limits_0^2\left[2f(x)-1)\right]\mathrm{\,d}x=6$}
% 	{\True Nếu $\displaystyle\int_0^2f(x)\mathrm{\,d}x=3$ và $\displaystyle\int_0^2g(x)\mathrm{\,d}x=7$ thì $\displaystyle\int_0^2\left[f(x)+3g(x)\right]\mathrm{\,d}x=24$}
% 	{\True Nếu $\displaystyle\int\limits_0^1\left[f(x)+2x\right]\mathrm{\,d}x=3$ thì $\displaystyle\int\limits_0^1f(x)\mathrm{\,d}x=2$}
% 	\loigiai{
% 		\begin{itemchoice}
% 			\itemch Sai. Ta có $\displaystyle\int _0^2\left[2f(x)-1\right]\mathrm{\,d}x=2\displaystyle\int _0^2f(x)\mathrm{\,d}x-\displaystyle\int _0^2\mathrm{\,d}x=2\cdot 6-2=10$.
% 			\itemch Đúng. Ta có $\displaystyle\int\limits_0^2\left[2f(x)-1)\right]\mathrm{\,d}x=\displaystyle\int\limits_0^22f(x)\mathrm{\,d}x-\displaystyle\int\limits_0^2\mathrm{\,d}x=2\cdot 4-2=6$.
% 			\itemch Đúng. Ta có $\displaystyle\int_0^2\left[f(x)+3g(x)\right]\mathrm{\,d}x=\displaystyle\int_0^2f(x)\mathrm{\,d}x+3\displaystyle\int_0^2g(x)\mathrm{\,d}x=3+3\cdot 7=24$.
% 			\itemch Đúng. Ta có:\\ $\displaystyle\int\limits_0^1\left[f(x)+2x\right]\mathrm{\,d}x=3\Leftrightarrow \displaystyle\int\limits_0^1f(x)\mathrm{\,d}x+2\displaystyle\int\limits_0^1x\mathrm{\,d}x=3\Leftrightarrow \displaystyle\int\limits_0^1f(x)\mathrm{\,d}x+2\cdot \dfrac{x^2}{2}\bigg|_0^1=3$.\\
% 			Suy ra $\displaystyle\int\limits_0^1f(x)\mathrm{\,d}x=3-x^2\bigg|_0^1=3-(1-0)=2$.
% 		\end{itemchoice}
% 	}
% \end{ex}
% \begin{ex}%[Câu 28]%[2D4H2-1]
% 	Cho hàm số $f(x),g(x)$ liên tục trên $\mathbb{R}$.
% 	\choiceTF
% 	{\True Nếu $\displaystyle\int\limits_{-1}^5f(x)\mathrm{\,d}x=-3$ thì $\displaystyle\int\limits_5^{-1}f(x)\mathrm{\,d}x=3$}
% 	{\True Nếu $\displaystyle\int\limits_2^3f(x)\mathrm{\,d}x=-6$ thì $\displaystyle\int\limits_3^22f(x)\mathrm{\,d}x=12$}
% 	{Nếu $\displaystyle\int\limits_1^2f(x)\mathrm{\,d}x=2$ và $\displaystyle\int\limits_1^2g(x)\mathrm{\,d}x=6$ thì $\displaystyle\int\limits_2^1\left[f(x)-g(x)\right]\mathrm{\,d}x=-4$}
% 	{Nếu $\displaystyle\int\limits_0^1f(x)\mathrm{\,d}x=3$ và $\displaystyle\int\limits_0^1g(x)\mathrm{\,d}x=-4$ thì $\displaystyle\int\limits_1^0\left[f(x)+g(x)\right]\mathrm{\,d}x=-1$}
% 	\loigiai{
% 		\begin{itemchoice}
% 			\itemch Đúng. Ta có $\displaystyle\int _5^{-1}f(x)\mathrm{\,d}x=-\displaystyle\int _{-1}^5f(x)\mathrm{\,d}x=-(-3)=3$.
% 			\itemch Đúng. Ta có $\displaystyle\int\limits_3^22f(x)\mathrm{\,d}x=-\displaystyle\int\limits_2^32f(x)\mathrm{\,d}x=-2\displaystyle\int\limits_2^3f(x)\mathrm{\,d}x=-2\cdot (-6)=12$.
% 			\itemch Sai. Ta có \\
% 			$\displaystyle\int\limits_2^1\left[f(x)-g(x)\right]\mathrm{\,d}x=-\displaystyle\int\limits_1^2\left[f(x)-g(x)\right]\mathrm{\,d}x=-\displaystyle\int\limits_1^2f(x)\mathrm{\,d}x+\displaystyle\int\limits_1^2g(x)\mathrm{\,d}x=-2+6=4$.
% 			\itemch Sai. Ta có\\
% 			$\displaystyle\int\limits_1^0\left[f(x)+g(x)\right]\mathrm{\,d}x=-\displaystyle\int\limits_0^1\left[f(x)+g(x)\right]\mathrm{\,d}x=-\displaystyle\int\limits_0^1f(x)\mathrm{\,d}x-\displaystyle\int\limits_0^1g(x)\mathrm{\,d}x=-3+4=1$.
% 		\end{itemchoice}
% 	}
% \end{ex}
% \begin{ex}%[Câu 29]%[2D4V2-1]
% 	Cho hàm số $f(x),g(x)$ liên tục trên $\mathbb{R}$.
% 	\choiceTF
% 	{\True Nếu $\displaystyle\int\limits_0^1f(x)\mathrm{\,d}x=-1$ và $\displaystyle\int\limits_0^3f(x)\mathrm{\,d}x=5$ thì $\displaystyle\int\limits_1^3f(x)=6$}
% 	{Nếu $\displaystyle\int\limits_1^2f(x)\mathrm{\,d}x=-3$ và $\displaystyle\int\limits_2^3f(x)\mathrm{\,d}x=4$ thì $\displaystyle\int\limits_1^3f(x)\mathrm{\,d}x=-1$}
% 	{Nếu $\displaystyle\int\limits_{-1}^0f(x)\mathrm{\,d}x=3, \displaystyle\int\limits_{0}^3f(x)\mathrm{\,d}x=1$ thì $\displaystyle\int\limits_{-1}^3f(x)\mathrm{\,d}x=-4$}
% 	{Nếu $\displaystyle\int\limits_{-2}^{5}f(x)\mathrm{\,d}x=8$ và $\displaystyle\int\limits_5^{-2}g(x)\mathrm{\,d}x=3$ thì $\displaystyle\int\limits_{-2}^5\left(f(x)-4g(x)-1\right)\mathrm{\,d}x=-13$}
% 	\loigiai{
% 		\begin{itemchoice}
% 			\itemch Đúng. Ta có 
% 			$\displaystyle\int\limits_0^3f(x)\mathrm{\,d}x =\displaystyle\int\limits_0^1f(x)\mathrm{\,d}x +\displaystyle\int\limits_1^3f(x)\mathrm{\,d}x$.\\
% 			Do đó $\displaystyle\int\limits_1^3f(x)\mathrm{\,d}x =\displaystyle\int\limits_0^3f(x)\mathrm{\,d}x-\displaystyle\int\limits_0^1f(x)\mathrm{\,d}x = 5+ 1= 6$.
% 			\itemch Sai. Ta có $\displaystyle\int\limits_1^3f(x)\mathrm{\,d}x=\displaystyle\int\limits_1^2f(x)\mathrm{\,d}x+\displaystyle\int\limits_2^3f(x)\mathrm{\,d}x=-3+4=1$.
% 			\itemch Sai. Ta có $\displaystyle\int\limits_{-1}^0f(x)\mathrm{\,d}x=3;\displaystyle\int\limits_{0}^3f(x)\mathrm{\,d}x=1;\displaystyle\int\limits_{-1}^3f(x)\mathrm{\,d}x=\displaystyle\int\limits_{-1}^0f(x)\mathrm{\,d}x+\displaystyle\int\limits_{0}^3f(x)\mathrm{\,d}x=3+1=4$.
% 			\itemch Sai. Ta có 
% 			\begin{eqnarray*}
% 				&&\displaystyle\int\limits_{-2}^5\left[f(x)-4g(x)-1\right]\mathrm{\,d}x\\
% 				&&=\displaystyle\int\limits_{-2}^5f(x)\mathrm{\,d}x-\displaystyle\int\limits_{-2}^54g(x)\mathrm{\,d}x-\displaystyle\int\limits_{-2}^5\mathrm{\,d}x\\
% 				&&=\displaystyle\int\limits_{-2}^5f(x)\mathrm{\,d}x-4\displaystyle\int\limits_{-2}^5g(x)\mathrm{\,d}x-\displaystyle\int\limits_{-2}^5\mathrm{\,d}x\\
% 				&&=\displaystyle\int\limits_{-2}^5f(x)\mathrm{\,d}x+4\displaystyle\int\limits_5^{-2}g(x)\mathrm{\,d}x-\displaystyle\int\limits_{-2}^5\mathrm{\,d}x\\
% 				&&=8+4\cdot 3-x\bigg|_{-2}^5=8+4\cdot 3-7=13.
% 			\end{eqnarray*}
% 		\end{itemchoice}
% 	}
% \end{ex}
% \begin{ex}%[Câu 30]%[2D4V2-1]
% 	Cho hàm số $f(x),g(x)$ liên tục trên $\mathbb{R}$.
% 	\choiceTF
% 	{\True Biết $\displaystyle\int\limits_1^2f(x)\mathrm{\,d}x=2$. Giá trị của  $\displaystyle\int\limits_2^13f(x)\mathrm{\,d}x=-6$}
% 	{Biết $\displaystyle\int\limits_1^2f(x)\mathrm{\,d}x=-1$ và $\displaystyle\int\limits_1^2g(x)\mathrm{\,d}x=3$, khi đó $\displaystyle\int\limits_2^1\left[f(x)-g(x)\right]\mathrm{\,d}x=5$}
% 	{\True Nếu $\displaystyle\int\limits_1^2f(x)\mathrm{\,d}x=-2$ và $\displaystyle\int\limits_2^3f(x)\mathrm{\,d}x=1$ thì $\displaystyle\int\limits_1^3f(x)\mathrm{\,d}x=-1$}
% 	{\True Nếu $\displaystyle\int\limits_0^2(f(x)+3x^2)\mathrm{\,d}x=10$ thì $\displaystyle\int\limits_0^2f(x)\mathrm{\,d}x=2$}
% 	\loigiai{
% 		\begin{itemchoice}
% 			\itemch Đúng. Biết $\displaystyle\int\limits_1^2f(x)\mathrm{\,d}x=2$. Giá trị của $\displaystyle\int\limits_2^13f(x)\mathrm{\,d}x=-6$.\\
% 			Ta có $\displaystyle\int\limits_2^13f(x)\mathrm{\,d}x=-\displaystyle\int\limits_1^23f(x)\mathrm{\,d}x=-3\displaystyle\int\limits_1^2f(x)\mathrm{\,d}x=-3\cdot 2=-6$.
% 			\itemch Sai. Biết $\displaystyle\int\limits_1^2f(x)\mathrm{\,d}x=-1$ và $\displaystyle\int\limits_1^2g(x)\mathrm{\,d}x=3$.\\
% 			Ta có $\displaystyle\int\limits_1^2f(x)\mathrm{\,d}x=-1\Leftrightarrow \displaystyle\int\limits_2^1f(x)\mathrm{\,d}x=1$ và $\displaystyle\int\limits_1^2g(x)\mathrm{\,d}x=3\Leftrightarrow \displaystyle\int\limits_2^1g(x)\mathrm{\,d}x=-3$.\\
% 			Do vậy,  $\displaystyle\int\limits_2^1\left[f(x)-g(x)\right]\mathrm{\,d}x=\displaystyle\int\limits_2^1f(x)\mathrm{\,d}x-\displaystyle\int\limits_2^1g(x)\mathrm{\,d}x=1-(-3)=4$.
% 			\itemch Đúng. Nếu $\displaystyle\int\limits_1^2f(x)\mathrm{\,d}x=-2$ và $\displaystyle\int\limits_2^3f(x)\mathrm{\,d}x=1$ thì $\displaystyle\int\limits_1^3f(x)\mathrm{\,d}x=-1$.\\
% 			Ta có $\displaystyle\int\limits_1^3f(x)\mathrm{\,d}x=\displaystyle\int\limits_1^2f(x)\mathrm{\,d}x+\displaystyle\int\limits_2^3f(x)\mathrm{\,d}x=-2+1=-1$.
% 			\itemch Đúng. Ta có
% 			\begin{eqnarray*}
% 				&&\displaystyle\int\limits_0^2(f(x)+3x^2)\mathrm{\,d}x=10\\
% 				&\Leftrightarrow&\displaystyle\int\limits_0^2f(x)\mathrm{\,d}x+\displaystyle\int\limits_0^23x^2\mathrm{\,d}x=10\\
% 				&\Leftrightarrow&\displaystyle\int\limits_0^2f(x)\mathrm{\,d}x=10-\displaystyle\int\limits_0^23x^2\mathrm{\,d}x\\
% 				&\Leftrightarrow&\displaystyle\int\limits_0^2f(x)\mathrm{\,d}x=10-x^3\bigg|_0^2\\
% 				&\Leftrightarrow&\displaystyle\int\limits_0^2f(x)\mathrm{\,d}x=10-8=2.	\end{eqnarray*}
% 		\end{itemchoice}
% 	}
% \end{ex}
\Closesolutionfile{ans}
% \indapan{3}{ans/ans-2C4B2CD3-DS}
\Opensolutionfile{ans}[ans/ans-2-C4B2CD3-KQ]
\TNSA

\begin{ex}%[2D4H2-1]
	Cho $\displaystyle\int\limits_0^3f(x)\mathrm{\,d}x=4$. Tính $I=\displaystyle\int\limits_0^33f(x)\mathrm{\,d}x$.\\
	\shortans{$12$}
	\loigiai{
		Ta có $\displaystyle\displaystyle\int\limits_0^3 3 f(x){d}x=3\displaystyle\displaystyle\int\limits_0^3 f(x){d}x=12$.}
\end{ex}
\begin{ex}%[2D4H2-1]
	Cho $\displaystyle\int\limits_1^3f(x)\mathrm{\,d}x=2$. Tính $I=\displaystyle\int\limits_1^3\left[f(x)+2x\right]\mathrm{\,d}x$.\\
	\shortans{$10$}
	\loigiai{
		Ta có$\colon $ $\displaystyle\int\limits_1^3\left[f(x)+2x\right]\mathrm{\,d}x=\displaystyle\int\limits_1^3f(x)\mathrm{\,d}x+\displaystyle\int\limits_1^32x\mathrm{\,d}x=2+\left.x^2\right|_1^3=2+3^2-1^2=10$.}
\end{ex}

\begin{ex}%[2D4H2-1]
	Cho $\displaystyle\int\limits_{-1}^2f(x)\mathrm{\,d}x=2$ và $\displaystyle\int\limits_{-1}^2g(x)\mathrm{\,d}x=-1$. Tính $ I=\displaystyle\int\limits_{-1}^2\left[x+2f(x)+3g(x)\right]\mathrm{\,d}x$.\\
	\shortans{$2{,}5$}
	\loigiai{
		Ta có $\displaystyle\int\limits_{-1}^2\left[x+2f(x)+3g(x)\right]\mathrm{\,d}x=\displaystyle\int\limits_{-1}^2x\mathrm{\,d}x+2\displaystyle\int\limits_{-1}^2f(x)\mathrm{\,d}x+3\displaystyle\int\limits_{-1}^2g(x)\mathrm{\,d}x=\dfrac{3}{2}+4-3=\dfrac{5}{2}=2{.}5$.}
\end{ex}

\begin{ex}%[2D4H2-1]
	Cho $\displaystyle\int\limits_0^1f(x)\mathrm{\,d}x=1$. Tính tích phân $ I=\displaystyle\int\limits_0^1\left[2f(x)-3x^2\right]\mathrm{\,d}x.$\\
	\shortans{$1$}
	\loigiai{
		$\displaystyle\int\limits_0^1\left[2f(x)-3x^2\right]\mathrm{\,d}x=2\displaystyle\int\limits_0^1f(x)\mathrm{\,d}x-3\displaystyle\int\limits_0^1x^2\mathrm{\,d}x=2-1=1$.}
\end{ex}

\begin{ex}%[2D4H2-1]
	Biết $\displaystyle\int\limits_1^3f(x)\mathrm{\,d}x=3$. Tính giá trị của $ I=\displaystyle\int\limits_3^12f(x)\mathrm{\,d}x$.\\
	\shortans{$-6$}
	\loigiai{
		Ta có $\displaystyle\int\limits_3^12f(x)\mathrm{\,d}x=-\displaystyle\int\limits_1^32f(x)\mathrm{\,d}x=-2\displaystyle\int\limits_1^3f(x)\mathrm{\,d}x=-2\cdot3=-6$.}
\end{ex}

% \begin{ex}%[2D4H2-1]
% 	Biết $\displaystyle\int\limits_0^1f(x)\mathrm{\,d}x=-2$ và $\displaystyle\int\limits_1^0g(x)\mathrm{\,d}x=-3$.. Tính $ I=\displaystyle\int\limits_0^1\left[f(x)-g(x)\right]\mathrm{\,d}x$.\\
% 	\shortans{$-5$}
% 	\loigiai{
% 		$\displaystyle\int\limits_0^1\left[f(x)-g(x)\right]\mathrm{\,d}x=\displaystyle\int\limits_0^1f(x)\mathrm{\,d}x-\displaystyle\int\limits_0^1g(x)\mathrm{\,d}x=-2-3=-5$.}
% \end{ex}

% \begin{ex}%[2D4H2-1]
% 	Biết $\displaystyle\int\limits_1^2f(x)\,\mathrm{\,d}x=3$ và $\displaystyle\int\limits_1^2g(x)\mathrm{\,d}x=2$ và $\displaystyle\int\limits_1^2h(x)\mathrm{\,d}x=2022$. Tính $\linebreak I=\displaystyle\int\limits_1^2\left[f(x)-g(x)+h(x)\right]\mathrm{\,d}x$.\\
% 	\shortans{$2023$}
% 	\loigiai{
% 		Ta có $\displaystyle\int\limits_1^2\left[f(x)-g(x)+h(x)\right]\,\mathrm{\,d}x=\displaystyle\int\limits_1^2f(x)\,\mathrm{\,d}x-\displaystyle\int\limits_1^2g(x)\mathrm{\,d}x+\displaystyle\int\limits_1^2h(x)\mathrm{\,d}x$\\
% 		$=3-2+2022=2023$.}
% \end{ex}

% \begin{ex}%[2D4H2-1]
% 	Cho $\displaystyle\int\limits_{-1}^2f(x)\mathrm{\,d}x=2$ và $\displaystyle\int\limits_2^5f(x)\mathrm{\,d}x=-5$. Tính $ I=\displaystyle\int\limits_{-1}^5f(x)\mathrm{\,d}x$.\\
% 	\shortans{$-3$}
% 	\loigiai{
% 		Ta có $\displaystyle\int\limits_{-1}^5f(x)\mathrm{\,d}x=\displaystyle\int\limits_{-1}^2f(x)\mathrm{\,d}x+\displaystyle\int\limits_2^5f(x)\mathrm{\,d}x=2-5=-3$.}
% \end{ex}

% \begin{ex}%[2D4H2-1]
% 	Cho $ f$, $ g$ là hai hàm liên tục trên đoạn $\left[1;\,3\right]$ thoả$\colon $ $\displaystyle\int\limits_1^3\left[f(x)+3g(x)\right]\mathrm{\,d}x=10$, $\displaystyle\int\limits_1^3\left[2f(x)-g(x)\right]\mathrm{\,d}x=6$. Tính $I=\displaystyle\int\limits_1^3\left[f(x)+g(x)\right]\mathrm{\,d}x$.\\
% 	\shortans{$6$}
% 	\loigiai{
% 		Đặt $ a=\displaystyle\int\limits_1^3f(x)\mathrm{\,d}x$ và $ b=\displaystyle\int\limits_1^3g(x)\mathrm{\,d}x$.\\
% 		Khi đó, $\displaystyle\int\limits_1^3\left[f(x)+3g(x)\right]\mathrm{\,d}x=a+3b$, $\displaystyle\int\limits_1^3\left[2f(x)-g(x)\right]\mathrm{\,d}x=2a-b$.\\
% 		Theo giả thiết, ta có $\heva{
% 			& a+3b=10\\ 
% 			& 2a-b=6\\ 
% 		}\Leftrightarrow\heva{
% 			& a=4\\ 
% 			& b=2.\\ 
% 		}$\\
% 		Vậy $ I=a+b=6$.}
% \end{ex}

% \begin{ex}%[2D4H2-1]
% 	Cho hàm số $ f(x)$ liên tục trên $\mathbb{R}$ thoả mãn $\displaystyle\int\limits_1^8f(x)\,\mathrm{\,d}x=9$, $\displaystyle\int\limits_4^{12}{f(x)}\,\mathrm{\,d}x=3$, $\displaystyle\int\limits_4^8f(x)\,\mathrm{\,d}x=5$. Tính $ I=\displaystyle\int\limits_1^{12}{f(x)}\,\mathrm{\,d}x$.\\
% 	\shortans{$7$}
% 	\loigiai{
% 		Ta có $ I=\displaystyle\int\limits_1^{12}{f(x)}\,\mathrm{\,d}x=\displaystyle\int\limits_1^8f(x)\,\mathrm{\,d}x+\displaystyle\int\limits_8^{12}{f(x)}\,\mathrm{\,d}x$ $=\displaystyle\int\limits_1^8f(x)\,\mathrm{\,d}x+\displaystyle\int\limits_4^{12}{f(x)}\,\mathrm{\,d}x-\displaystyle\int\limits_4^8f(x)\,\mathrm{\,d}x$\\$=9+3-5=7$.}
% \end{ex}

% \begin{ex}%[2D4H2-1]
% 	Cho hàm số $ f(x)$ liên tục trên $\left[0;10\right]$ thỏa mãn $\displaystyle\int\limits_0^{10}{f(x)\mathrm{\,d}x}=7$, $\displaystyle\int\limits_2^6f(x)\mathrm{\,d}x=3$. Tính $ P=\displaystyle\int\limits_0^2f(x)\mathrm{\,d}x+\displaystyle\int\limits_6^{10}{f(x)\mathrm{\,d}x}$.\\
% 	\shortans{$4$}
% 	\loigiai{
% 		Ta có $\displaystyle\int\limits_0^{10}{f(x)\mathrm{\,d}x}=\displaystyle\int\limits_0^2f(x)\mathrm{\,d}x+\displaystyle\int\limits_2^6f(x)\mathrm{\,d}x+\displaystyle\int\limits_6^{10}{f(x)\mathrm{\,d}x}$\\
% 		Suy ra $\displaystyle\int\limits_0^2f(x)\mathrm{\,d}x+\displaystyle\int\limits_6^{10}{f(x)\mathrm{\,d}x}=\displaystyle\int\limits_0^{10}{f(x)\mathrm{\,d}x}-\displaystyle\int\limits_2^6f(x)\mathrm{\,d}x=7-3=4$.}
% \end{ex}

% \begin{ex}%[2D4H2-1]
% 	Giả sử $\displaystyle\int\limits_0^1f(x)\mathrm{\,d}x=3$ và $\displaystyle\int\limits_0^5f(z)\mathrm{\,d}z=9$. Tổng $I=\displaystyle\int\limits_1^3f(t)\mathrm{\,d}t+\displaystyle\int\limits_3^5f(t)\mathrm{\,d}t$ bằng\\
% 	\shortans{$6$}
% 	\loigiai{
% 		$\displaystyle\int\limits_0^1f(x)\mathrm{\,d}x=3\Leftrightarrow\displaystyle\int\limits_0^1f(t)\mathrm{\,d}t=3\Leftrightarrow\displaystyle\int\limits_1^0f(t)\mathrm{\,d}t=-3.$\\
% 		$\displaystyle\int\limits_0^5f(z)\mathrm{\,d}z=9\Leftrightarrow\displaystyle\int\limits_0^5f(t)\mathrm{\,d}t=9.$\\
% 		$\Rightarrow\displaystyle\int\limits_1^0f(t)\mathrm{\,d}t+\displaystyle\int\limits_0^5f(t)\mathrm{\,d}t=6\Leftrightarrow\displaystyle\int\limits_1^5f(t)\mathrm{\,d}t=6.$\\
% 		$I=\displaystyle\int\limits_1^3f(t)\mathrm{\,d}t+\displaystyle\int\limits_3^5f(t)\mathrm{\,d}t=\displaystyle\int\limits_1^5f(t)\mathrm{\,d}t=6.$}
% \end{ex}
\Closesolutionfile{ans}
% \indapan{6}{ans/ans-2-C4B2CD3-KQ}
\begin{dang}{Tích phân hàm số sơ cấp}	
\end{dang}
\TN
\Opensolutionfile{ans}[ans/ans-C4B2CD1]
\begin{ex}%[2D4N2-2]%Câu 1
	Tích phân $ I=\displaystyle\int\limits_0^2(2x+1)\mathrm{\,d}x$ bằng
	\choice
	{$ I=5$}
	{\True $ I=6$}
	{$ I=2$}
	{$ I=4$}
	\loigiai{
		Ta có $ I=\displaystyle\int\limits_0^2(2x+1)\mathrm{\,d}x=\left(x^2+x\right)\big|_0^2=4+2=6$.}
\end{ex}
%
\begin{ex}%[2D4H2-2]%Câu 2
	Tích phân $\displaystyle\int\limits_0^1\left(3x+1\right)\left(x+3\right)\mathrm{\,d}x$ bằng
	\choice
	{$ 12$}
	{\True $ 9$}
	{$ 5$}
	{$ 6$}
	\loigiai{
		Ta có $\displaystyle\int\limits_0^1\left(3x+1\right)\left(x+3\right)\mathrm{\,d}x=\displaystyle\int\limits_0^1\left(3x^2+10x+3\right)\mathrm{\,d}x=\left(x^3+5x^2+3x\right)\big|_0^1=9$.\\
		Vậy $\displaystyle\int\limits_0^1\left(3x+1\right)\left(x+3\right)\mathrm{\,d}x=9$.}
\end{ex}
%
\begin{ex}%[2D4N2-2]%Câu 3
	Tính tích phân $ I=\displaystyle\int\limits_1^\mathrm{e}{\left(\dfrac{1}{x}-\dfrac{1}{x^2}\right)}\mathrm{\,d}x$
	\choice
	{\True $I=\dfrac{1}{\mathrm{e}}$}
	{$I=\dfrac{1}{\mathrm{e}}+1$}
	{$I=1$}
	{$I=\mathrm{e}$}
	\loigiai{
		$ I=\displaystyle\int\limits_1^\mathrm{e}{\left(\dfrac{1}{x}-\dfrac{1}{x^2}\right)}\mathrm{\,d}x=\left(\ln \left| x\right|+\dfrac{1}{x}\right)\Big|_1^\mathrm{e}=\dfrac{1}{\mathrm{e}}$.}
\end{ex}

\begin{ex}%[2D4N2-2]%Câu 4
	Biết $\displaystyle\int\limits_1^3\dfrac{x+2}{x}\mathrm{\,d}x=a+b\ln c,$ với $a$, $b$, $c\in\mathbb{Z}$, $c<9.$ Tính tổng $S=a+b+c.$
	\choice
	{\True $ S=7$}
	{$ S=5$}
	{$ S=8$}
	{$ S=6$}
	\loigiai{
		Ta có $\displaystyle\int\limits_1^3\dfrac{x+2}{x}\mathrm{\,d}x=\displaystyle\int\limits_1^3\left(1+\dfrac{2}{x}\right)\mathrm{\,d}x=\displaystyle\int\limits_1^3\mathrm{d}x+\displaystyle\int\limits_1^3\dfrac{2}{x}\mathrm{d}x=2+2\ln \left| x\right|\big|_1^3=2+2\ln 3.$\\
		Do đó $ a=2$, $b=2$, $c=3\Rightarrow S=7.$}
\end{ex}
%
\begin{ex}%[2D4H2-4]%Câu 5
	Tích phân $\displaystyle\int\limits_0^1\mathrm{e}^{3x+1}\mathrm{\,d}x$ bằng
	\choice
	{$\dfrac{1}{3}\left(\mathrm{e}^4+\mathrm{e}\right)$}
	{$\mathrm{e}^3-\mathrm{e}$}
	{\True $\dfrac{1}{3}\left(\mathrm{e}^4-\mathrm{e}\right)$}
	{$\mathrm{e}^4-\mathrm{e}$}
	\loigiai{
		$\displaystyle\int\limits_0^1\mathrm{e}^{3x+1}\mathrm{\,d}x=\dfrac{1}{3}\displaystyle\int\limits_0^1\mathrm{e}^{3x+1}\mathrm{\,d}\left(3x+1\right)=\dfrac{1}{3}{\mathrm{e}^{3x+1}}\big|_0^1=\dfrac{1}{3}\left(\mathrm{e}^4-\mathrm{e}\right)$.}
\end{ex}

\begin{ex}%[2D4H2-4]%Câu 6
	Biết $\displaystyle\int\limits_0^1\dfrac{\mathrm{e}^x}{2^x}\mathrm{\,d}x=\dfrac{\mathrm{e-1}}{a-\ln b }$, $\left(a,b\in\mathbb{Z}\right)$. Khi đó giá trị của $ P=a+b$ là
	\choice
	{$ P=-3$}
	{\True $ P=6$}
	{$ P=-1$}
	{$ P=3$}
	\loigiai{
		$ I=\displaystyle\int\limits_0^1\dfrac{\mathrm{e}^x}{2^x}\mathrm{\,d}x=\displaystyle\int\limits_0^1\left(\dfrac{\mathrm{e}}{2}\right)^x\mathrm{\,d}x=\left[\left(\dfrac{\mathrm{e}}{2}\right)^x\cdot\dfrac{1}{1-\ln 2}\right]\Big|_0^1=\dfrac{\mathrm{e}-1}{2-\ln 4}$.}
\end{ex}

\begin{ex}%[2D4H2-4]%Câu 7
	Giá trị của $ I=\displaystyle\int\limits_0^1\dfrac{\mathrm{e}^{2x}-4}{\mathrm{e}^x+2}\mathrm{\,d}x$ bằng
	\choice
	{$ I=2\left(\mathrm{e}+3\right)$}
	{$ I=\dfrac{1}{2}\left(\mathrm{e}+3\right)$}
	{\True $ I=\mathrm{e}-3$}
	{$ I=2\left(\mathrm{e}-3\right)$}
	\loigiai{
		$ I=\displaystyle\int\limits_0^1\dfrac{\mathrm{e}^{2x}-4}{\mathrm{e}^x+2}\mathrm{\,d}x=\displaystyle\int\limits_0^1\dfrac{\left(\mathrm{e}^x-2\right)\left(\mathrm{e}^x+2\right)}{\mathrm{e}^x+2}\mathrm{\,d}x=\displaystyle\int\limits_0^1\left(\mathrm{e}^x-2\right)\mathrm{\,d}x=\left(\mathrm{e}^x-2x\right)\big|_0^1=e-3$.}
\end{ex}
%
\begin{ex}%[2D4H2-4]%Câu 8
	Biết $\displaystyle\int\limits_1^2\mathrm{e}^x\left(1-\dfrac{\mathrm{e}^{-x}}{x}\right)\mathrm{d}x=\mathrm{e}^2+a\cdot \mathrm{e}+b\ln 2$, $\left(a,b\in\mathbb{Z}\right)$. Khi đó giá trị của $ P=\dfrac{a+b}{a\cdot b}$ là
	\choice
	{$ P=-3$}
	{$ P=1$}
	{$ P=-1$}
	{\True $ P=-2$}
	\loigiai{
		$ I=\displaystyle\int\limits_1^2\mathrm{e}^x\left(1-\dfrac{\mathrm{e}^{-x}}{x}\right)\mathrm{\,d}x=\displaystyle\int\limits_1^2\left(\mathrm{e}^x-\dfrac{1}{x}\right)\mathrm{\,d}x=\left(\mathrm{e}^x-\ln \left| x\right|\right)\big|_1^2=\mathrm{e}^2-\mathrm{e}-\ln 2$.}
\end{ex}

\begin{ex}%[2D4H2-4]%Câu 9
	Biết $ I=\displaystyle\int\limits_0^1\dfrac{\mathrm{e}^{2x-1}-\mathrm{e}^{-3x}+1}{\mathrm{e}^x}\mathrm{\,d}x=\dfrac{1}{a}+b$, $\left(a,b\in\mathbb{R}\right)$. Khi đó giá trị của $ P=\dfrac{a+b}{a\cdot b}$ là
	\choice
	{$ P=\mathrm{e}^4-1$}
	{$ P=\dfrac{\mathrm{e}^4-1}{\mathrm{e}^2}$}
	{$ P=\dfrac{\mathrm{e}^4-1}{\mathrm{e}^4}$}
	{\True $ P=\dfrac{1-\mathrm{e}^4}{\mathrm{e}^4}$}
	\loigiai{
		\allowdisplaybreaks
		\begin{eqnarray*} I&=&\displaystyle\int\limits_0^1\dfrac{\mathrm{e}^{2x-1}-\mathrm{e}^{-3x}+1}{\mathrm{e}^x}\mathrm{\,d}x=\displaystyle\int\limits_0^1\left(\mathrm{e}^{x-1}-\mathrm{e}^{-4x}+\mathrm{e}^{-x}\right)\mathrm{\,d}x\\
			&=&\left(\mathrm{e}^{x-1}-\dfrac{\mathrm{e}^{-4x}}{-4}+\dfrac{\mathrm{e}^{-x}}{-1}\right)\Big|_0^1=\dfrac{1-\mathrm{e}^4}{\mathrm{e}^4}=\dfrac{1}{\mathrm{e}^4}-1
		\end{eqnarray*}
		$\Rightarrow P=\dfrac{a+b}{a\cdot b}=\dfrac{1-\mathrm{e}^4}{\mathrm{e}^4}$.}
\end{ex}
%
\begin{ex}%[2D4N2-3]%Câu 10
	Giá trị của $\displaystyle\int\limits_0^{\frac{\pi}{2}}{\sin x\mathrm{\,d}x}$ bằng
	\choice
	{0}
	{\True 1}
	{$-1$}
	{$\dfrac{\pi}{2}$}
	\loigiai{
		Tính được $\displaystyle\int\limits_0^{\frac{\pi}{2}}{\sin x\mathrm{\,d}x}=-\cos x\Big|_0^{\frac{\pi}{2}}=1$.}
\end{ex}

\begin{ex}%[2D4H2-3]%Câu 11
	Biết $\displaystyle\int\limits_{\tfrac{\pi}{3}}^{\tfrac{\pi}{2}}{\left(2\sin x+3\cos x+x\right)\mathrm{\,d}x}=\dfrac{a+b\sqrt{3}}{2}+\dfrac{\pi^2}{c}$, $\left(a,b,c\in\mathbb{Z}\right)$. Khi đó giá trị của $ P=a+2b+3c$ là
	\choice
	{$ P=45$}
	{\True $ P=60$}
	{$ P=65$}
	{$ P=70$}
	\loigiai{
		$\displaystyle\int\limits_{\tfrac{\pi}{3}}^{\tfrac{\pi}{2}}\left(2\sin x+3\cos x+x\right)\mathrm{\,d}x=\left(-2\cos x+3\sin x+\dfrac{1}{2}{x^2}\right)\Big|_{\tfrac{\pi}{3}}^{\tfrac{\pi}{2}}=\dfrac{12-3\sqrt{3}}{2}+\dfrac{\pi^2}{18}$\\
		$\Rightarrow P=a+2b+3c=60$.
	}
\end{ex}

\begin{ex}%[2D4H2-3]%Câu 12
	Biết $\displaystyle\int\limits_{\tfrac{\pi}{4}}^{\tfrac{\pi}{3}}{3\tan^2x\mathrm{\,d}x}=a\sqrt{3}+b+\dfrac{\pi}{c}$, $\left(a,b,c\in\mathbb{Z}\right)$. Khi đó giá trị của $ P=a+b+c$ là
	\choice
	{$ P=6$}
	{\True $ P=-4$}
	{$ P=4$}
	{$ P=-6$}
	\loigiai{
		$\displaystyle\int\limits_{\tfrac{\pi}{4}}^{\tfrac{\pi}{3}}{3\tan^2x\mathrm{\,d}x}=3\displaystyle\int\limits_{\tfrac{\pi}{4}}^{\tfrac{\pi}{3}}{\left(\dfrac{1}{\cos^2x}-1\right)\mathrm{\,d}x= 3\left(\tan x-x\right)\big|_{\tfrac{\pi}{4}}^{\tfrac{\pi}{3}}=3\sqrt{3}-3-\dfrac{\pi}{4}}$\\
		$\Rightarrow P=a+b+c=3-3-4=-4$.}
\end{ex}
%
\begin{ex}%[2D4H2-3]%Câu 13
	Biết $\displaystyle\int\limits_{\tfrac{\pi}{6}}^{\tfrac{\pi}{4}}{\left(2\cot^2x+5\right)\mathrm{\,d}x}=\dfrac{\pi}{a}+b\sqrt{3}+c$, $\left(a,b,c\in\mathbb{Z}\right)$. Khi đó giá trị của \break $ P=a+b+c$ là
	\choice
	{$ P=6$}
	{$ P=-4$}
	{\True $ P=4$}
	{$ P=-6$}
	\loigiai{\allowdisplaybreaks
		\begin{eqnarray*}
			\displaystyle\int\limits_{\tfrac{\pi}{6}}^{\tfrac{\pi}{4}}{\left(2\cot^2x+5\right)\mathrm{\,d}x}&=&\displaystyle\int\limits_{\tfrac{\pi}{6}}^{\tfrac{\pi}{4}}{\left(2\left(\dfrac{1}{\sin^2x}-1\right)+5\right)\mathrm{\,d}x}\\
			&=&\displaystyle\int\limits_{\dfrac{\pi}{6}}^{\dfrac{\pi}{4}}{\left(3-\dfrac{-2}{\sin^2x}\right)\mathrm{\,d}x=\left(3x-\cot x\right)\Big|_{\tfrac{\pi}{6}}^{\tfrac{\pi}{4}}=\dfrac{\pi}{4}+\sqrt{3}-1}.
	\end{eqnarray*}}
\end{ex}

\begin{ex}%[2D4H2-3]%Câu 14
	Biết $\displaystyle\int\limits_0^{\tfrac{\pi}{2}}\sin^2\dfrac{x}{4}{\cos^2}\dfrac{x}{4}\mathrm{\,d}x=\dfrac{\pi}{c}+\dfrac{a}{b}$ với $a$, $b\in\mathbb{Z}$ và $\dfrac{a}{b}$ là phân số tối giản. Khi đó giá trị của $ P=a+b+c$ là
	\choice
	{$ P=17$}
	{$ P=16$}
	{$ P=32$}
	{\True $ P=49$}
	\loigiai{\allowdisplaybreaks
		\begin{eqnarray*}
			\displaystyle\int\limits_0^{\tfrac{\pi}{2}}{\sin^2\dfrac{x}{4}{\cos^2}\dfrac{x}{4}\mathrm{\,d}x}&=&\dfrac{1}{4}\displaystyle\int\limits_0^{\tfrac{\pi}{2}}\sin^2\dfrac{x}{2}\mathrm{\,d}x\\
			&=&\dfrac{1}{4}\displaystyle\int\limits_0^{\tfrac{\pi}{2}}\left(\dfrac{1-\cos x}{2}\right)\mathrm{\,d}x\\
			&=&\dfrac{1}{8}\left(x-\dfrac{1}{4}\sin x\right)\Big|_0^{\tfrac{\pi}{2}}=\dfrac{\pi}{16}+\dfrac{1}{32}.
		\end{eqnarray*}
		$\Rightarrow P=a+b+c=1+32+16=49$.}
\end{ex}
\Closesolutionfile{ans}
% \indapan{6}{ans/ans-C4B2CD1}
\TNTF
\Opensolutionfile{ans}[ans/ans-C4B2CD1-DS]
\begin{ex}%[2D4H2-1]%Câu 15
	Cho hàm số $y=f(x)$ liên tục trên $\left[a;b\right]$. Các mệnh đề sau đây đúng hay sai?
	\choiceTF
	{$\displaystyle\int\limits_a^b{f(x)\mathrm{\,d}x}=\displaystyle\int\limits_b^a{f(x)\mathrm{\,d}x}$}
	{\True $\displaystyle\int\limits_a^b{f(x)\mathrm{\,d}x}=-\displaystyle\int\limits_b^a{f(x)\mathrm{\,d}x}$}
	{$\displaystyle\int\limits_a^bf(x)\mathrm{\,d}x=2\displaystyle\int\limits_a^bf(x)\mathrm{\,d}\left(2x\right)$}
	{\True $\displaystyle\int\limits_a^a{2024f(x)\mathrm{\,d}x=0}$}
	\loigiai{
		\begin{itemchoice}
			\itemch Sai. Vì
			$\displaystyle\int\limits_a^b{f(x)\mathrm{\,d}x}=-\displaystyle\int\limits_b^a{f(x)\mathrm{\,d}x}$.
			\itemch Đúng. Vì $\displaystyle\int\limits_a^b{f(x)\mathrm{\,d}x}=-\displaystyle\int\limits_b^a{f(x)\mathrm{\,d}x}$.
			\itemch Sai. Vì $2\displaystyle\int\limits_a^bf(x)\mathrm{\,d}\left(2x\right)=4\displaystyle\int\limits_a^bf(x)\mathrm{\,d}\left(x\right)$.
			\itemch Đúng. 
			$\displaystyle\int\limits_a^a2024f(x)\mathrm{\,d}x=0.$
		\end{itemchoice}
	}
\end{ex}
%
\begin{ex}%[2D4H2-1]%Câu 16
	Cho hàm số $y=f(x)$, $y=g(x)$ liên tục trên $\left[a;b\right]$. Các mệnh đề sau đây đúng hay sai?
	\choiceTF
	{\True $\displaystyle\int\limits_a^b{\left[f(x)+g(x)\right]\mathrm{\,d}x}=\displaystyle\int\limits_a^b{f(x)}\mathrm{\,d}x+\displaystyle\int\limits_a^b{g(x)\mathrm{\,d}x}$}
	{$\displaystyle\int\limits_a^b{f(x)\cdot g(x)\mathrm{\,d}x}=\displaystyle\int\limits_a^b{f(x)\mathrm{\,d}x}\cdot\displaystyle\int\limits_a^b{g(x)\mathrm{\,d}x}$}
	{\True $\displaystyle\int\limits_a^b{kf(x)\mathrm{\,d}x=k\displaystyle\int\limits_a^b{f(x)\mathrm{\,d}x}}$}
	{$\displaystyle\int\limits_a^b{\dfrac{f(x)}{g(x)}\mathrm{\,d}x}=\dfrac{\displaystyle\int\limits_a^bf(x)\mathrm{\,d}x}{\displaystyle\int\limits_a^bg(x)\mathrm{\,d}x}$}
	\loigiai{
		\begin{itemchoice}
			\itemch Đúng.
			$\displaystyle\int\limits_a^b{\left[f(x)+g(x)\right]\mathrm{\,d}x}=\displaystyle\int\limits_a^b{f(x)}\mathrm{\,d}x+\displaystyle\int\limits_a^b{g(x)\mathrm{\,d}x}$.
			\itemch Sai. Vì không có tính chất.
			\itemch Đúng.
			$\displaystyle\int\limits_a^b{kf(x)\mathrm{\,d}x=k\displaystyle\int\limits_a^b{f(x)\mathrm{\,d}x}}$.
			\itemch Sai.
	\end{itemchoice}}
\end{ex}
%
% \begin{ex}%[2D4H2-1]%Câu 17
% 	Cho hàm số $y=f(x)$ liên tục trên $\mathbb{R}$ và $a$, $b$, $c\in\mathbb{R}$ thỏa mãn $a<b<c$. Các mệnh đề sau đây đúng hay sai?
% 	\choiceTF
% 	{$\displaystyle\int\limits_a^c{f(x)\mathrm{\,d}x=\displaystyle\int\limits_a^b{f(x)\mathrm{\,d}x}}\cdot \displaystyle\int\limits_b^c{f(x)\mathrm{\,d}x}$}
% 	{\True $\displaystyle\int\limits_a^c{f(x)\mathrm{\,d}x=\displaystyle\int\limits_a^b{f(x)\mathrm{\,d}x}}+\displaystyle\int\limits_b^c{f(x)\mathrm{\,d}x}$}
% 	{$\displaystyle\int\limits_a^c{f(x)\mathrm{\,d}x=\displaystyle\int\limits_a^b{f(x)\mathrm{\,d}x}}-\displaystyle\int\limits_b^c{f(x)\mathrm{\,d}x}$}
% 	{$\displaystyle\int\limits_a^c{f(x)\mathrm{\,d}x=\displaystyle\int\limits_a^b{f(x)\mathrm{\,d}x}}+\displaystyle\int\limits_c^b{f(x)\mathrm{\,d}x}$}
% 	\loigiai{\begin{itemchoice}
% 			\itemch Sai. Không đúng với lý thuyết.
% 			\itemch Đúng. $\displaystyle\int\limits_a^c{f(x)\mathrm{\,d}x=\displaystyle\int\limits_a^b{f(x)\mathrm{\,d}x}}+\displaystyle\int\limits_b^c{f(x)\mathrm{\,d}x}$.
% 			\itemch Sai.
% 			\itemch Sai.
% 	\end{itemchoice}}
% \end{ex}
% %
% \begin{ex}%[2D4H2-1]%Câu 18
% 	Cho $f(x)$, $g(x)$ là hai hàm số liên tục trên $\mathbb{R}$. Các mệnh đề sau đây đúng hay sai?
% 	\choiceTF
% 	{\True $\displaystyle\int\limits_a^bf(x)\mathrm{\,d}x=\displaystyle\int\limits_a^bf(y)\mathrm{\,d}y$}
% 	{\True $\displaystyle\int\limits_a^b{\left(f(x)+g(x)\right)\mathrm{\,d}x}=\displaystyle\int\limits_a^b{f(x)\mathrm{\,d}x+\displaystyle\int\limits_a^b{g(x)\mathrm{\,d}x}}$}
% 	{$\displaystyle\int\limits_a^b{f(x)\mathrm{\,d}x=\displaystyle\int\limits_a^b{f(t)\mathrm{\,d}x}}$}
% 	{$\displaystyle\int\limits_a^b{\left(f(x)g(x)\right)\mathrm{\,d}x}=\displaystyle\int\limits_a^b{f(x)\mathrm{\,d}x\displaystyle\int\limits_a^b{g(x)\mathrm{\,d}x}}$}
% 	\loigiai{
% 		\begin{itemchoice}
% 			\itemch Đúng. $\displaystyle\int\limits_a^b{f(x)\mathrm{\,d}x=\displaystyle\int\limits_a^b{f(y)\mathrm{\,d}}y}$
% 			\itemch Đúng. $\displaystyle\int\limits_a^b{\left(f(x)+g(x)\right)\mathrm{\,d}x}=\displaystyle\int\limits_a^bf(x)\mathrm{\,d}x+\displaystyle\int\limits_a^b g(x)\mathrm{\,d}x$.
% 			\itemch Sai. Không đúng với lý thuyết.
% 			\itemch Sai. Không đúng với lý thuyết.
% 		\end{itemchoice}
% 	}
% \end{ex}

% \begin{ex}%[2D4H2-1]%Câu 19
% 	Các mệnh đề sau đây đúng hay sai?
% 	\choiceTF
% 	{\True $\displaystyle\int\limits_{-2024}^{2024}\mathrm{\,d}x=4048$}
% 	{$\displaystyle\int\limits_a^bf_1(x)\cdot f_2(x)\mathrm{\,d}x=\displaystyle\int\limits_a^bf_1(x)\mathrm{\,d}x\cdot\displaystyle\int\limits_a^bf_2(x)\mathrm{\,d}x$}
% 	{\True Cho hàm số $f(x)$ liên tục trên đoạn $\left[a;b\right]$. Khi đó $\dfrac{1}{b-a}\displaystyle\int\limits_a^bf(x)\mathrm{\,d}x$ được gọi là giá trị trung bình của hàm số $f(x)$ trên đoạn $\left[a;b\right]$}
% 	{\True Nếu hàm số $f(x)$ có đạo hàm $f'(x)$ và $f'(x)$ liên tục trên đoạn $\left[a;b\right]$ thì $f(b)-f(a)=\displaystyle\int\limits_a^bf'(x)\mathrm{\,d}x$}
% 	\loigiai{\begin{itemchoice}
% 			\itemch Đúng.
% 			\itemch Sai. 
% 			\itemch Đúng.
% 			\itemch Đúng.
% 		\end{itemchoice}
		
% 	}
% \end{ex}
%
\begin{ex}%[2D4H2-1]%Câu 20
	Cho hàm $ f(x)$ là hàm liên tục trên đoạn $\left[a;b\right]$ với $ a<b$ và $F(x)$ là một nguyên hàm của hàm $ f(x)$ trên $\left[a;b\right]$. Các mệnh đề sau đây đúng hay sai?
	\choiceTF
	{\True $\displaystyle\int\limits_a^b{kf(x)\mathrm{\,d}x}=k\left[F(b)-F(a)\right]$}
	{$\displaystyle\int\limits_b^af(x)\mathrm{\,d}x=F(b)-F(a)$}
	{Diện tích $S$ của hình phẳng giới hạn bởi đường thẳng $x=a$; $x=b$; đồ thị của hàm số $ y=f(x)$ và trục hoành được tính theo công thức $ S=F(b)-F(a)$}
	{$\displaystyle\int\limits_a^b{f\left(2x+3\right)\mathrm{\,d}x}=F\left(2x+3\right)\big|_a^b$}
	\loigiai{
		\begin{itemchoice}
			\itemch Đúng.
			\itemch Sai. $\displaystyle\int\limits_b^a{f(x)\mathrm{\,d}x}=F(a)-F(b)$. 
			\itemch Sai. Diện tích $S$ của hình phẳng giới hạn bởi đường thẳng $x=a$; $x=b$; đồ thị của hàm số $ y=f(x)$ và trục hoành được tính theo công thức $ S=|F(b)-F(a)|$.
			\itemch Sai. $\displaystyle\int\limits_a^bf\left(2x+3\right)\mathrm{\,d}x=\dfrac12 F\left(2x+3\right)\big|_a^b$
		\end{itemchoice}
	}
\end{ex}
%
\begin{ex}%[2D4H2-4]%Câu 21
	Các mệnh đề sau đây đúng hay sai.
	\choiceTF
	{\True $\displaystyle\int\limits_0^1\dfrac{\mathrm{e}^{2x}-4}{\mathrm{e}^x+2}\mathrm{\,d}x=\mathrm{e}-3$}
	{$\displaystyle\int\limits_0^1\dfrac{\mathrm{e}^x}{2^x}\mathrm{\,d}x=\dfrac{\mathrm{e}}{2}+1$}
	{\True $\displaystyle\int\limits_1^2\mathrm{e}^x\left(1-\dfrac{\mathrm{e}^{-x}}{x}\right)\mathrm{\,d}x=\mathrm{e}^2-\mathrm{e}-\ln 2$}
	{$\displaystyle\int\limits_0^1\dfrac{\mathrm{e}^{2x-1}-\mathrm{e}^{-3x}+1}{\mathrm{e}^x}\mathrm{\,d}x=\mathrm{e}^4-1$}
	\loigiai{\begin{itemchoice}
			\itemch Đúng. \allowdisplaybreaks
			\begin{eqnarray*} \displaystyle\int\limits_0^1\dfrac{\mathrm{e}^{2x}-4}{\mathrm{e}^x+2}\mathrm{\,d}x&=&\displaystyle\int\limits_0^1\dfrac{\left(\mathrm{e}^x-2\right)\left(\mathrm{e}^x+2\right)}{\mathrm{e}^x+2}\mathrm{\,d}x\\
				&=&\displaystyle\int\limits_0^1\left(\mathrm{e}^x-2\right)\mathrm{\,d}x=\left(\mathrm{e}^x-2x\right)\big|_0^1=\mathrm{e}-3.
			\end{eqnarray*}
			\itemch Sai.  $\displaystyle\int\limits_0^1\dfrac{\mathrm{e}^x}{2^x}\mathrm{\,d}x=\displaystyle\int\limits_0^1\left(\dfrac{\mathrm{e}}{2}\right)^x\mathrm{\,d}x=\left[\left(\dfrac{\mathrm{e}}{2}\right)^x\right]\Big|_0^1=\dfrac{\mathrm{e}}{2}-1$.
			\itemch Đúng. $\displaystyle\int\limits_1^2\mathrm{e}^x\left(1-\dfrac{\mathrm{e}^{-x}}{x}\right)\mathrm{\,d}x=\displaystyle\int\limits_1^2\left(\mathrm{e}^x-\dfrac{1}{x}\right)\mathrm{\,d}x=\left(\mathrm{e}^x-\ln \left| x\right|\right)\big|_1^2=\mathrm{e}^2-\mathrm{e}-\ln 2$.
			\itemch Sai.\allowdisplaybreaks
			\begin{eqnarray*} \displaystyle\int\limits_0^1\dfrac{\mathrm{e}^{2x-1}-\mathrm{e}^{-3x}+1}{\mathrm{e}^x}\mathrm{\,d}x&=&\displaystyle\int\limits_0^1\left(\mathrm{e}^{x-1}-\mathrm{e}^{-4x}+\mathrm{e}^{-x}\right)\mathrm{\,d}x\\
				&=&\left(\mathrm{e}^{x-1}-\mathrm{e}^{-4x}+\mathrm{e}^{-x}\right)\big|_0^1=\dfrac{1-\mathrm{e}^4}{\mathrm{e}^4}=\mathrm{e}^{-4}-1.
			\end{eqnarray*}
		\end{itemchoice}
	}
\end{ex}
\Closesolutionfile{ans}
% \indapan{3}{ans/ans-C4B2CD1-DS}
\TNSA
\Opensolutionfile{ans}[ans/ans-C4B2CD1-KQ]
\begin{ex}%[2D4H2-2]%Câu 22
	Với $a$, $b$ là các tham số thực. Tích phân $$I=\displaystyle\int\limits_0^b\left(3x^2-2ax-1\right)\mathrm{\,d}x=b^t-b^ya+zb.$$ Tính $t+y+z$.
	\shortans{$4$}
	\loigiai{
		Ta có $\displaystyle\int\limits_0^b{\left(3x^2-2ax-1\right)\mathrm{\,d}x}=\left(x^3-a{x^2}-x\right)\big|_0^b=b^3-a{b^2}-b$. \\
		Suy ra $t=3$, $y=2$, $z=-1$ nên $t+y+z=4$.}
\end{ex}

\begin{ex}%[2D4H2-2]%Câu 23
	Cho $\displaystyle\int\limits_0^m{\left(3x^2-2x+1\right)}\mathrm{\,d}x=6$. Tính giá trị của tham số $m$.
	\shortans{$2$}
	\loigiai{
		Ta có $\displaystyle\int\limits_0^m{\left(3x^2-2x+1\right)}\mathrm{\,d}x=6\Leftrightarrow\left.\left(x^3-x^2+x\right)\right|_0^m=6\Leftrightarrow{m^3}-m^2+m-6=0\Leftrightarrow m=2$.}
\end{ex}
%%%==============EX_1============%%%
\begin{ex}%[2D4H2-2]
	Tính tích phân $I=\displaystyle\int\limits\limits_1^2\dfrac{x-1}{x} \mathrm{d}x$ (\textit{\textit{làm tròn đến hàng phần trăm}}).
	\shortans{$0{,}31$}	
	\loigiai{
		\begin{eqnarray*}
			I	&= &\displaystyle\int\limits_1^2\dfrac{x-1}{x} \mathrm{d}x\\
			&=& \displaystyle\int\limits_1^2\left(1-\dfrac{1}{x} \right) \mathrm{d}x\\
			&= & \left(x-\ln |x|\right)\Bigg|_1^2\\
			&=&	\left(2-\ln 2\right)-\left(1-\ln 1\right)=1-\ln 2.
	\end{eqnarray*}}
\end{ex}
%%%==============EX_2============%%%
\begin{ex}%[2D4H2-2]
	Tính $I=\displaystyle\int\limits_1^2\left(\dfrac{x-\sqrt[{4}]{x^3}}{x} \right)^2 \mathrm{\,d}x$ (\textit{\textit{làm tròn đến hàng phần trăm}}).
	\shortans{$0{,}01$}	
	\loigiai{
		\begin{eqnarray*}
			I	&= &\displaystyle\int\limits_1^2\left(\dfrac{x-\sqrt[{4}]{x^3}}{x} \right)^2 \mathrm{\,d}x\\
			&=& \displaystyle\int\limits_1^2\left(1-x^{-\tfrac{1}{4}}\right)^2 \mathrm{\,d}x\\
			&= &\displaystyle\int\limits_1^2\left(1-2x^{-\tfrac{1}{4}}+x^{-\tfrac{1}{8}}\right) \mathrm{\,d}x\\
			&=&	\left(x-\dfrac{8}{3}x^{\tfrac{3}{4}}+\dfrac{8}{7}x^{\tfrac{7}{8}} \right)\Bigg|_1^2\\
			&\approx& 0{,}01. 
		\end{eqnarray*}
	}
\end{ex}
%%%==============EX_3============%%%
\begin{ex}%[2D4H2-2]
	Tính $I=\displaystyle\int\limits_1^2\left(\sqrt{x}+1\right)\left(\sqrt[{3}]{x}-1\right)\mathrm{\,d}x$ (\textit{\textit{làm tròn đến hàng phần trăm}}).
	\shortans{$0{,}32$}	
	\loigiai{
		\begin{eqnarray*}
			I&= &\displaystyle\int\limits_1^2\left(\sqrt{x}+1\right)\left(\sqrt[{3}]{x}-1\right)\mathrm{\,d}x\\
			&=& \displaystyle\int\limits_1^2\left(x^{\tfrac{5}{6}}-x^{\tfrac{1}{2}}+x^{\tfrac{1}{3}}-1\right) \mathrm{\,d}x\\
			&=&	\left(\dfrac{6}{11}x^{\tfrac{11}{6}}-\dfrac{2}{3}x^{\tfrac{3}{2}}+\dfrac{3}{4}x^{\tfrac{4}{3}}-x \right)\Bigg|_1^2\\
			&\approx& 0{,}32. 
		\end{eqnarray*}
	}
\end{ex}
%%%==============EX_4============%%%
\begin{ex}%[2D4H2-2]
	Tính $I=\displaystyle\int\limits_1^2\dfrac{(x^2+1)^3}{x^2} \mathrm{\,d}x$ (\textit{làm tròn đến hàng phần chục}).
	\shortans{$16{,}7$}	
	\loigiai{
		\begin{eqnarray*}
			I&= &\displaystyle\int\limits_1^2\dfrac{(x^2+1)^3}{x^2} \mathrm{\,d}x\\
			&=& \displaystyle\int\limits_1^2\left(x^4+3x^2+3+\dfrac{1}{x^2}\right) \mathrm{\,d}x\\
			&=&	\left(\dfrac{x^5}{5}+x^3+3x-\dfrac{1}{x}\right)\Bigg|_1^2\\
			&=& 16{,}7. 
		\end{eqnarray*}
	}
\end{ex}
%%%==============EX_5============%%%
\begin{ex}%[2D4H2-4]
	Tính $I=\displaystyle\int\limits _0^15^{x+1}\cdot7^{2x-1} \mathrm{\,d}x$ (\textit{làm tròn đến hàng đơn vị}).
	\shortans{$959$}	
	\loigiai{
		\begin{eqnarray*}
			I&= &\displaystyle\int\limits _0^15^{x+1}\cdot7^{2x-1} \mathrm{\,d}x\\
			&=&\dfrac{5}{7} \displaystyle\int\limits_0^15^x\cdot49^x \mathrm{\,d}x\\
			&=&	\dfrac{5}{7} \displaystyle\int\limits_0^1245^x \mathrm{\,d}x\\
			&=&	\dfrac{5}{7}\left(245^x\ln 245\right)\Bigg|_0^1\\
			&=&\dfrac{5}{7}\left(245\ln 245-\ln 245\right)\approx 959. 
		\end{eqnarray*}
	}
\end{ex}
%%%==============EX_6============%%%
\begin{ex}%[2D4H2-4]
	Tính $I=\displaystyle\int\limits _0^1\left(x+\mathrm{e}^{-x-2} \right)\mathrm{\,d}x$ (\textit{\textit{làm tròn đến hàng phần trăm}}).
	\shortans{$0{,}59$}	
	\loigiai{
		\begin{eqnarray*}
			I&= &\displaystyle\int\limits _0^1\left(x+\mathrm{e}^{-x-2} \right)\mathrm{\,d}x\\
			&=&	\left(\dfrac{x^2}{2}-\mathrm{e}^{-x-2}\right)\Bigg|_0^1\\
			&=&\left(\dfrac{1}{2}+\mathrm{e}^{-2}-\mathrm{e}^{-3}\right)\approx 0{,}59. 
		\end{eqnarray*}
	}
\end{ex}
%%%==============EX_7============%%%
\begin{ex}%[2D4H2-3]
	Tính $I=\displaystyle\int\limits _{\tfrac{\pi}{6}}^{\tfrac{\pi}{3}}x^2 \left(1-\dfrac{\sin x}{x^2} \right)\mathrm{\,d}x$ (\textit{\textit{làm tròn đến hàng phần trăm}}).
	\shortans{$-0{,}03$}	
	\loigiai{
		\begin{eqnarray*}
			I&= &\displaystyle\int\limits _{\tfrac{\pi}{6}}^{\tfrac{\pi}{3}}x^2 \left(1-\dfrac{\sin x}{x^2} \right)\mathrm{\,d}x\\
			&=&\displaystyle\int\limits _{\tfrac{\pi}{6}}^{\tfrac{\pi}{3}}\left(x^2-\sin x \right)\mathrm{\,d}x\\
			&=&\left(\dfrac{x^3}{3}+\cos x\right)\Bigg| _{\tfrac{\pi}{6}}^{\tfrac{\pi}{3}}\approx -0{,}03.  
		\end{eqnarray*}
	}
\end{ex}
%%%==============EX_8============%%%
\begin{ex}%[2D4H2-3]
	Tính $I=\displaystyle\int\limits _{\tfrac{\pi}{6}}^{\tfrac{\pi}{2}}\left(\sin x-\dfrac{1}{\sqrt[{3}]{x^2}} \right) \mathrm{\,d}x$ \textit{(\textit{làm tròn đến hàng phần trăm})}.
	\shortans{$0{,}38$}	
	\loigiai{
		\begin{eqnarray*}
			I&= &\displaystyle\int\limits _{\tfrac{\pi}{6}}^{\tfrac{\pi}{2}}\left(\sin x-\dfrac{1}{\sqrt[{3}]{x^2}} \right) \mathrm{\,d}x\\
			&=&\left(-\cos x-3\sqrt[{3}]{x}\right)\Bigg| _{\tfrac{\pi}{6}}^{\tfrac{\pi}{2}}\approx 0{,}38.  
		\end{eqnarray*}
	}
\end{ex}
%%%==============EX_9============%%%
\begin{ex}%[2D4H2-4]
	Biết $\displaystyle\int\limits _0^1\dfrac{\left(e^{-x}+2\right)^2}{e^{x-1}} \mathrm{\,d}x=ae+b+\dfrac{c}{e}+\dfrac{1}{e^2}$ $\left(a,b,c\in \mathbb{Z}\right)$. Tính giá trị của $P=a+b+c$.
	\shortans{$-1$}	
	\loigiai{
		\begin{eqnarray*}
			I	&=& \displaystyle\int\limits _0^1\dfrac{\left(e^{-x}+2\right)^2}{e^{x-1}} \mathrm{\,d}x\\
			&= & \displaystyle\int\limits _0^1\dfrac{e^{-2x}+4e^{-x}+4}{e^{x-1}} \mathrm{\,d}x\\
			&=&	\displaystyle\int\limits _0^1\left(e^{-3x+1}+4e^{-2x+1}+4e^{-x+1} \right)\mathrm{\,d}x\\
			&=&\left. \left(\dfrac{e^{-3x+1}}{-3}+\dfrac{4e^{-2x+1}}{-2}+\dfrac{4e^{-x+1}}{-1} \right)\right|_0^1\\
			&=&\dfrac{-9e^3+4e^2+4e+1}{e^2}=-9e+4+\dfrac{4}{e}+\dfrac{1}{e^2}.
		\end{eqnarray*}
		Vậy $ P=a+b+c=-1$.
	}
\end{ex}
%%%==============EX_10============%%%
\begin{ex}%[2D4H2-3]
	Biết $\displaystyle\int\limits _0^{\tfrac{\pi}{3}}\dfrac{1-\cos 2x}{1+\cos 2x} \mathrm{\,d}x=a\sqrt{3}+\dfrac{\pi}{b}$ $\left(a,b\in \mathbb{Z}\right)$. Tính $a+b$.
	\shortans{$0$}	
	\loigiai{
		\begin{eqnarray*}
			I&= & \displaystyle\int\limits _0^{\tfrac{\pi}{3}}\dfrac{1-\cos 2x}{1+\cos 2x} \mathrm{\,d}x\\
			&= & \displaystyle\int\limits _0^{\tfrac{\pi}{3}}\dfrac{2\sin^2 x}{2\cos^2 x} \mathrm{\,d}x\\
			&=& \displaystyle\int\limits _0^{\tfrac{\pi}{3}}\left(\dfrac{1}{\cos^2 x}-1\right)\mathrm{\,d}x\\
			&=&  \left(\tan x-x\right)\Bigg|_0^{\tfrac{\pi}{3}}=\sqrt{3}-\dfrac{\pi}{3}.
		\end{eqnarray*}
		Vậy  $\heva{&a=1\\
			&b=-1}\Rightarrow a+b=0.$
	}
\end{ex}
%%%==============EX_11============%%%
\begin{ex}%[2D4H2-4]
	Tính $I=\displaystyle\int\limits _0^1\dfrac{\left(2024^x+1\right)^2}{e^{-3x}} \mathrm{\,d}x$ (\textit{làm tròn đến hàng phần trăm}).
	\shortans{$0$}	
	\loigiai{
		\begin{eqnarray*} 
			I&=&\int_0^1 \frac{\left(2024^x+1\right)^2}{e^{-3 x}} d x\\
			&=&\int_0^1 \frac{2024^{2 x}+2 \cdot 2024^x+1}{e^{-3 x}} d x\\
			&=&\left[\left(\frac{2024^2}{e^{-3}}\right)^x+2 \cdot\left(\frac{2024}{e^{-3}}\right)^x+e^{3 x}\right]\Bigg|_0 ^1 \\ 
			& =&\dfrac{\left(\dfrac{2024^2}{e^{-3}}\right)^x}{\ln \dfrac{2024^2}{e^{-3}}}+\dfrac{2 \cdot\left(\dfrac{2024}{e^{-3}}\right)^x}{\ln \dfrac{2024}{e^{-3}}}+\dfrac{1}{3} e^{3 x}\\
			&=&\dfrac{2024^{2 x} e^{3 x}}{2 \ln 2024-3}+\dfrac{2.2024^{2 x} e^{3 x}}{\ln 2024-3}+\dfrac{1}{3} e^{3 x} \\ 
			& =&\left(\dfrac{2024^{2 x}}{2 \ln 2024-3}+\dfrac{2\cdot2024^{2 x}}{\ln 2024-3}+\dfrac{1}{3}\right) e^{3 x}. 
		\end{eqnarray*}
	}
\end{ex}
%%%==============EX_12============%%%
\begin{ex}%[2D4H2-4]
	Tính $I=\dfrac{1}{1000}\displaystyle\int\limits _0^1\dfrac{\left(e^{-x}+2\right)^2}{e^{x-1}} \mathrm{\,d}x$ (\textit{làm tròn đến hàng đơn vị}).
	\shortans{$4522$}	
	\loigiai{
		\begin{eqnarray*}
			I&= &\dfrac{1}{1000}\displaystyle\int\limits _0^1\dfrac{\left(e^{-x}+2\right)^2}{e^{x-1}} \mathrm{\,d}x\\
			&=& \dfrac{1}{1000}\displaystyle\int\limits _0^1\dfrac{e^{-2x}+4e^{-x}+4}{e^{x-1}} \mathrm{\,d}x\\
			&= &\dfrac{1}{1000} \displaystyle\int\limits _0^1\left(e^{-3x+1}+4e^{-2x+1}+4e^{-x+1} \right)\mathrm{\,d}x\\
			&=& \dfrac{1}{1000}\left(e^{-3x+1}+4e^{-2x+1}+4e^{-x+1} \right)\Bigg|_0^1\\
			&=&\dfrac{1}{1000} \dfrac{-9e^3+4e^2+4e+1}{e^2}\approx 4522.
		\end{eqnarray*}
	}
\end{ex}
%%%==============EX_13============%%%
\begin{ex}%[2D4H2-4]
	Tính $I=\dfrac{1}{100}\displaystyle\int\limits_1^2e^{2x} \left(2023+\dfrac{2024e^{-2x}}{x^3} \right) \mathrm{\,d}x$ (\textit{làm tròn đến hàng phần chục}).
	\shortans{$48{,}5$}	
	\loigiai{
		\begin{eqnarray*}
			I&= &\dfrac{1}{100}\displaystyle\int\limits_1^2e^{2x} \left(2023+\dfrac{2024e^{-2x}}{x^3} \right) \mathrm{\,d}x\\
			&=&\dfrac{1}{100}\displaystyle\int\limits_1^2\left(2023e^{2x} +\dfrac{2024}{x^3} \right) \mathrm{\,d}x\\
			&=&\dfrac{1}{100}\left(2023\dfrac{e^{2x}}{2}-\dfrac{1012}{x}\right)\Bigg|_1^2\\
			&\approx& 48{,}5.
		\end{eqnarray*}
	}
\end{ex}
%%%==============EX_14============%%%
\begin{ex}%[2D4H2-4]
	Tính $I=\displaystyle\int\limits_1^2\left(4x^3-2\cdot3^{x+1}+\dfrac{1}{x^2} \right) \mathrm{\,d}x$ (\textit{làm tròn đến hàng phần chục}).
	\shortans{$-17{,}3$}	
	\loigiai{
		\begin{eqnarray*}
			I&= &\displaystyle\int\limits_1^2\left(4x^3-2\cdot3^{x+1}+\dfrac{1}{x^2} \right) \mathrm{\,d}x\\
			&=&\left(x^4-\dfrac{2\cdot3^{x+1}}{\ln 3}-\dfrac{1}{x}\right)\Bigg|_1^2\\
			&\approx&-17{,}3.
		\end{eqnarray*}
	}
\end{ex}
\Closesolutionfile{ans}
% \indapan{3}{ans/ans-C4B2CD1-KQ}

\begin{dang}{Tích phân hàm chứa trị tuyệt đối}
	Tính tích phân $I=\displaystyle\int\limits_a^b|f(x)| \mathrm{\,d}x$?\\
	\textbf{Phương pháp}
	\begin{itemize}
		\item \textbf{Bước 1.} Xét dấu $f(x)$ trên đoạn $[a ; b]$.
		\item \textbf{Bước 2.} Dựa vào bảng xét dấu trên đoạn $[a ; b]$ để khử $|f(x)|$. Sau đó sử dụng các phương pháp tính tích phân đã học để tính $I=\displaystyle\int\limits_a^b|f(x)| \cdot \mathrm{\,d}x$.
	\end{itemize}
\end{dang}

\Opensolutionfile{ans}[ans/ans-C4B2CD1-Dang2]
\TN
%%%==============EX_1============%%%
\begin{ex}%[2D4V2-3]
	Giá trị của $I=\displaystyle\int\limits _0^{2\pi}\sqrt{1-\cos 2x} \mathrm{\,d}x$ bằng
	\choice
	{$\sqrt{3}$}
	{\True $4\sqrt{2}$}
	{$2\sqrt{3}$}
	{$\dfrac{\pi}{2}$}
	\loigiai{
		Ta có	$I=\displaystyle\int\limits_0^{2\pi}\sqrt{1-\cos 2x} \mathrm{\,d}x=\displaystyle\int\limits _0^{2\pi}\sqrt{2\sin^2 x} \mathrm{\,d}x=\sqrt{2} \displaystyle\int\limits_0^{2\pi}\left|\sin x\right|\mathrm{\,d}x.$
		\\
		Vì $x\in \left[0;\pi \right]\to \sin x > 0\Rightarrow \left|\sin x\right|=\sin x$;\\
		$x\in \left[\pi;2\pi \right]\to \sin x < 0\Rightarrow \left|\sin x\right|=-\sin x$.
		\\		
		Vậy $I=\sqrt{2} \left(\displaystyle\int\limits_0^{\pi}\sin x \mathrm{\,d}x+\displaystyle\int\limits_{\pi}^{2\pi}-\sin x\mathrm{\,d}x \right)=\sqrt{2} \left(-\cos x\Bigg|_0^\pi+\cos x\Bigg|_\pi^{2\pi}\right) =4\sqrt{2}$.
	}
\end{ex}
%%%==============EX_2============%%%
\begin{ex}%[2D4H2-2]
	Tính tích phân $I=\displaystyle\int\limits _0^2\left|x-2\right|\mathrm{\,d}x$.
	\choice
	{$I=-2$}
	{$I=4$}
	{\True $I=2$}
	{$I=0$}
	\loigiai{
		Ta có $I=\displaystyle\int\limits _0^2\left|x-2\right|\mathrm{\,d}x.$\\
		Do $x\in \left[0;2\right]\Rightarrow x-2< 0\Leftrightarrow \left|x-2\right|=2-x$.\\
		Vậy $I=\displaystyle\int\limits _0^2\left(2-x\right)\mathrm{\,d}x=\left(2x-\dfrac{1}{2} x^2 \right)\Bigg|_0^2=4-2=2$.
	}
\end{ex}


%%%==============EX_3============%%%
\begin{ex}%[2D4H2-2]
	Tính tích phân $I=\displaystyle\int\limits _0^2\left|x^3-x\right|\mathrm{\,d}x$.
	\choice
	{$I=-\dfrac{1}{2}$}
	{$I=5$}
	{$I=\dfrac{1}{2}$}
	{\True $I=\dfrac{5}{2}$}
	\loigiai{
		Ta có $I=\displaystyle\int\limits _0^2\left|x^3-x\right|\mathrm{\,d}x.$\\
		Ta có $f(x)=x^3-x=x\left(x^2-1\right)=0\leftrightarrow \hoac{&x=0\\&x=-1\\&x=1.}$\\
		\[\Rightarrow f(x) > 0\forall x\in \left[1;2\right];\quad f(x) < 0\forall x\in \left[0;1\right].
		\]
		Vậy $I=\displaystyle\int\limits _0^1\left(x-x^3 \right)\mathrm{\,d}x+\displaystyle\int\limits_1^2\left(x^3-x\right)\mathrm{\,d}x=\left(\dfrac{1}{2} x^2-\dfrac{1}{4} x^4 \right)\Bigg|_0^1+\left(\dfrac{1}{4} x^4-\dfrac{1}{2}^2 \right)\Bigg|_1^2=\dfrac{5}{2}$.
	}
\end{ex}

%%%==============EX_4============%%%
\begin{ex}%[2D4H2-2]
	Tính tích phân $I=\displaystyle\int\limits _0^2\left|x^2+2x-3\right|\mathrm{\,d}x$.
	\choice
	{$I=-2$}
	{$I=4$}
	{$I=5$}
	{\True $I=-4$}
	\loigiai{
		Ta có		$I=\displaystyle\int\limits _0^2\left|x^2+2x-3\right|\mathrm{\,d}x.$\\
		Ta có $f(x)=x^2+2x-3=0\Rightarrow\hoac{&x=1\\&x=-3}\Rightarrow f(x) > 0$, $\forall x\in \left[1;2\right]$; $f(x) < 0$, $\forall x\in \left[0;1\right]$.
		\begin{eqnarray*} 
			I&=&\displaystyle\int\limits _0^1-f(x)\mathrm{\,d}x+\displaystyle\int\limits_1^2f(x)\mathrm{\,d}x\\
			&=&\displaystyle\int\limits _0^1\left(3-2x-x^2 \right)\mathrm{\,d}x+\displaystyle\int\limits_1^2\left(x^2+2x-3\right)\mathrm{\,d}x\\
			&=&	\left(3x-x^2-\dfrac{1}{3} x^3 \right)\Bigg|_0^1+\left(\dfrac{1}{3} x^3+x^2-3x\right)\Bigg|_1^2\\
			&=&\left(3-1-\dfrac{1}{3} \right)+\left[\left(\dfrac{8}{3}+4-6\right)-\left(\dfrac{1}{3}+1-3\right)\right]=4.
		\end{eqnarray*} 	
	}
\end{ex}
%%%==============EX_5============%%%

\begin{ex}%[2D4V2-2]
	Cho tích phân $I=\left(\sqrt{3}+\sqrt{2} \right)\displaystyle\int\limits _{-3}^3\left|x^2-1\right|\mathrm{\,d}x=a\sqrt{3}+b\sqrt{2}$ với $a,b\in \mathbb{Q}$. Tính $P=a+b$.
	\choice
	{$P=\dfrac{44}{3}$}
	{\True $P=\dfrac{88}{3}$}
	{$P=\dfrac{17}{3}$}
	{$P=\dfrac{98}{3}$}
	\loigiai{
		Ta có		$I=\left(\sqrt{3}+\sqrt{2} \right)\displaystyle\int\limits _{-3}^3\left|x^2-1\right|\mathrm{\,d}x$.\\
		Tính $J=\displaystyle\int\limits _{-3}^3\left|x^2-1\right|\mathrm{\,d}x$.\\
		Ta có $f(x)=x^2-1=0\Rightarrow \hoac{&x=1\\&x=-1.}$\\
		$\Rightarrow f(x) > 0$, $\forall x\in \left[-3;-1\right]\cup \left[1;3\right]$; và $f(x) < 0$, $\forall x\in \left[-1;1\right]$.\\
		Vậy
		\begin{eqnarray*} 
			I&=&\displaystyle\int\limits _{-3}^{-1}\left(x^2-1\right)\mathrm{\,d}x+\displaystyle\int\limits _{-1}^1\left(1-x^2 \right)\mathrm{\,d}x+\displaystyle\int\limits_1^3\left(x^2-1\right)\mathrm{\,d}x\\
			&=&\left(\dfrac{1}{3} x^3-x\right)\Bigg|_{-3}^{-1}+\left(x-\dfrac{1}{3} x^3 \right)\Bigg|_{-1}^{1}+\left(\dfrac{1}{3} x^3-x\right)\Bigg|_{1}^3\\
			&=&\dfrac{20}{3}+\dfrac{4}{3}+\dfrac{20}{3}=\dfrac{44}{3}.
		\end{eqnarray*}
		\[\Rightarrow I=\left(\sqrt{3}+\sqrt{2} \right)\displaystyle\int\limits _{-3}^3\left|x^2-1\right|\mathrm{\,d}x=\dfrac{44}{3} \sqrt{3}+\dfrac{44}{3} \sqrt{2}.
		\]
		Khi đó $a=\dfrac{44}{3}$, $b=\dfrac{44}{3}$. Suy ra $P=a+b=\dfrac{88}{3}$.
	}
\end{ex}
%%%==============EX_6============%%%
\begin{ex}%[2D4V2-2]
	Tính tích phân $I=\displaystyle\int\limits _{-2}^5\left(\left|x+2\right|-\left|x-2\right|\right)\mathrm{\,d}x$.
	\choice
	{$I=18$}
	{\True $I=12$}
	{$I=28$}
	{$I=30$}
	\loigiai{
		Ta có $I=\displaystyle\int\limits _{-2}^5\left(\left|x+2\right|-\left|x-2\right|\right)\mathrm{\,d}x.$\\
		Gọi $f(x)=\left|x+2\right|-\left|x-2\right|$ trên $x\in [-2;5]$. Khi đó
		\begin{itemize}
			\item Với $ x\in \left[-2;2\right]$ thì $f(x)=2x$.
			\item Với $ x\in \left[2;5\right]$ thì $f(x)=4$.
		\end{itemize} 
		Vậy $\displaystyle\int\limits _{-2}^5f(x)\mathrm{\,d}x=\displaystyle\int\limits _{-2}^2 2x\mathrm{\,d}x+\displaystyle\int\limits_2^5 4\mathrm{\,d}x=x^2\Bigg|_{-2}^2+4x\Bigg|_2^5=0+12=12$.
	}
\end{ex}
%%%==============EX_7============%%%
\begin{ex}%[2D4V2-4]
	Cho tích phân $I=\displaystyle\int\limits _0^3\left|2^x-4\right|\mathrm{\,d}x=a+\dfrac{b}{c\ln 2}$ với $a,b,c\in \mathbb{Z}$ và $\dfrac{b}{c}$ là phân số tối giản. Tính $P=a^2+b^2+c^2$.
	\choice
	{$P=15$}
	{$P=10$}
	{$P=5$}
	{\True $P=18$}
	\loigiai{
		Ta có $I=\displaystyle\int\limits _0^3\left|2^x-4\right|\mathrm{\,d}x$.
		Ta có $2^x-4> 0\Leftrightarrow x > 2\Rightarrow f(x) > 0,~\forall x\in \left[2;3\right]$; và $f(x) < 0,~\forall x\in \left[0;2\right]$.\\
		Vậy
		\begin{eqnarray*} 
			I&=&\displaystyle\int\limits _0^2\left(4-2^x \right)\mathrm{\,d}x+\displaystyle\int\limits_2^3\left(2^x-4\right)\mathrm{\,d}x\\
			&=&\left(4x-\dfrac{1}{\ln 2} 2^x \right)\Bigg|_0^2+\left(\dfrac{1}{\ln 2} 2^x-4x\right)\Bigg|_2^3\\
			&=&\left(8-\dfrac{3}{\ln 2} \right)+\left(\dfrac{4}{\ln 2}-4\right)=4+\dfrac{1}{\ln 2}.
		\end{eqnarray*} 
		\[\Rightarrow P=a^2+b^2+c^2=4^2+1^2+1^2=18.
		\]
	}
\end{ex}
%%%==============EX_8============%%%
\begin{ex}%[2D4V2-4]
	Tính tích phân $I=\displaystyle\int\limits _{-1}^1\left|2^x-2^{-x} \right|\mathrm{\,d}x$.
	\choice
	{\True $\dfrac{1}{\ln 2}$}
	{$\ln 2$}
	{$2\ln 2$}
	{$\dfrac{2}{\ln 2}$}
	\loigiai{
		$I=\displaystyle\int\limits _{-1}^1\left|2^x-2^{-x} \right|\mathrm{\,d}x$.\\
		Ta có $2^x-2^{-x}=0$ $\Rightarrow x=0$.
		\begin{eqnarray*} 
			I&=&\displaystyle\int\limits _{-1}^1\left|2^x-2^{-x} \right|\mathrm{\,d}x\\
			&=&\displaystyle\int\limits _{-1}^0\left|2^x-2^{-x} \right|\mathrm{\,d}x+\displaystyle\int\limits _0^1\left|2^x-2^{-x} \right|\mathrm{\,d}x\\
			&=&\left|\displaystyle\int\limits _{-1}^0\left(2^x-2^{-x} \right) \mathrm{\,d}x\right|+\left|\displaystyle\int\limits _0^1\left(2^x-2^{-x} \right) \mathrm{\,d}x\right|\\
			&=&\left|\left(\dfrac{2^x+2^{-x}}{\ln 2} \right)\Bigg|_{-1}^0 \right|+\left| \left(\dfrac{2^x+2^{-x}}{\ln 2} \right)\Bigg|_0^1 \right|=\dfrac{1}{\ln 2}.
		\end{eqnarray*} 		
	}
\end{ex}
%%%==============EX_9============%%%

\begin{ex}%[2D4V2-2]
	Tính tích phân $I=\displaystyle\int\limits _{-1}^2\left(\left|x\right|-\left|x-1\right|\right)\mathrm{\,d}x$.
	\choice
	{\True $I=0$}
	{$I=2$}
	{$I=-2$}
	{$I=-3$}
	\loigiai{
		Ta có $I=\displaystyle\int\limits _{-1}^2\left(\left|x\right|-\left|x-1\right|\right)\mathrm{\,d}x$.
		\begin{eqnarray*} 
			I&=&\displaystyle\int\limits _{-1}^2\left(\left|x\right|-\left|x-1\right|\right)\mathrm{\,d}x\\
			&=&\displaystyle\int\limits _{-1}^2\left|x\right|\mathrm{\,d}x-\displaystyle\int\limits _{-1}^2\left|x-1\right|\mathrm{\,d}x\\
			&=&-\displaystyle\int\limits _{-1}^0x\mathrm{\,d}x+\displaystyle\int\limits _0^2x\mathrm{\,d}x+\displaystyle\int\limits _{-1}^1(x-1)\mathrm{\,d}x-\displaystyle\int\limits_1^2(x-1)\mathrm{\,d}x\\
			&=&-\dfrac{x^2}{2}\Bigg|_{-1}^0+ \dfrac{x^2}{2}\Bigg|_0^2+ \left(\dfrac{x^2}{2}-x\right)\Bigg|_{-1}^1- \left(\dfrac{x^2}{2}-x\right)\Bigg|_1^2=0.
		\end{eqnarray*} 		
	}
\end{ex}


%%%==============EX_10============%%%
\begin{ex}%[2D4V2-2]
	Cho $a$ là số thực dương, tính tích phân $I=\displaystyle\int\limits _{-1}^a\left|x\right|\mathrm{d}x$ theo $a$.
	\choice
	{\True $I=\dfrac{a^2+1}{2}$}
	{$I=\dfrac{a^2+2}{2}$}
	{$I=\dfrac{-2a^2+1}{2}$}
	{$I=\dfrac{\left|3a^2-1\right|}{2}$}
	\loigiai{
		Vì $a > 0$ nên $I=-\displaystyle\int\limits_{-1}^0x \mathrm{\,d}x+\displaystyle\int\limits_0^ax \mathrm{\,d}x=\dfrac{1}{2}+\dfrac{a^2}{2}=\dfrac{1+a^2}{2}$.
	}
\end{ex}
%%%==============EX_11============%%%
\begin{ex}%[2D4V2-2]
	Cho số thực $m > 1$ thỏa mãn $\displaystyle\int\limits_1^m\left|2mx-1\right|\mathrm{\,d}x=1$. Khẳng định nào sau đây đúng?
	\choice
	{$m\in \left(4;6\right)$}
	{$m\in \left(2;4\right)$}
	{$m\in \left(3;5\right)$}
	{\True $m\in \left(1;3\right)$}
	\loigiai{
		Do $m > 1\Rightarrow 2m > 2\Rightarrow \dfrac{1}{2m} < 1$. Do đó với $m > 1, x\in \left[1;m\right]\Rightarrow 2mx-1> 0$.\\		
		Vậy
		\begin{eqnarray*} 
			\displaystyle\int\limits_1^m\left|2mx-1\right|\mathrm{\,d}x&=&\displaystyle\int\limits_1^m\left(2mx-1\right)\mathrm{\,d}x\\
			&=&\left(mx^2-x\right)\Bigg|_1^m\\
			&=&m^3-m-m+1=m^3-2m+1.
		\end{eqnarray*}
		Từ đó theo bài ra ta có $m^3-2m+1=1\Leftrightarrow \hoac{&m=0 \\&m=\pm \sqrt{2}.} $\\ Do $m > 1$ vậy $m=\sqrt{2}$.
	}
\end{ex}
%%%==============EX_12============%%%
\begin{ex}%[2D4V2-2]
	Khẳng định nào sau đây là đúng?
	\choice
	{$\displaystyle\int\limits _{-1}^1\left|x\right|^3 \mathrm{d}x=\left|\displaystyle\int\limits _{-1}^1x^3 \mathrm{d}x \right|$}
	{\True $\displaystyle\int\limits _{-1}^{2024}\left|x^4-x^2+1\right|\mathrm{d}x=\displaystyle\int\limits _{-1}^{2024}\left(x^4-x^2+1\right)\mathrm{d}x$}
	{$\displaystyle\int\limits _{-2}^3\left|e^x \left(x+1\right)\mathrm{d}x\right|=\displaystyle\int\limits _{-2}^3e^x \left(x+1\right)\mathrm{d}x$}
	{$\displaystyle\int\limits _{-\tfrac{\pi}{2}}^{\tfrac{\pi}{2}}\sqrt{1-\cos^2 x} \mathrm{d}x=\displaystyle\int\limits _{-\tfrac{\pi}{2}}^{\tfrac{\pi}{2}}\sin x\mathrm{d}x$}
	\loigiai{
		Ta có: $x^4-x^2+1=x^4-2\cdot x^2\cdot\dfrac{1}{2}+\dfrac{1}{4}+\dfrac{3}{4}$ $=\left(x^2-\dfrac{1}{2} \right)^2+\dfrac{3}{4} > 0,\forall x\in {\bf \mathbb{R}}$.\\
		Do đó $\displaystyle\int\limits _{-1}^{2024}\left|x^4-x^2+1\right|\mathrm{d}x=\displaystyle\int\limits _{-1}^{2024}\left(x^4-x^2+1\right)\mathrm{d}x$.
	}
\end{ex}
%%%==============EX_13============%%%
\begin{ex}%[2D4V2-2]
	Tính tích phân $I=\displaystyle\int\limits_1^4\sqrt{x^2-6x+9} \mathrm{\,d}x$.
	\choice
	{\True $I=\dfrac{5}{2}$}
	{$I=-\dfrac{1}{2}$}
	{$I=-2$}
	{$I=\dfrac{1}{2}$}
	\loigiai{
		Ta có $I=\displaystyle\int\limits_1^4\sqrt{x^2-6x+9} \mathrm{\,d}x=\displaystyle\int\limits_1^4\left|x-3\right|\mathrm{\,d}x$.\\
		Ta có $x-3> 0,~\forall x\in \left[3;4\right];~x-3< 0,~\forall x\in \left[1;3\right]$.\\
		Vậy
		\begin{eqnarray*} 
			I&=&\displaystyle\int\limits_1^3\left(3-x\right)\mathrm{\,d}x+\displaystyle\int\limits_3^4\left(x-3\right)\mathrm{\,d}x\\
			&=&\left(3x-\dfrac{1}{2} x^2 \right)\Bigg|_1^3+\left(\dfrac{1}{2} x^2-3x\right)\Bigg|_3^4\\
			&=&2+\dfrac{1}{2}=\dfrac{5}{2}.
		\end{eqnarray*}
	}
\end{ex}
\Closesolutionfile{ans}
% \indapan{6}{ans/ans-C4B2CD1-Dang2}



\Opensolutionfile{ans}[ans/ans-C4B2CD1-Dang2-KQ]
\TNSA

\begin{ex}%[2D4V2-2]
	Tính tích phân $I=\displaystyle\int\limits _{-3}^{3}\left|x^{2} -1\right|\mathrm{\,d}x $ (tính gần đúng đến hàng phần chục).
	\shortans{$13{,}3$}	
	\loigiai{
		\[I=\displaystyle\int\limits _{-3}^{3}\left|x^{2} -1\right|\mathrm{\,d}x.\] 
		Vì  $f(x)=x^{2} -1=0\to\hoac{&x=-1\\&x=1} \Rightarrow f(x)>0,~\forall x\in \left[-3;-1\right]\cup \left[1;3\right]$; $f(x)<0,~\forall x\in \left[-1;1\right]$.\\
		Vậy  
		\begin{eqnarray*} 
			I&=&\displaystyle\int\limits _{-3}^{-1}\left(x^{2} -1\right)\mathrm{\,d}x+\displaystyle\int\limits _{-1}^{1}\left(1-x^{2} \right)\mathrm{\,d}x+\displaystyle\int\limits _{1}^{3}\left(x^{2} -1\right)\mathrm{\,d}x\\
			&=&\left(\frac{1}{3} x^{3} -x\right)\Bigg|_{-3}^{-1}+\left(x-\frac{1}{3} x^{3} \right)\Bigg|_{-1}^1+\left(\frac{1}{3} x^{3} -x\right)\Bigg|_1^3\\
			&=&\frac{20}{3} +\frac{4}{3} +\frac{16}{3} =\frac{40}{3}\approx 13{,}3.
		\end{eqnarray*}
	}
\end{ex}

\begin{ex}%[2D4V2-2]
	Tính tích phân $I=\displaystyle\int\limits _{-1}^{2}\left|-x^{2} -2x+3\right|\mathrm{\,d}x $ (tính gần đúng đến hàng phần trăm).
	\shortans{$7{,}67$}	
	\loigiai{
		Vì  $f(x)=-x^{2} -2x+3=0\Rightarrow\hoac{&x=1\\&x=-3} \Rightarrow f(x)>0,~\forall x\in \left[-1;-1\right]$; $f(x)<0,~\forall x\in \left[1;2\right]$\\
		Vậy  
		\begin{eqnarray*} 
			I&=&\displaystyle\int\limits _{-1}^{1}\left(-x^{2} -2x+3\right)\mathrm{\,d}x+\displaystyle\int\limits _{1}^{2}\left(x^{2}+2x-3 \right)\mathrm{\,d}x\\
			&=&\left(-\dfrac{1}{3} x^{3}-x^2 +3x\right)\Bigg|_{-1}^{1}+\left(\dfrac{1}{3} x^{3}+x^2 -3x \right)\Bigg|_{1}^2\\
			&=&-\dfrac{1}{3}-1+3-\dfrac{1}{3}+1+3 +\dfrac{8}{3}+4-6-\dfrac{1}{3}-1+3 \approx7{,}67.
		\end{eqnarray*}		
	}
\end{ex}

\begin{ex}%[2D4V2-2]
	Tính tích phân $I=\displaystyle\int\limits _{1}^{2}\left|\frac{x+1}{x} \right|\mathrm{\,d}x $ (tính gần đúng đến hàng phần trăm).
	\shortans{$1{,}69$}	
	\loigiai{
		Vì $\frac{x+1}{x}>0$, $\forall x\in [1;2]$ nên
		\[I=\displaystyle\int\limits _{1}^{2}\left(\dfrac{x+1}{x}\right)\mathrm{\,d}x=\displaystyle\int\limits _{1}^{2}\left(1+\dfrac{1}{x}\right)\mathrm{\,d}x=\left(x+\ln x \right)\Bigg|1^2=2+\ln 2-1=1+\ln 2\approx1{,}69.  \]		
	}
\end{ex}

\begin{ex}%[2D4V2-2]
	Tính tích phân $I=\displaystyle\int\limits _{2}^{6}\sqrt{x^{2} -8x+16} \mathrm{\,d}x $.
	\shortans{$4$}	
	\loigiai{
		Ta có $I=\displaystyle\int\limits _{2}^{6}\left| x-4\right|  \mathrm{\,d}x $.\\
		Ta có $x-4\le 0$, $\forall x\in [2;4]$	và 	 $x-4\ge 0$, $\forall x\in [4;6]$. Khi đó
		\[I=\displaystyle\int\limits _{2}^{4}\left( 4- x\right)   \mathrm{\,d}x+\displaystyle\int\limits _{4}^{6}\left( x- 4\right)   \mathrm{\,d}x=\left( 4 x-\dfrac{x^2}{2}\right)\Bigg|_{2}^{4}+\left( -4 x+\dfrac{x^2}{2}\right)\Bigg|_{4}^{6}=4.\]
	}
\end{ex}

\begin{ex}%[2D4V2-2]
	Tính tích phân $I=\displaystyle\int\limits _{-2}^{1}\sqrt{4x^{2} +6x+9} \mathrm{\,d}x $ (\textit{làm tròn đến hàng phần trăm}).
	\shortans{$9{,}38$}	
	\loigiai{
		Ta có $I=\displaystyle\int\limits _{-2}^{1}\sqrt{4x^{2} +6x+9} \mathrm{\,d}x=\displaystyle\int\limits _{-2}^{1}\left|2x+3 \right|  \mathrm{\,d}x$.\\
		Ta có $2x+3\le 0$, $\forall x\in \left[-2;-\dfrac{3}{2} \right] $	và 	 $2x+3\ge 0$, $\forall x\in \left[-\dfrac{3}{2};1 \right]$. Khi đó
		\begin{eqnarray*} 
			I&=&\displaystyle\int\limits _{-2}^{-\tfrac{3}{2}}\left( -2x-3 \right)   \mathrm{\,d}x+\displaystyle\int\limits _{-\tfrac{3}{2}}^{1}\left( 2x+3 \right)  \mathrm{\,d}x\\
			&=&\left(-x^2-3x\right)\Bigg|_{-2}^{-\tfrac{3}{2}}+\left(x^2+3x \right)\Bigg|_{-\tfrac{3}{2}}^{1}\\
			&\approx&9{,}38.
		\end{eqnarray*}				
	}
\end{ex}

\begin{ex}%[2D4V2-2]
	Tính tích phân $I=\displaystyle\int\limits _{0}^{1}\sqrt{9x^{2} -6x+1} \mathrm{\,d}x $ (\textit{làm tròn đến hàng phần trăm}).
	\shortans{$0{,}83$}	
	\loigiai{
		Ta có $I=\displaystyle\int\limits _{0}^{1}\sqrt{9x^{2} -6x+1} \mathrm{\,d}x =\displaystyle\int\limits _{0}^{1}\left| 3x-1\right| \mathrm{\,d}x $.\\
		Ta có $3x+1\le 0$, $\forall x\in \left[1;\dfrac{1}{3} \right] $	và 	 $3x+1\ge 0$, $\forall x\in \left[\dfrac{1}{3};1 \right]$. Khi đó
		\begin{eqnarray*} 
			I&=&\displaystyle\int\limits _{0}^{\tfrac{1}{3}}\left(-3x-1 \right)   \mathrm{\,d}x+\displaystyle\int\limits _{\tfrac{1}{3}}^{1}\left( 3x+1 \right)  \mathrm{\,d}x\\
			&=&\left(-\dfrac{3x^2}{2}-x\right)\Bigg| _{0}^{\tfrac{1}{3}}+\left(\dfrac{3x^2}{2}+x\right)\Bigg|_{\tfrac{1}{3}}^{1}\\
			&\approx&0{,}83.
		\end{eqnarray*}		
	}
\end{ex}

\begin{ex}%[2D4V2-3]
	Tính tích phân $I=\displaystyle\int\limits _{0}^{2\pi }\sqrt{1+\cos 2x} \mathrm{\,d}x $ (\textit{làm tròn đến hàng phần trăm}).
	\shortans{$5{,}66$}	
	\loigiai{
		Ta có  $I=\displaystyle\int\limits _{0}^{2\pi }\sqrt{1+\cos 2x} \mathrm{\,d}x =\sqrt{2}\displaystyle\int\limits _{0}^{2\pi }|\cos x|\mathrm{\,d}x $.\\
		Ta có $\cos x\ge 0, \forall x\in \left[0;\dfrac{\pi}{2} \right]\cup\left[\dfrac{3\pi}{2};2\pi \right] $ và $\cos x\le 0, \forall x\in \left[\dfrac{\pi}{2};\dfrac{3\pi}{2} \right] $. Khi đó
		\begin{eqnarray*} 
			I&=&\sqrt{2}\displaystyle\int\limits _{0}^{\tfrac{\pi}{2} }\cos x\mathrm{\,d}x-\sqrt{2}\displaystyle\int\limits _{\tfrac{\pi}{2} }^{\tfrac{3\pi}{2} }\cos x\mathrm{\,d}x+\sqrt{2}\displaystyle\int\limits _{\tfrac{3\pi}{2} }^{2\pi}\cos x\mathrm{\,d}x\\
			&=&\sqrt{2}\sin x\Bigg|_{0}^{\tfrac{\pi}{2} } -\sqrt{2}\sin x\Bigg|_{\tfrac{\pi}{2} }^{\tfrac{3\pi}{2} }+\sqrt{2}\sin x\Bigg|_{\tfrac{3\pi}{2} }^{2\pi}\\
			&=&4\sqrt{2}\approx5{,}66.
	\end{eqnarray*}		}
\end{ex}

\begin{ex}%[2D4V2-3]
	Tính tích phân $I=\displaystyle\int\limits_{0}^{2\pi }\sqrt{1-\cos 2x} \mathrm{\,d}x $ (\textit{làm tròn đến hàng phần trăm}).
	\shortans{$5{,}66$}	
	\loigiai{
		Ta có  $I=\displaystyle\int\limits _{0}^{2\pi }\sqrt{1-\cos 2x} \mathrm{\,d}x =2\displaystyle\int\limits _{0}^{2\pi }|\sin  x|\mathrm{\,d}x $.\\
		Ta có $\sin x\ge 0, \forall x\in \left[0;\pi \right]$ và $\sin x\le 0, \forall x\in \left[\pi;2\pi \right]$. Khi đó
		\begin{eqnarray*} 
			I&=&\sqrt{2}\displaystyle\int\limits _{0}^{\pi}\sin x\mathrm{\,d}x-\sqrt{2}\displaystyle\int\limits _{\pi}^{2\pi}\sin x\mathrm{\,d}x\\
			&=&-\sqrt{2}\cos x\Bigg|_{0}^{\pi } +\sqrt{2}\cos x\Bigg|_{\pi }^{2\pi}\\
			&=&4\sqrt{2}\approx5{,}66.
		\end{eqnarray*}			
	}
\end{ex}


\begin{ex}%[2D4V2-3]
	Tính tích phân $I=\displaystyle\int\limits _{0}^{2\pi }\sqrt{1-\sin 2x} \mathrm{\,d}x $, (\textit{làm tròn đến hàng phần trăm}).
	\shortans{$0{,}31$}	
	\loigiai{
		Ta có  $I=\displaystyle\int\limits _{0}^{2\pi }\sqrt{1-\sin 2x} \mathrm{\,d}x =\displaystyle\int\limits _{0}^{2\pi }|\sin  x-\cos x|\mathrm{\,d}x $.\\
		Ta có $\sin x-\cos x\le 0, \forall x\in \left[0;\dfrac{\pi}{4} \right]\cup\left[\dfrac{5\pi}{4};2\pi \right] $ và $\sin x-\cos x\ge 0, \forall x\in \left[\dfrac{\pi}{4};\dfrac{5\pi}{4} \right]$. Khi đó
		\begin{eqnarray*} 
			I&=&\displaystyle\int\limits _{0}^{\tfrac{\pi}{4} }\left( \cos x-\sin x\right) \mathrm{\,d}x+\displaystyle\int\limits _{\tfrac{\pi}{4} }^{\tfrac{5\pi}{4} }\left( \sin x-\cos x\right) \mathrm{\,d}x+\displaystyle\int\limits _{\tfrac{5\pi}{4} }^{2\pi}\left( \cos x-\sin x\right) \mathrm{\,d}x\\
			&=&\left( \sin x+\cos x\right) \Bigg|_{0}^{\tfrac{\pi}{4} }+\left( -\cos x-\sin x\right) \Bigg|_{\tfrac{\pi}{4} }^{\tfrac{5\pi}{4} }+\left( \sin x+\cos x\right) \Bigg|_{\tfrac{5\pi}{4} }^{2\pi}\\
			&=&4\sqrt{2}\approx5{,}66.
		\end{eqnarray*}				
	}
\end{ex}

\begin{ex}%[2D4V2-3]
	Tính tích phân $I=\displaystyle\int\limits _{0}^{2\pi }\sqrt{1+\sin 2x} \mathrm{\,d}x $ (\textit{làm tròn đến hàng phần trăm}).
	\shortans{$5{,}66$}	
	\loigiai{
		Ta có  $I=\displaystyle\int\limits _{0}^{2\pi }\sqrt{1+\sin 2x} \mathrm{\,d}x =\displaystyle\int\limits _{0}^{2\pi }|\sin  x+\cos x|\mathrm{\,d}x $.\\
		Ta có $\sin x+\cos x\ge 0, \forall x\in \left[0;\dfrac{3\pi}{4} \right]\cup\left[\dfrac{7\pi}{4};2\pi \right] $ và $\sin x+\cos x\le 0, \forall x\in \left[\dfrac{3\pi}{4};\dfrac{7\pi}{4} \right]$. \\
		Khi đó:
		\begin{eqnarray*} 
			I&=&\displaystyle\int\limits _{0}^{\tfrac{3\pi}{4} }\left( \cos x+\sin x\right) \mathrm{\,d}x-\displaystyle\int\limits _{\tfrac{3\pi}{4} }^{\tfrac{7\pi}{4} }\left( \sin x+\cos x\right) \mathrm{\,d}x+\displaystyle\int\limits _{\tfrac{7\pi}{4} }^{2\pi}\left( \cos x+\sin x\right) \mathrm{\,d}x\\
			&=&\left( \sin x-\cos x\right) \Bigg|_{0}^{\tfrac{3\pi}{4} }-\left( \sin x-\cos x\right) \Bigg|_{\tfrac{3\pi}{4} }^{\tfrac{7\pi}{4} }+\left( \sin x-\cos x\right) \Bigg|_{\tfrac{7\pi}{4} }^{2\pi}\\
			&=&4\sqrt{2}\approx5{,}66.
		\end{eqnarray*}				
	}
\end{ex}
\Closesolutionfile{ans}
% \indapan{6}{ans/ans-C4B2CD1-Dang2-KQ}

% \begin{dang}{Tích phân có điều kiện}
\end{dang}
\TN
\Opensolutionfile{ans}[ans/ans-2C4B2CD2-LC]
\begin{ex}%[2D4H1-1]
	Nếu $F'(x) = \dfrac{1}{2x}$ và $F(1) = 1$ thì giá trị của $F(4)$ bằng
	\choice
	{$\ln 2$}
	{\True $1 + \ln 2$}
	{$1 + \dfrac{1}{2} \ln 2 $}
	{$ \dfrac{1}{2} \ln 2 $}
\loigiai{	
	Ta có
	$$
	\displaystyle\int\limits_{1}^{4} F'(x) \, \mathrm{d}x = 	\displaystyle\int\limits_{1}^{4} \dfrac{1}{2x} \, \mathrm{d}x = \dfrac{1}{2} \ln \left| x \right| \bigg|_{1}^{4} = \ln 2 .
	$$	
	Lại có 
	$$
	\displaystyle\int\limits_{1}^{4} F'(x) \, dx = F(x) \bigg|_{1}^{4} = F(4) - F(1) .
	$$	
	Suy ra 
	$	F(4) - F(1) = \ln 2	$	.
	Do đó 
	$	F(4) = F(1) + \ln 2 = 1 + \ln 2 $.
}
\end{ex}
\begin{ex}%[2D4H1-1]
	Cho $F(x)$ là một nguyên hàm của $f(x) = \dfrac{2}{x}$. Biết $F(-1) = 0$. Tính $F(2)$ kết quả là
	\choice
	{$2 \ln 2 + 1$}
	{$\ln 2$}
	{$ 2 \ln 3 + 2$}
	{\True $2 \ln 2 $}
\loigiai{
	Ta có 
\allowdisplaybreaks
\begin{eqnarray*}
&&	\displaystyle\int\limits_{-1}^{2} f(x) \, \mathrm{d}x = F(x) \bigg|_{-1}^{2} = F(2) - F(-1)\\
&&	\displaystyle\int\limits_{-1}^{2} \dfrac{2}{x} \, dx = 2 \ln \left| x \right| \bigg|_{-1}^{2} = 2 \ln 2 - 2 \ln 1 = 2 \ln 2\\
&	\Rightarrow& F(2) - F(-1) = 2 \ln 2\\
&\Leftrightarrow& F(2) = 2 \ln 2 \, \text{(do } F(-1) = 0).	
\end{eqnarray*}
}
\end{ex}
\begin{ex}%[2D4H1-1]
	Cho hàm số $f(x)$ liên tục, có đạo hàm trên $[-1;2]$, $f(-1) = 8$, $f(2) = -1$. Tích phân $\displaystyle\int\limits_{-1}^{2} f'(x) \, \mathrm{d}x$ bằng
	\choice
	{$1$}
	{$7$}
	{\True $-9$}
	{$9$}
\loigiai{
	Ta có 
	$$
	\displaystyle\int\limits_{-1}^{2} f'(x) \, \mathrm{d}x = f(x) \bigg|_{-1}^{2} = f(2) - f(-1) = -1 - 8 = -9
	.$$
}
\end{ex}
\begin{ex}%[2D4H1-1]
	Biết $F(x) = x^2$ là một nguyên hàm của hàm số $f(x)$ trên $\mathbb{R}$. Giá trị của $\displaystyle\int\limits_{1}^{3} \left[ 1 + f(x) \right] \mathrm{d}x$ bằng
	\choice
	{\True $10$}
	{$8$}
	{$ \dfrac{26}{3}$}
	{$ \dfrac{32}{3}$}


\loigiai{
	Ta có 
	$$\displaystyle\int\limits_{1}^{3} \left[ 1 + f(x) \right] \mathrm{d}x = (x + F(x)) \bigg|_{1}^{3} = (x + x^2) \bigg|_{1}^{3} = 12 - 2 = 10.$$
}
\end{ex}
\begin{ex}%[2D4H2-2]
	Biết $F(x) = x^3$ là một nguyên hàm của hàm số $f(x)$ trên $\mathbb{R}$. Giá trị của $\displaystyle\int\limits_{1}^{3} \left[ 1 + f(x) \right] \mathrm{d}x$ bằng
	\choice
	{$20$}
	{$22$}
	{$26$}
	{\True $28$}
\loigiai{	
	Ta có 
	$$
	\displaystyle\int\limits_{1}^{3} \left[ 1 + f(x) \right] \mathrm{d}x = \left[ x + F(x) \right] \bigg|_{1}^{3} = \left[ x + x^3 \right] \bigg|_{1}^{3} = 30 - 2 = 28
	.$$
}

\end{ex}
\begin{ex}%[2D4H2-2]
	Biết $F(x) = x^2$ là một nguyên hàm của hàm số $f(x)$ trên $\mathbb{R}$. Giá trị của $\displaystyle\int\limits_{1}^{2} \left[ 2 + f(x) \right] \mathrm{d}x$ bằng
	\choice
	{\True$5$}
	{$3$}
	{$\dfrac{13}{3}$}
	{$\dfrac{7}{3}$}
\loigiai{
	Ta có 
	$$	\displaystyle\int\limits_{1}^{2} \left[ 2 + f(x) \right] \mathrm{d}x = (2x + x^2) \bigg|_{1}^{2} = 8 - 3 = 5	.$$
}
\end{ex}
\begin{ex}%[2D4H2-2]
	Biết $F(x) = x^3$ là một nguyên hàm của hàm số $f(x)$ trên $\mathbb{R}$. Giá trị của $\displaystyle\int\limits_{1}^{2} \left[ 2 + f(x) \right] \mathrm{d}x$ bằng
	\choice
	{$\dfrac{23}{4}$}
	{$7$}
	{\True$9$}
	{$\dfrac{15}{4}$}
\loigiai{
	Ta có 
	$$	\displaystyle\int\limits_{1}^{2} \left[ 2 + f(x) \right] \mathrm{d}x = \displaystyle\int\limits_{1}^{2} 2 \, \mathrm{d}x + \displaystyle\int\limits_{1}^{2} f(x) \, \mathrm{d}x = 2x \bigg|_{1}^{2} + F(x) \bigg|_{1}^{2} = 2x \bigg|_{1}^{2} + x^3 \bigg|_{1}^{2} = 9	.$$
}
\end{ex}
\begin{ex}%[2D4H2-3]
	Cho hàm số $f(x)$. Biết $f(0) = 4$ và $f'(x) = 2 \sin^2 \dfrac{x}{2} + 1$, $\forall x \in \mathbb{R}$, khi đó $\displaystyle\int_{0}^{\frac{\pi}{4}} f(x) \mathrm{d}x$ bằng
	\choice
	{\True $\dfrac{\pi^2 + 16\pi + 8\sqrt{2} - 16}{16}$}
	{$\dfrac{\pi^2 + 16\pi + 2\sqrt{2} - 4}{16}$}
	{$\dfrac{\pi^2 + 16\pi + 8\sqrt{2}}{16}$}
	{$\dfrac{\pi^2 + 16\pi - 16}{16}$}
\loigiai{	
	Ta có 	$$
	f(x) = \displaystyle\int \left( 2 \sin^2 \dfrac{x}{2} + 1 \right) \mathrm{d}x = \displaystyle\int (2 - \cos x) \mathrm{d}x = 2x - \sin x + C	.$$	
	Vì $f(0) = 4 \Rightarrow C = 4 \Rightarrow f(x) = 2x - \sin x + 4$.\\	
	Suy ra 
	\allowdisplaybreaks
	\begin{eqnarray*}
&&		\displaystyle\int\limits_{0}^{\frac{\pi}{4}} f(x) \mathrm{d}x = \displaystyle\int\limits_{0}^{\frac{\pi}{4}} (2x - \sin x + 4) \mathrm{d}x\\
&	= &\left( x^2 + \cos x + 4x \right) \bigg|_{0}^{\frac{\pi}{4}} = \dfrac{\pi^2}{16} + \dfrac{\sqrt{2}}{2} + \pi - 1 = \dfrac{\pi^2 + 16\pi + 8\sqrt{2} - 16}{16}.		
	\end{eqnarray*}
}
\end{ex}
\begin{ex}%[2D4H2-3]
	Cho hàm số $f(x)$. Biết $f(0) = 4$ và $f'(x) = 2 \cos^2 \dfrac{x}{2} + 3$, $\forall x \in \mathbb{R}$, khi đó $\displaystyle\int\limits_{0}^{\frac{\pi}{4}} f(x) \mathrm{d}x$ bằng?
	\choice
	{$\dfrac{\pi^2 + 8\pi - 8 - \sqrt{2}}{8}$}
	{\True$\dfrac{\pi^2 + 8\pi - 8 - 4\sqrt{2}}{8}$}
	{$\dfrac{\pi^2 + 6\pi + 8}{8}$}
	{$\dfrac{\pi^2 + 8\pi - 4\sqrt{2}}{8}$}

\loigiai{
	Ta có 
\allowdisplaybreaks
\begin{eqnarray*}
		f(x) &= &\displaystyle\int f'(x) \mathrm{d}x = \displaystyle\int (2 \cos^2 \dfrac{x}{2} + 3) \mathrm{d}x\\
	&=&\displaystyle\int \left( 2 \cdot \dfrac{1 + \cos x}{2} + 3 \right) \mathrm{d}x = \displaystyle\int (\cos x + 4) \mathrm{d}x\\
	&	\Rightarrow& f(x) = \sin x + 4x + C	.
\end{eqnarray*}	
Do $f(0) = 4 \Rightarrow C = 4\Rightarrow  	f(x) = \sin x + 4x + 4$. Vậy\\
$$ \displaystyle\int\limits_{0}^{\frac{\pi}{4}} f(x) \mathrm{d}x = \displaystyle\int\limits_{0}^{\frac{\pi}{4}} (\sin x + 4x + 4) \mathrm{d}x	= \left( -\cos x + 2x^2 + 4x \right) \bigg|_{0}^{\frac{\pi}{4}} = \dfrac{\pi^2 + 8\pi - 8 - 4\sqrt{2}}{8}.$$
}
\end{ex}
\begin{ex}%[2D4H2-4]
	Cho hàm số $f(x) = \heva{
		&e^{2x} \text{ khi } x \geq 0 \\
		&x^2 + x + 2 \text{ khi } x < 0 
	}$. Biết tích phân $\displaystyle\int\limits_{-1}^{1} f(x) \mathrm{d}x = \dfrac{a}{b} + \dfrac{e^2}{c}$ ($\dfrac{a}{b}$ là phân số tối giản). Giá trị $a + b + c$ bằng
	\choice
	{$7$}
	{$8$}
	{\True$9$}
	{$10$}
\loigiai{
	Ta có
	$$
	I = \displaystyle\int\limits_{-1}^{1} f(x) \mathrm{d}x = \displaystyle\int\limits_{-1}^{0} (x^2 + x + 2) \mathrm{d}x + \displaystyle\int\limits_{0}^{1} e^{2x} \mathrm{d}x = \dfrac{4}{3} + \dfrac{e^2}{2}
	.$$	
	Vậy $a + b + c = 9$.
}
\end{ex}
\begin{ex}%[2D4H2-2]
	Cho hàm số $f(x) = \heva{
		&x^2 - 1 \text{ khi } x \geq 2 \\
		&x^2 - 2x + 3 \text{ khi } x < 2 
	}$. Tích phân $I = \dfrac{1}{2} \displaystyle\int\limits_{1}^{3} f(x) \mathrm{d}x$ bằng:
	\choice
	{$\dfrac{23}{3}$}
	{\True $\dfrac{23}{6}$}
	{$\dfrac{17}{6}$}
	{$\dfrac{17}{3}$}
\loigiai{
Ta có
	$$I = \dfrac{1}{2} \displaystyle\int\limits_{1}^{3} f(x) \mathrm{d}x = \dfrac{1}{2} \left[ \displaystyle\int\limits_{1}^{2} (x^2 - 2x + 3) \mathrm{d}x + \displaystyle\int\limits_{2}^{3} (x^2 - 1) \mathrm{d}x \right] = \dfrac{23}{6}.$$
}
\end{ex}
\begin{ex}%[2D4H2-2]
	Cho hàm số $f(x) = \heva{
		&\dfrac{x(1 + x^2)}{x - 4} \text{ khi } x \geq 3 \\
		&\dfrac{1}{x - 4} \text{ khi } x < 3 
	}$. Tích phân $I = \displaystyle\int\limits_{2}^{4} f(t) \mathrm{d}t$ bằng:
	\choice
	{$\dfrac{40}{3} - \ln 2$}
	{$\dfrac{95}{6} + \ln 2$}
	{$\dfrac{189}{4} + \ln 2$}
	{\True $\dfrac{189}{4} - \ln 2$}
\loigiai{
Ta có
$$	I = \displaystyle\int\limits_{2}^{4} f(t) \mathrm{d}t = \displaystyle\int\limits_{2}^{3} \dfrac{1}{x - 4} \mathrm{d}x + \displaystyle\int\limits_{3}^{4} \dfrac{x(1 + x^2)}{x - 4} \mathrm{d}x = \dfrac{189}{4} - \ln 2
	.$$
}
\end{ex}
\begin{ex}%[2D4H2-2]
	Cho số thực $a$ và hàm số $f(x) = \heva{
		&2x \text{ khi } x \leq 0 \\
		&a(x - x^2) \text{ khi } x > 0 
	}$. Tính tích phân $\displaystyle\int\limits_{-1}^{1} f(x) \mathrm{d}x$ bằng:
	\choice
	{\True $\dfrac{a}{6} - 1$}
	{$\dfrac{2a}{3} + 1$}
	{$\dfrac{a}{6} + 1$}
	{$\dfrac{2a}{3} - 1$}
\loigiai{
	Ta có
\allowdisplaybreaks
\begin{eqnarray*}
&&	\displaystyle\int\limits_{-1}^{1} f(x) \mathrm{d}x = \displaystyle\int\limits_{-1}^{0} f(x) \mathrm{d}x + \displaystyle\int\limits_{0}^{1} f(x) \mathrm{d}x = \displaystyle\int\limits_{-1}^{0} 2x \mathrm{d}x + \displaystyle\int\limits_{0}^{1} a(x - x^2) \mathrm{d}x\\
&	=& (x^2) \bigg|_{-1}^{0} + a \left( \dfrac{x^2}{2} - \dfrac{x^3}{3} \right) \bigg|_{0}^{1} = -1 + a \left( \dfrac{1}{6} \right) = \dfrac{a}{6} - 1.	
\end{eqnarray*}
}
\end{ex}
\Closesolutionfile{ans}
\indapan{6}{ans/ans-2C4B2CD2-LC}
\TNTF
\Opensolutionfile{ans}[ans/ans-2C4B2CD2-DS]
\begin{ex}%[2D4H2-2]
	Cho hàm số $f(x) = \heva{
		&2x^2 + 3 \text{ khi } x \geq 1 \\
		&2 - x^3 \text{ khi } x < 1 
	}$.
	\choiceTF
	{\True $\displaystyle\int\limits_{1}^{2024} f(x) \mathrm{d}x = \displaystyle\int\limits_{1}^{2024} (2x^2 + 3) \mathrm{d}x$}
{\True $\displaystyle\int\limits_{-2024}^{1} f(x) \mathrm{d}x = \displaystyle\int\limits_{-2024}^{1} (2 - x^3) \mathrm{d}x$}
	{$\displaystyle\int\limits_{-2024}^{2024} f(x) \mathrm{d}x = \displaystyle\int\limits_{1}^{2024} (2x^2 + 3) \mathrm{d}x + \displaystyle\int\limits_{-2024}^{1} (2 - x^3) \mathrm{d}x$}
	{\True $\displaystyle\int_{-2024}^{2024} f(x) \mathrm{d}x = \displaystyle\int\limits_{1}^{2024} (2x^2 + 3) \mathrm{d}x + \displaystyle\int\limits_{-2024}^{1} (2 - x^3) \mathrm{d}x$}
\loigiai{
Do $f(x) = \heva{
	&2x^2 + 3 \text{ khi } x \geq 1 \\
	&2 - x^3 \text{ khi } x < 1 
}$ nên\\
\begin{itemize}
	\item $\displaystyle\int\limits_{1}^{2024} f(x) \mathrm{d}x = \displaystyle\int\limits_{1}^{2024} (2x^2 + 3) \mathrm{d}x.
	$
	\item $\displaystyle\int\limits_{-2024}^{1} f(x) \mathrm{d}x = \displaystyle\int\limits_{-2024}^{1} (2 - x^3) \mathrm{d}x.
	$
\item $
\displaystyle\int\limits_{-2024}^{2024} f(x) \mathrm{d}x = \displaystyle\int\limits_{1}^{2024} (2x^2 + 3) \mathrm{d}x + \displaystyle\int\limits_{-2024}^{1} (2 - x^3) \mathrm{d}x.
$
\item $
\displaystyle\int\limits_{-2024}^{2024} f(x) \mathrm{d}x = \displaystyle\int\limits_{1}^{2024} (2x^2 + 3) \mathrm{d}x + \displaystyle\int\limits_{-2024}^{1} (2 - x^3) \mathrm{d}x.
$
\end{itemize}
}

\end{ex}

\begin{ex}%[2D4H2-2]
	Cho hàm số $f(x) = \heva{
		&x^2 - 2x + 3 \text{ khi } x \geq 2 \\
		&x + 1 \text{ khi } x < 2 
	}$.
	\choiceTF
	{\True $\displaystyle\int\limits_{1}^{2} f(x) \mathrm{d}x = \displaystyle\int\limits_{1}^{2} (x + 1) \mathrm{d}x$}
	{\True $\displaystyle\int\limits_{2}^{3} f(x) \mathrm{d}x = \displaystyle\int\limits_{2}^{3} (x^2 - 2x + 3) \mathrm{d}x$}
	{\True $\displaystyle\int\limits_{1}^{3} \dfrac{1}{2} f(x) \mathrm{d}x = \dfrac{41}{12}$}
	{$\displaystyle\int\limits_{1}^{2} f(x) \mathrm{d}x = \displaystyle\int\limits_{1}^{2} (x^2 - 2x + 3) \mathrm{d}x$}
\loigiai{
	Do $f(x) = \heva{
		&x^2 - 2x + 3 \text{ khi } x \geq 2 \\
		&x + 1 \text{ khi } x < 2 
	}$ nên\\
\begin{itemize}
	\item $
	\displaystyle\int\limits_{1}^{2} f(x) \mathrm{d}x = \displaystyle\int\limits_{1}^{2} (x + 1) \mathrm{d}x	$.
\item 	$
\displaystyle\int\limits_{2}^{3} f(x) \mathrm{d}x = \displaystyle\int\limits_{2}^{3} (x^2 - 2x + 3) \mathrm{d}x$.
\item 	$
 \displaystyle\int\limits_{1}^{3} \dfrac{1}{2} f(x) \mathrm{d}x = \dfrac{1}{2} \left( \displaystyle\int\limits_{1}^{2} (x + 1) \mathrm{d}x + \displaystyle\int\limits_{2}^{3} (x^2 - 2x + 3) \mathrm{d}x \right) = \dfrac{41}{12}
$.
\end{itemize}
}
\end{ex}
\Closesolutionfile{ans}
\indapan{2}{ans/ans-2C4B2CD2-DS}
\TNSA
\Opensolutionfile{ans}[ans/ans-2C4B2CD2-KQ]
\begin{ex}%[2D4H1-2]
	Cho hàm số $f(x) = \heva{
		&\dfrac{1}{x} \text{ khi } x \geq 1 \\
		&x + 1 \text{ khi } x < 1 
	}$. Tích phân $I = \displaystyle\int\limits_{2}^{0} -3t^2 f(t) \mathrm{d}t$. (\textit{\textit{làm tròn đến hàng phần trăm}})
\shortans{$2{,}08$}
\loigiai{
	Ta có\\
	$
	I = -3 \displaystyle\int\limits_{2}^{0} t^2 f(t) \mathrm{d}t = 3 \displaystyle\int\limits_{0}^{2} t^2 f(t) \mathrm{d}t = 3 \left[ \displaystyle\int\limits_{0}^{1} x^2 (x + 1) \mathrm{d}x + \displaystyle\int\limits_{1}^{2} x^2 \cdot \dfrac{1}{x} \mathrm{d}x \right] = \dfrac{25}{12}\approx 2{,}08
	$.
}
\end{ex}
\begin{ex}%[2D4H1-2]
	Cho hàm số $f(x) = \heva{
		&2x^2 - 1 \text{ khi } x < 0 \\
		&x - 1 \text{ khi } 0 \leq x \leq 2 \\
		&5 - 2x \text{ khi } x > 2 
	}$. Tính tích phân $I = \displaystyle\int\limits_{-5}^{9} \dfrac{1}{7} f(t) \mathrm{d}t$. (\textit{làm tròn đến hàng phần trăm})
\shortans{$5{,}19$}
\loigiai{
	Ta có
\allowdisplaybreaks
\begin{eqnarray*}
		I &=& \dfrac{1}{7} \displaystyle\int\limits_{-5}^{9} f(t) \mathrm{d}t = \dfrac{1}{7} \displaystyle\int\limits_{-5}^{9} f(x) \mathrm{d}x = \dfrac{1}{7} \left( \displaystyle\int\limits_{-5}^{0} f(x) \mathrm{d}x + \displaystyle\int\limits_{0}^{2} f(x) \mathrm{d}x + \displaystyle\int\limits_{2}^{9} f(x) \mathrm{d}x \right)\\
&	=& \dfrac{1}{7} \displaystyle\int\limits_{-5}^{0} (2x^2 - 1) \mathrm{d}x + \dfrac{1}{7} \displaystyle\int\limits_{0}^{2} (x - 1) \mathrm{d}x + \dfrac{1}{7} \displaystyle\int\limits_{2}^{9} (5 - 2x) \mathrm{d}x = \dfrac{109}{21}	\approx 5{,}19.
\end{eqnarray*}
}
\end{ex}
\begin{ex}%[2D4H1-2]
	Cho hàm số $f(x) = \heva{
		&x^2 - x \text{ khi } x \geq 0 \\
		&x \text{ khi } x < 0 
	}$. Khi đó $I = \displaystyle\int\limits_{-1}^{1} f(x) \mathrm{d}x + \displaystyle\int\limits_{-1}^{3} f(x) \mathrm{d}x$ bằng bao nhiêu? (\textit{làm tròn đến hàng phần trăm})
\shortans{$3{,}33$}
\loigiai{
	Đặt 	$	I_1 = \displaystyle\int\limits_{-1}^{1} f(x) \mathrm{d}x$ và 	$	I_2 = \displaystyle\int\limits_{-1}^{3} f(x) \mathrm{d}x
	$. \\	
	Vì  $f(x) = \heva{
		&x^2 - x \text{ khi } x \geq 0 \\
		&x \text{ khi } x < 0 
	}$ nên \\
	$$ I_1 = \displaystyle\int\limits_{-1}^{0} x \mathrm{d}x + \displaystyle\int\limits_{0}^{1} (x^2 - x) \mathrm{d}x = -\dfrac{2}{3}.$$
Và
	$$
 I_2 = \displaystyle\int\limits_{-1}^{0} x \mathrm{d}x + \displaystyle\int_{0}^{3} (x^2 - x) \mathrm{d}x = 4.$$	
	Vậy $I = I_1 + I_2 = \dfrac{10}{3}\approx 3{,}33$.
}
\end{ex}

\begin{ex}%[2D4H1-2]
	Cho hàm số $f(x) = \heva{
		&4x \text{ khi } x > 2 \\
		&-2x + 12 \text{ khi } x \leq 2 
	}$. Tính tích phân $I = \displaystyle\int\limits_{1}^{2} f(t) \mathrm{d}t + \dfrac{1}{2} \displaystyle\int\limits_{5}^{10} f(t) \mathrm{d}t$.
\shortans{$84$}
\loigiai{
Đặt 	$I_1 = \displaystyle\int\limits_{1}^{2} f(t) \mathrm{d}t = \displaystyle\int\limits_{1}^{2} f(x) \mathrm{d}x$ và 	$
I_2 = \dfrac{1}{2} \displaystyle\int\limits_{5}^{10} f(t) \mathrm{d}t = \dfrac{1}{2} \displaystyle\int\limits_{5}^{10} f(x) \mathrm{d}x$.\\
	Vì  $f(x) = \heva{
		&4x \text{ khi } x > 2 \\
		&-2x + 12 \text{ khi } x \leq 2 
	}$ nên\\	
$$I_1 = \displaystyle\int\limits_{1}^{2} (-2x + 12) \mathrm{d}x = 9.$$
Và 
	$$
I_2 = \dfrac{1}{2} \displaystyle\int\limits_{5}^{10} 4x \mathrm{d}x = 75.$$	
	Vậy $I = I_1 + I_2 = 84$.
}
\end{ex}

\begin{ex}%[2D4H1-2]
	Biết rằng hàm số $f(x) = mx + n$ thỏa mãn $\displaystyle\int\limits_{0}^{1} f(x) \mathrm{d}x = 3$, $\displaystyle\int\limits_{0}^{2} f(x) \mathrm{d}x = 8$. Tính $m + n$.
\shortans{$4$}
\loigiai{
	Ta có 
	$
	\displaystyle\int f(x) \mathrm{d}x = \displaystyle\int (mx + n) \mathrm{d}x = \dfrac{m}{2} x^2 + nx + C
	$.\\	
	Lại có
	$
	\displaystyle\int\limits_{0}^{1} f(x) \mathrm{d}x = 3 \Rightarrow \left( \dfrac{m}{2} x^2 + nx \right) \bigg|_{0}^{1} = 3 \Rightarrow \dfrac{1}{2} m + n = 3 \quad (1)
	$.\\
	$	\displaystyle\int\limits_{0}^{2} f(x) \mathrm{d}x = 8 \Rightarrow \left( \dfrac{m}{2} x^2 + nx \right) \bigg|_{0}^{2} = 8 \Rightarrow 2m + 2n = 8 \quad (2)
	$.\\	
	Từ (1) và (2) ta có hệ phương trình
$$
\heva{&\dfrac{1}{2} m + n = 3 \\
		&2m + 2n = 8 }
	\Rightarrow \heva{&	m = 2 \\&	n = 2.}
	$$	
Vậy $ m + n = 4$.
}
\end{ex}
\begin{ex}%[2D4V1-2]
	Biết rằng hàm số $f(x) = ax^2 + bx + c$ thỏa mãn $\displaystyle\int\limits_{0}^{1} f(x) \mathrm{d}x = -\dfrac{7}{2}$, $\displaystyle\int\limits_{0}^{2} f(x) \mathrm{d}x = -2$ và $\displaystyle\int\limits_{0}^{3} f(x) \mathrm{d}x = \dfrac{13}{2}$. Tính $P = a + b + c$. (\textit{làm tròn đến hàng phần trăm}).
	\shortans{$-1{,}33$}
\loigiai{
	Ta có
	$	\displaystyle\int f(x) \mathrm{d}x = \displaystyle\int (ax^2 + bx + c) \mathrm{d}x = \dfrac{a}{3} x^3 + \dfrac{b}{2} x^2 + cx + C
	$.\\	
	Lại có
	$	\displaystyle\int\limits_{0}^{1} f(x) \mathrm{d}x = -\dfrac{7}{2} \Rightarrow \left( \dfrac{a}{3} x^3 + \dfrac{b}{2} x^2 + cx \right) \bigg|_{0}^{1} = -\dfrac{7}{2} \Rightarrow \dfrac{1}{3} a + \dfrac{1}{2} b + c = -\dfrac{7}{2} \quad (1)
	$.	
	$	\displaystyle\int\limits_{0}^{2} f(x) \mathrm{d}x = -2 \Rightarrow \left( \dfrac{a}{3} x^3 + \dfrac{b}{2} x^2 + cx \right) \bigg|_{0}^{2} = -2 \Rightarrow \dfrac{8}{3} a + 2b + 2c = -2 \quad (2)
	$.\\	
	$	\displaystyle\int\limits_{0}^{3} f(x) \mathrm{d}x = \dfrac{13}{2} \Rightarrow \left( \dfrac{a}{3} x^3 + \dfrac{b}{2} x^2 + cx \right) \bigg|_{0}^{3} = \dfrac{13}{2} \Rightarrow 9a + \dfrac{9}{2} b + 3c = \dfrac{13}{2} \quad (3)
	$.\\	
	Từ (1), (2) và (3) ta có hệ phương trình:
	$$\heva{&\dfrac{1}{3} a + \dfrac{1}{2} b + c = -\dfrac{7}{2} \\&
		\dfrac{8}{3} a + 2b + 2c = -2 \\&
		9a + \dfrac{9}{2} b + 3c = \dfrac{13}{2}}
	\Rightarrow \heva{&	a = 1 \\&
		b = 3 \\&
		c = -\dfrac{16}{3}.}	$$	
Vậy	$P = a + b + c = 1 + 3 + \left( -\dfrac{16}{3} \right) = -\dfrac{4}{3}\approx -1{,}33$.
}
\end{ex}

% \begin{ex}%[2D4H1-2]
% 	Có hai giá trị của số thực $a$ là $a_1$, $a_2$ ($0 < a_1 < a_2$) thỏa mãn $\displaystyle\int\limits_{1}^{a} (2x - 3) \mathrm{d}x = 0$. Hãy tính $T = 3^{a_1} + 3^{a_2} + \log_2 \left( \dfrac{a_2}{a_1} \right)$.
% 	\shortans{$13$}
% \loigiai{
% 	Ta có
% 	$\displaystyle\int\limits_{1}^{a} (2x - 3) \mathrm{d}x = \left( x^2 - 3x \right) \bigg|_{1}^{a} = a^2 - 3a + 2	$.\\	
% 	Vì $\displaystyle\int\limits_{1}^{a} (2x - 3) \mathrm{d}x = 0$ nên $a^2 - 3a + 2 = 0$, suy ra $\hoac{&a = 1 \\ &a = 2.}$\\	
% 	Lại có $0 < a_1 < a_2$ nên $a_1 = 1$, $a_2 = 2$.\\	
% 	Như vậy $T = 3^{a_1} + 3^{a_2} + \log_2 \left( \dfrac{a_2}{a_1} \right) = 3^1 + 3^2 + \log_2 \left( \dfrac{2}{1} \right) = 13$.
% }
% \end{ex}
\begin{ex}%[2D4H1-2]
	Cho $\displaystyle\int\limits_{0}^{m} (3x^2 - 2x + 1) \mathrm{d}x = 6$. Tính giá trị của tham số $m$.
	\shortans{$2$}
\loigiai{
	Ta có\\
	$	\displaystyle\int\limits_{0}^{m} (3x^2 - 2x + 1) \mathrm{d}x = \left( x^3 - x^2 + x \right) \bigg|_{0}^{m} = m^3 - m^2 + m	$.\\	
	$
	\displaystyle\int\limits_{0}^{m} (3x^2 - 2x + 1) \mathrm{d}x = 6 \Leftrightarrow m^3 - m^2 + m - 6 = 0 \Leftrightarrow m = 2
	$.
}
\end{ex}
\begin{ex}%[2D4V1-2]
	Cho $I = \displaystyle\int\limits_{0}^{1} (4x - 2m^2) \mathrm{d}x$. Có bao nhiêu giá trị nguyên của $m$ để $I + 6 > 0$?
\shortans{$3$}
\loigiai{
	Theo định nghĩa tích phân ta có:
	$	I = \displaystyle\int\limits_{0}^{1} (4x - 2m^2) \mathrm{d}x = \left( 2x^2 - 2m^2 x \right) \bigg|_{0}^{1} = -2m^2 + 2
	$.\\
		Khi đó $I + 6 > 0 \Leftrightarrow -2m^2 + 2 + 6 > 0 \Leftrightarrow -2m^2 + 8 > 0  \Leftrightarrow -2 < m < 2$.\\	
	Mà $m$ là số nguyên nên $m \in \{-1; 0; 1\}$.\\	
	Vậy có $3$ giá trị nguyên của $m$ thỏa mãn yêu cầu.
}
\end{ex}
\begin{ex}%[2D4V1-2]
	Có bao nhiêu giá trị nguyên dương của $a$ để $\displaystyle\int\limits_{0}^{a} (2x - 3) \mathrm{d}x \leq 4$?
\shortans{$4$}
\loigiai{
	Ta có
	$
	\displaystyle\int\limits_{0}^{a} (2x - 3) \mathrm{d}x = \left( x^2 - 3x \right) \bigg|_{0}^{a} = a^2 - 3a
	$.\\	
	Khi đó
	$
	\displaystyle\int\limits_{0}^{a} (2x - 3) \mathrm{d}x \leq 4 \Leftrightarrow a^2 - 3a \leq 4 \Leftrightarrow -1 \leq a \leq 4
	$.\\	
	Mà $a \in \mathbb{N}^*$ nên $a \in \{1; 2; 3; 4\}$.\\	
	Vậy có $4$ giá trị của $a$ thỏa đề bài.
}
\end{ex}
\begin{ex}%[2D4V1-2]
	Có bao nhiêu số thực $b$ thuộc khoảng $(\pi; 3\pi)$ sao cho $\displaystyle\int\limits_{\pi}^{b} 4 \cos 2x \mathrm{d}x = 1$?
\shortans{$4$}
\loigiai{
	Ta có
	$
	\displaystyle\int\limits_{\pi}^{b} 4 \cos 2x \mathrm{d}x = 1 \Leftrightarrow 2 \sin 2x \bigg|_{\pi}^{b} = 1 \Leftrightarrow \sin 2b - \sin 2\pi = \dfrac{1}{2} \Leftrightarrow \sin 2b = \dfrac{1}{2}
	$.\\	
	$
	\Rightarrow 2b = \dfrac{\pi}{6} + k2\pi \quad \text{ hoặc } \quad 2b = \dfrac{5\pi}{6} + k2\pi
	$\\	
	$
	\Rightarrow b = \dfrac{\pi}{12} + k\pi \quad \text{ hoặc } \quad b = \dfrac{5\pi}{12} + k\pi\qquad (k\in \mathbb{Z})
	$.\\
Khi $	b = \dfrac{\pi}{12} + k\pi$, ta xét\\
\allowdisplaybreaks
\begin{eqnarray*}
&& \pi< \dfrac{\pi}{12} + k\pi<3\pi\\
&\Leftrightarrow& \dfrac{11}{12}<k<\dfrac{35}{12}\\
&\Leftrightarrow& k \in \{1;2\}.
\end{eqnarray*}
Khi $	b = \dfrac{5\pi}{12} + k\pi$, ta xét\\
\allowdisplaybreaks
\begin{eqnarray*}
&& \pi< \dfrac{5\pi}{12} + k\pi<3\pi\\
	&\Leftrightarrow& \dfrac{7}{12}<k<\dfrac{31}{12}\\
	&\Leftrightarrow& k \in \{1;2\}.
\end{eqnarray*}
	Vậy có $4$ số thực $b$ thỏa mãn yêu cầu bài toán.
}
\end{ex}
\Closesolutionfile{ans}
\indapan{2}{ans/ans-2C4B2CD2-KQ}
% \Opensolutionfile{ans}[ans/ans-2C4B2CD3-LC]
\begin{dang}{Ứng dụng tích phân trong thực tiễn}
    \begin{itemize}
        \item Cho hàm số$f\left(x \right)$ liên tục trên đoạn $\left[a;b \right]$. Khi đó $\dfrac{1}{b-a}\displaystyle\int\limits_a^b{f\left(x \right)dx}$ được gọi là giá trị trung bình của hàm số $f\left(x \right)$ trên đoạn $\left[a;b \right]$.
        \item Đạo hàm của quãng đường di chuyển của vật theo thời gian bằng tốc độ của chuyển động tại mọi thời điểm $v(t)=s'(t)$. Do đó, nếu biết tốc độ $v(t)$ tại mọi thời điểm $t\in \left[a;b \right]$ thì tính được quãng đường di chuyển trong khoảng thời gian từ $a$ đến $b$ theo công thức
        $$s=s\left(b \right)-s\left(a \right)=\displaystyle\int\limits_a^b v(t)\mathrm{\,d}t.$$
        \item Giả sử là vận tốc của vật tại thời điểm và là quãng đường vật đi được sau khoảng thời gian tính từ lúc bắt đầu chuyển động. Ta có mối liên hệ giữa vận tốc và quãng đường như sau
        \begin{itemize}
            \item Đạo hàm của quãng đường là vận tốc $s'(t)=v(t)$.
            \item Nguyên hàm của vận tốc là quãng đường $s(t)= \displaystyle\int v(t)\mathrm{\,d}t$.
        \end{itemize}
        $\Rightarrow$ Từ đây ta cũng có quãng đường vật đi được trong khoảng thời gian từ $a$ đến $b$ là 
        $$\displaystyle\int\limits_a^b v(t)\mathrm{\,d}t=s(b)-s(a).$$ 
        Nếu gọi $a(t)$ là gia tốc của vật thì ta có mối liên hệ giữa gia tốc và vận tốc như sau
        \begin{itemize}
            \item Đạo hàm của vận tốc là gia tốc $v'(t)=a(t)$.
            \item Nguyên hàm của gia tốc là vận tốc $v(t)= \displaystyle\int a(t)\mathrm{\,d}t$.
        \end{itemize}
    \end{itemize}
\end{dang}
\TN
\begin{ex}%[2D4H2-6] 
    Một ô tô đang chạy với vận tốc $10\,m/s$ thì gặp chướng ngại vật, người lái xe đạp phanh. Từ thời điểm đó, ô tô chuyển động chậm dần đều với vận tốc $v\,\left(t \right)=-2t+10\,\left(m/s \right)$, trong đó $t$ là khoảng thời gian tính bằng giây, kể từ lúc bắt đầu đạp phanh. Tính quãng đường ô tô di chuyển được trong $8$ giây cuối cùng.
    \choice
    {\True $55\,m$}
    {$25\,m$}
    {$50\,m$}
    {$16\,m$}
    \loigiai{
    Ta có $-2t+10=0\Leftrightarrow t=5\Rightarrow$ thời gian tính từ lúc bắt đầu đạp phanh đến khi dừng hẳn là $5$ giây.\\ 
    Vậy trong $8$ giây cuối cùng thì có $3$ giây ô tô chuyển động với vận tốc $10\,m/s$ và $5$ giây chuyển động chậm dần đều với vận tốc $v\left(t \right)=-2t+10\,\left(m/s \right)$.\\
    Khi đó quãng đường ô tô di chuyển là $$S=3\cdot 10+\displaystyle\int\limits_0^5 \left(-2t+10\right)\mathrm{\,d}t=30+25=55\,m.$$
    }
\end{ex}

\begin{ex}%[2D4H2-6]
    Một ô tô đang chạy với tốc độ $20\,\left(m/s \right)$ thì gặp chướng ngại vật, người lái đạp phanh, từ thời điểm đó ô tô chuyển động chậm dần đều với vận tốc $v\left(t \right)=-5t+20\,\left(m/s \right)$, trong đó $t$ là khoảng thời gian tính bằng giây, kể từ lúc bắt đầu đạp phanh. Hỏi từ lúc đạp phanh đến khi dừng hẳn, ô tô còn di chuyển bao nhiêu mét ($m$)?
    \choice
    {$20\,m$}
    {$30\,m$}
    {$10\,m$}
    {\True $40\,m$}
    \loigiai{
    Khi ô tô dừng hẳn thì $v\left(t \right)=0\Leftrightarrow-5t+20=0\Leftrightarrow t=4\,\left(s \right)$.\\
    Vậy từ lúc đạp phanh đến khi dừng hẳn, ô tô di chuyển được 
    $$s=\displaystyle\int\limits_0^4 \left(-5t+20\right)\mathrm{\,d}t=40\,\left(m \right).$$
}
\end{ex}

% \begin{ex}%[2D4V2-6]
%     Một chất điểm $A$ xuất phát từ $O$, chuyển động thẳng với vận tốc biến thiên theo thời gian bởi quy luật $v\left(t \right)=\dfrac{1}{120}t^2+\dfrac{58}{45}t\,\left(m/s \right)$, trong đó $t$ (giây) là khoảng thời gian tính từ lúc $A$ bắt đầu chuyển động. Từ trạng thái nghỉ, một chất điểm $B$ cũng xuất phát từ $O$, chuyển động thẳng cùng hướng với $A$ nhưng chậm hơn $3$ giây so với $A$ và có gia tốc bằng $a\,\left(m/s^2 \right)$ ($a$ là hằng số). Sau khi $B$ xuất phát được $15$ giây thì đuổi kịp $A$. Vận tốc của $B$ tại thời điểm đuổi kịp $A$ bằng
%     \choice
%     {$21\,\left(m/s \right)$}
%     {$25\,\left(m/s \right)$}
%     {$36\,\left(m/s \right)$}
%     {\True $30\,\left(m/s \right)$}
%     \loigiai{
%     Thời điểm chất điểm $B$ đuổi kịp chất điểm $A$ thì chất điểm $B$ đi được $15$ giây, chất điểm $A$ đi được $18$ giây.\\
%     Biểu thức vận tốc của chất điểm $B$ có dạng $v_B\left(t \right)=\displaystyle\int a\mathrm{\,d}t =at+C$ mà $v_B\left(0\right)=0$ nên $v_B\left(t \right)=at$.\\
%     Do từ lúc chất điểm $A$ bắt đầu chuyển động cho đến khi chất điểm $B$ đuổi kịp thì quãng đường hai chất điểm đi được bằng nhau.\\
%     Do đó $\displaystyle\int\limits_0^{18} \left(\dfrac{1}{120}t^2+\dfrac{58}{45} \right)\mathrm{\,d}t=\displaystyle\int\limits_0^{15} at\mathrm{\,d}t \Leftrightarrow 225=a\cdot\dfrac{225}{2}\Leftrightarrow a=2$.\\
%     Vậy vận tốc của chất điểm $B$ tại thời điểm đuổi kịp $A$ bằng 
%     $$v_B\left(t \right)=2\cdot 15=30\,\left(m/s \right).$$
%     }
% \end{ex}

\begin{ex}%[2D4V2-6]
    Một chất điểm $A$ xuất phát từ $O$, chuyển động thẳng với vận tốc biến thiên theo thời gian bởi quy luật $v\left(t \right)=\dfrac{1}{150}t^2+\dfrac{59}{75}t\,\left(m/s \right)$, trong đó $t$ (giây) là khoảng thời gian tính từ lúc $a$ bắt đầu chuyển động. Từ trạng thái nghỉ, một chất điểm $B$ cũng xuất phát từ $O$, chuyển động thẳng cùng hướng với $A$ nhưng chậm hơn $3$ giây so với $A$ và có gia tốc bằng $a\,\left(m/s^2 \right)$ ($a$ là hằng số). Sau khi $B$ xuất phát được $12$ giây thì đuổi kịp $A$. Vận tốc của $B$ tại thời điểm đuổi kịp $A$ bằng
    \choice
    {$15\,\left(m/s \right)$}
    {$20\,\left(m/s \right)$}
    {\True $16\,\left(m/s \right)$}
    {$13\,\left(m/s \right)$}
    \loigiai{
    Quãng đường chất điểm $A$ đi từ đầu đến khi $B$ đuổi kịp là 
    $$S=\displaystyle\int\limits_0^{15} \left(\dfrac{1}{150}t^2+\dfrac{59}{75}t \right)\mathrm{\,d}t=96\,\left(m \right).$$
    Vận tốc của chất điểm $B$ là 
    $$v_B\left(t \right)=\displaystyle\int a\mathrm{\,d}t=at+C.$$
    Tại thời điểm $t=3$ vật $B$ bắt đầu từ trạng thái nghỉ nên $v_B\left(3\right)=0\Leftrightarrow C=-3a$.\\
    Lại có quãng đường chất điểm $B$ đi được đến khi gặp $A$ là 
    $$S_2=\displaystyle\int\limits_3^{15} \left(at-3a \right)\mathrm{\,d}t=\left. \left(\dfrac{at^2}{2}-3at \right) \right|_3^{15}=72a\,\left(m \right).$$
    Vậy $72a=96\Leftrightarrow a=\dfrac{4}{3}\,\left(m/s^2 \right)$.\\
    Tại thời điểm đuổi kịp $A$ thì vận tốc của $B$ là $v_B\left(15\right)=16\,\left(m/s \right)$.
    }
\end{ex}

\begin{ex}%[2D4V2-6]
    Một ô tô bắt đầu chuyển động thẳng đều với vận tốc $v_0$, sau $6$ giây chuyển động thì gặp chướng ngại vật nên bắt đầu giảm tốc độ với vận tốc chuyển động $v(t)=-\dfrac{5}{2}t+a\,(m/s)$ với $t\ge 6$ cho đến khi dừng hẳn. Biết rằng kể từ lúc chuyển động đến lúc dừng hẳn thì ô tô đi được quãng đường là $80\,m$. Tìm $v_0$.
    \choice
    {$v_0=35\,m/s$}
    {$v_0=25\,m/s$}
    {\True $v_0=10\,m/s$}
    {$v_0=20\,m/s$}
    \loigiai{
    Tại thời điểm $t=6$ vật đang chuyển động với vận tốc $v_0$ nên có 
    $$v(6)=v_0 \Leftrightarrow -\dfrac{5}{2}\cdot 6+a=v_0 \Leftrightarrow a=v_0+15 \Rightarrow v(t)=-\dfrac{5}{2}t+v_0+15.$$
    Gọi $k$ là thời điểm vật dừng hẳn, ta có 
    $$v(k)=0 \Leftrightarrow k=\dfrac{2}{5}\cdot\left(v_0+15\right)\Leftrightarrow k=\dfrac{2v_0}{5}+6.$$
    Tổng quãng đường vật đi được là 
    \allowdisplaybreaks 
    \begin{eqnarray*}
        && 80=6\cdot v_0+\displaystyle\int\limits_6^k \left(-\dfrac{5}{2}t+v_0+15\right)\mathrm{\,d}t\\
        &\Leftrightarrow& 80=6\cdot v_0+\left. \left(-\dfrac{5}{4}t^2+v_0\cdot t+15t \right) \right|_6^k \\ 
        &\Leftrightarrow& 80=6\cdot v_0-\dfrac{5}{4}\left(k^2-6^2\right)+v_0\cdot (k-6)+15(k-6) \\ 
        &\Leftrightarrow& 80=6\cdot v_0-\dfrac{5}{4}\left(\dfrac{4\left(v_0 \right)^2}{25}+\dfrac{24v_0}{5} \right)+v_0\cdot\dfrac{2v_0}{5}+15\cdot\dfrac{2v_0}{5} \\
        &\Leftrightarrow& \left(v_0 \right)^2+36\cdot v_0-400=0\\ 
        &\Leftrightarrow& v_0=10. 
    \end{eqnarray*}
    }
\end{ex}

\begin{ex}%[2D4H2-6]
    Để đảm bảo an toàn khi lưu thông trên đường, các xe ô tô khi dừng đèn đỏ phải cách nhau tối thiểu $1\,m$. Một ô tô $A$ đang chạy với vận tốc $16\,m/s$ bỗng gặp ô tô $B$ đang dừng đèn đỏ nên ô tô $A$ hãm phanh và chuyển động chậm dần đều với vận tốc được biểu thị bởi công thức $v_A\left(t \right)=16-4t$ (đơn vị tính bằng $m/s$), thời gian tính bằng giây. Hỏi rằng để hai ô tô $A$ và $B$ đạt khoảng cách an toàn khi dừng lại thì ô tô $A$ phải hãm phanh khi cách ô tô $B$ một khoảng ít nhất là bao nhiêu?
    \choice
    {$33$}
    {$12$}
    {$31$}
    {\True $32$}
    \loigiai{
    Ta có $v_A\left(0\right)=16\,m/s$.\\
    Khi xe $A$ dừng hẳn $v_A\left(t \right)=0 \Leftrightarrow t=4\,s$.\\
    Quãng đường từ lúc xe $A$ hãm phanh đến lúc dừng hẳn là 
    $$s=\displaystyle\int\limits_0^4 \left(16-4t \right)\mathrm{\,d}t=32\,m.$$
    }
\end{ex}

\begin{ex}%[2D4H2-6]
    Do các xe phải cách nhau tối thiểu $1\,m$ để đảm bảo an toàn nên khi dừng lại ô tô $A$ phải hãm phanh khi cách ô tô $B$ một khoảng ít nhất là $33\,m$. Một chất điểm đang chuyển động với vận tốc $v_0=15\,m/s$ thì tăng tốc với gia tốc $a\left(t \right)=t^2+4t\, \left(m/s^2\right)$. Tính quãng đường chất điểm đó đi được trong khoảng thời gian $3$ giây kể từ lúc bắt đầu tăng vận tốc.
    \choice
    {$70{,}25\, {m}$}
    {$68{,}25\, {m}$}
    {$67{,}25\, {m}$}
    {\True $69{,}75\, {m}$}
    \loigiai{
    Ta có 
    $$a\left(t \right)=t^2+4t \Rightarrow v\left(t \right)=\displaystyle\int a\left(t \right)\mathrm{\,d}t=\dfrac{t^3}{3}+2t^2+C,\, \left(C\in \mathbb{R} \right).$$
    Mà $v\left(0\right)=C=15 \Rightarrow v\left(t \right)=\dfrac{t^3}{3}+2t^2+15$.\\
    Vậy $S=\displaystyle\int\limits_0^3 \left(\dfrac{t^3}{3}+2t^2+15\right)\mathrm{\,d}t=69{,}75\, {m}$.
    }
\end{ex}

\begin{ex}%[2D4V2-6]
    Một vật chuyển động với vận tốc $10\,m/s$ thì tăng tốc với gia tốc được tính theo thời gian là $a\left(t \right)=t^2+3t$. Tính quãng đường vật đi được trong khoảng thời gian $6$ giây kể từ khi vật bắt đầu tăng tốc.
    \choice
    {$136\,{m}$}
    {$126\,{m}$}
    {$276\,{m}$}
    {\True $216\,{m}$}
    \loigiai{
    Ta có $v\left(0\right)=10\,m/s$ và 
    $$v\left(t \right)=\displaystyle\int\limits_0^t a\left(t \right)\mathrm{\,d}t=\displaystyle\int\limits_0^t \left(t^2+3t \right)\mathrm{\,d}t=\left. \left(\dfrac{t^3}{3}+\dfrac{3t^2}{2} \right) \right|_0^t=\dfrac{1}{3}t^3+\dfrac{3}{2}t^2.$$
    Quãng đường vật đi được là 
    $$S=\displaystyle\int\limits_0^6 v\left(t \right)\mathrm{\,d}t=\displaystyle\int\limits_0^6 \left(\dfrac{1}{3}t^3+\dfrac{3}{2}t^2 \right)\mathrm{\,d}t=\left. \left(\dfrac{1}{12}t^4+\dfrac{1}{2}t^3 \right) \right|_0^6=216\,{m}.$$
    }
\end{ex}

% \begin{ex}%[2D4V2-6]
%     Một chiếc máy bay chuyển động trên đường băng với vận tốc $v\left(t \right)=t^2+10t$ $\left(m/s \right)$ với $t$ là thời gian được tính theo đơn vị giây kể từ khi máy bay bắt đầu chuyển động. Biết khi máy bay đạt vận tốc $200\,\left(m/s \right)$ thì rời đường băng. Quãng đường máy bay đã di chuyển trên đường băng là
%     \choice
%     {\True $\dfrac{2500}{3}\,\left(m \right)$}
%     {$2000\,\left(m \right)$}
%     {$500\,\left(m \right)$}
%     {$\dfrac{4000}{3}\,\left(m \right)$}
%     \loigiai{
%     Thời điểm máy bay đạt vận tốc $200\,\left(m/s \right)$ là 
%     $$v\left(t \right)=200 \Leftrightarrow t^2+10t=200 \Leftrightarrow \hoac{& t=10\\ & t=-20}\Leftrightarrow t=10.$$
%     Quãng đường máy bay đã di chuyển trên đường băng là
%     $$s=\displaystyle\int\limits_0^{10} \left(t^2+10t \right)\mathrm{\,d}t=\left.\left(\dfrac{t^3}{3}+5t \right)\right|_0^{10}=\dfrac{2500}{3}\,\left(m \right).$$
%     }
% \end{ex}

\begin{ex}%[2D4V2-6]
    Một ô tô bắt đầu chuyển động nhậnh dần đều với vận tốc $v_1\left(t \right)=7t\,\left(m/s \right)$. Đi được $5\,s$, người lái xe phát hiện chướng ngại vật và phanh gấp, ô tô tiếp tục chuyển động chậm dần đều với gia tốc $a=-70\,\left(m/s^2 \right)$. Tính quãng đường $S$ đi được của ô tô từ lúc bắt đầu chuyển bánh cho đến khi dừng hẳn.
    \choice
    {\True $S=96{,}25\,\left(m\right)$}
    {$S=87{,}5\,\left(m\right)$}
    {$S=94\,\left(m\right)$}
    {$S=95{,}7\,\left(m\right)$}
    \loigiai{
    Chọn gốc thời gian là lúc ô tô bắt đầu đi.\\ 
    Sau $5\,s$ ô tô đạt vận tốc là $v\left(5\right)=35\,\left(m/s\right)$.\\
    Sau khi phanh vận tốc ô tô là $v\left(t\right)=35-70\left(t-5\right)$.\\
    Ô tô dừng tại thời điểm $t=5{,}5\,s$.\\
    Quãng đường ô tô đi được là 
    $$S=\displaystyle\int\limits_0^5 7t\mathrm{\,d}t+\displaystyle\int\limits_5^{5{,}5} \left[35-70\left(t-5\right) \right]\mathrm{\,d}t=96{,}25\,\left(m\right).$$
    }
\end{ex}

\begin{ex}%[2D4V2-6]
    Một ô tô bắt đầu chuyển động nhanh dần đều với vận tốc $v_1\left(t \right)=2t\,\left(m/s\right)$. Đi được $12$ giây, người lái xe gặp chướng ngại vật và phanh gấp, ô tô tiếp tục chuyển động chậm dần đều với gia tốc $a=-12\,\left(m/s^2\right)$. Tính quãng đường $s\left(m\right)$ đi được của ôtô từ lúc bắt đầu chuyển động đến khi dừng hẳn.
    \choice
    {\True $s=168\,\left(m\right)$}
    {$s=166\,\left(m\right)$}
    {$s=144\,\left(m\right)$}
    {$s=152\,\left(m\right)$}
    \loigiai{
    \textbf{Giải đoạn 1:} Xe bắt đầu chuyển động đến khi gặp chướng ngại vật.\\
    Quãng đường xe đi được là
    $$S_1=\displaystyle\int\limits_0^{12} v_1\left(t \right)\mathrm{\,d}t=\displaystyle\int\limits_0^{12} 2t\mathrm{\,d}t =\left. t^2 \right|_0^{12}=144\,\left(m\right).$$
    \textbf{Giải đoạn 2:} Xe gặp chướng ngại vật đến khi dừng hẳn.\\
    Ôtô chuyển động chậm dần đều với vận tốc 
    $$v_2\left(t \right)=\displaystyle\int a\mathrm{\,d}t=-12t+c.$$
    Vận tốc của xe khi gặp chướng ngại vật là $$v_2\left(0\right)=v_1\left(12\right)=2\cdot 12=24\,\left(m/s\right).$$
    Suy ra $-12\cdot 0+c=24 \Rightarrow c=24\Rightarrow v_2\left(t \right)=-12t+24$.\\
    Thời gian khi xe gặp chướng ngại vật đến khi xe dừng hẳn là nghiệm phương trình
    $$-12t+24=0\Leftrightarrow t=2.$$
    Khi đó, quãng đường xe đi được là
    $$S_2=\displaystyle\int\limits_0^2 v_2\left(t \right)\mathrm{\,d}t=\displaystyle\int\limits_0^2 \left(-12t+24\right)\mathrm{\,d}t=\left. \left(-6t^2+24t \right) \right|_0^2=24\,\left(m\right).$$
    Vậy tổng quãng đường xe đi được là $S=S_1+S_2=168\,\left(m\right)$.
    }
\end{ex}
\begin{ex}%Cau12D%[2D4H2-6]
	Một ô tô đang dừng và bắt đầu chuyển động theo một đường thẳng với gia tốc $a\left(t\right)= 6-2t$ (m/s$^2$), trong đó $t$ là khoảng thời gian tính bằng giây kể từ lúc ô tô bắt đầu chuyển động. Hỏi quảng đường ô tô đi được từ lúc bắt đầu chuyển động đến khi vận tốc của ô tô đạt giá trị lớn nhất là bao nhiêu mét?
	\choice
	{\True $18$ m}
	{$36$ m}
	{$22{,}5$ m}
	{$6{,}75$ m}
	\loigiai{
		$a\left(t\right) = 6-2t$ (m/s$^2$) $\Rightarrow v\left(t\right) = \displaystyle\int \left(6-2t\right) \mathrm{\,d}t = 6t - t^2+C$.\\
		Xe dừng và bắt đầu chuyển động nên khi $t=0$ thì $v=0 \Rightarrow C=0 \Rightarrow v\left(t\right) = 6t-t^2$.\\
		$v\left(t\right) = 6t-t^2$ là hàm số bậc $2$ nên đạt giá trị lớn nhất khi $t=-\dfrac{b}{2a}=3$ (s).\\
		Quãng đường xe đi trong $3$ giây đầu là: $S= \displaystyle\int\limits_0^3 \left(6t-t^2\right) \mathrm{\,d}t = 18$ (m).
	}
\end{ex}

% \begin{ex}%Cau13D%[2D4V2-6]
% 	Một chất điểm $A$ xuất phát từ $O$, chuyển động thẳng với vận tốc biến thiên theo thời gian bởi quy luật $v \left(t\right) = \dfrac{1}{180}t^2 + \dfrac{11}{18}t$ (m/s), trong đó $t$ (giây) là khoảng thời gian tính từ lúc $A$ bắt đầu chuyển động. Từ trạng thái nghỉ, một chất điểm $B$ cũng xuất phát từ $O$, chuyển động thẳng cùng hướng với $A$ nhưng chậm hơn $5$ giây so với $A$ và có gia tốc bằng $a$ (m/s$^2$) ($a$ là hằng số). Sau khi $B$ xuất phát được $10$ giây thì đuổi kịp $A$. Vận tốc của $B$ tại thời điểm đuổi kịp $A$ bằng
% 	\choice
% 	{\True $15$ (m/s)}
% 	{$10$ (m/s)}
% 	{$7$ (m/s)}
% 	{$22$(m/s)}
% 	\loigiai{
% 		Thời gian tính từ khi $A$ xuất phát đến khi bị $B$ đuổi kịp là $15$ giây, suy ra quãng đường đi được tới lúc đó là:
% 		$$\displaystyle\int\limits_0^{15} v\left(t\right) \mathrm{\,d}t = \displaystyle\int\limits_0^{15} \left(\dfrac{1}{180}t^2 + \dfrac{11}{18}t \right) \mathrm{\,d}t= \left(\dfrac{1}{540}t^3 + \dfrac{11}{36}t^2 \right)\Big|_0^{15} = 75 \left(\text{m}\right).$$
% 		Vận tốc của chất điểm $B$ là $y\left(t\right) = \displaystyle\int a \mathrm{\,d}t = a \cdot t+C$ ( $C$ là hằng số); do $B$ xuất phát từ trạng thái nghỉ nên có $y\left(0\right)=0 \Leftrightarrow C=0$.\\
% 		Quãng đường của $B$ từ khi xuất phát đến khi đuổi kịp $A$ là
% 		$$\displaystyle\int\limits_0^{10} y\left(t\right) \mathrm{\,d}t = 75 \Leftrightarrow \displaystyle\int\limits_0^{10} a \cdot t \mathrm{\,d}t =75 \Leftrightarrow \dfrac{a \cdot t^2}{2}\Big|_0^{10} = 75 \Leftrightarrow 50a=75 \Leftrightarrow a = \dfrac{3}{2}.$$
% 		Vậy có $y\left(t\right) = \dfrac{3t}{2}$; suy ra vận tốc của $B$ tại thời điểm đuổi kịp $A$ bằng $y\left(10\right) = 15$ (m/s).
% 	}
% \end{ex}

% \begin{ex}%Cau14D%[2D4V2-6]
% 	Một vật chuyển động trong $3$ giờ với vận tốc $v$ (km/h) phụ thuộc thời gian $t$ (h) có đồ thị là một phần của đường parabol có đỉnh $I\left(2;9\right)$ và trục đối xứng song song với trục tung như hình bên. Tính quãng đường $s$ mà vật di chuyển được trong $3$ giờ đó.
% 	\begin{center}
% 		\begin{tikzpicture}[>=stealth, font=\footnotesize, line join=round, line cap=round, thick, smooth, samples=250, scale=0.6, yscale=.7]
% 			% Vẽ 2 trục, điền các số lên trục
% 			\draw[->] (-0.5,0)--(0,0) node[below left]{$O$}--(4,0) node[above]{$t$};
% 			\foreach \x in {2,3}
% 			\draw[shift={(\x,0)},color=black] (0pt,2pt)--(0pt,-2pt) 
% 			node[below] { $\x$};
% 			\draw[->,color=black] (0,-0.5)--(0,10) node[right]{$v$};
% 			\foreach \y in {6,9}
% 			\draw[shift={(0,\y)},color=black] (2pt,0pt) -- (-2pt,0pt) 
% 			node[left] {$\y$};
% 			\clip(-1,-1) rectangle (3,10); %vùng đồ thị
% 			%\draw[gray!50,thin,opacity=.5] (-1,-1) grid (4,10); %ô vuông
% 			%Vẽ đồ thị
% 			\draw[smooth,samples=100,domain=0:10] 
% 			plot(\x,{(-0.75)*(\x)^2+3*(\x)+6});
% 			\draw[dashed] (3,0)--(3,8.25) circle(1.5pt);  \draw[dashed] (2,0)--(2,9) circle(1.5pt) node[above]{$I$}--(0,9) circle(1.5pt);
% 			% Vẽ thêm mấy cái râu ria
			
% 			%Vẽ dấu chấm tròn 
% 			\fill (0cm,0cm) circle (1.5pt); 
% 		\end{tikzpicture}
% 	\end{center}
% 	\choice
% 	{$s = 25{,}25$ (km)}
% 	{$s = 24{,}25$ (km)}
% 	{\True $s = 24{,}75$ (km)}
% 	{$s = 26{,}75$ (km)}
% 	\loigiai{Gọi $v \left(t\right) = at^2 +bt +c$.\\
% 		Đồ thị $v\left(t\right)$ là một phần parabol có đỉnh $I\left(2;9\right)$ và đi qua điểm $A\left(0;6\right)$ nên\\
% 		$\heva{&\dfrac{-b}{2a} = 2\\&a \cdot 2^2 +b\cdot 2 +c=9\\&a \cdot 0^2 + b \cdot 0+c =6} \Rightarrow \heva{&a = -\dfrac{3}{4}\\&b=3\\&c=6}$. Tìm được $v\left(t\right) = -\dfrac{3}{4}t^2 + 3t +6$.\\
% 		Vậy $S = \displaystyle\int\limits_0^{3} \left(-\dfrac{3}{4} t^2 + 3t +6 \right) \mathrm{\,d}t = 24{,}75$ (km).
% 	}
% \end{ex}

\begin{ex}%Cau15D%[2D4H2-6]
	Một vật chuyển động trong $3$ giờ với vận tốc $v$ (km/h) phụ thuộc vào thời gian $t$ (h) có đồ thị vận tốc như hình bên. Trong thời gian $1$ giờ kể từ khi bắt đầu chuyển động, đồ thị đó là một phần của đường parabol có đỉnh $I\left(2;9\right)$ và trục đối xứng song song với trục tung, khoảng thời gian còn lại đồ thị là một đoạn thẳng song song với trục hoành. Tính quãng đường $s$ mà vật chuyển động được trong $3$ giờ đó (kết quả làm tròn đến hàng phần trăm).
	\begin{center}
		\begin{tikzpicture}[>=stealth,scale=0.6, yscale=.7]
			% Vẽ 2 trục, điền các số lên trục
			\draw[->] (-0.5,0)--(0,0) node[below left]{$O$}--(4,0) node[above]{$t$}; %định dạng trục Ox
			\foreach \x in {1,2,3}
			\draw[shift={(\x,0)},color=black] (0pt,2pt)--(0pt,-2pt) 
			node[below] { $\x$};
			\draw[->,color=black] (0,-0.5)--(0,10) node[right]{$v$};  %định dạng trục Oy
			\foreach \y in {4,9}
			\draw[shift={(0,\y)},color=black] (2pt,0pt) -- (-2pt,0pt) 
			node[left] {$\y$};
			\clip(-1,-1) rectangle (3,10); %vùng đồ thị
			%\draw[gray!50,thin,opacity=.5] (-1,-1) grid (4,10); %ô vuông
			%Vẽ đồ thị
			\draw[smooth,samples=100,domain=0:1,font=\footnotesize, line join=round, line cap=round, thick] 
			plot(\x,{(-5/4)*(\x)^2+5*(\x)+4});
			\draw[smooth,samples=100,domain=1:3,dashed] 
			plot(\x,{(-5/4)*(\x)^2+5*(\x)+4});
			\draw[smooth,samples=100,domain=1:3,font=\footnotesize, line join=round, line cap=round, thick] 
			plot(\x,{31/4});
			% Vẽ thêm mấy cái râu ria
			\draw[dashed] (3,0)--(3,31/4) circle(1.5pt);  \draw[dashed] (2,0)--(2,9) circle(1.5pt) node[above]{$I$}--(0,9) circle(1.5pt); \draw[dashed] (1,0)--(1,31/4) circle(1.5pt) --(0,31/4);
			%Vẽ dấu chấm tròn 
			\fill (0cm,0cm) circle (1.5pt); 
		\end{tikzpicture}
	\end{center}
	\choice
	{\True $s = 21{,}58$ (km)}
	{$s = 23{,}25$ (km)}
	{$s = 13{,}83$ (km)}
	{$s = 15{,}50$ (km)}
	\loigiai{Gọi phương trình parabol $v = at^2+bt+c$ ta có hệ như sau
		$$\heva{&c=4\\&4a+2b+c=9\\&-\dfrac{b}{2a}=2} \Leftrightarrow \heva{&b=5\\&c=4\\&a=-\dfrac{5}{4}.}$$
		Với $t=1$ ta có $v = \dfrac{31}{4}$.
		Vậy quãng đường vật chuyển động được là
		$$s = \displaystyle\int\limits_0^1 \left(-\dfrac{5}{4}t^2 + 5t +4 \right) \mathrm{\,d}t + \displaystyle\int\limits_1^3 \dfrac{31}{4} \mathrm{\,d}t = \dfrac{259}{12} \approx 21{,}58.$$
	}
\end{ex}

% \begin{ex} %Cau16D %[2D4H2-6]
% 	Một người chạy trong $2$ giờ, vận tốc $v$ (km/h) phụ thuộc vào thời gian $t$ (h) có đồ thị là $1$ phần của đường Parabol với đỉnh $I\left(1;5\right)$ và trục đối xứng song song với trục tung $Ov$ như hình vẽ. Tính quảng đường $S$ người đó chạy được trong $1$ giờ $30$ phút kể từ lúc bắt đầu chạy (kết quả làm tròn đến $2$ chữ số thập phân).
% 	\begin{center}
% 		\begin{tikzpicture}[>=stealth, font=\footnotesize, line join=round, line cap=round, thick, smooth, samples=250, scale=0.7]
% 			% Vẽ 2 trục, điền các số lên trục
% 			\draw[->] (-0.5,0)--(0,0) node[below left]{$O$}--(3,0) node[above]{$t$}; %định dạng trục Ox
% 			\foreach \x in {2,1}
% 			\draw[shift={(\x,0)},color=black] (0pt,2pt)--(0pt,-2pt) 
% 			node[below] { $\x$};
% 			\draw[->,color=black] (0,-0.5)--(0,6) node[right]{$v$};  %định dạng trục Oy
% 			\foreach \y in {5}
% 			\draw[shift={(0,\y)},color=black] (2pt,0pt) -- (-2pt,0pt) 
% 			node[left] {$\y$};
% 			\clip(-0.5,-0.5) rectangle (3,6); %vùng đồ thị
% 			%\draw[gray!50,thin,opacity=.5] (-1,-1) grid (4,10); %ô vuông
% 			%Vẽ đồ thị
% 			\draw[smooth,samples=100,domain=0:2] 
% 			plot(\x,{-5*(\x)^2+10*(\x)});
% 			% Vẽ thêm mấy cái râu ria
% 			\draw[dashed] (1,0)--(1,5) circle(1.5pt)--(0,5);
% 			%Vẽ dấu chấm tròn 
% 			\fill (0cm,0cm) circle (1.5pt); 
% 		\end{tikzpicture} 
% 	\end{center}
% 	\choice
% 	{$2{,}11$ km}
% 	{$6{,}67$ km}
% 	{\True $5{,}63$ km}
% 	{$6{,}63$ km}
% 	\loigiai{
% 		Ta có $1$ giờ $30$ phút = $1,5$ giờ $\Rightarrow S = \displaystyle\int\limits_0^{1,5} v\left(t\right) \mathrm{\,d}t$.\\
% 		Đồ thị $v = v\left(t\right)$ đi qua gốc tọa độ nên $v\left(t\right)$ có dạng $v\left(t\right) = at^2+bt$.\\
% 		Đồ thị $v\left(t\right)$ có đỉnh là $I\left(1;5\right)$ nên $\heva{&-\dfrac{b}{2a}=1\\&a+b=5} \Leftrightarrow \heva{&b=-2a\\&a+b=5} \Leftrightarrow \heva{&a=-5\\&b=10.}$\\
% 		Suy ra $v\left(t\right) = -5t^2+10$. Do đó
% 		$$S = \displaystyle\int\limits_0^{1,5} \left(-5t^2+10\right) \mathrm{\,d}t = \dfrac{45}{8} \approx 5{,}63.$$
% 	}
% \end{ex}
%--------------------------------------------------------------------
\Closesolutionfile{ans}
\indapan{6}{ans/ans-2C4B2CD3-LC}
\TNSA
\Opensolutionfile{ans}[ans/ans-2C4B2CD3-KQ]
\begin{ex}%Cau17D%[2D4H2-6] 
	Một ô tô đang chạy với vận tốc là $12$ (m/s) thì người lái đạp phanh; từ thời điểm đó ô tô chuyển động chậm dần đều với vận tốc $v \left(t\right) = -6t+12$ (m/s), trong đó $t$ là khoảng thời gian tính bằng giây kể từ lúc đạp phanh. Hỏi từ lúc đạp phanh đến lúc ô tô dừng hẳn, ô tô còn di chuyển được bao nhiêu mét?
	\shortans{$12$}
	\loigiai{
		Lấy mốc thời gian $\left(t=0\right)$ là lúc đạp phanh.\\
		Khi ô tô dừng hẳn thì vận tốc $v\left(t\right)=0$, tức là $v\left(t\right) = -6t+12 = 0 \Leftrightarrow t =2$.\\
		Vậy từ lúc đạp phanh đến lúc ô tô dừng hẳn, ô tô còn di chuyển được quãng đường là:
		$$\displaystyle\int\limits_0^2 \left(-6t+12\right) \mathrm{\,d}t = \left(-3t^2 +12t \right) \Big|_0^2 = 12 \left(\text{m}\right).$$
	}
\end{ex}

\begin{ex}%Cau18D%[2D4H2-6]
	Một ô tô đang chạy với vận tốc $10$ m/s thì người lái đạp phanh; từ thời điểm đó, ô tô chuyển động chậm dần đều với vận tốc $v \left(t\right) = -5t+10$ (m/s), trong đó $t$ là khoảng thời gian tính bằng giây, kể từ lúc bắt đầu đạp phanh. Hỏi từ lúc đạp phanh đến khi dừng hẳn, ô tô còn di chuyển bao nhiêu mét?
	\shortans{$10$}
	\loigiai{
		Xét phương trình $-5t+10=0 \Leftrightarrow t=2$. Do vậy, kể từ lúc người lái đạp phanh thì sau $2s$ ô tô dừng hẳn.\\
		Quãng đường ô tô đi được kể từ lúc người lái đạp phanh đến khi ô tô dừng hẳn là:
		$$s = \displaystyle\int\limits_0^2 \left(-5t+10\right) \mathrm{\,d}t = \left(-\dfrac{5}{2}t^2 + 10t \right) \Big|_0^2 = 10 (\text{m}).$$
	}
\end{ex}

% \begin{ex}%Cau19D%[2D4V2-6]
% 	Một chất điểm $A$ xuất phát từ $O$, chuyển động thẳng với vận tốc biến thiên theo thời gian bởi quy luật $v \left(t\right) = \dfrac{1}{100}t^2 + \dfrac{13}{30}t$ (m/s), trong đó $t$ (giây) là khoảng thời gian tính từ lúc $A$ bắt đầu chuyển động. Từ trạng thái nghỉ, một chất điểm $B$ cũng xuất phát từ $O$, chuyển động thẳng cùng hướng với $A$ nhưng chậm hơn $10$ giây so với $A$ và có gia tốc bằng $a$ (m/s$^2$ ) ( $a$ là hằng số). Sau khi $B$ xuất phát được $15$ giây thì đuổi kịp $A$. Vận tốc của $B$ tại thời điểm đuổi kịp $A$ bằng bao nhiêu m/s?
% 	\shortans{$25$}
% 	\loigiai{
% 		Ta có $v_{B}(t) = \displaystyle\int a \cdot \mathrm{\,d}t = at + C$, $v_{B} (0) = 0 \Rightarrow C = 0 \Rightarrow v_{B} \left(t\right) = at$.\\
% 		Quãng đường chất điểm $A$ đi được trong $25$ giây là
% 		$$S_{A} = \displaystyle\int\limits_0^{25} \left(\dfrac{1}{100}t^2 + \dfrac{13}{30}t \right) \mathrm{\,d}t = \left(\dfrac{1}{300}t^3 + \dfrac{13}{60}t^2 \right) \Big|_0^{25} = \dfrac{375}{2}.$$
% 		Quãng đường chất điểm $B$ đi được trong $15$ giây là
% 		$$S_{B} = \displaystyle\int\limits_0^{15} at \cdot \mathrm{\,d}t = \dfrac{at^2}{2} \Big|_0^{15} = \dfrac{225a}{2}.$$
% 		Ta có $\dfrac{375}{2} = \dfrac{225a}{2} \Leftrightarrow a = \dfrac{5}{3}$.\\
% 		Vận tốc của $B$ tại thời điểm đuổi kịp $A$ là $v_{B} \left(15\right) = \dfrac{5}{3} \cdot 15 = 25$ (m/s).
% 	}
% \end{ex}

\begin{ex}%Cau20D%[2D4H2-6] 
	Một ô tô chuyển động nhanh dần đều với vận tốc $v \left(t\right) = 7t$ (m/s). Đi được $5$ (s) người lái xe phát hiện chướng ngại vật và phanh gấp, ô tô tiếp tục chuyển động chậm dần đều với gia tốc $a = -35$ (m/s$^2$). Tính quãng đường của ô tô đi được từ lúc bắt đầu chuyển bánh cho đến khi dừng hẳn (đơn vị tính bằng mét)?
	\shortans{$105$}
	\loigiai{
		Quãng đường ô tô đi được trong $5 \left(s\right)$ đầu là $s_1 = \displaystyle\int\limits_0^5 7t \mathrm{\,d}t = 7 \dfrac{t^2}{2} \Big|_0^5 = 87{,}5$ (mét).\\
		Phương trình vận tốc của ô tô khi người lái xe phát hiện chướng ngại vật là $v_2 \left(t\right) = 35-35t$ (m/s). Khi xe dừng lại hẳn thì $v_2 \left(t\right) = 0 \Leftrightarrow 35-35t = 0 \Leftrightarrow t =1$.\\
		Quãng đường ô tô đi được từ khi phanh gấp đến khi dừng lại hẳn là
		$$s_2 = \displaystyle\int\limits_0^1 \left(35-35t \right) \mathrm{\,d}t = \left(35-35t\right) \Big|_0^1 = 17,5 \left(\text{mét}\right).$$
		Vậy quãng đường của ô tô đi được từ lúc bắt đầu chuyển bánh cho đến khi dừng hẳn là
		$$s = s_1 + s_2 = 87,5 + 17,5 = 105 \left(\text{mét}\right).$$ 
	}
\end{ex}

% \begin{ex}%Cau21D%[2D4H2-6] 
% 	\immini{Một người chạy trong thời gian $1$ giờ, vận tốc $v$ (km/h) phụ thuộc vào thời gian $t \left(h\right)$ có đồ thị là một phần parabol với đỉnh $I \left(\dfrac{1}{2};8\right)$ và trục đối xứng song song với trục tung như hình bên. Tính quảng đường $s$ người đó chạy được trong khoảng thời gian $45$ phút, kể từ khi chạy (đơn vị tính bằng km)?
% 	}{
% 		\begin{tikzpicture}[>=stealth, font=\footnotesize, line join=round, line cap=round, thick, smooth, samples=250, scale=0.6,yscale=.5]
% 			% Vẽ 2 trục, điền các số lên trục
% 			\draw[->] (-0.5,0)--(0,0) node[below left]{$O$}--(2,0) node[above]{$t$}; %định dạng trục Ox
% 			\foreach \x in {1}
% 			\draw[shift={(\x,0)},color=black] (0pt,2pt)--(0pt,-2pt) 
% 			node[below] { $\x$};
% 			\draw[->,color=black] (0,-0.5)--(0,9) node[right]{$v$};  %định dạng trục Oy
% 			\foreach \y in {8}
% 			\draw[shift={(0,\y)},color=black] (2pt,0pt) -- (-2pt,0pt) 
% 			node[left] {$\y$};
% 			\clip(-1,-1) rectangle (2,9); %vùng đồ thị
% 			%\draw[gray!50,thin,opacity=.5] (-1,-1) grid (4,10); %ô vuông
% 			%Vẽ đồ thị
% 			\draw[smooth,samples=50,domain=0:1] 
% 			plot(\x,{-32*(\x)^2+32*(\x)});
% 			% Vẽ thêm mấy cái râu ria
% 			\draw[dashed] (1/2,0)--(1/2,8) circle(1.5pt)--(0,8);
% 			%Vẽ dấu chấm tròn 
% 			\fill (0cm,0cm) circle (1.5pt); 
% 		\end{tikzpicture} 
% 	}
% 	\shortans{$4,5$}
% 	\loigiai{
% 		\immini{
% 			Gọi parabol là $\left(P\right) \colon y = ax^2 + bx + c$. Từ hình vẽ ta có $\left(P\right)$ đi qua $O\left(0;0\right)$, $A \left(1;0\right)$ và điểm $I\left(\dfrac{1}{2};8\right)$.\\
% 			Ta có hệ: $\heva{&c=0\\&a+b+c=0\\&\dfrac{a}{4}+\dfrac{b}{2}+c = 8} \Leftrightarrow \heva{&a=-32\\&b=32\\&c=0.}$\\
% 			Suy ra $\left(P\right) \colon y = -32x^2 + 32x$.\\
% 			Vậy quãng đường người đó đi được là $s = \displaystyle\int\limits_0^{\tfrac{3}{4}} \left(-32x^2 + 32x \right) \mathrm{\,d}x = 4{,}5$ (km).
% 		}
% 		{\begin{tikzpicture}[>=stealth, font=\footnotesize, line join=round, line cap=round, thick, smooth, samples=250, scale=0.6]
% 				% Vẽ 2 trục, điền các số lên trục
% 				\draw[->] (-0.5,0)--(0,0) node[below left]{$O$}--(2,0) node[above]{$t$}; %định dạng trục Ox
% 				\foreach \x in {1}
% 				\draw[shift={(\x,0)},color=black] (0pt,2pt)--(0pt,-2pt) 
% 				node[below] { $\x$};
% 				\draw[->,color=black] (0,-0.5)--(0,9) node[right]{$v$};  %định dạng trục Oy
% 				\foreach \y in {8}
% 				\draw[shift={(0,\y)},color=black] (2pt,0pt) -- (-2pt,0pt) 
% 				node[left] {$\y$};
% 				\clip(-1,-1) rectangle (2,9); %vùng đồ thị
% 				%\draw[gray!50,thin,opacity=.5] (-1,-1) grid (4,10); %ô vuông
% 				%Vẽ đồ thị
% 				\draw[smooth,samples=50,domain=0:1] 
% 				plot(\x,{-32*(\x)^2+32*(\x)});
% 				% Vẽ thêm mấy cái râu ria
% 				\draw[dashed] (1/2,0)--(1/2,8) circle(1.5pt)--(0,8);
% 				%Vẽ dấu chấm tròn 
% 				\fill (0cm,0cm) circle (1.5pt); 
% 		\end{tikzpicture} }
% 	}
% \end{ex}

% \begin{ex}%Cau22D%[2D4V2-6]
% 	Một vật chuyển động trong $4$ giờ với vận tốc $v$ (km/h) phụ thuộc thời gian $t$ (h) có đồ thị của vận tốc như hình bên. Trong khoảng thời gian $3$ giờ kể từ khi bắt đầu chuyển động, đồ thị đó là một phần của đường parabol có đỉnh $I \left(2;9\right)$ với trục đối xứng song song với trục tung, khoảng thời gian còn lại đồ thị là một đoạn thẳng song song với trục hoành. Tính quãng đường $s$ mà vật di chuyển được trong $4$ giờ đó (đơn vị tính bằng km).
% 	\begin{center}
% 		\begin{tikzpicture}[>=stealth,scale=0.45]
% 			% Vẽ 2 trục, điền các số lên trục
% 			\draw[->] (-0.5,0)--(0,0) node[below left]{$O$}--(5,0) node[above]{$t$};
% 			\foreach \x in {2,3,4}
% 			\draw[shift={(\x,0)},color=black] (0pt,2pt)--(0pt,-2pt) 
% 			node[below] { $\x$};
% 			\draw[->,color=black] (0,-0.5)--(0,10) node[right]{$v$};
% 			\foreach \y in {9}
% 			\draw[shift={(0,\y)},color=black] (2pt,0pt) -- (-2pt,0pt) 
% 			node[left] {$\y$};
% 			\clip(-1,-1) rectangle (5,10); %vùng đồ thị
% 			%\draw[gray!50,thin,opacity=.5] (-1,-1) grid (4,10); %ô vuông
% 			%Vẽ đồ thị
% 			\draw[smooth,samples=100,domain=0:3, font=\footnotesize, line join=round, line cap=round, thick, smooth] 
% 			plot(\x,{(-9/4)*(\x)^2+9*(\x)});
% 			\draw[smooth,samples=100, font=\footnotesize, line join=round, line cap=round, thick, smooth,domain=3:4] 
% 			plot(\x,{27/4});
% 			% Vẽ thêm mấy cái râu ria
% 			\draw[dashed] (3,0)--(3,27/4) circle(1.5pt);  \draw[dashed] (2,0)--(2,9) circle(1.5pt) node[above]{$I$}--(0,9) circle(1.5pt); \draw[dashed] (4,0)--(4,27/4) circle(1.5pt);
% 			%Vẽ dấu chấm tròn 
% 			\fill (0cm,0cm) circle (1.5pt); 
% 		\end{tikzpicture} 
% 	\end{center}
% 	\shortans{$27$}
% 	\loigiai{
% 		Gọi $\left(P\right) \colon y = ax^2+bx+c$.\\
% 		Vì $\left(P\right)$ qua $O\left(0;0\right)$ và có đỉnh $I\left(2;9\right)$ nên dễ tìm được phương trình là $y = \dfrac{-9}{4}x^2 + 9x$.\\
% 		Ngoài ra tại $x=3$ ta có $y = \dfrac{27}{4}$.\\
% 		Vậy quãng đường cần tìm là: $S = \displaystyle\int\limits_0^3 \left(\dfrac{-9}{4}x^2 +9x \right) \mathrm{\,d}x + \displaystyle\int\limits_3^4 \dfrac{27}{4} \mathrm{\,d}x = 27$ (km).
% 	}
% \end{ex}

% \begin{ex}%Cau23D%[2D4V2-6]
% 	Một vật chuyển động trong $6$ giờ với vận tốc $v$ (km/h) phụ thuộc vào thời gian $t$ (h) có đồ thị như hình bên dưới. Trong khoảng thời gian $2$ giờ từ khi bắt đầu chuyển động, đồ thị là một phần đường Parabol có đỉnh $I\left(3;9\right)$ và có trục đối xứng song song với trục tung. Khoảng thời gian còn lại, đồ thị vận tốc là một đường thẳng có hệ số góc bằng $\dfrac{1}{4}$. Tính quảng đường $s$ mà vật di chuyển được trong $6$ giờ? (đơn vị tính bằng km, làm tròn đến chữ số thập phân thứ nhất).
% 	\begin{center}
% 		\begin{tikzpicture}[>=stealth,scale=0.5]
% 			% Vẽ 2 trục, điền các số lên trục
% 			\draw[->] (-0.5,0)--(0,0) node[below left]{$O$}--(7,0) node[above]{$t$}; %định dạng trục Ox
% 			\foreach \x in {2,3,6}
% 			\draw[shift={(\x,0)},color=black] (0pt,2pt)--(0pt,-2pt) 
% 			node[below] { $\x$};
% 			\draw[->,color=black] (0,-0.5)--(0,10) node[right]{$v$};  %định dạng trục Oy
% 			\foreach \y in {8,9}
% 			\draw[shift={(0,\y)},color=black] (2pt,0pt) -- (-2pt,0pt) 
% 			node[left] {$\y$};
% 			\clip(-1,-1) rectangle (7,10); %vùng đồ thị
% 			%\draw[gray!50,thin,opacity=.5] (-1,-1) grid (4,10); %ô vuông
% 			%Vẽ đồ thị
% 			\draw[smooth,samples=100,domain=0:2,font=\footnotesize, line join=round, line cap=round, thick] 
% 			plot(\x,{(-1)*(\x)^2+6*(\x)});
% 			\draw[smooth,domain=2:6, line join=round, line cap=round,dashed] 
% 			plot(\x,{(-1)*(\x)^2+6*(\x)});
% 			\draw[smooth,samples=100,domain=2:6,font=\footnotesize, line join=round, line cap=round, thick] 
% 			plot(\x,{(1/4)*(\x)+15/2});
% 			% Vẽ thêm mấy cái râu ria
% 			\draw[dashed] (3,0)--(3,9) circle(1.5pt) node[above]{$I$}--(0,9) circle(1.5pt); 
% 			\draw[dashed] (2,0)--(2,8) circle(1.5pt) --(0,8);
% 			\draw[dashed] (6,0)--(6,9) circle(1.5pt) --(0,9);
% 			%Vẽ dấu chấm tròn 
% 			\fill (0cm,0cm) circle (1.5pt); 
% 		\end{tikzpicture}
% 	\end{center}
% 	\shortans{$43{,}3$}
% 	\loigiai{
% 		Vì Parabol đi qua $O\left(0;0\right)$ và có tọa độ đỉnh $I\left(3;9\right)$ nên thiết lập được phương trình Parabol là $\left(P\right) \colon y = v\left(t\right) = -t^2+6t$; $\forall t \in \left[0;2\right]$.\\
% 		Sau $2$ giờ đầu thì hàm vận tốc có dạng là hàm bậc nhất $y = \dfrac{1}{4}t + m$, dựa trên đồ thị ta thấy đi qua điểm có tọa độ $\left(6;9\right)$ nên thế vào hàm số và tìm được $m = \dfrac{15}{2}$.\\
% 		Nên hàm vận tốc từ giờ thứ $2$ đến giờ thứ $6$ là: $y = \dfrac{1}{4}t + \dfrac{15}{2},\forall t \in \left[2;6\right]$.\\
% 		Quảng đường vật đi được bằng tổng đoạn đường $2$ giờ đầu và đoạn đường $4$ giờ sau.
% 		$$S = S_1 +S_2 = \displaystyle\int\limits_0^2 \left(-t^2+6t\right) \mathrm{\,d}t + \displaystyle\int\limits_2^6 \left(\dfrac{1}{4}t+\dfrac{15}{2} \right) \mathrm{\,d}t = \dfrac{130}{3} \approx 43{,}3 \left(\text{km}\right).$$
% 	}
% \end{ex}

% \begin{ex}%Cau24D%[2D4H2-6]
% 	\immini{Một người chạy trong thời gian $1$ giờ, với vận tốc $v$ (km/h) phụ thuộc vào thời gian $t$ (h) có đồ thị là một phần của parabol có đỉnh $I \left(\dfrac{1}{2};8\right)$ và trục đối xứng song song với trục tung như hình vẽ. Tính quãng đường $S$ người đó chạy được trong thời gian $45$ phút, kể từ khi bắt đầu chạy (đơn vị tính bằng km).
% 	}{
% 		\begin{tikzpicture}[>=stealth, scale=0.6, yscale=.5]
% 			% Vẽ 2 trục, điền các số lên trục
% 			\draw[->] (-0.5,0)--(0,0) node[below left]{$O$}--(2,0) node[above]{$t$}; %định dạng trục Ox
% 			\foreach \x in {1}
% 			\draw[shift={(\x,0)},color=black] (0pt,2pt)--(0pt,-2pt) 
% 			node[below] { $\x$};
% 			\draw[->,color=black] (0,-0.5)--(0,9) node[right]{$v$};  %định dạng trục Oy
% 			\foreach \y in {8}
% 			\draw[shift={(0,\y)},color=black] (2pt,0pt) -- (-2pt,0pt) 
% 			node[left] {$\y$};
% 			\clip(-1,-1) rectangle (2,9); %vùng đồ thị
% 			\draw[smooth,samples=50,domain=0:1, font=\footnotesize, line join=round, line cap=round, thick] 
% 			plot(\x,{-32*(\x)^2+32*(\x)});
% 			\draw[dashed] (1/2,0)--(1/2,8) circle(1.5pt)--(0,8);
% 			\fill (0cm,0cm) circle (1.5pt); 
% 		\end{tikzpicture}
% 	}
% 	\shortans{$4{,}5$}
% 	\loigiai{
% 		Trước hết ta tìm công thức biểu thị vận tốc theo thời gian, giả sử $v\left(t\right) = at^2+bt+c$.\\  .
% 		Khi đó dựa vào hình vẽ ta có hệ phương trình\\
% 		$$\heva{&c=0\\&a\left(\dfrac{1}{2}\right)^2+b\left(\dfrac{1}{2}\right)+c =8\\&a+b+c=0} \Leftrightarrow \heva{&a=-32\\&b=32\\&c=0.}$$
% 		Do đó quãng đường người đó đi được sau $45$ phút là $S = \displaystyle\int\limits_0^{\tfrac{45}{60}} \left(32t-32t^2\right) \mathrm{\,d}t = 4{,}5$ (km).
% 	}
% \end{ex}

\begin{ex}%Cau25D%[2D4H2-6]
	Một vật chuyển động trong $4$ giờ với vận tốc $v$ (km/h) phụ thuộc thời gian $t$ (h) có đồ thị là một phần của đường parabol có đỉnh $I\left(1;1\right)$ và trục đối xứng song song với trục tung như hình bên. Tính quãng đường $s$ mà vật di chuyển được trong $4$ giờ kể từ lúc xuất phát (làm tròn đến chữ số thập phân thứ nhất).
	\begin{center}
		\begin{tikzpicture}[>=stealth,scale=0.5,yscale=.7]
			% Vẽ 2 trục, điền các số lên trục
			\draw[->] (-0.5,0)--(0,0) node[below left]{$O$}--(5,0) node[above]{$t$};
			\foreach \x in {1,4}
			\draw[shift={(\x,0)},color=black] (0pt,2pt)--(0pt,-2pt) 
			node[below] { $\x$};
			\draw[->,color=black] (0,-0.5)--(0,11) node[right]{$v$};
			\foreach \y in {1,2,10}
			\draw[shift={(0,\y)},color=black] (2pt,0pt) -- (-2pt,0pt) 
			node[left] {$\y$};
			\clip(-1,-1) rectangle (5,11); %vùng đồ thị
			\draw[smooth,samples=100,domain=0:10, font=\footnotesize, line join=round, line cap=round, thick] 
			plot(\x,{(\x)^2-2*(\x)+2});
			% Vẽ thêm mấy cái râu ria
			\draw[dashed] (4,0)--(4,10) circle(1.5pt)--(0,10);  \draw[dashed] (1,0)--(1,1) circle(1.5pt) node[above]{$I$}--(0,1);
			%Vẽ dấu chấm tròn 
			\fill (0cm,0cm) circle (1.5pt); 
		\end{tikzpicture} 
	\end{center}
	\shortans{$13{,}3$}
	\loigiai{Hàm biểu diễn vận tốc có dạng $v\left(t\right) = at^2+bt+c$. Dựa vào đồ thị ta có
		$$\heva{&c=2\\&-\dfrac{b}{2a}=1\\&a+b+c=1} \Leftrightarrow \heva{&a=1\\&b=-2\\&c=2} \Leftrightarrow v\left(t\right) = t^2 -2t+2.$$
		Với $t=4 \Rightarrow v\left(4\right) = 10$ (thõa mãn).\\
		Từ đó $s = \displaystyle\int\limits_0^4 \left(t^2-2t+2\right) \mathrm{\,d}t = \dfrac{40}{3} \approx 13{,}3$ (km).
	}
\end{ex}

% \begin{ex}%Cau26D%[2D4V2-6]
% 	Chất điểm chuyển động theo quy luật vận tốc $v\left(t\right)$ (m/s) có dạng đường Parapol khi $0 \leq t\leq 5$ (s) và $v\left(t\right)$ có dạng đường thẳng khi $5 \leq t \leq 10$ (s). Cho đỉnh Parapol là $I\left(2;3\right)$. Hỏi quãng đường đi được chất điểm trong thời gian $0 \leq t \leq 10$ (s) là bao nhiêu mét? (làm tròn đến hàng đơn vị)
% 	\begin{center}
% 		\begin{tikzpicture}[>=stealth,scale=0.3,yscale=.7]
% 			% Vẽ 2 trục, điền các số lên trục
% 			\draw[->] (-0.5,0)--(0,0) node[below left]{$O$}--(13,0) node[above]{$t$};
% 			\foreach \x in {2,5,10}
% 			\draw[shift={(\x,0)},color=black] (0pt,2pt)--(0pt,-2pt) 
% 			node[below] { $\x$};
% 			\draw[->,color=black] (0,-0.5)--(0,22) node[right]{$v$};
% 			\foreach \y in {3,11}
% 			\draw[shift={(0,\y)},color=black] (2pt,0pt) -- (-2pt,0pt) 
% 			node[left] {$\y$};
% 			\clip(-1,-1) rectangle (11,23); %vùng đồ thị
% 			\draw[smooth,samples=100,domain=0:5, font=\footnotesize, line join=round, line cap=round, thick] 
% 			plot(\x,{2*(\x)^2-8*(\x)+11});
% 			\draw[smooth,samples=100,domain=5:10, font=\footnotesize, line join=round, line cap=round, thick] 
% 			plot(\x,{(-21/5)*(\x)+42});
% 			% Vẽ thêm mấy cái râu ria
% 			\draw[dashed] (5,0)--(5,21) circle(1.5pt);  \draw[dashed] (2,0)--(2,3) circle(1.5pt)--(0,3);
% 			%Vẽ dấu chấm tròn 
% 			\fill (0cm,0cm) circle (1.5pt); 
% 		\end{tikzpicture}
% 	\end{center}
% 	\shortans{$91$}
% 	\loigiai{
% 		Gọi Parapol $\left(P\right) \colon y = ax^2+bx+c$ khi $0 \leq t \leq 5 \left(s\right)$.\\  
% 		Do $\left(P\right) \colon y = ax^2+bx+c$ đi qua $I\left(3;2\right)$; $A \left(0;11\right)$ nên
% 		$$\heva{&4a+2b+c=3\\&c=11\\&4a+b=0} \Rightarrow \heva{&a=2\\&b=-8\\&c=11.}$$
% 		Khi đó quãng đường vật di chuyển trong khoảng thời gian từ $0 \leq t \leq 5$ (s) là
% 		$$S_1 = \displaystyle\int\limits_0^5 \left(2x^2 - 8x +11\right) \mathrm{\,d}x = \dfrac{115}{3} \left(\text{m}\right).$$
% 		Ta có $f\left(5\right) = 21$.\\
% 		Gọi $d \colon y = ax+b$ khi $5 \leq t \leq 10$ (s), do $d$ đi qua điểm $B\left(5;21\right)$ và $C\left(10;0\right)$ nên
% 		$$\heva{&5a+b=11\\&10a+b=0} \Rightarrow \heva{&a=-\dfrac{21}{5}\\&b=42.}$$
% 		Khi đó quãng đường vật di chuyển trong khoảng thời gian từ $5 \leq t \leq 10$ (s) là:
% 		$$S_2 = \displaystyle\int\limits_5^{10} \left(-\dfrac{21}{5}x+42\right) \mathrm{\,d}x = \dfrac{105}{2} \left(\text{m}\right).$$
% 		Quãng đường đi được chất điểm trong thời gian  $0 \leq t \leq 10$ (s) là:
% 		$$S = \dfrac{115}{3}+\dfrac{105}{2}= \dfrac{545}{6} \approx 91 \left(\text{m}\right).$$
% 	}
% \end{ex}
\Closesolutionfile{ans}
\indapan{6}{ans/ans-2C4B2CD3-KQ}
% \subsection{Tích phân hàm ẩn biến đổi phức tạp}
% \begin{tomtat}
% 	Cần nhớ các công thức đạo hàm của hàm hợp
% 	\begin{itemize}
% 		\item $\displaystyle\int f'(x)\mathrm{\,d}x=f(x)+C$
% 		\item $f'(x)\cdot g(x)+f(x)\cdot g'(x)=\left[f(x)\cdot g(x)\right]'$
% 		\item $\dfrac{f'(x)\cdot g(x)-f(x)\cdot g'(x)}{g^2(x)}=\left[\dfrac{f(x)}{g(x)}\right]'$
% 		\item $\dfrac{f'(x)}{f(x)}=\left[\ln \left(f(x)\right)\right]'$
% 		\item $ -\dfrac{f'(x)}{f^2(x)}=\left[\dfrac 1{f(x)}\right]'$
% 		\item $-\dfrac{f'(x)}{f^n(x)}=\left[\dfrac 1{(n-1)[f(x)]^{n-1}}\right]'$
% 		\item $n\cdot f'(x)\cdot \left(f(x)\right)^{n-1}=\left[f(x)^n\right]'$
% 		\item $\dfrac{f'(x)}{\sqrt{f(x)}}=\left[2\sqrt{f(x)}\right]'$
% 	\end{itemize}	 
% \end{tomtat}
% \begin{dang}{.}
% \begin{enumerate}
		
% \item[1.]  Điều kiện hàm ẩn có dạng$\colon $ $\hoac{&f'(x)=g(x) \cdot h\left[f(x)\right] \\ & f'(x) \cdot h[f(x)]=g(x).}$

% Phương pháp giải$\colon $
% 	\begin{itemize}
% 	\item $\dfrac{f'(x)}{h[f(x)]}=g(x) \Leftrightarrow \displaystyle\int \dfrac{f'(x)}{h[f(x)]} \mathrm{\,d}x=\displaystyle\int g(x) \mathrm{\,d}x \Leftrightarrow \displaystyle\int \dfrac{d[f(x)]}{h[f(x)]}=\displaystyle\int g(x)\mathrm{\,d}x.$
% 	\item $f'(x) h[f(x)]=g(x) \Leftrightarrow \displaystyle\int f'(x) h[f(x)] \mathrm{\,d}x=\displaystyle\int g(x)\mathrm{\,d}x$\\$ \Leftrightarrow \displaystyle\int h[f(x)]  d\left[f'(x)\right]=\displaystyle\int g(x).$
% 	\end{itemize}
% Chú ý$\colon$ Ngoài việc nguyên hàm hai vế, ta có thể lấy tích phân hai vế (tùy câu hỏi của bài toán).\\
% \item[2.] Điều kiện hàm ẩn có dạng$\colon $ $\hoac{&f'(x)+p(x) \cdot f(x)=0 \\ & f'(x)+p(x) \cdot[f(x)]^n=0.}$

% Phương pháp giải$\colon $
% \begin{itemize}
% 	\item $f'(x)+p(x) \cdot f(x)=0.$\\
% Chia hai vế với $f(x)$ ta đựơc $\dfrac{f'(x)}{f(x)}+p(x)=0 \Leftrightarrow \dfrac{f'(x)}{f(x)}=-p(x).$\\
% Suy ra $\displaystyle\int \dfrac{f'(x)}{f(x)} \mathrm{d} x=-\displaystyle\int p(x) \mathrm{d} x \Leftrightarrow \ln |f(x)|=-\displaystyle\int p(x) \mathrm{d} x$.\\
% Từ đây ta dễ dàng tính được $f(x).$
% 	\item $f'(x)+p(x) \cdot[f(x)]^n=0$\\
% Chia hai vế với $[f(x)]^n$ ta được $\dfrac{f'(x)}{[f(x)]^n}+p(x)=0 \Leftrightarrow \dfrac{f'(x)}{[f(x)]^n}=-p(x).$
% \end{itemize}
% \end{enumerate}
% \end{dang}
\setcounter{ex}{0}
\Opensolutionfile{ans}[ans/ans-2-B1]
\TN  
\begin{ex}%[2D4C2-2]
	Cho hàm số $f(x)$ nhận giá trị không âm và có đạo hàm liên tục trên $\mathbb{R}$ thỏa mãn $f'(x)=(2x+1){{\left[f(x) \right]}^2},\forall x\in \mathbb{R}$ và $f(0)=-1$. Tính tích phân $\displaystyle\int\limits_0^1\left(x^3-1\right)f(x)\mathrm{\,d}x$.
	\choice
	{$1$}
	{$\dfrac{2}{3}$}
	{\True $\dfrac{1}{2}$}
	{$\dfrac{3}{2}$}
	\loigiai{
		Ta có
		$$
		\begin{aligned}
			 &&f'(x)=(2x+1)[f(x)]^2,\forall x\in\mathbb{R}\\
			&\Rightarrow&\dfrac{-f'(x)}{[f(x)]^2}=-(2x+1),\forall x\in\mathbb{R}\\ 			
			&\Rightarrow&\left[\dfrac 1{f(x)}\right]'=-(2x+1),\forall x\in\mathbb{R}.
		\end{aligned}
		$$
		Suy ra $\dfrac{1}{f(x)}=-\displaystyle\int{\left(2x+1\right)}\mathrm{\,d}x=-x^2-x+C\Rightarrow f(x)=\dfrac{1}{-x^2-x+C}$.\\
		Vì  $f(0)=-1\Rightarrow C=-1$.\\
		Suy ra $f(x)=-\dfrac{1}{x^2+x+1}$.\\
		$\displaystyle\int\limits_0^1\left(x^3-1\right)f(x)\mathrm{\,d}x=-\displaystyle\int\limits_0^1\left(x^3-1\right)\left(\dfrac{1}{x^2+x+1}\right)\mathrm{\,d}x=\displaystyle\int\limits_0^1\left(1-x\right)\mathrm{\,d}x$\\
		$=\left.\left(x-\dfrac{x^2}{2}\right)\right|_0^1=\dfrac{1}{2}$.}
\end{ex}

\begin{ex}%[2D4C2-2]
	Cho hàm số $f(x)\ne 0$, liên tục trên đoạn $\left[1;2\right]$ và thỏa mãn $f(1)=\dfrac{1}{3}$; $\linebreak x^2\cdot f'(x)=f^2(x)$ với $\forall x\in\left[1;2\right]$. Tính tích phân $I=\displaystyle\int\limits_1^2\left(2x+1\right)^2f(x)\mathrm{\,d}x$.
	\choice
	{$I=\dfrac{7}{6}$}
	{$I=\dfrac{5}{6}$}
	{\True $I=\dfrac{37}{6}$}
	{$I=\dfrac{1}{6}$}
	\loigiai{
		Ta có
		$$
		\begin{aligned}
		&x^2\cdot f'(x)=f^2(x)\\ 
		\Rightarrow&\dfrac{f'(x)}{f^2(x)}=\dfrac 1{x^2}\\ 
		\Rightarrow&{\left[-\dfrac 1{f(x)}\right]'}=\dfrac 1{x^2}\\ 
		\Rightarrow&-\dfrac 1{f(x)}=\displaystyle\int{\dfrac 1{x^2}}\mathrm{\,d}x\\
		 \Rightarrow&\dfrac 1{f(x)}=-\displaystyle\int{\dfrac 1{x^2}}\mathrm{\,d}x\\
		  \Rightarrow&\dfrac 1{f(x)}=\dfrac 1 x+C.\\ 
		\end{aligned}
		$$
		Mà $f(1)=\dfrac{1}{3}$ $\Rightarrow 3=1+C\Rightarrow C=2.$\\
		Do đó $\dfrac{1}{f(x)}=\dfrac{1}{x}+2 \Rightarrow f(x)=\dfrac{x}{2x+1}.$\\
		Vậy $I=\displaystyle\int\limits_1^2\left(2x+1\right)^2f(x)\mathrm{\,d}x=\displaystyle\int\limits_1^2\left(2x+1\right)^2\dfrac{x}{2x+1}\mathrm{\,d}x=\displaystyle\int\limits_1^2\left(2x^2+x\right)\mathrm{\,d}x=\dfrac{37}{6}$.}
\end{ex}

\begin{ex}%[2D4C2-2]
	Cho hàm số $f(x)$ có đạo hàm trên $\mathbb{R}$ thỏa mãn $3f'(x)\cdot \mathrm{e}^{f^3(x)}-\dfrac{2x}{f^2(x)}=0$ với $\forall x\in\mathbb{R}$. Biết $f(1)=0$, tính tích phân $I=\displaystyle\int\limits_0^{2024}{\dfrac{1}{\sqrt[3]{2\ln x}}\cdot f(x){\mathrm{\,d}}x}$.
	\choice
	{$1$}
	{$\dfrac{1}{2024}$}
	{\True $2024$}
	{$0$}
	\loigiai{
		Ta có
		$$
		\begin{aligned}
		&3f'(x)\cdot\mathrm{e}^{f^3(x)}-\dfrac{2x}{f^2(x)}=0\\ 
		\Rightarrow& 3f^2(x)\cdot f'(x)\cdot\mathrm{e}^{f^3(x)}=2x \\
		\Rightarrow&\left[\mathrm{e}^{f^3(x)}\right]'=2x \\
		\Rightarrow&\mathrm{e}^{f^3(x)}=\displaystyle\int{2x}\mathrm{\,d}x \\
		\Rightarrow&\mathrm{e}^{f^3(x)}=x^2+C.\\ 
		\end{aligned}
		$$
		Mặt khác $f(1)=0\Rightarrow\mathrm{e}^{f^3(1)}=1+C\Rightarrow C=0.$\\
		Suy ra $\mathrm{e}^{f^3(x)}=x^2\Rightarrow{f^3}(x)=\ln {x^2}\Rightarrow f(x)=\sqrt[3]{2\ln x}$.\\
		Vậy $I=\displaystyle\int\limits_0^{2024}\dfrac 1{\sqrt[3]{2\ln x}}\cdot f(x)\mathrm{\,d}x=\displaystyle\int\limits_0^{2024}\dfrac 1{\sqrt[3]{2\ln x}}\cdot \sqrt[3]{2\ln x}\mathrm{\,d}x=\displaystyle\int\limits_0^{2024}\mathrm{\,d}x=2024$}
\end{ex}

\begin{ex}%[2D4C2-2]
	Cho hàm số $f(x)$ đồng biến, có đạo hàm trên đoạn $\left[1;4\right]$ và thoả mãn $x+2x\cdot f(x)=\left[f'(x)\right]^2$ với $\forall x\in\left[1;4\right]$. Biết $f(1)=\dfrac{3}{2}$, tính $I=\displaystyle\int\limits_1^4f(x)\mathrm{\,d}x$.
	\choice
	{\True $I=\dfrac{1186}{45}$}
	{$I=\dfrac{1186}{9}$}
	{$I=\dfrac{1186}{5}$}
	{$I=\dfrac{1186}{41}$}
	\loigiai{
		Do $f(x)$ đồng biến trên đoạn $\left[1;4\right]$ $\Rightarrow f'(x)\ge 0,\forall x\in\left[1;4\right].$\\
		Ta có  $x+2x \cdot f(x)=\left[f'(x)\right]^2
		\Leftrightarrow x\left(1+2\cdot f(x)\right)=\left[f'(x)\right]^2$, \\Do $x\in\left[1;4\right]$ và $f'(x)\ge 0,\forall x\in\left[1;4\right]$
		$\Rightarrow f(x) >\dfrac{-1}{2}$ và
		$$
		\begin{aligned}
			&f'(x)=\sqrt x \cdot \sqrt{1+2f(x)}\\
			\Leftrightarrow&\dfrac{f'(x)}{\sqrt{1+2f(x)}}=\sqrt x\\
			\Leftrightarrow&\left(\sqrt{1+2f(x)}\right)'=\sqrt x \\
			\Leftrightarrow&\sqrt{1+2f(x)}=\displaystyle\int{\sqrt x}\mathrm{\,d}x\\
			\Leftrightarrow&\sqrt{1+2f(x)}=\dfrac{2}{3}x\sqrt x+C.
		\end{aligned}
		$$
		Vì $f(1)=\dfrac{3}{2}\Rightarrow\sqrt{1+2\cdot\dfrac{3}{2}}=\dfrac{2}{3}+C\Leftrightarrow C=\dfrac{4}{3}$.\\
		Suy ra
		$$
		\begin{aligned}
			&\sqrt{1+2f(x)}=\dfrac{2}{3}x\sqrt x+\dfrac{4}{3}\\
			\Leftrightarrow & 1+2f(x)=\left(\dfrac{2}{3}x\sqrt x+\dfrac{4}{3}\right)^2\\
			\Leftrightarrow & f(x)=\dfrac{2}{9}{x^3}+\dfrac{8}{9}{x^{\dfrac{3}{2}}}+\dfrac{7}{18}.
		\end{aligned}
		$$
		Khi đó\\ $I=\displaystyle\int\limits_1^4f(x)\mathrm{\,d}x=\displaystyle\int\limits_1^4\left(\dfrac{2}{9}{x^3}+\dfrac{8}{9}{x^{\tfrac{3}{2}}}+\dfrac{7}{18}\right)\mathrm{\,d}x=\left.\left(\dfrac{1}{18}{x^4}+\dfrac{16}{45}{x^{\tfrac{5}{2}}}+\dfrac{7}{18}x\right)\right|_1^4=\dfrac{1186}{45}$.}
\end{ex}

\begin{ex}%[2D4C2-2]
	Cho hàm số $f(x)$ nhận giá trị dương và thỏa mãn $f(0)=1$, $\left[f'(x)\right]^3=\mathrm{e}^x\left[f(x)\right]^2,\forall x\in\mathbb{R}$.
	Tính $I=\displaystyle\int\limits_1^2f(x)\mathrm{\,d}x$.
	\choice
	{$I=\mathrm{e}^2+1$}
	{$I=\mathrm{e}-1$}
	{\True $I=\mathrm{e}^2-e$}
	{$I=\mathrm{e}$}
	\loigiai{
		Ta có
		$$
		\begin{aligned}
			&\left[f'(x)\right]^3=\mathrm{e}^x\left[f(x)\right]^2\\
		\Leftrightarrow& f'(x)=\sqrt[3]{\mathrm{e}^x}\cdot\sqrt[3]{\left[f(x)\right]^2}\\ 
		\Leftrightarrow&\dfrac{f'(x)}{\sqrt[3]{\left[f(x)\right]^2}}=\sqrt[3]{\mathrm{e}^x}\\
		 \Leftrightarrow&\dfrac{f'(x)}{\sqrt[3]{\left[f(x)\right]^2}}=\sqrt[3]{\mathrm{e}^x}\\
		  \Leftrightarrow &f'(x)\cdot \left[f(x)\right]^{-\tfrac 23}=\sqrt[3]{\mathrm{e}^x}\\ 
		  \Leftrightarrow& 3\left[\left(f(x)\right)^{\tfrac 13}\right]'=\sqrt[3]{\mathrm{e}^x}\\ 
		  \Leftrightarrow&\left[\left(f(x)\right)^{\tfrac 13}\right]'=\dfrac 13\sqrt[3]{\mathrm{e}^x}\\ 
		  \Leftrightarrow&\left[f(x)\right]^{\tfrac 13}=\dfrac 13\displaystyle\int{\sqrt[3]{\mathrm{e}^x}}\mathrm{\,d}x \\
		  \Leftrightarrow&\left[f(x)\right]^{\tfrac 13}=\mathrm{e}^{\tfrac x3}+C.
		\end{aligned}
		$$	
		Mà $f(0)=1\Rightarrow 1=1+C\Rightarrow C=0$.\\
		Do đó $\left[f(x)\right]^{\tfrac{1}{3}}=\mathrm{e}^{\tfrac{x}{3}}\Rightarrow f(x)=\mathrm{e}^x$.\\
		Vậy $I=\displaystyle\int\limits_1^2\mathrm{e}^x\mathrm{\,d}x=\mathrm{e}^2-\mathrm{e}$.}
\end{ex}

\begin{ex}%[2D4C2-2]
	Cho hàm số $y=f(x)$ có đạo hàm liên tục trên $\mathbb{R}$ và thỏa mãn điều kiện ${{x}^6}{{\left[f'(x) \right]}^3}+27{{\left[f(x)-1 \right]}^4}=0\,,\,\forall x\in \mathbb{R}$ và $f(1)=0$. Tính $I=\displaystyle\int\limits_2^3f(x)\mathrm{\,d}x$.
	\choice
	{$I=\dfrac{31}{2}$}
	{$I=-\dfrac{31}{2}$}
	{$I=\dfrac{61}{4}$}
	{\True $I=-\dfrac{61}{4}$}
	\loigiai{
		Ta có
		$$
		\begin{aligned}
		&x^6\left[f'(x)\right]^3+27\left[f(x)-1\right]^4=0\\ 
		\Leftrightarrow&{x^6}{\left[f'(x)\right]^3}=-27\left[f(x)-1\right]^4\\ 
		\Leftrightarrow&\dfrac{\left[f'(x)\right]^3}{\left[f(x)-1\right]^4}=-\dfrac{27}{x^6}\\ 
		\Leftrightarrow&\dfrac{\left[f'(x)\right]^3}{\left[f(x)-1\right]^3\left[f(x)-1\right]}=-\dfrac{27}{x^6}\\ 
		\Leftrightarrow&\dfrac{f'(x)}{\left[f(x)-1\right]\sqrt[3]{f(x)-1}}=-\dfrac 3{x^2}\\
		 \Leftrightarrow&\dfrac{f'(x)}{-3\left[f(x)-1\right]\sqrt[3]{f(x)-1}}=\dfrac 1{x^2}\\ 
		 \Leftrightarrow&{\left[\dfrac 1{\sqrt[3]{f(x)-1}}\right]'}=\dfrac 1{x^2}.\\ 
		\end{aligned}
		$$
		Do đó $\displaystyle\int{\left[\dfrac{1}{\sqrt[3]{f(x)-1}}\right]'}\mathrm{\,d}x=\displaystyle\int{\dfrac{1}{x^2}\mathrm{\,d}x}=-\dfrac{1}{x}+C.$\\
		Suy ra $\dfrac{1}{\sqrt[3]{f(x)-1}}=-\dfrac{1}{x}+C$.\\
		Mà  $f(1)=0\Rightarrow C=0$.\\
		Nên  $f(x)=1-x^3$.\\
		Khi đó $I=\displaystyle\int\limits_2^3f(x)\mathrm{\,d}x=\displaystyle\int\limits_2^3(1-x^3)\mathrm{\,d}x=-\dfrac{61}{4}$.}
\end{ex}
\begin{ex}%[2D4C2-4]
	Cho hàm số $f(x) > 0$ và thỏa mãn $\left[f'(x)\right]^2+f(x)\cdot f''(x)=\mathrm{e}^x$, $\forall x\in \mathbb{R}$ và $f(0)=f'(0)=1$. Tính $I=\displaystyle\int\limits_1^2 f(x) \mathrm{\,d}x$.
	\choice
	{$I=2\sqrt{\mathrm{e}}$}
	{$I=\mathrm{e}-\sqrt{\mathrm{e}}$}
	{\True $I=2\mathrm{e}-2\sqrt{\mathrm{e}}$}
	{$I=2\mathrm{e}+2\sqrt{\mathrm{e}}$}
	\loigiai{
		Ta có
		\allowdisplaybreaks
		\begin{eqnarray*}
			&&\left[f'(x)\right]^2+f(x)\cdot f''(x)=\mathrm{e}^x\\
			&\Leftrightarrow& \left[f(x)\cdot f'(x)\right]'=\mathrm{e}^x\\
			&\Rightarrow& f(x)\cdot f'(x)=\displaystyle\int\limits_{\mathrm{e}}^x \mathrm{e}^x \mathrm{\,d}x\\
			&\Rightarrow& f(x)\cdot f'(x)=\mathrm{e}^x+C.
		\end{eqnarray*}
		Từ $f(0)=f'(0)=1$ ta suy ra $C=0$.\\
		Vậy $f(x)\cdot f'(x)=\mathrm{e}^x$\\
		Tiếp đến có
		\allowdisplaybreaks
		\begin{eqnarray*}
			&&2f(x)\cdot f'(x)=\mathrm{e}^x\\
			&\Leftrightarrow& \left[f^2(x)\right]'=\mathrm{e}^x\\
			&\Rightarrow& f^2(x)=\displaystyle\int\limits_{\mathrm{e}}^x \mathrm{e}^x \mathrm{\,d}x\\
			&\Rightarrow& f^2(x)=\mathrm{e}^x+C
		\end{eqnarray*}
		Từ $f(0)=1$ ta suy ra $C=0$.\\
		Vậy $f^2(x)=\mathrm{e}^x\Rightarrow f(x)=\sqrt{\mathrm{e}^x}$ (do $f(x) > 0$).\\
		Khi đó $I=\displaystyle\int\limits_1^2 f(x) \mathrm{\,d}x = \displaystyle\int\limits_1^2 \sqrt{\mathrm{e}^x}\mathrm{\,d}x = \displaystyle\int\limits_1^2 \mathrm{e}^{\tfrac{x}{2}} \mathrm{\,d}x = \left.2\mathrm{e}^{\tfrac{x}{2}}\right|_1^2 = 2\mathrm{e}-2\sqrt{\mathrm{e}}$.
	}
\end{ex}

\begin{ex}%[2D4C2-2]
	Cho hàm số $f(x)$ thỏa mãn $\left[f'(x)\right]^2+f(x)\cdot f''(x)=2x$, và $f(0)=f'(0)=2$. Tính $I=\displaystyle\int\limits_1^2f^2(x)\mathrm{\,d}x$.
	\choice
	{\True $I=\dfrac{15}{2}$}
	{$I=\dfrac{1}{2}$}
	{$I=\dfrac{19}{2}$}
	{$I=15$}
	\loigiai{
		Ta có $\left[f(x)f'(x)\right]'=\left[f'(x)\right]^2+f(x)f''(x)$.\\
		Do đó theo giả thiết ta được $\left[f(x)f'(x)\right]'=2x$.\\
		Suy ra $f(x)f'(x)=x^2+C$.\\
		Hơn nữa $f(0)=f'(0)=2$ suy ra $C=1$.\\
		$\Rightarrow f(x)f'(x)=x^2+1$.\\
		Tương tự vì $\left[f^2(x)\right]'=2f(x)f'(x)$ nên $\left[f^2(x)\right]'=2\left(x^2+1\right)$.\\
		Suy ra $f^2(x)=\displaystyle\int 2\left(x^2+1\right) \mathrm{\,d}x \Rightarrow f^2(x)=\dfrac{2}{3}{x^3}+2x+C$.\\
		Mặt khác $f(0)=2$ nên  suy ra $C=2$.\\
		$\Rightarrow f^2(x)=\dfrac{2}{3}{x^3}+2x+2$.\\
		Vậy $I=\displaystyle\int\limits_1^2 f^2(x)\mathrm{\,d}x=\displaystyle\int\limits_1^2 \left(\dfrac{2}{3}{x^3}+2x+2\right)\mathrm{\,d}x=\dfrac{15}{2}$.
	}
\end{ex}

\begin{ex}%[2D4C2-2]
	Cho hàm số $f(x)$ thỏa mãn: $\left[f'(x)\right]^2+f(x)\cdot f''(x)=15x^4+12x$, $\forall x\in\mathbb{R}$ và $f(0)=f'(0)=1$. Giá trị của $f^2(1)$ bằng
	\choice
	{$\dfrac{5}{2}$}
	{\True $8$}
	{$10$}
	{$4$}
	\loigiai{
		Theo giả thiết
		\allowdisplaybreaks
		\begin{eqnarray*}
			& & \forall x\in\mathbb{R}\colon \left[f'(x)\right]^2+f(x)\cdot f''(x)=15x^4+12x\\
			&\Leftrightarrow& f'(x)\cdot f'(x)+f(x)\cdot f''(x)=15x^4+12x\\
			&\Leftrightarrow& \left[f(x)\cdot f'(x)\right]'=15x^4+12x\\
			&\Leftrightarrow& f(x)\cdot f'(x)=\displaystyle\int \left(15x^4+12x\right)\mathrm{\,d}x=3x^5+6x^2+C.\quad (1)
		\end{eqnarray*}
		Thay $x=0$ vào $(1)$, ta được $f(0)\cdot f'(0)=C \Leftrightarrow C=1$.\\
		Khi đó $(1)$ trở thành $\begin{aligned}[t]& f(x)\cdot f'(x)=3x^5+6x^2+1\\
			&\Rightarrow \displaystyle\int\limits_0^1 f(x)\cdot f'(x) \mathrm{\,d}x = \displaystyle\int\limits_0^1 \left(3x^5+6x^2+1\right) \mathrm{\,d}x\\
			&\Leftrightarrow \left.\left[\dfrac{1}{2} f^2(x)\right] \right|_0^1 = \left.\left(\dfrac{1}{2}{x^6}+2x^3+x\right) \right|_0^1
			\Leftrightarrow \dfrac{1}{2}\left[f^2(1)-f^2(0)\right]=\dfrac{7}{2} \\
			&\Leftrightarrow f^2(1)-1=7\Leftrightarrow f^2(1)=8.\end{aligned}$\\
		Vậy $f^2(1)=8$.
	}
\end{ex}

\begin{ex}%[2D4C2-2]
	Cho hàm số $y=f(x)$ thỏa mãn $\left[f'(x)\right]^2+f(x)\cdot f''(x)=x^3-2x,\,\forall x\in \mathbb{R}$ và $f(0)=f'(0)=2$. Tính giá trị của $T=f^2(2)$.
	\choice
	{$\dfrac{160}{15}$}
	{\True $\dfrac{268}{15}$}
	{$\dfrac{4}{15}$}
	{$\dfrac{268}{30}$}
	\loigiai{
		Ta có $\left[f'(x)\right]^2+f(x)\cdot f''(x)=x^3-2x,\,\forall x\in \mathbb{R} \Leftrightarrow \left[f'(x)\cdot f(x)\right]'=x^3-2x,\,\forall x\in \mathbb{R}$.\\
		Lấy nguyên hàm hai vế ta có 
		\allowdisplaybreaks
		\begin{eqnarray*}
			& & \displaystyle\int \left[f'(x)\cdot f(x)\right]' \mathrm{\,d}x= \displaystyle\int \left(x^3-2x\right)\mathrm{\,d}x\\ &\Leftrightarrow& f'(x)\cdot f(x)=\dfrac{x^4}{4}-x^2+C.
		\end{eqnarray*}
		Theo đề ra ta có $f(0)\cdot f(0)=C=4$.\\
		Suy ra $\displaystyle\int\limits_0^2 f'(x)\cdot f(x)\mathrm{\,d}x = \displaystyle\int\limits_0^2 \left(\dfrac{x^4}{4}-x^2+4\right)\mathrm{\,d}x \Leftrightarrow \left.\dfrac{f^2(x)}{2}\right|_0^2=\dfrac{104}{15}$ $\Leftrightarrow f^2(2)=\dfrac{268}{15}$.
	}
\end{ex}
\Closesolutionfile{ans}
% \indapan{10}{ans/ans-2-B1}
% \begin{dang}{}
% 	\begin{enumerate}
% 		\item Điều kiện hàm ẩn có dạng: $A(x)f(x)+B(x)f'(x)=h(x)$.\quad$(1)$\\
% 		\textit{Ý tưởng giải:}
% 		\begin{itemize}
% 			\item Ta cần nhân thêm một lượng $u(x)$ vào $(1)$ để tạo thành \break  $u'(x)f(x)+u(x)f'(x)=u(x).h(x)$ và lúc này:
% 			\allowdisplaybreaks
% 			\begin{eqnarray*}
% 				& & u'(x)f(x)+u(x)f'(x)=u(x)\cdot h(x)\\
% 				&\Leftrightarrow & \left[u(x)f(x)\right]'=u(x)\cdot .h(x)\\
% 				&\Rightarrow& \displaystyle\int \left[u(x)f(x)\right]'\mathrm{\,d}x= \displaystyle\int u(x)\cdot h(x)\mathrm{\,d}x\\
% 				&\Rightarrow& u(x)f(x)=\displaystyle\int u(x)\cdot h(x)\mathrm{\,d}x\\
% 				&\Rightarrow& f(x)=\dfrac{\displaystyle\int u(x)\cdot h(x)\mathrm{\,d}x}{u(x)}
% 			\end{eqnarray*}
% 			\item Cách tìm $u(x)$\\
% 			$u(x)$ được chọn sao cho: $\heva{&u'(x)=A(x)\\&u(x)=B(x)}$\\
% 			$\Rightarrow \dfrac{u'(x)}{u(x)}=\dfrac{A(x)}{B(x)} \Rightarrow \displaystyle\int \dfrac{u'(x)}{u(x)}\mathrm{\,d}x =\displaystyle\int\dfrac{A(x)}{B(x)}\mathrm{\,d}x$\\ $\Rightarrow \ln \left|u(x)\right|=\displaystyle\int \dfrac{A(x)}{B(x)}\mathrm{\,d}x \Rightarrow u(x)=\mathrm{e}^{\displaystyle\int \dfrac{A(x)}{B(x)}\mathrm{\,d}x}$.\\
% 		\end{itemize}
% 		\textbf{Tóm lại phương pháp giải:} $A(x)f(x)+B(x)f'(x)=h(x)$ $(1)$ như sau:
% 		\begin{itemize}
% 			\item Tìm $u(x)$: $u(x)=\mathrm{e}^{\displaystyle\int \dfrac{A(x)}{B(x)} \mathrm{\,d}x}$.
% 			\item Nhân $u(x)$ vào $(1)$ $\Rightarrow f(x)=\dfrac{\displaystyle\int{u(x)\cdot h(x)} \mathrm{\,d}x}{u(x)}$. 
% 		\end{itemize}
% 		\item Một số dạng đặc biệt của $(1)$
% 		\begin{enumerate}
% 			\item Điều kiện hàm ẩn có dạng: $\heva{&f'(x)+f(x)=h(x)\\&f'(x)-f(x)=h(x).}$\\
% 			\textbf{Phương pháp giải}
% 			\begin{itemize}
% 				\item $f'(x)+f(x)=h(x)$.\\
% 				Nhân hai vế với $\mathrm e^x$ ta được $$\mathrm e^x\cdot f'(x)+\mathrm e^x\cdot f(x)=\mathrm e^x\cdot h(x)\Leftrightarrow \left[\mathrm e^x\cdot f(x)\right]'=\mathrm e^x\cdot h(x).$$
% 				Suy ra $\mathrm e^x\cdot f(x)=\displaystyle\int \mathrm e^x\cdot h(x) \mathrm{\,d}x$.\\
% 				Từ đây ta dễ dàng tính được $f(x)$.
% 				\item $f'(x)-f(x)=h(x)$.\\
% 				Nhân hai vế với $\mathrm e^{-x}$ ta được $$\mathrm e^{-x}\cdot f'(x)-\mathrm e^{-x}\cdot f(x)=\mathrm e^{-x}\cdot h(x)\Leftrightarrow \left[e^{-x}\cdot f(x)\right]'=\mathrm e^{-x}\cdot h(x).$$
% 				Suy ra $\mathrm e^{-x}\cdot f(x)=\displaystyle\int \mathrm e^{-x}\cdot h(x) \mathrm{\,d}x$.\\
% 				Từ đây ta dễ dàng tính được $f(x)$.
% 			\end{itemize}
% 			\item Điều kiện hàm ẩn có dạng: $f'(x)+p(x)\cdot f(x)=h(x)$.\\
% 			\textbf{Phương pháp giải}\\
% 			Nhân hai vế với $\mathrm e^{\displaystyle\int p (x)\mathrm{\,d}x}$ ta được
% 			\allowdisplaybreaks
% 			\begin{eqnarray*}
% 				& & f'(x)\cdot \mathrm e^{\displaystyle\int p(x)\mathrm{\,d}x}+p(x)\cdot \mathrm e^{\displaystyle\int p (x)dx}\cdot f(x)=h(x)\cdot{\mathrm e^{\displaystyle\int p (x)dx}}\\
% 				&\Leftrightarrow& \left[f(x)\cdot{e^{\displaystyle\int p (x)dx}}\right]'=h(x)\cdot \mathrm e^{\displaystyle\int p (x)\mathrm{\,d} x}.
% 			\end{eqnarray*}		
% 			Suy ra $f(x)\cdot \mathrm e^{\displaystyle\int p(x)\mathrm{\,d}x}=\displaystyle\int \mathrm e^{\displaystyle\int p (x)\mathrm{\,d}x}h(x) \mathrm{\,d}x$.\\
% 			Từ đây ta dễ dàng tính được $f(x)$.
% 		\end{enumerate}
% 	\end{enumerate}
% \end{dang}
\Opensolutionfile{ans}[ans/ans-2-B1-D2]
% \TN
\begin{ex}%[2D4C2-4]
	Cho hàm số $f(x)$ thỏa mãn $f(x)+f'(x)=\mathrm{e}^{-x}$, $\forall x\in\mathbb{R}$ và $f(0)=2$. Tính $I=\displaystyle\int\limits_1^2 \dfrac{f(x) \mathrm{e}^x}{x}\mathrm{\,d}x$.
	\choice
	{$I=2\ln 2$}
	{$I=\ln 2$}
	{$I=1+\ln 2$}
	{\True $I=1+2\ln 2$}
	\loigiai{
		Ta có
		\allowdisplaybreaks
		\begin{eqnarray*}
			& & f(x)+f'(x)=\mathrm{e}^{-x}\\
			&\Leftrightarrow& f(x) \mathrm{e}^x+f'(x)\mathrm{e}^x=1\\
			&\Leftrightarrow& \left[f(x) \mathrm{e}^x\right]'=1\\
			&\Rightarrow& f(x)\mathrm{e}^x=\displaystyle\int x \mathrm{\,d}x\\
			&\Leftrightarrow& f(x) \mathrm{e}^x=x+C.
		\end{eqnarray*}
		Vì $f(0)=2$ nên $C=2$.\\
		$\Rightarrow f(x)\mathrm{e}^x=x+2$.\\
		Vậy 
		$I=\displaystyle\int\limits_1^2 \dfrac{f(x) \mathrm{e}^x}{x} \mathrm{\,d}x = \displaystyle\int\limits_1^2 \dfrac{x+2}{x}\mathrm{\,d}x=\displaystyle\int\limits_1^2 \left(1+\dfrac{2}{x}\right)\mathrm{\,d}x= \left(x+2\ln | x|\right)\bigg|_1^2=1+2\ln 2$.
	}
\end{ex}

\begin{ex}%[2D4C2-4]
	Cho hàm số $f(x)$ có đạo hàm trên $\mathbb{R}$ thỏa mãn $\left(x+2\right)f(x)+\left(x+1\right)f'(x)=\mathrm{e}^x$ và $f(0)=\dfrac{1}{2}$. Tính $I=\displaystyle\int\limits_1^2 \left(2x+2\right)f(x)\mathrm{\,d}x$.
	\choice
	{$I=\mathrm{e}^2$}
	{$I=1+\mathrm{e}$}
	{$I=1+\mathrm{e}^2$}
	{\True $I=\mathrm{e}^2-\mathrm{e}$}
	\loigiai{
		Ta có
		\allowdisplaybreaks
		\begin{eqnarray*}
			& & \left(x+2\right)f(x)+\left(x+1\right)f'(x)=\mathrm{e}^x\\
			&\Leftrightarrow& \left(x+1\right)f(x)+f(x)+\left(x+1\right)f'(x)=\mathrm{e}^x\\
			&\Leftrightarrow& \left[\left(x+1\right)f(x)\right]+\left[\left(x+1\right)f(x)\right]'=\mathrm{e}^x\\
			&\Leftrightarrow& \mathrm{e}^x\left[\left(x+1\right)f(x)\right]+\mathrm{e}^x\left[\left(x+1\right)f(x)\right]'=\mathrm{e}^{2x}\\
			&\Leftrightarrow& \left[\mathrm{e}^x\left(x+1\right)f(x)\right]'=\mathrm{e}^{2x}\\
			&\Rightarrow& \displaystyle\int \left[\mathrm{e}^x\left(x+1\right)f(x)\right]'\mathrm{\,d}x=\displaystyle\int \mathrm{e}^{2x}\mathrm{\,d}x\\
			&\Leftrightarrow& \mathrm{e}^x\left(x+1\right)f(x)=\dfrac{1}{2}{\mathrm{e}^{2x}}+C.
		\end{eqnarray*}
		Mà $f(0)=\dfrac{1}{2}$ $\Rightarrow C=0$.\\
		Vậy $f(x)=\dfrac{1}{2}\cdot \dfrac{\mathrm{e}^x}{x+1}$.\\
		Do đó 
		$I=\displaystyle\int\limits_1^2 \left(2x+2\right)\dfrac{1}{2}\cdot \dfrac{\mathrm{e}^x}{x+1}\mathrm{\,d}x=\displaystyle\int\limits_1^2 \mathrm{e}^x\mathrm{\,d}x=\mathrm e^2-\mathrm e$.
	}
\end{ex}

\begin{ex}%[2D4C2-3]
	Cho hàm số $y=f(x)$ liên tục, có đạo hàm trên $\mathbb{R}$ thỏa mãn điều kiện \break $f(x)+x\left[f'(x)-2\sin x\right]=x^2\cos x$, $x\in \mathbb{R}$ và $f\left(\dfrac{\pi}{2}\right)=\dfrac{\pi}{2}$. Tính $I=\displaystyle\int\limits_0^{\tfrac{\pi}{2}} \dfrac{f(x)}{x}\mathrm{\,d}x$.
	\choice
	{\True $I=1$}
	{$I=\dfrac{\pi}{2}$}
	{$I=-1$}
	{$I=-\pi$}
	\loigiai{
		Từ giả thiết $\begin{aligned}[t] &f(x)+x\left(f'(x)-2\sin x\right)=x^2\cos x\\
			&\Leftrightarrow f(x)+xf'(x)=x^2\cos x+2x\sin x\\
			&\Leftrightarrow \left(xf(x)\right)'=\left(x^2\sin x\right)'\\
			&\Leftrightarrow xf(x)=x^2\sin x+C.\end{aligned}$\\
		Mặt khác $f\left(\dfrac{\pi}{2}\right)=\dfrac{\pi}{2}\Rightarrow C=0\Rightarrow f(x)=x\sin x$.\\
		Vậy
		$I=\displaystyle\int\limits_0^{\tfrac{\pi}{2}}{\dfrac{f(x)}{x}\mathrm{\,d}x}=\displaystyle\int\limits_0^{\tfrac{\pi}{2}}{\dfrac{x\sin x}{x}\mathrm{\,d}x}=\displaystyle\int\limits_0^{\tfrac{\pi}{2}}{\sin x\mathrm{\,d}x}=1$.
	}
\end{ex}

\begin{ex}%[2D4C2-4]
	Cho hàm số $y=f(x)$ có đạo hàm trên $(0;+\infty)$ thỏa mãn $2xf'(x)+f(x)=2x$, $\forall x\in(0;+\infty)$, $f(1)=1$. Giá trị của biểu thức $f(4)$ là
	\choice
	{$\dfrac{25}{6}$}
	{$\dfrac{25}{3}$}
	{\True $\dfrac{17}{6}$}
	{$\dfrac{17}{3}$}
	\loigiai{
		Xét phương trình $2xf'(x)+f(x)=2x$ $(1)$ trên $(0;+\infty)$ ta có $$(1)\Leftrightarrow f'(x)+\dfrac{1}{2x}\cdot f(x)=1.\quad(2)$$
		Đặt $g(x)=\dfrac{1}{2x}$, ta tìm một nguyên hàm $G(x)$ của $g(x)$.\\
		Ta có $\displaystyle\int g(x)\mathrm{\,d}x=\displaystyle\int \dfrac{1}{2x}\mathrm{\,d}x=\dfrac{1}{2}\ln x+C=\ln \sqrt x+C$. Ta chọn $G(x)=\ln \sqrt x $.\\
		Nhân cả 2 vế của $(2)$ cho $\mathrm{e}^{G(x)}=\sqrt x$, ta được 
		$$\sqrt x\cdot f'(x)+\dfrac{1}{2\sqrt x}\cdot f(x)=\sqrt x \Leftrightarrow \left[\sqrt x \cdot f(x)\right]'=\sqrt x. \quad(3)$$
		Lấy tích phân 2 vế của $(3)$ từ $1$ đến $4$, ta được \\
		$\displaystyle\int\limits_1^4 \left[\sqrt x \cdot f(x)\right]'\mathrm{\,d}x= \displaystyle\int\limits_1^4 \sqrt x\mathrm{\,d}x$ $\Rightarrow \left[\sqrt x \cdot f(x)\right]\bigg|_1^4=\left.\left(\dfrac{2}{3}\sqrt{x^3}\right)\right|_1^4\Rightarrow 2f(4)-f(1)=\dfrac{14}{3}$\\
		$\Rightarrow f(4)=\dfrac{1}{2}\left(\dfrac{14}{3}+1\right)=\dfrac{17}{6}$ (vì $f(1)=1$).\\
		Vậy $f(4)=\dfrac{17}{6}$.
	}
\end{ex}

\begin{ex}%[2D4C2-2]
	Cho hàm số $f(x)$ không âm, có đạo hàm trên đoạn $[0;1]$ và thỏa mãn $f(1)=1$, $\left[2f(x)+1-x^2\right]f'(x)=2x\left[1+f(x)\right]$, $\forall x\in[0;1]$. Tích phân $\displaystyle\int\limits_0^1 f(x)\mathrm{\,d}x$ bằng
	\choice
	{$1$}
	{$2$}
	{\True $\dfrac{1}{3}$}
	{$\dfrac{3}{2}$}
	\loigiai{
		Xét trên đoạn $[0;1]$, theo đề bài ta có
		\allowdisplaybreaks
		\begin{eqnarray*}
			&&\left[2f(x)+1-x^2\right]f'(x)=2x\left[1+f(x)\right]\\
			&\Leftrightarrow& 2f(x)\cdot f'(x)=2x+\left(x^2-1\right)\cdot f'(x)+2x\cdot f(x)\\
			&\Leftrightarrow& \left[f^2(x)\right]'=\left[x^2+\left(x^2-1\right)\cdot f(x)\right]'\\
			&\Leftrightarrow& f^2(x)=x^2+\left(x^2-1\right)\cdot f(x)+C.\quad (1)
		\end{eqnarray*}
		Thay $x=1$ vào $(1)$ ta được $f^2(1)=1+C\Leftrightarrow C=0$ (vì $f(1)=1$).\\
		Do đó, $(1)$ trở thành
		\allowdisplaybreaks
		\begin{eqnarray*} &&f^2(x)=x^2+\left(x^2-1\right)\cdot f(x)\\
			&\Leftrightarrow& f^2(x)-1=x^2-1+\left(x^2-1\right)\cdot f(x)\\
			&\Leftrightarrow& \left[f(x)-1\right]\cdot \left[f(x)+1\right]=\left(x^2-1\right)\cdot \left[f(x)+1\right]\\
			&\Leftrightarrow& f(x)-1=x^2-1 ~(\text{vì } f(x)\ge 0\Rightarrow f(x)+1 > 0,\,\forall x\in[0;1])\\
			&\Leftrightarrow& f(x)=x^2.
		\end{eqnarray*}
		Vậy $\displaystyle\int\limits_0^1 f(x)\mathrm{\,d}x=\displaystyle\int\limits_0^1 x^2\mathrm{\,d}x=\left.\dfrac{x^3}{3}\right|_0^1=\dfrac{1}{3}$.
	}
\end{ex}

\begin{ex}%[2D4C2-2]
	Cho hàm số $y=f(x)$ có đạo hàm liên tục trên $[0;1]$, thỏa mãn \break  $\left[f'(x)\right]^2+4f(x)=8x^2+4,\,\forall x\in[0;1]$ và $f(1)=2$. Tính $\displaystyle\int\limits_0^1 f(x) \mathrm{\,d}x$.
	\choice
	{$\dfrac{1}{3}$}
	{$2$}
	{\True $\dfrac{4}{3}$}
	{$\dfrac{21}{4}$}
	\loigiai{
		Ta có
		\allowdisplaybreaks
		\begin{eqnarray*}
			& & \left[f'(x)\right]^2+4f(x)=8x^2+4\\
			&\Rightarrow& \displaystyle\int\limits_0^1\left[f'(x)\right]^2\mathrm{\,d}x+4\displaystyle\int\limits_0^1 f(x)\mathrm{\,d}x =\displaystyle\int\limits_0^1\left(8x^2+4\right)\mathrm{\,d}x=\dfrac{20}{3}.\quad (1)	
		\end{eqnarray*}
		Và 
		\allowdisplaybreaks
		\begin{eqnarray*}
			&&\displaystyle\int\limits_0^1 xf'(x)\mathrm{\,d}x=xf(x)\big|_0^1-\displaystyle\int\limits_0^1 f(x)\mathrm{\,d}x=2-\displaystyle\int\limits_0^1 f(x)\mathrm{\,d}x\\
			&\Rightarrow&-4\displaystyle\int\limits_0^1 xf'(x)\mathrm{\,d}x=-8+4\displaystyle\int\limits_0^1 f(x)\mathrm{\,d}x.\quad (2)
		\end{eqnarray*}
		Lại có $$\displaystyle\int\limits_0^1\left(2x\right)^2\rm{d}x=\dfrac{4}{3}.\quad (3)$$
		Cộng vế với vế của (1), (2), (3) ta được $$\displaystyle\int\limits_0^1\left(f'(x)-2x\right)^2\mathrm{\,d}x=0\Rightarrow f'(x)=2x\Rightarrow f(x)=x^2+C.$$
		Mặt khác $f(1)=C+1=2\Rightarrow C=1\Rightarrow f(x)=x^2+1$.\\
		Do đó $\displaystyle\int\limits_0^1 f(x)\mathrm{\,d}x=\displaystyle\int\limits_0^1 \left(x^2+1\right)\mathrm{\,d}x=\dfrac{4}{3}$.
	}
\end{ex}

\begin{ex}%[2D4C2-2]
	Cho hàm số $y=f(x)$ có đạo hàm liên tục trên $[0;1]$ thỏa mãn \break $3f(x)+xf'(x)\ge x^{2018}$, $\forall x\in[0;1]$. Tìm giá trị nhỏ nhất của $\displaystyle\int_0^1 f(x)\mathrm{\,d}x$.
	\choice
	{$\dfrac{1}{2018\cdot2020}$}
	{$\dfrac{1}{2019\cdot2020}$}
	{$\dfrac{1}{2020\cdot2021}$}
	{\True $\dfrac{1}{2019\cdot2021}$}
	\loigiai{
		Ta có
		\allowdisplaybreaks
		\begin{eqnarray*}
			& & 3f(x)+xf'(x)\ge{x^{2018}},\, \forall x\in[0;1]\\
			&\Leftrightarrow& 3x^2f(x)+x^3\cdot f'(x)\ge x^{2020},\, \forall x\in[0;1]\\
			&\Leftrightarrow& \left[x^3f(x)\right]'\ge x^{2020}, \, \forall x\in[0;1]\\
			&\Rightarrow& x^3f(x)\ge\displaystyle\int x^{2020}\mathrm{\,d}x,\, \forall x\in[0;1]\\
			&\Rightarrow& x^3f(x)\ge\dfrac{x^{2021}}{2021}+C,\, \forall x\in[0;1].
		\end{eqnarray*}
		Cho $x=0\Rightarrow C=0\Rightarrow x^3f(x)\ge\dfrac{x^{2021}}{2021}$, $\forall x\in[0;1]$ $\Rightarrow f(x)\ge\dfrac{x^{2018}}{2021},\,\forall x\in[0;1]$.\\
		$\Rightarrow\displaystyle\int_0^1 f(x) \mathrm{\,d}x\ge\displaystyle\int_0^1 \dfrac{x^{2018}}{2021}\mathrm{\,d}x=\left.\left(\dfrac{x^{2019}}{2019\cdot2021}\right)\right|_0^1=\dfrac{1}{2019\cdot2021}$.
	}
\end{ex}
\Closesolutionfile{ans}
% \indapan{10}{ans/ans-2-B1-D2}
% \begin{dang}{MỘT SỐ DẠNG KHÁC}
% \end{dang}
\Opensolutionfile{ans}[ans/ans-2-B1-D3]
% \TNSA
\begin{ex}%[2D4C2-2]
	Cho hàm số $y=f(x)$ có đạo hàm trên $\mathbb{R}$ thỏa mãn $$\heva{&f(0)=f'(0)=1\\&		f(x+y)=f(x)+f(y)+3xy(x+y)-1} \text{ với }x,y\in\mathbb{R}$$
	Tính $\displaystyle\int\limits_0^1 f(x-1)\mathrm{\,d}x$.
	\choice
	{$\dfrac{1}{2}$}
	{$-\dfrac{1}{4}$}
	{\True $\dfrac{1}{4}$}
	{$\dfrac{7}{4}$}
	\loigiai{
		Lấy đạo hàm theo hàm số $y$ ta được $f'(x+y)=f'(y)+3x^2+6xy$, $\forall x\in\mathbb{R}$.\\
		Cho $y=0\Rightarrow f'(x)=f'(0)+3x^2\Rightarrow f'(x)=1+3x^2$\\
		$\Rightarrow f(x)=\displaystyle\int f'(x)\mathrm{\,d}x=x^3+x+C$ mà $f(0)=1 \Rightarrow C=1$.\\
		Do đó $f(x)=x^3+x+1\Rightarrow f(x-1)=(x-1)^3+x-1+1=x^3-3x^2+4x-1$.\\
		Vậy $\displaystyle\int\limits_0^1 f(x-1)\mathrm{\,d}x=\displaystyle\int\limits_0^1 \left(x^3-3x^2+4x-1\right)\mathrm{\,d}x=\dfrac{1}{4} \displaystyle\int\limits_{-1}^0 f(x)\mathrm{\,d}x= \displaystyle\int\limits_{-1}^0 \left(x^3+x+1\right)\mathrm{\,d}x=\dfrac{1}{4}$.
	}
\end{ex}

\begin{ex}%[2D4C2-2]
	Cho hai hàm $f(x)$ và $g(x)$ có đạo hàm trên $[1;4]$, thỏa mãn $\heva{&f(1)+g(1)=4\\&g(x)=-xf'(x)\\&f(x)=-xg'(x)}$, 
	với mọi $x\in[1;4]$. Tính tích phân $I=\displaystyle\int\limits_1^4\left[f(x)+g(x)\right]\mathrm{\,d}x$.
	\choice
	{$3\ln 2$}
	{$4\ln 2$}
	{$6\ln 2$}
	{\True $8\ln 2$}
	\loigiai{
		Từ giả thiết ta có 
		\allowdisplaybreaks
		\begin{eqnarray*}
			&& f(x)+g(x)=-x\cdot f'(x)-x\cdot g'(x)\\
			&\Leftrightarrow& \left[f(x)+x\cdot f'(x)\right]+\left[g(x)+x\cdot g'(x)\right]=0\\ &\Leftrightarrow& \left[x\cdot f(x)\right]'+\left[x\cdot g(x)\right]'=0\\
			&\Rightarrow& x\cdot f(x)+x\cdot g(x)=C\\
			&\Rightarrow& f(x)+g(x)=\dfrac{C}{x}
		\end{eqnarray*}
		Mà $f(1)+g(1)=4\Rightarrow C=4\Rightarrow f(x)+g(x)=\dfrac{4}{x}$.\\
		Vậy $I=\displaystyle\int\limits_1^4 \left[f(x)+g(x)\right]\mathrm{\,d}x=\displaystyle\int\limits_1^4\dfrac{4}{x}\mathrm{\,d}x=8\ln 2$.
	}
\end{ex}
\begin{ex}%[2D4C2-5]
	Cho hai hàm $f(x)$ và $g(x)$ có đạo hàm trên $\left[1;2\right]$ thỏa mãn $f(1)=g(1)=0$ và $\heva{& \dfrac{x}{(x+1)^2}g(x)+2023x=(x+1)f'(x) \\ & \dfrac{x^3}{x+1}g'(x)+f(x)=2024x^2}\,,\forall x\in \left[1;2\right]$. \\
	Tính tích phân $I=\displaystyle\int\limits_1^2 \left[\dfrac{x}{x+1}g(x)-\dfrac{x+1}{x}f(x) \right]\mathrm{\,d}x$.
	\choice
	{\True $I=\dfrac{1}{2}$}
	{$I=1$} 
	{$I=\dfrac{3}{2}$}
	{$I=2$}
	\loigiai{
		Từ giả thiết ta có $\heva{& \dfrac{1}{(x+1)^2}g(x)-\dfrac{x+1}{x}f'(x)=-2023\\ & \dfrac{x}{x+1}g'(x)+\dfrac{1}{x^2}f(x)=2024}\,,\forall x\in \left[1;2\right]$.\\
		Suy ra
		\allowdisplaybreaks 
		\begin{eqnarray*}
			&& \left[\dfrac{1}{(x+1)^2}g(x)+\dfrac{x}{x+1}g'(x) \right]-\left[\dfrac{x+1}{x}f'(x)-\dfrac{1}{x^2}f(x) \right]=1\\  
			&\Leftrightarrow& \left[\dfrac{x}{x+1}g(x) \right]'-\left[\dfrac{x+1}{x}f(x) \right]'=1\\ 
			&\Rightarrow& \dfrac{x}{x+1}g(x)-\dfrac{x+1}{x}f(x)=x+C.
		\end{eqnarray*}
		Mà $f(1)=g(1)=0\Rightarrow C=-1 \Rightarrow \dfrac{x}{x+1}g(x)-\dfrac{x+1}{x}f(x)=x-1$.\\
		Vậy $I=\displaystyle\int\limits_1^2 \left[\dfrac{x}{x+1}g(x)-\dfrac{x+1}{x}f(x) \right]\mathrm{\,d}x=\displaystyle\int\limits_1^2 (x-1)\mathrm{\,d}x=\dfrac{1}{2}$.
	}
\end{ex}

\begin{ex}%[2D4C2-5]
	Cho hàm số $f\left(x \right)$ xác định và liên tục trên $\mathbb{R}\setminus \left\{0\right\}$ thỏa mãn $x^2f^2\left(x \right)+\left(2x-1\right)f\left(x \right)=xf'\left(x \right)-1$, với mọi $x\in \mathbb{R}\setminus \left\{0\right\}$ đồng thời thỏa mãn $f\left(1\right)=-2$. Tính $\displaystyle\int\limits_1^2 f\left(x \right)\mathrm{\,d}x$.
	\choice
	{$-\dfrac{\ln 2}{2}-1$}
	{\True $-\ln 2-\dfrac{1}{2}$}
	{$-\ln 2-\dfrac{3}{2}$}
	{$-\dfrac{\ln 2}{2}-\dfrac{3}{2}$}
	\loigiai{
		Ta có 
		\allowdisplaybreaks 
		\begin{eqnarray*}
			&& x^2f^2\left(x \right)+2xf\left(x \right)+1=xf'\left(x \right)+f\left(x \right) \\ 
			&\Leftrightarrow& \left(xf\left(x \right)+1\right)^2=\left(xf\left(x \right)+1\right)'.
		\end{eqnarray*}
		Do đó
		\allowdisplaybreaks 
		\begin{eqnarray*}
			&& \dfrac{\left(xf\left(x \right)+1\right)'}{\left(xf\left(x \right)+1\right)^2}=1\\ 
			&\Rightarrow& \displaystyle\int \dfrac{\left(xf\left(x \right)+1\right)'}{\left(xf\left(x \right)+1\right)^2}\mathrm{\,d}x=\displaystyle\int 1\mathrm{\,d}x\\
			&\Rightarrow& -\dfrac{1}{xf\left(x \right)+1}=x+C\\
			&\Rightarrow& xf\left(x \right)+1=-\dfrac{1}{x+C}.
		\end{eqnarray*}
		Mặt khác $f\left(1\right)=-2$ nên $-2+1=-\dfrac{1}{1+C}\Rightarrow C=0$.\\
		Nên suy ra $xf\left(x \right)+1=-\dfrac{1}{x}\Rightarrow f\left(x \right)=-\dfrac{1}{x^2}-\dfrac{1}{x}$.\\
		Vậy $\displaystyle\int\limits_1^2 f\left(x \right)\mathrm{\,d}x=\displaystyle\int\limits_1^2 \left(-\dfrac{1}{x^2}-\dfrac{1}{x} \right)\mathrm{\,d}x=\left.\left(-\ln x+\dfrac{1}{x} \right)\right|_1^2=-\ln 2-\dfrac{1}{2}$.
	}
\end{ex}

\begin{ex}%[2D4C2-5]
	Cho hàm số $y=f(x)$ có đạo hàm liên tục trên $\mathbb{R}$ thỏa mãn $x\cdot f(x)\cdot f'(x)=f^2(x)-x,\,\forall x\in \mathbb{R}$ và có $f(2)=1$. Tích phân $\displaystyle\int\limits_0^2 f^2(x)\mathrm{\,d}x$ bằng
	\choice
	{$\dfrac{3}{2}$}
	{$\dfrac{4}{3}$}
	{\True $2$}
	{$4$}
	\loigiai{
		Ta có
		\allowdisplaybreaks 
		\begin{eqnarray*}
			x\cdot f(x)\cdot f'(x)=f^2(x)-x &\Leftrightarrow& 2x\cdot f(x)\cdot f'(x)=2f^2(x)-2x \\
			&\Leftrightarrow& 2x\cdot f(x)\cdot f'(x)+f^2(x)=3f^2(x)-2x \\ 
			&\Leftrightarrow& \displaystyle\int\limits_0^2 \left(x\cdot f^2(x) \right)'\mathrm{\,d}x=3\displaystyle\int\limits_0^2 f^2(x)\mathrm{\,d}x-\displaystyle\int\limits_0^2 2x\mathrm{\,d}x \\ 
			&\Leftrightarrow& \left.\left(x\cdot f^2(x) \right)\right|_0^2 =3I-4\\ 
			&\Leftrightarrow& 2=3I-4\\ 
			&\Leftrightarrow& I=2.
		\end{eqnarray*}
	}
\end{ex}

\begin{ex}%[2D4C2-5]
	Cho hàm số $f\left(x \right)$ có đạo hàm liên tục trên $\mathbb{R}$, $f\left(0\right)=0$, $f'\left(0\right)\ne 0$ và thỏa mãn hệ thức $f\left(x \right)\cdot f'\left(x \right)+18x^2=\left(3x^2+x \right)f'\left(x \right)+\left(6x+1\right)f\left(x \right),\,\forall x \in \mathbb{R}$. Biết $\displaystyle\int\limits_0^1 \left(x+1\right)\mathrm{e}^{f\left(x \right)}\mathrm{\,d}x=a\mathrm{e}^2+b,\,\left(a,b\in \mathbb{Q} \right)$. Giá trị của $a-b$ bằng
	\choice
	{\True $1$}
	{$2$}
	{$0$}
	{$\dfrac{2}{3}$}
	\loigiai{
		Ta có $f\left(x \right)\cdot f'\left(x \right)+18x^2=\left(3x^2+x \right)f'\left(x \right)+\left(6x+1\right)f\left(x \right)$.\\
		Lấy nguyên hàm hai vế ta được 
		\allowdisplaybreaks 
		\begin{eqnarray*}
			\dfrac{f^2\left(x \right)}{2}+6x^3=\left(3x^2+x \right)f\left(x \right)
			&\Rightarrow& f^2\left(x \right)-2\left(3x^2+x \right)f\left(x \right)+12x^3=0\\
			&\Rightarrow& \hoac{& f\left(x \right)=6x^2 \\ & f\left(x \right)=2x.}
		\end{eqnarray*}
		\begin{enumerate}[\bf TH1:]
			\item $f\left(x \right)=6x^2$ không thoả mãn kết quả $\displaystyle\int\limits_0^1 \left(x+1\right)\mathrm{e}^{f\left(x \right)}\mathrm{\,d}x=a\mathrm{e}^2+b,\,\left(a,b\in \mathbb{Q} \right)$.
			\item $f\left(x \right)=2x \Rightarrow \displaystyle\int\limits_0^1 \left(x+1\right)\mathrm{e}^{f\left(x \right)}\mathrm{\,d}x= \displaystyle\int\limits_0^1 \left(x+1\right)\mathrm{e}^{2x}\mathrm{\,d}x=\dfrac{3}{4}\mathrm{e}^2-\dfrac{1}{4}$.\\ 
			Suy ra $a=\dfrac{3}{4};b=-\dfrac{1}{4}$.
		\end{enumerate}
		Vậy $a-b=1$.
	}
\end{ex}

\begin{ex}%[2D4C2-5]
	Cho hàm số $y=f(x)$ xác định và có đạo hàm $f'\left(x \right)$ liên tục trên $[1;3]$; $f\left(x \right)\ne 0,\,\forall x\in \left[1;3\right]$; $f'\left(x \right)\left[1+f\left(x \right) \right]^2=\left(x-1\right)^2\left[f\left(x \right) \right]^4$ và $f\left(1\right)=-1$. Biết rằng $\displaystyle\int\limits_{\mathrm{e}}^3 f\left(x \right)\mathrm{\,d}x=a\ln 3+b\,\left(a,b\in \mathbb{Z} \right)$. Giá trị của $a+b^2$ bằng
	\choice
	{$4$}
	{\True $0$}
	{$2$}
	{$-1$}
	\loigiai{
		Ta có 
		\allowdisplaybreaks 
		\begin{eqnarray*}
			f'(x)\left[1+f(x)\right]^2=(x-1)^2\left[f(x)\right]^4
			&\Rightarrow& \dfrac{f'(x)}{f^4(x)}+\dfrac{2f'(x)}{f^3(x)}+\dfrac{f'(x)}{f^2(x)}=(x-1)^2\\
			&\Rightarrow& \displaystyle\int \left(\dfrac{f'(x)}{f^4(x)}+\dfrac{2f'(x)}{f^3(x)}+\dfrac{f'(x)}{f^2(x)} \right) \mathrm{\,d}x=\displaystyle\int (x-1)^2\mathrm{\,d}x\\
			&\Rightarrow& -\left(\dfrac{1}{3f^3(x)}+\dfrac{1}{f^2(x)}+\dfrac{1}{f(x)} \right)=\dfrac{1}{3}(x-1)^3+C. \quad (*)
		\end{eqnarray*}
		Do $f(1)=-1$ nên $C=\dfrac{1}{3}$.\\ 
		Thay vào $(*)$ ta được $\left(\dfrac{1}{f(x)}+1\right)^3=-(x-1)^3 \Rightarrow f(x)=\dfrac{-1}{x}$.\\
		Khi đó $\displaystyle\int\limits_{\mathrm{e}}^3 \dfrac{-1}{x}\mathrm{\,d}x=\left.-\ln \left|x \right|\right|_{\mathrm{e}}^3=-\ln 3+1\Rightarrow a=-1,b=1$.\\ 
		Vậy $a+b^2=0$.\\
	}
\end{ex}
\Closesolutionfile{ans}
% \indapan{10}{ans/ans-2-B1-D3}

% %%Bài 3. Ứng dụng
% \section{ỨNG DỤNG HÌNH HỌC CỦA TÍCH PHÂN}
\subsection{Diện tích hình thang cong}
\subsubsection{Hình phẳng giới hạn bởi đồ thị hàm số, trục hoành và hai đường thẳng $x=a$ và $x=b$}
\begin{center}
	\begin{tikzpicture}[scale=1,font=\footnotesize,line join=round,line cap=round,>=stealth]
	% Draw axes
	\draw[->] (-0.5,0) -- (6,0) node[right] {$x$};
	\draw[->] (0,-0.5) -- (0,4) node[above] {$y$};
	
	% Labels
	\node at (0,0) [below left] {$O$};
	\node at (0.7,0) [below] {$a$};
	\node at (4.3,0) [below] {$b$};
	\node at (5.3,2) {$y = f(x)$};
	
	% Draw function curve
	\draw[thick,domain=0.5:4.7,samples=100] plot (\x,{-0.3*(\x-0.8)*(\x-2.5)*(\x-5)+1.5});
	
	% Draw vertical lines
	\draw[dashed] (0.7,0) -- (0.7,{-0.3*(0.7-0.8)*(0.7-2.5)*(0.7-5)+1.5});
	\draw[dashed] (4.3,0) -- (4.3,{-0.3*(4.3-0.8)*(4.3-2.5)*(4.3-5)+1.5});
	
	% Draw shaded area
	\fill[pattern=north east lines, pattern color=black!50] 
	(0.7,0) -- plot[domain=0.7:4.3,samples=100] (\x,{-0.3*(\x-0.8)*(\x-2.5)*(\x-5)+1.5}) -- (4.3,0) -- cycle;
	
	% Additional labels
	\node at (0.5,2.7) [below] {$x = a$};
	\node at (4.5,3.5) [below] {$x = b$};
	\node at (3.5,-0.1) [below left] {$y = 0$};
\end{tikzpicture}
\end{center}
Cho hàm số $y=f(x)$ liên tục trên $[a;b]$. Khi đó, diện tích hình phẳng giới hạn bởi đồ thị hàm số $y=f(x)$, trục hoành $Ox$ $(y=0)$ và hai đường thẳng $x=a$ và $x=b$ được tính bởi công thức

	$$S=\displaystyle\int\limits_a^b \left|f(x)\right|\mathrm{\,d}x$$

\textbf{Chú ý:} Giả sử hàm số $y=f(x)$ liên tục trên $[a;b]$. Nếu $f(x)$ không đổi dấu trên $[a;b]$ thì

	$$\displaystyle\int\limits_a^b \left|f(x)\right|\mathrm{\,d}x=\displaystyle\left|\int\limits_a^b f(x)\mathrm{\,d}x\right|.$$

\subsubsection{Hình phẳng giới hạn bởi hai đồ thị hàm số và hai đường thẳng $x=a$ và $x=b$}

\begin{center}
	\begin{tikzpicture}[scale=1,font=\footnotesize,line join=round,line cap=round,>=stealth]
	% Draw axes
\draw[->] (-0.5,0) -- (6,0) node[right] {$x$};
\draw[->] (0,-0.5) -- (0,4) node[above] {$y$};

% Labels
\node at (0,0) [below left] {$O$};
\node at (0.7,0) [below] {$a$};
\node at (4.3,0) [below] {$b$};
\node at (5,3.1) {$y = f(x)$};
\node at (5,2.1) {$y = g(x)$};
% Draw function curve
\draw[thick,domain=1:3.7,samples=100] plot (\x,{sqrt(4-(\x-2.5)^2)+1.5});
\draw[thick,domain=0.9:3.9,samples=100] plot (\x,{-sqrt(4-(\x-2.5)^2)+3.3});

% Draw vertical lines
\draw[dashed] (1.2,0) -- (1.2,{sqrt(4-(1.2-2.5)^2)+1.5});
\draw[dashed] (3.4,0) -- (3.4,{sqrt(4-(3.4-2.5)^2)+1.5});

% Draw shaded area
\fill[pattern=north east lines, pattern color=black!50] 
(1.2,1.78) -- plot[domain=1.2:3.4,samples=100] (\x,{sqrt(4-(\x -2.5)^2)+1.5}) -- (3.4,1.514) -- plot[domain=1.2:3.4,samples=100] (\x,{-sqrt(4-(\x-2.5)^2)+3.3}) -- cycle;

% Additional labels
\node at (1.2,4) [below] {$x = a$};
\node at (3.4,4) [below] {$x = b$};

\end{tikzpicture}
\end{center}
Cho 2 hàm số $y=f(x)$ và $y=g(x)$ liên tục trên $[a;b]$. Khi đó diện tích của hình phẳng giới hạn bởi đồ thị hai hàm số $y=f(x)$ và $y=g(x)$ và hai đường thẳng $x=a$ và $x=b$ được tính bởi công thức
$$
S=\displaystyle\int_a^b|f(x)-g(x)|\mathrm{\,d}x
$$
\subsection{Thể tích hình khối}
\subsubsection{Thể tích của vật thể}
\begin{center}
\begin{tikzpicture}[scale=1,font=\footnotesize,line join=round,line cap=round,>=stealth]
	\draw plot[smooth,tension=.65] coordinates{(1,2) (2.5,2.3) (3.5,2.2)};
	\draw[dashed] plot[smooth,tension=.65] coordinates{(3.5,2.2) (4,2)};
	\draw plot[smooth,tension=.65] coordinates{(4,2) (5,2.2) (5.5,2.1)};
	\draw[dashed] plot[smooth,tension=.65] coordinates{(5.5,2.1) (6,2)};
	\draw plot[smooth,tension=.65] coordinates{(1,1) (2.3,0.5) (3.5,0.8)};
	\draw[dashed] plot[smooth,tension=.65] coordinates{(3.5,0.8) (4,1)};
	\draw plot[smooth,tension=.65] coordinates{(4,1) (5,0.7) (5.5,0.8)};
	\draw[dashed] plot[smooth,tension=.65] coordinates{(5.5,0.8) (6,1)};
	\draw[dashed] (1,1) arc (-90:90:.2 and 0.5);
	\draw (1,2) arc (90:270:.2 and 0.5);
	\draw[dashed] (4,1) arc (-90:90:.2 and 0.5);
	\draw (4,2) arc (90:270:.2 and 0.5);
	\draw (6,1) arc (-90:270:.2 and 0.5);
	\fill[pattern=north east lines] (4,1) arc (-90:90:.2 and 0.5)--(4,2) arc (90:270:.2 and 0.5)--cycle;
	\draw (-.5,0)--(0.5,0) (1,0)--(3.5,0) (4,0)--(5.5,0);
	\draw[dashed] (0.5,0)--(1,0) (3.5,0)--(4,0) (5.5,0)--(6,0);
	\draw[->] (6,0)--(7,0)node[below]{$x$};
	\draw (0.5,-1)--(0.5,3)--(1.5,3.5)--(1.5,2.2) (1.5,.8)--(1.5,-0.5)--(0.5,-1);
	\draw[dashed](1.5,2.2)--(1.5,.8);
	\draw[dashed] (1,1)--(1,0)node[below]{$a$};
	\coordinate (A) at (0.5,3);
	\coordinate (B) at (1.5,3.5);
	\coordinate (C) at (1.5,2.2);
	%\tkzMarkAngle[size=.6](A,B,C);
	\draw pic[draw=black, angle eccentricity=1.6, angle radius=0.5cm]{angle=A--B--C};
	\draw (1.3,3.2) node {\footnotesize $P$};
	\draw (3.5,-1)--(3.5,3)--(4.5,3.5)--(4.5,2) (4.5,1)--(4.5,-0.5)--(3.5,-1);
	\draw[dashed](4.5,2)--(4.5,1);
	\draw[dashed] (4,1)--(4,0)node[below]{$x$};
	\coordinate (D) at (3.5,3);
	\coordinate (E) at (4.5,3.5);
	\coordinate (F) at (4.5,2);
	%	\tkzMarkAngle[size=.6](D,E,F);
	\draw pic[draw=black, angle eccentricity=1.6, angle radius=0.5cm]{angle=D--E--F};
	\draw (4.3,3.2) node {\footnotesize $R$};
	\draw (5.5,-1)--(5.5,3)--(6.5,3.5)--(6.5,-0.5)--(5.5,-1);
	\draw[dashed] (6,1)--(6,0)node[below]{$b$};
	\coordinate (G) at (5.5,3);
	\coordinate (H) at (6.5,3.5);
	\coordinate (K) at (6.5,-0.5);
	%	\tkzMarkAngle[size=.6](G,H,K);
	\draw pic[draw=black, angle eccentricity=1.6, angle radius=0.5cm]{angle=G--H--K};	\draw (6.3,3.2) node {\footnotesize $Q$};
	\draw (0,.3) node {$O$};
	\fill (0,0) circle(1pt);
	\draw[->] (4,1.5)--(4.7,1.7) node[right] {\scriptsize $S(x)$};
\end{tikzpicture}
\end{center}
Trong không gian, cho một vật thể nằm trong khoảng không gian giữa hai mặt phẳng $(P)$ và $(Q)$ cùng vuông góc với trục $Ox$ tại các điểm $a$ và $b$. Mặt phẳng vuông góc với trục $Ox$ tại điểm $x(a\leq x \leq b)$ cắt vật thể theo mặt cắt có diện tích $S(x)$. Khi đó, nếu $S(x)$ là hàm số liên tục trên $[a;b]$ thì thể tích của vật thể được tính bởi công thức
	$$V=\displaystyle\int\limits_a^b S(x)\mathrm{\,d}x$$
\subsubsection{Thể tích khối tròn xoay}
\begin{center}
	\begin{tikzpicture}[scale=1,font=\footnotesize,line join=round,line cap=round,>=stealth]
	 \draw[->] (-1,0) -- (6,0) node[right] {$x$};
	\draw[->] (0,-3) -- (0,3) node[above] {$y$};
	
	\draw[black, thick] plot[domain=0.5:5] (\x, {0.8*(0.4*(\x-1)-0.4)^2+1});
	\draw[black, thick, dashed] plot[domain=0.5:5] (\x, {-0.8*(0.4*(\x-1)-0.4)^2-1});
	
	\draw[black, thick,dashed, domain=4.54:5.54, samples=100] plot (\x, {sqrt(4.63 * (1 - 4 * (\x - 5.04)^2))});
	\draw[black, thick,dashed, domain=4.54:5.54, samples=100] plot (\x, {-sqrt(4.63 * (1 - 4 * (\x - 5.04)^2))});
	
	
	\draw[thick,dashed,domain=0.5:1.50 ,samples=100] plot (\x,{sqrt(0.5^2-(\x -1)^2)*1.5*1.5});
	\draw[thick,dashed,domain=0.5:1.50,samples=100] plot (\x,-{sqrt(0.5^2-(\x-1)^2)*1.5*1.5});
	
	\draw[thick,dashed,domain=2.55:3.551,samples=100] plot (\x,-{sqrt(1.274-5.09*(\x-3.05)^2)});
	\draw[thick,dashed,domain=2.55:3.551,samples=100] plot (\x,{sqrt(1.274-5.09*(\x-3.05)^2)});
	
	% Additional labels
	\node at (1,0) [below] {$a$};
	\node at (3,0) [below] {$x$};
	\node at (5,0) [below] {$b$};
	\node at (0,0) [below left] {$O$};
	\node at (3,2) [below] {$y=f(x)$};
	% Draw vertical lines
	\draw[thick] (1,0) -- (1,{0.8*(0.4*(1-1)-0.4)^2+1});
	\draw[thick] (3,0) -- (3,{0.8*(0.4*(3-1)-0.4)^2+1});
	\draw[thick] (5,0) -- (5,{0.8*(0.4*(5-1)-0.4)^2+1});
	
	%fill
	% Draw shaded area
	\fill[pattern=north east lines, pattern color=black!50] 
	(1,0) -- plot[domain=1:5,samples=100] (\x,{0.8*(0.4*(\x-1)-0.4)^2+1}) -- (5,0) -- cycle;
	
\end{tikzpicture}
\end{center}

Cho hàm số $y=f(x)$ liên tục, không âm trên $[a;b]$. Hình phẳng $(H)$ giới hạn bởi đồ thị hàm số $y=f(x)$, trục hoành $O x$ và hai đường thẳng $x=a$ và $x=b$ quay quanh trục $O x$ tạo thành một khối tròn xoay có thể tích bằng

	$$V=\displaystyle\pi\int\limits_a^b \left[f(x)\right]^2\mathrm{\,d}x$$

\begin{dang}{TÍNH DIỆN TÍCH HÌNH GIỚI HẠN BỞI CÁC ĐƯỜNG CONG}
\end{dang}

%\TN
\Opensolutionfile{ans}[ans/ans2-C4B3CD1-D1]
\begin{ex}%[2D4N3-1]
	Cho hai hàm số $f(x)$ và $g(x)$ liên tục trên $[a;b]$. Diện tích hình phẳng giới hạn bởi đồ thị của các hàm số $y=f(x), y=g(x)$ và các đường thẳng $x=a, x=b$ bằng
\choice
{$\left|\displaystyle\int\limits_a^b \left[f(x)-g(x)\right]\mathrm{\,d}x\right|$}
{$\displaystyle\int\limits_a^b \left|f(x)+g(x)\right|\mathrm{\,d}x$}
{\True $\displaystyle\int\limits_a^b \left|f(x)-g(x)\right|\mathrm{\,d}x$}
{$\displaystyle\int\limits_a^b \left[f(x)-g(x)\right]\mathrm{\,d}x$}
\loigiai{
Theo lý thuyết thì diện tích hình phẳng được giới hạn bởi đồ thị của các đường\\ $y=f(x), y=g(x)$, $x=a, x=b$ 
\\Được tính theo công thức $S=\displaystyle\int\limits_a^b \left|f(x)-g(x)\right|\mathrm{\,d}x$.
}
\end{ex}
\begin{ex}%[2D4N3-1]
	Gọi $S$ là diện tích của hình phẳng giới hạn bởi các đường $y=3^x$, $y=0$, $x=0$, $x=2$. Mệnh đề nào dưới đây đúng?
\choice
{\True $\displaystyle\int\limits_0^2 3^x\mathrm{\,d}x$}
{$S=\pi\displaystyle\int\limits_0^2 3^{2x}\mathrm{\,d}x$}
{$S=\pi\displaystyle\int\limits_0^2 3^x\mathrm{\,d}x$}
{$S=\displaystyle\int\limits_0^2 3^{2x}\mathrm{\,d}x$}

\loigiai{
Diện tích hình phẳng đã cho được tính bởi công thức $S=\displaystyle\int\limits_0^2 3^x\mathrm{\,d}x$.
}
\end{ex}%[2D4N3-1]
\begin{ex}%[2D4N3-1]
	Diện tích hình phẳng giới hạn bởi đồ thị hàm số $y=(x-2)^2-1$, trục hoành và hai đường thẳng $x=1, x=2$ bằng
\choice
{\True $\dfrac{2}{3}$}
{$\dfrac{3}{2}$}
{$\dfrac{1}{3}$}
{$\dfrac{7}{3}$}
\loigiai{
Ta có $S=\displaystyle\int\limits_1^2 \left|(x-2)^2-1\right|\mathrm{\,d}x=\displaystyle\int\limits_1^2\left|x^2-4 x+3\right| \mathrm{d}x=\left|\displaystyle\int\limits_1^2\left(x^2-4 x+3\right) \mathrm{d}x\right|=\dfrac{2}{3}$.
}

\end{ex}
\begin{ex}%[2D4H3-1]
	Tính diện tích $S$ hình phẳng giới hạn bởi các đường $y=x^2+1$, $x=-1$, $x=2$ và trục hoành.
\choice
{\True $S=6$}
{$S=16$}
{$S=\dfrac{13}{6}$}
{$S=13$}
\loigiai{
Ta có $S=\displaystyle\int\limits_{-1}^2\left|x^2+1\right| \mathrm{d}x=\displaystyle\int\limits_{-1}^2\left(x^2+1\right) \mathrm{d}x=6$.
}
\end{ex}
\begin{ex}%[2D4H3-1]
	Gọi $S$ là diện tích hình phẳng giới hạn bởi các đường $y=x^2+5, y=6 x, x=0, x=1$. Tính $S$.
\choice
{$\dfrac{4}{3}$}
{\True $\dfrac{7}{3}$}
{$\dfrac{8}{3}$}
{$\dfrac{5}{3}$}
\loigiai{
Diện tích hình phẳng cần tìm $S=\displaystyle\int\limits_0^1\left|x^2-6 x+5\right| \mathrm{d}x=\dfrac{7}{3}$.
}
\end{ex}
\begin{ex}%[2D4V3-1]
	Diện tích hình phẳng giới hạn bởi đồ thị các hàm số $y=\ln x, y=1$ và hai đường thẳng $x=1, x=e$ bằng
\choice
{$e^2$}
{$e+2$}
{$2 e$}
{\True $e-2$}
\loigiai{
\begin{eqnarray*}
S&=&\displaystyle\int\limits_1^e|\ln x-1|\mathrm{d}x\\&=&\left|\displaystyle\int\limits_1^e(\ln x-1) \mathrm{d}x\right|\\&=&|x(\ln x-1)|_1^e-\displaystyle\int\limits_1^e \mathrm{d}x|\\&=&| 1-\left.x\right|_1 ^e|\\&=&| 1-(e-1)|=| 2-e \mid\\&=&e-2.
\end{eqnarray*}
}
\end{ex}
\begin{ex}%[2D4H3-1]
	Diện tích hình phẳng giới hạn bởi đồ thị của hàm số $y=4 x-x^2, y=2 x$ và hai đường thẳng $x=1, x=e$ bằng
\choice
{$4$}
{$\dfrac{20}{3}$}
{\True $\dfrac{4}{3}$}
{$\dfrac{16}{3}$}
\loigiai{
	Diện tích hình phẳng cần tìm là\[S=\displaystyle\int\limits_0^2\left|x^2-2 x\right| \mathrm{d}x=\displaystyle\int\limits_0^2\left(2 x-x^2\right) \mathrm{d}x=\left.\left(x^2-\dfrac{x^3}{3}\right)\right|_0 ^2=\dfrac{4}{3}.\]
}
\end{ex}
\begin{ex}%[2D4V3-1]
	Tính diện tích $S$ của hình phẳng giới hạn bởi các đường $y=x^2-2 x$, $y=0$, $x=-10$, $x=10$.
\choice
{$S=\dfrac{2000}{3}$}
{$S=2008$}
{$S=2000$}
{\True $S=\dfrac{2008}{3}$}
\loigiai{
Phương trình hoành độ giao điểm của hai đường $(C)\colon y=x^2-2 x$ và $(d)\colon y=0$ là
$$
x^2-2 x=0 \Leftrightarrow\hoac{
	x=0 \\
	x=2
}.
$$
Bảng xét dấu\\
\begin{center}
	\begin{tikzpicture}
	\tkzTabInit[nocadre=false,lgt=2.5,espcl=2.5,deltacl=0.6]
	{$x$ /0.6, VT/0.6}
	{$-\infty$, $0$, $2$, $+\infty$}
	\tkzTabLine{,+,,-,,+,}
\end{tikzpicture}
\end{center}
Diện tích cần tìm

\begin{eqnarray*}
	S&=&\displaystyle\int\limits_{-10}^{10}\left|x^2-2 x\right| \mathrm{d}x\\&=&\displaystyle\int\limits_{-10}^0\left(x^2-2 x\right) \mathrm{d}x-\displaystyle\int\limits_0^2\left(x^2-2 x\right) \mathrm{d}x+\displaystyle\int\limits_2^{10}\left(x^2-2 x\right) \mathrm{d}x 
	\\&=&\left.\left(\dfrac{x^3}{3}-x^2\right)\right|_{-10} ^0-\left.\left(\dfrac{x^3}{3}-x^2\right)\right|_0 ^2+\left.\left(\dfrac{x^3}{3}-x^2\right)\right|_2 ^{10}\\&=&\dfrac{1300}{3}+\dfrac{4}{3}+\dfrac{704}{3}\\&=&\dfrac{2008}{3} .
\end{eqnarray*}
}
\end{ex}
\Closesolutionfile{ans}
% \indapan{10}{ans/ans2-C4B3CD1-D1}
%\TNTF
\Opensolutionfile{ans}[ans/ans2-C4B3CD1-D1-DS]
\begin{ex}%[2D4N3-1]
	Gọi $S$ là diện tích của hình phẳng giới hạn bời các đường $y=2^x$, $y=0$, $x=0$, $x=2$. Các mệnh đề sau đây đúng hay sai?
\choiceTF
{\True $S=\displaystyle\int\limits_0^2 2^x \mathrm{d}x$}
{\True $S=\dfrac{3}{\ln 2}$}
{$S=\pi \displaystyle\int\limits_0^2 2^x \mathrm{d}x$}
{$S=\dfrac{3 \pi}{\ln 2}$}
\loigiai{
\[
S=\displaystyle\int\limits_0^2\left|2^x\right| \mathrm{d}x=\displaystyle\int\limits_0^2 2^x \mathrm{d}x=\dfrac{2^2}{\ln 2}-\dfrac{2^0}{\ln 2}=\dfrac{3}{\ln 2}\left( \text {do } 2^x>0, \forall x \in[0;2]\right) .
\]
}
\end{ex}
\begin{ex}%[2D4N3-1]
	Gọi $S$ là diện tích hình phẳng giới hạn bởi các đường $y=\mathrm{e}^x, y=0, x=0, x=2$. Các mệnh đề sau đây đúng hay sai?
\choiceTF
{\True $S=\displaystyle\int\limits_0^2 \mathrm{e}^x \mathrm{d}x$}
{$S=e^2$}
{$S=\pi \displaystyle\int\limits_0^2 \mathrm{e}^x \mathrm{d}x$}
{$S=\left(e^2-1\right) \pi$}
\loigiai{
Diện tích hình phẳng giới hạn bời các đường $y=\mathrm{e}^x, y=0, x=0, x=2$ là
\[
S=\displaystyle\int\limits_0^2 e^x \mathrm{d}x=e^2-1.
\]
}
\end{ex}
\begin{ex}%[2D4V3-1]
	Các mệnh đề sau đây đúng hay sai
\choiceTF
{\True Diện tích hình phẳng giới hạn bởi đồ thị hàm số $y=x^2$, $y=2 x$, $x=0$, $x=1$ là $\dfrac{4}{3}$}
{\True Diện tích hình phẳng giới hạn bởi đồ thị hàm số $y=-x^2+2 x+1$, $y=2 x^2-4 x+1$, $x=0$, $x=2$ là $4$}
{\True Diện tích hình phẳng giới hạn bởi đồ thị hàm số $y=\dfrac{x-1}{x+1}$, trục hoành, $x=0$, $x=1$ là $2 \ln 2-1$}
{\True Diện tích hình phẳng giới hạn bởi đồ thị hàm số $y=-x^3+12 x$, $y=-x^2$, $x=-3$, $x=4$ là $\dfrac{937}{12}$}
\loigiai{
\begin{itemchoice}
\itemch Đúng.\\Diện tích hình phẳng giới hạn bời đồ thị hàm số $y=x^2$, $y=2 x$, $x=0$, $x=1$ là
\begin{eqnarray*}
S&=&\displaystyle\int\limits_0^1\left|x^2-x\right| \mathrm{d}x\\&=&\left|\displaystyle\int\limits_0^1\left(x^2-x\right) \mathrm{d}x\right|\\&=&\dfrac{4}{3}.
\end{eqnarray*}

\itemch Đúng.\\Diện tích hình phẳng giới hạn bởi đồ thị hàm số $y=-x^2+2x+1$, $y=2 x^2-4 x+1$, $x=0$, $x=2$ là
\begin{eqnarray*}
&&\displaystyle\int\limits_0^2\left|2x^2-4x+1-\left(-x^2+2 x+1\right)\right| \mathrm{d}x
\\&=&\displaystyle\int\limits_0^2\left|3 x^2-6 x\right| \mathrm{d}x
\\&=&\displaystyle\int\limits_0^2\left(6 x-3 x^2\right) \mathrm{d}x
\\&=&\left(3 x^2-x^3\right)|_0^2=4.
\end{eqnarray*}

\itemch Đúng.\\Diện tích hình phẳng giới hạn bởi đồ thị hàm số $y=\dfrac{x-1}{x+1}$, trục hoành, $x=0$, $x=1$ là

\begin{eqnarray*}
S&=&\displaystyle\int\limits_0^1\left|\dfrac{x-1}{x+1}\right| \mathrm{d}x
\\&=&\left|\displaystyle\int\limits_0^1\left(\dfrac{x-1}{x+1}\right)\mathrm{d}x\right|
\\&=&\left|\displaystyle\int\limits_0^1\left(1-\dfrac{2}{x+1}\right)\mathrm{d}x\right|
\\&=&|(x-2 \ln |x+1|)|_0^1|
\\&=&2 \ln {2}-1.
\end{eqnarray*}
\itemch Đúng.\\Diện tích hình phẳng giới hạn bởi đồ thị hàm số $y=-x^3+12$, $y=-x^2$, $x=-3$ là

\begin{eqnarray*}
	S&=&\displaystyle\int\limits_{-3}^4\left|x^3-x^2-12x\right| \mathrm{d}x
	\\&=&\displaystyle\int\limits_{-3}^0\left|x^3-x^2-12x\right| \mathrm{d}x+\displaystyle\int\limits_0^4\left|x^3-x^2-12 x\right| \mathrm{d}x 
	\\&=&\left|\displaystyle\int\limits_{-3}^0\left(x^3-x^2-12 x\right) \mathrm{d}x\right|+\left|\displaystyle\int\limits_0^4\left(x^3-x^2-12 x\right) \mathrm{d}x\right|
	\\&=&\left| \left|\left(\dfrac{x^4}{4}-\dfrac{x^3}{3}-6 x^2\right)\right|_{-3}^0\right| +\left| \left(\dfrac{x^4}{4}-\dfrac{x^3}{3}-6 x^2\right)\right|_0^4
	\\&=&\left|\dfrac{-99}{4}\right|+\left|\dfrac{-160}{3}\right|=\dfrac{937}{12}.
\end{eqnarray*}
\end{itemchoice}
}
\end{ex}
\Closesolutionfile{ans}
% \indapan{3}{ans/ans2-C4B3CD1-D1-DS}
%\TN
\Opensolutionfile{ans}[ans/ans2-C4B3CD1-D1-KQ]
\begin{ex}%[2D4V3-1]
	Tính diện tích hình phẳng giới hạn bời đồ thị hàm số $y=x^2+x-1$, $y=x^4+x-1$, $x=-1$, $x=1$.
\shortans{$0,27$}
\loigiai{
Diện tích hình phẳng giới hạn bởi đồ thị hàm số $y=x^2+x-1, y=x^4+x-1, x=-1, x=1$ là
\begin{eqnarray*}
	S&=&\displaystyle\int\limits_{-1}^1\left|x^2-x^4\right| \mathrm{d}x\\
	&=&\displaystyle\int\limits_{-1}^0\left|x^2-x^4\right| \mathrm{d}x+\displaystyle\int\limits_0^1\left|x^2-x^4\right| \mathrm{d}x\\
	&=&\left|\displaystyle\int\limits_{-1}^0\left(x^2-x^4\right) \mathrm{d}x\right|+\left|\displaystyle\int\limits_0^1\left(x^2-x^4\right) \mathrm{d}x\right|\\
	&=&\left|\left(\dfrac{x^3}{3}-\dfrac{x^5}{5}\right)\right| 0|+|\left(\dfrac{x^3}{3}-\dfrac{x^5}{5}\right)|0|\\
	&=&\dfrac{2}{15}+\dfrac{2}{15}=\dfrac{4}{15}\approx0,27.
\end{eqnarray*}
}
\end{ex}

\begin{ex}%[2D4V3-1]
	Kí hiệu $S(t)$ là diện tích của hình phẳng giới hạn bởi các đường $y=2 x+1$, $y=0$, $x=1$, $x=t\left(t>1\right)$. Tìm $t$ để $S(t)=10$.
\shortans{$3$}
\loigiai{
\textbf{Cách 1.} Ta có $S(t)=\displaystyle\int\limits_1^t|2 x+1| \mathrm{d}x=\displaystyle\int\limits_1^t(2 x+1) \mathrm{d}x$.\\
Suy ra $S(t)=\left.\left(x^2+x\right)\right|_1^t=t^2+t-2$.\\
Do đó $S(t)=10 \Leftrightarrow t^2+t-2=10 \Leftrightarrow t^2+t-12=0 \Leftrightarrow\hoac{&t=3 \\ &t=-4\text{ (L)}}.$\\
Vậy $t=3$.\\
\textbf{Cách 2}. Hình phẳng đã cho là hình thang có đáy nhỏ bằng $y(1)=3$, đáy lớn bằng $y(t)=2 t+1$ và chiều cao bằng $t-1$.
Ta có \[\dfrac{(3+2t+1)(t-1)}{2}=10 \Leftrightarrow 2 t^2+2 t-24=0 \Leftrightarrow\hoac{t=3 \\ t=-4}.\]\\ 
Vì $t>1$ nên $t=3$.
}
\end{ex}
\begin{ex}%[2D4V3-1]
	 Gọi $S$ là diện tích hình phẳng giới hạn bởi các đường $m y=x^2$, $m x=y^2(m>0)$. Tìm giá trị của $m$ để $S=3$.

\shortans{$3$}
\loigiai{
Tọa độ giao điểm của hai đồ thị hàm số là nghiệm của hệ phương trình $\heva{my=x^2 &\quad(1)\\mx=y^2 &\quad(2)}$
Thế $(1)$ vào $(2)$ ta được $m x=\left(\dfrac{x^2}{m}\right)^2 \Leftrightarrow m^3 x-x^4=0 \Leftrightarrow\hoac{&x=0\\&x=m>0.}$\\
Vì $y=\dfrac{x^2}{m}>0$ nên $m x=y^2 \text{ (với $y>0$) }  \Leftrightarrow y=\sqrt{m x}$\\
Khi đó diện tích hình phẳng cần tìm là 
\begin{eqnarray*}
	S&=&\displaystyle\int\limits_0^m\left|\sqrt{m x}-\dfrac{x^2}{m}\right| \mathrm{d}x=\left|\displaystyle\int\limits_0^m\left(\sqrt{m x}-\dfrac{x^2}{m}\right) \mathrm{d}x\right|\\
	&=&\left|\left(\dfrac{2 \sqrt{m}}{3} \cdot x^{\dfrac{3}{2}}-\dfrac{x^3}{3 m}\right)\right|_0^m\\
	&=& \left|\dfrac{1}{3} m^2 \right|=\dfrac{1}{3} m^2
\end{eqnarray*}
	Yêu cầu bài toán $S=3 \Leftrightarrow \dfrac{1}{3} m^2=3 \Leftrightarrow m^2=9 \Leftrightarrow m=3$.

}
\end{ex}
\begin{ex}%[2D4V3-1]
	Giá trị dương của tham số $m$ sao cho diện tích hình phẳng giới hạn bởi đồ thị của hàm số $y=2 x+3$ và các đường thẳng $y=0$, $x=0$, $x=m$ bằng 10 là?

\shortans{$2$}
\loigiai{
Vì $m>0$ nên $2 x+3>0, \forall x \in[0;m]$.
Diện tích hình phẳng giới hạn bởi đồ thị hàm số $y=2 x+3$ và các đường thẳng $y=0$, $x=0$, $x=m$ là
\[S=\displaystyle\int\limits_0^m(2 x+3) \cdot \mathrm{d}x=\left.\left(x^2+3 x\right)\right|_0^m=m^2+3m.\]
Theo giả thiết ta có
\[S=10 \Leftrightarrow m^2+3 m=10 \Leftrightarrow m^2+3 m-10=0 \Leftrightarrow \hoac{&m=2\\&m=-5}
 \Leftrightarrow m=2 \text { do } m>0.\]
}
\end{ex}
\begin{ex}%[2D4V3-1]
	Cho hàm số  $f(x)=\heva{&7-4x^3\text{ khi }  0 \leq x \leq 1\\&4-x^2 \text{ khi } x>1}$. Tính diện tích hình phẳng giới hạn bởi đồ thị hàm số $f(x)$ và các đường thẳng $x=0$, $x=3$, $y=0$.
\shortans{10}
\loigiai{

\begin{center}
	\begin{tikzpicture}[scale=0.6, font=\footnotesize, line join=round, line cap=round, >=stealth]
		% Vẽ trục
		\draw[->] (-0.5,0) -- (4,0) node[right] {$x$};
		\draw[->] (0,-5) -- (0,7) node[above] {$y$};
		
		% Đánh dấu các điểm trên trục x
		\foreach \x in {1,2,3}{
			\draw[fill=black] (\x,0) circle(0.03) node[below]{$\x$};
		}
		
		% Đánh dấu các điểm trên trục y
		\foreach \y in {-5,-4,-3,-2,-1,1,2,3,4,5,6,7}{
			\draw[fill=black] (0,\y) circle(0.03) node[left]{$\y$};
		}
		% Đánh dấu gốc tọa độ
		\draw[fill=black] (0,0) circle(0.03) node[below left] {$0$};
		
		
			
		% Draw function curve
		\draw[thick,domain=0:1,samples=100] plot (\x,{-4*(\x)*(\x)+7});
		\draw[thick,domain=1:3,samples=100] plot (\x,{-(\x)*(\x)+4});
		% Draw lines
		\draw (1,0) -- (1,3);
		\draw (3,0) -- (3,-5);
		\draw[dashed] (0,3) -- (1,3);
		\draw[dashed] (0,-5) -- (3,-5);
		% Draw shaded area
		\fill[pattern=north east lines, pattern color=black!50] 
		(0,0) -- plot[domain=0:1,samples=100] (\x,{-4*(\x)*(\x)+7}) -- (1,0) -- cycle;
		\fill[pattern=north east lines, pattern color=black!50] 
		(1,0) -- plot[domain=1:2,samples=100] (\x,{-(\x)*(\x)+4}) -- (2,0) -- cycle;
		\fill[pattern=north east lines, pattern color=black!50] 
		(2,0) -- plot[domain=2:3,samples=100] (\x,{-(\x)*(\x)+4}) -- (3,0) -- cycle;
	\end{tikzpicture}
\end{center}

\begin{eqnarray*}
	 S&=&\displaystyle\int\limits_0^1\left(7-4 x^3\right) \mathrm{d}x+\displaystyle\int\limits_1^2\left(4-x^2\right) \mathrm{d}x+\displaystyle\int\limits_2^3\left(x^2-4\right) \mathrm{d}x \\ & =&\left.\left(7 x-x^4\right)\right|_0 ^1+\left.\left(4 x-\dfrac{x^3}{3}\right)\right|_1 ^2+\left.\left(\dfrac{x^3}{3}-4 x\right)\right|_2 ^3
	\\&=&6+4-\dfrac{7}{3}-3-\dfrac{8}{3}+8=10 .
\end{eqnarray*}
}
\end{ex}
\Closesolutionfile{ans}
% \indapan{5}{ans/ans2-C4B3CD1-D1-KQ}
% \begin{dang}
% 	{TÍNH DIỆN TÍCH GIỚI HẠN BỞI CÁC ĐƯỜNG CONG KHI BIẾT ĐỒ THỊ HÀM SỐ CỦA CÁC ĐƯỜNG CONG}
% \end{dang}
\Opensolutionfile{ans}[ans/ans-2-C4B3CD1_10-19]

%\TN
%%%%-------------Câu 17
\begin{ex}%[2D4N3-1]
	\immini{
		Gọi $S$  là diện tích hình phẳng giới hạn bởi đồ thị hàm số  $y=f(x)$, trục hoành, đường thẳng $x=a$, $x=b$  (như hình vẽ bên). Hỏi cách tính $S$  nào dưới đây đúng?
	}{
		\begin{tikzpicture}[scale=.7,>=stealth, font=\footnotesize, line join=round, line cap=round]
			\def\a{-0.25} \def\b{2.5} \def\c{-6.75} \def\d{4.5} % Hệ số
			\def\xmin{-1} \def\xmax{7}
			\def\ymin{-2} \def\ymax{3} 
			\draw[->] (\xmin,0)--(\xmax,0) node [below]{$x$};
			\draw[->] (0,\ymin)--(0,\ymax) node [left]{$y$};
			\node at (0,0) [below left]{$O$};
			\draw[smooth,samples=300] plot[domain=1:6](\x,{\a*(\x)^3+\b*(\x)^2+\c*(\x)+\d});
			\draw[pattern=north east lines] plot[domain=1:6](\x,{\a*(\x)^3+\b*(\x)^2+\c*(\x)+\d});
			\node[below left] at (1,0) {$a$};
			\node[below right] at (3,0) {$c$};
			\node[below] at (6,0) {$b$};
			\node[] at (4,2.5) {$y=f(x)$};
		\end{tikzpicture}
	}
	\choice
	{$S=\displaystyle\int\limits_a^b f(x) \mathrm{\,d}x$}
	{$ S= \left|\displaystyle\int\limits_a^c f(x) \mathrm{\,d}x + \displaystyle\int\limits_c^b f(x) \mathrm{\,d}x \right|$}
	{\True  $S=-\displaystyle\int\limits_a^c f(x) \mathrm{\,d}x + \displaystyle\int\limits_c^b f(x) \mathrm{\,d}x$}
	{$S=\displaystyle\int\limits_a^c f(x) \mathrm{\,d}x + \displaystyle\int\limits_c^b f(x) \mathrm{\,d}x$}
	\loigiai{
		Ta có $y=f(x)$ liên tục trên đoạn $\left[a; b\right]$.\\
		Dựa vào đồ thị ta có $\left|f(x)\right|=\heva{& -f(x), & a\le x \le c\\& f(x), & c< x \le b.}$\\
		Suy ra 
		$S= \displaystyle\int\limits_a^b \left|f(x)\right| \mathrm{\,d}x = 
		\displaystyle\int\limits_a^c \left|f(x)\right| \mathrm{\,d}x +\displaystyle\int\limits_c^b \left|f(x)\right| \mathrm{\,d}x = -\displaystyle\int\limits_a^c f(x) \mathrm{\,d}x +
		\displaystyle\int\limits_c^b f(x) \mathrm{\,d}x$.
		
	}
\end{ex}	

%%%%%-------------Câu 18
\begin{ex}%[2D4N3-1]
	\immini{
		Cho hàm số $y=f(x)$  liên tục trên đoạn  $\left[a; b\right]$. Gọi $D$  là diện tích hình phẳng giới hạn bởi đồ thị  $\left(C\right)\colon y=f(x)$, trục hoành, hai đường thẳng  $x=a$, $x=b$ (như hình vẽ). Giả sử  $S_D$ là diện tích hình phẳng  $D$. Chọn phương án đúng trong các phương án {\bf A}, {\bf B}, {\bf C}, {\bf D} cho dưới đây?
	}{
		\begin{tikzpicture}[yscale=.7,xscale=1,>=stealth, font=\footnotesize, line join=round, line cap=round]
			\def\a{1/3} \def\b{0} \def\c{0} \def\d{0} % Hệ số
			\def\xmin{-3} \def\xmax{3}
			\def\ymin{-3} \def\ymax{3} 
			\draw[->] (\xmin,0)--(\xmax,0) node [below]{$x$};
			\draw[->] (0,\ymin)--(0,\ymax) node [left]{$y$};
			\node at (0,0) [below left]{$O$};
			\draw[smooth,samples=300] plot[domain=-2.1:2.1](\x,{\a*(\x)^3+\b*(\x)^2+\c*(\x)+\d});
			\draw[pattern=north east lines] plot[domain=0:-2](\x,{\a*(\x)^3+\b*(\x)^2+\c*(\x)+\d})--(-2,0)--cycle
			plot[domain=0:2](\x,{\a*(\x)^3+\b*(\x)^2+\c*(\x)+\d})--(2,0)--cycle;
			\node[above] at (-2,0) {$a$};
			\node[below] at (2,0) {$b$};
		\end{tikzpicture}
	}
	\choice
	{$S_D=\displaystyle\int\limits_a^0 f(x) \mathrm{\,d}x +\displaystyle\int\limits_0^b f(x) \mathrm{\,d}x$}
	{\True  $S_D=-\displaystyle\int\limits_a^0 f(x) \mathrm{\,d}x + \displaystyle\int\limits_0^b f(x) \mathrm{\,d}x$}
	{$S_D=\displaystyle\int\limits_a^0 f(x) \mathrm{\,d}x -\displaystyle\int\limits_0^b f(x) \mathrm{\,d}x$}
	{$S_D=-\displaystyle\int\limits_a^0 f(x) \mathrm{\,d}x -\displaystyle\int\limits_0^b f(x) \mathrm{\,d}x$}
	\loigiai{
		Ta có $y=f(x)$ liên tục trên đoạn $\left[a; b\right]$.\\
		Dựa vào đồ thị ta có $\left|f(x)\right|=\heva{& -f(x), & a\le x \le 0\\& f(x), & 0< x \le b.}$\\
		Suy ra $S_D= \displaystyle\int\limits_a^b \left|f(x)\right| \mathrm{\,d}x = 
		\displaystyle\int\limits_a^0 \left|f(x)\right| \mathrm{\,d}x +\displaystyle\int\limits_0^b \left|f(x)\right| \mathrm{\,d}x = -\displaystyle\int\limits_a^0 f(x) \mathrm{\,d}x + \displaystyle\int\limits_0^b f(x) \mathrm{\,d}x$.
	}
\end{ex}	

%%%%%-------------Câu 19
\begin{ex}%[2D4N3-1]
	\immini{
		Diện tích của hình phẳng được giới hạn bởi đồ thị hàm số $y=f(x)$, trục hoành và hai đường thẳng  $x=a$,  $x=b$  $(a<b)$ (phần tô đậm trong hình vẽ) tính theo công thức nào dưới đây?
	}{
		\begin{tikzpicture}[yscale=1,xscale=.8,>=stealth, font=\footnotesize, line join=round, line cap=round]
			\def\xmin{-3.5} \def\xmax{3}
			\def\ymin{-1.5} \def\ymax{2} 
			\draw[->] (\xmin,0)--(\xmax,0) node [below]{$x$};
			\draw[->] (0,\ymin)--(0,\ymax) node [left]{$y$};
			\node at (0,0) [below left]{$O$};
			\draw[smooth,samples=300] plot[domain=-3:2](\x,{((\x)+3)^.5-1});
			\draw[pattern=north west lines] plot[domain=-2:-3](\x,{((\x)+3)^.5-1})--(-3,0)--cycle
			plot[domain=-2:2](\x,{((\x)+3)^.5-1})--(2,0)--cycle;
			\node[above] at (-2,0) {$c$};
			\node[above] at (-3,0) {$a$};
			\node[below] at (2,0) {$b$};
			\node[left] at (0,1) {$(C)\colon y = f(x)$};
		\end{tikzpicture}
	}
	\choice
	{$S=\displaystyle\int\limits_a^c f(x) \mathrm{\,d}x + 
		\displaystyle\int\limits_c^b f(x) \mathrm{\,d}x$}
	{$S=\displaystyle\int\limits_a^b f(x) \mathrm{\,d}x$}
	{\True  $S=-\displaystyle\int\limits_a^c f(x) \mathrm{\,d}x + \displaystyle\int\limits_c^b f(x) \mathrm{\,d}x$}
	{$ S= \left|\displaystyle\int\limits_a^b f(x) \mathrm{\,d}x\right|$}
	
	\loigiai{
		Ta có $y=f(x)$ liên tục trên đoạn $\left[a; b\right]$.\\
		Dựa vào đồ thị ta có $\left|f(x)\right|=\heva{& -f(x), & a\le x \le c\\& f(x), & c< x \le b.}$\\
		Suy ra $S= \displaystyle\int\limits_a^b \left|f(x)\right| \mathrm{\,d}x = 
		\displaystyle\int\limits_a^c \left|f(x)\right| \mathrm{\,d}x +\displaystyle\int\limits_c^b \left|f(x)\right| \mathrm{\,d}x = -\displaystyle\int\limits_a^c f(x) \mathrm{\,d}x + \displaystyle\int\limits_c^b f(x) \mathrm{\,d}x$.
		
	}
\end{ex}	

%%%%%-------------Câu 20
\begin{ex}%[2D4H3-1]
	Diện tích phần hình phẳng gạch chéo trong hình vẽ bên dưới được tính theo công thức nào dưới đây?
	\begin{center}
		\begin{tikzpicture}[yscale=.8,xscale=.8,>=stealth, font=\footnotesize, line join=round, line cap=round]
			\def\xmin{-2} \def\xmax{3.5}
			\def\ymin{-2} \def\ymax{4} 
			\draw[->] (\xmin,0)--(\xmax,0) node [below]{$x$};
			\draw[->] (0,\ymin)--(0,\ymax) node [left]{$y$};
			\node [right] at (3,2){$y=x^2-2x-1$};
			\node [right] at (2.2,-2){$y=-x^2+3$};
			\clip (-2,-2) rectangle (3,3);
			\draw[smooth,samples=300] plot[domain=-1.4:3](\x,{(\x)^2-2*(\x)-1}) ;
			\draw[smooth,samples=300] plot[domain=-1.4:3](\x,{-(\x)^2+3});
			\draw[pattern=north west lines]plot[domain=-1:2](\x,{-(\x)^2+3})-- plot[domain=-1:2](\x,{(\x)^2-2*(\x)-1});
			\node at (0,0) [below right]{$O$};
			\draw[dashed] (-1,0) node [below]{$-1$}--(-1,2);
			\draw[dashed] (2,0) node [above]{$2$}--(2,-1);
		\end{tikzpicture}
	\end{center}
	
	\choice
	{$\displaystyle\int\limits_{-1}^{2} (-2x +2) \mathrm{\,d}x$}
	{$\displaystyle\int\limits_{-1}^{2} (2x -2) \mathrm{\,d}x$}
	{\True$\displaystyle\int\limits_{-1}^{2} (-2x^2 + 2x + 4) \mathrm{\,d}x$}
	{$\displaystyle\int\limits_{-1}^{2} (2x^2 -2x - 4) \mathrm{\,d}x$}
	
	\loigiai{
		Ta có 
		$S=\displaystyle\int\limits_{-1}^{2} \left|(-x^2 + 3) - (x^2 - 2x -1) \right| \mathrm{\,d}x = 
		\displaystyle\int\limits_{-1}^{2} \left|-2x^2 + 2x +4\right| \mathrm{\,d}x$.\\
		Vì $-2x^2 + 2x +4 > 0 , \forall x \in (-1; 2) $ nên ta có \\
		$S = \displaystyle\int\limits_{-1}^{2} \left|-2x^2 + 2x +4\right| \mathrm{\,d}x = \displaystyle\int\limits_{-1}^{2} (-2x^2 + 2x + 4) \mathrm{\,d}x$.
		
	}
\end{ex}	

%%%%%-------------Câu 21
\begin{ex}%[2D4N3-1]
	Cho hàm số  $y=f(x)$ liên tục trên $\mathbb{R}$. Gọi $S$  là diện tích hình phẳng giới hạn bởi các đường $y=f(x)$, $y=0$, $x= -1$, $x = 5$ (như hình vẽ bên dưới).
	\begin{center}
		\begin{tikzpicture}[scale=.7,>=stealth, font=\footnotesize, line join=round, line cap=round]
			\def\a{1/5} \def\b{-1} \def\c{-1/5} \def\d{1} % Hệ số
			\def\xmin{-2} \def\xmax{7}
			\def\ymin{-4} \def\ymax{2} 
			\draw[->] (\xmin,0)--(\xmax,0) node [below]{$x$};
			\draw[->] (0,\ymin)--(0,\ymax) node [left]{$y$};
			\node at (0,0) [below left]{$O$};
			\draw[smooth,samples=300] plot[domain=-1.7:5.4](\x,{\a*(\x)^3+\b*(\x)^2+\c*(\x)+\d}) node [left]{$y=f(x)$};
			\draw[pattern=north east lines] plot[domain=-1:5](\x,{\a*(\x)^3+\b*(\x)^2+\c*(\x)+\d});
			\node[above left] at (-1,0) {$-1$};
			\node[above right] at (1,0) {$1$};
			\node[below right] at (5,0) {$5$};
		\end{tikzpicture}
	\end{center}
	Mệnh đề nào sau đây đúng?
	\choice
	{$S=-\displaystyle\int\limits_{-1}^{1} f(x) \mathrm{\,d}x - \displaystyle\int\limits_{1}^{5} f(x) \mathrm{\,d}x$}
	{$S=\displaystyle\int\limits_{-1}^{1} f(x) \mathrm{\,d}x + \displaystyle\int\limits_{1}^{5} f(x) \mathrm{\,d}x$}
	{\True  $S=\displaystyle\int\limits_{-1}^{1} f(x) \mathrm{\,d}x - \displaystyle\int\limits_{1}^{5} f(x) \mathrm{\,d}x$}
	{$S=-\displaystyle\int\limits_{-1}^{1} f(x) \mathrm{\,d}x + \displaystyle\int\limits_{1}^{5} f(x) \mathrm{\,d}x$}
	\loigiai{
		Ta có $y=f(x)$ liên tục trên đoạn $\left[-1; 5\right]$.\\
		Dựa vào đồ thị ta có $\left|f(x)\right|=\heva{& f(x), & -1\le x \le 1\\& -f(x), & 1< x \le 5.}$\\
		Suy ra $S= \displaystyle\int\limits_{1}^{5} \left|f(x)\right| \mathrm{\,d}x = 
		\displaystyle\int\limits_{-1}^{1} \left|f(x)\right| \mathrm{\,d}x +\displaystyle\int\limits_{1}^{5} \left|f(x)\right| \mathrm{\,d}x = \displaystyle\int\limits_{-1}^{1} f(x) \mathrm{\,d}x - \displaystyle\int\limits_{1}^{5} f(x) \mathrm{\,d}x$.
		
	}
\end{ex}	
%%%%%-------------Câu 22
\begin{ex}%[2D4N3-1]
	Cho hàm số  $y=f(x)$ liên tục trên $\mathbb{R}$. Gọi $S$  là diện tích hình phẳng giới hạn bởi các đường $y=f(x)$, $y=0$, $x= -1$, $x = 2$ (như hình vẽ bên dưới).
	\begin{center}
		\begin{tikzpicture}[scale=1,>=stealth, font=\footnotesize, line join=round, line cap=round]
			\def\a{1} \def\b{-2} \def\c{-1} \def\d{2} % Hệ số
			\def\xmin{-2} \def\xmax{3}
			\def\ymin{-2} \def\ymax{3} 
			\draw[->] (\xmin,0)--(\xmax,0) node [below]{$x$};
			\draw[->] (0,\ymin)--(0,\ymax) node [left]{$y$};
			\node at (0,0) [below left]{$O$};
			\draw[smooth,samples=300] plot[domain=-1.2:2.5](\x,{\a*(\x)^3+\b*(\x)^2+\c*(\x)+\d}) node [left]{$y=f(x)$};
			\draw[pattern=north east lines] plot[domain=-1:2](\x,{\a*(\x)^3+\b*(\x)^2+\c*(\x)+\d});
			\node[above left] at (-1,0) {$-1$};
			\node[above right] at (1,0) {$1$};
			\node[below right] at (2,0) {$2$};
		\end{tikzpicture}
	\end{center}
	Mệnh đề nào sau đây đúng?
	\choice
	{$S=\displaystyle\int\limits_{-1}^{1} f(x) \mathrm{\,d}x + \displaystyle\int\limits_{1}^{2} f(x) \mathrm{\,d}x$}
	{$S=-\displaystyle\int\limits_{-1}^{1} f(x) \mathrm{\,d}x - \displaystyle\int\limits_{1}^{2} f(x) \mathrm{\,d}x$}
	{$S=-\displaystyle\int\limits_{-1}^{1} f(x) \mathrm{\,d}x + \displaystyle\int\limits_{1}^{2} f(x) \mathrm{\,d}x$}
	{\True  $S=\displaystyle\int\limits_{-1}^{1} f(x) \mathrm{\,d}x - \displaystyle\int\limits_{1}^{2} f(x) \mathrm{\,d}x$}
	\loigiai{
		Ta có $y=f(x)$ liên tục trên đoạn $\left[-1; 2\right]$.\\
		Dựa vào đồ thị ta có $\left|f(x)\right|=\heva{& f(x), & -1\le x \le 1\\& -f(x), & 1< x \le 2.}$\\
		Suy ra $S= \displaystyle\int\limits_{1}^{2} \left|f(x)\right| \mathrm{\,d}x = 
		\displaystyle\int\limits_{-1}^{1} \left|f(x)\right| \mathrm{\,d}x +\displaystyle\int\limits_{1}^{2} \left|f(x)\right| \mathrm{\,d}x = \displaystyle\int\limits_{-1}^{1} f(x) \mathrm{\,d}x - \displaystyle\int\limits_{1}^{2} f(x) \mathrm{\,d}x$.
		
	}
\end{ex}	
%%%%%-------------Câu 23
\begin{ex}%[2D4N3-1]
	\immini{
		Gọi $S$ là diện tích hình phẳng $(H)$ giới hạn bởi các đường $y=f(x)$, trục hoành và hai đường thẳng  $x=-1$, $x=2$. Đặt $a=\displaystyle\int\limits_{-1}^{0} f(x) \mathrm{\,d}x$, $b=\displaystyle\int\limits_{0}^{2} f(x) \mathrm{\,d}x$ (như hình vẽ bên). Mệnh đề nào sau đây đúng?  
		\choice
		{\True $S=b-a$}
		{$S=b+a$}
		{$S=-b+a$}
		{$S=-b-a$}
	}{
		\begin{tikzpicture}[yscale=.7,xscale=.8,>=stealth, font=\footnotesize, line join=round, line cap=round]
			\def\a{0.37} \def\b{0} \def\c{0.33} \def\d{0} % Hệ số
			\def\xmin{-2} \def\xmax{3}
			\def\ymin{-2.5} \def\ymax{4} 
			\draw[->] (\xmin,0)--(\xmax,0) node [below]{$x$};
			\draw[->] (0,\ymin)--(0,\ymax) node [left]{$y$};
			\node at (0,0) [below right]{$O$};
			\draw[smooth,samples=300] plot[domain=-1.5:2.1](\x,{\a*(\x)^3+\b*(\x)^2+\c*(\x)+\d});
			\draw[pattern=north east lines] plot[domain=0:-1](\x,{\a*(\x)^3+\b*(\x)^2+\c*(\x)+\d})--(-1,0)--cycle
			plot[domain=0:2](\x,{\a*(\x)^3+\b*(\x)^2+\c*(\x)+\d})--(2,0)--cycle;
			\node[above] at (-1,0) {$-1$};
			\node[below] at (2,0) {$2$};
		\end{tikzpicture}
	}
	\loigiai{
		Ta có $y=f(x)$ liên tục trên đoạn $\left[-1; 2\right]$.\\
		Dựa vào đồ thị ta có $\left|f(x)\right|=\heva{& -f(x), & -1\le x \le 0\\& f(x), & 0< x \le 2.}$\\
		Suy ra $S= \displaystyle\int\limits_{-1}^{2} \left|f(x)\right| \mathrm{\,d}x = 
		\displaystyle\int\limits_{-1}^{0} \left|f(x)\right| \mathrm{\,d}x +\displaystyle\int\limits_{0}^{2} \left|f(x)\right| \mathrm{\,d}x = -\displaystyle\int\limits_{-1}^{0} f(x) \mathrm{\,d}x + \displaystyle\int\limits_{0}^{2} f(x) \mathrm{\,d}x$.\\
		Hay $S=-a + b = b - a$.
		
	}
\end{ex}	
%%%%%-------------Câu 24
\begin{ex}%[2D4N3-1]
	\immini{
		Gọi $S$ là diện tích hình phẳng $(H)$ giới hạn bởi các đường $y=f(x)$, trục hoành và hai đường thẳng  $x=-3$, $x=2$. Đặt $a=\displaystyle\int\limits_{-3}^{1} f(x) \mathrm{\,d}x$, $b=\displaystyle\int\limits_{1}^{2} f(x) \mathrm{\,d}x$ (như hình vẽ bên). Mệnh đề nào sau đây đúng?  
	}{
		\begin{tikzpicture}[yscale=.7,xscale=.7,>=stealth, font=\footnotesize, line join=round, line cap=round]
			\def\a{-0.05} \def\b{-0.08} \def\c{1.07} \def\d{-.94} % Hệ số
			\def\xmin{-4} \def\xmax{3}
			\def\ymin{-4} \def\ymax{1} 
			\draw[->] (\xmin,0)--(\xmax,0) node [below]{$x$};
			\draw[->] (0,\ymin)--(0,\ymax) node [left]{$y$};
			\node at (0,0) [above left]{$O$};
			\draw[smooth,samples=300] plot[domain=-3:2](\x,{\a*(\x)^3+\b*(\x)^2+\c*(\x)+\d});
			\draw[pattern=north west lines] (-3,0)-- (-3,-3.5)-- plot[domain=-3:2](\x,{\a*(\x)^3+\b*(\x)^2+\c*(\x)+\d})-- (2,.5)--(2,0);
			\node[above] at (-3,0) {$-3$};
			\node[below] at (2,0) {$2$};
			\node[below] at (1,0) {$1$};
		\end{tikzpicture}
	}
	\choice
	{$S=a+b$}
	{$S=a-b$}
	{$S=-a-b$}
	{\True $S=b-a$}
	\loigiai{
		Ta có $y=f(x)$ liên tục trên đoạn $\left[-3; 2\right]$.\\
		Dựa vào đồ thị ta có $\left|f(x)\right|=\heva{& -f(x), & -3\le x \le 1\\& f(x), & 1< x \le 2.}$\\
		Suy ra $S= \displaystyle\int\limits_{-3}^{2} \left|f(x)\right| \mathrm{\,d}x = 
		\displaystyle\int\limits_{-3}^{1} \left|f(x)\right| \mathrm{\,d}x +\displaystyle\int\limits_{1}^{2} \left|f(x)\right| \mathrm{\,d}x = -\displaystyle\int\limits_{-3}^{1} f(x) \mathrm{\,d}x + \displaystyle\int\limits_{1}^{2} f(x) \mathrm{\,d}x$.\\
		Hay $S=-a + b = b - a$.
		
	}
\end{ex}	
%%%%%-------------Câu 25
% \begin{ex}%[2D4V3-1]
% 	\immini{
% 		Cho các số $p$, $q$  thỏa mãn các điều kiện $p>0$, $q>1$, $\dfrac{1}{p}+\dfrac{1}{q} = 1$ và các số dương $a, b$. Xét hàm số $y=x^{p-1}$ $(x>0)$ có đồ thị $(C)$. Gọi $S_1$  là diện tích hình phẳng giới hạn bởi  $(C)$, trục hoành, đường thẳng  $x=a$. Gọi $S_2$  là diện tích hình phẳng giới hạn bởi  $(C)$, trục tung, đường thẳng  $y=b$. Gọi $S$ là diện tích hình phẳng giới hạn bởi trục hoành, trục tung và hai đường thẳng  $x=a$,  $y=b$ (như hình vẽ bên). Khi so sánh  $S_1 + S_2$ và  $S$ ta nhận được bất đẳng thức nào trong các bất đẳng thức dưới đây?
% 	}{
% 		\begin{tikzpicture}[yscale=.7,xscale=.7,>=stealth, font=\footnotesize, line join=round, line cap=round]
% 			\def\a{1/8} \def\b{1} \def\c{0} \def\d{0} % Hệ số
% 			\def\xmin{-1} \def\xmax{4}
% 			\def\ymin{-1} \def\ymax{6} 
% 			\draw[->] (\xmin,0)--(\xmax,0) node [below]{$x$};
% 			\draw[->] (0,\ymin)--(0,\ymax) node [left]{$y$};
% 			\node at (0,0) [below left]{$O$};
% 			\draw[smooth,samples=300] plot[domain=0:2.2](\x,{\a*(\x)^3+\b*(\x)^2+\c*(\x)+\d}) node[above right]{$y=x^{p-1}$};
% 			\fill[pattern=north west lines] plot[domain=0:2](\x,{\a*(\x)^3+\b*(\x)^2+\c*(\x)+\d})--(2,5)--(2,0)--cycle;
% 			\fill[pattern=north east lines](0,4)-- plot[domain=0:1.802](\x,{\a*(\x)^3+\b*(\x)^2+\c*(\x)+\d})--cycle;
% 			\draw (-1,4)--(4,4) node[pos=.8,sloped,above]{$y=b$};
% 			\draw (2,-1)--(2,6)node[pos=.5,sloped,below]{$x=a$};
% 			\node[circle] at (.8,3){$S_2$};
% 			\node[circle] at (1.5,.5){$S_1$};
% 			\node[above left] at (0,4) {$b$};
% 			\node[below right] at (2,0) {$a$};
% 		\end{tikzpicture}
% 	}
% 	\choice
% 	{$\dfrac{a^p}{p}+\dfrac{b^q}{q}\le ab$}
% 	{$\dfrac{a^{p-1}}{p-1}+\dfrac{b^{q-1}}{q-1}\le ab$}
% 	{$\dfrac{a^{p+1}}{p+1}+\dfrac{b^{q+1}}{q+1}\le ab$}
% 	{\True $\dfrac{a^p}{p}+\dfrac{b^q}{q}\ge ab$}
% 	\loigiai{
% 		\begin{itemize}
% 			\item Diện tích hình phẳng giới hạn bởi trục hoành, trục tung và hai đường thẳng  $x=a$,  $y=b$ là $S = ab$.
% 			\item $S_1 = \displaystyle\int\limits_0^a x^{p-1} \mathrm{\,d}x=
% 			\left.\dfrac{x^p}{p}\right|_0^a = \dfrac{a^p}{p}$.
% 			\item Ta có $\dfrac{1}{p}+\dfrac{1}{q} = 1 \Leftrightarrow \dfrac{1}{q} = 1 - \dfrac{1}{p} = \dfrac{p - 1}{p} \Leftrightarrow q= \dfrac{p}{p-1}$. Tương tự $p=\dfrac{q}{q-1}$.\\
% 			Phương trình hoành độ giao điểm $ x^{p-1}=b\Leftrightarrow x= b^{\tfrac{1}{p-1}} \in (0;2)$. Suy ra\\
% 			$S_2 = \displaystyle\int\limits_0^{b^{\frac{1}{p-1}}} \left(b-x^{p-1}\right)\mathrm{\,d}x=
% 			\left.\left(bx -\dfrac{x^p}{p}\right)\right|_0^{b^{\frac{1}{p-1}}} $\\
% 			$= b\cdot b^{\frac{1}{p-1}}-
% 			\dfrac{\left( b^{\frac{1}{p-1}}\right)^p}{p}= b^{\frac{p}{p-1}}-
% 			\dfrac{b^{\frac{p}{p-1}}}{\dfrac{q}{q-1}} = b^q - \dfrac{ b^q(q-1)}{q} = \dfrac{b^q}{q}$.
% 			\item Dựa và hình vẽ đồ thị  ta có $S_1 + S_2 \ge S $.
% 			Vậy $\dfrac{a^p}{p}+\dfrac{b^q}{q}\ge ab $.
% 		\end{itemize}
% 	}
% \end{ex}	
%%%%%-------------Câu 26
\begin{ex}%[2D4N3-1]
	Diện tích phần hình phẳng được gạch sọc trong hình vẽ sau được tính theo công thức nào dưới đây?
	
	\begin{center}
		\begin{tikzpicture}[scale=1,>=stealth, font=\footnotesize, line join=round, line cap=round]
			\def\a{0} \def\b{1} \def\c{0} \def\d{-2} % Hệ số
			\def\xmin{-4} \def\xmax{4}
			\def\ymin{-3} \def\ymax{3} 
			\draw[->] (\xmin,0)--(\xmax,0) node [below]{$x$};
			\draw[->] (0,\ymin)--(0,\ymax) node [left]{$y$};
			\node at (0,0) [below left]{$O$};
			\draw[smooth,samples=300] plot[domain=-2:2](\x,{\a*(\x)^3+\b*(\x)^2+\c*(\x)+\d}) node [right]{$y=x^2 -2$};
			\draw[smooth,samples=300] plot[domain=0:3](\x,{-(\x)^.5})node [below]{$y=-\sqrt{|x|}$};
			\draw[smooth,samples=300] plot[domain=-3:0](\x,{-(-\x)^.5});
			\draw[pattern=north east lines] (-1,-1)--  plot[domain=-1:0](\x,{-(-\x)^.5})--(0,0)--plot[domain=0:-1](\x,{\a*(\x)^3+\b*(\x)^2+\c*(\x)+\d}) --cycle;
			\draw[pattern=north east lines] (0,0)--  plot[domain=0:1](\x,{-(\x)^.5})--(1,-1)--plot[domain=1:0](\x,{\a*(\x)^3+\b*(\x)^2+\c*(\x)+\d}) --cycle;
			\foreach \x in {-3,-2,-1,1,2,3} \draw[fill] (\x,0) circle (1pt) node [above] { $\x$};
			\foreach \y in {-2,1,2} \draw[fill] (0,\y) circle (1pt) node [ below left] { $\y$};
			\draw[dashed] (-1,0)--(-1,-1) (1,0)--(1,-1);
		\end{tikzpicture}
	\end{center}
	\choice
	{$\displaystyle\int\limits_{-1}^{1} \left( x^2 -2 + \sqrt{|x|}\right)\mathrm{\,d}x$}
	{$\displaystyle\int\limits_{-1}^{1} \left( x^2 -2 - \sqrt{|x|}\right)\mathrm{\,d}x$}
	{$\displaystyle\int\limits_{-1}^{1} \left( -x^2 + 2 + \sqrt{|x|}\right)\mathrm{\,d}x$}
	{\True $\displaystyle\int\limits_{-1}^{1} \left( -x^2 + 2 - \sqrt{|x|}\right)\mathrm{\,d}x$}
	\loigiai{
		Ta có $ -\sqrt{|x|}\ge x^2 -2$, $\forall x\in [-1; 1]$.\\
		Do đó $-\sqrt{|x|}- (x^2 -2) = -x^2  +2 -\sqrt{|x|} \ge 0, \forall x\in [-1; 1] $.\\
		Diện tích phần hình phẳng được gạch sọc trong hình vẽ là\\
		$\displaystyle\int\limits_{-1}^{1} \left|-\sqrt{|x|}- (x^2 -2)\right|\mathrm{\,d}x = \displaystyle\int\limits_{-1}^{1} \left( -x^2 + 2 - \sqrt{|x|}\right)\mathrm{\,d}x$
		
		
	}
\end{ex}

\Closesolutionfile{ans}
% \indapan{6}{ans/ans-2-C4B3CD1_10-19}


%\TNTF
\Opensolutionfile{ans}[ans/ans-2-C4B3CD1_10-19-DS]
\begin{ex}%[2D4H3-1]
	Cho hàm số  $y=f(x)$ liên tục trên  $\mathbb{R}$. Gọi $S$  là diện tích hình phẳng giới hạn bởi các đường  $y=f(x)$, $y=0$, $x=-1$, $x=4$ (như hình vẽ). Các mệnh đề sau đây đúng hay sai?
	\begin{center}
		\begin{tikzpicture}[scale=1,>=stealth, font=\footnotesize, line join=round, line cap=round]
			\def\a{1/3} \def\b{-4/3} \def\c{-1/3} \def\d{4/3} % Hệ số
			\def\xmin{-2} \def\xmax{5}
			\def\ymin{-3} \def\ymax{2.5} 
			\draw[->] (\xmin,0)--(\xmax,0) node [below]{$x$};
			\draw[->] (0,\ymin)--(0,\ymax) node [left]{$y$};
			\node at (0,0) [below left]{$O$};
			\draw[smooth,samples=300] plot[domain=-1.5:4.2](\x,{\a*(\x)^3+\b*(\x)^2+\c*(\x)+\d}) node [right]{$y=f(x)$};
			\draw[pattern=north east lines] plot[domain=-1:4](\x,{\a*(\x)^3+\b*(\x)^2+\c*(\x)+\d});
			\node[above left] at (-1,0) {$-1$};
			\node[above right] at (1,0) {$1$};
			\node[below right] at (4,0) {$4$};
		\end{tikzpicture}
	\end{center}
	\choiceTF
	{\True $S= \displaystyle\int\limits_{-1}^{1} f(x) \mathrm{\,d}x - \displaystyle\int\limits_{1}^{4} f(x) \mathrm{\,d}x$}
	{\True $S= \displaystyle\int\limits_{-1}^{1} \left|f(x)\right| \mathrm{\,d}x +\displaystyle\int\limits_{1}^{4} \left|f(x)\right| \mathrm{\,d}x$}
	{$S= \left|\displaystyle\int\limits_{-1}^{4} f(x)\mathrm{\,d}x\right|$}
	{$S= \displaystyle\int\limits_{-1}^{1} f(x) \mathrm{\,d}x + \displaystyle\int\limits_{1}^{4} f(x) \mathrm{\,d}x$}
	\loigiai{
		Ta có $y=f(x)$ liên tục trên đoạn $\left[-1; 4\right]$.\\
		Dựa vào đồ thị ta có $\left|f(x)\right|=\heva{& f(x), & -1\le x \le 1\\& -f(x), & 1< x \le 4.}$\\
		Suy ra
		$S= \displaystyle\int\limits_{-1}^{4} \left|f(x)\right| \mathrm{\,d}x = 
		\displaystyle\int\limits_{-1}^{1} \left|f(x)\right| \mathrm{\,d}x + \displaystyle\int\limits_{1}^{4} \left|f(x)\right| \mathrm{\,d}x = \displaystyle\int\limits_{-1}^{1} f(x) \mathrm{\,d}x - \displaystyle\int\limits_{1}^{4} f(x) \mathrm{\,d}x$.\\
		Do đó suy ra
		\begin{itemchoice}
			\itemch {\bf Đúng.}
			Vì $S= \displaystyle\int\limits_{-1}^{1} f(x) \mathrm{\,d}x - \displaystyle\int\limits_{1}^{4} f(x) \mathrm{\,d}x$ đúng.
			\itemch {\bf Đúng.}
			Vì $S= \displaystyle\int\limits_{-1}^{1} \left|f(x)\right| \mathrm{\,d}x +\displaystyle\int\limits_{1}^{4} \left|f(x)\right| \mathrm{\,d}x$ đúng.
			\itemch {\bf Sai.} 
			Vì $\left|f(x)\right|=\heva{& f(x), & -1\le x \le 1\\& -f(x), & 1< x \le 4.}$ nên $\left|\displaystyle\int\limits_{-1}^{4} f(x)\mathrm{\,d}x\right| \ne \displaystyle\int\limits_{-1}^{4} \left|f(x)\right| \mathrm{\,d}x$.
			\itemch {\bf Sai.} 
			Vì $S= \displaystyle\int\limits_{-1}^{1} f(x) \mathrm{\,d}x -\displaystyle\int\limits_{1}^{4} f(x) \mathrm{\,d}x$ sai.
		\end{itemchoice}
	}
\end{ex}

\begin{ex}%[2D4H3-1]
	Cho hình phẳng được gạch chéo trong hình bên dưới.
	\begin{center}
		\begin{tikzpicture}[yscale=.8,xscale=.8,>=stealth, font=\footnotesize, line join=round, line cap=round]
			\def\xmin{-2} \def\xmax{3.5}
			\def\ymin{-3} \def\ymax{3} 
			\draw[->] (\xmin,0)--(\xmax,0) node [below]{$x$};
			\draw[->] (0,\ymin)--(0,\ymax) node [left]{$y$};
			\node [right] at (3,1){$y=x^2-2x-2$};
			\node [right] at (2.2,-3){$y=-x^2+2$};
			\clip (-2,-3) rectangle (3,3);
			\draw[smooth,samples=300] plot[domain=-1.4:3](\x,{(\x)^2-2*(\x)-2}) ;
			\draw[smooth,samples=300] plot[domain=-1.4:3](\x,{-(\x)^2+2});
			\fill[pattern=north west lines]plot[domain=-1:2](\x,{-(\x)^2+2})-- plot[domain=-1:2](\x,{(\x)^2-2*(\x)-2});
			\node at (0,0) [below right]{$O$};
			\draw[dashed] (-1,0) node [below]{$-1$}--(-1,1);
			\draw[dashed] (2,0) node [above]{$2$}--(2,-2);
		\end{tikzpicture}
	\end{center}
	Các mệnh đề sau đây đúng hay sai?
	\choiceTF
	{\True Hình phẳng được gạch chéo trong hình trên được giới hạn các đồ thị $y=x^2-2x-2$, $y=-x^2+2$ và hai đường thẳng $x=-1$, $x=2$}
	{Diện tích hình phẳng gạch chéo trong hình vẽ là\\
		$S= \displaystyle\int\limits_{-1}^{2} \left|x^2 -2x -2\right|\mathrm{\,d}x+\displaystyle\int\limits_{-1}^{2} \left|-x^2 + 2\right|\mathrm{\,d}x$}
	{\True Hình phẳng được gạch chéo trong hình trên được giới hạn các đồ thị $y=x^2-2x-2$ và  $y=-x^2+2$}
	{\True Diện tích hình phẳng gạch chéo trong hình vẽ là $S=9$}
	\loigiai{
		\begin{itemchoice}
			\itemch {\bf Đúng.}
			Hình phẳng được gạch chéo trong hình trên được giới hạn các đồ thị $y=x^2-2x-2$, $y=-x^2+2$ và hai đường thẳng $x=-1$, $x=2$.
			\itemch {\bf Sai.}\\
			Vì $\displaystyle\int\limits_{-1}^{2} \left|x^2 -2x -2\right|\mathrm{\,d}x+
			\displaystyle\int\limits_{-1}^{2} \left|-x^2 + 2\right|\mathrm{\,d}x\ge
			\displaystyle\int\limits_{-1}^{2} \left|(x^2 -2x -2)- (-x^2 + 2) \right|\mathrm{\,d}x=S$
			\itemch {\bf Đúng.} 
			Phương trình hoành độ giao điểm\\
			$ x^2 -2x -2 = -x^2 +2 \Leftrightarrow
			2x^2 -2x - 4 = 0 \Leftrightarrow $ $x= -1$ hoặc $x=2$.\\
			Suy ra $S= \displaystyle\int\limits_{-1}^{2} \left|2x^2 -2x - 4\right|\mathrm{\,d}x$.
			\itemch {\bf Đúng}. 
			Vì $2x^2 -2x - 4<0, \forall x\in (-1;2)$.\\
			$S= \displaystyle\int\limits_{-1}^{2} \left|2x^2 -2x - 4\right|\mathrm{\,d}x=\displaystyle\int\limits_{-1}^{2} (-2x^2 + 2x + 4) \mathrm{\,d}x=\left.\left(\dfrac{2x^3}{3}+x^2+4x\right)\right|_{-1}^{2}= 9$.
		\end{itemchoice}
	}
\end{ex}


\begin{ex}%[2D4H3-1]
	Cho hình phẳng được gạch chéo trong hình bên dưới.
	\begin{center}
		\begin{tikzpicture}[yscale=.8,xscale=.8,>=stealth, font=\footnotesize, line join=round, line cap=round]
			\def\xmin{-3} \def\xmax{3}
			\def\ymin{-1} \def\ymax{5} 
			\node at (0,0) [below left]{$O$};
			\draw[->] (\xmin,0)--(\xmax,0) node [below]{$x$};
			\draw[->] (0,\ymin)--(0,\ymax) node [left]{$y$};
			\node [left] at (-2,4){$y=x^2$};
			\draw[smooth,samples=300] plot[domain=-2.2:2.2](\x,{(\x)^2}) ;
			\draw (1,-1)--(1,5) node[sloped,pos=.6,above]{$x=1$} 
			(2,-1)--(2,5) node[sloped,pos=.6,below]{$x=2$};
			\fill[pattern=north west lines](1,0)--(1,1)-- plot[domain=1:2](\x,{(\x)^2})--(2,4)--(2,0)--cycle;
			\foreach \x in {-1,-2,1,2} \draw[fill] (\x,0) circle (1pt) node [below left] { $\x$};
			\foreach \y in {1,2,3,4} \draw[fill] (0,\y) circle (1pt) node [left] { $\y$};
		\end{tikzpicture}
	\end{center}
	Các mệnh đề sau đây đúng hay sai?
	\choiceTF
	{\True Hình phẳng được gạch chéo trong hình trên được giới hạn các đồ thị $y=x^2$, $y=0$ và hai đường thẳng $x=1$, $x=2$}
	{\True  Diện tích hình phẳng gạch chéo trong hình vẽ là $S= \displaystyle\int\limits_{1}^{2} x^2 \mathrm{\,d}x$}
	{Diện tích hình phẳng gạch chéo trong hình vẽ là $S=\dfrac{4}{3}$}
	{Hình phẳng được gạch chéo trong hình trên được giới hạn đồ thị $y=x^2$ và hai đường thẳng $x=1$, $x=2$}
	\loigiai{
		\begin{itemchoice}
			\itemch {\bf Đúng}.
			Hình phẳng được gạch chéo trong hình trên được giới hạn các đồ thị $y=x^2$, $y=0$ và hai đường thẳng $x=1$, $x=2$.
			\itemch {\bf Đúng}.
			Vì $S= \displaystyle\int\limits_{1}^{2} \left|x^2\right| \mathrm{\,d}x = \displaystyle\int\limits_{1}^{2} x^2 \mathrm{\,d}x$.
			\itemch {\bf Sai}. 
			Vì $S= \displaystyle\int\limits_{1}^{2} x^2 \mathrm{\,d}x = \left.\dfrac{x^3}{3}\right|_1^2 = \dfrac{8}{3}-\dfrac{1}{3}=\dfrac{7}{3}$.
			\itemch {\bf Sai}. 
			Vì hình phẳng được giới hạn đồ thị $y=x^2$ và hai đường thẳng $x=1$, $x=2$ không xác định được diện tích.
		\end{itemchoice}
	}
\end{ex}



\begin{ex}%[2D4H3-1]
	Cho hình phẳng được gạch chéo trong hình bên dưới.
	\begin{center}
		\begin{tikzpicture}[yscale=.8,xscale=.8,>=stealth, font=\footnotesize, line join=round, line cap=round]
			\def\xmin{-1} \def\xmax{6}
			\def\ymin{-1} \def\ymax{6.5} 
			\draw[->] (\xmin,0)--(\xmax,0) node [below]{$x$};
			\draw[->] (0,\ymin)--(0,\ymax) node [left]{$y$};
			\node at (0,0) [below right]{$O$};
			\draw[smooth,samples=300] plot[domain=-.2:5.2](\x,{-(\x)^2+5*(\x)}) node[left]{$y=5x-x^2$};
			\draw[smooth,samples=300] plot[domain=-1:5.2](\x,{(\x)})node[below right]{$y=x$} ;
			\fill[pattern=north west lines] plot[domain=0:4](\x,{-(\x)^2+5*(\x)});
			\foreach \x/\y in {4/4} \draw[fill] (\x,\y) circle (1pt) node [right] { $(\x,\y)$};
		\end{tikzpicture}
	\end{center}
	Các mệnh đề sau đây đúng hay sai?
	\choiceTF
	{\True Hình phẳng được gạch chéo trong hình trên được giới hạn các đồ thị $y=5x-x^2$, $y=x$ và các đường thẳng $x=0$, $x=4$}
	{Diện tích hình phẳng gạch chéo trong hình vẽ là $S= \displaystyle\int\limits_{0}^{4} \left(x^2 - 4x\right) \mathrm{\,d}x$}
	{\True Diện tích hình phẳng gạch chéo trong hình vẽ là $S= \displaystyle\int\limits_{0}^{4} \left|x^2 - 4x \right| \mathrm{\,d}x $}
	{Diện tích hình phẳng gạch chéo trong hình vẽ $S= \dfrac{56}{3}$}
	\loigiai{
		\begin{itemchoice}
			\itemch {\bf Đúng.}
			Hình phẳng được gạch chéo trong hình trên được giới hạn các đồ thị $y=5x-x^2$, $y=x$ và các đường thẳng $x=0$, $x=4$.
			\itemch {\bf Sai}
			Phương trình hoành độ giao điểm\\
			$ x  = 5x -x^2 \Leftrightarrow
			x^2 -4x = 0 \Leftrightarrow $ $x= 0$ hoặc $x=4$.\\
			Vì $x^2-4x<0, \forall x\in (0;4)$.
			Do đó $S= \displaystyle\int\limits_{0}^{4} \left|x^2 - 4x \right| \mathrm{\,d}x = \displaystyle\int\limits_{0}^{4} \left(- x^2 + 4x \right) \mathrm{\,d}x$.
			\itemch {\bf Đúng}. 
			Vì $S= \displaystyle\int\limits_{0}^{4} \left|x^2 - 4x \right| \mathrm{\,d}x$.
			\itemch {\bf Sai}. 
			Vì $S= \displaystyle\int\limits_{0}^{4} \left|x^2 - 4x \right| \mathrm{\,d}x= \displaystyle\int\limits_{0}^{4} \left(- x^2 + 4x \right) \mathrm{\,d}x= \left.\left(-\dfrac{x^3}{3}+ 2x^2\right)\right|_{0}^{4}=\dfrac{32}{3}$.
		\end{itemchoice}
	}
\end{ex}


\begin{ex}%[2D4H3-1]
	Cho hình phẳng được gạch chéo trong hình bên dưới.
	\begin{center}
		\begin{tikzpicture}[yscale=1,xscale=1,>=stealth, font=\footnotesize, line join=round, line cap=round]
			\def\xmin{-1} \def\xmax{4}
			\def\ymin{-1} \def\ymax{3.5} 
			\draw[->] (\xmin,0)--(\xmax,0) node [below]{$x$};
			\draw[->] (0,\ymin)--(0,\ymax) node [left]{$y$};
			\node at (0,0) [below left]{$O$};
			\draw[smooth,samples=300] plot[domain=0.5:3](\x,{1+1/(\x)});
			\draw[pattern=north east lines](1,0)--(1,2)-- plot[domain=1:2](\x,{1+1/(\x)})--(2,1.5)--(2,0)--cycle;
			\foreach \x in {1,2} \draw[fill] (\x,0) circle (1pt) node [below] { $\x$};
			\foreach \y in {1,2} \draw[fill] (0,\y) circle (1pt) node [left] { $\y$};
			\node[above right] at (1,2) {$y=1+\dfrac{1}{x}$};
		\end{tikzpicture}
	\end{center}
	Các mệnh đề sau đây đúng hay sai?
	\choiceTF
	{Hình phẳng được gạch chéo trong hình trên được giới hạn đồ thị $y=1 + \dfrac{1}{x}$ và các đường thẳng $x=1$, $x=2$}
	{\True  Diện tích hình phẳng gạch chéo trong hình vẽ là $S= \displaystyle\int\limits_{1}^{2} \left(1 + \dfrac{1}{x}\right) \mathrm{\,d}x$}
	{Diện tích hình phẳng gạch chéo trong hình vẽ là $S=2$}
	{\True Diện tích hình phẳng gạch chéo trong hình vẽ là $S= 1+ \displaystyle\int\limits_{1}^{2} \dfrac{1}{x} \mathrm{\,d}x$}
	\loigiai{
		\begin{itemchoice}
			\itemch {\bf Sai.}
			Hình phẳng  giới hạn đồ thị
			$y=1 + \dfrac{1}{x}$ và các đường thẳng $x=1$, $x=2$ không xác định được diện tích.
			\itemch {\bf Đúng.}
			Vì $1+\dfrac{1}{x}>0, \forall x\in (1;2)$ nên $\left|1+\dfrac{1}{x}\right| = 1+\dfrac{1}{x}, \forall x\in (1;2)$.\\
			Do đó $S= \displaystyle\int\limits_{1}^{2} \left|1+\dfrac{1}{x}\right| \mathrm{\,d}x= \displaystyle\int\limits_{1}^{2} \left(1 + \dfrac{1}{x}\right) \mathrm{\,d}x$.
			\itemch {\bf Sai.}
			Vì $S= \displaystyle\int\limits_{1}^{2} \left(1 + \dfrac{1}{x}\right) \mathrm{\,d}x =\left(x+ \ln |x|\right)\Big|_{1}^{2} = 1+ \ln 2$.
			\itemch {\bf Đúng.} 
			Vì $ S=  \displaystyle\int\limits_{1}^{2} \left(1 + \dfrac{1}{x}\right) \mathrm{\,d}x = 1+ \displaystyle\int\limits_{1}^{2} \dfrac{1}{x} \mathrm{\,d}x = 1+ \ln 2$.
		\end{itemchoice}
	}
\end{ex}
%câu 33
\begin{ex}%[2D4N3-1]
	Cho hình phẳng được tô màu trong hình bên dưới
	\begin{center}
		\begin{tikzpicture}[font=\footnotesize, line join=round, line cap=round, >=stealth, scale = 0.8]
			\draw[->] (-1.5,0) --(0,0) node[below right]{$O$}--(1.5,0) node[below]{$x$};
			\draw[->] (0,-.7) --(0,3.2) node[right]{$y$};
			\draw[fill = black] (1,0) node[below]{$1$} circle (1pt);
			\draw[fill = black] (-1,0) node[below left]{$-1$} circle (1pt);
			\draw[fill = black] (0,1) node[left]{$1$} circle (1pt);
			\draw[fill = black] (0,0) circle (1pt);
			\draw[dashed](1,0)--(1,2.71) (-1,0)--(-1,0.37);
			%	\clip (-1.3,-.4) rectangle (1.5,4);
			\draw [samples=100, domain=-1.2:1.1] plot (\x, {e^(\x)});
			\fill[pattern = north west lines] (-1,0) -- plot[smooth,samples=100,domain=-1:1] (\x, {e^(\x)}) -- (1,0) -- cycle;
			\draw (1.3,3)node[above]{\scriptsize $ y=e^x $};
		\end{tikzpicture}
	\end{center}
	Các mệnh đề sau đây đúng hay sai?
	\choiceTF
	{Hình phẳng được tô màu trong hình vẽ trên được giới hạn bởi các đồ thị $ y=\mathrm{e}^x $; $ y=0 $; $ x=0 $; $ x=1 $}
	{\True Diện tích hình phẳng tô màu trong hình vẽ là $ \displaystyle\int\limits_{-1}^1 \mathrm{e}^x \mathrm{\,d}x$}
	{Diện tích hình phẳng tô màu trong hình vẽ là $ \displaystyle\int\limits_0^1 \mathrm{e}^x \mathrm{\,d}x$}
	{\True Hình phẳng được tô màu trong hình vẽ trên được giới hạn bởi các đồ thị $ y=\mathrm{e}^x $; $ y=0 $; $ x=-1 $; $ x=1 $}
	\loigiai{
		\begin{itemchoice}
			\itemch Sai. Vì hình phẳng được tô màu trong hình vẽ trên được giới hạn bởi các đồ thị $ y=\mathrm{e}^x $; $ y=0 $; $ x=-1 $; $ x=1 $.
			\itemch Đúng. Ta có $ S=\displaystyle\int\limits_{-1}^1 \mathrm{e}^x \mathrm{d}x $.
			\itemch Sai.
			\itemch Đúng.
	\end{itemchoice}}
\end{ex}

%CÂU 34
\begin{ex}%[2D4N3-1]
	Cho hình phẳng được tô màu trong hình bên dưới.
	\begin{center}
		\begin{tikzpicture}[font=\footnotesize, line join=round, line cap=round, >=stealth, scale = 0.8]
			\draw[->] (-.5,0) --(0,0) node[below right]{$O$}--(3.3,0) node[below]{$x$};
			\draw[->] (0,-.7) --(0,4) node[right]{$y$};
			\draw[fill = black] (2,0) node[below left]{$2$} circle (1pt);
			\draw[fill = black] (0,1) node[left]{$1$} circle (1pt);
			\draw[fill = black] (0,0) circle (1pt);
			\draw[dashed](2,0)--(2,3);
			\fill[pattern=north west lines] plot[domain=2:0](\x,{0.5^(\x)})-- plot[domain=0:2](\x,{(\x)+1}) --cycle;
			\draw [samples=100, domain=-.2:2.6] plot (\x, {(\x)+1});
			\draw [samples=100, domain=-.2:2.8] plot (\x, {0.5^(\x)});
			\draw (1.5,2.6)node[above,rotate=45]{\scriptsize $ y=x+1 $};
			\draw (3.3,.02)node[above]{\scriptsize $ y=\left(\dfrac{1}{2}\right)^x $};
		\end{tikzpicture}
	\end{center}
	Các mệnh đề sau đúng hay sai?
	\choiceTF
	{\True Hình phẳng được tô màu trong hình vẽ trên được giới hạn bởi các đồ thị $ y=x+1 $; $ y=\left(\dfrac{1}{2}\right)^x $; $ x=0 $; $ x=2 $}
	{Diện tích hình phẳng tô màu trong hình vẽ là $ \displaystyle\int\limits_0^2\left[\left(\dfrac{1}{2}\right)^x-x-1\right]\mathrm{\,d}x$}
	{\True Diện tích hình phẳng tô màu trong hình vẽ bằng $ S=4-\dfrac{3}{4\ln 2}$}
	{Hình phẳng được tô màu trong hình vẽ trên được giới hạn bởi các đồ thị $ y=x+1 $; $ y=\left(\dfrac{1}{2}\right)^x $; $ x=1 $; $ x=2 $}
	\loigiai{
		\begin{itemchoice}
			\itemch Đúng. Vì hình phẳng được tô màu trong hình vẽ trên được giới hạn bởi các đồ thị $ y=x+1 $; $ y=\left(\dfrac{1}{2}\right)^x $; $ x=0 $; $ x=2 $.
			\itemch Sai. Trên đoạn $ \left[0;2\right] $, đồ thị hàm số $ y=x+1 $ nằm trên đồ thị hàm số $ y=\left(\dfrac{1}{2}\right)^x $ nên với mọi $ x \in \left[0;2\right]$ ta có $ x+1 \ge \left(\dfrac{1}{2}\right)^x  \Rightarrow \left|\left(\dfrac{1}{2}\right)^x-(x+1)\right|=x+1-\left(\dfrac{1}{2}\right)^x$.\\
			Vậy diện tích hình phẳng tô màu là $ \displaystyle\int\limits_0^2\left[ x+1-\left(\dfrac{1}{2}\right)^x\right]\mathrm{\,d}x$.
			\itemch Đúng. Ta có $ S=\displaystyle\int\limits_0^2\left[ x+1-\left(\dfrac{1}{2}\right)^x\right]\mathrm{\,d}x=\left( \dfrac{x^2}{2}+x+\dfrac{2^{-x}}{\ln 2}\right) \Bigg|_0^2 =4+\dfrac{1}{4\ln 2}-\dfrac{1}{\ln 2}=4-\dfrac{3}{4\ln 2}$.
			\itemch Sai.
		\end{itemchoice}
	}
\end{ex}

%Câu 35
\begin{ex}%[2D4H3-1]
	Cho đồ thị hàm số $y=f(t)$ như hình vẽ.
	\begin{center}
		\begin{tikzpicture}[font=\footnotesize, line join=round, line cap=round, >=stealth, scale = 0.8]
			\draw[->] (-.5,0) --(0,0) node[below left]{$O$}--(5.3,0) node[below]{$t$};
			\draw[->] (0,-2.5) --(0,3) node[right]{$y$};
			\draw[fill = black] (1,0) node[below left]{$1$} circle (1pt) (2,0) node[below left]{$2$} circle (1pt) (3,0) node[below left]{$3$} circle (1pt) (4,0) node[below left]{$4$} circle (1pt) (5,0) node[below left]{$5$} circle (1pt);
			\draw[fill = black] (0,2) node[left]{$2$} circle (1pt) (0,-2) node[left] {$-2$} circle (1pt);
			\draw[line width = 1pt, red] (0,0)--(1,2)--(2,2);
			\draw[fill = black] (0,0) circle (1pt);
			\draw[line width = 1pt, red] 
			plot[domain=2:5, samples=100] (\x, {2/3*((\x)^3-9*(\x)^2+23*(\x)-15)});
			\draw[dashed] (1,0)--(1,2)--(0,2) (2,0)--(2,2);
		\end{tikzpicture}
	\end{center}
	Các mệnh đề sau đây đúng hay sai?
	\choiceTF
	{\True Diện tích hình phẳng được giới hạn các đồ thị hàm số $y=f(t)$, trục $O t$ và hai đường thẳng $t=0$; $t=1$ là $S=\dfrac{1}{2} \displaystyle\int\limits_{0}^{1} t \mathrm{\,d} t=\dfrac{1}{4}$}
	{\True Diện tích hình phẳng được giới hạn các đồ thị hàm số $y=f(t)$, trục $Ot$ và hai đường thẳng $t=1$; $t=2$ là $S=\displaystyle\int\limits_{1}^{2} 2 \mathrm{\,d}t=2$}
	{\True Tích phân $\displaystyle\int\limits_{2}^{3} f(x) \mathrm{\,d} x$ biểu thị cho phần diện tích của hình phẳng giới hạn các đồ thị hàm số $y=f(t)$, trục $O t$ và hai đường thẳng $t=2$; $ t=3$}
	{Tích phân $\displaystyle\int\limits_{3}^{5} f(x) \mathrm{\,d} x$ biểu thị cho phần diện tích của hình phẳng giới hạn các đồ thị hàm số $y=f(t)$, trục $O t$ và hai đường thẳng $t=3$; $ t=5$}
	\loigiai{
		\begin{itemchoice}
			\itemch Đúng. Vì đồ thị hàm số $y=f(t)$ trên đoạn $\left[0 ; 1\right]$ là $y=\dfrac{1}{2} t$. Do đó diện tích hình phẳng được giới hạn các đồ thị hàm số $y=f(t)$, trục $Ot$ và hai đường thẳng $t=0$; $t=1$ là $S=\dfrac{1}{2}\displaystyle\int\limits_{0}^{1} t \mathrm{\,d} t=\dfrac{1}{4}$.
			\itemch Đúng. Vì trên đoạn $ \left[1;2\right] $ đồ thị hàm số $y=f(t)=2$ nên hình phẳng được giới hạn bởi các đồ thị hàm số $ y=f(t) $, trục $O t$ và hai đường thẳng $t=1$; $t=2$ có diện tích là $S=\displaystyle\int\limits_{1}^{2} 2 \mathrm{\,d}t=2$.
			\itemch Đúng. Tích phân $\displaystyle\int\limits_{2}^{3} f(x) \mathrm{\,d} x=\displaystyle\int\limits_{2}^{3} f(t) \mathrm{\,d} t$ nên giá trị của tích phân $\displaystyle\int\limits_{2}^{3} f(t) \mathrm{\,d} t$ là diện tích của hình phẳng giới hạn các đồ thị hàm số $y=f(t)$, trục $Ot$ và hai đường thẳng $t=2$; $t=3$.
			\itemch Sai. Tích phân $\displaystyle\int\limits_{3}^{5} f(x) \mathrm{\,d} x=\displaystyle\int\limits_{3}^{5} f(t) \mathrm{\,d} t$.\\
			Diện tích hình phẳng được giới hạn các đồ thị hàm số $y=f(t)$, trục $O t$ và hai đường thẳng $t=3$; $t=5$ là $S=\displaystyle\int\limits_{3}^{5} \left|f(t)\right| \mathrm{\,d} t$.
		\end{itemchoice}
	}
\end{ex}

\Closesolutionfile{ans}
% \indapan{3}{ans/ans-2-C4B3CD1_10-19-DS}

\Opensolutionfile{ans}[ans/ans-C4B3CD1_20-26-KQ]
%\TNSA
%Câu 36

\begin{ex}%[2D4N3-1]
	Tính diện tích hình phẳng được tô màu trong hình bên dưới.
	\begin{center}
		\begin{tikzpicture}[font=\footnotesize, line join=round, line cap=round, >=stealth, scale = 1]
			\draw[->] (-.5,0) --(0,0) node[below left]{$O$}--(2.7,0) node[below]{$x$};
			\draw[->] (0,-.5) --(0,2.5) node[right]{$y$};
			\draw[fill = black] (1,0) node[below]{$1$} circle (1pt) (2,0) node[below]{$2$} circle (1pt);
			\draw[fill = black] (0,1) node[left]{$1$} circle (1pt)  (0,2) node[left]{$2$} circle (1pt);
			\draw[fill = black] (0,0) circle (1pt);
			\draw[dashed](0,2)--(2,2);
			\draw[line width =0.5 pt] (0,1) node[above right]{$ A $}--(2,2) node[above ]{$ B $}--(2,0) node[above right]{$ C $};
			\fill[pattern=north west lines] (0,1)--(2,2)--(2,0)--(0,0)--cycle;
			
		\end{tikzpicture}
	\end{center}
	\shortans{$ 3 $}
	\loigiai{
		\textbf{Cách 1:} Hình phẳng đã cho là hình thang vuông $ AOCB $, vuông tại $ A $, $ O $. Ta có
		$$ S=\dfrac{\left(AO+BC\right)\cdot OC}{2}=3.$$
		\textbf{Cách 2:} Đường thẳng $ AB $ đi qua hai điểm $ A\left(0;1\right) $ và $ B\left(2;2\right) $ nên đường thẳng $ AB $ có phương trình là $ y=\dfrac{1}{2}x+1 $.\\
		Hình phẳng đã cho giới hạn bởi đường thẳng $ y=\dfrac{1}{2}x+1 $, $ y=0 $, $ x=0 $, $ x=2 $ nên diện tích của hình phẳng là $ S=\displaystyle\int\limits_0^2 \left|\dfrac{1}{2}x+1 \right| \mathrm{\,d} x=3$.
	}
\end{ex}

%Câu 37
\begin{ex}%[2D4H3-1]
	Biết diện tích phần hình phẳng gạch chéo trong hình vẽ bên có diện tích là $ \dfrac{a}{b} $ với $ a$, $b \in \mathbb{Z} $ và phân số $ \dfrac{a}{b} $ tối giản. Tính tổng $ a+b $.
	\begin{center}
		\begin{tikzpicture}[font=\footnotesize, line join=round, line cap=round, >=stealth, scale = 1]
			\draw[->] (-.5,0) --(0,0) node[above left]{$O$}--(4.5,0) node[below]{$x$};
			\draw[->] (0,-.5) --(0,3.8) node[right]{$y$};
			\draw[fill = black] (1,0) node[below]{$1$} circle (1pt) (2,0) node[below]{$2$} circle (1pt);
			\draw[fill = black] (0,1) node[left]{$1$} circle (1pt)  (0,2) node[left]{$2$} circle (1pt);
			\draw[fill = black] (0,0) circle (1pt);
			\draw[dashed](1,0)--(1,1)--(0,1);
			\draw[line width = 0.5pt] plot[domain=0.2:3.8, samples=100] (\x, {((\x)-2)^2});
			\draw[line width = 0.5pt] plot[domain=-.5:3.6, samples=100] (\x, {(\x)});
			\fill[pattern=north west lines] (0,0)-- plot[domain=0:1](\x,{(\x)}) --(1,1)-- plot[domain=1:2](\x, {((\x)-2)^2}) --(2,0) --cycle;
			\draw (1.5,1.5)node[above,rotate=45]{\scriptsize $ y=x $};
			\draw (3.8,1)node[below]{\scriptsize $ y=\left(x-2\right)^2 $};
		\end{tikzpicture}
	\end{center}
	\shortans{$ 11 $}
	\loigiai{
		Dựa vào đồ thị, diện tích hình phẳng cần tìm là
		
		$S = \displaystyle\int\limits_{0}^{1} x \mathrm{\,d} x + \displaystyle\int\limits_{1}^{2}(x-2)^{2} \mathrm{\,d} x = \dfrac{1}{2} + \dfrac{1}{3} = \dfrac{5}{6}$.\\
		Vậy $ a=5 $; $ b=6 $ và $ a+b=11 $.
		
	}
\end{ex}

%Câu 38
\begin{ex}%[2D4H3-1]
	Biết diện tích phần tam giác cong $ OAB $ trong hình vẽ bên có diện tích là $ \dfrac{a}{b} $ với $ a$, $b \in \mathbb{Z} $ và phân số $ \dfrac{a}{b} $ tối giản. Tính hiệu $ b-a $.
	\begin{center}
		\begin{tikzpicture}[font=\footnotesize, line join=round, line cap=round, >=stealth, scale = 1]
			\draw[->] (-1.5,0) --(0,0) node[above left]{$O$}--(5.3,0) node[below]{$x$};
			\draw[->] (0,-1.5) --(0,4.5) node[right]{$y$};
			\foreach \x in {-1,1,2,3,4,5} \draw[fill] (\x,0) circle (1pt) node [below] { $\x$};
			\foreach \y in {-1,1,2,3,4} \draw[fill] (0,\y) circle (1pt) node [left] { $\y$};
			\draw[fill = red] (0,0) circle (1.2pt);
			\draw[line width = 0.5pt] plot[domain=-1.1:1.6, samples=100] (\x, {(\x)^3});
			\draw[line width = 0.5pt] plot[domain=-.1:4.1, samples=100] (\x, {((\x)-2)^2});
			\draw (1.5,3.5)node[left]{\scriptsize $ y=x^3 $};
			\draw (4,4.2)node[right]{\scriptsize $ y=x^2-4x+4 $};
			\draw[fill=red] (1,1) node[right]{$ A $} circle (1pt) (2,0) node[above right]{$ B $} circle (1pt);
		\end{tikzpicture}
	\end{center}
	\shortans{$ 5 $}
	\loigiai{
		Dựa vào hình vẽ ta thấy hình phẳng cần tính diện tích gồm 2 phần.\\
		Phần 1: Hình phẳng giới hạn bởi đồ thị hàm số $y=x^3$, trục $Ox$, $x=0$, $x=1$.\\
		Phần 2: Hình phẳng giới hạn bởi đồ thị hàm số $y=x^2-4 x+4$, trục $O x$, $x=1$, $x=2$.\\
		Do đó diện tích cần tính là 
		
		$S=\displaystyle\int\limits_{0}^{1}\left|x^3\right| \mathrm{\,d} x+\displaystyle\int\limits_{1}^{2}\left|x^2-4 x+4\right| \mathrm{\,d} x = \displaystyle\int\limits_{0}^{1} x^3 \mathrm{\,d} x + \displaystyle\int\limits_{1}^{2}\left(x^2-4 x+4\right) \mathrm{\,d} x = \dfrac{7}{12}$.\\
		Vậy $ a=7 $, $ b=12 $ và $ b-a=5 $.
	}
\end{ex}

%Câu 39
\begin{ex}%[2D4H3-1]
	Hình vuông $OABC$ có cạnh bằng $4$ được chia thành hai phần bởi đường cong $(C)$ có phương trình $y=\dfrac{1}{4} x^{2}$. Gọi $S_2$, $S_2$ lần lượt là diện tích của phần không tô màu và phần tô màu như hình vẽ bên dưới. Tỉ số $\dfrac{S_1}{S_2}$ bằng bao nhiêu?
	\begin{center}
		\begin{tikzpicture}[font=\footnotesize, line join=round, line cap=round, >=stealth, scale = 0.8]
			\draw[->] (-.5,0) --(0,0) node[above left]{$O$}--(5.3,0) node[below]{$x$};
			\draw[->] (0,-1) --(0,5) node[right]{$y$};
			\foreach \x in {2,4} \draw[fill] (\x,0) circle (1pt) node [below] { $\x$};
			\foreach \y in {2,4} \draw[fill] (0,\y) circle (1pt) node [left] { $\y$};
			\draw (0,0) circle (1.2pt);
			\draw[line width = 0.5pt] plot[domain=4:-.2, samples=100] (\x, {0.25*(\x)^2});
			\draw (4,0)--(4,4)--(0,4);
			\fill[gray] (0,0)-- plot[domain=0:4](\x,{0.25*(\x)^2})--(4,0)--cycle;
			\draw[fill=black] (0,4) node[above right]{$ A $} circle (1pt) (4,4) node[above right]{$ B $} circle (1pt) (4,0)node[above right]{$ C $} circle (1pt);
			\draw (1,2.5) node[right]{$ S_1 $} (3,1) node[above]{$ S_2 $};
		\end{tikzpicture}
	\end{center}
	\shortans{$ 2 $}
	\loigiai{Ta có diện tích hình vuông $O A B C$ là $ 16 $ và bằng $S_1+S_2$.\\
		Ta có $S_2=\displaystyle\int\limits_{0}^{4} \dfrac{1}{4} x^2 \mathrm{\,d} x =\left.\dfrac{x^3}{12}\right|_{0} ^{4}=\dfrac{16}{3} \Rightarrow \dfrac{S_1}{S_2}=\dfrac{16-S_2}{S_2}=\dfrac{16-\dfrac{16}{3}}{\dfrac{16}{3}}=2$.}
\end{ex}


%câu 40
\begin{ex}%[2D4V3-1]
	Cho hình thang cong $(H)$ giới hạn bởi các đường $y=\mathrm{e}^{x}$, $y=0$, $x=0$, $x=\ln 4$. Đường thẳng $x=k$, $(0<k<\ln 4)$ chia $(H)$ thành hai phần có diện tích là $S_1$ và $S_2$ như hình vẽ bên. Tìm $k$ để $S_1=2 S_2$ (làm tròn kết quả đến hàng phần chục).
	\begin{center}
		\begin{tikzpicture}[font=\footnotesize, line join=round, line cap=round, >=stealth, scale =1]
			\draw[->] (-1,0) --(0,0) node[below left]{$O$}--(2.5,0) node[below]{$x$};
			\draw[->] (0,-.7) --(0,4.6) node[right]{$y$};
			\draw[fill = black] (.8,0) node[below left]{$k$} circle (1pt);
			\draw[fill = black] (1.4,0) node[below right]{$\ln 4$} circle (1pt);
			\draw[fill = black] (0,1) node[above left]{$1$} circle (1pt);
			\draw[fill = black] (0,0) circle (1pt);
			\draw (.8,-.7)--(.8,4.4) (1.4,-.7)--(1.4,4.4);
			\draw [samples=100, domain=-.9:1.5] plot (\x, {e^(\x)});
			\fill[color=gray!10!black,shading=axis,opacity=0.2] (0,0) -- plot[smooth,samples=100,domain=0:1.4] (\x, {e^(\x)}) -- (1.4,0) -- cycle;
			\draw (0.2,.3) node[right]{$ S_1 $} (1,1) node[above]{$ S_2 $};
		\end{tikzpicture}
	\end{center}
	\shortans{$1,1$}
	\loigiai{
		Diện tích hình thang cong $(H)$ giới hạn bởi các đường $y=\mathrm{e}^{x}$, $y=0$, $x=0$, $x=\ln 4$ là
		$$S=\displaystyle\int\limits_{0}^{\ln 4} \mathrm{e}^{x} \mathrm{\,d} x = \mathrm{e}^{x}\Bigg|_{0}^{\ln 4}=\mathrm{e}^{\ln 4}-\mathrm{e}^{0}=4-1=3.$$\\
		Ta có $S=S_1+S_2=S_1+\dfrac{1}{2} S_1=\dfrac{3}{2} S_1$. Suy ra $S_1=\dfrac{2 S}{3}=\dfrac{2 \cdot 3}{3}=2$.\\
		Vì $S_1$ là phần diện tích được giới hạn bởi các đường $y=\mathrm{e}^{x}$, $y=0$, $x=0$, $x=k$ nên\\
		$
		2=S_1=\displaystyle\int\limits_{0}^{k} \mathrm{e}^{x} \mathrm{\,d} x=\mathrm{e}^{x}\Bigg|_{0}^{k}=\mathrm{e}^{k}-\mathrm{e}^{0}=\mathrm{e}^{k}-1$.\\
		Do đó $\mathrm{e}^{k}=3 \Leftrightarrow k=\ln 3 \approx 1{,}1$.}
\end{ex}

%câu 41
% \begin{ex}%[2D4V3-1]
% 	Cho hình phẳng $(H)$ giới hạn bởi các đường $y=\left|x^{2}-1\right|$ và $y=k$, với $0<k<1$. Tìm $k$ để diện tích hình phẳng $(H)$ gấp hai lần diện tích hình phẳng được kẻ sọc ở hình vẽ bên (làm tròn kết quả đến hàng phần trăm).
% 	\begin{center}
% 		\begin{tikzpicture}[font=\footnotesize, line join=round, line cap=round, >=stealth, scale =1]
% 			\draw[->] (-2,0) --(0,0) node[below left]{$O$}--(2,0) node[below]{$x$};
% 			\draw[->] (0,-3) --(0,3.3) node[right]{$y$};
% 			\draw[fill = black] (1,0) node[below left]{$1$} circle (1pt);
% 			\draw[fill = black] (0,1) node[above left]{$1$} circle (1pt);
% 			\draw[fill = black] (0,0) circle (1pt);
% 			\draw (-2,.4)--(2,.4) node[above right]{$ y=k $};
% 			\draw [samples=100, domain=-1:1] plot (\x, {-(\x)^2+1});
% 			\draw [samples=100, domain=-1:-1.8] plot (\x, {(\x)^2-1});
% 			\draw [samples=100, domain=1:1.8] plot (\x, {(\x)^2-1});
% 			\draw[dashed,samples=100,domain=-1:-1.8] plot (\x, {-(\x)^2+1});
% 			\draw[dashed,samples=100,domain=1:1.8] plot (\x, {-(\x)^2+1});
% 			\fill[pattern=north west lines] (-.77,.4) -- plot[smooth,samples=100,domain=-.77:.77] (\x, {-(\x)^2+1}) -- (.77,.4) -- cycle;
			
% 		\end{tikzpicture}
% 	\end{center}
% 	\shortans{$ 0{,}59 $}
% 	\loigiai{\begin{center}
% 			\begin{tikzpicture}[font=\footnotesize, line join=round, line cap=round, >=stealth, scale =1]
% 				\draw[->] (-2,0) --(0,0) node[below left]{$O$}--(2,0) node[below]{$x$};
% 				\draw[->] (0,-3) --(0,3.3) node[right]{$y$};
% 				\draw[fill = black] (1,0) node[below left]{$1$} circle (1pt);
% 				\draw[fill = black] (0,1) node[above left]{$1$} circle (1pt);
% 				\draw[fill = black] (0,0) circle (1pt);
% 				\draw (-2,.4)--(2,.4) node[above right]{$ y=k $};
% 				\draw [samples=100, domain=-1:1] plot (\x, {-(\x)^2+1});
% 				\draw [samples=100, domain=-1:-1.8] plot (\x, {(\x)^2-1});
% 				\draw [samples=100, domain=1:1.8] plot (\x, {(\x)^2-1});
% 				\draw[dashed,samples=100,domain=-1:-1.8] plot (\x, {-(\x)^2+1});
% 				\draw[dashed,samples=100,domain=1:1.8] plot (\x, {-(\x)^2+1});
% 				\fill[pattern=dots] (0,.4) -- plot[smooth,samples=100,domain=0:.77] (\x, {-(\x)^2+1}) -- (.77,.4) -- cycle;
% 				\fill[pattern=north west lines] (.77,.4)--plot[smooth,samples=100,domain=.77:1] (\x, {-(\x)^2+1}) -- plot[smooth,samples=100,domain=1:1.18] (\x, {(\x)^2-1}) -- (1.18,.4) -- cycle;
% 				\draw (0.77,.4) node[above]{$ A $} circle (1pt) (1.18,.4) node[above right]{$ B $} circle (1pt);
% 			\end{tikzpicture}
% 		\end{center}
% 		Gọi $S$ là diện tích hình phẳng $(H)$. Lúc dó $S=2 S_1+2 S_2$, trong đó $S_1$ là diện tích phần chấm bi và $S_2$ là diện tích phần gạch sọc trong hình vẽ bên.\\
% 		Gọi $A$, $ B$ là các giao điếm có hoành độ dương của đường thẳng $y=k$ và đồ thị hàm số $y=\left|x^{2}-1\right|$, trong đó $A(\sqrt{1-k}; k)$ và $B(\sqrt{1+k}; k)$.\\
% 		Theo yêu cầu bài toán
% 		\begin{eqnarray*}
% 			& & S=2 \cdot 2 S_1 \\
% 			& \Leftrightarrow & S_1=S_2 \\
% 			& \Leftrightarrow &\displaystyle\int\limits_{0}^{\sqrt{1-k}} \left(1-x^{2}-k\right) \mathrm{\,d} x =\displaystyle\int\limits_{\sqrt{1-k}}^{1}\left(k-1+x^{2}\right) \mathrm{\,d} x+\displaystyle\int\limits_{1}^{\sqrt{1+k}}\left(k-x^{2}+1\right) \mathrm{\,d} x \\
% 			& \Leftrightarrow & (1-k) \sqrt{1-k}-\dfrac{1}{3}(1-k) \sqrt{1-k}=\dfrac{1}{3}-(1-k)-\dfrac{1}{3}(1-k) \sqrt{1-k}+(1-k) \sqrt{1-k}+(1+k) \sqrt{1+k}-\dfrac{1}{3}(1+k) \sqrt{1+k}-(1+k)+\dfrac{1}{3} \\
% 			& \Leftrightarrow &\dfrac{2}{3}(1+k) \sqrt{1+k}=\dfrac{4}{3} \\
% 			& \Leftrightarrow & (\sqrt{1+k})^{3}=2\\
% 			& \Leftrightarrow & k=\sqrt[3]{4}-1\approx 0{,}59.
% 	\end{eqnarray*}}
	
% \end{ex}
\Closesolutionfile{ans}
% \indapan{6}{ans/ans-C4B3CD1_20-26-KQ}
% %\setcounter{chude}{1}
\begin{dang}{	THỂ TÍCH KHỐI TRÒN XOAY}
\end{dang}
% \begin{tomtat}
% 	\subsection{Thế tích của vật thế}
% 	\begin{center}
% 	\begin{tikzpicture}[>=stealth, scale=0.8]
% 		\draw plot[smooth,tension=.65] coordinates{(1,2) (2.5,2.3) (3.5,2.2)};
% 		\draw[dashed] plot[smooth,tension=.65] coordinates{(3.5,2.2) (4,2)};
% 		\draw plot[smooth,tension=.65] coordinates{(4,2) (5,2.2) (5.5,2.1)};
% 		\draw[dashed] plot[smooth,tension=.65] coordinates{(5.5,2.1) (6,2)};
% 		\draw plot[smooth,tension=.65] coordinates{(1,1) (2.3,0.5) (3.5,0.8)};
% 		\draw[dashed] plot[smooth,tension=.65] coordinates{(3.5,0.8) (4,1)};
% 		\draw plot[smooth,tension=.65] coordinates{(4,1) (5,0.7) (5.5,0.8)};
% 		\draw[dashed] plot[smooth,tension=.65] coordinates{(5.5,0.8) (6,1)};
% 		\draw[dashed] (1,1) arc (-90:90:.2 and 0.5);
% 		\draw (1,2) arc (90:270:.2 and 0.5);
% 		\draw[dashed] (4,1) arc (-90:90:.2 and 0.5);
% 		\draw (4,2) arc (90:270:.2 and 0.5);
% 		\draw (6,1) arc (-90:270:.2 and 0.5);
% 		\fill[pattern=north east lines] (4,1) arc (-90:90:.2 and 0.5)--(4,2) arc (90:270:.2 and 0.5)--cycle;
% 		\draw (-.5,0)--(0.5,0) (1,0)--(3.5,0) (4,0)--(5.5,0);
% 		\draw[dashed] (0.5,0)--(1,0) (3.5,0)--(4,0) (5.5,0)--(6,0);
% 		\draw[->] (6,0)--(7,0)node[below]{$x$};
% 		\draw (0.5,-1)--(0.5,3)--(1.5,3.5)--(1.5,2.2) (1.5,.8)--(1.5,-0.5)--(0.5,-1);
% 		\draw[dashed](1.5,2.2)--(1.5,.8);
% 		\draw[dashed] (1,1)--(1,0)node[below]{$a$};
% 		\coordinate (A) at (0.5,3);
% 		\coordinate (B) at (1.5,3.5);
% 		\coordinate (C) at (1.5,2.2);
% 		%\tkzMarkAngle[size=.6](A,B,C);
% 		\draw pic[draw=black, angle eccentricity=1.6, angle radius=0.5cm]{angle=A--B--C};
% 		\draw (1.3,3.2) node {\footnotesize $P$};
% 		\draw (3.5,-1)--(3.5,3)--(4.5,3.5)--(4.5,2) (4.5,1)--(4.5,-0.5)--(3.5,-1);
% 		\draw[dashed](4.5,2)--(4.5,1);
% 		\draw[dashed] (4,1)--(4,0)node[below]{$x$};
% 		\coordinate (D) at (3.5,3);
% 		\coordinate (E) at (4.5,3.5);
% 		\coordinate (F) at (4.5,2);
% 	%	\tkzMarkAngle[size=.6](D,E,F);
% 		\draw pic[draw=black, angle eccentricity=1.6, angle radius=0.5cm]{angle=D--E--F};
% 		\draw (4.3,3.2) node {\footnotesize $R$};
% 		\draw (5.5,-1)--(5.5,3)--(6.5,3.5)--(6.5,-0.5)--(5.5,-1);
% 		\draw[dashed] (6,1)--(6,0)node[below]{$b$};
% 		\coordinate (G) at (5.5,3);
% 		\coordinate (H) at (6.5,3.5);
% 		\coordinate (K) at (6.5,-0.5);
% 	%	\tkzMarkAngle[size=.6](G,H,K);
% 		\draw pic[draw=black, angle eccentricity=1.6, angle radius=0.5cm]{angle=G--H--K};	\draw (6.3,3.2) node {\footnotesize $Q$};
% 		\draw (0,.3) node {$O$};
% 		\fill (0,0) circle(1pt);
% 		\draw[->] (4,1.5)--(4.7,1.7) node[right] {\scriptsize $S(x)$};
% 	\end{tikzpicture}
% 	\end{center}
% 	Trong không gian, cho một vật thể nằm trong khoảng không gian giữa hai mặt phẳng $(P)$ và $(Q)$ cùng vuông góc với trục $O x$ tại các điểm $a$ và $b$. Mặt phẳng vuông góc với trục $O x$ tại điểm $x(a \leq x \leq b)$ cắt vật thể theo mặt cắt có diện tích $S(x)$. Khi đó, nếu $S(x)$ là hàm số liên tục trên $\left[a ; b\right]$ thì thể tích của vật thể được tính bởi công thức
% 	$$
% 	V=\displaystyle\int\limits_a^b S(x) \mathrm{\,d} x.
% 	$$
% \subsection{Thế tích khối tròn xoay}
% 	\begin{center}
% 		\begin{tikzpicture}[line join=round, line cap=round,>=stealth,thick,scale=.4]
% 		\tikzset{label style/.style={font=\normalsize}}
% 		%%Nhập giới hạn đồ thị và hàm số cần vẽ
% 		\def \xmin{-.5}
% 		\def \xmax{11}
% 		\def \ymin{-4}
% 		\def \ymax{4.5}
% 		%\draw[xstep=1 cm, ystep=1 cm,gray,thin] (\xmin,\ymin) grid (\xmax,\ymax);
% 		\def \hamso{0-0.01864083398323228*((\x)-1.0)^(3.0)+0.35126127715584465*((\x)-1.0)^(2.0)-1.5117402856674746*((\x)-1.0)+3.106314636847822}
% 		%%Tự động
% 		\draw[->] (\xmin,0)--(11,0) node[below left] {$x$};
% 		\draw[->] (0,\ymin)--(0,\ymax) node[below left] {$y$};
% 		\draw (0,0) node [below left] {$O$};
% 		\draw[dashed] (2.04,-1.89) arc(-90:90:.5 cm and 1.89 cm);
% 		\draw (2.04,1.89) arc(90:270:.5 cm and 1.89 cm);
		
% 		%\draw[dashed] (5.59,-1.77) arc(-90:90:.5 cm and 1.77 cm);
% 		%	\draw (5.59,1.77) arc(90:270:.5 cm and 1.77 cm);
		
% 		\draw (9,0) ellipse (1 cm and 3.96 cm);
		
% 		\draw[fill=black] (2.04,0) node [below] {$a$} circle (1.2pt);
% 		\draw[fill=black] (9,0) node [below] {$b$} circle (1.2pt);
% 		\draw[fill=black] (5.5,1.8) node [above,rotate=30] {\scriptsize $y=f(x)$};
% 		%%Tự động
% 		\begin{scope}
% 			\clip (\xmin+0.01,\ymin+0.01) rectangle (\xmax-0.01,\ymax-0.01);
% 			\draw[samples=350,domain=1:9.,smooth,variable=\x] plot (\x,{\hamso});
% 			\draw[samples=350,domain=1:9,smooth,variable=\x] plot (\x,{-0+0.01864083398323228*((\x)-1.0)^(3.0)-0.35126127715584465*((\x)-1.0)^(2.0)+1.5117402856674746*((\x)-1.0)-3.106314636847822});		
% 			\draw[pattern=north west lines,opacity=0.5] (2.04,0)--(2.04,1.89)plot[domain=2.04:9] (\x,{0-0.01864083398323228*((\x)-1.0)^(3.0)+0.35126127715584465*((\x)-1.0)^(2.0)-1.5117402856674746*((\x)-1.0)+3.106314636847822})--(9.0,0)--(2.04,0);
% 		\end{scope}
% 	\end{tikzpicture}
% 	\end{center}
% 	Cho hàm số $y=f(x)$ liên tục, không âm trên $\left[a ; b\right]$. Hình phẳng $(H)$ giới hạn bởi đồ thị hàm số $y=f(x)$, trục hoành $O x$ và hai đường thẳng $x=a$ và $x=b$ quay quanh trục $O x$ tạo thành một khối tròn xoay có thể tích bằng
% 	$$
% 	V=\pi \displaystyle\int\limits_a^b\left[f(x)\right]^2 \mathrm{\,d} x
% 	$$
% \end{tomtat}
%\TN
\Opensolutionfile{ans}[ans/ans-2-C4B3CD2-lc]
\begin{ex}%[2D4N3-3]
Viết công thức tính thể tích $V$ của khối tròn xoay được tạo ra khi quay hình thang cong, giới hạn bới đồ thị hàm số $y=f(x)$, trục $O x$ và hai đường thẳng $x=a$, $x=b$, $(a<b)$ xung quanh trục $O x$.
\choice
{$V=\displaystyle\int\limits_a^b \left|f(x)\right| \mathrm{\,d} x$}
{\True $V=\pi \displaystyle\int\limits_a^b f^2(x) \mathrm{\,d} x $}
{$V=\displaystyle\int\limits_a^b f^2(x) \mathrm{\,d} x$}
{$V=\pi \displaystyle\int\limits_a^b f(x) \mathrm{\,d} x$}
\loigiai{
Theo lí thuyết.
}
\end{ex}
\begin{ex}%[2D4N3-4]
	Cắt một vật thể bởi hai mặt phẳng vuông góc với trục $O x$ tại $x=1$ và $x=2$. Một mặt phẳng tùy ý vuông góc với trục $O x$ tại điểm có hoành độ $x$, $(1 \leq x \leq 2)$ cắt vật thể đó có diện tích $S(x)=2024 x$. Tính thể tích của phần vật thể giới hạn bởi hai mặt phẳng trên.
	\choice
	{\True $V=3036$}
	{$V=3036 \pi$}
	{$V=1518$}
	{$V=1518 \pi$}
\loigiai{Thể tích vật thể là $ V=\displaystyle\int\limits_1^2 2024x \mathrm{\,d}x=3036 $.}
\end{ex}

\begin{ex}%[2D4H3-4]
Cắt một vật thể bởi hai mặt phẳng vuông góc với trục $O x$ tại $x=1$ và $x=3$. Một mặt phẳng tùy ý vuông góc với trục $O x$ tại điểm có hoành độ $x$, $(1 \leq x \leq 3)$ cắt vật thể đó theo thiết diện là một hình chữ nhật có độ dài hai cạnh là $3 x$ và $3 x^2-2$. Tính thể tích của phần vật thể giới hạn bởi hai mặt phẳng trên.
\choice
{\True $V=156$}
{$V=156 \pi$}
{$ V=312 $}
{$V=312 \pi$}
\loigiai{Diện tích thiết diện là $S(x)=3 x \cdot \left(3 x^2-2\right)=9 x^3-6 x$.\\
Thể tích vật thể là $V=\displaystyle\int\limits_1^3\left(9 x^3-6 x\right) \mathrm{\,d} x=156$.
}
\end{ex}

\begin{ex}%[2D4N3-3]
Gọi $D$ là hình phẳng giới hạn bởi các đường $y=\mathrm{e}^{3 x}$, $y=0$, $x=0$ và $x=1$. Thể tích của khối tròn xoay tạo thành khi quay $D$ quanh trục $O x$ bằng
\choice
{$\pi \displaystyle\int\limits_0^1 \mathrm{e}^{3 x} \mathrm{\,d} x$}
{$\displaystyle\int\limits_0^1 \mathrm{e}^{6 x} \mathrm{\,d} x$}
{\True $\pi \displaystyle\int\limits_0^1 \mathrm{e}^{6 x} \mathrm{\,d} x$}
{$\displaystyle\int\limits_0^1 \mathrm{e}^{3 x} \mathrm{\,d} x$}
\loigiai{
Thể tích của khối tròn xoay tạo thành khi quay $D$ quanh trục $O x$ là\\
$$\pi \displaystyle\int\limits_0^1\left(\mathrm{e}^{3 x}\right)^2 \mathrm{\,d} x=\pi \displaystyle\int\limits_0^1 \mathrm{e}^{6 x} \mathrm{\,d} x.$$}
\end{ex}

\begin{ex}%[2D4N3-3]
Gọi $D$ là hình phẳng giới hạn bởi các đường $y=\mathrm{e}^{4 x}$, $y=0$, $x=0$ và $x=1$. Thể tích của khối tròn xoay tạo thành khi quay $D$ quanh trục $O x$ bằng
\choice
{$\displaystyle\int\limits_0^1 \mathrm{e}^{4 x} \mathrm{\,d} x$}
{\True $\pi \displaystyle\int\limits_0^1 \mathrm{e}^{8 x} \mathrm{\,d} x$}
{$\pi \displaystyle\int\limits_0^1 \mathrm{e}^{4 x} \mathrm{\,d} x$}
{$\displaystyle\int\limits_0^1 \mathrm{e}^{8 x} \mathrm{\,d} x$}
\loigiai{
Thể tích của khối tròn xoay tạo thành khi quay $D$ quanh trục $O x$ là 
$$V=\pi \displaystyle\int\limits_0^1\left(\mathrm{e}^{4 x}\right)^2 \mathrm{\,d} x=\pi \displaystyle\int\limits_0^1 \mathrm{e}^{8 x} \mathrm{\,d} x.$$}
\end{ex}

\begin{ex}%[2D4N3-3]
Cho hình phẳng $(H)$ giới hạn bởi các đường $y=x^2+3$, $y=0$, $x=0$, $x=2$. Gọi $V$ là thể tích của khối tròn xoay được tạo thành khi quay $(H)$ xung quanh trục $O x$. Mệnh đề nào dưới đây đúng?
\choice
{$V=\displaystyle\int\limits_0^2\left(x^2+3\right) \mathrm{\,d}x$}
{$V=\pi \displaystyle\int\limits_0^2\left(x^2+3\right) \mathrm{\,d}x$}
{$V=\displaystyle\int\limits_0^2\left(x^2+3\right)^2 \mathrm{\,d}x$}
{\True $V=\pi \displaystyle\int\limits_0^2\left(x^2+3\right)^2 \mathrm{\,d}x$}
\loigiai{Thể tích của khối tròn xoay được tạo thành khi quay $(H)$ xung quanh trục $O x$ là\\
$V=\pi \displaystyle\int\limits_0^2\left(x^2+3\right)^2 \mathrm{\,d} x$.
}
\end{ex}

\begin{ex}%[2D4H3-3]
Cho hình phẳng $D$ giới hạn bởi đường cong $y=\mathrm{e}^x$, trục hoành và các đường thẳng $x=0$, $x=1$. Khối tròn xoay tạo thành khi quay $D$ quanh trục hoành có thể tích $V$ bằng bao nhiêu?
\choice
{$V=\dfrac{\pi\left(\mathrm{e}^2+1\right)}{2}$}
{$V=\dfrac{\mathrm{e}^2-1}{2}$}
{$V=\dfrac{\pi \mathrm{e}^2}{3}$}
{\True $V=\dfrac{\pi\left(\mathrm{e}^2-1\right)}{2}$}
\loigiai{
$V=\pi \displaystyle\int\limits_0^1 \mathrm{e}^{2 x} \mathrm{\,d} x=\pi \dfrac{\mathrm{e}^{2x}}{2} \Bigg|_0 ^1=\dfrac{\pi\left(e^2-1\right)}{2}$.}
\end{ex}

\begin{ex}%[2D4H3-3]
Cho hình phẳng $D$ giới hạn bởi đường cong $y=\sqrt{x^2+1}$, trục hoành và các đường thẳng $x=0$, $x=1$. Khối tròn xoay tạo thành khi quay $D$ quanh trục hoành có thể tích $V$ bằng bao nhiêu?
\choice
{$ V=2 $}
{\True $V=\dfrac{4 \pi}{3} $}
{$V=2 \pi$}
{$V=\dfrac{4}{3}$}
\loigiai{Thể tích khối tròn xoay được tính theo công thức
	$$
	V=\pi \displaystyle\int\limits_0^1\left(\sqrt{x^2+1}\right)^2 \mathrm{\,d} x=\pi \displaystyle\int\limits_0^1\left(x^2+1\right) \mathrm{\,d} x=\pi\left(\dfrac{x^3}{3}+x\right)\Bigg|_0 ^1=\dfrac{4 \pi}{3} .
	$$}
\end{ex}

\begin{ex}%[2D4H3-3]
Cho hình phẳng $D$ giới hạn bởi đường cong $y=\sqrt{2+\cos x}$, trục hoành và các đường thẳng $x=0, x=\dfrac{\pi}{2}$. Khối tròn xoay tạo thành khi $D$ quay quanh trục hoành có thể tích $V$ bằng bao nhiêu?
\choice
{\True $V=(\pi+1) \pi$}
{$V=\pi-1$}
{$V=\pi+1$}
{$V=(\pi-1) \pi$}
\loigiai{Ta có 
	$$
	V=\pi \displaystyle\int\limits_0^{\tfrac{\pi}{2}}(\sqrt{2+\cos x})^2 \mathrm{\,d} x=\pi(2 x+\sin x)\Bigg|_0 ^{\tfrac{\pi}{2}}=\pi(\pi+1).
	$$}
\end{ex}

\begin{ex}%[2D4H3-3]
Cho hình phẳng $D$ giới hạn bởi đường cong $y=\sqrt{2+\sin x}$, trục hoành và các đường thẳng $x=0$, $ x=\pi$. Khối tròn xoay tạo thành khi quay $D$ quay quanh trục hoành có thể tích $V$ bằng bao nhiêu?
\choice
{\True $V=2 \pi(\pi+1)$}
{$V=2 \pi$}
{$V=2(\pi+1)$}
{$V=2 \pi^2$}
\loigiai{Ta có $V=\pi \displaystyle\int\limits_0^\pi\left(\sqrt{2+\sin x}\right)^2 \mathrm{\,d} x=\pi \displaystyle\int\limits_0^\pi\left(2+\sin x\right) \mathrm{\,d} x=\pi(2 x-\cos x)\Bigg|_0 ^\pi=2 \pi\left(\pi+1\right)$.}
\end{ex}

\begin{ex}%[2D4H3-3]
Tìm công thức tính thể tích của khối tròn xoay khi cho hình phẳng giới hạn bởi parabol $(P)\colon y=x^2$, đường thẳng $d\colon y=2 x$ và đường thẳng $x=0$, $x=2$ quay xung quanh trục $O x$.
\choice
{$\pi \displaystyle\int\limits_0^2\left(x^2-2 x\right)^2 \mathrm{\,d} x$}
{\True $\pi \displaystyle\int\limits_0^2 4 x^2 \mathrm{\,d} x-\pi \int_0^2 x^4 \mathrm{\,d} x$}
{ $\pi \displaystyle\int\limits_0^2 4 x^2 \mathrm{\,d} x+\pi \int_0^2 x^4 \mathrm{\,d} x$}
{$\pi \displaystyle\int\limits_0^2\left(2 x-x^2\right) \mathrm{\,d} x$}
\loigiai{Với mọi $ x \in \left[0;2\right] $ ta có $ 2x\ge 0 $, $ x^2\ge 0 $ và $ 2x\ge x^2 $ nên $V=\pi \displaystyle\int\limits_0^2 4 x^2 \mathrm{\,d} x-\pi \displaystyle\int\limits_0^2 x^4 \mathrm{\,d} x$.
}
\end{ex}

\begin{ex}%[2D4N3-3]
 Cho hình phẳng $(H)$ giới hạn bởi các đường $y=x^2+3$, $y=0$, $x=0$, $x=2$. Gọi $V$ là thể tích khối tròn xoay được tạo hành khi quay $ (H) $ xung quanh trục $ Ox $. Mệnh đề nào sau đây đúng?
 \choice
 {\True $ V=\pi \displaystyle\int\limits_0^2 \left(x^2+3\right)^2 \mathrm{\,d}x $}
 {$ V=\displaystyle\int\limits_0^2 \left(x^2+3\right)\mathrm{\,d}x $}
 {$ V=\displaystyle\int\limits_0^2 \left(x^2+3\right)^2 \mathrm{\,d}x $}
 {$ V=\pi \displaystyle\int\limits_0^2 \left(x^2+3\right) \mathrm{\,d}x $}
 \loigiai{Thể tích của vật tròn xoay là $ V=\pi \displaystyle\int\limits_0^2 \left(x^2+3\right)^2 \mathrm{\,d}x $.}
\end{ex}
	%Câu 13
\begin{ex}%[2D4N3-3]
	Gọi $V$ là thể tích của khối tròn xoay thu được khi quay hình thang cong, giới hạn bởi đồ thị hàm số $y=\sin x$, trục $Ox$, trục $Oy$ và đường thẳng $x=\dfrac{\pi}{2}$, xung quanh trục $Ox$. Mệnh đề nào dưới đây đúng?
	\choice
	{$V=\displaystyle\int\limits_0^{\tfrac{\pi}{2}}{\sin^2x\mathrm{\,d}x}$}
	{$V=\displaystyle\int\limits_0^{\tfrac{\pi}{2}}{\sin x\mathrm{\,d}x}$}
	{\True $V=\pi\displaystyle\int\limits_0^{\tfrac{\pi}{2}}{\sin^2x\mathrm{\,d}x}$}
	{$V=\pi\displaystyle\int\limits_0^{\tfrac{\pi}{2}}{\sin x\mathrm{\,d}x}$}
	\loigiai{
		Công thức tính $V=\pi\displaystyle\int\limits_a^b{f^2(x)\mathrm{\,d}x}$.
	}
\end{ex}

%Câu 14
\begin{ex}%[2D4H3-3]
	Thể tích khối tròn xoay được sinh ra khi quay hình phẳng giới hạn bởi đồ thị của hàm số $y=x^2-2x$, trục hoành, đường thẳng $x=0$ và $x=1$ quanh trục hoành bằng
	\choice
	{$\dfrac{16\pi}{15}$}
	{$\dfrac{2\pi}{3}$}
	{$\dfrac{4\pi}{3}$}
	{\True $\dfrac{8\pi}{15}$}
	\loigiai{
		Ta có
		\allowdisplaybreaks
		\begin{eqnarray*}
			V&=&\pi\displaystyle\int\limits_0^1\left(x^2-2x\right)^2\mathrm{\,d}x\\
			&=&\pi\displaystyle\int\limits_0^1\left(x^4-4x^3+4x^2\right)\mathrm{\,d}x\\
			&=&\pi \cdot \left(\dfrac{x^5}{5}-x^4+\dfrac{4x^3}{3}\right)\Bigg|_0^1\\
			&=&\pi \cdot \left(\dfrac{1}{5}-1+\dfrac{4}{3}\right)=\dfrac{8\pi}{15}.
		\end{eqnarray*}
	}
\end{ex}

%Câu 15
\begin{ex}%[2D4N3-3]
	Cho miền phẳng $(D)$ giới hạn bởi $y=\sqrt x$, hai đường thẳng $x=1$, $x=2$ và trục hoành. Tính thể tích khối tròn xoay tạo thành khi quay $(D)$ quanh trục hoành.
	\choice
	{$3\pi $}
	{\True $\dfrac{3\pi}{2}$}
	{$\dfrac{2\pi}{3}$}
	{$\dfrac{3}{2}$}
	\loigiai{
		$V=\pi\displaystyle\int\limits_1^2x\mathrm{\,d}x=\dfrac{\pi x^2}{2}\Bigg|_1^2=\dfrac{3\pi}{2}$.
	}
\end{ex}

%Câu 16
\begin{ex}%[2D4N3-3]
	Cho hình phẳng $(H)$ giới hạn bởi các đường $y=2x-x^2$, $y=0$. Quay $(H)$ quanh trục hoành tạo thành khối tròn xoay có thể tích là
	\choice
	{$\displaystyle\int\limits_0^2\left(2x-x^2\right)\mathrm{\,d}x$}
	{\True $\pi\displaystyle\int\limits_0^2\left(2x-x^2\right)^2\mathrm{\,d}x$}
	{$\displaystyle\int\limits_0^2\left(2x-x^2\right)^2\mathrm{\,d}x$}
	{$\pi\displaystyle\int\limits_0^2\left(2x-x^2\right)\mathrm{\,d}x$}
	\loigiai{
		Theo công thức ta chọn $V=\pi\displaystyle\int\limits_0^2\left(2x-x^2\right)^2\mathrm{\,d}x$.
	}
\end{ex}

%Câu 17
\begin{ex}%[2D4H3-3]
	Cho hình phẳng giới hạn bởi các đường $y=\sqrt{x}-2$, $y=0$ và $x=4$, $x=9$ quay xung quanh trục $Ox$. Tính thể tích khối tròn xoay tạo thành.
	\choice
	{$V=\dfrac{7}{6}$}
	{$V=\dfrac{5\pi}{6}$}
	{$V=\dfrac{7\pi}{11}$}
	{\True $V=\dfrac{11\pi}{6}$}
	\loigiai{
		Thể tích của khối tròn xoay tạo thành là
		\allowdisplaybreaks
		\begin{eqnarray*}
			V&=&\pi\displaystyle\int\limits_4^9\left(\sqrt{x}-2\right)^2\mathrm{\,d}x\\ &=&\pi\displaystyle\int\limits_4^9\left(x-4\sqrt{x}+4\right)\mathrm{\,d}x\\
			&=&\pi\cdot \left(\dfrac{x^2}{2}-\dfrac{8x\sqrt{x}}{3}+4x\right)\Bigg|_4^9\\
			&=&\pi\left(\dfrac{81}{2}-72+36\right)-\pi\left(\dfrac{16}{2}-\dfrac{64}{3}+16\right)=\dfrac{11\pi}{6}.
		\end{eqnarray*}
	}
\end{ex}

%Câu 18
\begin{ex}%[2D4H3-3]
	Cho hình phẳng $(H)$ giới hạn bởi các đường thẳng $y=x^2+2$, $y=0$, $x=1$, $x=2$. Gọi $V$ là thể tích của khối tròn xoay được tạo thành khi quay $(H)$ xung quanh trục $Ox$. Mệnh đề nào dưới đây đúng?
	\choice
	{$V=\displaystyle\int\limits_1^2\left(x^2+2\right)\mathrm{\,d}x$}
	{\True $V=\pi\displaystyle\int\limits_1^2\left(x^2+2\right)^2\mathrm{\,d}x$}
	{$V=\displaystyle\int\limits_1^2\left(x^2+2\right)^2\mathrm{\,d}x$}
	{$V=\pi\displaystyle\int\limits_1^2\left(x^2+2\right)\mathrm{\,d}x$}
	\loigiai{
		Ta có $V=\pi\displaystyle\int\limits_1^2\left(x^2+2\right)^2\mathrm{\,d}x$.
	}
\end{ex}
\Closesolutionfile{ans}
% \indapan{6}{ans/ans-2-C4B3CD2-lc}

\Opensolutionfile{ans}[ans/ans-2-C4B3CD2_5-10-KQ]
%\TNSA

%Câu 19
\begin{ex}%[2D4H3-4]
	Cắt một vật thể $(T)$ bởi hai mặt phẳng vuông góc với trục $Ox$ tại $x=0$ và $x=2$. Một mặt phẳng tùy ý vuông góc với trục $Ox$ tại điểm có hoành độ $x$ ($0\le x\le 2$) cắt vật thể đó có theo một thiết diện là một hình vuông có cạnh bằng $\sqrt{x^3}$. Thể tích vật thể $(T)$ là số hữu tỉ có dạng phân số tối giản $\dfrac{a}{b}$. Tính $a+b$.
	\shortans{$135$}
	\loigiai{
		Diện tích thiết diện là $S(x)=\sqrt{x^3}\cdot \sqrt{x^3}=x^6$.\\
		Thể tích của vật thể $(T)$ là $V=\displaystyle\int\limits_0^2S(x)\mathrm{\,d}x=\displaystyle\int\limits_0^2x^6\mathrm{\,d}x=\dfrac{128}{7}$.\\
		Suy ra $a=128$ và $b=7$. Khi đó, $a+b=135$.
	}
\end{ex}

%Câu 20
\begin{ex}%[2D4H3-4]
	Cắt một vật thể bởi hai mặt phẳng vuông góc với trục $Ox$ tại $x=1$; $x=3$. Khi cắt một vật thể bởi mặt phẳng vuông góc với trục $Ox$ tại điểm có hoành độ $x$ ($1\le x\le 3$), mặt cắt là tam giác vuông có một góc $45^\circ$ và độ dài một cạnh góc vuông là $\sqrt{4-\dfrac{1}{2} x^2}$. Thể tích vật thể trên là một số hữu tỉ có dạng phân số tối giản $\dfrac{a}{b}$. Tính $a\cdot b$.
	\shortans{$66$}
	\loigiai{
		Diện tích tam giác vuông cân là $S(x)=\dfrac{1}{2}\sqrt{4-\dfrac{1}{2} x^2}\cdot \sqrt{4-\dfrac{1}{2}x^2}=\dfrac{1}{2}\left(4-\dfrac{1}{2}x^2\right)$.\\
		Vậy thể tích vật thể là \[V=\displaystyle\int\limits_1^3\dfrac{1}{2}\left(4-\dfrac{1}{2}{x^2}\right)\mathrm{\,d}x=\dfrac{11}{6}.\]
		Suy ra $a=11$; $b=6$. Khi đó $a\cdot b=66$.
	}
\end{ex}

%Câu 21
\begin{ex}%[2D4H3-3]
	Tính thể tích khối tròn xoay khi quay hình phẳng $(H)$ xác định bởi các đường $y=\dfrac{1}{3}x^3-x^2$, $y=0$, $x=0$ và $x=3$ quanh trục $Ox$ (kết quả viết dưới dạng số thập phân và làm tròn đến hàng phần trăm).
	\shortans{$7{,}27$}
	\loigiai{
		Thể tích khối tròn xoay sinh ra khi quay hình phẳng $(H)$ quanh trục $Ox$ là
		$$V=\pi\displaystyle\int\limits_0^3\left(\dfrac{1}{3}x^3-x^2\right)^2\mathrm{\,d}x=\pi\displaystyle\int\limits_0^3\left(\dfrac{1}{9}x^6-\dfrac{2}{3}x^5+x^4\right)\mathrm{\,d}x=\dfrac{81\pi}{35} \approx 7{,}27.$$
	}
\end{ex}

%Câu 22
\begin{ex}%[2D4H3-3]
	Tính thể tích của vật thể tạo nên khi quay quanh trục $Ox$ hình phẳng $D$ giới hạn bởi đồ thị $(P)\colon y=2x-x^2$, trục $Ox$ và hai đường thẳng $x=0$, $x=2$ (Kết quả viết dưới dạng số thập phân và làm tròn đến hàng phần trăm).
	\shortans{$3{,}35$}
	\loigiai{
		Ta có
		\allowdisplaybreaks
		\begin{eqnarray*}
			V&=&\pi\displaystyle\int\limits_0^2\left(2x-x^2\right)^2\mathrm{\,d}x\\
			&=&\pi\displaystyle\int\limits_0^2\left(4x^2-4x^3+x^4\right)\mathrm{\,d}x\\
			&=&\pi\left(\dfrac{4}{3}{x^3}-x^4+\dfrac{1}{5}{x^5}\right)\Bigg|_0^2\\
			&=&\dfrac{16}{15}\pi\approx 3{,}35.
		\end{eqnarray*}
	}
\end{ex}

%Câu 23
\begin{ex}%[2D4H3-3]
	Cho hình phẳng giới hạn bởi các đường $y=\tan x$, $y=0$, $x=0$, $x=\dfrac{\pi}{4}$ quay xung quanh trục $Ox$. Tính thể tích vật thể tròn xoay được sinh ra (kết quả viết dưới dạng số thập phân và làm tròn một chữ số thập phân sau dấu phẩy).
	\shortans{$0{,}8$}
	\loigiai{
		Thể tích vật thể tròn xoay được sinh ra là
		\[V=\pi\displaystyle\int\limits_0^{\tfrac{\pi}{4}}{\tan^2x\mathrm{\,d}x}=\pi\displaystyle\int\limits_0^{\tfrac{\pi}{4}}{\left(\dfrac{1}{\cos^2x-1}\right)}\mathrm{\,d}x=\pi\left(\tan x-x\right)\Bigg|_0^{\tfrac{\pi}{4}}=\dfrac{4\pi-\pi^2}{4} \approx 0{,}8.\]
	}
\end{ex}

%Câu 24
\begin{ex}%[2D4H3-3]
	Gọi $V$ là thể tích khối tròn xoay tạo thành do quay xung quanh trục hoành một elip có phương trình $\dfrac{x^2}{25}+\dfrac{y^2}{16}=1$. Tính $V$ (Kết quả làm tròn đến hàng đơn vị).
	\shortans{$335$}
	\loigiai{
		Quay elip đã cho xung quanh trục hoành chính là quay hình phẳng $H$ giới hạn bởi $y=4\sqrt{1-\dfrac{x^2}{25}}$, $y=0$, $x=-5$, $x=5$.\\
		Vậy thể tích khối tròn xoay sinh ra bởi $H$ khi quay xung quanh trục hoành là
		\[V=\pi\displaystyle\int_{-5}^5\left(16-\dfrac{16x^2}{25}\right)\mathrm{\,d}x=\pi\left(16x-\dfrac{16x^3}{75}\right)\Bigg|^5_{-5}=\dfrac{320\pi}{3}\approx 335.\]
	}
\end{ex}

%Câu 25
\begin{ex}%[2D4H3-3]%Câu 13
	\immini{Cho hình phẳng $(H)$ được gạch chéo trong hình bên. Tính thể hình tròn xoay sinh ra bởi $(H)$ khi quay $(H)$ quanh trục $Ox$ (Kết quả viết dưới dạng số thập phân và làm tròn đến hàng phần chục).
	}{
		\begin{tikzpicture}[line join=round, line cap=round,>=stealth,thick,scale=0.7]
			\tikzset{every node/.style={scale=0.8}}
			\draw[->] (-3.1,0)--(3.1,0) node[below left] {$x$};
			\draw[->] (0,-1.1)--(0,5.1) node[below left] {$y$};
			\draw (0,0) node [below left] {$O$};
			\foreach \x/\nx in {1/1,2/2}
			\draw (\x,1pt)--(\x,-1pt) node [below left] {$\nx$};
			\foreach \y/\ny in {1/1,2/2,3/3,4/4}
			\draw (1pt,\y)--(-1pt,\y) node [left] {$\ny$};
			\begin{scope}
				\clip (-3,-1) rectangle (3,5);
				\draw[samples=200,domain=-2:2,smooth,variable=\x] plot (\x,{1*(\x)^2+0*(\x)+0});
				\fill[pattern=north east lines](1,0)--plot[samples=200,domain=1:2,smooth,variable=\x] (\x,{(\x)^2})--(2,0);
				\draw plot[samples=200,domain=-2:2.15,smooth,variable=\x] (\x,{(\x)^2}) node[left=2cm]{$y=x^2$};
				\draw (1,-0.8)--(1,4.3) (2,-0.8)--(2,4.3);
				\fill[black](1,1) circle (2pt);
				\fill[black](2,4) circle (2pt);
			\end{scope}
		\end{tikzpicture}
	}
	\shortans{$19{,}5$}
	\loigiai{
		Ta có $V=\pi\displaystyle\int_1^2{\left(x^2\right)^2\mathrm{\,d}x}=\pi\dfrac{x^5}{5}\Bigg|^2_1=\dfrac{31\pi}{5}\approx 19{,}5$.
	}
\end{ex}

%Câu 26
\begin{ex}%[2D4H3-3]
	\immini{Cho hình phẳng $(D)$ được tô màu trong hình bên. Tính thể hình tròn xoay sinh ra bởi $(D)$ khi quay $(D)$ quanh trục $Ox$ (Kết quả viết dưới dạng số thập phần và làm tròn đến hàng phần trăm).
	}{
		\begin{tikzpicture}[line join=round, line cap=round,>=stealth,thick]
			\tikzset{every node/.style={scale=0.9}}
			\draw[->] (-1.1,0)--(3.1,0) node[below left] {$x$};
			\draw[->] (0,-1.1)--(0,3.1) node[below left] {$y$};
			\draw (0,0) node [below left] {$O$};
			\foreach \x/\nx in {1/1,2/2}
			\draw[thin] (\x,1pt)--(\x,-1pt) node [below] {$\nx$};
			\foreach \y/\ny in {1/1,2/2}
			\draw[thin] (1pt,\y)--(-1pt,\y) node [left] {$\ny$};
			\begin{scope}
				\clip (-1,-1) rectangle (3,3);
				\draw[pattern=north east lines](1,0)--plot[samples=200,domain=1:2,smooth,variable=\x] (\x,{1+1/(\x)})--(2,0);
				\draw plot[samples=200,domain=0.1:2.7,smooth,variable=\x] (\x,{1+1/(\x)});
				\draw (1.7,2.5) node{$y=1+\dfrac{1}{x}$};
				\draw[dashed](1,2)--(0,2);			
			\end{scope}
			\draw (1.5,1) node[circle, fill=white] {$\mathrm{D}$};
		\end{tikzpicture}
	}
	\shortans{$9{,}08$}
	\loigiai{
		Ta có $V=\pi\displaystyle\int_1^2{\left(1+\dfrac{1}{x}\right)^2\mathrm{\,d}x}=\pi\displaystyle\int_1^2{\left(1+\dfrac{2}{x}+\dfrac{1}{x^2}\right)\mathrm{\,d}x}=\pi\left(x+\ln x-\dfrac{1}{x}\right)\Bigg|^2_1 \approx 9{,}08$.
	}
\end{ex}

%Câu 27
\begin{ex}%[2D4H3-3]
	\immini{Cho hình phẳng $(H)$ được tô màu trong hình bên. Tính thể hình tròn xoay sinh ra bởi $(H)$ khi quay $(H)$ quanh trục $Ox$ (Kết quả viết dưới dạng số thập phân và làm tròn đến hàng phần chục)
	}{
		\begin{tikzpicture}[line join=round, line cap=round,>=stealth,thick]
			\tikzset{every node/.style={scale=0.9}}
			\draw[->] (-1.6,0)--(2.1,0) node[below left] {$x$};
			\draw[->] (0,-1.1)--(0,3.1) node[below left] {$y$};
			\draw (0,0) node [below left] {$O$};
			\foreach \x/\nx in {-1/-1,1/1}
			\draw[thin] (\x,1pt)--(\x,-1pt) node [below] {$\nx$};
			\foreach \y/\ny in {1/1}
			\draw[thin] (1pt,\y)--(-1pt,\y) node [left] {$\ny$};
			\begin{scope}
				\clip (-1.5,-1) rectangle (2,3);
				\draw[pattern=north east lines](-1,0)--plot[samples=200,domain=-1:1,smooth,variable=\x] (\x,{e^(\x)})--(1,0);
				\draw[samples=200,domain=-1.5:2,smooth,variable=\x] plot (\x,{e^(\x)});
				\path (0,1)--(1,e) node[pos=0.7, above, sloped]{$y=\mathrm{e}^x$};
			\end{scope}
		\end{tikzpicture}
	}
	\shortans{$11{,}4$}
	\loigiai{
		Ta có $V=\pi\displaystyle\int_{-1}^1{\left(\mathrm{e}^x\right)^2\mathrm{\,d}x}=\pi\displaystyle\int_{-1}^1{\left(\mathrm{e}^{2x}\right)\mathrm{\,d}x}=\dfrac{\pi}{2}\mathrm{e}^{2x}\Bigg|^1_{-1} \approx 11{,}4$.
	}
\end{ex}

%Câu 28
\begin{ex}%[2D4H3-3]
	\immini{Cho hình phẳng $(H)$ được tô màu trong hình bên. Tính thể hình tròn xoay sinh ra bởi $(H)$ khi quay $(H)$ quanh trục $Ox$ (Kết quả viết dưới dạng số thập phân và làm tròn đến hàng phần chục).
	}{
		\begin{tikzpicture}[line join=round, line cap=round,>=stealth,thick]
			\tikzset{every node/.style={scale=0.9}}
			\draw[->] (-1.1,0)--(3.1,0) node[below left] {$x$};
			\draw[->] (0,-1.1)--(0,3.1) node[below left] {$y$};
			\draw (0,0) node [below left] {$O$};
			\foreach \x/\nx in {1/1,2/2}
			\draw[thin] (\x,1pt)--(\x,-1pt) node [below] {$\nx$};
			\draw[thin] (1pt,1)--(-1pt,1) node [below left] {$1$};
			\draw[thin] (1pt,2)--(-1pt,2) node [left] {$2$};
			\begin{scope}
				\clip (-1,-1) rectangle (3,3);
				\draw[pattern=north east lines](0,0)--(0,1)--(2,2)--(2,0);
				\draw[dashed](0,2)--(2,2);
				\fill[black](0,1) circle (1.5pt) node[above left]{$A$};
				\fill[black](2,2) circle (1.5pt) node[right]{$B$};
				\fill[black](2,0) circle (1.5pt) node[above right]{$C$};
			\end{scope}
		\end{tikzpicture}
	}
	\shortans{$14{,}7$}
	\loigiai{
		Gọi đường thẳng $d$ đi qua $A$ và $B$ có phương trình dạng $y=ax+b$.\\
		Ta có hệ phương trình $\heva{&b=1\\&2a+b=2} \Rightarrow \heva{&a=\dfrac{1}{2}\\&b=1.}$\\
		Suy ra $d \colon y=\dfrac{1}{2}x+1$.\\
		Khi đó
		$V=\pi\displaystyle\int_0^1{\left(\dfrac{1}{2}x+1\right)^2\mathrm{\,d}x} \approx 14{,}7$.
	}
\end{ex}

%Câu 29
\begin{ex}%[2D4V3-3]
	\immini{Cho hình phẳng $(H)$ là tam giác cong $OAB$ trong hình vẽ bên. Tính thể hình tròn xoay sinh ra bởi $(H)$ khi quay $(H)$ quanh trục $Ox$ (Kết quả viết dưới dạng số thập phân và làm tròn đến hàng phần trăm).
	}{
		\begin{tikzpicture}[line join=round, line cap=round,>=stealth,thick]
			\tikzset{every node/.style={scale=0.9}}
			\draw[dashed, step=1, gray!50,very thin] (-.5,-0.9) grid (4.5,4.5);
			\draw[->] (-1.6,0)--(5.1,0) node[below] {$x$};
			\draw[->] (0,-1.8)--(0,4.5) node[right] {$y$};
			\draw (0,0) node [below left] {$O$};
			\foreach \x/\nx in {-1/-1,1/1,2/2,3/3,4/4}
			\draw[thin] (\x,1pt)--(\x,-1pt) node [below] {$\nx$};
			\foreach \y/\ny in {-1/-1,1/1,2/2,3/3,4/4}
			\draw[thin] (1pt,\y)--(-1pt,\y) node [left] {$\ny$};
			\begin{scope}
				\clip (-1.5,-1.5) rectangle (4.5,4.5);
				\draw plot[samples=200,domain=-1.2:1.7,smooth,variable=\x] (\x,{(1*(\x)^3});
				\path (1,1)--(2,8) node[pos=0.4,above, sloped]{$y=x^3$};
				\draw plot[samples=200,domain=-0.3:4.3,smooth,variable=\x] (\x,{1*(\x)^2+-4*(\x)+4});
				\fill [pattern=north east lines](0,0)--plot[samples=200,domain=0:1,smooth,variable=\x] (\x,{(\x)^3})--plot[samples=200,domain=1:2,smooth,variable=\x] (\x,{(\x)^2-4*(\x)+4})--(2,0)--cycle;
				\path (3,1)--(4,4) node[pos=0.55,above, sloped]{$y=x^2-4x+4$};
				\fill[black](1,1) circle (1.5pt) node[right]{$A$};
				\fill[black](2,0) circle (1.5pt) node[above]{$B$};
			\end{scope}
		\end{tikzpicture}
	}
	\shortans{$1{,}08$}
	\loigiai{
		Ta có $V=\pi\displaystyle\int_0^1{\left(x^3\right)^2\mathrm{\,d}x}+\pi\displaystyle\int_1^2{\left(x^2-4x+4\right)^2\mathrm{\,d}x} \approx 1{,}08$.
	}
\end{ex}

%Câu 30
\begin{ex}%[2D4V3-3]
	\immini{Gọi $V$ là thể tích khối tròn xoay tạo thành khi quay hình phẳng giới hạn bởi các đường $y=\sqrt{x}$, $y=0$ và $x=4$ quanh trục $Ox$. Đường thẳng $x=a$, $\left(0<a<4\right)$ cắt đồ thị hàm số $y=\sqrt{x}$ tại $M$ (hình vẽ). Gọi $V_1$ là thể tích khối tròn xoay tạo thành khi quay tam giác $OMH$ quanh trục $Ox$. Biết rằng $V=2V_1$. Tìm $a$.
	}{
		\begin{tikzpicture}[line join=round, line cap=round,>=stealth,thick]
			\tikzset{every node/.style={scale=0.9}}
			\draw[->] (-0.6,0)--(5.1,0) node[below left] {$x$};
			\draw[->] (0,-0.4)--(0,2.5) node[below left] {$y$};
			\draw (0,0) node [below left] {$O$};
			\foreach \x/\nx in {3/a,4/4}
			\draw[thin] (\x,1pt)--(\x,-1pt) node [below] {$\nx$};
			\begin{scope}
				\clip (-1.0,-1.0) rectangle (4.5,2.5);
				\draw plot[samples=200,domain=0:4.5,smooth,variable=\x] (\x,{sqrt((\x))});
				\path (0,0)--(4,1) node[pos=0.45,above=0.8cm, sloped]{$y=\sqrt x$};
				\fill[black](3,1.732) circle (1.5pt) node[above right]{$M$} (4,0) node[above right]{$H$};
				\draw (3,2)--(3,-0.1);
				\draw[pattern=north east lines](0,0)--(3,1.732)--(4,0);
			\end{scope}
		\end{tikzpicture}
	}
	\shortans{$3$}
	\loigiai{
		\immini{
			Ta có $V=\pi\displaystyle\int\limits_0^4x\mathrm{\,d}x=\pi\dfrac{x^2}{2}\Bigg|_0^4=8\pi$.\\
			Mà $V=2V_1\Rightarrow{V_1}=4\pi$.\\
			Gọi $K$ là hình chiếu của $M$ trên $Ox$.\\
			Suy ra $OK=a$, $KH=4-a$, $MK=\sqrt a$.\\
			Khi xoay tam giác $OMH$ quanh $Ox$ ta được khối
		}{
			\begin{tikzpicture}[line join=round, line cap=round,>=stealth,thick]
				\tikzset{every node/.style={scale=0.9}}
				\draw[->] (-0.6,0)--(5.1,0) node[below left] {$x$};
				\draw[->] (0,-0.4)--(0,2.5) node[below left] {$y$};
				\draw (0,0) node [below left] {$O$};
				\foreach \x/\nx in {3/a,4/4}
				\draw[thin] (\x,1pt)--(\x,-1pt) node [below] {$\nx$};
				\begin{scope}
					\clip (-1.0,-1.0) rectangle (4.5,2.5);
					\draw plot[samples=200,domain=0:4.5,smooth,variable=\x] (\x,{sqrt((\x))});
					\path (0,0)--(4,1) node[pos=0.45,above=0.8cm, sloped]{$y=\sqrt x$};
					\fill[black](3,1.732) circle (1.5pt) node[above right]{$M$} (4,0) node[above right]{$H$} (3,0)node[below right]{$K$};
					\draw (3,2)--(3,-0.1);
					\draw[pattern=north east lines](0,0)--(3,1.732)--(4,0);
				\end{scope}
			\end{tikzpicture}
		}\hspace{-0.77cm}
		tròn xoay là sự lắp ghép của hai khối nón sinh bởi các tam giác $OMK$, $MHK$, hai khối nón đó có cùng mặt đáy và có tổng chiều cao là $OH=4$ nên thể tích của khối tròn xoay đó là $V_1=\dfrac{1}{3} \cdot \pi \cdot 4 \cdot \left(\sqrt a\right)^2=\dfrac{4\pi a}{3}$, từ đó suy ra $a=3$.
	}
\end{ex}
\Closesolutionfile{ans}
% \indapan{6}{ans/ans-2-C4B3CD2_5-10-KQ}
% \begin{dang}{Ứng dụng diện tích hình phẳng và thể tích khối tròn xoay trong bt thực tiễn}
\end{dang}

% \begin{dang}{Ứng dụng diện tích hình phẳng trong bài toán thực tiễn}
% \end{dang}

\Opensolutionfile{ans}[ans/ans-2-C4B3CD3_1-4-lc]
%\TN

%Câu 1
\begin{ex}%[2D4V3-2]
	Trường Nguyễn Văn Trỗi muốn làm một cái cửa nhà hình parabol có chiều cao từ mặt đất đến đỉnh là $2{,}25$\,mét, chiều rộng tiếp giáp với mặt đất là $3$\,mét. Giá thuê mỗi mét vuông là $1\,500\,000$\,đồng. Vậy số tiền nhà trường phải trả là
	\choice
	{$33\,750\,000$\,đồng}
	{$3\,750\,000$\,đồng}
	{$12\,750\,000$\,đồng}
	{\True $6\,750\,000$\,đồng}
	\loigiai{
		\immini{Gọi phương trình parabol
			\[(P)\colon y=ax^2+bx+c.\]
			Do tính đối xứng của parabol nên ta có thể chọn hệ trục tọa độ $Oxy$ sao cho $(P)$ có đỉnh $I\in Oy$ (như hình vẽ).\\
			Ta có hệ phương trình\\ $\heva{&\dfrac{9}{4}=c,\Big(I\in(P)\Big)\\&\dfrac{9}{4}a-\dfrac{3}{2}b+c=0\Big(A\in(P)\Big)\\&\dfrac{9}{4}a+\dfrac{3}{2}b+c=0\Big(B\in(P)\Big)} \Leftrightarrow \heva{&c=\dfrac{9}{4}\\&a=-1\\& b=0.}$\\
			Vậy $(P)\colon y=-x^2+\dfrac{9}{4}$.
		}{
			\begin{tikzpicture}[line join=round, line cap=round,>=stealth,thick]
				\tikzset{every node/.style={scale=0.9}}
				\begin{scope}
					\clip (-3,-1) rectangle (3,3.5);
					\draw[fill=green!20](-1.5,0)--plot[samples=200,domain=-1.5:1.5,smooth,variable=\x] (\x,{-1*(\x)^2+9/4})--(1.5,0);
					\draw (-1.5,0) circle (1.5pt) node[below]{$A\left(-\frac{3}{2},0\right)$};
					\draw (1.5,0) circle (1.5pt) node[below right]{$B\left(\frac{3}{2},0\right)$};
					\draw (0,2.25) circle (1.5pt) node[above right]{$I\left(\frac{3}{2},0\right)$};
				\end{scope}
				\draw[->] (-3.1,0)--(4.1,0) node[below left] {$x$};
				\draw[->] (0,-1.1)--(0,3.6) node[below left] {$y$};
				\draw (0,0) node [below left] {$O$};
				\foreach \x/\nx in {-1/-1,1/1}
				\draw[thin] (\x,1pt)--(\x,-1pt) node [above] {$\nx$};
				\foreach \y/\ny in {1/1,2/2}
				\draw[thin] (1pt,\y)--(-1pt,\y) node [left] {$\ny$};
			\end{tikzpicture}
		}
		\noindent
		Dựa vào đồ thị, diện tích của parabol là
		\[S=\displaystyle\int\limits_{-\tfrac{3}{2}}^{\tfrac{3}{2}}{\left(-x^2+\dfrac{9}{4}\right)\mathrm{\,d}x}=2\displaystyle\int\limits_0^{\tfrac{3}{2}}{\left(-x^2+\dfrac{9}{4}\right)\mathrm{\,d}x}=2\left(\dfrac{-x^3}{3}+\dfrac{9}{4}x\right)\Bigg|_0^{\tfrac{9}{4}}=\dfrac{9}{2}\,\mathrm{(m^2)}.\]
		Số tiền phải trả là $\dfrac{9}{2}\cdot 1\,500\,000=6\,750\,000$\,(đồng).
	}
\end{ex}

%Câu 2
\begin{ex}%[2D4V3-2]
	\immini{Chị Minh Hiền muốn làm một cái cổng hình Parabol như hình vẽ bên. Chiều cao $GH=4$\,m, chiều rộng $AB=4$\,m, $AC=BD=0{,}9$\,m. Chị Minh Hiền làm hai cánh cổng khi đóng lại là hình chữ nhật $CDEF$ tô đậm có giá là $1\,200\,000$\,đồng/$\mathrm{m^2}$, còn các phần để trắng làm xiên hoa có giá là $900\,000$\,đồng/$\mathrm{m^2}$. Hỏi tổng số tiền để làm hai phần nói trên gần nhất với số tiền nào dưới đây?
		\choice
		{\True $11\,445\,000$\,đồng}
		{$4\,077\,000$\,đồng}
		{$7\,368\,000$\,đồng}
		{$11\,370\,000$\,đồng}
	}{
		\begin{tikzpicture}[line join=round, line cap=round,>=stealth,thick]
			\tikzset{every node/.style={scale=0.9}}
			\begin{scope}
				\clip (-0.1,-0.5) rectangle (4.1,4.5);
				\draw(0,0)--plot[samples=200,domain=0:4,smooth,variable=\x] (\x,{-1*(\x)^2+4*(\x)})--(4,0);
				\draw[fill=black](0,0) circle (1.5pt) node[below]{$A$} (4,0) circle (1.5pt) node[below]{$B$} (2,4) circle (1.5pt) node[above]{$G$} (0.9,2.79) circle (1.5pt) node[left]{$F$} (3.1,2.79) circle (1.5pt) node[right]{$E$} (3.1,0) circle (1.5pt) node[below]{$D$} (0.9,0) circle (1.5pt) node[below]{$C$} (2,0) circle (1.5pt) node[below]{$H$};
				\draw[fill=gray!20](0.9,0)--(0.9,2.79)--(3.1,2.79)--(3.1,0) (0,0)--(4,0);
				\draw[dashed](2,0)--(2,4);
			\end{scope}
		\end{tikzpicture}
	}
	\loigiai{
		\immini{Gắn hệ trục tọa độ Oxy sao cho $AB$ trùng $Ox$, $A$ trùng $O$ khi đó parabol có đỉnh $G(2;4)$ và đi qua gốc tọa độ.\\
			Giả sử phương trình của parabol có dạng $y=ax^2+bx+c$, $(a\ne 0)$.\\
			Vì parabol có đỉnh là $G(2;4)$ và đi qua điểm $O(0;0)$ nên ta có \[\heva{&c=0\\&-\dfrac{b}{2a}=2\\&a\cdot 2^2+b\cdot 2+c=4}\Leftrightarrow\heva{&a=-1\\&b=4\\&c=0.}\]
		}{
			\begin{tikzpicture}[line join=round, line cap=round,>=stealth,thick]
				\tikzset{every node/.style={scale=0.9}}
				\begin{scope}
					\clip (-0.5,-0.5) rectangle (4.5,4.5);
					\draw(0,0)--plot[samples=200,domain=0:4,smooth,variable=\x] (\x,{-1*(\x)^2+4*(\x)})--(4,0);
					\draw plot[samples=200,domain=-0.5:4.5,smooth,variable=\x] (\x,{-1*(\x)^2+4*(\x)});
					\draw[fill=black](0,0) circle (1.5pt) node[below right]{$A$} (4,0) circle (1.5pt) node[below left]{$B$} (2,4) circle (1.5pt) node[above]{$G$} (0.9,2.79) circle (1.5pt) node[left]{$F$} (3.1,2.79) circle (1.5pt) node[right]{$E$} (3.1,0) circle (1.5pt) node[above left]{$D$} (0.9,0) circle (1.5pt) node[above left]{$C$} (2,0) circle (1.5pt) node[above left]{$H$};
					\draw(0.9,0)--(0.9,2.79)--(3.1,2.79)--(3.1,0) (0,0)--(4,0);
					\draw[dashed](2,0)--(2,4)--(0,4);
				\end{scope}
				\draw[->] (-1,0)--(5.1,0) node[below left] {$x$};
				\draw[->] (0,-1)--(0,5.1) node[below left] {$y$};
				\draw (0,0) node [above left] {$O$};
				\foreach \x/\nx in {0.9/0{,}9,2/2,3.1/3{,}1}
				\draw[thin] (\x,1pt)--(\x,-1pt) node [below] {$\nx$};
				\draw[thin] (4,1pt)--(4,-1pt) node [above right] {$4$};
				\foreach \y/\ny in {4/4}
				\draw[thin] (1pt,\y)--(-1pt,\y) node [left] {$\ny$};
			\end{tikzpicture}				
		}
		\noindent
		Suy ra phương trình parabol là $y=f(x)=-x^2+4x$.\\
		Diện tích của cả cổng là $S=\displaystyle\int\limits_0^4\left(-x^2+4x\right)\mathrm{\,d}x=\left(-\dfrac{x^3}{3}+2x^2\right)\Bigg|_0^4=\dfrac{32}{3}\,\mathrm{\left(m^2\right)}$.\\
		Mặt khác chiều cao $CF=DE=f(0{,}9)=2{,}79$\,(m); $CD=4-2\cdot 0{,}9=2{,}2$\,(m).\\
		Diện tích hai cánh cổng là $S_{CDEF}=CD\cdot CF=6{,}138\,\mathrm{(m^2)}$.\\
		Diện tích phần xiên hoa là $S_{xh}=S-S_{CDEF}=\dfrac{32}{3}-6{,}14=\dfrac{6793}{1500}\,\mathrm{(m^2)}$.\\
		Vậy tổng số tiền để làm cổng là $6{,}138\cdot 1\,200\,000+\dfrac{6793}{1500}\cdot 900\,000=11\,441\,400$\,(đồng).
	}
\end{ex}

%Câu 3
\begin{ex}%[2D4V3-2]
	\immini{Một cổng chào có dạng hình Parabol chiều cao $18$\,m, chiều rộng chân đế $12$\,m. Người ta căng hai sợi dây trang trí $AB$, $CD$ nằm ngang đồng thời chia hình giới hạn bởi Parabol và mặt đất thành ba phần có diện tích bằng nhau (xem hình vẽ bên). Tỉ số $\dfrac{AB}{CD}$ bằng
		\choice
		{$\dfrac{1}{\sqrt{2}}$}
		{$\dfrac{4}{5}$}
		{\True $\dfrac{1}{\sqrt[3]{2}}$}
		{$\dfrac{3}{1+2\sqrt{2}}$}
	}{
		\begin{tikzpicture}[line join=round, line cap=round,>=stealth,thick,scale=0.6]
			\tikzset{every node/.style={scale=0.9}}
			\begin{scope}
				\draw(-3,-9)--plot[samples=200,domain=-3:3,smooth,variable=\x] (\x,{-1*(\x)^2})--(3,-9)--cycle;
				\draw[dashed](0,0)--(3.5,0) (3,-9)--(3.5,-9) (-3,-9)--(-3,-9.5) (3,-9)--(3,-9.5);
				\draw[<->](3.5,0)--(3.5,-9) node[pos=0.5, right]{$18$\,m}; \draw[<->](-3,-9.5)--(3,-9.5) node[pos=0.5, below]{$12$\,m};
				\draw[fill=black](-2,-4) circle (1.5pt) node[left]{$A$} (2,-4) circle (1.5pt) node[right]{$B$} (-2.5,-6.25) circle (1.5pt) node[left]{$C$} (2.5,-6.25) circle (1.5pt) node[right]{$D$} (-3,-9) circle (1.5pt) (3,-9) circle (1.5pt);
				\draw (-2,-4)--(2,-4) (-2.5,-6.25)--(2.5,-6.25);
			\end{scope}
		\end{tikzpicture}
	}
	\loigiai{
		\immini{Chọn hệ trục tọa độ $Oxy$ như hình vẽ.
			Phương trình Parabol $(P)$ có dạng $y=ax^2$.\\
			$(P)$ đi qua điểm có tọa độ $(-6;-18)$.\\
			Suy ra $-18=a\cdot (-6)^2\Leftrightarrow a=-\dfrac{1}{2}.\\
			\Rightarrow(P)\colon y=-\dfrac{1}{2}x^2$.\\
			Từ hình vẽ ta có $\dfrac{AB}{CD}=\dfrac{x_1}{x_2}$.\\
			Diện tích hình phẳng giới bạn bởi Parabol và đường thẳng $AB\colon y=-\dfrac{1}{2}x_1^2$ là
			\allowdisplaybreaks
			\begin{eqnarray*}
				S_1&=&2\displaystyle\int\limits_0^{x_1}{\left[-\dfrac{1}{2}{x^2}-\left(-\dfrac{1}{2}x_1^2\right)\right]\mathrm{\,d}x}\\
				&=&2\left(-\dfrac{1}{2}\cdot \dfrac{x^3}{3}+\dfrac{1}{2}x_1^2x\right)\Bigg|_0^{x_1}=\dfrac{2}{3}x_1^3.
			\end{eqnarray*}
		}{
			\begin{tikzpicture}[line join=round, line cap=round,>=stealth,thick,scale=0.7]
				\tikzset{every node/.style={scale=0.8}}
				\begin{scope}
					\draw(-3,-9)--plot[samples=200,domain=-3:3,smooth,variable=\x] (\x,{-1*(\x)^2})--(3,-9)--cycle;
					\draw[dashed] (3,-9)--(3.5,-9) (-3,-9)--(-3,-9.5) (3,-9)--(3,-9.5);
					\draw[<->](3.5,0)--(3.5,-9) node[pos=0.5, right]{$18$\,m}; \draw[<->](-3,-9.5)--(3,-9.5) node[pos=0.7, below]{$12$\,m};
					\draw[fill=black](-2,-4) circle (1.5pt) node[left]{$A$} (2,-4) circle (1.5pt) node[right]{$B$} (-2.5,-6.25) circle (1.5pt) node[left]{$C$} (2.5,-6.25) circle (1.5pt) node[right]{$D$} (-3,-9) circle (1.5pt) (3,-9) circle (1.5pt);
					\draw (-2,-4)--(2,-4) (-2.5,-6.25)--(2.5,-6.25);
				\end{scope}
				\draw[->] (-4,0)--(4,0) node[below left] {$x$};
				\draw[->] (0,-10)--(0,2.0) node[below left] {$y$};
				\draw (0,0) node [above left] {$O$};
				\foreach \x/\nx in {-3/-6,2/x_1,2.5/x_2}
				\draw[thin] (\x,1pt)--(\x,-1pt) node [above] {$\nx$};
				\draw[thin] (1pt,-9)--(-1pt,-9) node [above left] {$-18$};
				\draw[dashed](-3,0)--(-3,-9) (2,0)--(2,-4) (2.5,0)--(2.5,-6.25);
			\end{tikzpicture}
		}
		\noindent
		Diện tích hình phẳng giới hạn bởi Parabol và đường thẳng $CD\colon y=-\dfrac{1}{2}x_2^2$ là
		\[S_2=2\displaystyle\int\limits_0^{x_2}{\left[-\dfrac{1}{2}{x^2}-\left(-\dfrac{1}{2}x_2^2\right)\right]\mathrm{\,d}x}=2\left(-\dfrac{1}{2}\cdot \dfrac{x^3}{3}+\dfrac{1}{2}x_2^2x\right)\Bigg|_0^{x_2}=\dfrac{2}{3}x_2^3.\]
		Từ giả thiết suy ra $S_2=2S_1\Leftrightarrow x_2^3=2x_1^3\Leftrightarrow\dfrac{x_1}{x_2}=\dfrac{1}{\sqrt[3]{2}}$.\\
		Vậy $\dfrac{AB}{CD}=\dfrac{x_1}{x_2}=\dfrac{1}{\sqrt[3]{2}}$.
	}
\end{ex}

%Câu 4
\begin{ex}%[2D4C3-2]
	\immini{Một họa tiết hình cánh bướm như hình vẽ bên. Phần tô đậm được đính đá với giá thành $500\,000$/$\,\mathrm{m^2}$. Phần còn lại được tô màu với giá thành $250\,000$/$\,\mathrm{m^2}$. Cho $AB=4$\,dm; $BC=8$\,dm. Hỏi để trang trí $1\,000$ họa tiết như vậy cần số tiền gần nhất với số nào sau đây.
		\choice
		{$105\,660\,667$}
		{\True $106\,666\,667$}
		{$ 107\,665\,667$}
		{$ 108\,665\,667$}
	}{
		\begin{tikzpicture}[line join=round, line cap=round,>=stealth,thick,scale=0.5]
			\tikzset{every node/.style={scale=0.9}}
			\begin{scope}
				\draw[fill=gray!35](-2,0)--plot[samples=200,domain=-2:2,smooth,variable=\x] (\x,{(\x)^2})--(2,0);
				\draw[fill=gray!35](-2,0)--plot[samples=200,domain=-2:2,smooth,variable=\x] (\x,{-1*(\x)^2})--(2,0);
				\draw[fill=black](-2,4) circle (1.5pt) node[left]{$A$} (2,4) circle (1.5pt) node[right]{$B$} (2,-4) circle (1.5pt) node[right]{$C$} (-2,-4) circle (1.5pt) node[left]{$D$};
				\draw (-2,4)--(2,4) (2,-4)--(-2,-4);
			\end{scope}
			\draw[->] (-3,0)--(3,0) node[below left] {$x$};
			\draw[->] (0,-5)--(0,5) node[below left] {$y$};
		\end{tikzpicture}
	}
	\loigiai{
		Vì $AB=4$\,dm; $BC=8$\,dm $\Rightarrow A(-2;4)$, $B(2;4)$, $C(2;-4)$, $D(-2;-4)$.\\
		parabol là $y=x^2$ hoặc $y=-x^2$.\\
		Diện tích phần tô đậm là $S_1=4\displaystyle\int\limits_0^2{x^2}\mathrm{\,d}x=\dfrac{32}{3}\mathrm{\,(dm^2)}$.\\
		Diện tích hình chữ nhật là $S=4\cdot 8=32\mathrm{\,(dm^2)}$.\\
		Diện tích phần trắng là $S_2=S-S_1=32-\dfrac{32}{3}=\dfrac{64}{3}\mathrm{\,(dm^2)}$.\\
		Tổng chi phí trang chí là $T=\left(\dfrac{32}{3}\cdot 5\,000+\dfrac{64}{3}\cdot 2\,500\right)\cdot 1\,000\approx 106\,666\,667$.
	}
\end{ex}
%%==========Câu 5
% \begin{ex}%[2D4C3-2]
% 	\immini{Một hoa văn trang trí được tạo ra từ một miếng bìa mỏng hình vuông cạnh bằng $10$ cm bằng cách khoét đi bốn phần bằng nhau có hình dạng parabol như hình bên. Biết $AB=5$ cm, $OH=4$ cm. Biết giá trang trí hoa văn $1$ cm$^2$ là $50000$ đồng, tính số tiền cần bỏ ra để trang trí hoa văn đó.
% 		\choice
% 		{$2\,553\,333$ đồng}
% 		{\True $2\,333\,333$ đồng}
% 		{$2\,780\,333$ đồng}
% 		{$2\,123\,333$ đồng}
% 	}{
% 		\begin{tikzpicture}[line join=round, line cap=round,>=stealth,thick,scale=0.5]
% 			\tikzset{every node/.style={scale=0.9}}
% 			\begin{scope}
% 				\fill[black] (0,0)--(10,0)--(10,10)--(0,10)--cycle;
% 				\fill[white](2.5,0)--plot[samples=200,domain=2.5:7.5,smooth,variable=\x] (\x,{-16/25*(\x-2.5)^2+16/5*(\x-2.5)})--(7.5,0);
% 				\fill[white](2.5,10)--plot[samples=200,domain=2.5:7.5,smooth,variable=\x] (\x,{16/25*(\x-2.5)^2-16/5*(\x-2.5)+10})--(7.5,10);
% 				\fill[white] plot[samples=200,domain=6:10,smooth,variable=\x] (\x,{sqrt(1.5625*((\x)-6))+5})--(10,5)--plot[samples=200,domain=6:10,smooth,variable=\x] (\x,{-sqrt(1.5625*((\x)-6))+5})--(10,5);
% 				\fill[white] plot[samples=200,domain=4:0,smooth,variable=\x] (\x,{sqrt(1.5625*(4-(\x)))+5})--(0,5)--plot[samples=200,domain=4:0,smooth,variable=\x] (\x,{-sqrt(1.5625*(4-(\x)))+5})--(0,5);				
% 			\end{scope}
% 			\draw[fill=white](6,5) circle (1.5pt) node[above right]{$O$} (10,2.5) circle (1.5pt) node[right]{$B$} (10,7.5) circle (1.5pt) node[right]{$A$} (10,5) circle (1.5pt) node[right]{$H$};
% 			\draw[dashed] (6,5)--(10,5) (10,2.5)--(10,7.5);
% 		\end{tikzpicture}
% 	}
% 	\loigiai{
% 		\immini{Đưa parabol vào hệ trục $Oxy$ ta tìm được phương trình là $(P)\colon y=-\dfrac{16}{25}x^2+\dfrac{16}{5}x$.\\
% 			Diện tích hình phẳng giới hạn bởi $(P)\colon y=-\dfrac{16}{25}x^2+\dfrac{16}{5}x$, trục hoành và các đường thẳng $x=0$, $x=5$ là
% 			\[S=\displaystyle\int\limits_0^5 \left(-\dfrac{16}{25}x^2+\dfrac{16}{5}x\right)\mathrm{\,d}x=\dfrac{40}{3}.\]
			
% 		}
% 		{\begin{tikzpicture}[line join=round, line cap=round,>=stealth,thick,scale=0.7]
% 				\tikzset{every node/.style={scale=0.8}}
% 				\draw[step=0.5, gray!50,very thin] (0,0) grid (5.5,4.5);
% 				\draw[->] (-0.5,0)--(6.1,0) node[below left] {$x$};
% 				\draw[->] (0,-0.5)--(0,5.1) node[below left] {$y$};
% 				\foreach \x/\nx in {1/1,2/2,3/3,4/4,5/5}
% 				\draw[thin] (\x,1pt)--(\x,-1pt) node [below] {$\nx$};
% 				\foreach \y/\ny in {1/1,2/2,3/3,4/4}
% 				\draw[thin] (1pt,\y)--(-1pt,\y) node [left] {$\ny$};
% 				\draw (0,0) node [below left] {$O$};
% 				\begin{scope}
% 					\clip (-0.5,-0.5) rectangle (5.5,5);
% 					\draw[samples=200,domain=-0.5:5.5,smooth,variable=\x] plot (\x,{-0.64*(\x)^2+3.2*(\x)+0});
% 				\end{scope}
% 				\draw[fill=black](0,0) circle (1.5pt) (2.5,4) circle (1.5pt) (5,0) circle (1.5pt);
% 			\end{tikzpicture}
% 		}\noindent
% 		Tổng diện tích phần bị khoét đi $S_1=4S=\dfrac{160}{3}$ cm$^2$.\\
% 		Diện tích của hình vuông là $S_{hv}=100$ cm$^2$.\\
% 		diện tích bề mặt hoa văn là $S_2=S_{hv}-S_1=100-\dfrac{160}{3}=\dfrac{140}{3}\mathrm{\,(cm^2)}$.\\
% 		Vậy số tiền cần bỏ ra để trang trí hoa văn đó là $\dfrac{140}{3}\cdot 50\,000\approx 2\,333\,333$ (đồng).
% 	}
% \end{ex}
%%==========Câu 6
\begin{ex}%[2D4V3-2]
	\immini{Một viên gạch hoa hình vuông cạnh $40$ cm. Người thiết kế đã sử dụng bốn đường parabol có chung đỉnh tại tâm viên gạch để tạo ra bốn cánh hoa (được tô đen như hình vẽ dưới). Diện tích mỗi cánh hoa của viên gạch bằng
		\choice
		{$800$ cm$^2$} 
		{$\dfrac{800}{3}$ cm$^2$} 
		{\True $\dfrac{400}{3}$ cm$^2$} 
		{$250$ cm$^2$}
	}
	{\begin{tikzpicture}[scale=1.2,>=stealth, line join = round, line cap = round,font=\footnotesize]
			\draw (-1,1) rectangle (1,-1);
			\foreach \i in {0,90,180,270}
			\draw [fill=gray, smooth, samples=100,rotate=\i] plot [domain=0:1] (\x, {(\x)^2})-- plot [domain=1:0] (\x, {sqrt(\x)});
			\draw [<-] (-1,-1.5)--(0,-1.5) node[below] {$40$ cm};
			\draw [->] (0,-1.5)--(1,-1.5);
		\end{tikzpicture}
	} 
	\loigiai{
		\immini{Chọn hệ tọa độ như hình vẽ (1 đơn vị trên trục bằng $10$ cm= $1$ dm), các cánh hoa tạo bởi các đường parabol có phương trình $ y=\dfrac{x^2}{2}$, $y=-\dfrac{x^2}{2}$, $x=-\dfrac{y^2}{2}$, $x=\dfrac{y^2}{2}$.\\ 
			Diện tích một cánh hoa (nằm trong góc phàn tư thứ nhất) bằng diện tích hình phẳng giới hạn bởi hai đồ thị hàm số $y=\dfrac{x^2}{2}$, $y=\sqrt{2x}$ và hai đường thẳng $ x=0$; $x=2$.
		}
		{\begin{tikzpicture}[scale=1.3,>=stealth, line join = round, line cap = round,font=\footnotesize]
				\draw[->] (-1.5,0)--(0,0)%
				node[below right]{$O$}--(1.5,0) node[below]{$x$};
				\draw[->] (0,-1.5) --(0,1.5) node[right]{$y$};
				\draw (-1,1) rectangle (1,-1);
				\foreach \i in {-1,1}
				\foreach \j in {-1,1}
				\draw[fill=black]  (\i,0) circle (0.5 pt) node [below] {\footnotesize $\i$};
				\draw [fill=gray, smooth, samples=100] plot [domain=0:1] (\x, {(\x)^2})-- plot [domain=1:0] (\x, {sqrt(\x)});
			\end{tikzpicture}
		}\noindent
		Do đó diện tích một cánh hoa bằng 
		\[\displaystyle\int\limits_0^2 \left( \sqrt{2x}-\dfrac{x^2}{2} \right) \mathrm{d}x = \left(\dfrac{2\sqrt{2}}{3}\sqrt{(2x)^3}-\dfrac{x^3}{6} \right) \Big|_0^2 = \dfrac{4}{3} \;\;\text{dm}^2= \dfrac{400}{3}\;\;\text{cm}^2.\]
	}
\end{ex} 
\Closesolutionfile{ans}
\indapan{6}{ans/ans-2-C4B3CD3_1-4-lc}

% \begin{dang}
% 	{Ứng dụng thể tích khối tròn xoay trong bài toán thực tiễn}
% \end{dang}
\Opensolutionfile{ans}[ans/ans-2-C4B3CD3_1-2-lc]
%%==========Câu 7
% \begin{ex}%[2D4V3-4]
% 	\immini{Khi cắt một vật thể hình chiếc niêm bởi mặt phẳng vuông góc với trục $Ox$ tại điểm có hoành độ $x$ ($-2 \le x\le 2$), mặt cắt là tam giác vuông có một góc $45^\circ$ và độ dài một cạnh góc vuông là $\sqrt{14-3x^2}$ (như hình vẽ). Tính thể tích vật thể hình chiếc niêm trên.
% 		\choice
% 		{\True $V=20$}
% 		{$V=20\pi$}
% 		{$V=10$}
% 		{$V=10\pi$}
% 	}
% 	{\begin{tikzpicture}[declare function={r=4;d=3;},scale=0.6]
% 			\path (0,0) coordinate (O)--++(85:r/3) coordinate (A)
% 			($(A)!2!(O)$) coordinate (B)
% 			($(O)+(0:r)$) coordinate (C)
% 			($(C)+(90:d)$) coordinate (D)
% 			;
% 			\draw (A)..controls +(40:1.5) and +(90:1)..(D)..controls +(-90:1) and +(40:1.5)..(B);
% 			\draw (A)--(O)node[midway,left]{} (O)--(B)node[midway,left]{} (C)--(D)
% 			(C)..controls +(-90:1) and +(0:2)..(B)
% 			;
% 			\draw[dashed] 
% 			(A)..controls +(0:2) and +(90:1)..(C)
% 			(D)--(O) (O)--(C)node[pos=0.7,above]{} ;
% 			\path pic[draw,angle radius=17pt,"$\alpha$"]{angle= C--O--D};
% 		\end{tikzpicture}
% 	}
% 	\loigiai{
% 		Diện tích tam giác vuông cân là $S(x)=\dfrac{1}{2}\sqrt{14-3x^2} \cdot \sqrt{14-3x^2} = \dfrac{1}{2}(14-3x^2)$.\\
% 		Vậy thể tích vật thể là $\displaystyle\int\limits_{-2}^2 \dfrac{1}{2}(14-3x^2)\mathrm{\,d}x=20$.
% 	}
% \end{ex}
%%==========Câu 8
% \begin{ex}%[2D4V3-4]
% 	\immini{Trong chương trình nông thôn mới của tỉnh Phú Yên, tại xã Hòa Mỹ Tây có xây một cây cầu bằng bê tông như hình vẽ (đường cong trong hình vẽ là các đường Parabol). Biết $1$ m$^3$ khối bê tông để đổ cây cầu có giá 5 triệu đồng. Tính số tiền mà tỉnh Phú Yên cần bỏ ra để xây cây cầu trên.
% 	}
% 	{\begin{tikzpicture}[scale=1.0, font=\footnotesize, line join=round, line cap=round,>=stealth,samples=100]
% 			\path 
% 			(0,0) coordinate (O)
% 			(-1,0) coordinate (B)
% 			(0,1) coordinate (C)
% 			(1,0) coordinate (D)
% 			(-1.4,0) coordinate (M)
% 			(0,1.96) coordinate (N)
% 			(1.4,0) coordinate (P)
% 			;
% 			\path (-165:4) coordinate (A);
% 			\foreach \x in {B,C,D,M,N,P,O}{\path ($(\x)+(-165:4)$) coordinate (\x_1);}
% 			\fill[gray!40] (N_1)--(N)--plot[domain=0:1.4] (\x,{-(\x)^2+1.96})--(P)--(P_1)--plot[shift={(A)},domain=1.4:0] (\x,{-(\x)^2+1.96});
% 			\fill[gray!40] (M_1)--plot[shift={(A)},domain=-1.4:1.4](\x,{-(\x)^2+1.96})--(P_1)--(D_1)--plot[shift={(A)},domain=1:-1] (\x,{-(\x)^2+1})--(B_1)--(M_1);
% 			\draw[dash pattern=on 2pt off 2pt] plot[domain=-1:1] (\x,{-(\x)^2+1})
% 			plot[domain=-1.4:1.4] (\x,{-(\x)^2+1.96})
% 			(M)--(P) (O)--(N)
% 			;
% 			\draw[shift={(A)}] plot[domain=-1:1] (\x,{-(\x)^2+1});
% 			\draw[shift={(A)}] plot[domain=-1.4:1.4] (\x,{-(\x)^2+1.96});
% 			\draw [dash pattern=on 2pt off 2pt] (B)--(B_1) (C)--(C_1) (D)--(D_1) (M)--(M_1)
% 			(O_1)--(N_1)
% 			;
% 			\draw (N)--(N_1) (P)--(P_1) (M_1)--(P_1);
% 			\foreach \t in {O,B,M,P,O_1,M_1,B_1,D_1,P_1}{
% 				\draw[fill=white] (\t) circle (1pt);}
% 			\path (M_1)--(B_1) node[midway,below]{$0,5$m};
% 			\path (B_1)--(D_1) node[midway,below]{$19$m};
% 			\path (D_1)--(P_1) node[midway,below]{$0,5$m};
% 			\path (O)--(C) node[midway,right]{$2$m};
% 			\path (C)--(N) node[pos=0.2,left]{$0,5$m};
% 		\end{tikzpicture}
% 	}
% 	\choice
% 	{$110$ triệu đồng}
% 	{$250$ triệu đồng}
% 	{$180$ triệu đồng}
% 	{\True $200$ triệu đồng}
% 	\loigiai{
% 		\immini{Chọn hệ trục $Oxy$ như hình vẽ.\\
% 			Gọi $(P_1)\colon y=a_1 x^2+b_1$ là Parabol đi qua hai điểm $A\left(\dfrac{19}{2};0\right)$, $B(0;2)$.\\
% 		}
% 		{\begin{tikzpicture}[scale=0.9, font=\footnotesize, line join=round, line cap=round,>=stealth,samples=100]
% 				\path 
% 				(0,0) coordinate (O)
% 				(-1,0) coordinate (B)
% 				(0,1) coordinate (C)
% 				(1,0) coordinate (D)
% 				(-1.4,0) coordinate (M)
% 				(0,1.96) coordinate (N)
% 				(1.4,0) coordinate (P)
% 				;
% 				\path (-165:4) coordinate (A);
% 				\foreach \x in {B,C,D,M,N,P,O}{\path ($(\x)+(-165:4)$) coordinate (\x_1);}
% 				\fill[gray!40] (N_1)--(N)--plot[domain=0:1.4] (\x,{-(\x)^2+1.96})--(P)--(P_1)--plot[shift={(A)},domain=1.4:0] (\x,{-(\x)^2+1.96});
% 				\fill[gray!40] (M_1)--plot[shift={(A)},domain=-1.4:1.4](\x,{-(\x)^2+1.96})--(P_1)--(D_1)--plot[shift={(A)},domain=1:-1] (\x,{-(\x)^2+1})--(B_1)--(M_1);
% 				\draw[-stealth,dash pattern=on 2pt off 2pt] (-1.4,0)--(0,0)node[below left]{$O$}--(2,0) node[below] {$x$};
% 				\draw[-stealth,dash pattern=on 2pt off 2pt] (0,0)--(0,2.5) node[left] {$y$};
% 				\draw[dash pattern=on 2pt off 2pt] plot[domain=-1:1] (\x,{-(\x)^2+1})
% 				plot[domain=-1.4:1.4] (\x,{-(\x)^2+1.96})
% 				;
% 				\draw[shift={(A)}] plot[domain=-1:1] (\x,{-(\x)^2+1});
% 				\draw[shift={(A)}] plot[domain=-1.4:1.4] (\x,{-(\x)^2+1.96});
% 				\draw [dash pattern=on 2pt off 2pt] (B)--(B_1) (C)--(C_1) (D)--(D_1) (M)--(M_1)
% 				(O_1)--(N_1)
% 				;
% 				\draw (N)--(N_1) (P)--(P_1) (M_1)--(P_1);
% 				\foreach \t in {O,B,M,P,O_1,M_1,B_1,D_1,P_1}{
% 					\draw[fill=white] (\t) circle (1pt);}
% 				\path (M_1)--(B_1) node[midway,below]{$0,5$m};
% 				\path (B_1)--(D_1) node[midway,below]{$19$m};
% 				\path (D_1)--(P_1) node[midway,below]{$0,5$m};
% 				\path (O)--(C) node[midway,right]{$2$m};
% 				\path (C)--(N) node[pos=0.2,left]{$0,5$m};
% 			\end{tikzpicture}
% 		}\noindent
% 		Nên ta có hệ phương trình sau
% 		\[\heva{& 0=a\cdot \left(\dfrac{19}{2}\right)^2+2 \\ & 2=b} \Leftrightarrow \heva{& a_1=-\dfrac{8}{361} \\ & b_1=2 }\Rightarrow (P_1)\colon y=-\dfrac{8}{361}{x^2}+2.\]
% 		Gọi $(P_2)\colon y=a_2 x^2+b_2$ là Parabol đi qua hai điểm $C(10;0)$, $D\left(0;\dfrac{5}{2}\right)$.\\
% 		Nên ta có hệ phương trình sau
% 		\[\heva{& 0=a_2 \cdot 10^2 + \dfrac{5}{2} \\ & \dfrac{5}{2}=b_2 } \Leftrightarrow \heva{& a_2=-\dfrac{1}{40} \\ & b_2=\dfrac{5}{2} } \Rightarrow (P_2)\colon y=-\dfrac{1}{40}{x^2}+\dfrac{5}{2}.\]
% 		Ta có thể tích của bê tông là
% 		\[V=5\cdot 2\left[\displaystyle\int\limits_0^{10}\left(-\dfrac{1}{40}{x^2}+\dfrac{5}{2} \right)\mathrm{\,d}x-\displaystyle\int\limits_0^{\tfrac{19}{2}}\left( -\dfrac{8}{361}x^2+2\right)\mathrm{\,d}x \right]=40\;\;\text{m}^3.\]
% 		Số tiền mà tỉnh Phú Yên cần bỏ ra để xây cây cầu là $5\cdot 40=200$ triệu đồng.
% 	}
% \end{ex}
%%==========Câu 9
\begin{ex}%[2D4V3-4]
	Để kỷ niệm ngày 26-3. Chi đoàn 12A dự định dựng một lều trại có dạng parabol, với kích thước: nền trại là một hình chữ nhật có chiều rộng là $3$ mét, chiều sâu là $6$ mét, đỉnh của parabol cách mặt đất là $3$ mét. Hãy tính thể tích phần không gian phía bên trong trại để lớp 12A cử số lượng người tham dự trại cho phù hợp.
	\choice
	{$30$ m$^3$}
	{\True $36$ m$^3$}
	{$40$ m$^3$}
	{$41$ m$^3$}
	\loigiai{
		Giả sử nền trại là hình chữ nhật $ABCD$ có $AB = 3$ m, $BC = 6$ m, đỉnh của parabol là $I$.\\
		Chọn hệ trục tọa độ $Oxy$ sao cho $O$ là trung điểm của cạnh $AB$, $A$, $B$ và $I$, phương trình của parabol có dạng $y=ax^2+b$, $a \ne 0$.\\
		Do $I$, $A$, $B$ thuộc nên ta có $y=-\dfrac{4}{3}x^2+3$.\\
		Vậy thể tích phần không gian phía trong trại là
		\[V=6 \cdot 2\displaystyle\int\limits_0^{\tfrac{3}{2}}\left(-\dfrac{4}{3}x^2+3\right)\mathrm{\,d}x=36.\]
	}
\end{ex}
%%==========Câu 10
\begin{ex}%[2D4V3-4]
	Cho một vật thể bằng gỗ có dạng hình trụ với chiều cao và bán kính đáy cùng bằng $R$. Cắt khối gỗ đó bởi một mặt phẳng đi qua đường kính của một mặt đáy của khối gỗ và tạo với mặt phẳng đáy của khối gỗ một góc $30^\circ$ ta thu được hai khối gỗ có thể tích là $V_1$ và $V_2$, với $V_1<V_2$. Thể tích $V_1$ bằng
	\choice
	{\True $V_1=\dfrac{2\sqrt{3}R^3}{9}$}
	{$V_1=\dfrac{\sqrt{3}\pi R^3}{27}$}
	{$V_1=\dfrac{\sqrt{3}\pi R^3}{18}$}
	{$V_1=\dfrac{\sqrt{3}R^3}{27}$}
	\loigiai{
		\immini{Khi cắt khối gỗ hình trụ ta được một hình nêm có thể tích $V_1$ như hình vẽ.\\
			Chọn hệ trục tọa độ $Oxy$ như hình vẽ.\\
			Nửa đường tròn đường kính $AB$ có phương trình là
			\[y=\sqrt{R^2-x^2}, x \in [-R;R].\]
		}
		{\begin{tikzpicture}[declare function={r=4;d=3;},scale=0.7]
				\path (0,0) coordinate (O)--++(85:r/3) coordinate (A)
				($(A)!2!(O)$) coordinate (B)
				($(O)+(0:r)$) coordinate (C)
				($(C)+(90:d)$) coordinate (D)
				;
				\draw (A)..controls +(40:1.5) and +(90:1)..(D)..controls +(-90:1) and +(40:1.5)..(B);
				\draw (A)--(O)node[midway,left]{$R$} (O)--(B)node[midway,left]{$R$} (C)--(D)
				(C)..controls +(-90:1) and +(0:2)..(B)
				;
				\draw[dashed] 
				(A)..controls +(0:2) and +(90:1)..(C)
				(D)--(O) (O)--(C)node[pos=0.7,above]{$R$} ;
				\path pic[draw,angle radius=17pt,"$\alpha$"]{angle= C--O--D};
			\end{tikzpicture}
		}\noindent
		Một mặt phẳng vuông góc với trục $Ox$ tại điểm $M$ có hoành độ $x$, cắt hình nêm theo thiết diện là $\triangle MNP$ vuông tại $N$ và có $\widehat{PMN}=30^\circ$.\\
		Ta có $NM=y=\sqrt{R^2-x^2} \Rightarrow NP=MN \cdot \tan 30^\circ = \dfrac{\sqrt{R^2-x^2}}{\sqrt{3}}$.\\
		Do $\triangle MNP$ có diện tích $S(x)=\dfrac{1}{2}NM \cdot NP =\dfrac{1}{2}\cdot \dfrac{R^2-x^2}{\sqrt{3}}$.\\
		Thể tích hình nêm là 
		\[V_1=\displaystyle\int\limits_{-R}^R S(x)\mathrm{\,d}x=\dfrac{1}{2}\displaystyle\int\limits_{-R}^R \dfrac{R^2-x^2}{\sqrt{3}}\mathrm{\,d}x=\dfrac{1}{2\sqrt{3}}\left(R^2 x-\dfrac{1}{3}x^3 \right) \Big|_{-R}^R=\dfrac{2\sqrt{3}{R^3}}{9}.\]
		\textbf{Chú ý:} Có thể ghi nhớ công thức tính thể tích hình nêm
		$V_1=\dfrac{2}{3}{R^2}h=\dfrac{2}{3}{R^3}\tan \alpha $, trong đó $R=\dfrac{AB}{2}$, $\alpha =\widehat{PMN}$.
	}
\end{ex}
%%==========Câu 11
% \begin{ex}%[2D4V3-4]
% 	\immini{Cho một mô hình $3-D$ mô phỏng một đường hầm như hình vẽ bên. Biết rằng đường hầm mô hình có chiều dài $5$ cm; khi cắt hình này bởi mặt phẳng vuông góc với đấy của nó, ta được thiết diện là một hình parabol có độ dài đáy gấp đôi chiều cao parabol. Chiều cao của mỗi thiết diện parobol cho bởi công thức $y=3-\dfrac{2}{5}x$ cm, với $x$ cm là khoảng cách tính từ lối vào lớn hơn của đường hầm mô hình. Tính thể tích (theo đơn vị cm$^3$) không gian bên trong đường hầm mô hình (làm tròn kết quả đến hàng đơn vị).
% 		\choice
% 		{\True $29$}
% 		{$27$}
% 		{$31$}
% 		{$33$}
% 	}
% 	{\begin{tikzpicture}[scale=0.6,declare function={a=0.8;b=0.6;c=0.4;d=0.2;}]
% 			\tikzset{
% 				homothety at/.style args={#1 scaled by #2}{shift={($(#1)!#2!(0,0)$)},scale=#2},
% 			}
% 			\def\mypath{(-120:2)..controls +(90:0.6) and +(-180:0.6)..(0,3)}
% 			\def\mydot{(0,3)..controls +(0:0.25) and +(95:0.05)..(60:2)}
% 			\draw \mypath;
% 			\draw[dashed] \mydot;
% 			\path (7,0) coordinate (c1);
% 			\begin{scope}[homothety at=c1 scaled by a]
% 				\draw \mypath;
% 				\draw[dashed] \mydot;
% 			\end{scope}
% 			\begin{scope}[homothety at=c1 scaled by b]
% 				\draw \mypath;
% 				\draw[dashed] \mydot;
% 			\end{scope}
% 			\begin{scope}[homothety at=c1 scaled by c]
% 				\draw \mypath;
% 				\draw[dashed] \mydot;
% 			\end{scope}
% 			\begin{scope}[homothety at=c1 scaled by d]
% 				\draw \mypath;
% 				\draw \mydot;
% 			\end{scope}
% 			\path 
% 			(-120:2) coordinate (A)
% 			(0,3) coordinate (B)
% 			(60:2) coordinate (C)
% 			(0,0) coordinate (O);
% 			\foreach \x in {A,B,C}{\path ($(c1)!a!(\x)$) coordinate (\x_1);}
% 			\foreach \x in {A,B,C}{\path ($(c1)!b!(\x)$) coordinate (\x_2);}
% 			\foreach \x in {A,B,C}{\path ($(c1)!c!(\x)$) coordinate (\x_3);}
% 			\foreach \x in {A,B,C}{\path ($(c1)!d!(\x)$) coordinate (\x_4);}
% 			\path ($(A_4)!0.5!(B_4)$) coordinate (D);
% 			\draw (A)--(A_4) (B)--(B_4) (A_4)--(C_4)
% 			;
% 			\draw[dashed] (B)node[above]{$3$}--(O)--(D)node[below right]{$5$} (A)--(C) (A_1)--(C_1) (A_2)--(C_2) (A_3)--(C_3)  (C)--(C_4);
% 		\end{tikzpicture}
% 	}
% 	\loigiai{
% 		\immini{Xét một thiết diện Parabol có chiều cao là $h$ và độ dài đáy $2h$ và chọn hệ trục $Oxy$ như hình vẽ trên.\\
% 			Parabol $(P)$ có phương trình $(P)\colon y=ax^2+h$, ($a<0$).\\
% 			Có $B(h;0)\in (P) \Leftrightarrow 0=ah^2+h \Leftrightarrow a=-\dfrac{1}{h}$ (do $h>0$).\\
% 			Diện tích $S$ của thiết diện là
% 			\[S=\displaystyle\int\limits_{-h}^h \left( -\dfrac{1}{h}x^2+h\right)\mathrm{\,d}x=\dfrac{4h^2}{3}, h=3-\dfrac{2}{5}x \Rightarrow S(x)=\dfrac{4}{3}\left(3-\dfrac{2}{5}x\right)^2.\]
% 		}
% 		{\begin{tikzpicture}[scale=0.8,font=\footnotesize]
% 				\path (0,0) coordinate (O)
% 				(2,0) coordinate (A)
% 				(0,2) coordinate (B)
% 				;
% 				\draw[-stealth] (-3.5,0)--(0,0)--(3,0)node[below]{$x$};
% 				\draw[-stealth] (0,-1.5)--(0,4)node[left]{$y$};
% 				\draw[smooth,samples=100] plot[domain=-2:2](\x,{(-1/2)*(\x)^2+2});
% 				\foreach \x in {O,A,B}{\draw[fill=blue!40] (\x) circle (1pt);}
% 				\foreach \x in {-3,-2,-1,1}{\draw (\x,0.05)--(\x,-0.05);}
% 				\foreach \x in {-1,1,3}{\draw (-0.05,\x)--(0.05,\x);}
% 				\node[above left] at (B) {$h$};
% 				\path (O)--(A)node[below]{$h$};
% 				\node at (0,0) [below left]{$O$};
% 			\end{tikzpicture}
% 		}\noindent
% 		Suy ra thể tích không gian bên trong của đường hầm mô hình là
% 		\[V=\displaystyle\int\limits_0^5 S(x)\mathrm{\,d}x=\displaystyle\int\limits_0^5\dfrac{4}{3}\left(3-\dfrac{2}{5}x\right)^2\mathrm{\,d}x\approx 28{,}888 \Rightarrow V\approx 29\;\;\text{cm}^3.\]
% 	}
% \end{ex}
%%==========Câu 12
\begin{ex}%[2D4V3-3]
	\immini{Chuẩn bị cho đêm hội diễn văn nghệ chào đón năm mới, bạn Minh Hiền đã làm một chiếc mũ “cách điệu” cho ông già Noel có dáng một khối tròn xoay. Mặt cắt qua trục của chiếc mũ như hình vẽ bên dưới. Biết rằng $OO'=5$ cm, $OA=10$ cm, $OB=20$ cm, đường cong $AB$ là một phần của parabol có đỉnh là điểm $A$. Thể tích của chiếc mũ bằng
		\choice
		{$\dfrac{2750\pi}{3}$ cm$^3$}
		{\True $\dfrac{2500\pi}{3}$ cm$^3$}
		{$\dfrac{2050\pi}{3}$ cm$^3$}
		{$\dfrac{2250\pi}{3}$ cm$^3$}
	}
	{\begin{tikzpicture}[line join=round, line cap=round,>=stealth,scale=0.2]
			\path (0,0) coordinate (O)
			(10,0) coordinate (A)
			(0,20) coordinate (B)
			(0,-5) coordinate (O')
			(10,-5) coordinate (I)
			(-10,-5) coordinate (C)
			(-10,0) coordinate (D)
			;
			\draw[smooth,samples=100,thick] plot[domain=0:10](\x,{(1/5)*((\x)-10)^2})
			plot[domain=-10:0](\x,{(1/5)*((\x)+10)^2})
			;
			\draw[dash pattern=on 2pt off 2pt] (O')--(O)--(B) (A)--(D);
			\draw[thick] (A)--(I)--(C)--(D);
			\foreach \t/\g in {A/90,B/45,O/-135,O'/-135}{
				\draw[fill=black] (\t) circle (1pt) node[shift={(\g:7pt)},font=\scriptsize]{$ \t $};
			}
		\end{tikzpicture}
	}
	\loigiai{
		\immini{Ta gọi thể tích của chiếc mũ là $V$.\\
			Thể tích của khối trụ có bán kính đáy bằng $OA=10$ cm và đường cao $OO'=5$ cm là $V_1$.\\
			Thể tích của vật thể tròn xoay khi quay hình phẳng giới hạn bởi đường cong $AB$ và hai trục tọa độ quanh trục $Oy$ là $V_2$.\\
			Ta có $V=V_1+V_2$; $V_1=5\cdot 10^2\pi =500\pi $ cm$^3$.\\
			Chọn hệ trục tọa độ như hình vẽ.\\
			Do parabol có đỉnh $A$ nên nó có phương trình dạng $(P)\colon y=a(x-10)^2$.
		}
		{\begin{tikzpicture}[line join=round, line cap=round,>=stealth,font=\scriptsize,scale=0.2]
				\path (0,0) coordinate (O)
				(10,0) coordinate (A)
				(0,20) coordinate (B)
				(0,-5) coordinate (O')
				(-10,0) coordinate (C)
				;
				\draw[-stealth] (-12,0)--(0,0)--(12,0)node[below]{$x$};
				\draw[-stealth] (0,-7)--(0,22)node[left]{$y$};
				\draw[smooth,samples=100,thick] plot[domain=0:10](\x,{(1/5)*((\x)-10)^2})
				plot[domain=-10:0](\x,{(1/5)*((\x)+10)^2})
				;
				\draw[thick] (10,0) rectangle (-10,-5);
				\foreach \t/\g in {O/-135,O'/-135}{
					\draw[fill=black] (\t) circle (1pt) node[shift={(\g:7pt)},font=\scriptsize]{$ \t $};
				}
				\draw[fill=black] (A) circle (1pt)node[shift={(45:10pt)}]{$A(10;0)$};
				\draw[fill=black] (B) circle (1pt)node[right]{$B(0;20)$};
				\path (O')--(O)node[pos=0.4,left]{$5$};
				\node at ($(A)+(100:10)$) {$y=\dfrac{1}{5}(x-10)^2$};
			\end{tikzpicture}
		}\noindent
		Vì $(P)$ qua điểm $B(0;20)$ nên $a=\dfrac{1}{5}$.\\
		Do đó, $(P)\colon y=\dfrac{1}{5}(x-10)^2$. Từ đó suy ra $x=10-\sqrt{5y}$ (do $ x<10$).\\
		Suy ra $V_2=\pi \displaystyle\int\limits_0^{20}\left(10-\sqrt{5y}\right)^2\mathrm{\,d}y=\pi(3000-\dfrac{8000}{3})=\dfrac{1000}{3}\pi$ cm$^3$.\\
		Do đó $V=V_1+V_2=\dfrac{1000}{3}\pi +500\pi =\dfrac{2500}{3}\pi$ cm$^3$.
		
	}
\end{ex}
%%==========Câu 13
\begin{ex}%[2D4V3-4]
	\immini{Một chi tiết máy được thiết kế như hình vẽ bên. Các tứ giác $ABCD$, $CDPQ$ là các hình vuông cạnh $2{,}5$ (cm). Tứ giác $ABEF$ là hình chữ nhật có $BE=3{,}5$ (cm). Mặt bên $PQEF$ được mài nhẵn theo đường parabol $(P)$ có đỉnh parabol nằm trên cạnh $ EF$. Thể tích của chi tiết máy bằng
		
	}
	{\begin{tikzpicture}[scale=0.8, font=\footnotesize, line join=round, line cap=round,>=stealth,declare function={a=3;h=4.5;}]
			\path (0,0) coordinate (A)
			--++(30:a) coordinate (B)
			--++(90:a) coordinate (C)
			($(A)+(C)-(B)$) coordinate (D)
			($(A)+(180:h)$) coordinate (F)
			($(B)+(F)-(A)$) coordinate (E)
			($(D)+(180:a)$) coordinate (P)
			($(C)+(P)-(D)$) coordinate (Q)
			;
			\draw (A)--(B)--(C)--(D)--cycle (F)--(A) (D)--(P)--(Q)--(C);
			\draw[dashed] (F)--(E)--(B);
			\draw (F)..controls +(30:1) and +(-100:1)..(P)node[midway,right]{$c$};
			\draw[dashed] (E)..controls +(30:1) and +(-100:1)..(Q)node[pos=0.3,right]{$c$};
			\foreach \t/\g in {A/-90,B/-90,C/0,D/-45,E/135,F/-90,P/180,Q/135}{
				\draw[fill=black] (\t) circle (1pt) node[shift={(\g:7pt)},font=\scriptsize]{$ \t $};
			}
		\end{tikzpicture}
	}
	\choice
	{$\dfrac{395}{24}$ cm$^3$}
	{$\dfrac{50}{3}$ cm$^3$}
	{$\dfrac{125}{8}$ cm$^3$}
	{\True $\dfrac{425}{24}$ cm$^3$}
	\loigiai{
		\immini{Gọi hình chiếu của $P$, $Q$ trên $AF$ và $BE$ là $S$ và $R$.\\
			Vật thể được chia thành hình lập phương $ABCD.PQRS$ có cạnh $2{,}5$ (cm), thể tích $V_1=\dfrac{125}{8}$ cm$^3$ và phần còn lại có thể tích $V_2$.\\
			Khi đó thể tích vật thể $V=V_1+V_2=\dfrac{125}{8}+V_2$.
		}
		{\begin{tikzpicture}[scale=0.7, font=\footnotesize, line join=round, line cap=round,>=stealth,declare function={a=3;h=6;}]
				\path (0,0) coordinate (A)
				--++(30:a) coordinate (B)
				--++(90:a) coordinate (C)
				($(A)+(C)-(B)$) coordinate (D)
				($(A)+(180:h)$) coordinate (F)
				($(B)+(F)-(A)$) coordinate (E)
				($(D)+(180:a)$) coordinate (P)
				($(C)+(P)-(D)$) coordinate (Q)
				($(A)+(P)-(D)$) coordinate (S)
				($(B)+(Q)-(C)$) coordinate (R)
				($(F)!0.5!(S)$) coordinate (M)
				($(M)+(90:2)$) coordinate (M_1)
				($(E)!0.5!(R)$) coordinate (K)
				($(K)+(90:2)$) coordinate (K_1)
				;
				\path[name path=d1] (F)..controls +(30:1) and +(-100:1)..(P);
				\path[name path=d2] (E)..controls +(30:1) and +(-100:1)..(Q);
				\path[name path=d3] (M)--(M_1);
				\path[name path=d4] (K)--(K_1);
				\path[name intersections={of=d1 and d3,by=N}];
				\path[name intersections={of=d2 and d4,by=H}];
				\fill[gray!30] (N)--(M)--(K)--(H)--cycle;
				\draw (A)--(B)--(C)--(D)--cycle (F)--(A) (D)--(P)--(Q)--(C) (P)--(S) (N)--(M);
				\draw[dashed] (F)--(E)--(B) (Q)--(R)--(S) (M)--(K)--(H) (N)--(H);
				\draw (F)..controls +(30:1) and +(-100:1)..(P);
				\draw[dashed] (E)..controls +(30:1) and +(-100:1)..(Q);
				\draw[-stealth] (A)--++(0:1)node[below]{$x$};
				\draw[-stealth] (F)--++(90:1.5*a)node[left]{$y$};
				\foreach \t/\g in {A/-90,B/-90,C/0,D/-45,E/115,F/-90,P/180,Q/135,S/-90,R/-90,N/90,M/-90,K/-90,H/135}{
					\draw[fill=black] (\t) circle (1pt) node[shift={(\g:7pt)},font=\scriptsize]{$ \t $};
				}
			\end{tikzpicture}
		}\noindent
		Đặt hệ trục $Oxyz$ sao cho $O$ trùng với $F$, $Ox$ trùng với $FA$, $Oy$ trùng với tia $Fy$ song song với $AD$. Khi đó Parabol $(P)$ có phương trình dạng $y=ax^2$, đi qua điểm $P\left(1;\dfrac{5}{2}\right)$ do đó $a=\dfrac{5}{2}\Rightarrow y=\dfrac{5}{2}{x^2}$.\\
		Cắt vật thể bởi mặt phẳng vuông góc với $Ox$ và đi qua điểm $M(x;0;0)$, $0\le x\le 1$ ta được thiết diện là hình chữ nhật $MNHK$ có cạnh là $ MN=\dfrac{5}{2}x^2$ và $ MK=\dfrac{5}{2}$ do đó diện tích $S(x)=\dfrac{25}{4}x^2$.\\
		Áp dụng công thức thể tích vật thể ta có $V_2=\displaystyle\int\limits_0^1\dfrac{25}{4}x^2\mathrm{\,d}x=\dfrac{25}{12}$.\\
		Từ đó $V=\dfrac{125}{8}+\dfrac{25}{12}=\dfrac{425}{24}\approx17{,}7$ cm$^3$.
		
	}
\end{ex}

%%==========Câu 14
\begin{ex}%[2D4V3-4]
	Bổ dọc một quả dưa hấu ta được thiết diện là hình elip có trục lớn $28$ cm, trục nhỏ $25$ cm. Biết cứ $1000$ m$^3$ dưa hấu sẽ làm được cốc sinh tố giá $20000$ đồng. Hỏi từ quả dưa hấu trên có thể thu được bao nhiêu tiền từ việc bán nước sinh tố? Biết rằng bề dày vỏ dưa không đáng kể.
	\choice
	{\True $183000$ đồng} 
	{$180000$ đồng} 
	{$185000$ đồng} 
	{$190000$ đồng}
	\loigiai{ 
		Đường elip có trục lớn $28$ cm, trục nhỏ $25$ cm có phương trình
		\[\dfrac{y^2}{\left(\dfrac{25}{2}\right)^2}=1\Leftrightarrow y^2=\left(\dfrac{25}{2}\right)^2\left(1-\dfrac{x^2}{14^2}\right)\Leftrightarrow y=\pm \dfrac{25}{2}\sqrt{1-\dfrac{x^2}{14^2}}.\]
		Do đó thể tích quả dưa là
		{\allowdisplaybreaks
			\begin{eqnarray*}
				V
				&=& \pi \int\limits_{-14}^{14}\left(\dfrac{25}{2}\sqrt{1-\dfrac{x^2}{14^2}} \right)^2\mathrm{d}x\\
				&=& \pi\left(\dfrac{25}{2}\right)^2\displaystyle\int\limits_{-14}^{14}\left(1-\dfrac{x^2}{14^2}\right)^2\mathrm{d}x\\
				&=&\pi\left(\dfrac{25}{2}\right)^2 \cdot \left(x-\dfrac{x^3}{3\cdot 14^2}\right) \Big|_{-14}^{14}\\
				&=& \pi\left(\dfrac{25}{2}\right)^2 \cdot \dfrac{56}{3}\\
				&=& \dfrac{8750\pi}{3} \;\;\text{cm}^3.
			\end{eqnarray*}
		}
		Do đó tiền bán nước thu được là $\dfrac{8750\pi \cdot 20000}{3\cdot 1000}\approx 183259$ đồng.
	} 
\end{ex} 
%------------------------------------------------------------
% \begin{ex}%[2D4V3-4]%Câu 1.
% 	\immini{Có một cốc nước thủy tinh hình trụ, bán kính trong lòng đáy cốc là $6\,\text{cm}$, chiều cao lòng cốc là $10\,\text{cm}$ đang đựng một lượng nước. Tính thể tích lượng nước trong cốc, biết khi nghiêng cốc nước vừa lúc khi nước chạm miệng cốc thì đáy mực nước trùng với đường kính đáy.
% 		\choice
% 		{\True $240$\,cm$^3$}
% 		{$240\pi$ \,cm$^3$}
% 		{$120$\,cm$^3$}
% 		{$120\pi$ \,cm$^3$}}
% 	{\begin{tikzpicture}[scale=0.7, font=\footnotesize,line join=round, line cap=round, >=stealth]
% 			\begin{scope}[shift={(0,0)}]
% 				\def\a{1.5}
% 				\def\b{.5}
% 				\def\h{4}
% 				\path
% 				(0,0) coordinate (M)
% 				($(M)+(2*\a,0)$) coordinate (N)
% 				($(M)!0.5!(N)$)coordinate (O)
% 				($(M)+(0,\h)$) coordinate (M')
% 				($(N)+(0,\h)$) coordinate (N')
% 				($(O)+(0,\h)$) coordinate (O')
% 				($(M')!0.6!(M)$)coordinate (A)
% 				($(N')!0.6!(N)$)coordinate (B)
% 				;
% 				\fill[black!15] (A) arc (180:0:\a cm and \b cm)--(N)--(M) arc (-180:0:\a cm and \b cm)--(M)--(A);
% 				\draw(M)--(M') (N)--(N');
% 				\draw[dashed,thin] (M) arc (180:0:\a cm and \b cm);
% 				\draw[dashed,thin] (A) arc (180:0:\a cm and \b cm);
% 				\draw (O') ellipse (\a cm and \b cm)	(M) arc (-180:0:\a cm and \b cm) (A) arc (-180:0:\a cm and \b cm);
% 			\end{scope}
% 			\begin{scope}[rotate=-70,shift={(0,6)}]
% 				\def\a{1.5}
% 				\def\b{.5}
% 				\def\h{4}
% 				\path
% 				(0,0) coordinate (M)
% 				($(M)+(2*\a,0)$) coordinate (N)
% 				($(M)!0.5!(N)$)coordinate (O)
% 				($(M)+(0,\h)$) coordinate (M')
% 				($(N)+(0,\h)$) coordinate (N')
% 				($(O)+(0,\h)$) coordinate (O')
% 				($(M')!0.6!(M)$) coordinate (A)
% 				($(N')!0.6!(N)$) coordinate (B)
% 				;
% 				\fill[black!15]  (2.1,-.45)--(.8,.45) .. controls +(70:2) and +(180:0) ..(N') (N').. controls +(-90:.3) and +(180:0) ..(N).. controls +(-90:.3) and +(0:0.3) ..(2.1,-.45);
% 				\draw(M)--(M') (N)--(N');
% 				\draw[dashed,thin] (M) arc (180:0:\a cm and \b cm);
% 				\draw (O') ellipse (\a cm and \b cm)	(M) arc (-180:0:\a cm and \b cm);
				
% 				\draw[dashed] (2.1,-.45)--(.8,.45) .. controls +(70:2) and +(180:0) ..(N') (1.5,0)--(N');
% 				\draw (2.1,-.45) .. controls +(40:1) and +(180:0) ..(N') ;
				
% 			\end{scope}
% 	\end{tikzpicture}} 
	
% 	\loigiai{
% 		\textbf{Cách 1.} 
% 		\begin{center}
% 			\begin{tikzpicture}[scale=1, font=\footnotesize,line join=round, line cap=round, >=stealth]
% 				\path
% 				(0,0) coordinate (A)
% 				(0,3) coordinate (B)
% 				;
				
% 				\draw (A) arc (0: -90: 5 and 2) coordinate (C);
% 				\draw[dashed] (A) arc (0: 90: 4.5 and 2) coordinate (D);
% 				\coordinate (I) at ($(C)!.5!(D)$);
% 				\draw (A)--(B) (C)--(D) (I)--(B);
% 				\draw[dashed] (I)--(A);
% 				\draw 
% 				(C) .. controls +(0:0) and +(-90:1.5) ..(B).. controls +(90:1.5) and +(60:0) ..(D);
% 				\draw pic[draw, angle radius=2mm]{right angle=B--A--I};
% 				\pic[draw,"$\alpha$", angle eccentricity=0.6,angle radius=0.8cm]{angle=A--I--B};
% 				\path (A)--(I) node[above,pos=.7,sloped]{$R$};
% 				\path (C)--(I) node[above,midway,sloped]{$R$};
% 				\path (D)--(I) node[above,midway,sloped]{$R$};
% 			\end{tikzpicture}
% 		\end{center}
% 		Xét thiết diện cắt cốc thủy tinh vuông góc với đường kính tại vị trí bất kỳ có  
% 		$$S(x)=\dfrac{1}{2}\sqrt{R^2-x^2}\cdot \sqrt{R^2-x^2}\cdot \tan \alpha= \dfrac{1}{2}\left( R^2-x^2 \right)\tan \alpha.$$
% 		Thể tích hình cái nêm là: $V=\dfrac{1}{2}\tan \alpha \displaystyle\int\limits_{-R}^{R}{\left( R^2-x^2 \right)}\mathrm{\,d}x=\dfrac{2}{3}R^3\tan \alpha $.\\
% 		Thể tích khối nước tạo thành khi nguyên cốc có hình dạng cái nêm nên $V_{kn}=\dfrac{2}{3}R^3\tan \alpha $. \\
% 		$\Rightarrow V_{kn}=\dfrac{2}{3}R^3\cdot \dfrac{h}{R}=240\,cm^3$.\\
% 		\textbf{Cách 2.} 
% 		\begin{center}
% 			\begin{tikzpicture}[scale=1, font=\footnotesize,line join=round, line cap=round, >=stealth]
% 				\path
% 				(0,0) coordinate (O)
% 				(0,4) coordinate (O')
% 				(7,0) coordinate (J)
% 				(7,4) coordinate (J')
% 				(9,0) coordinate (x)
% 				($(J)!.5!(J')$) coordinate (I)
% 				($(O)!.5!(O')$) coordinate (I')
% 				;
% 				\fill[cyan!20] (J)--(O) .. controls +(82:0.6) and +(180:0.6) ..
% 				(6.15,3.15)--(7.85,.9).. controls +(180:0.05) and +(0:.6) ..
% 				(7,0);
% 				\draw[->] (O)--(x);
% 				\draw[dashed,name path=OB] 
% 				(O) .. controls +(82:0.6) and +(180:0.6) ..
% 				(6.15,3.15)coordinate (B)--(7.85,.9) coordinate (A);
% 				\draw (0,2) ellipse (1 and 2) (O)--(O') (O')--(J') ;
% 				\draw (J) arc (-90: 90: 1 and 2);
% 				\draw[dashed] (J) arc (-90: 90: -1 and 2) (J)--(J') ;
% 				\draw[dashed,name path=II'] (I)--(I');
% 				\path [name intersections={of=OB and II',by=H}];
% 				\coordinate (E) at ($(J)!(H)!(O)$);
% 				\coordinate (N) at ($(O)!.55!(A)$);
				
% 				\path (intersection of H--E and O--I) coordinate (F);
% 				\coordinate (n) at ($(F)!-3!(N)$);
% 				\path[name path=FN] (N)--(n);
% 				\path [name intersections={of=OB and FN,by=M}];
				
% 				\fill[orange!50] (M) .. controls +(-90:1) and +(180:.5) ..(E) .. controls +(0:.5) and +(180:0) ..(N);
% 				\draw[dashed] (H)--(E) (M)--(N) (H)--(N) (O)--(I);
% 				\draw[dashed] (M) .. controls +(-90:1) and +(180:.5) ..(E);
% 				\draw (E) .. controls +(0:.5) and +(180:0) ..(N) (O)--(A);
% 				\draw[<->] ($(O)+(0,-.3)$)--($(E)+(0,-.3)$) node[fill=white,midway,sloped]{$x$};
% 				\draw[<->] ($(O')+(0,.3)$)--($(J')+(0,.3)$) node[fill=white,midway,sloped]{$10$ cm};
% 				\draw[<->] ($(J)+(1.5,0)$)--($(J')+(1.5,0)$) node[fill=white,midway,sloped]{$12$ cm};
% 				\draw[->] ($(E)+(.5,.3)$)--($(E)+(.8,-.5)$) node[below] {$S(x)$};
% 				\pic[draw,"$\alpha$", angle eccentricity=1.1,angle radius=2cm]{angle=J--O--I};
% 				\node[above right] at (F) {$\beta$};
% 				\foreach \x/\g in {H/90,E/-70,F/120,N/-90,M/90,I/0,J/-90,O/-120} \fill[black] (\x) circle (1pt)+(\g:.3) node {$\x$};
% 			\end{tikzpicture}
% 		\end{center}
% 		Dựng hệ trục tọa độ $Oxyz$.\\
% 		Gọi $S\left( x \right)$ là diện tích thiết diện do mặt phẳng có phương vuông góc với trục $Ox$ với khối nước, mặt phẳng này cắt trục $Ox$ tại điểm có hoành độ $h\ge x\ge 0$.\\
% 		Gọi $\widehat{IOJ}=\alpha ,\,\widehat{FHN}=\beta ,\,OE=x$\\
% 		$\tan \alpha =\dfrac{IJ}{OJ}=\dfrac{6}{10}=\dfrac{EF}{OE}\Rightarrow EF=\dfrac{6x}{10}\Rightarrow HF=6-\dfrac{6x}{10}$.\\
% 		$\cos \beta =\dfrac{HF}{HN}=\dfrac{6-\dfrac{6x}{10}}{6}=1-\dfrac{x}{10}\Rightarrow \beta =\arccos \left( 1-\dfrac{x}{10} \right)$\\
% 		%-------------------------------------
% 		$S\left( x \right)=S_{\text{hình quạt}}-S_{HMN}=\dfrac{1}{2}HN^2\cdot 2\beta -\dfrac{1}{2}HM\cdot HN\cdot \sin 2\beta $\\
% 		%-------------------------------------
% 		$\Rightarrow S\left( x \right)=6^2\arccos \left( 1-\dfrac{x}{10} \right)-\dfrac{1}{2}\cdot 6\cdot 6\cdot 2\left( 1-\dfrac{x}{10} \right)\sqrt{1-\left( 1-\dfrac{x}{10} \right)^2}$\\
% 		$\Rightarrow V=\displaystyle\int\limits_{0}^{10}{S\left( x \right) \,\mathrm{d}x}=\displaystyle\int\limits_{0}^{10}{\left( 36\arccos \left( 1-\dfrac{x}{10} \right)-36\left( 1-\dfrac{x}{10} \right)\sqrt{1-\left( 1-\dfrac{x}{10} \right)^2} \right)\,\mathrm{d}x}=240$.}
% \end{ex}
%------------------------------------------------------------
% \begin{ex}%[2D4V3-4]%Câu 2.
% 	\immini{Cho vật thể đáy là hình tròn có bán kính bằng 1 (tham khảo hình vẽ). Khi cắt vật thể bằng mặt phẳng vuông góc với trục $Ox$ tại điểm có hoành độ $x\ \left( -1\le x\le 1 \right)$ thì được thiết diện là một tam giác đều. Thể tích $V$ của vật thể đó là
% 		\choice
% 		{$V=\sqrt{3}$}
% 		{$V=3\sqrt{3}$}
% 		{\True $V=\dfrac{4\sqrt{3}}{3}$}
% 		{$V=\pi $}}
% 	{\includegraphics[scale=0.4]{images/Cau2_C4B3CD3.png} }
% 	\loigiai{
% 		\immini{
% 			Do vật thể có đáy là đường tròn và khi cắt bởi mặt phẳng vuông góc với trục $Ox$ được thiết diện là tam giác đều do đó vật thể đối xứng qua mặt phẳng vuông góc với trục $Oy$ tại điểm $O$.\\
% 			Cạnh của tam giác đều thiết diện là  $a=2\sqrt{1-x^2}$.\\
% 			Diện tích tam giác thiết diện là  
% 			$$S=\dfrac{a^2\sqrt{3}}{4}=\left( 1-x^2 \right)\sqrt{3}.$$
% 		}
% 		{\begin{tikzpicture}[scale=0.7, font=\footnotesize,line join=round, line cap=round, >=stealth]
% 				\path
% 				(0,0) coordinate (O)
% 				(60:3) coordinate (A)
% 				(-60:3) coordinate (B)
% 				($(A)!.5!(B)$) coordinate (x)
% 				;
% 				\draw (O) circle (3) (O)--(A)--(B);
% 				\draw[->] (-4,0) -- (4,0)node[below] {$x$};
% 				\draw[->] (0,-4) -- (0,4)node[right] {$y$};
% 				\path (A)--(B) node[above right,midway]{$\sqrt{1-x^2}$};
% 				\foreach \x/\g in {O/-120,x/-70} \fill[black] (\x) circle (1pt)+(\g:.3) node {$\x$};
% 		\end{tikzpicture}}
% 		\noindent
% 		Thể tích khối cần tìm là 
% 		$$V=2\displaystyle\int\limits_{0}^{1}{Sdx}=2\displaystyle\int\limits_{0}^{1}{\sqrt{3}\left( 1-x^2 \right)=\left. 2\sqrt{3}\left( x-\dfrac{x^3}{3} \right) \right|_{0}^{1}=\dfrac{4\sqrt{3}}{3}}.$$
% 	}
% \end{ex}
% %------------------------------------------------------------
% \begin{ex}%[2D4V3-4]%Câu 3.
% 	Sân vận động Sport Hub (Singapore) là sân có mái vòm kỳ vĩ nhất thế giới. Đây là nơi diễn ra lễ khai mạc Đại hội thể thao Đông Nam Á được tổ chức tại Singapore năm $2015$. Nền sân là một elip $\left( E \right)$ có trục lớn dài $150m$, trục bé dài $90m$ (hình vẽ). Nếu cắt sân vận động theo một mặt phẳng vuông góc với trục lớn của $\left( E \right)$và cắt elip ở $M,N$ (hình vẽ) thì ta được thiết diện luôn là một phần của hình tròn có tâm $I$ (phần tô đậm trong hình 4) với $MN$ là một dây cung và góc $\widehat{MIN}=90^{\circ}.$ Để lắp máy điều hòa không khí thì các kỹ sư cần tính thể tích phần không gian bên dưới mái che và bên trên mặt sân, coi như mặt sân là một mặt phẳng và thể tích vật liệu là mái không đáng kể. Hỏi thể tích xấp xỉ bao nhiêu?
% 	\begin{center}
% 		{\includegraphics[scale=0.8]{images/Cau3_C4B3CD3.png}\\
% 			\begin{tikzpicture}[scale=1, font=\footnotesize,line join=round, line cap=round, >=stealth]
% 				\path
% 				(0,0) coordinate (O)
% 				(-2,0) coordinate (A)
% 				(2,0) coordinate (B)
% 				(70: 2 and 1) ellipse  coordinate (M)
% 				(-70: 2 and 1) ellipse  coordinate (N)
% 				;
% 				\draw (O) ellipse (2 and 1) (M)--(N) (A)--(B);
% 				\fill[black] (B) circle (1pt);
% 				\fill[black] (A) circle (1pt);
% 				\foreach \x/\g in {M/90,N/-90} \fill[black] (\x) circle (1pt)+(\g:.3) node {$\x$};
% 				\begin{scope}[shift={(5,-0)}]
% 					\path (0,0) coordinate (I) (40:2)   coordinate (M)
% 					(140:2)  coordinate (N) ;
% 					\draw (I) circle (2) (M)--(N);
					
					
% 					\clip (-2,1.28) rectangle (2,2);
% 					\fill[black!15] (I) circle (2);
% 					\draw (I) circle (2) (M)--(N);
% 				\end{scope}
% 				\foreach \x/\g in {M/45,N/135,I/-90} \fill[black] (\x) circle (1pt)+(\g:.3) node {$\x$};
% 		\end{tikzpicture} }
% 	\end{center}
% 	\choice
% 	{$57793$ m$^3$}
% 	{\True $115586$ m$^3$}
% 	{$32162$ m$^3$}
% 	{$101793$ m$^3$}
% 	\loigiai{
% 		\begin{center}
% 			\includegraphics[scale=0.8]{images/Cau3_C4B3CD3_g.png}
% 		\end{center}
% 		Chọn hệ trục như hình vẽ\\
% 		Ta cần tìm diện tích của $S\left( x \right)$thiết diện.\\
% 		Gọi $d\left( O,MN \right)=x$\\
% 		$\left( E \right)\colon\dfrac{x^2}{75^2}+\dfrac{y^2}{45^2}=1.$\\
% 		Lúc đó $MN=2y=2\sqrt{45^2\left( 1-\dfrac{x^2}{75^2} \right)}=90\sqrt{1-\dfrac{x^2}{75^2}}$\\
% 		$\Rightarrow R=\dfrac{MN}{\sqrt{2}}=\dfrac{90}{\sqrt{2}}.\sqrt{1-\dfrac{x^2}{75^2}}\Rightarrow R^2=\dfrac{90^2}{2}\cdot \left( 1-\dfrac{x^2}{75^2} \right)$.\\
% 		$S\left( x \right)=\dfrac{1}{4}\pi R^2-\dfrac{1}{2}R^2=\left( \dfrac{1}{4}\pi -\dfrac{1}{2} \right)R^2=\left( \pi -2 \right)\dfrac{2025}{2}.\left( 1-\dfrac{x^2}{75^2} \right).$\\
% 		Thể tích khoảng không cần tìm là
% 		$$V=\displaystyle\int\limits_{-75}^{75}\left( \pi -2 \right)\dfrac{2025}{2}.\left( 1-\dfrac{x^2}{75^2} \right)\approx 115586 \,\text{m}^3.$$
% 	}
% \end{ex}
%------------------------------------------------------------
% \begin{ex}%[2D4V3-4]%Câu 4.
% 	\immini{Gọi $\left( H \right)$ là phần giao của hai khối $\dfrac{1}{4}$ hình trụ có bán kính $a$, hai trục hình trụ vuông góc với nhau như hình vẽ sau. Tính thể tích của khối $\left( H \right)$.
% 		\choice
% 		{${{V}_{\left( H \right)}}=\dfrac{a^3}{2}$}
% 		{${{V}_{\left( H \right)}}=\dfrac{3a^3}{4}$}
% 		{\True $V_{\left( H \right)}=\dfrac{2a^3}{3}$}
% 		{${{V}_{\left( H \right)}}=\dfrac{\pi a^3}{4}$}}
% 	{\includegraphics[scale=0.8]{images/Cau4_C4B3CD3_De.png}}
% 	\loigiai{
% 		\begin{center}
% 			\includegraphics[scale=0.8]{images/Cau4_C4B3CD3.png}
% 		\end{center}
% 		+ Đặt hệ toạ độ $Oxyz$ như hình vẽ, xét mặt cắt song song với mp $\left( Oyz \right)$ cắt trục $Ox$ tại $x$: thiết diện mặt cắt luôn là hình vuông có cạnh $\sqrt{a^2-x^2}$ $\left( 0\le x\le a \right)$.\\
% 		+ Do đó thiết diện mặt cắt có diện tích: $S\left( x \right)=a^2-x^2$.\\
% 		+ Vậy $V_{\left( H \right)}=\displaystyle\int\limits_{0}^{a}{S\left( x \right)\,\mathrm{\,d}x} =\displaystyle\int\limits_{0}^{a}{\left( a^2-x^2 \right)\,\mathrm{d}x} =\left. \left( a^2x-\dfrac{x^3}{3} \right) \right|_{0}^{a}$\\
% 		$=\dfrac{2a^3}{3}$}
% \end{ex}
%------------------------------------------------------------
\begin{ex}%[2D4H3-5]%Câu 5.
	Một bác thợ xây bơm nước vào bể chứa nước. Gọi $h\left( t \right)$ là thể tích nước bơm được sau $t$ giây. Cho ${h}'\left( t \right)=6at^2+2bt$ và ban đầu bể không có nước. Sau 3 giây thì thể tích nước trong bể là $90m^3$, sau $6$ giây thì thể tích nước trong bể là $504m^3$. Tính thể tích nước trong bể sau khi bơm được $9$ giây.
	\choice
	{\True $1458m^3$}
	{$600m^3$}
	{$2200m^3$}
	{$4200m^3$}
	\loigiai{
		$\displaystyle\int\limits_{0}^{3}\left( 6at^2+2bt \right)\,\mathrm{d}t=90\Leftrightarrow \left.\left( 2at^3+bt^2 \right) \right|_{0}^{3}=90\Leftrightarrow 54a+9b=90$\quad (1)\\
		$\displaystyle\int\limits_{0}^{6}\left( 6at^2+2bt \right)\,\mathrm{d}t=504 \Leftrightarrow  \left. \left( 2at^3+bt^2 \right) \right|_{0}^{6}=504\Leftrightarrow 432a+36b=504$\quad (2)\\
		Từ (1), (2) $\Rightarrow  \heva{
			& a=\dfrac{2}{3} \\ 
			& b=6.}$\\
		Sau khi bơm $9$ giây thì thể tích nước trong bể là\\
		$V=\displaystyle\int\limits_{0}^{9}\left(4t^2+12t \right)\,\mathrm{d}t =  \left. \left(\dfrac{4}{3}t^3+6t^2 \right) \right|_{0}^{9}=1458\; \left(m^3 \right)$.}
\end{ex}
%------------------------------------------------------------
\begin{ex}%[2D4H3-5]%Câu 6.
	Người ta thay nước mới cho một bể bơi có dạng hình hộp chữ nhật có độ sâu là $280$cm. Giả sử $h\left( t \right)$là chiều cao (tính bằng cm) của mực nước bơm được tại thời điểm $t$ giây, biết rằng tốc độ tăng của chiều cao mực nước tại giây thứ $t$ là ${h}'(t)=\dfrac{1}{500}\sqrt[3]{t}$ và lúc đầu hồ bơi không có nước. Hỏi sau bao lâu thì bơm được số nước bằng $\dfrac{3}{4}$độ sâu của hồ bơi (làm tròn đến giây)?
	\choice
	{$2$ giờ $36$ giây}
	{$2$ giờ $48$ giây}
	{\True $2$ giờ $38$ giây}
	{$2$ giờ $46$ giây}
	\loigiai{
		Gọi $x$ là thời điểm bơm được số nước bằng $\dfrac{3}{4}$ độ sâu của bể ($x$ tính bằng giây).
		Ta có
		\begin{eqnarray*}
			&&\displaystyle\int\limits_0^x{\dfrac{1}{500}\sqrt[3]{t}\mathrm{\,d}t}=\dfrac{3}{4}\cdot 280\left. \Rightarrow \dfrac{3}{4}t^{\dfrac{4}{3}} \right|_0^x=105000\\
			&\Rightarrow& x\sqrt[3]{x}=140000\Rightarrow \sqrt[3]{x^4}=140000\\
			&\Rightarrow& x=\sqrt[4]{140000^3}\Rightarrow x\approx 7237{,}6242.
		\end{eqnarray*}
		Suy ra $x= 2$ giờ $38$ giây.}
\end{ex}
%------------------------------------------------------------
\Closesolutionfile{ans}
\indapan{6}{ans/ans-2-C4B3CD3_1-2-lc}


%% Ôn KTTX chương IV
% \begin{name}
	{NGUYÊN HÀM - TÍCH PHÂN}
	{KT NGUYÊN HÀM}
	{\tentruong}
	{\thoigian}
\end{name}
\setcounter{ex}{0}\setcounter{bt}{0}
\Opensolutionfile{ans}[ans/ans-2-B11-De1-lc]
\TN
\begin{ex}%Câu 1%[2D4N1-2]
	Họ nguyên hàm của hàm số $f(x)=x^3$ là
	\choice
	{$4x^4+C$}
	{$3x^2+C$}
	{$x^4+C$}
	{\True $\dfrac{1}{4}x^4+C$}
	\loigiai{
Ta có 
	$\displaystyle\int x^3\mathrm{\,d}x=\dfrac{1}{4}x^4+C$.
}
\end{ex}

\begin{ex}%Câu 2%[2D4N1-3]
	Tìm nguyên hàm của hàm số $ f(x)=2\sin x$.
	\choice
	{\True $\displaystyle\int 2\sin x\mathrm{\,d}x=-2\cos x+C$}
	{$\displaystyle\int 2\sin x\mathrm{\,d}x=2\cos x+C$}
	{$\displaystyle\int 2\sin x\mathrm{\,d}x=\sin^2x+C$}
	{$\displaystyle\int 2\sin x\mathrm{\,d}x=\sin 2x+C$}
	\loigiai{
Ta có $\displaystyle\int 2\sin x\mathrm{\,d}x=2\displaystyle\int \sin x \mathrm{\,d}x=-2\cos x+C$.}
\end{ex}

\begin{ex}%Câu 3%[2D4N1-4]
	Họ nguyên hàm của hàm số $f(x)=\mathrm{e}^x+x$ là
	\choice
	{$\mathrm{e}^x+1+C$}
	{$\mathrm{e}^x+x^2+C$}
	{\True $\mathrm{e}^x+\dfrac{1}{2}{x^2}+C$}
	{$\dfrac{1}{x+1}{\mathrm{e}^x}+\dfrac{1}{2}{x^2}+C$}
	\loigiai{
Ta có $\displaystyle\int \left(\mathrm{e}^x+x\right) \mathrm{\,d}x=\displaystyle\int \mathrm{e}^x \mathrm{\,d}x+\displaystyle\int x \mathrm{\,d}x=\mathrm{e}^x+\dfrac{x^2}{2}+C$.}
\end{ex}

\begin{ex}%Câu 4%[2D4N1-2]
	Họ nguyên hàm của hàm số $y=x^2-3x+\dfrac{1}{x}$ là
	\choice
	{$\dfrac{x^3}{3}-\dfrac{3x^2}{2}-\ln\left|x\right|+C$}
	{$\dfrac{x^3}{3}-\dfrac{3x^2}{2}+\ln x+C$}
	{\True $\dfrac{x^3}{3}-\dfrac{3x^2}{2}+\ln\left|x\right|+C$}
	{$\dfrac{x^3}{3}-\dfrac{3x^2}{2}+\dfrac{1}{x^2}+C$}
	\loigiai{
	Ta có 
		$\displaystyle\int \left(x^2-3x+\dfrac{1}{x}\right) \mathrm{\,d}x=\dfrac{x^3}{3}-\dfrac{3x^2}{2}+\ln\left|x\right|+C$.
	
}
\end{ex}

\begin{ex}%Câu 5%[2D4H1-2]
	Tìm nguyên hàm của hàm số $f(x)=\dfrac{x^4+2}{x^2}$.
	\choice
	{$\displaystyle\int f(x)\mathrm{\,d}x=\dfrac{x^3}{3}-\dfrac{1}{x}+C$}
	{$\displaystyle\int f(x)\mathrm{\,d}x=\dfrac{x^3}{3}+\dfrac{2}{x}+C$}
	{$\displaystyle\int f(x)\mathrm{\,d}x=\dfrac{x^3}{3}+\dfrac{1}{x}+C$}
	{\True $\displaystyle\int f(x)\mathrm{\,d}x=\dfrac{x^3}{3}-\dfrac{2}{x}+C$}
	\loigiai{
Ta có
		$\displaystyle\int \dfrac{x^4+2}{x^2} \mathrm{\,d}x=\displaystyle\int \left(x^2+\dfrac{2}{x^2}\right) \mathrm{\,d}x=\dfrac{x^3}{3}-\dfrac{2}{x}+C$.}
\end{ex}

\begin{ex}%Câu 6%[2D4N1-3]
	Cho hàm số $ f(x)=1-\dfrac{1}{\cos^2 x}$. Khẳng định nào dưới đây đúng?
	\choice
	{$\displaystyle\int f(x)\mathrm{\,d}x=x+\tan x+C$}
	{$\displaystyle\int f(x)\mathrm{\,d}x=x+\cot x+C$}
	{\True $\displaystyle\int f(x)\mathrm{\,d}x=x-\tan x+C$}
	{$\displaystyle\int f(x)\mathrm{\,d}x=x-\cot x+C$}
	\loigiai{
Ta có $\displaystyle\int f(x) \mathrm{\,d}x=\displaystyle\int \left(1-\dfrac{1}{\cos^2x}\right)\mathrm{\,d}x=\displaystyle\int  \mathrm{\,d}x-\displaystyle\int \dfrac{\mathrm{\,d}x}{\cos^2x}=x-\tan x+C$.}
\end{ex}

\begin{ex}%Câu 7%[2D4N1-3]
	Họ nguyên hàm của hàm số $f(x)=\cos x+6x$ là
	\choice
	{\True $\sin x+3x^2+C$}
	{$-\sin x+3x^2+C$}
	{$\sin x+6x^2+C$}
	{$-\sin x+C$}
	\loigiai{
Ta có $\displaystyle\int f(x)\mathrm{\,d}x=\displaystyle\int \left(\cos x+6x\right)\mathrm{\,d}x=\sin x+3x^2+C$.}
\end{ex}

\begin{ex}%Câu 8%[2D4N1-2]
	$\displaystyle\int f(x)\mathrm{\,d}x=4x^3+x^2+C$ thì hàm số $f(x)$ bằng
	\choice
	{$f(x)=x^4+\dfrac{x^3}{3}+Cx$}
	{$f(x)=12x^2+2x+C$}
	{\True $f(x)=12x^2+2x$}
	{$f(x)=x^4+\dfrac{x^3}{3}$}
	\loigiai{
		Ta có $ f(x)=F'(x)=\left(4x^3+x^2+C\right)'=12x^2+2x$.}
\end{ex}

\begin{ex}%Câu 9%[2D4N1-4]
	Hàm số $F(x)=2x+3^x-1$ là nguyên hàm của hàm số nào trong các hàm số sau
	\choice
	{\True $f(x)=2+3^x\ln 3$}
	{$f(x)=x^2+\dfrac{3^x}{\ln 3}-x+C$}
	{$f(x)=x^2+\dfrac{3^x}{\ln 3}-x$}
	{$f(x)=2+3^x\ln 3+C$}
	\loigiai{
	Ta có $ f(x)=F'(x)\Rightarrow f(x)=\left(2x+3^x-1\right)'=2+3^x\ln 3$.}
\end{ex}

\begin{ex}%Câu 10%[2D4H1-4]
	Cho$\displaystyle\int \ln x \mathrm{\,d}x=F(x)+C$. Khẳng định nào dưới đây đúng?
	\choice
	{$F'(x)=\dfrac{1}{x}$}
	{$F'(x)=\dfrac{1}{x}+C$} 
	{\True $F'(x)=\ln x$}
	{$F'(x)=\ln x+1$}
	\loigiai{
Ta có $ F'(x)=f(x)=\ln x$.}
\end{ex}

\begin{ex}%Câu 11%[2D4H1-4]
	Cho $F(x)$ là một nguyên hàm của hàm số $f(x)=\mathrm{e}^x+2x$ thỏa mãn $F(0)=\dfrac{3}{2}$. Tìm $F(x)$.
	\choice
	{\True $F(x)=\mathrm{e}^x+x^2+\dfrac{1}{2}$}
	{$F(x)=\mathrm{e}^x+x^2+\dfrac{5}{2}$}
	{$F(x)=\mathrm{e}^x+x^2+\dfrac{3}{2}$}
	{$F(x)=2\mathrm{e}^x+x^2-\dfrac{1}{2}$}
	\loigiai{
Ta có $ F(x)=\displaystyle\int\left(\mathrm{e}^x+2x\right) \mathrm{\,d}x=e^x+x^2+C$.\\
		Theo bài ra ta có $F(0)=1+C=\dfrac{3}{2}\Rightarrow C=\dfrac{1}{2}$.}
\end{ex}

\begin{ex}%Câu 12%[2D4H1-6]
	Một viên đạn được bắn thẳng đứng lên trên từ mặt đất. Giả sử tại thời điểm $t$ giây (coi $t=0$ là thời điểm viên đạn được bắn lên), vận tốc của nó được cho bởi $v(t)=160-9{,}8t$ (m/s). Độ cao của viên đạn (tính từ mặt đất) sau $t=10$ giây là
	\choice
	{$620$ m}
	{$1\,240$ m}
	{$555$ m}
	{\True $1\,110$ m}
	\loigiai{
Gọi $S(t)$ là độ cao của viên đạn sau $t$ giây kể từ lúc bắt đầu bắn.\\
		Ta có $v(t)=S'(t)$. Do đó, $S(t)$ là một nguyên hàm của vận tốc $ v(t)$.\\
		$S(t)=\displaystyle\int v(t) \mathrm{\,d}t=\displaystyle\int \left(160-9{,}8t\right)\mathrm{\,d}t=160t-4{,}9t^2+C$.\\
		Theo giả thiết, $S(0)=0$ nên $C=0$ và ta được $S(t)=160t-4{,}9t^2$ (m).\\
		Độ cao của viên đạn sau $t=10$ giây là
		$S(10)=160\cdot 10-4{,}9\cdot10^2=1\,110$ (m).\\
		Vậy độ cao của viên đạn (tính từ mặt đất) sau $t=10$ giây là $ 1\,110$ (m).\\
}
\end{ex}
 \Closesolutionfile{ans}
\indapan{6}{ans/ans-2-B11-De1-lc}
\Opensolutionfile{ans}[ans/ans-2-B11-De1-ds]
\TNTF
\begin{ex}%Câu 13%[2D4H1-4]
	Cho hàm số $y=h(x)$ có đạo hàm $h'(x)=3x^2$ và hàm số $y=g(x)$ có đạo hàm $g'(x)=\mathrm{e}^x$.
	\choiceTF
	{Hàm số $y=h(x)=6x+C_1$, với $C_1\in \mathbb{R}$}
	{\True Hàm số $y=g(x)=\mathrm{e}^x+C_2$, với $C_2\in \mathbb{R}$}
	{\True $ I=\displaystyle\int \left[xh'(x)+2025\right]\mathrm{\,d}x=\dfrac{3}{4}{x^4}+2025x+C$ với $C\in \mathbb{R}$}
	{\True Cho $ f'(x)=3x^2+\mathrm{e}^x+m-1$. Cho $f(0)=2$; $ f(1)=2\mathrm{e}$ thì giá trị của $m\in (1;2)$}
\loigiai{
\begin{itemchoice}
	\itemch \textbf{Sai}. Ta có $ h(x)=\displaystyle\int h'(x) \mathrm{\,d}x=3\displaystyle\int x^2\mathrm{\,d}x=x^3+C_1$.
	\itemch \textbf{Đúng}. Ta có $ g(x)=\displaystyle\int g'(x)\mathrm{\,d}x=\displaystyle\int \mathrm{e}^x\mathrm{\,d}x=\mathrm{e}^x+C_2$.
	\itemch \textbf{Đúng}. Ta có $ I=\displaystyle\int \left[xh'(x)+2025\right]\mathrm{\,d}x=\displaystyle\int \left[3x^3+2025\right]\mathrm{\,d}x=\dfrac{3}{4}{x^4}+2025x+C$.
	\itemch \textbf{Đúng}. Ta có $ f(x)=\displaystyle\int f'(x)\mathrm{\,d}x=\displaystyle\int \left(3x^2+\mathrm{e}^x+m-1\right)\mathrm{\,d}x=x^3+\mathrm{e}^x+(m-1)x+C$.\\
	Vì $\heva{
		&f(0)=2\\
		&f(1)=2\mathrm{e}}
	\Rightarrow\heva{
		&1+C=2\\
		&1+\mathrm{e}+m-1+C=2\mathrm{e}}
\Rightarrow\heva{
		&C=1\\
		&m=\mathrm{e}-1.}
	$\\
	Vậy $m=\mathrm{e}-1\Rightarrow 1<m<2$.
	\end{itemchoice}
}
\end{ex}

\begin{ex}%Câu 14%[2D4H1-3]
Cho các hàm số $g(x)=\sin x$ , $h(x)=\cos x$.
\choiceTF
{$\displaystyle\int \left[2g(x)-3h(x)\right]\mathrm{\,d}x=3\displaystyle\int g(x)\mathrm{\,d}x-2\displaystyle\int h(x)\mathrm{\,d}x$}
{\True Một nguyên của hàm số $g(x)$ là $-\cos x$}
{Họ nguyên của hàm số $h(x)+2\sqrt{x}$ là $\sin x+\dfrac{3}{2}\sqrt{x^3}+C$}
{\True Họ nguyên hàm của hàm số $f(x)=g(x)\cdot h^2(x)$ là $F(x)=-\dfrac{1}{3}\cos^3 x+C$}
\loigiai{
\begin{itemchoice}
\itemch \textbf{Sai}. Ta có $\displaystyle\int \left[2g(x)-3h(x)\right]\mathrm{\,d}x=2\displaystyle\int g(x)\mathrm{\,d}x-3\displaystyle\int h(x)\mathrm{\,d}x$.
\itemch \textbf{Đúng}. Ta có $\displaystyle\int g(x)\mathrm{\,d}x=\displaystyle\int \sin x \mathrm{\,d}x=-\cos x+C$ nên một nguyên của hàm số $g(x)$ là $-\cos x$.
\itemch \textbf{Sai}. Ta có $\displaystyle\int \left[h(x)+2\sqrt x\right]\mathrm{\,d}x=\displaystyle\int \cos x\mathrm{\,d}x+2\displaystyle\int x^{\tfrac{1}{2}}\mathrm{\,d}x=\sin x+\dfrac{4}{3}\sqrt{x^3}+C$.
\itemch \textbf{Đúng}. Ta có $F'(x)=-\cos^2 x\cdot (\cos x)'=-\cos ^2 x\cdot (-\sin x)=\sin x\cdot \cos^2 x=g(x)\cdot h^2(x)$.
\end{itemchoice}}
\end{ex} 
\begin{ex}%Câu 15%[2D4H1-4]
Cho các hàm số $g(x)=\dfrac{1}{x^2}$, $h(x)=\ln (x+3)$.
\choiceTF
{ Biết $G(x)$ là một nguyên hàm của $g(x)$ và $ G(1)=1$. Khi đó $ G(2)=-\dfrac{1}{2}$}
{\True $ J=\displaystyle\int \left[h(x)+\ln\dfrac{1}{x+3}+2025\right]\mathrm{\,d}x=2025x+C$}
{$I=\displaystyle\int x\cdot h'(x)\mathrm{\,d}x=x-\ln (x+3)+C$ với $C\in \mathbb{R}$}
{\True Giả sử $F(x)$ là một nguyên hàm của $f(x)=\dfrac{x+3}{g(x)}$ và $F(1)=\dfrac{1}{4}$.\\ Khi đó $F(-1)=-\dfrac{7}{4}$}
\loigiai{
\begin{itemchoice}
\itemch \textbf{Sai}. Ta có $ G(x)=\displaystyle\int{g(x)}{\rm{d}}x=\displaystyle\int{\dfrac{1}{x^2}}{\rm{d}}x=\displaystyle\int{x^{-2}}{\rm{d}}x=-\dfrac{1}{x}+C$.\\
Mà $ G(1)=1\Rightarrow C=2$ $\Rightarrow G(x)=-\dfrac{1}{x}+2\Rightarrow G(2)=\dfrac{3}{2}$.
\itemch \textbf{Đúng}. Ta có 
\begin{eqnarray*}
 J&=&\displaystyle\int \left[h(x)+\ln\dfrac{1}{x+3}+2025\right]\mathrm{\,d}x=2025x+C\\
&=&\displaystyle\int \left[\ln (x+3) + \ln\dfrac{1}{x+3}+2025 \right]\mathrm{\,d}x\\
&=&\displaystyle\int \left(\ln 1+2025\right)\mathrm{\,d}x=2025x+C.
\end{eqnarray*}
\itemch \textbf{Sai}. Ta có $\left[x-\ln\left(x+3\right)+C\right]'=1-\dfrac{1}{x+3}=\dfrac{x+2}{x+3}. \quad(1)$ \\
Và $ x\cdot h'(x)=x\cdot \left(\ln (x+3)\right)'=\dfrac{x}{x+3}. \quad(2)$\\
Từ $(1)$ và $(2)$ suy ra $\displaystyle\int xh'(x)\mathrm{\,d}x \ne x-\ln (x+3)+C$.\\
\itemch \textbf{Đúng}. Ta có 
\begin{eqnarray*}
	\displaystyle\int \dfrac{x+3}{g(x)}\mathrm{\,d}x &=&\displaystyle\int \dfrac{x+3}{\dfrac{1}{x^2}}\mathrm{\,d}x\\
	&=&\displaystyle\int x^2(x+3)\mathrm{\,d}x=\displaystyle\int x^3\mathrm{\,d}x+3\displaystyle\int x^2\mathrm{\,d}x\\
	&=&\dfrac{1}{4}{x^4}+x^3+C.
\end{eqnarray*}
Mà $F(1)=\dfrac{1}{4}\Leftrightarrow C=-1$.\\
$F(x)=\dfrac{1}{4}{x^4}+x^3-1\Rightarrow F(-1)=-\dfrac{7}{4}$.
\end{itemchoice}
}
\end{ex}
\begin{ex}%Câu 16%[2D4H1-4]
Cho các hàm số $g(x)=\mathrm{e}^{\tfrac{x}{2}}$, $h(x)=2x^3+5x^2-2x+4$.
\choiceTF
{\True $\displaystyle\int \left[2g(x)+3h(x)\right]\mathrm{\,d}x=2\displaystyle\int g(x)\mathrm{\,d}x+3\displaystyle\int h(x)\mathrm{\,d}x$}
{\True Một nguyên của hàm số $3\cdot g^2(x)$ là $3\mathrm{e}^x$}
{Họ nguyên của hàm số $h(x)$ là $\dfrac{1}{4}{x^3}+\dfrac{5}{3}{x^3}-x^2+C$}
{\True Biết $\displaystyle\int g^4(x)\cdot h(x)\mathrm{\,d}x=(a{x^3}+b{x^2}+cx+d)\mathrm{e}^{2x}+C$. Khi đó $ a+b+c+d=3$}
\loigiai{
\begin{itemchoice}
\itemch \textbf{Đúng}. Ta có $\displaystyle\int \left[2g(x)+3h(x)\right]\mathrm{\,d}x=2\displaystyle\int g(x)\mathrm{\,d}x+3\displaystyle\int h(x)\mathrm{\,d}x$.
\itemch \textbf{Đúng}. Ta có $\displaystyle\int 3g^2(x)=3\displaystyle\int \mathrm{e}^x\mathrm{\,d}x=3\mathrm{e}^x+C$.
\itemch \textbf{Sai}. Ta có $\displaystyle\int h(x)\mathrm{\,d}x=\displaystyle\int \left(2x^3+5x^2-2x+4\right)\mathrm{\,d}x=\dfrac{1}{2}{x^4}+\dfrac{5}{3}{x^3}-x^2+4x+C$.
\itemch \textbf{Đúng}. Ta có 
\begin{eqnarray*}
	&&\displaystyle\int g^4(x)\cdot h(x)\mathrm{\,d}x=(a{x^3}+b{x^2}+cx+d){\mathrm{e}^{2x}}+C\\
	&\Leftrightarrow& \displaystyle\int{\mathrm{e}^{2x}}(2x^3+5x^2-2x+4)\mathrm{\,d}x=(a{x^3}+b{x^2}+cx+d){\mathrm{e}^{2x}}+C.
\end{eqnarray*}
Nên
\begin{eqnarray*}
	\left((ax^3+bx^2+cx+d)\mathrm{e}^{2x}+C\right)'&=&(3ax^2+2bx+c)\mathrm{e}^{2x}+2\mathrm{e}^{2x}\left(ax^3+bx^2+cx+d\right)\\
	&=&\left(2ax^3+(3a+2b)x^2+(2b+2c)x+c+2d\right)\mathrm{e}^{2x}\\
	&=&(2x^3+5x^2-2x+4)\mathrm{e}^{2x}.
\end{eqnarray*}
Do đó $\heva{&2a=2\\&3a+2b=5\\&2b+2c=-2\\
&c+2d=4}\Leftrightarrow\heva{&a=1\\&b=1\\&c=-2\\&d=3.}$\\
Vậy $a+b+c+d=3$.
\end{itemchoice}
}
\end{ex}
\Closesolutionfile{ans}
\indapan{2}{ans/ans-2-B11-De1-ds}
\Opensolutionfile{ans}[ans/ans-2-B11-De1-kq]
\TNSA
\begin{ex}%Câu 17%[2D4H1-3]
Cho $F(x)$ là một nguyên hàm của hàm $f(x)=\dfrac{\cos 2x}{\sin x+\cos x}$ thỏa mãn $F(0)=1$. Tính $F(\pi)$.
\shortans{-1}
\loigiai{
Ta có
\allowdisplaybreaks
\begin{eqnarray*}
	F(x)&=&\displaystyle\int \dfrac{\cos 2x}{\sin x+\cos x}\mathrm{\,d}x 
	=\displaystyle\int \dfrac{\cos^2x-\sin^2x}{\sin x+\cos x}\mathrm{\,d}x\\
	&=&\displaystyle\int \dfrac{(\sin x+\cos x)(\cos x-\sin x)}{\sin x+\cos x} \mathrm{\,d}x\\
	&=&\displaystyle\int (\cos x-\sin x)\mathrm{\,d}x=\sin x+\cos x+C.
\end{eqnarray*}
Do $F(0)=1$ nên $C=0\Rightarrow F(x)=\sin x+\cos x \Rightarrow F(\pi)=-1$.}
\end{ex}

\begin{ex}%Câu 18%[2D4V1-4]
$F(x)$ là một nguyên hàm của hàm số $f(x)=2^x$, thỏa mãn $ F(0)=\dfrac{1}{\ln 2}$. Biểu thức $ F(0)+F(1)+F(2)+\ldots+F(2024)=\dfrac{a^b-c}{\ln a}$ $(a$, $b$, $c\in{N^*})$. Tính $ T=a+b-2c$.
\shortans{2025}
\loigiai{
Ta có $F(x)=\displaystyle\int 2^x \mathrm{\,d}x=\dfrac{2^x}{\ln 2}+C$.\\
Theo giả thiết $F(0)=\dfrac{1}{\ln 2}\Leftrightarrow\dfrac{2^0}{\ln 2}+C=\dfrac{1}{\ln 2}\Leftrightarrow C=0 \Rightarrow F(x)=\dfrac{2^x}{\ln 2}$.\\
Khi đó 
\allowdisplaybreaks
\begin{eqnarray*}
	F(0)+F(1)+F(2)+\ldots+F(2024)&=&\dfrac{2^0}{\ln 2}+\dfrac{2^1}{\ln 2}+\dfrac{2^2}{\ln 2}+\ldots+\dfrac{2^{2024}}{\ln 2}\\
	&=&\dfrac{1}{\ln 2}(2^0+2^1+2^2+\ldots+2^{2024})\\
	&=&\dfrac{1}{\ln 2}\cdot \dfrac{1(1-2^{2025})}{1-2}=\dfrac{2^{2025}-1}{\ln 2}.
\end{eqnarray*}
$\Rightarrow a=2$, $b=2025$, $c=1$.\\
Vậy $ T=a+b-2c=2025$.}
\end{ex}

\begin{ex}%Câu 19%[2D4V1-2]
Gọi $F(x)$ là một nguyên hàm của hàm số $f(x)=\dfrac{(3x-1)^2}{x^2}$, biết đồ thị hàm số $y=F(x)$ đi qua điểm $M(1;-2)$. Tính $F\left(\mathrm{e}^2\right)$ (làm tròn kết quả đến hàng phần chục).
\shortans{44{,}4}
\loigiai{
Ta có
\allowdisplaybreaks
\begin{eqnarray*}
F(x)&=&\displaystyle\int \dfrac{(3x-1)^2}{x^2}\mathrm{\,d}x=\displaystyle\int \left(\dfrac{9x^2-6x+1}{x^2}\right) \mathrm{\,d}x\\
&=&\displaystyle\int \left(9-\dfrac{6}{x}+\dfrac{1}{x^2}\right) \mathrm{\,d}x\\
&=&9\displaystyle\int \mathrm{\,d}x-6\displaystyle\int\dfrac{1}{x} \mathrm{\,d}x+\displaystyle\int x^{-2} \mathrm{\,d}x\\
&=&9x-6\ln\left| x\right|-\dfrac{1}{x}+C.
\end{eqnarray*}
Theo giả thiết, đồ thị hàm số $y=F(x)$ đi qua điểm $M(1;-2)$ nên suy ra
\allowdisplaybreaks
\begin{eqnarray*}
F(1)=-2&\Rightarrow &9-6\ln 1-1+C=-2\\
&\Rightarrow & C=-10\Rightarrow F(x)=9x-6\ln\left|x\right|-\dfrac{1}{x}-10\\
&\Rightarrow& F\left(e^2\right)=9e^2-6\ln\left|\mathrm{e}^2\right|-\dfrac{1}{\mathrm{e}^2}-10\approx 44{,}4.
\end{eqnarray*}
}
\end{ex}

\begin{ex}%Câu 20%[2D4V1-6]
Một xe ô tô đang chạy với tốc độ $90$ km/h thì người lái xe bất ngờ phát hiện chướng ngại vật trên đường cách đó $150$ m. Người lái xe phản ứng $2$ giây sau đó bằng cách đạp phanh cho xe chạy chậm hơn. Kể từ thời điểm này, ô tô chuyển động chậm dần đều với tốc độ $v(t)=-\dfrac{25}{4}t+25$(m/s), trong đó $t$ là thời gian tính bằng giây kể từ lúc đạp phanh. Quãng đường xe ô tô đã di chuyển kể từ lúc người lái xe phát hiện chướng ngại vật trên đường đến khi xe ô tô dừng hẳn là bao nhiêu mét?\\
\shortans{100}
\loigiai{
Gọi $s(t)$ là quãng đường xe ô tô đi được trong $t$ (giây) kể từ lúc đạp phanh.\\ Khi đó
$ s(t)=\displaystyle\int v(t) \mathrm{\,d}t=\displaystyle\int \left(-\dfrac{25}{4}t+25\right) \mathrm{\,d}t=-\dfrac{25}{8}{t^2}+25t+C$.\\
Do $s(0)=0$ nên $C=0$ . Suy ra $s(t)=-\dfrac{25}{8}{t^2}+25t$.\\
Xe ô tô dừng hẳn khi $ v(t)=0\Leftrightarrow-\dfrac{25}{4}t+25=0\Leftrightarrow t=4$.\\
Suy ra quãng đường xe ô tô còn di chuyển được kể từ lúc đạp phanh đến khi xe dừng hẳn là 
$s(4)=-\dfrac{25}{8}{4^2}+25\cdot 4=50$ (m).\\
Ta có tốc độ $90$ km/h cũng là tốc độ $25$ m/s.\\
Do đó, quãng đường xe ô tô đã di chuyển kể từ lúc người lái xe phát hiện chướng ngại vật trên đường đến khi xe ô tô dừng hẳn là: $ 25\cdot 2+50=100$ (m).
}
\end{ex}

\begin{ex}%Câu 21%[2D4V1-6]
Một quần thể vi khuẩn ban đầu gồm $500$ vi khuẩn, sau đó bắt đầu tăng trưởng. Gọi $P(t)$ là số lượng vi khuẩn của quần thể đó tại thời điểm $t$, trong đó $ t$ tính theo ngày $(0\le t\le 10)$. Tốc độ tăng trưởng của quần thể vi khuẩn đó cho bởi hàm số $P'(t)=k\sqrt{t}$, trong đó k là hằng số. Sau 1 ngày, số lượng vi khuẩn của quần thể đó đã tăng lên thành 600 vi khuẩn (Nguồn: R. Larson and
Edwards, Calculus 10e, Cengage 2014). Tính số lượng vi khuẩn của quần thể đó sau $9$ ngày.

\shortans{3200}
\loigiai{
Ta có $P(t)=\displaystyle\int P'(t)\mathrm{\,d}t=\displaystyle\int k\sqrt{t} \mathrm{\,d}t=\displaystyle\int k\cdot t^{\tfrac{1}{2}}\mathrm{\,d}t=k\cdot\dfrac{2}{3}t\sqrt{t}+C$.\\
Từ giả thiết suy ra $\heva{&P(0)=500\\&P(1)=600
}\Rightarrow\heva{&k\cdot\dfrac{2}{3}\cdot0\sqrt 0+C=500\\&k\cdot\dfrac{2}{3}\cdot1\sqrt 1+C=600}\Rightarrow\heva{&C=500\\&
\dfrac{2}{3}k=100}\Rightarrow\heva{&C=500\\&k=150.}$\\
$\Rightarrow P(t)=100t\sqrt t+500$.\\
Do đó, số lượng vi khuẩn của quần thể đó sau $9$ ngày là $P(9)=100\cdot 9\sqrt{9}+500=3\,200$.}
\end{ex}

\begin{ex}%Câu 22%[2D4V1-6]
Cây cà chua khi trồng có chiều cao $5$ cm. Tốc độ tăng chiều cao của cây cà chua sau khi trồng được cho bởi hàm số $ v(t)=-0{,}1t^3+t^2$, trong đó $t$ tính theo tuần, $v(t)$ tính bằng cm/tuần. Gọi $h(t)$ (tính bằng centimét) là độ cao của cây cà chua ở tuần thứ $t$ (Nguồn:A. Bigalke et al., Mathematik, Grundkurs ma-I, Cornelsen 2016). Vào thời điểm cây cà chua đó phát triển nhanh nhất thì cây cà chua sẽ cao bao nhiêu? (làm tròn kết quả đến hàng phần chục).
\shortans{54{,}4}
\loigiai{
Ta có $h(t)=\displaystyle\int v(t) \mathrm{\,d}t=\displaystyle\int \left(-0{,}1t^3+t^2\right) \mathrm{\,d}t=-\dfrac{1}{40}{t^4}+\dfrac{t^3}{3}+C$.\\
Cây cà chua khi trồng có chiều cao $5$ cm nên $h(0)=5\Rightarrow C=5$.\\
Vậy độ cao của cây cà chua ở tuần thứ $t$ được cho bởi hàm số\\ 
\centerline {$h(t)=-\dfrac{1}{40}{t^4}+\dfrac{t^3}{3}+5$ $(t\ge 0)$.}\\
Ta tìm $ t$ $(t\ge 0)$ sao cho $v(t)$ đạt giá trị lớn nhất.\\
$v'(t)=-0{,}3t^2+2t$; $ v'(t)=0\Leftrightarrow-0{,}3t^2+2t=0\Leftrightarrow\hoac{&t=0\\&t=\dfrac{20}{3}.}$\\
Bảng biến thiên
\begin{center}
	\begin{tikzpicture}
	\tkzTabInit[espcl=2.5,lgt=1.5,nocadre]
	{$x$/0.7,$y'$/0.7,$y$/2.1}
	{$-\infty$,$0$,$\tfrac{20}{3}$,$+\infty$}
	\tkzTabLine{,-,0,+,0,-,}
	\tkzTabVar{+/$+\infty$,-/$0$,+/$\dfrac{400}{27}$,-/$-\infty$}
\end{tikzpicture}
\end{center}
Từ đó ta thấy $v(t)$ đạt giá trị lớn nhất tại $t=\dfrac{20}{3}$.\\
Khi đó, cây cà chua sẽ đạt chiều cao là $h\left(\dfrac{20}{3}\right)=\dfrac{4\,405}{81}\approx 54{,}4$ (cm).
}
\end{ex}
\Closesolutionfile{ans}
\indapan{6}{ans/ans-2-B11-De1-kq}
% \begin{name}
	{NGUYÊN HÀM - TÍCH PHÂN}
	{KT NGUYÊN HÀM}
	{\tentruong}
	{\thoigian}
\end{name}
\setcounter{ex}{0}\setcounter{bt}{0}
\TN
\Opensolutionfile{ans}[ans/ans-2C4B11-De2]
\begin{ex}%[2D4N1-2]
	Họ nguyên hàm của hàm số $f(x)=3x^2+1$ là
	\choice
	{$x^3+C$}
	{$\dfrac{x^3}{3}+x+C$}
	{$6x+C$}
	{\True $x^3+x+C$}
	\loigiai{
	$\displaystyle\int(3x^2+1)\mathrm{d}x=x^3+x+C$.
	}
\end{ex} 
\begin{ex}%[2D4N1-2]
	Hàm số nào sau đây là một nguyên hàm của hàm số $y=12x^5$?
	\choice
	{$y=12x^4$}
	{$y=60x^4$}
	{$y=12x^6+5$}
	{\True $y=2x^6+3$}
	\loigiai{		
		Ta có $\displaystyle\int{12x^5\mathrm{d}\,x}=12\cdot\dfrac{x^6}{6}+C=2x^6+C$.}
\end{ex} 
\begin{ex}%[2D4N1-2]
	Tìm họ nguyên hàm $F(x)$ của hàm số $f(x)=\dfrac{1}{x}$.
	\choice
	{\True  $F(x)=\ln \left| x \right|+C$}
	{$F(x)=\ln x+C$}
	{$F(x)=\ln \left| x \right|$}
	{$F(x)=-\dfrac{1}{x^2}+C$}
	\loigiai{		
		Áp dụng công thức nguyên hàm của hàm số ta có $\displaystyle\int{\frac{1}{x}\mathrm{d}\,x}=\ln \left| x \right|+C$.}
\end{ex} 
\begin{ex}%[2D4N1-3]
	Mệnh đề nào \textbf{sai} trong các mệnh đề sau?
	\choice
	{ $\displaystyle\int\cos x\,\mathrm{d}x=\sin x+C$}
	{\True $\displaystyle\int \sin x \, \mathrm{d}x=\cos x+C$}
	{$\displaystyle\int{\dfrac{1}{\cos^2x}\, \mathrm{d}x=\tan x+C}$}
	{$\displaystyle\int{\dfrac{1}{\sin^2x}\, \mathrm{d}x=-\cot x+C}$}
	\loigiai{
		
		Từ bảng nguyên hàm của các hàm cơ bản suy ra $\displaystyle\int \sin x \, \mathrm{d}x=\cos x+C$ sai}
\end{ex} 
\begin{ex}%[2D4N1-4]
	Tìm nguyên hàm của hàm số $f(x)=7^x$.
	\choice
	{$\displaystyle\int 7^x\mathrm{d}\,x=\dfrac{7^{x+1}}{x+1}+C$}
	{$\displaystyle\int 7^x\mathrm{d}\,x=7^x\ln 7+C$}
	{\True $\displaystyle\int 7^x \mathrm{d}\,x=\dfrac{7^x}{\ln 7}+C$}
	{$\displaystyle\int 7^x\mathrm{d}\,x=7^{x+1}+C$}
	\loigiai{		
		Áp dụng công thức nguyên $\displaystyle\int a^x\mathrm{d}\,x=\dfrac{a^x}{\ln a}+C \Rightarrow \displaystyle  \int 7^x\mathrm{d}\,x=\dfrac{7^x}{\ln 7}+C$.}
\end{ex} 
\begin{ex}%[2D4N1-4]
	Nguyên hàm của hàm số $F(x)=2^x+x$ là
	\choice
	{ $2^x+\dfrac{x^2}{2}+C$}
	{$2^x+x^2+C$}
	{$\dfrac{2^x}{\ln 2}+x^2+C$}
	{\True $\dfrac{2^x}{\ln 2}+\dfrac{x^2}{2}+C$}
	\loigiai{		
		Ta có $\displaystyle\int (2^x+x)\,\mathrm{d}\,x=\dfrac{2^x}{\ln 2}+\dfrac{1}{2} x^2+C$.}
\end{ex} 
\begin{ex}%[2D4N1-4]
	$\displaystyle\int (3^x+4^x)\mathrm{d}\,x$ bằng
	\choice
	{\True  $\dfrac{3^x}{\ln 3}+\dfrac{4^x}{\ln 4}+C$}
	{$\dfrac{3^x}{\ln 4}+\dfrac{4^x}{\ln 3}+C$}
	{$\dfrac{4^x}{\ln 3}-\dfrac{3^x}{\ln 4}+C$}
	{$\dfrac{3^x}{\ln 3}-\dfrac{4^x}{\ln 4}+C$}
	\loigiai{
		Áp dụng công thức $\displaystyle\int a^x\,\mathrm{d}x=\frac{a^x}{\ln a}+C$.\\
		Ta có $\displaystyle\int(3^x+4^x)\mathrm{d}\,x
		=\int 3^x\mathrm{d}\,x+\int 4^x\mathrm{d}\,x=\dfrac{3^x}{\ln 3}+\dfrac{4^x}{\ln 4}+C$.}
\end{ex} 
\begin{ex}%[2D4H1-4]
	Họ nguyên hàm của hàm số $f(x)=\mathrm{e}^x+2x$ là
	\choice
	{ $\dfrac{1}{x+1}\mathrm{e}^x+x^2+C$}
	{$\mathrm{e}^x+2x^2+C$}
	{\True $\mathrm{e}^x+x^2+C$}
	{$\mathrm{e}^x+\dfrac{1}{2} x^2+C$}
	\loigiai{
		Ta có $\displaystyle\int(\mathrm{e}^x+2x)\mathrm{d}\,x=\int\mathrm{e}^x\mathrm{d}\,x+\int 2x\mathrm{d}\,x=\mathrm{e}^x+x^2+C$.}
\end{ex} 
\begin{ex}%[2D4H1-4]
	Trong các mệnh đề sau, mệnh đề nào \textbf{sai}?
	\choice
	{\True  $\displaystyle\int\sin x\mathrm{d}x=\cos x+C$}
	{$\displaystyle\int 2x\mathrm{d}x=x^2+C$}
	{$\displaystyle\int \mathrm{e}^x\mathrm{d}x=\mathrm{e}^x+C$}
	{$\displaystyle\int \dfrac{1}{x}\mathrm{d}x=\ln \left| x \right|+C$}
	\loigiai{
		$\displaystyle\int{\sin x\mathrm{d}x}=-\cos x+C$.}
\end{ex} 
\begin{ex}%[2D4N1-1]
	Khẳng định nào sau đây là \textbf{sai}?
	\choice
	{ Mọi hàm số $f(x)$ liên tục trên đoạn $[a;b]$ đều có nguyên hàm trên đoạn $[a;b]$}
	{\True $\displaystyle\int x^\alpha \mathrm{d}x=\dfrac{x^{\alpha +1}}{\alpha +1}+C$ ($C$ là hằng số, $\alpha $ là hằng số)}
	{$\displaystyle\int \mathrm{e}^x\mathrm{d}x=\mathrm{e}^x+C$ ($C$ là hằng số)}
	{$\displaystyle\int{\dfrac{1}{x}\mathrm{d}x=\ln \left| x \right|+C}$ ($C$ là hằng số) với $x\ne 0$}
	\loigiai{		
		$\displaystyle\int x^{\alpha} \mathrm{d}\,x=\dfrac{x^{\alpha +1}}{\alpha +1}+C$ ($C$ là hằng số, $\alpha $ là hằng số và $\alpha \ne -1$).}
\end{ex} 
\begin{ex}%[2D4N1-4]
	Hàm số nào dưới đây là một nguyên hàm của hàm số $f(x)=\sqrt{x}-1$ trên $(0;+\infty)$?
	\choice
	{$F(x)=\dfrac{1}{2\sqrt{x}}$}
	{$F(x)=\dfrac{1}{2\sqrt{x}}-x$}
	{$F(x)=\dfrac{2}{3}\sqrt[3]{x^2}-x+1$}
	{\True $F(x)=\dfrac{2}{3}\sqrt{x^3}-x+2$}
	\loigiai{		
		Ta có : $\displaystyle\int (\sqrt{x}-1)\mathrm{d}x=\frac{2}{3}\sqrt{x^3}-x+C$.}
\end{ex} 
\begin{ex}%[2D4V2-6]
	Một vật chuyển động với gia tốc $a(t)=\dfrac{3}{t+1}$ (m/s$^2$), trong đó $t$ là khoảng thời gian tính từ thời điểm ban đầu. Vận tốc ban đầu của vật là $6$(m/s). Hỏi vận tốc của vật tại giây thứ $8$ là bao nhiêu?
	\choice
	{\True  $12{,}6$ (m/s)}
	{$12{,}2$ (m/s)}
	{$6{,}6$ (m/s)}
	{$12{,}4$ (m/s)}
	\loigiai{
		Vận tốc của vật tại thời điểm t được tính theo công thức
		$$v(t)=\int a(t)\mathrm{d}t=\displaystyle \int \dfrac{3}{t+1}\mathrm{d} t=3\ln (t+1)+C.$$
		Do vận tốc ban đầu của vật bằng $6$ (m/s) nên ta có:
		$$v(0)=3\ln (0+1)+C=6\Rightarrow C=6\Rightarrow v(t)=3\ln (t+1)+6.$$
		Vận tốc chuyển động của vật tại giây thứ $8$ là:
		$$v(8)=3\ln (8+1)+6=3\ln 9+6\approx 12{,}6 \text{ (m/s)}.$$
}
\end{ex} 

\Closesolutionfile{ans}
% \indapan{6}{ans/ans-2C4B11-De2}
\Opensolutionfile{ans}[ans/ans-2C4B11-De2-ds]
\TNTF
\setcounter{ex}{0}
\begin{ex}%[2D4H1-3]
	Cho hàm số $f(x)=\sin \dfrac{x}{2}$ và hàm số $g(x)=\cos \dfrac{x}{2}$ .
	\choiceTF
	{$F(x)=2\cos \dfrac{x}{2}$  là một nguyên hàm của hàm số $f(x)$}
	{\True $G(x)=2\sin \dfrac{x}{2}+\dfrac{1}{2}$  là một nguyên hàm của hàm số $g(x)$}
	{$\displaystyle\int \left[ f(x)-g(x) \right]^2 \mathrm{d}x=x+\cos x+C$ ($C$ là một hằng số)}
	{\True $\displaystyle\int \dfrac{1}{[2f(x)\cdot g(x)]^2}\mathrm{d}x=-\cot x+C$ ($C$ là một hằng số)}
\loigiai{
	\begin{itemchoice}
	\itemch Vì $F'(x)=-\sin \dfrac{x}{2},\forall x\in R$ nên  $F(x)=2\cos \dfrac{x}{2}$  không là một nguyên hàm của hàm số $F(x)$ trên $\mathbb{R}$.  Sai
	\itemch Vì $G'(x)=\cos \dfrac{x}{2}, \forall x\in \mathbb{R}$ nên  $G(x)=2\sin \dfrac{x}{2}+\dfrac{1}{2}$ là một nguyên hàm của hàm số $g(x)$ trên $R$.  Đúng
	\itemch $\displaystyle\int [f(x)-g(x)]^2\mathrm{d}x=\int \left( \sin\dfrac{x}{2}-\cos\frac{x}{2} \right)^2\mathrm{d}x=\int \left(\sin ^2\frac{x}{2}+2\sin\frac{x}{2}\cos\frac{x}{2}+\cos^2\frac{x}{2}\right)\mathrm{d}x=\int( 1+\sin x )\mathrm{d}x=x-\cos x+C$.  Sai
	\itemch $\displaystyle\int \frac{1}{[2f(x)\cdot g(x)]^2}\mathrm{d}x=\int \frac{1}{(2\sin\frac{x}{2}\cos\frac{x}{2})^2}\mathrm{d}x=\int \frac{1}{\sin^2 x}\mathrm{d}x=-\cot x+C$.  Đúng
	\end{itemchoice}
	}
	\end{ex} 
	\begin{ex}%[2D4H1-4]
		Cho hàm số $f(x)=\dfrac{1}{x}$ và $F(x)=\ln x+C_1$, $G(x)=\ln (-x)+C_2$ ($C_1,C_2$ là các hằng số).
		\choiceTF
{\True Trên $(0;+\infty)$, một nguyên hàm của hàm số $f(x)$ là $H(x)=\ln (x)+e$}
{\True Trên $(-\infty ;0)$, nguyên hàm của hàm số $f(x)$ là $G(x)$}
{\True Trên $(0;+\infty)$, nguyên hàm của hàm số $f(x)$ là $F(x)$}
{\True $\displaystyle\int \left[ f(x)+f^2(x) \right]\mathrm{d}x=\ln (3\left| x \right|)-\dfrac{1}{x}+C$ ($C$ là một hằng số)}
\loigiai{
		\begin{itemchoice}
\itemch Vì $H'(x)=\dfrac{1}{x}=F(x),\forall x\in (0;+\infty)$ nên $H(x)$ là một nguyên hàm của hàm số $F(x)$ trên ($0,+\infty $).  Đúng
\itemch $\displaystyle\int f(x)\mathrm{d}x=\int \frac{1}{x}\mathrm{d}x=\ln \left( \left| x \right| \right)+C_2=\ln (-x)+C_2,\forall x\in (-\infty ;0)$.  Đúng
\itemch $\displaystyle\int f(x)\mathrm{d}x=\int \frac{1}{x}\mathrm{d}x=\ln \left( \left| x \right| \right)+C_1=\ln x+C_1,\forall x\in( 0;+\infty)$.  Đúng
\itemch $\displaystyle\int \left[ f(x)+f^2(x) \right]\mathrm{d}x=\int \left( \frac{1}{x}+\frac{1}{x^2} \right)\mathrm{d}x=\ln (\left| x \right|)-\frac{1}{x}+C_3=\ln \left( \left| x \right| \right)-\frac{1}{x}+\ln 3+C=\ln ( 3\left| x \right|)-\frac{1}{x}+C$.  Đúng 
	\end{itemchoice}}
\end{ex} 
\begin{ex}%[2D4V1-4]
	Cho hàm số $f(x)=\cos x$ và hàm số $g(x)=\sin x$.
	\choiceTF
{\True $F(x)=\sin x+\mathrm{e}$ là một nguyên hàm của hàm số $f(x)$ trên $\mathbb{R}$}
{$G(x)={\mathrm{e}^{-\cos x}}+\ln 3$ là một nguyên hàm của hàm số $\mathrm{e}^{g(x)}$ trên $\mathbb{R}$}
{\True $\displaystyle\int \left[ 5f(x)+6g(x) \right]\mathrm{d}x=5\sin x-6\cos x+C$, ($C$ là một hằng số)}
{\True $\displaystyle\int \left[ 2+\left( \frac{g(x)}{f(x)} \right)^2 \right]\mathrm{d}x=x+\tan x+C$, ($C$ là một hằng số)}
\loigiai{
		\begin{itemchoice}
\itemch Vì $F'(x)=\cos x=f(x),\forall x\in \mathbb{R}$ nên $(x)$ là một nguyên hàm của hàm số $f(x)$ trên $\mathbb{R}$.  Đúng
\itemch Vì $G'(x)=\sin x \mathrm{e}^{-\cos x}\ne \mathrm{e}^{\sin x}, \forall x\in \mathbb{R}$ nên $G(x)$ không là một nguyên hàm của hàm số $\mathrm{e}^{g(x)}$ trên $\mathbb{R}$. Sai
\itemch $\displaystyle\int \left[ 5f(x)+6g(x) \right]\mathrm{d}x=\int \left( 5\cos x+6\sin x \right)\mathrm{d}x=5\sin x-6\cos x+C$.  Đúng
\itemch $\displaystyle\int \left[ 2+\left( \frac{g(x)}{f(x)} \right)^2 \right]\mathrm{d}x=\int \left(2+\frac{\sin^2 x}{\cos^2 x}\right)\mathrm{d}x=\int \left( 1+\frac{\sin^2 x+\cos^2 x}{\cos^2 x} \right)\mathrm{d}x\\
=\int \left( 1+\frac{1}{\cos^2 x} \right)\mathrm{d}x=x+\tan x+C$.  Đúng
	\end{itemchoice} }
\end{ex} 
\begin{ex}%[2D4V1-4]
	Cho hàm số $f(x)=3^{2x}$ và hàm số $g(x)=\tan x$.
	\choiceTF
{$F(x)=\dfrac{3^{2x}\ln 3}{2}$  là một nguyên hàm của hàm số $f(x)$ trên $\mathbb{R}$}
{\True $G(x)=-\ln (3\cos x)$ là một nguyên hàm của hàm số $g(x)$ trên $\left( -\dfrac{\pi}{2};\dfrac{\pi}{2} \right)$}
{\True $\displaystyle\int 3f(x)\mathrm{d}x=\dfrac{3^{2x+1}}{\ln 9}+C$, ($C$ là một hằng số)}
{\True $\displaystyle\int [f(x)+g(x)^2]\mathrm{d}x =\dfrac{9^x}{2\ln 3}-x+\tan x+C$, ($C$ là một hằng số)}
\loigiai{
	\begin{itemchoice}
\itemch Vì $F'(x)=\dfrac{2\cdot 3^{2x}\cdot\ln ^23}{2}=3^{2x}\cdot \ln ^2 3\ne f(x),\forall x\in \mathbb{R}$ nên $f(x)$ không là một nguyên hàm của hàm số $F(x)$ trên $\mathbb{R}$.  Sai
\itemch Vì $G'(x)=-\dfrac{-3\sin x}{3\cos x}=\tan x=g(x),\forall x\in \left( -\dfrac{\pi }{2};\dfrac{\pi }{2} \right)$ nên $G(x)$ là một nguyên hàm của hàm số $g(x)$ trên $\left( -\dfrac{\pi }{2};\dfrac{\pi }{2} \right)$.  Đúng
\itemch $\displaystyle\int 3f(x)\mathrm{d}x=\int 3\cdot 3^{2x}\mathrm{d}x=3\cdot 9^x\mathrm{d}x=3\cdot\dfrac{9^x}{\ln 9}+C=\dfrac{3\cdot3^{2x}}{\ln 9}+C=\dfrac{3^{2x+1}}{\ln 9}+C$.  Đúng
\itemch $\displaystyle\int \left[f(x)+g(x)^2\right]\mathrm{d}x=\int \left( 3^{2x}+\tan^2x\right)\mathrm{d}x=\int \left(9^x-1+1+\tan ^2\right)\mathrm{d}x\\
=\int \left(9^x-1+\dfrac{1}{\cos^2 x}\right)\mathrm{d}x
=\dfrac{9^x}{\ln 9}-x+\tan x+C=\dfrac{9^x}{2\ln 3}-x+\tan x+C$.  Đúng
	\end{itemchoice}
}
\end{ex} 

\Closesolutionfile{ans}
% \indapan{2}{ans/ans-2C4B11-De2-ds}
\Opensolutionfile{ans}[ans/ans-2C4B11-De2-kq]
\TNSA
\setcounter{ex}{0}
\begin{ex}%[2D4V2-4]
	Giả sử hàm số $y=f(x)$ liên tục và thỏa mãn: $f(1)=1$ và $f'(x)\sqrt[3]{x^{-1}}=1$, với mọi $x>0$. Tính $4f(8)$.
	\shortans{$47$}
	\loigiai{
		Ta có $f'(x)=\dfrac{1}{\sqrt[3]{x^{-1}}}=\dfrac{1}{x^{-\tfrac{1}{3}}}=x^{\tfrac{1}{3}}$\\
		$\Rightarrow F(x)=\displaystyle\int f'(x)\mathrm{d}x= \int x^{\tfrac{1}{3}}\mathrm{d}x=\dfrac{3}{4}x^{\frac{4}{3}}+C=\frac{3}{4}\sqrt[3]{x^4}+C$.\\
		$f(1)=1\Rightarrow \dfrac{3}{4}+C=1\Rightarrow C=-\dfrac{1}{4}.\\
		\Rightarrow f(x)=\dfrac{3}{4}\sqrt[3]{x^4}-\dfrac{1}{4}$\\
		$\Rightarrow 4f(8)=47$.}
\end{ex} 
\begin{ex}%[2D4V2-6]
Một ô tô đang chạy với vận tốc $10$(m/s) thì người lái xe đạp phanh. Từ thời điểm đó, ô tô chuyển động chậm dần đều với vận tốc $v(t)=10-2t$ (m/s), trong đó $t$ là khoảng thời gian tính bằng giây kể từ lúc đạp phanh. Tính quãng đường ô tô di chuyển được trong $8$ giây cuối cùng.
\shortans{$55$}
\loigiai{
	Chọn mốc thời gian và gốc tọa độ lúc ô tô bắt đầu đạp phanh. Suy ra $t=0;\,s=0$.\\
	$s(t)=\displaystyle \int v(t)\mathrm{d}t=\int (10-2t)\mathrm{d}t=10t-t^2+C$.\\
	$s(0)=0\Rightarrow C=0 \Rightarrow s(t)=10t-t^2$.\\ 
	Ô tô dừng hẳn khi $v(t)=0\Leftrightarrow 10-2t=0\Leftrightarrow t=5$.\\
	Trong $8$ giây cuối:
	\begin{itemize}
		\item ô tô chuyển động đều với vận tốc $10$(m/s) trong $3$ giây đầu.
		\item ô tô chuyển động chậm dần đều trong $5$ giây cuối.
	\end{itemize}
	Quãng đường ô tô di chuyển là: $s=3\cdot 10+10\cdot 5-5^2=55$ m.}

\end{ex} 
\begin{ex}%[2D4V2-4]
	Gọi $F(x)$ là một nguyên hàm của hàm số $f(x)=3^{2x+1} 2^{1+3x}$, biết $F(0)=\dfrac{8}{\ln 72}$. Tính $F(-2)$. (làm tròn kết quả đến hàng phần trăm).
	\shortans{$0{,}47$}
	\loigiai{
		Ta có:\\
		$F(x)=\displaystyle \int{\left(3^{2x+1}\cdot 2^{1+3x} \right)}\mathrm{d}x=\int\left(3\cdot3^{2x}\cdot2\cdot2^{3x}\right)\mathrm{d}x=\int\left(6\cdot9^x\cdot8^x \right)\mathrm{d}x\\
		=6\int 72^x\mathrm{d}x=6\cdot\dfrac{72^x}{\ln 72}+C$.\\
		Theo giả thiết, $F(0)=\dfrac{8}{\ln 72}\Rightarrow 6\cdot\dfrac{72^0}{\ln 72}+C=\dfrac{8}{\ln 72}\Rightarrow C=\dfrac{2}{\ln 72}$\\
		$\Rightarrow F(x)=6\cdot\dfrac{{{72}^{x}}}{\ln 72}+\dfrac{2}{\ln 72}\Rightarrow F\left( -2 \right)=6\cdot \dfrac{72^{-2}}{\ln 72}+\dfrac{2}{\ln 72}\approx 0{,}47$.
		}
\end{ex} 
\begin{ex}%[2D4V2-6]
Một viên đạn được bắn thẳng đứng lên từ độ cao $1{,}5$ mét so với mặt đất. Giả sử tại thời điểm $t$ giây (coi $t=0$ là thời điểm viên đạn được bắn lên), vận tốc của nó được cho bởi $v(t)=170-9{,}8\,t\,\left( \text{m/s} \right)$. Tìm độ cao lớn nhất của viên đạn (làm tròn kết quả đến hàng đơn vị).
	\shortans{$1476$}
	\loigiai{
		Gọi $h(t)$ là độ cao của viên đạn tại thời điểm $t$ giây sau khi bắn. Ta có:\\
		$h(t)=\displaystyle \int v(t)\mathrm{d}t=\int{(170-9{,}8t)}\mathrm{d}t=170t-4{,}9t^2+C$.\\
		Từ giả thiết suy ra: $h\left( 0 \right)=1,5\Rightarrow C=1{,}5\Rightarrow h(t)=170t-4{,}9t^2+1,5$.\\
		Viên đạn đạt độ cao lớn nhất khi $v(t)=0\Leftrightarrow 170-9,8\,t\,=0\Leftrightarrow t=\dfrac{850}{49}$.\\
		Khi đó, độ cao lớn nhất của viên đạn là:\\
		$h\left(\dfrac{850}{49}\right)=170 \cdot\dfrac{850}{49}-4{,}9\left( \dfrac{850}{49} \right)^2+1{,}5=\dfrac{144647}{98}\approx 1476$ (m).}
\end{ex} 
\begin{ex}%[2D4V2-6]
Một chiếc cốc chứa nước ở $95^\circ$ C được đặt trong phòng có nhiệt độ ${{20}^{0}}C$. Theo định luật làm mát của Newton, nhiệt độ của nước trong cốc sau $t$ phút (xem $t=0$ là thời điểm nước ở $95^\circ$ C là một hàm số $(t)$. Tốc độ giảm nhiệt độ của nước trong cốc tại thời điểm t phút được xác định bởi $T'(t)=\left(-\dfrac{3}{2} \mathrm\mathrm{e}^{-\tfrac{t}{50}}\right)^\circ$ C/phút). Tính nhiệt độ của nước tại thời điểm $t=40$ phút (làm tròn kết quả đến hàng phần chục).
	\shortans{$53{,}7$}
	\loigiai{
		Ta có:\\
$\displaystyle T(t)=\int T'(t)\mathrm{d}t
=\int\left( -\frac{3}{2}\mathrm{e}^{-\tfrac{t}{50}} \right)\mathrm{d}t
=-\frac{3}{2}\int\left({\mathrm{e}^{-\tfrac{1}{50}}} \right)^t\mathrm{d}t\\
=-\frac{3}{2}\cdot\frac{\left(\mathrm{e}^{-\tfrac{1}{50}} \right)^t}{\ln \left(\mathrm{e}^{-\tfrac{1}{50}}\right)}+C
=75\left(\mathrm{e}^{-\frac{1}{50}}\right)^t+C$.\\
		Vì $t=0$ là thời điểm nước ở $95^\circ$ C nên $T(0)=95\Rightarrow 75\left(\mathrm{e}^{-\tfrac{1}{50}} \right)^\circ+C=95\Rightarrow C=20$.\\ 
		Suy ra $T(t)=75\left(\mathrm{e}^{-\frac{1}{50}} \right)^t+20$.\\
		Do đó, nhiệt độ của nước tại thời điểm $t=40$ phút là: \\
		$T(40)=75\left(\mathrm{e}^{-\tfrac{1}{50}} \right)^{40}+20\approx 53{,}7 ^\circ$ C.}
\end{ex} 
\begin{ex}%[2D4V2-6]
	Doanh thu bán hàng của một công ty khi bán một loại sản phẩm là số tiền $R(x)$ (triệu đồng) thu được khi $x$ đơn vị sản phẩm được bán ra. Tốc độ biến động (thay đổi) của doanh thu khi $x$ đơn vị sản phẩm đã được bán là hàm số $M_R(x)=R'(x)$. Một công ty công nghệ cho biết, tốc độ biến đổi của doanh thu khi bán một loại con chip của hãng được cho bởi $M_R(x)=40-0{,}1x$, ở đó $x$ là số lượng chip đã bán. Hỏi doanh thu của công ty khi đã bán 500 con chip bằng bao nhiêu tỉ đồng?
	\shortans{$7{,}5$}
	\loigiai{
		Vì $R'(x)=M_R(x)$ nên doanh thu $R(x)$ là một nguyên hàm của $M_R(x)$.\\
		Ta có: $R(x)=\displaystyle \int M_R(x) \mathrm{d}x=\int{(40-0{,}1x)}\mathrm{d}x=40x-0{,}05 x^2+C$.\\
		Khi $x=0$, tức là chưa bán chip nào thì doanh thu sẽ bằng $0$ (triệu đồng), nên $R\left( 0 \right)=0\Rightarrow C=0$.\\
		Suy ra $R(x)=40x-0{,}05 x^2$.\\
		Do đó, doanh thu của công ty khi đã bán 500 con chip là:\\
		$R(500)=40\cdot 500-0{,}05\cdot 500^2=7500$ (triệu đồng) $=7{,}5$ (tỉ đồng).		
	}
\end{ex}
\Closesolutionfile{ans}
% \indapan{6}{ans/ans-2C4B11-De2-kq}
% \begin{name}
	{NGUYÊN HÀM - TÍCH PHÂN}
	{KT TÍCH PHÂN}
	{\tentruong}
	{\thoigian}
\end{name}
\setcounter{ex}{0}\setcounter{bt}{0}
\Opensolutionfile{ans}[ans/ans-2-B12-De1-NLC]
\TN
\begin{ex}%[2D4N2-1]
	Biết $\displaystyle\displaystyle\int\limits f(x) \mathrm{\,d} x=F(x)+C$. Trong các khẳng định sau, khẳng định nào đúng?
	\choice 
		{$\displaystyle\displaystyle\int\limits\limits_a^b f(x) \mathrm{\,d} x=F(b) \cdot F(a)$}
		{$\displaystyle\displaystyle\int\limits\limits_a^b f(x) \mathrm{\,d}x=F(a)-F(b)$}
		{\True $\displaystyle\displaystyle\int\limits\limits_a^b f(x) \mathrm{\,d}x=F(b)-F(a)$}
		{$\displaystyle\displaystyle\int\limits\limits_a^b f(x) \mathrm{\,d} x=F(b)+F(a)$}
	\loigiai{
		Ta có $\displaystyle\displaystyle\int\limits\limits_a^b f(x) \mathrm{\,d}x=F(b)-F(a)$.
	}
\end{ex}
\begin{ex}%[2D4N2-2]
	Tính tích phân $\displaystyle\displaystyle\int\limits\limits_1^2(2a x+b) \mathrm{\,d} x$.
	\choice 
		{\True $3a+b$}
		{$3a+2b$}
		{$a+2 b$}
		{$a+b$}
	\loigiai{
		Ta có $\displaystyle\displaystyle\int\limits\limits_1^2(2 a x+b) \mathrm{\,d} x=\left(a x^2+b x\right)\Big|_1 ^2=4 a+2 b-(a+b)=3 a+b$.
	}
\end{ex}
\begin{ex}%[2D4H2-1]
	Biết $\displaystyle\int\limits_1^8 f(x) \mathrm{\,d} x=-2$, $\displaystyle\int\limits_1^4 f(x) \mathrm{\,d} x=3$ và $\displaystyle\int\limits_1^4 g(x) \mathrm{\,d} x=7$. Mệnh đề nào sau đây \textbf{sai}?
	\choice 
		{$\displaystyle\int\limits_1^4\left[4 f(x)-2 g(x)\right] \mathrm{d} x=-2$}
		{$\displaystyle\int\limits_4^8 f(x) \mathrm{\,d} x=1$}
		{$\displaystyle\int\limits_1^4\left[f(x)+g(x)\right] \mathrm{d} x=10$}
		{\True $\displaystyle\int\limits_4^8 f(x) \mathrm{\,d} x=-5$}
	\loigiai{
		Ta có
		 $\displaystyle\int\limits_4^8 f(x) \mathrm{\,d} x=\displaystyle\int\limits_1^8 f(x) \mathrm{\,d} x-\displaystyle\int\limits_1^4 f(x) \mathrm{\,d} x=-2-3=-5$.
	}
\end{ex}
\begin{ex}%[2D4N2-4]
	Tích phân $I=\displaystyle\int\limits_0^{2018} 2^x \mathrm{\,d} x$ bằng
	 \choice 
		{$\dfrac{2^{2018}}{\ln 2}$}
		{$2^{2018}$}
		{$2^{2018}-1$}
		{\True $\dfrac{2^{2018}-1}{\ln 2}$}
	\loigiai{
		Ta  có
		$I=\displaystyle\int\limits_0^{2018} 2^x \mathrm{\,d} x=\dfrac{2^x}{\ln 2}\,\bigg|_0 ^{2018}=\dfrac{2^{2018}-1}{\ln 2}$.
	}
\end{ex}
\begin{ex}%[2D4N2-3]
	Tích phân $I=\displaystyle\int\limits_{\tfrac{\pi}{4}}^{\tfrac{\pi}{3}} \dfrac{\mathrm{\,d} x}{\sin ^2 x}$ bằng
	\choice 
		 {$\cot \dfrac{\pi}{3}-\cot \dfrac{\pi}{4}$} 
		 {$\cot \dfrac{\pi}{3}+\cot \dfrac{\pi}{4}$}
		 {\True $-\cot \dfrac{\pi}{3}+\cot \dfrac{\pi}{4}$}
	 	 {$-\cot \dfrac{\pi}{3}-\cot \dfrac{\pi}{4}$}
	\loigiai{
		Ta có $I=\displaystyle\int\limits_{\tfrac{\pi}{4}}^{\tfrac{\pi}{3}} \dfrac{\mathrm{\,d} x}{\sin ^2 x}=-\cot x\,\bigg|_{\tfrac{\pi}{4}} ^{\tfrac{\pi}{3}}=-\cot \dfrac{\pi}{3}+\cot \dfrac{\pi}{4}$.
	}
\end{ex}
\begin{ex}%[2D4N2-2]
	Tính tích phân $I=\displaystyle\int\limits_1^2\left(\dfrac{2}{x}-\dfrac{1}{x^2}\right) \mathrm{d} x$.
	\choice 
		{$I=2 \ln 2$}
		{\True $I=2 \ln 2-\dfrac{1}{2}$}
		{$I=2 \mathrm{e}+\dfrac{1}{2}$}
		{$I=0$}
	\loigiai{
		Ta có $I=\displaystyle\int\limits_1^2\left(\dfrac{2}{x}-\dfrac{1}{x^2}\right) \mathrm{d} x=\left(2 \ln |x|+\dfrac{1}{x}\right)\bigg|_1 ^2=\left(2 \ln 2+\dfrac{1}{2}\right)-(2 \ln 1+1)=2 \ln 2-\dfrac{1}{2}$.
	}
\end{ex}
\begin{ex}%[2D4H2-3]
	Tính tích phân $I=\displaystyle\int\limits_0^{\tfrac{\pi}{4}} \tan ^2 x \mathrm{\,d} x$. 
	\choice 
		{$I=2$}
		{$I=\ln 2$}
		{$I=\dfrac{\pi}{12}$}
		{\True $I=1-\dfrac{\pi}{4}$}
	\loigiai{
		Ta có
		\begin{eqnarray*}
			I=\displaystyle\int\limits_0^{\tfrac{\pi}{4}} \tan ^2 x \mathrm{\,d} x=\displaystyle\int\limits_0^{\tfrac{\pi}{4}}\left( \dfrac{1}{\cos^2x}-1\right)  \mathrm{d} x&=&\left( \tan x-x\right) \bigg|_0^{\tfrac{\pi}{4}}\\
			&=&\left( \tan \dfrac{\pi}{4}-\dfrac{\pi}{4}\right)-\left( \tan 0-0\right)=1-\dfrac{\pi}{4}. 
		\end{eqnarray*} 
		
	}
\end{ex}
\begin{ex}%[2D4H2-2]
	Cho $a$, $b$ là các số thực dương thỏa mãn $\sqrt{a}-\sqrt{b}+1=0$. Tính tích phân $I=\displaystyle\int\limits_a^b \dfrac{\mathrm{\,d} x}{\sqrt{x}}$.
	\choice 
		{$I=-2$}
		{$I=1$}
		{$I=\dfrac{1}{2}$}
		{\True $I=2$}
	\loigiai{
		Ta có\\ $I=\displaystyle\int\limits_a^b \dfrac{\mathrm{\,d} x}{\sqrt{x}}=\displaystyle\int\limits_a^b x^{-\tfrac{1}{2}} \mathrm{\,d} x=2 \sqrt{x}\,\bigg|_a ^b=2\left( \sqrt{b}-\sqrt{a}\right) =2\left( 1-\left(\sqrt{a}-\sqrt{b}+1\right) \right) =2\cdot 1=2 $.
	}
\end{ex}
\begin{ex}%[2D4N2-4]
	Cho $\displaystyle\int\limits_2^5 \dfrac{\mathrm{\,d} x}{x}=\ln a$. Tìm $a$. 
	\choice 
		{$2$}
		{$\dfrac{2}{5}$}
		{\True $\dfrac{5}{2}$}
		{$5$}
	\loigiai{ 
		Ta có $\displaystyle\int\limits_2^5 \dfrac{\mathrm{\,d} x}{x}=\ln a \Leftrightarrow \ln |x|\, \bigg|_2^5=\ln a \Leftrightarrow \ln 5-\ln 2=\ln a \Leftrightarrow \ln \dfrac{5}{2}=\ln a \Leftrightarrow a=\dfrac{5}{2}$.
	}
\end{ex}
\begin{ex}%[2D4H2-2]
	Cho hàm số $f(x)$ liên tục trên $\mathbb{R}$ và $\displaystyle\int\limits_0^2\left( f(x)+2 x\right)  \mathrm{d} x=5$. Tính $\displaystyle\int\limits_0^2 f(x) \mathrm{\,d} x$.
	\choice
		{$-9$}
		{$-1$}
		{$9$}
		{\True $1$}
	\loigiai{
		Ta có $\displaystyle\int\limits_0^2\left( f(x)+2 x\right)  \mathrm{d} x=\displaystyle\int\limits_0^2 f(x) \mathrm{\,d} x+\displaystyle\int\limits_0^2 2 x \mathrm{\,d} x=\displaystyle\int\limits_0^2 f(x) \mathrm{\,d} x+4=5$. Do đó $\displaystyle\int\limits_0^2 f(x) \mathrm{\,d} x=1$.
	}
\end{ex}
\begin{ex}%[2D4H2-1]
	Cho hai tích phân $\displaystyle\int\limits_{-2}^5 f(x) \mathrm{\,d} x=8$ và $\displaystyle\int\limits_5^{-2} g(x) \mathrm{\,d} x=3$. Tính $I=\displaystyle\int\limits_{-2}^5\left[ f(x)-4 g(x)-1\right] \mathrm{d} x$.
	\choice 
		{$I=-11$}
		{\True $I=13$}
		{ $I=27$}
		{ $I=3$}
	\loigiai{
		Ta có\\ $I=\displaystyle\int\limits_{-2}^5\left[ f(x)-4 g(x)-1\right]  \mathrm{d} x=\displaystyle\int\limits_{-2}^5 f(x) \mathrm{\,d} x+4 \displaystyle\int\limits_{5}^{-2} g(x) \mathrm{\,d} x-x\,\bigg|_{-2} ^5=8+4\cdot 3-(5+2)=13$.
	}
\end{ex}
\begin{ex}%[2D4H2-2]
	Cho hàm số $y=f(x)=\heva{&3 x^2 & \text { khi } 0 \leq x \leq 1 \\ &4-x & \text { khi } 1 \leq x \leq 2}$. Tính tích phân $\displaystyle\int\limits_0^2 f(x) \mathrm{\,d} x$.
	\choice 
		{\True $\dfrac{7}{2}$}
		{$1$}
		{$\dfrac{5}{2}$}
		{$\dfrac{3}{2}$}
	\loigiai{ 
		Ta có 
		\begin{eqnarray*}
		\displaystyle\int\limits_0^2 f(x) \mathrm{\,d} x&=&\displaystyle\int\limits_0^1 f(x) \mathrm{\,d} x+\displaystyle\int\limits_1^2 f(x) \mathrm{\,d} x=\displaystyle\int\limits_0^1\left(3 x^2\right) \mathrm{d} x+\displaystyle\int\limits_1^2(4-x) \mathrm{\,d} x\\
		&=& x^3\,\bigg|_0 ^1+\left(4 x-\dfrac{x^2}{2}\right)\bigg|_1 ^2=\left( 1^3-0^3\right) +\left[ \left(4\cdot 2-\dfrac{2^2}{2}\right)-\left( 4\cdot1-\dfrac{1^2}{2}\right) \right] = \dfrac{7}{2}.	
		\end{eqnarray*}
	}
\end{ex}
\Closesolutionfile{ans}
% \indapan{6}{ans/ans-2-B12-De1-NLC}
\TNTF
\Opensolutionfile{ans}[ans/ans-2-B12-De1-DS]
\begin{ex}%[2D4V2-2]
	Cho $f(x)$ và $g(x)$ là các hàm số liên tục bất kì trên đoạn $[a;b]$.
	\choiceTF
		{\True $\displaystyle\int\limits_a^b\left(f(x)-g(x)\right) \mathrm{d} x=\displaystyle\int\limits_a^b f(x) \mathrm{\,d} x-\displaystyle\int\limits_a^b g(x) \mathrm{\,d} x$}
		{$\displaystyle\int\limits_a^a\left[f(x)+g(x)\right] \mathrm{d} x=1$}
		{Nếu $\displaystyle\int\limits_a^b f(x) \mathrm{\,d} x=3$ và $\displaystyle\int\limits_a^b\left[3 f(x)-g(x)\right] \mathrm{d} x=10$ thì $\displaystyle\int\limits_a^b g(x) \mathrm{\,d} x=1$}
		{\True Nếu $f(x)+2 f\left(\dfrac{1}{x}\right)=3 x$ với $x \in\left[\dfrac{1}{2}; 2\right]$. Tính $\displaystyle\int\limits_{\tfrac{1}{2}}^2 \dfrac{f(x)}{x} \mathrm{\,d} x=\dfrac{3}{2}$}
	\loigiai{
		\begin{itemchoice}
			\itemch Đúng. Do tính chất tích phân.
			\itemch Sai. Ta có $\displaystyle\int\limits_a^a\left[f(x)+g(x)\right] \mathrm{d} x=0$.
			\itemch Sai. Ta có 
			\begin{eqnarray*}
				\displaystyle\int\limits_a^b\left[ 3 f(x)-g(x)\right]  \mathrm{d} x=10 &\Leftrightarrow& 3 \displaystyle\int\limits_a^b f(x) \mathrm{\,d} x-\displaystyle\int\limits_a^b g(x) \mathrm{\,d} x=10\\ &\Leftrightarrow& 3\cdot 3-\displaystyle\int\limits_a^b g(x) \mathrm{\,d} x=10 \Leftrightarrow\displaystyle\int\limits_a^b g(x) \mathrm{\,d} x=-1.
			\end{eqnarray*}
			\itemch Đúng. Ta có $f(x)+2 f\left(\dfrac{1}{x}\right)=3 x \Rightarrow f\left(\dfrac{1}{x}\right)+2 f(x)=\dfrac{3}{x}$.\\
			Suy ra $\heva{&f(x)+2 f\left(\dfrac{1}{x}\right)=3x \\& 4 f(x)+2 f\left(\dfrac{1}{x}\right)=\dfrac{6}{x}} \Rightarrow f(x)=\dfrac{2}{x}-x\Rightarrow \dfrac{f(x)}{x}=\dfrac{2}{x^2}-1$.\\
			Do đó $\displaystyle\int\limits_{\tfrac{1}{2}}^2 \dfrac{f(x)}{x} \mathrm{\,d} x=\displaystyle\int\limits_{\tfrac{1}{2}}^2\left(\dfrac{2}{x^2}-1\right) \mathrm{d} x=\dfrac{3}{2}$.
		\end{itemchoice}
	}
\end{ex}
\begin{ex}%[2D4V2-2]
	Cho các số thực $a$, $b$ $(a<b)$. Nếu hàm số $y=f(x)$ có đạo hàm là hàm liên tục trên $\mathbb{R}$ và $\displaystyle\displaystyle\int\limits f(x) \mathrm{\,d} x=F(x)+C$.
	\choiceTF
		{$\displaystyle\int\limits_a^b f(x) \mathrm{\,d} x=F(a)-F(b)$}
		{\True $\displaystyle\int\limits_a^b f'(x) \mathrm{\,d} x=f(b)-f(a)$}
		{Nếu $\displaystyle\int\limits_0^2 f(x) \mathrm{\,d} x=2$ thì $\displaystyle\int\limits_0^2\left[ 3 f(x)-2\right]  \mathrm{d} x=4$} 
		{\True Nếu $f(x)+f(2-x)=x^2-2 x+2,\, \forall x \in \mathbb{R}$ và $f(0)=3$ thì $\displaystyle\int\limits_0^2 f'(x) \mathrm{\,d} x=-4$}
	\loigiai{
		\begin{itemchoice}
			\itemch Sai. Ta có $\displaystyle\int\limits_a^b f(x) \mathrm{\,d} x=F(b)-F(a)$.
			\itemch Đúng. Ta có $\displaystyle\int\limits_a^b f'(x) \mathrm{\,d} x=f(x)\,\bigg|_a ^b=f(b)-f(a)$.
			\itemch Sai. Ta có $J=\displaystyle\int\limits_0^2\left[ 3 f(x)-2\right] \mathrm{d} x=3 \displaystyle\int\limits_0^2 f(x) \mathrm{\,d} x-2 \displaystyle\int\limits_0^2 \mathrm{\,d} x=3\cdot 2-2 x\,\bigg|_0 ^2=6-4=2$.
			\itemch Đúng. Ta có $f(x)+f(2-x)=x^2-2 x+2, \forall x \in \mathbb{R}\quad (1)$.\\
			Thay $x=0$ vào (1) ta được
			$f(0)+f(2)=2 \Rightarrow f(2)=2-f(0)=2-3=-1$.\\
			Từ đó có $ \displaystyle\int\limits_0^2 f'(x) \mathrm{\,d} x=f(2)-f(0)=-1-3=-4.$
		\end{itemchoice}
	}
\end{ex}
\begin{ex}%[2D4V2-2]
	Giả sử $f(x)$ và $g(x)$ là hai hàm số bất kỳ có đạo hàm liên tục trên $\mathbb{R}$ và $a$, $b$, $c$ là các số thực.
	\choiceTF
		{\True $\displaystyle\int\limits_a^b f(x) \mathrm{\,d} x=-\displaystyle\int\limits_b^a f(x) \mathrm{\,d} x$} 
		{Nếu $f(x)=\dfrac{1}{x}$ thì $\displaystyle\int\limits_{-3}^{-2} f(x) \mathrm{\,d} x=\ln x\,\bigg|_{-3} ^{-2}$}
		{\True $\displaystyle\int\limits_a^b f(x) \mathrm{\,d} x+\displaystyle\int\limits_b^c f(x) \mathrm{\,d} x+\displaystyle\int\limits_c^a f(x) \mathrm{\,d} x=0$}
		{Nếu $3 f(x)+x f'(x)=x^{2018}$ với mọi $x \in[0 ; 1]$ thì $\displaystyle\int\limits_0^1 f(x) \mathrm{\,d}x=\dfrac{1}{2020\cdot 2019}$}
	\loigiai{
		\begin{itemchoice}
			\itemch Đúng. Theo tính chất của tích phân.
			\itemch Sai. Ta có $\displaystyle\int\limits_{-3}^{-2} \dfrac{1}{x} \mathrm{\,d} x=\left( \ln |x|\right) \bigg|_{-3} ^{-2}$.
			\itemch Đúng. Ta có \\ $\displaystyle\int\limits_a^b f(x) \mathrm{\,d} x+\displaystyle\int\limits_b^c f(x) \mathrm{\,d} x+\displaystyle\int\limits_c^a f(x) \mathrm{\,d} x=\displaystyle\int\limits_a^c f(x) \mathrm{\,d} x+\displaystyle\int\limits_c^a f(x) \mathrm{\,d} x=\displaystyle\int\limits_a^a f(x) \mathrm{\,d} x=0$.
			\itemch Sai. Nhân hai vế của đẳng thức $3 f(x)+x f'(x)=x^{2018}$ với $x^2$ ta được $$3 x^2 f(x)+x^3 f'(x)=x^{2020} \Rightarrow\left[x^3 f(x)\right]'=x^{2020}\Rightarrow x^3 f(x)=\displaystyle\displaystyle\int\limits x^{2020} \mathrm{\,d} x=\dfrac{x^{2021}}{2021}+C\,(*).$$
			Thay $x=0$ vào hai vế $(*)$ ta được $C=0 \Rightarrow f(x)=\dfrac{x^{2018}}{2021}$.\\
			Vậy $\displaystyle\int\limits_0^1 f(x) \mathrm{\,d} x=\displaystyle\int\limits_0^1 \dfrac{1}{2021} x^{2018} \mathrm{\,d} x=\dfrac{1}{2021} \cdot \dfrac{1}{2019} x^{2019}\,\bigg|_0 ^1=\dfrac{1}{2021 \cdot 2019}$.
		\end{itemchoice}
	}
\end{ex}
\begin{ex}%[2D4V2-2]
	Cho $F(x)$ là nguyên hàm của hàm số $f(x)$.
	\choiceTF
		{\True $\displaystyle\int\limits_1^3 f(x) \mathrm{\,d} x=F(3)-F(1)$}
		{\True Nếu $f(x)=\dfrac{2}{x}+\dfrac{3}{x^2}\,(x \neq 0)$, $F(1)=1$ thì $F(3)=2 \ln 3+3$}
		{Nếu $F(-1)=1$ và $F(2)=4$ thì $\displaystyle\int\limits_{-1}^2\left[ f(x)+2 x\right]  \mathrm{d} x=9$}
		{\True Nếu hàm số $y=f(x)$ có đạo hàm liên tục trên $[0;1]$ thỏa $2 f(x)+3 f(1-x)=\sqrt{1-x^2}$ thì $\displaystyle\int\limits_0^1 f'(x) \mathrm{\,d} x=1$}
	\loigiai{
		\begin{itemchoice}
			\itemch  Đúng. Theo định nghĩa tích phân.
			\itemch Đúng. Ta có $\displaystyle\int\limits_1^3 f(x) \mathrm{\,d} x=F(3)-F(1)$. Suy ra $$F(3)=F(1)+\displaystyle\int\limits_1^3\left(\dfrac{2}{x}+\dfrac{3}{x^2}\right) \mathrm{\,d} x=1+\left(2 \ln x-\dfrac{3}{x}\right)\bigg|_1 ^3=2 \ln 3+3.$$
			\itemch Sai. Ta có $I=\displaystyle\int\limits_{-1}^2\left[ f(x)+2 x\right]  \mathrm{d} x=\left[F(x)+x^2\right]\bigg|_{-1} ^2=F(2)+4-F(-1)-1=6$.
			\itemch Đúng. Ta có 
			$\displaystyle\int\limits_0^1 f'(x) \mathrm{\,d} x=f(x)\,\bigg|_0 ^1=f(1)-f(0)$.\\
			Từ $2 f(x)+3 f(1-x)=\sqrt{1-x^2}\Rightarrow\heva{& 2f(0)+3 f(1)=1 \\ &2 f(1)+3 f(0)=0} \Leftrightarrow\heva{&f(0)=-\dfrac{2}{5} \\& f(1)=\dfrac{3}{5}.}$\\	
			Vậy $I=\displaystyle\int\limits_0^1 f'(x) \mathrm{\,d} x=f(1)-f(0)=\dfrac{3}{5}+\dfrac{2}{5}=1$.
		\end{itemchoice}
	}
\end{ex}
\Closesolutionfile{ans}
% \indapan{2}{ans/ans-2-B12-De1-DS}
\Opensolutionfile{ans}[ans/ans-2-B12-De1-KQ]
\TNSA
\begin{ex}%[2D4H2-6]
	Một xe ô tô đang di chuyển với tốc độ $22$ m/s thì gặp chướng ngại vật. Người lái xe phản ứng $3$ giây sau đó và đạp phanh khẩn cấp, kể từ thời điểm đạp phanh, ô tô chuyển động chậm dần đều với tốc độ $v(t)=36-6 t$ m/s, trong đó $t$ là thời gian tính bằng giây kể từ lúc đạp phanh. Hỏi quãng đường ô tô đi được từ lúc phát hiện chướng ngại vật đến khi ô tô dừng hẳn là bao nhiêu mét?
	\shortans{$174$} 
	\loigiai{
		Quãng đường ô tô đi được từ lúc phát hiện chướng ngại vật đến khi đạp phanh là $66$ m.\\
		Xe ô tô dừng hẳn khi $v(t)=0 \Leftrightarrow 36-6 t=0 \Leftrightarrow t=6$.\\
		Quãng đường ô tô đi được từ lúc đạp phanh đến lúc dừng lại là $\displaystyle\int\limits_0^6(36-6 t) \mathrm{\,d} t=108$ m.\\
		Vậy quãng đường ô tô đi được từ lúc phát hiện chướng ngại vật đến khi ô tô dừng hẳn là $66+108=174$ m.
	}
\end{ex}

\begin{ex}%[2D4V2-2]
	Cho hàm số $y=f(x)$ liên tục trên $\mathbb{R}$. Hàm số $y=f'(x)$ có đồ thị $(C)$ như hình vẽ, $(C)$ cắt trục $Ox$ tại ba điểm phân biệt có hoành độ $a<b<c$.
	\begin{center}
		\begin{tikzpicture}[line join = round, line cap = round,>=stealth,x = 1cm,y = .6cm] 
			%Vẽ hệ trục Oxy 
			\draw[->] (-2.5,0)--(0,0) node[below right]{$O$}--(4.5,0) node[below]{$x$}; 
			\draw (-1.1,.3) node {$a$} (1.1,.3) node {$b$}  (3.1,-.3) node{$c$} (2.9,2) node[rotate=70]{$(C)$};
			\draw[->] (0,-4.5)--(0,4.5) node[right]{$y$}; 
			\draw[samples=200,domain=-1.42:3.42,smooth] plot (\x, {(\x)^3-3*(\x)^2-\x+3}); 
		\end{tikzpicture}
	\end{center}
	Biết rằng diện tích hình phẳng giới hạn bởi $(C)\colon y=f'(x)$ và $O x$ bằng $15$, $f(a)=5$, $f(c)=6$. Tính $f(b)$.
	\shortans{$13$}
	\loigiai{
		Diện tích hình phẳng giới hạn bởi $(C)\colon y=f'(x)$ và $Ox$ bằng $15$, do đó
		$$
		15=\displaystyle\int\limits_a^c\left|f'(x)\right| \mathrm{\,d} x=\displaystyle\int\limits_a^b\left|f'(x)\right| \mathrm{\,d} x+\displaystyle\int\limits_b^c\left|f'(x)\right| \mathrm{\,d} x=2 f(b)-f(a)-f(c) .
		$$
		Suy ra $2 f(b)=15+f(a)+f(c) \Rightarrow f(b)=13$.
	}
\end{ex}

\begin{ex}%[2D4H2-2]
	Biết rằng $\displaystyle\int\limits_0^2 \dfrac{x^2}{x+1} \mathrm{\,d} x=a+\ln b$ với $a, b \in \mathbb{Z}$, $b>0$. Tính $2a+b$.
	\shortans{$3$}
	\loigiai{
		Ta có $\displaystyle\int\limits_0^2 \dfrac{x^2}{x+1} \mathrm{~d} x=\displaystyle\int\limits_0^2\left(x-1+\dfrac{1}{x+1}\right) \mathrm{d} x=\left(\dfrac{x^2}{2}-x+\ln |x+1|\right)\bigg|_0 ^2=\ln 3$.\\
		Suy ra $a=0$, $b=3$. Vậy $2 a+b=3$.
	}
\end{ex}

\begin{ex}%[2D4H2-2] 
	Cho $\displaystyle\int\limits_0^1 \dfrac{\mathrm{\,d} x}{\sqrt{x+2}+\sqrt{x+1}}=a \sqrt{b}-\dfrac{8}{3} \sqrt{a}+\dfrac{2}{3},\left(a, b \in \mathbb{N}^*\right)$. Tính $a+2b$.
	\shortans{$8$}
	\loigiai{
		Ta có
		 \begin{eqnarray*}
		 	\displaystyle\int\limits_0^1 \dfrac{\mathrm{\,d} x}{\sqrt{x+2}+\sqrt{x+1}}&=&\displaystyle\int\limits_0^1\left( \sqrt{x+2}-\sqrt{x+1}\right) \mathrm{d} x\\ &=&\dfrac{2}{3}\left(\sqrt{(x+2)^3}-\sqrt{(x+1)^3}\right)\bigg|_0 ^2 
			=2 \sqrt{3}-\dfrac{8}{3} \sqrt{2}+\dfrac{2}{3}.
	\end{eqnarray*}
	Vậy $a=2$, $b=3$, $a+2b=8$.
	}
\end{ex}

\begin{ex}%[2D4H2-6]
	Tại một nơi không có gió, một chiếc khí cầu đang đứng yên ở độ cao $162$ mét so với mặt đất đã được phi công cài đặt cho nó chế độ chuyển động đi xuống. Biết rằng, khí cầu đã chuyển động theo phương thẳng đứng với vận tốc tuân theo quy luật $v(t)=10 t-t^2$, trong đó $t$ phút là thời gian tính từ lúc bắt đầu chuyển động, $v(t)$ được tính theo đơn vị mét/phút. Tìm vận tốc $v$ của khí cầu khi bắt đầu tiếp đất.
	\shortans{$9$}
	\loigiai{
		Gọi thời điểm khí cầu bắt đầu chuyển động là $t=0$, thời điểm khinh khí cầu bắt đầu tiếp đất là $t_1$.
		Quãng đường khí cầu đi được từ thời điểm $t=0$ đến thời điểm khinh khí cầu bắt đầu tiếp đất  $t_1$ là
		\[
		\displaystyle\int\limits_0^{t_1}\left(10 t-t^2\right) \mathrm{d} t=5 t_1^2-\dfrac{t_1^3}{3}=162 \Leftrightarrow \hoac{&t_1 \approx-4{,}93\\& t_1 \approx 10{,}93 \\& t_1=9.}\]		
		Do $v(t) \geq 0$ nên $0 \leq t_1 \leq 10$, suy ra chọn $t_1=9$.\\
		Vậy khi bắt đầu tiếp đất vận tốc $v$ của khí cầu là $v(9)=10\cdot 9-9^2=9$ mét/phút.
	}
\end{ex}

\begin{ex}%[2D4V2-6] 
	Một ô tô chuyển động nhanh dần đều với vận tốc $v(t)=7 t$ m/s. Đi được $5$ s người lái xe phát hiện chướng ngại vật và phanh gấp, ô tô tiếp tục chuyển động chậm dần đều với gia tốc $a=-35 \mathrm{~m} / \mathrm{s}^2$. Tính quãng đường của ô tô đi được từ lúc bắt đầu chuyển bánh cho đến khi dừng hẳn? (quãng đường tính theo đơn vị m).
	\shortans{$105$}
	\loigiai{
		Quãng đường ô tô đi được trong $5$ s đầu là $s_1=\displaystyle\int\limits_0^5 7 t \mathrm{\,d} t=7 \dfrac{t^2}{2}\,\bigg|_0 ^5=87{,}5$.\\
		Phương trình vận tốc của ô tô khi người lái xe phát hiện chướng ngại vật là $v_2(t)=35-35 t$.\\
		Khi xe dừng lại hẳn thì $v_2(t)=0 \Leftrightarrow 35-35 t=0 \Leftrightarrow t=1$.\\
		Quãng đường ô tô đi được từ khi phanh gấp đến khi dừng lại hẳn là
		\[
		s_2=\displaystyle\int\limits_0^1\left(35-35 t\right) \mathrm{d} t=\left(35 t-35 \dfrac{t^2}{2}\right)\bigg|_0 ^1=17{,}5.\]
		Do đó quãng đường của ô tô đi được từ lúc bắt đầu chuyển bánh cho đến khi dừng hẳn là \[s=s_1+s_2=87{,}5+17{,}5=105.\]
	}
\end{ex}
\Closesolutionfile{ans}
% \indapan{6}{ans/ans-2-B12-De1-KQ}
% \begin{name}
	{NGUYÊN HÀM - TÍCH PHÂN}
	{KT TÍCH PHÂN}
	{\tentruong}
	{\thoigian}
\end{name}
\setcounter{ex}{0}\setcounter{bt}{0}
\Opensolutionfile{ans}[ans/ans-2-B12-De2-NLC]
\TN
\begin{ex}%[Cau-1]%[2D4N2-1]
	Cho hàm số $y=f(x)$ liên tục trên khoảng $K$ và $a$, $b$, $c\in K$. Mệnh đề nào sau đây \textbf{sai}?
	\choice
	{$\displaystyle \int\limits_{a}^{a} f(x)\mathrm{\,d}x=0$}
	{$\displaystyle \int\limits_{a}^{b} f(x)\mathrm{\,d}x=\displaystyle \int\limits_{a}^{b} f(t) \mathrm{\,d}t$}
	{$\displaystyle \int\limits_{a}^{b} f(x)\mathrm{\,d}x=-\displaystyle \int\limits_{b}^{a} f(x) \mathrm{\,d}x$}
	{\True $\displaystyle \int\limits_{a}^{b} f(x) \mathrm{\,d}x + \displaystyle \int\limits_{c}^{b} f(x) \mathrm{\,d}x=\displaystyle \int\limits_{a}^{c} f(x) \mathrm{\,d}x$}
	\loigiai{
		Mệnh đề sai là $\displaystyle \int\limits_{a}^{b} f(x)\mathrm{\,d}x + \displaystyle \int\limits_{c}^{b} f(x) \mathrm{\,d}x=\displaystyle \int\limits_{a}^{c} f(x) \mathrm{\,d}x$.
	}
\end{ex}
\begin{ex}%[Cau-2]%[2D4N2-1]
	Cho hàm số $f(x)$ liên tục trên $\mathbb{R}$ và $F(x)$ là nguyên hàm của $f(x)$, biết $\displaystyle \int\limits_{0}^{9} f(x)\mathrm{\,d}x=9$ và $F(0)=3$. Tính $F(9)$.
	\choice
	{$F(9)=-6$}
	{$F(9)=6$}
	{\True $F(9)=12$}
	{$F(9)=-12$}
	\loigiai{
		Ta có $I=\displaystyle\int\limits_{0}^{9} f(x)\mathrm{\,d}x = F(x)\Big|_0^9 = F(9)- F(0)=9\Leftrightarrow F(9)=9 + F(0)=9 + 3= 12$.
	}
\end{ex}
\begin{ex}%[Cau-3]%[2D4N2-2]
	Tính tích phân $I=\displaystyle\int\limits_{0}^{1} x^{2018}(1 + x)\mathrm{\,d}x$?
	\choice
	{$I=\dfrac{1}{2017}+\dfrac{1}{2018}$}
	{$I=\dfrac{1}{2018}+\dfrac{1}{2019}$}
	{$I=\dfrac{1}{2020}+\dfrac{1}{2021}$}
	{\True $I=\dfrac{1}{2019} + \dfrac{1}{2020}$}
	\loigiai{
		Ta có $I=\displaystyle \int\limits_{0}^{1} x^{2018}(1 + x) \mathrm{\,d}x= \displaystyle \int\limits_{0}^{1} \left(x^{2018} + x^{2019}\right) \mathrm{d}x= \left(\dfrac{x^{2019}}{2019} + \dfrac{x^{2020}}{2020}\right)\Bigg|_0^1 = \dfrac{1}{2019}+\dfrac{1}{2020}$.
	}
\end{ex}
\begin{ex}%[Cau-4]%[2D4N2-4]
	Tính $\displaystyle \int\limits_{0}^{1} 2\mathrm{e}^{x} \mathrm{\,d}x$?
	\choice
	{$I=\mathrm{e}^2-2\mathrm{e}$}
	{$I=2\mathrm{e}$}
	{$I=2\mathrm{e}+2$}
	{\True $I=2\mathrm{e}-2$}
	\loigiai{
		Ta có $\displaystyle \int\limits_{0}^{1} 2\mathrm{e}^{x} \mathrm{\,d}x = 2\mathrm{e}^x\Big|_0^1=2\mathrm{e}-2$.
	}
\end{ex}
\begin{ex}%[Cau-5]%[2D4N2-3]
	Cho $a\in \left(0;\dfrac{\pi}{2}\right)$. Tính $J=\displaystyle \int\limits_{0}^{a} \dfrac{29}{\cos^2 x} \mathrm{\,d}x$ theo $a$.
	\choice
	{$J=-29\tan a$}
	{$J=\dfrac{1}{29}\tan a$}
	{$J=29\cot a$}
	{\True $J=29\tan a$}
	\loigiai{
		Ta có $J=\displaystyle \int\limits_{0}^{a} \dfrac{29}{\cos^2 x}\mathrm{\,d}x=29\tan x\Big|_0^a = 29\tan a - 29\tan 0 = 29\tan a$.
	}
\end{ex}
\begin{ex}%[Cau-6]%[2D4H2-5]
	Tích phân $I=\displaystyle \int\limits_{-1}^{2}\left|x^2 - 2x\right|\mathrm{d}x$ có giá trị là
	\choice
	{\True $I=\dfrac{8}{3}$}
	{$I=\dfrac{4}{3}$}
	{$I=0$}
	{$I=-\dfrac{4}{3}$}
	\loigiai{
		Ta có $x^2 - 2x=0 \Leftrightarrow x=0$ hoặc $x=2$.\\
		Bảng xét dấu
		\begin{center}
			\begin{tikzpicture}
				\tkzTabInit[nocadre=false,lgt=2.5,espcl=2.5,deltacl=0.6]
				{$x$/0.7,$x^2-2x$/0.7}
				{$-\infty$,$0$,$2$,$+\infty$}
				\tkzTabLine{,+,0,-,0,+,}   
			\end{tikzpicture}
		\end{center}
		\begin{align*}
			I & = \displaystyle \int\limits_{-1}^{2} \left|x^2 - 2x\right| \mathrm{d}x = \displaystyle \int\limits_{-1}^{0} \left(x^2 - 2x\right) \mathrm{d}x - \displaystyle \int\limits_{0}^{2} \left(x^2 - 2x\right) \mathrm{d}x\\
			& = \left(\dfrac{x^3}{3} - x^2\right)\bigg|_{-1}^0 - \left(\dfrac{x^3}{3} - x^2\right)\bigg|_{0}^2 \\
			& = \left[0 - \left(-\dfrac{1}{3}-1\right)\right] - \left[\left(\dfrac{8}{3}-4\right)-0\right]\\
			& = \dfrac{8}{3}.
		\end{align*}
	}
\end{ex}
\begin{ex}%[Cau-7]%[2D4N2-3]
	Tính tích phân $I=\displaystyle \int\limits_{0}^{\tfrac{\pi}{4}} \sin x \mathrm{\,d}x$?
	\choice
	{$\dfrac{2+\sqrt{2}}{2}$}
	{\True $\dfrac{2-\sqrt{2}}{2}$}
	{$\dfrac{\sqrt{2}}{2}$}
	{$-\dfrac{\sqrt{2}}{2}$}
	\loigiai{
		Ta có $I=\displaystyle \int\limits_{0}^{\tfrac{\pi}{4}} \sin x \mathrm{\,d}x=-\cos x \bigg|_0^{\tfrac{\pi}{4}} = - \cos \left(\dfrac{\pi}{4}\right)+\cos 0= -\dfrac{\sqrt{2}}{2}+1=\dfrac{2-\sqrt{2}}{2}$.
	}
\end{ex}
\begin{ex}%[Cau-8]%[2D4N2-2]
	Tính tích phân $I=\displaystyle \int\limits_{1}^{2}\dfrac{x^2+4x}{x}\mathrm{\,d}x$?
	\choice
	{$I=\dfrac{29}{2}$}
	{$I=-\dfrac{11}{2}$}
	{\True $I=\dfrac{11}{2}$}
	{$I=-\dfrac{29}{2}$}
	\loigiai{
		Ta có $I=\displaystyle \int\limits_{1}^{2} \dfrac{x^2+4x}{x}\mathrm{\,d}x= \int\limits_{1}^{2}(x+4)\mathrm{d}x=\dfrac{11}{2}$.
	}
\end{ex}
\begin{ex}%[Cau-9]%[2D4N2-3]
	Cho tích phân $I=\displaystyle \int\limits_{0}^{\tfrac{\pi}{2}}(4x-1+\cos x)\mathrm{d}x =\pi\left(\dfrac{\pi}{a} -\dfrac{1}{b}\right)+c$, $(a$, $b$, $c\in \mathbb{Q})$. Tính $a-b+c$.
	\choice
	{\True $1$}
	{$-2$}
	{$\dfrac{1}{3}$}
	{$\dfrac{1}{2}$}
	\loigiai{
		Ta có
		\begin{align*}
			I &= \displaystyle \int\limits_{0}^{\tfrac{\pi}{2}} \left(4x - 1 + \cos x\right) \mathrm{d}x \\
			&= \left(2x^2 - x + \sin x\right)\Big|_0^{\tfrac{\pi}{2}} \\
			&= 2\cdot \left(\dfrac{\pi}{2}\right)^2 - \dfrac{\pi}{2} + \sin \left(\dfrac{\pi}{2}\right) \\
			&= \dfrac{\pi^2}{2} - \dfrac{\pi}{2} + 1 \\
			&= \pi\left(\dfrac{\pi}{2}-\dfrac{1}{2}\right)+1.
		\end{align*}
		Suy ra $a=2$, $b=2$, $c=1$. \\
		Vậy $a-b+c=2-2+1=1$.
	}
\end{ex}
\begin{ex}%[Cau-10]%[2D4H2-2]
	Cho $I=\displaystyle \int\limits_{0}^{1}\left(4x-2m^2\right)\mathrm{d}x$. Có bao nhiêu giá trị nguyên của $m$ để $I+6>0$?
	\choice
	{$1$}
	{$5$}
	{$2$}
	{\True $3$}
	\loigiai{
		Ta có $I=\displaystyle \int\limits_{0}^{1}\left(4x-2m^2\right)\mathrm{d}x= \left(2x^2-2m^2 x\right)\bigg|_0^1=-2m^2+2$.\\
		Khi đó $I+6>0\Leftrightarrow-2m^2+2+6>0\Leftrightarrow-m^2+4>0\Leftrightarrow-2<m<2$.\\
		Mà $m$ là số nguyên nên $m \in\{-1;0;1\}$.\\
		Vậy có $3$ giá trị nguyên của $m$ thỏa mãn yêu cầu.
	}
\end{ex}
\begin{ex}%[Cau-11]%[2D4H2-1]
	Cho $\displaystyle \int\limits_{1}^{2} \left[3f(x) + 2g(x)\right]\mathrm{d}x=1$, $\displaystyle \int\limits_{1}^{2} \left[2f(x) - g(x)\right]\mathrm{d}x=-3$. Khi đó $\displaystyle \int\limits_{1}^{2} f(x) \mathrm{\,d}x$ bằng
	\choice
	{$\dfrac{11}{7}$}
	{\True $-\dfrac{5}{7}$}
	{$\dfrac{6}{7}$}
	{$\dfrac{16}{7}$}
	\loigiai{
		Ta có $\displaystyle \int\limits_{1}^{2} \left[3f(x) + 2g(x)\right] \mathrm{d}x = 1 \Leftrightarrow \displaystyle 3\int\limits_{1}^{2} f(x) \mathrm{\,d}x + \displaystyle 2\int\limits_{1}^{2} g(x) \mathrm{\,d}x= 1$.\\
		Và $\displaystyle \int\limits_{1}^{2} \left[2f(x) - g(x)\right] \mathrm{d}x = -3 \Leftrightarrow \displaystyle 2\int\limits_{1}^{2} 2f(x) \mathrm{\,d}x - \displaystyle \int\limits_{1}^{2} g(x) \mathrm{\,d}x = -3$.\\
		Đặt $a = \displaystyle \int\limits_{1}^{2} f(x) \mathrm{\,d}x$, $b = \displaystyle \int\limits_{1}^{2} g(x) \mathrm{\,d}x$ ta có hệ phương trình
		$$\heva{&3a+2b=1\\&2a-b=-3\\} \Leftrightarrow \heva{&a=-\dfrac{5}{7}\\&b=\dfrac{11}{7}.}$$
		Vậy $\displaystyle \int\limits_{1}^{2} f(x) \mathrm{\,d}x=\dfrac{5}{7}$.
	}
\end{ex}
\begin{ex}%[Cau-12]%[2D4V2-6]
	Một vật chuyển động chậm với vận tốc $v(t)=160-10t$ (m/s). Quãng đường mà vật di chuyển được từ thời điểm $t=0$ (s) đến thời điểm mà vật dừng lại là
	\choice
	{$160$ (m)}
	{\True $1280$ (m)}
	{$0$ (m)}
	{$144$ (m)}
	\loigiai{
		Vật dừng lại đồng nghĩa với $v(t)=0 \Leftrightarrow 160-10t=0 \Leftrightarrow t=16$ (s).\\
		Quãng đường vật đi được là $s(t)=\displaystyle \int\limits_{0}^{16}(160-10t) \mathrm{d}t = \left(160t-5t^2\right)\bigg|_0^{16}=1280$ (m).
	}
\end{ex}
\Closesolutionfile{ans}
\indapan{6}{ans/ans-2-B12-De2-NLC}
\Opensolutionfile{ans}[ans/ans-2-B12-De2-DS]
\TNTF
\begin{ex}%[Cau-1]%[2D4H2-1]
	Cho hàm số $y=f(x)$ liên tục trên $\mathbb{R}$ và thỏa mãn $\displaystyle \int\limits_{-1}^{10} f(x)\mathrm{\,d}x=15$,\break $\displaystyle \int\limits_{3}^{5} f(x) \mathrm{\,d}x=-2$, $\displaystyle \int\limits_{-1}^{12} f(x) \mathrm{\,d}x=5$.
	\choiceTF[t]
	{$\displaystyle \int\limits_{10}^{-1} 2f(x)\mathrm{\,d}x=30$}
	{$\displaystyle \int\limits_{10}^{12} \left[f(x)-2\right] \mathrm{d}x=-12$}
	{\True $\displaystyle \int\limits_{-1}^{3} f(x) \mathrm{\,d}x + \int\limits_{5}^{10} f(x) \mathrm{\,d}x=17$}
	{Biết rằng $f(x)>0$, $\forall x>3$; $f(x)<0$, $\forall x<3$ và $\displaystyle \int\limits_{-1}^{12} \left|f(x)\right|\mathrm{d}x=5$. Khi đó $\displaystyle \int\limits_{-1}^{3} f(x)\mathrm{\,d}x - \int\limits_{5}^{12} f(x)\mathrm{\,d}x=3$}
	\loigiai{
		
		\begin{itemchoice}
			\itemch Sai. Vì $ \displaystyle \int\limits_{10}^{-1} 2f(x) \mathrm{\,d}x = -2\int\limits_{-1}^{10} f(x) \mathrm{\,d}x = -2\cdot 15=-30$.
			\itemch Sai. Vì 
			\begin{align*}
				\displaystyle \int\limits_{10}^{12} \left[f(x)-2\right] \mathrm{d}x &= \int\limits_{10}^{12} f(x) \mathrm{\,d}x - \int\limits_{10}^{12} 2 \mathrm{\,d}x \\
				&= \int\limits_{-1}^{12} f(x) \mathrm{\,d}x - \int\limits_{-1}^{10} f(x) \mathrm{\,d}x - 4 \\
				&= 5-15-4 \\
				&= -14.
			\end{align*}
			\itemch Đúng. Vì $\displaystyle \int\limits_{-1}^{3} f(x) \mathrm{\,d}x + \int\limits_{5}^{10} f(x) \mathrm{\,d}x = \int\limits_{-1}^{10} f(x) \mathrm{\,d}x - \int\limits_{3}^{5} f(x) \mathrm{\,d}x = 15 - (-2) = 17$.
			\itemch Sai.\\
			Ta có $f(x)>0$, $\forall x>3$; $f(x)<0$, $\forall x<3$.\\
			Suy ra $\displaystyle \int\limits_{-1}^{12} \left|f(x)\right|\mathrm{d}x= -\int\limits_{-1}^{3} f(x)\mathrm{\,d}x+\int\limits_{3}^{5} f(x)\mathrm{\,d}x + \int\limits_{5}^{12} f(x)\mathrm{\,d}x=5$.\\
			Khi đó\\
			\begin{align*}
				\displaystyle \int\limits_{-1}^{3} f(x) \mathrm{\,d}x -  \int\limits_{5}^{12} f(x) \mathrm{\,d}x &=  \int\limits_{3}^{5} f(x) \mathrm{\,d}x - 5\\
				&= -2-5 \\
				&= -7.
			\end{align*}
		\end{itemchoice}
	}
\end{ex}
\begin{ex}%[Cau-2]%[2D4H2-2]
	Cho hàm số $y = f(x)$ liên tục trên $\mathbb{R}$ thỏa mãn $f(x) = \heva{&\dfrac{4x^2-3}{x}& \text{khi} &x\ge 1\\&ax+b &\text{khi} &-2<x<1\\&x^2+4x-4 &\text{khi} &x \le -2.}$
	\choiceTF[t]
	{\True $\displaystyle \int\limits_{-5}^{-2} f(x) \mathrm{\,d}x = -15$}
	{\True $\displaystyle \int\limits_{3}^{4} f(x) \mathrm{\,d}x = 14 +3\ln 3 - 6\ln 2$}
	{$\displaystyle \int\limits_{0}^{1} f(x) \mathrm{\,d}x = a+b$}
	{\True $\displaystyle \int\limits_{-3}^{0} f(x) \mathrm{\,d}x = \dfrac{-53}{3}$}
	\loigiai{
		\begin{itemchoice}
			\itemch Đúng. Vì $\displaystyle \int\limits_{-5}^{-2} f(x) \mathrm{\,d}x =\int\limits_{-5}^{-2} \left(x^2+4x-4\right) \mathrm{d}x = -15$.
			\itemch Đúng. Vì $\displaystyle \int\limits_{3}^{4} f(x) \mathrm{\,d}x = \int\limits_{3}^{4} \dfrac{4x^2-3}{x} \mathrm{\,d}x = \int\limits_{3}^{4} \left(4x - \dfrac{3}{x} \right) \mathrm{d}x = 14 +3\ln 3 - 6\ln 2$.
			\itemch Sai. Vì $\displaystyle \int\limits_{0}^{1} f(x) \mathrm{\,d}x =\int\limits_{0}^{1}(ax+b)\mathrm{d}x=\left(\dfrac{ax^2}{2} + bx\right)\bigg|_0^1=\dfrac{a}{2}+b\ne a+b$.\\ (Chỉ đúng với $a=0$).
			\itemch Đúng. Vì $\displaystyle \int\limits_{-3}^{0} f(x) \mathrm{\,d}x= \int\limits_{-3}^{-2} \left(x^2+4x-4\right)\mathrm{d}x+\int\limits_{-2}^{0} (ax+b)\mathrm{d}x$.\\
			Do hàm số $y=f(x)$ liên tục trên $\mathbb{R}$ nên 
			$$\heva{&\lim\limits_{x\to 1^-} f(x) = \lim\limits_{x\to 1^+} f(x)\\&\lim\limits_{x\to -2^+} f(x) = \lim\limits_{x\to -2^-} f(x)\\}\Rightarrow \heva{&a+b=1\\&-2a+b=-8\\} \Rightarrow a=3;\, b=-2.$$
			Suy ra $\displaystyle \int\limits_{-3}^{0} f(x) \mathrm{\,d}x = \int\limits_{-3}^{-2} \left(x^2+4x-4\right)\mathrm{d}x+\int\limits_{-2}^{0} (3x-2)\mathrm{d}x=\dfrac{-53}{3}$.
		\end{itemchoice}
	}
\end{ex}
\begin{ex}%[Cau-3]%[2D4H2-4]
	Cho hàm số $f(x)$; $g(x)$ thỏa mãn $\displaystyle \int\limits_{2}^{6} f(x) \mathrm{\,d}x= 3$; $\displaystyle \int\limits_{2}^{6} g(x)\mathrm{\,d}x=-2$.
	\choiceTF[t]
	{\True $\displaystyle \int\limits_{2}^{6} \left[f(x)+g(x)\right]\mathrm{,d}x=1$}
	{$\displaystyle \int\limits_{2}^{6} \left[3f(x)-g(x)-3\right]\mathrm{d}x=10$}
	{\True $\displaystyle \int\limits_{2}^{6} \left[3\mathrm{e}^x-2f(x)\right]\mathrm{d}x= 3\mathrm{e}^6-3\mathrm{e}^2-6$}
	{Biết $\displaystyle \int\limits_{2}^{6}\left[3g(x)-\dfrac{2x-3}{x^2}\right]\mathrm{d}x =a+b\ln 3$, với $a$; $b\in\mathbb{Q}$.
		Khi đó $a^2+12b=-8$}
	\loigiai{
		\begin{itemchoice}
			\itemch Đúng. Vì $\displaystyle \int\limits_{2}^{6}\left[f(x)+g(x)\right] \mathrm{d}x=\displaystyle\int\limits_{2}^{6} f(x)\mathrm{\,d}x+\displaystyle\int\limits_{2}^{6} g(x) \mathrm{\,d}x=3-2=1$.
			\itemch Sai. Vì 
			\begin{align*}
				&\displaystyle \int\limits_{2}^{6}\left[3f(x)-g(x)-3\right]\mathrm{d}x \\
				= & 3\displaystyle\int\limits_{2}^{6} f(x) \mathrm{\,d}x-\displaystyle\int\limits_{2}^{6} g(x) \mathrm{\,d}x - \displaystyle\int\limits_{2}^{6} 3 \mathrm{\,d}x \\
				= & 3\cdot 3-(-2)-12=-1\ne 10.
			\end{align*}
			\itemch Đúng. Vì $\displaystyle \int\limits_{2}^{6} \left[3\mathrm{e}^x-2f(x)\right] \mathrm{\,d}x = 3\displaystyle\int\limits_{2}^{6} \mathrm{e}^x \mathrm{\,d}x - 2\displaystyle\int\limits_{2}^{6} f(x) \mathrm{\,d}x = 3\mathrm{e}^6-3\mathrm{e}^2-6$.
			\itemch Sai. Ta có
			\begin{align*}
				& \displaystyle \int\limits_{2}^{6} \left[3g(x)-\dfrac{2x-3}{x^2}\right] \mathrm{d}x \\
				= &\ 3\displaystyle\int\limits_{2}^{6} g(x)\mathrm{\,d}x- \displaystyle\int\limits_{2}^{6} \dfrac{2x-3}{x^2} \mathrm{\,d}x=-6- \left(2\ln \left|x\right|+\dfrac{3}{x}\right)\bigg|_2^6 \\
				= &-5 + 2\ln 3.
			\end{align*}
			Suy ra $a=-5$; $b=2$.\\
			Vậy $a^2+12b=25+24=49\ne 25$.
		\end{itemchoice}
	}
\end{ex}
\begin{ex}%[Cau-4]%[2D4V2-2]
	Cho hàm số $y=f(x)$ liên tục trên $\mathbb{R}$, đồ thị hàm số $(C)\colon y=f'(x)$ trên đoạn $[-3;6]$ là đường gấp khúc như hình vẽ. Khi đó
	\begin{center}
		\begin{tikzpicture}[scale=0.6,>=stealth, font=\footnotesize, line join=round, line cap=round]  
			%Vẽ hệ trục
			\draw[->] (-4.5,0)--(7.5,0) node[below]{$x$};
			\draw[->] (0,-3.5)--(0,4.5) node[left]{$y$};
			\node at (0,0) [below left]{$O$};
			\path 
				(-3,-2) coordinate (A)
				(-3,0) coordinate (Ax)
				(0,-2) coordinate (Ay)
				(2,3) coordinate (B)
				(2,0) coordinate (Bx)
				(0,3) coordinate (By)
				(6,-1) coordinate (C)
				(6,0) coordinate (Cx)
				(0,-1) coordinate (Cy)
				(5,0) coordinate (Ex)
			;
			\fill[black] (A) circle(1pt) node[below left]{$A$};
			\fill[black] (Ax) circle(1pt) node[above]{$-3$};
			\fill[black] (Ay) circle(1pt) node[right]{$-2$};
			\fill[black] (B) circle(1pt) node[above right]{$B$};
			\fill[black] (Bx) circle(1pt) node[below]{$2$};
			\fill[black] (By) circle(1pt) node[left]{$3$};
			\fill[black] (C) circle(1pt) node[below]{$C$};
			\fill[black] (Cx) circle(1pt) node[above]{$6$};
			\fill[black] (Ex) circle(1pt) node[above right]{$5$};
			\fill[black] (Ex) circle(0pt) node[below left]{$E$};
			\draw[dashed] (Ax)--(A)--(Ay) (Bx)--(B)--(By) (Cx)--(C);
			\draw (A)--(B)--(C);
		\end{tikzpicture}
	\end{center}
	\choiceTF[t]
	{\True $\displaystyle \int\limits_{-3}^{-1} f'(x)\mathrm{\,d}x=-2$}
	{\True $\displaystyle \int\limits_{0}^{1} f'(x)\mathrm{\,d}x=\dfrac{3}{2}$}
	{$f(2)-f(6)=4$}
	{$f(5)+f(-3)-2f(2)=-10$}
	\loigiai{
		\begin{itemchoice}
			\itemch Đúng.\\
			Ta có $A(-3;-2)$, $B(2;3)$ $\Rightarrow AB\colon y=x+1$.\\
			Khi đó $\displaystyle \int\limits_{-3}^{-1} f'(x)\mathrm{\,d}x= \displaystyle\int\limits_{-3}^{-1} (x+1)\mathrm{\,d}x=-2$.
			\itemch Đúng. Vì $\displaystyle \int\limits_{0}^{1} f'(x) \mathrm{\,d}x= \displaystyle\int\limits_{0}^{1}(x+1)\mathrm{\,d}x=\dfrac{3}{2}$.
			\itemch Sai.\\
			Ta có $B(2;3)$, $E(5;0)\Rightarrow BC\colon y=-x+5$.\\
			Khi đó $\displaystyle \int\limits_{2}^{6} f'(x)\mathrm{\,d}x= \displaystyle\int\limits_{2}^{6}(-x+5) \mathrm{\,d}x = 4$.\\
			Vì vậy $f(2)-f(6)=-\displaystyle \int\limits_{2}^{6} f'(x)\mathrm{\,d}x=-4$.
			\itemch Sai.\\
			Ta có $\displaystyle \int\limits_{-3}^{2} f'(x)\mathrm{\,d}x= \displaystyle\int\limits_{-3}^{2} (x+1)\mathrm{\,d}x=\dfrac{5}{2}=f(2)-f(-3)$.\\
			Mặt khác $\displaystyle \int\limits_{2}^{5} f'(x)\mathrm{\,d}x= \displaystyle\int\limits_{2}^{5}(-x+5)\mathrm{\,d}x=\dfrac{9}{2}=f(5)-f(2)$.\\
			Vì vậy $f(5)+f(-3)-2f(2)=\dfrac{9}{2}-\dfrac{5}{2}=2$.
		\end{itemchoice}
	}
\end{ex}
\Closesolutionfile{ans}
\indapan{3}{ans/ans-2-B12-De2-DS}
\Opensolutionfile{ans}[ans/ans-2-B12-De2-KQ]
\TNSA
\begin{ex}%[Cau-1]%[2D4V2-2]
	Cho $\displaystyle \int\limits_{1}^{4}\sqrt{\dfrac{1}{4x}+\dfrac{\sqrt{x}+ \mathrm{e}^x}{\sqrt{x}\cdot \mathrm{e}^{2x}}}\mathrm{\,d}x=a+\mathrm{e}^b-\mathrm{e}^c$ với $a$, $b$, $c$ là các số nguyên. Tính giá trị của biểu thức $S=a+b+c$.
	\shortans{$-4$}
	\loigiai{
		Ta có
		\begin{align*}
			\displaystyle \int\limits_{1}^{4} \sqrt{\dfrac{1}{4x} + \dfrac{\sqrt{x} + \mathrm{e}^x}{\sqrt{x}\cdot \mathrm{e}^{2x}}} \mathrm{\,d}x 
			&= \displaystyle\int\limits_{1}^{4} \sqrt{\left(\dfrac{1}{2\sqrt{x}}\right)^2 + 2\dfrac{1}{2\sqrt{x}\cdot \mathrm{e}^x}+\left(\dfrac{1}{\mathrm{e}^x}\right)^2} \mathrm{\,d}x\\
			&= \displaystyle\int\limits_{1}^{4}\sqrt{\left(\dfrac{1}{2\sqrt{x}} + \dfrac{1}{\mathrm{e}^x}\right)^2}\mathrm{d}x\\
			&=\displaystyle\int\limits_{1}^{4}\left(\dfrac{1}{2\sqrt{x}}+\dfrac{1}{\mathrm{e}^x}\right) \mathrm{\,d}x\\
			&=\left(\sqrt{x}-\mathrm{e}^{-x}\right)\bigg|_1^4\\
			&=1-\mathrm{e}^{-4}+\mathrm{e}^{-1}\\
			&=a+\mathrm{e}^b-\mathrm{e}^c.
		\end{align*}
		$\Rightarrow \heva{&a=1\\&b=-1\\&c=-4.}$\\
		Vậy $a+b+c=1+(-1)+(-4)=-4$.
	}
\end{ex}
\begin{ex}%[Cau-2]%[2D4V2-6]
	Tốc độ chuyển động của thang máy từ tầng $1$ lên tầng cao nhất theo thời gian $t$ (giây) được cho bởi công thức 
	$$v(t) = \heva{&t&\text{khi}&\ 0 \le t \le 2\\&2 &\text{khi}&\ 2 < t \le 20\\&12-0{,}5t &\text{khi}&\ 20 < t \le 24.}$$
	Tính vận tốc trung bình của thang máy.
	\shortans{$1{,}75$}
	\loigiai{
		Quãng đường chuyển động của thang máy là\\
		$s=\displaystyle \int\limits_{0}^{24} v(t)\mathrm{\,d}t = \displaystyle\int\limits_{0}^{2} v(t)\mathrm{\,d}t + \displaystyle\int\limits_{2}^{20} v(t)\mathrm{\,d}t + \displaystyle\int\limits_{20}^{24} v(t)\mathrm{\,d}t = \displaystyle\int\limits_{0}^{2} t\mathrm{\,d}t +\displaystyle\int\limits_{2}^{20} 2 \mathrm{\,d}t+\displaystyle\int\limits_{20}^{24} (12-0{,}5t)\mathrm{d}t = 42$.\\
		Tốc độ trung bình của thang máy là $v_{tb}=\dfrac{s}{t}=\dfrac{42}{24}=1{,}75$ (m/s).
	}
\end{ex}
\begin{ex}%[Cau-3]%[2D4H2-3]
	Biết $\displaystyle \int\limits_{0}^{\tfrac{\pi}{2}} \left(x-1+\sin 2x\right) \mathrm{d}x = \pi \left(\dfrac{\pi}{a} - \dfrac{1}{b}\right) + 1$, $(a$, $b\in \mathbb{Q})$. Tính $a+2b$.
	\shortans{$12$}
	\loigiai{
		Ta có
		\begin{align*}
			\displaystyle \int\limits_{0}^{\tfrac{\pi}{2}} (x-1+\sin 2x) \mathrm{d}x &= \left(\dfrac{1}{2}x^2-x-\dfrac{1}{2}\cos 2x\right)\bigg|_0^{\tfrac{\pi}{2}}\\
			&=\dfrac{1}{2}\cdot \left(\dfrac{\pi}{2}\right)^2 - \dfrac{\pi}{2} - \dfrac{1}{2} \cos \left(2\cdot \dfrac{\pi}{2}\right) + \dfrac{1}{2}\\
			&=\dfrac{\pi^2}{8}-\dfrac{\pi}{2}+1\\
			&=\pi\left(\dfrac{\pi}{8}-\dfrac{1}{2}\right)+1.
		\end{align*}
		Suy ra $a=8$; $b=2$.\\
		Vậy $a+2b=8+2\cdot 2=12$.
	}
\end{ex}
\begin{ex}%[Cau-4]%[2D4V2-3]
	Cho $M$, $N$ là các số thực, xét hàm số $f(x)=M\cdot \sin \pi x + N\cdot \cos \pi x$ thỏa mãn $f(1)=3$ và $\displaystyle \int\limits_{0}^{\tfrac{1}{2}} f(x) \mathrm{\,d}x= -\dfrac{1}{\pi}$. Tính $f'\left(\dfrac{1}{4}\right)$. (Kết quả làm tròn đến hàng phần mười).
	\shortans{$11{,}1$}
	\loigiai{
		Ta có $f(1)=3\Leftrightarrow M\cdot \sin \pi+N\cdot \cos \pi=3\Leftrightarrow N=-3$.\\
		Mặt khác \\
		\begin{align*}
			\displaystyle \int\limits_{0}^{\tfrac{1}{2}} f(x) \mathrm{\,d}x = -\dfrac{1}{\pi} &\Leftrightarrow \int\limits_{0}^{\tfrac{1}{2}} \left(M\cdot \sin \pi x + N \cdot \cos \pi x\right) \mathrm{d}x = -\dfrac{1}{\pi}\\
			&\Leftrightarrow \left(-\dfrac{M}{\pi}\cos \pi x - \dfrac{3}{\pi} \sin \pi x\right)\bigg|_0^{\tfrac{1}{2}} = -\dfrac{1}{\pi}\\
			&\Leftrightarrow -\dfrac{3}{\pi} + \dfrac{M}{\pi} = -\dfrac{1}{\pi}\\
			&\Leftrightarrow M = 2.
		\end{align*}
		Ta được $f(x) = 2\cdot \sin \pi x - 3\cdot \cos \pi x$ nên $f'(x) = 2\pi \cos \pi x + 3\pi \sin \pi x$.\\
		Vậy $f'\left(\dfrac{1}{4}\right) = \dfrac{5\pi \sqrt{2}}{2} \approx 11{,}1$.
	}
\end{ex}
\begin{ex}%[Cau-5]%[2D4V2-6]
	Ba Tí muốn làm cửa sắt được thiết kế như hình bên dưới. Vòm cổng có hình dạng là một Parabol. Giá $1$ m$^2$ cửa sắt là $660\,000$ đồng. Cửa sắt có giá (nghìn đồng) là bao nhiêu?
	\begin{center}
		\begin{tikzpicture}[smooth,samples=300,scale=1,>=stealth, font=\footnotesize]
			%\draw[->] (-3.0,0)--(4.0,0) node[below]{$x$};
			%\draw[->] (0,-1.8)--(0,1.5) node[right]{$y$};
			\path 
			(-2.5,0) coordinate (A)
			(2.5,0) coordinate (B)
			(0,0.5) coordinate (C)
			(2.5,-1.5) coordinate (E)
			(-2.5,-1.5) coordinate (D)
			(0,0) coordinate (O)
			;
			\draw[thick, magenta,domain=-2.5:2.5] plot(\x,{-0.08*(\x)^2+1/2});
			%\draw[thick, magenta,domain=-2.5:2.5] plot(\x,{-0.04*(\x)^2+2-0.2});
			\draw[dashed] (D)--($(D)+(0,-0.5)$)--($(E)+(0,-0.5)$)--(E);
			\draw[dashed] (C)--(3,0.5)--(3,-1.5)--(0,-1.5);
			%\draw[dashed] (A)--(B);
			\draw (A)--(D)--(E)--(B) (0,-1.5)--(0,0.5);
			\node[rotate=90] at (3.2,-0.8) {$2$ m};
			\node[rotate=90] at (2.7,-0.8) {$1{,}5$ m};
			\node[rotate=0] at (0.3,-2.2) {$5$ m};
		\end{tikzpicture}
	\end{center}
	\shortans{$6050$}
	\loigiai{
		\begin{center}
			\begin{tikzpicture}[smooth,samples=300,scale=1,>=stealth, font=\footnotesize]
				\draw[->] (-3.0,0)--(4.0,0) node[below]{$x$};
				\draw[->] (0,-1.8)--(0,1.5) node[right]{$y$};
				\path 
					(-2.5,0) coordinate (A)
					(2.5,0) coordinate (B)
					(0,0.5) coordinate (C)
					(2.5,-1.5) coordinate (E)
					(-2.5,-1.5) coordinate (D)
					(0,0) coordinate (O)
				;
				\draw[thick, magenta,domain=-2.5:2.5] plot(\x,{-0.08*(\x)^2+1/2});
				%\draw[thick, magenta,domain=-2.5:2.5] plot(\x,{-0.04*(\x)^2+2-0.2});
				\draw[dashed] (D)--($(D)+(0,-0.5)$)--($(E)+(0,-0.5)$)--(E);
				\draw[dashed] (C)--(3,0.5)--(3,-1.5)--(0,-1.5);
				\draw[dashed] (A)--(B);
				\draw (A)--(D)--(E)--(B);
				\node[rotate=90] at (3.2,-0.8) {$2$ m};
				\node[rotate=90] at (2.7,-0.8) {$1{,}5$ m};
				\node[rotate=0] at (0.3,-2.2) {$5$ m};
				\draw[fill=black] (O) circle(1pt) node[below right]{$O$};
				\draw[fill=black] (A) circle(1pt) node[below left]{$A$};
				\draw[fill=black] (B) circle(1pt) ($(B)+(60:0.3)$) node {$B$};
				\draw[fill=black] (C) circle(1pt) node[above right]{$C$};
				\draw[fill=black] (D) circle(1pt) node[left]{$D$};
				\draw[fill=black] (E) circle(1pt) node[below right]{$E$};
				\draw[->] (-3.5,0.25) node[left]{Phần 1}--(-1,0.25);
				\draw[->] (-3.5,-1) node[left]{Phần 2}--(-2,-1);
			\end{tikzpicture}
		\end{center}
		Từ hình vẽ ta chia cửa rào sắt ra thành $2$ phần như trên.\\
		Khi đó $S = S_1 + S_2 = S_1 + 5\cdot 1{,}5 = S_1 + 7{,}5$.\\
		Để tính $S_1$ ta vận dụng kiến thức tính diện tích hình phẳng của tích phân.\\
		Gắn hệ trục $Oxy$ trong đó $O$ trung với trung điểm của $AB$, $OB\subset Ox$, $OC \subset Oy$.\\
		Theo đề bài ta có đường cong có dạng hình Parabol. Giả sử $(P)\colon y=ax^2+bx+c$.\\
		Khi đó $\heva{&A\left(-\dfrac{5}{2}; 0\right)\in (P)\\&B\left(\dfrac{5}{2}; 0\right)\in (P)\\&C\left(0;\dfrac{1}{2}\right)\in (P)} \Leftrightarrow\heva{&\dfrac{25}{4}a-\dfrac{5}{2}b+c=0\\&\dfrac{25}{4}a+ \dfrac{5}{2}b+c=0\\&c=\dfrac{1}{2}} \Leftrightarrow \heva{&a= -\dfrac{2}{25}\\ &b = 0\\&c=\dfrac{1}{2}.}$\\
		$\Rightarrow (P) \colon y = -\dfrac{2}{25} x^2 + \dfrac{1}{2}$.\\
		Diện tích $S_2 = 2\displaystyle\int\limits_{0}^{2{,}5} \left(-\dfrac{2}{25} x^2 + \dfrac{1}{2}\right) \mathrm{d}x = \dfrac{10}{6}\ (\mathrm{m}^2)$.\\
		$\Rightarrow S = \dfrac{55}{6}$ (m$^2$).\\
		Vậy  giá tiền cửa sắt là $\dfrac{55}{6} \cdot 660\,000= 6050$ (nghìn đồng).
	}
\end{ex}
\begin{ex}%[Cau-6]%[2D4V2-6]
	Một ô tô đang chạy đều với vận tốc $15$ (m/s) thì phía trước xuất hiện chướng ngại vật nên người lái đạp phanh gấp. Kể từ thời điểm đó, ô tô chuyển động chậm dần đều với gia tốc $-a$ (m/s$^2$). Tìm giá trị của $a$ biết ô tô chuyển động thêm được $20$ (m) thì dừng hẳn. (Kết quả làm tròn đến hàng phần trăm).
	\shortans{$5{,}63$}
	\loigiai{
		Gọi $x(t)$ là hàm biểu diễn quãng đường, $v(t)$ là hàm vận tốc.\\
		Ta có $\displaystyle \int\limits_{0}^{t} (-a) \mathrm{\,d}x = -at \Rightarrow v(t) = -at + 15$.\\
		Mặt khác $x(t)-x(0)=\displaystyle \int\limits_{0}^{t}v(t)\mathrm{\,d}x =\displaystyle \int\limits_{0}^{t}(-at+15)\mathrm{d}x=-\dfrac{1}{2}at^2 + 15t$.\\
		$\Rightarrow x(t)=-\dfrac{1}{2}at^2 + 15t$.\\
		Ta có $\heva{&v(t)=0\\&x(t)=20}\Leftrightarrow\heva{&-at+15=0\\&-\dfrac{1}{2}at^2 + 15t = 20}\Rightarrow -\dfrac{15}{2}t+15t=20 \Rightarrow t=\dfrac{8}{3}$.\\
		$\Rightarrow a=\dfrac{45}{8}\approx 5{,}63$.
	}
\end{ex}
\Closesolutionfile{ans}
\indapan{6}{ans/ans-2-B12-De2-KQ}
% \begin{name}
	{NGUYÊN HÀM - TÍCH PHÂN}
	{KT ỨNG DỤNG NGUYÊN HÀM - TÍCH PHÂN}
	{\tentruong}
	{\thoigian}
\end{name}
\setcounter{ex}{0}\setcounter{bt}{0}

\Opensolutionfile{ans}[ans/ans-2-C4B13-D1]
\TN
\begin{ex}%[Dự án 2025 - đề cấu trúc mới, Hung Doan]%[2D4N3-1]
	Cho hàm số $y=f(x)$ liên tục trên đoạn $[a;b]$. Khi đó, diện tích hình phẳng giới hạn bởi đồ thị của hàm số $y=f(x)$, trục hoành và hai đường thẳng $x=a$, $x=b$ được tính bởi công thức.
	\choice
	{$S=\pi \displaystyle\int \limits_a^b f(x)\mathrm{\,d}x$}
	{$S=\pi \displaystyle\int \limits_a^b |f(x)|\mathrm{\,d}x$}
	{$S=\displaystyle\int \limits_a^b f(x)\mathrm{\,d}x$}
	{\True $S=\displaystyle\int \limits_a^b |f(x)|\mathrm{\,d}x$}
	\loigiai{
		Công thức diện tích hình phẳng giới hạn bởi đồ thị của hàm số $y=f(x)$, trục hoành và hai đường thẳng $x=a$, $x=b$ là $S=\displaystyle\int \limits_a^b |f(x)|\mathrm{\,d}x$.
	}
\end{ex}
\begin{ex}%[Dự án 2025 - đề cấu trúc mới, Hung Doan]%[2D4N3-1]
	Diện tích hình phẳng giới hạn bởi đồ thị của hàm số $y=x^2-4x+3$, trục hoành và hai đường thẳng $x=0$, $x=3$ là
	\choice
	{$S=\pi \displaystyle\int \limits_0^3 \left(x^2-4x+3\right)\mathrm{\,d}x$}
	{$S=\pi \displaystyle\int \limits_0^3 \left|x^2-4x+3\right|\mathrm{\,d}x$}
	{$S=\displaystyle\int \limits_0^3 \left(x^2-4x+3\right)\mathrm{\,d}x$}
	{\True $S=\displaystyle\int \limits_0^3 \left|x^2-4x+3\right|\mathrm{\,d}x$}
	\loigiai{
		Diện tích hình phẳng giới hạn bởi đồ thị của hàm số $y=x^2-4x+3$, trục hoành và hai đường thẳng $x=0$, $x=3$ là
		$$S=\displaystyle\int \limits_0^3 \left|x^2-4x+3\right|\mathrm{\,d}x.$$
	}
\end{ex}
\begin{ex}%[Dự án 2025 - đề cấu trúc mới, Hung Doan]%[2D4N3-1]
	Cho hai hàm số $y=f(x)$, $y=g(x)$ liên tục trên đoạn $[a;b]$. Khi đó, diện tích hình phẳng giới hạn bởi đồ thị của hai hàm số $y=f(x)$, $y=g(x)$ và hai đường thẳng $x=a$, $x=b$ được tính bởi công thức
	\choice
	{$S=\pi \displaystyle\int \limits_a^b [f(x)-g(x)]\mathrm{\,d}x$}
	{$S=\displaystyle\int \limits_a^b \left|f^2(x)-g^2(x)\right|\mathrm{\,d}x$}
	{\True $S=\displaystyle\int \limits_a^b |f(x)-g(x)|\mathrm{\,d}x$}
	{$S=\pi ^2\displaystyle\int \limits_a^b [f(x)-g(x)]\mathrm{\,d}x$}
	\loigiai{
		Diện tích hình phẳng giới hạn bởi đồ thị của hai hàm số $y=f(x)$, $y=g(x)$ và hai đường thẳng $x=a$, $x=b$ là
		$$S=\displaystyle\int \limits_a^b |f(x)-g(x)|\mathrm{\,d}x.$$
	}
\end{ex}
\begin{ex}%[Dự án 2025 - đề cấu trúc mới, Hung Doan]%[2D4N3-1]
	\immini{Cho hàm số $y=f(x)$ có đồ thị như hình vẽ. Diện tích $S$ của hình phẳng trong phần gạch sọc được tính theo công thức
		\choice
		{$S=-\displaystyle\int \limits_a^b f(x)\mathrm{d}x-\displaystyle\int \limits_b^c f(x)\mathrm{d}x$}
		{$S=\displaystyle\int \limits_a^c f(x)\mathrm{d}x$}
		{$S=\displaystyle\int \limits_a^b f(x)\mathrm{d}x+\displaystyle\int \limits_b^c f(x)\mathrm{d}x$}
		{\True $S=-\displaystyle\int \limits_a^b f(x)\mathrm{d}x+\displaystyle\int \limits_b^c f(x)\mathrm{d}x$}}{
		\begin{tikzpicture}[scale=1, font=\footnotesize, line join=round, line cap=round, >=stealth]
			\def\xt{-2.5} \def\xp{3} \def\yt{3.5} \def\yd{-1.5}
			\draw[->, line width=0.8pt](\xt, 0)--(\xp, 0) node[below]{$x$};
			\draw[->, line width=0.8pt](0,\yd)--(0,\yt) node[left]{$y$};
			\node at (0, 0) [below left]{$O$};
			\clip(\xt+0.1,\yd+0.1) rectangle(\xp-0.1,\yt-0.1);
			\draw[smooth,samples=300] plot(\x,{-0.5*(\x)^3+0.2*(\x)^2+2*(\x)+1});
			\node at (-2, 2) [right]{$y=f(x)$};
			\draw[fill=black](-1.423,0) circle (.04)+(-135:.25) node{$a$};
			\draw[fill=black](-0.584,0) circle (.04)+(100:.25) node{$b$};
			\draw[fill=black](2.41,0) circle (.04)+(-135:.25) node{$c$};
			\fill[pattern=north east lines,smooth] (-1.423,0)--plot[domain=-1.423:2.41](\x,{-0.5*(\x)^3+0.2*(\x)^2+2*(\x)+1})--(2.41,0)--cycle;
		\end{tikzpicture}
	}
	\loigiai{
		Diện tích $S$ của hình phẳng trong phần gạch sọc được tính theo công thức là
		$$S=-\displaystyle\int \limits_a^b f(x)\mathrm{d}x+\displaystyle\int \limits_b^c f(x)\mathrm{d}x.$$
	}
\end{ex}
\begin{ex}%[Dự án 2025 - đề cấu trúc mới, Hung Doan]%[2D4N3-3]
	Cho hàm số $y=f(x)$ liên tục trên đoạn $[a;b]$. Gọi $D$ là hình phẳng giới hạn bởi đồ thị hàm số $y=f(x)$, trục hoành và hai đường thẳng $x=a$, $x=b$ $(a<b)$. Thể tích khối tròn xoay tạo thành khi quay $D$ quanh trục $Ox$ được tính theo công thức
	\choice
	{$V=\displaystyle\int \limits_a^b f^2(x)\mathrm{\,d}x$}
	{$V=\pi \displaystyle\int \limits_a^b f(x)\mathrm{\,d}x$}
	{$V=\displaystyle\int \limits_a^b |f(x)|\mathrm{\,d}x$}
	{\True $V=\pi \displaystyle\int \limits_a^b f^2(x)\mathrm{\,d}x$}
	\loigiai{
		Thể tích khối tròn xoay tạo thành khi quay $D$ quanh trục $Ox$ được tính theo công thức là
		$$V=\pi \displaystyle\int \limits_a^b f^2(x)\mathrm{\,d}x.$$
	}
\end{ex}
\begin{ex}%[Dự án 2025 - đề cấu trúc mới, Hung Doan]%[2D4N3-1]
	Diện tích hình phẳng giới hạn bởi đồ thị của hai hàm số $y=x^3-3x$, $y=x$ và hai đường thẳng $x=-1$, $x=3$ được xác định bởi công thức
	\choice
	{$S=\displaystyle\int \limits_{-1}^3 \left(x^3-3x+x\right)\mathrm{\,d}x$}
	{$S=\displaystyle\int \limits_{-1}^3 \left(x^3-3x-x\right)\mathrm{\,d}x$}
	{$S=\displaystyle\int \limits_{-1}^3 \left|x^3-3x+x\right|\mathrm{\,d}x$}
	{\True $S=\displaystyle\int \limits_{-1}^3 \left|x^3-4x\right|\mathrm{\,d}x$}
	\loigiai{
		Diện tích hình phẳng giới hạn bởi $y=x^3-3x$, $y=x$ và $x=-1$, $x=3$ là
		$$S=\displaystyle\int \limits_{-1}^3 \left|x^3-3x-x\right|\mathrm{\,d}x=\displaystyle\int \limits_{-1}^3 \left|x^3-4x\right|\mathrm{\,d}x.$$
	}
\end{ex}
\begin{ex}%[Dự án 2025 - đề cấu trúc mới, Hung Doan]%[2D4N3-1]
	\immini{Cho hàm số $f(x)$ liên tục trên $\mathbb{R}$. Gọi $S$ là diện tích hình phẳng giới hạn bởi các đường $y=f(x)$, $y=0$,$ x=-1$, $x=2$ (như hình vẽ bên). Mệnh đề nào dưới đây đúng?
		\choice
		{$S=-\displaystyle\int \limits_{-1}^1 f(x)\mathrm{\,d}x+\displaystyle\int \limits_1^2 f(x)\mathrm{\,d}x$}
		{$S=\displaystyle\int \limits_{-1}^1 f(x) \mathrm{\,d}x+\displaystyle\int \limits_1^2 f(x) \mathrm{\,d}x$}
		{\True $S=\displaystyle\int \limits_{-1}^1 f(x) \mathrm{\,d}x-\displaystyle\int \limits_1^2 f(x) \mathrm{\,d}x$}
		{$S=-\displaystyle\int \limits_{-1}^1 f(x) \mathrm{\,d}x-\displaystyle\int \limits_1^2 f(x) \mathrm{\,d}x$}}{
		\begin{tikzpicture}[scale=1, font=\footnotesize, line join=round, line cap=round, >=stealth]
			\def\xt{-1.5} \def\xp{3} \def\yt{3} \def\yd{-1.5}
			\draw[->, line width=0.8pt](\xt, 0)--(\xp, 0) node[below]{$x$};
			\draw[->, line width=0.8pt](0,\yd)--(0,\yt) node[left]{$y$};
			\node at (0, 0) [below left]{$O$};
			\clip(\xt+0.1,\yd+0.1) rectangle(\xp-0.1,\yt-0.1);
			\draw[smooth,samples=300] plot(\x,{(\x)^3-2*(\x)^2-(\x)+2});
			\fill[pattern=north east lines,smooth] (-1,0)--plot[domain=-1:2](\x,{(\x)^3-2*(\x)^2-(\x)+2})--(2,0)--cycle;
			\draw[fill=black](-1,0) circle (.04)+(145:.35) node{$-1$};
			\draw[fill=black](1,0) circle (.04)+(-125:.3) node{$1$};
			\draw[fill=black](2,0) circle (.04)+(-45:.3) node{$2$};
		\end{tikzpicture}
	}
	\loigiai{
		Diện tích hình phẳng giới hạn bởi các đường $y=f(x),y=0,x=-1,x=2$ là
		$$S=\displaystyle\int \limits_{-1}^1 f(x) \mathrm{\,d}x-\displaystyle\int \limits_1^2 f(x) \mathrm{\,d}x.$$
	}
\end{ex}
\begin{ex}%[Dự án 2025 - đề cấu trúc mới, Hung Doan]%[2D4N3-1]
	Diện tích hình phẳng giới hạn bởi đồ thị của hai hàm số $y=x^3+2x+1$, $y=x^3+x+3$ và hai đường thẳng $x=1$, $x=3$ được xác định bởi công thức
	\choice
	{$S=\displaystyle\int \limits_1^3 (2x^3+3x+4)\mathrm{\,d}x$}
	{$S=\displaystyle\int \limits_1^3 (x-2)\mathrm{\,d}x$}
	{\True $S=\displaystyle\int \limits_1^3 |x-2|\mathrm{\,d}x$}
	{$S=\displaystyle\int \limits_1^3 |2x^3+3x+4|\mathrm{\,d}x$}
	\loigiai{
		Diện tích hình phẳng giới hạn bởi $y=x^3+2x+1$, $y=x^3+x+3$, $x=1$, $x=3$ là
		$$S=\displaystyle\int \limits_1^3 \left|x^3+2x+1-(x^3+x+3)\right|\mathrm{\,d}x=\displaystyle\int \limits_1^3 |x-2|\mathrm{\,d}x.$$
	}
\end{ex}
\begin{ex}%[Dự án 2025 - đề cấu trúc mới, Hung Doan]%[2D4H3-1]
	Diện tích hình phẳng giới hạn bởi hai đường $y=x^2+2x$ và $y=-x+4$ bằng
	\choice
	{$\dfrac{13}{2}$}
	{$\dfrac{63}{2}$}
	{$\dfrac{205}{6}$}
	{\True $\dfrac{125}{6}$}
	\loigiai{
		\begin{itemize}
			\item Phương trình hoành độ giao điểm của hai đồ thị hàm số $y=x^2+2x$ và $y=-x+4$ là
			$$x^2+2x=-x+4\Leftrightarrow x^2+3x-4=0\Leftrightarrow \hoac{&x=1\\&x=-4.}$$
			\item Diện tích hình phẳng cần tìm là
			\begin{align*}
				S&=\displaystyle\int \limits_{-4}^1 |x^2+2x-(-x+4)|\mathrm{d}x\\
				&=\displaystyle\int \limits_{-4}^1 |x^2+3x-4|\mathrm{d}x\\
				&=\displaystyle\int \limits_{-4}^1 |x^2+3x-4|\mathrm{d}x\\
				&=\displaystyle\int \limits_{-4}^1 (4-3x-x^2)\mathrm{d}x\\
				&=\left(4x-\dfrac{3}{2}x^2-\dfrac{1}{3}x^3\right)\bigg|^1_{-4}=\dfrac{125}{6}.
			\end{align*}
		\end{itemize}
	}
\end{ex}
\begin{ex}%[Dự án 2025 - đề cấu trúc mới, Hung Doan]%[2D4H3-1]
	Diện tích hình phẳng được giới hạn bởi các đường $y=x^2+x-1$ và $y=x^4+x-1$ là
	\choice
	{$\dfrac{8}{15}$}
	{$\dfrac{7}{15}$}
	{$\dfrac{2}{5}$}
	{\True $\dfrac{4}{15}$}
	\loigiai{
		\begin{itemize}
			\item Phương trình hoành độ giao điểm của $y=x^2+x-1$ và $y=x^4+x-1$ là
			$$x^2+x-1=x^4+x-1 \Leftrightarrow x^2-x^4=0\Leftrightarrow \hoac{&x=0\\&x=1\\&x=-1.}$$
			\item Diện tích hình phẳng cần tìm là
			\begin{align*}
				S&=\displaystyle\int \limits_{-1}^1 |x^2-x^4|\mathrm{d}x\\
				&=\displaystyle\int \limits_{-1}^0 |x^2-x^4|\mathrm{d}x+\displaystyle\int \limits_0^1 |x^2-x^4|\mathrm{d}x\\
				&=\left|\displaystyle\int \limits_{-1}^0 (x^2-x^4)\mathrm{d}x\right|+\left|\displaystyle\int \limits_0^1 (x^2-x^4)\mathrm{d}x\right|\\
				&=\left|\left(\dfrac{x^3}{3}-\dfrac{x^5}{5}\right)\bigg|^0_{-1}\right|+\left|\left(\dfrac{x^3}{3}-\dfrac{x^5}{5}\right)\bigg|^1_0 \right|\\
				&=\dfrac{2}{15}+\dfrac{2}{15}=\dfrac{4}{15}.
			\end{align*}
	\end{itemize}}
\end{ex}
\begin{ex}%[Dự án 2025 - đề cấu trúc mới, Hung Doan]%[2D4H3-3]
	Tính thể tích khối tròn xoay được tạo bởi hình phẳng giới hạn bởi đồ thị hàm số $y=3x-x^2$ và trục hoành khi quay quanh trục hoành.
	\choice
	{$\dfrac{85\pi}{7}$}
	{$\dfrac{8\pi}{7}$}
	{\True $\dfrac{81\pi}{10}$}
	{$\dfrac{41\pi}{7}$}
	\loigiai{
		Phương trình hoành độ giao điểm của đồ thị hàm số $y=3x-x^2$ và trục hoành là $$3x-x^2=0\Leftrightarrow \hoac{&x=0\\&x=3.}$$
		Thể tích của khối tròn xoay là $V=\pi \displaystyle\int \limits_0^3 (3x-x^2)^2\mathrm{\,d}x=\dfrac{81\pi}{10}$.}
\end{ex}
\begin{ex}%[Dự án 2025 - đề cấu trúc mới, Hung Doan]%[2D4H3-1]
	Giá trị dương của tham số $m$ sao cho diện tích hình phẳng giới hạn bởi đồ thị của hàm số $y=2x+3$ và các đường thẳng $y=0$, $x=0$, $x=m$ bằng $10$ là
	\choice
	{$m=\dfrac{7}{2}$}
	{$m=5$}
	{\True $m=2$}
	{$m=1$}
	\loigiai{
		Vì $m>0$ nên $2x+3>0$, $\forall x\in [0;m]$.\\
		Diện tích hình phẳng giới hạn bởi đồ thị hàm số $y=2x+3$ và các đường thẳng $y=0$, $x=0$, $x=m$ là
		$$S=\displaystyle\int \limits_0^m (2x+3)\mathrm{d}x=(x^2+3x)\bigg|_0^m=m^2+3m.$$
		Theo giả thiết ta có
		\begin{align*}
			S=10&\Leftrightarrow m^2+3m=10\\
			&\Leftrightarrow m^2+3m-10=0\\
			&\Leftrightarrow \hoac{&m=2\\&m=-5}\\
			&\Leftrightarrow m=2\, (\text{do} m>0).
		\end{align*}
	}
\end{ex}
\Closesolutionfile{ans}
% \indapan{6}{ans/ans-2-C4B13-D1}
\TNTF
\Opensolutionfile{ans}[ans/ans-2-C4B13-D1-DS]
\begin{ex}%[Dự án 2025 - đề cấu trúc mới, Hung Doan]%[2D4H3-1]
	\immini[thm]{Cho đồ thị hàm số $y=f(x)$, và hình phẳng $(H)$ được gạch chéo như hình vẽ. Đặt $a=\displaystyle\int \limits_{-1}^0 f(x)\mathrm{d}x $, $b=\displaystyle\int \limits_0^2 f(x)\mathrm{d}x$.
		\choiceTF
		{Hình phẳng $(H)$ được giới hạn bởi các đường $x=-1$, $x=2$, $y=f(x)$}
		{Hình phẳng $(H)$ có diện tích $S=\left|\displaystyle\int \limits_{-1}^2 f(x)\mathrm{d}x\right|$}
		{\True Hình phẳng $(H)$ có diện tích $S=b-a$}
		{\True $\displaystyle\int \limits_{-1}^2 f(x)\mathrm{d}x>0$}}{
		\begin{tikzpicture}[scale=1, font=\footnotesize, line join=round, line cap=round, >=stealth]
			\def\xt{-2} \def\xp{3} \def\yt{5} \def\yd{-1.5}
			\draw[->, line width=0.8pt](\xt, 0)--(\xp, 0) node[below]{$x$};
			\draw[->, line width=0.8pt](0,\yd)--(0,\yt) node[left]{$y$};
			\node at (0, 0) [below left]{$O$};
			\clip(\xt+0.1,\yd+0.1) rectangle(\xp-0.1,\yt-0.1);
			\draw[smooth,samples=300] plot(\x,{0.4*(\x)^3});
			\fill[pattern=north east lines,smooth] (-1,0)--plot[domain=-1:2](\x,{0.4*(\x)^3})--(2,0)--cycle;
			\draw[fill=black](-1,0) circle (.04)+(145:.35) node{$-1$};
			\draw[fill=black](2,0) circle (.04)+(-90:.3) node{$2$};
			\draw (-1,-0.4)--(-1,0) (2,0)--(2,3.2);
			\node at (2, 3.5) [left]{$f(x)$};
		\end{tikzpicture}
	}
	\loigiai{
		\begin{itemchoice}
			\itemch \textbf{Sai}.\\
			Ta có hình phẳng $(H)$ được giới hạn bởi các đường $x=-1$, $x=2$, $y=f(x)$ và trục $Ox$.
			\itemch \textbf{Sai}.\\
			Hình phẳng $(H)$ có diện tích $S=\displaystyle\int \limits_{-1}^2 \left|f(x)\right|\mathrm{d}x$.
			\itemch \textbf{Đúng}.\\
			Hình phẳng $(H)$ có diện tích $S=b-a$.
			\itemch \textbf{Đúng}.\\
			Ta có $b>a$ nên $S>0$ hay $\displaystyle\int \limits_{-1}^2 f(x)\mathrm{d}x>0$.
		\end{itemchoice}
	}
\end{ex}
\begin{ex}%[Dự án 2025 - đề cấu trúc mới, Hung Doan]%[2D4V3-3]
	\immini{Cho đồ thị của hai hàm số $y=f(x)$, $y=g(x)$ và phần tô màu như hình vẽ.
		\choiceTF
		{Phần hình phẳng tô màu được giới hạn bởi các đường $y=f(x)$, $y=g(x)$, $x=-3$, $x=3$}
		{\True Hình phẳng giới hạn bởi $y=f(x)$, trục $Ox$ có diện tích $S_1=\dfrac{32}{3}$}
		{\True Phần hình phẳng tô màu có diện tích $S_2=\dfrac{9}{2}$}
		{Quay hình phẳng tô màu quanh trục $Ox$ ta được khối tròn xoay có thể tích $V=\dfrac{9}{2}\pi $}}{
		\begin{tikzpicture}[scale=1, font=\footnotesize, line join=round, line cap=round, >=stealth]
			\def\xt{-4} \def\xp{2} \def\yt{4.5} \def\yd{-1}
			\draw[->, line width=0.8pt](\xt, 0)--(\xp, 0) node[below]{$x$};
			\draw[->, line width=0.8pt](0,\yd)--(0,\yt) node[left]{$y$};
			\node at (0, 0) [below left]{$O$};
			\clip(\xt+0.1,\yd+0.1) rectangle(\xp-0.1,\yt-0.1);
			\draw[smooth,samples=300] plot(\x,{-(\x)^2-2*(\x)+3});
			\draw[smooth,samples=300] plot(\x,{(\x)+3});
			\foreach \x/\g in{-3/135,1/45}\draw[fill=black](\x,0) circle (.04)+(\g:.35) node{$\x$};
			\draw[fill=black](0,3) circle (.04)+(0:.35) node{$3$};
			\fill[gray,smooth] (-3,0)--plot[domain=-3:0](\x,{-(\x)^2-2*(\x)+3})--plot[domain=-3:0](\x,{(\x)+3})--cycle;
		\end{tikzpicture}
	}
	\loigiai{
		\begin{itemchoice}
			\itemch \textbf{Sai}.\\
			Phần hình phẳng tô màu được giới hạn bởi các đường $y=f(x)$, $y=g(x)$, $x=-3$, $x=0$.
			\itemch \textbf{Đúng}.\\
			Từ đồ thị suy ra $f(x)=-x^2-2x+3$, $g(x)=x+3$.\\
			Suy ra diện tích $S_1=\displaystyle\int \limits_{-3}^1 \left(-x^2-2x+3\right)\mathrm{d}x=\dfrac{32}{3}$.
			\itemch \textbf{Đúng}.\\
			Phần hình phẳng tô màu có diện tích $S_2=\displaystyle\int \limits_{-3}^0 \left(f(x)-g(x)\right)\mathrm{d}x=\dfrac{9}{2}$.
			\itemch \textbf{Sai}.\\
			Quay hình phẳng tô màu quanh trục $Ox$ ta được khối tròn xoay có thể tích
			$$\begin{aligned}
				V&=\pi \displaystyle\int \limits_{-3}^0 \left[(-x^2-2x+3)^2-(x+3)^2\right]\mathrm{d}x\\
				&=\pi \displaystyle\int \limits_{-3}^0 \left(x^4+4x^3-3x^2-18x\right)\mathrm{d}x\\
				&=\pi \left(\dfrac{x^5}{5}+x^4-x^3-9x^2\right)\bigg|_{-3}^0=\dfrac{108\pi }{5}.
			\end{aligned}$$
		\end{itemchoice}
	}
\end{ex}
\begin{ex}%[Dự án 2025 - đề cấu trúc mới, Hung Doan]%[2D4V3-1]
	Cho hàm số $y=f(x)$ có đồ thị $y=f'(x)$ cắt trục $Ox$ tại ba điểm có hoành độ $a<b<c$ như hình vẽ bên dưới.
	\begin{center}
		\begin{tikzpicture}[scale=1, font=\footnotesize, line join=round, line cap=round, >=stealth]
			\def\xt{-0.5} \def\xp{5} \def\yt{2} \def\yd{-2.5}
			\draw[->](\xt, 0)--(\xp, 0) node[below]{$x$};
			\draw[->](0,\yd)--(0,\yt) node[left]{$y$};
			\node at (0, 0) [below left]{$O$};
			\clip(\xt+0.1,\yd+0.1) rectangle(\xp-0.1,\yt-0.1);
			\draw[smooth,samples=300] plot(\x,{((\x)-1)*((\x)-2)*((\x)-4)});
			\fill (1,0) node[above left]{$a$} circle (1pt);
			\fill (2,0) node[above right]{$b$} circle (1pt);
			\fill (4,0) node[above left]{$c$} circle (1pt);
			\fill[pattern=north east lines,smooth] (1,0)--plot[domain=1:4](\x,{((\x)-1)*((\x)-2)*((\x)-4)})--(4,0)--cycle;
		\end{tikzpicture}
	\end{center}
	\choiceTF
	{\True Hình phẳng gạch sọc được giới hạn bởi các đường $y=f'(x)$ và trục $Ox$}
	{Diện tích hình phẳng gạch sọc $S=\displaystyle\int \limits_a^b f(x)\mathrm{d}x-\displaystyle\int \limits_b^c f(x)\mathrm{d}x$}
	{$\displaystyle\int \limits_a^b f'(x)\mathrm{d}x<\displaystyle\int \limits_b^c f'(x)\mathrm{d}x$}
	{\True $f(b)>f(a)>f(c)$}
	\loigiai{
		\begin{itemchoice}
			\itemch \textbf{Đúng}.\\
			Hình phẳng gạch sọc được giới hạn bởi các đường $y=f'(x)$ và trục $Ox$.
			\itemch \textbf{Sai}.\\
			Diện tích hình phẳng gạch sọc phải là $S=\displaystyle\int \limits_a^b f'(x)\mathrm{d}x-\displaystyle\int \limits_b^c f'(x)\mathrm{d}x$.
			\itemch \textbf{Sai}.
			\begin{center}
				\begin{tikzpicture}[scale=1, font=\footnotesize, line join=round, line cap=round, >=stealth]
					\def\xt{-0.5} \def\xp{5} \def\yt{2} \def\yd{-2.5}
					\draw[->](\xt, 0)--(\xp, 0) node[below]{$x$};
					\draw[->](0,\yd)--(0,\yt) node[left]{$y$};
					\node at (0, 0) [below left]{$O$};
					\clip(\xt+0.1,\yd+0.1) rectangle(\xp-0.1,\yt-0.1);
					\draw[smooth,samples=300] plot(\x,{((\x)-1)*((\x)-2)*((\x)-4)});
					\fill (1,0) node[above left]{$a$} circle (1pt);
					\fill (2,0) node[above right]{$b$} circle (1pt);
					\fill (4,0) node[above left]{$c$} circle (1pt);
					\fill[pattern=north east lines,smooth] (1,0)--plot[domain=1:4](\x,{((\x)-1)*((\x)-2)*((\x)-4)})--(4,0)--cycle;
					\node at (1.5, -0.1) [above]{$S_1$};
					\node at (3, -0.5) [below]{$S_2$};
				\end{tikzpicture}
			\end{center}
			Ta có diện tích $S_1=\displaystyle\int \limits_a^b f'(x)\mathrm{d}x$, diện tích $S_2=-\displaystyle\int \limits_b^c f'(x)\mathrm{d}x$.\\
			Từ hình vẽ ta có $S_1<S_2 \Leftrightarrow \displaystyle\int \limits_a^b f'(x)\mathrm{d}x<-\displaystyle\int \limits_b^c f'(x)\mathrm{d}x$.\\
			\itemch \textbf{Đúng}.\\
			Ta có $\displaystyle\int \limits_a^b f'(x)\mathrm{d}x<-\displaystyle\int \limits_b^c f'(x)\mathrm{d}x\Leftrightarrow f(b)-f(a)<f(b)-f(c) \Leftrightarrow f(a)>f(c)$.\\
			Mặt khác, từ đồ thị hàm $f'(x)$ ta có bảng biến thiên
			\begin{center}
				\begin{tikzpicture}
					\tkzTabInit[nocadre=true,lgt=1.5,espcl=3,deltacl=.55]
					{$x$/0.7, $f'(x)$/0.7, $f(x)$/2}
					{$-\infty$,$a$,$b$,$c$,$+\infty$}
					\tkzTabLine{,-,$0$,+,$0$,-,$0$,+,}
					\tkzTabVar{+/$+\infty$,-/$f(a)$,+/$f(b)$,-/$f(c)$,+/$+\infty$}	
				\end{tikzpicture}
			\end{center}
			Suy ra $f(b)$ lớn hơn $f(a)$ và $f(c)$.\\
			Vậy $f(b)>f(a)>f(c)$.
		\end{itemchoice}
	}
\end{ex}
\begin{ex}%[Dự án 2025 - đề cấu trúc mới, Hung Doan]%[2D4H3-1]
	\immini[thm]{Cho hình vuông $ABCD$ tâm $O$, độ dài cạnh là $4$ cm. Đường cong $BOC$ là một phần của parabol đỉnh $O$ chia hình vuông thành hai hình phẳng có diện tích lần lượt là $S_1$ và $S_2$ (tham khảo hình vẽ).
		\choiceTF
		{Diện tích hình phẳng $S_1=4$}
		{Diện tích hình phẳng $S_2=12$}
		{\True $S_2=2S_1$}
		{$S_2=3S_1$}}{
		\begin{tikzpicture}[scale=1, font=\footnotesize, line join=round, line cap=round, >=stealth]
			\node at (0, 0) [below]{$O$};
			\node at (0, 2.2) {$4$\,cm};
			\node at (2.4, 0) {$4$\,cm};
			\node at (0,1) {$S_1$};
			\node at (0,-1) {$S_2$};
			\fill (-2,-2) node[below left]{$A$};
			\fill (-2,2) node[above left]{$B$};
			\fill (2,2) node[above right]{$C$};
			\fill (2,-2) node[below right]{$D$};
			\clip(-2.1,-2.1) rectangle(2.1,2);
			\draw[smooth,samples=300] plot(\x,{0.5*(\x)^2});
			\draw (2,2)--(2,-2)--(-2,-2)--(-2,2)--(2,2);
		\end{tikzpicture}
		
	}
	\loigiai{
		Gắn hệ trục toạ độ như hình vẽ
		\begin{center}
			\begin{tikzpicture}[scale=1, font=\footnotesize, line join=round, line cap=round, >=stealth]
				\draw[->](-3, 0)--(3, 0) node[below]{$x$};
				\draw[->](0,-3)--(0,3) node[left]{$y$};
				\node at (0, 0) [below left]{$O$};
				\fill (-2,-2) node[below left]{$A$};
				\fill (-2,2) node[above left]{$B$};
				\fill (2,2) node[above right]{$C$};
				\fill (2,-2) node[below right]{$D$};
				\fill (-2,0) node[below left]{$-2$} circle (1pt);
				\fill (0,2) node[above left]{$2$};
				\fill (2,0) node[above right]{$2$};
				\fill (0,-2) node[below right]{$-2$};
				\clip(-2.1,-2.1) rectangle(2.1,2);
				\draw[smooth,samples=300] plot(\x,{0.5*(\x)^2});
				\draw (2,2)--(2,-2)--(-2,-2)--(-2,2)--(2,2);
			\end{tikzpicture}
		\end{center}
		Ta có phương trình parabol $(P)\colon y=\dfrac{1}{2}x^2$.\\
		Suy ra $S_1=\displaystyle\int \limits_{-2}^2 \left(2-\dfrac{1}{2}x^2\right)\mathrm{\,d}x=\dfrac{16}{3}$ (đvdt).\\
		Diện tích hình vuông $ABCD$ là $S_{ABCD}=4^2=16$ (đvdt).\\
		Do đó diện tích $S_2$ là $S_2=S_{ABCD}-S_1=16-\dfrac{16}{3}=\dfrac{32}{3}$ (đvdt).\\
		Vậy tỉ số $\dfrac{S_1}{S_2}=\dfrac{16}{3}\colon \dfrac{32}{3}=\dfrac{1}{2}\Rightarrow S_2=2S_1$.\\
		Khi đó, ta có
		\begin{itemchoice}
			\itemch \textbf{Sai}.
			\itemch \textbf{Sai}.
			\itemch \textbf{Đúng}.
			\itemch \textbf{Sai}.
		\end{itemchoice}
	}
\end{ex}
\Closesolutionfile{ans}
% \indapan{2}{ans/ans-2-C4B13-D1-DS}
\Opensolutionfile{ans}[ans/ans-2-C4B13-D1-KQ]
\TNSA
\begin{ex}%[Dự án 2025 - đề cấu trúc mới, Hung Doan]%[2D4H3-1]
	Tính diện tích hình phẳng giới hạn bởi các đường $y=x^2$, $y=-\dfrac{1}{3}x+\dfrac{4}{3}$ và trục hoành (làm tròn kết quả đến hàng phần trăm).
	\shortans{$1{,}83$}
	\loigiai{
		\begin{center}
			\begin{tikzpicture}[scale=1, font=\footnotesize, line join=round, line cap=round, >=stealth]
				\def\xt{-2} \def\xp{5} \def\yt{3} \def\yd{-1}
				\draw[->, line width=0.8pt](\xt, 0)--(\xp, 0) node[below]{$x$};
				\draw[->, line width=0.8pt](0,\yd)--(0,\yt) node[left]{$y$};
				\node at (0, 0) [below left]{$O$};
				\clip(\xt+0.1,\yd+0.1) rectangle(\xp-0.1,\yt-0.1);
				\draw[ smooth,samples=300] plot(\x,{(\x)^2});
				\draw[ smooth,samples=300] plot(\x,{-1/3*(\x)+4/3});
				\foreach \x in {1,4}
				\draw[shift ={ (\x,0)}]node[below]{$\x$} (0pt,2pt) --(0pt,-2pt);
				\fill[pattern=north east lines,smooth] (0,0)--plot[domain=0:1](\x,{(\x)^2})--(1,0)--cycle;
				\fill[pattern=north east lines,smooth] (1,0)--plot[domain=1:4](\x,{-1/3*(\x)+4/3})--(4,0)--cycle;
				\draw[dashed] (1,0)--(1,1);
			\end{tikzpicture}
		\end{center}
		Phương trình hoành độ giao điểm của các đường là
		\begin{itemize}
			\item $x^2=0\Leftrightarrow x=0$.
			\item $-\dfrac{1}{3}x+\dfrac{4}{3}\Leftrightarrow x=4$.
			\item $x^2=-\dfrac{1}{3}x+\dfrac{4}{3} \Leftrightarrow 3x^2+x-4=0 \Leftrightarrow \hoac{&x=1\\&x=-\dfrac{4}{3}.}$
		\end{itemize}
		Diện tích hình phẳng cần tìm là
		$$S=\displaystyle\int \limits_0^1 x^2\mathrm{d}x+\displaystyle\int \limits_1^4 \left(-\dfrac{1}{3}x+\dfrac{4}{3}\right)\mathrm{d}x =\dfrac{x^3}{3}\bigg|_0^1+\left(-\dfrac{1}{6}x^2+\dfrac{4}{3}x\right)\bigg|_1^4 =\dfrac{11}{6}\approx 1{,}83.$$
	}
\end{ex}
\begin{ex}%[Dự án 2025 - đề cấu trúc mới, Hung Doan]%[2D4V3-1]
	Cho hàm số $y=f(x)$. Hàm số có đồ thị hàm số $y=f'(x)$ như hình vẽ dưới đây.
	\begin{center}
		\begin{tikzpicture}[scale=1, font=\footnotesize, line join=round, line cap=round, >=stealth]
			\def\xt{-2.75} \def\xp{4.75} \def\yt{2.5} \def\yd{-3}
			\draw[->](\xt, 0)--(\xp, 0) node[below]{$x$};
			\draw[->](0,\yd)--(0,\yt) node[left]{$y$};
			\node at (0, 0) [below left]{$O$};
			\node at (-1.5, 1.4) {$y=f'(x)$};
			\foreach \x in {1,4}
			\draw[shift ={ (\x,0)}]node[above]{$\x$} (0pt,2pt) --(0pt,-2pt);
			\draw[shift ={ (-2,0)}]node[above right]{$-2$} (0pt,2pt) --(0pt,-2pt);
			\clip(\xt+0.1,\yd+0.1) rectangle(4,\yt-0.1);
			\draw plot[smooth,tension=0.7] coordinates{(-2.4,2) (-1.2,-1.8) (1,0) (3,-2.5) (4,0)};			
		\end{tikzpicture}
	\end{center}
	Biết diện tích hình phẳng giới hạn bởi trục $Ox$ và đồ thị hàm số $y=f'(x)$ trên đoạn $[-2;1]$ và $[1;4]$ lần lượt bằng $9$ và $12$. Cho biết $f(1)=3$. Tính giá trị biểu thức $P=f(-2)+f(4)$.
	\shortans{$3$}
	\loigiai{
		Ta có $\displaystyle\int \limits_{-2}^1 |f'(x)|\mathrm{d}x=9\Leftrightarrow \displaystyle\int \limits_{-2}^1 f'(x)\mathrm{d}x=-9\Rightarrow f(1)-f(-2)=-9$.\\
		Mà $f(1)=3 \Rightarrow f(-2)=12$.\\
		Ta có $\displaystyle\int \limits_1^4 |f'(x)|\mathrm{d}x=12\Leftrightarrow \displaystyle\int \limits_1^4 f'(x)\mathrm{d}x=-12\Rightarrow f(4)-f(1)=-12$.\\
		Mà $f(1)=3 \Rightarrow f(4)=-9$.\\
		Vậy $P=f(-2)+f(4)=3$.
	}
\end{ex}
\begin{ex}%[Dự án 2025 - đề cấu trúc mới, Hung Doan]%[2D4C3-1]
	Cho hàm số $f(x)=x^3+ax^2+bx+c$ với $a$, $b$, $c$ là các số thực. Biết hàm số $g(x)=f(x)+f'(x)+f''(x)$ có hai giá trị cực trị là $5$ và $2$. Tính diện tích hình phẳng giới hạn bởi đường $y=\dfrac{f(x)}{g(x)+6}$ và $y=1$, kết quả làm tròn đến hàng phần trăm.
	\shortans{$2{,}08$}
	\loigiai{
		Ta có $f'''(x)=6$, khi đó $g'(x)=f'(x)+f''(x)+f'''(x)=f'(x)+f''(x)+6$.\\
		Giả sử $x_1$, $x_2$ ($x_1<x_2$) là hai điểm cực trị của hàm số $g(x)$.\\
		Vì $\lim \limits_{x\to +\infty } g(x)=+\infty $ và $-5$ và $2$ là hai giá trị cực trị của hàm số $g(x)$ nên $\heva{&g(x_1)=2\\&g(x_2)=-5.}$\\
		Phương trình hoành độ giao điểm của $y=\dfrac{f(x)}{g(x)+6}$ và $y=1$ là
		\begin{align*}
			\dfrac{f(x)}{g(x)+6}=1 &\Leftrightarrow g(x)+6=f(x)\\
			&\Leftrightarrow f(x)+f'(x)+f''(x)+6=f(x)\\
			&\Leftrightarrow f'(x)+f''(x)+6=0\\
			&\Leftrightarrow \hoac{&x=x_1\\&x=x_2.} 
		\end{align*}
		Khi đó diện tích hình phẳng cần tìm là
		\begin{align*}
			S&=\displaystyle\int \limits_{x_1}^{x_2} \left|\dfrac{f(x)}{g(x)+6}-1\right|\mathrm{d}x\\
			&=\left|\displaystyle\int \limits_{x_1}^{x_2} \dfrac{f'(x)+f''(x)+6}{g(x)+6}\mathrm{d}x\right|\\
			&=\left|\displaystyle\int \limits_{x_1}^{x_2} \dfrac{g'(x)}{g(x)+6}\mathrm{d}x\right|\\
			&=\left|\ln |g(x)+6|\bigg|_{x_1}^{x_2}\right| \\
			&=\left|\ln |g(x_2)+6|-\ln |g(x_1)+6|\right|\\
			&=\ln 8\approx 2{,}08.
		\end{align*}
	}
\end{ex}
\begin{ex}%[Dự án 2025 - đề cấu trúc mới, Hung Doan]%[2D4V3-2]
	Một viên gạch hoa hình vuông cạnh $40$ cm. Người thiết kế đã sử dụng bốn đường parabol có chung đỉnh tại tâm viên gạch để tạo ra bốn cánh hoa (được tô màu sẫm như hình vẽ bên).
	\begin{center}
		\begin{tikzpicture}[scale=1, font=\footnotesize, line join=round, line cap=round, >=stealth]
			\draw (2,2)--(2,-2)--(-2,-2)--(-2,2)--(2,2);
			\clip(-2,-2) rectangle(2,2);
			\draw[smooth,samples=300] plot(\x,{0.5*(\x)^2});
			\draw[smooth,samples=300] plot(\x,{-0.5*(\x)^2});
			\draw[samples=200,domain=0:2,smooth] plot (\x,{sqrt(2*(\x))});
			\draw[samples=200,domain=0:2,smooth] plot (\x,{-sqrt(2*(\x))});
			\draw[samples=200,domain=-2:0,smooth] plot (\x,{sqrt(-2*(\x))});
			\draw[samples=200,domain=-2:0,smooth] plot (\x,{-sqrt(-2*(\x))});
			\fill[gray,smooth] (-2,0)--plot[domain=-2:0](\x,{-sqrt(-2*(\x))})--(0,0)--cycle;
			\fill[gray,smooth] (-2,0)--plot[domain=-2:0](\x,{sqrt(-2*(\x))})--(0,0)--cycle;
			\fill[gray,smooth] (0,0)--plot[domain=0:2](\x,{-sqrt(2*(\x))})--(2,0)--cycle;
			\fill[gray,smooth] (0,0)--plot[domain=0:2](\x,{sqrt(2*(\x))})--(2,0)--cycle;
			\fill[white,smooth] (-2,0)--plot[domain=-2:2](\x,{-0.5*(\x)^2})--(2,0)--cycle;
			\fill[white,smooth] (-2,0.01)--plot[domain=-2:2](\x,{0.5*(\x)^2})--(2,0.01)--cycle;
		\end{tikzpicture}
	\end{center}
	Diện tích mỗi cánh hoa của viên gạch bằng bằng $\dfrac{a}{b}$ (cm$^2$), với $\dfrac{a}{b}$ là phân số tối giản thì $a$ bằng bao nhiêu?
	\shortans{$400$}
	\loigiai{
		\begin{center}
			\begin{tikzpicture}[scale=1, font=\footnotesize, line join=round, line cap=round, >=stealth]
				\begin{scope}
					\clip(-2,-2) rectangle(2,2);
					\draw[smooth,samples=300] plot(\x,{0.5*(\x)^2});
					\draw[smooth,samples=300] plot(\x,{-0.5*(\x)^2});
					\draw[samples=200,domain=0:2,smooth] plot (\x,{sqrt(2*(\x))});
					\draw[samples=200,domain=0:2,smooth] plot (\x,{-sqrt(2*(\x))});
					\draw[samples=200,domain=-2:0,smooth] plot (\x,{sqrt(-2*(\x))});
					\draw[samples=200,domain=-2:0,smooth] plot (\x,{-sqrt(-2*(\x))});
					\fill[gray,smooth] (-2,0)--plot[domain=-2:0](\x,{-sqrt(-2*(\x))})--(0,0)--cycle;
					\fill[gray,smooth] (-2,0)--plot[domain=-2:0](\x,{sqrt(-2*(\x))})--(0,0)--cycle;
					\fill[gray,smooth] (0,0)--plot[domain=0:2](\x,{-sqrt(2*(\x))})--(2,0)--cycle;
					\fill[gray,smooth] (0,0)--plot[domain=0:2](\x,{sqrt(2*(\x))})--(2,0)--cycle;
					\fill[white,smooth] (-2,0)--plot[domain=-2:2](\x,{-0.5*(\x)^2})--(2,0)--cycle;
					\fill[white,smooth] (-2,0.01)--plot[domain=-2:2](\x,{0.5*(\x)^2})--(2,0.01)--cycle;
				\end{scope}
				\draw[->](-2.75, 0)--(3, 0) node[below]{$x$};
				\draw[->](0,-2.5)--(0,2.75) node[left]{$y$};
				\node at (0, 0) [below left]{$O$};
				\draw[shift ={ (-2,0)}]node[below left]{$-2$} (0pt,2pt) --(0pt,-2pt);
				\draw[shift ={ (2,0)}]node[below right]{$2$} (0pt,2pt) --(0pt,-2pt);
				\draw[shift ={ (0,2)}]node[above right]{$2$} (2pt,0pt) --(-2pt,0pt);
				\draw[shift ={ (0,-2)}]node[below left]{$-2$} (2pt,0pt) --(-2pt,0pt);
				\draw[smooth] (2,2)--(2,-2)--(-2,-2)--(-2,2)--(2,2);
			\end{tikzpicture}
		\end{center}
		Chọn hệ tọa độ như hình vẽ ($1$ đơn vị trên trục bằng $10$ cm=$1$ dm), các cánh hoa tạo bởi các đường parabol có phương trình $y=\dfrac{x^2}{2}$, $y=-\dfrac{x^2}{2}$, $x=-\dfrac{y^2}{2}$, $x=\dfrac{y^2}{2}$.\\
		Diện tích một cánh hoa (nằm trong góc phần tư thứ nhất) bằng diện tích hình phẳng giới hạn bởi hai đồ thị hàm số $y=\dfrac{x^2}{2}$, $y=\sqrt{2x}$ và hai đường thẳng $x=0$; $x=2$.\\
		Do đó diện tích một cánh hoa bằng
		\begin{align*}
			\displaystyle\int \limits_0^2 \left(\sqrt{2x}-\dfrac{x^2}{2}\right)\mathrm{d}x &=\left(\dfrac{2\sqrt{2}}{3}\sqrt{(2x)^3}-\dfrac{x^3}{6}\right)\bigg|\bigg|_0^2\\
			&=\dfrac{4}{3}(\text{dm}^2)=\dfrac{400}{3}\,(\text{cm}^2).
		\end{align*}
		Suy ra $a=400$.
	}
\end{ex}
\begin{ex}%[Dự án 2025 - đề cấu trúc mới, Hung Doan]%[2D4C3-2]
	\immini[thm]{Một bức tường lớn kích thức $8$m $\times$ $8$m trước đại sảnh của một tòa biệt thự được sơn các loại sơn đặc biệt. Người ta vẽ hai nửa đường tròn đường kính $AD$, $AB$ cắt nhau tại $H$; đường tròn tâm $D$, bán kính $AD$, cắt nửa đường tròn đường kính $AB$ tại $K$. Biết tam giác cong $AHK$ được sơn màu xanh và các phần còn lại được sơn màu trắng (như hình vẽ) và một mét vuông sơn trắng, sơn xanh lần lượt có giá là $1$ triệu đồng và $1{,}5$ triệu đồng. Số tiền phải trả là bao nhiêu triệu đồng? (làm tròn đến hàng triệu).}{
		\begin{tikzpicture}[scale=1, font=\footnotesize, line join=round, line cap=round, >=stealth]
			\begin{scope}
				\clip (0,0) rectangle (4,4);
				\fill[pattern=north east lines,smooth] (0,0)--plot[domain=0:4](\x,{sqrt(16-(\x)^2)})--(4,0)--cycle;
				\fill[white,smooth] (0,0)--plot[domain=0:4](\x,{4-sqrt(4*(\x)-(\x)^2)})--(4,0)--cycle;
				\draw[fill=white,samples=200,domain=0:4,smooth] (0,2) circle (2);
				\draw[samples=200,domain=0:4,smooth,variable=\x] plot (\x,{sqrt(16-(\x)^2)});
				\draw[samples=200,domain=0:4,smooth,variable=\x] plot (\x,{4-sqrt(4*(\x)-(\x)^2)});
				%\draw[samples=200,domain=0:4,smooth] (0,2) circle (2);
			\end{scope}
			\draw (0,0)rectangle (4,4);
			\node at (0, 2) [left]{$8$};
			\node at (2,0) [below]{$8$};
			\fill (0,4) node[above left]{$A$} circle(1pt);
			\fill (0,0) node[below left]{$B$} circle(1pt);
			\fill (4,0) node[below right]{$C$} circle(1pt);
			\fill (4,4) node[above right]{$D$} circle(1pt);
			\fill (2,2) node[below left]{$H$} circle(1pt);
			\fill (3.2,2.4) node[below]{$K$} circle(1pt);
			\draw (0,0)rectangle (0.2,0.2);
			\draw (0,4)rectangle (0.2,3.8);
			\draw (4,0)rectangle (3.8,0.2);
			\draw (4,4)rectangle (3.8,3.8);
		\end{tikzpicture}
	}
	\shortans{$67$}
	\loigiai{
		Chọn hệ toạ độ ${Oxy}$ như hình vẽ sau
		\begin{center}
			\begin{tikzpicture}[scale=1, font=\footnotesize, line join=round, line cap=round, >=stealth]
				\begin{scope}
					\clip (0,0) rectangle (4,4);
					\fill[pattern=north east lines,smooth] (0,0)--plot[domain=0:4](\x,{sqrt(16-(\x)^2)})--(4,0)--cycle;
					\fill[white,smooth] (0,0)--plot[domain=0:4](\x,{4-sqrt(4*(\x)-(\x)^2)})--(4,0)--cycle;
					\draw[fill=white,samples=200,domain=0:4,smooth] (0,2) circle (2);
					\draw[samples=200,domain=0:4,smooth,variable=\x] plot (\x,{sqrt(16-(\x)^2)});
					\draw[samples=200,domain=0:4,smooth,variable=\x] plot (\x,{4-sqrt(4*(\x)-(\x)^2)});
					%\draw[samples=200,domain=0:4,smooth] (0,2) circle (2);
				\end{scope}
				\draw (0,0)rectangle (4,4);
				\node at (0, 2) [left]{$8$};
				\node at (2,0) [below]{$8$};
				\fill (0,4) node[above left]{$A$} circle(1pt);
				\fill (0,0) node[below left]{$B$} circle(1pt);
				\fill (4,0) node[below right]{$C$} circle(1pt);
				\fill (4,4) node[above right]{$D$} circle(1pt);
				\fill (2,2) node[below left]{$H$} circle(1pt);
				\fill (3.2,2.4) node[below]{$K$} circle(1pt);
				\draw (0,0)rectangle (0.2,0.2);
				\draw (0,4)rectangle (0.2,3.8);
				\draw (4,0)rectangle (3.8,0.2);
				\draw (4,4)rectangle (3.8,3.8);
				\draw[->](-0.5, 0)--(5, 0) node[below]{$x$};
				\draw[->](0,-0.5)--(0,5) node[left]{$y$};
				\node at (0, 0) [above left]{$O$};
				\draw[dashed] (2,2)--(2,3.4641)node[above]{$E$};
			\end{tikzpicture}
		\end{center}
		Dễ thấy cung $AB$ có phương trình $y=f(x)=8-\sqrt{16-(x-4)^2}$; cung $AH$ có phương trình $y=g(x)=4+\sqrt{16-x^2}$; cung $AC$ có phương trình $y=h(x)=\sqrt{64-x^2}$ và tọa độ các điểm $H(4;4)$ và $K\left(6{,}4;\dfrac{24}{5}\right)$.\\
		Diện tích tam giác $AHK$ là
		\begin{align*}
			S&=S_{AHE}+S_{HEX}\\
			&=\displaystyle\int \limits_0^4 (\sqrt{64-x^2}-4-\sqrt{16-x^2})\mathrm{d}x+\displaystyle\int \limits_4^{6\cdot 4} (\sqrt{64-x^2}-8+\sqrt{16-(x-4)^2})\mathrm{d}x\\
			&\approx 6,25\,5085\,231.
		\end{align*}
		Số tiền cần trả là $S\cdot 1{,}5+(8^2-S)\cdot 1=67{,}12\,754\,262$.\\
		Vậy số tiền cần trả là $67$ (triệu đồng).
	}
\end{ex}
\begin{ex}%[Dự án 2025 - đề cấu trúc mới, Hung Doan]%[2D4V3-5]
	\immini{Một cốc có hình dạng tròn xoay và kích thước như hình vẽ, thiết diện dọc của mặt bên trong cốc (bổ dọc cốc thành $2$ phần bằng nhau) là một đường Parabol. Tính thể tích tối đa mà cốc có thể chứa được (kết quả làm tròn đến chữ số hàng đơn vị).}{
		\definecolor{almond}{rgb}{0.94, 0.87, 0.8}%màu ly
		\definecolor{anti-flashwhite}{rgb}{0.95, 0.95, 0.96}%màu miệng ly
		\begin{tikzpicture}[line join=round, line cap=round,scale=0.5,transform shape,line width=.3mm]
			
			\tikzset{co_vat/.pic={
					\path 
					(1.85,4)coordinate (A)			
					(-1.85,4) coordinate (B)	
					(-3,3)coordinate (C)
					(-3,-1.5)coordinate (D)		
					
					;
					\draw (A)--(B) (C)--(D);
					\foreach\p in {A,B,C,D}
					{\draw[fill=black](\p) circle (1pt);}	
					\node at (-3.8,1) {$10$ cm};
					\node at (0,4.5) {$8$ cm};
					\draw (2,1)--(5,2.5) node[above] {Parabol};
					
					\draw[fill=almond!50!black] (1.75,-4.93)  arc (0:360:1.75 cm and .54cm);
					\draw[fill=almond] (1.75,-4.8)  arc (0:360:1.75 cm and .4cm);
					
					\draw[fill=anti-flashwhite] (1.85,3)  arc (0:360:1.85 cm and .3cm);
					\draw[fill=almond] (1.85,3)
					..controls +(-95:1.3) and +(30:1.7) ..(.4,-1.6) 
					..controls +(-150:.1) and +(90:2) ..(.2,-4.2) 
					..controls +(-150:.1) and +(-30:.1) ..(-.2,-4.2) 
					..controls +(90:2) and +(-30:.1) ..(-.4,-1.6) 
					..controls +(150:1.7) and +(-85:1.3) ..(-1.85,3)
					arc (-180:0:1.85 cm and .3cm);
					;
					\draw[fill=almond] (.2,-4.2) 
					..controls +(-150:.1) and +(-30:.1) ..(-.2,-4.2) 
					..controls +(180:.1) and +(60:.1) ..(-.4,-4.4) 
					..controls +(180:.3) and +(60:.1) ..(-.9,-4.8) 
					..controls +(-30:.4) and +(-150:.4) ..(.9,-4.8) 
					..controls +(120:.1) and +(0:.3) ..(.4,-4.4)
					..controls +(150:.1) and +(-30:.1) ..(.2,-4.2) 
					;
					
			}}
			
			\path
			(0,0)pic[scale=1]{co_vat};
		\end{tikzpicture}
	}
	\shortans{$251$}
	\loigiai{
		\immini{Parabol có phương trình $y=\dfrac{5}{8}x^2\Leftrightarrow x^2=\dfrac{8}{5}y$.\\
			Thể tích tối đa cốc $V=\pi \displaystyle\int \limits_0^{10} \left(\dfrac{8}{5}y\right)\cdot \mathrm{\,d}y\approx 251$.}{
			\begin{tikzpicture}[scale=0.7, font=\footnotesize, line join=round, line cap=round, >=stealth]
				\draw[->, line width=0.8pt](-3.5, 0)--(3.75, 0) node[below]{$x$};
				\draw[->, line width=0.8pt](0,-0.5)--(0,6.5) node[left]{$y$};
				\node at (0, 0) [below left]{$O$};
				\fill (3,0) node[below]{$4$} circle(1pt);
				\fill (-3,0) node[below]{$-4$} circle(1pt);
				\fill (0,5.6125) node[below left]{$10$} circle(1pt);
				\clip(-3,-0.1) rectangle(3,5.75);
				\draw[ smooth,samples=300] plot(\x,{5/8*(\x)^2});
				\draw[dashed] (-3,0)--(-3,5.6125)--(3,5.6125)--(3,0);
			\end{tikzpicture}
		}
	}
\end{ex}
\Closesolutionfile{ans}
% \indapan{6}{ans/ans-2-C4B13-D1-KQ}


% \begin{name}
	{NGUYÊN HÀM - TÍCH PHÂN}
	{KT ỨNG DỤNG NGUYÊN HÀM - TÍCH PHÂN}
	{\tentruong}
	{\thoigian}
\end{name}
\setcounter{ex}{0}\setcounter{bt}{0}
\Opensolutionfile{ans}[ans/ans-2-B13-De2-TN]
\TN
\begin{ex}%[Vovanle]%[2D4N3-1]
Diện tích hình phẳng giới hạn bởi đồ thị hàm số $y=\sin x$, trục hoành và hai đường thẳng $x=0$, $x=2\pi$ được xác định bởi công thức
	\choice
	{$S=\displaystyle\displaystyle\int\limits_0^{2\pi}\sin x\mathrm{\,d}x$}
	{$S=\pi\displaystyle\int\limits_0^{2\pi}\sin x\mathrm{\,d}x$}
	{$S=\pi\displaystyle\int\limits_0^{2\pi}\sin^2 x\mathrm{\,d}x$}
	{\True $S=\displaystyle\int\limits_0^{2\pi}\left| \sin x \right|\mathrm{\,d}x$}
	\loigiai{
Diện tích hình phẳng được tính theo công thức 
$$S=\displaystyle\int\limits_0^{2\pi}\left|\sin x\right|\mathrm{\,d}x.$$
}
\end{ex}
\begin{ex}%[Vovanle]%[2D4N3-1]
Diện tích hình phẳng giới hạn bởi parabol $y=x^2-4$, trục hoành và hai đường thẳng $x=0$, $x=3$ bằng
	\choice
	{\True $\dfrac{23}{3}$}
	{$S=3$}
	{$\dfrac{7}{3}$}
	{$\dfrac{16}{3}$}
	\loigiai{
Diện tích hình phẳng là $$S=\displaystyle\int\limits_0^{3}\left|x^2-4\right|\mathrm{\,d}x=\dfrac{23}{3}.$$
}
\end{ex}
\begin{ex}%[Vovanle]%[2D4N3-3]
Thể tích khối tròn xoay do hình phẳng giới hạn bởi các đường thẳng $y=\sqrt{x}$, trục $Ox$ và hai đường thẳng $x=1$ và $x=2$. Khi quay quanh trục hoành được tính theo công thức nào?
	\choice
	{\True $V=\pi\displaystyle\int\limits_1^2 x\mathrm{\,d}x$}
	{$V=\pi \displaystyle\int\limits_1^2 \sqrt{x}\mathrm{\,d}x$}
	{$V=\pi^2\displaystyle\int\limits_1^2 x\mathrm{\,d}x$}
	{$V=\displaystyle\int\limits_1^2 \left|\sqrt{x}\right|\mathrm{\,d}x$}
	\loigiai{
Thể tích khối tròn xoay do hình phẳng được tính theo công thức
$$V=\pi \displaystyle\int\limits_1^2 \left(\sqrt{x}\right)^2\mathrm{\,d}x=\pi \displaystyle\int\limits_1^2 x\mathrm{\,d}x.$$
}
\end{ex}
\begin{ex}%[Vovanle]%[2D4H3-1]
\immini{Hình phẳng $(H)$ được giới hạn bởi đồ thị hàm số bậc ba và trục hoành được chia thành hai phần có diện tích lần lượt là $S_1$ và $S_2$ (như hình vẽ).\\ 
Biết $\displaystyle\int\limits_{-1}^1f(x)\mathrm{\,d}x=\dfrac{8}{3}$ và $\displaystyle\int\limits_1^4f(x)\mathrm{\,d}x=-\dfrac{63}{8}$. Khi đó diện tích $S$ của hình phẳng $(H)$ bằng
}{
\begin{tikzpicture}[line join=round,line cap=round, font=\footnotesize,scale=0.75,>=stealth]
	\draw[-stealth](-1.5,0)--(4.5,0)node[above]{$x$};
	\draw[-stealth](0,-2.5)--(0,2)node[right]{$y$};			
	\fill (0,0) circle(1pt)node[below left]{$O$}(-0.3,0.3)node[above]{$S_1$}(2.5,-1.2)node[above]{$S_2$};
	\draw[smooth,samples=300,domain=-1.5:4.25] plot(\x,{0.3*((\x)^2-1)*(\x-4)})node[above]{$y=f(x)$};
	\fill[pattern=north east lines]plot[domain=-1:4](\x,{0.3*((\x)^2-1)*(\x-4)})--(-1,0);
	\foreach \x/\g in {-1/140,1/60,4/130}\fill[black] (\x,0) circle (1pt)+(\g:.3)node{$\x$};		
	\end{tikzpicture}
}
	\choice
	{$\dfrac{125}{24}$}
	{$\dfrac{8}3$}
	{\True $\dfrac{253}{24}$}
	{$\dfrac{63}{8}$}
	\loigiai{
Ta có 
$$S_1=\displaystyle\int\limits_{-1}^1f(x)\mathrm{\,d}x=\dfrac{8}{3};
\,S_2=-\displaystyle\int\limits_1^4f(x)\mathrm{\,d}x=\dfrac{63}{8}.$$
Suy ra $S=S_1+S_2=\dfrac{8}{3}+\dfrac{63}{8}=\dfrac{253}{24}$.
}
\end{ex}
\begin{ex}%[Vovanle]%[2D4N3-3]
Hình phẳng giới hạn bởi các đường $y=-x^2+9$, $y=0$, $x=-3$, $x=3$ quay quanh trục $Ox$ tạo thành một khối tròn xoay có thể tích $V$. Khẳng định nào sau đây là đúng?
	\choice
	{$V=\displaystyle\int\limits_{-3}^3\left|-x^2+9\right|\mathrm{\,d}x$}
	{$V=\pi \displaystyle\int\limits_{-3}^3\left|-x^2+9\right|\mathrm{\,d}x$}
	{$V=\displaystyle\int\limits_{-3}^3\left(-x^2+9\right)^2\mathrm{\,d}x$}
	{\True $V=\pi\displaystyle\int\limits_{-3}^3\left(-x^2+9\right)^2\mathrm{\,d}x$}
	\loigiai{
Thể tích khối tròn xoay là
$$V=\pi\displaystyle\int\limits_{-3}^3\left(-x^2+9\right)^2\mathrm{\,d}x.$$
}
\end{ex}
\begin{ex}%[Vovanle]%[2D4N3-1]
Diện tích của hình phẳng giới hạn bởi đồ thị hàm số $y=x^2-4$, trục hoành và hai đường thẳng $x=-2$, $x=2$ bằng
	\choice
	{$S=\pi\displaystyle\int\limits_{-2}^2\left(x^2-4\right)\mathrm{\,d}x$}
	{\True $S=\displaystyle\int\limits_{-2}^2\left|x^2-4\right|\mathrm{\,d}x$}
	{$S=\displaystyle\int\limits_{-2}^2\left(x^2-4\right)\mathrm{\,d}x$}
	{$S=\pi\displaystyle\int\limits_{-2}^2\left(x^2-4\right)^2\mathrm{\,d}x$}
	\loigiai{
Diện tích của hình phẳng là
$$S=\displaystyle\int\limits_{-2}^2\left|x^2-4\right|\mathrm{\,d}x.$$
}
\end{ex}

\begin{ex}%[Vovanle]%[2D4H3-1]
Diện tích hình phẳng giới hạn bởi parabol $y=x^2-4x+5$ và đường thẳng $y=x+1$ được tính theo công thức nào sau đây?
	\choice
	{$S=\displaystyle\int\limits_1^4\left(x^2-5x+4\right)\mathrm{\,d}x$}
	{$S=\displaystyle\int\limits_1^4\left(x^2-5x+4\right)^2\mathrm{\,d}x$}
	{$S=\displaystyle\int\limits_1^4\left|x^2-5x+4\right|\mathrm{\,d}x$}
	{\True $S=\displaystyle\int\limits_1^4\left(x^2+5x+4\right)\mathrm{\,d}x$}
	\loigiai{
Phương trình hoành độ giao điểm của parabol $y=x^2-4x+5$ và đường thẳng $y=x+1$ là
$$x^2-4x+5=x+1\Leftrightarrow x^2-5x+4=0\Leftrightarrow\hoac{&x=1\\&x=4.}$$ 
Diện tích hình phẳng giới hạn bởi parabol $y=x^2-4x+5$ và đường thẳng $y=x+1$ là
$$S=\displaystyle\int\limits_1^4\left|x^2-4x+5-\left(x+1\right)\right|\mathrm{\,d}x=\displaystyle\int\limits_1^4\left|x^2-5x+4\right|\mathrm{\,d}x.$$
}
\end{ex}

\begin{ex}%[Vovanle]%[2D4H3-1]
Diện tích hình phẳng giới hạn bởi đồ thị hàm số $y=x^2$ và đường thẳng $y=2x$ là 
	\choice
	{\True $\dfrac{4}{3}$}
	{$\dfrac{5}{3}$}
	{$\dfrac{3}{2}$}
	{$\dfrac{23}{15}$}
	\loigiai{
Xét phương trình $x^2=2x\Leftrightarrow\hoac{&x=0\\&x=2.}$\\ 
Diện tích hình phẳng giới hạn bởi đồ thị hàm số $y=x^2$ và đường thẳng $y=2x$ là  $$S=\displaystyle\int\limits_0^2\left|x^2-x\right|\mathrm{\,d}x=\left| \displaystyle\int\limits_0^2\left(x^2-x\right)\mathrm{\,d}x\right|=\dfrac{4}{3}.$$
}
\end{ex}

\begin{ex}%[Vovanle]%[2D4H3-1]
\immini{Diện tích phần hình phẳng phần gạch sọc trong hình vẽ được tính theo công thức nào dưới đây? 
	\choice
	{$\displaystyle\int\limits_{-2}^3 \left[f(x)-g(x)\right]\mathrm{\,d}x$}
	{$\displaystyle\int\limits_{-2}^{5} \left[f(x)-g(x)\right]\mathrm{\,d}x+\displaystyle\int\limits_{5}^3 \left[ g(x)-f(x)\right]\mathrm{\,d}x$}
	{\True $\displaystyle\int\limits_{-2}^0 \left[f(x)-g(x)\right]\mathrm{\,d}x+\displaystyle\int\limits_0^3 \left[g(x)-f(x)\right]\mathrm{\,d}x$}
	{$\displaystyle\int\limits_{-2}^0 \left[g(x)-f(x)\right]\mathrm{\,d}x+\displaystyle\int\limits_0^3 \left[f(x)-g(x)\right]\mathrm{\,d}x$}
}{
\begin{tikzpicture}[line join=round,line cap=round, font=\footnotesize,scale=0.5,>=stealth]
	\draw[-stealth](-2.5,0)--(4,0)node[below]{$x$};
	\draw[-stealth](0,-3.7)--(0,7)node[right]{$y$};			
	\fill (0,0) circle(1pt)node[below left]{$O$};
	\draw[smooth,samples=300,domain=-2.2:3.5] plot(\x,{(5/6)*(\x+2)*(\x-1)*(\x-3)})node[above]{$y=f(x)$};
	\draw[smooth,samples=300,domain=-2.4:3.25] plot(\x,{(-5/6)*(\x+2)*(\x-3)})node[below right]{$y=g(x)$};
	\fill[pattern=north east lines]plot[domain=-2:3](\x,{(5/6)*(\x+2)*(\x-1)*(\x-3)})--plot[domain=3:-2](\x,{(-5/6)*(\x+2)*(\x-3)});
	\foreach \x/\g in {-2/150}\fill[black] (\x,0) circle (1pt)+(\g:.6)node{$\x$};
	\foreach \x/\g in {1/-120,3/50}\fill[black] (\x,0) circle (1pt)+(\g:.4)node{$\x$};
	\foreach \x/\g in {5/60}\fill[black] (0,\x) circle (1pt)+(\g:.6)node{$\x$};		
	\end{tikzpicture}
}	
	\loigiai{
Diện tích phần hình phẳng là
$$\displaystyle\int\limits_{-2}^0 \left[f(x)-g(x)\right]\mathrm{\,d}x+\displaystyle\int\limits_0^3 \left[g(x)-f(x)\right]\mathrm{\,d}x.$$
}
\end{ex}

\begin{ex}%[Vovanle]%[2D4V3-1]
Diện tích $S$ của hình phẳng giới hạn bởi đồ thị hai hàm số $y=-x^3$ và $y=x^2-2x$ là
	\choice
	{$S=\dfrac{9}{4}$}
	{$S=\dfrac{7}{3}$}
	{\True $S=\dfrac{37}{12}$}
	{$S=\dfrac{4}{3}$}
	\loigiai{
Hoành độ giao điểm của hai đồ thị là nghiệm của phương trình
$$-x^3=x^2-2x\Leftrightarrow x^3+x^2-2x=0\Leftrightarrow \hoac{&x=-2\\&x=0\\&   x=1.}$$
Diện tích hình phẳng cần tìm là 
\allowdisplaybreaks
\begin{eqnarray*}
S&=&\displaystyle\int\limits_{-2}^0 \left|\left(x^3+x^2-2x\right)\right|\mathrm{\,d}x+\displaystyle\int\limits_0^1 \left|\left(x^3+x^2-2x\right) \right|\mathrm{\,d}x\\
&=&\displaystyle\int\limits_{-2}^0\left(x^3+x^2-2x\right)\mathrm{\,d}x-\displaystyle\int\limits_0^1\left(x^3+x^2-2x \right)\mathrm{\,d}x\\
&=&\left.\left(\dfrac{x^4}4+\dfrac{x^3}3-x^2\right)\right|_{-2}^0-\left.\left(\dfrac{x^4}4+\dfrac{x^3}3-x^2\right)\right|_0^1\\
&=&\dfrac{37}{12}.
\end{eqnarray*}
}
\end{ex}

\begin{ex}%[Vovanle]%[2D4H3-3]
Thể tích vật tròn xoay khi quay hình phẳng $(H)$ xác định bởi các đường $y=\dfrac{1}{3}x^3-x^2$, $y=0$, $x=0$ và $x=3$ quanh trục $Ox$ là
	\choice
	{\True $\dfrac{81\pi}{35}$}
	{$\dfrac{81}{35}$}
	{$\dfrac{71\pi}{35}$}
	{$\dfrac{71}{35}$}
	\loigiai{
Phương trình hoành độ giao điểm 
$$\dfrac{1}{3}x^3-x^2=0\Leftrightarrow \hoac{&x=0\\&x=3.}$$
$$V=\pi\displaystyle\int\limits_0^3\left(\dfrac{1}{3}x^3-x^2\right)^2\mathrm{\,d}x=\pi\displaystyle\int\limits_0^3\left(\dfrac{1}{9}x^6-\dfrac{2}{3}x^5+x^4\right)\mathrm{\,d}x=\dfrac{81\pi}{35}.$$
}
\end{ex}

\begin{ex}%[Vovanle]%[2D4H3-3]
Cho $(H)$ là hình phẳng giới hạn bởi các đường $y=\sqrt{x}$, $y=x-2$ và trục hoành. Biết diện tích của $(H)$ bằng $\dfrac{a}{b}$. Tính giá trị biểu thức $T=a+b$.
	\choice
	{$T=11$}
	{\True $T=13$}
	{$T=10$}
	{$T=19$}
	\loigiai{
\immini{Diện tích của $(H)$ bằng 
$$S=\displaystyle\int\limits_0^2\sqrt{x}\mathrm{\,d}x+\displaystyle\int\limits_2^4\left(\sqrt{x}-x+2\right)\mathrm{\,d}x=\dfrac{10}{3}.$$
Vậy $a=10$; $b=3\Rightarrow a+b=13$.
}{
\begin{tikzpicture}[line join=round,line cap=round, font=\footnotesize,scale=1,>=stealth]
	\draw[-stealth](-0.5,0)--(4.6,0)node[below]{$x$};
	\draw[-stealth](0,-0.5)--(0,2.3)node[right]{$y$};			
	\fill (0,0) circle(1pt)node[below left]{$O$}(4,0) circle(1pt)node[below]{$4$}(0,2) circle(1pt)node[left]{$2$};
	\draw[smooth,samples=300,domain=0:4.5] plot(\x,{sqrt (\x)});
	\draw[smooth,samples=300,domain=4.3:1.4] plot(\x,{\x-2})node[below]{$y=x-2$};
	\draw (1,1)node[above,rotate=30]{$y=\sqrt{x}$};
	\draw [dashed] (4,0)|-(0,2);
	\fill[pattern=north east lines]plot[domain=0:4](\x,{sqrt (\x)})--(2,0)--cycle;			
	\end{tikzpicture}
}	
}
\end{ex}
\Closesolutionfile{ans}
% \indapan{6}{ans/ans-2-B13-De2-TN}

\TNTF
\Opensolutionfile{ans}[ans/ans-2-B13-De2-DS]
\setcounter{ex}{0}
\begin{ex}%[Vovanle]%[2D4H3-1]
\immini{Cho đồ thị hàm số $y=\left(\dfrac{1}{2}\right)^x$, $y=x+1$ và hình phẳng được gạch sọc như hình vẽ.
}{
\begin{tikzpicture}[line join=round,line cap=round, font=\footnotesize,scale=0.75,>=stealth]
	\draw[-stealth](-0.5,0)--(4.6,0)node[below]{$x$};
	\draw[-stealth](0,-0.5)--(0,4)node[right]{$y$};			
	\fill (0,0) circle(1pt)node[below left]{$O$}(2,0) circle(1pt)node[below]{$2$}(0,1) circle(1pt)node[left]{$1$};
	\draw[smooth,samples=300,domain=-0.2:2.5] plot(\x,{\x+1})node[above]{$y=x+1$};
	\draw[smooth,samples=300,domain=-0.2:4] plot(\x,{(0.5)^(\x)})node[above]{$y=\left(\dfrac{1}{2}\right)^x$};	
	\draw (2,3)|-(2,0.25);
	\draw [dashed] (2,0.25)--(2,0);
	\fill[pattern=north east lines]plot[domain=0:2](\x,{\x+1})--(2,0.25)--plot[domain=2:0](\x,{(0.5)^(\x)});			
	\end{tikzpicture}
}
	\choiceTF
	{\True Hình phẳng được gạch sọc giới hạn bởi các đường $x=0$; $x=2$; $y=x+1$; $y=\left(\dfrac{1}{2}\right)^x$}
	{\True Gọi $S_1$ là diện hình phẳng giới hạn bởi trục $Ox$, hai đường thẳng $x=0,\,x=2$ và đồ thị hàm số $y=x+1$. Khi đó $S_1=4$}
	{Gọi $S_2$ là diện hình phẳng giới hạn bởi trục $Ox$, hai đường thẳng $x=0,\,x=2$ và đồ thị hàm số $y=\left(\dfrac{1}{2}\right)^x$. Khi đó $S_2=\dfrac{3}{\ln 2}$}
	{Diện tích hình phẳng được giới hạn bởi các đường $x=0$; $x=2$; $y=x+1$; $y=\left( \dfrac{1}{2}\right)^x$ bằng $4-\dfrac{3}{\ln 2}$}
	\loigiai{

	\begin{itemchoice}
	\itemch Đúng. Hình phẳng được gạch sọc giới hạn bởi các đường $x=0$; $x=2$; $y=x+1$; $y=\left(\dfrac{1}{2}\right)^x$.
	\itemch Đúng. Gọi $S_1$ là diện hình phẳng giới hạn bởi trục $Ox$, hai đường thẳng $x=0,\,x=2$ và đồ thị hàm số $y=x+1$. Khi đó $S_1=\displaystyle\int\limits_0^2\left(x+1\right)\mathrm{\,d}x=\left.\left(\dfrac{x^2}2+x\right)\right|_0^2=2+2=4$.
	\itemch Sai. Gọi $S_2$ là diện hình phẳng giới hạn bởi trục $Ox$, hai đường thẳng $x=0,\,x=2$ và đồ thị hàm số $y=\left( \dfrac{1}{2}\right)^x$. Khi đó $S_2=\displaystyle\int\limits_0^2\left(\dfrac{1}{2}\right)^x\mathrm{\,d}x=\left.\dfrac{\left(\tfrac12\right)^x}{\ln \tfrac12}\right|_0^2=\dfrac{\tfrac{1}{4}-1}{-\ln 2}=\dfrac{3}{4\ln 2}$.
	\itemch Sai. Diện tích hình phẳng được giới hạn bởi các đường $x=0$; $x=2$; $y=x+1$; $y=\left(\dfrac{1}{2}\right)^x$.\\
Ta có $x+1>\left(\dfrac{1}{2}\right)^x$ với mọi $x\in\left[0;2\right]$.\\
Do đó $S=S_1-S_2=4-\dfrac{3}{4\ln 2}$.
	\end{itemchoice}
}
\end{ex}

\begin{ex}%[Vovanle]%[2D4H3-1]
Cho đồ thị các hàm số $y=4-x^2$, $y=x^2$.
\begin{center}
\begin{tikzpicture}[line join=round,line cap=round, font=\footnotesize,scale=1,>=stealth]
\draw[-stealth](-2,0)--(3,0)node[below]{$x$};
	\draw[-stealth](0,-0.5)--(0,4.5)node[right]{$y$};	
\fill[pattern=north east lines]plot[domain=-1:1](\x,{(\x)^2})--plot[domain=1:-1](\x,{-(\x)^2+4});
\draw[smooth,samples=300,domain=-2:2] plot(\x,{(\x)^2})node[above]{$y=x^2$};
\draw[smooth,samples=300,domain=-2.2:2.2] plot(\x,{4-(\x)^2})node[below]{$y=4-x^2$};		
	\fill (0,0) circle(1pt)node[below right]{$O$}(-1,0)circle(1pt)+(-0.1,0) node[below]{$-1$}(1,0)circle(1pt)node[below]{$1$};
	\draw (1,1)--(1,3)(-1,1)--(-1,3);
	\draw[dashed] (1,0)--(1,1)(-1,0)--(-1,1);		
	\end{tikzpicture}
\end{center}
	\choiceTF
	{Hình phẳng được gạch sọc, giới hạn bởi các đường $x=-1$; $x=2$; $y=x^2$; $y=4-x^2$}
	{Gọi $S_1$ là diện hình phẳng giới hạn bởi trục $Ox$, hai đường thẳng $x=-1$, $x=1$ và đồ thị hàm số $y=4-x^2$. Khi đó $S_1=\dfrac{22}{3}$}
	{Gọi $S_2$ là diện hình phẳng giới hạn bởi các đường $y=x^2$; $y=4-x^2$. Khi đó $S_2=16\sqrt2$}
	{Diện tích hình phẳng được giới hạn bởi các đường $x=-1$; $x=1$; $y=x^2$; $y=4-x^2$ là $S=\dfrac{20}3$}
	\loigiai{
	\begin{itemchoice}
	\itemch Sai. Hình phẳng được gạch sọc, giới hạn bởi các đường $x=-1$; $x=1$; $y=x^2$; $y=4-x^2$.	
	\itemch Đúng. $S_1=\displaystyle\int\limits_{-1}^1{\left|4-x^2\right|}\mathrm{\,d}x=\displaystyle\int\limits_{-1}^1{\left(4-x^2\right)}\mathrm{\,d}x=\left.\left( 4x-\dfrac{x^3}3 \right) \right|_{-1}^1=\dfrac{22}3$.
	\itemch Sai. Xét phương trình hoành độ giao điểm $$x^2=4-x^2\Leftrightarrow 2x^2=4\Leftrightarrow x^2=2\Leftrightarrow \hoac{&x=2\\&x=-2.}$$
Do đó 
\allowdisplaybreaks
\begin{eqnarray*}
S_2&=&\displaystyle\int\limits_{-\sqrt2}^{\sqrt2}\left|\left(4-x^2\right)-x^2 \right|\mathrm{\,d}x=\displaystyle\int\limits_{-\sqrt2}^{\sqrt2}\left|4-2x^2\right|\mathrm{\,d}x=\displaystyle\int\limits_{-\sqrt2}^{\sqrt2}\left(4-2x^2\right)\mathrm{\,d}x\\
&=&\left.\left(4x-\dfrac{2x^3}{3}\right)\right|_{-\sqrt2}^{\sqrt2}=\dfrac{16\sqrt2}{3}.
\end{eqnarray*}
	\itemch Đúng. Ta có 
\allowdisplaybreaks
\begin{eqnarray*}	
S&=&\displaystyle\int\limits_{-1}^1\left|\left(4-x^2 \right)-x^2\right|\mathrm{\,d}x=\displaystyle\int\limits_{-1}^1\left|4-2x^2\right|\mathrm{\,d}x=\displaystyle\int\limits_{-1}^1\left(4-2x^2\right)\mathrm{\,d}x\\
&=&\left. \left(4x-\dfrac{2x^3}3\right)\right|_{-1}^1=\dfrac{20}{3}.
\end{eqnarray*}
	\end{itemchoice}
}
\end{ex}

\begin{ex}%[Vovanle]%[2D4H3-3]
Cho đồ thị hàm số $y=5x-x^2$, đường thẳng $y=x$ và phần hình phẳng được gạch sọc như hình vẽ
\begin{center}
\begin{tikzpicture}[line join=round,line cap=round, font=\footnotesize,scale=0.6,>=stealth]
\draw[-stealth](-1,0)--(6,0)node[below]{$x$};
	\draw[-stealth](0,-0.5)--(0,6.5)node[right]{$y$};	
\fill[pattern=north east lines]plot[domain=0:4](\x,\x)--plot[domain=4:0](\x,{-(\x)^2+5*\x});
\draw[smooth,samples=300,domain=-0.5:4.7] plot(\x,\x)node[above]{$y=x$};
\draw[smooth,samples=300,domain=-0.1:5.1] plot(\x,{-(\x)^2+5*\x})node[below]{$y=5x-x^2$};		
	\fill (0,0) circle(1pt)node[below right]{$O$}(4,0)circle(1pt) node[below]{$4$}(0,4)circle(1pt)node[left]{$4$};
	\draw[dashed] (4,0)|-(0,4);		
	\end{tikzpicture}
\end{center}
	\choiceTF
	{\True Diện tích phần hình phẳng được gạch sọc trong hình vẽ là $\dfrac{32}{3}$}
	{Diện tích hình phẳng giới hạn bởi đường cong $y=5x-x^2$, trục hoành và hai đường thẳng $x=0$, $x=5$ là $\dfrac{125}{3}$}
	{\True Thể tích khi quay phần hình phẳng giới hạn bởi đồ thị hàm số $y=5x-x^2$ và đường thẳng $y=x$ quanh trục $Ox$ là $\dfrac{384\pi}{5}$}
	{\True Thể tích khi quay phần hình phẳng giới hạn bởi đường thẳng $y=x$, trục $Ox$, hai đường thẳng $x=2$, $x=5$ quanh trục $Ox$ là $39\pi$}
	\loigiai{
	\begin{itemchoice}
	\itemch Đúng. Xét phương trình hoành độ giao điểm 
	$$5x-x^2=x\Leftrightarrow 4x-x^2=0\Leftrightarrow \hoac{&x=0\\&x=4.}$$
Diện tích hình phẳng được gạch sọc giới hạn bởi hai đường là 
$$\displaystyle\int\limits_0^4{\left(5x-x^2-x \right)}\mathrm{\,d}x=\displaystyle\int\limits_0^4{\left(4x-x^2 \right)}\mathrm{\,d}x=\left.\left( 2x^2-\dfrac{x^3}{3}\right)\right|_0^4=\dfrac{32}3.$$
	\itemch Sai. Diện tích hình phẳng giới hạn bởi đường cong $y=5x-x^2$, trục hoành và hai đường thẳng $x=0$, $x=5$ là 
	$$\displaystyle\int\limits_0^{5}{\left(5x-x^2\right)\mathrm{\,d}x}=\left. \left(\dfrac{5x^2}{2}-\dfrac{x^3}{3}\right)\right|_0^{5}=\dfrac{125}{6}.$$
	\itemch Đúng. Phương trình hoành độ giao điểm hai đường là 
	$$5x-x^2=x\Leftrightarrow 4x-x^2=0\Leftrightarrow \hoac{&x=0\\&x=4.}$$
Thể tích khi quay phần hình phẳng giới hạn bởi đồ thị hàm số $y=5x-x^2$ và đường thẳng $y=x$ quanh trục $Ox$ là
\allowdisplaybreaks
\begin{eqnarray*}
V&=&\pi \displaystyle\int\limits_0^4{{{\left( 5x-x^2 \right)}^2}}\mathrm{\,d}x-\pi \displaystyle\int\limits_0^4{x^2}\mathrm{\,d}x=\pi\displaystyle\int\limits_0^4{\left(x^4-10x^3+24x^2 \right)}\mathrm{\,d}x\\
&=&\pi \left. \left( \dfrac{{x^{5}}}{5}-\dfrac{5x^4}2+8x^3 \right) \right|_0^4=\dfrac{384\pi}{5}.
\end{eqnarray*} 
	\itemch Đúng. Thể tích khi quay phần hình phẳng giới hạn bởi đường thẳng $y=x$, trục $Ox$, hai đường thẳng $x=2$, $x=5$ quanh trục $Ox$ là
$$V_1=\pi \displaystyle\int\limits_2^{5}x^2\mathrm{\,d}x=\pi\left.\dfrac{x^3}{3}\right|_2^{5}=39\pi.$$
	\end{itemchoice}
}
\end{ex}

\begin{ex}%[Vovanle]%[2D4H3-3]
Cho hai đồ thị hàm số $y=x^2-2x-1$ và $y=-x^2+3$ và phần hình phẳng được gạch chéo như hình vẽ.
\begin{center}
\begin{tikzpicture}[line join=round,line cap=round, font=\footnotesize,scale=1,>=stealth]
\draw[-stealth](-2,0)--(3,0)node[below]{$x$};
	\draw[-stealth](0,-2.5)--(0,3.5)node[right]{$y$};	
\fill[pattern=north east lines]plot[domain=2:-1](\x,{(\x)^2-2*\x-1})--plot[domain=-1:2](\x,{-(\x)^2+3});
\draw[smooth,samples=300,domain=-1.2:3] plot(\x,{(\x)^2-2*\x-1})node[above]{$y=x^2-2x-1$};
\draw[smooth,samples=300,domain=-1.8:2.2] plot(\x,{-(\x)^2+3})node[below]{$y=-x^2+3$};		
	\fill (0,0) circle(1pt)node[below right]{$O$}(-1,0)circle(1pt)+(0.1,0) node[below]{$-1$}(2,0)circle(1pt)node[above]{$2$};
	\draw[dashed] (2,0)--(2,-1)(-1,0)--(-1,2);		
	\end{tikzpicture}
\end{center}
	\choiceTF
	{\True Biểu thức diện tích phần hình phẳng gạch chéo trong hình vẽ là 
	$\displaystyle\int\limits_{-1}^2 \left(-2x^2+2x+4 \right)\mathrm{\,d}x$}
	{\True Diện tích hình phẳng giới hạn bởi đường cong $y=x^2-2x-1$, trục hoành và hai đường thẳng $x=0$, $x=1$ là $\dfrac{5}{3}$}
	{Thể tích khi quay phần hình phẳng giới hạn bởi đồ thị hàm số $y=-x^2+3$, trục $Ox$, hai đường thẳng $x=1$, $x=2$ quanh trục $Ox$ là $\dfrac{\pi}{5}$}
	{Thể tích khi quay phần hình phẳng giới hạn bởi đồ thị hàm số $y=x^2-2x-1$, trục $Ox$, hai đường thẳng $x=-1$, $x=2$ quanh trục $Ox$ là $\dfrac{33}{5}$}
	\loigiai{
	\begin{itemchoice}
	\itemch Đúng. Dựa vào hình vẽ ta có diện tích hình phẳng được gạch chéo trong hình vẽ được xác định là biểu thức 
	$$\displaystyle\int_{-1}^2\left[\left(-x^2+2\right)-\left(x^2-2x-2\right)\right]\mathrm{\,d}x=\displaystyle\int_{-1}^2\left(-2x^2+2x+4\right)\mathrm{\,d}x.$$ 	
	\itemch Đúng. Diện tích hình phẳng giới hạn bởi đường cong $y=x^2-2x-1$, trục hoành và hai đường thẳng $x=0,x=1$ là 
	$$\displaystyle\int\limits_0^1\left| x^2-2x-1\right|\mathrm{\,d}x=\dfrac{5}{3}.$$	
	\itemch Sai. Thể tích khi quay phần hình phẳng giới hạn bởi đồ thị hàm số $y=-x^2+3$, trục $Ox$, hai đường thẳng $x=1$, $x=2$ quanh trục $Ox$ là 
	$$\pi\displaystyle\int\limits_1^2(-x^2+3)^2\mathrm{\,d}x=\dfrac{6\pi}{5}.$$	
	\itemch Sai. Thể tích khi quay phần hình phẳng giới hạn bởi đồ thị hàm số $y=x^2-2x-1$, trục $Ox$, hai đường thẳng $x=-1$, $x=2$ quanh trục $Ox$ là 
	$$\pi\displaystyle\int\limits_{-1}^2(x^2-2x-1)^2\mathrm{\,d}x=\dfrac{33\pi }{5}.$$
	\end{itemchoice}
}
\end{ex}
\Closesolutionfile{ans}
\indapan3{ans/ans-2-B13-De2-DS}

\Opensolutionfile{ans}[ans/ans-2-B13-De2-KQ]
\TNSA
\setcounter{ex}{0}
\begin{ex}%[Vovanle]%[2D4H3-1]
Tính diện tích hình phẳng giới hạn bởi đồ thị của hàm số $y=x^3-3x$; $y=x$, hai đường thẳng $x=-1$; $x=2$.
\shortans{$5{,}75$}
	\loigiai{
Diện tích hình phẳng cần tìm là $S=\displaystyle\int\limits_{-1}^2{\left| x^3-3x-x \right|\mathrm{\,d}x=\displaystyle\int\limits_{-1}^2{\left| x^3-4x \right|\mathrm{\,d}x.}}$\\
Ta có $$x^3-3x=x\Leftrightarrow x(x^2-4)=0\Leftrightarrow \hoac{&x=0\\ &x=-2\notin[-1;2]\\&x=2.}$$
Phương trình có hai nghiệm thuộc đoạn $\left[-1;2\right]$ là $x=0$; $x=2$.
\allowdisplaybreaks
\begin{eqnarray*}
  S&=&\displaystyle\int\limits_{-1}^2{\left|x^3-4x\right|}\mathrm{\,d}x=\displaystyle\int\limits_{-1}^0{\left|x^3-4x\right|}\mathrm{\,d}x+\displaystyle\int\limits_0^2{\left| x^3-4x\right|}\mathrm{\,d}x\\
  &=&\left|\displaystyle\int\limits_{-1}^0{(x^3-4x)\mathrm{\,d}x}\right|+\left| \displaystyle\int\limits_0^2{(x^3-4x)\mathrm{\,d}x}\right|\\ 
 &=&\left| \left.\left(\dfrac{x^4}4-2x^2 \right)\right|_0^1\right|+\left|\left. \left(\dfrac{x^4}4-2x^2 \right)\right|_0^2 \right|=\dfrac{23}{4}\approx 5{,}75. 
\end{eqnarray*}
}
\end{ex}

\begin{ex}%[Vovanle]%[2D4H3-3]
Cho hình phẳng giới hạn bởi các đường $y=\sqrt{x}-2$, $y=0$ và $x=9$ quay xung quanh trục $Ox$. Tính thể tích khối tròn xoay tạo thành (làm tròn kết quả thể tích đến hàng phần trăm).
\shortans{$5{,}76$}
	\loigiai{
Phương trình hoành độ giao điểm của đồ thị hàm số $y=\sqrt{x}-2$ và trục hoành 
$$\sqrt{x}-2=0\Leftrightarrow \sqrt{x}=2\Leftrightarrow x=4.$$
Thể tích của khối tròn xoay tạo thành là
\allowdisplaybreaks
\begin{eqnarray*}
V&=&\pi \displaystyle\int\limits_4^{9}{{{\left(\sqrt{x}-2\right)}^2}\mathrm{\,d}x}\\
&=&\pi\displaystyle\int\limits_4^{9}{\left(x-4\sqrt{x}+4\right)}\mathrm{\,d}x\\
&=&\pi\left.\left(\dfrac{x^2}2-\dfrac{8x\sqrt{x}}3+4x\right)\right|_4^{9}\\
&=&\pi\left(\dfrac{81}{2}-72+36\right)-\pi\left(\dfrac{16}{2}-\dfrac{64}{3}+16\right)\\
&=&\dfrac{11\pi}{6}\approx 5{,}76.
\end{eqnarray*}
}
\end{ex}

\begin{ex}%[Vovanle]%[2D4V3-1]
\immini{Cho hàm số $y=ax^4+bx^2+c$ có đồ thị $(C)$, biết rằng $(C)$ đi qua điểm $A(-1;0)$, tiếp tuyến $d$ tại $A$ của $(C)$, cắt $(C)$ tại hai điểm có hoành độ lần lượt là $0$ và $2$. Diện tích hình phẳng giới hạn bởi $d$, đồ thị $(C)$ và hai đường thẳng $x=0$; $x=2$ có diện tích bằng $\dfrac{28}{5}$ (phần gạch sọc trong hình vẽ).
 
Tính diện tích hình phẳng giới hạn bởi $(C)$, trục hoành và hai đường thẳng $x=-1$; $x=0$.
}{
\begin{tikzpicture}[line join=round,line cap=round, font=\footnotesize,scale=0.5,>=stealth]
\draw[-stealth](-2,0)--(2.5,0)node[below]{$x$};
	\draw[-stealth](0,-0.7)--(0,7)node[right]{$y$};	
\fill[pattern=north east lines]plot[domain=0:2](\x,{(\x)^4-3*(\x)^2+2})--cycle;
\draw[smooth,samples=300,domain=-2.02:2.02] plot(\x,{(\x)^4-3*(\x)^2+2});
\draw[smooth,samples=300,domain=-1.5:2.3] plot(\x,{2*(\x+1)});		
	\fill (0,0) circle(1pt)node[below right]{$O$}(-1,0)circle(1pt)+(0.1,0) node[below]{$-1$}(2,0)circle(1pt)node[below]{$2$};
	\draw[dashed] (2,0)--(2,6);		
	\end{tikzpicture}
}	
\shortans{$0{,}2$}
	\loigiai{
Ta có $y'=4ax^3+2bx$ $\Rightarrow d\colon y=\left(-4a-2b\right)\left(x+1\right)$.
Phương trình hoành độ giao điểm của $d$ và $(C)$ là $\left(-4a-2b\right)\left(x+1\right)=ax^4+bx^2+c.\hfill(1)$\\
Phương trình $(1)$ phải cho $2$ nghiệm là $x=0$, $x=2$.
$$\Rightarrow\heva{&-4a-2b=c\\&-12a-6b=16a+4b+c}
\Leftrightarrow \heva{&-4a-2b-c=0&(2)\\&28a+10b+c=0&(3).}$$
Mặt khác, diện tích phần gạch sọc là 
\allowdisplaybreaks
\begin{eqnarray*}
&&\dfrac{28}{5}=\displaystyle\int\limits_0^2{\left[\left(-4a-2b \right)\left( x+1 \right)-ax^4-bx^2-c\right]\mathrm{\,d}x}\\
&\Leftrightarrow& \dfrac{28}{5}=4\left(-4a-2b\right)-\dfrac{32}{5}a-\dfrac{8}3b-2c\\
&\Leftrightarrow& \dfrac{112}{5}a+\dfrac{32}3b+2c=-\dfrac{28}{5}\qquad(4)
\end{eqnarray*}
Giải hệ 3 phương trình $(2)$, $(3)$ và $(4)$ ta được $a=1$, $b=-3$, $c=2$.\\
Khi đó, $(C)\colon y=x^4-3x^2+2$, $d\colon y=2\left(x+1\right)$.\\
Diện tích cần tìm là 
$$S=\displaystyle\int\limits_{-1}^0\left[x^4-3x^2+2-2\left(x+1\right)\right]\mathrm{\,d}x=\displaystyle\int\limits_{-1}^0\left(x^4-3x^2-2x\right)\mathrm{\,d}x=\dfrac1{5}=0{,}2.$$
}
\end{ex}

\begin{ex}%[Vovanle]%[2D4V3-2]
\immini{Một khuôn viên dạng nửa hình tròn có đường kính bằng $4\sqrt{5}$ (m). Trên đó người thiết kế hai phần để trồng hoa có dạng của một cánh hoa hình parabol có đỉnh trùng với tâm nửa hình tròn và hai đầu mút của cánh hoa nằm trên nửa đường 
}{
\begin{tikzpicture}[line join=round,line cap=round, font=\footnotesize,scale=0.5,>=stealth]
\path
({-2*sqrt (5)},0) coordinate (A)
({2*sqrt (5)},0) coordinate (B)
(2,4) coordinate (M)
(-2,4) coordinate (N)
(-2,0) coordinate (C)
(2,0) coordinate (D)
($(M)!0.5!(D)$) coordinate (G)node[right]{$4$ m}
;
\fill[pattern=north east lines]plot[domain=-2:2](\x,{sqrt (20-(\x)^2)})--plot[domain=2:-2](\x,{(\x)^2});	
	\draw (A) arc(180:0:{2*sqrt (5)});
	\draw plot[domain=-2:2](\x,{(\x)^2});
	\draw[dashed] (N)--(C)(M)--(D)(M)--(N);
	\path ($(M)!0.5!(N)$) coordinate (H)node[below]{$4$ m};
	\draw (A)--(B);			
	\end{tikzpicture}
}
\noindent tròn (phần gạch sọc), cách nhau một khoảng bằng $4\,\mathrm{m}$, phần còn lại của khuôn viên (phần không gạch sọc) dành để trang trí cỏ nhân tạo. Biết các kích thước cho như hình vẽ và kinh phí cỏ nhân tạo là $100\,000$ đồng/m$^2$. Hỏi cần bao nhiêu tiền để trang trí cỏ trên phần đất đó? (Số tiền được làm tròn đến hàng nghìn).	
\shortans{$1948$}
	\loigiai{
\immini{Đặt hệ trục tọa độ như hình vẽ. Khi đó phương trình nửa đường tròn là
$$y=\sqrt{R^2-x^2}=\sqrt{\left(2\sqrt{5}\right)^2-x^2}=\sqrt{20-x^2}.$$
Phương trình parabol $(P)$ có đỉnh là gốc $O$ sẽ có dạng $y=ax^2$. Mặt khác $(P)$ qua điểm $M(2;4)$. 
}{
\begin{tikzpicture}[line join=round,line cap=round, font=\footnotesize,scale=0.6,>=stealth]
\path
({-2*sqrt (5)},0) coordinate (A)
({-2*sqrt (5)},0) coordinate (B)
(2,4) coordinate (M)node[above right]{$M(2,4)$}
(-2,4) coordinate (N)
(-2,0) coordinate (C)
(2,0) coordinate (D)
;
\fill[pattern=north east lines]plot[domain=-2:2](\x,{sqrt (20-(\x)^2)})--plot[domain=2:-2](\x,{(\x)^2});
	\draw[-stealth](-5,0)--(5,0)node[below]{$x$};
	\draw[-stealth](0,-0.7)--(0,5)node[right]{$y$};	
	\fill (0,0) circle(1pt)node[below left]{$O$}(-2,0)circle(1pt) node[below]{$-2$}(2,0)circle(1pt)node[below]{$2$}(0,4)circle(1pt)node[below left]{$4$};
	\draw (A) arc(180:0:{2*sqrt (5)});
	\draw plot[domain=-2:2](\x,{(\x)^2});
	\draw[dashed] (N)--(C)(M)--(D)(M)--(N);				
	\end{tikzpicture}
}	
\noindent Do đó $4=a\cdot(-2)^2\Rightarrow a=1$.\\
Phần diện tích của hình phẳng giới hạn bởi $(P)$ và nửa đường tròn.(phần gạch sọc).\\
Ta có công thức $S_1=\displaystyle\int\limits_{-2}^2{\left(\sqrt{20-x^2}-x^2 \right)\mathrm{\,d}x}\approx 11{,}94\,\mathrm{m}^2$.\\
Vậy phần diện tích trồng cỏ là $S_{\text{cỏ}}=\dfrac{1}{2}{{S}_{\text{htron}}}-S_1=\dfrac{1}{2}\cdot 20\cdot \pi-11{,}94\approx 19{,}476\,\mathrm{m}^2$.\\
Số tiền cần có là $S_{\text{cỏ}}\times 100000\approx 1947592\text{ (đồng)}\approx 1948$ (nghìn đồng).
}
\end{ex}

\begin{ex}%[Vovanle]%[2D4V3-4]
Một téc nước hình trụ, đang chứa nước được đặt nằm ngang, có chiều dài $3$ m và đường kính đáy $1$ m. Hiện tại mặt nước trong téc cách phía trên đỉnh của téc $0{,}25$ m (xem hình vẽ). 
\begin{center}
\begin{tikzpicture}[line join=round,line cap=round, font=\footnotesize,scale=1,>=stealth]
\def \x{0.5}
\def \y{1.5}
\def \z{6}
\path
(80:{\x} and {\y}) coordinate (A)
(80:{\x} and {\y})+(\z,0) coordinate (B)
(-80:{\x} and {\y}) coordinate (C)
(-80:{\x} and {\y})+(\z,0) coordinate (D)
(10:{\x} and {\y}) coordinate (E)
(200:{\x} and {\y}) coordinate (F)
($(B)!0.5!(D)$) coordinate (K)
($(A)!0.5!(C)$) coordinate (H)
(40:{\x} and {\y}) coordinate (M)
(160:{\x} and {\y}) coordinate (N)
(40:{\x} and {\y})+(\z,0) coordinate (U)
(160:{\x} and {\y})+(\z,0) coordinate (V)
(C)+(0,-0.7) coordinate (I)
(D)+(0,-0.7) coordinate (J)
(B)+(2,0) coordinate (P)
(D)+(2,0) coordinate (Q)
(intersection of B--D and U--V) coordinate (T)
(B)+(0.6,0) coordinate (G)
(T)+(0.6,0) coordinate (R)
($(P)!0.5!(Q)$) coordinate (m)node[right]{$1$ m}
($(I)!0.5!(J)$) coordinate (X)node[above]{$3$ m}
($(G)!0.5!(R)$) coordinate (Z)node[right]{$0{,}25$ m}
;
\fill[blue!20] plot [domain=-90:-200] (0.5*cos \x,{1.5*sin \x})--(M)--(U)--plot [domain=40:-90] (\z+0.5*cos \x,{1.5*sin \x})--cycle;
\draw  (A) arc (80:280:{\x} and {\y});
\draw[dashed] (C) arc (-80:80:{\x} and {\y});
\draw  (B) arc (80:280:{\x} and {\y});
\draw (D) arc (-80:80:{\x} and {\y});
\draw (A)--(B)(C)--(D)(U)--(V)(N)--(V);
\draw[<->](I)--(J);
\draw[<->](P)--(Q);
\draw[<->](G)--(R);
\draw[dashed](C)--(I)(D)--(J)(B)--(P)(C)--(Q)(M)--(N)(M)--(U)(T)--(R);

\end{tikzpicture}
\end{center}
Tính thể tích của nước trong téc (kết quả làm tròn đến hàng phần trăm)?
\shortans{$1{,}9$}
	\loigiai{
\immini{Thế tích phần dầu còn lại sẽ bằng diện tích hình phẳng gạch sọc trong hình nhân với chiều dài của bồn (chiều cao của trụ).
 
Đường tròn có tâm $O(0;0)$, $R=0{,}5$ có phương trình là 
$$x^2+y^2=0{,}25 \Leftrightarrow y=\pm \sqrt{0{,}25-x^2}.$$
 Diện tích hình gạch sọc chính là diện tích hình phẳng giới hạn bởi các đường 
 $$y=\sqrt{0{,}25-x^2};\,y=-\sqrt{0{,}25-x^2};\,x=-0{,}5;\,x=0{,}25.$$
Do đó 
$$V=Sh=3 \displaystyle\int_{-0{,}5}^{0{,}25}\left|\sqrt{0{,}25-x^2}-\left(-\sqrt{0{,}25-x^2}\right)\right|\mathrm{\,d}x \approx 1{,}896\mathrm{\,m}^3 \approx 1{,}9\mathrm{\,m}^3.$$
}{
\begin{tikzpicture}[line join=round,line cap=round, font=\footnotesize,scale=1,>=stealth]
\def \r{1.5}
\fill[pattern=north east lines]plot[domain=60:300](\r*cos \x,\r*sin \x)--cycle;
	\draw[-stealth](-2,0)--(2.5,0)node[below]{$x$};
	\draw[-stealth](0,-2)--(0,2)node[right]{$y$};	
	\fill (0,0) circle(1pt)node[below left]{$O$}(\r,0)circle(1pt)+(-0.1,0.2) node[right]{$0{,}5$}(-\r,0)circle(1pt)+(0.1,0.2) node[left]{$-0{,}5$}(0,-\r)circle(1pt)+(0.1,-0.2) node[left]{$-0{,}5$}(0,\r)circle(1pt)+(0.1,0.2) node[left]{$0{,}5$}(0.5*\r,0)circle(1pt)+(-0.1,-0.2) node[right]{$0{,}25$};
	\draw (60:\r)--(-60:\r);
	\draw (0,0) circle(\r);			
	\end{tikzpicture}
}	
}
\end{ex}

\begin{ex}%[Vovanle]%[2D4V3-1]
\immini{Cho hai đường tròn $\left(O_1;5\right)$ và $\left(O_2;3\right)$ cắt nhau tại hai điểm $A$, $B$ sao cho $AB$ là một đường kính của đường tròn $\left(O_2\right)$. Gọi $(D)$ là hình thẳng được giới hạn bởi hai đường tròn (phần ở ngoài đường tròn lớn, được gạch chéo như hình vẽ). Một vật trang trí có dạng một khối tròn xoay được tạo thành khi quay miền $(D)$ quanh trục $O_1O_2$. Thể tích của khối tròn xoay được tạo thành có $V=\dfrac{a\pi}{b}$ ($\dfrac{a}{b}$ là phân số tối giản) thì $a^2+b^3$ bằng bao nhiêu?
}{
\begin{tikzpicture}[line join=round,line cap=round, font=\footnotesize,scale=1,>=stealth]
\def \r{0.3}
 \path
    (-4*\r,0) coordinate (O_1)
    (0,0) coordinate (O_2)
    (90:3*\r) coordinate (A)
    (-90:3*\r) coordinate (B)    
    ;
    \pgfmathsetmacro\g{atan (3/4)}
    \fill[pattern=north east lines]plot[domain=-\g:\g](5*\r*cos \x-4*\r,5*\r*sin \x)--plot[domain=90:-90](3*\r*cos \x,3*\r*sin \x);	
	\draw (O_1) circle(5*\r);
	\draw (O_2) circle(3*\r);
	\draw (2*\r,0) coordinate (D)node[above]{$(D)$};
	\draw (A)--(B)(-9*\r,0)--(3*\r,0);
	\foreach \x/\g in {O_1/90,O_2/140,A/60,B/-60}\fill[black] (\x) circle (1pt)+(\g:.3)node{$\x$};		
	\end{tikzpicture}
}
\shortans{$1627$}
	\loigiai{
\immini{Chọn hệ tọa độ $Oxy$ với \\
$O_2\equiv O$, $O_2C\equiv Ox$, $O_2A\equiv Oy$.\\
Đoạn $O_1O_2=\sqrt{O_1A^2-O_2A^2}=\sqrt{5^2-3^2}=4$.\\
Suy ra $\left(O_1\right):{{\left( x+4 \right)}^2}+y^2=25$.\\
Kí hiệu $\left(H_1\right)$ là hình phẳng giới hạn bởi các đường $\left(O_1\right)\colon \left(x+4\right)^2+y^2=25$, $Oy\colon x=0$, $x\geq 0$.\\
Kí hiệu $\left(H_2\right)$ là hình phẳng giới hạn bởi các đường $\left(O_2\right)\colon x^2+y^2=9$, $Oy\colon x=0$, $x\geq 0$.
}{
\begin{tikzpicture}[line join=round,line cap=round, font=\footnotesize,scale=1,>=stealth]
\def \r{0.4}
 \path
    (-4*\r,0) coordinate (O_1)
    (0,0) coordinate (O_2)
    (90:3*\r) coordinate (A)
    (-90:3*\r) coordinate (B)    
    ;
    \pgfmathsetmacro\g{atan (3/4)}
    \fill[pattern=north east lines]plot[domain=-\g:\g](5*\r*cos \x-4*\r,5*\r*sin \x)--plot[domain=90:-90](3*\r*cos \x,3*\r*sin \x);
	\draw[-stealth](-9.5*\r,0)--(4*\r,0)node[below]{$x$};
	\draw[-stealth](0,-5.5*\r)--(0,5.5*\r)node[right]{$y$};		
	\draw (O_1) circle(5*\r);
	\draw (O_2) circle(3*\r);
	\draw (2*\r,0) coordinate (D)node[above]{$(D)$};
	\foreach \x/\g in {O_1/90,O_2/140,A/60,B/-60}\fill[black] (\x) circle (1pt)+(\g:.3)node{$\x$};		
	\end{tikzpicture}
}
\noindent Khi đó thể tích $V$ cần tìm chính bằng thể tích $V_2$ của khối tròn xoay thu được khi quay hình $\left(H_2\right)$ xung quanh trục $Ox$ trừ đi thể tích $V_1$ của khối tròn xoay thu được khi quay hình $\left(H_1\right)$ xung quanh trục $Ox$.\\
Ta có $V_2=\dfrac{1}{2}\cdot \dfrac{4}{3}\pi r^3=\dfrac{2}{3}\pi {\cdot 3^3}=18\pi$.\\
Lại có $V_1=\pi\displaystyle\int\limits_0^1y^2\mathrm{\,d}x=\pi\displaystyle\int\limits_0^1\left[25-\left(x+4\right)^2\right]\mathrm{\,d}x=\left.\pi \left[25x-\dfrac{\left(x+4\right)^3}{3}\right]\right|_0^1
=\dfrac{14\pi}{3}$.\\
Do đó $V=V_2-V_1=18\pi-\dfrac{14\pi}{3}=\dfrac{40\pi}{3}$.\\
Vậy $a^2+b^3=1627$.
}
\end{ex}
\Closesolutionfile{ans}
% \indapan{6}{ans/ans-2-B13-De2-KQ}


%Chương V. Mặt phẳng, đt trong kg
%%Bài 1.
% \chapter{PHƯƠNG PHÁP TỌA ĐỘ TRONG KHÔNG GIAN}
\section{PHƯƠNG TRÌNH MẶT PHẲNG}
% \chude{Xác định các yếu tố cơ bản liên quan đến mặt phẳng}
\begin{dang}{Xác định véctơ pháp tuyến của mặt phẳng. Xác định điểm thuộc và không thuộc mặt phẳng}
	\begin{enumerate}[label=\bf\arabic*.]
		\item \textbf{véctơ pháp tuyến của mặt phẳng:}
		\begin{itemize}
			\item Mặt phẳng $(\alpha)\colon A x+B y+C z+D=0$ có véctơ pháp tuyến $\overrightarrow{n}=(A; B; C)$.
			\item Nếu mặt phẳng $(\alpha)$ có cặp véctơ chỉ phương là $\overrightarrow{a}, \overrightarrow{b}$ thì $(\alpha)$ có véctơ pháp tuyến là $\overrightarrow{n}=\left[\overrightarrow{a}, \overrightarrow{b}\right]$.
			\item véctơ pháp tuyến của mặt phẳng $(\alpha)$ là véctơ có giá vuông góc với $(\alpha)$.
			\item véctơ chỉ phương của mặt phẳng $(\alpha)$ là véctơ có giá song song hoặc trùng với $(\alpha)$.
			\item Nếu $\overrightarrow{n}$ là một véctơ pháp tuyến của $(\alpha)$ thì $k \cdot \overrightarrow{n}$ cũng là một véctơ pháp tuyến của $(\alpha)$.
			\item Nếu $\overrightarrow{a}$ là một véctơ chỉ phương của $(\alpha)$ thì $k \cdot \overrightarrow{a}$ cũng là một véctơ chỉ phương của $(\alpha)$.
			\item[] \textbf{Chú ý:}
			\item Trục $O x$ có véctơ chỉ phương là $\overrightarrow{i}=(1; 0; 0)$.
			\item Trục $O y$ có véctơ chỉ phương là $\overrightarrow{j}=(0; 1; 0)$.
			\item Trục $O z$ có véctơ chỉ phương là $\overrightarrow{k}=(0; 0; 1)$.
			\item Mặt phẳng $(O x y)$ có véctơ pháp tuyến là $\overrightarrow{k}=(0; 0; 1)$.
			\item Mặt phẳng $(O x z)$ có véctơ pháp tuyến là $\overrightarrow{j}=(0; 1; 0)$.
			\item Mặt phẳng $(O y z)$ có véctơ pháp tuyến là $\overrightarrow{i}=(1; 0; 0)$.
		\end{itemize}
		\item \textbf{Điểm thuộc và không thuộc mặt phẳng:}\\
		Cho mặt phẳng $(\alpha)$ có phương trình $A x+B y+C z+D=0$. Khi đó: 
		\begin{itemize}
			\item $N_0\left(x_0; y_0; z_0\right) \in(\alpha) \Leftrightarrow A x_0+B y_0+C z_0+D=0$.
			\item $N_0\left(x_0; y_0; z_0\right) \notin(\alpha) \Leftrightarrow A x_0+B y_0+C z_0+D \neq 0$.	
		\end{itemize}
	\end{enumerate}
\end{dang}

\TN
\Opensolutionfile{ans}[ans/ans2C5B1CD1]
\begin{ex}%[2H2H2-5]
	Trong không gian $O x y z$, tọa độ một véctơ $\overrightarrow{n}$ vuông góc với cả hai véctơ $\overrightarrow{a}=(1; 1;-2), \overrightarrow{b}=(1; 0; 3)$ là
	\choice
	{$(2; 3;-1)$}
	{$(3; 5;-2)$}
	{$(2;-3;-1)$}
	{\True $(3;-5;-1)$}
	\loigiai{
		véctơ $\overrightarrow{n}$ vuông góc với cả hai véctơ $\overrightarrow{a}, \overrightarrow{b}$.\\
		Do đó $\overrightarrow{n}=\left[\overrightarrow{a}, \overrightarrow{b}\right]$.\\
		Ta có $\left[\overrightarrow{a}, \overrightarrow{b}\right]=(3;-5;-1)$.	
	}
\end{ex}

\begin{ex}%[2H2H2-5]
	Trong không gian với hệ tọa độ $O x y z$, cho hai véctơ $\overrightarrow{a}=(2; 1;-2)$ và véctơ $\overrightarrow{b}=(1; 0; 2)$. Tìm tọa độ véctơ $\overrightarrow{c}$ là tích có hướng của $\overrightarrow{a}$ và $\overrightarrow{b}$.
	\choice
	{$\overrightarrow{c}=(2; 6;-1)$}
	{$\overrightarrow{c}=(4; 6;-1)$}
	{$\overrightarrow{c}=(4;-6;-1)$}
	{\True $\overrightarrow{c}=(2;-6;-1)$}
	\loigiai{
		Áp dụng công thức tính tích có hướng trong hệ trục tọa độ $O x y z$, ta được
		$$\overrightarrow{c}=\left[\overrightarrow{a}, \overrightarrow{b}\right]=(2;-6;-1).$$	
	}
\end{ex}

\begin{ex}%[2H2H2-5]
	Trong không gian với hệ trục tọa độ $O x y z$, cho $A(2; 1;-3), B(0;-2; 5)$ và $C(1; 1; 3)$. Tìm tọa độ véctơ $\overrightarrow{n}$ có phương vuông góc với hai véctơ $\overrightarrow{A B}$ và $\overrightarrow{A C}$.
	\choice
	{$\overrightarrow{n}=(8; 4;-3)$}
	{$\overrightarrow{n}=(-18; 0;-3)$}
	{\True $\overrightarrow{n}=(-18; 4;-3)$}
	{$\overrightarrow{n}=(1; 4;-3)$}
	\loigiai{
		Ta có $\overrightarrow{A B}=(-2;-3; 8)$ và $\overrightarrow{A C}=(-1; 0; 6)$. Suy ra $\left[\overrightarrow{A B}, \overrightarrow{A C}\right]=(-18; 4;-3)$.\\
		Vậy $\overrightarrow{n}=\left[\overrightarrow{A B}, \overrightarrow{A C}\right]=(-18; 4;-3)$.
	}
\end{ex}

\begin{ex}%[2H5N1-3]
	Trong không gian $O x y z$, phương trình nào sau đây là phương trình tổng quát của mặt phẳng?
	\choice
	{$x-3 y^2+z-1=0$}
	{$x^2+2 y+4 z-2=0$}
	{\True $2 x-3 y+4 z-2024=0$}
	{$2 x-3 y+4 z^2-2025=0$}
	\loigiai{
		Phương trình tổng quát của mặt phẳng là $2 x-3 y+4 z-2024=0$.	
	}
\end{ex}

\begin{ex}%[2H5H1-3]
	Trong không gian $O x y z$, cho mặt phẳng $(P)\colon 3 x-y+2 z-1=0$. véctơ nào dưới đây \textbf{không phải} là một véctơ pháp tuyến của $(P)$?
	\choice
	{$\overrightarrow{n}=(-3; 1;-2)$}
	{\True $\overrightarrow{n}=(3; 1; 2)$}
	{$\overrightarrow{n}=(3;-1; 2)$}
	{$\overrightarrow{n}=(6;-2; 4)$}
	\loigiai{
		Véctơ pháp tuyến của $(P)$ là $\overrightarrow{n}=(3;-1; 2)$.\\
		$\overrightarrow{n}=(-3; 1;-2)=-1(3;-1; 2)$ là một véctơ pháp tuyến của $(P)$.\\
		$\overrightarrow{n}=(6;-2; 4)=2(3;-1; 2)$ là một véctơ pháp tuyến của $(P)$. 	
	}
\end{ex}

\begin{ex}%[2H5H1-3]
	Trong không gian với hệ tọa độ $O x y z$, véctơ nào dưới đây là một véctơ pháp tuyến của mặt phẳng $(O x y)$?
	\choice
	{$\overrightarrow{i}=(1; 0; 0)$}
	{$\overrightarrow{m}=(1; 1; 1)$}
	{$\overrightarrow{j}=(0; 1; 0)$}
	{\True $\overrightarrow{k}=(0; 0; 1)$}
	\loigiai{
		Do mặt phẳng $(O x y)$ vuông góc với trục $O z$ nên nhận véctơ $\overrightarrow{k}=(0; 0; 1)$ làm một véctơ pháp tuyến.	
	}
\end{ex}

\begin{ex}%[2H5H1-3]
	Trong không gian $O x y z$, véctơ nào dưới đây có giá vuông góc với mặt phẳng $(\alpha)\colon 2 x-3 y+1=0$?
	\choice
	{$\overrightarrow{a}=(2;-3; 1)$}
	{$\overrightarrow{b}=(2; 1;-3)$}
	{\True $\overrightarrow{c}=(2;-3; 0)$}
	{$\overrightarrow{d}=(3; 2; 0)$}
	\loigiai{
		Mặt phẳng $(\alpha)$ có một véctơ pháp tuyến là $\overrightarrow{n}=(2;-3; 0)=\overrightarrow{c}$.	
	}
\end{ex}

\begin{ex}%[2H5H1-3]
	Trong không gian $O x y z$, một véctơ pháp tuyến của mặt phẳng $\dfrac{x}{-2}+\dfrac{y}{-1}+\dfrac{z}{3}=1$ là
	\choice
	{\True $\overrightarrow{n}=(3; 6;-2)$}
	{$\overrightarrow{n}=(2;-1; 3)$}
	{$\overrightarrow{n}=(-3;-6;-2)$}
	{$\overrightarrow{n}=(-2;-1; 3)$}
	\loigiai{
		Phương trình $\dfrac{x}{-2}+\dfrac{y}{-1}+\dfrac{z}{3}=1 \Leftrightarrow-\dfrac{1}{2} x-y+\dfrac{1}{3} z-1=0 \Leftrightarrow 3 x+6 y-2 z+6=0.$\\
		Do đó mặt phẳng đã cho có một véctơ pháp tuyến là $\overrightarrow{n}=(3; 6;-2)$.	
	}
\end{ex}

\begin{ex}%[2H5H1-3]
	Trong không gian $O x y z$, điểm nào dưới đây nằm trên mặt phẳng $(P)\colon 2 x-y+z-2=0$.
	\choice
	{$Q(1;-2; 2)$}
	{$P(2;-1;-1)$}
	{$M(1; 1;-1)$}
	{\True $N(1;-1;-1)$}
	\loigiai{
		Thay toạ độ điểm $Q$ vào phương trình mặt phẳng $(P)$ ta được $2\cdot 1-(-2)+2-2=4 \neq 0$ nên $Q \notin(P)$.\\
		Thay toạ độ điểm $P$ vào phương trình mặt phẳng $(P)$ ta được $2\cdot2-(-1)+(-1)-2=2 \neq 0$ nên $P \notin(P)$.\\
		Thay toạ độ điểm $M$ vào phương trình mặt phẳng $(P)$ ta được $2\cdot1-1+(-1)-2=-2 \neq 0$ nên $M \notin(P)$.\\
		Thay toạ độ điểm $N$ vào phương trình mặt phẳng $(P)$ ta được $2 \cdot 1-(-1)+(-1)-2=0$ nên $N \in(P)$.	
	}
\end{ex}

\begin{ex}%[2H5H1-3]
	Trong không gian với hệ tọa độ $O x y z$, cho mặt phẳng $(\alpha)\colon x+y+z-6=0$. Điểm nào dưới đây \textbf{không thuộc} $(\alpha)$?
	\choice
	{$Q(3; 3; 0)$}
	{$N(2; 2; 2)$}
	{$P(1; 2; 3)$}
	{\True $M(1;-1; 1)$}
	\loigiai{
		\begin{itemize}
			\item  Thay $Q(3; 3; 0)$  vào phương trình mặt phẳng $(\alpha)$, ta được $3+3+0-6=0 \Rightarrow Q \in(\alpha)$.
			\item  Thay $N(2; 2; 2)$ vào phương trình mặt phẳng $(\alpha)$, ta được  $2+2+2-6=0 \Rightarrow N \in(\alpha)$.
			\item Thay $P(1; 2; 3)$ vào phương trình mặt phẳng $(\alpha)$, ta được $1+2+3-6=0 \Rightarrow P \in(\alpha)$.
			\item  Thay $M(1;-1; 1)$ toạ độ vào phương trình mặt phẳng $(\alpha)$, ta được $1-1+1-6\neq 0 \Rightarrow M \notin(\alpha)$.	 
		\end{itemize}
	}
\end{ex}

\begin{ex}%[2H5H1-3]
	Trong không gian với hệ tọa độ $O x y z$, cho mặt phẳng $(P)\colon x-2 y+z-5=0$. Điểm nào dưới đây thuộc $(P)$?
	\choice
	{$P(0; 0;-5)$}
	{\True $M(1; 1; 6)$}
	{$Q(2;-1; 5)$}
	{$N(-5; 0; 0)$}
	\loigiai{
		Ta có $1-2 \cdot 1+6-5=0$ nên $M(1; 1; 6)$ thuộc mặt phẳng $(P)$.	
	}
\end{ex}

\begin{ex}%[2H5H1-3]
	Trong không gian $O x y z$, mặt phẳng $(P)\colon \dfrac{x}{1}+\dfrac{y}{2}+\dfrac{z}{3}=1$ \textbf{không} đi qua điểm nào dưới đây?
	\choice
	{$P(0; 2; 0)$}
	{\True $N(1; 2; 3)$}
	{$M(1; 0; 0)$}
	{$Q(0; 0; 3)$}
	\loigiai{
		Thế tọa độ điểm $N$ vào phương trình mặt phẳng $(P)$ ta có $\dfrac{1}{1}+\dfrac{2}{2}+\dfrac{3}{3}=1$ (sai).\\
		Vậy mặt phẳng $(P)\colon \dfrac{x}{1}+\dfrac{y}{2}+\dfrac{z}{3}=1$ không đi qua điểm $N(1; 2; 3)$.	
	}
\end{ex}

\begin{ex}%[2H5H1-3]
	Trong không gian $O x y z$, mặt phẳng $(\alpha)\colon x-y+2 z-3=0$ đi qua điểm nào dưới đây?
	\choice
	{\True $M\left(1; 1; \dfrac{3}{2}\right)$}
	{$N\left(1;-1;-\dfrac{3}{2}\right)$}
	{$P(1; 6; 1)$}
	{$Q(0; 3; 0)$}
	\loigiai{
		Xét điểm $M\left(1; 1; \dfrac{3}{2}\right)$, ta có $1-1+2 \cdot \dfrac{3}{2}-3=0$ (đúng) nên $M \in(\alpha)$ .\\
		Xét điểm $N\left(1;-1;-\dfrac{3}{2}\right)$, ta có $1+1+2.\left(-\dfrac{3}{2}\right)-3=0$ (sai) nên $N \notin(\alpha)$.\\
		Xét điểm $P(1; 6; 1)$, ta có $1-6+2.1-3=0$ (sai) nên $P \notin(\alpha)$.\\
		Xét điểm $Q(0; 3; 0)$, ta có $0-3+2.0-3=0$ (sai) nên $Q \notin(\alpha)$.	
	}
\end{ex}
\Closesolutionfile{ans}
\indapan{10}{ans/ans2C5B1CD1}
\TNTF
\Opensolutionfile{ans}[ans/ans2C5B1CD1-DS]
\begin{ex}%[2H5H1-2]
	Trong không gian cho hệ tọa độ $O x y z$. Các mệnh đề sau đây đúng hay sai?
	\choiceTF
	{\True Mặt phẳng $(O x y)$ có một véctơ pháp tuyến là $\overrightarrow{n}=(0; 0; 1)$}
	{\True Mặt phẳng $(O x z)$ có véctơ pháp tuyến là $\overrightarrow{n}=(0; 3; 0)$}
	{\True Mặt phẳng $(O y z)$ có véctơ pháp tuyến là $\overrightarrow{n}=(-2; 0; 0)$}
	{\True Trục $O z$ có véctơ chỉ phương là $\overrightarrow{a}=(0; 0;-2024)$}
	\loigiai{
		\begin{itemchoice}
			\itemch Mặt phẳng $(O x y)$ có một véctơ pháp tuyến là $\overrightarrow{n}=(0; 0; 1)$.
			\itemch Mặt phẳng $(O x z)$ có véctơ pháp tuyến là $\overrightarrow{n}=(0; 3; 0)$.
			\itemch Mặt phẳng $(O y z)$ có véctơ pháp tuyến là $\overrightarrow{n}=(-2; 0; 0)$.
			\itemch Trục $O z$ có véctơ chỉ phương là $\overrightarrow{a}=(0; 0;-2024)$.
		\end{itemchoice}
	}
\end{ex}

\begin{ex}%[2H2H2-5]%[2H2H2-1] 
	Trong không gian với hệ toạ độ $O x y z$, cho $\overrightarrow{a}=(1;-2; 3)$ và $\overrightarrow{b}=(1; 1;-1)$. Các mệnh đề sau đây đúng hay sai? 
	\choiceTF
	{\True $\left|\overrightarrow{a}+\overrightarrow{b}\right|=3$}
	{\True $\overrightarrow{a} \cdot \overrightarrow{b}=-4$}
	{\True $\left|\overrightarrow{a}-\overrightarrow{b}\right|=5$}
	{$\left[\overrightarrow{a}, \overrightarrow{b}\right]=(-1;-4; 3)$}
	\loigiai{
		\begin{itemchoice}
			\itemch $\left|\overrightarrow{a}+\overrightarrow{b}\right|=\left|\overrightarrow{a}+\overrightarrow{b}\right|=\sqrt{(1+1)^2+(-2+1)^2+(3-1)^2}=\sqrt{4+1+4}=3$.
			\itemch $\overrightarrow{a} \cdot \overrightarrow{b}=1 \cdot 1+(-2) \cdot 1+3 \cdot(-1)=1-2-3=-4$.
			\itemch $\left|\overrightarrow{a}+\overrightarrow{b}\right|=\left|\overrightarrow{a}+\overrightarrow{b}\right|=\sqrt{(1-1)^2+(-2-1)^2+(3+1)^2}=\sqrt{0+9+16}=5$.
			\itemch 
			$\left[\overrightarrow{a}, \overrightarrow{b}\right]=\left(\left|\begin{array}{cc}-2 & 3 \\ 1 &-1\end{array}\right|;\left|\begin{array}{cc}3 & 1 \\-1 & 1\end{array}\right|;\left|\begin{array}{cc}1 &-2 \\ 1 & 1\end{array}\right|\right)=(-1; 4; 3)$.
		\end{itemchoice}
	}
\end{ex}
\begin{ex}%[2H2H2-4]%[2H2H2-1]
	Trong không gian với hệ trục tọa độ $O x y z$, cho ba véctơ $\overrightarrow{a}=(1; 2;-1), \overrightarrow{b}=(3;-1; 0), \overrightarrow{c}=(1;-5; 2)$. Các mệnh đề sau đây đúng hay sai?
	\choiceTF
	{$\overrightarrow{a}$ cùng phương với $\overrightarrow{b}$}
	{$\left[\overrightarrow{a}, \overrightarrow{b}\right] \cdot \overrightarrow{c}=0$}
	{$\overrightarrow{a}$ không cùng phương với $\overrightarrow{b}$}
	{$\overrightarrow{a}$ vuông góc với $\overrightarrow{b}$}
	\loigiai{   
		\begin{itemchoice}
			\itemch Ta có:
			$\left[\overrightarrow{a}, \overrightarrow{b}\right]=(-1;-3;-7) \neq \overrightarrow{0}$.
			\itemch Hai véctơ $\overrightarrow{a}, \overrightarrow{b}$ không cùng phương.
			\itemch $\left[\overrightarrow{a}, \overrightarrow{b}\right] \cdot \overrightarrow{c}=-1+15-14=0$.
			\itemch Ba véctơ $\overrightarrow{a}, \overrightarrow{b}, \overrightarrow{c}$ đồng phẳng.
		\end{itemchoice}
	}
\end{ex}
\begin{ex}%[2H5H1-2]
	Trong không gian $O x y z$, cho mặt phẳng $(P)\colon 2 x+3 y+z-2024=0$. Các mệnh đề sau đây đúng hay sai?
	\choiceTF
	{\True Mặt phẳng $(P)$ có một véctơ pháp tuyến là $\overrightarrow{n}=(2; 3; 1)$}
	{\True Mặt phẳng $(P)$ có véctơ pháp tuyến là $\overrightarrow{n}=(6; 9; 3)$}
	{\True Mặt phẳng $(P)$ có véctơ pháp tuyến là $\overrightarrow{n}=(-4;-6;-2)$}
	{Điểm $M(0; 0; 2024)$ không thuộc mặt phẳng $(P)$}
	\loigiai{
		\begin{itemchoice}
			\itemch Véctơ pháp tuyến của $(P)$ là $\overrightarrow{n}=(2; 3; 1)$.
			\itemch $\overrightarrow{n}=(6; 9; 3)=3(2; 3; 1).$
			\itemch $\overrightarrow{n}=(-4;-6;-2)=-2(2; 3; 1).$
			\itemch Thay điểm $M(0; 0; 2024)$ vào mặt phẳng $(P)\colon 2\cdot0+3\cdot 0+2024-2024=0 \Rightarrow M \in(P)$.
		\end{itemchoice}
	}
\end{ex}
\begin{ex}%[2H5H1-3] 
	Trong không gian $O x y z$, cho mặt phẳng $(P)\colon x+y+z-3=0$. Các mệnh đề sau đây đúng hay sai?
	\choiceTF
	{\True Điểm $M(-1;-1;-1)$ \textbf{không thuộc} mặt phẳng $(P)$}
	{\True Điểm $N(1; 1; 1)$ \textbf{thuộc} mặt phẳng $(P)$}
	{\True Điểm $K(-3; 0; 0)$ \textbf{không thuộc} mặt phẳng $(P)$}
	{Điểm $Q(0; 0;-3)$ \textbf{thuộc} mặt phẳng $(P)$}
	\loigiai{
		\begin{itemchoice}
			\itemch Điểm $M(-1;-1;-1)$ có tọa độ không thỏa mãn phương trình mặt phẳng $(P)$ nên $M \notin(P)$.
			\itemch Điểm $N(1; 1; 1)$ có tọa độ thỏa mãn phương trình mặt phẳng $(P)$ nên $N \in(P)$.
			\itemch Điểm $K(-3; 0; 0)$ có tọa độ không thỏa mãn phương trình mặt phẳng $(P)$ nên $K \notin(P)$.
			\itemch Điểm $Q(0; 0;-3)$ có tọa độ không thỏa mãn phương trình mặt phẳng $(P)$ nên $Q \notin(P)$.
		\end{itemchoice}
	}
\end{ex}
\Closesolutionfile{ans}
\indapan{2}{ans/ans2C5B1CD1-DS}
\TNSA
\Opensolutionfile{ans}[ans/ans2C5B1CD1-KQ]
\begin{ex}%[2H2H2-5]
	Trong không gian với hệ trục tọa độ $O x y z$, cho $A(0; 1;-1)$, $B(1; 1; 2)$ và $C(1;-1; 0)$. Biết  $\vec{u}=\left[\overrightarrow{B C}, \overrightarrow{B D}\right]$. Khi đó, độ dài của $\vec{u}$ bằng bao nhiêu?
	\shortans[0]{$4$}
	\loigiai{
		Ta có $\overrightarrow{B C}=(0;-2;-2)$ và  $\overrightarrow{B D}=(-1;-1;-1)$.\\
		Khi đó $\vec{u}=\left[\overrightarrow{B C}, \overrightarrow{B D}\right]=(0; 2;-2)$.\\
		Suy ra $\left|\vec{u}\right|=\sqrt{0^2+2^2+(-2)^2}=4$. 	
	}
\end{ex}

\begin{ex}%[2H2V2-5]
	Trong không gian với hệ trục tọa độ $Oxyz$, cho $A(2; 0; 2)$, $B(1;-1;-2)$ và $C(-1; 1; 0)$. Một véctơ $\overrightarrow{n}=(a; b; 2)$ có phương vuông góc với hai véctơ $\overrightarrow{AB}$ và $\overrightarrow{AC}$. Tính giá trị của $a+b$.
	\shortans[0]{$-8$}
	\loigiai{
		Ta có $\overrightarrow{A C}=(-3; 1;-2)$ và $\overrightarrow{A B}=(-1;-1;-4)$.\\
		Vì $\vec{n}$ có phương vuông góc với $\overrightarrow{AB}$ và $\overrightarrow{AC}$ nên $\vec{n}$ cùng phương với vectơ $\left[\overrightarrow{AB},\overrightarrow{AC}\right]=(-6;-10; 4)$.\\
		Suy ra $\overrightarrow{n}=(-3; -5; 2)$
		Vậy $a+b=-3-5=-8$.
	}
\end{ex}

\begin{ex}%[2H2V2-5]
	Hệ trục tọa độ $Oxyz$, cho bốn điểm $A(1;-2; 0)$, $B(2; 0; 3)$, $C(-2; 1; 3)$ và $D(0; 1; 1)$. Tính giá trị của phép tính $\left[\overrightarrow{AB}, \overrightarrow{AC}\right] \cdot \overrightarrow{AD}$.
	\shortans[0]{$-24$}
	\loigiai
	{
		Ta có $\overrightarrow{AB}=(1; 2; 3)$; $\overrightarrow{AC}=(-3; 3; 3)$; $\overrightarrow{A D}=(-1; 3; 1)$.\\
		Khi đó $\left[\overrightarrow{A B}, \overrightarrow{A C}\right]=(-3;-12; 9)$.\\
		Và $\left[\overrightarrow{A B}, \overrightarrow{A C}\right] \cdot \overrightarrow{A D}=(-3) \cdot(-1)+(-12) \cdot 3+9 \cdot 1=-24$.
	}
\end{ex}
\begin{ex}%[2H5H1-2] 
	Trong mặt phẳng tọa độ $O x y z$, mặt phẳng $(P)\colon 2 x-6 y-8 z+1=0$ có một véctơ pháp tuyến $\vec{n}=(1;a;b)$. Khi đó tổng $a+b$ bằng bao nhiêu? 
	\shortans[0]{$-7$}
	\loigiai
	{
		Phương trình tổng quát của mặt phẳng $(P)\colon 2 x-6 y-8 z+1=0$ nên một véctơ pháp tuyến của mặt phẳng $(P)$ có tọa độ là $(2;-6;-8)=2\cdot (1;-3;-4)$.\\
		Suy ra $\vec{n}=(1;-3;-4)$, nên $a+b=-3-4=-7$.	
	}
\end{ex}

\begin{ex}%[2H2V2-5]
	Trong không gian với hệ tọa độ $O x y z$, cho $\overrightarrow{u}=(1; 1; 2), \overrightarrow{v}=(-1; m; m-2)$. Tìm giá trị của $m$ dương sao cho $|[\overrightarrow{u}, \overrightarrow{v}]|=\sqrt{14}$.
	\shortans[0]{$1$}
	\loigiai
	{ Ta có {\allowdisplaybreaks
			\begin{eqnarray*}
				&& [\overrightarrow{u}, \overrightarrow{v}]=(-m-2;-m; m+1)\\ &\Rightarrow& |[\overrightarrow{u}, \overrightarrow{v}]|=\sqrt{(m+2)^2+m^2+(m+1)^2}=\sqrt{3 m^2+6 m+5}.
		\end{eqnarray*}}
		Khi đó $$|[\overrightarrow{u}, \overrightarrow{v}]|=\sqrt{14} \Leftrightarrow 3 m^2+6 m+5=14 \Leftrightarrow 3 m^2+6 m-9=0 \Leftrightarrow \hoac{&m=1 \\&m=-3.}$$
		
	}
\end{ex}

\begin{ex}%[2H2V2-5]
	Trong không gian với hệ tọa độ $O x y z$, cho hai véctơ $\overrightarrow{m}=(4; 3; 1), \overrightarrow{n}=(0; 0; 1)$. Gọi $\overrightarrow{p}=\left(a;b;c\right)$ là véctơ cùng hướng với $[\overrightarrow{m}, \overrightarrow{n}]$ (tích có hướng của hai véctơ $\overrightarrow{m}$ và $\overrightarrow{n}$). Biết $|\overrightarrow{p}|=15$, giá trị của tổng $a+b+c$ bằng bao nhiêu?
	\shortans[0]{$3$}
	\loigiai
	{
		Ta có  $[\overrightarrow{m}; \overrightarrow{n}]=(3;-4; 0)$, suy ra $|[\overrightarrow{m}; \overrightarrow{n}]|=5$.\\
		Do $\overrightarrow{p}$ là véctơ cùng hướng với $[\overrightarrow{m}; \overrightarrow{n}]$ nên $\overrightarrow{p}=k[\overrightarrow{m}; \overrightarrow{n}]$, $k>0$.\\
		Mặt khác $|\overrightarrow{p}|=15 \Leftrightarrow k \cdot|[\overrightarrow{m}, \overrightarrow{n}]| =15 \Leftrightarrow k\cdot 5=15 \Leftrightarrow k=3$.\\
		Suy ra $\overrightarrow{p}=(9;-12; 0)$.	\\
		Vậy $a+b+c=9-12+0=3$.
	}
\end{ex}
\Closesolutionfile{ans}
\indapan{6}{ans/ans2C5B1CD1-KQ}
\begin{dang}{Hai mặt phẳng song song, vuông góc. Khoảng cách một điểm đến mặt phẳng}
	\begin{enumerate}[label=\bf\arabic*.]
		\item \textbf{Điều kiện hai mặt phẳng song song, vuông góc:}\\
		Cho 2 mặt phẳng $\left(\alpha_1\right)\colon A_1 x+B_1 y+C_1 z+D_1=0$ và $\left(\alpha_2\right)\colon A_2 x+B_2 y+C_2 z+D_2=0$ có vectơ pháp tuyến lần lượt là $\overrightarrow{n}_1=\left(A_1; B_1; C_1\right), \overrightarrow{n}_2=\left(A_2; B_2; C_2\right)$. Khi đó:
		\begin{itemize}
			\item $\left(\alpha_1\right) \parallel \left(\alpha_2\right) \Leftrightarrow\heva{&\overrightarrow{n}_1=k \overrightarrow{n}_2 \\ &D_1 \neq k D_2} \quad (k \in \mathbb{R})$.
			\item $\left(\alpha_1\right) \equiv\left(\alpha_2\right) \Leftrightarrow\heva{&\overrightarrow{n}_1=k \overrightarrow{n}_2 \\& D_1=k D_2} \quad (k \in \mathbb{R})$.
			\item $\left(\alpha_1\right)$ cắt $\left(\alpha_2\right) \Leftrightarrow \overrightarrow{n}_1$ và $\overrightarrow{n}_2$ không cùng phương.
			\item $\left(\alpha_1\right) \perp\left(\alpha_2\right) \Leftrightarrow \overrightarrow{n}_1 \cdot \overrightarrow{n}_2=0 \Leftrightarrow A_1 A_2+B_1 B_2+C_1 C_2=0$. 	
		\end{itemize}
		\begin{tikzpicture}[line cap=round,line join=round,>=stealth,x=1.0cm,y=1.0cm,scale=0.6]
			\path
			(1,1) coordinate (A)
			(3,3) coordinate (B)
			(8,3) coordinate (C)
			($(A)+(C)-(B)$) coordinate (D)
			(2,4) coordinate (A')
			(4,6) coordinate (B')
			(9,6) coordinate (C')
			($(A')+(C')-(B')$) coordinate (D')
			(5,5) coordinate (K)
			(6,5) coordinate (M)
			($(K)+(0,1.5)$) coordinate (N)
			(6,2) coordinate (H)
			($(C)!0.5!(D)$) coordinate (H')
			;
			\draw (A)--(B)--(C)--(D)--cycle (A')--(B')--(C')--(D')--cycle ;
			\draw[->] (K)--(N) node[right]{$\vec{n}_1$};
			\draw[->] (H)--($(H)+(0,1.5	 )$) node[right]{$\vec{n}_2$};	
			\draw pic[draw,blue,"$\alpha_1$",angle radius=8mm]{angle=D--A--B};
			\draw pic[draw,blue,"$\alpha_2$",angle radius=8mm]{angle=D'--A'--B'};
			\draw pic[draw,blue,,angle radius=3mm]{right angle=M--K--N};
			\draw pic[draw,blue,,angle radius=3mm]{right angle=M--H--H'};
		\end{tikzpicture}
		\begin{tikzpicture}[line cap=round,line join=round,>=stealth,x=1.0cm,y=1.0cm,scale=0.6]
			\path
			(1,1) coordinate (A)
			(3,3) coordinate (B)
			(8,3) coordinate (C)
			($(A)+(C)-(B)$) coordinate (D)
			($(A)+(-2,3)$) coordinate (E)
			($(B)+(-2,3)$) coordinate (F)
			($(A)!0.5!(C)$) coordinate (G)
			($(G)+(0,1.5)$) coordinate (H)
			($(A)!0.5!(F)$) coordinate (I)
			($(I)+(1.7,1.5)$) coordinate (J)
			($(C)!0.5!(D)$) coordinate (K)
			($(A)!0.5!(B)$) coordinate (L)
			;
			\draw (A)--(B)--(C)--(D)--cycle (A)--(E)--(F)--(B) ;
			\draw[->] (G)--(H) node[right]{$\vec{n}_1$};
			\draw[->] (I)--(J) node[right]{$\vec{n}_2$};	
			\draw pic[draw,blue,"$\alpha_1$",angle radius=8mm]{angle=B--C--D};
			\draw pic[draw,blue,"$\alpha_2$",angle radius=5mm]{angle=A--E--F};
			\draw pic[draw,blue,,angle radius=3mm]{right angle=L--I--J};
			\draw pic[draw,blue,,angle radius=3mm]{right angle=H--G--K};
		\end{tikzpicture}
		\begin{tikzpicture}[line cap=round,line join=round,>=stealth,x=1.0cm,y=1.0cm,scale=0.6]
			\path
			(1,1) coordinate (A)
			(3,3) coordinate (B)
			(8,3) coordinate (C)
			($(A)+(C)-(B)$) coordinate (D)
			($(A)+(0,4)$) coordinate (E)
			($(B)+(0,4.)$) coordinate (F)
			($(A)!0.5!(C)$) coordinate (G)
			($(G)+(0,1.5)$) coordinate (H)
			($(A)!0.5!(F)$) coordinate (I)
			($(I)+(1.7,0)$) coordinate (J)
			($(C)!0.5!(D)$) coordinate (K)
			($(A)!0.5!(B)$) coordinate (L)
			;
			\draw (A)--(B)--(C)--(D)--cycle (A)--(E)--(F)--(B) ;
			\draw[->] (G)--(H) node[right]{$\vec{n}_1$};
			\draw[->] (I)--(J) node[above]{$\vec{n}_2$};	
			\draw pic[draw,blue,"$\alpha_1$",angle radius=8mm]{angle=B--C--D};
			\draw pic[draw,blue,"$\alpha_2$",angle radius=5mm]{angle=A--E--F};
			\draw pic[draw,blue,,angle radius=3mm]{right angle=L--I--J};
			\draw pic[draw,blue,,angle radius=3mm]{right angle=H--G--K};
		\end{tikzpicture}
		\begin{note}
			\textbf{Chú ý:}
			\begin{itemize}
				\item $\overrightarrow{a}$ cùng phương với $\overrightarrow{b} \Leftrightarrow[\overrightarrow{a}, \overrightarrow{b}]=\overrightarrow{0}$.
				\item Nếu $\overrightarrow{n}=[\overrightarrow{a}, \overrightarrow{b}]$ thì vectơ $\overrightarrow{n}$ vuông góc với cả hai vectơ $\overrightarrow{a}$ và $\overrightarrow{b}$.
			\end{itemize}
		\end{note}
		\item \textbf{Khoảng cách từ một điểm đến một mặt phẳng}
		\immini{
			Trong không gian $O x y z$, cho điểm $M_0\left(x_0; y_0; z_0\right)$ và mặt phẳng $(\alpha)\colon A x+B y+C z+D=0$. Khi đó khoảng cách từ điểm $M_0$ đến mặt phẳng $(\alpha)$ được tính: $$d\left(M_0,(\alpha)\right)=\dfrac{\left|A x_0+B y_0+C z_0+D\right|}{\sqrt{A^2+B^2+C^2}}.$$
		}{
			\begin{tikzpicture}[line cap=round,line join=round,>=stealth,x=1.0cm,y=1.0cm,scale=0.6]
				\path
				(1,1) coordinate (A)
				(3,3) coordinate (B)
				(9,3) coordinate (C)
				($(A)+(C)-(B)$) coordinate (D)
				(4,2) coordinate (E)
				(6,2) coordinate (F)
				($(E)+(0,2.5)$) coordinate (G)
				;
				\draw (A)--(B)--(C)--(D)--cycle ;
				\draw[->] (E)--($(E)+(0,2.5)$) node[right]{$M_0$};
				\draw[->] (F)--($(F)+(0,1.5)$) node[right]{$\vec{n}$};	
				\draw pic[draw,blue,"$\alpha$",angle radius=8mm]{angle=D--A--B};
				\draw pic[draw,blue,,angle radius=3mm]{right angle=F--E--G};
				%			\draw pic[draw,blue,,angle radius=3mm]{right angle=H--G--K};
			\end{tikzpicture}
		}
		\begin{note}
			\textbf{Chú ý:}
			\begin{itemize}
				\item Mặt phẳng $(O x y)$ có phương trình: $z=0$.
				\item Mặt phẳng $(O x z)$ có phương trình: $y=0$.
				\item Mặt phẳng $(O y z)$ có phương trình: $x=0$.
			\end{itemize}
		\end{note}
		\item \textbf{Khoảng cách hai mặt phẳng song song}\\
		Khoảng cách giữa mặt phẳng song song là khoảng cách từ một điểm thuộc mặt phẳng này đến mặt phẳng kia (Thực chất là khoảng cách từ một điểm đến mặt phẳng).\\
		Để tính khoảng cách mặt phẳng $\left(\alpha_1\right)$ song song với $\left(\alpha_2\right)$, ta thực hiện như sau:
		\begin{enumerate}
			\item[] \textbf{Bước 1:} Chọn điểm $M \in\left(\alpha_1\right)$.
			\item[] \textbf{Bước 2:} Tính khoảng cách điểm $M$ đến $\left(\alpha_2\right)$.
			\item[] \textbf{Bước 3:} Kết luận: $d\left(\left(\alpha_1\right),\left(\alpha_2\right)\right)=d\left(M,\left(\alpha_2\right)\right)$.
		\end{enumerate}
		\begin{note}
			\textbf{Chú ý:}
			Cho 2 mặt phẳng $\left(\alpha_1\right)\colon A x+B y+C z+D_1=0$ và $\left(\alpha_2\right)\colon A x+B y+C z+D_2=0$ có cùng vectơ pháp tuyến là $\overrightarrow{n}=(A; B; C)$. Khi đó khoảng cách giữa hai mặt phẳng đó là: $$d\left(\left(\alpha_1\right),(\alpha_2)\right)=\dfrac{\left|D_1-D_2\right|}{\sqrt{A^2+B^2+C^2}}.$$ 
		\end{note}
	\end{enumerate}
\end{dang}
\textbf{Khoảng cách hai mặt phẳng song song}
\begin{itemize}
	\item Khoảng cách giữa mặt phẳng song song là khoảng cách từ một điểm thuộc mặt phẳng này đến mặt phẳng kia (Thực chất là khoảng cách từ một điểm đến mặt phẳng).
	\item Để tính khoảng cách mặt phẳng $(\alpha_1)$ song song với $(\alpha_2)$, ta thực hiện như sau:
	\begin{enumEX}[\hspace*{1cm}\bf Bước 1:]{1}
		\item Chọn điểm $M\in (\alpha_1)$
		\item Tính khoảng cách điểm $M$ đến $(\alpha_2)$
		\item Kết luận $\mathrm{d}\left((\alpha_1),(\alpha_2)\right)=\mathrm{d}\left(M,(\alpha_2)\right)$
	\end{enumEX}
	\textbf{Chú ý:} Cho 2 mặt phẳng $(\alpha_1)\colon Ax+By+Cz+D_1=0$ và $(\alpha_2)\colon Ax+By+Cz+D_2=0$ có cùng vectơ pháp tuyến là $\vec{n}=(A;B;C)$.\\
	Khi đó khoảng cách giữa hai mặt phẳng đó là: $\mathrm{d}((\alpha_1),(\alpha))=\dfrac{|D_1-D_2|}{\sqrt{A^2+B^2+C^2}}$.
\end{itemize}
\TN
\Opensolutionfile{ans}[ans/ans2C5B1CD1-D2]
%%==========Câu 27
\begin{ex}%[Câu 2]%[2H5N1-5]
	Khoảng cách từ điểm $M\left(3;2;1\right)$ đến mặt phẳng $(P)\colon Ax+Cz+D=0$, $A.C.D\ne 0$. Chọn khẳng định đúng trong các khẳng định sau:
	\choice
	{\True $\mathrm{d}(M,(P))=\dfrac{\left| 3A+C+D\right|}{\sqrt{A^2+C^2}}$}
	{$\mathrm{d}(M,(P))=\dfrac{\left| A+2B+3C+D\right|}{\sqrt{A^2+B^2+C^2}}$}
	{$\mathrm{d}(M,(P))=\dfrac{\left| 3A+C\right|}{\sqrt{A^2+C^2}}$}
	{$\mathrm{d}(M,(P))=\dfrac{\left| 3A+C+D\right|}{\sqrt{3^2+1^2}}$}
	\loigiai{
		Áp dung công thức $\mathrm{d}(M_0,(\alpha))=\dfrac{\left |Ax_0+By_0+Cz_0+D\right |}{\sqrt{A^2+B^2+C^2}}$.\\
		Ta được: $\mathrm{d}(M,(P))=\dfrac{\left| 3A+C+D\right|}{\sqrt{A^2+C^2}}$.}
\end{ex}

%%==========Câu 28
\begin{ex}%[Câu 3]%[2H5N1-5]
	Trong không gian với hệ tọa độ $Oxyz$, cho mặt phẳng $(P)$ có phương trình: $3x+4y+2z+4=0$ và điểm $A(1;-2;3)$. Tính khoảng cách $\mathrm{d}$ từ $A$ đến $(P)$.
	\choice
	{$\mathrm{d}=\dfrac{5}{9}$}
	{$\mathrm{d}=\dfrac{5}{29}$}
	{\True $\mathrm{d}=\dfrac{5}{\sqrt{29}}$}
	{$\mathrm{d}=\dfrac{\sqrt{5}}{3}$}
	\loigiai{
		Khoảng cách $\mathrm{d}$ từ $A$ đến $(P)$ là $$\mathrm{d}(A,(P))=\dfrac{\left| 3x_A+4y_A+2z_A+4\right|}{\sqrt{3^2+4^2+2^2}}=\dfrac{\left| 3-8+6+4\right|}{\sqrt{29}}=\dfrac{5}{\sqrt{29}}.$$}
\end{ex}

%%==========Câu 29
\begin{ex}%[Câu 4]%[2H5N1-5]
	Trong không gian $Oxyz$, cho mặt phẳng $(P)\colon 2x-2y+z-1=0$. Khoảng cách từ điểm $M\left(-1;2;0\right)$ đến mặt phẳng $(P)$ bằng
	\choice
	{$5$}
	{$2$}
	{\True $\dfrac{5}{3}$}
	{$\dfrac{4}{3}$}
	\loigiai{
		Ta có: $\mathrm{d}\left(M,(P)\right)=\dfrac{\left| 2\cdot\left(-1\right)-2\cdot2+0-1\right|}{\sqrt{2^2+\left(-2\right)^2+1^2}}=\dfrac{5}{3}$.}
\end{ex}

%%==========Câu 30
\begin{ex}%[Câu 5]%[2H5N1-5]
	Trong không gian $Oxyz$, tính khoảng cách từ $M\left(1;2;-3\right)$ đến mặt phẳng $(P)\colon x+2y+2z-10=0$.
	\choice
	{\True $\dfrac{11}{3}$}
	{$3$}
	{$\dfrac{7}{3}$}
	{$\dfrac{4}{3}$}
	\loigiai{
		Ta có: $\mathrm{d}\left(M;(P)\right)=\dfrac{\left| 1+2\cdot 2+2\cdot\left(-3\right)-10\right|}{\sqrt{1^2+2^2+2^2}}=\dfrac{\left| -11\right|}{3}=\dfrac{11}{3}$.}
\end{ex}

%%==========Câu 31
\begin{ex}%[Câu 6]%[2H5H1-5]
	Trong không gian $Oxyz$, cho mặt phẳng $(P)\colon 2x-y+2z-4=0$. Gọi $H$ là hình chiếu vuông góc của điểm $M\left(3;1;-2\right)$ lên mặt phẳng $(P)$. Độ dài đoạn thẳng $MH$ là
	\choice
	{$2$}
	{$\dfrac{1}{3}$}
	{\True $1$}
	{$3$}
	\loigiai{
		Độ dài đoạn thẳng $MH$ là $MH=\mathrm{d}\left(M,(P)\right)=\dfrac{\left| 2\cdot 3-1+2\cdot (-2)-4\right|}{\sqrt{2^2+(-1)^2+2^2}}=1$.}
\end{ex}

%%==========Câu 32
\begin{ex}%[Câu 7]%[2H5H1-5]
	Trong không gian với hệ trục tọa độ $Oxyz$, gọi $H$ là hình chiếu vuông góc của điểm $A(1;-2;3)$ lên mặt phẳng $(P)\colon 2x-y-2z+5=0$. Độ dài đoạn thẳng $AH$ bằng
	\choice
	{$3$}
	{$7$}
	{$4$}
	{$1$}
	\loigiai{
		Độ dài đoạn thẳng $AH$ là $AH=\mathrm{d}\left(A,(P)\right)=\dfrac{\left| 2+2-6+5\right|}{\sqrt{2^2+(-1)^2+(-2)^2}}=1$.}
\end{ex}

%%==========Câu 33
\begin{ex}%[Câu 8]%[2H5H1-5]
	Khoảng cách từ điểm $M(-4;-5;6)$ đến mặt phẳng $(Oxy)$, $(Oyz)$ lần lượt bằng
	\choice
	{\True $6$ và $4$}
	{$6$ và $5$}
	{$5$ và $4$}
	{$4$ và $6$}
	\loigiai{
		Ta có: $\mathrm{d}\left(M,(Oxy)\right)=\left|z_M\right|=6$ và $\mathrm{d}(M,(Oyz))=\left|x_M\right|=4$.}
\end{ex}

%%==========Câu 34
\begin{ex}%[Câu 9]%[2H5H1-5]
	Tính khoảng cách $\mathrm{d}$ từ điểm $B\left(x_0;y_0;z_0\right)$ đến mặt phẳng $(P)\colon y + 1 = 0$ ta được:
	\choice
	{$y_0$}
	{$\left| y_0\right|$}
	{$\dfrac{\left| y_0+1\right|}{\sqrt{2}}$}
	{\True $\left| y_0+1\right|$}
	\loigiai{
		Ta có: $\mathrm{d}\left (M,(P)\right )=\dfrac{\left |y_0+1\right |}{\sqrt{1^2}}=\left| y_0+1\right|$.
	}
\end{ex}

%%==========Câu 35
\begin{ex}%[Câu 10]%[2H5H1-5]
	Khoảng cách từ điểm $C(-2;0;0)$ đến mặt phẳng $(Oxy)$ bằng
	\choice
	{\True $0$}
	{$2$}
	{$1$}
	{$\sqrt{2}$}
	\loigiai{
		Điểm $C$ thuộc mặt phẳng $(Oxy)$ nên $\mathrm{d}\left(C,(Oxy)\right)=0$.}
\end{ex}

%%==========Câu 36
\begin{ex}%[Câu 11]%[2H5H1-5]
	Trong không gian $Oxyz$, khoảng cách giữa hai mặt phẳng $(P)\colon x+2y+2z-10=0$ và $(Q)\colon x+2y+2z-3=0$ bằng
	\choice
	{$\dfrac{4}{3}$}
	{$\dfrac{8}{3}$}
	{\True $\dfrac{7}{3}$}
	{$3$}
	\loigiai{
		Ta có $\dfrac{1}{1}=\dfrac{2}{2}=\dfrac{2}{2}\ne \dfrac{-10}{-3}$ nên $(P)\parallel (Q)$.\\
		Lấy $A\left(2;1;3\right)\in \left(P\right)$. 
		Ta có: $\mathrm{d}\left(\left(P\right),\left(Q\right)\right)=\mathrm{d}\left(A,\left(Q\right)\right)=\dfrac{\left| 2+2\cdot 1+2\cdot3-3\right|}{\sqrt{1^2+2^2+2^2}}=\dfrac{7}{3}$.}
\end{ex}

%%==========Câu 37
\begin{ex}%[Câu 12]%[2H5H1-5]
	Trong không gian $Oxyz$, khoảng cách giữa hai mặt phẳng $(P)\colon x+2y+3z-1=0$ và $(Q)\colon x+2y+3z+6=0$ là
	\choice
	{\True $\dfrac{7}{\sqrt{14}}$}
	{$\dfrac{8}{\sqrt{14}}$}
	{$14$}
	{$\dfrac{5}{\sqrt{14}}$}
	\loigiai{
		Ta có $\dfrac{1}{1}=\dfrac{2}{2}=\dfrac{3}{3}\ne \dfrac{-1}{6}$ nên $(P)\parallel (Q)$.\\
		Khi đó: $\mathrm{d}\left((P);(Q)\right)$ =$\dfrac{\left| D_2-D_1\right|}{\sqrt{A^2+B^2+C^2}}
		=\dfrac{\left| -1-6\right|}{\sqrt{1^2+2^2+3^2}}=\dfrac{7}{\sqrt{14}}$.}
\end{ex}

%%==========Câu 38
\begin{ex}%[Câu 13]%[2H5H1-5]
	Trong không gian $Oxyz$, khoảng cách giữa hai mặt phẳng $(P)\colon x+2y+2z-8=0$ và $(Q)\colon x+2y+2z-4=0$ bằng
	\choice
	{$1$}
	{\True $\dfrac{4}{3}$}
	{$2$}
	{$\dfrac{7}{3}$}
	\loigiai{
		Ta có $\dfrac{1}{1}=\dfrac{2}{2}=\dfrac{2}{2}\ne \dfrac{-8}{-4}$ nên $(P)\parallel (Q)$.\\
		Khi đó: $\mathrm{d}\left((P);(Q)\right)=\dfrac{\left| -8-(-4)\right|}{\sqrt{1^2+2^2+2^2}}=\dfrac{4}{3}$.\\
	}
\end{ex}

%%==========Câu 39
\begin{ex}%[Câu 14]%[2H5H1-4]
	Trong không gian $Oxyz$, mặt phẳng $(P)\colon 2x+y+z-2=0$ vuông góc với mặt phẳng nào dưới đây?
	\choice
	{$2x-y-z-2=0$}
	{\True $x-y-z-2=0$}
	{$x+y+z-2=0$}
	{$2x+y+z-2=0$}
	\loigiai{
		Mặt phẳng $(P)$ có một vectơ pháp tuyến $\overrightarrow{n_P}=\left(2;1;1\right)$.\\
		Mặt phẳng $(Q)\colon x-y-z-2=0$ có một vectơ pháp tuyến $\overrightarrow{n_Q}=\left(1;-1;-1\right)$.\\
		Mà $\overrightarrow{n_P}\cdot\overrightarrow{n_Q}=2-1-1=0\Rightarrow \overrightarrow{n_P}\perp \overrightarrow{n_Q}\Rightarrow (P)\perp (Q)$.\\
		Vậy mặt phẳng $(Q)\colon x-y-z-2=0$ là mặt phẳng cần tìm.}
\end{ex}

%%==========Câu 40
\begin{ex}%[Câu 15]%[2H5H1-4]
	Trong không gian với hệ tọa độ $Oxyz$, cho hai mặt phẳng $(P)\colon 2x+my+3z-5=0$ và $(Q)\colon nx-8y-6z+2=0$, với $m,n\in \mathbb{R}$. Xác định $m,n$ để $(P)$ song song với $(Q)$.
	\choice
	{$m=n=-4$}
	{\True $m=4;n=-4$}
	{$m=- 4;n=4$}
	{$m=n=4$}
	\loigiai{
		Mặt phẳng $(P)$ có véc tơ pháp tuyến $\vec{n_1}=(2;m;3)$.\\
		Mặt phẳng $(Q)$ có véc tơ pháp tuyến $\vec{n_2}=(n;-8;-6)$.\\
		Mặt phẳng $(P)\parallel (Q)\Rightarrow \vec{n_1}=k\cdot \vec{n_2}\, (k\in \mathbb{R})\Leftrightarrow \heva{&2=kn \\&m=- 8k \\&3=- 6k}\Leftrightarrow \heva{&k=-\dfrac{1}{2} \\&m=4 \\&n=- 4.}$}
\end{ex}

%%==========Câu 41
\begin{ex}%[Câu 16]%[2H5H1-4]
	Trong không gian $Oxyz$, cho hai mặt phẳng $(P)\colon x-2y+2z-3=0$ và $(Q)\colon mx+y-2z+1=0$. Với giá trị nào của $m$ thì hai mặt phẳng đó vuông góc với nhau?
	\choice
	{$m=1$}
	{$m=-1$}
	{$m=-6$}
	{\True $m=6$}
	\loigiai{
		Ta có: $(P)\perp (Q)\Leftrightarrow 1\cdot m-2\cdot 1+2\cdot (-2)=0\Leftrightarrow m=6$.}
\end{ex}

%%==========Câu 42
\begin{ex}%[Câu 17]%[2H5V1-4]
	Trong không gian $Oxyz$, cho ba mặt phẳng $(P)\colon x+y+z-1=0$, $(Q)\colon 2x+my+2z+3=0$ và $(R)\colon -x+2y+nz=0$. Tính tổng $m+2n$, biết rằng $(P)\perp (R)$ và $(P)\parallel (Q)$.
	\choice
	{$-6$}
	{$1$}
	{\True $0$}
	{$6$}
	\loigiai{
		$(P)$ có vectơ pháp tuyến $\vec{a}=(1;1;1)$.\\
		$(Q)$ có vectơ pháp tuyến $\vec{b}=(2;m;2)$.\\
		$(R)$ có vectơ pháp tuyến $\vec{c}=(-1;2;n)$.\\
		Ta có: $(P)\perp (R)\Leftrightarrow \vec{a}\cdot \vec{c}=0\Leftrightarrow n=-1$.\\
		$(P)\parallel (Q)\Leftrightarrow \dfrac{2}{1}=\dfrac{m}{1}=\dfrac{2}{1}\Leftrightarrow m=2$.\\
		Vậy $m+2n=2+2\left(-1\right)=0$}
\end{ex}

%%==========Câu 43
\begin{ex}%[Câu 18]%[2H5V1-4]
	Trong không gian $Oxyz$, cho $(P)\colon x+y-2z+5=0$ và $(Q)\colon 4x+(2-m)y+mz-3=0$, $m$ là tham số thực. Tìm tham số $m$ sao cho mặt phẳng $(Q)$ vuông góc với mặt phẳng $(P)$.
	\choice
	{$m=-3$}
	{$m=-2$}
	{$m=3$}
	{\True $m=2$}
	\loigiai{
		Mặt phẳng $(P)$ có véctơ pháp tuyến là $\vec{n_{P}}=(1;1;-2)$.\\
		Mặt phẳng $(Q)$ có véctơ pháp tuyến là $\vec{n_{Q}}=(4;2-m;m)$.\\
		Ta có $(P)\perp (Q)\Leftrightarrow \vec{n_{P}}\perp \vec{n_{Q}}\Leftrightarrow \vec{n_{P}}\cdot \vec{n_{Q}}=0\Leftrightarrow 4\cdot 1+2-m-2m=0\Leftrightarrow m=2$.
	}
\end{ex}

%%==========Câu 44
\begin{ex}%[Câu 19]%[2H5V1-4]
	Trong không gian $Oxyz$ cho hai mặt phẳng $(\alpha)\colon x+2y-z-1=0$ và $(\beta)\colon 2x+4y-mz-2=0$. Tìm $m$ để hai mặt phẳng $(\alpha)$ và $(\beta)$ song song với nhau.
	\choice
	{$m=1$}
	{\True Không tồn tại $m$}
	{$m=-2$}
	{$m=2$}
	\loigiai{
		Ta có vectơ pháp tuyến của $(\alpha)$ là $\overrightarrow{n_1}=(1;2;-1)$, vectơ pháp tuyến của $(\beta)$ là $\overrightarrow{n_2}=(2;4;-m)$.\\
		Hai mặt phẳng $(\alpha)$ và $(\beta)$ song song khi $\dfrac{2}{1}=\dfrac{4}{2}=\dfrac{-m}{-1}\ne \dfrac{-2}{-1}$.\\
		Vậy không có giá trị nào của $m$ thỏa mãn điều kiện trên.}
\end{ex}

%%==========Câu 45
\begin{ex}%[Câu 20]%[2H5V1-4]
	Trong không gian toạ độ $Oxyz$, cho mặt phẳng $(P)\colon x+2y-2z-1=0$, mặt phẳng nào dưới đây song song với $(P)$ và cách $(P)$ một khoảng bằng $3$.
	\choice
	{\True $(Q)\colon x+2y-2z+8=0$}
	{$(Q)\colon x+2y-2z+5=0$}
	{$(Q)\colon x+2y-2z+1=0$}
	{$(Q)\colon x+2y-2z+2=0$}
	\loigiai{
		+ Chọn $A\left(1;0;0\right)\in (P)$.\\
		+ Xét đáp án \textbf{A.}, ta có $\mathrm{d}\left(A;(Q)\right)=\dfrac{\left| 1+8\right|}{\sqrt{1^2+2^2+\left(-2\right)^2}}=3$.
	}
\end{ex}
\Closesolutionfile{ans}
\indapan{10}{ans/ans2C5B1CD1-D2}
\TNTF
\Opensolutionfile{ans}[ans/ans2C5B1CD1-D2-DS]
%%==========Câu 46
\begin{ex}%[Câu 21]%[2H5N1-5]
	Trong không gian toạ độ $Oxyz$, cho điểm $M\left(1;2;0\right)$ và các mặt phẳng $(Oxy)$, $(Oyz)$, $(Oxz)$. Các mệnh đề sau đây đúng hay \textbf{sai}?
	\choiceTF
	{\True $\mathrm{d}\left(M,(Oxz)\right)=2$}
	{\True $\mathrm{d}\left(M,(Oyz)\right)=1$}
	{$\mathrm{d}\left(M,(Oxy)\right)=1$}
	{\True $\mathrm{d}\left(M,(Oxz)\right)>d\left(M,(Oyz)\right)$}
	\loigiai{
		\begin{itemchoice}
			\itemch $\mathrm{d}\left(M,(Oxz)\right)=|2|=2$.	ĐÚNG
			\itemch $\mathrm{d}\left(M,(Oyz)\right)=|1|=1$. ĐÚNG
			\itemch $\mathrm{d}\left(M,(Oxy)\right)=|0|=0$.	SAI
			\itemch $\mathrm{d}\left(M,(Oxz)\right)>d\left(M,(Oyz)\right)$. ĐÚNG
		\end{itemchoice}
	}
\end{ex}
%%==========Câu 48
\begin{ex}%[Câu 23]%[2H5H1-4]
	Trong không gian $Oxyz$, cho hai mặt phẳng $(P)\colon x+2y-2z-6=0$ và $(Q)\colon x+2y-2z+3=0$. Các mệnh đề sau đây đúng hay \textbf{sai}?
	\choiceTF
	{\True Hai mặt phẳng $(P)$ và $(Q)$ song song với nhau}
	{Hai mặt phẳng $(P)$ và $(Q)$ vuông góc với nhau}
	{Khoảng cách giữa hai mặt phẳng $(P)$ và $(Q)$ bằng $2$}
	{\True Khoảng cách giữa hai mặt phẳng $(P)$ và $(Q)$ bằng $3$}
	\loigiai{
		\begin{itemize}
			\item Ta có: $\dfrac{1}{1}=\dfrac{2}{2}=\dfrac{-2}{-2}\ne \dfrac{-6}{3}$ nên $(P)\parallel (Q)$.
			\item $\mathrm{d}\left ((P),(Q)\right )=\dfrac{|-6-3|}{\sqrt{1^2+2^2+(-2)^2}}=3$.
		\end{itemize}
		\begin{itemchoice}
			\itemch Hai mặt phẳng $(P)$ và $(Q)$ song song với nhau. ĐÚNG
			\itemch Hai mặt phẳng $(P)$ và $(Q)$ vuông góc với nhau. SAI
			\itemch Khoảng cách giữa hai mặt phẳng $(P)$ và $(Q)$ bằng $2$. SAI
			\itemch Khoảng cách giữa hai mặt phẳng $(P)$ và $(Q)$ bằng $3$. ĐÚNG
		\end{itemchoice}
	}
\end{ex}

%%==========Câu 47
\begin{ex}%[2H5N1-5]%[2H5H1-5]
	Trong không gian toạ độ $Oxyz$, Biết khoảng cách từ điểm $O$ đến mặt phẳng $(Q)$ bằng 1. Các mệnh đề sau đây đúng hay \textbf{sai}?
	\choiceTF
	{Mặt phẳng $(Q)$ có phương trình là $x + y + z-3 = 0$}
	{\True Mặt phẳng $(Q)$ có phương trình là $2x + y + 2z-3 = 0$}
	{Mặt phẳng $(Q)$ có phương trình là $2x + y- 2z + 6 = 0$}
	{\True Mặt phẳng $(Q)$ có phương trình là $x + 2y + 2z-3= 0$}
	\loigiai{
		\begin{itemchoice}
			\itemch {Ta có $\mathrm{d}(O,(Q))=\dfrac{|-3|}{\sqrt{1^2+1^2+1^2}}=\sqrt{3}\ne 1$. SAI}
			\itemch {Ta có $\mathrm{d}(O,(Q))=\dfrac{|-3|}{\sqrt{2^2+1^2+2^2}}= 1$. ĐÚNG}
			\itemch {Ta có $\mathrm{d}(O,(Q))=\dfrac{|6|}{\sqrt{2^2+1^2+(-2)^2}}=2\ne 1$. SAI}
			\itemch {Ta có $\mathrm{d}(O,(Q))=\dfrac{|-3|}{\sqrt{1^2+2^1+2^2}}=1$. ĐÚNG
			}
		\end{itemchoice}
	}
\end{ex}

%%==========Câu 49
\begin{ex}%[Câu 24]%[2H5H1-4]
	Trong không gian $Oxyz$, cho điểm $N(0;1;0)$ và hai mặt phẳng $(P)\colon 2x-y-2z-9=0$, $(Q)\colon 4x-2y-4z-6=0$. Các mệnh đề sau đây đúng hay \textbf{sai}?
	\choiceTF
	{\True Hai mặt phẳng $(P)$ và $(Q)$ song song với nhau}
	{Khoảng cách từ điểm $N$ đến mặt phẳng $(Q)$ bằng $\dfrac{1}{2}$}
	{\True Khoảng cách giữa hai mặt phẳng $(P)$ và $(Q)$ bằng $2$}
	{Khoảng cách giữa hai mặt phẳng $(P)$ và $(Q)$ bằng $3$}
	\loigiai{
		\begin{itemize}
			\item Ta có $\dfrac{2}{4}=\dfrac{-1}{-2}=\dfrac{-2}{-4}\ne \dfrac{-9}{-6}$ nên $(P)\parallel (Q)$.	
			\item $\mathrm{d}\left(N,\left(Q\right)\right)=\dfrac{\left| -2\cdot 1-6\right|}{\sqrt{4^2+\left(-2\right)^2+\left(-4\right)^2}}=\dfrac{4}{3}$.
			\item $\mathrm{d}\left((P),(Q)\right)=\dfrac{|-9-(-3)|}{\sqrt{2^2+(-1)^2+(-2)^2}}=2$.
		\end{itemize}
		\begin{itemchoice}
			\itemch Hai mặt phẳng $(P)$ và $(Q)$ song song với nhau.	ĐÚNG
			\itemch Khoảng cách điểm đến mặt phẳng $(Q)$ bằng $\dfrac{1}{2}$.	SAI
			\itemch Khoảng cách giữa hai mặt phẳng $(P)$ và $(Q)$ bằng $2$. ĐÚNG
			\itemch Khoảng cách giữa hai mặt phẳng $(P)$ và $(Q)$ bằng $3$. SAI
		\end{itemchoice}
	}
\end{ex}

%%==========Câu 50
\begin{ex}%[Câu 25]%[2H5H1-5]
	Khoảng cách từ điểm $A(2;4;3)$ đến mặt phẳng $(\alpha)\colon 2x+y+2z+1=0$ và $(\beta)\colon x=0$ lần lượt là $\mathrm{d}(A,(\alpha))$, $\mathrm{d}(A,(\beta))$. Các mệnh đề sau đây đúng hay \textbf{sai}?
	\choiceTF
	{$\mathrm{d}\left(A,(\alpha)\right)=3\cdot \mathrm{d}\left(A,(\beta)\right)$}
	{$\mathrm{d}\left(A,(\alpha)\right)>\mathrm{d}\left(A,(\beta)\right)$}
	{$\mathrm{d}\left(A,(\alpha)\right)=\mathrm{d}\left(A,(\beta)\right)$}
	{\True $2\cdot\mathrm{d}\left(A,(\alpha)\right) = \mathrm{d}\left(A,(\beta)\right)$}
	\loigiai{
		Ta có: $\mathrm{d}\left(A,(\alpha)\right)=\dfrac{\left| 2.x_A+y_A+2.z_A+1\right|}{\sqrt{2^2+1^2+2^2}}=1$ và $\mathrm{d}\left(A,(\beta)\right)=\dfrac{\left|x_A\right|}{\sqrt{1^2}}=2$.\\
		Kết luận: $\mathrm{d}\left(A,(\beta)\right)=2\cdot \mathrm{d}\left(A,(\alpha)\right)$.
		\begin{itemchoice}
			\itemch $\mathrm{d}\left(A,(\alpha)\right)=3\cdot \mathrm{d}\left(A,(\beta)\right)$. SAI
			\itemch $\mathrm{d}\left(A,(\alpha)\right)>\mathrm{d}\left(A,(\beta)\right)$. SAI
			\itemch $\mathrm{d}\left(A,(\alpha)\right) =\mathrm{d}\left(A,(\beta)\right)$. SAI
			\itemch $2\cdot \mathrm{d}\left(A,(\alpha)\right)=\mathrm{d}\left(A,(\beta)\right)$. ĐÚNG
		\end{itemchoice}
	}
\end{ex}

%%==========Câu 51
\begin{ex}%Câu 51.%[2H5H1-4]
	Trong không gian $Oxyz$, cho điểm $I(2; 6;-3)$ và các mặt phẳng: $(\alpha)\colon x-2=0$; $(\beta)\colon y-6=0$; $(\gamma): z-3=0$. Các mệnh đề sau đây đúng hay sai?
	\choiceTF
	{\True $(\alpha) \perp(\beta)$}
	{$(\beta) \parallel (Oyz)$}
	{$(\gamma) \parallel Oz$}
	{\True $(\alpha)$ qua $I$}
	\loigiai{
		Ta có:
		\begin{itemize}
			\item $(\alpha): x-2=0$ có véctơ pháp tuyến $\vec{a}=(1 ; 0 ; 0)$.
			\item $(\beta): y-6=0$ có véctơ pháp tuyến $\vec{b}=(0 ; 1 ; 0)$.
			\item $(\gamma): z+3=0$ có véctơ pháp tuyến $\vec{c}=(0 ; 0 ; 1)$.
		\end{itemize}
		\begin{itemchoice}
			\itemch đúng vì ta có $\vec{a} \cdot \vec{b}=1\cdot 0+0\cdot 1+0=0 \Rightarrow(\alpha) \perp(\beta)$.
			\itemch sai vì $(Oyz)$ có véctơ pháp tuyến $\vec{i}=(1 ; 0 ; 0)$ không cùng phương với $\vec{b}=(0 ; 1 ; 0)$ nên $(\beta)$ không song song với mặt phẳng $(Oyz)$. 
			\itemch sai vì trục $Oz$ có vectơ chỉ phương $\vec{k}=(0 ; 0 ; 1)=\vec{c}$ nên $(\gamma) \perp Oz$.
			\itemch đúng vì thay tọa độ điểm $I$ vào $(\alpha)$ ta thấy thỏa thỏa mãn nên $I \in(\alpha)$.	
		\end{itemchoice}
	}
\end{ex}

%%==========Câu 52
\begin{ex}%Câu 52.%[2H5H1-4]
	Trong không gian $Oxyz$, cho hai mặt phẳng $(P)\colon y-9=0$. Xét các mệnh đề sau:
	\begin{multicols}{2}
		\item \hspace*{1cm}(I) $(P) \parallel (Oxz)$.
		\item (II) $(P) \perp Oy$
	\end{multicols}
	\choiceTF
	{Cả (I) và (II) đều sai}
	{(I) đúng, (II) sai}
	{(I) sai, (II) đúng}
	{\True Cả (I) và (II) đều đúng}
	\loigiai{
		Ta có: mặt phẳng $(Oxz)$ có véctơ pháp tuyến $\vec{j}=(0 ; 1 ; 0)$.\\
		Mặt phẳng $(P)$ có véctơ pháp tuyến là $\vec{a}=(0;1;1)=\vec{j}$ nên $(P)\parallel (Oxz)$.\\
		Trục $Oz$ có vectơ chỉ phương là $\vec{j}=(0;1;0)$ nên $(P)\perp Oy$.
		\begin{itemchoice}
			\itemch Cả (I) và (II) đều sai. SAI
			\itemch (I) đúng, (II) sai. SAI
			\itemch (I) sai, (II) đúng. SAI
			\itemch Cả (I) và (II) đều đúng. ĐÚNG
		\end{itemchoice}
	}
\end{ex}

%%==========Câu 53
\begin{ex}%Câu 53.%[2H5H1-4]
	Trong không gian $Oxyz$, Cho ba mặt phẳng $(\alpha)\colon x+y+2z+1=0;(\beta)\colon x+y-z+2=0$; $(\gamma)\colon x-y+5=0$. Các mệnh đề sau đây đúng hay sai?
	\choiceTF
	{$(\alpha) \parallel (\gamma)$}
	{\True $(\alpha) \perp(\beta)$}
	{\True $(\gamma) \perp(\beta)$}
	{\True $(\alpha) \perp(\gamma)$}
	\loigiai{ Ta có:
		\begin{itemize}
			\item Mặt phẳng $(\alpha)$ có véctơ pháp tuyến là $\vec{a}=(1;1;2)$.
			\item Mặt phẳng $(\beta)$ có có véctơ pháp tuyến là $\vec{b}=(1;1;-1)$.
			\item Mặt phẳng $(\gamma)$ có có véctơ pháp tuyến là $\vec{c}=(1;-1;0)$.
			\item $\left [\vec{a},\vec{c}\right ]=(2;2;-2)\ne \vec{0}$ nên $(\alpha)$ và $(\gamma)$ không song song nhau.
			\item $\vec{a} \cdot \vec{b}=0 \Rightarrow(\alpha) \perp(\beta)$.
			\item $\vec{a} \cdot \vec{c}=0 \Rightarrow(\alpha) \perp(\gamma)$.
			\item $\vec{b} \cdot \vec{c}=0 \Rightarrow(\beta) \perp(\gamma)$.
		\end{itemize}
		\begin{itemchoice}
			\itemch $(\alpha)\parallel(\gamma)$. SAI
			\itemch $(\alpha) \perp(\beta)$. ĐÚNG
			\itemch $(\gamma) \perp(\beta)$. ĐÚNG
			\itemch $(\alpha) \perp(\gamma)$. ĐÚNG
		\end{itemchoice}
	}
\end{ex}
\Closesolutionfile{ans}
\indapan{2}{ans/ans2C5B1CD1-D2-DS}
\TNSA
\Opensolutionfile{ans}[ans/ans2C5B1CD1-D2-KQ]
%%==========Câu 54
\begin{ex}%Câu 54.%[2H5H1-5]
	Trong không gian $Oxyz$, cho điểm $M(-1; 2-3)$ và mặt phẳng $(P)\colon 2 x-2 y+z+5=0$. Tính khoảng cách từ điểm $M$ đến mặt phẳng $(P)$ (kết quả viết dưới dạng số thập phân, lấy gần đúng đến hàng phần mười).
	\shortans[0]{$1{,}3$}
	\loigiai{
		Khoảng cách từ điểm $M$ đến mặt phẳng $(P)$ là 
		$$\mathrm{d}\left(M,(P)\right)=\dfrac{\left| 2\cdot (-1)-2\cdot 2+1\cdot (-3)+5\right|}{\sqrt{2^2+(-2)^2+1^2}}=\dfrac{4}{3}.$$
	}
\end{ex}

%%==========Câu 55
\begin{ex}%Câu 55.%[2H5H1-4]
	Trong không gian $Oxyz$, khoảng cách giữa hai mặt phẳng $(P)\colon x+2 y-2 z-16=0$ và $(Q)\colon x+2 y-2 z-1=0$ bằng bao nhiêu?
	\shortans[0]{$5$}
	\loigiai{
		Ta có $\heva{&(P)\parallel (Q) \\ &A(16;0;0)\in (P)}\Rightarrow \mathrm{d}\left((P),(Q)\right)=\mathrm{d}\left(A,(Q)\right)=\dfrac{\left| 16+2\cdot 0-2\cdot 0-1\right|}{\sqrt{1^2+2^2+2^2}}=5$.
	}
\end{ex}
\Closesolutionfile{ans}
\indapan{2}{ans/ans2C5B1CD1-D2-DS}
\TNSA
\Opensolutionfile{ans}[ans/ans2C5B1CD1-D2-KQ]
\begin{ex} %Cau 56D  %[2H5H1-4]
	Trong không gian $Oxyz$, điểm $M \left(0;a;0\right)$ thuộc trục $Oy$ và cách đều hai mặt phẳng: $\left(P\right) \colon x+y-z+1=0$ và $\left(Q\right) \colon x-y+z-5=0$. Khi đó $a$ có giá trị bằng
	\shortans{$-3$}
	\loigiai{
		Ta có $M \in Oy \Rightarrow M\left(0;a;0\right)$.\\
		Theo giả thiết: $\mathrm{d} \left(M,\left(P\right)\right) = \mathrm{d} \left(M,\left(Q\right)\right) \Leftrightarrow \dfrac{\vert a+1 \vert}{\sqrt{3}} = \dfrac{\vert -a-5 \vert}{\sqrt{3}} \Leftrightarrow a = -3$.\\
		Vậy $a = -3$ thì thỏa mãn đề bài.
	}
\end{ex}

\begin{ex} %Cau 57D %[2H5V1-5]
	Trong không gian với hệ trục tọa độ $Oxy$, cho $A \left(1;2;3\right)$, $B \left(3;4;4\right)$. Khi đó giá trị của tham số $m$ bằng bao nhiêu để khoảng cách từ điểm $A$ đến mặt phẳng $\left(P\right)\colon 2x + y + mz -1=0$ bằng độ dài đoạn thẳng $AB$.
	\shortans{$2$}
	\loigiai{
		Ta có $\overrightarrow{AB} = \left(2;2;1\right) \Rightarrow AB = \sqrt{2^2+2^2+1^2} = 3$ \quad(1)\\
		Khoảng cách từ điểm $A$ đến mặt phẳng $\left(P\right)$:\\
		$\mathrm{d} \left(A;\left(P\right)\right) = \dfrac{\vert 2 \cdot 1 + 2 + m \cdot 3 -1 \vert}{\sqrt{2^2+1^2+m^2}} = \dfrac{\vert 3m+3 \vert}{\sqrt{5+m^2}}$ \quad(2).\\
		Để $AB = \mathrm{d} \left(A;\left(P\right)\right) \Rightarrow 3 = \dfrac{\vert 3m+3 \vert}{\sqrt{5+m^2}} \Leftrightarrow 9 \left(5+m^2\right)= 9 \left(m+1\right)^2 \Leftrightarrow m =2$.
	}
\end{ex}

\begin{ex} %Cau 58D %[2H5H1-5]
	Gọi điểm $M \left(0;a;0\right)$ trên trục $Oy$ sao cho khoảng cách từ điểm $M$ đến mặt phẳng $\left(P\right) \colon 2x-y+3z-4=0$ nhỏ nhất. Khi đó giá trị của $a$ là
	\shortans{$-4$}
	\loigiai{
		Khoảng cách từ $M$ đến $\left(P\right)$ nhỏ nhất khi $M$ thuộc $\left(P\right)$. Nên $M$ là giao điểm của trục $Oy$ với mặt phẳng $\left(P\right)$.\\
		Thay $x=0$, $z=0$ vào phương trình ta được $y = -4$. Khi đó $M \left(0;-4;0\right)$\\
		Vậy giá trị của $a = -4$.
	}
\end{ex}

\begin{ex} %Cau 59D %[2H5H1-5]
	Cho điểm $M \left(0;0;m\right)$ thuộc trục $Oz$ sao cho điểm $M$ cách đều điểm $A \left(2;3;4\right)$ và mặt phẳng $\left(P\right) \colon 2x+3y+z-17=0$. Khi đó giá trị của $m$ là
	\shortans{$3$}
	\loigiai{
		Ta có $MA = \sqrt{2^2+3^2+\left(4-m\right)^2}$; $\mathrm{d} \left(M,\left(P\right)\right) = \dfrac{\vert m-17 \vert}{\sqrt{14}}$.\\
		$M$ cách đều điểm $A \left(2;3;4\right)$ và mặt phẳng $\left(P\right) \colon 2x+3y+z-17=0$ khi và chỉ khi\\
		$$\sqrt{2^2+3^2+\left(4-m\right)^2} = \dfrac{\vert m-17 \vert}{\sqrt{14}} \Leftrightarrow 13 \left(m-3\right)^2 = 0 \Leftrightarrow m=3$$
		Vậy $m=3$.
	}
\end{ex}

\begin{ex} %Cau 60D %[2H5V1-5]
	Trong không gian với hệ trục tọa độ $Oxyz$, cho hai điểm $A \left(1;2;3\right)$, $B\left(5;-4;-1\right)$ và mặt phẳng $\left(P\right)$ qua $Ox$ sao cho $\mathrm{d} \left(B;\left(P\right)\right) = 2\mathrm{d} \left(A;\left(P\right)\right)$, $\left(P\right)$ cắt $AB$ tại $I\left(a;b;c\right)$ nằm giữa $AB$. Tính $a+b+c$.
	\shortans{$4$}
	\loigiai{
		Vì $\mathrm{d} \left(B;\left(P\right)\right) = 2\mathrm{d} \left(A;\left(P\right)\right)$ và $\left(P\right)$ cắt đoạn $AB$ tại $I$ nên\\
		$\overrightarrow{BI} = -2 \overrightarrow{AI} \Leftrightarrow \heva{&a-5 = -2\left(a-1\right)\\&b+4 = -2\left(b-2\right)\\&c+1 = -2\left(c-3\right)} \Leftrightarrow \heva{&a=\dfrac{7}{3}\\&b=0\\&c=\dfrac{5}{3}} \Rightarrow a+b+c = 4$.
	}
\end{ex}

\begin{ex} %Cau 61D %[2H5V1-5]
	Trong không gian $Oxyz$, cho mặt phẳng $\left(P\right) \colon 3x+4y-12z+5=0$ và điểm $A \left(2;4;-1\right)$. Trên mặt phẳng $\left(P\right)$ lấy điểm $M$. Gọi $B$ là điểm sao cho $\overrightarrow{AB} = 3\cdot \overrightarrow{AM}$. Tính khoảng cách $\mathrm{d}$ từ $B$ đến mặt phẳng $\left(P\right)$
	\shortans{$6$}
	\loigiai{Ta có: $\overrightarrow{AB} = 3 \cdot \overrightarrow{AM} \Rightarrow BM=2\cdot AM \Rightarrow \dfrac{\mathrm{d} \left(B,\left(P\right)\right)}{\mathrm{d} \left(A,\left(P\right)\right)} = \dfrac{BM}{AM} = 2$
		\immini{
			\begin{eqnarray*}
				&\Rightarrow \mathrm{d} \left(B,\left(P\right)\right) &= 2 \cdot \mathrm{d} \left(A,\left(P\right)\right)\\
				& &= 2 \cdot \dfrac{\vert 3 \cdot 2 + 4 \cdot 4 -12 \cdot \left(-1\right)+5\vert}{\sqrt{3^2+4^2+\left(-12\right)^2}}\\
				& & = 2 \cdot 3 = 6
			\end{eqnarray*}
			Vậy $\mathrm{d} \left(B,\left(P\right)\right) = 6$.}{
			\begin{tikzpicture}[>=stealth,line join=round, line cap=round, scale=0.7]
				\coordinate (A) at (1,3);
				\coordinate (B) at (-1,0);
				\coordinate (C) at (5,0);
				\coordinate (D) at (7,3);
				\coordinate (H) at (1.5,2);
				\coordinate (K) at (4.5,2);
				\coordinate (M) at (4.5,-1.5);
				\coordinate (N) at (1.5,4.5);
				\path[name path=d1] (N)--(M);
				\path[name path=d2] (H)--(K);
				\path[name path=d3] (B)--(C);
				\path[name intersections={of=d1 and d2,by=l}];
				\path[name intersections={of=d1 and d3,by=z}];
				\draw (A)--(B)--(C)--(D)--(A); \draw[dashed] (K)--(4.5,0); \draw (4.5,0)--(M)--(z);
				\draw (l) node[below left]{$M$} circle (1pt)--(N) node[above]{$A$}  circle (1pt)--(H) node[below left]{$H$}  circle (1pt)--(K) node[below right]{$K$} circle (1pt); \draw (M) node[below]{$B$} circle (1pt); \draw [dashed] (l)--(z);
				\begin{scope}
					\clip (A)--(B)--(C);
					\draw (B) circle (1.1);
					\draw (-0.8,0) node[above right]{$P$} ;
				\end{scope}
			\end{tikzpicture}
	}}
\end{ex}

\begin{ex} %Cau 62D %[2H5H1-4]
	Trong không gian $Oxyz$, cho hai mặt phẳng $\left(P\right) \colon 2x+my+2mz-9=0$ và $\left(Q\right) \colon 6x-y-z-10=0$. Tìm $m$ để $\left(P\right) \perp \left(Q\right)$
	\shortans{$4$}
	\loigiai{
		$\left(P\right) \colon 2x+my+2mz-9=0$ có véc-tơ pháp tuyến là $\overrightarrow{a} = \left(2;m;2m\right)$\\
		$\left(Q\right) \colon 6x-y-z-10=0$ có véc-tơ pháp tuyến là $\overrightarrow{b} = \left(6;-1;-1\right)$\\
		Khi đó $\left(P\right) \perp \left(Q\right) \Leftrightarrow \overrightarrow{a} \cdot \overrightarrow{b} =0 \Leftrightarrow 2 \cdot 6 + m \cdot \left(-1\right)+2m \cdot \left(-1\right) =0 \Leftrightarrow m=4$.
	}
\end{ex}

\begin{ex} %Cau 63D %[2H5H1-4]
	Trong không gian $Oxyz$, cho hai mặt phẳng $\left(P\right) \colon 5x+my+z-5=0$ và $\left(Q\right) \colon nx-3y-2z+7=0$. Để $\left(P\right) \parallel \left(Q\right)$ thì giá trị của $m+n$ là (làm tròn đến chữ số thập phân thứ nhất)
	\shortans{$-8{,}5$}
	\loigiai{
		$\left(P\right) \colon 5x+my+z-5=0$ có véc-tơ pháp tuyến là $\overrightarrow{a} = \left(5;m;1\right)$\\
		$\left(Q\right) \colon nx-3y-2z+7=0$ có véc-tơ pháp tuyến là $\overrightarrow{b} = \left(n;-3;-2\right)$\\
		Để $\left(P\right) \parallel \left(Q\right) \Leftrightarrow \left[a;b\right] = \overrightarrow{0} \Leftrightarrow \heva{&-2m+3=0\\&n+10=0\\&-15-mn=0} \Leftrightarrow \heva{&m=\dfrac{3}{2}\\&n=-10}$\\
		Khi đó $m+n = \dfrac{3}{2} + \left(-10\right)= -8,5$.
	}
\end{ex}

\begin{ex} %Cau 64D %[2H5V1-4]
	Trong không gian $Oxyz$, cho hai mặt phẳng $\left(P\right) \colon 2x-my-4z-6+m=0$ và $\left(Q\right) \colon \left(m+3\right)x +y+\left(5m+1\right)z -7=0$. Tìm $m$ để $\left(P\right) \equiv \left(Q\right)$.
	\shortans{$-1$}
	\loigiai{
		$\left(P\right) \colon 2x-my-4z-6+m=0$ có véc-tơ pháp tuyến là $\overrightarrow{a} = \left(2;-m;-4\right)$\\
		$\left(Q\right) \colon \left(m+3\right)x +y+\left(5m+1\right)z -7=0$ có véc-tơ pháp tuyến là $\overrightarrow{b} = \left(m+3;1;5m+1\right)$\\
		Khi đó với $m \neq -3$, $m \neq -\dfrac{1}{5}$ ta có $\left(P\right) \equiv \left(Q\right) \Leftrightarrow \dfrac{2}{m+3} = \dfrac{-m}{1} = \dfrac{-4}{5m+1} \Leftrightarrow m = -1$.
	}
\end{ex}

\begin{ex} %Cau 65D %[2H5H1-3]
	Trong không gian $Oxyz$, cho hai mặt phẳng $\left(P\right) \colon x-2y-z+3=0$ và $\left(Q\right) \colon 2x +y+z -1=0$. Mặt phẳng $\left(R\right)$ đi qua điểm $M\left(1;1;1\right)$ chứa giao tuyến của $\left(P\right)$ và $\left(Q\right)$; phương trình của $\left(R\right) \colon m\left(x-2y-z+3\right) + \left(2x+y+z-1\right)=0$. Khi đó giá trị của $m$ là bao nhiêu?
	\shortans{$-3$}
	\loigiai{
		Vì $\left(R\right) \colon m\left(x-2y-z+3\right) + \left(2x+y+z-1\right)=0$ đi qua điểm $M \left(1;1;1\right)$ nên ta có:\\
		$m\left(1-2 \cdot 1-1+3\right) + \left(2 \cdot 1+1+1-1\right)=0 \Leftrightarrow m = -3$\\
		Vậy $m = -3$.
	}
\end{ex}

\begin{ex} %Cau 66D %[2H5V1-4]
	Trong không gian $Oxyz$, cho $3$ điểm $A\left(1;0;0\right)$,$B\left(0;b;0\right)$,$C\left(0;0;c\right)$ trong đó $b \cdot c \neq 0$ và mặt phẳng $\left(P\right) \colon y-z+1=0$. Giá trị của $\dfrac{2b}{c}$ bằng bao nhiêu để mặt phẳng $\left(ABC\right)$ vuông góc với mặt phẳng $\left(P\right)$.
	\shortans{$2$}
	\loigiai{
		Phương trình mặt phẳng $\left(ABC\right) \colon \dfrac{x}{1}+\dfrac{y}{b}+\dfrac{z}{c}=1$ có véc-tơ pháp tuyến là $\overrightarrow{n} = \left(1;\dfrac{1}{b};\dfrac{1}{c}\right)$.\\
		Phương trình mặt phẳng $\left(P\right) \colon y-z+1=0$ có véc-tơ pháp tuyến là $\overrightarrow{n'} = \left(0;1;-1\right)$.\\
		Do đó $\left(ABC\right) \perp \left(P\right) \Leftrightarrow \overrightarrow{n} \cdot \overrightarrow{n'} = 0 \Leftrightarrow \dfrac{1}{b}-\dfrac{1}{c} = 0 \Leftrightarrow b = c$.\\
		Vậy $\dfrac{2b}{c} = 2$.
	}
\end{ex}

\begin{ex} %Cau 67D %[2H5V1-3]
	Trong không gian $Oxyz$, cho mặt phẳng $\left(\alpha\right) \colon ax-y+2z+b=0$ đi qua giao tuyến của hai mặt phẳng $\left(P\right) \colon x-y-z+1=0$ và $\left(Q\right) \colon x+2y+z-1=0$. Tính $a+4b$
	\shortans{$-16$}
	\loigiai{
		Trên giao tuyến $\Delta$ của hai mặt phẳng $\left(P\right)$, $\left(Q\right)$ ta lấy lần lượt hai điểm $A$, $B$ như sau\\
		Lấy $A \left(x;y;1\right) \in \Delta$, ta có hệ phương trình $\heva{&x-y=0\\&x+2y=0} \Rightarrow x=y=0 \Rightarrow A \left(0;0;1\right)$.\\
		Lấy $B \left(-1;y;z\right) \in \Delta$, ta có hệ phương trình $\heva{&y+z=0\\&2y+z=0} \Rightarrow \heva{&y=2\\&z=2} \Rightarrow B \left(-1;2;-2\right)$.\\
		Vì $\Delta \subset \left(\alpha\right)$ nên $A$, $B \in \left(\alpha\right)$. Do đó ta có: $\heva{&2+b=0\\&-a+b-6=0} \Rightarrow \heva{&a=-8\\&b=-2}$.\\
		Vậy $a+4b = -8 + 2 \cdot \left(-2\right) = -16$.
	}
\end{ex}

\begin{ex} %Cau 68D %[2H5V1-4]
	Gọi $m$, $n$ là hai giá trị thực thỏa mãn giao tuyến của hai mặt phẳng $\left(P_m\right) \colon mx+2y+nz+1=0$ và $\left(Q_m\right) \colon x-my + nz + 2=0$ vuông góc với mặt phẳng $\left(\alpha\right) \colon 4x -y -6z +3=0$. Tính $m+n$
	\shortans{$3$}
	\loigiai{
		\begin{description}
			\item[+] $\left(P_m\right) \colon mx+2y+nz+1=0$ có véc-tơ pháp tuyến $\overrightarrow{n}_1 = \left(m;2;n\right)$
			\item[+] $\left(Q_m\right) \colon x-my+nz+2=0$ có véc-tơ pháp tuyến $\overrightarrow{n}_2 = \left(1;-m;n\right)$
			\item[+] $\left(\alpha \right) \colon 4x-y-6z+3=0$ có véc-tơ pháp tuyến $\overrightarrow{n}_{\alpha} = \left(4;-1;-6\right)$.
			\item[+] Giao tuyến của hai mặt phẳng $\left(P_m\right)$ và $\left(Q_m\right)$ vuông góc với mặt phẳng $\left(\alpha\right)$ nên
			$$\heva{&\left(P_m\right) \perp \left(\alpha\right)\\&\left(Q_m\right) \perp \left(\alpha\right)} \Leftrightarrow \heva{&\overrightarrow{n}_1 \perp \overrightarrow{n}_{\alpha}\\&\overrightarrow{n}_2 \perp \overrightarrow{n}_{\alpha}} \Leftrightarrow \heva{&\overrightarrow{n}_1 \cdot \overrightarrow{n}_{\alpha} = 0\\&\overrightarrow{n}_2 \cdot \overrightarrow{n}_{\alpha} = 0} \Leftrightarrow \heva{&4m-2-6n=0\\&4+m-6n=0} \Leftrightarrow \heva{&m=2\\&n=1}$$
		\end{description}
		Vậy $m+n = 3$.
	}
\end{ex}

\begin{ex} %Cau 69D %[2H5V1-4]
	Trong không gian với hệ tọa độ $Oxyz$ có bao nhiêu mặt phẳng song song với mặt phẳng $\left(Q\right) \colon x+y+z+3=0$, cách điểm $M \left(3;2;1\right)$ một khoảng bằng $3\sqrt{3}$ biết rằng tồn tại một điểm $X\left(a;b;c\right)$ trên mặt phẳng đó, khi đó $a+b+c$ có giá trị bằng
	\shortans{$15$}
	\loigiai{
		Ta có mặt phẳng cần tìm là $\left(P\right) \colon x+y+z+d=0$ với $d \neq 3$.\\
		Mặt phẳng $\left(P\right)$ cách điểm $M \left(3;2;1\right)$ một khoảng bằng $3\sqrt{3}$ nên\\
		$\mathrm{d} \left(M,\left(P\right)\right) = \dfrac{\vert 6+d \vert}{\sqrt{3}} = 3\sqrt{3} \Leftrightarrow \hoac{&d=3 \quad(L)\\&d=-15} \Rightarrow d = -15$.\\
		Suy ra $\left(P\right) \colon x+y+z-15=0$.\\
		Theo giả thiết $X\left(a;b;c\right) \in \left(P\right) \Leftrightarrow a+b+c = 15$.
	}
\end{ex}

\begin{ex} %Cau 70D %[2H5C1-4]
	Biết rằng trong không gian với hệ tọa độ $Oxyz$ có hai mặt phẳng $\left(P\right)$ và $\left(Q\right)$ cùng thỏa mãn các điều kiện sau: đi qua hai điểm $A\left(1;1;1\right)$ và $B\left(0;-2;2\right)$, đồng thời cắt các trục tọa độ $Ox$, $Oy$ tại hai điểm cách đều $O$. Giả sử $\left(P\right) \colon x+b_{1}y+c_{1}z+d_1=0$ và $\left(Q\right) \colon x+b_2 y+c_2 z+d_2=0$. Tính giá trị biểu thức $b_1b_2 + c_1c_2$
	\shortans{$-9$}
	\loigiai{
		$\textbf{Cách 1}$\\
		Xét mặt phẳng $\left(\alpha\right) \colon x+by+cz+d=0$ thỏa mãn các điều kiện: đi qua hai điểm $A \left(1;1;1\right)$ và $B\left(0;-2;2\right)$, đồng thời cắt các trục tọa độ $Ox$, $Oy$ tại hai điểm cách đều $O$.\\
		Vì $\left(\alpha\right)$ đi qua $A \left(1;1;1\right)$ và $B \left(0;-2;2\right)$ nên ta có hệ phương trình:
		$$\heva{&1+b+c+d=0\\&-2b+2c+d=0} \quad(*)$$
		Mặt phẳng $\left(\alpha\right)$ cắt các trục tọa độ $Ox$, $Oy$ lần lượt tại $M \left(-d;0;0\right)$, $N \left(0;\dfrac{-d}{c};0\right)$.\\
		Vì $M$, $N$ cách đều $O$ nên $OM = ON$. Suy ra: $\vert d \vert = \left\vert \dfrac{d}{b} \right\vert$.\\
		Nếu $d=0$ thì chỉ tồn tại duy nhất một mặt phẳng thỏa mãn yêu cầu bài toán (mặt phẳng này sẽ đi qua điểm $O$).\\
		Do đó để tồn tại hai mặt phẳng thỏa mãn yêu cầu bài toán thì $\vert d \vert = \left\vert \dfrac{d}{b} \right\vert \Leftrightarrow b = \pm 1$.
		\begin{description}
			\item[$\bullet$] Với $b=1$, $\left(*\right) \Leftrightarrow \heva{&c+d=-2\\&2c+d=2} \Leftrightarrow \heva{&c=4\\&d=-6}$. Ta được mặt phẳng $\left(P\right) \colon x+y+4z-6=0$.
			\item[$\bullet$] Với $b=-1$, $\left(*\right) \Leftrightarrow \heva{&c+d=0\\&2c+d=-2} \Leftrightarrow \heva{&c=-2\\&d=2}$. Ta có mặt phẳng $\left(P\right) \colon x-y-2z+2=0$.
		\end{description}
		Vậy $b_1b_2 + c_1c_2 = 1 \cdot \left(-1\right) + 4 \cdot \left(-2\right) = -9$.\\
		$\textbf{Cách 2}$\\
		Ta có $\overrightarrow{AB} = \left(-1;-3;1\right)$.\\
		Xét mặt phẳng $\left(\alpha\right) \colon x+by+cz+d=0$ thõa mãn các điều kiện: đi qua hai điểm $A \left(1;1;1\right)$ và $B\left(0;-2;2\right)$, đồng thời cắt các trục tọa độ $Ox$, $Oy$ tại hai điểm cách đều $O$ lần lượt tại $M$, $N$. Vì $M$, $N$ cách đều $O$ nên ta có hai trường hợp sau
		\begin{description}
			\item[TH1] $M \left(a;0;0\right)$, $N \left(0;a;0\right)$ với $a \neq 0$ khi đó $\left(\alpha\right)$ chính là $\left(P\right)$. Ta có $\overrightarrow{MN} = \left(-a;a;0\right)$, chọn $\overrightarrow{u}_1 = \left(-1;1;0\right)$ là một véc-tơ cùng phương với $\overrightarrow{MN}$.\\
			Khi đó $\overrightarrow{n}_P = \left[\overrightarrow{AB},\overrightarrow{u}_1\right] = \left(-1;-1;-4\right)$
			suy ra $\left(P\right) \colon x+y+4z+d_1 = 0$.
			\item[TH2] $M \left(-a;0;0\right)$, $N \left(0;a;0\right)$ với $a \neq 0$ khi đó $\left(\alpha\right)$ chính là $\left(Q\right)$. Ta có $\overrightarrow{MN} = \left(a;a;0\right)$, chọn $\overrightarrow{u}_2 = \left(1;1;0\right)$ là một véc-tơ cùng phương với $\overrightarrow{MN}$.\\
			Khi đó $\overrightarrow{n}_Q = \left[\overrightarrow{AB},\overrightarrow{u}_2\right] = \left(-1;1;2\right)$
			suy ra $\left(Q\right) \colon x-y-2z+d_2 = 0$.
		\end{description}
		Vậy $b_1b_2 + c_1c_2 = 1 \cdot \left(-1\right) + 4 \cdot \left(-2\right) = -9$.
	}
\end{ex}
%-------------HetCD1---------------
\Closesolutionfile{ans}
\indapan{6}{ans/ans2C5B1CD1-D2-KQ}
% % \begin{dang}{LẬP PHƯƠNG TRÌNH TỔNG QUÁT MẶT PHẲNG}
% Để lập phương trình tổng quát của mặt phẳng $\left(\alpha\right)$ thông thường ta có 3 trường hợp cơ bản sau:\\
% $\textbf{Trường hợp 1:}$ Khi bài toán cho biết mặt phẳng $\left(\alpha\right)$ đi qua điểm $M_0 \left(x_0;y_0;z_0\right)$ và có một vectơ pháp tuyến $\overrightarrow{n} = \left(A;B;C\right)$ hoặc có hai vectơ chỉ phương $\overrightarrow{a}$, $\overrightarrow{b}$ (với $\overrightarrow{n} = \left[\overrightarrow{a},\overrightarrow{b}\right]$) thì viết dưới dạng sau:
% $$\left(\alpha\right) \colon A\left(x-x_0\right)+B\left(y-y_0\right)+C\left(z-z_0\right)=0$$
% $\textbf{Trường hợp 2:}$ Khi bài toán cho biết mặt phẳng $\left(\alpha\right)$ có một vectơ pháp tuyến $\overrightarrow{n} = \left(A;B;C\right)$ hoặc có hai vectơ chỉ phương $\overrightarrow{a}$, $\overrightarrow{b}$ (với $\overrightarrow{n} = \left[\overrightarrow{a},\overrightarrow{b}\right]$) và không tìm được điểm $M_0 \left(x_0;y_0;z_0\right) \in \left(\alpha\right)$ thì ta thực hiện các bước sau:
% \begin{itemize}
% 	\item $\textbf{Bước 1:}$ Viết phương trình mặt phẳng $\left(\alpha\right)$ dưới dạng:
% 	$$Ax+By+Cz+D=0$$
% 	\item $\textbf{Bước 2:}$ Sau đó dựa vào giả thiết bài toán để tìm giá trị $D$.
% \end{itemize}
% \begin{note}
% 	Dạng này, giả thiết có liên quan đến khoảng cách và góc liên quan đến mặt phẳng.
% \end{note}
% $\textbf{Trường hợp 3:}$ Khi bài toán cho biết mặt phẳng $\left(\alpha\right)$ đi qua điểm $M_0 \left(x_0;y_0;z_0\right)$ và giả thiết bài toán không cho vectơ pháp tuyến $\overrightarrow{n}$ hoặc không cho hai vectơ chỉ phương $\overrightarrow{a}$, $\overrightarrow{b}$ thì ta thực hiện các bước sau:
% \begin{itemize}
% 	\item $\textbf{Bước 1:}$ Gọi vectơ pháp tuyến của mặt phẳng $\left(\alpha\right)$ là $\overrightarrow{n} = \left(A;B;C\right)$ với $A^2+B^2+C^2 \neq 0$
% 	\item $\textbf{Bước 2:}$ Viết phương trình mặt phẳng $\left(\alpha\right)$ dưới dạng:
% 	$$\left(\alpha\right) \colon A\left(x-x_0\right)+B\left(y-y_0\right)+C\left(z-z_0\right)=0$$
% 	\item $\textbf{Bước 3:}$ Sau đó dựa vào giả thiết bài toán để tìm hai phương trình chứa $3$ ẩn $A$, $B$, $C$.
% 	\begin{note}
% 		\begin{itemize}
% 			\item Dạng này, giả thiết có liên quan đến khoảng cách và góc liên quan đến mặt phẳng
% 			\item Để giải tìm vectơ pháp tuyến của mặt phẳng đơn giản hơn thì gọi vectơ pháp tuyến của mặt phẳng là $\overrightarrow{n} = \left(1;B;C\right)$.
% 		\end{itemize}
% 	\end{note}
% \end{itemize}

\begin{dang}{Viết PTTQ MP khi biết điểm đi qua và một VTPT hoặc hai VTCP}
	\textbf{1. Lập phương trình tổng quát} của mặt phẳng đi qua điểm $M_0 \left(x_0;y_0;z_0\right)$ và biết một vectơ pháp tuyến $\overrightarrow{n} = \left(A;B;C\right)$\\
	Trong KG $Oxyz$, phương trình tổng quát của mặt phẳng đi qua điểm $M_0 \left(x_0;y_0;z_0\right)$ và có vectơ pháp tuyến $\overrightarrow{n}= \left(A;B;C\right)$ là:\\
	$$A\left(x-x_0\right) + B\left(y-y_0\right)+C\left(z-z_0\right) = 0$$
	hay $Ax+By+Cz+D=0$ với $D= -Ax_0-By_0-Cz_0$
	\begin{center}
		\begin{tikzpicture}[>=stealth,line join=round, line cap=round, scale=0.8]
			\coordinate (A) at (1.5,3);
			\coordinate (B) at (-1,0);
			\coordinate (C) at (5,0);
			\coordinate (D) at (7.5,3);
			\coordinate (I) at (1.9,1.5); \coordinate (J) at (1.9,4.5); \coordinate (K) at (1.9,-0.8);
			\coordinate (M) at (3.5,1.5);
			\coordinate (N) at (5.5,2);
			\begin{scope}
				\clip (A)--(B)--(C);
				\draw (B) circle (1);
			\end{scope}
			\draw (A)--(B)--(C)--(D)--(A);
			\draw (-0.8,0) node[above right]{$\alpha$};
			\draw (K)--(1.9,0); \draw [dashed] (1.9,0)--(I) circle (.8pt);  \draw (I)--(J);
			\draw (M) node[below]{$M$} circle (.8pt)--(N) node[below]{$N$} circle (.8pt);
			\draw[->,line width=1] (M)--(N);
			\draw[->,line width=1.4] (1.9,2.5)--(1.9,3.8) node[right]{$\overrightarrow{n}$};
		\end{tikzpicture}
	\end{center}
	$\textbf{Chú ý:}$
	\begin{enumerate}[a.]
		\item Mặt phẳng $\left(\alpha\right)$ có cặp vectơ chỉ phương $\overrightarrow{a}$, $\overrightarrow{b}$ ($\overrightarrow{a}$, $\overrightarrow{b}$ không cùng phương) thì mặt phẳng $\left(\alpha\right)$ có vectơ pháp tuyến $\overrightarrow{n}= \left[\overrightarrow{a},\overrightarrow{b}\right]$.
		\item Mặt phẳng $\left(\alpha\right)$ đi qua ba điểm $A$, $B$, $C$ không thẳng hàng thì có cặp vectơ chỉ phương $\overrightarrow{AB}$, $\overrightarrow{AC}$ nên mặt phẳng $\left(\alpha\right)$ có vectơ pháp tuyến $\overrightarrow{n} = \left[\overrightarrow{AB}, \overrightarrow{AC}\right]$.
		\item Dựa vào tính chất vuông góc, song song giữa mặt phẳng với mặt phẳng, giữa đường thẳng với mặt phẳng trong không gian để tìm vectơ chỉ phương, vectơ pháp tuyến của mặt phẳng cần lập.
		\begin{itemize}
			\item Hai mặt phẳng song song thì có cùng vectơ pháp tuyến.
			\item Hai mặt phẳng vuông góc thì vectơ chỉ phương của mặt phẳng này là vectơ pháp tuyến của mặt phẳng kia.
			\item Đường thẳng song song mặt phẳng thì vectơ chỉ phương của đường thẳng là vectơ chỉ phương của mặt phẳng.
			\item Đường thẳng vuông góc mặt phẳng thì vectơ chỉ phương của đường thẳng là vectơ pháp tuyến của mặt phẳng.
		\end{itemize}
	\end{enumerate}
	\textbf{2. Các trường hợp đặc biệt của mặt phẳng}
	\begin{enumerate}[a.]
		\item Phương trình mặt phẳng theo đoạn chắn\\
		Mặt phẳng $\left(\alpha\right)$ không đi qua gốc tọa độ $O$ và lần lượt cắt trục $Ox$ tại $A \left(a;0;0\right)$, cắt trục $Oy$ tại $B \left(0;b;0\right)$, cắt trục $Oz$ tại $C \left(0;0;c\right)$ có $\textbf{phương trình mặt phẳng theo đoạn chắn}$ là: $\dfrac{x}{a}+\dfrac{y}{b}+\dfrac{z}{c}=1$ với $a \cdot b \cdot c \neq 0$
		\begin{center}
			\begin{tikzpicture}[>=stealth,line join=round, line cap=round, scale=0.7]
				\coordinate (O) at (0,0); 
				\coordinate (A) at (-1.9,-1.9); \coordinate (C) at (0,3); \coordinate (B) at (3,0);
				\path[name path=trucy] (O)--(5,0);
				\path[name path=d1] (A)--(C);
				\path[name path=d2] (B)--(C);
				\path[name path=d3] (A)--(B);
				\path[name path=trucz] (O)--(0,5); \path[name path=trucx] (O)--(-4,-4);
				\path[name intersections={of=d1 and trucz, by=I}];  
				\path[name intersections={of=d1 and trucx, by=J}];
				\path[name intersections={of=d2 and trucy, by=K}]; 
				\draw [fill = blue!5] (I)--(J)--(K)--(I);
				\draw [->] (I)--(0,5) node[right]{$z$};\draw [->] (K)--(4,0) node[above right]{$y$}; \draw [->] (J)--(-3,-3) node[right]{$x$};
				\draw [dashed] (O) node[below]{$O$} circle (.8pt)--(I) node[left]{$C(0;0;c)$};
				\draw [dashed] (O)--(J) node[above left]{$A(a;0;0)$};
				\draw [dashed] (O)--(K) node[below right]{$B(0;b;0)$};
			\end{tikzpicture} 
		\end{center}
		\item Phương trình mặt phẳng đặc biệt\\
		Xét phương trình mặt phẳng $\left(\alpha\right) \colon Ax+By+Cz+D=0$ với $A^2+B^2+C^2 \neq 0$
		\begin{itemize}
			\item Nếu $D = 0$ thì mặt phẳng $\left(\alpha\right)$ đi qua gốc tọa độ $O$ và có dạng $\left(\alpha\right) \colon Ax+By+Cz=0$.
			\begin{center}
				\begin{tikzpicture}[>=stealth,line join=round, line cap=round, scale=0.8]
					%p1
					\coordinate (O) at (0,0); 
					\coordinate (x) at (-2,-2); 
					\coordinate (y) at (4,0); 
					\coordinate (z) at (0,3.5); 
					%p2
					\draw [->] (O) node[below]{$O$}--(x) node[below right]{$x$}; 
					\draw [->] (O)--(y) node[below]{$y$}; 
					\coordinate (A) at (3.3,2.5); \coordinate (B) at (-3,2);
					\draw [fill = blue!5] (A)--(O)--(B)--cycle;
					\path[name path=d1] (A)--(B);
					\path[name path=d2] (O)--(z);
					\path[name intersections={of=d1 and d2, by=M}];
					\draw [->] (M)--(z) node[right]{$z$}; \draw [dashed] (O)--(M);
					\draw (-2.5,2.1) node[below right]{$\left(\alpha\right)$};
					\draw (0.2,-2) node[right]{$Ax+By+Cz=0$};
				\end{tikzpicture}
			\end{center}
			\item Nếu $A=0$, $B \neq 0$, $C \neq 0$ thì mặt phẳng $\left(\alpha\right)$ song song hoặc chứa trục $Ox$.
			\item[+] Mặt phẳng $\left(\alpha\right)$ song song $Ox$ thì có dạng $\left(\alpha\right) \colon By+Cz+D=0$.(Hình 1)
			\item[+] Mặt phẳng $\left(\alpha\right)$ chứa trục $Ox$ thì có dạng $\left(\alpha\right) \colon By+Cz=0$.
			\item Nếu $A \neq 0$, $B = 0$, $C \neq 0$ thì mặt phẳng $\left(\alpha\right)$ song song hoặc chứa trục $Oy$.
			\item[+] Mặt phẳng $\left(\alpha\right)$ song song $Oy$ thì có dạng $\left(\alpha\right) \colon Ax+Cz+D=0$.(Hình 2)
			\item[+] Mặt phẳng $\left(\alpha\right)$ chứa trục $Oy$ thì có dạng $\left(\alpha\right) \colon Ax+Cz=0$.
			\item Nếu $A\neq 0$, $B \neq 0$, $C = 0$ thì mặt phẳng $\left(\alpha\right)$ song song hoặc chứa trục $Oz$.
			\item[+] Mặt phẳng $\left(\alpha\right)$ song song $Oz$ thì có dạng $\left(\alpha\right) \colon Ax+By+D=0$.(Hình 3)
			\item[+] Mặt phẳng $\left(\alpha\right)$ chứa trục $Oz$ thì có dạng $\left(\alpha\right) \colon Ax+By=0$.
			\item Nếu $A=B= 0$, $C \neq 0$ thì mặt phẳng $\left(\alpha\right)$ song song hoặc trùng với $\left(Oxy\right)$.
			\item[+] Mặt phẳng $\left(\alpha\right)$ song song $\left(Oxy\right)$ thì có dạng $\left(\alpha\right) \colon Cz+D=0$.(Hình 4)
			\item[+] Mặt phẳng $\left(\alpha\right)$ chứa $\left(Oxy\right)$ thì có dạng $\left(\alpha\right) \colon z=0$.
			\item Nếu $A=C= 0$, $B \neq 0$ thì mặt phẳng $\left(\alpha\right)$ song song hoặc trùng với $\left(Oxz\right)$.
			\item[+] Mặt phẳng $\left(\alpha\right)$ song song $\left(Oxz\right)$ thì có dạng $\left(\alpha\right) \colon By+D=0$.(Hình 5)
			\item[+] Mặt phẳng $\left(\alpha\right)$ chứa $\left(Oxz\right)$ thì có dạng $\left(\alpha\right) \colon y=0$.
			\item Nếu $B=C= 0$, $A \neq 0$ thì mặt phẳng $\left(\alpha\right)$ song song hoặc trùng với $\left(Oyz\right)$.
			\item[+] Mặt phẳng $\left(\alpha\right)$ song song $\left(Oyz\right)$ thì có dạng $\left(\alpha\right) \colon Ax+D=0$.(Hình 6)
			\item[+] Mặt phẳng $\left(\alpha\right)$ chứa $\left(Oyz\right)$ thì có dạng $\left(\alpha\right) \colon x=0$.
		\end{itemize}
		
		\begin{tabular}{*{2}{c}}
			\begin{tikzpicture}[>=stealth,line join=round, line cap=round, scale=0.9]
				%p1
				\coordinate (O) at (0,0); 
				\coordinate (x) at (-2.6,-2.6); 
				\coordinate (y) at (4,0); 
				\coordinate (z) at (0,3); 
				%p2
				\coordinate (M) at ($(O)!0.5!(x)$);
				\coordinate (N) at ($(O)!0.6!(y)$);
				\coordinate (P) at ($(O)!0.7!(z)$);
				\coordinate (P') at ($(P)-(2,2)$);\coordinate (N') at ($(N)-(2,2)$);
				\coordinate (L) at ($(P')-(P)$);
				%p3
				\path[name path=d1] (N')--(P');
				\path[name path=d2] (O)--(x);
				\path[name intersections={of=d1 and d2, by=S}];
				\draw [fill = blue!5, line width = .7pt] (N)--(P)--(P')--(N')--cycle;
				\draw [->] (N)--(y) node[below right]{$y$}; \draw [->] (P)--(z) node[below right]{$z$};
				\draw [->] (S)--(x) node[below right]{$x$};
				\draw [dashed] (N)--(O) node[below right]{$O$} circle (1pt)--(P);
				\draw [dashed] (S)--(O);\draw [dashed] (P') node[right]{$(\alpha)$}--(L)--(N');
				\draw (3,3) node [below]{$By+Cz+D=0$};
				%p4
				\draw [->,dashed,line width = 1.2pt] (O)--(-0.5,-0.5) node[above]{$\overrightarrow{i}$};
			\end{tikzpicture} & \begin{tikzpicture}[>=stealth,line join=round, line cap=round, scale=0.9]
				%p1
				\coordinate (O) at (0,0); 
				\coordinate (x) at (-2,-2); 
				\coordinate (y) at (4,0); 
				\coordinate (z) at (0,4); 
				%p2
				\coordinate (M) at ($(O)!0.65!(x)$);
				\coordinate (N) at ($(O)!0.6!(y)$);
				\coordinate (P) at ($(O)!0.5!(z)$);
				\coordinate (M') at ($(M)+(3,0)$);\coordinate (P') at ($(P)+(3,0)$);
				\coordinate (L) at ($(P')-(P)$);
				%p3
				\path[name path=d1] (P')--(M');
				\path[name path=d2] (O)--(y);
				\path[name intersections={of=d1 and d2, by=S}];
				\draw [fill = blue!5, line width = .7pt] (P)--(M)--(M')--(P')--cycle;
				\draw [->] (P)--(z) node[below right]{$z$}; \draw [->] (M)--(x) node[below right]{$x$};
				\draw [->] (S)--(y) node[below right]{$y$};
				\draw [dashed] (M)--(O) node[below right]{$O$} circle (1pt)--(P);
				\draw [dashed] (S)--(O); \draw [dashed] (P') node[below left]{$(\alpha)$}--(L)--(M');
				%datten
				\draw (3.3,3.3) node [below]{$Ax+Cz+D=0$};
				%p4
				\draw [->,dashed,line width = 1.2pt] (O)--(0.8,0) node[above]{$\overrightarrow{j}$};
			\end{tikzpicture}\\
			$\textbf{Hình 1}$    & $\textbf{Hình 2}$    \\
			\begin{tikzpicture}[>=stealth,line join=round, line cap=round, scale=0.7]
				%p1
				\coordinate (O) at (0,0); 
				\coordinate (x) at (-2,-2); 
				\coordinate (y) at (4,0); 
				\coordinate (z) at (0,4); 
				%p2
				\coordinate (M) at ($(O)!0.65!(x)$);
				\coordinate (N) at ($(O)!0.6!(y)$);
				\coordinate (P) at ($(O)!0.5!(z)$);
				\coordinate (M') at ($(M)+(0,3)$);\coordinate (N') at ($(N)+(0,3)$);
				\coordinate (L) at ($(M')-(M)$);
				%p3
				\path[name path=d1] (N')--(M');
				\path[name path=d2] (O)--(z);
				\path[name intersections={of=d1 and d2, by=S}];
				\draw [fill = blue!5, line width = .7pt] (N)--(M)--(M')--(N')--cycle;
				\draw [->] (N)--(y) node[below right]{$y$}; \draw [->] (M)--(x) node[below right]{$x$};
				\draw [->] (S)--(z) node[below right]{$z$};
				\draw [dashed] (M)--(O) node[below right]{$O$} circle (1pt)--(N);
				\draw [dashed] (S)--(O); \draw [dashed] (N') node[below]{$(\alpha)$}--(L)--(M');
				%datten
				\draw (2,-1.5) node [below]{$Ax+By+D=0$};
				%p4
				\draw [->,dashed,line width = 1.2pt] (O)--(0,0.8) node[right]{$\overrightarrow{k}$};
			\end{tikzpicture}   & \begin{tikzpicture}[>=stealth,line join=round, line cap=round, scale=0.9]
				%p1
				\coordinate (O) at (0,0); 
				\coordinate (x) at (-3,-3); 
				\coordinate (y) at (2.6,0); 
				\coordinate (z) at (0,2.6); 
				%p2
				\coordinate (A) at ($(O)!0.4!(z)$);
				\coordinate (B) at ($(A)+(2,0)$);
				\coordinate (D) at ($(A)-(2,2)$);
				\coordinate (C) at ($(B)-(2,2)$);
				\coordinate (B') at ($(B)-(A)$);
				\coordinate (C') at ($(C)-(A)$);
				\coordinate (D') at ($(D)-(A)$);
				%p3
				\path[name path=d1] (B)--(C);
				\path[name path=d2] (D)--(C);
				\path[name path=d3] (O)--(y);
				\path[name path=d4] (O)--(x);
				\path[name intersections={of=d2 and d4, by=M}];
				\path[name intersections={of=d1 and d3, by=N}];
				%p5
				\draw [fill = green!15, line width = .7pt] (A)--(D)--(C)--(B)--cycle;
				\draw [->] (A)--(z) node[below right]{$z$}; \draw [->] (N)--(y) node[below right]{$y$};
				\draw [->] (M)--(x) node[below right]{$x$};
				\draw [dashed] (M)--(O) node[below right]{$O$} circle (1pt)--(N); \draw [dashed] (O)--(A);
				\draw [dashed] (D)--(D')--(C')--(C); \draw [dashed] (C')--(B')--(B) node[below left]{$(\alpha)$};
				%datten
				\draw (2,2.4) node [below]{$Cz+D=0$};
			\end{tikzpicture}     \\
			$\textbf{Hình 3}$  & $\textbf{Hình 4}$   \\
			\begin{tikzpicture}[>=stealth,line join=round, line cap=round, scale=0.9]
				%p1
				\coordinate (O) at (0,0); 
				\coordinate (x) at (-2.5,-2.5); 
				\coordinate (y) at (2.6,0); 
				\coordinate (z) at (0,2.6); 
				%p2
				\coordinate (A) at ($(O)!0.45!(y)$);
				\coordinate (B) at ($(A)+(0,2)$);
				\coordinate (D) at ($(A)-(2,2)$);
				\coordinate (C) at ($(B)-(2,2)$);
				\coordinate (B') at ($(B)-(A)$);
				\coordinate (C') at ($(C)-(A)$);
				\coordinate (D') at ($(D)-(A)$);
				%p3
				\path[name path=d1] (B)--(C);
				\path[name path=d2] (D)--(C);
				\path[name path=d3] (O)--(z);
				\path[name path=d4] (O)--(x);
				\path[name intersections={of=d2 and d4, by=M}];
				\path[name intersections={of=d1 and d3, by=N}];
				%p5
				\draw [fill = green!15, line width = .7pt] (A)--(D)--(C)--(B)--cycle;
				\draw [->] (A)--(y) node[below right]{$y$}; \draw [->] (M)--(x) node[below right]{$x$};
				\draw [->] (N)--(z) node[below right]{$z$};
				\draw [dashed] (M)--(O) node[below right]{$O$} circle (1pt)--(N); \draw [dashed] (O)--(A);
				\draw [dashed] (D)--(D')--(C')--(C); \draw [dashed] (C')--(B')--(B);
				%datten
				\draw (2,-1.7) node [below]{$By+D=0$};
				\draw (B) node[below]{$(\alpha)$};
			\end{tikzpicture}   & \begin{tikzpicture}[>=stealth,line join=round, line cap=round, scale=0.9]
				%p1
				\coordinate (O) at (0,0); 
				\coordinate (x) at (-2,-2); 
				\coordinate (y) at (2.6,0); 
				\coordinate (z) at (0,2.6); 
				%p2
				\coordinate (A) at ($(O)!0.6!(x)$);
				\coordinate (B) at ($(A)+(2.2,0)$);
				\coordinate (D) at ($(A)+(0,2)$);
				\coordinate (C) at ($(B)+(0,2)$);
				\coordinate (B') at ($(B)-(A)$);
				\coordinate (C') at ($(C)-(A)$);
				\coordinate (D') at ($(D)-(A)$);
				%p3
				\path[name path=d1] (B)--(C);
				\path[name path=d2] (D)--(C);
				\path[name path=d3] (O)--(y);
				\path[name path=d4] (O)--(z);
				\path[name intersections={of=d1 and d3, by=M}];
				\path[name intersections={of=d2 and d4, by=N}];
				%p5
				\draw [fill = green!15, line width = .7pt] (A)--(D)--(C)--(B)--cycle;
				\draw [->] (A)--(x) node[right]{$x$}; \draw [->] (M)--(y) node[below]{$y$};
				\draw [->] (N)--(z) node[right]{$z$};
				\draw [dashed] (M)--(O) node[above left]{$O$} circle (1pt)--(N); \draw [dashed] (O)--(A);
				\draw [dashed] (D)--(D')--(C')--(C); \draw [dashed] (C')--(B')--(B);
				%datten
				\draw (2,-1.7) node [below]{$Ax+D=0$};
				\draw (B) node[above left]{$(\alpha)$};
			\end{tikzpicture}   \\
			$\textbf{Hình 5}$    &$\textbf{Hình 6}$ \\
		\end{tabular}
		
	\end{enumerate}
	$\textbf{Nhận xét:}$
	\begin{itemize}
		\item Để nhớ các phương trình mặt phẳng đặc biệt thì lấy phương trình $\left(\alpha\right) \colon Ax+By+Cz+D=0$  làm chuẩn.
		\item[+] Mặt phẳng $\left(\alpha\right)$ chứa gốc tọa độ $O\left(0;0;0\right)$ thì $D=0$.
		\item[+] Mặt phẳng $\left(\alpha\right)$ chứa trục tương ứng nào (trục $Ox$, $Oy$, $Oz$) thì ẩn đó không có (không chứa $Ax$, $By$, $Cz$) và $D=0$.
		\item[+] Mặt phẳng $\left(\alpha\right)$ song song với trục tương ứng nào (trục $Ox$, $Oy$, $Oz$) thì ẩn đó không có (không chứa $Ax$, $By$, $Cz$) và $D \neq 0$.
		\item Nếu không nhớ các phương trình mặt phẳng đặc biệt thì nhớ vec-tơ chỉ phương của các trục $Ox$, $Oy$, $Oz$ và vectơ pháp tuyến các mặt phẳng tọa độ $\left(Oxy\right)$, $\left(Oxz\right)$, $\left(Oyz\right)$ để chuyển bài toán lập phương trình mặt phẳng khi biết một điểm và một vectơ pháp tuyến.
		\item[+] Trục $Ox$ có vectơ chỉ phương là $\overrightarrow{i} = \left(1;0;0\right)$.
		\item[+] Trục $Oy$ có vectơ chỉ phương là $\overrightarrow{j} = \left(0;1;0\right)$.
		\item[+] Trục $Ox$ có vectơ chỉ phương là $\overrightarrow{k} = \left(0;0;1\right)$.
		\item[+] Mặt phẳng $\left(Oxy\right)$ có vectơ pháp tuyến là $\overrightarrow{k} = \left(0;0;1\right)$.
		\item[+] Mặt phẳng $\left(Oxz\right)$ có vectơ pháp tuyến là $\overrightarrow{j} = \left(0;1;0\right)$.
		\item[+] Mặt phẳng $\left(Oyz\right)$ có vectơ pháp tuyến là $\overrightarrow{i} = \left(1;0;0\right)$.
	\end{itemize}
\end{dang}

\Opensolutionfile{ans}[ans/CD3_17-23]
\TN

\begin{ex}%[2H5H1-3]
	Trong không gian với hệ tọa độ $O x y z$, phương trình nào dưới đây là phương trình mặt phẳng đi qua điểm $M(1 ; 2 ;-3)$ và có một vectơ pháp tuyến $\vec{n}=(1 ;-2 ; 3)$.
	\choice
	{\True $x-2 y+3 z+12=0$}
	{$x-2 y-3 z-6=0$}
	{$x-2 y+3 z-12=0$}
	{$x-2 y-3 z+6=0$}
	\loigiai{
		Phương trình mặt phẳng đi qua điểm $M(1 ; 2 ;-3)$ và có một vectơ pháp tuyến $\vec{n}=(1 ;-2 ; 3)$ là $$1(x-1)-2(y-2)+3(z+3)=0 \Leftrightarrow x-2y+3 z+12=0.$$
	}
\end{ex}
\begin{ex}%[2H5H1-3] 
	Trong không gian với hệ trục tọa độ $Oxyz$, phương trình mặt phẳng đi qua điểm $A(1 ; 2 ;-3)$ có vectơ pháp tuyến $\vec{n}=(2 ;-1 ; 3)$ là
	\choice
	{\True $2 x-y+3 z+9=0$}
	{$2 x-y+3 z-4=0$}
	{$x-2 y-4=0$}
	{$2 x-y+3 z+4=0$}
	\loigiai{Phương trình mặt phẳng đi qua điểm $A(1 ; 2 ;-3)$ có vectơ pháp tuyến $\vec{n}=(2 ;-1 ; 3)$ là
		\allowdisplaybreaks
		\begin{eqnarray*}
			&&2(x-1)-1 (y-2)+3 (z+3)=0\\
			&\Leftrightarrow& 2 x-2-y+2+3 z+9=0\\
			&\Leftrightarrow& 2 x-y+3 z+9=0.
		\end{eqnarray*}
	}
\end{ex}

\begin{ex}%[2H5H1-3] 
	Trong không gian $O x y z$, phương trình của mặt phẳng đi qua điểm $A(3 ; 0 ;-1)$ và có vectơ pháp tuyến $\vec{n}=(4 ;-2 ;-3)$ là
	\choice
	{$4x-2 y+3z-9=0$}
	{\True $4x-2y-3z-15=0$}
	{$3x-z-15=0$}
	{$4x-2y-3z+15=0$}
	\loigiai{
		Mặt phẳng đi qua điểm $A(3 ; 0 ;-1)$ và có vectơ  pháp tuyến $\vec{n}=(4 ;-2 ;-3)$ có phương trình:
		$$4(x-3)-2(y-0)-3(z+1)=0 \Leftrightarrow 4 x-2 y-3 z-15=0.$$
	}
\end{ex}

\begin{ex}%[2H5H1-3]
	Trong KG $Oxyz$, phương trình mặt phẳng qua $A(-1 ; 1 ;-2)$ và có vectơ  pháp tuyến $\vec{n}=(1 ;-2 ;-2)$ là
	\choice
	{\True $x-2 y-2 z-1=0$}
	{$-x+y-2z-1=0$}
	{$x-2y-2z+7=0$}
	{$-x+y-2z+1=0$}
	\loigiai{
		Mặt phẳng $(P)$ đi qua $A(-1 ; 1 ;-2)$ và có vectơ  pháp tuyến $\vec{n}=(1 ;-2 ;-2)$ nên có phương trình
		$$1(x+1)-2(y-1)-2(z+2)=0 \Leftrightarrow x-2y-2z-1=0.$$
	}
\end{ex}

\begin{ex}%[2H5N1-1] 
	Trong KG $Oxyz$, phương trình mặt phẳng $(Oyz)$ là
	\choice
	{$z=0$}
	{\True $x=0$}
	{$x+y+z=0$}
	{$y=0$}
	\loigiai{
		Mặt phẳng $(Oyz)$ nhận $\vec{i}=(1;0;0)$ làm vectơ  pháp tuyến và đi qua gốc tọa độ $O(0;0;0)$ có phương trình là $x=0$.
	}
\end{ex}

\begin{ex}%[2H5N1-1]
	Trong KG $Oxyz$, phương trình của mặt phẳng $(Oxy)$ là
	\choice
	{\True $z=0$}
	{$x=0$}
	{$y=0$}
	{$x+y=0$}
	\loigiai{
		Phương trình của mặt phẳng $(Oxy)$ là $z=0$.
	}
\end{ex}

\begin{ex}%[2H5N1-1] 
	Trong không gian với hệ toạ độ $Oxyz$, phương trình nào dưới đây là phương trình của mặt phẳng $(Oyz)$?
	\choice
	{$y=0$}
	{\True $x=0$}
	{$y-z=0$}
	{$z=0$}
	\loigiai{
		Mặt phẳng $(Oyz)$ đi qua điểm $O(0 ; 0 ; 0)$ và có vectơ  pháp tuyến là $\vec{i}=(1 ; 0 ; 0)$ nên ta có phương trình mặt phẳng $(O y z)$ là  $1(x-0)+0(y-0)+0(z-0)=0 \Leftrightarrow x=0$.
	}
\end{ex}
\begin{ex}%[2H5N1-1] 
	Trong không gian với hệ tọa độ $O x y z$, phương trình nào sau đây là phương trình của mặt phẳng $O z x$ ?
	\choice
	{$x=0$}
	{$y-1=0$}
	{\True $y=0$}
	{$z=0$}
	\loigiai{
		Ta có mặt phẳng $(Oxz)$ đi qua điểm $O(0 ; 0 ; 0)$ và vuông góc với trục $O y$ nên có VTPT $\vec{n}=(0 ; 1 ; 0)$.\\
		Do đó phương trình của mặt phẳng $(Oxz)$ là $y=0$.
	}
\end{ex}

\begin{ex}%[2H5H1-3] 
	Trong không gian với hệ tọa độ $O x y z$, phương trình mặt phẳng $(P)$ qua $M(0 ;-2 ; 1)$ và có cặp vectơ  chỉ phương $\vec{a}=(1 ; 1 ;-2),$ $ \vec{b}=(1 ; 0 ; 3)$ là
	\choice
	{\True $3 x-5 y-z-6=0$}
	{$3 x-5 y-z+6=0$}
	{$3 x+5 y-z+6=0$}
	{$3 x-5 y+z-6=0$}
	\loigiai{
		Ta có $\vec{n}=[\vec{a}, \vec{b}]=(3 ;-5 ;-1)$.\\
		Mặt phẳng $(P)$ đi qua $M(0 ;-2 ; 1)$ và có vectơ  pháp tuyến $\vec{n}=(3 ;-5 ;-1)$ nên có phương trình $$3(x-0)-5(y+2)-(z-1)=0 \Leftrightarrow 3 x-5 y-z-6=0.$$
	}
\end{ex}
\begin{ex}%[2H5H1-3] 
	Trong không gian với hệ tọa độ $O x y z$, cặp vectơ  $\vec{a}=(2 ; 1 ;-2), $ $\vec{b}=(1 ; 0 ; 2)$ có giá song song với mặt phẳng $(P)$. Phương trình mặt phẳng $(P)$ qua $C(1 ; 1 ; 3)$ là
	\choice
	{$2 x+6 y-z-7=0$}
	{$2 x-6 y-z+5=0$}
	{$2 x+6 y+z+5=0$}
	{\True $2 x-6 y-z+7=0$}
	\loigiai{
		Ta có $\vec{n}=[\vec{a}, \vec{b}]=(2 ;-6 ;-1)$.\\
		Mặt phẳng $(P)$ đi qua $C(1 ; 1 ; 3)$ và có vectơ  pháp tuyến $\vec{n}=(2 ;-6 ;-1)$ nên có phương trình $$2(x-1)-6(y-1)-1(z-3)=0 \Leftrightarrow 2 x-6 y-z+7=0.$$
	}
\end{ex}

\begin{ex}%[2H5H1-3] 
	Trong không gian $O x y z$, cho ba điểm $A(3 ; 0 ; 0),$ $ B(0 ; 1 ; 0)$ và $C(0 ; 0 ;-2)$. Mặt phẳng $(A B C)$ có phương trình là
	\choice
	{$\dfrac{x}{3}+\dfrac{y}{-1}+\dfrac{z}{2}=1$}
	{\True $\dfrac{x}{3}+\dfrac{y}{1}+\dfrac{z}{-2}=1$}
	{$\dfrac{x}{3}+\dfrac{y}{1}+\dfrac{z}{2}=1$}
	{$\dfrac{x}{-3}+\dfrac{y}{1}+\dfrac{z}{2}=1$}
	\loigiai{Theo công thức phương trình mặt chắn, ta có
		$(A B C)\colon  \dfrac{x}{3}+\dfrac{y}{1}+\dfrac{z}{-2}=1$.}
\end{ex}
\begin{ex}%[2H5H1-3] 
	Trong không gian với hệ tọa độ $O x y z$, cho ba điểm $A(0 ; 1 ; 2), $ $B(2 ;-2 ; 1),$ $ C(-2 ; 1 ; 0)$. Khi đó, phương trình mặt phẳng $(A B C)$ là $a x+y-z+d=0$. Hãy xác định $a$ và $d$.
	\choice
	{\True $a=1,$ $ d=1$}
	{$a=6, $ $d=-6$}
	{$a=-1, $ $d=-6$}
	{$a=-6, $ $d=6$}
	\loigiai{
		Ta có $\overrightarrow{A B}=(2 ;-3 ;-1) ; \overrightarrow{A C}=(-2 ; 0 ;-2)$.
		
		$$[\overrightarrow{A B}, \overrightarrow{A C}]=\left(\left|\begin{array}{cc}-3 & -1 \\ 0 & -2\end{array}\right| ;\left|\begin{array}{cc}-1 & 2 \\ -2 & -2\end{array}\right| ;\left|\begin{array}{cc}2 & -3 \\ -2 & 0\end{array}\right|\right)=(6 ; 6 ;-6).$$
		Chọn $\vec{n}=\dfrac{1}{6}[\overrightarrow{A B} ; \overrightarrow{A C}]=(1 ; 1 ;-1)$ là một VTPT của mp$(A B C)$. Ta có 
		$$(A B C)\colon x+y-1-z+2=0 \Leftrightarrow x+y-z+1=0.$$ Vậy $a=1,$ $ d=1$.
	}
\end{ex}

\begin{ex}%[2H5H1-3] 
	Trong không gian $O x y z$, cho điểm $A(0 ;-3 ; 2)$ và mặt phẳng $(P)\colon 2 x-y+3 z+5=0$. Mặt phẳng đi qua $A$ và song song với $(P)$ có phương trình là
	\choice
	{$2 x-y+3 z+9=0$}
	{$2 x+y+3 z-3=0$}
	{$2 x+y+3 z+3=0$}
	{\True $2 x-y+3 z-9=0$}
	\loigiai{
		Gọi $(Q)$ là mặt phẳng cần tìm.\\
		Theo bài $(Q) \parallel (P) \Rightarrow(Q)\colon 2 x-y+3 z+m=0\,(m \neq 5)$.\\
		Mà $(Q)$ qua $A \Leftrightarrow 2\cdot 0-(-3)+3\cdot 2+m=0 \Leftrightarrow m=-9$.\\
		Vậy $(Q)\colon 2 x-y+3 z-9=0$.
	}
\end{ex}
\begin{ex}%[2H5H1-3] 
	Trong không gian $O x y z$, cho hai điểm $A(0 ; 0 ; 1)$ và $B(1 ; 2 ; 3)$. Mặt phẳng đi qua $A$ và vuông góc với $A B$ có phương trình là
	\choice
	{$x+2 y+2 z-11=0$}
	{\True $x+2 y+2 z-2=0$}
	{$x+2 y+4 z-4=0$}
	{$x+2 y+4 z-17=0$}
	\loigiai{
		Ta có $\overrightarrow{A B}=(1 ; 2 ; 2)$.\\
		Mặt phẳng đi qua $A$ và vuông góc với $A B$ nên nhận $\overrightarrow{A B}=(1 ; 2 ; 2)$ làm vectơ pháp tuyến có phương trình $$1(x-0)+2(y-0)+2(z-1)=0 \Leftrightarrow x+2 y+2 z-2=0.$$
	}
\end{ex}

\begin{ex}%[2H5H1-3] 
	Trong mặt phẳng $O x y z$, cho hai điểm $A(1 ; 0 ; 0)$ và $B(3 ; 2 ; 1)$. Mặt phẳng đi qua $A$ và vuông góc với $A B$ có phương trình là
	\choice
	{\True $2 x+2 y+z-2=0$}
	{$4 x+2 y+z-17=0$}
	{$4 x+2 y+z-4=0$}
	{$2 x+2 y+z-11=0$}
	\loigiai{
		Mặt phẳng đi qua $A$ và vuông góc với $A B$ nên nhận $\overrightarrow{A B}=(2 ; 2 ; 1)$ làm vectơ pháp tuyến.\\
		Vậy phương trình mặt phẳng cần tìm là $$2(x-1)+2 y+z=0 \Leftrightarrow 2 x+2 y+z-2=0.$$
	}
\end{ex}
\begin{ex}%[2H5H1-3] 
	Trong KG $Oxyz$, cho hai điểm $A(0 ; 1 ; 1)$  và $B(1 ; 2 ; 3)$. Viết phương trình của mặt phẳng $(P)$ đi qua $A$ và vuông góc với đường thẳng $A B$.
	\choice
	{\True $x+y+2 z-3=0$}
	{$x+y+2z-6=0$}
	{$x+3y+4z-7=0$}
	{$x+3y+4z-26=0$}
	\loigiai{
		Mặt phẳng $(P)$ đi qua $A(0 ; 1 ; 1)$ và nhận vectơ $\overrightarrow{A B}=(1 ; 1 ; 2)$ là vectơ pháp tuyến
		$$(P)\colon 1(x-0)+1(y-1)+2(z-1)=0 \Leftrightarrow x+y+2 z-3=0.$$
	}
\end{ex}

\begin{ex}%[2H5H1-3] 
	Trong không gian $O x y z$, cho ba điểm $A(-1 ; 1 ; 1),$ $ B(2 ; 1 ; 0),$ $ C(1 ;-1 ; 2)$. Mặt phẳng đi qua $A$ và vuông góc với đường thẳng $B C$ có phương trình là
	\choice
	{$3 x+2 z+1=0$}
	{\True $x+2 y-2 z+1=0$}
	{$x+2 y-2 z-1=0$}
	{$3 x+2 z-1=0$}
	\loigiai{
		Ta có $\overrightarrow{B C}=(-1 ;-2 ; 2)$ là một vectơ  pháp tuyến của mặt phẳng $(P)$ cần tìm.\\
		$\vec{n}=-\overrightarrow{B C}=(1 ; 2 ;-2)$ cũng là một vectơ  pháp tuyến của mặt phẳng $(P)$.\\
		Vậy phương trình mặt phẳng $(P)$ là $x+2 y-2 z+1=0$.
	}
\end{ex}
\begin{ex}%[2H5H1-3] 
	Trong không gian với hệ tọa độ $O x y z$, cho các điểm $A(0 ; 1 ; 2), $ $B(2 ;-2 ; 1)$, $C(-2 ; 0 ; 1)$. Phương trình mặt phẳng đi qua $A$ và vuông góc với $B C$ là
	\choice
	{$y+2 z-5=0$}
	{$2 x-y-1=0$}
	{\True $2 x-y+1=0$}
	{$-y+2 z-5=0$}
	\loigiai{
		Ta có vectơ  pháp tuyến của mặt phẳng $(P)$ là $\overrightarrow{B C}=(-4 ; 2 ; 0)$.\\
		Phương trình mặt phẳng $(P)$ là
		$$-4(x-0)+2(y-1)+0(z-2)=0 \Leftrightarrow-4 x+2 y-2=0 \Leftrightarrow 2 x-y+1=0.$$}
\end{ex}

\begin{ex}%[2H5H1-3] 
	Trong không gian $O x y z$, mặt phẳng $(P)$ đi qua hai điểm $A(0 ; 1 ; 0)$, $B(2 ; 3 ; 1)$ và vuông góc với mặt phẳng $(Q)\colon x+2 y-z=0$ có phương trình là
	\choice
	{$4x-3y+2z+3=0$}
	{\True $4 x-3 y-2 z+3=0$}
	{$2 x+y-3 z-1=0$}
	{$4 x+y-2 z-1=0$}
	\loigiai{
		Ta có $\overrightarrow{A B}=(2 ; 2 ; 1)$, vectơ  pháp tuyến mặt phẳng $(Q)\colon \overrightarrow{n}_Q=(1 ; 2 ;-1)$.\\
		Theo đề bài ta có vectơ  pháp tuyến mặt phẳng $(P)\colon \overrightarrow{n}_P=\left[\overrightarrow{n}_Q , \overrightarrow{A B}\right]=(4 ;-3 ;-2)$.\\
		Phương trình mặt phẳng $(P)$ có dạng $4 x-3 y-2 z+C=0$.\\
		Mặt phẳng $(P)$ đi qua $A(0 ; 1 ; 0)$ nên $-3+C=0 \Leftrightarrow C=3$.\\
		Vậy phương trình mặt phẳng $(P)$ là $4 x-3 y-2 z+3=0$.
	}
\end{ex}
\begin{ex}%[2H5H1-3] 
	Cho hai mặt phẳng $(\alpha)\colon  3 x-2 y+2 z+7=0,$ $(\beta)\colon 5 x-4 y+3 z+1=0$. Phương trình mặt phẳng đi qua gốc tọa độ $O$ đồng thời vuông góc với cả $(\alpha)$ và $(\beta)$ là
	\choice
	{$2 x-y-2 z=0$}
	{$2 x-y+2 z=0$}
	{\True $2 x+y-2 z=0$}
	{$2 x+y-2 z+1=0$}
	\loigiai{
		vectơ pháp tuyến của hai mặt phẳng lần lượt là $\overrightarrow{n}_\alpha=(3 ;-2 ; 2), \overrightarrow{n}_\beta=(5 ;-4 ; 3)$.\\
		Suy ra $\left[\overrightarrow{n}_\alpha ; \overrightarrow{n}_\beta\right]=(2 ; 1 ;-2)$ là vectơ pháp tuyến của mặt phẳng cần tìm.\\
		Phương trình mặt phẳng đi qua gốc tọa độ $O, $ có vectơ pháp tuyến $\vec{n}=(2 ; 1 ;-2)$ là $2 x+y-2 z=0$.
	}
\end{ex}

\begin{ex}%[2H5H1-3] 
	Trong không gian với hệ tọa độ $O x y z$, cho điểm $A(2 ; 4 ; 1) ;$ $ B(-1 ; 1 ; 3)$ và mặt phẳng $(P)\colon x-3 y+2 z-5=0$. Một mặt phẳng $(Q)$ đi qua hai điểm $A, B$ và vuông góc với mặt phẳng $(P)$ có dạng $a x+b y+c z-11=0$. Khẳng định nào sau đây là đúng?
	\choice
	{\True $a+b+c=5$}
	{$a+b+c=15$}
	{$a+b+c=-5$}
	{$a+b+c=-15$}
	\loigiai{Vì $(Q)$ vuông góc với $(P)$ nên $(Q)$ nhận vectơ pháp tuyến $\vec{n}=(1 ;-3 ; 2)$ của $(P)$ làm vectơ chỉ phương.\\
		Mặt khác $(Q)$ đi qua $A$ và $B$ nên $(Q)$ nhận $\overrightarrow{A B}=(-3 ;-3 ; 2)$ làm vectơ chỉ phương.\\
		$(Q)$ nhận $\overrightarrow{n}_Q=[\vec{n}, \overrightarrow{A B}]=(0 ; 8 ; 12)$ làm vectơ pháp tuyến.\\
		Vậy phương trình mặt phẳng $(Q)\colon  0(x+1)+8(y-1)+12(z-3)=0\Leftrightarrow 2 y+3 z-11=0$.\\
		Vậy $a+b+c=5$.}
\end{ex}

\begin{ex}%[2H5V1-3]
	Trong không gian $O x y z$, cho hai mặt phẳng $(P)\colon  x-3 y+2 z-1=0$, 
	$(Q)\colon  x-z+2=0$. Mặt phẳng $(\alpha)$ vuông góc với cả $(P)$ và $(Q)$ đồng thời cắt trục $O x$ tại điểm có hoành độ bằng 3 . Phương trình của $(\alpha)$ là
	\choice
	{\True $x+y+z-3=0$}
	{$x+y+z+3=0$}
	{$-2 x+z+6=0$}
	{$-2 x+z-6=0$}
	\loigiai{
		$(P)$ có vectơ pháp tuyến $\overrightarrow{n}_P=(1 ;-3 ; 2),(Q)$ có vectơ pháp tuyến $\overrightarrow{n}_Q=(1 ; 0 ;-1)$.\\
		Vì mặt phẳng $(\alpha)$ vuông góc với cả $(P)$ và $(Q)$ nên $(\alpha)$ có một vectơ pháp tuyến là $\left[\overrightarrow{n}_P, \overrightarrow{n}_Q\right]=(3 ; 3 ; 3)=3(1 ; 1 ; 1)$.\\
		Vì mặt phẳng $(\alpha)$ cắt trục $O x$ tại điểm có hoành độ bằng $3$ nên $(\alpha)$ đi qua điểm $M(3 ; 0 ; 0)$.\\
		Vậy $(\alpha)$ đi qua điểm $M(3 ; 0 ; 0)$ và có vectơ pháp tuyến $\overrightarrow{n}_\alpha=(1 ; 1 ; 1)$ nên $(\alpha)$ có phương trình: $x+y+z-3=0$.
	}
\end{ex}

\begin{ex}%[2H5H1-3] 
	Trong không gian với hệ trục tọa độ $O x y z$, cho mặt phẳng $(P)\colon  a x+b y+c z-9=0$ chứa hai điểm $A(3 ; 2 ; 1),$ $ B(-3 ; 5 ; 2)$ và vuông góc với mặt phẳng $(Q)\colon  3 x+y+z+4=0$. Tính tổng $S=a+b+c$?
	\choice
	{$S=-12$}
	{$S=2$}
	{\True $S=-4$}
	{$S=-2$}
	\loigiai{
		$\overrightarrow{A B}=(-6 ; 3 ; 1)$.\\
		$\overrightarrow{n}_{(Q)}=(3 ; 1 ; 1)$ là vectơ pháp tuyến  của $(Q)$.\\
		Mặt phẳng $(P)$ chứa hai điểm $A(3 ; 2 ; 1),$ $ B(-3 ; 5 ; 2)$ và vuông góc với mặt phẳng $(Q)$.\\
		Suy ra $ \overrightarrow{n}_{(P)}=\left[\overrightarrow{A B}, \overrightarrow{n}_{(Q)}\right]=(2 ; 9 ;-15)$ là vectơ pháp tuyến  của $(P)$.\\
		$A(3 ; 2 ; 1) \in(P)\Rightarrow(P)\colon 2 x+9 y-15 z-9=0$ hoặc $(P)\colon -2 x-9 y+15 z+9=0$.\\
		Mặt khác $(P)\colon a x+b y+c z-9=0 \Rightarrow a=2 ; $ $b=9 ;$ $ c=-15$.\\
		Vậy $S=a+b+c=2+9+(-15)=-4$.
	}
\end{ex}
\begin{ex}%[2H5H1-3] 
	Trong không gian $O x y z$, phương trình của mặt phẳng $(P)$ đi qua điểm $B(2 ; 1 ;-3)$, đồng thời vuông góc với hai mặt phẳng $(Q)\colon x+y+3 z=0,$ $(R)\colon 2 x-y+z=0$ là
	\choice
	{$4 x+5 y-3 z+22=0$}
	{$4 x-5 y-3 z-12=0$}
	{$2 x+y-3 z-14=0$}
	{\True $4 x+5 y-3 z-22=0$}
	\loigiai{
		Mặt phẳng $(Q)\colon x+y+3 z=0,$ $(R)\colon 2 x-y+z=0$ có các vectơ pháp tuyến lần lượt là $\overrightarrow{n}_1=(1 ; 1 ; 3)$ và $\overrightarrow{n}_2=(2 ;-1 ; 1)$.\\
		Vì $(P)$ vuông góc với hai mặt phẳng $(Q),$ $(R)$ nên $(P)$ có vectơ pháp tuyến là $\vec{n}=\left[\overrightarrow{n}_1, \overrightarrow{n}_2\right]=(4 ; 5 ;-3)$.\\
		Ta lại có $(P)$ đi qua điểm $B(2 ; 1 ;-3)$ nên $$(P)\colon 4(x-2)+5(y-1)-3(z+3)=0\Leftrightarrow 4 x+5 y-3 z-22=0.$$
	}
\end{ex}
\Closesolutionfile{ans}
\indapan{10}{ans/CD3_17-23}

\Opensolutionfile{ans}[ans/CD3_17-23DS]
\TNTF

\begin{ex}%[2H5H1-3]
	Trong KG $Oxyz$, cho điểm $A(1; -2; 3)$ và hai vectơ  $\overrightarrow{v}=(-1; 2; 3)$, $\overrightarrow{u}=(-2; 0; 1)$.
	\choiceTF
	{\True $\overrightarrow{v}=-\overrightarrow{i}+2\overrightarrow{j}+3\overrightarrow{k}$}
	{$\overrightarrow{u}\perp \overrightarrow{v}$}
	{\True Phương trình mặt phẳng đi qua điểm $A(1; -2; 3)$ và vuông góc với giá của vectơ  $\overrightarrow{v}=(-1; 2; 3)$ là $x-2y-3z+4=0$}
	{Phương trình mặt phẳng đi qua điểm $A(1; -2; 3)$ và vuông góc với giá của vectơ $\overrightarrow{u}=(-2; 0; 1)$ là $2x-y+1=0$}
	\loigiai{
		\begin{itemchoice}
			\itemch Đúng. \\Ta có $\overrightarrow{v}=(-1; 2; 3) \Leftrightarrow \overrightarrow{v}=-\overrightarrow{i}+2\overrightarrow{j}+3\overrightarrow{k}$.
			\itemch Sai.\\ Ta có $\overrightarrow{u}\cdot \overrightarrow{v} = 2 + 0 + 3 = 5\neq 0 \Rightarrow \overrightarrow{u}\not \perp \overrightarrow{v}$.
			\itemch Đúng.\\ Mặt phẳng đi qua điểm $A(1; -2; 3)$ và vuông góc với giá của vectơ  $\overrightarrow{v}=(-1; 2; 3)$ có phương trình
			\[-1(x-1)+2(y+2)+3(z-3)=0\Leftrightarrow x-2y-3z+4=0.\]
			\itemch Sai.\\ Mặt phẳng đi qua điểm $A(1; -2; 3)$ và vuông góc với giá của vectơ $\overrightarrow{u}=(-2; 0; 1)$ có phương trình
			\[ -2(x-1)+0(y+2)+1(z-3)=0\Leftrightarrow 2x -z +1=0.\]
		\end{itemchoice}
	}
\end{ex}
\begin{ex}%[2H5H1-3]
	Trong KG $Oxyz$, cho ba điểm $A(1;1;4)$, $B(2;7;9)$, $C(0;9;13)$.
	\choiceTF
	{\True $\overrightarrow{AB}=\overrightarrow{i}+6\overrightarrow{j}+5\overrightarrow{k}$}
	{$\overrightarrow{AB}\perp \overrightarrow{AC}$}
	{\True Phương trình mặt phẳng đi qua ba điểm $A$, $B$, $C$ là $x-y+z-4=0$}
	{Phương trình mặt phẳng đi qua ba điểm $A$, $B$, $C$ là $2x+y-z-2=0$}
	\loigiai{
		\begin{itemchoice}
			\itemch $\overrightarrow{AB}=(1; 6; 5) \Rightarrow \overrightarrow{AB}=\overrightarrow{i}+6\overrightarrow{j}+5\overrightarrow{k}$.
			\itemch Ta có $\overrightarrow{AC}=(-1; 8; 9)$, khi đó $\overrightarrow{AB}\cdot \overrightarrow{AC}= -1 + 48 + 45 = 92 \neq 0 \Rightarrow \overrightarrow{AB}\not\perp \overrightarrow{AC}$.
			\itemch Ta có $[\overrightarrow{AB}, \overrightarrow{AC}]= (14;-14;14)=14(1;-1;1)$.\\
			Mặt phẳng $(ABC)$ đi qua điểm $A$ và có vectơ pháp tuyến $\overrightarrow{n}=(1;-1;1)$ là $x - y + z - 4 = 0$.
			\itemch Phương trình mặt phẳng đi qua ba điểm $A,B,C$ là $x - y + z - 4 = 0$.			
		\end{itemchoice}
	}
\end{ex}
\begin{ex}%[2H5H1-3]
	Trong KG $Oxyz$, cho điểm $M(2; -1; 4)$ và mặt phẳng $(P)\colon 3x - 2y+z+1=0$.
	\choiceTF
	{\True Mặt phẳng $(P)$ có một vec-tơ pháp tuyến là $\overrightarrow{n}=(-3; 2; -1)$}
	{Mặt phẳng $(P)$ đi qua điểm $B(-1; 1; 2)$}
	{\True Phương trình của mặt phẳng $(Q)$ đi qua điểm $M$ và song song với mặt phẳng $(P)$ là $3x-2y+z-12=0$}
	{Phương trình của mặt phẳng $(R)$ đi qua điểm $O$, $M$ và vuông góc với mặt phẳng $(P)$ là $7x+my+nz=0$. Khi đó $m+n=8$}
	\loigiai{
		\begin{itemchoice}
			\itemch Mặt phẳng $(P)$ có vec-tơ pháp tuyến là $\overrightarrow{n}=(3;-2;1)=-(-3;2;-1)$.
			\itemch Ta có $3\cdot (-1)-2\cdot(1)+2+1=-2\neq 0$. Suy ra mặt phẳng $(P)$ không đi qua điểm $B$.
			\itemch Mặt phẳng $(Q)$ song song với mặt phẳng $(P)$ có dạng $3x-2y+z+d=0$.\\
			Vì $M\in (Q) \Rightarrow d = -12$. Vậy phương trình mặt phẳng $(Q)\colon 3x-2y+z-12=0$.
			\itemch Ta có mặt phẳng $(R)$ đi qua điểm $O$, $M$ và vuông góc với mặt phẳng $(P)$ cho nên mặt phẳng $(R)$ có vec-tơ pháp tuyến là $\overrightarrow{n}_R=\left[\overrightarrow{OM}, \overrightarrow{n}_P\right] =(7;10;-1)$.\\
			Mặt phẳng $(R)$ đi qua điểm $O$ và có vec-tơ pháp tuyến $\overrightarrow{n}_R=(7;10;-1)$ có phương trình $7x+10y-z=0$. Khi đó $m+n=9$.
	\end{itemchoice}}
\end{ex}
\begin{ex}%[2H5H1-3]
	Trong KG $Oxyz$, cho hai điểm $A(1;0;0)$, $B(4;1;2)$.
	\choiceTF
	{$\overrightarrow{AB}=(5;1;2)$}
	{\True Nếu $I$ là trung điểm đoạn thẳng $AB$ thì $I\left(\dfrac{5}{2};\dfrac{1}{2};1\right)$}
	{\True Mặt phẳng $(\alpha) $ đi qua $A$ và vuông góc với $AB$ có phương trình là $3x+y+2z-3=0$}
	{Mặt phẳng trung trực của đoạn thẳng $AB$ có phương trình là $3x+y+2z-12=0$}
	\loigiai{
		\begin{itemchoice}			
			\itemch Ta có $\overrightarrow{AB}=(3;1;2)$.
			\itemch Nếu $I$ là trung điểm đoạn thẳng $AB$ thì $I\left(\dfrac{5}{2};\dfrac{1}{2};1\right)$.
			\itemch Mặt phẳng $(\alpha) $ vuông góc với $AB$ cho nên mặt phẳng $(\alpha)$ có vec-tơ pháp tuyến $\overrightarrow{n}=\overrightarrow{AB}=(3;1;2)$.\\
			Mặt phẳng $(\alpha)$ đi qua $A$ và có vec-tơ pháp tuyến $\overrightarrow{n}=(3;1;2)$ có phương trình là $3x+y+2z-3=0$.
			\itemch Mặt phẳng trung trực của đoạn thẳng $AB$ là mặt phẳng đi qua điểm $I$ và vuông góc $AB$ nên có phương trình là
			\[\begin{array}{l} {3\left(x-\dfrac{5}{2} \right)+y-\dfrac{1}{2}+2\left(z-1\right)=0} \\ {\Leftrightarrow 3x+y+2z-10=0} \end{array}\]
	\end{itemchoice}}
\end{ex}
\begin{ex}%[2H5H1-3]
	Trong không gian với hệ trục tọa độ $Oxyz$, cho điểm $M(1;2;3)$. Gọi $A$, $B$, $C$ lần lượt là hình chiếu vuông góc của $M$ trên các trục $Ox$, $Oy$, $Oz$.
	\choiceTF
	{\True Điểm $A$ có tọa độ là $A\left(1;0;0\right)$}
	{Điểm $B$ có tọa độ là $B\left(1;2;0\right)$}
	{$\overrightarrow{BC}=(-1;-2;3)$}
	{Phương trình mặt phẳng $(ABC)$ là $\dfrac{x}{1}+\dfrac{y}{2}+\dfrac{z}{3}=0$}
	\loigiai{
		\begin{itemchoice}
			\itemch Điểm $A$ có tọa độ là $A\left(1;0;0\right)$.
			\itemch Điểm $B$ có tọa độ là $B\left(0;2;0\right)$.
			\itemch Ta có $C(0;0;3)$. Suy ra $\overrightarrow{BC}=(0;-2;3)$.
			\itemch Mặt phẳng $(ABC)$ là $\dfrac{x}{1}+\dfrac{y}{2}+\dfrac{z}{3}=1$.
		\end{itemchoice}
	}
\end{ex}
\begin{ex}%[2H5H2-3]%Câu 28
	Trong KG $Oxyz$, cho hai điểm $A(1 ; 0 ; 0), B(4 ; 1 ; 2)$. Mệnh đề nào sau đây đúng hay sai?
	\choiceTF
	{\True $\overrightarrow{AB}=(3 ; 1 ; 2)$}
	{\True Mặt phẳng đi qua $\mathrm{A}$ và vuông góc với $AB$ có phương trình là $3x+y+2z-3=0$}
	{\True Nếu $I$ là trung điểm đoạn thẳng $AB$ thì $I\left(\dfrac{5}{2} ; \dfrac{1}{2} ; 1\right)$}
	{\True Mặt phẳng trung trực đoạn thẳng $AB$ có phương trình là $3x+y+2z-12=0$}
	\loigiai{
		\begin{itemchoice}
			\itemch Đúng.\\Do $A(1 ; 0 ; 0), B(4 ; 1 ; 2)$ nên ta có $\overrightarrow{A B}=(3 ; 1 ; 2)$.
			\itemch Đúng.\\Gọi $(Q)$ là mặt phẳng đi qua $A(1 ; 0 ; 0)$ và vuông góc với $A B$ suy ra mặt phẳng $(Q)$ nhận vectơ $\overrightarrow{A B}=(3 ; 1 ; 2)$ làm vectơ pháp tuyến.\\ 
			Vậy phương trình mặt phẳng $(Q)$ cần tìm có dạng: $3(x-1)+y+2 z=0 \Leftrightarrow 3 x+y+2 z-3=0$.
			\itemch Đúng.\\$I$ là trung điểm đoạn thẳng $A B$ nên $I\left(\dfrac{5}{2} ; \dfrac{1}{2} ; 1\right)$.
			\itemch Đúng. \\Mặt phẳng trung trực đoạn thẳng $AB$ là mặt phẳng đi qua $\mathrm{I}$ và vuông góc $AB$ nên có phương trình là
			$$3\left(x-\dfrac{5}{2}\right)+y-\dfrac{1}{2}+2(z-2)=0 \\
			\Leftrightarrow 3x+y+2z-12=0..$$
		\end{itemchoice}
	}
\end{ex}
\begin{ex} %[2H5H2-3]%Cau 29
	Trong không gian với hệ trục tọa độ $Oxyz$, cho điểm $M(1 ; 2 ; 3)$. Gọi $A, B, C$ lần lượt là hình chiếu vuông góc của $M$ trên các trục $Ox, Oy, Oz$. Mệnh đề nào sau đây đúng hay sai?
	\choiceTF
	{\True Điểm $A$ có tọa độ là $A(1 ; 0 ; 0)$}
	{Điểm $B$ có tọa độ là $B(1 ; 2 ; 0)$}
	{Phương trình mặt phẳng $(A B C)$ là $\dfrac{x}{1}+\dfrac{y}{2}+\dfrac{z}{3}=0$}
	{\True Phương trình mặt phẳng $(A B C)$ là $\dfrac{x}{1}+\dfrac{y}{2}+\dfrac{z}{3}=1$}
	\loigiai{
		\begin{itemchoice}
			\itemch Đúng.\\Do $A$ là hình chiếu vuông góc của $M$ trên trục $Ox \Rightarrow A(1 ; 0 ; 0)$.
			\itemch Sai.\\Do $B$ là hình chiếu vuông góc của $M$ trên trục $Oy \Rightarrow B(0 ; 2 ; 0)$.
			\itemch Sai.\\$C$ là hình chiếu vuông góc của $M$ trên trục $Oz \Rightarrow C(0 ; 0 ; 3)$.
			\itemch Đúng.\\Vì 3 điểm $A(1;0;0);B(0;2;0);C(0;0;3)$ thuộc $Ox;Oy;Oz$ nên phương trình mặt phẳng $(A B C)$ là $\dfrac{x}{1}+\dfrac{y}{2}+\dfrac{z}{3}=1$.
		\end{itemchoice}
	}
\end{ex}	

\begin{ex}%[2H5H2-3]%Cau 30
	Trong KG $Oxyz$, cho điểm $A(3 ; 5 ; 2)$.  Gọi $A_{1}, A_{2}, A_{3}$ lần lượt là hình chiếu của điểm $A$ lên các mặt phẳng $(Oxy),(Oyz),(Oxz)$. Mệnh đề nào sau đây đúng hay sai?
	\choiceTF
	{\True Điểm $A_{1}$ có tọa độ là $(3 ; 5 ; 0)$}
	{\True Phương trình mặt phẳng đi qua các điểm $A_{1}, A_{2}, A_{3}$ là $10 x+6 y+15z-60=0$}
	{Phương trình mặt phẳng đi qua các điểm $A_{1}, A_{2}, A_{3}$ là $10 x+6 y+15 z-90=0$}
	{Phương trình mặt phẳng đi qua các điểm $A_{1}, A_{2}, A_{3}$ là $\dfrac{x}{3}+\dfrac{y}{5}+\dfrac{z}{2}=1$}
	\loigiai{
		\begin{enumerate}
			\itemch Đúng.\\Vì $A_1$ là hình chiếu của $A$ trên mặt phẳng $(Oxy)$ nên $A_1$ có tọa độ là $(3;5;0)$.
			\itemch Đúng.\\Mặt phẳng đi qua $A_1(3;5;0);A_2(0;5;2),A_3(3;0;2)$ có vectơ pháp tuyến được tính từ tích có hướng của hai vectơ
			$$\overrightarrow{A_1A_2}=(-3;0;2)$$
			$$\overrightarrow{A_1A_3}=(0;-5;2).$$
			Tích có hướng của hai vectơ này là
			$$\overrightarrow{n}=\left[ \overrightarrow{A_1A_2}, \overrightarrow{A_1A_3}\right]=(10;6;15).$$
			Phương trình mặt phẳng là
			$10(x-3)+6(y-5)+15(z-10)=0$\\$\Rightarrow 10x+6y+15-60=0$.
			\itemch Sai.\\Vì phương trình mặt phẳng là $10(x-3)+6(y-5)+15(z-10)=0$\\ $\Rightarrow 10x+6y+15-60=0$.
			\itemch Sai.\\Phương trình mặt phẳng đi qua các điểm $A_{1}, A_{2}, A_{3}$ là $\dfrac{x}{3}+\dfrac{y}{5}+\dfrac{z}{2}=1$\\
			Để kiểm tra phương trình này, ta nhân cả hai vế phương trình $\dfrac{x}{3}+\dfrac{y}{5}+\dfrac{z}{2}=1$ với 30 ta được
			$$10x+6y+15z-30=0 \neq 10x+6y+15-60=0.$$
		\end{enumerate}
	}
\end{ex}
\begin{ex} %[2H5H2-3]%Cau 31
	Trong KG $Oxyz$, cho hai điểm $A(4 ; 0 ; 1)$ và $B(-2 ; 2 ; 3)$. Mệnh đề nào sau đây đúng hay sai?
	\choiceTF
	{\True $\overrightarrow{AB}=(-6 ; 2 ; 2)$}
	{\True Nếu $I$ là trung điểm đoạn thẳng $AB$ thì $I(1 ; 1 ; 2)$}
	{ Mặt phẳng trung trực của đoạn thẳng $AB$ có phương trình là $x+y+2z-6=0$}
	{\True Mặt phẳng trung trực của đoạn thẳng $AB$ có phương trình là $3x-y-z=0$}
	
	\loigiai{
		\begin{enumerate}
			\itemch Đúng.\\Vì $\overrightarrow{AB}=(-6;2;2)$.
			\itemch Đúng.\\Vì tọa độ trung điểm $I=\left(\dfrac{4-2}{2};\dfrac{0+2}{2};\dfrac{1+3}{2}\right)=\left(1;1;2\right)$.
			\itemch Sai.\\
			Mặt phẳng trung trực của đoạn thẳng $AB$ là mặt phẳng đi qua trung điểm $I$ và vuông góc với $\overrightarrow{AB}$.\\
			Phương trình mặt phẳng có dạng
			$$a(x-1)+b(y-1)+c(z-2)=0.$$
			Với $\overrightarrow{n}=(a;b;c)$ là các vectơ pháp tuyến của mặt phẳng trung trực.\\
			Vì mặt phẳng trung trực vuông góc với $\overrightarrow{AB}=(-6;2;2)$ nên ta chọn vectơ pháp tuyến là $(-6;2;2)$.\\
			Do đó phương trình mặt phẳng là
			$$-6(x-1)+2(y-1)+2(z-2)=0 \Leftrightarrow 3x-y-z=0.$$
			\itemch Đúng.\\Vì phương trình mặt phẳng là
			$-6(x-1)+2(y-1)+2(z-2)=0 \Leftrightarrow 3x-y-z=0$.
		\end{enumerate}
	}
\end{ex}
\begin{ex} %[2H5H2-6]%Cau 32
	Trong không gian hệ tọa độ $Oxyz$, cho $A(1 ; 2 ;-1) ; B(-1 ; 0 ; 1)$ và mặt phẳng $(P)\colon x+2y-z+1=0$. Mệnh đề nào sau đây đúng hay sai?
	\choiceTF
	{\True $\overrightarrow{AB}=(1 ; 1;-1)$}
	{\True Phương trình mặt phẳng $(Q)$ qua $A,B$ và vuông góc với $(P)$ là $x+z=0$}
	{\True Khoảng cách từ điểm $A$ đến mặt phẳng $(P)$ là: $\mathrm{d}(A,(P))=\dfrac{7 \sqrt{6}}{6}$}
	{ Phương trình mặt phẳng $(Q)$ qua $A, B$ và vuông góc với $(P)$ là $3x-y+z=0$}
	\loigiai{
		\begin{enumerate}
			\itemch Đúng.\\Vì $\overrightarrow{AB}=(-2;-2;2)=-\dfrac{1}{2}(-2;-2;2)=(1 ; 1;-1)$.
			\itemch Đúng.\\
			vectơ pháp tuyến của mặt phẳng $(P)$ là $(1;2;-1)$\\
			Mặt phẳng $(Q$ chứa $\overrightarrow{AB}$ và vuông góc với $(P)$ nên vectơ pháp tuyến của $(Q)$ là tích có hướng của $\overrightarrow{AB}$ và vectơ pháp tuyến của $(P)$
			$$\overrightarrow{n_Q}=\left[ \overrightarrow{AB}, \overrightarrow{n_P}\right] =(-2;0;-2)=(1;0;1).$$
			Vậy phương trình mặt phẳng $(Q)$ qua $A,B$ và vuông góc với $(P)$ là $1(x-1)+0+1(z+1)=x+z=0$.
			\itemch Đúng.\\Khoảng cách từ điểm $A(x_1;y_1;z_1)$ đến mặt phẳng $(P)=ax+by+cz+d=0$ là
			$$\mathrm{d}(A,P)=\dfrac{\left|1\cdot 1+2\cdot 2-(-1)+1\right|}{\sqrt{1^2+2^2+(-1)^2}}=\dfrac{7}{\sqrt{6}}=\dfrac{7\sqrt{6}}{6}.$$
			\itemch Sai.\\Vì phương trình mặt phẳng $(Q)$ qua $A,B$ và vuông góc với $(P)$ là $x+z=0$.
		\end{enumerate}
	}
\end{ex}
\Closesolutionfile{ans}
\indapan{3}{ans/CD3_17-23DS}

\Opensolutionfile{ans}[ans/CD3-14-25-KQ]
\TNSA

\begin{ex} %[2H5H2-3]%câu 33
	Trong KG $Oxyz$, phương trình tổng quát mặt phẳng $(P)\colon ax+by+cz+d=0$ đi qua điểm $M(3 ;-1 ; 4)$ đồng thời vuông góc với giá của vectơ $\overrightarrow{a}=(1 ;-1 ; 2)$. Tính $a+b+c$.
	\shortans{$2$}
	\loigiai{
		Mặt phẳng $(P)$ đi qua điểm $M(3 ;-1 ; 4)$ đồng thời vuông góc với giá của $\overrightarrow{a}=(1 ;-1 ; 2)$ nên nhận $\overrightarrow{a}=(1 ;-1 ; 2)$ làm vectơ pháp tuyến. \\Do đó, $(P)$ có phương trình là
		$$1(x-3)-1(y+1)+2(z-4)=0 \Leftrightarrow x-y+2 z-12=0.$$
		Suy ra $a+b+c=2$.
	}
\end{ex}
\begin{ex} %[2H5H2-3]%Câu 34.
	Trong KG $Oxyz$, phương trình mặt phẳng $(P)\colon ax+by+cz+d=0$ qua $M(0 ;-2 ; 1)$ và có cặp vectơ chỉ phương $\overrightarrow{a}=(-2 ;-3 ; 8), \overrightarrow{b}=(-1 ; 0 ; 6)$. Tính $a+b+c$.
	\shortans{$17$} 
	\loigiai{
		Ta có $\overrightarrow{n}=\left[\overrightarrow{a}, \overrightarrow{b}\right]=(-18 ; 4 ;-3)$. \\
		Mặt phẳng $(P)$ đi qua $M(0 ;-2 ; 1)$ và có vectơ pháp tuyến $\overrightarrow{n}=(-18 ; 4 ;-3)$ nên có phương trình $-18(x-0)+4(y+2)-3(z-1)=0 \Leftrightarrow 18 x-4 y+3 z-11=0$.\\
		Vậy mặt phẳng cần tìm có phương trình: $18x-4y+3z-11=0$.\\
		Suy ra $a+b+c=17$.
	}
\end{ex}
\begin{ex} %[2H5H2-3] %Câu 35
	Trong KG $Oxyz$, cho $A(1 ; 1 ; 0), B(0 ; 2 ; 1), C(1 ; 0 ; 2), D(1 ; 1 ; 1)$. Mặt phẳng $(\alpha)\colon ax+by+cz+d=0$ đi qua $A(1 ; 1 ; 0), B(0 ; 2 ; 1),(\alpha)$ song song với đường thẳng $CD$. Tính $a+b+c$.
	\shortans{$4$}
	\loigiai{
		$\overrightarrow{AB}=(-1 ; 1 ; 1), \overrightarrow{CD}=(0 ; 1 ;-1) \Rightarrow \left[ \overrightarrow{ B}, \overrightarrow{D}\right] =(-2 ;-1 ;-1)$.\\
		$(\alpha)$ đi qua $A(1 ; 1 ; 0)$ và có một VTPT là $\overrightarrow{n}=(2 ; 1 ; 1) \Rightarrow(\alpha)\colon 2 x+y+z-3=0$.\\
		Suy ra $a+b+c=4$.
	}
\end{ex}
\begin{ex} %[2H5H2-3]%Câu 36 
	Trong KG $Oxyz$, cho điểm $M(2 ; 1 ;-3)$ và mặt phẳng $(P)\colon 3 x-2 y+z-3=0$. Phương trình của mặt phẳng đi qua $M$ và song song với $(P)$ có dạng $(Q)\colon ax+by+cz+d=0$. Tính $a+b+c$.
	\shortans{$2$}
	\loigiai{
		Mặt phẳng $(Q)$ cần tìm song song với mặt phẳng $(P)\colon 3 x-2 y+z-3=0$ nên có phương trình dạng
		$$(Q)\colon 3x-2y+z+m=0, m \neq -3.$$
		Vì $M$ $\in(Q)$ nên $(Q)\colon 3\cdot2-2\cdot1+(-3)+m=0 \Leftrightarrow m=-1$.\\
		Vậy $(Q)\colon 3x-2y+z-1=0$.\\
		Suy ra $a+b+c=2$.
	}
\end{ex}
\begin{ex} %[2H5H2-3]%Câu 37.
	Trong KG $Oxyz$, cho ba điểm $A(3 ;-2 ;-2), B(3 ; 2 ; 0), C(0 ; 2 ; 1)$. Phương trình mặt phẳng $(ABC)$ có dạng $=ax+by+cz+d=0$. Tính $a+b+c$.
	\shortans{$5$}
	\loigiai{
		Ta có $\overrightarrow{AB}=(0 ; 4 ; 2), \overrightarrow{AC}=(-3 ; 4 ; 3), \overrightarrow{n}=\left[ \overrightarrow{B} ; \overrightarrow{C}\right]=(4 ;-6 ; 12)$.\\
		Ta có $\overrightarrow{n}=(4 ;-6 ; 12)$ cùng phương $\overrightarrow{n}_{1}=(2 ;-3 ; 6)$.\\
		Mặt phẳng $(ABC)$ đi qua điểm $C(0 ; 2 ; 1)$ và có một vectơ pháp tuyến $\overrightarrow{n}_{1}=(2 ;-3 ; 6)$ nên $(ABC)$ có phương trình là
		$$2(x-0)-3(y-2)+6(z-1)=0 \Leftrightarrow 2 x-3 y+6 z=0.$$
		Vậy phương trình mặt phẳng cần tìm là $2x-3y+6z=0$.\\
		Suy ra $a+b+c=5$. 
	}
\end{ex}
\begin{ex}  %[2H5H2-4]%Câu 38
	Trong không gian, cho hai điểm $A(0 ; 0 ; 1)$ và $B(2 ; 1 ; 3)$. Phương trình mặt phẳng đi qua $A$ và vuông góc với $ABC\colon ax+by+cz+d=0$. Tính $a+b+c$.
	\shortans{$5$}
	\loigiai{
		Mặt phẳng đi qua $A(0 ; 0 ; 1)$ và nhận vectơ $\overrightarrow{AB}=(2 ; 1 ; 2)$ làm vectơ pháp tuyến nên có phương trình là
		$$2(x-0)+(y-0)+2(z-1)=0 \Leftrightarrow 2x+y+2z-2=0.$$
		Suy ra $a+b+c=5$.}
\end{ex}
\begin{ex} %[2H5H2-4]%Câu 39
	Trong KG $Oxyz$, cho hai điểm $A(2 ; 4 ; 1), B(-1 ; 1 ; 3)$ và mặt phẳng $(P)\colon x-3y+2z-5=0$. Lập phương trình mặt phẳng $(Q)$ đi qua hai điểm $A, B$ và vuông góc với mặt phẳng $(P)\colon ax+by+cz+d=0$. Tính $a+b+c$.
	\shortans{$5$}
	\loigiai{
		Ta có: $\overrightarrow{AB}=(-3 ;-3 ; 2)$, vectơ pháp tuyến của $(P)$ là $\overrightarrow{n}_{P}=(1 ;-3 ; 2)$.\\
		Từ giả thiết suy ra $\overrightarrow{n}=\left[\overrightarrow{AB}, \overrightarrow{n}_{P}\right]=(0 ; 8 ; 12)$ là vectơ pháp tuyến của $(Q)$. \\
		$(Q)$ đi qua điểm $A(2 ; 4 ; 1)$ suy ra phương trình tổng quát của $(Q)$ là
		$$0(x-2)+8(y-4)+12(z-1)=0 \Leftrightarrow 2y+3z-11=0.$$
		Suy ra $a+b+c=5$.
	}
\end{ex}
\begin{ex}%[2H5H2-3]%Câu 40
	Trong KG $Oxyz$, gọi $M, N, P$ lần lượt là hình chiếu vuông góc của $A(2 ;-3 ; 1)$ lên các mặt phẳng tọa độ. Tính $a+b+c$ của phương trình mặt phẳng $(MNP)\colon ax+by+cz+d=0$. 
	\shortans{$7$}
	\loigiai{
		Không mất tính tổng quát, ta giả sử $M, N, P$ lần lượt là hình chiếu vuông góc của $A(2 ;-3 ; 1)$ lên các mặt phẳng tọa độ $(Oxy),(Oxz),(Oyz)$. \\
		Khi đó $M(2 ;-3 ; 0), N(2 ; 0 ; 1)$ và $P(0 ;-3 ; 1).$\\
		$\overrightarrow{MN}=(0 ; 3 ; 1)$ và $\overrightarrow{MP}=(-2 ; 0 ; 1)$. \\
		Ta có $\overrightarrow{MN}$ và $\overrightarrow{MP}$ là cặp vectơ không cùng phương và có giá nằm trong $(MNP)$.\\
		Do đó $(MNP)$ có một vectơ pháp tuyến là $\overrightarrow{n}=\left[\overrightarrow{M N}, \overrightarrow{MP}\right]=(3 ;-2 ; 6)$.\\
		Mặt khác $(MNP)$ đi qua $M(2 ;-3 ; 0)$ nên có phương trình là
		$$3(x-2)-2(y+3)+6(z-0)=0 \Leftrightarrow 3x-2y+6z-12=0.$$
		Suy ra $a+b+c=7$.
	}
\end{ex}
\Closesolutionfile{ans}
\indapan{8}{ans/CD3-14-25-KQ}
\begin{dang}{Viết PTTQ MP khi biết VTPT, VTCP nhưng không biết điểm đi qua}
	\begin{itemize}
		\item Viết phương trình mặt phẳng $(\alpha)$ dưới dạng
		$$
		Ax+By+Cz+D=0
		.$$
		\item Sau đó dựa vào giả thiết bài toán để tìm giá trị $D$.\\
		Chú ý: Dạng này giả thiết có liên quan đến khoảng cách và góc liên quan đến mặt phẳng.
	\end{itemize}
\end{dang}

\Opensolutionfile{ans}[ans/CD3-B46-B49-KQ]
\TN
\begin{ex}%[2H5V2-5]%Câu 41
	Trong KG $Oxyz$, cho mặt phẳng $(P)\colon 2 x+2y-z-1=0$ Mặt phẳng nào sau đây song song với $(P)$ và cách $(P)$ một khoảng bằng $3$?
	\choice
	{$(Q)\colon 2x+2y-z+10=0$}
	{$(Q)\colon 2x+2y-z+4=0$}
	{\True $(Q)\colon 2x+2y-z+8=0$}
	{$(Q)\colon 2x+2y-z-8=0$}
	\loigiai{
		Mặt phẳng $(P)$ đi qua điểm $M(0 ; 0 ;-1)$ và có một vectơ pháp tuyến $\overrightarrow{n}=(2 ; 2 ;-1)$.\\
		Mặt phẳng $(Q)$ song song với $(P)$ và cách $(P)$ một khoảng bằng $3$ nên có dạng
		$$(Q)\colon 2x+2y-z+d=0,\quad(d \neq -1).$$
		Mặt khác ta có $\mathrm{d}(M,(Q))=3$ 
		\begin{align*}
			\Leftrightarrow & \dfrac{|1+d|}{\sqrt{4+4+1}}=3\\
			\Leftrightarrow &|d+1|=9\\
			\Leftrightarrow &\hoac{d&=8\\d&=-10} \text{(thỏa mãn)}.
		\end{align*}
		Do đó $(Q)\colon 2x+2y-z+8=0$ hoặc $(Q)\colon 2x+2y-z-10=0$. 
	}
\end{ex}
\begin{ex} %[2H5V2-5]%Câu 42
	Trong KG $Oxyz$, cho ba điểm $A(2 ; 0 ; 0), B(0 ; 3 ; 0), C(0 ; 0 ;-1)$. Phương trình của mặt phẳng $(P)$ qua $D(1 ; 1 ; 1)$ và song song với mặt phẳng $(ABC)$ là
	\choice
	{$2x+3y-6z+1=0$}
	{\True $3x+2y-6z+1=0$}
	{$3x+2y-5z=0$}
	{$6x+2y-3z-5=0$}
	\loigiai{
		Phương trình đoạn chắn của mặt phẳng $(ABC)$ là $\dfrac{x}{2}+\dfrac{y}{3}+\dfrac{z}{-1}=1$.\\
		Mặt phẳng $(P)$ song song với mặt phẳng $(ABC)$ nên\\
		$(P)\colon \dfrac{1}{2} x+\dfrac{1}{3} y-z+m=0\quad(m \neq-1)$.\\
		Do $D(1 ; 1 ; 1) \in(P)$ có $\dfrac{1}{2}\cdot 1+\dfrac{1}{3} \cdot 1-1+m=0 \Leftrightarrow m-\dfrac{1}{6}=0 \Leftrightarrow m=\dfrac{1}{6}$.\\
		Vậy $(P)\colon \dfrac{1}{2}x+\dfrac{1}{3}y-z+\dfrac{1}{6}=0 \Leftrightarrow (P)\colon 3x+2y-6z+1=0$.
	}
\end{ex}
\begin{ex} %[2H5V2-5]%Câu 43.
	Trong KG $Oxyz$ cho $A(2 ; 0 ; 0), B(0 ; 4 ; 0), C(0 ; 0 ; 6), D(2 ; 4 ; 6)$. Gọi $(P)$ là mặt phẳng song song với mặt phẳng $(A B C),(P)$ cách đều $D$ và mặt phẳng $(ABC)$. Phương trình của $(P)$ là
	\choice
	{\True $6x+3y+2z-24=0$}
	{$6x+3y+2z-12=0$}
	{$6x+3y+2z=0$}
	{$6x+3y+2z-36=0$}
	\loigiai{
		$(ABC)\colon \dfrac{x}{2}+\dfrac{y}{4}+\dfrac{z}{6}=1 \Leftrightarrow 6x+3y+2z-12=0$.\\
		$(P)\parallel(ABC) \Rightarrow(P)\colon 6x+3y+2z+m=0\quad(m\neq-12)$.\\
		$(P)$ cách đều $D$ và mặt phẳng $(ABC) \Rightarrow \mathrm{d}(D,(P))=\mathrm{d}(A,(P))$.
		\begin{align*}
			\Leftrightarrow& \dfrac{|6\cdot 2+3\cdot 4+2\cdot 6+m|}{\sqrt{6^{2}+3^{2}+2^{2}}}=\dfrac{|6\cdot 2+3\cdot 0+2\cdot 0+m|}{\sqrt{6^{2}+3^{2}+2^{2}}}\\
			\Leftrightarrow&|36+m|=|12+m|\\ \Leftrightarrow& \hoac{36+m=12+m \\ 36+m=-12-m}\\
			\Leftrightarrow& m=-24 \text{(cách)  (nhận).}
		\end{align*}
		Vậy phương trình của $(P)$ là $6x+3y+2z-24=0$.
	}
\end{ex}
\begin{ex} %[2H5V2-5]%Cau 44 
	Trong không gian với hệ trục tọa độ $Oxyz$, cho mặt phẳng $(Q)\colon x+2y+2z-3=0$, mặt phẳng $(P)$ không qua $O$, song song với mặt phẳng $(Q)$ và $\mathrm{d}((P),(Q))=1$. Phương trình mặt phẳng $(P)$ là
	\choice
	{$x+2y+2z+1=0$}
	{$ x+2y+2z=0$}
	{\True $ x+2y+2z-6=0$}
	{$ x+2y+2z+3=0$}
	\loigiai{
		Vì mặt phẳng $(P)$ song song với mặt phẳng $(Q)$.\\
		$\Rightarrow$ vtpt $\overrightarrow{n}_{P}=$ vtpt $\overrightarrow{n}_{Q}=(1 ; 2 ; 2)$.\\
		Phương trình mặt phẳng $(P)$ có dạng $x+2y+2z+d=0\quad(d \ne 0).$\\
		Gọi $A(3 ; 0 ; 0) \in (Q)$\\
		$\Rightarrow \mathrm{d}((P),(Q))=\mathrm{d}(A,(P))=1$\\
		$\Leftrightarrow \dfrac{|3+D|}{3}=1 \Leftrightarrow\hoac{3+d&=3 \\ 3+d&=-3} \Leftrightarrow\hoac{d&=0 &(\text{loại})&O\\ d&=-6 &(\text{nhận})&}.$
	}
\end{ex}
\begin{ex} %[2H5V2-5]%Cau 45
	Trong KG $Oxyz$, cho mặt phẳng $(P)\colon 2 x-2 y+z-5=0$.  Viết phương trình mặt phẳng $(Q)$ song song với mặt phẳng $(P)$, cách $(P)$ một khoảng bằng $3$ và cắt trục $Ox$ tại điểm có hoành độ dương. 
	\choice
	{$(Q)\colon 2x-2y+z+4=0$}
	{\True $(Q)\colon 2x-2y+z-14=0$}
	{$(Q)\colon 2x-2y+z-19=0$}
	{$(Q)\colon 2x-2y+z-8=0$}
	\loigiai{
		Ta có, $(Q)$ song song $(P)$ nên phương trình mặt phẳng $(Q)\colon 2x-2y+z+d=0$; $d\ne -5$.\\
		Chọn $M(0 ; 0 ; 5)\in(P)$.\\
		Ta có $\mathrm{d}((P),(Q))=\mathrm{d}(M),(Q))=\dfrac{|5+d|}{\sqrt{2^{2}+(-2)^{2}+1^{2}}}=3 \Leftrightarrow\hoac{d&=4 \\ d&=-14.}\\
		\\d=4 \Rightarrow(Q)\colon 2x-2y+z+4=0$ khi đó $(Q)$ cắt $Ox$ tại điểm $M_{1}(-2 ; 0 ; 0)$ có hoành độ âm nên trường hợp này $(Q)$ không thỏa đề bài.\\
		$d=-14 \Rightarrow(Q)\colon 2x-2y+z-14=0$ khi đó $(Q)$ cắt $Ox$ tại điểm $M_{2}(7 ; 0 ; 0)$ có hoành độ dương do đó $(Q)\colon 2x-2y+z-14=0$ thỏa đề bài.\\
		Vậy phương trình mặt phẳng $(Q)\colon 2x-2y+z-14=0$.
	}
\end{ex}
\Closesolutionfile{ans}
\indapan{10}{ans/CD3-B46-B49-KQ}

\TNSA
\Opensolutionfile{ans}[ans/CD3-14-25-KQ2]
\begin{ex} %[2H5V2-5]%Câu 46
	Trong không gian hệ toạ độ $Oxyz$, lập phương trình các mặt phẳng song song với mặt phẳng $(\beta)\colon x+y-z+3=0$ và cách $(\beta)$ một khoảng bằng $\sqrt{3}$ có dạng $ax+by+cz+d=0\quad (d\neq 0)$. Tính $a+b+c$.
	\shortans{$1$}
	\loigiai{
		Gọi mặt phẳng $(\alpha)$ cần tìm.\\
		Vì $(\alpha)\parallel(\beta)$ nên phương trình $(\alpha)$ có dạng: $x+y-z+c=0$ với $c$ khác $\backslash\{3\}$.\\
		Lấy điểm $I(-1 ;-1 ; 1) \in(\beta)$.\\
		Vì khoảng cách từ $(\alpha)$ đến $(\beta)$ bằng $\sqrt{3}$ nên ta có
		$$\mathrm{d}(I,(\alpha))=\sqrt{3} \Leftrightarrow \dfrac{|-1-1-1+c|}{\sqrt{3}}=\sqrt{3} \Leftrightarrow \dfrac{|c-3|}{\sqrt{3}}=\sqrt{3} \Leftrightarrow\hoac{c=0 \\ c=6}. (\text{thỏa điều kiện } c \in \mathbb{R} \backslash\{3\} ).$$
		Vậy phương trình $(\alpha)\colon x+y-z+6=0$ hoặc $(\alpha)\colon x+y-z=0$.\\
		Suy ra $a+b+c=1$.
	}
\end{ex}
\begin{ex} %[2H5V2-5]%Câu 47.
	Trong không gian với hệ trục tọa độ $Oxyz$, cho hai mặt phẳng $\left(Q_{1}\right)\colon 3x-y+4z+2=0$ và $\left(Q_{2}\right)\colon 3x-y+4z+8=0$. Viết phương trình mặt phẳng $(P)\colon ax+by+cz=0$ song song và cách đều hai mặt phẳng $\left(Q_{1}\right)$ và $\left(Q_{2}\right)$. Tính $a+b+c$.
	\shortans{$6$}
	\loigiai{
		Mặt phẳng $(P)$ có dạng $3x-y+4z+d=0$. \\
		Lấy $M(0 ; 2 ; 0) \in\left(Q_{1}\right)$ và $N(0 ; 8 ; 0) \in\left(Q_{2}\right)$. Do $\left(Q_{1}\right)\parallel\left(Q_{2}\right)$ trung điểm $I(0 ; 5 ; 0)$ của $MN$ phải thuộc vào $(P)$ nên ta tìm được $D=5$. Vậy $(P)\colon 3x-y+4z+5=0$.\\
		Suy ra $a+b+c=6$.
	}
\end{ex}
\begin{ex}  %[2H5V2-5]%Câu 48
	Trong KG $Oxyz$, gọi $(\gamma)$ là mặt phẳng cách đều hai mặt phẳng sau đây:
	$4x-y-2z-3=0$, $4x-y-2z-5=0$. lập mặt phẳng $(\gamma)$ có dạng $ax+by+cz=0$. Tính $a+b+c+d$.
	\shortans{$-3$}
	\loigiai{
		Gọi điểm $A(0;-3; 0) \in (\alpha)\colon4x-y-2z-3=0$ và $B(0 ;-5 ; 0) \in (\beta)\colon4x-y-2z-5=0$.\\
		Mặt phẳng cách đều hai mặt phẳng trên có dạng: $(\gamma)\colon 4x-y-2z+m=0$.\\
		Để mặt phẳng $(\gamma)$ cách đều hai mặt phẳng trên thì
		$$\mathrm{d}(A\colon(\beta))=2 \mathrm{d}(A\colon(\gamma))
		\Leftrightarrow|m+3|=1 \Leftrightarrow\hoac{m=-2 \\ m=-4.}$$ 
		Mặt khác điểm hai điểm $A, B$ phải nằm về hai phía của mặt phẳng $(\gamma)$.\\
		Do đó:
		\begin{itemize}
			\item Với $m=-2$ ta có: $(4\cdot0+3-2\cdot0-2)(4\cdot0+5-2\cdot0-2)>0$ nên $A, B$ cùng phía.
			\item Với $ m=-4$ ta có: $(4\cdot0+3-2\cdot0-4)(4\cdot 0+5-2\cdot 0-4)<0$ nên $A, B$ khác phía.
		\end{itemize}
		Vậy phương trình mặt phẳng cần tìm là $(\gamma)\colon 4x-y-2z-4=0$.\\
		Suy ra $a+b+c+d=-3$.
	}
\end{ex}
\begin{ex} %[2H5V2-5]%Câu 49.
	Trong KG $Oxyz$ cho các điểm $A(2 ; 0 ; 0), B(0 ; 4 ; 0), C(0 ; 0 ; 6), D(2 ; 4 ; 6)$. Gọi $(P)$ là mặt phẳng song song với mặt phẳng $(A B C),(P)$ cách đều $D$ và mặt phẳng $(A B C)$. Viêt phương trình của mặt phẳng $(P)\colon ax+by+cz+d=0$. Tính $a+b+c$.
	\shortans{$11$}
	\loigiai{
		Phương trình mặt phẳng $(ABC)$ là $\dfrac{x}{2}+\dfrac{y}{4}+\dfrac{z}{6}=1 \Leftrightarrow 6x+3y+2z-12=0$
		\begin{itemize}
			\item $(P)$ song song với mặt phẳng $(ABC)$ nên $(P)$ có dạng $$6x+3y+2z+d=0\quad(d \ne q-12).$$
			\item Khoảng cách từ $D$ đến mặt phẳng $(P)$ là
			\begin{align*}
				&\mathrm{d}(D),(P))=\mathrm{d}((ABC),(P))\\
				&\Leftrightarrow \mathrm{d}(D),(P))=\mathrm{d}(A,(P))\\
				&\Leftrightarrow|36+d|=|12+d|\\
				&\Leftrightarrow d=-24.
			\end{align*}
		\end{itemize}
		Vậy $(P)\colon 6x+3y+2z-24=0$.\\
		Suy ra $a+b+c=11$.
	}
\end{ex}
\Closesolutionfile{ans}
\indapan{4}{ans/CD3-14-25-KQ2}

\begin{dang}{Viết PTTQ khi biết điểm đi qua nhưng không biết vectơ}
\end{dang}
\begin{tomtat}
	Khi bài toán cho biết mặt phẳng $(\alpha)$ đi qua điềm $M_0\left(x_0 ; y_0 ; z_0\right)$ và giả thiết bài toán không cho vectơ pháp tuyến $\overrightarrow{n}$ hoặc không cho hai vectơ chỉ phương $\overrightarrow{a}, \overrightarrow{b}$ thì ta thực hiện các bước sau:
	\begin{itemize}
		\item Gọi vectơ pháp tuyến của mặt phẳng $(\alpha)$ là $\overrightarrow{n}=(A ; B ; C)$ với $A^2+B^2+C^2 \neq 0$.
		\item Viết phương trình mặt phẳng $(\alpha)$ dưới dạng:
		$$
		(\alpha)\colon A\left(x-x_0\right)+B\left(y-y_0\right)+C\left(z-z_0\right)=0.
		$$
		\item Sau đó dựa vào giả thiết bài toán để tìm \textbf{hai} phương trình chứa 3 ẩn $A, B, C$.
	\end{itemize}
	Chú ý:
	\begin{itemize}
		\item Dạng này, giả thiết có liên quan đến khoảng cách và góc liên quan đến mặt phẳng.
		\item Để giải tìm vectơ pháp tuyến của mặt phẳng đơn giàn hơn thì gọi vectơ pháp tuyến của mặt phẳng là $\overrightarrow{n}=(1 ; B ; C)$.
	\end{itemize}
\end{tomtat}

\Opensolutionfile{ans}[ans/CD3-50-50]
\TN

\begin{ex} %[2H5V2-5]%Câu 50
	Trong KG $Oxyz$, cho $3$ điểm $A(1 ; 0 ; 0), B(0 ;-2 ; 3), C(1 ; 1 ; 1)$. Gọi $(P)$ là mặt phẳng chứa $A, B$ sao cho khoảng cách từ $C$ tới mặt phẳng $(P)$ bằng $\dfrac{2}{\sqrt{3}}$.  Phương trình mặt phẳng $(P)$ là
	\choice
	{$\hoac{&2x+3y+z-1=0 \\& 3x+y+7z+6=0}$}
	{$\hoac{&x+2y+z-1=0 \\ &-2x+3y+6z+13=0}$}
	{$\hoac{&x+y+2z-1=0 \\& -2x+3y+7z+23=0}$}
	{\True $\hoac{&x+y+z-1=0 \\& -23x+37y+17z+23=0}$} 
	\loigiai{
		Gọi $(P)\colon \heva{&\text{ qua } A(1 ; 0 ; 0)\\ &\text{ VTPT } \overrightarrow{n}=(A ; B ; C) \neq \overrightarrow{0}}$\\
		$(P)\colon A \cdot(x-1)+By+Cz=0$.\\
		$B\in(P)\colon -A-2B+3=0 \Leftrightarrow A=-2B+3C$.\\
		$ \mathrm{d}(C\colon(P))=\dfrac{2}{\sqrt{3}} \Leftrightarrow \dfrac{|B+C|}{\sqrt{A^{2}+B^{2}+C^{2}}}=\dfrac{2}{\sqrt{3}}$\\
		$\Leftrightarrow 3\left(B^{2}+C^{2}+2BC\right)=4\left(A^{2}+B^{2}+C^{2}\right)$\\
		$\Leftrightarrow B^{2}+C^{2}-6BC+4A^{2}=0$.\\
		Thay $A=-2B+3C$ vào $B^{2}+C^{2}-6BC+4A^{2}=0$\\ 
		Ta có: $B^{2}+C^{2}-6BC+4(-2B+3C)^{2}=0 \Leftrightarrow 17B^{2}-54BC+37C^{2}=0$\\
		Cho $C=1$ từ đó suy ra $17 B^{2}-54 B+37=0 \Leftrightarrow\hoac{&B=1& &\Rightarrow& &A=1&\\ &B=\dfrac{37}{17}& &\Rightarrow& &A=\dfrac{-23}{17}.&}$\\
		Suy ra $\hoac{&(P)\colon x+y+x-1=0\\&(P)\colon-23x+37y+17z+23=0.}$
	}
\end{ex}
\begin{ex}%[2H5H1-3]
	Trong hệ trục tọa độ $O x y z$ cho $3$ điểm $M(4 ; 2 ; 1)$, $N(0 ; 0 ; 3)$, $Q(2 ; 0 ; 1)$. Viết phương trình mặt phẳng chứa $O Q$ và cách đều $2$ điểm $M$, $N$.
	\choice
	{$x-2 y-2 z=0$ hoặc $x+4 y-2 z=0$}
	{$x+2 y+2 z=0$ hoặc $x-4 y-2 z=0$}
	{$x+2 y-2 z=0$ hoặc $x+4 y-2 z=0$}
	{\True $x+2 y-2 z=0$ hoặc $x-4 y-2 z=0$}
	\loigiai{
		Gọi $(\alpha)\colon A x+B y+C z+D=0$ $\left(A^2+B^2+C^2 \neq 0\right)$.\\
		$O \in(\alpha)$ nên ta có $D=0$, $Q \in(\alpha)$ nên ta có $2 A+C=0 \Rightarrow C=-2 A$.\\
		Theo đề bài
		$$\mathrm{d}(M,(\alpha))=\mathrm{d}(N,(\alpha))	\Leftrightarrow|2 A+2 B|=|-6 A| \Leftrightarrow \hoac{&2 A + 2B = 6 A\\&2 A + 2B = - 6 A}\Leftrightarrow \hoac{&B=2 A& (*)\\&B=-4 A	 & (* *).}$$
		Từ $(*)$ chọn $A=1 \Rightarrow B=2$, $C=-2 \Rightarrow(\alpha)\colon x+2 y-2 z=0$.\\
		Từ $(**)$ chọn $A=1 \Rightarrow B=-4$, $C=-2 \Rightarrow(\alpha)\colon x-4 y-2 z=0$.
	}
\end{ex}
\Closesolutionfile{ans}
\indapan{10}{ans/CD3-50-50}
\TNSA
\begin{ex}%[2H5H1-3]
	Trong không gian với hệ toạ độ $O x y z$, biết mặt phẳng $(P)\colon Ax+By+Cz+D=0$ ($A$, $B$, $C \in \mathbb{Z}$, $A$ và $C$ trái dấu) qua $O$, vuông góc với mặt phẳng $(Q)\colon x+y+z=0$ và cách điểm $M(1 ; 2 ;-1)$ một khoảng bằng $\sqrt{2}$. Tính giá trị của $A+B+C$.
	\shortans{$0$}
	\loigiai{
		$(P)$ qua $O$ nên phương trình có dạng $A x+B y+C z=0$ (với $A^2+B^2+C^2 \neq 0$ ).\\
		Vì $({P}) \perp({Q})$ nên $1 \cdot A+1 \cdot B+1 \cdot C=0 \Leftrightarrow C=-A-B \quad (1)$.\\
		Do $\mathrm{d}(M,(P))=\sqrt{2} \Leftrightarrow \dfrac{|A+2 B-C|}{\sqrt{A^2+B^2+C^2}}=\sqrt{2} \Leftrightarrow(A+2 B-C)^2=2\left(A^2+B^2+C^2\right) \quad (2)$.\\
		Từ $(1)$ và $(2)$ ta được $8 A B+5 B^2=0 \Leftrightarrow\hoac{&B=0 &(3)\\ &8 A+5 B=0&(4).}$\\
		Từ $(3)$, ta có ${B}=0 \Rightarrow {C}=-{A}$ (nhận do $A$ và $C$ trái dấu). \\
		Chọn ${A}=1$, ${C}=-1 \Rightarrow({P})\colon  x-z=0$.\\
		Khi đó $A+B+C=0$.\\
		Từ $(4)$, ta có $8 {A}+5 {B}=0$. \\
		Chọn ${A}=5$, ${B}=-8 \Rightarrow {C}=3 \Rightarrow({P})\colon 5 x-8 y+3 z=0$. (loại  do $A$ và $C$ cùng dấu).
	}
\end{ex}

\begin{ex}%[2H5H1-3]
	Trong không gian với hệ toạ độ $Oxyz$, cho các điểm $M(-1 ; 1 ; 0)$, $N(0 ; 0 ;-2)$, $I(1 ; 1 ; 1)$. Biết mặt phẳng $({P})$ qua ${A}$ và ${B}$, đồng thời khoảng cách từ ${I}$ đến $({P})$ bằng $\sqrt{3}$. Giả sử phương trình mặt phẳng $(P)$ có dạng $ax+by+z+d=0$ với $b>0$. Tính $\dfrac{a}{b}$ viết dưới dạng số thập phân.
	\shortans{$1{,}4$}
	\loigiai{
		Phương trình mặt phẳng $({P})$ có dạng $a x+b y+ z+d=0$ $\left(a^2+b^2+1 \neq 0\right)$.\\
		Ta có $\heva{ &M \in(P) \\ &N \in(P) \\ &\mathrm{d}(I,(P))=\sqrt{3}} \Leftrightarrow \hoac{&a=-b,\ 2 =a-b,\ d=a-b &(1)\\ &5 a=7 b,\ 2 =a-b,\ d=a-b &(2).}$
		\begin{itemize}
			\item Với $(1) \Rightarrow $ Phương trình mặt phẳng $(P)\colon x-y+z+2=0$ (loại do $b<0$).
			\item Với $(2) \Rightarrow $ Phương trình mặt phẳng $(P)\colon 7 x+5 y+z+2=0$ (nhận do $b=5>0$).\\
			Khi đó $\dfrac{a}{b}=\dfrac{7}{5}=1{,}4$.
		\end{itemize}
	}
\end{ex}

\begin{ex}%[2H5H1-3]
	Trong không gian với hệ toạ độ $O x y z$, cho tứ diện $ABCD$ với $A(1 ;-1 ; 2)$, $B(1 ; 3 ; 0)$, $C(-3 ; 4 ; 1)$, $D(1 ; 2 ; 1)$. Mặt phẳng $({P})$ đi qua ${A}$, ${B}$ sao cho khoảng cách từ ${C}$ đến $({P})$ bằng khoảng cách từ ${D}$ đến $({P})$. Biết có hai mặt phẳng $(P)$ thỏa yêu cầu đề bài là $x+b_1y+c_1z+d_1=0$ và $x+b_2y+c_2z+d_2=0$. Tính $S=b_1+c_1+b_2+c_2$.
	\shortans{$9$}
	\loigiai{
		Phương trình mặt phẳng $({P})$ có dạng $a x+b y+c z+d=0$ với $\left(a^2+b^2+c^2 \neq 0\right)$.\\
		Ta có $\heva{&A \in(P) \\ &B \in(P) \\ &\mathrm{d}(C,(P))=\mathrm{d}(D,(P))} \Leftrightarrow\heva{&a-b+2 1+d=0 \\&a+3 b+d=0 \\&\dfrac{|-3 a+4 b+1+d|}{\sqrt{a^2+b^2+1^2}}=\dfrac{|a+2 b+1+d|}{\sqrt{a^2+b^2+1^2}}}$\\
		$ \Leftrightarrow\hoac{&b=2 a,\ c=4 a,\ d=-7 a \\& c=2 a,\ b=a,\ d=-4 a}$
		\begin{itemize}
			\item Với $b=2 a$, $c=4 a$, $d=-7 a$ và ta đã có $a=1$ nên $({P}) \colon x+2 y+4 z-7=0$.\\
			Khi đó $b_1=2$, $c_1=4$.
			\item Với $c=2 a$, $b=a$, $d=-4 a$ và ta đã có $a=1$ nên $({P})\colon x+y+2 z-4=0$.\\
			Khi đó $b_2=1$, $c_2=2$.
		\end{itemize}
		Vậy $S=2+4+1+2=9$.
	}
\end{ex}

\begin{ex}%[2H5H1-3]
	Trong không gian với hệ trục tọa độ $O x y z$, cho các điểm $A(1 ; 2 ; 3)$, $B(0 ;-1 ; 2)$, $C(1 ; 1 ; 1)$. Mặt phẳng $(P)$ đi qua $A$ và gốc tọa độ $O$ sao cho khoảng cách từ $B$ đến $(P)$ bằng khoảng cách từ $C$ đến $(P)$. Biết phương trình mặt phẳng $(P)$ có dạng $ax+by-4z+d=0$. Hỏi $a$ có bao nhiêu ước nguyên?
	\shortans{$12$}
	\loigiai{
		Vì $O \in(P)$ nên $(P)\colon a x+by-4 z=0$, với $a^2+b^2+16 \neq 0$.\\
		Do $A \in(P) \Rightarrow a+2 b -12=0$ $(1)$\\
		Và $\mathrm{d}(B,(P))=\mathrm{d}(C,(P)) \Leftrightarrow|-b-8|=|a+b-4|$ $(2)$.\\
		Từ $(1)$ và $(2) \Rightarrow b=0$.	Khi đó ta được $a=-3 \cdot (-4) =12$.\\
		Các ước nguyên của $12$ là $\{\pm 1;\pm 2; \pm 3; \pm 4; \pm 6; \pm 12\}$ có $12$ ước nguyên.
	}
\end{ex}

\begin{ex}%[2H5H1-3]
	Trong không gian với hệ trục tọa độ $O x y z$, cho ba điểm $A(1 ; 1 ;-1)$, $B(1 ; 1 ; 2)$, $C(-1 ; 2 ;-2)$ và mặt phẳng $({P})\colon x-2 y+2 z+1=0$. Mặt phẳng $(\alpha)$ đi qua ${A}$, vuông góc với mặt phẳng $({P})$, cắt đường thẳng ${BC}$ tại ${I}$ sao cho $I B=2 I C$. Biết có hai mặt phẳng $(\alpha)$ thỏa yêu cầu đề bài có phương trình lần lượt là $4x+b_1y+c_1+d_1=0$ và $2x+b_2y+c_2+d_2=0$ với $b_1<b_2$. Hỏi có bao nhiêu giá trị nguyên thuộc tập $(b_1;b_2)$?
	\shortans{$4$}
	\loigiai{
		Phương trình mặt phẳng $(\alpha)$ có dạng $a x+b y+c z+d=0$, với $a^2+b^2+c^2 \neq 0$.\\
		Do $A(1 ; 1 ;-1) \in(\alpha)$ nên $a+b-c+d=0$. $\quad (1)$;\\
		$(\alpha) \perp(P)$ nên $a-2 b+2 c=0\quad (2)$.
		\begin{eqnarray*}
			&I B=2 I C &\Rightarrow \mathrm{d}(B,(\alpha))=2 \mathrm{d}(C ;(\alpha))\\  & &\Rightarrow \dfrac{|a+b+2 c+d|}{\sqrt{a^2+b^2+c^2}}=2 \dfrac{|-a+2 b-2 c+d|}{\sqrt{a^2+b^2+c^2}}\\
			& &\Leftrightarrow\hoac{&3 a-3 b+6 c-d=0 \\&-a+5 b-2 c+3 d=0.}
		\end{eqnarray*}
		Từ $(1)$, $(2)$, $(3)$ ta có 2 trường hợp sau
		\begin{itemize}
			\item $\heva{&a+b-c+d=0 \\&a-2 b+2 c=0 \\& 3 a-3 b+6 c-d=0} \Leftrightarrow \heva{&b=\dfrac{-1}{2} a \\& c=-a \\& d=\dfrac{-3}{2} a.}$
			\item $\heva{&a+b-c+d=0 \\ &a-2 b+2 c=0 \\ &-a+5 b-2 c+3 d=0} \Leftrightarrow \heva{&b=\dfrac{3}{2} a \\ &c=a \\ &d=\dfrac{-3}{2} a.} \quad (3)$
		\end{itemize}
		Do theo đề bài, ta có $a>0$ nên ta có thể có được $4x+b_1y+c_1+d_1=0$ là mặt phẳng ở trường hợp $1$ và $2x+b_2y+c_2+d_2=0$ là mặt phẳng ở trường hợp $2$.\\
		Khi đó
		\begin{itemize}
			\item Chọn $a=4 \Rightarrow b_1=-2$; $c_1=-4$; $d_1=-6 \Rightarrow(\alpha)\colon 4 x-2y-4 z-6=0$.
			\item Với $a=2 \Rightarrow b=3 $; $c=2 $; $d=-3 \Rightarrow(\alpha)\colon 2 x+3 y+2 z-3=0$.
		\end{itemize}
		Vậy ta có tập $(-2;3)$ có tất cả $4$ giá trị nguyên là $-1$, $0$, $1$, $2$.
	}
\end{ex}
\Closesolutionfile{ans}
\indapan{6}{ans/ans-2-C5B1CD2-D3}
\begin{dang}{Một số dạng khác}
	
\end{dang}
\Opensolutionfile{ans}[ans/ans-2-C5B1CD2-D4]
\TN

\begin{ex}%[2H5H1-3]
	Trong không gian $O x y z$ cho điểm $M(1 ; 2 ; 3)$. Viết phương trình mặt phẳng $(P)$ đi qua điểm $M$ và cắt các trục tọa độ $O x$, $O y$, $O z$ lần lượt tại $A$, $B$, $C$ sao cho $M$ là trọng tâm của tam giác $A B C$.
	\choice
	{$(P)\colon 6 x+3 y+2 z+18=0$}
	{$(P)\colon 6 x+3 y+2 z+6=0$}
	{\True $(P)\colon 6 x+3 y+2 z-18=0$}
	{$(P)\colon 6 x+3 y+2 z-6=0$}
	\loigiai{
		Theo giả thiết $A \in O x$, $B \in O y$, $C \in O z$ nên ta có thể đặt $A(a ; 0 ; 0)$, $B(0 ; b ; 0)$, $C(0 ; 0 ; c)$.\\
		Vì $M(1 ; 2 ; 3)$ là trọng tâm tam giác $A B C$ nên $\heva{&a=3 \\ &b=6 \\ &c=9.}$\\
		Từ đó ta có phương trình mặt phẳng theo đoạn chắn là
		$$	(P)\colon \dfrac{x}{3}+\dfrac{y}{6}+\dfrac{z}{9}=1 \Leftrightarrow 6 x+3 y+2 z-18=0.		$$
	}
\end{ex}

\begin{ex}%[2H5H1-3]
	Trong không gian với hệ trục tọa độ $O x y z$, cho điểm $G(1 ; 4 ; 3)$. Mặt phẳng nào sau đây cắt các trục $O x$, $O y$, $O z$ lần lượt tại $A$, $B$, $C$ sao cho $G$ là trọng tâm tứ diện $O A B C$?
	\choice
	{$\dfrac{x}{3}+\dfrac{y}{12}+\dfrac{z}{9}=1$}
	{\True $12 x+3 y+4 z-48=0$}
	{$\dfrac{x}{4}+\dfrac{y}{16}+\dfrac{z}{12}=0$}
	{$12 x+3 y+4 z=0$}
	\loigiai{
		Mặt phẳng $(P)$ cắt các trục $O x$, $O y$, $O z$ lần lượt tại $A$, $B$, $C$ nên $A(a ; 0 ; 0)$, $B(0 ; b ; 0)$, $C(0 ; 0 ; c)$.
		Vì $G$ là trọng tâm tứ diện $O A B C$ nên $$\heva{&x_G=\dfrac{x_A+x_B+x_C+x_O}{4}=\dfrac{a}{4} \\& y_G=\dfrac{y_A+y_B+y_C+y_O}{4}=\dfrac{b}{4} \\& z_G=\dfrac{z_A+z_B+z_C+z_O}{4}=\dfrac{c}{4}} \Rightarrow\heva{&a=4 \\ &b=16 \\ &c=12.}$$
		Khi đó mặt phẳng $(P)$ có phương trình là $\dfrac{x}{4}+\dfrac{y}{16}+\dfrac{z}{12}=1$ hay $12 x+3 y+4 z-48=0$.\\
		Vậy mặt phẳng $(P)$ thỏa mãn là $12 x+3 y+4 z-48=0$.
	}
\end{ex}

\begin{ex}%[2H5V1-3]
	Viết phương trình mặt phẳng $(\alpha)$ đi qua $M(2 ; 1 ;-3)$, biết $(\alpha)$ cắt trục $O x$, $O y$, $O z$ lần lượt tại $A$, $B$, $C$ sao cho tam giác $A B C$ nhận $M$ làm trực tâm.
	\choice
	{$2 x+5 y+z-6=0$}
	{$2 x+y-6 z-23=0$}
	{\True $2 x+y-3 z-14=0$}
	{ $3 x+4 y+3 z-1=0$}
	\loigiai{
		Giả sử $A(a ; 0 ; 0)$, $B(0 ; b ; 0)$, $C(0 ; 0 ; c)$, $a b c \neq 0$.\\
		Khi đó mặt phẳng $(\alpha)$ có dạng $\dfrac{x}{a}+\dfrac{y}{b}+\dfrac{z}{c}=1$.\\
		Do $M \in(\alpha) \Rightarrow \dfrac{2}{a}+\dfrac{1}{b}-\dfrac{3}{c}=1$.\\
		Ta có $\overrightarrow{A M}=(2-a ; 1 ;-3)$, $\overrightarrow{B M}=(2 ; 1-b ;-3)$, $\overrightarrow{B C}=(0 ;-b ; c)$, $\overrightarrow{A C}=(-a ; 0 ; c)$.\\
		Do $M$ là trực tâm tam giác $A B C$ nên $\heva{&\overrightarrow{A M} \cdot \overrightarrow{B C}=0 \\ &\overrightarrow{B M} \cdot \overrightarrow{A C}=0}\Leftrightarrow\heva{&-b-3 c=0 \\ &-2 a-3 c=0} \Leftrightarrow\heva{&b=-3 c \\ &a=-\dfrac{3 c}{2}.}$\\
		Thay $(2)$ vào $(1)$ ta có $-\dfrac{4}{3 c}-\dfrac{1}{3 c}-\dfrac{3}{c}=1 \Leftrightarrow c=-\dfrac{14}{3} \Rightarrow a=7$, $b=14$.\\
		Do đó $(\alpha)\colon \dfrac{x}{7}+\dfrac{y}{14}-\dfrac{3 z}{14}=1 \Leftrightarrow 2 x+y-3 z-14=0$.
	}
\end{ex}

\begin{ex}%[2H5V1-3]
	Trong không gian với hệ trục toạ độ $Oxyz,$ điểm $M\left(a,b,c\right)$ thuộc mặt phẳng $(P)\colon x+y+z-6=0$ và cách đều các điểm $A\left(1;6;0\right)$, $B\left(-2;2;-1\right)$, $C\left(5;-1;3\right).$ Tích $abc$ bằng
	\choice
	{\True $6$}
	{$-6$}
	{$0$}
	{$5$}
	\loigiai{
		Ta có
		\begin{eqnarray*}
			&\heva{&a+b+c=6\\&MA^2=MB^2\\&MA^2=MC^2}&\Leftrightarrow\heva{&a+b+c=6\\&
				\left(a-1\right)^2+\left(b-6\right)^2+b^2=\left(a+2\right)^2+\left(b-2\right)^2+\left(c+1\right)^2\\&
				\left(a-1\right)^2+\left(b-6\right)^2+c^2=\left(a-5\right)^2+\left(b+1\right)^2+\left(c-3\right)^2}\\
			&	&\Leftrightarrow\heva{&	a+b+c=6\\&3a+4b+c=14\\&4a-7b+3b=-1}\\ &&\Leftrightarrow\heva{&a=1\\&b=2\\&c=3}\Rightarrow abc=6.
	\end{eqnarray*}}
\end{ex}

\begin{ex}%[2H5V1-3]
	Trong không gian với hệ tọa độ $Oxyz,$ cho điểm $M\left(3;2;1\right)$. Mặt phẳng $(P)$ đi qua $M$ và cắt các trục tọa độ $Ox$, $Oy$, $Oz$ lần lượt tại các điểm $A$, $B$, $C$ không trùng với gốc tọa độ sao cho $M$ là trực tâm tam giác $ABC$. Trong các mặt phẳng sau, tìm mặt phẳng song song với mặt phẳng $(P)$.
	\choice
	{\True $3x+2y+z+14=0$}
	{$2x+y+3z+9=0$}
	{$3x+2y+z-14=0$}
	{$2x+y+z-9=0$}
	\loigiai{
		Gọi $A\left(a;0;0\right);B\left(0;b;0\right);C\left(0;0;c\right)$.\\
		Phương trình mặt phẳng $(P)$ có dạng $\dfrac{x}{a}+\dfrac{y}{b}+\dfrac{z}{c}=1$ $\left(abc\ne 0\right)$.\\
		Vì $(P)$ qua $M$ nên $\dfrac{3}{a}+\dfrac{2}{b}+\dfrac{1}{c}=1\quad(1)$.\\
		Ta có $\overrightarrow{MA}=\left(a-3;-2;-1\right)$; $\overrightarrow{MB}=\left(-3;b-2;-1\right)$; $\overrightarrow{BC}=\left(0;-b;c\right)$; $\overrightarrow{AC}=\left(-a;0;c\right)$.\\
		Vì $M$ là trực tâm của tam giác $ABC$ nên $$\heva{&				\overrightarrow{MA}\cdot \overrightarrow{BC}=0\\
			&\overrightarrow{MB}\cdot \overrightarrow{AC}=0}\Leftrightarrow\heva{&		2b=c\\&	3a=c} \quad (2).$$
		Từ $(1)$ và $(2)$ suy ra $a=\dfrac{14}{3}$; $b=\dfrac{14}{2}$; $c=14$.\\
		Khi đó phương trình $(P)\colon 3x+2y+z-14=0$.\\
		Vậy mặt phẳng song song với $(P)$ là $3x+2y+z+14=0$.}
\end{ex}

\begin{ex}%[2H5V1-3]
	Trong không gian với hệ tọa độ $O x y z$, cho các điểm $A(0 ; 1 ; 2)$, $B(2 ;-2 ; 0)$, $C(-2 ; 0 ; 1)$. Mặt phẳng $(P)$ đi qua $A$, trực tâm $H$ của tam giác $A B C$ và vuông góc với mặt phẳng $(A B C)$ có phương trình là
	\choice
	{\True $4 x-2 y-z+4=0$}
	{$4 x-2 y+z+4=0$}
	{$4 x+2 y+z-4=0$}
	{$4 x+2 y-z+4=0$}
	\loigiai{
		Ta có $\overrightarrow{A B}=(2 ;-3 ;-2)$, $\overrightarrow{A C}=(-2 ;-1 ;-1)$ nên $\left[\overrightarrow{A B}, \overrightarrow{A C}\right]=(1 ; 6 ;-8)$.\\
		Phương trình mặt phẳng $(A B C)$ là $x+6 y-8 z+10=0$.\\
		Phương trình mặt phẳng qua $B$ và vuông góc với $A C$ là $2 x+y+z-2=0$.\\
		Phương trình mặt phẳng qua $C$ và vuông góc với $A B$ là $2 x-3 y-2 z+6=0$.\\
		Giao điểm của ba mặt phẳng trên là trực tâm $H$ của tam giác $A B C$ nên $H\left(-\dfrac{22}{101} ; \dfrac{70}{101} ; \dfrac{176}{101}\right)$.\\
		Mặt phẳng $(P)$ đi qua $A$, $H$ nên $\overrightarrow{n_P} \perp \overrightarrow{A H}=\left(-\dfrac{22}{101} ;-\dfrac{31}{101} ;-\dfrac{26}{101}\right)=-\dfrac{1}{101}(22 ; 31 ; 26)$.\\
		Mặt phẳng $(P) \perp(A B C)$ nên $\overrightarrow{n_P} \perp \overrightarrow{n}_{(A B C)}=(1 ; 6 ;-8)$.\\
		Vậy $\left[\overrightarrow{n}_{(A B C)} ; \overrightarrow{u}_{A H}\right]=(404 ;-202 ;-101)$ là một vectơ pháp tuyến của $(P)$.\\
		Chọn $\overrightarrow{n}_P=(4 ;-2 ;-1)$ nên phương trình mặt phẳng $(P)$ là $4 x-2 y-z+4=0$.
	}
\end{ex}

\begin{ex}%[2H5V1-3]
	Trong không gian với hệ tọa độ $O x y z$, viết phương trình mặt phẳng $(P)$ đi qua $A(1 ; 1 ; 1)$ và $B(0 ; 2 ; 2)$ đồng thời cắt các tia $O x$, $O y$ lần lượt tại hai điểm $M$, $N$ ( không trùng với gốc tọa độ $O$ ) sao cho $O M=2 O N$.
	\choice
	{$(P)\colon 3x+y+2z-6=0$}
	{$(P)\colon 2x+3y-z-4=0$}
	{$(P)\colon 2x+y+z-4=0$}
	{\True $(P)\colon x+2 y-z-2=0$}
	\loigiai{
		Giả sử $(P)$ đi qua 3 điểm $M(a ; 0 ; 0)$, $N(0 ; b ; 0)$, $P(0 ; 0 ; c)$.\\
		Suy ra $(P)\colon \dfrac{x}{a}+\dfrac{y}{b}+\dfrac{z}{c}=1$.\\
		Mà $(P)$ đi qua $A(1 ; 1 ; 1)$ và $B(0 ; 2 ; 2)$ nên ta có hệ $\heva{&\dfrac{1}{a}+\dfrac{1}{b}+\dfrac{1}{c}=1 \\ &\dfrac{2}{b}+\dfrac{2}{c}=1} \Leftrightarrow\heva{&a=2 \\ &\dfrac{2}{b}+\dfrac{2}{c}=1.}$\\
		Theo giả thuyết ta có $O M=2 O N \Leftrightarrow|a|=2|b| \Leftrightarrow|b|=1$.
		\begin{itemize}
			\item \textbf{TH1.} $b=1 \Rightarrow c=-2$ suy ra $(P)\colon x+2 y-z-2=0$.
			\item \textbf{TH2.} $b=-1 \Rightarrow c=-\dfrac{2}{3}$ suy ra $(P)\colon x-2 y+3 z-2=0$.
		\end{itemize}
	}
\end{ex}

\begin{ex}%[2H5V1-3]
	Trong không gian $O x y z$, cho mặt phẳng $(\alpha)$ đi qua điểm $M(1 ; 2 ; 3)$ và cắt các trục $O x$, $O y$, $O z$ lần lượt tại $A$, $B$, $C$ (khác gốc tọa độ $O$ ) sao cho $M$ là trực tâm tam giác $A B C$. Mặt phẳng $(\alpha)$ có phương trình dạng $a x+b y+c z-14=0$. Tính tổng $T=a+b+c$.
	\choice
	{$8$}
	{$14$}
	{\True $6$}
	{$11$}
	\loigiai{
		Do $M$ là trực tâm tam giác $ABC$, nên ta có
		\begin{itemize}
			\item $OA \perp BC$ và $AM \perp BC$ nên $(OAM) \perp BC \Rightarrow OM \perp BC$.
			\item $OB \perp AC$ và $BM \perp AC$ nên $(OBM) \perp AC \Rightarrow OM \perp AC$.
		\end{itemize}
		Từ đó ta được $OM \perp (ABC)$ nên $\overrightarrow{OM}=(1;2;3)$ là vectơ pháp tuyến của $(ABC)$.\\
		Vậy phương trình mặt phẳng $(ABC)$ là $$1\cdot (x-1)+2\cdot (y-2)+3\cdot (x-3)=0 \Leftrightarrow x+2y+3z-14=0.$$
		Dẫn đến $a=1$, $b=2$, $c=3$ nên $T=1+2+3=6$.
	}
\end{ex}

\begin{ex}%[2H5V1-3]
	Trong không gian $O x y z$, cho hai mặt phẳng $(P)\colon x+4 y-2 z-6=0$, $(Q)\colon x-2 y+4 z-6=0$. Mặt phẳng $(\alpha)$ chứa giao tuyến của $(P)$, $(Q)$ và cắt các trục tọa độ tại các điểm $A$, $B$, $C$ sao cho hình chóp $O. A B C$ là hình chóp đều. Phương trình mặt phẳng $(\alpha)$ là
	\choice
	{\True $x+y+z-6=0$}
	{$x+y+z+6=0$}
	{$x+y+z-3=0$}
	{$x+y-z-6=0$}
	\loigiai{
		Mặt phẳng $(P)\colon x+4 y-2 z-6=0$ có vectơ pháp tuyến $\overrightarrow{n_P}=(1 ; 4 ;-2)$.\\
		Mặt phẳng $(Q)\colon x-2 y+4 z-6=0$ có vectơ pháp tuyến $\overrightarrow{n_Q}=(1 ;-2 ; 4)$.\\
		Ta có $\left[\overrightarrow{n}_P ; \overrightarrow{n}_Q\right]=(12 ;-6 ;-6)$, cùng phương với $\overrightarrow{u}=(2 ;-1 ;-1)$.\\
		Gọi $d=(P) \cap(Q)$. Ta có đường thẳng $d$ có vectơ chỉ phương là $\overrightarrow{u}=(2 ;-1 ;-1)$ và đi qua điểm $M(6 ; 0 ; 0)$.\\
		Mặt phẳng $(\alpha)$ cắt các trục tọa độ tại các điểm $A(a ; 0 ; 0)$, $B(0 ; b ; 0)$, $C(0 ; 0 ; c)$ với $a b c \neq 0$.\\
		Phương trình mặt phẳng $(\alpha)\colon \dfrac{x}{a}+\dfrac{y}{b}+\dfrac{z}{c}=1$.\\
		Mặt phẳng $(\alpha)$ có vectơ pháp tuyến $\vec{n}=\left(\dfrac{1}{a} ; \dfrac{1}{b} ; \dfrac{1}{c}\right)$.\\
		Mặt phẳng $(\alpha)$ chứa $d$ nên $$\heva{&\vec{n} \perp \vec{u} \\ &M \in(\alpha)} \Leftrightarrow\heva{&\dfrac{2}{a}-\dfrac{1}{b}-\dfrac{1}{c}=0 \\ &\dfrac{6}{a}=1} \Leftrightarrow\heva{&a=6 \\ &\dfrac{1}{b}+\dfrac{1}{c}=\dfrac{1}{3}.\quad (*)}$$
		Ta lại có hình chóp $O.ABC$ là hình chóp đều $$\Leftrightarrow O A=O B=O C \Leftrightarrow|a|=|b|=|c| \Leftrightarrow|b|=|c|=6.$$
		Kêt hợp với điều kiện $(*)$ ta được $b=c=6$.\\
		Vậy phương trình của mặt phẳng $(\alpha)\colon \dfrac{x}{6}+\dfrac{y}{6}+\dfrac{z}{6}=1 \Leftrightarrow x+y+z-6=0$.
	}
\end{ex}
\begin{ex}%[2H5V1-3]
	Trong không gian tọa độ $O x y z$, cho mặt phẳng $(\alpha)$ đi qua $M(1 ;-3 ; 8)$ và chắn trên $O z$ một đoạn dài gấp đôi các đoạn chắn trên các tia $O x$, $O y$. Giả sử $(\alpha)\colon a x+b y+c z+d=0$ ($a$, $b$, $c$, $d$ là các số nguyên). Tính $S=\dfrac{a+b+c}{d}$.
	\choice
	{$3$}
	{$-3$}
	{$\dfrac{5}{4}$}
	{\True $-\dfrac{5}{4}$}
	\loigiai{
		Giả sử mặt phẳng $(\alpha)$ cắt các tia $Ox$, $Oy$, $Oz$ lần lượt tại $A(m ; 0 ; 0)$, $B(0 ; n ; 0)$, $C(0 ; 0 ; p)$ (với $m$, $n$, $p>0$).\\
		Theo giả thiết có $O C=2 O A=2 O B \Rightarrow p=2 m=2 n. \quad(1)$\\
		Phương trình mặt phẳng $(\alpha)$ có dạng $\dfrac{x}{m}+\dfrac{y}{n}+\dfrac{z}{p}=1. \quad (2)$\\
		Do mặt phẳng $(\alpha)$ đi qua $M(1 ;-3 ; 8)$ nên $\dfrac{1}{m}-\dfrac{3}{n}+\dfrac{8}{p}=1$.\\
		Thay $(1)$ vào $(2)$ ta được $\dfrac{1}{m}-\dfrac{3}{m}+\dfrac{8}{2 m}=1 \Leftrightarrow \dfrac{2}{m}=1 \Leftrightarrow m=2 \Rightarrow m=n=2,\ p=4$.
		Phương trình mặt phẳng $(\alpha)$ có dạng $\dfrac{x}{2}+\dfrac{y}{2}+\dfrac{z}{4}=1 \Leftrightarrow 2 x+2 y+z-4=0$.\\
		Từ đó suy ra $a=2 t$, $b=2 t$, $c=t$, $d=-4 t \quad(t \neq 0)$.\\
		Vậy $S=\dfrac{a+b+c}{d}=-\dfrac{5}{4}$.
	}
\end{ex}
\Closesolutionfile{ans}
\indapan{10}{ans/ans-2-C5B1CD2-D4}

\Opensolutionfile{ans}[ans/ans-2-C5B1CD2-D4]
\TNSA
\begin{ex}%[2H5V1-3]
	Trong không gian với hệ trục tọa độ $O x y z$, cho hai điểm $A(3 ; 1 ; 7)$, $B(5 ; 5 ; 1)$ và mặt phẳng $(P)\colon 2 x-y-z+4=0$. Điểm $M$ thuộc $(P)$ sao cho $M A=M B=\sqrt{35}$. Biết $M$ có hoành độ nguyên, tính $O M$ (làm tròn đến chữ số hàng phần trăm).
	\shortans{$2{,}83$}
	\loigiai{
		Gọi $M(a ; b ; c)$ với $a \in \mathbb{Z}$, $b \in \mathbb{R}$, $c \in \mathbb{R}$.\\
		Ta có $\overrightarrow{A M}=(a-3 ; b-1 ; c-7)$ và $\overrightarrow{B M}=(a-5 ; b-5 ; c-1)$.\\
		Vì $\heva{&M \in ( P )\\&M A = M B = \sqrt { 3 5 }}\Leftrightarrow \heva{&M \in(P) \\&M A^2=M B^2\\&M A^2=35}$ nên ta có hệ phương trình sau
		\begin{eqnarray*}
			\allowdisplaybreaks
			& &\heva{&2a - b - c + 4 = 0\\&(a-3)^ {2} + ( b - 1) ^ {2} + (c-7)^{2} = (a-5)^{2} + ( b - 5 ) ^ { 2 } + ( c - 1 ) ^ { 2 }\\&( a - 3 ) ^ { 2 } + ( b - 1 ) ^ { 2 } + ( c - 7 ) ^ { 2 } = 3 5 }\\
			&\Leftrightarrow &\heva{&2 a-b-c=-4 \\&4 a+8 b-12 c=-8 \\&(a-3)^2+(b-1)^2+(c-7)^2=35}\\
			&\Leftrightarrow &\heva{&b=c\\&c=a+2 \\&(a-3)^2+(b-1)^2+(c-7)^2=35}\Leftrightarrow \heva{&b=a+2 \\&c=a+2 \\&3a^2-14=0}
			\Leftrightarrow \heva{&a=0 \\&b=2 \ (\text{do }a \in \mathbb{Z})\\&c=2.}
		\end{eqnarray*}
		Ta có $M(2 ; 2 ; 0)$. Suy ra $O M=2 \sqrt{2}\approx 2{,}83$.\\
	}
\end{ex}
\begin{ex}%[2H5V1-3]
	Trong không gian với hệ tọa độ $O x y z$, mặt phẳng $(P)$ chứa điểm $M(1 ; 3 ;-2)$, cắt các tia $O x$, $O y$, $O z$ lần lượt tại $A$, $B$, $C$ sao cho $\dfrac{O A}{1}=\dfrac{O B}{2}=\dfrac{O C}{4}$. Biết phương trình mặt phẳng $(P)$ có dạng $ax+by+cz-8=0$. Tính $P=\dfrac{a+c}{2b}$ (kết quả được viết dưới dạng số thập phân).
	\shortans{$1{,}25$}
	\loigiai{
		Phương trình mặt chắn cắt tia $O x$ tại $A(a ; 0 ; 0)$, cắt tia $O y$ tại $B(0 ; b ; 0)$, cắt tia $O z$ tại $C(0 ; 0 ; c)$ có dạng là $(P)\colon \dfrac{x}{a}+\dfrac{y}{b}+\dfrac{z}{c}=1$ (với $a>0, b>0, c>0$).\\
		Theo đề $\dfrac{O A}{1}=\dfrac{O B}{2}=\dfrac{O C}{4} \Leftrightarrow \dfrac{a}{1}=\dfrac{b}{2}=\dfrac{c}{4} \Rightarrow\heva{&a=\dfrac{b}{2} \\ &c=2 b.}$\\
		Vì $M(1 ; 3 ;-2)$ nằm trên mặt phẳng $(P)$ nên ta có $$\dfrac{1}{\frac{b}{2}}+\dfrac{3}{b}+\dfrac{-2}{2 b}=1 \Leftrightarrow \dfrac{4}{b}=1 \Leftrightarrow b=4.$$
		Khi đó $a=2$, $c=8$.\\
		Vậy phương trình mặt phẳng $(P)$ là $\dfrac{x}{2}+\dfrac{y}{4}+\dfrac{z}{8}=1 \Leftrightarrow 4 x+2 y+z-8=0$.\\
		Khi đó $=\dfrac{a+c}{2b}=\dfrac{4+1}{2\cdot 2}=1{,}25$
	}
\end{ex}
\begin{ex}%[2H5V1-3]
	Trong không gian với hệ tọa độ $O x y z$ cho mặt phẳng $(P)$ đi qua điểm $M(9 ; 1 ; 1)$ cắt các tia $O x$, $O y$, $O z$ tại $A$, $B$, $C$ ($A$, $B$, $C$ không trùng với gốc tọa độ ). Thể tích tứ diện $O A B C$ đạt giá trị nhỏ nhất là bao nhiêu  (kết quả được viết dưới dạng số thập phân)?
	\shortans{$40{,}5$}
	\loigiai{
		Giả sử $A(a ; 0 ; 0)$, $B(0 ; b ; 0)$, $C(0 ; 0 ; c)$ với $a$, $b$, $c>0$.\\
		Mặt phẳng $(P)$ có phương trình ( theo đoạn chắn) $$\dfrac{x}{a}+\dfrac{y}{b}+\dfrac{z}{c}=1.$$
		Vì mặt phẳng $(P)$ đi qua điểm $M(9 ; 1 ; 1)$ nên $$\dfrac{9}{a}+\dfrac{1}{b}+\dfrac{1}{c}=1.$$
		Ta có $1=\dfrac{9}{a}+\dfrac{1}{b}+\dfrac{1}{c} \geq 3 \sqrt[3]{\dfrac{9}{abc}} \Rightarrow abc\geq 243$.\\
		$$
		V_{O A B C}=\dfrac{1}{6} abc \geq \dfrac{243}{6}=\dfrac{81}{2}.$$
		Vậy thể tích tứ diện $O A B C$ đạt giá trị nhỏ nhất là $\dfrac{81}{2}=40{,}5$.
	}
\end{ex}

%Câu 70
\begin{ex}%[2H5H1-3]
	Trong không gian với hệ trục tọa độ $Oxyz$, cho ba điểm $A(a;0;0)$, $B(0;b;0)$, $C(0;0;c)$ với $a$, $b$, $c$ là ba số thực dương thay đổi, thỏa mãn điều kiện $\dfrac{1}{a}+\dfrac{1}{b}+\dfrac{1}{c}=2017$. Khi đó, mặt phẳng $(ABC)$ luôn đi qua một điểm cố định có tọa độ là $M(m;m;m)$. Tính giá trị $P=2017m+2$.
	\shortans[\kindSA]{$3$}	
	\loigiai{
		Phương trình mặt phẳng đi qua ba điểm $A(a;0;0)$, $B(0;b;0)$, $C(0;0;c)$ có dạng  $$(ABC) \colon \dfrac{x}{a}+\dfrac{y}{b}+\dfrac{z}{c}=1.$$
		Giả sử $M(m;m;m)$ là một điểm cố định nằm trên $(ABC)$. Khi đó ta có $$ M \in (ABC) \Leftrightarrow \dfrac{m}{a}+\dfrac{m}{b}+\dfrac{m}{c}=1 \Leftrightarrow m \left(\dfrac{1}{a}+\dfrac{1}{b}+\dfrac{1}{c}\right)=1 \Leftrightarrow m \cdot 2017 =1 \Leftrightarrow m =\dfrac{1}{2017}.$$
		Vậy $P=2017m+2=2017 \cdot \dfrac{1}{2017}+2=3$.
	}
\end{ex}
%Câu 71
\begin{ex}%[2H5V1-3]
	Trong không gian với hệ trục tọa độ $Oxyz$, cho ba điểm $M(1;2;5)$. Tính số mặt phẳng $(\alpha)$ đi qua $M$ và cắt các trục $Ox$, $Oy$, $Oz$ lần lượt tại $A$, $B$, $C$ sao cho $OA=OB=OC \neq 0$.
	\shortans[\kindSA]{$4$}	
	\loigiai{
		Gọi $A(a;0;0)$, $B(0;b;b)$, $C(0;0;c)$ lần lượt là giao điểm của mặt phẳng $(\alpha)$ với các trục $Ox$, $Oy$ và $Oz$ (với $abc \neq 0$). \\
		Khi đó $(\alpha) \equiv (ABC) \colon \dfrac{x}{a}+\dfrac{y}{b}+\dfrac{z}{c}=1$.\\
		Ta có $OA=\sqrt{a^2+0^2+0^2}=|a|$. Tương tự $OB=|b|$, $OC=|c|$.\\
		Vì $OA=OB=OC$ nên $\heva{&OA=OB\\ &OC=OB} \Leftrightarrow \heva{&|a|=|b|\\&|c|=|b|} \Leftrightarrow \heva{&a= \pm b \\ &c = \pm b.}$
		\begin{itemize}
			\item Trường hợp $1$: $a=b$, $c=b$. \\
			Khi đó $\dfrac{x}{b}+\dfrac{y}{b}+\dfrac{z}{b}=1$ mà $M(1;2;5) \in (ABC)$ nên $\dfrac{1}{b}+\dfrac{2}{b}+\dfrac{5}{b}=1 \Leftrightarrow b=8.$
			\item Trường hợp $2$: $a=b$, $c=-b$. \\
			Khi đó $\dfrac{x}{b}+\dfrac{y}{b}-\dfrac{z}{b}=1$ mà $M(1;2;5) \in (ABC)$ nên $\dfrac{1}{b}+\dfrac{2}{b}-\dfrac{5}{b}=1 \Leftrightarrow b=-2.$
			\item Trường hợp $3$: $a=-b$, $c=b$. \\
			Khi đó $\dfrac{x}{b}-\dfrac{y}{b}+\dfrac{z}{b}=1$ mà $M(1;2;5) \in (ABC)$ nên $\dfrac{1}{b}-\dfrac{2}{b}+\dfrac{5}{b}=1 \Leftrightarrow b=4.$
			\item Trường hợp $4$: $a=-b$, $c=-b$. \\
			Khi đó $\dfrac{x}{b}-\dfrac{y}{b}+\dfrac{z}{b}=1$ mà $M(1;2;5) \in (ABC)$ nên $\dfrac{1}{b}-\dfrac{2}{b}-\dfrac{5}{b}=1 \Leftrightarrow b=-6.$
		\end{itemize}
		Vậy có bốn mặt phẳng $(\alpha)$ thỏa yêu cầu bài toán.
	}
\end{ex}
\begin{ex}%[2H5V1-3]
	Trong không gian với hệ trục tọa độ $Oxyz$, có bao nhiêu mặt phẳng $(P)$ đi qua ba điểm $M(2;1;3)$, $A(0;0;4)$ và cắt hai trục $Ox$, $Oy$ lần lượt tại $B$, $C$ khác $O$ thỏa mãn diện tích tam giác $OBC$ bằng $1$?	
	\shortans[\kindSA]{$2$}	
	\loigiai{
		Gọi $B(b;0;0)$ và $C(0;c;0)$ lần lượt là giao điểm của $(P)$ với các trục $Ox$, $Oy$.\\
		Khi đó ta có phương trình mặt phẳng $(P) \colon \dfrac{x}{b}+\dfrac{y}{c}+\dfrac{z}{4}=1$.\\
		Vì $M(2;1;3) \in (P)$ nên ta có $\dfrac{2}{b}+\dfrac{1}{c}+\dfrac{3}{4}=1 \Leftrightarrow \dfrac{2}{b}+\dfrac{1}{c} = \dfrac{1}{4} \Leftrightarrow 4b+8c=bc$. \quad (1)\\
		Diện tích tam giác $OBC$ bằng $1$ nên $\dfrac{1}{2} \cdot OB \cdot OC =1 \Leftrightarrow |b| \cdot |c|=2 \Leftrightarrow |bc|=2.$ \quad (2)\\
		Từ (1) và (2), ta có hệ phương trình $\heva{&4b+8c=bc\\&|bc|=2.} \quad (I)$\\
		\begin{itemize}
			\item Xét trường hợp $bc>0$. \\
			Khi đó
			$$(I) \Leftrightarrow \heva{&4b+8c=bc\\&bc=2} \Leftrightarrow \heva{&4a+8b=2\\&2bc=4} \Leftrightarrow \heva{&2b=1-4c\\ &(1-4c)c=4}
			\Leftrightarrow \heva{&2b=1-4c\\ &4c^2-c+4=0  \; (\text{pt vô nghiệm}).}$$
			\item Xét trường hợp $bc<0$. \\
			Khi đó
			\begin{align*}
				(I) \Leftrightarrow \heva{&4b+8c=bc\\&bc=-2} \Leftrightarrow \heva{&4a+8b=2\\&2bc=-4} &\Leftrightarrow  \heva{&2b=1-4c\\ &(1-4c)c=-4}\\
				&\Leftrightarrow \heva{&2b=1-4c\\ &4c^2-c-4=0.} \\
				&\Leftrightarrow \heva {&2b=1-4c\\&c=\dfrac{1 \pm \sqrt{65}}{2}}\\
				&\Leftrightarrow \heva{&c=\dfrac{1 + \sqrt{65}}{2} \\ &b=\dfrac{-1-2\sqrt{65}}{2}} \; \text{hay} \;\heva{&c=\dfrac{1 - \sqrt{65}}{2} \\ &b=\dfrac{-1+2\sqrt{65}}{2}.} 
			\end{align*}
		\end{itemize}
		Vậy có $2$ cặp số $(b;c)$ thỏa yêu cầu bài toán nên có $2$ mặt phẳng $(P)$ thỏa yêu cầu bài tán.
	}
\end{ex}
\Closesolutionfile{ans}
\indapan{6}{ans/ans-2-C5B1CD2-D4}
% \begin{dang}{Bài toán thực tế}
Gắn hệ trục toạ độ vào mô hình. Đặt gốc toạ độ tại vị trí có "3 góc vuông"
% \newcommand{\gv}[4][black]{\draw[#1,thick] ($(#3)!8pt!(#2)$)--($(#3)!2!($($(#3)!8pt!(#2)$)!.5!($(#3)!8pt!(#4)$)$)$)--($(#3)!8pt!(#4)$);}
% \begin{longtable}{|>{\raggedright\arraybackslash}p{5.2cm}|>{\raggedright\arraybackslash}p{5.4cm}|>{\raggedright\arraybackslash}p{5.7cm}|}
% 		\hline
% 	    \multicolumn{3}{|>{\centering\arraybackslash}p{16.5cm}|}{\textbf{I. Gắn trục tọa độ đối với hình chóp}} \\ \hline    
% 	    \multicolumn{3}{|>{\centering\arraybackslash}p{16.5cm}|}{\textbf{1. Hình chóp có cạnh bên (SA) vuông góc với mặt đáy}} \\ \hline                                                                                                                                                                               
% 		\multicolumn{1}{|>{\raggedright\arraybackslash}p{5.2cm}|}{\begin{tabular}[l]{>{\raggedright\arraybackslash}p{5.2cm}} \textbf{Đáy là tam giác đều}
% 							\begin{tikzpicture}[>=stealth,font=\footnotesize]
% 							\def\a{3}
% 							\def\b{2}
% 							\def\h{2}
% 							\path (0:0) coordinate (A)
% 							++(0:\a) coordinate (C)
% 							++(-130:\b) coordinate (B)
% 							($(A)+(90:\h)$) coordinate (S)
% 							($(B)!1/2!(C)$) coordinate (O)
% 							($(O)+(90:3.5)$) coordinate (O1)
% 							($(S)+(O)-(A)$) coordinate (H);
% 							\draw[dashed,thick] (A)--(C);
% 							\draw[thick] (S)--(A)--(B)--(C)--(S)--(B);
% 							\draw[dashed,thick](A)--(O);
% 							\draw[thick](S)--(H);
% 							%Ve truc Ox,Oy, Oz
% 							\draw[thick,->](C)--($(O)!2!(C)$) node [pos=0.9,above ]{$x$};
% 							\draw[thick,->](A)--($(O)!1.2!(A)$) node [pos=0.9,above right]{$y$};
% 							\draw[thick,->](O)--(O1) node [pos=0.9,above right]{$z$};
% 							%Các góc vuông
% 							\gv{S}{H}{O}
% 							\gv{C}{O}{H}
% 							\gv{A}{O}{H}
% 							\gv{O}{A}{S}
% 							\foreach \x/\g in {A/-90,B/0,C/0,S/180,O/-10,H/-10}
% 							\fill[black] (\x) circle (1pt) ($(\g:4mm)+(\x)$) node {$\x$};	
% 						\end{tikzpicture}
				 
% 				 - Gọi $O$ là trung điểm $BC$. Chọn hệ trục tọa độ như hình vẽ, $AB=a=1$.
				
% 				- Tọa độ các điểm là:
						
% 						$O(0;0;0)$, $A \left(0;\dfrac{\sqrt{3}}{2};0\right)$, $B \left(\dfrac{-1}{2};0;0\right)$, $C \left(\dfrac{1}{2};0;0\right)$, $S \left(0;\dfrac{\sqrt{3}}{2};\underbrace {OH}_{ = SA}\right)$.
% 				\end{tabular}} &\multicolumn{1}{l|}{\begin{tabular}[l]{>{\raggedright\arraybackslash}p{5.2cm}}\textbf{Đáy là tam giác cân tại A}
				
% 					\begin{tikzpicture}[>=stealth,font=\footnotesize]
% 						\def\a{3.5}
% 						\def\b{2.5}
% 						\def\h{3.5}
% 						\path (0:0) coordinate (B)
% 						++(0:\a) coordinate (C)
% 						++(-150:\b) coordinate (A)
% 						($(B)!1/2!(C)$) coordinate (O)
% 						($(A)+(90:\h)$) coordinate (S)
% 						($(O)+(90:3.8)$) coordinate (O1)
% 						($(S)+(O)-(A)$) coordinate (H);
% 						\draw[dashed,thick] (B)--(C);
% 						\draw[thick] (S)--(B)--(A)--(C)--(S)--(A);
% 						\draw[dashed,thick](A)--(O);
% 						\draw[thick](S)--(H);
% 						%Ve truc Ox,Oy, Oz
% 						\draw[thick,->](C)--($(O)!1.4!(C)$) node [pos=0.9,below]{$x$};
% 						\draw[thick,->](A)--($(O)!1.4!(A)$) node [pos=0.9,below left]{$y$};
% 						\draw[thick,->](O)--(O1) node [pos=0.9,above right]{$z$};
% 						%Các góc vuông
% 						\gv{B}{A}{S}
% 						\gv{A}{O}{C}
% 						\foreach \x/\g in {A/-20,B/120,C/-50,S/180,O/40,H/180}
% 						\fill[black] (\x) circle (1pt) ($(\g:4mm)+(\x)$) node {$\x$};	
% 					\end{tikzpicture}
					
% 			- Gọi $O$ là trung điểm $BC$. Chọn hệ trục tọa độ như hình vẽ, $a=1$.
			
% 			- Tọa độ các điểm là:
				
% 				$O(0;0;0)$, $A \left(0;OA;0\right)$, $B \left(-OB;0;0\right)$, $C \left(OC;0;0\right)$, $S \left(0;OA;\underbrace {OH}_{ = SA}\right)$.
% 			\end{tabular}} & \begin{tabular}[l]{>{\raggedright\arraybackslash}p{5.6cm}}\textbf{ Đáy là tam giác cân tại B}
			
% 				\begin{tikzpicture}[>=stealth,font=\footnotesize]
% 					\def\a{3}
% 					\def\b{2}
% 					\def\h{2}
% 					\path (0:0) coordinate (A)
% 					++(0:\a) coordinate (C)
% 					++(-150:\b) coordinate (B)
% 					($(A)!1/2!(C)$) coordinate (O)
% 					($(A)+(90:\h)$) coordinate (S)
% 					($(O)+(90:2.5)$) coordinate (O1)
% 					($(S)+(O)-(A)$) coordinate (H);
% 					\draw[dashed,thick] (A)--(C);
% 					\draw[thick] (S)--(A)--(B)--(C)--(S)--(B);
% 					\draw[dashed,thick](B)--(O) ;
% 					\draw[thick](S)--(H);
% 					%Ve truc Ox,Oy, Oz
% 					\draw[thick,->](C)--($(O)!1.4!(C)$) node [pos=0.9,below]{$x$};
% 					\draw[thick,->](B)--($(O)!1.2!(B)$) node [pos=0.9,below left]{$y$};
% 					\draw[dashed,thick](O)--($(O)!1/2!(H)$);
% 					\draw[thick,->]($(O)!1/2!(H)$)--(O1) node [pos=0.9,above right]{$z$};
% 					%Các góc vuông
% 					\gv{S}{A}{C}
% 					\gv{B}{O}{C}
% 					\foreach \x/\g in {B/-20,A/120,C/-50,S/180,O/40,H/-10}
% 					\fill[black] (\x) circle (1pt) ($(\g:4mm)+(\x)$) node {$\x$};	
% 				\end{tikzpicture}
				
% 			- Gọi $O$ là trung điểm $BC$. Chọn hệ trục tọa độ như hình vẽ, $AB=a=1$.
			
% 			- Tọa độ các điểm là:
			
% 			$O(0;0;0)$, $A \left(\dfrac{-1}{2};0;0\right)$, $B \left(0;\dfrac{\sqrt{3}}{2};0\right)$, $C \left(\dfrac{1}{2};0;0\right)$, $S \left(0;\dfrac{\sqrt{3}}{2};\underbrace {OH}_{ = SA}\right)$.
% 	\end{tabular} \\ \hline
% 	\multicolumn{1}{|>{\raggedright\arraybackslash}p{5.2cm}|}{\begin{tabular}[l]{>{\raggedright\arraybackslash}p{5.2cm}} \textbf{Đáy là tam giác vuông tại $B$}
			
% 			\begin{tikzpicture}[>=stealth,font=\footnotesize,scale=1]
% 				\def\a{4}
% 				\def\b{3}
% 				\def\h{2.6}
% 				\path (0:0) coordinate (A)
% 				++(0:\a) coordinate (C)
% 				++(-150:\b) coordinate (B)
% 				($(C)!1.01!(B)$) coordinate (O)
% 				($(A)+(90:\h)$) coordinate (S)
% 				($(O)+(90:3.5)$) coordinate (O1)
% 				($(S)+(O)-(A)$) coordinate (H);
% 				\draw[dashed,thick] (A)--(C);
% 				\draw[thick] (S)--(A)--(B)--(C)--(S)--(B);
% 				\draw[thick](S)--(H);
% 				%Ve truc Ox,Oy, Oz
% 				\draw[thick,->](C)--($(O)!1.1!(C)$) node [pos=0.9,above ]{$x$};
% 				\draw[thick,->](A)--($(O)!1.2!(A)$) node [pos=0.9, above]{$y$};
% 				\draw[thick,->](O)--(O1) node [above]{$z$};
% 				%Các góc vuông
% 				\gv{H}{O}{A}
% 				\gv{C}{B}{A}
% 				\gv{C}{A}{S}
% 				\gv{S}{H}{B}
% 				\foreach \x/\g in {A/-90,B/0,C/-40,S/90,O/-110,H/-10}
% 				\fill[black] (\x) circle (1pt) ($(\g:4mm)+(\x)$) node {$\x$};	
% 			\end{tikzpicture}
% 	\end{tabular}} &\multicolumn{1}{l|}{\begin{tabular}[l]{>{\raggedright\arraybackslash}p{5.2cm}}\textbf{Đáy là tam giác vuông tại $A$}
	
% 	\begin{tikzpicture}[>=stealth,font=\footnotesize,scale=1]
% 		\def\a{3.5}
% 		\def\b{4}
% 		\def\h{3}
% 		\path (0:0) coordinate (O)
% 		++(0:\a) coordinate (C)
% 		++(-165:\b) coordinate (B)
% 		($(A)+(90:\h)$) coordinate (S)
% 		($(O)+(90:3.3)$) coordinate (O1)
% 		($(C)!1!(O)$) coordinate (A);
% 		\draw[dashed,thick] (B)--(A)--(C) (O)--(S);
% 		\draw[thick] (S)--(B)--(C)--(S);
% 		%Ve truc Ox,Oy, Oz
% 		\draw[thick,->](C)--($(O)!1.2!(C)$) node [pos=0.9,below]{$x$};
% 		\draw[thick,->](B)--($(O)!1.25!(B)$) node [pos=0.9,right]{$y$};
% 		\draw[thick,->](S)--(O1) node [pos=0.9,above right]{$z$};
% 		%Các góc vuông
% 		\gv{B}{O}{C}
% 		\gv{A}{O}{C}
% 		\foreach \x/\g in {A/-45,B/120,C/-50,S/180,O/145}
% 		\fill[black] (\x) circle (1pt) ($(\g:4mm)+(\x)$) node {$\x$};	
% 	\end{tikzpicture}
% 	\end{tabular}} & \begin{tabular}[l]{>{\raggedright\arraybackslash}p{5.6cm}}\textbf{Đáy là tam giác thường}
% 	\begin{tikzpicture}[>=stealth,font=\footnotesize,scale=1]
% 	\def\a{3.8}
% 	\def\b{3}
% 	\def\h{2}
% 	\path (0:0) coordinate (A)
% 	++(0:\a) coordinate (C)
% 	++(-150:\b) coordinate (B)
% 	($(A)!1/2!(C)$) coordinate (O)
% 	($(A)+(90:\h)$) coordinate (S)
% 	($(O)+(90:2.5)$) coordinate (O1)
% 	($(S)+(O)-(A)$) coordinate (H);
% 	\draw[dashed,thick] (A)--(C);
% 	\draw[thick] (S)--(A)--(B)--(C)--(S)--(B);
% 	\draw[dashed,thick](B)--(O) ;
% 	\draw[thick](S)--(H);
% 	%Ve truc Ox,Oy, Oz
% 	\draw[thick,->](C)--($(O)!1.4!(C)$) node [pos=0.9,below]{$x$};
% 	\draw[thick,->](B)--($(O)!1.2!(B)$) node [pos=0.9,below left]{$y$};
% 	\draw[dashed,thick](O)--($(O)!1/2!(H)$);
% 	\draw[thick,->]($(O)!1/2!(H)$)--(O1) node [pos=0.9,above right]{$z$};
% 	%Các góc vuông
% 	\gv{S}{A}{C}
% 	\gv{B}{O}{C}
% 	\foreach \x/\g in {B/-20,A/120,C/-50,S/180,O/40,H/-10}
% 	\fill[black] (\x) circle (1pt) ($(\g:4mm)+(\x)$) node {$\x$};	
% 	\end{tikzpicture}
% 	\end{tabular} \\ \hline
% 	\end{longtable}
% \begin{longtable}{|>{\raggedright\arraybackslash}p{5.4cm}|>{\raggedright\arraybackslash}p{5.2cm}|>{\raggedright\arraybackslash}p{5.8cm}|}
% 	\hline
% 	{\begin{tabular}[l]{>{\raggedright\arraybackslash}p{4.8cm}} \textbf{Đáy là tam giác vuông tại $B$}
			
% 			\begin{tikzpicture}[>=stealth,font=\footnotesize,scale=1]
% 				\def\a{4}
% 				\def\b{3}
% 				\def\h{2.6}
% 				\path (0:0) coordinate (A)
% 				++(0:\a) coordinate (C)
% 				++(-150:\b) coordinate (B)
% 				($(C)!1.01!(B)$) coordinate (O)
% 				($(A)+(90:\h)$) coordinate (S)
% 				($(O)+(90:3.5)$) coordinate (O1)
% 				($(S)+(O)-(A)$) coordinate (H);
% 				\draw[dashed,thick] (A)--(C);
% 				\draw[thick] (S)--(A)--(B)--(C)--(S)--(B);
% 				\draw[thick](S)--(H);
% 				%Ve truc Ox,Oy, Oz
% 				\draw[thick,->](C)--($(O)!1.1!(C)$) node [pos=0.9,above ]{$x$};
% 				\draw[thick,->](A)--($(O)!1.2!(A)$) node [pos=0.9, above]{$y$};
% 				\draw[thick,->](O)--(O1) node [above]{$z$};
% 				%Các góc vuông
% 				\gv{H}{O}{A}
% 				\gv{C}{B}{A}
% 				\gv{C}{A}{S}
% 				\gv{S}{H}{B}
% 				\foreach \x/\g in {A/-90,B/0,C/-40,S/90,O/-110,H/-10}
% 				\fill[black] (\x) circle (1pt) ($(\g:4mm)+(\x)$) node {$\x$};	
% 			\end{tikzpicture}
		
% 			- Chọn hệ trục tọa độ như hình vẽ, $a=1$.
			
% 			- Tọa độ các điểm là:
				
% 				$B \equiv O(0;0;0)$, $A \left(0;AB;0\right)$, $C \left(BC;0;0\right)$, $S \left(0;AB;\underbrace {BH}_{ = SA}\right)$.
	
% 	\end{tabular}}&{\begin{tabular}[l]{>{\raggedright\arraybackslash}p{5cm}}\textbf{Đáy là tam giác vuông tại $A$}
			
% 			\begin{tikzpicture}[>=stealth,font=\footnotesize,scale=1]
% 				\def\a{3.5}
% 				\def\b{4}
% 				\def\h{3}
% 				\path (0:0) coordinate (O)
% 				++(0:\a) coordinate (C)
% 				++(-165:\b) coordinate (B)
% 				($(A)+(90:\h)$) coordinate (S)
% 				($(O)+(90:3.3)$) coordinate (O1)
% 				($(C)!1!(O)$) coordinate (A);
% 				\draw[dashed,thick] (B)--(A)--(C) (O)--(S);
% 				\draw[thick] (S)--(B)--(C)--(S);
% 				%Ve truc Ox,Oy, Oz
% 				\draw[thick,->](C)--($(O)!1.2!(C)$) node [pos=0.9,below]{$x$};
% 				\draw[thick,->](B)--($(O)!1.25!(B)$) node [pos=0.9,right]{$y$};
% 				\draw[thick,->](S)--(O1) node [pos=0.9,above right]{$z$};
% 				%Các góc vuông
% 				\gv{B}{O}{C}
% 				\gv{A}{O}{C}
% 				\foreach \x/\g in {A/-45,B/120,C/-50,S/180,O/145}
% 				\fill[black] (\x) circle (1pt) ($(\g:4mm)+(\x)$) node {$\x$};	
% 			\end{tikzpicture}

% 			- Chọn hệ trục tọa độ như hình vẽ, $a=1$.
			
% 			- Tọa độ các điểm là:
				
% 				$A \equiv O(0;0;0)$, $B \left(0;OB;0\right)$, $C \left(AC;0;0\right)$, $S \left(0;0;SA \right)$.
	
% 	\end{tabular}}&{\begin{tabular}[l]{>{\raggedright\arraybackslash}p{5cm}} \textbf{Đáy là tam giác thường}
% 			\begin{tikzpicture}[>=stealth,font=\footnotesize,scale=1]
% 				\def\a{3.8}
% 				\def\b{3}
% 				\def\h{2}
% 				\path (0:0) coordinate (A)
% 				++(0:\a) coordinate (C)
% 				++(-150:\b) coordinate (B)
% 				($(A)!1/2!(C)$) coordinate (O)
% 				($(A)+(90:\h)$) coordinate (S)
% 				($(O)+(90:2.5)$) coordinate (O1)
% 				($(S)+(O)-(A)$) coordinate (H);
% 				\draw[dashed,thick] (A)--(C);
% 				\draw[thick] (S)--(A)--(B)--(C)--(S)--(B);
% 				\draw[dashed,thick](B)--(O) ;
% 				\draw[thick](S)--(H);
% 				%Ve truc Ox,Oy, Oz
% 				\draw[thick,->](C)--($(O)!1.4!(C)$) node [pos=0.9,below]{$x$};
% 				\draw[thick,->](B)--($(O)!1.2!(B)$) node [pos=0.9,below left]{$y$};
% 				\draw[dashed,thick](O)--($(O)!1/2!(H)$);
% 				\draw[thick,->]($(O)!1/2!(H)$)--(O1) node [pos=0.9,above right]{$z$};
% 				%Các góc vuông
% 				\gv{S}{A}{C}
% 				\gv{B}{O}{C}
% 				\foreach \x/\g in {B/-20,A/120,C/-50,S/180,O/40,H/-10}
% 				\fill[black] (\x) circle (1pt) ($(\g:4mm)+(\x)$) node {$\x$};	
% 			\end{tikzpicture}
			
% 				- Dựng đường cao $BO$ của $\triangle ABC$. Chọn hệ trục tọa độ như hình vẽ, $a=1$.\\
% 				- Tọa độ các điểm là:
			
% 				$O(0;0;0)$, $A \left(-OA;0;0\right)$, $B \left(0;OB;0\right)$, $C \left(OC;0;0\right)$, $S \left(-OA;0;\underbrace {OH}_{ = SA}\right)$.
% 	\end{tabular}}\\ \hline
% 	{\begin{tabular}[l]{>{\raggedright\arraybackslash}p{5cm}} \textbf{Đáy là hình vuông, hình chữ nhật}
			
% 			\begin{tikzpicture}[>=stealth,font=\footnotesize,scale=1]
% 				\def\a{3}
% 				\def\b{2}
% 				\def\h{2}
% 				\path 	(0:0) coordinate (A)
% 				++(0:\a) coordinate (D)
% 				++(-130:\b) coordinate (C)
% 				($(A)+(C)-(D)$) coordinate (B)
% 				($(A)+(90:\h)$) coordinate (S);
% 				\draw[dashed,thick] (S)--(A)--(B) (D)--(A);
% 				\draw[thick] (S)--(B)--(C)--(D)--(S)--(C);
% 				%Ve truc Ox,Oy, Oz
% 				\draw[thick,->](D)--($(A)!1.2!(D)$) node [pos=0.9,above ]{$x$};
% 				\draw[thick,->](B)--($(A)!1.2!(B)$) node [pos=0.9, above]{$y$};
% 				\draw[thick,->](S)--($(A)!1.2!(S)$) node [pos=0.9,above ]{$z$};
% 				%Các góc vuông
% 				\gv{D}{A}{S}
% 				\gv{D}{A}{B}
% 				\foreach \x/\g in {A/-90,B/-40,C/-40,D/-90,S/180}
% 				\fill[black] (\x) circle (1pt) ($(\g:4mm)+(\x)$) node {$\x$};	
% 			\end{tikzpicture}
		
% 			- Chọn hệ trục tọa độ như hình vẽ, $a=1$.
			
% 			- Tọa độ các điểm là:
			
% 			$A \equiv O(0;0;0)$, $B \left(0;AB;0\right)$, $C \left(AD;AB;0\right)$, $D(AD;0;0)$, $S \left(0;0;SA\right)$.
% \end{tabular}}&{\begin{tabular}[l]{>{\raggedright\arraybackslash}p{5cm}}\textbf{Đáy là hình thoi}
		
% 		\begin{tikzpicture}[>=stealth,font=\footnotesize,scale=1]
% 			\def\a{3}
% 			\def\b{2}
% 			\def\h{2}
% 			\path 	(0:0) coordinate (A)
% 			++(0:\a) coordinate (D)
% 			++(-130:\b) coordinate (C)
% 			($(A)+(C)-(D)$) coordinate (B)
% 			($(A)+(90:\h)$) coordinate (S)
% 			($(A)!1/2!(C)$) coordinate (O)
% 			($(S)+(O)-(A)$) coordinate (H);
% 			\draw[dashed,thick] (S)--(A)--(B) (D)--(A);
% 			\draw[thick] (S)--(B)--(C)--(D)--(S)--(C) (S)--(H);
% 			\draw[dashed,thick] (A)--(C) (B)--(D);
% 			%Ve truc Ox,Oy, Oz
% 			\draw[thick,->](A)--($(A)!-1/5!(C)$) node [pos=0.9,above ]{$x$};
% 			\draw[thick,->](B)--($(B)!-1/10!(D)$) node [pos=0.9, above]{$y$};
% 			\draw[thick,dashed](O)--($(O)!1/2!(H)$);
% 			\draw[thick,->]($(O)!1/2!(H)$)--($(O)!3/2!(H)$) node [pos=0.9,right ]{$z$};
	
% 			%Các góc vuông
% 			\gv{D}{A}{S}
% 			\gv{S}{H}{O}
% 			\gv{A}{O}{D}
% 			\foreach \x/\g in {A/-90,B/-40,C/-40,D/-90,S/180,O/-90,H/-40}
% 			\fill[black] (\x) circle (1pt) ($(\g:4mm)+(\x)$) node {$\x$};	
% 		\end{tikzpicture}
	
% 			- Chọn hệ trục tọa độ như hình vẽ, $a=1$.
			
% 			- Tọa độ các điểm là:
			
% 			$O(0;0;0)$, $A(OA;0;0)$, $B \left(0;OB;0\right)$, $C \left(-OC;0;0\right)$, $D(0;-OD;0)$, $S \left(OA;0;\underbrace {OH}_{ = SA} \right)$.
		
% \end{tabular}}&{\begin{tabular}[l]{>{\raggedright\arraybackslash}p{5.8cm}} \textbf{Đáy là hình thang vuông}
% 	\begin{tikzpicture}[>=stealth,font=\footnotesize,scale=1]
% 		\def\a{3}
% 		\def\b{2.1}
% 		\def\h{2.2}
% 		\path 	(0:0) coordinate (A)
% 		++(0:\a) coordinate (D)
% 		($(A)+(-140:\b)$) coordinate (B)
% 		($(A)+(90:\h)$) coordinate (S)
% 		($(A)!0.78!(D)$) coordinate (H)
% 		($(H)+(B)-(A)$) coordinate (C);
% 		\draw[dashed,thick] (S)--(A)--(B) (D)--(A) (C)--(H);
% 		\draw[thick] (S)--(B)--(C)--(D)--(S)--(C);
% 		%Ve truc Ox,Oy, Oz
% 		\draw[thick,->](D)--($(A)!1.2!(D)$) node [pos=0.9,above ]{$x$};
% 		\draw[thick,->](B)--($(A)!1.2!(B)$) node [pos=0.9, above]{$y$};
% 		\draw[thick,->](S)--($(A)!1.2!(S)$) node [pos=0.9,above ]{$z$};
% 		%Các góc vuông
% 		\gv{D}{A}{S}
% 		\gv{D}{A}{B}
% 		\gv{C}{H}{A}
% 		\foreach \x/\g in {A/-90,B/-40,C/-40,D/-90,S/180,H/120}
% 		\fill[black] (\x) circle (1pt) ($(\g:4mm)+(\x)$) node {$\x$};	
% 	\end{tikzpicture}
	
% 	 	- Chọn hệ trục tọa độ như hình vẽ, $a=1$.
	 
% 	 - Tọa độ các điểm là:
	 
% 	 $A \equiv O(0;0;0)$, $B \left(0;AB;0\right)$, $C \left(AH;AB;0\right)$, $D(AD;0;0)$, $S \left(0;0;SA\right)$.
% \end{tabular}}\\ \hline
% \end{longtable}
% \newpage
% 	\begin{longtable}{|>{\raggedright\arraybackslash}p{5cm}|>{\raggedright\arraybackslash}p{5cm}|>{\raggedright\arraybackslash}p{5cm}|}
% 	\hline
% 	\multicolumn{3}{|>{\centering\arraybackslash}p{16.5cm}|}{\textbf{2. Hình chóp có cạnh mặt bên $(SAB)$ vuông góc với mặt đáy}}                                                                                                                                                                                 \\ \hline
% 	\multicolumn{1}{|>{\raggedright\arraybackslash}p{5cm}|}{\begin{tabular}[l]{>{\raggedright\arraybackslash}p{4.5cm}} \textbf{Đáy là tam giác, mặt bên là tam giác thường}
% 			\begin{tikzpicture}[>=stealth,font=\footnotesize,scale=1]
% 				\def\a{4}
% 				\def\b{3}
% 				\def\h{3}
% 				\path (0:0) coordinate (A)
% 				++(0:\a) coordinate (C)
% 				++(-140:\b) coordinate (B)
% 				($(A)!0.55!(B)$) coordinate (O)
% 				($(A)!1/3!(B)$) coordinate (H)
% 				($(O)+(90:3.7)$) coordinate (O1)
% 				($(H)+(90:\h)$) coordinate (S)
% 				($(S)+(O)-(H)$) coordinate (K);
% 				\draw[dashed,thick] (A)--(C) (C)--(O);
% 				\draw[thick] (S)--(A)--(B)--(C)--(S)--(B);
% 				\draw[thick](S)--(H) (S)--(K);
% 				%Ve truc Ox,Oy, Oz
% 				\draw[thick,->](C)--($(O)!1.1!(C)$) node [pos=0.9,above ]{$x$};
% 				\draw[thick,->](A)--($(O)!1.3!(A)$) node [pos=0.9, above]{$y$};
% 				\draw[thick,->](O)--(O1) node [above]{$z$};
% 				%Các góc vuông
% 				\gv{C}{O}{B}
% 				\gv{S}{H}{B}
% 				\foreach \x/\g in {A/-90,B/0,C/-40,S/90,O/-110,H/-110,K/-45}
% 				\fill[black] (\x) circle (1pt) ($(\g:4mm)+(\x)$) node {$\x$};	
% 			\end{tikzpicture}
			
% 		- Vẽ đường cao $CO$ trong $\triangle ABC$. Chọn hệ trục như hình vẽ, $a=1$.
		
% 		- Tọa độ các điểm là:
				
% 				$O(0;0;0)$, $A \left(0;OA;0\right)$, $B \left(0;-OB;0\right)$, $C \left(OC;0;0\right)$, $S \left(0;OH;\underbrace {OK}_{ =SH}\right)$.
% 	\end{tabular}} &\multicolumn{1}{l|}{\begin{tabular}[l]{>{\raggedright\arraybackslash}p{5cm}}\textbf{Đáy là tam giác cân tại $C$ (hoặc đều), mặt bên là tam giác cân tại $S$ (hoặc đều)}
% 			\begin{tikzpicture}[>=stealth,font=\footnotesize,scale=1]
% 				\def\a{4}
% 				\def\b{3}
% 				\def\h{3.3}
% 				\path (0:0) coordinate (A)
% 				++(0:\a) coordinate (C)
% 				++(-140:\b) coordinate (B)
% 				($(A)!1/2!(B)$) coordinate (O)
% 				($(O)+(90:3.7)$) coordinate (O1)
% 				($(O)+(90:\h)$) coordinate (S);
% 				\draw[dashed,thick] (A)--(C) (C)--(O);
% 				\draw[thick] (S)--(A)--(B)--(C)--(S)--(B);
% 				\draw[thick](S)--(O);
% 				%Ve truc Ox,Oy, Oz
% 				\draw[thick,->](C)--($(O)!1.1!(C)$) node [pos=0.9,above ]{$x$};
% 				\draw[thick,->](A)--($(O)!1.3!(A)$) node [pos=0.9, above]{$y$};
% 				\draw[thick,->](O)--(O1) node [above]{$z$};
% 				%Các góc vuông
% 				\gv{C}{O}{B}
% 				\gv{S}{O}{A}
% 				\foreach \x/\g in {A/-90,B/0,C/-40,S/180,O/-110}
% 				\fill[black] (\x) circle (1pt) ($(\g:4mm)+(\x)$) node {$\x$};	
% 			\end{tikzpicture}
% 			- Gọi $O$ là trung điểm $BC$. Chọn hệ trục như hình vẽ, $a=1$.
			
% 			- Tọa độ các điểm là:
			
% 			$O(0;0;0)$, $A \left(0;OA;0\right)$, $B \left(0;-OB;0\right)$, $C \left(OC;0;0\right)$, $S \left(0;0;SO\right)$.
% 	\end{tabular}} & \begin{tabular}[l]{>{\raggedright\arraybackslash}p{5.4cm}} \textbf{Đáy là hình chữ nhật, hình vuông, mặt bên là tam giác thường}
% 			\begin{tikzpicture}[>=stealth,font=\footnotesize,scale=1]
% 			\def\a{3}
% 			\def\b{2}
% 			\def\h{2.5}
% 			\path (0:0) coordinate (A)
% 			++(0:\a) coordinate (B)
% 			++(-130:\b) coordinate (C)
% 			($(A)+(C)-(B)$) coordinate (D)
% 			($(A)!1/3!(B)$) coordinate (H)
% 			($(H)+(90:\h)$) coordinate (S)
% 			($(A)+(S)-(H)$) coordinate (K);
% 			\draw[dashed,thick] (S)--(A)--(D) (B)--(A) (S)--(H);
% 			\draw[thick] (S)--(D)--(C)--(B)--(S)--(C) (K)--(S);
% 			%Ve truc Ox,Oy, Oz
% 			\draw[thick,->](B)--($(A)!1.2!(B)$) node [pos=0.9,above ]{$x$};
% 			\draw[thick,->](D)--($(A)!1.2!(D)$) node [pos=0.9, above]{$y$};
% 			\draw[thick,->](A)--($(A)!1.2!(K)$) node [right]{$z$};
% 			%Các góc vuông
% 			\gv{S}{H}{B}
% 			\gv{D}{A}{B}
% 			\foreach \x/\g in {A/-90,D/-40,C/-40,B/-90,S/90,H/-90,K/180}
% 			\fill[black] (\x) circle (1pt) ($(\g:4mm)+(\x)$) node {$\x$};	
% 		\end{tikzpicture}
% 		- Chọn hệ trục tọa độ như hình vẽ, $a=1$.
		
% 		- Tọa độ các điểm là:
			
% 			$A \equiv O(0;0;0)$, $B \left(AB;0;0\right)$, $C \left(AB;AD;0\right)$, $D \left(0;AD;0\right)$, $S \left(AH;0;\underbrace {AK}_{ = SH}\right)$.
% 	\end{tabular} \\\hline
% \end{longtable}
% \newpage
% \begin{longtable}{|>{\raggedright\arraybackslash}p{8.5cm}|>{\raggedright\arraybackslash}p{8.5cm}|}
% 	\hline
% 	{\begin{tabular}[l]{>{\raggedright\arraybackslash}p{8.5cm}} \textbf{Hình chóp tam giác đều}
% 		\begin{tikzpicture}[>=stealth,font=\footnotesize,scale=1]
% 			\def\a{4}
% 			\def\b{3}
% 			\def\h{4}
% 			\path (0:0) coordinate (A)
% 			++(0:\a) coordinate (C)
% 			++(-150:\b) coordinate (B)
% 			($(B)!1/2!(C)$) coordinate (O)
% 			($(A)!2/3!(O)$) coordinate (H)
% 			($(H)+(90:\h)$) coordinate (S)
% 			($(S)+(O)-(H)$) coordinate (K);
% 			\draw[dashed,thick] (A)--(C) (A)--(O) (S)--(H);
% 			\draw[thick] (S)--(A)--(B)--(C)--(S)--(B) (S)--(K);
% 			\foreach \x/\g in {A/180,B/-90,C/0,S/180,H/-120}
% 				%Ve truc Ox,Oy, Oz
% 				\draw[thick,->](C)--($(C)!-0.2!(B)$) node [pos=0.9,above ]{$x$};
% 				\draw[thick,->](A)--($(A)!-0.2!(O)$) node [pos=0.9, above]{$y$};
% 				\draw[thick,->](O)--($(O)!1.15!(K)$) node [above]{$z$};
% 				%Các góc vuông
% 				\gv{S}{H}{O}
% 				\gv{A}{O}{K}
% 				\gv{K}{O}{C}
% 				\foreach \x/\g in {A/-90,B/-90,C/-90,S/180,H/-90,O/0,K/0}
% 				\fill[black] (\x) circle (1pt) ($(\g:4mm)+(\x)$) node {$\x$};	
% 			\end{tikzpicture}
			
% 			Gọi $O$ là trung điểm $BC$. Chọn hệ trục như hình vẽ, $a=1$.
			
% 			- Tọa độ các điểm là:
			
% 			$O(0;0;0)$, $A \left(0;\dfrac{AB \sqrt{3}}{2};0\right)$, $B \left(-\dfrac{BC}{2};0;0\right)$, $C \left(0;0;OC\right)$, $S \left(0;\underbrace {\dfrac{AB \sqrt{3}}{6}}_{ =SH};\underbrace {OK}_{ =SH}\right)$.
% 	\end{tabular}} &{\begin{tabular}[l]{>{\raggedright\arraybackslash}p{8.5cm}}\textbf{Hình chóp tứ giác đều}
% 			\begin{tikzpicture}[>=stealth,font=\footnotesize,scale=1]
% 					\def\a{3}
% 				\def\b{2.5}
% 				\def\h{3.4}
% 				\path (0:0) coordinate (D)
% 				++(0:\a) coordinate (A)
% 				++(-150:\b) coordinate (B)
% 				($(D)+(B)-(A)$) coordinate (C)
% 				($(D)!1/2!(B)$) coordinate (O)
% 				($(O)+(90:\h)$) coordinate (S);
% 				\draw[dashed,thick] (C)--(D)--(A) (D)--(S) (D)--(B) (A)--(C);
% 				\draw[thick] (S)--(C)--(B)--(A)--(S) (B)--(S);
% 				%Ve truc Ox,Oy, Oz
% 				\draw[thick,->](A)--($(C)!1.2!(A)$) node [pos=0.9,above ]{$x$};
% 				\draw[thick,->](B)--($(D)!1.4!(B)$) node [pos=0.9, above right]{$y$};
% 				\draw[dashed,thick] (S)--(O);
% 				\draw[thick,->](S)--($(O)!1.15!(S)$) node [above]{$z$};
% 				%Các góc vuông
% 				\gv{D}{O}{S}
% 				\gv{B}{O}{A}
% 				\foreach \x/\g in {A/-40,C/180,D/180,B/-140,S/180,O/-120}
% 				\fill[black] (\x) circle (1pt) ($(\g:4mm)+(\x)$) node {$\x$};	
% 			\end{tikzpicture}
			
% 			- Chọn hệ trục như hình vẽ, $a=1$.
			
% 			- Tọa độ các điểm là:
			
% 			$O(0;0;0)$, $A \left(\underbrace{\dfrac{AB \sqrt{2}}{2}}_{ =OA};0;0\right)$, $B\left(0;\underbrace{\dfrac{AB \sqrt{2}}{2}}_{ =OB};0\right)$, $C \left(\underbrace{-\dfrac{AB \sqrt{2}}{2}}_{ =-OA};0;0 \right)$,  $D \left(0;\underbrace{-\dfrac{AB \sqrt{2}}{2}}_{ =-OA};0 \right)$, $S \left(0;0;SO\right)$.
% 	\end{tabular}}\\ \hline
% \multicolumn{2}{|>{\centering\arraybackslash}p{17cm}|}{\textbf{II. Gắn tọa độ đối với hình lăng trụ}}                                                                                                                                                                                 \\ \hline
% \multicolumn{2}{|>{\centering\arraybackslash}p{17cm}|}{\textbf{1. Hình lăng trụ đứng}}                                                                                                                                                            \\ \hline
% {\begin{tabular}[l]{>{\raggedright\arraybackslash}p{8.5cm}} \textbf{Hình lập phương, hình hộp chữ nhật}
% 		\begin{tikzpicture}[>=stealth,font=\footnotesize,scale=0.9]
% 		\def\a{4} 
% 		\def\b{1.8}
% 		\def\h{2}
% 		\path 	(0:0) coordinate (A)
% 		++(0:\a) coordinate (D)
% 		++(-130:\b) coordinate (C)
% 		($(A)+(C)-(D)$) coordinate (B)
% 		($(A)+(90:\h)$) coordinate (A')
% 		($(B)+(90:\h)$) coordinate (B')
% 		($(C)+(90:\h)$) coordinate (C')
% 		($(D)+(90:\h)$) coordinate (D');
% 		\draw[dashed,thick] 	(B)--(A)--(D)	(A)--(A');
% 		\draw[thick] (C)--(C') 	(D)--(D') 	(B)--(B') 	(B)--(C)--(D) (A')--(B')--(C')--(D')--cycle;
% 			%Ve truc Ox,Oy, Oz
% 			\draw[thick,->](D)--($(A)!1.2!(D)$) node [pos=0.9,above ]{$x$};
% 			\draw[thick,->](B)--($(A)!1.4!(B)$) node [pos=0.9,right]{$y$};
% 			\draw[thick,->](A')--($(A)!1.2!(A')$) node [right]{$z$};
% 			%Các góc vuông
% 			\gv{B}{A}{D}
% 			\gv{D}{A}{A'}
% 			\foreach \x/\g in  {A/180,B/180,C/0,D/-85,A'/180,B'/180,C'/0,D'/0}
% 			\fill[black] (\x) circle (1pt) ($(\g:4mm)+(\x)$) node {$\x$};	
% 		\end{tikzpicture}
		
% 		- Chọn hệ trục như hình vẽ, $a=1$.
		
% 		- Tọa độ các điểm là:
		
% 	$A \equiv O \left(0;0;0\right)$, $B \left(0;AB;0\right)$, $C \left(AD;AB;0\right)$,  $D \left(AD;0;0\right)$, $A' \left(0;0;AA'\right)$, $B' \left(0;AB;AA'\right)$, $C' \left(AD;AB;AA'\right)$, $D' \left(AD;0;AA'\right)$.
% \end{tabular}} &{\begin{tabular}[l]{>{\raggedright\arraybackslash}p{8.5cm}}\textbf{\textbf{Hình lăng trụ đứng đáy là hình thoi}}
% 		\begin{tikzpicture}[>=stealth,font=\footnotesize,scale=0.9]
% 			\def\a{4} 
% 			\def\b{1.5}
% 			\def\h{1.8}
% 			\path 	(0:0) coordinate (A)
% 			++(0:\a) coordinate (D)
% 			++(-130:\b) coordinate (C)
% 			($(A)+(C)-(D)$) coordinate (B)
% 			($(A)+(90:\h)$) coordinate (A')
% 			($(B)+(90:\h)$) coordinate (B')
% 			($(C)+(90:\h)$) coordinate (C')
% 			($(D)+(90:\h)$) coordinate (D')
% 			($(A)!1/2!(C)$) coordinate (O)
% 			($(A')!1/2!(C')$) coordinate (O');
% 			\draw[dashed,thick] 	(B)--(A)--(D)	(A)--(A') (A)--(C) (B)--(D);
% 			\draw[thick] (C)--(C') 	(D)--(D') 	(B)--(B') 	(B)--(C)--(D) (A')--(B')--(C')--(D')--cycle (A')--(C') (B')--(D');
% 			%Ve truc Ox,Oy, Oz
% 			\draw[thick,->](C)--($(A)!1.2!(C)$) node [pos=0.9,below]{$x$};
% 			\draw[thick,->](B)--($(D)!1.1!(B)$) node [below right]{$y$};
% 			\draw[thick,dashed](O)--(O');
% 			\draw[thick,->](O')--($(O)!1.6!(O')$) node [right]{$z$};
% 			%Các góc vuông
% 			\gv{A'}{A}{D}
% 			\gv{B}{A}{D}
% 			\gv{A}{O}{D}
% 			\foreach \x/\g in  {A/180,B/159,C/15,D/-85,A'/180,B'/180,C'/0,D'/0,O/-90}
% 			\fill[black] (\x) circle (1pt) ($(\g:4mm)+(\x)$) node {$\x$};	
% 		\end{tikzpicture}
		
% 	- Chọn hệ trục như hình vẽ, $a=1$.
	
% 	- Tọa độ các điểm là:
	
% 	$O \left(0;0;0\right)$, $A \left(-OA;0;0\right)$, $B \left(0;OB;0\right)$, $C \left(OC;0;0\right)$,  $D \left(0;-OD:0\right)$, $A' \left(-OA;0;AA'\right)$, $B' \left(0;OB;AA'\right)$, $C' \left(OC;0;CC'\right)$, $D' \left(0;-OD;DD'\right)$.
% \end{tabular}}\\ \hline
% {\begin{tabular}[l]{>{\raggedright\arraybackslash}p{8.5cm}} \textbf{Lăng trụ tam giác đều}
% 		\begin{tikzpicture}[>=stealth,font=\footnotesize,scale=0.9]
% 			\def\a{4.3} 
% 			\def\b{2}
% 			\def\h{2.5}
% 			\path 	(0:0) coordinate (A)
% 			++(0:\a) coordinate (C)
% 			(A)	++(-50:\b) coordinate (B)
% 			($(A)+(90:\h)$) coordinate (A')
% 			($(B)+(90:\h)$) coordinate (B')
% 			($(C)+(90:\h)$) coordinate (C')
% 			($(A)!1/2!(C)$) coordinate (O)
% 			($(A')!1/2!(C')$) coordinate (O');
% 			\draw[dashed,thick] 	(A)--(C) (B)--(O) (O)--(O');
% 			\draw[thick] (C)--(C')--(B')--(A') (B)--(B') (B')--(O')	(B)--(C) (C')--(A')--(A)--(B) ;
% 			%Ve truc Ox,Oy, Oz
% 			\draw[thick,->](C)--($(A)!1.2!(C)$) node [pos=0.9,above ]{$x$};
% 			\draw[thick,->](B)--($(O)!1.4!(B)$) node [pos=0.9,right]{$y$};
% 			\draw[thick,->](O')--($(O)!1.2!(O')$) node [right]{$z$};
% 			%Các góc vuông
% 			\gv{B}{O}{C}
% 			\gv{C}{O}{O'}
% 				\foreach \x/\g in {A/180,B/180,C/-50,A'/180,B'/180,C'/0,O/130}
% 			\fill[black] (\x) circle (1pt) ($(\g:4mm)+(\x)$) node {$\x$};	
% 		\end{tikzpicture}
		
% 		- Chọn hệ trục như hình vẽ, $a=1$.
		
% 		- Tọa độ các điểm là:
		
% 		$O \left(0;0;0\right)$, $A \left(-\dfrac{AC}{2};0;0\right)$, $B \left(0;OB;0\right)$,  $C \left(\dfrac{AC}{2};0;0\right)$, $A' \left(-\dfrac{AC}{2};0;AA' \right)$, $B' \left(0;OB;AA'\right)$, $C' \left(\dfrac{AC}{2};0;AA'\right)$.
% \end{tabular}} &{\begin{tabular}[l]{>{\raggedright\arraybackslash}p{8.4cm}}\textbf{\textbf{Lăng trụ đứng có đáy là tam giác thường}}

% \begin{tikzpicture}[>=stealth,font=\footnotesize,scale=0.9]
% 	\def\a{4.3} 
% 	\def\b{2}
% 	\def\h{2.5}
% 	\path 	(0:0) coordinate (A)
% 	++(0:\a) coordinate (C)
% 	(A)	++(-50:\b) coordinate (B)
% 	($(A)+(90:\h)$) coordinate (A')
% 	($(B)+(90:\h)$) coordinate (B')
% 	($(C)+(90:\h)$) coordinate (C')
% 	($(A)!0.4!(C)$) coordinate (O)
% 	($(A')!0.4!(C')$) coordinate (O');
% 	\draw[dashed,thick] 	(A)--(C) (B)--(O) (O)--(O');
% 	\draw[thick] (C)--(C')--(B')--(A') (B)--(B') (B')--(O')	(B)--(C) (C')--(A')--(A)--(B) ;
% 	%Ve truc Ox,Oy, Oz
% 	\draw[thick,->](C)--($(A)!1.2!(C)$) node [pos=0.9,above ]{$x$};
% 	\draw[thick,->](B)--($(O)!1.4!(B)$) node [pos=0.9,right]{$y$};
% 	\draw[thick,->](O')--($(O)!1.2!(O')$) node [right]{$z$};
% 	%Các góc vuông
% 	\gv{B}{O}{C}
% 	\gv{C}{O}{O'}
% 	\foreach \x/\g in {A/180,B/180,C/-50,A'/180,B'/180,C'/0,O/130}
% 	\fill[black] (\x) circle (1pt) ($(\g:4mm)+(\x)$) node {$\x$};	
% \end{tikzpicture}
		
% 		- Vẽ đường cao $CO$ của $\triangle ABC$. Chọn hệ trục như hình vẽ, $a=1$.
		
% 		- Tọa độ các điểm là:
		
% 		$O \left(0;0;0\right)$, $A \left(-OA;0;0\right)$, $B \left(0;OB;0\right)$, $C \left(OC;0;0\right)$, $A' \left(-OA;0;AA'\right)$, $B' \left(0;OB;AA'\right)$, $C' \left(OC;0;AA'\right)$.
% \end{tabular}} \\ \hline
% \multicolumn{2}{|>{\centering\arraybackslash}p{17cm}|}{\textbf{2. Hình lăng trụ xiên}}                                                                          \\ \hline
% {\begin{tabular}[l]{>{\raggedright\arraybackslash}p{8.4cm}} \textbf{Lăng trụ có đáy là tam giác đều, hình chiếu của các đỉnh trên mặt đối diện là trung điểm của một cạnh tam giác đáy}
% 		\begin{tikzpicture}[>=stealth,font=\footnotesize,scale=0.9]
% 			\def\a{4.3} 
% 			\def\b{2}
% 			\def\h{2.5}
% 			\path 	(0:0) coordinate (A)
% 			++(0:\a) coordinate (C)
% 			(A)	++(-50:\b) coordinate (B)
% 			($(A)!1/2!(C)$) coordinate (O)
% 			($(O)+(90:\h)$) coordinate (A')
% 			($(A')+(B)-(A)$) coordinate (B')
% 			($(A')+(C)-(A)$) coordinate (C')
% 			($(A')!1/2!(C')$) coordinate (O');
% 			\draw[dashed,thick] 	(A)--(C) (B)--(O) (O)--(A');
% 			\draw[thick] (C)--(C')--(B')--(A') (B)--(B') (B')--(O')	(B)--(C) (C')--(A')--(A)--(B) ;
% 			%Ve truc Ox,Oy, Oz
% 			\draw[thick,->](C)--($(A)!1.2!(C)$) node [pos=0.9,above ]{$x$};
% 			\draw[thick,->](B)--($(O)!1.4!(B)$) node [pos=0.9,right]{$y$};
% 			\draw[thick,->](A')--($(O)!1.25!(A')$) node [right]{$z$};
% 			%Các góc vuông
% 			\gv{A}{O}{A'}
% 			\gv{A}{O}{B}
% 			\foreach \x/\g in {A/180,B/180,C/-50,A'/180,B'/180,C'/0,O/30}
% 			\fill[black] (\x) circle (1pt) ($(\g:4mm)+(\x)$) node {$\x$};	
% 		\end{tikzpicture}
		
% 		- Chọn hệ trục như hình vẽ, ta dễ xác định tọa đọ các điểm $O$, $A$, $B$, $C$, $A'$.
		
% 		- Tìm tọa độ các điểm còn lại thông qua $\overrightarrow{AA'}=\overrightarrow{BB'}=\overrightarrow{CC'}$.
% \end{tabular}} &{\begin{tabular}[l]{>{\raggedright\arraybackslash}p{8.4cm}}\textbf{\textbf{Lăng trụ xiên có đáy là hình vuông hoặc hình chữ nhật, hình chiếu của một đỉnh là một điểm thuộc cạnh đáy không chứa đỉnh đó}}
		
% 		\begin{tikzpicture}[>=stealth,font=\footnotesize,scale=0.9]
% 			\def\a{4} 
% 			\def\b{1.5}
% 			\def\h{2.8}
% 			\path (0:0) coordinate (A)
% 			++(0:\a) coordinate (D)
% 			++(-150:\b) coordinate (C)
% 			($(A)+(C)-(D)$) coordinate (B)
% 			($(B)!1/2!(C)$) coordinate (O)
% 			($(O)+(90:\h)$) coordinate (A')
% 			($(A')+(B)-(A)$) coordinate (B')
% 			($(A')+(C)-(A)$) coordinate (C')
% 			($(A')+(D)-(A)$) coordinate (D')
% 			($(A)!1/2!(D)$) coordinate (O');
% 			\draw[dashed,thick] 	(B)--(A)--(D)	(A)--(A') (A')--(O) (O)--(O');
% 			\draw[thick] (C)--(C') 	(D)--(D') 	(B)--(B') 	(B)--(C)--(D) (A')--(B')--(C')--(D')--cycle;
% 			%Ve truc Ox,Oy, Oz
% 			\draw[thick,->](C)--($(C)!-0.3!(B)$) node [pos=0.9,below]{$x$};
% 			\draw[thick,->,dashed](O')--($(O)!1.5!(O')$) node [right]{$y$};
% 			\draw[thick,->](A')--($(O)!1.3!(A')$) node [pos=0.9,right]{$z$};
% 			%Các góc vuông
% 			\gv{C}{O}{A'}
% 			\gv{C}{B}{A}
% 			\foreach \x/\g in  {A/180,B/180,C/-90,D/-85,A'/180,B'/180,C'/0,D'/0,O/-90}
% 			\fill[black] (\x) circle (1pt) ($(\g:4mm)+(\x)$) node {$\x$};		
% 		\end{tikzpicture}
		
% 		- Chọn hệ trục như hình vẽ, ta dễ xác định tọa đọ các điểm $O$, $A$, $B$, $C$, $A'$.
	
% 		- Tìm tọa độ các điểm còn lại thông qua $\overrightarrow{AA'}=\overrightarrow{BB'}=\overrightarrow{CC'}=\overrightarrow{DD'}$.
% \end{tabular}} \\ \hline
% \end{longtable}

\end{dang}
\TN
\Opensolutionfile{ans}[ans/ans-2C5B1CD3]
\begin{ex}%[2H5H1-5]
	Cho tứ diện $O.ABC$, có $OA$, $OB$, $OC$ đôi một vuông góc và $OA=5$, $OB=2$, $OC=4$. Gọi $M$, $N$ lần lượt là trung điểm của $OB$ và $OC$. Gọi $G$ là trọng tâm của tam giác $ABC$. Khoảng cách từ $G$ đến mặt phẳng $(AMN)$ là
	\choice
	{\True $\dfrac{20}{3\sqrt{129}}$}
	{$\dfrac{20}{\sqrt{129}}$}
	{$\dfrac{1}{4}$}
	{$\dfrac{1}{2}$}
	\loigiai{
		\begin{center}
			\begin{tikzpicture}[>=stealth,font=\footnotesize,scale=1]
				\def\a{3.5}
				\def\b{4.5}
				\def\h{3}
				\path (0:0) coordinate (O)
				++(0:\a) coordinate (C)
				++(-165:\b) coordinate (B)
				($(O)+(90:\h)$) coordinate (A)
				($(O)!1/2!(B)$) coordinate (M)
				($(O)!1/2!(C)$) coordinate (N) ;
				\draw[dashed,thick] (B)--(O)--(C) (O)--(A) (A)--(M)--(N)--(A);
				\draw[thick] (A)--(B)--(C)--(A);
				%Ve truc Ox,Oy, Oz
				\draw[thick,->](B)--($(O)!1.6!(B)$) node [pos=0.9,right]{$x$};
				\draw[thick,->](C)--($(O)!1.2!(C)$) node [pos=0.9,below]{$y$};
				\draw[thick,->](A)--($(O)!1.2!(A)$) node [pos=0.9,above right]{$z$};
				\foreach \x/\g in {B/120,C/-90,A/180,O/45,M/-30,N/60}
				\fill[black] (\x) circle (1pt) ($(\g:4mm)+(\x)$) node {$\x$};	
			\end{tikzpicture}
		\end{center}
	Chọn hệ trục tọa độ $Oxyz$ như hình vẽ.\\
	Ta có $O(0;0;0)$, $A \in Oz$, $B \in Ox$, $C \in Oy$ sao cho $OA=5$, $OB=2$, $OC=4$.\\
	Do đó $A(0;0;5)$, $B(2;0;0)$, $C(0;4;0)$.\\
	Khi đó $G$ là trọng tâm tam giác $ABC$ nên $G\left(\dfrac{2}{3};\dfrac{4}{3};\dfrac{5}{3}\right)$.\\
	Vì $M$ là trung điểm $OB$ nên $M(1;0;0)$.\\
	Vì $N$ là trung điểm $OC$ nên $N(0;2;0)$.\\
	Phương trình mặt phẳng $(AMN)$ là $\dfrac{x}{1}+\dfrac{y}{2}+\dfrac{z}{5}=1$ hay $10x+5y+2z-10=0$.\\
	Vậy khoảng cách từ $G$ đến mặt phẳng $(AMN)$ là
	$$\mathrm{d}(G,(AMN))=\dfrac{\left|\dfrac{20}{3}+\dfrac{20}{3}+\dfrac{10}{3}-10 \right|}{\sqrt{100+25+4}}=\dfrac{20}{3\sqrt{129}}.$$
	}
\end{ex}
\begin{ex}%[2H5V1-5]
Cho hình chóp $S.ABCD$ có đáy là hình thang vuông tại $A$ và $D$, $SA \perp (ABCD)$. Góc giữa $SB$ và mặt phẳng đáy bằng $45^\circ$, $E$ là trung điểm của $SD$, $AB=2a$, $AD=DC=a$. Tính khoảng cách từ điểm $B$ đến mặt phẳng $(ACE)$.
	\choice
	{ $\dfrac{2a}{2}$}
	{\True$\dfrac{4a}{3}$}
	{$a$}
	{$\dfrac{3a}{4}$}
	\loigiai{
\begin{center}
		\begin{tikzpicture}[>=stealth,font=\footnotesize,scale=1]
		\def\a{6.5}
		\def\b{3}
		\def\h{3.4}
		\path (0:0) coordinate (A)
		++(0:\a) coordinate (B)
		($(A)+(-145:\b)$) coordinate (D)
		($(A)+(90:\h)$) coordinate (S)
		($(A)!0.5!(B)$) coordinate (H)
		($(S)!0.5!(D)$) coordinate (E)
		($(H)+(D)-(A)$) coordinate (C);
		\draw[dashed,thick] (S)--(A)--(D) (B)--(A) (C)--(A)--(E);
		\draw[thick] (S)--(D)--(C)--(B)--(S)--(C) (C)--(E);
		\draw [thick]($(A)!7/8!(B)$) arc (180:135:0.5) node [pos=0.5,left]{$45^\circ$};
		%Ve truc Ox,Oy, Oz
		\draw[thick,->](B)--($(A)!1.2!(B)$) node [pos=0.9,above ]{$x$};
		\draw[thick,->](D)--($(A)!1.2!(D)$) node [pos=0.9, above]{$y$};
		\draw[thick,->](S)--($(A)!1.2!(S)$) node [pos=0.9,above ]{$z$};
		\foreach \x/\g in {A/40,D/-40,C/-40,B/-90,S/180,E/180}
		\fill[black] (\x) circle (1pt) ($(\g:4mm)+(\x)$) node {$\x$};	
	\end{tikzpicture}
\end{center}		
Hình chiếu của $SB$ trên mặt phẳng $(ABCD)$ là $AB$ nên góc giữa $SB$ và mặt đáy là góc giữa $SB$ và $AB$ bằng $\widehat{SBA}=45^\circ$.\\
Vì tam giác $SAB$ vuông cân tại $A$ nên $SA=2a$.\\
Chọn hệ trục tọa độ như hình vẽ, ta có $A(0;0;0)$, $B(0;2a;0)$, $C(a;a;0)$, $D(a;0;0)$, $S(0;0;2a)$, $E \left(\dfrac{a}{2}\;0;a \right)$.\\
Ta có $\overrightarrow{AC}=(a;a;0)$, $\overrightarrow{AE}= \left(\dfrac{a}{2};0;a\right)$. Do đó $\left[\overrightarrow{AC},\overrightarrow{AE}\right]=\left(a^2;-a^2;-\dfrac{a^2}{2}\right)$.\\
Mặt phẳng $(ACE)$ có véc-tơ pháp tuyến là $\overrightarrow{n}=(2;-2;-1)$ nên $(ACE) \colon 2x-2y-z=0$.\\
Vậy $\mathrm{d}(B,(ACE))=\dfrac{|2 \cdot 2a|}{\sqrt{4+4+1}}=\dfrac{4a}{3}$.
	}
\end{ex}
\begin{ex}%[2H5V1-5]
Trong KG $Oxyz$, cho hình chóp $SABCD$ có đáy $ABCD$ là hình chữ nhật. Biết $A(0;0;0)$, $D(2;0;0)$, $B(0;4;0)$, $S(0;0;4)$. Gọi $M$ là trung điểm của $SB$. Tính khoảng cách từ $B$ đến mặt phẳng $(CDM)$.
	\choice
	{$\mathrm{d}(B,(CDM))=2$}
	{$\mathrm{d}(B,(CDM))=2 \sqrt{2}$}
	{$\mathrm{d}(B,(CDM))=\dfrac{1}{\sqrt{2}}$}
	{\True$\mathrm{d}(B,(CDM))=\sqrt{2}$}
	\loigiai{
		\begin{center}
		\begin{tikzpicture}[>=stealth,font=\footnotesize,scale=1]
			\def\a{5}
			\def\b{2.5}
			\def\h{4}
			\path 	(0:0) coordinate (A)
			++(0:\a) coordinate (D)
			++(-130:\b) coordinate (C)
			($(A)+(C)-(D)$) coordinate (B)
			($(A)+(90:\h)$) coordinate (S)
			($(S)!1/2!(B)$) coordinate (M);
			\draw[dashed,thick] (S)--(A)--(B) (D)--(A) (C)--(D)--(M)--(C);
			\draw[thick] (S)--(B)--(C)--(D)--(S)--(C);
			%Ve truc Ox,Oy, Oz
			\draw[thick,->](D)--($(A)!1.2!(D)$) node [pos=0.9,above ]{$x$};
			\draw[thick,->](B)--($(A)!1.4!(B)$) node [pos=0.9, above]{$y$};
			\draw[thick,->](S)--($(A)!1.2!(S)$) node [pos=0.9,above ]{$z$};
			
			\foreach \x/\g in {A/-90,B/-40,C/-40,D/-90,S/180,M/180}
			\fill[black] (\x) circle (1pt) ($(\g:4mm)+(\x)$) node {$\x$};	
		\end{tikzpicture}		
		\end{center}
	Tứ giác $ABCD$ là hình chữ nhật nên $\heva{&x_A+x_C=x_B+x_D\\&y_A+y_C=y_B+y_D\\&z_A+z_C=z_B+z_D} \Leftrightarrow \heva{&x_C=2\\&y_C=4\\&z_C=0} \Leftrightarrow C(2;4;0)$.\\
	Vì $M$ là trung điểm $SB$ nên $M(0;2;2)$.\\
	Ta có $\overrightarrow{CD}=(0;-4;0)$, $\overrightarrow{CM}=(-2;-2;2)$. Do đó $\left[\overrightarrow{CD},\overrightarrow{CM}\right]=(-8;0;-8)$.\\
	Mặt phẳng $(CDM)$ có véc-tơ pháp tuyến là $\overrightarrow{n}=(1;0;1)$.\\
	Suy ra $(CDM)$ có phương trình $x+z-2=0$.\\
	Vậy $\mathrm{d}(B,(CDM))=\dfrac{|0+0-2|}{\sqrt{1^2+0^2+1^2}}=\sqrt{2}$.
	}
\end{ex}
\begin{ex}%[2H5V1-5]
Một phần sân trường được định vị bởi các điểm $A$, $B$, $C$, $D$ như hình vẽ.
\begin{center}
	\begin{tikzpicture}[>=stealth,font=\footnotesize,scale=1]
		\def\a{5}
		\def\b{2.5}
		\def\h{4}
		\path 	(0:0) coordinate (A)
		++(0:\a) coordinate (B)
		++(-90:\b*1.4) coordinate (C)
		($(A)+(-90:\b)$) coordinate (D);
		\draw[thick] (A)--(B) node[pos=0.5,above]{$2500$cm};
		\draw[thick] (B)--(C) node[pos=0.5,right]{$1500$cm};
		\draw[thick] (A)--(D) node[pos=0.5,left]{$1600$cm};
		\draw[thick] (D)--(C);
		%Ve truc Ox,Oy, Oz
		%\draw[thick,->](D)--($(A)!1.2!(D)$) node [pos=0.9,above ]{$x$};
		%\draw[thick,->](B)--($(A)!1.4!(B)$) node [pos=0.9, above]{$y$};
		%\draw[thick,->](S)--($(A)!1.2!(S)$) node [pos=0.9,above ]{$z$};
		
		\foreach \x/\g in {A/180,B/-40,C/-40,D/-90}
		\fill[black] (\x) circle (1pt) ($(\g:4mm)+(\x)$) node {$\x$};	
	\end{tikzpicture}
\end{center}
Bước đầu chúng được lấy "thăng bằng" để có cùng độ cao, biết $ABCD$ là hình thang vuông ở $A$ và $B$ với độ dài $AB=25$ m, $AD=15$ m, $BC=18$ m. Do yêu cầu kĩ thuật, khi lát phẳng phần sân trường phải thoát nước về góc sân ở $C$ nên người ta lấy độ cao ở các điểm $B$, $C$, $D$ xuống thấp hơn so với độ cao ở $A$ là $10$ cm, $a$ cm, $6$ cm tương ứng. Giá trị của $a$ là số nào sau đây?
	\choice
	{$15{,}7$ cm}
	{\True $17{,}2$ cm}
	{$18{,}1$ cm}
	{$17{,}5$ cm}
	\loigiai{
	Chọn hệ trục tọa độ $Oxyz$ sao cho $O \equiv A$, tia $Ox \equiv AD$, tia $Oy \equiv AB$.
	\begin{center}
		\begin{tikzpicture}[>=stealth,font=\footnotesize,scale=0.7]
			\def\a{5}
			\def\b{2.5}
			\def\h{4}
			\path 	(0:0) coordinate (A)
			++(0:\a) coordinate (B)
			++(-120:\b*1.4) coordinate (C)
			($(A)+(-123:\b)$) coordinate (D)
			($(B)+(-90:1.2)$) coordinate (B')
			($(C)+(-90:2.2)$) coordinate (C')
			($(D)+(-90:1.8)$) coordinate (D');
			\draw[thick] (A)--(B)--(C)--(D)--(A);
			\draw[dashed] (B)--(B')--(C')--(C)--(B);
			\draw[dashed] (D)--(D')--(C');
			%Ve truc Ox,Oy, Oz
			\draw[thick,->](D)--($(A)!1.8!(D)$) node [pos=0.9, above left]{$x$};
			\draw[thick,->](B)--($(A)!1.4!(B)$) node [pos=0.9,above ]{$y$};
			\draw[thick,->](A)--($(A)+(90:\h)$) node [pos=0.9,right]{$z$};
			
			\foreach \x/\g in {A/180,B/-40,C/140,D/180,B'/0,C'/0,D'/-90}
			\fill[black] (\x) circle (1pt) ($(\g:4mm)+(\x)$) node {$\x$};	
		\end{tikzpicture}
	\end{center}
	Khi đó $A(0;0;0)$, $B(0;2500;0)$, $C(1800;2500;0)$, $D(1500;0;0)$.\\
	Khi hạ độ cao các điểm ở các điểm $B$, $C$, $D$ xuống thấp hơn so với độ cao ở $A$ là $10$ cm, $a$ cm, $6$ cm tương ứng ta có các điểm mới $B'(0;2500;-10)$, $C(1800;2500;-a)$, $D'(1500;0;-6)$.\\
	Theo bài ta có bốn điểm $A$, $B'$, $C'$, $D'$ đồng phẳng.\\
	Phương trình mặt phẳng $(AB'D') \colon x+y+250z=0$.\\
	Do $C'(1800;2500;-a) \in (AB'D')$ nên có $1800+2500-250a=0 \Leftrightarrow a=17{,}2$.\\
	Vậy $a=17{,}2$ cm.
	}
\end{ex}
\Closesolutionfile{ans}
\indapan{10}{ans/ans-2C5B1CD3}
\Opensolutionfile{ans}[ans/ans-0-B15-KQ]
\TNSA
\begin{ex}%[2H2V2-6]
	Một sân vận động được xây dựng theo mô hình là hình chóp cụt $OAGD.BCFE$ có hai đáy song song với nhau. Mặt sân $OAGD$ là hình chữ nhật và được gắn hệ trục $Oxyz$ như hình vẽ dưới (đơn vị trên mỗi trục tọa độ là mét). Mặt sân $OAGD$ có chiều dài $OA=100 \mathrm{~m}$, chiều rộng $OD=60 \mathrm{~m}$ và tọa độ điểm $B(10;10;8)$. Tính khoảng cách từ điểm $G$ đến mặt phẳng $(OBED)$ (kết quả làm tròn đến hàng phần chục).
	\begin{center}
		\tikzset{every picture/.style={line width=0.75pt}}         
		\begin{tikzpicture}[x=0.75pt,y=0.75pt,yscale=-1,xscale=1]
			\draw (250,250) -- (250,77) ;
			\draw [shift={(250,75)}, rotate = 90] [color={rgb, 255:red, 0; green, 0; blue, 0 }  ][line width=0.75]    (10.93,-3.29) .. controls (6.95,-1.4) and (3.31,-0.3) .. (0,0) .. controls (3.31,0.3) and (6.95,1.4) .. (10.93,3.29)   ;
			\draw    (250,250) -- (398.1,200.63) ;
			\draw [shift={(400,200)}, rotate = 161.57] [color={rgb, 255:red, 0; green, 0; blue, 0 }  ][line width=0.75]    (10.93,-3.29) .. controls (6.95,-1.4) and (3.31,-0.3) .. (0,0) .. controls (3.31,0.3) and (6.95,1.4) .. (10.93,3.29)   ;
			\draw    (250,250) -- (126.86,200.74) ;
			\draw [shift={(125,200)}, rotate = 21.8] [color={rgb, 255:red, 0; green, 0; blue, 0 }  ][line width=0.75]    (10.93,-3.29) .. controls (6.95,-1.4) and (3.31,-0.3) .. (0,0) .. controls (3.31,0.3) and (6.95,1.4) .. (10.93,3.29)   ;
			\draw  [dash pattern={on 4.5pt off 4.5pt}]  (152,212) -- (275,175) ;
			\draw  [dash pattern={on 4.5pt off 4.5pt}]  (275,175) -- (375,207) ;
			\draw    (175,175) -- (152,212) ;
			\draw    (175,175) -- (275,212) ;
			\draw    (250,250) -- (275,212) ;
			\draw    (275,212) -- (350,186) ;
			\draw    (350,186) -- (375,207) ;
			\draw    (175,175) -- (275,150) ;
			\draw    (275,150) -- (350,186) ;
			\draw  [dash pattern={on 4.5pt off 4.5pt}]  (275,150) -- (275,175) ;
			
			\draw (402,203) node [anchor=north west][inner sep=0.75pt]   [align=left] {x};
			
			\draw (129,215) node [anchor=north west][inner sep=0.75pt]   [align=left] {y};
			
			\draw (252,78) node [anchor=north west][inner sep=0.75pt]   [align=left] {z};
			
			\draw (252,253) node [anchor=north west][inner sep=0.75pt]   [align=left] {O};
			
			\draw (377,210) node [anchor=north west][inner sep=0.75pt]   [align=left] {A};
			
			\draw (154,215) node [anchor=north west][inner sep=0.75pt]   [align=left] {D};
			
			\draw (264,177) node [anchor=north west][inner sep=0.75pt]   [align=left] {G};
			
			\draw (177,178) node [anchor=north west][inner sep=0.75pt]   [ below] {E};
			
			\draw (277,215) node [anchor=north west][inner sep=0.75pt]   [align=left] {B};
			
			\draw (351,163) node [anchor=north west][inner sep=0.75pt]   [align=left] {C};
			
			\draw (282,133) node [anchor=north west][inner sep=0.75pt]   [align=left] {F};		
		\end{tikzpicture}
	\end{center}
	\shortans{$62{,}5$}
	\loigiai{Gắn hình chóp cụt $OAGD.BCFE$ vào hệ trục $Oxyz$, ta có: $O(0;0;0), A(100;0;0), G(100; 60;0), \\D(0;60;0), B(10;10;8)$, $\overrightarrow{OD}=(0 ; 60 ; 0), \overrightarrow{OB}=(10 ; 10 ; 8)$.\\
		Véc-tơ pháp tuyến của mặt phẳng $(OBED)$ là $\vec{n}=[\overrightarrow{OD}, \overrightarrow{OB}]=(480 ; 0 ;-600) = 120(4 ; 0 ;-5)$.\\
		Phương trình mặt phẳng $(OBED)$ đi qua điểm $O(0 ; 0 ; 0)$ và có véc-tơ pháp tuyến $\vec{n}=(4 ; 0 ;-5)$ là  $4x-5z=0$.\\
		Khoảng cách từ điểm $G$ đến mặt phẳng $(OBED)$ là $$\mathrm{d}(G,(O B E D))=\dfrac{|4\cdot 100-5\cdot 0|}{\sqrt{16+25}}=\dfrac{400 \sqrt{41}}{41} \approx 62{,}5.$$
	}
\end{ex}
\begin{ex}%[2H2V2-6]
	Một công trình đang xây dựng được gắn hệ trục $Oxyz$ như hình vẽ dưới (đơn vị trên mỗi trục tọa độ là mét). Mỗi cột bê tông có dạng hình lăng trụ tứ giác đều và có tâm của mặt đáy trên lần lợt là $A(3 ; 2 ; 3), B(6 ; 3 ; 3), C(9 ; 4 ; 2), D\left(6 ; 0 ; \dfrac{5}{2}\right)$. Tính khoảng cách từ điểm $D$ đến mặt phẳng $(ABC)$ (kết quả làm tròn tới hàng phần trăm).
	\begin{center}
		\includegraphics[width=0.5\linewidth]{image/h3.png}
	\end{center}
	\shortans{$2,85$}
	\loigiai{Ta có $\overrightarrow{AB}=(3;1;0); \overrightarrow{AC}=(6;2;-1)$. Phương trình mặt phẳng $(ABC)$ qua $A$ và có véc-tơ pháp tuyến $[\overrightarrow{AB},\overrightarrow{AC}]=(-1;3;0)$ là $-x + 3y - 3 = 0$. \\
		Khoảng cách từ $D$ tới mặt phẳng $(ABC)$ là $\mathrm{d}(D,(ABC))=\dfrac{\left|-6-3\right|}{\sqrt{1^2 + 3^2}}=\dfrac{9\sqrt{10}}{10} \approx 2{,}85$.
	}
\end{ex}

\begin{ex}%[2H2V2-6]
	Một công trình đang xây dựng được gắn hệ trục $Oxyz$ (đơn vị trên mỗi trục tọa độ là mét). Ba bức tường $(P),(Q),(R)$ (như hình vẽ) của tòa nhà lần lượt có phương trình $(P)\colon x+2y-2z+1=0$, $(Q)\colon 2x+y+2z-3=0,(R)\colon 2x+4y-4z-19=0$. Tính khoảng cách giữa hai bức tường $(P)$ và $(R)$ của tòa nhà.
	\begin{center}
		\includegraphics[width=0.5\linewidth]{image/h1.png}
	\end{center}
	\shortans{$3{,}5$}
	\loigiai{Tính khoảng cách giữa hai bức tường $(P)$ và $(R)$ của tòa nhà.\\
		Chọn điểm $M(-1 ; 0 ; 0) \in(P)$. Do hai bức tường $(P)$ và $(R)$ song song nhau nên \\
		$$\mathrm{d}((P),(R))=\mathrm{d}(M,(R))=\dfrac{|2\cdot (-1)+4\cdot 0-4\cdot 0-19|}{\sqrt{4+16+16}}=\dfrac{21}{6}=3{,}5 \text{m.}$$}
\end{ex}


\begin{ex}%[2H2V2-6]
	Một công trình đang xây dựng được gắn hệ trục $O x y z$ (đơn vị trên mỗi trục tọa độ là mét). Ba bức tường $(P),(Q),(R),(T)$ (như hình vẽ) của tòa nhà lần lượt có phương trình $(P)\colon 2x-y-z+1=0$, $(Q)\colon x+3y-z-2=0,(R)\colon 4x-2y-2 z+9=0,(T)\colon 2x+6y-2z+15=0$. Tính chiều rộng bức tường $(Q)$ của tòa nhà (kết quả làm tròn đến hàng phần chục).
	\begin{figure}[!ht]
		\centering
		\includegraphics[width=0.5\linewidth]{image/h2.png}
	\end{figure}
	\shortans{$2{,}9$}
	\loigiai{
		Do hai bức tường $(P)$ và $(R)$ song song nhau nên chiều rộng bức tường $(Q)$ là khoảng cách giữa hai bức tường $(P)$ và $(R)$. Chọn điểm $N(0 ; 0 ; 1) \in(P)$.\\
		Do hai bức tường $(P)$ và $(R)$ song song nhau nên $$\mathrm{d}((P),(R))=\mathrm{d}(N,(R))=\dfrac{|4\cdot 0-2\cdot 0-2\cdot 1+9|}{\sqrt{4+1+1}}=\dfrac{7}{\sqrt{6}} \approx 2{,}9.$$}
\end{ex}
\begin{ex}%[2H2V2-6]
	Cho hình lập phương $ABCD.A'B'C'D'$ có độ dài cạnh bằng 1. Gọi $M, N, P, Q$ lần lượt là trung điểm của $AB, BC, C'D',DD'$. Chọn hệ tọa độ $Oxyz$ như hình vẽ, xác định tọa độ các điểm $M, N, P, Q$. 
	Tính khoảng cách từ điểm $Q$ đến mặt phẳng $(MNP)$. Kết quả làm tròn đến hàng phần chục.
	\begin{center}
		\tikzset{every picture/.style={line width=0.75pt}} %set default line width to 0.75pt        
		\begin{tikzpicture}[x=0.75pt,y=0.75pt,yscale=-1,xscale=1]
			%uncomment if require: \path (0,797); %set diagram left start at 0, and has height of 797
			%Straight Lines [id:da8382470567419189] 
			\draw    (250,400) -- (250,327) ;
			\draw [shift={(250,325)}, rotate = 90] [color={rgb, 255:red, 0; green, 0; blue, 0 }  ][line width=0.75]    (10.93,-3.29) .. controls (6.95,-1.4) and (3.31,-0.3) .. (0,0) .. controls (3.31,0.3) and (6.95,1.4) .. (10.93,3.29)   ;
			%Straight Lines [id:da22849660996364696] 
			\draw    (175,575) -- (375,575) ;
			%Straight Lines [id:da3429340193043935] 
			\draw    (375,575) -- (450,500) ;
			%Straight Lines [id:da6171158479105385] 
			\draw    (250,400) -- (175,475) ;
			%Straight Lines [id:da20159315767924957] 
			\draw    (175,475) -- (175,575) ;
			%Straight Lines [id:da09157475506424606] 
			\draw    (175,475) -- (375,475) ;
			%Straight Lines [id:da7794866428713987] 
			\draw    (375,475) -- (375,575) ;
			%Straight Lines [id:da5730802129402581] 
			\draw    (375,475) -- (450,400) ;
			%Straight Lines [id:da4069170044141508] 
			\draw    (250,400) -- (450,400) ;
			%Straight Lines [id:da7895384060713708] 
			\draw    (450,400) -- (450,500) ;
			%Straight Lines [id:da3822968918446086] 
			\draw    (212.5,437.5) -- (350,400) ;
			%Straight Lines [id:da5114427996681619] 
			\draw  [dash pattern={on 4.5pt off 4.5pt}]  (350,400) -- (412.5,537.5) ;
			%Straight Lines [id:da5639353335968631] 
			\draw  [dash pattern={on 4.5pt off 4.5pt}]  (212.5,437.5) -- (412.5,537.5) ;
			%Straight Lines [id:da7141193916967588] 
			\draw  [dash pattern={on 4.5pt off 4.5pt}]  (350,400) -- (375,525) ;
			%Straight Lines [id:da6592626042182406] 
			\draw  [dash pattern={on 4.5pt off 4.5pt}]  (250,500) -- (175,575) ;
			%Straight Lines [id:da4180570193860802] 
			\draw    (175,575) -- (126.41,623.59) ;
			\draw [shift={(125,625)}, rotate = 315] [color={rgb, 255:red, 0; green, 0; blue, 0 }  ][line width=0.75]    (10.93,-3.29) .. controls (6.95,-1.4) and (3.31,-0.3) .. (0,0) .. controls (3.31,0.3) and (6.95,1.4) .. (10.93,3.29)   ;
			%Straight Lines [id:da9167552187295651] 
			\draw  [dash pattern={on 4.5pt off 4.5pt}]  (250,500) -- (450,500) ;
			%Straight Lines [id:da8527980278686933] 
			\draw    (450,500) -- (523,500) ;
			\draw [shift={(525,500)}, rotate = 180] [color={rgb, 255:red, 0; green, 0; blue, 0 }  ][line width=0.75]    (10.93,-3.29) .. controls (6.95,-1.4) and (3.31,-0.3) .. (0,0) .. controls (3.31,0.3) and (6.95,1.4) .. (10.93,3.29)   ;
			%Straight Lines [id:da8000503760713495] 
			\draw  [dash pattern={on 4.5pt off 4.5pt}]  (250,400) -- (250,500) ;
			\draw (177,478) node [anchor=north west][inner sep=0.75pt]   [align=left] {A};
			
			\draw (252,403) node [anchor=north west][inner sep=0.75pt]   [align=left] {B};
			
			\draw (452,403) node [anchor=north west][inner sep=0.75pt]   [align=left] {C};
			
			\draw (376,481) node [anchor=north west][inner sep=0.75pt]   [align=left] {D};
			
			\draw (177,578) node [anchor=north west][inner sep=0.75pt]   [align=left] {A'};
			
			\draw (252,503) node [anchor=north west][inner sep=0.75pt]   [align=left] {B'};
			
			\draw (451,506) node [anchor=north west][inner sep=0.75pt]   [align=left] {C'};
			
			\draw (377,578) node [anchor=north west][inner sep=0.75pt]   [align=left] {D'};
			
			\draw (195,419) node [anchor=north west][inner sep=0.75pt]   [align=left] {M};
			
			\draw (351,381) node [anchor=north west][inner sep=0.75pt]   [align=left] {N};
			
			\draw (414.5,540.5) node [anchor=north west][inner sep=0.75pt]   [align=left] {P};
			
			\draw (351,527) node [anchor=north west][inner sep=0.75pt]   [align=left] {Q};
			
			\draw (513,518) node [anchor=north west][inner sep=0.75pt]   [align=left] {x};
			
			\draw (140,620) node [anchor=north west][inner sep=0.75pt]   [align=left] {y};
			
			\draw (263,331) node [anchor=north west][inner sep=0.75pt]   [align=left] {z};
		\end{tikzpicture}
	\end{center}
	\shortans{$1,4$}
	\loigiai{Thiết lập hệ tọa độ $O x y z$ như hình vẽ, gốc $O \equiv B'$. Khi đó $M\left(0 ; \dfrac{1}{2} ; 1\right), N\left(\dfrac{1}{2} ; 0 ; 1\right), P\left(1 ; \dfrac{1}{2} ; 0\right),\\
		Q\left(1 ; 1 ; \dfrac{1}{2}\right)$. Phương trình mặt phẳng $(MNP)$ đi qua $M\left(0;\dfrac{1}{2};1\right)$ và có véc-tơ pháp tuyến $[\overrightarrow {MN} ,\overrightarrow {MP}]=\left( \dfrac{1}{{2}};\dfrac{1}{{2}};\dfrac{1}{{2}}\right)$ là $2x + 2y + 2z - 3 = 0$.\\
		Khoảng cách từ điểm $Q$ đến mặt phẳng $(MNP)$ là $$\mathrm{d}(Q;(MNP)=\dfrac{\left|2\cdot 1+2\cdot 1+2\cdot \dfrac{1}{2}\right|}{\sqrt{2^2 +2^2 +2^2}}= \dfrac{5\sqrt{3}}{6}\approx 1{,}4.$$
	}
\end{ex}
\begin{ex}%[2H2V2-6]
	Cho hình chóp $S.ABCD$ có đáy $ABCD$ là hình vuông cạnh $a, SAD$ là tam giác đều và nằm trong mặt phẳng với đáy. Gọi $M$ và $N$ lần lượt là trung điểm của $BC$ và $CD$. Chọn hệ tọa độ $Oxyz$ như hình vẽ dưới. Gọi $Q$ là trung điểm $S D$. Tính khoảng cách giữa hai mặt phẳng $(SAC)$ và mặt phẳng $(ONQ)$ (kết quả làm tròn đến hàng phần chục).
	\begin{center}
		% Gradient Info
		\tikzset {_0pmvymc3o/.code = {\pgfsetadditionalshadetransform{ \pgftransformshift{\pgfpoint{-198 bp } { 158.4 bp }  }  \pgftransformscale{1.32 }  }}}
		\pgfdeclareradialshading{_ffechrcqo}{\pgfpoint{160bp}{-128bp}}{rgb(0bp)=(0,0,0);
			rgb(0bp)=(0,0,0);
			rgb(6.785714285714286bp)=(0,0,0);
			rgb(14.017857142857142bp)=(0,0,0);
			rgb(20bp)=(0,0,0);
			rgb(25bp)=(1,1,1);
			rgb(400bp)=(1,1,1)}
		\tikzset{every picture/.style={line width=0.75pt}} %set default line width to 0.75pt        
		\begin{tikzpicture}[x=0.75pt,y=0.75pt,yscale=-1,xscale=1]
			%uncomment if require: \path (0,300); %set diagram left start at 0, and has height of 300
			%Straight Lines [id:da1986885226481041] 
			\draw    (108,216) -- (297,216) ;
			\draw [shift={(202.5,216)}, rotate = 0] [color={rgb, 255:red, 0; green, 0; blue, 0 }  ][line width=0.75]      (0, 0) circle [x radius= 1.34, y radius= 1.34]   ;
			%Straight Lines [id:da8495468285480916] 
			\draw [color={rgb, 255:red, 0; green, 0; blue, 0 }  ,draw opacity=1 ][shading=_ffechrcqo,_0pmvymc3o]   (135,54) -- (171,63) -- (207,72) -- (243,81) -- (279,90) -- (315,99) -- (351,108) ;
			%Straight Lines [id:da7064838767550801] 
			\draw  [dash pattern={on 4.5pt off 4.5pt}]  (135,162) -- (324,162) ;
			%Straight Lines [id:da7232875579325917] 
			\draw  [dash pattern={on 4.5pt off 4.5pt}]  (162,108) -- (108,216) ;
			%Straight Lines [id:da6308494004055998] 
			\draw  [dash pattern={on 4.5pt off 4.5pt}]  (162,108) -- (351,108) ;
			%Straight Lines [id:da6682204451648537] 
			\draw    (351,108) -- (297,216) ;
			%Straight Lines [id:da3749747303473572] 
			\draw  [dash pattern={on 4.5pt off 4.5pt}]  (135,54) -- (135,162) ;
			%Straight Lines [id:da8031053998975075] 
			\draw    (135,54) -- (108,216) ;
			\draw [shift={(121.5,135)}, rotate = 99.46] [color={rgb, 255:red, 0; green, 0; blue, 0 }  ][line width=0.75]      (0, 0) circle [x radius= 1.34, y radius= 1.34]   ;
			%Straight Lines [id:da6847944951979155] 
			\draw    (135,54) -- (297,216) ;
			%Straight Lines [id:da6134510126128467] 
			\draw  [dash pattern={on 4.5pt off 4.5pt}]  (135,54) -- (162,108) ;
			%Straight Lines [id:da8173261646582135] 
			\draw    (324,162) -- (376,162) ;
			\draw [shift={(378,162)}, rotate = 180] [color={rgb, 255:red, 0; green, 0; blue, 0 }  ][line width=0.75]    (10.93,-3.29) .. controls (6.95,-1.4) and (3.31,-0.3) .. (0,0) .. controls (3.31,0.3) and (6.95,1.4) .. (10.93,3.29)   ;
			%Straight Lines [id:da8552794618253725] 
			\draw    (108,216) -- (81.89,268.21) ;
			\draw [shift={(81,270)}, rotate = 296.57] [color={rgb, 255:red, 0; green, 0; blue, 0 }  ][line width=0.75]    (10.93,-3.29) .. controls (6.95,-1.4) and (3.31,-0.3) .. (0,0) .. controls (3.31,0.3) and (6.95,1.4) .. (10.93,3.29)   ;
			%Straight Lines [id:da44029097674743123] 
			\draw    (135,54) -- (135,2) ;
			\draw [shift={(135,0)}, rotate = 90] [color={rgb, 255:red, 0; green, 0; blue, 0 }  ][line width=0.75]    (10.93,-3.29) .. controls (6.95,-1.4) and (3.31,-0.3) .. (0,0) .. controls (3.31,0.3) and (6.95,1.4) .. (10.93,3.29)   ;
			\draw (164,111) node [anchor=north west][inner sep=0.75pt]   [align=left] {A};
			
			\draw (353,111) node [anchor=north west][inner sep=0.75pt]   [align=left] {B};
			
			\draw (299,219) node [anchor=north west][inner sep=0.75pt]   [align=left] {C};
			
			\draw (110,219) node [anchor=north west][inner sep=0.75pt]   [align=left] {D};
			
			\draw (326,165) node [anchor=north west][inner sep=0.75pt]   [align=left] {M};
			
			\draw (137,165) node [anchor=north west][inner sep=0.75pt]   [align=left] {O};
			
			\draw (204.5,219) node [anchor=north west][inner sep=0.75pt]   [align=left] {N};
			
			\draw (366,164) node [anchor=north west][inner sep=0.75pt]   [align=left] {x};
			
			\draw (96,251) node [anchor=north west][inner sep=0.75pt]   [align=left] {y};
			
			\draw (136,8) node [anchor=north west][inner sep=0.75pt]   [align=left] {z};
			
			\draw (122,35) node [anchor=north west][inner sep=0.75pt]   [align=left] {S};
			
			\draw (97.5,137) node [anchor=north west][inner sep=0.75pt]   [align=left] {Q};
		\end{tikzpicture}
	\end{center}
	\shortans{$0{,}3$}
	\loigiai{
		Với hệ trục toạ độ như hình vẽ ta có $S\left(0 ; 0 ; \dfrac{a \sqrt{3}}{2}\right) ; M(a ; 0 ; 0) ; N\left(\dfrac{a}{2} ; \dfrac{a}{2} ; 0\right); A(0;-\dfrac{a}{2};0);$\\
		$B\left(a;-\dfrac{-a}{2};0\right); C\left(a; \dfrac{a}{2};0\right); D\left(0;\dfrac{a}{2};0\right);Q\left(0;\dfrac{a}{4};\dfrac{a\sqrt{3}}{4}\right)$. \\
		Lấy $a=1.$ Mặt phẳng $(SAC)$ qua $A$ và có véc-tơ pháp tuyến $[\overrightarrow{SA},\overrightarrow{AC}]$ là
		$2\sqrt3 x - 2\sqrt3 y + 2z - \sqrt3 = 0.$\\
		Khoảng cách cần tìm 
		$$\mathrm{d}((SAC);(OQN))=\mathrm{d}(O;(SAC))= \dfrac{\sqrt{21}}{14}\approx 0{,}3.$$
	}
\end{ex}
\begin{ex}%[2H2V2-6]
	Cho tứ diện $OABC$, có $OA, OB, OC$ đôi một vuông góc và $OA=5, OB=2, OC=4$. Gọi $M, N$ lần lượt là trung điểm của $OB$ và $OC$. Chọn hệ tọa độ $Oxyz$ như hình vẽ dưới. Tính khoảng cách từ điểm $B$ đến mặt phẳng $(AMN)$. Kết quả làm tròn đến hàng phần chục.
	\begin{center}		
		
		\tikzset{every picture/.style={line width=0.75pt}} %set default line width to 0.75pt        
		
		\begin{tikzpicture}[x=0.75pt,y=0.75pt,yscale=-1,xscale=1]
			%uncomment if require: \path (0,300); %set diagram left start at 0, and has height of 300
			
			%Straight Lines [id:da3679447504076472] 
			\draw  [dash pattern={on 4.5pt off 4.5pt}]  (252,140) -- (364,140) ;
			%Straight Lines [id:da13962597702284096] 
			\draw  [dash pattern={on 4.5pt off 4.5pt}]  (252,140) -- (196,196) ;
			%Straight Lines [id:da8580092790638094] 
			\draw  [dash pattern={on 4.5pt off 4.5pt}]  (252,56) -- (252,140) ;
			%Straight Lines [id:da7130062908773143] 
			\draw    (252,56) -- (196,196) ;
			%Straight Lines [id:da012778923343735649] 
			\draw    (252,56) -- (364,140) ;
			%Straight Lines [id:da3173437648870021] 
			\draw    (196,196) -- (364,140) ;
			%Straight Lines [id:da528126976168177] 
			\draw  [dash pattern={on 4.5pt off 4.5pt}]  (224,168) -- (308,140) ;
			%Straight Lines [id:da44405621585220234] 
			\draw  [dash pattern={on 4.5pt off 4.5pt}]  (252,56) -- (308,140) ;
			%Straight Lines [id:da4474952587057264] 
			\draw  [dash pattern={on 4.5pt off 4.5pt}]  (252,56) -- (224,168) ;
			%Straight Lines [id:da6666103459826491] 
			\draw    (252,56) -- (252,30) ;
			\draw [shift={(252,28)}, rotate = 90] [color={rgb, 255:red, 0; green, 0; blue, 0 }  ][line width=0.75]    (10.93,-3.29) .. controls (6.95,-1.4) and (3.31,-0.3) .. (0,0) .. controls (3.31,0.3) and (6.95,1.4) .. (10.93,3.29)   ;
			%Straight Lines [id:da6450164393512425] 
			\draw    (364,140) -- (390,140) ;
			\draw [shift={(392,140)}, rotate = 180] [color={rgb, 255:red, 0; green, 0; blue, 0 }  ][line width=0.75]    (10.93,-3.29) .. controls (6.95,-1.4) and (3.31,-0.3) .. (0,0) .. controls (3.31,0.3) and (6.95,1.4) .. (10.93,3.29)   ;
			%Straight Lines [id:da9897161761780247] 
			\draw    (196,196) -- (169.41,222.59) ;
			\draw [shift={(168,224)}, rotate = 315] [color={rgb, 255:red, 0; green, 0; blue, 0 }  ][line width=0.75]    (10.93,-3.29) .. controls (6.95,-1.4) and (3.31,-0.3) .. (0,0) .. controls (3.31,0.3) and (6.95,1.4) .. (10.93,3.29)   ;
			
			
			\draw (169,226) node [anchor=north west][inner sep=0.75pt]   [align=left] {x};
			
			\draw (380,142) node [anchor=north west][inner sep=0.75pt]   [align=left] {y};
			
			\draw (268,20) node [anchor=north west][inner sep=0.75pt]   [align=left] {z};
			
			\draw (226.8,46.8) node [anchor=north west][inner sep=0.75pt]   [align=left] {A};
			
			\draw (198,199) node [anchor=north west][inner sep=0.75pt]   [align=left] {B};
			
			\draw (360.2,149) node [anchor=north west][inner sep=0.75pt]   [align=left] {C};
			
			\draw (253,124.6) node [anchor=north west][inner sep=0.75pt]   [align=left] {O};
			
			\draw (209.8,155.6) node [anchor=north west][inner sep=0.75pt]   [align=left] {M};
			
			\draw (309,121) node [anchor=north west][inner sep=0.75pt]   [align=left] {N};
			
			
		\end{tikzpicture}
	\end{center}	
	\shortans{$0{,}9$}
	\loigiai{Chọn hệ trục tọa độ Oxyz như hình vẽ.
		
		\begin{center}
			
			\tikzset{every picture/.style={line width=0.75pt}} %set default line width to 0.75pt        
			
			\begin{tikzpicture}[x=0.75pt,y=0.75pt,yscale=-1,xscale=1]
				
				
				%Straight Lines [id:da3679447504076472] 
				\draw  [dash pattern={on 4.5pt off 4.5pt}]  (252,140) -- (364,140) ;
				%Straight Lines [id:da13962597702284096] 
				\draw  [dash pattern={on 4.5pt off 4.5pt}]  (252,140) -- (196,196) ;
				%Straight Lines [id:da8580092790638094] 
				\draw  [dash pattern={on 4.5pt off 4.5pt}]  (252,56) -- (252,140) ;
				%Straight Lines [id:da7130062908773143] 
				\draw    (252,56) -- (196,196) ;
				%Straight Lines [id:da012778923343735649] 
				\draw    (252,56) -- (364,140) ;
				%Straight Lines [id:da3173437648870021] 
				\draw    (196,196) -- (364,140) ;
				%Straight Lines [id:da528126976168177] 
				\draw  [dash pattern={on 4.5pt off 4.5pt}]  (224,168) -- (308,140) ;
				%Straight Lines [id:da44405621585220234] 
				\draw  [dash pattern={on 4.5pt off 4.5pt}]  (252,56) -- (308,140) ;
				%Straight Lines [id:da4474952587057264] 
				\draw  [dash pattern={on 4.5pt off 4.5pt}]  (252,56) -- (224,168) ;
				%Straight Lines [id:da6666103459826491] 
				\draw    (252,56) -- (252,30) ;
				\draw [shift={(252,28)}, rotate = 90] [color={rgb, 255:red, 0; green, 0; blue, 0 }  ][line width=0.75]    (10.93,-3.29) .. controls (6.95,-1.4) and (3.31,-0.3) .. (0,0) .. controls (3.31,0.3) and (6.95,1.4) .. (10.93,3.29)   ;
				%Straight Lines [id:da6450164393512425] 
				\draw    (364,140) -- (390,140) ;
				\draw [shift={(392,140)}, rotate = 180] [color={rgb, 255:red, 0; green, 0; blue, 0 }  ][line width=0.75]    (10.93,-3.29) .. controls (6.95,-1.4) and (3.31,-0.3) .. (0,0) .. controls (3.31,0.3) and (6.95,1.4) .. (10.93,3.29)   ;
				%Straight Lines [id:da9897161761780247] 
				\draw    (196,196) -- (169.41,222.59) ;
				\draw [shift={(168,224)}, rotate = 315] [color={rgb, 255:red, 0; green, 0; blue, 0 }  ][line width=0.75]    (10.93,-3.29) .. controls (6.95,-1.4) and (3.31,-0.3) .. (0,0) .. controls (3.31,0.3) and (6.95,1.4) .. (10.93,3.29)   ;
				
				
				\draw (169,226) node [anchor=north west][inner sep=0.75pt]   [align=left] {x};
				
				\draw (380,142) node [anchor=north west][inner sep=0.75pt]   [align=left] {y};
				
				\draw (268,20) node [anchor=north west][inner sep=0.75pt]   [align=left] {z};
				
				\draw (226.8,46.8) node [anchor=north west][inner sep=0.75pt]   [align=left] {A};
				
				\draw (198,199) node [anchor=north west][inner sep=0.75pt]   [align=left] {B};
				
				\draw (360.2,149) node [anchor=north west][inner sep=0.75pt]   [align=left] {C};
				
				\draw (253,124.6) node [anchor=north west][inner sep=0.75pt]   [align=left] {O};
				
				\draw (209.8,155.6) node [anchor=north west][inner sep=0.75pt]   [align=left] {M};
				
				\draw (309,121) node [anchor=north west][inner sep=0.75pt]   [align=left] {N};
				
				
			\end{tikzpicture}
		\end{center}
		Ta có $O(0 ; 0 ; 0), A \in {Oz}, B \in O x, C \in O y$ sao cho $A O=5, OB=2, OC=4\Rightarrow A(0 ; 0 ; 5), B(2 ; 0 ; 0), C(0 ; 4 ; 0)$. $M$ là trung điểm $OB$ nên $M(1 ; 0 ; 0)$. $N$ là trung điểm $OC$ nên $N(0 ; 2 ; 0)$.\\
		Phương trình mặt phẳng $(AMN)$ qua $A$ và có véc-tơ pháp tuyến $[\overrightarrow{AM},\overrightarrow{AN}]=(10;5;2)$ là $10x + 5y + 2z - 10 = 0$.\\
		Ta có $\mathrm{d}(B;(AMN))=\mathrm{d}(O;(AMN))=\dfrac{10}{\sqrt{129}}\approx 0{,}9$.
		
	}
\end{ex}

\begin{ex}%[2H2V2-6]
	Cho hình chóp $S.ABCD$ đáy là hình thang vuông tại $A$ và $D, SA \perp(ABCD)$. Góc giữa $SB$ và mặt phẳng đáy bằng $45^{\circ}, E$ là trung điểm của $SD, AB=2a, AD = DC = a$. Chọn hệ tọa độ $Oxyz$ như hình vẽ dưới. Tính khoảng cách từ điểm $B$ đến mặt phẳng $(AEC)$ (kết quả làm tròn đến hàng phần chục).
	\begin{center}	
		
		\tikzset{every picture/.style={line width=0.75pt}} %set default line width to 0.75pt        
		
		\begin{tikzpicture}[x=0.75pt,y=0.75pt,yscale=-1,xscale=1]
			%uncomment if require: \path (0,300); %set diagram left start at 0, and has height of 300
			
			%Straight Lines [id:da8913682824234928] 
			\draw  [dash pattern={on 4.5pt off 4.5pt}]  (252,140) -- (364,140) ;
			%Straight Lines [id:da039948733048584595] 
			\draw  [dash pattern={on 4.5pt off 4.5pt}]  (252,140) -- (196,196) ;
			%Straight Lines [id:da6718633500675624] 
			\draw  [dash pattern={on 4.5pt off 4.5pt}]  (252,56) -- (252,140) ;
			%Straight Lines [id:da9759242944770992] 
			\draw    (252,56) -- (196,196) ;
			\draw [shift={(224,126)}, rotate = 111.8] [color={rgb, 255:red, 0; green, 0; blue, 0 }  ][line width=0.75]      (0, 0) circle [x radius= 1.34, y radius= 1.34]   ;
			%Straight Lines [id:da563221306551871] 
			\draw    (252,56) -- (364,140) ;
			%Straight Lines [id:da15516892739193677] 
			\draw    (280,196) -- (364,140) ;
			%Straight Lines [id:da7881469638756575] 
			\draw    (252,56) -- (252,30) ;
			\draw [shift={(252,28)}, rotate = 90] [color={rgb, 255:red, 0; green, 0; blue, 0 }  ][line width=0.75]    (10.93,-3.29) .. controls (6.95,-1.4) and (3.31,-0.3) .. (0,0) .. controls (3.31,0.3) and (6.95,1.4) .. (10.93,3.29)   ;
			%Straight Lines [id:da6714011604516419] 
			\draw    (364,140) -- (390,140) ;
			\draw [shift={(392,140)}, rotate = 180] [color={rgb, 255:red, 0; green, 0; blue, 0 }  ][line width=0.75]    (10.93,-3.29) .. controls (6.95,-1.4) and (3.31,-0.3) .. (0,0) .. controls (3.31,0.3) and (6.95,1.4) .. (10.93,3.29)   ;
			%Straight Lines [id:da5385347511589023] 
			\draw    (196,196) -- (169.41,222.59) ;
			\draw [shift={(168,224)}, rotate = 315] [color={rgb, 255:red, 0; green, 0; blue, 0 }  ][line width=0.75]    (10.93,-3.29) .. controls (6.95,-1.4) and (3.31,-0.3) .. (0,0) .. controls (3.31,0.3) and (6.95,1.4) .. (10.93,3.29)   ;
			%Straight Lines [id:da05923631932446871] 
			\draw    (196,196) -- (280,196) ;
			%Straight Lines [id:da8592722804571848] 
			\draw    (252,56) -- (280,196) ;
			%Straight Lines [id:da802931795948812] 
			\draw  [dash pattern={on 4.5pt off 4.5pt}]  (252,140) -- (280,196) ;
			%Straight Lines [id:da4994838123991385] 
			\draw  [dash pattern={on 4.5pt off 4.5pt}]  (224,126) -- (252,140) ;
			%Straight Lines [id:da23640159060964017] 
			\draw  [dash pattern={on 4.5pt off 4.5pt}]  (224,126) -- (280,196) ;
			
			
			\draw (169,226) node [anchor=north west][inner sep=0.75pt]   [align=left] {x};
			
			\draw (380,142) node [anchor=north west][inner sep=0.75pt]   [align=left] {y};
			
			\draw (268,20) node [anchor=north west][inner sep=0.75pt]   [align=left] {z};
			
			\draw (238,44.4) node [anchor=north west][inner sep=0.75pt]   [align=left] {S};
			
			\draw (198,199) node [anchor=north west][inner sep=0.75pt]   [align=left] {D};
			
			\draw (360.2,149) node [anchor=north west][inner sep=0.75pt]   [align=left] {B};
			
			\draw (253,124.6) node [anchor=north west][inner sep=0.75pt]   [align=left] {A};
			
			\draw (281,198) node [anchor=north west][inner sep=0.75pt]   [align=left] {C};
			
			\draw (211,114) node [anchor=north west][inner sep=0.75pt]   [align=left] {E};
			
			
		\end{tikzpicture}
	\end{center}
	\shortans{$1{,}3$}
	\loigiai{Lấy $a=1$. Ta có $(SB,(ABCD))=\widehat{SBA}=45^0 \Rightarrow \triangle ASB$ vuông cân tại $A.$ Suy ra $SA=AB=2$.\\
		Ta có $A(0;0;0); S(0;0;2); C(1;1;0); B(0;2;0); D(1;0;0); E\left(\dfrac{1}{2};0;1\right)$.\\
		Phương trình mặt phẳng $(AEC)$ qua $A$ và có véc-tơ pháp tuyến $[\overrightarrow{AE}, \overrightarrow{AC}]=\left(-1;1;\dfrac{1}{2}\right)$ là $-2x + 2y + z = 0$.\\
		Khoảng cách từ điểm $B$ đến mặt phẳng $(AEC)$ là 
		$$\mathrm{d}(B,(AEC))=\dfrac{|2\cdot 2|}{\sqrt{2^2 +2^2 +1^2}}=\dfrac{4}{3}\approx 1{,}3.$$}
\end{ex}

\begin{ex}%[2H2V2-6]
	Trong không gian với hệ trục tọa độ $Oxyz$, cho bốn điểm $S(-1 ; 6 ; 2), A(0 ; 0 ; 6),\\ B(0 ; 3 ; 0)$, $C(-2 ; 0 ; 0)$. Gọi $H$ là chân đường cao vẽ từ $S$ của tứ diện $S.ABC$. Giả sử phương trình mặt phẳng đi qua ba điểm $S,B,H$ có dạng $x+by+cz+d=0$ với $b,c,d \in \mathbb{Z}$. Tính $b+c+d.$
	\shortans{$-17$}
	\loigiai{Phương trình mặt phẳng $(ABC): \dfrac{x}{-2}+\dfrac{y}{3}+\dfrac{z}{6}=1 \Leftrightarrow-3 x+2 y+z-6=0$. \\
		$H$ là chân đường cao vẽ từ $S$ của tứ diện $S.ABC$ nên $H$ là hình chiếu vuông góc của $S$ lên mặt phẳng $(A B C) \Rightarrow H\left(\dfrac{19}{14} ; \dfrac{31}{7} ; \dfrac{17}{14}\right)$.\\
		Mặt phẳng $(SBH)$ qua $B(0;3;0)$ và có véc-tơ pháp tuyến $$[\overrightarrow{BH}, \overrightarrow{SB}]=\left(\dfrac{11}{14} ; \dfrac{55}{14} ;-\dfrac{11}{2}\right)=\dfrac{11}{14}(1 ; 5 ;-7).$$
		Phương trình mặt phẳng $(SBH)$ là $ x+5(y-3)-7 z=0\\ \Leftrightarrow x+5 y-7 z-15=0$. Ta có $b+c+d =-17.$
	}
\end{ex}
\begin{ex}%[2H2V2-6]
	Trong KG $Oxyz$, cho hình chóp $S.ABCD$, đáy $ABCD$ là hình chữ nhật. Biết $A(0 ; 0 ; 0), D(2 ; 0 ; 0), B(0 ; 4 ; 0), S(0 ; 0 ; 4)$. Gọi $M$ là trung điểm của $SB$ và $G$ là trọng tâm của tam giác $SCD$. Tính khoảng cách từ điểm $B$ đến mặt phẳng $(AMG)$. Kết quả làm tròn đến hàng phần chục.
	\shortans{$2{,}8$}
	\loigiai{
		\begin{center}
			\tikzset{every picture/.style={line width=0.75pt}} %set default line width to 0.75pt        
			\begin{tikzpicture}[x=0.75pt,y=0.75pt,yscale=-1,xscale=1]
				%uncomment if require: \path (0,300); %set diagram left start at 0, and has height of 300
				
				%Straight Lines [id:da8697087570722153] 
				\draw  [dash pattern={on 4.5pt off 4.5pt}]  (252,140) -- (364,140) ;
				%Straight Lines [id:da4072889759548881] 
				\draw  [dash pattern={on 4.5pt off 4.5pt}]  (252,140) -- (196,196) ;
				%Straight Lines [id:da5044623873055065] 
				\draw  [dash pattern={on 4.5pt off 4.5pt}]  (252,56) -- (252,140) ;
				%Straight Lines [id:da6764918883827946] 
				\draw    (252,56) -- (196,196) ;
				%Straight Lines [id:da4329647923634008] 
				\draw    (252,56) -- (308,98) ;
				%Straight Lines [id:da011038461362765872] 
				\draw    (308.27,98.2) -- (364,140) ;
				\draw [shift={(308,98)}, rotate = 36.87] [color={rgb, 255:red, 0; green, 0; blue, 0 }  ][line width=0.75]      (0, 0) circle [x radius= 1.34, y radius= 1.34]   ;
				%Straight Lines [id:da5196781175343534] 
				\draw    (308,196) -- (364,140) ;
				%Straight Lines [id:da3160409919917737] 
				\draw    (252,56) -- (252,30) ;
				\draw [shift={(252,28)}, rotate = 90] [color={rgb, 255:red, 0; green, 0; blue, 0 }  ][line width=0.75]    (10.93,-3.29) .. controls (6.95,-1.4) and (3.31,-0.3) .. (0,0) .. controls (3.31,0.3) and (6.95,1.4) .. (10.93,3.29)   ;
				%Straight Lines [id:da9102255389540188] 
				\draw    (364,140) -- (390,140) ;
				\draw [shift={(392,140)}, rotate = 180] [color={rgb, 255:red, 0; green, 0; blue, 0 }  ][line width=0.75]    (10.93,-3.29) .. controls (6.95,-1.4) and (3.31,-0.3) .. (0,0) .. controls (3.31,0.3) and (6.95,1.4) .. (10.93,3.29)   ;
				%Straight Lines [id:da3969763351140396] 
				\draw    (196,196) -- (169.41,222.59) ;
				\draw [shift={(168,224)}, rotate = 315] [color={rgb, 255:red, 0; green, 0; blue, 0 }  ][line width=0.75]    (10.93,-3.29) .. controls (6.95,-1.4) and (3.31,-0.3) .. (0,0) .. controls (3.31,0.3) and (6.95,1.4) .. (10.93,3.29)   ;
				%Straight Lines [id:da06997294598351322] 
				\draw    (196,196) -- (308,196) ;
				%Straight Lines [id:da06948427828035464] 
				\draw    (252,56) -- (308,196) ;
				
				
				\draw (169,226) node [anchor=north west][inner sep=0.75pt]   [align=left] {x};
				
				\draw (380,142) node [anchor=north west][inner sep=0.75pt]   [align=left] {y};
				
				\draw (268,20) node [anchor=north west][inner sep=0.75pt]   [align=left] {z};
				
				\draw (238,44.4) node [anchor=north west][inner sep=0.75pt]   [align=left] {S};
				
				\draw (198,199) node [anchor=north west][inner sep=0.75pt]   [align=left] {D};
				
				\draw (360.2,149) node [anchor=north west][inner sep=0.75pt]   [align=left] {B};
				
				\draw (253,124.6) node [anchor=north west][inner sep=0.75pt]   [align=left] {A};
				
				\draw (309,198) node [anchor=north west][inner sep=0.75pt]   [align=left] {C};
				
				\draw (314.6,82) node [anchor=north west][inner sep=0.75pt]   [align=left] {M};
			\end{tikzpicture}
		\end{center}
		Chọn hệ trục tọa độ như hình vẽ. Ta có $A(0 ; 0 ; 0), D(2 ; 0 ; 0), B(0 ; 4 ; 0), S(0 ; 0 ; 4)$.\\
		$M$ là trung điểm của $S B \Rightarrow M(0 ; 2 ; 2)$.\\
		Tứ giác $A B C D$ là hình chữ nhật nên $\left\{\begin{array}{l}x_{A}+x_{C}=x_{B}+x_{D} \\ y_{A}+y_{C}=y_{B}+y_{D} \\ z_{A}+z_{C}=z_{B}+z_{D}\end{array} \Rightarrow\left\{\begin{array}{l}x_{C}=2 \\ y_{C}=4 \\ z_{C}=0\end{array} \Rightarrow C(2 ; 4 ; 0)\right.\right.$.\\
		$G$ là trọng tâm của tam giác $S C D \Rightarrow G\left(\dfrac{4}{3}; \dfrac{4}{3} ; \dfrac{4}{3}\right)$.\\
		Phương trình mặt phẳng $(AMG)$ qua $A$ và có véc-tơ pháp tuyến $[\overrightarrow{AM},\overrightarrow{AG}]=\left(0;\dfrac{-8}{3};\dfrac{8}{3} \right)$ là $y-z=0.$\\
		Khoảng cách từ điểm B đến mặt phẳng $(AMG)$ là $\mathrm{d}(B,(AMG))= \dfrac{|4|}{\sqrt{1^2+1^2}}=\dfrac{4}{\sqrt{2}}\approx 2{,}8$.
	}
\end{ex}

\begin{ex}%[2H2V2-6]
	Cho hình hộp chữ nhật $ABCD \cdot A'B'C'D'$ có các kích thước $AB=4, AD=3, AA'=5$. Gọi $G$ là trọng tâm của tam giác $ACB'$. Gọi $m$ là khoảng cách từ điểm $G$ đến mặt phẳng $\left(AB'C\right)$ và $n$ là khoảng cách giữa hai mặt phẳng $(AB'D')$ và $(CB'D')$. Tính $m+n$.
	\shortans{$0$}
	\loigiai{
		\begin{center}
			\tikzset{every picture/.style={line width=0.75pt}} %set default line width to 0.75pt        
			\begin{tikzpicture}[x=0.75pt,y=0.75pt,yscale=-1,xscale=1]
				%Straight Lines [id:da09989011048457908] 
				\draw    (250,400) -- (250,327) ;
				\draw [shift={(250,325)}, rotate = 90] [color={rgb, 255:red, 0; green, 0; blue, 0 }  ][line width=0.75]    (10.93,-3.29) .. controls (6.95,-1.4) and (3.31,-0.3) .. (0,0) .. controls (3.31,0.3) and (6.95,1.4) .. (10.93,3.29)   ;
				%Straight Lines [id:da5887885003342179] 
				\draw    (175,575) -- (375,575) ;
				%Straight Lines [id:da38181530618442316] 
				\draw    (375,575) -- (450,500) ;
				%Straight Lines [id:da5520368192051237] 
				\draw    (250,400) -- (175,475) ;
				%Straight Lines [id:da857881333421624] 
				\draw    (175,475) -- (175,575) ;
				%Straight Lines [id:da1055531864002952] 
				\draw    (175,475) -- (375,475) ;
				%Straight Lines [id:da9921967082690255] 
				\draw    (375,475) -- (375,575) ;
				%Straight Lines [id:da830188535856651] 
				\draw    (375,475) -- (450,400) ;
				%Straight Lines [id:da8543854930333936] 
				\draw    (250,400) -- (450,400) ;
				%Straight Lines [id:da010540648916782303] 
				\draw    (450,400) -- (450,500) ;
				%Straight Lines [id:da5916479203423954] 
				\draw  [dash pattern={on 4.5pt off 4.5pt}]  (250,500) -- (175,575) ;
				%Straight Lines [id:da3953089475850997] 
				\draw    (175,575) -- (126.41,623.59) ;
				\draw [shift={(125,625)}, rotate = 315] [color={rgb, 255:red, 0; green, 0; blue, 0 }  ][line width=0.75]    (10.93,-3.29) .. controls (6.95,-1.4) and (3.31,-0.3) .. (0,0) .. controls (3.31,0.3) and (6.95,1.4) .. (10.93,3.29)   ;
				%Straight Lines [id:da9157272689802225] 
				\draw  [dash pattern={on 4.5pt off 4.5pt}]  (250,500) -- (450,500) ;
				%Straight Lines [id:da9303391433588712] 
				\draw    (450,500) -- (523,500) ;
				\draw [shift={(525,500)}, rotate = 180] [color={rgb, 255:red, 0; green, 0; blue, 0 }  ][line width=0.75]    (10.93,-3.29) .. controls (6.95,-1.4) and (3.31,-0.3) .. (0,0) .. controls (3.31,0.3) and (6.95,1.4) .. (10.93,3.29)   ;
				%Straight Lines [id:da1786416964548232] 
				\draw  [dash pattern={on 4.5pt off 4.5pt}]  (250,400) -- (250,500) ;
				\draw (177,478) node [anchor=north west][inner sep=0.75pt]   [align=left] {D'};
				
				\draw (252,403) node [anchor=north west][inner sep=0.75pt]   [align=left] {A'};
				
				\draw (452,403) node [anchor=north west][inner sep=0.75pt]   [align=left] {B'};
				
				\draw (376,481) node [anchor=north west][inner sep=0.75pt]   [align=left] {C'};
				
				\draw (177,578) node [anchor=north west][inner sep=0.75pt]   [align=left] {D};
				
				\draw (252,503) node [anchor=north west][inner sep=0.75pt]   [align=left] {A};
				
				\draw (452,503) node [anchor=north west][inner sep=0.75pt]   [align=left] {B};
				
				\draw (377,578) node [anchor=north west][inner sep=0.75pt]   [align=left] {C};
				
				\draw (513,518) node [anchor=north west][inner sep=0.75pt]   [align=left] {x};
				
				\draw (140,620) node [anchor=north west][inner sep=0.75pt]   [align=left] {y};
				
				\draw (263,331) node [anchor=north west][inner sep=0.75pt]   [align=left] {z};
			\end{tikzpicture}
		\end{center}			
		Chọn hệ trục tọa độ như hình vẽ. Ta có $A(0;0 ; 0), C(4 ; 3 ; 0), B^{\prime}(4 ; 0 ; 5), B(4 ; 0 ; 0); D'(0;3;5).$ $G$ là trọng tâm của tam giác $ACB' \Rightarrow G\left(\dfrac{8}{3} ; 1 ; \dfrac{5}{3}\right).$\\
		Vì $G \in (ACB')$ nên $\mathrm{d}(G, (ACB'))=0.$\\
		Vì hai mặt phẳng $(AB'D')$ và $(CB'D')$ cắt nhau nên khoảng cách của chúng  bằng $0.$ \\
		Vậy $m+n=0$.}
\end{ex}
\Closesolutionfile{ans}
\indapan{6}{ans/ans-0-B15-KQ}

%%%==============Bai_BT1==============%%%
\begin{ex}%[2H5C1-5]
	Cho hình chóp $S.ABCD$ có đáy $ABCD$ là hình vuông cạnh $a$, cạnh bên $SA=a$ và vuông góc với mặt phẳng đáy. Gọi $M$, $N$ lần lượt là trung điểm của $SB$ và $SD$ và $G$ là trọng tâm của tam giác $AMN$.  Biết độ dài đoạn $BG$ có dạng $x\cdot a$. Hỏi giá trị $x$ bằng bao nhiêu? (Kết quả được làm tròn đến hàng phần trăm).
	
	\shortans{$0{,}87$}
	\loigiai{
		\immini{Đặt hệ trục tọa độ $Oxyz$ như hình vẽ. Khi đó\\ 
			$A\equiv O (0;0;0)$, $B\left(a;0;0\right)$, $D\left(0;a;0\right)$, $S\left(0;0;a\right)$.\\ 
			Suy ra $M\left(\dfrac{a}{2};0;\dfrac{a}{2} \right)$ và $N\left(0;\dfrac{a}{2};\dfrac{a}{2} \right)$.\\
			Vì $G$ là trọng tâm của tam giác $AMN$ nên $G\left(\dfrac{a}{6};\dfrac{a}{6};\dfrac{a}{3} \right)$.\\
			Khi đó độ dài đoạn $BG$ là
			$$BG=\sqrt{\left(\dfrac{5a}{6}\right)^2+\left(\dfrac{a}{6}\right)^2+\left(\dfrac{a}{6}\right)^2}=\dfrac{\sqrt{3}}{2}a\approx 0{,}87 a.$$}{\begin{tikzpicture}[scale=0.9]
				\def\a{3.5}
				\def\h{3.5}
				\path 	(0:0) coordinate (A)
				++(0:\a) coordinate (D)
				++(-130:\a/2) coordinate (C)
				($(A)+(C)-(D)$) coordinate (B)
				($(A)+(90:\h)$) coordinate (S)
				(intersection of A--C and B--D) coordinate (O)
				($(S)!0.5!(B)$) coordinate (M)
				($(S)!0.5!(D)$) coordinate (N)
				($(M)!0.5!(N)$) coordinate (I)
				($(A)!2/3!(I)$) coordinate (G);
				\draw[dashed,thick] 	(B)--(A)--(D)	(A)--(S) (A)--(M)--(N)--(A) (B)--(D);
				\draw [-stealth,thick]  (B) -- ($(B)!-1/2!(A)$)node[left, below]{$x$};	
				\draw [-stealth,thick]  (S) -- ($(S)!-1/3!(A)$)node[above]{$z$};
				\draw [-stealth,thick]  (D) -- ($(D)!-1/4!(A)$)node[right]{$y$};
				\draw[thick] 			(B)--(C)--(D)
				(B)--(S)	(C)--(S)	(D)--(S);
				\foreach \x/\g in {A/-65,B/150,C/-45,D/45,S/180,M/150,N/20,G/0}
				\fill[black] 	(\x) circle (1.5pt)
				($(\g:3mm)+(\x)$) node {$\x$};
		\end{tikzpicture}}
	}
\end{ex}
%%%==============HetBai_BT1==============%%%

%%%==============Bai_BT2==============%%%
\begin{ex}%[2H5C1-5]
	Cho hình chóp $S.ABCD$ có đáy $ABCD$ là hình vuông cạnh $a$, cạnh bên $SA=a$ và vuông góc với mặt phẳng đáy. Gọi $M$, $N$ lần lượt là trung điểm của $SB$ và $SD$ và $G$ là trọng tâm của tam giác $AMN$. Khoảng cách từ điểm $G$ đến mặt phẳng $\left(SBC\right)$ là bao nhiêu nếu $a=6\sqrt{3}$?
	
	\shortans{$2$}
	\loigiai{
		\immini{Chọn hệ trục tọa độ $Oxyz$ thỏa mãn:\\ 
			$A\equiv O(0;0;0)$, $B\left(a;0;0\right)$, $D\left(0;a;0\right)$, $S\left(0;0;a\right)$.\\ 
			Do đó $C(a;a;0)$.\\ 
			Suy ra $M\left(\dfrac{a}{2};0;\dfrac{a}{2} \right)$ và $N\left(0;\dfrac{a}{2};\dfrac{a}{2} \right)$.\\
			Vì $G$ là trọng tâm của tam giác $AMN$ nên $G\left(\dfrac{a}{6};\dfrac{a}{6};\dfrac{a}{3} \right)$.\\
			Phương trình mặt phẳng $(SBD)$ là 
			$$\dfrac{x}{a}+\dfrac{y}{a}+\dfrac{z}{a}=1.$$
			Do đó khoảng cách từ $G$ đến mặt phẳng $\left(SBC\right)$ là
			$$\mathrm{d}\left(G,(SBD)\right)=\dfrac{\left| \dfrac{1}{a}\cdot\dfrac{a}{6}+\dfrac{1}{a}\cdot\dfrac{a}{6}+\dfrac{1}{a}\cdot\dfrac{a}{3}-1\right|}{\sqrt{\left(\dfrac{1}{a}\right)^2+\left(\dfrac{1}{a}\right)^2+\left(\dfrac{1}{a}\right)^2}}=\dfrac{a}{3\sqrt{3}}=2.$$}{\begin{tikzpicture}[scale=0.9]
				\def\a{3.5}
				\def\h{3.5}
				\path 	(0:0) coordinate (A)
				++(0:\a) coordinate (D)
				++(-130:\a/2) coordinate (C)
				($(A)+(C)-(D)$) coordinate (B)
				($(A)+(90:\h)$) coordinate (S)
				(intersection of A--C and B--D) coordinate (O)
				($(S)!0.5!(B)$) coordinate (M)
				($(S)!0.5!(D)$) coordinate (N)
				($(M)!0.5!(N)$) coordinate (I)
				($(A)!2/3!(I)$) coordinate (G);
				\draw[dashed,thick] 	(B)--(A)--(D)	(A)--(S) (A)--(M)--(N)--(A) (B)--(D);
				\draw [-stealth,thick]  (B) -- ($(B)!-1/2!(A)$)node[left, below]{$x$};	
				\draw [-stealth,thick]  (S) -- ($(S)!-1/3!(A)$)node[above]{$z$};
				\draw [-stealth,thick]  (D) -- ($(D)!-1/4!(A)$)node[right]{$y$};
				\draw[thick] 			(B)--(C)--(D)
				(B)--(S)	(C)--(S)	(D)--(S);
				\foreach \x/\g in {A/-65,B/150,C/-45,D/45,S/180,M/150,N/20,G/10}
				\fill[black] 	(\x) circle (1.5pt)
				($(\g:3mm)+(\x)$) node {$\x$};
		\end{tikzpicture}}
	}
\end{ex}
%%%==============HetBai_BT2==============%%%

%%%==============Bai_BT3==============%%%
\begin{ex}%[2H5C1-5]
	Cho hình chóp $S.ABCD$ có đáy $ABCD$ là hình vuông cạnh $a$, cạnh bên $SA=a$ và vuông góc với mặt phẳng đáy. Gọi $M$, $N$ lần lượt là trung điểm của $SB$ và $SD$ và $G$ là trọng tâm của tam giác $AMN$. Tính khoảng cách từ điểm $C$ đến mặt phẳng $\left(AMN\right)$ biết $a=\sqrt{3}$.
	
	\shortans{$2$}
	\loigiai{
		\immini{Chọn hệ trục tọa độ $Oxyz$ thỏa mãn: $A\equiv O$, $B\left(a;0;0\right)$, $D\left(0;a;0\right)$, $S\left(0;0;a\right)$. Do đó $C(a;a;0)$.\\ 
			Suy ra $M\left(\dfrac{a}{2};0;\dfrac{a}{2} \right)$ và $N\left(0;\dfrac{a}{2};\dfrac{a}{2} \right)$.\\
			Vì $G$ là trọng tâm của tam giác $AMN$ nên $G\left(\dfrac{a}{6};\dfrac{a}{6};\dfrac{a}{3} \right)$.\\
			Ta có $AC$ là hình chiếu vuông góc của $SC$ lên mặt phẳng $(ABCD)$. Mà $AC \perp BD$ nên $SC \perp BD$.\\
			Hơn nữa vì $MN \parallel BD$ (tính chất đường trung bình) nên $SC \perp MN$. \quad(1)\\
			Lại có do $\triangle SAB$ cân tại $A$ có $M$ là trung điểm $SB$ nên $AM \perp SB$.\\
			Hơn nữa vì $BC \perp (SAB)$ nên $BC \perp AM$.\\ 
			Do đó $AM \perp (SBC)$.\\
			Suy ra $AM \perp SC$. \quad(2)}{\begin{tikzpicture}[scale=1]
				\def\a{3.5}
				\def\h{3.5}
				\path 	(0:0) coordinate (A)
				++(0:\a) coordinate (D)
				++(-130:\a/2) coordinate (C)
				($(A)+(C)-(D)$) coordinate (B)
				($(A)+(90:\h)$) coordinate (S)
				(intersection of A--C and B--D) coordinate (O)
				($(S)!0.5!(B)$) coordinate (M)
				($(S)!0.5!(D)$) coordinate (N)
				($(M)!0.5!(N)$) coordinate (I)
				($(A)!2/3!(I)$) coordinate (G);
				\draw[dashed,thick] 	(B)--(A)--(D)	(A)--(S) (A)--(M)--(N)--(A) (B)--(D);
				\draw [-stealth,thick]  (B) -- ($(B)!-1/2!(A)$)node[left, below]{$x$};	
				\draw [-stealth,thick]  (S) -- ($(S)!-1/3!(A)$)node[above]{$z$};
				\draw [-stealth,thick]  (D) -- ($(D)!-1/4!(A)$)node[right]{$y$};
				\draw[thick] 			(B)--(C)--(D)
				(B)--(S)	(C)--(S)	(D)--(S);
				\foreach \x/\g in {A/-65,B/150,C/-45,D/45,S/180,M/150,N/20,G/0}
				\fill[black] 	(\x) circle (1.5pt)
				($(\g:3mm)+(\x)$) node {$\x$};
		\end{tikzpicture}}
		\noindent Từ (1) và (2) ta có $SC \perp (AMN)$, hay $\overrightarrow{SC}$ là véc-tơ pháp tuyến của mặt phẳng $(AMN)$.\\ 
		Hay mặt phẳng $(AMN)$ có một véc-tơ pháp tuyến $\overrightarrow{n} = (1;1;-1)$.\\
		Phương trình mặt phẳng $(AMN)$ là 
		$$x+y-z=0.$$
		Do đó khoảng cách từ $C$ đến mặt phẳng $\left(AMN\right)$ là
		$$\mathrm{d}\left(C,(AMN)\right)=\dfrac{\left| a+a-0\right|}{\sqrt{1^2+1^2+\left(-1\right)^2}}=\dfrac{2a}{\sqrt{3}}=2.$$
	}
\end{ex}
%%%==============HetBai_BT3==============%%%

%%%==============Bai_BT4==============%%%
\begin{ex}%[2H5C1-5]
	Cho hình chóp $S.ABCD$ có đáy $ABCD$ là hình chữ nhật, $AB=a$, $BC=a\sqrt{3} $, $SA=a$ và $SA$ vuông góc với đáy $ABCD$. Tính khoảng cách từ điểm $C$ đến mặt phẳng $\left(SBD\right)$ biết $a=\sqrt{21}$.
	
	\shortans{$6$}
	\loigiai{
		\immini{
			Đặt hệ trục tọa độ $Oxyz$ như hình vẽ.\\ 
			Khi đó, ta có
			\[A\left(0;0;0\right), B\left(a;0;0\right), C\left(a;a\sqrt{3} ;0\right), D\left(0;a\sqrt{3} ;0\right), S\left(0;0;a\right).\] 
			Phương trình mặt phẳng $(SBD)$ là
			$$\dfrac{x}{a}+\dfrac{y}{a\sqrt{3}} + \dfrac{z}{a}=1.$$
			Do đó khoảng cách từ $C$ đến mặt phẳng $\left(SBD\right)$ là
			$$\mathrm{d}\left(C,(SBD)\right)=\dfrac{\left| \dfrac{1}{a}\cdot a+\dfrac{1}{a\sqrt{3}}\cdot a\sqrt{3}+\dfrac{1}{a}\cdot0\right|}{\sqrt{\left(\dfrac{1}{a}\right)^2+\left(\dfrac{1}{a\sqrt{3}}\right)^2+\left(\dfrac{1}{a}\right)^2}}=\dfrac{2\sqrt{21}a}{7}=6.$$}{\begin{tikzpicture}[scale=0.85]
				\def\a{3.5}
				\def\h{\a}
				\path 	(0:0) coordinate (A)
				++(0:\a) coordinate (D)
				++(-130:\a/2) coordinate (C)
				($(A)+(C)-(D)$) coordinate (B)
				($(A)+(90:\h)$) coordinate (S)
				(intersection of A--C and B--D) coordinate (O)
				($(S)!2/3!(O)$) coordinate (G);
				\draw[dashed,thick] 	(B)--(A)--(D)	(A)--(S) (B)--(D) (A)--(C);
				\draw [-stealth,thick]  (B) -- ($(B)!-1/2!(A)$)node[left]{$x$};	
				\draw [-stealth,thick]  (S) -- ($(S)!-1/3!(A)$)node[above]{$z$};
				\draw [-stealth,thick]  (D) -- ($(D)!-1/4!(A)$)node[right]{$y$};
				\draw[thick] 			(B)--(C)--(D) (B)--(S)	(C)--(S)	(D)--(S);
				\foreach \x/\g in {A/180,B/150,C/-45,D/45,S/180}
				\fill[black] 	(\x) circle (1.5pt) ($(\g:3mm)+(\x)$) node {$\x$};
		\end{tikzpicture}}
	}
\end{ex}
%%%==============HetBai_BT4==============%%%

%%%==============Bai_BT5==============%%%
\begin{ex}%[2H5C1-5]
	Cho hình chóp $S.ABCD$ có đáy $ABCD$ là hình chữ nhật, $AB=a$, $BC=a\sqrt{3} $, $SA=a$ và $SA$ vuông góc với đáy $ABCD$. Gọi $G$ là trọng tâm của tam giác $SBD$. Tính khoảng cách từ điểm $G$ đến mặt phẳng $\left(SCD\right)$ biết $a=\sqrt{3}$.
	
	\shortans{$0{,}5$}
	\loigiai{
		\immini{ Đặt hệ trục tọa độ $Oxyz$ như hình vẽ.\\ 
			Khi đó, ta có
			\[A\left(0;0;0\right), B\left(a;0;0\right), C\left(a;a\sqrt{3} ;0\right), D\left(0;a\sqrt{3} ;0\right), S\left(0;0;a\right).\] 
			$G$ là trọng tâm của tam giác $SBD$ $\Rightarrow G\left(\dfrac{a}{3} ;\dfrac{a\sqrt{3} }{3} ;\dfrac{a}{3} \right)$.\\
			Gọi phương trình mặt phẳng $(SCD)$ có dạng
			$$Ax+By+Cz+D=0.$$
			Vì $S,C,D\in (SCD)$ nên ta có hệ
		}{\begin{tikzpicture}[scale=0.9]
				\def\a{3.5}
				\def\h{\a}
				\path 	(0:0) coordinate (A)
				++(0:\a) coordinate (D)
				++(-130:\a/2) coordinate (C)
				($(A)+(C)-(D)$) coordinate (B)
				($(A)+(90:\h)$) coordinate (S)
				(intersection of A--C and B--D) coordinate (O)
				($(S)!2/3!(O)$) coordinate (G);
				\draw[dashed,thick] 	(B)--(A)--(D)	(A)--(S) (B)--(D) (A)--(C) (S)--(O);
				\draw [-stealth,thick]  (B) -- ($(B)!-1/2!(A)$)node[left, below]{$x$};	
				\draw [-stealth,thick]  (S) -- ($(S)!-1/3!(A)$)node[above]{$z$};
				\draw [-stealth,thick]  (D) -- ($(D)!-1/4!(A)$)node[right]{$y$};
				\draw[thick] 			(B)--(C)--(D) (B)--(S)	(C)--(S)	(D)--(S);
				\foreach \x/\g in {A/180,B/150,C/-45,D/45,S/180,G/180,O/-90}
				\fill[black] 	(\x) circle (1.5pt) ($(\g:3mm)+(\x)$) node {$\x$};
		\end{tikzpicture}}
		$$\heva{
			& 	Ca		+D	=0\\
			&Aa 	+a\sqrt{3}B +D =0\\
			&a\sqrt{3}B +D=0
		} \Leftrightarrow \heva{&A=0\\&C=B\sqrt{3}\\
			&Ca+D=0.}$$
		Vì vậy phương trình mặt phẳng $(SCD)$ là
		$$y+\sqrt{3}z-a\sqrt{3}=0.$$
		Vậy khoảng cách từ $G$ đến mặt phẳng $\left(SCD\right)$ là
		$$\mathrm{d}\left(C,(SBD)\right)=\dfrac{\left|\dfrac{a\sqrt{3}}{3}+ \dfrac{a\sqrt{3}}{3}-a\sqrt{3}\right|}{\sqrt{1^2+\left(\sqrt{3}\right)^2}}=\dfrac{a\sqrt{3}}{6}=0{,}5.$$
	}
\end{ex}
%%%==============HetBai_BT5==============%%%

%%%==============Bai_BT6==============%%%
\begin{ex}%[2H5C1-5]
	Cho hình chóp $S.ABCD$ có đáy $ABCD$ là hình vuông tâm $I$, có độ dài đường chéo bằng $a\sqrt{2} $ và $SA$ vuông góc với mặt phẳng $\left(ABCD\right)$. Gọi $\alpha $ là góc giữa hai mặt phẳng $\left(SBD\right)$ và $\left(ABCD\right)$ và $\tan \alpha =\sqrt{2} $. Khoảng cách từ điểm $I$ đến mặt phẳng $\left(SAB\right)$ có dạng $x\cdot a$. Tìm giá trị của $x$.
	
	\shortans{$0{,}5$}
	\loigiai{
		\immini{ Hình vuông $ABCD$ có độ dài đường chéo bằng $a\sqrt{2} $ suy ra hình vuông đó có cạnh bằng $a$.\\
			Ta có 
			$\heva{&\left(SBD\right)\cap \left(ABCD\right)=BD \\ &{SI\bot BD} \\ &{AI\bot BD} } $\\
			$\Rightarrow {\left(\left(SBD\right); \left(ABCD\right)\right)}={\left(SI; AI\right)}=\widehat{SIA}$
			Ta có $\tan \alpha =\tan \widehat{SIA}=\dfrac{SA}{AI} \Leftrightarrow SA=a$.\\
			Ta xét hệ trục tọa độ $Oxyz$ như hình vẽ với\\ 
			$A\left(0; 0; 0\right)$, $B\left(a; 0; 0\right)$, $C\left(a; a; 0\right)$, $D(0;a;0)$, $S\left(0; 0; a\right)$.\\
			Suy ra $I\left(\dfrac{a}{2} ;\dfrac{a}{2};0 \right)$.\\
			Phương trình mặt phẳng $(SAB)$ là $y=0$.\\
			Vì vậy khoảng cách từ $I$ đến mặt phẳng $\left(SAB\right)$ là $\dfrac{a}{2}=0{,}5a$.}{\begin{tikzpicture}[scale=1]
				\def\a{3.5}
				\def\h{\a}
				\path 	(0:0) coordinate (A)
				++(0:\a) coordinate (D)
				++(-130:\a/2) coordinate (C)
				($(A)+(C)-(D)$) coordinate (B)
				($(A)+(90:\h)$) coordinate (S)
				(intersection of A--C and B--D) coordinate (I);
				\draw[dashed,thick] 	(B)--(A)--(D)	(A)--(S) (B)--(D) (A)--(C) (S)--(I);
				\draw [-stealth,thick]  (B) -- ($(B)!-1/2!(A)$)node[left, below]{$x$};	
				\draw [-stealth,thick]  (S) -- ($(S)!-1/3!(A)$)node[above]{$z$};
				\draw [-stealth,thick]  (D) -- ($(D)!-1/4!(A)$)node[right]{$y$};
				\draw[thick] 			(B)--(C)--(D) (B)--(S)	(C)--(S)	(D)--(S);
				\foreach \x/\g in {A/180,B/150,C/-45,D/45,S/180,I/-90}
				\fill[black] 	(\x) circle (1.5pt) ($(\g:3mm)+(\x)$) node {$\x$};
		\end{tikzpicture}}
	}
\end{ex}
%%%==============HetBai_BT6==============%%%

%%%==============Bai_BT7==============%%%
\begin{ex}%[2H5C1-5]
	Cho hình chóp $S.ABCD$ có đáy $ABCD$ là hình vuông tâm $I$, có độ dài đường chéo bằng $a\sqrt{2} $ và $SA$ vuông góc với mặt phẳng $\left(ABCD\right)$. Gọi $\alpha $ là góc giữa hai mặt phẳng $\left(SBD\right)$ và $\left(ABCD\right)$ và $\tan \alpha =\sqrt{2} $. Tính khoảng cách từ điểm $I$ đến mặt phẳng $\left(SCD\right)$ biết $a=2\sqrt{2}$.
	
	\shortans{$1$}
	\loigiai{
		\immini{ Hình vuông $ABCD$ có độ dài đường chéo bằng $a\sqrt{2} $ suy ra hình vuông đó có cạnh bằng $a$.\\
			Ta có 
			$\heva{&\left(SBD\right)\cap \left(ABCD\right)=BD \\ &{SI\bot BD} \\ &{AI\bot BD} }$\\
			$ \Rightarrow {\left(\left(SBD\right); \left(ABCD\right)\right)}={\left(SI; AI\right)}=\widehat{SIA}.$
			Ta có $\tan \alpha =\tan \widehat{SIA}=\dfrac{SA}{AI} \Leftrightarrow SA=a$.\\
			Ta có $A\left(0; 0; 0\right)$, $B\left(a; 0; 0\right)$, $C\left(a; a; 0\right)$, $D(0;a;0)$, $S\left(0; 0; a\right)\Rightarrow I\left(\dfrac{a}{2} ;\dfrac{a}{2};0 \right)$.\\
			Phương trình mặt phẳng $(SCD)$ có dạng
			$$Ax+By+Cz+D=0.$$}{\begin{tikzpicture}[scale=1]
				\def\a{3.5}
				\def\h{\a}
				\path 	(0:0) coordinate (A)
				++(0:\a) coordinate (D)
				++(-130:\a/2) coordinate (C)
				($(A)+(C)-(D)$) coordinate (B)
				($(A)+(90:\h)$) coordinate (S)
				(intersection of A--C and B--D) coordinate (I);
				\draw[dashed,thick] 	(B)--(A)--(D)	(A)--(S) (B)--(D) (A)--(C) (S)--(I);
				\draw [-stealth,thick]  (B) -- ($(B)!-1/2!(A)$)node[left, below]{$x$};	
				\draw [-stealth,thick]  (S) -- ($(S)!-1/3!(A)$)node[above]{$z$};
				\draw [-stealth,thick]  (D) -- ($(D)!-1/4!(A)$)node[right]{$y$};
				\draw[thick] 			(B)--(C)--(D) (B)--(S)	(C)--(S)	(D)--(S);
				\foreach \x/\g in {A/180,B/150,C/-45,D/45,S/180,I/-90}
				\fill[black] 	(\x) circle (1.5pt) ($(\g:3mm)+(\x)$) node {$\x$};
		\end{tikzpicture}}
		\noindent Vì $S,C,D\in (SCD)$ nên ta có hệ
		$$\heva{
			& 	Ca		+D	=0\\
			&aA 	aB+D =0\\
			&aB +D=0
		} \Leftrightarrow \heva{&A=0\\&C=B\\
			&Ca+D=0.}$$
		Vì vậy phương trình mặt phẳng $(SCD)$ là
		$$y+z-a=0.$$
		Vậy khoảng cách từ $I$ đến mặt phẳng $\left(SCD\right)$ là
		$$\mathrm{d}\left(C,(SCD)\right)=\dfrac{\left|\dfrac{a}{2}+0-a\right|}{\sqrt{1^2+1^2}}=\dfrac{a\sqrt{2}}{4}=1.$$
	}
\end{ex}
%%%==============HetBai_BT7==============%%%

%%%==============Bai_BT8==============%%%
\begin{ex}%[2H5C1-5]
	Cho hình chóp $S.ABCD$ có đáy $ABCD$ là hình vuông cạnh $a$, mặt bên $SAB$ là tam giác đều và nằm trong mặt phẳng vuông góc với mặt phẳng $\left(ABCD\right)$. Tính khoảng cách từ điểm $A$ đến mặt phẳng $\left(SBD\right)$ biết $a=\sqrt{21}$.
	
	\shortans{$3$}
	\loigiai{
		\begin{center}
			\begin{tikzpicture}
				\def\a{4}
				\def\h{4.5}
				\path 	(0:0) coordinate (A)
				++(0:\a) coordinate (D)
				++(-130:\a/2) coordinate (C)
				($(A)+(C)-(D)$) coordinate (B)
				($(A)!0.5!(B)$) coordinate (H)
				($(C)!0.5!(D)$) coordinate (K)
				($(H)+(90:\h)$) coordinate (S)
				(intersection of A--C and B--D) coordinate (O);%giao điểm O
				\draw[dashed,thick] 	(B)--(A)--(D)	(A)--(S) (S)--(H)	(H)--(K);
				\draw [-stealth,thick]  (B) -- ($(B)!-1/2!(A)$)node[left, below]{$x$};	
				\draw [-stealth,thick]  (S) -- ($(S)!-1/3!(H)$)node[above]{$z$};
				\draw [-stealth,thick]  (K) -- ($(K)!-1/3!(H)$)node[right]{$y$};
				\draw[thick] 			(B)--(C)--(D)
				(B)--(S)	(C)--(S)	(D)--(S);
				\foreach \x/\g in {A/45,B/185,C/-45,D/45,S/180}
				\fill[black] 	(\x) circle (1.5pt)
				($(\g:4mm)+(\x)$) node {$\x$};
				\fill[black] 	(H) circle (1.5pt)
				($(-30:5mm)+(H)$) node {\footnotesize{$H\equiv O$}};
			\end{tikzpicture}
		\end{center}
		Chọn hệ trục tọa độ $Oxyz$ như hình vẽ. Khi đó
		\[S\left(0; 0; \dfrac{a\sqrt{3} }{2} \right); A\left(\dfrac{-a}{2} ;0;0\right); B\left(\dfrac{a}{2} ;0; 0\right);C\left(\dfrac{a}{2} ;a; 0\right); D\left(\dfrac{-a}{2} ;a; 0\right).\] 
		Phương trình mặt phẳng $(SBD)$ có dạng
		$$Ax+By+Cz+D=0.$$
		Vì $S,B,D\in (SBD)$ nên ta có hệ
		$$\heva{
			& \dfrac{a\sqrt{3} }{2}C		+D	=0\\
			&\dfrac{a}{2} A 	  +D=0\\
			&-\dfrac{a}{2} A  +aB +D=0
		} \Leftrightarrow \heva{&A=-\dfrac{2}{a}D\\ &B=-\dfrac{2}{a} D\\
			&C=-\dfrac{2\sqrt{3}}{3a}D.}$$
		Vì vậy phương trình mặt phẳng $(SBD)$ là
		$$x+y+\dfrac{\sqrt{3}}{3}z-\dfrac{a}{2}=0.$$
		Vậy khoảng cách từ $A$ đến mặt phẳng $\left(SBD\right)$ là
		$$\mathrm{d}\left(A,(SBD)\right)=\dfrac{\left|-\dfrac{a}{2}-\dfrac{a}{2}\right|}{\sqrt{1^2+1^2+\left(\dfrac{\sqrt{3}}{3}\right)^2 }}=\dfrac{a\sqrt{21}}{7}=3.$$
	}
\end{ex}
%%%==============HetBai_BT8==============%%%

%%%==============Bai_BT9==============%%%
\begin{ex}%[2H5C1-5]  
	Cho hình chóp $S.ABCD$ có đáy $ABCD$ là hình vuông cạnh $a$, mặt bên $SAB$ là tam giác đều và nằm trong mặt phẳng vuông góc với mặt phẳng $\left(ABCD\right)$. Gọi $G$ là trọng tâm của tam giác $SAB$ và $M$, $N$ lần lượt là trung điểm của $SC$, $SD$. Tính khoảng cách từ điểm $S$ đến mặt phẳng $\left(GMN\right)$ biết $a=\sqrt{14}$.
	
	\shortans{$2$}
	\loigiai{\;
		\begin{center}
			\begin{tikzpicture}[scale=1]
				\def\a{4}
				\def\h{4.5}
				\path 	(0:0) coordinate (A)
				++(0:\a) coordinate (D)
				++(-130:\a/2) coordinate (C)
				($(A)+(C)-(D)$) coordinate (B)
				($(A)!0.5!(B)$) coordinate (H)
				($(C)!0.5!(D)$) coordinate (K)
				($(H)+(90:\h)$) coordinate (S)
				(intersection of A--C and B--D) coordinate (O)
				($(S)!2/3!(H)$) coordinate (G)
				($(S)!1/2!(D)$) coordinate (N)
				($(S)!1/2!(C)$) coordinate (M);
				\draw[dashed,thick] 	(B)--(A)--(D)	(A)--(S) (S)--(H)	(H)--(K) (G)--(N);
				\draw [-stealth,thick]  (B) -- ($(B)!-1/2!(A)$)node[left, below]{$x$};	
				\draw [-stealth,thick]  (S) -- ($(S)!-1/3!(H)$)node[above]{$z$};
				\draw [-stealth,thick]  (K) -- ($(K)!-1/3!(H)$)node[right]{$y$};
				\draw[thick] 			(B)--(C)--(D)
				(B)--(S)	(C)--(S)	(D)--(S) (G)--(M)--(N);
				\foreach \x/\g in {A/45,B/185,C/-45,D/45,S/180,M/-10,N/0,G/-45}
				\fill[black] 	(\x) circle (1.5pt)
				($(\g:4mm)+(\x)$) node {$\x$};
				\fill[black] 	(H) circle (1.5pt)
				($(-30:5mm)+(H)$) node {\footnotesize{$H\equiv O$}};
			\end{tikzpicture}
		\end{center}
		Chọn hệ trục tọa độ $Oxyz$ như hình vẽ.\\ 
		Khi đó
		$S\left(0; 0; \dfrac{a\sqrt{3} }{2} \right)$, $A\left(\dfrac{-a}{2} ;0;0\right)$, $ B\left(\dfrac{a}{2} ;0; 0\right)$, $C\left(\dfrac{a}{2} ;a; 0\right)$ và $D\left(\dfrac{-a}{2} ;a; 0\right)$.\\
		Suy ra $G\left(0; 0; \dfrac{a\sqrt{3} }{6} \right)$, $M\left(\dfrac{a}{4} ;\dfrac{a}{2} ; \dfrac{a\sqrt{3} }{4} \right)$, $N\left(-\dfrac{a}{4} ;\dfrac{a}{2} ; \dfrac{a\sqrt{3} }{4} \right)$.\\
		Phương trình mặt phẳng $(GMN)$ có dạng
		$$Ax+By+Cz+D=0.$$
		Vì $G,M,N\in (GMN)$ nên ta có hệ
		$$\heva{
			& 	\dfrac{a\sqrt{3} }{6}C		+D	=0\\
			&\dfrac{a}{4} A 	+\dfrac{a}{2}B + \dfrac{a\sqrt{3} }{4}C +D=0\\
			&-\dfrac{a}{4} A  +\dfrac{a}{2} B+\dfrac{a\sqrt{3} }{4}C +D=0
		} \Leftrightarrow \heva{&A=0\\&B=\dfrac{1}{a} D\\
			&C=-\dfrac{2\sqrt{3}}{a} D.}$$
		Vì vậy phương trình mặt phẳng $(GMN)$ là
		$$y-2\sqrt{3}z+a=0.$$
		Vậy khoảng cách từ $S$ đến mặt phẳng $\left(GMN\right)$ là
		$$\mathrm{d}\left(S,(GMN)\right)=\dfrac{\left|-2a\right|}{\sqrt{1^2+1^2+\left(-2\sqrt{3}\right)^2 }}=\dfrac{2a\sqrt{14}}{14}=2.$$
	}
\end{ex}
\Closesolutionfile{ans}
\indapan{6}{ans/ans-0-B15-KQ}
%%%==============HetBai_BT9==============%%%
%%Bài 2.
% \foreach \i in {1,2,...,7} {\input{data/12/C5B2/C5B2CD\i.tex}}
%%Bài 3.
% \foreach \i in {1,2,3,4,7} {\input{data/12/C5B3/C5B3CD\i.tex}}
%%%%%12C
%%Bài 1.
\setcounter{section}{13}
\setcounter{dang}{0}
\section{PHƯƠNG TRÌNH MẶT PHẲNG}
\subsection{LÝ THUYẾT CẦN NHỚ}
\subsubsection{Vectơ pháp tuyến của mặt phẳng}
\begin{itemize}
	\immini{\item [\iconMT] \indam{Định nghĩa:} Vectơ pháp tuyến $\vec{n}$ của mặt phẳng $(P)$ là những vectơ khác $\vec{0}$ và có giá vuông góc với $(P)$. 
		\item [\iconMT] \indam{Chú ý:} 
		\begin{boxdn}
			\begin{itemize}
				\item [$\bullet$] $\vec{n} \ne \vec{0}$ và có giá vuông với $(P)$;
				\item [$\bullet$] Nếu $\vec{n}$ và $\vec{n'}$ cùng là vectơ pháp tuyến của $(P)$ thì $\vec{n'} = k \cdot \vec{n}$ (tọa độ tỉ lệ nhau).
			\end{itemize}
		\end{boxdn}
	}{
		\begin{tikzpicture}[scale=0.8, line join=round, line cap=round,>=stealth]
			\tkzDefPoints{0/0/A,4/0/B,5/2/C}
			\coordinate (D) at ($(A)+(C)-(B)$);
			\tkzDrawPolygon(A,B,C,D)
			\tkzMarkAngles[size=0.7cm,arc=l](B,A,D)
			\tkzLabelAngles[pos=0.5,rotate=10](B,A,D){\scriptsize$P$}
			\draw[->] (2,1)--(2,2.5)node[right]{\scriptsize$\vec{n}$};
			\draw[->] (3,1.5)--(3,3)node[right]{\scriptsize$\vec{n'}$};
	\end{tikzpicture}}
\end{itemize}

\subsubsection{Cặp vectơ chỉ phương của mặt phẳng}
\begin{itemize}
	\item [\iconMT] \indam{Định nghĩa:} Trong không gian $Oxyz$, cho hai vectơ $\vec u$, $\vec v$ được gọi là cặp vectơ chỉ phương của mặt phẳng $(P)$ nếu chúng không cùng phương và có giá nằm trong hoặc song song với mặt phẳng $(P)$.
	\item [\iconMT] \indam{Chú ý:} 
	\begin{boxdn}
\immini{		\begin{itemize}
			\item [$\bullet$] Cho hai vectơ $\vec u = (a; b; c)$ và $\vec v = (a'; b'; c')$. Khi đó 
			$$\vec n = (bc' - b'c;ca' - c'a; ab' - a'b)$$
			vuông góc với cả hai vectơ $\vec u$ và $\vec v$, được gọi là tích có hướng của $\vec u$ và $\vec v$, ký hiệu là $[\vec u, \vec v]$.
			\item [$\bullet$] Nếu $\vec u$, $\vec v$ là cặp vectơ chỉ phương của $(P)$ thì $[\vec u,\vec v]$ là một vectơ pháp tuyến của $(P)$.
		\end{itemize}}{
	\begin{tikzpicture}[>=stealth, line join=round, line cap = round,scale=0.8]
	\def\d{4}
	\def\r{3}
	\path (0:0) coordinate (B)
	++(0:\d) coordinate (C)
	++(50:\r) coordinate (D)
	($(B)+(D)-(C)$) coordinate (A)
	(2.5,2) coordinate (M)
	(3.5,2) coordinate (N)
	(3.,1.6) coordinate (P)
	;
	\draw[->] (M)--($(M)+(-130:1.5)$) node[pos=0.4,left] {$\vec u$};
	\draw[->] (N)--($(N)+(0:1.8)$) node[pos=0.4,below] {$\vec v$};
	\draw[->] (P)--($(P)+(90:1.8)$) node[pos=0.9,right] {${[\vec u,\vec v]}$};
	\draw (A)--(B)--(C)--(D)--cycle;
	\begin{scope}
		\clip (A)--(B)--(C);
		\draw[opacity=0.7] (B) circle(0.8cm)node[black,shift={(25:4mm)}]{$P$};
	\end{scope}
	%		\foreach \x/ \goc in {A/180,B/180,C/0,D/0} 
	%		\fill (\x) circle (1pt) ($(\x)+(\goc:3mm)$) node {$\x$};
\end{tikzpicture}}
	\end{boxdn}

\end{itemize}
\subsubsection{Phương trình tổng quát của mặt phẳng}
\begin{itemize}
	\item [\iconMT] \indam{Công thức:} Mặt phẳng $(P)$ đi qua điểm $M(x_0;y_0;z_0)$ và nhận $\vec{n}=(a;b;c)$ làm vectơ pháp tuyến có phương trình là 
	\boxmini{$a(x-x_0)+b(y-y_0)+c(z-z_0)=0$}
	Thu gọn ta được dạng 
	$$ax+by+cz+d=0$$
	\item [\iconMT] \indam{Chú ý:}
	\begin{boxdn}
		\begin{itemize}
			\item [\ding{172}] Phương trình các mặt phẳng tọa độ: 
			\begin{listEX}[2]
				\item [$\bullet$] $(Oxy) \colon z=0$.
				\item [$\bullet$] $(Oxz) \colon y=0$.
				\item [$\bullet$] $(Oyz) \colon x=0$.
			\end{listEX}
			\item [\ding{173}] Phương trình mặt phẳng $(\alpha)$ song song với mặt phẳng tọa độ: 
			\begin{listEX}[2]
				\item [$\bullet$] $(\alpha) \parallel (Oxy) \Rightarrow z=a \quad a \ne 0$.
				\item [$\bullet$] $(\alpha) \parallel (Oxz) \Rightarrow y=b \quad b \ne 0 $.
				\item [$\bullet$] $(\alpha) \parallel (Oyz) \Rightarrow x=c \quad c \ne 0$.
			\end{listEX}
		\end{itemize}
	\end{boxdn}
\end{itemize}

\subsubsection{Vị trị tương đối giữa hai mặt phẳng}
\begin{itemize}
	\item [] Cho hai mặt phẳng $(P) \colon a_1x + b_1y + c_1z + d_1=0$ và $(Q) \colon a_2x + b_2y + c_2z + d_2=0$. \\
	Gọi $\vec{n_1}=(a_1;b_1;c_1)$, $\vec{n_2}=(a_2;b_2;c_2)$ lần lượt là vectơ pháp tuyến của $(P)$ và $(Q)$.\\
	\begin{boxdn}
		\begin{listEX}[1]
			\item [\ding{172}] Nếu $\heva{&\vec{n_1}= k \cdot \vec{n_2}\\& d_1 =k\cdot d_2}$ thì $(P)$ trùng $(Q)$.
			\item [\ding{173}] Nếu $\heva{&\vec{n_1}= k \cdot \vec{n_2}\\& d_1 \ne k\cdot d_2}$ thì $(P)$ song song $(Q)$.
			\item [\ding{174}] Nếu $\vec{n_1}$ không cùng phương với $\vec{n_2}$ thì $(P)$ cắt $(Q)$.
			\item [\ding{175}] Nếu $\vec{n_1} \perp \vec{n_2}$ hay $a_1a_2+b_1b_2+c_1c_2=0$ thì $(P) \perp (Q)$.
		\end{listEX}
	\end{boxdn}  
\end{itemize}

\subsubsection{Khoảng cách từ một điểm đến mặt phẳng}
\begin{itemize}
	\immini{\item [\iconMT] \indam{Định nghĩa:} Cho điểm $M(x_0;y_0;z_0)$ và mặt phẳng $(P) \colon ax+by+cz+d=0$. Gọi $H$ là hình chiếu vuông góc của điểm $M$ lên mặt phẳng $(P)$. Khi đó độ dài đoạn $MH$ được gọi là khoảng cách từ điểm $M$ đến $(P)$. Kí hiệu $\mathrm{d}\left(M,(P) \right)$.
		\item [\iconMT] \indam{Công thức tính:}
		\boxmini{$\mathrm{d}\left(M,(P) \right)=\dfrac{\bigg|ax_0+by_0+cz_0+d\bigg|}{\sqrt{a^2+b^2+c^2}}$}
	}{
		\begin{tikzpicture}[scale=0.8, line join=round, line cap=round]
			\tkzDefPoints{0/0/A,4/0/B,5/2/C}
			\coordinate (D) at ($(A)+(C)-(B)$);
			\tkzDrawPolygon(A,B,C,D)
			\tkzMarkAngles[size=0.7cm,arc=l](B,A,D)
			\tkzLabelAngles[pos=0.5,rotate=10](B,A,D){$P$}
			\draw (2,1)node[right]{$H$}--(2,3)node[above]{$M$};
			\draw[fill=black] (2,1) circle(1.5pt) (2,3) circle(1.5pt);
	\end{tikzpicture}}
	\item [\iconMT] \indam{Đặc biệt:} 
	\begin{listEX}[3]
		\item [\ding{172}] $\mathrm{d}\left(M,(Oxy) \right)=\big|z_M\big|$.
		\item [\ding{173}]  $\mathrm{d}\left(M,(Oxz) \right)=\big|y_M\big|$.
		\item [\ding{174}]  $\mathrm{d}\left(M,(Oyz) \right)=\big|x_M\big|$.
	\end{listEX}
\end{itemize}
\subsection{PHÂN LOẠI, PHƯƠNG PHÁP GIẢI TOÁN}
\begin{dang}{Xác định vectơ pháp tuyến và điểm thuộc mặt phẳng}
	Cho mặt phẳng $(\alpha)$.
	\begin{itemize}
		\item  [\ding{172}] Nếu véctơ $\overrightarrow{n}$ khác $\overrightarrow{0}$ và có giá vuông góc với $(\alpha)$ thì $\overrightarrow{n}$ được gọi là véctơ pháp tuyến của $(\alpha)$.
		\item  [\ding{173}] Nếu hai véctơ $\overrightarrow{a}, \overrightarrow{b}$ không cùng phương, có giá song song hoặc nằm trong $(\alpha)$ thì $\overrightarrow{a}, \overrightarrow{b}$ được gọi là cặp véctơ chỉ phương của $(\alpha)$. Khi đó, nếu $\vec{a}=(a_1;a_2;a_3)$, $\vec{b}=(b_1;b_2;b_3)$ thì
		$$\vec{n}= [\vec{a}, \vec{b}]=\left(\left|\begin{array}{ll}a_2 & a_3 \\ b_2 & b_3\end{array}\right| ;\left|\begin{array}{ll}a_3 & a_1 \\ b_3 & b_1\end{array}\right| ;\left|\begin{array}{ll}a_1 & a_2 \\ b_1 & b_2\end{array}\right|\right)$$ là một vectơ pháp tuyến của mặt phẳng $(P)$.
		\item [\ding{174}] Nếu $(\alpha) \colon ax + by + cz + d = 0$ thì vectơ pháp tuyến của $(\alpha)$ là $\vec{n}=(a;b;c)$.
	\end{itemize}
\end{dang}
\boxmini{BÀI TẬP TỰ LUẬN}

\begin{vd}
	\immini{Cho hình lập phương $ABCD.A'B'C'D'$. 
		\begin{tasks}
			\task Xác định vectơ pháp tuyến của các mặt phẳng $(ABCD)$,  $(ABB'A')$,  $(ACC'A')$,  $(ADD'A')$.
			\task Chứng minh $\vec{AB'}$ là một vectơ pháp tuyến của $(BCD'A')$.
		\end{tasks}
	}{
	\begin{tikzpicture}[scale=0.7, font=\footnotesize, line join=round, line cap=round, >=stealth]
		\def\bc{4} % cạnh BC
		\def\ba{2} % cạnh BA
		\def\h{3} % đường cao
		\def\gocB{35} % góc B của đáy
		\coordinate[label=below left:$B$] (B) at (0,0);
		\coordinate[label=above left:$A$] (A) at (\gocB:\ba);
		\coordinate[label=below:$C$] (C) at (\bc,0);
		\coordinate[label=right:$D$] (D) at ($(C)-(B)+(A)$);
		\coordinate[label=above left:$A'$] (A') at ($(A)+(90:\h)$);
		\coordinate[label=left:$B'$] (B') at ($(B)-(A)+(A')$);
		\coordinate[label=below right:$C'$] (C') at ($(C)-(A)+(A')$);
		\coordinate[label=right:$D'$] (D') at ($(D)-(A)+(A')$);
		\draw (B')--(B)--(C)--(D)--(D')--(A')--(B')--(C')--(D') (C)--(C');
		\draw[dashed] (A')--(A)--(D) (A)--(B);
		\foreach \diem in {A,B,C,D,A',B',C',D'}	\fill (\diem)circle(1.5pt);
\end{tikzpicture}}
\end{vd}

\dongcham{9}
\begin{vd}
	Cho mặt phẳng $(P): 2x-3y+4z+5=0$. Hãy chỉ ra một vectơ pháp tuyến của $(P)$ và hai điểm thuộc $(P)$. 

	\loigiai{vectơ $\overrightarrow{n}=(2;-3;4) $ là một vectơ pháp tuyến của mặt phẳng $(P)$. 
		
	}
\end{vd}
\dongcham{4}

\begin{vd}
	Cho $(P)$ là mặt phẳng trung trực của $MN$ với $M(1;-2;3)$, $N(1;4;1)$. Hãy chỉ ra một vectơ pháp tuyến của $(P)$ và một điểm thuộc $(P)$.
	\loigiai{
		Mặt phẳng trung trực của $MN$ đi qua trung điểm của $MN$ và vuông góc với $MN$.\\
		Vậy $\vec{MN}=(0; 6; -2)$ là một vectơ pháp tuyến của mặt phẳng $(P)$.
	}
\end{vd}
\dongcham{4}
\begin{vd}
	Chỉ ra một vectơ pháp tuyến của mặt phẳng $(\alpha)$ biết
	\begin{listEX}[1]
		\item $(\alpha)$ đi qua $A(-1; 3; 5)$, $B(3;2;-2)$ và $C(0; 3; 0)$
		\item $(\alpha)$ đi qua $M(0; 3; 1)$, $N(-3;2;5)$ và $P(-2; 0; 0)$
	\end{listEX}
	\loigiai{
		\begin{listEX}[1]
			\item 	Ta có $\vec{AB} = (4; -1;-7)$, $\vec{AC} = (1; 0; -5)$.\\
			Xét vectơ $\vec{n}=[\vec{AB}, \vec{AC}] = \left(\left|\begin{array}{cc}-1 & -7 \\ 0 & -5\end{array}\right| ;\left|\begin{array}{cc}-7 & 4 \\ -5 & 1\end{array}\right| ;\left|\begin{array}{cc}4 & -1 \\ 1 & 0\end{array}\right|\right)=(5 ; 13 ; 1)$.\\
			Vậy $\vec{n} = (5 ; 13 ; 1)$ là một vectơ pháp tuyến của mặt phẳng $(\alpha)$.
			\item Ta có $\vec{MN} = (-3; -1;4)$, $\vec{MP} = (-2; -3; -1)$.\\
			Xét vectơ $\vec{n}=[\vec{MN}, \vec{MP}] = \left(\left|\begin{array}{cc}-1 & 4 \\ -3 & -1\end{array}\right| ;\left|\begin{array}{cc}4 & -3 \\ -1 & -2\end{array}\right| ;\left|\begin{array}{cc}-3 & -1 \\ -2 & -3\end{array}\right|\right)=(13; -11; 7)$.\\
			Vậy $\vec{n} = (13 ; -11 ; 7)$ là một vectơ pháp tuyến của mặt phẳng $(\alpha)$.
		\end{listEX}
	}
\end{vd}
\dongcham{6}


\begin{vd}
	Cho tứ diện $ABCD$ có các đỉnh là $A(5 ; 1 ; 3)$, $B(1 ; 6 ; 2)$, $C(5 ; 0 ; 4)$ và $D(4 ; 0 ; 6)$. Gọi $(\alpha)$ là mặt phẳng chứa cạnh $AB$ và song song với cạnh $CD$. Hãy tìm một điểm thuộc $(\alpha)$ và một vectơ pháp tuyến của $(\alpha)$.
	\loigiai{
		Ta có $\overrightarrow{AB}=(-4 ; 5 ;-1)$, $\overrightarrow{CD}=(-1 ; 0 ; 2)$ nên $\overrightarrow{AB}, \overrightarrow{CD}$ không cùng phương.\\
		Mà giá của $\overrightarrow{AB}$ nằm trong mặt phẳng $(\alpha)$ và giá của $\overrightarrow{CD}$ song song với mặt phẳng $(\alpha)$ nên $\overrightarrow{AB}$, $\overrightarrow{CD}$ là một cặp vectơ chỉ phương của mặt phẳng $(\alpha)$.
		Vậy một vectơ pháp tuyến của $(\alpha)$ là
		$$
		\begin{aligned}
			\vec{n}=\left[\overrightarrow{AB}, \overrightarrow{CD}\right]&=
			\left(\left|\begin{array}{cc}
				5 & -1 \\
				0 & 2
			\end{array}\right| ;
			\left|\begin{array}{cc}
				-1 & -4 \\
				2 & -1
			\end{array}\right| ;
			\left|\begin{array}{cc}
				-4 & 5 \\
				-1 & 0
			\end{array}\right|\right) \\
			&=\left(5 \cdot 2-0 \cdot(-1) ; (-1)\cdot (-1) - 2 \cdot (-4) ; (-4) \cdot 0-5 \cdot (-1)\right) \\
			&=(10 ; 9 ; 5).
		\end{aligned}
		$$
	}
\end{vd}
\dongcham{4}
\boxmini{BÀI TẬP TRẮC NGHIỆM}
\setcounter{ex}{0}

\begin{ex}
	Cho mặt phẳng $(\alpha) \colon 2x-y+3z-2=0$. Điểm nào sau đây thuộc mặt phẳng $(\alpha)$?
	\choice
	{$A(1;-3;1)$}
	{\True $B(2;-1;-1)$}
	{$C(2;-1;1)$}
	{$D(1;2;3)$}
	\loigiai{
	Thay tọa độ các điểm vào phương trình $(\alpha)$, tọa độ $B(2;-1;-1)$ thỏa mãn.}
\end{ex}

\begin{ex}%[2H3Y2-7]
	Cho mặt phẳng $(\alpha) \colon x+y+z-6=0$. Điểm nào dưới đây \textbf{không} thuộc $(\alpha)$?
	\choice
	{\True $M(1;-1;1)$}
	{$N(2;2;2)$}
	{$P(1;2;3)$}
	{$Q(3;3;0)$}
	\loigiai
	{
		Ta có $1-1+1-6=-5 \neq 0$ nên $M(1;-1;1)$ không thuộc $(\alpha)$.
	}
\end{ex}

\begin{ex}
	Cho $(\alpha)$ vuông góc với giá của $\vec{a}=(2;-1;3)$. Vectơ nào dưới đây là vectơ pháp tuyến của $(\alpha)$?
	\choice
	{$\vec{n_1}=(-2;1;3)$}
	{\True $\vec{n_2}=(-2;1;-3)$}
	{$\vec{n_3}=(4;2;6)$}
	{$\vec{n_4}=(4;-2;-6)$}
	\loigiai{
		$(\alpha)$ vuông góc với giá của $\vec{a}=(2;-1;3)$ nên $\vec{a}$ là một vectơ pháp tuyến của $(\alpha)$.\\
		Do đó $\vec{n_2}=-\vec{a}$ cũng là một vectơ pháp tuyến của $(\alpha)$.
	}
\end{ex}

\begin{ex}%[2H3B2-2]
	vectơ nào sau đây \textbf{không} phải là vectơ pháp tuyến của mặt phẳng $(P):x+3y-5z+2=0$.
	\choice
	{$\overrightarrow{n}_1=(-1;-3;5)$}
	{\True $\overrightarrow{n}_2=(-2;-6;-10)$}
	{$\overrightarrow{n}_3=(-3;-9;15)$}
	{$\overrightarrow{n}_4=(2;6;-10)$}
	\loigiai{Mặt phẳng $(P)$ nhận vectơ $\overrightarrow{a}=(1;3;-5)$ làm vectơ pháp tuyến.\\
		Xét $\overrightarrow{n}_2=(-2;-6;-10)$ có $\dfrac{-2}{1}\ne\dfrac{-6}{3}\ne\dfrac{-10}{-5}$ nên $\overrightarrow{n}_2$ không cùng phương với $\overrightarrow{a}$.\\
		Suy ra $\overrightarrow{n}_2$ không là vectơ pháp tuyến của $(P)$.}
\end{ex}

\begin{ex}
	Trong không gian  $Oxyz$, mặt phẳng tọa độ $(Oxy)$ có một vectơ pháp tuyến là
	\choice
	{$\overrightarrow{n}=(0;1;0)$}
	{\True $\overrightarrow{n}=(0;0;1)$}
	{$\overrightarrow{n}=(1;0;0)$}
	{$\overrightarrow{n}=(1;1;0)$}
	\loigiai{
		Mặt phẳng tọa độ $(Oxy)\colon x=0\Rightarrow$ 1 vectơ pháp tuyến là $\overrightarrow{n}=(0;0;1)$.}
\end{ex}

\begin{ex}
	Trong không gian $Oxyz$, cho điểm $A(4;-3;7)$ và $B(2;1;3)$. Một vectơ pháp tuyến của mặt phẳng trung trực của đoạn $AB$ là
	\choice
	{\True $\vec{n}=(1;-2;2)$}
	{$\vec{n}=(2;4;4)$}
	{$\vec{n}=(6;-2;10)$}
	{$\vec{n}=(-2;-4;4)$}
	\loigiai{
		Ta có $\overrightarrow{AB}(-2;4;-4)$ cùng phương với  $\vec{n}=(1;-2;2)$. Suy ra  $\vec{n}=(1;-2;2)$ là một vectơ pháp tuyến.
	}
\end{ex}

\begin{ex}%[HK1 - Chuyên Huỳnh Mẫn Đạt - Kiên Giang - 20-21]%[Phan Quốc Trí-EX5]%[2H3B2-2]%
	Trong không gian $Oxyz$, $(P)$ là mặt phẳng trung trực của đoạn $AB$, biết $A(1;3;0)$, $B(-2;1;-1)$. vectơ nào sau đây là vectơ pháp tuyến của $(P)$?
	\choice
	{$\overrightarrow{n}_4=(3;-2;-1)$}
	{$\overrightarrow{n}_2=(-3;2;-1)$}
	{$\overrightarrow{n}_3=(-3;4;1)$}
	{\True $\overrightarrow{n}_1=(3;2;1)$}
	\loigiai{
		Mặt phẳng $(P)$ có vectơ pháp tuyến là  $\overrightarrow{BA}= (3;2;1)$.
	}
\end{ex}

\begin{ex}%[Trần Bình Thuận - DA2]%[2H3B2-2]% câu 2
	Trong không gian $Oxyz$, vectơ nào sau đây là một vectơ pháp tuyến của $(P)$. Biết $\vec{u}=(1;-2;0)$, $\vec{v}=(0;2;-1)$ là cặp vectơ chỉ phương của $(P)$.
	\choice
	{$\vec{n}=(1;2;0)$}
	{\True $\vec{n}=(2;1;2)$}
	{$\vec{n}=(2;-1;2)$}
	{$\vec{n}=(0;1;2)$}
	\loigiai{
		Ta có $(P)$ có một vectơ pháp tuyến là $\vec{n}=\left[\vec{u},\vec{v}\right]=\left(
		\begin{vmatrix}
			-2&0\\
			2&-1
		\end{vmatrix};
		\begin{vmatrix}
			0&1\\
			-1&0
		\end{vmatrix};
		\begin{vmatrix}
			1&-2\\
			0&2
		\end{vmatrix}\right)=(2;1;2)$.
	}
\end{ex}

\begin{ex}
	Trong không gian  $Oxyz$, cho $(\alpha)$ song song với giá của $\vec{a}=(1;-2;-3)$, $\vec{b}=(-4;2;0)$. Vectơ nào dưới đây \textbf{không phải} là vectơ pháp tuyến của $(\alpha)$?
	\choice
	{$\vec{n_1}=(6;12;-6)$}
	{$\vec{n_2}=(1;2;-1)$}
	{$\vec{n_3}=(-2;-4;2)$}
	{\True $\vec{n_4}=(-3;-6;-3)$}
	\loigiai{
		$(\alpha)$ song song với giá của $\vec{a}=(1;-2;-3)$, $\vec{b}=(-4;2;0)$ nên $\vec{a}$, $\vec{b}$ là cặp vectơ chỉ phương của $(\alpha)$.\\
		Một vectơ pháp tuyến của mặt phẳng $(\alpha)$ là
		$$
		\begin{aligned}
			\vec{n}=[\vec{a}, \vec{b}] & =\left(\left|\begin{array}{cc}
				-2 & -3 \\ 2 & 0
			\end{array}\right| ;\left|\begin{array}{cc}
				-3 & 1 \\ 0 & -4
			\end{array}\right| ;\left|\begin{array}{cc}
				1 & -2 \\ -4 & 2
			\end{array}\right|\right) \\
			& =(6 ;12 ; -6) .
		\end{aligned}
		$$
		Ta có $\vec{n_1}=\vec{n}$; $\vec{n_2}=\dfrac{1}{6}\vec{n}$; $\vec{n_3}=-\dfrac{1}{3}\vec{n}$ là các vectơ pháp tuyến của $(\alpha)$.\\
		Vậy $\vec{n_4}=(-3;-6;-3)$ không phải là vectơ pháp tuyến của $(\alpha)$.
	}
\end{ex}


\begin{ex}%[Thi thử, Sở GD và ĐT - Hậu Giang, 2020]%[Trần Thành Thống, 12EX10]%[2H3B2-2]%
	Trong không gian $Oxyz$, cho ba điểm $A(2;0;0)$, $B(0;-3;0)$, $C(0;0;6)$. Tọa độ một vectơ pháp tuyến của mặt phẳng $(ABC)$ là
	\choice
	{$\overrightarrow{n}=(1;-2;3)$}
	{$\overrightarrow{n}=(3;2;1)$}
	{\True $\overrightarrow{n}=(3;-2;1)$}
	{$\overrightarrow{n}=(2;-3;6)$}
	\loigiai{
		Ta có $\overrightarrow{AB}=\left(-2;-3;0\right)\; \overrightarrow{AC}=\left(0;3;6\right)$.\\
		$\Rightarrow$ vectơ pháp tuyến của mặt phẳng $\left(ABC\right)$ là $\overrightarrow{v}=\left[\overrightarrow{AB};\overrightarrow{AC}\right]=\left(-18;12;-6\right)$.\\
		Ta có $\overrightarrow{v}=\left(-18;12;-6\right)$ cùng phương với $\overrightarrow{n}=\left(3;-2;1\right)$.
	}
\end{ex}

\begin{ex}
	Trong không gian $Oxyz$, cho ba điểm $A(2;-1;3)$, $B(4;0;1)$ và $C(-10;5;3)$. vectơ nào dưới đây là vectơ pháp tuyến của mặt phẳng $(ABC)$?
	\choice
	{$\overrightarrow{n}=(1;2;0)$}
	{$\overrightarrow{n}=(1;-2;2)$}
	{$\overrightarrow{n}=(1;8;2)$}
	{$\True \overrightarrow{n}=(1;2;2)$}
	\loigiai{
		Ta có $\overrightarrow{AB}=(2;1;-2)$, $\overrightarrow{AC}=(-12;6;0)$, $\left[\overrightarrow{AB},\overrightarrow{AC}\right]=(12;24;24)$.\\
		$\Rightarrow (ABC)$ có một vectơ pháp tuyến là $\overrightarrow{n}=\dfrac{1}{12}\left[\overrightarrow{AB},\overrightarrow{AC}\right]=(1;2;2)$.}
\end{ex}


\begin{ex}%[2H3B2-2]%[50 dạng toán đề minh họa 2020-Nguyễn Tâm Phục]%Câu 6.
	Trong không gian $Oxyz$, cho hai điểm $A(2;-1;5)$, $B(1;-2;3)$. Mặt phẳng $(\alpha)$ đi qua hai điểm $A$, $B$ và song song với trục $Ox$ có vectơ pháp tuyến $\overrightarrow{n}=(0;a;b)$. Khi đó tỉ số $\dfrac{a}{b}$ bằng
	\choice
	{\True $-2$}
	{$-\dfrac{3}{2}$}
	{$\dfrac{3}{2}$}
	{$2$}
	\loigiai{
		$\overrightarrow{BA}=(1;1;2)$; $\overrightarrow{i}=(1;0;0)$ là vectơ đơn vị của trục $Ox$.\\
		Vì $(\alpha)$ đi qua hai điểm $A$, $B$ và song song với trục $Ox$ nên $\left[\overrightarrow{BA},\overrightarrow{i}\right]=(0;2;-1)$ là một vectơ pháp tuyến của $(\alpha)$. Do đó $\dfrac{a}{b}=-2$.}
\end{ex}

\begin{dang}{Lập phương trình mặt phẳng khi biết các yếu tố liên quan}
	\begin{itemize}
		\item [\iconCV] \indamm{Công thức:} Cho $(P)$ qua điểm $M(x_0,y_0,z_0)$ và một vectơ pháp tuyến $\overrightarrow{n_P}=(a,b,c)$. Khi đó, phương trình $(P)$ là
		\begin{align*}
			\boxed{(P):a(x-x_0)+b(y-y_0)+c(z-z_0)=0}
		\end{align*}
		\item [\iconCV] \indamm{Một số cách xác định vectơ pháp tuyến thường gặp:}
		\begin{listEX}[1]
			\item [\ding{172}] Nếu $(P)\bot AB$ thì $\overrightarrow{n_P}=\overrightarrow{AB}$;
			\item [\ding{173}] Nếu $(P)$ là mặt phẳng trung trực của đoạn $AB$ thì $(P)$ qua trung điểm $I$ của $AB$ và $\overrightarrow{n_P}=\overrightarrow{AB}$;
			\item [\ding{174}] Nếu $(P)$ có cặp vectơ chỉ phương $\vec u$, $\vec v$ thì $\overrightarrow{n_P}=[\vec u,\vec v]$ là một vectơ pháp tuyến của $(P)$.
			\item [\ding{175}] Nếu $(P)$ qua ba điểm $A,B,C$ phân biệt và không thẳng hàng thì $\overrightarrow{n_P}=\left[ \overrightarrow{AB},\overrightarrow{AC} \right]$;
			\item [\ding{176}] Nếu $(P)$ qua hai điểm $A,B$ phân biệt và song song với $d$ thì $\overrightarrow{n_P}=\left[ \overrightarrow{AB},\overrightarrow{u_d} \right]$;
			\item [\ding{177}] Nếu $(P)$ qua điểm $A$ và chứa $d$ thì $\overrightarrow{n_P}=\left[ \overrightarrow{AM},\overrightarrow{u_d} \right]$, với $M \in d$.
		\end{listEX}
	\item [\iconCV] \indamm{Phương trình theo đoạn chắn:}
		Cho $(P)$ đi qua $A(a;0;0),\,B(0;b;0),\,C(0;0;c)$ với $abc \neq 0$ thì $(P):\dfrac{x}{a}+\dfrac{y}{b}+\dfrac{z}{c}=1$ (phương trình theo đoạn chắn)
	\end{itemize}
\end{dang}
\boxmini{BÀI TẬP TỰ LUẬN}
\setcounter{vd}{0}
\begin{vd}
	Trong không gian $Oxyz$, cho ba điểm $A(3;-2;-2)$, $B(3;2;0)$, $C(0;2;1)$.
	\begin{tasks}
		\task Lập phương trình mặt phẳng qua $A$ và vuông góc với $BC$.
		\task Lập phương trình mặt phẳng trung trực của đoạn $AB$.
		\task Lập phương trình mặt phẳng $(ABC)$.
	\end{tasks} 
	\loigiai{
		Ta có
		$\vec{AB}=(0;4;2)$, $\vec{AC}=(-3;4;3)$ là cặp vectơ chỉ phương của $(ABC)$.\\
		$\vec{n}=\left[\vec{AB},\vec{AC}\right]=(4;-6;12)$.\\
		Chọn $\vec{n}_1=\dfrac{1}{2} \vec{n}=(2;-3;6)$ là một vectơ pháp tuyến của $(ABC)$.\\
		Mặt phẳng $(ABC)$ đi qua điểm $C(0;2;1)$ và có một vectơ pháp tuyến $\vec{n}_1=(2;-3;6)$ nên $(ABC)$ có phương trình là
		$$2(x-0)-3(y-2)+6(z-1)=0\Leftrightarrow 2x-3y+6z=0.$$
		Vậy phương trình mặt phẳng cần tìm là $2x-3y+6z=0$.
	}
\end{vd}
\dongcham{11}

\begin{vd}
	Cho tứ diện $ ABCD $ có các đỉnh $ A(5;1;3)$, $B(1;6;2)$, $C(5;0;4),D(4;0;6) $.
	\begin{listEX}
		\item Hãy viết phương trình của các mặt phẳng $ (ACD) $ và $ (BCD) $;
		\item  Hãy viết phương trình mặt phẳng $ (\alpha) $ chứa cạnh $ AB $ và song song với cạnh $ CD $;
		\item Gọi $A'$, $B'$, $C'$ lần lượt là hình chiếu vuông góc của $A$, $B$, $C$ lên các trục $Ox$, $Oy$, $Oz$. Hãy viết phương trình mặt phẳng $(A'B'C')$.
	\end{listEX}
	\loigiai{
		\begin{listEX}
			\item Ta có $ \vec{AC}=(0;-1;1),\vec{AD}=(-1;-1;3),\vec{BC}=(4;-6;2),\vec{BD}=(4;-6;4) $.
			\begin{itemize}
				\item 	Mặt phẳng $ (ACD) $ qua $ A(5;1;3) $ và có vectơ pháp tuyến $ \vec{n}=\left[\vec{AC},\vec{AD}\right]=(-2;-1;-1) $ có phương trình là 
				$$ -2(x-5)-(y-1)-(z-3)=0\Leftrightarrow 2x+y+z-14=0 .$$
				\item 	Mặt phẳng $ (BCD) $ qua $ B(1;6;2) $ và có vectơ pháp tuyến $ \vec{n}=\left[\vec{BC},\vec{BD}\right]=(-12;-8;0) $ có phương trình là 
				$$ -12(x-1)-8(y-6)-0(z-3)=0\Leftrightarrow 3x+2y-15=0 .$$
			\end{itemize}
			\item Ta có $ \vec{AB}=(-4;5;-1) $, $\vec{CD}=(-1;0;2)$.\\
			Mặt phẳng $ (\alpha) $ chứa cạnh $ AB $ và song song với cạnh $ CD $ qua $ A(5;1;3) $ và có vectơ pháp tuyến $ \vec{n}=\left[\vec{AB},\vec{CD}\right]=(10;9;5) $ có phương trình là 
			$$ 10(x-5)+9(y-1)+5(z-3)=0\Leftrightarrow 10x+9y+5z-74=0 .$$
			\item Ta có $A'(5;0;0)$, $B'(0;6;0)$, $C'(0;0;4)$. Phương trình mặt phẳng $(A'B'C')$ là
			$$\dfrac{x}{5}+\dfrac{y}{6}+\dfrac{z}{4}=1.$$
		\end{listEX}
	}
\end{vd}
\dongcham{17}
\begin{vd}
	Viết phương trình của mặt phẳng
	\begin{tasks}(2)
		\task Chứa trục $ Ox $ và điểm $ M(-4;1;2) $;
		\task Chứa trục $ Oz $ và điểm $ P(3;0;-7) $.
	\end{tasks}
	\loigiai{
		\begin{listEX}
			\item Mặt phẳng $ (P) $ chứa trục $ Ox $ và điểm $ M(-4;1;2) $ nên có vectơ pháp tuyến \\$ \vec{n}=\left[\vec{i},\vec{OM}\right]=(0;-2;1) $, phương trình của $ (P) $ là $ -2y+z=0 $.
			\item Mặt phẳng $ (R) $ chứa trục $ Oz $ và điểm $ P(3;0;-7) $ nên có vectơ pháp tuyến \\$ \vec{n}=\left[\vec{k},\vec{OP}\right]=(0;3;0) $, phương trình của $ (R) $ là $ y=0 $.
		\end{listEX}
	}
\end{vd}
\dongcham{12}

\begin{vd}
	Một phần sân nhà bác An có dạng hình thang $ABCD$ vuông tại $A$ và $B$ với độ dài $AB=9$ m, $AD=5$ m và $BC=6$ m như Hình bên dưới. Theo thiết kế ban đầu thì mặt sân bằng phẳng và $A$, $B$, $C$, $D$ có độ cao như nhau. Sau đó bác An thay đổi thiết kế để nước có thể thoát về phía góc sân ở vị trí $C$ bằng cách giữ nguyên độ cao ở $A$, giảm độ cao của sân ở vị trí $B$ và $D$ xuống thấp hơn độ cao ở $A$ lần lượt là $6$ cm và $3{,}6$ cm. Để mặt sân sau khi lát gạch vẫn là bề mặt phẳng thì bác An cần phải giảm độ cao ở $C$ xuống bao nhiêu centimét so với độ cao ở $A$?
	\begin{center}
		% \includegraphics[scale=.4]{images/2P5-1-H5-9}
		% \hspace{0.5cm}
		\begin{tikzpicture}[scale=0.5, font=\footnotesize,line join=round, line cap=round, >=stealth]
			\path
			(0,0) coordinate (A) 
			(9,0) coordinate(B)
			(9,-6) coordinate(C)
			(0,-5) coordinate(D)
			;
			\draw[thick] (A)--(B)--(C)--(D)--cycle;
			\node [above] at ($(A)!0.5!(B)$) {$9$ m};
			\node [right] at ($(B)!0.5!(C)$) {$6$ m};
			\node [left] at ($(A)!0.5!(D)$) {$5$ m};
			\foreach \i/\g in {A/90,B/90,C/-90,D/-90}{\draw[fill=black](\i) circle (0pt) ($(\i)+(\g:4mm)$) node[scale=1]{$\i$};}
		\end{tikzpicture}
	\end{center}
	\loigiai{
		Tại vị trí ban đầu $A$, $B$, $C$, $D$ có độ cao như nhau, chọn hệ trục tọa độ có gốc tọa độ là điểm $A$ và các trục tọa độ lần lượt là $AD$, $AB$ và $Az$, với $Az \perp (ABCD)$.\\
		Khi đó $A(0 ; 0 ; 0)$, $D(5; 0 ; 0)$, $B(0; 9 ; 0)$, $C(6; 9 ; 0)$.\\
		Sau đó bác An thay đổi thiết kế để nước có thể thoát về phía góc sân ở vị trí $C$ bằng cách giữ nguyên độ cao ở $A$, giảm độ cao của sân ở vị trí $B$ và $D$ xuống thấp hơn độ cao ở $A$ lần lượt là $6$ cm và $3{,}6$ cm.\\
		Khi đó, $A(0 ; 0 ; 0)$, $D(5; 0 ; -3{,}6)$, $B(0; 9 ; -6)$.\\
		Ta có $\overrightarrow{AB}=(0 ; 9 ; -6)$, $ \overrightarrow{AD}=(5 ; 0 ; -3{,}6)$ là cặp vectơ chỉ phương của mặt phẳng $(ABD)$ nên một vectơ pháp tuyến của $(ABD)$ là $\left[\overrightarrow{AB}, \overrightarrow{AD}\right]=(-32{,}4 ; -30 ; -45)$.\\
		Vậy mặt phẳng $(ABD)$ qua $A(0 ; 0 ; 0)$ và có vectơ pháp tuyến $\vec{n}=(-32{,}4 ; -30 ; -45)$ nên có phương trình là
		\allowdisplaybreaks
		\begin{eqnarray*}
			-32{,}4 (x-2)-30(y+1)-45(z-3)=0 \qquad \text{hay } -32{,}4 x -30y -45z=0.
		\end{eqnarray*}
		Để mặt sân sau khi lát gạch vẫn là bề mặt phẳng thì bác An cần phải giảm độ cao ở $C$ xuống $k$ centimét so với độ cao ở $A$ nên suy ra $C(6; 9 ; -k)$.\\
		Ta có $A$, $B$, $C$, $D$ đồng phẳng\\
		$\Leftrightarrow C \in (ABD)$\\
		$\Leftrightarrow -32{,}4\cdot 6 -30 \cdot 9 -45\cdot (-k)=0$\\
		$\Leftrightarrow k=10{,}32$.\\
		Vậy bác An cần phải giảm độ cao ở $C$ xuống $10{,}32$ centimét so với độ cao ở $A$.
	}
\end{vd}
\dongcham{14}
\boxmini{BÀI TẬP TRẮC NGHIỆM}
\setcounter{ex}{0}
\begin{ex}%[2H3Y2-3]
	Phương trình mặt phẳng đi qua điểm $A(1;2;3)$ và có vectơ pháp tuyến $\overrightarrow{n}=(-2;0;1)$ là
	\choice
	{$-2x+z+1=0$}
	{$-2y+z-1=0$}
	{\True $-2x+z-1=0$}
	{$-2x+y-1=0$}
	\loigiai
	{
		Phương trình của mặt phẳng cần tìm là $-2(x-1)+0(y-2)+1(z-3)=0 \Leftrightarrow -2x+z-1=0$.
	}
\end{ex}
\cham{2}

\begin{ex}%[2H3B2-3]
	Phương trình nào được cho dưới đây là phương trình mặt phẳng $(Oyz)$?
	\choice
	{$x=y+z$}
	{$y-z=0$}
	{$y+z=0$}
	{\True $x=0$}
	\loigiai{
		Trong không gian với hệ tọa độ $Oxyz$, phương trình của mặt phẳng $(Oyz)$ là $x=0$.
	}
\end{ex}
\cham{2}

\begin{ex}%[Lê Quý Đôn, Hà Nội, lần 1, 2018]%[2H3B2-3]%[Nguyễn Bình Nguyên-12Ex7]
	Cho các điểm $A(0;1;2)$, $B(2;- 2;1)$, $C(- 2;0;1)$. Phương trình mặt phẳng đi qua $A$ và vuông góc với $BC$ là
	\choice
	{$2x - y - 1 = 0$}
	{$ - y + 2z - 3 = 0$}
	{\True $2x - y + 1 = 0$}
	{$y + 2z - 5 = 0$}
	\loigiai{
		Ta có $\overrightarrow{n}=\dfrac{1}{2}\overrightarrow{BC}=(-2;1;0)$.\\
		Vậy phương trình mặt phẳng đi qua $A$ và vuông góc với $BC$ có dạng
		$ - 2(x - 0) + 1(y - 1) = 0 \Leftrightarrow  - 2x + y - 1 = 0$ $ \Leftrightarrow 2x - y + 1 = 0$.}
\end{ex}
\cham{3}

\begin{ex}%[2H3B2-3]
	Cho hai điểm $A(4;0;1)$ và $B(-2;2;3)$. Phương trình nào dưới đây là phương trình mặt phẳng trung trực của đoạn thẳng $AB$?
	\choice
	{$3x-y-z+1=0$}
	{$3x+y+z-6=0$}
	{\True $3x-y-z=0$}
	{$6x-2y-2z-1=0$}
	\loigiai
	{
		Gọi $(\alpha)$ là mặt phẳng trung trực của đoạn thẳng $AB$. Khi đó $(\alpha)$ đi qua điểm $M(1;1;2)$, là trung điểm của $AB$, và nhận $\overrightarrow{AB}=(-6;2;2)$ làm vectơ pháp tuyến. Phương trình của mặt phẳng $(\alpha)$ là
		$$-6(x-1)+2(y-1)+2(z-2)=0 \Leftrightarrow -6x+2y+2z=0 \Leftrightarrow 3x-y-z=0.$$
	}
\end{ex}
\cham{3}
\begin{ex}
	Trong không gian $Oxyz$, cho hai điểm $A(1;1;1)$ và $B(1;3;5)$. Viết phương trình mặt phẳng trung trực của đoạn $AB$.
	\choice
	{$y-2z-6=0$}
	{$y-2z+2=0$}
	{$y-3z+4=0$}
	{\True $y+2z-8=0$}
	\loigiai{
		Ta có $I(1;2;3)$ là trung điểm của đoạn $AB$.\\
		Mặt phẳng trung trực của đoạn thẳng $AB$ đi qua $I$ và có vectơ pháp tuyến $\overrightarrow{AB}=(0;2;4)=2(0;1;2)$, suy ra phương trình mặt phẳng trung trực cần tìm là
		\begin{center}
			$0(x-1)+1(y-2)+2(z-3)=0\Leftrightarrow y+2z-8=0$.
		\end{center}
	}
\end{ex}

\begin{ex}%[Thi thử lần 1, THPT Văn Giang - Hưng Yên, 2019]%[Đỗ Đường Hiếu, 12EX-8-2019]%[2H3B2-3]%
	Trông không gian $Oxyz$, phương trình mặt phẳng $(P)$ đi qua $A(0;-1;4)$ và song song với giá của hai vectơ $\vec{u}=(3;2;1)$, $\vec{v}=(-3;0;1)$ là
	\choice
	{\True $x-3y+3z-15=0$}
	{$x-2y+3z-14=0$}
	{$x-y-z+3=0$}
	{$x-3y+3z-9=0$}
	\loigiai{
		Mặt phẳng $(P)$ có vectơ pháp tuyến là $\left[ \vec{u}; \vec{v}\right] =(2;-6;6)$. Hay $(P)$ có vectơ pháp tuyến là $\vec n=(1;-3;3)$.\\
		Phương trình mặt phẳng $(P)$ là
		$$1\cdot(x-0)-3\cdot (y+1)+3\cdot (z-4)=0\;\text{hay}\; (P)\colon x-3y+3z-15=0.$$
	}
\end{ex}

\begin{ex}
	Trong không gian $Oxyz$, cho ba điểm $A(3;-2;-2)$, $B(3;2;0)$, $C(0;2;1)$. Phương trình mặt phẳng $(ABC)$ là
	\choice
	{$2x-3y+6z+12=0$}
	{$2x+3y-6z-12=0$}
	{\True $2x-3y+6z=0$}
	{$2x+3y+6z+12=0$}
	\loigiai{
		Ta có
		$\vec{AB}=(0;4;2)$, $\vec{AC}=(-3;4;3)$ là cặp vectơ chỉ phương của $(ABC)$.\\
		$\vec{n}=\left[\vec{AB},\vec{AC}\right]=(4;-6;12)$.\\
		Chọn $\vec{n}_1=\dfrac{1}{2} \vec{n}=(2;-3;6)$ là một vectơ pháp tuyến của $(ABC)$.\\
		Mặt phẳng $(ABC)$ đi qua điểm $C(0;2;1)$ và có một vectơ pháp tuyến $\vec{n}_1=(2;-3;6)$ nên $(ABC)$ có phương trình là
		$$2(x-0)-3(y-2)+6(z-1)=0\Leftrightarrow 2x-3y+6z=0.$$
		Vậy phương trình mặt phẳng cần tìm là $2x-3y+6z=0$.
	}
\end{ex}
\cham{4}

\begin{ex}%[12-TN-BGD-3]%[Nguyễn Thành Khang, dự án 12-TN-BGD-3]%[2H3B2-3]%
	Trong không gian với hệ trục toạ độ $Oxyz$, cho ba điểm $A(1;0;0),B(0;-1;-1),C(5;-1;1)$. Mặt phẳng $(ABC)$ có phương trình là
	\choice
	{$2x+3y+5z-2=0$}
	{$2x-3y-5z-2=0$}
	{$2x-3y-5z+2=0$}
	{\True $2x+3y-5z-2=0$}
	\loigiai{
		Ta có $\vec{AB}=(-1;-1;-1), \vec{AC}=(4;-1;1)$ nên vectơ pháp tuyến của mặt phẳng $(ABC)$ là $\vec{n}=\left[\vec{AC},\vec{AB}\right]=(2;3;-5)$, mà mặt phẳng $(ABC)$ đi qua $A(1;0;0)$ nên có phương trình là $2x+3y-5z-2=0$.
	}
\end{ex}
\cham{4}
\begin{ex}
	Mặt phẳng $(\alpha)$ đi qua $A(-1; 4; -6)$ và chứa trục $Oy$ có phương trình là
	\choice
	{$-2x+y+z=0$}
	{$6x+z=0$}
	{$3x-y-6z+1=0$}
	{\True $6x-z=0$}
	\loigiai{
		Ta có $\vec{OA} = (-1; 4; -6)$, $\vec{j} = (0; 1; 0)$ song song hoặc trùng với $(\alpha)$. Nên $\vec{OA}$ và $\vec{j}$ là cặp vectơ chỉ phương của $(\alpha)$.\\
		Xét vectơ $\vec{n}=[\vec{OA}, \vec{j}] = \left(\left|\begin{array}{cc}4 & -6 \\ 1 & 0\end{array}\right| ;\left|\begin{array}{cc}-6 & -1 \\ 0 & 0\end{array}\right| ;\left|\begin{array}{cc}-1 & 4 \\ 0 & 1\end{array}\right|\right)=(6 ; 0 ; -1)$.\\
		Do đó $\vec{n} = (6; 0 ; -1)$ là một vectơ pháp tuyến của mặt phẳng $(\alpha)$.\\
		Phương trình mặt phẳng $(\alpha)$ là $6x-z=0$.
	}
\end{ex}
\cham{6}
\begin{ex}%[Thi thử L1, Cụm chuyên môn, Sở GDDT Hải Phòng, 2019]%[Nguyễn Quang Tân, dự án 12-EX-7-2019]%[2H3B2-3]%
	Trong không gian ${Oxyz}$, mặt phẳng chứa trục ${Ox}$ và đi qua điểm $A(1;1;-1)$ có phương trình là
	\choice
	{\True $ y+z=0$}
	{$ z+1=0$}
	{$ x+z=0$}
	{$ x-y=0$}
	\loigiai{
		Gọi $\vec n$ là vectơ pháp tuyến của mặt phẳng $(P)$ chứa trục ${Ox}$ và đi qua điểm $A(1;1;-1)$.\\
		Ta có $\heva{&\vec n \bot \overrightarrow {OA}  = \left(1;1; - 1\right)\\& \vec n \perp \vec i = \left( 1;0;0 \right).}$\\
		Chọn một vectơ pháp tuyến của mặt phẳng $(P)$ là $\vec{n} = \left[ \vec i, \vec{OA} \right] = \left(0;1;1\right)$.\\
		Vậy phương trình mặt phẳng là $y + z = 0$.
	}
\end{ex}
\cham{6}
\begin{ex}%[Đề thi thử - Trường THPT chuyên Lương Thế Vinh - Đồng Nai - Lần 1 - 2018]%[2H3B2-3]%[Kim Minh Bui - 12EX8]%
	Trong không gian $Oxyz,$ cho ba điểm $A(2;1;1),\ B(3;0;-1),\ C(2;0;3)$. Mặt phẳng $(\alpha)$ đi qua hai điểm $A,\ B$ và song song với đường thẳng $OC$ có phương trình là
	\choice
	{$3x+y-2z-5=0$}
	{$4x+2y+z-11=0$}
	{$x-y+z-2=0$}
	{\True $3x+7y-2z-11=0$}
	\loigiai{
		Gọi $\overrightarrow{n}$ là vtpt của mặt phẳng $(\alpha)$.\\
		Ta có $\begin{cases} AB \subset (\alpha) \\ OC \parallel (\alpha) \end{cases} \Rightarrow
		\begin{cases} \overrightarrow{n} \perp \overrightarrow{AB} \\ \overrightarrow{n} \perp \overrightarrow{OC} \end{cases}$ nên $\overrightarrow{n}$ cùng phương với $\overrightarrow{AB} \wedge \overrightarrow{OC}$.\\
		Ta có $\overrightarrow{AB}=(1;-1;-2),\ \overrightarrow{OC}=(2;0;3) \Rightarrow \overrightarrow{AB} \wedge \overrightarrow{OC}=(-3;-7;2) = (-1) \cdot (3;7;-2).$ Ta chọn $\overrightarrow{n} = (3;7;-2)$. \\ Phương trình mặt phẳng $(\alpha)$ là: $3x+7y-2z-11=0.$
	}
\end{ex}
\cham{6}
\begin{ex}
	Mặt phẳng đi qua hai điểm $A(1;2;-1)$, $B(0;4;3)$ và song song với trục $Oz$ có phương trình là
	\choice
	{\True $2x + y -4 =0$}
	{$4x - 4y +3 z+7 =0$}
	{$x + 2y -5=0$}
	{$2x + y+z -3 =0$}
	\loigiai{
		$\vec{AB}=(-1;2;4)$, $\vec{k}=(0;0;1)$. vectơ pháp tuyến của mặt phẳng cần tìm là $\vec{n}=\big[\vec{AB}, \vec{k}\big]=(2;1;0)$.\\
		Mặt phẳng qua $A(1;2;-1)$, nhận $\vec{n}=(2;1;0)$ làm vectơ pháp tuyến có phương trình là
		$$2(x-1)+1(y-2)+0(z+1)=0 \Leftrightarrow 2x+y-4=0.$$
		
	}
\end{ex}
\cham{6}

\begin{ex}%[2H3B2-3]
	Cho điểm $M(1;2;-3)$. Gọi $M_{1}$, $M_{2}$, $M_{3}$ lần lượt là hình chiếu vuông góc của $M$ lên trục $Ox$, $Oy$, $Oz$. Phương trình mặt phẳng đi qua ba điểm $M_{1}$, $M_{2}$, $M_{3}$ là
	\choice
	{\True $x+\dfrac{y}{2}-\dfrac{z}{3}=1$}
	{$\dfrac{x}{3}+\dfrac{y}{2}+\dfrac{z}{1}=1$}
	{$x+\dfrac{y}{2}+\dfrac{z}{3}=1$}
	{$x+\dfrac{y}{2}+\dfrac{z}{3}=-1$}
	\loigiai{
		Ta có $M_{1}(1;0;0)$, $M_{2}(0;2;0)$, $M_{3}(0;0;-3)$.\\
		Phương trình mặt phẳng đi qua $M_{1}$, $M_{2}$, $M_{3}$ là $x+\dfrac{y}{2}-\dfrac{z}{3}=1$.
	}
\end{ex}
\cham{4}

\begin{ex}%[2H3K2] 
	Mặt phẳng nào sau đây cắt các trục $Ox$, $Oy$, $Oz$ lần lượt tại các điểm $A$, $B$, $C$ sao cho tam giác $ABC$ nhận điểm $G\big(1; 2; 1\big)$ là trọng tâm?
	\choice{$x + 2y + 2z  - 6 = 0$}
	{\True $2x + y + 2z  - 6 = 0$}
	{$2x + 2y + z  - 6 = 0$}
	{$2x + 2y + 6z - 6 = 0$} 
\end{ex}
\cham{6}
\begin{ex}%[2H3B2-3]
	Cho mặt phẳng  $\left(P\right)$ đi qua điểm $M\left(2; - 4; 1\right)$  và chắn trên các trục tọa độ $Ox$, $Oy$, $Oz$ theo ba đoạn có độ dài đại số lần lượt là $a$, $b$, $c$. Phương trình tổng quát của mặt phẳng $\left(P\right)$ khi $a$, $b$, $c$ theo thứ tự tạo thành một cấp số nhân có công bội bằng $2$ là 
	\choice
	{$4x + 2y - z - 1 = 0$}
	{$4x -  2y + z +  1 = 0$}
	{$16x + 4y - 4z - 1 = 0$}
	{\True $4x + 2y +  z - 1 = 0$}
	\loigiai{Do giả thiết suy ra $a, b, c \neq 0$ và $b = 2a$, $c = 2b$. Giả sử $A\left(a; 0;0\right)$, $B\left(0; b;0\right)$ và $C\left(0; 0;c\right)$ khi đó phương trình mặt phẳng $\left(P\right)\colon \dfrac{x}{a} + \dfrac{y}{b} + \dfrac{z}{c} = 1$.  Do $M$ thuộc $\left(P\right)$  nên
		$$\dfrac{2}{a} - \dfrac{4}{b} + \dfrac{1}{c} = 1\Leftrightarrow \dfrac{2}{a} - \dfrac{4}{2a} + \dfrac{1}{4a} = 1\Leftrightarrow a = \dfrac{1}{4}.$$
		Suy ra $b = \dfrac{1}{2}$ và $c = 1$ do đó phương trình mặt phẳng $\left(P\right)\colon 4x + 2y + z - 1 = 0$.
	}
\end{ex}
\cham{8}

\begin{dang}{Vị trí tương đối của hai mặt phẳng}
	Cho hai mặt phẳng $(P) \colon a_1x+b_1y+c_1z+d_1=0$ và $(Q) \colon a_2x+b_2y+c_2z+d_2=0$.
		\begin{listEX}[1]
		\item [\ding{172}] Nếu $\heva{&\vec{n_1}= k \cdot \vec{n_2}\\& d_1 =k\cdot d_2}$ thì $(P)$ trùng $(Q)$.
		\item [\ding{173}] Nếu $\heva{&\vec{n_1}= k \cdot \vec{n_2}\\& d_1 \ne k\cdot d_2}$ thì $(P)$ song song $(Q)$.
		\item [\ding{174}] Nếu $\vec{n_1}$ không cùng phương với $\vec{n_2}$ thì $(P)$ cắt $(Q)$.
		\item [\ding{175}] Nếu $\vec{n_1} \perp \vec{n_2}$ hay $a_1a_2+b_1b_2+c_1c_2=0$ thì $(P) \perp (Q)$.
	\end{listEX}
\end{dang}
\setcounter{ex}{0}
\setcounter{vd}{0}
\boxmini{BÀI TẬP TỰ LUẬN}

\begin{vd}%[2H5H1-4]	
	Tìm các cặp mặt phẳng song song hoặc vuông góc trong các mặt phẳng sau
	\begin{listEX}[2]
		\item [] $(P)\colon 2x+3y-2z+7=0$
		\item [] $(Q)\colon 3x-2y-11=0$
		\item [] $(R)\colon 4x+6y-4z-9=0$
		\item [] $(T)\colon 7x+y-z+1=0$
	\end{listEX}
	\loigiai{
		Các mặt phẳng $(P)$, $(Q)$, $(R)$, $(T)$ có các vectơ pháp tuyến lần lượt là $\overrightarrow{n}_1=(2;3;-2)$, $\overrightarrow{n}_2=(3;-2;0)$, $\overrightarrow{n}_3=(4;6;-4)$, $\overrightarrow{n}_4=(7;1;-1)$.\\
		Ta có $\overrightarrow{n}_1\cdot \overrightarrow{n}_2=2 \cdot3 +3\cdot (-2)+(-2)\cdot0 =0$, suy ra $(P)\perp (Q)$.\\
		Vì $\dfrac{4}{2}=\dfrac{6}{3}=\dfrac{-4}{-2}\ne \dfrac{-9}{7}$ nên $(P) \parallel (R)$.\\
		Ta lại có $ (P)\perp (Q)$ và $(P) \parallel (R)$, suy ra $(Q) \perp (R)$.\\
		Ta có $\dfrac{2}{7}\ne \dfrac{3}{1}$ suy ra $\overrightarrow{n}_2$ và $\overrightarrow{n}_4$ không cùng phương.\\
		Mặt khác, $\overrightarrow{n}_1\cdot \overrightarrow{n}_4=2\cdot7+3\cdot1+(-2)\cdot(-1)=19\ne 0$. Suy ra $(P)$ và $(T)$ cắt nhau nhưng không vuông góc. Tương tự, ta cũng có $(Q)$ và $(T)$ cắt nhau nhưng không vuông góc.}
\end{vd}
\dongcham{11}
\begin{vd}
	Trong không gian $Oxyz$, cho mặt phẳng $(\alpha)\colon 2x-3y+z+5=0$.
	\begin{listEX}[1]
		\item Chứng minh rằng mặt phẳng $\left(\alpha'\right)\colon-4 x+6 y-2 z+7=0$ song song với $(\alpha)$.
		\item Viết phương trình mặt phẳng $(\beta)$ đi qua điểm $M(1 ; -2 ; 3)$ và song song với $(\alpha)$.
	\end{listEX}
	\loigiai{
		\begin{listEX}[1]
			\item Xét $(\alpha)\colon 2x-3y+z+5=0$ và $\left(\alpha'\right)\colon -4x+6y-2z+7=0$.\\
			Ta có $\dfrac{2}{-4}=\dfrac{-3}{6}=\dfrac{1}{-2} \neq \dfrac{5}{7}$ nên $(\alpha) \parallel \left(\alpha'\right)$.
			\item Mặt phẳng $(\alpha)$ có vectơ pháp tuyến $\vec{n}=(2 ;-3 ; 1)$.\\
			Vì $(\beta) \parallel (\alpha)$ nên $(\beta)$ có vectơ pháp tuyến $\vec{n}=(2 ;-3 ; 1)$.\\
			Vậy mặt phẳng $(\beta)$ đi qua điểm $M(1 ;-2 ; 3)$ và có vectơ pháp tuyến $\vec{n}=(2 ;-3 ; 1)$ nên có phương trình là
			\allowdisplaybreaks
			\begin{eqnarray*}
				2(x-1)-3(y+2)+(z-3)=0 \qquad \text{hay } 2x-3y+z-11=0.
			\end{eqnarray*}
		\end{listEX}
		
	}
\end{vd}

\dongcham{10}

\begin{vd}%[Thi thử, Sở GD và ĐT-THANH HÓA, 2020]%[Nguyễn Hữu Tính]%[2H3B2-3]%
	Trong không gian $Oxyz$, cho  hai mặt phẳng $(Q) \colon x+y+3z=0$, $(R) \colon  2x-y+z=0$.
	\begin{enumEX}[a)]{1}
		\item Xét vị trí tương đối của $(Q)$ và $(R)$;
		\item Viết trình của mặt phẳng $(P)$ đi qua điểm $B(2;1;-3)$, đồng thời vuông góc với $(Q)$ và $(R)$.
	\end{enumEX}
	\loigiai{
		vectơ pháp tuyến $(P)$ là $n_{\overrightarrow{P}}= \left[ n_{\overrightarrow{Q}},n_{\overrightarrow{R}} \right]= (4;5;-3)$.\\
		Phương trình mặt phẳng $(P)$ là $4(x-2)+5(y-1)-3(z+3)=0 \Leftrightarrow 4x+5y-3z-22=0$.
	}
\end{vd}
\dongcham{10}
\begin{vd}%[Đề tập huấn, Sở GD - ĐT tỉnh Quảng Bình, 2019]%[Nguyễn Tiến, dự án 12EX5]%[2H3K2-3]%
	Trong không gian với hệ tọa độ $Oxyz$, cho hai điểm $A(-2;4;-1)$, $B(1;1;3)$ và mặt phẳng $(P)$ có phương trình $x-3y+2z-5=0$. Viết phương trình mặt phẳng $(Q)$ đi qua hai điểm $A$, $B$ và vuông góc với mặt phẳng $(P)$.
	\loigiai{
		$\left.\begin{array}{l} \overrightarrow{AB}=(3;-3;4)\\ \overrightarrow{n}_{(P)}=(1;-3;2)\end{array}\right\}\Rightarrow\left[\overrightarrow{AB},\overrightarrow{n}_{(P)}\right]=(6;-2;-6)=2(3;-1;-3)$.\\
		Mặt phẳng $(Q)$ đi qua điểm $A$ và có vectơ pháp tuyến $\overrightarrow{n}_{(Q)}=(3;-1;-3)$ có phương trình
		\begin{eqnarray*}
			& & 3(x+2)-(y-4)-3(z+1)=0\\
			&\Leftrightarrow & 3x-y-3z+7=0.
		\end{eqnarray*}
	}
\end{vd}
\dongcham{10}

\boxmini{BÀI TẬP TRẮC NGHIỆM}

\begin{ex}%[2H3B2-7]
	Cho mặt phẳng $(P)\colon -x+y+3z+1=0$. Mặt phẳng song song với mặt phẳng $(P)$ có phương trình nào sau đây?
	\choice
	{\True $2x-2y-6z+7=0$}
	{$-2x+2y+3z+5=0$}
	{$x-y+3z-3=0$}
	{$-x-y+3z+1=0$}
	\loigiai{
		vectơ pháp tuyến của mặt phẳng $ (P) $ là $ \overrightarrow{n}=(-1;1;3)$ cùng phương với vectơ $\overrightarrow{n}=(2;-2;-6) $. Vì $ \dfrac{2}{-1}\neq \dfrac{7}{1} $ nên phương trình mặt phẳng song song với $ (P) $ là $2x-2y-6z+7=0$.
	}
\end{ex}
\cham{2}

\begin{ex}%[2H3B2-7]
	Cho hai mặt phẳng $(P) \colon 2x+4y+3z-5=0$ và $(Q) \colon mx-ny-6z+2-0$. Giá trị của $m,n$ sao cho $(P) \parallel (Q)$ là
	\choice
	{$m=4;n=-8$}
	{$m=n=4$}
	{\True $m=-4;n=8$}
	{$m=n=-4$}
	\loigiai{
		$(P)$ có vectơ chỉ phương $\overrightarrow{u}_{(P)}=(2;4;3)$, $(Q)$ có vectơ chỉ phương $\overrightarrow{u}_{(Q)}=(m;-n;-6)$.\\ Để hai mặt phẳng trên song song thì $\overrightarrow{u}_{(Q)}=k\overrightarrow{u}_{(P)}\,(k \neq 0) \Leftrightarrow \heva{&m=2k\\&-n=4k\\&-6=3k} \Rightarrow \heva{&k=-2\\&m=-4\\&n=8.}$
	}
	
\end{ex}
\cham{3}

\begin{ex}
	Cho hai mặt phẳng $(P)\colon x+my+(m-1)z+1=0$ và $(Q)\colon x+y+2z=0$. Tập hợp tất cả các giá trị $m$ để hai mặt phẳng này \textbf{không} song song là
	\choice
	{$(0;+\infty)$}
	{$\mathbb{R}\setminus\{-1;1;2\}$}
	{$(-\infty;3)$}
	{\True $\mathbb{R}$}
	\loigiai{
		Ta có $A(0;0;0)\in (Q)$.\\
		$(P)\parallel (Q)\Leftrightarrow \heva{&\dfrac{1}{1}=\dfrac{m}{1}=\dfrac{m-1}{2}\\&A(0;0;0)\notin (P)}$. Hệ này vô nghiệm. Do đó $(P)$ không song song với $(Q)$, với mọi giá trị của $m$. 
	} 
\end{ex}
\cham{4}

\begin{ex}
	Cho mặt phẳng $(\alpha)\colon x+y+z-1=0$. Trong các mặt phẳng sau, tìm mặt phẳng vuông góc với mặt phẳng $(\alpha)$.
	\choice
	{$2x-y+z+1=0$}
	{\True $2x-y-z+1=0$}
	{$2x+2y+2z-1=0$}
	{$x-y-z+1=0$}
	\loigiai{
		Mặt phẳng $(\alpha)$ có $\overrightarrow{n}_{(\alpha)}=(1;1;1)$.\\
		Mặt phẳng $2x-y-z+1=0$ có vectơ pháp tuyến $\overrightarrow{n}_1=(2;-1;-1)\Rightarrow\overrightarrow{n}_{(\alpha)}\cdot\overrightarrow{n}_1=0$ nên mặt phẳng $(\alpha)$ vuông góc với mặt phẳng $2x-y-z+1=0$.
	}
\end{ex}

\begin{ex}%[2H3B2-7]
	Cho mặt phẳng $(P)\colon 2x-y+2z-3=0$ và $(Q) \colon x+my+z-1=0$. Tìm tham số $m$ để hai mặt phẳng $P$ và $Q$ vuông góc với nhau.
	\choice
	{$m=-4$}
	{$m=- \dfrac{1}{2}$}
	{$m=\dfrac{1}{2}$}
	{\True $m=4$}
	\loigiai
	{ Mặt phẳng $(P)$ và $(Q)$ có vectơ pháp tuyến lần lượt là $\overrightarrow{n}_1=(2;-1;2)$ và $\overrightarrow{n}_2=(1;m;1)$. \\
		Do đó $(P) \perp (Q) \Leftrightarrow \overrightarrow{n}_1 \cdot \overrightarrow{n}_2=0 \Leftrightarrow 2-m+2=0 \Leftrightarrow m=4$.
		
	}
\end{ex}
\cham{2}

\begin{ex}
	Cho hai mặt phẳng $(P)\colon x+2y-z-1=0$, $(Q)\colon 3x-(m+2)y+(2m-1)z+3=0$. Tìm $m$ để hai mặt phẳng $(P)$ và $(Q)$ vuông góc với nhau.
	\choice
	{\True $m=0$}
	{$m=2$}
	{$m=-2$}
	{$m=-1$}
	\loigiai{
		vectơ pháp tuyến của $(P)$, $(Q)$ lần lượt là $\overrightarrow{n}_P=(1;2;-1)$ và $\overrightarrow{n}_Q=(3;-m-2;2m-1)$.\\
		$(P)\perp (Q)\Leftrightarrow \overrightarrow{n}_P\cdot\overrightarrow{n}_Q=0\Leftrightarrow 3-2(m+2)-2m+1=0\Leftrightarrow m=0$.
	}
\end{ex}

\begin{ex}%[2H3B2-3]%
	Mặt phẳng đi qua $A(1;3;-2)$ và song song với mặt phẳng $(P) \colon 2x-y+3z+4=0$ có phương trình là
	\choice
	{\True $2x-y+3z+7=0$}
	{$2x-y+3z-7=0$}
	{$2x+y-3z+7=0$}
	{$2x+y+3z+7=0$}
	\loigiai{
		Ta có $\vec{n}=\vec{n_{(P)}}=(2;-1;3)$. Khi đó phương trình mặt phẳng qua $A(1;3;-2)$ và song song $(P)$ là
		$$2(x-1)-1(y-3)+3(z+2)=0\Leftrightarrow 2x-y+3z+7=0.$$
	}
\end{ex}


\begin{ex}%[KSCL giữa HK2 Cụm trường THPT TP Nam Định]%[Nguyễn Tiến, 12EX7]%[2H3B2-3]%
	Cho điểm $A(2;-1;-3)$ và mặt phẳng $(P)\colon 3x-2y+4z-5=0$. Mặt phẳng $(Q)$ đi qua $A$ và song song với mặt phẳng $(P)$ có phương trình là
	\choice
	{\True $(Q)\colon 3x-2y+4z+4=0$}
	{$(Q)\colon 3x+2y+4z+8=0$}
	{$(Q)\colon 3x-2y+4z+5=0$}
	{$(Q)\colon 3x-2y+4z-4=0$}
	\loigiai{
		Do mặt phẳng $(Q)$ song song với mặt phẳng $(P)$ nên có vectơ pháp tuyến là $\overrightarrow{n}=(3;-2;4)$.\\
		Phương trình mặt phẳng $(Q)\colon 3(x-2)-2(y+1)+4(z+3)=0 \Leftrightarrow 3x-2y+4z+4=0$.
	}
\end{ex}

\begin{ex}
	Cho mặt phẳng $(P)$ đi qua các điểm $A(-2; 0; 0)$, $B(0; 3; 0)$, $C(0; 0; -3)$. Mặt phẳng $(P)$ vuông góc với mặt phẳng nào trong các mặt phẳng sau?
	\choice
	{\True $2x+2y-z-1=0$}
	{$x+y+z+1=0$}
	{$3x-2y+2z+6=0$}
	{$x-2y-z-3=0$}
	\loigiai{
		Mặt phẳng $(P)\colon \dfrac{x}{-2}+\dfrac{y}{3}+\dfrac{z}{-3}=1$ hay $(P)\colon 3x-2y+2z+6=0$ có vectơ pháp tuyến $\vec{n}=(3;-2;2)$.\\
		Ta có $3\cdot 2 -2\cdot 2-2\cdot 1=0$ nên $(P)$ vuông góc với mặt phẳng $2x+2y-z-1=0$.
	}
\end{ex}

\begin{ex}
	Mặt phẳng qua $A(1;2;-1)$ và vuông góc với các mặt phẳng $(P) \colon 2x-y+3z-2=0$; $(Q) \colon x+y+z-1=0$ có phương trình là
	\choice
	{$x-y+z+2=0$}
	{$4x-y+z-1=0$}
	{$x+y+2z-1=0$}
	{\True $4x-y-3z-5=0$}
	\loigiai{
		vectơ pháp tuyến của mặt phẳng $(P)$ và $(Q)$ lần lượt là $\overrightarrow{n_1}=(2;-1;3)$ và $\overrightarrow{n_2}=(1;1;1)$.
		\\
		Ta có $\left[\overrightarrow{n_1};\overrightarrow{n_2}\right]=(-4;1;3)$.
		Mặt phẳng cần tìm qua $A(1;2;-1)$ và có vectơ pháp tuyến là $\overrightarrow{n}=(-4;1;3)$ nên có phương trình là
		$$-4 \cdot (x-1)+1 \cdot (y-2)+3 \cdot (z+1)=0\Leftrightarrow4x-y-3z-5=0.$$}
\end{ex}

\begin{ex}
	Cho hai mặt phẳng $(P)$, $(Q)$ lần lượt có phương trình là $x+y-z=0$, $x-2y+3z=4$ và cho điểm $M(1;-2;5)$. Tìm phương trình mặt phẳng $(\alpha)$ đi qua điểm $M$ và đồng thời vuông góc với hai mặt phẳng $(P)$, $(Q)$.
	\choice
	{$5x+2y-z+14=0$}
	{\True $x-4y-3z+6=0$}
	{$x-4y-3z-6=0$}
	{$5x+2y-z+4=0$}
	\loigiai{
		Ta có $\vec{n}_{(P)} = (1;1;-1)$ và
		$\vec{n}_{(Q)} = (1;-2;3).$\\
		Suy ra $\left[\vec{n}_{(P)},\vec{n}_{(Q)}\right]= (1;-4;-3).$\\
		Do $(\alpha)$ vuông góc với $(P)$ và $(Q)$ nên $\heva{&\vec{n}_{(\alpha)} \perp \vec{n}_{(P)}\\&\vec{n}_{(\alpha)} \perp \vec{n}_{(Q)}}$. \\
		Chọn $\vec{n}_{(\alpha)} = \left[\vec{n}_{(P)},\vec{n}_{(Q)}\right]=(1;-4;-3)$. Hơn nữa, $(\alpha)$ đi qua $M(1;-2;5)$ nên có phương trình là $$(x-1)-4(y+2)-3(z-5)=0\Leftrightarrow x-4y-3z+6=0.$$
	}
\end{ex}

\begin{ex}%[Thi thử, THPT chuyên Quang Trung, 2020]%[Phạm Doãn Lê Bình, 12EX2-2020]%[2H3B2-3]%
	Cho điểm $A(-4;1;1)$ và mặt phẳng $(P)\colon x-2y-z+4=0$. Mặt phẳng $(Q)$ đi qua điểm $A$ và song song với mặt phẳng $(P)$ có phương trình là
	\choice
	{$(Q)\colon x-2y-z+7=0$}
	{\True $(Q)\colon x-2y-z-7=0$}
	{$(Q)\colon x-2y+z+5=0$}
	{$(Q)\colon x-2y+z-5=0$}
	\loigiai{
		Do $(Q)\parallel (P)$ nên phương trình của $(Q)$ có dạng $x-2y-z+c=0$ ($c\ne 4$).\\
		Do $A \in (Q)$ nên $-4-2\cdot 1 - 1 + c = 0 \Leftrightarrow c = 7$ (thỏa).\\
		Vậy $(Q)\colon x-2y-z+7=0$.
	}
\end{ex}

\begin{ex}%[Thi thử L1, THPT Chuyên ĐH Vinh, Nghệ An, 2019]%[Nguyễn Đắc Giáp, dự án 12EX6]%[2H3B2-3]%
	Cho hai mặt phẳng $(P)\colon x-3y+2z-1=0$, $(Q)\colon x-z+2=0$. Mặt phẳng $\left(\alpha\right)$ vuông góc với hai mặt phẳng $(P),(Q)$ đồng thời cắt trục $Ox$ tại điểm có hoành độ bằng $3$. Phương trình của $\left(\alpha\right)$ là
	\choice
	{$-2x+z+6=0$}
	{$-2x+z-6=0$}
	{\True $x+y+z-3=0$}
	{$x+y+z+3=0$}
	\loigiai{
		Mặt phẳng $(P)$ có một vectơ pháp tuyến là $\overrightarrow{n}_P=(1;-3;2)$.\\
		Mặt phẳng $(Q)$ có một vectơ pháp tuyến là $\overrightarrow{n}_Q=(1;0;-1)$.\\
		Vì mặt phẳng $(\alpha)$ vuông góc với hai mặt phẳng $(P)$ và $(Q)$ nên $(\alpha)$ có một vectơ pháp tuyến là	 $\overrightarrow{n}_{\alpha}=\left[\overrightarrow{n}_P,\overrightarrow{n}_Q\right]=3(1;1;1)$.\\
		Mà mặt phẳng $(\alpha)$ đi qua $A(3;0;0)$, nên suy ra phương trình là $\left(\alpha\right)\colon x+y+z-3=0$.
	}
\end{ex}

\begin{ex}%[2H3K2-3]%
	Cho $A\left( 1;-1;2 \right);\ B\left( 2;1;1 \right)$ và mặt phẳng $\left( P \right):x+y+z+1=0$. Mặt phẳng $\left( Q \right)$ chứa $A,\ B$ và vuông góc với mặt phẳng $\left( P \right)$. Mặt phẳng $\left( Q \right)$ có phương trình là
	\choice
	{$3x-2y-z+3=0$}
	{\True $3x-2y-z-3=0$}
	{$-x+y=0$}
	{$x+y+z-2=0$}
	\loigiai
	{
		Ta có $\vec{A B}=(1 ; 2 ;-1)$ và vectơ pháp tuyến của $(P)$ là $\vec{n}_{P}=(1 ; 1 ; 1)$. Gọi vectơ pháp tuyến của $(Q)$ là $\vec{n}_{Q}$.\\
		Vì $(Q)$ chứa $A,B$ nên $\vec{n_{Q}} \perp \vec{A B}$, mặt khác $(Q) \perp(P)$ nên $\vec{n_{Q}} \perp \vec{n_{P}}$. \\
		Từ đó suy ra $\vec{n_{Q}}=\left[\vec{A B}, \vec{n_{P}}\right]=(3 ;-2 ;-1)$.\\
		$(Q)$ đi qua $A(1;-1;2)$ và có vectơ pháp tuyến $\vec{n_{Q}}=(3 ;-2 ;-1)$ nên $(Q)$ có phương trình là
		$$(Q):3(x-1)-2(y+1)-(z-2)=0 \Leftrightarrow 3 x-2 y-z-3=0.$$
	}
\end{ex}

\begin{ex}%[2H3K2-3]%[Đề thi thử L2, THPT Nguyễn Quang Diêu, 2018]%[Đỗ Đường Hiếu, 12EX-9]%
	Cho hai điểm $A(2;4;1)$, $B(-1;1;3)$ và mặt phẳng $(P)\colon x-3y+2z-5=0$. Một mặt phẳng $(Q)$ đi qua hai điểm $A$, $B$ và vuông góc với mặt phẳng $(P)$ có dạng là $ax+by+cz-11=0$. Tính $a+b+c$.
	\choice
	{$a+b+c=-7$}
	{$a+b+c=10$}
	{\True $a+b+c=5$}
	{$a+b+c=3$}
	\loigiai{
		Ta có $\overrightarrow{AB}=\left(-3;-3;2\right)$ và vectơ pháp tuyến của mặt phẳng $(P)$ là $\overrightarrow{n}_P=\left(1;-3;2\right)$.\\
		Mặt phẳng $(Q)$ đi qua hai điểm $A$, $B$ và vuông góc với mặt phẳng $(P)$ có một vectơ chỉ phương là
		$$\overrightarrow{n}_Q=\left[\overrightarrow{AB}, \overrightarrow{n}_P\right]=\left(0;8;12\right) =4\left(0;2;3\right).$$
		Phương trình mặt phẳng $(Q)$ là
		$$0\cdot (x-2)+2\cdot (y-4)+3\cdot (z-1)=0.$$
		Hay $(Q)\colon 2y+3z-11=0$.
		Từ đó suy ra $a=0$, $b=2$, $c=3$. Do đó $a+b+c=0+2+3=5$.
	}
\end{ex}

\begin{dang}{Khoảng cách từ một điểm đến mặt phẳng, khoảng cách giữa hai mặt phẳng song song}
	\begin{itemize}
		\item [\iconMT] \indam{Khoảng cách từ một điểm đến mặt phẳng:} Cho điểm $M(x_0;y_0;z_0)$ và mặt phẳng $(P) \colon ax+by+cz+d=0$. Khi đó
		\boxmini{$\mathrm{d}\left(M,(P) \right)=\dfrac{\bigg|ax_0+by_0+cz_0+d\bigg|}{\sqrt{a^2+b^2+c^2}}$}
		\item [\iconMT] \indam{Khoảng cách giữa hai mặt phẳng song song:} 	Cho hai mặt phẳng $(P) \colon ax + by + cz + d_1=0$ và $(Q) \colon ax + by + cz + d_2=0$ song song nhau. 
		Khi đó
		\boxmini{$\mathrm{d}\left((P),(Q) \right)=\dfrac{\bigg|d_1-d_2\bigg|}{\sqrt{a^2+b^2+c^2}}$}
	\end{itemize}
\end{dang}
\setcounter{ex}{0}
\setcounter{vd}{0}
\boxmini{BÀI TẬP TỰ LUẬN}
\begin{vd}%[2H5H1-5] %[Dang]
	Tính khoảng cách từ điểm $A(1;2;3)$ đến các mặt phẳng sau
	\begin{enumEX}[a)]{3}
		\item $\left(P\right) \colon 3x + 4z + 10 = 0$;
		\item $\left(Q\right) \colon 2x - 10 = 0$;
		\item $\left(R\right) \colon 2x + 2y + z - 3 = 0$.
	\end{enumEX}
	
	\loigiai{
		\begin{listEX}
			\item $\mathrm{d}\left( A;\left( P \right) \right) = \dfrac{\left| {3 \cdot 1 + 0 \cdot 2 + 4 \cdot 3 + 10} \right|}{\sqrt {3^2 + 4^2}} = 5$.
			\item $\mathrm{d}\left(A;\left( Q \right) \right) = \dfrac{\left| {2 \cdot 1 - 10} \right|}{\sqrt {2^2} } = 2$.
			\item $\mathrm{d}\left( A;\left( R \right) \right) = \dfrac{\left| {2 \cdot 1 + 2 \cdot 2 + 1 \cdot 3 - 3} \right|}{\sqrt {2^2 + 2^2 + 1^2} } = 2$.
		\end{listEX}
	}
\end{vd}
\dongcham{3}
\begin{vd}%[2H5N1-4]%[2H5H1-5]%[Dự án tex hóa sách bài tập Toán 12 CTST]%[Lê Thị Thúy Hằng]
	Cho hai mặt phẳng $(P) \colon 2x+y+2z+12=0$, $(Q) \colon 4x+2y+4z-6=0$.
	\begin{enumerate}
		\item Chứng minh $(P) \parallel (Q)$.
		\item Tính khoảng cách giữa hai mặt phẳng $(P)$ và $(Q)$.
	\end{enumerate}
	\loigiai{
		\begin{enumerate}
			\item Xét hai mặt phẳng $(P) \colon 2x+y+2z+12=0$ và $(Q) \colon 4x+2y+4z-6=0$, ta có
			$\dfrac{4}{2} = \dfrac{2}{1} = \dfrac{4}{2} \ne \dfrac{-6}{12}$, suy ra $(P) \parallel (Q)$.
			\item Trên mặt phẳng $(Q)$, lấy điểm $M(0;1;1)$.\\
			Ta có
			$$ \mathrm{d}((P),(Q)) = \mathrm{d} (M, (P)) = \dfrac{\left| 2 \cdot 0 + 1 \cdot 1 + 2 \cdot 1 + 12 \right|}{\sqrt{2^2+1^2+2^2}}=\dfrac{15}{3}=5.$$
		\end{enumerate}
	}
\end{vd}
\dongcham{7}
\begin{vd}
	\immini{
		Một kĩ sư xây dựng thiết kế khung một ngôi nhà trong không gian $Oxyz$ như Hình 9 nhờ một phần mềm đồ họa máy tính.
		\begin{enumerate}
			\item Viết phương trình mặt phẳng mái nhà $(DEMN)$.
			\item Tính khoảng cách từ điểm $B$ đến mái nhà $(DEMN)$.
		\end{enumerate}
	}
	{
		\begin{tikzpicture}[scale=.7, font=\footnotesize, line join=round, line cap=round, >=stealth]
			\path 
			(0:0) coordinate (O)
			(-135:2) coordinate (A)
			(0:4) coordinate (C)
			($(A)+(-135:0.5)$) coordinate (x)
			($(A)+(45:2)$) coordinate (O)
			($(A)+(C)-(O)$) coordinate (B)
			($(O)+(90:4)$) coordinate (D)
			($(D)+(90:0.8)$) coordinate (z)
			($(C)+(0:0.5)$) coordinate (y)
			($(A)+(90:4)$) coordinate (E)
			($(B)+(90:4)$) coordinate (F)
			($(C)+(90:4)$) coordinate (H)
			($(D)+(20:2)$) coordinate (N)
			($(E)+(20:2)$) coordinate (M)
			($(O)+(-50:4.5)$) coordinate (h)
			;
			\draw[dashed] (A)--(O)--(C) (O)--(D)--(H);
			\draw[->] (A)--(x) node[below]{$x$};
			\draw[->](C)--(y) node[above]{$y$};
			\draw[->](D)--(z) node[right]{$z$};
			\draw 	(B)--(F)--(E)--(A)--cycle
			(D)--(E)--(M)--(N)--cycle
			(B)--(C)--(H)--(F)--(M) (N)--(H) 
			;
			\draw (A) node[left]{$A(6;0;0)$}
			(O) node[below right]{$O(0;0;0)$}
			(C) node[below right]{$C(0;4;0)$}
			(D) node[left]{$D(0;0;4)$}
			(M) node[right]{$M(6;2;6)$}
			(N) node[above right]{$N(0;2;6)$}
			(E) node[left]{$E(6;0;4)$}
			(B) node[right]{$B$}
			(H) node[right]{$H$}
			(h) node[above]{Hình 9}
			;
			%pic[draw,angle radius=2mm]{right angle=C--O--O'}
			%pic[draw,angle radius=2mm]{right angle=A--O--C}
			%pic[draw,angle radius=2mm]{right angle=O'--O--A}
			%;
			%\foreach \x/\g in {A/170,B/-15,C/-30,O/180,E/170,F/-15,H/0,D/180}
			%\draw[fill=black] 	(\x) circle (.5pt)
			%($(\g:.5)+(\x)$) node {$\x$};	
		\end{tikzpicture}
	}
	\loigiai{
		\begin{enumerate}
			\item Mặt phẳng $(DEMN)$ có cặp vectơ chỉ phương là $\overrightarrow{DE} =(6;0;0)$, $\overrightarrow{DN} = (0;2;2)$. Ta có $\left[ \overrightarrow{DE}, \overrightarrow{DN} \right] =(0;-12;12)$, suy ra $(DEMN)$ có vectơ pháp tuyến là $$\overrightarrow{n} = -\dfrac{1}{12} \left[ \overrightarrow{DE}, \overrightarrow{DN} \right] = (0;1;-1).$$
			Phương trình của mặt phẳng $(DEMN)$ là $y-z+4=0$.
			\item $B(6;4;0)$, suy ra $\mathrm{d} (B,(DEMN)) = \dfrac{\left| 4+4 \right|}{\sqrt{0^2+1^2+(-1)^2}} = \dfrac{8}{\sqrt{2}} = 4\sqrt{2}$.
		\end{enumerate}
	}
\end{vd}
\dongcham{7}
\begin{vd}
	Cho hình hộp chữ nhật $ABCD.A'B'C'D'$ có $DA=2$, $DC=3$, $DD'=2$. Tính khoảng cách từ đỉnh $B'$ đến mặt phẳng $(BA'C')$.
	\loigiai{
		\begin{center}
			\begin{tikzpicture}[scale=1, font=\footnotesize, line join=round, line cap=round, >=stealth]
				\path 
				(0,0) coordinate (D)
				($(D)+(-135:2)$) coordinate (A)
				($(A)+(-135:1)$) coordinate (x)
				(0:4.5) coordinate (C)
				($(C)+(0:1)$) coordinate (y)
				($(A)+(C)-(D)$) coordinate (B)
				($(D)+(90:3)$) coordinate (D')
				($(D')+(90:1)$) coordinate (z)
				($(A)+(90:3)$) coordinate (A')
				($(B)+(90:3)$) coordinate (B')
				($(C)+(90:3)$) coordinate (C')
				;
				\draw[dashed] (A)--(D)--(C)--cycle (D)--(D');
				\draw (A')--(A)--(B)--(B')--(A')--(B')--(C')--(D') (B)--(C)--(C')--(B') (D')--(A')--(C')--(B);
				\draw[->] (A')--(B);
				\draw[->] (A)--(x) node[below]{$x$};
				\draw[->] (C)--(y) node[below]{$y$};
				\draw[->] (D')--(z) node[left]{$z$};
				\foreach \x/\g in {A/170,B/-15,C/-60,D/180,D'/180,A'/180,B'/90,C'/12}
				\draw	(\x)
				($(\g:.2)+(\x)$) node {$\x$};	
			\end{tikzpicture}
		\end{center}
		Chọn hệ tọa độ $Oxyz$ sao cho gốc tọa độ $O$ trùng với điểm $D$.\\
		Khi đó, tọa độ các đỉnh của hình hộp chữ nhật $ABCD.A'B'C'D'$ là 
		$D(0,0,0)$, $A(2,0,0)$, $C(0,3,0)$, $B(2,3,0)$, $D'(0,0,2)$, $A'(2,0,2)$, $B'(2,3,2)$, $C'(0,3,2)$.
		Mặt phẳng $(BA'C')$ có cặp vectơ chỉ phương là
		$\overrightarrow{BA'}=(0;-3;2)$, $\overrightarrow{BC'}=(-2;0;2)$. \\
		Ta có $\left[ \overrightarrow{BA'}, \overrightarrow{BC'} \right] =(-6;-4;-6)$, suy ra $(BA'C')$ có vectơ pháp tuyến là 
		$$\overrightarrow{n} = -\dfrac{1}{2} \left[ \overrightarrow{BA'}, \overrightarrow{BC'} \right] = (3;2;3).$$\\
		Phương trình của $(BA'C')$ là
		$$3(x-2)+2(y-3)+3z=0 \text{ hay } 3x+2y+3z-12=0.$$
		Khoảng cách từ đỉnh $B'$ đến mặt phẳng $(BA'C')$ là
		$$\mathrm{d} (B', (BA'C')) = \dfrac{\left| 3 \cdot 2 + 2 \cdot 3 + 3 \cdot 2 - 12 \right|}{\sqrt{3^2+2^2+3^2}}=\dfrac{6}{\sqrt{22}}=\dfrac{3 \sqrt{22}}{11}.$$
	} 
\end{vd}
\dongcham{10}
\boxmini{BÀI TẬP TRẮC NGHIỆM}

\begin{ex}
	Khoảng cách từ $ A(-2;1;-6) $ đến mặt phẳng $ (Oxy) $ là 
	\choice
	{\True $ 6 $}
	{$ 2 $}
	{$ 1 $}
	{$ \dfrac{7}{\sqrt{41}} $}
	\loigiai{
		Ta có $ (Oxy) \colon z=0 $. Ta được $ d(A,(Oxy)) = \dfrac{|-6|}{1} = 6 $.
	}	
\end{ex}
\cham{3}
\begin{ex}
	Cho hai điểm $A(-2;1;3)$, $B(4;1;-1)$. Khoảng cách từ trung điểm $I$ của đoạn $AB$ đến mặt phẳng $(Oyz)$ là
	\choice
	{$0$}
	{$2$}
	{$4$}
	{\True $1$}
	\loigiai{
		Ta có trung điểm của đoạn $AB$ là $I(1;1;1)$ nên $\mathrm{d}(I,(Oyz))=|x_I|=1$.
	}
\end{ex}


\begin{ex}%[2H3Y2-6]%
	Cho mặt phẳng $(P)\colon 2x+3y+4z-5=0$ và điểm $A(1;-3;1)$. Khoảng cách từ điểm $A$ đến mặt phẳng $(P)$ bằng
	\choice
	{\True $\dfrac{8}{\sqrt{29}}$}
	{$\dfrac{8}{9}$}
	{$\dfrac{3}{\sqrt{29}}$}
	{$\dfrac{8}{29}$}
	\loigiai{
		Ta có
		\[\mathrm{d}(A, (P))=\dfrac{|2\cdot 1+3\cdot (-3)+4\cdot 1-5|}{\sqrt{2^2+3^2+4^2}}=\dfrac{8}{\sqrt{29}}.\]
	}
\end{ex}

\begin{ex}
	Gọi $H$ là hình chiếu vuông góc của điểm $A(2;3;-1)$ trên mặt phẳng $(\alpha)\colon 16x+12y-15z+7=0$. Tính độ dài đoạn thẳng $AH$.
	\choice
	{$\dfrac{19}{25}$}
	{\True $\dfrac{12}{25}$}
	{$\dfrac{19}{625}$}
	{$\dfrac{12}{625}$}
	\loigiai{
		Độ dài đoạn thẳng $AH$ bằng $\mathrm{d}\left(A;(\alpha)\right)=\dfrac{|16\cdot 2+12\cdot (-3)-15\cdot 1+7|}{\sqrt{16^2+12^2+(-15)^2}}=\dfrac{12}{25}$.
	}
\end{ex}

\begin{ex}%[2H3B2-6]
	Cho hai mặt phẳng $(P)\colon x+2y-2z+3=0$ và $(Q)\colon x+2y-2z-1=0$. Khoảng cách giữa hai mặt phẳng $(P)$ và $(Q)$ là 
	\choice 
	{$\dfrac{4}{9}$}
	{$\dfrac{2}{3}$}
	{\True $\dfrac{4}{3}$}
	{$-\dfrac{4}{3}$}
	
	\loigiai{
		Lấy $M(-3;0;0)\in (P)$. Vì $(P)\parallel (Q)$ nên khoảng cách giữa hai mặt phẳng $(P)$ và $(Q)$ bằng khoảng cách từ điểm $M$ đến mặt phẳng $(Q)$.\\
		Ta có $\mathrm{d}(M,(Q))=\dfrac{|x_M+2y_M-2z_M-1|}{\sqrt{1^2+2^2+(-2)^2}}=\dfrac{4}{3}$.
	}
\end{ex}
\cham{4}

\begin{ex}
	Biết rằng hai mặt phẳng $4x-4y+2z-7=0$ và $2x-2y+z+4=0$ chứa hai mặt của hình lập phương. Thể tích khối lập phương đó bằng
	\choice
	{$V=\dfrac{9 \sqrt{3}}{2}$}
	{$V=\dfrac{27}{8}$}
	{$V=\dfrac{81 \sqrt{3}}{8}$}
	{\True $V=\dfrac{125}{8}$}
	\loigiai{
		Khoảng cách giữa hai mặt phẳng trên bằng độ dài cạnh của hình lập phương.\\
		Gọi $(P)\colon 4x-4y+2z-7=0$ và $(Q)\colon 2x-2y+z+4=0$.\\
		Lây $M(0;0;-4) \in (Q)$ và $\mathrm{d}(M,(P))=\dfrac{5}{2}$.\\
		Vậy $V=\dfrac{125}{8}$.
	}
\end{ex}

\begin{ex}
	Cho hai điểm $A(2;2;-2)$ và $B(3;-1;0)$. Đường thẳng $AB$ cắt mặt phẳng $(P)\colon x+y-z+2=0$ tại điểm $I$. Tỉ số $\dfrac{IA}{IB}$ bằng
	\choice
	{\True $2$}
	{$4$}
	{$6$}
	{$3$}
	\loigiai{
		Ta có $\dfrac{IA}{IB}=\dfrac{d(A,(P))}{d(B,(P))}=\dfrac{8}{\sqrt{3}} : \dfrac{4}{\sqrt{3}}=2$.
	}
\end{ex}
\cham{4}

\begin{ex}%[2H3B2-6]
	Cho hai mặt phẳng $(P) \colon x+y-z+1=0$ và $(Q) \colon x-y+z-5=0.$ Có bao nhiêu điểm $M$ trên trục $Oy$ thỏa mãn $M$ cách đều hai mặt phẳng $(P)$ và $(Q)$?
	\choice
	{$0$}
	{\True $1$}
	{$2$}
	{$3$}
	\loigiai{
		Vì $M\in Oy$ nên $M(0;y;0).$\\
		Ta có $\mathrm{d}(M;(P))=\dfrac{|y+1|}{\sqrt{3}}$ và $\mathrm{d}(M;(Q))=\dfrac{|-y-5|}{\sqrt{3}}.$\\
		Theo giả thiết $\mathrm{d}(M;(P))=\mathrm{d}(M;(Q))\Leftrightarrow |y+1|=|-y-5|\Leftrightarrow \hoac{&y+1=-y-5\\&y+1=y=5}\Leftrightarrow \hoac{&y=-3\\& 0y=4 \,(\text{vô nghiệm})}$\\
		$\Rightarrow M(0;-3;0).$\\
		Vậy có $1$ điểm $M$ thỏa mãn bài.
	}
\end{ex}
\cham{6}
\begin{ex}%[2H3B2-3]
	Cho điểm $A(1;2;3)$ và mặt phẳng $(P)\colon x+y+z-2=0$. Mặt phẳng $(Q)$ song song với mặt phẳng $(P)$ và $(Q$) cách điểm $A$ một khoảng bằng $3\sqrt{3}$. Phương trình mặt phẳng $(Q)$ là
	\choice
	{$x+y+z+3=0$ và $x+y+z-3=0$}
	{$x+y+z+3=0$ và $x+y+z+15=0$}
	{\True $x+y+z+3=0$ và $x+y+z-15=0$}
	{$x+y+z+3=0$ và $x+y-z-15=0$}
	\loigiai{
		Do $(Q)\parallel (P)\Rightarrow (Q)\colon x+y+z+d=0,\quad d\neq -2$.\\
		Mà $d\left(A,(Q)\right)=3\sqrt{3}\Leftrightarrow |6+d|=9 \Leftrightarrow \hoac{&d=3\\&d=-15.}$\\
		Vậy $(Q_1)\colon x+y+z+3=0$ và $(Q_2)\colon x+y+z-15=0$.
		
	} 
\end{ex}
\cham{6}
\begin{ex}
	Cho mặt phẳng $ (P)\colon x+2y+z-4=0 $ và điểm $ D(1;0;3) $. Mặt phẳng $ (Q) $ song song với $ (P) $ và cách $ D $ một khoảng bằng $ \sqrt{6} $ có phương trình là
	\choice
	{$ \hoac{&x+2y-z-10=0\\&x+2y-z+2=0} $}
	{$ x+2y+z+2=0 $}
	{\True $ \hoac{&x+2y+z+2=0\\&x+2y+z-10=0} $}
	{$ x+2y+z-10=0 $}
	\loigiai{
		Vì $(Q)\parallel (P)$ nên $(Q)$ có phương trình dạng $(Q)\colon x+2y+z+D=0$ $(D\neq -4)$.\\
		Lại có $\mathrm{d}(D,(Q))=\sqrt{6}\Leftrightarrow \dfrac{|1+3+D|}{\sqrt{1+1+4}}=\sqrt{6}\Leftrightarrow |D+4|=6\Leftrightarrow \hoac{&D=2\\&D=-10}$.\\
		Vậy $(Q)\colon x+2y+z+2=0$ hoặc $(Q)\colon x+2y+z-10=0$.
	}
\end{ex}

\begin{ex}
	Cho hình chóp $S.ABCD$ có đáy $ABCD$ là hình chữ nhật. Biết $A(0; 0; 0), D(2; 0; 0), B(0; 4; 0), S(0; 0; 4)$. Gọi $M$ là trung điểm của $SB$. Tính khoảng cách từ $B$ đến mặt phẳng $(CDM)$.
	\choice
	{$d(B,(CDM))=\sqrt{2}$}
	{$d(B,(CDM))=2$}
	{$d(B,(CDM))=\dfrac{1}{\sqrt{2}}$}
	{\True $d(B,(CDM))=2\sqrt{2}$}
	\loigiai
	{
		Do $ABCD$ là hình chữ nhật nên $C\left(2;4;0\right)$. Và $M$ là trung điểm $SB$ nên $M\left(0;2;2\right)$.\\
		Phương trình mặt phẳng $\left(CDM\right)$ đi qua $M$ và nhận $\vec{n}=\left[\overrightarrow{MC},\overrightarrow{MD}\right]=\left(-8;0;-8\right)$ làm vectơ pháp tuyến là $x+z-8=0$.\\
		Khi đó $\mathrm{d}\left(B,\left(MCD\right)\right)=\dfrac{|4-8|}{\sqrt{1^2+0^2+1^2}}=2\sqrt{2}$.
	}
\end{ex}

\begin{ex}%[Thi thử, Chuyên Phan Bội Châu - Nghệ An, 2019-L1]%[Duong Xuan Loi, 12-EX-5-2019]%[2H3B4-1]
	Cho hình lập phương $ABCD.A'B'C'D'$ có cạnh bằng $2$. Khoảng cách giữa hai mặt phẳng $(AB'D')$ và $(BC'D)$ bằng
	\choice
	{$\dfrac{\sqrt{3}}{3}$}
	{\True $\dfrac{2\sqrt{3}}{3}$}
	{$\dfrac{\sqrt{3}}{2}$}
	{$\sqrt{3}$}
	\loigiai{
		\immini{
			Chọn hệ trục toạ độ như hình vẽ.\\
			Ta có $A(0;0;0),B(2;0;0),C(2;2;0),D(0;2;0)$.\\
			$A'(0;0;2),B'(2;0;2),C'(2;2;2),D'(0;2;2)$.\\
			Mặt phẳng $(AB'D')$ qua $A$ và có một vectơ pháp tuyến là $-\dfrac{1}{4}\left[\overrightarrow{AB'}, \overrightarrow{AD'}\right]=(1;1;-1)$ nên có phương trình $x+y-z=0.$				
		}{
			\begin{tikzpicture}[scale=0.8, font=\footnotesize, line join=round, line cap=round, >=stealth]
				\tkzDefPoints{0/0/A,-1.1/-1.1/B,2/-1.1/C}
				\coordinate (D) at ($(A)+(C)-(B)$);
				\coordinate (A') at ($(A)+(0,2.5)$);
				\coordinate (x) at ($(A)!1.5!(B)$);
				\coordinate (y) at ($(A)!1.3!(D)$);
				\coordinate (z) at ($(A)!1.4!(A')$);
				\tkzDefPointsBy[translation=from A to A'](B,C,D){B'}{C'}{D'}
				\tkzDrawPolygon(A',B',B,C,D,D')
				\tkzDrawSegments(B',C' C',D' C,C' B',D' C',B C',D)
				\tkzDrawSegments[dashed](A,B A,D A,A' B,D A,B' A,D')
				\tkzDrawPoints[fill=black](A,B,D,C,A',B',C',D')
				\tkzLabelPoints[above](D')
				\tkzLabelPoints[below](C,D)
				\tkzLabelPoints[above left](A')
				\tkzLabelPoints[left](A,B',B)
				\tkzLabelPoints[right](C')
				\draw[->] (B)--(x)node[right]{$x$};
				\draw[->] (D)--(y)node[above]{$y$};
				\draw[->] (A')--(z)node[right]{$z$};
			\end{tikzpicture}
		}\noindent
		Mặt phẳng $\left(BC'D\right)$ qua $B$ và có một vectơ pháp tuyến là $-\dfrac{1}{4}\left[\overrightarrow{BC'}, \overrightarrow{BD}\right]=(1;1;-1)$ nên có phương trình $x+y-z-2=0$.\\
		Ta có $(AB'D')\parallel (BC'D)$ nên
		$$\mathrm{d}((AB'D'),(BC'D))=\mathrm{d}(A,(BC'D))=\dfrac{|-2|}{\sqrt{1^2+1^2+t(-1)^2}}=\dfrac{2\sqrt{3}}{3}.$$
	}
\end{ex}

\begin{ex}%[Thi thử L1, Star Education HCM, 2019]%[Nguyễn Ngọc Dũng, dự án 12EX6]%[2H3K4-1]
	Cho hình hộp chữ nhật $ABCD.A’B’C’D’$ có $AB=a$, $AD=2a$, $AA'=3a$. Gọi $M$, $N$, $P$ lần lượt là trung điểm của $BC$, $C’D’$ và $DD’$. Tính khoảng cách từ $A$ đến $(MNP)$.
	\choice
	{\True $ \dfrac{15}{11}a $}
	{$\dfrac{15}{22}a$}
	{$\dfrac{9}{11}a$}
	{$\dfrac{3}{4}a$}
	\loigiai{
		\immini{
			Gán hệ trục tọa độ như hình vẽ với độ dài đơn vị trên trục là $a$. Khi đó, ta tính được tọa độ các điểm như sau
			$$A(0;0;0), M(1;1;0), N\left(2;\dfrac{1}{2}; 3\right), P\left( 2;0;\dfrac{3}{2} \right).$$
			Ta có $\overrightarrow{MN} = \left( 1;-\dfrac{1}{2};3\right)$ và $\overrightarrow{MP} = \left( 1;-1;\dfrac{3}{2}\right)$. \\
			Chọn $\left[ \overrightarrow{MN}, \overrightarrow{MP}\right] = \left( \dfrac{9}{4}; \dfrac{3}{2}; -\dfrac{1}{2}\right)$ là vtpt của $(MNP)$.\\
			Suy ra $(MNP)\colon 9x + 6y - 2z - 15 = 0$.\\
			Do đó $\mathrm{d}(A,(MNP)) = \dfrac{15}{11}$.\\
			Vậy $\mathrm{d}(A,(MNP)) = \dfrac{15a}{11}$.	
		}{
			\begin{tikzpicture}[scale=0.8, font=\footnotesize, line join=round, line cap=round, >=stealth]
				\tikzset{label style/.style={font=\footnotesize}}
				\def\h{4} \def\r{5} \def\x{2.2} \def\y{1.5}
				\coordinate[label={below}:$B$] (A) at (-3,-3);
				\coordinate[label={below,xshift=2mm}:{$A\equiv O$}] (B) at ($(A)+(\x,\y)$);
				\coordinate[label={below right}:$D$] (C) at ($(B)+(\r,0)$);
				\coordinate[label={below right}:$C$] (D) at ($(A)+(\r,0)$);
				\coordinate[label={above left}:$B'$] (A') at ($(A)+(0,\h)$);
				\coordinate[label={above left}:$A'$] (B') at ($(A')+(\x,\y)$);
				\coordinate[label={above right}:$D'$] (C') at ($(B')+(\r,0)$);
				\coordinate[label={above}:$C'$] (D') at ($(A')+(\r,0)$);
				\coordinate[label={above}:{$x$}] (x) at ($(B)!1.2!(C)$);
				\coordinate[label={below}:{$y$}] (y) at ($(B)!1.2!(A)$);
				\coordinate[label={right}:{$z$}] (z) at ($(B)!1.2!(B')$);
				
				\coordinate[label={below}:{$M$}] (M) at ($(A)!0.5!(D)$);
				\coordinate[label={above}:{$N$}] (N) at ($(C')!0.5!(D')$);
				\coordinate[label={right}:{$P$}] (P) at ($(C)!0.5!(C')$);
				
				\draw (A)--(A') (C)--(C') (D)--(D') (A)--(D)--(C) (A')--(B')--(C')--(D')--(A') (P)--(N);
				\draw[dashed] (B)--(B') (A)--(B)--(C) (P)--(M)--(N);
				\draw[->] (C)--(x);
				\draw[->] (B')--(z);
				\draw[->] (A)--(y);
				
				\tkzLabelSegment[below](B,C){$2a$}
				\tkzLabelSegment[right](D,C){$a$}
				\tkzLabelSegment[left](A,A'){$3a$}
				\tkzDrawPoints[fill=black](A,B,C,D,A',B',C',D',M,N,P)
			\end{tikzpicture}
		}
	}
\end{ex}

\begin{ex}%[1H3B5-2]
	Cho hình chóp $S.ABCD$ có đáy $ABCD$ là hình vuông cạnh $a$, $SA\perp (ABCD)$ và $SA=a\sqrt{3}$. Tính khoảng cách từ điểm $B$ đến mặt phẳng $(SCD)$.
	\choice
	{\True $\dfrac{a\sqrt{3}}{2}$}
	{$\dfrac{a\sqrt{2}}{4}$}
	{$\dfrac{a\sqrt{2}}{3}$}
	{$\dfrac{a}{2}$}
	\loigiai{
	\immini{
		Chuẩn hóa $a=1$. Với hệ trục đã chọn như hình vẽ thì $B(1;0;0)$, $S(0;0;\sqrt{3})$, $C(1;1;0)$, $D(0;1;0)$.\\
		Ta có 
		\begin{enumEX}[]{1}
			\item $\vec{CD}=(-1;0;0)$, $\vec{CS}=(-1;-1;\sqrt{3})$,
			\item $\vec{CB}=(0;-1;0)$; $[\vec{CD},\vec{CS}]=(0;\sqrt{3};1)$
		\end{enumEX}
			Khoảng cách từ điểm $B$ đến $(SCD)$ được tính theo công thức:
		$$d=\dfrac{\big|[\vec{CD},\vec{CS}]\cdot \vec{CB}\big|}{\big|[\vec{CD},\vec{CS}]\big|}=\dfrac{\sqrt{3}}{2}$$
		}{
		\begin{tikzpicture}[scale=0.6, font=\footnotesize,>=stealth]
			\path
			(0,0) coordinate (A)
			(-2,-2) coordinate (B)
			(5,0) coordinate (D)
			($(B)+(D)-(A)$)coordinate (C)
			%($(A)!0.5!(C)$)coordinate (I)
			($(A)+(0,3)$)coordinate (S)
			;
			\draw[->] (D)--(7,0) node[below]{$y$};
			\draw[->] (B)--(-3,-3) node[below]{$x$};
			\draw[->] (S)--(0,4) node[left]{$z$};
			\draw (C)--(D)--(S)--(C)--(B)--(S);
			\draw[dashed] (S)--(A)--(D) (B)--(A);
			\pic[draw,thin,angle radius=2mm] {right angle = B--A--D};
			\foreach \x/\g in {A/180,B/-90,C/-100,D/-80,S/170}\draw[fill=black] (\x) circle (.04) +(\g:.5)node{\footnotesize$\x$};
	\end{tikzpicture}}	
	}
\end{ex}

\begin{ex}
	Cho hình chóp $S. ABCD$ có đáy $ABCD$ là hình vuông cạnh $a$, $SD=\dfrac{3a}{2}$, hình chiếu vuông góc của $S$ lên mặt phẳng $(ABCD)$ là trung điểm của cạnh $AB$. Tính khoảng cách $d$ từ $A$ đến mặt phẳng $(SBD)$. 
	\choice
	{$d=\dfrac{a}{3}$}
	{$d=\dfrac{a}{6}$}
	{$d=\dfrac{3a}{2}$}
	{\True $d=\dfrac{2a}{3}$}
	\loigiai{
		
	}
\end{ex}

\subsection{BÀI TẬP TRẮC NGHIỆM TỰ LUYỆN}
\TN
	\setcounter{ex}{0}
	\Opensolutionfile{ans}[ans/B1-De2-1]

\begin{ex}%[2H5N1-4]
	Trong không gian với hệ tọa độ $Oxyz$, cho mặt phẳng $(P)\colon x-z+3=0$. Mặt phẳng nào sau đây vuông góc với mặt phẳng $(P)$?
	\choice
	{\True $(\alpha)\colon 2x-y+2z=0$}
	{$(\beta)\colon 2x-y-2z=0$}
	{$(Q)\colon -2x-y+2z=0$}
	{$(R)\colon 2x+y-2z=0$}
	\loigiai{
		Mặt phẳng $(P)$ có véc-tơ pháp tuyến là $\overrightarrow{n}_{(P)}=(1;0;-1)$.\\
		Mặt phẳng $(\alpha)$ có véc-tơ pháp tuyến là $\overrightarrow{n}_{(\alpha)}=(2;-1;2)$.\\
		Ta có $\overrightarrow{n}_{(P)}\cdot \overrightarrow{n}_{(\alpha)}=1\cdot 2+0\cdot (-1)+(-1)\cdot 2=0$. Do đó, $(P)\perp (\alpha)$.
	}
\end{ex}

%G:\My Drive\CODE12-2024\DE-ON-THEO BAI\2H5-TACH DE\Bai1-De2.tex
\begin{ex}%[2H5N1-5]
	Trong không gian với hệ tọa độ $Oxyz$, cho điểm $A(1;3;-2)$ và mặt phẳng $(P)\colon 2x+y-2z-3=0$. Khoảng cách từ điểm $A$ đến mặt phẳng $(P)$ bằng
	\choice
	{\True $2$}
	{$1$}
	{$\dfrac{2}{3}$}
	{$3$}
	\loigiai{
		Ta có $\mathrm{d}\big(A,(P)\big)=\dfrac{\left|2\cdot 1+3-2\cdot (-2)-3\right|}{\sqrt{2^2+1^2+(-2)^2}}=2$.
	}
\end{ex}

%G:\My Drive\CODE12-2024\DE-ON-THEO BAI\2H5-TACH DE\Bai1-De2.tex
\begin{ex}%[2H5N1-2]
	Trong không gian với hệ tọa độ $Oxyz$, cho mặt phẳng $(P)\colon x-y+3=0$. Véc-tơ nào sau đây \textbf{không phải} là véc-tơ pháp tuyến của mặt phẳng $(P)$?
	\choice
	{$\overrightarrow{a}=(3;-3;0)$}
	{$\overrightarrow{a}=(1;-1;0)$}
	{\True $\overrightarrow{a}=(1;-1;3)$}
	{$\overrightarrow{a}=(-1;1;0)$}
	\loigiai{
		Véc-tơ pháp tuyến của mặt phẳng $(P)$ là $\overrightarrow{n}=(1;-1;0)$.\\
		Ta có $\overrightarrow{a}=(-1;1;0)=-(1;-1;0)=-\overrightarrow{n}$. Vậy $\overrightarrow{a}=(-1;1;0)$ là một véc-tơ pháp tuyến của mặt phẳng $(P)$.\\
		Tương tự $\overrightarrow{a}=(3;-3;0)=3(1;-1;0)=3\overrightarrow{n}$. Vậy $\overrightarrow{a}=(3;-3;0)$ là một véc-tơ pháp tuyến của mặt phẳng $(P)$.\\
		Do véc-tơ $\overrightarrow{a}=(1;-1;3)$ không cùng phương với véc-tơ $\overrightarrow{n}=(1;-1;0)$. Nên $\overrightarrow{a}=(1;-1;3)$ không là véc-tơ pháp tuyến của mặt phẳng $(P)$.
	}
\end{ex}

%G:\My Drive\CODE12-2024\DE-ON-THEO BAI\2H5-TACH DE\Bai1-De2.tex
\begin{ex}%[2H5H1-3]
	Trong không gian với hệ tọa độ $Oxyz$ với điểm $M(-3;1;4)$ và gọi $A$, $B$, $C$ lần lượt là hình chiếu của $M$ lên các trục $Ox$, $Oy$, $Oz$. Phương trình nào dưới đây là phương trình mặt phẳng song song với mặt phẳng $(ABC)$?
	\choice
	{$4x-12y+3z-12=0$}
	{$4x-12y-3z+12=0$}
	{$4x+12y-3z-12=0$}
	{\True $4x-12y-3z-12=0$}
	\loigiai{
		Vì $A$, $B$, $C$ lần lượt là hình chiếu của $M(-3;1;4)$ các trục $Ox$, $Oy$, $Oz$ nên $A(-3;0;0)$, $B(0;1;0)$, $C(0;0;4)$.\\
		Phương trình mặt phẳng $(ABC)\colon \dfrac{x}{-3}+\dfrac{y}{1}+\dfrac{z}{4}=1$ $\Leftrightarrow$ $4x-12y-3z+12=0$.\\
		Vậy mặt phẳng $4x-12y-3z-12=0$ song song với mặt phẳng $(ABC)$.
	}
\end{ex}

%G:\My Drive\CODE12-2024\DE-ON-THEO BAI\2H5-TACH DE\Bai1-De2.tex
\begin{ex}%[2H5H1-2]
	Trong không gian với hệ tọa độ $Oxyz$, cho ba điểm $A(2;0;0)$, $B(0;-3;0)$, $C(0;0;1)$. Một véc-tơ pháp tuyến của mặt phẳng $(ABC)$ là
	\choice
	{\True $\overrightarrow{n}=(3;-2;6)$}
	{$\overrightarrow{n}=(2;-3;-1)$}
	{$\overrightarrow{n}=(2;3;1)$}
	{$\overrightarrow{n}=(2;-3;1)$}
	\loigiai{
		Phương trình mặt phẳng $(ABC)\colon \dfrac{x}{2}+\dfrac{y}{-3}+\dfrac{z}{1}=1$ $\Leftrightarrow$ $3x-2y+6z-6=0$.\\
		Vậy mặt phẳng $(ABC)$ có một véc-tơ pháp tuyến $\overrightarrow{n}=(3;-2;6)$.
	}
\end{ex}

%G:\My Drive\CODE12-2024\DE-ON-THEO BAI\2H5-TACH DE\Bai1-De2.tex
\begin{ex}%[2H2N2-2]
	Trong không gian với hệ tọa độ $Oxyz$, cho điểm $M(2024;0;-1)$. Mệnh đề nào dưới đây đúng?
	\choice
	{\True $M\in(Oxz)$}
	{$M\in Oy$}
	{$M\in(Oxy)$}
	{$M\in(Oyz)$}
	\loigiai{
		Do tung độ của điểm $M(2024;0;-1)$ bằng $0$ nên $M\in(Oxz)$.
	}
\end{ex}

%G:\My Drive\CODE12-2024\DE-ON-THEO BAI\2H5-TACH DE\Bai1-De2.tex
\begin{ex}%[2H5H1-3]
	Trong không gian với hệ tọa độ $Oxyz$, cho ba điểm $A(-2;0;0)$, $B(0;3;0)$ và $C(0;0;4)$. Mặt phẳng $(ABC)$ có phương trình là
	\choice
	{\True $\dfrac{x}{-2}+\dfrac{y}{3}+\dfrac{z}{4}=1$}
	{$\dfrac{x}{2}+\dfrac{y}{3}+\dfrac{z}{-4}=1$}
	{$\dfrac{x}{2}+\dfrac{y}{-3}+\dfrac{z}{4}=1$}
	{$\dfrac{x}{2}+\dfrac{y}{3}+\dfrac{z}{4}=1$}
	\loigiai{
		Vì ba điểm $A(-2;0;0)$, $B(0;3;0)$ và $C(0;0;4)$ lần lượt nằm trên các trục tọa độ $Ox$, $Oy$, $Oz$ và không trùng với gốc tọa độ $O$ nên mặt phẳng $(ABC)$ có phương trình là
		\begin{align*}
			\dfrac{x}{-2}+\dfrac{y}{3}+\dfrac{z}{4}=1.
		\end{align*}
	}
\end{ex}

%G:\My Drive\CODE12-2024\DE-ON-THEO BAI\2H5-TACH DE\Bai1-De2.tex
\begin{ex}%[2H5H1-5]
	Trong không gian với hệ tọa độ $Oxyz$, cho mặt phẳng $(\alpha)\colon 2x+2y-z+m=0$ ($m$ là tham số). Tìm giá trị của tham số $m$ dương để khoảng cách từ gốc tọa độ đến mặt phẳng $(\alpha)$ bằng $1$.
	\choice
	{$-6$}
	{$-3$}
	{\True $3$}
	{$6$}
	\loigiai{
		Ta có $\mathrm{d}(O,(\alpha))=\dfrac{\left| m\right|}{3}=1$ $\Leftrightarrow$ $\left| m\right|=3$ $\Leftrightarrow$ $m=\pm 3$.\\
		Do $m>0$ nên $m=3$.
	}
\end{ex}

%G:\My Drive\CODE12-2024\DE-ON-THEO BAI\2H5-TACH DE\Bai1-De2.tex
\begin{ex}%[2H2N2-2]
	Trong không gian với hệ tọa độ $Oxyz$, hình chiếu vuông góc của điểm $M(2;3;-4)$ trên mặt phẳng $(Oyz)$ có tọa độ là
	\choice
	{$(2;0;-4)$}
	{\True $(0;3;-4)$}
	{$(2;3;0)$}
	{$(0;3;0)$}
	\loigiai{
		Trong không gian với hệ tọa độ $Oxyz$, hình chiếu vuông góc của điểm $M_0(x_0;y_0;z_0)$ trên mặt phẳng $(Oyz)$ có tọa độ là $(0;y_0;z_0)$.\\
		Do đó, hình chiếu vuông góc của điểm $M(2;3;-4)$ trên mặt phẳng $(Oyz)$ có tọa độ là $(0;3;-4)$.
	}
\end{ex}

%G:\My Drive\CODE12-2024\DE-ON-THEO BAI\2H5-TACH DE\Bai1-De2.tex
\begin{ex}%[2H5H1-3]
	Trong không gian với hệ tọa độ $Oxyz$, cho hai điểm $A(1;2;-1)$, $B(-1;0;1)$ và mặt phẳng $(P)\colon x+2y-z+1=0$. Viết phương trình mặt phẳng $(Q)$ qua $A$, $B$ và vuông góc với $(P)$.
	\choice
	{$(Q)\colon 3x-y+z=0$}
	{\True $(Q)\colon x+z=0$}
	{$(Q)\colon 2x-y+3=0$}
	{$(Q)\colon -x+y+z=0$}
	\loigiai{
		Ta có $\overrightarrow{AB}=(-2;-2;2)$, $\overrightarrow{n}_P=(1;2;-1)$.\\
		$\left[\overrightarrow{AB},\overrightarrow{n}_P\right]=(-2;0;-2)$.\\
		Mặt phẳng $(Q)$ đi qua $A$, $B$ và vuông góc với $(P)$ nhận véc-tơ $\overrightarrow{n}_Q=\left[\overrightarrow{AB},\overrightarrow{n}_P\right]$ là véc-tơ pháp tuyến.\\
		Mặt phẳng $(Q)$ có phương trình là
		\begin{align*}
			-2(x-1)+0(x-2)-2(z+1)=0 \Leftrightarrow -2x-2z=0 \Leftrightarrow x+z=0.
		\end{align*}
	}
\end{ex}

%G:\My Drive\CODE12-2024\DE-ON-THEO BAI\2H5-TACH DE\Bai1-De2.tex
\begin{ex}%[2H5N1-2]
	Trong không gian với hệ tọa độ $Oxyz$, cho hai véc-tơ $\overrightarrow{u}=(1;2;3)$, $\overrightarrow{v}=(0;-1;1)$. Mặt phẳng $(\alpha)$ đi qua điểm $A(1;2;5)$ và song song với giá của hai véc-tơ $\overrightarrow{u}$ và $\overrightarrow{v}$. Véc-tơ nào dưới đây là một véc-tơ pháp tuyến của mặt phẳng $(\alpha)$?
	\choice
	{$\overrightarrow{n}_3=(-1;-1;-1)$}
	{\True $\overrightarrow{n}_2=(5;-1;-1)$}
	{$\overrightarrow{n}_1=(5;1;-1)$}
	{$\overrightarrow{n}_4=(-1;-1;5)$}
	\loigiai{
		Vì mặt phẳng $(\alpha)$ song song với giá của hai véc-tơ $\overrightarrow{u}$ và $\overrightarrow{v}$ nên có một véc-tơ pháp tuyến là $\overrightarrow{n}_2=\left[\overrightarrow{u},\overrightarrow{v}\right]=(5;-1;-1)$.
	}
\end{ex}

%G:\My Drive\CODE12-2024\DE-ON-THEO BAI\2H5-TACH DE\Bai1-De2.tex
\begin{ex}%[2H5H1-3]
	Trong không gian với hệ tọa độ $Oxyz$, cho ba điểm $A(1;-1;0)$, $B(-1;0;1)$, $C(2;1;-1)$. Phương trình mặt phẳng $(ABC)$ là
	\choice
	{\True $3x+y+5z-2=0$}
	{$x+3y+z+2=0$}
	{$3x-y+5z-2=0$}
	{$3x+y+5z+2=0$}
	\loigiai{
		Mặt phẳng $(ABC)$ đi qua $A(1;-1;0)$ và nhận $\overrightarrow{n}=\left[\overrightarrow{AB},\overrightarrow{AC}\right]=(-3;-1;-5)=-(3;1;5)$ làm véc-tơ pháp tuyến có phương trình: $3(x-1)+1(y+1)+5(z-0)=0$.\\
		Do đó, $(ABC)\colon 3x+y+5z-2=0$.
	}
\end{ex}
	\Closesolutionfile{ans}

\TNTF
	\setcounter{ex}{0}
	\Opensolutionfile{ans}[ans/B1-De2-2]
\begin{ex}%[2H5V1-5]
	Trong không gian với hệ trục tọa độ $Oxyz$, cho hai điểm $A(1;2;3)$, $B(3;4;4)$ và mặt phẳng $(\alpha)\colon 2x+y+mz-1=0$. Các mệnh đề sau đúng hay sai?
	\choiceTF
	{\True Mặt phẳng đi qua $3$ điểm là hình chiếu vuông góc của $A(1;2;3)$ lên ba trục tọa độ có phương trình là $6x+3y+2z-6=0$}
	{Điểm $A$ cách đều mặt phẳng $(\gamma)\colon 2x+y+mz-1=0$ và điểm $B$ khi $m=-2$}
	{\True Biết mặt phẳng $(\beta)\colon 4x+(n-2)y+z-3=0$ song song với mặt phẳng $(\alpha)$. Khi đó, $2m+n=5$}
	{Khi $B\in (\alpha)\colon 2x+y+mz-1=0$ thì $m=-2$}
	\loigiai{
		\begin{itemchoice}
			\itemch \textbf{Đúng}.\\
			Ta có hình chiếu vuông góc của $A(1;2;3)$ lên ba trục tọa độ lần lượt $M(1;0;0)$, $N(0;2;0)$, $P(0;0;3)$.\\
			Do đó, phương trình mặt phẳng cần tìm là $\dfrac{x}{1}+\dfrac{y}{2}+\dfrac{z}{3}=1$ $\Leftrightarrow$ $6x+3y+2z-6=0$.
			\itemch \textbf{Sai}.\\
			Ta có $\overrightarrow{AB}=(2;2;1)$ $\Rightarrow$ $AB=\sqrt{2^2+2^2+1^2}=3$. \hfill$(1)$\\
			Khoảng cách từ $A$ đến mặt phẳng $(P)$ là
			\begin{align*}
				\mathrm{d}\big(A,(P)\big)=\dfrac{\left|2\cdot 1+2+m\cdot 3-1\right|}{\sqrt{2^2+1^2+ m^2}}=\dfrac{\left|3m+3\right|}{\sqrt{5+m^2}}. \tag{2}
			\end{align*}
			Do đó, $AB=\mathrm{d}\big(A,(P)\big)$ khi và chỉ khi
			\begin{align*}
				3=\dfrac{\left|3m+3\right|}{\sqrt{5+ m^2}} \Leftrightarrow 9(5+m^2)=9(m+1)^2 \Leftrightarrow m=2.
			\end{align*}
			\itemch \textbf{Đúng}.\\
			Vì $(\alpha)\parallel (\beta)$ nên $\dfrac{2}{4}=\dfrac{1}{n-2}=\dfrac{m}{1}\ne\dfrac{-1}{-3}$ $\Leftrightarrow$ $\heva{
				& m=\dfrac{1}{2}\\
				& n=4
			}$ $\Rightarrow$ $2m+n=5$.
			\itemch \textbf{Sai}.\\
			Ta có $B\in (\alpha)\colon 2x+y+mz-1=0$ $\Leftrightarrow$ $2\cdot 3+4+m\cdot 4-1=0$ $\Leftrightarrow$ $m=-\dfrac{9}{4}$.
		\end{itemchoice}
	}
\end{ex}

%G:\My Drive\CODE12-2024\DE-ON-THEO BAI\2H5-TACH DE\Bai1-De2.tex
\begin{ex}%[2H5H1-5]
	Trong không gian với hệ tọa độ $Oxyz$, cho $A(1;2;-1)$, $B(-1;0;1)$ và mặt phẳng $(P)\colon x+2y-z+1=0$. Các mệnh đề sau đúng hay sai?
	\choiceTF
	{\True Biết điểm $M$ nằm trên tia $Ox$ mà khoảng cách từ $M$ đến mặt phẳng $(P)$ bằng $\sqrt{6}$. Khi đó, hoành độ điểm $M$ là $x_M=5$}
	{\True Mặt phẳng $(Q)$ qua $A$, $B$ và vuông góc với $(P)$ có phương trình là $x+z=0$}
	{\True Mặt phẳng $(P)$ có một véc-tơ pháp tuyến là $(1;2;-1)$}
	{Khi $m=-4$ thì mặt phẳng $(R)\colon 2x-my+3=0$ vuông góc với mặt phẳng $(P)$}
	\loigiai{
		\begin{itemchoice}
			\itemch \textbf{Đúng}.\\
			$M$ nằm trên tia $Ox$ $\Rightarrow$ $M(x;0;0)$, $x>0$.\\
			Khi đó, khoảng cách từ $M$ đến mặt phẳng $(P)$ bằng $\sqrt{6}$ khi và chỉ khi
			\begin{align*}
				\dfrac{\left| x+1\right|}{\sqrt{6}}=\sqrt{6} \Leftrightarrow x=5, (x>0).
			\end{align*}
			\itemch \textbf{Đúng}.\\
			Ta có $\overrightarrow{AB}=(-2;-2;2)$, $\overrightarrow{u}=-\dfrac{1}{2}\overrightarrow{AB}=(1;1;-1)$; $\overrightarrow{n}_P=(1;2;-1)$.\\
			Do đó, $\overrightarrow{n}_Q=\left[\overrightarrow{u},\overrightarrow{n}_{(P)}\right]=(1;0;1)$ là một véc-tơ pháp tuyến của mặt phẳng $(Q)$.\\
			Vậy $(Q)\colon 1\cdot(x-1)+0\cdot(y-2)+1\cdot(z+1)=0 \Leftrightarrow x+z=0$.
			\itemch \textbf{Đúng}.\\
			Mặt phẳng $(P)\colon x+2y-z+1=0$ có một véc-tơ pháp tuyến là $(1;2;-1)$.
			\itemch \textbf{Sai}.\\
			$(P)$ có một véc-tơ pháp tuyến là $\overrightarrow{n}_P=(1;2;-1)$; $(R)$ có một véc-tơ pháp tuyến là $\overrightarrow{n}_R=(2;-m;0)$.\\
			Hai mặt phẳng $(P)$ và $(R)$ vuông góc với nhau khi và chỉ khi
			\begin{align*}
				\overrightarrow{n}_P\cdot \overrightarrow{n}_R=0 \Leftrightarrow 1\cdot 2+2\cdot (-m)+(-1)\cdot 0=0 \Leftrightarrow m=1.
			\end{align*}
		\end{itemchoice}
	}
\end{ex}

%G:\My Drive\CODE12-2024\DE-ON-THEO BAI\2H5-TACH DE\Bai1-De2.tex
\begin{ex}%[2H5V1-3]
	Trong không gian với hệ tọa độ $Oxyz$, cho hai điểm $A(1;2;1)$ và $B(3;-1;5)$. Các mệnh đề sau đúng hay sai?
	\choiceTF
	{\True Phương trình mặt phẳng trung trực của đoạn thẳng $AB$ là $2x-3y+4z-\dfrac{29}{2}=0$}
	{Điểm $N(1;2;-1)$ đối xứng với $A(1;2;1)$ qua mặt phẳng $(Oyz)$}
	{\True Mặt phẳng $(P)$ vuông góc với đường thẳng $AB$ và cắt các trục $Ox$, $Oy$ và $Oz$ lần lượt tại các điểm $D$, $E$ và $F$. Khi thể tích của tứ diện $ODEF$ bằng $\dfrac{3}{2}$, phương trình mặt phẳng $(P)$ là $2x-3y+4z\pm 6=0$}
	{Véc-tơ $\overrightarrow{AB}$ là một véc-tơ pháp tuyến của mặt phẳng $(\alpha)\colon 2x+3y+4z-2=0$}
	\loigiai{
		\begin{itemchoice}
			\itemch \textbf{Đúng}.\\
			Trung điểm của đoạn thẳng $AB$ là điểm $I\left(2;\dfrac{1}{2};3\right)$.\\
			Vậy mặt phẳng trung trực của $AB$ đi qua $I\left(2;\dfrac{1}{2};3\right)$ và nhận $\overrightarrow{AB}=(2;-3;4)$ làm véc-tơ pháp tuyến nên có phương trình là
			\begin{align*}
				2(x-2)-3\left(y-\dfrac{1}{2}\right)+4(z-3)=0 \Leftrightarrow 2x-3y+4z-\dfrac{29}{2}=0.
			\end{align*}
			\itemch \textbf{Sai}.\\
			Điểm đối xứng với $A(1;2;1)$ qua mặt phẳng $(Oyz)$ có tọa độ là $(-1;2;1)$.
			\itemch \textbf{Đúng}.\\
			Vì $AB\perp (P)$ nên mặt phẳng $(P)$ có một véc-tơ pháp tuyến là $\overrightarrow{AB}=(2;-3;4)$. Do đó, phương trình mặt phẳng $(P)$ có dạng $2x-3y+4z+d=0$.\\
			Từ đây tìm được $D\left(-\dfrac{d}{2};0;0\right)$, $E\left(0;\dfrac{d}{3};0\right)$, $F\left(0;0;-\dfrac{d}{4}\right)$ suy ra $OD=\dfrac{\left|d\right|}{2}$, $OE=\dfrac{\left|d\right|}{3}$, $OF=\dfrac{\left|d\right|}{4}$.\\
			Mặt khác, tứ diện $ODEF$ có $OD$, $OE$, $OF$ đôi một vuông góc nên thể tích của tứ diện $ODEF$ là $V_{ODEF}=\dfrac{1}{6}OD. OE. OF=\dfrac{(\left|d\right|)^3}{144}$.\\
			Do đó, $V_{ODEF}=\dfrac{3}{2}$ $\Leftrightarrow$ $\dfrac{(\left|d\right|)^3}{144}=\dfrac{3}{2}$ $\Leftrightarrow$ $\left|d\right|=6$ $\Leftrightarrow$ $d=\pm 6$.\\
			Vậy phương trình mặt phẳng $(P)$ là $2x-3y+4z\pm 6=0$.
			\itemch \textbf{Sai}.\\
			Véc-tơ $\overrightarrow{AB}=(2;-3;4)$ không cùng phương với $\overrightarrow{n}=(2;3;4)$ là véc-tơ pháp tuyến của mặt phẳng $(\alpha)$ nên $\overrightarrow{AB}$ không là một véc-tơ pháp tuyến của mặt phẳng $(\alpha)$.
		\end{itemchoice}
	}
\end{ex}

%G:\My Drive\CODE12-2024\DE-ON-THEO BAI\2H5-TACH DE\Bai1-De2.tex
\begin{ex}%[2H5V1-4]
	Trong không gian với hệ tọa độ $Oxyz$, cho hai mặt phẳng $(\alpha)\colon 3x-2y+2z+7=0$ và $(\beta)\colon 5x-4y+3z+1=0$. Các mệnh đề sau đúng hay sai?
	\choiceTF
	{Hai mặt phẳng $(\alpha)$, $(\beta)$ song song với nhau}
	{Điểm $A(1;2;-1)$ nằm trên mặt phẳng $(\alpha)\colon 3x-2y+2z+7=0$}
	{\True Phương trình mặt phẳng qua $O$, đồng thời vuông góc với cả $(\alpha)$ và $(\beta)$ có phương trình là $2x+y-2z=0$}
	{\True Mặt phẳng $(\gamma)$ đi qua điểm $I(1;0;-1)$ và song song với $(\alpha)\colon 3x-2y+2z+7=0$ có phương trình là $(\gamma)\colon 3x-2y+2z-1=0$}
	\loigiai{
		\begin{itemchoice}
			\itemch \textbf{Sai}.\\
			Ta có $\dfrac{3}{5}\ne\dfrac{-2}{-4}$ nên hai mp $(\alpha)$, $(\beta)$ cắt nhau. Vậy chúng không song song.
			\itemch \textbf{Sai}.\\
			Ta có $3\cdot 1-2\cdot 2+2\cdot (-1)+7\ne 0$ nên $A(1;2;-1)$ không nằm trên mặt phẳng $(\alpha)$.
			\itemch \textbf{Đúng}.\\
			Mặt phẳng $(\alpha)$ có một véc-tơ pháp tuyến là $\overrightarrow{n}_1=(3;-2;2)$.\\
			Mặt phẳng $(\beta)$ có một véc-tơ pháp tuyến là $\overrightarrow{n}_2=(5;-4;3)$.\\
			Do mặt phẳng $(Q)$ vuông góc với cả $(\alpha)$ và $(\beta)$ nên $\overrightarrow{n}=\left[\overrightarrow{n}_1,\overrightarrow{n}_2\right]=(2;1;-2)$ là một véc-tơ pháp tuyến của mặt phẳng $(Q)$.\\
			Mặt phẳng $(Q)$ đi qua $O(0;0;0)$ và có véc-tơ pháp tuyến $\overrightarrow{n}=(2;1;-2)$ có phương trình là $2x+y-2z=0$.
			\itemch \textbf{Đúng}.\\
			Mặt phẳng $(\gamma)$ song song với mặt phẳng $(\alpha)\colon 3x-2y+2z+7=0$ nên phương trình của mặt phẳng $(\gamma)$ có dạng $3x-2y+2z+d=0$, với $d\ne7$.\\
			Vì $I(1;0;-1)\in(\gamma)$ nên $3\cdot 1-2\cdot 0+2\cdot(-1)+d=0$ $\Leftrightarrow$ $d=-1$ (thỏa mãn điều kiện $d\ne7$).\\
			Vậy $(\gamma)\colon 3x-2y+2z-1=0$.
		\end{itemchoice}
	}
\end{ex}
	\Closesolutionfile{ans}

\TNSA
	\setcounter{ex}{0}
	\Opensolutionfile{ans}[ans/B1-De2-3]
\begin{ex}%[2H5V1-7]
	Từ mặt nước trong một bể nước, tại ba vị trí đôi một cách nhau $2$ m, người ta lần lượt thả dây dọi để quả dọi chạm đáy bể. Phần dây dọi (thẳng) nằm trong nước tại ba vị trí đó lần lượt có độ dài $4$ m; $4{,}4$ m; $4{,}8$ m. Biết đáy bể là phẳng. Hỏi đáy bể nghiêng so với mặt phẳng nằm ngang một góc bao nhiêu độ (làm tròn đến hàng phần chục)?\\
	\shortans[oly]{$21{,}8$}
	\loigiai{
		Gọi ba vị trí trên mặt nước là $A$, $B$, $C$ thì tam giác $ABC$ là tam giác đều cạnh bằng $2$ m. Gọi dây dọi lần lượt là $AA'$, $BB'$, $CC'$ có độ dài lần lượt là $4$ m; $4{,}4$ m; $4{,}8$ m.\\
		Chọn hệ trục toạ độ $Oxyz$ sao cho $O$ là trung điểm của $BC$, tia $Ox$ chứa điểm $A$, tia $Oy$ chứa điểm $B$, tia $Oz$ đi qua trung điểm của $B'C'$ và đơn vị trên các trục là mét.\\
		Ta có $OB=OC=1$, $OA=\sqrt{3}$ $\Rightarrow$ $A'\left(\sqrt{3};0;4\right)$, $B'(0;1;4{,}4)$, $C'(0;-1;4{,}8)$.\\
		Khi đó, $\overrightarrow{A'B'}=\left(-\sqrt{3};1;0{,}4\right)$, $\overrightarrow{A'C'}=\left(-\sqrt{3};-1;0{,}8\right)$.\\
		Mặt phẳng $(A'B'C')$ có một véc-tơ pháp tuyến là $\overrightarrow{n}=\left[\overrightarrow{A'B'},\overrightarrow{A'C'}\right]=0{,}4\sqrt{3}\left(\sqrt{3};1;5\right)$.\\
		Mặt phẳng $(ABC)$ có một véc-tơ pháp tuyến là $\overrightarrow{k}=(0;0;1)$.\\
		Do đó, $\cos\big((ABC),(A'B'C')\big)=\left|\cos\left(\overrightarrow{n},\overrightarrow{k}\right)\right|=\dfrac{5}{\sqrt{29}}$. Góc cần tìm gần bằng $21{,}8^\circ$.
	}
\end{ex}

\begin{ex}%[2H5V1-3]
	Trong không gian với hệ tọa độ $Oxyz$, cho mặt cầu $(S)\colon (x-1)^2+(y+1)^2+z^2=11$ và hai véc-tơ $\overrightarrow{u}_1=(1;1;2)$, $\overrightarrow{u}_2=(1;2;1)$. Gọi $(P)$ là mặt phẳng tiếp xúc với mặt cầu $(S)$ đồng thời song song với giá của hai véc-tơ $\overrightarrow{u}_1$, $\overrightarrow{u}_2$. Phương trình mặt phẳng $(P)$ có dạng $3x+by+cz+d=0$, với $b,\,c,\,d\in\mathbb{Z}$ và $d\ne -15$. Khi đó, $b+c+d$ bằng bao nhiêu?\\
	\shortans[oly]{$5$}
	\loigiai{
		Mặt cầu $(S)$ có tâm $I(1;-1;0)$, bán kính $R=\sqrt{11}$.\\
		Mặt phẳng $(P)$ song song với của hai véc-tơ $\overrightarrow{u}_1$, $\overrightarrow{u}_2$ nên $(P)$ có véc-tơ pháp tuyến là $\overrightarrow{n}=\left[\overrightarrow{u}_1,\overrightarrow{u}_2\right]=(-3;1;1)$.\\
		Phương trình mặt phẳng $(P)$ có dạng $-3x+y+z+d=0$$\Leftrightarrow$ $3x-y-z-d=0$, $d\ne 15$.\\
		Mặt khác, mặt phẳng $(P)$ tiếp xúc với mặt cầu $(S)$ nên ta có
		\begin{align*}
			\mathrm{d}\big(I,(P)\big)=R \Leftrightarrow \dfrac{\left|3+1-0-d\right|}{\sqrt{9+1+1}}=\sqrt{11} \Leftrightarrow \left|-d+4\right|=11 \Leftrightarrow \hoac{&d=15& (\text{loại})\\ &d=-7.&}
		\end{align*}
		Với $d=-7$, ta có phương trình mặt phẳng $(P)$ là $-3x+y+z-7=0$ $\Leftrightarrow$ $3x-y-z+7=0$.\\
		Vậy $b+c+d=-1-1+7=5$.
	}
\end{ex}

\begin{ex}%[2H5C1-3]
	Trong không gian với hệ tọa độ $Oxyz$, có hai mặt phẳng $(P)$ và $(Q)$ cùng thỏa mãn các điều kiện sau: đi qua hai điểm $A(1;1;1)$ và $B(0;-2;2)$, đồng thời cắt các trục tọa độ $Ox$, $Oy$ tại hai điểm cách đều $O$. Giả sử $(P)$ có phương trình $x+b_1y+c_1z+d_1=0$ và $(Q)$ có phương trình $x+b_2y+c_2z+d_2=0$. Tính giá trị biểu thức $b_1b_2+c_1c_2$.\\
	\shortans[oly]{$-9$}
	\loigiai{
		Xét mặt phẳng $(\alpha)$ có phương trình $x+by+cz+d=0$ thỏa mãn các điều kiện: đi qua hai điểm $A(1;1;1)$ và $B(0;-2;2)$, đồng thời cắt các trục tọa độ $Ox,Oy$ tại hai điểm cách đều $O$.\\
		Vì $(\alpha)$ đi qua $A(1;1;1)$ và $B(0;-2;2)$ nên ta có hệ phương trình
		\begin{align*}
			\heva{& 1+b+c+d=0\\ & -2b+2c+d=0.} \tag{*}
		\end{align*}
		Mặt phẳng $(\alpha)$ cắt các trục tọa độ $Ox$, $Oy$ lần lượt tại $M(-d;0;0)$, $N\left(0;\dfrac{-d}{b};0\right)$.\\
		Vì $M$, $N$ cách đều $O$ nên $OM=ON$. Suy ra $\left|d\right|=\left|\dfrac{d}{b}\right|$.\\
		Nếu $d=0$ thì chỉ tồn tại duy nhất một mặt phẳng thỏa mãn yêu cầu bài toán (mặt phẳng này sẽ đi qua điểm $O$).\\
		Do đó, để tồn tại hai mặt phẳng thỏa mãn yêu cầu bài toán thì $\left|d\right|=\left|\dfrac{d}{b}\right|$ $\Leftrightarrow$ $b=\pm 1$.
		\begin{itemize}
			\item Với $b=1$, $(*)$ $\Leftrightarrow$ $\heva{
				& c+d=-2\\
				& 2c+d=2
			}$ $\Leftrightarrow$ $\heva{
				& c=4\\
				& d=-6
			}$. Ta được $(P)\colon x+y+4z-6=0$.
			\item Với $b=-1$, $(*)$ $\Leftrightarrow$ $\heva{
				& c+d=0\\
				& 2c+d=-2
			}$ $\Leftrightarrow$ $\heva{
				& c=-2\\
				& d=2
			}$. Ta được $(Q)\colon x-y-2z+2=0$.
		\end{itemize}
		Vậy $b_1 b_2+ c_1 c_2=1\cdot (-1)+4\cdot (-2)=-9$.
	}
\end{ex}

\begin{ex}%[2H5V1-3]
	Trong không gian với hệ tọa độ $Oxyz$, cho mặt cầu $(S)\colon (x+1)^2+(y-2)^2+(z-3)^2=8$ và điểm $A(1;3;2)$.
	Mặt phẳng $(P)$ đi qua $A$ và cắt $(S)$ theo giao tuyến là đường tròn có bán kính nhỏ nhất. Biết phương trình của $(P)$ có dạng $ax+by+cz+6=0$. Tính $a+b+c$.\\
	\shortans[oly]{$-4$}
	\loigiai{
		Mặt cầu $(S)$ có tâm $I(-1;2;3)$, bán kính $R=2\sqrt{2}$.\\
		Ta có $\overrightarrow{IA}=(2;1;-1)$; $AI=\sqrt{6}<R$, suy ra điểm $A$ nằm trong mặt cầu $(S)$.\\
		Gọi $H$ là hình chiếu vuông góc của $I$ trên mặt phẳng $(P)$. Khi đó mặt phẳng $(P)$ đi qua $A$ và cắt $(S)$ theo giao tuyến là đường tròn có bán kính $r=\sqrt{R^2-IH^2}$. Do đó, $r$ nhỏ nhất khi và chỉ khi $IH$ lớn nhất.\\
		Mặt khác, ta luôn có $IH\le IA$, dấu bằng xảy ra khi và chỉ khi $H$ trùng với $A$, hay $(P)\perp IA$.\\
		Mặt phẳng $(P)$ có véc-tơ pháp tuyến $\overrightarrow{IA}=(2;1;-1)$ và qua $A(1;3;2)$ có phương trình $2(x-1)+(y-3)-1(z-2)=0$ $\Leftrightarrow$ $2x+y-z-3=0$ $\Leftrightarrow$ $-4x-2y+2z+6=0$.\\
		Vậy $a+b+c=-4$.
	}
\end{ex}

\begin{ex}%[2H5V1-3]
	Trong không gian với hệ tọa độ $Oxyz$, cho hai điểm $A(2;-3;1)$, $B(-1;1;0)$ và mặt phẳng $(P)\colon x-y+z-2=0$. Một mặt phẳng $(Q)$ đi qua hai điểm $A$, $B$ và vuông góc với $(P)$ có dạng là $ax+by+cz+2=0$. Tính $a^2+ b^2+ c^2$.\\
	\shortans[oly]{$56$}
	\loigiai{
		$\overrightarrow{AB}=(-3;4;-1)$, $(P)$ có một véc-tơ pháp tuyến là $\overrightarrow{n}_{(P)}=(1;-1;1)$.\\
		$\left[\overrightarrow{AB},\overrightarrow{n}_{(P)}\right]=(3;2;-1)$.\\
		$(Q)$ đi qua $B(-1;1;0)$ và có một véc-tơ pháp tuyến $\overrightarrow{n}_{(Q)}=(3;2;-1)$ nên có phương trình
		\begin{align*}
			3(x+1)+2(y-1)-z=0 \Leftrightarrow 3x+2y-z+1=0 \Leftrightarrow 6x+4y-2z+2=0.
		\end{align*}
		Suy ra $a=6$, $b=4$, $c=-2$ hay $a^2+b^2+c^2=56$.
	}
\end{ex}

\begin{ex}%[2H5H1-3]
	Trong không gian với hệ trục $Oxyz$, cho ba điểm $A(1;2;1)$, $B(2;-1;0)$, $C(1;1;3)$. Phương trình mặt phẳng đi qua ba điểm $A$, $B$, $C$ có dạng $ax+by+cz-12=0$. Khi đó, $a-b-2c$ bằng\\
	\shortans[oly]{$3$}
	\loigiai{
		Ta có $\overrightarrow{AB}=(1;-3;-1)$, $\overrightarrow{AC}=(0;-1;2)$ suy ra $\left[\overrightarrow{AB},\overrightarrow{AC}\right]=(-7;-2;-1)=-1(7;2;1)$.\\
		Mặt phẳng $(ABC)$ đi qua điểm $A(1;2;1)$ có véc-tơ pháp tuyến $\overrightarrow{n}=(7;2;1)$ có phương trình là $7x+2y+z-12=0$. Khi đó, $a-b-2c=3$.
	}
\end{ex}

\centerline{---HẾT---}
\Closesolutionfile{ans}
%\newpage
%%=====================
%\begin{center}
%\textbf{\large BẢNG ĐÁP ÁN}
%\end{center}
%\noindent\textbf{ĐÁP ÁN PHẦN I}
%\inputansbox{10}{ans/B1-De2-1}
	
%\noindent\textbf{ĐÁP ÁN PHẦN II}
%\inputansbox[2]{2}{ans/B1-De2-2}
	
%\noindent\textbf{ĐÁP ÁN PHẦN III}
%\inputansbox[3]{6}{ans/B1-De2-3}



%%Bài 2.
\setcounter{dang}{0}
\newpage
\section{PHƯƠNG TRÌNH ĐƯỜNG THẲNG}
\subsection{LÝ THUYẾT CẦN NHỚ}
\subsubsection{Vectơ chỉ phương của đường thẳng}
\begin{itemize}
	\immini{\item [\iconMT] \indam{Định nghĩa:} Vectơ chỉ phương  $\vec{u}$ của đường thẳng $d$ là những vectơ khác $\vec{0}$ và có giá song song hoặc trùng với $d$. 
		\item [\iconMT] \indam{Chú ý:} 
		\begin{boxdn}
			\begin{itemize}
				\item [$\bullet$] $\vec{u} \ne \vec{0}$ và có giá song song hoặc trùng với $d$. 
				\item [$\bullet$] Nếu $\vec{u}$ và $\vec{u'}$ cùng là vectơ chỉ phương của $d$ thì $\vec{u'} = k \cdot \vec{u}$ (\textit{tọa độ tỉ lệ nhau}).
			\end{itemize}
		\end{boxdn}
	}{
		\begin{tikzpicture}[scale=0.8, line join=round, line cap=round,>=stealth]
			\draw[thick] (0,0)--(4,0)node[below right]{$d$};
			\draw[->,blue] (1,1)--(3,1)node[below]{\scriptsize$\vec{u}$};
			\draw[->,violet] (2.3,1.5)--(3.7,1.5)node[above]{\scriptsize$\vec{u'}$};
	\end{tikzpicture}}
\end{itemize}
\subsubsection{Phương trình tham số của đường thẳng}
\begin{itemize}
	\item [\iconMT] \indam{Công thức:} Đường thẳng $d$ đi qua điểm $M(x_0;y_0;z_0)$ và nhận $\vec{u}=(u_1;u_2;u_3)$ làm vectơ chỉ phương có phương trình là 
	\boxmini{$\heva{&x=x_0+u_1t\\&y=y_0+u_2t\\&z=z_0+u_3t} \quad \left( t \in \mathbb{R}\right) \quad (1) $}
	\item [\iconMT] \indam{Chú ý:}
	\begin{boxdn}
		\begin{itemize}
			\item [\ding{172}] Phương trình các trục tọa độ: 
			\begin{listEX}[3]
				\item [$\bullet$] $Ox \colon \heva{&x=t\\&y=0\\&z=0}$ .
				\item [$\bullet$] $Oy \colon \heva{&x=0\\&y=t\\&z=0}$ .
				\item [$\bullet$] $Oz \colon \heva{&x=0\\&y=0\\&z=t}$ .
			\end{listEX}
			\item [\ding{173}] Nếu $u_1$, $u_2$ và $u_3$ đều khác $0$ thì $(1)$ có thể được viết dưới dạng
			\boxmini{$\dfrac{x-x_0}{u_1}=\dfrac{y-y_0}{u_2}=\dfrac{z-z_0}{u_3} \quad (2)$}
			$(2)$ được gọi là phương trình chính tắc của đường thẳng $d$.
		\end{itemize}
	\end{boxdn}
\end{itemize}
\subsubsection{Vị trị tương đối giữa hai đường thẳng}
Cho hai đường thẳng 
\begin{itemize}
	\item [$\bullet$] $\Delta_1$ qua điểm $M(x_0;y_0;z_0)$, vectơ chỉ phương $\vec{u}=(u_1;u_2;u_3)$;
	\item [$\bullet$] $\Delta_2$ qua điểm $N(x_0';y_0';z_0')$, vectơ chỉ phương $\vec{v}=(v_1;v_2;v_3)$.
\end{itemize}
	\begin{listEX}[1]
		\item [] \indamm{Trường hợp 1:} Nếu $\bigg[\vec{u},\vec{v}\bigg] = \vec{0}$ và 
		\begin{itemize}
			\item [$\bullet$] $\bigg[\vec{u},\vec{MN}\bigg]\ne \vec{0}$  thì $\Delta_1$ song song $\Delta_2$; 
			\item [$\bullet$] $\bigg[\vec{u},\vec{MN}\bigg]  =\vec{0}$  thì $\Delta_1$ trùng $\Delta_2$.
		\end{itemize}
		\item [] \indamm{Trường hợp 2:} Nếu $\bigg[\vec{u},\vec{v}\bigg] \ne \vec{0}$ và 
		\begin{itemize}
			\item [$\bullet$] $\bigg[\vec{u},\vec{v}\bigg] \cdot \vec{MN} \ne 0$  thì $\Delta_1$ chéo $\Delta_2$; 
			\item [$\bullet$] $\bigg[\vec{u},\vec{v}\bigg] \cdot \vec{MN} =0$  thì $\Delta_1$ cắt $\Delta_2$.
		\end{itemize}
	\end{listEX}
\subsection{PHÂN LOẠI, PHƯƠNG PHÁP GIẢI TOÁN}
\begin{dang}{Xác định điểm thuộc và vectơ chỉ phương của đường thẳng}
	Cho đường thẳng $d$.
	\begin{listEX}[1]
		\item [\ding{172}] Nếu $\vec{u} \ne \vec{0}$ và có giá song song hoặc trùng với $d$ thì $\vec{u}$ là vectơ chỉ phương của $d$.
		\item [\ding{173}] Nếu $d$ qua hai điểm $AB$ thì $d$ có một vectơ chỉ phương là $\vec{AB}=\left(x_B-x_A; y_B-y_A;z_B-z_A \right)$. 
		\item [\ding{174}] Nếu $d$ vuông góc với giá của hai vectơ $\vec{a}$, $\vec{b}$ không cùng phương thì $d$ có một vectơ chỉ phương là $\vec{u}=[\vec{a},\vec{b}]$.
		\item [\ding{175}] Cho đường thẳng  $d \colon \heva{&x=x_0+u_1t\\&y=y_0+u_2t\\&z=z_0+u_3t} \quad \left( t \in \mathbb{R}\right)$ thì
		\begin{itemize}
			\item [$\bullet$] Một vectơ chỉ phương của $d$ là $\vec{u}=(u_1;u_2;u_3)$ (hệ số của $t$).
			\item [$\bullet$] Muốn xác định tọa độ một điểm thuộc $d$, ta chỉ cần cho trước giá trị cụ thể của tham số $t$, thay vào hệ phương trình tính $x$, $y$ và $z$.
		\end{itemize}
	\end{listEX}
\end{dang}
\boxmini{BÀI TẬP TỰ LUẬN}
\setcounter{vd}{0}

\begin{vd}
	Cho đường thẳng $d:\heva{x=1-t\\y=2+3t\\z=2+t}\quad (t\in\mathbb{R})$. Tìm một vectơ chỉ phương và hai điểm thuộc đường thẳng $d$.
	\loigiai{
	}
\end{vd}
\dongcham{2}

\begin{vd}
	Trong không gian $Oxyz$, cho hình chóp $O.ABC$ có $A\left( 2;0;0 \right),B\left( 0;4;0 \right)$ và $C\left( 0;0;7 \right)$.
	\begin{enumerate}
		\item Tìm tọa độ một vectơ chỉ phương của đường thẳng $AB$, $AC$.
		\item Vectơ $\overrightarrow{v}=\left( -1;2;0\right)$ có là vectơ chỉ phương của đường thẳng $AB$ không?
	\end{enumerate}
	\loigiai{
		\immini{\begin{enumerate}
				\item Ta có $\overrightarrow{AB}=\left( -2;4;0 \right)$ là một vectơ chỉ phương của đường thẳng $AB$; $\overrightarrow{AC}=\left( -2;0;7 \right)$ là một vectơ chỉ phương của đường thẳng $AC$.
				\item Vì $\overrightarrow{v}=\left( -1;2;0 \right)=\dfrac{1}{2}\overrightarrow{AB}$ nên $\overrightarrow{v}$ là một vectơ chỉ phương của đường thẳng $AB$.
		\end{enumerate}}{\begin{tikzpicture}[xscale=0.8,yscale=0.8]
				\path
				(0,0) coordinate (A)
				(2,-1.5) coordinate (B)
				(5,0) coordinate (C)
				(2.5,3) coordinate (O)
				;
				\draw[dashed] (A)--(C)
				;
				\draw (A)--(B) (A)--(O) (C)--(B) (B)--(O) (C)--(O)
				;
				\foreach \x/\g in {O/90, A/180, B/-90, C/0} \path (\x) circle (.05) +(\g:.3) node {$\x$}
				;
				\draw (2.5,-2.5) node{\textit{Hình 2}}
				;
				\foreach \x/\g in {O/90, A/90, B/90, C/90} \draw[fill=black] (\x) circle (.05) +(\g:.3) node {}
				;
		\end{tikzpicture}}
	}
\end{vd}
\dongcham{5}

\begin{vd}%[2H3H3-1][Mức độ 2]
	Trong không gian $Oxyz$, cho hai mặt phẳng $(P)\colon 2x+y-z-1=0$ và $(Q)\colon x-2y+z-5=0$. Gọi $\Delta$ là giao tuyến của $(P)$ và $(Q)$. Tìm một điểm thuộc $\Delta$ và một vectơ chỉ phương của $\Delta$.
	\loigiai{
		Mặt phẳng $(P)$ và $(Q)$ có VTPT lần lượt là $\overrightarrow{n}_P=(2;1;-1)$ và $\overrightarrow{n}_Q=(1;-2;1)$.\\
		Vậy vectơ chỉ phương của đường thẳng $d$ là giao tuyến của $(P)$ và $(Q)$ là
		$\overrightarrow{u}=\left[\overrightarrow{n}_P,\overrightarrow{n}_Q\right]=(1;3;5)$.}
\end{vd}
\dongcham{5}
\boxmini{BÀI TẬP TRẮC NGHIỆM}
\setcounter{ex}{0}

\begin{ex}
	Cho đường thẳng $d\colon \heva{&x = 1 + 2t\\&y = - t\\&z = 4 + 5t}$. Đường thẳng $d$ có một vectơ chỉ phương là
	\choice
	{\True $\overrightarrow{u_2} = \left(2;-1;5\right)$}
	{$\overrightarrow{u_4} = \left(1;-1;4\right)$}
	{$\overrightarrow{u_3} = \left(1;-1;5\right)$}
	{$\overrightarrow{u_1} = \left(1;0;4\right)$}
	\loigiai{
		Đường thẳng $d$ có một vectơ chỉ phương là $\overrightarrow{u} = \left(2;-1;5\right)$.}
\end{ex}

\begin{ex}
	Cho đường thẳng $d\colon \dfrac{x-2}{-1}=\dfrac{y-1}{2}=\dfrac{z}{1}$. Đường thẳng $d$ có một vectơ chỉ phương là
	\choice
	{\True $\vec{u}=(-1;2;1)$}
	{ $\vec{u}=(2;1;0)$}
	{ $\vec{u}=(-1;2;0)$}
	{ $\vec{u}=(2;1;1)$}
	\loigiai{
		\textbf{Cần nhớ}: Đường thẳng $d\colon \dfrac{x-x_0}{a} =\dfrac{y-y_0}{b}= \dfrac{z-z_0}{c}$ có một VTCP là $\vec{u}=(a;b;c)$ và đi qua điểm $M(x_0;y_0;z_0)$.\\
		Đường thẳng $d\colon \dfrac{x-2}{-1}=\dfrac{y-1}{2}=\dfrac{z}{1}$ có một vectơ chỉ phương là $\vec{u}=(-1;2;1)$.
	}
\end{ex}

\begin{ex}
	Cho đường thẳng $d: \dfrac{x-1}{2}=\dfrac{y+1}{3}=\dfrac{z}{2}$. Điểm nào trong các điểm dưới đây nằm trên đường thẳng $d$?
	\choice
	{$P(5;2;5)$}
	{$Q(1;0;0)$}
	{\True $M(3;2;2)$}
	{$N(1;-1;2)$}
	\loigiai{
	}
\end{ex}

\begin{ex}
	Cho đường thẳng $d: \begin{cases}
		x=1+2t \\
		y=2+3t \\
		z=5-t
	\end{cases}(t\in \mathbb{R})$. Đường thẳng $d$ \textbf{không} đi qua điểm nào sau đây?
	\choice{$M(1;2;5)$}
	{\True $N(2;3;-1)$}
	{$P(3;5;4)$}
	{$Q(-1;-1;6)$} 
	\loigiai{
	}
\end{ex}

\begin{ex}%[Phát triển đề minh họa, 2021]%[Đoàn Minh Tân]%[2H3Y3-1]%
	Cho hai điểm $A(2;-1;4)$ và $B(-1;3;2)$. Đường thẳng $AB$ có một vectơ chỉ phương là
	\choice
	{$\overrightarrow{u}_1=(1;2;2)$}
	{$\overrightarrow{u}_3=(1;2;6)$}
	{\True $\overrightarrow{u}_2=(3;-4;2)$}
	{$\overrightarrow{u}_4=(1;-4;2)$}
	\loigiai{
		Đường thẳng $AB$ nhận $\overrightarrow{AB}=(-3;4;-2)$ làm một vectơ chỉ phương.\\
		Do đó $\overrightarrow{u}_2=(3;-4;2)=-\overrightarrow{AB}$ cũng là một vectơ chỉ phương của $AB$.
	}
\end{ex}

\begin{ex}%[Paul Hieu Nguyen]%[2H3B3-1]%Câu 35.10
	Cho tam giác $ABC$ với $A(1;0;-2)$, $B(2;-3;-4)$, $C(3;0;-3)$. Gọi $G$ là trọng tâm tam giác $ABC$. vectơ nào sau đây là một vectơ chỉ phương của đường thẳng $OG$?
	\choice
	{\True $(-2;1;3)$}
	{$(3;-2;1)$}
	{$(2;1;3)$}
	{$(-1;-3;2)$}
	\loigiai{
		$G$ là trọng tâm tam giác $ABC \Rightarrow G(2;-1;-3)\Rightarrow \overrightarrow{OG}=(2;-1;-3)$.\\
		Đường thẳng $OG$ nhận $\vec{u}=-\overrightarrow{OG}=(-2;1;3)$ làm một vectơ chỉ phương.
	}
\end{ex}

\begin{ex}
	Cho đường thẳng $d$ song song với trục $Oy$. Đường thẳng $d$ có một vectơ chỉ phương là
	\choice
	{$\overrightarrow{u}_4=(2019;0;2019)$}
	{$\overrightarrow{u}_1=(2019;0;0)$}
	{\True $\overrightarrow{u}_2=(0;2019;0)$}
	{$\overrightarrow{u}_3=(0;0;2019)$}
	\loigiai{
		Trục $Oy$ có vectơ chỉ phương $\overrightarrow{j}=(0;1;0)$, mà $d\parallel Oy$ nên $d$ có một vectơ chỉ phương là \[\overrightarrow{u}_2=2019\overrightarrow{j}=(0;2019;0)\]
	}
\end{ex}

\begin{ex}%[2-TT-41-Vted-lan6-2019]%[Trần Nhân Kiệt, dự án tex đề W-T-B]%[2H3Y3-1]%
	Cho đường thẳng $\Delta$ vuông góc với mặt phẳng $(\alpha) \colon x+2z+3=0$. Một vectơ chỉ phương của $\Delta$ là
	\choice
	{$\overrightarrow{v}=(1;2;3)$}
	{\True $\overrightarrow{a}=(1;0;2)$}
	{$\overrightarrow{u}=(2;0;-1)$}
	{$\overrightarrow{b}=(2;-1;0)$}
	\loigiai{
		Ta có $\Delta$ vuông góc với $(\alpha) \Rightarrow \overrightarrow{a}=(1;0;2)$ là một vectơ chỉ phương của $\Delta$.}
\end{ex}

\begin{ex}%[2H3V3-1][Mức độ 3]
	vectơ chỉ phương của đường thẳng vuông góc với mặt phẳng đi qua ba điểm $A(1;2;4)$, $B(-2;3;5)$, $C(-9;7;6)$ có toạ độ là
	\choice
	{$(3;4;-5)$}
	{$(3;-4;5)$}
	{$(-3;4;-5)$}
	{\True $(3;4;5)$}
	\loigiai{
		Gọi $(P)$ là mặt phẳng đi qua ba điểm $A$, $B$, $C$. \\
		Gọi $\overrightarrow{a}$ là vectơ chỉ phương của đường thẳng $d$ là đường thẳng vuông góc với $(P)$.\\
		Ta có $\overrightarrow{AB}=(-3;1;1),\overrightarrow{AC}=(-10;5;2)$.\\
		Vì $d$ vuông góc với $(P)$ nên $d$ có vectơ chỉ phương là
		$\overrightarrow{a}=\left[\overrightarrow{AB},\overrightarrow{AC}\right]=(-3;-4;-5)=-1(3;4;5)$.}
\end{ex}

\begin{ex}%[2H3B3-1]%
	Cho hai mặt phẳng $(P): 3x-2y+2z-5=0$, $(Q): 4x+5y-z+1=0$. Các điểm $A, B$ phân biệt thuộc giao tuyến của hai mặt phẳng $(P)$ và $(Q)$. Khi đó $\overrightarrow{AB}$ cùng phương với vectơ nào sau đây?
	\choice
	{\True $\overrightarrow{u}=(8;-11;-23)$}
	{$\overrightarrow{k}=(4;5;-1)$}
	{$\overrightarrow{w}=(3;-2;2)$}
	{$\overrightarrow{v}=(-8;11;-23)$}
	\loigiai{
		Ta có $\overrightarrow{n}=(3;-2;2)$ và $\overrightarrow{n'}=(4;5;-1)$ lần lượt là các vectơ pháp tuyến của các mặt phẳng $(P), (Q)$. Do đó $\left[\overrightarrow{n}, \overrightarrow{n'}\right]=(-8;11;23)$ là một vectơ chỉ phương của giao tuyến của $(P)$ và $(Q)$.\\
		Từ đó suy ra $\overrightarrow{AB}$ cùng phương với vectơ $\overrightarrow{u}=(8;-11;-23)$.
	}
\end{ex}

\begin{dang}{Viết phương trình đường thẳng $d$ khi biết vài yếu tố liên quan}
	\begin{itemize}
		\item [\iconCH] \indamm{Phương pháp chung:} Ta cần xác định vectơ chỉ phương $\vec{u}$  và một điểm $M$ thuộc đường thẳng.
		\item [\iconCH] \indamm{Một số kiểu xác định vectơ $\vec{u}$ thường gặp:} 
		\begin{listEX}[1]
			\item [\ding{172}] $d$ qua hai điểm $A$, $B$ thì $\vec{u}=\vec{AB}=(x_B-x_A;y_B-y_A;z_B-z_A)$.
			\item [\ding{173}] $d$ song song với $\Delta$ thì $\vec{u}=\vec{u_{\Delta}}$.
			\item [\ding{174}] $d$ vuông góc với $(P)$ thì $\vec{u}=\vec{n}_{P}$.
			\item [\ding{175}]  $d$ vuông góc với giá của hai vectơ $\vec{a}$ và $\vec{b}$ (không cùng phương) thì $\vec{u}=[\vec{a},\vec{b}]$.
		\end{listEX}
	\end{itemize}
\end{dang}
\boxmini{BÀI TẬP TỰ LUẬN}
\setcounter{vd}{0}

\begin{vd}%[2H5H2-3]%[Dự án tex hóa sách bài tập Toán 12 CTST]%[Lê Thị Thúy Hằng]
	Lập phương trình chính tắc của đường thẳng $d$ trong mỗi trường hợp sau
	\begin{enumerate}
		\item $d$ đi qua điểm $A(4;-2;5)$ và có vectơ chỉ phương $\overrightarrow{a}=(7;3;-9)$.
		\item $d$ đi qua hai điểm $M(0;0;1)$, $N(3;3;6)$.
		\item $d$ có phương trình tham số là $\heva{&x=8+5t\\&y=7+4t\\&z=11+9t.}$
	\end{enumerate}
	\loigiai{
		\begin{enumerate}
			\item Đường thẳng $d$ đi qua điểm $A(4;-2;5)$ và có vectơ chỉ phương $\overrightarrow{a}=(7;3;-9)$ nên $d$ có phương trình chính tắc là
			$\dfrac{x-4}{7}=\dfrac{y+2}{3}=\dfrac{z-5}{-9}$.
			\item Đường thẳng $d$ đi qua hai điểm $M(0;0;1)$, $N(3;3;6)$ nên $d$ có vectơ chỉ phương là $\overrightarrow{MN}=(3;3;5)$.\\
			Suy ra phương trình chính tắc của đường thẳng $d$ là $\dfrac{x}{3}=\dfrac{y}{3}=\dfrac{z-1}{5}$.
			\item Đường thẳng $d$ có phương trình tham số là $\heva{&x=8+5t\\&y=7+4t\\&z=11+9t}$, suy ra $d$ có phương trình chính tắc là $\dfrac{x-8}{5}=\dfrac{y-7}{4}=\dfrac{z-11}{9}$.
		\end{enumerate}
	}
\end{vd}
\dongcham{8}
\begin{vd}
\immini{Trong một khu du lịch, người ta cho du khách trải nghiệm thiên nhiên bằng cách đu theo đường trượt zipline từ vị trí $A$ cao 15 m của tháp 1 này sang vị trí $B$ cao 10 m của tháp 2 trong khung cảnh tuyệt đẹp xung quanh. Với hệ trucuc toạ độ $O x y z$ cho trưóc (đơn vị: mét), toạ độ của $A$ và $B$ lần lượt là $(3 ; 2,5 ; 15)$ và $(21 ; 27,5 ; 10)$.
\begin{enumEX}[a)]{1}
	\item Viết phương trình đường thẳng chứa đường trượt zipline này.
	\item Xác định toạ độ của du khách khi ở độ cao 12 mét.
\end{enumEX}}{
\includegraphics[scale=0.6]{images/2P5-B2-NguyenHuuTinh-H5-23}
}
\end{vd}
\dongcham{5}
\begin{vd}
	Trong không gian $Oxyz$, Lập phương trình tham số và phương trình chính tắc (nếu có) của đường thẳng $d$ trong các trường hợp sau:
	\begin{enumEX}[a)]{1}
		\item $d$ đi qua điểm $M$ và song song với đường thẳng $\Delta \colon \dfrac{x-1}{2}=\dfrac{y+1}{1}=\dfrac{z}{-1}$
		\item $d$ qua điểm $M(3;2;-1)$ và vuông góc với mặt phẳng $(P)\colon x+z-2=0$.
		\item $d$ đi qua điểm $M(1; 2; 1)$, đồng thời vuông góc với cả hai đường thẳng $\Delta_{1}\colon \dfrac{x-2}{1}=\dfrac{y+1}{-1}=\dfrac{z-1}{1}$ và $\Delta_{2}\colon \dfrac{x+1}{1}=\dfrac{y-3}{2}=\dfrac{z-1}{-1}$.
	\end{enumEX}

	\loigiai
	{
		\begin{enumerate}[a)]
			\item Đường thẳng $\Delta$ có vectơ chỉ phương là $\vec{u}=(2;1;-1)$.\\
			Đường thẳng $d$ qua $M(2;1;0)$ và song song với đường thẳng $\Delta$ cũng nhận $\vec{u}=(2;1;-1)$ làm vectơ chỉ phương của nó.\\
			Vậy phương trình đường thẳng cần tìm là
			\[\dfrac{x-2}{2}=\dfrac{y-1}{1}=\dfrac{z}{-1} \Leftrightarrow\dfrac{x-2}{4}=\dfrac{y-1}{2}=\dfrac{z}{-2}.\]
			\item Mặt phẳng $(P)\colon x+z-2=0$ có vectơ pháp tuyến là $\overrightarrow{n}_{(P)}=(1;0;1)$.\\
			Đường thẳng $\Delta$ đi qua $M$ và vuông góc với $(P)$ nhận $\overrightarrow{n}_{(P)}$ làm vectơ chỉ phương có phương trình là
			$$\heva{& x=3+t \\ & y=2\\& z=-1+t.}$$
			\item Đường thẳng $\Delta_{1}$ và $\Delta_{2}$ có vectơ chỉ phương là $\vec{u}_1 =(1;-1;1)$ và $\vec{u}_2=(1;2;-1)$.\\
			Vì $d$ vuông góc với cả hai đường thẳng $\Delta_{1}$ và $\Delta_{2}$ nên $d$ có vectơ chỉ phương là $\vec{u}=\left[\vec{u}_{1}, \vec{u}_{2}\right]=(-1; 2; 3)$.\\
			Vậy phương trình đường thẳng $d$ là $\heva{&x=1-t \\&y=2+2 t\\&z=1+3 t}.$
		\end{enumerate}
		
	}
\end{vd}
\dongcham{18}
\begin{vd}%[2H3V3-2]
	Trong không gian $Oxyz$, cho điểm $A(1;-2;0)$, mặt phẳng $(P)\colon 2x-3y+z+5=0$ và đường thẳng $d\colon \dfrac{x-1}{2}=\dfrac{y}{-1}=\dfrac{z+1}{1}$. Viết phương trình đường thẳng $\Delta$ đi qua $A$, cắt $d$ và song song với mặt phẳng $(P)$.
	\loigiai
	{
		Mặt phẳng $(P)$ có VTPT $\overrightarrow{n}=(2 ;-3 ; 1)$.\\
		Gọi $M$ là giao điểm của $\Delta$ và $d\colon \heva{&x=1+2t\\&y=-t\\&z=-1+t}$ là $M(1+2t ;-t ;-1+t)$.\\
		Đường thẳng $\Delta$ nhận $\overrightarrow{AM}=(2 t ;-t+2 ; t-1)$ làm VTCP.\\
		Đường thẳng $\Delta$ song song với mặt phẳng $(P)$ nên
		$$\overrightarrow{AM} \cdot \overrightarrow{n}=0 \Leftrightarrow 2t \cdot 2+(-t+2)\cdot(-3)+(t-1) \cdot 1=0 \Leftrightarrow t=\dfrac{7}{8}.$$
		Suy ra $\overrightarrow{AM}=\left(\dfrac{7}{4} ; \dfrac{9}{8} ;-\dfrac{1}{8}\right)=\dfrac{1}{8}(14 ; 9 ;-1)$.\\
		Đường thẳng $\Delta$ qua $A$ và nhận $\overrightarrow{AM}=(14 ; 9 ;-1)$ làm VTCP nên phương trình $\Delta\colon \heva{&x=1+14t\\&y=-2+9t\\&z=-t.}$	
	}
\end{vd}
\dongcham{10}

\begin{vd}%[2H3K3-2]%Câu 27%[Đình Phúc ]%
	Trong Không gian với hệ tọa độ $Oxyz$, cho hai đường thẳng $d_1 \colon \dfrac{x-2}{1}=\dfrac{y-1}{-1}=\dfrac{z-2}{-1}$ và $d_2 \colon \heva{&x=t\\&y=3\\&z=-2+t}$. Viết phương trình đường vuông góc chung của hai đường thẳng $d_1, d_2$.
	\loigiai{
		Gọi $d$ là đường thẳng cần tìm.
		Gọi $A=d \cap d_1,B=d \cap d_2$\\
		$\begin{aligned}
			& A \in d_1 \Rightarrow A(2+a;1-a;2-a)\\
			& B \in d_2 \Rightarrow B(b;3;-2+b)\\
			& \overrightarrow{AB}=(-a+b-2;a+2;a+b-4)
		\end{aligned}$\\
		$d_1$ có vectơ chỉ phương $\overrightarrow{a}_1=(1;-1;-1)$,
		$d_2$ có vectơ chỉ phương $\overrightarrow{a}_2=(1;0;1)$\\
		$\heva{&d \perp d_1\\&d \perp d_2} \Leftrightarrow \heva{&\overrightarrow{AB} \perp \overrightarrow{a}_1\\&\overrightarrow{AB} \perp \overrightarrow{a}_2} \Leftrightarrow \heva{&\overrightarrow{AB} \cdot \overrightarrow{a}_1=0\\&\overrightarrow{AB} \cdot \overrightarrow{a}_2=0} \Leftrightarrow \heva{&a=0\\&b=3} \Rightarrow A(2;1;2);B(3;3;1)$\\
		$d$ đi qua điểm $A(2;1;2)$ và có vectơ chỉ phương $\overrightarrow{a}_d=\overrightarrow{AB}=(1;2;-1)$\\
		Vậy phương trình của $d$ là $\heva{&x=2+t\\&y=1+2t\\&z=2-t}$}
\end{vd}
\dongcham{10}
\boxmini{BÀI TẬP TRẮC NGHIỆM}
\setcounter{ex}{0}

\begin{ex}
	Cho đường thẳng $\Delta$ đi qua điểm $M\left(2;0;-1\right)$ và có vectơ chỉ phương $\overrightarrow{a}=\left(4;-6;2\right)$. Phương trình tham số của đường thẳng $\Delta$ là
	\choice
	{$\heva{&x=-2+2t\\& y=-3t\\& z=1+t}$}
	{$\heva{&x=2+2t\\ &y=-3t\\& z=-1+t}$}
	{$\heva{&x=-2+4t\\& y=-6t\\ &z=1+2t}$}
	{\True $\heva{&x=4+2t\\ &y=-3t\\ &z=2+t}$}
	\loigiai{
	}
\end{ex}
\cham{3}

\begin{ex}
	Cho hai điểm $A(2;-1;3), B(3;2;-1)$. Phương trình nào sau đây là phương trình đường thẳng $AB$?
	\choice
	{$\heva{&x=1+2t\\&y=3-t\\&z=-4+3t}$}
	{\True $\heva{&x=2+t\\&y=-1+3t\\&z=3-4t}$}
	{$\heva{&x=2+t\\&y=-1+t\\&z=3-4t}$}
	{$\heva{&x=1+2t\\&y=1-t\\&z=-4+3t}$}
	\loigiai{
	}
\end{ex}
\cham{3}

\begin{ex}
	Cho đường thẳng $\Delta: \dfrac{2x-1}{2}=\dfrac{y}{1}=\dfrac{z+1}{-1}$, điểm $A(2;-3;4)$. Đường thẳng qua $A$ và song song với $\Delta$ có phương trình là
	\choice
	{ $\heva{&x=2+t  \\&	y=-3+t  \\&	z=4-t}$}
	{\True $\heva{&x=2-2t  \\&y=-3-t  \\&	z=4+t}$}
	{$\heva{&x=2+2t  \\&y=-3+t  \\&	z=4+t}$}
	{ $\heva{&x=2+2t  \\&y=1-3t  \\&z=-1+4t}$}
	\loigiai{
	}
\end{ex}
\cham{3}

\begin{ex}
	Viết phương trình đường thẳng đi qua điểm $N(2;-3;-5)$ và vuông góc với mặt phẳng $(P): 2x-3y-z+2=0$.
	\choice
	{\True $\dfrac{x-2}{2}=\dfrac{y+3}{-3}=\dfrac{z+5}{-1}$}
	{$\dfrac{x+2}{2}=\dfrac{y-3}{-3}=\dfrac{z-5}{-1}$}
	{$\dfrac{x+2}{2}=\dfrac{y-3}{-3}=\dfrac{z-1}{-5}$}
	{$\dfrac{x-2}{2}=\dfrac{y+3}{-3}=\dfrac{z+1}{-5}$}
	\loigiai{
	}
\end{ex}
\cham{3}

\begin{ex}
	Cho tam giác $ABC$ có $A(3;2;-4), B(4;1;1)$ và $C(2;6;-3).$ Viết phương trình đường thẳng $d$ đi qua trọng tâm $G$ của tam giác $ABC$ và vuông góc với mặt phẳng $(ABC)$.
	\choice
	{$d:\dfrac{x-3}{3}=\dfrac{y-3}{2}=\dfrac{z+2}{-1}$}
	{$d:\dfrac{x+12}{3}=\dfrac{y+7}{2}=\dfrac{z-3}{-1}$}
	{\True $d:\dfrac{x-3}{7}=\dfrac{y-3}{2}=\dfrac{z+2}{-1}$}
	{$d:\dfrac{x+7}{3}=\dfrac{y+3}{2}=\dfrac{z-2}{-1}$}
	\loigiai{
	}
\end{ex}
\cham{3}

\begin{ex}%[2H3V3-2]
	Cho hai điểm $A(1;-1;1)$ và $B(-1;2;3)$ và đường thẳng $\Delta\colon\dfrac{x+1}{-2}=\dfrac{y-2}{1}=\dfrac{z-3}{3}$. Phương trình đường thẳng đi qua điểm $A$, đồng thời vuông góc với hai đường thẳng $AB$ và $\Delta$ là
	\choice
	{$\dfrac{x+1}{7}=\dfrac{y-1}{-2}=\dfrac{z+1}{4}$}
	{$\dfrac{x+1}{7}=\dfrac{y-1}{2}=\dfrac{z+1}{4}$}
	{$\dfrac{x-7}{1}=\dfrac{y-2}{-1}=\dfrac{z-4}{1}$}
	{\True $\dfrac{x-1}{7}=\dfrac{y+1}{2}=\dfrac{z-1}{4}$}
	\loigiai{
		Ta có: $\overrightarrow{AB}=(-2;3;2)$.\\
		Đường thẳng $\Delta$ có vectơ chỉ phương là $\overrightarrow{u}_{\Delta}=(-2;1;3)\Rightarrow\left[\overrightarrow{AB},\overrightarrow{u}_{\Delta}\right]=(7;2;4)$.\\
		Đường thẳng đi qua điểm $A$, đồng thời vuông góc với hai đường thẳng $AB$ và $\Delta$ có vectơ chỉ phương $\left[\overrightarrow{AB},\overrightarrow{u}_{\Delta}\right]$ có phương trình là $\dfrac{x-1}{7}=\dfrac{y+1}{2}=\dfrac{z-1}{4}$.}
\end{ex}
\cham{4}

\begin{ex}
	Cho điểm $A(1;2;3)$ và đường thẳng $d: \dfrac{x+1}{2}=\dfrac{y}{1}=\dfrac{z-3}{-2}$. Gọi $\Delta$ là đường thẳng đi qua điểm $A$, vuông góc với đường thẳng $d$ và cắt trục hoành. Tìm một vectơ chỉ phương $\overrightarrow{u}$ của đường thẳng $\Delta$.
	\choice
	{$\overrightarrow{u}=(0;2;1)$}
	{$\overrightarrow{u}=(1;0;1)$}
	{$\overrightarrow{u}=(1;-2;0)$}
	{\True $\overrightarrow{u}=(2;2;3)$}
	\loigiai{
	}
\end{ex}
\cham{5}

\begin{ex}
	Cho hai đường thẳng $d_1: \dfrac{x-1}{1}=\dfrac{y+1}{2}=\dfrac{z}{-1}$ và $d_2: \dfrac{x-2}{1}=\dfrac{y}{2}=\dfrac{z+3}{2}$. Viết phương trình đường thẳng $\Delta$ đi qua điểm $A(1;0;2)$, cắt $d_1$ và vuông góc với $d_2$.
	\choice
	{$\dfrac{x-1}{-2}=\dfrac{y}{3}=\dfrac{z-2}{4}$}
	{\True $\dfrac{x-3}{2}=\dfrac{y-3}{3}=\dfrac{z+2}{-4}$}
	{$\dfrac{x-5}{-2}=\dfrac{y-6}{-3}=\dfrac{z-2}{4}$}
	{$\dfrac{x-1}{-2}=\dfrac{y}{3}=\dfrac{z-2}{-4}$}
	\loigiai{
		Gọi $B(1+t;-1+2t;-t) \in d_1$. Do đó $\overrightarrow{AB}=(t;2t-1;-t-2)$.\\
		Xét $\overrightarrow{AB}.\overrightarrow{u}_{d_2}=0 \Longleftrightarrow 1.t +2(2t-1)+2(-t-2)=0\Longleftrightarrow t=2$.\\
		Ta có $\overrightarrow{AB}=(2;3;-4)$ chính là vec-tơ chỉ phương của đường thẳng $\Delta$.\\
		Vậy phương trình đường thẳng $\Delta$ là $\heva{x=1+2t\\y=3t\\z=2-4t}$.\\
		Với $t=1$ thì đường thẳng $\Delta$ đi qua điểm $C(3;3;-2)$.\\
		Vậy phương trình đường thẳng $\Delta$ là $\dfrac{x-3}{2}=\dfrac{y-3}{3}=\dfrac{z+2}{-4}$.
	}
\end{ex}
\cham{5}
\begin{ex}%[Paul Hieu Nguyen]%[2H3K3-2]%Câu 35.16
	Cho đường thẳng $\Delta$ đi qua $M(1;2;2)$, song song với mặt phẳng
	$(P)\colon x-y+z+3=0$ đồng thời cắt đường thẳng $d\colon \dfrac{x-1}{1}=\dfrac{y-2}{1}=\dfrac{z-3}{1}$ có phương trình là
	\def\dotEX{}
	\choice
	{\True $\heva{&x=1-t\\&y=2-t\\&z=2.}$}
	{$\heva{&x=1\\&y=2-t\\&z=2-t.}$}
	{$\heva{&x=1-t\\&y=2+t\\&z=2.}$}
	{$\heva{&x=-1+t\\&y=-1+2t\\&z=2t.}$}
	\loigiai{
		Đường thẳng $d$ có phương trình tham số là $\heva{&x=1+t\\&y=2+t\\&z=3+t.}$\\
		Mặt phẳng $(P)$ có một VTPT là $\vec{n}_P=(1;-1;1)$.\\
		Giả sử $\Delta$ cắt $d$ tại $A$
		$\Rightarrow A(1+t;2+t;3+t)$ và $\overrightarrow{MA}=(t;t;1+t)$.\\
		Vì $\Delta$ song song $(P)$ nên $\overrightarrow{MA}\cdot\vec{n}=0\Leftrightarrow t=-1\Rightarrow \overrightarrow{MA}=(-1;-1;0)$.\\
		Chọn một VTCP của $\Delta$ là $\vec{u}_{\Delta}=\overrightarrow{MA}=(-1;-1;0)$.\\
		Đường thẳng $\Delta$ có phương trình tham số là $\heva{&x=1-t\\&y=2-t\\&z=2.}$
	}
\end{ex}
\begin{ex}%%[THPT Việt Đức_Hà Nội_HK2] %[2H3K3]
	Cho đường thẳng $d: x=y=z$. Viết phương trình đường thẳng $d'$ là hình chiếu vuông góc của $d$ lên mặt phẳng tọa độ $(Oyz)$. 
	\choice
	{$\heva{&x=0\\&	y=t \\&	z=2t}$}
	{$\heva{&x=t\\&y=t \\&z=2t}$}
	{$\heva{&x=0\\&y=2+t \\&z=1+t}$}
	{\True $\heva{&x=0\\&y=t \\&z=t}$}
	\loigiai{
	}
\end{ex}
\cham{5}

\begin{ex}
	Cho đường thẳng $d:\dfrac{x+1}{2}=\dfrac{y}{1}=\dfrac{z-2}{1},$ mặt phẳng $(P):x+y-2z+5=0$ và điểm $A(1;-1;2).$ Viết phương trình đường thẳng $\Delta$ cắt $d$ và $(P)$ lần lượt tại $M$ và $N$ sao cho $A$ là trung điểm của đoạn thẳng $MN$.
	\choice
	{\True $\Delta:\dfrac{x-3}{2}=\dfrac{y-2}{3}=\dfrac{z-4}{2}$}
	{$\Delta:\dfrac{x-1}{6}=\dfrac{y+1}{1}=\dfrac{z-2}{2}$}
	{$\Delta:\dfrac{x+5}{6}=\dfrac{y+2}{1}=\dfrac{z}{2}$}
	{$\Delta:\dfrac{x+1}{2}=\dfrac{y+4}{3}=\dfrac{z-3}{2}$}
	\loigiai{
		Ta loại ngay được $1$ phương án vì điểm A không thuộc đường thẳng này.\\
		Đường thẳng $d$ có phương trình tham số $d:\left\{\begin{aligned}x&=-1+2t\\y&=t\\z&=2+t\end{aligned}\right.$\\
		$M\in d\Rightarrow M(-1+2t;t;2+t), A$ là trung điểm $MN\Rightarrow N(3-2t;-2-t;2-t)$\\
		$N\in (P)\Rightarrow 3-2t-2-t-2(2-t)+5=0\Leftrightarrow t=2\Rightarrow N(-1;-4;0)\Rightarrow \overrightarrow{NA}=(2;3;2).\Rightarrow $ Loại được hai phương án không thỏa mãn điều kiện này.\\
		Còn duy nhất $1$ phương án cần chọn.
	}
\end{ex}
\cham{5}
\begin{ex}%[Đề thi HK2, môn Toán THPT Yên Lạc, Vĩnh phúc]%[Nguyễn Quang Dũng, dự án 12-EX-6-2020]%[2H3K3-2]%
	Trong không gian $Oxyz$, đường vuông góc chung của hai đường thẳng chéo nhau $d_1\colon \dfrac{x-2}{2}=\dfrac{y-3}{3}=\dfrac{z+4}{-5}$ và $d_2\colon \dfrac{x+1}{3}=\dfrac{y-4}{-2}=\dfrac{z-4}{-1}$ có phương trình là
	\choice
	{$\dfrac{x-2}{2}=\dfrac{y+2}{2}$}
	{$\dfrac{x-2}{2}=\dfrac{y-2}{3}=\dfrac{z-3}{4}$}
	{\True $\dfrac{x}{1}=\dfrac{y}{1}=\dfrac{z-1}{1}$}
	{$\dfrac{x}{2}=\dfrac{y-2}{3}=\dfrac{z-3}{-1}$}
	\loigiai{
		Giả sử $M\in d_1$, $N\in d_2$ và $MN$ là đoạn vuông góc chung.\\
		Ta có $M(2+2t;3+3t;-4-5t)\,\,N(-1+3s;4-2s;4-s)$, $\overrightarrow{MN}=(3s-2t-3;-2s-3t+1;-s+5t+8)$.\\
		vectơ chỉ phương của $d_1$, $d_2$ lần lượt là $\overrightarrow{u}_1=(2;3;-5)$, $\overrightarrow{u}_2=(3;-2;-1)$. \\
		Ta có $MN$ là đoạn vuông góc chung của $d_1$ và $d_2$ khi và chỉ khi
		\[\heva{&\overrightarrow{MN}\cdot\overrightarrow{u}_1=0\\&\overrightarrow{MN}\cdot\overrightarrow{u}_2=0}\Leftrightarrow\heva{&5s-38t-43=0\\&14s-5t-19=0}\Leftrightarrow\heva{&t=-1\\&s=1.}\]
		Suy ra $M(0;0;1)$, $\overrightarrow{MN}=(2;2;2)$. Ta có phương trình của đường vuông góc chung $MN$ là
		\[\dfrac{x}{1}=\dfrac{y}{1}=\dfrac{z-1}{1}.\]
	}
\end{ex}
\begin{dang}{Vị trí tương đối của hai đường thẳng}
	Cho $d$ qua điểm $M$ và có vectơ chỉ phương $\vec{u}$; $d'$ qua điểm $N$ và có vectơ chỉ phương $\vec{v}$.
	\begin{listEX}[1]
		\item [\ding{172}] Nếu $\vec{u}$ cùng phương $\vec{v}$ ($\vec{u}=k\vec{v}$) và $M \notin d'$ thì $d \parallel d'$.
		\item [\ding{173}] Nếu $\vec{u}$ cùng phương $\vec{v}$ ($\vec{u}=k\vec{v}$) và $M \in d'$ thì $d$ trùng với $d'$.
		\item [\ding{174}] Nếu $\left[\vec{u},\vec{v}\right] \cdot \vec{MN} \ne 0$ thì $d$ và $d'$ chéo nhau.
		\item [\ding{175}] Nếu $\left[\vec{u},\vec{v}\right] \cdot \vec{MN} = 0$ thì $d$ và $d'$ cắt nhau.
		\item [\ding{176}] Nếu $\vec{u} \cdot \vec{v} =0$ thì $d$ và $d'$ vuông góc nhau.
	\end{listEX}
\end{dang}
\boxmini{BÀI TẬP TỰ LUẬN}
\begin{vd}%[2H5V2-4]
	\immini{
		Trên phần mềm mô phỏng 3D một máy khoan trong không gian $Oxyz$, cho biết phương trình trục $a$ của mũi khoan và một đường rãnh $b$ trên vật cần khoan (tham khảo hình vẽ bên) lần lượt là $$a\colon\heva{& x=1\\&y=2\\&z=3t }\text{ và }b\colon\heva{& x=1+4t'\\&y=2+2t'\\&z=6. }$$
		\begin{enumerate}
			\item Chứng minh $a$, $b$ vuông góc và cắt nhau.
			\item Tìm giao điểm của $a$ và $b$.
		\end{enumerate}
	}{
		\includegraphics[scale=0.3]{images/12-SGK-CTST-5-2-13}
	}
	\loigiai
	{
		\begin{enumerate}
			\item $a$ có vectơ chỉ phương $\overrightarrow{m}=(0;0;3)$ và $b$ có vectơ chỉ phương $\overrightarrow{n}=(4;2;0)$.\\
			Ta có $\overrightarrow{m}\cdot\overrightarrow{n}=0\cdot 4+0\cdot 2+3\cdot 0=0$.\\
			Do đó $\overrightarrow{m}\perp\overrightarrow{n}$ hay $a\perp b$.
			\item Gọi $M$ là giao điểm của $a$ và $b$, ta có $\heva{& M\in a\\&M\in b }\Rightarrow\heva{& M(1;2;3t)\\&M(1+4t';2+2t';6)}$. Khi đó ta có $$\heva{& 1=1+4t'\\&2=2+2t'\\&3t=6 }\Rightarrow\heva{& t=2\\&t'=0 }\Rightarrow M(1;2;6).$$
			Vậy giao điểm của $a$ và $b$ là $M(1;2;6)$.
		\end{enumerate}
	}
\end{vd}
\dongcham{6}

\setcounter{vd}{0}
\begin{vd}
	Trong khôn gian $Oxyz$, xét vị trí tương đối giữa hai đường thẳng $d$ và $d'$ trong mỗi trường hợp sau. Nếu chúng cắt nhau, hãy xác định tọa độ giao điểm.
	\begin{enumerate}
		\item $d \colon \heva{&x=2+3t\\&y=3+2t\\&z=4+2t}$ và $d' \colon \heva{&x=8+9t'\\&y=7+6t'\\&z=8+6t';}$
		\item $d \colon \dfrac{x}{4}=\dfrac{y-3}{3}=\dfrac{z-1}{2}$ và $d' \colon \dfrac{x-5}{8}=\dfrac{y-5}{6}=\dfrac{z-3}{4}$;
		\item $d \colon \heva{&x=2\\&y=3+2t\\&z=1-t}$ và $d' \colon \dfrac{x-4}{3}=\dfrac{y-1}{4}=\dfrac{z-5}{5}$;
		\item $d \colon \dfrac{x-2}{3}=\dfrac{y-3}{4}=\dfrac{z-2}{3}$ và $d' \colon \heva{&x=5\\&y=7+2t\\&z=5-t.}$
	\end{enumerate}
	\loigiai{
		\begin{enumerate}
			\item Đường thẳng $d$ đi qua điểm $M(2;3;4)$ và nhận $\overrightarrow{a}=(3;2;2)$ làm vectơ chỉ phương.\\
			Đường thẳng $d'$ đi qua điểm $M'(8;7;8)$ và nhận $\overrightarrow{a'}=(9;6;6)$ làm vectơ chỉ phương.\\
			Ta có $\overrightarrow{MM'}=(6;4;4)$, $\overrightarrow{a'} = 3 \overrightarrow{a} = \dfrac{3}{2} \overrightarrow{MM'}$, suy ra ba vectơ $\overrightarrow{a}$, $\overrightarrow{a'}$, $\overrightarrow{MM'}$ cùng phương. Do đó, $d \equiv d'$.
			\item Đường thẳng $d$ đi qua điểm $M(0;3;1)$ và nhận $\overrightarrow{a}=(4;3;2)$ làm vectơ chỉ phương.\\
			Đường thẳng $d'$ đi qua điểm $M'(5;5;3)$ và nhận $\overrightarrow{a'}=(8;6;4)$ làm vectơ chỉ phương.\\
			Ta có $\overrightarrow{MM'}=(5;2;2)$, $\overrightarrow{a'} = 2\overrightarrow{a}$, suy ra $\overrightarrow{a}$, $\overrightarrow{a'}$ cùng phương. Mặt khác, $\dfrac{4}{5} \ne \dfrac{3}{2}$, suy ra $\overrightarrow{a}$, $\overrightarrow{MM'}$ không cùng phương. Do đó $d \parallel d'$.
			\item Đường thẳng $d$ đi qua điểm $M(2;3;1)$ và nhận $\overrightarrow{a}=(0;2;-1)$ làm vectơ chỉ phương.\\
			Đường thẳng $d'$ đi qua điểm $M'(4;1;5)$ và nhận $\overrightarrow{a'}=(3;4;5)$ làm vectơ chỉ phương.\\
			Ta có $\overrightarrow{MM'}=(2;-2;4)$, $\left[ \overrightarrow{a}, \overrightarrow{a'} \right] = (14;-3;-6) \ne \overrightarrow{0}$, $\left[ \overrightarrow{a}, \overrightarrow{a'} \right] \cdot \overrightarrow{MM'} =10 \ne 0$, suy ra $d$ và $d'$ chéo nhau.
			item Đường thẳng $d$ đi qua điểm $M(2;3;2)$ và nhận $\overrightarrow{a}=(3;4;3)$ làm vectơ chỉ phương.\\
			Đường thẳng $d'$ đi qua điểm $M'(5;7;5)$ và nhận $\overrightarrow{a'}=(0;2;-1)$ làm vectơ chỉ phương.\\
			Ta có $\overrightarrow{MM'}=(3;4;3)$, $\left[ \overrightarrow{a}, \overrightarrow{a'} \right] = (-10;3;6) \ne \overrightarrow{0}$, $\left[ \overrightarrow{a}, \overrightarrow{a'} \right] \cdot \overrightarrow{MM'} =0 \ne 0$, suy ra $d$ và $d'$ cắt nhau.
		\end{enumerate}
	}
\end{vd}
\dongcham{29}
\boxmini{BÀI TẬP TRẮC NGHIỆM}
\setcounter{ex}{0}

\begin{ex} %[2H3B3-6]%Câu 9:
	Cho hai đường thẳng $d\colon \heva{&x=1+t \\& y=2 t \\& z=2-t} \text{ và } d'\colon \heva{&x=2+2t' \\& y=3+4t' \\& z=5-2t'.}$ Mệnh đề nào sau đây đúng?
	\choice
	{$d$ và $d'$ chéo nhau}
	{$d$ trùng $d'$}
	{\True $d$ song song $d'$}
	{$d$ cắt $d'$}
	\loigiai{
		$d$ đi qua $A=(1;0;2)$, có vectơ chỉ phương $\vec{a}=(1;2;-1).$\\
		$d'$ đi qua $B=(2;3;5)$, có vectơ chỉ phương $\vec{a'}=(2;4;-2).$\\
		Ta có $\vec{a}$ cùng phương $\vec{a'}$ nên loại B. D.\\
		$A \notin d'$ nên $d$ song song $d'$.
	}
\end{ex}
\cham{5}

\begin{ex} %[2H3B3-6]%Câu 11:
	Cho hai đường thẳng $d_1 \colon \dfrac{x-1}{1}=\dfrac{y+3}{-2}=\dfrac{z+3}{-3}$ và $d_2\colon \heva{&x=3t \\& y=-1+2t \\& z=0}$. Mệnh đề nào đưới đây đúng?
	\choice
	{\True $d_1$ cắt và không vuông góc với $d_2$}
	{$d_1$ cắt và vuông góc với $d_2$}
	{$d_1$ song song $d_2$}
	{$d_1$ chéo $d_2$}
	\loigiai{
		$d_1$ đi qua $A=(1;-3;-3)$, có vectơ chỉ phương $\vec{a}_1=(1;-2;-3).$\\
		$d_2$ đi qua $B=(0;-1;0)$, có vectơ chỉ phương $\vec{a}_2=(3;2;0).$\\
		Ta có $\vec{a}_1$ không cùng phương $\vec{a}_2$.\\
		$\vec{a}_1 \cdot \vec{a}_2 \ne 0 $.\\
		$[\vec{a}_1 , \vec{a}_2]\cdot \vec{AB} = 0$ nên $d_1$ cắt và không vuông góc với $d_2$.
	}
\end{ex}
\cham{5}
\begin{ex}
	Cho hai đường thẳng $d_1\colon\heva{&x=1-2t\\&y=1+t\\&z=1-t}$ và $d_2\colon\dfrac{x+1}{2}=\dfrac{y-2}{-1}=\dfrac{z}{1}$. Chọn khẳng định đúng.
	\choice
	{$d_1\parallel d_2$}
	{\True $d_1\equiv d_2$}
	{$d_1$, $d_2$ chéo nhau}
	{$d_1$, $d_2$ cắt nhau}
	\loigiai{
		Ta có $d_1$, $d_2$ có vectơ chỉ phương lần lượt là $\overrightarrow{u}_1=(-2;1;-1)$, $\overrightarrow{u}_2=(2;-1;1)$. \\
		Và $A(1;1;1)\in d_1$, $B(-1;2;0)\in d_2\Rightarrow \overrightarrow{AB}=(-2;1;-1)$.\\
		Khi đó $\overrightarrow{u}_1$, $\overrightarrow{u}_2$, $\overrightarrow{AB}$ cùng phương nên $d_1\equiv d_2$.
	}
\end{ex}

\begin{ex}
	Vị trí tương đối của hai đường thẳng $\Delta _1 \colon \dfrac{x-1}{3}=y=\dfrac{z+1}{2}$ và $\Delta _2 \colon \dfrac{x}{2}=\dfrac{y-1}{-1}=\dfrac{z}{1}$,
	\choice
	{Trùng nhau }
	{\True Chéo nhau}
	{Song song}
	{Cắt nhau}
	\loigiai{
		Đường thẳng	$\Delta _1$ đi qua điểm $M_1\left(1;0;-1\right)$ và có vectơ chỉ phương $\overrightarrow{u_1}=\left(3;1;2\right)$.\\
		Đường thẳng $\Delta _2$ đi qua điểm $M_2\left(0;1;0\right)$ và có vectơ chỉ phương $\overrightarrow{u_2}=\left(2;-1;1\right)$.\\
		Ta có $\heva{& \overrightarrow{u_1}\wedge \overrightarrow{u_2}=\left(3;1;-5\right)\\& \overrightarrow{M_1M_2}=(-1;1;1)} \Rightarrow \left(\overrightarrow{u_1}\wedge \overrightarrow{u_2}\right)\cdot \overrightarrow{M_1M_2}=-3+1-5=-7\ne 0$.\\
		Do đó $\Delta _1$ và $\Delta _2$ chéo nhau.
	}
\end{ex}
\begin{ex}
	Cho hai đường thẳng $d_1:\dfrac{x+1}{2}=\dfrac{y-1}{-m}=\dfrac{z-2}{-3}$ và $d_2: \dfrac{x-3}{1}=\dfrac{y}{1}=\dfrac{z-1}{1}$. Tìm tất cả các giá trị thực của $m$ để $d_1$ vuông góc $d_2$.
	\choice
	{$m=5$}
	{$m=1$}
	{$m=-5$}
	{\True$m=-1$}
	\loigiai{
	}
\end{ex}
\cham{5}

\begin{ex}%[Kiểm tra, Sở GD và ĐT - Khánh Hòa, 2019]%[Thanh Ta, 12EX6]%[2H3B3-6]%
	Cho hai đường thẳng $\Delta_1\colon \dfrac{x-1}{2}=\dfrac{y-2}{3}=\dfrac{z-3}{4}$ và $\Delta_2\colon \dfrac{x-4}{1}=\dfrac{y-3}{-2}=\dfrac{z-5}{-2}$. Tọa độ giao điểm $M$ của hai đường thẳng đã cho là
	\choice
	{$M(5;1;3)$}
	{$M(0;-1;-1)$}
	{\True $M(3;5;7)$}
	{$M(2;3;7)$}
	\loigiai{
		Phương trình tham số của đường thẳng $\Delta_1$ là $\Delta_1\colon\heva{& x=1+2t \\ & y=2+3t\\ &z=3+4t}$, thay vào phương trình $\Delta_2$ ta được
		$$\dfrac{2t-3}{1}=\dfrac{3t-1}{-2}=\dfrac{4t-2}{-2}\Rightarrow t=1.$$
		Vậy giao điểm của $\Delta_1$ và $\Delta_2$ là $M(3;5;7)$.
	}
\end{ex}
\begin{ex}%[Thi Thử Lần 2, THPT Lương Thế Vinh - Hà Nội, 2019]%[Dương BùiĐức, dự án 12EX6]%[2H3B3-6]%
	Cho hai đường thẳng $d_1\colon\heva{
		&x=1+t\\ &y=2-t\\ &z=3+2t
	}$ và $d_2\colon\dfrac{x-1}{2}=\dfrac{y-m}{1}=\dfrac{z+2}{-1}$ (với $m$ là tham số). Tìm $m$ để hai đường thẳng $d_1,\ d_2$ cắt nhau.
	\choice
	{\True $m=5$}
	{$m=7$}
	{$m=9$}
	{$m=4$}
	\loigiai{
		Ta có $ M_{1}(1;2;3)\in d_{1} $ và vectơ chỉ phương của $ d_{1} $ là $ \overrightarrow{u}_{1}=(1;-1;2) $, $ M_{2}(1;m;-2)\in d_{2} $ và vectơ chỉ phương của $ d_{2} $ là $ \overrightarrow{u}_{2}=(2;1;-2) $. Suy ra $ \overrightarrow{M_{1}M_{2}}=(0;m-2;-5) $ và $ [\overrightarrow{u}_{1},\overrightarrow{u}_{2}]=(0;6;3) $.\\
		Để $ d_{1} $ cắt $ d_{2} $ thì $ [\overrightarrow{u}_{1},\overrightarrow{u}_{2}]\cdot \overrightarrow{M_{1}M_{2}}=0\Leftrightarrow 6(m-2)-15=0\Leftrightarrow m=5 $.
	}
\end{ex}
\cham{6}
\begin{ex}
	Cho hai đường thẳng $d: \left \lbrace \begin{aligned} &x=1+mt\\ &y=t  \\&z=-1+2t   \end{aligned} \right. (t \in \mathbb{R})$ và $d': \left \lbrace \begin{aligned} &x=1-t'\\ &y=2+2t'\\&z=3-t'   \end{aligned} \right. (t' \in \mathbb{R})$. Giá trị của $m$ để hai đường thẳng $d$ và $d'$ cắt nhau là
	\choice
	{\True $m=0$}
	{$m=1$}
	{$m=-1$}
	{$m=2$}
	\loigiai{
		Đường thẳng $d$ đi qua $A(1;0;-1)$, có vectơ chỉ phương $\overrightarrow{u_1}=(m;1;2)$.\\
		Đường thẳng $d'$ đi qua $B(1;2;3)$, có vectơ chỉ phương $\overrightarrow{u_2}=(-1;2;-1)$.\\
		Ta có $[\overrightarrow{u_1}, \overrightarrow{u_2}]=(-5;m-2;2m+1)$ và $\overrightarrow{AB}=(0;2;4)$.\\
		Hai đường thẳng $d$ và $d'$ cắt nhau $\Leftrightarrow [\overrightarrow{u_1}, \overrightarrow{u_2}]\cdot AB=0\Leftrightarrow m=0$.
	}
\end{ex}
\begin{dang}{Vị trí tương đối của đường thẳng và mặt phẳng}
	Xét đường thẳng $d \colon \heva{&x=x_0+u_1t\\&y=y_0+u_2t\\&z=z_0+u_3t}$ và mặt phẳng $(P) \colon Ax + By + Cz + D =0$.
	\begin{itemize}
		\item [\iconMT] \indam{Phương pháp:} Xét $d \cap (P) \Rightarrow A(x_0+u_1t)+B(y_0+u_2t)+C(z_0+u_3t)+D=0 \quad (*)$
		\begin{boxdn}
			\begin{itemize}
				\item [$\bullet$] Nếu (*) có đúng 1 nghiệm $t$ thì $d$ cắt $(P)$;
				\item [$\bullet$] Nếu (*) vô nghiệm  thì $d$ song song $(P)$;
				\item [$\bullet$] Nếu (*) nghiệm đúng với mọi $t$ thì $d$ nằm trong $(P)$.
			\end{itemize}
		\end{boxdn}
		\item [\iconMT] \indam{Đặc biệt:} Với $\vec{u}$ là vectơ chỉ phương của $d$ và $\vec{n}$ là vectơ pháp tuyến của $(P)$ thì
		$$d \perp (P) \Leftrightarrow \vec{u} \text{ cùng phương với } \vec{n} \text{ hay } \vec{u}=k \cdot\vec{n}$$
	\end{itemize}
\end{dang}
\boxmini{BÀI TẬP TỰ LUẬN}
\setcounter{vd}{0}

\begin{vd}
	Xét vị trí tương đối giữa đường thẳng và mặt phẳng được chỉ ra ở các câu sau:
	\begin{enumEX}[a)]{1}
		\item $(\alpha) \colon y+2z=0$ và $d \colon \heva{& x=2-t \\ & y=4+2t \\ &z=1} $.
		\item $(P)\colon3x-3y+2z-5=0$ và $d\colon\heva{&x=-1+2t\\&y=3+4t\\&z=3t}~(t\in\mathbb{R})$.
		\item $(P)\colon 3x-3y+2z+1=0$ và $d\colon \dfrac{x+1}{1}=\dfrac{y}{-1}=\dfrac{z-1}{-3}$.
	\end{enumEX}
	\loigiai{
			\begin{enumEX}[a)]{1}
			\item Gọi $M(2-t;4+2t;1) \in d$, thay tọa độ $M$ vào phương trình của $(\alpha)$ ta được $4+2t+2=0 \Leftrightarrow t=-3$. Từ đó tìm được $M(5;-2;1)$. Suy ra $d$ cắt $(\alpha)$.
			\item Xét phương trình $3(-1+2t)-3(3+4t)+2\cdot3t-5=0\Leftrightarrow 0\cdot t-17=0$ (vô nghiệm).\\
			Vậy $d\parallel (P)$.
			\item Viết lại đường thẳng $d$ ở dạng tham số $\heva{&x=-1+t\\&y=-t\\&z=1-3t}$.\\
			Xét phương trình $3\cdot (-1+t)-3\cdot (-t)+2\cdot (1-3t)+1=0 \Leftrightarrow 0=0$. Kết luận phương trình có vô số nghiệm $\Rightarrow d \subset (P)$.
		\end{enumEX}
	}
\end{vd}
\dongcham{15}

\begin{vd}
	Tìm điều kiện của tham số $m$ để
	\begin{enumEX}[a)]{1}
		\item $ \Delta:\dfrac{x-10}{5}=\dfrac{y-2}{1}=\dfrac{z+2}{1}$ vuông góc với $(P):10x+2y+mz+11=0$.
		\item $d\colon \dfrac{x-1}{2}=\dfrac{y+1}{3}=\dfrac{z-1}{-1}$ song song với $(\alpha)\colon -x+m^2y+mz+1=0$.
	\end{enumEX}
	\loigiai{
	\begin{enumEX}[a)]{1}
		\item Đường thẳng $ \Delta $ có VTCP ${\overrightarrow{u}}_{\Delta}=(5;1;1)$. \\
		Mặt phẳng $(P)$ có VTPT ${\overrightarrow{n}}_P=(10;2;m)$. \\
		Để $ \Delta \perp (P) \Leftrightarrow {\overrightarrow{u}}_{\Delta}\parallel {\overrightarrow{n}}_P \Leftrightarrow \dfrac{10}{5}=\dfrac{2}{1}=\dfrac{m}{1} \Leftrightarrow m=2$.
		\item Đường thẳng $d$ có phương trình tham số $\heva{&x=1+2t\\&y=-1+3t\\&z=1-t.}$\\
		Xét phương trình $-(1+2t)+m^2(-1+3t)+m(1-t)+1=0\Leftrightarrow \left(3m^2-m-2\right)t-m^2+m=0.\quad (1)$\\
		Ta có $d\parallel (\alpha)$ khi và chỉ khi $(1)$ vô nghiệm $\Leftrightarrow \heva{&3m^2-m-2=0\\&-m^2+m\neq 0}\Leftrightarrow m=-\dfrac{2}{3}$.
	\end{enumEX}}
\end{vd}
\dongcham{12}
\boxmini{BÀI TẬP TRẮC NGHIỆM}
\setcounter{ex}{0}
\begin{ex}
	Cho đường thẳng $d:\,\displaystyle\frac{x-1}{2}=\frac{y}{-2}=\dfrac{z-1}{1}$. Tìm tọa độ giao điểm $ M $ của đường thẳng $ d $ với mặt phẳng ($ Oxy $).
	\choice
	{\True$M(-1;2;0) $}
	{$ M(1;0;0)$}
	{$ M(2;-1;0)$}
	{$ M(3;-2;0)$}
	\loigiai{
	}
\end{ex}
\cham{3}

\begin{ex}
	Cho đường thẳng $d: \dfrac{x-1}{-1}=\dfrac{y+3}{2}=\dfrac{z-3}{1}$ và mặt phẳng $(P): 2x+y-2z+9=0$. Tìm toạ độ giao điểm của $d$ và $(P)$.
	\choice
	{$(2; 1; 1)$}
	{\True $(0;-1;4)$}
	{$(1; -3; 3)$}
	{$(2; -5; 1)$} 
	\loigiai{
	}
\end{ex}
\cham{4}
\begin{ex}
	Cho mặt phẳng $(\alpha)\colon x+2y+3z-6=0$ và đường thẳng $\Delta\colon\dfrac{x+1}{-1}=\dfrac{y+1}{-1}=\dfrac{z-3}{1}$. Mệnh đề nào sau đây đúng?
	\choice
	{$\Delta$ cắt và không vuông góc với $(\alpha)$}
	{$\Delta\parallel (\alpha)$}
	{\True $\Delta\subset (\alpha)$}
	{$\Delta \perp (\alpha)$}
	\loigiai{
		Mặt phẳng $(\alpha)$ có vectơ pháp tuyến $\overrightarrow{n}=(1;2;3)$.\\
		Đường thẳng $\Delta$ đi qua $M(-1;-1;3)$ và có vectơ chỉ phương $\overrightarrow{u}=(-1;-1;1)$.\\
		Ta có $\overrightarrow{n}\cdot\overrightarrow{u}=1\cdot (-1)+2\cdot (-1)+3\cdot 1=0$ và $M\in (\alpha)$.\\
		Vậy $\Delta\subset (\alpha)$.
	}
\end{ex}

\begin{ex}
	Cho đường thẳng $d:\dfrac{x-1}{1}=\dfrac{y-1}{4}=\dfrac{z-m}{-1}$ và mặt phẳng $(P): 2x+my-(m^2+1)z+m-2m^2=0$. Có bao nhiêu giá trị của $m$ để đường thẳng $d$ nằm trên $(P)$?
	\choice{$0$}{\True $1$}{$2$}{Vô số}
	\loigiai{
	}
\end{ex}
\cham{6}

\begin{ex}
	Cho mặt phẳng $(\alpha): x+y+z-6=0$ và đường thẳng $\Delta:\heva{&x=m+t\\&y=-1+nt\\&z=4+2t}$. Tìm điều kiện của $m$ và $n$ để đường thẳng $\Delta$ song song với mặt phẳng $(\alpha)$.
	\choice{\True $\heva{&m\neq 3\\&n=-3}$}
	{$\heva{&m=3\\&n\neq -3}$}
	{$\heva{&m=3\\&n=-3}$}
	{$\heva{&m\neq 3\\&n\neq -3}$}
	\loigiai{
	}
\end{ex}
\cham{6}
\begin{ex}
	Cho đường thẳng $d\colon \dfrac{x-1}{2}=\dfrac{y+2}{-1}=\dfrac{z+1}{1}.$ Trong các mặt phẳng dưới đây mặt phẳng nào vuông góc với đường thẳng $d$?
	\choice
	{$2x-2y+2z+4=0$}
	{$4x-2y-2z-4=0$}
	{$4x+2y+2z+4=0$}
	{\True $4x-2y+2z+4=0$}
	\loigiai{
		Đường thẳng $d$ có vec-tơ chỉ phương $\overrightarrow{u}=(2;-1;1)$.\\
		Xét mặt phẳng $4x-2y+2z+4=0$ có vec-tơ pháp tuyến $\overrightarrow{n}=(4;-2;2)\Rightarrow\overrightarrow{n}=2\overrightarrow{v}.$\\
		$\Rightarrow d$ vuông góc với mặt phẳng có phương trình $4x-2y+2z+4=0$.
	}
\end{ex}

\begin{ex}
	Cho đường thẳng $d\colon\heva{&x=3+2t\\&y=5-3mt\\&z=-1+t.}$ và mặt phẳng $(P)\colon4x-4y+2z-5=0$. Giá trị nào của $m$ để đường thẳng $d$ vuông góc với mặt phẳng $(P)$.
	\choice
	{$m=-\dfrac{5}{6}$}
	{\True $m=\dfrac{2}{3}$}
	{$m=\dfrac{3}{2}$}
	{$m=\dfrac{5}{6}$}
	\loigiai{
		\begin{itemize}
			\item Mặt phẳng $(P)$ có vectơ pháp tuyến là $\overrightarrow{n}=(4;-4;2)$.
			\item Đường thẳng $d$ có vectơ chỉ phương là $\overrightarrow{u}=(2;-3m;1)$.
		\end{itemize}
		Đường thẳng $d$ vuông góc với mặt phẳng $(P)$ khi và chỉ khi $\overrightarrow{n}$ cùng phương với $\overrightarrow{u}$\\
		$\Leftrightarrow\dfrac{2}{4}=\dfrac{-3m}{-4}=\dfrac{1}{2}\Leftrightarrow3m=2\Leftrightarrow m=\dfrac{2}{3}$.}
\end{ex}

\begin{ex}
	Cho điểm $A(1;2;3)$ và đường thẳng $d\colon\dfrac{x-2}{2}=\dfrac{y+2}{-1}=\dfrac{z-3}{1}$. Phương trình mặt phẳng $(P)$ đi qua $A$ và vuông góc với đường thẳng $d$ là
	\choice
	{\True $2x-y+z-3=0$}
	{$x+2y+3z-7=0$}
	{$x+2y+3z-1=0$}
	{$2x-y+z=0$}
	\loigiai{
		Đường thẳng $d$ có một vectơ chỉ phương $\overrightarrow{u}_d=(2;-1;1)$.\\
		Mặt phẳng $(P)$ vuông góc với đường thẳng $d$ nên có một vectơ pháp tuyến $\overrightarrow{n}_P=\overrightarrow{u}_d=(2;-1;1)$.\\
		Mặt phẳng $(P)$ đi qua $A(1;2;3)$ và có một vectơ pháp tuyến $\overrightarrow{n}_P=(2;-1;1)$ có phương trình là
		\[2(x-1)-1(y-2)+1(z-3)=0\Leftrightarrow 2x-y+z-3=0.\]
	}
\end{ex}

\begin{ex}%[2H3B3-6]%
	Cho hai đường thẳng chéo nhau  $d_1:\dfrac{x-2}{2}=\dfrac{y+2}{1}=\dfrac{z-6}{-2}$; $d_2:\dfrac{x-4}{1}=\dfrac{y+2}{-2}=\dfrac{z+1}{3}$. Phương trình mặt phẳng $(P)$ chứa $d_1$ và song song với $d_2$ là
	\choice
	{$(P):x+8y+5z+16=0$}
	{$(P):x+4y+3z-12=0$}
	{$(P):2x+y-6=0$}
	{\True $(P):x+8y+5z-16=0$}
	\loigiai
	{
		$d_1$ đi qua điểm $M(2;-2;6)$ và có vectơ chỉ phương $\vec{u}_1=(2;1;-2).$ \\
		$d_2$ đi qua điểm $N(4;-2;-1)$ và có vectơ chỉ phương $\vec{u}_2=(1;-2;3)$.\\
		Vì mặt phẳng $(P)$ chứa $d_1$ và song song với $d_2$ nên chọn một vectơ pháp tuyến của $(P)$ là $ \vec{n}_{(P)}=\left[\vec{u}_1,\vec{u}_2\right]=(-1;-8;-5)$.\\
		Mặt phẳng $(P)$ đi qua $M(2;-2;6)$ và nhận $\vec{n}_{(P)}=(-1;-8;-5)$ làm vectơ pháp tuyến nên có phương trình $$-1\cdot(x-2)-8\cdot(y+2)-5\cdot(z-6)=0  \Leftrightarrow x+8y+5z-16=0.$$
	}
\end{ex}

\begin{ex}%[HK2 (2017-2018),Sở GD Quảng Trị]%[Phan Minh Tâm ex9]%[2H3B2-3]%
	Cho hai đường thẳng $ d_1 \colon \dfrac{x-1}{2}=\dfrac{y+1}{3}=\dfrac{z-3}{-5} $ và $ d_2\colon \heva{&x=-1+t\\&y=4+3t\\&z=1+t} $. Tìm phương trình mặt phẳng chứa đường thẳng $ d_1 $ và song song với đường thẳng $ d_2. $
	\choice
	{$ 18x-7y+3z+34=0 $}
	{$ 18x+7y+3z-20=0 $}
	{$ 18x+7y+3z+20=0 $}
	{\True $ 18x-7y+3z-34=0 $}
	\loigiai
	{Đường thẳng $ d_1 $ qua $ M(1;-1;3) $ và nhận $ \vec{u_1} =(2;3;-5) $ làm vectơ chỉ phương; $ d_2 $ có vectơ chỉ phương $ \vec{u_2}=(1;3;1) $.\\
		Mặt phẳng $ (P) $ chứa $ d_1 $ và song song $ d_2 $ nên nhận vectơ $ \vec{n} = \left[\vec{u_1},\vec{u_2}\right]=(18;-7;3) $ làm vectơ pháp tuyến.\\
		Vậy phương trình tổng quát của $ (P)$ là \begin{eqnarray*}
			&&18(x-1)-7(y+1)+3(z-3)=0\\ &\Leftrightarrow& 18x-7y+3z-34=0.
		\end{eqnarray*}
	}
\end{ex}
\begin{dang}{Hình chiếu, đối xứng}
\begin{enumerate}[\iconCH]
	\item \indamm{Bài toán 1: Tìm hình chiếu vuông góc của điểm $M$ trên $(P)$:}
	\immini{\begin{itemize}
			\item [$\bullet$] Viết phương trình đường thẳng $MH$ qua $M$ và nhận $\overrightarrow{n_P}$ làm vectơ chỉ phương;
			\item [$\bullet$] Giải hệ giữa đường $MH$ với mặt phẳng $(P)$, tìm $t$. Từ đó, suy ra tọa độ $H$.
		\end{itemize}
		\begin{note}
			Gọi $M'$ đối xứng với $M$ qua mặt phẳng $(P)$ thì
			$$\heva{&x_M'=2x_M-x_H\\&y_M'=2y_M-y_H\\&z_M'=2z_M-z_H}.$$
	\end{note}}{
		\begin{tikzpicture}[scale=1, line join=round, line cap=round]
			\tkzDefPoints{0/0/A,4/0/B,5/1.5/C,2/1/H,2/3/M}
			\coordinate (D) at ($(A)+(C)-(B)$);
			\coordinate (M') at ($(H)+(H)-(M)$);
			\tkzInterLL(M,M')(A,B)\tkzGetPoint{I}
			\tkzDrawPolygon(A,B,C,D)
			\tkzDrawSegments(M,H I,M')
			\tkzDrawSegments[dashed](H,I)
			\draw[fill=black] (2,1) circle (1.5pt) (2,3) circle (1.5pt) (2,-1) circle (1.5pt);
			\draw[->] (3.5,1.1)--(3.5,2.5) node[above right]{$\vec{n_P}$};
			\tkzMarkAngles[size=0.7cm,arc=l](B,A,D)
			\tkzLabelAngles[pos=0.45,rotate=10](B,A,D){$P$}
			\tkzLabelPoints[right](M,M',H)
	\end{tikzpicture}}
	\item \indamm{Bài toán 2: Tìm hình chiếu vuông góc của điểm $M$ trên $d$:}
		\immini{
		\begin{itemize}
			\item [$\bullet$] Tham số điểm $H \in d$ theo ẩn $t$;
			\item [$\bullet$] Giải $\overrightarrow{MH} \cdot \overrightarrow{u_d}=0$, tìm $t$. Từ đó, suy ra tọa độ $H$.
		\end{itemize}
		\begin{note}
			Gọi $M'$ đối xứng với $M$ qua mặt phẳng $d$ thì
			$$\heva{&x_M'=2x_M-x_H\\&y_M'=2y_M-y_H\\&z_M'=2z_M-z_H}.$$
	\end{note}}{
		\begin{tikzpicture}[scale=1, line join=round, line cap=round]
			\tkzDefPoints{0/0/A,5/0/B,2/0/H,2/1.5/M}
			\coordinate (M') at ($(H)+(H)-(M)$);
			\tkzDrawSegments(A,B M,M')
			\draw[fill=black] (2,0) circle (1.5pt) (2,1.5) circle (1.5pt) (2,-1.5) circle (1.5pt);
			\draw[->] (2.5,0.3)--(4,0.3) node[above right]{$\vec{u_d}$};
			\tkzLabelPoints[right](M,M')
			\tkzLabelPoints[below right](H)
			\tkzLabelSegment[pos=0.9,below right](A,B){$d$}
	\end{tikzpicture}}
\end{enumerate}
\end{dang}
\boxmini{BÀI TẬP TỰ LUẬN}
\setcounter{vd}{0}

\begin{vd}
	Trong hệ tọa độ $Oxyz$, cho điểm $M(2;-3;1)$ và đường thẳng $d\colon\dfrac{x+1}{2}=\dfrac{y+2}{-1}=\dfrac{z}{2}$. 
	\begin{enumEX}[a)]{1}
		\item Tìm tọa độ hình chiếu vuông góc của điểm $M$ lên $d$.
		\item Tìm tọa độ điểm $M'$ đối xứng với điểm $M$ qua $d$.
	\end{enumEX}
	\loigiai{
		Gọi $H$ là điểm thuộc đường thẳng $d$, suy ra $H(-1+2t;-2-t;2t)$ với $t\in\mathbb{R}$.\\
		Ta có $\overrightarrow{MH}=(2t-3;1-t;2t-1)$ và một vectơ chỉ phương của đường thẳng là $\overrightarrow{u}=(2;-1;2)$.\\
		Điểm $H$ là hình chiếu của $M$ lên đường thẳng $d$ khi $2(2t-3)-(1-t)+2(2t-1)=0\Leftrightarrow 9t-9=0\Leftrightarrow t=1$.\\
		Suy ra $H(1;-3;2)$, do đó tọa độ điểm $M'$ đối xứng với $M$ qua $d$ là $M'(0;-3;3)$.
	}
\end{vd}
\dongcham{13}
\begin{vd}
	Trong không gian với hệ tọa độ $Oxyz$, cho điểm $M(2;7;-9)$ và mặt phẳng $(P)\colon x+2y-3z-1=0$. 
	\begin{enumEX}[a)]{1}
		\item Tìm tọa độ hình chiếu vuông góc của $M$ trên mặt phẳng $(P)$.
		\item Tìm tọa độ điểm $M'$ đối xứng với điểm $M$ qua $(P)$.
	\end{enumEX}
	\loigiai{
		Đường thẳng $d$ đi qua $M$ vuông góc với $(P)$ có phương trình $\heva{&x=2+t \\&y=7+2t \\&z=-9-3t} (t\in \mathbb{R})$. \\
		Gọi $H$ là hình chiếu vuông góc của điểm $M$ trên $(P)$ thì $H=d\cap(P)$.\\
		Xét phương trình: $2+t+2(7+2t)-3(-9-3t)-1=0 \Leftrightarrow 14t+42=0 \Leftrightarrow t=-3$. \\
		Với $t=-3 \Rightarrow \heva{&x=-1\\&y=1\\&z=0}$. Vậy $H(-1;1;0)$.\\
		Từ đây, suy ra $M'(-4;-5;9)$}
\end{vd}
\dongcham{15}

\boxmini{BÀI TẬP TRẮC NGHIỆM}
\setcounter{ex}{0}

\begin{ex}
	Hình chiếu vuông góc của điểm $A(3;-4;5)$ trên mặt phẳng $(Oxz)$ là điểm
	\choice
	{$M(3;0;0)$}
	{$M(0;-4;5)$}
	{$M(0;0;5)$}
	{\True $M(3;0;5)$}
	\loigiai{
		Hình chiếu vuông góc của điểm $A(3;-4;5)$ trên mặt phẳng $(Oxz)$ là điểm $M(3;0;5)$.}
\end{ex} \cham{3}

\begin{ex}
	Hình chiếu vuông góc của điểm $A(1;2;3)$ trên mặt phẳng $(Oxy)$ là điểm
	\choice
	{$M(0;0;3)$}
	{\True $N(1;2;0)$}
	{$Q(0;2;0)$}
	{$P(1;0;0)$}
	\loigiai
	{
		Hình chiếu vuông góc của điểm $A(1;2;3)$ trên mặt phẳng $(Oxy)$ là điểm $N(1;2;0)$.
	}
\end{ex} \cham{3}

\begin{ex}
	Hình chiếu vuông góc của điểm $M(2;1;-3)$ lên mặt phẳng $(Oyz)$ có tọa độ là
	\choice
	{$(2;0;0)$}
	{$(2;1;0)$}
	{\True $(0;1;-3)$}
	{$(2;0;-3)$}
	\loigiai{
		Điểm thuộc $(Oyz)$ có tọa độ $(0;y;z)$ nên hình chiếu của $M$ lên $(Oyz)$ có tọa độ là $(0;-1;3)$.
	}
\end{ex} \cham{3}

\begin{ex}
	Hình chiếu vuông góc của điểm $A(3;2;1)$ trên trục $Ox$ có tọa độ là
	\choice
	{$(0;2;1)$}
	{$(0;2;0)$}
	{\True $(3;0;0)$}
	{$(0;0;1)$}
	\loigiai{
		Hình chiếu vuông góc của điểm $A(3;2;1)$ lên trục $Ox$ là $A'(3;0;0)$.
	}
\end{ex} \cham{3}

\begin{ex}
	Hình chiếu của điểm $M(2;3;-2)$ trên trục $Oy$ có tọa độ là
	\choice
	{$ (2;0;0) $}
	{\True $ (0;3;0) $}
	{$ (0;0;-2) $}
	{$ (2;0;-2) $}
	\loigiai{
		Hình chiếu của điểm $M(2;3;-2)$ trên trục $Oy$ có tọa độ là $(0;3;0)$.
	}
\end{ex} \cham{3}

\begin{ex}
	Cho điểm $M(3;2;-1)$, điểm $M'(a;b;c)$ đối xứng của M qua trục $Oy$, khi đó $a+b+c$ bằng
	\choice
	{$6$}
	{$2$}
	{$4$}
	{\True $0$}
	\loigiai{
		Với $M(a;b;c)\Rightarrow$ điểm đối xứng của $M$ qua trục $Oy$ là $M'(-a;b;-c)$ \\
		$ \Rightarrow M'(-3;2;1)\Rightarrow a+b+c=0 $.}
\end{ex}

\begin{ex}%[Thi thử L2, Cụm-NBHL,2020]%[Võ Minh Tâm, EX-11-2020]%[2H3B1-1]%
	Điểm đối xứng với điểm $A(-2;7;5)$ qua mặt phẳng $(Oxz)$ là điểm $B$ có tọa độ là
	\choice
	{$B(2;7;-5)$}
	{\True$B(-2;-7;5)$}
	{$B(-2;7;-5)$}
	{$B(2;-7;-5)$}
	\loigiai{
		Hình chiếu vuông góc của điểm $A(-2;7;5)$ trên mặt phẳng $(Oxz)$ là điểm $H(-2;0;5)$.\\
		Điểm $B(-2;-7;5)$ đối xứng với điểm $A$ qua mặt phẳng $(Oxz)$ nên $H$ là trung điểm của $AB$. Vậy điểm đối xứng với điểm $A(-2;7;5)$ qua mặt phẳng $(Oxz)$ là điểm $B(-2;-7;5)$.
	}
\end{ex}

\begin{ex}
	Tọa độ hình chiếu vuông góc của điểm $A(2; -1; 0)$ lên mặt phẳng $(P): 3x - 2y + z + 6 = 0$ là
	\choice
	{$(5; -3; 1)$}
	{\True $(-1; 1; -1)$}
	{$(1; 1; 1)$}
	{$(3; -2; 1)$}
	\loigiai{
		Gọi $H (x; y; -6 -3x + 2y)$ là hình chiếu của $A$ lên mặt phẳng $P$.\\ Ta có $\overrightarrow{AH} = (x - 2; y + 1; -6 - 3x + 2y)$. \\Do $\overrightarrow{AH} \perp (P)$ nên hai vectơ $\overrightarrow{AH}$ và $\overrightarrow{n}_P$ cùng phương.\\ Suy ra ta có hệ phương trình \begin{center}$\dfrac{x - 2}{3} = \dfrac{y + 1}{-2} = \dfrac{-6-3x+2y}{1}.$\end{center}
		Giải hệ ta thu được một nghiệm là $(-1; 1; -1)$.
	}
\end{ex}
\cham{5}
\begin{ex}
	Gọi hình chiếu vuông góc của điểm $A(3; -1; -4)$ lên mặt phẳng $(P): 2x - 2y - z -3 = 0$ là điểm
	$H(a; b; c).$ Khi đó khẳng định nào sau đây đúng?
	\choice
	{\True $a+b+c=-1$}
	{$a+b+c=3$}
	{$a+b+c=5$}
	{$a+b+c=-\dfrac{5}{3}$}
	\loigiai{
		\begin{itemize}
			\item [$\bullet$] Đường thẳng $AH$ qua $A(3;-1;-4)$ và nhận $\vec{n}=(2;-2;-1)$ làm vectơ pháp tuyến nên có phương trình là
			$$\heva{&x=3+2t\\&y=-1-2t\\&z=-4-t}$$
			\item [$\bullet$] $H\left(3+2t;-1-2t;-4-t \right) =AH \cap (P)$. Phương trình để xác định $t$ (\textit{thay vào phương trình của} $(P)$) là
			$$2(3+2t)-2(-1-2t)-(-4-t)-3=0 \Leftrightarrow t=-1.$$
			Với $t=-1$ thì $H(1;1;-3)$. Suy ra $a+b+c=1+1-3=-1.$
		\end{itemize}
	}
\end{ex}
\cham{5}
\begin{ex}
	Cho mặt phẳng $(P): 2x+2y-z+9=0$ và điểm $A(-7; -6; 1)$. Tìm tọa độ điểm $A'$ đối xứng với điểm $A$ qua mặt phẳng $(P)$.
	\choice
	{\True $A'(1; 2; -3)$}
	{$A'(1; 2; 1)$}
	{$A'(5; 4; 9)$}
	{$A'(9; 0; 9)$}
	\loigiai{
		Họi $H$ là hình chiếu vuông góc của điểm $A$ lên $(P)$.
		\begin{itemize}
			\item [$\bullet$] Đường thẳng $AH$ qua $A(-7;-6;1)$ và nhận $\vec{n}=(2;2;-1)$ làm vectơ pháp tuyến nên có phương trình là
			$$\heva{&x=-7+2t\\&y=-6+2t\\&z=1-t}$$
			\item [$\bullet$] $H\left(-7+2t;-6+2t;1-t \right) =AH \cap (P)$. Phương trình để xác định $t$ (\textit{thay vào phương trình của} $(P)$) là
			$$2(-7+2t)+2(-6+2t)-(1-t)+9=0 \Leftrightarrow t=2.$$
			Với $t=2$ thì $H(-3;-2;-1)$ và $H$ là trung điểm của đoạn $AA'$ nên
			$$\heva{&x_{A'}=2x_H-x_A=1\\&y_{A'}=2y_H-y_A=2\\&z_{A'}=2z_H-z_A=-3} \Rightarrow A'(1;2;-3).$$
		\end{itemize}
	}
\end{ex}
\cham{5}


\begin{ex}
	Cho điểm $A\left(4; - 3; 2\right)$ và đường thẳng $d: \dfrac{x + 2}{3}=\dfrac{y + 2}{2}=\dfrac{z}{- 1}$. Gọi điểm $H$ là hình chiếu vuông góc của điểm $A$ lên đường thẳng $d$. Tọa độ điểm $H$ là
	\choice
	{$H\left(5; 4; - 1\right)$}
	{\True $H\left(1; 0; - 1\right)$}
	{$H\left(- 5; - 4; 1\right)$}
	{$H\left(- 2; - 2; 0\right)$}
	\loigiai{
		$d$ có vectơ chỉ phương $\vec{u}=(3;2;-1)$.
		\begin{itemize}
			\item [$\bullet$]  Gọi $H(-2+3t;-2+2t;-t) \in d$, ta có $\vec{AH}=(3t-6;2t+1;-t-2)$.
			\item [$\bullet$] $\vec{AH}$ vuông góc $\vec{u}$, suy ra $\vec{AH} \cdot \vec{u}=0$ hay
			$$ (3t-6) \cdot 3 + (2t+1) \cdot 2 + (-t-2) \cdot (-1)=0 \Leftrightarrow t=1.$$
		\end{itemize}
		Với $t=1$ thì $H(1;0;-1)$.
	}
\end{ex}
\cham{7}
\begin{ex}%[2-TT-ChuyenBacGiang-thang4-2019]%[Duong Xuan Loi, dự án tex đề W-T-B]%[2H3K3-8]%Câu 33
	Cho đường thẳng $d \colon \dfrac{x-1}{2}=\dfrac{y+1}{1}=\dfrac{z}{-1}$, $M(2;1;0)$. Gọi $H(a;b;c)$ là điểm thuộc $d$ sao cho $MH$ có độ dài nhỏ nhất. Tính $T=a^2+b^2+c^2$.
	\choice
	{$T=\sqrt{5}$}
	{$T=12$}
	{$T=21$}
	{\True $T=6$}
	\loigiai{
		Phương trình tham số của đường thẳng $d$ là $\left\{\begin{aligned}
			&x=1+2t\\
			&y=-1+t\\
			&z=-t.\\
		\end{aligned}\right. $\\
		Lấy $H \in d \Rightarrow H(1+2t;-1+t;-t)$ và $\overrightarrow{MH}=(2t-1;t-2;-t)$.\\
		$MH$ nhỏ nhất khi và chỉ khi $H$ là hình chiếu của $M$ xuống $d$, do đó
		$$\overrightarrow{MH} \cdot \overrightarrow{u}_d=0\Leftrightarrow 2(2t-1)+(t-2)+t=0 \Leftrightarrow t=\dfrac{2}{3}.$$
		Vậy $H\left(\dfrac{7}{3};-\dfrac{1}{3};-\dfrac{2}{3}\right) \Rightarrow \left\{\begin{aligned}
			&a=\dfrac{7}{3}\\
			&b=-\dfrac{1}{3}\\
			&c=-\dfrac{2}{3}\\
		\end{aligned}\right. \Rightarrow T=a^2+b^2+c^2=6$.}
\end{ex}
\begin{ex}
	Cho điểm $M\left(1; 2; - 6\right)$ và đường thẳng $d: \heva{ & x=2 + 2t \\ & y=1 - t \\  & z= - 3 + t} \left(t\in \mathbb{R}\right)$. Điểm $N$ là điểm đối xứng của $M$ qua đường thẳng $d$ có tọa độ là
	\choice
	{$N\left(0; 2; - 4\right)$}
	{\True $N\left(- 1; 2; - 2\right)$}
	{$N\left(1; - 2; 2\right)$}
	{$N\left(- 1; 0; 2\right)$}
	\loigiai{
		$d$ có vectơ chỉ phương $\vec{u}=(2;-1;1)$. Gọi $H$ là hình chiếu vuông góc của $m$ trên $d$. 
		\begin{itemize}
			\item [$\bullet$] Ta có $H(2+2t;1-t;-3+t) \in d$ và $\vec{MH}=(2t+1;-t-1;t+3)$.
			\item [$\bullet$] $\vec{MH}$ vuông góc $\vec{u}$, suy ra $\vec{MH} \cdot \vec{u}=0$ hay
			$$ (2t+1) \cdot 2 + (-t-1) \cdot (-1) + (t+3) \cdot (1)=0 \Leftrightarrow t=-1.$$
		\end{itemize}
		Với $t=-1$ thì $H(0;2;-4)$ và $H$ là trung điểm của đoạn $MN$ nên 
		$$\heva{&x_{N}=2x_H-x_M=-1\\&y_{M}=2y_H-y_M=2\\&z_{N}=2z_H-z_M=-2} \Rightarrow N(-1;2;-2).$$
	}
\end{ex}
\cham{7}

\begin{ex}
	Cho đường thẳng $\Delta: \dfrac{x}{2}=\dfrac{y+1}{1}=\dfrac{z-1}{-1}$ và hai điểm $A(1;0;1)$, $B(-1;1;2)$. Biết điểm $M(a;b;c)$ thuộc $\Delta $ sao cho $\left| \overrightarrow{MA}-3\overrightarrow{MB} \right|$ đạt giá trị nhỏ nhất. Khi đó, tổng $a+2b+4c$ bằng bao nhiêu?
	\choice
	{$0$}
	{$-1$}
	{\True $2$}
	{$1$}
	\loigiai{
		\begin{itemize}
			\item [\iconMT] \indam{Cách 1:} Gọi $I$ là điểm thỏa $\overrightarrow{IA}-3\overrightarrow{IB}=\vec{0}$, suy ra $I\left(-2;\dfrac{3}{2};\dfrac{5}{2}\right)$.\\
			Theo kết quả của \indamm{Bài toán 5} thì  $\left| \overrightarrow{MA}-3\overrightarrow{MB} \right|$ nhỏ nhất khi $M$ là hình chiếu vuông góc của điểm $I$ lên $\Delta$.
			\begin{itemize}
				\item Gọi $M(2t;-1+t;1-t) \in \Delta$ là hình chiếu vuông góc của $I$ lên $\Delta$. Ta có
				$$\vec{IM} \cdot \vec{u}_\Delta = 0 \Leftrightarrow t=-\dfrac{1}{2}.$$
				\item Với $t=-\dfrac{1}{2}$ thì $M\left( -1;-\dfrac{3}{2};\dfrac{3}{2}\right)$. Suy ra $a+2b+4c=2$.
			\end{itemize}
			\item [\iconMT] \indam{ Cách 2:} Ta tham số tọa độ điểm $M$, sau đó dùng khảo sát hàm để xứ lý max - min.\\
			Gọi $M(2m;-1+m;1-m)\in\Delta$. Ta có $$\overrightarrow{MA}-3\overrightarrow{MB}=(4+4m;-5+2m;-3-2m)$$
			Suy ra $$\left|\overrightarrow{MA}-3\overrightarrow{MB}\right|=\sqrt{24m^2+24m+50}=\sqrt{24\left(m+\dfrac{1}{2}\right)^2 +44}.$$
			Nhận xét $\left|\overrightarrow{MA}-3\overrightarrow{MB}\right|$ nhỏ nhất khi $m=-\dfrac{1}{2}$. Từ đó suy ra $M\left( -1;-\dfrac{3}{2};\dfrac{3}{2}\right)$.\\
			Vậy $a+2b+4c=2$.
		\end{itemize}
	}
\end{ex}

\begin{ex}
	Cho ba điểm $A(0;-2;-1)$, $B(-2;-4;3)$, $C(1;3;-1)$ và mặt phẳng $(P)\colon x+y-2z-3=0$. Gọi $M(a;b;c)\in (P)$ sao cho $\left|\vec{MA}+\vec{MB}+2\vec{MC}\right|$ đạt giá trị nhỏ nhất. Tính $a-b+2c$.
	\choice
	{$3$}
	{$-1$}
	{$4$}
	{\True $-2$}
	\loigiai{
		Gọi $I$ là điểm sao cho $\vec{IA}+\vec{IB}+2\vec{IC}=\vec{0}\Rightarrow I(0;0;0)$. Từ đó ta có
		\begin{eqnarray*}
			\left|\vec{MA}+\vec{MB}+2\vec{MC}\right|
			&=&\left|\left(\vec{IA}-\vec{IM}\right) +\left(\vec{IB}-\vec{IM}\right)+2\cdot\left(\vec{IC}-\vec{IM}\right)\right|\\
			&=&\left|\left( \vec{IA}+\vec{IB}+2\vec{IC}\right) -4\vec{IM}\right|\\
			&=&\left|\vec{0} -4\vec{IM}\right|=4IM.
		\end{eqnarray*}
		Bởi vậy $\left|\vec{MA}+\vec{MB}+2\vec{MC}\right|$ đạt giá trị nhỏ nhất $\Leftrightarrow IM$ đạt giá trị nhỏ nhất $\Leftrightarrow M$ là hình chiếu vuông góc của $I$ trên mặt phẳng $(P)$ $\Leftrightarrow M$ là giao điểm của đường thẳng $d$ đi qua $I$ và vuông góc với mặt phẳng $(P)$.\\
		Phương trình đường thẳng $d$ là $d\colon \heva{&x=t\\&y=t\\&z=-2t.}$\\
		Tọa độ giao điểm của $d$ và $(P)$ ứng với $t$ là nghiệm phương trình
		$$(t)+(t)-2\cdot(-2t)-3=0 \Leftrightarrow t=\dfrac{1}{2}.$$
		Tọa độ điểm $M$ cần tìm là $M\left(\dfrac{1}{2};\dfrac{1}{2};-1\right)$. Suy ra $a-b+2c=-2$
	}
\end{ex}
\Closesolutionfile{ans}
\subsection{BÀI TẬP TRẮC NGHIỆM TỰ LUYỆN}
\subsection*{\indam{PHẦN I. Câu trắc nghiệm nhiều phương án lựa chọn. Thí sinh trả lời từ câu 1 đến câu 12. Mỗi câu hỏi thí sinh chỉ chọn một phương án.}} 
	\setcounter{ex}{0}
	\Opensolutionfile{ans}[ans/B2-De2-1]
\begin{ex}%[2H5N2-2]
	Trong không gian với hệ trục tọa độ $O x y z$, cho hai điểm $A(0 ;-1 ;-2)$ và $B(2 ; 2 ; 2)$. Véc-tơ $\overrightarrow{a}$ nào dưới đây là một véc-tơ chỉ phương của đường thẳng $A B$?
	\choice
	{$\overrightarrow{a}=(-2 ; 1 ; 0)$}
	{$\overrightarrow{a}=(2 ; 3 ; 0)$}
	{$\overrightarrow{a}=(2 ; 1 ; 0)$}
	{\True $\overrightarrow{a}=(2 ; 3 ; 4)$}
	\loigiai{
		Ta có $\overrightarrow{A B}=(2 ; 3 ; 4)$ nên đường thẳng $A B$ có một véc-tơ chỉ phương là $\overrightarrow{a}=(2 ; 3 ; 4)$.}
\end{ex}

%G:\My Drive\CODE12-2024\DE-ON-THEO BAI\2H5-TACH DE\Bai2-De2.tex
\begin{ex}%[2H5H2-1]
	Đường thẳng $(\Delta)\colon \dfrac{x-1}{2}=\dfrac{y+2}{1}=\dfrac{z}{-1}$ \textbf{không} đi qua điểm nào dưới đây?
	\choice
	{$C(3 ;-1 ;-1)$}
	{$D(1 ;-2 ; 0)$}
	{\True $A(-1 ; 2 ; 0)$}
	{$B(-1 ;-3 ; 1)$}
	\loigiai{
		Ta có $\dfrac{-1-1}{2} \neq \dfrac{2+2}{1} \neq \dfrac{0}{-1}$ nên điểm $A(-1 ; 2 ; 0)$ không thuộc đường thẳng $(\Delta)$.
	}
\end{ex}

%G:\My Drive\CODE12-2024\DE-ON-THEO BAI\2H5-TACH DE\Bai2-De2.tex
\begin{ex}%[2H5H2-3]
	Cho đường thẳng $\Delta$ đi qua điểm $M(2 ; 0 ;-1)$ và có một véc-tơ chỉ phương $\overrightarrow{a}=(4 ;-6 ; 2)$. Phương trình tham số của đường thẳng $\Delta$ là
	\choice
	{\True $\heva{&x=2+2 t \\& y=-3 t \\& z=-1+t}$}
	{$\heva{&x=-2+4 t \\& y=-6 t \\& z=1+2 t}$}
	{$\heva{&x=4+2 t \\& y=-3 t \\& z=2+t}$}
	{$\heva{&x=-2+2 t \\& y=-3 t \\& z=1+t}$}
	\loigiai{
		Véc-tơ chỉ phương $\overrightarrow{a}=(4 ;-6 ; 2)=2(2 ;-3 ; 1)$ nên đường thẳng $\Delta$ có phương trình tham số là $\heva{&x=2+2 t \\& y=-3 t \\& z=-1+t.}$
	}
\end{ex}

%G:\My Drive\CODE12-2024\DE-ON-THEO BAI\2H5-TACH DE\Bai2-De2.tex
\begin{ex}%[2H5N2-2]
	Trong không gian $Oxyz$, cho đường thẳng $d\colon \dfrac{x-1}{2}=\dfrac{y}{-1}=\dfrac{z-1}{-3}$. Một véc-tơ chỉ phương của đường thẳng $d$ là
	\choice
	{\True $\overrightarrow{u}_3=(2 ; -1 ; -3)$}
	{$\overrightarrow{u}_4=(-2 ;-1 ; 3)$}
	{$\overrightarrow{u}_1=(2 ;-1 ; 3)$}
	{ $\overrightarrow{u}_2=(1 ; 0 ; 1)$}
	\loigiai{
		Một véc-tơ chỉ phương của đường thẳng $d$ là $\overrightarrow{u}_3=(2 ;-1 ;-3)$.
	}
\end{ex}

%G:\My Drive\CODE12-2024\DE-ON-THEO BAI\2H5-TACH DE\Bai2-De2.tex
\begin{ex}%[2H5H2-3]
	Trong không gian với hệ trục tọa độ $O x y z$, cho mặt phẳng $(P)$ có phương trình là $2 x+y-5 z+6=0$ . Phương trình đường thẳng $d$ đi qua điểm $M(1 ;-2 ; 7)$ và  vuông góc với $(P)$ là
	\choice
	{$d\colon  \dfrac{x+1}{2}=\dfrac{y-2}{-1}=\dfrac{z+7}{-5}$}
	{$d\colon  \dfrac{x-1}{2}=\dfrac{y-2}{1}=\dfrac{z-7}{-5}$}
	{\True $d\colon  \dfrac{x-1}{2}=\dfrac{y+2}{1}=\dfrac{z-7}{-5}$}
	{$d\colon  \dfrac{x-2}{1}=\dfrac{y-1}{-2}=\dfrac{z+5}{7}$}
	\loigiai{
		Ta có $d$ vuông góc với $(P)$ nên có véc-tơ chỉ phương là $\overrightarrow{u}=(2 ; 1 ;-5)$.\\
		Kết hợp với $d$ đi qua điểm $M(1 ;-2 ; 7)$ nên $d\colon  \dfrac{x-1}{2}=\dfrac{y+2}{1}=\dfrac{z-7}{-5}$.
	}
\end{ex}

%G:\My Drive\CODE12-2024\DE-ON-THEO BAI\2H5-TACH DE\Bai2-De2.tex
\begin{ex}%[2H5H2-4]
	Trong không gian với hệ trục tọa độ $O x y z$, cho hai đường thẳng $d_1\colon \dfrac{x-1}{2}=\dfrac{y-2}{3}=\dfrac{z-3}{4}$ và $d_2\colon\heva{&x=1+t \\& y=2+2 t \\& z=3-2 t}$. Mệnh đề nào sau đây đúng?
	\choice
	{\True $d_1$ và $d_2$  vừa cắt nhau vừa vuông góc}
	{$d_1$ và $d_2$ không vuông góc và không cắt nhau}
	{$d_1$ và $d_2$ cắt nhau nhưng không vuông góc}
	{$d_1$ và $d_2$ vuông góc nhưng không cắt nhau}
	\loigiai{
		Chọn $M(1 ; 2 ; 3), $ $N(0 ; 0 ; 5)$ là hai điểm lần lượt thuộc đường thẳng $d_1$ và $d_2$.\\
		Ta có $\overrightarrow{u}_{d_1}=(2 ; 3 ; 4)$ và $\overrightarrow{u}_{d_2}=(1 ; 2 ;-2)$ nên $\overrightarrow{u}_{d_1} \cdot \overrightarrow{u}_{d_2}=0$ nên $d_1 \perp d_2$.\\
		Mặt khác, ta có $\left[\overrightarrow{u}_{d_1} ; \overrightarrow{u}_{d_2}\right] \overrightarrow{M N}=0$ nên $d_1$ cắt $d_2$.\\
		Vậy hai đường thẳng vừa vuông góc, vừa cắt nhau.
	}
\end{ex}

%G:\My Drive\CODE12-2024\DE-ON-THEO BAI\2H5-TACH DE\Bai2-De2.tex
\begin{ex}%[2H5N2-2]
	Trong không gian với hệ trục tọa độ $Oxyz$, véc-tơ nào dưới đây là véc-tơ chỉ phương của trục $Oz$?
	\choice
	{$\overrightarrow{m}=(1 ; 1 ; 1)$}
	{\True $\overrightarrow{k}=(0 ; 0 ; 1)$}
	{$\overrightarrow{i}=(1 ; 0 ; 0)$}
	{$\overrightarrow{j}=(0 ; 1 ; 0)$}
	\loigiai{
		Trục $O z$ có một vectơ chỉ phương là $\overrightarrow{k}=(0 ; 0 ; 1)$.}
\end{ex}

%G:\My Drive\CODE12-2024\DE-ON-THEO BAI\2H5-TACH DE\Bai2-De2.tex
\begin{ex}%[2H5N2-1]
	Đường thẳng $(\Delta)\colon \dfrac{x-1}{2}=\dfrac{y+2}{1}=\dfrac{z}{-1}$ đi qua điểm nào dưới đây?
	\choice
	{$Q(-1 ;-2 ; 0)$}
	{$N(-1 ; 2 ; 0)$}
	{$P(3 ; 1 ;-1)$}
	{\True $M(1 ;-2 ; 0)$}
	\loigiai{
		Ta có $\dfrac{1-1}{2}=\dfrac{2-2}{1}=\dfrac{0}{-1}$ nên điểm $M(1 ;-2 ; 0)$ thuộc đường thẳng $(\Delta)$.
	}
\end{ex}

%G:\My Drive\CODE12-2024\DE-ON-THEO BAI\2H5-TACH DE\Bai2-De2.tex
\begin{ex}%[2H5H2-7]
	Đường thẳng $d\colon \dfrac{x-1}{2}=\dfrac{y+1}{-1}=\dfrac{z+3}{-1}$ vuông góc với đường thẳng nào dưới đây?
	\choice
	{$d_1\colon \heva{&x=2-3 t \\& y=-2 t \\& z=1+5 t}$}
	{\True $d_4\colon \heva{&x=1-3 t \\& y=2-t \\& z=5-5 t}$}
	{$d_3\colon \heva{&x=2+3 t \\& y=3-t \\& z=5 t}$}
	{$d_2\colon \heva{&x=2 \\& y=3-3 t \\& z=1+t}$}
	\loigiai{
		Đường thẳng $d$ có véc-tơ chỉ phương $\overrightarrow{u}=(2 ;-1 ;-1)$.\\
		Các đường thẳng $d_1,$ $ d_2, $ $d_3,$ $ d_4$ lần lượt có véc-tơ chỉ phương là
		$\overrightarrow{u}_1=(-3 ;-2 ; 5),$ $ \overrightarrow{u}_2=(0 ;-3 ; 1),$ $ \overrightarrow{u}_3=(3 ;-1 ; 5)$ và $\overrightarrow{u}_4=(-3 ;-1 ;-5)$.\\
		Vì $\overrightarrow{u} \cdot \overrightarrow{u}_4=0$ nên $d \perp d_4$.
	}
\end{ex}

%G:\My Drive\CODE12-2024\DE-ON-THEO BAI\2H5-TACH DE\Bai2-De2.tex
\begin{ex}%[2H5H2-5]
	Trong không gian với hệ trục tọa độ $Oxyz$, cho đường thẳng $d$ có véc-tơ chỉ phương $\overrightarrow{u}$ và mặt phẳng $(P)$ có véc-tơ pháp tuyến $\overrightarrow{n}$. Mệnh đề nào dưới đây đúng?
	\choice
	{$d$ song song với $(P)$ thì $\overrightarrow{u}$ cùng phương với $\overrightarrow{n}$}
	{$\overrightarrow{u}$ vuông góc với $\overrightarrow{n}$ thì $d$ song song với $(P)$}
	{\True $\overrightarrow{u}$ không vuông góc với $\overrightarrow{n}$ thì $d$ cắt $(P)$}
	{$d$ vuông góc với $(P)$ thì $\overrightarrow{u}$ vuông góc với $\overrightarrow{n}$}
	\loigiai{
		Ta có $\overrightarrow{u}$ không vuông góc với $\overrightarrow{n}$ thì $d$ cắt $(P)$.}
\end{ex}

%G:\My Drive\CODE12-2024\DE-ON-THEO BAI\2H5-TACH DE\Bai2-De2.tex
\begin{ex}%[2H5H2-3]
	Cho đường thẳng $d$ có phương trình tham số $\heva{&x=1+2 t \\& y=2-t \\& z=-3+t}$. Viết phương trình chính tắc của đường thẳng $d$.
	\choice
	{$d\colon  \dfrac{x-1}{2}=\dfrac{y-2}{-1}=\dfrac{z-3}{1}$}
	{$d\colon  \dfrac{x+1}{2}=\dfrac{y+2}{-1}=\dfrac{z-3}{1}$}
	{$d\colon \dfrac{x-1}{2}=\dfrac{y-2}{1}=\dfrac{z+3}{1}$}
	{\True $d\colon  \dfrac{x-1}{2}=\dfrac{y-2}{-1}=\dfrac{z+3}{1}$}
	\loigiai{
		Từ phương trình tham số ta thấy đường thẳng $d$ đi qua điểm tọa độ $(1 ; 2 ;-3)$ và có véc-tơ chỉ phương  $\overrightarrow{u}=(2 ;-1 ; 1)$.\\
		Suy ra phương trình chính tắc của $d$ là $ \dfrac{x-1}{2}=\dfrac{y-2}{-1}=\dfrac{z+3}{1}$.
	}
\end{ex}

%G:\My Drive\CODE12-2024\DE-ON-THEO BAI\2H5-TACH DE\Bai2-De2.tex
\begin{ex}%[2H5V2-3]
	Trong không gian $O x y z$, cho đường thẳng $d\colon \dfrac{x+3}{2}=\dfrac{y+1}{1}=\dfrac{z}{-1}$ và mặt phẳng $(P)\colon x+y-3 z-2=0$. Gọi $d'$ là đường thẳng nằm trong mặt phẳng $(P)$, cắt và vuông góc với $d$. Đường thẳng $d'$ có phương trình là $\dfrac{x+1}{a}=\dfrac{y}{5}=\dfrac{z+1}{c}$. Tính $S=a-c$.
	\choice
	{$3$}
	{$-7 $}
	{\True $-3$}
	{$4$}
	\loigiai{
		Phương trình tham số của $d\colon\heva{&x=-3+2 t \\& y=-1+t \\& z=-t.}$\\
		Tọa độ giao điểm của $d$ và $(P)$ là nghiệm của hệ
		\allowdisplaybreaks
		\begin{eqnarray*}
			&&\heva{&x=-3+2 t \\& y=-1+t \\& z=-t \\& x+y-3 z-2=0} \Leftrightarrow \heva{&x=-3+2 t \\& y=-1+t \\& z=-t \\& -3+2 t-1+t+3 t-2=0} \\&\Leftrightarrow&\heva{&t=1 \\& x=-1 \\& y=0 \\& z=-1} \Rightarrow d \cap(P)=M(-1 ; 0 ;-1).
		\end{eqnarray*}
		Theo đề bài, đường thẳng $d$ có véc-tơ chỉ phương $u_d=(2;1;-1)$, mặt phẳng $(P)$ có véc-tơ pháp tuyến $n_{(P)}=(1;1;-3)$.\\
		Vì $d'$ nằm trong mặt phẳng $(P)$, cắt và vuông góc với $d$ nên $d'$ đi qua $M$ và có véc-tơ chỉ phương $\overrightarrow{u}_{d'}=\left[\overrightarrow{n}_{(P)}, \overrightarrow{u}_d\right]=(2 ;-5 ;-1)$ hay $d'$ nhận véc-tơ $\overrightarrow{v}=(-2 ; 5 ; 1)$ làm véc tơ chỉ phương.\\
		Phương trình của $d'\colon  \dfrac{x+1}{-2}=\dfrac{y}{5}=\dfrac{z+1}{1}$.\\
		Do đó $S=a-c=-2-1=-3$.
} \end{ex}

	\Closesolutionfile{ans}

\subsection*{\indam{PHẦN II. Câu trắc nghiệm đúng sai. Thí sinh trả lời từ câu 1 đến câu 4. Trong mỗi ý a), b), c), d) ở mỗi câu, thí sinh chọn đúng hoặc sai.}}
	\setcounter{ex}{0}
	\Opensolutionfile{ans}[ans/B2-De2-2]

\begin{ex}%[2H5H2-5]
	Trong không gian $O x y z$ cho đường thẳng $d$ có phương trình tham số $\heva{&x=-1+2 t \\& y=1+t \\& z=3-2 t.}$
	\choiceTF
	{Giao điểm của đt $d$ và mặt phẳng $(P)\colon x+2 y-3 z+2=0$  là $I(0 ; 1 ; 2)$}
	{Véc-tơ $\overrightarrow{a}=(4 ; 2 ;-3)$ là một véc-tơ chỉ phương   của đường thẳng $d$}
	{\True  Đường thẳng $d$ đi qua điểm $A(-1 ; 1 ; 3)$}
	{\True Phương trình chính tắc của đường thẳng $d$ là $\dfrac{x+1}{2}=\dfrac{y-1}{1}=\dfrac{z-3}{-2}$}
	\loigiai{
		\begin{itemchoice}
			\itemch \textbf{Sai.} Vì  ta có $-1+2 t+2(1+t)-3(3-2 t)+2=0$ $\Leftrightarrow 10 t-6=0 \Leftrightarrow t=\dfrac{3}{5}$.\\
			Giao điểm của $d$ và $(P)$ là $B\left(\dfrac{1}{5} ; \dfrac{8}{5} ; \dfrac{9}{5}\right)$.\itemch \textbf{Sai.} Vì véc-tơ chỉ phương của đường thẳng $d$ là $\overrightarrow{u}_d=(2 ; 1 ;-2)$. \\Xét hai véc-tơ $\overrightarrow{a}=(4 ; 2 ;-3)$ và $\overrightarrow{u}_d=(2 ; 1 ;-2)$.\\
			Vì $\dfrac{1}{2} \neq \dfrac{-2}{-3}$ nên $\overrightarrow{a}=(4 ; 2 ;-3)$ và $\overrightarrow{u}_d$ không cùng phương. \\Do đó $\overrightarrow{a}$ không là véc-tơ chỉ phương của đường thẳng $d$.
			\itemch \textbf{Đúng.} Vì theo phương trình tham số của đường thẳng $d$ thì $d$ đi qua điểm $A(-1 ; 1 ; 3)$.
			\itemch \textbf{Đúng.} Vì  đường thẳng $d$ đi qua điểm $M(-1 ; 1 ; 3)$ và có một  véc-tơ chỉ phương   là $\overrightarrow{u}_d=(2 ; 1 ;-2)$ nên có phương trình chính tắc là $\dfrac{x+1}{2}=\dfrac{y-1}{1}=\dfrac{z-3}{-2}$.
		\end{itemchoice}
	}
\end{ex}

%G:\My Drive\CODE12-2024\DE-ON-THEO BAI\2H5-TACH DE\Bai2-De2.tex
\begin{ex}%[2H5H2-7]
	Trong không gian $O x y z$ cho đường thẳng $d\colon \dfrac{x-1}{2}=\dfrac{y+1}{-1}=\dfrac{z}{1}$ và mặt phẳng $(P)\colon x+y+2 z-3=0$.
	\choiceTF
	{\True Đường thẳng $d'$ đi qua điểm $A(1 ; 0 ;-1)$ và vuông góc với mặt phẳng $(P)$. Phương trình tham số của đường thẳng $d'$ là $\heva{&x=1+t \\& y=t \\& z=-1+2 t}$}
	{ Đường thẳng $d$ có một véc-tơ chỉ phương là $\overrightarrow{a}=(1 ;-1 ; 0)$}
	{ Đường thẳng $d$ đi qua điểm $M(2 ;-1 ; 1)$}
	{\True Góc giữa đường thẳng $d$ và mặt phẳng $(P)$ bằng $30^{\circ}$}
	\loigiai{
		\begin{itemchoice}
			\itemch \textbf{Đúng.} Vì đường thẳng $d'$ vuông góc với mặt phẳng $(P)$ nên có một véc-tơ chỉ phương là $\overrightarrow{u}_{d'}=\overrightarrow{n}_P=(1 ; 1 ; 2)$.\\
			Kết hợp với $d'$
			đi qua $A(1;0;-1)$ nên phương trình tham số của đường thẳng $d'$ là $\heva{&x=1+t \\& y=t \\& z=-1+2 t.}$
			\itemch \textbf{Sai.} Vì đường thẳng $d$ có một véc-tơ chỉ phương là $\overrightarrow{u}=(2 ;-1 ; 1)$. Mà $\overrightarrow{a},$ $ \overrightarrow{u}$ không cùng phương nên $\overrightarrow{a}$ không là véc-tơ chỉ phương  của $d$.
			\itemch \textbf{Sai.} Vì lấy $M\in d \Rightarrow M(1+2t; -1- t;  t).$ Không có giá trị nào của $t$ thỏa hệ $\heva{&1+2t=2 \\& -1- t=-1 \\&  t=1}$ nên đường thẳng $d$ không đi qua điểm $M(2 ;-1 ; 1)$.
			\itemch \textbf{Đúng.} Vì gọi $\alpha$ là góc giữa đường thẳng $d$ và mặt phẳng $(P)$. Ta có
			\allowdisplaybreaks
			\begin{eqnarray*}
				\sin \alpha&=&\bigg|\cos \left(\overrightarrow{u}_d, \overrightarrow{n}_P\right)\bigg|=\dfrac{\left|\overrightarrow{u}_d \cdot \overrightarrow{n}_P\right|}{\left|\overrightarrow{u}_d\right|\left|\overrightarrow{n}_P\right|}\\&=&\dfrac{|2 \cdot 1+(-1) \cdot 1+1 \cdot 2|}{\sqrt{2^{2}+(-1)^{2}+1^{2}} \cdot \sqrt{1^{2}+1^{2}+2^{2}}}=\dfrac{1}{2} \Rightarrow \alpha=30^{\circ}.
			\end{eqnarray*}
		\end{itemchoice}
	}
\end{ex}

%G:\My Drive\CODE12-2024\DE-ON-THEO BAI\2H5-TACH DE\Bai2-De2.tex
\begin{ex}%[2H5H2-6] 
	Trong không gian $Oxyz$ cho đường thẳng $d$ có phương trình tham số $\heva{&x=2+t \\& y=3-2 t \\& z=3 t.}$
	\choiceTF
	{\True Đường thẳng $d$ có một véc-tơ chỉ phương là $\overrightarrow{u}=(1 ;-2 ; 3)$}
	{ Đường thẳng $d$ đi qua điểm $M(1 ;-2 ; 3)$}
	{\True  Đường thẳng $d'$ đi qua điểm $A(1 ; 2 ;-2)$ và song song với đường thẳng $d$. Phương trình tham số của đường thẳng $d'$ là $\heva{&x=1+t \\& y=2-2 t \\& z=-2+3 t}$}
	{ Khoảng cách từ điểm $B(0 ; 1 ; 2)$ đến đường thẳng $d$ bằng $3$}
	\loigiai{
		\begin{itemchoice}
			\itemch \textbf{Đúng.} Vì  đường thẳng $d$ có một véc-tơ chỉ phương là $\overrightarrow{u}=(1 ;-2 ; 3)$.
			\itemch \textbf{Sai.} Vì lấy $M\in d \Rightarrow M(2+t; 3-2 t; 3 t).$ Không có giá trị nào của $t$ thỏa hệ $\heva{&2+t=1 \\& 3-2 t=-2 \\& 3 t=3}$ nên đường thẳng $d$ không đi qua điểm $M(1 ;-2 ; 3)$.
			\itemch \textbf{Đúng.} Vì $d'\parallel d$ nên $d'$ có một véc-tơ chỉ phương  là $\overrightarrow{u}_{d'}=\overrightarrow{u}_d=(1 ;-2 ; 3)$. Vậy đường thẳng $d'$ có phương trình tham số là $\heva{&x=1+t \\& y=2-2 t \\& z=-2+3 t.}$
			\itemch \textbf{Sai.} Vì lấy điểm $C (3 ; 1 ; 3)\in d,$ ta có $\overrightarrow{u}_d=(1 ;-2 ; 3),$ $ \overrightarrow{B C}=(3 ; 0 ; 1)$.\\
			Khoảng cách từ điểm $B$ đến đường thẳng $d$ là $h=\dfrac{\left[\overrightarrow{B C} , \overrightarrow{u}_d\right]}{\left|\overrightarrow{u}_d\right|}=\dfrac{\sqrt{364}}{7}$.\end{itemchoice}
	}
\end{ex}

%G:\My Drive\CODE12-2024\DE-ON-THEO BAI\2H5-TACH DE\Bai2-De2.tex
\begin{ex}%[2H5V2-5] 
	Trong không gian $O x y z$ cho đường thẳng $d$ có phương trình tham số $\heva{&x=2+2 t \\& y=1-t \\& z=1+2 t.}$
	\choiceTF
	{ Đường thẳng $d$ có một  véc-tơ chỉ phương là $\overrightarrow{a}=(2 ; 1 ; 1)$}
	{\True Điểm $B(4 ; 0 ; 3)$ thuộc đường thẳng $d$}
	{ Khoảng cách giữa đường thẳng $d$ và mặt phẳng $(P)\colon x+2 y-3=0$ bằng $1$}
	{\True  Đường thẳng $d$ và đường thẳng $d'\colon \dfrac{x}{4}=\dfrac{y-2}{-2}=\dfrac{z+1}{4}$ trùng nhau}
	\loigiai{
		\begin{itemchoice}
			\itemch \textbf{Sai.} Vì  véc-tơ chỉ phương của đường thẳng $d$ là $\overrightarrow{u}_d=(2 ;-1 ; 2)$.\\ Xét hai  véc-tơ $\overrightarrow{a}=(2 ; 1 ; 1)$ và $\overrightarrow{u}_d=(2 ;-1 ; 2)$.\\
			Vì $\dfrac{2}{2} \neq \dfrac{1}{-1}$ nên $\overrightarrow{a}=(2 ; 1 ; 1)$ và $\overrightarrow{u}_d$ không cùng phương. \\Do đó $\overrightarrow{a}$ không là  véc-tơ chỉ phương của đường thẳng $d$.
			\itemch \textbf{Đúng.} Vì thế tọa độ điểm $B(4 ; 0 ; 3)$ vào phương trình của đường thẳng $d$, ta có $\heva{&4=2+2 t \\& 0=1-t \\& 3=1+2 t}\Leftrightarrow t=1$ nên điểm $B(4 ; 0 ; 3)$ thuộc đường thẳng $d$. 
			\itemch \textbf{Sai.} Vì  ta có $\overrightarrow{u}_d=(2 ;-1 ; 2),$ $\overrightarrow{n}_{(P)}=(1 ; 2 ; 0)$ thỏa mãn $\overrightarrow{u}_d \cdot \overrightarrow{n}_{(P)}=0$. \\Suy ra $\overrightarrow{u}_d \perp \overrightarrow{n}_{(P)} \Rightarrow \hoac{&d \parallel (P) \\& d \subset(P).}$\\
			Mặt khác điểm $A(2 ; 1 ; 1) \in d$ nhưng $A \notin(P)$ nên $d \parallel(P)$.\\Theo câu \textbf{b)}, ta có $$B(4 ; 0 ; 3)\in d\Rightarrow \mathrm{d}(d,(P))=\mathrm{d}(B,(P))=\dfrac{4+2 \cdot 0-3}{\sqrt{1^2+2^2}}=\dfrac{\sqrt{5}}{5}.$$ 
			\itemch \textbf{Đúng.} Vì  ta có $\overrightarrow{u}_d=(2 ;-1 ; 2),$ $ \overrightarrow{u}_d'=(4 ;-2 ; 4)$ cùng phương.\hfill $(1)$\\
			Điểm $A(2 ; 1 ; 1) \in d$ và $A(2 ; 1 ; 1) \in d'$. \hfill $(2)$\\
			Từ $(1)$ và $(2)$ chứng tỏ $d,$ $ d'$ trùng nhau.
		\end{itemchoice}
	}
\end{ex}
\Closesolutionfile{ans}
\subsection*{\indam{PHẦN III. Câu trắc nghiệm trả lời ngắn. Thí sinh trả lời từ câu 1 đến câu 6 vào ô kết quả.}}
\setcounter{ex}{0}
\Opensolutionfile{ans}[ans/B2-De2-3]
\begin{ex}%[2H5V2-5] 
	Trong không gian $O x y z$, cho ba điểm $A(1 ;-2 ; 1),$ $ B(5 ; 0 ;-1),$ $ C(3 ; 1 ; 2)$ và mặt phẳng $(Q)\colon 3 x+y-z+3=0$. Gọi $M(a ; b ; c)$ là điểm thuộc $(Q)$ thỏa mãn $M A^2+M B^2+2 M C^2$ nhỏ nhất. Khi đó tổng $a+b+3 c$ bằng bao nhiêu?\\
	\shortans[oly]{$5$}
	\loigiai{
		Gọi $E$ là điểm thỏa mãn $\overrightarrow{E A}+\overrightarrow{E B}+2 \overrightarrow{E C}=\overrightarrow{0} \Rightarrow E(3 ; 0 ; 1)$.\\
		Ta có 
		\allowdisplaybreaks
		\begin{eqnarray*}
			S&=&M A^2+M B^2+2 M C^2=\overrightarrow{M A}^2+\overrightarrow{M B}^2+2 \overrightarrow{M C}^2\\&=&\left(\overrightarrow{M E}+\overrightarrow{E A}\right)^2+\left(\overrightarrow{M E}+\overrightarrow{E B}\right)^2+2\left(\overrightarrow{M E}+\overrightarrow{E C}\right)^2\\&=&4 M E^2+E A^2+E B^2+2 E C^2.
		\end{eqnarray*}
		Vì $E A^2+E B^2+2 E C^2$ không đổi nên $S$ nhỏ nhất khi và chỉ khi $M E$ nhỏ nhất.\\
		Suy ra $ M$ là hình chiếu vuông góc của $E$ lên $(Q)$. \\Do đó $ME\perp (Q)$ nên $\overrightarrow{u}_{ME}=\overrightarrow{n}_{(Q)}=(3;1;-1)$, và $E(3 ; 0 ; 1)$.\\
		Suy ra phương trình đường thẳng $M E\colon \heva{&x=3+3 t \\& y=t \\& z=1-t.}$\\
		Tọa độ điểm $M$ là nghiệm của hệ phương trình $\heva{&x=3+3 t \\& y=t \\& z=1-t \\& 3 x+y-z+3=0} \Leftrightarrow\heva{&x=0 \\& y=-1 \\& z=2 \\& t=-1.}$\\
		Vậy $M(0 ;-1 ; 2) \Rightarrow a=0,$ $ b=-1,$ $ c=2 \Rightarrow a+b+3 c=5$.
	}
\end{ex}

\begin{ex}%[2H5H2-8] 
	Trong không gian $O x y z$, một viên đạn được bắn ra từ điểm $A(3 ; 4 ; 2)$ và trong $4$ giây đầu đạn đi với vận tốc không đổi, véc-tơ vận tốc (trên giây) là $\overrightarrow{v}=(4 ; 5 ; 1)$. Biết viên đạn trúng mục tiêu tại điểm $M(13 ; b ; c)$, tính $b+2 c$.\\
	\shortans[oly]{$25{,}5$}
	\loigiai{
		Phương trình đường đi của viên đạn $\heva{&x=3+4 t \\& y=4+5 t \\& z=2+t}$ với $0 \leq t \leq 4$.\\
		Viên đạn trúng mục tiêu tại điểm $M(13 ; b ; c)$ khi $M$ nằm trên đường đi của viên đạn
		$$
		\Rightarrow\heva{& 1 3 = 3 + 4 t \\&
			b = 4 + 5 t  \\&
			c = 2 + t } \Leftrightarrow \heva{&
			t=\dfrac{5}{2} \\&
			b=\dfrac{33}{2} \\&
			c=\dfrac{9}{2}} \Rightarrow b+2 c=\dfrac{33}{2}+9=\dfrac{51}{2}=25{,}5.
		$$}
\end{ex}

\begin{ex}%[2H5V2-4]
	Trong không gian với hệ trục tọa độ $O x y z$, cho điểm $M(3 ; 3 ;-2)$ và hai đường thẳng $d_1\colon \dfrac{x-1}{1}=\dfrac{y-2}{3}=\dfrac{z}{1} ;$ $ d_2\colon \dfrac{x+1}{-1}=\dfrac{y-1}{2}=\dfrac{z-2}{4}$. Đường thẳng $d$ đi qua $M$ cắt $d_1,$ $ d_2$ lần lượt tại $A$ và $B$. Khi đó độ dài đoạn thẳng $A B$ bằng bao nhiêu?\\
	\shortans[oly]{$3$}
	\loigiai{
		Ta có
		\begin{itemize}
			\item Phương trình tham số của $d_1\colon \heva{&x=1+t_1 \\& y=2+3 t_1 \\& z=t_1} ;$ $ t_1 \in \mathbb{R},$\\ Do $A \in d_1$ nên $ A\left(1+t_1 ; 2+3 t_1 ; t_1\right).$
			\item Phương trình tham số của $d_2\colon \heva{&x=-1-t_2 \\& y=1+2 t_2 \\& z=2+4 t_2} ;$ $t_2 \in \mathbb{R}.$ \\Do $ B \in d_2$ nên $ B\left(-1-t_2 ; 1+2 t_2 ; 2+4 t_2\right)$.
		\end{itemize}
		$\overrightarrow{M A}=\left(t_1-2 ; 3 t_1-1 ; t_1+2\right) ;$ $ \overrightarrow{M B}=\left(-4-4 t_2 ;-2+2 t_2 ; 4+4 t_2\right)$.\\
		Vì $A,$ $ B,$ $ M$ thẳng hàng nên 
		\allowdisplaybreaks
		\begin{eqnarray*}
			&&\overrightarrow{M A}=k \overrightarrow{M B}, k \in \mathbb{R}
			\\&\Leftrightarrow&\heva{&t_1-2=-4 k-k t_2 \\& 3 t_1-1=-2 k+2 k t_2 \\& t_1+2=4 k+4 k t_2} \Leftrightarrow\heva{&t_1+4 k+k t_2=2 \\& 3 t_1+2 k-2 k t_2=1 \\& t_1-4 k-4 k t_2=-2}\\& \Leftrightarrow&\heva{&t_1=0 \\& k=\dfrac{1}{2} \\& k t_2=0} \Leftrightarrow\heva{&t_1=0 \\& k=\dfrac{1}{2} \\& t_2=0.}
		\end{eqnarray*}
		Vậy $A(1 ; 2 ; 0)$ và $B(-1 ; 1 ; 2) \Rightarrow \overrightarrow{A B}=(-2 ;-1 ; 2)$.\\ Độ dài đoạn thẳng $A B=\left|\overrightarrow{A B}\right|=3$.
	}
\end{ex}

\begin{ex}%[2H5V2-8] 
	Hình vẽ dưới đây là hình ảnh Cầu Cổng Vàng (The Golden Gate Bridge) ở Mỹ. Xét hệ trục toạ độ $O x y z$ với $O$ là bệ của chân cột trụ tại mặt nước, trục $O z$ trùng với cột trụ, mặt phẳng $O x y$ là mặt nước và xem như trục $O y$ cùng phương với cầu như hình vẽ. Dây cáp $A D$ (xem như là một đoạn thẳng) đi qua đỉnh $D$ thuộc trục $O z$ và điểm $A$ thuộc mặt phẳng $O y z$, trong đó điểm $D$ là đỉnh cột trụ cách mặt nước $227$ m, điểm $A$ cách mặt nước $75$ m và cách trục $O z$ khoảng $343$ m. \begin{flushright}
		\textit{(Nguồn: https://www.goldengate.org/assets/1/6/ggb-exhibit-chapter-statistics.pdf)}
	\end{flushright}
	\begin{center}
		\includegraphics[scale=0.7]{image/Cau-cong-vang}
	\end{center}
	Giả sử ta dùng một đoạn dây nối điểm $N$ trên dây cáp $A D$ và điểm $M$ trên thành cầu, biết $M$ cách mặt nước $75$ m và $M N$ song song với cột trụ. Tính độ dài $M N$ (đơn vị mét) biết điểm $M$ cách trục $O z$ một khoảng bằng $230$ m (kết quả làm tròn đến hàng phần mười).\\
	\shortans[oly]{$50{,}1$}
	\loigiai{
		Chọn một đơn vị trên các trục bằng $1$ m.\\
		Ta có $D(0 ; 0 ; 227),$ $ A(0 ;-343 ; 75),$ $ M(0 ;-230 ; 75)$, $\overrightarrow{A D}=(0 ; 343 ; 152)$.\\ Phương trình đường thẳng $A D\colon \heva{&x=0 \\& y=343 t \\& z=227+152 t} \Rightarrow N(0 ; 343 t ; 227+152 t)$.\\
		Ta có $\overrightarrow{M N}=(0 ; 343 t+230 ; 152+152 t)$, $M N$ song song với trục $O z$, suy ra $$ 343 t+230=0 \Rightarrow t=-\dfrac{230}{343} \Rightarrow M N=152+152 \cdot \left(-\dfrac{230}{343}\right) \approx 50,1(m).$$
	}
\end{ex}

\begin{ex}%[2H5V2-5] 
	Trong không gian với hệ trục tọa độ $O x y z$, cho hai điểm $A(-2 ;-1 ; 2)$ và $B(5 ;-1 ; 1)$. Đường thẳng $d'$ là hình chiếu của đường thẳng $A B$ lên mặt phẳng $(P)\colon x+2 y+z+2=0$ có một véc-tơ chỉ phương $\overrightarrow{u}=(a ; b ; 2)$. Tính $S=a+b$.\\
	\shortans[oly]{$-4$}
	\loigiai{
		Gọi $(Q)$ là mặt phẳng chứa đường thẳng $A B$ và vuông góc $(P)$. 
		\\Khi đó, đường thẳng $d'=(P) \cap(Q)$.\\
		Có $\heva{&\overrightarrow{n}_{(Q)} \perp \overrightarrow{A B}=(7 ; 0 ;-1) \\& \overrightarrow{n}_{(Q)} \perp \overrightarrow{n}_{(P)}=(1 ; 2 ; 1)}$. Suy ra chọn $\overrightarrow{n}_{(Q)}=\left[\overrightarrow{A B} ; \overrightarrow{n}_{(P)}\right]=(2 ;-8 ; 14)$.\\
		Mặt khác $\heva{&\overrightarrow{u}_{\left(d'\right)} \perp \overrightarrow{n}_{(P)} \\& \overrightarrow{u}_{\left(d'\right)} \perp \overrightarrow{n}_{(Q)}} $. Suy ra chọn $\overrightarrow{u}_{\left(d'\right)}=\left[\overrightarrow{n}_{(P)} ; \overrightarrow{n}_{(Q)}\right]=(36 ;-12 ;-12)$ cùng phương với $\overrightarrow{u}(-6 ; 2 ; 2)$.\\
		Như vậy $a=-6,$ $ b=2 \Rightarrow a+b=-4$.
	}
\end{ex}

\begin{ex}%[2H5V2-4]
	Trong không gian $O x y z$, cho điểm $A(1 ; 0 ; 2)$ và đường thẳng $d\colon \dfrac{x-1}{1}=\dfrac{y}{1}=\dfrac{z+1}{2}$. Đường thẳng $\Delta$ đi qua $A$, vuông góc và cắt $d$ đi qua điểm $M(a ; b ; 0)$. Tính $\dfrac{a}{b}$.\\
	\shortans[oly]{$1{,}5$}
	\loigiai{
		Đường thẳng $d$ có véc-tơ chỉ phương là $\overrightarrow{u}_d=(1 ; 1 ; 2)$.\\
		Gọi giao điểm của đường thẳng $\Delta$ và $d$ là $B$.\\
		Vì $B\in d$ nên $B(1+t ; t ;-1+2 t)\Rightarrow \overrightarrow{A B}=(t ; t ;-3+2 t)$.\\
		Vì đường thẳng $\Delta$ vuông góc với đường thẳng $d$ nên $$\overrightarrow{A B} \perp \overrightarrow{u}_d \Leftrightarrow \overrightarrow{A B} \cdot \overrightarrow{u}_d=0 \Leftrightarrow 1 \cdot t+1  \cdot  t+2  \cdot (-3+2 t)=0 \Leftrightarrow t=1.$$
		Do đó $B(2 ; 1 ; 1),$ $ \overrightarrow{A B}=(1 ; 1 ;-1)$.\\
		Đường thẳng $\Delta$ đi qua điểm $A(1 ; 0 ; 2)$ và có vectơ chỉ phương là $\overrightarrow{A B}=(1 ; 1 ;-1)$ nên có phương trình tham số $\heva{&x=1+t \\& y=0+t  \\& z=2-t.}$\\ $M(a ; b ; 0) \in \Delta \Rightarrow M(3 ; 2 ; 0) \Rightarrow a=3 ;$ $ b=2 ;$ $ \dfrac{a}{b}=1{,}5.$
	}
\end{ex}
\centerline{---HẾT---}
\Closesolutionfile{ans}
%\newpage
%%=====================
%\begin{center}
%\textbf{\large BẢNG ĐÁP ÁN}
%\end{center}
%\noindent\textbf{ĐÁP ÁN PHẦN I}
%\inputansbox{10}{ans/B2-De2-1}
	
%\noindent\textbf{ĐÁP ÁN PHẦN II}
%\inputansbox[2]{2}{ans/B2-De2-2}
	
%\noindent\textbf{ĐÁP ÁN PHẦN III}
%\inputansbox[3]{6}{ans/B2-De2-3}




%%Bài 3.
\setcounter{dang}{0}
\newpage
\section{CÔNG THỨC TÍNH GÓC TRONG KHÔNG GIAN}
\subsection{LÝ THUYẾT CẦN NHỚ}
\subsubsection{Góc giữa hai mặt phẳng}
\begin{itemize}
	\item [\iconMT] \indam{Công thức:} Gọi $\vec{n_1}=(a_1;b_1;c_1)$, $\vec{n_2}=(a_2;b_2;c_2)$ lần lượt là vectơ pháp tuyến của $(P)$ và $(Q)$; $\varphi$ là góc giữa hai mặt phẳng $(P)$ và $(Q)$, với $0^\circ \leq \varphi \leq 90^\circ$.
	Khi đó
	\boxmini{$\cos \varphi =\bigg|\cos\left(\vec{n_1}, \vec{n_2}\right) \bigg| =\dfrac{\bigg|a_1a_2+b_1b_2+c_1c_2\bigg|}{\sqrt{a_1^2+b_1^2+c_1^2} \cdot \sqrt{a_2^2+b_2^2+c_2^2}}$}
	\item [\iconMT] \indam{Chú ý:}
	\begin{itemize}
		\item [$\bullet$] Nếu $(P)$ song song hoặc trùng $(Q)$ thì $\varphi =0^\circ$.
		\item [$\bullet$] Nếu $(P)\perp (Q)$ thì $\varphi =90^\circ$. Khi đó $\vec{n_1}\cdot \vec{n_2}=0 \Leftrightarrow a_1a_2+b_1b_2+c_1c_2=0$.
	\end{itemize}
\end{itemize}
\subsubsection{Góc giữa hai đường thẳng}
\begin{itemize}
	\item [\iconMT] \indam{Công thức:}  Gọi $\vec{u}=(u_1;u_2;u_3)$, $\vec{v}=(v_1;v_2;v_3)$ lần lượt là vectơ chỉ phương của  $d_1$ và $d_2$; $\varphi$ là góc giữa hai đường thẳng $d_1$ và $d_2$, với $0^\circ \leq \varphi \leq 90^\circ$.
	Khi đó
	\boxmini{$\cos \varphi =\bigg|\cos\left(\vec{u}, \vec{v}\right) \bigg| =\dfrac{\bigg|u_1v_1+u_2v_2+u_3v_3\bigg|}{\sqrt{u_1^2+u_2^2+u_3^2} \cdot \sqrt{v_1^2+v_2^2+v_3^2}}$}
	\item [\iconMT] \indam{Chú ý:}
	\begin{itemize}
		\item [$\bullet$] Nếu $d_1$ song song hoặc trùng $d_2$ thì $\varphi =0^\circ$.
		\item [$\bullet$] Nếu $d_1\perp d_2$ thì $\varphi =90^\circ$. Khi đó $\vec{u} \cdot\vec{u} =0 \Leftrightarrow u_1v_1+u_2v_2+u_3v_3=0$.
	\end{itemize}
\end{itemize}

\subsubsection{Góc giữa đường thẳng và mặt phẳng}
\begin{itemize}
	\item [\iconMT] \indam{Công thức:}  Gọi $\vec{u}=(u_1;u_2;u_3)$, $\vec{n}=(A;B;C)$ lần lượt là vectơ chỉ phương của  $d$ và vectơ pháp tuyến của $(P)$; $\varphi$ là góc giữa đường thẳng $d$ và mặt phẳng $(P)$, với $0^\circ \leq \varphi \leq 90^\circ$.
	Khi đó
	\boxmini{$\sin \varphi =\bigg|\cos\left(\vec{u}, \vec{n}\right) \bigg| =\dfrac{\bigg|u_1A+u_2B+u_3C\bigg|}{\sqrt{u_1^2+u_2^2+u_3^2} \cdot \sqrt{A^2+B^2+C^2}}$}
	\item [\iconMT] \indam{Chú ý:}
	\begin{itemize}
		\item [$\bullet$] Nếu $d$ song song hoặc trùng $(P)$ thì $\varphi =0^\circ$, khi đó $\vec{u} \perp \vec{n}$
		\item [$\bullet$] Nếu $d$ vuông góc với $(P)$ thì $\varphi =90^\circ$, khi đó $\vec{u} =k \cdot \vec{n}$.
	\end{itemize}
\end{itemize}
\subsection{PHÂN LOẠI, PHƯƠNG PHÁP GIẢI TOÁN}
\begin{dang}{Tính góc trong không gian Oxyz}
	\begin{itemize}
		\item [$\bullet$] Xác định vectơ chỉ phương (vectơ pháp tuyến);
		\item [$\bullet$] Áp dụng đúng công thức.
	\end{itemize}
\end{dang}
\boxmini{BÀI TẬP TỰ LUẬN}
\setcounter{vd}{0}
\begin{vd}
	Trong không gian $Oxyz$, tính góc giữa hai mặt phẳng sau:
	\begin{enumEX}[a)]{1}
		\item $(P)\colon x+y+4z-2=0$ và $(Q)\colon 2x-2z+7=0$.
		\item $(P)\colon 2x - y-2z-9=0$ và $(Q)\colon  x - y - 6=0$.
	\end{enumEX}
	\loigiai
	{
	\begin{enumEX}[a)]{1}
		\item Ta có $\vec{n}_P=(1;1;4)$, $\vec{n}_Q=(2;0;-2)$ lần lượt là vectơ pháp tuyến của $(P)$ và $(Q)$.\\
		Suy ra $\cos\left((P),(Q)\right)=|\cos\left(\vec{n}_P,\vec{n}_Q\right)|=\dfrac{|\vec{n}_P\cdot \vec{n}_Q|}{|\vec{n}_P|\cdot |\vec{n}_Q|}=\dfrac{|2+0-8|}{\sqrt{18}\cdot \sqrt{8}}=\dfrac{1}{2}$.\\
		Vậy góc giữa $(P)$ và $(Q)$ bằng $60^\circ$.
		\item $(P)\colon 2x - y-2z-9=0$ có $1$ vectơ pháp tuyến là $\overrightarrow{n_1} = (2;-1;-2)$.\\
		$(Q)\colon  x - y - 6=0$ có $1$ vectơ pháp tuyến là $\overrightarrow{n_2} = (1;-1;0)$.\\
		$\cos \left(\left( P \right);\left( Q \right)\right) = \dfrac{\left|\overrightarrow{n_1}, \overrightarrow{n_2}\right|}{\left|\overrightarrow{n_1}\right|\left|\overrightarrow{n_2}\right|} $
		$= \dfrac{\left|2\cdot 1 + \left(- 1\right)\left(- 1\right) + 0\right|}{\sqrt{2^2 + 1^2 + 2^2}\cdot \sqrt{1^2+ 1^2 + 0}}= \dfrac{1}{\sqrt{2}} \Rightarrow \left(\left( P \right);\left( Q \right)\right) = 45^\circ$.
	\end{enumEX}
	}
\end{vd}
\dongcham{10}
\begin{vd}%[Phan Quốc Trí]%[2H3B3-4]%
	Trong không gian $Oxyz$, tính góc giữa hai đường thẳng sau:
	\begin{enumEX}[a)]{1}
		\item $d:\heva{&x=1-t\\&y=t \\&z=0}$ và $d': \dfrac{x}{-2}=\dfrac{y}{1}=\dfrac{z-1}{-2}$.
		\item $d_1\colon \heva{&x=2+t\\&y=-1+t\\&z=3}$ và $ d_2\colon \heva{&x=1-t'\\&y=2\\&z=-2+t'}$. 
	\end{enumEX} 
	\loigiai{
		\begin{enumEX}[a)]{1}
			\item Gọi $\varphi$ là góc giữa hai đường thẳng $d$ và $d'$. Ta có
			$$\cos \varphi = \dfrac{\left| (-1)\cdot (-2) + 1 \cdot 1+0\cdot (-2) \right|}{\sqrt{(-1)^2+1^2+0^2}\cdot \sqrt{(-2)^2+1^2+(-2)^2}} = \dfrac{\sqrt{2}}{2} \Rightarrow \varphi = 45^{\circ}.$$
			\item $d_1$ có VTCP $\overrightarrow{v}_1=(1;1;0)$ và $d_2$ có VTCP $\overrightarrow{v}_2=(-1;0;1)$, $\left|v_1 \right|=\sqrt{2} $, $\left|v_2 \right|=\sqrt{2}$.\\
			Khi đó góc giữa hai đường thẳng $d_1$ và $d_2$ là
			\[\cos \left(d_1, d_2 \right)=\dfrac{\left|\overrightarrow{v}_1\cdot \overrightarrow{v}_2 \right|}{\left|\overrightarrow{v}_1 \right|\cdot\left|\overrightarrow{v}_2 \right|}=\dfrac{1}{2}. \]
			Vậy góc giữa hai đường thẳng $d_1$ và $d_2$ là $60^\circ$.
		\end{enumEX}
	}
\end{vd}
\dongcham{10}
\begin{vd}%[2H3B3-4]%
	Trong không gian $Oxyz$, tính góc giữa đường thẳng và mặt phẳng sau:
	\begin{enumEX}[a)]{1}
		\item $d\colon\dfrac{x-1}{1}=\dfrac{y}{2}=\dfrac{z+1}{-1}$ và $(P)\colon x-y+2z+1=0$.
		\item $d\colon \dfrac{x-1}{4}=\dfrac{y-6}{3}=\dfrac{z+4}{1}$ và $(P)\colon 4x+3y-z+1=0$
	\end{enumEX}
	\loigiai{
		\begin{enumEX}[a)]{1}
			\item Ta có $\overrightarrow{n}_{(P)}=(1;-1; 2)$ và $\overrightarrow{u}_d=(1; 2;-1)$.\\
			Vậy $\sin\left(d;(P)\right)=\dfrac{\left|\overrightarrow{n}_{(P)}\cdot\overrightarrow{u}_d\right|}{\left|\overrightarrow{n}_{(P)}\right|\cdot\left|\overrightarrow{u}_d\right|} =\dfrac{|1-2-2|}{\sqrt{6}\cdot\sqrt{6}}=\dfrac{1}{2}\Rightarrow \left(d;(P)\right)=30^{\circ}$.
			\item Mặt phẳng $(P)$ có vectơ pháp tuyến $\overrightarrow{n}=(4;3;-1)$.\\
			Đường thẳng $d$ có vectơ chỉ phương $\overrightarrow{u}=(4;3;1)$.\\
			Gọi $\alpha$ là góc giữa $d$ và $(P)$, ta có $\sin\alpha=\dfrac{\left|\overrightarrow{n}\cdot \overrightarrow{u}\right|}{\left|\overrightarrow{n}\right|\cdot \left|\overrightarrow{u}\right|}=\dfrac{|16+9-1|}{\sqrt{16+9+1}\cdot \sqrt{16+9+1}}=\dfrac{12}{13} \Rightarrow \alpha \approx 67,38^\circ$.
		\end{enumEX}
		}
\end{vd}
\dongcham{17}
\boxmini{BÀI TẬP TRẮC NGHIỆM}
\setcounter{ex}{0}
\Opensolutionfile{ans}[ans/2H5-B3-d1]
\begin{ex}
	Cho mặt phẳng $(P):x+2y-2z+3=0$, mặt phẳng $(Q):x-3y+5z-2=0$. Cosin của góc giữa hai mặt phẳng $(P)$, $(Q)$ là
	\choice
	{$-\dfrac{\sqrt{35}}{7}$}
	{$\dfrac{5}{7}$}
	{\True $\dfrac{\sqrt{35}}{7}$}
	{$-\dfrac{5}{7}$}
	\loigiai{
		Mặt phẳng $(P)$ có ve-tơ pháp tuyến là $\overrightarrow{n}_1=(1;2;-2)$\\
		Mặt phẳng $(Q)$ có vec-tơ pháp tuyến là $\overrightarrow{n}_2=(1;-3;5)$.\\
		Ta có $\cos[(P),(Q)]=\left|\cos(\overrightarrow{n}_1,\overrightarrow{n}_2)\right|=\dfrac{|\overrightarrow{n}_1\cdot\overrightarrow{n}_2|}{|\overrightarrow{n}_1|\cdot|\overrightarrow{n}_2|}=\left|\dfrac{-15}{3\sqrt{35}}\right|=\dfrac{\sqrt{35}}{7}$.}
\end{ex}

\begin{ex}
	Góc giữa hai mặt phẳng $(P):x+2y+z+4=0$ và $(Q):-x+y+2z+3=0$ bằng
	\choice
	{$45^{\circ}$}
	{$90^{\circ}$}
	{$30^{\circ}$}
	{\True $60^{\circ}$}
	\loigiai{
		Gọi $\varphi$ là góc giữa $(P)$ và $(Q)$. Ta có
		$$\cos \varphi = \dfrac{\left| 1\cdot (-1)+2\cdot 1+1\cdot 2 \right|}{\sqrt{1^2+2^2+1^2} \cdot \sqrt{(-1)^2 +1^2+2^2}}=\dfrac{1}{2} \Rightarrow \varphi = 60^{\circ}.$$
	}
\end{ex}

\begin{ex}
	Tính góc $\alpha$ giữa mặt $(P)\colon x+z-4=0$ và mặt phẳng $(Oxy)$.
	\choice
	{\True $45^\circ$}
	{$30^\circ$}
	{$90^\circ$}
	{$60^\circ$}
	\loigiai{
		Ta có $\vec{n}_{(P)}=(1;0;1)$, $\vec{n}_{(Oxy)}=(0;0;1)$.\\
		Suy ra $\cos \alpha =\dfrac{\left| 1\cdot 0+0\cdot 0+1\cdot 1 \right|}{\sqrt{2}\cdot \sqrt{1}}=\dfrac{\sqrt{2}}{2}$.\\
		Vậy $\widehat{((P);(Q))}=45^\circ$.
	}
\end{ex}

\begin{ex}
	Cho điểm $H\left(2;1;2 \right)$, điểm $H$ là hình chiếu vuông góc của gốc tọa độ $O$ xuống mặt phẳng $\left(P \right)$, số đo góc giữa mặt phẳng $\left(P \right)$ và mặt phẳng $\left(Q \right):x+y-11=0$ là
	\choice
	{\True $45^\circ$}
	{$30^\circ$}
	{$60^\circ$}
	{$90^\circ$}
	\loigiai
	{
		Vì điểm $H$ là hình chiếu vuông góc của gốc tọa độ $O$ xuống mặt phẳng $(P)$ nên ta chọn
		$\vec{OH}=\vec{n}_{(P)}=(2;1;2)$.\\
		Phương trình mặt phẳng $(P)$ có dạng
		$$2\left(x-2\right)+\left(y-1\right)+2\left(z-2\right)=0\Leftrightarrow 2x+y+2z-9=0.$$
		Do đó, góc giữa 2 mặt phẳng $(P),(Q)$ tính như sau
		$$\cos \left((P),(Q)\right)=\dfrac{\left|{\vec{n}_{(P)}\cdot\vec{n}_{(Q)}}\right|}{\left|{\vec{n}_{(P)}}\right|\left|{\vec{n}_Q}\right|}=\dfrac{|2\cdot1+1\cdot1+2\cdot0|}{\sqrt{9}\cdot\sqrt{2}}=\dfrac{3}{3\sqrt{2}}=\dfrac{\sqrt{2}}{2}.$$
		Do đó góc giữa mặt phẳng $(P)$ và mặt phẳng $(Q)$ bằng $\cos 45^\circ=\dfrac{\sqrt{2}}{2}$.
	}
\end{ex}


\begin{ex}
	Cho hai đường thẳng $d_1\colon \dfrac{x}{-1}=\dfrac{y+1}{2}=\dfrac{z}{2}$, $d_2\colon\heva{& x=2t \\ & y=1\\ & z=1-t}$. Gọi $\varphi$ là góc
	giữa hai đường thẳng $d_1$, $d_2$. Tính $\cos\varphi$.
	\choice
	{$\cos \varphi=-\dfrac{4\sqrt{5}}{15}$}
	{\True $\cos \varphi=\dfrac{4\sqrt{5}}{15}$}
	{$\cos \varphi=\dfrac{\sqrt{6}}{9}$}
	{$\cos \varphi=-\dfrac{\sqrt{6}}{9}$}
	\loigiai
	{Đường thẳng $d_1$, $d_2$ lần lượt có vectơ chỉ phương là $\overrightarrow{u}_1=(-1; 2; 2)$ và $\overrightarrow{u}_2=(2; 0; -1)$.\\
		Vậy $\cos\varphi=\left|\cos\left(\overrightarrow{u}_1, \overrightarrow{u}_2\right)\right|=\dfrac{|(-1)\times 2 +2\times 0+ 2\times(-1)|}{\sqrt{(-1)^2+2^2+2^2}\cdot\sqrt{2^2+0^2+(-1)^2}}=\dfrac{|-4|}{3\sqrt{5}}=\dfrac{4\sqrt{5}}{15}$.
	}
\end{ex}

\begin{ex}%[2H3B3-4]%
	Cho đường thẳng $d_1\colon \dfrac{x}{-1}=\dfrac{y+1}{1}=\dfrac{z-1}{-2}$ và $d_2\colon \dfrac{x+1}{-1}=\dfrac{y}{1}=\dfrac{z-3}{1}$. Góc giữa hai đường thẳng bằng
	\choice
	{\True $90^\circ$}
	{$30^\circ$}
	{$60^\circ$}
	{$45^\circ$}
	\loigiai{
		Đường thẳng $d_1$ có VTCP là $\vec{a}=(-1;1;-2)$, đường thẳng $d_2$ có VTCP là $\vec{b}=(-1;1;1)$.\\
		Ta có $\vec{a}\cdot \vec{b}= 0\Rightarrow d_1\perp d_2\Rightarrow \left( d_1, d_2\right) =90^\circ$.
	}
\end{ex}

\begin{ex}
	Cho đường thẳng $d$ là giao tuyến của hai mặt phẳng\\ $(P) \colon x-z \cdot \sin \alpha +\cos \alpha =0$ và $(Q) \colon y-z \cdot \cos \alpha -\sin \alpha =0$, $\alpha \in \left(0;\dfrac{\pi}{2}\right)$. Góc giữa $d$ và trục $Oz$ là
	\choice
	{$90^{\circ}$}
	{$30^{\circ}$}
	{\True $45^{\circ}$}
	{$60^{\circ}$}
	\loigiai{
		Xét hệ phương trình $\left\{\begin{aligned}
			&x-z \cdot \sin \alpha +\cos \alpha =0\\
			&y-z \cdot \cos \alpha -\sin \alpha =0\\
		\end{aligned}\right. $. \\
		Ta có $\vec{n}_P=\left(1;0;-\sin \alpha \right)$ và $\vec{n}_Q=\left(0;1;-\cos \alpha \right)$.\\
		vectơ chỉ phương của $d$ là $\vec{u}_d=\left[\vec{n}_P;\vec{n}_Q \right]=\left(\sin \alpha;\cos \alpha;1\right)$.\\
		vectơ chỉ phương của trục $Oz$ là $\vec{k}=(0;0;1)$.\\
		Gọi $\varphi $ là góc giữa đường thẳng $d$ và trục $Oz$.\\
		Ta có $\cos \varphi =\dfrac{\left|\vec{u}_d \cdot \vec{k}\right|}{\left|\vec{u}_d\right| \cdot \left|\vec{k}\right|}=\dfrac{1}{\sqrt{2}}$. Suy ra $\varphi =45^{\circ}$.}
\end{ex}

\begin{ex}%[2-TT-5- Đề thi tháng 2-2019, Toán 12 trường THPT chuyên Bắc Giang- 2019]%[Nguyễn Thế Anh-EX6]%[2H3B3-4]%
	Cho đường thẳng $d\colon \heva{& x=1-t \\ & y=2+2t \\ & z=3+t}$ và mặt phẳng $(P)\colon x-y+3=0$. Tính số đo góc giữa đường thẳng $d$ và mặt phẳng $(P)$.
	\choice
	{$45^\circ$}
	{$120^\circ$}
	{\True $60^\circ$}
	{$30^\circ$}
	\loigiai
	{
		Ta có $\vec {u}=(-1;2;1)$ là vectơ chỉ phương của đường thẳng $d$ và $\vec{n}=(1;-1;0)$ là vectơ pháp tuyến của mặt phẳng $(P)$.\\
		Suy ra $\sin \left(d;(P)\right)=\left|\cos \left(\vec{u};\vec{n}\right)\right|=\dfrac{\left|\vec{u} \cdot \vec{n}\right|}{|\vec{u}|\cdot |\vec{n}|}=\dfrac{\sqrt{3}}{2}$. Vậy $(d;(P))=60^\circ$.
	}
\end{ex}

\begin{ex}
	Cho mặt phẳng $(P)\colon 3x+4y+5z-8=0$ và đường thẳng $d\colon\heva{&x=2-3t\\&y=-1-4t\\&z=5-5t}$. Góc giữa đường thẳng $d$ và mặt phẳng $(P)$ là
	\choice
	{\True $90^{\circ}$}
	{$45^{\circ}$}
	{$30^{\circ}$}
	{$60^{\circ}$}
	\loigiai{
		Mặt phẳng $(P)$ có một vectơ pháp tuyến là $\overrightarrow{n}=(3;4;5)$.\\
		Đường thẳng $d$ có một vectơ chỉ phương là $\overrightarrow{u}=(-3;-4;-5)$.\\
		Ta có $\overrightarrow{n}=-\overrightarrow{u}\Rightarrow d\perp(P)$ nên góc giữa đường thẳng $d$ và mặt phẳng $(P)$ là $90^{\circ}$.
	}
\end{ex}

\begin{ex}
	Cho mặt phẳng $(P)\colon x+y-\sqrt{2}z+5=0$. Tính góc $\varphi$ giữa mặt phẳng $(P)$ và trục $Oy$.
	\choice
	{$\varphi=60^{\circ}$}
	{$\varphi=45^{\circ}$}
	{$\varphi=90^{\circ}$}
	{\True $\varphi=30^{\circ}$}
	\loigiai{
		Ta có vectơ pháp tuyến của mặt phẳng $(P)$ là $\overrightarrow{n}=(1;1;-\sqrt{2})$ và vectơ chỉ phương của trục $Oy$ là $\overrightarrow{u}=(0;1;0)$. Suy ra $\sin\varphi=\dfrac{|\overrightarrow{u}\cdot \overrightarrow{n}|}{|\overrightarrow{u}|\cdot|\overrightarrow{n}|}=\dfrac{|1|}{\sqrt{4}\cdot 1}=\dfrac{1}{2}\Rightarrow\varphi=30^{\circ}$.
	}
\end{ex}


\begin{ex}
	Cho hai mặt phẳng $(P)\colon (m-1)x+y-2z+m=0$ và $(Q)\colon 2x-z+3=0.$ Tìm $m$ để $(P)$ vuông góc với $(Q).$
	\choice
	{\True $m=0$}
	{$m=\dfrac32$}
	{$m=5$}
	{$m=-1$}
	\loigiai{
		$(P)$ vuông góc với $(Q)$ khi và chỉ khi các vectơ pháp tuyến của chúng vuông góc với nhau, tức là $$(m-1;1;-2)\cdot(2;0;-1)=0\Leftrightarrow m=0.$$
	}
\end{ex}

\begin{ex}
	Cho mặt phẳng $(P) \colon x-3y+2z+1=0$ và $(Q) \colon (2m-1)x+m(1-2m)y+(2m-4)z+14=0$ với $m$ là tham số thực. Tổng các giá trị của $m$ để $(P)$ và $(Q)$ vuông góc nhau bằng
	\choice
	{$-\dfrac{3}{2}$}
	{\True $-\dfrac{1}{2}$}
	{$-\dfrac{5}{2}$}
	{$-\dfrac{7}{2}$}
	\loigiai{
		$(P)$ có vectơ pháp tuyến $\overrightarrow{n}_P =(1;-3;2)$.
		\\ $(Q)$ có vectơ pháp tuyến $\overrightarrow{n}_{Q} =\left( 2m-1;m(1-2m);2m-4 \right)$. \\
		$(P)$ và $(Q)$ vuông góc với nhau khi và chỉ khi $\overrightarrow{n}_P \perp \overrightarrow{n}_{Q}$.
		\\ Điều này tương đương với $\overrightarrow{n}_P \cdot \overrightarrow{n}_Q =0 \Leftrightarrow 6m^2+3m -9=0 \Leftrightarrow \hoac{&m=1 \\ &m=-\dfrac{3}{2}.}$
		\\ Tổng các giá trị của $m$ để $(P)$ và $(Q)$ vuông góc nhau bằng $1 -\dfrac{3}{2} = -\dfrac{1}{2}$.
	}
\end{ex}

\begin{ex}%[Trần Bình Thuận - DA2]%[2H3K2-5]% Câu 7
	Cho hai mặt phẳng $ (P)\colon x+2y-z+2=0 $ và $ (Q) \colon x-my+(m+1)z+m-2=0 $, với $ m $ là tham số. Gọi $ S $ là tập hơn tất cả các giá trị của $ m $ sao cho góc giữa $ (P) $ và $ (Q) $ bằng $ 60^{\circ} $. Tính tổng các phần tử của $ S $.
	\choice
	{$ 1 $}
	{$ -\dfrac{1 }{2} $}
	{$ \dfrac{3 }{2} $}
	{\True $ \dfrac{1 }{2} $}
	\loigiai{
		vectơ pháp tuyến của mặt phẳng $ (P): \vec{n}_{(P)} = (1;2;-1) $.\\
		vectơ pháp tuyến của mặt phẳng $ (Q): \vec{n}_{(Q)} = (1;-m;m+1) $.\\
		Góc giữa hai mặt phẳng $ (P) $ và $ (Q) $ bằng $ 60^{\circ} $ nên
		{\allowdisplaybreaks
			\begin{eqnarray*}
				&&\dfrac{|1-2m-m-1|}{\sqrt{6}\cdot\sqrt{2m^2+2m+2}}= \cos 60^{\circ}\\
				&\Leftrightarrow& |3m| = \dfrac{1}{2}\sqrt{6}\cdot\sqrt{2m^2+2m+2}\\
				&\Leftrightarrow& 9m^2 = 3(m^2+m+1)\\
				&\Leftrightarrow& 2m^2-m-1=0 \Leftrightarrow \hoac{&m=1\\&m=-\dfrac{1}{2}.}
		\end{eqnarray*}}
		Do đó $ S=1+\left( -\dfrac{1}{2} \right) =\dfrac{1}{2} $.
	}
\end{ex}

\begin{ex}
	Hãy tìm tham số thực $m$ để góc giữa hai đường thẳng  sau bằng $60^\circ$.
	$$d\colon \heva{&x=1+t\\&y=-\sqrt{2}t\\&z=1+t}\, ,t\in \mathbb{R} \text {và } d'\colon \heva{&x=1+t'\\&y=1+\sqrt{2}t'\\&z=1+mt'}\, , t'\in \mathbb{R}$$ 
	\choice
	{$\dfrac{1}{2}$}
	{$-1$}
	{\True $-\dfrac{1}{2}$}
	{$1$}
	\loigiai{
		Ta có $\heva{&\vec{u}_d=(1;-\sqrt{2};1)\\&\vec{u}_{d'}=(1;\sqrt{2};m)}\Rightarrow	\cos\alpha = \dfrac{\left|-1+m\right| }{\sqrt{4}\cdot \sqrt{3+m^2}}$
		\begin{eqnarray*}
			\cos\alpha = \cos 60^\circ &\Leftrightarrow& \dfrac{\left|-1+m\right| }{\sqrt{4}\cdot \sqrt{3+m^2}}=\dfrac{1}{2}\\
			&\Leftrightarrow& |-1+m|=\sqrt{3+m^2}\\
			&\Leftrightarrow& -2m+2=0\\
			&\Leftrightarrow& m=1
		\end{eqnarray*}
	}
\end{ex}

\begin{ex}
	Cho các điểm $A(-1;\sqrt{3};0)$, $B(1;\sqrt{3};0)$, $C(0;0;\sqrt{3})$ và điểm $M$ thuộc trục $Oz$ sao cho hai mặt phẳng $(MAB)$ và $(ABC)$ vuông góc với nhau. Tính góc giữa hai mặt phẳng $(MAB)$ và $(OAB)$.
	\choice
	{\True $45^\circ$}
	{$60^\circ$}
	{$15^\circ$}
	{$30^\circ$}
	\loigiai{
		$M(0;0;m)$ thuộc trục $Oz$.\\
		Ta có $\vv{AM}=(1;-\sqrt{3};m)$, $\vv{AB}=(2;0;0)$, $\vv{AC}=(1;-\sqrt{3};\sqrt{3})$.\\
		$\Rightarrow \vv{n}_1 =\left[ \vv{AB}, \vv{AC} \right] =(0;-2\sqrt{3};-2\sqrt{3})$,
		$\vv{n}_2 =\left[ \vv{AB}, \vv{AM} \right] =(0;-2m;-2\sqrt{3})$.\\
		Mặt phẳng $(ABC)$ có một vectơ pháp tuyến là $\vv{n}_1$, mặt phẳng $(MAB)$ có một vectơ pháp tuyến là $\vv{n}_2$.\\
		Hai mặt phẳng $(MAB)$ và $(ABC)$ vuông góc với nhau khi và chỉ khi
		$$\vv{n}_1 \perp \vv{n}_2
		\Leftrightarrow 0\cdot 0  +(-2\sqrt{3})\cdot (-2m) + (-2\sqrt{3})\cdot (-2\sqrt{3}) =0
		\Leftrightarrow m=-\sqrt{3}.$$
		Mặt phẳng $(OAB)$ có một vectơ pháp tuyến là $\vv{n}_3 = \left[ \vv{OA}, \vv{OB} \right] =(0;0;-2\sqrt{3})$.\\
		Gọi $\varphi$ là góc giữa hai mặt phẳng $(MAB)$ và $(OAB)$. Khi đó
		$$\cos \varphi =\left| \cos \left( \vv{n}_2, \vv{n}_3 \right) \right|
		= \dfrac{\left| \vv{n}_2 \cdot \vv{n}_3 \right|}{\left| \vv{n}_2 \right| \cdot \left| \vv{n}_3 \right|}
		=\dfrac{12}{2\sqrt{6} \cdot 2\sqrt{3}} =\dfrac{1}{\sqrt{2}}.$$
		Vậy góc giữa hai mặt phẳng $(MAB)$ và $(OAB)$ là $45^\circ$.
	}
\end{ex}



\Closesolutionfile{ans}
\begin{dang}{Tọa độ hóa một số bài toán hình không gian}
\end{dang}

\boxmini{BÀI TẬP TỰ LUẬN}
\setcounter{vd}{0}

\begin{vd}%[2H5V1-6]
	Cho hình lăng trụ đứng $OBC.O'B'C'$ có đáy là tam giác $OBC$ vuông tại $O$ và có $OB=3a$, $OC=a$, $OO'=2a$. Tính góc giữa
	\begin{enumerate}
		\item hai đường thẳng $BO'$ và $B'C$;
		\item hai mặt phẳng $(O'BC)$ và $(OBC)$;
		\item đường thẳng $B'C$ và mặt phẳng $(O'BC)$.
	\end{enumerate}
	\loigiai
	{
		\immini{
			Xét hệ trục tọa độ $Oxyz$ sao cho các điểm có tọa độ như sau: $O(0;0;0)$, $O'(2a;0;0)$, $B(0;3a;0)$, $C(0;0;1a)$.\\
			Trong không gian $Oxyz$ vừa chọn, ta có $B'(2a;3a;0)$, $C'(2a;0;1a)$, $\overrightarrow{BO}'=(0;-3a;0)$, $\overrightarrow{CB}'=(2a;3a;-a)$.
			\begin{enumerate}
				\item Hai đường thẳng $BO'$ và $B'C$ có vectơ chỉ phương lần lượt là $\overrightarrow{u}=(0;3;0)$, $\overrightarrow{v}=(2;3;-1)$.\\
				Ta có 
				\begin{eqnarray*}
					\cos (BO',B'C)&=&\dfrac{\left |\overrightarrow{u}\cdot\overrightarrow{v} \right |}{\left |\overrightarrow{u} \right |\cdot\left |\overrightarrow{v} \right |}\\&=&\dfrac{\left |0\cdot 2+3\cdot 3+0\cdot (-1) \right |}{\sqrt{3^2}\cdot\sqrt{2^2+3^2+(-1)^2}}=\dfrac{3}{\sqrt{14}}.
				\end{eqnarray*}
				Suy ra $(BO',B'C)\approx 36^{\circ}42'$.
				\item Ta có phương trình mặt phẳng $(O'BC)$ theo đoạn chắn là $\dfrac{x}{2a}+\dfrac{y}{3a}+\dfrac{z}{a}=1$ hay $3x+2y+6z-6a=0$.
			\end{enumerate}
		}{
			\begin{tikzpicture}[line cap=round,line join=round,scale=.8,>=stealth,font=\footnotesize ]
				\coordinate (O) at (0,0);
				\coordinate (O') at (0,4.5);
				\coordinate (B) at (5.5,0);
				\coordinate (C) at (2,-2);
				\coordinate (C') at ($(C)+(0,4.5)$);
				\coordinate (B') at ($(B)+(0,4.5)$);
				\coordinate (y) at ($(O)!1.3!(B)$);
				\coordinate (x) at ($(O)!1.3!(O')$);%trung điểm
				\coordinate (z) at ($(O)!1.5!(C)$);%trung điểm
				\draw (B')--(C)--(O')--(B')--(C')--(C)--(B)--(B')(B)--(C')--(O')--(O)--(C)(O)--(C');
				\draw[dashed](O)--(B)--(O');
				\foreach \x/\g in {O'/180,B'/75,C'/180,O/180,B/-80,C/-100}
				\fill[black](\x) circle (1pt)
				($(\x)+(\g:3mm)$) node{\x};
				\draw[->](B)--(y)node[below]{$y$};
				\draw[->](O')--(x)node[above]{$x$};
				\draw[->](C)--(z)node[below right]{$z$};
			\end{tikzpicture}
		}
		Mặt phẳng $(O'BC)$ có vectơ pháp tuyến $\overrightarrow{n}=(3;2;6)$, mặt đáy $(OBC)$ có vectơ pháp tuyến $\overrightarrow{k}=(0;0;1)$. Gọi $\alpha$ là góc giữa mặt phẳng $(O'BC)$ và mặt đáy.\\
		Ta có $\cos\alpha =\dfrac{\left |\overrightarrow{n}\cdot\overrightarrow{k} \right |}{\left |\overrightarrow{n} \right |\cdot\left |\overrightarrow{k} \right |}=\dfrac{\left |3\cdot 0+2\cdot 0+6\cdot 1 \right |}{\sqrt{3^2+2^2+6^2}\cdot\sqrt{1^2}}=\dfrac{6}{7}$.\\
		Suy ra $\left ((O'BC),(OBC) \right )\approx 31^{\circ}1'$.
		\begin{enumerate}
			\setcounter{enumi}{2}
			\item Gọi $\beta$ là góc giữa đường thẳng $B'C$ và mặt phẳng $(O'BC)$.\\
			Ta có $\sin\beta =\dfrac{\left |\overrightarrow{v}\cdot\overrightarrow{n} \right |}{\left |\overrightarrow{v} \right |\cdot\left |\overrightarrow{n} \right |}=\dfrac{\left |2\cdot 3+3\cdot 2+(-1)\cdot 6 \right |}{\sqrt{2^2+3^2+(-1)^2}\cdot\sqrt{3^2+2^2+6^2}}=\dfrac{3\sqrt{14}}{49}$.\\
			Suy ra $(B'C,(O'BC))\approx 13^{\circ}15'$.
		\end{enumerate}
	}
\end{vd}
\dongcham{31}
\begin{vd}%[2H5H2-7]
	Cho hình chóp $S.ABCD$ có đáy $ABCD$ là hình vuông cạnh bằng $4$. Mặt bên $SAB$ là tam giác cân tại $S$ có chiều cao bằng $6$ và nằm trong mặt phẳng vuông góc với đáy.
	\begin{listEX}  
		\item Tính góc $\alpha$ giữa hai đường thẳng $SD$ và $BC$;
		\item Tính góc $\beta$ giữa hai mặt phẳng $(SAD)$ và $(SCD)$.
	\end{listEX}
	\loigiai{
		\immini{Gọi $O$ là trung điểm của $AB$ suy ra $SO\perp(ABCD)$.\\
			Chọn hệ trục $Oxyz$ như hình bên. Ta có: $S(0;0;6)$, $A(2; 0; 0)$, $B(-2; 0; 0)$, $C(-2; 4; 0)$, $D(2; 4; 0)$.}
		{\begin{tikzpicture}[scale=.7,>=stealth, font=\footnotesize, line join=round, line cap=round]
				\coordinate (O) at (0, 0);
				\coordinate (I) at (4, 0);
				\coordinate (S) at (0,6);
				\coordinate (A) at (-1, -1);
				\coordinate (B) at (1, 1);
				\coordinate (C) at (5, 1);
				\coordinate (D) at ($(A)+(C)-(B)$);
				%\draw[color=gray!50,dashed] (\xmin,\ymin) grid (\xmax,\ymax);
				\draw[dashed](A)node[below]{$A$}--(B)node[above right]{$B$}--(C)node[right]{$C$}(O)--(S)node[right]{$S$}--(B)(O)node[left]{$O$}--(I)node[below ]{$I$};
				\draw(S)node[left]{$6$}--(A)node[above left]{$2$}--(D)--(C)--(S)--(D)node[below]{$D$};
				\draw[->](S)--++(0,1)node[left]{$z$};
				\draw[->](I)node[above]{$4$}--++(1,0)node[above]{$y$};
				\draw[->](A)--++(-1,-1)node[left]{$x$};
		\end{tikzpicture}}
		\begin{listEX}
			\item Ta có $\overrightarrow{SD}=(2;4;-6)$, $\overrightarrow{BC}=(0; 4; 0)$, suy ra
			\begin{eqnarray*}
				&\cos \alpha & =\dfrac{|\overrightarrow{S D} \cdot \overrightarrow{B C}|}{|\overrightarrow{S D}| \cdot|\overrightarrow{B C}|}=\dfrac{|2 \cdot 0+4 \cdot 4-6 \cdot 0|}{\sqrt{2^2+4^2+(-6)^2} \cdot \sqrt{4^2}} \\
				&& =\dfrac{\sqrt{14}}{7} \Rightarrow \alpha=57{,}7^{\circ}.
			\end{eqnarray*}
			\item Mặt phẳng $(S A D)$ có cặp vectơ chỉ phương là $\overrightarrow{S D}=(2; 4;-6)$,\\        
			$\overrightarrow{S A}=(2; 0;-6)$ nên có vectơ pháp tuyến $\overrightarrow{n}=-\dfrac{1}{8}[\overrightarrow{S D}, \overrightarrow{SA}]=(3; 0; 1)$.\\     
			Mặt phẳng $(SCD)$ có cặp vectơ chỉ phương là $\overrightarrow{SD}=(2; 4;-6)$,   $\overrightarrow{D C}=(-4; 0; 0)$ nên có vectơ pháp tuyến $\overrightarrow{u}=\dfrac{1}{8}[\overrightarrow{S D}, \overrightarrow{D C}]=(0; 3; 2)$.\\
			Suy ra $\cos \beta=\dfrac{\left|\overrightarrow{n} \cdot \overrightarrow{u}\right|}{|\overrightarrow{n}| \cdot\left|\overrightarrow{u}\right|}=\dfrac{|3 \cdot 0+0\cdot 3+1 \cdot 2|}{\sqrt{3^2+1^2} \cdot \sqrt{3^2+2^2}}=\dfrac{2}{\sqrt{130}} \Rightarrow \beta \approx 79{,}9^{\circ}$.
		\end{listEX}
	}
\end{vd}
\dongcham{15}

\begin{vd}%[2H5V2-8]
	\immini{Người ta muốn dựng một cột ăng-ten trên một sườn đồi. Ăng-ten được dựng thẳng đứng trong không gian $Oxyz$ với độ dài đơn vị trên mỗi trục bằng $1$ m. Gọi $O$ là gốc cột, $A$ là điểm buộc dây cáp vào cột ăng-ten và $M$, $N$ là hai điểm neo dây cáp xuống mặt sườn đồi (hình vẽ). Cho biết toạ độ các điểm nói trên lần lượt là $O(0;0;0)$, $A(0;0;6)$, $M(3;-4;3)$, $N(-5;-2;2)$.
	}{\begin{tikzpicture}[scale=.4,>=stealth, font=\footnotesize, line join=round, line cap=round]
			\filldraw[green!40!brown!50] 
			(-5,0) .. controls (-3,3) and (-1.5,2.5) .. (0,2) .. controls (1.5,2.5) and (3,2) .. (5,0) -- cycle;
			\filldraw[green!70!black] 
			(-6,0) .. controls (-4,2.5) and (-2,2) .. (0,0) -- cycle;
			\filldraw[green!80!black] 
			(-4,0) .. controls (0,4) and (3,3.5) .. (6,0) -- cycle;
			\draw[line width=.5, double](5,0.6)node[right]{$O$}circle(2pt)--(5,4.2);
			\draw[line width=1,red](4,5)--(6,3) (3.5,4.8)--(4.5,5.2) (4.5,3.8)--(5.5,4.2) (5.5,2.8)--(6.5,3.2);
			\draw(0,1)node[below]{$N$}circle(2pt)--(5,2.5)node[right]{$A$}circle(2pt)--(1,2)node[above]{$M$}circle(2pt);
			% Vẽ các ngọn đồi đậm hơn phía trước cùng
			\filldraw[green!90!black] 
			(-5,0) .. controls (-3.5,1.5) and (-1.5,1.5) .. (0,0) -- cycle;
			
	\end{tikzpicture}}
	\begin{listEX}
		\item Tính độ dài các đoạn dây cáp $MA$ và $NA$.
		\item Tính góc tạo bởi các sợi dây cáp $MA$, $NA$ với mặt phẳng sườn đồi.
	\end{listEX} 
	\loigiai{
		\begin{listEX}
			\item Ta có $\overrightarrow{MA}=(-3; 4; 3), \overrightarrow{NA}=(5; 2; 4)$, suy ra
			\begin{itemize}
				\item $MA=\sqrt{(-3)^2+4^2+3^2}=\sqrt{34} \approx 5{,}8$ m. 
				\item $NA=\sqrt{5^2+2^2+4^2}=\sqrt{45} \approx 6{,}7$ m.
			\end{itemize}
			\item  Mặt phẳng $(OMN)$ có cặp vectơ chỉ phương là $\overrightarrow{OM}=(3;-4; 3)$,  $\overrightarrow{ON}=(-5;-2; 2)$ nên có vectơ pháp tuyến $\overrightarrow{n}=[\overrightarrow{O M}, \overrightarrow{O N}]=(-2;-21;-26)$.\\        
			Gọi $\alpha$, $\beta$ lần lượt là góc tạo bởi $MA$, $NA$ với mặt phẳng $(AMN)$.\\
			Ta có
			\begin{eqnarray*}
				&\sin \alpha=\dfrac{|\overrightarrow{M A} \cdot \overrightarrow{n}|}{|\overrightarrow{M A}| \cdot|\overrightarrow{n}|} & =\dfrac{|-3 \cdot(-2)+4 \cdot(-21)+3 \cdot(-26)|}{\sqrt{(-3)^2+4^2+3^2} \cdot \sqrt{(-2)^2+(-21)^2+(-26)^2}} \\
				&&=\dfrac{156}{\sqrt{38\,114}} \Rightarrow \alpha \approx 53^{\circ}.
			\end{eqnarray*}
			Và  
			\begin{eqnarray*}
				&\sin \beta=\dfrac{|\overrightarrow{NA} \cdot \overrightarrow{n}|}{|\overrightarrow{N A}| \cdot|\overrightarrow{n}|} & =\frac{|5 \cdot(-2)+2 \cdot(-21)+4 \cdot(-26)|}{\sqrt{5^2+2^2+4^2} \cdot \sqrt{(-2)^2+(-21)^2+(-26)^2}} \\
				&&=\dfrac{156}{\sqrt{50\,445}} \Rightarrow \beta \approx 44^{\circ}. 
			\end{eqnarray*}
		\end{listEX}    
	}
\end{vd}
\dongcham{13}

% \begin{vd}
% 	Một khuôn nướng bánh mì được mô phỏng trong không gian $Oxyz$ như hình vẽ với  $S(0; 0; 0)$, $P(8; 0; 0)$, $Q(8; 18; 0)$, $T(-1;-1; 7)$, $R(9; 19; 7)$. Tính góc giữa hai cạnh kề nhau, giữa cạnh bên và mặt đáy, giữa mặt bên và mặt đáy của khuôn.
% 	\begin{center}
% 		% \includegraphics[width=6cm]{images/KhuonBanh.jpg}~
% 		\begin{tikzpicture}[>=stealth,scale=.8]
% 			\path (0,0) coordinate (S)
% 			(-20:0.5) coordinate (ex)
% 			(15:0.5) coordinate (ey)
% 			(0,0.65) coordinate (ez)
% 			($4*(ex)$) coordinate (P)
% 			($4*(ex)+9*(ey)$) coordinate (Q)
% 			($9*(ey)$) coordinate (H)
% 			($-0.5*(ex)-0.5*(ey)+3.5*(ez)$) coordinate (T)
% 			($4.5*(ex)+9.5*(ey)+3.5*(ez)$) coordinate (R)
% 			($4.5*(ex)-.5*(ey)+3.5*(ez)$) coordinate (E)
% 			($-.5*(ex)+9.5*(ey)+3.5*(ez)$) coordinate (K)
% 			;
			
% 			\foreach \x in {-3,...,12}{
% 				\draw[dashed,gray!65] ($-3*(ex)+\x*(ey)$)--($9*(ex)+\x*(ey)$);
% 				\ifnum\x>0
% 				\pgfmathsetmacro{\xt}{int(2*\x)}
% 				\draw ($\x*(ey)+(0,-1pt)$)--($\x*(ey)+(0,1pt)$) node[above,font=\tiny]{$\xt$};
% 				\ifnum \x<9
% 				\draw ($\x*(ex)+(0,-1pt)$)--($\x*(ex)+(0,1pt)$) node[below,font=\tiny]{$\xt$};
% 				\draw[dashed,gray!65] ($-3*(ey)+\x*(ex)$)--($12*(ey)+\x*(ex)$);
% 				\fi
% 				\else
% 				\pgfmathsetmacro{\xt}{int(2*\x)}
% 				\draw ($\x*(ey)+(0,-1pt)$)--($\x*(ey)+(0,1pt)$)
% 				+(-0.1,0) node[below,font=\tiny]{$\xt$};
% 				\draw ($\x*(ex)+(0,-1pt)$)--($\x*(ex)+(0,1pt)$) +(-0.1,0) node[below,font=\tiny]{$\xt$};
% 				\draw[dashed,gray!65] ($-3*(ey)+\x*(ex)$)--($12*(ey)+\x*(ex)$);
% 				\fi
% 			}
% 			\foreach \x in {-1,1,2,...,5}{
% 				\pgfmathsetmacro{\xt}{int(2*\x)}
% 				\draw ($\x*(ez)+(1pt,0)$)--($\x*(ez)+(-1pt,0)$) +(0.5pt,0) node[anchor=east,font=\tiny]{$\xt$};
% 			}
% 			\draw[->,cyan!65] ($-3*(ex)$)--($9*(ex)$);
% 			\draw[->,cyan!65] ($-3*(ey)$)--($12.5*(ey)$);
% 			\draw[->,cyan!65] ($-1.5*(ez)$)--($5.5*(ez)$);
% 			\draw[dashed](S)--(H)--(Q) (H)--(K);
% 			\draw (S)--(P)--(E)--(T)--cycle
% 			(P)--(Q)--(R)--(E)
% 			(R)--(K)--(T);
% 			\foreach \t/\g in {S/-65,P/45,Q/-60,H/-90,R/20,K/75,T/110,E/90}{
% 				\shade[ball color=blue] (\t) circle (1pt) node[shift={(\g:7pt)},font=\scriptsize]{$ \t $};
% 			}
% 		\end{tikzpicture}
% 	\end{center}
% 	\loigiai{
% 		\begin{listEX}[1]
% 			\item Tính góc giữa hai cạnh kề nhau\\
% 			$\overrightarrow{SP}=\left(8;0;0\right);\overrightarrow{SH}=\left(0;18;0\right);\overrightarrow{ST}=\left(-1;-1;7\right)$.\\
% 			$\overrightarrow{SP}\cdot\overrightarrow{SH}=0\Rightarrow \left(SP,SH\right)=90^\circ$.\\
% 			$\cos\left(SP,ST\right)=\dfrac{\left|\overrightarrow{SP}\cdot\overrightarrow{ST}\right|}{\left|\overrightarrow{SP}\right|\cdot\left|\overrightarrow{ST}\right|}=\dfrac{1}{\sqrt{51}}\Rightarrow \left(SP,ST\right)\approx 82^\circ$.\\
% 			$\cos\left(SH,ST\right)=\dfrac{\left|\overrightarrow{SH}\cdot\overrightarrow{ST}\right|}
% 			{\left|\overrightarrow{SH}\right|\cdot\left|\overrightarrow{ST}\right|}
% 			=\dfrac{1}{\sqrt{51}}\Rightarrow \left(SH,ST\right)\approx 82^\circ$.
% 			\item Tính góc giữa cạnh bên và mặt đáy\\
% 			Gọi $\alpha$ là góc giữa cạnh bên $ST$ và mặt phẳng đáy.\\
% 			$\sin\alpha=\dfrac{\left|\overrightarrow{ST}\cdot\vec{k}\right|}{\left|\overrightarrow{ST}\right|\cdot\left|\vec{k}\right|}=\dfrac{7}{\sqrt{51}}\Rightarrow \alpha\approx 78^\circ$.
% 			\item Tính góc giữa mặt bên và mặt đáy\\
% 			Gọi $\beta$ là góc giữa mặt bên và mặt phẳng đáy.\\
% 			$\vec{n}=\left[\overrightarrow{ST},\overrightarrow{SP}\right]=\left(0;56;8\right)$.\\
% 			$\cos\beta=\dfrac{\left|\overrightarrow{n}\cdot\vec{k}\right|}{\left|\overrightarrow{n}\right|\cdot\left|\vec{k}\right|}=\dfrac{\sqrt{2}}{10}\Rightarrow \beta\approx 82^\circ$.
% 		\end{listEX}
% 	}
% \end{vd}
% \dongcham{28}
\boxmini{BÀI TẬP TRẮC NGHIỆM}
\setcounter{ex}{0}
\Opensolutionfile{ans}[ans/2H5-B3-d2]

\begin{ex}
		Trong hệ trục toạ độ $Oxyz$, với mặt phẳng $(Ox y)$ là mặt đất, một máy bay cất cánh từ vị trí $A(0; 10; 0)$ với vận tốc $\vec{v}=(150; 150; 40)$. Tính góc nâng của máy bay (góc giữa hướng chuyển động bay lên của máy bay với đường băng và làm tròn kết quả đến hàng đơn vị).
		\choice
		{$10^{\circ}$}
		{$12^{\circ}$}
		{\True $11^{\circ}$}
		{$9^{\circ}$}
	\loigiai{Gọi $\alpha$ là góc nâng của máy bay.\\
		$\sin\alpha=\dfrac{\left|\overrightarrow{v}\cdot\vec{k}\right|}{\left|\overrightarrow{v}\right|\cdot\left|\vec{k}\right|}\Rightarrow \alpha\approx 10^\circ 40'$.
	}
\end{ex}


\begin{ex}
	Cho hình lập phương $ABCD.A'B'C'D'$có cạnh bằng $a$. Tính số đo góc giữa hai mặt phẳng $(BA'C)$ và $(DA'C)$.
	\choice
	{$30^\circ$}
	{$120^\circ$}
	{$90^\circ$}
	{\True $60^\circ$}
	\loigiai{
		\immini
		{ Chọn hệ tọa độ $Oxyz$ có $A\equiv O$, $ \overrightarrow{AB}$, $\overrightarrow{AD}$, $\overrightarrow{AA'}$ lần lượt cùng hướng với các vectơ đơn vị $\overrightarrow{i}$, $\overrightarrow{j}$, $\overrightarrow{k}$.\\
			Lấy $a=1$, suy ra $B(1;0;0)$, $D(0;1;0)$, $A'(0;0;1)$, $C(1;1;0)$.\\
			Mặt phẳng $(BA'C)$ có vectơ pháp tuyến là \\ $\overrightarrow{n_1}=\overrightarrow{BA'}\wedge \overrightarrow{BC}=(-1;0;-1)$.\\
			Mặt phẳng $(DA'C)$ có vectơ pháp tuyến là \\ $\overrightarrow{n_2}=\overrightarrow{DA'}\wedge \overrightarrow{DC}=(0;1;1)$.\\
			Gọi $\varphi $ là góc giữa hai mặt phẳng $(BA'C)$ và $(DA'C)$, ta có
			$ \cos\varphi =\left|{\cos\left(\overrightarrow{n_1},\overrightarrow{n_2}\right)}\right|=\dfrac{\left|{\overrightarrow{n_1} \cdot \overrightarrow{n_2}}\right|}{\left|{\overrightarrow{n_1}}\right|\cdot \left|{\overrightarrow{n_2}}\right|}=\dfrac{1}{\sqrt{2}\cdot \sqrt{2}}=\dfrac{1}{2}$.\\ Suy ra $\varphi =60^\circ$.
		}
		{ \begin{tikzpicture}[scale=1,>=stealth, font=\footnotesize, line join=round, line cap=round]
				\tkzDefPoints{0/0/A,-1.3/-1.1/B,2/-1.1/C}
				\coordinate (D) at ($(A)+(C)-(B)$);
				\coordinate (A') at ($(A)+(0,3.3)$);
				\coordinate (x) at ($(A)!1.6!(B)$);
				\coordinate (y) at ($(A)!1.2!(D)$);
				\coordinate (z) at ($(A)!1.3!(A')$);
				\tkzDrawSegments[vector style](B,x D,y A',z)
				\tkzDefPointsBy[translation=from A to A'](B,C,D){B'}{C'}{D'}
				\tkzDrawPolygon(A',B',B,C,D,D')
				\tkzDrawSegments(B',C' C',D' C,C')
				\tkzDrawSegments[dashed](A,B A,D A,A' A',C A,C B,D A',B A',D)
				\tkzDrawPoints[fill=black,size=4](A,B,D,C,A',B',C',D')
				\tkzLabelPoints[below](B,C,x)
				\tkzLabelPoints[left](B')
				\tkzLabelPoints[right](C',z)
				\tkzLabelPoints[above right](A,D,A',D',y)
			\end{tikzpicture}
		}
	}
\end{ex}

\begin{ex}
	Cho hình lập phương $MNPQ.M'N'P'Q'$ có $E$, $F$, $G$ lần lượt là trung điểm của $NN'$, $PQ$, $M'Q'$ Tính góc giữa hai đường thẳng $EG$ và $P'F$.
	\choice
	{$60^{\circ} $}
	{\True $90^{\circ} $}
	{$30^{\circ} $}
	{$45^{\circ} $}
	\loigiai{
		\immini{Chọn hệ trục tọa độ $Oxyz$ sao cho $M(0;0;0)$, $N(1;0;0)$, $Q(0;1;0)$ và $M'(0;0;1)$. Lúc đó $P(1;1;0), N'(1;0;1), Q'(0;1;1)$ và $P'(1;1;1)$.\\
			Vì $E, F, G$ lần lượt là trung điểm $NN', PQ$ và $M'Q'$ nên $E\left(1;0;\dfrac{1}{2}\right)$, $F\left(\dfrac{1}{2};1;0\right)$ và $G\left(0;\dfrac{1}{2};1\right)$.\\
			Suy ra $\vec{EG}=\left(-1;\dfrac{1}{2};\dfrac{1}{2}\right)$, $\vec{P'F}=\left(-\dfrac{1}{2};0;-1\right)$, do đó $\vec{EG}\cdot \vec{P'F}=0$ hay $(EG,P'F)=90^{\circ}$.}{\begin{tikzpicture}[line cap=round,line join=round,scale=1.5]%[Mai Hà Lan] %1.20
				%-------------- Đáy ABCD
				\tkzDefPoints{0/0/A, -0.5/-1/B, 2/0/D}
				\coordinate (C) at ($(B)+(D)-(A)$);
				%-------------- Đáy A'B'C'D'
				\tkzDefPointBy[rotation = center A angle 90](D) \tkzGetPoint{A'} %Phép quay tâm A, góc quay 90 độ, biến D thành A'
				\coordinate (B') at ($(B)+(A')-(A)$);
				\coordinate (D') at ($(D)+(A')-(A)$);
				\coordinate (C') at ($(B')+(D')-(A')$);
				%---------------
				\tkzDrawSegments[dashed](A,B A,D A,A')
				\tkzDrawPolygon(A',B',C',D')
				\tkzDrawPolygon(B,C,C',B')
				\tkzDrawSegments(D,D' C,D)
				\tkzDrawPoints(A,B,C,D,A',B',C',D')
				\tkzLabelPoint(A){$M$}
				\tkzLabelPoint(B){$N$}
				\tkzLabelPoint(C){$P$}
				\tkzLabelPoint(D){$Q$}
				\tkzLabelPoint(A'){$M'$}
				\tkzLabelPoint(B'){$N'$}
				\tkzLabelPoint(C'){$P'$}
				\tkzLabelPoint(D'){$Q'$}
				%\tkzLabelPoints(A,B,C,D,A',B',C',D')
				\tkzDefMidPoint(B,B') \tkzGetPoint{E}
				\tkzDefMidPoint(C,D) \tkzGetPoint{F}
				\tkzDefMidPoint(A',D') \tkzGetPoint{G}
				\tkzLabelPoints[above](G)
				\tkzLabelPoints[right](E,F)
				\tkzDrawSegments[dashed](E,G)
				\tkzDrawSegments(C',F)
				\tkzDrawPoints(G,E)
		\end{tikzpicture}}
	}
\end{ex}

\begin{ex}
	Cho hình hộp chữ nhật $ABCD.A'B'C'D'$ có các cạnh $AB=2,AD=3,AA'=4$. Góc giữa hai mặt phẳng $(AB'D')$ và $(A'C'D)$ là $\alpha$. Tính giá trị gần đúng của góc $\alpha$.
	\choice
	{$45,2^{\circ}$}
	{$38,1^{\circ}$}
	{\True $61,6^{\circ}$}
	{$53,4^{\circ}$}
	\loigiai{
		\immini{Gắn hình hộp chữ nhật $ABCD.A'B'C'D'$ vào hệ trục tọa độ $Oxyz$. Khi đó $A(0,0,0)$, $B(0;2;0)$, $C(3;2;0)$, $D(3;0;0)$, $A'(0;0;4)$, $B'(0;2;4)$, $C'(3;2;4)$, $D'(3;0;4)$.\\
			$\vec{AB'}=(0;2;4)$, $\vec{AD'}=(3;0;4)$, $\vec{A'C'}=(3;2;0),\vec{A'D}=(3;0;-4)$.\\
			Gọi $\overrightarrow{n}_{1}$ là vectơ pháp tuyến của $(AB'D')$. Ta có $\overrightarrow{n}_{1}=[\vec{AB'},\vec{AD'}]=(8;12;-6)$.\\
			Gọi $\overrightarrow{n}_{2}$ là vectơ pháp tuyến của $(A'C'D)$. Ta có $\overrightarrow{n}_{2}=[\vec{A'C'},\vec{A'D}]=(-8;12;-6)$.\\
			$\alpha$ là góc giữa hai mặt phẳng $(AB'D')$ và $(A'C'D)$, ta có \\
			$\cos \alpha=\left|\dfrac{\overrightarrow{n}_{1}\cdot \overrightarrow{n}_{2}}{|\overrightarrow{n}_{1}||\overrightarrow{n}_{2}|} \right|=\dfrac{29}{61}$.\\
			Vậy giá trị gần đúng của góc $\alpha$ là $61,6^{\circ}$.
		}{
	\begin{tikzpicture}
		\tkzDefPoints{0/0/A,-3/-2/D,2/-2/C,5/0/B,0/5/A',-3/3/D',2/3/C',5/5/B'}
		\tkzDrawPoints(A,B,C,D,A',B',C',D')
		\tkzLabelPoints[above right](A')
		\tkzLabelPoints[above left](B')
		\tkzLabelPoints[above](C')
		\tkzLabelPoints[above left](B,D',D)
		\tkzLabelPoints[below right](A,C)
		\tkzDrawSegments(B,C C,D D,D' C,C' B,B' A',B' B',C' C',D' D',A' B',D' A',C' C',D)
		\tkzDrawSegment[style=dashed](A,B)
		\tkzDrawSegment[style=dashed](A,D)
		\tkzDrawSegment[style=dashed](A,A')
		\tkzDrawSegment[style=dashed](A,B')
		\tkzDrawSegment[style=dashed](A,D)
		\tkzDrawSegment[style=dashed](A,D')
		\tkzDrawSegment[style=dashed](A',D)
\end{tikzpicture}}
	}
\end{ex}

\begin{ex}
		Cho hình chóp $SABCD$ có $ABCD$ là hình vuông cạnh $a$, $SA$ vuông góc $\left(ABCD\right)$, $SA=a$. Gọi $E$ và $F$ lần lượt là trung điểm $SB,SD$. Cô-sin của góc hợp bởi hai mặt phẳng $\left(AEF\right)$ và $\left(ABCD\right)$ là
		\choice
		{$\sqrt{3}$}
		{$\dfrac{\sqrt{3}}{2}$}
		{$\dfrac{1}{2}$}
		{\True $\dfrac{\sqrt{3}}{3}$}
	\loigiai{
		\immini{
			Chọn hệ trục tọa độ như hình vẽ. Ta có:
			$A\left(0;0;0\right)$, $B\left(a;0;0\right)$, $D\left(0;a;0\right)$, $S\left(0;0;a\right)$, $E\left(\dfrac{a}{2};0;\dfrac{a}{2}\right)$, $F\left(0;\dfrac{a}{2};\dfrac{a}{2}\right)$ và
			$\overrightarrow{AE}\left(\dfrac{a}{2};0;\dfrac{a}{2}\right)$, $\overrightarrow{AF}\left(0;\dfrac{a}{2};\dfrac{a}{2}\right)$.\\
			$\Rightarrow \left[\overrightarrow{AE},\overrightarrow{AF}\right]=\left(-\dfrac{a^2}{4};-\dfrac{a^2}{4};\dfrac{a^2}{4}\right)$.\\
			$\Rightarrow $ Một VTPT của mặt phẳng $\left(AEF\right)$ là $\left(1;1;-1\right)$.
			Phương trình mặt phẳng $\left(AEF\right)\colon $ $x+y-z=0$.\\
			Phương trình mặt phẳng $\left(ABCD\right)\colon z=0$.
		}{
			\begin{tikzpicture}[scale=0.8, font=\footnotesize, line join=round, line cap=round, >=stealth]
				\def \xa{-2}
				\def \xb{-1}
				\def \y{4}
				\def \z{3}
				\coordinate (A) at (0,0);
				\coordinate (B) at ($(A)+(\xa,\xb)$);
				\coordinate (Bx) at ($(B)+(\xa,\xb)$);
				\coordinate (D) at ($(A)+(\y,0)$);
				\coordinate (Dy) at ($(D)+(2,0)$);
				\coordinate (C) at ($ (B)+(D)-(A) $);
				\coordinate (S) at ($ (A)+(0,\z) $);
				\coordinate (Sz) at ($ (S)+(0,1.5) $);
				\tkzDefMidPoint(S,B)    \tkzGetPoint{E}
				\tkzDefMidPoint(S,D)    \tkzGetPoint{F}
				\draw [dashed] (B)--(A)--(D) (A)--(S) (A)--(E)--(F)--(A);
				\draw (S)--(B)--(C)--(D)--(S)--(C);
				\draw[->] (B)--(Bx) node [below right] {$x$};
				\draw[->] (D)--(Dy) node [below] {$y$};
				\draw[->] (S)--(Sz) node [right] {$z$};
				\tkzDrawPoints(S,A,B,C,D,E,F)
				\tkzLabelPoints[above right](D)
				\tkzLabelPoints[below right](C)
				\tkzLabelPoints[above left](S)
				\tkzLabelPoints[left](A)
				\tkzLabelPoints[below](B)
				\tkzLabelPoints[above right](F)
				\tkzLabelPoints[above left](E)
		\end{tikzpicture}}
		\noindent Góc hợp bởi hai mặt phẳng $\left(AEF\right)$ và $\left(ABCD\right)$ là $\alpha $, ta có $\cos \alpha =\left| \dfrac{-1}{\sqrt{3}\cdot  1}\right|=\dfrac{1}{\sqrt{3}}.$
	}
\end{ex}
\begin{ex}
	Cho hình chóp tam giác $O.ABC$ có $OA$, $OB$, $OC$ đôi một vuông góc và $OA=OB=OC$. Lấy $M$, $N$ lần lượt là trung điểm của $AB$, $OC$. Gọi $\alpha$ là góc tạo bởi $OA$ và $MN$. Tính $\cos\alpha$.
	\choice
	{\True $\dfrac{\sqrt{3}}{3}$}
	{$\dfrac{1}{3}$}
	{$\dfrac{\sqrt{3}}{4}$}
	{$\dfrac{\sqrt{3}}{2}$}
	\loigiai{
		\immini{
			Do $OA$, $OB$, $OC$ đôi một vuông góc nên chọn hệ trục toạ độ như hình vẽ. Khi đó, ta có hệ toạ độ của các điểm $O(0;0;0)$, $A(a;0;0)$, $B(0;a;0)$, $C(0;0;a)$.\\
			Suy ra $M\left(\dfrac{a}{2};\dfrac{a}{2};0\right)$, $N\left(0;0;\dfrac{a}{2}\right)$ nên $\overrightarrow{N M}=\left(\dfrac{a}{2} ; \dfrac{a}{2} ;-\dfrac{a}{2}\right)$.\\
			Suy ra $\cos \alpha=\dfrac{\dfrac{a^{2}}{2}}{\sqrt{0^{2}+0^{2}+a^{2}} \sqrt{\left(\dfrac{a}{2}\right)^{2}+\left(\dfrac{a}{2}\right)^{2}+\left(-\dfrac{a}{2}\right)^{2}}}=\dfrac{\sqrt{3}}{3}$.
		}{
			\begin{tikzpicture}[scale=0.8,>=stealth, font=\footnotesize, line join=round, line cap=round]
				%%\draw[color=gray!50,dashed] (-6,-6) grid (6,6);
				\tkzSetUpPoint[fill=black]
				\tkzDefPoints{0/0/O,3/0/C,0/3/A,-2/-2/B}
				\tkzDefMidPoint(A,B)\tkzGetPoint{M}
				\tkzDefMidPoint(O,C)\tkzGetPoint{N}
				\coordinate (x) at ($(O)!1.3!(B)$);
				\coordinate (y) at ($(O)!1.2!(C)$);
				\coordinate (z) at ($(O)!1.2!(A)$);
				\draw[line width = 0.6pt,->](B) -- (x);
				\draw[line width = 0.6pt,->](C) -- (y);
				\draw[line width = 0.6pt,->](A) -- (z);
				\tkzDrawPoints[fill=black,size=4](O,A,B,C,M,N)
				\tkzLabelPoints[above left](A)
				\tkzLabelPoints[left](B,O)
				\tkzLabelPoints[left](M)
				\tkzLabelPoints[above](N)
				\tkzLabelPoints[right](z,x)
				\tkzLabelPoints[below](y)
				\tkzLabelPoints[above right](C)
				\tkzDrawSegments[dashed](O,A O,B O,C M,N)
				\tkzDrawSegments(A,B B,C A,C)
				\tkzMarkRightAngles[size=0.16](B,O,C A,O,C B,O,A)
			\end{tikzpicture}
		}
	}
\end{ex}
\begin{ex}%[2H3K4-1]%
	Hình chóp $S.ABC$ có đáy là tam giác vuông tại $B$ có $AB=a$,  $AC =2a$. $SA$ vuông góc với mặt phẳng đáy, $SA = 2a$. Gọi $\psi$ là góc tạo bởi hai mặt phẳng $(SAC)$ và $(SBC)$. Tính $\cos\psi$.
	\choice
	{$\dfrac{1}{2}$}
	{$\dfrac{\sqrt{3}}{5}$}
	{$\dfrac{\sqrt{3}}{2}$}
	{\True $\dfrac{\sqrt{15}}{5}$}
	\loigiai{
		\immini{Chọn hệ trục tọa độ $Bxyz$ như hình vẽ.\\
			Ta tính được $B(0;0;0)$, $A(a;0;0)$, $C(0;a \sqrt{3};0)$, $S(a;0;2a)$,\\
			$\vv{SA}=(0;0;-2a)$, $\vv{SB}=(-a;0;-2a)$, $\vv{SC}=(-a;a\sqrt{3};-2a)$,\\
			$\vv{n_1}= \left[ \vv{SA} ; \vv{SC} \right] = (2a^2\sqrt{3};2a^2;0)$ là VTPT của $(SAC)$,\\
			$\vv{n_2}= \left[ \vv{SB} ; \vv{SC} \right] = (2a^2\sqrt{3};0;-a^2\sqrt{3})$ là VTPT của $(SBC)$.\\
			Ta có $$\cos \psi = \big|\cos \left[\vv{n_1};\vv{n_2}\right]\big|=\dfrac{|\vv{n_1}\vv{n_2}|}{|\vv{n_1}|\cdot |\vv{n_2}|}=\dfrac{\sqrt{15}}{5}.$$
		}{\begin{tikzpicture}[>=stealth,line join=round,line cap=round,font=\footnotesize,scale=0.8]
				\tkzDefPoints{0/0/A, 2/-2/B, 7/0/C}
				\coordinate (S) at ($(A)+(0,5)$);
				\tkzDefPointBy[homothety = center B ratio 1.7](A)
				\tkzGetPoint{E}
				\tkzDefPointBy[homothety = center B ratio 1.4](C)
				\tkzGetPoint{F}
				\coordinate (G) at ($(B)+(0,7)$);
				\tkzDrawSegments[dashed](A,C)
				\tkzDrawSegments(S,A S,B B,C S,C A,B)
				\tkzLabelPoints[below left](A,B)
				\tkzLabelPoints[below right](C)
				\tkzLabelPoints[above](S)
				\tkzMarkRightAngle(A,B,C)
				\draw[->](B)--(E);
				\draw[->](B)--(F);
				\draw[->](B)--(G);
				\draw (-1.5,1.5)node[below left]{$x$} (9,0)node[above]{$y$} (2.5,5)node[right]{$z$};
				\draw (0,2)node[above left]{$2a$};
				\draw (1.3,-1.5)node[above left]{$a$};
				\draw (3,0) node[above]{$2a$};
			\end{tikzpicture}
	}}
\end{ex}

\begin{ex}
	Cho hình chóp $S.ABCD$ có đáy $ ABCD $ là hình chữ nhật, $ AB=a $, $ BC=a\sqrt{3} $, $ SA=a $ và $ SA $ vuông góc với đáy $ ABCD $. Tính $ \sin \alpha $, với $ \alpha $ là góc tạo bởi giữa đường thẳng $ BD $ và mặt phẳng $ (SBC) $.
	\choice
	{\True $\dfrac{\sqrt{2}}{4} $}
	{$ \dfrac{\sqrt{3}}{3} $}
	{$ \dfrac{\sqrt{3}}{4} $}
	{$\dfrac{\sqrt{2}}{2} $}
	\loigiai{
		\immini{
			Đặt hệ trục tọa độ $ Oxyz $ như hình vẽ. Khi đó $A(0;0;0)$, $B(a;0;0)$, $D(0;a\sqrt{3};0)$, $S(0;0;a)$.\\
			Ta có $ \overrightarrow {BD}  = (-a;a\sqrt{3};0) = a(-1;\sqrt{3};0) $ nên đường thẳng $ BD $ có vectơ chỉ phương là $ \overrightarrow {u} = (-1;\sqrt{3};0) $.\\
			Ta có  $\overrightarrow {SB} = (a;0;-a)$, $ \overrightarrow {BC} = (0;a\sqrt{3};0) \Rightarrow \left[\overrightarrow {SB} ,\overrightarrow {BC} \right] = \left( {a^2 \sqrt{3}; 0; a^2\sqrt{3}} \right) = a^2 \sqrt{3}  (1;0;1)$. Nên mặt phẳng $ (SBC) $ có vectơ pháp tuyến là $ \overrightarrow {n} = (1;0;1) $.\\
			$ \alpha $ là góc tạo bởi giữa đường thẳng $ BD $ và mặt phẳng $ (SBC) $ thì \\
			$
			\sin \alpha = \dfrac{\left|\overrightarrow u \cdot \overrightarrow{n} \right|}{\left| {\overrightarrow{u}} \right| \cdot \left| {\overrightarrow {n}} \right|}
			= \dfrac{|( - 1)\cdot 1 + \sqrt 3 \cdot 0 + 0 \cdot1 |}{\sqrt {(- 1)^2  + (\sqrt 3)^2  + 0^2 } \cdot \sqrt {1^2  + 0^2  + 1^2}}
			= \dfrac{\sqrt 2}{4}.
			$
		}{
			\begin{tikzpicture}[scale=0.5,>=stealth]
				\tkzDefPoints{0/0/A, -3/-2/x, 7/0/y, 0/7/z}
				\coordinate (B) at ($(A)+(x)$);
				\coordinate (D) at ($(A)+(y)$);
				\coordinate (C) at ($(B)+(D)$);
				\coordinate (S) at ($(A)+(z)$);
				\coordinate (a) at ($(A)!1.4!(B)$);
				\coordinate (b) at ($(A)!1.2!(D)$);
				\coordinate (c) at ($(S)+(0,1)$);
				\tkzDrawPoints[fill=black](A,B,C,D,S)
				\draw (S)--(B)--(C)--(D)--(S) (S)--(C);
				\draw[dashed] (S)--(A)--(B)--(D)--(A);
				\draw[->] (B)--(a);
				\draw[->] (D)--(b);
				\draw[->] (S)--(c);
				\tkzLabelPoints[left](S,A,B)
				\tkzLabelPoints[below right](C)
				\tkzLabelPoints[above right](D)
				\node[above] at (a) {$y$};
				\node[below] at (b) {$x$};
				\node[left] at (c) {$z$};
			\end{tikzpicture}
		}
	}
\end{ex}

\begin{ex}
	Cho hình chóp tứ giác đều $S.ABCD$ có cạnh đáy bằng $a$, tâm $O$. Gọi $M$ và $N$ lần lượt là trung điểm $SA$ và $BC$. Biết góc giữa $MN$ và $\left(ABCD\right)$ bằng $60^{\circ}$, côsin góc giữa $MN$ và mặt phẳng $(SBD)$ bằng
	\choice
	{$\dfrac{\sqrt{41}}{41}$}
	{$\dfrac{2\sqrt{41}}{41}$}
	{$\dfrac{\sqrt{5}}{5}$}
	{\True $\dfrac{2\sqrt{5}}{5}$}
	\loigiai{
		\begin{center}
			\begin{tikzpicture}[scale=0.8, font=\footnotesize, line join=round, line cap=round, >=stealth]
				\coordinate (A) at (0,0);
				\coordinate (B) at ($(A) +(5,0)$);
				\coordinate (C) at ($(B) +(-140:2.5)$);
				\coordinate (D) at ($(A) +(-140:2.5)$);
				\coordinate (O) at ($(A)!0.5!(C)$);
				\coordinate (S) at ($(O)+(0,4)$);
				\coordinate (M) at ($(S)!0.5!(A)$);
				\coordinate (N) at ($(B)!0.5!(C)$);
				\coordinate (I) at ($(A)!0.5!(O)$);
				\coordinate (K) at ($(S)!0.5!(O)$);
				\coordinate (x) at ($(A)!1.4!(C)$);
				\coordinate (y) at ($(D)!1.4!(B)$);
				\coordinate (z) at ($(O)!1.4!(S)$);
				\foreach \x in {A,B,C,D,O,M,N}{
					\draw[fill] (\x) circle(1pt);
				}
				\draw (D)--(C)--(B) (S)--(B) (S)--(C) (S)--(D);
				\draw[dashed] (A)--(B) (A)--(D) (S)--(A) (S)--(O) (M)--(N) (B)--(D) (A)--(C);
				\draw[->]  (C)--(x) node[right]{$x$};
				\draw[->]  (B)--(y) node[above]{$y$};
				\draw[->]  (S)--(z) node[right]{$z$};
				\path
				node at (A) [left]{$A$}
				node at (B) [below]{$B$}
				node at (C) [below]{$C$}
				node at (D) [below left]{$D$}
				node at (S) [left]{$S$}
				node at (O) [below]{$O$}
				node at (M) [below left]{$M$}
				node at (N) [below right]{$N$};
			\end{tikzpicture}
		\end{center}
		Chọn hệ trục tọa độ như hình vẽ, đặt $OS=k$ $(k>0)$.\\
		Ta có $A\left(-\dfrac{a\sqrt{2}}{2};0;0\right)$, $B\left(0;\dfrac{a\sqrt{2}}{2};0\right)$, $C\left(\dfrac{a\sqrt{2}}{2};0;0\right)$, $D\left(0;-\dfrac{a\sqrt{2}}{2};0\right)$, $S\left(0;0;k\right)$,\\ $M\left(-\dfrac{a\sqrt{2}}{4};0;\dfrac{k}{2}\right)$ và $N\left(\dfrac{a\sqrt{2}}{4};\dfrac{a\sqrt{2}}{4};0\right)$.\\
		Ta có $\vec{u}_{MN}= \vec{MN}=\left(\dfrac{a\sqrt{2}}{2}; \dfrac{a\sqrt{2}}{4};-\dfrac{k}{2}\right)$ và $\vec{n}_{ABCD} = (0;0;1)$.\\
		Ta có \begin{eqnarray*}
			\sin \left(MN;(ABCD)\right)= \dfrac{\left|\vec{u}_{MN} \cdot \vec{n}_{ABCD}\right|}{\left|\vec{u}_{MN}\right| \cdot \left|\vec{n}_{ABCD}\right|} & \Leftrightarrow & \sin 60^{\circ} =\dfrac{\dfrac{k}{2}}{\sqrt{\dfrac{9a^2}{16}+\dfrac{k^2}{4}}} \\
			&\Leftrightarrow & k =  \dfrac{a\sqrt{30}}{2}.
		\end{eqnarray*}
		Ta có $\vec{n}_{SBD}=(1;0;0)$, do đó
		\begin{eqnarray*}
			\sin \left(MN,(SBD)\right)= \dfrac{\left|\vec{u}_{MN} \cdot \vec{n}_{SBD}\right|}{\left|\vec{u}_{MN}\right| \cdot \left|\vec{n}_{SBD}\right|} & \Leftrightarrow & \sin \left(MN,(SBD)\right)= \dfrac{\dfrac{a\sqrt{2}}{2}}{\sqrt{\dfrac{5a^2}{2}}} = \dfrac{\sqrt{5}}{5}.
		\end{eqnarray*}
		$\Rightarrow  \cos \left(MN,(SBD)\right) = \dfrac{2\sqrt{5}}{5}$. (do góc nhọn nên $cos$ dương)
	}
\end{ex}
\begin{ex}
	Cho hình chóp $S.ABCD$ có đáy là hình thang vuông tại $A$ và $B$, $AB=BC=a$, $AD=2a, \ \ SA$ vuông góc với mặt đáy $(ABCD)$, $SA=a$. Gọi $M,N$ lần lượt là trung điểm của $SB$ và $CD$. Tính cosin của góc giữa $MN$ và $(SAC)$.
	\choice
	{$\dfrac{2}{\sqrt{5}}$}
	{$\dfrac{1}{\sqrt{5}}$}
	{$\dfrac{3 \sqrt{5}}{10}$}
	{\True $\dfrac{\sqrt{55}}{10}$}
	\loigiai{
		\immini{
			Chọn hệ trục $Oxyz$ như hình vẽ, với $O \equiv A$. \\
			Khi đó ta có: $A(0;0;0)$, $B(a;0;0)$, $C(a;a;0)$, $D(0;2a;0)$, $S(0;0;a)$. \\
			Khi đó: $M \left(\dfrac{a}{2};0; \dfrac{a}{2} \right)$, $N \left(\dfrac{a}{2}; \dfrac{3a}{2};0 \right)$. \\
			Ta có: $- \dfrac{1}{a} \vec{SA}=(0;0;1)= \vec{u}$; $\dfrac{1}{a} \vec{SC}=(1;1;-1)= \vec{v}$. \\
			Gọi $\vec{n}$ là vectơ pháp tuyến của mặt phẳng $(SAC)$ ta có $\vec{n}= \left[\vec{u}, \vec{v} \right]=(-1;-1;0)$. \\
			Lại có: $\dfrac{2}{a} \vec{MN}=(0;3;-1)= \vec{w}$. \\
			Gọi $\alpha $ là góc giữa $MN$ và $(SAC)$ ta có: $\sin \alpha = \dfrac{\left| \vec{n}. \vec{w} \right|}{\left| \vec{n} \right|. \left| \vec{w} \right|}= \dfrac{3}{2 \sqrt{5}} \Rightarrow \cos \alpha = \dfrac{\sqrt{55}}{10}.$
		}
		{
			\begin{tikzpicture}[scale=0.7,>=stealth]
				\tkzDefPoints{0/0/A, 5/0/D, -2/-2/B, 1/-2/C}
				\tkzDefShiftPoint[A](90:4){S}
				\draw [thick] [->] (0,4)--(0,5) node[right] {$z$};
				\draw [thick] [->] (5,0)--(6,0) node[above] {$y$};
				\draw [thick] [->] (-2,-2)--(-3,-3) node[above] {$x$};
				\tkzDefMidPoint(S,B)\tkzGetPoint{M}
				\tkzDefMidPoint(C,D)\tkzGetPoint{N}
				\tkzDrawSegments(S,B S,C S,D B,C C,D)
				\tkzDrawSegments[dashed](B,A A,D S,A M,N)
				\tkzDrawPoints(A,B,C,D,S,M,N)
				\tkzLabelPoints[below](A, B, C)
				\tkzLabelPoints[above right](D,N)
				\tkzLabelPoints[left](S,M)
				\tkzLabelSegments[right](A,B S,A){$a$}
				\tkzLabelSegment[below](C,B){$a$}
				\tkzLabelSegment[above](A,D){$2a$}
			\end{tikzpicture}
		}
	}
\end{ex}
\Closesolutionfile{ans}
\subsection{BÀI TẬP TRẮC NGHIỆM TỰ LUYỆN}
\TN
	\setcounter{ex}{0}
	\Opensolutionfile{ans}[ans/B3-De2-1]

\begin{ex}%[2H2N2-4]
	Trong không gian $Oxyz$, cho đường thẳng $d\colon\heva{
		& x=6+5t \\
		& y=2+t \\
		& z=1
	}$ và mặt phẳng $\left( P \right)\colon3x-2y+1=0$. Tính góc hợp bởi đường thẳng $d$ và mặt phẳng $\left( P \right)$.
	\choice
	{$30^\circ $}
	{\True $45^\circ $}
	{$60^\circ $}
	{$90^\circ $}
	\loigiai{
		Đường thẳng $d\colon\heva{
			& x=6+5t \\
			& y=2+t \\
			& z=1 \\
		}$ có vectơ chỉ phương $\overrightarrow{u}=\left( 5;1;0 \right)$.\\
		Mặt phẳng $\left( P \right)\colon3x-2y+1=0$ có vectơ pháp tuyến $\overrightarrow{n}=\left( 3;-2;0 \right)$.\\
		Gọi $\alpha $ là góc hợp bởi đường thẳng $d$ và mặt phẳng $\left( P \right)$.\\
		Khi đó: $\sin \alpha =\dfrac{\left| \overrightarrow{u}\cdot\overrightarrow{n} \right|}{\left| {\overrightarrow{u}} \right|\cdot\left| {\overrightarrow{n}} \right|}=\dfrac{\left| 5\cdot3+1\cdot\left( -2 \right)+0\cdot0 \right|}{\sqrt{5^2+1^2}\cdot\sqrt{3^2+{{\left( -2 \right)}^2}}}=\dfrac{\sqrt{2}}{2}$.\\ Suy ra $\alpha =45^\circ $.}
\end{ex}

%G:\My Drive\CODE12-2024\DE-ON-THEO BAI\2H5-TACH DE\Bai3-De2.tex
\begin{ex}%[2H2N2-4]
	Trong không gian $Oxyz$, cho ba điểm $M\left( 2; 3; -1 \right)$, $N\left( -1; 1; 1 \right)$ và $P\left( 1; m-1; 2 \right)$. Tìm $m$ để tam giác $MNP$ vuông tại $N$.
	\choice
	{\True $m=0$}
	{$m=-4$}
	{$m=2$}
	{$m=-6$}
	\loigiai{
		Ta có\\
		$\overrightarrow{NM}=\left( 3; 2; -2 \right)$, $\overrightarrow{NP}=\left( 2; m-2; 1 \right)$.\\
		Tam giác $MNP$ vuông tại $N$ khi và chỉ khi
		\begin{eqnarray*}
			& & \overrightarrow{NM}\cdot\overrightarrow{NP}=0\\
			&\Leftrightarrow & 3\cdot2+2\cdot\left( m-2 \right)-2\cdot1=0\\
			&\Leftrightarrow & m=0.
		\end{eqnarray*}
		Vậy giá trị cần tìm của $m$ là $m=0$.}
\end{ex}

%G:\My Drive\CODE12-2024\DE-ON-THEO BAI\2H5-TACH DE\Bai3-De2.tex
\begin{ex}%[2H5N2-7]
	Trong không gian $Oxyz$, tính góc giữa hai đường thẳng $d_1\colon\dfrac{x}{1}=\dfrac{y+1}{-1}=\dfrac{z-1}{2}$ và $d_2\colon\dfrac{x+1}{-1}=\dfrac{y}{1}=\dfrac{z-3}{1}$.
	\choice
	{$60^\circ $}
	{$30^\circ $}
	{$45^\circ $}
	{\True $90^\circ $}
	\loigiai{
		Ta có $\overrightarrow{u}_{d_1}=\left( 1;-1;2 \right)$ và
		$\overrightarrow{u}_{d_2}=\left( -1;1;1 \right)$ lần lượt là véc tơ chỉ phương của $d_1$ và $d_2$.\\
		$\overrightarrow{u}_{d_1}\cdot\overrightarrow{u}_{d_2}=1\cdot\left( -1 \right)+\left( -1 \right)\cdot1+2\cdot1=0\Rightarrow {d_1}\bot {d_2}\Rightarrow \left( \widehat{d_1,d_2} \right)=90^\circ $.}
\end{ex}

%G:\My Drive\CODE12-2024\DE-ON-THEO BAI\2H5-TACH DE\Bai3-De2.tex
\begin{ex}%[2H5N2-7]
	Trong không gian $Oxyz$, cho đường thẳng $d\colon\dfrac{x-4}{1}=\dfrac{y-5}{2}=\dfrac{z}{3}$ mặt phẳng $\left( \alpha \right)$ chứa đường thẳng $d$ sao cho khoảng cách từ $O$ đến $\left( \alpha \right)$ đạt giá trị lớn nhất. Khi đó góc giữa mặt phẳng $\left( \alpha \right)$ và trục $Ox$ là $\varphi $ thỏa mãn.
	\choice
	{$\sin \varphi =\dfrac{2}{3\sqrt{3}}$}
	{$\sin \varphi =\dfrac{1}{3\sqrt{3}}$}
	{$\sin \varphi =\dfrac{1}{2\sqrt{3}}$}
	{\True $\sin \varphi =\dfrac{1}{\sqrt{3}}$}
	\loigiai{
		Đường thẳng $d$ có véc tơ chỉ phương $\overrightarrow{u}=\left( 1; 2; 3 \right)$.\\
		Gọi $H$ là hình chiếu của $O$ lên $d$, $K$ là hình chiếu của $O$ lên $\left( \alpha \right)$.\\
		Ta có
		$d\left( O,\left( \alpha \right) \right)=OK\le OH\Rightarrow d\left( O,\left( \alpha \right) \right)$ lớn nhất bằng $OH$ khi $K\equiv H$.\\
		Khi đó $\left( \alpha \right)$ chứa $d$ và nhận $\overrightarrow{n}=\overrightarrow{OH}$ làm véc tơ pháp tuyến.\\
		$H\in d\Rightarrow H\left( 4+t; 5+2t; 3t \right)\Rightarrow \overrightarrow{OH}=\left( 4+t; 5+2t; 3t \right)$.\\
		Vì $OH\bot d\Rightarrow \overrightarrow{OH}\cdot\overrightarrow{u}=0\Leftrightarrow 4+t+2\left( 5+2t \right)+3\cdot3t=0\Leftrightarrow 14t+14=0\Leftrightarrow t=-1$.\\
		$\Rightarrow H\left( 3; 3; -3 \right)$, $\overrightarrow{OH}=\left( 3; 3; -3 \right)$.\\
		Trục $Ox$ có véc tơ chỉ phương  $\overrightarrow{i}=\left( 1; 0; 0 \right)$.\\
		$\sin \varphi =\dfrac{\left| \overrightarrow{i}\cdot\overrightarrow{n} \right|}{\left| \overrightarrow{i} \right|\cdot\left| \overrightarrow{n} \right|}=\dfrac{\left| 3 \right|}{\sqrt{1}\cdot\sqrt{3^2+3^2+{{\left( -3 \right)}^2}}}=\dfrac{1}{\sqrt{3}}$.}
\end{ex}

%G:\My Drive\CODE12-2024\DE-ON-THEO BAI\2H5-TACH DE\Bai3-De2.tex
\begin{ex}%[2H5N1-3]
	Trong không gian $Oxyz$, gọi $\left( P \right)$ là mặt phẳng chứa trục $Oy$ và tạo với mặt phẳng $y+z+1=0$ một góc $60^\circ$. Phương trình mặt phẳng $\left( P \right)$ là
	\choice
	{$\hoac{
			& x-z-1=0 \\
			& x-z=0 \\
		}$}
	{$\hoac{
			& x-2z=0 \\
			& x+z=0 \\
		}$}
	{$\hoac{
			& x-y=0 \\
			& x+y=0 \\
		}$}
	{\True $\hoac{
			& x-z=0 \\
			& x+z=0 \\
		}$}
	\loigiai{
		+) Do $\left( P \right)$ chứa trục $Oy$ nên phương trình $\left( P \right)$ có dạng $\left( P \right)\colon ax+cz=0$, $\left( {a^2}+c^2>0 \right)$.\\
		+) Gọi $\left(Q\right)\colon y+z+1=0$.\\ Ta có $\cos \left( \left( P \right),\,\left( Q \right) \right)=\cos 60^\circ \Leftrightarrow \dfrac{\left| c \right|}{\sqrt{a^2+c^2}\cdot\sqrt{2}}=\dfrac{1}{2}\Leftrightarrow c=\pm a$.\\
		Khi đó $\hoac{
			& \left( P \right)\colon ax+az=0 \\
			& \left( P \right)\colon ax-az=0 \\
		}\Leftrightarrow \hoac{
			& \left( P \right)\colon x+z=0 \\
			& \left( P \right)\colon x-z=0 \\
		}$ .}
\end{ex}

%G:\My Drive\CODE12-2024\DE-ON-THEO BAI\2H5-TACH DE\Bai3-De2.tex
\begin{ex}%[2H5N1-4]
	Với giá trị nào của $m$ thì đường thẳng $\left( D \right)\colon\dfrac{x+1}{2}=\dfrac{y-3}{m}=\dfrac{z-1}{m-2}$ vuông góc với mặt phẳng $\left( P \right)\colon x+3y+2z=2$.
	\choice
	{\True $6$}
	{$5$}
	{$-7$}
	{$1$}
	\loigiai{
		Vectơ chỉ phương của $\left( D \right)$ là $\overrightarrow{a}=\left( 2;m;m-2 \right)$.\\
		Vectơ pháp tuyến của $\left( P \right)$ là $\overrightarrow{n}=\left( 1;3;2 \right)$.\\
		$\left( D \right)\bot \left( P \right)\Leftrightarrow $ $\overrightarrow{a}$ và $\overrightarrow{n}$ cùng phương.\\ $2=\dfrac{m}{3}=\dfrac{m-2}{2}\Leftrightarrow m=6$.}
\end{ex}

%G:\My Drive\CODE12-2024\DE-ON-THEO BAI\2H5-TACH DE\Bai3-De2.tex
\begin{ex}%[2H5N1-4]
	Trong không gian $Oxyz$, cho mặt phẳng $\left( P \right)\colon mx+ny-2z-1=0$ và đường thẳng $\dfrac{x}{n+1}=\dfrac{y}{m}=\dfrac{1-z}{1}$ với $m\ne 0$, $m\ne -1$. Khi $\left( P \right)\bot d$ thì tổng $m+n$ bằng bao nhiêu?
	\choice
	{$m+n=2$}
	{\True $m+n=-2$}
	{$m+n=-\dfrac{1}{2}$}
	{$m+n=-\dfrac{2}{3}$}
	\loigiai{
		Sử dụng tỷ lệ thức $\dfrac{m}{n+1}=\dfrac{n}{m}=\dfrac{-2}{-1}\Rightarrow \dfrac{m+n}{n+1+m}=2\Rightarrow m+n=-2$.}
\end{ex}

%G:\My Drive\CODE12-2024\DE-ON-THEO BAI\2H5-TACH DE\Bai3-De2.tex
\begin{ex}%[2H5N2-7]
	Trong không gian $Oxyz$, cho hai đường thẳng $d_1:\dfrac{x}{1}=\dfrac{y+1}{-1}=\dfrac{z-1}{2}$ và $d_2:\dfrac{x+1}{-1}=\dfrac{y}{1}=\dfrac{z-3}{1}$. Góc giữa hai đường thẳng đó bằng
	\choice
	{\True $90^\circ $}
	{$60^\circ $}
	{$30^\circ $}
	{$45^\circ $}
	\loigiai{
		Đường thẳng $d_1$ có véctơ chỉ phương $\overrightarrow{u}_1=\left( 1;-1;2 \right)$.\\
		Đường thẳng $d_2$ có véctơ chỉ phương $\overrightarrow{u}_2=\left( -1;1;1 \right)$.\\
		Gọi $\alpha$ là góc giữa hai đường thẳng trên.\\
		Khi đó ta có $\cos \alpha =\left| \cos \left( \overrightarrow{u}_1,\overrightarrow{u}_2 \right) \right|=\dfrac{\left| 1\cdot\left( -1 \right)+\left( -1 \right)\cdot1+2\cdot1 \right|}{\sqrt{1^2+{{\left( -1 \right)}^2}+2^2}\cdot\sqrt{{{\left( -1 \right)}^2}+1^2+1^2}}=0$.\\$\Rightarrow \left( \widehat{d_1,d_2} \right)=90^\circ $.}
\end{ex}

%G:\My Drive\CODE12-2024\DE-ON-THEO BAI\2H5-TACH DE\Bai3-De2.tex
\begin{ex}%[2H5N2-7]
	Trong không gian $Oxyz$, cho đường thẳng $\Delta \colon x=\dfrac{y}{2}=\dfrac{z-1}{3}$ và mặt phẳng $\left( P \right)\colon4x+2y+z-1=0$. Khi đó khẳng định nào sau đây là đúng?
	\choice
	{\True Góc tạo bởi $\left( \Delta \right)$ và $\left( P \right)$ lớn hơn $30^\circ $}
	{$ \Delta \parallel\left( P \right)$}
	{$ \Delta\bot \left( P \right)$}
	{$ \Delta \subset \left( P \right)$}
	\loigiai{
		Ta có $\sin \left( \widehat{\Delta ,\left( P \right)} \right)=\dfrac{11}{7\sqrt{6}}>\dfrac{1}{2}$.\\ Suy ra góc tạo bởi $ \Delta $ và $\left( P \right)$ lớn hơn $30^\circ $.}
\end{ex}

%G:\My Drive\CODE12-2024\DE-ON-THEO BAI\2H5-TACH DE\Bai3-De2.tex
\begin{ex}%[2H2N2-4]
	Trong không gian $Oxyz$, cho mặt phẳng $\left( P \right)\colon3x+4y+5z-8=0$ và đường thẳng $d\colon\heva{
		& x=2-3t \\
		& y=-1-4t \\
		& z=5-5t \\
	}$. Góc giữa đường thẳng $d$ và mặt phẳng $\left( P \right)$ là
	\choice
	{\True $90^\circ $}
	{$45^\circ $}
	{$60^\circ $}
	{$30^\circ $}
	\loigiai{
		Mặt phẳng $\left( P \right)$ có một véc tơ pháp tuyến là $\overrightarrow{n}=\left( 3;4;5 \right)$.\\
		Đường thẳng $d$ có một véc tơ chỉ phương là $\overrightarrow{u}=\left( -3;-4;-5 \right)$.\\
		Ta có $\overrightarrow{n}=-\overrightarrow{u}\Rightarrow d\bot \left( P \right)$ nên góc $90^\circ $}
\end{ex}

%G:\My Drive\CODE12-2024\DE-ON-THEO BAI\2H5-TACH DE\Bai3-De2.tex
\begin{ex}%[2H5N2-7]
	Trong không gian $Oxyz$, cho hai đường thẳng $d_1\colon\heva{
		& x=-1-t \\
		& y=3+4t \\
		& z=3+3t \\
	}$ và $d_2\colon\dfrac{x}{1}=\dfrac{y+8}{-4}=\dfrac{z+3}{-3}$. Tính góc hợp bởi đường thẳng $d_1$ và $d_2$.\\
	\choice
	{$90^\circ $}
	{$60^\circ $}
	{\True $0^\circ $}
	{$30^\circ $}
	\loigiai{
		Ta có đường thẳng $d_1\colon \heva{
			& x=-1-t \\
			& y=3+4t \\
			& z=3+3t \\
		}$ có vectơ chỉ phương là $\overrightarrow{u_1}=\left( -1; 4; 3 \right)$,\\
		đường thẳng $d_2\colon\dfrac{x}{1}=\dfrac{y+8}{-4}=\dfrac{z+3}{-3}$ có vectơ chỉ phương là $\overrightarrow{u_2}=\left( 1;-4;-3 \right)$.\\
		Vì $\overrightarrow{u_1}=\left( -1;4;3 \right)$ và $\overrightarrow{u_2}=\left( 1;-4;-3 \right)$ cùng phương với nhau nên góc hợp bởi đường thẳng $d_1$ và $d_2$ bằng $0^\circ $.}
\end{ex}

%G:\My Drive\CODE12-2024\DE-ON-THEO BAI\2H5-TACH DE\Bai3-De2.tex
\begin{ex}%[2H5N2-7]
	Trong không gian $Oxyz$, cho mặt phẳng $(P)\colon -\sqrt{3}x+y+1=0$. Tính góc tạo bởi $(P)$ với trục $Ox$?
	\choice
	{$120^\circ$}
	{$30^\circ$}
	{$150^\circ$}
	{\True $60^\circ$}
	\loigiai{
		Mặt phẳng $(P)$ có véc tơ pháp tuyến $\overrightarrow{n}=(-\sqrt{3};1;0)$\\
		Trục $Ox$ có có véc tơ pháp tuyến $\overrightarrow{i}=(1;0;0)$.\\
		Góc tạo bởi $(P)$ với trục $Ox$\\
		$\sin((P),Ox)=\left| \cos((P), Ox) \right|=\dfrac{\left| \overrightarrow{n}\cdot\overrightarrow{i} \right|}{\left| \overrightarrow{n} \right|\cdot\left| \overrightarrow{i} \right|}=\dfrac{\left| -\sqrt{3}\cdot1+1\cdot0+0\cdot0 \right|}{\sqrt{3+1}\cdot\sqrt{1}}=\dfrac{\sqrt{3}}{2}$.\\
		Vậy góc tạo bởi $(P)$ với trục $Ox$ bằng $60^\circ$.}
\end{ex}
	\Closesolutionfile{ans}

\TNTF
	\setcounter{ex}{0}
	\Opensolutionfile{ans}[ans/B3-De2-2]

\begin{ex}%[2H5N1-3]
	Trong không gian $Oxyz$, cho hai đường thẳng $d_1\colon\heva{
		& x=2+t \\
		& y=-1+t \\
		& z=3 \\
	}$ và $d_2\colon\heva{
		& x=1-t \\
		& y=2 \\
		& z=-2+t \\
	}$. Xét tính đúng sai của các khẳng định sau
	\choiceTF
	{Đường thẳng $d_1$ có một vectơ chỉ phương là $\overrightarrow{u_1}=\left( 1; 1; 3 \right)$}
	{\True Góc giữa hai đường thẳng $d_1$ và $d_2$ bằng $60^\circ $}
	{\True Đường thẳng $d_2$ có một vectơ chỉ phương là $\overrightarrow{u_2}=\left( -1; 0; 1 \right)$}
	{Giá trị cosin của góc giữa hai đường thẳng $d_1$ và $d_2$ bằng $-\dfrac{1}{2}$}
	
	\loigiai{
		\begin{itemchoice}
			\itemch Sai. Đường thẳng $d_1$ có một vectơ chỉ phương là $\overrightarrow{u_1}=\left( 1; 1; 0 \right)$.
			\itemch Đúng. Ta có $\cos \left( {d_1}, d_2 \right)=\dfrac{\left| 1\cdot\left( -1 \right)+0+0 \right|}{\sqrt{2}\cdot\sqrt{2}}=\dfrac{1}{2}\Rightarrow \left( d_1, d_2 \right)=60^\circ $.
			\itemch Đúng. Đường thẳng $d_2$ có một vectơ chỉ phương là $\overrightarrow{u_2}=\left( -1; 0; 1 \right)$.
			\itemch Sai. Ta có $\cos \left( {d_1},\,d_2 \right)=\dfrac{\left| 1\cdot\left( -1 \right)+0+0 \right|}{\sqrt{2}\cdot\sqrt{2}}=\dfrac{1}{2}$.
		\end{itemchoice}
	}
\end{ex}

%G:\My Drive\CODE12-2024\DE-ON-THEO BAI\2H5-TACH DE\Bai3-De2.tex
\begin{ex}%%[2H5N2-7]
	Trong không gian $Oxyz$, cho mặt phẳng $\left( P \right)\colon2x+y+2z-1=0$ và hai điểm $A\left( 1; -1; 2 \right)$, $B\left( 0; 1; -1 \right)$. Xét tính đúng sai của các khẳng định sau
	\choiceTF
	{Giá trị cosin của góc giữa đường thẳng $AB$ và mặt phẳng $\left( P \right)$ bằng $\dfrac{2}{\sqrt{21}}$}
	{Đường thẳng $AB$ vuông góc với mặt phẳng $\left( P \right)$}
	{\True Mặt phẳng $\left( OAB \right)$ có một vectơ pháp tuyến là $\overrightarrow{n}=\left( -1; 1; 1 \right)$}
	{\True Giá trị cosin của góc giữa mặt phẳng $\left( OAB \right)$ và mặt phẳng $\left( P \right)$ bằng $\dfrac{\sqrt{3}}{9}$}
	\loigiai{
		\begin{itemchoice}
			\itemch Sai. Ta có $\sin \left( AB, \left( P \right) \right)=\dfrac{\left| -1\cdot2+2\cdot1+\left( -3 \right)\cdot2 \right|}{\sqrt{14}\cdot3}=\dfrac{\sqrt{7}}{7}$.\\ $\Rightarrow \cos \left( d, \left( P \right) \right)=\sqrt{1-\dfrac{1}{7}}=\dfrac{\sqrt{42}}{7}$.
			\itemch Sai. Đường thẳng $AB$ có một véctơ chỉ phương $\overrightarrow{u}=\overrightarrow{AB}=\left( -1; 2; -3 \right)$, ta thấy véctơ $\overrightarrow{u}$ không cùng phương với véctơ $\overrightarrow{n}$ nên $AB$ không vuông góc với mặt phẳng $\left( P \right)$.
			\itemch Đúng. Ta có $A\left( 1; -1; 2 \right)\Rightarrow \overrightarrow{OA}=\left( 1; -1; 2 \right)$; $B\left( 0; 1; -1 \right)\Rightarrow \overrightarrow{OB}=\left( 0; 1; -1 \right).$\\
			Do đó, mặt phẳng $\left( OAB \right)$ có một vectơ pháp tuyến là $\overrightarrow{n}=\left[ \overrightarrow{OA},\overrightarrow{OB} \right]=\left( -1; 1;  \right)$.
			\itemch Đúng. Mặt phẳng $\left( OAB \right)$ có một vectơ pháp tuyến là $\overrightarrow{n}=\left( -1; 1 ; 1 \right)$ và mặt phẳng $\left( P \right)$ có một vectơ pháp tuyến là $\overrightarrow{n_1}=\left( 2; 1; 2 \right)$.\\ $\Rightarrow \cos \left( \left( OAB \right),\left( P \right) \right)=\dfrac{\left| 2\cdot\left( -1 \right)+1\cdot1+2\cdot1 \right|}{3\cdot\sqrt{3}}=\dfrac{\sqrt{3}}{9}.$
		\end{itemchoice}
	}
\end{ex}

%G:\My Drive\CODE12-2024\DE-ON-THEO BAI\2H5-TACH DE\Bai3-De2.tex
\begin{ex}%[2H5N2-7]
	Trong không gian $Oxyz$, cho đường thẳng $d\colon\dfrac{x}{1}=\dfrac{y}{-2}=\dfrac{z}{1}$ và mặt phẳng $\left( P \right)\colon5x+11y+2z-4=0$. Xét tính đúng sai của các khẳng định sau
	\choiceTF
	{\True Đường thẳng $d$ có một vectơ chỉ phương là $\overrightarrow{u}=\left( 1; -2; 1 \right)$}
	{Mặt phẳng $\left( P \right)$ có một vectơ pháp tuyến là $\overrightarrow{n}=\left( -5; -11; 2 \right)$}
	{Giá trị cosin của góc giữa đường thẳng $d$ và mặt phẳng $\left( P \right)$ bằng $\dfrac{1}{2}$}
	{\True Góc giữa đường thẳng $d$ và mặt phẳng bằng $30^\circ $}
	\loigiai{
		\begin{itemchoice}
			\itemch Đúng. Đường thẳng $d$ có một vectơ chỉ phương là $\overrightarrow{u}=\left( 1; -2; 1 \right)$.
			\itemch Sai. Mặt phẳng $\left( P \right)$ có một vectơ pháp tuyến là $\overrightarrow{n}=\left( 5; 11; 2 \right)$.
			\itemch Sai. Ta có $\sin \left( d, \left( P \right) \right)=\dfrac{\left| 1\cdot5+\left( -2 \right)\cdot11+1\cdot2 \right|}{\sqrt{6}\cdot5\sqrt{6}}=\dfrac{1}{2}\Rightarrow \cos \left( d, \left( P \right) \right)=\dfrac{\sqrt{3}}{2}$.
			\itemch Đúng. Ta có $\sin \left( d, \left( P \right) \right)=\dfrac{\left| 1\cdot5+\left( -2 \right)\cdot11+1\cdot2 \right|}{\sqrt{6}\cdot5\sqrt{6}}=\dfrac{1}{2}\Rightarrow \left( d, \left( P \right) \right)=30^\circ $.
		\end{itemchoice}
	}
\end{ex}

%G:\My Drive\CODE12-2024\DE-ON-THEO BAI\2H5-TACH DE\Bai3-De2.tex
\begin{ex}%[2H5N2-7]
	Trong không gian $Oxyz$, cho mặt phẳng $\left( P \right)\colon3x+4y+5z+2=0$ và đường thẳng $d$ là giao tuyến của hai mặt phẳng $\left( \alpha \right)\colon x-2y+1=0$ và $\left( \beta \right)\colon x-2z-3=0$. Xét tính đúng sai của các khẳng định sau
	\choiceTF
	{\True Mặt phẳng $\left( P \right)$ có một vectơ pháp tuyến là $\overrightarrow{n}=\left( 3; 4; 5 \right)$}
	{Góc giữa đường thẳng $d$ và mặt phẳng $\left( P \right)$ bằng $30^\circ $}
	{Đường thẳng $d$ có một vectơ chỉ phương là $\overrightarrow{u}=\left( 2\,;-1\,;1 \right)$}
	{\True Đường thẳng $d$ cắt mặt phẳng $\left( P \right)$ tại $A\left( \dfrac{7}{15}; \dfrac{11}{15}; -\dfrac{19}{15} \right)$}
	\loigiai{
		\begin{itemchoice}
			\itemch Đúng. Mặt phẳng $\left( P \right)$ có một vectơ pháp tuyến là $\overrightarrow{n}=\left( 3; 4; 5 \right)$.
			\itemch Sai. Ta có $\sin \left( d, \left( P \right) \right)=\dfrac{\left| 4\cdot3+2\cdot4+2\cdot5 \right|}{2\sqrt{6}\cdot5\sqrt{2}}=\dfrac{\sqrt{3}}{2}\Rightarrow \sin \left( d, \left( P \right) \right)=60^\circ $. 
			\itemch Sai.  Mặt phẳng $\left( \alpha \right)$ có một vectơ pháp tuyến là $\overrightarrow{n_1}=\left( 1; -2; 0 \right)$.\\
			Mặt phẳng $\left( \beta \right)$ có một vectơ pháp tuyến là $\overrightarrow{n_2}=\left( 1; 0; -2 \right)$.\\
			$\Rightarrow d$ có một vectơ chỉ phương là $\overrightarrow{u}=\left[ \overrightarrow{n_1}, \overrightarrow{n_2} \right]=\left( 4; 2; 2 \right)$.
			\itemch Đúng. Ta có\\
			$\heva{
				& x-2y+1=0 \\
				& x-2z-3=0 \\
				& 3x+4y+5z+2=0 \\
			}\Leftrightarrow \heva{
				& x=\dfrac{7}{15} \\
				& y=\dfrac{11}{15} \\
				& z=-\dfrac{19}{15} \\
			}$.\\ Suy ra đường thẳng $d$ cắt mặt phẳng $\left( P \right)$ tại $A\left( \dfrac{7}{15}; \dfrac{11}{15}; -\dfrac{19}{15} \right)$.
		\end{itemchoice}
	}
\end{ex}

\Closesolutionfile{ans}
\TNSA

	\setcounter{ex}{0}
	\Opensolutionfile{ans}[ans/B3-De2-3]
	
\begin{ex}%[2H5H2-7]
	Trong hệ tọa độ $Oxyz$, một vật chuyển động theo quĩ đạo là một đường thẳng. Tại thời điểm ban đầu, vật ở vị trí điểm $A(1;5;0)$, sau $10$ phút vật ở vị trí điểm $B(101;205;1250)$. Hỏi vật chuyển động theo phương hợp với mặt đất góc bao nhiêu độ ( giả sử mặt đất là mặt phẳng $Oxy$, kết quả làm tròn đến hàng phần chục).\\
	\shortans[oly]{$79{,}9$}
	\loigiai{
		Ta có $\overrightarrow{AB}=(100;200;1250)$.\\
		$\sin\left( AB,\left( Oxy \right) \right)=\dfrac{\left| \overrightarrow{AB}\cdot{{\overrightarrow{n}}_{Oxy}} \right|}{\left| \overrightarrow{AB} \right|\cdot\left| {{\overrightarrow{n}}_{Oxy}} \right|}=\dfrac{1250}{\sqrt{{{100}^2}+{{200}^2}+{{1250}^2}}}$ nên $\widehat{\left( AB;(Oxy) \right)}\approx 79,9^0$.}
\end{ex}

\begin{ex}%[2H5H2-7]
	Cho hình lăng trụ tam giác đều $ABC.A^\prime B^\prime C^\prime $ có cạnh bên $2a$, góc tạo bởi $A^\prime B$ và mặt đáy là $60^\circ$. Gọi $M$ là trung điểm $BC$. Ta có $\cos\left( A^\prime C, AM \right)=\dfrac{\sqrt{a}}{b}$ với $\dfrac{a}{b}$ là phân số tối giản, $a,b\in N$. Tổng $a+b$ bằng bao nhiêu?\\
	\shortans[oly]{$7$}
	\loigiai{
		\immini{	Chọn hệ trục tọa độ như hình vẽ.\\
			Ta có $AB=AC=BC=\dfrac{2a}{\tan 60^\circ}=\dfrac{2a}{\sqrt{3}}$.\\$\Rightarrow MC=\dfrac{BC}{2}=\dfrac{a}{\sqrt{3}}$.\\
			$AM=\dfrac{AB\sqrt{3}}{2}=a$.\\Khi đó: $M\left( 0; 0; 0 \right)$, $A\left( 0; a; 0 \right)$, $C\left( \dfrac{a}{\sqrt{3}}; 0; 0 \right)$, $A^\prime \left( 0; a; 2a \right)$.\\
			Ta có $\overrightarrow{A^\prime C}=\left( \dfrac{a}{\sqrt{3}} ;-a; -2a \right)$ $\Rightarrow A^\prime C=\dfrac{4a}{\sqrt{3}}$.\\
			$\overrightarrow{AM}=\left( 0; -a; 0 \right)$ $\Rightarrow AM=a$.\\
			Khi đó có $\cos \left( A^\prime C, AM \right)=\dfrac{\left| \overrightarrow{A^\prime C}\cdot\overrightarrow{AM} \right|}{\left| \overrightarrow{A^\prime C} \right|\cdot\left| \overrightarrow{AM} \right|}=\dfrac{\sqrt{3}}{4}$.\\ Vậy $a+b=7$.}{	\begin{tikzpicture}[scale=0.7, font=\footnotesize,line join=round, line cap=round, >=stealth]
				\path
				(0:0) coordinate (B)
				(-45:3) coordinate (C)
				(0:5) coordinate (A)
				\foreach \x in {A,B,C}{(\x)+(90:5.5) coordinate (\x')}
				($(B)!0.5!(C)$) coordinate (M)
				($(B')!0.5!(C')$) coordinate (M')
				($(M)!1.5!(M')$) coordinate (T1)
				($(M)!1.5!(C)$) coordinate (T2)
				($(M)!1.5!(A)$) coordinate (T3)
				;
				\draw[dashed] (A)--(B) (A)--(M) (C)--(A')--(B);
				\draw 
				(A)--(C) (B')--(B)--(C)
				(A')--(C')--(B')--cycle  (C)--(C') (A)--(A') 
				;
				\draw[thick,->] (M)--(T1) ;
				\draw[thick,->] (A)--(T3) ;
				\draw[thick,->] (C)--(T2) ;
				\foreach \x/\g in {A/-10,B/-90,C/180,A'/0,B'/180,C'/90,M/180}\draw[fill=white] (\x) circle (.03) +(\g:.3) node{$\x$};
				\draw[right]  (T1) node{$z$} (T3) node{$y$} (T2) node{$x$};
		\end{tikzpicture}}
	}
\end{ex}

\begin{ex}%[2H5H2-7]
	Trong không gian $Oxyz$, cho đường thẳng $d\colon\dfrac{x+1}{2}=\dfrac{y-1}{2}=\dfrac{z+2}{1}$ và mặt phẳng $(P)\colon3x+my-1=0$ ($m$ là tham số ).Tìm $m$ để đường thẳng $d$ tạo với mặt phẳng $\left( P \right)$ góc $\alpha $ thỏa mãn $\sin\alpha =\dfrac{2}{3}$.\\
	\shortans[oly]{$0$}
	\loigiai{
		Ta có đường thẳng $d$ có véc tơ chỉ phương $\overrightarrow{u_d}=(2; 2; 1)$.\\Mặt phẳng $\left(P\right)$ có véc tơ pháp tuyến  $\overrightarrow{n}_P=(3; m; 0)$.\\ Suy ra $\sin\left( d,\left( P \right) \right)=\dfrac{\left| 6+2m \right|}{3\cdot\sqrt{9+m^2}}$.\\
		Theo giả thiết ta có
		\begin{eqnarray*}
			& & \sin\left( d,\left( P \right) \right)=\dfrac{2}{3}\\
			&\Leftrightarrow & \dfrac{\left| 6+2m \right|}{3\cdot\sqrt{9+m^2}}=\dfrac{2}{3}\\
			&\Leftrightarrow & 4m^2+24m+36=4m^2+36 \\
			&\Leftrightarrow & m=0.
		\end{eqnarray*}
	}
\end{ex}

\begin{ex}%[2H5H2-7]
	Trong không gian, cho mặt phẳng $(P)$ có phương trình $ax+by+cz-1=0$ với $c<0$ đi qua hai điểm $A(0; 1; 0)$, $B(1; 0; 0)$ tạo với $\left( Oyz \right)$ một góc $60^\circ$. Khi đó $a+b-\sqrt{2}c$ bằng\\
	\shortans[oly]{$4$}
	\loigiai{
		Mặt phẳng $\left( P \right)$ đi qua $A$, $B$ nên $a=b=1\quad (1)$ .\\
		Ta có $\cos ((P),(Oyz))=\dfrac{\left| a \right|}{\sqrt{a^2+b^2+c^2}\cdot\sqrt{1}}=\dfrac{1}{2}\quad(2)$.\\
		Thay (1) vào (2) ta được:
		\begin{eqnarray*}
			& & \dfrac{1}{\sqrt{2+c^2}}=\dfrac{1}{2}\\
			&\Leftrightarrow & \sqrt{2+c^2}=2\\
			&\Leftrightarrow & c^2=2\\
			&\Leftrightarrow & \hoac{&c=\sqrt{2} \quad\left(\text{Loại}\right)\\&c=-\sqrt{2}}\\
			&\Leftrightarrow & c=-\sqrt{2}
		\end{eqnarray*}
		Vậy $a+b-\sqrt{2}c=4$.}
\end{ex}

\begin{ex}%[2H5H2-7]
	Có hai bức tường hình vuông cạnh $5m$, vuông góc với nhau và cùng vuông góc với mặt đất, hai mặt tường giao nhau tại cột $d$. Trên cột $d$ có một điểm $A$ cách mặt đất $2m$. Có một chiếc cột cao $1m$ đặt vuông góc với mặt đất, khoảng cách từ chân cột đến mỗi bức tường là $1$m. Người ta muốn căng một chiếc bạt phẳng hình tam giác đi qua điểm $A$ và đầu cột, hai đầu mút $M$, $N$ thuộc hai chân tường sao cho diện tích bạt bé nhất. Hỏi phải căng chiếc bạt hợp với mặt đất góc bao nhiêu độ ( Kết quả làm tròn đến hàng phần chục).\\
	\shortans[oly]{$54{,}7$}
	\loigiai{
		\immini{	Đặt hệ trục $Oxyz$ như hình vẽ.\\
			Ta có $A(0;0;2)$, $I(1;1;1)$. Gọi $M(m;0;0)$; $N(0;n;0)$ với $m>0$, $n>0$.\\
			Phương trình mặt phẳng $(AMN)\colon\dfrac{x}{m}+\dfrac{y}{n}+\dfrac{z}{2}=1$.\\
			Vì mặt phẳng $\left( AMN \right)$ đi qua điểm $I(1;1;1)$ nên ta có $\dfrac{1}{m}+\dfrac{1}{n}+\dfrac{1}{2}=1\Rightarrow \dfrac{1}{m}+\dfrac{1}{n}=\dfrac{1}{2}$.\\
			Ta có mặt phẳng $\left(AMN\right)$ có véc tơ pháp tuyến ${\overrightarrow{n}}_{\left( AMN \right)}=\left( \dfrac{1}{m};\dfrac{1}{n};\dfrac{1}{2} \right)$,\\mặt phẳng $\left(OMN\right)$ có véc tơ pháp tuyến${{\overrightarrow{n}}_{\left( OMN \right)}}=\left( 0;0;1 \right)$.\\$\Rightarrow \cos\left( \left( AMN \right),\left( OMN \right) \right)=\dfrac{\dfrac{1}{2}}{\sqrt{\dfrac{1}{m^2}+\dfrac{1}{n^2}+\dfrac{1}{4}}}$.}{	\begin{tikzpicture}[scale=0.7, font=\footnotesize,line join=round, line cap=round, >=stealth]
				\path
				(0:0) coordinate (D)
				(0:4) coordinate (C)
				(-135:4) coordinate (A)
				+(C) coordinate (B)
				\foreach \x in {A,B,C,D}{(\x)+(90:5.5) coordinate (\x')}
				($(D)!0.8!(C)$) coordinate (N)
				($(D)!0.5!(C)$) coordinate (T1)
				($(D)!0.8!(A)$) coordinate (M)
				($(D)!0.5!(A)$) coordinate (T2)+(T1) coordinate (E)
				(E)+(90:2.5) coordinate (I) (D)+(90:4) coordinate (T3)
				;
				\foreach \x/\g in {M/135,N/90,I/90}\draw[fill=white] (\x) circle (.03) +(\g:.4) node{$\x$};
				\draw (T1)--(E)--(T2) (I)--(E);
				\draw [->](D)--(C);
				\draw [->](D)--(A);
				\draw [->](D)--(D');
				\draw[right] (C) node{$y$} (D') node{$z$} (A) node{$x$} ;
				\draw[fill=white] (T3) circle (.03) +(180:.4) node{$A$};
		\end{tikzpicture}			}
		Mà ${S_{AMN}}\cdot \cos\left( \left( AMN \right),\left( OMN \right) \right)={S_{OMN}}\Rightarrow {S_{AMN}}=mn\sqrt{\dfrac{1}{m^2}+\dfrac{1}{n^2}+\dfrac{1}{4}}=\dfrac{\sqrt{\dfrac{1}{m^2}+\dfrac{1}{n^2}+\dfrac{1}{4}}}{\dfrac{1}{m}\cdot\dfrac{1}{n}}$.
		Đặt $\dfrac{1}{m}=a$, $\dfrac{1}{n}=b$ thì $a+b=\dfrac{1}{2}$ và ${S_{AMN}}=\sqrt{\dfrac{a^2+b^2+\dfrac{1}{4}}{a^2b^2}}=\sqrt{\dfrac{2a^2+2b^2+2ab}{a^2{b^2}}}$.\\
		Theo bất đẳng thức Cosi ta có $2a^2+2b^2+2ab\ge 3\sqrt[3]{8a^3{b^3}}=6ab$.\\
		Suy ra ${S_{AMN}}\ge \sqrt{\dfrac{6}{ab}}$.\\ Mà $a+b\ge 2\sqrt{ab}\Rightarrow \dfrac{1}{2}\ge 2\sqrt{ab}\Rightarrow ab\le \dfrac{1}{16}$ nên ${S_{AMN}}\ge \sqrt{96}$.\\
		Dấu bằng xảy ra khi $a=b=\dfrac{1}{4}$.\\ Thay vào (*) tc có $\cos\left( \left( AMN \right),\left( OMN \right) \right)=\dfrac{1}{\sqrt{3}}$ nên $\widehat{\left( \left( AMN \right),\left( OMN \right) \right)}\approx 54,7^\circ$\\
	}
\end{ex}

\begin{ex}%[2H5H2-7]
	Trong không gian, tìm $m$ để số đo góc giữa hai đường thẳng $d_1$, $d_2$ bằng $60^\circ$ biết $d_1\colon\heva{
		& x=1+t \\
		& y=1-t \\
		& z=-3+\sqrt{2}t \\
	}$, $d_2 \colon\heva{
		& x=2+mt \\
		& y=3+t \\
		& z=\sqrt{2}t \\
	}$.\\
	\shortans[oly]{$1$}
	\loigiai{
		Ta có $\cos \alpha =\cos 60^\circ=\dfrac{\left| 1\cdot m-1\cdot1+2 \right|}{\sqrt{1+1+2}\sqrt{m^2+1+2}}\Leftrightarrow \left| m-1 \right|=\sqrt{m^2+1}\Leftrightarrow m=1$.}
\end{ex}

\centerline{---HẾT---}
\Closesolutionfile{ans}
%\newpage
%%=====================
%\begin{center}
%\textbf{\large BẢNG ĐÁP ÁN}
%\end{center}
%\noindent\textbf{ĐÁP ÁN PHẦN I}
%\inputansbox{10}{ans/B3-De2-1}
	
%\noindent\textbf{ĐÁP ÁN PHẦN II}
%\inputansbox[2]{2}{ans/B3-De2-2}
	
%\noindent\textbf{ĐÁP ÁN PHẦN III}
%\inputansbox[3]{6}{ans/B3-De2-3}




%%Bài 4.
\setcounter{dang}{0}
\newpage
\section{PHƯƠNG TRÌNH MẶT CẦU}
\subsection{LÝ THUYẾT CẦN NHỚ}
\subsubsection{Định nghĩa}
\begin{itemize}
	\immini{	\item [\iconCH] Trong không gian, tập hợp tất cả các điểm $M$  cách điểm $I$ cố định một khoảng không đổi $r$ $(r>0)$  cho trước được gọi là mặt cầu tâm $I$ bán kính $R$. Kí hiệu $S(I;r)$ hay viết tắt là $(S)$. Vậy $S(I;R)=\{M|IM=r\}.$
		\item [\iconCH] Nhận xét: 
		\begin{itemize}
			\item Nếu $IM=r$ thì $M$ nằm trên mặt cầu.
			\item 	Nếu $IM<r$ thì $M$ nằm trong mặt cầu.
			\item 	Nếu $I M>r$ thì $M$ nằm ngoài mặt cầu.
		\end{itemize}
	}{
		\begin{tikzpicture}
			\draw[fill=black] (0,0) circle(1pt) node[left]{$I$};
			\draw (0,0) circle(1.5);
			\draw (-1.5,0)..controls (-1.45,-0.8) and (1.45,-0.8)..(1.5,0);
			\draw[dashed] (-1.5,0)..controls (-1.45,0.8) and (1.45,0.8)..(1.5,0);
			\draw[dashed] (0,0)--(1,-0.45);
			\draw[fill=black] (1,-0.45) circle(1pt) node[below]{$M$};
			\node at (0.6,0) {$r$};
	\end{tikzpicture}}
\end{itemize}
\subsubsection{Phương trình mặt cầu}
\begin{itemize}
	\item [\iconCH] Trong không gian $Oxyz$, mặt cầu $(S)$ tâm $I(a;b;c)$ bán kính $r$ có phương trình là $$(x-a)^2+(y-b)^2+(z-c)^2=r^2.$$
	\item [\iconCH] Dạng khai triển: $x^2+y^2+z^2-2ax-2by-2cz+d=0, \text{ với } d=a^2+b^2+c^2-r^2>0.$
\end{itemize}
\subsection{PHÂN LOẠI, PHƯƠNG PHÁP GIẢI TOÁN}
\begin{dang}{Xác định tâm $I$, bán kính $r$ của mặt cầu cho trước}
	\begin{itemize}
		\item [\iconCH] \indamm{Loại 1.} Cho $(S) \colon (x-a)^2+(y-b)^2+(z-c)^2=r^2$ . Khi đó
		\begin{listEX}[1]
			\item [\ding{172}] Tâm $I\left(a;b;c\right)$ (đổi dấu số trong dấu ngoặc);
			\item [\ding{173}] Bán kính $r$ (Rút căn vế phải).
		\end{listEX}
		\item [\iconCH] \indamm{Loại 2.} Cho $(S)\colon x^2+y^2+z^2-2ax-2by-2cz+d=0$. Khi đó
		\begin{listEX}[1]
			\item [\ding{172}] Điều kiện để (*) là mặt cầu là $a^2+b^2+c^2-d > 0$;
			\item [\ding{173}] Tâm $I\left(a,b,c\right)$ (đổi dấu hệ số của $x$, $y$, $z$ và chia đôi);
			\item [\ding{174}]  Bán kính $R=\sqrt{a^2+b^2+c^2-d}$ .
		\end{listEX}
	\end{itemize}
\end{dang}
\boxmini{BÀI TẬP TỰ LUẬN}
\setcounter{vd}{0}

\begin{vd}
	Trong các phương trình sau, phương trình nào là phương trình mặt cầu? Hãy xác định tâm và bán kính (nếu là phương trình mặt cầu).
	\begin{enumEX}[a)]{2}
		\item $(x-2)^2 + y^2 + (z+1)^2 = 4$.
		\item $x^2+y^2+z^2-2x-4y+6z-2=0$.
		\item $x^2 + y^2 + z^2 - 2x + 4y + 3z + 8 = 0$.
		\item $3x^2+3y^2+3z^2+6x+12y-9z+1=0$
	\end{enumEX}
	\loigiai{
		\begin{enumEX}[a)]{1}
			\item 
			\item Mặt cầu $(S)$ có tâm $I(1;2;-3)$ và bán kính $R=\sqrt{1^2+2^2+(-3)^2+2}=4$
			\item 
			\item Ta có 
			\begin{eqnarray*}
				&&3x^2+3y^2+3z^2+6x+12y-9z+1=0 \\
				&\Leftrightarrow& x^2+y^2+z^2+2x+4y-3z+\dfrac{1}{3}=0\\
				&\Leftrightarrow& (x^2 + 2x) + (y^2 +2\cdot2\cdot y) + \left(z^2-2\cdot\dfrac{3}{2}\cdot z\right) = \dfrac{-1}{3} \\
				&\Leftrightarrow& (x + 1)^2 + (y + 2)^2 + \left(z-\dfrac{3}{2}\right)^2 = 1+4+\dfrac{9}{4}-\dfrac{1}{3}\\
				&\Leftrightarrow& (x + 1)^2 + (y + 2)^2 + \left(z-\dfrac{3}{2}\right)^2 = \dfrac{83}{12}.
			\end{eqnarray*}
			Vậy đây là phương trình mặt cầu $(S)$ tâm $I\left(-1;-2;\dfrac{3}{2}\right)$, bán kính $r=\sqrt{\dfrac{83}{12}}=\dfrac{\sqrt{249}}{6}$.
		\end{enumEX}
	}
\end{vd}

\begin{vd}
	Trong không gian $Oxyz$, tìm tất cả giá trị của tham số $m$ để các phương trình sau là phương trình mặt cầu.
	\begin{enumEX}[a)]{1}
		\item  $x^2 + y^2 + z^2 - 2(m + 2)x + 4my - 2mz + 5m^2 + 9 = 0$;
		\item  ${x^{2}+y^{2}+z^{2}+2(m+2) x-2(m-1) z+3 m^{2}-5=0}$.
	\end{enumEX}
	\loigiai{
		\begin{enumEX}[a)]{2}
			\item Gọi phương trình đã cho có dạng $x^2 + y^2 + z^2 - 2ax - 2by - 2cz + d = 0$ với $a = m + 2$, $b = -2m$, $c = m$, $d = 5m^2 + 9$.\\
			Để phương trình đã cho là phương trình mặt cầu thì
			$$a^2 + b^2 + c^2 - d > 0 \Leftrightarrow m^2 + 4m + 4 + 4m^2 + m^2 - 5m^2 - 9 > 0 \Leftrightarrow m^2 + 4m - 5 > 0 \Leftrightarrow \hoac{&m < -5\\&m > 1.}$$
			\item Phương trình đã cho là phương trình của một mặt cầu khi và chỉ khi
			\[(m+2)^{2}+(m-1)^{2}-3 m^{2}+5>0 \Leftrightarrow m^{2}-2 m-10<0 \Leftrightarrow 1-\sqrt{11}<m<1+\sqrt{11}.\]
			Do $m \in \mathbb{Z}$ nên $m \in\{-2 ;-1 ; 0 ; 1 ; 2 ; 3 ; 4\}$. Vậy có $7$ giá trị nguyên của $m$ thỏa yêu cầu bài toán.
		\end{enumEX}
	}
\end{vd}

\dongcham{10}
\boxmini{BÀI TẬP TRẮC NGHIỆM}
\setcounter{ex}{0}
\Opensolutionfile{ans}[ans/2H5-B4-d1]

\begin{ex}
	Cho mặt cầu $(S)\colon (x+1)^2+(y-2)^2+(z-1)^2=9$. Tìm tọa độ tâm $I$ và tính bán kính $R$ của $(S)$.
	\choice
	{$I(1;-2;-1)$ và $R=3$}
	{$I(1;-2;-1)$ và $R=9$}
	{\True $I(-1;2;1)$ và $R=3$}
	{$I(-1;2;1)$ và $R=9$}
	\loigiai{
		Ta có mặt cầu $(S)$ có tâm $I(-1;2;1)$ và bán kính $R=3$.
	}
\end{ex}

\begin{ex}%[2H3Y1-3]%
	Cho mặt cầu $(S)\colon (x-1)^2+(y+2)^2+z^2=9$. Mặt cầu $(S)$ có thể tích bằng
	\choice
	{\True $V=36\pi$}
	{$V=14\pi$}
	{$V=\dfrac{4}{36}\pi$}
	{$V=16\pi$}
	\loigiai{
		Mặt cầu $(S)\colon (x-1)^2+(y+2)^2+z^2=9$ có tâm là $(1;-2;0)$, bán kính $R=3$.\\
		Thể tích mặt cầu $V=\dfrac{4}{3}\pi R^3=36\pi$.}
\end{ex}

\begin{ex}
	Cho mặt cầu $(S)\colon x^2+y^2+z^2-4x-6y+8z-7=0$. Tọa độ tâm và bán kính mặt cầu $(S)$ lần lượt là
	\choice
	{$I(-2;-3;4)$, $R=6$}
	{$I(-2;-3;4)$, $R=36$}
	{$I(2;3;-4)$, $R=36$}
	{\True $I(2;3;-4)$, $R=6$}
	\loigiai{
		Ta có
		$$x^2+y^2+z^2-4x-6y+8z-7=0\Leftrightarrow(x-2)^2+(y-3)^2+(z+4)^2=36.$$
		Nên mặt cầu $(S)$ có tâm $I(2;3;-4)$ và bán kính $R=6$.}
\end{ex}

\begin{ex}
	Cho mặt cầu $(S)\colon x^2+y^2+z^2-8x+2y+1=0$. Tìm tọa độ tâm và bán kính của mặt cầu $(S)$.
	\choice
	{\True $I(4;-1;0)$, $R=4$}
	{$I(-4;1;0)$, $R=4$}
	{$I(-4;1;0)$, $R=2$}
	{$I(4;-1;0)$, $R=2$}
	\loigiai{
		Mặt cầu $(S)$ có tâm $I(4;-1;0)$ và bán kính $R=\sqrt{4^2+(-1)^2+0^2-1}=4$.
	}
\end{ex}

\begin{ex} %[Word to LaTeX 3.2]
	Cho mặt cầu $(S)\colon 2x^2+2y^2+2z^2+12x-4y+4=0$. Mặt cầu $(S)$ có đường kính $AB$. Biết điểm $A(-1;-1;0)$ thuộc mặt cầu $(S)$. Tọa độ điểm $B$ là
	\choice
	{$B(-5;3;-2)$}
	{$B(-11;5;0)$}
	{$B(-11;5;-4)$}
	{\True$B(-5;3;0)$}
	\loigiai{
		\begin{itemize}
			\item [$\bullet$] Viết lại phương trình $(S) \colon x^2+y^2+z^2+6x-2y+2=0$. Khi đó tâm của mặt cầu là $I(-3;1;0)$.
			\item [$\bullet$] Vì $AB$ là đường kính nên $I$ là trung điểm của $AB$, suy ra $B(-5;3;0)$.
	\end{itemize}}
\end{ex}

\begin{ex}%[Thi HK2, Sở GD Bình Dương, 2018]%[2H3Y1-3]%[Trần Bá Huy, 12EX-8-2018]
	Phương trình nào dưới đây là phương trình mặt cầu?
	\choice
	{$x^2+y^2-z^2+4x-2y+6z+5=0$}
	{$x^2+y^2+z^2+4x-2y+6z+15=0$}
	{\True $x^2+y^2+z^2+4x-2y+z-1=0$}
	{$x^2+y^2+z^2-2x+2xy+6z-5=0$}
	\loigiai{
		Phương trình $x^2+y^2+z^2+4x-2y+z-1=0$ là phương trình mặt cầu vì có dạng là $x^2+y^2+z^2-2ax-2by-2cz+d=0$ và thỏa $a^2+b^2+c^2-d>0$ (dễ nhận biết vì $d=-1<0$).
	}
\end{ex}

\begin{ex}%[2H3K1-3]
	Cho phương trình $x^2+y^2+z^2-2mx-2(m+2)y-2(m+3)z+16m+13=0$. Tìm tất cả các giá trị thực của $m$ để phương trình trên là phương trình của một mặt cầu.
	\choice 
	{\True $m<0$ hay $m>2$}
	{$m \leq -2$ hay $m \geq 0$}
	{$m<-2$ hay $m>0$}
	{$m \leq 0$ hay $m \geq 2$}  
	\loigiai{ 
		Phương trình đã cho là phương trình của một mặt cầu khi và chỉ khi 
		$$\begin{aligned} &\, m^2+(m+2)^2+(m+3)^2-16m-13>0 \\ 
			\Leftrightarrow &\, 3m^3-6m>0\\
			\Leftrightarrow &\, \left[\begin{aligned} &m<0\\ &m>2  \\   \end{aligned}. \right. 
		\end{aligned}$$ 
	}
\end{ex}

\begin{ex}%[TT, TTLTĐH Diệu Hiền, Cần Thơ tháng 10, 2017]%[Thọ Bùi, dự án 12EX6]%[2H3B1-3]%
	Có tất cả bao nhiêu giá trị của tham số $m$ (biết $m \in \mathbb{N}$) để phương trình $x^2 + y^2 + z^2 + 2(m-2)y - 2(m+3)z + 3m^2 + 7 = 0$ là phương trình của một mặt cầu?
	\choice
	{$2$}
	{$3$}
	{\True $4$}
	{$5$}
	\loigiai
	{
		Đồng nhất hệ số của phương trình $x^2 + y^2 + z^2 + 2(m-2)y - 2(m+3)z + 3m^2 + 7 = 0$ (*) với phương trình $x^2 + y^2 + z^2 - 2ax - 2by - 2cz + d = 0$ ta được $a = 0$, $b = 2 - m$, $c = m + 3$ và $d = 3m^2 + 7$.\\
		Phương trình (*) là phương trình của một mặt cầu khi
		\begin{align*}
			a^2 + b^2 + c^2 - d > 0 & \Leftrightarrow (2-m)^2 + (m+3)^2 - (3m^2 + 7) > 0 \\
			& \Leftrightarrow -m^2 + 2m + 6 > 0 \\
			& \Leftrightarrow 1 - \sqrt{7} < m < 1 + \sqrt{7}.
		\end{align*}
		Do $1 - \sqrt{7} < m < 1 + \sqrt{7}$ và $m \in \mathbb{N}$ nên $m \in \{ 0; 1; 2; 3 \}$.
	}
\end{ex}

\begin{ex}%[HK2 (2017-2018), THPT LÊ QUÝ ĐÔN, HÀ NỘI]%[Trần Hòa, dự án EX9]%[2H3B1-3]
	Cho mặt cầu $(S)\colon x^2 + y^2+ z^2 -2x - 4y + 4z - m = 0$ ($m$ là tham số ). Biết mặt cầu có  bán kính bằng $5$. Tìm $m$.
	\choice%36
	{$m=25$}
	{$m=11$}
	{\True $m=16$}
	{$m=-16$}
	\loigiai{
		\begin{itemize}
			\item [$\bullet$] Công thức bán kính mặt cầu là $R=\sqrt{a^2+b^2+c^2-d}=\sqrt{1+4+4+m}$.
			\item [$\bullet$] Theo giả thiết $R=5\Leftrightarrow \sqrt{1+4+4+m}=5\Leftrightarrow m=16$.
		\end{itemize}
		
	}
\end{ex}

\begin{ex} 
	Mặt cầu $(S):x^2+y^2+z^2-4mx+4y+2mz+m^2+4m=0$ có bán kính nhỏ nhất khi $m$ bằng
	\choice
	{\True$\dfrac{1}{2}$}
	{$\dfrac{1}{3}$}
	{$\dfrac{\sqrt{3}}{2}$}
	{$0$}
	\loigiai{
		\begin{itemize}
			\item [$\bullet$] Công thức bán kính mặt cầu là
			\begin{eqnarray*}
				R &=&\sqrt{a^2+b^2+c^2-d}\\
				&=& \sqrt{4m^2+4+m^2-(m^2+4m)}\\
				&=&\sqrt{4m^2-4m+4}\\
				&=& \sqrt{(2m-1)^2+3}\quad(1).
			\end{eqnarray*} 
			\item [$\bullet$] Biểu thức (1) đạt giá trị nhỏ nhất khi $m=\dfrac{1}{2}$.
	\end{itemize}}
\end{ex}
\Closesolutionfile{ans}

\begin{dang}{Lập phương trình mặt cầu và ứng dụng thực tiễn}
	\begin{itemize}
		\item [\iconCH] \indamm{Phương pháp chung:} Cần xác định được tọa độ tâm $I\left(a;b;c\right)$ và độ dài bán kính $r$.
		\item [\iconCH] \indamm{Các bài toán cơ bản:}
		\begin{listEX}[1]
			\item [\ding{172}] Mặt cầu có tâm $I\left(a;b;c\right)$ và đi qua điểm $A\left(x_A;{y_A};{z_A}\right)$ thì bán kính $$r=IA=\sqrt{\left(x_A-x_I\right)^2+\left(y_A-y_I\right)^2+\left(z_A-z_I\right)^2}.$$
			\item [\ding{173}] Mặt cầu (S) có đường kính $AB$ thì
			\begin{itemize}
				\item [$\bullet$] Tâm $I\left(a;b;c\right)$ là trung điểm của $AB$ hay $I\left(\dfrac{x_A+x_B}{2};\dfrac{y_A+y_B}{2};\dfrac{z_A+z_B}{2}\right)$.
				\item [$\bullet$] Bán kính $r=\dfrac{AB}{2}=\dfrac{\sqrt{\left(x_B-x_A\right)^2+\left(y_B-y_A\right)^2+\left(z_B-z_A\right)^2}}{2}$.
			\end{itemize}
			\item [\ding{174}] Mặt cầu có tâm $I(a;b;c)$ và tiếp xúc với $(\alpha) \colon Ax+By+Cz+D=0$ thì bán kính $$r=\mathrm{d}\left(I,(\alpha) \right)= \dfrac{\big|Aa+Bb+Cc+D\big|}{\sqrt{A^2+B^2+C^2}}.$$
			\item [\ding{175}] Mặt cầu qua bốn điểm $A$, $B$, $C$, $D$ không đồng phẳng (ngoại tiếp tứ diện $ABCD$)\\
			Gọi $(S)$ có dạng $x^2+y^2+z^2-2ax-2by-2cz+d=0$ (*)\\
			Thay tọa độ 4 điểm $A$, $B$, $C$, $D$ vào (*), ta được hệ phương trình 4 ẩn số $a$, $b$, $c$, $d$;\\
			Giải tìm $a$, $b$, $c$, $d$. Suy ra tâm $I\left(a,b,c\right)$ , bán kính $R=\sqrt{a^2+b^2+c^2-d}$.
		\end{listEX}
	\end{itemize}
\end{dang}
\boxmini{BÀI TẬP TỰ LUẬN}
\setcounter{vd}{0}

\begin{vd}
	Trong không gian $Oxyz$, viết phương trình mặt cầu $(S)$
	\begin{enumEX}[a)]{1}
		\item Có tâm $I(2;-1;0)$ và đi qua điểm $M(4;1;-2)$;
		\item Có đường kính $AB$ với $A(0;1;3)$, $B(4;-5;-1)$;
		\item Có tâm $I(1;-2;3)$ và tiếp xúc với trục $Oy$;
		\item Có tâm $I(1;2;-1)$ và tiếp xúc với $(P)\colon x-2y-2z-8=0$.
	\end{enumEX}
	\loigiai{
		\begin{itemize}
			\item[a)] Bán kính mặt cầu là $r=IM=\sqrt{(4-2)^2+(1+1)^2+(-2-0)^2}=\sqrt{12}$.
			Phương trình mặt cầu tâm $I(2;-1;0)$, bán kính $r=\sqrt{12}$ là
			$$(x-2)^2+(y+1)^2+z^2=12.$$
			\item[b)] Tâm của mặt cầu $(S)$ là trung điểm $I$ của đoạn thẳng $AB$, suy ra $I(2;-2;1)$. Bán kính mặt cầu $(S)$ là $R=\dfrac{AB}{2}=\dfrac{\sqrt{4^2+(-6)^2+(-4)^2}}{2}=\sqrt{17}$. Vậy phương trình mặt cầu $(S)$ là
			$$(x-2)^2+(y+2)^2+(z-1)^2=17. $$
			\item[c)] Gọi $M$ là hình chiếu của $I\left( 1;-2;3 \right)$ lên $Oy$, ta có: $M\left( 0;-2;0 \right)$.\\
			$\overrightarrow{IM}=\left( -1;0;-3 \right)\Rightarrow R=d\left( I,Oy \right)=IM=\sqrt{10}$ là bán kính mặt cầu cần tìm.\\
			Phương trình mặt cầu là: ${{\left( x-1 \right)}^{2}}+{{\left( y+2 \right)}^{2}}+{{\left( z-3 \right)}^{2}}=10.$
			\item [d)] 
			Mặt cầu có tâm $I(1;2;-1)$ và tiếp xúc với $(P)\colon x-2y-2z-8=0$ sẽ có bán kính là
			$$R= \mathrm{d}(I,(P))
			=\dfrac{|1-2\cdot 2 -2 \cdot (-1) -8|}{\sqrt{1^2+(-2)^2+(-2)^2}}
			=3.$$
			Vậy phương trình mặt cầu là $(x-1)^2+(y-2)^2+(z+1)^2=9$.
		\end{itemize}
	}
\end{vd}
\dongcham{22}
\begin{vd}
Viết phương trình mặt cầu ngoại tiếp tứ diện $ABCD$, biết
\begin{enumEX}[a)]{1}
	\item $A(1 ; 1 ; 0)$, $B(1 ; 0 ; 1)$, $C(0 ; 1 ; 1)$, $D(1 ; 2 ; 3)$.
	\item $A (1; 2; -4)$; $B (1; -3; 1)$, $C (2; 2; 3)$, $D (1; 0; 4)$.
\end{enumEX}
\loigiai{
	\begin{enumEX}[a)]{1}
		\item Giả sử phương trình mặt cầu có dạng: $(S): x^2+y^2+z^2-2 a x-2 b y-2 c z+d=0$ + Với $a^2+b^2+c^2-d>0$, ta có $\mathrm{A}(1 ; 1 ; 0), \mathrm{B}(1 ; 0 ; 1), \mathrm{C}(0 ; 1 ; 1), \mathrm{D}(1 ; 2 ; 3) \in(S)$ :
		$\Rightarrow\left\{\begin{array}{l}1+1+0-a-2 b-0+d=0 \\ 1+0+1-2 a-0-2 c+d=0 \\ 0+1+1-0-2 b-2 c+d=0 \\ 1+2^2+3^2-2 a-4 b-6 c+d=0\end{array}\right.$
		$\Leftrightarrow\left\{\begin{array}{l}-2 a-2 b+d=-2 \\ -2 c-2 c+d=-2 \\ -2 b-2 c+d=-2 \\ -2 a-4 b-6 c+d=-14\end{array} \Leftrightarrow\left\{\begin{array}{l}a=\frac{3}{2} \\ b=\frac{3}{2} \\ c=\frac{3}{2} \\ d=4\end{array}\right.\right.$
		
		$$
		\begin{aligned}
			& \Rightarrow(S): x^2+y^2+z^2-2 \cdot \frac{3}{2} x-2 \cdot \frac{3}{2} y-2 \cdot \frac{3}{2} z+4=0 \\
			& \Leftrightarrow x^2+y^2+z^2-3 x-3 y-3 z+4=0 .
		\end{aligned}
		$$
		
		\item Gọi phương trình mặt cầu $(S)$ :
		
		$$
		x^2+y^2+z^2-2 a x-2 b y-2 c z+d=0\left(a^2+b^2+c^2-d>0\right)
		$$
		
		
		Do mặt cầu đi qua 4 điểm $A, B, C, D$ nên tọa độ của 4 điểm thỏa mãn phương trình mặt cầu
		
		$$
		\begin{aligned}
			& \Leftrightarrow\left\{\begin{array}{l}
				-2 a-4 b+8 c+d=-21 \\
				-2 a+6 b-2 c+d=-11 \\
				-4 a-4 b-6 c+d=-17 \\
				-2 a-8 c+d=-17
			\end{array}\right. \\
			& \Leftrightarrow\left\{\begin{array}{c}
				a=-2 \\
				b=1 \\
				c=0 \\
				d=-21
			\end{array}\right.
		\end{aligned}
		$$
		
		
		Vậy phương trình mặt cầu cần tìm là
		
		$$
		x^2+y^2+z^2+4 x-2 y-21=0
		$$
		
	\end{enumEX}
}
\end{vd}
\dongcham{17}
\begin{vd}%GV:Phan Phú Quý
	Giả sử người ta biểu diễn mô phỏng của tòa nhà Ericsson Globe ở phần Khởi động trong hệ trục tọa độ $Oxyz$ bởi một mặt cầu có tâm $I$, đường kính $110$ m và $OA=85$ m như hình vẽ (đơn vị trên trục là mét). Hãy viết phương trình của mặt cầu này.\\
	\includegraphics[width=7cm]{image/2P5-B4-PhuQuy-H2}
	%\includegraphics[width=7cm]{image/5372.pdf}\\
	\begin{tikzpicture}[scale=0.5]
		\def\r{5}
		\pgfmathsetmacro\a{\r *sin(60)}
		\pgfmathsetmacro\h{\r *cos(60)}
		\def\b{1}
		%		\draw[step=1,gray,very thin]
		%		(-7,-7) grid (7,5);
		\path
		(0,0) coordinate (O)
		(O)++(90:\h) coordinate (I)
		(I)++(90:\r) coordinate (A)
		(O)++(0:\a) coordinate (M)
		(O)++(180:\a) coordinate (N);
		\draw (M) arc (-30:210:\r);
		\draw[dashed] (M) arc (0:180:\a cm and \b cm);
		\draw (M) arc (0:-180:\a cm and \b cm);
		\coordinate (B) at ($(O)+({\a*cos(-120)},{\b*sin(-120)})$);
		\coordinate (R) at ($(I)+({\r*cos(-165)},{\r*sin(-165)})$);
		\coordinate (S) at ($(I)+({\r*cos(-15)},{\r*sin(-15)})$);
		\foreach \i/\j in {O/-50,I/180,A/150}
		\draw[fill=black] (\i) node[shift=(\j:.32)]{$\i$}circle (1.2pt);
		\draw[dashed] (O)--(B) (O)--(M) (O)--(A) (R)--(S);
		\draw[->] (B)--($ (B)+({1.7*\a*cos(-120)},{1.7*\b*sin(-120)}) $) node[below right]{$ x $};
		\draw[->] (M)--($ (M)+(0:2.5) $) node[below right]{$ y $};
		\draw[->] (A)--($ (A)+(90:2) $) node[right]{$ z $};
		\draw (-8,-2)--(5,-2)--($ (S)+(0:2) $)--(S)
		(R)--($ (R)+(180:1) $)--(-8,-2);
		\draw (-5.5,-0.5) node[below]{$ \text{Mặt đất} $};
	\end{tikzpicture}
	% \centering{\textit{Hình 5.37}}
	\loigiai{
		\begin{itemize}
			\item Bán kính của mặt cầu tâm $I$ là $R=IA=\dfrac{110}{2}=55$ m.
			\item Ta có $OA=OI+IA\Rightarrow OI=OA-IA=85-55=30$ m.\\
			Vì $I\in Oz$ nên toạ độ điểm $I(0;0;30)$.
			\item Phương trình mặt cầu tâm $I(0;0;30)$ có bán kính $R=55$ m là
			$$ x^2+y^2+(z-30)^2=55^2 \text{ hay } x^2+y^2+(z-30)^2=3025.  $$ 
		\end{itemize}
	}
\end{vd}
\dongcham{7}

\begin{vd}
	Bạn Bình đố bạn Nam tìm được đường kính của quả bóng rổ, biết rằng nếu đặt quả bóng ở một góc căn phòng hình hộp chữ nhật, sao cho quả bóng chạm (tiếp xúc) với hai bức tường và nền nhà của căn phòng đó (khi đó khoảng cách từ tâm quả bóng đến hai bức tường và nền nhà đều bằng bán kính của quả bóng) thì có một điểm $M$ trên quả bóng với khoảng cách lần lượt đến hai bức tường và nền nhà là $17$ cm, $18$ cm và $21$ cm (Hình bên dưới). Hãy giúp Nam xác định đường kính của quả bóng rổ đó. Biết rằng loại bóng rổ tiêu chuẩn có đường kính từ $23$ cm đến $24{,}5$ cm.
	\begin{center}
		\includegraphics[height=4cm]{image/2P5-B4-TheUt-H3}
	\end{center}
	\loigiai{
		\immini{Xét quả bóng tiếp xúc với các bức tường và chọn hệ trục $ Oxyz $ như hình vẽ bên.\\
			Gọi $ I(a;a;a) $ là tâm của mặt cầu và $ r=a>0 $.\\
			Phương trình mặt cầu của quả bóng là $$ (S)\colon (x-a)^2+(y-a)^2+(z-a)^2=a^2 .$$
			Giả sử $ M(x;y;z) $ nằm trên mặt cầu (bề mặt của quả bóng) sao cho $ \mathrm{d}\left(M,(Oxy) \right)=21  $, $ \mathrm{d}\left(M,(Oxz) \right)=18  $, $ \mathrm{d}\left(M,(Oyz) \right)=17  $. Khi đó $ z=21, y=18, x=17 $. Khi đó ta có phương trình
			\begin{eqnarray*}
				&&(17-a)^2+(18-a)^2+(21-a)^2=a^2\\
				&\Leftrightarrow & 2a^2-112a+1054=0\\
				&\Leftrightarrow & \hoac{&a\approx 11{,}97(\text{nhận})\\&a\approx 44{,}03(\text{loại})}
			\end{eqnarray*}
			Vậy đường kính của quả bóng rổ là $ 2a\approx 23{,}94 $ cm.
		}{
			\begin{tikzpicture}[font=\fontsize{8}{8},scale=1.2]
				\def\a{-135} \def\ra{1.8}
				\def\b{-13} \def\rb{2.4}
				\def\c{90}	\def\rc{2.4}
				\def\m{40}
				\def\rM{1.6}
				\path 
				(0,0) coordinate (O)
				(\a:2.8) coordinate (A) (\a:\ra) coordinate (A')
				(\b:3.2) coordinate (B) (\b:\rb) coordinate (B')
				(\c:3.1) coordinate (C) (\c:\rc) coordinate (C')
				($(A')+(\b:\rb)$) coordinate (D)
				($(D)+(90:\rc)$) coordinate (M)
				($(M)+({180+\b}:\rb)$) coordinate (E)
				($(M)+({180+\a}:\ra)$) coordinate (F);
				\fill (O) circle(.6pt) node[shift={(138:6pt)}]{$O$};
				\draw (M)--(D)node[midway, right]{$ 21 $} (M)--(E)node[midway, below]{$ 18 $} (M)--(F)node[midway, below]{$ 17 $};
				\begin{scope}[-stealth,line width=.5pt]
					\draw (O)--(A) node[shift={(160:4pt)}]{$x$};
					\draw (O)--(B) node[above]{$y$};
					\draw (O)--(C) node[shift={(140:4pt)}]{$z$};
				\end{scope}
				\fill[red] (M) circle(1pt) node[right]{$M(x,y,z)$};
				\fill[blue] (E) circle(1pt) node[left]{$B$};
				\fill[magenta!70!blue] (F) circle(1pt) node[right]{$A$};
				\fill[cyan!50!blue] (D) circle(1pt) node[right]{$C$};
			\end{tikzpicture}
		}
	}
\end{vd}
\dongcham{8}
\boxmini{BÀI TẬP TRẮC NGHIỆM}
\setcounter{ex}{0}
\Opensolutionfile{ans}[ans/2H5-B4-d2]

\begin{ex}%[2H3Y1-3]
	Mặt cầu tâm $I(3;-1;0)$, bán kính $R=5$ có phương trình là
	\choice
	{$(x+3)^2+(y-1)^2+z^2=5$}
	{$(x-3)^2+(y+1)^2+z^2=5$}
	{\True $(x-3)^2+(y+1)^2+z^2=25$}
	{$(x+3)^2+(y-1)^2+z^2=25$}
	\loigiai{
		Mặt cầu tâm $I(3;-1;0)$, bán kính $R=5$ có phương trình là $(x-3)^2+(y+1)^2+z^2=25$.
	}
\end{ex}

\begin{ex}%[2H3B1-3]
	Phương trình mặt cầu tâm $I(2; -3; -4)$, bán kính bằng $4$ là
	\choice
	{$(x+2)^2+(y-3)^2+(z-4)^2=16$}
	{\True $(x-2)^2+(y+3)^2+(z+4)^2=16$}
	{$(x+2)^2+(y-3)^2+(z-4)^2=4$}
	{$(x-2)^2+(y+3)^2+(z+4)^2=4$}
	\loigiai
	{
	}
\end{ex}

\begin{ex}%[2H3B1-3]
	Viết phương trình mặt cầu $(S)$ có tâm $I(-1;1;-2)$ và đi qua điểm $A(2;;1;2)$.
	\choice%34
	{$(S)\colon (x-1)^2 +(y+1)^2 +(z-2)^2=5$}
	{$(S)\colon (x-2)^2 +(y-1)^2 +(z-2)^2=25$}
	{\True $(S)\colon (x+1)^2 +(y-1)^2 +(z+2)^2=25$}
	{$(S)\colon x^2 +y^2 +z^2 +2x -2y +4z + 1 = 0$}
	\loigiai{
		Bán kính mặt cầu là $R=IA=\sqrt{9+0+16}=5$.\\ Vậy phương trình mặt cầu là $(x+1)^2 +(y-1)^2 +(z+2)^2=25$.
	}
\end{ex}

\begin{ex}%[2H3B1-3]
	Mặt cầu tâm $I(-3; 0; 4)$ và đi qua điểm $A(-3; 0; 0)$ có phương trình là
	\choice
	{$(x-3)^2+y^2+(z+4)^2=4$}
	{$(x-3)^2+ y^2 + (z+4)^2=16$}
	{\True $(x+3)^2+y^2+(z-4)^2=16$}
	{$(x+3)^2+y^2+(z-4)^2=4$}
	\loigiai{
		Bán kính mặt cầu $R=IA=4.$}
\end{ex}

\begin{ex}%[2H3Y1-3]
	Phương trình mặt cầu $\left(S\right)$ đường kính $AB$ với $A\left(4; -3; 5\right)$, $B\left(2; 1; 3\right)$ là
	\choice
	{$x^2 + y^2 + z^2 + 6x + 2y - 8z - 26 = 0$}
	{\True $x^2 + y^2 + z^2 - 6x + 2y - 8z + 20 = 0$}
	{$x^2 + y^2 + z^2 + 6x -  2y + 8z - 20 = 0$}
	{$x^2 + y^2 + z^2 - 6x + 2y - 8z +  26 = 0$}
	\loigiai{ Ta có $AB = \sqrt{\left(2 - 4\right)^2 + \left(1 + 4\right)^2 + \left(3 - 5\right)^2} = 2\sqrt{6}$.\\
		Gọi $I$, $R$ là tâm và bán kính của mặt cầu $\left(S\right)$ suy ra $R = \dfrac{AB}{2} = \sqrt{6}$ và $I\left(3; - 1; 4\right)$.\\
		Khi đó phương trình mặt cầu $\left(S\right)$ là
		$$\left(x - 3\right)^2 + \left(y + 1\right)^2 + \left(z - 4\right)^2 = 6\Leftrightarrow x^2 + y^2 + z^2 - 6x + 2y - 8z + 20 = 0$$
	}
\end{ex}

\begin{ex}%[2H3B1-3]
	Cho hai điểm $A(2; 4; 1)$ và $B(-2; 2; -3)$. Phương trình mặt cầu đường kính $AB$ là
	\choice
	{$x^2+(y-3)^2+(z-1)^2=9$}
	{$x^2+(y+3)^2+(z-1)^2=9$}
	{$x^2+(y-3)^2+(z+1)^2=3$}
	{\True $x^2+(y-3)^2+(z+1)^2=9$}
	\loigiai{
		Mặt cầu đường kính $AB$ có tâm là trung điểm của đoạn thẳng $AB.$\\
		Suy ra tọa độ tâm mặt cầu là $I\left(0;3;-1\right).$ Bán kính mặt cầu: $R=\dfrac{AB}{2}=3.$}
\end{ex}



\begin{ex}%[2H3B1-3]
	Viết phương trình mặt cầu $(S)$ có tâm $I(-1;4;2)$, biết thể tích khối cầu tương ứng là $V=972\pi$.
	\choice
	{\True $(x+1)^2+(y-4)^2+(z-2)^2=81$}
	{$(x+1)^2+(y-4)^2+(z-2)^2=9$}
	{$(x-1)^2+(y+4)^2+(z-2)^2=9$}
	{$(x-1)^2+(y+4)^2+(z+2)^2=81$}
	\loigiai{
		Thể tích khối cầu $V=\dfrac{4}{3}\pi R^3=972\pi\Leftrightarrow R=9$.\\
		Phương trình mặt cầu $(S)\colon (x+1)^2+(y-4)^2+(z-2)^2=81$.
	}
\end{ex}

\begin{ex}%[2H3B1-3]
	Mặt cầu $(S)$ có tâm $I(2; 1; -1)$, tiếp xúc với mặt phẳng tọa độ $(Oyz)$. Phương trình của mặt cầu $(S)$ là
	\choice
	{$(x+2)^2+(y+1)^2+(z-1)^2=4$}
	{$(x-2)^2+(y-1)^2+(z+1)^2=1$}
	{\True $(x-2)^2+(y-1)^2+(z+1)^2=4$}
	{$(x+2)^2+(y-1)^2+(z+1)^2=2$}
	\loigiai{
		Bán kính mặt cầu: $R=d\left[I,\left(Oyz\right)\right]=\left|x_I\right|=2.$}
\end{ex}

\begin{ex}%[2H3B1-3]
	Mặt cầu có tâm $I(1; 2; -3)$ và tiếp xúc với trục $Oy$ có bán kính bằng
	\choice
	{$2$}
	{$\sqrt{5}$}
	{\True $\sqrt{10}$}
	{$\sqrt{13}$}
	\loigiai{
		Bán kính mặt cầu: $R=d\left[I,Oy\right]=\sqrt{x_I^2+z_I^2}=\sqrt{10}.$}
\end{ex}

\begin{ex}
	Trong không gian $Oxyz$, mặt cầu tâm $I(-1;0;3)$ tiếp xúc với mặt phẳng $(\alpha) \colon 4y-3z+19=0$ có phương trình là
	\choice
	{$(x-1)^2+y^2+(z+3)^2=4$}
	{$(x+1)^2+y^2+(z-3)^2=2$}
	{\True $(x+1)^2+y^2+(z-3)^2=4$}
	{$(x-1)^2+y^2+(z+3)^2=2$}
	\loigiai{
		Mặt cầu tiếp xúc với mặt phẳng $\Leftrightarrow R=\mathrm{d}\left(I,(\alpha)\right)=\dfrac{|-3 \cdot 3+19|}{\sqrt{4^2+(-3)^2}}=2$.\\
		Vậy phương trình mặt cầu là $(x+1)^2+y^2+(z-3)^2=4$.
	}
\end{ex}

\begin{ex}%[2H3K1-3]
	Viết phương trình mặt cầu $(S)$ đi qua $A(-1;2;0)$, $B(-2;1;1)$ và có tâm nằm trên trục $Oz$.
	\choice 
	{\True $x^2+y^2+z^2-z-5=0$}
	{$x^2+y^2+z^2+5=0$}
	{$x^2+y^2+z^2-x-5=0$}
	{$x^2+y^2+z^2-y-5=0$}  
	\loigiai{ 
		Tâm $I$ của mặt cầu trên trục $Oz$ có tọa độ $I(0;0;c)$.\\
		Hai điểm $A, B$ nằm trên mặt cầu nên
		$$\begin{aligned} &\,IA^2=IB^2\\ 
			\Leftrightarrow &\,1^2+2^2+c^2=2^2+1^2+(1-c)^2  \\   
			\Leftrightarrow &\,c=\frac{1}{2}.
		\end{aligned}$$ 
		Từ đó, phương trình mặt cầu là $x^2+y^2+\left(z-\dfrac{1}{2}\right)^2=1^2+2^2+\left(\dfrac{1}{2}\right)^2$ \\
		hay $x^2+y^2+z^2-z-5=0$.
	} 
\end{ex}


\begin{ex}%[HK2 (2017-2018), THPT LÊ QUÝ ĐÔN, HÀ NỘI]%[Trần Hòa, dự án EX9]%[2H3B1-3]
	Cho mặt cầu $(S)$ tâm $I$ nằm trên mặt phẳng $(Oxy)$ đi qua ba điểm $A(1;2;-4)$, $B(1;-3;1)$, $C(2;2;3)$. Tìm tọa độ điểm $I$.
	\choice%8
	{$I(2;-1;0)$}
	{$I(0;0;1)$}
	{$I(0;0;-2)$}
	{\True $I(-2;1;0)$}
	\loigiai{
		Vì $I\in (Oxy)\Rightarrow I(a;b;0)$. Ta  có $\overrightarrow{AI}=(a-1;b-2;4);\overrightarrow{BI}=(a-1;b+3;-1);\overrightarrow{CI}=(a-2;b-2;-3)$.\\
		Do $I$ là tâm cầu nên
		\begin{eqnarray*}
			& & \heva{&IA = IB\\&IA = IC}\\
			&\Leftrightarrow& 
			\heva{&(a-1)^2+(b-2)^2+4^2=(a-1)^2+(b+3)^2+1\\&(a-1)^2+(b-2)^2+4^2=(a-2)^2+(b-2)^2+9}\\
			&\Leftrightarrow& \heva{&-4b+20=6b+10\\&-2a+17=-4a+13}\\
			&\Leftrightarrow& \heva{&b=1\\&a=-2}\\
			&\Rightarrow& I(-2;1;0).
		\end{eqnarray*}
	}
\end{ex}


\begin{ex}%[HK2 (2017-2018), THPT Tân Hiệp, Kiên Giang]%[Bùi Mạnh Tiến, dự án (12EX-9)]%[2H3B1-3]
	Cho $3$ điểm $A(2;3;0)$, $B(0;-4;1)$, $C(3;1;1)$. Mặt cầu đi qua ba điểm $A,B,C$ và có tâm $I$ thuộc mặt phẳng $(Oxz)$, biết $I(a;b;c)$. Tính tổng $T=a+b+c$.
	\choice
	{$T=3$}
	{$T=-3$}
	{\True $T=-1$}
	{$T=2$}
	\loigiai{
		Gọi phương trình mặt cầu có dạng $(S)\colon x^2+y^2+z^2-2ax-2by-2cz+d=0$.\\
		Mặt cầu có tâm $I(a;b;c)$. Vì $I\in (Oxz)$ và $A,B,C\in (S)$ nên ta có hệ
		\begin{align*}
			\heva{&13-4a-6b+d=0\\&17+8b-2c+d=0\\&11-6a-2b-2c+d=0\\&b=0}\Leftrightarrow \heva{&13-4a-6b+d=0\\&4a+14b-2c=-4\\&-2a+4b-2c=2\\&b=0}\Leftrightarrow \heva{&a=-1\\&b=0\\&c=0\\&d=-17.}
		\end{align*}
		Vậy $T=a+b+c=-1+0+0=-1$.
	}
\end{ex}

\begin{ex}%[TT lần 2, Chuyên KHTN, Hà Nội 2018]%[2H3B1-3]%[Nguyện Ngô và Hồ Như Vương,12EX7]
	Cho các điểm $A(1; 0; 0)$, $B(0; 2; 0)$, $C(0; 0; -2)$. Bán kính mặt cầu ngoại tiếp hình chóp $OABC$ là
	\choice
	{$\dfrac{7}{2}$}
	{$\dfrac{1}{2}$}
	{\True $\dfrac{3}{2}$}
	{$\dfrac{5}{2}$}
	\loigiai{
		\begin{enumerate}[\faCheckSquareO]
			\item \textbf{Cách 1.}	Gọi $I(x; y; z)$ là tâm mặt cầu ngoại tiếp hình chóp $OABC$. Khi đó
			$$\heva{&OI^2=AI^2\\&OI^2=BI^2\\&OI^2=CI^2}\Leftrightarrow
			\heva{&x^2+y^2+z^2=(x-1)^2+y^2+z^2\\&x^2+y^2+z^2=x^2+(y-2)^2+z^2\\&x^2+y^2+z^2=x^2+y^2+(z+2)^2}\Leftrightarrow\heva{&x=\dfrac{1}{2}\\&y=1\\&z=-1.}$$
			Suy ra bán kính $R=OI=\dfrac{3}{2}$.
			\item \textbf{Cách 2.} 
			\begin{itemize}
				\item [$\bullet$] Gọi mặt cầu $(S)$ có dạng $x^2+y^2+z^2-2ax-2by-2cz+d=0 \quad(1)$.
				\item [$\bullet$] Thay lần lượt tọa độ 4 điểm $O$, $A$, $B$, $C$ vào (1) và giải hệ, ta tìm được $a$, $b$, $c$, $d$.
				\item [$\bullet$] Tính $R=\sqrt{a^2+b^2+c^2-d}$.
			\end{itemize}
		\end{enumerate}
		
	}
\end{ex}


\begin{ex}%[2H3K1]
	Cho điểm $D(3; 4; -2).$ Gọi $A, B, C$ lần lượt là hình chiếu vuông góc của $D$ trên các trục tọa độ $Ox, Oy, Oz.$ Gọi $(S)$ là mặt cầu ngoại tiếp tứ diện $ABCD.$ Tính diện tích mặt cầu $(S).$
	\choice
	{$\dfrac{4\sqrt{29}\pi}{3}$}
	{$\dfrac{29\sqrt{29}\pi}{6}$}
	{$116\pi$}
	{\True $29\pi$}
	\loigiai{
		Nhận xét rằng, bốn điểm $A$, $B$, $C$, $D$ là 4 trong 8 đỉnh của một hình hộp chữ nhật có đường chéo là $OD$. Suy ra $R=\dfrac{OD}{2}=\dfrac{\sqrt{29}}{2}$.\\
		Diện tích mặt cầu $S=4 \pi R^2=29\pi$.
	}
\end{ex}


\begin{dang}{Vị trí tương đối của điểm, của mặt phẳng với mặt cầu}
	\begin{enumerate}[\iconCH]
		\item \indamm{Bài toán 1:} Xét điểm $M(x_0;y_0;z_0)$ và mặt cầu $S \colon (x-a)^2+(y-b)^2+(z-c)^2-r^2=0 \quad (1)$. Thay tọa độ điểm $M$ vào vế trái của (1), nếu
		\begin{listEX}[1]
			\item [\ding{172}] Kết quả bằng $0$ thì $M \in (S)$.
			\item [\ding{173}] Kết quả ra số âm thì $M$ nằm trong $(S)$.
			\item [\ding{174}] Kết quả ra số dương thì $M$ nằm trong $(S)$.
		\end{listEX}
		\item \indamm{Bài toán 2:} Cho mặt cầu $(S)$ có tâm $I(a;b;c)$, bán kính $r$ và mặt phẳng $(P)\colon Ax+By+Cz+D=0$.
		\begin{tcolorbox}[colframe=orange!3,colback=red!3!white,boxrule=0.2mm]
			\begin{listEX}[1]
				\item [\ding{172}] Nếu $\mathrm{d}\left(I,(P) \right)=\dfrac{\bigg|Aa+Bb+Cc+D\bigg|}{\sqrt{A^2+B^2+C^2}}>r$ thì $(P)$ và $(S)$ không có điểm chung.
				\item [\ding{173}] Nếu $\mathrm{d}\left(I,(P) \right)=\dfrac{\bigg|Aa+Bb+Cc+D\bigg|}{\sqrt{A^2+B^2+C^2}}=r$ thì $(P)$ tiếp xúc $(S)$.
				\item [\ding{174}] Nếu $\mathrm{d}\left(I,(P) \right)=\dfrac{\bigg|Aa+Bb+Cc+D\bigg|}{\sqrt{A^2+B^2+C^2}}<r$ thì $(P)$ cắt $(S)$.
			\end{listEX}
		\end{tcolorbox}
	\end{enumerate}
\end{dang}
\boxmini{BÀI TẬP TỰ LUẬN}
\setcounter{vd}{0}

\begin{vd}%[2H5N3-1]
	Cho mặt cầu $(S)$ có tâm $I(2;-1;4)$ và bán kính $R=5$. Các điểm $A(3;1;5)$, $B(-1;11;14)$, $C(6;2;4)$ nằm trong, nằm trên hay nằm ngoài mặt cầu $(S)$?
	\loigiai{
		\begin{itemize}
			\item $IA=\sqrt{1^2+2^2+1^2}=\sqrt{6} \approx 2{,}45<R=5$, suy ra $I A<R$. Do đó, điểm $A$ nằm trong mặt cầu $(S)$.
			\item 
			$IB=\sqrt{(-3)^2+12^2+10^2}=\sqrt{253} \approx 15,91>5$, suy ra $I B>R$. Do đó, điểm $B$ nằm ngoài mặt cầu $(S)$.
			\item $I C=\sqrt{4^2+3^2+0^2}=5=R$. Do đó, điểm $C$ nằm trên mặt cầu $(S)$.
		\end{itemize}
	}
\end{vd}
\dongcham{5}
\begin{vd}%[2H5H3-4]   
	\immini{Trong không gian $Oxyz$ (đơn vị trên mỗi trục là mét), một router phát sóng wifi có đầu thu phát được đặt tại điểm $I(4;2;10)$.
		\begin{listEX}[1]
			\item Cho biết bán kính phủ sóng wifi là $40$ m. Viết phương trình mặt cầu $(S)$ biểu diễn ranh giới của vùng phủ sóng.
			\item Một người sử dụng máy tính tại điểm $M(6;12;0)$. Hãy cho biết điểm $M$ nằm trong hay nằm ngoài mặt cầu $(S)$ và người đó có thể sử dụng được sóng wifi của router nói trên hay không?
			\item Câu hỏi tương tự đối với người sử dụng máy tính ở điểm $N(14;6;50)$.
	\end{listEX}}
	{\begin{tikzpicture}[scale=.8,line cap=round,line join=round,font=\footnotesize,>=stealth]
			\tikzset{laptop/.pic={
					% Manhinh
					\fill[cyan!20] (-4, 2.5) -- (4, 2.5) -- (3.5, -1.5) -- (-3.5, -1.5) -- cycle;
					\draw[thick] (-4, 2.5) -- (4, 2.5) -- (3.5, -1.5) -- (-3.5, -1.5) -- cycle;
					
					% vien
					\draw[thick, black] (-4.1, 2.6) -- (4.1, 2.6) -- (3.6, -1.6) -- (-3.6, -1.6) -- cycle;
					
					% Camera
					\fill[red] (0, 2.6) circle (0.1);
					
					% Keyboard 
					\fill[gray!30] (-3.5, -1.5) -- (3.5, -1.5) -- (4, -3) -- (-4, -3) -- cycle;
					\draw[thick] (-3.5, -1.5) -- (3.5, -1.5) -- (4, -3) -- (-4, -3) -- cycle;
					
					% Base
					\fill[blue!20] (-4, -3) rectangle (4, -3.2);
					\draw[thick] (-4, -3) -- (4, -3) -- (4, -3.2) -- (-4, -3.2) -- cycle;
					
					% Touchpad
					\fill[gray!40] (-1.5, -2.5) rectangle (1.5, -2);
					\draw[thick] (-1.5, -2.5) rectangle (1.5, -2);
					
					% Keys
					\foreach \x in {-3.2,-2.7,...,3.2} {
						\foreach \y in {-2.8,-2.5,...,-1.8} {
							\fill[gray!50] (\x,\y) rectangle ++(0.4,0.3);
						}
					}
					
			}}
			\tikzset{router/.pic={
					\fill[black!80] (-3,1,0) -- (3,1,0) -- (3,-2,-1) -- (-3,-2,-1) -- cycle;
					\fill[black!60] (-3,1,0) -- (3,1,0) -- (3,1,-0.2) -- (-3,1,-0.2) -- cycle;
					\fill[black!50] (3,1,0) -- (3,-2,-1) -- (3,-2.2,-1) -- (3,1,-0.2) -- cycle;
					\fill[black!50] (-3,-2,-1) -- (3,-2,-1) -- (3,-2.2,-1) -- (-3,-2.2,-1) -- cycle;
					% Mặt trên
					\foreach \i in {-2.7,-2.4,...,2.7}
					\draw[gray!90] (\i, 1, -0.05) -- (\i, -1.8, -1.05);
					% Anten
					\draw[thick, fill=cyan] (-2.5,1,0) -- (-2.3,1,0) -- (-2.3,3,0.2) -- (-2.5,3,0.2) -- cycle;
					\draw[thick, fill=cyan] (2.5,1,0) -- (2.3,1,0) -- (2.3,3,0.2) -- (2.5,3,0.2) -- cycle;
					% LED
					\foreach \i in {-1.8,-1.4,-1.0,-0.6,-0.2,0.2,0.6,1.0,1.4,1.8}
					\fill[yellow] (\i,0.5,-0.8) circle (0.05);
					% Mặt trước
					\foreach \i in {-2.2,-1.8,-1.4,-1.0,-0.6,-0.2,0.2,0.6,1.0,1.4,1.8}
					\fill[white] (\i,-2.1,-1.05) circle (0.05);
			}}
			\path (0:0) coordinate(O) (-135:3) coordinate(x) (0:6) coordinate(y) (90:3) coordinate(z)
			(-70:2.2) coordinate(B) (-27:4.4) coordinate(A) (50:1.7) coordinate(I);
			%\fill[cyan!15] (O)--(x)--($(x)+(y)-(O)$)--(y)--cycle;
			\draw[->] (O)--(x) node[shift={(150:.2)}]{$x$};
			\draw[->] (O)--(y) node[shift={(80:.2)}]{$y$};
			\draw[->] (O)--(z) node[shift={(160:.2)}]{$z$};
			\path (1.3,-.5) pic[rotate=15,scale=.15,shift=(-150:3.5)]{laptop};
			\path (4.7,-.8) pic[rotate=-15,scale=.15,shift=(-150:3.5)]{laptop};
			\path (1,2) pic[rotate=5, scale=.2,shift=(10:.5)]{router};
			\foreach \d/\g in {B/170, A/180, O/180, I/180 } \fill (\d) circle(1pt) node[shift={(\g:.3)}]{$\d$};
			\path pic[draw,angle radius=.15cm]{right angle=x--O--y} pic[draw,angle radius=.15cm]{right angle=z--O--y} pic[draw,angle radius=.15cm]{right angle=x--O--z};
	\end{tikzpicture}}
	\loigiai{
		\begin{listEX}[1]
			\item Mặt cầu $(S)$ có tâm $I(4;2;10)$, bán kính $R=40$ nên có phương trình là
			$$(x-4)^2+(y-2)^2+(z-10)^2=1600.$$
			\item  Ta có $I M=\sqrt{2^2+10^2+(-10)^2}=\sqrt{204} \approx 14{,}3<40$, suy ra $IM<R$. Do đó, điểm $M$ nằm trong mặt cầu $(S)$.\\
			Vậy người đó có thể sử dụng được sóng wifi của router nói trên.
			\item  Ta có $I N=\sqrt{10^2+4^2+40^2}=\sqrt{1716} \approx 41{,}4>40$, suy ra $IN>R$. Do đó, điểm $N$ nằm ngoài mặt cầu $(S)$.\\
			Vậy người đó \textbf{không} thể sử dụng được sóng wifi của router nói trên.
	\end{listEX}}
\end{vd}
\dongcham{10}

\begin{vd}
	Cho mặt cầu $(S)\colon (x - 2)^2 + y^2 + (z + 1)^2=9$ và mặt phẳng $(P)\colon 2x - y - 2z - 3=0$. 
	\begin{enumEX}[a)]{1}
		\item Chứng minh rằng mặt phẳng $(P)$ cắt mặt cầu $(S)$.
		\item Biết mặt cầu $(S)$ cắt $(P)$ theo giao tuyến là đường tròn $(C)$. Tính bán kính $r$ của đường tròn $(C)$.
	\end{enumEX}
	\loigiai{
		\immini
		{
			Mặt cầu $(S)$ có tâm $I(2;0;-1)$, bán kính $R=3$.\\
			Khoảng cách từ $I$ đến mặt phẳng $(P)$ là
			$$h=\mathrm{d}(I;(P))=\dfrac{|4-0+2-3|}{3}=1.$$
			Ta có $h^2+r^2=R^2\Leftrightarrow 1+r^2=9\Leftrightarrow r=2\sqrt{2}$.
		}
		{
			\begin{tikzpicture}[scale=0.7, font=\footnotesize, line join=round, line cap=round, >=stealth]
				\tikzset{label style/.style={font=\footnotesize}}
				\coordinate (A) at (2,0);
				\coordinate (I) at (0,2);
				\coordinate (K) at (0,0);
				\coordinate (B) at (-2,0);
				\draw[dashed] (A) arc (0:180:2cm and 0.35cm);
				\draw (A) arc (0:-180:2cm and 0.35cm);
				\draw (I) circle (2.828cm);
				\filldraw (I) node[above] {$I$} circle (1pt);
				%\filldraw (K) node[above right] {$K$} circle (1pt);
				%\filldraw (A) node[right] {$A$} circle (1pt);
				%\filldraw (B) node[left] {$B$} circle (1pt);
				\draw[dashed] (I)--(A)--(B)--(I)--(K);
				\tkzLabelSegment[above](I,A){$R$}
				\tkzLabelSegment[above=-0.1](K,A){$r$}
				\tkzLabelSegment[right](I,K){$h$}
			\end{tikzpicture}
		}
	}
\end{vd}
\dongcham{14}
\boxmini{BÀI TẬP TRẮC NGHIỆM}
\setcounter{ex}{0}
\Opensolutionfile{ans}[ans/2H5-B4-d3]
\begin{ex}
	Cho điểm $M\left(1;-1;3\right)$ và mặt cầu $(S)$ có phương trình$\left(x-1\right)^2+(y+2)^2+z^2=9$. Khẳng định đúng là
	\choice
	{\True $M$ nằm ngoài $(S)$}
	{$M$ nằm trong $(S)$}
	{$M$ nằm trên$(S)$}
	{$M$ trùng với tâm của $(S)$}
	\loigiai{
		Thay tọa độ $M$ vào vế trái của phương trình mặt cầu, ta được $(1-1)^2+(-1+2)^2+(3)^2=10>9$. Suy ra $M$ nằm ngoài $(S)$}
\end{ex}

\begin{ex}%[BG-12-New-4in1, Nguyễn Cường]%[2H5H3-3]
	Cho mặt cầu $(S)\colon x^2+y^2+z^2-2x-4y-6z = 0$ và ba điểm $O(0; 0; 0)$, $A(1; 2; 3)$, $B(2; -1; -1)$. Trong số ba điểm trên số điểm nằm trên mặt cầu là
	\choice
	{$2$}
	{$0$}
	{$3$}
	{\True $1$}
	\loigiai{Lần lượt thay tọa độ các điểm $O, A, B$ vào phương trình mặt cầu $(S)$ ta chỉ thấy duy nhất điểm $O$ thuộc mặt cầu $(S)$.
	}
\end{ex}

\begin{ex}%[BG-12-New-4in1, Nguyễn Cường]%[2H5H3-3]
	Cho mặt cầu $(S): x^2+y^2+z^2-4y+6z-2=0$ và mặt phẳng $(P)\colon x+y-z+4=0$. Trong các mệnh đề sau, mệnh đề nào đúng?
	\choice
	{ $(P)$ tiếp xúc $(S)$}
	{\True $(P)$ và $(S)$ không có điểm chung}
	{$(P)$ đi qua tâm của $(S)$}
	{$(P)$ cắt $(S)$}
	\loigiai{
		$(S)$ có tâm $I(0;2;-3)$ và bán kính $R=\sqrt{15}$.\\
		Ta có $\mathrm{d}[I,(P)]=\dfrac{\left|2+3+4 \right| }{\sqrt{1+1+1}}=3\sqrt{3} > \sqrt{15}=R$ nên $(P)$ và $(S)$ không có điểm chung.
	}
\end{ex}

\begin{ex}%[2H3B2-7]%Câu 9.
	Cho mặt phẳng $(P)$ và mặt cầu $(S)$ có phương trình lần lượt là $(P)\colon 2x+2y+z-m^2+4m-5=0$; $(S)\colon x^2+y^2+z^2-2x+2y-2z-6=0$. Giá trị của $m$ để $(P)$ tiếp xúc $(S)$ là
	\choice
	{$m=5$}
	{$m=-1$}
	{\True $m=-1$ hoặc $m=5$}
	{$m=1$ hoặc $m=-5$}
	\loigiai{
		Mặt cầu $(S)\colon x^2+y^2+z^2-2x+2y-2z-6=0$ có tâm $I(1;-1;1)$ và bán kính $R=3$.\\
		$(P)$ tiếp xúc $(S)\Leftrightarrow\mathrm{d}\left(I,(P)\right)=R$
		\begin{eqnarray*}
			&\Leftrightarrow&\dfrac{\left|2\cdot 1+2\cdot (-1)+1-m^2+4m-5\right|}{\sqrt{2^2+2^2+1^2}}=3\\
			&\Leftrightarrow&\left|m^2-4m+4\right|=9\\
			&\Leftrightarrow&\hoac{&m^2-4m+4=9\\&m^2-4m+4=-9\quad \text{(vô nghiệm)}}\\
			&\Leftrightarrow& m^2-4m-5=0\Leftrightarrow\hoac{&m=-1\\&m=5.}
		\end{eqnarray*}
	}
\end{ex}

\begin{ex}%[Thi thử L3, Yên Lạc 2, Vĩnh Phúc 2018]%[Hoàng Trình,12EX6]%[2H3B2-7]%
	Trong không gian với hệ tọa độ $Oxyz$, cho mặt phẳng $(P)\colon 2x+2y+z-2=0$ và mặt cầu $(S)$ tâm $I\left(2;1;-1\right)$ bán kính $R=2$. Bán kính đường tròn giao của mặt phẳng $(P)$ và mặt cầu $(S)$ là
	\choice
	{\True $r=\sqrt{3}$}
	{$r=\sqrt{5}$}
	{$r=1$}
	{$r=3$}
	\loigiai{
		\immini{
			Gọi bán kính đường tròn giao của mặt phẳng $(P)$\\
			và mặt cầu $(S)$ là $r$.\\
			Ta có $h=\mathrm{d}(I,(P))=\dfrac{\left|2\cdot 2+2\cdot (-1)-1-2 \right| }{\sqrt{2^2+2^2+1^2}}=1$.\\
			Suy ra $r=\sqrt{2^2-1^2}=\sqrt{3}$.
		}
		{
			\begin{tikzpicture}[line width=0.6pt]
				\draw[fill=black] (0,0)coordinate(I) circle(1pt) node[above left]{$I$};
				\draw[fill=black] (0,-0.9)coordinate(H) circle(1pt);
				\node at (0,-0.9) [left]{$H$};
				\draw[dashed] (0,0)--(0,-0.9)--(1,-1.3)coordinate(M)--(0,0);
				\tkzMarkRightAngle(I,H,M)
				\node at (0.83,-0.7) {$R$};
				\node at (0.4,-1.23) {$r$};
				\draw[line width=0.6pt] (1.61,-1.01) arc [radius=1.9, start angle=-32.1, end angle=212];
				\draw[line width=0.6pt] (-1.164,-1.5) arc [radius=1.9, start angle=-127.9, end angle=-52];
				\draw[dashed,line width=0.6pt] (-1.61,-1.01) arc [radius=1.9, start angle=212, end angle=232.1];
				\draw[dashed,line width=0.6pt] (1.61,-1.01) arc [radius=1.9, start angle=-32.1, end angle=-52];
				\coordinate (A) at (-2.7,-1.5);
				\coordinate (B) at (-1.7,-0.3);
				\coordinate (C) at (2.6,-0.3);
				\coordinate (D) at ($(A)+(C)-(B)$);
				\draw (-1.84,-0.47)--(A)--(D)--(C)--(1.87,-0.3);
				\draw[line width=0.5pt,dashed] (-1.84,-0.47)--(B)--(1.87,-0.3);
				\draw[dashed] (-1.64,-0.9)..controls (-1.57,-0.2) and (1.57,-0.2)..(1.64,-0.9);
				\draw (-1.64,-0.9)..controls (-1.57,-1.6) and (1.57,-1.6)..(1.64,-0.9);
				\node at (-2.25,-1.3) {$(P)$};
				\node at (-1.9,1.3) {$(S)$};
				\draw (-2.1,-1.5)..controls (-1.95,-1.1)..(-2.34,-1.05);
			\end{tikzpicture}
		}
	}
\end{ex}

\begin{ex}%[Dự án EX-7-2019]%[Phạm Tuấn]%[2H3B2-7]%
	Trong không gian $Oxyz $, mặt cầu có phương trình $x^2+y^2+z^2-2x+2y-6z+2=0$
	cắt mặt phẳng $(Oxz)$ theo một đường tròn có bán kính bằng
	\choice
	{$3\sqrt{2}$}
	{\True $2\sqrt{2}$}
	{$5$}
	{$4\sqrt{2}$}
	\loigiai{
		Mặt cầu đã cho có tâm $I(1;-1;3)$ và bán kính $R=3$. \\
		Khoảng cách từ $I$ đến mặt phẳng $(Oxz)$ bằng $1$, do đó bán kính của đường tròn bằng \[\sqrt{3^2-1^2}=2\sqrt{2}.\]
	}
\end{ex}

\begin{ex}%[Thi thử L1, Chuyên Ngoại Ngữ, Hà Nội, 2018]%[2H3B2-7]%[Nguyễn Bình Nguyên-12Ex8]%
	Trong không gian với hệ trục tọa độ $Oxyz$, cho mặt cầu $(S) \colon (x-1)^2+(y-2)^2+(z-2)^2=9$ và mặt phẳng $(P) \colon 2x-y-2z+1=0$. Biết $(P)$ cắt $(S)$ theo giao tuyến là đường tròn có bán kính $r$. Tính $r$.
	\choice
	{$r=3$}
	{$r=2$}
	{\True $r=2\sqrt{2}$}
	{$r=\sqrt{3}$}
	\loigiai
	{Ta có $(S) \colon (x-1)^2+(y-2)^2+(z-2)^2=9$ $\Rightarrow \heva{&I(1;2;2)\\&R=3}.$\\ $d=\mathrm{d}\left(I,(\alpha)\right)=\dfrac{|2\cdot1-2-2\cdot2+1|}{\sqrt{2^2+(-1)^2+(-2)^2}}=1.$\\
		Vậy	$r=\sqrt{R^2-d^2}=2\sqrt{2}$.}
\end{ex}

\begin{ex}%[Thi thử L2, Lương Thế Vinh, Hà Nội, 2018]%[Phạm Toàn, Dự án (12EX-9)]%[2H3B2-7]%
	Trong không gian $Oxyz$ cho mặt cầu $(S)\colon x^2+y^2+z^2-6x-4y-12z=0$ và mặt phẳng $(P)\colon 2x+y-z-2=0$. Tính diện tích thiết diện của mặt cầu $(S)$ cắt bởi mặt phẳng $(P)$.
	\choice
	{$50\pi $}
	{\True $S=49\pi $}
	{$25\pi $}
	{$36\pi $}
	\loigiai{
		Mặt cầu $(S)$ có tâm $I(3;2;6)$ và bán kính $R=\sqrt{3^2+2^2+6^2}=7$. Vì $I$ thuộc $(P)$ nên $(P)$ cắt $(S)$ theo thiết diện là đường tròn có bán kính bằng $7$. Diện tích thiết diện bằng $49\pi$.
	}
\end{ex}

\begin{ex}%[Đề thi thử Tốt nghiệp THPT lần 1, Sở GD&ĐT Hà Nội, 2020]%[Nguyễn Đắc Giáp, 12EX8]%[2H3B2-7]%
	Trong không gian với hệ tọa độ $Oxyz$, cho điểm $I(2;1;1)$ và mặt phẳng $(P)\colon 2x+y+2z-1=0$. Mặt cầu $(S)$ có tâm $I$, cắt $(P)$ theo một đường tròn có bán kính $r=4$. Mặt cầu $(S)$ có phương trình là
	\choice
	{$(x-2)^2 + (y-1)^2 + (z-1)^2 =18$}
	{$(x-2)^2 + (y-1)^2 +(z-1)^2 = 2\sqrt{5}$}
	{\True $(x-2)^2 + (y-1)^2 + (z-1)^2 =20$}
	{$(x+2)^2 + (y+1)^2 + (z+1)^2 =20$}
	\loigiai{
		Ta có $$\mathrm{d}(I;(P)) = \dfrac{\left | 2\cdot 2+1+2\cdot 1-1  \right |}{\sqrt{2^2+1^2+2^2}} =2.$$
		Vì mặt cầu $(S)$ có tâm $I$, cắt $(P)$ theo một đường tròn có bán kính $r=4$ nên mặt cầu $(S)$ có bán kính
		$$R=\sqrt{r^2+\mathrm{d}^2(I,(P))} = \sqrt{4^2+2^2} = 2\sqrt{5}.$$
		Vậy phương trình mặt cầu $(S)$ là $(x-2)^2 + (y-1)^2 + (z-1)^2 =20$.
	}
\end{ex}

\begin{ex}%[HK2, Sở GD&ĐT Đà Nẵng, 2019]%[Trần Chiến, 12EX6-2020]%[2H3B2-7]%
	Trong không gian $Oxyz$, cho mặt cầu $(S)$ có tâm $I(1;2;1)$ và cắt mặt phẳng $(P)\colon 2x-y+2z+7 =0$ theo một đường tròn có đường kính bằng $8$. Phương trình mặt cầu $(S)$ là
	\choice
	{\True $(x-1)^2+(y-2)^2 +(z-1)^2=25$}
	{$(x-1)^2+(y-2)^2 +(z-1)^2=81$}
	{$(x+1)^2+(y+2)^2 +(z+1)^2=9$}
	{$(x-1)^2+(y-2)^2 +(z-1)^2=5$}
	\loigiai{
		\immini{
			Gọi $R$, $r$, $d$ lần lượt là bán kính của mặt cầu, bán kính của đường tròn và khoảng cách từ tâm $I$ đến mp$(P)$. Khi đó $R = \sqrt{r^2+d^2}$.\\
			Ta có $d= \mathrm{d}(I,(P)) =\dfrac{|2\cdot 1 -2+2\cdot 1+7|}{\sqrt{2^2+ (-1)^2+2^2}} = 3$ và $r=4$. \\
			Suy ra $R =\sqrt{3^2+4^2}=5$.\\
			Vậy phương trình mặt cầu $(S)$ là $(x-1)^2+(y-2)^2 +(z-1)^2=25$.
		}{
			\begin{tikzpicture}[scale=0.7, font=\footnotesize, line join=round, line cap=round, >=stealth]
				\tkzDefPoints{0/0/O}
				\def\a{3}
				\def\b{0.6}
				\coordinate (M) at (0:\a cm and \b cm);
				\coordinate (I) at ($(O)+(0,1.5)$);
				\tkzDrawCircle(I,M)
				\draw (M) arc (0:-180:\a cm and \b cm);
				\draw[dashed] (M) arc (0:180:\a cm and \b cm);
				\tkzDrawSegments[dashed](I,O I,M M,O)
				\tkzDrawPoints[fill=black](I,O,M)
				\tkzLabelSegment[left](I,O){$d$}
				\tkzLabelSegment[above right](I,M){$R$}
				\tkzLabelSegment[below](M,O){$r$}
			\end{tikzpicture}
		}
	}
\end{ex}
\Closesolutionfile{ans}


\subsection{BÀI TẬP TRẮC NGHIỆM TỰ LUYỆN}
\subsection*{\indam{PHẦN I. Câu trắc nghiệm nhiều phương án lựa chọn. Thí sinh trả lời từ câu 1 đến câu 12. Mỗi câu hỏi thí sinh chỉ chọn một phương án.}} 
	\setcounter{ex}{0}
	\Opensolutionfile{ans}[ans/B4-De2-1]

\begin{ex}%[2H5H3-1]
	Trong không gian $Oxyz$, điểm nào sau đây nằm trong mặt cầu $(S)\colon \left(x-1\right)^2+\left(y-4\right)^2+\left(z-3\right)^2=16$?
	\choice
	{$M\left(0;7;-3\right)$}
	{$P\left(1;0;0\right)$}
	{\True $N\left(0;4;3\right)$}
	{$Q\left(1;0;3\right)$}
	\loigiai{
		Mặt cầu $(S)$ có tâm $I\left(1;4;3\right)$ và bán kính $R=4$.\\
		$IM=\sqrt{(-1)^2+3^2+(-6)^2}=\sqrt{46} > R\Rightarrow M$ nằm ngoài mặt cầu.\\
		$IN=\sqrt{(-1)^2+0^2+0^2}=1< R\Rightarrow N$ nằm trong mặt cầu.\\
		$IP=\sqrt{0^2+\left(-4\right)^2+\left(-3\right)^2}=5> R\Rightarrow P$ nằm ngoài mặt cầu.\\
		$IQ=\sqrt{0^2+\left(-4\right)^2+0^2}=4=R\Rightarrow Q$ thuộc mặt cầu.\\
	}
\end{ex}

%G:\My Drive\CODE12-2024\DE-ON-THEO BAI\2H5-TACH DE\Bai4-De2.tex
\begin{ex}%[2H5H3-3]
	Trong không gian $Oxyz$, cho mặt cầu $(S)$ có tâm $A\left(3;-1;1\right)$ và đi qua $M\left(2;-2;4\right)$. Phương trình mặt cầu $(S)$ là
	\choice
	{\True $\left(x-3\right)^2+\left(y+1\right)^2+\left(z-1\right)^2=11$}
	{$\left(x+3\right)^2+\left(y-1\right)^2+\left(z+1\right)^2=11$}
	{$\left(x+3\right)^2+\left(y-1\right)^2+\left(z+1\right)^2=\sqrt{11}$}
	{$\left(x-3\right)^2+\left(y+1\right)^2+\left(z-1\right)^2=\sqrt{11}$}
	\loigiai{
		Bán kính $R=AM=\sqrt{(-1)^2+(-1)^2+3^2}=\sqrt{11}$.\\
		Phương trình mặt cầu  là  $(S)\colon \left(x-3\right)^2+\left(y+1\right)^2+\left(z-1\right)^2=11$.
	}
\end{ex}

%G:\My Drive\CODE12-2024\DE-ON-THEO BAI\2H5-TACH DE\Bai4-De2.tex
\begin{ex}%[2H5N3-2]
	Trong không gian $Oxyz$, cho mặt cầu $(S):\left(x-2\right)^2+\left(y+2\right)^2+\left(z-1\right)^2=18$. Bán kính của $(S)$ bằng
	\choice
	{$9$}
	{$18$}
	{$6\sqrt{2}$}
	{\True $3\sqrt{2}$}
	\loigiai{
		Bán kính $R=\sqrt{18}=3\sqrt{2}$.
	}
\end{ex}

%G:\My Drive\CODE12-2024\DE-ON-THEO BAI\2H5-TACH DE\Bai4-De2.tex
\begin{ex}%[2H5H3-3]
	Trong không gian $Oxyz$, cho hai điểm $M\left(3;-2;5\right)$, $N\left(-1;6;-3\right)$. Mặt cầu đường kính $MN$ có phương trình là
	\choice
	{$\left(x+1\right)^2+\left(y+2\right)^2+\left(z+1\right)^2=6$}
	{$\left(x-1\right)^2+\left(y-2\right)^2+\left(z-1\right)^2=6$}
	{\True $\left(x-1\right)^2+\left(y-2\right)^2+\left(z-1\right)^2=36$}
	{$\left(x+1\right)^2+\left(y+2\right)^2+\left(z+1\right)^2=36$}
	\loigiai{
		Tâm $I$ của mặt cầu là trung điểm đoạn $MN$ $\Rightarrow$ $I\left(1;2;1\right)$.\\
		Bán kính mặt cầu $R=\dfrac{MN}{2}=\dfrac{\sqrt{\left(-1-3\right)^2+\left(6+2\right)^2+\left(-3-5\right)^2}}{2}=6$.\\
		Vậy phương trình mặt cầu là $\left(x-1\right)^2+\left(y-2\right)^2+\left(z-1\right)^2=36$.
	}
\end{ex}

%G:\My Drive\CODE12-2024\DE-ON-THEO BAI\2H5-TACH DE\Bai4-De2.tex
\begin{ex}%[2H5H3-2]
	Trong không gian $Oxyz$, cho mặt cầu $(S)\colon \left(x-5\right)^2+\left(y-1\right)^2+\left(z+2\right)^2=9$. Đường kính của mặt cầu $(S)$ là
	\choice
	{$9$}
	{$3$}
	{\True $6$}
	{$18$}
	\loigiai{
		Bán kính $R=\sqrt{9}=3\Rightarrow$ đường kính bằng $2R=2\cdot 3=6$.
	}
\end{ex}

%G:\My Drive\CODE12-2024\DE-ON-THEO BAI\2H5-TACH DE\Bai4-De2.tex
\begin{ex}%[2H5H3-2]
	Trong không gian $Oxyz$, phương trình nào sau đây là phương trình của mặt cầu?
	\choice
	{$x^2+z^2+3x-2y+4z-1=0$}
	{$x^2+y^2+z^2-2x+2y-4z+8=0$}
	{\True $x^2+y^2+z^2-2x+4z-1=0$}
	{$x^2+y^2+z^2+2xy-4y+4z-1=0$}
	\loigiai{
		Điều kiện để phương trình $x^2+y^2+z^2-2ax-2by-2cz+d=0$ là phương trình mặt cầu là
		\[
		a^2+b^2+c^2-d > 0.
		\]
		Xét phương trình  $x^2+y^2+z^2-2x+4z-1=0$ có  $1^2+0^2+(-2)^2-(-1)=6> 0$. \\
		Suy ra phương trình  $x^2+y^2+z^2-2x+4z-1=0$  là phương trình của một mặt cầu.
	}
\end{ex}

%G:\My Drive\CODE12-2024\DE-ON-THEO BAI\2H5-TACH DE\Bai4-De2.tex
\begin{ex}%[2H5H3-2]
	Trong không gian với hệ toạ độ $Oxyz$,  cho mặt cầu $(S):(x+2)^2+y^2+\left(z-3\right)^2=4$. Tâm của $(S)$ có toạ độ là
	\choice
	{$\left(-1;0;\dfrac{3}{2} \right)$}
	{$\left(2;0;-3\right)$}
	{$\left(1;0;\dfrac{3}{2} \right)$}
	{\True $\left(-2;0;3\right)$}
	\loigiai{
		Phương trình mặt cầu $\left(x-a\right)^2+\left(y-b\right)^2+\left(z-c\right)^2=R^2$ có tâm $I\ (a;b;c)$.\\
		Suy ra $(S)\colon (x+2)^2+y^2+\left(z-3\right)^2=4$ có tâm $\left(-2;0;3\right)$.
	}
\end{ex}

%G:\My Drive\CODE12-2024\DE-ON-THEO BAI\2H5-TACH DE\Bai4-De2.tex
\begin{ex}%[2H5H3-4]
	Trong không gian $Oxyz$, một thiết bị phát sóng đặt tại vị trí $A(2;0;0)$. Vùng phủ sóng của thiết bị có bán kính bằng $1$. Điểm nào sau đây thuộc vùng phủ sóng của thiết bị nói trên?
	\choice
	{\True $P\left(1;0;0\right)$}
	{$O\left(0;0;0\right)$}
	{$N\left(0;1;1\right)$}
	{$M\left(1;0;3\right)$}
	\loigiai{
		Ta có $AM=\sqrt{(-1)^2+0^2+3^2}=\sqrt{10} > R$ nên $M$ không thuộc vùng phủ sóng.\\
		$AN=\sqrt{0^2+1^2+1^2}=\sqrt{2} > R$ nên $N$ không thuộc vùng phủ sóng.\\
		$AP=\sqrt{(-1)^2+0^2+0^2}=1=R$ nên $P$ thuộc vùng phủ sóng.\\
		$AO=\sqrt{(-2)^2+0^2+0^2}=2> R$ nên $O$ không thuộc vùng phủ sóng.
	}
\end{ex}

%G:\My Drive\CODE12-2024\DE-ON-THEO BAI\2H5-TACH DE\Bai4-De2.tex
\begin{ex}%[2H5H3-2]
	Trong không gian $Oxyz$, cho mặt cầu $(S)\colon x^2+y^2+z^2-2x+4y+2z-3=0$. Bán kính $R$ của mặt cầu $(S)$ bằng
	\choice
	{$\sqrt{3}$}
	{$9$}
	{$3\sqrt{3}$}
	{\True $3$}
	\loigiai{
		Mặt cầu có dạng $x^2+y^2+z^2-2ax-2by-2cz+d=0$ có bán kính là $R=\sqrt{a^2+b^2+c^2-d}$.\\
		Suy ra $R=\sqrt{1^2+(-2)^2+1^2-(-3)}=3$.
	}
\end{ex}

%G:\My Drive\CODE12-2024\DE-ON-THEO BAI\2H5-TACH DE\Bai4-De2.tex
\begin{ex}%[2H5H3-3]
	Trong không gian $Oxyz$, phương trình mặt cầu tâm $I\left(1;2;3\right)$, bán kính $R=2$ có dạng
	\choice
	{\True $\left(x-1\right)^2+\left(y-2\right)^2+\left(z-3\right)^2=4$}
	{$\left(x-1\right)^2+\left(y-2\right)^2+\left(z-3\right)^2=2$}
	{$\left(x+1\right)^2+\left(y+2\right)^2+\left(z+3\right)^2=2$}
	{$\left(x+1\right)^2+\left(y+2\right)^2+\left(z+3\right)^2=4$}
	\loigiai{
		Phương trình mặt cầu tâm $I\left(1;2;3\right)$, bán kính $R=2$ có dạng
		\[
		\left(x-1\right)^2+\left(y-2\right)^2+\left(z-3\right)^2=4.
		\]
	}
\end{ex}

%G:\My Drive\CODE12-2024\DE-ON-THEO BAI\2H5-TACH DE\Bai4-De2.tex
\begin{ex}%[2H5H3-2]
	Trong không gian $Oxyz$, tìm $m$ để phương trình $x^2+y^2+z^2-2x-y+4z-m=0$ là phương trình của mặt cầu.
	\choice
	{$m\le \dfrac{21}{4}$}
	{\True $m > \dfrac{21}{4}$}
	{$m\ge \dfrac{21}{4}$}
	{$m < \dfrac{21}{4}$}
	\loigiai{
		Phương trình đã cho là phương trình mặt cầu khi và chỉ khi
		\[
		a^2+b^2+c^2-d > 0 \Leftrightarrow 1^2+\left(\dfrac{1}{2} \right)^2+(-2)^2+m > 0 \Leftrightarrow m > \dfrac{21}{4}.
		\]
	}
\end{ex}

%G:\My Drive\CODE12-2024\DE-ON-THEO BAI\2H5-TACH DE\Bai4-De2.tex
\begin{ex}%[2H5N3-1]
	Trong không gian với hệ toạ độ $Oxyz$, mặt cầu  $(S)\colon x^2+y^2+z^2-2ax-2by-2cz+d=0$ có bán kính $R$ bằng
	\choice
	{$a^2+b^2+c^2+d$}
	{$\sqrt{a^2+b^2+c^2+d}$}
	{\True $\sqrt{a^2+b^2+c^2-d}$}
	{$a^2+b^2+c^2-d$}
	\loigiai{
		Mặt cầu  $(S)\colon x^2+y^2+z^2-2ax-2by-2cz+d=0$ có bán kính $R = \sqrt{a^2+b^2+c^2-d}$.
	}
\end{ex}
	\Closesolutionfile{ans}

\subsection*{\indam{PHẦN II. Câu trắc nghiệm đúng sai. Thí sinh trả lời từ câu 1 đến câu 4. Trong mỗi ý a), b), c), d) ở mỗi câu, thí sinh chọn đúng hoặc sai.}}
	\setcounter{ex}{0}
	\Opensolutionfile{ans}[ans/B4-De2-2]
\begin{ex}%[2H5H3-3]
	Trong không gian $Oxyz$, cho mặt cầu $(S): \left(x-2\right)^2+y^2+\left(z+1\right)^2=1$ và mặt phẳng $(P)\colon  x+2y-z+1=0$. Các mệnh đề sau đúng hay sai?
	\choiceTF
	{Khoảng cách từ tâm $I$ đến mặt phẳng $(P)$ bằng $\dfrac{\sqrt{6}}{3}$}
	{\True Mặt cầu $(S)$ có tâm $I\left(2;0;-1\right)$ và bán kính $R=1$}
	{Mặt phẳng $(P)$ tiếp xúc mặt cầu $(S)$}
	{\True Phương trình mặt cầu tâm $I\left(2;0;-1\right)$ và tiếp xúc mặt phẳng $(P)$ là: $\left(S'\right): x^2+y^2+z^2-4x+2z+\dfrac{7}{3}=0$}
	\loigiai{
		\begin{itemchoice}
			\itemch \textbf{Sai}.\\
			$\mathrm{d}\left(I,(P)\right)=\dfrac{\left|2+1+1\right|}{\sqrt{1+4+1}}=\dfrac{2\sqrt{6}}{3}$.
			\itemch \textbf{Đúng}.\\
			Mặt cầu $(S)\colon \left(x-2\right)^2+y^2+\left(z+1\right)^2=1$ có tâm $I\left(2;0;-1\right)$ và bán kính $R=1$.
			\itemch \textbf{Sai}.\\
			Vì $\mathrm{d}\left(I,(P)\right)=\dfrac{2\sqrt{6}}{3} > R$ nên $(P)$ không cắt mặt cầu $(S)$.
			\itemch \textbf{Đúng}.\\
			Phương trình mặt cầu cần tìm có bán kính $R=\mathrm{d}\left(I,(P)\right)=\dfrac{2\sqrt{6}}{3}$, tâm $I\left(2;0;-1\right)$ nên có phương trình 
			\[
			\left(x-2\right)^2+y^2+\left(z+1\right)^2=\left(\dfrac{2\sqrt{6}}{3} \right)^2 \Leftrightarrow x^2+y^2+z^2-4x+2z+\dfrac{7}{3}=0.
			\]
		\end{itemchoice}
	}
\end{ex}

%G:\My Drive\CODE12-2024\DE-ON-THEO BAI\2H5-TACH DE\Bai4-De2.tex
\begin{ex}%[2H5H3-3]
	Trong không gian $Oxyz$, cho mặt cầu $(S)\colon \left(x-1\right)^2+(y+3)^2+\left(z-2\right)^2=49$. 
	\choiceTF
	{\True Mặt cầu $(S)$ có bán kính $R=7$}
	{\True Điểm $A\left(1;4;2\right)$ nằm trên mặt cầu $(S)$}
	{Mặt cầu $(S)$ có tâm $I\left(1;3;2\right)$}
	{Mặt cầu $(S)$ còn có phương trình: $x^2+y^2+z^2-2x+6y-4z-49=0$}
	\loigiai{
		\begin{itemchoice}
			\itemch \textbf{Đúng}.\\
			Mặt cầu $(S)\colon \left(x-1\right)^2+(y+3)^2+\left(z-2\right)^2=49=7^2$ nên mặt cầu $(S)$ có bán kính $R=7$.
			\itemch \textbf{Đúng}.\\
			Thay toạ độ điểm $A\left(1;4;2\right)$ vào vế trái của mặt cầu $(S)$ được: $VT=7^2=49$.
			\itemch \textbf{Sai}.\\
			Mặt cầu $(S)$ có tâm là $I\left(1;-3;2\right)$.
			\itemch \textbf{Sai}.\\
			Ta có
			\begin{eqnarray*}
				& & (S) \colon \left(x-1\right)^2+(y+3)^2+\left(z-2\right)^2=49\\
				& \Leftrightarrow &  x^2+y^2+z^2-2x+6y-4z+1+9+4=49\\
				& \Leftrightarrow & x^2+y^2+z^2-2x+6y-4z-35=0.
			\end{eqnarray*}
		\end{itemchoice}
	}
\end{ex}

%G:\My Drive\CODE12-2024\DE-ON-THEO BAI\2H5-TACH DE\Bai4-De2.tex
\begin{ex}%[2H5H3-3]
	Trong không gian $Oxyz$, cho mặt cầu $(S): x^2+y^2+z^2+2x+8y+1=0$. Các mệnh đề sau đúng hay sai?
	\choiceTF
	{Mặt cầu $(S)$ có tâm $I\left(1;4;0\right)$}
	{\True Mặt cầu $(S)$ còn có phương trình: $(S): \left(x+1\right)^2+\left(y+4\right)^2+z^2=16$}
	{\True Điểm $M\left(0;3;4\right)$ nằm bên ngoài mặt cầu $(S)$}
	{\True Mặt cầu $(S)$ có bán kính $R=4$}
	\loigiai{
		\begin{itemchoice}
			\itemch \textbf{Sai}.\\
			Mặt cầu $(S)$ có tâm là $I\left(-1;-4;0\right)$.
			\itemch \textbf{Đúng}.\\
			Vì mặt cầu $(S)$ có tâm $I\left(-1;-4;0\right)$, bán kính $R=4$ nên có phương trình là
			\[
			(S)\colon  \left(x+1\right)^2+\left(y+4\right)^2+z^2=16.
			\]
			\itemch \textbf{Đúng}.\\
			Ta có $IM=\sqrt{1^2+7^2+4^2} > R$ nên điểm $M\left(0;3;4\right)$ nằm bên ngoài mặt cầu $(S)$.
			\itemch \textbf{Đúng}.\\
			Mặt cầu $(S)$ có bán kính $R=\sqrt{a^2+b^2+c^2-d}=\sqrt{1+16-1}=4$.
		\end{itemchoice}
	}
\end{ex}

%G:\My Drive\CODE12-2024\DE-ON-THEO BAI\2H5-TACH DE\Bai4-De2.tex
\begin{ex}%[2H5H3-2]
	Trong không gian $Oxyz$, cho ba điểm $A(1; 2; -4)$, $B(1; -3; 1)$, $C(2; 2; 3)$.
	\choiceTF
	{\True Bán kính của mặt cầu $(S_4)$ đi qua ba điểm $A$, $B$, $C$  và có tâm nằm trên mặt phẳng $(Oxy)$ là $R = \sqrt{26}$}
	{Mặt cầu $\left(S_1 \right)$ tâm $A$, bán kính $R=1$ có phương trình là: $\left(x-1\right)^2+\left(y-2\right)^2+\left(z-4\right)^2=1$}
	{\True Bán kính của mặt cầu $\left(S_2 \right)$ có tâm là $A$ và đi qua điểm $C$ là $\sqrt{50}$}
	{\True Mặt cầu $\left(S_3 \right)$ nhận $AB$ làm đường kính có phương trình là: $\left(x-1\right)^2+\left(y+\dfrac{1}{2} \right)^2+\left(z+\dfrac{3}{2} \right)^2=\dfrac{25}{2}$}
	\loigiai{
		\begin{itemchoice}
			\itemch \textbf{Đúng}.\\
			Gọi phương trình mặt cầu $(S_4)$ có dạng $x^2+y^2+z^2-2ax-2by-2cz+d=0$, với tâm $I(a; b; c)$.\\
			Ta có $I(a;b;c)\in (Oxy) \Rightarrow c=0$. \\
			Vì $ \heva{&{A\in (S)} \\&{B\in (S)} \\&{C\in (S)}}
			\Rightarrow	 \heva{&{-2a-4b+d=-21} \\ &{-2a+6b+d=-11} \\&{-4a-4b+d=-17}}
			\Leftrightarrow \heva{&{a=-2} \\&{b=1} \\&{d=-21}}$\\
			Suy ra $R=\sqrt{a^2+b^2+c^2-d}=\sqrt{4+1+0+21}=\sqrt{26}$.
			\itemch \textbf{Sai}.\\
			Mặt cầu $\left(S_1 \right)$ tâm $A$, bán kính $R=1$ có phương trình là $\left(x-1\right)^2+\left(y-2\right)^2+\left(z+4\right)^2=1$.
			\itemch \textbf{Đúng}.\\
			Ta có $\overrightarrow{AC}=\left(1;0;7\right) \Rightarrow AC=\sqrt{50}$.\\
			Suy ra bán kính của mặt cầu $\left(S_2 \right)$ có tâm là $A$ và đi qua điểm $C$ là $AC=\sqrt{50}$.
			\itemch \textbf{Đúng}.\\
			Ta có $\overrightarrow{AB}=\left(0;-5;5\right) \Rightarrow AB=5\sqrt{2}$.\\
			Mặt cầu $\left(S_3 \right)$ nhận $AB$ làm đường kính nên có tâm $I\left(1;-\dfrac{1}{2};-\dfrac{3}{2} \right)$, $R=\dfrac{AB}{2}=\dfrac{5\sqrt{2}}{2}$, có phương trình là
			\[
			\left(x-1\right)^2+\left(y+\dfrac{1}{2} \right)^2+\left(z+\dfrac{3}{2} \right)^2=\dfrac{25}{2}.
			\]
		\end{itemchoice}
	}
\end{ex}
\Closesolutionfile{ans}

\subsection*{\indam{PHẦN III. Câu trắc nghiệm trả lời ngắn. Thí sinh trả lời từ câu 1 đến câu 6 vào ô kết quả.}}
	\setcounter{ex}{0}
	\Opensolutionfile{ans}[ans/B4-De2-3]

\begin{ex}%[2H5H3-4]
	Trong không gian $Oxyz$, cho phương trình $x^2+y^2+z^2-4x+2my+3m^2-2m=0$ với $m$ là tham số. Tính tổng tất cả các giá trị nguyên của $m$ để phương trình đã cho là phương trình mặt cầu.\\
	\shortans[oly]{$1$}
	\loigiai{
		Phương trình  $x^2+y^2+z^2-4x+2my+3m^2-2m=0$ là phương trình mặt cầu khi và chỉ khi 
		\[
		-2m^2+2m+4> 0\Leftrightarrow m\in \left(-1; 2\right).
		\]
		Do $m\in \mathbb{Z}\Rightarrow m\in \left\{0; 1\right\}$.
		Vậy tổng tất cả các giá trị nguyên của $m$ bằng $1$.
	}
\end{ex}

\begin{ex}%[2H5V3-4]
	Trong không gian $Oxyz$ (đơn vị của các trục tọa độ là ki--lô-mét), đài kiểm soát không lưu sân bay có tọa độ $\left(-64;128;64\right)$. Máy bay bay trong phạm vi cách đài kiểm soát $500$ km thì sẽ hiển thị trên màn hình ra đa. Một máy bay $N$ xuất hiện trên màn hình ra đa và một máy bay $M$ nằm trong mặt phẳng $(P)\colon x-2y+2z-1458=0$ sao cho hai máy bay $M$, $N$ thuộc đường thẳng có vectơ chỉ phương là $\overrightarrow{u}=\left(1;1;1\right)$. Khoảng cách nhỏ nhất giữa hai máy bay $M$, $N$ là bao nhiêu km? (kết quả làm tròn đến hàng đơn vị)\\
	\shortans[oly]{$260$}
	\loigiai{
		\immini{
			Máy bay $N$ xuất hiện trên màn hình ra đa nên $N$ thuộc mặt cầu $(S)$ có tâm $I\left(-64;128;64\right)$, bán kính $R=500$.\\
			Ta có 
			\[	
			\mathrm{d}\left(I,(P)\right)=\dfrac{\left|x_I-2y_I+2z_I-1458\right|}{\sqrt{1^2+2^2+2^2}}=550> R
			\] nên $(P)$ và $(S)$ không giao nhau.\\	
			Gọi $\alpha$ là góc tạo bởi $MN$ và mặt phẳng $(P)$.\\
			Gọi $H$ là hình chiếu của $N$ lên mặt phẳng $(P)$.
		}{\begin{tikzpicture}[scale=1, font=\footnotesize,line join=round, line cap=round, >=stealth, every node/.style={scale=0.8}]
				\coordinate (I) at (0,0);
				\coordinate (N) at (0,-1);
				\coordinate (H) at (0,-3);
				\coordinate (M) at (-1,-3);
				\path let \p1=(I),\p2=(N),\n1={veclen(\x2-\x1,\y2-\y1)} in \pgfextra{\xdef\Temp{\n1}};
				\draw (I) circle (\Temp);
				\coordinate (A) at (-2,-3);
				\coordinate (B) at (4,-3);
				\coordinate (C) at (1.5,-2);
				\coordinate (D) at ($(N)+(C)-(M)$);
				\draw[->] (C)--(D) node[midway,right]{$\vec{u}$};
				\draw(A)--(B) (I)--(H) (M)--(N);
				\foreach \i/\g in {M/-90,H/-90,N/-50,I/180} \fill (\i) circle (1.2pt) node[shift={(\g:3mm)},scale=0.8]{$\i$};
			\end{tikzpicture}
		}
		\noindent
		Mặt khác $\overrightarrow{MN}$ cùng phương với véc-tơ $\overrightarrow{u}=\left(1;1;1\right)$ và $\overrightarrow{n_P}=(1;-2;2)$ suy ra 
		\[
		\sin \alpha=\dfrac{\left|\overrightarrow{u}\cdot \overrightarrow{n_P}\right|}{\left|\overrightarrow{u}\right|\cdot \left|\overrightarrow{n_P}\right|}=\dfrac{1}{3\sqrt{3}}.
		\]
		Khi đó $MN=\dfrac{NH}{\sin \alpha} \ge \dfrac{NH_{\min}}{\sin \alpha}=\dfrac{\mathrm{d}\left(I,(P)\right)-R}{\sin \alpha}=\dfrac{550-500}{\dfrac{1}{3\sqrt{3}}}=150\sqrt{3} \approx 260$.
	}
\end{ex}

\begin{ex}%[2H5V3-2]
	Một vỏ kem ốc quế là một loại bánh khô, hình nón $(N)$ trong không gian $Oxyz$, thường được làm bằng một chiếc bánh xốp dùng để đặt kem vào và cầm ăn mà không cần bát hoặc muỗng. Người ta thả vào vỏ kem $(N)$ một viên kem vani hình cầu có đính hai viên socola nhỏ tại hai vị trí $A(2; 1; 3)$ và $B(6; 5; 5)$ sao cho đường kính $AB$ có $B$ là tâm đường tròn đáy khối nón. Khi thể tích của khối nón $(N)$ nhỏ nhất thì mặt phẳng qua đỉnh $S$ của khối nón $(N)$ và song song với mặt phẳng chứa đường tròn đáy của $(N)$ có phương trình $2x+by+cz+d=0$. Tính giá trị của biểu thức $T=b+c+d$.\\
	\shortans[oly]{$12$}
	\loigiai{
		\immini{
			Gọi chiều cao khối nón $SB=h \left(h > 0\right)$ và bán kính đường tròn đáy $BC=R$.\\
			Ta có $V=\dfrac{1}{3} \pi R^2 h $. \hfill (1)\\
			Ta có $	\overrightarrow{AB}=(4; 4; 2)\Rightarrow AB=6$.\\
			Xét mặt cầu có đường kính $AB$.\\
			Bán kính là $r=\dfrac{AB}{2}=3$ và tâm $I(4; 3; 4)$.\\
			Vì $\triangle SHI$ đồng dạng với $\triangle SBC$ nên
			\[ 
			\dfrac{SI}{SC}=\dfrac{IH}{BC} \Leftrightarrow \dfrac{h-3}{\sqrt{h^2+R^2}}=\dfrac{3}{R}.
			\]
			\[\Leftrightarrow \dfrac{\left(h-3\right)^2}{h^2+R^2}=\dfrac{9}{R^2} \Leftrightarrow R^2\left[\left(h-3\right)^2-9\right]=9h^2\Leftrightarrow R^2=\dfrac{9h^2}{h^2-6h} \tag{2}.
			\]
			Thay $(2)$ vào $(1)$ ta có\\
			$V=\dfrac{1}{3} \pi\cdot \dfrac{9h^2}{h^2-6h}\cdot h=3\pi\cdot \dfrac{h^2}{h-6}$ với $h > 6$.
		}{
			\begin{tikzpicture}[scale=0.8, font=\footnotesize,line join=round, line cap=round, >=stealth, every node/.style={scale=0.8}]
				\def\a{3}
				\def\b{1.0}
				\pgfmathsetmacro\h{\a*sqrt(3)}
				\pgfmathsetmacro\r{\h/3}
				\pgfmathsetmacro\g{asin(\b/\h)}
				\pgfmathsetmacro\xo{\a *cos(\g)}
				\pgfmathsetmacro\yo{\b *sin(\g)}
				\begin{scope}[rotate = 180]
					\path 
					(\xo,\yo) coordinate (Mr) (0,0) coordinate (B) (90:\h) coordinate (S) (0:\a) coordinate (C)  (-\xo,\yo) coordinate (Ml) ($(B)!1/3!(S)$) coordinate (I) ($(B)!2!(I)$) coordinate (A);
					\coordinate (H) at ($(S)!(I)!(C)$);
					\draw[dash pattern=on 2pt off 1.5pt] (S)--(B) (I)--(H)
					(I) circle (\r) (I) ellipse ({\r} and {\r/3});
					\draw (Mr) arc (\g:180-\g:{\a} and {\b}) (S)--(Ml) arc(180-\g:360+\g:{\a} and {\b})--cycle (B)--(C);
					\foreach \x /\gN in {B/-90,S/90,C/0,I/180,A/-40,H/60}
					\fill[black] (\x) circle (1.2pt)($(\x)+(\gN:4mm)$) node {$\x$};
				\end{scope}
			\end{tikzpicture}
		}
		Xét $V'=3\pi\cdot \dfrac{2h\left(h-6\right)-h^2}{\left(h-6\right)^2}=3\pi\cdot \dfrac{h^2-12h}{\left(h-6\right)^2}$.\\
		Ta được BBT như sau:
		\begin{center}
			\begin{tikzpicture}[scale=0.8, font=\footnotesize,line join=round, line cap=round, >=stealth, every node/.style={scale=0.8}]
				\tkzTabInit[nocadre=true,lgt=1.2,espcl=2.5,deltacl=0.6]
				{$h$ /0.6,$V'$ /0.6,$V$ /2.5}
				{$-\infty$,$0$,$6$,$12$,$+\infty$}
				\tkzTabLine{,+,$0$,-,d,-,$0$,+,}
				\tkzTabVar{-/$-\infty$,+/$V(0)$,-D+/$-\infty$/$+\infty$,-/$V(12)$,+/$+\infty$}
				
			\end{tikzpicture}
		\end{center}
		Do $h > 6$ nên $V_{\min}$ khi $SB=h=12$ $\Rightarrow A$ là trung điểm của $SB$ $\Rightarrow S(-2; -3; 1)$.\\
		Khi đó $(P)$ đi qua $S$, vuông góc với $AB$ nên có một VTPT $\overrightarrow{n}=\overrightarrow{AB}=(4; 4; 2)$ hay $\overrightarrow{n}=(2; 2;1)$.\\
		Suy ra $(P)\colon  2(x+2)+2(y+3)+z-1=0\Leftrightarrow (P)\colon 2x+2y+z+9=0$.\\
		Vậy $T=b+c+d=2+1+9=12$.
	}
\end{ex}

\begin{ex}%[2H5V3-2]
	Trong không gian $Oxyz$, cho mặt cầu $(S)\colon x^2+y^2+z^2-2x-4y+6z-13=0$ và đường thẳng $d\colon \heva{& x=-1+t \\ & y=-2+t \\ & z=1+t}$. Gọi $M(a; b; c)$ với $a < 0$ là điểm thuộc đường thẳng $d$ sao cho từ $M$ kẻ được ba tiếp tuyến $MA$, $MB$, $MC$ đến mặc cầu $(S)$ ($A,B,C$ là các tiếp điểm) thỏa mãn $\widehat{AMB}=60^{\circ}$; $\widehat{BMC}=90^{\circ}$; $\widehat{CMA}=120^{\circ}$. Tính giá trị của biểu thức $P=a+b+c$.\\
	\shortans[oly]{$-2$}
	\loigiai{
		\immini{
			Ta có mặt cầu $(S)$ có tâm $I\left(1;2;-3\right)$, bán kính $R=3\sqrt{3}$.\\
			Đặt $MA=MB=MC=a$.\\
			Tam giác $MAB$ đều nên $AB=a$.\\
			Tam giác $MBC$ vuông cân tại $M$ nên $BC=a\sqrt{2}$.\\
			Tam giác $MCA$ có $\widehat{CMA}=120^{\circ}$ nên $CA=a\sqrt{3}$.\\
			Xét tam giác $ABC$ có $AB^2+BC^2=AC^2$ nên tam giác $ABC$ vuông tại $B$ hay tam giác $ABC$ nội tiếp đường tròn đường kính $AC$ và $AH=\dfrac{1}{2} AC=\dfrac{a\sqrt{3}}{2}$.
		}{
			\begin{tikzpicture}[scale=0.7, line join=round, line cap=round,>=stealth]
				\coordinate (H) at (0,0);
				\coordinate (I) at (0,-3);
				\coordinate (M) at (0,5);
				%\draw (H) ellipse (4 cm and 1.5 cm);
				\coordinate (A) at ($(H) + (110:4 and 1.5)$);
				\coordinate (C) at ($(H) + (-30:4 and 1.5)$);
				\coordinate (B) at ($(H) + (-150:4 and 1.5)$);
				\coordinate (E) at ($(H) + (-90:4 and 1.5)$);
				\draw pic[draw,angle radius=0.3cm]{right angle=I--B--M};
				\draw pic[draw,angle radius=0.3cm]{right angle=I--A--M};
				\draw pic[draw,angle radius=0.3cm]{right angle=M--C--I};
				\draw  (B)--(M)--(C) (I)--(B) (I)--(C) (B)--(C);
				\draw[dashed] (A)--(B) (C)--(A) (H)--(A) (H)--(B) (H)--(C) (A)--(M)--(H) (I)--(A) (H)--(I);
				\foreach \i/\g in {A/130,B/-130,C/0,M/90,I/-90,H/45}
				\fill[black] (\i) circle(1pt)+(\g:3mm)node[scale=0.7]{$\i$};
			\end{tikzpicture}
		}	
		\noindent
		Xét tam giác vuông $IAM$ có
		\[
		\dfrac{1}{AH^2}=\dfrac{1}{AM^2}+\dfrac{1}{AI^2} \Rightarrow MA=a=3\Rightarrow IM^2=AM^2+AI^2=36.
		\]
		Mặt khác $M\in (d)\Rightarrow M\left(-1+t;-2+t;1+t\right)$.\\
		Hay $\left(t-2\right)^2+\left(t-4\right)^2+\left(t+4\right)^2=36$ $\Leftrightarrow 3t^2-4t=0\Leftrightarrow \hoac{ & t=0 \\ & t=\dfrac{4}{3}.}$\\
		Suy ra $M\left(-1;-2;1\right)$ hay $M\left(\dfrac{1}{3};-\dfrac{2}{3};\dfrac{7}{3} \right)$.\\
		Vì giả thiết cho $a < 0$ nên $a=-1;b=-2;c=1$. Khi đó $P=a+b+c=-2$.
	}
\end{ex}

\begin{ex}%[2H5V3-4]
	Trong không gian $Oxyz$ (đơn vị của các trục tọa độ là ki--lô-mét), một trạm thu phát sóng điện thoại di động có đầu thu đặt tại điểm $I\left(1;2;2\right)$ biết rằng bán kính phủ sóng của trạm là $3$ km. Hai người sử dụng điện thoại lần lượt tại $M\left(4;-4;2\right)$ và $N\left(6;0;6\right)$. Gọi $E(a; b; c)$ với $a < 0$ là một điểm thuộc ranh giới vùng phủ sóng của trạm sao cho tổng khoảng cách từ $E$ đến vị trí $M$ và $N$ lớn nhất. Tính $T=a+b+c$.\\
	\shortans[oly]{$4$}
	\loigiai{
		\immini{
			Xét mặt cầu $(S)$ có tâm $I\left(1;2;2\right)$, bán kính $R=3$.\\
			Ta có $IM=IN=3\sqrt{3} > R$ nên cả hai điểm $M$, $N$ đều nằm ngoài mặt cầu $(S)$.\\
			Gọi $H$ là trung điểm của $MN$ suy ra $H\left(5;-2;4\right)$ và 
			\[
			EH^2=\dfrac{EM^2+EN^2}{2}-\dfrac{MN^2}{4}.
			\]
			Ta có $\left(EM+EN\right)^2\le 2\left(EM^2+EN^2\right)=2\left(EH^2+\dfrac{MN^2}{4} \right)$.
		}{
			\begin{tikzpicture}[scale=0.6, font=\footnotesize,line join=round, line cap=round, >=stealth, every node/.style={scale=0.8}]
				\def\r{2.0}
				\def\x{2.0}
				\def\y{0.8}
				\path
				(0,0) coordinate (I)
				(-4,0) coordinate (M)
				(-\r,0) coordinate (A)
				(\r,0) coordinate (B)
				($(I)+({\x*cos(-120)},{\y*sin(-120)})$) coordinate (E)
				($(I)+({2.5*\x*cos(-120)},{2.5*\y*sin(-120)})$) coordinate (H)
				($(E)!2!(I)$) coordinate (E')
				($(M)!2!(H)$) coordinate (N)
				;
				\draw (I) circle(\r)
				(B) arc (0:-180:\x cm and \y cm)
				(E)--(H) (M)--(N)
				;
				\draw[dashed] (B) arc (0:180:\x cm and \y cm)
				(I)--(B)node[below,midway]{$R$} (E')--(I)--(E);
				\foreach \i/\g in {I/90,E/-90,E'/90,H/-120,M/90,N/-90}
				\fill[black] (\i) circle(1pt)+(\g:3mm)node[scale=1]{$\i$};
			\end{tikzpicture}
		}
		\noindent
		Khi đó $\left(EM+EN\right)$ lớn nhất khi $EH$ lớn nhất khi và chỉ khi $E$ là giao điểm của $IH$ và mặt cầu $(S)$.\\
		$IH$ có phương trình là $\heva{& x=1+2t \\ & y=2-2t \\ & z=2+t.}$ \hfill (1)\\
		Mặt cầu $(S)$ có phương trình $\left(x-1\right)^2+\left(y-2\right)^2+\left(z-2\right)^2=9.$ \hfill (2)\\
		Thay (1) vào (2), ta được
		\[(2t)^2+\left(-2t\right)^2+t^2=9\Leftrightarrow t=\pm 1.
		\]
		Suy ra có hai điểm $E'\left(3;0;3\right)$ hoặc $E\left(-1;4;1\right)$. \\
		Vì $a < 0$ nên $E\left(-1;4;1\right)$ hay $T=a+b+c=4$.
	}
\end{ex}

\begin{ex}%[2H5H3-4]
	\immini[thm]{Người ta muốn thiết kế một bồn chứa khí hoá lỏng hình cầu bằng phần mềm 3D (tham khảo hình vẽ). Cho biết phương trình bề mặt của bồn chứa là $(S)\colon  \left(x-2\right)^2+y^2+\left(z+1\right)^2=1$. Phương trình mặt phẳng chứa nắp là $(P)\colon  z-6=0$. Tính khoảng cách từ tâm bồn chứa đến mặt phẳng chứa nắp.\\
	\shortans[oly]{$7$}
}{
\includegraphics[scale=0.7]{image/bonkhi}}
	\loigiai{
		Tâm của bồn chứa $I\left(2;0;-1\right)$.\\
		Khoảng cách từ tâm bồn chứa đến mặt phẳng chứa nắp là
		\[
		\mathrm{d}\left(I,(P)\right)=\dfrac{\left|-1-6\right|}{\sqrt{1^2}}=7.
		\]
	}
\end{ex}

\centerline{---HẾT---}
\Closesolutionfile{ans}
%\newpage
%%=====================
%\begin{center}
%\textbf{\large BẢNG ĐÁP ÁN}
%\end{center}
%\noindent\textbf{ĐÁP ÁN PHẦN I}
%\inputansbox{10}{ans/B4-De2-1}
	
%\noindent\textbf{ĐÁP ÁN PHẦN II}
%\inputansbox[2]{2}{ans/B4-De2-2}
	
%\noindent\textbf{ĐÁP ÁN PHẦN III}
%\inputansbox[3]{6}{ans/B4-De2-3}




%Chuong VI. Xs
% \setcounter{section}{0}
\section{XÁC SUẤT CÓ ĐIỀU KIỆN}
%%%%%%%%%%%%%%%%
\subsection{Trọng tâm kiến thức}
\begin{tomtat}
	\subsubsection{Xác suất có điều kiện}
	\begin{boxdn}
	\begin{itemize}
	\item
	Cho hai biến cố $A$ và $B$. Xác suất của biến cố $A$, tính trong điều kiện biết rằng biến cố $B$ đã xảy ra, được gọi là xác suất của $A$ với điều kiện $B$ và kí hiệu là $\mathrm{P}(A\mid B)$.
	\item
	Cho hai biến cố $A$ và $B$ bất kì, với $\mathrm{P}(B)>0$. Khi đó
	$$\mathrm{P}(A \mid B)=\dfrac{\mathrm{P}(A B)}{\mathrm{P}(B)}.$$
	\end{itemize}
	\end{boxdn}
	\subsubsection{Công thức nhân xác suất}
	\begin{boxdn}
	Vậy với hai biến cố $A$ và $B$ bất kì, ta có
	$$\mathrm{P}(A B)=\mathrm{P}(B) \cdot \mathrm{P}(A \mid B).$$
	Công thức trên được gọi là \textbf{\textit{công thức nhân xác suất}}.
	\end{boxdn}
	\begin{note}
	\begin{itemize}
	\item Vì $AB=BA$ nên với hai biến cố $A$ và $B$ bất kì, ta cũng có
	$$\mathrm{P}(A B)=\mathrm{P}(A) \cdot \mathrm{P}(B \mid A) \text{. }$$
	\item Nếu $A$ và $B$ là hai biến cố độc lập thì
	$$\mathrm{P}(A B)=\mathrm{P}(A) \cdot \mathrm{P}(B).$$
	\end{itemize}
	\end{note}
\end{tomtat}
%%%%%%%%%%%%%%
\subsection{Các dạng bài tập}
%============================
\begin{dang}{Tính xác suất có điều kiện theo định nghĩa}
%	\begin{itemize}
%	\item Cho hai biến cố $A$ và $B$. Xác suất của biến cố $B$ khi biến cố $A$ đã xảy ra được gọi là \textbf{xác suất của $B$ với điều kiện $A$}, kí hiệu là $\mathrm{P}(B \mid A)$.
%	\item Sử dụng định nghĩa để tính xác suất có điều kiện (áp dụng với các bài có thể tìm được số phần tử của các biến cố).
%	\end{itemize}
\end{dang}
%----------------------------
\subsubsection{Ví dụ minh hoạ}
\begin{vd}%[2D5H1-2]
	Có hai hộp chứa các thẻ được đánh số. Hộp thứ nhất có các thẻ được đánh số từ $1$ đến $4$, hộp thứ hai có các thẻ được đánh số từ $5$ đến $6$. Các thẻ có cùng kích thước và khối lượng. Bạn Phương lấy ngẫu nhiên một thẻ từ hộp thứ nhất bỏ vào hộp thứ hai. Sau đó bạn lại lấy ngẫu nhiên một thẻ từ hộp thứ hai. Liệt kê các kết quả của phép thử biết lần thứ nhất bạn Phương lấy được một thẻ đánh số chẵn.
	\loigiai{
	Vì đã biết lần thứ nhất bạn Phương lấy được một thẻ đánh số chẵn. Nghĩa là lúc đó bạn Phương có thể lấy được thẻ đánh số $2$ hoặc $4$.\\
	Nếu bạn Phương lấy được thẻ đánh số $2$ và bỏ vào hộp thứ hai, thì lúc này trong hộp thứ hai có các thẻ đánh số từ $5$ đến $6$ và $2$. Do đó ta có các khả năng $(2;5),(2;6),(2;7),(2;2)$.\\
	Tương tự như vậy nếu bạn Phương lấy được thẻ đánh số $4$, ta có các khả năng $(4;5),(4;6),(4,7),(4,4)$.\\
	Vậy các kết quả của phép thử biết lần thứ nhất bạn Phương lấy được một thẻ đánh số chẵn là 
	$$(2;5),(2;6),(2;7),(2;2),(4;5),(4;6),(4,7),(4,4).$$
	}
\end{vd}
\begin{vd}%[2D5H1-2]
	Một hộp có $5$ viên bi cùng kích thước và khối lượng, trong đó có $3$ viên bi màu đỏ và $2$ viên bi màu xanh. Lấy ngẫu nhiên lần lượt $2$ viên bi và không hoàn lại. Tính xác suất để lấy được viên bi thứ hai có màu xanh, biết rằng viên bi thứ nhất có màu đỏ.
	\loigiai{
	Gọi
	\begin{itemize}
	\item $A$ là biến cố \lq\lq  Lấy được viên bi thứ hai có màu xanh\rq\rq;
	\item $B$ là biến cố \lq\lq  Lấy được viên bi thứ nhất có màu đỏ\rq\rq.
	\end{itemize}
	Khi đó xác suất để lấy được viên bi thứ hai có màu xanh, biết rằng viên bi thứ nhất có màu đỏ chính là xác suất của $A$ với điều kiện $B$.\\
	Vì một viên bi đỏ đã được lấy ra ở lần thứ nhất nên trong hộp còn lại $4$ viên bi, trong đó có $2$ viên bi xanh.\\
	Từ đó ta có $\mathrm{P}(A \mid B)=\dfrac{2}{4}=0{,}5$.\\
	Vậy xác suất để lấy được viên bi thứ hai có màu xanh, biết rằng viên bi thứ nhất có màu đỏ là $0{,}5$.
	}	
\end{vd}
\begin{vd}%[2D5H2-2]
	Một hộp có $20$ viên bi trắng và $10$ viên bi đen, các viên bi có cùng kích thước và khối lượng. Bạn Bình lấy ngẫu nhiên một viên bi trong hộp, không trả lại. Sau đó bạn An lấy ngẫu nhiên một viên bi trong hộp đó.\\
	Gọi $A$ là biến cố: ``An lấy được viên bi trắng''; $B$ là biến cố: ``Bình lấy được viên bi trắng''. Tính $\mathrm{P}(A\mid B)$, $P\left(A\mid\overline{B}\right)$.
	\loigiai{
		Nếu $B$ xảy ra tức là Bình lấy được viên bi trắng. \\
		Khi đó, trong hộp còn lại $29$ viên bi với $19$ viên bi trắng và $10$ viên bi đen.\\ 
		Vậy $\mathrm{P}(A\mid B)=\dfrac{19}{29}\approx 0{,}67$.\\
		Nếu $\overline{B}$ xảy ra tức là Bình lấy được viên bi đen.\\
	Khi đó trong hộp còn lại $29$ viên bi với $20$ viên bi trắng và $9$ viên bi đen.\\
	Vậy $\mathrm{P}(A\mid\overline{B})=\dfrac{20}{29}$.
	}
\end{vd}
\begin{vd}%[2D5H2-2]
	Câu lạc bộ cờ của nhà trường gồm $35$ thành viên, mỗi thành viên biết chơi ít nhất một trong hai môn cờ vua hoặc cờ tướng. Biết rằng có $25$ thành viên biết chơi cờ vua và $20$ thành viên biết chơi cờ tướng. Chọn ngẫu nhiên $1$ thành viên của câu lạc bộ. Tính xác suất thành viên được chọn biết chơi cờ vua, biết rằng thành viên đó biết chơi cờ tướng.
	\loigiai{
	Gọi $A$ là biến cố \lq\lq  Thành viên được chọn biết chơi cờ tướng\rq\rq \,và $B$ là biến cố \lq\lq  Thành viên được chọn biết chơi cờ vua\rq\rq.\\
	Số thành viên của câu lạc bộ biết chơi cả hai môn cờ là $20+25-35=10$.\\
	Do đó, trong số $20$ thành viên biết chơi cờ tướng, có đúng $10$ thành viên biết chơi cờ vua.\\ 
	Vậy nên xác suất thành viên được chọn biết chơi cờ vua, biết rằng thành viên đó biết chơi cờ tướng là 
	$$\mathrm{P}(B \mid A)=\dfrac{10}{20}=0{.}5.$$
	}
\end{vd}
\begin{vd}%[2D5H1-2]
	Hộp thứ nhất chứa $4$ viên bi xanh và $3$ viên bi đỏ. Hộp thứ hai chứa $3$ viên bi xanh và $5$ viên bi đỏ. Các viên bi có cùng kích thước và khối lượng. Bạn Thanh lấy ra ngẫu nhiên $1$ viên bi từ hộp thứ nhất bỏ vào hộp thứ hai, sau đó lại lấy ra ngẫu nhiên $1$ viên bi từ hộp thứ hai. Tính xác suất để viên bi lấy ra ở lần thứ hai là viên bi đỏ, biết viên bi lấy ra ở lần thứ nhất là viên bi đỏ.
	\loigiai{
	Gọi $A$ là biến cố \lq\lq  viên bi lấy ra lần thứ hai là viên bi đỏ\rq\rq; $B$ là biến cố \lq\lq  viên bi lấy ra lần thứ hai là viên bi đỏ\rq\rq.\\
	Biến cố $B$ xảy ra, nghĩa là lần thứ nhất lấy được viên bi đỏ và bỏ vào hộp thứ hai. Khi đó trong hộp thứ hai sẽ có $3$ viên bi xanh và $6$ viên bi đỏ.\\
	Vậy $\mathrm{P}(A \mid B)= \dfrac{6}{9}=\dfrac{2}{3}$.
	}
\end{vd}
%=========================
% \setcounter{subsubsection}{1}
\begin{dang}{Tính xác suất có điều kiện theo công thức}
	Nếu $\mathrm{P}(B)>0$ thì xác suất của biến cố $A$ với điều kiện $B$ được xác định bởi công thức
	$$\mathrm{P}(A \mid B)=\dfrac{\mathrm{P}(A B)}{\mathrm{P}(B)}.$$
\end{dang}
%----------------------------
\subsubsection{Ví dụ minh hoạ}
\begin{vd}%[2D5H1-2]
	Cho hai biến cố $A$, $B$ có $\mathrm{P}(A)=0,4$; $\mathrm{P}(B)=0,6$; $\mathrm{P}(AB)=0,2$. Tính các xác suất $\mathrm{P}(A\mid B)$ và $\mathrm{P}(B\mid A)$.
	\loigiai{
	Ta có
	\begin{itemize}
	\item $\mathrm{P}(A\mid B)=\dfrac{\mathrm{P}(AB)}{\mathrm{P}(B)}=\dfrac{0{,}2}{0{,}6}=\dfrac{1}{3}$.
	\item $\mathrm{P}(B|A)=\dfrac{\mathrm{P}(AB)}{\mathrm{P}(A)}=\dfrac{0{,}2}{0{,}4}=0{,}5$.
	\end{itemize}
	}
\end{vd}
\begin{vd}%[2D5H1-2]
	Cho hai biến độc lập $A,B$ với $\mathrm{P}(A)=0{,}8$. Tính $\mathrm{P}(A\mid B)$.
	\loigiai{
	Vì $A$ và $B$ là hai biến cố độc lập, do đó
	\[\mathrm{P}(A\mid B)=\dfrac{\mathrm{P}(AB)}{\mathrm{P}(B)}=\dfrac{\mathrm{P}(A)\cdot \mathrm{P}(B)}{\mathrm{P}(B)}=\mathrm{P}(A)=0{,}8.\]
	\begin{note} Ta có thể dùng công thức $\mathrm{P}\left(A\mid B\right)=\mathrm{P}(A)$ với $A$ và $B$ là hai biến cố độc lập.
	\end{note}
	}
\end{vd}
\begin{vd}%[2D5H2-2]
	Một lô sản phẩm có $20$ sản phẩm, trong đó có $5$ sản phẩm chất lượng thấp. Lấy liên tiếp $2$ sản phẩm trong lô sản phẩm trên, trong đó sản phẩm lấy ra ở lần thứ nhất không được bỏ lại vào lô sản phẩm. Tính xác suất để cả hai sản phẩm được lấy ra đều có chất lượng thấp.
	\loigiai{
	Gọi $A$ là biến cố \lq\lq  sản phẩm thứ nhất có chất lượng thấp\rq\rq, và $B$ là biến cố \lq\lq  sản phẩm thứ hai có chất lượng thấp\rq\rq.\\
	Xác suất của $A$ là xác suất để lấy ra một sản phẩm chất lượng thấp trong lần đầu tiên:
	$$\mathrm{P}(A)=\dfrac{n(A)}{n(\Omega)}=\dfrac{5}{20}=\dfrac{1}{4}.$$
	Sau khi lấy một sản phẩm chất lượng thấp, số sản phẩm chất lượng thấp giảm còn $4$ trong tổng số $19$ sản phẩm.\\
	Xác suất của $B$ khi đã xảy ra $A$ là xác suất để lấy ra một sản phẩm chất lượng thấp trong lần thứ hai:
	$$\mathrm{P}(B \mid A)=\dfrac{4}{19}.$$
	Áp dụng quy tắc nhân xác suất:
	$$\mathrm{P}(AB)=\mathrm{P}(A)\cdot \mathrm{P}(B \mid A)=\dfrac{1}{4}\cdot\dfrac{4}{19}=\dfrac{1}{19}.$$
	}
\end{vd}
\begin{vd}%[2D5H2-2]
	Trong cuộc khảo sát $300$ gia đình ở một khu vực, người ta nhận thấy rằng có $90\%$ gia đình có tivi và $60\%$ gia đình có máy tính bàn. Mỗi gia đình đều có ít nhất một trong hai thiết bị này. Chọn ngẫu nhiên một gia đình. Tính xác suất gia đình có máy tính bàn trong nhóm các gia đình có tivi.
	\loigiai{
	Gọi $A$ là biến cố \lq\lq  Gia đình được chọn có máy tính bàn\rq\rq; $B$ là biến cố \lq\lq  Gia đình được chọn có tivi\rq\rq. Khi đó $AB$ là biến cố \lq\lq  Gia đình được chọn có cả máy tính bàn và tivi\rq\rq. \\
	Ta có $n(B)=0{,}9\cdot300=270$ và $n(AB)=0{,}9\cdot300+0{,}6\cdot300-300=150$. \\
	Do đó $\mathrm{P}(A\mid B)=\dfrac{n\left(A\cap B\right)}{n(B)}=\dfrac{150}{270}=\dfrac{5}{9}$.
	}
\end{vd}
%----------------------------
\subsubsection{Bài tập áp dụng}

\begin{bt}%[2D5H2-2]
	Một phòng nghiên cứu dược học cho $500$ người bị bệnh $H$ dùng hai loại thuốc $X, Y$ để điều trị. Một số người được điều trị bằng thuốc $X$ và số người còn lại được điều trị bằng thuốc $Y$. Kết quả nghiên cứu được trình bày ở bảng $6.2$.
	\begin{center}
	\begin{tikzpicture}
	\begin{scope}[xscale=4.4,yscale=0.85]
	\path
	(0,0) foreach \i[count=\k] in {$X$,$Y$} {++(1,0)node(1\k){\i}}
	(0,-1) node {Khỏi bệnh} foreach \i[count=\k] in {$180$,$190$} {++(1,0)node(2\k){\i}}
	(0,-2) node{Không khỏi bệnh} foreach \i[count=\k] in {$60$,$70$} {++(1,0)node(3\k){\i}}
	%(0,-3) node{tiện} foreach \i[count=\k] in {,$O_1$,,$O_4$,} {++(1,0)node(4\k){\i}}
	%(0,-4) node{điện} foreach \i[count=\k] in {,$O_1$,,$O_4$,} {++(1,0)node(5\k){\i}}
	%(0,-5) node{nước} foreach \i[count=\k] in {,$O_1$,,$O_4$,} {++(1,0)node(6\k){\i}}
	;
	%\path
	%(23.south east) node{$\times$}
	%;
	\draw[shift={(-0.5,.5)}] (0,0) grid (3.,-3)
	(0,0)--(1.,-1)
	(0,-1) node[above right]{Tình trạng}
	(1,0) node[below left]{Loại thuốc}
	;
	\end{scope}
	\end{tikzpicture}
	\end{center}
	Chọn ngẫu nhiên một người trong số này. Gọi $A$ là biến cố \lq\lq  Người được chọn khỏi bệnh\rq\rq, $B$ là biến cố \lq\lq  Người được chọn điều trị bằng thuốc $X$\rq\rq, $C$ là biến cố \lq\lq  Người được chọn điều trị bằng thuốc $Y$\rq\rq.
	\begin{listEX}
	\item Tính và giải thích ý nghĩa của $\mathrm{P}(A \mid B)$ và $\mathrm{P}(A \mid C)$.
	\item Có thể nói loại thuốc nào có hiệu quả hơn trong việc điều trị bệnh $H$?
	\end{listEX}
	\loigiai{
	\begin{listEX}
	\item $\mathrm{P}(A \mid B)=\dfrac{n\left(AB\right)}{n(B)}=\dfrac{180}{240}=\dfrac{3}{4}$ và $\mathrm{P}(A \mid C)=\dfrac{n\left(A\cap C\right)}{n(C)}=\dfrac{190}{260}=\dfrac{19}{26}$. \\
	Theo các kết quả trên, xác suất để một người khỏi bệnh khi được chọn điều trị bằng thuốc $X$ là $\dfrac{3}{4}$ và xác suất để một người khỏi bệnh khi được chọn điều trị bằng thuốc $Y$ là $\dfrac{19}{26}$.
	\item Do $\dfrac{3}{4}>\dfrac{19}{26}$ nên loại thuốc $X$ có hiệu quả hơn loại thuốc $Y$ trong việc điều trị bệnh $H$.
	\end{listEX}
	}
\end{bt}
\begin{bt}%[2D5H2-2]
	Một xí nghiệp dệt may có những dải của một loại vải đang sản xuất theo một quy trình đặc biệt. Những dải này có thể bị lỗi theo hai hướng: lỗi chiều dài và lỗi kết cấu. Thông qua đợt kiểm tra quy trình sản xuất, người ta thấy rằng có $10\%$ dải không đạt yêu cầu về chiều dài, $5\%$ dải không đạt yêu cầu và kết cấu và chỉ có $0{,}8\%$ dải không đạt yêu cầu về cả chiều dài và kết cấu.
	\begin{listEX}
	\item Nếu chọn ngẫu nhiên một dải từ quy trình này thì xác suất không đạt yêu cầu về kết cấu là bao nhiêu?
	\item Nếu một dải được chọn ngẫu nhiên từ quy trình này và phép đo nhanh xác định dải đó không đạt yêu cầu về chiều dài, tính xác suất để dải đó không đạt yêu cầu về kết cấu.
	\end{listEX}
	\loigiai{
	\begin{listEX}
	\item Nếu chọn ngẫu nhiên một dải từ quy trình này thì xác suất không đạt yêu cầu về kết cấu là $\dfrac{5}{100}+\dfrac{0{,}8}{100}=\dfrac{29}{500}$.
	\item Gọi $A$ là biến cố \lq\lq  Dải được chọn từ quy trình không đạt yêu cầu về kết cấu\rq\rq;\\
	$B$ là biến cố \lq\lq  Dải được chọn từ quy trình không đạt yêu cầu về chiều dài\rq\rq.\\
	Khi đó $AB$ là biến cố \lq\lq  Một dải từ quy trình không đạt yêu cầu về cả kết cấu và chiều dài\rq\rq. Ta có $\mathrm{P}(AB)=0{,}8\%=0{,}008$ và $\mathrm{P}(B)=0{,}1$. \\
	Do đó $\mathrm{P}(A\mid B)=\dfrac{\mathrm{P}\left(AB\right)}{\mathrm{P}(B)}=\dfrac{0{,}008}{0{,}1}=0{,}08$.
	\end{listEX}
	}
\end{bt}
\begin{bt}%[2D5H2-2]
	Trong một lọ có chứa bi đen và bi trắng cùng kích thước và khối lượng, lấy ngẫu nhiên lần lượt hai viên bi ra ngoài và không bỏ vào lại. Biết rằng xác suất để lần đầu lấy được bi đen là $0{,}47$; xác suất để lần đầu lấy được bi đen và lần thứ hai lấy được bi trắng là $0{,34}$. Tính xác suất để lấy được bi trắng ở lần thứ hai với điều kiện lần đầu lấy được bi đen.
	\loigiai{
	Gọi $A$ là biến cố \lq\lq  Lấy được bi trắng ở lần thứ hai\rq\rq; $B$ là biến cố \lq\lq  Lấy được bi đen ở lần đầu\rq\rq. \\
	Do đó $\mathrm{P}(A \mid B)=\dfrac{\mathrm{P}(AB)}{\mathrm{P}(B)}=\dfrac{0{,34}}{0{,47}}=\dfrac{34}{47}$.
	}
\end{bt}
%=======================
\begin{dang}{Tính xác suất có điều kiện nhờ sơ đồ hình cây}
	Trên sơ đồ hình cây:
	\begin{itemize}
	\item Xác suất của các nhánh trong sơ đồ hình cây từ đỉnh thứ hai là xác suất có điều kiện.
	\item Xác suất xảy ra của mỗi kết quả bằng tích các xác suất trên các nhánh của cây đi đến kết quả đó.
	\end{itemize}
\end{dang}
%----------------------------
\subsubsection{Ví dụ minh hoạ}
\begin{vd}%[2D5H2-3]
	Một hộp có $8$ bi màu đỏ và $5$ viên bi màu vàng; các viên bi có kích thước và khối lượng như nhau. Có $5$ viên bi trong hộp được đánh số, trong đó có $3$ viên bi màu đỏ và $2$ viên bi màu vàng. Lấy ngẫu nhiên một viên bi trong hộp. Dùng sơ đồ hình cây, tính xác suất để viên bi được lấy ra, có màu đỏ, biết rằng viên bi đó được đánh số.
	\loigiai{
	Xét các biến cố sau:
	\begin{itemize}
	\item $A \colon$ \lq\lq  Viên bi được lấy ra có đánh số\rq\rq.
	\item $B \colon$ \lq\lq  Viên bi được lấy ra có màu đỏ \rq\rq.
	\end{itemize}
	Khi đó, xác suất để viên bi được lấy ra có màu đỏ, biết ràng viên bi đó được đánh số, chính là xác suất có điều kiện $\mathrm{P}(B\mid A)$.\\
	Sơ đồ hình cây biểu thị cách tính xác suất có điều kiện $\mathrm{P}(B\mid A)$, được vẽ như sau:
	\begin{center}
	\begin{tikzpicture}[scale=0.8]
	\def\gocm{20}
	\def\gocn{10}
	\def\r{4}
	\tikzset{s/.style={outer sep=0.5 mm,draw=magenta,rectangle,minimum width=2.75cm,rounded corners=1mm}}
	\path(0,0)node(O){}++(\gocm:\r)node[s](A1){A}++(\gocn:\r)node[s](A2){$B$}++(0:\r)node[s](A3){$AB$};
	\path(A1)++({-\gocn}:\r)node[s](a2){$\overline{B}$}++(0:\r)node[s](a3){$A\overline{B}$};
	\path(O)++(-\gocm:\r)node[s](B1){$\overline{A}$}++(\gocn:\r)node[s](B2){$B$}++(0:\r)node[s](B3){$\overline{A}B$};
	\path(B1)++({-\gocn}:\r)node[s](b2){$\overline{B}$}++(0:\r)node[s](b3){$\overline{A}\overline{B}$};
	\foreach \x/\y in {
	O/A1,A1/A2,
	O/B1,B1/B2,
	A1/a2,
	B1/b2}
	\draw[-stealth](\x.east)--(\y.west);
	\path(O)--(A1.west)node[pos=0.5,above,sloped]{$\frac{5}{13}$}(O)--(B1.west)node[pos=0.5,below,sloped]{$\frac{8}{13}$}(B1.east)--(B2.west)node[pos=0.5,above,sloped]{$\frac{5}{8}$}(A1.east)--(A2.west)node[pos=0.5,above,sloped]{$\frac{3}{5}$}
	(A1.east)--(a2.west)node[pos=0.5,below,sloped]{$\frac{2}{5}$}
	(B1.east)--(b2.west)node[pos=0.5,below,sloped]{$\frac{3}{8}$};
	\end{tikzpicture}
	\end{center}
	Vậy xác suất để viên bi được lấy ra có màu đỏ, biết rằng viên bi đó có đánh số, là $ 0{,}6$.
	}
\end{vd}
\begin{vd}%[2D5H2-3]
	Ở một sân bay, người ta sử dụng một loại máy soi tự động phát hiện hàng cấm trong hành lí kí gửi. Máy phát chuông cảnh báo với $95 \%$ các kiện hành lí có chứa hàng cấm và $2 \%$ các kiện hành lí không chứa hàng cấm. Tỉ lệ các kiện hành lí có chứa hàng cấm là $1 \%$.\\
	Chọn ngẫu nhiên một kiện hành lí để soi bằng máy trên. Sử dụng sơ đồ hình cây, tính xác suất của các biến cố:\\
	M: \lq\lq  Kiện hành lí có chứa hàng cấm và máy phát chuông cảnh báo \rq\rq;
	N : \lq\lq  Kiện hành lí không chứa hàng cấm và máy phát chuông cảnh báo\rq\rq.
	\loigiai
	{
	Gọi $A$ là biến cố \lq\lq  Kiện hành lí có chứa hàng cấm\rq\rq và $B$ là biến cố \lq\lq  Máy phát chuông cành báo\rq\rq. Ta có
	$$
	\mathrm{P}(B \mid A)=0{,}95 ; \mathrm{P}(B \mid \overline{A})=0{,}02 ; \mathrm{P}(A)=0{,}01.
	$$
	Do đó $\mathrm{P}(\overline{A})=1-\mathrm{P}(A)=0{,}99 ; \mathrm{P}(\overline{B} \mid A)=1-\mathrm{P}(B \mid A)=0{,}05 ; \mathrm{P}(\overline{B} \mid \overline{A})=1-\mathrm{P}(B \mid \overline{A})=0{,}98$.
	Ta có sơ đồ hình cây như sau:
	\begin{center}
	\begin{tikzpicture}[yscale=0.7]
	\def\gocm{20}
	\def\gocn{10}
	\def\r{3.5}
	\tikzset{s/.style={outer sep=0.5 mm,draw=magenta,rectangle,minimum width=3cm,rounded corners=1mm}}
	\path(0,0)node(O){}++(\gocm:\r)node[s](A1){A}++(\gocn:\r)node[s](A2){$B$}++(0:\r)node[s](A3){$AB$}++(0:\r)node[s](A4){$0{,}0095$};
	\path(A1)++({-\gocn}:\r)node[s](a2){$\overline{B}$}++(0:\r)node[s](a3){$A\overline{B}$}++(0:\r)node[s](a4){$0{,}0005$};
	\path(O)++(-\gocm:\r)node[s](B1){$\overline{A}$}++(\gocn:\r)node[s](B2){$B$}++(0:\r)node[s](B3){$\overline{A}B$}++(0:\r)node[s](B4){$0{,}0198$};
	\path(B1)++({-\gocn}:\r)node[s](b2){$\overline{B}$}++(0:\r)node[s](b3){$\overline{A}\overline{B}$}++(0:\r)node[s](b4){$0{,}9702$};
	\foreach \x/\y in {
	O/A1,A1/A2,
	O/B1,B1/B2,
	A1/a2,
	B1/b2}
	\draw[-stealth](\x.east)--(\y.west);
	\path(O)--(A1.west)node[pos=0.5,above,sloped]{$0{,}01$}(O)--(B1.west)node[pos=0.5,below,sloped]{$0{,}99$}(B1.east)--(B2.west)node[pos=0.5,above,sloped]{\tiny$0{,}02$}(A1.east)--(A2.west)node[pos=0.5,above,sloped]{\tiny$0{,}95$}
	(A1.east)--(a2.west)node[pos=0.5,below,sloped]{\tiny$0{,}05$}
	(B1.east)--(b2.west)node[pos=0.5,below,sloped]{\tiny$0{,}98$};
	\end{tikzpicture}
	\end{center}
	Do $M=A B$ nên $\mathrm{P}(M)=\mathrm{P}(A B)=0{,}0095$.\\
	Do $N=\overline{A} B$ nên $\mathrm{P}(N)=\mathrm{P}(\overline{A} B)=0{,}0198$.
	}
\end{vd}
\begin{vd}%[2D5H2-3]
	Theo kết quả từ trạm nghiên cứu khí hậu tại địa phương $T$, xác suất để một ngày có gió là $0{,}6$; nếu ngày có gió thì xác suất có mưa là $0{,}4$; nếu ngày không có gió thì xác suất có mưa là $0{,}2$. Gọi $G$ là biến cố \lq\lq  Ngày có gió\rq\rq~ và $M$ là biến cố \la\la Ngày có mưa\rq\rq.
	\begin{listEX}
	\item Vẽ lại sơ đồ hình cây sau và điền vào ô? các giá trị xác suất tương ứng.
	\begin{center}
	\begin{tikzpicture}[node distance=1.5cm, every node/.style={fill=none}, align=center,scale=0.6]
	\definecolor{diagram_bg_green}{HTML}{d5e8d4}
	\definecolor{diagram_bg_blue}{HTML}{dae8fc}
	\definecolor{diagram_bg_pink}{HTML}{f8cecc}
	\definecolor{diagram_bd_green}{HTML}{82b366}
	\definecolor{diagram_bd_blue}{HTML}{7494c2}
	\definecolor{diagram_bd_pink}{HTML}{b85450}
	\tikzset{>={Latex[width=2mm,length=2mm]}, base/.style = {rectangle, rounded corners, draw=black, text centered, drop shadow={shadow xshift=0.6mm, shadow yshift=-0.6mm}},
	Style1/.style = {base, fill=diagram_bg_blue, draw=diagram_bd_blue},
	Style2/.style = {base, fill=diagram_bg_pink, draw=diagram_bd_pink},
	Style3/.style = {base, fill=diagram_bg_green, draw=diagram_bd_green},
	Style4/.style = {base, minimum width=2.5cm, fill=orange!15, draw=orange},
	}
	\node (B0) [Style1, text width=1cm] {Ngày};
	\node (B1) [Style2, right of=B0, xshift=3cm, yshift=1cm, text width=3.5cm] {Có gió $(G)$};
	\node (B3) [Style2, right of=B0, xshift=3cm, yshift=-1cm, text width=3.5cm] {Không có gió $(\overline{G})$};
	\node (B11) [Style3, right of=B1, xshift=6cm, yshift=0.5cm, text width=3cm] {Có mưa};
	\node (B13) [Style3, right of=B1, xshift=6cm, yshift=-0.5cm, text width=3cm] {Không có mưa};
	\node (B31) [Style3, right of=B3, xshift=6cm, yshift=0.5cm, text width=3cm] {Có mưa};
	\node (B33) [Style3, right of=B3, xshift=6cm, yshift=-0.5cm, text width=3cm] {Không có mưa};
	\draw[->] (B0) -- (B1.west)node[sloped,above,pos=0.5]{$\mathrm{P}(G)=?$};
	\draw[->] (B0) -- (B3.west)node[sloped,below,pos=0.5]{$\mathrm{P}(\overline{G})=?$};
	\draw[->] (B1) -- (B11.west)node[sloped,above,pos=0.5]{$\mathrm{P}(M\mid G)=?$};
	\draw[->] (B1) -- (B13.west)node[sloped,below,pos=0.5]{$\mathrm{P}(\overline{M}\mid G)=?$};
	\draw[->] (B3) -- (B31.west)node[sloped,above,pos=0.5]{$\mathrm{P}(M\mid\overline{G})=?$};
	\draw[->] (B3) -- (B33.west)node[sloped,below,pos=0.5]{$\mathrm{P}(\overline{M}\mid\overline{G})=?$};
	\end{tikzpicture}
	\end{center}
	\item Tính xác suất $\mathrm{P}(G M)$ và $\mathrm{P}(G \overline{M})$. Nêu ý nghĩa của các xác suất này.
	\end{listEX}
	\loigiai{
	\begin{listEX}
	\item Theo đề bài, nếu ngày có gió thì xác suất có mưa là 0{,}4 nên $\mathrm{P}(M \mid G)=0{,}4$.\\
	Suy ra $\mathrm{P}(\overline{M} \mid G)=1-0{,}4=0{,}6$.
	Ngày không có gió thì xác suất có mưa là $0{,}2$ nên $\mathrm{P}(M \mid \overline{G})=0{,}2$.\\
	Suy ra $\mathrm{P}(\overline{M} \mid \overline{G})=1-0{,}2=0{,}8$.
	\begin{center}
	\begin{tikzpicture}[node distance=1.5cm, every node/.style={fill=none}, align=center,scale=0.6]
	\definecolor{diagram_bg_green}{HTML}{d5e8d4}
	\definecolor{diagram_bg_blue}{HTML}{dae8fc}
	\definecolor{diagram_bg_pink}{HTML}{f8cecc}
	\definecolor{diagram_bd_green}{HTML}{82b366}
	\definecolor{diagram_bd_blue}{HTML}{7494c2}
	\definecolor{diagram_bd_pink}{HTML}{b85450}
	\tikzset{>={Latex[width=2mm,length=2mm]}, base/.style = {rectangle, rounded corners, draw=black, minimum width=1cm, text centered, drop shadow={shadow xshift=0.6mm, shadow yshift=-0.6mm}},
	Style1/.style = {base, fill=diagram_bg_blue, draw=diagram_bd_blue},
	Style2/.style = {base, fill=diagram_bg_pink, draw=diagram_bd_pink},
	Style3/.style = {base, fill=diagram_bg_green, draw=diagram_bd_green},
	Style4/.style = {base, minimum width=2.5cm, fill=orange!15, draw=orange},
	}
	\node (B0) [Style1, text width=1cm] {Ngày};
	\node (B1) [Style2, right of=B0, xshift=3cm, yshift=1cm, text width=3cm] {Có gió $(G)$};
	\node (B3) [Style2, right of=B0, xshift=3cm, yshift=-1cm, text width=3cm] {Không có gió $(\overline{G})$};
	\node (B11) [Style3, right of=B1, xshift=6cm, yshift=0.5cm, text width=4cm] {Có mưa};
	\node (B13) [Style3, right of=B1, xshift=6cm, yshift=-0.5cm, text width=4cm] {Không có mưa};
	\node (B31) [Style3, right of=B3, xshift=6cm, yshift=0.5cm, text width=4cm] {Có mưa};
	\node (B33) [Style3, right of=B3, xshift=6cm, yshift=-0.5cm, text width=4cm] {Không có mưa};
	\draw[->] (B0) -- (B1.west)node[sloped,above,pos=0.5]{$\mathrm{P}(G)=0{,}6$};
	\draw[->] (B0) -- (B3.west)node[sloped,below,pos=0.5]{$\mathrm{P}(\overline{G})=0{,}4$};
	\draw[->] (B1) -- (B11.west)node[sloped,above,pos=0.5]{$\mathrm{P}(M\mid G)=0{,}4$};
	\draw[->] (B1) -- (B13.west)node[sloped,below,pos=0.5]{$\mathrm{P}(\overline{M}\mid G)=0{,}6$};
	\draw[->] (B3) -- (B31.west)node[sloped,above,pos=0.5]{$\mathrm{P}(M\mid\overline{G})=0{,}2$};
	\draw[->] (B3) -- (B33.west)node[sloped,below,pos=0.5]{$\mathrm{P}(\overline{M}\mid\overline{G})=0{,}8$};
	\end{tikzpicture}
	\end{center}
	\item $\mathrm{P}(G M)=\mathrm{P}(G) \cdot \mathrm{P}(M \mid G)=0{,}6 \cdot 0{,}4=0{,}24 $.\\
	$\mathrm{P}(G \overline{M})=\mathrm{P}(G) \cdot \mathrm{P}(\overline{M} \mid G)=0{,}6 \cdot 0{,}6=0{,}36$.\\
	Điểu này có nghĩa là tại địa phương $T$, trong một ngày, xác suất để trời vừa có gió và vừa có mưa là $0{,}24$; xác suất để trời có gió nhưng không có mưa là $0{,}36$.
	\end{listEX}
	}
\end{vd}
%----------------------------
\subsubsection{Bài tập áp dụng}
\begin{bt}%[2D5H2-3]
	Bạn Việt chuẩn bị đi tham quan một hòn đảo trong hai ngày thứ Bảy và Chủ nhật. Ở hòn đảo đó, mỗi ngày chỉ có nắng hoặc mưa, nếu một ngày là nắng thì khả năng xảy ra mưa ở ngày tiếp theo là $20 \%$, còn nếu một ngày là mưa thì khả năng ngày hôm sau vẫn mưa là $30 \%$. Theo dự báo thời tiết, xác suất trời sẽ nắng vào thứ Bảy là $0{,}7$.
	Hãy tìm các giá trị thích hợp thay vào \mbox{?} ở sơ đồ hình cây sau:
	\begin{center}
	\begin{tikzpicture}[yscale=0.7]
	\def\gocm{20}
	\def\gocn{10}
	\def\r{4}
	\tikzset{s/.style={outer sep=0.5 mm,draw=magenta,rectangle,minimum width=2.75cm,rounded corners=1mm}}
	\path(0,0)node(O){}++(\gocm:\r)node[s](A1){Nắng}++(\gocn:\r)node[s](A2){Nắng};
	\path(A1)++({-\gocn}:\r)node[s](a2){Mưa};
	\path(O)++(-\gocm:\r)node[s](B1){Mưa}++(\gocn:\r)node[s](B2){Nắng};
	\path(B1)++({-\gocn}:\r)node[s](b2){Mưa};
	\foreach \x/\y in {
	O/A1,A1/A2,
	O/B1,B1/B2,
	A1/a2,
	B1/b2}
	\draw[-stealth](\x.east)--(\y.west);
	\path(O)--(A1.west)node[pos=0.5,above,sloped]{$\mbox{0{,}7}$}(O)--(B1.west)node[pos=0.5,below]{$\mbox{?}$}(B1.east)--(B2.west)node[pos=0.5,above]{$\mbox{?}$}(A1.east)--(A2.west)node[pos=0.5,above]{$\mbox{?}$}
	(A1.east)--(a2.west)node[pos=0.5,below,sloped]{$\mbox{0{,}2}$}
	(B1.east)--(b2.west)node[pos=0.5,below,sloped]{$\mbox{0{,}3}$};
	%%Node dòng trên
	\path(A2)++(0,1)node{\textbf{Chủ nhật}}++(180:4)node{\textbf{Thứ bảy}};
	\end{tikzpicture}
	\end{center}
	\loigiai{
	Gọi $A$ là biến cố \lq\lq  Ngày thứ Bảy trời nắng\rq\rq và $B$ là biến cố \lq\lq  Ngày Chủ nhật trời mưa\rq\rq.\\
	Ta có $\mathrm{P}(A)=0{,}7 ; \mathrm{P}(B \mid A)=0{,}2 ; \mathrm{P}(B \mid \overline{A})=0{,}3$.\\
	Do đó $\mathrm{P}(\overline{A})=1-\mathrm{P}(A)=0{,}3 ;\, \mathrm{P}(\overline{B} \mid A)=1-\mathrm{P}(B \mid A)=0{,}8 ;\, \mathrm{P}(\overline{B} \mid \overline{A})=1-\mathrm{P}(B \mid \overline{A})=0{,}7$.
	Áp dụng công thức nhân xác suất, ta có xác suất trời nắng vào thứ Bảy và trời mưa vào Chủ nhật là
	$$
	\mathrm{P}(A B)=\mathrm{P}(A) \mathrm{P}(B \mid A)=0{,}7\cdot 0{,}2=0{,}14 .
	$$
	Tương tự, ta có
	\allowdisplaybreaks
	\begin{eqnarray*}
	&&\mathrm{P}(A \overline{B})=\mathrm{P}(A) \mathrm{P}(\overline{B} \mid A)=0{,}7\cdot 0{,}8=0{,}56 ; \\
	&&\mathrm{P}(\overline{A} B)=\mathrm{P}(\overline{A}) \mathrm{P}(B \mid \overline{A})=0{,}3\cdot 0{,}3=0{,}09 ; \\
	&&\mathrm{P}(\overline{A} \overline{B})=\mathrm{P}(\overline{A}) \mathrm{P}(\overline{B} \mid \overline{A})=0{,}3\cdot0{,}7=0{,}21.
	\end{eqnarray*}
	Ta có thể biểu diễn các kết quả trên theo sơ đồ hình cây như sau:
	\begin{center}
	\begin{tikzpicture}[yscale=0.7]
	\def\gocm{20}
	\def\gocn{10}
	\def\r{4}
	\tikzset{s/.style={outer sep=0.5 mm,draw=magenta,rectangle,minimum width=2.75cm,rounded corners=1mm}}
	\path(0,0)node(O){}++(\gocm:\r)node[s](A1){A}++(\gocn:\r)node[s](A2){$\overline{B}$}++(0:\r)node[s](A3){$A\overline{B}$}++(0:\r)node[s](A4){$0{,}56$};
	\path(A1)++({-\gocn}:\r)node[s](a2){B}++(0:\r)node[s](a3){$AB$}++(0:\r)node[s](a4){$0{,}14$};
	\path(O)++(-\gocm:\r)node[s](B1){$\overline{A}$}++(\gocn:\r)node[s](B2){$\overline{B}$}++(0:\r)node[s](B3){$\overline{A}\overline{B}$}++(0:\r)node[s](B4){$0{,}21$};
	\path(B1)++({-\gocn}:\r)node[s](b2){$B$}++(0:\r)node[s](b3){$\overline{A}B$}++(0:\r)node[s](b4){$0{,}09$};
	\foreach \x/\y in {
	O/A1,A1/A2,
	O/B1,B1/B2,
	A1/a2,
	B1/b2}
	\draw[-stealth](\x.east)--(\y.west);
	\path(O)--(A1.west)node[pos=0.5,above,sloped]{$0{,}7$}(O)--(B1.west)node[pos=0.5,below,sloped]{$0{,}3$}(B1.east)--(B2.west)node[pos=0.5,above,sloped]{$0{,}7$}(A1.east)--(A2.west)node[pos=0.5,above,sloped]{$0{,}8$}
	(A1.east)--(a2.west)node[pos=0.5,below,sloped]{$0{,}2$}
	(B1.east)--(b2.west)node[pos=0.5,below,sloped]{$0{,}3$};
	%%Node dòng trên
	\path(A2)++(0,1)node{\textbf{Chủ nhật}}++(180:4)node{\textbf{Thứ bảy}}(A3)++(0,1)node{\textbf{Kết quả}}(A4)++(0,1)node{\textbf{Xác suất}};
	\end{tikzpicture}
	\end{center}}
\end{bt}
\begin{bt}%[2D5H2-3]
	Hộp thứ nhất có $4$ viên bi xanh và $6$ viên bi đỏ. Hộp thứ hai có $5$ viên bi xanh và $4$ viên bi đỏ. Các viên bi có cùng kích thước và khối lượng. Lấy ra ngẫu nhiên $1$ viên bi từ hộp thứ nhất chuyển sang hộp thứ hai. Sau đó lại lấy ra ngẫu nhiên $1$ viên bi từ hộp thứ hai.
	Sử dụng sơ đồ hình cây, tính xác suất của các biến cố:
	$A$ : \lq\lq  Viên bi lấy ra từ hộp thứ nhất có màu xanh và viên bi lấy ra từ hộp thứ hai có màu đỏ\rq\rq ;
	$B$ : \lq\lq  Hai viên bi lấy ra có cùng màu\rq\rq.
	\loigiai{
	Gọi $X$ là biến cố: \lq\lq  Viên bi lấy ra từ hộp thứ nhất có màu xanh\rq\rq.\\
	$Y$ là biến cố: \lq\lq  Viên bi lấy ra từ hộp thứ hai có màu đỏ\rq\rq.\\
	Ta có
	$
	\mathrm{P}(Y|X)=0{,}4; \mathrm{P}(Y \mid \overline{X})=0{,}5 ; \mathrm{P}(X)=0{,}4
	$.\\
	Do đó $\mathrm{P}(\overline{X})=1-\mathrm{P}(X)=0{,}6 ; \mathrm{P}(\overline{Y} |X)=1-\mathrm{P}(Y|X)=0{,}6 ; \\\mathrm{P}(\overline{Y} \mid \overline{X})=1-\mathrm{P}(Y \mid \overline{X})=0{,}5$.\\
	Ta có sơ đồ hình cây như sau
	\begin{center}
	\begin{tikzpicture}[yscale=0.7]
	\def\gocm{20}
	\def\gocn{10}
	\def\r{4}
	\tikzset{s/.style={outer sep=0.5 mm,draw=magenta,rectangle,minimum width=2.75cm,rounded corners=1mm}}
	\path(0,0)node(O){}++(\gocm:\r)node[s](A1){X}++(\gocn:\r)node[s](A2){$Y$};
	\path(A1)++({-\gocn}:\r)node[s](a2){$\overline{Y}$};
	\path(O)++(-\gocm:\r)node[s](B1){$\overline{X}$}++(\gocn:\r)node[s](B2){$Y$};
	\path(B1)++({-\gocn}:\r)node[s](b2){$\overline{Y}$};
	\foreach \x/\y in {
	O/A1,A1/A2,
	O/B1,B1/B2,
	A1/a2,
	B1/b2}
	\draw[-stealth](\x.east)--(\y.west);
	\path(O)--(A1.west)node[pos=0.5,above,sloped]{$0{,}4$}(O)--(B1.west)node[pos=0.5,below,sloped]{$0{,}6$}(B1.east)--(B2.west)node[pos=0.5,above,sloped]{$0{,}5$}(A1.east)--(A2.west)node[pos=0.5,above,sloped]{$0{,}4$}
	(A1.east)--(a2.west)node[pos=0.5,below,sloped]{$0{,}6$}
	(B1.east)--(b2.west)node[pos=0.5,below,sloped]{$0{,}5$};
	\end{tikzpicture}
	\end{center}
	Khi đó $\mathrm{P}(A)=\mathrm{P}(XY)=0{,}4\cdot 0{,}4=0{,}16$;\\ $\mathrm{P}(B)=\mathrm{P}(X\overline{Y})+\mathrm{P}(\overline{X}Y)=0{,}4\cdot 0{,}6+0{,}6\cdot 0{,}5=0{,}54$.
	}
\end{bt}
\begin{bt}%[2D5V2-3]
	Một trường đại học tiến hành khảo sát tình trạng việc làm sau khi tốt nghiệp của sinh viên. Kết quả khảo sát cho thấy tỉ lệ người tìm được việc làm đúng chuyên ngành là $85 \%$ đối với sinh viên tốt nghiệp loại giỏi và $70 \%$ đối với sinh viên tốt nghiệp loại khác.
	Tỉ lệ sinh viên tốt nghiệp loại giỏi là $30 \%$. Gặp ngẫu nhiên một sinh viên đã tốt nghiệp của trường.
	Sử dụng sơ đồ hình cây, tính xác suất của các biến cố:
	C: \lq\lq  Sinh viên tốt nghiệp loại giỏi và tìm được việc làm đúng chuyên ngành\rq\rq;
	$D$ : \lq\lq  Sinh viên không tốt nghiệp loại giỏi và tìm được việc làm đúng chuyên ngành\rq\rq.
	\loigiai{
	Gọi $X$ là biến cố: \lq\lq  Sinh viên tốt nghiệp loại Giỏi\rq\rq.\\
	$Y$ là biến cố: \lq\lq  Sinh viên tìm được việc làm đúng chuyên ngành\rq\rq.\\
	Ta có
	$
	\mathrm{P}(Y|X)=0{,}85; \mathrm{P}(Y \mid \overline{X})=0{,}7 ; \mathrm{P}(X)=0{,}3
	$.\\
	Do đó $\mathrm{P}(\overline{X})=1-\mathrm{P}(X)=0{,}7 ; \mathrm{P}(\overline{Y} |X)=1-\mathrm{P}(Y|X)=0{,}15 ; \\\mathrm{P}(\overline{Y} \mid \overline{X})=1-\mathrm{P}(Y \mid \overline{X})=0{,}3$.\\
	Ta có sơ đồ hình cây như sau
	\begin{center}
	\begin{tikzpicture}[yscale=0.7]
	\def\gocm{20}
	\def\gocn{10}
	\def\r{4}
	\tikzset{s/.style={outer sep=0.5 mm,draw=magenta,rectangle,minimum width=2.75cm,rounded corners=1mm}}
	\path(0,0)node(O){}++(\gocm:\r)node[s](A1){X}++(\gocn:\r)node[s](A2){$Y$};
	\path(A1)++({-\gocn}:\r)node[s](a2){$\overline{Y}$};
	\path(O)++(-\gocm:\r)node[s](B1){$\overline{X}$}++(\gocn:\r)node[s](B2){$Y$};
	\path(B1)++({-\gocn}:\r)node[s](b2){$\overline{Y}$};
	\foreach \x/\y in {
	O/A1,A1/A2,
	O/B1,B1/B2,
	A1/a2,
	B1/b2}
	\draw[-stealth](\x.east)--(\y.west);
	\path(O)--(A1.west)node[pos=0.5,above,sloped]{$0{,}3$}(O)--(B1.west)node[pos=0.5,below,sloped]{$0{,}7$}(B1.east)--(B2.west)node[pos=0.5,above,sloped]{$0{,}7$}(A1.east)--(A2.west)node[pos=0.5,above,sloped]{$0{,}85$}
	(A1.east)--(a2.west)node[pos=0.5,below,sloped]{$0{,}15$}
	(B1.east)--(b2.west)node[pos=0.5,below,sloped]{$0{,}3$};
	\end{tikzpicture}
	\end{center}
	Khi đó $\mathrm{P}(C)=\mathrm{P}(XY)=0{,}3\cdot 0{,}85=0{,}255$; $\mathrm{P}(B)=\mathrm{P}(\overline{X}Y)=0{,}7\cdot 0{,}7=0{,}49$.
	}
\end{bt}
%=====================
\begin{dang}{Công thức nhân xác suất}
	Với hai biến cố $A$ và $B$ bất kì, ta có
	$$\mathrm{P}(A B)=\mathrm{P}(B) \cdot \mathrm{P}(A \mid B).$$
\end{dang}
%----------------------------
\subsubsection{Ví dụ minh hoạ}
\begin{vd}%[2D5H2-4]
	Trong một hộp kín có 7 chiếc bút bi xanh và 5 chiếc bút bi đen, các chiếc bút có cùng kích thước và khối lượng. Bạn Sơn lấy ngẫu nhiên một chiếc bút bi từ trong hộp, không trả lại. Sau đó bạn Tùng lấy ngẫu nhiên một trong 11 chiếc bút còn lại. Tính xác suất để Sơn lấy được bút bi đen và Tùng lấy được bút bi xanh.
	\loigiai{
	Gọi $A$ là biến cố: ``Bạn Sơn lấy được bút bi đen'';\\
	$B$ là biến cố: ``Bạn Tùng lấy được bút bi xanh''.\\
	Ta cần tính $\mathrm{P}(AB)$.\\
	Vì $n(A)=5$ nên $\mathrm{P}(A)=\dfrac{5}{12}$.\\
	Nếu $A$ xảy ra tức là bạn Sơn lấy được bút bi đen thì trong hộp có 11 bút bi với 7 bút bi xanh.\\
	Vậy $\mathrm{P}(B\mid A)=\dfrac{7}{11}$.\\
	Theo công thức nhân xác suất: $\mathrm{P}(AB)=\mathrm{P}(A)\cdot \mathrm{P}(B\mid A)=\dfrac{5}{12}\cdot\dfrac{7}{11}=\dfrac{35}{132}$.\\}
\end{vd}
\begin{vd}
	Cho hai biến cố $A$ và $B$ có $\mathrm{P}(A)=0{,}3 ; \mathrm{P}(B)=0{,}5$ và $\mathrm{P}(A \mid B)=0{,}4$. Tính $\mathrm{P}(\overline{A} B)$ và $\mathrm{P}(\overline{A} \mid B)$.
	\loigiai
	{
	\immini
	{
	Theo công thức nhân xác suất, ta có $\mathrm{P}(A B)=\mathrm{P}(B) \cdot \mathrm{P}(A \mid B)=0{,}2$.\\
	Vì $\overline{A} B$ và $A B$ là hai biến cố xung khắc và $\overline{A} B \cup A B=B$ nên theo tính chất của xác suất, ta có $\mathrm{P}(\overline{A} B)=\mathrm{P}(B)-\mathrm{P}(A B)=0{,}3$.\\
	Theo công thức tính xác suất có điều kiện,
	$$
	\mathrm{P}(\overline{A} \mid B)=\dfrac{\mathrm{P}(\overline{A} B)}{\mathrm{P}(B)}=\dfrac{0{,}3}{0{,}5}=0{,}6.$$
	}
	{
	\begin{tikzpicture}[scale=0.54]
	\def\firstven{(0,0) ellipse (3cm and 2cm)}
	\def\secondven{(2.5,1) ellipse (2.8cm and 2cm)}
	\begin{scope}
	\clip \firstven;
	\fill[gray!50,opacity=0.85] \secondven;
	\end{scope}
	\draw \firstven \secondven;
	\node at (-2.2,2) {$A$};
	\node at (5.6,2.2){$B$};
	\node at (1.3,0.5){$AB$};
	\node at (4,1){$\overline{A} B$};
	\end{tikzpicture}
	}
	}
\end{vd}
\begin{vd}%[2D5H2-4]
	Ô cửa bí mật (Let's Make a Deal) là một trò chơi trên truyền hình nổi tiếng ở Mỹ, đã được mua bản quyền và phát sóng ở nhiều nước trên thế giới. Nội dung trò chơi như sau:
	\begin{itemize}
	\item Người chơi được mời lên sân khấu và đứng trước ba cánh cửa đóng kín. Sau một cánh cửa có chiếc ô tô, sau mỗi cánh cửa còn lại là một con lừa. Người chơi được yêu cầu chọn ngẫu nhiên một cánh cửa, nhưng không được mở ra.
	\item Tiếp đó người quản trò tuyên bố sẽ mở ngẫu nhiên một trong hai cánh cửa người chơi không chọn mà sau cửa đó là con lừa. Người quản trò hỏi người chơi muốn giữ nguyên sự lựa chọn ban đầu của mình hay muốn chuyển sang cửa chưa mở còn lại.
	\end{itemize}
	Giả sử người chơi chọn cửa số 1 và người quản trò mở cửa số 3. Kí hiệu $E_1$; $E_2$; $E_3$ tương ứng là các biến cố: ``Sau ở cửa số 1 có ô tô''; ``Sau ở cửa số 2 có ô tô''; ``Sau ở cửa số 3 có ô tô'' và $H$ là biến cố: ``Người quản trò mở ở cửa số 3 thấy con lừa''.
	Sau khi người quản trò mở cánh cửa số 3 thấy con lừa, tức là khi $H$ xảy ra. Để quyết định thay đổi lựa chọn hay không, người chơi cần so sánh hai xác suất có điều kiện: $P\left(E_1\mid H\right)$ và $P\left(E_2\mid H\right)$
	\begin{listEX}
	\item Chứng minh rằng:
	\begin{enumEX}[\itemCI]{2}
	\item $\mathrm{P}(E_1)=\mathrm{P}(E_2)=\mathrm{P}(E_3)=\dfrac{1}{3}$;
	\item $P\left(H \mid E_1\right)=\dfrac{1}{2} ~\text{và}~P\left(H \mid E_2\right)=1$.
	\end{enumEX}
	\item Sử dụng công thức tính xác suất có điều kiện và công thức nhân xác suất, chứng minh rằng:
	$$ \mathrm{P}\left(E_1 \mid H\right)=\dfrac{\mathrm{P}(E_1) \cdot P\left(H \mid E_1\right)}{\mathrm{P}(H)}$$
	%	\item $P\left(E_2 \mid H\right)=\dfrac{\mathrm{P}(E_2) \cdot P\left(H \mid E_2\right)}{\mathrm{P}(H)$.
	\item Từ các kết quả trên hãy suy ra:
	$$P\left(E_2 \mid H\right)=2 P\left(E_1 \mid H\right)$$
	Từ đó hãy đưa ra lời khuyên cho người chơi: Nên giữ nguyên sự lựa chọn ban đầu hay chuyển sang cửa chưa mở còn lại?\\
	\end{listEX}
	\loigiai{
	\begin{listEX}
	\item Không gian mẫu $\Omega$ là tập hợp gồm 3 phần thưởng (1 ô tô + 2 con lừa) $\Rightarrow n(\Omega)=3$.
	Ta có $n(E_1)=n(E_2)=n(E_3)=1$.\\
	Suy ra $\mathrm{P}(E_1)=\mathrm{P}(E_2)=\mathrm{P}(E_3)=\dfrac{1}{3}$.\\
	Nếu $E_1$ xảy ra, tức là sau cửa số 1 có ô tô. Khi đó, sau cửa số 2 và 3 là con lừa. Người quản trò chọn ngẫu nhiên một trong hai cửa số 2 và 3 để mở ra. Do đó, việc chọn cửa số 2 hay cửa số 3 có khả năng như nhau. \\
	Vậy $P\left(H\mid E_1\right)=\dfrac{1}{2}$.\\
	Nếu $E_2$ xảy ra, tức là cửa số 2 có ô tô. Khi đó, người quản trò chắc chắn phải mở cửa số 3.\\
	Do đó $P\left(H\mid E_2\right)=1$.
	\item Ta có $$\begin{aligned} &P\left(E_1\mid H\right)=\dfrac{\mathrm{P}(E_1H)}{\mathrm{P}(H)}\\\Leftrightarrow &\mathrm{P}(E_1H)=\mathrm{P}(E_1)\cdot P\left(H\mid E_1\right)\\\Leftrightarrow &P\left(E_1\mid H\right)=\dfrac{\mathrm{P}(E_1)\cdot P\left(H\mid E_1\right)}{\mathrm{P}(H)}.\end{aligned}$$
	\item Ta có $$P\left(E_2 \mid H\right)=\dfrac{\mathrm{P}(E_2) \cdot P\left(H\mid E_2\right)}{\mathrm{P}(H)}.$$
	Suy ra
	$$\begin{aligned}
	\dfrac{P\left(E_2 \mid H\right)}{P\left(E_1 \mid H\right)}&=\dfrac{\mathrm{P}(E_2) \cdot P\left(H \mid E_2\right)}{\mathrm{P}(E_1) \cdot P\left(H \mid E_1\right)}\\
	&=\dfrac{\dfrac{1}{3} \cdot 1}{\dfrac{1}{3} \cdot\dfrac{1}{2}}=2.
	\end{aligned}$$
	Suy ra $P\left(E_2\mid H\right)=2\cdot P\left(E_1\mid H\right)$.\\
	\textbf{\textit{Nhận xét: }}Từ kết quả trên ta thấy người chơi nên chuyển sang cửa chưa mở còn lại để tăng gấp đôi khả năng trúng thưởng chiếc ô tô.
	\end{listEX}
	}
\end{vd}
\begin{vd}%[2D5H2-4]
	Một nhóm $5$ học sinh nam và $4$ học sinh nữ tham gia lao động trên sân trường. Cô giáo chọn ngẫu nhiên đồng thời $2$ bạn trong nhóm đi tưới cây. Tính xác suất để hai bạn được chọn có cùng giới tính, biết rằng có ít nhất $1$ bạn nam được chọn.
	\loigiai{
	Số phần tử của không gian mẫu là $n(\Omega)=C^2_9=36$.\\
	Gọi A là biến cố \lq\lq  Hai bạn được chọn có cùng giới tính\rq\rq.\\
	B là biến cố \lq\lq  Có ít nhất một bạn nam được chọn\rq\rq.\\
	Ta có $n(B)=C^2_5+C^1_5\cdot C^1_4=30$ suy ra $\mathrm{P}(B)=\dfrac{30}{36}$.\\
	Ta có $n(AB)=C^2_5=10$ suy ra $\mathrm{P}(AB)=\dfrac{10}{36}$.\\
	Vậy $\mathrm{P}(A|B)=\dfrac{\mathrm{P}(AB)}{\mathrm{P}(B)}=\dfrac{10}{30}=\dfrac{1}{3}$.
	}
\end{vd}
%----------------------------
\subsubsection{Bài tập áp dụng}
\begin{bt}%[2D5H1-2]
	Cho hai biến cố $A$ và $B$ có $\mathrm{P}(A)=0{,}4; \mathrm{P}(B)=0{,}8$ và $\mathrm{P}(A|\overline{B})=0{,}5$. Tính $\mathrm{P}(A\overline{B})$ và $\mathrm{P}(A|B)$.
	\loigiai{
	Ta có $\mathrm{P}(A\overline{B})=\mathrm{P}(A|\overline{B})\cdot \mathrm{P}(\overline{B})=0{,}5\cdot 0{,}2=0{,}1$.\\
	Vì $AB$ và $A\overline{B}$ là hai biến cố xung khắc và $AB\cup A\overline{B}=A$ nên theo tính chất của xác suất, ta có\\ $\mathrm{P}(AB)=\mathrm{P}(A)-\mathrm{P}(A\overline{B})=0{,}4-0{,}1=0{,}3$.\\
	Khi đó: $\mathrm{P}(A|B)=\dfrac{\mathrm{P}(AB)}{\mathrm{P}(B)}=\dfrac{0{,}3}{0{,}8}=0{,}375$.
	}
\end{bt}
\begin{bt}%[2D5H2-3]
	\immini{Máy tính và thiết bị lưu điện (UPS) được kết nối như hình 5. Khi xảy ra sự cố điện, UPS bị hỏng với xác suất $0{,}02$. Nếu UPS bị hỏng khi xảy ra sự cố điện, máy tính sẽ bị hỏng với xác suất $0{,}1$; ngược lại, nếu UPS không bị hỏng, máy tính sẽ không bị bỏng.
	\begin{listEX}
	\item Tính xác suất để cả UPS và máy tính đều không bị hỏng khi xảy ra sự cố điện.
	\item Tính xác suất để cả UPS và máy tính đều bị hỏng khi xảy ra sự cố điện.
	\end{listEX}
	}
	{\includegraphics[width=6.5cm,height=4cm]{images/12-SGK-CTST-6-1-5}}
	\loigiai{
	Gọi $A$ là biến cố \lq\lq  UPS bình thường\rq\rq\, và $B$ là biến cố: \lq\lq  Máy tính bình thường\rq\rq.\\
	Ta có
	$
	\mathrm{P}(B|A)=1; \mathrm{P}(B\mid \overline{A})=0{,}9 ; \mathrm{P}(A)=0{,}98
	$.\\
	Do đó $\mathrm{P}(\overline{A})=1-\mathrm{P}(A)=0{,}02 ; \mathrm{P}(\overline{B} \mid \overline{A})=1-\mathrm{P}(B \mid \overline{A})=0{,}1$.\\
	Ta có sơ đồ hình cây như sau
	\begin{center}
	\begin{tikzpicture}[yscale=0.6]
	\def\gocm{20}
	\def\gocn{10}
	\def\r{4}
	\tikzset{s/.style={outer sep=0.5 mm,draw=magenta,rectangle,minimum width=2.75cm,rounded corners=1mm}}
	\path(0,0)node(O){}++(\gocm:\r)node[s](A1){A}++(\gocn:\r)node[s](A2){$B$};
	\path(O)++(-\gocm:\r)node[s](B1){$\overline{A}$}++(\gocn:\r)node[s](B2){$B$};
	\path(B1)++({-\gocn}:\r)node[s](b2){$\overline{B}$};
	\foreach \x/\y in {
	O/A1,A1/A2,
	O/B1,B1/B2,
	B1/b2}
	\draw[-stealth](\x.east)--(\y.west);
	\path(O)--(A1.west)node[pos=0.5,above,sloped]{$0{,}98$}(O)--(B1.west)node[pos=0.5,below,sloped]{$0.02$}(B1.east)--(B2.west)node[pos=0.5,above,sloped]{$0{,}9$}(A1.east)--(A2.west)node[pos=0.5,above,sloped]{$1$}
	(B1.east)--(b2.west)node[pos=0.5,below,sloped]{$0{,}1$};
	\end{tikzpicture}
	\end{center}
	\begin{listEX}
	\item $\mathrm{P}(AB)=0{,}98\cdot 1=0{,}98$
	\item $\mathrm{P}(\overline{A}.\overline{B})=0{,}02\cdot 0{,}1=0{,}002$.
	\end{listEX}
	}
\end{bt}
\begin{bt}%[2D5H2-4]
	Công ty nước giải khát $X$ tổ chức một chương trình khuyến mại như sau: Trong mỗi thùng 24 chai nước giải khát đều có hai chai trúng thưởng (giải thưởng được viết ở dưới nắp chai), người tham gia chương trình được mở nắp một cách ngẫu nhiên lần lượt hai chai trong một thùng. Tính xác suất để một người tham gia chương trình mở được cả hai chai đều trúng thưởng.
	\loigiai{
		Gọi $A$ là biến cố \lq\lq  chai thứ nhất có trúng thưởng\rq\rq, và $B$ là biến cố \lq\lq  chai thứ hai có trúng thưởng\rq\rq.\\
		Xác suất của $A$ là xác suất để lấy ra một chai có trúng thưởng lần đầu tiên là $$\mathrm{P}(A)=\dfrac{n(A)}{n(\Omega)}=\dfrac{2}{24}=\dfrac{1}{12}.$$
		Sau khi lấy một chai trúng thưởng, số chai trúng trưởng còn $1$ trong tổng số $23$ chai nước.
		Xác suất của $B$ khi đã xảy ra $A$ là xác suất để lấy ra một chai trúng thưởng lần thứ hai là $$\mathrm{P}(B \mid A)=\dfrac{1}{23}.$$
		Áp dụng quy tắc nhân xác suất, ta có
		$$\mathrm{P}(A \cap B)=\mathrm{P}(A) \cdot \mathrm{P}(B \mid A)=\dfrac{1}{2} \cdot \dfrac{1}{23}=\dfrac{1}{46}.$$
		Vậy, xác suất để cả hai chai nước đều trúng thưởng là $\dfrac{1}{46}$.
	}
\end{bt}


% %%%%%%%
% \newpage
\subsection{Bài tập tự luận}
% \BTTL
\setcounter{bt}{0}
%%==========Bài 1
\begin{bt}
	Cho $P(A)=0{,}2;P(B)=0{,}51;P(B\mid A)=0{,}8$. Tính $P(A\mid B)$.
	\loigiai{
		Ta có
		$P(AB)=P(A) \cdot P(B \mid A)=0{,}2\cdot 0{,}8=0{,}16$.\\
		$P(AB)=P(B) \cdot P(A \mid B) \Rightarrow P(A \mid B)=\dfrac{P(AB)}{P(B)}=\dfrac{0{,}16}{0{,}51}\approx 0{,}314$.
	}
\end{bt}

%%==========Bài 2
\begin{bt}
	Cho hai biến độc lập $A,B$ với $\mathrm{P}(A)=0{,}8$, $\mathrm{P}(B)=0{,}25$. Tính $\mathrm{P}(A\mid B)$.
	\loigiai{
		Vì $A$ và $B$ là hai biến cố độc lập, do đó
		\[\mathrm{P}(A\mid B)=\dfrac{\mathrm{P}(A\cap B)}{\mathrm{P}(B)}=\dfrac{\mathrm{P}(A)\cdot\mathrm{P}(B)}{\mathrm{P}(B)}=\mathrm{P}(A)=0{,}8.\]	
	}
\end{bt}
%%==========Bài 3
\begin{bt}
	Cho hai biến cố $A$, $B$ có $\mathrm{P}(A)=0{,}6;\mathrm{P}(B)=0{,}8;\mathrm{P}(A\cap B)=0{,}4$. Tính các xác suất sau:
	\begin{listEX}[2]
		\item $\mathrm{P}(B\mid A);\mathrm{P}(\overline{B}\mid A)$.
		\item $\mathrm{P}(A \cap \overline{B})$.
	\end{listEX}
	\loigiai{
		\begin{enumerate}
			\item $\mathrm{P}(B\mid A)=\dfrac{\mathrm{P}(A \cap B)}{\mathrm{P}(A)}=\dfrac{0{,}4}{0{,}6}=\dfrac{2}{3} 
			\Rightarrow \mathrm{P}(\overline{B}\mid A)=1-\mathrm{P}(B\mid A) = 1 -\dfrac{2}{3}=\dfrac{1}{3}$.
			\item Ta có 
			\begin{eqnarray*}
				\mathrm{P}(A \cap \overline{B}) &=& \mathrm{P}\left(\overline{B}\mid A\right)\cdot \mathrm{P} (A)=\dfrac{1}{3}\cdot 0{,}6=0{,}2.
			\end{eqnarray*}
	\end{enumerate}}
\end{bt}
%%==========Bài 5
\begin{bt}%[2D5H1-2]
	Cho hai biến cố $A$ và $B$ có $P(A)=0{.}4; P(B)=0{.}8$ và $P(A|B)=0{.}5$. Tính $P(A\overline{B})$ và $P(A|B)$.
	\loigiai{
		Ta có $P(A\overline{B})=P(A|\overline{B})\cdot P(\overline{B})=0{.}5\cdot 0{.}2=0{.}1$.\\
		Vì $AB$ và $A\overline{B}$ là hai biến cố xung khắc và $AB\cup A\overline{B}=A$ nên theo tính chất của xác suất, ta có\\ $P(AB)=P(A)-P(A\overline{B})=0{.}4-0{.}1=0{.}3$.\\
		Khi đó: $P(A|B)=\dfrac{P(AB)}{P(B)}=\dfrac{0{.}3}{0{.}8}=0{.}375$.
	}
\end{bt}
%%==========Bài 4
\begin{bt}%[2D5B1-2]
	Một thư viện có $35\%$ tổng số sách là sách khoa học, $14\%$ tổng số sách là sách khoa học tự nhiên. Chọn ngẫu nhiên một quyển sách của thư viện. Tính xác suất để quyển sách được chọn là sách khoa học tự nhiên, biết rằng đó là quyển sách về khoa học.
	\loigiai
	{Gọi $A$ là biến cố \lq\lq  Sách được chọn là sách khoa học tự nhiên\rq\rq.\\
	Gọi $B$ là biến cố \lq\lq  Sách được chọn là sách khoa học\rq\rq.\\
	Do có $35\%$ tổng số sách là sách khoa học nên $P(B)=0{.}35$.\\
	Do có $14\%$ tổng số sách là sách khoa học tự nhiên nên $P(AB)=0{.}14$.\\
	Vậy $P(A|B)=\dfrac{P(AB)}{P(B)}=\dfrac{0{.}14}{0{.}35}=0{.}4$.
	}
\end{bt}


%%==========Bài 7
\begin{bt}
	Một hộp kín đựng 20 tấm thẻ giống hệt nhau đánh số từ 1 đến 20. Một người rút ngẫu nhiên ra một tấm thẻ từ trong hộp. Người đó được thông báo rằng thẻ rút ra mang số chẵn. Tính xác suất để người đó rút được thẻ số 10.
	\loigiai{
	Gọi $A$ là biến cố: ``Người đó rút được thẻ số 10''.\\
	Gọi $B$ là biến cố: ``Người đó rút được thẻ mang số chẵn''.\\
	Không gian mẫu mà 20 tấm thẻ đánh số từ 1 đến 20 $\Rightarrow$ $n(\Omega)=20$.\\
	Ta cần tính $P\left(A\mid B\right)$.\\
	Ta có, từ 1 đến 20 có 10 số chẵn nên $n(B)=10$.\\
	Vậy $P(B)=\dfrac{n(B)}{n(\Omega)}=\dfrac{10}{20}=\dfrac{1}{2}$.\\
	Trong số 10 số chẵn có một số 10 nên $n(AB)=1$.\\
	Vậy $P(AB)=\dfrac{n(AB)}{n(\Omega)}=\dfrac{1}{20}$.\\
	Do đó $P(A\mid B)=\dfrac{P(AB)}{P(B)}=\dfrac{1}{10}\approx 0{,}1$.
	}
\end{bt}

%%==========Bài 8
\begin{bt}
	Gieo hai con xúc xắc cân đối, đồng chất. Tính xác suất để:
	\begin{enumerate}[a)]
	\item Tổng số chấm xuất hiện trên hai con xúc xắc bằng 7 nếu biết rằng ít nhất có một con xúc xắc xuất hiện mặt 5 chấm;
	\item Có ít nhất có một con xúc xắc xuất hiện mặt 5 chấm nếu biết rằng tổng số chấm xuất hiện trên hai con xúc xắc bằng 7. 
	\end{enumerate}
	\loigiai{
	\begin{enumerate}[a)]
	\item Không gian mẫu $n(\Omega)=6\cdot 6=36$.\\
	Gọi $A$ là biến cố: ``Tổng số chấm xuất hiện trên hai con xúc xắc bằng 7''.\\
	Gọi $B$ là biến cố: ``Có ít nhất một con xúc xắc xuất hiện mặt 5 chấm''.\\
	Ta có $n(B)=6+6+1=13$ ứng với các trường hợp $(5;x)$; $(x;5)$; $(5;5)$.\\
	Vậy $P(B)=\dfrac{n(B)}{n(\Omega)}=\dfrac{13}{36}$.\\
	Ta có tổng số chấm xuất hiện trên hai con xúc xắc bằng 7 trong đó có ít nhất một con xúc xắc xuất hiện mặt 5 chấm ứng với các trường hợp $(5;2)$ và $(2;5)$. \\
	Suy ra $n(AB)=2$.\\
	Vậy $P(AB)=\dfrac{n(AB)}{n(\Omega)}=\dfrac{2}{36}=\dfrac{1}{18}$.\\
	Do đó 
	$P(A\mid B)=\dfrac{P(AB)}{P(B)}=\dfrac{2}{13}$.
	\item Ta tính $P(B\mid A)$.\\
	Biến cố tổng hai mặt là $7: A=\{(1;6);(2;5);(3;4);(4;3);(5;2);(6;1)\}$ nên $n(A)=6$.\\
	Vậy $P(A)=\dfrac{n(A)}{n(\Omega)}=\dfrac{6}{36}$.\\
	Ta có $P(BA)=P(AB)=\dfrac{1}{18}$.\\
	Do đó 
	$P(B\mid A)=\dfrac{P(BA)}{P(A)}=\dfrac{2}{6}=\dfrac{1}{3}$.
	\end{enumerate}
	}
\end{bt}
%%==========Bài 9
\begin{bt}
	Gieo hai con xúc xắc cân đối, đồng chất. Tính xác suất để tổng số chấm xuất hiện trên hai con xúc xắc đó không nhỏ hơn 10 nếu biết rằng có ít nhất một con xúc xắc xuất hiện mặt 5 chấm.
	\loigiai{
	Không gian mẫu $n(\Omega)=6\cdot 6=36$.\\
	Gọi $A$ là biến cố: ``Tổng số chấm xuất hiện trên hai con xúc xắc không nhỏ hơn 10''.\\
	Gọi $B$ là biến cố: ``Có ít nhất một con xúc xắc xuất hiện mặt 5 chấm''.\\
	Ta có $n(B)=6+6+1=13$ ứng với các trường hợp $(5;x)$; $(x;5)$; $(5;5)$.\\
	Vậy $P(B)=\dfrac{n(B)}{n(\Omega)}=\dfrac{13}{36}$.\\
	Ta có tổng số chấm xuất hiện trên hai con xúc xắc không nhỏ hơn 10 trong đó có ít nhất một con xúc xắc xuất hiện mặt 5 chấm ứng với các trường hợp $(5;5);(5;6);(6;5)$.\\
	Suy ra $n(AB)=3$.\\
	Vậy $P(AB)=\dfrac{n(AB)}{n(\Omega)}=\dfrac{3}{36}=\dfrac{1}{12}$.\\
	Do đó 
	$P(A\mid B)=\dfrac{P(AB)}{P(B)}=\dfrac{3}{13}$.
	}
\end{bt}



%%==========Bài 12
\begin{bt}
	Một hộp có $3$ quả bóng màu xanh, $4$ quả bóng màu đỏ; các quả bóng có kích thước và khối lượng như nhau. Lấy bóng ngẫu nhiên hai lần liên tiếp, trong đó mỗi lần lấy ngẫu nhiên một quả bóng trong hộp, ghi lại màu của quả bóng lấy ra và bỏ lại quả bóng đó vào hộp. Xét các biến cố:
	\begin{enumerate}
	\item $A$ \lq\lq  \,Quả bóng màu xanh được lấy ra ở lần thứ nhất\rq\rq;
	\item $B$ \lq\lq  \,Quả bóng màu đỏ được lấy ra ở lần thứ hai \rq\rq.
	\end{enumerate}
	Chứng minh rằng $A$, $B$ là hai biến cố độc lập.
	\loigiai{
	Gọi $A_i$ là biến cố \lq\lq  Lần thứ $i$ lấy được bóng màu xanh\rq\rq:\\
	$A=A_1A_2 \cup \overline{A_1}~\overline{A_2} \Rightarrow P\left(A\right)=\dfrac{3}{7}\cdot\dfrac{3}{7}+\dfrac{3}{7}\cdot\dfrac{4}{7}=\dfrac{3}{7}$.\\
	Gọi $B_i$ là biến cố \lq\lq  Lần thứ $i$ lấy được bóng màu đỏ\rq\rq:\\
	$B=B_1B_2 \cup \overline{B_1}B_2 \Rightarrow P\left(B\right)=\dfrac{4}{7}\cdot\dfrac{4}{7}+\dfrac{4}{7}\cdot\dfrac{3}{7}=\dfrac{4}{7}$.\\
	Ta có, $\mathrm{P}(A\mid B)=\dfrac{n\left(A \cap B\right)}{n\left(B\right)}=\dfrac{4\cdot 3}{7\cdot 4}=\dfrac{3}{7}=P\left(A\right) \Rightarrow A, B$ là hai biến cố độc lập.}
\end{bt}
%%==========Bài 13
\begin{bt}
	Cho hai con xúc xắc cân đối và đồng chất. Gieo lần lượt từng xúc xắc trong hai xúc xắc đó. Tính xác suất để tổng số chấm xuất hiện trên hai xúc xắc bằng $6$, biết rằng xúc xắc thứ nhất xuất hiện mặt $4$ chấm.
	\loigiai{
	Gọi $A$ là biến cố \lq\lq  xúc xắc thứ nhất xuất hiện mặt $4$ chấm\rq\rq ~và $B$ là biến cố \lq\lq  tổng số chấm xuất hiện trên hai xúc xắc bằng $6$\rq\rq.\\
	Xác suất của $A$ là $\mathrm{P}(A)$ là xác suất để xúc xắc thứ nhất xuất hiện mặt $4$ chấm. Vì xúc xắc cân đối và đồng chất, nên \[\mathrm{P}(A)=\dfrac{1}{6}.\]
	Xác suất của $B$ khi biết $A$ đã xảy ra là $\mathrm{P}(B \mid A)$. Trong trường hợp này, để tổng số chấm là $6$, xúc xắc thứ hai phải xuất hiện mặt $2$ chấm. Do đó, $\mathrm{P}(B \mid A)=\dfrac{1}{6}$.\\
	Vậy, theo quy tắc xác suất điều kiện, ta có:\\
	$$\mathrm{P}(B \mid A)=\dfrac{\mathrm{P}(A \cap B)}{\mathrm{P}(A)} \Rightarrow \mathrm{P}(A \cap B)=\mathrm{P}(B \mid A)\cdot \mathrm{P}(A)=\dfrac{1}{6}\cdot \dfrac{1}{6}=\dfrac{1}{36}.$$
	}
\end{bt}
%%==========Bài 14
\begin{bt}
	Một doanh nghiệp trước khi xuất khẩu áo sơ mi phải qua hai lần kiểm tra chất lượng sản phẩm, nếu cả hai lần đều đạt thì chiếc áo đó mới đủ tiêu chuẩn xuất khẩu. Biết rằng bình quân $98\%$ sản phẩm làm ra qua được lần kiểm tra thứ nhất và $95\%$ sản phẩm qua được lần kiểm tra thứ nhất sẽ tiếp tục qua được lần kiểm tra thứ hai. Tính xác suất để một chiếc áo sơ mi đủ tiêu chuẩn xuất khẩu.
	\loigiai{
	$A$ là biến cố \lq\lq  sản phẩm qua được lần kiểm tra thứ nhất\rq\rq.\\
	$B$ là biến cố \lq\lq  sản phẩm qua được lần kiểm tra thứ hai\rq\rq.\\
	Bài toán yêu cầu tính xác suất của biến cố $A\cap B$, tức là sản phẩm vừa qua được lần kiểm tra thứ nhất, và sau đó qua được lần kiểm tra thứ hai.\\
	Xác suất của $A$ là $\mathrm{P}(A)=0{,}98$.\\
	Xác suất của $B$ khi đã qua được $A$ là $\mathrm{P}(B \mid A)=0{,}95$.\\
	Áp dụng công thức xác suất có điều kiện, ta có:
	$$\mathrm{P}(A \cap B)=\mathrm{P}(A)\cdot \mathrm{P}(B \mid A)=0{,}98\cdot 0{,}95=0{,}931.$$
	Vậy, xác suất để một chiếc áo sơ mi đủ tiêu chuẩn xuất khẩu là $93{,}1\%$
	}
\end{bt}
%%==========Bài 17
\begin{bt}
	Một nhóm $50$ học sinh có $23$ bạn biết chơi cầu lông mà không biết chơi bóng đá và $21$ bạn biết chơi bóng đá mà không biết chơi cầu lông. Biết rằng mỗi học sinh trong nhóm này biết chơi bóng đá hoặc cầu lông. Chọn ngẫu nhiên một học sinh trong nhóm. Tính xác suất học sinh này biết chơi bóng đá, biết rằng bạn ấy biết chơi cầu lông. 
	\loigiai{
		Gọi $A$ là biến cố "Học sinh được chọn biết chơi bóng đá"; $B$ là biến cố "Học sinh được chọn biết chơi cầu lông".\\
		Ta có $n\left(A\cap B\right)=50-(23+21)=6$ và $n(B)=23+6=29$. Do đó 
		$\mathrm{P}(A|B)=\dfrac{n\left(A\cap B\right)}{n(B)}=\dfrac{6}{29}$.
	}
\end{bt}
%%==========Bài 16
\begin{bt}
	Có $2$ linh kiện điện tử, xác suất để mỗi linh kiện hỏng trong một thời điểm bất kì lần lượt là: $0{,}01$; $0{,}02$. Hai linh kiện đó được lắp vào một mạch điện theo sơ đồ ở \textit{Hình a, b}. Trong mỗi trường hợp, hãy tính xác suất để trong mạch điện có dòng điện chạy qua
	\begin{center}
	\tikzset{noratorW/.style={voosource,
	bipoles/oosource/circlesize=0.5,
	bipoles/oosource/circleoffset=0.5},
	nullatorW/.style={esource, sources/scale=0.8}}
	\begin{tikzpicture}
	\draw (0,0) to[nullatorW] ++(2,0) to[nullatorW] ++(2, 0)--(4,-2)to[battery1](0,-2)--(0,0);
	\draw (2,-2.5) node[below] {\textit{a)}};
	\end{tikzpicture}
	\quad
	\begin{tikzpicture}
	\draw (0,0) to[battery1](4,0)--(4,-2)to[nullatorW](0,-2)--(0,0);
	\draw (4,-1) to[nullatorW](0,-1);
	\draw (2,-2.5) node[below] {\textit{b)}};
	\end{tikzpicture}
	\end{center}
	\loigiai{
	Gọi
	\begin{itemize}
	\item	$A$: \lq\lq  Linh kiện thứ nhất không hỏng\rq\rq .
	\item 	$B$: \lq\lq  Linh kiện thứ hai không hỏng\rq\rq.
	\end{itemize}
	\begin{enumerate}
	\item \textbf{Hai linh kiện mắc nối tiếp.}\\
	Xác suất để cả hai linh kiện đều không hỏng là:
	$$\mathrm{P}(A \cap B)=\mathrm{P}(A)\cdot \mathrm{P}(B \mid A)$$
	Ta có\\
	$\mathrm{P}(A)=1-P(\overline{A})=1-0{,}01=0{,}99$.\\
	$\mathrm{P}(B \mid A)=1-P(\overline{B}\mid A)=1-0{,}02=0{,}98$.\\
	$\Rightarrow \mathrm{P}(A \cap B)=\mathrm{P}(A)\cdot \mathrm{P}(B \mid A)=0{,}99\cdot0{,}98=0{,}9702$.
	\item \textbf{Hai linh kiện mắc song song.}\\
	Xác suất để ít nhất một linh kiện không hỏng là:
	$$\mathrm{P}	(A \cup B)=\mathrm{P}(A)+\mathrm{P}(B)-\mathrm{P}(A \cap B) = 0{,}99+0{,}98-0{,}9702=0{,}9998.$$
	\end{enumerate}
	}
\end{bt}
%%==========Bài 6
\begin{bt}%[2D5V1-3]
	Mỗi bạn học sinh trong lớp của Minh lựa chọn một trong hai ngoại ngữ là tiếng Anh hoặc tiếng Nhật. Xác suất chọn tiếng Anh của mỗi bạn học sinh nữ là $0{.}6$ và của mỗi bạn học sinh nam là $0{.}7$. Lớp của Minh có $25$ bạn nữ và $20$ bạn nam. Chọn ra ngẫu nhiên một bạn trong lớp.
	Sử dụng sơ đồ hình cây, tính xác suất của các biến cố 
	A:\lq\lq  Bạn được chọn là nam và học tiếng Nhật\rq\rq.\\
	B:\lq\lq  Bạn được chọn là nữ và học tiếng Anh\rq\rq .
	\loigiai{ 
		Gọi $A$ là biến cố \lq\lq  Bạn được chọn là nữ\rq\rq và $B$ là biến cố: \lq\lq  Bạn được chọn học tiếng Anh\rq\rq. \\
		Ta có
		$
		P(Y|X)=0{.}6; P(Y \mid \bar{X})=0{.}7 ; P(X)=\dfrac{5}{9}
		$.\\
		Do đó $P(\bar{X})=1-P(X)=\dfrac{4}{9} ; P(\bar{Y} |X)=1-P(Y|X)=0{.}4; \\P(\bar{Y} \mid \bar{X})=1-P(Y \mid \bar{X})=0{.}3$.\\
		Ta có sơ đồ hình cây như sau
		\begin{center}
			\begin{tikzpicture}[yscale=0.9]
				\def\gocm{20}
				\def\gocn{10}
				\def\r{4}
				\tikzset{s/.style={outer sep=0.5 mm,draw=magenta,rectangle,minimum width=2.75cm,rounded corners=1mm}}
				\path(0,0)node(O){}++(\gocm:\r)node[s](A1){X}++(\gocn:\r)node[s](A2){$Y$};
				\path(A1)++({-\gocn}:\r)node[s](a2){$\bar{Y}$};
				\path(O)++(-\gocm:\r)node[s](B1){$\bar{X}$}++(\gocn:\r)node[s](B2){$Y$};
				\path(B1)++({-\gocn}:\r)node[s](b2){$\bar{Y}$};
				\foreach \x/\y in {
					O/A1,A1/A2,
					O/B1,B1/B2,
					A1/a2,
					B1/b2}
				\draw[-stealth](\x.east)--(\y.west);
				\path(O)--(A1.west)node[pos=0.5,above]{$\dfrac{5}{9}$}(O)--(B1.west)node[pos=0.5,below]{$\dfrac{4}{9}$}(B1.east)--(B2.west)node[pos=0.5,above,sloped]{$0{.}7$}(A1.east)--(A2.west)node[pos=0.5,above,sloped]{$0{.}6$}
				(A1.east)--(a2.west)node[pos=0.5,below,sloped]{$0{.}4$}
				(B1.east)--(b2.west)node[pos=0.5,below,sloped]{$0{.}3$};
			\end{tikzpicture}
		\end{center}
		Do $A=\overline{X}\cdot\overline{Y}$ nên $P(\overline{X}\cdot\overline{Y})=\dfrac{4}{9}\cdot 0{.}3=\dfrac{2}{15}$.\\
		Do $B=XY$ nên $P(XY)=\dfrac{5}{9}\cdot 0{.}6=\dfrac{1}{3}$.\\
	}
\end{bt}
%%==========Bài 10
\begin{bt}
	Bạn An phải thực hiện hai thí nghiệm liên tiếp. Thí nghiệm thứ nhất có xác suất thành công là 0,7. Nếu thí nghiệm thứ nhất thành công thì xác suất thành công của thí nghiệm thứ hai là 0,9. Nếu thí nghiệm thứ nhất không thành công thì xác suất thành công của thí nghiệm thứ hai chỉ là 0,4. Tính xác suất để:
	\begin{enumerate}[a)]
		\item Cả hai thí nghiệm đều thành công;
		\item Cả hai thí nghiệm đều không thành công;
		\item Thí nghiệm thứ nhất thành công và thí nghiệm thứ hai không thành công. 
	\end{enumerate}
	\loigiai{
		Ta có sơ đồ hình cây
		\begin{center}
			\begin{tikzpicture}[line join = round,line cap = round, >=stealth, thick, font = \small, yscale = 0.85]
				%	\draw[gray!50,xstep = 1, ystep = 1] (-5,-5) grid (5,5);
				\path
				(0,0) coordinate (0) node[above=3mm] {TN}
				+(-2,-3) coordinate (11) node[left=3mm] {TB1}
				+(2,-3) coordinate (12) node[right=3mm] {TC1}
				(11)+(-1,-3) coordinate (21) node[below=3mm] {TB2} node[below=9mm] {TB1-TB2}
				+(1,-3) coordinate (22) node[below=3mm] {TC2} node[below=9mm] {TB1-TC2}
				(12)+(-1,-3) coordinate (23) node[below=3mm] {TB2} node[below=9mm] {TC1-TB2}
				+(1,-3) coordinate (24) node[below=3mm] {TC2} node[below=9mm] {TC1-TC2}
				(0)--(11) node[midway,left=3mm] {$\dfrac{3}{10}$}
				(0)--(12) node[midway,right=3mm] {$\dfrac{7}{10}$}
				(11)--(21) node[midway,left=2mm] {$\dfrac{6}{10}$}
				(11)--(22) node[midway,right=2mm] {$\dfrac{4}{10}$}
				(12)--(23) node[midway,left=2mm] {$\dfrac{1}{10}$}
				(12)--(24) node[midway,right=2mm] {$\dfrac{9}{10}$}
				;
				\draw (11)--(0)--(12)
				(21)--(11)--(22)
				(23)--(12)--(24)
				;
				\node[circle, line width = .2 mm, draw = black, text = black, fill = yellow!60, anchor = center, outer sep = 0pt, minimum size = .2cm] (c) at (0) {};
				\node[circle, line width = .2 mm, draw = black, text = black, fill = black, anchor = center, outer sep = 0pt, minimum size = .2cm] (c) at (11) {};
				\node[circle, line width = .2 mm, draw = black, text = black, fill = cyan, anchor = center, outer sep = 0pt, minimum size = .2cm] (c) at (12) {};
				\node[circle, line width = .2 mm, draw = black, text = black, fill = black, anchor = center, outer sep = 0pt, minimum size = .2cm] (c) at (21) {};
				\node[circle, line width = .2 mm, draw = black, text = black, fill = cyan, anchor = center, outer sep = 0pt, minimum size = .2cm] (c) at (22) {};
				\node[circle, line width = .2 mm, draw = black, text = black, fill = black, anchor = center, outer sep = 0pt, minimum size = .2cm] (c) at (23) {};
				\node[circle, line width = .2 mm, draw = black, text = black, fill = cyan, anchor = center, outer sep = 0pt, minimum size = .2cm] (c) at (24) {};
			\end{tikzpicture}
		\end{center}
		\begin{enumerate}[a)]
			\item Xác suất cả hai thí nghiệm đều thành công là $$\dfrac{7}{10}\cdot \dfrac{9}{10}=\dfrac{63}{100}.$$
			\item Xác suất cả hai thí nghiệm đều không thành công là $$\dfrac{3}{10}\cdot \dfrac{6}{10}=\dfrac{18}{100}.$$
			\item Xác suất thí nghiệm thứ nhất thành công và thí nghiệm thứ hai không thành công là $$\dfrac{7}{10}\cdot \dfrac{1}{10}=\dfrac{7}{100}.$$
		\end{enumerate}
	}
\end{bt}
%%==========Bài 11
\begin{bt}
	Trong một túi có một số chiếc kẹo cùng loại, chỉ khác màu, trong đó có 6 cái kẹo màu cam, còn lại là kẹo màu vàng. Hà lấy ngẫu nhiên một cái kẹo từ trong túi, không trả lại.
	Sau đó Hà lại lấy ngẫu nhiên thêm một cái kẹo khác từ trong túi. Biết rằng xác suất Hà lấy được cả hai cái kẹo màu cam là $\dfrac{1}{3}$. Hỏi ban đầu trong túi có bao nhiêu cái kẹo?
	\loigiai{
		Gọi số kẹo là $x~(x>6)$.\\
		Số kẹo màu vàng là $x-6$.\\
		Khi Hà lấy được chiếc kẹo màu cam thì số kẹo trong túi là $x-1$ và số kẹo cam còn lại trong túi là 5 cái.\\
		Ta có sơ đồ cây
		\begin{center}
			\begin{tikzpicture}[line join = round,line cap = round, >=stealth, thick, font = \small, yscale = 0.85]
				%	\draw[gray!50,xstep = 1, ystep = 1] (-5,-5) grid (5,5);
				\path
				(0,0) coordinate (0) node[above=3mm] {Túi kẹo}
				+(-2,-3) coordinate (11) node[left=3mm] {C}
				+(2,-3) coordinate (12) node[right=3mm] {V}
				(11)+(-1,-3) coordinate (21) node[below=3mm] {C} node[below=9mm] {CC}
				+(1,-3) coordinate (22) node[below=3mm] {V} node[below=9mm] {CV}
				(12)+(-1,-3) coordinate (23) node[below=3mm] {C} node[below=9mm] {VC}
				+(1,-3) coordinate (24) node[below=3mm] {V} node[below=9mm] {VV}
				(0)--(11) node[midway,left=3mm] {$\dfrac{6}{x}$}
				(0)--(12) node[midway,right=3mm] {$\dfrac{x-6}{x}$}
				(11)--(21) node[midway,left=2mm] {$\dfrac{5}{x-1}$}
				(11)--(22) node[midway,right=2mm] {$\dfrac{x-6}{x-1}$}
				(12)--(23) node[midway,left=2mm] {$\dfrac{6}{x-1}$}
				(12)--(24) node[midway,right=2mm] {$\dfrac{x-7}{x-1}$}
				;
				\draw (11)--(0)--(12)
				(21)--(11)--(22)
				(23)--(12)--(24)
				;
				\node[circle, line width = .2 mm, draw = black, text = black, fill = yellow!50!orange, anchor = center, outer sep = 0pt, minimum size = .2cm] (c) at (0) {};
				\node[circle, line width = .2 mm, draw = black, text = black, fill=orange, anchor = center, outer sep = 0pt, minimum size = .2cm] (c) at (11) {};
				\node[circle, line width = .2 mm, draw = black, text = black, fill = yellow, anchor = center, outer sep = 0pt, minimum size = .2cm] (c) at (12) {};
				\node[circle, line width = .2 mm, draw = black, text = black, fill = orange, anchor = center, outer sep = 0pt, minimum size = .2cm] (c) at (21) {};
				\node[circle, line width = .2 mm, draw = black, text = black, fill = yellow, anchor = center, outer sep = 0pt, minimum size = .2cm] (c) at (22) {};
				\node[circle, line width = .2 mm, draw = black, text = black, fill = orange, anchor = center, outer sep = 0pt, minimum size = .2cm] (c) at (23) {};
				\node[circle, line width = .2 mm, draw = black, text = black, fill = yellow, anchor = center, outer sep = 0pt, minimum size = .2cm] (c) at (24) {};
			\end{tikzpicture}
		\end{center}
		Xác suất để Hà lấy được cả hai cái kẹo màu cam là
		$$
		\dfrac{6}{x}\cdot \dfrac{5}{x-1}=\dfrac{1}{3} \Rightarrow x^2-x-90=0 \Rightarrow \hoac{&x=-9~\text{(loại)}\\&x=10~\text{thỏa mãn}.}
		$$
		Vậy ban đầu trong túi có 10 cái kẹo.
	}
\end{bt}











%%==========Bài 15
\begin{bt}
	Trên giá sách có $10$ quyển sách Khoa học và $15$ quyển sách Nghệ thuật. Có $9$ quyển sách viết bằng tiếng Anh, trong đó $3$ quyển sách Khoa học có $6$ quyển sách Nghệ thuật, các quyển sách còn lại viết bằng tiếng Việt. Lấy ngẫu nhiên một quyển sách. Dùng sơ đồ hình cây, tính xác suất để quyển sách được lấy ra là sách viết bằng tiếng Việt, biết rằng quyển sách đó là sách Khoa học.
	\loigiai{
		Gọi các biến cố:\\
		$A$: \lq\lq  Sách lấy ra là sách tiếng Việt\rq\rq.\\
		$B$: \lq\lq  Sách lấy ra là sách khoa học\rq\rq.\\
		Khi đó, xác suất để cuốn sách được lấy ra là tiếng Việt, biết rằng cuốn sách đó là sách khoa học là $\mathrm{P}\left(A\mid B\right)$.
		Ta có sơ đồ cây
		\begin{center}
			\begin{tikzpicture}[scale=.2,>=stealth,every node/.style={shape=rectangle,draw,rounded corners, color=blue, fill=blue!10}]
			%-------------
			\tikzstyle{block} = [rectangle, draw, fill=blue!10, rounded corners, text centered, text width = 10em, minimum height = 2em]
			%-------------
			\node (c1) {Cuốn sách được lấy};
			\node (c2) [block, above right = 4cm of c1]{$B\colon$ \,\lq\lq  Sách lấy ra là sách khoa học \rq\rq};
			\node at (7,7.5){$\mathrm{P}(B)=2/5$};
			\node at (7,-7.5){$\mathrm{P}(\overline{B})=3/5$};
			\node (c3) [block, below right= 4cm of c1]{$\overline{B}\colon$\,\lq\lq  Sách lấy ra là sách nghệ thuật \rq\rq};
			\node at (37.8,25){$\mathrm{P}(A\mid B)=7/10$};
			\node (c4) [above right = 2cm of c2]{$A\colon $\,\lq\lq  Sách lấy ra là sách tiếng Việt \rq\rq };
			\node (c5) [below right = 2cm of c2]{$\overline{A}\colon $\,\lq\lq  Sách lấy ra là sách tiếng Anh \rq\rq};
			\node at (39,9.5){$\mathrm{P}(\overline{A}\mid B)=3/10$};
			\node (c6) [block, above right =2cm of c3]{$A\colon $\,\lq\lq  Sách lấy ra là sách tiếng Việt \rq\rq };
			\node at (39,-8.5){$\mathrm{P}(A\mid \overline{B})=3/5$};
			\node (c7) [block, below right = 2cm of c3]{$\overline{A}\colon $\,\lq\lq  Sách lấy ra là sách tiếng Anh \rq\rq};
			\node at (39,-27.8){$\mathrm{P}(\overline{A}\mid \overline{B})=2/5$};
			%--------------
			\draw[->] (c1.east) -- (c2.west);
			\draw[->] (c1.east) -- (c3.west);
			\draw[->] (c2.east) -- (c4.west);
			\draw[->] (c2.east) -- (c5.west);
			\draw[->] (c3.east) -- (c6.west);
			\draw[->] (c3.east) -- (c7.west);
		\end{tikzpicture}
		\end{center}
		Từ đó ta có xác suất để cuốn sách lấy ra là tiếng Việt, biết rằng cuốn sách đó là sách khoa học là $\mathrm{P}(A\mid B)=\dfrac{7}{10}=0{,}7$.
	}
\end{bt}
% %%%%%%%%%%%%%%%%%
% \newpage
\subsection{Bài tập trắc nghiệm}%\BTTN
%
\Opensolutionfile{ans}[ans/ans-2-B18]
%\TN
\begin{ex}%[2D5N1-2]
	Cho hai biến cố $A$, $B$ có xác suất $\mathrm{P}(A)=0{,}4$, $\mathrm{P}(B)=0{,}6$, $\mathrm{P}(AB)=0{,}2$. Tính xác suất $\mathrm{P}(A|B)$.
	\choice
	{\True$\dfrac{1}{3}$}
	{$\dfrac{1}{2}$}
	{$0{,}3$}
	{$0{,}25$}
	\loigiai{
	Theo định nghĩa xác suất có điều kiện, ta có
	$$\mathrm{P}(A \mid B)=\dfrac{\mathrm{P}(A B)}{\mathrm{P}(B)}=\dfrac{0{,}2}{0{,}6}=\dfrac{1}{3}.$$	}
\end{ex}
\begin{ex}%[2D5N1-2]
	Cho hai biến cố $A$, $B$ có xác suất $\mathrm{P}(A)=0{,}4$, $\mathrm{P}(B)=0{,}3$, $\mathrm{P}(A\mid B)=0{,}25$. Tính xác suất $\mathrm{P}(B\mid A)$.
	\choice
	{\True $0{,}1875$}
	{$0{,}48$}
	{$0{,}333$}
	{$0{,}95$}
	\loigiai{
	Ta có
	\allowdisplaybreaks
	\begin{eqnarray*}
	\mathrm{P}(A \mid B)&=&\dfrac{\mathrm{P}(A B)}{\mathrm{P}(B)}\\
	\Rightarrow \mathrm{P}(AB)&=&\mathrm{P}(B)\cdot 	\mathrm{P}(A \mid B)=0{,}3\cdot 0{,}25=0{,}075.
	\end{eqnarray*}
	Từ đó suy ra
	$$\mathrm{P}(B\mid A)=\dfrac{\mathrm{P}(AB)}{\mathrm{P}(A)}=\dfrac{0{,}075}{0{,}4}=0{,}1875.$$	
	}
\end{ex}
\begin{ex}%[2D5H1-2]
	Cho hai biến độc lập $A,B$ với $\mathrm{P}(A)=0{,}8$, $\mathrm{P}(B)=0{,}25$. Khi đó, $\mathrm{P}(B\mid A)$ bằng
	\choice
	{$0{,}2$}
	{$0{,}8$}
	{\True $0{,}25$}
	{$0{,}75$}
	\loigiai{
	Vì $A$ và $B$ là hai biến cố độc lập, do đó
	\[\mathrm{P}(B\mid A)=\mathrm{P}(B)=0{,}25.\]	
	}
\end{ex}	
\begin{ex}%[2D5H1-2]
	Cho một hộp kín có $6$ thẻ ATM của ACB và 4 thẻ ATM của Vietcombank. Lấy ngẫu nhiên lần lượt $2$ thẻ (lấy không hoàn lại). Tìm xác suất để lần thứ hai lấy được thẻ ATM của Vietcombank nếu biết lần thứ nhất đã lấy được thẻ ATM của ACB.
	\choice
	{ $\dfrac{1}{3}$}
	{$\dfrac{2}{3}$}
	{$\dfrac{2}{9}$}
	{\True$\dfrac{4}{9}$}
	\loigiai{
	Gọi $A$ là biến cố \lq\lq  lần thứ hai lấy được thẻ ATM Vietcombank\rq\rq, $B$ là biến cố \lq\lq  lần thứ nhất lấy được thẻ ATM của ACB\rq\rq. Ta cần tìm $\mathrm{P}(A\mid B)$.\\	
	Sau khi lấy lần thứ nhất (biến cố $B$ đã xảy ra) trong hộp còn lại $9$ thẻ (trong đó $4$ thẻ Vietcombank) nên $\mathrm{P}(A\mid B)=\dfrac{4}{9}$.}
\end{ex}
\begin{ex}%[2D5H1-2]
	Một nhóm $50$ học sinh có $23$ bạn biết chơi cầu lông mà không biết chơi bóng đá và $21$ bạn biết chơi bóng đá mà không biết chơi cầu lông. Biết rằng mỗi học sinh trong nhóm này biết chơi bóng đá hoặc cầu lông. Chọn ngẫu nhiên một học sinh trong nhóm. Tính xác suất học sinh này biết chơi bóng đá, biết rằng bạn ấy biết chơi cầu lông. 
	\choice
	{$\dfrac{23}{29}$}
	{\True$\dfrac{6}{29}$}
	{$\dfrac{21}{29}$}
	{$\dfrac{6}{23}$}
	\loigiai{
	Gọi $A$ là biến cố \lq\lq  Học sinh được chọn biết chơi bóng đá\rq\rq; $B$ là biến cố \lq\lq  Học sinh được chọn biết chơi cầu lông\rq\rq.\\
	Ta có $n\left(A\cap B\right)=50-(23+21)=6$ và $n(B)=23+6=29$. Do đó 
	$$\mathrm{P}(A\mid B)=\dfrac{P\left(A B\right)}{\mathrm{P}(B)}=\dfrac{6}{29}.$$
	}
\end{ex}
\begin{ex}%[2D5V1-2]
	Một bình đựng $3$ bi xanh và $2$ bi trắng. Lấy ngẫu nhiên lần $1$ một viên bi (không bỏ vào lại), rồi lần $2$ một viên bi. Tính xác suất để lần $1$ lấy một viên bi xanh, lần $2$ lấy một viên bi trắng.
	\choice
	{$\dfrac{1}{5}$}
	{$\dfrac{1}{10}$}
	{$\dfrac{1}{3}$}
	{\True $\dfrac{3}{10}$}
	\loigiai{
	Gọi $A$ là biến cố lấy một bi xanh lần thứ nhất thì $\mathrm{P}(A)=\dfrac{3}{5}$.\\
	Gọi $B$ là biến cố lấy một bi trắng lần thứ hai.
	Gọi $C$ là biến cố lấy lần $1$ lấy một viên bi xanh, lần $2$ lấy một viên bi trắng.
	Nếu $A$ đã xảy ra thì trong bình chi còn $2$ bi xanh, $2$ bi trằng. Khi đó $\mathrm{P}(B\mid A)=\dfrac{2}{4}$.\\
	Mà $C=AB$, do đó theo công thức nhân ta có:
	$$\mathrm{P}(C)=\mathrm{P}(AB)=\mathrm{P}(A)\cdot \mathrm{P}(B\mid A)=\dfrac{3}{5} \cdot \dfrac{1}{2}=\dfrac{3}{10}.
	$$}
\end{ex}	
\begin{ex}%[2D5H1-2]
	Một công ty đấu thầu $2$ dự án. Khả năng thắng thầu của các dự án $I$ và $II$ lần lượt là $0{,}4$ và $0{,}5$. Khả năng thắng thầu của hai dự án là $0{,}3$. Gọi $A$, $B$ lần lượt là biến cố thắng thầu dự án $I$ và dự án $II$. Biết công ty thắng thầu dự án $I$, tìm xác suất công ty thắng thầu dự án $II$.
	\choice
	{$0{,}25$}
	{$0{,}5$}
	{\True$0{,}75$}
	{$0{,}125$}
	\loigiai{
	Gọi $C$ là biến cố công ty thắng dự $II$ biết công ty thắng dự án $I$. Ta có
	$$\mathrm{P}(C)=\mathrm{P}(B|A)=\dfrac{\mathrm{P}(AB)}{\mathrm{P}(A)}=\dfrac{0{,}3}{0{,}4}=0{,}75.$$
	}
\end{ex}
\begin{ex}%[2D5H1-2]
	Một sinh viên làm $2$ bài tập kế tiếp. Xác suất làm đúng bài thứ nhất là $0{,}7$. Nếu làm đúng bài thứ nhất thì khả năng làm đúng bài thứ $2$ là $0{,}8$, nhưng nếu làm sai bài thứ $1$ thì khả năng làm đúng bài thứ $2$ là $0{,}2$. Tính xác suất sinh viên làm ít nhất một bài.
	\choice
	{$ 0{,}903$}
	{$0{,}737$}
	{\True $0{,}76$}
	{$0{,}62$}
	\loigiai{
	Gọi $A$, $B$ lần lượt là biến cố sinh viên làm đúng bài $1$, bài $2$.\\ Theo đề bài ta có $\mathrm{P}(A)=0{,}7$; $\mathrm{P}(B\mid A)=0{,}8$	và $\mathrm{P}(B\mid \overline{A})=0{,}2$.\\
	Suy ra 
	$$ \mathrm{P}(\overline{B}\mid \overline{A})=1-\mathrm{P}(B\mid \overline{A})=1-0{,}2=0{,}8.$$
	Ta có
	\allowdisplaybreaks
	\begin{eqnarray*}
	\mathrm{P}(A\cup B)&=&1-\mathrm{P}(\overline{A\cup B})\\
	&=&1-\mathrm{P}(\overline{A}\cdot \overline{B})\\
	&=&1- \mathrm{P}(\overline{A})\cdot \mathrm{P}(\overline{B}\mid \overline{A})\\
	&=& 1- 0{,}3 \cdot 0{,}8=0{,}76.
	\end{eqnarray*}
	}
\end{ex}
\begin{ex}%[2D5H1-2]
	Một thư viện có $35\%$ tổng số sách là sách khoa học, $14\%$ tổng số sách là sách khoa học tự nhiên. Chọn ngẫu nhiên một quyển sách của thư viện. Tính xác suất để quyển sách được chọn là sách khoa học tự nhiên, biết rằng đó là quyển sách về khoa học.
	\choice
	{$0{,}2$}
	{\True $0{,}4$}
	{$0,{4}$}
	{$0{,}8$}
	\loigiai
	{Gọi $A$ là biến cố \lq\lq  Sách được chọn là sách khoa học tự nhiên\rq\rq.\\
	Gọi $B$ là biến cố \lq\lq  Sách được chọn là sách khoa học\rq\rq.\\
	Do có $35\%$ tổng số sách là sách khoa học nên $\mathrm{P}(B)=0{,}35$.\\
	Do có $14\%$ tổng số sách là sách khoa học tự nhiên nên $\mathrm{P}(AB)=0{,}14$.\\
	Vậy $\mathrm{P}(A\mid B)=\dfrac{\mathrm{P}(AB)}{\mathrm{P}(B)}=\dfrac{0{,}14}{0{,}35}=0{,}4$.
	}
\end{ex}
\begin{ex}%[2D5V1-2]
	Trong một kì thi, thí sinh được phép thi $3$ lần. Xác suất lần đầu vượt qua kì thi là $0{,}9$. Nếu trượt lần đầu thì xác suất vượt qua kì thi lần hai là $0{,}7$. Nếu trượt cả hai lần thì xác suất vượt qua kì thi ở lần thứ ba là $0{,}3$. Tính xác suất để thí sinh thi đậu.
	\choice
	{\True $0{,}979$}
	{$ 0{,97}$ }
	{$ 0{,79}$ }
	{$ 0{,797}$ }
	\loigiai{
	Gọi $A$, $B$, $C$ lần lượt là biến cố thí sinh thi đậu lần thứ $1$, lần thứ $2$, lần thứ $3$.
	Gọi $D$ là biến cố để thí sinh thi đậu. Ta có
	\allowdisplaybreaks
	\begin{eqnarray*}
	D&=&A\cup (\overline{A}B)\cup (\overline{A}\,\overline{B}C)\\
	\Rightarrow D&=&\mathrm{P}(A)+P (\overline{A}B)+P (\overline{A}\,\overline{B}C).
	\end{eqnarray*}
	Trong đó ta có
	\begin{itemize}
	\item $\mathrm{P}(A)=0{,}9$.
	\item $P (\overline{A}B)=\mathrm{P}(\overline{A})\cdot P (B\mid \overline{A})=0{,}1\cdot 0{,}7=0{,}07$.
	\item $P (\overline{A}\,\overline{B}C)=\mathrm{P}(\overline{A}\,\overline{B})\cdot \mathrm{P}(C\mid \overline{A}\,\overline{B})=\mathrm{P}(\overline{A})\cdot P (\overline{B}\mid \overline{A}) \cdot \mathrm{P}(C\mid \overline{A}\,\overline{B})=0{,}1\cdot (1-0{,}7)\cdot 0{,}3=0{,}009$.
	\end{itemize}
	Vậy $\mathrm{P}(D)=0{,}9+0{,}07+0{,}009=0{,}979$.
	}
\end{ex}
\begin{ex}%[2D5V1-3]
	Trong một hộp kín có $7$ chiếc bút bi xanh và $5$ chiếc bút bi đen, các chiếc bút có cùng kích thước và khối lượng. Bạn Sơn lấy ngẫu nhiên một chiếc bút bi từ trong hộp, không trả lại. Sau đó bạn Tùng lấy ngẫu nhiên một trong $11$ chiếc bút còn lại. Tính xác suất để Sơn lấy được bút bi đen và Tùng lấy được bút bi xanh.
	\choice
	{$\dfrac{139}{132}$}
	{\True$\dfrac{35}{132}$}
	{$\dfrac{25}{132}$}
	{$\dfrac{49}{132}$}
	\loigiai{
	Gọi $A$ là biến cố: \lq\lq  Bạn Sơn lấy được bút bi đen\rq\rq ;\\
	$B$ là biến cố: ``Bạn Tùng lấy được bút bi xanh''.\\
	Ta cần tính $\mathrm{P}(AB)$.\\
	Vì $n(A)=5$ nên $\mathrm{P}(A)=\dfrac{5}{12}$.\\
	Nếu $A$ xảy ra tức là bạn Sơn lấy được bút bi đen thì trong hộp có 11 bút bi với 7 bút bi xanh.\\
	Vậy $\mathrm{P}(B\mid A)=\dfrac{7}{11}$.\\
	Theo công thức nhân xác suất: $\mathrm{P}(AB)=\mathrm{P}(A)\cdot \mathrm{P}(B\mid A)=\dfrac{5}{12}\cdot\dfrac{7}{11}=\dfrac{35}{132}$.\\
	Một phương pháp mô tả trực quan lời giải trên là dùng sơ đồ hình cây
	\begin{center}
	\begin{tikzpicture}[line join = round,line cap = round, >=stealth, thick, font = \small, scale = 1,yscale=0.85]
	%	\draw[gray!50,xstep = 1, ystep = 1] (-5,-5) grid (5,5);
	\path
	(0,0) coordinate (0) node[above=3mm] {$O$}
	+(-2,-3) coordinate (11) node[left=3mm] {Đ}
	+(2,-3) coordinate (12) node[right=3mm] {X}
	(11)+(-1,-3) coordinate (21) node[below=3mm] {Đ} node[below=9mm] {ĐĐ}
	+(1,-3) coordinate (22) node[below=3mm] {X} node[below=9mm] {ĐX}
	(12)+(-1,-3) coordinate (23) node[below=3mm] {Đ} node[below=9mm] {XĐ}
	+(1,-3) coordinate (24) node[below=3mm] {X} node[below=9mm] {XX}
	(0)--(11) node[midway,left=3mm] {$\dfrac{5}{12}$}
	(0)--(12) node[midway,right=3mm] {$\dfrac{7}{12}$}
	(11)--(21) node[midway,left=2mm] {$\dfrac{4}{11}$}
	(11)--(22) node[midway,right=2mm] {$\dfrac{7}{11}$}
	(12)--(23) node[midway,left=2mm] {$\dfrac{5}{11}$}
	(12)--(24) node[midway,right=2mm] {$\dfrac{6}{11}$}
	;
	\draw (11)--(0)--(12)
	(21)--(11)--(22)
	(23)--(12)--(24)
	;
	\node[circle, line width = .2 mm, draw = black, text = black, fill = yellow!60, anchor = center, outer sep = 0pt, minimum size = .2cm] (c) at (0) {};
	\node[circle, line width = .2 mm, draw = black, text = black, fill = black, anchor = center, outer sep = 0pt, minimum size = .2cm] (c) at (11) {};
	\node[circle, line width = .2 mm, draw = black, text = black, fill = cyan, anchor = center, outer sep = 0pt, minimum size = .2cm] (c) at (12) {};
	\node[circle, line width = .2 mm, draw = black, text = black, fill = black, anchor = center, outer sep = 0pt, minimum size = .2cm] (c) at (21) {};
	\node[circle, line width = .2 mm, draw = black, text = black, fill = cyan, anchor = center, outer sep = 0pt, minimum size = .2cm] (c) at (22) {};
	\node[circle, line width = .2 mm, draw = black, text = black, fill = black, anchor = center, outer sep = 0pt, minimum size = .2cm] (c) at (23) {};
	\node[circle, line width = .2 mm, draw = black, text = black, fill = cyan, anchor = center, outer sep = 0pt, minimum size = .2cm] (c) at (24) {};
	\end{tikzpicture}
	\end{center}
	Trên nhánh OĐ và OX tương ứng ghi xác suất lấy được bút đen và bút xanh.\\
	Trên nhánh ĐĐ, ĐX tương ứng ghi xác suất lấy được bút đen, bút xanh với điều kiện đã lấy được bút đen.\\
	Trên nhánh XĐ, XX tương ứng ghi xác suất lấy được bút đen, bút xanh với điều kiện đã lấy được bút xanh.\\
	Vậy xác suất cần tính là $\dfrac{5}{12}\cdot\dfrac{7}{11}=\dfrac{35}{132}$.
	}
\end{ex}
\begin{ex}%[2D5H1-2]
	Một hộp kín đựng $20$ tấm thẻ giống hệt nhau đánh số từ $1$ đến $20$. Một người rút ngẫu nhiên ra một tấm thẻ từ trong hộp. Người đó được thông báo rằng thẻ rút ra mang số chẵn. Tính xác suất để người đó rút được thẻ số $10$.
	\choice
	{$0{,}4$}
	{$0{,}3$}
	{$0{,}2$}
	{\True$0{,}1$}
	\loigiai
	{	Gọi $A$ là biến cố: \lq\lq  Người đó rút được thẻ số 10\rq\rq.\\
	Gọi $B$ là biến cố: \lq\lq  Người đó rút được thẻ mang số chẵn\rq\rq .\\
	Không gian mẫu mà 20 tấm thẻ đánh số từ 1 đến 20 $\Rightarrow$ $n(\Omega)=20$.\\
	Ta cần tính $P\left(A\mid B\right)$.\\
	Ta có, từ 1 đến 20 có 10 số chẵn nên $n(B)=10$.\\
	Vậy $\mathrm{P}(B)=\dfrac{n(B)}{n(\Omega)}=\dfrac{10}{20}=\dfrac{1}{2}$.\\
	Trong số 10 số chẵn có một số 10 nên $n(AB)=1$.\\
	Vậy $\mathrm{P}(AB)=\dfrac{n(AB)}{n(\Omega)}=\dfrac{1}{20}$.\\
	Do đó $\mathrm{P}(A\mid B)=\dfrac{\mathrm{P}(AB)}{\mathrm{P}(B)}=\dfrac{1}{10}= 0{,}1$.
	}
\end{ex}
\begin{ex}%[2D6N1-2]
	Cho hai biến cố $A$, $B$ xung khắc với nhau thỏa $\mathrm{P}(A)=0{,}2$; $\mathrm{P}(B)=0{,}4$. Khi đó $\mathrm{P}\left(A|B\right)$ bằng
	\choice
	{$0{,}5$}
	{$0{,}2$}
	{$0{,}4$}
	{\True $0$}
	\loigiai{
	Do $A$, $B$ xung khắc nên $\mathrm{P}\left(A\cap B\right)=0$. \\
	Vậy $\mathrm{P}\left(A|B\right)=\dfrac{\mathrm{P}\left(A\cap B\right)}{\mathrm{P}(B)}=0$.
	}
\end{ex}	
\begin{ex}%[2D6H1-2] 
	Trong một kì thi học sinh giỏi môn Toán của một tỉnh có $200$ học sinh tham gia, trong đó có $95$ học sinh nữ và $105$ học sinh nam. Kết quả của kì thi cho biết có $50$ học sinh đạt giải (bao gồm nhất, nhì và ba), trong đó có $24$ học sinh nữ và $26$ học sinh nam. Chọn ngẫu nhiên một học sinh trong số $200$ học sinh đó. Tính xác suất để học sinh được chọn có giải, biết rằng học sinh đó là nam.
	\choice
	{$\dfrac{8}{35}$}
	{$\dfrac{24}{95}$}
	{$\dfrac{26}{95}$}
	{\True $\dfrac{26}{105}$}
	\loigiai{
	Xét các biến cố:\\
	$A$: "Học sinh được chọn đạt giải."\\
	$B$: "Học sinh được chọn là nam."\\
	Ta có: $\mathrm{P}(A\cap B)=\dfrac{26}{200}=0{,}13$ và $\mathrm{P}(B)=\dfrac{105}{200}=0{,}525$.\\
	Xác xuất để học sinh được chọn đạt giải, biết rằng học sinh đó là nam, là
	$$\mathrm{P}(A|B)=\dfrac{\mathrm{P}(A\cap B)}{\mathrm{P}(B)}=\dfrac{0{,}13}{0{,}525}=\dfrac{26}{105}.$$
	}
\end{ex}
\begin{ex}%[2D6H1-2]
	Trong một lô hàng áo thun trơn gồm $1000$ chiếc của một công ty dệt may có $100$ áo thun có màu đỏ. Các áo thun đỏ đó gồm có các kích thước là $X$, $M$ và $L$, trong đó có $30$ áo kích thước $M$. Chọn ngẫu nhiên một chiếc áo trong lô hàng đó. Giả sử chiếc áo được chọn là màu đỏ, tính xác suất để chiếc áo đó có kích thước $M$.
	\choice
	{$0{,}03$}
	{\True $0{,}3$}
	{$0{,}01$}
	{$0{,}3$}
	\loigiai{
	Xét các biến cố:\\
	$A$: "Áo được chọ có kích thước $M$.";\\
	$B$: "Áo được chọn là màu đỏ."\\
	Khi đó, xác suất để chiếc áo được chọn có kích thước $M$, biết rằng áo thun đó có màu đỏ, là xác suất có điều kiện $\mathrm{P}(A|B)$, ta có:
	$$\mathrm{P}(A|B)=\dfrac{n(A\cap B)}{n(B)}=\dfrac{30}{100}=0,3$$
	}
\end{ex}
\begin{ex}%[2D6H1-2]
	Trong một hộp đựng $500$ chiếc thẻ cùng loại có $200$ chiếc thẻ màu vàng. Trên mỗi chiếc thẻ màu vàng có ghi một trong năm số: $1$, $2$, $3$, $4$, $5$. Có $40$ chiếc thẻ màu vàng ghi số $5$. Chọn ra ngẫu nhiên một chiếc thẻ trong hộp đựng thẻ. Giả sử chiếc thẻ chọn ra có màu vàng. Tính xác suất để chiếc thẻ đó ghi số $5$.
	\choice
	{\True $0{,}2$}
	{$\dfrac{2}{15}$}
	{$0{,}4$}
	{$\dfrac{4}{15}$}
	\loigiai{
	Gọi biến cố $A$: \lq\lq  Thẻ được chọn ghi số $5$\rq\rq. \\
	Gọi biến cố $B$: \lq\lq  Thẻ được chọn có màu vàng\rq\rq. \\
	Có $\mathrm{P}\left(A|B\right)=\dfrac{n\left(A\cap B\right)}{n(B)}=\dfrac{40}{200}=0{,}2$. 
	}
\end{ex}	
\begin{ex}%[2D6H1-2]
	Một hộp đựng $8$ viên bi màu đỏ và $5$ viên bi màu vàng (các viên bi có kích thước và khối lượng như nhau). Có $5$ viên bi trong hộp được đánh số, trong đó có $3$ viên bi màu đỏ và $2$ viên bi màu vàng. Lấy ngẫu nhiên một viên bi trong hộp. Tính xác suất để viên bi được lấy ra có màu đỏ, biết rằng viên bi đó được đánh số.
	\choice
	{$0{,}4$}
	{\True $0{,}6$}
	{$0{,}2$}
	{$0{,}3$}
	\loigiai{
	Gọi biến cố $A$: \lq\lq  Viên bi được lấy ra có màu đỏ \rq\rq. \\
	Gọi biến cố $B$: \lq\lq  Viên bi được lấy ra có đánh số \rq\rq. \\
	Có $\mathrm{P}\left(A|B\right)=\dfrac{n\left(A\cap B\right)}{n(B)}=\dfrac{3}{5}=0{,}6$. 
	}
\end{ex}	
\begin{ex}%[2D6H1-2]
	Gieo lần lượt hai con xúc xắc cân đối và đồng chất. Tính xác suất để tổng số chấm xuất hiện trên hai con xúc xắc không nhỏ hơn $6$, biết rằng xúc xắc thứ nhất xuất hiện mặt $4$ chấm.
	\choice
	{$\dfrac{5}{24}$}
	{$\dfrac{5}{12}$}
	{$\dfrac{5}{36}$}
	{\True $\dfrac{5}{6}$}
	\loigiai{
	Gọi biến cố $A$: \lq\lq  Tổng số chấm xuất hiện trên hai con xúc xắc bằng $6$ \rq\rq. \\
	Gọi biến cố $B$: \lq\lq  Xúc xắc thứ nhất xuất hiện mặt $4$ chấm\rq\rq. \\
	Có $\mathrm{P}\left(A|B\right)=\dfrac{n\left(A\cap B\right)}{n(B)}=\dfrac{5}{6}$. 
	}
\end{ex}
\begin{ex}%[2D6H2-2] 
	Cho hai biến cố $A$ và $B$ thỏa $\mathrm{P}(A)=0{,}4$; $\mathrm{P}(B)=0{,}8$; $\mathrm{P}\left(A| \overline{B}\right)=0{,}5$. Tính $\mathrm{P}(A|B)$.
	\choice
	{$0{,}4$}
	{\True $0{,}375$}
	{$0{,}5$}
	{$0{,}625$}
	\loigiai{
	Ta có
	\begin{eqnarray*}
	& & \mathrm{P}\left(A| \overline{B}\right) = \dfrac{\mathrm{P}(A \overline{B})}{\mathrm{P}(\overline{B})}\\
	&\Leftrightarrow & 0{,}5 = \dfrac{\mathrm{P}(A)-\mathrm{P}(AB)}{1-P(B)}\\
	&\Leftrightarrow & \mathrm{P}(AB)=\mathrm{P}(A) - 0{,}5\cdot[1-P(B)]\\
	&\Leftrightarrow & \mathrm{P}(AB)=0{,}3.
	\end{eqnarray*}
	Khi đó 
	$$\mathrm{P}\left(A|B\right)=\dfrac{\mathrm{P}\left(AB\right)}{\mathrm{P}(B)}=0{,}375.$$ }
\end{ex}
\begin{ex}%[2D6V1-4]
	Một công ty vừa ra mắt sản phẩm $X$ và tổ chức ngày trải nghiệm sản phẩm. Họ thống kê được trong $200$ người đến tham quan ngày trải nghiệm có $60$ người là nam giới và $140$ người là nữ giới. Trong số những người được thống kê này, có $120$ người mua sản phẩm $X$, gồm $40$ khách hàng nam và $80$ khách hàng nữ, còn lại là không mua sản phẩm $X$. Chọn ngẫu nhiên một người trong số $200$ người được thống kê. Tính xác suất để người này mua sản phẩm $X$, biết rằng người được chọn là nữ giới.
	\choice
	{$\dfrac{2}{3}$}
	{$\dfrac{7}{10}$}
	{$\dfrac{2}{5}$}
	{\True $\dfrac{4}{7}$}
	\loigiai{
	Xét các biến cố:\\
	$A$: "Người được chọn mua sản phẩm $X$.";\\
	$B$: "Người được chọn là nữ giới."\\
	Khi đó xác suất để người này mua sản phẩm $X$, biết rằng người được chọn là nữ giới là xác suất của $A$ với điều kiện $B$.\\
	Có $80$ người mua sản phẩm $X$ là nữ giới nên $\mathrm{P}(AB)=\dfrac{80}{200}=0{,}4$.\\
	Có $140$ người là nữ giới trong số lượng thống kê nên $\mathrm{P}(B)=\dfrac{140}{200}=0{,}7$.\\
	Vậy $\mathrm{P}(A|B)=\dfrac{\mathrm{P}(AB)}{\mathrm{P}(B)}=\dfrac{4}{7}$.
	}
\end{ex}
\begin{ex}%[2D6V2-4] 
	Theo kết quả từ trạm nghiên cứu khí hậu tại một địa phương, xác suất để một ngày có gió là $0{,}6$. Nếu ngày đó có gió thì xác suất có mưa là $0{,}4$. Tính xác suất để trời có gió nhưng không có mưa ở địa phương đó trong một ngày.
	\choice
	{$0{,}6$}
	{\True $0{,}36$}
	{$0{,}24$}
	{$0{,}16$}
	\loigiai{
	Xét các biến cố:\\
	$A$:"Ngày có gió" và $B$: "Ngày có mưa".\\
	Xác suất để trời có gió nhưng không có mưa ở địa phương đó trong một ngày là $P(A\overline{B})$.\\
	Theo đề bài, nếu ngày có gió thì xác suất có mưa là $0{,}4$ nên $\mathrm{P}(B|A)=0{,}4$.\\
	Suy ra $\mathrm{P}(\overline{B}|A)=1-0{,}4=0{,}6$.\\
	Ta có
	$$ \mathrm{P}(\overline{B}|A)=\dfrac{\mathrm{P}(A\overline{B})}{\mathrm{P}(A)} \Rightarrow \mathrm{P}(A\overline{B})=\mathrm{P}(A)\cdot \mathrm{P}(\overline{B}|A) = 0{,}6 \cdot 0{,}6 =0{,}36 $$}
\end{ex}
\Closesolutionfile{ans}
\indapan{6}{ans/ans-2-B18}
%\TNTF
\Opensolutionfile{ans}[ans/ans-2-B18-DS]
\begin{ex}%[2D5H1-2]
	Hai xạ thủ An và Bình bắn vào cùng một mục tiêu ở hai thời điểm khác nhau với xác suất bắn trúng mục tiêu lần lượt là $0{,}6$ và $0{,}7$. Xét các biến cố\\ 
	$A$: \lq\lq  Xạ thủ An bắn trúng mục tiêu\rq\rq;\\
	$B$: \lq\lq  Xạ thủ Bình bắn trúng mục tiêu \rq\rq.\\
	Xét tính đúng sai của các khẳng định sau?
	\choiceTF
	{$\mathrm{P}(\overline{A})=0{,}6$; $\mathrm{P}(\overline{B})=0{,}7$}
	{\True Hai biến cố $\overline{A}, \overline{B}$ là độc lập}
	{Xác suất cả hai xạ thủ đều không bắn trúng mục tiêu là $0{,}42$}
	{Xác suất cả hai xạ thủ đều bắn trúng mục tiêu là $0{,}58$}
	\loigiai{
	\begin{itemchoice}
	\itemch Sai. Do $\mathrm{P}(A)=0{,}6$; $\mathrm{P}(B)=0{,}7$.
	\itemch Đúng. Vì hai xạ thủ bắn ở hai thời điểm khác nhau nên các cặp biến cố $A$ và $B, \overline{A}$ và $\overline{B}$ là độc lập.
	\itemch Sai. Vì $\mathrm{P}(\overline{A} \cap \overline{B})=0{,}4 \cdot 0{,}3=0{,}12$.
	\itemch Sai. Vì $\mathrm{P}(A \cap B)=0{,}6 \cdot 0{,}7=0{,}42$.
	\end{itemchoice}
	}
\end{ex}
\begin{ex}%[2D5H1-2]
	Một xạ thủ bắn vào bia số $1$ và bia số $2$. Xác suất để xạ thủ đó bắn trúng bia số $1$, bia số $2$ lần lượt là $0{,}8$; $0{,}9$. Xác suất để xạ thủ đó bắn trúng cả hai bia là $0{,}8$. Xét hai biến cố
	\begin{itemize}
	\item $A$: \lq\lq  Xạ thủ đó bắn trúng bia số $1$\rq\rq;
	\item $B$: \lq\lq  Xạ thủ đó bắn trúng bia số $2$\rq\rq.
	\end{itemize}
	Xét tính đúng sai của các khẳng định sau?
	\choiceTF
	{Hai biến cố $A$ và $B$ có độc lập}
	{Biết xạ thủ đó bắn trúng bia số $1$ thì xác suất xạ thủ đó bắn trúng bia số $2$ là $0{,}72$}
	{\True Biết xạ thủ đó không bắn trúng bia số $1$, thì xác suất xạ thủ đó bắn trúng bia số $2$ bằng $0{,}9$}
	{Biết xạ thủ đó không bắn trúng bia số $1$ thì xác suất xạ thủ đó bắn không trúng bia số $2$ bằng $0{,}9$}
	\loigiai{
	\begin{itemchoice}
	\itemch Đúng. $A$ và $B$ là hai biến cố độc lập.
	\itemch Sai. Xác suất xạ thủ đó bắn trúng bia số $2$ và bia số $1$ là $$\mathrm{P}(B|A)=\dfrac{\mathrm{P}(B\cap A)}{\mathrm{P}(A)}=\dfrac{\mathrm{P}(B) \cdot \mathrm{P}(A)}{\mathrm{P}(A)}=\mathrm{P}(B)=0{,}9.$$
	\itemch Đúng. Xác suất xạ thủ đó bắn không bắn trứng bia số $1$ mà trúng bia số $2$ là $$\mathrm{P}(B|\overline{A})=\dfrac{\mathrm{P}(B\cap \overline{A})}{\mathrm{P}(\overline{A})}=\dfrac{\mathrm{P}(B) \cdot \mathrm{P}(\overline{A})}{\mathrm{P}(\overline{A})}=\mathrm{P}(B)=0{,}9.$$
	\itemch Sai. Xác suất xạ thủ đó bắn không bắn trúng bia số $1$ và không trúng bia số $2$ là
	$$\mathrm{P}(\overline{B}|\overline{A})=\dfrac{\mathrm{P}(\overline{B}\cap \overline{A})}{\mathrm{P}(\overline{A})}=\dfrac{\mathrm{P}(\overline{B}) \cdot \mathrm{P}(\overline{A})}{\mathrm{P}(\overline{A})}=\mathrm{P}(\overline{B})=1-\mathrm{P}(B)=1-0{,}9=0{,}1.$$
	\end{itemchoice}
	}
\end{ex}
\begin{ex}%[2D6C1-2] 
	Cho $A$ và $B$ là hai biến cố độc lập với $\mathrm{P}(A)=0{,}7$ và $\mathrm{P}(B)=0{,}4$. Xét tính đúng sai của các khẳng định sau:
	\choiceTF[4]
	{$\mathrm{P}(A|B)=0{,}6$}
	{\True $\mathrm{P}(B|\overline{A})=0{,}4$}
	{\True $\mathrm{P}(\overline{A}|B)=0{,}3$}
	{\True $\mathrm{P}(\overline{B}|\overline{A})=0{,}6$}
	\loigiai{
	\begin{itemchoice}
	\itemch $A$ và $B$ độc lập nên $\mathrm{P}(A|B)= \dfrac{\mathrm{P}(A)\cdot \mathrm{P}(B)}{\mathrm{P}(B)} =\mathrm{P}(A)=0{,}7$.
	\itemch $\overline{A}$ và $B$ độc lập nên $\mathrm{P}(B|\overline{A})=\dfrac{\mathrm{P}(B)\cdot \mathrm{P}(\overline{A})}{\mathrm{P}(\overline{A})}=\mathrm{P}(B)=0{,}4$.
	\itemch $\overline{A}$ và $B$ độc lập nên $\mathrm{P}(\overline{A}|B)= \dfrac{\mathrm{P}(B)\cdot \mathrm{P}(\overline{A})}{\mathrm{P}(B)}= \mathrm{P}(\overline{A})=1-\mathrm{P}(A)=0{,}3$.
	\itemch $\overline{A}$ và $\overline{B}$ độc lập nên $\mathrm{P}(\overline{B}|\overline{A})=\dfrac{\mathrm{P}(\overline{B})\cdot \mathrm{P}(\overline{A})}{\mathrm{P}(\overline{A})}=\mathrm{P}(\overline{B})=0{,}6$.
	\end{itemchoice}
	}
\end{ex}
\begin{ex}%[2D6C1-2] 
	Cho $A$ và $B$ với $\mathrm{P}(\overline{A})=0{,}4$; $\mathrm{P}(B)=0{,}8$ và $\mathrm{P}(AB)=0{,}4$. Xét tính đúng sai của các khẳng định sau:
	\choiceTF
	{\True $\mathrm{P}(A|B)=\dfrac{1}{2}$}
	{$\mathrm{P}(B|A)=\dfrac{1}{2}$}
	{\True $\mathrm{P}(\overline{B}|A)=\dfrac{1}{3}$}
	{$\mathrm{P}(\overline{A}B)=\dfrac{3}{5}$}
	\loigiai{
	\begin{itemchoice}
	\itemch $\mathrm{P}(A|B)=\dfrac{\mathrm{P}(AB)}{\mathrm{P}(B)}=\dfrac{1}{2}$.
	\itemch $\mathrm{P}(B|A)=\dfrac{P(AB)}{P(A)}=\dfrac{P(AB)}{1-P(\overline{A})}=\dfrac{2}{3}$.
	\itemch $\mathrm{P}(\overline{B}|A)=\dfrac{\mathrm{P}(A\overline{B})}{\mathrm{P}(A)}=\dfrac{\mathrm{P}(A)-\mathrm{P}(AB)}{\mathrm{P}(A)}=\dfrac{1}{3}$.
	\itemch $\mathrm{P}(\overline{A}B)=\mathrm{P}(\overline{A}|B) \cdot \mathrm{P}(B)= \dfrac{\mathrm{P}(\overline{A}B)}{\mathrm{P}(B)} \cdot \mathrm{P}(B) = \mathrm{P}(\overline{A}B) = \mathrm{P}P(B)-\mathrm{P}(AB) =\dfrac{2}{5}$.
	\end{itemchoice}
	}
\end{ex}
\begin{ex}%[2D6C1-4] 
	Một công ty truyền thông đấu thầy $2$ dự án. Khả năng thắng thầu của dự án $1$ là $0{,}5$ và dự án $2$ là $0{,}6$. Khả năng thắng thầu cả $2$ dự án là $0{,}4$. Gọi $A$, $B$ lần lượt là biến cố thắng thầu dự án $1$ và $2$. Xét tính đúng sai của các khẳng định sau:
	\choiceTF
	{$A$, $B$ là hai biến cố độc lập.}
	{\True Xác xuất công ty thắng thầu đúng $1$ dự án là $0{,}3$}
	{\True Biết công ty thắng thầu dự án $1$, xác suất công ty thắng thầu dự án $2$ là $0{,}8$}
	{Biết công ty không thắng thầu dự án $1$, xác suất công ty thắng thầu dự án $2$ là $0{,}6$}
	\loigiai{
	\begin{itemchoice}
	\itemch Theo đề bài: $\mathrm{P}(A)=0{,}5$; $\mathrm{P}(B)=0{,}6$ và $\mathrm{P}(AB)=0{,}4$.\\
	Vì $\mathrm{P}(AB) \ne \mathrm{P}(A)\cdot \mathrm{P}(B)$ nên $A$, $B$ không độc lập.
	\itemch Gọi biến cố $C$: "Thắng thầu đúng một dự án."\\
	Khi đó
	\begin{eqnarray*}
	& \mathrm{P}(C) & =\mathrm{P}(\overline{A}B)+\mathrm{P}(A\overline{B})\\
	& & = [\mathrm{P}(B)-\mathrm{P}(AB)] + [ \mathrm{P}(A)-\mathrm{P}(AB) ]\\
	& & = \mathrm{P}(A) + \mathrm{P}(B) - 2\mathrm{P}(AB)\\
	&& = 0{,}3.
	\end{eqnarray*}
	\itemch Biết công ty thắng thầu dự án $1$, xác suất công ty thắng thầu dự án $2$ là
	$$\mathrm{P}(B|A)=\dfrac{\mathrm{P}(AB)}{\mathrm{P}(A)}=\dfrac{0{,}4}{0{,}5}=0{,}8.$$
	\itemch Biết công ty không thắng thầu dự án $1$, xác suất công ty thắng thầu dự án $2$ là
	$$\mathrm{P}(B|\overline{A})=\dfrac{\mathrm{P}(\overline{A}B)}{\mathrm{P}(\overline{A})}=\dfrac{\mathrm{P}(B)-\mathrm{P}(AB)}{\mathrm{P}(\overline{A})}=\dfrac{0{,}6-0{,}4}{0{,}5}=0{,}4.$$
	\end{itemchoice}
	}
\end{ex}
\begin{ex}%[2D6C1-4] 
	Ở một sân bay, người ta sử dụng một loại máy soi tự động phát hiện hàng cấm trong hành lí kí gửi. Máy phát chuông cảnh báo với $95 \%$ các kiện hành lí có chứa hàng cấm và $2\%$ các kiện hành lí không chứa hàng cấm. Tỉ lệ các kiện hành lí có chứa hàng cấm là $0{,1}\%$. Chọn ngẫu nhiên một kiện hành lí để soi bằng máy trên. Xét tính đúng sai của các mệnh đề sau:
	\choiceTF
	{\True Máy không phát chuông cảnh báo với $5\%$ các kiện hành lí có chứa hàng cấm}
	{\True Máy không phát chuông cảnh báo với $98\%$ các kiện hành lí không chứa hàng cấm}
	{Xác suất chọn được kiện hành lí có chứa hàng cấm và máy phát chuông cảnh báo là $0{,}0095$}
	{\True Xác suất chọn được kiện hành lí không chứa hàng cấm và máy phát chuông cảnh báo là $0{,}01998$}
	\loigiai{
	Xét các biến cố:\\
	$A$: "Kiện hành lí có chứa hàng cấm."\\
	$B$: "Máy phát chuông cảnh báo."\\
	Theo đề, ta có $\mathrm{P}(B|A)=0{,}95$; $\mathrm{P}(B| \overline{A})=0{,}02$ và $\mathrm{P}(A)=0{,}001$.
	\begin{itemchoice}
	\itemch Xác suất máy không phát chuông cảnh báo với các kiện hành lí có chứa hàng cấm là
	$$P(\overline{B}|A)=1-P(B|A)=0{,}05=5\%.$$
	\itemch Xác suất máy không phát chuông cảnh báo với các kiện hành lí không chứa hàng cấm là
	$$\mathrm{P}(\overline{B}|\overline{A})=1-\mathrm{P}(B|\overline{A})=0{,}05=1-0{,}02=98\%.$$
	\itemch 
	Xác suất chọn được kiện hành lí có chứa hàng cấm và máy phát chuông cảnh báo là
	$$ \mathrm{P}(AB)=P(B|A) \cdot \mathrm{P}(A) =0{,}00095 = 0{,}095\% .$$
	\itemch Ta có
	\begin{eqnarray*}
	& & \mathrm{P}(B| \overline{A})=0,{,}02\\
	&\Leftrightarrow & \dfrac{\mathrm{P}(B)-\mathrm{P}(AB)}{\mathrm{P}(\overline{A})}=0{,}02\\
	&\Leftrightarrow & \mathrm{P}(B)= 0{,}02 \cdot \mathrm{P}(\overline{A}) + \mathrm{P}(AB)\\
	&\Leftrightarrow & \mathrm{P}(B)= 0{,}02093.
	\end{eqnarray*}
	Ta có $\mathrm{P}(AB)=\mathrm{P}(B|A) \cdot \mathrm{P}(A) =0{,}00095 = 0{,}095\% $.\\
	Xác suất chọn được kiện hành lí không chứa hàng cấm và máy phát chuông cảnh báo là
	$$\mathrm{P}(\overline{AB})=\mathrm{P}(B)-\mathrm{P}(AB)= 0{,}02093 - 0{,}00095= 0{,}01998.$$
	\end{itemchoice}
	}
\end{ex}
\begin{ex}%[2D5H1-2]
	Một lớp học có $17$ học sinh nam và $24$ học sinh nữ. Cô giáo gọi ngẫu nhiên lần lượt $2$ học sinh (có thứ tự) lên trả lời câu hỏi. Xét các biến cố\\
	$A$: \lq\lq  Lần thứ nhất cô giáo gọi $1$ học sinh nam\rq\rq;\\
	$B$: \lq\lq  Lần thứ hai cô giáo gọi $1$ học sinh nữ\rq\rq.\\
	Xét tính đúng sai của các khẳng định sau?
	\choiceTF
	{$\mathrm{P}(B \mid A)=0{,}575$}
	{$\mathrm{P}(B \mid \overline{A})=0{,}6$}
	{$\mathrm{P}(\overline{B} \mid A)=0{,}425$}
	{$\mathrm{P}(\overline{B} \mid \overline{A})=0{,}4$}
	\loigiai{
	Nếu lần thứ nhất gọi $1$ học sinh nam thì số học sinh còn lại là $40$ , số học sinh nam còn lại là $16$, số học sinh nữ giữ nguyên; nếu lần thứ nhất gọi $1$ học sinh nữ thì số học sinh còn lại là $40$, số học sinh nam giữ nguyên, số học sinh nữ còn lại là $23$.
	\begin{itemchoice}
	\itemch Sai. Vì $\mathrm{P}(B \mid A)=\dfrac{24}{40}=0{,}6$.
	\itemch	Sai. Vì $\mathrm{P}(B \mid \overline{A})=\dfrac{23}{40}=0{,}575$.
	\itemch Sai. Vì $\mathrm{P}(\overline{B} \mid A)=\dfrac{16}{40}=0{,}4$.
	\itemch Sai. Vì $\mathrm{P}(\overline{B} \mid \overline{A})=\dfrac{17}{40}=0{,}425$.
	\end{itemchoice}
	}
\end{ex}
\begin{ex}%[2D5H1-2]
	Gieo một xúc xắc cân đối và đồng chất $1$ lần. Xét các biến cố:\\
	$A$: \lq\lq  Mặt xuất hiện của xúc xắc ghi số $5$\rq\rq;\\
	$B$: \lq\lq  Mặt xuất hiện của xúc xắc ghi số lẻ\rq\rq.\\
	Xét tính đúng sai của các khẳng định sau?
	\choiceTF
	{$\mathrm{P}(A)=\dfrac{5}{6}$}
	{\True $\mathrm{P}(A \cap B)=\dfrac{1}{6}$}
	{\True $\mathrm{P}(B \mid A)=1$}
	{$\mathrm{P}(A \mid B)=\dfrac{1}{2}$}
	\loigiai{
	\begin{itemchoice}
	\itemch Sai. Vì $\mathrm{P}(A)=\dfrac{1}{6}$ .
	\itemch	Đúng. Vì $\mathrm{P}(A \cap B)=\dfrac{1}{6}$ .
	\itemch Đúng. Vì $\mathrm{P}(B \mid A)=1$.
	\itemch Sai. Vì $\mathrm{P}(B)=\dfrac{1}{2}$. Khi đó, $\mathrm{P}(A \mid B)=\dfrac{\frac{1}{6}}{\frac{1}{2}}=\dfrac{1}{3}$.
	\end{itemchoice}
	}
\end{ex}
\begin{ex}%[2D5V1-2]
	Một cửa hàng kinh doanh tổ chức rút thăm trúng thưởng cho hai loại sản phẩm. Tỉ lệ trúng thưởng của các loại sản phẩm I, II lần lượt là $6 \%$; $4 \%$. Trong một hộp kín gồm các thăm cùng loại, người ta để lẫn lộn $200$ chiếc thăm cho sản phẩm loại I và $300$ chiếc thăm cho sản phẩm loại II. Một khách hàng lấy ngẫu nhiên $1$ chiếc thăm từ chiếc hộp đó. Xét tính đúng sai của các khẳng định sau
	\choiceTF
	{Xác suất để chiếc thăm được lấy ra là trúng thưởng bằng $10\%$}
	{Xác suất thăm được lấy ra trúng thưởng là thăm cho sản phẩm loại I bằng $6\%$ }
	{\True Xác suất thăm được lấy ra trúng thưởng là thăm cho sản phẩm loại II bằng $50\%$}
	{Khả năng lấy ra được thăm trúng thưởng là thăm sản phẩm loại II cao hơn khả năng lấy ra được thăm trúng thưởng là thăm sản phẩm loại I}
	\loigiai{
	\begin{itemchoice}
	\itemch Sai. Xét biến cố $A$: \lq\lq  Chiếc thăm được lấy ra là trúng thưởng\rq\rq.\\	
	Khi đó, ta có\\	
	$\mathrm{P}(A)=\dfrac{6\% \cdot 200 + 4\% \cdot 300}{200+300}=0{,}048$.	
	\itemch Sai. Xét biến cố $B$: \lq\lq  Chiếc thăm được lấy ra là thăm cho sản phẩm loại I\rq\rq.\\
	Khi đó, ta có:\\
	$\mathrm{P}(B|A)=\dfrac{n(B\cap A)}{n(A)}=\dfrac{6\% \cdot 200}{6\% \cdot 200 + 4\% \cdot 300}=0{,}5$.
	\itemch	Đúng. Xét biến cố $C$: \lq\lq  Chiếc thăm được lấy ra là thăm cho sản phẩm loại II\rq\rq.\\
	Khi đó, ta có:\\
	$\mathrm{P}(C|A)=\dfrac{n(C\cap A)}{n(A)}=\dfrac{4\% \cdot 300}{6\% \cdot 200 + 4\% \cdot 300}=0{,}5$.
	\itemch Sai. Do $\mathrm{P}(B|A) = \mathrm{P}(C|A)$ nên xác suất hai chiếc thăm lấy được là như nhau.
	\end{itemchoice}	
	}
\end{ex}
\begin{ex}%[2D5V1-2]
	Một phòng nghiên cứu dược học cho $500$ người bị bệnh $H$ dùng hai loại thuốc $X$, $Y$ để điều trị. Một số người được điều trị bằng thuốc $X$ và số người còn lại được điều trị bằng thuốc $Y$. Kết quả nghiên cứu được trình bày ở bảng sau 
	\begin{center}
	\begin{tikzpicture}
	\begin{scope}[xscale=4.4]
	\path
	(0,0) foreach \i[count=\k] in {$X$,$Y$} {++(1,0)node(1\k){\i}}
	(0,-1) node {Khỏi bệnh} foreach \i[count=\k] in {$180$,$190$} {++(1,0)node(2\k){\i}}
	(0,-2) node{Không khỏi bệnh} foreach \i[count=\k] in {$60$,$70$} {++(1,0)node(3\k){\i}};
	\draw[shift={(-0.5,.5)}] (0,0) grid (3.,-3)
	(0,0)--(1.,-1)
	(0,-1) node[above right]{Tình trạng}
	(1,0) node[below left]{Loại thuốc}
	;
	\end{scope}
	\end{tikzpicture}
	\end{center}
	Chọn ngẫu nhiên một người trong số này. Gọi $A$ là biến cố "Người được chọn khỏi bệnh", $B$ là biến cố "Người được chọn điều trị bằng thuốc $X$", $C$ là biến cố "Người được chọn điều trị bằng thuốc $Y$". \\ Xét tính đúng sai của các khẳng định sau?
	\choiceTF
	{Xác suất để một người khỏi bệnh khi điều trị bằng thuốc $Y$ có kí hiệu là $\mathrm{P}(A|B)$ }
	{\True $\mathrm{P}(A|B)=\dfrac{3}{4}$}
	{Xác xuất để một người khỏi bênh khi điều trị bằng thuốc $Y$ bằng $\dfrac{7}{26}$}
	{\True Thuốc $X$ có hiệu quả hơn thuốc $Y$ trong điều trị bệnh}
	\loigiai{
	\begin{itemchoice}
	\itemch Sai. Vì xác suất để một người khỏi bệnh khi điều trị bằng thuốc $X$ có kí hiệu là $\mathrm{P}(A|B)$. 
	\itemch Đúng. Vì $\mathrm{P}(A|B)=\dfrac{n\left(AB\right)}{n(B)}=\dfrac{180}{240}=\dfrac{3}{4}$.
	\itemch Sai. Vì $\mathrm{P}(A|C)=\dfrac{n\left(A C\right)}{n(C)}=\dfrac{190}{260}=\dfrac{19}{26}$. \\
	\itemch Đúng. Vì xác suất để một người khỏi bệnh khi được chọn điều trị bằng thuốc $X$ là $\dfrac{3}{4}$ và xác suất để một người khỏi bệnh khi được chọn điều trị bằng thuốc $Y$ là $\dfrac{19}{26}$. Do $\dfrac{3}{4}>\dfrac{19}{26}$ nên loại thuốc $X$ có hiệu quả hơn loại thuốc $Y$ trong việc điều trị bệnh $H$. 
	\end{itemchoice}
	}
\end{ex}
\Closesolutionfile{ans}
\indapan{3}{ans/ans-2-B18-DS}
\Opensolutionfile{ans}[ans/ans-2-B18-KQ]
%\TNSA
\begin{ex}%[2D5N1-2]
	Cho hai biến cố $A$, $B$ có xác suất $\mathrm{P}(A)=0{,}4$, $\mathrm{P}(B)=0{,}6$; $\mathrm{P}(AB)=0{,}2$. Xác suất $\mathrm{P}(\overline{A}\mid B)=\dfrac{a}{b}$ với $\dfrac{a}{b}$ là phân số tối giản. Tính $M=a^2+b^2$.
	\shortans{$13$}
	\loigiai{
	Theo định nghĩa và tính chất xác suất có điều kiện, ta có
	$$\mathrm{P}(\overline{A}\mid B)=1-\mathrm{P}(A \mid B)=1-\dfrac{\mathrm{P}(A B)}{\mathrm{P}(B)}=1-\dfrac{0{,}2}{0{,}6}=1-\dfrac{1}{3}=\dfrac{2}{3}.$$
	Từ đó ta có $a=2$, $b=3$. Vậy $M=a^2+b^2=2^2+3^2=13$.
	}
\end{ex}
\begin{ex}%[2D5H1-2]
	Một công ty bảo hiểm nhận thấy có $48 \%$ số người mua bảo hiểm ô tô là phụ nữ và có $36 \%$ số người mua bảo hiểm ô tô là phụ nữ trên $45$ tuổi. Biết một người mua bảo hiểm ô tô là phụ nữ, tính xác suất người đó trên $45$ tuổi.
	\shortans{$0{,}75$}
	\loigiai
	{
	Gọi $A$ là biến cố \lq\lq  Người mua bảo hiểm ô tô là phụ nữ\rq\rq, $B$ là biến cố \lq\lq  Người mua bảo hiểm ô tô trên 45 tuổi\rq\rq. Ta cần tính $\mathrm{P}(B \mid A)$.\\
	Do có $48 \%$ người mua bảo hiểm ô tô là phụ nữ nên $\mathrm{P}(A)=0{,}48$.\\
	Do có $36 \%$ số người mua bảo hiểm ô tô là phụ nữ trên $45$ tuổi nên $\mathrm{P}(A B)=0{,}36$.\\
	Vậy $\mathrm{P}(B \mid A)=\dfrac{\mathrm{P}(A B)}{\mathrm{P}(A)}=\dfrac{0{,}36}{0{,}48}=0{,}75$. 
	}
\end{ex}
\begin{ex}%[2D5V1-2]
	Một nhóm $5$ học sinh nam và $4$ học sinh nữ tham gia lao động trên sân trường. Cô giáo chọn ngẫu nhiên đồng thời $2$ bạn trong nhóm đi tưới cây. Tính xác suất để hai bạn được chọn có cùng giới tính, biết rằng có ít nhất $1$ bạn nam được chọn.	(Kết quả làm tròn đến hai chữ số thập phân)
	\shortans{$0{,}67$}
	\loigiai{
	Số phần tử của không gian mẫu là $n(\Omega)=\mathrm{C}^2_9=36$.\\
	Gọi A là biến cố \lq\lq  Hai bạn được chọn có cùng giới tính\rq\rq.\\
	B là biến cố \lq\lq  Có ít nhất một bạn nam được chọn\rq\rq.\\
	Ta có $n(B)=\mathrm{C}^2_5+\mathrm{C}^1_5=15$ suy ra $\mathrm{P}(B)=\dfrac{15}{36}$.\\
	Ta có $n(AB)=\mathrm{C}^2_5=10$ suy ra $\mathrm{P}(AB)=\dfrac{10}{36}$.\\
	Vậy $\mathrm{P}(A|B)=\dfrac{\mathrm{P}(AB)}{\mathrm{P}(B)}=\dfrac{10}{15}=\dfrac{2}{3}\approx 0{,}67$.
	} 
\end{ex}
\begin{ex}%[2D5V1-2]
	Kết quả khảo sát những bệnh nhân bị tai nạn xe máy về mối liên hệ giữa việc đội mũ bảo hiểm và khả năng bị chấn thương vùng đầu cho thấy:\\
	- Tỉ lệ bệnh nhân bị chấn thương vùng đầu khi gặp tai nạn là $80 \%$;\\
	- Tỉ lệ bệnh nhân đội mũ bảo hiểm đúng cách khi gặp tai nạn là $90 \%$;\\
	- Tỉ lệ bệnh nhân đội mũ bảo hiểm đúng cách bị chấn thương vùng đầu là $18 \%$.\\
	Hỏi theo kết quả điều tra trên, việc đội mũ bảo hiểm đúng cách sẽ làm giảm khả năng bị chấn thương vùng đầu bao nhiêu lần? (Kết quả làm tròn đến hàng phần mười).
	\shortans{$4{,}6$}
	\loigiai{
	Gọi A là biến cố: \lq\lq  Bệnh nhân bị chấn thương vùng đầu khi gặp tai nạn\rq\rq\, và B là biến cố: \lq\lq  Bệnh nhân đội mũ bảo hiểm đúng cách khi gặp tai nạn\rq\rq.\\
	Theo đề bài ta có $\mathrm{P}(A)=0{,}8, \mathrm{P}(B)=0{,}9, \mathrm{P}(B|A)=0{,}18$.\\
	Suy ra $\mathrm{P}(B\mid A)=\dfrac{\mathrm{P}(AB)}{\mathrm{P}(A)}\Rightarrow \mathrm{P}(AB)=\mathrm{P}(A)\cdot \mathrm{P}(B\mid A)=0{,}8\cdot 0{,}18=0{,}144$.\\
	Vì $A\bar{B}$ và $AB$ là hai biến cố xung khắc nên $A\overline{B}\cup AB=A$.\\
	Suy ra $\mathrm{P}(\overline{B}A)=\mathrm{P}(A)-\mathrm{P}(AB)=0{,}8-0{,}144=0{,}656$.\\
	Ta có $\mathrm{P}(\overline{B}\mid A)=\dfrac{\mathrm{P}(\overline{B}A)}{\mathrm{P}(A)}=\dfrac{0{,}656}{0{,}8}=0{,}82$.
	Khi đó $\dfrac{\mathrm{P}(\overline{B}\mid A)}{\mathrm{P}(B|A)}=\dfrac{0{,}82}{0{,18}}\approx 4{,}6$.\\
	Như vậy việc đội mũ bảo hiểm đúng cách sẽ làm giảm khả năng chấn thương vùng đầu xuống $4{,}6$ lần.
	}
\end{ex} 
\begin{ex}%[2D5H1-2]
	Một công ty đấu thầu $2$ dự án. Khả năng thắng thầu của các dự án $I$ và $II$ lần lượt là $0{,}4$ và $0{,}5$. Khả năng thắng thầu của hai dự án là $0{,}3$. Gọi $A$, $B$ lần lượt là biến cố thắng thầu dự án $I$ và dự án $II$. Biết công ty không thắng thầu dự án $I$, tìm xác suất công ty thắng thầu dự án $II$.(Kết quả làm tròn đến hai chữ số thập phân)
	\shortans{$0{,}33$}
	\loigiai{
	Gọi $D$ là biến cố công ty thắng dự $II$ khi công ty đã không thắng dự án $I$. Ta có
	$$\mathrm{P}(D)=\mathrm{P}(B\mid\overline{A})=\dfrac{\mathrm{P}(\overline{A}B)}{\mathrm{P}(\overline{A})}=\dfrac{\mathrm{P}(B)-\mathrm{P}(AB)}{1-\mathrm{P}(\overline{A})}=\dfrac{0{,}5-0{,}3}{1-0{,}4}=\dfrac{1}{3}\approx 0{,}33.$$
	}
\end{ex}
\begin{ex}%[2D5V1-3]
	Hộp thứ nhất có $4$ viên bi xanh và $6$ viên bi đỏ. Hộp thứ hai có $5$ viên bi xanh và $4$ viên bi đỏ. Các viên bi có cùng kích thước và khối lượng. Lấy ra ngẫu nhiên $1$ viên bi từ hộp thứ nhất chuyển sang hộp thứ hai. Sau đó lại lấy ra ngẫu nhiên $1$ viên bi từ hộp thứ hai.
	Sử dụng sơ đồ hình cây, tính xác suất của biến cố
	$B\colon$ \lq\lq  Hai viên bi lấy ra có cùng màu\rq\rq.
	\shortans{$0{,}54$}
	\loigiai{
	Gọi $X$ là biến cố: \lq\lq  Viên bi lấy ra từ hộp thứ nhất có màu xanh\rq\rq.\\
	$Y$ là biến cố: \lq\lq  Viên bi lấy ra từ hộp thứ hai có màu đỏ\rq\rq.\\
	Ta có
	$
	\mathrm{P}(Y|X)=0{,}4$; $\mathrm{P}(Y \mid \overline{X})=0{,}5 $; $\mathrm{P}(X)=0{,}4
	$.\\
	Do đó $\mathrm{P}(\overline{X})=1-\mathrm{P}(X)=0{,}6$; $\mathrm{P}(\overline{Y} |X)=1-\mathrm{P}(Y|X)=0{,}6$; \\
	$\mathrm{P}(\overline{Y} \mid \overline{X})=1-\mathrm{P}(Y \mid \overline{X})=0{,}5$.\\
	Ta có sơ đồ hình cây như sau
	\begin{center}
	\begin{tikzpicture}
	\def\gocm{20}
	\def\gocn{10}
	\def\r{4}
	\tikzset{s/.style={outer sep=0.5 mm,draw=magenta,rectangle,minimum width=2.75cm,rounded corners=1mm}}
	\path(0,0)node(O){}++(\gocm:\r)node[s](A1){X}++(\gocn:\r)node[s](A2){$Y$};
	\path(A1)++({-\gocn}:\r)node[s](a2){$\overline{Y}$};
	\path(O)++(-\gocm:\r)node[s](B1){$\overline{X}$}++(\gocn:\r)node[s](B2){$Y$};
	\path(B1)++({-\gocn}:\r)node[s](b2){$\overline{Y}$};
	\foreach \x/\y in {
	O/A1,A1/A2,
	O/B1,B1/B2,
	A1/a2,
	B1/b2}
	\draw[-stealth](\x.east)--(\y.west);
	\path(O)--(A1.west)node[pos=0.5,above,sloped]{$0{,}4$}(O)--(B1.west)node[pos=0.5,below,sloped]{$0{,}6$}(B1.east)--(B2.west)node[pos=0.5,above,sloped]{$0{,}5$}(A1.east)--(A2.west)node[pos=0.5,above,sloped]{$0{,}4$}
	(A1.east)--(a2.west)node[pos=0.5,below,sloped]{$0{,}6$}
	(B1.east)--(b2.west)node[pos=0.5,below,sloped]{$0{,}5$};
	\end{tikzpicture}
	\end{center}
	Khi đó $\mathrm{P}(B)=\mathrm{P}(X\overline{Y})+\mathrm{P}(\overline{X}Y)=0{,}4\cdot 0{,}6+0{,}6\cdot 0{,}5=0{,}54$.
	}
\end{ex}
\begin{ex}%[2D6H1-2] 
	Cho hai biến cố $A$, $B$ có $P(A)=0{,}51$; $P(B)=0{,}2$; $P(A | B)=0{,}8$. Tính $P(B | A)$, làm tròn kết quả đến hàng phần trăm.
	\shortans{$0{,}31$}
	\loigiai{
	Ta có
	\begin{eqnarray*}
	&&P(A | B)=0{,}8\Leftrightarrow \dfrac{P(A\cap B)}{P(B)}=0{,}8\\
	&\Leftrightarrow& P(A \cap B)=0{,}8\cdot P(B)=0{,}8\cdot 0{,}2=0{,}16\\
	&\Rightarrow& P(B|A)=\dfrac{P(A\cap B)}{P(A)}=\dfrac{0{,}16}{0{,}51} \approx 0{,}31.
	\end{eqnarray*}
	}
\end{ex}
\begin{ex}%[2D6H1-2] 
	Cho hai biến cố $A$, $B$ có $P(A)=0{,}8$; $P(B)=0{,}5$; $P(A \cap B)=0{,}2$. Tính xác suất biến cố $B$ không xảy ra với điều kiện biến cố $A$ xảy ra.
	\shortans{$0{,}75$}
	\loigiai{
	Xác suất biến cố $B$ không xảy ra với điều kiện biến cố $A$ xảy ra là
	$$P(\overline{B} |A)=\dfrac{P(A\overline{B})}{P(A)}=\dfrac{P(A)-P(A\cap B)}{P(A)}=\dfrac{3}{4}=0{,}75.$$
	}
\end{ex}
\begin{ex}%[2D6V1-4] 
	Gieo một con xúc xắc cân đối và đồng chất hai lần. Tính xác suất để tổng số chấm xuất hiện trong hai lần gieo nhỏ hơn $8$. Biết rằng con lần gieo thứ nhất xuất hiện mặt $4$ chấm.
	\shortans{$0{,}5$}
	\loigiai{
	Xét các biến cố:\\
	$A$: "Lần thứ nhất xuất hiện mặt 4 chấm."\\
	$B$: "Tổng số chấm trong hai lần gieo nhỏ hơn $8$."\\
	Khi đó $$A=\{(4;1), (4;2), (4;3), (4;4), (4;5), (4;6)\};$$
	$$A\cap B= \{(4;1), (4;2), (4;3)\}.$$
	Suy ra $n(A)=6$ và $n(A\cap B)=3$.\\
	Vậy xác suất để tổng số chấm xuất hiện trong hai lần gieo bằng 6, biết rằng con lần gieo thứ nhất xuất hiện mặt 4 chấm là
	$$P(B|A)=\dfrac{n(A\cap B)}{n(A)}=\dfrac{3}{6}=0{,}5.$$
	}
\end{ex}
\begin{ex}%[2D6V1-4] 
	Một công ty bảo hiểm nhận thấy có $56 \%$ số người mua bảo hiểm sức khỏe là phụ nữ và có $42 \%$ số người mua bảo hiểm sức khỏe là phụ nữ trên 50 tuổi. Tính tỉ lệ người trên $50$ tuổi trong số những người phụ nữ mua bảo hiểm sức khỏe.
	\shortans{ $0{,}75$}
	\loigiai{
	Xét các biến cố:\\
	A: "Người mua bảo hiểm sức khỏe là phụ nữ."\\
	B: "Người mua bảo hiểm sức khỏe là phụ nữ trên $50$ tuổi."\\
	Khi đó $P(A)=0{,}56$ và $P(A \cap B)=0{,}42$.\\
	Xác suất người mua bảo hiểmTỉ lệ người trên $45$ tuổi trong số những người phụ nữ mua bảo hiểm sức khỏe là
	$$P(B|A)=\dfrac{P(A\cap B)}{P(A)}=\dfrac{0{,}42}{0{,}56}=0{,}75.$$}
\end{ex}
\begin{ex}%[2D6C1-4] 
	Tại một khu phố có $100$ căn nhà, trong đó có $40$ căn nhà gắn biển số lẻ. Biết rằng có $25$ căn nhà gắn biển số lẻ và $15$ nhà gắn biển số chẵn có ô tô. Chọn ngẫu nhiên một nhà trong khu phố đó. Tính xác suất nhà được chọn gắn biển số lẻ, biết rằng nhà đó không có ô tô.
	\shortans{$0{,}25$}
	\loigiai{
	Xét các biến cố:\\
	$A$: "Nhà được chọn gắn số lẻ."\\
	$B$: "Nhà được chọn có ô tô."\\
	Khi đó xác suất nhà được chọn gắn biển số lẻ, biết rằng nhà đó không có ô tô, là $P(A | \overline{B})$.\\
	Số căn nhà gắn số lẻ và không có ô tô là $n(A\cap \overline{B})=40-25=15$.\\
	Số căn nhà không có ô tô là $n(\overline{B})=100-(25+15)=60$.\\
	Ta có
	$$P(A | \overline{B})=\dfrac{n(A\cap \overline{B})}{n(\overline{B})}=0{,}25.$$
	}
\end{ex}
\begin{ex}%[2D6C1-4] 
	Kết quả một cuộc khảo sát các vụ tai nạn giao thông ô tô về mối quan hệ giữa việc thắt dây an toàn của người lái xe khi xảy ra tai nạn giao thông và nguy cơ tử vong của người lái xe khi xảy ra tai nạn giao thông cho thấy:
	\begin{itemize}
	\item Tỉ lệ người lái xe tử vong khi xảy ra tai nạn giao thông là $0{,4}\%$.
	\item Tỉ lệ người lái xe không thắt dây an toàn giao thông khi xảy ra tai nạn giao thông là $28 \%$.
	\item Tỉ lệ người lái xe tử vong khi xảy ra tai nạn giao thông trong trường hợp không thắt dây an toàn là $0{,}3 \%$.
	\end{itemize}
	Hỏi theo kết quả khảo sát trên, việc thắt dây an toàn của người lái xe ô tô sẽ làm giảm khả năng tử vong là bao nhiêu lần? (làm tròn đến hàng phần mười).
	\shortans{$7{,}7$}
	\loigiai{
	Chọn ngẫu nhiên một một vụ tai nạn giao thông của cuộc khảo sát trên. Xét các biến cố:\\
	$A$: "Người lái xe đó tử vong khi xảy ra tai nạn giao thông."\\
	$B$: "Người lái xe đó không thắt dây an toàn khi xảy ra tai nạn giao thông."\\
	Ta có $P(A)= 0{,4}\%$; $P(B)= 28\%$; $P(A\cap B)=0{,}3\%$.\\
	Xác suất người lái xe đó tử vong khi xảy ra tai nạn giao thông trong trường hợp không thắt dây an toàn là
	$$P(A|B)=\dfrac{P(A\cap B)}{P(B)}=\dfrac{3}{280}.$$
	Xác suất người lái xe đó có thắt dây an toàn giao thông là $P(\overline{B})=72\%$.\\
	Xác suất người lái xe đó tử vong khi xảy ra tai nạn giao thông trong trường hợp có thắt dây an toàn là
	$$P(A|\overline{B})=\dfrac{P(A\cap \overline{B})}{P(\overline{B})}=\dfrac{P(A)-P(A\cap B)}{P(\overline{B}}=\dfrac{1}{720}.$$
	Ta có
	$$\dfrac{P(A|B)}{P(A|\overline{B})}=\dfrac{54}{7}\approx7{,}7.$$
	Vậy theo khảo sát trên, việc thắt dây an toàn của người lái xe ô tô sẽ làm giảm khả năng tử vong khoảng $7{,}7$ lần.
	}
\end{ex}
\Closesolutionfile{ans}
\indapan{6}{ans/ans-2-B18-KQ}
% \setcounter{section}{1}
\section{CÔNG THỨC XÁC SUẤT TOÀN PHẦN VÀ CÔNG THỨC BAYES}
%%%%%%%%%%%%%%%%
\subsection{Trọng tâm kiến thức}
\begin{tomtat}
\subsubsection{Công thức xác suất toàn phần}
\begin{boxdn}
Cho hai biến cố $A$ và $B$ là hai biến cố tùy ý. Khi đó
$$\mathrm{P}(A)=\mathrm{P}(B)\cdot \mathrm{P}(A|B)+\mathrm{P}(\overline{B})\cdot \mathrm{P}(A|\overline{B}).$$
Công thức trên được gọi là công thức xác suất toàn phần.
\end{boxdn}
\subsubsection{Công thức Bayes}
\begin{boxdn}
Cho hai biến cố $A$ và $B$ với $\mathrm{P}(A)>0$. Khi đó
$$\mathrm{P}\left(B|A\right)=\dfrac{\mathrm{P}(B)\cdot\mathrm{P}\left(A|B\right)}{\mathrm{P}(A)}.$$
\end{boxdn}
\begin{note}
Công thức Bayes còn được viết dưới dạng 
$$\mathrm{P}(B | A)=\dfrac{\mathrm{P}(B) \cdot \mathrm{P}(A \mid B)}{\mathrm{P}(B)\cdot \mathrm{P}(A | B)+\mathrm{P}\left(\overline{B}\right)\cdot \mathrm{P}\left(A | \overline{B}\right)}.$$
\end{note}
\end{tomtat}
%%%%%%%%%%%%%%
\subsection{Các dạng bài tập}
\setcounter{dang}{0}
\begin{dang}{Tính xác suất theo công thức xác suất toàn phần}
	Với hai biến cố $A$ và $B$ tùy ý thì
	$\mathrm{P}(A)=\mathrm{P}(B)\cdot\mathrm{P}(A|B)+\mathrm{P}(\overline{B})\cdot\mathrm{P}(A|\overline{B}).$
\end{dang}
%----------------------------
\subsubsection{Ví dụ minh hoạ}
\begin{vd}%[2D5H2-2]
	Cho hai biến cố $A$, $B$ với $\mathrm{P}(B)=0{,}6$; $\mathrm{P}(A|B) =0{,}7$ và $\mathrm{P}\left(A|\overline{B}\right)=0{,}4$. Tính $\mathrm{P}(A)$
	\loigiai{
		Ta có $\mathrm{P}\left(\overline{B}\right)= 1- \mathrm{P}(B) = 1-0{,}6 = 0{,}4$.\\
		Áp dụng công thức xác suất toàn phần, ta có
		\[\mathrm{P}(A) = \mathrm{P}(A|B)\cdot \mathrm{P}(B) + \mathrm{P}\left(A|\overline{B}\right)\cdot \mathrm{P}\left(\overline{B}\right)= 0{,}7\cdot 0{,}6 + 0{,}4\cdot 0{,}4=0{,}58.\]
	}
\end{vd}
\begin{vd}%[2D5H2-2]
	Trong một kì thi tốt nghiệp trung học phổ thông, một tỉnh X có $80 \%$ học sinh lựa chọn tổ hợp A00 (gồm các môn Toán, Vật lí, Hoá học). Biết rằng, nếu một học sinh chọn tổ hợp A00 thì xác suất để học sinh đó đỗ đại học là 0{,}6; còn nếu một học sinh không chọn tổ hợp A00 thì xác suất để học sinh đó đỗ đại học là 0{,}7. Chọn ngẫu nhiên một học sinh của tỉnh X đã tốt nghiệp trung học phổ thông trong kì thi trên. Tính xác suất để học sinh đó đỗ đại học.
	\loigiai{Gọi $A$ là biến cố: \lq \lq Học sinh đó chọn tổ hợp A00\rq \rq~; $B$ là biến cố: \lq \lq Học sinh đó đỗ đại học\rq \rq.\\
		Ta cần tính $\mathrm{P}(B)$. Theo công thức xác suất toàn phần, ta cần biết: $\mathrm{P}(A), \mathrm{P}(\overline{A}), \mathrm{P}(B \mid A)$ và $\mathrm{P}(B \mid \overline{A})$.\\
		Ta có: $\mathrm{P}(A)=0{,}8; \mathrm{P}(\overline{A})=1-\mathrm{P}(A)=1-0{,}8=0{,}2$.\\
		$\mathrm{P}(B \mid A)$ là xác suất để một học sinh đỗ đại học với điều kiện học sinh đó chọn tổ hợp $A 00$ \\$\Rightarrow \mathrm{P}(B \mid A)=0{,}6$.\\
		$\mathrm{P}(B \mid \overline{A})$ là xác suất để một học sinh đỗ đại học với điều kiện học sinh đó không chọn tổ hợp $\mathrm{A} 00$\\$ \Rightarrow \mathrm{P}(B \mid \overline{A})=0{,}7$.\\
		Thay vào công thức xác suất toàn phần ta được:$$\mathrm{P}( B)={\mathrm{P}(A) \cdot \mathrm{P}(B \mid A)+\mathrm{P}(\overline{A}) \cdot \mathrm{P}(B \mid \overline{A})}={0{,}8 \cdot 0{,}6+0{,}2 \cdot 0{,}7} = 0{,}62.$$}
\end{vd}
\begin{vd}%[2D5H2-2]
	\immini{
		Số khán giả đến xem buổi biểu diễn ca nhạc ngoài trời phụ thuộc vào thời tiết. Giả sử, nếu trời không mưa thì xác suất để bán hết vé là $0{,}9$; còn nếu trời mưa thì xác suất để bán hết vé chỉ là $0{,}4$. Dự báo thời tiết cho thấy xác suất để trời mưa vào buổi biểu diễn là $0{,}75$. Nhà tổ chức sự kiện quan tâm đến xác suất để bán được hết vé là bao nhiêu.\\
			Gọi $A$ là biến cố \lq \lq Trời mưa\rq \rq~và $B$ là biến cố \lq \lq Bán hết vé\rq \rq~trong tình huống.
		\begin{listEX}
			\item Tính $\mathrm{P}(A), \mathrm{P}(\overline{A}), \mathrm{P}(B \mid A), \mathrm{P}(B \mid \overline{A})$.
			\item Tính xác suất để nhà tổ chức sự kiện bán hết vé.
		\end{listEX}
	}{	\includegraphics[width=8cm,height=6cm]{images/im2D5-2-1.png}}
	\loigiai{
		\begin{listEX}
			\item $\mathrm{P}(A)=0{,}75$; $\mathrm{P}(\overline{A})=1-\mathrm{P}(A)=0{,}25$; $\mathrm{P}(B\mid A)=0{,}4$; $\mathrm{P}(B\mid \overline{A})=0{,}9$.
			\item 	Ta có $\mathrm{P}(B)= \mathrm{P}(A) \cdot \mathrm{P}(B \mid A)+\mathrm{P}(\overline{A}) \cdot \mathrm{P}(B \mid \overline{A}) = 0{,}75\cdot 0{,}4 +0{,}25\cdot 0{,}9=0{,}525$.
	\end{listEX}}
\end{vd}
\begin{vd}%[2D5H2-2]
	Một hộp có $60$ viên bi màu xanh và $40$ viên bi màu đỏ; các viên bi có kích thước và khối lượng như nhau. Sau khi thống kê, người ta thấy: có $50\%$ số viên bi màu xanh có dán nhãn và $75\%$ số viên bi màu đỏ có dán nhãn; những viên bi còn lại không có dán nhãn.
	\begin{listEX}
		\item Chọn số thích hợp cho $\boxed{?}$ trong bảng (đơn vị: viên bi).
		\begin{center}
			\begin{tabular}{|c|C{3cm}|C{3cm}|}
				\hline
				\diaghead{Dán nhãn Màu bi}{\normalsize Màu bi}{\normalsize Dán nhãn} & Có dán nhãn & Không dán nhãn \\
				\hline
				Đỏ & $\boxed{?}$ & $\boxed{?}$ \\
				\hline
				Xanh & $\boxed{?}$ & $\boxed{?}$\\
				\hline
			\end{tabular}
		\end{center}
		\item Lấy ra ngẫu nhiên một viên bi trong hộp. Sử dụng công thức xác suất toàn phần, tính xác suất để viên bi được lấy ra có dán nhãn.
	\end{listEX}
	\loigiai{
		\begin{listEX}
			\item Số viên bi màu đỏ có dán nhãn là $75\%\cdot 40 = 30$ (viên bi).\\
			Số viên bi màu xanh có dãn nhẫn là $50\%\cdot 60 = 30$ (viên bi).\\
			\begin{center}
				\begin{tabular}{|c|C{2.75cm}|C{2.75cm}|}
					\hline
					\diaghead{Dán nhãn Màu bi}{\normalsize Dán nhãn}{\normalsize Màu bi} & Có dán nhãn & Không dán nhãn \\
					\hline
					Đỏ & $30$ & $10$ \\
					\hline
					Xanh & $30$ & $30$\\
					\hline
				\end{tabular}
			\end{center}
			Sau khi hoàn thiện bảng $3$ ta được bảng $4$ (đơn vị: viên bi).
			\item Xét hai biến cố sau
			\begin{itemize}
				\item $A$: \lq\lq  Viên bi được chọn ra có dãn nhãn\rq\rq.
				\item $B$: \lq\lq  Viên bi được chọn ra có màu đỏ\rq\rq.
			\end{itemize}
			Khi đó, ta có
		$$\mathrm{P}(B)=\dfrac{40}{100} = \dfrac{2}{5};
		\quad \mathrm{P}\left(\overline{B}\right)= 1 - \mathrm{P}(B)=1-\dfrac{2}{5}=\dfrac{3}{5};
		\quad \mathrm{P}(A|B) = \dfrac{30}{40}=\dfrac{3}{4};
		\quad \mathrm{P}\left(A|\overline{B}\right)=\dfrac{30}{60}=\dfrac{1}{2}.$$
			Áp dụng công thức tính xác suất toàn phần, ta có
			\[\mathrm{P}(A) = \mathrm{P}(B)\cdot\mathrm{P}\left(A|B\right) + \mathrm{P}\left(\overline{B}\right)\cdot\mathrm{P}\left(A|\overline{B}\right) = \dfrac{2}{3}\cdot \dfrac{3}{4} + \dfrac{3}{5}\cdot \dfrac{1}{2}=\dfrac{4}{5}.\]
			Vậy xác suất để viên bi được lấy ra có dãn nhãn bằng $\dfrac{4}{5}$.
		\end{listEX}
	}
\end{vd}
%Ví dụ 3
\begin{vd}%[2D5H2-2]
	Trong trò chơi hái hoa có thưởng của lớp 12A, cô giáo treo $10$ bông hoa trên cành cây, trong đó có $5$ bông hoa chưa phiếu có thưởng. Bạn Bình hái bông hoa đầu tiên, sau đó bạn An hái bông hoa thứ hai.
	\begin{listEX}
		\item Vẽ sơ đồ cây biểu thị tình huống trên.
		\item Từ đó, tính xác suất bạn An hái được bông hoa chứa phiếu có thưởng.
	\end{listEX}
	\loigiai{
		Xét hai biến cố
		\begin{itemize}
			\item $A$: \lq\lq  Bông hoa bạn An hái được chứa phiếu có thưởng\rq\rq.
			\item $B$: \lq\lq  Bông hoa bạn Bình hái được chứa phiếu có thưởng\rq\rq.
		\end{itemize}
		Khi đó, ta có
		$$\mathrm{P}(B)=\dfrac{5}{10}=\dfrac{1}{2}; 
		\quad \mathrm{P}\left(\overline{B}\right) = 1 - \mathrm{P}(B)=1-\dfrac{1}{2}=\dfrac{1}{2}; 
		\quad \mathrm{P}(A|B)=\dfrac{4}{9};
		\quad \mathrm{P}\left(A|\overline{B}\right)=\dfrac{5}{9}.$$
		\begin{listEX}
			\item Sơ đồ hình cây biểu thị tình huống đã cho là
			\begin{center}
				\begin{tikzpicture}[->,>=stealth,line join=round,line cap=round,font=\footnotesize,scale=1]
					\def\xmot{3}
					\def\xhai{8}
					\node (O) at (0,0){};
					\node (B) at (\xmot,1){$B$};
					\node (B1) at (\xmot,-1){$\overline{B}$};
					\node (BA) at (\xhai,2){$A$};
					\node (BA1) at (\xhai,0.3){$\overline{A}$};
					\node (B1A) at (\xhai,-0.3){$A$};
					\node (B1A1) at (\xhai,-1.75){$\overline{A}$};
					\foreach \x/\y/\p/\l in
					{
						O/B/above/$\mathrm{P}(B)=\dfrac{1}{2}$,
						B/BA/above/$\mathrm{P}(A|B)=\dfrac{4}{9}$,
						B/BA1//,
						O/B1/below/$\mathrm{P}\left(\overline{B}\right)=\dfrac{1}{2}$,
						B1/B1A/above/$\mathrm{P}\left(A|\overline{B}\right)=\dfrac{5}{9}$,
						B1/B1A1//
					}
					{
						\draw[->] (\x)--(\y)node[midway,\p,scale=0.8,sloped]{\l};
					}
					\node (mot) at (\xmot,4) {Bông hoa bạn Bình hái ra};
					\node (hai) at (\xhai,4) {Bông hoa bạn An hái ra};
					\draw[->,thick,orange] (mot)--(B);
					\draw[->,thick,orange] (hai)--(BA);
				\end{tikzpicture}
			\end{center}
			\item Áp dụng công thức tính xác suất toàn phần, ta có:
			\[\mathrm{P}(A) = \mathrm{P}(B)\cdot\mathrm{P}\left(A|B\right) + \mathrm{P}\left(\overline{B}\right)\cdot\mathrm{P}\left(A|\overline{B}\right) = \dfrac{1}{2}\cdot \dfrac{4}{9} + \dfrac{1}{2}\cdot\dfrac{5}{9}=\dfrac{1}{2}.\]
			Vậy xác suất bạn An hái được bông hoa chứa phiếu có thưởng bằng $\dfrac{1}{2}$.
		\end{listEX}
	}
\end{vd}
\begin{vd}%[2D5H2-2]
	Một hộp có $5$ quả cầu trắng và $10$ quả cầu đen cùng kích thước và khối lượng. Lấy ngẫu nhiên lần lượt hai quả cầu (không hoàn lại) từ hộp. Xác suất để lần thứ hai lấy được quả cầu trắng là
	\loigiai{
		Xét phép thử lấy ngẫu nhiên lần lượt hai quả cầu (không hoàn lại) từ hộp. Gọi:
		\begin{itemize}
			\item $A$ là biến cố \lq \lq Lần thứ hai lấy được quả cầu trắng\rq \rq~;
			\item $B$ là biến cố \lq \lq Lần thứ nhất lấy được quả cầu trắng\rq \rq~;
			\item$\overline{B}$ là biến cố \lq \lq Lần thứ nhất lấy được quả cầu đen\rq \rq~.
		\end{itemize}
		Ta có:
		$$\mathrm{P}(B)=\dfrac{5}{15}=\dfrac{1}{3};\quad \mathrm{P}(\overline{B})=\dfrac{10}{15}=\dfrac{2}{3}.$$
		Nếu lần thứ nhất lấy được quả cầu trắng thì trong hộp còn $4$ quả cầu trắng và $10$ quả cầu đen. Do đó $\mathrm{P}(A|B)=\dfrac{4}{14}=\dfrac{2}{7}$.\\
		Nếu lần thứ nhất lấy được quả cầu đen thì trong hộp còn $5$ quả cầu trắng và $9$ quả cầu đen. Do đó $\mathrm{P}(A|\overline{B})=\dfrac{5}{14}$.\\
		Áp dụng công thức xác suất toàn phần, ta có:
		$$\mathrm{P}(A) = \mathrm{P}(B)\cdot \mathrm{P}(A|B) + \mathrm{P}(\overline{B})\cdot \mathrm{P}(A|\overline{B}) =\dfrac{1}{3}\cdot \dfrac{2}{7}+\dfrac{2}{3}\cdot \dfrac{5}{14}=\dfrac{1}{3}.$$
		Vậy xác suất để lần thứ hai lấy được quả cầu trắng bằng $\dfrac{1}{3}$.
	}
\end{vd}
%----------------------------
\subsubsection{Bài tập áp dụng}
\begin{bt}
	Cho hai biến cố $A$, $B$ sao cho $\mathrm{P}(A) = 0{,}6$; $\mathrm{P}(B)=0{,}4$; $\mathrm{P}(B\mid\overline{A}) = 0{,}3$. Tính~$\mathrm{P}(B|A)$.
	\loigiai{
		Áp dụng công thức xác suất toàn phần ta có	$$\mathrm{P}(B)=\mathrm{P}(A)\cdot \mathrm{P}(B \mid A)+\mathrm{P}(\overline{A}) \cdot \mathrm{P}(B \mid \overline{A})
		\Leftrightarrow 0{,}4=0{,}6\cdot\mathrm{P}(B \mid A) +0{,}4 \cdot 0{,}3.$$
		Suy ra $\mathrm{P}(B|A)=\dfrac{7}{15} . $}
\end{bt}
\begin{bt}%[2D5H2-2]
	Một loại linh kiện do hai nhà máy số I, số II cùng sản xuất. Tỉ lệ phế phẩm của các nhà máy I, II lần lượt là $4\%$; $3\%$. Trong một lô linh kiện để lẫn lộn $80$ sản phẩm của nhà máy số I và $120$ sản phẩm của nhà máy số II. Một khách hàng lấy ngẫu nhiên một linh liện từ lô hàng đó.
	Tính xác suất để linh kiện được lấy ra là linh kiện tốt.
	\loigiai{
		Xét các biến cố
		\begin{itemize}
			\item $A$: \lq\lq  Linh kiện lấy ra là linh kiện tốt\rq\rq.
			\item $B$: \lq\lq  Linh kiện lấy ra là linh kiện từ nhà máy số I\rq\rq.
			\item $\overline{B}$: \lq\lq  Linh kiện lấy ra là linh kiện từ nhà máy số II\rq\rq.
		\end{itemize}
		Theo đề bài, ta có
		$$\mathrm{P}(A|B) =1 - 0{,}04 = 0{,}96;
		\quad\quad \mathrm{P}\left(A|\overline{B}\right) = 1-0{,}03 = 0{,}97;
		\quad\quad \mathrm{P}(B)=\dfrac{80}{200}=0{,}4;
		\quad\quad \mathrm{P}\left(\overline{B}\right) = \dfrac{120}{200}=0{,}6.
		$$
		Khi đó áp dụng công thức xác suất toàn phần, ta có
		\[\mathrm{P}(A) = \mathrm{P}(A|B)\cdot \mathrm{P}(B) + \mathrm{P}\left(A|\overline{B}\right)\cdot\mathrm{P}\left(\overline{B}\right)=0{,}96\cdot 0{,}4 + 0{,}97\cdot 0{,}6=0{,}966.\]
	}
\end{bt}
\begin{bt}%[2D5H2-2]
	Người ta khảo sát khả năng chơi nhạc cụ của một nhóm học sinh tại trường X. Nhóm này có $60\%$ học sinh là nam. Kết quả khảo sát cho thấy có $20\%$ học sinh nam và $15\%$ học sinh nữ biết chơi ít nhất một nhạc cụ. Chọn ngẫu nhiên một học sinh trong nhóm này. Tính xác suất để chọn được học sinh biết chơi ít nhất một nhạc cụ.
	\loigiai{
		Xét phép thử chọn ngẫu nhiên một học sinh trong nhóm.\\
		Gọi $A$ là biến cố \lq \lq Chọn được một học sinh biết chơi ít nhất một nhạc cụ\rq \rq~ và $B$, $\overline{B}$ lần lượt là các biến cố \lq \lq Chọn được một học sinh nam\rq \rq~ và \lq \lq Chọn được một học sinh nữ\rq \rq~.\\
		Theo đề bài:
		$$\mathrm{P}(B) = 60\% = 0{,}6;\quad \mathrm{P}(\overline{B}) = 1 - 0{,}6 = 0{,}4;$$
		$$\mathrm{P}(A|B) = 20\% = 0{,}2;\quad \mathrm{P}(A|\overline{B}) = 15\% = 0{,}15.$$
		Áp dụng công thức xác suất toàn phần, ta có:
		$$\mathrm{P}(A) = \mathrm{P}(B)\cdot \mathrm{P}(A|B) + \mathrm{P}(\overline{B})\cdot \mathrm{P}(A|\overline{B}) = 0{,}6\cdot 0{,}2 + 0{,}4\cdot 0{,}15 = 0{,}18.$$
		Vậy xác suất để chọn được một học sinh biết chơi nhạc cụ là $0{,}18$.
	}
\end{bt}
\begin{bt}%[2D5V2-2]
	Một doanh nghiệp có $45 \%$ nhân viên là nữ. Tỉ lệ nhân viên nữ và tỉ lệ nhân viên nam mua bảo hiểm nhân thọ lần lượt là $7 \%$ và $5 \%$. Gặp ngẫu nhiên một nhân viên của doanh nghiệp. Tính xác suất nhân viên đó có mua bảo hiểm nhân thọ.
	\loigiai{
	\begin{itemize}
		\item \textbf{Cách 1:} Giả sử doanh nghiệp có $100$ nhân viên, trong đó có $45$ nhân viên là nữ và $55$ nhân viên là nam. Tỉ lệ nhân viên nữ mua bảo hiểm nhân thọ là $7 \%$, tức là có $3{,}15$ nhân viên nữ mua bảo hiểm nhân thọ. Tỉ lệ nhân viên nam mua bảo hiểm nhân thọ là $5 \%$, tức là có $2{,}75$ nhân viên nam mua bảo hiểm nhân thọ. Tổng số nhân viên mua bảo hiểm nhân thọ là $3{,}15+2{,}75=5{,}9$.\\
		Xác suất để ngẫu nhiên chọn được một nhân viên mua bảo hiểm nhân thọ trong doanh nghiệp là $\dfrac{5{,}9}{100}=0{,}059$.
		\item \textbf{Cách 2:} Xét các biến cố
		\begin{itemize}
			\item $A$: \lq\lq  Nhân viên có mua bảo hiểm nhân thọ\rq\rq.
			\item $B$: \lq\lq  Nhân viên là nữ \rq\rq.
		\end{itemize}
		Do doanh nghiệp có $45\%$ nhân viên là nữ cho nên \[\mathrm{P}(B)=0{,}45 ~\text{và}~ \mathrm{P}(\overline{B})=0{,}55.\]
		Mặt khác tỉ lệ nhân viên nữ và tỉ lệ nhân viên nam mua bảo hiểm nhân thọ lần lượt là $7 \%$ và $5 \%$ cho nên ta có \[\mathrm{P}(A\mid B)=0{,}07~ \text{và}~ \mathrm{P}(A\mid \overline{B})=0{,}05.\]
		Xác suất nhân viên có mua bảo hiểm nhân thọ là
		\[ \mathrm{P}(A)=\mathrm{P}(B)\mathrm{P}(A\mid B)+\mathrm{P}(\overline{B})\mathrm{P}(A\mid \overline{B})=0{,}45\cdot 0{,}07
		+ 0{,}55\cdot 0{,}05 = 0{,}059.\]
	\end{itemize}
	}
\end{bt}
\begin{bt}%[2D5H2-2]%[2D5V2-2]%[2D5V2-3]
	Có hai đội thi đấu môn Bắn súng. Đội I có 5 vận động viên, đội II có 7 vận động viên. Xác suất đạt huy chương vàng của mỗi vận động viên đội I và đội II tương ứng là 0{,}65 và 0{,}55. Chọn ngẫu nhiên một vận động viên.
	Tính xác suất để vận động viên này đạt huy chương vàng;
	\loigiai{
		Xét biến cố $A$: \lq \lq Vận động viên này thuộc đội I\rq \rq~. Xét biến cố $B$: \lq \lq Vận động viên này đạt huy chương vàng\rq \rq~.
		Ta có $\mathrm{P}(B)=\mathrm{P}(A)\cdot \mathrm{P}(B \mid A)+\mathrm{P}(\overline{A}) \cdot \mathrm{P}(B \mid \overline{A})$.
		\begin{itemize}
			\item Tính $\mathrm{P}(A)$: Đây là xác suất để vận động viên đó thuộc đội I. Vậy $\mathrm{P}(A)=\dfrac{5}{12}$.
			\item Tính $\mathrm{P}(\overline{A})$: $\mathrm{P}(\overline{A})=1-\mathrm{P}(A)=\dfrac{7}{12}$.
			\item Tính $\mathrm{P}(B\mid A)$: Đây là xác suất để vận động viên thuộc đội I đạt huy chương vàng.\\ Vậy $\mathrm{P}(B\mid A)=0{,}65$.
			\item Tính $\mathrm{P}(B\mid \overline{A})$: Đây là xác suất để vận động viên thuộc đội I đạt huy chương vàng. \\ Vậy $\mathrm{P}(B\mid \overline{A})=0{,}55$.
		\end{itemize}
		Vậy $\mathrm{P}(B)=\mathrm{P}(A)\cdot \mathrm{P}(B \mid A)+\mathrm{P}(\overline{A}) \cdot \mathrm{P}(B \mid \overline{A})=\dfrac{5}{12}\cdot 0{,}65+\dfrac{7}{12}\cdot 0{,}55=\dfrac{71}{120}\approx 0{,}59$. \\
		Vậy xác suất để vận động viên này đạt huy chương vàng là khoảng $0{,}59$.
	}
\end{bt}
\begin{bt}%[2D5H2-2]
	Chuồng I có 5 con gà mái, 2 con gà trống. Chuồng II có 3 con gà mái, 5 con gà trống. Bác Mai bắt một con gà trong số đó theo cách sau: Bác tung một con xúc xắc cân đối, đồng chất. Nếu số chấm chia hết cho 3 thì bác chọn chuồng I, nếu số chấm không chia hết cho 3 thì bác chọn chuồng II. Sau đó, từ chuồng đã chọn bác bắt ngẫu nhiên một con gà. Tính xác suất để bác Mai bắt được con gà mái.
	\loigiai{
		Gọi $A$ là biến cố: \lq\lq  Bác Mai bắt được con gà mái\rq\rq.\\
		Gọi $B$ là biến cố: \lq\lq  tung con xúc xắc được số chấm chia hết cho $3$\rq\rq\, suy ra $\mathrm{P}(B)=\dfrac{2}{6}=\dfrac{1}{3}$.\\
		Gọi $\overline{B}$ là biến cố: \lq\lq  tung con xúc xắc được số chấm không chia hết cho $3$\rq\rq\, suy ra $\mathrm{P}(\overline{B})=\dfrac{4}{6}=\dfrac{2}{3}$.\\
		Ta có xác suất bắt được gà mái từ chuồng I là $\mathrm{P}\left(A\mid B\right)=\dfrac{5}{7}$.\\
		Ta có xác suất bắt được gà mái từ chuồng II là $\mathrm{P}\left(A\mid \overline{B}\right)=\dfrac{3}{8}$.\\
		Áp dụng công thức xác suất toàn phần ta có
		$$\mathrm{P}(A)=\mathrm{P}(B)\cdot \mathrm{P}(A\mid B)+\mathrm{P}(\overline{B})\cdot \mathrm{P}(A\mid \overline{B})=\dfrac{1}{3}\cdot \dfrac{5}{7}+\dfrac{2}{3}\cdot \dfrac{3}{8}=\dfrac{41}{84}.$$
		Vậy xác suất để bác Mai bắt được con gà mái là $\dfrac{41}{84}$.
	}
\end{bt}
\begin{bt}%[2D5H2-2]%[2D5H2-2]
	Tại nhà máy X sản xuất linh kiện điện tử tỉ lệ sản phẩm đạt tiêu chuẩn là $80 \%$. Trước khi xuất xưởng ra thị trường, các linh kiện điện tử đều phải qua khâu kiểm tra chất lượng để đóng dấu OTK. Vì sự kiểm tra không tuyệt đối hoàn hảo nên nếu một linh kiện điện tử đạt tiêu chuẩn thì nó có xác suất 0{,}99 được đóng dấu OTK; nếu một linh kiện điện tử không đạt tiêu chuẩn thì nó có xác suất 0{,}95 không được đóng dấu OTK. Chọn ngẫu nhiên một linh kiện điện tử của nhà máy X trên thị trường.
	\begin{listEX}
		\item Tính xác suất để linh kiện điện tử đó được đóng dấu OTK.
		\item Dùng sơ đồ hình cây, hãy mô tả cách tính xác suất để linh kiện điện tử được chọn không được đóng dấu OTK.
	\end{listEX}
	\loigiai{Gọi $A$ là biến cố: \lq \lq Linh kiện điện tử đó đạt tiêu chuẩn\rq \rq~. Gọi $B$ là biến cố: \lq \lq Linh kiện điện tử đó được đóng dấu OTK\rq \rq~.
		\begin{listEX}
			\item Ta có $\mathrm{P}(B)=\mathrm{P}(A)\cdot \mathrm{P}(B \mid A)+\mathrm{P}(\overline{A}) \cdot \mathrm{P}(B \mid \overline{A})$.
			\begin{itemize}
				\item Tính $\mathrm{P}(A)$: Đây là xác suất để linh kiện đó đạt tiêu chuẩn. Vậy $\mathrm{P}(A)=0{,}8$.
				\item Tính $\mathrm{P}(\overline{A})$: $\mathrm{P}(\overline{A})=1-\mathrm{P}(A)=0{,}2$.
				\item Tính $\mathrm{P}(B\mid A)$: Đây là xác suất để linh kiện điện tử đó được đóng dấu OTK với điều kiện nó đạt tiêu chuẩn. Vậy $\mathrm{P}(B\mid A)=0{,}99$.
				\item Tính $\mathrm{P}(B\mid \overline{A})$: Đây là xác suất để linh kiện điện tử đó được đóng dấu OTK với điều kiện nó không đạt tiêu chuẩn. Vậy $\mathrm{P}(B\mid \overline{A})=1-0{,}95=0{,}05$.
			\end{itemize}
			Vậy $\mathrm{P}(B)=\mathrm{P}(A)\cdot \mathrm{P}(B \mid A)+\mathrm{P}(\overline{A}) \cdot \mathrm{P}(B \mid \overline{A})=0{,}8\cdot 0{,}99+0{,}2\cdot 0{,}05=0{,}802$. \\
			Vậy xác suất để linh kiện điện tử đó được đóng dấu OTK là $0{,}802$.
			\item Ta có sơ đồ hình cây
			\begin{center}
				\begin{tikzpicture}[yscale=0.65, font=\footnotesize, line join=round, line cap=round, >=stealth]
					\draw (0,0)--(-2,-2.5) node [midway, shift=(140:4mm)] {$0{,}8$} --(-3,-5) node [midway, shift=(160:4mm)] {$0{,}99$};
					\draw (0,0)--(2,-2.5) node [midway, shift=(40:4mm)] {$0{,}2$} --(3,-5) node [midway, shift=(20:4mm)] {$0{,}95$};
					\draw (-2,-2.5)--(-1,-5) node [midway, shift=(20:4mm)] {$0{,}01$};
					\draw (2,-2.5)--(1,-5) node [midway, shift=(160:4mm)] {$0{,}05$};
					\draw[draw=none,fill=yellow] (0,0) circle [radius=7pt] node[shift=(90:6mm)] {Gốc $O$};
					\draw[draw=none,fill=cyan] (-2,-2.5) circle [radius=7pt] node[shift=(140:6mm)] {$A$};
					\draw[draw=none,fill=green] (2,-2.5) circle [radius=7pt] node[shift=(40:6mm)] {$\overline{A}$};
					\draw[draw=none,fill=magenta] (1,-5) circle [radius=7pt] node[shift=(-90:6mm)] {$B$};
					\draw[draw=none,fill=magenta!40!black!30] (-1,-5) circle [radius=7pt] node[shift=(-90:6mm)] {$\overline{B}$};
					\draw[draw=none,fill=magenta] (-3,-5) circle [radius=7pt] node[shift=(-90:6mm)] {$B$};
					\draw[draw=none,fill=magenta!40!black!30] (3,-5) circle [radius=7pt] node[shift=(-90:6mm)] {$\overline{B}$};
				\end{tikzpicture}
			\end{center}
			Có hai nhánh cây đi từ $O$ tới $\overline{B}$ là $OA\overline{B}$ và $O\overline{A}\cdot\overline{B}$. Vậy xác suất để linh kiện điện tử được chọn không được đóng dấu OTK là $$\mathrm{P}(\overline{B})=0{,}8\cdot 0{,}01+0{,}2\cdot 0{,}95=0{,}198.$$
		\end{listEX}
	}
\end{bt}
\begin{bt}%[2D5H2-2]
	Trong một cuộc khảo sát tình trạng công việc trên $900$ người đã có bằng tốt nghiệp trung học phổ thông ở một địa phương cho cả nam lẫn nữ, người ta thu được số liệu thống kê trong bảng sau.
	\begin{center}
		\begin{tabular}{|l|c|c|}
			\hline
			\diagbox{Giới tính}{Tình trạng} & Có việc làm & Thất nghiệp\\\hline
			Nam & $460$ & $40$\\\hline
			Nữ & $140$ & $260$\\\hline
		\end{tabular}
	\end{center}
	Chọn ngẫu nhiên một người trong nhóm này. Gọi $A$ là biến cố \lq \lq Người được chọn là nữ\rq \rq , $B$ là biến cố \lq \lq Người được chọn có việc làm\rq \rq .
	\begin{listEX}
		\item Vẽ lại sơ đồ hình cây sau đây và hoàn thành kết quả ở các ô \fbox{?}.
		\begin{center}
			\begin{tikzpicture}[line join = round, line cap = round, >=stealth, font=\footnotesize, yscale=0.7]
				\begin{scope}[every node/.style={draw, rounded corners=5pt}]
					\node (A) at (0,0){Chọn một người};
					\def \gocA{30}
					\def \kcA{4}
					\node (B1) at ($(A)+(\gocA:\kcA)$){$A$};
					\node (B2) at ($(A)+(-\gocA:\kcA)$){$\overline{A}$};
					\def \gocB{15}
					\def \kcB{4}
					\node (B11) at ($(B1)+(\gocB:\kcB)$){$B$};
					\node (B12) at ($(B1)+(-\gocB:\kcB)$){$\overline{B}$};
					\node (B21) at ($(B2)+(\gocB:\kcB)$){$B$};
					\node (B22) at ($(B2)+(-\gocB:\kcB)$){$\overline{B}$};
				\end{scope}
				\begin{scope}[every node/.style={midway,sloped},every path/.style={->}]
					\draw (A)--(B1) node[above]{$\mathrm{P}(A)=$\fbox{?}};
					\draw (A)--(B2) node[below]{$\mathrm{P}(\overline{A})=$\fbox{?}};
					\draw (B1)--(B11) node[above]{$\mathrm{P}(B|A)=$\fbox{?}};
					\draw (B1)--(B12) node[below]{$\mathrm{P}(\overline{B}|A)=$\fbox{?}};
					\draw (B2)--(B21) node[above]{$\mathrm{P}(B|\overline{A})=$\fbox{?}};
					\draw (B2)--(B22) node[below]{$\mathrm{P}(\overline{B}|\overline{A})=$\fbox{?}};
				\end{scope}
				\def \kcC{1.7}
				\foreach \i/\j in {B11/AB,B12/{A\overline{B}},B21/{\overline{A}B},B22/{\overline{A}\,\overline{B}}}% Tạo nội dung lặp
				{
					\node at ($(\i)+(\kcC,0)$)[]{$\j$};
					\node at ($(\i)+({2*\kcC},0)$)[]{\fbox{?}};
				}
				\node (B) at ($(B11)+(\kcC,0.7)$){\textbf{Kết quả}};
				\node (C) at ($(B)+(\kcC,0)$){\textbf{Xác suất}};
			\end{tikzpicture}\\
			$A \colon$ nữ; $\overline{A} \colon$ nam; $B \colon$ có việc; $\overline{B} \colon$ thất nghiệp.
		\end{center}
		\item Tính xác suất để chọn được một người có việc làm.
	\end{listEX}
	\loigiai{
		\begin{listEX}
			\item Theo đề bài xác suất để chọn được một người nữ là $\mathrm{P}(A)=\dfrac{4}{9}$, suy ra $\mathrm{P}(\overline{A})=\dfrac{5}{9}$.\\
			Xác suất chọn được người có việc làm nếu người đó là nữ $\mathrm{P}(B|A)=\dfrac{140}{400}=\dfrac{7}{20}$. Suy ra $\mathrm{P}(\overline{B}|A)=\dfrac{13}{20}$.\\
			Xác suất chọn được người có việc làm nếu người đó không là nữ $\mathrm{P}(B|\overline{A})=\dfrac{460}{500}=\dfrac{23}{25}$.\\
			Suy ra $\mathrm{P}(\overline{B}|\overline{A})=\dfrac{2}{25}.$\\
			\begin{center}
				\begin{tikzpicture}[line join = round, line cap = round, >=stealth, font=\footnotesize, scale=1]
					\begin{scope}[every node/.style={draw, rounded corners=5pt}]
						\node (A) at (0,0){Chọn một người};
						\def \gocA{30}
						\def \kcA{4}
						\node (B1) at ($(A)+(\gocA:\kcA)$){$A$};
						\node (B2) at ($(A)+(-\gocA:\kcA)$){$\overline{A}$};
						\def \gocB{15}
						\def \kcB{4}
						\node (B11) at ($(B1)+(\gocB:\kcB)$){$B$};
						\node (B12) at ($(B1)+(-\gocB:\kcB)$){$\overline{B}$};
						\node (B21) at ($(B2)+(\gocB:\kcB)$){$B$};
						\node (B22) at ($(B2)+(-\gocB:\kcB)$){$\overline{B}$};
					\end{scope}
					\begin{scope}[every node/.style={midway,sloped},every path/.style={->}]
						\draw (A)--(B1) node[above]{$\mathrm{P}(A)=$\fbox{$\dfrac{4}{9}$}};
						\draw (A)--(B2) node[below]{$\mathrm{P}(\overline{A})=$\fbox{$\dfrac{5}{9}$}};
						\draw (B1)--(B11) node[above]{$\mathrm{P}(B|A)=$\fbox{$\dfrac{7}{20}$}};
						\draw (B1)--(B12) node[below]{$\mathrm{P}(\overline{B}|A)=$\fbox{$\dfrac{13}{20}$}};
						\draw (B2)--(B21) node[above]{$\mathrm{P}(B|\overline{A})=$\fbox{$\dfrac{23}{25}$}};
						\draw (B2)--(B22) node[below]{$\mathrm{P}(\overline{B}|\overline{A})=$\fbox{$\dfrac{2}{25}$}};
					\end{scope}
					\def \kcC{1.7}
					\foreach \i/\j/\k in {B11/AB/{\dfrac{7}{45}},B12/{A\overline{B}}/{\dfrac{13}{45}},B21/{\overline{A}B}/{\dfrac{23}{45}},B22/{\overline{A}\,\overline{B}}/{\dfrac{2}{45}}}% Tạo nội dung lặp
					{
						\node at ($(\i)+(\kcC,0)$)[]{$\j$};
						\node at ($(\i)+({2*\kcC},0)$)[]{$\k$};
					}
					\node (B) at ($(B11)+(\kcC,0.7)$){\textbf{Kết quả}};
					\node (C) at ($(B)+(\kcC,0)$){\textbf{Xác suất}};
				\end{tikzpicture}
			\end{center}
			\item Xác suất để chọn được một người có việc làm $$\mathrm{P}(B)=\mathrm{P}(A)\mathrm{P}(B|A)+\mathrm{P}(\overline{A})\mathrm{P}(B|\overline{A})=\dfrac{4}{9}\cdot \dfrac{7}{20}+\dfrac{5}{9}\cdot \dfrac{23}{25}=\dfrac{2}{3}.$$
		\end{listEX}
	}
\end{bt}
\begin{bt}%[2D5V2-2]%[2D5V2-4]
	Hộp thứ nhất có $1$ viên bi xanh và $5$ viên bi đỏ. Hộp thứ hai có $3$ viên bi xanh và $5$ viên bi đỏ. Các viên bi có cùng kích thước và khối lượng. Lấy ra ngẫu nhiên đồng thời $2$ viên bi từ hộp thứ nhất chuyển sang hộp thứ hai. Sau đó lại lấy ra ngẫu nhiên $2$ viên bi từ hộp thứ hai.
	Tính xác suất để hai viên bi lấy ra từ hộp thứ hai là bi đỏ.
	\loigiai{
		Xét các biến cố
		\begin{itemize}
			\item $A$: \lq\lq  Hai viên bi lấy ra từ hộp thứ hai là bi đỏ\rq\rq.
			\item $B_1$: \lq\lq  Hai viên bi lấy ra từ hộp thứ nhất có cả màu xanh và màu đỏ\rq\rq.
			\item $B_2$: \lq\lq  Hai viên bi lấy ra từ hộp thứ nhất có màu đỏ\rq\rq.
		\end{itemize}
		Ta có
		\[\mathrm{P}(B_1) = \dfrac{\mathrm{C}^1_{5}}{\mathrm{C}^2_{6}}=\dfrac{1}{3};\quad
		\mathrm{P}(B_2) = \dfrac{\mathrm{C}^2_5}{\mathrm{C}^2_{6}}=\dfrac{2}{3};\quad
		\mathrm{P}(A\mid B_1) = \dfrac{\mathrm{C}^2_6}{\mathrm{C}^2_{10}}=\dfrac{1}{3};\quad
		\mathrm{P}(A\mid B_2) = \dfrac{\mathrm{C}^2_7}{\mathrm{C}^2_{10}}=\dfrac{7}{15}.
		\]
		Áp dụng công thức xác suất toàn phần, ta có
		\allowdisplaybreaks
		\begin{eqnarray*}
			\mathrm{P}(A)
			&=& \mathrm{P}(A|B_1)\cdot \mathrm{P}(B_1) + \mathrm{P}(A|B_2)\cdot\mathrm{P}(B_2)\\
			&=& \dfrac{1}{3}\cdot \dfrac{1}{3} + \dfrac{7}{15}\cdot \dfrac{2}{3}\\
			&=& \dfrac{19}{45}.
		\end{eqnarray*}
}\end{bt}
%Bài 2

%========================
\begin{dang}{Công thức Bayes tính xác suất}
	\begin{itemize}
		\item Giả sử $A$ và $B$ là hai biến cố ngẫu nhiên thoả mãn $\mathrm{P}(A) > 0$ và $0 < \mathrm{P}(B) < 1$. Khi đó
		$$\mathrm{P}(B|A)=\dfrac{\mathrm{P}(B)\mathrm{P}(A|B)}{\mathrm{P}(B)\mathrm{P}(A|B)+\mathrm{P}(\overline{B})\mathrm{P}(A|\overline{B})}$$ gọi là \textbf{\textit{công thức Bayes}}.
	\end{itemize}
	\begin{note}
		Với $\mathrm{P}(A) > 0$, công thức $\mathrm{P}(B|A)=\dfrac{\mathrm{P}(B)\mathrm{P}(A|B)}{\mathrm{P}(A)}$ cũng được gọi là công thức Bayes.
	\end{note}
\end{dang}
%----------------------------
\subsubsection{Ví dụ minh hoạ}
\begin{vd}%[2D5N2-3]
	Cho hai biến cố $A$, $B$ sao cho $\mathrm{P}(A) = 0{,}6$; $\mathrm{P}(B)=0{,}4$; $\mathrm{P}(A|B) = 0{,}3$. Tính $\mathrm{P}(B|A)$.
	\loigiai{
		Áp dụng công thức Bayes, ta có
		\[\mathrm{P}(B|A)=\dfrac{\mathrm{P}(B)\cdot \mathrm{P}(A|B)}{\mathrm{P}(A)}=\dfrac{0{,}4\cdot 0{,}3}{0{,}6}=0{,}2.\]
	}
\end{vd}
\begin{vd}%[2D5N2-3]
	Cho $\mathrm{P}(A)=\dfrac{2}{5}$; $\mathrm{P}\left( B\mid A\right)=\dfrac{1}{3}$; $\mathrm{P}\left(B\mid \overline{A}\right)=\dfrac{1}{4}$. Giá trị của $\mathrm{P}(A\mid B)$ là
	\loigiai{Áp dụng công thức Bayes ta có
		$$\mathrm{P}(A\mid B)=\dfrac{\mathrm{P}(A)\cdot \mathrm{P}(B\mid A)}{\mathrm{P}(A)\cdot \mathrm{P}(B\mid A)+\mathrm{P}\left( \overline{A}\right) \cdot \mathrm{P}\left( B\mid \overline{A}\right) }=\dfrac{\dfrac{2}{5}\cdot \dfrac{1}{3}}{\dfrac{2}{5}\cdot \dfrac{1}{3}+\dfrac{3}{5}\cdot \dfrac{1}{4}}=\dfrac{8}{17}.$$
	}
\end{vd}
\begin{vd}%[2D5H2-3]
	\immini{Cho sơ đồ hình cây như hình bên. Biết $\mathrm{P}(A)=0{,}6$. Tính $\mathrm{P}(A \mid B)$.}
	{\begin{tikzpicture}[xscale=.2,yscale=0.15,,>=stealth]
			\tikzstyle{block} = [rectangle, draw, fill=blue!10\text{,} rounded corners, text centered, text width = 10em, minimum height = 2em]
			\node (c1) {};
			\node (c2)[above right = 1.5cm of c1] {$A$};
			%			\node at (0.5,5){\fbox{$0\text{,}7$}};
			%			\node at (0.5,-5){\fbox{$0\text{,}3$}};
			\node (c3) [below right= 1.5cm of c1]{$\overline{A}$};
			\node at (12,11.5){$0\text{,}9$};
			\node (c4) at (21.5, 12){$B$};
			\node (c5) at (21.5, 2){$\overline{B}$};
			\node at (12,3){$0\text{,}1$};
			\node (c6) at (21.5, -4){$B$};
			\node at (12,-4){$0\text{,}3$};
			\node (c7) at (21.5, -14){$\overline{B}$};
			\node at (12,-13){$0\text{,}7$};
			\draw[->] (c1.east) -- (c2.west);
			\draw[->] (c1.east) -- (c3.west);
			\draw[->] (c2.east) -- (c4.west);
			\draw[->] (c2.east) -- (c5.west);
			\draw[->] (c3.east) -- (c6.west);
			\draw[->] (c3.east) -- (c7.west);
	\end{tikzpicture}}
	\loigiai{
		Theo công thức Bayes ta có\begin{eqnarray*}
			\mathrm{P}\left( {A|B} \right) &=& \dfrac{{\mathrm{P}\left( A \right) \mathrm{P}\left( {B|A} \right)}}{{P\left( B \right)}} \\
			&=& \dfrac{{\mathrm{P}\left( A \right)\mathrm{P}\left( {B|A} \right)}}{\mathrm{P}(A) \cdot P\mathrm{P}(B \mid A)+\mathrm{P}(\overline{A}) \cdot \mathrm{P}(B \mid \overline{A})} \\
			&=& \dfrac{0{,}6 \cdot 0{,}9}{0{,}6 \cdot 0{,}9+0{,}4 \cdot 0{,}3} = \dfrac{9}{11}.
		\end{eqnarray*}
	}
\end{vd}
\begin{vd}%[2D5H2-3]
	Một bệnh viện có hai phòng khám là phòng A và phòng B với khả năng lựa chọn của bệnh nhân là như nhau. Tỉ lệ bệnh nhân nam có ở phòng A và phòng B lần lượt là $60\%$ và $40\%$. Một người bệnh được chọn ngẫu nhiêu từ hai phòng khám và biết người này là nam, xác suất để người bệnh được chọn đến từ phòng A là
	\loigiai{Một người bệnh được chọn ngẫu nhiên từ hai phòng khám.\\
		Gọi $X$ là biến cố \lq \lq Người đó đến từ phòng khám A\rq \rq \, và $Y$, $\overline{Y}$ lần lượt là biến cố \lq \lq Người đó là nam\rq \rq \; và \lq \lq Người đó không là nam\rq \rq.\\
		Ta có sơ đồ hình cây sau
		\begin{center}
			\begin{tikzpicture}[>=stealth,xscale=0.8,yscale=0.5]
				%Khung 1
				\draw (-3.5,-1) rectangle (2.2,0);
				\draw (-0.8,-0.5) node{Bệnh nhân được chọn} ;
				%Mui ten 1,2
				\draw [->] (2.2,-0.5)--(3.8,1.6) node[pos=0.5,sloped,above]{$0{,}5$};
				\draw [->] (2.2,-0.5)--(3.8,-2.6) node[pos=0.5,sloped,below]{$0{,}5$};
				%Khung 2.1
				\draw (3.8,1.1) rectangle (5.1,2.1);
				\draw (8.9/2,1.6) node{$X$} ;
				%Khung 2.2
				\draw (3.8,-2.1) rectangle (5.1,-3.1);
				\draw (8.9/2,-2.6) node{$\overline{X}$} ;
				%Mui ten 3,4
				\draw [->] (5.1,1.6)--(6.5,2.6) node[pos=0.5,sloped,above]{$0{,}6$};
				\draw [->] (5.1,1.6)--(6.5,0.6) node[pos=0.5,sloped,below]{$0{,}4$};
				%Mui ten 5,6
				\draw [->] (5.1,-2.6)--(6.5,-1.6) node[pos=0.5,sloped,above]{$0{,}4$};
				\draw [->] (5.1,-2.6)--(6.5,-3.6) node[pos=0.5,sloped,below]{$0{,}6$};
				%Khung 3.1
				\draw (6.5,2.2) rectangle (7.7,3.2);
				\draw (7.1,5.4/2) node{$Y$} ;
				%Khung 3.2
				\draw (6.5,1.2) rectangle (7.7,0.2);
				\draw (7.1,1.4/2) node{$\overline{Y}$} ;
				%Khung 3.3
				\draw (6.5,-1.1) rectangle (7.7,-2.1);
				\draw (7.1,-3.2/2) node{$Y$} ;
				%Khung 3.3
				\draw (6.5,-2.9) rectangle (7.7,-3.9);
				\draw (7.1,-3.4) node{$\overline{Y}$} ;
				%Kết quả
				\draw (9.5,3.7) node{\textbf{Kết quả}};
				\draw (9.5,2.7) node{$XY$};
				\draw (9.5,0.7) node{$X \overline{Y}$};
				\draw (9.5,-1.6) node{$\overline{X}Y$};
				\draw (9.5,-3.4) node{$\overline{X} \,\,\overline{Y}$};
				%Xác suất
				\draw (12.5,3.7) node{\textbf{Xác suất}};
				\draw (12.5,2.7) node{$0{,}3$};
				\draw (12.5,0.7) node{$0{,}2$};
				\draw (12.5,-1.6) node{$0{,}2$};
				\draw (12.5,-3.4) node{$0{,}3$};
			\end{tikzpicture}
		\end{center}
		Theo công thức Bayes, ta có $$\mathrm{P}(X|Y)=\dfrac{\mathrm{P}(X)\mathrm{P}(Y|X)}{\mathrm{P}(X)\mathrm{P}(Y|X)+\mathrm{P}(\overline{X})\mathrm{P}(Y|\overline{X})}=\dfrac{0{,}3}{0{,}3+0{,}2}=0{,}6.$$
		Vậy với một người bệnh được chọn ngẫu nhiêu từ hai phòng khám và biết người này là nam, xác suất để người đó đến từ phòng A là $0{,}6$.}
\end{vd}
\begin{vd}%[2D5H2-3]
	Trong một kì thi tốt nghiệp trung học phổ thông, một tỉnh X có $80 \%$ học sinh lựa chọn tổ hợp A00 (gồm các môn Toán, Vật lí, Hoá học). Biết rằng, nếu một học sinh chọn tổ hợp A00 thì xác suất để học sinh đó đỗ đại học là 0{,}6; còn nếu một học sinh không chọn tổ hợp A00 thì xác suất để học sinh đó đỗ đại học là 0{,}7. Chọn ngẫu nhiên một học sinh của tỉnh X đã tốt nghiệp trung học phổ thông trong kì thi trên. Biết rằng học sinh này đã đỗ đại học. Tính xác suất để học sinh đó chọn tổ hợp A00. (làm tròn hai chữ số thập phân)
	\loigiai{Gọi $A$ là biến cố: ``Học sinh đó chọn tổ hợp A00''; $B$ là biến cố: ``Học sinh đó đỗ đại học''.\\
		Ta cần tính $\mathrm{P}(A \mid B)$. Theo công thức Bayes, ta cần biết: $\mathrm{P}(A), \mathrm{P}(\overline{A}), \mathrm{P}(B \mid A)$ và $\mathrm{P}(B \mid \overline{A})$.\\
		Ta có:
		\begin{itemize}
			\item $\mathrm{P}(A)=0{,}8; \mathrm{P}(\overline{A})=1-\mathrm{P}(A)=1-0{,}8=0{,}2$.\\
			\item $\mathrm{P}(B \mid A)$ là xác suất để một học sinh đỗ đại học với điều kiện học sinh đó chọn tổ hợp $A 00$.\\
			Suy ra $\mathrm{P}(B \mid A)=0{,}6$.
			\item $\mathrm{P}(B \mid \overline{A})$ là xác suất để một học sinh đỗ đại học với điều kiện học sinh đó không chọn tổ hợp $\mathrm{A} 00$. Suy ra $\mathrm{P}(B \mid \overline{A})=0{,}7$.
		\end{itemize}
		Thay vào công thức Bayes ta được:
		$$\mathrm{P}(A \mid B)=\frac{\mathrm{P}(A) \cdot \mathrm{P}(B \mid A)}{\mathrm{P}(A) \cdot \mathrm{P}(B \mid A)+\mathrm{P}(\overline{A}) \cdot \mathrm{P}(B \mid \overline{A})}=\frac{0{,}8 \cdot 0{,}6}{0{,}8 \cdot 0{,}6+0{,}2 \cdot 0{,}7} \approx 0{,}77.$$}
\end{vd}
\begin{vd}%[2D5H2-3]
	Kết quả khảo sát tại một xã cho thấy có $20 \%$ cư dân hút thuốc lá. Tỉ lệ cư dân thường xuyên gặp các vấn đề sức khoẻ về đường hô hấp trong số những người hút thuốc lá và không hút thuốc lá lần lượt là $70\%$, $15\%$. Giả sử ta gặp một cư dân của xã, gọi $A$ là biến cố \lq\lq  Người đó có hút thuốc lá\rq\rq\, và $B$ là biến cố \lq\lq  Người đó thường xuyên gặp các vấn đề sức khoẻ về đường hô hấp\rq\rq.
	\begin{listEX}
		\item Vẽ sơ đồ hình cây.
		\item Nếu ta gặp một cư dân của xã thì xác suất người đó thường xuyên gặp các vấn đề sức khoẻ về đường hô hấp là bao nhiêu?
		\item  Nếu ta gặp một cư dân của xã thường xuyên gặp các vấn đề sức khoẻ về đường hô hấp thì xác suất người đó có hút thuốc lá là bao nhiêu?
	\end{listEX}
	\loigiai{
		\begin{listEX}
			\item Ta có sơ đồ hình cây sau.
			\begin{center}
				\begin{tikzpicture}[scale=.3,>=stealth]
					%-------------
					\tikzstyle{block} = [rectangle, draw, fill=none, rounded corners, minimum height = 2em]
					%-------------
					\node[block] (c1) {Gặp một cư dân};
					\node[block] (c2) [above right = 2cm of c1]{$A$};
					\node[block] (c3) [ below right= 2cm of c1]{$\overline{A}$};
					\node[block] (c4) [above right = 1cm of c2]{$B$} ;
					\node[block] (c5) [below right = 1cm of c2]{$\overline{B}$};
					\node[block] (c6) [ above right =1cm of c3]{$B$};
					\node[block] (c7) [ below right = 1cm of c3]{$\overline{B}$};
					%--------------
					\draw
					(17.5,15) node[right] {\text{Kết quả}}
					(20,11.3) node[right] {$AB$}
					(20,2.5) node[right] {$A\overline{B}$}
					(20,-2.5) node[right] {$\overline{A}B$}
					(20,-11.5) node[right] {$\overline{A} \,\,\overline{B}$};
					%--------------
					\draw
					(25,15) node[right] {\text{Xác suất}}
					(25,11.3) node[right] {$0{,}14$}
					(25,2.5) node[right] {$0{,}06$}
					(25,-2.5) node[right] {$0{,}12$}
					(25,-11.5) node[right] {$0{,}68$};
					%------------
					\draw[->] (c1.east) --node[above left]{$0{,}2$} (c2.west);
					\draw[->] (c1.east) --node[below left]{$0{,}8$} (c3.west);
					\draw[->] (c2.east) --node[above left]{$0{,}7$} (c4.west);
					\draw[->] (c2.east) --node[below left]{$0{,}3$} (c5.west);
					\draw[->] (c3.east) --node[above left]{$0{,}15$} (c6.west);
					\draw[->] (c3.east) -- node[below left]{$0{,}85$} (c7.west);
				\end{tikzpicture}
			\end{center}
			\item Ta có $\mathrm{P}(B)=\mathrm{P}(A) \cdot \mathrm{P}(B | A)+\mathrm{P}(\overline{A}) \cdot \mathrm{P}\left(B | \overline{A}\right)=0{,}14+0{,}12=0{,}26$.\\
			Vậy nếu ta gặp một cư dân của xã thì xác suất người đó thường xuyên gặp các vấn đề sức khoẻ về đường hô hấp là $26\%$.
			\item Theo công thức Bayes, ta có $\mathrm{P}\left(A | B\right)=\dfrac{\mathrm{P}(A)\cdot \mathrm{P}\left(B| A\right)}{\mathrm{P}(B)}=\dfrac{0{,}14}{0{,}26} \approx 0{,}54$.\\
			Vậy nếu ta gặp một cư dân của xã thường xuyên gặp các vấn đề sức khoẻ về đường hô hấp thì xác suất người đó có hút thuốc lá là khoảng $54\%$.
		\end{listEX}
	}
\end{vd}
\begin{vd}%[2D5H2-3]
	Một nhà máy có hai phân xưởng I và II. Phân xưởng I sản xuất $40\%$ số sản phẩm
	và phân xưởng II sản xuất $60\%$ số sản phẩm. Tỉ lệ sản phẩm bị lỗi của phần xưởng I
	là $2\%$ và của phân xưởng II là $1\%$.
	\begin{listEX}
		\item  Kiểm tra ngẫu nhiên 1 sản phẩm của nhà máy và tính xác suất để sản phẩm đó bị lỗi.
		\item Biết rằng sản phẩm được chọn bị lỗi. Hỏi xác suất sản phẩm đó do phân xưởng nào
		sản xuất cao hơn?
	\end{listEX}
	\loigiai{\begin{listEX}
			\item  Gọi $A$ là biến cố “Sản phẩm bị lỗi” và $B$ là biến cố “Sản phẩm lấy ra do phân xưởng I
			sản xuất”.\\
			Do phân xưởng I sản xuất $40\%$ số sản phẩm và phân xưởng II sản xuất $60\%$ số sản phẩm nên
			$$\mathrm{P}(B)=0{,}4 \text{ và } \mathrm{P}(\overline{B})=1-0{,}4=0{,}6.$$
			Do tỉ lệ sản phẩm bị lỗi của phân xưởng I là $2\%$ và của phân xưởng II là $1\%$ nên
			$$\mathrm{P}(A|B)=0{,}02 \text{ và } \mathrm{P}(A|\overline{B})=0{,}01.$$
			Xác suất để sản phẩm lấy ra bị lỗi là
			$$\mathrm{P}(A)=\mathrm{P}(B)\mathrm{P}(A|B)+\mathrm{P}(\overline{B})\mathrm{P}(A|\overline{B})=0{,}4\cdot0{,}02+0{,}6 \cdot 0{,}01=0{,}014.$$
			\item Nếu sản phẩm lấy ra bị lỗi thì xác suất sản phẩm đó do phân xưởng I sản xuất là
			$$\mathrm{P}(B|A)=\dfrac{\mathrm{P}(B)\mathrm{P}(A|B)}{\mathrm{P}(A)}=\dfrac{0{,}4 \cdot 0{,}02}{0{,}014}=\dfrac{4}{7}.$$
			Nếu sản phẩm lấy ra bị lỗi thì xác suất sản phẩm đó do phân xưởng II sản xuất là
			$$\mathrm{P}(\overline{B}|A)=1-\mathrm{P}(B|A)=\dfrac{3}{7}.$$
			Vậy nếu sản phẩm lấy ra bị lỗi thì xác suất sản phẩm đó do phân xưởng I sản xuất cao hơn
			xác suất sản phẩm đó do phân xưởng II sản xuất.
	\end{listEX}}
\end{vd}
\begin{vd}%[2D5H2-3]
	Giả sử có một loại bệnh mà tỉ lệ người mắc bệnh là $0{,}1\%$. Giả sử có một loại xét nghiệm, mà ai mắc bệnh khi xét nghiệm cũng có phản ứng dương tính, nhưng tỉ lệ phản ứng dương tính giả là $5\%$ (tức là trong số những người không bị bệnh có $5\%$ số người xét nghiệm lại có phản ứng dương tính).
	\begin{listEX}
		\item Vẽ sơ đồ cây biểu thị tình huống trên.
		\item Khi một người xét nghiệm có phản ứng dương tính thì khả năng mắc bệnh của người đó là bao nhiêu phần trăm (làm tròn kết quả đến hàng phần trăm).
	\end{listEX}
	\loigiai{
		\begin{listEX}
			\item Xét hai biến cố
			\begin{itemize}
				\item $K$: \lq\lq  Người được chọn ra không mắc bệnh\rq\rq.
				\item $D$: \lq\lq  Người được chọn ra có phản ứng dương tính\rq\rq.
			\end{itemize}
			Do tỉ lệ người mắc bệnh là là $0,1\%=0{,}001$ nên $\mathrm{P}(K)=1-0{,}001 = 0{,}999$.\\
			Trong số những người không mắc bệnh có $5\%$ số người có phản ứng dương tính nên \break $\mathrm{P}(D|K) = 5\% = 0{,}05$. Vì ai mắc bệnh khi xét nghiệm cũng có phản ứng dương tính nên $\mathrm{P}\left(\overline{K}\right)=1$.\\
			Ta có sơ đồ cây biểu thị tình huống đã cho
			\begin{center}
				\begin{tikzpicture}[->,>=stealth,line join=round,line cap=round,font=\footnotesize,scale=1]
					\def\xmot{4}
					\def\xhai{8}
					\node (O) at (0,0){};
					\node (B) at (\xmot,1){$K$};
					\node (B1) at (\xmot,-1){$\overline{K}$};
					\node (BA) at (\xhai,2){$D$};
					\node (BA1) at (\xhai,0.3){$\overline{D}$};
					\node (B1A) at (\xhai,-0.3){$D$};
					\node (B1A1) at (\xhai,-1.75){$\overline{D}$};
					\foreach \x/\y/\p/\l in
					{
						O/B/above/$\mathrm{P}(K)=0{,}999$,
						B/BA/above/$\mathrm{P}(D|K)=0{,}05$,
						B/BA1//,
						O/B1/below/$\mathrm{P}\left(\overline{K}\right)=0{,}001$,
						B1/B1A/above/$\mathrm{P}\left(D|\overline{K}\right)=1$,
						B1/B1A1//
					}
					{
						\draw[->] (\x)--(\y)node[midway,\p,scale=0.8,sloped]{\l};
					}
				\end{tikzpicture}
			\end{center}
			\item Ta thấy rằng khả năng mắc bệnh của một người xét nghiệm có phản ứng dương tính chính là $\mathrm{P}\left(\overline{K}|D\right)$. Áp dụng công thức Bayes, ta có:
			\[\mathrm{P}\left(\overline{K}|D\right)=\dfrac{\mathrm{P}\left(\overline{K}\right)\cdot \mathrm{P}\left(D|\overline{K}\right)}{\mathrm{P}\left(\overline{K}\right)\cdot \mathrm{P}\left(D|\overline{K}\right) + \mathrm{P}(K)\cdot\mathrm{P}(D|K)}=\dfrac{0{,}001\cdot 1}{0{,}001\cdot 1 + 0{,}999 \cdot 0{,}05}\approx 1{,}96\%.\]
			Vậy xác suất mắc bệnh của một người xét nghiệm có phản ứng dương tính là $1{,}96\%$.
		\end{listEX}
	}
\end{vd}

\begin{vd}%[2D5H2-3]
	\immini{Một loại xét nghiệm nhanh SARS-CoV-2 cho
		kết quả dương tính với $76,2\%$ các ca thực sự
		nhiễm virus và kết quả âm tính với $99,1\%$ các
		ca thực sự không nhiễm virus (nguồn: https://
		tapchiyhocvietnam.vn/index.php/vmj/article/
		view/2124/1921). Giả sử tỉ lệ người nhiễm virus
		SARS-CoV-2 trong một cộng đồng là $1\%$.\\
		Một người làm xét nghiệm và nhận được kết quả dương tính.
		Tính xác suất người đó thực sự nhiễm virus (kết quả làm tròn đến hàng phần nghìn).
	}{\includegraphics[scale=0.5]{images/12-SGK-CTST-6-2-1.png}}
	\loigiai{
		Gọi $A$ là biến cố “Người làm xét nghiệm có kết quả dương tính” và $B$ là biến cố
		“Người làm xét nghiệm thực sự nhiễm virus”.\\
		Do xét nghiệm cho kết quả dương tính với $76,2\%$ các ca thực sự nhiễm virus nên $\mathrm{P}(A|B)=0{,}762$.\\
		Do xét nghiệm cho kết quả âm tính với $99,1\%$ các ca thực sự không nhiễm virus nên
		$\mathrm{P}(\overline{A}|\overline{B})=0{,}991$. Suy ra $$\mathrm{P}(A|\overline{B})=1-0{,}991=0{,}009.$$
		Do tỉ lệ người nhiễm virus trong cộng đồng là $1\%$ nên $\mathrm{P}(B)=0,01$ và $\mathrm{P}(\overline{B}) = 0,99$.
		Áp dụng công thức xác suất toàn phần, ta có xác suất người làm xét nghiệm có kết quả
		dương tính là $$\mathrm{P}(A)=\mathrm{P}(B)\mathrm{P}(A|B)+\mathrm{P}(\overline{B})\mathrm{P}(A|\overline{B})=0{,}01\cdot0{,}762+0{,}99\cdot0{,}009=0{,}01653.$$
		Xác suất một người thực sự nhiễm virus khi người đó có kết quả xét nghiệm dương tính
		là $\mathrm{P}(B|A)$. Ta có $$\mathrm{P}(B|A)=\dfrac{\mathrm{P}(B)\mathrm{P}(A|B)}{\mathrm{P}(A)}=\dfrac{0{,}01\cdot0{,}762}{0{,}01653}\approx0{,}461.$$}
\end{vd}
%----------------------------
\subsubsection{Bài tập áp dụng}
\begin{bt}%[2D5H2-3]
	Trong một kì sát hạch lái xe có $65 \%$ thí sinh nam. Biết rằng $80 \%$ thí sinh nam và $70\% $ thí sinh nữ đỗ kì sát hạch này.
	\begin{listEX}
		\item Tính tỉ lệ thí sinh đỗ kì sát hạch này.
		\item Chọn ngẫu nhiên một thí sinh đã đỗ kì sát hạch. Tính xác suất thí sinh đó là nữ.
	\end{listEX}
	\loigiai{
		\begin{listEX}
			\item Xét các biến cố sau
			\begin{itemize}
				\item $D$: \lq\lq  Thí sinh đỗ kì sát hạch\rq\rq
				\item $M$: \lq\lq  Thí sinh là nam giới\rq\rq\,
				\item $\overline{M}$: \lq\lq  Thí sinh là nữ giới\rq\rq\,
				\item $D|M$: \lq\lq  Thí sinh nam đỗ kì sát hạch\rq\rq\,
				\item $D|\overline{M}$: \lq\lq  Thí sinh nữ đỗ kì sát hạch\rq\rq\,.
			\end{itemize}
			Theo đề bài ta có
			\begin{center}
				$\mathrm{P}(M) =0{,}65$;  $\mathrm{P}(D|M) =0{,}8$; $\mathrm{P}(\overline{M}) =0{,}35$; $\mathrm{P}(D|\overline{M}) =0{,}7$.
			\end{center}
			Theo công thức xác suất toàn phần, ta có
			\begin{align*}
				\mathrm{P}(D) &= \mathrm{P}(M) \cdot \mathrm{P}(D|M) + \mathrm{P}(\overline{M}) \cdot \mathrm{P}(D|\overline{M}).
			\end{align*}
			Thay vào giá trị đã cho
			\begin{align*}
				\mathrm{P}(D) &= 0{,}65 \cdot 0{,}8 + 0{,}35 \cdot 0{,}7 \\
				&= 0{,}765.
			\end{align*}
			\item Xác suất một thí sinh đỗ là nữ\\
			Để tính xác suất này, ta sử dụng công thức Bayes
			\begin{align*}
				\mathrm{P}(\overline{M}|D) &= \dfrac{\mathrm{P}(D|\overline{M}) \cdot \mathrm{P}(\overline{M})}{\mathrm{P}(D)}.
			\end{align*}
			Thay vào giá trị đã cho và kết quả tính được ở trên
			\begin{align*}
				\mathrm{P}(\overline{M}|D) &= \dfrac{0{,}7 \cdot 0{,}35}{0{,}765} \\
				&\approx 0{,}321.
			\end{align*}
			Vậy, xác suất một thí sinh đỗ kì sát hạch là nữ là khoảng $0{,}321$, hay chấp nhận được là khoảng $32{,}1\%$.
		\end{listEX}
	}
\end{bt}
\begin{bt}%[2D5H2-3]
	Bạn Nam tham gia một gian hàng trò chơi dân gian trong hội xuân của trường. Trò chơi có hai lượt chơi. Xác suất để Nam thắng ở lượt chơi thứ nhất là $0{,}6$. Nếu Nam thắng ở lượt chơi thứ nhất thì xác suất Nam thắng ở lượt chơi thứ hai là $0{,}8 $. Ngược lại, nếu Nam thua ở lượt chơi thứ nhất thì xác suất Nam thắng ở lượt chơi thứ hai là $0{,}3$.
	\begin{listEX}
		\item Vẽ sơ đồ hình cây mô tả các khả năng xảy ra và xác suất tương ứng khi Nam tham gia trò chơi này.
		\item 	Biết Nam đã thắng ở lượt chơi thứ hai, tính xác suất Nam thắng ở lượt chơi thứ nhất.
	\end{listEX}
	\loigiai{
		\begin{listEX}
			\item Sơ đồ hình cây
			\begin{center}
				\begin{tikzpicture}[scale=.2,>=stealth]
					%-------------
					\tikzstyle{block} = [rectangle, draw, fill=cyan!20, rounded corners, text centered, text width = 10em, minimum height = 2em]
					%-------------
					\node (c1) [block] {Trò chơi};
					\node (c2) [block, above right = 3cm of c1]{Nam thắng lượt chơi thứ nhất};
					\node (c3) [block, below right= 3cm of c1]{Nam thua lượt chơi thứ nhất };
					\node (c4) [block,above right = 1.5cm of c2]{Nam thắng lượt chơi thứ hai};
					\node (c5) [block,below right = 1.5cm of c2]{Nam thua lượt chơi thứ hai};
					\node (c6) [block, above right =1.5cm of c3]{Nam thắng lượt chơi thứ hai};
					\node (c7) [block, below right = 1.5cm of c3]{Nam thua lượt chơi thứ hai};
					%--------------
					\draw[->] (c1.east) -- (c2.west);
					\draw[->] (c1.east) -- (c3.west);
					\draw[->] (c2.east) -- (c4.west);
					\draw[->] (c2.east) -- (c5.west);
					\draw[->] (c3.east) -- (c6.west);
					\draw[->] (c3.east) -- (c7.west);
					\draw ($(c1.east)!.5!(c2.west)$) node [below right=-.1]{\color{red}$\mathrm{P}(A) =0{,}6$};
					\draw ($(c1.east)!.5!(c3.west)$) node [above right=-.1]{\color{red}$\mathrm{P}(\overline{A}) =0{,}4$};
					\draw ($(c2.east)!.5!(c4.west)$) node [below right=-.1]{\color{red}$\mathrm{P}(B|A) =0{,}8$};
					\draw ($(c2.east)!.5!(c5.west)$) node [above right=-.1]{\color{red}$\mathrm{P}(\overline{B}|A) =0{,}2$};
					\draw ($(c3.east)!.5!(c6.west)$) node [below right=-.1]{\color{red}$\mathrm{P}(B|\overline{A}) =0{,}3$};
					\draw ($(c3.east)!.5!(c7.west)$) node [above right=-.1]{\color{red}$\mathrm{P}(\overline{B}|\overline{A}) =0{,}7$};
				\end{tikzpicture}
			\end{center}
			\item
			Công thức Bayes cho sự kiện $A$ và $B$ là $\mathrm{P}(A | B)=\dfrac{\mathrm{P}(B | A) \cdot \mathrm{P}(A)}{\mathrm{P}(B)}	$.\\
			Trong đó
			\begin{itemize}
				\item $A$ là Nam thắng ở lượt chơi thứ nhất.
				\item $B$ là Nam thắng ở lượt chơi thứ hai.
				\item $\mathrm{P}(A)$: Xác suất Nam thắng ở lượt chơi thứ nhất $0{,}6$.
				\item $\mathrm{P}(B | A)$: Xác suất Nam thắng ở lượt chơi thứ hai khi đã thắng ở lượt chơi thứ nhất $0{,}8$.
				\item $\mathrm{P}(B | \overline{A})$: Xác suất Nam thắng ở lượt chơi thứ hai khi đã thua ở lượt chơi thứ nhất $0.3$.
			\end{itemize}
			Công thức Bayes:
			$$
			\begin{aligned}[t]
				\mathrm{P}(A | B)&=\dfrac{\mathrm{P}(B | A) \cdot \mathrm{P}(A)}{\mathrm{P}(B | A) \cdot \mathrm{P}(A)+\mathrm{P}(B | \overline{A}) \cdot \mathrm{P}(\overline{A})} \\
				& =\dfrac{0{,}8 \cdot 0{,}6}{0{,}8 \cdot 0{,}6+0{,}3 \cdot 0{,}4} \\
				& = \dfrac{0{,}48}{0{,}48+0{,}12} \\
				& = \dfrac{0{,}48}{0{,}6} \\
				& = 0{,}8.
			\end{aligned}
			$$
			Vậy, xác suất Nam thắng ở lượt chơi thứ nhất khi đã thắng ở lượt chơi thứ hai là khoảng $0{,}8$ hoặc $80 \%$.
		\end{listEX}
	}
\end{bt}

\begin{bt}%[2D5H2-3]
	Năm $2001$, Cộng đồng châu Âu có làm một đợt kiểm tra rất rộng rãi các con bò để phát hiện những con bị bệnh bò điên. Không có xét nghiệm nào cho kết quả chính xác $100 \%$. Một loại xét nghiệm, mà ở đây ta gọi là xét nghiệm $A$ cho kết quả như sau: khi con bò bị bệnh bò điên thì xác suất để có phản ứng dương tính trong xét nghiệm A là $70 \%$ còn khi con bò không bị bệnh thì xác suất để có phản ứng dương tính trong xét nghiệm $A$ là $10 \%$. Biết rằng tỉ lệ bò bị mắc bệnh bò điên ở Hà Lan là $13$ con trên $1~000~000$ con \textit{(Nguồn: F. M. Dekking et al., Amodern introduction to probability and statistics Understanding why and how, Springer, $2005$)}. Hỏi khi một con bò ở Hà Lan có phản ứng dương tính với xét nghiệm $A$ thì xác suất để nó bị mắc bệnh bò điên là bao nhiêu?
	\loigiai{
		Xét hai biến cố\\
		$N$: \lq\lq  Con bò được chọn bị nhiễm bệnh\rq\rq.\\
		$D$: \lq\lq  Con bò được chọn có phản ứng dương tính\rq\rq.\\
		Khi đó, ta có\\
		\[\mathrm{P}(N)=\dfrac{13}{1 000 000}=0{,}000013; \qquad \mathrm{P}(\overline{N})=1-\mathrm{P}(N)=0{,}999987;\]
		\[\mathrm{P}(D|N)=70\%=0{,}7; \qquad \mathrm{P}(D|\overline{N})=10\%=0{,}1.\]
		Áp dụng công thức Bayes, ta có\\
		$\mathrm{P}(N|D)=\dfrac{\mathrm{P}(D|N) \cdot \mathrm{P}(N)}{\mathrm{P}(N) \cdot \mathrm{P}(D|N)+\mathrm{P}(\overline{N})\mathrm{P}(D|\overline{N})}=\dfrac{0{,}7 \cdot 0{,}000013}{0{,}7 \cdot 0{,}000013+0{,}1 \cdot 0{,}999987}\approx 0{,}009\%$.
	}
\end{bt}
\begin{bt}%[2D5H2-3]
	\immini{Khi phát hiện một vật thể bay, xác suất một hệ thống radar phát
		cảnh báo là $0,9$ nếu vật thể bay đó là mục tiêu thật và là $0,05$
		nếu đó là mục tiêu giả. Có $99\%$ các vật thể bay là mục tiêu giả.
		Biết rằng hệ thống radar đang phát cảnh báo khi phát hiện một
		vật thể bay. Tính xác suất vật thể đó là mục tiêu thật.}{\includegraphics[scale=0.7]{images/12-SGK-CTST-6-2-4.png}}	\loigiai{Gọi $ A $ là biến cố \lq\lq  Hệ thống radar phát cảnh báo\rq\rq \, và $ B $ là biến cố \lq\lq  Vật thể bay là mục tiêu thật\rq\rq.\\
		Do xác suất một hệ thống radar cảnh báo nếu vật thể bay là mục tiêu thật là $ 0,9 $ nên $ \mathrm{P}(A|B)=0,9 $. \\
		Do xác suất một hệ thống radar cảnh báo nếu vật thể bay là mục tiêu giả là $ 0,05 $ nên $ \mathrm{P}(A|\overline{B})=0,05 $. \\
		Do có $ 99\% $ các vật thể bay là mục tiêu giả nên $ \mathrm{P}(\overline{B})=0,99 $ và $ \mathrm{P}(B)= 0,01$.\\
		Áp dụng công thức xác suất toàn phần, ta có xác suất để hệ thống radar phát cảnh báo là
		$$ \mathrm{P}(A)=\mathrm{P}(B)\mathrm{P}(A|B)+\mathrm{P}(\overline{B})\mathrm{P}(A|\overline{B})=0,01 \cdot 0,9+0,99 \cdot 0,05= 0,0585$$
		Xác suất vật bay là mục tiêu thật khi hệ thống radar đang phát cảnh báo là $ \mathrm{P}(B|A) $. Ta có
		$$ \mathrm{P}(B|A)=\dfrac{\mathrm{P}(B)\mathrm{P}(A|B)}{\mathrm{P}(A)}=\dfrac{0,01 \cdot 0,9}{0,0585}=\dfrac{2}{13}. $$}
\end{bt}
\begin{bt}%[2D5H2-3]
	Trong một kho rượu có 30\% là rượu loại I. Chọn ngẫu nhiên một chai rượu đưa cho ông Tùng, một người sành rượu, để nếm thử. Biết rằng, một chai rượu loại I có xác suất 0{,}9 để ông Tùng xác nhận là loại I; một chai rượu không phải loại I có xác suất 0{,}95 để ông Tùng xác nhận đây không phải là loại I. Sau khi nếm, ông Tùng xác nhận đây là rượu loại I. Tính xác suất để chai rượu đúng là rượu loại I.
	\loigiai{Xét biến cố $A$: ``Chai rượu đó là chai rượu loại I''. Xét biến cố $B$: ``Ông Tùng xác nhận chai rượu đó là rượu loại I''.\\
		Ta cần tính $\mathrm{P}(A\mid B)$.
		Áp dụng công thức Bayes
		$$\mathrm{P}(A\mid B) = \dfrac{\mathrm{P}(A)\cdot \mathrm{P}(B\mid A)}{\mathrm{P}(A)\cdot \mathrm{P}(B\mid A) + \mathrm{P}(\overline{A})\cdot \mathrm{P}(B \mid \overline{A})}.$$
		\begin{itemize}
			\item Tính $P\mathrm{P}(A)$: Đây là chai rượu đó là chai rượu loại I. Vậy $\mathrm{P}(A)=0{,}3$.
			\item Tính $\mathrm{P}(\overline{A})$: $\mathrm{P}(\overline{A})=1-\mathrm{P}(A)=0{,}7$.
			\item Tính $\mathrm{P}(B \mid A)$: Đây là xác suất ông Tùng xác nhận đúng một chai rượu loại I là một chai rượu loại I. Vậy $\mathrm{P}(B \mid A)=0{,}9$.
			\item Tính $\mathrm{P}(B \mid \overline{A})$: Đây là xác suất ông Tùng xác nhận sai một chai rượu không phải loại I là một chai rượu loại I. Vậy $\mathrm{P}(B \mid \overline{A})=1-0{,}95=0{,}05$.
		\end{itemize}
		Vậy ta có
		$$\mathrm{P}(A\mid B) = \dfrac{\mathrm{P}(A)\cdot \mathrm{P}(B\mid A)}{\mathrm{P}(A)\cdot \mathrm{P}(B\mid A) + \mathrm{P}(\overline{A})\cdot \mathrm{P}(B \mid \overline{A})}=\dfrac{0{,}3\cdot 0{,}9}{0{,}3\cdot 0{,}9+0{,}7\cdot 0{,}05}=\dfrac{270}{277}\approx 0{,}8852.$$
		Vậy xác suất để chai rượu mà ông Tùng xác nhận là rượu loại I đúng là rượu loại I là khoảng $0{,}8852$.}
\end{bt}
\begin{bt}%[2D5H2-3]
	Một loại linh kiện do hai nhà máy số I, số II cùng sản xuất. Tỉ lệ phế phẩm của các nhà máy I, II lần lượt là $4\%$; $3\%$. Trong một lô linh kiện để lẫn lộn $80$ sản phẩm của nhà máy số I và $120$ sản phẩm của nhà máy số II. Một khách hàng lấy ngẫu nhiên một linh liện từ lô hàng đó.
	\begin{listEX}
		\item Tính xác suất để linh kiện được lấy ra là linh kiện tốt.
		\item Giả sử linh kiện được lấy ra là linh kiện phế phẩm. Xác suất linh kiện đó do nhà máy nào sản xuất là cao nhất?
	\end{listEX}
	\loigiai{
		\begin{listEX}
			\item Xét các biến cố
			\begin{itemize}
				\item $A$: \lq\lq  Linh kiện lấy ra là linh kiện tốt\rq\rq.
				\item $B_1$: \lq\lq  Linh kiện lấy ra là linh kiện từ nhà máy số I\rq\rq.
				\item $B_2$: \lq\lq  Linh kiện lấy ra là linh kiện từ nhà máy số II\rq\rq.
			\end{itemize}
			Theo đề bài, ta có
			\begin{listEX}[2]
				\item[] $\mathrm{P}(A|B_1) =1 - 0{,}04 = 0{,}96$.
				\item[] $\mathrm{P}\left(A|B_2\right) = 1-0{,}03 = 0{,}97$.
				\item[] $\mathrm{P}(B_1)=\dfrac{80}{200}=0{,}4$.
				\item[] $\mathrm{P}\left(B_2\right) = \dfrac{120}{200}=0{,}6$.
			\end{listEX}
			Khi đó áp dụng công thức xác suất toàn phần, ta có
			\[\mathrm{P}(A) = \mathrm{P}(A|B_1)\cdot \mathrm{P}(B_1) + \mathrm{P}\left(A|B_2\right)\cdot\mathrm{P}\left(B_2\right)=0{,}96\cdot 0{,}4 + 0{,}97\cdot 0{,}6=0{,}966.\]
			\item Ta có $\mathrm{P}\left(\overline{A}\right)=1-\mathrm{P}(A) = 0{,}034$.\\
			Áp dụng công thức Bayes, ta có
			\begin{itemize}
				\item $\mathrm{P}\left(B_1|\overline{A}\right)=\dfrac{\mathrm{P}\left(\overline{A}|B_1\right)\cdot \mathrm{P}(B_1)}{\mathrm{P}\left(\overline{A}\right)} = \dfrac{0{,}04\cdot 0{,}4}{0{,}034} = \dfrac{8}{17}\approx 0{,}048$.
				\item $\mathrm{P}\left(B_2|\overline{A}\right)=\dfrac{\mathrm{P}\left(\overline{A}|B_2\right)\cdot \mathrm{P}(B_2)}{\mathrm{P}\left(\overline{A}\right)} = \dfrac{0{,}03\cdot 0{,}6}{0{,}034} = \dfrac{8}{167}\approx 0{,}054$.
			\end{itemize}
			Vậy với điều kiện linh kiện lấy ra là linh kiện phế phẩm thì xác suất linh kiện đó do nhà máy II sản xuất là cao nhất.
		\end{listEX}
	}
\end{bt}
\begin{bt}%[2D5V2-3]
	Một nghiên cứu đã chỉ ra rằng tỉ lệ người bị lao phổi trong nhóm $X$ những người mắc phải hội chứng suy giảm miễn dịch $ {H}$ là $15{,}2 \%$. Kết quả nghiên cứu về một số triệu chứng lâm sàng như có ho trong vòng bốn tuần, hoặc có bị sốt trong vòng bốn tuần, hoặc ra mồ hôi ban đêm từ ba tuần trở lên của nhóm $X$ cho thấy.
	\begin{itemize}
		\item Trong số những người mắc bệnh lao phổi, có $93{,}2 \%$ trường hợp có ít nhất một triệu chứng;
		\item Trong số những người không mắc bệnh lao phổi, có $35{,}8 \%$ trường hợp không có triệu chứng nào.
	\end{itemize}
	Nếu bác sĩ gặp một bệnh nhân thuộc nhóm $X$ và bệnh nhân đó có ít nhất một triệu chứng trên thì xác suất bệnh nhân này mắc bệnh lao phổi là bao nhiêu?
	\loigiai{
		Gọi $A$ là biến cố \lq\lq  Người bị lao phổi\lq\lq .\\
		$\overline{A}$ là biến cố \lq\lq  Người không mắc lao phổi\lq\lq .\\
		$B$ là biến cố \lq\lq  Những người có ít nhất một triệu chứng\rq\rq.\\
		$\overline{B}$ là biến cố \lq\lq  Những người không có triệu chứng\rq\rq.\\
		Ta có $\mathrm{P}(A)=0{,}152$. \\
		Khi đó, xác suất những người không mắc lao phổi là $$\mathrm{P}\left(\overline{A}\right)=1-0{,}152=0{,}848.$$
		Ta có xác suất những người có ít một triệu chứng trong những người mắc lao phổi là $$\mathrm{P}\left(B|A\right)=0{,}932.$$
		Khi đó, xác suất những người không có triệu chứng trong những người mắc lao phổi là  $$\mathrm{P}\left(\overline{B}|A\right)=1-0{,}932=0{,}068.$$
		Mặt khác, ta có xác suất những người không có triệu chứng  $$\mathrm{P}\left(\overline{B}|\overline{A}\right)=0{,}358.$$
		Khi đó, xác suất những người có ít nhất một triệu chứng trong những người không mắc bệnh lao phổi là $$\mathrm{P}\left(B|\overline{A}\right)=1-0{,}358=0{,}642.$$
		Ta có $\mathrm{P}(B)=\mathrm{P}\left(B|A\right)\cdot \mathrm{P}(A)+\mathrm{P}\left(B|\overline{A}\right)\cdot \mathrm{P}\left(\overline{A}\right)=0{,}932\cdot 0{,}152+0{,}642\cdot 0{,}848=0{,}68608$.\\
		Theo công thức Bayes, ta có $\mathrm{P}(A|B)=\dfrac{\mathrm{P}(A)\cdot\mathrm{P}\left(B|A\right)}{\mathrm{P}(B)}=\dfrac{0{,}152\cdot 0{,}932}{0{,}68608}\approx 0{,}2065$.
	}
\end{bt}
\begin{bt}%[2D5V2-3]
	Có hai hộp đựng các viên bi cùng kích thước và khối lượng. Hộp thứ nhất chứa $5$ viên bi đỏ và $5$ viên bi xanh, hộp thứ hai chứa $6$ viên bi đỏ và $4$ viên bi xanh. Lấy ngẫu nhiên một viên bi từ hộp thứ nhất chuyển sang hộp thứ hai, sau đó lấy ra ngẫu nhiên một viên bi từ hộp thứ hai (giả sử viên bi được lấy ra từ hộp thứ hai là bi đỏ). Tính xác suất viên bi đỏ đó là của hộp thứ nhất.
	\loigiai{
		Gọi $A$ là biến cố \lq\lq  Viên bi được lấy ra từ hộp thứ hai là bi đỏ\rq\rq.\\
		$B$ là biến cố \lq\lq  Viên bi được lấy ra từ hộp thứ nhất chuyển sang hộp thứ hai là viên bi đỏ\rq\rq.\\
		$\overline{B}$ là biến cố \lq\lq  Viên bi được lấy ra từ hộp thứ nhất chuyển sang hộp thứ hai là bi xanh\rq\rq.\\
		Ta có $\mathrm{P}(B)=\dfrac{5}{10}=\dfrac{1}{2}$; $\mathrm{P}(\overline{B})=\dfrac{5}{10}=\dfrac{1}{2}$.\\
		Nếu viên bi được lấy ra từ hộp thứ nhất chuyển sang hộp thứ hai là bi đỏ thì sau khi chuyển, hộp thứ hai có $7$ bi đỏ và $4$ bi xanh. Do đó $\mathrm{P}(A\mid B) =\dfrac{7}{11}$.\\
		Nếu viên bi được lấy ra từ hộp thứ nhất chuyển sang hộp thứ hai là bi xanh thì sau khi chuyển, hộp thứ hai có $6$ bi đỏ và $5$ bi xanh. Do đó $\mathrm{P}(A\mid \overline{B})=\dfrac{6}{11}$.\\
		Áp dụng công thức xác suất toàn phần ta có
		$$\mathrm{P})(A) = \mathrm{P}(B) \cdot \mathrm{P}(A\mid B) + \mathrm{P}(\overline{B}) \cdot \mathrm{P}(A\mid \overline{B}) = \dfrac{1}{2} \cdot \dfrac{7}{11} + \dfrac{1}{2}\cdot \dfrac{6}{11} = \dfrac{13}{22}.$$
		Xác suất viên bi được lấy ra từ hộp thứ nhất chuyển sang hộp thứ hai là viên bi đỏ khi biết viên bi lấy ra từ hộp thứ hai là bi đỏ là
		$$\mathrm{P}(B\mid A) =\dfrac{\mathrm{P}(B) \cdot \mathrm{P}(A\mid B)}{\mathrm{P}(A)} = \dfrac{\dfrac{1}{2} \cdot \dfrac{7}{11}}{\dfrac{13}{22}} = \dfrac{7}{13}.$$
		Vì khi viên bi lấy sang hộp thứ II là bi đỏ thì hộp II có $7$ bi đỏ, do vậy xác suất viên bi đỏ lấy ra là của hộp thứ nhất là $\dfrac{1}{7}\cdot\dfrac{7}{13}=\dfrac{1}{13}$.
	}
\end{bt}
\begin{bt}%[2D5V2-3]
	Có hai chiếc hộp, hộp I có $5$ viên bi màu trắng và $5$ viên bi màu đen; hộp II có $6$ viên bi màu trắng và $4$ viên bi màu đen. Các viên bi có cùng kích thước và khối lượng. Lấy ngẫu nhiên đồng thời hai viên bi từ hộp I bỏ sang hộp II. Sau đó lấy ngẫu nhiên một viên bi từ hộp II.
	\begin{listEX}
		\item Tính xác suất để viên bi được lấy ra là viên bi màu trắng.
		\item Giả sử viên bi được lấy ra là viên bi màu trắng. Tính xác suất viên bi màu trắng đó thuộc hộp I.
	\end{listEX}
	\loigiai{
		\begin{listEX}
			\item Xét các biến cố
			\begin{itemize}
				\item $A$: \lq\lq  Viên bi lấy ra là viên màu trắng\rq\rq.
				\item $B_1$: \lq\lq  2 viên bi lấy ra từ hộp I có màu trắng\rq\rq.
				\item $B_2$: \lq\lq  2 viên bi lấy ra từ hộp I có màu đen\rq\rq.
				\item $B_3$: \lq\lq  2 viên bi lấy ra từ hộp I có cả hai màu đen trắng\rq\rq.
			\end{itemize}
			Ta có
			\[\mathrm{P}(B_1) = \dfrac{\mathrm{C}^2_{5}}{\mathrm{C}^2_{10}}=\dfrac{2}{9};\quad
			\mathrm{P}(B_2) = \dfrac{\mathrm{C}^2_5}{\mathrm{C}^2_{10}}=\dfrac{2}{9};\quad
			\mathrm{P}(B_3) = \dfrac{\mathrm{C}^1_5\cdot\mathrm{C}^1_5}{\mathrm{C}^2_{10}}=\dfrac{5}{9}.
			\]
			Áp dụng công thức xác suất toàn phần, ta có
			\allowdisplaybreaks
			\begin{eqnarray*}
				\mathrm{P}(A)
				&=& \mathrm{P}(A|B_1)\cdot \mathrm{P}(B_1) + \mathrm{P}(A|B_2)\cdot\mathrm{P}(B_2) + \mathrm{P}(A|B_3)\cdot\mathrm{P}(B_3)\\
				&=& \dfrac{8}{12}\cdot \dfrac{2}{9} + \dfrac{6}{12}\cdot \dfrac{2}{9} + \dfrac{7}{12}\cdot \dfrac{5}{9}\\
				&=& \dfrac{7}{12}.
			\end{eqnarray*}
			\item Gọi $C$ là biến cố \lq\lq  Viên bi được lấy ra là viên màu trắng thuộc hộp I\rq\rq.\\
			Ta cần tính $\mathrm{P}(C|A)$. \\
			Khi $B_1$ xảy ra, hộp II có $2$ viên bi trắng là từ hộp I nên $P(C|B_1)=\dfrac{2}{12}$.\\
			Khi $B_2$ xảy ra, hộp II không có viên bi trắng nào là từ hộp I nên $P(C|B_2)=0$.\\
			Khi $B_3$ xảy ra, hộp II có $1$ viên bi trắng là từ hộp I nên $P(C|B_3)=\dfrac{1}{12}$.\\
			Do đó, ta có
			\allowdisplaybreaks
			\begin{eqnarray*}
				\mathrm{P}(C) &=& \mathrm{P}(C|B_1)\cdot \mathrm{P}(B_1) + \mathrm{P}(C|B_2)\cdot \mathrm{P}(B_2) + \mathrm{P}(C|B_3)\cdot \mathrm{P}(B_3)\\
				&=& \dfrac{2}{12}\cdot \dfrac{2}{9} + 0 + \dfrac{1}{12}\cdot \dfrac{5}{9}\\
				&=& \dfrac{1}{12}.
			\end{eqnarray*}
			Đồng thời vì khi $C$ xảy ra thì $A$ cũng xảy ra nên $\mathrm{P}(A|C)=1$.\\
			Áp dụng công thức Bayes, ta có
			$$\mathrm{P}(C|A) = \dfrac{\mathrm{P}(C)\cdot \mathrm{P}(A|C)}{\mathrm{P}(A)} 
				= \dfrac{\dfrac{1}{12}\cdot 1}{\dfrac{7}{12}} = \dfrac{1}{7}.$$
		\end{listEX}
	}
\end{bt}
%===================

% %%%%%%%
\setcounter{bt} {0}
\subsection{Bài tập tự luận}
%%==========Bài 1
\begin{bt}%%%%[2D5H2-2]
	Trong quân sự, một máy bay chiến đấu của đối phương có thể xuất hiện ở vị trí X với xác suất 0{,}55. Nếu máy bay đó không xuất hiện ở vị trí X thì nó xuât hiện ở vị trí Y. Để phòng thủ, các bệ phóng tên lửa được bố trí tại các vị trí X và Y. Khi máy bay đối phương xuất hiện ở vị trí X hoặc Y thì tên lửa sẽ được phóng để hạ máy bay đó.\\
	Xét phương án tác chiến sau: Nếu máy bay xuất hiện tại X thì bắn 2 quả tên lửa và nếu máy bay xuất hiện tại Y thì bắn 1 quả tên lửa.\\
	Biết rằng, xác suất bắn trúng máy bay của mỗi quả tên lửa là 0{,}8 và các bệ phóng tên lửa hoạt động độc lập. Máy bay bị bắn hạ nếu nó trúng ít nhất 1 quả tên lửa. Tính xác suất bắn hạ máy bay đối phương trong phương án tác chiến nêu trên.
	\loigiai{Xét biến cố $A$: ``Máy bay xuất hiện ở vị trí X'', điều đó có nghĩa là biến cố $\overline{A}$: ``Máy bay xuất hiện ở vị trí Y''. Xét biến cố $B$: ``Máy bay bị bắn hạ''. \\
	Ta có $P(B)=P(A)\cdot P(B \mid A)+P(\overline{A}) \cdot P(B \mid \overline{A})$.
	\begin{itemize}
	\item Tính $P(A)$, $P(\overline{A})$: $P(A)=0{,}55$ và $P(\overline{A})=0{,}35$.
	\item Tính $P(B\mid A)$: Đây là xác suất để máy bay bị bắn hạ tại vị trí X. Máy bay bị bắn hạ nếu nó trúng ít nhất một 1 quả tên lửa (trong 2 quả tên lửa đối với máy bay ở vị trí X), mà xác suất bắn trúng máy bay của mỗi quả tên lửa là $0{,}8$, vậy $P(B\mid A)=1-\left(1-0{,}8\right)\left(1-0{,}8\right)=0{,}96$.
	\item Tính $P(B\mid \overline{A})$: Đây là xác suất để máy bay bị bắn hạ tại vị trí Y. Máy bay bị bắn hạ nếu nó trúng ít nhất một 1 quả tên lửa (trong 1 quả tên lửa đối với máy bay ở vị trí Y), mà xác suất bắn trúng máy bay của mỗi quả tên lửa là $0{,}8$, vậy $P(B\mid \overline{A})=0{,}8$.
	\end{itemize}
	Vậy $P(B)=P(A)\cdot P(B \mid A)+P(\overline{A}) \cdot P(B \mid \overline{A})=0{,}55\cdot 0{,}96+0{,}35\cdot 0{,}8=0{,}808$. \\
	Vậy xác suất để máy bay bị bắn hạ là $0{,}808$.
	}
\end{bt}
%%==========Bài 2
\begin{bt}%%%%[2D5H2-2]
	Có hai chuồng thỏ. Chuồng I có 5 con thỏ đen và 10 con thỏ trắng. Chuồng II có 7 con thỏ đen và 3 con thỏ trắng. Trước tiên, từ chuồng II lấy ra ngẫu nhiên 1 con thỏ rồi cho vào chuồng I. Sau đó, từ chuồng I lấy ra ngẫu nhiên 1 con thỏ. Tính xác suất để con thỏ được lấy ra là con thỏ trắng.
	\loigiai{Xét biến cố $A$: ``Con thỏ được lấy ra từ chuồng II để cho vào chuồng I là con thỏ trắng''. 	Xét biến cố $B$: ``Con thỏ được lấy ra từ chuồng I là con thỏ trắng''. \\	
	Ta có $P(B)=P(A)\cdot P(B \mid A)+P(\overline{A}) \cdot P(B \mid \overline{A})$.
	\begin{itemize}
	\item Tính $P(A)$: Đây là xác suất để lấy ra ngẫu nhiên 1 con thỏ trắng từ chuồng II rồi cho vào chuồng I. Có $n\left(\Omega\right)=\mathrm{C}^1_{10}$, $n\left(A\right)=\mathrm{C}^1_3$. Vậy $P(A)=\dfrac{3}{10}$.
	\item Tính $P(\overline{A})$: $P(\overline{A})=1-P(A)=\dfrac{7}{10}$.
	\item Tính $P(B\mid A)$: Đây là xác suất để lấy ra ngẫu nhiên 1 con thỏ trắng từ chuồng I với điều kiện đã chọn ra 1 con thỏ trắng từ chuồng II rồi cho vào chuồng I, tức là có 5 con thỏ đen và 11 con thỏ trắng ở trong chuồng I. Tương tự như trên ta có $P(B\mid A)=\dfrac{11}{16}$.
	\item Tính $P(B\mid \overline{A})$: Đây là để lấy ra ngẫu nhiên 1 con thỏ trắng từ chuồng I với điều kiện đã chọn ra 1 con thỏ đen từ chuồng II rồi cho vào chuồng I, tức là có 6 con thỏ đen và 10 con thỏ trắng ở trong chuồng I. Tương tự như trên ta có $P(B\mid \overline{A})=\dfrac{10}{16}$.
	\end{itemize}
	Vậy $P(B)=P(A)\cdot P(B \mid A)+P(\overline{A}) \cdot P(B \mid \overline{A})=\dfrac{3}{10}\cdot \dfrac{11}{16}+\dfrac{7}{10}\cdot \dfrac{10}{16}=\dfrac{103}{160}=0{,}64375$. \\
	Vậy xác suất để con thỏ được lấy ra là con thỏ trắng là $0{,}64375$.
	}
\end{bt}
%%==========Bài 3
\begin{bt}%%%%[2D5V2-3]
	Một bộ lọc được sử dụng để chặn thư rác trong các tài khoản thư điện tử. Tuy nhiên, vì bộ lọc không tuyệt đối hoàn hảo nên một thư rác bị chặn với xác suất 0{,}95 và một thư đúng (không phải là thư rác) bị chặn với xác suất 0{,}01. Thống kê cho thấy tỉ lệ thư rác là $3 \%$.
	\begin{listEX}
	\item Chọn ngẫu nhiên một thư bị chặn. Tính xác suất để đó là thư rác.
	\item Chọn ngẫu nhiên một thư không bị chặn. Tính xác suất để đó là thư đúng.
	\item Trong số các thư bị chặn, có bao nhiêu phần trăm là thư đúng? Trong số các thư không bị chặn, có bao nhiêu phần trăm là thư rác?
	\end{listEX}
	\loigiai{Xét biến cố $A$: ``Thư đó là thư rác''. Xét biến cố $B$: ``Thư đó bị chặn''.
	\begin{listEX}
	\item Ta cần tính $P(A\mid B)$.
	Áp dụng công thức Bayes
	$$P(A\mid B) = \dfrac{P(A)\cdot P(B\mid A)}{P(A)\cdot P(B\mid A) + P(\overline{A})\cdot P(B \mid \overline{A})}.$$
	\begin{itemize}
	\item Tính $P(A)$: Đây là xác suất thư đó là thư rác. Vậy $P(A)=0{,}03$.
	\item Tính $P(\overline{A})$: $P(\overline{A})=1-P(A)=0{,}97$.
	\item Tính $P(B \mid A)$: Đây là xác suất thư đó là thư rác bị chặn. Vậy $P(B \mid A)=0{,}95$.
	\item Tính $P(B \mid \overline{A})$: Đây là xác suất thư đó là thư đúng bị chặn. Vậy $P(B \mid \overline{A})=0{,}01$.
	\end{itemize}
	Vậy $P(A\mid B) = \dfrac{P(A)\cdot P(B\mid A)}{P(A)\cdot P(B\mid A) + P(\overline{A})\cdot P(B \mid \overline{A})}=\dfrac{0{,}03\cdot 0{,}95}{0{,}03\cdot 0{,}95+0{,}97\cdot 0{,}01}=\dfrac{285}{382}\approx 0{,}746$.\\
	Vậy xác suất để chọn ngẫu nhiên một thư bị chặn mà thư đó là thư rác là khoảng $0{,}746$.
	\item Ta cần tính $P(\overline{A}\mid \overline{B})$.
	Áp dụng công thức Bayes
	$$P(\overline{A}\mid \overline{B}) = \dfrac{P(\overline{A})\cdot P(\overline{B}\mid \overline{A})}{P(\overline{A})\cdot P(\overline{B}\mid \overline{A}) + P(A)\cdot P(\overline{B} \mid A)}.$$
	\begin{itemize}
	\item Tính $P(\overline{B} \mid A)$: Ta có $(\overline{B} \mid A)=1-P(B \mid A)=0{,}05$.
	\item Tính $P(\overline{B} \mid \overline{A})$: Ta có $P(\overline{B} \mid \overline{A})=1-P(B \mid \overline{A})=0{,}99$.
	\end{itemize}
	Vậy $P(\overline{A}\mid \overline{B}) = \dfrac{P(\overline{A})\cdot P(\overline{B}\mid \overline{A})}{P(\overline{A})\cdot P(\overline{B}\mid \overline{A}) + P(A)\cdot P(\overline{B} \mid A)}=\dfrac{0{,}97\cdot 0{,}99}{0{,}97\cdot 0{,}99+0{,}03\cdot 0{,}05}=\dfrac{3201}{3206}\approx 0{,}998$.\\
	Vậy xác suất để chọn ngẫu nhiên một thư không bị chặn mà thư đó là thư đúng là khoảng $0{,}998$.
	\item Trong số các thư bị chặn, có $74{,}6 \%$ là thư rác, có $25{,}4 \%$ là thư đúng. \\
	Trong số các thư không bị chặn, có $0{,}2 \%$ là thư rác, có $99{,}8 \%$ là thư đúng.
	\end{listEX}
	}
\end{bt}
%%==========Bài 4
\begin{bt}%%%%[2D5N2-2]
	Trong một trường học, tỉ lệ học sinh nữ là $52\%$. Tỉ lệ học sinh nữ và tỉ lệ học sinh nam
	tham gia câu lạc bộ nghệ thuật lần lượt là $18\%$ và $15\%$. Gặp ngẫu nhiên 1 học sinh của trường. 
	\begin{listEX}
	\item Tính xác suất học sinh đó có tham gia câu lạc bộ nghệ thuật. 
	\item Biết rằng học sinh có tham gia câu lạc bộ nghệ thuật. Tính xác suất học sinh đó là nam.
	\end{listEX}
	\loigiai{
	\begin{listEX}
	\item Gọi $A$ là biến cố "Học sinh đó là nữ" và $B$ là biến cố "Học sinh đó tham gia câu lạc bộ nghệ thuật".\\
	Do tỉ lệ học sinh nữ là $52\%$ nên
	\begin{center}
	$P(A)=0{,}52$ và $P(\overline{A})=1-0{,}52=0{,}48$.
	\end{center}
	Do tỉ lệ học sinh nữ và tỉ lệ học sinh nam tham gia câu lạc bộ nghệ thuật lần lượt là $18\%$ và $15\%$ nên
	\begin{center}
	$P(B|A)=0{,}18$ và $P(B|\overline{A})=0{,}15$.
	\end{center}
	Xác suất để học sinh đó có tham gia câu lạc bộ nghệ thuật là
	$$P(B)=P(A)P(B|A)+P(\overline{A})P(B|\overline{A})=0{,}52\cdot0{,}18+0{,}48\cdot0{,}15=0{,}1656.$$
	\item Do học sinh có tham gia câu lạc bộ nghệ thuật nên xác suất học sinh đó là nam là
	$$P(\overline{A}|B)=\dfrac{P(\overline{A})P(B|\overline{A})}{P(B)}=\dfrac{0{,}48\cdot0{,}15}{0{,}1656}=\dfrac{10}{23}.$$
	\end{listEX}}
\end{bt}
%%==========Bài 5
\begin{bt}%%%%[2D5Y2-3]
	Tỉ lệ người dân đã tiêm vắc xin phòng bệnh A ở một địa phương là $65\%$. Trong số những 
	người đã tiêm phòng, tỉ lệ mắc bệnh A là $5\%$ còn trong số những người chưa tiêm, tỉ lệ 
	mắc bệnh A là $17\%$. Gặp ngẫu nhiên một người ở địa phương đó.
	\begin{listEX}
	\item Tính xác suất người đó mắc bệnh A.
	\item Biết rằng người đó mắc bệnh A. Tính xác suất người đó không tiêm vắc xin phòng bệnh A.
	\end{listEX}
	\loigiai{Gọi $H_1$ là biến cố "Gặp được người đã tiêm vắc xin phòng bệnh $A$", $H_2$ là biến cố "Gặp được người chưa tiêm vắc xin phòng bệnh $A$", $A$ là biến cố" người đó mắc bệnh A".
	\begin{listEX}
	\item Theo công thức Bayes, ta có:
	$$\mathrm{P}(A)=\mathrm{P}(H_1).\mathrm{P}(A|H_1)+\mathrm{P}(H_2).\mathrm{P}(A|H_2)$$
	$$ \Leftrightarrow \mathrm{P}(A)= 0,65.0,05+0,35.0,17=0,092. $$
	\item $$\mathrm{P}(H_2|A)=\dfrac{\mathrm{P}(AH_2)}{\mathrm{P}(A)}=\dfrac{\mathrm
	P(H_2)\mathrm{P}(A|H_2)}{\mathrm{P}(A)}$$
	$$\Leftrightarrow \mathrm{P}(H_2|A)= \dfrac{0,35.0,17}{0,092}=\dfrac{119}{184}\approx 0,65.
	$$
	\end{listEX}}
\end{bt}
%%==========Bài 6
\begin{bt}%%%%[2D5N2-3]
	Ở một khu rừng nọ có 7 chú lùn, trong đó có 4 chú luôn nói thật, 3 chú còn lại nói thật với 
	xác suất 0,5. Bạn Tuyết gặp ngẫu nhiên một chú lùn. Gọi $A$ là biến cố “Chú lùn đó luôn
	nói thật” và $B$ là biến cố “Chú lùn đó tự nhận mình luôn nói thật”.
	\begin{listEX}
	\item Tính xác suất của các biến cố $A$ và $B$.
	\item Biết rằng chú lùn mà bạn Tuyết gặp tự nhận mình là người luôn nói thật. Tính xác suất để 
	chú lùn đó luôn nói thật.
	\end{listEX}
	\loigiai{Gọi $C$ là biến cố "bạn Tuyết gặp được chú lùn nói thật với xác suất $0,5$".
	\begin{listEX}
	\item Ta có
	$\mathrm{P}(A)=\dfrac{4}{7}$
	Theo công thức Bayes, ta có
	$\mathrm{P}(B)=\mathrm{P}(A).\mathrm{P}(B|A)+\mathrm{P}(C).\mathrm{P}(B|C)$
	$\Leftrightarrow \mathrm{P}(B)=\dfrac{4}{7}.1+\dfrac{3}{7}.0,5=\dfrac{11}{14}\approx 0,79 $
	\item $$\mathrm{P}(A|B)=\dfrac{\mathrm{P}(AB)}{\mathrm{P(B)}}=\dfrac{\mathrm{P}(A).\mathrm{P}(B|A)}{\mathrm{P}(B)}=\dfrac{\dfrac{4}{7}.1}{\dfrac{11}{14}}=\dfrac{8}{11}\approx 0,73.$$
	\end{listEX}}
\end{bt}
%%==========Bài 7
\begin{bt}%%%
	Giả sử có khoảng $40 \%$ thư điện tử (email) gửi đến một địa chỉ là thư rác. Người ta sử dụng một thuật toán để phân loại thư rác, biết rằng thuật toán này có thể phân loại đến $99 \%$ thư rác và tỉ lệ sai sót khi phân loại thư bình thường thành thư rác là $5 \%$. Tính xác suất một thư điện tử là thư bình thường nếu thư này đã được phân loại đúng.
	\loigiai{
	Ta có công thức
	$$	P(A | B)=\dfrac{P(B | A) \cdot P(A)}{P(B)}$$
	Trong đó
	\begin{itemize}
	\item $A$: Thư điện tử là thư bình thường.
	\item $B$: Thư đã được phân loại đúng.
	\item 	$\mathrm{P}(A)$: Xác suất một thư điện tử là thư bình thường ban đầu.\\
	Vì có $40 \%$ thư rác, nên $\mathrm{P}(A)=$ $1-0{,}4=0{,}6$.
	\item $\mathrm{P}(B | A)$: Xác suất một thư bình thường được phân loại đúng.\\
	Do tỉ lệ sai sót là $5 \%$, nên $\mathrm{P}(B | A)=1-0{,}05=0{,}95$.
	\item $P(B)$: Xác suất một thư nào đó được phân loại đúng, tính bằng tổng xác suất một thư rác được phân loại đúng và xác suất một thư bình thường được phân loại đúng
	\end{itemize}
	$$\begin{aligned}[t]
	\mathrm{P}(B)&=\mathrm{P}(B | A) \cdot \mathrm{P}(A)+\mathrm{P}(B |\overline{A}) \cdot \mathrm{P}(\overline{A}) \\&
	=0{,}95 \cdot 0{,}6+0{,}99 \cdot 0{,}4=0{,}97.
	\end{aligned}$$
	Áp dụng định lý Bayes: $\mathrm{P}(A | B)=\dfrac{\mathrm{P}(B | A) \cdot \mathrm{P}(A)}{\mathrm{P}(B)}=\dfrac{0{,}95 \cdot 0{,}6}{0{,}97}=	\dfrac{57}{97}$.
	}
\end{bt}
%%==========Bài 8
\begin{bt}%%%
	Một chiếc hộp có $20$ chiếc thẻ cùng loại, trong đó có $2$ chiếc thẻ màu xanh và $18$ chiếc thẻ màu trắng. Bạn Châu rút thẻ hai lần một cách ngẫu nhiên, mỗi lần rút một thẻ và thẻ được rút ra không bỏ lại hộp. Tính xác suất để cả hai lần bạn Châu đều rút được thẻ màu xanh.
	\loigiai{
	Xét hai biến cố\\
	$A$: \lq\lq  Thẻ thứ nhất rút được màu xanh\rq\rq.\\
	$B$: \lq\lq  Thẻ thứ hai rút được màu xanh\rq\rq.\\
	Khi đó, ta có xác suất để cả hai lần bạn Châu đều rút được thẻ màu xanh là \[\mathrm{P}(A\cap B)=\dfrac{C_2^2}{C_{20}^2}=\dfrac{1}{190}.\]
	}
\end{bt}
% \subsection{Bài tập trắc nghiệm}
% \BTTN
% \PHIEUTRACNGHIEM
\Opensolutionfile{ans}[ans/ans-2-B19]
\begin{ex}%[2D6N2-1]Câu 1
Cho hai biến cố $A$ và $B$ với $0<P(B)<1$. Khi đó công thức xác suất toàn phần cho biến cố $A$ là
	\choice
	{\True $\mathrm{P}(A)=\mathrm{P}(B)\mathrm{P}(A|B)+\mathrm{P}(\overline{B})\mathrm{P}(A|\overline{B})$}
	{$\mathrm{P}(A)=\mathrm{P}(A)\mathrm{P}(A|B)+\mathrm{P}(\overline{A})\mathrm{P}(A|\overline{B})$}
	{$\mathrm{P}(A)=\mathrm{P}(\overline{B})\mathrm{P}(A|B)+\mathrm{P}(B)\mathrm{P}(A|\overline{B})$}
	{$\mathrm{P}(B)=\mathrm{P}(\overline{B})\mathrm{P}(A|B)+\mathrm{P}(B)\mathrm{P}(B|\overline{B})$}
	\loigiai{Cho hai biến cố $A$ và $B$ với $0<P(B)<1$. Khi đó 
	$\mathrm{P}(A)=\mathrm{P}(B)P(A|B)+\mathrm{P}(\overline{B})\mathrm{P}(A|\overline{B})$ 
	gọi là {\bf công thức xác suất toàn phần}.
	}
\end{ex}

\begin{ex}%[2D6H2-1]Câu 2
Cho hai biến cố $A=A_1+A_2$ và biến cố $B=B_1+B_2$ biểu diễn theo đồ ven như sau
\begin{center}
\tikzset{every picture/.style={line width=0.75pt}} %set default line width to 0.75pt 
\begin{tikzpicture}[x=0.75pt,y=0.75pt,yscale=-1,xscale=1]
	%uncomment if require: \path (0,300); %set diagram left start at 0, and has height of 300
	%Shape: Rectangle [id:dp41068169978107316] 
	\draw [fill=red!30] (94,104) -- (354,104) -- (354,176) -- (94,176) -- cycle ;
	%Shape: Ellipse [id:dp24667066386593728] 
	\draw [fill=green!30] (144,140) .. controls (144,128.95) and (176.24,120) .. (216,120) .. controls (255.76,120) and (288,128.95) .. (288,140) .. controls (288,151.05) and (255.76,160) .. (216,160) .. controls (176.24,160) and (144,151.05) .. (144,140) -- cycle ;
	%Straight Lines [id:da2324528203840277] 
	\draw (192,104) -- (192,176) ;
	% Text Node
	\draw (302,115) node [anchor=north west][inner sep=0.75pt] [align=left] {A\textsubscript{2}};
	% Text Node
	\draw (121,114) node [anchor=north west][inner sep=0.75pt] [align=left] {A\textsubscript{1}};
	% Text Node
	\draw (169,131) node [anchor=north west][inner sep=0.75pt] [align=left] {B\textsubscript{1}};
	% Text Node
	\draw (230,130) node [anchor=north west][inner sep=0.75pt] [align=left] {B\textsubscript{2}};
\end{tikzpicture}
\end{center}
Tính xác xuất của $P(A)$.
	\choice
	{$\mathrm{P}(A)=\mathrm{P}(B_1)\mathrm{P}(A_1|B_1)+\mathrm{P}(B_2)\mathrm{P}(A_1|B_2)$}
	{\True $\mathrm{P}(A)=\mathrm{P}(B_1)\mathrm{P}(A|B_1)+\mathrm{P}(B_2)\mathrm{P}(A|B_2)$}
	{$\mathrm{P}(A)=\mathrm{P}(B)\mathrm{P}(A_1|B_1)+\mathrm{P}(B)\mathrm{P}(A_2|B_2)$}
	{$\mathrm{P}(A)=\mathrm{P}(A_1)\mathrm{P}(A|B_1)+\mathrm{P}(B_2)\mathrm{P}(A|B_2)$}
	\loigiai{ 
	Cho hai biến cố $A$ và $B$ với $0<P(B)<1$. Khi đó 
	$\mathrm{P}(A)=\mathrm{P}(B)\mathrm{P}(A|B)+\mathrm{P}(\overline{B})\mathrm{P}(A|\overline{B})$ 
	gọi là {\bf công thức xác suất toàn phần}.\\
	 $\Rightarrow \mathrm{P}(A)=\mathrm{P}(B_1)\mathrm{P}(A|B_1)+\mathrm{P}(B_2)\mathrm{P}(A|B_2)$}
\end{ex}

\begin{ex}%[2D6N2-1]Câu 3
Giả sử hai biến cố $A$ và $B$ ngẫu nhiên thỏa mãn $\mathrm{P}(A)>0$ với $0<\mathrm{P}(B)<1$. Khi đó công thức {\bf Bayes} là
	\choice
	{$\mathrm{P}(A|B)=\dfrac{\mathrm{P}(B)\mathrm{P}(A|B)}{\mathrm{P}(B)\mathrm{P}(A|B)+\mathrm{P}(\overline{B})\mathrm{P}(A|\overline{B})}$}
	{$\mathrm{P}(B|A)=\dfrac{\mathrm{P}(A)P(A|B)}{\mathrm{P}(B)\mathrm{P}(A|B)+\mathrm{P}(\overline{B})\mathrm{P}(A|\overline{B})}$}
	{\True $\mathrm{P}(B|A)=\dfrac{\mathrm{P}(B)P(A|B)}{\mathrm{P}(B)\mathrm{P}(A|B)+\mathrm{P}(\overline{B})\mathrm{P}(A|\overline{B})}$}
	{$\mathrm{P}(B|A)=\dfrac{\mathrm{P}(B)\mathrm{P}(B|A)}{\mathrm{P}(B)\mathrm{P}(A|B)+\mathrm{P}(\overline{B})\mathrm{P}(A|\overline{B})}$}
	\loigiai{Giả sử hai biến cố $A$ và $B$ ngẫu nhiên thỏa mãn $P(A)>0$ với $0<P(B)<1$. Khi đó\\
	$\mathrm{P}(B|A)=\dfrac{\mathrm{P}(B)\mathrm{P}(A|B)}{\mathrm{P}(B)P(A|B)+\mathrm{P}(\overline{B})\mathrm{P}(A|\overline{B})}$
	 gọi là công thức {\bf Bayes}. 
	}
\end{ex}

\begin{ex}%[2D6N2-2]
	Cho $\mathrm{P}(A)=\dfrac{2}{5}$; $\mathrm{P}(B \mid A)=\dfrac{1}{3}$; $\mathrm{P}(B \mid \overline{A})=\dfrac{1}{4}$. Giá trị của $\mathrm{P}(B)$ là 
	\choice
	{$\dfrac{19}{60}$}
	{\True $\dfrac{17}{60}$}
	{$\dfrac{9}{20}$}
	{$\dfrac{7}{30}$}
	\loigiai{
	Áp dụng công thức xác suất toàn phần ta có
	$$
	\mathrm{P}(B)=\mathrm{P}(A) \cdot \mathrm{P}(B \mid A)+\mathrm{P}\left(\overline{A}\right) \cdot \mathrm{P}(B \mid \overline{A})=\dfrac{2}{5} \cdot \dfrac{1}{3}+\dfrac{3}{5} \cdot \dfrac{1}{4}=\dfrac{17}{60}.
	$$
	}
\end{ex}

\begin{ex}%[2D6N2-2]
	Cho hai biến cố $A$, $B$ với $\mathrm{P}(B)=0{,}6$; $\mathrm{P}(A \mid B)=0{,}7$ và $\mathrm{P}(A \mid \overline{B})=0{,}4$. Khi đó, $\mathrm{P}(A)$ bằng 
	\choice
	{$0{,}7$}
	{$0{,}4$}
	{\True $0{,}58$}
	{$0{,}52$}
	\loigiai{
	Ta có $\mathrm{P}(\overline{B})=1-\mathrm{P}(B)=1-0{,}6=0{,}4$.\\
	Áp dụng công thức xác suất toàn phần, ta có
	$$
	\mathrm{P}(A)=\mathrm{P}(A \mid B) \cdot \mathrm{P}(B)+\mathrm{P}(A \mid \overline{B}) \cdot \mathrm{P}(\overline{B})=0{,}7 \cdot 0{,}6+0{,}4 \cdot 0{,}4=0{,}58.
	$$
	}
\end{ex}

\begin{ex}%[2D6N2-1]
	Theo công thức Bayes ta có 
	\choice
	{$\mathrm{P}(A \mid B)=\dfrac{\mathrm{P}(B)\cdot \mathrm{P}(B \mid A)}{\mathrm{P}(A)}$}
	{$\mathrm{P}(B \mid A)=\dfrac{\mathrm{P}(A)\cdot \mathrm{P}(B \mid A)}{\mathrm{P}(B)}$}
	{\True $\mathrm{P}(A \mid B)=\dfrac{\mathrm{P}(A)\cdot \mathrm{P}(B \mid A)}{\mathrm{P}(B)}$}
	{$\mathrm{P}(A \mid B)=\dfrac{\mathrm{P}(B)\cdot \mathrm{P}(A \mid B)}{\mathrm{P}(A)}$}
	\loigiai{
	Theo công thức Bayes ta có $\mathrm{P}(A \mid B)=\dfrac{\mathrm{P}(A)\cdot \mathrm{P}(B \mid A)}{\mathrm{P}(B)}$
	}
\end{ex}

\begin{ex}%[2D6N2-3]
	Cho $\mathrm{P}(A)=\dfrac{4}{5}$; $\mathrm{P}(B \mid A)=\dfrac{2}{3}$; $\mathrm{P}(B \mid \overline{A})=\dfrac{1}{4}$. Giá trị của $\mathrm{P}(A \mid B)$ là 
	\choice
	{$\dfrac{33}{35}$}
	{\True $\dfrac{32}{35}$}
	{$\dfrac{9}{35}$}
	{$\dfrac{26}{35}$}
	\loigiai{
	Áp dụng công thức Bayes ta có
	$$
	\mathrm{P}(A \mid B)=\dfrac{\mathrm{P}(A) \cdot \mathrm{P}(B \mid A)}{\mathrm{P}(A) \cdot \mathrm{P}(B \mid A)+\mathrm{P}(\overline{A}) \cdot \mathrm{P}(B \mid \overline{A})}=\dfrac{\dfrac{4}{5} \cdot \dfrac{2}{3}}{\dfrac{4}{5} \cdot \dfrac{2}{3}+\dfrac{1}{5} \cdot \dfrac{1}{4}}=\dfrac{32}{35}.
	$$
	}
\end{ex}

\begin{ex}%[2D6H2-2]
	Tỉ lệ người dân đã tiêm vắc xin phòng bệnh A ở một địa phương là $65 \%$. Trong số những người đã tiêm phòng, tỉ lệ mắc bệnh A là $5 \%$ còn trong số những người chưa tiêm, tỉ lệ mắc bệnh A là $17 \%$. Gặp ngẫu nhiên một người ở địa phương đó. Xác suất người đó mắc bệnh A là 
	\choice
	{$0{,}0325$}
	{$0{,}018$}
	{\True $0{,}092$}
	{$0{,}0525$}
	\loigiai{
	Gọi $H_1$ là biến cố \lq\lq  Gặp được người đã tiêm vắc xin phòng bệnh A\rq\rq, $H_2$ là biến cố \lq\lq  Gặp được người chưa tiêm vắc xin phòng bệnh A\rq\rq, $K$ là biến cố \lq\lq  Người đó mắc bệnh A\rq\rq. \\
	Theo công thức xác suất toàn phần, ta có
	\begin{eqnarray*}
	\mathrm{P}(K)&=&\mathrm{P}\left(H_1\right) \cdot \mathrm{P}\left(K \mid H_1\right)+\mathrm{P}\left(H_2\right) \cdot \mathrm{P}\left(K \mid H_2\right) \\
	&=& 0{,}65 \cdot 0{,}05+0{,}35 \cdot 0{,}17=0{,}092.
	\end{eqnarray*}
	}
\end{ex}

\begin{ex}%[2D6H2-2]
	Một nhà máy có hai phân xưởng I và II. Phân xưởng I sản xuất $40 \%$ số sản phẩm và phân xưởng II sản xuất $60 \%$ số sản phẩm. Tỉ lệ sản phẩm bị lỗi của phần xưởng I là $2 \%$ và của phân xưởng II là $1 \%$. Kiểm tra ngẫu nhiên $1$ sản phẩm của nhà máy và xác suất để sản phẩm đó bị lỗi là 
	\choice
	{$0{,}02$}
	{$0{,}6$}
	{\True $0{,}014$}
	{$0{,}01$}
	\loigiai{
	Gọi $A$ là biến cố \lq\lq  Sản phẩm bị lỗi\rq\rq, $B$ là biến cố \lq\lq  Sản phẩm lấy ra do phân xưởng $I$ sản xuất\rq\rq.\\
	Do phân xưởng I sản xuất $40 \%$ số sản phẩm và phân xưởng II sản xuất $60 \%$ số sản phẩm nên
	$$
	\mathrm{P}(B)=0{,}4 \,\,\text{và}\,\, \mathrm{P}(\overline{B})=1-0{,}4=0{,}6.
	$$
	Do tỉ lệ sản phẩm bị lỗi của phân xưởng I là $2 \%$ và của phân xưởng II là $1 \%$ nên
	$$
	\mathrm{P}(A \mid B)=0{,}02 \,\, \text{và} \,\, \mathrm{P}(A \mid \overline{B})=1-0{,}01.
	$$
	Xác suất để sản phẩm lấy ra bị lỗi là
	$$
	\mathrm{P}(A)=\mathrm{P}(B) \mathrm{P}(A \mid B)+\mathrm{P}(\overline{B}) \mathrm{P}(A \mid \overline{B})=0{,}4 \cdot 0{,}02+0{,}6 \cdot 0{,}01=0{,}014.
	$$
	}
\end{ex}

\begin{ex}%[2D6V2-4]
	Cho $A$, $B$ là các biến cố thỏa mãn $\mathrm{P}(\overline{A}\cap\overline{B})=0{,}35$; $\mathrm{P}(A)=0{,}25$; $\mathrm{P}(B)=0{,}6$. Giá trị của $\mathrm{P}(A|B)$ bằng
	\choice
	{$\dfrac{1}{5}$}
	{\True$\dfrac{1}{3}$}
	{$\dfrac{7}{15}$}
	{$\dfrac{2}{3}$}
	\loigiai{Ta có $\mathrm{P}(\overline{A}\cap\overline{B}) = \mathrm{P}\left( \overline A \right)\mathrm{P}\left( \overline B |\overline A \right) \Rightarrow \mathrm{P}\left( \overline B |\overline A \right) = \dfrac{\mathrm{P}(\overline{A}\cap\overline{B})}{\mathrm{P}\left( \overline A \right)} = \dfrac{0{,}35}{0{,}75} = \dfrac{7}{15}.$ \\
	Suy ra $\mathrm{P}\left( B|\overline A \right) = 1 - \dfrac{7}{{15}} = \dfrac{8}{{15}}$.\\
	Theo công thức xác suất toàn phần, ta có
	\begin{eqnarray*}
	&&\mathrm{P}\left( B \right) = \mathrm{P}\left( B|A \right)\mathrm{P}\left( A \right) + \mathrm{P}\left( B|\overline A \right)\mathrm{P}\left( \overline A\right)\\
	&\Rightarrow& \mathrm{P}\left(B|A\right) = \dfrac{\mathrm{P}\left( B \right) - \mathrm{P}\left( {B|\overline A } \right)\mathrm{P}\left( {\overline A } \right)}{\mathrm{P}\left( A \right)} = \dfrac{0{,}6 - \dfrac{8}{{15}} \cdot 0{,}75}{0{,}25} = 0{,}8.
	\end{eqnarray*}
	Theo công thức Bayes, ta được
	$$\mathrm{P}\left( A|B\right) = \dfrac{\mathrm{P}\left( A \right)\mathrm{P}\left( {B|A} \right)}{\mathrm{P}\left( B \right)} = \dfrac{0{,}25 \cdot 0{,}8}{0{,}6} = \dfrac{1}{3}.$$}
\end{ex}

\begin{ex}%[2D6H2-3]
	Một bệnh viện có hai phòng khám là phòng A và phòng B với khả năng lựa chọn của bệnh nhân là như nhau. Tỉ lệ bệnh nhân nam có ở phòng A và phòng B lần lượt là $60\%$ và $40\%$. Một người bệnh được chọn ngẫu nhiêu từ hai phòng khám và biết người này là nam, xác suất để người bệnh được chọn đến từ phòng A là
	\choice
	{\True $0{,}6$}
	{$0{,}5$}
	{$0{,}4$}
	{$0{,}3$}
	\loigiai{Một người bệnh được chọn ngẫu nhiên từ hai phòng khám.\\
	Gọi $X$ là biến cố \lq \lq Người đó đến từ phòng khám A\rq \rq \, và $Y$, $\overline{Y}$ lần lượt là biến cố \lq \lq Người đó là nam\rq \rq \; và \lq \lq Người đó không là nam\rq \rq.\\
	Ta có sơ đồ hình cây sau
	\begin{center}
	\begin{tikzpicture}[>=stealth,scale=0.6]
	%Khung 1
	\draw (-2,-1) rectangle (2.2,0);
	\draw (0.1,-0.5) node{Bệnh nhân được chọn} ;
	%Mui ten 1,2
	\draw [->] (2.2,-0.5)--(3.8,1.6) node[pos=0.5,sloped,above]{$0{,}5$};
	\draw [->] (2.2,-0.5)--(3.8,-2.6) node[pos=0.5,sloped,below]{$0{,}5$};
	%Khung 2.1
	\draw (3.8,1.1) rectangle (5.1,2.1);
	\draw (8.9/2,1.6) node{$X$} ;
	%Khung 2.2
	\draw (3.8,-2.1) rectangle (5.1,-3.1);
	\draw (8.9/2,-2.6) node{$\overline{X}$} ;
	%Mui ten 3,4
	\draw [->] (5.1,1.6)--(6.5,2.6) node[pos=0.5,sloped,above]{$0{,}6$};
	\draw [->] (5.1,1.6)--(6.5,0.6) node[pos=0.5,sloped,below]{$0{,}4$};
	%Mui ten 5,6
	\draw [->] (5.1,-2.6)--(6.5,-1.6) node[pos=0.5,sloped,above]{$0{,}4$};
	\draw [->] (5.1,-2.6)--(6.5,-3.6) node[pos=0.5,sloped,below]{$0{,}6$};
	%Khung 3.1
	\draw (6.5,2.2) rectangle (7.7,3.2);
	\draw (7.1,5.4/2) node{$Y$} ;
	%Khung 3.2
	\draw (6.5,1.2) rectangle (7.7,0.2);
	\draw (7.1,1.4/2) node{$\overline{Y}$} ;
	%Khung 3.3
	\draw (6.5,-1.1) rectangle (7.7,-2.1);
	\draw (7.1,-3.2/2) node{$Y$} ;
	%Khung 3.3
	\draw (6.5,-2.9) rectangle (7.7,-3.9);
	\draw (7.1,-3.4) node{$\overline{Y}$} ;
	%Kết quả
	\draw (9.5,3.7) node{\textbf{Kết quả}};	
	\draw (9.5,2.7) node{$XY$};
	\draw (9.5,0.7) node{$X \overline{Y}$};
	\draw (9.5,-1.6) node{$\overline{X}Y$};
	\draw (9.5,-3.4) node{$\overline{X}\overline{Y}$};
	%Xác suất
	\draw (12.5,3.7) node{\textbf{Xác suất}};	
	\draw (12.5,2.7) node{$0{,}3$};
	\draw (12.5,0.7) node{$0{,}2$};
	\draw (12.5,-1.6) node{$0{,}2$};
	\draw (12.5,-3.4) node{$0{,}3$};	
	\end{tikzpicture}
	\end{center}
	Theo công thức Bayes, ta có $$\mathrm{P}(X|Y)=\dfrac{\mathrm{P}(X)\mathrm{P}(Y|X)}{\mathrm{P}(X)\mathrm{P}(Y|X)+\mathrm{P}(\overline{X})\mathrm{P}(Y|\overline{X})}=\dfrac{0{,}3}{0{,}3+0{,}2}=0{,}6.$$
	Vậy với một người bệnh được chọn ngẫu nhiêu từ hai phòng khám và biết người này là nam, xác suất để người đó đến từ phòng A là $0{,}6$.}
\end{ex}

\begin{ex}%[2D6V2-3]
	Một bệnh viện đang xét nghiệm cho một số bệnh nhân để xác định liệu họ có nhiễm virus $X$ hay không. Xác suất để một bệnh nhân bị nhiễm virus $X$ là $0{,}05$. Khi xét nghiệm, nếu một bệnh nhân bị nhiễm thì xác suất để kết quả xét nghiệm dương tính là $0{,}95$. Nếu một bệnh nhân không bị nhiễm thì xác suất để kết quả xét nghiệm âm tính là $0{,}98$. Một bệnh nhân được chọn ngẫu nhiên và có kết quả xét nghiệm dương tính. Xác suất để bệnh nhân đó thực sự bị nhiễm virus $X$ là
	\choice
	{$\dfrac{133}{2000}$}
	{$\dfrac{19}{400}$}
	{\True $\dfrac{5}{7}$}
	{$\dfrac{2}{7}$}
	\loigiai{Một bệnh nhân đến một bệnh viên để xét nghiệm.\\
	Gọi $A$ là biến cố \lq \lq Bệnh nhân bị nhiễm virus $X$\rq \rq \, và $B$, $\overline{B}$ lần lượt là biến cố \lq \lq Kết quả xét nghiệm dương tính\rq \rq \; và \lq \lq Kết quả xét nghiệm âm tính\rq \rq.\\
	Ta xét sơ đồ hình cây như sau
	\begin{center}
	\begin{tikzpicture}[>=stealth,scale=0.6]
	%Khung 1
	\draw (-3.5,-1) rectangle (2.2,0);
	\draw (-1.3/2,-0.5) node{Bệnh nhân được xét nghiệm} ;
	%Mui ten 1,2
	\draw [->] (2.2,-0.5)--(3.8,1.6) node[pos=0.5,sloped,above]{$0{,}05$};
	\draw [->] (2.2,-0.5)--(3.8,-2.6) node[pos=0.5,sloped,below]{$0{,}95$};
	%Khung 2.1
	\draw (3.8,1.1) rectangle (5.1,2.1);
	\draw (8.9/2,1.6) node{$A$} ;
	%Khung 2.2
	\draw (3.8,-2.1) rectangle (5.1,-3.1);
	\draw (8.9/2,-2.6) node{$\overline{A}$} ;
	%Mui ten 3,4
	\draw [->] (5.1,1.6)--(6.5,2.6) node[pos=0.5,sloped,above]{$0{,}95$};
	\draw [->] (5.1,1.6)--(6.5,0.6) node[pos=0.5,sloped,below]{$0{,}05$};
	%Mui ten 5,6
	\draw [->] (5.1,-2.6)--(6.5,-1.6) node[pos=0.5,sloped,above]{$0{,}02$};
	\draw [->] (5.1,-2.6)--(6.5,-3.6) node[pos=0.5,sloped,below]{$0{,}98$};
	%Khung 3.1
	\draw (6.5,2.2) rectangle (7.7,3.2);
	\draw (7.1,5.4/2) node{$B$} ;
	%Khung 3.2
	\draw (6.5,1.2) rectangle (7.7,0.2);
	\draw (7.1,1.4/2) node{$\overline{B}$} ;
	%Khung 3.3
	\draw (6.5,-1.1) rectangle (7.7,-2.1);
	\draw (7.1,-3.2/2) node{$B$} ;
	%Khung 3.3
	\draw (6.5,-2.9) rectangle (7.7,-3.9);
	\draw (7.1,-3.4) node{$\overline{B}$} ;
	%Kết quả
	\draw (9.5,3.7) node{\textbf{Kết quả}};	
	\draw (9.5,2.7) node{$AB$};
	\draw (9.5,0.7) node{$A\overline{B}$};
	\draw (9.5,-1.6) node{$\overline{A}B$};
	\draw (9.5,-3.4) node{$\overline{A}\cap\overline{B}$};
	%Xác suất
	\draw (12.5,3.7) node{\textbf{Xác suất}};	
	\draw (12.5,2.7) node{$0{,}0475$};
	\draw (12.5,0.7) node{$0{,}0025$};
	\draw (12.5,-1.6) node{$0{,}019$};
	\draw (12.5,-3.4) node{$0{,}931$};
	\end{tikzpicture}
	\end{center}
	Theo công thức Bayes, ta có $$\mathrm{P}(A|B)=\dfrac{\mathrm{P}(A)\mathrm{P}(B|A)}{\mathrm{P}(A)\mathrm{P}(B|A)+\mathrm{P}(\overline{A})\mathrm{P}(B|\overline{A})}=\dfrac{0{,}0475}{0{,}0475+0{,}019}=\dfrac{5}{7}.$$	 
	Vậy với một bệnh nhân có kết quả xét nghiệm dương tính, xác suất để bệnh nhân đó thực sự bị nhiễm virus $X$ là $\dfrac{5}{7}$.}
\end{ex}

\begin{ex}%[2D6V2-2]
	Kết quả khảo sát tại một xã cho thấy có $20 \%$ cư dân hút thuốc lá. Tỉ lệ cư dân thường xuyên gặp các vấn đề sức khoẻ về đường hô hấp trong số những người hút thuốc lá và không hút thuốc lá lần lượt là $70 \%$, $15 \%$. Tỉ lệ gặp một cư dân của xã thì xác suất người đó thường xuyên gặp các vấn đề sức khoẻ về đường hô hấp là bao nhiêu phần trăm? 
	\choice
	{\True $26 \%$}
	{$12 \%$}
	{$68 \%$}
	{$24 \%$}
	\loigiai{
	Giả sử ta gặp một cư dân của xã, gọi $A$ là biến cố \lq\lq  Người đó có hút thuốc lá\rq\rq\, và $B$ là biến cố \lq\lq  Người đó thường xuyên gặp các vấn đề sức khoẻ về đường hô hấp\rq\rq. Ta có sơ đồ hình cây sau.
	\begin{center}
	\begin{tikzpicture}[scale=.3,>=stealth,every node/.style={shape=rectangle,draw,rounded corners, color=blue, fill=blue!10}]
	%-------------
	\tikzstyle{block} = [rectangle, draw, fill=blue!10, rounded corners, text centered, text width = 10em, minimum height = 2em]
	%-------------
	\node (c1) {Gặp một cư dân};
	\node (c2) [above right = 2cm of c1]{$A$};
	\node (c3) [ below right= 2cm of c1]{$\overline{A}$};
	\node (c4) [above right = 1cm of c2]{$B$} ;
	\node (c5) [below right = 1cm of c2]{$\overline{B}$};
	\node (c6) [ above right =1cm of c3]{$B$};
	\node (c7) [ below right = 1cm of c3]{$\overline{B}$};
	%--------------
	\draw 
	(17.5,15) node[right] {\text{Kết quả}} 
	(20,11.3) node[right] {$AB$} 
	(20,2.5) node[right] {$A\overline{B}$} 
	(20,-2.5) node[right] {$\overline{A}B$} 
	(20,-11.5) node[right] {$\overline{A}\cap \overline{B}$};
	%--------------
	\draw 
	(25,15) node[right] {\text{Xác suất}} 
	(25,11.3) node[right] {$0{,}14$} 
	(25,2.5) node[right] {$0{,}06$} 
	(25,-2.5) node[right] {$0{,}12$} 
	(25,-11.5) node[right] {$0{,}68$};
	%------------
	\draw[->] (c1.east) --node[above left]{$0{,}2$} (c2.west);
	\draw[->] (c1.east) --node[below left]{$0{,}8$} (c3.west);
	\draw[->] (c2.east) --node[above left]{$0{,}7$} (c4.west);
	\draw[->] (c2.east) --node[below left]{$0{,}3$} (c5.west);
	\draw[->] (c3.east) --node[above left]{$0{,}15$} (c6.west);
	\draw[->] (c3.east) -- node[below left]{$0{,}85$} (c7.west);
	\end{tikzpicture}
	\end{center}
	Ta có $\mathrm{P}(B)=\mathrm{P}(A) \cdot \mathrm{P}(B | A)+\mathrm{P}(\overline{A}) \cdot \mathrm{P}\left(B | \overline{A}\right)=0{,}14+0{,}12=0{,}26$.\\
	Vậy nếu ta gặp một cư dân của xã thì xác suất người đó thường xuyên gặp các vấn đề sức khoẻ về đường hô hấp là $26\%$.
	}
\end{ex}

\begin{ex}%[2D6V2-3]
	Ở một địa phương $X$, xác suất để một người lớn trên $40$ tuổi mắc bệnh ung thư là $0{,}05$. Xác suất bác sĩ chẩn đoán đúng một người mắc bệnh ung thư là $0{,}78$ và chẩn đoán sai (không bị ung thư nhưng được chẩn đoán mắc bệnh) là $0{,}06$. Xác suất để một người thật sự mắc bệnh ung thư khi nhận được kết quả chẩn đoán bị ung thư bằng
	\choice
	{\True$0{,}40625$}
	{$0{,}096$}
	{$0{,}904$}
	{$0{,}59375$}
	\loigiai{Một bệnh nhân trên 40 tuổi ở địa phương X đến bác sĩ để khám bệnh ung thư.\\
	Gọi $A$ là biến cố \lq \lq Người đó mắc bệnh ung thư\rq \rq \, và $B$, $\overline{B}$ lần lượt là biến cố \lq \lq Bác sĩ chẩn đoán người đó bị ung thư\rq \rq \;và \lq \lq Bác sĩ chẩn đoán người đó không bị ung thư\rq \rq.\\
	Ta xét sơ đồ hình cây như sau
	\begin{center}
	\begin{tikzpicture}[>=stealth,scale=0.7]
	%Khung 1
	\draw (-3.5,-1) rectangle (2.2,0);
	\draw (-1.3/2,-0.5) node{Bệnh nhân được chẩn đoán} ;
	%Mui ten 1,2
	\draw [->] (2.2,-0.5)--(3.8,1.6) node[pos=0.5,sloped,above]{$0{,}05$};
	\draw [->] (2.2,-0.5)--(3.8,-2.6) node[pos=0.5,sloped,below]{$0{,}95$};
	%Khung 2.1
	\draw (3.8,1.1) rectangle (5.1,2.1);
	\draw (8.9/2,1.6) node{$A$} ;
	%Khung 2.2
	\draw (3.8,-2.1) rectangle (5.1,-3.1);
	\draw (8.9/2,-2.6) node{$\overline{A}$} ;
	%Mui ten 3,4
	\draw [->] (5.1,1.6)--(6.5,2.6) node[pos=0.5,sloped,above]{$0{,}78$};
	\draw [->] (5.1,1.6)--(6.5,0.6) node[pos=0.5,sloped,below]{$0{,}22$};
	%Mui ten 5,6
	\draw [->] (5.1,-2.6)--(6.5,-1.6) node[pos=0.5,sloped,above]{$0{,}06$};
	\draw [->] (5.1,-2.6)--(6.5,-3.6) node[pos=0.5,sloped,below]{$0{,}94$};
	%Khung 3.1
	\draw (6.5,2.2) rectangle (7.7,3.2);
	\draw (7.1,5.4/2) node{$B$} ;
	%Khung 3.2
	\draw (6.5,1.2) rectangle (7.7,0.2);
	\draw (7.1,1.4/2) node{$\overline{B}$} ;
	%Khung 3.3
	\draw (6.5,-1.1) rectangle (7.7,-2.1);
	\draw (7.1,-3.2/2) node{$B$} ;
	%Khung 3.3
	\draw (6.5,-2.9) rectangle (7.7,-3.9);
	\draw (7.1,-3.4) node{$\overline{B}$} ;
	%Kết quả
	\draw (9.5,3.7) node{\textbf{Kết quả}};	
	\draw (9.5,2.7) node{$AB$};
	\draw (9.5,0.7) node{$A\overline{B}$};
	\draw (9.5,-1.6) node{$\overline{A}B$};
	\draw (9.5,-3.4) node{$\overline{A}\cap\overline{B}$};
	%Xác suất
	\draw (12.5,3.7) node{\textbf{Xác suất}};	
	\draw (12.5,2.7) node{$0{,}039$};
	\draw (12.5,0.7) node{$0{,}011$};
	\draw (12.5,-1.6) node{$0{,}057$};
	\draw (12.5,-3.4) node{$0{,}893$};
	\end{tikzpicture}
	\end{center}
	Theo công thức Bayes, ta có $$\mathrm{P}(A|B)=\dfrac{\mathrm{P}(A)\mathrm{P}(B|A)}{\mathrm{P}(A)\mathrm{P}(B|A)+\mathrm{P}(\overline{A})\mathrm{P}(B|\overline{A})}=\dfrac{0{,}039}{0{,}039+0{,}057}=0{,}40625.$$	 
	Vậy xác suất để một người thật sự mắc bệnh ung thư khi nhận được kết quả chẩn đoán bị ung thư bằng $0{,}40625$.}
\end{ex}

\begin{ex}%[2D6V2-2]
	Trung tâm kiểm soát và phòng ngừa dịch bệnh Hoa Kỳ (Centers for Disease Control and Prevention, viết tắt là (CDC) thống kê vào thời điểm năm $2020 - 2021$ về số lượng sốc phản vệ sau khi tiêm vaccine ở một số nơi tại Hoa Kỳ và châu Âu như sau: Trong $360{,}19$ triệu liều vaccine $P$ được sử dụng có $581$ ca sốc phản vệ (có khả năng gây tử vong) và $4\,259$ ca phản ứng phụ (không sốc phản vệ, không gây tử vong); trong $67{,}72$ triệu liều vaccine $A$ được sử dụng có $195$ ca sốc phản vệ và $1\,118$ ca phản ứng phụ.\\
	\textit{(Nguồn: https://www.ncbi.nlm.nih.gov/pmc/articles/PMC8626274/)}
	\choice
	{$1{,}9 \cdot 10^{-6}$}
	{$2{,}8 \cdot 10^{-6}$}
	{\True $1{,}81 \cdot 10^{-6}$}
	{$2{,}81 \cdot 10^{-6}$}
	\loigiai{
	Xét ngẫu nhiên một người trong số được thống kê ở trên. Tính xác suất để người đó thuộc trường hợp sốc phản vệ (có khả năng gây tử vong).\\
	Gọi $X$ là biến cố “Người được chọn tiêm vaccine $P$”, khi đó $\overline X $ là biến cố “Người được chọn tiêm vaccine $A$”.\\
	$Y$ là biến cố “Người được chọn thuộc trường hợp sốc phản vệ”.\\
	Khi đó, xác suất chọn được người tiêm vaccine $P$ là $\mathrm{P}(X)=\dfrac{360{,}19 \cdot 10^6}{360{,}19 \cdot 10^6+67{,}72 \cdot 10^6}$.\\
	Xác suất chọn được người tiêm vaccine $A$ là $\mathrm{P}(\overline X)=\dfrac{67{,}72 \cdot 10^6}{360{,}19 \cdot 10^6+67{,}72 \cdot 10^6}$.\\
	Xác suất chọn được người bị sốc phản vệ, nếu người đó tiêm vaccine $P$ là $\mathrm{P}(Y|X)=\dfrac{581}{360{,}19 \cdot 10^6}$.\\
	Xác suất chọn được người bị sốc phản vệ, nếu người đó tiêm vaccine $A$ là $\mathrm{P}(Y|\overline X)=\dfrac{195}{67{,}72 \cdot 10^6}$.\\
	Áp dụng công thức tính xác suất toàn phần, ta có:
	$$\mathrm{P}(Y)=\mathrm{P}(X)\cdot \mathrm{P}(Y|X)+\mathrm{P}(\overline{X})\cdot \mathrm{P}(Y|\overline{X}) \approx 1{,}81 \cdot 10^{-6}.$$
	}
\end{ex}

\begin{ex}%[2D6V2-2]Câu 4
	Trên bàn có hai hộp $B_1$ và $B_2$ đều đựng đá cẩm thạch. Hộp $B_1$ chứa 7 viên xanh và 4 viên trắng. Hộp $B_2$ chứa 3 viên xanh và 10 viên vàng. Các hộp được sắp xếp sao cho xác suất chọn hộp $B_1$ là $\dfrac{1}{3}$ và xác suất chọn hộp $B_2$ là $\dfrac{2}{3}$. Kathy bị bịt mắt và dược yêu cầu chọn một viên đá cẩm thạch. Cô ấy sẽ được thưởng một chiếc TV nếu chọn được một viên màu xanh. Xác suất cô ấy được thưởng một chiếc TV màu xanh là bao nhiêu
	\choice
	{\True $\dfrac{157}{429}$}
	{$\dfrac{158}{429}$}
	{$\dfrac{159}{429}$}
	{$\dfrac{59}{429}$}
	\loigiai{Gọi $A$ là biến cố \lq\lq  Kathy được thưởng một chiếc TV\rq\rq.\\
	Gọi $B_j$ là biến cố \lq\lq  hộp $B_j$ được chọn\rq\rq, $j=1,2$.\\
	Xác suất $\mathrm{P}(B_1)=\dfrac{1}{3}, P(B_2)=\dfrac{2}{3}$.\\
	Xác suất cần tìm là \\ $\mathrm{P}(A)=\mathrm{P}(B_1)\cdot \mathrm{P}(A|B_1)+\mathrm{P}(B_2)\cdot \mathrm{P}(A|B_2)=\dfrac{1}{3}\cdot\dfrac{7}{11}+\dfrac{2}{3}\cdot\dfrac{3}{13}=\dfrac{157}{429}$.
	}
\end{ex}

\begin{ex}%[2D6V2-2]Câu 5
Có $0{,}5\%$ dân số mắc bệnh $X$. Có một xét nghiệm để phát hiện bệnh $X$. Đối với những người mắc bệnh $X$, Xác suất xét nghiệm này không dương tính là $2\%$. Đối vơi những người không mắc bệnh $X$, xác suất xét nghiệm này dương tính là $3\%$. Xác suất một người được chọn ngẫu nhiên có kết quả dương tính với xét nghiệm phát hiện bệnh $X$ là bao nhiêu?
	\choice
	{$\dfrac{13}{400}$}
	{\True $\dfrac{139}{4000}$}
	{$\dfrac{137}{4000}$}
	{$\dfrac{13}{200}$}
	\loigiai{ Gọi $A$ là biến cố \lq\lq  người được chọn có kết quả dương tính với xét nghiệm phát hiện bệnh $X$ \rq\rq .\\
	Gọi $B_1$, $B_2$ lần lượt là biến cố \lq\lq  người được chọn mắc bệnh $X$\rq\rq. \\
	Xác suất $\mathrm{P}(B_1)=\dfrac{5}{1000}$, $\mathrm{P}(B_2)=\dfrac{995}{1000}$.\\
	Xác suất cần tìm là \\ $\mathrm{P}(A)=\mathrm{P}(B_1)\cdot \mathrm{P}(A|B_1)+\mathrm{P}(B_2)\cdot\mathrm{P}(A|B_2)=\dfrac{5}{1000}\cdot\dfrac{98}{1000}+\dfrac{995}{1000}\cdot\dfrac{3}{1000}=\dfrac{139}{4000}$
	}
\end{ex}

\begin{ex}%[2D6V2-2]Câu 6
Một cặp sinh đôi có thể do cùng một trứng sinh ra (sinh đôi thật), hoặc do hai trứng
khác nhau sinh ra (sinh đôi giả). Các cặp sinh đôi thật luôn cùng giới tính. Đối với
cặp sinh đôi giả thì khả năng cùng giới tính và khác giới tính là như nhau. Thống kê
cho thấy $34\%$ cặp sinh đôi đều là trai, $30\%$ cặp sinh đôi đều là gái, và $36\%$ cặp sinh đôi có giới tính khác nhau. Tìm xác suất sinh đôi thật?
	\choice
	{$0{,}29$}
	{$0{,}27$}
	{\True$0{,}28$}
	{$0{,}3$}
	\loigiai{Gọi $B_1$, $B_2$, $A$ lần lượt là biến cố \lq\lq  cặp sinh đôi là thật\rq\rq,\, \lq\lq  cặp sinh đôi là giả\rq\rq,\, \lq\lq  cặp sinh đôi cùng giới\rq\rq.
	\begin{eqnarray*}
	\mathrm{P}(A)& =& \mathrm{P}(B_1)\cdot \mathrm{P}(A|B_1)+\mathrm{P}(B_2)\mathrm{P}(A|B_2)\\
\Leftrightarrow	\, \, \, 0{,}34+0{,}3	&= & P(B_1)\cdot1+[1-P(B_1)]\cdot0{,}5\\
	\Rightarrow \quad \quad P(B_1) &= & 0{,28}
	\end{eqnarray*}
	}
\end{ex}

\begin{ex}%[2D6V2-2]Câu 7
	Một lô hạt giống được chia làm $2$ loại: loại $1$ chiếm $\dfrac{2}{3}$ số hạt của lô, còn lại là loại $2$. Hạt trong loại $1$ có tỉ lệ nảy mầm là $80\%$, loại $2$ có tỉ lệ nảy mầm là $60\%$. Hỏi tỉ lệ nảy mầm chung của lô hạt giống này là bao nhiêu? (Hay nói cách khác: Ta lấy ngẫu nhiên từ lô ra $1$ hạt giống. Tìm xác suất để chọn được hạt nảy mầm).
	\choice
	{$\dfrac{11}{16}$}
	{$\dfrac{11}{14}$}
	{$\dfrac{11}{11}$}
	{\True$\dfrac{11}{15}$}
	\loigiai{ Gọi $A_i$ là biến cố \lq\lq  hạt giống được lấy ra từ lô thứ $i$\rq\rq \, (với $i=1,2$).\\
	Gọi $A$ là biến cố \lq\lq  hạt giống được lấy ra là hạt nảy mầm\rq\rq .
	Dễ thấy rằng $A_1$, $A_2$, $A_3$ tạo thành một hệ đầy đủ các biến cố.\\
	Theo đề bài: $\mathrm{P}(A_1)=\dfrac{2}{3}$, $\mathrm{P}(A_2)=\dfrac{1}{3}$ và $\mathrm{P}(A|A_1)=0{,}8$, $\mathrm{P}(A|A_2)=0{,}6$.\\
	Áp dụng công thức xác suất đầy đủ
	$$\mathrm{P}(A)=\mathrm{P}(A_1)\cdot \mathrm{P}(A|A_1)+\mathrm{P}(A_2)\mathrm{P}(A|A_2)=\dfrac{2}{3}\cdot0{,}8+\dfrac{1}{3}\cdot0{,}6=\dfrac{11}{15}.$$
	}
\end{ex}

\begin{ex}%[2D6V2-2]Câu 8
Một trạm cấp cứu có $80\%$ nạn nhân bỏng nóng và $20\%$ bỏng do hóa chất. Loại bỏng do nóng có $30\%$ biến chứng và bỏng do hóa chất có $50\%$ biến chứng. Tìm xác suất của nạn nhân bị biến chứng.
	\choice
	{$0{,}35$}
	{$0{,}36$}
	{\True $0{,}34$}
	{$0{,}37$}
	\loigiai{ Gọi $A$ là biến cố \lq\lq  nạn nhân bị biến chứng\rq\rq.
	Gọi $A_1$ là biến cố \lq\lq  nạn nhân bị bỏng do nóng\rq\rq.
	Gọi $A_2$ là biến cố \lq\lq  nạn nhân bị bỏng do hóa chất\rq\rq .
	Ta thấy rằng $A_1$ và $A_2$ lập thành một hệ đầy đủ.\\
	Theo đề bài	$\mathrm{P}(A_1)=0{,}8$;\, $\mathrm{P}(A_2)=0{,}2$; \,$\mathrm{P}(A|A_1)=0{,}3$; \,$\mathrm{P}(A|A_2)=0{,}5$.\\
	Sử dụng công thức xác suất đầy đủ
	$$\mathrm{P}(A)=\mathrm{P}(A_1)\mathrm{P}(A|A_1)+P(A_2)\mathrm{P}(A|A_2)=0{,}8\cdot0{,}3+0{,}2\cdot0{,}5=0{,}34.$$
	}
\end{ex}

\begin{ex}%[2D6C2-2]Câu 9
Chuồng gà thứ nhất có $9$ con mái và $1$ con trống. Chuồng gà thứ hai có $1$ con mái và $5$ con trống. Từ mỗi chuồng gà bắt ra ngẫu nhiên một con. Các con gà còn lại được dồn vào chuồng thứ ba. Bắt ngẫu nhiên một con gà trong chuồng thứ ba. Tính xác suất để bắt được gà trống.
	\choice
	{$\dfrac{38}{107}$}
	{\True$\dfrac{38}{105}$}
	{$\dfrac{38}{109}$}
	{$\dfrac{39}{105}$}
	\loigiai{ 
	$B_1$ là biến cố $2$ con gà được bắt đều là trống.\\
	$B_2$ là biến cố $2$ con gà được bắt đều là mái.\\
	$B_3$ là biến cố $2$ con gà được bắt gồm $1$ trống và $1$ mái.\\
	$A$ là biến cố bắt được gà trống ở chuồng thứ ba.\\
	 $B_1$ xảy ra $\Rightarrow$ Chuồng thứ ba gồm: $10$ gà mái và $4$ gà trống.\\
	 $B_2$ xảy ra $\Rightarrow$ Chuồng thứ ba gồm: $8$ gà mái và $6$ gà trống.\\
	 $B_3$ xảy ra $\Rightarrow$ Chuồng thứ ba gồm: $9$ gà mái và $5$ gà trống.\\
	$\mathrm{P}(B_1)=\dfrac{1}{10}\cdot \dfrac{\mathrm{C}_{5}^{1}}{6}=\dfrac{1}{12}$;
	$\mathrm{P}(B_2)=\dfrac{\mathrm{C}_{9}^{1}}{10}\cdot \dfrac{1}{6}=\dfrac{3}{20}$;
	$\mathrm{P}(B_3)=1-\mathrm{P}(B_1)-\mathrm{P}(B_2)=\dfrac{23}{30}$.\\
	Xác suất để bắt được gà trống ở chuồng thứ ba là
	\begin{eqnarray*}
	\mathrm{P}(A)&=& \mathrm{P}(B_1)\mathrm{P}(A|B_1)+\mathrm{P}(B_2)\mathrm{P}(A|B_2)+\mathrm{P}(B_3)\mathrm{P}(A|B_3)\\
	&=& \dfrac{1}{12}\cdot\dfrac{4}{14}+\dfrac{3}{20}\cdot\dfrac{6}{14}+\dfrac{23}{30}\cdot\dfrac{5}{14}\\
	&= & \dfrac{38}{105}.
	\end{eqnarray*}
	}
\end{ex}

\begin{ex}%[2D6V2-3]Câu 10
Dây chuyền lắp ráp được các chi tiết do hai máy sản xuất. Trung bình máy thứ nhất cung cấp $60\%$ chi tiết, máy thứ hai cung cấp $40\%$ chi tiết. Khoảng $90\%$ chi tiết do máy thứ nhất sản xuất là đạt tiêu chuẩn, còn $85\%$ chi tiết do máy thứ hai sản xuất là đạt tiêu chuẩn. Lấy ngẫu nhiên từ dây chuyền một sản phẩm, lấy nó đạt tiêu chuẩn. Tìm xác suất để sản phẩm đó do máy thứ nhất sản xuất.
	\choice
	{$0{,}713$}
	{$0{,}715$}
	{$0{,}814$}
	{\True$0{,}614$}
	\loigiai{Gọi $A$ là biến cố \lq\lq  Chi tiết lấy từ dây chuyền đạt tiêu chuẩn\rq\rq.\\ $B_1$ là biến cố \lq\lq  Chi tiết do máy thứ nhất sản xuất \rq\rq.\\
	$B_2$ là biến cố \lq\lq  Chi tiết do máy thứ 2 sản xuất \rq\rq.\\
	Ta cần tính xác suất $\mathrm{P}(B_1|A)$.\\
	Theo điều kiện bài toán $\mathrm{P}(B_1)=0{,}6$; $\mathrm{P}(B_2)=0{,}4$; $\mathrm{P}(A|B_1)=0{,}9$; $\mathrm{P}(A|B_2)=0{,}85$. \\
	 Theo công thức Bayes
	 $$\mathrm{P}(B_1|A)=\dfrac{\mathrm{P}(B_1)\mathrm{P}(A|B_1)}{\mathrm{P}(B_1)\mathrm{P}(A|B_1)+\mathrm{P}(B_2)\mathrm{P}(A|B_2)}=\dfrac{0{,}6\cdot0{,9}}{0{,}6\cdot0{,9}+0{,}4\cdot0{,}85}=0{,}614.$$
	}
\end{ex}

\begin{ex}%[2D6C2-3]Câu 11
 Tan giờ học buổi chiều một sinh viên có $60\%$ về nhà ngay, nhưng do giờ cao điểm nên có $30\%$ ngày bị tắc đường nên bị về nhà muộn (từ $30$ phút trở lên) còn $20\%$ số ngày sinh viên đó vào quán Internet cạnh trường để chơi game, những ngày này xác suất về nhà muộn là $80\%$. Còn lại những ngày khác sinh viên đó đi chơi với bạn bè có xác suất về muộn là $90\%$. Tính xác suất sinh viên đó đi chơi với bạn và về muộn.
	\choice
	{$0{,}275$}
	{$0{,}575$}
	{\True $0{,}375$}
	{$0{,}475$}
	\loigiai{ 
	Gọi $B$ là biến cố sinh viên đi học về muộn \\
	Gọi $\overline{B}$ là biến cố sinh viên đó đi học không về muộn\\
	$E_1$ là biến cố tan học về nhà ngay $\Rightarrow \mathrm{P}(E_1)=0{,}6$; $\mathrm{P}(B|E_1)=0{,}3$.\\
 	$E_2$ là biến cố tan học đi chơi game $\Rightarrow \mathrm{P}(E_2)=0{,}2$; $\mathrm{P}(B|E_2)=0{,}8$.\\
	 $E_3$ là biến cố tan học về đi chơi với bạn $\Rightarrow \mathrm{P}(E_3)=0{,}2$; $\mathrm{P}(B|E_3)=0{,}9$.\\
	$B$ có thể xảy ra một trong 3 biến cố
	 \begin{eqnarray*}
	 	\mathrm{P}(B)&= & \mathrm{P}(E_1)\cdot \mathrm{P}(B|E_1)+\mathrm{P}(E_2)\cdot \mathrm{P}(B|E_2)+\mathrm{P}(E_3)\cdot \mathrm{P}(B|E_3)\\
	 	&= & 0{,}6\cdot0{,}3+0{,}2\cdot0{,}8+0{,}2\cdot0{,}9\\
	 	&= & 0{,}52.
	 \end{eqnarray*}
	Xác suất để sinh viên đó đi chơi với bạn và về muộn là
	$\mathrm{P}(E_3|B)=\dfrac{\mathrm{P}(E_3)\cdot \mathrm{P}(B|E_3)}{\mathrm{P}(B)}=0{,}375$.
	}
\end{ex}

\begin{ex}%[2D6V2-3]Câu 12
	Dây chuyền lắp ráp máy vô tuyến điện gồm các linh kiện là sản phẩm từ $2$ nhà máy sản xuất ra. Số linh kiện nhà máy $1$ sản xuất chiếm $55\%$, số linh kiện nhà máy $2$ sản xuất chiếm $45\%$; tỷ lệ sản phẩm đạt tiêu chuẩn của nhà máy $1$ là $90\%$, nhà máy $2$ là $87\%$. Lấy ngẫu nhiên ra $1$ linh kiện từ dây chuyền lắp ráp đó ra kiểm tra thì được kết quả linh kiện đạt chuẩn. Tìm xác suất để linh kiện đó do nhà máy $1$ sản xuất?
	\choice
	{$0{,}6683$}
	{\True$0{,}5583$}
	{$0{,}7583$}
	{$0{,}4583$}
	\loigiai{ Gọi $A_i$ là biến cố \lq\lq  linh kiện do nhà máy thứ $i$ sản xuất\rq\rq \, $i=1,2$.\\
	Gọi $B$ là biến cố \lq\lq  linh kiện đạt chuẩn\rq\rq, ta cần tìm $P(A_1|B)$\\
	Ta có\\ $\mathrm{P}(B)=\mathrm{P}(A_1)\mathrm{P}(B|A_1)+\mathrm{P}(A_2)\mathrm{P}(B|A_2)=0{,}55\cdot0{,}9+0{,}45\cdot0{,}87=0{,}8865$.\\
	$\mathrm{P}(A_1|B)=\dfrac{\mathrm{P}(A_1)\mathrm{P}(B|A_1)}{\mathrm{P}(B)}=\dfrac{0{,}55\cdot0{,}9}{0{,}8865}=0{,}5583$.
	}
\end{ex}

\begin{ex}%[2D6H2-2]
	Người ta khảo sát khả năng chơi nhạc cụ của một nhóm học sinh tại trường X. Nhóm này có $60\%$ học sinh là nam. Kết quả khảo sát cho thấy có $20\%$ học sinh nam và $15\%$ học sinh nữ biết chơi ít nhất một nhạc cụ. Chọn ngẫu nhiên một học sinh trong nhóm này. Gọi $A$ là biến cố \lq\lq  Chọn được một học sinh biết chơi ít nhất một nhạc cụ\rq\rq\,và $B$, $\overline{B}$ lần lượt là các biến cố \lq\lq  Chọn được một học sinh nam\rq\rq\,và \lq\lq  Chọn được một học sinh nữ\rq\rq. 
	\choiceTF
	{\True Xác suất $\mathrm{P}(B) = 60\% = 0{,}6$}
	{$\mathrm{P}(A|B) = 0{,}8$}
	{\True $\mathrm{P}(A|\overline{B}) = 0{,}15$}
	{\True Xác suất để chọn được học sinh biết chơi ít nhất một nhạc cụ là $18\%$}
	\loigiai{
	Xét phép thử chọn ngẫu nhiên một học sinh trong nhóm.\\
	Gọi $A$ là biến cố \lq\lq  Chọn được một học sinh biết chơi ít nhất một nhạc cụ\rq\rq \,và $B$, $\overline{B}$ lần lượt là các biến cố \lq\lq  Chọn được một học sinh nam\rq\rq\,và \lq\lq  Chọn được một học sinh nữ\rq\rq.
	Theo đề bài 
	\begin{itemchoice}
	\itemch $\mathrm{P}(B) = 60\% = 0{,}6$.
	$\mathrm{P}(\overline{B}) = 1-0{,}6 = 0{,}4$.
	\itemch $\mathrm{P}(A|B) = 20\% = 0{,}2$.
	\itemch $\mathrm{P}(A|\overline{B}) = 15\% = 0{,}15$. 
	\itemch Áp dụng công thức xác suất toàn phần, ta có
	$$\mathrm{P}(A) = \mathrm{P}(B)\cdot \mathrm{P}(A|B) + \mathrm{P}(\overline{B})\cdot \mathrm{P}(A|\overline{B}) = 0{,}6\cdot0{,}2 + 0{,}4\cdot 0{,}15 = 0{,}18.$$
	Vậy xác suất để chọn được một học sinh biết chơi nhạc cụ là $0{,}18$ hay $18\%$.
	\end{itemchoice}
	}
\end{ex}

\begin{ex}%[2D6H2-4]
	Kết quả khảo sát tại một xã cho thấy có $20\%$ cư dân hút thuốc lá. Tỉ lệ cư dân thường xuyên gặp các vấn đề sức khoẻ về đường hô hấp trong số những người hút thuốc lá và không hút thuốc lá lần lượt là $70\%$, $15\%$. Giả sử ta gặp một cư dân của xã, gọi $A$ là biến cố \lq\lq  Người đó có hút thuốc lá\rq\rq\,và $B$ là biến cố \lq\lq  Người đó thường xuyên gặp các vấn đề sức khoẻ về đường hô hấp\rq\rq.
	\choiceTF
	{$\mathrm{P}(AB)=0{,}13$}
	{$\mathrm{P}(\overline{A}B)=0{,}14$}
	{\True Nếu ta gặp một cư dân của xã thì xác suất người đó thường xuyên gặp các vấn đề sức khoẻ về đường hô hấp là $0{,}26$}
	{\True Nếu ta gặp một cư dân của xã thường xuyên gặp các vấn đề sức khoẻ về đường hô hấp thì xác suất người đó có hút thuốc lá xấp xỉ $54\%$}
	\loigiai{
	Giả sử ta gặp một cư dân của xã, gọi $A$ là biến cố \lq\lq  Người đó có hút thuốc lá\rq\rq\,và $B$ là biến cố \lq\lq  Người đó thường xuyên gặp các vấn đề sức khoẻ về đường hô hấp\rq\rq.\\
	Khi đó, ta có $\mathrm{P}(A)=0{,}2$; $\mathrm{P}(\overline{A})=0{,}8$; $\mathrm{P}(B|A)=0{,}7$, $\mathrm{P}(B|\overline{A})=0{,}15$.
	\begin{itemchoice}
	\itemch $\mathrm{P}(AB)=\mathrm{P}(A)\cdot \mathrm{P}(B|A)=0{,}2\cdot 0{,}7=0{,}14$.
	\itemch $\mathrm{P}(\overline{A}B)=\mathrm{P}(\overline{A})\cdot \mathrm{P}(B|\overline{A})=0{,}15\cdot0{,}8=0{,}12$.
	\itemch	
	Ta có $\mathrm{P}(B) = \mathrm{P}(A)\cdot \mathrm{P}(B|A) + \mathrm{P}(\overline{A})\cdot \mathrm{P}(B|\overline{A}) = 0{,}14 + 0{,}12 = 0{,}26$.\\
	Vậy nếu ta gặp một cư dân của xã thì xác suất người đó thường xuyên gặp các vấn đề sức khoẻ về đường hô hấp là $26\%$.
	\itemch Theo công thức Bayes, ta có $\mathrm{P}(A|B) = \dfrac{\mathrm{P}(A)\mathrm{P}(B|A)}{\mathrm{P}(B)} =\dfrac{ 0{,}2\cdot 0{,}7}{0{,} 26}\approx 0{,}54$.\\
	Vậy nếu ta gặp một cư dân của xã thường xuyên gặp các vấn đề sức khoẻ về đường hô hấp thì xác suất người đó có hút thuốc lá là khoảng $54\%$.
	\end{itemchoice}
	}
\end{ex}

\begin{ex}%[2D6H2-3]%1
	Tỉ lệ người dân đã tiêm vắc xin phòng bệnh $A$ ở một địa phương là $75\%$. Trong số những người đã tiêm phòng, tỉ lệ mắc bệnh $A$ là $10\%$; trong số những người chưa tiêm phòng, tỉ lệ mắc bệnh $A$ là $32\%$. Chọn ngẫu nhiên một người ở địa phương đó. Gọi $A$ là biến cố: \lq\lq  Người được chọn đã tiêm vắc xin phòng bệnh\rq\rq \, và $B$ là biến cố: \lq\lq  Người được chọn mắc bệnh $A$\rq\rq.
	\choiceTF
	{ $\mathrm{P}(A)=0{,}25$}
	{\True $\mathrm{P}\left(B|A\right)=0{,}1$}
	{\True $\mathrm{P}\left(B|\overline{A}\right)=0{,}32$}
	{ \True $\mathrm{P}\left(\overline{A}|B\right)=\dfrac{16}{31}$}
	\loigiai{
	Vì tỉ lệ người dân đã tiêm vắc xin phòng bệnh $A$ ở địa phương là $75\%$ nên $\mathrm{P}(A)=0{,}75$ và $\mathrm{P}\left(\overline{A}\right)=0{,}25$.\\
	Vì tỉ lệ mắc bệnh $A$ trong số những người đã tiêm phòng là $10\%$ và trong số những người chưa tiêm phòng là $32\%$ nên $\mathrm{P}(B|A)=0{,}1$ và $\mathrm{P}\left(B|\overline{A}\right)=0{,}32$.\\
	Theo công thức Bayes
	\[
	\mathrm{P}\left(\overline{A}|B\right)=\dfrac{\mathrm{P}\left(\overline{A}\right)\cdot \mathrm{P}\left(B|\overline{A}\right)}{\mathrm{P}(A) \cdot \mathrm{P}(B|A)+\mathrm{P}(\overline{A}) \cdot \mathrm{P}\left(B|\overline{A}\right)}=\dfrac{0{,}25\cdot 0{,}32}{0{,}75\cdot 0{,}1+0{,}25\cdot 0{,}32}=\dfrac{16}{31}.
	\]
	}
\end{ex}

\begin{ex}%[2D6H2-2]%2
	Ở một địa phương, tỉ lệ nam và nữ là $2 : 3$. Số người mắc bệnh bạch tạng của địa phương này chiếm tỉ lệ $0{,}45\%$ dân cư. Biết tỉ lệ nữ giới mắc bệnh bạch tạng là $0{,}35\%$. Xét phép thử chọn ngẫu nhiên một người ở địa phương, gọi
	\begin{itemize}
	\item $A$ là biến cố \lq\lq  Người được chọn mắc bệnh bạch tạng\rq\rq;
	\item $B$ là biến cố \lq\lq  Người được chọn là nam\rq\rq.
	\end{itemize}
	\choiceTF
	{ $\mathrm{P}(B)=\dfrac{3}{5}$}
	{$\mathrm{P}(A)=0{,}35\%$}
	{\True $\mathrm{P}(A) = \mathrm{P}(B)\cdot \mathrm{P}(A|B) + \mathrm{P}\left(\overline{B}\right)\cdot \mathrm{P}\left(A|\overline{B}\right)$}
	{\True Tỉ lệ nam giới mặc bệnh bạch tạng bằng $0{,}65\%$}
	\loigiai{
	Ta có
	\[
	\mathrm{P}(B)=\dfrac{2}{5}; \quad \mathrm{P}\left(\overline{B}\right) =\dfrac{3}{5};
	\]
	\[
	\mathrm{P}(A)=0{,}45\%;\quad \mathrm{P}\left(A|\overline{B}\right)=0{,}35\%.
	\]
	Áp dụng công thức xác suất toàn phần, ta có
	\[
	\mathrm{P}(A) = \mathrm{P}(B)\cdot \mathrm{P}(A|B) + \mathrm{P}\left(\overline{B}\right)\cdot \mathrm{P}\left(A|\overline{B}\right).
	\] 
	Từ đó suy ra
	$$\mathrm{P}(A|B) =\dfrac{\mathrm{P}(A) - \mathrm{P}\left(\overline{B}\right)\cdot \mathrm{P}\left(A|\overline{B}\right)}{\mathrm{P}(B)}=\dfrac{0{,}45\%-\dfrac{3}{5}\cdot 0{,}35\%}{\dfrac{2}{5}}=0{,}65\%.$$
	Vậy tỉ lệ nam giới mắc bệnh bạch tạng của địa phương đó là $0{,}65\%$.
	}
\end{ex}

\begin{ex}%[2D6H2-3]%3
	Bạn Nam tham gia một gian hàng trò chơi dân gian trong hội xuân của trường. Trò chơi có hai lượt chơi. Xác suất để Nam thắng ở lượt chơi thứ nhất là $0{,}6$. Nếu Nam thắng ở lượt chơi thứ nhất thì xác suất Nam thắng ở lượt chơi thứ hai là $0{,}8 $. Ngược lại, nếu Nam thua ở lượt chơi thứ nhất thì xác suất Nam thắng ở lượt chơi thứ hai là $0{,}3$. Xét các biến cố
	\begin{itemize}
	\item $A$: \lq\lq  Nam thắng ở lượt chơi thứ nhất\rq\rq.
	\item $B$: \lq\lq  Nam thắng ở lượt chơi thứ hai\rq\rq.
	\end{itemize}
	\choiceTF
	{ $\mathrm{P}(A)=0{,}8$}
	{ $\mathrm{P}\left(B|A\right) = 0{,}6$}
	{\True $\mathrm{P}\left(B|\overline{A}\right) = 0{,}3$}
	{\True Xác suất Nam thắng ở lượt chơi thứ nhất khi đã thắng ở lượt chơi thứ hai là khoảng $80 \%$}
	\loigiai{
	Theo đề bài, ta có
	\begin{listEX}[3]
	\item $\mathrm{P}(A)=0{,}6$
	\item $\mathrm{P}\left(B|A\right) = 0{,}8$.
	\item $\mathrm{P}\left(B|\overline{A}\right) = 0{,}3$.
	\end{listEX}
	Áp dụng công thức Bayes, ta có
	\begin{eqnarray*}
	\mathrm{P}(A | B) &= & \dfrac{\mathrm{P}(B | A) \cdot \mathrm{P}(A)}{\mathrm{P}(B | A) \cdot \mathrm{P}(A)+\mathrm{P}(B | \overline{A}) \cdot \mathrm{P}(\overline{A})} \\
	& = & \dfrac{0{,}8 \cdot 0{,}6}{ 0{,}8 \cdot 0{,}6 + 0{,}3 \cdot 0{,}4 } \\
	& \approx & 0{,}8.
	\end{eqnarray*}
	Vậy xác suất Nam thắng ở lượt chơi thứ nhất khi đã thắng ở lượt chơi thứ hai là khoảng $0{,}8$ hay $80 \%$.
	}
\end{ex}

\begin{ex}%[2D6V2-4]%4
	Một loại linh kiện do hai nhà máy số I, số II cùng sản xuất. Tỉ lệ phế phẩm của các nhà máy I, II lần lượt là $4\%$; $3\%$. Trong một lô linh kiện để lẫn lộn $80$ sản phẩm của nhà máy số I và $120$ sản phẩm của nhà máy số II. Một khách hàng lấy ngẫu nhiên một linh liện từ lô hàng đó. Xét các biến cố sau
	\begin{itemize}
	\item $A$: \lq\lq  Linh kiện lấy ra là linh kiện tốt\rq\rq.
	\item $B_1$: \lq\lq  Linh kiện lấy ra là linh kiện từ nhà máy số I\rq\rq.
	\item $B_2$: \lq\lq  Linh kiện lấy ra là linh kiện từ nhà máy số II\rq\rq.
	\end{itemize}
	\choiceTF
	{\True $\mathrm{P}(B_1)=0{,}4$}
	{\True $\mathrm{P}\left(A|B_2\right) = 0{,}97$}
	{\True $\mathrm{P}(A) = 0{,}966$}
	{Nếu linh kiện được lấy ra là linh kiện phế phẩm thì xác suất linh kiện đó do nhà máy II sản xuất là cao nhất}
	\loigiai{
	Theo đề bài, ta có
	\begin{listEX}[2]
	\item $\mathrm{P}(A|B_1) =1 - 0{,}04 = 0{,}96$.
	\item $\mathrm{P}\left(A|B_2\right) = 1-0{,}03 = 0{,}97$.
	\item $\mathrm{P}(B_1)=\dfrac{80}{200}=0{,}4$.
	\item $\mathrm{P}\left(B_2\right) = \dfrac{120}{200}=0{,}6$.
	\end{listEX}
	Khi đó áp dụng công thức xác suất toàn phần, ta có
	\[\mathrm{P}(A) = \mathrm{P}(A|B_1)\cdot \mathrm{P}(B_1) + \mathrm{P}\left(A|B_2\right)\cdot\mathrm{P}\left(B_2\right)=0{,}96\cdot 0{,}4 + 0{,}97\cdot 0{,}6=0{,}966.\]
	Ta có $\mathrm{P}\left(\overline{A}\right)=1-\mathrm{P}(A) = 0{,}034$.\\
	Áp dụng công thức Bayes, ta có
	\begin{itemize}
	\item $\mathrm{P}\left(B_1|\overline{A}\right)=\dfrac{\mathrm{P}\left(\overline{A}|B_1\right)\cdot \mathrm{P}(B_1)}{\mathrm{P}\left(\overline{A}\right)} = \dfrac{0{,}04\cdot 0{,}4}{0{,}034} = \dfrac{8}{17}\approx 0{,}048$.
	\item $\mathrm{P}\left(B_2|\overline{A}\right)=\dfrac{\mathrm{P}\left(\overline{A}|B_2\right)\cdot \mathrm{P}(B_2)}{\mathrm{P}\left(\overline{A}\right)} = \dfrac{0{,}03\cdot 0{,}6}{0{,}034} = \dfrac{8}{167}\approx 0{,}054$.
	\end{itemize}
	Vậy với điều kiện linh kiện lấy ra là linh kiện phế phẩm thì xác suất linh kiện đó do nhà máy I sản xuất là cao nhất.
	}
\end{ex}

\begin{ex}%[2D6H2-4]
	Xác suất để một chuyến bay khởi hành đúng giờ là $\mathrm{P}(D)=0{,}83$; xác suất để nó đến đúng giờ là $\mathrm{P}(A)=0{,}82$; xác suất để nó khởi hành và đến đều đúng giờ là $\mathrm{P}(D \cap A)=0{,}78$. 
	\choiceTF
	{\True Xác suất để một máy bay đến đúng giờ biết rằng nó đã khởi hành đúng giờ là $0{,}94$}
	{Xác suất để một máy bay khởi hành đúng giờ biết rằng nó sẽ đến đúng giờ là $0{,}85$}
	{\True Xác suất để một máy bay đến đúng giờ biết rằng nó khởi hành không đúng giờ là $0{,}24$}
	{Xác suất để một máy bay khởi hành đúng giờ biết rằng nó sẽ đến không đúng giờ là $0{,}95$}
	\loigiai{
	Ta có $\mathrm{P}(A \cap \overline{D})=\mathrm{P}(A)-\mathrm{P}(A \cap D)=0{,}82-0{,}78=0{,}04$.\\
	$\mathrm{P}(D \cap \overline{A})=\mathrm{P}(D)-\mathrm{P}(D \cap A)=0{,}83-0{,}78=0{,}05$.
	\begin{itemchoice}
	\itemch Xác suất để một máy bay đến đúng giờ biết rằng nó đã khởi hành đúng giờ là
	$$
	\mathrm{P}(A \mid D)=\frac{\mathrm{P}(D \cap A)}{\mathrm{P}(D)}=\frac{0{,}78}{0{,}83}=0{,}94.
	$$
	\itemch Xác suất để một máy bay khởi hành đúng giờ biết rằng nó đã đến đúng giờ là
	$$
	\mathrm{P}(D \mid A)=\frac{\mathrm{P}(D \cap A)}{\mathrm{P}(A)}=\frac{0{,}78}{0{,}82}=0{,}95.
	$$
	\itemch Xác suất để máy bay đến đúng giờ khi nó khởi hành không đúng giờ là
	$$
	\mathrm{P}(A \mid \overline{D})=\dfrac{\mathrm{P}(A \cap \overline{D})}{\mathrm{P}(\overline{D})}=\dfrac{0{,}82-0{,}78}{0{,}17}=0{,}24.
	$$
	\itemch Xác suất để một máy bay khởi hành đúng giờ biết rằng nó sẽ đến đúng giờ là $$
	\mathrm{P}(D \mid \overline{A})=\dfrac{\mathrm{P}(D \cap \overline{A})}{\mathrm{P}(\overline{A})}=\frac{0{,}83-0{,}78}{1-0{,}82}=0{,}28.
	$$
	\end{itemchoice}
	}
\end{ex}

\begin{ex}%[2D6H2-4]
	Cho hai biến cố $A$, $B$ sao cho $\mathrm{P}(A)=0{,}6$; $\mathrm{P}(B)=0{,}4$; $\mathrm{P}(A \mid B)=0{,}3$. 
	\choiceTF
	{\True $\mathrm{P}(B \mid A)=0{,}2$}
	{\True $\mathrm{P}(A \mid \overline{B})=0{,}8$}
	{$\mathrm{P}(B \mid \overline{A})=0{,}8$}
	{$\mathrm{P}(B \cap A)=0{,}24$}
	\loigiai{\begin{itemchoice}
	\itemch	Áp dụng công thức Bayes, ta có
	$$
	\mathrm{P}(B \mid A)=\dfrac{\mathrm{P}(B) \cdot \mathrm{P}(A \mid B)}{\mathrm{P}(A)}=\frac{0{,}4 \cdot 0{,}3}{0{,}6}=0{,}2.
	$$
	\itemch Ta có
	$\mathrm{P}(A)=\mathrm{P}(B)\cdot \mathrm{P}(A \mid B) + \mathrm{P}(\overline{B})\cdot \mathrm{P}(A \mid \overline{B})$.\\
	Suy ra $\mathrm{P}(A \mid \overline{B})=\dfrac{0{,}6-0{,}4\cdot0{,}3}{1-0{,}4}=0{,}8$.
	\itemch Ta có
	$\mathrm{P}(B)=\mathrm{P}(A)\cdot \mathrm{P}(B \mid A) + \mathrm{P}(\overline{A})\cdot \mathrm{P}(B \mid \overline{A})$.\\
	Suy ra $\mathrm{P}(B \mid \overline{A})=\dfrac{0{,}4-0{,}6\cdot0{,}2}{1-0{,}6}=0{,}7$.
	\itemch Ta có $\mathrm{P}(B \cap A)=\mathrm{P}(A\mid B)\cdot \mathrm{P}(B)=0{,}3\cdot 0{,}4=0{,}12$.
	\end{itemchoice}
	}
\end{ex}

\begin{ex}%%[2D6V2-3]Câu 1
Có $7$ hộp bi, trong đó có $4$ hộp loại $1$, $3$ hộp loại $2$. Mỗi hộp loại $1$ có $3$ bi trắng và $5$ bi đỏ, mỗi hộp loại $2$ có $4$ bi trắng và $6$ bi đỏ. Chọn ngẫu nhiên $1$ hộp và từ đó lấy ra $1$ bi thì được bi trắng. Tìm xác suất để bi lấy ra này thuộc hộp loại $2$. (Làm tròn đến kết quả hàng phần trăm).
	\shortans{$0{,}44$}
	\loigiai{ 
	Gọi $B$ là biến cố \lq\lq  lấy được bi trắng\rq\rq.\\
	$A_1$ là biến có lấy hộp loại 1.\\
	$A_2$ là biến có lấy hộp loại 2.\\
	$\mathrm{P}(A_1)=\dfrac{ \mathrm{C} _{4}^{1}}{ \mathrm{C} _{7}^{1}}=\dfrac{4}{7}$;
	$\mathrm{P}(A_2)=\dfrac{ \mathrm{C} _{3}^{1}}{ \mathrm{C} _{7}^{1}}=\dfrac{3}{7}$;
	$\mathrm{P}(B|A_1)=\dfrac{ \mathrm{C} _{3}^{1}}{ \mathrm{C} _{8}^{1}}=\dfrac{3}{8}$;
	$\mathrm{P}(B|A_2)=\dfrac{ \mathrm{C} _{4}^{1}}{ \mathrm{C} _{10}^{1}}=\dfrac{4}{10}=\dfrac{2}{5}$.\\
	$\Rightarrow \mathrm{P}(B)=\mathrm{P}(A_1)\mathrm{P}(B|A_1)+\mathrm{P}(A_2)\mathrm{P}(B|A_2)=\dfrac{4}{7}\cdot\dfrac{3}{8}+\dfrac{3}{7}\cdot\dfrac{2}{5}=\dfrac{27}{70}$.\\
	Xác suất để bi lấy ra này thuộc hộp loại $2$ là\\
	$\mathrm{P}(A_2|B)=\dfrac{\mathrm{P}(A_2)\mathrm{P}(B|A_2)}{\mathrm{P}(B)}=\dfrac{\dfrac{3}{7}\cdot\dfrac{2}{5}}{\dfrac{27}{70}}=\dfrac{4}{9}\approx 0{,}44$.
	}
\end{ex}

\begin{ex}%[2D6V2-2]Câu 2
Trong $12$ xạ thủ có $5$ người bắn trúng hồng tâm với xác suất $0{,}8$; $7$ người bắn trúng hồng tâm với xác suất $0{,}7$. Chọn ngẫu nhiên $1$ xạ thủ. Tìm xác suất để người này bắn trúng hồng tâm. (Làm tròn đến kết quả hàng phần trăm).
	\shortans{$0{,}74$}
	\loigiai{ Gọi $A$ là biến cố \lq\lq  Xạ thủ bắn trúng hồng tâm\rq\rq.\\
	$A_1$ là biến cố \lq\lq  nhóm $5$ người bắn trúng hồng tâm\rq\rq.\\
	$A_2$ là biến cố \lq\lq  nhóm $7$ người bắn trúng hồng tâm\rq\rq.\\
	$\mathrm{P}(A_1)=\dfrac{5}{12}$; $\mathrm{P}(A_2)=\dfrac{7}{12}$;
	$\mathrm{P}(A|A_1)=0{,}8$; $P(A|A_2)=0{,}7$.\\
	Xác suất để người này bắn trúng hồng tâm
	$$\mathrm{P}(A)=\mathrm{P}(A_1)\mathrm{P}(A|A_1)+\mathrm{P}(A_1)P(A|A_2)=\dfrac{5}{12}\cdot0{,}8+\dfrac{7}{12}\cdot0{,}7=\dfrac{89}{120}\approx 0{,}74.$$
	}
\end{ex}

\begin{ex}%[2D6V2-2]Câu 3
 Có $3$ hộp phấn. Hộp thứ nhất có $7$ viên trắng và $3$ viên vàng; hộp thứ hai có $16$ viên trắng và $4$ viên vàng; hộp thứ $3$ có $22$ viên trắng và $8$ viên vàng. Ta tung đồng thời $3$ đồng xu cân đối và đồng chất: nếu được cả $3$ mặt sấp thì chọn hộp thứ nhất; nếu được $1$ mặt sấp và $2$ mặt ngửa thì chọn hộp thứ hai; trường hợp còn lại thì chọn hộp thứ ba. Từ hộp đã chọn ta lấy ngẫu nhiên ra $1$ viên phấn. Tính xác suất để lấy được viên phấn trắng. (Làm tròn đến kết quả hàng phần trăm).
	\shortans{$0{,}75$}
	\loigiai{ 
	Gọi $A$ là biến cố \lq\lq  Lấy được một viên phấn trắng\rq\rq.\\
	 $A_1$ là biến cố \lq\lq  Chọn hộp 1\rq\rq.\\
	 $A_2$ là biến cố \lq\lq  Chọn hộp 2\rq\rq.\\
	 $A_3$ là biến cố \lq\lq  Chọn hộp 3\rq\rq.\\
	 $\mathrm{P}(A_1)=\left ( \dfrac{1}{2} \right )^3$.\\ 
	 $\mathrm{P}(A_2)=3\cdot\dfrac{1}{2}\left ( \dfrac{1}{2} \right )^2=\dfrac{3}{8}$.\\ 
 $\mathrm{P}(A_3)=1-\dfrac{3}{8}-\dfrac{1}{8}=\dfrac{1}{2}$.\\	 
	 	 	$\mathrm{P}(A|A_1)=\dfrac{7}{10}$; $\mathrm{P}(A|A_2)=\dfrac{16}{20}=\dfrac{4}{5}$;$\mathrm{P}(A|A_3)=\dfrac{22}{30}=\dfrac{11}{15}$.\\
	 	 	Xác suất để lấy được viên phấn trắng
	 	 	\begin{eqnarray*}
	 	 	P(A)	& =& \mathrm{P}(A_1)\mathrm{P}(A|A_1)+\mathrm{P}(A_2)\mathrm{P}(A|A_2)+\mathrm{P}(A_3)\mathrm{P}(A|A_3)\\
	 	 	&= & \dfrac{1}{8}\cdot\dfrac{7}{10}+\dfrac{3}{8}\cdot\dfrac{4}{5}+\dfrac{1}{2}\cdot\dfrac{11}{5}\\
	 	 	& =& \dfrac{181}{240}\\
	 	 	&\approx &0{,75}.
	 	 	\end{eqnarray*}
	}
\end{ex}

\begin{ex}%[2D6C2-3]Câu 5
Một loại linh kiện do $3$ nhà máy số $I$, số $II$, số $III$ cùng sản xuất. Tỷ lệ phế phẩm của các nhà máy lần lượt là: $I$: $0{,}04$; $II$: $0{,}03$ và $III$: $0{,}05$. Trong $1$ lô linh kiện để lẫn lộn $80$ sản phẩm của nhà máy số $I$, $120$ của nhà máy số $II$ và $100$ của nhà máy số $III$. Khách hàng lấy phải một linh kiện loại phế phẩm từ lô hàng đó. Khả năng linh kiện đó do nhà máy nào sản xuất là cao nhất? (Làm tròn đến kết quả hàng phần trăm).
\shortans{$3$}
\loigiai{
	Gọi $E_1$ là biến cố \lq\lq  phế phẩm máy số $I$\rq\rq $\Rightarrow \mathrm{P}(E_1)=0{,}04,\mathrm{P}(\overline{E_1})=0{,}96$.\\
	Gọi $E_2$ là biến cố \lq\lq  phế phẩm máy số $II$\rq\rq $\Rightarrow \mathrm{P}(E_2)=0{,}03,\mathrm{P}(\overline{E_2})=0{,}97$.\\
	Gọi $E_3$ là biến cố \lq\lq  phế phẩm máy số $III$\rq\rq $\Rightarrow \mathrm{P}(E_1)=0{,}05,\mathrm{P}(\overline{E_3})=0{,}95$\\
	Gọi $B$ là biến cố khách hàng lấy được $1$ linh kiện tốt\\
	$\mathrm{P}(B|\overline{E_1})=\dfrac{\mathrm{C}_{80}^{1}}{\mathrm{C}_{300}^{1}}=\dfrac{4}{15}$;
	$\mathrm{P}(B|\overline{E_2})=\dfrac{\mathrm{C}_{120}^{1}}{\mathrm{C}_{300}^{1}}=\dfrac{2}{5}$;
	$\mathrm{P}(B|\overline{E_3})=\dfrac{\mathrm{C}_{100}^{1}}{\mathrm{C}_{300}^{1}}=\dfrac{1}{3}$.\\
	Xác suất để khách hàng lấy được linh kiện tốt là\allowdisplaybreaks
	\begin{eqnarray*}
	\mathrm{P}(B)&=&\mathrm{P}(\overline{E_1})\mathrm{P}(B|\overline{E_1})+P(\overline{E_2})\mathrm{P}(B|\overline{E_2})+\mathrm{P}(\overline{E_3})\mathrm{P}(B|\overline{E_3})\\
	&=&0{,}96\cdot\dfrac{4}{15}+0{,}97\cdot\dfrac{2}{5}+0{,}95\cdot\dfrac{1}{3}\approx 0{,}96.
	\end{eqnarray*}
	Gọi $\overline{B}$ là biến cố khách hàng lấy $1$ linh kiện loại không tốt\\
	 $\mathrm{P}(\overline{B})=1-\mathrm{P}(B)=0{,}04$.\\
	$\mathrm{P}(E_1|\overline{B})=\dfrac{\mathrm{P}(E_1)P(\overline{B}|E_1)}{\mathrm{P}(\overline{B})}=\dfrac{0{,}04 \cdot \dfrac{\mathrm{C}_{80}^{1}}{\mathrm{C}_{300}^{1}}}{0{,}04}=0{,}26$.\\
	$\mathrm{P}(E_2|\overline{B})=\dfrac{\mathrm{P}(E_2)\mathrm{P}(\overline{B}|E_2)}{\mathrm{P}(\overline{B})}=\dfrac{0{,}03 \cdot \dfrac{\mathrm{C}_{120}^{1}}{\mathrm{C}_{300}^{1}}}{0{,}04}=0{,}3$.\\
	$\mathrm{P}(E_3|\overline{B})=\dfrac{\mathrm{P}(E_3)P(\overline{B}|E_3)}{\mathrm{P}(\overline{B})}=\dfrac{0{,}05 \cdot \dfrac{\mathrm{C}_{100}^{1}}{\mathrm{C}_{300}^{1}}}{0{,}04}=0{,}41$.\\
	Vậy linh kiện đó do máy $III$ là cao nhất.
	}
\end{ex}

\begin{ex}%[2D6V2-3]Câu 6
	Trong $1$ đám đông, số người nam bằng số người nữ. Xác suất mắc cận thị của nam là $0{,}4$ và nữ là $0{,}6$. Chọn ngẫu nhiên $1$ người. Xác suất chọn được nam không cận thị. (Làm tròn đến kết quả hàng phần trăm).
	\par\shortans{$0{,}6$}
	\loigiai{ Goi tiêp $H_1$ là biến cố chọn nam; $H_2$ là biến cố chọn được nữ.\\
	$C$ là biến cố người được chọn là không cận thị.\\
	Ta có $\mathrm{P}(H_1)=\mathrm{P}(H_2)=\dfrac{1}{2}$\\
	 $\mathrm{P}(C)=\mathrm{P}(H_1)\mathrm{P}(C|H_1)+\mathrm{P}(H_2)\mathrm{P}(C|H_2)=\dfrac{1}{2}\cdot0{,}6+\dfrac{1}{2}\cdot0{,}4=0{,}5$.\\
	Xác suất để chọn ngẫu nhiên ra $1$ người mà người đó là nam không cận thị là\\ (Áp dụng định lý Bayes)\\
	$\mathrm{P}(H_1|C)=\dfrac{\mathrm{P}(H_1)\mathrm{P}(C|H_1)}{\mathrm{P}(C)}=\dfrac{\dfrac{1}{2}\cdot0{,}6}{0{,}5}=0{,}6$.}
	\end{ex}

\begin{ex}%[2D6H2-2]
	Khi phát hiện một vật thể bay, xác suất một hệ thống radar phát 
	cảnh báo là $0{,}9$ nếu vật thể bay đó là mục tiêu thật và là $0{,}05$ 
	nếu đó là mục tiêu giả. Có $99\%$ các vật thể bay là mục tiêu giả. 
	Tính xác suất để radar phát hiện cảnh báo khi phát hiện một vật thể bay (Làm tròn đến hàng phần trăm).
	\shortans{$0{,}96$}
	\loigiai{
	Gọi $ A $ là biến cố \lq\lq  Hệ thống radar phát cảnh báo\rq\rq\, và $ B $ là biến cố \lq\lq  Vật thể bay là mục tiêu thật\rq\rq.\\
	Do xác suất một hệ thống radar cảnh báo nếu vật thể bay là mục tiêu thật là $ 0{,}9 $ nên $ \mathrm{P}(A|B)=0{,}9 $. \\
	Do xác suất một hệ thống radar cảnh báo nếu vật thể bay là mục tiêu giả là $ 0{,}05 $ nên $ \mathrm{P}(A|\overline{B})=0{,}05 $. \\
	Do có $ 99\% $ các vật thể bay là mục tiêu giả nên $ \mathrm{P}(\overline{B})=0{,}99 $ và $ \mathrm{P}(B)= 0{,}01$.\\
	Áp dụng công thức xác suất toàn phần, ta có xác suất để hệ thống radar phát cảnh báo là
	$$ \mathrm{P}(A)=\mathrm{P}(B)\mathrm{P}(A|B)+\mathrm{P}(\overline{B})\mathrm{P}(A|\overline{B})=0{,}01 \cdot 0{,}9+0{,}99 \cdot 0{,}05= 0{,}9595.$$
	}
\end{ex}

\begin{ex}%[2D6H2-2]
	Một loại vaccine được tiêm ở địa phương X. Người có bệnh nền thì với xác suất $0{,}35$ có phản ứng phụ sau tiêm, người không có bệnh nền thì chỉ có phản ứng phụ sau tiêm với xác suất $0{,}16$. Chọn ngẫu nhiên một người được tiêm vaccine. Tính xác suất người này có phản ứng phụ, biết rằng tỉ lệ người có bệnh nền ở địa phương X là $18\%$. (Làm tròn đến hàng phần trăm).
	\shortans{$0{,}19$}
	\loigiai{
	Gọi $A$ là biến cố \lq\lq  Người được chọn có bệnh nền\rq\rq\, và $B$ là biến cố \lq\lq  Người này có phản ứng phụ sau tiêm\rq\rq.\\
	Ta có $\mathrm{P}(A)=0{,}18$; $\mathrm{P}(\overline{A})=0{,}82$.\\
	$\mathrm{P}(B\mid A)$ là xác suất để một người bệnh có phản ứng sau tiêm với điều kiện có bệnh nền, suy ra $\mathrm{P}(B\mid A)=0{,}35$.\\
	$\mathrm{P}(B\mid \overline{A})$ là xác suất để một người bệnh có phản ứng sau tiêm với điều kiện không có bệnh nền, suy ra $\mathrm{P}(B\mid \overline{A})=0{,}16$.\\
	Theo công thức xác suất toàn phần, ta được 
	$$\mathrm{P}(B)=\mathrm{P}(A)\cdot \mathrm{P}(B\mid A)+\mathrm{P}(\overline{A})\cdot \mathrm{P}(B\mid \overline{A})=0{,}18\cdot 0{,}35+0{,}82\cdot 0{,}16=\dfrac{971}{5\,000}\approx 0{,}19.$$
	}
\end{ex}

\begin{ex}%[2D5V2-2]
	Có hai chuồng thỏ. Chuồng I có $5$ con thỏ đen và $10$ con thỏ trắng. Chuồng II có $7$ con thỏ đen và $3$ con thỏ trắng. Trước tiên, từ chuồng II lấy ra ngẫu nhiên $1$ con thỏ rồi cho vào chuồng I. Sau đó, từ chuồng I lấy ra ngẫu nhiên $1$ con thỏ. Tính xác suất để con thỏ được lấy ra là con thỏ trắng (Làm tròn đến hàng phần trăm).
	\par\shortans{$0{,}64$}
	\loigiai{Xét biến cố $A$ \lq\lq  Con thỏ được lấy ra từ chuồng II để cho vào chuồng I là con thỏ trắng\rq\rq.\\
	Xét biến cố $B$ \lq\lq  Con thỏ được lấy ra từ chuồng I là con thỏ trắng\rq\rq. \\	
	Ta có $\mathrm{P}(B)=\mathrm{P}(A)\cdot \mathrm{P}(B \mid A)+\mathrm{P}(\overline{A}) \cdot \mathrm{P}(B \mid \overline{A})$.
	\begin{itemize}
	\item Tính $\mathrm{P}(A)$: Đây là xác suất để lấy ra ngẫu nhiên 1 con thỏ trắng từ chuồng II rồi cho vào chuồng I. Có $n\left(\Omega\right)=\mathrm{C}^1_{10}$, $n\left(A\right)=\mathrm{C}^1_3$. Vậy $\mathrm{P}(A)=\dfrac{3}{10}$.
	\item Tính $\mathrm{P}(\overline{A})$: $\mathrm{P}(\overline{A})=1-\mathrm{P}(A)=\dfrac{7}{10}$.
	\item Tính $\mathrm{P}(B\mid A)$: Đây là xác suất để lấy ra ngẫu nhiên 1 con thỏ trắng từ chuồng I với điều kiện đã chọn ra 1 con thỏ trắng từ chuồng II rồi cho vào chuồng I, tức là có 5 con thỏ đen và 11 con thỏ trắng ở trong chuồng I. Tương tự như trên ta có $\mathrm{P}(B\mid A)=\dfrac{11}{16}$.
	\item Tính $\mathrm{P}(B\mid \overline{A})$: Đây là để lấy ra ngẫu nhiên 1 con thỏ trắng từ chuồng I với điều kiện đã chọn ra 1 con thỏ đen từ chuồng II rồi cho vào chuồng I, tức là có 6 con thỏ đen và 10 con thỏ trắng ở trong chuồng I. Tương tự như trên ta có $\mathrm{P}(B\mid \overline{A})=\dfrac{10}{16}$.
	\end{itemize}
	Vậy $\mathrm{P}(B)=\mathrm{P}(A)\cdot \mathrm{P}(B \mid A)+\mathrm{P}(\overline{A}) \cdot \mathrm{P}(B \mid \overline{A})=\dfrac{3}{10}\cdot \dfrac{11}{16}+\dfrac{7}{10}\cdot \dfrac{10}{16}=\dfrac{103}{160}=0{,}64375$. \\
	Vậy xác suất để con thỏ được lấy ra là con thỏ trắng xấp xỉ là $0{,}64$.
	}
\end{ex}

\begin{ex}%[2D6V2-3]
	Người ta điều tra thấy ở một địa phương nọ có $2\%$ tài xế sử dụng điện thoại di động khi lái xe. Trong các vụ tai nạn ở địa phương đó, người ta nhận thấy có $10\%$ là do tài xế có sử dụng điện thoại khi lái xe gây ra. Hỏi việc sử dụng điện thoại di động khi lái xe làm tăng xác suất gây tai nạn lên bao nhiêu lần?
	\par\shortans{$5$}
	\loigiai{
	Gọi $ A $ là biến cố \lq\lq  Tài xế sử dụng điện thoại di động khi lái xe\rq\rq\,và $ B $ là biến cố \lq\lq  Địa phương có tai nạn\rq\rq.\\
	Do có $ 2\% $ tài xế sử dụng điện thoại di động khi lái xe nên $ \mathrm{P}(A)=0{,}02 $.\\
	Do trong các vụ tai nạn ở địa phương, có $ 10\% $ là do tài xế có sử dụng điện thoại khi lái xe nên $ \mathrm{P}(A|B)=0{,}1 $.
	Xác suất gây tai nạn do tài xế sử dụng điện thoại di động khi lái xe là
	$$ \mathrm{P}(B|A)= \dfrac{\mathrm{P}(B)\mathrm{P}(A|B)}{\mathrm{P}(A)}$$
	Do đó $ \dfrac{\mathrm{P}(B|A)}{\mathrm{P}(B)}=\dfrac{\mathrm{P}(A|B)}{\mathrm{P}(A)}=\dfrac{0{,}1}{0{,}02}= 5$. Suy ra $ \mathrm{P}(B|A)=5\mathrm{P}(B) $.\\
	Vậy việc sử dụng điện thoại di động khi lái xe làm tăng xác suất gây tai nạn lên $5$ lần.
	}
\end{ex}

\begin{ex}%[2D6C2-4]
	Tỉ lệ người dân đã tiêm vắc xin phòng bệnh A ở một địa phương là $65\%$. Trong số những người đã tiêm phòng, tỉ lệ mắc bệnh A là $5\%$ còn trong số những người chưa tiêm, tỉ lệ mắc bệnh A là $17\%$. Gặp ngẫu nhiên một người ở địa phương đó. Biết rằng người đó mắc bệnh X. Khi đó xác suất người đó không tiêm vắc xin phòng bệnh X có dạng $\dfrac{a}{b}$. Giá trị $b-a$ là?
	\shortans{$65$}
	\loigiai{
	Gọi $A$ là biến cố \lq\lq  người đó mắc bệnh $X$\rq\rq\,và $B$ là biến cố \lq\lq  Gặp được người đã tiêm vắc xin phòng bệnh X\rq\rq.\\
	Theo công thức xác suất toàn phần, ta có
	\begin{eqnarray*}
	\mathrm{P}(A) & = &\mathrm{P}(B) \cdot \mathrm{P}(A \mid B)+\mathrm{P}(\overline{B}) \cdot \mathrm{P}(A \mid \overline{B}) \\
	& = &0,65 \cdot 0{,}05+0{,}35 \cdot 0{,}17=0{,}092.
	\end{eqnarray*}
	Suy ra
	\begin{eqnarray*}
	\mathrm{P}(\overline{B} \mid A) & =& \dfrac{\mathrm{P}(A \overline{B})}{\mathrm{P}(A)}=\dfrac{\mathrm{P}(\overline{B}) \mathrm{P}(A \mid \overline{B})}{\mathrm{P}(A)} \\
	& =& \dfrac{0,35 \cdot 0{,}17}{0{,}092}=\dfrac{119}{184}.
	\end{eqnarray*}
	Khi đó $a=119$ và $b=184$, suy ra $b-a=65$.
	}
\end{ex}

\begin{ex}%[2D6C2-4]
	Ở một khu rừng nọ có $7$ chú lùn, trong đó có $4$ chú luôn nói thật, $3$ chú còn lại nói thật với xác suất $0{,}5$. Bạn Tuyết gặp ngẫu nhiên một chú lùn. Gọi $A$ là biến cố \lq\lq  Chú lùn đó luôn nói thật\rq\rq\,và $B$ là biến cố \lq\lq  Chú lùn đó tự nhận mình luôn nói thật\rq\rq. Biết rằng chú lùn mà bạn Tuyết gặp tự nhận mình là người luôn nói thật. Tính xác suất để chú lùn đó luôn nói thật (làm tròn hai chữ số thập phân).
	\shortans{$0{,}73$}
	\loigiai{
	Ta có $\mathrm{P}(A)=\dfrac{4}{7}$; $\mathrm{P}(\overline{A})=\dfrac{3}{7}$; $\mathrm{P}(B \mid A)=1$; $\mathrm{P}(B \mid \overline{A})=0{,}5$.\\
	Theo công thức xác suất toàn phần, ta có
	\begin{eqnarray*}
	\mathrm{P}(B) & =& \mathrm{P}(A) \cdot \mathrm{P}(B \mid A)+\mathrm{P}(\overline{A}) \cdot \mathrm{P}(B \mid \overline{A}) \\
	& = & \dfrac{4}{7} \cdot 1+\dfrac{3}{7} \cdot 0{,}5=\dfrac{11}{14}.
	\end{eqnarray*}
	Khi đó
	$$
	\mathrm{P}(A \mid B)=\dfrac{\mathrm{P}(AB)}{\mathrm{P}(B)}=\dfrac{\mathrm{P}(A) \cdot \mathrm{P}(B \mid A)}{\mathrm{P}(B)}=\dfrac{\dfrac{4}{7} \cdot 1}{\dfrac{11}{14}}=\dfrac{8}{11} \approx 0{,}73.
	$$
	}
\end{ex}
\Closesolutionfile{ans}
% \section*{ÔN TẬP CHƯƠNG VI}
\subsubsection{Bài tập tự luận}
\setcounter{bt}{0}
\begin{bt}%[2D5H2-1]
	Một khu dân cư có $85 \%$ các hộ gia đình sử dụng điện để đun nấu. Hơn nữa, có $21 \%$ các hộ gia đình sử dụng bếp từ để đun nấu. Chọn ngẫu nhiên một hộ gia đình, tính xác suất hộ đó sử dụng bếp từ để đun nấu, biết hộ đó sử dụng điện để đun nấu.
	\loigiai{
	Gọi $A$ là biến cố \lq\lq  Hộ gia đình sử dụng điện để đun nấu\rq\rq, $B$ là biến cố \lq\lq  Hộ gia đình sử dụng bếp từ để đun nấu\rq\rq.\\
	Ta cần tính $\mathrm{P} (B \mid A)$.
	\begin{itemize}
	\item Do có $85 \%$ các hộ gia đình sử dụng điện để đun nấu nên $\mathrm{P} (A)=0{,}85$.
	\item Do trong các hộ gia đình sử dụng điện để đun nấu, có $21 \%$ các hộ gia đình sử dụng bếp từ để đun nấu nên $\mathrm{P} (AB)=0{,}21$.
	\item Vậy
	$\mathrm{P} (B \mid A)= \dfrac{\mathrm{P}(AB)}{\mathrm{P}(A)}=\dfrac{0{,}21}{0{,}85}=\dfrac{21}{85}.$
	\end{itemize}
	}
\end{bt}

\begin{bt}%[2D5H2-1]
	Cho hai biến cố ngẫu nhiên $A$ và $B$. Biết rằng $\mathrm{P} (A \mid B)=2 \mathrm{P}(B \mid A)$ và $\mathrm{P}(A B) \neq 0$.
	Tính tỉ số $\dfrac{\mathrm{P}(A)}{\mathrm{P}(B)}$.
	\loigiai{Ta có $$\mathrm{P} (A \mid B)=2 \mathrm{P}(B \mid A)\Leftrightarrow \dfrac{\mathrm{P}(AB)}{\mathrm{P}(B)}=2\dfrac{\mathrm{P}(AB)}{\mathrm{P}(A)}\Leftrightarrow \dfrac{\mathrm{P} (A)}{\mathrm{P}(B)} =2 .$$
}
\end{bt}

\begin{bt}%[2D5V2-3]
	Phòng công nghệ của một công ty có $4$ kĩ sư và $6$ kĩ thuật viên. Chọn ra ngẫu nhiên đồng thời $3$ người từ phòng. Tính xác suất để cả $3$ người được chọn đều là kĩ sư, biết rằng trong $3$ người được chọn có ít nhất $2$ kĩ sư.
	\loigiai{
	Để giải bài toán này, ta sử dụng công thức Bayes:
	$$
	\mathrm{P}(A \mid B)=\frac{\mathrm{P}(B \mid A) \mathrm{P}(A)}{\mathrm{P}(B)}
	$$
	Trong đó
	\begin{itemize}
	\item $A$ là biến cố \lq\lq  Cả 3 người được chọn đều là kĩ sư\rq\rq .
	\item $B$ là là biến cố \lq\lq  Trong 3 người được chọn có ít nhất 2 kĩ sư\rq\rq .
	\item $\mathrm{P}(A \mid B)$ là xác suất cần tìm.
	\end{itemize}
	Ta có
	\begin{itemize}
	\item $\mathrm{P}(A)=\dfrac{\mathrm{C} _4^3}{\mathrm{C}_{10}^3}=\dfrac{1}{30}$
	\item $\mathrm{P}(B)=\dfrac{\mathrm{C}_4^2\cdot \mathrm{C}_6^1+\mathrm{C}_4^3}{\mathrm{C}_{10}^3}=\dfrac{1}{3}$.
	\item $\mathrm{P}(B \mid A)=1$.
	\end{itemize}
	Áp dụng công thức Bayes, ta có
	$$
	\mathrm{P}(A \mid B)=\dfrac{\mathrm{P}(B \mid A) \mathrm{P}(A)}{\mathrm{P}(B)}=\dfrac{1 \cdot \dfrac{1}{30}}{\dfrac{1}{3}} =\dfrac{1}{10}.
	$$
	Vậy xác suất để cả $3$ người được chọn đều là kĩ sư, biết rằng trong $3$ người được chọn có ít nhất $2$ kĩ sư, là khoảng $10\%$.
	}
\end{bt}

\begin{bt}%[2D5V2-4]
	Có hai cái hộp giống nhau, hộp thứ nhất chứa $5$ quả bóng bàn màu trắng và $3$ quả bóng bàn màu vàng, hộp thứ hai chứa $4$ quả bóng bàn màu trắng và $6$ quả bóng bàn màu vàng. Minh lấy ra ngẫu nhiên $1$ quả bóng từ hộp thứ nhất. Nếu quả bóng đó là bóng vàng thì Minh lấy ra ngẫu nhiên đồng thời $2$ quả bóng từ hộp thứ hai, còn nếu quả bóng đó màu trắng thì Minh lấy ra ngẫu nhiên đồng thời $3$ quả bóng từ hộp thứ hai.
	\begin{enumEX}[\hspace*{.5cm}a)]{1}
	\item Sử dụng sơ đồ hình cây, tính xác suất để có đúng $1$ quả bóng màu vàng trong các quả bóng lấy ra từ hộp thứ hai.
	\item Biết rằng các quả bóng lấy ra từ hộp thứ hai đều có màu trắng. Tính xác suất để quả bóng lấy ra từ hộp thứ nhất có màu vàng.
	\end{enumEX}
	\loigiai{\begin{enumEX}[\hspace*{.5cm}a)]{1}
	\item\, \begin{center}
	\begin{tikzpicture}[scale=0.7, line join=round, line cap=round, >=stealth]
	\draw [->](0,0)--(4,4) node[right]{$\text{Vàng}$};\draw (2,2) node[above]{$\dfrac{3}{8}$};
	\draw [->](0,0)--(4,-4) node[right]{$\text{Trắng}$};\draw (2,-2) node[below]{$\dfrac{5}{8}$};
	\draw [->](6,4)--(9,6) node[right]{$\text{Vàng,\, Trắng}$};\draw (7.5,5) node[above]{$\frac{8}{15}$};\draw [->](15,6)--(17,6) node[right]{$\dfrac{1}{5}$};
	\draw [->](6,4)--(9,4) node[right]{$\text{Vàng,\, Vàng}$};\draw (7.5,4) node[above]{$\frac{1}{3}$};\draw [->](15,4)--(17,4) node[right]{$\dfrac{1}{8}$};
	\draw [->](6,4)--(9,2) node[right]{$\text{Trắng,\, Trắng}$};\draw (7.5,3) node[above]{$\frac{2}{15}$};\draw [->](15,2)--(17,2) node[right]{$\dfrac{1}{20}$};
	\draw [->](6,-4)--(9,-1) node[right]{$\text{Vàng,\, Trắng,\, Trắng}$};\draw (7.5,-2.5) node[above]{$\frac{3}{10}$};\draw [->](15,-1)--(17,-1) node[right]{$\dfrac{3}{16}$};
	\draw [->](6,-4)--(9,-3) node[right]{$\text{Vàng,\,Vàng,\,Trắng}$};\draw (7.5,-3.5) node[above]{$\frac{1}{2}$};\draw [->](15,-3)--(17,-3) node[right]{$\dfrac{5}{16}$};
	\draw [->](6,-4)--(9,-5) node[right]{$\text{Vàng,\, Vàng,\,Vàng}$};\draw (7.5,-4.5) node[above]{$\frac{1}{6}$};\draw [->](15,-5)--(17,-5) node[right]{$\dfrac{5}{48}$};
	\draw [->](6,-4)--(9,-7) node[right]{$\text{Trắng,\, Trắng,\,Trắng}$};\draw (7.5,-5.5) node[above]{$\frac{1}{30}$};\draw [->](15,-7)--(17,-7) node[right]{$\dfrac{1}{48}$};
	\end{tikzpicture}
	\end{center}
	Vậy xác suất để có đúng $1$ quả bóng màu vàng trong các quả bóng lấy ra từ hộp thứ hai là $$ \dfrac{1}{5} + \dfrac{3}{16}=\dfrac{31}{80}.$$
	\item Xét các biến cố
	\begin{itemize}
	\item $A$: \lq\lq  Quả bóng lấy ra từ hộp thứ nhất có màu vàng\rq\rq.
	\item $B$: \lq\lq  Các quả bóng lấy ra từ hộp thứ hai đều có màu trắng\rq\rq.
	\end{itemize}
	Ta có
	\[\mathrm{P}(A) = \dfrac{3}{8};\quad
	\mathrm{P}(B) = \dfrac{2}{15};\quad
	\mathrm{P}(B\mid A) = \dfrac{1}{20} \]
	Áp dụng công thức Bayes, ta có
	\[
	\mathrm{P} (A \mid B)=\dfrac{\mathrm{P} (B \mid A) \mathrm{P} (A)}{\mathrm{P} (B)}=\dfrac{\dfrac{3}{8} \cdot \dfrac{1}{20}}{\dfrac{2}{15}} =\dfrac{9}{64}.\]
	\end{enumEX}
}\end{bt}

\begin{bt}%[2D5V2-2]%[2D5V2-4]
	Hộp thứ nhất có $1$ viên bi xanh và $5$ viên bi đỏ. Hộp thứ hai có $3$ viên bi xanh và $5$ viên bi đỏ. Các viên bi có cùng kích thước và khối lượng. Lấy ra ngẫu nhiên đồng thời $2$ viên bi từ hộp thứ nhất chuyển sang hộp thứ hai. Sau đó lại lấy ra ngẫu nhiên $2$ viên bi từ hộp thứ hai.
	\begin{enumEX}[\hspace*{.5cm}a)]{1}
	\item Tính xác suất để hai viên bi lấy ra từ hộp thứ hai là bi đỏ.
	\item Biết rằng $2$ viên bi lấy ra từ hộp thứ hai là bi đỏ. Tính xác suất để $2$ viên bi lấy ra từ hộp thứ nhất cũng là bi đỏ.
	\end{enumEX}
	\loigiai{
	\begin{enumEX}[\hspace*{.5cm}a)]{1}
	\item Xét các biến cố
	\begin{itemize}
	\item $A$: \lq\lq  Hai viên bi lấy ra từ hộp thứ hai là bi đỏ\rq\rq.
	\item $B_1$: \lq\lq  Hai viên bi lấy ra từ hộp thứ nhất có cả màu xanh và màu đỏ\rq\rq.
	\item $B_2$: \lq\lq  Hai viên bi lấy ra từ hộp thứ nhất có màu đỏ\rq\rq.
	\end{itemize}
	Ta có
	\[\mathrm{P}(B_1) = \dfrac{\mathrm{C}^1_{5}}{\mathrm{C}^2_{6}}=\dfrac{1}{3};\quad
	\mathrm{P}(B_2) = \dfrac{\mathrm{C}^2_5}{\mathrm{C}^2_{6}}=\dfrac{2}{3};\quad
	\mathrm{P}(A\mid B_1) = \dfrac{\mathrm{C}^2_6}{\mathrm{C}^2_{10}}=\dfrac{1}{3};\quad
	\mathrm{P}(A\mid B_2) = \dfrac{\mathrm{C}^2_7}{\mathrm{C}^2_{10}}=\dfrac{7}{15}.
	\]
	Áp dụng công thức xác suất toàn phần, ta có
	\allowdisplaybreaks
	\begin{eqnarray*}
	\mathrm{P}(A)
	&=& \mathrm{P}(A|B_1)\cdot \mathrm{P}(B_1) + \mathrm{P}(A|B_2)\cdot\mathrm{P}(B_2)\\
	&=& \dfrac{1}{3}\cdot \dfrac{1}{3} + \dfrac{7}{15}\cdot \dfrac{2}{3}\\
	&=& \dfrac{19}{45}.
	\end{eqnarray*}
	\item Từ yêu cầu bài toán, ta cần tìm $\mathrm{P}(B_2|A)$.\\
	Áp dụng công thức Bayes, ta có
	\allowdisplaybreaks
	\begin{eqnarray*}
	\mathrm{P}(B_2|A)
	&=& \dfrac{\mathrm{P}(A|B_2)\cdot \mathrm{P}(B_2)}{\mathrm{P}(A)}\\
	&=& \dfrac{\dfrac{7}{15}\cdot \dfrac{2}{3}}{\dfrac{19}{45}}\\
	&=& \dfrac{14}{19}.
	\end{eqnarray*}
	\end{enumEX}
}\end{bt}

\begin{bt}
	Một cửa hàng kinh doanh tổ chức rút thăm trúng thưởng cho hai loại sản phẩm. Tỉ lệ trúng thưởng của các loại sản phẩm I, II lần lượt là $6 \%$; $4 \%$. Trong một hộp kín gồm các thăm cùng loại, người ta để lẫn lộn $200$ chiếc thăm cho sản phẩm loại I và $300$ chiếc thăm cho sản phẩm loại II. Một khách hàng lấy ngẫu nhiên $1$ chiếc thăm từ chiếc hộp đó.
	\begin{enumerate}
	\item Tính xác suất để chiếc thăm được lấy ra là trúng thưởng.
	\item Giả sử chiếc thăm được lấy ra là trúng thưởng. Xác suất chiếc thăm đó thuộc loại sản phẩm nào là cao nhất?
	\end{enumerate}
	\loigiai{
	\begin{enumerate}
	\item
	Xét biến cố $A$: \lq\lq  Chiếc thăm được lấy ra là trúng thưởng\rq\rq.\\
	Khi đó, ta có
	$$\mathrm{P}(A)=\dfrac{6\% \cdot 200 + 4\% \cdot 300}{200+300}=0{,}048.$$
	\item
	Xét hai biến cố:\\
	$B$: \lq\lq  Chiếc thăm được lấy ra là thăm cho sản phẩm loại I\rq\rq.\\
	$C$: \lq\lq  Chiếc thăm được lấy ra là thăm cho sản phẩm loại II\rq\rq.\\
	Khi đó, ta có:
	$$\mathrm{P}(B|A)=\dfrac{n(B\cap A)}{n(A)}=\dfrac{6\% \cdot 200}{6\% \cdot 200 + 4\% \cdot 300}=0{,}5.$$
	$$\mathrm{P}(C|A)=\dfrac{n(C\cap A)}{n(A)}=\dfrac{4\% \cdot 300}{6\% \cdot 200 + 4\% \cdot 300}=0{,}5.$$
	Vậy xác suất hai chiếc thăm lấy được là như nhau.
	\end{enumerate}
	}
\end{bt}

\begin{bt}
	Một xạ thủ bắn vào bia số $1$ và bia số $2$. Xác suất để xạ thủ đó bắn trúng bia số $1$, bia số $2$ lần lượt là $0{,}8$; $0{,}9$. Xác suất để xạ thủ đó bắn trúng cả hai bia là $0{,}8$. Xét hai biến cố sau:
	\begin{itemize}
	\item $A$: \lq\lq  Xạ thủ đó bắn trúng bia số $1$\rq\rq;
	\item $B$: \lq\lq  Xạ thủ đó bắn trúng bia số $2$\rq\rq.
	\end{itemize}
	\begin{enumerate}
	\item Hai biến cố $A$ và $B$ có độc lập hay không?
	\item Biết xạ thủ đó bắn trúng bia số $1$, tính xác suất xạ thủ đó bắn trúng bia số $2$.
	\item Biết xạ thủ đó không bắn trúng bia số $1$, tính xác suất xạ thủ đó bắn trúng bia số $2$.
	\end{enumerate}
	\loigiai{
	\begin{enumerate}
	\item $A$ và $B$ là hai biến cố độc lập.
	\item Xác suất xạ thủ đó bắn trúng bia số $2$ và bia số $1$ là $$\mathrm{P}(B|A)=\dfrac{\mathrm{P}(B\cap A)}{\mathrm{P}(A)}=\dfrac{\mathrm{P}(B) \cdot \mathrm{P}(A)}{\mathrm{P}(A)}=\mathrm{P}(B)=0{,}9.$$
	\item Xác suất xạ thủ đó bắn trúng bia số $2$ và không bắn trứng bia số $1$ là $$\mathrm{P}(B|\overline{A})=\dfrac{\mathrm{P}(B\cap \overline{A})}{\mathrm{P}(\overline{A})}=\dfrac{\mathrm{P}(B) \cdot \mathrm{P}(\overline{A})}{\mathrm{P}(\overline{A})}=\mathrm{P}(B)=0{,}9.$$
	\end{enumerate}
	}
\end{bt}

\begin{bt}
	Một chiếc hộp có $40$ viên bi, trong đó có $12$ viên bi màu đỏ và $28$ viên bi màu vàng; các viên bi có kích thước và khối lượng như nhau. Bạn Ngân lấy ngẫu nhiên viên bi từ chiếc hộp đó hai lần, mỗi lần lấy ra một viên bi và viên bi được lấy ra không bỏ lại hộp. Tính xác suất để cả hai lần bạn Ngân đều lấy ra được viên bi màu vàng.
	\loigiai{
	Xét hai biến cố\\
	$A$: \lq\lq  Viên bi thứ nhất màu vàng\rq\rq.\\
	$B$: \lq\lq  Viên bi thứ hai màu vàng\rq\rq.\\
	Khi đó, ta có xác suất để cả hai lần bạn Ngân đều lấy ra được viên bi màu vàng là\\
	$\mathrm{P}(A\cap B)=\dfrac{\mathrm{C}_{28}^2}{\mathrm{C}_{40}^2}=\dfrac{63}{130}$.
	}
\end{bt}

\begin{bt}
	Giả sử trong một nhóm người có $2$ người nhiễm bệnh, $58$ người còn lại là không nhiễm bệnh. Để phát hiện ra người nhiễm bệnh, người ta tiến hành xét nghiệm tất cả mọi người của nhóm đó. Biết rằng đối với người nhiễm bệnh, xác suất xét nghiệm có kết quả dương tính là $85 \%$ nhưng đối với người không nhiễm bệnh thì xác suất để bị xét nghiệm có phản ứng dương tính là $7 \%$.
	\begin{enumerate}
	\item Vẽ sơ đồ hình cây biểu thị tình huống trên.
	\item Giả sử $X$ là một người trong nhóm bị xét nghiệm có kết quả dương tính. Tính xác suất để $X$ là người nhiễm bệnh.
	\end{enumerate}
	\loigiai{
	\begin{enumerate}
	\item Xét hai biến cố\\
	$N$: \lq\lq  Người được chọn bị nhiễm bệnh\rq\rq.\\
	$D$: \lq\lq  Người được chọn có phản ứng dương tính\rq\rq.\\
	Khi đó, ta có
	\[\mathrm{P}(N)=\dfrac{2}{60}=\dfrac{1}{30}; \qquad \mathrm{P}(\overline{N})=\dfrac{58}{60}=\dfrac{29}{30}.\]
	\[\mathrm{P}(D|N)=85\%=0{,}85; \qquad \mathrm{P}(D|\overline{N})=7\%=0{,}07.\]
	Ta có sơ đồ cây biểu thị tình huống đã cho là
	\begin{center}
	\begin{tikzpicture}[->,>=stealth,line join=round,line cap=round,font=\footnotesize,scale=1]
	\def\xmot{4}
	\def\xhai{8}
	\node (O) at (0,0){};
	\node (B) at (\xmot,1){$N$};
	\node (B1) at (\xmot,-1){$\overline{N}$};
	\node (BA) at (\xhai,2){$D$};
	\node (BA1) at (\xhai,0.3){$\overline{D}$};
	\node (B1A) at (\xhai,-0.3){$D$};
	\node (B1A1) at (\xhai,-1.75){$\overline{D}$};
	\foreach \x/\y/\p/\l in
	{
	O/B/above/$\mathrm{P}(N)=\dfrac{1}{30}$,
	B/BA/above/$\mathrm{P}(D|N)=0{,}85$,
	B/BA1//,
	O/B1/below/$\mathrm{P}\left(\overline{N}\right)=\dfrac{29}{30}$,
	B1/B1A/above/$\mathrm{P}\left(D|\overline{N}\right)=0{,}07$,
	B1/B1A1//
	}
	{
	\draw[->] (\x)--(\y)node[midway,\p,scale=0.8,sloped]{\l};
	}
	\end{tikzpicture}
	\end{center}
	\item $\mathrm{P}(N|D)=\dfrac{\mathrm{P}(D|N) \cdot \mathrm{P}(N)}{\mathrm{P}(N) \cdot \mathrm{P}(D|N)+\mathrm{P}(\overline{N})\mathrm{P}(D|\overline{N})}=\dfrac{0{,}85 \cdot \dfrac{1}{30}}{0{,}85 \cdot \dfrac{1}{30}+0{,}07 \cdot \dfrac{29}{30}}\approx 29{,}5\%$.
	\end{enumerate}
	}
\end{bt}

\begin{bt}
	Trong một cuộc khảo sát trên một nhóm gồm $50$ học sinh chơi cầu lông có cả các bạn nam và các bạn nữ, số liệu thống kê các bạn thuận tay trái và thuận tay phải được cho như bảng sau:
	\begin{center}
	\begin{tabular}{|l|c|c|}
	\hline
	\diagbox{Giới tính}{Tay thuận}& Tay trái & Tay phải\\\hline
	Nam& $5$ & $32$\\\hline
	Nữ & $2$ & $11$\\\hline
	\end{tabular}
	\end{center}
	Chọn ngẫu nhiên một bạn học sinh trong nhóm này. Gọi $A$ là biến cố "Người được chọn là bạn nam", $B$ là biến cố "Chọn được người thuận tay trái", $C$ là biến cố "Chọn được người thuận tay phải."\\
	Tính và giải thích ý nghĩa của $P(A|B)$ và $P(A|C)$
	\loigiai{
	Ta có $P(AB)=\dfrac{5}{50}=\dfrac{1}{10}$, $P(B)=\dfrac{7}{50}.$\\
	Vậy $P(A|B)=\dfrac{P(AB)}{P(B)}=\dfrac{\dfrac{1}{10}}{\dfrac{7}{50}}=\dfrac{5}{7}.$\\
	Do đó xác suất để chọn ra một bạn nam với điều kiện bạn đó thuận tay trái là $\dfrac{5}{7}.$\\
	Ta có $P(AC)=\dfrac{32}{50}=\dfrac{16}{25}$, $P(C)=\dfrac{43}{50}.$\\
	Vậy $P(A|C)=\dfrac{P(AC)}{P(C)}=\dfrac{\dfrac{16}{25}}{\dfrac{43}{50}}=\dfrac{32}{43}.$\\
	Do đó xác suất để chọn ra một bạn nam với điều kiện bạn thuận tay phải là $\dfrac{32}{43}.$
	}
\end{bt}

\begin{bt}
	Một hãng hàng không sau khi nghiên cứ các chuyến bay cho kết quả như sau$\colon$ Xác suất để một chuyến bay khởi hành đúng giờ là $0,83$; xác suất để một chuyến bay đến nơi đúng giờ là $0,82$; xác suất để chuyến bay khởi hành đúng giờ và đến nơi đúng giờ là $0,78$. Gọi $A$ là biến cố "Chuyến bay khởi hành đúng giờ" và $B$ là biến cố "Chuyến bay đến nơi đúng giờ".
	\begin{enumerate}
	\item Tính và giải thích ý nghĩa của $P(A|B)$.
	\item Tính và giải thích ý nghĩa của $P(B|A)$.
	\item Tính $P(B|\overline{A})$ và cho biết xác suất chuyến bay đến nơi đúng giờ là tăng hay giảm khi có thêm thông tin chuyến bay khở hành không đúng giờ.
	\end{enumerate}
	\loigiai{
	\begin{enumerate}
	\item Ta có $P(A)=0,83,P(B)=0,82$ và $P(AB)=0,78$.\\
	Vậy $P(A|B)=\dfrac{P(AB)}{P(B)}=\dfrac{0,78}{0,82}=\dfrac{39}{41}=\approx 0,95$.\\
	Do đó, xác suất để chuyến bay khởi hành đúng giờ, biết rằng chuyến bay đến nơi đúng giờ là $0,95$
	\item Ta có $P(B|A)=\dfrac{P(AB)}{P(A)}=\dfrac{0.78}{0,83}=\dfrac{78}{83} \approx 0,94$.\\
	Vậy xác suất để chuyến bay đến nơi đúng giờ với điều kiện chuyến bay đã khởi hành đúng giờ là $0,94$.
	\item Ta có $P(\overline{A})=1-P(A)=1-0,83=0,17$.\\
	Ta có $P(\overline{A}|B)=1-P(A|B)=1-\dfrac{39}{41}=\dfrac{2}{41}$.\\
	Theo công thức Bayes, ta có\\
	$P(B|\overline{A})=\dfrac{P(B)\cdot P(\overline{A}|B)}{P(\overline{A})}=\dfrac{0,82\cdot \dfrac{2}{41}}{0,17}=\dfrac{4}{17}\approx 0,235$.\\
	Vì $0,235<0,82$ nên xác suất chuyến bay đến nơi đúng giờ là giảm khi có thêm thông tin chuyến bay khở hành không đúng giờ.
	\end{enumerate}
	}
\end{bt}

\begin{bt}
	Trong một cuộc khảo sát tình trạng công việc trên $900$ người đã có bằng tốt nghiệp trung học phổ thông ở một địa phương cho cả nam lẫn nữ, người ta thu được số liệu như bảng sau:
	\begin{center}
	\begin{tabular}{|l|c|c|}
	\hline
	\diagbox{Giới tính}{Tình trạng} & Có việc làm & Thất nghiệp\\\hline
	Nam & $460$ & $40$\\\hline
	Nữ & $140$ & $260$\\\hline
	\end{tabular}
	\end{center}
	Chọn ngẫu nhiên một người trong nhóm này. Gọi $A$ là biến cố "Người được chọn là nữ", $B$ là biến cố "Người được chọn có việc làm".
	\begin{enumerate}
	\item Vẽ lại sơ đồ hình cây sau đây và hoàn thành kết quả ở các ô \fbox{?}.
	\begin{center}
	\begin{tikzpicture}[line join = round, line cap = round, >=stealth, font=\footnotesize, scale=1]
	\begin{scope}[every node/.style={draw, rounded corners=5pt}]
	\node (A) at (0,0){Chọn một người};
	\def \gocA{30}
	\def \kcA{4}
	\node (B1) at ($(A)+(\gocA:\kcA)$){$A$};
	\node (B2) at ($(A)+(-\gocA:\kcA)$){$\overline{A}$};
	\def \gocB{15}
	\def \kcB{4}
	\node (B11) at ($(B1)+(\gocB:\kcB)$){$B$};
	\node (B12) at ($(B1)+(-\gocB:\kcB)$){$\overline{B}$};
	\node (B21) at ($(B2)+(\gocB:\kcB)$){$B$};
	\node (B22) at ($(B2)+(-\gocB:\kcB)$){$\overline{B}$};
	\end{scope}
	\begin{scope}[every node/.style={midway,sloped},every path/.style={->}]
	\draw (A)--(B1) node[above]{$\mathrm{P}(A)=$\fbox{?}};
	\draw (A)--(B2) node[below]{$\mathrm{P}(\overline{A})=$\fbox{?}};
	\draw (B1)--(B11) node[above]{$\mathrm{P}(B|A)=$\fbox{?}};
	\draw (B1)--(B12) node[below]{$\mathrm{P}(\overline{B}|A)=$\fbox{?}};
	\draw (B2)--(B21) node[above]{$\mathrm{P}(B|\overline{A})=$\fbox{?}};
	\draw (B2)--(B22) node[below]{$\mathrm{P}(\overline{B}|\overline{A})=$\fbox{?}};
	\end{scope}
	\def \kcC{1.7}
	\foreach \i/\j in {B11/AB,B12/{A\overline{B}},B21/{\overline{A}B},B22/{\overline{A}\,\overline{B}}}% Tạo nội dung lặp
	{
	\node at ($(\i)+(\kcC,0)$)[]{$\j$};
	\node at ($(\i)+({2*\kcC},0)$)[]{\fbox{?}};
	}
	\node (B) at ($(B11)+(\kcC,0.7)$){\textbf{Kết quả}};
	\node (C) at ($(B)+(\kcC,0)$){\textbf{Xác suất}};
	\end{tikzpicture}\\
	$A \colon$ nữ; $\overline{A} \colon$ nam; $B \colon$ có việc; $\overline{B} \colon$ thất nghiệp.
	\end{center}
	\item Tính xác suất để chọn được một người có việc làm.
	\item Biết rằng đã chọn được một người có việc làm, tính xác suất để người này là nữ.
	\end{enumerate}
	\loigiai{
	\begin{enumerate}
	\item Theo đề bài xác suất để chọn được một người nữ là $P(A)=\dfrac{4}{9}$, suy ra $P(\overline{A})=\dfrac{5}{9}$.\\
	Xác suất chọn được người có việc làm nếu người đó là nữ $P(B|A)=\dfrac{140}{400}=\dfrac{7}{20}$. Suy ra $P(\overline{B}|A)=\dfrac{13}{20}$.\\
	Xác suất chọn được người có việc làm nếu người đó không là nữ $P(B|\overline{A})=\dfrac{460}{500}=\dfrac{23}{25}$.\\
	Suy ra $P(\overline{B}|\overline{A})=\dfrac{2}{25}.$\\
	\begin{center}
	\begin{tikzpicture}[line join = round, line cap = round, >=stealth, font=\footnotesize, scale=1]
	\begin{scope}[every node/.style={draw, rounded corners=5pt}]
	\node (A) at (0,0){Chọn một người};
	\def \gocA{30}
	\def \kcA{4}
	\node (B1) at ($(A)+(\gocA:\kcA)$){$A$};
	\node (B2) at ($(A)+(-\gocA:\kcA)$){$\overline{A}$};
	\def \gocB{15}
	\def \kcB{4}
	\node (B11) at ($(B1)+(\gocB:\kcB)$){$B$};
	\node (B12) at ($(B1)+(-\gocB:\kcB)$){$\overline{B}$};
	\node (B21) at ($(B2)+(\gocB:\kcB)$){$B$};
	\node (B22) at ($(B2)+(-\gocB:\kcB)$){$\overline{B}$};
	\end{scope}
	\begin{scope}[every node/.style={midway,sloped},every path/.style={->}]
	\draw (A)--(B1) node[above]{$\mathrm{P}(A)=$\fbox{$\dfrac{4}{9}$}};
	\draw (A)--(B2) node[below]{$\mathrm{P}(\overline{A})=$\fbox{$\dfrac{5}{9}$}};
	\draw (B1)--(B11) node[above]{$\mathrm{P}(B|A)=$\fbox{$\dfrac{7}{20}$}};
	\draw (B1)--(B12) node[below]{$\mathrm{P}(\overline{B}|A)=$\fbox{$\dfrac{13}{20}$}};
	\draw (B2)--(B21) node[above]{$\mathrm{P}(B|\overline{A})=$\fbox{$\dfrac{23}{25}$}};
	\draw (B2)--(B22) node[below]{$\mathrm{P}(\overline{B}|\overline{A})=$\fbox{$\dfrac{2}{25}$}};
	\end{scope}
	\def \kcC{1.7}
	\foreach \i/\j/\k in {B11/AB/{\dfrac{7}{45}},B12/{A\overline{B}}/{\dfrac{13}{45}},B21/{\overline{A}B}/{\dfrac{23}{45}},B22/{\overline{A}\,\overline{B}}/{\dfrac{2}{45}}}% Tạo nội dung lặp
	{
	\node at ($(\i)+(\kcC,0)$)[]{$\j$};
	\node at ($(\i)+({2*\kcC},0)$)[]{$\k$};
	}
	\node (B) at ($(B11)+(\kcC,0.7)$){\textbf{Kết quả}};
	\node (C) at ($(B)+(\kcC,0)$){\textbf{Xác suất}};
	\end{tikzpicture}
	\end{center}
	\item Xác suất để chọn được một người có việc làm $$P(B)=P(A)P(B|A)+P(\overline{A})P(B|\overline{A})=\dfrac{4}{9}\cdot \dfrac{7}{20}+\dfrac{5}{9}\cdot \dfrac{23}{25}=\dfrac{2}{3}.$$
	\item Theo công thức Bayes, ta có\\
	$$P(A|B)=\dfrac{P(A)P(B|A)}{P(B)}=\dfrac{\dfrac{4}{9}\cdot \dfrac{7}{20}}{\dfrac{2}{3}}=\dfrac{7}{30}.$$
	\end{enumerate}
	}
\end{bt}

\begin{bt}
	Theo thống kê, tỉ lệ khách hàng thân thiết của một siêu thị là $35\%$. Trong nhóm khách hàng thân thiết này, có $74\%$ khách hàng mua rau sạch. Trong nhóm khách hàng còn lại, tỉ lệ mua rau sạch là $28\%$
	\begin{enumerate}
	\item Tính tỉ lệ khách hàng mua rau sạch của siêu thị đó.
	\item Trong một dịp đặc biệt, người ta đã chọn được một khách hàng mua rau sạch. Tính xác suất người này là khách hàng thân thiết.
	\end{enumerate}
	\loigiai{
	Gọi $A$ là biến cố "Khách hàng được chọn là khách hàng thân thiết của siêu thị".\\
	Gọi $B$ là biến cố "Khách hàng được chọn là khách hàng mua rau sạch".\\
	\begin{enumerate}
	\item Ta có $P(A)=0,35$ và $P(\overline{A})=1-P(A)=1-0,35=0,65$.\\
	Vì trong nhóm khách hàng thân thiết này, có $74\%$ khách hàng mua rau sạch nên $P(B|A)=0,74$.\\
	Vì trong nhóm khách hàng còn lại, tỉ lệ mua rau sạch là $28\%$ nên $P(B|\overline{A})=0,28$.\\
	Xác suất khách hàng mua rau sạch $$P(B)=P(A)P(B|A)+P(\overline{A})P(B|\overline{A})=0,35\cdot 0,74+0,65\cdot 0,28=0,441.$$
	Vậy tỉ lệ khách hàng mua rau sạch của siêu thị đó là $44,1\%$
	\item Khi chọn được một khách hàng mua rau sạch thì xác suất người này là khách hàng thân thiết là$\colon$
	$$P(A|B)=\dfrac{P(A)P(B|A)}{P(B)}=\dfrac{0,35\cdot 0,74 }{0,441}=\dfrac{37}{63}.$$
	\end{enumerate}
	}
\end{bt}

\begin{bt}
	Trung tâm kiểm soát và phòng ngừa dịch bệnh Hoa Kỳ (Centers for Disease Control and Prevention, viết tắt là (CDC) thống kê vào thời điểm năm $2020 - 2021$ về số lượng sốc phản vệ sau khi tiêm vaccine ở một số nơi tại Hoa Kỳ và châu Âu như sau: Trong $360{,}19$ triệu liều vaccine $P$ được sử dụng có $581$ ca sốc phản vệ (có khả năng gây tử vong) và $4\,259$ ca phản ứng phụ (không sốc phản vệ, không gây tử vong); trong $67{,}72$ triệu liều vaccine $A$ được sử dụng có $195$ ca sốc phản vệ và $1\,118$ ca phản ứng phụ.\\
	\textit{(Nguồn: https://www.ncbi.nlm.nih.gov/pmc/articles/PMC8626274/)}
	\begin{enumerate}
	\item Xét ngẫu nhiên một người trong số được thống kê ở trên. Tính xác suất để người đó thuộc trường hợp sốc phản vệ (có khả năng gây tử vong).
	\item Nếu gặp một người có biểu hiện sốc phản vệ (có khả năng gây tử vong) trong số này thì có thể nói khả năng cao là người đó đã tiêm vaccine $P$ hay $A$ ?
	\end{enumerate}
	\loigiai{
	%Bài này thực chất không cần dùng đến công thức xác suất toàn phần và công thức Bayes, tuy nhiên trong chương nên mình chọn giải theo lý thuyết của chương
	\begin{enumerate}
	\item Xét ngẫu nhiên một người trong số được thống kê ở trên. Tính xác suất để người đó thuộc trường hợp sốc phản vệ (có khả năng gây tử vong).\\
	Gọi $X$ là biến cố “Người được chọn tiêm vaccine $P$”, khi đó $\overline X $ là biến cố “Người được chọn tiêm vaccine $A$”.\\
	$Y$ là biến cố “Người được chọn thuộc trường hợp sốc phản vệ”.\\
	Khi đó, xác suất chọn được người tiêm vaccine $P$ là $P(X)=\dfrac{360{,}19 \cdot 10^6}{360{,}19 \cdot 10^6+67{,}72 \cdot 10^6}$.\\
	Xác suất chọn được người tiêm vaccine $A$ là $P(\overline X)=\dfrac{67{,}72 \cdot 10^6}{360{,}19 \cdot 10^6+67{,}72 \cdot 10^6}$.\\
	Xác suất chọn được người bị sốc phản vệ, nếu người đó tiêm vaccine $P$ là $P(Y|X)=\dfrac{581}{360{,}19 \cdot 10^6}$.\\
	Xác suất chọn được người bị sốc phản vệ, nếu người đó tiêm vaccine $A$ là $P(Y|\overline X)=\dfrac{195}{67{,}72 \cdot 10^6}$.\\
	Áp dụng công thức tính xác suất toàn phần, ta có:
	$$P(Y)=P(X)\cdot P(Y|X)+P(\overline{X})\cdot P(Y|\overline{X}) \approx 1{,}81 \cdot 10^{-6}$$
	\item So sánh khả năng người đó đã tiêm vaccine $P$ hay $A$\\
	Theo công thức Bayes, ta có xác suất người được chọn bị sốc phản vệ tiêm vaccine $P$ là:
	$$P(X|Y)=\dfrac{P(X)\cdot P(Y|X)}{P(Y)} \approx 0{,}75$$
	Vậy khả năng người đó đã tiêm vaccine $P$ cao hơn so với khả năng tiêm vaccine $A$.
	\end{enumerate}
	}
\end{bt}

\begin{bt}
	Một nhà máy có hai phân xưởng cùng sản xuất một loại sản phẩm. Phân xưởng thứ nhất sản xuất $60 \%$ và phân xưởng thứ hai sản xuất $40 \%$ tổng số sản phẩm của cả nhà máy. Tỉ lệ phế phẩm của từng phân xưởng lần lượt là $16 \%$ và $20 \%$. Lấy ngẫu nhiên một sản phẩm trong kho hàng của nhà máy.
	\begin{enumerate}
	\item Tính xác suất để lấy được phế phẩm.
	\item Giả sử đã lấy được phế phẩm, tính xác suất phế phẩm đó do phân xưởng thứ nhất sản xuất.
	\item Nếu lấy được sản phẩm tốt, khả năng sản phẩm đó do phân xưởng nào sản xuất là cao hơn?
	\end{enumerate}
	\loigiai{
	\begin{enumerate}
	\item Gọi $A$ là biến cố “Chọn được sản phẩm từ phân xưởng thứ nhất”, khi đó $\overline A $ là biến cố “Chọn được sản phẩm từ phân xưởng thứ hai”.\\
	$B$ là biến cố “Chọn được sản phẩm là phế phẩm”.\\
	Khi đó: $P(A)=60 \% = 0{,}6$; $P(\overline{A})=40 \% = 0{,}4$; $P(B|A)=16 \%=0{,}16$; $P(\overline B|A)=0{,}84$; $P(B|\overline A)=20 \%=0{,}2$ .\\
	Áp dụng công thức tính xác suất tính xác suất toàn phần, ta có:
	$$P(B)=P(A)\cdot P(B|A)+P(\overline{A})\cdot P(B|\overline{A})=0{,}6 \cdot 0{,}16+0{,}4 \cdot 0{,}2=0,176.$$
	Vậy xác suất lấy được phế phẩm là $0{,}176$.
	\item Chọn được phế phẩm, biến cố phế phẩm đó do phân xưởng thứ nhất sản xuất là $A|B$, áp dụng công thức Bayes, ta được:
	$$P(A|B)=\dfrac{P(A)\cdot P(B|A)}{P(B)}=\dfrac{0{,}6\cdot 0{,}16}{0{,}176}=\dfrac{6}{11} \approx 0{,}55.$$
	\item Khi lấy được sản phẩm tốt, để so sánh khả năng sản phẩm thuộc phân xưởng, ta tính xác suất để sản phẩm tốt được chọn ấy thuộc phân xưởng thứ nhất \\
	Từ ý a suy ra $P(\overline{B})=1-0{,}176=0,824$.
	Theo công thức Bayes, ta có:
	$$P(A|\overline{B})=\dfrac{P(A)\cdot P(\overline{B}|A)}{P(\overline{B})}=\dfrac{0{,}6 \cdot 0{,}84}{0{,}824} \approx 0{,}61.$$
	Vậy khả năng sản phẩm tốt được chọn từ phân xưởng thứ nhất cao hơn.
	\end{enumerate}
	}
\end{bt}

\begin{bt}
Để thử nghiệm tác dụng điều trị bệnh mất ngủ của hai loại thuốc $X$ và $Y$, người ta tiến hành thử nghiệm trên 4000 người bệnh tình nguyện. Kết quả được cho trong bảng thống kê sau:
	\begin{center}
	\begin{tabular}{|c|c|c|}
	\hline \diagbox{Kết quả}{Dùng thuốc}&$X$ & $Y$\\
	\hline Khỏi bệnh & $1600$ & $1200$\\
	\hline Không khỏi bệnh & $800$ & $400$\\
	\hline
	\end{tabular}
	\end{center}
	Chọn ngẫu nhiên 1 người bệnh tham gia tình nguyện thử nghiệm thuốc.
	\begin{enumEX}{1}
	\item Tính xác suất để người đó khỏi bệnh nếu biết người đó uống thuốc $X$.
	\item Tính xác suất để người bệnh đó uống thuốc $Y$, biết rằng người đó khỏi bệnh.
	\end{enumEX}
	\loigiai{
	\begin{enumEX}{1}
	\item Không gian mẫu là tập hợp gồm $4000$ người điều trị bệnh nên $n(\Omega)=4000$.\\
	Gọi $A$ là biến cố: \lq\lq  Người đó khỏi bệnh\rq\rq\, và $B$ là biến cố: \lq\lq  Người đó uống thuốc $X$\rq\rq.\\
	Khi đó $AB$ là biến cố: \lq\lq  Người đó khỏi bệnh và uống thuốc $X$\rq\rq.\\
	Ta có số người uống thuốc $X$ là $1600+800=2400$ nên $n(B)=2400\Rightarrow \mathrm{P}(B)=\dfrac{2400}{4000}=\dfrac{3}{5}$.\\
	Trong những người khỏi bệnh và uống thuốc $X$ có $1600$ người nên $n(AB)=1600\Rightarrow \mathrm{P}(AB)=\dfrac{1600}{4000}=\dfrac{2}{5}$.\\
	Ta có $\mathrm{P}(A\mid B)=\dfrac{\mathrm{P}(AB)}{\mathrm{P}(B)}=\dfrac{2}{5}:\dfrac{3}{5}=\dfrac{2}{3}$.\\
	Vậy xác suất để người đó khỏi bệnh nếu biết người đó uống thuốc $X$ là $\dfrac{2}{3}$.
	\item Gọi $C$ là biến cố: \lq\lq  Người đó uống thuốc $Y$\rq\rq\, và $D$ là biến cố: \lq\lq  Người đó khỏi bệnh\rq\rq.\\
	Khi đó $CD$ là biến cố: \lq\lq  Người đó uống thuốc $Y$ và khỏi bệnh\rq\rq.\\
	Ta có số người khỏi bệnh là $1600+1200=2800$ nên $$n(D)=2800\Rightarrow \mathrm{P}(D)=\dfrac{2800}{4000}=\dfrac{7}{10}.$$\\
	Có $1200$ người uống thuốc $Y$ và khỏi bệnh nên $n(CD)=1200\Rightarrow \mathrm{P}(CD)=\dfrac{1200}{4000}=\dfrac{3}{10}$.\\
	Ta có $\mathrm{P}(C\mid D)=\dfrac{\mathrm{P}(CD)}{\mathrm{P}(D)}=\dfrac{3}{10}:\dfrac{7}{10}=\dfrac{3}{7}$.\\
	Vậy xác suất để người bệnh đó uống thuốc $Y$, biết rằng người đó khỏi bệnh là $\dfrac{3}{7}$.
	\end{enumEX}
	}
\end{bt}

\begin{bt}
Một nhóm có $25$ học sinh, trong đó $14$ học sinh học khá môn Toán, $16$ học sinh học khá môn Vật lí, $1$ em không học khá cả hai môn Toán và Vật lí. Chọn ngẫu nhiên một học sinh trong số đó, tính xác suất để học sinh đó
	\begin{enumEX}{1}
	\item Học khá môn Toán, đồng thời học khá môn Vật lí.
	\item Học khá môn Toán, nhưng không học khá môn Vật lí.
	\item Học khá môn Toán, biết rằng học sinh đó học khá môn Vật lí.
	\end{enumEX}
	\loigiai{
	Chọn bất kì một học sinh trong $25$ học sinh nên số phần tử không gian mẫu là $n(\Omega)=25$.
	\begin{enumEX}{1}
	\item Gọi $C$ là biến cố: \lq\lq  Học sinh được chọn học khá môn Toán, đồng thời học khá môn Vật lí\rq\rq.\\
	Ta có số học sinh học khá cả Toán và Vật lí là $14+16-24=6$ nên $n(C)=6$.\\
	Xác suất để chọn được học sinh học khá môn Toán, đồng thời học khá môn Vật lí là $$\mathrm{P}(C)=\dfrac{n(C)}{n(\Omega)}=\dfrac{6}{25}.$$
	\item
	Gọi $D$ là biến cố: \lq\lq  Học sinh được chọn học khá môn Toán, nhưng không học khá môn Vật lí\rq\rq.\\
	Vì số học sinh chỉ học khá môn Toán là $14-6=8$ nên $n(D)=8$.\\
	Xác suất để chọn được học sinh học khá môn Toán, nhưng không học khá môn Vật lí là $$\mathrm{P}(D)=\dfrac{8}{25}.$$
	\item Gọi $A$ là biến cố: \lq\lq  Học sinh được chọn học khá môn Toán\rq\rq\, và $B$ là biến cố: \lq\lq  Học sinh được chọn học khá môn Vật lí\rq\rq\, nên $AB$ là biến cố: \lq\lq  Học sinh được chọn học khá môn Toán và Vật lí\rq\rq.\\
	Có $16$ học sinh học khá môn Vật lí nên $n(B)=16\Rightarrow \mathrm{P}(B)=\dfrac{16}{25}$.\\
	Ta có $\mathrm{P}(AB)=\dfrac{6}{25}$.\\
	Khi đó xác suất để chọn được học sinh học khá môn Toán, biết rằng học sinh đó học khá môn Vật lí là
	$$\mathrm{P}(A\mid B)=\dfrac{\mathrm{P}(AB)}{\mathrm{P}(B)}=\dfrac{\frac{6}{25}}{\frac{16}{25}}=\dfrac{3}{8}.$$
	\end{enumEX}
	}
\end{bt}

\begin{bt}
Chuồng I có 5 con gà mái, 2 con gà trống. Chuồng II có 3 con gà mái, 5 con gà trống. Bác Mai bắt một con gà trong số đó theo cách sau: Bác tung một con xúc xắc cân đối, đồng chất. Nếu số chấm chia hết cho 3 thì bác chọn chuồng I, nếu số chấm không chia hết cho 3 thì bác chọn chuồng II. Sau đó, từ chuồng đã chọn bác bắt ngẫu nhiên một con gà. Tính xác suất để bác Mai bắt được con gà mái.
	\loigiai{
	Gọi $A$ là biến cố: \lq\lq  Bác Mai bắt được con gà mái\rq\rq.\\
	Gọi $B_1$ là biến cố: \lq\lq  tung con xúc xắc được số chấm chia hết cho $3$\rq\rq\, suy ra $\mathrm{P}(B_1)=\dfrac{2}{6}=\dfrac{1}{3}$.\\
	Gọi $B_2$ là biến cố: \lq\lq  tung con xúc xắc được số chấm không chia hết cho $3$\rq\rq\, suy ra $\mathrm{P}(B_2)=\dfrac{4}{6}=\dfrac{2}{3}$.\\
	Ta có xác suất bắt được gà mái từ chuồng I là $\mathrm{P}\left(A\mid B_1\right)=\dfrac{5}{7}$.\\
	Ta có xác suất bắt được gà mái từ chuồng II là $\mathrm{P}\left(A\mid B_2\right)=\dfrac{3}{8}$.\\
	Áp dụng công thức xác suất toàn phần ta có
	$$\mathrm{P}(A)=\mathrm{P}(B_1)\cdot \mathrm{P}(A\mid B_1)+\mathrm{P}(B_2)\cdot \mathrm{P}(A\mid B_2)=\dfrac{1}{3}\cdot \dfrac{5}{7}+\dfrac{2}{3}\cdot \dfrac{3}{8}=\dfrac{41}{84}.$$
	Vậy xác suất để bác Mai bắt được con gà mái là $\dfrac{41}{84}$.
	}
\end{bt}

\begin{bt}
	Một loại vaccine được tiêm ở địa phương $X$. Người có bệnh nền thì với xác suất $0{,}35$ có phản ứng phụ sau tiêm, người không có bệnh nền thì chỉ có phản ứng phụ sau tiêm với xác suất $0{,}16$. Chọn ngẫu nhiên một người được tiêm vaccine và người này có phản ứng phụ. Tính xác suất để người này có bệnh nền, biết rằng tỉ lệ người có bệnh nền ở địa phương $X$ là $18\%$.
	\loigiai{
	Gọi $A$ là biến cố: \lq\lq  Người được chọn có bệnh nền\rq\rq\, và $B$ là biến cố: \lq\lq  Người này có phản ứng phụ sau tiêm\rq\rq.\\
	Ta có $\mathrm{P}(A)=0{,}18$; $\mathrm{P}(\overline{A})=0{,}82$.\\
	$\mathrm{P}(B\mid A)$ là xác suất để một người bệnh có phản ứng sau tiêm với điều kiện có bệnh nền, suy ra $\mathrm{P}(B\mid A)=0{,}35$.\\
	$\mathrm{P}(B\mid \overline{A})$ là xác suất để một người bệnh có phản ứng sau tiêm với điều kiện không có bệnh nền, suy ra $\mathrm{P}(B\mid \overline{A})=0{,}16$.\\
	Theo công thức Bayes, ta được
	$$\mathrm{P}(A\mid B)=\dfrac{\mathrm{P}(A)\cdot \mathrm{P}(B\mid A)}{\mathrm{P}(A)\cdot \mathrm{P}(B\mid A)+\mathrm{P}(\overline{A})\cdot \mathrm{P}(B\mid \overline{A})}=\dfrac{0{,}18\cdot 0{,}35}{0{,}18\cdot 0{,}35+0{,}82\cdot 0{,}16}=\dfrac{315}{971}.$$
	}
\end{bt}

\subsubsection{Bài tập trắc nghiệm}
\setcounter{ex}{0}
\Opensolutionfile{ans}[ans12]
\begin{ex}%[2D5N2-1]
	Cho hai biến cố $A$ và $B$ có $\mathrm{P}(A)=0\text{,}8$; $\mathrm{P}(B)=0\text{,}5$ và $\mathrm{P}(A B)=0\text{,}2$. Xác suất của biến cố $A$ với điều kiện $B$ là
	\choice
	{\True$0\text{,}4$}
	{$0\text{,}5$}
	{$0\text{,}25$}
	{$0\text{,}625$}
	\loigiai{Ta có
	$\mathrm{P}(A\mid B)=\dfrac{\mathrm{P}(A B)}{\mathrm{P}(B)}=\dfrac{0\text{,}2}{0\text{,}5}=0\text{,}4$.}
\end{ex}

\begin{ex}%[2D5N2-1]
	Cho hai biến cố $A$ và $B$ có $\mathrm{P}(A)=0\text{,}8$; $\mathrm{P}(B)=0\text{,}5$ và $\mathrm{P}(A B)=0\text{,}2$. Xác suất biến cố $B$ không xảy ra với điều kiện biến cố $A$ xảy ra là
	\choice
	{$0\text{,}6$}
	{$0\text{,}5$}
	{\True$0\text{,}75$}
	{$0\text{,}25$}
	\loigiai{Ta có $\mathrm{P}(B\mid A)=\dfrac{\mathrm{P}(A B)}{\mathrm{P}(A)}=\dfrac{0\text{,}2}{0\text{,}8}=0\text{,}25$.\\
	Mặt khác $\mathrm{P}(A)>0\Rightarrow \mathrm{P}(\overline{B}\mid A)=1-\mathrm{P}(B\mid A)=1-0\text{,}25=0\text{,}75$.}
\end{ex}

\begin{ex}%[2D5H2-1]
	Cho hai biến cố $A$ và $B$ có $\mathrm{P}(A)=0\text{,}8$; $\mathrm{P}(B)=0\text{,}5$ và $\mathrm{P}(A B)=0\text{,}2$. Giá trị của biểu thức $
	\dfrac{\mathrm{P}(A \mid B)}{\mathrm{P}(A)}-\dfrac{\mathrm{P}(B \mid A)}{\mathrm{P}(B)}$ là
	\choice
	{$-0\text{,}5$}
	{\True$0$}
	{$0\text{,}5$}
	{$1$}
	\loigiai{Ta có $$
	\dfrac{\mathrm{P}(A \mid B)}{\mathrm{P}(A)}-\dfrac{\mathrm{P}(B \mid A)}{\mathrm{P}(B)}=\dfrac{\mathrm{P}(AB)}{\mathrm{P}(A)\cdot \mathrm{P}(B)}- \dfrac{\mathrm{P}(BA)}{\mathrm{P}(B)\cdot \mathrm{P}(A)}=\dfrac{0\text{,}2}{0\text{,}8\cdot 0\text{,}5}-\dfrac{0\text{,}2}{0\text{,}8\cdot 0\text{,}5}=0.$$}
\end{ex}

\begin{ex}
	Cho hai biến cố xung khắc $A$, $B$ với $\mathrm{P}(A)=0{,}2$; $\mathrm{P}(B)=0{,}4$. Khi đó, $\mathrm{P}(A\mid B)$ bằng
	\choice{$0{,}5$}
	{$0{,}2$}
	{$0{,}4$}
	{$0$}
	\loigiai{
	Vì $A$ và $B$ là 2 biến cố xung khắc nên $\mathrm{P}(A\cap B)=0$.\\
	$\Rightarrow \mathrm{P}(A|B)=\dfrac{\mathrm{P}(A\cap B)}{\mathrm{P}(B)}=\dfrac{0}{0{,}4}=0$.
	}
\end{ex}

\begin{ex}
	Cho $\mathrm{P}(A)=\dfrac{2}{5}$; $\mathrm{P}\left( B\mid A\right)=\dfrac{1}{3}$. Giá trị của $\mathrm{P}(AB)$ là
	\choice
	{\True $\dfrac{2}{15} $}
	{$ \dfrac{3}{16}$}
	{$ \dfrac{1}{5}$}
	{$ \dfrac{4}{15}$}
	\loigiai{
	$\mathrm{P}(AB)=\mathrm{P}(A)\cdot \mathrm{P}\left( B\mid A\right)=\dfrac{2}{5}\cdot \dfrac{1}{3}=\dfrac{2}{15}$.
	}
\end{ex}

\begin{ex}
	Cho $\mathrm{P}(A)=\dfrac{2}{5}$; $\mathrm{P}\left(B\mid \overline{A}\right)=\dfrac{1}{4}$. Giá trị của $\mathrm{P}\left(B\overline{A}\right)$ là
	\choice
	{$\dfrac{1}{7} $}
	{$ \dfrac{4}{19}$}
	{$ \dfrac{4}{21}$}
	{\True $ \dfrac{3}{20}$}
	\loigiai{
	$\mathrm{P}\left(B\overline{A}\right)=\mathrm{P}\left( B\mid A\right)\cdot \mathrm{P}\left( \overline{A}\right) =\dfrac{1}{4}\cdot \dfrac{3}{5}=\dfrac{3}{20}$.
	}
\end{ex}

\begin{ex}
	Cho $\mathrm{P}(A)=\dfrac{2}{5}$; $\mathrm{P}\left( B\mid A\right)=\dfrac{1}{3}$; $\mathrm{P}\left(B\mid \overline{A}\right)=\dfrac{1}{4}$. Giá trị của $\mathrm{P}(B)$ là
	\choice
	{$\dfrac{19}{60} $}
	{\True $ \dfrac{17}{60}$}
	{$ \dfrac{9}{20}$}
	{$ \dfrac{7}{30}$}
	\loigiai{Áp dụng công thức Bayes ta có
	$$\mathrm{P}(A\mid B)=\dfrac{\mathrm{P}(A)\cdot \mathrm{P}(B\mid A)}{\mathrm{P}(A)\cdot \mathrm{P}(B\mid A)+\mathrm{P}\left( \overline{A}\right) \cdot \mathrm{P}\left( B\mid \overline{A}\right) }=\dfrac{\frac{2}{5}\cdot \frac{1}{3}}{\frac{2}{5}\cdot \frac{1}{3}+\frac{3}{5}\cdot \frac{1}{4}}=\dfrac{8}{17}.$$
	Khi đó
	$\mathrm{P}(B)=\dfrac{\mathrm{P}(AB)}{\mathrm{P}(A\mid B)}=\dfrac{2}{15}:\dfrac{8}{17}=\dfrac{17}{60}$.
	}
\end{ex}

\begin{ex}
	Cho $A$, $B$ là các biến cố của một phép thử $T$. Biết rằng $P(B)>0$, xác suất của biến cố $A$ với điều kiện biến cố $B$ đã xảy ra được tính theo công thức nào sau đây?
	\choice
	{$P(A|B)=\dfrac{P(A)}{P(B)}$}
	{$P(A|B)=\dfrac{P(A)}{P(A B)}$}
	{\True $P(A| B)=\dfrac{P(A B)}{P(B)}$}
	{$P(A|B)=\dfrac{P(A B)}{P(A) \cdot P(B)}$}
	\loigiai{ Dựa theo công thức tính xác suất biến cố $A$ với điều kiện $B$ thì $P(A| B)=\dfrac{P(A B)}{P(B)}$ là đáp án đúng.
	}
\end{ex}

\begin{ex}
	Cho $A$, $B$ là các biến cố thỏa mãn $P(\overline{AB})=0{,}35$, $P(A)=0{,}25$, $P(B)=0{,}6$. Giá trị của $P(A|B)$ bằng
	\choice
	{$\dfrac{1}{5}$}
	{\True$\dfrac{1}{3}$}
	{$\dfrac{7}{15}$}
	{$\dfrac{2}{3}$}
	\loigiai{Ta có $P\left( {\overline {AB} } \right) = P\left( {\overline A } \right)P\left( {\overline B |\overline A } \right) \Rightarrow P\left( {\overline B |\overline A } \right) = \dfrac{{P\left( {\overline {AB} } \right)}}{{P\left( {\overline A } \right)}} = \dfrac{{0{,}35}}{{0{,}75}} = \dfrac{7}{{15}}.$ \\
	Suy ra $P\left( {B|\overline A } \right) = 1 - \dfrac{7}{{15}} = \dfrac{8}{{15}}$.\\
	Theo công thức xác suất toàn phần, ta có
	$$\begin{array}{l}
	P\left( B \right) = P\left( {B|A} \right)P\left( A \right) + P\left( {B|\overline A } \right)P\left( {\overline A } \right)\\
	\Rightarrow P\left( {B|A} \right) = \dfrac{{P\left( B \right) - P\left( {B|\overline A } \right)P\left( {\overline A } \right)}}{{P\left( A \right)}} = \dfrac{{0{,}6 - \dfrac{8}{{15}} \cdot 0{,}75}}{{0{,}25}} = 0{,}8.
	\end{array}.$$
	Theo công thức Bayes, ta được
	$$P\left( {A|B} \right) = \dfrac{{P\left( A \right)P\left( {B|A} \right)}}{{P\left( B \right)}} = \dfrac{{0{,}25 \cdot 0{,}8}}{{0{,}6}} = \dfrac{1}{3}.$$}
\end{ex}

\begin{ex}%[2D5V2-1]
	Một nhà máy thực hiện khảo sát toàn bộ công nhân về sự hài lòng của họ về điều kiện làm việc tại phân xưởng. Kết quả khảo sát như sau:
	\begin{center}
	\begin{tabular}{|c|c|c|}
	\hline
	\diagbox {Khảo sát công nhân}{Kết quả khảo sát}	& Hài lòng & Không hài lòng \\
	\hline
	\textbf{Số công nhân phân xưởng I}	& $37$ & $13$ \\
	\hline
	\textbf{Số công nhân phân xưởng II}	& $63$ & $27$\\
	\hline
	\end{tabular}
	\end{center}
	Gặp ngẫu nhiên một công nhân của nhà máy. Gọi $A$ là biến cố \lq\lq  Công nhân đó làm việc tại phân xưởng I \rq\rq.
	Xác suất của biến cố $A$ là
	\choice
	{$\dfrac{37}{140}$}
	{$\dfrac{37}{50}$}
	{\True$\dfrac{5}{14}$}
	{$\dfrac{1}{2}$}
	\loigiai{
	\begin{itemize}
	\item Không gian mẫu $n(\Omega)=50+90=140$.
	\item Ta có $n(A)=37+13=50$.
	\item Xác suất của biến cố $A$ là $\mathrm{P}(A)=\dfrac{n(A)}{n(\Omega)}=\dfrac{50}{140}=\dfrac{5}{14}$.
	\end{itemize}
	}
\end{ex}

\begin{ex}%[2D5V2-1]
	Một nhà máy thực hiện khảo sát toàn bộ công nhân về sự hài lòng của họ về điều kiện làm việc tại phân xưởng. Kết quả khảo sát như sau:
	\begin{center}
	\begin{tabular}{|c|c|c|}
	\hline
	\diagbox {Khảo sát công nhân}{Kết quả khảo sát}	& Hài lòng & Không hài lòng \\
	\hline
	\textbf{Số công nhân phân xưởng I}	& $37$ & $13$ \\
	\hline
	\textbf{Số công nhân phân xưởng II}	& $63$ & $27$\\
	\hline
	\end{tabular}
	\end{center}
	Gặp ngẫu nhiên một công nhân của nhà máy. Gọi $A$ là biến cố \lq\lq  Công nhân đó làm việc tại phân xưởng I \rq\rq \, và $B$ là biến cố \lq\lq  Công nhân đó hài lòng với điều kiện làm việc tại phân xưởng\rq\rq.
	Xác suất của biến cố $B$ với điều kiện $A$ không xảy ra là
	\choice
	{$\dfrac{2}{7}$}
	{$0\text{,}9$}
	{\True$0\text{,}7$}
	{$\dfrac{9}{20}$}
	\loigiai{
	\begin{itemize}
	\item Ta có $n(B\cap \overline{A})=63$ và $n(\overline{A})=63+27=90$.
	\item $\mathrm{P}(B\mid \overline{A})=\dfrac{n(B\cap \overline{A})}{n(\overline{A})}=\dfrac{63}{90}=0\text{,}7$.
	\end{itemize}
	}
\end{ex}

\begin{ex}
	Người ta nhập hai lô hàng vào kho. Lô thứ nhất chứa $10$ sản phẩm, trong đó có $3$ phế phẩm. Lô thứ hai có $4$ phế phẩm và $8$ sản phẩm tốt. Chọn ngẫu nhiên một sản phẩm. Xác suất chọn được một sản phẩm tốt là
	\choice
	{\True $\dfrac{15}{22}$}
	{$\dfrac{7}{15}$}
	{$\dfrac{7}{22}$}
	{$\dfrac{83}{242}$}
	\loigiai{
	%Bài này thực sự không cần thiết phải dùng xác suất có điều kiện vì:
	Gọi $A$ là biến cố “Chọn được sản phẩm tốt”, theo đề ra, kho có $22$ sản phẩm, trong đó có $15$ sản phẩm tốt nên:\\
	$n(A)=15$, $n(\Omega)=22$. Vậy $P(A)=\dfrac{n(A)}{n(\Omega)}=\dfrac{15}{22}$.
	}
\end{ex}

\begin{ex}
	An có một túi gồm một số chiếc kẹo cùng loại, chỉ khác màu, trong đó có $6$ chiếc kẹo sô-cô-la đen, còn lại là $4$ chiếc kẹo sô-cô-la trắng. An lấy ngẫu nhiên $1$ chiếc kẹo trong túi để cho Bình, rồi lại lấy ngẫu nhiên tiếp 1 chiếc kẹo nữa trong túi và cũng đưa cho Bình. Xác suất để Bình nhận được $2$ chiếc kẹo sô-cô-la đen là
	\choice
	{\True $\dfrac{1}{3} $}
	{$ \dfrac{1}{4}$}
	{$ \dfrac{2}{5}$}
	{$ \dfrac{3}{7}$}
	\loigiai{
	Gọi $A$ là biến cố: \lq\lq  An lấy lần 1 được 1 chiếc kẹo sô-cô-la đen\rq\rq\, và $B$ là biến cố: \lq\lq  An lấy lần 2 được 1 chiếc kẹo sô-cô-la đen\rq\rq.\\
	Khi đó $AB$ là biến cố: \lq\lq  Cả hai lần đều lấy được kẹo sô-cô-la đen\rq\rq.\\
	Ta có $\mathrm{P}(A)=\dfrac{n(A)}{n(\Omega)}=\dfrac{6}{10}$.\\
	Sau khi lấy 1 chiếc kẹo sô-cô-la đen thì xác suất để chọn 1 chiếc kẹo sô-cô-la đen trong hộp đựng $5$ chiếc kẹo sô-cô-la đen, còn lại là $4$ chiếc kẹo sô-cô-la trắng là $\mathrm{P}(B\mid A)=\dfrac{5}{9}$.\\
	Khi đó $\mathrm{P}(AB)=\mathrm{P}(A)\cdot \mathrm{P}(B\mid A)=\dfrac{6}{10}\cdot \dfrac{5}{9}=\dfrac{1}{3}$.\\
	Xác suất để Bình nhận được $2$ chiếc kẹo sô-cô-la đen là $\dfrac{1}{3}$.
	}
\end{ex}

\begin{ex}
	An có một túi gồm một số chiếc kẹo cùng loại, chỉ khác màu, trong đó có $6$ chiếc kẹo sô-cô-la đen, còn lại là $4$ chiếc kẹo sô-cô-la trắng. An lấy ngẫu nhiên $1$ chiếc kẹo trong túi để cho Bình, rồi lại lấy ngẫu nhiên tiếp 1 chiếc kẹo nữa trong túi và cũng đưa cho Bình. Xác suất để Bình nhận được chiếc kẹo sô-cô-la đen ở lần thứ nhất và chiếc kẹo sô-cô-la trắng ở lần thứ hai là
	\choice
	{$\dfrac{1}{5} $}
	{$ \dfrac{3}{16}$}
	{$ \dfrac{1}{4}$}
	{$ \dfrac{4}{17}$}
	\loigiai{
	Gọi $A$ là biến cố: \lq\lq  Bình nhận được chiếc kẹo ở lần đầu tiên là đen\rq\rq\, và $B$ là là biến cố: \lq\lq  Bình nhận được chiếc kẹo ở lần 2 là trắng \rq\rq.\\
	Ta có $\mathrm{P}(A)=\dfrac{3}{5}$ và $\mathrm{P}(B\mid A)=\dfrac{4}{9}$.\\
	Khi đó $\mathrm{P}(B\mid A)=\dfrac{\mathrm{P}(AB)}{\mathrm{P}(A)}\Rightarrow \mathrm{P}(AB)=\mathrm{P}(B\mid A)\cdot \mathrm{P}(A)=\dfrac{3}{5}\cdot \dfrac{4}{9}=\dfrac{4}{15}$.
	}
\end{ex}

\begin{ex}
	Một bệnh viện có hai phòng khám là phòng A và phòng B với khả năng lựa chọn của bệnh nhân là như nhau. Tỉ lệ bệnh nhân nam có ở phòng A và phòng B lần lượt là $60\%$ và $40\%$. Một người bệnh được chọn ngẫu nhiêu từ hai phòng khám và biết người này là nam, xác suất để người bệnh được chọn đến từ phòng A là
	\choice
	{\True $0{,}6$}
	{$0{,}5$}
	{$0{,}4$}
	{$0{,}3$}
	\loigiai{Một người bệnh được chọn ngẫu nhiên từ hai phòng khám.\\
	Gọi $X$ là biến cố \lq \lq Người đó đến từ phòng khám A\rq \rq \, và $Y$, $\overline{Y}$ lần lượt là biến cố \lq \lq Người đó là nam\rq \rq \; và \lq \lq Người đó không là nam\rq \rq.\\
	Ta có sơ đồ hình cây sau
\begin{center}
	\begin{tikzpicture}[>=stealth,scale=0.7]
	%Khung 1
	\draw (-3.8,-1) rectangle (2.2,0);
	\draw (-0.8,-0.5) node{Bệnh nhân được chọn} ;
	%Mui ten 1,2
	\draw [->] (2.2,-0.5)--(3.8,1.6) node[pos=0.5,sloped,above]{$0{,}5$};
	\draw [->] (2.2,-0.5)--(3.8,-2.6) node[pos=0.5,sloped,below]{$0{,}5$};
	%Khung 2.1
	\draw (3.8,1.1) rectangle (5.1,2.1);
	\draw (8.9/2,1.6) node{$X$} ;
	%Khung 2.2
	\draw (3.8,-2.1) rectangle (5.1,-3.1);
	\draw (8.9/2,-2.6) node{$\overline{X}$} ;
	%Mui ten 3,4
	\draw [->] (5.1,1.6)--(6.5,2.6) node[pos=0.5,sloped,above]{$0{,}6$};
	\draw [->] (5.1,1.6)--(6.5,0.6) node[pos=0.5,sloped,below]{$0{,}4$};
	%Mui ten 5,6
	\draw [->] (5.1,-2.6)--(6.5,-1.6) node[pos=0.5,sloped,above]{$0{,}4$};
	\draw [->] (5.1,-2.6)--(6.5,-3.6) node[pos=0.5,sloped,below]{$0{,}6$};
	%Khung 3.1
	\draw (6.5,2.2) rectangle (7.7,3.2);
	\draw (7.1,5.4/2) node{$Y$} ;
	%Khung 3.2
	\draw (6.5,1.2) rectangle (7.7,0.2);
	\draw (7.1,1.4/2) node{$\overline{Y}$} ;
	%Khung 3.3
	\draw (6.5,-1.1) rectangle (7.7,-2.1);
	\draw (7.1,-3.2/2) node{$Y$} ;
	%Khung 3.3
	\draw (6.5,-2.9) rectangle (7.7,-3.9);
	\draw (7.1,-3.4) node{$\overline{Y}$} ;
	%Kết quả
	\draw (9.5,3.7) node{\textbf{Kết quả}};
	\draw (9.5,2.7) node{$XY$};
	\draw (9.5,0.7) node{$X \overline{Y}$};
	\draw (9.5,-1.6) node{$\overline{X}Y$};
	\draw (9.5,-3.4) node{$\overline{X}\overline{Y}$};
	%Xác suất
	\draw (12.5,3.7) node{\textbf{Xác suất}};
	\draw (12.5,2.7) node{$0{,}3$};
	\draw (12.5,0.7) node{$0{,}2$};
	\draw (12.5,-1.6) node{$0{,}2$};
	\draw (12.5,-3.4) node{$0{,}3$};
	\end{tikzpicture}
\end{center}
	Theo công thức Bayes, ta có $$P(X|Y)=\dfrac{P(X)P(Y|X)}{P(X)P(Y|X)+P(\overline{X})P(Y|\overline{X})}=\dfrac{0{,}3}{0{,}3+0{,}2}=0{,}6.$$
	Vậy với một người bệnh được chọn ngẫu nhiêu từ hai phòng khám và biết người này là nam, xác suất để người đó đến từ phòng A là $0{,}6$.}
\end{ex}

\begin{ex}
	Một bệnh viện đang xét nghiệm cho một số bệnh nhân để xác định liệu họ có nhiễm virus $X$ hay không. Xác suất để một bệnh nhân bị nhiễm virus $X$ là $0{,}05$. Khi xét nghiệm, nếu một bệnh nhân bị nhiễm thì xác suất để kết quả xét nghiệm dương tính là $0{,}95$. Nếu một bệnh nhân không bị nhiễm thì xác suất để kết quả xét nghiệm âm tính là $0{,}98$. Một bệnh nhân được chọn ngẫu nhiên và có kết quả xét nghiệm dương tính. Xác suất để bệnh nhân đó thực sự bị nhiễm virus $X$ là
	\choice
	{$\dfrac{133}{2000}$}
	{$\dfrac{19}{400}$}
	{\True $\dfrac{5}{7}$}
	{$\dfrac{2}{7}$}
	\loigiai{Một bệnh nhân đến một bệnh viên để xét nghiệm.\\
	Gọi $A$ là biến cố \lq \lq Bệnh nhân bị nhiễm virus $X$\rq \rq \, và $B$, $\overline{B}$ lần lượt là biến cố \lq \lq Kết quả xét nghiệm dương tính\rq \rq \; và \lq \lq Kết quả xét nghiệm âm tính\rq \rq.\\
	Ta xét sơ đồ hình cây như sau
\begin{center}
		\begin{tikzpicture}[>=stealth,scale=0.7]
	%Khung 1
	\draw (-4.8,-1) rectangle (2.2,0);
	\draw (-1.3,-0.5) node{Bệnh nhân được xét nghiệm} ;
	%Mui ten 1,2
	\draw [->] (2.2,-0.5)--(3.8,1.6) node[pos=0.5,sloped,above]{$0{,}05$};
	\draw [->] (2.2,-0.5)--(3.8,-2.6) node[pos=0.5,sloped,below]{$0{,}95$};
	%Khung 2.1
	\draw (3.8,1.1) rectangle (5.1,2.1);
	\draw (8.9/2,1.6) node{$A$} ;
	%Khung 2.2
	\draw (3.8,-2.1) rectangle (5.1,-3.1);
	\draw (8.9/2,-2.6) node{$\overline{A}$} ;
	%Mui ten 3,4
	\draw [->] (5.1,1.6)--(6.5,2.6) node[pos=0.5,sloped,above]{$0{,}95$};
	\draw [->] (5.1,1.6)--(6.5,0.6) node[pos=0.5,sloped,below]{$0{,}05$};
	%Mui ten 5,6
	\draw [->] (5.1,-2.6)--(6.5,-1.6) node[pos=0.5,sloped,above]{$0{,}02$};
	\draw [->] (5.1,-2.6)--(6.5,-3.6) node[pos=0.5,sloped,below]{$0{,}98$};
	%Khung 3.1
	\draw (6.5,2.2) rectangle (7.7,3.2);
	\draw (7.1,5.4/2) node{$B$} ;
	%Khung 3.2
	\draw (6.5,1.2) rectangle (7.7,0.2);
	\draw (7.1,1.4/2) node{$\overline{B}$} ;
	%Khung 3.3
	\draw (6.5,-1.1) rectangle (7.7,-2.1);
	\draw (7.1,-3.2/2) node{$B$} ;
	%Khung 3.3
	\draw (6.5,-2.9) rectangle (7.7,-3.9);
	\draw (7.1,-3.4) node{$\overline{B}$} ;
	%Kết quả
	\draw (9.5,3.7) node{\textbf{Kết quả}};
	\draw (9.5,2.7) node{$AB$};
	\draw (9.5,0.7) node{$A\overline{B}$};
	\draw (9.5,-1.6) node{$\overline{A}B$};
	\draw (9.5,-3.4) node{$\overline{A}\overline{B}$};
	%Xác suất
	\draw (12.5,3.7) node{\textbf{Xác suất}};
	\draw (12.5,2.7) node{$0{,}0475$};
	\draw (12.5,0.7) node{$0{,}0025$};
	\draw (12.5,-1.6) node{$0{,}019$};
	\draw (12.5,-3.4) node{$0{,}931$};
	\end{tikzpicture}
\end{center}
	Theo công thức Bayes, ta có $$P(A|B)=\dfrac{P(A)P(B|A)}{P(A)P(B|A)+P(\overline{A})P(B|\overline{A})}=\dfrac{0{,}0475}{0{,}0475+0{,}019}=\dfrac{5}{7}.$$
	Vậy với một bệnh nhân có kết quả xét nghiệm dương tính, xác suất để bệnh nhân đó thực sự bị nhiễm virus $X$ là $\dfrac{5}{7}$.}
\end{ex}

\begin{ex}
	Ở một địa phương $X$, xác suất để một người lớn trên $40$ tuổi mắc bệnh ung thư là $0{,}05$. Xác suất bác sĩ chẩn đoán đúng một người mắc bệnh ung thư là $0{,}78$ và chẩn đoán sai (không bị ung thư nhưng được chẩn đoán mắc bệnh) là $0{,}06$. Xác suất để một người thật sự mắc bệnh ung thư khi nhận được kết quả chẩn đoán bị ung thư bằng
	\choice
	{\True$0{,}40625$}
	{$0{,}096$}
	{$0{,}904$}
	{$0{,}59375$}
	\loigiai{Một bệnh nhân trên 40 tuổi ở địa phương X đến bác sĩ để khám bệnh ung thư.\\
	Gọi $A$ là biến cố \lq \lq Người đó mắc bệnh ung thư\rq \rq \, và $B$, $\overline{B}$ lần lượt là biến cố \lq \lq Bác sĩ chẩn đoán người đó bị ung thư\rq \rq \;và \lq \lq Bác sĩ chẩn đoán người đó không bị ung thư\rq \rq.\\
	Ta xét sơ đồ hình cây như sau
\begin{center}
		\begin{tikzpicture}[>=stealth,scale=0.7]
	%Khung 1
	\draw (-4.7,-1) rectangle (2.2,0);
	\draw (-1.3,-0.5) node{Bệnh nhân được chẩn đoán} ;
	%Mui ten 1,2
	\draw [->] (2.2,-0.5)--(3.8,1.6) node[pos=0.5,sloped,above]{$0{,}05$};
	\draw [->] (2.2,-0.5)--(3.8,-2.6) node[pos=0.5,sloped,below]{$0{,}95$};
	%Khung 2.1
	\draw (3.8,1.1) rectangle (5.1,2.1);
	\draw (8.9/2,1.6) node{$A$} ;
	%Khung 2.2
	\draw (3.8,-2.1) rectangle (5.1,-3.1);
	\draw (8.9/2,-2.6) node{$\overline{A}$} ;
	%Mui ten 3,4
	\draw [->] (5.1,1.6)--(6.5,2.6) node[pos=0.5,sloped,above]{$0{,}78$};
	\draw [->] (5.1,1.6)--(6.5,0.6) node[pos=0.5,sloped,below]{$0{,}22$};
	%Mui ten 5,6
	\draw [->] (5.1,-2.6)--(6.5,-1.6) node[pos=0.5,sloped,above]{$0{,}06$};
	\draw [->] (5.1,-2.6)--(6.5,-3.6) node[pos=0.5,sloped,below]{$0{,}94$};
	%Khung 3.1
	\draw (6.5,2.2) rectangle (7.7,3.2);
	\draw (7.1,5.4/2) node{$B$} ;
	%Khung 3.2
	\draw (6.5,1.2) rectangle (7.7,0.2);
	\draw (7.1,1.4/2) node{$\overline{B}$} ;
	%Khung 3.3
	\draw (6.5,-1.1) rectangle (7.7,-2.1);
	\draw (7.1,-3.2/2) node{$B$} ;
	%Khung 3.3
	\draw (6.5,-2.9) rectangle (7.7,-3.9);
	\draw (7.1,-3.4) node{$\overline{B}$} ;
	%Kết quả
	\draw (9.5,3.7) node{\textbf{Kết quả}};
	\draw (9.5,2.7) node{$AB$};
	\draw (9.5,0.7) node{$A\overline{B}$};
	\draw (9.5,-1.6) node{$\overline{A}B$};
	\draw (9.5,-3.4) node{$\overline{A}\overline{B}$};
	%Xác suất
	\draw (12.5,3.7) node{\textbf{Xác suất}};
	\draw (12.5,2.7) node{$0{,}039$};
	\draw (12.5,0.7) node{$0{,}011$};
	\draw (12.5,-1.6) node{$0{,}057$};
	\draw (12.5,-3.4) node{$0{,}893$};
	\end{tikzpicture}
\end{center}
	Theo công thức Bayes, ta có $$P(A|B)=\dfrac{P(A)P(B|A)}{P(A)P(B|A)+P(\overline{A})P(B|\overline{A})}=\dfrac{0{,}039}{0{,}039+0{,}057}=0{,}40625.$$
	Vậy xác suất để một người thật sự mắc bệnh ung thư khi nhận được kết quả chẩn đoán bị ung thư bằng $0{,}40625$.}
\end{ex}
\Closesolutionfile{ans}
% % \newpage
% \setcounter{ex}{0}
\Opensolutionfile{ans}[ans/ans-0-B15]
%\TN
%%==========PHẦN 1=============================================
%%==========Câu 1
\begin{ex}%[2D6H1-2]
	Cho $A$, $B$ là các biến cố của một phép thử $T$. Biết rằng $\mathrm{P}(B)>0$, xác suất của biến cố $A$ với điều kiện biến cố $B$ đã xảy ra được tính theo công thức nào sau đây?
	\choice
	{$\mathrm{P}(A\mid B)=\dfrac{\mathrm{P}(A)}{\mathrm{P}(B)}$}
	{$\mathrm{P}(A\mid B)=\dfrac{\mathrm{P}(A)}{\mathrm{P}(A B)}$}
	{\True $\mathrm{P}(A\mid B)=\dfrac{\mathrm{P}(A B)}{\mathrm{P}(B)}$}
	{$\mathrm{P}(A\mid B)=\dfrac{\mathrm{P}(A B)}{\mathrm{P}(A) \cdot \mathrm{P}(B)}$}
	\loigiai{ Dựa theo công thức tính xác suất biến cố $A$ với điều kiện $B$ thì $\mathrm{P}(A\mid B)=\dfrac{\mathrm{P}(A B)}{\mathrm{P}(B)}$ là đáp án đúng.
	}
\end{ex}
%%==========Câu 2
\begin{ex}%[2D6H1-2]
	Cho hai biến cố độc lập $A$, $B$ với $\mathrm{P}(A)=0{,}3$; $\mathrm{P}(B)=0{,}4$. Khi đó, $\mathrm{P}(A\mid B)$ bằng
	\choice{$0{,}7$}
	{$0{,}12$}
	{$0{,}4$}
	{\True $0{,}3$}
	\loigiai{
	Vì $A$ và $B$ là $2$ biến cố độc lập nên $\mathrm{P}(A\cap B)=\mathrm{P}(A)\cdot \mathrm{P}(B)=0{,}12$.\\
	$$\Rightarrow \mathrm{P}(A \mid B)=\dfrac{\mathrm{P}(A\cap B)}{\mathrm{P}(B)}=\mathrm{P}(A)=\dfrac{0{,}12}{0{,}4}=0{,}3.$$
	}
\end{ex}
%%==========Câu 3
\begin{ex}%[2D6H1-2]
	Cho hai biến cố xung khắc $A$, $B$ với $\mathrm{P}(A)=0{,}2$; $\mathrm{P}(B)=0{,}4$. Khi đó, $\mathrm{P}(A\mid B)$ bằng
	\choice{$0{,}5$}
	{$0{,}2$}
	{$0{,}4$}
	{\True $0$}
	\loigiai{
	Vì $A$ và $B$ là $2$ biến cố xung khắc nên $\mathrm{P}(A\cap B)=0$.\\
	$$\Rightarrow \mathrm{P}(A\mid B)=\dfrac{\mathrm{P}(A\cap B)}{\mathrm{P}(B)}=\dfrac{0}{0{,}4}=0.$$
	}
\end{ex}
%%==========Câu 4
\begin{ex}%[2D6H1-2]
	Cho $\mathrm{P}(A)=\dfrac{2}{5}$; $\mathrm{P}\left( B\mid A\right)=\dfrac{1}{3}$. Giá trị của $\mathrm{P}(AB)$ là 
	\choice
	{\True $\dfrac{2}{15} $}
	{$ \dfrac{3}{16}$}
	{$ \dfrac{1}{5}$}
	{$ \dfrac{4}{15}$}
	\loigiai{
	Ta có $\mathrm{P}(AB)=\mathrm{P}(A)\cdot \mathrm{P}\left( B\mid A\right)=\dfrac{2}{5}\cdot \dfrac{1}{3}=\dfrac{2}{15}$.
	}
\end{ex}
%%==========Câu 5
\begin{ex}%[2D6H1-2]
	Cho $\mathrm{P}(A)=\dfrac{2}{5}$; $\mathrm{P}\left(B\mid \overline{A}\right)=\dfrac{1}{4}$. Giá trị của $\mathrm{P}\left(B\overline{A}\right)$ là 
	\choice
	{$\dfrac{1}{7} $}
	{$ \dfrac{4}{19}$}
	{$ \dfrac{4}{21}$}
	{\True $ \dfrac{3}{20}$}
	\loigiai{
	Ta có $\mathrm{P}\left(B\overline{A}\right)=\mathrm{P}\left( B\mid \overline{A} \right)\cdot \mathrm{P}\left( \overline{A}\right) =\dfrac{1}{4}\cdot \dfrac{3}{5}=\dfrac{3}{20}$.
	}
\end{ex}
%%%%%----------Câu 1
\begin{ex}%[2D6N1-1]%[Võ Thanh Hiệp]
	Cho hai biến cố $A$ và $B$. Xác suất của biến cố $B$, tính trong điều kiện biết rằng biến cố $A$ đã xảy ra, được gọi là xác suất của $B$ với điều kiện $A$ kí hiệu là 
	\choice
	{$\mathrm{P}\left(A\mid B\right)$}
	{\True $\mathrm{P}\left(B\mid A\right)$}
	{$\mathrm{P}\left(AB\right)$}
	{$\mathrm{P}\left(B\right)$}
	\loigiai{
	Theo định nghĩa xác suất có điều kiện. Xác suất của $B$ với điều kiện $A$ kí hiệu là $\mathrm{P}\left(B\mid A\right)$.
	}	
\end{ex}
%%%%%----------Câu 2
\begin{ex}%[2D6N1-2]%[Võ Thanh Hiệp]
	Cho hai biến cố $A$ và $B$. Biết rằng xác suất của biến cố $A$ bằng $0{,}6$; xác suất của biến cố biến cố $B$ trong điều kiện biến cố $A$ đã xảy ra bằng $0{,}2$. Tính xác suất của $A$ và $B$ đều xảy ra. 
	\choice
	{\True $\dfrac{3}{25}$}
	{$\dfrac{3}{10}$}
	{$\dfrac{1}{3}$}
	{$\dfrac{2}{3}$}
	\loigiai{
	Ta có $\mathrm{P}\left(A\right) = 0{,}6$; $\mathrm{P}\left(B\mid A\right) = 0{,}2$.\\
	Suy ra 
	$\mathrm{P}\left(AB\right)=
	\mathrm{P}\left(A\right)\cdot \mathrm{P}\left(B\mid A\right)=0{,}6\cdot 0{,}2 =
	0{,}12 = \dfrac{3}{25} $.\\
	Vậy xác suất của $A$ và $B$ đều xảy ra là $\mathrm{P}\left(AB\right)= \dfrac{3}{25}$.
	}	
\end{ex}
%%%%%----------Câu 3
\begin{ex}%[2D6H1-2]%[Võ Thanh Hiệp]
	Gieo hai con xúc xắc cân đối đồng chất. Tính xác suất để tổng số chấm trên hai con xúc xắc bằng bằng $8$ nếu biết rằng ít nhất có một con xúc xắc xuất hiện mặt $3$ chấm.
	\choice
	{$\dfrac{5}{11}$}
	{$\dfrac{2}{5}$}
	{$\dfrac{11}{5}$}
	{\True $\dfrac{2}{11}$}
	\loigiai{
	Gọi $A$ là biến cố: \lq\lq  Tổng số chấm trên hai con xúc xắc bằng bằng $8$\rq\rq.\\
	Gọi $B$ là biến cố: \lq\lq  Có ít nhất có một con xúc xắc xuất hiện mặt $3$ chấm\rq\rq.\\
	Gieo hai con xúc xắc cân đối đồng chất, ta có $n\left(\Omega\right)=6\cdot 6 = 36$.\\
	$A=\{(2,6); (6,2); (3,5); (5,3); (4;4)\}$	$\Rightarrow n\left(A\right) = 5
	\Rightarrow \mathrm{P}\left(A\right)= \dfrac{5}{36}$.\\	
	$B=\{(3,1);(3,2); (3,3); (3,4); (3,5); (3,6); (1,3); (2,3); (4,3); (5,3); (6,3)\}$.\\
	$\Rightarrow n\left(B\right) =11 \Rightarrow \mathrm{P}\left(B\right)= \dfrac{11}{36}$.\\
	$AB=\{(3,5); (5,3)\}$
	$\Rightarrow n\left(AB\right) =2\Rightarrow \mathrm{P}\left(AB\right)=\dfrac{2}{36}=\dfrac{1}{18}$.\\
	Xác suất cần tính là 	
	$\mathrm{P}\left(A\mid B\right) =\dfrac{\mathrm{P}\left(AB\right) }{\mathrm{P}\left(B\right)}= \dfrac{2}{11}$.
	}	
\end{ex}
%%%%%----------Câu 4
\begin{ex}%[2D6H1-1]%[Võ Thanh Hiệp]
	Cho $A$ và $B$ là hai biến cố. Trong các mệnh đề sau, mệnh đề nào {\bf sai}?
	\choice
	{\True Với $\mathrm{P}\left(B\right)>0$. Khi đó $\mathrm{P}\left(A\mid B\right)=\mathrm{P}\left(B\right)\cdot \mathrm{P}\left(AB\right)$}
	{Nếu $A$ và $B$ là hai biến cố độc lập thì $\mathrm{P}\left(B\right)=\mathrm{P}\left(B \mid A\right)$}
	{Với $\mathrm{P}\left(B\right)>0$. Khi đó $\mathrm{P}\left(\overline{A}\mid B\right)= 1-\mathrm{P}\left(A\mid B\right)$}
	{Nếu $A$ và $B$ là hai biến cố độc lập thì $\mathrm{P}\left(A\mid B\right)=\mathrm{P}\left(A \mid \overline{B}\right)$}
	\loigiai{
	\begin{itemize}[\color{blue}\checkmark]
	\item Theo công thức nhân xác suất ta có $\mathrm{P}\left(AB \right)= \mathrm{P}\left(B\right)\cdot \mathrm{P}\left(A \mid B\right)$.\\
	Vậy $\mathrm{P}\left(A\mid B\right)=\mathrm{P}\left(B\right)\cdot \mathrm{P}\left(AB\right)$ sai.
	\item Nếu $A$ và $B$ là hai biến cố độc lập thì $\mathrm{P}\left(AB\right) =\mathrm{P}\left(A\right) \cdot \mathrm{P}\left(B\right)$.\\
	Suy ra $\mathrm{P}\left(B \mid A\right)=\dfrac{\mathrm{P}\left(BA\right)}{\mathrm{P}\left(A\right) }
	=	\dfrac{\mathrm{P}\left(B\right) \cdot \mathrm{P}\left(A\right) }{\mathrm{P}\left(A\right) }
	=\mathrm{P}\left(B\right)$.
	\item 
	\immini{Vì $\overline{A}B$ và $AB$ là hai biến cố xung khắc và $\overline{A}B \cup AB = B$ nên\\ $\mathrm{P}\left(\overline{A}B\right)=\mathrm{P}\left(B\right)-\mathrm{P}\left(AB\right)$.\\
	Với $\mathrm{P}\left(B\right)>0$.\\
	Ta có $\begin{aligned}[t]
	\mathrm{P}\left(\overline{A}\mid B\right)&=
	\dfrac{\mathrm{P}\left(\overline{A}B\right)}{\mathrm{P}\left(B\right) }\\
	&=\dfrac{\mathrm{P}\left(B\right)-\mathrm{P}\left(AB\right)}{\mathrm{P}\left(B\right) }\\
	&= 1- \dfrac{\mathrm{P}\left(AB\right)}{\mathrm{P}\left(B\right) }\\
	&=
	1-\mathrm{P}\left(A\mid B\right)
	\end{aligned}$.
	}{
	\begin{tikzpicture}[scale=1.5,>=stealth, line join=round, line cap=round]
	\draw[ pattern=north east lines]
	(0,0) to [bend left=90] (2,2) to [bend left=90] (0,0) (0,-.5) node {$A$} ;
	\draw[ pattern=north west lines]
	(1,0) to [bend left=90] (3,2) to [bend left=90] (1,0)
	(2.5,-.5) node {$B$};
	\def\miena{(0,-3.5) to [bend left=90] (2,-1.5) to [bend left=90] (0,-3.5)};
	\def\mienb{(1,-3.5) to [bend left=90] (3,-1.5) to [bend left=90] (1,-3.5)};	
	\node [circle,draw,fill=white] at (2.7,1.5){$\small \overline{A} B$};
	\node [circle,draw,fill=white] at (1.6,1.2){$\small AB$};
	\end{tikzpicture}
	}
	\item Nếu $A$ và $B$ là hai biến cố độc lập thì $A$ và $\overline{B}$ cũng là biến cố độc lập. Do đó\\
	$\mathrm{P}\left(A \mid B\right)=
	\dfrac{\mathrm{P}\left(AB\right)}{\mathrm{P}\left(B\right) }=
	\dfrac{\mathrm{P}\left(A\right) \cdot \mathrm{P}\left(B\right) }{\mathrm{P}\left(B\right) }
	=\mathrm{P}\left(A\right)$.\\
	$\mathrm{P}\left(A \mid \overline{B}\right)=
	\dfrac{\mathrm{P}\left(A\overline{B}\right)}{\mathrm{P}\left(\overline{B}\right) }=
	\dfrac{\mathrm{P}\left(A\right) \cdot \mathrm{P}\left(\overline{B}\right) }{\mathrm{P}\left(\overline{B}\right) }
	=\mathrm{P}\left(A\right)$.\\
	$\Rightarrow \mathrm{P}\left(A\mid B\right)=
	\mathrm{P}\left(A \mid \overline{B}\right)=\mathrm{P}\left(A\right)$.
	\end{itemize}
	}
\end{ex}
%%%%%----------Câu 5
\begin{ex}%[2D6N2-1]%[Võ Thanh Hiệp]
	Cho $A$ và $B$ là hai biến cố. Công thức nào dưới đây là công thức tính xác suất toàn phần?
	\choice
	{$\mathrm{P}\left(A\right)= \mathrm{P}\left(B\right)\cdot \mathrm{P}\left(A\mid B\right) + 
	\mathrm{P}\left(\overline{B}\right)\cdot \mathrm{P}\left(A\mid B\right)$}
	{$\mathrm{P}\left(A\right)= \mathrm{P}\left(B\right)\cdot \mathrm{P}\left(A\mid B\right) + 
	\mathrm{P}\left(\overline{B}\right)\cdot \mathrm{P}\left(\overline{A}\mid \overline{B}\right)$}
	{$\mathrm{P}\left(A\right)= \mathrm{P}\left(B\right)\cdot \mathrm{P}\left(A\mid B\right) + 
	\mathrm{P}\left(\overline{B}\right)\cdot \mathrm{P}\left(\overline{A}\mid B\right)$}
	{\True $\mathrm{P}\left(A\right)= \mathrm{P}\left(B\right)\cdot \mathrm{P}\left(A\mid B\right) + 
	\mathrm{P}\left(\overline{B}\right)\cdot \mathrm{P}\left(A\mid \overline{B}\right)$}
	\loigiai{
	Công thức tính xác suất toàn phần là 	
	$$\mathrm{P}\left(A\right)= \mathrm{P}\left(B\right)\cdot \mathrm{P}\left(A\mid B\right) + 
	\mathrm{P}\left(\overline{B}\right)\cdot \mathrm{P}\left(A\mid \overline{B}\right).$$
	}
\end{ex}
%%%%%----------Câu 6
\begin{ex}%[2D6N2-2]%[Võ Thanh Hiệp]
	Cho hai biến cố $A$ và $B$ có $\mathrm{P}\left(A\right) = 0{,}3$; $\mathrm{P}\left(B\right) = 0{,}5$ và
	$\mathrm{P}\left(B\mid A\right)=0{,}4$. Tính $\mathrm{P}\left(A\mid B\right)$. 
	\choice
	{$0{,}6$}
	{\True $0{,}24$}
	{$0{,}15$}
	{$0{,}5$}
	\loigiai{
	Áp dụng công thức Bayes	ta có
	\begin{eqnarray*}
	\mathrm{P}\left(A\mid B\right)&=&
	\dfrac{\mathrm{P}\left(A\right) \cdot \mathrm{P}\left(B\mid A\right) }{\mathrm{P}\left(A\right)\cdot \mathrm{P}\left(B\mid A\right)
	+ \mathrm{P}\left(\overline{A}\right)\cdot \mathrm{P}\left(B\mid \overline{A}\right)}\\
	&=& \dfrac{\mathrm{P}\left(A\right) \cdot \mathrm{P}\left(B\mid A\right) }{\mathrm{P}\left(B\right)}\\
	&=&\dfrac{0{,}3\cdot 0{,}4}{0{,}5}=0{,}24.
	\end{eqnarray*}
	}
\end{ex}
%%%%%----------Câu 7
\begin{ex}%[2D6N1-2]%[Võ Thanh Hiệp]
	Cho hai biến cố $A$ và $B$ có $\mathrm{P}\left(A\right) = 0{,}7$; $\mathrm{P}\left(B\right) = 0{,}4$ và
	$\mathrm{P}\left(AB\right)=0{,}2$. Tính xác suất biến cố $B$ với điều kiện $A$. 
	\choice
	{$\dfrac{1}{2}$}
	{$\dfrac{4}{7}$}
	{\True $\dfrac{2}{7}$}
	{$\dfrac{7}{10}$}
	\loigiai{
	Xác suất của biến cố $B$ với điều kiện $A$ là $\mathrm{P}\left(B\mid A\right)$.\\
	Ta có $\mathrm{P}\left(B\mid A\right) 
	= \dfrac{\mathrm{P}\left(AB\right) }{\mathrm{P}\left(A\right)}
	=\dfrac{0{,}2}{0{,}7}=\dfrac{2}{7}$.
	}
\end{ex}
%%%%%----------Câu 8
\begin{ex}%[2D6H1-2]%[Võ Thanh Hiệp]
	Cho hai biến cố $A$ và $B$ có $\mathrm{P}\left(A\right) = 0{,}6$; $\mathrm{P}\left(B\right) = 0{,}4$ và $\mathrm{P}\left(AB\right)=0{,}3$. Xác suất biến cố $A$ không xảy ra với điều kiện $B$ là 
	\choice
	{$\dfrac{7}{10}$}
	{$\dfrac{3}{4}$}
	{$\dfrac{1}{2}$}
	{\True $\dfrac{1}{4}$}
	\loigiai{
	Xác suất biến cố $A$ không xảy ra với điều kiện $B$ là 
	$\mathrm{P}\left(\overline{A}\mid B\right)$.\\
	Ta có $\mathrm{P}\left(\overline{A}B\right)= 1- \mathrm{P}\left(A \mid B\right)
	=1-\dfrac{\mathrm{P}\left(AB\right)}{\mathrm{P}\left(B\right)}=1-\dfrac{0{,}3}{0{,}4}=\dfrac{1}{4}$.
	}
\end{ex}
%%%%%----------Câu 9
\begin{ex}%[2D6H1-2]%[Võ Thanh Hiệp]
	Trong hộp có $7$ viên bi màu xanh, $5$ viên bi màu đỏ, các viên bi có cùng kích thước và khối lượng. An lấy ngẫu nhiên một viên bi từ hộp không trả lại, sau đó Bình lấy một viên bi trong các bi còn lại. Tính xác suất để An lấy được bi màu xanh và Bình lấy được bi màu đỏ.
	\choice
	{\True $\dfrac{35}{132}$}
	{$\dfrac{35}{144}$}
	{$\dfrac{1}{11}$}
	{$\dfrac{3}{132}$}
	\loigiai{
	Gọi $A$ là biến cố: \lq\lq  An lấy được bi màu xanh\rq\rq.\\
	$B$ là biến cố: \lq\lq  Bình lấy được bi màu đỏ\rq\rq.\\
	Ta cần tính $\mathrm{P}\left(AB\right)$.\\
	Ta có $\mathrm{P}\left(A\right) = \dfrac{7}{12}$, 
	$\mathrm{P}\left(B\mid A\right) = \dfrac{5}{11}$.\\
	Theo công thức nhân xác suất
	$\mathrm{P}\left(AB\right)= \mathrm{P}\left(A\right)\cdot \mathrm{P}\left(B\mid A\right)
	=\dfrac{7}{12} \cdot \dfrac{5}{11}=\dfrac{35}{132}$.
	}
\end{ex}
%%%%%----------Câu 10
\begin{ex}%[2D6H1-3]%[Võ Thanh Hiệp]
	\immini{ Cho sơ đồ cây như hình vẽ bên. Khẳng định nào sau đây {\bf sai}?
	\choice
	{$\mathrm{P}\left(A\right)= 0{,}3$}
	{$\mathrm{P}\left(B\right)= 0{,}54$}
	{\True $\mathrm{P}\left(A\mid B\right)=0{,}6$}
	{$\mathrm{P}\left(\overline{B}\mid \overline{A}\right)=0{,}2$}
	}{
	\tikzstyle{xs} = [rectangle ,fill=white,draw=black,rounded corners,align=center]
	\tikzstyle{bc} = [circle ,fill=white,draw=black,rounded corners,align=center]
	\begin{tikzpicture}[scale=0.7,>=stealth, font=\footnotesize, line join=round, line cap=round]
	\node[bc,text width=2.5mm] (O) at(0,0) { };
	\node[bc] (A) at ($(O)+(30:3)$) {$A$};
	\node[bc] (A1) at ($(O)+(-30:3)$){$\overline{A}$};
	\node[bc] (B) at ($(A)+(20:3)$) {$B$};
	\node[bc] (B1) at ($(A)+(-20:3)$) {$\overline{B}$};
	\node[bc] (B2) at ($(A1)+(20:3)$) {$B$};
	\node[bc] (B3) at ($(A1)+(-20:3)$) {$\overline{B}$};
	\draw[->] {(O)--node [above,xs,sloped] {$0{,}3$}(A)} ;
	\draw[->] {(O)--node [below,xs,sloped] {$0{,}7$}(A1)} ;
	\draw[->] {(A)--node [above,xs,sloped] {$0{,}4$}(B)} ;
	\draw[->] {(A)--node [below,xs,sloped] {$0{,}6$}(B1)} ;
	\draw[->] {(A1)--node [above,xs,sloped] {$0{,}8$}(B2)} ;
	\draw[->] {(A1)--node [below,xs,sloped] {$0{,}2$}(B3)} ;
	\end{tikzpicture}
	}
	\loigiai{
	\begin{center}	
	\tikzstyle{xs} = [rectangle ,fill=white,draw=black,rounded corners,align=center]
	\tikzstyle{bc} = [circle ,fill=white,draw=black,rounded corners,align=center]
	\begin{tikzpicture}[scale=0.7,>=stealth, font=\footnotesize, line join=round, line cap=round]
	\node[bc,text width=2.5mm] (O) at(0,0) { };
	\node[bc] (A) at ($(O)+(32:3)$) {$A$};
	\node[bc] (A1) at ($(O)+(-32:3)$){$\overline{A}$};
	\node[bc] (B) at ($(A)+(20:3)$) {$B$};
	\node[bc] (B1) at ($(A)+(-20:3)$) {$\overline{B}$};
	\node[bc] (B2) at ($(A1)+(20:3)$) {$B$};
	\node[bc] (B3) at ($(A1)+(-20:3)$) {$\overline{B}$};
	\draw[->] {(O)--node [above,xs,sloped] {$0{,}3$}(A)} ;
	\draw[->] {(O)--node [below,xs,sloped] {$0{,}7$}(A1)} ;
	\draw[->] {(A)--node [above,xs,sloped] {$0{,}4$}(B)} ;
	\draw[->] {(A)--node [below,xs,sloped] {$0{,}6$}(B1)} ;
	\draw[->] {(A1)--node [above,xs,sloped] {$0{,}8$}(B2)} ;
	\draw[->] {(A1)--node [below,xs,sloped] {$0{,}2$}(B3)} ;
	\node[bc] (AB) at ($(B)+(0:2)$) {$AB$};
	\node[bc] (AB1) at ($(B1)+(0:2)$) {$A\overline{B}$};
	\node[bc] (AB2) at ($(B2)+(0:2)$) {$\overline{A}B$};
	\node[bc] (AB3) at ($(B3)+(0:2)$) {$\overline{A}\,\overline{B}$};
	\end{tikzpicture}
	\end{center}
	Từ sơ đồ cây suy ra\\
	$\mathrm{P}\left(A\right)= 0{,}3$; $\mathrm{P}\left(\overline{A}\right)= 0{,}7$;
	$\mathrm{P}\left(A B\right)= 0{,}12$;\\
	$\mathrm{P}\left(B\mid A\right)= 0{,}4$; $\mathrm{P}\left(B\mid \overline{A}\right)= 0{,}8$;
	$\mathrm{P}\left(\overline{B}\mid \overline{A}\right)=0{,}2$.\\
	Suy ra
	$\mathrm{P}\left(B\right)= \mathrm{P}\left(A\right)\cdot \mathrm{P}(B\mid A)
	+ \mathrm{P}\left(\overline{A}\right)\cdot \mathrm{P}(B\mid \overline{A})
	= 0{,}3\cdot 0{,}4 + 0{,}7\cdot 0{,}6 = 0{,}54 $.\\
	$\mathrm{P}\left(A\mid B\right)= \dfrac{\mathrm{P}\left(AB\right)}{\mathrm{P}\left(B\right)}
	=\dfrac{0{,}12}{0{,}54}=\dfrac{2}{9}$.\\
	Vậy $\mathrm{P}\left(A\mid B\right)=0{,}6$ sai.
	}
\end{ex}
%%%%%----------Câu 11
\begin{ex}%[2D6H2-3]%[Võ Thanh Hiệp]
	Cho hai biến cố ngẫu nhiên $A$ và $B$. Biết rằng $\mathrm{P}\left(A\mid B\right)= \dfrac{2}{3} \mathrm{P}\left(B\mid A\right)$ và $P\left(AB\right) \ne 0$. Tính tỉ số $\dfrac{\mathrm{P}\left(A\right)}{\mathrm{P}\left(B\right)}$.
	\choice
	{$\dfrac{3}{2}$}
	{\True $\dfrac{2}{3}$}
	{$\dfrac{1}{3}$}
	{$\dfrac{1}{2}$}
	\loigiai{
	Theo công thức Bayes ta có\\
	$\mathrm{P}\left(B\mid A\right)=
	\dfrac{\mathrm{P}\left(B\right)\cdot \mathrm{P}\left(A\mid B\right)}{\mathrm{P}\left(A\right)}
	\Leftrightarrow 
	\dfrac{\mathrm{P}\left(A\mid B\right)}{\mathrm{P}\left(B\mid A\right)}
	=\dfrac{\mathrm{P}\left(A\right) }{\mathrm{P}\left(B\right)}=\dfrac{2}{3}$. 
	}
\end{ex}
%%%%%----------Câu 12
\begin{ex}%[2D6H2-2]%[Võ Thanh Hiệp]
	Cho hai biến cố ngẫu nhiên $A$ và $B$. Biết rằng $\mathrm{P}\left(A\right)= \dfrac{3}{5}$; $\mathrm{P}\left(B\mid A\right)= \dfrac{1}{4}$ và $\mathrm{P}\left(B\mid \overline{A}\right)= \dfrac{1}{3}$. Tính $\mathrm{P}\left(B\overline{A}\right)$.
	\choice
	{$\dfrac{43}{180}$}
	{\True$\dfrac{2}{15}$}
	{$\dfrac{3}{15}$}
	{$\dfrac{17}{180}$}
	\loigiai{
	Ta có $\mathrm{P}\left(A\right)= \dfrac{3}{5}\Rightarrow \mathrm{P}\left(\overline{A}\right)=\dfrac{2}{5}$.\\
	Áp dụng công thức xác suất toàn phần ta có\\
	$\mathrm{P}\left(B\right)=\mathrm{P}\left(A\right)\cdot \mathrm{P}\left(B\mid A\right)
	+\mathrm{P}\left(\overline{A}\right)\cdot \mathrm{P}\left(B\mid \overline{A}\right)
	=\dfrac{3}{5}\cdot\dfrac{1}{4}+\dfrac{2}{5}\cdot\dfrac{1}{3}=\dfrac{17}{60}$.\\
	Áp dụng công thức nhân xác suất ta có\\
	$\mathrm{P}\left(B\overline{A}\right)=\mathrm{P}\left(\overline{A}\right)\cdot \mathrm{P}\left(B\mid \overline{A}\right)
	=\dfrac{2}{5}\cdot\dfrac{1}{3}=\dfrac{2}{15}$.
	}
\end{ex}
%%==========Câu 6
\begin{ex}%[2D6H1-2]
	Cho $\mathrm{P}(A)=\dfrac{2}{5}$; $\mathrm{P}\left( B\mid A\right)=\dfrac{1}{3}$; $\mathrm{P}\left(B\mid \overline{A}\right)=\dfrac{1}{4}$. Giá trị của $\mathrm{P}(B)$ là 
	\choice
	{$\dfrac{19}{60} $}
	{\True $ \dfrac{17}{60}$}
	{$ \dfrac{9}{20}$}
	{$ \dfrac{7}{30}$}
	\loigiai{Áp dụng công thức Bayes ta có
	$$\mathrm{P}(A\mid B)=\dfrac{\mathrm{P}(A)\cdot \mathrm{P}(B\mid A)}{\mathrm{P}(A)\cdot \mathrm{P}(B\mid A)+\mathrm{P}\left( \overline{A}\right) \cdot \mathrm{P}\left( B\mid \overline{A}\right) }=\dfrac{\dfrac{2}{5}\cdot \dfrac{1}{3}}{\dfrac{2}{5}\cdot \dfrac{1}{3}+\dfrac{3}{5}\cdot \dfrac{1}{4}}=\dfrac{8}{17}.$$
	Khi đó 
	$\mathrm{P}(B)=\dfrac{\mathrm{P}(AB)}{\mathrm{P}(A\mid B)}=\dfrac{2}{15}:\dfrac{8}{17}=\dfrac{17}{60}$.
	}
\end{ex}
%%==========Câu 7
\begin{ex}%[2D6H1-2]
	An có một túi gồm một số chiếc kẹo cùng loại, chỉ khác màu, trong đó có $6$ chiếc kẹo sô-cô-la đen, còn lại là $4$ chiếc kẹo sô-cô-la trắng. An lấy ngẫu nhiên $1$ chiếc kẹo trong túi để cho Bình, rồi lại lấy ngẫu nhiên tiếp $1$ chiếc kẹo nữa trong túi và cũng đưa cho Bình. Xác suất để Bình nhận được $2$ chiếc kẹo sô-cô-la đen là 
	\choice
	{\True $\dfrac{1}{3} $}
	{$ \dfrac{1}{4}$}
	{$ \dfrac{2}{5}$}
	{$ \dfrac{3}{7}$}
	\loigiai{
	Gọi $A$ là biến cố \lq\lq  An lấy lần $1$ được $1$ chiếc kẹo sô-cô-la đen\rq\rq\, và $B$ là biến cố \lq\lq  An lấy lần $2$ được $1$ chiếc kẹo sô-cô-la đen\rq\rq.\\
	Khi đó $AB$ là biến cố \lq\lq  Cả hai lần đều lấy được kẹo sô-cô-la đen\rq\rq.\\
	Ta có $\mathrm{P}(A)=\dfrac{n(A)}{n(\Omega)}=\dfrac{3}{5}$.\\
	Sau khi lấy $1$ chiếc kẹo sô-cô-la đen thì xác suất để chọn $1$ chiếc kẹo sô-cô-la đen trong hộp đựng $5$ chiếc kẹo sô-cô-la đen, còn lại là $4$ chiếc kẹo sô-cô-la trắng là $\mathrm{P}(B\mid A)=\dfrac{5}{9}$.\\
	Khi đó $\mathrm{P}(AB)=\mathrm{P}(A)\cdot \mathrm{P}(B\mid A)=\dfrac{3}{5}\cdot \dfrac{5}{9}=\dfrac{1}{3}$.\\
	Xác suất để Bình nhận được $2$ chiếc kẹo sô-cô-la đen là $\dfrac{1}{3}$.
	}
\end{ex}
%%==========Câu 8
\begin{ex}%[0D0H2-9]
	Người ta nhập hai lô hàng vào kho. Lô thứ nhất chứa $10$ sản phẩm, trong đó có $3$ phế phẩm. Lô thứ hai có $4$ phế phẩm và $8$ sản phẩm tốt. Chọn ngẫu nhiên một sản phẩm. Xác suất chọn được một sản phẩm tốt là
	\choice
	{\True $\dfrac{15}{22}$}
	{$\dfrac{7}{15}$}
	{$\dfrac{7}{22}$}
	{$\dfrac{83}{242}$}
	\loigiai{
	Gọi $A$ là biến cố “chọn được sản phẩm tốt”, theo đề bầi, kho có $22$ sản phẩm, trong đó có $15$ sản phẩm tốt nên $n(A)=15$, $n(\Omega)=22.$\\
	Vậy $\mathrm{P}(A)=\dfrac{n(A)}{n(\Omega)}=\dfrac{15}{22}$.
	}
\end{ex}
%%==========Câu 9
\begin{ex}%[2D6V2-2]
	Cho $A$, $B$ là các biến cố thỏa mãn $\mathrm{P}(\overline{A}\cap \overline{B})=0{,}35$; $\mathrm{P}(A)=0{,}25$; $\mathrm{P}(B)=0{,}6$. Giá trị của $\mathrm{P}(A\mid B)$ bằng
	\choice
	{$\dfrac{1}{5}$}
	{\True$\dfrac{1}{3}$}
	{$\dfrac{7}{15}$}
	{$\dfrac{2}{3}$}
	\loigiai{
	Ta có $$\mathrm{P}(\overline{A}\cap \overline{B}) = \mathrm{P} \left(\overline A \right) \mathrm{P} \left( \overline B \mid \overline A \right) \Rightarrow \mathrm{P} \left(\overline B \mid \overline A \right) = \dfrac{\mathrm{P}(\overline{A}\cap \overline{B})}{\mathrm{P} \left( \overline A\right)} = \dfrac{0{,}35}{0{,}75} = \dfrac{7}{15}.$$
	Suy ra $$\mathrm{P} \left(B\mid \overline A\right) = 1 - \dfrac{7}{15} = \dfrac{8}{15}.$$
	Theo công thức xác suất toàn phần, ta có
	\begin{eqnarray*}
	&&	\mathrm{P}\left( B \right) = \mathrm{P}\left(B\mid A\right)\mathrm{P}\left( A \right) + \mathrm{P}\left( B\mid \overline A \right)\mathrm{P}\left( \overline A \right)\\
	&\Rightarrow& \mathrm{P}\left( B\mid A \right) = \dfrac{\mathrm{P}\left( B \right) - \mathrm{P} \left( {B\mid \overline A } \right)\mathrm{P}\left( {\overline A } \right)}{\mathrm{P}\left( A \right)} = \dfrac{0{,}6 - \dfrac{8}{{15}} \cdot 0{,}75}{0{,}25} = 0{,}8.	
	\end{eqnarray*}
	Theo công thức Bayes, ta được
	$$\mathrm{P}\left( {A\mid B} \right) = \dfrac{{\mathrm{P}\left( A \right)\mathrm{P}\left( {B\mid A} \right)}}{{\mathrm{P}\left( B \right)}} = \dfrac{{0{,}25 \cdot 0{,}8}}{{0{,}6}} = \dfrac{1}{3}.$$}
\end{ex}
%%==========Câu 10
\begin{ex}%[2D6V1-3]
	Một bệnh viện có hai phòng khám là phòng A và phòng B với khả năng lựa chọn của bệnh nhân là như nhau. Tỉ lệ bệnh nhân nam có ở phòng A và phòng B lần lượt là $60\%$ và $40\%$. Một người bệnh được chọn ngẫu nhiêu từ hai phòng khám và biết người này là nam, xác suất để người bệnh được chọn đến từ phòng A là
	\choice
	{\True $0{,}6$}
	{$0{,}5$}
	{$0{,}4$}
	{$0{,}3$}
	\loigiai{Một người bệnh được chọn ngẫu nhiên từ hai phòng khám.\\
	Gọi $X$ là biến cố \lq \lq Người đó đến từ phòng khám A\rq \rq \, và $Y$, $\overline{Y}$ lần lượt là biến cố \lq \lq Người đó là nam\rq \rq \; và \lq \lq Người đó không là nam\rq \rq.\\
	Ta có sơ đồ hình cây sau
	\begin{center}
		\begin{tikzpicture}[>=stealth,scale=0.8]
	%Khung 1
	\draw (-2,-1) rectangle (2.2,0);
	\draw (0.1,-0.5) node{Bệnh nhân được chọn} ;
	%Mui ten 1,2
	\draw [->] (2.2,-0.5)--(3.8,1.6) node[pos=0.5,sloped,above]{$0{,}5$};
	\draw [->] (2.2,-0.5)--(3.8,-2.6) node[pos=0.5,sloped,below]{$0{,}5$};
	%Khung 2.1
	\draw (3.8,1.1) rectangle (5.1,2.1);
	\draw (8.9/2,1.6) node{$X$} ;
	%Khung 2.2
	\draw (3.8,-2.1) rectangle (5.1,-3.1);
	\draw (8.9/2,-2.6) node{$\overline{X}$} ;
	%Mui ten 3,4
	\draw [->] (5.1,1.6)--(6.5,2.6) node[pos=0.5,sloped,above]{$0{,}6$};
	\draw [->] (5.1,1.6)--(6.5,0.6) node[pos=0.5,sloped,below]{$0{,}4$};
	%Mui ten 5,6
	\draw [->] (5.1,-2.6)--(6.5,-1.6) node[pos=0.5,sloped,above]{$0{,}4$};
	\draw [->] (5.1,-2.6)--(6.5,-3.6) node[pos=0.5,sloped,below]{$0{,}6$};
	%Khung 3.1
	\draw (6.5,2.2) rectangle (7.7,3.2);
	\draw (7.1,5.4/2) node{$Y$} ;
	%Khung 3.2
	\draw (6.5,1.2) rectangle (7.7,0.2);
	\draw (7.1,1.4/2) node{$\overline{Y}$} ;
	%Khung 3.3
	\draw (6.5,-1.1) rectangle (7.7,-2.1);
	\draw (7.1,-3.2/2) node{$Y$} ;
	%Khung 3.3
	\draw (6.5,-2.9) rectangle (7.7,-3.9);
	\draw (7.1,-3.4) node{$\overline{Y}$} ;
	%Kết quả
	\draw (9.5,3.7) node{\textbf{Kết quả}};	
	\draw (9.5,2.7) node{$XY$};
	\draw (9.5,0.7) node{$X \overline{Y}$};
	\draw (9.5,-1.6) node{$\overline{X}Y$};
	\draw (9.5,-3.4) node{$\overline{X}\overline{Y}$};
	%Xác suất
	\draw (12.5,3.7) node{\textbf{Xác suất}};	
	\draw (12.5,2.7) node{$0{,}3$};
	\draw (12.5,0.7) node{$0{,}2$};
	\draw (12.5,-1.6) node{$0{,}2$};
	\draw (12.5,-3.4) node{$0{,}3$};	
	\end{tikzpicture}
	\end{center}
	\noindent Theo công thức Bayes, ta có $$\mathrm{P}(X\mid Y)=\dfrac{\mathrm{P}(X)\mathrm{P}(Y\mid X)}{\mathrm{P}(X)\mathrm{P}(Y\mid X)+\mathrm{P}(\overline{X})\mathrm{P}(Y\mid \overline{X})}=\dfrac{0{,}3}{0{,}3+0{,}2}=0{,}6.$$
	Vậy với một người bệnh được chọn ngẫu nhiêu từ hai phòng khám và biết người này là nam, xác suất để người đó đến từ phòng A là $0{,}6$.}
\end{ex}
%%==========Câu 11
\begin{ex}%[2D6V1-3]
	Ở một địa phương $X$, xác suất để một người lớn trên $40$ tuổi mắc bệnh ung thư là $0{,}05$. Xác suất bác sĩ chẩn đoán đúng một người mắc bệnh ung thư là $0{,}78$ và chẩn đoán sai (không bị ung thư nhưng được chẩn đoán mắc bệnh) là $0{,}06$. Xác suất để một người thật sự mắc bệnh ung thư khi nhận được kết quả chẩn đoán bị ung thư bằng
	\choice
	{\True$0{,}40625$}
	{$0{,}096$}
	{$0{,}904$}
	{$0{,}59375$}
	\loigiai{Một bệnh nhân trên 40 tuổi ở địa phương X đến bác sĩ để khám bệnh ung thư.\\
	Gọi $A$ là biến cố \lq \lq Người đó mắc bệnh ung thư\rq \rq \, và $B$, $\overline{B}$ lần lượt là biến cố \lq \lq Bác sĩ chẩn đoán người đó bị ung thư\rq \rq \;và \lq \lq Bác sĩ chẩn đoán người đó không bị ung thư\rq \rq.\\
	Ta xét sơ đồ hình cây như sau
	\begin{center}
		\begin{tikzpicture}[>=stealth,scale=0.8]
	%Khung 1
	\draw (-3.5,-1) rectangle (2.2,0);
	\draw (-1.3/2,-0.5) node{Bệnh nhân được chẩn đoán} ;
	%Mui ten 1,2
	\draw [->] (2.2,-0.5)--(3.8,1.6) node[pos=0.5,sloped,above]{$0{,}05$};
	\draw [->] (2.2,-0.5)--(3.8,-2.6) node[pos=0.5,sloped,below]{$0{,}95$};
	%Khung 2.1
	\draw (3.8,1.1) rectangle (5.1,2.1);
	\draw (8.9/2,1.6) node{$A$} ;
	%Khung 2.2
	\draw (3.8,-2.1) rectangle (5.1,-3.1);
	\draw (8.9/2,-2.6) node{$\overline{A}$} ;
	%Mui ten 3,4
	\draw [->] (5.1,1.6)--(6.5,2.6) node[pos=0.5,sloped,above]{$0{,}78$};
	\draw [->] (5.1,1.6)--(6.5,0.6) node[pos=0.5,sloped,below]{$0{,}22$};
	%Mui ten 5,6
	\draw [->] (5.1,-2.6)--(6.5,-1.6) node[pos=0.5,sloped,above]{$0{,}06$};
	\draw [->] (5.1,-2.6)--(6.5,-3.6) node[pos=0.5,sloped,below]{$0{,}94$};
	%Khung 3.1
	\draw (6.5,2.2) rectangle (7.7,3.2);
	\draw (7.1,5.4/2) node{$B$} ;
	%Khung 3.2
	\draw (6.5,1.2) rectangle (7.7,0.2);
	\draw (7.1,1.4/2) node{$\overline{B}$} ;
	%Khung 3.3
	\draw (6.5,-1.1) rectangle (7.7,-2.1);
	\draw (7.1,-3.2/2) node{$B$} ;
	%Khung 3.3
	\draw (6.5,-2.9) rectangle (7.7,-3.9);
	\draw (7.1,-3.4) node{$\overline{B}$} ;
	%Kết quả
	\draw (9.5,3.7) node{\textbf{Kết quả}};	
	\draw (9.5,2.7) node{$AB$};
	\draw (9.5,0.7) node{$A\overline{B}$};
	\draw (9.5,-1.6) node{$\overline{A}B$};
	\draw (9.5,-3.4) node{$\overline{A}\overline{B}$};
	%Xác suất
	\draw (12.5,3.7) node{\textbf{Xác suất}};	
	\draw (12.5,2.7) node{$0{,}039$};
	\draw (12.5,0.7) node{$0{,}011$};
	\draw (12.5,-1.6) node{$0{,}057$};
	\draw (12.5,-3.4) node{$0{,}893$};
	\end{tikzpicture}
	\end{center}
	Theo công thức Bayes, ta có $$\mathrm{P}(A\mid B)=\dfrac{\mathrm{P}(A)\mathrm{P}(B\mid A)}{\mathrm{P}(A)\mathrm{P}(B\mid A)+\mathrm{P}(\overline{A})\mathrm{P}(B\mid \overline{A})}=\dfrac{0{,}039}{0{,}039+0{,}057}=0{,}40625.$$	 
	Vậy xác suất để một người thật sự mắc bệnh ung thư khi nhận được kết quả chẩn đoán bị ung thư bằng $0{,}40625$.}
\end{ex} 
%%==========Câu 12
\begin{ex}%[2D6V2-3]
	Một loại vaccine được tiêm ở địa phương $X$. Người có bệnh nền thì với xác suất $0{,}35$ có phản ứng phụ sau tiêm, người không có bệnh nền thì chỉ có phản ứng phụ sau tiêm với xác suất $0{,}16$. Chọn ngẫu nhiên một người được tiêm vaccine và người này có phản ứng phụ. Tính xác suất để người này có bệnh nền, biết rằng tỉ lệ người có bệnh nền ở địa phương $X$ là $18\%$.
	\choice
	{\True $\dfrac{315}{971}$ }
	{ $\dfrac{31}{971}$}
	{$0{,}16$} 
	{$0{,}063$}
	\loigiai{
	Gọi $A$ là biến cố \lq\lq  Người được chọn có bệnh nền\rq\rq\, và $B$ là biến cố \lq\lq  Người này có phản ứng phụ sau tiêm\rq\rq.\\
	Ta có $\mathrm{P}(A)=0{,}18$; $\mathrm{P}(\overline{A})=0{,}82$.\\
	$\mathrm{P}(B\mid A)$ là xác suất để một người bệnh có phản ứng sau tiêm với điều kiện có bệnh nền, suy ra $$\mathrm{P}(B\mid A)=0{,}35.$$
	$\mathrm{P}(B\mid \overline{A})$ là xác suất để một người bệnh có phản ứng sau tiêm với điều kiện không có bệnh nền, suy ra $$\mathrm{P}(B\mid \overline{A})=0{,}16.$$
	Theo công thức Bayes, ta được 
	$$\mathrm{P}(A\mid B)=\dfrac{\mathrm{P}(A)\cdot \mathrm{P}(B\mid A)}{\mathrm{P}(A)\cdot \mathrm{P}(B\mid A)+\mathrm{P}(\overline{A})\cdot \mathrm{P}(B\mid \overline{A})}=\dfrac{0{,}18\cdot 0{,}35}{0{,}18\cdot 0{,}35+0{,}82\cdot 0{,}16}=\dfrac{315}{971}.$$
	}
\end{ex}
%==============================================================
\Closesolutionfile{ans}
\indapan{6}{ans/ans-0-B15}
%%==========HẾT PHẦN 1=========================================
\Opensolutionfile{ans}[ans/ans-0-B15-DS]

%%%%%----------Câu 13
\begin{ex}%[2D6V1-2]%[Nguyễn Khánh Trọng]
	Một nhà máy thực hiện khảo sát toàn bộ công nhân về sự hài lòng của họ về điều kiện làm việc tại phân xưởng. Kết quả khảo sát như sau:
	\begin{center}
	\begin{tabular}{|c|ccll|}
	\hline
	\multirow{2}{*}{Khảo sát công nhân} & \multicolumn{4}{c|}{Kết quả khảo sát} \\ \cline{2-5} 
	& \multicolumn{1}{c|}{Hài lòng} & \multicolumn{3}{c|}{Không hài lòng} \\ \hline
	Số công nhân xưởng I & \multicolumn{1}{c|}{23} & \multicolumn{3}{c|}{12} \\ \hline
	Số công nhân xưởng II & \multicolumn{1}{c|}{25} & \multicolumn{3}{c|}{15} \\ \hline
	\end{tabular}
	\end{center}
	Gặp ngẫu nhiên một công nhân của nhà máy. Gọi $A$ là biến cố \lq\lq  Công nhân đó làm việc tại phân xưởng I\rq\rq \, và $B$ là biến cố \lq\lq  Công nhân đó hài lòng với điều kiện làm việc tại phân xưởng\rq\rq. Xét tính đúng sai của các phát biểu sau.
	\choiceTF
	{\True Xác suất của biến cố $A$ là $\dfrac{7}{15}$}
	{Xác suất của biến cố $B$ là $0{,}65$}
	{\True Xác suất gặp được công nhân không hài lòng với điều kiện làm việc tại phân xưởng biết công nhân đó thuộc xưởng I là $\dfrac{12}{35}$}
	{Xác suất gặp được công nhân thuộc xưởng II biết công nhân đó hài lòng với điều kiện làm việc tại phân xưởng là $0{,}52$}
	\loigiai{
	Gặp ngẫu nhiên một công nhân của nhà máy, ta có $n(\Omega)=23+12+25+15=75$. 
	\begin{itemchoice}
	\itemch {\bf Đúng.}\\
	Ta có $n(A)=23+12=35$.\\
	Xác suất của biến cố $A$ là 
	$\mathrm{P}(A)=\dfrac{35}{75}=\dfrac{7}{15}$.
	\itemch {\bf Sai.} \\
	Ta có $n(B)=23+25=48$. \\
	Xác suất của biến cố $B$ là 
	$\mathrm{P}(B)=\dfrac{48}{75}=\dfrac{16}{25}$.
	\itemch Đúng. Ta có $n(\overline{B}A)=12\Rightarrow \mathrm{P}(\overline{B}A)=\dfrac{12}{75}=\dfrac{4}{25}$.\\
	Do đó
	$\mathrm{P}(\overline{B}\mid A)=\dfrac{\mathrm{P}(\overline{B}A)}{\mathrm{P}(A)}=\dfrac{\dfrac{4}{25}}{\dfrac{7}{15}}=\dfrac{12}{35}$.
	\itemch {\bf Sai.}\\
	Ta có $n(\overline{A}B)=25\Rightarrow\mathrm{P}(\overline{A}B)
	=\dfrac{25}{75}=\dfrac{1}{3}$.\\
	Do đó
	$\mathrm{P}(\overline{A}\mid B)=\dfrac{\mathrm{P}(\overline{A}B)}{\mathrm{P}(B)}
	=\dfrac{\dfrac{1}{3}}{\dfrac{16}{25}}=\dfrac{25}{48}$.
	\end{itemchoice}
	}
\end{ex}
%%%%%----------Câu 14
\begin{ex}%[2D6H1-2]%[Nguyễn Khánh Trọng]
	Cho hai biến cố $A$ và $B$ có $P(A) = 0{,}35$; $P(B \mid A)= 0{,}6$ và $P(B \mid \overline{A})= 0{,}2$. Mỗi phát biểu dưới đây đúng hay sai?
	\choiceTF
	{\True $\mathrm{P}(\overline{A})=0{,}65$}
	{$\mathrm{P}(AB)=0{,}2$}
	{\True $\mathrm{P}(\overline{B}\mid A)=0{,}4$}
	{\True $\mathrm{P}(B)=0{,}34$}
	\loigiai{
	\begin{itemchoice}
	\itemch {\bf Đúng.}\\
	Vì $\mathrm{P}(\overline{A})=1-\mathrm{P}(A)=1-0{,}35=0{,}65$.
	\itemch {\bf Sai.}\\
	Vì $\mathrm{P}(AB)=\mathrm{P}(B\mid A)\cdot\mathrm{P}(A)
	= 0{,}35\cdot0{,}6=0{,}21$.
	\itemch {\bf Đúng.}\\
	Vì $\mathrm{P}(\overline{B}\mid A)
	=1-\mathrm{P}(B\mid A)=1-0{,}6=0{,}4$.
	\itemch {\bf Đúng.} \\
	Vì $\mathrm{P}(B)=\mathrm{P}(A)\cdot\mathrm{P}(B\mid A)
	+\mathrm{P}(\overline{A})\cdot\mathrm{P}(B\mid\overline{A})
	=0{,}35\cdot 0{,}6+0{,}65\cdot0{,}2=0{,}34$.
	\end{itemchoice}
	}
\end{ex}
%%%%%----------Câu 15
\begin{ex}%[2D6H1-3]%[Nguyễn Khánh Trọng]
	\immini{Cho sơ đồ cây như hình vẽ. Xét tính đúng sai của các phát biểu sau.
	\choiceTF
	{\True $ \mathrm{P}\left(A\right)= 0{,}25$}
	{$\mathrm{P}\left(A\overline{B}\right)=\dfrac{1}{8}$}
	{$\mathrm{P}\left(B\right)= 0{,}65$}
	{\True $\mathrm{P}\left(A\mid B\right)=0{,}16$}
	}{
	\tikzstyle{xs} = [rectangle ,fill=white,draw=black,rounded corners,align=center]
	\tikzstyle{bc} = [circle ,fill=white,draw=black,rounded corners,align=center]
	\begin{tikzpicture}[scale=1.2,>=stealth, font=\footnotesize, line join=round, line cap=round]
	\node[bc,text width=2.5mm] (O) at(0,0) { };
	\node[bc] (A) at ($(O)+(30:3)$) {$A$};
	\node[bc] (A1) at ($(O)+(-30:3)$){$\overline{A}$};
	\node[bc] (B) at ($(A)+(20:3)$) {$B$};
	\node[bc] (B1) at ($(A)+(-20:3)$) {$\overline{B}$};
	\node[bc] (B2) at ($(A1)+(20:3)$) {$B$};
	\node[bc] (B3) at ($(A1)+(-20:3)$) {$\overline{B}$};
	\draw[->] {(O)--node [above,xs,sloped] {$?$}(A)} ;
	\draw[->] {(O)--node [below,xs,sloped] {$\mathrm{P}(\overline{A})=0{,}75$}(A1)} ;
	\draw[->] {(A)--node [above,xs,sloped] { $\mathrm{P}(B\mid A)=0{,}4$}(B)} ;
	\draw[->] {(A)--node [below,xs,sloped] {$?$}(B1)} ;
	\draw[->] {(A1)--node [above,xs,sloped] {$?$}(B2)} ;
	\draw[->] {(A1)--node [below,xs,sloped] {$\mathrm{P}(\overline{B}\mid \overline{A})=0{,}3$}(B3)} ;
	\end{tikzpicture}
	}
	\loigiai{
	\begin{center}	
	\tikzstyle{xs} = [rectangle ,fill=white,draw=black,rounded corners,align=center]
	\tikzstyle{bc} = [circle ,fill=white,draw=black,rounded corners,align=center]
	\begin{tikzpicture}[scale=1,>=stealth, font=\footnotesize, line join=round, line cap=round]
	\node[bc,text width=2.5mm] (O) at(0,0) { };
	\node[bc] (A) at ($(O)+(32:3)$) {$A$};
	\node[bc] (A1) at ($(O)+(-32:3)$){$\overline{A}$};
	\node[bc] (B) at ($(A)+(20:3)$) {$B$};
	\node[bc] (B1) at ($(A)+(-20:3)$) {$\overline{B}$};
	\node[bc] (B2) at ($(A1)+(20:3)$) {$B$};
	\node[bc] (B3) at ($(A1)+(-20:3)$) {$\overline{B}$};
	\draw[->] {(O)--node [above,xs,sloped] {$0{,}25$}(A)} ;
	\draw[->] {(O)--node [below,xs,sloped] {$0{,}75$}(A1)} ;
	\draw[->] {(A)--node [above,xs,sloped] {$0{,}4$}(B)} ;
	\draw[->] {(A)--node [below,xs,sloped] {$0{,}6$}(B1)} ;
	\draw[->] {(A1)--node [above,xs,sloped] {$0{,}7$}(B2)} ;
	\draw[->] {(A1)--node [below,xs,sloped] {$0{,}3$}(B3)} ;
	\node[bc] (AB) at ($(B)+(0:2)$) {$AB$};
	\node[bc] (AB1) at ($(B1)+(0:2)$) {$A\overline{B}$};
	\node[bc] (AB2) at ($(B2)+(0:2)$) {$\overline{A}B$};
	\node[bc] (AB3) at ($(B3)+(0:2)$) {$\overline{A}\,\overline{B}$};
	\end{tikzpicture}
	\end{center}
	Từ sơ đồ cây suy ra\\
	\begin{itemchoice}
	\itemch {\bf Đúng.}\\
	$\mathrm{P}\left(A\right) = 1 - \mathrm{P}\left(\overline{A}\right) = 1- 0{,}75 = 0{,}25$.
	\itemch {\bf Sai.}\\
	Vì $\mathrm{P}\left(A \overline{B}\right)
	=\mathrm{P}\left(A\right) \cdot \mathrm{P}\left(\overline{B}\mid A\right)
	= 0{,}25\cdot 0{,}6=0{,}15$.
	\itemch {\bf Sai.}\\
	Ta có $\mathrm{P}\left(B\right)= \mathrm{P}\left(A\right)\cdot \mathrm{P}(B\mid A)
	+ \mathrm{P}\left(\overline{A}\right)\cdot \mathrm{P}(B\mid \overline{A})
	= 0{,}25\cdot 0{,}4 + 0{,}75\cdot 0{,}7 = 0{,}625$.
	\itemch {\bf Đúng.}\\
	Ta có $\mathrm{P}\left(A B\right)= \mathrm{P}\left(A\right)\cdot \mathrm{P}\left(B\mid A\right)
	=0{,}25\cdot0{,}4=0{,}1$.\\
	Suy ra $\mathrm{P}\left(A\mid B\right)= \dfrac{\mathrm{P}\left(AB\right)}{\mathrm{P}\left(B\right)}
	=\dfrac{0{,}1}{0{,}625}=0{,}16$.
	\end{itemchoice}
	}
\end{ex}
%%%%%----------Câu 16
\begin{ex}%[2D6V2-2]%[Nguyễn Khánh Trọng]
	Trong một trường học, tỉ lệ học sinh nữ là $55\%$. Tỉ lệ học sinh nữ và tỉ lệ học sinh nam tham gia câu lạc bộ tiếng anh lần lượt là $20\%$ và $15\%$. Gặp ngẫu nhiên $1$ học sinh của trường. 
	Gọi $A$ là biến cố \lq\lq  Học sinh đó là nữ\rq\rq\, và $B$ là biến cố \lq\lq  Học sinh đó tham gia câu lạc bộ tiếng Anh\rq\rq. Xét tính đúng sai của các phát biểu sau.
	\choiceTF
	{\True $\mathrm{P}(\overline{A})=0{,}45$}
	{$\mathrm{P}(B\mid \overline{A}) = 0{,}15$ và $\mathrm{P}(\overline{B}\mid A) = 0{,}2$}
	{Xác suất để học sinh đó có tham gia câu lạc bộ tiếng Anh là $0{,}1675$}
	{\True Biết rằng học sinh có tham gia câu lạc bộ tiếng Anh. Xác suất học sinh đó là nam bằng $\dfrac{27}{71}$}
	\loigiai{
	\begin{itemchoice}
	\itemch {\bf Đúng.}\\
	Do tỉ lệ học sinh nữ là $55\%$ nên
	$\mathrm{P}(A) = 0{,}55$ và $\mathrm{P}(\overline{A}) = 1 - 0{,}55 = 0{,}45$.
	\itemch {\bf Sai.}\\
	Do tỉ lệ học sinh nữ và tỉ lệ học sinh nam tham gia câu lạc bộ tiếng Anh lần lượt là $20\%$ và $15\%$ nên $\mathrm{P}(B\mid A) = 0{,}2$ và $\mathrm{P}(B\mid \overline{A}) = 0{,}15$.\\
	Suy ra $\mathrm{P}(\overline{B}\mid A) =1-\mathrm{P}(B\mid A)=1-0{,}2=0{,}8$.
	\itemch {\bf Sai.}\\
	Xác suất để học sinh đó có tham gia câu lạc bộ tiếng Anh là
	$$\mathrm{P}(B) = \mathrm{P}(A)\cdot\mathrm{P}(B\mid A) + \mathrm{P}(\overline{A})\cdot\mathrm{P}(B\mid\overline{A}) = 0{,}55 \cdot 0{,}2 + 0{,}45 \cdot 0{,}15 = 0{,}1775.$$
	\itemch {\bf Đúng.}\\
	Do học sinh có tham gia câu lạc bộ tiếng Anh nên xác suất học sinh đó là nam là
	$$\mathrm{P}(\overline{A}\mid B) = \dfrac{\mathrm{P}(\overline{A})\cdot \mathrm{P}(B\mid \overline{A})}{\mathrm{P}(B)}=
	\dfrac{0{,}45\cdot 0{,}15}{0{,}1775}=\dfrac{27}{71}.$$
	\end{itemchoice}
	}
\end{ex}
%%==========Câu 13
\begin{ex}%[2D6H1-2]
	Lớp $12A$ có $40$ học sinh, trong đó có $25$ học sinh tham gia câu lạc bộ Tiếng Anh, $16$ học sinh tham gia câu lạc bộ Toán, $12$ học sinh vừa tham gia câu lạc bộ tiếng Anh vừa tham gia câu lạc bộ Toán. Chọn ngẫu nhiên $1$ học sinh. Xét các biến cố sau\\
	$A\colon$ \lq\lq  Học sinh được chọn tham gia câu lạc bộ Tiếng Anh\rq\rq;\\
	$B\colon$ \lq\lq  Học sinh được chọn tham gia câu lạc bộ Toán\rq\rq.\\
	Xét tính đúng, sai của các khẳng định sau
	\choiceTF
	{$\mathrm{P}(A)=0{,}4$}
	{$\mathrm{P}(B)=0{,}625$}
	{\True $\mathrm{P}(A | B)=0{,}75$}
	{\True $\mathrm{P}(B | A)=0{,}48$}
	\loigiai{
	\begin{itemchoice}
	\itemch Sai, vì xác suất của biến cố $A$ là $\mathrm{P}(A)=\dfrac{25}{40}=0{,}625$.
	\itemch Sai, vì xác suất của biến cố $B$ là $\mathrm{P}(B)=\dfrac{16}{40}=0{,}4$.
	\itemch Đúng, vì số học sinh vừa tham gia câu lạc bộ tiếng Anh vừa tham gia câu lạc bộ Toán là $12$, số học sinh tham gia câu lạc bộ Toán là $16$ nên $\mathrm{P}(A | B)=\dfrac{12}{16}=0{,}75$.
	\itemch Đúng, vì số học sinh vừa tham gia câu lạc bộ tiếng Anh vừa tham gia câu lạc bộ Toán là $12$, số học sinh tham gia câu lạc bộ Tiếng Anh là $25$ nên $\mathrm{P}(B | A)=\dfrac{12}{25}=0{,}48$.
	\end{itemchoice}
	}
\end{ex}
%%==========Câu 14
\begin{ex}%[2D6V1-3]
	\immini{Cho sơ đồ hình cây như hình bên. Xét tính đúng, sai của các khẳng định sau
	\choiceTF
	{$\mathrm{P(AB)}=0{,}48$}
	{ $\mathrm{P(A| B)}=0{,}5$}
	{$\mathrm{P(\overline{A}| B)}=0{,}3$}
	{\True $\dfrac{\mathrm{P}(B) \mathrm{P}(\overline{A} | B)}{\mathrm{P}(\overline{A})}=0{,}6$}}
	{\begin{tikzpicture}[scale=.2,>=stealth]
	\tikzstyle{block} = [rectangle, draw, fill=blue!10\text{,} rounded corners, text centered, text width = 10em, minimum height = 2em]
	\node (c1) {};
	\node (c2)[above right = 1.5cm of c1] {$A$};
	\node at (0.5,5){\fbox{$0\text{,}2$}};
	\node at (0.5,-5){\fbox{$0\text{,}8$}};
	\node (c3) [below right= 1.5cm of c1]{$\overline{A}$};
	\node at (12,11.5){\fbox{$0\text{,}7$}};
	\node (c4) at (21.5, 12){$B$};
	\node (c5) at (21.5, 2){$\overline{B}$};
	\node at (12,3){\fbox{$0\text{,}3$}};
	\node (c6) at (21.5, -4){$B$};
	\node at (12,-4){\fbox{$0\text{,}6$}};
	\node (c7) at (21.5, -14){$\overline{B}$};
	\node at (12,-13){\fbox{$0\text{,}4$}};
	\draw[->] (c1.east) -- (c2.west);
	\draw[->] (c1.east) -- (c3.west);
	\draw[->] (c2.east) -- (c4.west);
	\draw[->] (c2.east) -- (c5.west);
	\draw[->] (c3.east) -- (c6.west);
	\draw[->] (c3.east) -- (c7.west);
	\end{tikzpicture}}
	\loigiai{
	\begin{itemchoice}
	\itemch Sai, vì $\mathrm{P}(B)=0\text{,}2\cdot 0\text{,}7+0\text{,}8\cdot 0\text{,}6=0\text{,}62$, $\mathrm{P}(\overline{A})=0\text{,}8$ và $\mathrm{P}(AB)=0\text{,}2\cdot 0\text{,}7=0\text{,}14$.
	\itemch Sai, vì $\mathrm{P}(A| B)=\dfrac{\mathrm{P}(AB)}{\mathrm{P}(B)}=\dfrac{0\text{,}14}{0\text{,}62}=\dfrac{7}{31}$.
	\itemch Sai, vì $\mathrm{P}(\overline{A} | B)=1-\mathrm{P}(A| B)=1-\dfrac{7}{31}=\dfrac{24}{31}$.
	\itemch Đúng, vì $\dfrac{\mathrm{P}(B) \mathrm{P}(\overline{A} | B)}{\mathrm{P}(\overline{A})}=\dfrac{0\text{,}62\cdot \dfrac{24}{31}}{0\text{,}8}=0\text{,}6$.
	\end{itemchoice}	
	}
\end{ex}
%%==========Câu 15
\begin{ex}%[2D6V1-3]
	Trong một hộp có $18$ quả bóng bàn loại I và $2$ quả bóng bàn loại II, các quả bóng bàn có hình dạng và kích thước như nhau. Một học sinh lấy ngẫu nhiên lần lượt $2$ quả bóng bàn (lấy không hoàn lại) trong hộp.\\ Xét tính đúng, sai của các khẳng định sau
	\choiceTF
	{Xác suất để lần thứ nhất lấy được quả bóng bàn loại II là $\dfrac{9}{10}$}
	{\True Xác suất để lần thứ hai lấy được quả bóng bàn loại II, biết lần thứ nhất lấy được quả bóng bàn loại II, là $\dfrac{1}{19}$}
	{Xác suất để cả hai lần đều lấy được quả bóng bàn loại II là $\dfrac{9}{190}$}
	{\True Xác suất để ít nhất $1$ lần lấy được quả bóng bàn loại I là $\dfrac{189}{190}$}
	\loigiai{
	Xét các biến cố\\
	$A\colon$ \lq\lq  Lần thứ nhất lấy được quả bóng bàn loại II\rq\rq;\\
	$B\colon$ \lq\lq  Lần thứ hai lấy được quả bóng bàn loại II\rq\rq. 
	\begin{itemchoice}
	\itemch Sai, vì Xác suất để lần thứ nhất lấy được quả bóng bàn loại II là $\mathrm{P}(A)=\dfrac{2}{20}=\dfrac{1}{10}$.
	\itemch Đúng, vì sau khi lấy $1$ quả bóng bàn loại II thì chỉ còn $1$ quả bóng bàn loại II trong hộp.\\ Suy ra xác suất để lần thứ hai lấy được quả bóng bàn loại II, biết lần thứ nhất lấy được quả bóng bàn loại II là $\mathrm{P}(B | A)=\dfrac{1}{19}$.
	\itemch Sai, vì xác suất để cả hai lần đều lấy được quả bóng bàn loại II là
	$$
	\mathrm{P}(C)=\mathrm{P}(A \cap B)=\mathrm{P}(A) \cdot \mathrm{P}(B | A)=\dfrac{1}{10} \cdot \dfrac{1}{19}=\dfrac{1}{190}.
	$$
	\itemch Đúng, vì để ít nhất $1$ lần lấy được quả bóng bàn loại I là
	$$
	\mathrm{P}(\overline {C})=1-\mathrm{P}(C)=1-\frac{1}{190}=\frac{189}{190} \text {. }
	$$
	\end{itemchoice}
	}
\end{ex}
%%==========Câu 16
\begin{ex}%[2D6V2-3]
	Một xưởng máy sử dụng một loại linh kiện được sản xuất từ hai cơ sở I và II. Số linh kiện do cơ sở I sản xuất chiếm $61 \%$, số linh kiện do cơ sở II sản xuất chiếm $39 \%$. Tỉ lệ linh kiện đạt tiêu chuẩn của cơ sở I, cơ sở II lần lượt là $93 \%$, $82 \%$. Kiểm tra ngẫu nhiên $1$ linh kiện ở xường máy. Xét các biến cố\\
	$A_1\colon$ \lq\lq  Linh kiện được kiểm tra do cơ sở I sản xuất\rq\rq;\\
	$A_2\colon$\lq\lq  Linh kiện được kiểm tra do cơ sở II sản xuất\rq\rq;\\
	$B\colon$ \lq\lq  Linh kiện được kiểm tra đạt tiêu chuẩn\rq\rq.\\
	Xét tính đúng, sai của các khẳng định sau
	\choiceTF
	{$\mathrm{P}\left(A_1\right)=0{,}39$} 
	{\True $\mathrm{P}\left(B | A_2\right)=0{,}82$}
	{$\mathrm{P}(B)=0{,}89$}
	{\True $\mathrm{P}\left(A_1 | B\right)=0{,}55$}
	\loigiai{
	\begin{itemchoice}
	\itemch Sai, vì $\mathrm{P}\left(A_1\right)=0{,}61$.
	\itemch Đúng, vì $\mathrm{P}\left(A_2\right)=0{,}39$;~ $\mathrm{P}\left(B | A_1\right)=0{,}93 ;~\mathrm{P}\left(B | A_2\right)=0{,}82$.
	\itemch Đúng, vì theo công thức xác suất toàn phần, ta có
	$$
	\mathrm{P}(B)=\mathrm{P}\left(A_1\right) \cdot \mathrm{P}\left(B | A_1\right)+\mathrm{P}\left(A_2\right) \cdot \mathrm{P}\left(B | A_2\right)=0{,}61\cdot 0{,}93+0{,}39\cdot 0{,}82=0{,}8871.
	$$
	\itemch Sai, vì theo công thức Bayes, ta có $$\mathrm{P}\left(A_1 | B\right)=\dfrac{\mathrm{P}\left(A_1\right) \cdot \mathrm{P}\left(B | A_1\right)}{\mathrm{P}(B)}=\dfrac{0{,}61 \cdot 0{,}93}{0{,}8871} \approx 0{,}64.$$
	\end{itemchoice}
	}
\end{ex}
%==============================================================
\Closesolutionfile{ans}
\indapan{3}{ans/ans-0-B15-DS}
%%==========HẾT PHẦN 2=========================================
\Opensolutionfile{ans}[ans/ans-0-B15-KQ]
%\TNSA
%%==========PHẦN 3=============================================
%%==========Câu 17
\begin{ex}%[2D6V2-2]
	Có hai hộp đựng các viên bi cùng kích thước và khối lượng. Hộp thứ nhất chứa $5$ viên bi đỏ và $5$ viên bi xanh, hộp thứ hai chứa $6$ viên bi đỏ và $4$ viên bi xanh. Lấy ngẫu nhiên một viên bi từ hộp thứ nhất chuyển sang hộp thứ hai, sau đó lấy ra ngẫu nhiên một viên bi từ hộp thứ hai. Tính xác suất để viên bi được lấy ra từ hộp thứ hai là viên bi đỏ (làm tròn đến hàng phần trăm).
	\shortans{$0{,}59$}
	\loigiai{
	Xét phép thử lấy ngẫu nhiên một viên bi từ hộp thứ nhất chuyển sang hộp thứ hai, sau đó lấy ra ngẫu nhiên một viên bi từ hộp thứ hai. Xét các biến cố sau
	\begin{itemize}
	\item $A$ là biến cố \lq \lq Viên bi được lấy ra từ hộp thứ hai là bi đỏ \rq \rq;
	\item $C$ là biến cố \lq \lq Viên bi được lấy ra từ hộp thứ hai là bi của hộp thứ nhất \rq \rq;
	\item $\overline{C}$ là biến cố \lq \lq Viên bi được lấy ra từ hộp thứ hai là bi của hộp thứ hai\rq \rq.
	\end{itemize}
	Sau khi chuyển một viên bi từ hộp thứ nhất sang hộp thứ hai thì hộp thứ hai có $11$ viên bi. Ta có 
	$$\mathrm{P}(C)=\dfrac{1}{11};\, \mathrm{P}(\overline{C})=\dfrac{10}{11}.$$
	Xác suất để viên bi được lấy ra từ hộp thứ hai là bi đỏ của hộp thứ nhất 
	$$\mathrm{P}(A\mid C)=\dfrac{5}{10}=\dfrac{1}{2}.$$
	Xác suất để viên bi được lấy ra từ hộp thứ hai là bi đỏ của hộp thứ hai 
	$$\mathrm{P}(A\mid \overline{C})=\dfrac{6}{10}=\dfrac{3}{5}.$$
	Áp dụng công thức xác suất toàn phần, ta có
	$$\mathrm{P}(A) = \mathrm{P}(C)\cdot \mathrm{P}(A\mid C) + \mathrm{P}(\overline{C})\cdot P(A\mid \overline{C}) =\dfrac{1}{11}\cdot \dfrac{1}{2}+\dfrac{10}{11}\cdot \dfrac{3}{5}=\dfrac{13}{22}\approx 0{,}59.$$
	}
\end{ex}
%%==========Câu 18
\begin{ex} %[2D6H1-4]
	Trong số $40$ học sinh lớp $12$A, có $22$ em đăng kí thi ngành Kinh tế, $25$ em đăng kí thi ngành Luật, $3$ em không đăng kí thi cả hai ngành này. Chọn ngẫu nhiên một học sinh, biết rằng em đó đăng kí thi ngành luật. Tính xác suất để em đó đăng kí thi ngành kinh tế. 	
	\shortans{$0{,}4$}
	\loigiai{	Gọi $A$ và $B$ lần lượt là tập hợp các học sinh đăng kí thi ngành kinh tế và ngành luật.
	Ta có $|A\cup B| =40-3 = 37$.\\
	Số sinh viên đăng kí cả hai ngành là $ |A\cap B| = |A|+|B|-|A\cup B|=10$.\\
	Vậy chọn ngẫu nhiên một học sinh, biết rằng em đó đăng kí thi ngành luật thì xác suất để em đó đăng kí thi ngành kinh tế là	 $\dfrac{10}{25}=\dfrac{2}{5}=0{,}4$.
	}
\end{ex}
%%==========Câu 19
\begin{ex} %[2D6V2-2]
	Trong một tuần, Sơn chọn ngẫu nhiên ba ngày chạy bộ buổi sáng. Nếu chạy bộ thì xác suất Sơn ăn thêm một quả trứng vào bữa sáng hôm đó là $0{,}7$ . Nếu không chạy bộ thì xác suất Sơn ăn thêm một quả trứng vào bữa sáng hôm đó là $0{,}25$. Chọn ngẫu nhiên một ngày trong tuần của Sơn. Tính xác suất để hôm đó Sơn chạy bộ nếu biết rằng bữa sáng hôm đó Sơn có ăn thêm một quả trứng (làm tròn đến hàng phần trăm).
	\par\shortans{$0{,}68$}	
	\loigiai{
	Gọi $A$	là biến cố \lq \lq Chọn ngày Sơn ăn trứng \rq \rq .\\
	Gọi $B_1$ là biến cố \lq \lq Ngày Sơn chạy bộ \rq \rq và $B_2$ là biến cố \lq \lq Ngày Sơn không chạy bộ \rq \rq.\\
	Ta có $\mathrm{P}(B_1) = \dfrac{3}{7}$ và $\mathrm{P}(B_2)=\dfrac{4}{7}$.\\
	Xác suất ngày ăn trứng và chạy bộ là
	$\mathrm{P}(A\mid B_1) = 0{,}7$.\\
	Xác suất ngày ăn trứng và không chạy bộ là
	$\mathrm{P}(A\mid B_2) = 0{,}25$.\\
	Khi đó ta có $\mathrm{P}(A) =\mathrm{P}(B_1)\cdot \mathrm{P}(A\mid B_1)+ \mathrm{P}(B_2)\cdot \mathrm{P}(A\mid B_2) = \dfrac{3}{7}\cdot 0{,}7+ \dfrac{4}{7}\cdot 0{,}25 = \dfrac{31}{70}$.\\
	Theo công thức Bayes, ta có xác suất hôm chọn Sơn chạy bộ mà trong bữa sáng có ăn một quả trứng là 
	$$\mathrm{P}(B_1\mid A) = \dfrac{\mathrm{P}(B_1A)}{P(A)} = \dfrac{\mathrm{P}(B_1)\cdot \mathrm{P}(A\mid B_1)}{\mathrm{P}(A)} = \dfrac{\dfrac{3}{7}\cdot 0{,}7}{ \dfrac{31}{70}}=\dfrac{21}{31}\approx 0{,}68.$$	
	}
\end{ex}
%%==========Câu 20
\begin{ex}%[2D6V2-2]
	Giả sử có khoảng $40 \%$ thư điện tử (email) gửi đến một địa chỉ là thư rác. Người ta sử dụng một thuật toán để phân loại thư rác, biết rằng thuật toán này có thể phân loại đến $99 \%$ thư rác và tỉ lệ sai sót khi phân loại thư bình thường thành thư rác là $5 \%$. Tính xác suất một thư điện tử là thư bình thường nếu thư này đã được phân loại đúng (làm tròn đến hàng phần trăm).
	\shortans{$0{,}59$}
	\loigiai{
	Ta có công thức
	$$	\mathrm{P}(A \mid B)=\dfrac{\mathrm{P}(B \mid A) \cdot \mathrm{P}(A)}{\mathrm{P}(B)}$$
	Trong đó
	\begin{itemize}
	\item $A$: Thư điện tử là thư bình thường.
	\item $B$: Thư đã được phân loại đúng.
	\item 	$\mathrm{P}(A)$: Xác suất một thư điện tử là thư bình thường ban đầu.\\
	Vì có $40 \%$ thư rác, nên $\mathrm{P}(A)=$ $1-0{,}4=0{,}6$.
	\item $\mathrm{P}(B \mid A)$: Xác suất một thư bình thường được phân loại đúng.\\
	Do tỉ lệ sai sót là $5 \%$, nên $\mathrm{P}(B \mid A)=1-0{,}05=0{,}95$.
	\item $\mathrm{P}(B)$: Xác suất một thư nào đó được phân loại đúng, tính bằng tổng xác suất một thư rác được phân loại đúng và xác suất một thư bình thường được phân loại đúng
	\end{itemize}
	$$\begin{aligned}[t]
	\mathrm{P}(B)&=\mathrm{P}(B \mid A) \cdot \mathrm{P}(A)+\mathrm{P}(B \mid \overline{A}) \cdot \mathrm{P}(\overline{A}) \\&
	=0{,}95 \cdot 0{,}6+0{,}99 \cdot 0{,}4=0{,}97.
	\end{aligned}$$
	Áp dụng định lý Bayes, ta có
	$$\mathrm{P}(A \mid B)=\dfrac{\mathrm{P}(B \mid A) \cdot \mathrm{P}(A)}{\mathrm{P}(B)}=\dfrac{0{,}95 \cdot 0{,}6}{0{,}97}=\dfrac{57}{97}\approx0{,}59.$$
	}
\end{ex}
% \begin{ex}%[2D6V2-1]
% 	\immini{Cho sơ đồ hình cây như hình bên. Tính giá trị của biểu thức $\dfrac{\mathrm{P}(B) \mathrm{P}(\overline{A} \mid B)}{\mathrm{P}(\overline{A})}$.
% 	\par\shortans{$0{,}6$
% 	}
% 	}
% 	{\begin{tikzpicture}[scale=.2,>=stealth]
% 	\tikzstyle{block} = [rectangle, draw, fill=blue!10\text{,} rounded corners, text centered, text width = 10em, minimum height = 2em]
% 	\node (c1) {};
% 	\node (c2)[above right = 1.5cm of c1] {$A$};
% 	\node at (0.5,5){\fbox{$0\text{,}2$}};
% 	\node at (0.5,-5){\fbox{$0\text{,}8$}};
% 	\node (c3) [below right= 1.5cm of c1]{$\overline{A}$};
% 	\node at (12,11.5){\fbox{$0\text{,}7$}};
% 	\node (c4) at (21.5, 12){$B$};
% 	\node (c5) at (21.5, 2){$\overline{B}$};
% 	\node at (12,3){\fbox{$0\text{,}3$}};
% 	\node (c6) at (21.5, -4){$B$};
% 	\node at (12,-4){\fbox{$0\text{,}6$}};
% 	\node (c7) at (21.5, -14){$\overline{B}$};
% 	\node at (12,-13){\fbox{$0\text{,}4$}};
% 	\draw[->] (c1.east) -- (c2.west);
% 	\draw[->] (c1.east) -- (c3.west);
% 	\draw[->] (c2.east) -- (c4.west);
% 	\draw[->] (c2.east) -- (c5.west);
% 	\draw[->] (c3.east) -- (c6.west);
% 	\draw[->] (c3.east) -- (c7.west);
% 	\end{tikzpicture}}
% 	\loigiai{
% 	\begin{itemize}
% 	\item Ta có $\mathrm{P}(B)=0{,}2\cdot 0{,}7+0{,}8\cdot 0{,}6=0{,}62$; $\mathrm{P}(\overline{A})=0{,}8$ và $\mathrm{P}(AB)=0{,}2\cdot 0{,}7=0{,}14$.
% 	\item $\mathrm{P}(A\mid B)=\dfrac{\mathrm{P}(AB)}{\mathrm{P}(B)}=\dfrac{0{,}14}{0{,}62}=\dfrac{7}{31}$.
% 	\item $\mathrm{P}(\overline{A} \mid B)=1-\mathrm{P}(A\mid B)=1-\dfrac{7}{31}=\dfrac{24}{31}$.
% 	\item $\dfrac{\mathrm{P}(B) \mathrm{P}(\overline{A} \mid B)}{\mathrm{P}(\overline{A})}=\dfrac{0{,}62\cdot \dfrac{24}{31}}{0{,}8}=0{,}6$.
% 	\end{itemize}	
% 	}
% \end{ex}
\begin{ex}%[2D6V2-4]
	Năm $2001$, Cộng đồng châu Âu có làm một đợt kiểm tra rất rộng rãi các con bò để phát hiện những con bị bệnh bò điên. Không có xét nghiệm nào cho kết quả chính xác $100 \%$. Một loại xét nghiệm, mà ở đây ta gọi là xét nghiệm A cho kết quả như sau: khi con bò bị bệnh bò điên thì xác suất để có phản ứng dương tính trong xét nghiệm A là $70 \%$ còn khi con bò không bị bệnh thì xác suất để có phản ứng dương tính trong xét nghiệm A là $10 \%$. Biết rằng tỉ lệ bò bị mắc bệnh bò điên ở Hà Lan là $13$ con trên $1~000~000$ con \textit{(Nguồn: F. M. Dekking et al., Amodern introduction to probability and statistics Understanding why and how, Springer, $2005$)}. Khi con bò ở Hà Lan có phản ứng dương tính với xét nghiệm A thì xác suất để nó bị mắc bệnh bò điên là $\mathrm{P}$, tính $1000P$ (lấy gần đúng đến hàng phần trăm).
	\shortans{$ 0{,}09$}
	\loigiai{
	Xét hai biến cố\\	
	$N$: \lq\lq  Con bò được chọn bị nhiễm bệnh\rq\rq.\\	
	$D$: \lq\lq  Con bò được chọn có phản ứng dương tính\rq\rq.\\	
	Khi đó, ta có\\	
	\[\mathrm{P}(N)=\dfrac{13}{1 000 000}=0{,}000013; \qquad \mathrm{P}(\overline{N})=1-\mathrm{P}(N)=0{,}999987;\]	
	\[\mathrm{P}(D|N)=70\%=0{,}7; \qquad \mathrm{P}(D|\overline{N})=10\%=0{,}1.\]
	Áp dụng công thức Bayes, ta có\\
	$\mathrm{P}(N|D)=\dfrac{\mathrm{P}(D|N) \cdot \mathrm{P}(N)}{\mathrm{P}(N) \cdot \mathrm{P}(D|N)+\mathrm{P}(\overline{N})\mathrm{P}(D|\overline{N})}=\dfrac{0{,}7 \cdot 0{,}000013}{0{,}7 \cdot 0{,}000013+0{,}1 \cdot 0{,}999987}\approx 0{,}009\%$.\\
	Do đó $1000\mathrm{P}\approx1000\cdot 0{,}009\%\approx 0{,}09$.
	}
\end{ex}

%\TNSA
%%%%%----------Câu 17
\begin{ex}%[2D6V1-2]%[Võ Thanh Hiệp]
	Lớp 12A có $40$ học sinh, trong đó có $22$ bạn nữ và $18$ bạn nam. Có $3$ tên Hiền gồm hai bạn nam và một bạn nữ. Thầy giáo chọn ngẫu nhiên một bạn lên bảng làm bài tập. Tính xác suất để chọn đúng bạn tên Hiền là bạn nam ({\it kết quả làm tròn đến hàng phần trăm}).
	\shortans{$0{,}25$}
	\loigiai{
	Gọi $A$ là biến cố: \lq\lq  Chọn bạn tên Hiền\rq\rq.\\
	Gọi $B$ là biến cố: \lq\lq  Chọn bạn nam\rq\rq.\\
	Ta có $\mathrm{P}\left(A\right) = \dfrac{3}{40}$,
	$\mathrm{P}\left(B\right) = \dfrac{18}{40}=\dfrac{9}{20}$, 
	$\mathrm{P}\left(AB\right) = \dfrac{2}{18}=\dfrac{1}{9}$.\\
	Xác suất chọn đúng bạn tên Hiền với điều kiện là bạn nam $\mathrm{P}\left(A\mid B\right)$.\\
	Ta có $\mathrm{P}\left(A\mid B\right)=\dfrac{\mathrm{P}\left(AB\right)}{\mathrm{P}\left(B\right)}=
	\dfrac{20}{81} \approx 0{,}25$.
	}
\end{ex}
%%%%%----------Câu 18
\begin{ex}%[2D6V1-2]%[Võ Thanh Hiệp]
	Một hộp đựng $30$ viên bi kích thước, chất liệu như nhau, trong đó có $20$ viên bi xanh và $10$ viên bi trắng. Lấy ngẫu nhiên ra một viên bi không bỏ lại trong hộp, rồi lại lấy ngẫu nhiên ra một viên bi nữa. Tính xác suất để lấy được một viên bi trắng ở lần thứ nhất và một viên bi xanh ở lần thứ hai ({\it kết quả làm tròn đến hàng phần trăm}).
	\shortans{$0{,}23$}
	\loigiai{
	Gọi $A$ là biến cố: \lq\lq  Lấy được một viên bi trắng ở lần thứ nhất\rq\rq.\\
	Gọi $B$ là biến cố: \lq\lq  Lấy được một viên bi xanh ở lần thứ hai\rq\rq.\\
	Ta có $P\left(A\right) = \dfrac{10}{30}=\dfrac{1}{3}$.\\
	Nếu $A$ đã xảy ra, tức là một viên bi trắng đã được lấy ra ở lần thứ nhất, thì còn lại trong hộp $29$ viên bi trong đó số viên bi xanh là 20, do đó $P\left(B\mid A\right)=\dfrac{20}{29}$.\\
	Xác suất để lấy được một viên bi trắng ở lần thứ nhất và một viên bi xanh ở lần thứ hai là $P\left(A B\right)$.\\
	Ta có $P\left(A B\right) = P\left(A\right) \cdot P\left(B\mid A\right)=
	\dfrac{1}{3} \cdot \dfrac{20}{29}=\dfrac{20}{87}\approx 0{,}23$.
	}
\end{ex}
%%%%%----------Câu 19
\begin{ex}%[2D6V2-2]%[Võ Thanh Hiệp]
	Khảo sát tỉ lệ người dân trong một xã nghiện thuốc lá là $20\%$; tỉ lệ người bị bệnh phổi trong số người nghiện thuốc lá là $70\%$, trong số người không nghiện thuốc lá là $15\%$. Hỏi khi ta gặp ngẫu nhiên một người dân của của xã đó thì khả năng mà người đó bị bệnh phổi là bao nhiêu $\%$?
	\shortans{$26$}
	\loigiai{
	Gọi $A$ là biến cố: \lq\lq  Người nghiện thuốc lá\rq\rq.\\
	Suy ra $\overline{A}$ là biến cố: \lq\lq  Người không nghiện thuốc lá\rq\rq. \\
	Gọi $B$ là biến cố: \lq\lq  Người bị bệnh phổi\rq\rq.\\
	Xác suất người nghiện thuốc lá là $\mathrm{P}\left(A\right) = 20\% = 0{,}2$.\\
	Xác suất người không nghiện thuốc lá là $\mathrm{P}\left(\overline{A}\right) =1- \mathrm{P}\left(A\right) = 0{,}8$.\\
	Xác suất người bị bệnh phổi trong số người nghiện thuốc lá là $70\% \Rightarrow \mathrm{P}\left(B\mid A\right) = 0{,}7$.\\
	Xác suất người bị bệnh phổi không nghiện thuốc lá là $15\% \Rightarrow \mathrm{P}\left(B\mid \overline{A}\right) = 0{,}15$.\\
	Xác suất người bị bệnh phổi là\\
	$\mathrm{P}\left(B\right) = \mathrm{P}\left(A\right) \cdot \mathrm{P}\left(B\mid A\right)+
	\mathrm{P}\left(\overline{A}\right) \cdot \mathrm{P}\left(B\mid \overline{A}\right)
	= 0{,}2\cdot 0{,}7+0{,}8\cdot 0{,}15=0{,}26=26\%$.
	}
\end{ex}
%%%%%----------Câu 20
\begin{ex}%[2D6V2-2]%[Võ Thanh Hiệp]
	Có $2$ xạ thủ loại I và $8$ xạ thủ loại II, xác suất bắn trúng đích của các loại xạ thủ loại I là $0{,}9$ và loại II là $0{,}7$. Chọn ngẫu nhiên ra một xạ thủ và xạ thủ đó bắn một viên đạn. Tìm xác suất để viên đạn đó trúng đích.
	\par\shortans{$0{,}74$}
	\loigiai{
	Gọi $A$ là biến cố: \lq\lq  Viên đạn bắn trúng đích\rq\rq.\\	
	Gọi $B$ là biến cố: \lq\lq  Chọn xạ thủ loại I\rq\rq.\\
	Gọi $C$ là biến cố: \lq\lq  Chọn xạ thủ loại II\rq\rq.\\	
	Xác suất biến cố $B$ là $\mathrm{P}\left(B\right)=\dfrac{2}{10}=0{,}2$.
	Xác suất biến cố $C$ là $\mathrm{P}\left(C\right)=\dfrac{8}{10}=0{,}8$.
	Xác suất biến cố viên đạn đó trúng đích với điều kiện là xạ thủ loại I là
	$\mathrm{P}\left(A\mid B\right)= 0{,}9$.\\
	Xác suất biến cố viên đạn đó trúng đích với điều kiện là xạ thủ loại II là
	$\mathrm{P}\left(A\mid C\right)= 0{,}7$.\\
	$\mathrm{P}\left(A\right) = \mathrm{P}\left(B\right) \cdot \mathrm{P}\left(A\mid B\right)+
	\mathrm{P}\left(C\right) \cdot \mathrm{P}\left(A\mid C\right)
	= 0{,}2\cdot 0{,}9+0{,}8\cdot 0{,}7=0{,}74$.
	}
\end{ex}
%%%%%----------Câu 21
\begin{ex}%[2D6V2-3]%[Võ Thanh Hiệp]
	Khảo sát sự yêu thích môn Toán của hai lớp 12 của một trường. Lớp 12A1 có 40 học sinh và có $80\%$ học sinh thích môn Toán, lớp 12A2 có 32 học sinh và có $75\%$ học sinh thích môn Toán. Chọn ngẫu nhiên một học sinh. Biết rằng bạn đó yêu thích môn Toán, tính xác suất bạn đó học lớp 12A1 ({\it kết quả làm tròn đến hàng phần trăm}).
	\shortans{$0{,}57$}
	\loigiai{
	Gọi $A$ là biến cố: \lq\lq  Học sinh yêu thích môn Toán\rq\rq.\\
	Gọi $B$ là biến cố: \lq\lq  Học sinh lớp 12A1\rq\rq.\\
	Gọi $C$ là biến cố: \lq\lq  Học sinh lớp 12A2\rq\rq.\\	
	Theo đề bài ta có\\
	$\mathrm{P}\left(B\right) = \dfrac{40}{72} =\dfrac{5}{9}$;
	$\mathrm{P}\left(C\right) = \dfrac{32}{72} =\dfrac{4}{9}$;
	$\mathrm{P}\left(A\mid B\right) = 80\% =\dfrac{4}{5}$;
	$\mathrm{P}\left(A\mid C\right) = 75\% =\dfrac{3}{4}$.\\ 
	Áp dụng công thức xác suất toàn phần, ta có\\
	$\mathrm{P}\left(A\right) = \mathrm{P}\left(B\right) \cdot \mathrm{P}\left(A\mid B\right)+
	\mathrm{P}\left(C\right) \cdot \mathrm{P}\left(A\mid C\right)
	= \dfrac{5}{9}\cdot \dfrac{4}{5}+ \dfrac{4}{9}\cdot \dfrac{3}{4}=\dfrac{7}{9}$.\\
	Xác suất cần tìm là 
	$\mathrm{P}\left(B\mid A\right)=
	\dfrac{\mathrm{P}\left(A\mid B\right)\cdot \mathrm{P}\left(B\right) }{\mathrm{P}\left(A\right)}=
	\dfrac{\dfrac{4}{5}\cdot\dfrac{5}{9}}{\dfrac{7}{9}}=\dfrac{4}{7}\approx 0{,}57$.
	}
\end{ex}
%%%%%----------Câu 22
\begin{ex}%[2D6V2-3]%[Võ Thanh Hiệp]
	Hộp thứ nhất có 6 viên bi đỏ và 4 viên bi xanh, hộp thứ hai có 4 viên bi đỏ và 6 viên bi xanh, các viên bi có cùng khối lượng và kích thước. Lấy ngẫu nhiên 1 viên bi từ hộp thứ nhất bỏ sang hộp thứ hai. Sau đó từ hộp thứ hai lấy ngẫu nhiên ra một viên bi. Biết rằng viên bi lấy ra từ hộp hai là viên bi màu đỏ. Tính xác suất viên bi bỏ từ hộp thứ nhất sang hộp thứ hai là màu xanh ({\it kết quả làm tròn đến hàng phần trăm}).
	\shortans{$0{,}35$}
	\loigiai{
	Gọi $A$ là biến cố: \lq\lq  Bi bỏ từ hộp thứ nhất sang hộp thứ hai là bi màu xanh \rq\rq.\\	
	Suy ra $\overline{A}$ là biến cố: \lq\lq  Bi bỏ từ hộp thứ nhất sang hộp thứ hai là bi màu đỏ \rq\rq.\\
	Gọi $B$ là biến cố: \lq\lq  Bi lấy từ hộp thứ hai là bi màu đỏ \rq\rq.\\	
	Theo đề bài ta có\\
	$\mathrm{P}\left(A\right)= \dfrac{4}{10}$;
	$\mathrm{P}\left(\overline{A}\right)= \dfrac{6}{10}$;
	$\mathrm{P}\left(B\mid A\right)=\dfrac{4}{11}$;
	$\mathrm{P}\left(B\mid \overline{A}\right)=\dfrac{5}{11}$.\\
	Áp dụng công thức xác suất toàn phần, ta có\\
	$\mathrm{P}\left(B\right) =\mathrm{P}\left(A\right) \cdot \mathrm{P}\left(B\mid A\right)+
	\mathrm{P}\left(\overline{A}\right) \cdot \mathrm{P}\left(B\mid \overline{A}\right)=
	\dfrac{4}{10}\cdot\dfrac{4}{11}+ \dfrac{6}{10}\cdot\dfrac{5}{11}=\dfrac{23}{55}$.\\
	Xác suất cần tìm là 
	$\mathrm{P}\left(A\mid B\right)=
	\dfrac{\mathrm{P}\left(B\mid A\right)\cdot \mathrm{P}\left(A\right) }{\mathrm{P}\left(B\right)}=
	\dfrac{\dfrac{4}{11}\cdot\dfrac{4}{10}}{\dfrac{23}{55}}=\dfrac{8}{23}\approx 0{,}35$.
	}
\end{ex}
\Closesolutionfile{ans}
\indapan{3}{ans/ans-2-B6-De2-TLN}

%HK1
% \foreach \i in {1,...,5} {\input{data/12/CK1/De_\i.tex}}

%C1
% \begin{name}
	{\tenchude}
	{ĐỀ ÔN TẬP CHƯƠNG I}
	{LỚP TOÁN THẦY PHÁT}
	{\thoigian}
\end{name}
\TN
\Opensolutionfile{ans}[ans/ans\currfilebase-Phan-I]
\begin{ex}%[2-D1B5-SO-13-2425]%[VN-MT-7, Lê Hải Phụng]%[2D1N1-2]
Cho hàm số $y=f(x)$ có bảng biến thiên như sau:
\begin{center}
\begin{tikzpicture}
\tkzTabInit[nocadre=true,lgt=1.2,espcl=2.5,deltacl=0.5]
{$x$/0.7,$f'(x)$/0.7,$f(x)$/2}
{$-\infty$,$-1$,$0$,$1$,$+\infty$}
\tkzTabLine{,-,0,+,0,-,0,+,}
\tkzTabVar{+/$+\infty$,-/$-1$,+/$4$,-/$-1$,+/$+\infty$}
\end{tikzpicture}
\end{center}
Hàm số đã cho đồng biến trên khoảng nào dưới đây?
\choice
{$(-\infty;-1)$}
{$(-1;1)$}
{$(0;1)$}
{\True $(-1;0)$}

\loigiai{
Dựa vào bảng biến thiên, ta thấy hàm số đã cho đồng biến trên $(-1;0)$ và $(1;+\infty)$
}
\end{ex}

\begin{ex}%[2-D1B5-SO-13-2425]%[VN-MT-7, Lê Hải Phụng]%%[2D1N2-2]
 Cho hàm số $f(x)$ có bảng biến thiên như sau:
 \begin{center}
 \begin{tikzpicture}
 \tkzTabInit[nocadre=true,lgt=1,espcl=3,deltacl=0.5]
 {$x$/0.7,$y'$/0.7,$y$/2}
 {$-\infty$,$1$,$3$,$+\infty$}
 \tkzTabLine{,+,0,-,0,+,} 
 \tkzTabVar{-/$-\infty$,+/$3$,-/$-2$,+/$+\infty$}
 \end{tikzpicture}
 \end{center}
 Hàm số $f(x)$ đạt cực đại tại 
 \choice
 {$x=-2$}
 {$x=3$}
 {\True $x=1$}
 {$x=2$}
 
 \loigiai{
 Hàm số $f(x)$ đạt cực đại tại $x=1$.
 }
\end{ex}

\begin{ex}%[2-D1B5-SO-13-2425]%[VN-MT-7, Lê Hải Phụng]%[2D1H5-1]
 Hàm số $y = f(x)$ có bảng biến thiên như sau:
 \begin{center}
 \begin{tikzpicture}
 \tkzTabInit[nocadre=true,lgt=1,espcl=3,deltacl=0.5]
 {$x$/0.7,$y'$/0.7,$y$/2}{$-\infty$,$1$,$+\infty$}
 \tkzTabLine{,+,d,+,}
 \tkzTabVar{-/$2$,+D-/$+\infty$/$-\infty$,+/$2$}
 \end{tikzpicture}
 \end{center}
 Hàm số đồng biến trên
 \choice
 {\True $(1;+\infty)$}
 {$(-\infty;2)$}
 {$\mathbb{R}$}
 {$\mathbb{R} \setminus \{1\}$}
 \loigiai{
 Dựa vào bảng biến thiên và bốn đáp án, hàm số đồng biến chỉ đúng với đáp án là khoảng $(1;+\infty)$.
 }
\end{ex}

\begin{ex}%[2-D1B5-SO-13-2425]%[VN-MT-7, Lê Hải Phụng]%[2D1N3-2]
 Giá trị lớn nhất của hàm số $y=x+\dfrac{4}{x}$ trên $(-4;0)$ là
 \choice
 {\True $-4$}
 {$4$}
 {$-5$}
 {$5$}
 \loigiai{Tập xác định $\mathscr{D}=\mathbb{R}\setminus \{0\}$.\\
 Đạo hàm $y'=1-\dfrac{4}{x^2}=\dfrac{x^2-4}{x^2}$.\\
 Xét $y'=0\Leftrightarrow \dfrac{x^2-4}{x^2}=0\Leftrightarrow x^2-4=0\Leftrightarrow\hoac{&x=2\\&x=-2.}$\\
 Suy ra $x=-2$ vì $x\in (-4;0)$.\\
 Ta có $y(-2)=-4$.
\begin{center}
 \begin{tikzpicture}
 \tkzTabInit[nocadre=true,lgt=1.2,espcl=2.5,deltacl=0.5]
 {$x$/0.7,$y'$/0.7,$y$/2}
 {$-4$,$-2$,$0$}
 \tkzTabLine{,+,0,-,d,}
 \tkzTabVar{-/$-5$,+/$-4$,-D/$-\infty$}
 \end{tikzpicture}
 \end{center}
 Vậy giá trị lớn nhất của hàm số $y=x+\dfrac{4}{x}$ trên $(-4;0)$ là $-4$.}
\end{ex}

\begin{ex}%[2-D1B5-SO-13-2425]%[VN-MT-7, Lê Hải Phụng]%[2D1H3-1]
\immini{ Cho hàm số $f(x)$ có đồ thị trên $[-3;3]$ như hình vẽ.
 Giá trị lớn nhất $M$ và giá trị nhỏ nhất $m$ của hàm số $f(x)$ trên $[-3;3]$ lần lượt là
 \choice[2]
 {$M=3$; $m=-1$}
 {\True $M=4$; $m=-2$}
 {$M=3$; $m=-3$}
 {$M=-1$; $m=1$}
 }{ \begin{tikzpicture}[scale=0.7, font=\normalsize, line join=round, line cap=round,>=stealth]
 \def\xmin{-4} \def\xmax{4}
 \def\ymin{-3} \def\ymax{5}
 \draw[->] (\xmin,0)--(0,0)node[above right]{$O$}--(\xmax,0)node[below]{$x$};
 \draw[->] (0,\ymin)--(0,\ymax) node [left]{$y$};
 \clip (\xmin+0.1,\ymin+0.1) rectangle (\xmax-0.1,\ymax-0.1);
 \draw[smooth,samples=100]plot[domain=-3:-1](\x,{-1.25*(\x)^2-2.5*(\x)+1.75});
 \draw[smooth,samples=100]plot[domain=-1:1](\x,{-2*(\x)^3+1});
 \draw[smooth,samples=100]plot[domain=1:3](\x,{-3.5+2.5*(\x)});
 \draw[dashed] (-3,0)--(-3,-2)--(0,-2) (-1,0)--(-1,3)--(0,3) (0,-1)--(1,-1)--(1,0) (3,0)--(3,4)--(0,4);
 \path 
 (-3,0) node[below left]{$-3$}
 (-1,0) node[below]{$-1$}
 (1,0) node[above]{$1$}
 (3,0) node[below]{$3$}
 (0,-2) node[right]{$-2$}
 (0,-1) node[left]{$-1$}
 (0,3) node[right]{$3$}
 (0,4) node[left]{$4$};
 \fill (-3,-2) circle (1pt);
 \fill (-1,3) circle (1pt);
 \fill (1,-1) circle (1pt);
 \fill (3,4) circle (1pt);
 \end{tikzpicture}}
 \loigiai{Từ đồ thị, ta có giá trị lớn nhất $M=4$ và giá trị nhỏ nhất $m=-2$ của hàm số $f(x)$ trên $[-3;3]$.}
\end{ex}

\begin{ex}%[2-D1B5-SO-13-2425]%[VN-MT-7, Lê Hải Phụng]%[2D1H4-1]
 Đồ thị hàm số $y=\dfrac{x+1}{x^2+x-2}$ có bao nhiêu đường tiệm cận đứng?
 \choice
 {$1$}
 {\True $2$}
 {$3$}
 {$4$}
 \loigiai{Ta có $y=\dfrac{x+1}{x^2+x-2}=\dfrac{x+1}{(x-1)(x+2)}$.\\
 Hàm số đã cho có tập xác định là $\mathbb{R}\setminus\{1;-2\}$.\\
 Ta có \begin{itemize}
 \item $\lim\limits_{x\to1^-}f(x)=\lim\limits_{x\to1^-}\dfrac{x+1}{(x-1)(x+2)}=-\infty$.
 \item $\lim\limits_{x\to1^+}f(x)=\lim\limits_{x\to1^-}\dfrac{x+1}{(x-1)(x+2)}=+\infty$.
 \end{itemize} 
Suy ra $x=1$ là tiệm cận đứng của đồ thị hàm số.\\
Ta có \begin{itemize}
 \item $\lim\limits_{x\to-2^-}f(x)=\lim\limits_{x\to-2^-}\dfrac{x+1}{(x-1)(x+2)}=-\infty$.
 \item $\lim\limits_{x\to-2^+}f(x)=\lim\limits_{x\to-2^-}\dfrac{x+1}{(x-1)(x+2)}=+\infty$.
\end{itemize} 
Suy ra $x=-2$ là tiệm cận đứng của đồ thị hàm số.\\
Vậy đồ thị hàm số đã cho có $2$ đường tiệm cận đứng.}
\end{ex}

\begin{ex}%[2-D1B5-SO-13-2425]%[VN-MT-7, Lê Hải Phụng]%[2D1H4-1]
 Cho hàm số $y=f(x)=\dfrac{6x^2+7x-2023}{2x^2+3x+2024}$. Đồ thị hàm số có tiệm cận ngang là
 \choice
 {\True $y=3$}
 {$y=0$}
 {$y=1$}
 {$y=2$}
 \loigiai{Tập xác định của hàm số đã cho là $\mathbb{R}$.\\
 Ta có \begin{itemize}
 \item $\lim\limits_{x\to+\infty}f(x)=\lim\limits_{x\to+\infty}\dfrac{6x^2+7x-2023}{2x^2+3x+2024}=\dfrac{6}{2}=3$.
 \item $\lim\limits_{x\to-\infty}f(x)=\lim\limits_{x\to-\infty}\dfrac{6x^2+7x-2023}{2x^2+3x+2024}=\dfrac{6}{2}=3$.
 \end{itemize} 
 Suy ra $y=3$ là tiệm cận ngang của đồ thị hàm số.}
\end{ex}

\begin{ex}%[2-D1B5-SO-13-2425]%[VN-MT-7, Lê Hải Phụng]%[2D1H4-1]
 Tiệm cận xiên của đồ thị hàm số $y=\dfrac{x^3+x^2-2x-1}{x^2-2}$ là đường thẳng có phương trình
 \choice
 {$y=2x+1$}
 {\True $y=x+1$}
 {$y=-x+1$}
 {$y=x$}
 \loigiai{Tập xác định của hàm số $\mathscr{D}=\mathbb{R}\setminus\left\lbrace \pm\sqrt{2}\right\rbrace $.\\
 Phương trình đường tiệm cận xiên có dạng $y=ax+b$.\\
 Trong đó
 \begin{itemize}
 \item $a=\lim\limits_{x\to +\infty} \dfrac{f(x)}{x}=\lim\limits_{x\to +\infty}\dfrac{x^3+x^2-2x-1}{x^3-2x}=1$.
 \item $b=\lim\limits_{x\to +\infty} \left[f(x)-ax\right]=\lim\limits_{x\to +\infty} \left(\dfrac{x^3+x^2-2x-1}{x^2-2}-x\right)=\lim\limits_{x\to +\infty}\dfrac{x^2+1}{x^2-2}=1$.
 \end{itemize}
 Ta cũng có \begin{itemize}
 \item $a=\lim\limits_{x\to -\infty} \dfrac{f(x)}{x}=\lim\limits_{x\to -\infty}\dfrac{x^3+x^2-2x-1}{x^3-2x}=1$.
 \item $b=\lim\limits_{x\to -\infty} \left[f(x)-ax\right]=\lim\limits_{x\to -\infty} \left(\dfrac{x^3+x^2-2x-1}{x^2-2}-x\right)=\lim\limits_{x\to -\infty}\dfrac{x^2+1}{x^2-2}=1$.
 \end{itemize}
 Và $\lim\limits_{x\to \pm\infty} \left[f(x)-(ax+b)\right]=\lim\limits_{x\to \pm\infty} \left[\dfrac{x^3+x^2-2x-1}{x^2-2}-(x+1)\right]=0$.\\
 Do đó, đồ thị hàm số có tiệm cận xiên là đường thẳng $y=x+1$. }
\end{ex}

\begin{ex}%[2-D1B5-SO-13-2425]%[VN-MT-7, Lê Hải Phụng]%[2D1H5-1]
\immini{Đồ thị của hàm số nào dưới đây có dạng như đường cong trong hình bên?
 \choice[2]
 {\True $y=x^3-2024x$}
 {$y=-x^3+3x$}
 {$y=x^3-3x^2+2024$}
 {$y=-x^3+3x^2-2$}
}{ \begin{tikzpicture}[scale=0.7, line join=round, line cap=round,>=stealth]
 \tikzset{every node/.style={scale=0.7}}
 \draw[->] (-2.5,0)--(2.5,0) node[below left] {$x$};
 \draw[->] (0,-2.5)--(0,2.5) node[below left] {$y$};
 \draw (0,0) node [below left] {$O$};
 \fill (0,0) circle (1pt);
 \begin{scope}
 \clip (-2.5,-2.5) rectangle (2.5,2.5);
 \draw[samples=200,domain=-2:2,smooth,variable=\x] plot (\x,{1*((\x)^3)+0*((\x)^2)+-3*(\x)+0});
 \end{scope}
\end{tikzpicture} }
 \loigiai{Đường cong có dạng của đồ thị hàm số bậc ba với hệ số $a>0$ và đi qua $O(0;0)$. Do đó đồ thị trên của hàm số $y=x^3-2024x$.}
\end{ex}

\begin{ex}%[2-D1B5-SO-13-2425]%[VN-MT-7, Lê Hải Phụng]%[2D1H5-1]
 \immini{Cho hàm số $y=\dfrac{ax+b}{cx+d}$ có đồ thị là đường cong trong hình vẽ bên. Tọa độ giao điểm của đồ thị
 hàm số đã cho và trục tung là
 \choice[2]
 {\True $(0;-2)$}
 {$(2;0)$}
 {$(-2;0)$}
 {$(0;2)$}
 }{ \begin{tikzpicture}[scale=.5, font=\normalsize, line join=round, line cap=round,>=stealth]
 \def\xmin{-7} \def\xmax{6}
 \def\ymin{-5} \def\ymax{6}
 \draw[->] (\xmin,0)--(0,0)node[below left]{$O$}--(\xmax,0)node[below]{$x$};
 \draw[->] (0,\ymin)--(0,\ymax) node [left]{$y$};
 \clip (\xmin+0.1,\ymin+0.1) rectangle (\xmax-0.1,\ymax-0.1);
 \draw[smooth,samples=100]plot[domain=\xmin:\xmax](\x,{(\x-2)/(\x+1)}) (\xmin,1)--(\xmax,1);
 \path 
 (-1,0) node[below left]{$-1$}
 (2,0) node[below]{$2$}
 (0,1) node[above right]{$1$}
 (0,-2) node[right]{$-2$};
 \fill (0,0) circle (1pt);
 \fill (-1,1) circle (1pt);
 \fill (0,1) circle (1pt);
 \fill (-1,0) circle (1pt);
 \fill (2,0) circle (1pt);
 \fill (0,-2) circle (1pt);
 \end{tikzpicture}}
 \loigiai{Từ đồ thị hàm số đã cho, ta có tọa độ giao điểm của đồ thị hàm số đã cho và trục tung là $(0;-2)$.}
\end{ex}

\begin{ex}%[2-D1B5-SO-13-2425]%[VN-MT-7, Lê Hải Phụng]%[2D1H5-1]
 \immini{Đồ thị của hàm số nào dưới đây có dạng như đường cong trong hình bên dưới?
 \choice
 {$y=x+2$}
 {$y=\dfrac{x^2-2x+2}{x+1}$}
 {$y=x^2-2x+2$}
 {\True $\dfrac{x^2+2x+2}{x+1}$}
 }{ \begin{tikzpicture}[scale=.6, font=\normalsize, line join=round, line cap=round,>=stealth]
 \def\xmin{-5} \def\xmax{5}
 \def\ymin{-4} \def\ymax{5}
 \draw[->] (\xmin,0)--(0,0)node[below right]{$O$}--(\xmax,0)node[below]{$x$};
 \draw[->] (0,\ymin)--(0,\ymax) node [left]{$y$};
 \clip (\xmin+0.1,\ymin+0.1) rectangle (\xmax-0.1,\ymax-0.1);
 \draw[smooth,samples=125]plot[domain=\xmin:\xmax](\x,{((\x)^2+2*(\x)+2)/(\x+1)});
 \draw[smooth,samples=125]plot[domain=\xmin:\xmax](\x,{(\x)+1});
 \draw[dashed] (-2,0)--(-2,-2)--(0,-2);
 \path 
 (-1,0) node[below right]{$-1$}
 (2,0) node[below]{$2$}
 (-2,0) node[above]{$-2$}
 (0,1) node[right]{$1$}
 (0,-2) node[right]{$-2$};
 \fill (0,0) circle (1pt);
 \fill (0,1) circle (1pt);
 \fill (0,-2) circle (1pt);
 \fill (2,0) circle (1pt);
 \fill (-2,0) circle (1pt);
 \fill (-2,-2) circle (1pt);
 \end{tikzpicture}}
 \loigiai{Từ đồ thị ta thấy $x=-1$ là tiệm cận đứng của đồ thị hàm số đã cho và đi qua $(-2;-2)$.\\
 Vậy đồ thị trên của hàm số $y=\dfrac{x^2+2x+2}{x+1}$.}
\end{ex}

\begin{ex}%[2-D1B5-SO-13-2425]%[VN-MT-7, Lê Hải Phụng]%[2D1H5-1]
 \immini{Cho hàm số $y=\dfrac{x^2+a}{x+b}$ có đồ thị là đường cong trong hình vẽ bên. Giá trị của $T=a+b$ bằng
 \choice
 {$T=0$}
 {$T=-2$}
 {\True $T=-1$}
 {$T=2$}
 }{ \begin{tikzpicture}[scale=.5, font=\normalsize, line join=round, line cap=round,>=stealth]
 \def\xmin{-4} \def\xmax{7}
 \def\ymin{-4} \def\ymax{7}
 \draw[->] (\xmin,0)--(0,0)node[above right]{$O$}--(\xmax,0)node[below]{$x$};
 \draw[->] (0,\ymin)--(0,\ymax) node [left]{$y$};
 \clip (\xmin+0.1,\ymin+0.1) rectangle (\xmax-0.1,\ymax-0.1);
 \draw[smooth,samples=125]plot[domain=\xmin:\xmax](\x,{((\x)^2)/(\x-1)});
 \draw[smooth,samples=100]plot[domain=\xmin:\xmax](\x,{(\x)+1});
 \draw[dashed] (0,2)--(1,2);
 \path 
 (0,1) node[left]{$1$}
 (0,2) node[left]{$2$}
 (1,\ymin) node[above right]{$x=1$};
 \fill (0,0) circle (1pt);
 \fill (0,1) circle (1pt);
 \fill (0,2) circle (1pt);
 \end{tikzpicture}}
 \loigiai{Từ đồ thị ta thấy $x=1$ là tiệm cận đứng của đồ thị hàm số nên $b=-1$. Suy ra $y=\dfrac{x^2+a}{x-1}$.\\
 Hàm số đi qua $(0;0)$ nên $\dfrac{0^2+a}{0-1}=0\Leftrightarrow a=0$.\\
 Vậy $T=a+b=0+(-1)=-1$.}
\end{ex}
\Closesolutionfile{ans}

\TNTF
\Opensolutionfile{ans}[ans/ans\currfilebase-Phan-II]
\begin{ex}%[2-D1B5-SO-13-2425]%[VN-MT-7, Lê Hải Phụng]%[2D1H2-1]
Cho hàm số $y=f(x)=x^4-2x^2-5$. Các khẳng định sau là đúng hay sai?
\choiceTF
{\True Hàm số có $3$ điểm cực trị}
{Hàm số đồng biến trên $(0;+\infty)$}
{Điểm $M(0;1)$ là điểm cực đại của đồ thị hàm số $y=f(x)$}
{Hàm số $y=f(x)$ và $y=f(2x)$ có cùng điểm cực đại}
\loigiai{
 Tập xác định $\mathscr{D}=\mathbb{R}$.\\
 Đạo hàm $y'=4x^3-4x$.\\
 Xét $y'=0\Leftrightarrow 4x^3-4x=0\Leftrightarrow\hoac{&x=0\\&x=1\\&x=-1.}$\\
 Bảng biến thiên
 \begin{center}
 \begin{tikzpicture}
 \tkzTabInit[nocadre=true,lgt=1.2,espcl=2.5,deltacl=0.5]
 {$x$/0.7,$f'(x)$/0.7,$f(x)$/2}
 {$-\infty$,$-1$,$0$,$1$,$+\infty$}
 \tkzTabLine{,-,0,+,0,-,0,+,}
 \tkzTabVar{+/$+\infty$,-/$-6$,+/$-5$,-/$-6$,+/$+\infty$}
 \end{tikzpicture}
 \end{center}
\begin{itemchoice}
\itemch \textbf{Đúng}.\\
Hàm số có $3$ điểm cực trị là $x=-1$, $x=0$, $x=1$.
\itemch \textbf{Sai}.\\
Hàm số đồng biến trên $(-1;0)$ và $(1;+\infty)$.
\itemch \textbf{Sai}.\\
Điểm $M(0;-5)$ là điểm cực đại của đồ thị hàm số $y=f(x)$
\itemch \textbf{Đúng}.\\
Xét $y=f(2x)$, ta có $y'=2f'(2x)$.\\
Xét $y'=0\Leftrightarrow 2f'(2x)=0\Leftrightarrow f'(2x)=0\Leftrightarrow \hoac{&2x=-1\\&2x=0\\&2x=1}\Leftrightarrow\hoac{&x=-\dfrac{1}{2}\\&x=0\\&x=\dfrac{1}{2}.}$\\
Bảng biến thiên
\begin{center}
 \begin{tikzpicture}
 \tkzTabInit[nocadre=true,lgt=1.2,espcl=2.5,deltacl=0.5]
 {$x$/0.7,$y'$/0.7,$y$/2}
 {$-\infty$,$-\tfrac{1}{2}$,$0$,$\tfrac{1}{2}$,$+\infty$}
 \tkzTabLine{,-,0,+,0,-,0,+,}
 \tkzTabVar{+/$+\infty$,-/$f(-1)$,+/$f(0)$,-/$f(1)$,+/$+\infty$}
 \end{tikzpicture}
\end{center}
Suy ra $x=0$ là điểm cực đại của hàm số $y=f(2x)$.\\
Vậy hàm số $y=f(x)$ và $y=f(2x)$ có cùng điểm cực đại.
\end{itemchoice}
}
\end{ex}

\begin{ex}%[2-D1B5-SO-13-2425]%[VN-MT-7, Lê Hải Phụng]%[2D1H3-1]
 Cho hàm số $y=f(x)=x^3-3x+2$. Các khẳng định sau là đúng hay sai?
 \choiceTF
 {\True $\min\limits_{[0;1]} y=0$}
 {\True $\min\limits_{[0;2]} y=y(0)$}
 {$\min\limits_{[-1;0]} y+\max\limits_{[0;1]} y=4$}
 {$\min\limits_{\left[-\frac{3}{2};0 \right] } \dfrac{1}{y}=\dfrac{8}{25}$}
 \loigiai{ Tập xác định $\mathscr{D}=\mathbb{R}$.\\
 Đạo hàm $y'=3x^2-3$.\\
 Xét $y'=0\Leftrightarrow 3x^2-3=0\Leftrightarrow\hoac{&x=1\\&x=-1.}$\\
 Bảng biến thiên
 \begin{center}
 \begin{tikzpicture}
 \tkzTabInit[nocadre=true,lgt=1.2,espcl=2.5,deltacl=0.5]
 {$x$/0.7,$f'(x)$/0.7,$f(x)$/2}
 {$-\infty$,$-1$,$1$,$+\infty$}
 \tkzTabLine{,+,0,-,0,+,}
 \tkzTabVar{-/$+\infty$,+/$4$,-/$0$,+/$-\infty$}
 \end{tikzpicture}
 \end{center}
 \begin{itemchoice}
 \itemch \textbf{Đúng}.\\
 Ta xét trên $[0;1]$, ta có $y(0)=2$ và $y(1)=0$. Vậy $\min\limits_{[0;1]} y=0$.
 \itemch \textbf{Đúng}.\\
 Ta xét trên $[0;2]$, ta có $y(0)=2$, $y(1)=0$ và $y(2)=4$. Vậy $\min\limits_{[0;2]} y=0=y(0)$.
 \itemch \textbf{Sai}.\\
 Ta xét trên $[0;1]$, ta có $y(0)=2$ và $y(1)=0$. Vậy $\min\limits_{[0;1]} y=0$ và $\max\limits_{[0;1]} y=2$, khi đó tổng bằng $0+2=2$.
 \itemch \textbf{Sai}.\\
 Ta có $g(x)=\dfrac{1}{y}=\dfrac{1}{x^3-3x+2}$.\\
 Tập xác định $\mathscr{D}=\mathbb{R}\setminus\{-2;1\}$.\\
 $g'(x)=\left(\dfrac{1}{y} \right)'=\dfrac{-3x^2-3}{x^3-3x+2}$.\\
 Bảng biến thiên
 \begin{center}
 \begin{tikzpicture}
 \tkzTabInit[nocadre=true,lgt=1.2,espcl=2.5,deltacl=0.5]
 {$x$/0.7,$g'(x)$/0.7,$g(x)$/2}
 {$-\tfrac{3}{2}$,$-1$,$0$}
 \tkzTabLine{,-,0,+,}
 \tkzTabVar{+/$\dfrac{8}{25}$,-/$\dfrac{1}{4}$,+/$\dfrac{1}{2}$}
 \end{tikzpicture}
 \end{center}
 Vậy $\min\limits_{\left[-\frac{3}{2};0 \right] } \dfrac{1}{y}=\dfrac{1}{4}$ khi $x=-1$.
 \end{itemchoice}
 }
\end{ex}

\begin{ex}%[2-D1B5-SO-13-2425]%[VN-MT-7, Lê Hải Phụng]%[2D1H4-1]
 Hàm số $y = f(x)$ có bảng biến thiên như sau
 \begin{center}
 \begin{tikzpicture}
 \tkzTabInit[nocadre=true,lgt=1.2,espcl=2.5,deltacl=0.5]
 {$x$/0.7,$y'$/0.7,$y$/2}{$-\infty$,$2$,$+\infty$}
 \tkzTabLine{,+,d,+,}
 \tkzTabVar{-/$1$,+D-/$+\infty$/$-\infty$,+/$1$}
 \end{tikzpicture}
 \end{center}
 \choiceTF
 {\True Tập xác định của hàm số là $\mathscr{D}=\mathbb{R}\setminus\{2\}$}
 {Hàm số đồng biến trên $\mathbb{R}$}
 {\True Tiệm cận ngang của hàm số là $y = 1$}
 {Hàm số đạt cực đại tại $x = 2$}
 \loigiai{
 \begin{itemchoice}
 \itemch \textbf{Đúng}.\\
 Tập xác định của hàm số là $\mathscr{D}=\mathbb{R}\setminus\{2\}$.
 \itemch \textbf{Sai}.\\
 Hàm số đống biến trên $(-\infty; 2)$ và $(2; +\infty)$.
 \itemch \textbf{Đúng}.\\
 Vì $\lim\limits_{x \to -\infty} f(x) = 1$ nên tiệm cận ngang của hàm số là $y = 1$.
 \itemch \textbf{Sai}.\\
 Hàm số không có cực trị.
 \end{itemchoice}
 }
\end{ex}

\begin{ex}%[2-D1B5-SO-13-2425]%[VN-MT-7, Lê Hải Phụng]%[2D1H5-7]
 \immini{
 Cho hàm số $y = f(x)$ có đồ thị như sau. Các mệnh đề sau đúng hay sai?
 \choiceTF
 {Hàm số đồng biến trên $(-\infty; -1)$}
 {Hàm số đạt cực đại tại $x = -2$}
 {\True Giá trị nhỏ nhất của hàm số $y = f(x)$ trên $(-\infty;-1)$ là $\dfrac{3}{2}$}
 {\True Điểm cực tiểu của hàm số là $x = -2$}}{ \begin{tikzpicture}[scale=0.7, font=\footnotesize, line join=round, line cap=round, >=stealth]
 \def \a{1}
 \def \b{1}
 \def \c{1}
 \def \u{-2}
 \def \v{-2}
 \def \f{((\a)*(\x)^2+(\b)*(\x)+(\c))/((\u)*(\x)+(\v))} %%Hàm số
 \def \tcx{-(\x)/2} %Tiệm cận xiên
 \def \xo{-{\v/\u}} %Tiệm cận đứng
 \def \yo{0.5} %Tiệm cận ngang
 \def \kx{4.5} %độ rộng của đồ thị theo x
 \draw[->] (\xo-\kx,0)--(\xo+\kx,0) node[below left] {$x$};
 \draw[->] (0,\yo-\kx)--(0,\yo+\kx) node[below left] {$y$};
 \draw (0,0) node [above right] {\scriptsize $O$};
 \draw[dashed] (-1,1pt)--(-1,-1pt) + (-30:6mm) node {\scriptsize $-1$};
 \draw[dashed] (-2,1pt)--(-2,-1pt) + (-90:6mm) node {\scriptsize $-2$};
 \draw[dashed] (1pt,1.5)--(-1pt,1.5) + (0:6mm) node {\scriptsize $\dfrac{3}{2}$};
 \draw[dashed] (1pt,-0.5)--(-1pt,-0.5) + (-45:10mm) node {\scriptsize $-\dfrac{1}{2}$};
 \begin{scope}
 \clip (\xo-\kx,\yo-\kx) rectangle (\xo+\kx,\yo+\kx);
 \draw[smooth,samples=200,domain=\xo-\kx:\xo-0.1,smooth,variable=\x] plot (\x,{\f});
 \draw[smooth,samples=200,domain=\xo+0.1:\xo+\kx,smooth,variable=\x] plot (\x,{\f});
 %Tiệm cận
 \draw[thin] (\xo,\yo-\kx)--(\xo,\yo+\kx);
 \draw[smooth,samples=200,domain=\xo-\kx:\xo+\kx,variable=\x] plot (\x,{\tcx});
 \end{scope}
 %Vẽ đường dóng
 \draw[dashed, thin] (-2,0) -- (-2,1.5) -- (0,1.5);
 \fill (0,0) circle (1pt);
 \fill (-2,0) circle (1pt);
 \fill (-1,0) circle (1pt);
 \fill (0,-0.5) circle (1pt);
 \fill (0,1.5) circle (1pt);
 \fill (-2,1.5) circle (1pt); 
 \end{tikzpicture}}
 \loigiai{
 \begin{itemchoice}
 \itemch \textbf{Sai}.\\
 Hàm số đồng biến trên $(-2;-1)$, $(-1;0)$ và nghịch biến trên $(-\infty;-2)$, $(0;+\infty)$.
 \itemch \textbf{Sai}.\\
 Hàm số đạt cực tiểu tại $x = -2$.
 \itemch \textbf{Đúng}.\\
 Giá trị nhỏ nhất của hàm số $y = f(x)$ trên $(-\infty;-1)$ là $\dfrac{3}{2}$.
 \itemch \textbf{Đúng}.\\
 Điểm cực tiểu của hàm số là $x = -2$.
 \end{itemchoice}
 }
\end{ex}
\Closesolutionfile{ans}

\TNSA
\Opensolutionfile{ans}[ans/ans\currfilebase-Phan-III]
\begin{ex}%[2-D1B5-SO-13-2425]%[VN-MT-7, Lê Hải Phụng]%[2D1N1-1]
 Cho hàm số $y=x^3-3x^2+1$. Tính tổng của tất cả các giá trị cực đại và giá trị cực tiểu của hàm số trên. 
 
 \shortans{-2}
 
 \loigiai{Tập xác định của hàm số $\mathscr{D}=\mathbb{R}$.\\
 Ta có đạo hàm $y'=3x^2-6x$.\\
 Xét $y'=0\Leftrightarrow3x^2-6x=0\Leftrightarrow\hoac{&x=0\\&x=2.}$\\
 Bảng biến thiên:\\
 \begin{center}
 \begin{tikzpicture}
 \tkzTabInit[nocadre=true,lgt=1,espcl=3,deltacl=0.5]
 {$x$/0.7,$y'$/0.7,$y$/2}
 {$-\infty$,$0$,$2$,$+\infty$}
 \tkzTabLine{,+,0,-,0,+,} 
 \tkzTabVar{-/$-\infty$,+/$1$,-/$-3$,+/$+\infty$}
 \end{tikzpicture}
 \end{center}
 Suy ra giá trị cực đại và cực tiểu lần lượt là $y=1$ và $y=-3$. Khi đó $1+(-3)=-2$.}
\end{ex}

\begin{ex}%[2-D1B5-SO-13-2425]%[VN-MT-7, Lê Hải Phụng]%[2D1H4-1]
 Tiệm cận xiên của đồ thị hàm số $y=f(x)=\dfrac{3x-x^2}{2x-1}$ là đường thẳng $y=ax+b$. Tính giá trị của biểu thức $P=a^2-b$.
 
 \shortans{-1}
 
 \loigiai{Tập xác định của hàm số $\mathscr{D}=\mathbb{R}\setminus\left\lbrace \dfrac{1}{2}\right\rbrace $.\\
 Phương trình đường tiệm cận xiên có dạng $y=ax+b$.\\
 Trong đó
 \begin{itemize}
 \item $a=\lim\limits_{x\to +\infty} \dfrac{f(x)}{x}=\lim\limits_{x\to +\infty}\dfrac{3x-x^2}{2x^2-x}=-\dfrac{1}{2}$.
 \item $b=\lim\limits_{x\to +\infty} \left[f(x)-ax\right]=\lim\limits_{x\to +\infty} \left(\dfrac{3x-x^2}{2x-1}+\dfrac{1}{2}x\right)=\dfrac{5}{4}$.
 \end{itemize}
 Ta cũng có \begin{itemize}
 \item $a=\lim\limits_{x\to -\infty} \dfrac{f(x)}{x}=\lim\limits_{x\to -\infty}\dfrac{3x-x^2}{2x^2-x}=-\dfrac{1}{2}$.
 \item $b=\lim\limits_{x\to -\infty} \left[f(x)-ax\right]=\lim\limits_{x\to -\infty} \left(\dfrac{3x-x^2}{2x-1}+\dfrac{1}{2}x\right)=\dfrac{5}{4}$.
 \end{itemize}
 Và $\lim\limits_{x\to \pm\infty} \left[f(x)-(ax+b)\right]=\lim\limits_{x\to \pm\infty} \left[\dfrac{3x-x^2}{2x-1}-\left(-\dfrac{1}{2}x+\dfrac{5}{4} \right) \right]=0$.\\
 Do đó, đồ thị hàm số có tiệm cận xiên là đường thẳng $y=-\dfrac{1}{2}x+\dfrac{5}{4}$.\\
 Do đó $a=-\dfrac{1}{2}$ và $b=\dfrac{5}{4}$. Vậy $P=-1$.}
\end{ex}

\begin{ex}%[2-D1B5-SO-13-2425]%[VN-MT-7, Lê Hải Phụng]%[2D1N5-4]
 Hàm số $y=f(x)=-x^3+2x^2-x+1$ có đồ thị $(C)$ và hàm số $y=g(x)=1$ có đồ thị là $(d)$. Số giao điểm của $(C)$ và $(d)$ là
 
 \shortans{2}
 \loigiai{Ta xét phương trình hoành độ giao điểm:
 \[-x^3+2x^2-x+1=1\Leftrightarrow-x^3+2x^2-x=0\Leftrightarrow\hoac{&x=0\\&x=1.}\]
 Suy ra giao điểm của $(C)$ và $(d)$ là $(0;1)$ và $(1;1)$.\\
 Vậy số giao điểm của $(C)$ và $(d)$ là $2$.}
\end{ex}

\begin{ex}%[2-D1B5-SO-13-2425]%[VN-MT-7, Lê Hải Phụng]%[2D1V2-7]
 Giả sử doanh số (tính bằng sản phẩm) của một sản phẩm mới (trong một năm nhất định) tuân theo quy luật logistic được mô hình hóa bằng hàm số \[f(t)=\dfrac{5000}{1+5\mathrm{e}^{-t}},\, t\ge0,\]
 trong đó thời gian $t$ được tính bằng năm, kể từ khi phát hành sản phẩm mới. Khi đó đạo hàm $f'(t)$ biểu thị tốc độ bán hàng. Hỏi sau khi phát hành bao nhiêu năm thì tốc độ bán hàng là cực đại? (làm tròn đến chữ số thập phân thứ nhất)
 
 \shortans{1{,}6}
 
 \loigiai{
 Gọi $g(t)$ là hàm tốc độ bán hàng.\\
 Khi đó $g(t)=f'(t)=\dfrac{25\,000\mathrm{e}^{-t}}{(1+5\mathrm{e}^{-t})^2}$, $t\ge0$.\\
 Ta có $g'(t)=\dfrac{25\,000\mathrm{e}^{-t}(1+5\mathrm{e}^{-t})(5\mathrm{e}^{-t}-1)}{(1+5\mathrm{e}^{-t})^4}$; $g'(t)=0\Leftrightarrow t=-\ln{\dfrac{1}{5}}$.\\
 Bảng biến thiên hàm số
 \begin{center}
 \begin{tikzpicture}
 \tkzTabInit[nocadre=true,lgt=1.3,espcl=2.5,deltacl=0.6]
 {$t$ /0.6, $g'(t)$ /0.6, $g(t)$ /2.5}
 {$0$, $-\ln{\tfrac{1}{5}}$, $+\infty$}
 \tkzTabLine{,+,0,-,}
 \tkzTabVar{-/ $694{,}4$,+/$1250$,-/$0$}
 \end{tikzpicture}
 \end{center}
 Hàm số đạt cực đại tại $t=-\ln{\dfrac{1}{5}}\approx1{,}6$.\\
 Vậy sau khi phát hành $1{,}6$ năm thì tốc độ bán hàng là cực đại.
 }
\end{ex}

\begin{ex}%[2-D1B5-SO-13-2425]%[VN-MT-7, Lê Hải Phụng]%[2D1C5-8]
 Một tàu đổ bộ tiếp cận Mặt Trăng theo cách tiếp cận thẳng đứng và đốt cháy các tên lửa hãm ở độ cao $677{,}6$ km so với bề mặt của Mặt Trăng được tính (gần đúng) bởi hàm
 \[h(t) = 0{,}01t^3 - 1{,}16t^2 + 34{,}52t - 46{,}4\]
 Trong khoảng thời gian $t$ ở $50$ giây đầu $(0 \le t \le 50)$. Khoảng cách con tàu lớn nhất so với bề mặt của Mặt Trăng là bao nhiêu?
 
 \shortans{260}
 
 \loigiai{
 Hàm số $h(t) = 0{,}01t^3 - 1{,}16t^2 + 34{,}52t - 46{,}4$.\\
 + Tập xác định $\mathscr{D} = \mathbb{R}$.\\
 + Đạo hàm $h'(t) = 0{,}03t^2 - 2{,}32t + 34{,}52 = 0 \Leftrightarrow \hoac{&x = \dfrac{-10\sqrt{31}+116}{3} \quad\in (0;50)\\&x = \dfrac{10\sqrt{31}+116}{3} \quad \not\in (0;50).}$\\
 + $\heva{&h(0) = 34{,}52\\&h(50) = 29{,}6\\&h\left(\dfrac{-10\sqrt{31}+116}{3}\right) = 260} \Rightarrow \max\limits_{0\le t \le 50} = 260$.\\
 Vậy trong khoảng thời gian $t$ ở $50$ giây đầu $(0 \le t \le 50)$. Khoảng cách con tàu lớn nhất so với bề mặt của Mặt Trăng là $260$ km.
 }
\end{ex}

\begin{ex}%[2-D1B5-SO-13-2425]%[VN-MT-7, Lê Hải Phụng]%[2D1V3-6]
 \immini{
 Sự phân huỷ của rác thải hữu cơ có trong nước sẽ làm tiêu hao oxygen hoà tan trong nước. Nồng độ oxygen (mg/l) trong một hồ nước sau $t$ giờ $(t \geq 0)$ khi một lượng rác thải hữu cơ bị xả vào hồ được xấp xỉ bởi hàm số (có đồ thị như đường cong ở hình bên)
 \[
 y(t)=5-\dfrac{15t}{9t^2+1}.
 \]
 }{
 \begin{tikzpicture}[>=stealth,x=1cm,y=0.3cm,scale=2,font=\footnotesize]
 \draw[->] (-0.5,0) -- (4,0) node[below] {$t$};
 \draw[->] (0,-1) -- (0,6) node[left] {$y$};
 \filldraw (0,0) circle (1pt)node[below left]{$O$};
 \draw[domain=0:4,samples=200,red] plot (\x,{5-(15*(\x))/(9*(\x)^2+1)});
 \draw[dashed] (0,5) node [left] {$5$}--(4,5);
 \foreach \x/\g in {1/-90,2/-90,3/-90}
 \draw[thin] (\x,2pt)--(\x,-2pt) + (\g:3mm) node {$\x$};
 \end{tikzpicture}
 }
%  \noindent
%  (Theo: https://www.researchgate.net/publication/264903978$\_$Microrespirometric$\_$ characterization$\_$\\of$\_$activated$\_$sludge$\_$inhibition$\_$by$\_$copper$\_$and$\_$zinc)\\
 Trong đó, đạo hàm $y'(t)$ biểu thị tốc độ thay đổi nồng độ oxigen trong nước. Tốc độ thay đổi nồng độ oxigen lớn nhất khi $t=\dfrac{\sqrt{a}}{b}$ giờ. Tính giá trị của $a-b$ biết $a$ và $b$ là các số nguyên tố.
 
 \shortans{0}
 
 \loigiai{
 Ta có $y'(t)=\dfrac{135t^2-15}{(9t^2+1)^2}$.\\
 Suy ra $y''(t)=\dfrac{-2430t^3+810t}{(9t^2+1)^3}$.\\
 Cho $y''(t)=0\Leftrightarrow -2430t^3+810t=0\Leftrightarrow \hoac{&t=\pm\dfrac{\sqrt{3}}{3}\\&t=0.}$
 \begin{center}
 \begin{tikzpicture}[>=stealth]
 \tkzTabInit[nocadre=true,lgt=1.5,espcl=3,deltacl=0.6]{$t$/1 ,$y''(t)$/.6,$y'(t)$/2}
 {$0$ , $\tfrac{\sqrt{3}}{3}$ , $+\infty$}
 \tkzTabLine{ $0$, + , $0$ , - , }
 \tkzTabVar{-/$-15$ , +/$\dfrac{15}{8}$ , -/$0$}
 \end{tikzpicture}
 \end{center}
 Từ bảng biến thiên ta có $\max\limits_{t\in[0;+\infty)}y'(t)=y'\left(\dfrac{\sqrt{3}}{3}\right)=\dfrac{15}{8}$. \\
 Vậy tốc độ thay đổi nồng độ oxigen lớn nhất khi $t=\dfrac{\sqrt{3}}{3}$ giờ.\\
 Vậy $a=b=3$. Khi đó $a-b=0$.
 }
\end{ex}
\Closesolutionfile{ans}
% \begin{indapan}
% 	{ans/ans\currfilebase}
% \end{indapan}


% \begin{name}
 {Biên soạn: Đỗ Minh Phúc \\ Phản biện: Đoàn Thị Lý}
 {Đề ôn tập chương I}
\end{name}

\caulc
\Opensolutionfile{ans}[ans/ans\currfilebase-Phan-I]
\begin{ex}%[2-D1B5-SO-14-2425]%[VN-MT-7, Đỗ Minh Phúc]%[2D1H1-1]
 Cho hàm số $y=f(x)$ có đạo hàm $f'(x)=-x^2-4$, $\forall x \in \mathbb{R}$. Mệnh đề nào dưới đây đúng?
\choice
{Hàm số đồng biến trên khoảng $(2;+\infty)$}
{Hàm số đồng biến trên khoảng $(-2;2)$}
{\True Hàm số nghịch biến trên khoảng $(-\infty;+\infty)$}
{Hàm số đồng biến trên khoảng $(-\infty;-2)$}
\loigiai{
Do hàm số $y=f(x)$ có đạo hàm $f'(x)=-x^2-4<0$, $\forall x \in \mathbb{R}$ nên hàm số nghịch biến trên khoảng $(-\infty;+\infty)$.
}
\end{ex}

\begin{ex}%[2-D1B5-SO-14-2425]%[VN-MT-7, Đỗ Minh Phúc]%[2D1N2-2]
 Cho hàm số $y=f(x)$ có bảng biến thiên như sau.
 \begin{center}
 \begin{tikzpicture}
 \tkzTabInit[nocadre,lgt=1.2,espcl=2.5,deltacl=0.6]
 {$x$/0.6,$f'(x)$/0.6,$f(x)$/2}
 {$-\infty$,$-1$,$2$,$+\infty$}
 \tkzTabLine{,+,0,-,0,+,}
 \tkzTabVar{-/$-\infty$,+/$1$,-/$-2$,+/$+\infty$}
 \end{tikzpicture}
 \end{center}
 Giá trị cực tiểu của hàm số đã cho bằng
 \choice
 {$-1$}
 {$2$}
 {\True $-2$}
 {$1$}
\loigiai{
 Dựa vào bảng biến thiên, ta có giá trị cực tiểu của hàm số đã cho bằng $-2$.}
\end{ex}

\begin{ex}%[2-D1B5-SO-14-2425]%[VN-MT-7, Đỗ Minh Phúc]%[2D1N3-2]
 Cho hàm số $y=f(x)$ liên tục trên $\mathbb{R}$ và có bảng biến thiên như sau.
 \begin{center}
 \begin{tikzpicture}
 \tkzTabInit[nocadre,lgt=1.2,espcl=2.5,deltacl=0.6]
 {$x$/0.6,$y'$/0.6,$y$/2}
 {$-\infty$,$0$,$2$,$+\infty$}
 \tkzTabLine{,+,0,-,0,+,}
 \tkzTabVar{-/$2$,+/$4$,-/$-5$,+/$2$}
 \end{tikzpicture}
\end{center}
 Tổng giá trị lớn nhất và giá trị nhỏ nhất của hàm số $y=f(x)$ trên $\mathbb{R}$ bằng
 \choice
 {$6$}
 {$9$}
 {$-3$}
 {\True $-1$}
 \loigiai{
 Trên $\mathbb{R}$, ta có giá trị lớn nhất của hàm số $y=f(x)$ bằng $4$ tại $x=0$ và giá trị nhỏ nhất bằng $-5$ tại $x=2$.\\
 Khi đó tổng giá trị lớn nhất và giá trị nhỏ nhất của hàm số $y=f(x)$ trên $\mathbb{R}$ bằng $-1$.
}
\end{ex}

\begin{ex}%[2-D1B5-SO-14-2425]%[VN-MT-7, Đỗ Minh Phúc]%[2D1N3-1]
 \immini[thm]{Cho hàm số $y=f(x)$ có đồ thị như hình vẽ. Giá trị lớn nhất của hàm số trên đoạn $[0;3]$ bằng
 \choice
 {\True $4$}
 {$2$}
 {$3$}
 {$0$}}{\begin{tikzpicture}[scale=0.6, font=\footnotesize, line join=round, line cap=round, >=stealth]
 \def\xmin{-2}\def\xmax{4}\def\ymin{-1}\def\ymax{5}
 \draw[->] (\xmin-0.2,0)--(\xmax+0.2,0) node[below] {$x$};
 \draw[->] (0,\ymin-0.2)--(0,\ymax+0.2) node[right] {$y$};
 \draw (0,0) node [below left] {$O$};
 \foreach \x in {1,2}
 \fill (\x,0)circle (1pt) node [below] {$\x$};
 \foreach \x in {3}
 \fill (\x,0)circle (1pt) node [above right] {$\x$};
 \foreach \y in {2,4}
 \fill (0,\y)circle (1pt) node [left] {$\y$};
 \clip (\xmin,\ymin) rectangle (\xmax,\ymax);
 \draw[smooth,samples=200,domain=\xmin:\xmax] plot (\x,{-1*((\x)^3)+3*((\x)^2)+0*(\x)+0});
 \draw[dashed] (1,0)--(1,2)--(0,2);\fill (1,2) circle (1pt);
 \draw[dashed] (0,0)--(0,0)--(0,0);\fill (0,0) circle (1pt);
 \draw[dashed] (2,0)--(2,4)--(0,4);\fill (2,4) circle (1pt);
\end{tikzpicture}}
 \loigiai{
 Từ đồ thị hàm số $f(x)$ ta có $\max_{[0;3]}f(x)=4$ tại $x=2$.
 }
\end{ex}

\begin{ex}%[2-D1B5-SO-14-2425]%[VN-MT-7, Đỗ Minh Phúc]%[2D1H4-1]
 Đường tiệm cận ngang của đồ thi hàm số $y=\dfrac{2024x+2025}{x-5}$ là
 \choice
 {$y=2025$}
 {\True $y=2024$}
 {$y=1$}
 {$y=-5$}
 \loigiai{
 Ta có $\lim\limits_{x \to +\infty} \dfrac{2024x+2025}{x-5}=2024$ và $\lim\limits_{x \to-\infty} \dfrac{2024x+2025}{x-5}=2024$ nên đồ thị hàm số có đường tiệm cận ngang là $y=2024$. 
 }
\end{ex}

\begin{ex}%[2-D1B5-SO-14-2425]%[VN-MT-7, Đỗ Minh Phúc]%[2D1N4-1]
 Đường tiệm cận đứng của đồ thị hàm số $y=\dfrac{15x-6}{10x+5}$ là
 \choice
 {$x=\dfrac{3}{2}$}
 {$x=-\dfrac{6}{5}$}
 {\True $x=-\dfrac{1}{2}$}
 {$x=\dfrac{2}{5}$}
 \loigiai{
 Điều kiện xác định $x \neq-\dfrac{1}{2}$.\\
 Ta có $\lim\limits_{x \to\left(-\tfrac{1}{2}\right)^+} \dfrac{15x-6}{10x+5}=-\infty$ và $\lim\limits_{x \to\left(-\tfrac{1}{2}\right)^-} \dfrac{15x-6}{10x+5}=+\infty$ nên đồ thị hàm số có đường tiệm cận đứng là $x=-\dfrac{1}{2}$. 
 }
\end{ex}

\begin{ex}%[2-D1B5-SO-14-2425]%[VN-MT-7, Đỗ Minh Phúc]%[2D1H4-1]
 Tiệm cận xiên của đồ thị hàm số $y=\dfrac{-x^2-3x+4}{x}$ là đường thẳng có phương trình nào sau đây?
 \choice
 {\True $y=-x-1$}
 {$y=x-1$}
 {$y=-x+1$}
 {$y=x+1$}
 \loigiai{
 Ta có $a=\lim\limits_{x \to+\infty}\left(\dfrac{-x^2-3x+4}{x+2}:x\right)=\lim\limits_{x \to+\infty} \dfrac{-x^2-3x+4}{x^2+2x}=-1$.\\
 Lại có $b=\lim\limits_{x \to+\infty}\left[\dfrac{-x^2-3x+4}{x+2}-(-1)x\right]=\lim\limits_{x \to+\infty} \dfrac{-x+4}{x+2}=-1$.\\
 (Tương tự, $\lim\limits_{x \to-\infty}\left(\dfrac{-x^2-3x+4}{x+2}:x\right)=-1$, $\lim\limits_{x \to-\infty}\left[\dfrac{-x^2-3x+4}{x+2}-(-1)x\right]=-1$).\\
 Tiệm cận xiên của đồ thị hàm số $y=\dfrac{-x^2-3x+4}{x+2}$ là đường thẳng có phương trình $y=-x-1$.
 }
\end{ex}

\begin{ex}%[2-D1B5-SO-14-2425]%[VN-MT-7, Đỗ Minh Phúc]%[2D1H5-1]
 \immini[thm]{Đường cong ở hình sau là đồ thi của hàm số nào?
 \choice
 {\True $y=-x^3+3x^2-4$}
 {$y=x^3-4$}
 {$y=x^2-4$}
 {$y=-x^2-4$}}{\begin{tikzpicture}[scale=0.6, font=\footnotesize, line join=round, line cap=round, >=stealth]
 \def\xmin{-2}\def\xmax{4}\def\ymin{-4.5}\def\ymax{1}
 \draw[->] (\xmin-0.2,0)--(\xmax+0.2,0) node[below] {$x$};
 \draw[->] (0,\ymin-0.2)--(0,\ymax+0.2) node[right] {$y$};
 \draw (0,0) node [below left] {$O$};
 \foreach \x in {-1}
 \fill (\x,0)circle (1pt) node [below left] {$\x$};
 \foreach \x in {2}
 \fill (\x,0)circle (1pt) node [below] {$\x$};
 \foreach \y in {-4}
 \fill (0,\y)circle (1pt) node [below left] {$\y$};
 \clip (\xmin,\ymin) rectangle (\xmax,\ymax);
 \draw[smooth,samples=200,domain=\xmin:\xmax] plot (\x,{-1*((\x)^3)+3*((\x)^2)+0*(\x)+-4});
 \fill (0,0)circle (1pt);
\end{tikzpicture}}
 \loigiai{
 Xét dáng hình của đồ thị, ta loại được hàm số $y=x^2-4$ và $y=-x^2-4$.\\
 Do $\lim\limits_{x \to+\infty} y=-\infty$ nên ta loại hàm số $y=x^3-4$ và nhận hàm số $y=-x^3+3x^2-4$. 
 }
\end{ex}

\begin{ex}%[2-D1B5-SO-14-2425]%[VN-MT-7, Đỗ Minh Phúc]%[2D1H5-1]
 Hàm số nào sau đây có bảng biến thiên như hình bên dưới?
 \begin{center}
 \begin{tikzpicture}
 \tkzTabInit[nocadre,lgt=1.2,espcl=2.5,deltacl=0.6]
 {$x$/0.6,$y'$/0.6,$y$/2}
 {$-\infty$,$2$,$+\infty$}
 \tkzTabLine{,-,d,-,}
 \tkzTabVar{+/$2$,-D+/$-\infty$/$+\infty$,-/$2$}
 \end{tikzpicture}
 \end{center}
 \choice
 {\True $y=\dfrac{2x+1}{x-2}$}
 {$y=\dfrac{2x-5}{x-2}$}
 {$y=\dfrac{2x+1}{x+2}$}
 {$y=\dfrac{2x-1}{x+2}$}
 \loigiai{
 Từ bảng biến thiên, ta nhận thấy đồ thị hàm số có tiệm cận đứng là $x=2$ là nên loại hàm số $y=\dfrac{2x+1}{x+2}$ và $y=\dfrac{2x-1}{x+2}$.\\
 Ta nhận thấy hàm số nghịch biến trên từng khoảng xác định nên loại hàm số $y=\dfrac{2x-5}{x-2}$ và nhận hàm số $y=\dfrac{2x+1}{x-2}$.
 }
\end{ex}

\begin{ex}%[2-D1B5-SO-14-2425]%[VN-MT-7, Đỗ Minh Phúc]%[2D1H5-1]
 \immini[thm]{Đường cong ở hình bên là đồ thị của hàm số nào sau đây?
 \choice
 {$y=-x^3+x^2-2x+1$}
 {$y=\dfrac{x^2-x+3}{x-1}$}
 {\True $y=\dfrac{x^2-3x+6}{x-1}$}
 {$y=\dfrac{2x+3}{x-1}$}}{\begin{tikzpicture}[scale=0.5, font=\footnotesize, line join=round, line cap=round, >=stealth]
 \def\xmin{-3}\def\xmax{5}\def\ymin{-8}\def\ymax{6}
 \draw[->] (\xmin-0.2,0)--(\xmax+0.2,0) node[below] {$x$};
 \draw[->] (0,\ymin-0.2)--(0,\ymax+0.2) node[right] {$y$};
 \draw (0,0) node [below left] {$O$};
 \foreach \x in {-2,2,3,4}
 \fill (\x,0)circle (1pt) node [below] {$\x$};
 \foreach \y in {-6,-4,-2,2,4,6}
 \fill (0,\y)circle (1pt) node [left] {$\y$};
 \clip (\xmin,\ymin) rectangle (\xmax,\ymax);
 \draw (1,\ymin)--(1,\ymax);
 \draw[domain=\xmin:\xmax] plot (\x,{1*(\x)+-2});
 \draw[smooth,samples=200,domain=\xmin:0.9] plot (\x,{(1*((\x)^2)+-3*(\x)+6)/(1*(\x)+-1)});
 \draw[smooth,samples=200,domain=1.1:\xmax] plot (\x,{(1*((\x)^2)+-3*(\x)+6)/(1*(\x)+-1)});
 \draw[dashed] (-1,0)--(-1,-5)--(0,-5);\fill (-1,-5) circle (1pt);
 \draw[dashed] (3,0)--(3,3)--(0,3);\fill (3,3) circle (1pt);
 \fill (0,0)circle (1pt);
\end{tikzpicture}}
 \loigiai{
 \begin{itemize}
 \item Xét hàm số $y=-x^3+x^2-2x+1$. Vì đồ thị hàm số $y=-x^3+x^2-2x+1$ không có đường tiệm cận. Suy ra phương án $y=-x^3+x^2-2x+1$ sai.
 \item Xét hàm số $y=\dfrac{x^2-x+3}{x-1}=x+\dfrac{3}{x-1}$.\\
 Ta có
 $\lim\limits_{x \to+\infty}[y-x]=\lim\limits_{x \to+\infty} \dfrac{3}{x-1}=0$ và $\lim\limits_{x \to-\infty}[y-x]=\lim\limits_{x \to-\infty} \dfrac{3}{x-1}=0$.\\
 Do đó đường thẳng $y=x$ là đường tiệm cận xiên của đồ thị hàm số. Suy ra phương án $y=\dfrac{x^2-x+3}{x-1}$ sai.
 \item Xét hàm số $y=\dfrac{x^2-3x+6}{x-1}=x-2+\dfrac{4}{x-1}$.\\
 Ta có $\lim\limits_{x \to 1^-} y=-\infty$ và $\lim\limits_{x \to 1^+} y=+\infty$.\\
 Do đó đường thẳng $x=1$ là đường tiệm cận đứng của đồ thị hàm số.\\
 Lại có $\lim\limits_{x \to+\infty}[y-(x-2)]=\lim\limits_{x \to+\infty} \dfrac{4}{x-1}=0$ và $\lim\limits_{x \to-\infty}[y-(x-2)]=\lim\limits_{x \to-\infty} \dfrac{4}{x-1}=0$.\\
 Do đó đường thẳng $y=x-2$ là đường tiệm cận xiên của đồ thị hàm số.\\
 Hơn nữa, đồ thị hàm số cắt trục tung tại điểm có tung độ bằng $-6$ nên suy ra phương án $y=\dfrac{x^2-3x+6}{x-1}$ đúng.
 \item Xét hàm số $y=\dfrac{2x+3}{x-1}$. Vì đồ thị hàm số $y=\dfrac{2x+3}{x-1}$ không có đường tiệm cận xiên nên phương án $y=\dfrac{2x+3}{x-1}$ sai.
 \end{itemize}
 }
\end{ex}

\begin{ex}%[2-D1B5-SO-14-2425]%[VN-MT-7, Đỗ Minh Phúc]%[2D1V3-6]
 Khi nuôi cá thí nghiệm trong hồ, một nhà khoa học đã nhận thấy rằng: nếu trên mỗi đơn vị diện tích của mặt hồ có $n$ con cá thì trung bình mỗi con cá sau một vụ cân nặng là $P(n)=800-20n$ (g). Hỏi phải thả bao nhiêu con cá trên một đơn vị diện tích của mặt hồ để sau một vụ thu hoạch được nhiều cá nhất?
 \choice
 {$19$}
 {\True $20$}
 {$21$}
 {$22$}
 \loigiai{
 Gọi $F(n)$ là hàm cân nặng của $n$ con cá sau vụ thu hoạch trên một đơn vị diện tích.\\
 Ta có $F(n)=(800-20n) \cdot n=800n-20n^2$.\\
 Để sau một vụ thu hoạch được nhiều cá nhất thì cân nặng của $n$ con cá trên một đơn vị điện tích của mặt hồ là lớn nhất.\\
 Bài toán trở thành tìm $n\in \mathbb{N}^*$ sao cho $F(n)$ đạt giá trị lớn nhất.\\
 Ta có $F'(n)=800-40n$.\\
 Cho $F'(n)=0 \Leftrightarrow 800-40n=0 \Leftrightarrow n=20$.\\
 Ta có bảng biến thiên
 \begin{center}
 \begin{tikzpicture}
 \tkzTabInit[nocadre,lgt=1.2,espcl=2.5,deltacl=0.6]
 {$n$/0.6,$F'(n)$/0.6,$F(n)$/2}{$-\infty$,$20$,$+\infty$}
 \tkzTabLine{,+,0,-,}
 \tkzTabVar{-/$-\infty$,+/$8\,000$,-/$-\infty$}
 \end{tikzpicture}
 \end{center}
 Vậy phải thả $20$ con cá trên một đơn vị diện tích của mặt hồ để sau một vụ thu hoạch được nhiều cá nhất.
 }
\end{ex}

\begin{ex}%[2-D1B5-SO-14-2425]%[VN-MT-7, Đỗ Minh Phúc]%[2D1H2-1]
 Hàm số $f(x)=x^3-3x^2-9x+1$ đạt cực đại tại điểm
 \choice
 {\True $x=-1$}
 {$x=1$}
 {$x=3$}
 {$x=-3$}
 \loigiai{
 Ta có $f'(x)=3x^2-6x-9$.\\
 Cho $f'(x)=0\Leftrightarrow 3x^2-6x-9=0\Leftrightarrow\hoac{&x=-1\\&x=3.}$\\
 Ta có bảng biến thiên
 \begin{center}
 \begin{tikzpicture}
 \tkzTabInit[nocadre,lgt=1.2,espcl=2.5,deltacl=0.6]
 {$x$/0.6,$y'$/0.6,$y$/2}
 {$-\infty$,$-1$,$3$,$+\infty$}
 \tkzTabLine{,+,0,-,0,+,}
 \tkzTabVar{-/$-\infty$,+/$6$,-/$-26$,+/$+\infty$}
 \end{tikzpicture}
 \end{center}
 Dựa vào bảng biến thiên, ta thấy hàm số đạt cực đại tại $x=-1$. 
 }
\end{ex}
\Closesolutionfile{ans}

\cauds
\Opensolutionfile{ans}[ans/ans\currfilebase-Phan-II]
\begin{ex}%[2-D1B5-SO-14-2425]%[VN-MT-7, Đỗ Minh Phúc]%[2D1H5-3]
 Cho hàm số $y=2x^{3}+x^{2}-\dfrac{1}{2}x-3$ có đồ thị $(C)$.
 \choiceTF
 {\True Hàm số xác định trên $\mathbb{R}$}
 {Hàm số đồng biến trên $(-\infty;+\infty)$}
 {Hàm số không có cực trị}
 {\True Đồ thị hàm số cắt đường thẳng $y=m$ tại $3$ điểm khi và chỉ khi $-\dfrac{329}{108}<m<-\dfrac{11}{4}$}
 \loigiai{
 Ta có $y'=6x^{2}+2x-\dfrac{1}{2}$.\\
 Cho $y'=0 \Leftrightarrow 6x^{2}+2x-\dfrac{1}{2}=0 \Leftrightarrow \hoac{&x=-\dfrac{1}{2}\\&x=\dfrac{1}{6}.}$\\
 Bảng biến thiên
 \begin{center}
 \begin{tikzpicture}[>=stealth]
 \tkzTabInit[nocadre,lgt=1,espcl=2,deltacl=0.5]{$x$/1,$y'$/.7,$y$/2}
 {$-\infty$,$-\dfrac{1}{2}$,$\dfrac{1}{6}$, $+\infty$}
 \tkzTabLine{,+,$0$,-,$0$,+,}
 \tkzTabVar{-/$-\infty$,+/$-\dfrac{11}{4}$,-/$-\dfrac{329}{108}$,+/$+\infty$}
 \end{tikzpicture}
 \end{center}
 \begin{itemchoice}
 \itemch {\bf Đúng}.\\
 Tập xác định $\mathbb{R}$.
 \itemch {\bf Sai}.
 \begin{itemize}
 \item Hàm số đồng biến trên $\left(-\infty; -\dfrac{1}{2}\right)$ và $\left(\dfrac{1}{6}; +\infty\right)$.
 \item Hàm số nghịch biến trên $ \left(-\dfrac{1}{2}; \dfrac{1}{6}\right)$.
 \end{itemize}
 \itemch {\bf Sai}.\\
 Hàm số đạt cực đại tại $x_{\text{CĐ}}=-\dfrac{1}{2}$, $y_{\text{CĐ}}=-\dfrac{11}{4}$; hàm số đạt cực tiểu tại $x_{\text{CT}}=\dfrac{1}{6}$, $y_{\mathrm{CT}}=-\dfrac{329}{108}$.
 \itemch {\bf Đúng}.\\
 Dựa vào bảng biến thiên, đồ thị hàm số cắt đường thẳng $y=m$ tại $3$ điểm khi và chỉ khi $-\dfrac{329}{108}<m<-\dfrac{11}{4}$.
 \end{itemchoice} 
 }
\end{ex}

\begin{ex}%[2-D1B5-SO-14-2425]%[VN-MT-7, Đỗ Minh Phúc]%[2D1V5-3]
 \immini[thm]{Cho hàm số $y=f(x)$ xác định trên $\mathbb{R}$. Đồ thị hàm số $y=f'(x)$ cắt trục hoành tại $3$ điểm phân biệt $a$, $b$, $c$ ($a<b<c)$ như hình bên.
 \choiceTF
 {Hàm số $y=f(x)$ đồng biến trên $(-\infty;a)$}
 {Hàm số có $2$ điểm cực trị}
 {\True Giá trị cực đại của hàm số là $f(b)$}
 {\True Biết $ f(b) < 0$. Đồ thị hàm số $ y = f (x)$ cắt trục hoành tại hai điểm phân biệt}
 }
 {\begin{tikzpicture}[scale=0.7, font=\footnotesize, line join=round, line cap=round, >=stealth]
 \draw[->,black] (-2.5,0) -- (3,0)node[above left] {$x$};
 \draw[->,black] (0,-2.5) -- (0,3.1)node[below right] {$y$};
 \node at (-1.62,0) [above left] {$a$};
 \node at (0.41,0) [above] {$b$};
 \node at (1.82,0) [above] {$c$};
 \node at (0,0) [below left] {$O$};
 \draw[smooth,samples=100,domain=-2.1:2.5] plot(\x,{0.6*(\x)^3-0.4*(\x)^2-1.92*(\x)+0.8});
 \fill %vẽ các điểm rỗng ruột
 (-1.68,0) circle (1pt)
 (0.41,0) circle (1pt)
 (1.94,0) circle (1pt)
 (0,0) circle (1pt);
 \end{tikzpicture}}
 \loigiai{
 Ta có $f'(x)=0\Leftrightarrow\hoac{&x=a\\&x=b\\&x=c.} $
 \begin{center}
 \begin{tikzpicture}
 \tkzTabInit[nocadre,lgt=1.5,espcl=2.5,deltacl=0.6]
 {$x$/1,$y'$/1,$y$/2}
 {$-\infty$,$a$,$b$,$c$,$+\infty$}
 \tkzTabLine{,-,$0$,+,$0$,-,$0$,+}
 \tkzTabVar{+/ $+\infty$ ,-/$f\left(a\right)$,+/$f\left(b\right)$,-/$f\left(c\right)$,+/ $+\infty$}
 \end{tikzpicture}
 \end{center}
 \begin{itemchoice}
 \itemch {\bf Sai}.\\
 Theo bảng biến thiên, hàm số $y=f(x)$ nghịch biến trên $(-\infty;a)$.
 \itemch {\bf Sai}.\\
 Theo bảng biến thiên, hàm số $y=f(x)$ có $3$ điểm cực trị.
 \itemch {\bf Đúng}.\\
 Theo bảng biến thiên, giá trị cực đại của hàm số là $f(b)$.
 \itemch {\bf Đúng}.\\
 Do $ f(b)<0$ nên đồ thị hàm số cắt trục hoành tại hai điểm phân biệt.
 \end{itemchoice}
 }
\end{ex}

\begin{ex}%%[2-D1B5-SO-14-2425]%[VN-MT-7, Đỗ Minh Phúc]%[2D1H5-4]
 \immini[thm]{Cho hàm số $y=f(x)=\dfrac{ax+b}{cx-1}$ có đồ thị như hình vẽ bên. Các khẳng định sau là đúng hay sai?
 \choiceTF
 {$b=-2$}
 {\True $a+b+c=2$}
 {\True Phương trình $f(x)=1$ có duy nhất một nghiệm}
 {\True Đồ thị hàm số nhận điểm $I(1;-1)$ là tâm đối xứng}
 }
 {\begin{tikzpicture}[scale=0.7,>=stealth, font=\footnotesize, line join=round, line cap=round]
 \def\a{-1} \def\b{2} \def\c{1} \def\d{-1} % Hệ số
 \def\xmin{-3} \def\xmax{5}
 \def\ymin{-4} \def\ymax{4}
 \draw[->] (\xmin,0)--(\xmax,0) node [below]{$x$};
 \draw[->] (0,\ymin)--(0,\ymax) node [left]{$y$};
 \node at (0,0) [above left]{$O$};
 \clip (\xmin+0.1,\ymin+0.1) rectangle (\xmax-0.1,\ymax-0.1);
 \draw[smooth,samples=300,domain=\xmin:(-\d/\c-0.1)] plot(\x,{(\a*(\x)+\b)/(\c*(\x)+\d)});
 \draw[smooth,samples=300,domain=(-\d/\c+0.1:\xmax)] plot(\x,{(\a*(\x)+\b)/(\c*(\x)+\d)});
 \draw (-\d/\c,\ymin)--(-\d/\c,\ymax);
 \draw (\xmin,\a/\c)--(\xmax,\a/\c);
 \foreach \d/\g in{1/135,2/60}
 \draw[fill=black](\d,0)circle(1pt)node[shift={(\g:0.35)}]{$\d$};
 \foreach \d/\g in{-2/180,-1/135}
 \draw[fill=black](0,\d)circle(1pt)node[shift={(\g:0.35)}]{$\d$};
 \fill (0,0) circle (1pt);
 \end{tikzpicture}}
 \loigiai{
 \begin{itemchoice}
 \itemch {\bf Sai}.\\
 Vì điểm $(0;-2)$ thuộc đồ thị hàm số $y=f(x)$ nên ta có $\dfrac{b}{-1}=-2\Leftrightarrow b=2$.
 \itemch {\bf Đúng}.\\
 Vì điểm $(0;-2)$ thuộc đồ thị hàm số $y=f(x)$ nên ta có $\dfrac{b}{-1}=-2\Leftrightarrow b=2$.\\
 Đồ thị hàm số $y=f(x)$ có tiệm cận ngang $y=\dfrac{a}{c}$ và tiệm cận đứng $x=\dfrac{1}{c}$, do đó
 \[\heva{&\dfrac{a}{c}=-1\\&\dfrac{1}{c}=1}\Leftrightarrow \heva{&a=-1\\&c=1.}\]
 Vậy $a+b+c=2$.
 \itemch {\bf Đúng}.\\
 Vẽ đường thẳng $y=1$ trên mặt phẳng tọa độ, ta thấy đường thẳng $y=1$ cắt đồ thị hàm số $y=f(x)$ tại duy nhất một điểm.
 \itemch {\bf Đúng}.\\
 Đồ thị hàm số nhận đường thẳng $y=-1$ làm tiệm cận ngang và $x=1$ là tiệm cận đứng, do đó điểm $(1;-1)$ là tâm đối xứng của đồ thị.
 \end{itemchoice}
 }
\end{ex}

\begin{ex}%[2-D1B5-SO-14-2425]%[VN-MT-7, Đỗ Minh Phúc]%[2D1H4-1]
 Cho hàm số $y=f(x)=\dfrac{x^2+mx-1}{x-1}$.
 \choiceTF
 {Hàm số có cực trị khi và chỉ khi $m\geq 0$}
 {\True Tiệm cận xiên của đồ thị hàm số là đường thẳng $y=x+m+1$}
 {Với $m=1$, hàm số nghịch biến trên khoảng $(0;2)$}
 {Tổng các giá trị nguyên dương của tham số $m$ để hàm số đồng biến trên khoảng $(3;5)$ bằng $6$}
 \loigiai{\begin{itemchoice}
 \itemch {\bf Sai}.\\
 Có $y'=\dfrac{x^2-2x-m+1}{(x-1)^2}$.\\
 Hàm số có hai cực trị khi và chỉ khi phương trình $x^2-2x-m+1=0 $ có hai nghiệm phân biệt khác $1\Leftrightarrow \heva{&\Delta' >0\\&1^2-2\cdot 1-m+1\ne 0} \Leftrightarrow \heva{&m>0\\&m \ne 0} \Leftrightarrow m>0$. 
 \itemch {\bf Đúng}.\\
 Ta có $y=x+m+1+\dfrac{m}{x-1}$.\\
 $\lim\limits_{x\to \pm \infty}\left[y-(x+m+1)\right]=\lim\limits_{x\to\pm \infty}\dfrac{m}{x-1}=0$.\\
 Vậy đồ thị hàm số có tiệm cận xiên $y=x+m+1$.
 \itemch {\bf Sai}.\\
 Với $m=1$, hàm số trở thành $y=\dfrac{x^2+x-1}{x-1}$ không xác định trên khoảng $(0;2)$ nên không nghịch biến trên khoảng $(0;2)$.
 \itemch {\bf Sai}.\\
 Hàm số đồng biến trên khoảng $(3;5)$
 \begin{eqnarray*}
 \Leftrightarrow x^2-2x-m+1\geq 0,\ \forall x\in (3;5)
 &\Leftrightarrow& m\leq x^2-2x+1,\ \forall x\in (3;5)\\
 &\Leftrightarrow& m\leq \min\limits_{[3;5]} \left(x^2-2x+1\right).
 \end{eqnarray*}
 Xét hàm $g(x)=x^2-2x+1 $ có bảng biến thiên
 \begin{center}
 \begin{tikzpicture}
 \tkzTabInit[nocadre,lgt=1.2,espcl=2.5,deltacl=0.6]
 {$x$/.7 ,$g'(x)$/.7,$g(x)$/2}
 {$-\infty$,$1$,$+\infty$}
 \tkzTabLine{,-,0,+,}
 \tkzTabVar{+/,-/$g(1)$,+/ /}
 \end{tikzpicture}
 \end{center}
 Từ bảng biến thiên suy ra $m\leq g(3) \Leftrightarrow m\leq 4$.\\
 Vì $m$ nguyên dương nên $m \in \{1;2;3;4\}$.\\
 Vậy tổng các giá trị $m$ thỏa mãn yêu cầu đề bài bằng $10$.
 \end{itemchoice}}
\end{ex}
\Closesolutionfile{ans}

\caukq
\Opensolutionfile{ans}[ans/ans\currfilebase-Phan-III]
\begin{ex}%[2-D1B5-SO-14-2425]%[VN-MT-7, Đỗ Minh Phúc]%[2D1V1-3]
 Có bao nhiêu giá trị nguyên của tham số $m$ để hàm số $y=(m-1)x^3-(m-1)x^2+3x+2024$ đồng biến trên tập xác định?
\shortans{10}
\loigiai{
Tập xác định $\mathscr{D}=\mathbb{R}$.\\
Ta có $y'=3(m-1)x^2-2(m-1)x+3$.\\
Hàm số đồng biến trên $\mathbb{R}$ khi $y' \geq 0$, $\forall x \in \mathbb{R} \Leftrightarrow 3(m-1)x^2-2(m-1)x+3 \geq 0$, $\forall x \in \mathbb{R}$.
\begin{itemize}
 \item Nếu $m-1=0 \Leftrightarrow m=1$. Khi đó $y' \geq 0 \Leftrightarrow 3 \geq 0$ luôn đúng $\forall x \in \mathbb{R}$.\\
 Suy ra $m=1$ thoả mãn yêu cầu bài toán.
 \item Nếu $m-1 \neq 0 \Leftrightarrow m \neq 1$.\\
 Khi đó 
 \begin{eqnarray*}
 3(m-1) x^2-2(m-1) x+3 \geq 0, \forall x \in \mathbb{R}
 &\Leftrightarrow&\heva{&\Delta'=(m-1)^2-9(m-1) \leq 0 \\ &a=m-1>0}\\
 &\Leftrightarrow&\heva{&1 \leq m \leq 10 \\ &m>1} \Leftrightarrow 1<m \leq 10\text{ (thỏa mãn)}.
 \end{eqnarray*}
 Mà $m \in \mathbb{Z} \Rightarrow m \in\{2;3;4;5;6;7;8;9;10\}$.
\end{itemize}
Vậy có tất cả $10$ giá trị nguyên của tham số $m$ thoả mãn yêu cầu bài toán.
}
\end{ex}

\begin{ex}%[2-D1B5-SO-14-2425]%[VN-MT-7, Đỗ Minh Phúc]%[2D1V1-3]
 Cho hàm số $y=f(x)$ liên tục trên $\mathbb{R}$ thoả mãn $f'(x)=x(x-1)^2(x-2)^3$. Hàm số $g(x)=f\left(x^2-2x+2\right)$ có bao nhiêu điểm cực trị?
 \shortans{3}
 \loigiai{
 Ta có $f'(x)=0 \Leftrightarrow\hoac{&x=0\\&x=1\\&x=2.}$\\
 Bảng biến thiên
 \begin{center}
 \begin{center}
 \begin{tikzpicture}[scale=1, font=\footnotesize, line join=round, line cap=round, >=stealth]
 \tkzTabInit[nocadre,lgt=1.2,espcl=2.5,deltacl=0.6]
 {$x$/.6,$f'(x)$/.6,$f(x)$/2}
 {$-\infty$,$0$,$1$,$2$,$+\infty$}
 \tkzTabLine{,+,$0$,-,$0$,-,$0$,+}
 \tkzTabVar{-/$-\infty$,+/$ $,R,-/$ $,+/$+\infty$}
 \end{tikzpicture}
 \end{center}
 \end{center}
 Ta có $g'(x)=(2x-2)f'\left(x^2-2x+2\right)$.\\
 Cho $g'(x)=0 \Leftrightarrow\hoac{&2x-2=0\\&f'\left( x^2-2x+2\right)=0}\Leftrightarrow\hoac{&x=1\\&x^2-2x+2=0\\&x^2-2x+2=1\\&x^2-2x+2=2}\Leftrightarrow\hoac{&x=1\\&x=0\\&x=2.}$\\
 Bảng biến thiên
 \begin{center}
 \begin{tikzpicture}
 \tkzTabInit[nocadre,lgt=1.2,espcl=2.5,deltacl=0.6]
 {$x$/0.6,$g'(x)$/0.6,$g(x)$/2}
 {$-\infty$,$0$,$1$,$2$,$+\infty$}
 \tkzTabLine{,-,0,+,0,-,0,+,}
 \tkzTabVar{+/$+\infty$,-/,+/,-/,+/$+\infty$}
 \end{tikzpicture}
 \end{center}
 Vây hàm số $g(x)=f\left(x^2-2x+2\right)$ có $3$ điểm cực trị.
 }
\end{ex}

\begin{ex}%[2-D1B5-SO-14-2425]%[VN-MT-7, Đỗ Minh Phúc]%[2D1V3-1]
 Tìm $m$ để giá trị lớn nhất của hàm số $y=\dfrac{x-m}{x+1}$ trên đoạn $[1;3]$ bằng $2$.
 \shortans{-3}
 \loigiai{
 Ta có $y'=\dfrac{1+m}{(x+1)^2}$.
 \begin{itemize}
 \item Trường hợp 1: $1+m>0 \Leftrightarrow m>-1$.\\
 Khi đó $y'>0$, $\forall x \in[1;3]$ nên hàm số $y=\dfrac{x-m}{x+1}$ đồng biến trên đoạn $[1;3]$.\\
 Suy ra $\max _{[1;3]}y=y(3)=\dfrac{3-m}{4}=2 \Leftrightarrow m=-5$ (loại).
 \item Trường hợp 2: $1+m<0 \Leftrightarrow m<-1$.\\
 Khi đó $y'<0$, $\forall x \in[1;3]$ nên hàm số $y=\dfrac{x-m}{x+1}$ nghịch biến trên đoạn $[1;3]$.\\
 Suy ra $\max _{[1;3]}y=y(1)=\dfrac{1-m}{2}=2 \Leftrightarrow m=-3$ (thỏa mãn).
 \end{itemize}
 Vậy $m=-3$ là giá trị cần tìm.
 }
\end{ex}

\begin{ex}%[2-D1B5-SO-14-2425]%[VN-MT-7, Đỗ Minh Phúc]%[2D1V3-6]
 Chị Hà dự định sử dụng hết $4$ m$^2$ kính để làm một bể cá bằng kính có dạng hình hộp chữ nhật không nắp, chiều dài gấp đôi chiều rộng (các mối ghép có kích thước không đáng kể). Bể cá có dung tích lớn nhất bằng bao nhiêu mét khối (kết quả làm tròn đến hàng phần trăm)?
 \shortans{0{,}73}
 \loigiai{
 \begin{center}
 \begin{tikzpicture}[scale=0.8, font=\footnotesize, line join=round, line cap=round, >=stealth]
 \coordinate (B) at (0,0); 
 \coordinate (A) at (2,2);
 \coordinate (C) at (4,0);
 \coordinate (D) at ($(C)-(B)+(A)$);
 \coordinate (B') at ($(B)+(90:3)$);
 \coordinate (C') at ($(C)+(90:3)$);
 \coordinate (D') at ($(D)+(90:3)$);
 \coordinate (A') at ($(A)+(90:3)$);
 \draw (A')--(B')--(C')--(D')--(A') (B)--(B') (C)--(C') (D)--(D')node[right,midway]{$h$} (B)--(C)node[below,midway,sloped]{$2x$}--(D)node[right,midway]{$x$};
 \draw [dashed] (B)--(A)--(D) (A')--(A);
 \end{tikzpicture}
 \end{center}
 Giả sử bể cá có kích thước như hình vẽ, với $x$, $h>0$.\\
 Theo đề bài ta có $2x^2+2xh+4xh=4 \Leftrightarrow h=\dfrac{4-2x^2}{6x}$.\\
 Do $x>0$, $h>0$ nên $4-2x^2>0 \Leftrightarrow 0<x<\sqrt{2}$.\\
 Thể tích của bể cá là $V=2x^2h=\dfrac{4x-2x^3}{3}=f(x)$, với $x \in\left(0;\sqrt{2}\right)$.\\
 Ta có $f'(x)=\dfrac{4}{3}-2x^2$.\\
 Cho $f'(x)=0 \Leftrightarrow \dfrac{4}{3}-2x^2=0\Leftrightarrow x=\dfrac{\sqrt{6}}{3}$ (vì $x>0$).\\
 Bảng biến thiên
 \begin{center}
 \begin{tikzpicture}
 \tkzTabInit[nocadre,lgt=1.2,espcl=2.5,deltacl=0.6]
 {$x$/0.6,$f'(x)$/0.6,$f(x)$/2}{$0$,$\tfrac{\sqrt{6}}{3}$,$\sqrt{2}$}
 \tkzTabLine{,+,0,-,}
 \tkzTabVar{-/$0$,+/$\tfrac{8\sqrt{6}}{27}$,-/$0$}
 \end{tikzpicture}
 \end{center}
 Vậy bể cá có dung tích lớn nhất bằng $\dfrac{8\sqrt{6}}{27} \mathrm{~m}^3 \approx 0{,}73 \mathrm{~m}^3$.
}
\end{ex}

\begin{ex}%[2-D1B5-SO-14-2425]%[VN-MT-7, Đỗ Minh Phúc]%[2D1V4-2]
 Có bao nhiêu giá trị thực của tham số $m$ để đồ thị hàm số $y=\dfrac{x^2-1}{x^2+(2-m)x+2m+1}$ có đúng hai đường tiệm cận?
 \shortans{3}
 \loigiai{
 Ta có $\lim\limits_{x \to \pm \infty} y=\lim\limits_{x \to \pm \infty} \dfrac{x^2-1}{x^2+(2-m) x+2 m+1}=\lim\limits_{x \to \pm \infty} \dfrac{1-\dfrac{1}{x^2}}{1+(2-m) \dfrac{1}{x}+(2 m+1) \dfrac{1}{x^2}}=1$.\\
 Suy ra đồ thị của hàm số đã cho có đường tiệm cận ngang $y=1$, do vậy đồ thị đó có đúng hai đường tiệm cận khi và chỉ khi đồ thị hàm số có đúng một đường tiệm cận đứng
 $\Leftrightarrow$ phương trình $x^2+(2-m) x+2 m+1=0$ (*) có nghiệm kép hoặc có một nghiệm $x=-1$ và một nghiệm khác $1$ hoặc có một nghiệm $x=1$ và một nghiệm khác $-1$.
 \begin{itemize}
 \item Trường hợp 1: Phương trình (*) có nghiệm kép
 \[\Leftrightarrow \Delta=0 \Leftrightarrow(2-m)^2-4(2m+1)=0 \Leftrightarrow m^2-12 m=0 \Leftrightarrow\hoac{&m=0\\&m=12.}\]
 \item Trường hợp 2: Phương trình (*) một có nghiệm $x=1$ và một nghiệm khác $-1$
 \[\Leftrightarrow\heva{&m=-4\\&m \neq 0}\Leftrightarrow m=-4.\]
 \item Trường hợp 3: Phương trình (*) một có nghiệm $x=-1$ và một nghiệm khác $1$ \[\Leftrightarrow\heva{&m=0\\&m\neq-4}\Leftrightarrow m=0.\]
 \end{itemize}
Vậy có $3$ giá trị của $m$ thỏa mãn yêu cầu bài toán là $m=-4$, $m=0$, $m=12$.
}
\end{ex}

\begin{ex}%[2-D1B5-SO-14-2425]%[VN-MT-7, Đỗ Minh Phúc]%[2D1V1-2]
 \immini[thm]{Cho hàm số $f(x)$ liên tục trên $\mathbb{R}$ và có đồ thị hàm số $y=f'(x)$ như hình vẽ bên. Hàm số $y=f(x)-\dfrac{1}{3}x^3+6x$ đồng biến trên khoảng $(a;b)$. Khi đó giá trị của biểu thức $b-a$ bằng bao nhiêu?}{\begin{tikzpicture}[scale=0.6, font=\footnotesize, line join=round,line cap=round,>=stealth]
 \def\xmin{-5}\def\xmax{3}\def\ymin{-3}\def\ymax{3}
 \draw[->] (\xmin-0.2,0)--(\xmax+0.2,0) node[above] {$x$};
 \draw[->] (0,\ymin-0.2)--(0,\ymax+0.2) node[right] {$y$};
 \draw (0,0) node [below left] {$O$};
 \foreach \x in {-2,2}
 \fill (\x,0)circle (1pt) node [above] {$\x$};
 \foreach \y in {2}
 \fill (0,\y)circle (1pt) node [left] {$\y$};
 \clip (\xmin,\ymin) rectangle (\xmax,\ymax);
 \draw[smooth,samples=200,domain=-4:0] plot (\x,{1*((\x)^2)+4*\x+2});
 \draw[smooth,samples=200,domain=0:3] plot (\x,{-1*((\x)^2)+0*\x+2});
 \draw[dashed] (-2,0)--(-2,-2)--(2,-2)--(2,0);
 \fill (-2,-2) circle (1pt);
 \fill (2,-2) circle (1pt);
 \fill (0,0) circle (1pt);
 \end{tikzpicture}}
 \shortans{4}
 \loigiai{
 Ta có $y=f(x)-\dfrac{1}{3} x^3+6 x$ nên $y'=f'(x)-x^2+6$.\\
 Quan sát đồ thị hàm số $y=f'(x)$ và parabol $(P)\colon y=x^2-6$ trên cùng một hệ trục tọa độ như hình vẽ.
 \begin{center}
 \begin{tikzpicture}[scale=0.8, font=\footnotesize, line join=round, line cap=round, >=stealth]
 \def\xmin{-5}\def\xmax{3}\def\ymin{-7}\def\ymax{3}
 \draw[->] (\xmin-0.2,0)--(\xmax+0.2,0) node[above] {$x$};
 \draw[->] (0,\ymin-0.2)--(0,\ymax+0.2) node[right] {$y$};
 \draw (0,0) node [below left] {$O$};
 \foreach \x in {-2,2}
 \fill (\x,0)circle (1pt) node [above] {$\x$};
 \foreach \y in {2}
 \fill (0,\y)circle (1pt) node [left] {$\y$};
 \foreach \y in {-6,-2}
 \fill (0,\y)circle (1pt) node [below right] {$\y$};
 \clip (\xmin,\ymin) rectangle (\xmax,\ymax);
 \draw[smooth,samples=200,domain=-4:0] plot (\x,{1*((\x)^2)+4*\x+2});
 \draw[smooth,samples=200,domain=0:3] plot (\x,{-1*((\x)^2)+0*\x+2});
 \draw[smooth,samples=200,domain=-3:3] plot (\x,{1*((\x)^2)+0*\x+-6});
 \draw[dashed] (-2,0)--(-2,-2)--(2,-2)--(2,0);
 \fill (-2,-2) circle (1pt);
 \fill (2,-2) circle (1pt);
 \fill (0,0) circle (1pt);
 \end{tikzpicture}
 \end{center}
 Từ đồ thị ta có $y'=f'(x)-x^2+6>0 \Leftrightarrow f'(x)>x^2-6 \Leftrightarrow-2<x<2$.\\
 Vậy hàm số $y=f(x)-\dfrac{1}{3}x^3+6x$ đồng biến trên khoảng $(-2;2)$.
}
\end{ex}
\Closesolutionfile{ans}
\begin{indapan}
	{ans/ans\currfilebase}
\end{indapan}


% \begin{name}
	{\tenchude}
	{ĐỀ ÔN TẬP CHƯƠNG I}
	{LỚP TOÁN THẦY PHÁT}
	{\thoigian}
\end{name}
\TN
\Opensolutionfile{ans}[ans/ans\currfilebase-Phan-I]
\begin{ex}%[2-D1B5-SO-15-2425]%[VN-MT-7, Đoàn Thị Lý]%[2D1N1-1]
 Cho hàm số $y=f(x)$ có đạo hàm $f'(x)=x(x-2)^3$, với mọi $x \in \mathbb{R}$. Hàm số đã cho nghịch biến trên khoảng nào dưới đây?
 \choice 
 {$(1;3)$}
 {$(-1;0)$}
 {\True $(0;1)$}
 {$(-2;0)$}
 \loigiai{
 Ta có $f'(x)=0 \Leftrightarrow\hoac{&x=0 \\&x=2.}$ \\
 Bảng xét dấu $f'(x)$
 \begin{center}
 \begin{tikzpicture}
 \tkzTabInit[nocadre=true,lgt=1.2,espcl=2.5,deltacl=0.5]
 {$x$ /.7, $f'(x)$ /.7}
 {$-\infty$,$0$,$2$,$+\infty$}
 \tkzTabLine{ ,+,0,-,0, +, }
 \end{tikzpicture}
 \end{center}
 Dựa vào bảng xét dấu $f'(x)$ ta thấy hàm số đã cho nghịch biến trên $(0;1)$.
 }
\end{ex}

\begin{ex}%[2-D1B5-SO-15-2425]%[VN-MT-7, Đoàn Thị Lý]%[2D1N1-2] 
 Cho hàm số $y=f(x)$ có tập xác định là $\mathscr{D}=\mathbb{R}\setminus\big\{-1\big\}$ và có bảng xét dấu của đạo hàm như hình sau:
 \begin{center}
 \begin{tikzpicture}
 \tkzTabInit[nocadre=true,lgt=1.2,espcl=2.5,deltacl=0.5]
 {$x$ /.7, $f'(x)$ /.7}
 {$-\infty$,$-1$,$1$,$+\infty$}
 \tkzTabLine{ ,-,d,-,0, +, }
 \end{tikzpicture}
 \end{center}
 Hàm số đã cho nghịch biến trên khoảng nào dưới đây? 
 \choice 
 {$(1 ;+\infty)$}
 {$(-\infty; 1)$}
 {$(-1;+\infty)$}
 {\True $(-\infty;-1)$}
 \loigiai{ 
 Từ bảng xét dấu ta thấy hàm số đã cho nghịch biến trên khoảng $(-\infty;-1)$.
 }
\end{ex}

\begin{ex}%[2-D1B5-SO-15-2425]%[VN-MT-7, Đoàn Thị Lý]%[2D1N3-1]
 \immini{
 Cho hàm số $y=f(x)$ liên tục trên đoạn $[-2;2]$ và có đồ thị như hình vẽ bên. Gọi $M$ và $m$ lần lượt là giá trị lớn nhất và giá trị nhỏ nhất của hàm số đã cho trên đoạn $[-2;2]$. Giá trị của $M+m$ bằng
 \choice[2]
 {$0$}
 {$1$}
 {$4$}
 {\True $3$}
 }
 {\begin{tikzpicture} [scale=.85, font=\footnotesize, line join=round, line cap=round, >=stealth]
 \draw[->] (-3,0)--(0,0) node[below right]{$O$}--(3,0) node[below]{$x$};
 \draw[->] (0,-1)--(0,4) node[right]{$y$};
 \draw (-2,3) .. controls ++(-80:1) and ++(170:.5) .. (-1.4,0).. controls (-1.2,.2) and (-1.1,1).. (-1,1).. controls (-.8,1.1) and (-.6,0) .. (-.4,0) .. controls (-.3,0.1) .. (0,1).. controls (-.1,.75) and (.6,3) .. (1,3).. controls (1.15,2.9).. (2,1);
 \draw[dashed] (-2,0) node[below]{$-2$}--(-2,3) --(0,3)node[above left]{$3$}--(1,3)--(1,0)node[below]{$1$} (-1,0)node[below]{$-1$}--(-1,1)--(0,1)node[above left]{$1$}--(2,1)--(2,0)node[below]{$2$};
 \foreach \i in {(0,3),(-2,3),(1,3),(-1,1),(0,1),(-2,0), (-1,0),(1,0),(2,0),(2,1),(-1.4,0),(-.4,0),(0,0)}\fill \i circle (1pt); 
 \end{tikzpicture}
 }
 \loigiai{ 
 Quan sát đồ thị ta thấy $M=\max\limits_{[-2;2]} f(x)=3$ và $m=\min\limits_{[-2 ; 2]} f(x)=0$. Vậy $M+m=3+0=3$.
 }
\end{ex}

\begin{ex}%[2-D1B5-SO-15-2425]%[VN-MT-7, Đoàn Thị Lý]%[2D1N4-1]
 Cho hàm số $y=f(x)$ có bảng biến thiên như sau:
 \begin{center}
 \begin{tikzpicture}
 \tkzTabInit[nocadre=true,lgt=1.2,espcl=3, deltacl=0.5]
 {$x$/0.7,$f’(x)$/0.7,$f(x)$/2}
 {$-\infty$,$0$,$1$,$+\infty$}
 \tkzTabLine{,+,0,-,d,+,}
 \tkzTabVar{-/$0$,+/$2$,-D-/$-\infty$/$3$,+/$5$}
 \end{tikzpicture}
 \end{center}
 Tổng số tiệm cận ngang và tiệm cận đứng của đồ thị hàm số đã cho là 
 \choice 
 {$4$}
 {\True $3$}
 {$2$}
 {$1$}
 \loigiai{
 Từ bảng biến thiên ta có
 \begin{itemize}
 \item $\lim\limits _{x \to-\infty} y=0$, suy ra đồ thị hàm số có tiệm cận ngang là $y=0$.
 \item $\lim\limits _{x\to+\infty} y=5$, suy ra đồ thị hàm số có tiệm cận ngang là $y=5$.
 \item $\lim\limits _{x \to 1^-} y=+\infty$, suy ra đồ thị hàm số có tiệm cận đứng là $x=1$.
 \end{itemize}
 Vậy tổng số tiệm cận ngang và tiệm cận đứng của đồ thị hàm số đã cho là $3$.
 }
\end{ex}

\begin{ex}%[2-D1B5-SO-15-2425]%[VN-MT-7, Đoàn Thị Lý]%[2D1N4-1]
 Cho hàm số $y=\dfrac{2 x+1}{x-3}$. Tiệm cận ngang của đồ thị hàm số đã cho là
 \choice 
 {$x=3$}
 {$x=2$}
 {$x=-\dfrac{1}{2}$}
 {\True $y=2$}
 \loigiai{
 Ta có $\lim\limits _{x \to \pm \infty} y=\lim\limits _{x \to \pm \infty} \dfrac{2 x+1}{x-3}=\lim\limits _{x \to \pm \infty} \dfrac{2+\dfrac{1}{x}}{1-\dfrac{3}{x}}=2$, suy ra đồ thị hàm số có tiệm cận ngang $y=2$.
 }
\end{ex}

\begin{ex}%[2-D1B5-SO-15-2425]%[VN-MT-7, Đoàn Thị Lý]%[2D1N5-1]
 \immini{
 Biết hàm số $y=\dfrac{ax+b}{cx+d}$ có đồ thị như trong hình vẽ bên. Mệnh đề nào dưới đây đúng?
 \choice
 {\True $y'>0$, $\forall x \neq 1$}
 {$y'>0$, $\forall x \in \mathbb{R}$}
 {$y'<0$, $\forall x \in \mathbb{R}$}
 {$y'<0$, $\forall x \neq 1$}
 }{
 \begin{tikzpicture} [scale=.8, font=\footnotesize, line join=round, line cap=round, >=stealth]
 \draw[->] (-2,0)--(0,0) node[below right]{$O$}--(1,0) node[above right]{$1$}--(4,0) node[below]{$x$}; 
 \draw[->] (0,-2)--(0,4) node[left]{$y$}; 
 \fill (0,0) circle (1.0pt) (1,0) circle (1.0pt); 
 \begin{scope}
 \clip (-2,-2) rectangle (4,4);
 \draw[samples=300,domain=-2:4,smooth] plot (\x, {(\x-2)/(\x-1)}); 
 \end{scope}
 \draw (-2,1)--(4,1); 
 \end{tikzpicture}
 }
 \loigiai{ 
 Tập xác định của hàm số đã cho là $\mathscr{D}=\mathbb{R}\setminus\{1\}$.\\
 Từ đồ thị của hàm số suy ra hàm số đã cho đồng biến trên mỗi khoảng xác định vì vậy $y'>0, \forall x \neq 1$.
 }
\end{ex}

\begin{ex}%[2-D1B5-SO-15-2425]%[VN-MT-7, Đoàn Thị Lý]%[2D1N5-1]
 \immini{Đồ thị của hàm số nào dưới đây có dạng như đường cong trong hình bên?
 \choice 
 {\True $y=x^3-3 x$}
 {$y=-x^3+3 x$}
 {$y=x^4-2 x^2$}
 {$y=-x^4+2 x^2$}
 }
 {
 \begin{tikzpicture} [scale=.8, font=\footnotesize, line join=round, line cap=round, >=stealth]
 \draw[->] (-2.5,0)--(0,0) node[below left]{$O$}--(2.5,0) node[below]{$x$};
 \draw[->] (0,-2.5)--(0,2.5) node[right]{$y$};
 \draw[samples=100,domain=-2.05:2.05,smooth] plot (\x, {(\x)^3-3*(\x)});
 \fill (0,0) circle (1.0pt);
 \end{tikzpicture} 
 }
 \loigiai{
 Đường cong có dạng của đồ thị hàm số bậc ba với hệ số $a>0$ nên chỉ có hàm số $y=x^3-3 x$ thỏa yêu cầu bài toán.
 }
\end{ex}

\begin{ex}%[2-D1B5-SO-15-2425]%[VN-MT-7, Đoàn Thị Lý]%[2D1H1-3]
 Tập hợp tất cả các giá trị thực của tham số $m$ để hàm số $y=x^3+(m+1) x^2+3 x+2$ đồng biến trên $\mathbb{R}$ là
 \choice 
 {\True $[-4;2]$}
 {$(-4;2)$}
 {$(-\infty;-4] \cup[2;+\infty)$}
 {$(-\infty;-4)\cup(2;+\infty)$}
 \loigiai{ 
 Tập xác định $\mathscr{D}=\mathbb{R}$.\\
 Ta có $y'=3 x^2+2(m+1) x+3$.\\
 Hàm số $y=x^3+(m+1) x^2+3 x+2$ đồng biến trên $\mathbb{R}$ khi và chỉ khi 
 \[y' \geq 0, \forall x \in \mathbb{R}
 \Leftrightarrow \Delta'=(m+1)^2-9 \leq 0 \Leftrightarrow m^2+2 m-8 \leq 0 \Leftrightarrow-4 \leq m \leq 2.\] 
 Vậy $m \in[-4 ; 2]$.
 }
\end{ex}

\begin{ex}%[2-D1B5-SO-15-2425]%[VN-MT-7, Đoàn Thị Lý]%[2D1H3-1]
 Gọi $M$ và $m$ lần lượt là giá trị lớn nhất và giá trị nhỏ nhất của hàm số $y=f(x)=\dfrac{2x+1}{x-2}$ trên đoạn $[3;7]$ . Tính giá trị của $M^2+m$.
 \choice 
 {\True $52$}
 {$58$}
 {$6$}
 {$10$}
 \loigiai{
 Hàm số $y=f(x)=\dfrac{2 x+1}{x-2}$ liên tục trên $[3;7]$.\\
 Ta có 
 $y'=\dfrac{-5}{(x-2)^2}<0, \forall x \in[3 ; 7]$ nên hàm số $y=f(x)$ nghịch biến trên $[3;7]$.\\ Lúc đó
 \[M=\max\limits_{[3;7]} f(x) = f(3)=7,m=\min\limits_{[3;7]}f(x)=f(7)=3.\]
 Vậy $M^2+m=52$.
 }
\end{ex}

\begin{ex}%[2-D1B5-SO-15-2425]%[VN-MT-7, Đoàn Thị Lý]%[2D1H4-1]
 Đồ thị hàm số $y=\dfrac{x^2-3 x-4}{x^2-16}$ có bao nhiêu đường tiệm cận đứng? 
 \choice 
 {\True $1$}
 {$0$}
 {$2$}
 {$3$}
 \loigiai{
 Tập xác định $\mathscr{D}=\mathbb{R} \setminus\big\{- 4;4\big\}$. Ta có
 \begin{itemize}
 \item $\lim\limits _{x \to (-4)^-} y=\lim\limits _{x \to (-4)^-} \dfrac{x^2-3 x-4}{x^2-16}=\lim\limits _{x \to (-4)^-} \dfrac{(x+1)(x-4)}{(x+4)(x-4)}=\lim\limits _{x \to (-4)^-} \dfrac{x+1}{x+4}=+\infty$, suy ra $x=-4$ là tiệm cận đứng của đồ thị hàm số.
 \item $\lim\limits_{x \to 4} y=\lim\limits _{x \to 4} \dfrac{x^2-3 x-4}{x^2-16}=\lim\limits _{x \to 4} \dfrac{(x+1)(x-4)}{(x+4)(x-4)}=\lim\limits _{x \to 4} \dfrac{x+1}{x+4}=\dfrac{5}{8}$, suy ra $x=4$ không là tiệm cận đứng của đồ thị hàm số.
 \end{itemize}
 Vậy đồ thị hàm số có $1$ đường tiệm cận đứng.
 }
\end{ex}

\begin{ex}%[2-D1B5-SO-15-2425]%[VN-MT-7, Đoàn Thị Lý]%[2D1H5-4] 
 Đồ thị của hàm số $y=x^3-3 x+2$ cắt trục tung tại điểm có tung độ bằng 
 \choice
 {$0$}
 {$1$}
 {\True $2$}
 {$3$}
 \loigiai{
 Gọi $M\left(x_0;y_0\right)$ là giao điểm của đồ thị hàm số với trục tung. Ta có $x_0=0 \Rightarrow y_0=2$.
 }
\end{ex}

\begin{ex}%[2-D1B5-SO-15-2425]%[VN-MT-7, Đoàn Thị Lý]%[2D1H5-4]
 Số giao điểm của đồ thị hàm số $y=x^3-3 x+1$ và trục hoành là \choice 
 {\True $3$}
 {$0$}
 {$2$}
 {$1$}
 \loigiai{
 Tập xác định $\mathscr{D}=\mathbb{R}$.\\
 Ta có $y'=3 x^2-3=3\left(x^2-1\right)$, $y'=0 \Leftrightarrow x= \pm 1$.\\
 Bảng biến thiên của hàm số
 \begin{center}
 \begin{tikzpicture}
 \tkzTabInit[nocadre=true,lgt=1.2,espcl=3, deltacl=0.5]
 {$x$/0.7,$y'$/0.7,$y$/2}
 {$-\infty$,$-1$,$1$,$+\infty$}
 \tkzTabLine{,-,0,+,0,-,}
 \tkzTabVar{-/$-\infty$,+/$3$,-/$-1$,+/$+\infty$}
 \end{tikzpicture}
 \end{center}
 Từ bảng biến thiên ta thấy đồ thị hàm số cắt trục hoành tại $3$ điểm phân biệt.
 }
\end{ex}
\Closesolutionfile{ans}

\TNTF
\Opensolutionfile{ans}[ans/ans\currfilebase-Phan-II]
\begin{ex}%[2-D1B5-SO-15-2425]%[VN-MT-7, Đoàn Thị Lý]%[2D1N3-1]
 \immini{
 Cho hàm số $y=f(x)$ liên tục trên $\mathbb{R}$ và có đồ thị như hình vẽ bên. 
 \choiceTF
 {\True Hàm số đồng biến trên khoảng $(0;2)$}
 {Hàm số đạt cực đại tại $x=0$}
 {\True Giá trị nhỏ nhất của hàm số trên $[-1;1]$ bằng $-4$} 
 {Hàm số $g(x)=f(3-x)$ nghịch biến trên $(2;5)$}
 }
 {
 \begin{tikzpicture} [scale=.8, font=\footnotesize, line join=round, line cap=round, >=stealth]
 \draw[->] (-1.8,0)--(-1,0) node[below left]{$-1$}--(0,0) node[below left]{$O$}--(2,0) node[above]{$2$}--(3.2,0) node[below]{$x$};
 \draw[->] (0,-4.8)--(0,-4) node[below left]{$-4$}--(0,1.2) node[right]{$y$};
 \draw[samples=100,domain=-1.1:3.05,smooth] plot (\x, {-(\x)^3+3*(\x)^2-4});
 \foreach \x in {(0,0),(-1,0),(0,-4),(2,0)}
 \fill \x circle (1pt);
 \end{tikzpicture}
 }
 \loigiai{
 \begin{itemchoice}
 \itemch \textbf{Đúng}.\\
 Hàm số đồng biến trên khoảng $(0;2)$.
 \itemch \textbf{Sai}.\\
 Hàm số đạt cực đại tại $x=2$.
 \itemch \textbf{Đúng}.\\
 Ta có $\min\limits_{[-1; 1]} f(x)=-4$.
 \itemch \textbf{Sai}.\\
 Xét hàm số $g(x)=f(3-x)$. Vì $f(x)$ liên tục trên $\mathbb{R}$ nên $g(x)$ liên tục trên $\mathbb{R}$.\\
 Từ đồ thị của hàm số ta có bảng xét dấu của $f'(x)$ như sau:
 \begin{center}
 \begin{tikzpicture}
 \tkzTabInit[nocadre=true,lgt=1.2,espcl=2.5,deltacl=0.5]
 {$x$ /.7, $f'(x)$ /.7}
 {$-\infty$,$0$,$2$,$+\infty$}
 \tkzTabLine{ ,-,0,+,0, -, }
 \end{tikzpicture}
 \end{center}
 Ta có $g'(x)=(3-x)' f'(3-x)=-f'(3-x)$.\\
 Cho $g'(x)=0 \Leftrightarrow-f'(3-x)=0 \Leftrightarrow\hoac{&3-x=0 \\ &3-x=2} \Leftrightarrow\hoac{&x=3 \\& x=1.}$\\
 Từ bảng xét dấu của $f'(x)$ suy ra được bảng xét dấu của $g'(x)$ như sau:
 \begin{center}
 \begin{tikzpicture}
 \tkzTabInit[nocadre=true,lgt=1.2,espcl=2.5,deltacl=0.5]
 {$x$ /.7, $g'(x)$ /.7}
 {$-\infty$,$1$,$3$,$+\infty$}
 \tkzTabLine{ ,+,0,-,0, +, }
 \end{tikzpicture}
 \end{center}
 Vậy hàm số $g(x)$ không nghịch biến trên $(2;5)$.
 \end{itemchoice}
 }
\end{ex}

\begin{ex}%[2-D1B5-SO-15-2425]%[VN-MT-7, Đoàn Thị Lý]%[2D1H2-2]
 Cho hàm số $y=f(x)$ liên tục trên $\mathbb{R}$ và có bảng xét dấu của đạo hàm $f'(x)$ như hình sau: 
 \begin{center}
 \begin{tikzpicture}
 \tkzTabInit[nocadre=true,lgt=1.2,espcl=2.5,deltacl=0.5]
 {$x$ /.7, $f'(x)$ /.7}
 {$-\infty$,$-2$,$0$,$1$,$2$,$+\infty$}
 \tkzTabLine{ ,+,0,-,d,+,0,-,0,-, }
 \end{tikzpicture}
 \end{center}
 \choiceTF
 {Hàm số đã cho đồng biến trên các khoảng $(-\infty;0)$ và $(0;1)$}
 {\True Hàm số đã cho nghịch biến trên khoảng $(3 ;+\infty)$}
 {Hàm số đã cho có $2$ điểm cực trị}
 {\True Hàm số đã cho đạt cực tiểu tại điểm $x=0$}
 \loigiai{
 Dựa vào bảng xét dấu của $f'(x)$ ta có
 \begin{itemchoice}
 \itemch \textbf{Sai}.\\
 Hàm số không đồng biến trên khoảng $(-\infty;0)$.
 \itemch \textbf{Đúng}.\\
 Hàm số nghịch biến trên khoảng $(1;+\infty)$ chứa khoảng $(3;+\infty)$.
 \itemch \textbf{Sai}.\\
 Hàm số đã cho liên tục trên $\mathbb{R}$ và $f'(x)$ đổi dấu ba lần nên hàm số đã cho có $3$ điểm cực trị.
 \itemch \textbf{Đúng}.\\
 Hàm số đã cho liên tục trên $\mathbb{R}$ và tại điểm $x=0$, $f'(x)$ đổi dấu từ âm sang dương nên hàm số đã cho đạt cực tiểu tại $x=0$.
 \end{itemchoice} 
 }
\end{ex}

\begin{ex}%[2-D1B5-SO-15-2425]%[VN-MT-7, Đoàn Thị Lý]%[2D1H3-1]
 Cho hàm số $y=f(x)=\dfrac{3 x-1}{x-3}$. Gọi $M$ và $m$ lần lượt là giá trị lớn nhất và nhỏ nhất của hàm số $y=f(x)$ trên đoạn $[0;2]$.
 \choiceTF
 {Hàm số đã cho đồng biến trên khoảng $(0;2)$} 
 {$M=f(1)=\dfrac{1}{3}$}
 {\True $m=f(2)=-5$}
 {\True Có $5$ giá trị nguyên dương bé hơn $10$ của $t$ sao cho $f(x)\le t,\,\forall x\in [0;2]$}
 \loigiai{
 Hàm số đã cho liên tục trên đoạn $[0;2]$.\\
 Ta có $y'=-\dfrac{8}{(x-3)^2}<0$, $ \forall x \neq 3$, suy ra hàm số nghịch biến trên đoạn $[0 ; 2]$.\\
 Vậy $M=\max\limits_{[0 ; 2]}f(x)=f(0)=\dfrac{1}{3}$ và $m=\min\limits_{[0 ; 2]}f(x)=f(2)=-5$.
 \begin{itemchoice}
 \itemch \textbf{Sai}.\\
 Vì hàm số nghịch biến trên các khoảng $\left( -\infty;3\right)$ và $\left(3;+\infty\right)$ nên hàm số nghịch biến trên $(0;2)~\subset~ \left( -\infty;3\right)$.
 \itemch \textbf{Sai}.\\
 Vì $M=\max\limits_{[0;2]}f(x)=f(0)=\dfrac{1}{3}$.
 \itemch \textbf{Đúng}.\\
 Vì $m=\min\limits_{[0;2]}f(x)=f(2)=-5$.
 \itemch \textbf{Sai}.\\
 Ta có $f(x)\le t,\,\forall x\in [0;2]\Leftrightarrow \max\limits_{[0;2]}f(x)\le t\Leftrightarrow t\ge \dfrac{1}{3}$.\\
 Vì $t$ nguyên dương và bé hơn $6$ nên $t\in\left\lbrace 1;2;3;4;5\right\rbrace$.\\ Vậy có $5$ giá trị của $t$ thỏa mãn.
 \end{itemchoice} 
 }
\end{ex}

\begin{ex}%[2-D1B5-SO-15-2425]%[VN-MT-7, Đoàn Thị Lý]%[2D1N4-1]
 Cho hàm số $y=f(x)$ xác định và liên tục trên $\mathbb{R}\setminus\{-1\}$, có bảng biến thiên như sau:
 \begin{center}
 \begin{tikzpicture}
 \tkzTabInit[nocadre=true, lgt=1, espcl=4, deltacl=0.5]
 {$x$/0.7,$y’$/0.7,$y$/2}
 {$-\infty$,$-1$,$+\infty$}
 \tkzTabLine{,+,d,+,}
 \tkzTabVar{-/$-2$,+D-/$+\infty$/$-\infty$,+/$-2$}
 \end{tikzpicture}
 \end{center}
 \choiceTF
 {Hàm số có $2$ cực trị}
 {\True Đồ thị hàm số cắt đường thẳng $y=1$ tại đúng $1$ điểm}
 {Hàm số đồng biến trên $\left(-2;3\right)$}
 {\True Đồ thị hàm số có tiệm cận đứng $x=-1$ và tiệm cận ngang $y=-2$}
 \loigiai{ 
 Từ bảng biến thiên, ta có
 \begin{itemchoice}
 \itemch \textbf{Sai}.\\
 Vì $f'(x)$ không đổi dấu trên $\mathbb{R}\setminus\{-1\}$ nên hàm số không có cực trị.
 \itemch \textbf{Đúng}.\\
 Đồ thị hàm số cắt đường thẳng $y=1$ tại đúng $1$ điểm.
 \itemch \textbf{Sai}.\\
 Vì hàm số không xác định trên $\left(-2;3\right)$ nên hàm số không đồng biến trên $\left(-2;3\right)$.
 \itemch \textbf{Đúng}.
 \begin{itemize}
 \item $\lim\limits _{x \to(-1)^-} f(x)=+\infty$ và $\lim\limits _{x \to(-1)^+} f(x)=-\infty$, suy ra $x=-1$ là tiệm cận đứng.
 \item $\lim\limits _{x \to-\infty} y=-2$ và $\lim\limits_{x\to \infty} y=-2$, suy ra $y=-2$ là tiệm cận ngang. 
 \end{itemize}
 Vậy đồ thị hàm số có tiệm cận đứng là $x=-1$ và tiệm cận ngang là $y=-2$.
 \end{itemchoice}
 }
\end{ex}
\Closesolutionfile{ans}

\TNSA
\Opensolutionfile{ans}[ans/ans\currfilebase-Phan-III]
\begin{ex}%[2-D1B5-SO-15-2425]%[VN-MT-7, Đoàn Thị Lý]%[2D1H2-1]
 Cho hàm số $y=\dfrac{2 x^2+5 x+4}{x+2}$. Độ dài của đoạn thẳng nối hai điểm cực trị của đồ thị hàm số bằng bao nhiêu (kết quả làm tròn đến hàng phần trăm)?
 \shortans{8{,}24}
 \loigiai{ 
 Điều kiện $x \neq-2$.\\
 Ta có $y'=\dfrac{2 x^2+8 x+6}{(x+2)^2}$ $(x \neq-2)$.
 Cho $y'=0 \Rightarrow 2 x^2+8 x+6=0 \Rightarrow\hoac{&x=-1 \\& x=-3.}$\\
 Với $x=-1\Rightarrow y=1$.\\
 Với $x=-3\Rightarrow y=-7$.\\
 Đồ thị hàm số có hai điểm cực trị $A(-1;1)$ và $B(-3;-7)$. Suy ra $AB=2 \sqrt{17}\approx 8{,}24$.
 }
\end{ex}

\begin{ex}%[2-D1B5-SO-15-2425]%[VN-MT-7, Đoàn Thị Lý]%[2D1H2-2]
 \immini{
 Cho hàm số $y=f(x)$ có đồ thị $y=f'(x)$ như hình vẽ bên. Hàm số $y=f(x)$ có bao nhiêu điểm cực trị?
 }
 {
 \begin{tikzpicture} [scale=.8, font=\footnotesize, line join=round, line cap=round, >=stealth]
\def \xmin{-3}\def \xmax{4}\def \ymin{-1.3}\def \ymax{3.5} 
\draw[->] (\xmin,0)--(\xmax,0) node[shift=(-110:0.2)] {$x$};
\draw[->] (0,\ymin)--(0,\ymax) node[shift=(-150:0.2)] {$y$};
\fill (0,0) circle(1pt) node[shift=(-135:0.25)]{$O$}
 (-1,0) circle(1pt) node[shift=(-90:0.2)]{$-1$}
 (2,0) circle(1pt) node[shift=(-90:0.2)]{$2$};
\clip (\xmin,\ymin) rectangle (\xmax,\ymax); 
\draw[yscale=0.2, smooth,samples=100,domain=\xmin:\xmax] plot(\x, {(\x)^3-1.5*(\x)^2-6*(\x)+10});
 \draw[dashed] (-1,0)--(-1,2.7)--(0,2.7);
 \draw (3,2.1) node[rotate=73]{$y=f'(x)$};
 \end{tikzpicture}
 }
 \shortans{1}
 \loigiai{
 Số điểm cực trị của hàm số đã cho là $1$. Vì dựa vào đồ thị của $f'(x)$ ta thấy $f'(x)$ đổi dấu một lần từ âm sang dương nên hàm số đã cho có một cực trị (một cực tiểu).
 }
\end{ex}

\begin{ex}%[2-D1B5-SO-15-2425]%[VN-MT-7, Đoàn Thị Lý]%[2D1H4-1]
 Tiệm cận xiên của đồ thị hàm số $y=f(x)=\dfrac{x^2-x+1}{x-1}$ có dạng $y=a x+b,(a, b \in \mathbb{Z})$. Tính giá trị biểu thức $P=5 a+2024 b$.
 \shortans{5}
 \loigiai{
 Giả sử hàm số có đồ thị là $(C)$. Ta có $y=\dfrac{x^2-x+1}{x-1}=x+\dfrac{1}{x-1}$. Từ đó có
 \[\lim\limits _{x \to \pm \infty}[f(x)-x]=\lim\limits _{x \to \pm \infty}\left[\dfrac{x^2-x+1}{x-1}-x\right]=\lim\limits _{x \to \pm \infty} \dfrac{1}{x-1}=\lim\limits _{x \to \pm \infty} \dfrac{\dfrac{1}{x}}{1-\dfrac{1}{x}}=0.\] 
 Suy ra $(C)$ có tiệm cận xiên là đường thẳng $y=x$.\\
 Vậy $a=1$, $b=0$, $P=5a+2024 b=5\cdot 1+2024\cdot 0=5$.
 }
\end{ex}

\begin{ex}%[2-D1B5-SO-15-2425]%[VN-MT-7, Đoàn Thị Lý]%[2D1V1-3]
 Cho hàm số $y=f(x)$ có đạo hàm trên $\mathbb{R}$ và bảng xét dấu đạo hàm như hình vẽ sau:
 \begin{center}
 \begin{tikzpicture}
 \tkzTabInit[nocadre=true,lgt=1.2,espcl=2.5,deltacl=0.5]
 {$x$ /.7, $f'(x)$ /.7}
 {$-\infty$,$-10$,$-2$,$3$,$8$,$+\infty$}
 \tkzTabLine{ ,+,0,+,0,-,0,-,0,+, }
 \end{tikzpicture}
 \end{center} 
 Tìm $m$ để hàm số $y=f\left( x^3+4x+m \right)$ nghịch biến trên khoảng $(-1;1)$?
 \shortans{3}
 \loigiai{
 Đặt $t=x^3+4x+m\Rightarrow t'=3x^2+4>0,\,\forall x\in (-1;1)$ nên $t$ đồng biến trên $(-1;1)$ và $t\in \left(m-5;m+5\right)$.\\
 Yêu cầu bài toán trở thành tìm $m$ để hàm số $f(t)$ nghịch biến trên khoảng $\left( m-5;m+5 \right)$.\\
 Dựa vào bảng xét dấu ta được $\heva{& m-5\ge -2 \\ 
 & m+5\le 8}\Leftrightarrow \heva{& m\ge 3 \\ 
 & m\le 3}\Leftrightarrow m=3$.
 }
\end{ex}

\begin{ex}%[2-D1B5-SO-15-2425]%[VN-MT-7, Đoàn Thị Lý]%[2D1V1-3]
 \immini{
 Cho hàm số $y=f(x)$ có đạo hàm liên tục trên $\mathbb{R}$ và có đồ thị $y=f'(x)$ như hình vẽ bên. Đặt $g(x)=f(x-m)-\dfrac{1}{2}{(x-m-1)^2}+2019$, với $m$ là tham số thực. Gọi $S$ là tập hợp các giá trị nguyên dương của $m$ để hàm số $y=g(x)$ đồng biến trên khoảng $(5;6)$. Tính tổng tất cả các phần tử thuộc $S$.
 }
 {
 \begin{tikzpicture} [scale=.8, font=\footnotesize, line join=round, line cap=round, >=stealth]
 \draw[->] (-1.8,0)--(0,0) node[below left]{$O$}--(4,0) node[below]{$x$};
 \draw[->] (0,-3)--(0,3) node[right]{$y$};
 \draw[samples=100,domain=-1.1:3.1,smooth] plot (\x, {(\x)^3-3*(\x)^2+2});
 \draw[dashed] (-1,0) node[below left]{$-1$}--(-1,-2)--(0,-2)node[below left]{$-2$}--(2,-2)--(2,0)node[above]{$2$} (0,2)node[above left]{$2$}--(3,2)--(3,0) node[below]{$3$};
 \draw (1,0) node[below] {$1$};
 \foreach \i in {(0,0),(3,2),(2,-2),(0,2),(-1,-2),(-1,0),(2,0),(0,-2),(1,0),(3,0)}\fill \i circle (1pt);
 \draw (3.3,2.2) node[rotate=82]{$y=f'(x)$};
 \end{tikzpicture}
}
 \shortans{14}
 \loigiai{
 Ta có 
 $g'(x)=f'(x-m)-(x-m-1)$.\\
 Xét phương trình $g'(x)=0\Leftrightarrow f'(x-m)-(x-m-1)=0\qquad (1)$.\\
 Đặt $x-m=t$, phương trình (1) trở thành $f'(t)-(t-1)=0\Leftrightarrow f'(t)=t-1\qquad(2)$.\\
 Nghiệm của phương trình $(2)$ là hoành độ giao điểm của hai đồ thị hàm số $y=f'(t)$ và $y=t-1$.\\
 Ta có đồ thị các hàm số $y=f'( t)$ và $y=t-1$ như sau: 
 \begin{center}
 \begin{tikzpicture} [scale=.8, font=\footnotesize, line join=round, line cap=round, >=stealth]
 \draw[->] (-1.8,0)--(0,0) node[below left]{$O$}--(4,0) node[below]{$x$};
 \draw[->] (0,-3)--(0,3) node[right]{$y$};
 \draw[samples=100,domain=-1.1:3.1,smooth] plot (\x, {(\x)^3-3*(\x)^2+2});
 \draw[samples=100,domain=-1.4:3.2,smooth] plot (\x, {(\x)-1});
 \draw[dashed] (-1,0) node[below left]{$-1$}--(-1,-2)--(0,-2)node[below left]{$-2$}--(2,-2)--(2,0)node[above]{$2$} (0,2)node[above left]{$2$}--(3,2)--(3,0) node[below]{$3$};
 \draw (1,0) node[below]{$1$};
 \foreach \i in {(0,0),(1,0),(3,2),(2,-2),(0,2),(-1,-2),(-1,0),(2,0),(0,-2),(1,0),(3,0)}\fill \i circle (1pt);
 \draw (3.3,2.2) node[rotate=82]{$y=f'(x)$};
 \end{tikzpicture}
 \end{center} 
 Căn cứ đồ thị các hàm số ta có phương trình (2) có nghiệm là $\hoac{& t=-1 \\ 
 & t=1 \\ 
 & t=3 }
 \Rightarrow\hoac{&x=m-1 \\ 
 & x=m+1 \\ 
 & x=m+3.}$\\
 Ta có bảng biến thiên của $y=g(x)$ như sau:
 \begin{center}
 \begin{tikzpicture}
 \tkzTabInit[nocadre=true,lgt=1.2,espcl=2.5,deltacl=0.5]
 {$x$/0.7,$g’(x)$/0.7,$g(x)$/2}
 {$-\infty$,$m-1$,$m+1$,$m+3$,$+\infty$}
 \tkzTabLine{,-,0,+,0,-,0,+,}
 \tkzTabVar{+/$+\infty$,-/,+/,-/,+/$+\infty$}
 \end{tikzpicture}
 \end{center}
 Để hàm số $y=g(x)$ đồng biến trên khoảng $(5;6)$ cần 
 $\hoac{
 & \heva{
 & m-1\le 5 \\ 
 & m+1\ge 6} \\ 
 & m+3\le 5 \\ }\Leftrightarrow \hoac{
 & 5\le m\le 6 \\ 
 & m\le 2.}$\\
 Vì $m\in \mathbb{N}^*$ nên $m\in \big\{1;2;5;6\big\}$. Suy 
 $S=14$.
 }
\end{ex}

\begin{ex}%[2-D1B5-SO-15-2425]%[VN-MT-7, Đoàn Thị Lý]%[2D1V3-6]
 Một hộp sữa dung tích $1 \ell$ (lít) có dạng hình hộp chữ nhật với đáy là hình vuông cạnh bằng $x$ cm và chiều cao $h$ cm. Tìm giá trị của $x$ để diện tích toàn phần của hình hộp là nhỏ nhất.
 \shortans[]{10}
 \loigiai{
 \begin{center}
 \begin{tikzpicture}
 \def\a{3}
 \def\b{2}
 \def\g{30}
 \def\h{2}
 \path
 (0:0) coordinate (A)--++(\g:\b) coordinate (B)--++(0:\a) coordinate (C)--++(\g-180:\b) coordinate (D)
 \foreach \x in {A,B,C,D}{
 ($(\x)+(90:\h)$) coordinate (\x')}
 ;
 \foreach \x/\g in {A/180,B/150,C/0,D/-30,A'/180,B'/150,C'/0,D'/-30}
 \fill[black](\x) circle (1pt);
 \draw[dashed] (B')--(B)--(A)
 (B)--(C);
 \draw
 (A)--(D)--(D')--(A')--cycle
 (A')--(B')--(C')--(D')
 (D)--(C)--(C');
 \draw (1.6,0.15) node {$x$} (4,.4) node {$x$} (5,2) node {$h$}
 ;
 \end{tikzpicture} 
 \end{center}
 Thể tích của hộp sữa là $V=x^2h$ $\mathrm{cm}^3$.\\
 Theo bài ra, ta có $V=1\mathrm{\ell}=1000 \mathrm{\,cm}^3$. Suy ra $x^2h=1000\Rightarrow h=\dfrac{1000}{x^2}$.\\
 Ta có diện tích toàn phần của hộp sữa là \[S_{\rm{tp}}=S_{\rm{xq}}+S_{\rm{d}}=4hx+2x^2=4\cdot\dfrac{1000}{x^2}\cdot x+2x^2=2x^2+\dfrac{4000}{x}.\]
 Đặt $y=f(x)=2x^2+\dfrac{4000}{x}$, suy ra $f'(x)=4x-\dfrac{4000}{x^2}$.\\
 Xét $f'(x)=0\Leftrightarrow 4x-\dfrac{4000}{x^2}=0\Leftrightarrow 4x^3-4000=0\Leftrightarrow x=10$.\\
 Ta có bảng biến thiên của hàm số $y=f(x)$ như sau:
 \begin{center}
 \begin{tikzpicture}
 \tkzTabInit[nocadre=true,lgt=1.2,espcl=2.5,deltacl=0.5]
 {$x$/0.7,$f’(x)$/0.7,$f(x)$/2}
 {$0$,$10$,$+\infty$}
 \tkzTabLine{d,-,0,+,}
 \tkzTabVar{-D+//$+\infty$,-/$600$,+/$+\infty$}
 \end{tikzpicture}
 \end{center}
 Vậy để hộp sữa có diện tích toàn phần nhỏ nhất thì $x=10$.
 } 
\end{ex}
\Closesolutionfile{ans}
% \begin{indapan}
% 	{ans/ans\currfilebase}
% \end{indapan}


% \begin{name}
{Biên soạn: Dương Phước Sang \\ Phản biện: Dương Công Tạo}
{Đề ôn tập chương I}
\end{name}

\caulc
\Opensolutionfile{ans}[ans/ans\currfilebase-Phan-I]
\begin{ex}%[2-D1B5-SO-16-2425]%[VN-MT-7, Dương Phước Sang]%[2D1H1-1]
Hàm số $y=x^3-3x^2+2$ đồng biến trên khoảng nào dưới đây?
\choice
{\True $(-2;0)$}
{$(0;+\infty)$}
{$(-\infty;2)$}
{$(0;2)$}
\loigiai{
Ta có $y'=3x^2-6x$.\\
$y' \geq 0 \Leftrightarrow 3x^2-6x \geq 0 \Leftrightarrow \hoac{&x \geq 2\\&x \leq 0.}$\\
Suy ra hàm số đồng biến trên các khoảng $(-\infty;0)$ và $(2;+\infty)$ nên đồng biến trên khoảng $(-2;0)$.
}
\end{ex}

\begin{ex}%[2-D1B5-SO-16-2425]%[VN-MT-7, Dương Phước Sang]%[2D1H2-1]
Cho hàm số $y=27x^3+108x^2-81x+189$. Điểm cực tiểu của hàm số là
\choice
{$-3$}
{\True $\dfrac{1}{3}$}
{$175$}
{$675$}
\loigiai{
Ta có $y'=81x^2+216x-81$.\\
$y'=0 \Leftrightarrow \hoac{&x=\dfrac{1}{3}\\&x=-3.}$
\begin{center}
\begin{tikzpicture}
\tkzTabInit[nocadre=true,lgt=1.2,espcl=2.5,deltacl=0.6]
{$x$/0.7, $y'$/0.6, $y$/2}
{$-\infty$,$-3$,$\tfrac{1}{3}$,$+\infty$}
\tkzTabLine{,+,0,-,0,+,}
\tkzTabVar{-/,+/,-/,+/}
\end{tikzpicture}
\end{center}
Vậy điểm cực tiểu của hàm số là $x_{_\text{CT}}=\dfrac{1}{3}$.
}
\end{ex}

\begin{ex}%[2-D1B5-SO-16-2425]%[VN-MT-7, Dương Phước Sang]%[2D1H3-1]
Giá trị lớn nhất của hàm số $f(x)=x^3-8x^2+16x-9$ trên đoạn $[1;3]$ là
\choice
{$\max\limits_{[1; 3]} f(x)=0$}
{\True $\max\limits_{[1; 3]} f(x)=\dfrac{13}{27}$}
{$\max\limits_{[1; 3]} f(x)=-6$}
{$\max\limits_{[1; 3]} f(x)=5$}
\loigiai{
Hàm số $f(x)$ liên tục trên $[1;3]$.\\
Ta có $f'(x)=3x^2-16x+16$; $f'(x)=0 \Leftrightarrow \hoac{&x=4 \notin (1;3)\\&x=\dfrac{4}{3}\in (1;3).}$\\
$f(1)=0$; $f\left( \dfrac{4}{3} \right)=\dfrac{13}{27}$; $f(3)=-6$.\\
Do đó $\max\limits_{x \in [1;3]} f(x)=f\left( \dfrac{4}{3} \right)=\dfrac{13}{27}$.
}
\end{ex}

\begin{ex}%[2-D1B5-SO-16-2425]%[VN-MT-7, Dương Phước Sang]%[2D1H3-1]
Giá trị lớn nhất của hàm số $f(x)=x^4-4x^2+1$ trên đoạn $[1;3]$ bằng
\choice
{\True $46$}
{$64$}
{$3$}
{$\sqrt{2}$}
\loigiai{
Ta có hàm số $f(x)=x^4-4x^2+1$ liên tục trên đoạn $[1;3]$.\\
$f'(x)=4x^3-8x$.\\
$f'(x)=0 \Leftrightarrow 4x^3-8x=0 \Leftrightarrow \hoac{&x=0 \notin (1;3)\\&x=\sqrt{2} \in (1;3)\\&x=-\sqrt{2} \notin (1;3).}$\\
$f(1)=-2$; $f\left( \sqrt{2} \right)=-3$; $f(3)=46$.\\
Vậy giá trị lớn nhất của hàm đã cho trên đoạn $[1;3]$ bằng $46$.
}
\end{ex}

\begin{ex}%[2-D1B5-SO-16-2425]%[VN-MT-7, Dương Phước Sang]%[2D1N4-1]
Tiệm cận ngang của đồ thị hàm số $y=\dfrac{2x+3}{x-1}$ là
\choice
{$y=1$}
{\True $y=2$}
{$x=1$}
{$x=2$}
\loigiai{
Tập xác định của hàm số là $\mathscr{D}=\mathbb{R} \setminus \{1\}$.\\
Ta có 
$\lim\limits_{x \to \pm\infty} y=\lim\limits_{x \to \pm\infty} \dfrac{2x+3}{x-1}=\lim\limits_{x \to \pm\infty} \dfrac{2+\dfrac{3}{x}}{1-\dfrac{1}{x}}=2$.\\
Vậy đồ thị của hàm số có tiệm cận ngang là đường thẳng $y=2$.
}
\end{ex}

\begin{ex}%[2-D1B5-SO-16-2425]%[VN-MT-7, Dương Phước Sang]%[2D1N4-1]
Tiệm cận đứng của đồ thị hàm số $y=\dfrac{2x+1}{x-1}$ là
\choice
{\True $x=1$}
{$y=2$}
{$x=2$}
{$x=-1$}
\loigiai{
Tập xác định của hàm số $\mathscr{D}=\mathbb{R} \setminus \{1\}$.\\
Ta có 
$\lim\limits_{x \to 1^+} y=\lim\limits_{x \to 1^+} \dfrac{2x+1}{x-1}=\lim\limits_{x \to 1^+} \left(2+\dfrac{3}{x-1}\right)=+\infty$.\\
Vậy đồ thị của hàm số có tiệm cận đứng là đường thẳng $x=1$.
}
\end{ex}

\begin{ex}%[2-D1B5-SO-16-2425]%[VN-MT-7, Dương Phước Sang]%[2D1H4-1]
Đường thẳng nào sau đây là tiệm cận xiên của đồ thị hàm số $y=\dfrac{2x^2-3x+1}{x+2}$?
\choice
{$y=2x$}
{$y=2$}
{\True $y=2x-7$}
{$x=-2$}
\loigiai{
Ta có $y=f(x)=\dfrac{2x^2-3x+1}{x+2}=2x-7+\dfrac{15}{x+2}$.\\
$\lim\limits_{x \to \pm\infty} [f(x)-(2x-7)]=\lim\limits_{x \to \pm\infty} \dfrac{15}{x+2}=0$.\\
Vậy đồ thị hàm số có tiệm cận xiên là $y=2x-7$.
}
\end{ex}

\begin{ex}%[2-D1B5-SO-16-2425]%[VN-MT-7, Dương Phước Sang]%[2D1N1-2]
Cho hàm số $y=f(x)$ có bảng biến thiên như sau:
\begin{center}
\begin{tikzpicture}
\tkzTabInit[nocadre=true,lgt=1.2,espcl=2.5,deltacl=0.6]
{$x$/0.6, $y'$/0.6, $y$/2}
{$-\infty$,$-2$,$0$,$+\infty$}
\tkzTabLine{,+,0,-,0,+,}
\tkzTabVar{-/$-\infty$,+/$1$,-/$-3$,+/$+\infty$}
\end{tikzpicture} 
\end{center}
Hàm số $y=f(x)$ nghịch biến trên khoảng nào dưới đây?
\choice
{$(-\infty;-2)$}
{$(0;+\infty)$}
{$(-3;1)$}
{\True $(-2;0)$}
\loigiai{
Dựa vào bảng biến thiên, ta thấy $f'(x)<0 \Leftrightarrow x \in (-2;0)$ nên hàm số nghịch biến trên khoảng $(-2;0)$.
}
\end{ex}

\begin{ex}%[2-D1B5-SO-16-2425]%[VN-MT-7, Dương Phước Sang]%[2D1H5-1]
Cho bảng biến thiên của hàm số $y=f(x)$ như sau:
\begin{center}
\begin{tikzpicture}
\tikzset{double style/.append style={double distance=1.5pt}}
\tkzTabInit[nocadre=true,lgt=1.2,espcl=4,deltacl=0.6]
{$x$/0.6,$y'$/0.6,$y$/2}
{$-\infty$,$1$,$+\infty$}
\tkzTabLine{,-,d,-,}
\tkzTabVar{+/$1$,-D+/$-\infty$/$+\infty$,-/$1$}
\end{tikzpicture} 
\end{center}
Hỏi đây là bảng biến thiên của hàm số nào trong các hàm số sau?
\choice
{$y=\dfrac{x-3}{x-1}$}
{$y=\dfrac{-x+2}{x-1}$}
{$y=\dfrac{x+2}{x+1}$}
{\True $y=\dfrac{x+2}{x-1}$}
\loigiai{
Bảng biến thiên được cung cấp có đặc điểm:
\begin{itemize}
\item Đồ thị hàm số có đường tiệm cận đứng là $x=1$, loại $y=\dfrac{x+2}{x+1}$.
\item Đồ thị hàm số có đường tiệm cận ngang là $y=1$, loại $y=\dfrac{-x+2}{x-1}$.
\item $y'<0,\,\forall x \ne 1$, trong khi $\left(\dfrac{x-3}{x-1}\right)'=\dfrac{2}{(x-1)^2}>0,\,\forall x \neq 1$, loại $y=\dfrac{x-3}{x-1}$.
\end{itemize}
Chỉ có hàm số $y=\dfrac{x+2}{x-1}$ thỏa mãn các đặc điểm trên.
}
\end{ex}

\begin{ex}%[2-D1B5-SO-16-2425]%[VN-MT-7, Dương Phước Sang]%[2D1H1-1]
Cho hàm số $y=\dfrac{x^2-2x}{1-x}$. Khẳng định nào sau đây đúng?
\choice
{Hàm số đồng biến trên $\mathbb{R}$}
{\True Hàm số nghịch biến trên các khoảng $(-\infty;1)$ và $(1;+\infty)$}
{Hàm số nghịch biến trên $\mathbb{R}$}
{Hàm số đồng biến trên các khoảng $(-\infty;1)$ và $(1;+\infty)$}
\loigiai{
Tập xác định $\mathscr{D}=\mathbb{R} \setminus \big\{1\big\}$.\\
$y'=\dfrac{-x^2+2x-2}{(1-x)^2}=\dfrac{-(x-1)^2-1}{(1-x)^2}<0,\,\forall x \in \mathscr{D}$.\\
Vậy hàm số nghịch biến trên các khoảng $(-\infty;1)$ và $(1;+\infty)$.
}
\end{ex}

\begin{ex}%[2-D1B5-SO-16-2425]%[VN-MT-7, Dương Phước Sang]%[1D7H2-8]
Cho chuyển động được xác định bởi phương trình $s(t)=3t^3+4t^2-t$, trong đó $t$ được tính bằng giây (s) và $s(t)$ được tính bằng mét. Vận tốc của chuyển động khi $t=4$\,s bằng
\choice
{\True $175$ m/s}
{$41$ m/s}
{$176$ m/s}
{$20$ m/s}
\loigiai{
Ta có $v(t)=s'(t)=9t^2+8t-1$.\\
Vận tốc của chuyển động khi $t=4$\,s bằng $v(4)=9 \cdot 4^2+8 \cdot 4-1=175$ m/s.
}
\end{ex}

\begin{ex}%[2-D1B5-SO-16-2425]%[VN-MT-7, Dương Phước Sang]%[2D1H2-1]
Cho hàm số $y=f(x)$ có đạo hàm $f'(x)=x^2(x-1)$ với mọi số thực $x$. Số điểm cực tiểu của hàm số $f(x)$ là
\choice
{$0$}
{\True $1$}
{$2$}
{$3$}
\loigiai{
Ta có $f'(x)=x^2(x-1)=0 \Leftrightarrow \hoac{&x=0&&\text{(nghiệm kép)}\\&x=1.}$\\
Bảng biến thiên:
\begin{center}
\begin{tikzpicture}
\tkzTabInit[nocadre=true,lgt=1.2,espcl=2.5,deltacl=0.6]
{$x$/0.6, $f'(x)$/0.6, $f(x)$/2}
{$-\infty$,$0$,$1$,$+\infty$}
\tkzTabLine{,-,0,-,0,+,}
\tkzTabVar{+/,R,-/,+/}
\end{tikzpicture} 
\end{center}
Từ bảng biến thiên ta thấy hàm số có một điểm cực tiểu duy nhất là $x=1$.
}
\end{ex}
\Closesolutionfile{ans}

\cauds
\Opensolutionfile{ans}[ans/ans\currfilebase-Phan-II]
\begin{ex}%[2-D1B5-SO-16-2425]%[VN-MT-7, Dương Phước Sang]%[2D1V2-1]
Cho các hàm số $f(x)=x^3-3x^2+2025$ và $g(x)=\dfrac{x^2-2x+1}{x-2}$.
\choiceTF
{\True Hàm số $y=f(x)$ nghịch biến trên khoảng $(0;2)$}
{Hàm số $y=g(x)$ nghịch biến trên khoảng $(1;3)$}
{\True Điểm cực đại của hàm số $y=f(x)$ là $x=0$}
{\True Đường thẳng đi qua $2$ điểm cực trị của đồ thị hàm số $y=g(x)$ cũng đi qua điểm $N(2;2)$}
\loigiai{
\begin{itemchoice}
\itemch \textbf{Đúng}.\\
Với $f(x)=x^3-3x^2+2025$, ta có $f'(x)=3x^2-6x$ và $f'(x)=0 \Leftrightarrow \hoac{&x=0\\&x=2.}$\\
Từ đó, ta có bảng xét dấu của $f'(x)$ như sau:
\begin{center}
\begin{tikzpicture}
\tkzTabInit[nocadre=true,lgt=1.2,espcl=2,deltacl=0.6]
{$x$/0.6,$f'(x)$/0.6}
{$-\infty$,$0$,$2$,$+\infty$}
\tkzTabLine{,+,0,-,0,+,}
\end{tikzpicture}
\end{center}
Suy ra hàm số nghịch biến trên khoảng $(0;2)$.
\itemch \textbf{Sai}.\\
Hàm số $y=g(x)=\dfrac{x^2-2x+1}{x-2}$ có tập xác định: $\mathscr{D}=\mathbb{R} \setminus \big\{2\big\}$.\\
Ta có $y'=\dfrac{x^2-4x+3}{(x-2)^2}$ và $y'=0 \Leftrightarrow x^2-4x+3=0 \Leftrightarrow \hoac{&x=1\\&x=3.}$\\
Bảng xét dấu của $g'(x)$:
\begin{center}
\begin{tikzpicture}
\tikzset{double style/.append style={double distance=1.5pt}}
\tkzTabInit[nocadre=true,lgt=1.2,espcl=2.5,deltacl=0.6]
{$x$/0.6,$g'(x)$/0.6}
{$-\infty$,$1$,$2$,$3$,$+\infty$}
\tkzTabLine{,+,0,-,d,-,0,+,}
\end{tikzpicture}
\end{center}
Suy ra hàm số nghịch biến trên khoảng $(1;2)$ và $(2;3)$.
\itemch \textbf{Đúng}.\\
Với $f(x)=x^3-3x^2+2025$, ta có $f'(x)=3x^2-6x$ và $f'(x)=0 \Leftrightarrow \hoac{&x=0\\&x=2.}$\\
Bảng biến thiên của hàm số $y=f(x)$:
\begin{center}
\begin{tikzpicture}
\tkzTabInit[nocadre=true,lgt=1.2,espcl=2.5,deltacl=0.6]
{$x$/0.6, $f'(x)$/0.6, $f(x)$/2}
{$-\infty$,$0$,$2$,$+\infty$}
\tkzTabLine{,+,0,-,0,+,}
\tkzTabVar{-/,+/,-/,+/}
\end{tikzpicture}
\end{center}
Suy ra điểm cực đại của hàm số là $x=0$.
\itemch \textbf{Đúng}.\\
Hàm số $y=g(x)=\dfrac{x^2-2x+1}{x-2}$ có tập xác định: $\mathscr{D}=\mathbb{R} \setminus \big\{2\big\}$.\\
Ta có $y'=\dfrac{x^2-4x+3}{(x-2)^2}$ và $y'=0 \Leftrightarrow x^2-4x+3=0 \Leftrightarrow \hoac{&x=1\\&x=3.}$\\
Bảng biến thiên của $g(x)$:
\begin{center}
\begin{tikzpicture}
\tikzset{double style/.append style={double distance=1.5pt}}
\tkzTabInit[nocadre=true,lgt=1.2,espcl=2.5,deltacl=0.6]
{$x$/0.6,$y'$/0.6,$y$/2}
{$-\infty$,$1$,$2$,$3$,$+\infty$}
\tkzTabLine{,+,0,-,d,-,0,+,}
\tkzTabVar{-/$-\infty$,+/$0$,-D+/$-\infty$/$+\infty$,-/$4$,+/$+\infty$}
\end{tikzpicture}
\end{center}
Hai điểm cực trị của đồ thị hàm số $y=g(x)$ là $A(1;0)$ và $B(3;4)$, cùng thuộc $AB\colon y=2x-2$.\\
Đường thẳng $AB$ đó đi qua điểm $N(2;2)$.
\end{itemchoice}
}
\end{ex}

\begin{ex}%[2-D1B5-SO-16-2425]%[VN-MT-7, Dương Phước Sang]%[2D1V3-1]
Cho các hàm số $f(x)=x^3-8x^2+16x-9$ và $h(x)=\dfrac{x^2-x+1}{x-1}$.
\choiceTF
{\True Giá trị lớn nhất của hàm số $y=f(x)$ trên đoạn $[-1;1]$ là $0$}
{Gọi giá trị lớn nhất và giá trị nhỏ nhất của hàm số $y=f(x)$ trên đoạn $[1;3]$ lần lượt là $a$, $b$. Khi đó giá trị của $27a-b$ bằng $13$}
{\True Giá trị nhỏ nhất của hàm số $y=h(x)$ trên khoảng $(1;+\infty)$ là $3$}
{\True Giá trị nhỏ nhất của hàm số $y=f\big(h(x)\big)$ trên khoảng $(1;3)$ là $-9$}
\loigiai{
\begin{itemchoice}
\itemch \textbf{Đúng}.\\
Với $f(x)=x^3-8x^2+16x-9$, ta có $f'(x)=3x^2-16x+16$.\\
Cho $f'(x)=0 \Leftrightarrow 3x^2-16x+16=0 \Leftrightarrow \hoac{&x=4 \notin [-1;1]\\&x=\dfrac{4}{3} \notin [-1;1].}$\\
Do $\heva{&f(-1)=-34\\&f(1)=0}$ nên $\max\limits_{x \in [-1;1]} f(x) =0$.
\itemch \textbf{Sai}.\\
Với $f(x)=x^3-8x^2+16x-9$, ta có $f'(x)=3x^2-16x+16$.\\
Cho $f'(x)=0 \Leftrightarrow 3x^2-16x+16=0 \Leftrightarrow \hoac{&x=4 \notin [1;3]\\&x=\dfrac{4}{3} \in [1;3].}$\\
Vì $\heva{&f(1)=0\\&f(\dfrac{4}{3})=\dfrac{13}{27}\\&f(3)=-6}$ nên $\heva{&\max\limits_{x \in [1;3]} f(x) =\dfrac{13}{27}=a\\&\min\limits_{x \in [1;3]} f(x) =-6=b}$. Từ đó $27a-b=19$.
\itemch \textbf{Đúng}.\\
Với $h(x)=\dfrac{x^2-x+1}{x-1}$, ta có $h'(x)=\dfrac{x^2-2x}{(x-1)^2}$.\\
Cho $h'(x)=0 \Rightarrow x^2-2x=0 \Leftrightarrow \hoac{&x=0 \notin (1;+\infty)\\&x=2 \in (1;+\infty).}$
\begin{center}
\begin{tikzpicture}
\tkzTabInit[nocadre=true,lgt=1.2,espcl=2.5,deltacl=0.6]
{$x$/0.6,$h'(x)$/0.6,$h(x)$/2}
{$1$,$2$,$+\infty$}
\tkzTabLine{,-,0,+,}
\tkzTabVar{+/$+\infty$,-/$3$,+/$+\infty$}
\end{tikzpicture}
\end{center}
Từ bảng biến thiên, suy ra giá trị nhỏ nhất của hàm số $y=h(x)$ trên khoảng $(1;+\infty)$ là $3$.
\itemch \textbf{Đúng}.\\
Với $t=h(x)=\dfrac{x^2-x+1}{x-1}$, ta có $h'(x)=\dfrac{x^2-2x}{(x-1)^2}$.\\
Cho $h'(x)=0 \Rightarrow x^2-2x=0 \Leftrightarrow \hoac{&x=0 \notin (1;3)\\&x=2 \in (1;3).}$
\begin{center}
 \begin{tikzpicture}
 \tkzTabInit[nocadre=true,lgt=1.2,espcl=2.5,deltacl=0.6]
 {$x$/0.6,$h'(x)$/0.6,$h(x)$/2}
 {$1$,$2$,$3$}
 \tkzTabLine{,-,0,+,}
 \tkzTabVar{+/$+\infty$,-/$3$,+/$\tfrac{7}{2}$}
 \end{tikzpicture}
\end{center}
Như thế đặt $t=h(x)$, $x\in (1;3)$ thì $t \in [3;+\infty)$ và $y=f(t)=t^3-8t^2+16t-9$.\\
Ta có $f'(t)=3t^2-16t+16$.\\
Cho $f'(t)=0 \Leftrightarrow 3t^2-16t+16=0 \Leftrightarrow \hoac{&t=4 \in [3;+\infty)\\&x=\dfrac{4}{3} \notin [3;+\infty).}$
\begin{center}
 \begin{tikzpicture}
 \tkzTabInit[nocadre=true,lgt=1.2,espcl=2.5,deltacl=0.6]
 {$t$/0.6,$f'(t)$/0.6,$f(t)$/2}
 {$3$,$4$,$+\infty$}
 \tkzTabLine{,-,0,+,}
 \tkzTabVar{+/$-6$,-/$-9$,+/$+\infty$}
 \end{tikzpicture}
\end{center}
Vậy $\min\limits_{1<x<3} f\big(h(x)\big)=\min\limits_{t \geq 3} f(t)=f(4)=-9$.
\end{itemchoice}
}
\end{ex}

\begin{ex}%[2-D1B5-SO-16-2425]%[VN-MT-7, Dương Phước Sang]%[2D1V4-1]
Cho các hàm số $f(x)=\dfrac{x-2}{x+3}$ và $g(x)=\dfrac{x^2-3x}{x+1}$. 
\choiceTF
{\True Đồ thị hàm số $y=f(x)$ có đường tiệm cận ngang là đường thẳng $y=1$}
{\True Đồ thị hàm số $y=g(x)$ có đường tiệm cận đứng là đường thẳng $x=-1$}
{\True Đồ thị hàm số $y=g(x)$ có đường tiệm cận xiên là đường thẳng $y=x-4$}
{\True Đồ thị hàm số $y=g\big(f(x)\big)$ không có đường tiệm cận xiên nào cả}
\loigiai{
\begin{itemchoice}
\itemch \textbf{Đúng}.\\
Ta có $\lim\limits_{x \to \pm\infty} \dfrac{x-2}{x+3}=\lim\limits_{x \to \pm\infty} \dfrac{1-\dfrac{2}{x}}{1+\dfrac{3}{x}}=1$ nên đồ thị hàm số $y=\dfrac{x-2}{x+3}$ có đường tiệm cận ngang là đường thẳng $y=1$.
\itemch \textbf{Đúng}.\\
Ta có $\lim\limits_{x \to -1^+} \dfrac{x^2-3x}{x+1}=\lim\limits_{x \to -1^+} \left(x-4+\dfrac{4}{x+1}\right)=+\infty$ nên đồ thị hàm số $y=\dfrac{x^2-3x}{x+1}$ có đường tiệm cận đứng là đường thẳng $x=-1$.
\itemch \textbf{Đúng}.\\
Ta có $y=\dfrac{x^2-3x}{x+1}=x-4+\dfrac{4}{x+1}$ nên $\lim\limits_{x \to \pm\infty} \left(\dfrac{x^2-3x}{x+1}-(x-4)\right)=\lim\limits_{x \to \pm\infty} \dfrac{4}{x+1}=0$.\\
Vậy đường thẳng $y=x-4$ là tiệm cận xiên của đồ thị hàm số $y=\dfrac{x^2-3x}{x+1}$.
\itemch \textbf{Đúng}.\\
Ta có $y=g\big(f(x)\big)=\dfrac{\big(f(x)\big)^2-3\big(f(x)\big)}{f(x)+1}$ nên $\lim\limits_{x \to \pm\infty} g\big(f(x)\big)=\lim\limits_{x \to \pm\infty} \dfrac{\big(f(x)\big)^2-3\big(f(x)\big)}{f(x)+1}$.\\
Mà $\lim\limits_{x \to \pm\infty} f(x)=\lim\limits_{x \to \pm\infty} \dfrac{x-2}{x+3}=\lim\limits_{x \to \pm\infty} \dfrac{1-\dfrac{2}{x}}{1+\dfrac{3}{x}}=1$ nên
$\lim\limits_{x \to \pm\infty} g\big(f(x)\big)=\dfrac{1^2-3\cdot 1}{1+1}=-1$.\\
Vậy $(C)\colon y=g\big(f(x)\big)$ có tiệm cận ngang $y=-1$ mà không có tiệm cận xiên.
\end{itemchoice}
}
\end{ex}

\begin{ex}%[2D1C4-3]
Cho hàm số $y=f(x)$ xác định trên tập $\mathscr{D}=\mathbb{R} \setminus \{2\}$, có bảng biến thiên như sau:
\begin{center}
\begin{tikzpicture}
\tikzset{double style/.append style={double distance=1.5pt}}
\tkzTabInit[nocadre=true,lgt=1.2,espcl=2]
{$x$/0.6,$f'(x)$/0.6,$f(x)$/2}
{$-\infty$,$1$,$2$,$3$,$+\infty$}
\tkzTabLine{,+,0,-,d,-,0,+,}
\tkzTabVar{-/$-\infty$,+/$1$,-D+/$-\infty$/$+\infty$,-/$5$,+/$+\infty$}
\end{tikzpicture}
\end{center}
\choiceTF
{\True Hàm số $y=f(x)$ có cực đại nhỏ hơn cực tiểu}
{\True Hàm số $f(x)=\dfrac{x^2-x-1}{x-2}$ có bảng biến thiên như trên}
{\True Đồ thị hàm số $y=f(x)$ luôn có đúng $1$ tiệm cận đứng}
{Đồ thị hàm số $y=f(x)$ luôn có $1$ hoặc $2$ tiệm cận xiên}
\loigiai{
\begin{itemchoice}
\itemch \textbf{Đúng}.\\
Hàm số $y=f(x)$ có cực đại bằng $1$ và cực tiểu bằng $5$ nên cực đại nhỏ hơn cực tiểu.\itemch \textbf{Đúng}.\\
Xét hàm số $y=\dfrac{x^2-x-1}{x-2}$ có tập xác định $\mathscr{D}=\mathbb{R} \setminus \big\{2\big\}$.\\
Ta có $y'=\dfrac{x^2-4x+3}{(x-2)^2}$ và
$y'=0 \Leftrightarrow x^2-4x+3=0 \Leftrightarrow \hoac{&x=1\\&x=3.}$\\
Bảng biến thiên của hàm số $y=\dfrac{x^2-x-1}{x-2}$ đúng như bảng biến thiên được cung cấp.
\begin{center}
\begin{tikzpicture}
\tikzset{double style/.append style={double distance=1.5pt}}
\tkzTabInit[nocadre=true,lgt=1.2,espcl=2,deltacl=0.6]
{$x$/0.6,$y'$/0.6,$y$/1.8}
{$-\infty$,$1$,$2$,$3$,$+\infty$}
\tkzTabLine{,+,0,-,d,-,0,+,}
\tkzTabVar{-/$-\infty$,+/$1$,-D+/$-\infty$/$+\infty$,-/$5$,+/$+\infty$}
\end{tikzpicture}
\end{center}
\itemch \textbf{Đúng}.\\
Tại mọi $x_0 \neq 2$, bảng biến thiên hàm số thể hiện $\lim\limits_{x \to x_0} f(x)=f(x_0)$ nên $x=x_0$ không là tiệm cận của đồ thị hàm số.\\
Và chỉ có $\lim\limits_{x \to 2^-} f(x)=-\infty$ nên chỉ có $x=2$ là tiệm cận đứng duy nhất của đồ thị hàm số.
\itemch \textbf{Sai}.\\
Không có đủ cơ sở nào để khẳng định được hàm số $y=f(x)$ có $1$ tiệm cận xiên.\\
Ít nhất có hàm số $f(x)=\dfrac{(x-2)^6-5(x-2)^4+15(x-2)^2+24(x-2)+5}{8(x-2)}$
\[\text{Có }
\begin{aligned}[t]
f'(x)&=\dfrac{1}{8}\left[5(x-2)^4-15(x-2)^2+15-\dfrac{5}{(x-2)^2}\right]\\
&=\dfrac{5}{8}\cdot\dfrac{\left(x^2-4x+3\right)^3}{(x-2)^2}.
\end{aligned}\]
Bảng biến thiên của hàm số $f(x)$ này đúng như bảng biến thiên được cung cấp nhưng đồ thị hàm số không hề có tiệm cận xiên do $\lim\limits_{x \to \pm\infty} \dfrac{f(x)}{x}=+\infty$.
\end{itemchoice}
}
\end{ex}
\Closesolutionfile{ans}

\caukq
\Opensolutionfile{ans}[ans/ans\currfilebase-Phan-III]
\begin{ex}%[2-D1B5-SO-16-2425]%[VN-MT-7, Dương Phước Sang]%[2D1C5-5]
\immini[thm]{
Cho hàm số $y=f(x)$ có đạo hàm trên $\mathbb{R}$, thỏa mãn $f(-1)=f(3)=0$ và đồ thị của hàm số 
$y=f'(x)$ có dạng như hình bên đây. Có tất cả bao nhiêu cặp số nguyên $\big\{a;b\}$ thuộc đoạn $[-10;10]$ để hàm số $y=\big[f(x)\big]^2$ nghịch biến trên khoảng $(a;b)$?
\shortans{48}}
{\begin{tikzpicture}[xscale=0.6, yscale=0.3, font=\footnotesize, samples=200, >=stealth]
\draw[->] (-2,0)--(4.1,0) node[below]{$x$};
\draw[->] (0,-4.6)--(0,4.6) node[left]{$y$};
\node at (0,0) [below left=-2pt]{$O$};
\draw plot[domain=-1.42:3.42] (\x,{-(\x)^3+3*(\x)^2+\x-3});
\fill[yscale=2] 
(-1,0) circle(1.0pt) node[below,xshift=-7]{$-1$}
(1,0) circle(1.0pt) node[below,xshift=2]{$1$}
(3,0) circle(1.0pt) node[below,xshift=-4]{$3$};
\end{tikzpicture}}
\loigiai{
Từ đồ thị và giả thiết, ta có bảng biến thiên của hàm số $y=f(x)$ như sau:
\begin{center}
\begin{tikzpicture}
\tkzTabInit[nocadre=true,lgt=1.2,espcl=2.5,deltacl=0.6]
{$x$/0.6,$y'$/0.6,$y$/2}
{$-\infty$,$-1$,$1$,$3$,$+\infty$}
\tkzTabLine{,+,0,-,0,+,0,-,}
\tkzTabVar{-/,+/$0$,-/,+/$0$,-/}
\end{tikzpicture}
\end{center}
Như thế $f(x) \leq 0$ với mọi $x \in \mathbb{R}$.\\
Với $y=\big[f(x)\big]^2$, ta có $y'=\left[\big(f(x)\big)^2\right]'=2f(x) \cdot f'(x)$.\\
Bảng xét dấu của $y'=\left[\big(f(x)\big)^2\right]'$:
\begin{center}
\begin{tikzpicture}
\tkzTabInit[nocadre=true,lgt=2.3,espcl=2,deltacl=0.6]
{$x$ /0.6, $f'(x)$ /0.6, $f(x)$/0.6, $\left[\big(f(x)\big)^2\right]'$ /0.9}
{$-\infty$,$-1$,$1$,$3$,$+\infty$}
\tkzTabLine{,+,0,-,0,+,0,-,}
\tkzTabLine{,-,0,-,t,-,0,-,}
\tkzTabLine{,-,0,+,0,-,0,+,}
\end{tikzpicture}
\end{center}
Như vậy hàm số $y=\big[f(x)\big]^2$ nghịch biến trên các khoảng $(-\infty;-1)$ và $(1;3)$.\\
Từ đó, số cặp số nguyên $\{a;b\}$ là số cách chọn $2$ từ $3$ số $\{1;2;3\}$ hoặc từ $10$ số $\{-10;-9;\ldots;-1\}$.\\
Số cặp số $\{a;b\}$ là $\mathrm{C}_3^2+\mathrm{C}_{10}^2=48$.
}
\end{ex}

\begin{ex}%[2-D1B5-SO-16-2425]%[VN-MT-7, Dương Phước Sang]%[2D1V5-4]
Đồ thị hàm số $y=x^3-3x^2-9x+5$ có điểm cực đại và điểm cực tiểu lần lượt là $A$ và $B$. 
Gọi $I$ là giao điểm của $AB$ với trục $Ox$. Đặt tỷ số $\dfrac{IA}{IB}=\dfrac{b}{c}$ tối giản ($b,c \in \mathbb{N}$). Tính $T=b+c$.
\par\shortans{16}
\loigiai{
Hàm số $y=x^3-3x^2-9x+5$ có tập xác định $\mathscr{D}=\mathbb{R}$.\\
$y'=3x^2-6x-9$. Cho $y'=0 \Leftrightarrow 3x^2-6x-9=0 \Leftrightarrow \hoac{&x=-1\\&x=3.}$\\
Với $x=-1$ ta có $y=y(-1)=10$. Đặt $A(-1;10)$.\\
Với $x=3$ ta có $y=y(3)=-22$. Đặt $B(3;-22)$.\\
Vì $AB$ cắt $Ox$ tại $I$ nên $\dfrac{IA}{IB}=\dfrac{\mathrm{d}(A,Ox)}{\mathrm{d}(B,Ox)}=\dfrac{\left|y_A\right|}{\left|y_B\right|}=\dfrac{10}{22}=\dfrac{5}{11}$.\\
Như vậy $b=5$ và $c=11$ nên $T=b+c=16$.
}
\end{ex}

\begin{ex}%[2-D1B5-SO-16-2425]%[VN-MT-7, Dương Phước Sang]%[2D1V3-1]
Gọi $M$, $m$ lần lượt là giá trị lớn nhất và giá trị nhỏ nhất của hàm số $y=\dfrac{3\sin x+2}{\sin x+1}$ trên đoạn $\left[ 0;\dfrac{\pi}{2} \right]$. Xác định giá trị làm tròn đến hàng phần mười của biểu thức $M^2+m^2$.
\shortans{10{,}3}
\loigiai{
Đặt $t=\sin x$, ta có $x \in \left[ 0;\dfrac{\pi}{2} \right]$ nên $t \in [0;1]$.\\
Xét hàm $f(t)=\dfrac{3t+2}{t+1}$ trên đoạn $[0;1]$ có $f'(t)=\dfrac{1}{(t+1)^2}>0,\, \forall t \in [0;1]$.\\
Suy ra hàm số $f(t)$ đồng biến trên $[0;1]$.\\
Từ đó ta có
$M=\max\limits_{[0;1]} f(t)=f(1)=\dfrac{5}{2}$
và 
$m=\min\limits_{[0;1]} f(t)=f(0)=2$.\\
Khi đó, $M^2+m^2=\left( \dfrac{5}{2} \right)^2+2^2=\dfrac{41}{4}=10{,}25 \approx 10{,}3$.
}
\end{ex}

\begin{ex}%[2-D1B5-SO-16-2425]%[VN-MT-7, Dương Phước Sang]%[2D1V3-6]
Vận tốc của một tàu con thoi từ lúc cất cánh tại thời điểm $t=0$\,s cho đến thời điểm $t=126$\,s được cho bởi công thức $v(t)=0{,}001302t^3-0{,}09029t^2+83$ (vận tốc được tính bằng đơn vị ft/s). Gọi $v_{\min}$ là vận tốc nhỏ nhất của tàu con thoi. Xác định kết quả làm tròn đến hàng phần mười của $v_{\min}$.
\shortans{18{,}7}
\loigiai{
Hàm số $v(t)=0{,}001302t^3-0{,}09029t^2+83$ liên tục trên đoạn $[0;126]$.\\
Ta có $v'(t)=0{,}003906t^2-0{,}18058t$.\\
Cho $v'(t)=0 \Leftrightarrow 0{,}003906t^2-0{,}18058t=0 \Leftrightarrow \hoac{&t=0\\&t=\dfrac{0{,}18058}{0{,}003906}.}$\\
Trên đoạn $[0;126]$, ta có 
$v(0)=83$; $v\left(\dfrac{0{,}18058}{0{,}003906}\right) \approx 18{,}67301185$; $v(126) \approx 1254{,}045512$.\\
Tàu con thoi đạt vận tốc nhỏ nhất bằng $v\left(\dfrac{0{,}18058}{0{,}003906}\right) \approx 18{,}7$ ft/s.
}
\end{ex}

\begin{ex}%[2-D1B5-SO-16-2425]%[VN-MT-7, Dương Phước Sang]%[2D1H4-1]
Một mảnh vườn hình chữ nhật có diện tích bằng $900$\,m$^2$. Biết chiều dài của mảnh vườn là $x$\,(m). Gọi biểu thức tính chu vi của mảnh vườn là $P(x)$\,(m). Biết rằng phương trình tiệm cận xiên của đồ thị hàm số $P(x)$ là $y=ax+b$. Tính giá trị biểu thức $T=10^a+b$.
\shortans{100}
\loigiai{
Mảnh vườn có chiều dài $x$\,(m) nên có chiều rộng là $\dfrac{900}{x}$\,(m).\\
Điều kiện: $x \geq \dfrac{900}{x} \Leftrightarrow x \geq 30$.\\
Ta có $P(x)=2\left( x+\dfrac{900}{x} \right)=2x+\dfrac{1800}{x}$.\\
Vì $\lim\limits_{x \to +\infty} [P(x)-2x]=\lim\limits_{x \to +\infty} \dfrac{1800}{x}=0$ nên đồ thị hàm số $P(x)$ có tiệm cận xiên là đường thẳng $y=2x$.\\
Suy ra $a=2$, $b=0$. Do vậy, $T=100$.
}
\end{ex}

\begin{ex}%[2-D1B5-SO-16-2425]%[VN-MT-7, Dương Phước Sang]%[2D1V5-4]
Biết rằng đồ thị hàm số $y=\dfrac{x+1}{x-1}$ cắt đường thẳng $y=2x-1$ tại hai điểm phân biệt $A$, $B$. Tính diện tích tam giác $OAB$.
\shortans{1}
\loigiai{
Phương trình hoành độ giao điểm của đồ thị hai hàm số $y=\dfrac{x+1}{x-1}$ và $y=2x-1$ là
\[\dfrac{x+1}{x-1}=2x-1 \Leftrightarrow 2x^2-4x=0 \Leftrightarrow \hoac{&x=0\\&x=2.}\]
Suy ra toạ độ các giao điểm của đồ thị hai hàm số đó là $A(0;-1)$, $B(2;3)$.\\
Diện tích cần tìm là 
$S=\dfrac{1}{2}|0 \cdot 3-(-1) \cdot 2|=1$.
}
\end{ex}
\Closesolutionfile{ans}
\begin{indapan}
	{ans/ans\currfilebase}
\end{indapan}


% \setcounter{deso}{4}
\begin{name}
	{\tenchude}
	{ĐỀ ÔN TẬP CHƯƠNG I}
	{LỚP TOÁN THẦY PHÁT}
	{\thoigian}
\end{name}

\TN
\Opensolutionfile{ans}[ans/ansc1l4-Phan-I]
\begin{ex}%[2-D1B5-SO-17-2425]%[VN-MT-7, VM031]%[2D1N5-1]
 \immini{Đường cong cho trong hình bên là đồ thị của hàm số nào trong các hàm số dưới đây?
 \choice[2]
 {$y=-x^3+2x-1$}
 {$y=-x^3+3x+1$}
 {$y=2x^3-6x+1$}
 {\True $y=x^3-3x+1$}}
 {\begin{tikzpicture}[font=\footnotesize,line join=round, line cap=round, >=stealth,scale=0.6] 
 \def \xmin{-3}\def \xmax{3}\def \ymin{-2}\def \ymax{4} 
 \draw[->] (\xmin,0)--(\xmax,0) node[shift=(-110:0.2)] {$x$};
 \draw[->] (0,\ymin)--(0,\ymax) node[shift=(-150:0.2)] {$y$};
 \fill (0,0) circle(1pt) node[shift=(135:0.25)]{$O$}
 (-1,0) circle(1pt) node[shift=(-90:0.2)]{$-1$}
 (1,0) circle(1pt) node[shift=(90:0.2)]{$1$}
 (0,-1) circle(1pt) node[shift=(-150:0.28)]{$-1$}
 (0,1) circle(1pt) node[shift=(0:0.2)]{$1$}
 (0,3) circle(1pt) node[shift=(0:0.2)]{$3$}
 (-1,3) circle(1pt) (1,-1) circle(1pt); 
 \draw[dashed] (-1,0)|-(0,3) (1,0)|-(0,-1);
 \clip (\xmin,\ymin) rectangle (\xmax,\ymax);
 \draw[smooth,samples=100,domain=\xmin:\xmax] plot(\x,{(\x)^3-3*(\x)+1}); 
 \end{tikzpicture}}
 \loigiai{
 Quan sát đồ thị, ta thấy
 \begin{itemize}
 \item Đây là đồ thị của hàm số $y=ax^3+bx^2+cx+d$ $ (a\ne 0)$ có $a > 0$.
 \item Đồ thị hàm số có hai điểm cực trị $(-1;3)$ và $(1;-1)$.
 \end{itemize}
 Vậy đường cong trong hình vẽ là đồ thị hàm số $y=x^3-3x+1$.
 }
\end{ex}

\begin{ex}%[2-D1B5-SO-17-2425]%[VN-MT-7, VM031]%[2D1H5-1]
 \immini{Cho hàm số $y=\dfrac{ax+b}{cx-1}$ có đồ thị như hình vẽ bên dưới. Trong các hệ số $a$, $b$, $c$ có bao nhiêu số dương?
 \choice[2]
 {$0$}
 {\True $2$}
 {$1$}
 {$3$}}
 {\begin{tikzpicture}[font=\footnotesize,line join=round, line cap=round, >=stealth,scale=0.6] 
 \def \xmin{-3}\def \xmax{4.5}\def \ymin{-4}\def \ymax{3} 
 \draw[->] (\xmin,0)--(\xmax,0) node[shift=(-110:0.2)] {$x$};
 \draw[->] (0,\ymin)--(0,\ymax) node[shift=(-150:0.2)] {$y$};
 \fill (0,0) circle(1pt) node[shift=(-135:0.25)]{$O$}
 (1,0) circle(1pt) node[shift=(-135:0.25)]{$1$}
 (0,-1) circle(1pt) node[shift=(-145:0.28)]{$-1$}
 (2,0) circle(1pt) node[shift=(-135:0.25)]{$2$}; 
 \draw (\xmin,-1)--(\xmax,-1);
 \clip (\xmin,\ymin) rectangle (\xmax,\ymax); 
 \draw[smooth,samples=100,domain=\xmin:0.99] plot(\x,{(-1*(\x)+2)/((\x)-1)});
 \draw[smooth,samples=100,domain=1.01:\xmax] plot(\x,{(-1*(\x)+2)/((\x)-1)}); 
 \draw (1,\ymin)--(1,\ymax);
 \end{tikzpicture}}
 \loigiai{
 \begin{itemize}
 \item Tiệm cận đứng $x=\dfrac{1}{c}=1\Leftrightarrow c=1$.
 \item Tiệm cận ngang $y=\dfrac{a}{c}=-1\Leftrightarrow a=-c\Rightarrow a=-1$.
 \item Đồ thị cắt trục hoành tại $x=2$ nên $2a+b=0$ hay $b=-2a=2$.
 \end{itemize}
 Vậy có hai số dương.
 }
\end{ex}

\begin{ex}%[2-D1B5-SO-17-2425]%[VN-MT-7, VM031]%[2D1H5-1]
 \immini{Đường cong cho trong hình bên là đồ thị của hàm số nào trong các hàm số dưới đây?
 \choice[2]
 {$y=\dfrac{x^2-2x+2}{x+1}$}
 {$y=\dfrac{-x^2+x+2}{x-1}$}
 {\True $y=\dfrac{x^2-x+1}{-x+1}$}
 {$y=\dfrac{-x^2-x+1}{x-1}$}}
 {\begin{tikzpicture}[font=\footnotesize,line join=round, line cap=round, >=stealth,scale=0.6] 
 \def \xmin{-3}\def \xmax{5}\def \ymin{-5}\def \ymax{3} 
 \draw[->] (\xmin,0)--(\xmax,0) node[shift=(-110:0.2)] {$x$};
 \draw[->] (0,\ymin)--(0,\ymax) node[shift=(-150:0.2)] {$y$};
 \fill (0,0) circle(1pt) node[shift=(-135:0.25)]{$O$}
 (1,0) circle(1pt) node[shift=(-120:0.22)]{$1$}
 (2,0) circle(1pt) node[shift=(90:0.2)]{$2$}
 (0,1) circle(1pt) node[shift=(-45:0.21)]{$1$}
 (0,-3) circle(1pt) node[shift=(180:0.25)]{$-3$}
 (2,-3) circle(1pt);
 \draw[dashed] (2,0)|-(0,-3);
 \clip (\xmin,\ymin) rectangle (\xmax,\ymax); 
 \clip (\xmin,\ymin) rectangle (\xmax,\ymax);
 \draw[smooth,samples=100,domain=\xmin:0.99] plot(\x,{((\x)^2-1*(\x)+1)/(-1*(\x)+1)}); 
 \draw[smooth,samples=100,domain=1.01:\xmax] plot(\x,{((\x)^2-1*(\x)+1)/(-1*(\x)+1)}); 
 \draw[smooth,samples=100,domain=\xmin:\xmax] plot(\x,{-(\x)}); 
 \draw (1,\ymin)--(1,\ymax);
 \end{tikzpicture}}
 \loigiai{
 \begin{itemize}
 \item Đồ thị hàm số có tiệm cận đứng $x=1$.
 \item Đồ thị hàm số có tiệm cận xiên $y=-x$.
 \item Đồ thị hàm số đi qua điểm $\left(2;-3\right)$.
 \end{itemize}
 Vậy đường cong trong hình vẽ là đồ thị hàm số $y=\dfrac{x^2-x+1}{-x+1}$.
 }
\end{ex}

\begin{ex}%[2-D1B5-SO-17-2425]%[VN-MT-7, VM031]%[2D1H4-1]
 \immini{Cho hàm số $y=\dfrac{ax^2+bx+1}{cx+2}$ có đồ thị như hình vẽ bên dưới. Tính giá trị biểu thức $T=2a+3b-c$.
 \choice[2]
 {$9$}
 {\True $10$}
 {$8$}
 {$11$}}
 {\begin{tikzpicture}[font=\scriptsize, line join=round, line cap=round, >=stealth,scale=0.6] 
 \def \xmin{-6}\def \xmax{2.5}\def \ymin{-4}\def \ymax{2.5} 
 \draw[->] (\xmin,0)--(\xmax,0) node[shift=(-110:0.2)] {$x$};
 \draw[->] (0,\ymin)--(0,\ymax) node[shift=(-150:0.2)] {$y$};
 \fill (0,0) circle(1pt) node[shift=(-45:0.25)]{$O$}
 (-2,0) circle(1pt) node[shift=(-140:0.25)]{$-2$}
 (-1,0) circle(1pt) node[shift=(110:0.2)]{$-1$}
 (0,1) circle(1pt) node[shift=(150:0.19)]{$1$};
 \clip (\xmin,\ymin) rectangle (\xmax,\ymax); 
 \draw[smooth,samples=100,domain=\xmin:-2.01] plot(\x,{((\x)^2+3*(\x)+1)/((\x)+2)}); 
 \draw[smooth,samples=100,domain=-1.99:\xmax] plot(\x,{((\x)^2+3*(\x)+1)/((\x)+2)}); 
 \draw[smooth,samples=100,domain=\xmin:\xmax] plot(\x,{(\x)+1}); 
 \draw(-2,\ymin)--(-2,\ymax);
 \end{tikzpicture}}
 \loigiai{
 \begin{center}
 
 \end{center}
 \begin{itemize}
 \item Đồ thị có tiệm cận đứng $x=-2$. Suy ra $-\dfrac{2}{c}=-2\Leftrightarrow c=1$.
 \item Đồ thị có tiệm cận xiên đi qua hai điểm $(0;1)$ và $(-1;0)$ nên có phương trình \[\dfrac{x}{-1}+\dfrac{y}{1}=1\Leftrightarrow y=x+1.\]
 \end{itemize}
 Khi đó ta có
 \begin{itemize}
 \item $\lim\limits_{x\to+\infty} \dfrac{ax^2+bx+1}{x\left(x+2\right)}=1\Leftrightarrow a=1$;
 \item $\lim\limits_{x\to+\infty} \left(\dfrac{x^2+bx+1}{x+2}-x\right)=\lim\limits_{x\to+\infty}\dfrac{\left(b-2\right)x+1}{x+2}=b-2=1\Leftrightarrow b=3$. 
 \end{itemize}
 Vậy $T=2a+3b-c=2+9-1=10$.
 }
\end{ex}

\begin{ex}%[2-D1B5-SO-17-2425]%[VN-MT-7, VM031]%[2D1N1-2]
 Cho hàm số $f(x)$ có bảng biến thiên như sau
 \begin{center}
 \begin{tikzpicture}
 \tkzTabInit[nocadre=true,lgt=1.2,espcl=2.5,deltacl=0.5]
 {$x$/0.7,$f'(x)$/0.7,$f(x)$/2}
 {$-\infty$,$0$,$2$,$+\infty$}
 \tkzTabLine{,+,0,-,0,+,}
 \tkzTabVar{-/$-\infty$,+/$1$,-/$-3$,+/$+\infty$}
 \end{tikzpicture}
 \end{center}
 Hàm số đã cho nghịch biến trên khoảng nào dưới đây?
 \choice
 {$(2;+\infty)$}
 {\True $(0;2)$}
 {$(-3;1)$}
 {$(-\infty;1)$}
 \loigiai{
 Dựa vào bảng biến thiên của hàm số, ta có hàm số đã cho nghịch biến trên khoảng $(0;2)$. 
 }
\end{ex}

\begin{ex}%[2-D1B5-SO-17-2425]%[VN-MT-7, VM031]%[2D1H1-1]
 \immini{Hàm số $y=-x^3+3x^2$ đồng biến trên khoảng nào dưới đây?
 \choice
 {$(0;4)$}
 {$(-\infty;0)$}
 {$(2;+\infty)$}
 {\True $(0;2)$} }
 
 \loigiai{
 Ta có $y'=-3x^2+6x=0\Leftrightarrow \hoac{&x=0\\&x=2.}$\\
 Hàm số đồng biến khi $y' > 0$ $\Leftrightarrow 0< x < 2$.
 }
\end{ex}

\begin{ex}%[2-D1B5-SO-17-2425]%[VN-MT-7, VM031]%[2D1N2-2]
 \immini{Cho hàm số $y=f(x)$ xác định và liên tục trên đoạn $[-2; 2]$ và có đồ thị là đường cong trong hình vẽ sau.
 Điểm cực tiểu của đồ thị hàm số $y=f(x)$ là
 \choice[2]
 {$x=1$}
 {$x=-2$}
 {\True $M(1;-2)$}
 {$M(-2;-4)$} }
 {\begin{tikzpicture}[font=\footnotesize,line join=round, line cap=round, >=stealth,scale=0.6] 
 \def \xmin{-2.8}\def \xmax{3}\def \ymin{-4.7}\def \ymax{5} 
 \draw[->] (\xmin,0)--(\xmax,0) node[shift=(-110:0.2)] {$x$};
 \draw[->] (0,\ymin)--(0,\ymax) node[shift=(-150:0.2)] {$y$};
 \fill (0,0) circle(1pt) node[shift=(45:0.25)]{$O$}
 (-2,0) circle(1pt) node[shift=(90:0.2)]{$-2$}
 (-1,0) circle(1pt) node[shift=(-90:0.2)]{$-1$}
 (1,0) circle(1pt) node[shift=(90:0.2)]{$1$}
 (2,0) circle(1pt) node[shift=(-90:0.2)]{$2$}
 (0,-4) circle(1pt) node[shift=(0:0.25)]{$-4$}
 (0,-2) circle(1pt) node[shift=(180:0.25)]{$-2$}
 (0,2) circle(1pt) node[shift=(0:0.2)]{$2$}
 (0,4) circle(1pt) node[shift=(180:0.2)]{$4$}
 (-1,2) circle(1pt) (1,-2) circle(1pt) (-2,-4) circle(1pt) (2,4) circle(1pt);
 \clip (\xmin,\ymin) rectangle (\xmax,\ymax); 
 \draw[smooth,samples=100,domain=-2:2] plot(\x,{4/3*(\x)^3-10/3*(\x)}); 
 \draw[dashed] (-2,0)|-(0,-4) (-1,0)|-(0,2) (1,0)|-(0,-2) (2,0)|-(0,4);
 \end{tikzpicture}}
 \loigiai{
 Dựa vào đồ thị hàm số ta thấy điểm cực tiểu của đồ thị hàm số $y=f(x)$ là $M(1;-2)$.
 }
\end{ex}

\begin{ex}%[2-D1B5-SO-17-2425]%[VN-MT-7, VM031]%[2D1N3-1]
 \immini{Cho hàm số $f(x)$ liên tục trên đoạn $[-2;2]$ có đồ thị như hình vẽ.
 Giá trị nhỏ nhất của hàm số trên đoạn $[-2;2]$ là
 \choice[2]
 {$1$}
 {\True $-1$}
 {$-2$}
 {$3$}}
 {\begin{tikzpicture}[font=\footnotesize,line join=round, line cap=round, >=stealth,scale=0.6] 
 \def \xmin{-3}\def \xmax{3}\def \ymin{-2}\def \ymax{4} 
 \draw[->] (\xmin,0)--(\xmax,0) node[shift=(-110:0.2)] {$x$};
 \draw[->] (0,\ymin)--(0,\ymax) node[shift=(-150:0.2)] {$y$};
 \fill (0,0) circle(1pt) node[shift=(135:0.25)]{$O$}
 (-2,0) circle(1pt) node[shift=(135:0.25)]{$-2$}
 (-1,0) circle(1pt) node[shift=(-90:0.2)]{$-1$}
 (1,0) circle(1pt) node[shift=(90:0.2)]{$1$}
 (2,0) circle(1pt) node[shift=(-90:0.2)]{$2$}
 (0,-1) circle(1pt) node[shift=(-145:0.28)]{$-1$}
 (0,3) circle(1pt) node[shift=(135:0.25)]{$3$}
 (-2,-1) circle(1pt) (1,-1) circle(1pt) (-1,3) circle(1pt) (2,3) circle(1pt);
 \clip (\xmin,\ymin) rectangle (\xmax,\ymax); 
 \draw[smooth,samples=100,domain=\xmin:\xmax] plot(\x,{(\x)^3-3*(\x)+1}); 
 \draw[dashed] (-2,0)--(-2,-1)--(1,-1)--(1,0) (-1,0)--(-1,3)--(2,3)--(2,0);
 \end{tikzpicture}}
 \loigiai{
 Từ đồ thị ta thấy $\min\limits_{[-2;2]} f(x)=f(1)=-1$.
 }
\end{ex}

\begin{ex}%[2-D1B5-SO-17-2425]%[VN-MT-7, VM031]%[2D1H3-1]
 Giá trị nhỏ nhất của hàm số $y=x^2-2x+3$ trên đoạn $[2;4]$ là
 \choice
 {\True $3$}
 {$-1$}
 {$0$}
 {$1$}
 \loigiai{
 \begin{itemize}
 \item $y'=\left(x^2-2x+3\right)'=2x-2$;
 \item $y'=0\Leftrightarrow 2x-2=0\Leftrightarrow x=1\notin [2;4]$.
 \end{itemize}
 Ta có $y(2)=3$; $y(4)=11$.\\
 Vậy $\min\limits_{[2;4]}y=y(2)=3$.
 }
\end{ex}

\begin{ex}%[2-D1B5-SO-17-2425]%[VN-MT-7, VM031]%[2D1N4-1]
 Đồ thị hàm số $y=\dfrac{1+2x}{x-1}$ có đường tiệm cận ngang là
 \choice
 {$x=1$}
 {$y=1$}
 {$x=2$}
 {\True $y=2$}
 \loigiai{
 Ta có $\lim\limits_{x\to \pm \infty} \dfrac{1+2x}{x-1}=\lim\limits_{x\to \pm \infty}\dfrac{\tfrac{1}{x}+2}{1-\tfrac{1}{x}}=2$.\\
 Nên $y=2$ là tiệm cận ngang của đồ thị hàm số.
 }
\end{ex}

\begin{ex}%[2-D1B5-SO-17-2425]%[VN-MT-7, VM031]%[2D1N4-1]
 Đường tiệm cận xiên của đồ thị hàm số $y=\dfrac{x^2-2x+3}{x+1}$ là
 \choice
 {\True $y=x-3$}
 {$y=x+1$}
 {$y=-3x+1$}
 {$x=-3y+1$}
 \loigiai{
 Tập xác định $\mathscr{D}=\mathbb{R}\setminus \{-1\}$.\\
 Phương trình đường tiệm cận xiên có dạng $y=ax+b$.\\
 Trong đó
 \begin{itemize}
 \item $a=\lim\limits_{x\to +\infty} \dfrac{f(x)}{x}=\lim\limits_{x\to +\infty}\dfrac{x^2-2x+3}{x^2+x}=1$;
 \item $b=\lim\limits_{x\to +\infty} \left[f(x)-ax\right]=\lim\limits_{x\to +\infty} \left(\dfrac{x^2-2x+3}{x+1}-x\right)=\lim\limits_{x\to +\infty}\dfrac{-3x+3}{x+1}=-3$.
 \end{itemize}
 Ta cũng có 
 \begin{itemize}
 \item $\lim\limits_{x\to -\infty}\dfrac{f(x)}{x}=1$;
 \item $\lim\limits_{x\to -\infty}\left[f(x)-x\right]=-3$.
 \end{itemize}
 Khi đó $\lim\limits_{x\to \pm\infty} \left[f(x)-(x-3)\right]=\lim\limits_{x\to \pm\infty} \dfrac{6}{x+1}=0$.\\
 Do đó, đồ thị hàm số có tiệm cận xiên là đường thẳng $y=x-3$.
 }
\end{ex}

\begin{ex}%[2-D1B5-SO-17-2425]%[VN-MT-7, VM031]%[2D1H4-1]
 Tổng số đường tiệm cận của đồ thị hàm số $y=\dfrac{\sqrt{x}+1}{3x-9\sqrt{x}+6}$ là
 \choice
 {\True $3$}
 {$4$}
 {$2$}
 {$1$}
 \loigiai{
 Tập xác định $\mathscr{D}=[0;+\infty)\setminus \left\{1;4\right\}$.\\
 Ta có $\lim\limits_{x\to +\infty}\dfrac{\sqrt{x}+1}{3x-9\sqrt{x}+6}=0$.\\
 Nên đồ thị hàm số có $1$ đường tiệm cận ngang là $y=0$.
 \begin{itemize}
 \item $\lim\limits_{x\to 1^+}\dfrac{\sqrt{x}+1}{3x-9\sqrt{x}+6}=\lim\limits_{x\to 1^+} \dfrac{\sqrt{x}+1}{3\left(\sqrt{x}-1\right)\left(\sqrt{x}-2\right)}=+\infty$;
 \item $\lim\limits_{x\to 1^-}\dfrac{\sqrt{x}+1}{3x-9\sqrt{x}+6}=-\infty$.
 \end{itemize}
 Suy ra đường thẳng $x=1$ là $1$ tiệm cận đứng của đồ thị hàm số.
 \begin{itemize}
 \item $\lim\limits_{x\to 4^+} \dfrac{\sqrt{x}+1}{3\left(\sqrt{x}-1\right)\left(\sqrt{x}-2\right)}=+\infty$;
 \item $\lim\limits_{x\to 4^-} \dfrac{\sqrt{x}+1}{3x-9\sqrt{x}+6}=-\infty$.
 \end{itemize}
 Suy ra đường thẳng $x=4$ là $1$ tiệm cận đứng của đồ thị hàm số.\\
 Đồ thị hàm số không có tiệm cận xiên.\\
 Vậy tổng số đường tiệm cận của đồ thị hàm số là $3$.
 }
 \end{ex}
\Closesolutionfile{ans}

\TNTF
\Opensolutionfile{ans}[ans/ansc1l4-Phan-II]
\begin{ex}%[2-D1B5-SO-17-2425]%[VN-MT-7, VM031]%[2D1V5-5]
 \immini{Cho hàm số $y=f(x)$ có đạo hàm trên $\mathbb{R}$ và hàm số $y=f'(x)$ là hàm số bậc ba có đồ thị là đường cong trong hình vẽ.
 \choiceTF
 {Hàm số $y=f(x)$ đồng biến trên khoảng $(-\infty;-2)$}
 {Hàm số $y=f(x)$ có hai điểm cực trị}
 {$f'(2)=4$}
 {\True Hàm số $g(x)=f(x)-\dfrac{1}{2} x^2+x+2024$ đồng biến trên khoảng $\left(-\dfrac{5}{2};-\dfrac{3}{2} \right)$}}
 {\begin{tikzpicture}[font=\footnotesize,line join=round, line cap=round, >=stealth,scale=0.8] 
 \def \xmin{-4}\def \xmax{2}\def \ymin{-5}\def \ymax{1} 
 \draw[->] (\xmin,0)--(\xmax,0) node[shift=(-110:0.2)] {$x$};
 \draw[->] (0,\ymin)--(0,\ymax) node[shift=(-150:0.2)] {$y$};
 \fill (0,0) circle(1pt) node[shift=(-135:0.25)]{$O$}
 (-3,0) circle(1pt) node[shift=(90:0.2)]{$-3$}
 (-2,0) circle(1pt) node[shift=(90:0.2)]{$-2$}
 (-1,0) circle(1pt) node[shift=(90:0.2)]{$-1$}
 (1,0) circle(1pt) node[shift=(110:0.22)]{$1$}
 (0,-4) circle(1pt) node[shift=(-145:0.3)]{$-4$}
 (0,-2) circle(1pt) node[shift=(-145:0.3)]{$-2$}
 (-1,-2) circle(1pt) (-3,-4) circle(1pt); 
 \clip (\xmin,\ymin) rectangle (\xmax,\ymax); 
 \draw[smooth,samples=100,domain=\xmin:\xmax] plot(\x,{(\x)^3+3*(\x)^2-4}); 
 \draw[dashed] (-3,0)|-(0,-4) (-1,0)|-(0,-2);
 \end{tikzpicture}}
 \loigiai{
 \begin{itemchoice}
 \itemch \textbf{Sai}.\\
 Vì từ đồ thị của hàm số $y=f'(x)$ ta thấy $f'(x)\ge 0$ với $\forall x\ge 1$ nên hàm số đồng biến trên khoảng $(1;+\infty)$.
 \itemch \textbf{Sai}.\\
 Vì từ đồ thị của hàm số $y=f'(x)$ ta thấy $f'(x)$ chỉ đổi dấu một lần qua $x=1$ nên hàm số có một điểm cực trị.
 \itemch \textbf{Sai}.\\
 Từ đồ thị ta có hàm số $f'(x)$ có dạng: $f'(x)=a(x+2)^2 (x-1)$.\\
 Đồ thị hàm số $y=f'(x)$ đi qua $(0;-4)$ nên $-4=a(0+2)^2 (0-1)\Leftrightarrow a=1$.\\
 Vậy $f'(x)=(x+2)^2(x-1)\Rightarrow f'(2)=(2+2)^2 (2-1)=16$.
 \itemch \textbf{Đúng}.\\
 Ta có $g'(x)=f'(x)-x+1=0\Leftrightarrow f'(x)=x-1$.\\
 Vẽ đường thẳng $y=x-1$ trên cùng hệ trục tọa độ với đồ thị hàm số $y=f'(x)$.
 \begin{center}
 \begin{tikzpicture}[line cap=butt,line join=miter,>=stealth,scale=.6,font=\footnotesize]
 \tikzset{declare function={xmin=-4;xmax=2;ymin=-5;ymax=1.5;
 f(\x)=(\x)^3+3*(\x)^2-4;
 g(\x)=(\x)-1;
 },
 smooth,samples=450
 }
 \draw[->] (xmin,0)--(xmax,0) node[shift={(-100:7pt)}]{$ x $};
 \draw[->] (0,ymin)--(0,ymax) node[shift={(190:7pt)}]{$ y $};
 \fill (0,0) node[shift={(225:9pt)}]{$ O $};
 \draw[dashed](-3,0)node[above]{$-3$}|-(0,-4)node[below left]{$-4$}
 (-1,0)node[above]{$-1$}|-(0,-2)node[below left]{$-2$}
 (-2,0)node[above]{$-2$} (1,0)node[above left]{$1$};
 \fill(-2,0) circle (1pt) node[above]{$-2$} 
 (1,0)circle (1pt) node[above left]{$1$}
 (-3,-4) circle (1pt) (-1,-2) circle (1pt) 
 (1,0) circle (1pt) (0,0) circle (1pt) (0,-4) circle (1pt) (-3,0) circle (1pt) (-1,0) circle (1pt)
 (0,-2) circle (1pt);
 \clip (xmin,ymin) rectangle (xmax,ymax);
 \draw plot[domain=xmin:xmax] (\x, {f(\x)});
 \draw plot[domain=xmin:xmax] (\x, {g(\x)});
 \end{tikzpicture}
 \end{center}
 Khi đó $f'(x)=x-1\Leftrightarrow \hoac{&x=-3\\&x=-1\\&x=1.}$\\
 Bảng biến thiên của hàm số $g(x)$
 \begin{center}
 \begin{tikzpicture}
 \tkzTabInit[nocadre=true,lgt=1.2,espcl=2.5,deltacl=0.5]
 {$x$/0.7,$f'(x)$/0.7,$f(x)$/2}
 {$-\infty$,$-3$,$-1$,$1$,$+\infty$}
 \tkzTabLine{,-,$0$,+,$0$,-,$0$,+,}
 \tkzTabVar{+/$+\infty$,-/$g(-3)$,+/$g(-1)$,-/$g(1)$,+/$+\infty$}
 \end{tikzpicture}
 \end{center}
 Ta có hàm số $g(x)$ đồng biến trên khoảng $(-3;-1)$ nên $g(x)$ đồng biến trên khoảng $\left(-\dfrac{5}{2};-\dfrac{3}{2} \right)$.
 \end{itemchoice}
 }
\end{ex}

\begin{ex}%[2-D1B5-SO-17-2425]%[VN-MT-7, VM031]%[2D1V2-1]
 Cho hàm số $y=x^3-3x+1$.
 \choiceTF
 {\True Điểm cực tiểu của hàm số là $x=1$}
 {Hàm số đồng biến trên khoảng $(-1;1)$}
 {\True Giả sử hàm số đã cho có hai điểm cực trị là $x_1$; $x_2$. Khi đó giá trị $x_1 \cdot x_2=-1$}
 {Gọi $A$, $B$ lần lượt là điểm cực đại và điểm cực tiểu của đồ thị hàm số. Khi đó, diện tích tam giác $ABC$ là $12$ với $C(-1;2)$}
 \loigiai{
 \begin{itemchoice}
 \itemch \textbf{Đúng}.\\
 Ta có $y'=3x^2-3$\\
 $y'=0\Leftrightarrow 3x^2-3=0\Leftrightarrow \hoac{&x=-1\\&x=1} \Leftrightarrow \hoac{&y(-1)=3\\&y(1)=-1.}$\\
 Ta có bảng biến thiên
 \begin{center}
 \begin{tikzpicture}
 \tkzTabInit[nocadre=true,lgt=1.2,espcl=2.5,deltacl=0.5]
 {$x$/0.7,$f'(x)$/0.7,$f(x)$/2}
 {$-\infty$,$-1$,$1$,$+\infty$}
 \tkzTabLine{,+,0,-,0,+,}
 \tkzTabVar{-/$-\infty$,+/$3$,-/$-1$,+/$+\infty$}
 \end{tikzpicture}
 \end{center}
 Từ bảng biến thiên ta có điểm cực tiểu của hàm số là $x=1$.
 \itemch \textbf{Sai}.\\
 Vì từ bảng biến thiên ta có hàm số nghịch biến trên khoảng $(-1;1)$.
 \itemch \textbf{Đúng}.\\
 Vì $x_1 \cdot x_2=1\cdot (-1)=-1$.
 \itemch \textbf{Sai}.\\
 Vì $A(-1;3)$, $B(1;-1)$, $C(-1;2)$ nên
 \begin{itemize}
 \item $\left|\overrightarrow{AB}\right|=\sqrt{2^2+(-4)^2}=2\sqrt{5}$;
 \item $\left|\overrightarrow{AC}\right|=\sqrt{0^2+(-1)^2}=1$;
 \item $\cos \widehat{BAC}=\cos\big(\overrightarrow{AB},\overrightarrow{AC}\big)=\dfrac{x_1 x_2+y_1 y_2}{\sqrt{x_1^2+y_1^2} \sqrt{x_2^2+y_2^2}}=\dfrac{2\cdot 0+(-4)(-1)}{\sqrt{2^2+(-4)^2} \sqrt{0^2+(-1)^2}}=\dfrac{2\sqrt{5}}{5}$.
 \item $\sin \widehat{BAC}=\sqrt{1-\cos^2 \widehat{BAC}}=\dfrac{\sqrt{5}}{5}$.
 \item $S_{\triangle ABC}=\dfrac{1}{2}\cdot AB\cdot AC\cdot\sin \widehat{BAC}=\dfrac{1}{2}\cdot 2\sqrt{5}\cdot 1\cdot \dfrac{\sqrt{5}}{5}=1$.
 \end{itemize}
 \end{itemchoice}
 }
\end{ex}

\begin{ex}%[2-D1B5-SO-17-2425]%[VN-MT-7, VM031]%[2D1V3-1]
 Cho hàm số $y=\dfrac{x+m}{x-1}$ ($m$ là tham số thực).
 \choiceTF
 {\True Khi $m=2$ thì giá trị lớn nhất của hàm số trên đoạn $[2;5]$ là $4$}
 {\True Khi $m=2$ thì giá trị nhỏ nhất của hàm số trên đoạn $[2;5]$ là $\dfrac{7}{4}$}
 {Khi $m <-1$ thì giá trị nhỏ nhất của hàm số trên đoạn $[2;4]$ là $y(4)$}
 {Khi $\min\limits_{[2;4]}y=3$ thì giá trị của tham số $m$ là $1\le m < 3$}
 \loigiai{
 Tập xác định $\mathscr{D}=\mathbb{R}\setminus \{1\}$.\\
 Ta có $y'=\dfrac{-1-m}{(x-1)^2}$.
 \begin{itemchoice}
 \itemch \textbf{Đúng}.\\
 Khi $m=2$ thì $y'=\dfrac{-1-2}{\left(x-1\right)^2}=\dfrac{-3}{\left(x-1\right)^2} < 0\; \forall x\in \mathscr{D}\Rightarrow$ hàm số nghịch biến trên từng khoảng xác định, do đó hàm số cũng nghịch biến trên $[2;5]$.\\
 Vậy $\max\limits_{[2;5]}y=y(2)=4$.
 \itemch \textbf{Đúng}.\\
 Khi $m=2$ thì $y'=\dfrac{-1-2}{\left(x-1\right)^2}=\dfrac{-3}{\left(x-1\right)^2} < 0\; \forall x\in \mathscr{D}\Rightarrow$ hàm số nghịch biến trên từng khoảng xác định, do đó hàm số cũng nghịch biến trên $[2;5]$.\\
 Vậy $\min\limits_{[2;5]} y=y(5)=\dfrac{7}{4}$.
 \itemch \textbf{Sai}.\\
 Với $m <-1$ $\Rightarrow-1-m > 0\Rightarrow y' > 0$ nên hàm số đã cho đồng biến trên trên từng khoảng xác định, do đó hàm số cũng đồng biến trên $\left[2;4\right]$ suy ra $\min\limits_{[2;4]}y=y(2)$.
 \itemch \textbf{Sai}.
 \begin{itemize}
 \item Trường hợp 1.\\
 $-1-m > 0\Leftrightarrow m <-1$ $\Rightarrow y' > 0$ nên hàm số đã cho đồng biến trên $[2;4]$.\\
 Khi đó $\min\limits_{[2;4]} y=y(2)\Leftrightarrow 3=2+m\Leftrightarrow m=1\text{ (không thoả mãn)}$.
 \item Trường hợp 2.\\
 $-1-m < 0\Leftrightarrow m >-1$ $\Rightarrow y' < 0$ nên hàm số đã cho nghịch biến trên $[2;4]$.
 \end{itemize}
 Khi đó $\min\limits_{[2;4]} y=y(4)\Leftrightarrow 3=\dfrac{4+m}{3} \Leftrightarrow m=5\text{ (thoả mãn)}$.\\
 Suy ra $m\notin [1;3)$.
 \end{itemchoice}
 }
\end{ex}

\begin{ex}%[2-D1B5-SO-17-2425]%[VN-MT-7, VM031]%[2D1V4-1]
 \immini{Cho hàm số $y=\dfrac{ax+b}{x+c}.(a,b,c\in\mathbb{R})$ có đồ thị như hình vẽ. Khi đó 
 \choiceTF
 {\True Đồ thị hàm số có tiệm cận ngang là $y=-1$}
 {\True Đồ thị hàm số có tiệm cận đứng là $x=1$}
 {$a+b+c=1$}
 {Hàm số đồng biến trên các khoảng xác định}}
 {\begin{tikzpicture}[>=stealth,scale=.6]
 \draw[->] (-4,0) --(4,0);
 \draw[->](0,-4)--(0,4);
 \draw (0,0)circle (1pt) node[above left]{$O$};
 \draw (4,0) node[below]{$x$};
 \draw (0,4) node[left]{$y$};
 \draw (1,0)circle (1pt) node[above left]{$1$};
 \draw (0,-1)circle (1pt) node[above left]{$-1$};
 \draw (2,0)circle (1pt) node[above]{$2$};
 \clip (-4,-4) rectangle(4,4);
 \draw[thick,samples=100] plot[domain=-4:4]
 (\x,{(-(\x)+2)/((\x)-1)});
 \draw[thick,samples=100] plot[domain=-4:4]
 (\x,-1);
 \draw (2,-2) node
 {$x=1$};
 \end{tikzpicture}} 
 \loigiai{
 \begin{itemchoice}
 \itemch \textbf{Đúng}.\\
 Dựa vào đồ thị ta thấy đường thẳng $y=-1$ là tiệm cận ngang.
 \itemch \textbf{Đúng}.\\
 Dựa vào đồ thị ta thấy đường thẳng $x=1$ là tiệm cận đứng.
 \itemch \textbf{Đúng}.\\
 Dựa vào đồ thị hàm số ta có 
 \begin{itemize}
 \item Tiệm cận ngang $y=-1\Rightarrow a=-1$.
 \item Tiệm cận đứng $x=1\Rightarrow c=-1$.
 \item Đồ thị hàm số đi qua điểm $(2;0)$ nên $0=\dfrac{-2+b}{2-1}\Rightarrow b=2$. 
 \end{itemize}
 Vậy $a+b+c=-1+2-1=0$.
 \itemch \textbf{Sai}.\\
 Ta có $y=\dfrac{-x+2}{x-1}\Rightarrow y'=\dfrac{-1}{(x-1)^2}< 0$, $\forall x\ne 1$.\\
 Vậy hàm số nghịch biến trên các khoảng xác định.
 \end{itemchoice}
 }
\end{ex}
\Closesolutionfile{ans}

\TNSA
\Opensolutionfile{ans}[ans/ansc1l4-Phan-III]
\begin{ex}%[Đề ôn tập chương1-3]%[Nguyễn Tấn Linh]%[2D1C1-3]
	\immini
	{
	Cho hàm số $y=f(x)$ có đạo hàm liên tục trên $\mathbb{R}$ và có đồ thị $y=f'\left(x\right)$ như hình vẽ. Đặt $g\left(x\right)=f\left(x-m\right)-\dfrac{1}{2}\left(x-m-1\right)^2+2019$, với $m$ là tham số thực. Gọi $S$ là tập hợp các giá trị nguyên dương của $m$ để hàm số $y=g\left(x\right)$ đồng biến trên khoảng $\left(5;6\right)$. Tính tổng tất cả các phần tử trong $S$.\shortans{$14$}
	}
	{
	\begin{tikzpicture}[line join = round, line cap = round,>=stealth,font=\footnotesize,scale=.75,declare function={f(\x)=\a*(\x)^3+(\b)*(\x)^2+(\c)*\x+(\d);}]
	\def\xt{-2} \def\xp{4} \def\yd{-3} \def\yt{3}
	\def\a{1}
	\def\b{-3}
	\def\c{0}
	\def\d{2}
	\draw[->] (\xt,0)--(\xp,0) node[below]{$x$};
	\draw[->] (0,\yd)--(0,\yt) node[left]{$y$};
	\fill (0,0) circle (1.5pt) node[below left]{$O$};
	\begin{scope}
	\clip (\xt+0.1,\yd+0.1) rectangle (\xp-0.1,\yt-0.1);
	\draw[samples=150,smooth,domain=\xt:\xp] plot(\x,{f(\x)});
	\end{scope}
	\draw[dashed] (-1,0) node[above]{$-1$}|-(2,-2)--(2,0) node[above]{$2$} (3,0) node[below]{$3$}|-(0,2) node[above left]{$2$};
	\node at (1,0)[below]{$1$};
	\end{tikzpicture}
	}
	\loigiai{
	Xét hàm số $g\left(x\right)=f\left(x-m\right)-\dfrac{1}{2}\left(x-m-1\right)^2+2019$.\\
	$g'\left(x\right)=f'\left(x-m\right)-\left(x-m-1\right)$.\\
	Xét phương trình $g'\left(x\right)=0$.\hfill$\left(1\right)$\\
	Đặt $x-m=t$, phương trình $\left(1\right)$ trở thành $f'\left(t\right)-\left(t-1\right)=0\Leftrightarrow f'\left(t\right)=t-1$.\hfill$\left(2\right)$\\
	Nghiệm của phương trình $\left(2\right)$ là hoành độ giao điểm của hai đồ thị hàm số $y=f'\left(t\right)$ và $y=t-1$.\\
	Ta có đồ thị các hàm số $y=f'\left(t\right)$ và $y=t-1$ như sau
	\begin{center}
	\begin{tikzpicture}[line join = round, line cap = round,>=stealth,font=\footnotesize,scale=.75,declare function={f(\x)=\a*(\x)^3+(\b)*(\x)^2+(\c)*\x+(\d);}]
	\def\xt{-2} \def\xp{4} \def\yd{-3} \def\yt{3}
	\def\a{1}
	\def\b{-3}
	\def\c{0}
	\def\d{2}
	\draw[->] (\xt,0)--(\xp,0) node[below]{$x$};
	\draw[->] (0,\yd)--(0,\yt) node[left]{$y$};
	\fill (0,0) circle (1.5pt) node[below left]{$O$};
	\begin{scope}
	\clip (\xt+0.1,\yd+0.1) rectangle (\xp-0.1,\yt-0.1);
	\draw[samples=150,smooth,domain=\xt:\xp] plot(\x,{f(\x)});
	\draw[samples=150,smooth,domain=\xt:\xp] plot(\x,{\x-1});
	\end{scope}
	\draw[dashed] (-1,0) node[above]{$-1$}|-(2,-2)--(2,0) node[above]{$2$} (3,0) node[below]{$3$}|-(0,2) node[above left]{$2$};
	\node at (1,0)[below]{$1$};
	\end{tikzpicture}
	\end{center}
	Căn cứ đồ thị các hàm số ta có phương trình $\left(2\right)$ có nghiệm là $\hoac{&t=-1 \\&t=1 \\&t=3}\Rightarrow \hoac{&x=m-1 \\&x=m+1 \\&x=m+3.}$\\
	Ta có bảng biến thiên của $y=g\left(x\right)$
	\begin{center}
	\begin{tikzpicture}
	\tkzTabInit[lgt=2,espcl=2.5]{$x$/1, $y'$/1, $y$/2}{$-\infty$, $m-1$, $m+1$, $m+3$, $+\infty$}
	\tkzTabLine{,-,$0$,+,$0$,-,$0$,+,}
	\tkzTabVar{+/ $+\infty$ , -/ , +/ , -/ , +/ $+\infty$}
	\end{tikzpicture}
	\end{center}
	Để hàm số $y=g\left(x\right)$ đồng biến trên khoảng $\left(5;6\right)$ cần $\hoac{&\heva{&m-1\le 5 \\&m+1\ge 6} \\&m+3\le 5}\Leftrightarrow \hoac{&5\le m\le 6 \\&m\le 2.}$\\
	Vì $m\in \mathbb{N}^*\Rightarrow S=\{1;2;5;6\}\Rightarrow$ Tổng các phần tử trong $S$ bằng $14$.}
	\end{ex}

\begin{ex}%[2-D1B5-SO-17-2425]%[VN-MT-7, VM031]%[2D1V3-6]
 Trong một trò chơi, mỗi đội chơi được phát một tấm bìa hình chữ nhật kích thước 21 cm, 29,5 cm. Nhiệm vụ của mỗi đội là cắt ở bốn góc của tấm bìa này bốn hình vuông bằng nhau, rồi gập tấm bìa lại và dán keo để được một cái hộp không nắp có dạng hình hộp chữ nhật như hình vẽ. 
 \begin{center}
 \begin{tabular}{p{8cm}p{8cm}}
 \begin{tikzpicture}[scale=0.6, font=\footnotesize, line join=round, line cap=round, >=stealth]
 \foreach \x/\y/\pos in {0/0/A, 1/0/B, 1/1/C, 0/1/D, 6/0/A', 0/4/M, 1/4/N, 1/5/P, 0/5/Q, 6/4/M'} \path ($(\x,\y)$) coordinate (\pos); 
 \coordinate (B') at ($(B)+(A')-(A)$);
 \coordinate (C') at ($(C)+(A')-(A)$);
 \coordinate (D') at ($(D)+(A')-(A)$);
 \coordinate (N') at ($(N)+(M')-(M)$);
 \coordinate (P') at ($(P)+(M')-(M)$);
 \coordinate (Q') at ($(Q)+(M')-(M)$);
 \fill[gray!40] (A)--(B)--(C)--(D)--cycle;
 \fill[gray!40] (A')--(B')--(C')--(D')--cycle;
 \fill[gray!40] (M)--(N)--(P)--(Q)--cycle;
 \fill[gray!40] (M')--(N')--(P')--(Q')--cycle;
 \draw (A)--(B')--(P')--(Q)--(A) (D)--(C') (M)--(N') (B)--(P) (A')--(Q');
 \draw[|<->|] ([xshift=-3mm]A)--([xshift=-3mm]Q)node[pos=.5,left]{$21$\,cm};
 \draw[|<->|] ([yshift=3mm]Q)--([yshift=3mm]P')node[pos=.5,above]{$29{,}5$\,cm};
 \end{tikzpicture} & 
 \begin{tikzpicture}[scale=0.6, font=\footnotesize, line join=round, line cap=round, >=stealth]
 \foreach \x/\y/\pos in {0/0/A, -1.8/-1.8/B, 3/-1.8/C} \path ($(\x,\y)$) coordinate (\pos); 
 \coordinate (D) at ($(A)+(C)-(B)$);
 \coordinate (A') at ($(A)+(0,1)$);
 \coordinate (B') at ($(B)+(A')-(A)$);
 \coordinate (C') at ($(C)+(A')-(A)$);
 \coordinate (D') at ($(D)+(A')-(A)$); 
 \draw (A')--(B')--(B)--(C)--(D)--(D')--(A');
 \draw (B')--(C')--(D') (C)--(C');
 \draw [dashed] (B)--(A)--(D) (A)--(A');
 \end{tikzpicture}
 \end{tabular}
 \end{center}
%\begin{center}
% \hspace{3cm}
% 
%\end{center}
 Đội nào thiết kế được chiếc hộp có thể tích lớn nhất sẽ dành chiến thắng. Hãy xác định cạnh của hình vuông bị cắt để thu được hộp có thể tích lớn nhất. (Coi mép dán không đáng kể, kết quả làm tròn đến hàng phần trăm).
 \shortans{4{,}03}
 \loigiai{
 Gọi cạnh của hình vuông bị cắt ở bốn góc là $x$.\\
 Điều kiện $0< 2x < 21\Leftrightarrow 0< x < 10,5$, đơn vị cm.\\
 Ta có kích thước của khối hộp chữ nhật là $x$; $21-2x$;\ $29{,}5-2x$.\\
 Thể tích của khối hộp là $V=(21-2x)\cdot (29{,}5-2x)\cdot x=619{,}5x-101x^2+4x^3=f(x)$.\\
 Thể tích khối hộp lớn nhất khi hàm số $f(x)$ đạt giá trị lớn nhất.\\
 Xét hàm số $f(x)=619{,}5x-101x^2+4x^3$ trên khoảng $(0;10{,}5)$
 \begin{eqnarray*}
 &&f'(x)=12x^2-202x+619{,}5=0\\
 &\Leftrightarrow& \hoac{&x_1 \approx 4{,}03\\&x_2 \approx 12{,}80.}
 \end{eqnarray*}
 Ta có bảng biến thiên
 \begin{center}
 \begin{tikzpicture}
 \tkzTabInit[nocadre=true,lgt=1.2,espcl=2.5,deltacl=0.5]
 {$x$/0.7,$f'(x)$/0.7,$f(x)$/2}
 {$0$,$x_1$,$10{,}5$}
 \tkzTabLine{,+,0,-,}
 \tkzTabVar{-/$0$,+/$f(x_1)$,-/$f(10{,}5)$}
 \end{tikzpicture}
 \end{center}
 Suy ra $\max\limits_{(0;10{,}5)} f(x)=f(x_1)$.\\
 Vậy cạnh của hình vuông xấp xỉ $4{,}03$ cm.
 }
\end{ex}

\begin{ex}%[2-D1B5-SO-17-2425]%[VN-MT-7, VM031]%[2D1H2-1]
 Điểm cực tiểu $x_{\text{CT}}$ của hàm số $y=x^3+3x^2-9x$ là
 \shortans{1}
 \loigiai{
 Ta có $y'=3x^2+6x-9=0$.\\
 $y'=0 \Leftrightarrow \hoac{&x=1\\&x=-3} \Rightarrow \hoac{&y(1)=-5\\&y(-3)=27.}$\\
 Bảng biến thiên
 \begin{center}
 \begin{tikzpicture}
 \tkzTabInit[nocadre=true,lgt=1.2,espcl=2.5,deltacl=0.5]
 {$x$/0.7,$f'(x)$/0.7,$f(x)$/2}
 {$-\infty$,$-3$,$1$,$+\infty$}
 \tkzTabLine{,+,0,-,0,+,}
 \tkzTabVar{-/$-\infty$,+/$27$,-/$-5$,+/$+\infty$}
 \end{tikzpicture}
 \end{center}
 Vậy $x=1$ là điểm cực tiểu.
 }
\end{ex}

\begin{ex}%[Vovanle]%[2D1V5-4]
	Một đường thẳng cắt đồ thị hàm số $y=3x^4-4x^2$ tại bốn điểm phân biệt có hoành độ $0;1;a;b$. Tính $S=ab-a-b$. (làm tròn 2 chữ số thậm phân)
	\shortans{$0{,}67$}
	\loigiai{
		Đường thẳng $d$ cắt đồ thị $(C)$ của hàm số $y=f(x)=3x^4-4x^2$ lần lượt tại các điểm $A$, $B$ có hoành độ $0;1$ nên
		$y_A=f(0)=0$; $y_B=f(1)=-1$.\\
		$\Rightarrow A(0;0),\,B(1;-1)$.\\
		Suy ra PTĐT $d$ là $y=-x$.\\
		Phương trình hoành độ giao điểm của $d$ và $(C)$ là
		\allowdisplaybreaks
		\begin{eqnarray*}
			&&3x^4-4x^2=-x\\
			& \Leftrightarrow& 3x^4-4x^2+x=0 \\ 
			& \Leftrightarrow& x\left(3x^3-4x+1\right)=0 \\ 
			& \Leftrightarrow& x(x-1)\left(3x^2+3x-1\right)=0 \\ 
			&\Leftrightarrow& \hoac{& x=0\\& x-1=0 \\& 3x^2+3x-1=0}\Leftrightarrow \hoac{& x=0 \\& x=1\\& x=\dfrac{-3-\sqrt{21}}{6}\\& x=\dfrac{-3+\sqrt{21}}{6}.}
		\end{eqnarray*}
		Từ đó suy ra $a=\dfrac{-3-\sqrt{21}}{6};b=\dfrac{-3+\sqrt{21}}{6}\Rightarrow S=ab-a-b=\dfrac{2}{3}$.\\
		\textbf{Nhận xét:} Do biểu thức $S$ đối xứng nên ta có thể áp dụng định lí Vi-ét để tính nhanh hơn\\
		Cụ thể $a,\,b$ là nghiệm của phương trình $3x^2+3x-1=0$ nên $ab=-\dfrac{1}{3};\,a+b=-1$.\\
		Từ đó suy ra $S=ab-a-b=ab-(a+b)=-\dfrac{1}{3}-(-1)=\dfrac{2}{3}\approx0{,}67$.
	}
\end{ex}

\begin{ex}%[2-D1B5-SO-17-2425]%[VN-MT-7, VM031]%[2D1V3-1]
 Cho hàm số $y=\dfrac{x-m^2-1}{x-m}$ có bao nhiêu giá trị nguyên $m$ thỏa mãn $\max\limits_{[0;4]}y=-6$.
 \shortans{1}
 \loigiai{
 Tập xác định $\mathscr{D}=\mathbb{R}\setminus \{m\}$.\\
 Ta có $y'=\dfrac{m^2-m+1}{\left(x-m\right)^2} > 0$, $\forall x\in \mathscr{D}$ (do $m^2-m+1=\left(m-\dfrac{1}{2} \right)^2+\dfrac{3}{4} > 0$, $\forall m\in \mathbb{R}$).\\
 Do đó hàm số đồng biến trên các khoảng $(-\infty; m)$ và $(m;+\infty)$.\\
 Khi đó $\max\limits_{[0;4]}y=y(4)$.\\
 Để hàm số đã cho có giá trị lớn nhất trên $[0;4]$ bằng $-6$ thì
 \[\heva{&m\notin [0;4]\\&y(4)=-6} \Leftrightarrow \heva{&m\notin [0;4]\\
 &\dfrac{3-m^2}{4-m}=-6} \Leftrightarrow \heva{&m\notin [0;4]\\
 &m^2+6m-27=0} \Leftrightarrow \heva{&m\notin [0;4]\\&\hoac{&m=3\\&m=-9.}} \Leftrightarrow m=-9.
 \]
 Vậy có $1$ giá trị của $m$ thỏa mãn yêu cầu bài toán.
 }
\end{ex}

\begin{ex}%[2-D1B5-SO-17-2425]%[VN-MT-7, VM031]%[2D1V4-2]
 Biết tích các giá trị của tham số $m$ để đồ thị của hàm số $y=\dfrac{2x-4}{x^2+2(m-2)x+m^2+1}$ có đúng $2$ đường tiệm cận là $\dfrac{a}{b}$, $\dfrac{a}{b}$ là phân số tối giản. Tính $P=a^2+b^2$.
 \shortans{85}
 \loigiai{
 Đặt $f(x)=x^2+2(m-2)x+m^2+1$.\\
 Dễ thấy đồ thị không có tiệm cận xiên.\\
 Đồ thị có $1$ tiệm cận ngang là $y=0$ do $\lim\limits_{x\to+\infty} \dfrac{2x-4}{x^2+2(m-2)x+m^2+1}=0$.\\
 Do đó, để đồ thị hàm số có đúng hai đường tiệm cận thì đồ thị hàm số chỉ có đúng $1$ đường tiệm cận đứng.\\
 Khi đó, $f(x)=0$ có $2$ nghiệm phân biệt trong đó có $1$ nghiệm $x=2$ hoặc $f(x)=0$ có nghiệm kép
 \begin{eqnarray*}
 &\Leftrightarrow& \hoac{&\heva{&\Delta' > 0\\&f(2)=0}\\&\Delta '=0} \Leftrightarrow\hoac{&\heva{&(m-2)^2-m^2-1> 0\\&4+2(m-2)\cdot 2+m^2+1=0}\\&(m-2)^2-m^2-1=0}\\
 &\Leftrightarrow&\hoac{&\heva{&-4m+3> 0\\&m^2+4m-3=0} \\&-4m+3=0} \Leftrightarrow \hoac{&\heva{&m < \dfrac{3}{4} \\&m=-2\pm \sqrt{7}} \\&m=\dfrac{3}{4}} \Leftrightarrow \hoac{&m=-2\pm \sqrt{7} \\&m=\dfrac{3}{4}.}
 \end{eqnarray*}
 Vậy tích tất cả các giá trị thực của tham số $m$ là $P=\left(-2+\sqrt{7} \right)\cdot\left(-2-\sqrt{7} \right)\cdot\dfrac{3}{4}=-3\cdot\dfrac{3}{4}=\dfrac{-9}{4}$.\\
 Do đó $a=-9$, $b=4$ nên $P=a^2+b^2=81+4=85$.
 }
\end{ex}
\Closesolutionfile{ans}
% \begin{indapan}
% 	{ans/ansc1l4}
% \end{indapan}


% \begin{name}
	{\tenchude}
	{ĐỀ ÔN TẬP CHƯƠNG I}
	{LỚP TOÁN THẦY PHÁT}
	{\thoigian}
\end{name}

\TN
\Opensolutionfile{ans}[ans/ansc101]
\Opensolutionfile{ans}[ans/ansDe1-TN1]
\begin{ex}%[2D1H5-3]
	Cho hàm số $y=f(x)$ có đồ thị như hình. Tìm số nghiệm của phương trình $2f(x)+3=0$.
	\begin{center}
		\begin{tikzpicture}[line join=round, line cap = round, >=stealth, scale=1,font=\footnotesize,transform shape]
			\pgfmathsetmacro\a{sqrt(2)}
			\draw[->] (-2.5,0) -- (2.5,0)node[above]{$x$};
			\draw[->] (0,-2.5) -- (0,2.5)node[right]{$y$};
			%\draw[-] (-2.1,-1.5) -- (2.2,-1.5) node[right]{$y=-\frac{3}{2}$};
			\draw[fill=black]
			(0,0) circle(1pt) node[below right]{$O$}
			(0,2) circle(1pt) node[above left]{$2$}
			(0,-2) circle(1pt) node[below left]{$-2$}
			(-\a,0) circle(1pt) node[above]{$-\sqrt{2}$}
			(\a,0) circle(1pt) node[above]{$\sqrt{2}$}
			;
			\draw[dashed]
			(-\a,0)--(-\a,-2)--(\a,-2)--(\a,0)
			;
			\draw[smooth,samples=100,domain=-2.005:2.005] plot(\x,{(\x)^4-4*(\x)^2+2});
		\end{tikzpicture}
	\end{center}
	\choice
	{\True $4$}
	{$2$}
	{$0$}
	{$3$}
	\loigiai{
		Ta có $2f(x)+3=0 \Leftrightarrow f(x)=-\dfrac{3}{2}.\quad (*)$
		\begin{center}
			\begin{tikzpicture}[line join=round, line cap = round, >=stealth, scale=1,font=\footnotesize,transform shape]
				\pgfmathsetmacro\a{sqrt(2)}
				\draw[->] (-2.5,0) -- (2.5,0)node[above]{$x$};
				\draw[->] (0,-2.5) -- (0,2.5)node[right]{$y$};
				\draw[-] (-2.1,-1.5) -- (2.2,-1.5) node[right]{$y=-\frac{3}{2}$};
				\draw[fill=black]
				(0,0) circle(1pt) node[below right]{$O$}
				(0,2) circle(1pt) node[above left]{$2$}
				(0,-2) circle(1pt) node[below left]{$-2$}
				(-\a,0) circle(1pt) node[above]{$-\sqrt{2}$}
				(\a,0) circle(1pt) node[above]{$\sqrt{2}$}
				;
				\draw[dashed]
				(-\a,0)--(-\a,-2)--(\a,-2)--(\a,0)
				;
				\draw[smooth,samples=100,domain=-2.005:2.005] plot(\x,{(\x)^4-4*(\x)^2+2});
			\end{tikzpicture}
		\end{center}
		Số nghiệm của phương trình $(*)$ là số giao điểm của đồ thị hàm số $f(x)$ và đường thẳng nằm ngang $y=-\dfrac{3}{2}$. Quan sát hình vẽ, nhận thấy số giao điểm là $4$. Suy ra số nghiệm của phương trình là $4$.
	}
\end{ex}

\begin{ex}%[2D1N5-3]
	\immini
	{
		Cho hàm số có đồ thị là đường cong trong hình bên. Tọa độ giao điểm của đồ thị hàm số đã cho và trục tung là
		\choice
		{$(0;-2)$}
		{$(-1;0)$}
		{\True $(0;-1)$}
		{$(-2;0)$}
	}
	{
		\begin{tikzpicture}[>=stealth,x=1cm,y=1cm,scale=1,font=\footnotesize,line cap=round,line join=round]
			\draw[->] (-2.5,0)--(0,0)%
			node[below right]{$O$}--(2.5,0) node[below]{$x$};
			\draw[->] (0,-3) --(0,2) node[right]{$y$};
			\foreach \x in {-2,2}{
					\draw[fill=black] (\x,0) node[above]{$\x$} circle (1pt);%Ox
				}
			\foreach \y in {-2}{
					\draw[fill=black] (0,\y) node[below left]{$\y$} circle (1pt);%Oy
				}
			\draw[fill=black] (0,-1) node[above left]{$-1$} circle (1pt) (0,1) node[left] {$1$} circle (1pt);
			\draw [domain=-1.7:1.7, samples=100]%
			plot (\x, {(\x)^4-2*(\x)^2-1});
			\draw [dashed] (1,0) node[above]{$1$}%
			--(1,-2)--(-1,-2)--(-1,0)
			node[above]{$-1$};
			\draw[fill=black] (0,0) circle(1pt);
		\end{tikzpicture}
	}
	\loigiai{
		Từ đồ thị ta thấy đồ thị hàm số cắt trục tung tại điểm có tọa độ $(0;-1)$.
	}
\end{ex}

\begin{ex}%[2D1V5-4]
	Cho hàm số $y=x^3-3mx^2+\left(3m-1\right)x+6m$ có đồ thị là $(C)$. Tìm tất cả các giá trị thực của tham số $m$ để $(C)$ cắt trục hoành tại ba điểm phân biệt có hoành độ $x_1, x_2, x_3$ thỏa mãn điều kiện $x_1^2+x_2^2+x_3^2+x_1x_2x_3=20$.
	\choice
	{$m=\dfrac{3\pm \sqrt{33}}{3}$}
	{$m=\dfrac{2\pm \sqrt{3}}{3}$}
	{$m=\dfrac{5\pm \sqrt{5}}{3}$}
	{\True $m=\dfrac{2\pm \sqrt{22}}{3}$}
	\loigiai{
		Phương trình hoành độ giao điểm của $(C)$ và trục hoành là
		\begin{eqnarray*}
			& & x^3-3mx^2+\left(3m-1\right)x+6m=0\\
			&\Leftrightarrow & \left(x+1\right)\left(x^2-\left(3m+1\right)x+6m\right)=0\\
			&\Leftrightarrow & \hoac{& x=-1=x_3 \\
				& g(x)=x^2-\left(3m+1\right)x+6m=0.\quad (*)
			}
		\end{eqnarray*}
		Điều kiện để $(C)$ cắt trục hoành tại ba điểm phân biệt có hoành độ $x_1, x_2, x_3$ là $(*)$ có $2$ nghiệm phân biệt khác $-1$. Khi đó ta có
		\[ \heva{& \Delta >0 \\
				& g\left(-1\right)\ne 0}\Leftrightarrow \heva{& 9m^2-18m+1>0 \\
				& 9m+2\ne 0}\Leftrightarrow \heva{& \hoac{& m<\dfrac{3-2\sqrt{2}}{3} \\
					& m>\dfrac{3+2\sqrt{2}}{3}} \\
				& m\ne -\dfrac{2}{9}
				.}\]
		Khi đó
		\begin{eqnarray*}
			& & x_1^2+x_2^2+x_3^2+x_1x_2x_3=20\\
			&\Leftrightarrow & x_1^2+x_2^2-x_1x_2=19\\
			&\Leftrightarrow & \left(x_1+x_2\right)^2-3x_1x_2-19=0\\
			&\Leftrightarrow & \left(3m+1\right)^2-18m-19=0\\
			&\Leftrightarrow &9m^2-12m-18=0\\
			&\Leftrightarrow & m=\dfrac{2\pm \sqrt{22}}{3} \text{ (thỏa mãn điều kiện)}.
		\end{eqnarray*}
	}
\end{ex}

\begin{ex}%[BG-12NEW-4in1, Nguyen Huynh]%[2D1H4-1]
	Đồ thị của hàm số nào dưới đây \textbf{không} có tiệm cận ngang?
	\choice
	{$y=3^x$}
	{$y=\dfrac{\sqrt{x^2+1}}{2x+3}$}
	{\True $y=\log_3x$}
	{$y=\dfrac{1}{1+x}$}
	\loigiai{
		Hàm số $y=\log_3x$ có tập xác định $(0; +\infty)$ và $\displaystyle\lim\limits_{x\to +\infty}y=+\infty$ nên đồ thị hàm số không có tiệm cận ngang.\\
		Hàm số $y=3^x$ có tập xác định $(-\infty; +\infty)$ và $\displaystyle\lim\limits_{x\to -\infty}y=0$ nên đồ thị hàm số có tiệm cận ngang là đường thẳng $y=0$.\\
		Hàm số $y=\dfrac{1}{1+x}$ có tập xác định $(-\infty;-1)\cup (-1; +\infty)$ và $\displaystyle\lim\limits_{x\to +\infty}y=\displaystyle\lim\limits_{x\to -\infty}y=0$ nên đồ thị hàm số có tiệm cận ngang là đường thẳng $y=0$.\\
		Hàm số $y=\dfrac{\sqrt{x^2+1}}{2x+3}$ có tập xác định $\left(-\infty;-\dfrac{3}{2}\right)\cup \left(-\dfrac{3}{2}; +\infty\right)$ và $\displaystyle\lim\limits_{x\to +\infty}y=\dfrac{1}{2}$ và $\displaystyle\lim\limits_{x\to -\infty}y=-\dfrac{1}{2}$ nên đồ thị hàm số có $2$ đường tiệm cận ngang là đường thẳng $y=-\dfrac{1}{2}$ và $y=\dfrac{1}{2}$.
	}
\end{ex}

\begin{ex}%[MĐ2]%[2D1H2-6]
	Hàm số $y=\ln \left(x^3-3x^2+1\right)$ có bao nhiêu điểm cực trị?
	\choice
	{$ 2 $}
	{$ 3 $}
	{$ 0 $}
	{\True $ 1 $}
	\loigiai
	{
		Điều kiện xác định $ x^3-3x^2+1>0 $\\
		Ta có $ y'=\dfrac{3x^2-6x}{x^3-3x^2+1} $, $ y'=0\Leftrightarrow \hoac{& x=0 \\ & x=2 \text{ (không thỏa mãn)}.} $\\
		Ta có $ y''=\dfrac{-3x^4+12x^3-18x^2+6x-6}{\left(x^3-3x^2+1\right)^2} $,
		nên $ y''(0)=-6<0 $ do đó hàm số đạt cực đại tại $ x=0 $.\\
		Hàm số đã cho có một điểm cực trị.
	}
\end{ex}

\begin{ex}%[Mức độ N]%[2D1N1-1]
	Cho hàm số $y=f(x)$ có đạo hàm $f'(x)=x^2+1\text{, } \forall x \in \mathbb{R}$. Mệnh đề nào dưới đây đúng?
	\choice{Hàm số nghịch biến trên khoảng $(1;+\infty)$}{Hàm số nghịch biến trên khoảng $(-1;1)$}{\True Hàm số đồng biến trên khoảng $(-\infty;+\infty)$}{Hàm số nghịch biến trên khoảng $(-\infty;0)$}
	\loigiai{Vì $f'(x)=x^2+1>0,$ $\forall x \in \mathbb{R}$ nên hàm số đồng biến trên khoảng $(-\infty;+\infty)$.}
\end{ex}

\begin{ex}%[SGK 12 - CTST, Mức độ 2]%[BG12-4IN1, Nguyễn Khánh Trọng]%[2D1H3-6]
	Khi làm nhà kho, bác An muốn cửa sổ có dạng hình chữ nhật với chu vi bằng $4 \mathrm{~m}$. Tìm kích thước khung cửa sổ sao cho diện tích cửa sổ lớn nhất (để hứng được nhiều ánh sáng nhất)?
	\choice
	{$3$ m}
	{\True $1$ m}
	{$2$ m}
	{$1{,}5$ m}
	\loigiai{
		Gọi chiều dài của khung cửa sổ là $x$ (mét). Điều kiện $0<x<2$.\\
		Suy ra chiều rộng của khung cửa sổ là $2-x$ (mét).\\
		Khi đó diện tích của khung cửa sổ là $x\left(2-x\right)=-x^2+2x$.\\
		Đặt $f(x)=-x^2+2x\Rightarrow f'(x)=-2x+2=0\Leftrightarrow x=1$. Ta có bảng biến thiên như sau
		\begin{center}
			\begin{tikzpicture}[font=\normalsize,t style/.style={style=solid}]
				%dòng khai báo
				\tkzTabInit[lgt=1.2,espcl=2.5,deltacl=0.5]
				{$x$ /0.75, $f'(x)$/0.75, $f(x)$/2}
				{$ 0$,$ 1 $,$ 2$}
				%dòng xét dấu
				\tkzTabLine{  , +,0 , -,  }  % z, t, d;
				%dòng biến thiên
				\tkzTabVar{-/$0$,+/$1$,-/$0$} %+ hoac-
			\end{tikzpicture}
		\end{center}
		Như bảng biến thiên ta thấy được diện tích khung của sổ lớn nhất khi $x=1$ hay khung cửa có dạng hình vuông cạnh $1$ mét.}
\end{ex}

\begin{ex}%[Mức độ 2]%[2D1H1-5]
	Sau khi phát hiện một bệnh dịch, các chuyên gia y tế ước tính số người nhiễm bệnh kể từ ngày xuất hiện bệnh nhân đầu tiên đến ngày thứ $t$ là $f(t)=45t^2-t^3$ (kết quả khảo sát được trong 8 tháng vừa qua). Xem $f'(t)$ là tốc độ truyền bệnh (người/ngày) tại thời điểm $t$.
	\choice
	{\True Từ ngày đầu tiên đến ngày thứ 10 tốc độ truyền bệnh tăng dần}
	{Từ ngày thứ 10 đến ngày thứ 20 tốc độ truyền bệnh giảm dần}
	{Từ ngày thứ 15 đến ngày thứ 20 tốc độ truyền bệnh tăng dần}
	{Từ ngày thứ 15 đến ngày thứ 20 tốc độ truyền bệnh tăng dần rồi giảm dần kể từ ngày thứ 21}
	\loigiai
	{
		$f'(t)=90t-3t^2 \ge 0 \Rightarrow 0\le t \le 30$.\\
		$
			f''(t)=90-6t=0 \Rightarrow t=15.
		$\\
		Bảng biến thiên
		\begin{center}
			\begin{tikzpicture}[scale=1, font=\footnotesize]%<DTools>
				\tkzTabInit[nocadre=false, lgt=1.2, espcl=4, deltacl=0.6]
				{$t$/0.8,$f'(t)$/0.6,$f(t)$/2}
				{$0$,$15$,$30$};
				\tkzTabLine{,+,$0$,-,};
				\tkzTabVar{-/$0$,+/$675$,-/$0$};
			\end{tikzpicture}
		\end{center}
		Từ bảng biến thiên ta thấy từ ngày đầu tiên đến ngày thứ 10 tốc độ truyền bệnh tăng dần.
	}
\end{ex}

\begin{ex}%[CKP]giảng 12-4in1, Nhật Thiện]%[2D1H2-7]
	Một công ty tiến hành khai thác $17$ giếng dầu trong khu vực được chỉ định. Trung bình mỗi giếng dầu chiết xuất được $245$ thùng dầu mỗi ngày. Công ty	có thể khai thác nhiều hơn $17$ giếng dầu nhưng	cứ khai thác thêm một giếng thì lượng dầu mỗi giếng chiết xuất được hằng ngày sẽ giảm $9$ thùng.	Để giám đốc công ty có thể quyết định số giếng cần thêm cho phù hợp với tài chính, hãy chỉ ra số giếng công ty có thể khai thác thêm để sản lượng
	dầu chiết xuất đạt cực đại.
	\choice
	{\True $5$}
	{$3$}
	{$4$}
	{$6$}
	\loigiai{
		Gọi $x$ ($x>0$) là số giếng dầu khai thác thêm.\\
		Sản lượng dầu khi khai thác thêm $x$ giếng là $(17+x)\cdot (245-9\cdot x)$ (thùng).\\
		Xét hàm số $f(x)=(17+x)(245-9x)=-9x^2+92x+4\,165$ mô tả sản lượng dầu.\\
		Ta có $f'(x)=0\Leftrightarrow -18x+92=0\Leftrightarrow x=\dfrac{46}{9}$.\\
		Bảng biến thiên
		\begin{center}
			\begin{tikzpicture}
				\tkzTabInit[nocadre=false,lgt=1.2,espcl=2.5,deltacl=0.6]
				{$x$ /0.6,$f’(x)$ /0.6,$f(x)$ /2}
				{$0$,$\tfrac{46}{9}$,$+\infty$}
				\tkzTabLine{,+,0,-,}
				\tkzTabVar{-/,+/$\dfrac{39\,601}{9}$,-/}
			\end{tikzpicture}
		\end{center}
		Dựa vào bảng biến thiên, để sản lượng dầu chiết suất đạt cực đại, công ty có thể khai thác thêm $5$ giếng dầu.
	}
\end{ex}

\begin{ex}%[Mức độ 3]giảng 12, Phạm Tiến Long]%[2D1V4-3]
	Gọi $d$ là tiệm cận xiên của đồ thị hàm số $f(x)=\dfrac{mx^2+nx+1}{x-1}$, với $m$, $n$ là tham số. Biết rằng $d$ song song với đường thẳng $\Delta \colon y=3x+2$ và đi qua điểm $M(-1;4)$. Khi đó $m+n$ bằng
	\choice
	{$5$}
	{$6$}
	{\True $7$}
	{$8$}
	\loigiai{	Hàm số đã cho có tập xác định $\mathscr{D}=\mathbb{R}\backslash\{1\}$.\\
		Ta có $\begin{aligned}[t]
				a & =\lim\limits_{x \rightarrow+\infty} \dfrac{f(x)}{x}=\lim\limits_{x \rightarrow+\infty} \dfrac{mx^2+nx+1}{x^2-x}=m;                                                                  \\
				b & =\lim\limits_{x \rightarrow+\infty}[f(x)-ax]=\lim\limits_{x \rightarrow+\infty}\left(\dfrac{mx^2+nx+1}{x-1}-mx\right)=\lim\limits_{x \rightarrow+\infty} \dfrac{(m+n)x+1}{x-1}=m+n.
			\end{aligned}$\\
		Ta cũng có $\lim\limits_{x \rightarrow-\infty} \dfrac{f(x)}{x}=m$; $\lim\limits_{x \rightarrow-\infty}[f(x)-x]=m+n$.\\
		Do đó, tiệm cận xiên của đồ thị hàm số là đường thẳng $d\colon y=mx+m+n$.\\
		Vì $d$ song song với đường thẳng $\Delta \colon y=3x+2$ và đi qua điểm $M(-1;4)$ nên ta có
		\[\heva{&m=3\\&-m+m+n=4\\&m+n\ne 2}\Leftrightarrow \heva{&m=3\\&n=4.}\]
		Vậy $m+n=7$.
	}
\end{ex}

\begin{ex}%[Mức độ 1]%[BG12-4IN1, Nguyễn Khánh Trọng]%[2D1N3-4]
	\immini[thm]{
		Cho hàm số $f(x)$ liên tục trên đoạn $[-1;3]$ và có đồ thị như hình vẽ bên. Có bao nhiêu giá trị nguyên dương của tham số $m$ để bất phương trình $f(x)\ge m$ có nghiệm trên $[-1;2]$.
		\choice
		{$3$}
		{\True $2$}
		{$1$}
		{$0$}
	}
	{\begin{tikzpicture}[scale=0.8, font=\footnotesize, line join=round, line cap=round, >=stealth]
			\draw[->] (-2.1,0)--(3.5,0) node[above left] {$x$};
			\draw[->] (0,-2.5)--(0,4.0) node[below right] {$y$};
			\draw (0,0) node [below right] {$O$};
			\foreach \x in {-2,-1,1,2,3}
			\draw[thin] (\x,1pt)--(\x,-1pt) node [below] {$\x$};
			\foreach \y in {-2,-1,1,2,3}
			\draw[thin] (1pt,\y)--(-1pt,\y) node [above left] {$\y$};
			\draw[dashed,thin](2,0)--(2,-2)--(0,-2);
			\draw[dashed,thin](-1,0)--(-1,1)--(0,1);
			\draw[dashed,thin](3,0)--(3,3)--(0,3);
			\draw[line width = 0.5pt] (2,-2)--(3,3);
			\begin{scope}
				\clip (-3,-3) rectangle (4,3.5);
				\draw[samples=200,domain=-1:2,smooth,variable=\x] plot (\x,{-1*(\x)^2+0*(\x)+2});
			\end{scope}
		\end{tikzpicture}
	}
	\loigiai{
		Dựa vào đồ thị ta có $\max \limits_{[-1; 2]} f(x)=f(0)=2$.\\
		Bất phương trình $f(x)\ge m$ có nghiệm trên $[-1;2]$ khi và chỉ khi
		\[\max\limits_{[-1;2]}f(x)\ge m\Leftrightarrow 2\ge m.\]
		Suy ra $m\in\{1;2\}$. Vậy có $2$ giá trị nguyên dương của $m$ thỏa mãn.}
\end{ex}

\begin{ex}%[Dự án BG 4in1, Nguyễn Văn Nay]%[2D1H3-2]
	\immini
	{Cho hàm số $f(x)$ có đạo hàm là $f'(x)$. Đồ thị của hàm số $y=f'(x)$ cắt $Ox$ tại các điểm có hoành độ bằng $0,2$ như hình vẽ. Biết $f(2)+f(4)=f(3)+f(0)$. Giá trị nhỏ nhất của $f(x)$ trên $[0;4]$ là
		\choice
		{ $f(1)$}
		{\True $f(4)$}
		{ $f(2)$}
		{ $f(0)$}
	}
	{\begin{tikzpicture}[scale=0.75, font=\footnotesize,line join=round, line cap=round,>=stealth]
			\draw[->] (-2.5,0)--(0,0)node[below right]{$O$}--(5.5,0) node[above]{$x$};
			\draw[->] (0,-1.6)--(0,2.5) node [left]{$y$};
			\begin{scope}
				\clip (-3.5,-1.5) rectangle (7.5,2.5);
				\draw[smooth,samples=100] plot[domain=-3.5:7.5](\x,{-(0.5*\x-1)^2+1});
			\end{scope}
			\draw (2,0)node[below]{$1$}(4,0)node[below]{$2$}(4,0)node[below]{$2$};
			\fill[black] (2,0) circle (2pt);
			\fill[black] (4,0) circle (2pt);
			\fill[black] (0,0) circle (2pt);
		\end{tikzpicture}}
	\loigiai{
		Ta có bảng biến thiên của hàm số
		\begin{center}
			\begin{tikzpicture}[scale=0.7, font=\footnotesize, line join=round, line cap=round, >=stealth]
				\tkzTabInit[nocadre=false,lgt=1.5,espcl=2.5,deltacl=0.6]
				{$x$ /0.6,$y'$ /0.6,$y$ /1.8}
				{$-\infty$,$0$,$2$,$+\infty$}
				\tkzTabLine{,-,$0$,+,$0$,-,}
				\tkzTabVar{+/$+\infty$ , -/$f(0)$ , +/$f(2)$ , -/$-\infty$}
			\end{tikzpicture}
		\end{center}
		Từ bảng biến thiên ta thấy hàm số đồng biến trên $[0;2]$, hàm số nghịch biến trên $[2;4]$ \[do \]vậy ta có
		\[\heva{&f(0)<f(2) \\&f(2)>f(3)>f(4)}\Rightarrow\heva{&f(3)-f(2)<0\\&f(4)-f(0)=f(3)-f(2)<0 }\Rightarrow f(4)<f(0)\Rightarrow\heva{&f(2)>f(3)>f(4)\\&f(2)>f(0)>f(4).}\]
		Vậy $\max\limits_{[0;4]}f(x)=f(4)$.}
\end{ex}

\begin{ex}%[Mức độ 2]%[2D1H1-5]
	\immini{
		Một vật được ném từ mặt đất lên trời xiên góc $\alpha$ so với phương nằm ngang với vận tốc ban đầu $v_0=9$ m/s (Hình vẽ). Khi đó quỹ đạo chuyển động của vật tuân theo phương trình $y=\dfrac{-g}{2v_0^2\cos^2\alpha}x^2+x\tan \alpha$, ở đó $x$ (mét) là khoảng cách vật bay được theo phương ngang từ điểm ném, $y$ (mét) là độ cao so với mặt đất của vật trong quá trình bay, $g$ là gia tốc trọng trường (theo Vật lí đại cương, Nhà xuất bản Giáo dục Việt Nam, $2016$).
	}{
		\begin{tikzpicture}[scale=1.1, font=\footnotesize, line join=round, line cap=round, >=stealth]
			\draw[->](-0.5,0)--(3.5,0) node[below]{$x$};
			\draw[->](0,-0.5)--(0,3) node[left]{$y$};
			\draw[fill=black] (0,0) circle(1pt) node[above left]{$O$};
			\draw[black,samples=200,domain=0:3,smooth,variable=\x] plot (\x,{-1*((\x)^2)+3*(\x)});
			\path
			(0,0) coordinate (O)
			(1,0) coordinate (A)
			(0.5,1.5) coordinate (B)
			;
			\draw[->] (O)--(B) node[above]{$\overrightarrow{v}$};
			\draw[black] pic["$\alpha$", draw=black, angle eccentricity=0.5, angle radius=0.6cm]
				{angle=A--O--B};
			%					\draw(1.5,-0.5) node[below]{Hình 2.10};
		\end{tikzpicture}
	}
	Khi góc $\alpha=60^\circ$, thì $y$ đồng biến trên khoảng nào? (giả sử gia tốc trọng trường là $g=9{,}8$ m/s$^2$).
	\choice
	{\True $(0;3{,}58)$}
	{$(3{,}58;5)$}
	{$(0;4)$}
	{$(0;+\infty)$}
	\loigiai{
		Đồ thị là đường parabol có đỉnh tại $x=-\dfrac{b}{2a}=-\dfrac{\tan \alpha}{\dfrac{-g}{v_0^2\cos^2\alpha}}=\dfrac{v_0^2\cos^2\alpha \tan \alpha}{g}\approx 3{,}58$.

	}
\end{ex}

\begin{ex}%[BG12new-4in1, Trần Hoà]%[2D1H1-1]
	Cho hàm số $y=\dfrac{3-x}{x+1}$. Mệnh đề nào sau đây đúng?
	\choice
	{\True Hàm số nghịch biến trên khoảng $(-\infty;-1)$}
	{Hàm số nghịch biến trên $\mathbb{R}$}
	{Hàm số đồng biến trên khoảng $(-\infty;-1)$}
	{Hàm số đồng biến trên $\mathbb{R}$}
	\loigiai{
		Tập xác định của hàm số là $\mathscr D =\mathbb{R} \setminus \{-1\}$. Ta có $y'=-\dfrac{4}{(x+1)^2}<0,~\forall x \neq -1$.\\ Do đó, hàm số đã cho nghịch biến trên mỗi khoảng $(-\infty;-1)$, $(-1;+\infty)$.
	}
\end{ex}

\begin{ex}%[BG-12NEW-4in1, Nguyen Huynh]%[2D1N4-1]
	Tiệm cận đứng của đồ thị hàm số $y=\dfrac{x+2}{x+1}$ là
	\choice
	{\True $x=-1$}
	{$x=-2$}
	{$x=1$}
	{$x=2$}
	\loigiai{
		Tập xác định của hàm số là $\mathbb{R}\setminus\{-1\}$.\\
		Ta có
		\begin{itemize}
			\item $\lim\limits_{x\to (-1)^+}\dfrac{x+2}{x+1}=+\infty$;\\
			\item $\lim\limits_{x\to (-1)^-}\dfrac{x+2}{x+1}=-\infty$.
		\end{itemize}
		Vậy $=x-1$, là tiệm cận đứng của đồ thị.}
\end{ex}

\begin{ex}%[Dự án Giảng 12 Nhóm Toán & LaTex, Lê Minh Thiện Anh]%[2D1H5-1]
	\immini{Cho hàm số $y=a{x^3}+b{x^2}+cx+d$ $(a,\,b,\,c,\,d \in \mathbb{R})$ có bảng biến thiên như hình bên.
	Có bao nhiêu số dương trong các số $a,\,b,\,c,\,d$ ?
	\choice
	{\True $2$}
	{$4$}
	{$1$}
	{$3$}}
	{\begin{tikzpicture}[scale=1]
		\tkzTabInit[nocadre=false,lgt=1.2,espcl=2.5,deltacl=0.6]
		{$x$ /.6,$y'$/.6,$y$/2.5}{$-\infty$,$0$,$4$,$+\infty$}
		\tkzTabLine{,+,0,-,0,+,}
		\tkzTabVar{-/$-\infty$,+/$3$,-/$-5$,+/$+\infty$}
		%\tkzTabVar{+/$+\infty$,-/$-3$,+/$2$,-/$-\infty$}
	\end{tikzpicture}}
	\loigiai{
		Từ bảng biến thiên, ta có\\
		$\heva{
				&f(0)=3\\
				&f(4)=-5\\
				&f'(0)=0\\
				&f'(4)=0}\Leftrightarrow\heva{
				&d=3\\
				&64a+16b+4c+d=-5\\
				&c=0\\
				&48a+8b+c=0}\Leftrightarrow\heva{
				&a=\dfrac{1}{4}\\
				&b=-\dfrac{3}{2}\\
				&c=0\\
				&d=3.}$\\
		Vậy trong các số $ a,b,c,d$ có 2 số dương.
	}
\end{ex}

\begin{ex}%[TEX NBV, Phạm Hoài]%[2D1N1-2]
	\immini[thm]{
		Biết hàm số $y=\dfrac{x+a}{x+1}$ ($a$ là số thực cho trước, $a\neq 1$ có đồ thị như hình bên). Mệnh đề nào dưới đây đúng?
		\choice
		{$y'<0, \,\forall x\neq -1$}
		{\True  $y'>0, \,\forall x\neq -1$}
		{$y'<0, \,\forall x\in \mathbb{R}$}
		{$y'>0, \,\forall x\in \mathbb{R}$}
	}{\begin{tikzpicture}[line join=round, line cap=round,>=stealth,thick,scale=0.75]
			\tikzset{every node/.style={scale=0.9}}
			\draw[->] (-4.1,0)--(4.2,0) node[below left] {$x$};
			\draw[->] (0,-4.1)--(0,5.2) node[below left] {$y$};
			\draw (0,0) node [below left] {$O$};
			\foreach \x/\nx in {1/1,2/2,3/3}
			\draw[thin] (\x,1pt)--(\x,-1pt) node [below] {$\nx$};
			\foreach \x/\nx in {-1/-1,-2/-2}
			\draw[thin] (\x,1pt)--(\x,-1pt) node [above right] {$\nx$};
			\foreach \x/\nx in {-4/-4,-3/-3}
			\draw[thin] (\x,1pt)--(\x,-1pt) node [above] {$\nx$};
			\foreach \y/\ny in {-1/-1,1/1,2/2,3/3,4/4}
			\draw[thin] (1pt,\y)--(-1pt,\y) node [left] {$\ny$};
			\foreach \y/\ny in {-4/-4,-3/-3,-2/-2}
			\draw[thin] (1pt,\y)--(-1pt,\y) node [right] {$\ny$};
			%\draw[dashed,thin](2,0)--(2,-6)--(0,-6);
			%\draw[dashed,thin] (1.01,-10)--(1.01,2);
			\begin{scope}
				\clip (-4,-4) rectangle (4,5);
				\draw[samples=200,domain=-5:-1.1,smooth,variable=\x] plot (\x,{(-1*(\x)-3)/(2*(\x)+2)});
				\draw[samples=200,domain=-.7:5,smooth,variable=\x] plot (\x,{(-1*(\x)-3)/(2*(\x)+2)});
				\draw (-1,-4)--(-1,5) (-5,-0.5)--(5,-0.5);
			\end{scope}
		\end{tikzpicture}}
	\loigiai{Dựa vào đồ thị, hàm số đồng biến trên từng khoảng xác định. Do đó $y'>0\, \forall x\ne -1$ suy ra $1-a>0\Rightarrow a<1$.
	}
\end{ex}

\begin{ex}%[BG12, Tran Tony]%[2D1H2-2]
	\immini{
		Cho hàm số bậc ba $y=f(x)$ có đồ thị như hình vẽ. Số điểm cực trị của hàm số $y=|f(x)|$ là
		\choice
		{$3$}
		{$2$}
		{$4$}
		{\True $5$}
	}
	{
		\begin{tikzpicture}[scale=0.5, font=\footnotesize, line join=round, line cap=round, >=stealth,yscale=0.7]
			\draw[->] (-3.5,0)--(5,0) node[below]{$x$} ;
			\draw[->] (0,-3)--(0,4) node[left]{$y$};
			\draw[fill=black] (0,0) circle(1pt) node[below right=-2pt] {$O$} ;
			\clip (-3.5,-3) rectangle (5,4) ;
			\draw[smooth, samples=100, domain=-3.5:5] plot(\x,{(\x-1.3)*(\x-1.3)*(\x-1.3) - 3*(\x-1.3)}) ;
		\end{tikzpicture}
	}
	\loigiai{
		\immini{
			Từ đồ thị hàm số $y=f(x)$, ta suy ra đồ thị hàm số $y=|f(x)|$ như hình vẽ bên. Dễ thấy hàm số $y=|f(x)|$ có $5$ điểm cực trị.
		}
		{
			\begin{tikzpicture}[scale=0.5, font=\footnotesize, line join=round, line cap=round, >=stealth]
				\draw[->] (-3.5,0)--(5,0) node[below]{$x$} ;
				\draw[->] (0,-1)--(0,4) node[left]{$y$};
				\draw[fill=black] (0,0) circle(1pt) node[below right=-2pt] {$O$} ;
				\clip (-3.5,-1.5) rectangle (5,4) ;
				\draw[smooth, samples=100, domain=-3.5:-0.432051] plot(\x,{-(\x-1.3)*(\x-1.3)*(\x-1.3) + 3*(\x-1.3)}) ;
				\draw[smooth, samples=100, domain=-0.432051:1.3] plot(\x,{(\x-1.3)*(\x-1.3)*(\x-1.3) - 3*(\x-1.3)}) ;
				\draw[smooth, samples=100, domain=1.3:3.03205] plot(\x,{-(\x-1.3)*(\x-1.3)*(\x-1.3) + 3*(\x-1.3)}) ;
				\draw[smooth, samples=100, domain=3.03205:5] plot(\x,{(\x-1.3)*(\x-1.3)*(\x-1.3) - 3*(\x-1.3)}) ;
			\end{tikzpicture}
		}
	}
\end{ex}

\begin{ex}%[Dự án TL12New-4in1-NCT]%[2D1V4-2]
	Cho hàm số $y=\dfrac{2mx+m}{x-1}$. Tìm tất cả các giá trị của tham số $m$ để đường tiệm cận đứng, tiệm cận ngang của đồ thị hàm số cùng hai trục tọa độ tạo thành một hình chữ nhật có diện tích bằng $8$.
	\choice
	{$m\neq\pm 2$}
	{$m=\pm\dfrac{1}{2}$}
	{$m=2$}
	{\True $m=\pm 4$}
	\loigiai{
		Đồ thị hàm số có đường TCĐ là $x=1$ và đường TCN là $y=2m$.\\
		Diện tích hình chữ nhật tạo bởi hai đường tiện cận và hai trục tọa độ có diện tích bằng $8$ khi và chỉ khi \[1\cdot|2m|=8\Leftrightarrow m=\pm 4.\]}
\end{ex}

\begin{ex}%[2D1H5-7]
	Khoảng cách giữa hai điểm cực trị của đồ thị hàm số $y=\dfrac{x^2+x+1}{x+1}$ bằng
	\choice
	{\True $2\sqrt{5}$}
	{$2\sqrt{3}$}
	{$3\sqrt{2}$}
	{$5\sqrt{2}$}
	\loigiai{
		Tập xác định $\mathscr{D}=\mathbb{R}\setminus\{-1\}$.\\
		Ta có $y'=\dfrac{x^2+2x}{(x+1)^2}$ và $y'=0\Leftrightarrow x^2+2x=0\Leftrightarrow\hoac{& x=0 \\ & x=-2.}$
		\begin{center}
			\begin{tikzpicture}
				\tkzTabInit[nocadre=false, lgt=1.2, espcl=2.5, deltacl=0.6]{$x$/0.6,$y'$/0.6,$y$/2}
				{$-\infty$, $-2$, $-1$, $0$, $+\infty$}
				\tkzTabLine {,+,0,-,d,-,0,+,}
				\tkzTabVar{-/$-\infty$, +/$-3$, -D+/$-\infty$ /$+\infty$,-/$1$, +/$+\infty$}
			\end{tikzpicture}
		\end{center}
		Từ bảng biến thiên ta có tọa độ hai điểm cực trị là $A(-2;-3)$ và $B(0;1)$.\\
		Vậy khoảng cách giữa hai điểm cực trị là $AB=2\sqrt{5}$.
	}
\end{ex}

\begin{ex}%[Mức độ N]%[2D1N4-1]
	Cho hàm số $y=f(x)$ có bảng biến thiên như sau
	\begin{center}
		\begin{tikzpicture}
			\tkzTabInit[nocadre=false, lgt=1.5,espcl=4.5]
			{$x$/1,$f'(x)$/1,$f(x)$/2}
			{$-\infty$,$1$, $+\infty$}
			\tkzTabLine{,,d,,}
			\tkzTabVar{-/$3$, +D-/$+\infty$/$\--\infty$/,+/$3$/}
		\end{tikzpicture}
	\end{center}
	Tiệm cận đứng của đồ thị hàm số đã cho có phương trình là
	\choice
	{$x=-1$}
	{$x=-3$}
	{$x=3$}
	{\True $x=1$}
	\loigiai{Ta có $\lim\limits_{x \to 1^-}y=+\infty$; $\lim\limits_{x \to 1^+}y=-\infty$.\\
		Vậy tiệm cận đứng của đồ thị hàm số đã cho có phương trình là $x=1$.}
\end{ex}

\begin{ex}%[Mức độ 1]%[2D1N1-5]
	Đồ thị dưới mô tả sự thay đổi độ cao của một máy bay. Độ cao của máy bay giảm trong khoảng thời gian nào?
	\begin{center}
		\begin{tikzpicture}
			\pgfplotsset{/pgf/number format/use comma}
			\begin{axis}[
					title={Sự thay đổi độ cao của máy bay theo thời gian},
					title style={at={(1,1.2)},anchor=north east},
					xlabel={Thời gian (phút)},
					ylabel={Độ cao (mét)},
					xmin=0, xmax=100,
					ymin=0, ymax=12500,
					xtick={0,20,40,60,80,100},
					ytick={0,2500,5000,7500,10000,12500},
					%			yticklabel style={/pgf/number format/sci},
					legend pos=north west,
					ymajorgrids=true,
					grid style={dashed,black},
				]
				\addplot[
					color=black,
					domain=0:100,
					smooth
				]
				{500*x - 5*x^2};
			\end{axis}
		\end{tikzpicture}
	\end{center}
	\choice
	{$(0;50)$}
	{\True $(50;100)$}
	{$(0;100)$}
	{$(40;60)$}
	\loigiai{
		Từ đồ thị ta thấy độ cao máy bay giảm trong khoảng thời gian $(50;100)$ phút.
	}
\end{ex}

\begin{ex}%[2D1H5-3]
	\immini{Cho hàm số $y=f(x)$ có đồ thị như hình vẽ bên cạnh. Tìm $m$ để phương trình $f(x)=m$ có bốn nghiệm phân biệt.
		\choice
		{$-4<m\le -3$}
		{\True $-4<m<-3$}
		{$-4\le m<-3$}
		{$m>-4$}
	}{\begin{tikzpicture}[line cap=round,line join=round,>=stealth,x=1cm,
				y=1cm]
			% Vẽ 2 trục, điền các số lên trục
			\draw[->] (-3.08,0) -- (3.06,0); %Vẽ trục Ox
			\foreach \x in {1,-1} %Đánh số trên trục
			\draw[shift={(\x,0)},color=black] (0pt,2pt) -- (0pt,-2pt)
			node[above] { $\x$};
			\draw[->,color=black] (0,-5.06) -- (0,0.98); %Vẽ trục Oy
			\foreach \y in {-3,-4} %đánh số trên trục
			\draw[shift={(0,\y)},color=black] (2pt,0pt) -- (-2pt,0pt)
			node[above left] {\normalsize $\y$};
			\draw[color=black] (3,.2) node[right] {\normalsize $x$}; %đặt tên trục
			\draw[color=black] (.1,0.8) node[right] {\normalsize $y$}; %đặt tên trục
			\draw[color=black] (0pt,-8pt) node[right] {\normalsize $O$}; %gốc tọa độ
			\clip(-3.08,-4.06) rectangle (2.06,0.98); %cắt khung đồ thị
			%Vẽ đồ thị
			\draw[smooth,samples=100,domain=-2.08:2.06]
			plot(\x,{(\x)^4-2*(\x)^2-3}); %Vẽ đồ thị
			% Vẽ thêm mấy cái râu ria
			\draw[dashed] (-1,0)--(-1,-4)--(1,-4)--(1,0);
			%Vẽ dấu chấm tròn
			\fill (0cm,0cm) circle (1.5pt);
		\end{tikzpicture}}
	\loigiai{
		Từ đồ thị ta thấy  phương trình $f(x)=m$ có bốn ngiệm phân biệt khi $-4<m<-3$.}
\end{ex}

\begin{ex}%[2D1H2-7]
	Giả sử chi phí tiền xăng C (đồng) phụ thuộc tốc độ trung bình  $v\left( {{\rm{\;km}}/{\rm{h}}} \right)$ theo công thức
	\[C(v) = \dfrac{16000}{v} + \dfrac{5}{2}v \quad (0 < v \le 120)\]
	Tính tốc độ trung bình để chi phí tiền xăng đạt cực tiểu.
	\choice
	{$60$ km/h}
	{$70$ km/h}
	{$50$ km/h}
	{\True $80$ km/h}
	\loigiai{
		Tập xác định: $D=(0; 120]$.\\
		Đạo hàm $C'(v)=-\dfrac{16000}{v^2}+\dfrac{5}{2}=\dfrac{5(v-80)(v+80)}{2v^2}$; $C'(v)=0\Leftrightarrow v=-80$ (loại) hoặc $v=80$.\\
		Bảng biến thiên
		\begin{center}
			\begin{tikzpicture}[>=stealth]
				\tkzTabInit[nocadre=false,lgt=1.5,espcl=2,deltacl=0.5]{$v$/.7 ,$C'(v)$/.7,$C(v)$/2}
				{$0$ , $80$ , $120$}
				\tkzTabLine{ d, - , $0$ , + , }
				\tkzTabVar{+D+/$ $/$+\infty$ , -/$400$ , +/$\dfrac{1300}{3}$}
			\end{tikzpicture}
		\end{center}
		Quan sát bảng biến thiên, ta nhận thấy hàm số đạt cực tiểu khi $v=80$.\\
		Như vậy, để chi phí tiền xăng đạt cực tiểu, tài xế nên chạy xe với tốc độ trung bình là $80$ km/h.
	}
\end{ex}

\begin{ex}%[Mức độ 3]%[BG12-4IN1, Nguyễn Khánh Trọng]%[2D1V3-6]
	Ông An dự định làm một cái bể chứa nước hình trụ bằng inox có nắp đậy với thể tích là $k$ m$^3$ $(k>0)$. Chi phí mỗi m$^2$ đáy là $600$ nghìn đồng, mỗi m$^2$ nắp là $200$ nghìn đồng và mỗi m$^2$ mặt bên là $400$ nghìn đồng. Hỏi ông An cần chọn bán kính đáy của bể là bao nhiêu để chi phí làm bể là ít nhất? (Biết bề dày vỏ inox không đáng kể)
	\choice
	{$\sqrt[3]{\dfrac{k}{\pi}}$}
	{$\sqrt[3]{\dfrac{2\pi}{k}}$}
	{\True $\sqrt[3]{\dfrac{k}{2\pi}}$}
	{$\sqrt[3]{\dfrac{k}{2}}$}
	\loigiai{
		\immini{
			Gọi $r, h$ $(r,h>0)$ lần lượt là bán kính đáy và chiều cao của hình trụ.\\
			Thể tích khối trụ $V=\pi r^2h=k\Rightarrow h=\dfrac{k}{\pi r^2}$.\\
			Diện tích đáy và nắp là $S_d=S_n=\pi r^2$; diện tích xung quanh là $S_{xq}=2\pi rh$.\\
			Khi đó chi phí làm bể là\\
			\[C=(600+200)\pi r^2+400\cdot 2\pi rh=800\pi r^2+800\pi r\dfrac{k}{\pi r^2}=800\left(\pi r^2+\dfrac{k}{r}\right).\]
		}{
			\begin{tikzpicture}[scale=0.7, font=\footnotesize, line join=round, line cap=round, >=stealth]
				\def \x{1.8} %bán kính trục lớn elip
				\def \y{0.8} %bán kính trục bé elip
				\def \h{3.5} %chiều cao hình trụ
				\coordinate (A) at (0,0);
				\coordinate (B) at (2*\x,0);
				\coordinate (O) at ($(A)!0.5!(B)$);
				\coordinate (O') at ($(O)+(0,\h)$);
				\coordinate (A') at ($(A)+(0,\h)$);
				\coordinate (B') at ($(B)+(0,\h)$);
				\draw[dashed] (B) arc(0:180:\x cm and \y cm);
				\draw (B) arc(0:-180:\x cm and \y cm);
				\draw (O') ellipse (\x cm and \y cm);
				\coordinate (I) at ($(O)+(-30:\x cm and \y cm)$);
				\tkzDrawSegments(A,A' B,B')
				\tkzDrawSegments[dashed](O,I)
				\tkzDrawPoints[fill=black,size=3](O)
				\tkzLabelSegment[above](O,I){$r$}
				\draw[<->] (-0.7,0)--(-0.7,\h);
				\node at (-0.7,\h/2) [left]{$h$};
			\end{tikzpicture}
		}
		\noindent
		Đặt $f(r)=\pi r^2+\dfrac{k}{r}$, $r>0\Rightarrow f'(r)=2\pi r-\dfrac{k}{r^2}=\dfrac{2\pi r^3-k}{r^2}$;\\
		Ta có $f'(r)=0\Leftrightarrow r=\sqrt[3]{\dfrac{k}{2\pi}}$, $(k>0)$.\\
		Lập bảng biến thiên, ta thấy $f(r)$ đạt giá trị nhỏ nhất khi $r=\sqrt[3]{\dfrac{k}{2\pi}}$.\\
		Vậy với bán kính đáy là $r=\sqrt[3]{\dfrac{k}{2\pi}}$ thì chi phí làm bể là ít nhất.}
\end{ex}

\begin{ex}%[Dự Án Giảng 12 4 in 1, Lê Văn Toàn]%[2D1CV5-6]%[2D1V5-6]
	Cho hàm số $y=\dfrac{x+2}{x+1}$ có đồ thị $(C)$. Gọi $d$ là khoảng cách từ giao điểm hai tiệm cận của đồ thị $(C)$ đến một tiếp tuyến của $(C)$. Giá trị lớn nhất của $d$ có thể đạt được là
	\choice
	{$\sqrt{3}$}
	{\True $\sqrt{2}$}
	{$3\sqrt{3}$}
	{$2\sqrt{2}$}
	\loigiai{
		Ta có $y=\dfrac{x+2}{x+1}\Rightarrow y'=\dfrac{-1}{(x+1)^2}$.\\
		Đồ thị $(C)$ của hàm số $y=\dfrac{x+2}{x+1}$ có đường tiệm cận đứng là $x=-1$ và đường tiệm cận ngang là $y=1$.\\
		Suy ra giao điểm hai đường tiệm cận là $I(-1;1)$.\\
		Lấy $M(x_0;y_0)\in (C)$ tùy ý với $x_0\ne -1$, $y_0=\dfrac{x_0+2}{x_0+1}$.\\
		Ta có tiếp tuyến của đồ thị $(C)$ tại điểm $M(x_0;y_0)$ là
		\[\Delta\colon y=\dfrac{-1}{(x_0+1)^2}(x-x_0)+y_0\Leftrightarrow \Delta\colon x+(x_0+1)^2y-x_0^2-4x_0-2=0.\]
		Khoảng cách từ điểm $I$ đến tiếp tuyến của đồ thị $(C)$ tại điểm $M(x_0;y_0)$ là
		\allowdisplaybreaks
		\begin{eqnarray*}
			d=\mathrm{d}(I,\Delta)&=&\dfrac{\left|-1+(x_0+1)^2-x_0^2-4x_0-2\right|}{\sqrt{1+(x_0+1)^4}}\\
			&=&\dfrac{2|x_0+1|}{\sqrt{1+(x_0+1)^4}}\\
			&=&\dfrac{2}{\sqrt{\dfrac{1}{(x_0+1)^2}+(x_0+1)^2}}\\
			&\le& \dfrac{2}{\sqrt{2\sqrt{\dfrac{1}{(x_0+1)^2}\cdot (x_0+1)^2}}}=\sqrt{2}.
		\end{eqnarray*}
		Dấu ``$=$'' xảy ra khi và chỉ khi $\dfrac{1}{(x_0+1)^2=(x_0+1)^2}\Leftrightarrow (x_0+1)^4=1\Leftrightarrow \hoac{&x_0=0\\&x_0=-2}$ (nhận).\\
		Vậy giá trị lớn nhất của $d$ có thể đạt được là $\sqrt{2}$.
	}
\end{ex}

\begin{ex}%[TEX NBV, Trương Đăng Khoa]%[2D1V3-6]
	\immini{Cho nửa đường tròn đường kính $AB=2$ và hai điểm $C$, $D$ thay đổi trên nửa đường tròn đó sao cho $ABCD$ là hình thang. Diện tích lớn nhất của hình thang $ABCD$ bằng
		\choice
		{ $\dfrac{1}{2}$}
		{ \True $\dfrac{3\sqrt{3}}{4}$}
		{ $1$}
		{ $\dfrac{3\sqrt{3}}{2}$
		}}{
		\begin{tikzpicture}[scale=1, font=\footnotesize, line join=round, line cap=round,>=stealth]
			\coordinate (O) at (2.5,0);
			\coordinate (A) at (0,0);
			\coordinate (B) at (5,0);
			\coordinate (D) at ($(O) + (120:2.5)$);
			\coordinate (C) at ($(O) + (60:2.5)$);
			\coordinate (I) at ($(D)!0.5!(C)$);
			\coordinate (H) at ($(A)!(D)!(O)$);
			\draw (5,0) arc (0:180:2.5);
			\draw (A)--(B)--(C)--(D)--(A) (O)--(I) (O)--node[above]{$1$}(D) (D)--node[left]{$x$}(H);
			\draw pic[draw,angle radius=0.3cm]{right angle=D--H--A};
			\draw pic[draw,angle radius=0.3cm]{right angle=I--O--B};
			\foreach \x/\y in {A/-90, B/-90, D/90,C/90, I/-45,O/-90, H/-90}{\fill (\x) circle(1.2pt) ($(\x)+(\y:0.3cm)$) node{$\x$};}
		\end{tikzpicture}
	}
	\loigiai{
		Gọi $H$ là hình chiếu vuông góc của $D$ lên $AB$, $I$ là trung điểm của đoạn $CD$ và $O$ là trung điểm của $AB$.\\
		Đặt $DH=x$, $0<x<1$.\\
		Ta có $DC=2DI=2OH=2\sqrt{OD^2-DH^2}=2\sqrt{1-x^2}$.\\
		Diện tích của hình thang $ABCD$ là $S=f( x )=\dfrac{( AB+CD )\cdot DH}{2}=\left( 1+\sqrt{1-x^2}\right)x$.\\
		Ta có $f'( x )=\dfrac{\sqrt{1-x^2}+1-2x^2}{\sqrt{1-x^2}}$.\\
		$f'( x )=0\Leftrightarrow \sqrt{1-x^2}+1-2x^2=0$. ($\ast$)\\
		Đặt $t=\sqrt{1-x^2}$, ($t\ge 0$) khi đó phương trình ($\ast$) trở thành $2t^2+t-1=0\Leftrightarrow \hoac{
				& t=-1 \text{ (loại)}\\
				& t=\dfrac{1}{2}.
			}$\\
		Khi đó  $\sqrt{1-x^2}=\dfrac{1}{2}\Leftrightarrow x^2=\dfrac{3}{4}\Leftrightarrow x=\pm \dfrac{\sqrt{3}}{2}$.\\
		Bảng biến thiên
		\begin{center}
			\begin{tikzpicture}[scale=1, font=\footnotesize]
				\tkzTabInit[nocadre=false, lgt=1.2, espcl=2, deltacl=0.6]
				{$x$/0.8,$f'(x)$/0.6,$f(x)$/2}
				{$0$,$\dfrac{\sqrt{3}}{2}$,$1$};
				\tkzTabLine{,+,$0$,-,};
				\tkzTabVar{-/$0$,+/$\dfrac{3\sqrt{3}}{4}$,-/$1$};
			\end{tikzpicture}
		\end{center}
		Vậy diện tích lớn nhất của hình thang $ABCD$ bằng $\dfrac{3\sqrt{3}}{4}$.}
\end{ex}

\begin{ex}%[BG-12NEW-4in1, Nguyen Huynh]%[2D1H4-1]
	Trong mặt phẳng $Oxy$, tổng khoảng cách từ gốc tọa độ đến tất cả các đường tiệm cận của đồ thị hàm số $y=\log_2\dfrac{2x+3}{x-1}$ bằng
	\choice
	{$2$}
	{$3$}
	{$\dfrac{5}{2}$}
	{\True $\dfrac{7}{2}$}
	\loigiai{
		Điều kiện $\dfrac{2x+3}{x-1}>0\Leftrightarrow\hoac{&x>1\\&x<-\dfrac{3}{2}.}$\\
		Ta xét các giới hạn sau
		\begin{itemize}
			\item $\lim\limits_{x\to{1^+}}\left(\log_2\dfrac{2x+3}{x-1}\right)=+\infty$.
			\item $\lim\limits_{x\to{\left(-\tfrac{3}{2}\right)^-}}\left(\log_2\dfrac{2x+3}{x-1}\right)=-\infty$.
		\end{itemize}
		Từ đó suy ra tiệm cận đứng là $d_1\colon x=-\dfrac{3}{2}$; $d_2\colon x=1$.\\
		Mặt khác $\lim\limits_{x\to+\infty}\left(\log_2\dfrac{2x+3}{x-1}\right)=\lim\limits_{x\to-\infty}\left(\log_2\dfrac{2x+3}{x-1}\right)=1$.\\
		Từ đó suy ra tiệm cận ngang là $\left(d_3\right)\colon y=1$.\\
		Ta có $T=\mathrm{d}\left(O,d_1\right)+\mathrm{d}\left(O,d_2\right)+\mathrm{d}\left(O,d_3\right)=\dfrac{3}{2}+1+1=\dfrac{7}{2}$.}
\end{ex}

\begin{ex}%[Dự án Giảng 12 Nhóm Toán & LaTex, Lê Minh Thiện Anh]%[2D1N5-1]
	\immini{Đường cong bên là đồ thị của một trong bốn hàm số đã cho sau đây. Hỏi đó là hàm số nào?
		\choice
		{$y=-x^3+x^2-2$}
		{\True $y=x^3+3x^2-2$}
		{$y=x^3-3x+2$}
		{$y=x^2-3x-2$}
	}{
		\begin{tikzpicture}[scale=0.6, font=\footnotesize,line join=round, line cap=round,>=stealth]
			\draw[->] (-3.7,0.) -- (2,0) node[below]{$x$};
			\draw[->] (0,-3) -- (0,3) node[right]{$y$};
			\fill (0,0) node[above left]{$O$};
			\fill (0,-2) circle(2pt) node[below left]{$-2$};
			\draw[smooth,samples=300,domain=-3.1:1.1] plot(\x,{(\x+2)^3-3*(\x+2)^2+2});
		\end{tikzpicture}
	}
	\loigiai{
		Dựa vào hình dáng đồ thị, ta thấy đây là đồ thị của hàm số bậc ba $y=ax^3+bx^2+cx+d$ với $a>0$ nên loại các hàm $y=x^2-3x-2$, $y=-x^3+x^2-2$.\\
		Mặt khác, đồ thị đi qua điểm $(0;-2)$ nên loại hàm $y=x^3-3x+2$.
	}
\end{ex}

\begin{ex}%[2D1N5-1]
	Bảng biến thiên sau là của hàm số nào dưới đây?
	\begin{center}
		\begin{tikzpicture}
			\tkzTabInit[nocadre=true,espcl=2.5,lgt=1.2,deltacl=0.5]
			{$x$/0.7,$y'$/0.7,$y$/2}
			{$-\infty$,$0$,$1$,$2$,$+\infty$}
			\tkzTabLine{,+,$0$,-,d,-,$0$,+,}
			\tkzTabVar{-/$-\infty$,+/$2$,-D+/$-\infty$/$+\infty$,-/$6$,+/$+\infty$}
		\end{tikzpicture}
	\end{center}
	\choice
	{$y=\dfrac{x^2-4x+2}{x-1}$}
	{\True $y=\dfrac{x^2+2x-2}{x-1}$}
	{$y=\dfrac{x^2+2x-2}{x+1}$}
	{$y=\dfrac{x^2+2}{x-1}$}
	\loigiai{
		Từ bảng biến thiên ta thấy $\heva{&\lim\limits_{x\to 1^-}y=-\infty\\&\lim\limits_{x\to 1^+}y=+\infty}\Rightarrow x=1$ là đường tiệm cận đứng nên loại đáp án \textbf{C}.\\
		Đồ thị hàm số có điểm cực đại $(0;2)$ nên loại đáp án \textbf{D}.\\
		Đồ thị hàm số có điểm cực tiểu $(2;6)$ nên loại đáp án \textbf{A}.
	}
\end{ex}

\begin{ex}%[BG-12NEW-4in1, Nguyen Huynh]%[2D1N4-1]
	Tiệm cận xiên của đồ thị hàm số $y=\dfrac{x^2+3x+5}{x+2}$ là
	\choice
	{$y=x$}
	{$y=x+1$}
	{\True $y=x+2$}
	{$y=x+3$}
	\loigiai{
		Tập xác định của hàm số là $\mathbb{R}\setminus\{-2\}$.
		Ta thấy
		\begin{itemize}
			\item $\lim\limits_{x\to +\infty}\dfrac{x^2+3x+5}{x(x+2)}=\lim\limits_{x\to +\infty}\dfrac{1+\tfrac{3}{x}+\tfrac{5}{x^2}}{1+\tfrac{2}{x}}=1$.\\
			\item $\lim\limits_{x\to+\infty}(y-x)=\lim\limits_{x\to +\infty}\dfrac{2x+3}{x+2}=2$.
		\end{itemize}
		Vậy $y=x+2$ là tiệm cận xiên của đồ thị hàm số.\\
		Tương tự, ta thấy $y=x+2$ là tiệm cận xiên của đồ thị hàm số.\\
		Vậy $y=x+2$ là tiệm cận xiên của đồ thị hàm số.

	}
\end{ex}

\begin{ex}%[BG-12NEW-4in1, Nguyen Huynh]%[2D1H4-1]
	Cho hàm số $y=a^x$ với $0<a\ne1$. Mệnh đề nào sau đây \textbf{sai}?
	\choice
	{Đồ thị hàm số $y=a^x$ và đồ thị hàm số $y=\log_ax$ đối xứng nhau qua đường thẳng $y=x$}
	{Hàm số $y=a^x$ có tập xác định là $\mathbb{R}$ và tập giá trị là $(0;+\infty)$}
	{Hàm số $y=a^x$ đồng biến trên tập xác định của nó khi $a>1$}
	{\True Đồ thị hàm số $y=a^x$ có tiệm cận đứng là trục tung}
	\loigiai{Theo lý thuyết, ta có $\lim\limits_{x\to0^+}a^x=1$ và $\lim\limits_{x\to0^-}a^x=1$ nên không nhận trục tung làm tiệm cận đứng.}
\end{ex}

\begin{ex}%[Khảo sát chất lượng L1 - THPT Nguyễn Viết Xuân - Vĩnh Phúc - 2019, Mức độ H]%[Dự án giảng 12 - Trung Anh]%[2D1H2-5]
	Biết đồ thị hàm số $y = x^4 - 2mx^2 + 2$ có ba điểm cực trị là ba đỉnh của một tam giác vuông cân. Tính giá trị của biểu thức $P = m^2 + 2m + 1$.
	\choice
	{$P = 1$}
	{\True $P = 4$}
	{$P = 2$}
	{$P = 0$}
	\loigiai{
		Tập xác định: $\mathscr{D} = \mathbb{R}$.
		$y' = 4x^3 - 4mx$.\\
		$y' = 0 \Leftrightarrow \hoac{&x = 0 \\	&x^2 = m.}$\\
		Hàm số có ba điểm cực trị $\Leftrightarrow m > 0$.\\
		Khi đó ba điểm cực trị của hàm số là $x_1 = 0$, $x_2 = \sqrt{m}$, $x_3 = -\sqrt{m}$.\\
		Vậy ba điểm cực trị của đồ thị hàm số là $A(0;2)$, $B\left(\sqrt{m}; 2 - m^2 \right)$, $C\left(-\sqrt{m}; 2 - m^2 \right)$. Ba điểm này luôn tạo thành tam giác cân tại $A$. Vậy tam giác này vuông cân khi và chỉ khi $\widehat{BAC} = 90^\circ$.\\
		Tương đương $\overrightarrow{AB} \cdot \overrightarrow{AC} = 0$, hay $\sqrt{m} \cdot \left(-\sqrt{m}\right) + (-m^2) \cdot (-m^2) = 0$.\\
		Giải phương trình này ta có $m = 1$ là nghiệm duy nhất. Do đó $P = 4$.
	}
\end{ex}
\begin{ex}%[SGK12-KNTT ]%[2D1H5-8]
	Khi máu di chuyển từ tim qua các động mạch chính rồi đến các mao mạch và quay trở lại qua các tĩnh mạch, huyết áp tâm thu (tức là áp lực của máu lên động mạch khi tim co bóp) liên tục giảm xuống. Giả sử một người có huyết áp tâm thu $P$ (tính bằng mmHg) được cho bởi hàm số
	\[
		P(t)=\dfrac{25 t^2+125}{t^2+1}, 0 \leq t \leq 10,
	\]
	trong đó thời gian $t$ được tính bằng giây. Tính tốc độ thay đổi của huyết áp sau $5$ giây kể từ khi máu rời tim.
	\choice
	{$-\dfrac{20}{17}$}
	{\True $-\dfrac{250}{169}$}
	{$-\dfrac{120}{163}$}
	{$-\dfrac{19}{132}$}
	\loigiai{
		Ta có tốc độ thay đổi của huyết áp là $P'(t)=\dfrac{-200t}{(t^2+1)^2}$.\\
		Do đó tốc độ thay đổi huyết áp sau $5$ giây là $P'(5)=-\dfrac{250}{169}$.
	}
\end{ex}
\begin{ex}%[KNTT]giảng 12 New-4in1, Toàn Phan]%[2D1H5-8]KNTT
	Khi bỏ qua sức cản của không khí, độ cao (mét) của một vật được phóng thẳng đứng lên trên từ điểm cách mặt đất $2$ m với vận tốc ban đầu $24{,}5$ m/s là $h(t)=2+24{,}5t-4{,}9t^2$ (theo Vật lí đại cương, NXB Giáo dục Việt Nam, $2016$). Tìm vận tốc của vật sau $2$ giây.
	\choice
	{\True $4{,}9$}
	{$3{,}2$}
	{$1{,}3$}
	{$5{,}5$}
	\loigiai{
	Theo ý nghĩa cơ học của đạo hàm, vận tốc của vật là $v=h'(t)=24{,}5-9{,}8t$ m/s.\\
	Do đó, vận tốc của vật sau $2$ giây là $v(2)=24{,}5-9{,}8\cdot 2=4{,}9$ m/s.
	}
\end{ex}

\TL

\begin{ex}%[SGK12-CTST, Mức độ 2]%[2D1H2-1]
	Tìm cực trị của hàm số $g(x)=\dfrac{x^2+x+4}{x+1}$.
	\loigiai{
		Tập xác định $\mathscr{D}=\mathbb{R}\setminus \{-1\}$.\\
		Ta có $g(x)=x+\dfrac{4}{x+1} \Rightarrow g'(x)=1-\dfrac{4}{(x+1)^2}=\dfrac{x^2+2x-3}{(x+1)^2}$;\\
		$g'(x)=0 \Leftrightarrow x^2+2x-3=0 \Leftrightarrow \hoac{& x=-3\\& x=1.}$\\
		Bảng biến thiên
		\begin{center}
			\begin{tikzpicture}
				\tkzTabInit[nocadre=false,lgt=1.2,espcl=2.5,deltacl=0.6]
				{$x$ /0.6,$g'(x)$ /0.6,$g(x)$ /2}
				{$-\infty$,$-3$,$-1$,$1$,$+\infty$}
				\tkzTabLine{,+,$0$,-,d,-,$0$,+,}
				\tkzTabVar{-/$-\infty$,+/$-5$,-D+/$-\infty$/$+\infty$,-/$3$,+/$+\infty$}
			\end{tikzpicture}
		\end{center}
		Vậy hàm số đạt cực đại tại $x=-3$, $y_{\text{CĐ}}=g(-3)=-5$; và hàm số đạt cực tiểu tại $x=1$, $y_{\text{CT}}=g(1)=3$.
	}
\end{ex}
\begin{ex}%[SGK12-CTST, Mức độ 2]%[2D1H1-5]
	Kim ngạch xuất khẩu rau quả của Việt Nam trong các năm từ $2010$ đến $2017$ có thể được tính xấp xỉ bằng công thức $f(x)=0{,}01x^3-0{,}04x^2+0{,}25x+0{,}44$ (tỉ USD) với $x$ là số năm tính từ $2010$ đến $2017$ $(0\leq x\leq 7)$.
	\begin{flushright}
		(\textit{Theo:} https://infographics.vn/interactive-xuat-khau-rau-qua-
		du-bao-bung-no-dat-4-ty-usd-trong-nam-2023/116220.vna)
	\end{flushright}
	\begin{enumerate}
		\item Tính đạo hàm của hàm số $y=f(x)$.
		\item Chứng minh rằng kim ngạch xuất khẩu rau quả của Việt Nam tăng liên tục trong các năm từ $2010$ đến $2017$.
	\end{enumerate}
	\loigiai{
		\begin{enumerate}
			\item Ta có $f'(x)=0{,}03x^2-0{,}08x+0{,}25$.
			\item Xét $f'(x)=0 \Leftrightarrow 0{,}03x^2-0{,}08x+0{,}25=0$ (vô nghiệm).\\
			      Bảng biến thiên
			      \begin{center}
				      \begin{tikzpicture}
					      \tkzTabInit[nocadre=false,lgt=1.2,espcl=5,deltacl=0.6]
					      {$x$ /0.6,$f'(x)$ /0.6,$f(x)$ /2}
					      {$0$,$7$}
					      \tkzTabLine{,+,}
					      \tkzTabVar{-/$0{,}44$,+/$3{,}66$}
				      \end{tikzpicture}
			      \end{center}
			      Từ bảng biến thiên trên, ta thấy $f'(x)>0$, $\forall x\in [0;7]$.\\
			      Vậy kim ngạch xuất khẩu rau quả của Việt Nam tăng liên tục trong các năm từ $2010$ đến $2017$.
		\end{enumerate}
	}
\end{ex}
\begin{ex}%[Mức độ 4]%[BG12-4IN1, Nguyễn Khánh Trọng]%[2D1C3-6]
	\immini[thm]{
		Cho một tấm gỗ hình vuông cạnh $200$ cm. Người ta cắt một tấm gỗ có hình một tam giác vuông $ABC$ từ tấm gỗ hình vuông đã cho như hình vẽ bên. Biết $AB=x$ ($0<x<60$ cm) là một cạnh góc vuông của tam giác $ABC$ và tổng độ dài cạnh góc vuông $AB$ với cạnh huyền $BC$ bằng $120$ cm. Tìm $x$ để tam giác $ABC$ có diện tích lớn nhất.
		% \shortans{$40$}
	}{
		\begin{tikzpicture}[scale=0.72, font=\footnotesize, line join=round, line cap=round, >=stealth]
			\draw[dashed] (0,0)--(4,0)--(0,1)--(0,0);
			\draw (4,0)--(5,0)--(5,5)--(0,5)--(0,1);
			\node at (0,0.5)[below left] {$x$}; \node at (2,0.5)[above,rotate=-13] {$120-x$}; \node at (2.5,5)[above] {$200$};
			\fill (0,0) circle (1.5pt) node[below left]{$A$} (4,0) circle (1.5pt) node[below]{$C$} (0,1) circle (1.5pt) node[left]{$B$};
		\end{tikzpicture}
	}

	\loigiai{
		Độ dài cạnh huyền $BC$ là $120-x$.\\
		Khi đó độ dài cạnh $AC=\sqrt{BC^2-AB^2}=\sqrt{(120-x)^2-x^2}=\sqrt{14400-240x}$.\\
		Diện tích tam giác $ABC$ là $S=\dfrac{1}{2}AB\cdot AC=\dfrac{1}{2}x\sqrt{14400-240x}$.\\
		Xét hàm số $f(x)=x\sqrt{14400-240x}$ với $0<x<60$.\\
		Ta có $f'(x)=\sqrt{14400-240x}-\dfrac{120x}{\sqrt{14400-240x}}=\dfrac{14400-360x}{\sqrt{14400-240x}}$;\\
		$f'(x)=0\Leftrightarrow x=40\in(0;60)$.\\
		Bảng biến thiên
		\begin{center}
			\begin{tikzpicture}
				\tkzTabInit[nocadre=false,lgt=1.2,espcl=2.5,deltacl=0.6]
				{$x$ /0.6,$f'(x)$ /0.6,$f(x)$ /2}
				{$0$,$40$,$60$}
				\tkzTabLine{,+,$0$,-,}
				\tkzTabVar{-/, +/,-/}
			\end{tikzpicture}
		\end{center}
		Vậy tam giác $ABC$ có diện tích lớn nhất khi $AB=40$ cm.
	}
\end{ex}
\begin{ex}%[Dự án đề ôn tập GK1, Mui Doan]%[2D1V3-6]
	Có hai xã $A$, $B$ cùng ở một bên bờ sông Lam, khoảng cách từ hai xã đó đến bờ sông lần lượt là $AA'=500$  m, $BB'=600$  m và người ta đo được $A'B'=2\,200$  m. Các kĩ sư muốn xây một trạm cung cấp nước sạch nằm bên bờ sông Lam cho dân hai xã. Để tiết kiệm chi phí, các kĩ sư cần phải chọn vị trí $M$ của trạm cung cấp nước sạch đó trên đoạn $A'B'$ sao cho tổng khoảng cách từ hai xã đến vị trí $M$ là nhỏ nhất. Hãy tìm vị trí tối ưu đó.
	% \shortans{$2460$}
	\begin{center}
		\begin{tikzpicture}
			\path
			(0:0) coordinate (A')
			(0:6) coordinate (B')
			(0:2) coordinate (M)
			($(A')+(90:2.5)$) coordinate (A)
			($(B')+(90:3)$) coordinate (B)
			;
			\fill[cyan!50] (-1.5,-1) rectangle (7.5,0);
			\draw[thick] (A')--node[left]{$500$ m}(A)--(M)--(B)--node[right]{$600$ m}(B');

			\foreach \i/\j in{A'/-100,B'/-80,A/100,B/80,M/-90}{\fill [black](\i) circle (1pt) ($(\i)+(\j:3mm)$) node {$\i$};}

			\draw [dashed,<->]	(0,.6)--(6,.6) node[pos=0.75,sloped,above]{$2\,200$ m};
		\end{tikzpicture}
	\end{center}
	\loigiai{
		\begin{center}
			\begin{tikzpicture}
				\path
				(0:0) coordinate (A')
				(0:6) coordinate (B')
				(0:2) coordinate (M)
				($(A')+(90:2.5)$) coordinate (A)
				($(B')+(90:3)$) coordinate (B)
				;
				\draw[thick] (A')--node[left]{$500 \text{(m)}$}(A)--(M)--(B)--node[right]{$600 \text{(m)}$}(B') (A')--(B');

				\foreach \i/\j in{A'/-100,B'/-80,A/100,B/80,M/-90}{\fill [black](\i) circle (1pt) ($(\i)+(\j:3mm)$) node {$\i$};}

				\draw [dashed,<->]	(0,.6)--(6,.6) node[pos=0.75,sloped,above]{$2\,200\text{(m)}$}; %Tùy chọn sloped,above,below
				\node at (3,-1.5){\it Hình 37};
			\end{tikzpicture}
		\end{center}
		Đặt $A'M=x$, $(0<x<2200)$,  $B'M=2200-x$.\\
		Ta có  $AM=\sqrt{x^2+500^2}$, $BM=\sqrt{(2200-x)^2+600^2}$.\\
		Khi đó tổng khoảng cách từ hai xã đến vị trí $M$ là $AM+BM= \sqrt{x^2+500^2}+\sqrt{(2200-x)^2+600^2} $.\\
		Xét hàm số $f(x)= \sqrt{x^2+500^2}+\sqrt{(2200-x)^2+600^2}$ trên khoảng $(0<x<2200)$.\\
		%$f(x)=\sqrt{x^2+500}+\sqrt{x^2-4400x+4840600}$ .\\
		$f'(x)=\dfrac{x}{\sqrt{x^2+500^2}}-\dfrac{2200-x}{\sqrt{(2200-x)^2+600^2}}$,
		\allowdisplaybreaks
		\begin{eqnarray*}
			f'(x)=0&\Leftrightarrow&\dfrac{x}{\sqrt{x^2+500^2}}=\dfrac{2200-x}{\sqrt{(2200-x)^2+600^2}}
			\\
			&\Leftrightarrow&\dfrac{x^2}{x^2+500^2}=\dfrac{(2200-x)^2}{(2200-x)^2+600^2}\\
			&\Leftrightarrow&\dfrac{x^2+500^2}{x^2}=\dfrac{(2200-x)^2+600^2}{(2200-x)^2}\\
			&\Leftrightarrow& 1+\dfrac{500^2}{x^2}=1+\dfrac{600^2}{(2200-x)^2}
			\\
			&\Leftrightarrow& \dfrac{25}{x^2}=\dfrac{36}{(2200-x)^2}
			\\
			&\Leftrightarrow& \dfrac{5}{x}=\dfrac{6}{2200-x}
			\\
			&\Leftrightarrow& x=1\,000~ \text{vì}~  x>0.
		\end{eqnarray*}
		Bảng biến thiên hàm số $f(x)$ trên khoảng $( 0;2\,200)$.
		\begin{center}
			\begin{tikzpicture}
				\tkzTabInit[lgt=1.2,espcl=4.5,deltacl=0.6]
				{$x$/1,$f'(x)$/1,$f(x)$/3} {$0$,$1000$,$2200$}
				\tkzTabLine{,-,0,+,}
				\tkzTabVar{+/$2780$,-/$2460$,+/$2856$}
			\end{tikzpicture}
		\end{center}
		Vậy giá trị nhỏ nhất của tổng khoảng cách từ hai xã đó đến bờ sông  là khoảng $2\,460$  m, tại vị trí $M$ cách điểm $A'$  là $1\,000$  m.
	}
\end{ex}
\Closesolutionfile{ans}

% \begin{indapan}
% 	{ans/ansc1l4}
% \end{indapan}


% \begin{name}
	{\tenchude}
	{ĐỀ ÔN TẬP CHƯƠNG I}
	{LỚP TOÁN THẦY PHÁT}
	{\thoigian}
\end{name}

\TN
\Opensolutionfile{ans}[ans/ansc101]
\begin{ex}%[Mức độ N]%[2D1N2-2]
	\immini{	Cho hàm số $y=f(x)$ có đồ thị như hình bên. Hàm số đạt cực đại tại
		\choice
		{$x=-3$}
		{\True $x=-2$}
		{$y=0$}
		{$y=4$}}{\begin{tikzpicture}[>=stealth]
			\draw [->] (-4,0)--(1.5,0);
			\draw [->] (0,-1)--(0,5);
			\draw (0,0) node[below right]{$O$};
			\draw (1.5,0) node[below]{$x$};
			\draw (0,5) node[below left]{$y$};
			\foreach \x in {-3,-2}{\draw (\x,-.1)--(\x,.1) node[below left,black]{$\x$};}
			\foreach \y in {4}{\draw [-] (-.1,\y)--(.1,\y) node[right,black]{$\y$};}
			\clip (-4,-1) rectangle (5,5);
			\draw [thick,samples=100] plot[domain=-5:5](\x,{(\x)^3+3*(\x)^2});
			\draw[dashed] (0,4) -- (-2,4) --(-2,0);
			\fill[black] (-2,4) circle(2pt);
			\draw (1,5) node[below right]{$f(x)$};
		\end{tikzpicture}}
	\loigiai{Từ hình vẽ, ta thấy hàm số $y=f(x)$ đạt cực đại tại điểm $x=-2$.}
\end{ex}

\begin{ex}%[2-D1B2-SO-5-2425]%[VN-MT-7, VM026]%[2D1N3-1]
	Cho hàm số $y=f\left( x \right)$ xác định và liên tục trên đoạn $\left[ -4;5 \right]$, có bảng biến thiên như hình sau
	\begin{center}
		\begin{tikzpicture}
			\tkzTabInit[nocadre=true,lgt=1.0,espcl=3, deltacl=0.5]
			{$x$/0.7,$y’$/0.7,$y$/2}
			{$-4$,$-2$,$4$,$5$}
			\tkzTabLine{,+,0,-,0,+,}
			\tkzTabVar{-/$\dfrac{2}{3}$,+/$\dfrac{46}{3}$,-/$-\dfrac{62}{3}$,+/$-\dfrac{52}{3}$}
		\end{tikzpicture}
	\end{center}
	Gọi $M$, $N$ lần lượt là giá trị lớn nhất, giá trị nhỏ nhất của hàm số $y=f(x)$ xác định trên đoạn $[-4;5]$. Tính $M+N$?
	\choice
	{\True $-\frac{16}{3}$}
	{$-\frac{50}{3}$}
	{$2$}
	{$-20$}
	\loigiai{
		Dựa vào bảng biến thiên, ta có $M+N=\dfrac{46}{3}-\dfrac{62}{3}=-\dfrac{16}{3}$.
	}
\end{ex}

\begin{ex}%[2D1N5-5]
	\immini{Cho hàm số $f(x)$ có đạo hàm $f’(x)$ xác định, liên tục trên $\mathbb{R}$ và $f’(x)$ có đồ thị như hình vẽ bên. Khẳng định nào sau đây là đúng?
		\choice
		{Hàm số $f(x)$ đồng biến trên $(-\infty;1)$}
		{Hàm số $f(x)$ đồng biến trên $(-\infty;1)$ và $(1;+\infty)$}
		{\True Hàm số $f(x)$ đồng biến trên $(1;+\infty)$}
		{Hàm số $f(x)$ đồng biến trên $\mathbb{R}$}}
	{\begin{tikzpicture}[>=stealth,line join=round,line cap=round,font=\normalsize,scale=0.6]
			\draw[-stealth] (-1,0)--(0,0)node[above left]{$O$}--(3,0)node[below]{$x$};
			\draw[-stealth] (0,-3)--(0,3)node[left]{$y$};
			\draw (0,-2.5)--(1,0)--(2.5,2.5);
			\draw
			(1,0) node[below right]{$1$}
			;
		\end{tikzpicture}}
	\loigiai{
		Dựa vào đồ thị hàm số $f’(x)$, ta thấy $f’(x)>0,\forall x\in(1;+\infty)$ suy ra hàm số $f(x)$ đồng biến trên $(1;+\infty)$.}
\end{ex}

\begin{ex}%[2D1N5-1]
	\immini[thm]{Đồ thị hình bên là của một trong bốn hàm số sau. Hỏi đó là hàm số nào?
		\choice
		{$y=\dfrac{x^2+x-1}{x-1}$}
		{\True $y=\dfrac{x^{2}-x+1}{x-1}$}
		{$y=\dfrac{x^2-4x-1}{-x+1}$}
		{$y=\dfrac{x^2-3x-1}{-x+1}$}}{
		\begin{tikzpicture}[line cap=butt,line join=miter,>=stealth,scale=0.57,font=\footnotesize]
			\tikzset{declare function={xmin=-3.5;xmax=4.7;ymin=-3.5;ymax=6;},
				smooth,samples=450}
			\draw[->] (xmin,0)--(xmax,0) node[shift={(0:7pt)}]{$ x $};
			\draw[->] (0,ymin-.2)--(0,ymax) node[shift={(90:7pt)}]{$ y $};
			\fill (0,0) node[shift={(140:6pt)}]{$ O $};
			\clip (xmin,ymin) rectangle (xmax,ymax);
			\foreach \i in {-3,-2,2,3,4}{
					\draw(\i,1.5pt)--(\i,-1.5pt)node[below]{$\i$};}
			\foreach \j in {-2,1,2,3,4,5}{
					\draw(-1.5pt,\j)--(1.5pt,\j) node[left]{$\j$};}
			\draw(-1.5pt,-1)--(1.5pt,-1)node[shift={(160:6.5pt)}]{$-1$};
			\draw(1,-1.5pt)--(1,1.5pt)node[shift={(-75:7pt)}]{$1$};
			\draw(-1,-1.5pt)--(-1,1.5pt)node[shift={(100:5pt)}]{$-1$};
			\def\f(#1){((#1)^2-(#1)+1)/((#1)-1)}
			\def\a{-1}
			\def\b{0}
			\def\c{0.5}
			\def\d{1.5}
			\def\e{2}
			\def\g{3}
			\pgfmathsetmacro\fa{\f(\a)}
			\pgfmathsetmacro\fb{\f(\b)}
			\pgfmathsetmacro\fc{\f(\c)}
			\pgfmathsetmacro\fd{\f(\d)}
			\pgfmathsetmacro\fe{\f(\e)}
			\pgfmathsetmacro\fg{\f(\g)}
			\draw[samples=100] plot[domain=-5.3:0.9] (\x,{\f(\x)});
			\draw[samples=100] plot[domain=1.05:5.2] (\x,{\f(\x)});
			\draw[] (1,ymin)--(1,ymax) node [pos=0.95,sloped, above]{$x=1$};
			\draw[] (xmin,ymin)--(6,ymax) node [pos=0.08,sloped, above]{$y=x$};
		\end{tikzpicture}
	}
	\loigiai{
	}
\end{ex}

\begin{ex}%[2-D1B5-SO-16-2425]%[VN-MT-7, Dương Phước Sang]%[2D1H5-1]
	Cho bảng biến thiên của hàm số $y=f(x)$ như sau:
	\begin{center}
		\begin{tikzpicture}
			\tikzset{double style/.append style={double distance=1.5pt}}
			\tkzTabInit[nocadre=true,lgt=1.2,espcl=4,deltacl=0.6]
			{$x$/0.6,$y'$/0.6,$y$/2}
			{$-\infty$,$1$,$+\infty$}
			\tkzTabLine{,-,d,-,}
			\tkzTabVar{+/$1$,-D+/$-\infty$/$+\infty$,-/$1$}
		\end{tikzpicture}
	\end{center}
	Hỏi đây là bảng biến thiên của hàm số nào trong các hàm số sau?
	\choice
	{$y=\dfrac{x-3}{x-1}$}
	{$y=\dfrac{-x+2}{x-1}$}
	{$y=\dfrac{x+2}{x+1}$}
	{\True $y=\dfrac{x+2}{x-1}$}
	\loigiai{
		Bảng biến thiên được cung cấp có đặc điểm:
		\begin{itemize}
			\item Đồ thị hàm số có đường tiệm cận đứng là $x=1$, loại $y=\dfrac{x+2}{x+1}$.
			\item Đồ thị hàm số có đường tiệm cận ngang là $y=1$, loại $y=\dfrac{-x+2}{x-1}$.
			\item $y'<0,\,\forall x \ne 1$, trong khi $\left(\dfrac{x-3}{x-1}\right)'=\dfrac{2}{(x-1)^2}>0,\,\forall x \neq 1$, loại $y=\dfrac{x-3}{x-1}$.
		\end{itemize}
		Chỉ có hàm số $y=\dfrac{x+2}{x-1}$ thỏa mãn các đặc điểm trên.
	}
\end{ex}

\begin{ex}%[BG-12NEW-4in1, Nguyen Huynh]%[2D1N4-1]
	\immini{
		Cho hàm số $y=\dfrac{2x^2}{x^2-1}$ có đồ thị là đường cong như hình vẽ bên. Số các đường tiệm cận đứng, tiệm cận ngang và tiệm cận xiên (nếu có) của đồ thị hàm số đã cho là	\choice{\True $4$}{ $2$}{ $3$}{$5$}}
	{	\begin{tikzpicture}[x=1cm,y=1cm,scale=.5]
			\draw[->] (-4,0)--(4,0)node[below right]{$x$};
			\draw[->] (0,-4)--(0,4)node[left]{$y$};
			\fill (0,0)node[above left]{ $O$};
			\draw (-4,2)--(4,2) node[below]{ $y=2$};
			\node at (-1,-4)[above left]{ $x=-1$};
			\node at (1,-4)[above right]{ $x=1$};
			\clip (-4,-4)rectangle(4,4);
			\draw[black,samples=150,smooth,domain=-3.85:3.85] plot(\x,{2*(\x)^(2)/((\x)^(2)-1)});
		\end{tikzpicture}
	}
	\loigiai{Đồ thị hàm số đã cho có tiệm cận đứng là các đường thẳng $x=-1$, $x=1$, tiệm cận ngang là đường thẳng $y=2$ và không có tiệm cận xiên}
\end{ex}

\begin{ex}%[Mức độ 3]%[2D1H2-1]
	Đồ thị của hàm số $y=x^3-3x^2-9x+1$ có hai điểm cực trị $A$ và $B$. Điểm nào dưới đây thuộc đường thẳng $AB$?
	\choice
	{$P (1;0)$}
	{$M (0;1)$}
	{\True $N(1;-10)$}
	{$Q (-1;10)$}
	\loigiai
	{
		Ta có $y'=3x^2-6x-9$.\\
		$y'=0\Leftrightarrow\hoac{&x=-1\Rightarrow y=6\\&x=3\Rightarrow y=-26.}$\\
		$\Rightarrow AB: y=-8x-2$
	}
\end{ex}

\begin{ex}%[De-chuan-hoa-so-1]%[Dương Quang]%[2D1N4-1]
	Đường tiệm cận ngang của đồ thị hàm số $y=\dfrac{2x-4}{-x+2}$ là
	\choice
	{ $y=2$}
	{ $x=2$}
	{$x=-2$}
	{\True$y=-2$}
	\loigiai{$\underset{x\to -\infty }{\mathop{\lim }}\,\dfrac{2x-4}{x+2}=-2$ và $\underset{x\to +\infty }{\mathop{\lim }}\,\dfrac{2x-4}{x+2}=-2$ nên đồ thị hàm số có tiệm cận ngang là $y=-2$.
	}
\end{ex}

\begin{ex}%[Mức độ H]%[2D1H3-4]
	Gọi $m$, $M$ lần lượt là giá trị nhỏ nhất, giá trị lớn nhất của hàm số $y=\sqrt{4-x^2}$. Tổng $m+M$ bằng
	\choice{\True $2$}{$0$}{$4$}{$1$}
	\loigiai{Tập xác định $\mathscr{D}=[-2;2]$.\\
	Ta có $y'=\dfrac{-x}{\sqrt{4-x^2} }\Rightarrow y'=0 \Leftrightarrow x = 0 \in [-2;2]$.\\
	Bảng biến thiên
	\begin{center}
		\begin{tikzpicture}
			\tkzTabInit[nocadre=false, lgt=1.5,espcl=3.5]
			{$x$/1,$y'$/1,$y$/2}
			{$-2$,$0$,$2$}
			\tkzTabLine{,+,0,-, }
			\tkzTabVar{-/$0$,+/$2$,-/$0$/}
		\end{tikzpicture}
	\end{center}
	Dựa vào bảng biến thiên ta thấy $ \underset{[-2;2]}{\text{max}}y=2; \underset{[-2;2]}{\text{min}}y=0$.\\
	Vậy $m+M=0+2=2$.}
\end{ex}

\begin{ex}%[2D1H5-3]
	Cho hàm số $y=f(x)$ xác định, liên tục trên $\mathbb{R}$ và có bảng biến thiên sau
	\begin{center}
		\begin{tikzpicture}
			\tkzTabInit[nocadre=false,lgt=0.7,espcl=2.1]
			{$x$ /0.6,$y'$ /0.6,$y$ /2}
			{$-\infty$,$-1$,$0$,$1$,$+\infty$}
			\tkzTabLine{,-,$0$,+,$0$,-,$0$,+,}
			\tkzTabVar{+/$-\infty$, -/$-1$,+/$0$,-/$-1$,+/$-\infty$}
		\end{tikzpicture}
	\end{center}
	Tìm tất cả các giá trị của tham số $m$ để phương trình $f(x)-1=m$ có đúng hai nghiệm.
	\choice
	{\True $m=-2,m>-1$}
	{$m=-2,m\ge -1$}
	{$-2<m<-1$}
	{$m>0,m=-1$}
	\loigiai{
	$f( x )-1=m\Leftrightarrow f( x )=m+1$. \\
	Dựa vào bảng biến thiên, để phương trình $f\left( x \right)-1=m$ có đúng hai nghiệm thì\\
	$\left[ \begin{aligned}
			 & m+1>0  \\
			 & m+1=-1 \\
		\end{aligned} \right.\Leftrightarrow \left[ \begin{aligned}
			 & m>-1  \\
			 & m=-2.
		\end{aligned} \right.$}
\end{ex}

\begin{ex}%[2-D1B3-SO-8-2425]%[VN-MT-7, Nguyễn Hồng Thạch]%[2D1H4-2]
	Tìm tất cả các giá trị thực của tham số $m$ để đồ thị hàm số $y=\dfrac{mx-8}{x+2}$ có hai đường tiệm cận.
	\choice
	{$m\neq 4$}
	{\True $m\neq -4$}
	{$m=4$}
	{$m=-4$}
	\loigiai{
		Ta có $x+2=0\Leftrightarrow x=-2$.\\
		Đồ thị hàm số đã cho có hai đường tiệm cận $\Leftrightarrow m\cdot(-2)-8\ne 0\Leftrightarrow m\neq -4$.}
\end{ex}

\begin{ex}%[De-chuan-hoa-so-15]%[Nguyễn Cường]%[2D1H3-6]
	Tại trường THPT Y, để giảm nhiệt độ trong các phòng học từ nhiệt độ ban đầu là $28^\circ C$, một hệ thống điều hòa làm mát được phép hoạt động trong $10$ phút. Gọi $T$ (đơn vị $^\circ C$) là nhiệt độ phòng ở phút thứ $t$ (tính từ thời điểm bật máy) được cho bởi công thức $T=-0{,}008t^3-0{,}16t+28$ $\left(t \in \left[0;10\right]\right)$. Nhiệt độ thấp nhất trong phòng có thể đạt được trong khoảng thời gian $10$ phút đó gần đúng là
	\choice
	{$27{,}832^\circ C$}
	{\True $18{,}4^\circ C$}
	{$26{,}2^\circ C$}
	{$25{,}312^\circ C$}
	\loigiai{
	Ta có $T'=-0{,}024t^2-0{,}16 < 0 \quad \forall t \in [0;10]$.\\
	$\Rightarrow T \ge T(10)=18{,}4^\circ C$.\\
	Do đó, nhiệt độ thấp nhất phòng có thể đạt được trong khoảng thời gian $10$ phút đó là $18{,}4^\circ C$.
	}
\end{ex}

\begin{ex}%[2D1V5-5]
	\immini{
		Cho hàm số $y=f(x)$ xác định trên $\mathbb{R}$, hàm số $y=f'(x)$ liên tục trên $\mathbb{R}$ và có đồ thị như hình vẽ bên. Hàm số $g(x)=f\left(3-\mathrm{e}^{x}\right)$ đồng biến trên khoảng nào dưới đây.
		\choice
		{\True$(2; 5)$}
		{$(-1; 0)$}
		{$(0; 1)$}
		{$(1; 2)$}
	}
	{
		\begin{tikzpicture}[scale=0.7, line join=round, line cap=round, >=stealth]
			\tikzset{every node/.style={scale=1}}
			\def\xmin{-2}\def\xmax{4}\def\ymin{-2}\def\ymax{3}
			\draw[->] (\xmin-0.2,0)--(\xmax+0.2,0) node[below]{$x$};
			\draw[->] (0,\ymin-0.2)--(0,\ymax+0.2) node[right]{$y$};
			\draw (0,0) node[below left]{$O$};
			\foreach \x in {1}\draw (\x,0.1)--(\x,-0.1) node[below]{$\x$};
			\foreach \x in {-1}\draw (\x,-0.1)--(\x,0.1) node[above left]{$\x$};
			\foreach \x in {2}\draw (\x,0.1)--(\x,-0.1) node[below right]{$\x$};
			\foreach \y in {-1,1}\draw (0.1,\y)--(-0.1,\y) node[left]{$\y$};
			\foreach \y in {2}\draw (0.1,\y)--(-0.1,\y) node[above left]{$\y$};
			\clip (\xmin,\ymin) rectangle (\xmax,\ymax);
			\draw[thick,smooth,samples=200,domain=\xmin:\xmax] plot(\x,{0.98*((\x)^2-1)*((\x)-2)});
		\end{tikzpicture}
	}
	\loigiai{
		Ta có $g'(x)=-\mathrm{e}^{x} \cdot f'\left(3-\mathrm{e}^{x}\right)$.\\
		Để hàm số đồng biến thì \[-\mathrm{e}^{x} \cdot f'\left(3-\mathrm{e}^{x}\right)>0 \Leftrightarrow f'\left(3-\mathrm{e}^{x}\right)<0.\]
		Dựa vào đồ thị hàm số ta được
		\[
			f'\left(3-\mathrm{e}^{x}\right)<0 \Leftrightarrow \hoac{&3-\mathrm{e}^{x}<-1\\&1<3-\mathrm{e}^{x}<2} \Leftrightarrow \hoac{&\mathrm{e}^{x}>4\\&1<\mathrm{e}^{x}<2} \Leftrightarrow \hoac{&x>\ln 4\\&0<x<\ln 2.}
		\]
		Suy ra hàm số đồng biến trên $(0;\ln 2)$ và $(\ln 4;+\infty)$.\\
		Mà $(2;5)\subset (\ln 4;+\infty)$ nên hàm số đồng biến trên $(2;5)$.
	}

\end{ex}

\begin{ex}%[Mức độ 2]%[2D1H1-5]
	Cho chuyển động thẳng xác định bởi phương trình $S=t^3-3t^2+4t$, trong đó $t$ tính bằng giây $(s)$ và $S$ được tính bằng mét $~\mathrm{(m)}$. Gia tốc của chất điểm từ thời điểm $t=1$ đến thời điểm $t=2$ giây thay đổi như thế nào?
	\choice
	{Gia tốc tăng rồi giảm}
	{Gia tốc giảm}
	{\True Gia tốc tăng}
	{Gia tốc không thay đổi}
	\loigiai
	{
		Vận tốc của chất điểm là $v=s'=3t^2-6t+4$.\\
		Gia tốc của chất điểm là $a=v'=6t-6$.\\
		Vậy từ thời điểm $t=1$ giây đến $t=2$ giây, gia tốc của vật luôn tăng.
	}
\end{ex}

\begin{ex}%[BG12, Tran Tony]%[2D1H2-2]
	\immini{
		Cho hàm số $y=f(x)$ có bảng biến thiên như hình vẽ bên. Điểm cực tiểu của hàm số $y=f(3x)$ là
		\choice
		{\True $x=\dfrac{2}{3}$}
		{$x=2$}
		{$y=-3$}
		{$x=-\dfrac{2}{3}$}
	}{
		\begin{tikzpicture}[font=\footnotesize, line join=round, line cap=round, >=stealth]
			\tkzTabInit[nocadre=false,lgt=1.2,espcl=2.2,deltacl=0.6]{$x$ /0.6,$f'(x)$ /0.6,$f(x)$ /2}{$-\infty$,$-1$,$2$,$+\infty$}
			\tkzTabLine{,+,0,-,+,}
			\tkzTabVar{-/$-\infty$,+/$4$,-/$-3$,+/$+\infty$}
		\end{tikzpicture}
	}
	\loigiai{
		Ta có $y'=3f'(3x)$, $f'(3x)=0\Leftrightarrow \hoac{&3x=-1\\&3x=2}\Leftrightarrow\hoac{&x=-\dfrac{1}{3}\\&x=\dfrac{2}{3}.}$\\
		Do $f'(x)$ và $3f'(3x)$ cùng dấu nên hàm số $y=f(3x)$ có điểm cực tiểu là $x=\dfrac{2}{3}$.
	}
\end{ex}

\begin{ex}%[2-D1B5-SO-14-2425]%[VN-MT-7, Đỗ Minh Phúc]%[2D1V3-6]
	Khi nuôi cá thí nghiệm trong hồ, một nhà khoa học đã nhận thấy rằng: nếu trên mỗi đơn vị diện tích của mặt hồ có $n$ con cá thì trung bình mỗi con cá sau một vụ cân nặng là $P(n)=800-20n$ (g). Hỏi phải thả bao nhiêu con cá trên một đơn vị diện tích của mặt hồ để sau một vụ thu hoạch được nhiều cá nhất?
	\choice
	{$19$}
	{\True $20$}
	{$21$}
	{$22$}
	\loigiai{
		Gọi $F(n)$ là hàm cân nặng của $n$ con cá sau vụ thu hoạch trên một đơn vị diện tích.\\
		Ta có $F(n)=(800-20n) \cdot n=800n-20n^2$.\\
		Để sau một vụ thu hoạch được nhiều cá nhất thì cân nặng của $n$ con cá trên một đơn vị điện tích của mặt hồ là lớn nhất.\\
		Bài toán trở thành tìm $n\in \mathbb{N}^*$ sao cho $F(n)$ đạt giá trị lớn nhất.\\
		Ta có $F'(n)=800-40n$.\\
		Cho $F'(n)=0 \Leftrightarrow 800-40n=0 \Leftrightarrow n=20$.\\
		Ta có bảng biến thiên
		\begin{center}
			\begin{tikzpicture}
				\tkzTabInit[nocadre,lgt=1.2,espcl=2.5,deltacl=0.6]
				{$n$/0.6,$F'(n)$/0.6,$F(n)$/2}{$-\infty$,$20$,$+\infty$}
				\tkzTabLine{,+,0,-,}
				\tkzTabVar{-/$-\infty$,+/$8\,000$,-/$-\infty$}
			\end{tikzpicture}
		\end{center}
		Vậy phải thả $20$ con cá trên một đơn vị diện tích của mặt hồ để sau một vụ thu hoạch được nhiều cá nhất.
	}
\end{ex}

\begin{ex}%[BG12new-4in1, Trần Hoà]%[2D1H1-1]
	Cho hàm số $y=f(x)$ có  $f'(x)=(x+1)^2(x-1)^3(2-x), \forall x\in \mathbb{R}$. Hàm số $y=f(x)$ đồng biến trên khoảng nào dưới đây?
	\choice
	{\True $(1;2)$}
	{$(-\infty;-1)$}
	{$(-1;1)$}
	{$(2;+\infty)$}
	\loigiai
	{
		Dựa vào bảng xét dấu của $f'(x)$:
		\begin{center}
			\begin{tikzpicture}
				\tkzTabInit[lgt=1.2,espcl=1.2]
				{$x$ /1, $f'(x)$ /1}
				{$-\infty$, $-1$,$1$, $2$, $+\infty$}
				\tkzTabLine{ ,-,z,-,z,+,z,-, }
			\end{tikzpicture}
		\end{center}
		ta suy ra hàm số $y=f(x)$ đồng biến trên $(1;2)$.
	}
\end{ex}

\begin{ex}%[Dự Án Giảng 12 4 in 1, Lê Văn Toàn]%[2D1N5-7]
	Đồ thị hàm số $y=\dfrac{2x+1}{x-1}$ cắt trục tung tại điểm có tung độ bằng
	\choice
	{$1$}
	{$-\dfrac{1}{2}$}
	{\True $-1$}
	{$2$}
	\loigiai{
		Gọi $M$ là giao điểm của đồ thị với trục tung, suy ra $x_M=0$.\\
		Thay vào biểu thức của đồ thị hàm số ta được $y_M=-1$.}
\end{ex}

\begin{ex}%[2D1H2-7]
	\immini{Đường dây điện $110KV$ kéo từ trạm phát (điểm $A$) trong đất liền ra Côn Đảo (điểm $C$). Biết khoảng cách ngắn nhất từ $C$ đến $B$ là $60$ km, khoảng cách từ $A$ đến $B$ là $100$ km, mỗi km dây điện dưới nước chi phí là $5000$ USD, chi phí cho mỗi km dây điện trên bờ là $3000$ USD. Hỏi điểm $D$ cách điểm $A$ bao nhiêu để mắc dây điện từ $A$ đến $D$ rồi từ $D$ đến $B$ chi phí đạt cực tiểu? (hình vẽ bên)
		\choice
		{$40$ km}
		{$50$ km}
		{\True $55$ km}
		{$45$ km}}{
		\begin{tikzpicture}[scale=1, font=\footnotesize, line join=round, line cap=round, >=stealth]
			\path
			(0,0) coordinate (A)
			(3,0) coordinate (D)
			(5,0) coordinate (B)
			(B)+(90:4) coordinate (C)
			;

			\draw (A)--(B)--(C)--(D) (A)--(C);

			\foreach \p/\r in {A/90,B/-120,D/-90,C/90}
			\fill (\p) circle (1.5pt) node[shift={(\r:3mm)}]{$\p$};
		\end{tikzpicture}
	}
	\loigiai{
		\immini{
			Đặt khoảng cách từ $D$ đến $B$ là $x$, $0\le x\le 100$. Khi đó khoảng cách từ $A$ đến $D$ là $100 - x$ km và khoảng cách từ $D$ đến $C$ là $\sqrt{x^2 + 3600}$ km. \\
			Chi thí cho việc kéo đường dây điện từ $A$ đến $D$ rồi đến $C$ được tính theo công thức
			\[f(x)=3000\left(100 - x\right) + 5000\sqrt{x^2 + 3600}=300000 - 3000x + 5000\sqrt{x^2 + 3600}.\]
			Ta xác định $x$ sao cho $f$ đạt cực tiểu. \\
			Ta có $f'(x)= - 3000 + \dfrac{5000x}{\sqrt{x^2 + 3600}}=0\Leftrightarrow x=45$.
		}
		{
			\begin{tikzpicture}[scale=1, font=\footnotesize, line join=round, line cap=round, >=stealth]
				\path
				(0,0) coordinate (A)
				(3,0) coordinate (D)
				(5,0) coordinate (B)
				(B)+(90:4) coordinate (C)
				;

				\draw (A)--node[midway,below]{$100-x$}(D)--node[midway,below]{$x$}(B)--(C)--(D) (A)--(C);

				\foreach \p/\r in {A/90,B/-120,D/-90,C/90}
				\fill (\p) circle (1.5pt) node[shift={(\r:3mm)}]{$\p$};
			\end{tikzpicture}
		}
		Bảng biến thiên
		\begin{center}
			\begin{tikzpicture}
				\tkzTabInit[nocadre=false,lgt=1.2,espcl=2.5,deltacl=0.6]
				{$x$ /0.6,$f'(x)$ /0.6,$f(x)$ /2}
				{$0$,$45$,$100$}
				\tkzTabLine{,+,0,-,}
				\tkzTabVar{-/$f(0)$,+/$f(45)$,-/$f(100)$}
			\end{tikzpicture}
		\end{center}
		\noindent
		Dựa vào bảng biến thiên, hàm số $f(x)$ đạt cực tiểu khi $x=45$. Vậy khoảng cách cần tính để chi phí kéo dây là $55$ km.
	}
\end{ex}

\begin{ex}%[2D1H5-1]
	\immini{Người ta muốn chế tạo một chiếc hộp hình hộp chữ nhật có thể tích $500$ $\mathrm{cm}^3$. Chiều cao hộp phải là $2$ cm, các kích thước khác là $x, y$ với $x > 0$ và $y > 0$.
		Công thức xác định diện tích toàn phần $S(x)$ của chiếc hộp theo $x$ là
		\choice
		{$S(x)= 500+4 x-\dfrac{1000}{x}$}
		{\True $S(x)= 500+4 x+\dfrac{1000}{x}$}
		{$S(x)= 250+4 x+\dfrac{1000}{x}$}
		{$S(x)= 500+2 x+\dfrac{1000}{x}$}
	}{\begin{tikzpicture}[declare function={a=2;b=4;h=2;},line join=round,scale =0.8]
			\path (0,0) coordinate (B)
			(35:a) coordinate (A)
			(b,-1) coordinate (C)
			($(C)-(B)+(A)$) coordinate (D)
			($(A)+(90:h)$) coordinate (A')
			($(B)-(A)+(A')$) coordinate (B')
			($(C)-(A)+(A')$) coordinate (C')
			($(D)-(A)+(A')$) coordinate (D');
			\fill[orange!10] (B)--(B')--(A')--(D')--(D)--(C)--cycle;
			\draw ( B')--(B)--(C)--(D)--(D')--(A')--(B')--(C')--(D')  (C)--(C');
			\path (B)--(C)node[pos=0.5,sloped,black,below]{$x$};
			\path (C)--(D)node[pos=0.5,sloped,black,below]{$y$};
			\path (D)--(D')node[pos=0.5,sloped,black,below,scale=0.8]{$2$ cm};
			\draw[dashed]  (A')--(A)--(D)  (A)--(B);
		\end{tikzpicture}}
	\loigiai{
		\begin{itemize}
			\item Biểu thị $y$ theo $x$.\\
			      Ta có $500=x\cdot y \cdot 2 \Rightarrow y=\dfrac{250}{x}$.
			\item Diện tích toàn phần của chiếc hộp là
			      \[S(x)= 2\cdot 2\cdot x+2\cdot 2\cdot y+2\cdot x\cdot y= 500+4 x+\dfrac{1000}{x}.\]
		\end{itemize}
	}
\end{ex}

\begin{ex}%[2-D1B5-SO-16-2425]%[VN-MT-7, Dương Phước Sang]%[2D1H3-1]
	Giá trị lớn nhất của hàm số $f(x)=x^3-8x^2+16x-9$ trên đoạn $[1;3]$ là
	\choice
	{$\max\limits_{[1; 3]} f(x)=0$}
	{\True $\max\limits_{[1; 3]} f(x)=\dfrac{13}{27}$}
	{$\max\limits_{[1; 3]} f(x)=-6$}
	{$\max\limits_{[1; 3]} f(x)=5$}
	\loigiai{
		Hàm số $f(x)$ liên tục trên $[1;3]$.\\
		Ta có $f'(x)=3x^2-16x+16$; $f'(x)=0 \Leftrightarrow \hoac{&x=4 \notin (1;3)\\&x=\dfrac{4}{3}\in (1;3).}$\\
		$f(1)=0$; $f\left( \dfrac{4}{3} \right)=\dfrac{13}{27}$; $f(3)=-6$.\\
		Do đó $\max\limits_{x \in [1;3]} f(x)=f\left( \dfrac{4}{3} \right)=\dfrac{13}{27}$.
	}
\end{ex}

\begin{ex}%[2-D1B3-SO-7-2425]%[VN-MT-7, VM024]%[2D1N4-1]
	Cho hàm số $ y=f(x)$ có $\lim\limits_{x\to +\infty}f(x)=2$, $\lim\limits_{x\to -\infty}f(x)=+\infty$.
	\choice
	{Đồ thị hàm số đã cho có hai đường tiệm cận ngang}
	{Đồ thị hàm số đã cho có đúng một tiệm cận ngang là đường thẳng $x=2$}
	{\True Đồ thị hàm số đã cho có đúng một tiệm cận ngang}
	{Đồ thị hàm số đã cho không có tiệm cận ngang}
	\loigiai{
		Ta có $\lim\limits_{x\to +\infty}f(x)=2$. Do đó, đường thẳng $y=2$ là tiệm cận ngang của đồ thị hàm số $y=f(x)$.
	}
\end{ex}

\begin{ex}%[Mức độ N]%[2D1N3-1]
	Tìm giá trị lớn nhất của hàm số $y=\dfrac{3\sin x+2}{\sin x+1}$ trên đoạn $\left[0;\dfrac{\pi}{2}\right]$.
	\choice{\True $\dfrac{5}{2}$}{$\dfrac{11}{2}$}{$\dfrac{31}{2}$}{$2$}
	\loigiai{Đặt $t=\sin x$, $t \in [0;1]$.\\
		Xét hàm số $f(t)=\dfrac{3t+2}{t+1}$, $t \in [0;1].$\\Ta có $f'(t)=\dfrac{1}{(t+1)^2}>0$, $t\in [0;1]$.\\
		Vậy	$\underset{[0;1]}{\text{max}}f(t)=f(1)=\dfrac{5}{2}$.}
\end{ex}

\begin{ex}%[BG12new-4in1, Trần Hoà]%[2D1H1-2]
	\immini{Cho hàm số $y=f(x)$ xác định, liên tục trên $\mathbb{R}$ và có đạo hàm $f'(x)$. Biết rằng $f'(x)$ có đồ thị như hình vẽ bên. Mệnh đề nào sau đây đúng?
		\choice
		{\True Hàm số $y=f(x)$ nghịch biến trên khoảng $\left(0; + \infty\right)$}
		{Hàm số $y=f(x)$ nghịch biến trên khoảng $(- 3;-2)$}
		{Hàm số $y=f(x)$ đồng biến trên khoảng $\left(- \infty; 3\right)$}
		{Hàm số $y=f(x)$ đồng biến trên khoảng $(- 2; 0)$}}{\begin{tikzpicture}[>=stealth,line join=round,line cap=round,font=\footnotesize,scale=1]
			\draw[->] (-4.1,0)--(2.1,0) node[below left] {$x$};
			\draw[->] (0,-2.6)--(0,2.6) node[below left] {$y$};
			\draw[fill=black] (0,0) circle (1pt) node[above left] {$O$};
			\foreach \x in {-3,-2}
			\draw[thin] (\x,1pt)--(\x,-1pt) node [below left] {$\x$};
			\begin{scope}
				\clip (-4,-2.6) rectangle (2,2.6);
				\draw[samples=200,domain=-4:2,smooth,variable=\x] plot (\x,{(-(\x)-3)*((\x)+2)*((\x)^2)});
			\end{scope}
		\end{tikzpicture}}
	\loigiai{
		Từ đồ thị của hàm số, ta nhận thấy
		Với $\forall x\in \left(- 3; - 2\right)$, $f'(x)>0$ nên hàm số đồng biến.
		Với $\forall x\in \left(- \infty; - 3\right)$ và $(- 2; 0)$ và $\left(0; + \infty\right)$, $f'(x)<0$ nên hàm số nghịch biến.
		Vậy hàm số nghịch biến trên $\left(0; + \infty\right)$.}
\end{ex}

\begin{ex}%[Mức độ N]%[2D1N2-1]
	Hàm số nào dưới đây không có cực trị?
	\choice{$y=\dfrac{x^2+1}{x}$}{\True $y=\dfrac{2x-2}{x+1}$}{$y=x^2-2x+1$}{$y=-x^3+x+1$}
	\loigiai{Xét hàm số $y=\dfrac{2x-2}{x+1}\text{, } \forall x \ne-1$.\\ Ta có $y'=\dfrac{4}{(x+1)^2}>0\text{, } \forall x \ne -1$.\\
		Vậy hàm số $y=\dfrac{2x-2}{x+1}$ không cực trị.}
\end{ex}

\begin{ex}%[2D1N5-1]
	\immini[thm]{Bảng biến thiên ở hình bên là của một trong bốn hàm số sau đây. Hỏi đó là hàm số nào?
		\choice
		{$y=-x^3-2x^2+5$}
		{\True $y=x^3-3x^2+5$}
		{$y=-x^3-3x+5$}
		{$y=x^3+3x^2+5$}}{
		\begin{tikzpicture}
			\tkzTabInit[nocadre=false, lgt=1.2, espcl=1.6]{$x$ /0.6,$f'(x)$ /0.6,$f(x)$ /1.5}{$-\infty$,$0$,$2$,$+\infty$}
			\tkzTabLine{,+,$0$,-,$0$,+,}
			\tkzTabVar{-/ $-\infty$/, +/$5$ , -/$1$  , +/$+\infty$/}
		\end{tikzpicture}}
	\loigiai{

	}
\end{ex}

\begin{ex}%[Mức độ 2]%[Dự án giảng new 4in1, Trần Quang Thạnh]%[2D1H1-4]
	Cho hàm số $y=f(x)$ có bảng biến thiên sau
	\begin{center}
		\begin{tikzpicture}
			\tkzTabInit[nocadre=false,lgt=1.2,espcl=2.5]
			{$x$ /0.7,$f(x)$ /2}{$-\infty$,$0$,$4$,$+\infty$}
			\tkzTabVar{+/$+\infty$,-/$-5$,+/$-1$,-/$0$}
		\end{tikzpicture}
	\end{center}
	Bất phương trình $f(8x) < f(3x-185)$ có bao nhiêu nghiệm nguyên âm?
	\choice
	{$39$}
	{$38$}
	{$37$}
	{\True $36$}
	\loigiai{
		Ta thấy $f(x)$ nghịch biến trên khoảng $(0;+\infty)$.\\
		Với $x<0$, ta có $8x<0$ và $3x-185<0$, do đó
		\[f(8x) < f(3x-185) \Leftrightarrow 8x>3x-185 \Leftrightarrow x>-37.\]
		Vì $x$ nguyên âm nên $x\in\{-36;-35;\ldots;-1\}$.
	}
\end{ex}

\begin{ex}%[TEX NBV, Phạm Hoài]%[2D1N1-2]
	\immini[thm]{
		Biết hàm số $y=\dfrac{x+a}{x+1}$ ($a$ là số thực cho trước, $a\neq 1$ có đồ thị như hình bên). Mệnh đề nào dưới đây đúng?
		\choice
		{$y'<0, \,\forall x\neq -1$}
		{\True  $y'>0, \,\forall x\neq -1$}
		{$y'<0, \,\forall x\in \mathbb{R}$}
		{$y'>0, \,\forall x\in \mathbb{R}$}
	}{\begin{tikzpicture}[line join=round, line cap=round,>=stealth,thick,scale=0.75]
			\tikzset{every node/.style={scale=0.9}}
			\draw[->] (-4.1,0)--(4.2,0) node[below left] {$x$};
			\draw[->] (0,-4.1)--(0,5.2) node[below left] {$y$};
			\draw (0,0) node [below left] {$O$};
			\foreach \x/\nx in {1/1,2/2,3/3}
			\draw[thin] (\x,1pt)--(\x,-1pt) node [below] {$\nx$};
			\foreach \x/\nx in {-1/-1,-2/-2}
			\draw[thin] (\x,1pt)--(\x,-1pt) node [above right] {$\nx$};
			\foreach \x/\nx in {-4/-4,-3/-3}
			\draw[thin] (\x,1pt)--(\x,-1pt) node [above] {$\nx$};
			\foreach \y/\ny in {-1/-1,1/1,2/2,3/3,4/4}
			\draw[thin] (1pt,\y)--(-1pt,\y) node [left] {$\ny$};
			\foreach \y/\ny in {-4/-4,-3/-3,-2/-2}
			\draw[thin] (1pt,\y)--(-1pt,\y) node [right] {$\ny$};
			%\draw[dashed,thin](2,0)--(2,-6)--(0,-6);
			%\draw[dashed,thin] (1.01,-10)--(1.01,2);
			\begin{scope}
				\clip (-4,-4) rectangle (4,5);
				\draw[samples=200,domain=-5:-1.1,smooth,variable=\x] plot (\x,{(-1*(\x)-3)/(2*(\x)+2)});
				\draw[samples=200,domain=-.7:5,smooth,variable=\x] plot (\x,{(-1*(\x)-3)/(2*(\x)+2)});
				\draw (-1,-4)--(-1,5) (-5,-0.5)--(5,-0.5);
			\end{scope}
		\end{tikzpicture}}
	\loigiai{Dựa vào đồ thị, hàm số đồng biến trên từng khoảng xác định. Do đó $y'>0\, \forall x\ne -1$ suy ra $1-a>0\Rightarrow a<1$.
	}
\end{ex}

\begin{ex}%[Sách tham khảo, Mức độ H]%[Dự án giảng 12 - Trung Anh]%[2D1H2-4]
	Tìm tất cả các giá trị thực của tham số $m$ để hàm số $y=x^3-3(m+1)x^2+12mx+2019$ có hai điểm cực trị $x_1,\ x_2$ thỏa mãn $x_1+x_2+2x_1x_2=-8$.
	\choice
	{\True $m=-1$}
	{$m=2$}
	{$m=1$}
	{$m=-2$}
	\loigiai{
		Ta có $y'=3x^2-6(m+1)x+12m,\ y'=0\Leftrightarrow 3x^2-6(m+1)x+12m=0$. \\
		Hàm số có hai điểm cực trị $\Leftrightarrow \Delta '=9m^2-18m+9>0\Leftrightarrow m\ne 1$.\tagEX{1}
		Giả sử $x_1,\ x_2$ là hai nghiệm của phương trình $y'=0$, theo định lí Vi-ét ta có
		\[\heva{&x_1+x_2=-\dfrac{b}{a}=2(m+1)\\&x_1\cdot x_2=\dfrac{c}{a}=4m.}\]
		Do đó $x_1+x_2+2x_1\cdot x_2=-8\Leftrightarrow 2(m+1)+8m=-8\Leftrightarrow 10m=-10\Leftrightarrow m=-1$ thỏa mãn $(1)$.\\
		Vậy $m=-1$ là giá trị cần tìm của $m$.}
\end{ex}

\begin{ex}%[ĐỀ CHUẨN HÓA CHƯƠNG 1-GIẢI TÍCH 12]%[Huỳnh Đức Vũ]%[2D1V5-6]
	Cho hàm số $y=-x^3+3x^2+2$ có đồ thị $(C)$. Biết rằng, tại điểm $M$ thuộc $(C)$ tiếp tuyến của $(C)$ có hệ số góc lớn nhất. Tìm phương trình tiếp tuyến đó.
	\choice
	{\True $y=3x+1$}
	{$y=-3x+1$}
	{$y=-3x-1$}
	{$y=3x-1$}
	\loigiai{
		$y'=-3x^2+6x=-3(x-1)^2+3\leq 3$. \\
		Tiếp tuyến của $(C)$ có hệ số góc lớn nhất bằng $3$ tại điểm $M(1;4)$ có phương trình là
		$y=3(x-1)+4=3x+1$.}
\end{ex}

\begin{ex}%[Mức độ 3]giảng 12, Phạm Tiến Long]%[2D1V4-3]
	Tìm tham số $m$ để đồ thị hàm số $f(x)=\dfrac{x^2-mx+1}{x-2}$ có tiệm cận xiên cắt hai trục tọa độ $Ox$, $Oy$ tại hai điểm $A$, $B$ sao cho tam giác $OAB$  có diện tích bằng $8$.
	\choice
	{$m=2$ hoặc $m=6$}
	{\True $m=-2$ hoặc $m=6$}
	{$m=2$ hoặc $m=-6$}
	{$m=-2$ hoặc $m=-6$}
	\loigiai{	Hàm số đã cho có tập xác định $\mathscr{D}=\mathbb{R}\backslash\{-1\}$.\\
		Ta có $\begin{aligned}[t]
				a & =\lim\limits_{x \rightarrow+\infty} \dfrac{f(x)}{x}=\lim\limits_{x \rightarrow+\infty} \dfrac{x^2-mx+1}{x^2-2x}=1;                                                                \\
				b & =\lim\limits_{x \rightarrow+\infty}[f(x)-ax]=\lim\limits_{x \rightarrow+\infty}\left(\dfrac{x^2-mx+1}{x-2}-x\right)=\lim\limits_{x \rightarrow+\infty} \dfrac{(2-m)x+1}{x-2}=2-m.
			\end{aligned}$\\
		Ta cũng có $\lim\limits_{x \rightarrow-\infty} \dfrac{f(x)}{x}=1$; $\lim\limits_{x \rightarrow-\infty}[f(x)-x]=2-m$.\\
		Do đó, tiệm cận xiên của đồ thị hàm số là đường thẳng $d\colon y=x+2-m$.\\
		Đường thẳng $d$ cắt hai trục tọa độ tại hai điểm $A(0;2-m)$ và $B(m-2;0)$.\\
		Dễ thấy tam giác $OAB$ vuông cân tại $O$.\\
		Với điều kiện $m\ne 2$, ta có
		\begin{eqnarray*}
			& & \dfrac{1}{2}\cdot OA^2=8\\
			&\Leftrightarrow & \dfrac{1}{2} \cdot (2-m)^2=8\\
			&\Leftrightarrow & (2-m)^2=16\\
			&\Leftrightarrow & \hoac{&2-m=4\\&2-m=-4}\\
			&\Leftrightarrow & \hoac{&m=-2\text{ (thỏa điều kiện)}\\&m=6.\text{ (thỏa điều kiện)}}
		\end{eqnarray*}
		Vậy $m=-2$ hoặc $m=6$.
	}
\end{ex}

\begin{ex}%[Dự án TL12New-4in1-NCT]%[2D1V4-1]
	Cho hàm số $y=f(x)$ có bảng biến thiên như sau
	\begin{center}
		\begin{tikzpicture}[>=stealth]
			\tkzTabInit[nocadre=false,lgt=1,espcl=3,deltacl=0.5]{$x$/.7 ,$y'$/.7,$y$/2}
			{$-\infty$ , $-2$ , $2$ , $+\infty$}
			\tkzTabLine{, + , $0$ , - , $0$ , + ,}
			\tkzTabVar{-/$-\infty$ , +/$3$ , -/$0$ , +/$+\infty$}
		\end{tikzpicture}
	\end{center}
	Đồ thị hàm số $y=\dfrac{1}{f(3-x)-2}$ có bao nhiêu tiệm cận đứng?
	\choice
	{$0$}
	{$2$}
	{\True $3$}
	{$1$}
	\loigiai{
		Ta thấy $f(x)=2$ có $3$ nghiệm $\Rightarrow$ đồ thị hàm số $y=\dfrac{1}{f(3-x)-2}$ có $3$ tiệm cận đứng.}
\end{ex}

\begin{ex}%[Dự án TL12New-4in1-NCT]%[2D1H4-2]
	Tìm tất cả các giá trị thực của tham số $m$ để đồ thị hàm số $y=\dfrac{mx+3}{\sqrt{mx^2-5}}$ có hai đường tiệm cận ngang.
	\choice
	{$m\geq 0$}
	{$m>\sqrt{5}$}
	{$m<0$}
	{\True $m>0$}
	\loigiai{
		Ta có $\lim\limits_{x\to+\infty}\dfrac{mx+3}{\sqrt{mx^2-5}}=\lim\limits_{x\to+\infty}\dfrac{m+\dfrac{3}{x}}{\sqrt{m-\dfrac{5}{x^2}}}$ và $\lim\limits_{x\to-\infty}\dfrac{mx+3}{\sqrt{mx^2-5}}=\lim\limits_{x\to-\infty}\dfrac{m+\dfrac{3}{x}}{-\sqrt{m-\dfrac{5}{x^2}}}$.\\
		Để đồ thị hàm số $y=\dfrac{mx+3}{\sqrt{mx^2-5}}$ có hai đường tiệm cận ngang thì $m>0$.\\
		Khi đó hai đường tiệm cận ngang là $y=\pm\sqrt{m}$.}
\end{ex}

\begin{ex}%[Mức độ 2]giảng 12, Phạm Tiến Long]%[2D1H4-3]
	Gọi $d$ là tiệm cận xiên của đồ thị hàm số $f(x)=2x-4+\dfrac{1}{3x+4}$. Giao điểm của $d$ với trục tung là
	\choice
	{$M(2;0)$}
	{$N(-2;0)$}
	{$P(0;4)$}
	{\True $Q(0;-4)$}
	\loigiai{
		Hàm số đã cho có tập xác định là $\mathbb{R}\backslash \left\{-\dfrac{4}{3}\right\}$.\\
		Ta có $\lim\limits_{x\to +\infty}[f(x)-(2x-4)]=0$ và $\lim\limits_{x\to -\infty}[f(x)-(2x-4)]=0$.\\
		Do đó, đồ thị hàm số có tiệm cận xiên là đường thẳng $d\colon y=2x-4$.\\
		Giao điểm của $d$ với trục tung là $Q(0;-4)$.
	}
\end{ex}
\begin{ex}%[CD12 - CTST, Mức độ 2] %[2D1H1-5]
	Chi phí sản xuất $x$ sản phẩm mỗi tháng của một công ty cho bởi hàm $\overline{C(x)}$ có bảng biến thiên như sau
	\begin{center}
		\begin{tikzpicture}
			\tikzset{double style/.append style={double distance=2pt}}
			\tkzTabInit[lgt=1.2, espcl=2]
			{$x$/0.6,$\overline{C'}(x)$/0.6,$\overline{C}(x)$/1.5}{$0$,$1000$,$+\infty$}
			\tkzTabLine{,-,0,+,}
			\tkzTabVar{+/,-/$60$,+/$+\infty$}
		\end{tikzpicture}
	\end{center}
	Hỏi khi số sản phẩm mỗi tháng vượt qua giá trị bao nhiêu thì chi phí sản xuất bắt đầu tăng.
	\choice
	{\True $1\,000$}
	{$60$}
	{$500$}
	{$3\,60$}
	\loigiai{
		Từ bảng biến thiên ta thấy khi mỗi tháng xưởng sản xuất vượt quá $1\,000$ sản phẩm thì chi phí trung bình  sản xuất một sản phẩm thấp bắt đầu tăng.
	}
\end{ex}
\Closesolutionfile{ans}
\TL
\begin{ex}%[SGK12-CTST, Mức độ 2]%[2D1H2-1]
	Xét sự biến thiên và các điểm cực trị của hàm số $g(x)=\dfrac{x^2+x+4}{x+1}$.
	\loigiai{
		Tập xác định $\mathscr{D}=\mathbb{R}\setminus \{-1\}$.\\
		Ta có $g(x)=x+\dfrac{4}{x+1} \Rightarrow g'(x)=1-\dfrac{4}{(x+1)^2}=\dfrac{x^2+2x-3}{(x+1)^2}$;\\
		$g'(x)=0 \Leftrightarrow x^2+2x-3=0 \Leftrightarrow \hoac{& x=-3\\& x=1.}$\\
		Bảng biến thiên
		\begin{center}
			\begin{tikzpicture}
				\tkzTabInit[nocadre=false,lgt=1.2,espcl=2.5,deltacl=0.6]
				{$x$ /0.6,$g'(x)$ /0.6,$g(x)$ /2}
				{$-\infty$,$-3$,$-1$,$1$,$+\infty$}
				\tkzTabLine{,+,$0$,-,d,-,$0$,+,}
				\tkzTabVar{-/$-\infty$,+/$-5$,-D+/$-\infty$/$+\infty$,-/$3$,+/$+\infty$}
			\end{tikzpicture}
		\end{center}
		Vậy hàm số đồng biến trên mỗi khoảng $(-\infty;-3)$ và $(1;+\infty)$; nghịch biến trên mỗi khoảng $(-3;-1)$ và $(-1;1)$. Hàm số đạt cực đại tại $x=-3$, $y_{\text{CĐ}}=g(-3)=-5$; và hàm số đạt cực tiểu tại $x=1$, $y_{\text{CT}}=g(1)=3$.
	}
\end{ex}
\begin{ex}%[2-D1B4-SO-10-2425]%[VN-MT-7, VM012]%[2D1V5-8]
	Giả sử chi phí cho xuất bản $x$ cuốn tạp chí (gồm: lương cán bộ, công nhân viên, giấy in,\ldots) được cho bởi công thức
	$C(x)=0{,}0\,001x^2-0{,}2x+10\,000$,
	trong đó $C(x)$ được tính theo đơn vị là vạn đồng ($1$ vạn đồng $=$ 10\,000 đồng). Chi phí phát hành cho mỗi cuốn là $4$ nghìn đồng. Tỉ số $M(x)=\dfrac{T(x)}{x}$ được gọi là chi phí trung bình cho một cuốn tạp chí khi xuất bản $x$ cuốn và tổng chi phí $T(x)$ (xuất bản và phát hành) cho $x$ cuốn tạp chí. Tìm chi phí trung bình thấp nhất cho một cuốn tạp chí là bao nhiêu vạn đồng, biết rằng nhu cầu hiện tại xuất bản không quá 30\,000 cuốn?
	% \shortans{2{,}2}
	\loigiai{
	Chi phí phát hành cho mỗi cuốn là $4$ nghìn đồng, tức là $0{,}4$ vạn đồng.\\
	Suy ra chi phí phát hành cho $x$ cuốn là $0{,}4x$ (vạn đồng).\\
	Theo đề bài, ta có tổng chi phí xuất bản và phát hành cho $x$ cuốn tạp chí là\\
	$T(x)=C(x)+0{,}4x=0{,}0\,001x^2+0{,}2x+10\,000$, với $x > 0$.\\
	Ta có $f(x)=M(x)=\dfrac{T(x)}{x}=0{,}0\,001x+0{,}2+\dfrac{10\,000}{x}$.\\
	Xét hàm số $f(x)=0{,}0\,001x+0{,}2+\dfrac{10\,000}{x}$, với $0< x\le 30\,000$.\\
	$f'(x)=0{,}0\,001-\dfrac{10\,000}{x^2}=\dfrac{0{,}0\,001x^2-10\,000}{x^2}$, $f'(x)=0\Leftrightarrow x=10\,000$ (do $x>0$).\\
	$\lim\limits_{x\to 0^+} f(x)=+\infty$.\\
	Bảng biến thiên:
	\begin{center}
		\begin{tikzpicture}[>=stealth]
			\tkzTabInit[nocadre=true,lgt=1.2,espcl=2.5,deltacl=0.6]{$x$/.7 ,$f'(x)$/.7,$f(x)$/2}
			{$0$ , $10\,000$ , $30\,000$}
			\tkzTabLine{ , - , $0$ , + , }
			\tkzTabVar{+/$+\infty$ , -/$f(10\,000)$ , +/$f(30\,000)$}
		\end{tikzpicture}
	\end{center}
	Dựa vào bảng biến thiên, ta thấy giá trị của $M(x)$ nhỏ nhất khi $x=10\,000$.\\
	Do đó, số lượng tạp chí cần xuất bản sao cho chi phí trung bình thấp nhất là $x=10\,000$ (cuốn).\\
	Vậy chi phí trung bình cho một cuốn tạp chí khi xuất bản $10\,000$ cuốn là $M(10\,000)=2{,}2$ (vạn đồng).
	}
\end{ex}
\begin{ex}%[2-D1B1-SO-1-2425]%[VN-MT-7, Nguyễn Cao Cường]%[2D1C2-7]
	\immini[thm]{Người ta muốn thiết kế một lồng nuôi cá có bề mặt hình chữ nhật bao gồm phần mặt nước có diện tích bằng $54$ m$^2$ và phần đường đi xung quanh có thiết kế như hình vẽ (đơn vị: mét). Khi kích thước $a$ thay đổi trong khoảng $(3;+\infty)$ thì giá trị hàm số mô tả diện tích lối đi theo kích thước $a$ sẽ giảm đến giá trị $S_0$ rồi tăng lên. Xác định giá trị $S_0$.
	}
	{\begin{tikzpicture}[>=stealth,line join=round,line cap=round,font=\footnotesize,scale=0.75]
			\path
			(0,0) coordinate (A)
			(5,0) coordinate (B)
			(0,4) coordinate (D)
			($(D)+(B)-(A)$) coordinate (C)
			($(B)+(-0.5,0)$) coordinate (M)
			($(B)+(0,0.5)$) coordinate (N)
			($(A)+(1,0)$) coordinate (H)
			($(A)+(0,0.5)$) coordinate (T)
			($(C)+(-0.5,0)$) coordinate (Q)
			($(C)+(0,-0.5)$) coordinate (P)
			($(D)+(1,0)$) coordinate (R)
			($(D)+(0,-0.5)$) coordinate (S)
			($(T)+(H)-(A)$) coordinate (A')
			($(M)+(N)-(B)$) coordinate (B')
			($(Q)+(P)-(C)$) coordinate (C')
			($(R)+(S)-(D)$) coordinate (D')
			($(D)+(0,0.5)$) coordinate (x)
			($(C)+(0,0.5)$) coordinate (y)
			($(D)+(-0.5,0)$) coordinate (u)
			($(A)+(-0.5,0)$) coordinate (v)
			($(A)+(0,-0.5)$) coordinate (x')
			($(H)+(0,-0.5)$) coordinate (y')
			($(M)+(0,-0.5)$) coordinate (u')
			($(B)+(0,-0.5)$) coordinate (v')
			($(C)+(0.5,0)$) coordinate (x'')
			($(P)+(0.5,0)$) coordinate (y'')
			($(N)+(0.5,0)$) coordinate (u'')
			($(B)+(0.5,0)$) coordinate (v'')
			;
			\draw[fill=cyan!20!brown](A)--(B)--(C)--(D)--(A);
			\draw[fill=cyan!90!blue](A')--(B')--(C')--(D')--(A');
			\draw(D)--(x) (C)--(y)(D)--(u)(A)--(v)(A)--(x')(H)--(y')(M)--(u')(B)--(v')(C)--(x'') (P)--(y'') (N)--(u'') (B)--(v'');
			\draw[<->] (x)--(y)node[pos=0.5,above]{$a$};
			\draw[<->] (u)--(v)node[pos=0.5,left]{$b$};
			\draw[<->] (x')--(y')node[pos=0.5,below]{$2$};
			\draw[<->] (u')--(v')node[pos=0.5,below]{$1$};
			\draw[<->] (u'')--(v'')node[pos=0.5,right]{$1$};
			\draw[<->] (x'')--(y'')node[pos=0.5,right]{$1$};
		\end{tikzpicture}}
	% \shortans{42}
	\loigiai{
		Gọi $x$, $y$ lần lượt là độ dài, rộng của mặt nước. Điều kiện $x$, $y>0$.\\
		Phần mặt nước có diện tích bằng $54$ m$^2$ nên ta có $xy=54$. \quad\quad $(*)$\\
		Theo đề bài ta có $x=a-3$, $y=b-2$.\\
		Từ $(*)$ suy ra \[(a-3)(b-2)=54\Rightarrow b=\dfrac{54}{a-3}+2=\dfrac{2a+48}{a-3}.\]
		Diện tích lối đi là
		\allowdisplaybreaks
		\begin{eqnarray*}
			S(a)&=&a\cdot b-x\cdot y\\
			&=&ab-54\\
			&=&a\cdot \dfrac{2a+48}{a-3}-54\\
			&=&\dfrac{2a^2+48a}{a-3}-54.
		\end{eqnarray*}
		$S'(a)=\dfrac{2a^2-12a-144}{\left(a-3\right)^2}$.\\
		Xét $S'(a)=0\Leftrightarrow \hoac{&a=-6\\&a=12.}$\\
		Bảng biến thiên
		\begin{center}
			\begin{tikzpicture}
				\tkzTabInit[nocadre=true,lgt=1.2,espcl=4,deltacl=0.5]
				{$a$ /0.7,$S'(a)$ /0.7,$S(a)$ /2}
				{$3$,$12$,$+\infty$}
				\tkzTabLine{,-,$0$,+,}
				\tkzTabVar{+/$+\infty$,-/$42$,+/$+\infty$}
			\end{tikzpicture}
		\end{center}
		Vậy $S_0=42$.
	}
\end{ex}
\begin{ex}%[Dự Án Giảng 12 4 in 1, Lê Văn Toàn]%[2D1C5-6]
	Cho hàm số $y=\dfrac{1}{4}x^4-\dfrac{7}{2}x^2$ có đồ thị $(C)$. Tiếp tuyến tại điểm $A$ thuộc $(C)$ cắt $(C)$ tại hai điểm phân biệt $M\left(x_1;y_1\right)$, $N\left(x_2;y_2\right)$ ($M$, $N$ khác $A)$ thỏa mãn $y_1-y_2=6\left(x_1-x_2\right)$. Các điểm $A$ thỏa mãn có tổng các hoành độ là
	% \shortans{$-3$}
	\loigiai{
		Gọi $A\left(x_0;y_0\right)\in\,(C)$ là tọa độ tiếp điểm của phương trình tiếp tuyến.\\
		Ta có hệ số góc $k=y'\left(x_0\right)=x_0^3-7x_0$.\\
		Phương trình tiếp tuyến $y=k\left(x-x_0\right)+y_0=\left(x_0^3-7x_0\right)\left(x-x_0\right)+y_0$.\\
		Ta có
		\begin{eqnarray*}
			&&y_1-y_2=6\left(x_1-x_2\right)\\
			&\Leftrightarrow& k\left(x_1-x_0\right)+y_0-\left[k\left(x_2-x_0\right)+y_0\right]=6\left(x_1-x_2\right)\\
			&\Leftrightarrow& k\left(x_1-x_2\right)=6\left(x_1-x_2\right)\\
			&\Leftrightarrow& k=6\\
			&\Leftrightarrow& x_0^3-7x_0=6\\
			&\Leftrightarrow& x_0^3-7x_0-6=0\\
			&\Leftrightarrow& \hoac{&x_0=3\Rightarrow y_0=-\dfrac{45}{4}\\&x_0=-1\Rightarrow y_0=-\dfrac{13}{4}\\&x_0=-2\Rightarrow y_0=-10.}
		\end{eqnarray*}
		Khi đó các phương trình tiếp tuyến tương ứng là
		\[\hoac{&d_1\colon y=6(x-3)-\dfrac{45}{4}=6x-\dfrac{117}{4}\\&d_2\colon y=6(x+1)-\dfrac{13}{4}=6x+\dfrac{11}{4}\\&d_3\colon y=6(x+2)-10=6x+2.}\]
		Phương trình hoành độ giao điểm của $(C)$ với các tiếp tuyến là
		\[\hoac{&\dfrac{1}{4}x^4-\dfrac{7}{2}x^2-6x+\dfrac{117}{4}=0\text{ (có 1 nghiệm nên không thỏa)}\\&\dfrac{1}{4}x^4-\dfrac{7}{2}x^2-6x-\dfrac{11}{4}=0\text{ (có 3 nghiệm nên thỏa mãn)}\\&\dfrac{1}{4}x^4-\dfrac{7}{2}x^2-6x-2=0\text{ (có 3 nghiệm nên thỏa mãn).}}\]
		Do đó tổng các hoành độ điểm các tiếp điểm là $-1-2=-3$.
	}
\end{ex}
% \begin{indapan}
% 	{ans/ansc1l4}
% \end{indapan}


%C2
% \begin{name}
	{\tenchude}
	{ĐỀ ÔN TẬP CHƯƠNG II}
	{LỚP TOÁN THẦY PHÁT}
	{\thoigian}
\end{name}

\TN
\Opensolutionfile{ans}[ans/ans\currfilebase-Phan-I]

\begin{ex}%[2-H2B4-SO-10-2425 (Nguồn: Bài 4 - Đề 1 - Ôn Tập Chương II)]%[VN-MT-7, Trần Bảo Hiên]%[2H2H1-2]
Cho tứ diện $ABCD$. Gọi $G$ là trọng tâm tam giác $BCD$ và điểm $M$ thuộc cạnh $AB$ sao cho $AM=2BM$. Đẳng thức nào sau đây là đúng?
\choice
{$\overrightarrow{MG}=\overrightarrow{AB}+\overrightarrow{AC}+\overrightarrow{AD}$}
{$\overrightarrow{MG}=\dfrac{1}{3}\overrightarrow{AB}-\dfrac{1}{3}\overrightarrow{AC}-\dfrac{1}{3}\overrightarrow{AD}$}
{\True $\overrightarrow{MG}=-\dfrac{1}{3}\overrightarrow{AB}+\dfrac{1}{3}\overrightarrow{AC}+\dfrac{1}{3}\overrightarrow{AD}$}
{$\overrightarrow{MG}=\dfrac{4}{3}\overrightarrow{AB}-\dfrac{1}{3}\overrightarrow{AC}-\dfrac{1}{3}\overrightarrow{AD}$}
\loigiai{
\begin{center}
\begin{tikzpicture}[scale=0.8,font=\footnotesize,line join=round,line cap=round,>=stealth]
\coordinate (A) at (-1,4);
\coordinate (B) at (-3,0);
\coordinate (C) at (1,-2);
\coordinate (D) at (3,0);
\coordinate (I) at ($(C)!1/2!(D)$);
\coordinate (G) at ($(B)!2/3!(I)$);
\coordinate (M) at ($(A)!2/3!(B)$);
\draw(A)--(B)--(C)--(D)--(A)--(C);
\draw[dashed](I)--(B)--(D) (M)--(G);
\foreach \i/\g in {A/90,B/180,C/-90,D/0,G/-90,M/135}{\fill (\i) circle (1.0pt)($(\i)+(\g:3mm)$) node[scale=1]{$\i$};}
\end{tikzpicture}
\end{center}
Ta có $M$ thuộc cạnh $AB$ và $AM=2BM$ nên $\overrightarrow{AM}=\dfrac{2}{3}\overrightarrow{AB}$.\\
Do $G$ là trọng tâm tam giác $BCD$ nên $3\overrightarrow{AG}=\overrightarrow{AB}+\overrightarrow{AC}+\overrightarrow{AD}$ hay $\overrightarrow{AG}=\dfrac{1}{3}\left(\overrightarrow{AB}+\overrightarrow{AC}+\overrightarrow{AD}\right)$.\\
Mà $\overrightarrow{MG}=\overrightarrow{AG}-\overrightarrow{AM}$ nên $\overrightarrow{MG}=\dfrac{1}{3}\left(\overrightarrow{AB}+\overrightarrow{AC}+\overrightarrow{AD}\right)-\dfrac{2}{3}\overrightarrow{AB}=-\dfrac{1}{3}\overrightarrow{AB}+\dfrac{1}{3}\overrightarrow{AC}+\dfrac{1}{3}\overrightarrow{AD}$.
}
\end{ex}

\begin{ex}%[2-H2B4-SO-10-2425 (Nguồn: Bài 4 - Đề 1 - Ôn Tập Chương II)]%[VN-MT-7, Trần Bảo Hiên]%[2H2H1-2]
Cho hình lập phương $ABCD.EFGH$. Hãy xác định góc giữa cặp vectơ $\overrightarrow{AB}$ và $\overrightarrow{EG}$?
\choice
{$60^\circ$}
{\True $45^\circ$}
{$90^\circ$}
{$120^\circ$}
\loigiai{
\begin{center}
\begin{tikzpicture}[scale=0.8, font=\footnotesize, line join=round, line cap=round, >=stealth]
\path
(0:0) coordinate (B)
(0:4) coordinate (C)
($(B)+(45:2)$) coordinate (A)
($(A)+(C)-(B)$) coordinate (D)
($(A)+(90:3)$) coordinate (E)
($(B)+(90:3)$) coordinate (F)
($(C)+(90:3)$) coordinate (G)
($(D)+(90:3)$) coordinate (H);
\draw[dashed] (B)--(A)--(E) (A)--(D)--(B) (A)--(C);
\draw (E)--(F)--(G)--(H)--(D)--(C)--(B)--(F) (E)--(H) (E)--(G)--(C) (F)--(H);
\foreach \x/\g in {A/170,B/170,C/0,D/0,E/170,F/170,G/0,H/0}
\draw[fill=black] (\x) circle (.5pt)($(\g:.3)+(\x)$) node {$\x$};
\end{tikzpicture}
\end{center}
Ta có $\overrightarrow{EG}=\overrightarrow{AC}$ (do $ACGE$ là hình chữ nhật)
$\Rightarrow\left(\overrightarrow{AB},\overrightarrow{EG}\right)=\left(\overrightarrow{AB},\overrightarrow{AC}\right)=\widehat{BAC}=45^\circ$.
}
\end{ex}

\begin{ex}%[2-H2B4-SO-10-2425 (Nguồn: Bài 4 - Đề 1 - Ôn Tập Chương II)]%[VN-MT-7, Trần Bảo Hiên]%[2H2N2-2]
Trong không gian với hệ tọa độ $Oxyz$, cho điểm $A(2;3;-2)$. Gọi $A_1$ là hình chiếu vuông góc của điểm $A$ lên mặt phẳng $(Oyz)$. Khi đó tọa độ của điểm $A_1$ là
\choice
{$(2;3;0)$}
{$(2;0;0)$}
{$(-2;3;-2)$}
{\True $(0;3;-2)$}
\loigiai{
Hình chiếu vuông góc của điểm $A$ lên mặt phẳng $(Oyz)$ là $A_1(0;3;-2)$.
}
\end{ex}

\begin{ex}%[2-H2B4-SO-10-2425 (Nguồn: Bài 4 - Đề 1 - Ôn Tập Chương II)]%[VN-MT-7, Trần Bảo Hiên]%[2H2H2-2]
Trong không gian với hệ tọa độ $Oxyz$, cho vectơ $\overrightarrow{a}=\left(2;\dfrac{1}{3};-5\right)$ và điểm $M(2;3;4)$. Tọa độ điểm $N$ thỏa mãn $\overrightarrow{MN}=\overrightarrow{a}$ là
\choice
{$\left(2;\dfrac{5}{3};-\dfrac{1}{2}\right)$}
{$\left(0;\dfrac{8}{3};9\right)$}
{\True $\left(4;\dfrac{10}{3};-1\right)$}
{$\left(0;-\dfrac{8}{3};-9\right)$}
\loigiai{
Gọi tọa độ điểm $N$ là $\left(x_N;y_N;z_N\right)$, ta có $\overrightarrow{MN}=\left(x_N-2;y_N-3;z_N-4\right)$.\\
Ta có $\overrightarrow{MN}=\overrightarrow{a}\Leftrightarrow\heva{&x_N-2=2\\&y_N-3=\dfrac{1}{3}\\&z_N-4=-5}\Leftrightarrow\heva{&x_N=2+2\\&y_N=\dfrac{1}{3}+3\\&z_N=-5+4}\Leftrightarrow\heva{&x_N=4\\&y_N=\dfrac{10}{3}\\&z_N=-1.}$\\
Vậy $N\left(4;\dfrac{10}{3};-1\right)$.
}
\end{ex}

\begin{ex}%[2-H2B4-SO-10-2425 (Nguồn: Bài 4 - Đề 1 - Ôn Tập Chương II)]%[VN-MT-7, Trần Bảo Hiên]%[2H2N2-3]
Trong không gian với hệ toạ độ $Oxyz$, cho các vectơ $\overrightarrow{a}=(1;1;2)$ và $\overrightarrow{b}=(-2;0;1)$. Tọa độ của vectơ $\overrightarrow{u}=\overrightarrow{a}-\overrightarrow{b}$ là
\choice
{\True $\overrightarrow{u}=(3;1;1)$}
{$\overrightarrow{u}=(-1;1;1)$}
{$\overrightarrow{u}=(3;1;-3)$}
{$\overrightarrow{u}=(1;3;3)$}
\loigiai{
Ta có $\overrightarrow{u}=\overrightarrow{a}-\overrightarrow{b} \Rightarrow \overrightarrow{u}=\left(1-(-2);1-0;2-1\right) \Rightarrow \overrightarrow{u}=(3;1;1)$.
}
\end{ex}

\begin{ex}%[2-H2B4-SO-10-2425 (Nguồn: Bài 4 - Đề 1 - Ôn Tập Chương II)]%[VN-MT-7, Trần Bảo Hiên]%[2H2H2-2]
Trong không gian với hệ tọa độ $Oxyz$, cho điểm $M(4;1;-2)$ và vectơ $\overrightarrow{u}=(4;-2;6)$. Tìm tọa độ điểm $N$ biết rằng $\overrightarrow{MN}=-\dfrac{1}{2}\overrightarrow{u}$.
\choice
{$(3;3;3)$}
{$(3;-3;3)$}
{\True $(2;2;-5)$}
{$(-3;-3;3)$}
\loigiai{
Ta có $-\dfrac{1}{2}\overrightarrow{u}=(-2;1;-3)$.\\
Gọi tọa độ điểm $N$ là $\left(x_N;y_N;z_N\right)$, ta có $\overrightarrow{MN}=\left(x_N-4;y_N-1;z_N+2\right)$.\\
Ta có $\overrightarrow{MN}=-\dfrac{1}{2}\overrightarrow{u}\Leftrightarrow\heva{&x_N-4=-2\\&y_N-1=1\\&z_N+2=-3}\Leftrightarrow\heva{&x_N=2\\&y_N=2\\&z_N=-5.}$\\
Vậy $N(2;2;-5)$.
}
\end{ex}

\begin{ex}%[2-H2B4-SO-10-2425 (Nguồn: Bài 4 - Đề 1 - Ôn Tập Chương II)]%[VN-MT-7, Trần Bảo Hiên]%[2H2N2-3]
Trong không gian với hệ tọa độ $Oxyz$, cho hai điểm $A(2;-1;4)$, $B(5;3;-8)$. Độ dài của vectơ $\overrightarrow{AB}$ là
\choice
{$5$}
{$8$}
{$9$}
{\True $13$}
\loigiai{
Ta có $\overrightarrow{AB}=(3;4;-12)$.\\
Độ dài của vectơ $\overrightarrow{AB}$ là $\left\vert\overrightarrow{AB}\right\vert=\sqrt{3^2+4^2+(-12)^2}=13$.
}
\end{ex}

\begin{ex}%[2-H2B4-SO-10-2425 (Nguồn: Bài 4 - Đề 1 - Ôn Tập Chương II)]%[VN-MT-7, Trần Bảo Hiên]%[2H2H2-3]
Trong không gian với hệ tọa độ $Oxyz$, cho hai vectơ $\overrightarrow{a}=(1;-2;-3)$, $\overrightarrow{b}=(-2;m-1;2)$. Tìm tham số $m$ để vectơ $\overrightarrow{a}$ vuông góc với vectơ $\overrightarrow{b}$.
\choice
{\True $m=-3$}
{$m=1$}
{$m=5$}
{$m=0$}
\loigiai{
Ta có $\overrightarrow{a}\perp\overrightarrow{b}\Leftrightarrow\overrightarrow{a}\cdot\overrightarrow{b}=0\Leftrightarrow 1\cdot(-2)+(-2)\cdot(m-1)+(-3)\cdot2=0\Leftrightarrow-2-2m+2-6=0\Leftrightarrow m=-3$.
}
\end{ex}

\begin{ex}%[2-H2B4-SO-10-2425 (Nguồn: Bài 4 - Đề 1 - Ôn Tập Chương II)]%[VN-MT-7, Trần Bảo Hiên]%[2H2N2-2]
Trong không gian với hệ tọa độ $Oxyz$, cho điểm $A(4;0;0)$, $B(0;2;0)$. Tâm đường tròn ngoại tiếp tam giác $OAB$ là
\choice
{$I(2;-1;0)$}
{$I\left(\dfrac{4}{3};\dfrac{2}{3};0\right)$}
{$I(-2;1;0)$}
{\True $I(2;1;0)$}
\loigiai{
Ta có $A(4;0;0)\in Ox$, $B(0;2;0)\in Oy$ nên tam giác $OAB$ vuông tại $O$.\\
Do đó, tâm đường tròn ngoại tiếp tam giác $OAB$ là trung điểm $I$ của cạnh $AB$.\\
Vậy $I(2;1;0)$.
}
\end{ex}

\begin{ex}%[2-H2B4-SO-10-2425 (Nguồn: Bài 4 - Đề 1 - Ôn Tập Chương II)]%[VN-MT-7, Trần Bảo Hiên]%[2H2H2-2]
Cho hai điểm $A(1;2;3)$ và $B(3;0;-5)$. Gọi $M$ là điểm đối xứng của $A$ qua $B$. Tọa độ của điểm $M$ là
\choice
{$(2;-2;-8)$}
{\True $(5;-2;-13)$}
{$(2;1;-1)$}
{$(7;2;-7)$}
\loigiai{
Vì $M$ là điểm đối xứng của $A$ qua $B$ nên $B$ là trung điểm của $AM$.
Gọi $M(x_M;y_M;z_M)$, ta có
\begin{center}
$\heva{&x_B=\dfrac{x_A+x_M}{2}\\&y_B=\dfrac{y_A+y_M}{2}\\&z_B=\dfrac{z_A+z_M}{2}}\Leftrightarrow\heva{&x_M=2x_B-x_A\\&y_M=2y_B-y_A\\&z_M=2z_B-z_A}\Leftrightarrow\heva{&x_M=2\cdot3-1=5\\&y_M=2\cdot0-2=-2\\&z_M=2\cdot(-5)-3=-13.}$
\end{center}
Vậy $M(5;-2;-13)$.
}
\end{ex}

\begin{ex}%[2-H2B4-SO-10-2425 (Nguồn: Bài 4 - Đề 1 - Ôn Tập Chương II)]%[VN-MT-7, Trần Bảo Hiên]%[2H2H2-2]
Cho tam giác $MNP$ có $M(-1;3;2)$, $N(2;2;0)$ và $P(-1;1;1)$. Biết $N$ là trọng tâm của tam giác $MNQ$. Điểm $Q$ có tọa độ là
\choice
{\True $(8;2;-3)$}
{$(4;-2;0)$}
{$(2;0;-2)$}
{$(0;-2;-2)$}
\loigiai{
Do $N$ là trọng tâm của của tam giác $MPQ$ nên $\heva{&2=\dfrac{-1-1+x_Q}{3}\\&2=\dfrac{3+1+y_Q}{3}\\&0=\dfrac{2+1+z_Q}{3}}\Leftrightarrow\heva{&x_Q=8\\&y_Q=2\\&z_Q=-3.}$\\
Vậy $N(8;2;-3)$.
}
\end{ex}

\begin{ex}%[2-H2B4-SO-10-2425 (Nguồn: Bài 4 - Đề 1 - Ôn Tập Chương II)]%[VN-MT-7, Trần Bảo Hiên]%[2H2C2-6]
Một chiếc máy ảnh được đặt trên giá đỡ ba chân với điểm đặt $E(0;0;8)$ và các điểm tiếp xúc với mặt đất của ba chân lần lượt là $A_1(0;1;0)$, $A_2\left(\dfrac{\sqrt{3}}{2};-\dfrac{1}{2};0\right)$, $A_3\left(-\dfrac{\sqrt{3}}{2};-\dfrac{1}{2};0\right)$.\\
\begin{center}
\begin{tikzpicture}[line join = round, line cap=round,>=stealth,font=\footnotesize,scale=1]
\draw[dashed,orange] (0,0) ellipse (2cm and 1cm);
\path
(0,0) coordinate (O)
(170:2cm and 1cm) coordinate (Q1)
(-10:2cm and 1cm) coordinate (Q1')
(60:2cm and 1cm) coordinate (Q4)
(-120:2cm and 1cm) coordinate (Q4')
(-70:2cm and 1cm) coordinate (Q2)
(10:2cm and 1cm) coordinate (Q3)
($(0,0)+(0,4)$) coordinate (S)
($(Q1)!-0.5!(Q1')$) coordinate (L1)
($(Q4)!1.5!(Q4')$)coordinate (L2)
($(L1)+(L2)-(O)$) coordinate (L3)
($2*(O)-(L3)$) coordinate (L4)
($2*(L1)-(L3)$) coordinate (L5)
($2*(L2)-(L3)$) coordinate (L6)
(-150:2cm and 1cm) coordinate (A2)
(80:2cm and 1cm) coordinate (A3)
(-70:2cm and 1cm) coordinate (Q2)
(-7:2cm and 1cm) coordinate (A1) ;
\draw[fill=cyan,opacity=0.2] (L3)--(L5)--(L4)--(L6)--cycle;;
\draw[->] ($(Q1)!-0.5!(Q1')$)--($(Q1)!1.5!(Q1')$) node[right]{$y$};
\draw[->] ($(Q4)!-0.5!(Q4')$)--($(Q4)!1.5!(Q4')$) node[below]{$x$};
\fill
(A1) circle(2pt) node[above right]{$A_1$}
(A3) circle(2pt) node[above right]{$A_3$}
(A2) circle(2pt) node[below left]{$A_2$}
(S) circle(2pt) node[below right,xshift=0.4cm]{$E\left(0;0;8\right)$}
(O) circle(2pt) node[below right]{$O$};
\draw[line width=1pt] (S)--(A1) (S)--(A2) (S)--(A3) ;
\draw[->,line width=1.5pt,black] (0,0)--($(O)!1.5!(S)$)node[right]{$z$};
\draw[->,line width=1.5pt,red] (S)--($(S)!0.6!(A1)$) node[above right]{$\overrightarrow{F}_1$};
\draw[->,line width=1.5pt,red] (S)--($(S)!0.6!(A2)$) node[left]{$\overrightarrow{F}_2$};
\draw[->,line width=1.5pt,red] (S)--($(S)!0.5!(A3)$) node[below right]{$\overrightarrow{F}_3$};
\draw ($(S)+(-0.2,0.5)$) node[]{
\begin{tikzpicture}[line join = round, line cap=round,>=stealth,font=\footnotesize,scale=1]
\draw[fill=black] (-0.75,-0.5)rectangle (0.5,0.5);
\draw[fill=cyan] (-0.65,-0.35)rectangle (0.35,-0.25);
\draw[fill=white] (-0.65,0.35)rectangle (0.35,0.25);
\draw[fill=black] (-0.65,0.5)rectangle (-0.5,0.75);
\draw[fill=black] (-0.65,0.65)rectangle (0.5,0.85);
\draw[fill=black] (-0.4,0)--(-1,0.5)--(-1,-0.5)--cycle;
\end{tikzpicture}
};
\end{tikzpicture}
\end{center}
Biết rằng trọng lượng của chiếc máy là $240N$. Tọa độ của lực $\overrightarrow{F_1}$ là
\choice
{\True $\overrightarrow{F_1}=(0;10;-80)$}
{$\overrightarrow{F_1}=(0;10;80)$}
{$\overrightarrow{F_1}=(0;-10;-80)$}
{$\overrightarrow{F_1}=(10;0;-80)$}
\loigiai{
Ta có $\overrightarrow{EA_1}=(0;1;-8)$, $\overrightarrow{EA_2}=\left(\dfrac{\sqrt{3}}{2};-\dfrac{1}{2};-8\right)$, $\overrightarrow{EA_3}=\left(-\dfrac{\sqrt{3}}{2};-\dfrac{1}{2};-8\right)$.\\
Nên $EA_1=EA_2=EA_3=\sqrt{65}$.\\
Mặt khác, $\left\vert\overrightarrow{F_1}\right\vert=\left\vert\overrightarrow{F_2}\right\vert=\left\vert\overrightarrow{F_3}\right\vert$ vì đèn cân bằng và trọng lực của đèn tác dụng đều lên 3 chân của giá đỡ.\\
Do đó $\overrightarrow{F_1}=k\overrightarrow{EA_1}=(0;k;-8k)$, $\overrightarrow{F_2}=k\overrightarrow{EA_2}=\left(\dfrac{\sqrt{3}}{2}k;-\dfrac{1}{2}k;-8k\right)$, $\overrightarrow{F_3}=k\overrightarrow{EA_3}=\left(-\dfrac{\sqrt{3}}{2}k;-\dfrac{1}{2}k;-8k\right)$.\\ $\Rightarrow\overrightarrow{F_1}+\overrightarrow{F_2}+\overrightarrow{F_3}=(0;0;-24k)$.\\
Mà $\overrightarrow{F_1}+\overrightarrow{F_2}+\overrightarrow{F_3}=\overrightarrow{P}=(0;0;-240)\Rightarrow -24k=-240\Rightarrow k=10$.\\
Vậy $\overrightarrow{F_1}=(0;10;-80)$.
}
\end{ex}
\Closesolutionfile{ans}

\TNTF
\Opensolutionfile{ans}[ans/ans\currfilebase-Phan-II]

\begin{ex}%[2-H2B4-SO-10-2425 (Nguồn: Bài 4 - Đề 1 - Ôn Tập Chương II)]%[VN-MT-7, Trần Bảo Hiên]%[2H2V1-2]
Cho hình lập phương $ABCD.A'B'C'D'$ có cạnh bằng $a$. Trên các cạnh $CD$ và $BB'$ ta lần lượt lấy các điểm $M$ và $N$ sao cho $DM=BN=x$ với $0\le x\le a$.
\choiceTF
{\True $\overrightarrow{AC'}=\overrightarrow{AA'}+\overrightarrow{AB}+\overrightarrow{AD}$}
{\True Gọi $K$ là trung điểm $AD$. Khi đó $\overrightarrow{C'K}=\overrightarrow{C'C}+\overrightarrow{C'D'}+\dfrac{1}{2}\overrightarrow{C'B'}$}
{$\overrightarrow{AB}\cdot\overrightarrow{B'D'}=a^2$}
{\True Góc giữa vectơ $\overrightarrow{AC'}$ và $\overrightarrow{MN}$ bằng $90^\circ$}
\loigiai{
\begin{center}
\begin{tikzpicture}[scale=0.85, font=\footnotesize, line join=round, line cap=round, >=stealth]
\path
(0:0) coordinate (B)
(0:4) coordinate (C)
($(B)+(45:2)$) coordinate (A)
($(A)+(C)-(B)$) coordinate (D)
($(A)+(90:4)$) coordinate (A')
($(B)+(90:4)$) coordinate (B')
($(C)+(90:4)$) coordinate (C')
($(D)+(90:4)$) coordinate (D')
($(A)!1/2!(D)$) coordinate (K)
($(D)!2/3!(C)$) coordinate (M)
($(B)!2/3!(B')$) coordinate (N);
\draw[dashed] (B)--(A)--(A') (D)--(A)--(C')--(K) (M)--(N);
\draw (D')--(A')--(B')--(C')--(D')--(D)--(C)--(B)--(B') (C')--(C);
\foreach \x/\g in {A/170,B/170,C/0,D/0,A'/170,B'/170,C'/0,D'/0,K/120,M/0,N/180}
\draw[fill=black] (\x) circle (.5pt)($(\g:.3)+(\x)$) node {$\x$};
\end{tikzpicture}
\end{center}
\begin{itemchoice}
\itemch \textbf{Đúng}.\\
Ta có $\overrightarrow{AC'}=\overrightarrow{AA'}+\overrightarrow{AC}$ (do $A'C'CA$ là hình bình hành).\\
Ngoài ra, ta có $\overrightarrow{AC}=\overrightarrow{AB}+\overrightarrow{AD}$ (do $ABCD$ là hình bình hành).\\
Suy ra $\overrightarrow{AC'}=\overrightarrow{AA'}+\overrightarrow{AB}+\overrightarrow{AD}$.
\itemch \textbf{Đúng}.\\
Ta có
\begin{align*}
\overrightarrow{C'K}=\overrightarrow{C'C}+\overrightarrow{CK}&=\overrightarrow{C'C}+\dfrac{1}{2}\left(\overrightarrow{CA}+\overrightarrow{CD}\right)\\ &=\overrightarrow{C'C}+\dfrac{1}{2}\left(\overrightarrow{C'A'}+\overrightarrow{C'D'}\right)\\
&=\overrightarrow{C'C}+\dfrac{1}{2}\left(\overrightarrow{C'B'}+\overrightarrow{C'D'}+\overrightarrow{C'D'}\right)\\
&=\overrightarrow{C'C}+\dfrac{1}{2}\overrightarrow{C'B'}+\overrightarrow{C'D'}.
\end{align*}
\itemch \textbf{Sai}.\\
Ta có
\begin{align*}
\overrightarrow{AB}\cdot\overrightarrow{B'D'}&=\overrightarrow{AB}\cdot\left(\overrightarrow{B'A'}+\overrightarrow{B'C'}\right)\\
&=\overrightarrow{AB}\cdot\left(-\overrightarrow{AB}+\overrightarrow{AD}\right)\\
&=-\overrightarrow{AB}\cdot\overrightarrow{AB}+\overrightarrow{AB}\cdot\overrightarrow{AD}\\
&=-{\overrightarrow{AB}}^2=-a^2.
\end{align*}
\itemch \textbf{Đúng}.\\
Ta đặt $\overrightarrow{AA'}=\overrightarrow{a}$, $\overrightarrow{AB}=\overrightarrow{b}$, $\overrightarrow{AD}=\overrightarrow{c}$. Ta có $\left\vert\overrightarrow{a}\right\vert=\left\vert\overrightarrow{b}\right\vert=\left\vert\overrightarrow{c}\right\vert=a$.\\
$\overrightarrow{AC'}=\overrightarrow{AA'}+\overrightarrow{AB}+\overrightarrow{AD}$ hay $\overrightarrow{AC'}=\overrightarrow{a}+\overrightarrow{b}+\overrightarrow{c}$.\\
Mặt khác $\overrightarrow{MN}=\overrightarrow{AN}-\overrightarrow{AM}=\left(\overrightarrow{AB}+\overrightarrow{BN}\right)-\left(\overrightarrow{AD}+\overrightarrow{DM}\right)$ với $\overrightarrow{BN}=\dfrac{x}{a}\overrightarrow{a}$ và $\overrightarrow{DM}=\dfrac{x}{a}\overrightarrow{b}$.\\
Do đó $\overrightarrow{MN}=\left(\overrightarrow{b}+\dfrac{x}{a}\overrightarrow{a}\right)-\left(\overrightarrow{c}+\dfrac{x}{a}\overrightarrow{b}\right)=\dfrac{x}{a}\overrightarrow{a}+\left(1-\dfrac{x}{a}\right)\overrightarrow{b}-\overrightarrow{c}$.\\
Ta có $\overrightarrow{AC'}\cdot\overrightarrow{MN}=\left(\overrightarrow{a}+\overrightarrow{b}+\overrightarrow{c}\right)\left[\dfrac{x}{a}\overrightarrow{a}+\left(1-\dfrac{x}{a}\right)\overrightarrow{b}-\overrightarrow{c}\right]$.\\
Vì $\overrightarrow{a}\cdot\overrightarrow{b}=\overrightarrow{a}\cdot\overrightarrow{c}=\overrightarrow{b}\cdot\overrightarrow{c}=0$ nên ta có
\begin{center}
$\overrightarrow{AC'}\cdot\overrightarrow{MN}=\dfrac{x}{a}{\overrightarrow{a}}^2+\left(1-\dfrac{x}{a}\right){\overrightarrow{b}}^2-{\overrightarrow{c}}^2=x\cdot a+\left(1-\dfrac{x}{a}\right)a^2-a^2=0$.
\end{center}
Vậy góc giữa vectơ $\overrightarrow{AC'}$ và $\overrightarrow{MN}$ bằng $90^\circ$.
\end{itemchoice}
}
\end{ex}

\begin{ex}%[2-H2B4-SO-10-2425 (Nguồn: Bài 4 - Đề 1 - Ôn Tập Chương II)]%[VN-MT-7, Trần Bảo Hiên]%[2H2V2-5]
Trong không gian với hệ tọa độ $Oxyz$, cho hình bình hành $ABCD$ có $A(-3;4;2)$, $B(-5;6;2)$, $C(-10;17;-7)$.
\choiceTF
{\True Tọa độ trung điểm của $AB$ là $I(-4;5;2)$}
{\True Tọa độ trọng tâm của tam giác $ABC$ là $G(-6;9;-1)$}
{$\overrightarrow{AB}\cdot\overrightarrow{AD}=10$}
{Tọa độ trực tâm của tam giác $ABD$ là $H(-5;12;4)$}
\loigiai{
\begin{itemchoice}
\itemch \textbf{Đúng}.\\
Gọi $I$ là trung điểm của $AB$. Khi đó $\heva{&x_I=\dfrac{x_A+x_B}{2}=\dfrac{-3+(-5)}{2}=-4\\&y_I=\dfrac{y_A+y_B}{2}=\dfrac{4+6}{2}=5\\&z_I=\dfrac{z_A+z_B}{2}=\dfrac{2+2}{2}=2.}$\\
Vậy $I(-4;5;2)$.
\itemch \textbf{Đúng}.\\
Gọi $G$ là trọng tâm của tam giác $ABC$.\\
Khi đó $\heva{&x_G=\dfrac{x_A+x_B+x_C}{3}=\dfrac{-3+(-5)+(-10)}{3}=-6\\&y_I=\dfrac{y_A+y_B+y_C}{3}=\dfrac{4+6+17}{3}=9\\&z_I=\dfrac{z_A+z_B+z_C}{3}=\dfrac{2+2+(-7)}{3}=-1.}$\\
Vậy $G(-6;9;-1)$.
\itemch \textbf{Sai}.\\
Ta có $\overrightarrow{AB}=(-2;2;0)$, $\overrightarrow{DC}=(-10-x_D;17-y_D;-7-z_D)$.
Vì $ABCD$ là hình bình hành nên $\overrightarrow{AB}=\overrightarrow{DC}\Leftrightarrow\heva{&-10-x_D=-2\\&17-y_D=2\\&-7-z_D=0}\Leftrightarrow\heva{&x_D=-8\\&y_D=15\\&z_D=-7}\Rightarrow D(-8;15;-7)$.\\
$\overrightarrow{AD}=(-5;11;-9)$. Do đó $\overrightarrow{AB}\cdot\overrightarrow{AD}=-2\cdot(-5)+2\cdot11+0\cdot(-9)=32$.
\itemch \textbf{Sai}.\\
Gọi $H(a;b;c)$ là trực tâm tam giác $ABD$. Do $\left[\overrightarrow{AB},\overrightarrow{AD}\right]$ có giá vuông góc với $(ABC)$ nên nó vuông góc với vectơ $\overrightarrow{AH}$.\\
Do đó $\heva{&\overrightarrow{AH}\perp\overrightarrow{BD}\\&\overrightarrow{BH}\perp\overrightarrow{AD}\\&\left[\overrightarrow{AB},\overrightarrow{AD}\right]\cdot\overrightarrow{AH}=0.}$\\
Ta có 
$\overrightarrow{AH}=(a+3;b-4;c-2)$, $\overrightarrow{BH}=(a+5;b-6;c-2)$, $\overrightarrow{BD}=(-3;9;-9)$, $\overrightarrow{AD}=(-5;11;-9)$, $\overrightarrow{AB}=(-2;2;0)$, 
$\left[\overrightarrow{AB},\overrightarrow{AD}\right]=(-18;-18;-12)$.
Suy ra 
\begin{center}
$\heva{&-3(a+3)+9(b-4)-9(c-2)=0\\&-5(a+5)+11(b-6)-9(c-2)=0\\&-18(a+3)-18(b-4)-12(c-2)=0}\Leftrightarrow\heva{&-3a+9b-9c=27\\&-5a+11b-9c=73\\&-18a-18b-12c=-42}\Leftrightarrow\heva{&a=-\dfrac{153}{11}\\&b=\dfrac{100}{11}\\&c=\dfrac{118}{11}.}$
\end{center}
Vậy $H\left(-\dfrac{153}{11};\dfrac{100}{11};\dfrac{118}{11}\right)$.
\end{itemchoice}
}
\end{ex}

\begin{ex}%[2-H2B4-SO-10-2425 (Nguồn: Bài 4 - Đề 1 - Ôn Tập Chương II)]%[VN-MT-7, Trần Bảo Hiên]%[2H2H2-4]
Trong không gian với hệ tọa độ $Oxyz$, cho các điểm $A(2;1;-1)$, $B(3;1;0)$, $C(-1;1;3)$.
\choiceTF
{\True Ba điểm $A$, $B$, $C$ không thẳng hàng}
{\True Ba điểm $A$, $B$, $D(4;1;1)$ thẳng hàng}
{Góc $\widehat{ABC}=45^\circ$}
{\True $\left[\overrightarrow{AB},\overrightarrow{AC}\right]=(0;-7;0)$}
\loigiai{
\begin{itemchoice}
\itemch \textbf{Đúng}.\\
Ta có $\overrightarrow{AB}=(1;0;1)$, $\overrightarrow{AC}=(-3;0;4)$, $\overrightarrow{AB}\ne k\overrightarrow{AC}=(-3k;0;4k)$ với mọi $k$ nên hai vectơ $\overrightarrow{AB}$ và $\overrightarrow{AC}$ không cùng phương. Do đó ba điểm $A$, $B$, $C$ không thẳng hàng.
\itemch \textbf{Đúng}.\\
Ta có $\overrightarrow{AB}=(1;0;1)$, $\overrightarrow{AD}=(2;0;2)\Rightarrow\overrightarrow{AD}=2\overrightarrow{AB}$. Do đó ba điểm $A$, $B$, $D$ thẳng hàng.
\itemch \textbf{Sai}.\\
Ta có $\overrightarrow{BA}=(-1;0;-1)$, $\overrightarrow{BC}=(-4;0;3)$, suy ra
\begin{center}
$\cos\widehat{ABC}=\cos\left(\overrightarrow{BA},\overrightarrow{BC}\right)=\dfrac{\overrightarrow{BA}\cdot\overrightarrow{BC}}{\left\vert\overrightarrow{BA}\right\vert\cdot\left\vert\overrightarrow{BC}\right\vert}=\dfrac{1}{5\sqrt{2}}\Rightarrow\widehat{ABC}\approx 82^\circ$.\\
\end{center}

\itemch \textbf{Đúng}.\\
Ta có $\overrightarrow{AB}=(1;0;1)$, $\overrightarrow{AC}=(-3;0;4)\Rightarrow\left[\overrightarrow{AB},\overrightarrow{AC}\right]=(0;-7;0)$.
\end{itemchoice}
}
\end{ex}

\begin{ex}%[2-H2B4-SO-10-2425 (Nguồn: Bài 4 - Đề 1 - Ôn Tập Chương II)]%[VN-MT-7, Trần Bảo Hiên]%[2H2V2-5]
Trong không gian với hệ tọa độ $Oxyz$, cho các điểm $A(1;1;2)$, $B(3;-1;2)$, $C(2;0;1)$.
\choiceTF
{\True Ba điểm $A$, $B$, $C$ không thẳng hàng}
{\True Điểm $M(a;b;3)$ thỏa mãn ba điểm $A$, $C$, $M$ thẳng hàng thì $a+b=2$}
{Góc $\alpha$ là góc tạo bởi hai vectơ $\overrightarrow{AB}$, $\overrightarrow{BC}$ thì $\cos\alpha=-1$}
{Gọi điểm $M(a;b;3)$ thỏa mãn ba điểm $A$, $C$, $M$ thẳng hàng. Khi đó $\left[\overrightarrow{AB},\overrightarrow{AM}\right]=(1;1;2)$}
\loigiai{
\begin{itemchoice}
\itemch \textbf{Đúng}.\\
Ta có $\overrightarrow{AB}=(2;-2;0)$, $\overrightarrow{BC}=(-1;1;-1)$. Suy ra $\overrightarrow{AB}\ne k\cdot\overrightarrow{BC}$ với mọi $k\in\mathbb{R}$ nên ba điểm $A$, $B$, $C$ không thẳng hàng.
\itemch \textbf{Đúng}.\\
Ta có $\overrightarrow{AC}=(1;-1;-1)$, $\overrightarrow{AM}=(a-1;b-1;1)$.\\
Ba điểm $A$, $C$, $M$ thẳng hàng khi và chỉ khi $\overrightarrow{AB}=k\overrightarrow{AM}\Leftrightarrow\heva{&1=k(a-1)\\&-1=k(b-1)\\&-1=k}\Leftrightarrow\heva{&a=0\\&b=2\\&k=-1.}$\\
Vậy $a+b=2$.
\itemch \textbf{Sai}.\\
\begin{center}
$\cos\alpha=\cos\left(\overrightarrow{AB},\overrightarrow{BC}\right)=\dfrac{\overrightarrow{AB}\cdot\overrightarrow{BC}}{\left\vert\overrightarrow{AB}\right\vert\cdot\left\vert\overrightarrow{BC}\right\vert}=\dfrac{2\cdot(-1)+(-2)\cdot1+0\cdot(-1)}{2\sqrt{2}\cdot\sqrt{3}}=-\dfrac{\sqrt{6}}{3}$.
\end{center}
\itemch \textbf{Sai}.\\
$\overrightarrow{AB}=(2;-2;0)$, $\overrightarrow{AM}=(-1;1;1)$.\\
$\left[\overrightarrow{AB},\overrightarrow{AM}\right]=(-2;-2;0)$.
\end{itemchoice}
}
\end{ex}
\Closesolutionfile{ans}

\TNSA
\Opensolutionfile{ans}[ans/ans\currfilebase-Phan-III]

\begin{ex}%[2-H2B4-SO-10-2425 (Nguồn: Bài 4 - Đề 1 - Ôn Tập Chương II)]%[VN-MT-7, Trần Bảo Hiên]%[2H2V2-4]
Cho hình chóp $S.ABC$ có $SA=SB=SC=AB=AC=a$, $BC=a\sqrt{2}$. Góc giữa hai véctơ $\overrightarrow{AB}$ và $\overrightarrow{SC}$ bằng bao nhiêu độ?
\shortans{120}
\loigiai{
\begin{center}
\begin{tikzpicture}[scale=0.8,font=\footnotesize,line join=round,line cap=round,>=stealth]
\coordinate (S) at ($(A)+(50:3)$);
\coordinate (A) at (0,0);
\coordinate (B) at ($(A)+(-60:3)$);
\coordinate (C) at ($(A)+(0:5)$);
\draw(S)--(A)--(B)--(C)--(S)--(B);
\draw[dashed](A)--(C);
\foreach \i/\g in {S/90,A/180,B/-90,C/0}{\fill (\i) circle (1.0pt) ($(\i)+(\g:3mm)$)node[scale=1]{$\i$};}
\end{tikzpicture}
\end{center}
Tam giác $ABC$ có $AB=AC=a$, $BC=a\sqrt{2}\Rightarrow\triangle ABC$ vuông tại $A\Rightarrow\overrightarrow{AB}\cdot\overrightarrow{AC}=0$.\\
\begin{align*}
\cos\left(\overrightarrow{SC},\overrightarrow{AB}\right)&=\dfrac{\overrightarrow{SC}\cdot\overrightarrow{AB}}{\left\vert\overrightarrow{SC}\right\vert\cdot\left\vert\overrightarrow{AB}\right\vert}=\dfrac{\left(\overrightarrow{SA}+\overrightarrow{AC}\right)\cdot\overrightarrow{AB}}{SC\cdot AB}\\
&=\dfrac{\overrightarrow{SA}\cdot\overrightarrow{AB}+\overrightarrow{AC}\cdot\overrightarrow{AB}}{SC\cdot AB}=\dfrac{SA\cdot AB\cdot \cos120^\circ}{SC\cdot AB}=-\dfrac{1}{2}.
\end{align*}
Suy ra $\left(\overrightarrow{SC},\overrightarrow{AB}\right)=120^\circ$.
}
\end{ex}

\begin{ex}%[2-H2B4-SO-10-2425 (Nguồn: Bài 4 - Đề 1 - Ôn Tập Chương II)]%[VN-MT-7, Trần Bảo Hiên]%[2H2H2-2]
Trong không gian với hệ tọa độ $Oxyz$, cho hình hộp $ABCD.A'B'C'D'$ có $A(1;0;1)$, $B(2;1;2)$, $D(1;-1;1)$, $C'(4;5;-5)$. Giả sử $A'(x;y;z)$, tính $x+y+z$.
\shortans{2}
\loigiai{
\begin{center}
\begin{tikzpicture}[scale=0.8, font=\footnotesize, line join=round, line cap=round, >=stealth]
\path
(0:0) coordinate (B)
(0:4) coordinate (C)
($(B)+(45:2)$) coordinate (A)
($(A)+(C)-(B)$) coordinate (D)
($(A)+(80:3)$) coordinate (A')
($(B)+(80:3)$) coordinate (B')
($(C)+(80:3)$) coordinate (C')
($(D)+(80:3)$) coordinate (D');
\draw[dashed] (B)--(A)--(A') (A)--(D);
\draw (D')--(A')--(B')--(C')--(D')--(D)--(C)--(B)--(B') (C)--(C');
\foreach \x/\g in {A/170,B/170,C/0,D/0,A'/170,B'/170,C'/0,D'/0}
\draw[fill=black] (\x) circle (.5pt)
($(\g:.3)+(\x)$) node {$\x$};
\end{tikzpicture}
\end{center}
Ta có $\overrightarrow{AC'}=(3;5;-6)$, $\overrightarrow{AB}=(1;1;1)$, $\overrightarrow{AD}=(0;-1;0)$.\\
Theo quy tắc hình bình hành ta có $\overrightarrow{AB}+\overrightarrow{AD}+\overrightarrow{AA'}=\overrightarrow{AC'}$, suy ra
\begin{center}
$\overrightarrow{AA'}=\overrightarrow{AC'}-\overrightarrow{AB}-\overrightarrow{AD}\Rightarrow\overrightarrow{AA'}=(2;5;-7)$.
\end{center}
Ta có $A'(x;y;z)\Rightarrow\overrightarrow{AA'}=(2;5;-7)\Leftrightarrow\heva{&x-1=2\\&y=5\\&z-1=-7}\Leftrightarrow\heva{&x=3\\&y=5\\&z=-6}\Rightarrow A'(3;5;-6)$.\\
Vậy $x+y+z=3+5+(-6)=2$.
}
\end{ex}

\begin{ex}%[2-H2B4-SO-10-2425 (Nguồn: Bài 4 - Đề 1 - Ôn Tập Chương II)]%[VN-MT-7, Trần Bảo Hiên]%[2H2V2-5]
Cho hình hộp $ABCD.A'B'C'D'$ có $A(1;0;1)$, $B(2;1;2)$, $D(1;-1;1)$, $C'(4;5;-5)$. Biết rằng có một vectơ $\overrightarrow{v}=(a;b;6)$ vuông góc với cả hai vectơ $\overrightarrow{CC'}$ và $\overrightarrow{C'D'}$. Tính $a+b$.
\shortans{-6}
\loigiai{
\begin{center}
\begin{tikzpicture}[scale=0.8, font=\footnotesize, line join=round, line cap=round, >=stealth]
\path
(0:0) coordinate (B)
(0:4) coordinate (C)
($(B)+(45:2)$) coordinate (A)
($(A)+(C)-(B)$) coordinate (D)
($(A)+(80:3)$) coordinate (A')
($(B)+(80:3)$) coordinate (B')
($(C)+(80:3)$) coordinate (C')
($(D)+(80:3)$) coordinate (D');
\draw[dashed] (B)--(A)--(A') (A)--(D);
\draw (D')--(A')--(B')--(C')--(D')--(D)--(C)--(B)--(B') (C)--(C');
\foreach \x/\g in {A/170,B/170,C/0,D/0,A'/170,B'/170,C'/0,D'/0}
\draw[fill=black] (\x) circle (.5pt)($(\g:.3)+(\x)$) node {$\x$};
\end{tikzpicture}
\end{center}
Ta có $\overrightarrow{AB}=(1;1;1)$, $\overrightarrow{AD}=(0;-1;0)$.\\
Gọi $C(x;y;z)\Rightarrow\overrightarrow{AC}=(x-1;y;z-1)$. Theo quy tắc hình bình hành ta có
\begin{center}
$\overrightarrow{AC}=\overrightarrow{AB}+\overrightarrow{AD}\Leftrightarrow\heva{&x-1=1\\&y=0\\&z-1=1}\Leftrightarrow\heva{&x=2\\&y=0\\&z=2}\Rightarrow\overrightarrow{CC'}=(2;5;-7)$.
\end{center}
Mặt khác $\overrightarrow{C'D'}=\overrightarrow{BA}=(-1;-1;-1)$.\\
Suy ra $\left[\overrightarrow{CC'},\overrightarrow{C'D'}\right]=(-12;9;3)$ là một vectơ thoả mãn yêu cầu bài toán.\\
Ta có $\left[\overrightarrow{CC'},\overrightarrow{C'D'}\right]$ và $\overrightarrow{v}$ cùng phương nên có số thực $k$ để $\left[\overrightarrow{CC'};\overrightarrow{C'D'}\right]=k\cdot\overrightarrow{v}$.\\
Suy ra $\heva{&-12=k\cdot a\\&9=k\cdot b\\&3=k\cdot6}\Leftrightarrow\heva{&a = -24\\&b = 18\\&k=\dfrac{1}{2}.}$\\
Vậy $a+b=-6$.
}
\end{ex}

\begin{ex}%[2-H2B4-SO-10-2425 (Nguồn: Bài 4 - Đề 1 - Ôn Tập Chương II)]%[VN-MT-7, Trần Bảo Hiên]%[2H2V2-6]
Trong một căn phòng dạng hình hộp chữ nhật với chiều dài $8$ m, rộng $6$ m và cao $4$ m có cây quạt treo tường. Cây quạt $A$ treo chính gữa bức tường $8$ m và cách trần $1$ m, cây quạt $B$ treo chính giữa bức tường $6$ m và cách trần $1{,}5$ m. Chọn hệ trục tọa độ $Oxyz$ như hình vẽ bên dưới (đơn vị: mét). Hãy tính độ dài vectơ $\overrightarrow{AB}$ (làm tròn đến hàng đơn vị).
\begin{center}
\definecolor{amber}{rgb}{1.0, 0.75, 0.0}%mau non
\definecolor{antiquebrass}{rgb}{0.8, 0.58, 0.46}%mau da
\definecolor{antiquewhite}{rgb}{0.98, 0.92, 0.84}%mau ao
\definecolor{cadmiumgreen}{rgb}{0.0, 0.42, 0.24}%mau quan
\definecolor{cadetblue}{rgb}{0.37, 0.62, 0.63}%mau but
\definecolor{brown(traditional)}{rgb}{0.59, 0.29, 0.0}%mau giay
\definecolor{brilliantlavender}{rgb}{0.96, 0.73, 1.0}%màu sơn tím
\definecolor{brightube}{rgb}{0.82, 0.62, 0.91}%màu sơn tím đậm
%---------------màu quạt
\definecolor{burntorange}{rgb}{0.8, 0.33, 0.0}
\definecolor{arsenic}{rgb}{0.23, 0.27, 0.29}
\definecolor{battleshipgrey}{rgb}{0.52, 0.52, 0.51}
\begin{tikzpicture}[line join=round, line cap=round,scale=0.7,transform shape,font=\large]
\clip (1,-1) rectangle (16,13);
%\draw[gray!50] (-3,-3) grid (3,4);
\definecolor{burntsienna}{rgb}{0.91, 0.45, 0.32}
\tikzset{mai/.pic={
\def\mainha{
(.5,2)
foreach \n in {1,2,...,22} { -- ++ (0,0) -- ++ (0,1) -- ++ (1,0) -- ++ (0,-1) } -- cycle
(.5,1)
foreach \n in {1,2,...,22} { -- ++ (0,0) -- ++ (0,1) -- ++ (1,0) -- ++ (0,-1) } -- cycle
(.5,0)
foreach \n in {1,2,...,22} { -- ++ (0,0) -- ++ (0,1) -- ++ (1,0) -- ++ (0,-1) } -- cycle
(.5,-1)
foreach \n in {1,2,...,22} { -- ++ (0,0) -- ++ (0,1) -- ++ (1,0) -- ++ (0,-1) } -- cycle
(.5,-2)
foreach \n in {1,2,...,22} { -- ++ (0,0) -- ++ (0,1) -- ++ (1,0) -- ++ (0,-1) } -- cycle
(.5,-3)
foreach \n in {1,2,...,22} { -- ++ (0,0) -- ++ (0,1) -- ++ (1,0) -- ++ (0,-1) } -- cycle
(.5,-4)
foreach \n in {1,2,...,22} { -- ++ (0,0) -- ++ (0,1) -- ++ (1,0) -- ++ (0,-1) } -- cycle;
}
\clip (-4,-3)--(17.5,-3)--(17.5,3)--(.5,3)--cycle;
\draw[white,fill=burntsienna!70] \mainha;
\draw (17.5,-3)--(17.5,3)--(17.5,9);}}
\fill[brilliantlavender] (18,-3)--(18,13)--(12,13)--(12,-3)--cycle;
\fill[brightube] (12,13)--(12,3)--(-18,3)--(-18,13)--cycle;
\path(-2.5,0)pic[scale=1,xslant=-1]{mai};
\path
(12,3) coordinate (O)
(3,3) coordinate (x)
(12,13) coordinate (z)
(15,0) coordinate (y);
\foreach\p/\g in {y/160,x/45, z/-45,O/-90}
{
\node at (\p) [shift=(\g:4mm)] {$\p$};
}
\draw[line width=.5mm,->] (O)--(z) ;
\draw[line width=.5mm,->] (O)--(x);
\draw[line width=.5mm,->] (O)--(y);
\node at ($(O)+(-6,4.5)$) {$8$ m};
\draw[red,line width=.5mm,<->] ($(O)+(0,4)$)--($(x)+(-2,4)$);
%===========================A FAN
\tikzset{fan/.pic={
%chân quạt
\draw[fill=battleshipgrey](-.2,.8)--(.2,.8)--(.4,-2.2)
..controls +(-120:.3) and +(-60:.3) ..(-.4,-2.2)--cycle;
%---Nút bấm
\foreach \i in{-1.95,-1.7}{%-1.45
\draw[fill=arsenic](-.15,\i) rectangle (.15,\i+.15);
}
%-----------------------------------------------------
\draw[black](0,.8) circle (2.25cm);
\draw[black](0,.8) circle (2.15cm);
\draw[black](0,.8) circle (1.42cm);
\draw[black](0,.8) circle (1.48cm);
\draw[fill=black](0,.8) circle (6mm);
\draw[fill=arsenic](0,.8) circle (5mm);
\def\N{
(0,.8)
..controls +(145:1.3) and +(170:1) ..(0,2.8)
..controls +(-10:1.4) and +(-20:1) ..(.6,1.76)
..controls +(160:.4) and +(100:.4) ..(.2,.9)--cycle;
}
\foreach \i/\j/\k in {0/0/0,120/.7/-1.2,240/-.7/-1.2}
{
\draw[black,rotate=\i,shift={(\j,\k)}]\N;
\fill[burntorange,rotate=\i,shift={(\j,\k)}] \N;
}
%lồng quạt
\def\r{2.15}
\foreach \i in {0,15,25,35,...,365}
\draw[double] ($(\i:\r)+(0,.8)$)--(0,.8);
\draw[fill=arsenic](0,.8) circle (3.5mm);
}}
\path
(6,11)pic[scale=.6]{fan}
(14,9)pic[scale=.55,yslant=-.3]{fan};
\end{tikzpicture}
\end{center}
\shortans{5}
\loigiai{
Từ hình vẽ $A\in(Oxz)$ nên $A(x;0;z)$ và $B\in(Oyz)$ nên $B(0;y;z)$.\\
Cây quạt $A$ treo chính giữa bức tường $8$ m và cách trần $1$ m nên $A(4;0;3)$.\\
Cây quạt $B$ treo chính giữa bức tường $6$ m và cách trần $1{,}5$ m nên $B\left(0;3;\dfrac{5}{2}\right)$.\\
Khi đó $\overrightarrow{AB}=\left(-4;3;-\dfrac{1}{2}\right)\Rightarrow\left\vert\overrightarrow{AB}\right\vert=\sqrt{(-4)^2+3^2+\left(-\dfrac{1}{2}\right)^2}\approx 5$ m.
}
\end{ex}

\begin{ex}%[2-H2B4-SO-10-2425 (Nguồn: Bài 4 - Đề 1 - Ôn Tập Chương II)]%[VN-MT-7, Trần Bảo Hiên]%[2H2V2-4]
Một chi tiết trong bộ trang sức được gắn hệ trục tọa độ $Oxyz$ như hình vẽ. Các hình chóp $S.ABCD$ và $I.ABCD$ là các hình chóp đều cạnh $1$ cm. Tính số đo góc nhị diện $[S,CD,I]$ theo đơn vị độ, làm tròn đến hàng đơn vị.
\shortans{109}
\loigiai{
\begin{center}
\begin{tikzpicture}[scale=0.8,font=\footnotesize,line join=round,line cap=round,>=stealth]
\def \a{6}
\def \h{5}
\path (0:0) coordinate (A)
++(0:\a) coordinate (D)
++(-138:0.45*\a) coordinate (C)
($(A)+(C)-(D)$) coordinate (B)
($(A)!0.5!(C)$) coordinate (O)
($(O)+(90:\h)$) coordinate (S)
($(S)!-0.2!(O)$) coordinate (H)
($(C)!-0.4!(A)$) coordinate (K)
($(D)!-0.2!(B)$) coordinate (L)
($(D)!0.5!(C)$) coordinate (M)
($(O)!1!180:(S)$) coordinate (I);
\draw [dashed] (B)--(A)--(D) (A)--(S) (A)--(C) (B)--(D) (S)--(O)--(I)--(A);
\draw (B)--(C)--(D) (B)--(S) (C)--(S) (D)--(S) (C)--(I)--(B) (I)--(D);
\draw[->] (S)--(H)node[right]{$z$};
\draw[->] (C)--(K)node[above]{$x$};
\draw[->] (D)--(L)node[above]{$y$};
\foreach \x/\g in {A/135,B/180,C/80,D/70,S/40,O/50,I/-90,M/150}
\fill (\x) circle (1pt)
($(\g:3mm)+(\x)$) node {$\x$};
\end{tikzpicture}
\end{center}
Ta có $ABCD$ là hình vuông cạnh $1$ cm nên $OC=OD=\dfrac{\sqrt{2}}{2}$.\\
Xét $\triangle SOC$ vuông tại $O$, ta có $OS=\sqrt{SC^2-OC^2}=\sqrt{1^2-\left(\dfrac{\sqrt{2}}{2}\right)^2}=\dfrac{\sqrt{2}}{2}$.\\
Xét $\triangle IOC$ vuông tại $O$, ta có $OI=\sqrt{IC^2-OC^2}=\sqrt{1^2-\left(\dfrac{\sqrt{2}}{2}\right)^2}=\dfrac{\sqrt{2}}{2}$.\\
Vậy $C\left(\dfrac{\sqrt{2}}{2};0;0\right)$, $D\left(0;\dfrac{\sqrt{2}}{2};0\right)$, $S\left(0;0;\dfrac{\sqrt{2}}{2}\right)$, $I\left(0;0;-\dfrac{\sqrt{2}}{2}\right)$.\\
Gọi $M$ là trung điểm của $CD$ thì $M\left(\dfrac{\sqrt{2}}{4};\dfrac{\sqrt{2}}{4};0\right)$.\\
Ta có $\heva{&CD\perp MI\\&CD\perp MS}\Rightarrow[S,CD,I]=\widehat{SMI}$.\\
Ta có $\overrightarrow{MS}=\left(-\dfrac{\sqrt{2}}{4};-\dfrac{\sqrt{2}}{4};\dfrac{\sqrt{2}}{2}\right)$, $\overrightarrow{MI}=\left(-\dfrac{\sqrt{2}}{4};-\dfrac{\sqrt{2}}{4};-\dfrac{\sqrt{2}}{2}\right)$.\\
$\Rightarrow\overrightarrow{MS}\cdot\overrightarrow{MI}=-\dfrac{1}{4}$, $\left\vert\overrightarrow{MS}\right\vert=\dfrac{\sqrt{3}}{2}$, $\left\vert\overrightarrow{MI}\right\vert=\dfrac{\sqrt{3}}{2}$.
\begin{align*}
\cos\widehat{SMI}=\cos\left(\overrightarrow{MS},\overrightarrow{MI}\right)=\dfrac{\overrightarrow{MS}\cdot\overrightarrow{MI}}{\left\vert\overrightarrow{MS}\right\vert\cdot\left\vert\overrightarrow{MI}\right\vert}=\dfrac{-\dfrac{1}{4}}{\dfrac{\sqrt{3}}{2}\cdot\dfrac{\sqrt{3}}{2}}=-\dfrac{1}{3}\Rightarrow\widehat{SMI}\approx109^\circ.
\end{align*}
}
\end{ex}

\begin{ex}%[2-H2B4-SO-10-2425 (Nguồn: Bài 4 - Đề 1 - Ôn Tập Chương II)]%[VN-MT-7, Trần Bảo Hiên]%[2H2C1-4]
\immini[thm]{Một chiếc ô tô được đặt trên mặt đáy dưới của một khung sắt có dạng hình hộp chữ nhật với đáy trên là hình chữ nhật $ABCD$, mặt phẳng $(ABCD)$ song song với mặt phẳng nằm ngang. Khung sắt đó được buộc vào móc $E$ của chiếc cần cẩu sao cho các đoạn dây cáp $EA$, $EB$, $EC$ và $ED$ có độ dài bằng nhau và cùng tạo với mặt phẳng $(ABCD)$ một góc bằng $60^\circ$ (hình minh họa). Chiếc cần cẩu đang kéo khung sắt lên theo phương thẳng đứng. Biết rằng các lực căng $\overrightarrow{F_1}$, $\overrightarrow{F_2}$, $\overrightarrow{F_3}$, $\overrightarrow{F_4}$ đều có cường độ là $4{,}7$ kN và trọng lượng của khung sắt là $3$ kN. Tính trọng lượng của chiếc xe ô tô (làm tròn đến hàng phần chục)?
}
{
\begin{tikzpicture}[>=latex,line join=round, line cap=round,scale=0.8,transform shape]
\definecolor{bostonuniversityred}{rgb}{0.8, 0.0, 0.0}
\definecolor{charcoal}{rgb}{0.21, 0.27, 0.31}
\definecolor{bananayellow}{rgb}{1.0, 0.88, 0.21}
\definecolor{anti-flashwhite}{rgb}{0.95, 0.95, 0.96}
% \clip (-6,-3) rectangle (6,3);
\tikzset{%
xeoto/.pic={%
%--------------------------
\tikzset{xe/.pic={
\def\N{
(-2.7,.56)--(-2.5,.56)
..controls +(50:1.5) and +(165:1.5) .. (2.1,1.88)--(2.05,2)
..controls +(-10:.1) and +(130:.1) .. (3.25,1.75)--(3.15,1.65)
..controls +(-4:.2) and +(130:.15) .. (4.05,.7)--(4.25,.75)
..controls +(-40:.2) and +(130:.15) .. (4.55,.35)--(4.35,.26)
..controls +(-40:.2) and +(130:.15) .. (4.8,-.45)--(4.92,-.4)
..controls +(-40:.25) and +(73:.17) .. (4.8,-1.8)--(-4.4,-1.8)
..controls +(175:.7) and +(-175:3.2) ..cycle;
}
\fill[bostonuniversityred] \N;
\draw \N;
\def\K{
(-2.2,.56)--(3.3,.7)
..controls +(100:1.18) and +(43:3) .. cycle;
}
\fill[bottom color=charcoal,top color=charcoal!20!white, middle color=charcoal!80!white] \K;
\draw \K;
\def\K1{
(-2.2,.56)..controls +(43:.2) and +(43:.2) .. (-1.58,1.05)--(-1.53,.57)--cycle;
}
\draw \K1;
\fill[charcoal] \K1;
\def\K2{
(1.2,1.85)..controls +(-10:.1) and +(160:.1) .. (1.58,1.8)--(1.8,.65)--(1.25,.65)--cycle;
}
\draw \K2;
\fill[charcoal] \K2;
\def\Kt{
(-2.5,.56)..controls +(50:1.5) and +(165:1.5) .. (2.1,1.88)--(2.05,2)
..controls +(170:2.2) and +(45:1.5) .. (-2.7,.56)--cycle;
}
\fill[charcoal!50] \Kt;
\draw \Kt;
\def\Ks{
(3.25,1.75)--(3.15,1.65)
..controls +(-4:.2) and +(130:.15) .. (4.05,.7)--(4.22,.75)
..controls +(120:.3) and +(-35:.3) .. cycle;
}
\fill[charcoal!50] \Ks;
\draw \Ks;
%Đèn sau
\def\D{
(4.55,.35)--(4.35,.26)
..controls +(-40:.2) and +(130:.15) .. (4.8,-.45)--(4.92,-.4)
..controls +(110:.2) and +(-40:.15) ..cycle;
}
\fill[bananayellow] \D;
\draw \D;
\def\M{
(2.2,-1.3)--(-1.8,-1.4)--(-1.78,-1.7)
..controls +(-5:.3) and +(-90:.6) ..cycle;
}
\draw \M;
\fill[charcoal!90] \M;
\draw (-1.6,.55)..controls +(-170:.5) and +(95:.4) .. (-1.78,-1.7)
(1.6,.65)..controls +(-30:.5) and +(35:.3) .. (1.7,-1.3);
%gương
\def\G{ (-1.5,.45)--(-1.4,.6)..controls +(85:1) and +(20:.6) .. (-1.25,.5)--(-1.4,.33);
}
\draw \G;
\fill[bostonuniversityred] \G;
%Đèn trước
\def\Dt{
(-4.85,-.7)..controls +(75:1) and +(65:.8) .. (-4.5,-.7)
..controls +(-115:.6) and +(-105:.4) .. cycle;
}
\fill[bananayellow] \Dt;
\draw \Dt;
\def\Dt2{
(-4.85,-.7)
..controls +(75:.6) and +(65:.4) .. (-4.7,-.7)
..controls +(-115:.3) and +(-105:.2) .. cycle;
}
\fill[anti-flashwhite] \Dt2;
\draw \Dt2;
\draw[fill=anti-flashwhite] (-4.86,-1.45)--(-4.82,-1.5)--(-4.55,-1.3)
..controls +(90:.3) and +(45:.2) .. cycle;
}}
\tikzset{banh_xe/.pic={
\draw[fill=charcoal] (-3.25,-1.65) circle (1) ;
\draw[fill=anti-flashwhite] (-3.25,-1.65) circle (.7) ;
\draw[fill=charcoal] (-3.25,-1.65) circle (.4) ;
}}
%----------------
\path
(0,0)pic[scale=1]{xe}(0,0)pic[scale=1]{banh_xe}(6.9,0)pic[scale=1]{banh_xe};
}}
\def\bc{4.25} % cạnh BC
\def\ba{1.5} % cạnh BA
\def\h{4} % đường cao
\def\gocnghieng{90} % góc nghiêng
\def\gocB{160} % góc B của đáy
\coordinate (B1) at (0,0);
\coordinate (A1) at (\gocB:\ba);
\coordinate (C1) at (\bc,0.25);
\coordinate (D1) at ($(C1)-(B1)+(A1)$);
\coordinate[label=above left:$A$] (A) at ($(A1)+(\gocnghieng:\h)$);
\coordinate[label=below left:$B$] (B) at ($(B1)-(A1)+(A)$);
\coordinate[label=right:$C$] (C) at ($(C1)-(A1)+(A)$);
\coordinate[label=above right:$D$] (D) at ($(D1)-(A1)+(A)$);
\coordinate (E) at ($(A)!0.5!(C)+(\gocnghieng:\h)$);
%------------
\draw[->,blue,very thick] (E)--($(E)!0.4!(A)$) node[above left]{$\overrightarrow{F_1}$};
\draw[->,blue,very thick] (E)--($(E)!0.4!(B)$) node[right]{$\overrightarrow{F_2}$};
\draw[->,blue,very thick] (E)--($(E)!0.4!(C)$) node[above right]{$\overrightarrow{F_3}$}; \draw[->,blue,very thick] (E)--($(E)!0.4!(D)$) node[left=2pt]{$\overrightarrow{F_4}$};
%------------
\path (E) node[left=1mm]{$E$};
\draw[blue,very thick] (A)--(B)--(C)--(D)--cycle
(A1)--(A) (D1)--(D) (C1)--(C)
(A)--(E)--(B) (C)--(E)--(D);
\draw[fill=teal] (A1)--(B1)--(C1)--(D1)--cycle;
\draw[fill=teal!30] (A1)--(B1)--(C1)--++(0,-0.3)--([yshift=-0.3cm]B1)--([yshift=-0.3cm]A1)--cycle;
\foreach \diem in {A1,B1,C1,D1,A,B,C,D,E} \fill (\diem)circle(1.5pt);
%phần móc và dây
\def\r{0.3}\def\rr{0.25}
\coordinate (tam) at ([yshift=6mm]E);
\draw[brown,fill=brown,line width=1pt] (tam) circle (\r cm);
\fill (tam) circle (2pt);
\draw[brown,line width=1pt] (tam)++(\r,0)--++(0,0.7)(tam)++(-\r,0)--++(0,0.7);
\draw[line width=1.5pt] (tam)--++(0,-1.35*\r) arc(90:370:1mm);
%%%%%%%%%%%%%%%%%%%
\pic[scale=0.45,rotate=4] at (1.6,1.3) [pic type = xeoto];
%--------
\draw[blue,very thick] (B)--(B1);
\end{tikzpicture}
}
\shortans{13{,}3}
\loigiai{
\begin{center}
\begin{tikzpicture}[scale=1,font=\footnotesize,line join=round,line cap=round,>=stealth]
\def \a{4.5}
\def \h{4}
\path (0:0) coordinate (A_1)
++(0:\a) coordinate (D_1)
++(-138:0.45*\a) coordinate (C_1)
($(A_1)+(C_1)-(D_1)$) coordinate (B_1)
($(A_1)!0.5!(C_1)$) coordinate (O)
($(O)+(90:\h)$) coordinate (E)
($(A_1)!0.5!(E)$) coordinate (F_1)
($(B_1)!0.5!(E)$) coordinate (F_2)
($(C_1)!0.5!(E)$) coordinate (F_3)
($(D_1)!0.5!(E)$) coordinate (F_4);
\draw [dashed] (B_1)--(A_1)--(D_1) (A_1)--(C_1) (B_1)--(D_1) (E)--(O);
\draw (B_1)--(C_1)--(D_1);
\draw[dashed,->] (E)--(A_1);
\draw[->] (E)--(B_1);
\draw[->] (E)--(C_1);
\draw[->] (E)--(D_1);
\foreach \x/\g in {A_1/-90,B_1/-135,C_1/-45,D_1/45,E/90,O/-90}
\fill (\x) circle (1pt)($(\g:3mm)+(\x)$) node {$\x$};
\foreach \y/\m in {F_1/20,F_2/135,F_3/45,F_4/45}
\draw ($(\m:3mm)+(\y)$) node {$\overrightarrow{\y}$};
\end{tikzpicture}
\end{center}
Gọi $A_1$, $B_1$, $C_1$, $D_1$ lần lượt là các điểm sao cho $\overrightarrow{EA_1}=\overrightarrow{F_1}$, $\overrightarrow{EA_2}=\overrightarrow{F_2}$, $\overrightarrow{EA_3}=\overrightarrow{F_3}$, $\overrightarrow{EA_4}=\overrightarrow{F_4}$.\\
Vì $EA=EB=EC=ED$ và cùng tạo với mặt phẳng $(ABCD)$ một góc bằng $60^\circ$, các lực căng $\overrightarrow{F_1}$, $\overrightarrow{F_2}$, $\overrightarrow{F_3}$, $\overrightarrow{F_4}$ đều có cường độ là $4{,}7$ kN nên $EA_1=EB_1=EC_1=ED_1$ và cùng tạo với mặt phẳng $\left(A_1B_1C_1D_1\right)$ một góc bằng $60^\circ$.\\
Vì $ABCD$ là hình chữ nhật nên $A_1B_1C_1D_1$ cũng là hình chữ nhật.\\
Gọi $O$ là tâm của hình chữ nhật $A_1B_1C_1D_1$.\\
Ta suy ra $EO\perp \left(A_1B_1C_1D_1\right)$.\\
Do đó, $\left(EA_1,\left(A_1B_1C_1D_1\right)\right)=\widehat{EA_1O}=60^\circ$.\\
Ta có $\left\vert\overrightarrow{F_1}\right\vert=\left\vert\overrightarrow{F_2}\right\vert=\left\vert\overrightarrow{F_3}\right\vert=\left\vert\overrightarrow{F_4}\right\vert=4{,}7$ kN nên $EA_1=EB_1=EC_1=ED_1=4{,}7$.\\
Tam giác $EA_1O$ vuông tại $O$ nên $EO=EA_1\cdot\sin\widehat{EA_1O}=2{,}35\sqrt{3}$.\\
Ta có
\begin{align*}
\overrightarrow{F_1}+\overrightarrow{F_2}+\overrightarrow{F_3}+\overrightarrow{F_4}&=\overrightarrow{EA_1}+\overrightarrow{EA_2}+\overrightarrow{EA_3}+\overrightarrow{EA_4}\\
&=\overrightarrow{EO}+\overrightarrow{OA_1}+\overrightarrow{EO}+\overrightarrow{OA_2}+\overrightarrow{EO}+\overrightarrow{OA_3}+\overrightarrow{EO}+\overrightarrow{OA_4}\\
&=4\overrightarrow{EO}+\left(\overrightarrow{OA_1}+\overrightarrow{OC_1}\right)+\left(\overrightarrow{OB_1}+\overrightarrow{OD_1}\right)\\
&=4\overrightarrow{EO}.
\end{align*}
Gọi $\overrightarrow{P}$ là trọng lực của khung sắt có chứa chiếc ô tô. Khi đó ta có 
\begin{align*}
 \overrightarrow{P} = \overrightarrow{F}_1 + \overrightarrow{F}_2 + \overrightarrow{F}_3 + \overrightarrow{F}_4 = 4\overrightarrow{EO}.
\end{align*}
Suy ra trọng lượng của khung sắt có chứa ô tô là $\left|\overrightarrow{P}\right| = 4\left\vert\overrightarrow{EO}\right\vert=4\cdot2{,}35\sqrt{3}=9{,}4\sqrt{3}$ kN.\\
Vì trọng lượng của khung sắt là $3$ kN nên trọng lượng của chiếc xe ô tô là $9{,}4\sqrt{3}-3\approx13{,}3$ kN.
}
\end{ex}
\Closesolutionfile{ans}
 
% \begin{indapan}
% 	{ans/ans\currfilebase}
% \end{indapan}


% \begin{name}
	{\tenchude}
	{ĐỀ ÔN TẬP CHƯƠNG II}
	{LỚP TOÁN THẦY PHÁT}
	{\thoigian}
\end{name}

\TN
\Opensolutionfile{ans}[ans/ans\currfilebase-Phan-I]
\begin{ex}%[2-H2B4-SO-11-2425]%[VN-MT-7, Đào Trung Kiên]%[2H2N1-2]
Cho tứ diện $ABCD$. Có bao nhiêu vectơ có điểm đầu là $A$ và điểm cuối là một trong các đỉnh còn lại của tứ diện?
\choice{$1$}
{$2$}
{\True $3$}
{$4$}
\loigiai{
\begin{center}
\begin{tikzpicture}[font=\footnotesize, line join=round, line cap=round, >=stealth, scale=0.9]
 \def\a{4}
 \path (0:0) coordinate (B)
 ++(0:\a) coordinate (D)
 ++(-120:\a/2) coordinate (C)
 ($(B)+(60:\a)$) coordinate (A);
 \draw[dashed] (B)--(D);
 \draw[->] (A)--(C);
 \draw[->] (A)--(D);
 \draw[->] (A)--(B);
 \draw (B)--(C)--(D);
 \foreach \x/\g in {A/90,B/180,C/-45,D/0}
 \fill (\x) circle (1pt)
 ($(\g:3mm)+(\x)$) node {$\x$};
\end{tikzpicture}
\end{center}
Ba vectơ $\overrightarrow{AB}$, $\overrightarrow{AC}$, $\overrightarrow{AD}$.
}
\end{ex}

\begin{ex}%[2-H2B4-SO-11-2425]%[VN-MT-7, Đào Trung Kiên]%[2H2N1-1]
Cho hình lập phương $ABCD.A'B'C'D'$. Hai vectơ nào dưới đây có giá cùng nằm trong mặt phẳng $(ABCD)$?
 \choice{$\overrightarrow{DD'}$, $\overrightarrow{AC}$}
 {$\overrightarrow{AD'}$, $\overrightarrow{AD}$}
 {$\overrightarrow{AD'}$, $\overrightarrow{AC}$}
 {\True $\overrightarrow{AC}$, $\overrightarrow{AD}$}
 \loigiai{
\begin{center}
\begin{tikzpicture}[font=\footnotesize, line join=round, line cap=round, >=stealth, scale=1]
 \def\a{3.5}
 \path (0:0) coordinate (A)
 ++(0:\a) coordinate (D)
 ++(-130:\a/2) coordinate (C)
 ($(A)+(C)-(D)$) coordinate (B)
 ($(A)+(90:\a)$) coordinate (A')
 ($(B)+(90:\a)$) coordinate (B')
 ($(C)+(90:\a)$) coordinate (C')
 ($(D)+(90:\a)$) coordinate (D');
 \draw[dashed] (B)--(A) (A)--(A');
 \draw[dashed,->] (A)--(C);
 \draw[dashed,->] (A)--(D);
 \draw (C)--(C') (D)--(D') (B)--(B') (B)--(C)--(D) (A')--(B')--(C')--(D')--cycle;
 \foreach \x/\g in {A/180,B/180,C/0,D/0,A'/180,B'/180,C'/0,D'/0}
 \fill (\x) circle (1pt)
 ($(\g:4mm)+(\x)$) node {$\x$}; 
\end{tikzpicture}
\end{center}
Hai vectơ $\overrightarrow{AC}$, $\overrightarrow{AD}$ có giá cùng nằm trong mặt phẳng $(ABCD)$.
 }
\end{ex}

\begin{ex}%[2-H2B4-SO-11-2425]%[VN-MT-7, Đào Trung Kiên]%[2H2N1-1]
Cho hình lập phương $ABCD.A'B'C'D'$ có cạnh là $a$. Hai vectơ nào dưới đây có cùng độ dài?
 \choice{$\overrightarrow{DD'}$, $\overrightarrow{AC}$}
 {$\overrightarrow{AD'}$, $\overrightarrow{AD}$}
 {\True $\overrightarrow{AD'}$, $\overrightarrow{AC}$}
 {$\overrightarrow{AC}$, $\overrightarrow{AD}$}
 \loigiai{
\begin{center}
 \begin{tikzpicture}[font=\footnotesize, line join=round, line cap=round, >=stealth, scale=1]
 \def\a{3.5}
 \path (0:0) coordinate (A)
 ++(0:\a) coordinate (B)
 ++(-130:\a/2) coordinate (C)
 ($(A)+(C)-(B)$) coordinate (D)
 ($(A)+(-90:\a)$) coordinate (A')
 ($(B)+(-90:\a)$) coordinate (B')
 ($(C)+(-90:\a)$) coordinate (C')
 ($(D)+(-90:\a)$) coordinate (D');
 \draw[dashed] (B)--(A) (A)--(A')--(B') (A')--(D');
 \draw[->] (A)--(C);
 \draw[->] (A)--(D);
 \draw[dashed,->] (A)--(D');
 \draw (C)--(C')--(B') (D)--(D')--(C') (A)--(B)--(B') (B)--(C)--(D) ;
 \foreach \x/\g in {A/180,D/180,C/0,B/0,A'/180,D'/180,C'/0,B'/0}
 \fill (\x) circle (1pt)
 ($(\g:4mm)+(\x)$) node {$\x$}; 
 \end{tikzpicture}
\end{center}
Vì $\left|\overrightarrow{AD'}\right|=\left|\overrightarrow{AC}\right|=a\sqrt{2}$ nên hai vectơ $\overrightarrow{AD'}$ và $\overrightarrow{AC}$ có cùng độ dài.
 }
\end{ex}

\begin{ex}%[2-H2B4-SO-11-2425]%[VN-MT-7, Đào Trung Kiên]%[2H2N1-1]
Cho hình lập phương $ABCD.A'B'C'D'$ có cạnh là $a$. Vectơ nào bằng vectơ $\overrightarrow{D'C'}$?
 \choice{$\overrightarrow{DD'}$}
 {$\overrightarrow{AD}$}
 {\True $\overrightarrow{AB}$}
 {$\overrightarrow{CD}$}
 \loigiai{
\begin{center}
 \begin{tikzpicture}[font=\footnotesize, line join=round, line cap=round, >=stealth, scale=1]
 \def\a{3.5}
 \path (0:0) coordinate (A)
 ++(0:\a) coordinate (B)
 ++(-130:\a/2) coordinate (C)
 ($(A)+(C)-(B)$) coordinate (D)
 ($(A)+(-90:\a)$) coordinate (A')
 ($(B)+(-90:\a)$) coordinate (B')
 ($(C)+(-90:\a)$) coordinate (C')
 ($(D)+(-90:\a)$) coordinate (D');
 \draw[dashed] (B)--(A) (A)--(A')--(B') (A')--(D');
 \draw[->] (A)--(B);
 \draw[->] (D')--(C');
 \draw (C)--(C')--(B') (A)--(D)--(D') (A)--(B)--(B') (B)--(C)--(D) ;
 \foreach \x/\g in {A/180,D/180,C/0,B/0,A'/180,D'/180,C'/0,B'/0}
 \fill (\x) circle (1pt)
 ($(\g:4mm)+(\x)$) node {$\x$}; 
 \end{tikzpicture}
\end{center}
Vì hai vectơ $\overrightarrow{AB}$ và $\overrightarrow{D'C'}$ có cùng hướng và cùng độ dài nên $\overrightarrow{AB}=\overrightarrow{D'C'}$.
 }
\end{ex}

\begin{ex}%[2-H2B4-SO-11-2425]%[VN-MT-7, Đào Trung Kiên]%[2H2N2-1]
Trong không gian với hệ tọa độ $Oxyz$, cho hai điểm $A(1; -1; 2)$ và $B(2; 1; -4)$. Vectơ $\overrightarrow{AB}$ có tọa độ là
 \choice{\True $(1;2;-6)$}
 {$(1; 0; -6)$}
 {$(-1; -2; 6)$}
 {$(3; 0; -2)$}
 \loigiai{
Ta có $\overrightarrow{AB}=(2-1; 1-(-1); -4-2)\Rightarrow \overrightarrow{AB}=(1; 2; -6)$.
 }
\end{ex}

\begin{ex}%[2-H2B4-SO-11-2425]%[VN-MT-7, Đào Trung Kiên]%[2H2N2-3]
Trong không gian với hệ tọa độ $Oxyz$, cho biểu diễn của vectơ $\overrightarrow{a}$ qua các vectơ đơn vị là $\overrightarrow{a}=2\overrightarrow{i}+\overrightarrow{k}-3\overrightarrow{j}$. Tọa độ của vectơ $\overrightarrow{a}$ là
 \choice{\True $(2; -3; 1)$}
 {$(1; -3; 2)$}
 {$(2; 1; -3)$}
 {$(1; 2; -3)$}
 \loigiai{ Ta có $\overrightarrow{a}=2\cdot\overrightarrow{i}-3\cdot\overrightarrow{j}+1\cdot \overrightarrow{k}$ nên $\overrightarrow{a}=(2; -3; 1)$.
 }
\end{ex}

\begin{ex}%[2-H2B4-SO-11-2425]%[VN-MT-7, Đào Trung Kiên]%[2H2H2-2]
Trong không gian với hệ tọa độ $Oxyz$, cho hình bình hành $ABCD$ với các đỉnh có tọa độ là $A(3; 1; 2)$, $B(1; 0; 1)$, $C(2; 3; 0)$. Tọa độ đỉnh $D$ là
 \choice{$D(1; 1; 0)$}
 {$D(0; 2; -1)$}
 {\True $D(4; 4; 1)$}
 {$D(1; 3; -1)$}
 \loigiai{
Ta có $ABCD$ là hình bình hành nên $\overrightarrow{AD}=\overrightarrow{BC}\Leftrightarrow\heva{&x_D-3=1\\&y_D-1=3\\&z_D-2=-1}\Leftrightarrow\heva{&x_D=4\\&y_D=4\\&z_D=1}\Rightarrow D(4; 4; 1)$.
 }
\end{ex}

\begin{ex}%[2-H2B4-SO-11-2425]%[VN-MT-7, Đào Trung Kiên]%[2H2H2-2]
Trong không gian với hệ tọa độ $Oxyz$, cho vectơ $\overrightarrow{a}=(-3; 2; 1)$ và điểm $A(4; 6; -3)$. Tọa độ điểm $B$ thỏa mãn $\overrightarrow{AB}=\overrightarrow{a}$ là
 \choice{$(-1; -8; 2)$}
 {$(7; 4; -4)$}
 {\True $(1; 8; -2)$}
 {$(-7; -4; 4)$}
 \loigiai{
Đặt $B(x; y; z)$. Ta có $\overrightarrow{AB}=(x-4; y-6; z+3)$.\\
Khi đó $\overrightarrow{AB}=\overrightarrow{a}\Leftrightarrow \heva{&x-4=-3\\&y-6=2\\&z+3=1}\Leftrightarrow \heva{&x=1\\&y=8\\&z=-2.}$\\
Vậy $B(1; 8; -2)$.
 }
\end{ex}

\begin{ex}%[2-H2B4-SO-11-2425]%[VN-MT-7, Đào Trung Kiên]%[2H2H2-3]
Trong không gian với hệ tọa độ $Oxyz$, cho ba vectơ $\overrightarrow{a}=(1; 2; 3)$, $\overrightarrow{b}=(-2; 0; 1)$,  $\overrightarrow{c}=(-1; 0; 1)$. Tìm tọa độ của vectơ $\overrightarrow{n}=\overrightarrow{a}+\overrightarrow{b}+2\overrightarrow{c}-3\overrightarrow{i}$.
 \choice{$\overrightarrow{n}=(6; 2; 6)$}
 {$\overrightarrow{n}=(6; 2; -6)$}
 {$\overrightarrow{n}=(0; 2; 6)$}
 {\True $\overrightarrow{n}=(-6; 2; 6)$}
 \loigiai{Vì $2\overrightarrow{c}=(-2; 0; 2)$ và $-3\overrightarrow{i}=(-3; 0; 0)$ nên $\overrightarrow{n}=\overrightarrow{a}+\overrightarrow{b}+2\overrightarrow{c}-3\overrightarrow{i}$ có tọa độ $(-6; 2; 6)$.
 }
\end{ex}

\begin{ex}%[2-H2B4-SO-11-2425]%[VN-MT-7, Đào Trung Kiên]%[2H2H2-2]
Trong không gian với hệ tọa độ $Oxyz$, cho ba điểm $A(3; 5; -1)$, $B(7; x; 1)$ và $C(9; 2; y)$. Để $A$, $B$, $C$ thẳng hàng thì $x+y$ bằng
 \choice{\True $5$}
 {$6$}
 {$4$}
 {$7$}
 \loigiai{
Ta có $\overrightarrow{AB}=(4; x-5; 2)$, $\overrightarrow{AC}=(6; -3; y+1)$.\\
Vì $\overrightarrow{AB}\neq \overrightarrow{0}$ nên $A$, $B$, $C$ thẳng hàng khi $\overrightarrow{AB}$, $\overrightarrow{AC}$ cùng phương\\
\[\Leftrightarrow \overrightarrow{AB}=k\overrightarrow{AC}\Leftrightarrow \heva{&4=k\cdot 6\\&x-5=k\cdot(-3)\\&2=k(y+1)}\Leftrightarrow \heva{&k=\dfrac{2}{3}\\&x=3\\&y=2.}\]
Vậy $x+y=5$.
 }
\end{ex}

\begin{ex}%[2-H2B4-SO-11-2425]%[VN-MT-7, Đào Trung Kiên]%[2H2H2-4]
Trong không gian với hệ tọa độ $Oxyz$, điểm $M$ thuộc trục $Ox$ và cách đều hai điểm $A(4; 2; -1)$ và $B(2; 1; 0)$ là
 \choice{$M(-4; 0; 0)$}
 {$M(5; 0; 0)$}
 {\True $M(4; 0; 0)$}
 {$M(-5; 0; 0)$}
 \loigiai{$M\in Ox\Rightarrow M(x; 0; 0)$. Ta có $\overrightarrow{MA}=(4-x; 2; -1)$, $\overrightarrow{MB}=(2-x; 1; 0)$.\\
$M$ cách đều hai điểm $A$, $B$ khi \[MA=MB\Leftrightarrow \sqrt{(4-x)^2+2^2+(-1)^2}=\sqrt{(2-x)^2+1^2+0^2}\Leftrightarrow x=4\]
 }
\end{ex}

\begin{ex}%[2-H2B4-SO-11-2425]%[VN-MT-7, Đào Trung Kiên]%[2H2H2-2]
Trong không gian với hệ tọa độ $Oxyz$, cho ba điểm $A(1; 3; 4)$, $B(1; 0; -2)$ và $C(4; 0; 1)$. Tọa độ trọng tâm $G$ của tam giác $ABC$ là
 \choice{$G(3; 0; 2)$}
 {\True $G(2; 1; 1)$}
 {$G(1; 1; 3)$}
 {$G(3; 0; -1)$}
 \loigiai{
Tọa độ trọng tâm của tam giác $ABC$ là $G(2; 1; 1)$.
 }
\end{ex}
\Closesolutionfile{ans}

\TNTF
\Opensolutionfile{ans}[ans/ans\currfilebase-Phan-II]
\begin{ex}%[2-H2B4-SO-11-2425]%[VN-MT-7, Đào Trung Kiên]%[2H2V1-4]
\immini{Một chất điểm ở vị trí $A$ của hình lập phương $ABCD.A'B'C'D'$. Chất điểm chịu tác động bởi ba lực $\overrightarrow{a}$, $\overrightarrow{b}$, $\overrightarrow{c}$ lần lượt cùng hướng với $\overrightarrow{AD}$, $\overrightarrow{AB}$, $\overrightarrow{AC'}$ như hình vẽ bên. Độ lớn của lực $\overrightarrow{a}$, $\overrightarrow{b}$ và $\overrightarrow{c}$ tương ứng là $10$ N, $10$ N và $10\sqrt{3}$ N.
\choiceTF
{$\overrightarrow{a}+\overrightarrow{b}=\overrightarrow{c}$}
{$\left|\overrightarrow{a}+\overrightarrow{b}\right|=20$ (N)}
{\True $\left|\overrightarrow{a}+\overrightarrow{c}\right|=\left|\overrightarrow{b}+\overrightarrow{c}\right|$}
{\True $\left|\overrightarrow{a}+\overrightarrow{b}+\overrightarrow{c}\right|=30$ (N)}
}{ \begin{tikzpicture}[font=\footnotesize, line join=round, line cap=round, >=stealth, scale=0.7]
 \def\a{3.5}
 \path (0:0) coordinate (A)
 ++(0:\a) coordinate (B)
 ++(-130:\a/2) coordinate (C)
 ($(A)+(C)-(B)$) coordinate (D)
 ($(A)+(-90:\a)$) coordinate (A')
 ($(B)+(-90:\a)$) coordinate (B')
 ($(C)+(-90:\a)$) coordinate (C')
 ($(D)+(-90:\a)$) coordinate (D')
 ($(B)!0.5!(A)$) coordinate (M)
 ($(C')!0.5!(A)$) coordinate (N)
 ($(D)!0.5!(A)$) coordinate (P);
 \draw[dashed] (B)--(A)--(C') (A)--(A')--(B') (A')--(D');
 \draw[->,thick] (A)--(M)node[midway, above] {$\overrightarrow{b}$};
 \draw[->] (D')--(C');
 \draw[->,thick] (A)--(N)node[midway, above right] {$\overrightarrow{c}$};
 \draw[->,thick] (A)--(P)node[midway, above left] {$\overrightarrow{a}$};
 \draw (C)--(C')--(B') (B)--(A)--(D)--(D') (A)--(B)--(B') (B)--(C)--(D) ;
 \foreach \x/\g in {A/90,D/180,C/0,B/0,A'/180,D'/180,C'/0,B'/0}
 \fill (\x) circle (1pt)
 ($(\g:4mm)+(\x)$) node {$\x$}; 
 \end{tikzpicture}
}
\loigiai{
Xét hình lập phương $ABCD.A'B'C'D'$ với cạnh bằng $x>0$, ta có $AC'=\sqrt{AB^2+AD^2+AA'^2}=x\sqrt{3}$.\\
Vì $\triangle ADC'$ vuông tại $D$ nên
$\cos\left(\overrightarrow{a},\overrightarrow{c}\right)=\cos \widehat{DAC'}=\dfrac{AD}{AC'}=\dfrac{1}{\sqrt{3}}$.\\
Tương tự, $\triangle ABC'$ vuông tại $B$ nên $\cos\left(\overrightarrow{b},\overrightarrow{c}\right)=\cos\widehat{BAC'}=\dfrac{AB}{AC'}=\dfrac{1}{\sqrt{3}}$.
 \begin{itemchoice}
 \itemch \textbf{Sai}.\\
Giả sử $\overrightarrow{a}+\overrightarrow{b}=\overrightarrow{d}$. Theo quy tắc hình bình hành thì $\overrightarrow{d}$ cùng hướng với $\overrightarrow{AC}$ nên $\overrightarrow{d}$ không cùng phương với $\overrightarrow{AC'}$. Suy ra $\overrightarrow{a}+\overrightarrow{b}=\overrightarrow{c}$ là sai.
 \itemch \textbf{Sai}.\\
Ta có $\left( \overrightarrow{a} + \overrightarrow{b}\right)^2 = {\overrightarrow{a}}^2 + {\overrightarrow{b}}^2 + 2\overrightarrow{a}\cdot \overrightarrow{b} = 10^2+10^2+0 =200$, suy ra $\left|\overrightarrow{a}+\overrightarrow{b}\right|=10\sqrt{2}$.
 \itemch \textbf{Đúng}.\\
Ta có \begin{itemize}
\item $\left(\overrightarrow{a}+\overrightarrow{c}\right)^2=\left|\overrightarrow{a}\right|^2+2\overrightarrow{a}\cdot\overrightarrow{c}+\left|\overrightarrow{c}\right|^2=10^2+2\cdot 10\cdot 10\sqrt{3}\cdot\dfrac{1}{\sqrt{3}}+\left(10\sqrt{3}\right)^2=600$.\\
Suy ra $\left|\overrightarrow{a}+\overrightarrow{c}\right|=\sqrt{600}$.
\item $\left(\overrightarrow{b}+\overrightarrow{c}\right)^2=\left|\overrightarrow{b}\right|^2+2\overrightarrow{b}\cdot\overrightarrow{c}+\left|\overrightarrow{c}\right|^2=10^2+2\cdot 10\cdot 10\sqrt{3}\cdot\dfrac{1}{\sqrt{3}}+\left(10\sqrt{3}\right)^2=600$.\\
Suy ra $\left|\overrightarrow{a}+\overrightarrow{c}\right|=\sqrt{600}$.
 \end{itemize}
Vậy $\left|\overrightarrow{a}+\overrightarrow{c}\right|=\left|\overrightarrow{b}+\overrightarrow{c}\right|$.
 \itemch \textbf{Đúng}.\\
Giả sử lực tổng hợp là $\overrightarrow{m}$, tức là $\overrightarrow{m}=\overrightarrow{a}+\overrightarrow{b}+\overrightarrow{c}$. Do đó
\begin{eqnarray*}
&& \big|\overrightarrow{m}\big|^2=\left(\overrightarrow{a}+\overrightarrow{b}+\overrightarrow{c}\right)^2\\
&\Leftrightarrow& \left|\overrightarrow{m}\right|^2={\overrightarrow{a}}^2+{\overrightarrow{b}}^2+{\overrightarrow{c}}^2+2\overrightarrow{a}\cdot\overrightarrow{b}+2\overrightarrow{b}\cdot\overrightarrow{c}+2\overrightarrow{c}\cdot\overrightarrow{a}\\
&\Leftrightarrow& \left|\overrightarrow{m}\right|^2=10^2+10^2+(10\sqrt{3})^2+2\cdot 10\cdot10\sqrt{3}\cdot\dfrac{1}{\sqrt{3}}+2\cdot 10\cdot10\sqrt{3}\cdot\dfrac{1}{\sqrt{3}}\\
&\Leftrightarrow& \left|\overrightarrow{m}\right|^2=900.
\end{eqnarray*}
Suy ra cường độ lực tổng hợp $\overrightarrow{a}+\overrightarrow{b}+\overrightarrow{c}$ bằng $30$ N.
\end{itemchoice}
}
\end{ex}

\begin{ex}%[2-H2B4-SO-11-2425]%[VN-MT-7, Đào Trung Kiên]%[2H2H2-4]
Trong không gian với hệ tọa độ $Oxyz$, cho ba điểm $A(2; 3; 1)$, $B(-1; 2; 0)$, $C(1; 1; -2)$.
 \choiceTF
 {\True $\overrightarrow{OA}=2\overrightarrow{i}+3\overrightarrow{j}+\overrightarrow{k}$}
 {$\overrightarrow{AB}=(3; -1; -1)$}
 {\True Gọi $D$ là đỉnh của hình bình hành $ABCD$, khi đó $D(4; 2; -1)$}
 {\True $G$ là trọng tâm của tam giác $ABC$, khi đó $OG=\dfrac{\sqrt{41}}{3}$}
 \loigiai{
 \begin{itemchoice}
 \itemch \textbf{Đúng}.\\
Vì $A(2; 3; 1)$ nên $\overrightarrow{OA}=2\overrightarrow{i}+3\overrightarrow{j}+\overrightarrow{k}$.
 \itemch \textbf{Sai}.\\
$\overrightarrow{AB}=(-3; -1; -1)$.
 \itemch \textbf{Đúng}.\\
Gọi $D(x; y; z)$. Khi đó $\overrightarrow{AB}=(-3; -1; -1)$ và $\overrightarrow{DC}=(1-x; 1-y; -2-z)$.\\
Vì $ABCD$ là hình bình hành nên $\overrightarrow{AB}=\overrightarrow{DC}\Leftrightarrow \heva{&-3=1-x\\&-1=1-y\\&-1=-2-z}\Leftrightarrow \heva{&x=4\\&y=2\\&z=-1.}$\\
Vậy $D(4; 2; -1)$.
 \itemch \textbf{Đúng}.\\
Gọi $G(x; y; z)$ là trọng tâm của tam giác $ABC$. Khi đó $\heva{&x=\dfrac{2-1+1}{3}=\dfrac{2}{3}\\&y=\dfrac{3+2+1}{3}=2\\&z=\dfrac{1+0-2}{3}=-\dfrac{1}{3}.}$\\
Vậy $G\left(\dfrac{2}{3}; 2; -\dfrac{1}{3}\right)$ nên $OG=\sqrt{\left(\dfrac{2}{3}\right)^2+2^2+\left(-\dfrac{1}{3}\right)^2}=\dfrac{\sqrt{41}}{3}$.

\end{itemchoice}
 }
\end{ex}

\begin{ex}%[2-H2B4-SO-11-2425]%[VN-MT-7, Đào Trung Kiên]%[2H2H2-4]
Trong không gian với hệ tọa độ $Oxyz$.
 \choiceTF
 {\True Cho hai vectơ $\overrightarrow{u}=m\overrightarrow{i}+2\overrightarrow{j}-3\overrightarrow{k}$, $\overrightarrow{v}=m\overrightarrow{j}+2\overrightarrow{i}+4\overrightarrow{k}$. Biết rằng $\overrightarrow{u}\cdot \overrightarrow{v}=8$, khi đó $m=5$}
 {Góc giữa hai vectơ $\overrightarrow{u}=(1; -2;1)$ và $\overrightarrow{v}=(-2; 1; 1)$ bằng $60^\circ$}
 {\True Cho lăng trụ đứng $ABC.A'B'C'$ có $A(0; 0; 0)$, $B(2; 0; 0)$, $C(0; 2; 0)$ và $A'(0; 0; 2)$. Góc giữa $BC'$ và $A'C$ bằng $90^\circ$}
 {Gọi $\varphi$ là góc giữa hai vectơ $\overrightarrow{a}$ và $\overrightarrow{b}$ (với $\overrightarrow{a}$ và $\overrightarrow{b}$ khác $\overrightarrow{0}$), khi đó $\cos\varphi=\dfrac{|\overrightarrow{a}|\cdot|\overrightarrow{b}|}{\overrightarrow{a}\cdot\overrightarrow{b}}$}
\loigiai{
 \begin{itemchoice}
 \itemch \textbf{Đúng}.\\
Từ giả thiết ta có $\overrightarrow{u}=(m; 2; -3)$, $\overrightarrow{v}=(2; m; 4)$.\\
Do đó $\overrightarrow{u}\cdot\overrightarrow{v}=8\Leftrightarrow 2m+2m-3\cdot 4=8\Leftrightarrow m=5$.
 \itemch \textbf{Sai}.\\
Ta có $\cos(\overrightarrow{u},\overrightarrow{v})=\dfrac{\overrightarrow{u}\cdot\overrightarrow{v}}{|\overrightarrow{u}|\cdot|\overrightarrow{v}|}=\dfrac{-3}{\sqrt{6}\cdot\sqrt{6}}=-\dfrac{1}{2}\Rightarrow \left(\overrightarrow{u},\overrightarrow{v}\right)=120^\circ$.
 \itemch \textbf{Đúng}.\\
Gọi $C'(x; y; z)$, vì $ABC.A'B'C'$ là hình lăng trụ đứng nên $\overrightarrow{AA'}=\overrightarrow{CC'}\Leftrightarrow \heva{&x-0=0\\&y-2=0\\&z-0=2}$.\\
Từ đó ta có $B(2; 0; 0)$, $C'(0; 2; 2)$ nên $\overrightarrow{BC'}=(-2; 2; 2)$.\\
Vì $A'(0; 0; 2)$ và $C(0; 2; 0)$ nên $\overrightarrow{A'C}=(0; 2; -2)$.\\
Từ đó suy ra $\overrightarrow{BC'}\cdot\overrightarrow{A'C}=0$ nên góc giữa $BC'$ và $A'C$ bằng $90^\circ$.
\itemch \textbf{Sai}.\\
Công thức tính côsin của góc giữa hai vectơ $\overrightarrow{a}$ và $\overrightarrow{b}$ (với $\overrightarrow{a}$ và $\overrightarrow{b}$ khác $\overrightarrow{0}$) là  $\cos\left(\overrightarrow{a},\overrightarrow{b}\right)=\dfrac{\overrightarrow{a}\cdot\overrightarrow{b}}{|\overrightarrow{a}|\cdot|\overrightarrow{b}|}$.
\end{itemchoice}
 }
\end{ex}

\begin{ex}%[2-H2B4-SO-11-2425]%[VN-MT-7, Đào Trung Kiên]%[2H2V2-6]
Hình minh họa sơ đồ ngôi nhà Trong không gian với hệ tọa độ $Oxyz$, trong đó nền nhà, bốn bức tường và hai mái nhà đều là hình chữ nhật.
\begin{center}
\begin{tikzpicture}[font=\footnotesize, line join=round, line cap=round, >=stealth, scale=1.2]
 \def\a{3}
 \def\b{5}
 \def\h{3}
 \path (0:0) coordinate (C)
 ++(0:\a) coordinate (B)
 ++(-160:\b) coordinate (O)
 ($(O)+(B)-(C)$) coordinate (A)
 ($(O)+(90:\h)$) coordinate (E)
 ($(B)+(90:\h)$) coordinate (G)
 ($(C)+(90:\h)$) coordinate (H)
 ($(A)+(90:\h)$) coordinate (F)
 ($(A)+(0:1)$) coordinate (x)
 ($(H)+(35:2)$) coordinate (Q)
 ($(E)+(35:2)$) coordinate (P)
 ($(E)+(90:1)$) coordinate (z)
 ($(O)!1.3!(C)$) coordinate (y);
 \draw[dashed] (G)--(H)--(C)--(B) (C)--(O);
 \draw (G)--(Q)--(H)--(E)--(F)--(G)--(B)--(A)--(O)--(E) (F)--(A) (F)--(P)--(E) (P)--(Q);
 \draw [->] (A)--(x);
 \draw [->] (E)--(z);
 \draw [->,dashed] (C)--(y);
 \draw (Q)node[above]{$Q(2; 5; 4)$} (G)node[right]{$G(4; 5; 3)$} (B)node[right]{$B(4; 5; 0)$} (P)node[right]{$P(2; 0; 4)$} (O)node[below]{$O(0; 0; 0)$} (E)node[left]{$E(0; 0; 3)$} (x)node[below]{$x$} (y)node[above]{$y$} (z)node[left]{$z$};
 \foreach \x/\g in {A/-90,C/180,F/0,H/90}
 \fill (\x) circle (1pt)
 ($(\g:4mm)+(\x)$) node {$\x$}; 
 \fill (E) circle (1pt) (Q) circle (1pt) (O) circle (1pt) (G) circle (1pt) (P) circle (1pt) (B) circle (1pt);
\end{tikzpicture}
\end{center}
 \choiceTF
 {\True Tọa độ điểm $F(4; 0; 3)$}
 {Tọa độ vectơ $\overrightarrow{AH}=(4; 5; 3)$}
 {$\overrightarrow{AH}\cdot\overrightarrow{AF}=3$}
 {\True Góc đốc của mái nhà, tức là số đo của góc nhị diện có cạnh là đường thẳng $FG$, hai mặt lần lượt là $(FGQP)$ và $(FGHE)$ bằng $26{,}6^\circ$ (làm tròn đến hàng phần mười của đơn vị độ)}
 \loigiai{
 \begin{itemchoice}
 \itemch \textbf{Đúng}.\\
Vì nền nhà là hình chữ nhật nên $OACB$ là hình chữ nhật, suy ra $x_A=x_B=4, y_C=y_B=5$.\\
Do điểm $A$ nằm trên trục $O x$ nên tọa độ điểm $A(4; 0; 0)$; điểm $C$ nằm trên trục $Oy$ nên tọa độ điểm $C(0; 5; 0)$.\\
Tường nhà là hình chữ nhật nên $OCHE$ là hình chữ nhật, suy ra $y_H=y_C=5$, $z_H=z_E=3$.\\
Do $H$ nằm trên mặt phẳng $(Oyz)$ nên tọa độ điểm $H(0; 5; 3)$.\\
Tứ giác $OAFE$ là hình chữ nhật nên $x_F=x_A=4, z_F=z_E=3$.\\
Do $F$ nằm trên mặt phẳng $(Oxz)$ nên tọa độ điểm $F(4; 0; 3)$.
 \itemch \textbf{Sai}.\\
Ta có toạ độ vectơ $\overrightarrow{AH}=(-4; 5; 3)$.
 \itemch \textbf{Sai}.\\
Ta có $\overrightarrow{AF}=(0; 0; 3)$. Suy ra $\overrightarrow{AH}\cdot\overrightarrow{AF}=0+0+9=9$.
 \itemch \textbf{Đúng}.\\
Để tính góc đốc của mái nhà, ta tính số đo của góc nhị diện có cạnh là đường thẳng $FG$, hai mặt lần lượt là $(FGQP)$ và $(FGHE)$.\\
Do mặt phẳng $(O z x)$ vuông góc với hai mặt phẳng $(FGQP)$ và $(F G H E)$ nên $\widehat{PFE}$ là góc phẳng nhị diện cần tìm.\\
Ta có $\overrightarrow{FP}=(-2; 0; 1), \overrightarrow{FE}=(-4; 0; 0)$ suy ra 
\[\cos\widehat{PFE}=\cos \left(\overrightarrow{FP}, \overrightarrow{FE}\right)=\dfrac{\overrightarrow{FP} \cdot \overrightarrow{FE}}{\left|\overrightarrow{FP}\right|\cdot\left|\overrightarrow{FE}\right|}= \dfrac{(-2)(-4)+0\cdot 0+1\cdot 0}{\sqrt{(-2)^2+0^2+1^2} \cdot \sqrt{(-4)^2+0^2+0^2}}=\dfrac{2 \sqrt{5}}{5}.\]
Do đó, $\widehat{PFE} \approx 26{,}6^{\circ}$.\\
Vậy góc đốc mái nhà khoảng $26{,}6^{\circ}$.
\end{itemchoice}
 }
\end{ex}
\Closesolutionfile{ans}

\TNSA
\Opensolutionfile{ans}[ans/ans\currfilebase-Phan-III]
\begin{ex}%[2-H2B4-SO-11-2425]%[VN-MT-7, Đào Trung Kiên]%[2H2H1-3]
Cho hai vectơ $\overrightarrow{a}$, $\overrightarrow{b}$ thỏa mãn $\left|\overrightarrow{a}\right|=3$, $\left|\overrightarrow{b}\right|=4$, $\left|\overrightarrow{a}+\overrightarrow{b}\right|=6$. Tính $\left|\overrightarrow{a}-\overrightarrow{b}\right|$ (làm tròn kết quả đến hàng phần trăm).
\shortans{3{,}74}
\loigiai{
Ta có $\left|\overrightarrow{a}+\overrightarrow{b}\right|^2=\left(\overrightarrow{a}+\overrightarrow{b}\right)^2=\left|\overrightarrow{a}\right|^2+2\overrightarrow{a}\overrightarrow{b}+\left|\overrightarrow{b}\right|^2\Rightarrow 2\overrightarrow{a}\overrightarrow{b}=\left|\overrightarrow{a}+\overrightarrow{b}\right|^2-\left|\overrightarrow{a}\right|^2-\left|\overrightarrow{b}\right|^2=11$.\\
$\left|\overrightarrow{a}-\overrightarrow{b}\right|^2=\left(\overrightarrow{a}-\overrightarrow{b}\right)^2=\left|\overrightarrow{a}\right|^2-2\overrightarrow{a}\overrightarrow{b}+\left|\overrightarrow{b}\right|^2=9-11+16=14\Rightarrow \left|\overrightarrow{a}-\overrightarrow{b}\right|=\sqrt{14}\approx 3{,}74$.
}
\end{ex}

\begin{ex}%[2-H2B4-SO-11-2425]%[VN-MT-7, Đào Trung Kiên]%[2H2V1-4]
Một chiếc đèn trang trí hình tròn được treo song song với mặt phẳng trần nhà nằm ngang bởi ba sợi dây không giãn $OA$, $OB$, $OC$ đôi một vuông góc (như hình vẽ dưới đây). Biết lực căng của sợi dây tương ứng trên mỗi dây $OA$, $OB$, $OC$ lần lượt là $\overrightarrow{F_1}$, $\overrightarrow{F_2}$, $\overrightarrow{F_3}$ thỏa mãn $\left|\overrightarrow{F_1}\right|=\left|\overrightarrow{F_2}\right|=\left|\overrightarrow{F_3}\right|=16$ (N). Tính trọng lượng (đơn vị: N) của chiếc đèn đó (làm tròn kết quả đến hàng phần mười).
\begin{center}
\begin{tikzpicture}[line join=round, line cap=round,>=stealth,xscale=1,yscale=0.4]
 \path (0:0) coordinate (O')
 ++(0:2) coordinate (C)
 ($(O')+(150:2)$) coordinate (B)
 ($(O')+(220:2)$) coordinate (A)
 ($(O')+(90:8)$) coordinate (O)
 ($(O)!0.5!(A)$) coordinate (A1)
 ($(O)!0.5!(B)$) coordinate (B1)
 ($(O)!0.5!(C)$) coordinate (C1);
 \draw[fill=blue,opacity=0.15] (O') circle (2 cm);
 \draw (O') circle (2 cm);
 \draw (O)--(A) (O)--(B) (O)--(C) (2,0)--(2,-1) (-2,0)--(-2,-1);
 \draw (2,-1) arc (0:-180:2);
 \draw[color=red,dashed] (C)--(O');
 \draw[dashed] (O')--(0,-3);
 \draw[fill=blue,opacity=0.4] (-2,8) rectangle (2,8.6);
 \draw [->] (0,-3)--(0,-7)node[midway, right]{$\overrightarrow{P}$};
 \draw [->] (O)--(A1)node[above right]{$\overrightarrow{F_1}$};
 \draw [->] (O)--(B1)node[above left]{$\overrightarrow{F_2}$};
 \draw [->] (O)--(C1)node[above right]{$\overrightarrow{F_3}$};
 \draw (0,7.5) node[below] {$O$};
 \foreach \x/\g in {A/60,B/100,C/0}
 \fill (\x) circle (1pt)
 ($(\g:5mm)+(\x)$) node {$\x$};
\end{tikzpicture}
\end{center}
 \shortans{27{,}7}
 \loigiai{
\begin{center}
\begin{tikzpicture}[line join=round, line cap=round,>=stealth,scale=1]
 \def\a{3.5}
 \path (0:0) coordinate (O)
 ++(0:\a) coordinate (B)
 ++(-160:\a*1.3) coordinate (A)
 ($(B)+(A)-(O)$) coordinate (E)
 ($(O)+(90:\a)$) coordinate (C)
 ($(E)+(90:\a)$) coordinate (F)
 ($(A)+(90:\a)$) coordinate (A')
 ($(B)+(90:\a)$) coordinate (B');
 \draw[dashed] (O)--(A) (O)--(B) (O)--(C) (O)--(F);
 \draw[thick] (C)--(F) (C)--(A') (C)--(B') (F)--(E) (A)--(E)--(B)--(B')--(F)--(A')--cycle;
 \foreach \x/\g in {O/170,A/-90,E/-90,B/-60,C/100,F/0}
 \fill[black] (\x) circle (1pt)
 ($(\g:4mm)+(\x)$) node {$\x$}; 
\end{tikzpicture}
\end{center}
Gọi $P$ là trọng lượng của đèn, ta có $P=\left|\overrightarrow{F_1}+\overrightarrow{F_2}+\overrightarrow{F_3}\right|=\left|\overrightarrow{OA}+\overrightarrow{OB}+\overrightarrow{OC}\right|$.\\
Vẽ hình vuông $OAEB$, ta có $\overrightarrow{OA}+\overrightarrow{OB}=\overrightarrow{OE}$ (quy tắc hình bình hành).\\
Vẽ hình chữ nhật $OCFE$, ta có $\overrightarrow{OC}+\overrightarrow{OE}=\overrightarrow{OF}$ (quy tắc hình bình hành).\\
Suy ra $P=\left|\overrightarrow{OF}\right|=OF$.\\
Xét hình vuông $OAEB$, cạnh bằng $16$ và có đường chéo $OE=16\sqrt{2}$.\\
Xét tam giác vuông $OEF$, vuông tại $E$, có $OF=\sqrt{OE^2+EF^2}=\sqrt{\left(16\sqrt{2}\right)^2+16^2}=16\sqrt{3}\approx 27{,}7$.\\
Vậy $P\approx 27{,}7$ N.
 }
\end{ex}

\begin{ex}%[2-H2B4-SO-11-2425]%[VN-MT-7, Đào Trung Kiên]%[2H2H2-4]
Trong không gian với hệ tọa độ $Oxyz$, cho hai điểm $B(2; 1; 0)$, $C(1; 4; 5)$. Điểm $M(x; y; z)$ thuộc trục hoành sao cho $MB=MC$. Khi đó giá trị $2x+y+z$ bằng bao nhiêu?
 \shortans{-37}
 \loigiai{
Do điểm $M \in Ox$ nên $M(x; 0; 0)$, ta có 
\begin{eqnarray*}
MB=MC&\Leftrightarrow& MB^2=MC^2\Leftrightarrow (2-x)^2+1^2+0^2=(1-x)^2+4^2+5^2\\
&\Leftrightarrow&x^2-4x+5=x^2-2x+42\Leftrightarrow x=-\dfrac{37}{2}. 
\end{eqnarray*}
Vậy $M\left(-\dfrac{37}{2};0;0\right)\Rightarrow 2x+y+z=-37$.
 }
\end{ex}

\begin{ex}%[2-H2B4-SO-11-2425]%[VN-MT-7, Đào Trung Kiên]%[2H2H1-4]
Trong không gian tọa độ $Oxyz$ cho $\overrightarrow{a}$ và $\overrightarrow{b}$ tạo với nhau một góc $120^{\circ}$. Biết rằng $|\overrightarrow{a}|=4$; $|\overrightarrow{b}|=3$, tính giá trị của biểu thức $A=|\overrightarrow{a}-\overrightarrow{b}|+|\overrightarrow{a}+\overrightarrow{b}|$ ( làm tròn kết quả đến hàng phần trăm).

\shortans{9{,}69}
 \loigiai{
Ta có $|\overrightarrow{a}-\overrightarrow{b}|^2=\left(\overrightarrow{a}-\overrightarrow{b}\right)^2=|\overrightarrow{a}|^2-2 \overrightarrow{a} \cdot \overrightarrow{b}+|\overrightarrow{b}|^2=16-2|\overrightarrow{a}| \cdot|\overrightarrow{b}|\cdot\cos 120^{\circ}+9=37$.\\
Tương tự $|\overrightarrow{a}+\overrightarrow{b}|^2=\left(\overrightarrow{a}+\overrightarrow{b}\right)^2=|\overrightarrow{a}|^2+2 \overrightarrow{a} \cdot \overrightarrow{b}+|\overrightarrow{b}|^2=16+2|\overrightarrow{a}| \cdot|\overrightarrow{b}|\cdot\cos 120^{\circ}+9=13$.\\
Do đó $A=|\overrightarrow{a}-\overrightarrow{b}|+|\overrightarrow{a}+\overrightarrow{b}|=\sqrt{37}+\sqrt{13} \approx 9{,}69$.
 }
\end{ex}

\begin{ex}%[2-H2B4-SO-11-2425]%[VN-MT-7, Đào Trung Kiên]%[2H2V2-6]
Người ta cần lắp một camera phía trên sân bóng để phát sóng truyền hình một trận bóng đá, camera có thể di động để luôn thu được hình ảnh rõ nét về diễn biến trên sân. Các kĩ sư dự định trồng bốn chiếc cột cao 30 m và sử dụng hệ thống cáp gắn vào bốn đầu cột để giữ camera ở vị trí mong muốn.\\
Mô hình thiết kế được xây dựng như sau\\
Trong hệ trục toạ độ $Oxyz$ (đơn vị độ dài trên mỗi trục là $1$ m), các đỉnh của bốn chiếc cột lần lượt là các điểm $M(90; 0; 30)$, $N(90; 120; 30)$, $P(0; 120; 30)$, $Q(0; 0; 30)$.\\
Giả sử $K_0$ là vị trí ban đầu của camera có cao độ bằng $25$ và $K_0M=K_0N=K_0P=K_0Q$. Để theo dõi quả bóng đến vị trí $A$, camera được hạ thấp theo phương thẳng đứng xuống điểm $K_1$ cao độ bằng $19$.\\
Tọa độ của vectơ $\overrightarrow{K_0K_1}=(a; b; c)$ với $a$, $b$, $c$ là các số thực. Tính $P=a+b-c$.
\begin{center}
 \begin{tikzpicture}
 \def\a{5.5}
 \def\b{2}
 \def\h{3.5}
 \path (0:0) coordinate (O)
 ++(0:\a) coordinate (P')
 ($(O)+(220:\b)$) coordinate (M')
 ($(O)+(0:\a+1)$) coordinate (y)
 ($(P')+(M')-(O)$) coordinate (F)
 ($(O)+(90:\h)$) coordinate (Q)
 ($(M')+(90:\h)$) coordinate (M)
 ($(F)+(90:\h)$) coordinate (N)
 ($(P')+(90:\h)$) coordinate (P)
 ($(O)!1.3!(M')$) coordinate (x)
 ($(O)!1.3!(Q)$) coordinate (z)
 ($(O)!0.5!(Q)$) coordinate (Q_1)
 ($(O)!0.7!(Q)$) coordinate (Q_0)
 ($(F)!0.5!(N)$) coordinate (N_1)
 ($(F)!0.7!(N)$) coordinate (N_0)
 ($(O)!0.1!(F)$) coordinate (A)
 ($(O)!0.9!(F)$) coordinate (C)
 ($(M')!0.1!(P')$) coordinate (B)
 ($(M')!0.9!(P')$) coordinate (D);
 \coordinate (H) at (intersection of Q--N and M--P);
 \coordinate (H') at (intersection of F--O and M'--P');
 \coordinate (K_0) at (intersection of Q_0--N_0 and H--H');
 \coordinate (K_1) at (intersection of Q_1--N_1 and H--H');
 \draw [->] (O)--(y) node[below]{$y$};
 \draw [->] (O)--(x) node[below]{$x$};
 \draw [->] (O)--(z) node[left]{$z$};
 \draw [fill=green] (A)--(B)--(C)--(D)--(A);
 \draw[dashed] (M')--(P') (F)--(O) (K_1)--(Q) (K_1)--(M) (K_1)--(P) (K_1)--(N);
 \draw (Q)--(K_0)--(N) (M')--(M)--(K_0)--(P) (N)--(F) (P)--(P');
 \foreach \x/\g in {O/45,F/-90,P/40,Q/60,M/160,K_0/100,K_1/-90,N/0}
 \fill (\x) circle (1pt)
 ($(\g:4mm)+(\x)$) node {$\x$}; 
 \fill[black](4.3,-0.4) node [below right]{$A$} circle (1.5pt);
 \end{tikzpicture}
\end{center} 
\shortans{6}
 \loigiai{
\begin{center}
 \begin{tikzpicture}
 \def\a{5.5}
 \def\b{2}
 \def\h{3.5}
 \path (0:0) coordinate (O)
 ++(0:\a) coordinate (P')
 ($(O)+(220:\b)$) coordinate (M')
 ($(O)+(0:\a+1)$) coordinate (y)
 ($(P')+(M')-(O)$) coordinate (F)
 ($(O)+(90:\h)$) coordinate (Q)
 ($(M')+(90:\h)$) coordinate (M)
 ($(F)+(90:\h)$) coordinate (N)
 ($(P')+(90:\h)$) coordinate (P)
 ($(O)!1.3!(M')$) coordinate (x)
 ($(O)!1.3!(Q)$) coordinate (z)
 ($(O)!0.5!(Q)$) coordinate (Q_1)
 ($(O)!0.7!(Q)$) coordinate (Q_0)
 ($(F)!0.5!(N)$) coordinate (N_1)
 ($(F)!0.7!(N)$) coordinate (N_0)
 ($(O)!0.1!(F)$) coordinate (A)
 ($(O)!0.9!(F)$) coordinate (C)
 ($(M')!0.1!(P')$) coordinate (B)
 ($(M')!0.9!(P')$) coordinate (D);
 \coordinate (H) at (intersection of Q--N and M--P);
 \coordinate (H') at (intersection of F--O and M'--P');
 \coordinate (K_0) at (intersection of Q_0--N_0 and H--H');
 \coordinate (K_1) at (intersection of Q_1--N_1 and H--H');
 \draw [->] (O)--(y) node[below]{$y$};
 \draw [->] (O)--(x) node[below]{$x$};
 \draw [->] (O)--(z) node[left]{$z$};
 \draw [fill=green] (A)--(B)--(C)--(D)--(A);
 \draw[dashed] (M')--(P') (F)--(O) (K_1)--(Q) (K_1)--(M) (K_1)--(P) (K_1)--(N);
 \draw (Q)--(K_0)--(N) (M')--(M)--(K_0)--(P) (N)--(F) (P)--(P');
 \foreach \x/\g in {O/45,F/-90,P/40,Q/60,M/160,K_0/100,K_1/-90,N/0}
 \fill (\x) circle (1pt)
 ($(\g:4mm)+(\x)$) node {$\x$}; 
 \fill[black](4.3,-0.4) node [below right]{$A$} circle (1.5pt);
\end{tikzpicture}

\end{center} 
Gọi $K_0(x; y; 25)$ và $K_1(x; y; 19)$ suy ra $\overrightarrow{K_0K_1}=(0; 0; -6)$.\\
Vậy $a =0$, $b=0$, $c=-6$ nên $P=a+b-c=6$.}
\end{ex}

\begin{ex}%[2-H2B4-SO-11-2425]%[VN-MT-7, Đào Trung Kiên]%[2H2H1-3]
\immini{Cho tứ diện $OABC$ có các cạnh $OA$, $OB$, $OC$ đôi một vuông góc và $OA=OB=OC=1$. Gọi $M$ là trung điểm của cạnh $AB$. Côsin của góc giữa hai vectơ $\overrightarrow{OM}$ và $\overrightarrow{AC}$ bằng $-\dfrac{a}{b}$ với $\dfrac{a}{b}$ là phân số tối giản. Tính $Q = a\cdot b$.}{
\begin{tikzpicture}[line join=round, line cap=round,>=stealth,scale=0.8]
 \def\a{4} %Khai báo cạnh
 \def\h{3}
 \path (0:0) coordinate (O)
 ++(0:\a) coordinate (B)
 ($(O)+(-50:2.4)$) coordinate (A)
 ($(O)+(90:\h)$) coordinate (C)
 ($(A)!0.5!(B)$) coordinate (M);
 \draw (C)--(O)--(A)--(C)--(B)--(A);
 \draw[dashed,->] (O)--(M);
 \draw[dashed] (O)--(B) ;
 \draw[->] (A)--(C);
 \foreach \x /\goc in {A/-90,B/0,C/170,M/-40,O/180}
 \fill (\x) circle (1pt)
 ($(\x)+(\goc:3mm)$) node {$\x$};
% \draw pic[draw,angle radius=2mm]{right angle=B--A--S};%Theo chiều dương
\end{tikzpicture}
}
 \shortans{2}
 \loigiai{
Đặt $\overrightarrow{OA}=\overrightarrow{a}$, $\overrightarrow{OB}=\overrightarrow{b}$, $\overrightarrow{OC}=\overrightarrow{c}$.\\
Khi đó, $\left|\overrightarrow{a}\right|=\left|\overrightarrow{b}\right|=\left|\overrightarrow{c}\right|=1$ và $\overrightarrow{a}\cdot\overrightarrow{b}=\overrightarrow{a}\cdot\overrightarrow{c}=\overrightarrow{b}\cdot\overrightarrow{c}=0$.\\
Ta có $\cos\left(\overrightarrow{OM},\overrightarrow{AC}\right)=\dfrac{\overrightarrow{OM}\cdot\overrightarrow{AC}}{\left|\overrightarrow{OM}\right|\cdot\left|\overrightarrow{AC}\right|}$.\\
Mặt khác, do $\overrightarrow{OM}=\dfrac{1}{2}\left(\overrightarrow{OA}+\overrightarrow{OB}\right)=\dfrac{1}{2}\left(\overrightarrow{a}+\overrightarrow{b}\right)$ và $\overrightarrow{AC}=\overrightarrow{OC}-\overrightarrow{OA}=\overrightarrow{c}-\overrightarrow{a}$ nên 
\[\overrightarrow{OM}\cdot\overrightarrow{AC}=\dfrac{1}{2}\left(\overrightarrow{a}+\overrightarrow{b}\right)\cdot\left(\overrightarrow{c}-\overrightarrow{a}\right)=\dfrac{1}{2}\left(\overrightarrow{a}\cdot\overrightarrow{c}-{\overrightarrow{a}}^2+\overrightarrow{b}\cdot\overrightarrow{c}-\overrightarrow{b}\cdot\overrightarrow{a}\right)=-\dfrac{1}{2}.\]
Ta có $AC=\sqrt{OA^2+OC^2}=\sqrt{2}$, $OM=\dfrac{1}{2}AB=\dfrac{1}{2}\sqrt{OA^2+OC^2}=\dfrac{\sqrt{2}}{2}$.\\
Từ đó $\cos\left(\overrightarrow{OM},\overrightarrow{AC}\right)=-\dfrac{1}{2}$ nên $a=1$ và $b=2$.\\
Vậy $Q=a\cdot b=2$.
 }
\end{ex}
\Closesolutionfile{ans}
 
% \begin{indapan}
% 	{ans/ans\currfilebase}
% \end{indapan}


% \begin{name}
 {Biên soạn: Vũ Hồng Toàn \\ Phản biện: Lê Văn Hiếu}
{Đề ôn tập chương II}
\end{name}

\TN
\Opensolutionfile{ans}[ans/ans\currfilebase-Phan-I]

\begin{ex}%[2-H2B4-SO-12-2425 (Nguồn Đề 3 - Bài 4)]%[VN-MT-7, Vũ Hồng Toàn]%[2H2H1-2]
Cho hình chóp $ S.ABCD$ có đáy là hình bình hành tâm $O$. Đẳng thức nào sau đây \textbf{sai}?
\choice
{$\overrightarrow{BC}+\overrightarrow{DA}=\overrightarrow{BA}+\overrightarrow{DC}$}
{$\overrightarrow{SA}+\overrightarrow{SC}=\overrightarrow{SB}+\overrightarrow{SD}$}
{$\overrightarrow{SA}+\overrightarrow{SB}+\overrightarrow{SC}+\overrightarrow{SD}=4\overrightarrow{SO}$}
{\True $\overrightarrow{AB}+\overrightarrow{CD}=\overrightarrow{AC}+\overrightarrow{BD}$}
\loigiai{
\begin{center}
\begin{tikzpicture}[scale=1, font=\footnotesize, line join=round, line cap=round, >=stealth]
\def\a{4} \def\b{2.5} \def\h{4.5} \def\g{30}
\path
(0,0) coordinate (A)
(0:\a) coordinate (D)++
(\g-180:\b) coordinate (C)++(180:\a) coordinate (B)
($(A)!.5!(C)$) coordinate (O)++ (90:\h) coordinate (S)
;
\draw[dashed](B)--(A)--(D)--cycle (A)--(C);
\draw[->,dashed] (S)--(O);
\draw[->,dashed] (S)--(A);
\draw[->] (S)--(B);
\draw[->] (S)--(C);
\draw[->] (S)--(D);
\draw (B)--(C)--(D);
\foreach \x /\gN in {A/160,B/180,C/-20,S/90,O/-90,D/0}
\fill(\x) circle (1pt)($(\x)+(\gN:3mm)$) node {$\x$};
\end{tikzpicture}
\end{center}
Với $O$ là trung điểm của $AC$, ta có $\overrightarrow{SA}+\overrightarrow{SC}=2\overrightarrow{SO}$.\\
Với $O$ là trung điểm của $BD$, ta có $\overrightarrow{SB}+\overrightarrow{SD}=2\overrightarrow{SO}$.\\
Từ đó suy ra
\begin{itemize}
\item $\overrightarrow{SA}+\overrightarrow{SC}=\overrightarrow{SB}+\overrightarrow{SD}$.
\item $\overrightarrow{SA}+\overrightarrow{SB}+\overrightarrow{SC}+\overrightarrow{SD}=4\overrightarrow{SO}$.
\item
$\overrightarrow{BC}+\overrightarrow{DA}=\overrightarrow{BA}+\overrightarrow{AC}+\overrightarrow{DC}+\overrightarrow{CA}=\overrightarrow{BA}+\overrightarrow{DC}+\left(\overrightarrow{AC}+\overrightarrow{CA}\right)=\overrightarrow{BA}+\overrightarrow{DC}$.
\item $\overrightarrow{AB}+\overrightarrow{CD}=\overrightarrow{AC}+\overrightarrow{CB}+\overrightarrow{CB}+\overrightarrow{BD}=\overrightarrow{AC}+\overrightarrow{BD}+2\overrightarrow{CB}$.
\end{itemize}
}
\end{ex}


\begin{ex}%[2-H2B4-SO-12-2425 (Nguồn Đề 3 - Bài 4)]%[VN-MT-7, Vũ Hồng Toàn]%[2H2H1-3]
Cho hình lập phương $ABCD.A_1B_1C_1D_1$ có cạnh $a$. Gọi $M$ là trung điểm $AD$. Giá trị $\overrightarrow{B_1M}\cdot\overrightarrow{BD_1}$ bằng
\choice
{$a^2$}
{\True $\dfrac{1}{2} a^2$}
{$\dfrac{3}{2} a^2$}
{$\dfrac{3}{4} a^2$}
\loigiai{
\begin{center}
\begin{tikzpicture}[scale=1, font=\footnotesize, line join=round, line cap=round, >=stealth]
\def\a{4} \def\b{2.5} \def\h{3.5} \def\g{30}
\path
(0,0) coordinate (A)
(0:\a) coordinate (D)++
(\g-180:\b) coordinate (C)++(180:\a) coordinate (B)
(90:\h) coordinate (A_1) ++(0:\a) coordinate (D_1)++
(\g-180:\b) coordinate (C_1)++(180:\a) coordinate (B_1)
($(A)!.5!(D)$) coordinate (M)
;
\draw[dashed](B)--(A)--(D)--cycle (A_1)--(A)--(C);
\draw[dashed,->] (B_1)--(M);
\draw[dashed,->] (B)--(D_1);
\draw (A_1)--(B_1)--(C_1)--(D_1) --cycle (B_1)--(B)--(C)--(C_1)--(A_1) (C)--(D)--(D_1)--(B_1) ;
\foreach \x /\gN in {A/40,B/180,C/-20,M/60,A_1/90,D_1/0,C_1/90,B_1/180,D/0}
\fill(\x) circle (1pt)($(\x)+(\gN:3mm)$) node {$\x$};
\end{tikzpicture}
\end{center}
Áp dụng quy tắc cộng ta có
\begin{itemize}
 \item $\overrightarrow{B_1M}=\overrightarrow{B_1B}+\overrightarrow{BA }+\overrightarrow{AM}$;
 \item $\overrightarrow{BD_1}=\overrightarrow{BA}+\overrightarrow{AD}+\overrightarrow{DD_1}$.
\end{itemize}
Khi đó
\allowdisplaybreaks
\begin{eqnarray*}
\overrightarrow{B_1M}\cdot \overrightarrow{BD_1}&=&\left(\overrightarrow{B_1 B}+\overrightarrow{BA}+\overrightarrow{AM}\right)\cdot\left(\overrightarrow{BA}+\overrightarrow{AD}+\overrightarrow{DD_1}\right)\\
&=&\overrightarrow{B_1B}\cdot\overrightarrow{BA}+\overrightarrow{B_1B}\cdot\overrightarrow{AD}+\overrightarrow{B_1B}\cdot\overrightarrow{DD_1}\\&&+{\overrightarrow{BA}}^2+\overrightarrow{BA}\cdot \overrightarrow{AD}+\overrightarrow{BA}\cdot \overrightarrow{DD_1}\\&&+\overrightarrow{AM}\cdot\overrightarrow{BA}+\overrightarrow{AM}\cdot\overrightarrow{AD}+\overrightarrow{AM}\cdot\overrightarrow{DD_1}\\
&=&\overrightarrow{B_1B}\cdot\overrightarrow{DD_1}+{\overrightarrow{BA}}^2+\overrightarrow{AM}\cdot\overrightarrow{AD}\\
&=&-{\overrightarrow{B_1B}}^2+{\overrightarrow{BA}}^2+\dfrac{1}{2}{\overrightarrow{AD}}^2\\
&=&-a^2+a^2+\dfrac{a^2}{2} \\
&=&\dfrac{a^2}{2}.
\end{eqnarray*}
}
\end{ex}


\begin{ex}%[2-H2B4-SO-12-2425 (Nguồn Đề 3 - Bài 4)]%[VN-MT-7, Vũ Hồng Toàn]%[2H2N2-3]
Trong KG $Oxyz$, cho điểm $A(3;-1;5)$. Tọa độ của vectơ $\overrightarrow{OA}$ là
\choice
{$(3;1;5)$}
{\True $(3;-1;5)$}
{$(-3;-1;5)$}
{$(-3;1;-5)$}
\loigiai{Ta có
$A(3;-1;5)\Rightarrow \overrightarrow{OA}=(3;-1;5)$.
}
\end{ex}


\begin{ex}%[2-H2B4-SO-12-2425 (Nguồn Đề 3 - Bài 4)]%[VN-MT-7, Vũ Hồng Toàn]%[2H2N2-3]
Trong KG $Oxyz$, cho hai điểm $M\left(\dfrac{1}{2};1;-3\right)$ và $N\left(\dfrac{1}{2};-2;4\right)$. Tọa độ của vectơ $\overrightarrow{MN}$ là
\choice
{$(1;-1;1)$}
{\True $(0;-3;7)$}
{$(0;3;-7)$}
{$\left(\dfrac{1}{2};-\dfrac{1}{2};\dfrac{1}{2} \right)$}
\loigiai{
Ta có $\overrightarrow{MN}=\left(\dfrac{1}{2}-\dfrac{1}{2};(-2)-1;4-(-3)\right)=(0;-3;7)$.
}
\end{ex}


\begin{ex}%[2-H2B4-SO-12-2425 (Nguồn Đề 3 - Bài 4)]%[VN-MT-7, Vũ Hồng Toàn]%[2H2H2-3]
Trong KG $Oxyz$, cho các vectơ $\overrightarrow{a}=(1;2;3)$, $\overrightarrow{b}=(2;1;-3)$ và $\overrightarrow{c}=(-1;1;5)$. Vectơ $\overrightarrow{x}=\overrightarrow{a}-4\overrightarrow{b}+2\overrightarrow{c}$ có tọa độ là
\choice
{$\overrightarrow{x}=(9;0;25)$}
{$\overrightarrow{x}=(-9;0;-25)$}
{$\overrightarrow{x}=(9;0;5)$}
{\True $\overrightarrow{x}=(-9;0;25)$}
\loigiai{
Ta có
\begin{itemize}
\item $-4\overrightarrow{b}=(-8;-4;12)$.
\item $2\overrightarrow{c}=(-2;2;10)$.
\end{itemize}
Vậy $ \overrightarrow{x}=\overrightarrow{a}-4\overrightarrow{b}+2\overrightarrow{c}=(-9;0;25) $.
}
\end{ex}


\begin{ex}%[2-H2B4-SO-12-2425 (Nguồn Đề 3 - Bài 4)]%[VN-MT-7, Vũ Hồng Toàn]%[2H2N2-3]
Trong không gian với hệ toạ độ $Oxyz$, cho hai điểm $A(0;-1;2)$ và $B(1;-2;3)$. Toạ độ của vectơ $3\overrightarrow{AB}$ là
\choice
{$(3;3;3)$}
{\True $(3;-3;3)$}
{$(-3;3;3)$}
{$(-3;-3;3)$}
\loigiai{
Ta có $\overrightarrow{AB}=(1;-1;1)$.\\
Vậy $3\overrightarrow{AB}=(3;-3;3)$.
}
\end{ex}


\begin{ex}%[2-H2B4-SO-12-2425 (Nguồn Đề 3 - Bài 4)]%[VN-MT-7, Vũ Hồng Toàn]%[2H2H2-3]
Trong KG $Oxyz$, cho hai vectơ $\overrightarrow{u}=(3;-1;1)$ và $\overrightarrow{v}=(1;2;-2)$. Độ dài của vectơ $\overrightarrow{u}+\overrightarrow{v}$ là
\choice
{$\sqrt{10}$}
{$\sqrt{11}+3$}
{\True $3\sqrt{2}$}
{$5$}
\loigiai{
Có $\overrightarrow{u}+\overrightarrow{v}=(4;1;-1)$.\\
Độ dài của vectơ $\overrightarrow{u}+\overrightarrow{v}$ là $\left|\overrightarrow{u}+\overrightarrow{v}\right|=\sqrt{4^2+1^2+(-1)^2}=3\sqrt{2}$.
}
\end{ex}


\begin{ex}%[2-H2B4-SO-12-2425 (Nguồn Đề 3 - Bài 4)]%[VN-MT-7, Vũ Hồng Toàn]%[2H2H2-4]
Trong KG $Oxyz$, cho ba điểm $A(-2; 1; 0)$, $B(0;-2; 5)$, $C(6;-2;1)$. Tích vô hướng của hai vectơ $\overrightarrow{AB}$ và $\overrightarrow{BC}$ là
\choice
{$\sqrt{38}\cdot\sqrt{52}$}
{$-\sqrt{38}\cdot\sqrt{52}$}
{$8$}
{\True $-8$}
\loigiai{
Ta có $\overrightarrow{AB}=(2;-3; 5)$ và $\overrightarrow{BC}=(6; 0;-4)$.\\
Tích vô hướng của hai vectơ $\overrightarrow{AB}$ và $\overrightarrow{BC}$ là $\overrightarrow{AB}\cdot\overrightarrow{BC}=2\cdot 6+(-3)\cdot 0+5\cdot (-4)=-8$.
}
\end{ex}


\begin{ex}%[2-H2B4-SO-12-2425 (Nguồn Đề 3 - Bài 4)]%[VN-MT-7, Vũ Hồng Toàn]%[2H2H2-2]
Trong không gian với hệ trục tọa độ $Oxyz$, cho hai điểm $A(2;1;1)$ và $B(-1;2;1)$. Tìm tọa độ $A'$ đối xứng với $A$ qua $B$.
\choice
{$A'(3;4;-3)$}
{\True $A'(-4;3;1)$}
{$A'(4;-3;3)$}
{$A'(4;33)$}
\loigiai{
Vì $A'$ đối xứng với $A$ qua $B$ nên $B$ là trung điểm của $AA'$.\\
Do đó $\heva{&x_B=\dfrac{x_A+x_{A'}}{2}\\&y_B=\dfrac{y_A+y_{A'}}{2}\\&z_B=\dfrac{z_A+z_{A'}}{2}}\Rightarrow\heva{&x_{A'}=2x_B-x_A=2\cdot(-1)-2=-4\\&y_{A'}=2y_B-y_A=2\cdot 2-1=3\\&z_{A'}=2z_B-z_A=2\cdot 1-1=1.}$\\
Vậy $A'(-4;3;1)$.
}
\end{ex}


\begin{ex}%[2-H2B4-SO-12-2425 (Nguồn Đề 3 - Bài 4)]%[VN-MT-7, Vũ Hồng Toàn]%[2H2H2-2]
Trong KG $Oxyz$, cho hai điểm $M(0;0;2)$ và $N(4;-2;6)$. Tìm tọa độ điểm $P$ sao cho $N$ là trung điểm của $MP$.
\choice
{$P(2;-1;4)$}
{$(4;-2;4)$}
{$(2;-1;2)$}
{\True $P(8;-4; 10)$}
\loigiai{
Vì $N$ là trung điểm của $MP$ nên\\
\centerline{$\heva{&x_N=\dfrac{x_M+x_P}{2}\\&y_N=\dfrac{y_M+y_P}{2}\\&z_N=\dfrac{z_M+z_P}{2}}\Rightarrow\heva{&4=\dfrac{0+x_P}{2}\\&-2=\dfrac{0+y_P}{2}\\&6=\dfrac{2+z_P}{2}}\Rightarrow\heva{&x_P=8\\&y_P=-4\\&z_P=10.}$}
Vậy $P(8;-4; 10)$.
}
\end{ex}


\begin{ex}%[2-H2B4-SO-12-2425 (Nguồn Đề 3 - Bài 4)]%[VN-MT-7, Vũ Hồng Toàn]%[2H2H2-2]
Trong KG $Oxyz$, cho tam giác $MNP$ có $M(-1;3;0)$, $N(2;2;1)$, $P(-1;1;2)$. Trọng tâm $G$ của tam giác $MNP$ có tọa độ là
\choice
{\True $(0;2;1)$}
{$(0;6;3)$}
{$(2;0;1)$}
{$(0;-2;1)$}
\loigiai{
Theo công thức tính tọa độ trong tâm của tam giác ta có\\
\centerline{$\heva{&x_G=\dfrac{-1+2-1}{3}\\&y_G=\dfrac{3+2+1}{3}\\&z_G=\dfrac{0+1+2}{3}}\Rightarrow\heva{&x_G=0\\&y_G=2\\&z_G=1.}$}
Vậy $G(0;2;1)$.
}
\end{ex}


\begin{ex}%[2-H2B4-SO-12-2425 (Nguồn Đề 3 - Bài 4)]%[VN-MT-7, Vũ Hồng Toàn]%[2H2H2-6]
Trong không gian chọn hệ trục tọa độ cho trước, đơn vị đo là kilômét, rađa phát hiện một máy bay chiến đấu của Nga di chuyển với vận tốc và hướng không đổi từ điểm $M(600;400;20)$ đến điểm $N(800;500;30)$ trong $30$ phút. Nếu máy bay tiếp tục giữ nguyên vận tốc và hướng bay thì tọa độ của máy bay sau $15$ phút tiếp theo bằng bao nhiêu?
\choice
{$(700;500;30)$}
{$(900;650;55)$}
{\True $(900;550;35)$}
{$(800;540;30)$}
\loigiai{
Gọi $Q(x;y;z)$ là tọa độ của máy bay sau $15$ phút tiếp theo.\\
Ta có $\overrightarrow{MN}=(200;100;10)$ và $\overrightarrow{NQ}=(x-800;y-500;z-30)$.\\
Vì máy bay giữ nguyên hướng bay nên $\overrightarrow{MN}$ và $\overrightarrow{NQ}$ cùng hướng.
Do máy bay tiếp tục giữ nguyên vận tốc và thời gian bay từ $M\to N$ gấp $2$ lần thời gian bay từ $N\to Q$ nên\\
\centerline{$MN=2NQ
\Rightarrow\overrightarrow{MN}=2\overrightarrow{NQ}
\Rightarrow\heva{&200=2(x-800)\\&100=2(y-500)\\&10=2(z-30)}
\Rightarrow\heva{&x=900\\&y=550\\&z=35.}
\Rightarrow Q(900;550;35)$.}
Tọa độ của máy bay sau $15$ phút tiếp theo là $(900;550;35)$ .
}
\end{ex}

\Closesolutionfile{ans}

\TNTF
\Opensolutionfile{ans}[ans/ans\currfilebase-Phan-II]

\begin{ex}%[2-H2B4-SO-12-2425 (Nguồn Đề 3 - Bài 4)]%[VN-MT-7, Vũ Hồng Toàn]%[2H2H2-4]
Cho hình lăng trụ tam giác đều $ABC.A'B'C'$ có $AB=a$ và $AA'=a\sqrt{2}$. Gọi $M$ là trung điểm $BC$.
\choiceTF
{\True $\overrightarrow{AC}=\overrightarrow{AB}+\overrightarrow{BC}$}
{\True $\overrightarrow{A'M}=\overrightarrow{A'A}+\overrightarrow{A'B'}-\overrightarrow{CM}$}
{$\overrightarrow{A'M}\cdot\overrightarrow{AC}=\dfrac{a^2\sqrt{3}}{4}$}
{\True Góc giữa vectơ $\overrightarrow{AB'}$ và $\overrightarrow{BC'}$ bằng $60^\circ$}
\loigiai{
\begin{center}
\begin{tikzpicture}[scale=1, font=\footnotesize, line join=round, line cap=round, >=stealth]
\def\a{4} \def\b{2.2} \def\h{3.5} \def\g{30}
\path
(0,0) coordinate (A)
(0:\a) coordinate (C)++
(\g-180:\b) coordinate (B)
(90:\h) coordinate (A') ++(0:\a) coordinate (C')++
(\g-180:\b) coordinate (B')
($(C)!.5!(B)$) coordinate (M);
\draw[dashed](C)--(A)--(M)--(A');
\draw (A')--(A)--(B)--(B')--cycle (B)--(C)--(C')--(B') (A')--(C');
\foreach \x /\gN in {A/160,B/-90,C/-20,M/-30,A'/90,C'/90,B'/90}
\fill(\x) circle (1pt)($(\x)+(\gN:3mm)$) node {$\x$};
\end{tikzpicture}
\end{center}
Do $ABC.A'B'C'$ là lăng trụ tam giác đều cạnh $a$ nên $\triangle ABC$ đều cạnh $a$ và $AA'\perp (ABC)$.\\
Ta có $M$ là trung điểm của $BC$ nên $AM=\dfrac{a\sqrt{3}}{2}$.
\begin{itemchoice}
\itemch \textbf{Đúng}.\\ Áp dụng quy tắc cộng ta có $\overrightarrow{AC}=\overrightarrow{AB}+\overrightarrow{BC}$.
\itemch \textbf{Đúng}.\\Ta có $\overrightarrow{A'A}+\overrightarrow{A'B'}-\overrightarrow{CM}=\overrightarrow{A'A}+\overrightarrow{AB}+\overrightarrow{BM}=\overrightarrow{A'B}+\overrightarrow{BM}=\overrightarrow{A'M}$.
\itemch \textbf{Sai}.\\Ta có
\allowdisplaybreaks
\begin{eqnarray*}
\overrightarrow{A'M}\cdot \overrightarrow{AC}&=&\left(\overrightarrow{A'A}+\overrightarrow{AM}\right)\cdot\overrightarrow{AC}\\&=&\overrightarrow{A'A}\cdot\overrightarrow{AC}+\overrightarrow{AM}\cdot\overrightarrow{AC}\\&=&\overrightarrow{AM}\cdot\overrightarrow{AC}\\&=&\dfrac{a\sqrt{3}}{2}\cdot a\cdot \cos 30^\circ\\&=&\dfrac{3a^2}{4}.
\end{eqnarray*}
\itemch \textbf{Đúng}.\\Ta có
\begin{itemize}
 \item $\triangle ABB'$ vuông tại $B$ nên $AB'=\sqrt{AB^2+BB'^2}=\sqrt{a^2+2a^2}=a\sqrt{3}$.
 \item $\triangle BCC'$ vuông tại $C$ nên $BC'=\sqrt{BC^2+CC'^2}=\sqrt{a^2+2a^2}=a\sqrt{3}$.
 \item $\overrightarrow{AB}\cdot\overrightarrow{BC}=\left|\overrightarrow{AB}\right|\cdot \left|\overrightarrow{BC}\right|\cdot \cos 120^\circ=a\cdot a\cdot\dfrac{-1}{2}=-\dfrac{a^2}{2}$.
 \item $\overrightarrow{BB'}\cdot\overrightarrow{CC'}=\overrightarrow{BB'}^2=2a^2$.
\end{itemize}
Khi đó
\allowdisplaybreaks
\begin{eqnarray*}
\overrightarrow{AB'}\cdot\overrightarrow{BC'}&=&\left(\overrightarrow{AB}+\overrightarrow{BB'}\right)\cdot\left(\overrightarrow{BC}+\overrightarrow{CC'}\right)\\ &=&\overrightarrow{AB}\cdot\overrightarrow{BC}+\overrightarrow{AB}\cdot\overrightarrow{CC'}+\overrightarrow{BB'}\cdot\overrightarrow{BC}+\overrightarrow{BB'}\cdot\overrightarrow{CC'}\\
&=&-\dfrac{a^2}{2}+0+0+2a^2\\&=&\dfrac{3a^2}{2}.
\end{eqnarray*}
Suy ra $\cos \left(\overrightarrow{AB'},\overrightarrow{BC'}\right)=\dfrac{\overrightarrow{AB'}\cdot\overrightarrow{BC'}}{\left|\overrightarrow{AB'}\right|\cdot\left|\overrightarrow{BC'}\right|}=\dfrac{\dfrac{3a^2}{2}}{a\sqrt{3}\cdot a\sqrt{3}}=\dfrac{1}{2} \Rightarrow \left(\overrightarrow{AB'}, \overrightarrow{BC'}\right)=60^\circ$.
\end{itemchoice}
}
\end{ex}


\begin{ex}%[2-H2B4-SO-12-2425 (Nguồn Đề 3 - Bài 4)]%[VN-MT-7, Vũ Hồng Toàn]%[2H2H2-3]
Trong KG $Oxyz$, cho tam giác $ABC$ có các đỉnh $A(1;-2;0)$, $B(2;1;-2)$, $C(0;3;4)$. 
\choiceTF
{\True Tọa độ của vectơ $\overrightarrow{AB}$ là $(1;3;-2)$}
{\True Tọa độ trọng tâm của tam giác $ABC$ là $G\left(1;\dfrac{2}{3};\dfrac{2}{3} \right)$}
{Tọa độ hình chiếu của điểm $B$ trên mặt phẳng $(Oxy)$ là $H(0;0;-2)$}
{$\overrightarrow{x}=2\overrightarrow{AB}-3\overrightarrow{BC}$. Tọa độ của vectơ $\overrightarrow{x}=(-4;12;14)$}
\loigiai{
\begin{itemchoice}
\itemch \textbf{Đúng}.\\ Ta có $\overrightarrow{AB}=(1;3;-2)$.
\itemch \textbf{Đúng}.\\ Ta có $\heva{&x_G=\dfrac{x_A+x_B+x_C}{3}=1\\&y_G=\dfrac{y_A+y_B+y_C}{3}=\dfrac{2}{3}\\&z_G=\dfrac{z_A+z_B+z_C}{3}=\dfrac{2}{3}}\Rightarrow G\left(1;\dfrac{2}{3};\dfrac{2}{3} \right)$.
\itemch \textbf{Sai}.\\Tọa độ hình chiếu của điểm $B(2;1;-2)$ trên mặt phẳng $(Oxy)$ là $H(2;1;0)$.
\itemch \textbf{Sai}.\\Ta có
\begin{itemize}
\item $\overrightarrow{AB}=(1;3;-2)\Rightarrow 2\overrightarrow{AB}=(2;6;-4)$.
\item $\overrightarrow{BC}=(-2;2;6)\Rightarrow-3\overrightarrow{BC}=(6;-6;-18)$.
\end{itemize}
Vậy $\overrightarrow{x}=2\overrightarrow{AB}-3\overrightarrow{BC}=(8;0;-22)$.
\end{itemchoice}
}
\end{ex}


\begin{ex}%[2-H2B4-SO-12-2425 (Nguồn Đề 3 - Bài 4)]%[VN-MT-7, Vũ Hồng Toàn]%[2H2V2-5]
Trong không gian với hệ trục tọa độ $Oxyz$, cho bốn điểm $A(0;-1; 1)$, $B(-2; 1;-1)$, $C(-1; 3; 2)$, $D(-1; 0; 0)$. 
\choiceTF
{\True Ba điểm $A$, $B$, $C$ không thẳng hàng}
{\True Ba điểm $A$, $B$, $D$ thẳng hàng}
{Côsin của góc giữa $\overrightarrow{AB}$ và $\overrightarrow{CB}$ bằng $-\dfrac{\sqrt{42}}{21}$}
{Bốn điểm $A$, $B$, $C$, $D$ không đồng phẳng}
\loigiai{
\begin{itemchoice}
\itemch \textbf{Đúng}.\\Ta có $\overrightarrow{AB}=(-2; 2;-2)$, $ \overrightarrow{BC}=(1; 2; 3)$.\\
Giả sử tồn tại số $k\ne 0$ sao cho $\overrightarrow{AB}=k\overrightarrow{BC}\Rightarrow \heva{&-2=k\\&2=2k\\&-2=3k.}
$\\Hệ vô nghiệm suy ra không tồn tại $k$.
Vậy ba điểm $A$, $B$, $C$ không thẳng hàng.
\itemch \textbf{Đúng}.\\ Ta có $\overrightarrow{AB}=(-2; 2;-2)$, $ \overrightarrow{BD}=(1;-1; 1)$.\\
Vì $\overrightarrow{AB}=-2\overrightarrow{BD}$.
Suy ra điểm $A$, $B$, $D$ thẳng hàng
\itemch \textbf{Sai}.\\Ta có $\overrightarrow{AB}=(-2; 2;-2)$, $\overrightarrow{CB}=(-1;-2;-3)$.\\
Khi đó $\cos\left (\overrightarrow{AB},\overrightarrow{CB}\right)=\dfrac{\overrightarrow{AB}\cdot\overrightarrow{CB}}{\left|\overrightarrow{AB}\right|\cdot\left|\overrightarrow{CB}\right|} =\dfrac{(-2)\cdot(-1)+2\cdot(-2)+(-2)\cdot(-3)}{\sqrt{12}\cdot\sqrt{14}}=\dfrac{\sqrt{42}}{21}$.
\itemch \textbf{Sai}.\\Ta có $\overrightarrow{AB}=(-2; 2;-2)$, $ \overrightarrow{BD}=(1;-1; 1)$.\\
Vì $\overrightarrow{AB}=-2\overrightarrow{BD}$.
Suy ra điểm $A$, $B$, $D$ thẳng hàng.\\
Khi đó luôn tồn tại một mặt phẳng qua $C$ và chứa đường thẳng đi qua ba điểm $A$, $B$, $D$.\\
Vậy bốn điểm $A$, $B$, $C$, $D$ đồng phẳng.
\end{itemchoice}
}
\end{ex}


\begin{ex}%[2-H2B4-SO-12-2425 (Nguồn Đề 3 - Bài 4)]%[VN-MT-7, Vũ Hồng Toàn]%[2H2V2-5]
Trong không gian với hệ trục tọa độ $Oxyz$, cho ba điểm $A(-1; 2; 1)$; $B(2;-2;4)$; $C(0;-4;1)$. 
\choiceTF
{\True Ba điểm $A$, $B$, $C$ không thẳng hàng}
{\True Biết điểm $D(5;-6;7)$. Khi đó ba điểm $A$, $B$, $D$ thẳng hàng}
{$\cos\left(\overrightarrow{AB},\overrightarrow{AC}\right)=\dfrac{37}{\sqrt{1258}}$}
{Cho $\overrightarrow{u}=(x-1;2y+1;3z-5)$ thoả mãn $\overrightarrow{u}\perp \overrightarrow{AB}$ và $\overrightarrow{u}\perp \overrightarrow{AC}$. Khi đó $x^2+y^2+z^2=2024$}
\loigiai{
\begin{itemchoice}
\itemch \textbf{Đúng}.\\Ta có $\overrightarrow{AB}=(3;-4;3)$, $\overrightarrow{AC}=(1;-6;0)$.\\ Giả sử tồn tại số $k\ne 0$ sao cho $\overrightarrow{AB}=k\overrightarrow{AC}\Rightarrow \heva{&3=k\\&-4=-6k\\&3=0k.}
$\\
Hệ vô nghiệm suy ra không tồn tại $k$. Vậy ba điểm $A$, $B$, $C$ không thẳng hàng.
\itemch \textbf{Đúng}.\\Ta có $\overrightarrow{AB}=(3;-4;3)$, $\overrightarrow{AD}=(6;-8;6)\Rightarrow \overrightarrow{AD}=2\overrightarrow{AB}$.\\ 
Vậy ba điểm $A$, $B$, $D$ thẳng hàng.
\itemch \textbf{Sai}.\\
Ta có $\cos\left(\overrightarrow{AB},\overrightarrow{AC}\right)=\dfrac{\overrightarrow{AB}\cdot\overrightarrow{AC}}{\left|\overrightarrow{AB}\right|\cdot\left|\overrightarrow{AC}\right|}=\dfrac{3\cdot 1+(-4)\cdot (-6)+3\cdot 0}{\sqrt{9+16+9}\cdot\sqrt{1+36+0}}=\dfrac{27}{\sqrt{1\,258}}$.
\itemch \textbf{Sai}.\\Ta có $\overrightarrow{u}\perp \overrightarrow{AB}$ và $\overrightarrow{u}\perp \overrightarrow{AC}$ suy ra $\overrightarrow{u}$ cùng phương với $\left[\overrightarrow{AB},\overrightarrow{AC}\right]=(18;3;-14)$.\\
Xét trường hợp $\overrightarrow{u}=(18;3;-14)$ ta có\\
\centerline{$\heva{&x-1=18\\&2y+1=3\\&3z-5=-14}\Rightarrow\heva{&x=19\\&y=1\\&z=-3.}$}\\
Vậy $x^2+y^2+z^2=19^2+1+9=371$.
\end{itemchoice}
}
\end{ex}

\Closesolutionfile{ans}

\TNSA
\Opensolutionfile{ans}[ans/ans\currfilebase-Phan-III]


\begin{ex}%[2-H2B4-SO-12-2425 (Nguồn Đề 3 - Bài 4)]%[VN-MT-7, Vũ Hồng Toàn]%[2H2V2-4]
Cho hình lập phương $ABCD.A'B'C'D'$ có cạnh là $a$. Gọi $G$ là trọng tâm tam giác $B'C'D'$, $I$ là trung điểm của $AB'$. Tính $\cos\left(\overrightarrow{A'D}, \overrightarrow{IG}\right)$ (làm tròn kết quả đến hàng phần trăm).
\shortans{0{,}14}
\loigiai{
\begin{center}
\begin{tikzpicture}[scale=1, font=\footnotesize, line join=round, line cap=round, >=stealth]
\def\a{4} \def\b{2.5} \def\h{3.5} \def\g{30}
\path
(0,0) coordinate (A)
(0:\a) coordinate (D)++
(\g-180:\b) coordinate (C)++(180:\a) coordinate (B)
(90:\h) coordinate (A') ++(0:\a) coordinate (D')++
(\g-180:\b) coordinate (C')++(180:\a) coordinate (B')
($(A')!.5!(C')$) coordinate (M)
($(C')!2/3!(M)$) coordinate (G)
($(A)!.5!(B')$) coordinate (I)
;
\draw[dashed](B)--(A)--(D) (A')--(A)--(B') (I)--(G) (A')--(D);
\draw (A')--(B')--(C')--(D') --cycle (B')--(B)--(C)--(C')--(A') (C)--(D)--(D')--(B') ;
\foreach \x /\gN in {A/60,B/180,C/-20,G/0,A'/90,D'/0,C'/-40,B'/180,D/0,I/180}
\fill(\x) circle (1pt)($(\x)+(\gN:3mm)$) node {$\x$};
\end{tikzpicture}
\end{center}
Ta có cạnh hình lập phương là $a\Rightarrow A'D=a\sqrt{2}$ và $\overrightarrow{A'D}=\overrightarrow{AD}-\overrightarrow{AA'}$.
\allowdisplaybreaks
\begin{eqnarray*}
\overrightarrow{IG}&=&\overrightarrow{IB'}+\overrightarrow{B'G}\\&=&\dfrac{1}{2} \left(\overrightarrow{AA'}+\overrightarrow{AB}\right)+\dfrac{1}{3} \left(\overrightarrow{B'D'}+\overrightarrow{B'C'}\right)\\&=&\dfrac{1}{2} \left(\overrightarrow{AA'}+\overrightarrow{AB}\right)+\dfrac{1}{3} \left(\overrightarrow{BD}+\overrightarrow{AD}\right) \\ &=&\dfrac{1}{2} \left(\overrightarrow{AA'}+\overrightarrow{AB}\right)+\dfrac{1}{3} \left(2\overrightarrow{AD}-\overrightarrow{AB}\right)\\&=&\dfrac{1}{2} \overrightarrow{AA'}+\dfrac{1}{6} \overrightarrow{AB}+\dfrac{2}{3} \overrightarrow{AD} \\
\Rightarrow{\overrightarrow{IG}}^2&=&\left(\dfrac{1}{2} \overrightarrow{AA'}+\dfrac{1}{6} \overrightarrow{AB}+\dfrac{2}{3} \overrightarrow{AD}\right)^2=\dfrac{13a^2}{18}\\
\Rightarrow IG&=&\dfrac{a\sqrt{26}}{6}.
\end{eqnarray*}
Mà\\
\centerline{$\overrightarrow{A'D}\cdot\overrightarrow{IG}=\left(\dfrac{1}{2} \overrightarrow{AA'}+\dfrac{1}{6} \overrightarrow{AB}+\dfrac{2}{3} \overrightarrow{AD}\right)\cdot\left(\overrightarrow{AD}-\overrightarrow{AA'}\right)=\dfrac{a^2}{6} $.}
Vậy\\
\centerline{$ \cos\left(\overrightarrow{A'D}, \overrightarrow{IG}\right)=\dfrac{\overrightarrow{A'D}\cdot\overrightarrow{IG}}{A'D\cdot IG}=\dfrac{\dfrac{a^2}{6}}{\dfrac{a\sqrt{26}}{6}\cdot a\sqrt{2}}=\dfrac{\sqrt{13}}{26}\approx 0{,}14 $.}
}
\end{ex}


\begin{ex}%[2-H2B4-SO-12-2425 (Nguồn Đề 3 - Bài 4)]%[VN-MT-7, Vũ Hồng Toàn]%[2H2H2-2]
Trong KG $Oxyz$, cho hình hộp $ABCD.A'B'C'D'$. Biết $A(2; 4; 0)$, $B(4; 0; 0)$, $C(-1; 4;-7)$ và $D'(6; 8; 10)$. Tọa độ đỉnh $B'$ của hình hộp có dạng $B'(a;b;c)$. Tính $a+b+c$.
\shortans{30}
\loigiai{
\begin{center}
\begin{tikzpicture}[scale=1, font=\footnotesize, line join=round, line cap=round, >=stealth]
\def\a{3} \def\b{2} \def\h{3} \def\g{30}
\path
(0,0) coordinate (A)
(0:\a) coordinate (D)++
(\g-180:\b) coordinate (C)++(180:\a) coordinate (B)
(80:\h) coordinate (A') ++(0:\a) coordinate (D')++
(\g-180:\b) coordinate (C')++(180:\a) coordinate (B')
;
\draw[dashed](B)--(A)--(D) (A')--(A);
\draw (A')--(B')--(C')--(D') --cycle (B')--(B)--(C)--(C') (C)--(D)--(D') ;
\foreach \x /\gN in {A/60,B/180,C/-20,A'/180,D'/0,C'/-40,B'/180,D/0}
\fill(\x) circle (1pt)($(\x)+(\gN:3mm)$) node {$\x$};
\end{tikzpicture}
\end{center}
Ta có $\overrightarrow{BC}=(-5; 4;-7)$. Gọi $D(x; y; z)\Rightarrow \overrightarrow{AD}=(x-2; y-4; z)$.\\
Vì $ABCD$ là hình bình hành nên\\
\centerline{$\overrightarrow{AD}=\overrightarrow{BC}\Rightarrow\heva{&x-2=-5\\&y-4=4\\&z=-7}\Rightarrow\heva{&x=-3\\&y=8\\&z=-7}\Rightarrow D(-3;8;-7)$.}
Ta có $\overrightarrow{DD'}=(9; 0; 17)$ và $\overrightarrow{BB'}=(a-4; b; c)$.\\
Vì $BB'D'D$ là hình bình hành nên\\
\centerline{$\overrightarrow{BB'}=\overrightarrow{DD'}\Rightarrow \heva{&a-4=9\\&b=0\\&c=17}\Rightarrow\heva{&a=13\\&b=0\\&c=17}\Rightarrow B'(13; 0; 17)$.}
Vậy $a+b+c=13+0+17=30$.
}
\end{ex}


\begin{ex}%[2-H2B4-SO-12-2425 (Nguồn Đề 3 - Bài 4)]%[VN-MT-7, Vũ Hồng Toàn]%[2H2H2-4]
Trong KG $Oxyz$, cho ba điểm $A(1;6;2)$, $B(5; 1; 3)$ và $C(4; 0; 6)$. Biết $\overrightarrow{u}=(14;a;b)$ vuông góc với với cả hai vectơ $\overrightarrow{AB}$ và $\overrightarrow{AC}$. Tính $a-b$.
\shortans{4}
\loigiai{
Ta có $\overrightarrow{AB}=(4;-5; 1)$ và $\overrightarrow{AC}=(3;-6; 4)$.\\
Vì $\overrightarrow{u}$ vuông góc với cả hai vectơ $\overrightarrow{AB}$ và $\overrightarrow{AC}$ nên $\overrightarrow{u}$ cùng phương với vectơ \linebreak $\left[\overrightarrow{AB}, \overrightarrow{AC}\right]=(-14;-13;-9)$.
Do đó tồn tại số thực $k$ để $\overrightarrow{u}=k\left[\overrightarrow{AB}, \overrightarrow{AC}\right]$.\\
Khi đó ta có $k=-1$, $\overrightarrow{u}=(14;13;9)$.\\
Suy ra $a=13$ và $b=9$. Vậy $a-b=13-9=4$.
}
\end{ex}


\begin{ex}%[2-H2B4-SO-12-2425 (Nguồn Đề 3 - Bài 4)]%[VN-MT-7, Vũ Hồng Toàn]%[2H2V2-6]
Hình minh họa sơ đồ ngôi nhà Trong KG $Oxyz$, trong đó nền nhà, bốn bức tường và hai mái nhà đều là hình chữ nhật. Biết tọa độ của vectơ $\overrightarrow{AH}=(a;b;c)$. Tìm $a+b+c$.
\begin{center}
 \begin{tikzpicture}[font=\footnotesize, line join=round, line cap=round, >=stealth, scale=1.2]
 \def\a{3}
 \def\b{5}
 \def\h{3}
 \path (0:0) coordinate (C)
 ++(0:\a) coordinate (B)
 ++(-160:\b) coordinate (O)
 ($(O)+(B)-(C)$) coordinate (A)
 ($(O)+(90:\h)$) coordinate (E)
 ($(B)+(90:\h)$) coordinate (G)
 ($(C)+(90:\h)$) coordinate (H)
 ($(A)+(90:\h)$) coordinate (F)
 ($(A)+(0:1)$) coordinate (x)
 ($(H)+(35:2)$) coordinate (Q)
 ($(E)+(35:2)$) coordinate (P)
 ($(E)+(90:1)$) coordinate (z)
 ($(O)!1.3!(C)$) coordinate (y);
 \draw[dashed] (G)--(H)--(C)--(B) (C)--(O);
 \draw[] (G)--(Q)--(H)--(E)--(F)--(G)--(B)--(A)--(O)--(E) (F)--(A) (F)--(P)--(E) (P)--(Q);
 \draw [->] (A)--(x);
 \draw [->] (E)--(z);
 \draw [->,dashed] (C)--(y);
 \draw [] (Q)node[above]{$Q(2; 5; 4)$} (G)node[right]{$G(4; 5; 3)$} (B)node[right]{$B(4; 5; 0)$} (P)node[right]{$P(2; 0; 4)$} (O)node[below]{$O(0; 0; 0)$} (E)node[left]{$E(0; 0; 3)$} (x)node[below]{$x$} (y)node[above]{$y$} (z)node[left]{$z$};
 \foreach \x/\g in {A/-90,C/180,F/0,H/90}
 \fill[black] (\x) circle (1pt)
 ($(\g:4mm)+(\x)$) node {$\x$}; 
 \end{tikzpicture}
\end{center}
\shortans{4}
\loigiai{
Vì nền nhà là hình chữ nhật nên $OABC$ là hình chữ nhật, suy ra $x_A=x_B=4$, $y_C=y_B=5$.\\
Do điểm $A$ nằm trên trục $Ox$ nên tọa độ điểm $A(4;0;0)$; điểm $C$ nằm trên trục $Oy$ nên tọa độ điểm $C(0;5;0)$.\\
Tường nhà là hình chữ nhật nên $OCHE$ là hình chữ nhật, suy ra $y_H=y_C=5$.\\
Do $H$ nằm trên mặt phẳng $(Oyz)$ nên tọa độ điểm $H(0;5;3)$.\\
Khi đó $\overrightarrow{AH}=(0-4;5-0;3-0)\Rightarrow \overrightarrow{AH}=(-4;5;3)$.
Suy ra $a=-4$, $b=5$, $c=3$.\\
Vậy $a+b+c=-4+5+3=4$.
}
\end{ex}


\begin{ex}%[2-H2B4-SO-12-2425 (Nguồn Đề 3 - Bài 4)]%[VN-MT-7, Vũ Hồng Toàn]%[2H2V2-6]
 \immini{
 Một chiếc ô tô được đặt trên mặt đáy dưới một khung sắt có dạng hình hộp chữ nhật với đáy trên là hình chữ nhật $ABCD$, mặt phẳng $(ABCD)$ song song với mặt mặt phẳng nằm ngang. Khung sắt đó được buộc vào móc $E$ của chiếc cần cẩu sao cho các đoạn dây cáp $EA$, $EB$, $EC$, $ED$ có độ dài bằng nhau và cùng tạo với mặt phẳng $(ABCD)$ một góc $60^\circ$ như hình vẽ. Chiếc cần cẩu kéo khung sắt lên theo phương thẳng đứng. Biết lực căng $\overrightarrow{F_1}$, $\overrightarrow{F_2}$, $\overrightarrow{F_3}$, $\overrightarrow{F_4}$ đều có cường độ $5\, 000$ N và trọng lượng khung sắt là $2\, 000$ N. Biết trọng lượng của chiếc xe ô tô bằng $m\times 9{,}81$ N. Giá trị của $m$ làm tròn đến hàng đơn vị bằng bao nhiêu?
 }
 {
 \begin{tikzpicture}[scale=0.65, font=\footnotesize, line join=round, line cap=round, >=stealth,transform shape]
 \definecolor{bostonuniversityred}{rgb}{0.8, 0.0, 0.0}
 \definecolor{charcoal}{rgb}{0.21, 0.27, 0.31}
 \definecolor{bananayellow}{rgb}{1.0, 0.88, 0.21}
 \definecolor{anti-flashwhite}{rgb}{0.95, 0.95, 0.96}
 % \clip (-6,-3) rectangle (6,3);
 \tikzset{%
 xeoto/.pic={%
 %--------------------------
 \tikzset{xe/.pic={
 \def\N{
 (-2.7,.56)--(-2.5,.56)
 ..controls +(50:1.5) and +(165:1.5) .. (2.1,1.88)--(2.05,2)
 ..controls +(-10:.1) and +(130:.1) .. (3.25,1.75)--(3.15,1.65)
 ..controls +(-4:.2) and +(130:.15) .. (4.05,.7)--(4.25,.75)
 ..controls +(-40:.2) and +(130:.15) .. (4.55,.35)--(4.35,.26)
 ..controls +(-40:.2) and +(130:.15) .. (4.8,-.45)--(4.92,-.4)
 ..controls +(-40:.25) and +(73:.17) .. (4.8,-1.8)--(-4.4,-1.8)
 ..controls +(175:.7) and +(-175:3.2) ..cycle
 ;
 }
 \fill[bostonuniversityred] \N;
 \draw \N;
 \def\K{
 (-2.2,.56)--(3.3,.7)
 ..controls +(100:1.18) and +(43:3) .. cycle
 ;
 }
 \fill[bottom color=charcoal,top color=charcoal!20!white, middle color=charcoal!80!white] \K;
 \draw \K;
 \def\K1{
 (-2.2,.56)
 ..controls +(43:.2) and +(43:.2) .. (-1.58,1.05)--(-1.53,.57)--cycle
 ;
 }
 \draw \K1;
 \fill[charcoal] \K1;
 \def\K2{
 (1.2,1.85)
 ..controls +(-10:.1) and +(160:.1) .. (1.58,1.8)--(1.8,.65)--(1.25,.65)--cycle
 ;
 }
 \draw \K2;
 \fill[charcoal] \K2;
 \def\Kt{
 (-2.5,.56)
 ..controls +(50:1.5) and +(165:1.5) .. (2.1,1.88)--(2.05,2)
 ..controls +(170:2.2) and +(45:1.5) .. (-2.7,.56)--cycle
 ;
 }
 \fill[charcoal!50] \Kt;
 \draw \Kt;
 \def\Ks{
 (3.25,1.75)--(3.15,1.65)
 ..controls +(-4:.2) and +(130:.15) .. (4.05,.7)--(4.22,.75)
 ..controls +(120:.3) and +(-35:.3) .. cycle
 ;
 }
 \fill[charcoal!50] \Ks;
 \draw \Ks;
 %Đèn sau
 \def\D{
 (4.55,.35)--(4.35,.26)
 ..controls +(-40:.2) and +(130:.15) .. (4.8,-.45)--(4.92,-.4)
 ..controls +(110:.2) and +(-40:.15) ..cycle
 ;
 }
 \fill[bananayellow] \D;
 \draw \D;
 \def\M{
 (2.2,-1.3)--(-1.8,-1.4)--(-1.78,-1.7)
 ..controls +(-5:.3) and +(-90:.6) ..cycle
 ;
 }
 \draw \M;
 \fill[charcoal!90] \M;
 \draw (-1.6,.55)
 ..controls +(-170:.5) and +(95:.4) .. (-1.78,-1.7)
 (1.6,.65)
 ..controls +(-30:.5) and +(35:.3) .. (1.7,-1.3)
 ;
 %gương
 \def\G{
 (-1.5,.45)--(-1.4,.6)
 ..controls +(85:1) and +(20:.6) .. (-1.25,.5)--(-1.4,.33)
 ;
 }
 \draw \G;
 \fill[bostonuniversityred] \G;
 %Đèn trước
 \def\Dt{
 (-4.85,-.7)
 ..controls +(75:1) and +(65:.8) .. (-4.5,-.7)
 ..controls +(-115:.6) and +(-105:.4) .. cycle
 ;
 }
 \fill[bananayellow] \Dt;
 \draw \Dt;
 \def\Dt2{
 (-4.85,-.7)
 ..controls +(75:.6) and +(65:.4) .. (-4.7,-.7)
 ..controls +(-115:.3) and +(-105:.2) .. cycle
 ;
 }
 \fill[anti-flashwhite] \Dt2;
 \draw \Dt2;
 \draw[fill=anti-flashwhite] (-4.86,-1.45)--(-4.82,-1.5)--(-4.55,-1.3)
 ..controls +(90:.3) and +(45:.2) .. cycle;
 }}
 \tikzset{banh_xe/.pic={
 \draw[fill=charcoal] (-3.25,-1.65) circle (1) ;
 \draw[fill=anti-flashwhite] (-3.25,-1.65) circle (.7) ;
 \draw[fill=charcoal] (-3.25,-1.65) circle (.4) ;
 }}
 %----------------
 \path
 (0,0)pic[scale=1]{xe}(0,0)pic[scale=1]{banh_xe}(6.9,0)pic[scale=1]{banh_xe};
 %--------------------------------
 }}
 %%%%%%%%%%%%%%%%%%%
 \def\bc{4.25} % cạnh BC
 \def\ba{2} % cạnh BA
 \def\h{3.5} % đường cao
 \def\gocnghieng{90} % góc nghiêng
 \def\gocB{160} % góc B của đáy
 \coordinate (B1) at (0,0);
 \coordinate (A1) at (\gocB:\ba);
 \coordinate (C1) at (\bc,0.25);
 \coordinate (D1) at ($(C1)-(B1)+(A1)$);
 \coordinate[label=above left:$A$] (A) at ($(A1)+(\gocnghieng:\h)$);
 \coordinate[label=below left:$B$] (B) at ($(B1)-(A1)+(A)$);
 \coordinate[label=right:$C$] (C) at ($(C1)-(A1)+(A)$);
 \coordinate[label=above right:$D$] (D) at ($(D1)-(A1)+(A)$);
 \coordinate (E) at ($(A)!0.5!(C)+(\gocnghieng:\h)$);
 %------------
 \draw[->,blue,very thick] (E)--($(E)!0.65!(A)$) node[above left]{$\overrightarrow{F_1}$};
 \draw[->,blue,very thick] (E)--($(E)!0.65!(B)$) node[right]{$\overrightarrow{F_2}$};
 \draw[->,blue,very thick] (E)--($(E)!0.65!(C)$) node[above right]{$\overrightarrow{F_3}$};
 \draw[->,blue,very thick] (E)--($(E)!0.65!(D)$) node[left=2pt]{$\overrightarrow{F_4}$};
 %------------
 \path (E) node[left=1mm]{$E$};
 \draw[blue,very thick] (A)--(B)--(C)--(D)--cycle
 (A1)--(A) (D1)--(D) (C1)--(C)
 (A)--(E)--(B) (C)--(E)--(D);
 \draw[fill=teal] (A1)--(B1)--(C1)--(D1)--cycle;
 \draw[fill=teal!30] (A1)--(B1)--(C1)--++(0,-0.3)--([yshift=-0.3cm]B1)--([yshift=-0.3cm]A1)--cycle;
 \foreach \diem in {A1,B1,C1,D1,A,B,C,D,E} \fill (\diem)circle(1.5pt);
 %phần móc và dây
 \def\r{0.3}\def\rr{0.25}
 \coordinate (tam) at ([yshift=6mm]E);
 \draw[brown,fill=brown,line width=1pt] (tam) circle (\r cm);
 \fill (tam) circle (2pt);
 \draw[brown,line width=1pt] (tam)++(\r,0)--++(0,0.7)
 (tam)++(-\r,0)--++(0,0.7);
 \draw[line width=1.5pt] (tam)--++(0,-1.35*\r) arc(90:370:1mm);
 %%%%%%%%%%%%%%%%%%%
 \pic[scale=0.45,rotate=4] at (1.6,1.3) [pic type = xeoto];
 %--------
 \draw[blue,very thick] (B)--(B1);
 \end{tikzpicture}
 }
 \shortans{1562}
 \loigiai{
 \begin{center}
 \begin{tikzpicture}[scale=1, font=\footnotesize, line join=round, line cap=round, >=stealth]
 \def\a{4} \def\b{2.2} \def\h{4.5} \def\g{40}
 \path
 (0,0) coordinate (A)
 (0:\a) coordinate (D)++
 (-\g:\b) coordinate (C)++(180:\a) coordinate (B)
 ($(A)!.5!(C)$) coordinate (O)++ (90:\h) coordinate (E)
 ;
 \draw[dashed] (E)--(D)--(A)--(C)--(D)--(B) (E)--(O); 
 \draw (E)--(A)--(B)--(C)--(E)--(B);
 \foreach \i/\j [count =\k from 1] in {A/180,B/210,C/-10}{\draw[->] (E)-- ($(E)!0.65!(\i)$) coordinate(\i') node[midway, shift={(\j:4mm)}]{$\overrightarrow{F}_\k$};}
 \draw[->, dashed] (E)-- ($(E)!0.65!(D)$) coordinate(D') node[midway, shift={(250:4mm)}]{$\overrightarrow{F}_4$};
 \draw (A')--(B')--(C');
 \draw[dashed] (B')--(D')--(A')--(C')--(D');
 \draw[->, dashed] (E)--($(E)!0.65!(O)$) coordinate(O');
 \foreach \x /\gN in {A/160,B/-90,C/-20,E/90,O/-90,D/30, A'/160, B'/220, C'/10, D'/40, O'/-55}
 \fill(\x) circle (1pt)($(\x)+(\gN:3mm)$) node {$\x$};
 \end{tikzpicture}
 \end{center}
 Gọi $O$ là hình chiếu vuông góc của $E$ trên $(ABCD)$. Ta có $EA=EB=EC=ED$ nên các tam giác vuông $EOA$, $EOB$, $EOC$, $EOD$ bằng nhau. Suy ra $OA=OB=OC=OD$ hay $O$ là tâm hình chữ nhật $ABCD$.\\ 
 Gọi $A'$, $B'$, $C'$, $D'$ lần lượt là các điểm sao cho $\overrightarrow{EA'}=\overrightarrow{F}_1$, $\overrightarrow{EB'}=\overrightarrow{F}_2$, $\overrightarrow{EC'}=\overrightarrow{F}_3$ và $\overrightarrow{ED'}=\overrightarrow{F}_4$.\\
 Vì $\left|\overrightarrow{F}_1\right| = \left|\overrightarrow{F}_2\right| = \left|\overrightarrow{F}_3\right| = \left|\overrightarrow{F}_4\right| = 5\,000$ N nên $EA'=EB'=EC'=ED'=5\,000$. Do đó $E.A'B'C'D'$ là hình chóp có đáy $A'B'C'D'$ là hình chữ nhật.\\
 Gọi $O'$ là tâm hình chữ nhật $A'B'C'D'$, ta có $O'$ thuộc $EO$.\\ 
 Theo quy tắc hình bình hành $\overrightarrow{F_1}+\overrightarrow{F_3}=2\overrightarrow{EO'}$; $\overrightarrow{F_2}+\overrightarrow{F_4}=2\overrightarrow{EO'}$.\\
 Khi đó
 $\overrightarrow{F_1}+\overrightarrow{F_3}+\overrightarrow{F_2}+\overrightarrow{F_4}=4\overrightarrow{EO'}$.\\
 Các dây cáp $EA$, $EB$, $EC$, $ED$ có độ dài bằng nhau và cùng tạo với mặt phẳng $(ABCD)$ một góc $60^\circ$ nên $\widehat{EA'O'}=60^\circ$. Do đó\\
 \[EO'=EA'\cdot\sin 60^\circ=5\, 000\cdot\dfrac{\sqrt{3}}{2}=2\, 500\sqrt{3}.\]
 Gọi trọng lực của xe và khung sắt là $\overrightarrow{P}$. Vì chiếc xe ô tô và khung sắt ở vị trí cân bằng nên 
 \[\overrightarrow{P}=\overrightarrow{F_1}+\overrightarrow{F_2}+\overrightarrow{F_3}+\overrightarrow{F_4} = 4\overrightarrow{EO'}.\]
 Suy ra trọng lượng của xe và khung sắt là $\left|\overrightarrow{P}\right| = 4\left|\overrightarrow{EO'}\right| = 4\cdot 2\,500\cdot \sqrt{3} = 10\,000\sqrt{3}$ N.\\
 Vì khung sắt có trọng lượng bằng $2\,000$ N nên trọng lượng của xe ô tô là $10\,000\sqrt{3}-2\,000$ N.\\
 Vậy $m=\dfrac{10\,000\sqrt{3}-2\,000}{9{,}81}\approx 1\,562$.
 }
\end{ex}


\begin{ex}%[2-H2B4-SO-12-2425 (Nguồn Đề 3 - Bài 4)]%[VN-MT-7, Vũ Hồng Toàn]%[2H2V2-6]
 \immini{
 Một vật nặng có trọng lượng là $400$ N được đặt trên một khung sắt hình tròn như hình bên. Biết $ABCD$ là hình chữ nhật, mặt phẳng $(ABCD)$ song song với mặt phẳng nằm ngang. Khung sắt được móc vào điểm $S$ sao cho các đoạn dây cáp $SA$, $ SB$, $SC$, $SD$ có độ dài bằng nhau và cùng tạo với mặt phẳng $(ABCD)$ một góc bằng $45^\circ$. Chiếc cần cẩu kéo khung sắt lên theo phương thẳng đứng. Biết trọng lượng của khung sắt là $200$ N; cường độ các lực căng $\overrightarrow{F}_1$, $\overrightarrow{F}_2$, $\overrightarrow{F}_3$, $\overrightarrow{F}_4$ là bằng nhau. Tính cường độ của lực căng $\overrightarrow{F}_1$ (làm tròn đến hàng đơn vị).
 }
 {
 \begin{tikzpicture}[scale=.7,>=stealth, font=\footnotesize, line join=round, line cap=round]
 \tikzset{day/.pic=
 {\draw[shade,bottom color=brown!30,top color= white!30,rounded corners=0.5ex,line width=1.5pt,gray!80]
 (0,0) ellipse ({2.5pt} and {8pt});}
 }
 \def\h{6}
 \def\a{3}
 \def\b{1.5}
 \path
 (0,0) coordinate (O)
 ($(O)+(0,\h)$) coordinate (S)
 ($(O)+(10:\a cm and \b cm)$)coordinate (M)
 ($(O)+(180:\a cm and .8*\b cm)$)coordinate (A)
 ($(O)+(0:\a cm and .8*\b cm)$)coordinate (C)
 ($(O)+(60:\a cm and .8*\b cm)$)coordinate (B)
 ($(O)+(-120:\a cm and .8*\b cm)$)coordinate (D)
 ;
 
 \draw[fill=brown] (M) arc (10:-190:\a cm and \b cm);
 \draw[fill=white] (M) arc (10:-190:\a cm and .8*\b cm);
 \draw [fill=white] (M) arc (10:190:\a cm and .8*\b cm);
 \draw[fill,bottom color=black!30,top color= brown!70, ,left color=black!50] ($(S)-(.5,.3)$) rectangle ($(S)+(.5,.3)$);
 \foreach \m in {0,1,2,...,12}{\pic[rotate=-20] at ($(A)+(.25*\m,.5*\m)$) {day};}
 \foreach \m in {0,1,2,...,14}{\pic[rotate=-15] at ($(D)+(.11*\m,.5*\m)$) {day};}
 \foreach \m in {0,1,2,...,10}{\pic[rotate=5] at ($(B)+(-.15*\m,.5*\m)$) {day};}
 \foreach \m in {0,1,2,...,12}{\pic[rotate=18] at ($(C)+(-.25*\m,.5*\m)$) {day};}
 \foreach \x/\y in {A/180,B/150,C/-45,D/-90,S/90}
 \fill[black] (\x) circle (4pt) ($(\y:7mm)+(\x)$) node {$\x$};
 \foreach \i/\j [count =\k from 1] in {A/180,B/245,C/0,D/-65}{\draw[->] (S)-- ($(S)!0.6!(\i)$) coordinate(\i') node[midway, shift={(\j:5mm)}]{$\overrightarrow{F}_\k$};}
 \end{tikzpicture}
 }
 \shortans{212}
 \loigiai{
 \begin{center}
 \begin{tikzpicture}[scale=1, font=\footnotesize, line join=round, line cap=round, >=stealth]
 \def\a{4.5} \def\b{2.2} \def\h{4.5} \def\g{40}
 \path
 (0,0) coordinate (A)
 (0:\a) coordinate (B)++
 (-\g:\b) coordinate (C)++(180:\a) coordinate (D)
 ($(A)!.5!(C)$) coordinate (O)++ (90:\h) coordinate (S)
 ;
 \draw[dashed] (S)--(B)--(A)--(C)--(B)--(D) (S)--(O); 
 \draw (S)--(A)--(D)--(C)--(S)--(D);
 \foreach \i/\j [count =\k from 1] in {A/180,D/210,C/-10}{\draw[->] (S)-- ($(S)!0.65!(\i)$) coordinate(\i') node[midway, shift={(\j:4mm)}]{$\overrightarrow{F}_\k$};}
 \draw[->, dashed] (S)-- ($(S)!0.65!(B)$) coordinate(B') node[midway, shift={(250:4mm)}]{$\overrightarrow{F}_4$};
 \draw (A')--(D')--(C');
 \draw[dashed] (D')--(B')--(A')--(C')--(B');
 \draw[->, dashed] (S)--($(S)!0.65!(O)$) coordinate(O');
 \foreach \x /\g in {A/160,D/-90,C/-20,S/90,O/-90,B/30, A'/160, D'/220, C'/10, B'/40, O'/-55}
 \fill(\x) circle (1pt)($(\x)+(\g:3mm)$) node {$\x$};
 \end{tikzpicture}
 \end{center}
 Gọi $O$ là hình chiếu vuông góc của $S$ trên $(ABCD)$. Ta có $SA=SB=SC=SD$ nên các tam giác vuông $SOA$, $SOB$, $SOC$, $SOD$ bằng nhau. Suy ra $OA=OB=OC=OD$ hay $O$ là tâm hình chữ nhật $ABCD$.\\ 
 Gọi $A'$, $B'$, $C'$, $D'$ lần lượt là các điểm sao cho $\overrightarrow{SA'}=\overrightarrow{F}_1$, $\overrightarrow{SB'}=\overrightarrow{F}_2$, $\overrightarrow{SC'}=\overrightarrow{F}_3$ và $\overrightarrow{SD'}=\overrightarrow{F}_4$.\\
 Vì $\left|\overrightarrow{F}_1\right| = \left|\overrightarrow{F}_2\right| = \left|\overrightarrow{F}_3\right| = \left|\overrightarrow{F}_4\right|$ nên $SA'=SB'=SC'=SD'$. Do đó $S.A'B'C'D'$ là hình chóp có đáy $A'B'C'D'$ là hình chữ nhật.\\
 Gọi $O'$ là tâm hình chữ nhật $A'B'C'D'$, ta có $O'$ thuộc $SO$.\\
 Ta có
 \[\overrightarrow{F}_1 + \overrightarrow{F}_2 + \overrightarrow{F}_3 + \overrightarrow{F}_4 = \overrightarrow{SA'} + \overrightarrow{SB'} + \overrightarrow{SC'} + \overrightarrow{SD'} = \left(\overrightarrow{SA'} + \overrightarrow{SC'}\right) + \left(\overrightarrow{SB'} + \overrightarrow{SD'}\right) =4\overrightarrow{SO'}.\]
 Gọi $\overrightarrow{P}$ là trọng lực của vật nặng và khung sắt. Do vật và khung sắt ở vị trí cân bằng nên
 \[\overrightarrow{P} = \overrightarrow{F}_1 + \overrightarrow{F}_2 + \overrightarrow{F}_3 + \overrightarrow{F}_4 = 4\overrightarrow{SO'}.\]
 Theo giả thiết ta có $\left|\overrightarrow{P}\right|=400+200=600$ N nên $4\left|\overrightarrow{SO'}\right| =600 \Leftrightarrow SO'=150$.\\
 Lại có $\bigl(SA,(ABCD)\bigr)=\widehat{SAO}=\widehat{SA'O'}=45^\circ$.\\
 Vì $\triangle SO'A'$ vuông tại $O'$ nên $SA'=\dfrac{SO'}{\sin 45^{\circ}}=150\sqrt{2}$.\\
 Vậy cường độ của lực căng $\overrightarrow{F}_1$ là $\left|\overrightarrow{F}_1\right| = 150\sqrt{2}\approx 212$ N.
 }
\end{ex}


\Closesolutionfile{ans}
 
\begin{indapan}
	{ans/ans\currfilebase}
\end{indapan}


% \begin{name}
 {Biên soạn: Lê Văn Hiếu\\Phản biện: Bùi Lương Phúc}
{Đề ôn tập chương II}
\end{name}

\caulc
\Opensolutionfile{ans}[ans/ans\currfilebase-Phan-I]
\begin{ex}%[2-H2B4-SO-13-2425 (Nguồn Đề 13 - Bài 4)]%[VN-MT-7, Lê Văn Hiếu]%[2H2N1-2]
 Cho hình hộp chữ nhật $ABCD.A'B'C'D'$. Khẳng định nào sau đây đúng?
 \choice
 {$\overrightarrow{AC'}=\overrightarrow{AB}+\overrightarrow{AC}+\overrightarrow{AB}$}
 {\True $\overrightarrow{AC'}=\overrightarrow{AB}+\overrightarrow{AD}+\overrightarrow{AA'}$}
 {$\overrightarrow{AC'}=\overrightarrow{AB'}+\overrightarrow{AC}+\overrightarrow{AD'}$}
 {$\overrightarrow{AC'}=\overrightarrow{AB'}+\overrightarrow{AD'}+\overrightarrow{AA'}$}
 \loigiai{Theo quy tắc hình hộp ta có $\overrightarrow{AC'}=\overrightarrow{AB}+\overrightarrow{AD}+\overrightarrow{AA'}$.}
\end{ex}

\begin{ex}%[2-H2B4-SO-13-2425 (Nguồn Đề 13 - Bài 4)]%[VN-MT-7, Lê Văn Hiếu]%[2H2N1-3]
 Nếu một vật có khối lượng $m$ (kg) thì lực hấp dẫn $\overrightarrow{P}$ của trái đất tác dụng lên vật được xác định theo công thức $\overrightarrow P=m\overrightarrow g$, trong đó $\overrightarrow g$ là vectơ gia tốc rơi tự do có độ lớn $g=9{,}8$ (m/s$^2$). Độ lớn của lực hấp dẫn trái đất tác dụng lên một quả lê có khối lượng $105$ g là
 \choice
 {$102{,}9$ N}
 {$1029$ N}
 {\True $1{,}029$ N}
 {$10{,}29$ N}
 \loigiai{Đổi $105$ g$=0{,}105$ kg.\\
 Độ lớn của lực hấp dẫn của trái đất tác dụng lên quả lê là $\left|\overrightarrow P\right|=m\left|\overrightarrow g\right|=0{,}105\cdot9{,}8=1{,}029$ N.
 }
\end{ex}

\begin{ex}%[2-H2B4-SO-13-2425 (Nguồn Đề 13 - Bài 4)]%[VN-MT-7, Lê Văn Hiếu]%[2H2N2-3]
 Cho biết máy bay $A$ đang bay với vectơ vận tốc $\overrightarrow u=( 300;200;400)$ (đơn vị: km/h). Máy bay $B$ bay ngược hướng và có tốc độ gấp $2$ lần tốc độ của máy bay $A$. Tọa độ vectơ vận tốc $\overrightarrow v$ của máy bay $B$ là
 \choice
 {$\overrightarrow v=(600;400;800)$}
 {$\overrightarrow v=(150;100;200)$}
 {\True $\overrightarrow v=(-600;-400;-800)$}
 {$\overrightarrow v=(-150;-100;-200)$}
 \loigiai{Máy bay $B$ bay ngược hướng và có tốc độ gấp $2$ lần tốc độ của máy bay $A$ nên vectơ vận tốc $\overrightarrow{v}$ ngược hướng với vectơ vận tốc $\overrightarrow{u}$ và $|\overrightarrow{v}|=2\left|\overrightarrow{u}\right|$, do đó $\overrightarrow v=-2\overrightarrow u\Rightarrow\overrightarrow v=(-600;-400;-800)$.
}\end{ex}

\begin{ex}%[2-H2B4-SO-13-2425 (Nguồn Đề 13 - Bài 4)]%[VN-MT-7, Lê Văn Hiếu]%[2H2N2-2]
 Trong không gian $Oxyz$, cho hai điểm $A(-3;2;-1)$, $B(-1;0;5)$. Tọa độ trung điểm $I$ của đoạn thẳng $AB$ là
 \choice
 {$I(-1;1;2)$}
 {$I(2;1;-2)$}
 {$I(-2;-1;2)$}
 {\True $I(-2;1;2)$}
 \loigiai{Tọa độ trung điểm $I$ của đoạn thẳng $AB$ là $I(-2;1;2)$.
}\end{ex}

\begin{ex}%[2-H2B4-SO-13-2425 (Nguồn Đề 13 - Bài 4)]%[VN-MT-7, Lê Văn Hiếu]%[2H2N2-2]
 Trong không gian tọa độ $Oxyz$, biết $\overrightarrow{OM}=2\overrightarrow i-3\overrightarrow j+\overrightarrow k$. Toạ độ của điểm $M$ là
 \choice
 {$(-2;3;-1)$}
 {\True $(2;-3;1)$}
 {$(-3;2;1)$}
 {$(2;1;-3)$}
 \loigiai{Vì $\overrightarrow{OM}=2\overrightarrow i-3\overrightarrow j+\overrightarrow k$ nên $M(2;-3;1)$.
}\end{ex}

\begin{ex}%[2-H2B4-SO-13-2425 (Nguồn Đề 13 - Bài 4)]%[VN-MT-7, Lê Văn Hiếu]%[2H2N2-3]
 Trong không gian với hệ trục tọa độ $Oxyz$, cho hai điểm $A(-2;2;1)$, $B( 0;1;3)$. Toạ độ của vectơ $\overrightarrow{AB}$ là
 \choice
 {\True $\overrightarrow{AB}=(2;-1;2)$}
 {$\overrightarrow{AB}=(-2;3;4)$}
 {$\overrightarrow{AB}=(-2;1;-2)$}
 {$\overrightarrow{AB}=(-2;2;3)$}
 \loigiai{Ta có $\overrightarrow{AB}=(2;-1;2)$.
}\end{ex}

\begin{ex}%[2-H2B4-SO-13-2425 (Nguồn Đề 13 - Bài 4)]%[VN-MT-7, Lê Văn Hiếu]%[2H2N1-3]
 Cho tứ diện đều $ABCD$ có cạnh bằng $2$. Tính $\overrightarrow{AB}\cdot\overrightarrow{CD}$.
 \choice
 {$\overrightarrow{AB}\cdot\overrightarrow{CD}=-4$}
 {$\overrightarrow{AB}\cdot\overrightarrow{CD}=2$}
 {$\overrightarrow{AB}\cdot\overrightarrow{CD}=1$}
 {\True $\overrightarrow{AB}\cdot\overrightarrow{CD}=0$}
 \loigiai{
 Vì $ABCD$ là tứ diện đều nên các tam giác $ABC$ và $ABD$ là các tam giác đều.\\
 Khi đó
 \begin{eqnarray*}
 \overrightarrow{AB}\cdot\overrightarrow{CD}&=&\overrightarrow{AB}\cdot(\overrightarrow{AD}-\overrightarrow{AC})\\
 &=&\overrightarrow{AB}\cdot\overrightarrow{AD}-\overrightarrow{AB}\cdot\overrightarrow{AC}\\
 &=&2\cdot2\cdot\cos60^\circ-2\cdot2\cdot\cos60^\circ=0.
 \end{eqnarray*}
}\end{ex}

\begin{ex}%[2-H2B4-SO-13-2425 (Nguồn Đề 13 - Bài 4)]%[VN-MT-7, Lê Văn Hiếu]%[2H2N2-5]
 Trong không gian với hệ trục tọa độ $Oxyz$, cho hai điểm $M(-5;2;3)$, $I(2;3;1)$. Gọi $N$ là điểm đối xứng với $M$ qua $I$. Tính độ dài đoạn $ON$.
 \choice
 {$ON=6\sqrt2$}
 {$ON=5\sqrt2$}
 {\True $ON=7\sqrt2$}
 {$ON=3\sqrt2$}
 \loigiai{Vì $N$ là điểm đối xứng với $M$ qua $I$ nên $I$ là trung điểm của đoạn $MN$, do đó $N( 9;4;-1)$.\\
 Vậy $ON=\sqrt{9^2+4^2+(-1)^2}=7\sqrt2$.
}\end{ex}

\begin{ex}%[2-H2B4-SO-13-2425 (Nguồn Đề 13 - Bài 4)]%[VN-MT-7, Lê Văn Hiếu]%[2H2N2-5]
 Trong không gian với hệ trục tọa độ $Oxyz$, cho hai vectơ $\overrightarrow a=(1;-2;0)$ và $\overrightarrow b=(-2;3;1)$. Cho các mệnh đề sau.
 \begin{enumEX}{2}
 \item $\overrightarrow a\cdot\overrightarrow b=-8$.
 \item $2\overrightarrow a=(2;-4;1)$.
 \item $\overrightarrow a+\overrightarrow b=(-1;0;-1)$.
 \item $\left|\overrightarrow b\right|=14$.
 \end{enumEX}
 Số mệnh đề đúng là
 \choice
 {\True $1$}
 {$3$}
 {$2$}
 {$4$}
 \loigiai{Ta có
 \begin{enumerate}
 \item $\overrightarrow a\cdot\overrightarrow b=-8$.
 \item $2\overrightarrow a=(2;-4;0)$.
 \item $\overrightarrow a+\overrightarrow b=(-1;1;1)$.
 \item $\left|\overrightarrow b\right|=\sqrt{14}$.
 \end{enumerate}
 Vậy số mệnh đề đúng là $1$.
}\end{ex}

\begin{ex}%[2-H2B4-SO-13-2425 (Nguồn Đề 13 - Bài 4)]%[VN-MT-7, Lê Văn Hiếu]%[2H2N2-5]
 Trong không gian với hệ trục tọa độ $Oxyz$, cho $\overrightarrow a=(1;-2;3)$ và $\overrightarrow b=(2;-1;-1)$. Mệnh đề nào là mệnh đề \textbf{sai}?
 \choice
 {Vectơ $\overrightarrow{u}=(-5;-7;-3)$ cùng vuông góc với vectơ $\overrightarrow a$ và $\overrightarrow b$}
 {Vectơ $\overrightarrow a$ không cùng phương với vectơ $\overrightarrow b$}
 {Vectơ $\overrightarrow a$ không vuông góc với vectơ $\overrightarrow b$}
 {\True $\left|\overrightarrow a\right|=14$}
 \loigiai{
 Ta có $\left[\overrightarrow a,\overrightarrow b\right]=( 5;7;3)$, suy ra vectơ $\overrightarrow{u}=(-5;-7;-3)$ cùng phương với $\left[\overrightarrow a,\overrightarrow b\right]$ nên $\overrightarrow{u}$ vuông góc với hai vectơ $\overrightarrow{a}$ và $\overrightarrow{b}$.\\
 Do $\dfrac12\ne\dfrac{-2}{-1}$ nên vectơ $\overrightarrow a$ không cùng phương với vectơ $\overrightarrow b$.\\
 Do $\overrightarrow a\cdot\overrightarrow b=1\cdot2+(-2)(-1)+3(-1)=1$ nên vectơ $\overrightarrow a$ không vuông góc với vectơ $\overrightarrow b$.\\
 Ta có $\left|\overrightarrow a\right|=\sqrt{1^2+(-2)^2+3^2}=\sqrt{14}$.
}\end{ex}

\begin{ex}%[2-H2B4-SO-13-2425 (Nguồn Đề 13 - Bài 4)]%[VN-MT-7, Lê Văn Hiếu]%[2H2H1-2]
 Cho hình chóp $S.ABCD$ có đáy $ABCD$ là hình bình hành tâm $O$. Trong các mệnh đề sau mệnh đề nào là mệnh đề \textbf{sai}?
 \choice
 {$\overrightarrow{SA}+\overrightarrow{SB}+\overrightarrow{SC}+\overrightarrow{SD}=4\overrightarrow{SO}$}
 {$\overrightarrow{SA}-\overrightarrow{SB}+\overrightarrow{SC}-\overrightarrow{SD}=\overrightarrow0$}
 {\True $\overrightarrow{SA}+\overrightarrow{SB}+\overrightarrow{SC}+\overrightarrow{SD}=\overrightarrow0$}
 {$\overrightarrow{OA}+\overrightarrow{OB}+\overrightarrow{OC}+\overrightarrow{OD}=\overrightarrow0$}
 \loigiai{
 \begin{center}
 \begin{tikzpicture}[scale=1, font=\footnotesize, line join=round, line cap=round, >=stealth]
 \coordinate (A) at (0,0);
 \coordinate (B) at (-2,-1);
 \coordinate (C) at (3,-1);
 \coordinate (D) at ($(A)+(C)-(B)$);
 \coordinate (O) at ($(A)!1/2!(C)$);
 \coordinate (S) at ($(O)+(0,5)$);
 \draw(S)--(B)--(C)--(D)--(S)--(C);
 \draw[dashed](B)--(A)--(D)--(B)(C)--(A)--(S)--(O);
 \foreach \p/\g in {A/150, B/-90, C/-90, D/0, S/90, O/-90}\draw[fill=black] (\p) circle (1pt)node[shift={(\g:.3)},scale=1]{$\p$};
 \end{tikzpicture}
 \end{center}
 \begin{itemize}
 \item Ta có $O$ là trung điểm của $AC$ nên $\overrightarrow{SA}+\overrightarrow{SC}=2\overrightarrow{SO}$.\\
 $O$ là trung điểm của $BD$ nên $\overrightarrow{SB}+\overrightarrow{SD}=2\overrightarrow{SO}$.\\
 Do đó $\overrightarrow{SA}+\overrightarrow{SB}+\overrightarrow{SC}+\overrightarrow{SD}=4\overrightarrow{SO}$ là khẳng định đúng.
 \item $\overrightarrow{SA}-\overrightarrow{SB}+\overrightarrow{SC}-\overrightarrow{SD}=\overrightarrow{BA}+\overrightarrow{DC}=\overrightarrow0$ là khẳng định đúng.
 \item Ta có $\overrightarrow{SA}+\overrightarrow{SB}+\overrightarrow{SC}+\overrightarrow{SD}=4\overrightarrow{SO}$ như chứng minh trên.\\
 Do đó $\overrightarrow{SA}+\overrightarrow{SB}+\overrightarrow{SC}+\overrightarrow{SD}=\overrightarrow0$ là khẳng định sai.
 \item Ta có $O$ là trung điểm của $AC$ nên $\overrightarrow{OA}+\overrightarrow{OC}=\overrightarrow{O}$.\\
 $O$ là trung điểm của $BD$ nên $\overrightarrow{OB}+\overrightarrow{OD}=\overrightarrow0$.\\
 Do đó $\overrightarrow{OA}+\overrightarrow{OB}+\overrightarrow{OC}+\overrightarrow{OD}=\overrightarrow0$ là khẳng định đúng.
 \end{itemize}
 }
\end{ex}

\begin{ex}%[2-H2B4-SO-13-2425 (Nguồn Đề 13 - Bài 4)]%[VN-MT-7, Lê Văn Hiếu]%[2H2V1-4]
 \immini[thm]{
 Một chiếc đèn chùm treo có khối lượng $m=5$ kg được thiết kế với đĩa đèn được giữ bởi bốn đoạn xích $SA$, $SB$, $SC$, $SD$ sao cho $S.ABCD$ là hình chóp tứ giác đều có $\widehat{ASC}=60^\circ$ (Hình bên).\\
 Biết $\overrightarrow{P}=m\overrightarrow g$ trong đó $\overrightarrow g$ là vectơ gia tốc rơi tự do có độ lớn $10$ m/s$^2$, $\overrightarrow{P}$ là trọng lực tác động lên vật có đơn vị là N, $m$ là khối lượng của vật có đơn vị kg. Cho các kết luận dưới đây.
 \begin{enumerate}
 \item $SA$, $SB$ là hai vectơ cùng phương.
 \item $\left|\overrightarrow{SA}\right|=\left|\overrightarrow{SB}\right|=\left|\overrightarrow{SC}\right|=\left|\overrightarrow{SD}\right|$.
 \item Độ lớn của trọng lực $\overrightarrow{P}$ tác động lên chiếc đèn chùm bằng $50$ N.
 \item Độ lớn của lực căng cho mỗi sợi xích bằng $\dfrac{25\sqrt3}6$ N.
 \end{enumerate}
 }{
 \begin{tikzpicture}[scale=.65,>=stealth, font=\footnotesize, line join=round, line cap=round]
 \tikzset{day/.pic=
 {\draw[shade,bottom color=brown!30,top color= white!30,rounded corners=0.5ex,line width=1.5pt,gray!80]
 (0,0) ellipse ({2.5pt} and {8pt});}
 }
 \def\h{6}
 \def\a{3}
 \def\b{1.5}
 \path
 (0,0) coordinate (O)
 ($(O)+(0,\h)$) coordinate (S)
 ($(O)+(10:\a cm and \b cm)$)coordinate (M)
 ($(O)+(180:\a cm and .8*\b cm)$)coordinate (A)
 ($(O)+(0:\a cm and .8*\b cm)$)coordinate (C)
 ($(O)+(60:\a cm and .8*\b cm)$)coordinate (B)
 ($(O)+(-120:\a cm and .8*\b cm)$)coordinate (D)
 ;
 \draw[fill=brown] (M) arc (10:-190:\a cm and \b cm);
 \draw[fill=white] (M) arc (10:-190:\a cm and .8*\b cm);
 \draw [fill=white] (M) arc (10:190:\a cm and .8*\b cm);
 \draw[fill,bottom color=black!30,top color= brown!70, ,left color=black!50] ($(S)-(.5,.3)$) rectangle ($(S)+(.5,.3)$);
 \foreach \m in {0,1,2,...,12}{\pic[rotate=-20] at ($(A)+(.25*\m,.5*\m)$) {day};}
 \foreach \m in {0,1,2,...,14}{\pic[rotate=-15] at ($(D)+(.11*\m,.5*\m)$) {day};}
 \foreach \m in {0,1,2,...,10}{\pic[rotate=5] at ($(B)+(-.15*\m,.5*\m)$) {day};}
 \foreach \m in {0,1,2,...,12}{\pic[rotate=18] at ($(C)+(-.25*\m,.5*\m)$) {day};}
 \foreach \x/\y in {A/180,B/220,C/-45,D/-90,S/90}
 \fill[black] (\x) circle (4pt) ($(\y:7mm)+(\x)$) node {$\x$};
 \end{tikzpicture}
 }
 \noindent
 Số kết luận đúng là
 \choice
 {$1$}
 {\True $2$}
 {$3$}
 {$0$}
 \loigiai{
 \begin{enumerate}
 \item $SA$, $SB$ là hai vectơ cùng phương. \textbf{Sai}
 \item $\left|\overrightarrow{SA}\right|=\left|\overrightarrow{SB}\right|=\left|\overrightarrow{SC}\right|=\left|\overrightarrow{SD}\right|$. \textbf{Đúng}
 \item Độ lớn của trọng lực $\overrightarrow{P}$ tác động lên chiếc đèn chùm là $\left|\overrightarrow{P}\right|=m\cdot\left|\overrightarrow{g}\right|=5\cdot10=50$ N. \textbf{Đúng}
 \item Độ lớn của lực căng cho mỗi sợi xích bằng $\dfrac{25\sqrt3}6$ N. \textbf{Sai}\\
 Ta có $S.ABCD$ là hình chóp tứ giác đều $\Rightarrow SA=SB=SC=SD$.\\
 Mà $\widehat{ASC}=60^\circ\Rightarrow$ tam giác $SAC$ đều.\\
 Gọi $O$ là trung điểm $AC$.\\
 Ta có hợp lực của $4$ lực căng của $4$ sợi xích
 \[\overrightarrow{F}=\overrightarrow{SA}+\overrightarrow{SC}+\overrightarrow{SB}+\overrightarrow{SD}=2\overrightarrow{SO}+2\overrightarrow{SO}=4\overrightarrow{SO}.\]
 Để đèn chùm đứng yên thì hợp lực của các sợi xích phải cân bằng với trọng lực hay $4\overrightarrow{SO}=\overrightarrow{P}$ hay $4SO=50\Leftrightarrow SO=12{,}5$.\\
 Xét tam giác đều $SAC$ có $SA=\dfrac{\sqrt3}2SO=\dfrac{25\sqrt3}4$.\\
 Vậy độ lớn của lực căng cho mỗi sợi xích là $\dfrac{25\sqrt3}4$ N.
 \end{enumerate}
}\end{ex}
\Closesolutionfile{ans}

\cauds
\Opensolutionfile{ans}[ans/ans\currfilebase-Phan-II]
\begin{ex}%[2-H2B4-SO-13-2425 (Nguồn Đề 13 - Bài 4)]%[VN-MT-7, Lê Văn Hiếu]%[2H2H2-4]
 Trong không gian với hệ tọa độ $Oxyz$, cho hai vectơ $\overrightarrow a=(1;2;-2)$ và $\overrightarrow b=(-1;-1;0)$.
 \choiceTF
 {$\left|\overrightarrow a\right|=9$}
 {\True $\overrightarrow a+\overrightarrow b=( 0;1;-2)$}
 {$\overrightarrow a$ và $\overrightarrow b$ cùng phương}
 {\True $\left(\overrightarrow a,\overrightarrow b\right)=135^\circ$}
 \loigiai{
 \begin{itemchoice}
 \itemch \textbf{Sai}.\\
 Ta có $\left|\overrightarrow a\right|=\sqrt{1^2+2^2+(-2)^2}=3$.
 \itemch \textbf{Đúng}.\\
 Ta có $\overrightarrow a+\overrightarrow b=(1-1;2-1;-2+0)\Rightarrow \overrightarrow a+\overrightarrow b=( 0;1;-2)$.
 \itemch \textbf{Sai}.\\
 Ta có $\dfrac1{-1}\ne\dfrac2{-1}$ nên $\overrightarrow a$ và $\overrightarrow b$ không cùng phương.
 \itemch \textbf{Đúng}.\\
 Áp dụng công thức:
 \begin{align*}
 \cos(\overrightarrow a,\overrightarrow b)&=\dfrac{\overrightarrow a\cdot\overrightarrow b}{\left|\overrightarrow a\right|\left|\overrightarrow b\right|}\\
 &=\dfrac{1\cdot(-1)+2\cdot(-1)+(-2)\cdot0}{\sqrt{1^2+2^2+(-2)^2}\cdot\sqrt{(-1)^2+(-1)^2+0^2}}=\dfrac{-3}{3\sqrt2}=-\dfrac1{\sqrt2}.
 \end{align*}
 Suy ra $(\overrightarrow a,\overrightarrow b)=135^\circ$.
 \end{itemchoice}
}\end{ex}

\begin{ex}%[2-H2B4-SO-13-2425 (Nguồn Đề 13 - Bài 4)]%[VN-MT-7, Lê Văn Hiếu]%[2H2H2-4]
 Cho $4$ điểm $A(1;2;0)$, $B(5;1;4)$, $C(7;-2;-2)$, $D(3;m;2)$.
 \choiceTF
 {Độ dài đoạn $AB$ lớn hơn độ dài đoạn $AC$}
 {\True $ m=\dfrac32$ thì $D$ là trung điểm của $AB$}
 {$ m=5$ thì $AB\perp AD$}
 {$ m=-1$ thì $AB\parallel CD$}
 \loigiai{
 \begin{itemchoice}
 \itemch \textbf{Sai}.\\
 Ta có $AB=\sqrt{4^2+(-1)^2+4^2}=\sqrt{33}$ và $AC=\sqrt{6^2+(-4)^2+(-2)^2}=\sqrt{56}$.
 \itemch \textbf{Đúng}.\\
 Tọa độ trung điểm của đoạn $AB$ là $\left(3;\dfrac32;2\right)\Rightarrow m=\dfrac32$.
 \itemch \textbf{Sai}. Với $m=5$, ta có $D(3;5;2)$.\\
 Ta có $AB\perp AD\Leftrightarrow\overrightarrow{AB}\cdot\overrightarrow{AD}=0$.\\
 Vì $\heva{&\overrightarrow{AB}=(4;-1;4)\\
 &\overrightarrow{AD}=(2;3;2)}$ nên $\overrightarrow{AB}\cdot\overrightarrow{AD}=4\cdot2+(-1)\cdot3+4\cdot2=13\ne0$.
 \itemch \textbf{Sai}. Với $m=-1$, ta có $D(3;-1;2)$.\\
 Ta có $\heva{&\overrightarrow{AB}=(4;-1;4)\\
 &\overrightarrow{CD}=(-4;-1;4)} \Rightarrow\overrightarrow{AB}\ne k\overrightarrow{CD}$ (với mọi số thực $k$) $\Rightarrow AB$ không song song với $CD$.
 \end{itemchoice}
}\end{ex}

\begin{ex}%[2-H2B4-SO-13-2425 (Nguồn Đề 13 - Bài 4)]%[VN-MT-7, Lê Văn Hiếu]%[2H2H2-4]
 Trong không gian $Oxyz$, cho các điểm $A( 8;9;2)$, $B( 3;5;1)$ và $C(11;10;4)$.
 \choiceTF
 {Điểm $D$ thỏa mãn $ABCD$ là hình bình hành có tọa độ là $D( 6;6;3)$}
 {\True Độ dài trung tuyến $AM$ bằng $\dfrac{\sqrt{14}}2$}
 {$\widehat{BAC}=30^\circ$}
 {\True Điểm $N$ thuộc mp$( Oxy)$ sao cho ba điểm $A$, $B$, $N$ thẳng hàng có tọa độ là $N(-2;1;0)$}
 \loigiai{
 \begin{itemchoice}
 \itemch \textbf{Sai}. Giả sử $D(x;y;z)$.\\
 Ta có $\overrightarrow{AB}=(-5;-4;-1)$, $\overrightarrow{DC}=(11-x;10-y;4-z)$.\\
 Tứ giác $ABCD$ là hình bình hành $\Leftrightarrow \overrightarrow{AB}=\overrightarrow{DC}\Leftrightarrow\heva{
 &-5=11-x\\
 &-4=10-y\\
 &-1=4-z\\
 }\Leftrightarrow\heva{
 &x=16\\
 &y=14\\
 &z=5.}$\\
 Vậy $D(16;14;5)$.
 \itemch \textbf{Đúng}.\\
 Tọa độ trung điểm $M$ của $BC$ là $M\left(7;\dfrac{15}2;\dfrac52\right)$.\\
 Ta có $\overrightarrow{AM}=\left(-1;-\dfrac32;\dfrac12\right)$. Suy ra $AM=\sqrt{(-1)^2+\left(-\dfrac32\right)^2+\left(\dfrac12\right)^2}=\dfrac{\sqrt{14}}2$.
 \itemch \textbf{Sai}.\\
 Ta có $\overrightarrow{AB}=(-5;-4;-1)$; $\overrightarrow{AC}=( 3;1;2)$.\\
 Do đó
 \begin{align*}
 \cos\widehat{BAC}&=\cos\left(\overrightarrow{AB},\overrightarrow{AC}\right)=\dfrac{\overrightarrow{AB}\cdot\overrightarrow{AC}}{AB\cdot AC}\\
 &=\dfrac{(-5)\cdot3+(-4)\cdot1+(-1)\cdot2}{\sqrt{(-5)^2+(-4)^2+(-1)^2}\cdot\sqrt{3^2+1^2+2^2}}=-\dfrac{\sqrt3}2.
 \end{align*}
 Suy ra $\widehat{BAC}=150^\circ$.
 \itemch \textbf{Đúng}.\\
 Vì $N\in( Oxy)$ nên $N(x;y;0)$. Ta có $\overrightarrow{AB}=(-5;-4;-1)$; $\overrightarrow{AN}=(x-8;y-9;-2)$.\\
 Vì $3$ điểm $A$, $B$, $N$ thẳng hàng nên $\overrightarrow{AB}$ cùng phương với $\overrightarrow{AN}$. Khi đó
 \[\heva{
 &\dfrac{x-8}{-5}=2\\
 &\dfrac{y-9}{-4}=2\\
 }\Leftrightarrow\heva{
 &x-8=-10\\
 &y-9=-8\\
 }\Leftrightarrow\heva{
 &x=-2\\
 &y=1.\\
 }\]
 Vậy $N(-2;1;0)$.
 \end{itemchoice}
}\end{ex}

\begin{ex}%[2-H2B4-SO-13-2425 (Nguồn Đề 13 - Bài 4)]%[VN-MT-7, Lê Văn Hiếu]%[2H2V1-3]
 \immini[thm]{
 Cho hình hộp $ABCD.A'B'C'D'$. Gọi $M$, $N$ là các điểm lần lượt thuộc các đường thẳng $CA$ và $DC'$ sao cho $\overrightarrow{MC}=m\overrightarrow{MA}$, $\overrightarrow{ND}=m\overrightarrow{NC'}$. Đặt $\overrightarrow{BA}=\overrightarrow a$, $\overrightarrow{BB'}=\overrightarrow b$, $\overrightarrow{BC}=\overrightarrow c$.
 \choiceTF
 {$\overrightarrow{BD'}=\overrightarrow a+\overrightarrow b-\overrightarrow c$}
 {\True $\overrightarrow{BM}=\dfrac1{1-m}\overrightarrow c-\dfrac m{1-m}\overrightarrow a$}
 {\True $\overrightarrow{BN}=\dfrac1{1-m}\overrightarrow a-\dfrac m{1-m}\overrightarrow b+\overrightarrow c$}
 {$ m=\dfrac12$ thì $MN\parallel BD'$}
 }{
 \begin{tikzpicture}[scale=0.85,>=stealth, font=\footnotesize, line join=round, line cap=round]
 \coordinate (A) at (0,0);
 \coordinate (B) at (-2,-1);
 \coordinate (C) at ($(B)+(4,0)$);
 \coordinate (D) at ($(A)+(C)-(B)$);
 \coordinate (A') at ($(A)+(0,4)$);
 \coordinate (B') at ($(B)+(A')-(A)$);
 \coordinate (C') at ($(C)+(A')-(A)$);
 \coordinate (D') at ($(D)+(A')-(A)$);
 \coordinate (M) at ($(C)!1/3!(A)$);
 \coordinate (N) at ($(D)!1/3!(C')$);
 \draw(A')--(B')--(C')--(D')--(A')(B)--(B')(C)--(C')(D)--(D')(B)--(C)--(D)--(C');
 \draw[dashed](A)--(A')(A)--(B)(A)--(D)(C)--(A);
 \foreach \p/\g in {A/160, B/-90, C/-90, D/-90, A'/90, B'/90, C'/90, D'/90, M/90, N/90}\draw[fill=black] (\p) circle (1pt)node[shift={(\g:.3)},scale=1]{$\p$};
 \end{tikzpicture}
 }
 \loigiai{
 Dễ thấy $m\ne1$ vì nếu $m=1$, khi đó $\overrightarrow{MC}=\overrightarrow{MA}\Leftrightarrow \overrightarrow{AC}=\overrightarrow{0}\Leftrightarrow A\equiv C$ (vô lý).
 \begin{itemchoice}
 \itemch \textbf{Sai}.\\
 Theo quy tắc hình hộp ta có $\overrightarrow{BD'}=\overrightarrow a+\overrightarrow b+\overrightarrow c$.
 \itemch \textbf{Đúng}.\\
 Ta có
 \allowdisplaybreaks
 \begin{eqnarray*}
 &&\overrightarrow{MC}=m\overrightarrow{MA}\\
 &\Rightarrow&\overrightarrow{BC}-\overrightarrow{BM}=m\overrightarrow{BA}-m\overrightarrow{BM}\\
 &\Rightarrow&(1-m)\overrightarrow{BM}=\overrightarrow{BC}-m\overrightarrow{BA}\\
 &\Rightarrow&\overrightarrow{BM}=\dfrac1{1-m}\overrightarrow{BC}-\dfrac m{1-m}\overrightarrow{BA}=\dfrac1{1-m}\overrightarrow c-\dfrac m{1-m}\overrightarrow a.
 \end{eqnarray*}
 \itemch \textbf{Đúng}.\\
 Tương tự ta có
 \allowdisplaybreaks
 \begin{align*}
 \overrightarrow{BN}&=\dfrac1{1-m}\overrightarrow{BD}-\dfrac m{1-m}\overrightarrow{BC'}\\
 &=\dfrac1{1-m}\overrightarrow a+\dfrac1{1-m}\overrightarrow c-\dfrac m{1-m}(\overrightarrow b+\overrightarrow c)\\
 &=\dfrac1{1-m}\overrightarrow a-\dfrac m{1-m}\overrightarrow b+\overrightarrow c
 \end{align*}
 \itemch \textbf{Sai}.\\
 Ta có
 \allowdisplaybreaks
 \begin{eqnarray*}
 &&\overrightarrow{MN}=\overrightarrow{BN}-\overrightarrow{BM}\\
 &\Rightarrow&\overrightarrow{MN}=\dfrac{1+m}{1-m}\overrightarrow a-\dfrac m{1-m}\overrightarrow b-\dfrac m{1-m}\overrightarrow c.
 \end{eqnarray*}
 Vì $MN\parallel BD'$ nên $\overrightarrow{MN}$ cùng phương $\overrightarrow{BD'}$. Từ đó ta có
 \allowdisplaybreaks
 \begin{eqnarray*}
 &&\overrightarrow{MN}=k\overrightarrow{BD'}\\
 &\Rightarrow&\heva{
 &\dfrac{1+m}{1-m}=k\\
 &-\dfrac m{1-m}=k\\
 &-\dfrac m{1-m}=k\\
 }\\
 &\Rightarrow&m=-\dfrac12.
 \end{eqnarray*}
 \end{itemchoice}
}\end{ex}
\Closesolutionfile{ans}

\caukq
\Opensolutionfile{ans}[ans/ans\currfilebase-Phan-III]
\begin{ex}%[2-H2B4-SO-13-2425 (Nguồn Đề 13 - Bài 4)]%[VN-MT-7, Lê Văn Hiếu]%[2H2H2-3]
 Trong không gian với hệ tọa độ $Oxyz$, cho các điểm $A(1;0;3)$, $B(2;3;-4)$, $C(-3;1;2)$. Gọi $D(x;y;z)$ là điểm sao cho $ABCD$ là hình bình hành. Tính tổng $T=x+y+z$.
 \shortans[]{3}
 \loigiai{
 Ta có $\overrightarrow{AB}=(1;3;-7)$, $\overrightarrow{DC}=(-3-x;1-y;2-z)$.\\
 Tứ giác $ABCD$ là hình bình hành khi
 \[\overrightarrow{AB}=\overrightarrow{DC}\Leftrightarrow \heva{&1=-3-x\\ &3=1-y\\ &-7=2-z} \Leftrightarrow \heva{&x=-4\\ &y=-2\\ &z=9.}\]
 Vậy, $D(-4;-2;9)$.
 Khi đó $T=-4-2+9=3$.
}\end{ex}

\begin{ex}%[2-H2B4-SO-13-2425 (Nguồn Đề 13 - Bài 4)]%[VN-MT-7, Lê Văn Hiếu]%[2H2H2-2]
 Trong không gian với hệ tọa độ $Oxyz$, cho hình vuông $ABCD$ có $B(3;0;8)$, $D(-5;-4;0)$. Tính $\left|\overrightarrow{CA}+\overrightarrow{CB}\right|$ (kết quả làm tròn đến hàng đơn vị).
 \shortans[]{19}
 \loigiai{
 \begin{center}
 \begin{tikzpicture}[scale=1,>=stealth, font=\footnotesize, line join=round, line cap=round]
 \coordinate (A) at (0,0);
 \coordinate (B) at (-4,0);
 \coordinate (C) at (-4,-4);
 \coordinate (D) at ($(C)+(A)-(B)$);
 \coordinate (M) at ($(A)!1/2!(B)$);
 \draw(A)--(B)--(C)--(D)--(A)(B)--(D)(C)--(M);
 \foreach \p/\g in {A/90,B/90,C/180,D/0, M/90}\draw[fill=black] (\p) circle (1pt)node[shift={(\g:.3)},scale=1]{$\p$};
 \end{tikzpicture}
 \end{center}
 Ta có $\overrightarrow{BD}=(-8;-4;-8)$ $\Rightarrow BD=12$ $\Rightarrow AB=\dfrac{12}{\sqrt2}$ $=6\sqrt2$.\\
 Gọi $M$ là trung điểm $AB$ ta có $BM=\dfrac12AB=3\sqrt2$.\\
 Áp dụng định lí Pythagore ta có $MC=\sqrt{BC^2+BM^2}=\sqrt{72+18}=3\sqrt{10}$.\\
 Từ đó $\left|\overrightarrow{CA}+\overrightarrow{CB}\right|$ $=\left| 2\overrightarrow{CM}\right|$ $=2CM$ $=6\sqrt{10}\approx19$.
}\end{ex}

\begin{ex}%[2-H2B4-SO-13-2425 (Nguồn Đề 13 - Bài 4)]%[VN-MT-7, Lê Văn Hiếu]%[2H2H2-4]
 Trong không gian với hệ trục tọa độ $Oxyz$, cho hai vectơ $\overrightarrow a=(1;-2;0)$, $\overrightarrow b=(1;3;-2)$. Tính góc giữa hai vectơ $\overrightarrow a$ và $\overrightarrow b$ (tính theo độ làm tròn đến hàng đơn vị).
 \shortans[]{127}
 \loigiai{Ta có
 \[\cos(\overrightarrow a,\overrightarrow b)=\dfrac{\overrightarrow a\cdot\overrightarrow b}{\left|\overrightarrow a\right|\cdot\left|\overrightarrow b\right|}=\dfrac{1-6}{\sqrt{1^2+(-2)^2+0}\cdot\sqrt{1^2+3^2+(-2)^2}}=\dfrac{-5}{\sqrt5\cdot\sqrt{14}}.\]
 Vậy $(\overrightarrow a,\overrightarrow b)\approx 127^\circ$.
}\end{ex}

\begin{ex}%[2-H2B4-SO-13-2425 (Nguồn Đề 13 - Bài 4)]%[VN-MT-7, Lê Văn Hiếu]%[2H2H2-6]
 \immini[thm]{Trong một phòng học dạng hình hộp chữ nhật, với chiều dài $8$ m, chiều rộng $6$ m và chiều cao $3$ m. Hai bạn An và Bình làm nhiệm vụ trực nhật, mạng nhện cần quét ở góc ngoài cùng trên trần nhà, An bảo không nên đứng ngay vị trí đó ở nền nhà quét vì bụi sẽ rơi xuống người mình. An lại đố Bình ``nếu mình đứng ở giữa nhà quét thì chổi quét nhà dài mấy mét để quét được vị trí mạng nhện, biết đầu cán chổi (vị trí $B$ trên hình vẽ minh họa) cao $1{,}5$ m so với sàn nhà''. Bình trả lời đứng vị trí đó chổi dài $5$ m cũng không tới. Hỏi Bình đã tính được chổi cần dài bao nhiêu mét (làm tròn kết quả đến hàng phần trăm)?
 }{
 \begin{tikzpicture}[scale=.5,>=stealth, font=\footnotesize, line join=round, line cap=round]
 \coordinate (O) at (0,0);
 \coordinate (A') at (-140:3);
 \coordinate (A) at ($(A')+(0,5)$);
 \coordinate (B') at (8,0);
 \coordinate (C') at ($(A')+(8,0)$);
 \coordinate (b) at ($(B')+(0,5)$);
 \coordinate (c) at ($(C')+(0,5)$);
 \coordinate (o') at ($(O)+(0,5)$);
 \coordinate (x) at ($(O)!1.5!(A')$);
 \coordinate (y) at ($(O)!1.25!(B')$);
 \coordinate (z) at ($(O)!1.5!(o')$);
 \coordinate (b') at ($(A')!1/2!(B')$);
 \coordinate (B) at ($(b')+(0,2.5)$);
 \draw[dashed](O)--(A')(O)--(B')(O)--(o');
 \draw[-stealth](A')--(x);
 \draw[-stealth](B')--(y);
 \draw[-stealth](o')--(z);
 \draw[dashed](b')--(B)--(A);
 \draw(A')--(C') node[midway, below]{$8$m}--(B') node[midway, right, shift={(-45:.3)}]{$6$m}--(b)--(c)--(A)--(o')--(b)(C')--(c)(A)--(A');
 \foreach \p/\g in {A/90,x/-90,y/-90,z/0, O/-90, B/0}\draw[fill=black] (\p) circle (1pt)node[shift={(\g:.3)},scale=1]{$\p$};
 \end{tikzpicture}
 }
\shortans[]{5{,}22}
 \loigiai{
 Xét hệ tọa độ $Oxyz$ như hình vẽ, ta có vị trí mạng nhện ở $A(6;0;3)$ vị trí cầm chổi $B\left(3;4;\dfrac32\right)$.\\
 Vậy chổi phải có độ dài $AB=\sqrt{(3-6)^2+(4-0)^2+\left(\dfrac32-3\right)^2}=\dfrac{\sqrt{109}}2\approx 5{,}22$ m.
}\end{ex}

\begin{ex}%[2-H2B4-SO-13-2425 (Nguồn Đề 13 - Bài 4)]%[VN-MT-7, Lê Văn Hiếu]%[2H2V2-6]
 \immini[thm]{
 Với hệ trục tọa độ $Oxyz$ sao cho $O$ nằm trên mặt nước, mặt phẳng $(Oxy)$ là mặt nước, trục $Oz$ hướng lên trên (đơn vị đo: mét), một con chim bói cá đang ở vị trí $C$ cách mặt nước $2$ m, cách mặt phẳng $(Oxz)$, $(Oyz)$ lần lượt là $3$ m và $1$ m phóng thẳng xuống vị trí con cá, biết con cá cách mặt nước $50$ cm, cách mặt phẳng $(Oxz)$, $(Oyz)$ lần lượt là $1$ m và $1{,}5$ m. Gọi $B(a;b;0)$ là điểm lúc chim bói cá vừa tiếp xúc với mặt nước. Tính $T=a+b$.
 }{
 \begin{tikzpicture}[line join=round, line cap=round,scale=1,transform shape, >=stealth]
 \definecolor{columbiablue}{rgb}{0.61, 0.87, 1.0}%màu nước
 \definecolor{arsenic}{rgb}{0.23, 0.27, 0.29}%màu mỏ
 \definecolor{antiquewhite}{rgb}{0.98, 0.92, 0.84}%màu trắng
 \definecolor{cadmiumorange}{rgb}{0.93, 0.53, 0.18}%lông cam
 \definecolor{coolblack}{rgb}{0.0, 0.18, 0.39}%cánh đậm
 \definecolor{brandeisblue}{rgb}{0.0, 0.44, 1.0}%màu xanh đầu
 \definecolor{darkcoral}{rgb}{0.8, 0.36, 0.27}%màu chân
 
 %---------màu vẽ cá
 \definecolor{amber}{rgb}{1.0, 0.49, 0.0}
% \clip (-3,-3.5) rectangle (3.5,3);

 \tikzset{san/.pic={ 
 \path
 (-1.3,-1.5) coordinate (O)
 ($(O)+(-142:2)$) coordinate (y)
 ($(O)+(0:4.7)$) coordinate (x)
 ($(O)+(90:3)$) coordinate (z)
 ($(x)+(y)-(O)$) coordinate (t)
 
 (-.8,-2.2)coordinate (A)
 (1.6,0.8) coordinate (C)
 ($(A)!.13!(C)$) coordinate (B)
 ;
 \fill[columbiablue] (O)--(x)--(t)--(y)--cycle;
 
 \foreach\p/\g/\t in {x/-90/y, y/-90/x, z/0/z}
 {
 \node at (\p) [shift=(\g:2mm)] {\tiny $\t$};
 }
 
 \foreach\p/\g in {A/180,B/0,C/-50,O/-90}
 %\node at (\p) [shift=(\g:2mm)] {\tiny $\p$};
 {
 \draw[fill=black](\p) circle (.5pt) +(\g:2mm)node{\tiny $\p$};
 }
 
 \draw[->] (O)--(x) ;
 \draw[->] (O)--(y);
 \draw[->] (O)--(z);
 \draw[dashed] (A)--(B);
 \draw (B)--(C);
 %---------nước
 \draw (-1,-1.7)
 ..controls +(-120:.5) and +(-160:.5) ..(0,-2)
 (-.9,-1.9)
 ..controls +(-70:.2) and +(-160:.2) ..(-.2,-1.9)
 (-.95,-1.8)
 ..controls +(70:.5) and +(30:.5) ..(0,-1.9)
 (.1,-1.6)
 ..controls +(-20:.2) and +(30:.2) ..(.2,-1.9)
 (-.7,-1.75)
 ..controls +(-170:.2) and +(-160:.3) ..(-.5,-1.9)
 ;
 }}
 
 \path
 (0,0)pic[scale=1]{san}
 ;
 
 \tikzset{chim_boi_ca/.pic={
 %==============cánh trái
 \draw[fill=coolblack] %(-1,1.1)..controls +(90:.3) and +(170:.3) ..
 (-.55,1.4)
 ..controls +(110:.7) and +(100:.3) ..(-.9,1.8)
 ..controls +(135:.3) and +(120:.3) ..(-1.2,1.8)
 ..controls +(145:.25) and +(110:.25) ..(-1.45,1.7)
 ..controls +(165:.15) and +(85:.15) ..(-1.75,1.63)
 ..controls +(-165:.1) and +(85:.1) ..(-1.9,1.5)
 ..controls +(165:.15) and +(85:.1) ..(-2.1,1.3)
 ..controls +(-165:.1) and +(95:.1) ..(-2.2,1.1)
 ..controls +(-160:.1) and +(95:.1) ..(-2.35,1)--(-1,1.1)
 ;
 %======--------------------
 %Tô lông đầu
 \def\L{
 (2,.84)
 ..controls +(170:.2) and +(25:.3) ..(1,.65)
 ..controls +(-145:.5) and +(40:.6) ..(.3,.4)
 ..controls +(-140:.3) and +(-60:.3) ..(0,.6)
 ..controls +(120:.3) and +(-50:.7) ..(-1,1.1)
 ..controls +(90:.3) and +(170:.3) ..(-.55,1.4)%1
 ..controls +(-40:.2) and +(140:.1) ..(-.25,1.2)
 ..controls +(-70:.2) and +(-160:.35) ..(.15,.85)
 ..controls +(75:.7) and +(135:.8) ..(1.9,1.3)--(2.1,1)--cycle
 ;
 }
 %\draw[red]\L;
 \fill[brandeisblue] \L;
 %==============================
 \draw[fill=antiquewhite] (2,1.05)
 ..controls +(165:.2) and +(-35:.2)..(1.7,1.15)
 ..controls +(145:.2) and +(65:.2)..(1.2,1.15)
 ..controls +(-60:.1) and +(165:.1)..(1.4,1)
 ..controls +(-15:.2) and +(-145:.3)..cycle
 ;
 \draw[fill=arsenic] (1.6,1.2)
 ..controls +(155:.18) and +(55:.15)..(1.23,1.17)
 ..controls +(-75:.2) and +(-95:.2)..cycle
 ;
 \fill (1.44,1.14) circle(1mm);
 
 \fill[cadmiumorange] (2,1.05)
 ..controls +(165:.2) and +(-35:.2)..(1.7,1.15)
 ..controls +(145:.2) and +(95:.3)..cycle
 ;
 \fill[cadmiumorange] (2,.84)
 ..controls +(150:.2) and +(-25:.1) ..(1.7,1)
 ..controls +(155:.2) and +(-50:.3) ..(1.2,1.08)
 ..controls +(130:.2) and +(50:.2) ..(.75,1)
 ..controls +(-130:.2) and +(-10:.2) ..(.21,1)
 ..controls +(-120:.1) and +(70:.1) ..(.15,.84)
 ..controls +(-20:.3) and +(-150:.3) ..(.8,.85)
 ..controls +(-10:.3) and +(150:.5) ..(1.7,.85)
 ..controls +(-10:.1) and +(150:.1) ..cycle
 ;
 %==================
 %viền đen đầu
 \def\X{
 (0,.6)
 ..controls +(120:.3) and +(-50:.7) ..(-1,1.1)
 
 (-.55,1.4)%1
 ..controls +(-40:.2) and +(140:.1) ..(-.25,1.2)
 ..controls +(-70:.2) and +(-160:.35) ..(.15,.85)
 ..controls +(75:.7) and +(135:.8) ..(1.9,1.3)
 ;
 }
 \draw[black]\X;
 %====================
 %Tô mỏ
 \def\N{
 (1.9,1.3)
 ..controls +(-35:.4) and +(140:.3) ..(3.4,.78)
 ..controls +(-175:.2) and +(-10:.3) ..(2,.84)
 ..controls +(150:.2) and +(-25:.1) ..(1.7,1)
 ..controls +(20:.2) and +(175:.1) ..(2,1.05)
 ..controls +(-35:.2) and +(155:.1) ..cycle
 ;
 }
 %\draw[red]\N;
 \fill[arsenic] \N;
 %Mỏ
 \def\M{
 (1.9,1.3)
 ..controls +(-35:.4) and +(140:.3) ..(3.4,.78)
 ..controls +(-175:.2) and +(-10:.3) ..(2,.84)
 ..controls +(170:.2) and +(25:.3) ..(1,.65)
 ;
 }
 \draw[black]\M;
 %==============================
 
 %================Cánh phải
 \draw[fill=coolblack]
 %(-2.6,.95)
 %..controls +(-160:.2) and +(145:.3) ..(-1.6,.5)
 %..controls +(-35:.2) and +(145:.3) ..(-1,-.5)
 %..controls +(-35:.2) and +(-145:.2) ..
 (-.6,-.3)%3
 ..controls +(-130:.5) and +(-10:.2) ..(-1,-.95)
 ..controls +(170:.1) and +(-10:.15) ..(-1.2,-1.02)
 ..controls +(170:.1) and +(-40:.15) ..(-1.45,-.95)
 ..controls +(160:.1) and +(-80:.15) ..(-1.6,-.8)
 ..controls +(140:.1) and +(-70:.15) ..(-1.75,-.65)
 ..controls +(140:.1) and +(-70:.15) ..(-1.9,-.5)
 ..controls +(160:.1) and +(-70:.15) ..(-2.05,-.35)
 ..controls +(150:.1) and +(-70:.15) ..(-2.25,-.2)
 ..controls +(-160:.1) and +(-20:.15) ..(-2.5,-.18)
 ..controls +(160:.1) and +(-30:.15) ..(-2.75,-.16)
 ..controls +(160:.1) and +(-60:.15) ..(-2.95,-.05)
 ..controls +(160:.1) and +(-80:.15) ..(-3.2,.05)
 ..controls +(160:.1) and +(-90:.15) ..(-3.35,.15)
 ..controls +(160:.1) and +(-95:.15) ..(-3.55,.25)
 ..controls +(-160:.2) and +(-145:.2) ..(-3.7,.35)
 ..controls +(-160:.2) and +(-160:.4) ..(-3.7,.53)
 ..controls +(-160:.2) and +(-160:.3) ..(-3.85,.65)
 ..controls +(160:.5) and +(-170:.3) ..(-2.6,.95)
 ..controls +(0:.5) and +(120:.3) ..(-.6,-.3)%3
 ;
 %===================
 %===================
 \fill[brandeisblue]
 (-.8,-1.1)%lông đuôi xanh
 ..controls +(-80:.2) and +(140:.2) ..(-.5,-1.5)
 ..controls +(-40:.2) and +(140:.4) ..(-.3,-2.5)
 ..controls +(135:.7) and +(-85:.6) ..cycle
 ;
 \draw (-.3,-2.5)
 ..controls +(135:.7) and +(-85:.6) ..(-.8,-1.1)%lông đuôi xanh
 ;
 %=============đuôi
 \draw[fill=brandeisblue] (-.45,-2.3)
 ..controls +(-95:.2) and +(95:.2) ..(-.4,-2.8)
 --(-.32,-2.8)--(-.25,-2.2)
 (-.25,-2.2)--(-.32,-2.78)--(-.27,-2.78)--(-.16,-2.2)
 ;
 %================
 %Tô lông cam
 \def\C{
 (2,.84)
 ..controls +(170:.2) and +(25:.3) ..(1,.65)
 ..controls +(-145:.5) and +(40:.6) ..(.3,.4)
 ..controls +(-140:.3) and +(-60:.3) ..(0,.6)
 ..controls +(120:.3) and +(-50:.7) ..(-1,1.1)%2
 ..controls +(130:.2) and +(20:.3) ..(-2.6,.95)
 ..controls +(-160:.2) and +(145:.3) ..(-1.6,.5)
 ..controls +(-35:.2) and +(145:.3) ..(-1,-.5)
 ..controls +(-35:.2) and +(-145:.2) ..(-.6,-.3)%3
 ..controls +(-135:.2) and +(100:.2) ..(-.8,-1.1)%lông đuôi xanh
 ..controls +(-80:.2) and +(140:.2) ..(-.5,-1.5)
 ..controls +(-40:.2) and +(140:.4) ..(-.3,-2.5)%đuôi dưới
 ..controls +(40:.4) and +(-150:.5) ..(.5,-1.1)%chân
 ..controls +(-20:.2) and +(160:.2) ..(.8,-1.15)
 ..controls +(150:.1) and +(-70:.1) ..(.65,-.9)
 ..controls +(110:.2) and +(-95:.8) ..(1.26,.6)
 ..controls +(75:.1) and +(-160:.2) ..cycle
 ;
 }
 
 \fill[cadmiumorange] \C;
 \draw[black]\C;

 \draw[black] (-1,1.1)
 ..controls +(130:.2) and +(20:.3) ..(-2.6,.95)
 
 (-.6,-.3)%3
 ..controls +(-135:.2) and +(100:.2) ..(-.8,-1.1)
 ;
 
 %======================chân
 \draw[fill=darkcoral]
 (.5,-1.1)
 ..controls +(-20:.1) and +(170:.1) ..(1.5,-1.1)%chân
 ..controls +(-10:.1) and +(-10:.3) ..(1.2,-1.2)
 ;
 %móng 2
 \draw (1.47,-1.15)%chân
 ..controls +(-10:.1) and +(120:.1) ..(1.63,-1.25)
 ;
 %---------------------
 \draw[fill=darkcoral]
 (.7,-1.15)
 ..controls +(40:.1) and +(170:.1) ..(1.3,-1.08)%chân
 ..controls +(-10:.1) and +(-10:.1) ..(1.2,-1.2)
 ..controls +(170:.1) and +(30:.1) ..(1,-1.2)
 ;
 %móng 2
 \draw (1.27,-1.15)%chân
 ..controls +(-10:.1) and +(120:.1) ..(1.43,-1.25)
 ;
 %------------------------
 \draw[fill=darkcoral]
 (.25,-1.1)
 ..controls +(-20:.2) and +(-170:.1) ..(.5,-1.1)%chân
 ..controls +(-20:.2) and +(160:.2) ..(.8,-1.15)
 ..controls +(-20:.2) and +(160:.2) ..(1.1,-1.2)%móng 1
 ..controls +(-20:.1) and +(-50:.1) ..(1.03,-1.25)
 ..controls +(130:.05) and +(-10:.05) ..(.8,-1.26)
 ..controls +(170:.05) and +(10:.05) ..(.5,-1.28)
 ..controls +(-150:.1) and +(-160:.15) ..(.3,-1.25)
 ..controls +(160:.1) and +(-80:.1) ..(.1,-1.15)
 ;
 %móng 1
 \draw (1.1,-1.25)%móng 1
 ..controls +(-20:.1) and +(95:.1) ..(1.2,-1.35)
 ;
 }}
 %===========Vẽ cá
 \tikzset{ca/.pic={
 %vây
 \def\V{
 (-.35,.74)
 ..controls +(120:.12) and +(40:.22) ..(-.7,.72)--cycle
 (-.7,.32)
 ..controls +(-170:.1) and +(10:.1) ..(-.95,.3)
 ..controls +(60:.1) and +(-140:.1) ..(-.85,.45)--(-.65,.4)--cycle
 (-.3,.32)
 ..controls +(-170:.1) and +(10:.1) ..(-.45,.1)
 ..controls +(-40:.1) and +(-110:.1) ..(-.1,.37)--cycle
 ;
 }
 \fill[amber] \V;
 \draw\V;
 
 %-----------------
 \def\C{
 (-1.25,.83)
 ..controls +(-45:.2) and +(130:.2) ..(-1,.58)
 ..controls +(35:.2) and +(130:.52) ..(.05,.52)
 ..controls +(-90:.1) and +(-110:.1) ..(.05,.52)--(.04,.44)
 ..controls +(-150:.5) and +(-40:.2) ..(-1,.53)
 ..controls +(-140:.1) and +(40:.1) ..(-1.3,.42)
 ..controls +(60:.2) and +(-55:.2) ..cycle
 ;
 }
 
 \fill[amber] \C;
 \draw\C;
 %-----------------
 \def\Cn{
 (-1,.57)
 ..controls +(35:.1) and +(130:.4) ..(.01,.52)
 ..controls +(-140:.4) and +(-40:.2) ..cycle
 ;
 }
 %\draw[ecru!70!black]\Cn;
 \fill[amber!70] \Cn;
 \draw (-.3,.7)
 ..controls +(-120:.1) and +(120:.2) ..(-.3,.34);
 
 \draw[fill=white] (-.22,.55) circle (.08);
 \draw[fill=black] (-.22,.55) circle (.048);
 
 }}

 \path (-1,-2.5)pic[xscale=-.35,yscale=.35]{ca}
 (1.6,1)pic[xscale=-.15,yscale=.15,rotate=-40]{chim_boi_ca}; 
 \end{tikzpicture}
 }
\shortans[]{2{,}8}
 \loigiai{Ta có $A(1{,}5;1;-0{,}5)$ và $C(1;3;2)$ suy ra $\overrightarrow{AC}(-0{,}5;2;2{,}5)$ và $\overrightarrow{AB}=(a-1{,}5;b-1;0{,}5)$.\\
 Vì $A$, $B$, $C$ thẳng hàng nên ta có $\overrightarrow{AB}=k\overrightarrow{AC}$. Suy ra
 \[\heva{&a-1{,}5=k(-0{,}5)\\
 &b-1=2k\\
 &0{,}5=2{,}5k}
 \Leftrightarrow\heva{&k=\dfrac15\\
 &a=1{,}5-\dfrac{0{,}5}{5}=\dfrac75\\
 &b=1+\dfrac{2}{5}=\dfrac75.}
 \]
 Suy ra $B\left(\dfrac75;\dfrac75;0\right)$.\\
 Vậy $T=\dfrac75+\dfrac75=2{,}8$.
}\end{ex}

\begin{ex}%[2-H2B4-SO-13-2425 (Nguồn Đề 13 - Bài 4)]%[VN-MT-7, Lê Văn Hiếu]%[2H2H2-6]
 Một căn phòng dạng hình hộp chữ nhật với chiều dài $8$ m, rộng $6$ m và cao $4$ m có hai chiếc quạt treo tường. Chiếc quạt $A$ treo chính giữa bức tường $8$ m và cách trần $1$ m, chiếc quạt $B$ treo chính giữa bức tường $6$ m và cách trần $1{,}5$ m. (Tham khảo hình vẽ minh họa).
 \begin{center}
 \begin{tikzpicture}[line join=round, line cap=round,scale=.6,transform shape,>=stealth]
 \definecolor{amber}{rgb}{1.0, 0.75, 0.0}%mau non
 \definecolor{antiquebrass}{rgb}{0.8, 0.58, 0.46}%mau da
 \definecolor{antiquewhite}{rgb}{0.98, 0.92, 0.84}%mau ao
 \definecolor{cadmiumgreen}{rgb}{0.0, 0.42, 0.24}%mau quan
 \definecolor{cadetblue}{rgb}{0.37, 0.62, 0.63}%mau but
 \definecolor{brown(traditional)}{rgb}{0.59, 0.29, 0.0}%mau giay
 \definecolor{brilliantlavender}{rgb}{0.96, 0.73, 1.0}%màu sơn tím
 \definecolor{brightube}{rgb}{0.82, 0.62, 0.91}%màu sơn tím đậm
 %---------------màu quạt
 \definecolor{burntorange}{rgb}{0.8, 0.33, 0.0}
 \definecolor{arsenic}{rgb}{0.23, 0.27, 0.29}
 \definecolor{battleshipgrey}{rgb}{0.52, 0.52, 0.51}
 \clip (1,-1) rectangle (16,12);
 %\draw[gray!50] (-3,-3) grid (3,4);
 
 \definecolor{burntsienna}{rgb}{0.91, 0.45, 0.32}
 \tikzset{mai/.pic={
 \def\mainha{
 (.5,2)
 foreach \n in {1,2,...,22} { -- ++ (0,0) -- ++ (0,1) -- ++ (1,0) -- ++ (0,-1) } -- cycle
 
 (.5,1)
 foreach \n in {1,2,...,22} { -- ++ (0,0) -- ++ (0,1) -- ++ (1,0) -- ++ (0,-1) } -- cycle
 
 (.5,0)
 foreach \n in {1,2,...,22} { -- ++ (0,0) -- ++ (0,1) -- ++ (1,0) -- ++ (0,-1) } -- cycle
 
 (.5,-1)
 foreach \n in {1,2,...,22} { -- ++ (0,0) -- ++ (0,1) -- ++ (1,0) -- ++ (0,-1) } -- cycle
 
 (.5,-2)
 foreach \n in {1,2,...,22} { -- ++ (0,0) -- ++ (0,1) -- ++ (1,0) -- ++ (0,-1) } -- cycle
 
 (.5,-3)
 foreach \n in {1,2,...,22} { -- ++ (0,0) -- ++ (0,1) -- ++ (1,0) -- ++ (0,-1) } -- cycle
 
 (.5,-4)
 foreach \n in {1,2,...,22} { -- ++ (0,0) -- ++ (0,1) -- ++ (1,0) -- ++ (0,-1) } -- cycle
 ;
 }
 \clip (0.5,-3)--(17.5,-3)--(17.5,3)--(.5,3)--cycle;
 \draw[white,fill=burntsienna!70] \mainha;
 \draw (17.5,-3)--(17.5,3)--(17.5,9);
 }}
 
 \fill[brilliantlavender] (16,-1)--(16,9)--(12,12)--(12,3)--cycle;
 \fill[brightube] (12,13)--(12,3)--(-18,3)--(-18,13)--cycle;
 \begin{scope}
 \clip[draw] (1,3)--(12,3)--(16,-1)--(5,-1)--cycle;
 \path
 (-2.5,0)pic[scale=1,xslant=-1]{mai}%
 ;
 \end{scope}
 
 \path
 (12,3) coordinate (O)
 (3,3) coordinate (x)
 (12,12) coordinate (z)
 (15,0) coordinate (y)
 ;
 
 \foreach\p/\g in {y/70,x/80, z/-30,O/-120}
 {
 \node at (\p) [shift=(\g:3mm)] {$\p$};
 }
 
 \draw[line width=.5mm,->] (O)--(z) ;
 \draw[line width=.5mm,->] (O)--(x);
 \draw[line width=.5mm,->] (O)--(y); 
 \draw[red,line width=.5mm,<->] ($(O)+(0,4)$)--($(x)+(-2,4)$) node[midway, above]{$8$ m};
 
 %===========================A FAN
 \tikzset{fan/.pic={
 %chân quạt
 \draw[fill=battleshipgrey](-.2,.8)--(.2,.8)--(.4,-2.2)
 ..controls +(-120:.3) and +(-60:.3) ..(-.4,-2.2)--cycle;
 %---Nút bấm
 \foreach \i in{-1.95,-1.7}{%-1.45
 \draw[fill=arsenic](-.15,\i) rectangle (.15,\i+.15);
 }
 %-----------------------------------------------------
 \draw[black](0,.8) circle (2.25cm);
 \draw[black](0,.8) circle (2.15cm);
 
 \draw[black](0,.8) circle (1.42cm);
 \draw[black](0,.8) circle (1.48cm);
 
 \draw[fill=black](0,.8) circle (6mm);
 \draw[fill=arsenic](0,.8) circle (5mm);
 \def\N{
 (0,.8)
 ..controls +(145:1.3) and +(170:1) ..(0,2.8)
 ..controls +(-10:1.4) and +(-20:1) ..(.6,1.76)
 ..controls +(160:.4) and +(100:.4) ..(.2,.9)--cycle
 ;
 }
 \foreach \i/\j/\k in {0/0/0,120/.7/-1.2,240/-.7/-1.2}
 {
 \draw[black,rotate=\i,shift={(\j,\k)}]\N;
 \fill[burntorange,rotate=\i,shift={(\j,\k)}] \N;
 }
 
 %lồng quạt
 \def\r{2.15}
 \foreach \i in {0,15,25,35,...,365}
 \draw[double] ($(\i:\r)+(0,.8)$)--(0,.8);
 
 \draw[fill=arsenic](0,.8) circle (3.5mm);

 }}
 \path
 (6.5,9.5)pic[scale=.6]{fan}
 (14,7.5)pic[scale=.55,yslant=-.3]{fan};
 \end{tikzpicture}
 \end{center}
 Hỏi khoảng cách giữa hai chiếc quạt $A$, $B$ cách nhau bao nhiêu mét (làm tròn đến hàng phần trăm).
 
 \shortans[]{5{,}02}
 \loigiai{
 Chọn hệ trục tọa độ như hình vẽ, khi đó ta có tọa độ quạt $A$ là $A(4;0;3)$ và tọa độ quạt $B$ là $B\left(0;3;\dfrac{5}{2}\right)$.\\
 Khi đó $\overrightarrow{AB}=\left(-4;3;-\dfrac{1}{2}\right)$.\\
 Vậy khoảng cách giữa hai quạt $A$, $B$ là $AB= \sqrt{(-4)^2+3^2+\left(-\dfrac{1}{2}\right)^2} \approx 5{,}02$.}
\end{ex}
\Closesolutionfile{ans}
 
\begin{indapan}
	{ans/ans\currfilebase}
\end{indapan}


% \begin{name}
 {Biên soạn: Bùi Lương Phúc \\ Phản biện: Trần Bảo Hiên}
{Đề ôn tập chương II}
\end{name}


\TN
\Opensolutionfile{ans}[ans/ans\currfilebase-Phan-I]

\begin{ex}%[2-H2B4-SO-14-2425 (Nguồn Đề 5 - Bài 4- Ôn tập chương II)]%[VN-MT-7, Bùi Lương Phúc]%[2H2H1-1]
\immini{Cho hình hộp $ABCD.A'B'C'D'$ (tham khảo hình bên).
Vectơ $\overrightarrow{u}=\overrightarrow{BB'}+\overrightarrow{BA}+\overrightarrow{BC}$ bằng vectơ nào dưới đây?
\choice[2]
{$\overrightarrow{BD}$}
{\True $\overrightarrow{BD'}$}
{$\overrightarrow{BC}$}
{$\overrightarrow{BA'}$}
}
{\begin{tikzpicture}
[scale=0.8, font=\footnotesize, line join=round, line cap=round, >=stealth]
\coordinate (A) at (0,0);
\coordinate (B) at (0.8,1.3);
\coordinate (C) at (4.5,1.3);
\coordinate (D) at ($(A)+(C)-(B)$);
\coordinate (A') at ($(A)+(0.3,3)$);
\coordinate (B') at ($(A')+(B)-(A)$);
\coordinate (C') at ($(B')+(C)-(B)$);
\coordinate (D') at ($(A')+(D)-(A)$);
\draw (A')--(A)--(D)--(D')--(A')--(B')--(C')--(C)--(D) (C')--(D');
\draw[dashed,->](B)--(B');
\draw[dashed,->] (B)--(A);
\draw[dashed,->] (B)--(C);
\foreach \x/\y in {A/-180,B/180,C/0,D/0,A'/180,B'/180,C'/0,D'/0} \fill[black](\x) circle (1pt) ($(\x)+(\y:3mm)$) node{$\x$};
\end{tikzpicture}}
\loigiai{
Ta có $\overrightarrow{u}=\overrightarrow{BB'}+\overrightarrow{BA}+\overrightarrow{BC}=\overrightarrow{BB'}+\left(\overrightarrow{BA}+\overrightarrow{BC}\right)=\overrightarrow{BB'}+\overrightarrow{BD}=\overrightarrow{BD'}$.
}
\end{ex}

\begin{ex}%[2-H2B4-SO-14-2425 (Nguồn Đề 5 - Bài 4- Ôn tập chương II)]%[VN-MT-7, Bùi Lương Phúc]%[2H2H1-3]
\immini{Cho tứ diện $ABCD$ có $AB=AC=AD$ và $\widehat{BAC}=\widehat{BAD}=60^\circ$. Góc giữa hai vectơ $\overrightarrow{AB}$ và $\overrightarrow{CD}$ có số đo bằng
\choice[2]
{$60^\circ$}
{$45^\circ$}
{\True $90^\circ$}
{$120^\circ$}
}
{ \begin{tikzpicture}
[scale=0.8, font=\footnotesize, line join=round, line cap=round, >=stealth]
\coordinate (A) at (0.4,4);
\coordinate (B) at (-1.5,1.5);
\coordinate (D) at (3,1.5);
\coordinate (C) at (0,0);
\draw (A)--(B)--(C)--(D)--cycle (A)--(C);
\draw pic[draw,angle radius=5mm] {angle = B--A--C};
\draw[thick] pic[draw,angle radius=3.5mm] {angle = B--A--D};
\draw[dashed](B)--(D);
\foreach \x/\y in {A/90,B/-120,C/-90,D/-90} \fill[black](\x) circle (1pt) ($(\x)+(\y:3mm)$) node{$\x$};
\draw(-0.5,2.7)--(-0.6,2.8);
\draw (0.13,2) -- (0.27,2);
\draw (1.65,2.7) -- (1.75,2.8);
\end{tikzpicture}}
\loigiai{
\begin{center}
 \begin{tikzpicture}
 [scale=0.8, font=\footnotesize, line join=round, line cap=round, >=stealth]
 \coordinate (A) at (0.4,4);
 \coordinate (B) at (-1.5,1.5);
 \coordinate (D) at (3,1.5);
 \coordinate (C) at (0,0);
 \draw (A)--(B)--(C)--(D)--cycle (A)--(C);
 \draw pic[draw,angle radius=5mm] {angle = B--A--C};
 \draw[thick] pic[draw,angle radius=3.5mm] {angle = B--A--D};
 \coordinate (m) at ($(A)+(0.3,-0.8)$);
 \coordinate (n) at ($(A)+(-0.4,-1)$);
 \node [shift=(m)]{$60^\circ$};
 \node [shift=(n)]{$60^\circ$};
 \draw[dashed](B)--(D);
 \foreach \x/\y in {A/90,B/-120,C/-90,D/-90} \fill[black](\x) circle (1pt) ($(\x)+(\y:3mm)$) node{$\x$};
 \draw(-0.5,2.7)--(-0.6,2.8);
 \draw (0.13,2) -- (0.27,2);
 \draw (1.65,2.7) -- (1.75,2.8);
 \end{tikzpicture}
\end{center}
Ta có
\begin{align*}
 \overrightarrow{AB}\cdot \overrightarrow{CD}=&\, \overrightarrow{AB}\cdot (\overrightarrow{AD}-\overrightarrow{AC})\\
 =&\, \overrightarrow{AB}\cdot \overrightarrow{AD}-\overrightarrow{AB}\cdot \overrightarrow{AC} \\
 =&\, AB\cdot AD\cdot \cos 60{}^\circ -AB\cdot AC\cdot \cos 60^\circ =0 
\end{align*}
Suy ra $\overrightarrow{AB}\perp\overrightarrow{CD}\Rightarrow \left(\overrightarrow{AB},\overrightarrow{CD}\right)=90^\circ$.
}
\end{ex}

\begin{ex}%[2-H2B4-SO-14-2425 (Nguồn Đề 5 - Bài 4- Ôn tập chương II)]%[VN-MT-7, Bùi Lương Phúc]%[2H2N2-2]
Trong KG $Oxyz$, cho điểm $M$ thỏa mãn $\overrightarrow{OM}=2\overrightarrow{i}+3\overrightarrow{j}-\overrightarrow{k}$. Tọa độ của điểm $M$ là
\choice
{$(2;3;1)$}
{$(-2;-3;1)$}
{$(2;-1;3)$}
{\True $(2;3;-1)$}
\loigiai{
Ta có $\overrightarrow{OM}=2\overrightarrow{i}+3\overrightarrow{j}-\overrightarrow{k}\Rightarrow \overrightarrow{OM}=(2;3;-1)\Rightarrow M=(2;3;-1)$.
}
\end{ex}

\begin{ex}%[2-H2B4-SO-14-2425 (Nguồn Đề 5 - Bài 4- Ôn tập chương II)]%[VN-MT-7, Bùi Lương Phúc]%[2H2N2-3]
Trong KG $Oxyz$, cho vectơ $\overrightarrow{u}=2\overrightarrow{i}-\dfrac{1}{2}\overrightarrow{j}+4\overrightarrow{k}$. Tọa độ của vectơ $\overrightarrow{u}$ là
\choice
{\True $\left(2;-\dfrac{1}{2};4 \right)$}
{$\left(2;\dfrac{1}{2};4 \right)$}
{$(2;1;4)$}
{$\left(\dfrac{1}{2};-2;\dfrac{1}{4} \right)$}
\loigiai{
Áp dụng định lí: $\overrightarrow{u}=(a;b;c)\Leftrightarrow \overrightarrow{u}=a\overrightarrow{i}+b\overrightarrow{j}+c\overrightarrow{k}$. 
Ta có $\overrightarrow{u}=2\overrightarrow{i}-\dfrac{1}{2}\overrightarrow{j}+4\overrightarrow{k}=2\overrightarrow{i}+\left(-\dfrac{1}{2}\right)\overrightarrow{j}+4\overrightarrow{k}$, suy ra tọa độ của vectơ $\overrightarrow{u}$ là $\left(2;-\dfrac{1}{2};4\right)$.
}
\end{ex}

\begin{ex}%[2-H2B4-SO-14-2425 (Nguồn Đề 5 - Bài 4- Ôn tập chương II)]%[VN-MT-7, Bùi Lương Phúc]%[2H2H2-5]
Trong KG $Oxyz$, cho hai vectơ $\overrightarrow{u}=(-1;3;-2)$ và $\overrightarrow{v}=(2;5;-1)$. Vectơ nào dưới đây vuông góc với cả hai vectơ $\overrightarrow{u}$ và $\overrightarrow{v}$?
\choice
{${\overrightarrow{w}_4}=(-8;-9;-1)$}
{\True ${\overrightarrow{w}_2}=(7;-5;-15)$}
{${\overrightarrow{w}_1}=(7;5;-15)$}
{${\overrightarrow{w}_3}=(1;8;-3)$}
\loigiai{
Ta có $\left[ \overrightarrow{u},\overrightarrow{v} \right]=\left(\left| \begin{matrix}
3 & -2 \\
5 & -1 \\
\end{matrix} \right|;\left| \begin{matrix}
-2 & -1 \\
-1 & 2 \\
\end{matrix} \right|;\left| \begin{matrix}
-1 & 3 \\
2 & 5 \\
\end{matrix} \right|\right) \Rightarrow \left[\overrightarrow{u},\overrightarrow{v} \right]= (7;-5;-15)$.\\
Vì $\left[\overrightarrow{u},\overrightarrow{v} \right]$ vuông góc với cả hai vectơ $\overrightarrow{u}$ và $\overrightarrow{v}$ nên vectơ $\overrightarrow{w}_2 =(7;-5;-15)$ vuông góc với cả hai vectơ $\overrightarrow{u}$ và $\overrightarrow{v}$.
}
\end{ex}

\begin{ex}%[2-H2B4-SO-14-2425 (Nguồn Đề 5 - Bài 4- Ôn tập chương II)]%[VN-MT-7, Bùi Lương Phúc]%[2H2N2-3]
Trong KG $Oxyz$, cho vectơ $\overrightarrow{a}=(-1;4;2)$. Toạ độ của vectơ $-2\overrightarrow{a}$ là
\choice
{$(-2;8;4)$}
{\True $(2;-8;-4)$}
{$(-2;-8;-4)$}
{$(2;8;4)$}
\loigiai{
Ta có $-2\overrightarrow{a}=\left((-2)\cdot (-1);(-2)\cdot 4;(-2)\cdot 2\right) \Rightarrow -2\overrightarrow{a}=(2;-8;-4)$.
}
\end{ex}

\begin{ex}%[2-H2B4-SO-14-2425 (Nguồn Đề 5 - Bài 4- Ôn tập chương II)]%[VN-MT-7, Bùi Lương Phúc]%[2H2N2-4]
Trong KG $Oxyz$, cho vectơ $\overrightarrow{u}=(2;-1;2)$. Độ dài của vectơ $\overrightarrow{u}$ bằng
\choice
{$\sqrt{7}$}
{\True $3$}
{$9$}
{$2$}
\loigiai{
Độ dài của vectơ $\overrightarrow{u}$ là
$\left| \overrightarrow{u} \right|=\sqrt{2^2+(-1)^2+2^2}=3$.
}
\end{ex}

\begin{ex}%[2-H2B4-SO-14-2425 (Nguồn Đề 5 - Bài 4- Ôn tập chương II)]%[VN-MT-7, Bùi Lương Phúc]%[2H2N2-4]
Trong KG $Oxyz$, cho hai vectơ $\overrightarrow{u}=(-2;1;5)$ và $\overrightarrow{v}=(0;-3;1)$. Tích vô hướng của hai vectơ $\overrightarrow{u}$ và $\overrightarrow{v}$ bằng
\choice
{$10\sqrt{3}$}
{$0$}
{\True $2$}
{$-2$}
\loigiai{
Tích vô hướng của hai vectơ $\overrightarrow{u}$ và $\overrightarrow{v}$ là\\
\[\overrightarrow{u}\cdot \overrightarrow{v}=-2\cdot 0+1\cdot (-3)+5\cdot 1=2.\]
}
\end{ex}

\begin{ex}%[2-H2B4-SO-14-2425 (Nguồn Đề 5 - Bài 4- Ôn tập chương II)]%[VN-MT-7, Bùi Lương Phúc]%[2H2H2-2]
Trong KG $Oxyz$, cho điểm $A(2;-4;3)$ và vectơ $\overrightarrow{u}=(2;2;7)$. Biết rằng $\overrightarrow{u}=\overrightarrow{AB}$, tính tọa độ điểm $B$.
\choice
{$(2;6;4)$}
{$(1;3;2)$}
{\True $(4;-2;10)$}
{$(2;-1;5)$}
\loigiai{
Gọi $B(x;y;z)$, ta có $\overrightarrow{AB}=(x-2;y+4;z-3)$.\\
Vì $\overrightarrow{u}=\overrightarrow{AB}$ nên ta có \begin{center}
$\heva{&x-2=2 \\&y+4=2 \\&z-3=7}\Rightarrow \heva{&x=4 \\&y=-2 \\&z=10}\Rightarrow B(4;-2;10)$.
\end{center}
}
\end{ex}

\begin{ex}%[2-H2B4-SO-14-2425 (Nguồn Đề 5 - Bài 4- Ôn tập chương II)]%[VN-MT-7, Bùi Lương Phúc]%[2H2N2-2]
Trong KG $Oxyz$, cho hai điểm $A(2;3;-4)$ và $B(0;1;6)$. Trung điểm $M$ của đoạn thẳng $AB$ có tọa độ là
\choice
{$(-2;-2;10)$}
{$(1;2;2)$}
{\True $(1;2;1)$}
{$(2;2;-10)$}
\loigiai{
Gọi trung điểm của đoạn thẳng $AB$ là $M(x;y;z)$, ta có
\[x=\dfrac{1+0}{2}=1, 
y=\dfrac{3+1}{2}=2, 
z=\dfrac{-4+6}{2}=1.\]
Vậy $M=(1;2;1)$.
}
\end{ex}

\begin{ex}%[2-H2B4-SO-14-2425 (Nguồn Đề 5 - Bài 4- Ôn tập chương II)]%[VN-MT-7, Bùi Lương Phúc]%[2H2N2-2]
Trong KG $Oxyz$, cho tam giác $MNP$ có $M(1;1;2)$, $N(-1;0;2)$, $P(3;-7;5)$. Trọng tâm $G$ của tam giác $MNP$ có tọa độ là
\choice
{\True $(1;-2;3)$}
{$(-1;2;3)$}
{$(1;2;3)$}
{$(3;-6;7)$}
\loigiai{
Theo công thức tính tọa độ trọng tâm của tam giác ta có\\
\begin{center}
$\heva{& x_G=\dfrac{1-1+3}{3}=1 \\& y_G=\dfrac{1+0-7}{3}=-2 \\& z_G=\dfrac{2+2+5}{3}=3}\Rightarrow G(1;-2;3)$.
\end{center}
}
\end{ex}

\begin{ex}%[2-H2B4-SO-14-2425 (Nguồn Đề 5 - Bài 4- Ôn tập chương II)]%[VN-MT-7, Bùi Lương Phúc]%[2H2H2-2]
Trong KG $Oxyz$, cho hình hộp $ABCD.A'B'C'D'$ có $A(0;0;0)$, $B(3;0;0)$, $D(0;3;0)$ và $D'(0;3;-3)$. Tìm tọa độ đỉnh $A'$ của hình hộp.
\choice
{\True $(0;0;-3)$}
{$(0;3;0)$}
{$(3;3;0)$}
{$(0;0;3)$}
\loigiai{
\begin{center}
\begin{tikzpicture}
[scale=0.8, font=\footnotesize, line join=round, line cap=round, >=stealth]
\coordinate (A) at (0,0);
\coordinate (B) at (0.8,1.3);
\coordinate (C) at (4.5,1.3);
\coordinate (D) at ($(A)+(C)-(B)$);
\coordinate (A') at ($(A)+(0.3,3)$);
\coordinate (B') at ($(A')+(B)-(A)$);
\coordinate (C') at ($(B')+(C)-(B)$);
\coordinate (D') at ($(A')+(D)-(A)$);
\draw (A')--(A)--(D)--(D')--(A')--(B')--(C')--(C)--(D) (C')--(D');
\draw[dashed] (A)--(B)--(B');
\draw[dashed] (B)--(C);
\foreach \x/\y in {A/-180,B/180,C/0,D/0,A'/180,B'/180,C'/0,D'/0} \fill[black](\x) circle (1pt) ($(\x)+(\y:3mm)$) node{$\x$};
\end{tikzpicture}
\end{center}
Ta có $\overrightarrow{AD}=(0;3;0)$.\\
Gọi $A'(x_2;y_2;z_2)\Rightarrow \overrightarrow{A'D'}=(0-x_2;3-y_2;-3-z_2)$.\\
Vì $ADD'A'$ là hình bình hành $\Rightarrow \overrightarrow{A'D'}=\overrightarrow{AD}\Leftrightarrow
\heva{& 0-x_2=0 \\& 3-y_2=3 \\& -3-z_2=0}
\Leftrightarrow \heva{& x_2=0 \\& y_2=0 \\& z_2=-3}
\Rightarrow {A}'(0;0;\,-3)$.
}
\end{ex}
\Closesolutionfile{ans}


\TNTF
\Opensolutionfile{ans}[ans/ans\currfilebase-Phan-II]

\begin{ex}%[2-H2B4-SO-14-2425 (Nguồn Đề 5 - Bài 4- Ôn tập chương II)]%[VN-MT-7, Bùi Lương Phúc]%[2H2V1-3]
\immini{
Cho tứ diện $ABCD$ có $AB$, $AC$ và $AD$ đôi một vuông góc. Gọi $M$ là trung điểm của cạnh $BC$, $H$ là trung điểm của đoạn thẳng $MD$. Cho $AB=AC=a$.
\choiceTF
{\True $\overrightarrow{DM}=\dfrac{1}{2}\overrightarrow{DB}+\dfrac{1}{2}\overrightarrow{DC}$}
{$\overrightarrow{AH}=\dfrac{1}{4}\overrightarrow{AB}+\dfrac{1}{2}\overrightarrow{AC}+\dfrac{1}{2}\overrightarrow{AD}$}
{\True $\overrightarrow{AB}\cdot \overrightarrow{AH}=\dfrac{1}{4}a^2$}
{Góc giữa hai vectơ $\overrightarrow{AH}$ và $\overrightarrow{BC}$ bằng $60^\circ $}
}{\begin{tikzpicture}
[scale=0.7, font=\footnotesize, line join=round, line cap=round, >=stealth]
\coordinate (A) at (0,0);
\coordinate (B) at (-1,-2);
\coordinate (C) at (6,0);
\coordinate (D) at (0,4);
\draw (B)--(D)--(C);
\draw[->] (B)--(C);
\draw[dashed] (B)--(A)--(C) (A)--(D);
\coordinate (M) at ($(B)!0.5!(C)$);
\coordinate (H) at ($(M)!0.5!(D)$);
\draw (D)--(M);
\draw[->, dashed] (A)--(H);
\foreach \x/\g in {A/140,B/-90,C/0,D/90,M/-90,H/0} \fill[black](\x) circle (1pt) ($(\x)+(\g:5mm)$) node{$\x$};
\end{tikzpicture}}
\loigiai{
\begin{itemchoice}
\itemch \textbf{Đúng}.\\
Vì $M$ là trung điểm của cạnh $BC$ nên \[\overrightarrow{DM}=\dfrac{1}{2}\left(\overrightarrow{DB}+\overrightarrow{DC} \right)=\dfrac{1}{2}\overrightarrow{DB}+\dfrac{1}{2}\overrightarrow{DC}.\]
\itemch \textbf{Sai}.\\
Ta có
$\overrightarrow{AH}
=\dfrac{1}{2}\left(\overrightarrow{AM}+\overrightarrow{AD} \right), \overrightarrow{AM}
=\dfrac{1}{2}\left(\overrightarrow{AB}+\overrightarrow{AC} \right)$
\begin{align*}
\Rightarrow \overrightarrow{AH}
=&\, \dfrac{1}{2}\left[ \dfrac{1}{2}\left(\overrightarrow{AB}+\overrightarrow{AC} \right)+\overrightarrow{AD} \right]\\
=&\, \dfrac{1}{2}\cdot \dfrac{1}{2}\left(\overrightarrow{AB}+\overrightarrow{AC}\right)+\dfrac{1}{2}\overrightarrow{AD}\\
=&\, \dfrac{1}{4}\overrightarrow{AB}+\dfrac{1}{4}\overrightarrow{AC}+\dfrac{1}{2}\overrightarrow{AD}
\end{align*}
Rõ ràng $\overrightarrow{AH}=\dfrac{1}{4}\overrightarrow{AB}+\dfrac{1}{2}\overrightarrow{AC}+\dfrac{1}{2}\overrightarrow{AD}$ sai vì $\overrightarrow{AC} \ne\overrightarrow{0}$.
\itemch \textbf{Đúng}.\\
Ta có
\begin{align*}
\overrightarrow{AB}\cdot \overrightarrow{AH}
=&\, \overrightarrow{AB}\cdot \left[ \dfrac{1}{4}\overrightarrow{AB}+\dfrac{1}{4}\overrightarrow{AC}+\dfrac{1}{2}\overrightarrow{AD} \right]\\
=&\, \overrightarrow{AB}\cdot \left(\dfrac{1}{4}\overrightarrow{AB} \right)+\overrightarrow{AB}\cdot \left(\dfrac{1}{4}\overrightarrow{AC} \right)+\overrightarrow{AB}\cdot \left(\dfrac{1}{2}\overrightarrow{AD} \right)\\ =&\, \dfrac{1}{4}{{\overrightarrow{AB}}^2}+\dfrac{1}{4}\overrightarrow{AB}\cdot \overrightarrow{AC}+\dfrac{1}{2}\overrightarrow{AB}\cdot \overrightarrow{AD}.
\end{align*}
Vì $AB\perp AC$ nên $\overrightarrow{AB}\cdot \overrightarrow{AC}=0$, $AB\perp AD$ nên $\overrightarrow{AB}\cdot \overrightarrow{AD}=0$.\\
Vậy $\overrightarrow{AB}\cdot \overrightarrow{AH}=\dfrac{1}{4}{{\overrightarrow{AB}}^2}=\dfrac{1}{4}{\left| \overrightarrow{AB} \right|^2}=\dfrac{1}{4}a^2$.
\itemch \textbf{Sai}.\\
Ta có $\overrightarrow{BC}=\overrightarrow{AC}-\overrightarrow{AB}$, $\overrightarrow{AH}=\dfrac{1}{4}\overrightarrow{AB}+\dfrac{1}{4}\overrightarrow{AC}+\dfrac{1}{2}\overrightarrow{AD}$.\\
\begin{align*}
\overrightarrow{BC}\cdot \overrightarrow{AH}=&\, \left(\overrightarrow{AC}-\overrightarrow{AB} \right)\cdot \left(\dfrac{1}{4}\overrightarrow{AB}+\dfrac{1}{4}\overrightarrow{AC}+\dfrac{1}{2}\overrightarrow{AD} \right)\\
=&\, \dfrac{1}{4}\overrightarrow{AC}\cdot \overrightarrow{AB}+\dfrac{1}{4}{\overrightarrow{AC}}^2+\dfrac{1}{2}\overrightarrow{AC}\cdot \overrightarrow{AD}-\dfrac{1}{4}{\overrightarrow{AB}}^2-\dfrac{1}{4}\overrightarrow{AB}\cdot \overrightarrow{AC}-\dfrac{1}{2}\overrightarrow{AB}\cdot \overrightarrow{AD} \\
=&\, 0+\dfrac{1}{4}\left| \overrightarrow{AC} \right|^2+0-\dfrac{1}{4}\left| \overrightarrow{AB} \right|^2-0-0\\
=&\, \dfrac{1}{4}a^2-\dfrac{1}{4}a^2=0.
\end{align*} 
Vậy góc giữa hai vectơ $\overrightarrow{AH}$ và $\overrightarrow{BC}$ bằng $90^\circ $.
\end{itemchoice}
}
\end{ex}

\begin{ex}%[2-H2B4-SO-14-2425 (Nguồn Đề 5 - Bài 4- Ôn tập chương II)]%[VN-MT-7, Bùi Lương Phúc]%[2H2H2-3]
Trong KG $Oxyz$, cho hai điểm $M(-4;3;-1)$ và $N(2;-1;-3)$.
\choiceTF
{\True Tọa độ vectơ $\overrightarrow{OM}$ bằng $(-4;3;-1)$}
{\True Điểm đối xứng của $M$ qua trục cao có tọa độ là $(4;-3;-1)$}
{Gọi $G$ là trọng tâm của tam giác $OMN$. \\
Hình chiếu của điểm $G$ trên mặt phẳng $\left(Oxy \right)$ có tọa độ là $\left(0;0;-\dfrac{4}{3} \right)$}
{\True Nếu $\overrightarrow{v}=3\overrightarrow{i}-2\overrightarrow{j}-\overrightarrow{k}$ thì hai vectơ $\overrightarrow{MN}$ và $\overrightarrow{v}$ cùng hướng}
\loigiai{
\begin{itemchoice}

\itemch \textbf{Đúng}.\\
Tọa độ vectơ $\overrightarrow{OM}$ cũng là tọa độ điểm $M(-4;3;-1)$.
\itemch \textbf{Đúng}.\\
Điểm đối xứng của $M$ qua trục cao có tọa độ là $(4;-3;-1)$.
\itemch \textbf{Sai}.\\
Ta có trọng tâm tam giác $OMN$ là $G\left(-\dfrac{2}{3};\dfrac{2}{3};-\dfrac{4}{3} \right)$.\\
Hình chiếu của $G$ trên $\left(Oxy \right)$ có tọa độ là $\left(-\dfrac{2}{3};\dfrac{2}{3};0 \right)$.
\itemch \textbf{Đúng}.\\
$\overrightarrow{MN}=(6;-4;-2)$.\\
Vì $\overrightarrow{v}(3;-2;-1)$ nên $\overrightarrow{MN}=2\overrightarrow{v}$.\\
Vậy $\overrightarrow{MN}$ và $\overrightarrow{v}$ cùng hướng.
\end{itemchoice}
}
\end{ex}

\begin{ex}%[2-H2B4-SO-14-2425 (Nguồn Đề 5 - Bài 4- Ôn tập chương II)]%[VN-MT-7, Bùi Lương Phúc]%[2H2H2-4]
Trong KG $Oxyz$, cho vectơ $\overrightarrow{u}=(1;2;3)$ và điểm $A(2;7;4)$.
\choiceTF
{\True Hình chiếu của điểm $A$ trên trục tung có tọa độ là $(0;7;0)$}
{$\left| 2\overrightarrow{u}-\overrightarrow{j} \right|=8$}
{\True Một điểm $B$ nằm trên mặt phẳng $\left(Oxy \right)$ sao cho $\overrightarrow{AB}$ cùng phương với $\overrightarrow{u}=(1;2;3)$. Khi đó điểm $B$ có tọa độ là $\left(\dfrac{2}{3};\dfrac{13}{3};0 \right)$}
{$\sin^2\left(\overrightarrow{u},\overrightarrow{i} \right) +\sin^2\left(\overrightarrow{u},\overrightarrow{j} \right)+\sin ^2\left(\overrightarrow{u},\overrightarrow{k} \right)=1$}
\loigiai{
\begin{itemchoice}
\itemch \textbf{Đúng}.\\
Hình chiếu của điểm $A(2;7;4)$ trên trục tung có tọa độ là $(0;7;0)$.
\itemch \textbf{Sai}.\\
Ta có $2\overrightarrow{u}=(2;4;6);\overrightarrow{j}=(0;1;0)\Rightarrow 2\overrightarrow{u}-\overrightarrow{j}=(2;3;6)$.\\
Suy ra $\left| 2\overrightarrow{u}-\overrightarrow{j} \right|=\sqrt{2^2+3^2+6^2}=7$.
\itemch \textbf{Đúng}.\\
Vì điểm $B$ nằm trên mặt phẳng $\left(Oxy \right)$ nên ta gọi $B(x;y;0)$\\
$\Rightarrow \overrightarrow{AB}=(x-2;y-7;-4)$.\\
Vì $\overrightarrow{AB}$ cùng phương với $\overrightarrow{u}=(1;2;3)$ nên có số thực $k$ sao cho $\overrightarrow{AB}=k\overrightarrow{u}$
hay
\begin{align*}
\heva{& x-2=k \\& y-7=2k \\& -4=3k} \Leftrightarrow \heva{& k=-\dfrac{4}{3} \\& x=\dfrac{2}{3} \\& y=\dfrac{13}{3}.}
\end{align*}
Vậy $B\left(\dfrac{2}{3};\dfrac{13}{3};0 \right)$.
\itemch \textbf{Sai}.\\
Ta có
$\cos \left(\overrightarrow{u},\overrightarrow{i} \right)=\dfrac{\overrightarrow{u}\cdot \overrightarrow{i}}{\left| \overrightarrow{u} \right|\cdot\left| \overrightarrow{i} \right|}=\dfrac{1\cdot 1+2\cdot 0+3\cdot0}{\sqrt{1^2+2^2+3^2}\cdot 1}=\dfrac{1}{\sqrt{14}}$;\\
Tương tự, $\cos \left(\overrightarrow{u},\overrightarrow{j} \right)=\dfrac{2}{\sqrt{14}}$; $\cos \left(\overrightarrow{u},\overrightarrow{k} \right)=\dfrac{3}{\sqrt{14}}$\\
Vậy 
\begin{align*}
&\sin ^2\left(\overrightarrow{u},\overrightarrow{i} \right)+\sin ^2\left(\overrightarrow{u},\overrightarrow{j} \right)+\sin^2\left(\overrightarrow{u},\overrightarrow{k} \right)\\
=&\, 3-\left[ \cos^2\left(\overrightarrow{u},\overrightarrow{i} \right)+\cos ^2\left(\overrightarrow{u},\overrightarrow{j} \right)+\cos^2\left(\overrightarrow{u},\overrightarrow{k} \right) \right]\\
=&\, 3-\left(\dfrac{1}{14}+\dfrac{4}{14}+\dfrac{9}{14} \right)\\
=&\, 2.
\end{align*}
\end{itemchoice}
}
\end{ex}

\begin{ex}%[2-H2B4-SO-14-2425 (Nguồn Đề 5 - Bài 4- Ôn tập chương II)]%[VN-MT-7, Bùi Lương Phúc]%[2H2H2-4]
Trong KG $Oxyz$, cho bốn điểm $A(0;-2;1)$, $B(1;0;-2)$, $C(3;1;-2)$ và $D(-2;-2;-1)$.
\choiceTF
{Ba điểm $A$, $B$, $D$ thẳng hàng}
{\True Tam giác $ACD$ là tam giác vuông tại $A$}
{\True Góc giữa hai vectơ $\overrightarrow{AB}$ và $\overrightarrow{CD}$ là góc tù}
{Khoảng cách từ điểm $A$ đến đường thẳng $CD$ bằng $\dfrac{3\sqrt{210}}{35}$}
\loigiai{
\begin{itemchoice}
\itemch \textbf{Sai}.\\
Ta có $\overrightarrow{AB}=(1;2;-3)$; $\overrightarrow{AD}=(-2;0;-2)$.\\
Vì $\dfrac{1}{-2}\ne \dfrac{-3}{-2}$ nên
hai vectơ $\overrightarrow{AB}$ và $\overrightarrow{AD}$ không cùng phương. \\
Suy ra ba điểm $A$, $B$, $D$ không thẳng hàng.
\itemch \textbf{Đúng}.\\
Ta có $\overrightarrow{AC}=(3;3;-3)$, $\overrightarrow{AD}=(-2;0;-2)$.\\
$\overrightarrow{AC}\cdot\overrightarrow{AD}=3\cdot (-2)+3\cdot 0+(-3) \cdot (-2)=0\Rightarrow AC\perp AD$.\\
Suy ra tam giác $ACD$ là tam giác vuông tại $A$.
\itemch \textbf{Đúng}.\\
Ta có $\overrightarrow{AB}=(1;2;-3)$, $\overrightarrow{CD}=(-5;-3;1)$.\\
Vì $\overrightarrow{AB}\cdot \overrightarrow{CD}=1\cdot (-5)+2\cdot (-3)+(-3) \cdot 1=-14<0$ nên $\cos\left(\overrightarrow{AB},\overrightarrow{CD} \right)<0\Rightarrow \left(\overrightarrow{AB},\overrightarrow{CD} \right)$ là góc tù.
\itemch \textbf{Sai}.\\
Ta có $AC=\sqrt{3^2+3^2+(-3)^2}=3\sqrt{3}$;\\ $AD=\sqrt{(-2)^2+0^2+2^2}=2\sqrt{2}$;\\
$CD=\sqrt{(-5)^2+(-3)^2+1^2}=\sqrt{35}$.\\
Tam giác $ACD$ là tam giác vuông tại $A$ nên
$S_{ACD}=\dfrac{1}{2}AC\cdot AD=3\sqrt{6}$.\\
Khoảng cách từ điểm $A$ đến đường thẳng $CD$ chính là chiều cao kẻ từ $A$ của tam giác $ACD$.\\
\[\mathrm{d}(A,CD)=\dfrac{2S_{ACD}}{CD}=\dfrac{6\sqrt{6}}{\sqrt{35}}=\dfrac{6\sqrt{210}}{35}.\]
\end{itemchoice}
}
\end{ex}
\Closesolutionfile{ans}


\TNSA
\Opensolutionfile{ans}[ans/ans\currfilebase-Phan-III]

\begin{ex}%[2-H2B4-SO-14-2425 (Nguồn Đề 5 - Bài 4- Ôn tập chương II)]%[VN-MT-7, Bùi Lương Phúc]%[2H2V2-4]
Cho tứ diện $ABCD$ có $AB=AC=AD=a$, $\widehat{BAC}=\widehat{BAD}=60^\circ $ và $\widehat{CAD}=90^\circ $. Gọi $I$ là điểm trên cạnh $AB$ sao cho $AI=3IB$ và $J$ là trung điểm của $CD$. Gọi $\alpha$ là số đo của góc giữa hai vectơ $\overrightarrow{AB}$ và $\overrightarrow{IJ}$, hãy tính $\cos \alpha$ (kết quả làm tròn đến hàng phần mười).
\shortans[4]{-0{,}4}

\loigiai{
\begin{center}
\begin{tikzpicture}[line join = round, line cap=round,>=stealth,font=\footnotesize,scale=1]
\coordinate (0,0) at (A);
\coordinate (B) at (-1.5,-3);
\coordinate (C) at (-0.5,-4);
\coordinate (D) at (3,-3);
\coordinate (I) at ($(A)!0.75!(B)$);
\coordinate (J) at ($(C)!0.5!(D)$);
\fill (A) circle (1pt) node[above] {$A$};
\fill (B) circle (1pt) node[left] {$B$};
\fill (C) circle (1pt) node[below] {$C$};
\fill (D) circle (1pt) node[right] {$D$};
\fill (I) circle (1pt) node[left] {$I$};
\fill (J) circle (1pt) node[below] {$J$};
\draw (A) -- (B)--(C)--(D)--(A)--(C);
\draw[dashed] (B) -- (D) (I)--(J);
\draw (0.5,-3.6)--(0.5,-3.85) (2,-3.15)--(2,-3.4);
\end{tikzpicture}
\end{center}
Ta có 
$\overrightarrow{IJ}=\overrightarrow{IA}+\overrightarrow{AJ}=-\dfrac{3}{4}\overrightarrow{AB}+\dfrac{1}{2}\left(\overrightarrow{AC}+\overrightarrow{AD} \right)=-\dfrac{3}{4}\overrightarrow{AB}+\dfrac{1}{2}\overrightarrow{AC}+\dfrac{1}{2}\overrightarrow{AD}$ nên
\[\overrightarrow{IJ}\cdot \overrightarrow{AB}=\left(-\dfrac{3}{4}\overrightarrow{AB}+\dfrac{1}{2}\overrightarrow{AC}+\dfrac{1}{2}\overrightarrow{AD} \right)\cdot \overrightarrow{AB}=\dfrac{1}{2}\left(\overrightarrow{AC}\cdot \overrightarrow{AB}+\overrightarrow{AD}\cdot \overrightarrow{AB}-\dfrac{3}{2}{\overrightarrow{AB}}^2 \right).\]
Lại có $\overrightarrow{AB}\cdot \overrightarrow{AD}=AB\cdot AD\cdot \cos60^\circ =\dfrac{a^2}{2}$; 
$\overrightarrow{AC}\cdot \overrightarrow{AB}=AC\cdot AB\cdot \cos60^\circ =\dfrac{a^2}{2}$; ${\overrightarrow{AB}}^2=AB^2=a^2$.\\
Suy ra, $\overrightarrow{IJ}\cdot \overrightarrow{AB}=\dfrac{1}{2}\left(\dfrac{a^2}{2}+\dfrac{a^2}{2}-\dfrac{3}{2}a^2 \right)=-\dfrac{a^2}{4}$. \\
Ta có $\widehat{CAD}=90^\circ \Rightarrow \overrightarrow{AC}\cdot \overrightarrow{AD}=0$.
\begin{align*}
IJ^2=&\, {\overrightarrow{IJ}}^2=\dfrac{1}{4}\left(\overrightarrow{AC}+\overrightarrow{AD}-\dfrac{3}{2}\overrightarrow{AB} \right)^2 \\
=&\, \dfrac{1}{4}\left({\overrightarrow{AB}}^2+{\overrightarrow{AC}}^2+\dfrac{9}{4}{\overrightarrow{AB}}^2+2\overrightarrow{AC}\cdot \overrightarrow{AD}-3\overrightarrow{AC}\cdot \overrightarrow{AB}-3\overrightarrow{AB}\cdot \overrightarrow{AD} \right) \\
=&\, \dfrac{1}{4}\left(a^2+a^2+\dfrac{9}{4}a^2+2\cdot 0-3\cdot \dfrac{1}{2}a^2-3\cdot \dfrac{1}{2}a^2 \right)=\dfrac{5a^2}{16}.
\end{align*}
$\Rightarrow IJ=\dfrac{a\sqrt{5}}{4}$.\\
Vậy $\cos \alpha=\dfrac{\overrightarrow{AB}\cdot \overrightarrow{IJ}}{AB\cdot IJ}=\dfrac{-\dfrac{a^2}{4}}{\dfrac{a\sqrt{5}}{4}\cdot a}=-\dfrac{\sqrt{5}}{5}\approx -0{,}4$.
}
\end{ex}

\begin{ex}%[2-H2B4-SO-14-2425 (Nguồn Đề 5 - Bài 4- Ôn tập chương II)]%[VN-MT-7, Bùi Lương Phúc]%[2H2H2-4]
\immini{Trong không gian, xét hệ tọa độ $Oxyz$ có gốc $O$ trùng với vị trí của một giàn khoan trên biển, mặt phẳng $\left(Oxy \right)$ trùng với mặt biển (được coi là phẳng) với trục $Ox$ hướng về phía tây, trục $Oy$ hướng về phía nam và trục $Oz$ hướng thẳng đứng lên trời. Đơn vị đo Trong KG $Oxyz$ lấy theo kilômét. Một chiếc ra đa đặt tại giàn khoan và một chiếc tàu thám hiểm có tọa độ là $(30;25;-15)$ (tham khảo hình bên).\\
Tính khoảng cách theo đơn vị kilômét từ chiếc ra đa đến chiếc tàu thám hiểm (kết quả làm tròn đến hàng phần mười).}{
\begin{tikzpicture}[line join = round, line cap=round,>=stealth,font=\footnotesize,scale=0.7]
\definecolor{xanhdatroi}{RGB}{0,127,255}
\tikzset{icon-cot/.pic={
\path[black!80]
(0,0)coordinate (A)
($(A)+(0,8)$)coordinate (B)
($(B)+(-2,-1)$)coordinate (C)
($(A)+(C)-(B)$)coordinate (D)
($(C)!(A)!(D)$) coordinate (A1)
($2*(A1)-(A)+(0,1)$)coordinate (E)
($(B)+(E)-(A)$) coordinate (F)
;
%%%%%%%%%%%%%%%%%%%%%%%%%%%%%%%%%%%%
\draw (A)--(B)--(C)--(D)--cycle
(D)--(E)--(F)--(C);
\foreach \x/\y in {0/0.1,0.1/0.2,0.2/0.3,0.3/0.4,0.4/0.5,0.5/0.6,0.6/0.7,0.7/0.8,0.8/0.9,0.9/1}
\draw
($(A)!\x!(B)$)--($(D)!\y!(C)$)
($(D)!\x!(C)$)--($(A)!\y!(B)$)
($(E)!\x!(F)$)--($(D)!\y!(C)$)
($(D)!\x!(C)$)--($(E)!\y!(F)$)
($(A)!\x!(B)$)--($(E)!\y!(F)$)
($(E)!\x!(F)$)--($(A)!\y!(B)$)
(B)--(F) (A)--(E);
}}
%%%%%%%%%%%%%%%%%%%%%%%%%%%%%%%%%%%%
\fill[bottom color=blue!80!green!15!black!60, top color=blue!20!green!20, middle color=blue!80!green]
(-5,-3.5) rectangle (6,2);
\fill[inner color=xanhdatroi, outer color=xanhdatroi!50]
(-5,4) rectangle (6,2);

%%%%%%%%%%%%%%%%%%%%%%%%%%%%%%%%
%De
\draw[fill=orange!10,scale=1,yshift=-1.25cm,xshift=-0.3cm,opacity=0.2,decorate, decoration={snake, amplitude=0.4mm, segment length=2mm},scale=1.1]
(-1.5,0.5)--(0.5,-0.5)--(4.5,-0.5) --($(-1.5,0.5)+(4.5,-1)-(0.5,-1)$)--cycle
;
%%%%%%%%%%%%%%%%%%%%%%%%%%%%%%%%
%Chande
\path[black!80]
(0.75,-1.2) pic[scale=0.15]{icon-cot}
(0.75,-1.5) pic[scale=0.15]{icon-cot}
(1.8,-1.2) pic[scale=0.15]{icon-cot}
(1.8,-1.5) pic[scale=0.15]{icon-cot}
(2.8,-1.2) pic[scale=0.15]{icon-cot}
(2.8,-1.5) pic[scale=0.15]{icon-cot}
;
%%%%%%%%%%%%%%%%%%%%%%%%%%%%%%%%
%San
\draw[fill=orange!80!gray!60,scale=1]
(-1.5,0.5)--(0.5,-0.5)--(4.5,-0.5) --($(-1.5,0.5)+(4.5,-1)-(0.5,-1)$)--cycle
;
%%%%%%%%%%%%%%%%%%%%%%%%%%%%%%%%%%%%
%Giankhoan
\path[black!80]
(0.75,0) pic[scale=0.15]{icon-cot}
(0.75,0.2) pic[scale=0.15]{icon-cot}
(1.8,0) pic[scale=0.15]{icon-cot}
(1.8,0.2) pic[scale=0.15]{icon-cot}
(2.8,0) pic[scale=0.15]{icon-cot}
(2.8,0.2) pic[scale=0.15]{icon-cot}
;
%%%%%%%%%%%%%%%%%%%%%%%%%%%%%%%%%%%%
%Hetoado Oxyz
\begin{scope}
\draw[->,line width=1pt,red] (0,0)--(4,-2) node [below]{y};
\draw[->,line width=1pt,red] (0,0)--(-4,0) node [below]{x};
\draw[->,line width=1pt,red] (0,0)--(0,3.5) node [right]{z};
\fill (0,0) circle(1pt) node [above left, red]{O};
\end{scope}
\end{tikzpicture}
}
\shortans[4]{41{,}8}
\loigiai{
Theo đề bài ta có tọa độ của ra đa là $(0;0;0)$, tọa độ của tàu thám hiểm là $(30;25;-15)$.\\
Khi đó khoảng cách giữa ra đa và tàu thám hiểm là
\[\mathrm{d}=\sqrt{(30-0)^2+(25-0)^2+(-15-0)^2}=5\sqrt{70}\approx 41{,}8.\]
Vậy khoảng khoảng cách giữa ra đa và tàu thám hiểm là $41{,}8$\,(km).
}
\end{ex}

\begin{ex}%[2-H2B4-SO-14-2425 (Nguồn Đề 5 - Bài 4- Ôn tập chương II)]%[VN-MT-7, Bùi Lương Phúc]%[2H2V2-5]
Trong KG $Oxyz$, cho tứ diện $ABCD$ có $A(0;1;-1)$, $B(1;1;2)$, $C(1;-1;0)$, $D(0;0;1)$. Biết rằng có một vectơ $\overrightarrow{v}=(a;b;2)$ vuông góc với cả hai vectơ $\overrightarrow{BC}$ và $\overrightarrow{BD}$. Tính $3a+b$.
\shortans[4]{-2}
\loigiai{
Ta có $\overrightarrow{BC}=(0;-2;-2)$, $\overrightarrow{BD}=(-1;-1;-1)$.
\begin{align*}
\left[ \overrightarrow{BC},\overrightarrow{BD} \right]=\left(\left| \begin{matrix}
-2 & -2 \\
-1 & -1 \\
\end{matrix} \right|;\left| \begin{matrix}
-2 & 0 \\
-1 & -1 \\
\end{matrix} \right|;\left| \begin{matrix}
0 & -2 \\
-1 & -1 \\
\end{matrix} \right| \right) \Rightarrow \left[ \overrightarrow{BC},\overrightarrow{BD} \right]=(0;2;-2).
\end{align*}
Khi đó $\left[ \overrightarrow{BC},\overrightarrow{BD} \right]=(0;2;-2)$ là một vectơ vuông góc với cả hai vectơ $\overrightarrow{BC}$, $\overrightarrow{BD}$. \\
Ta có $\left[ \overrightarrow{BC},\overrightarrow{BD} \right]$ và $\overrightarrow{v}=(a;b;2)$ cùng phương nên có số thực $k$ để $\overrightarrow{v}= k\cdot \left[ \overrightarrow{BC},\overrightarrow{BD} \right]$.\\
Suy ra
$\heva{&a= k\cdot 0 \\&b=k\cdot 2 \\&2=k \cdot(-2)}$. \\
Giải ra ta được $a=0$, $b=-2$.\\
Vậy $3a+b=-2$.
}
\end{ex}

\begin{ex}%[2-H2B4-SO-14-2425 (Nguồn Đề 5 - Bài 4- Ôn tập chương II)]%[VN-MT-7, Bùi Lương Phúc]%[2H2V2-4]
Trong không gian với một hệ trục toạ độ cho trước (đơn vị đo lấy theo kilômét), ra đa phát hiện một chiếc máy bay di chuyển với vận tốc và hướng không đổi từ điểm $A(800;500;7)$ đến điểm $B(940;550;9)$ trong $10$ phút. Nếu máy bay tiếp tục giữ nguyên vận tốc và hướng bay thì toạ độ của máy bay sau $5$ phút tiếp theo là $C(x;y;z)$. Tính $x+y+z$.
\shortans[4]{1595}
\loigiai{
Vị trí của máy bay sau $5$ phút tiếp theo là $C(x;y;z)$.\\
Vì hướng của máy bay không đổi nên $\overrightarrow{AB}$ và $\overrightarrow{BC}$ cùng hướng.\\
Do vận tốc của máy bay không đổi và thời gian bay từ $A$ đến $B$ gấp đôi thời gian bay từ $B$ đến $C$ nên $AB=2BC$.\\
Do đó $\overrightarrow{BC}=\dfrac{1}{2}\overrightarrow{AB}$.\\
Mặt khác, $\dfrac{1}{2}\overrightarrow{AB}=\left(\dfrac{940-800}{2};\dfrac{550-500}{2};\dfrac{9-7}{2} \right) \Rightarrow \dfrac{1}{2}\overrightarrow{AB}=(70;25;1)$;\\
Mà $\overrightarrow{BC}=(x-940 ; y-550 ; z-9)$
nên
$\heva{&x-940=70 \\&y-550=25 \\&z-9=1} \Rightarrow \heva{&x=1010 \\&y=575 \\&z=10} \Rightarrow x+y+z=1\,595$.
}
\end{ex}

\begin{ex}%[2-H2B4-SO-14-2425 (Nguồn Đề 5 - Bài 4- Ôn tập chương II)]%[VN-MT-7, Bùi Lương Phúc]%[2H2V2-4]
\immini{Một tấm gỗ tròn được treo song song với mặt phẳng nằm ngang bởi ba sợi dây không dãn xuất phát từ điểm $O$ trên trần nhà và lần lượt buộc vào ba điểm $A$, $B$, $C$ trên tấm gỗ tròn sao cho các lực căng $\overrightarrow{F}_1$, $\overrightarrow{F}_2$, $\overrightarrow{F}_3$ lần lượt trên mỗi dây $OA$, $OB$, $OC$ đôi một vuông góc với nhau và có độ lớn $\left| \overrightarrow{F}_1 \right|=\left| \overrightarrow{F}_2 \right|=\left| \overrightarrow{F}_3 \right|=10$\,N (xem hình vẽ).\\
Tính trọng lượng $P$ của tấm gỗ tròn đó (kết quả làm tròn đến hàng phần mười).}{
\begin{tikzpicture}[line join = round, line cap=round,>=stealth,font=\footnotesize,scale=0.9]
\path
(0,-1) coordinate (O)
(0,-4) coordinate (O')
(0,-8) coordinate (O'')
($(100:2cm and 1cm)+(0,-4)$) coordinate (B)
($(220:2cm and 1cm)+(0,-4)$) coordinate (A)
($(-20:2cm and 1cm)+(0,-4)$) coordinate (C) ;
\filldraw[fill=gray!80, draw=gray] (0,-4.4) ellipse (2cm and 1.1cm);
\filldraw[fill=gray!40, draw=gray] (O') ellipse (2cm and 1cm);
\draw[]
(O)--(A)
(O)--(B) (O)--(C)
;
\filldraw[gray!80, line width=1pt]
($($(O')+(2,0)$)+(0,-0.5)$)--($(O')+(2,0)$)
($($(O')-(2,0)$)+(0,-0.45)$)--($(O')-(2,0)$)
;
\draw[->,line width=1pt] (O)--($(O)!1/2!(A)$)node[left]{$\overrightarrow{F}_1$};
\draw[->,line width=1pt] (O)--($(O)!0.5!(B)$)node[right]{$\overrightarrow{F}_2$};
\draw[->,line width=1pt] (O)--($(O)!1/2!(C)$)
node[right]{$\overrightarrow{F}_3$}
;
\draw[->]
($(270:2cm and 1cm)+(0,-4.52)$)--(O'')node[pos=0.2,right]{$\overrightarrow{P}$}
;
\draw[dashed] ($(270:2cm and 1cm)+(0,-3.9)$)--(O')--(C);
\foreach \p/\r in {O'/180,B/-90,O/90,A/-90,C/-100}
\fill (\p) circle (1.2pt) node[shift={(\r:3mm)}]{$\p$};
\filldraw[fill=gray!80, draw=gray] (-2,-1) rectangle (2,-0.9);
\end{tikzpicture}
}
\shortans[4]{17{,}3}
\loigiai{
\begin{center}
\begin{tikzpicture}
[scale=0.9, font=\footnotesize, line join=round, line cap=round, >=stealth]
\coordinate (O) at (0,0);
\coordinate (A_1) at (-1.5,-2);
\coordinate (M) at (0.5,-4);
\coordinate (C_1) at ($(O)+(M)-(A_1)$);
\coordinate (B_1) at ($(O)+(-0.5,-1.7)$);
\coordinate (D_1) at ($(A_1)+(B_1)-(O)$);
\coordinate (N) at ($(B_1)+(C_1)-(O)$);
\coordinate (Q) at ($(D_1)+(M)-(A_1)$);
\draw (M)--(A_1)--(D_1)--(Q)--(N)--(C_1)--(M)--(Q);
\draw[->,line width=1pt] (O)--(A_1) node[midway, above,xshift=-1mm] {$\overrightarrow{F}_1$};;
\draw[->,line width=1pt] (O) -- (C_1) node[midway, above] {$\overrightarrow{F}_3$};
\draw[dashed] (D_1)--(B_1)--(N);
\draw[dashed,->,line width=1pt] (O)--(B_1) node[midway, below, xshift=2mm] {$\overrightarrow{F}_2$};
\draw[dashed,red,->] (O)--(Q);
\foreach \x/\y in {O/90,B_1/-90,C_1/0,D_1/-90,A_1/120,M/0,N/0,Q/-90} \fill[black](\x) circle (1pt) ($(\x)+(\y:4mm)$) node{$\x$};
\end{tikzpicture}
\end{center}
Gọi $A_1$, $B_1$, $C_1$ lần lượt là các điểm sao cho $\overrightarrow{OA}_1=\overrightarrow{F}_1$, $\overrightarrow{OB}_1=\overrightarrow{F}_2$, $\overrightarrow{OC}_1=\overrightarrow{F}_3$.\\
Lấy các điểm $D_1$, $M$, $N$, $Q$ sao cho $OA_1D_1B_1.C_1MQN$ là hình hộp.\\
Theo quy tắc hình hộp ta có $\overrightarrow{OA}_1+\overrightarrow{OB}_1+\overrightarrow{OC}_1=\overrightarrow{OQ}$.\\
Do các lực căng $\overrightarrow{F}_1$, $\overrightarrow{F}_2$, $\overrightarrow{F}_3$ đôi một vuông góc với nhau và có độ lớn $\left| \overrightarrow{F}_1 \right|=\left| \overrightarrow{F}_2 \right|=\left| \overrightarrow{F}_3 \right|=10$\,N nên hình hộp $OA_1D_1B_1.C_1MQN$ có ba cạnh $OA_1$, $OB_1$, $OC_1$ đôi một vuông góc và đều có độ dài bằng $10$.\\
Vì thế $OA_1D_1B_1.C_1MQN$ là hình lập phương có độ dài cạnh bằng $10$.\\
Suy ra độ dài đường chéo bằng $10\sqrt{3}$.\\
Gọi $\overrightarrow{P}$ là trọng lượng tác dụng lên tấm gỗ.\\
Do tấm gỗ ở vị trí cân bằng nên $\overrightarrow{F}_1+\overrightarrow{F}_2+\overrightarrow{F}_3=\overrightarrow{P}$.\\
Suy ra $\left| \overrightarrow{P} \right|=\left| \overrightarrow{OQ} \right|=10\sqrt{3}\approx 17{,}3$.\\
Vậy trọng lượng của tấm gỗ tròn là $P=\left| \overrightarrow{P} \right|=10\sqrt{3}\approx 17{,}3$\,(N).}
\end{ex}

\begin{ex}%[2-H2B4-SO-14-2425 (Nguồn Đề 5 - Bài 4- Ôn tập chương II)]%[VN-MT-7, Bùi Lương Phúc]%[2H2V2-4]
 \immini{Một chậu cây được đặt trên một giá đỡ có bốn chân với điểm đặt $S(0;0;40)$ và các điểm chạm mặt đất của bốn chân lần lượt là $A(40;0;0)$, $B(0;40;0)$, $C(-40;0;0)$, $D(0;-40;0)$ (đơn vị là cm). Cho biết trọng lực tác dụng lên chậu cây có độ lớn $60$\,N và được phân bố thành bốn lực $\overrightarrow{F}_1,\overrightarrow{F}_2,\overrightarrow{F}_3,\overrightarrow{F}_4$ có độ lớn bằng nhau như hình vẽ. \\
 Tính $\left| \overrightarrow{F}_1+\overrightarrow{F}_2+\overrightarrow{F}_3+3\overrightarrow{F}_4 \right|$ (kết quả làm tròn đến hàng đơn vị).}{
 \hspace*{0.5cm}\begin{tikzpicture}[line join = round, line cap=round,>=stealth,font=\footnotesize,scale=1]
 \tikzset{chauhoa/.pic={\fill[orange!80!black!90] (-1,-1.1) -- (1,-1.1) -- (0.8,-1.8) -- (-0.8,-1.8) -- cycle;
 \draw[thick] (-1,-1.1) -- (1,-1.1) -- (0.8,-1.8) -- (-0.8,-1.8) -- cycle;
 \fill[brown] (-0.3,-1) -- (0.3,-1) -- (0.2,0) -- (-0.2,0) -- cycle;
 \draw[brown, very thick] (-0.3,-1) -- (0.3,-1) -- (0.2,-0.2) -- (-0.2,-0.2) -- cycle;
 \fill[brown!60!black!90] (-0.1,-0.7) circle (0.1);
 \fill[brown!60!black!90] (0.1,-0.9) circle (0.1);
 \fill[brown!60!black!90] (-0.2,-0.5) circle (0.1);
 \fill[brown!50!black!80] (0,-0.3) circle (0.1);
 \fill[green] (0,0.3) ellipse (0.6 and 0.3);
 \fill[green] (-0.5,0.3) ellipse (0.6 and 0.3);
 \fill[green] (0.5,0.3) ellipse (0.6 and 0.3);
 \fill[green] (0,-0.2) ellipse (0.6 and 0.3);
 \draw[green, very thick] (0,0.3) to[out=60,in=120] (0.5,1.5);
 \draw[green, very thick] (0,0.3) to[out=120,in=60] (-0.5,1.5);
 \draw[green, very thick] (0,-0.2) to[out=60,in=120] (0.5,1);
 \draw[green, very thick] (0,-0.2) to[out=120,in=60] (-0.5,1);
 \fill[green] (0.5,1.2) ellipse (0.4 and 0.2);
 \fill[green] (-0.5,1.2) ellipse (0.4 and 0.2);
 \fill[green] (0.5,0.8) ellipse (0.4 and 0.2);
 \fill[green] (-0.5,0.8) ellipse (0.4 and 0.2);}}
 \path
 (0,0) coordinate (O)
 (0,4) coordinate (S)
 (-115:2.3cm and 1.2cm) coordinate (A)
 (-10:2.3cm and 1.2cm) coordinate (B)
 ($2*(O)-(A)$) coordinate (C)
 ($2*(O)-(B)$) coordinate (D)
 ($(S)!0.55!(A)$) coordinate(F1)
 ($(S)!0.55!(B)$) coordinate(F2)
 ($(S)!0.55!(C)$) coordinate(F3)
 ($(S)!0.55!(D)$) coordinate(F4);
 \draw[->, shorten >=-0.5cm, shorten <=-0.5cm](C)--(A) node[below left=10pt]{$x$};
 \draw[->, shorten >=-0.5cm, shorten <=-0.5cm](D)--(B) node[below right=5pt]{$y$};
 \foreach \p in {A,B,C,D} {\draw (S)--(\p);}
 \foreach \f/\g [count =\i from 1] in {F1/200,F2/10,F3/-120,F4/160} {\draw[->] (S)--(\f)node [shift={(\g:0.4)}]{$\overrightarrow{F}_\i$};}
 \draw[->] (O)--(90:5.5) node[left]{$z$};
 \foreach \i/\j in {O/-90, A/-70, B/-90, C/-70, D/-90} \fill (\i) node[shift={(\j:0.3)}]{$\i$} circle(1pt);
 \fill (S) node[shift={(210:0.4)}]{$S$} circle(1pt);
 \draw ($(S)+(0,0.7)$) pic[scale=0.4]{chauhoa};
 \end{tikzpicture}}
 \shortans[4]{95}
 \loigiai{
 \begin{center}
 \begin{tikzpicture}[line join = round, line cap=round,>=stealth,font=\footnotesize,scale=1.2]
 \path
 (0,0) coordinate (O)
 (0,4) coordinate (S)
 (-115:2.5cm and 1.2cm) coordinate (A)
 (-10:2.5cm and 1.2cm) coordinate (B)
 ($2*(O)-(A)$) coordinate (C)
 ($2*(O)-(B)$) coordinate (D)
 ($(S)!0.55!(A)$) coordinate(A')
 ($(S)!0.55!(B)$) coordinate(B')
 ($(S)!0.55!(C)$) coordinate(C')
 ($(S)!0.55!(D)$) coordinate(D')
 ($(A')!1/2!(C')$) coordinate(O');
 \draw (S)--(D)--(A)--(B)--(S)--(A) (B')--(A')--(D');
 \draw[dashed] (S)--(C)--(D)--(B)--(C)--(A) (S)--(O) (D')--(C')--(B') (A')--(C') (B')--(D');
 \foreach \i/\j in {O/-70, A/-90, B/-60, C/0, D/200, S/90, A'/200, B'/20, C'/160, D'/160, O'/-50} \fill (\i) node[shift={(\j:0.3)}]{$\i$} circle(1pt);
 \draw[->, red] (S)--(A') node[midway, left=-4pt] {$\overrightarrow{F}_1$};
 \draw[->, red] (S)--(B') node[midway, right] {$\overrightarrow{F}_2$};
 \draw[->, red] (S)--(C') node[midway, shift={(250:0.3)}] {$\overrightarrow{F}_3$};
 \draw[->, red] (S)--(D') node[midway, left] {$\overrightarrow{F}_4$};
 \end{tikzpicture}
 \end{center}
 Tứ giác $ABCD$ có hai đường chéo bằng nhau và vuông góc với nhau tại trung điểm của mỗi đường nên là hình vuông.\\
 Ta có $\overrightarrow{SA}=(40;0;-40)$, $\overrightarrow{SB}=(0;40;-40)$, $\overrightarrow{SC}=(-40;0;-40)$, $\overrightarrow{SD}=(0;-40;-40)$\\
 $\Rightarrow SA=SB=SC=SD=40\sqrt{2}$. \\
 Do đó $S.ABCD$ là hình chóp tứ giác đều.\\
 Các vectơ $\overrightarrow{F}_1$, $\overrightarrow{F}_2$, $\overrightarrow{F}_3$, $\overrightarrow{F}_4$ có điểm đầu tại $S$ và điểm cuối lần lượt là $A'$ ,$B'$, $C'$, $D'$.\\
 Ta có $SA'=SB'=SC'=SD'$ nên $S.A'B'C'D'$ cũng là hình chóp tứ giác đều.\\
 Gọi $\overrightarrow{F}$ là trọng lực tác dụng lên chậu cây và ${O}'$ là tâm của hình vuông $A'B'C'D'$.\\
 Ta có 
 $\overrightarrow{F}=\overrightarrow{F}_1+\overrightarrow{F}_2+\overrightarrow{F}_3+\overrightarrow{F}_4=\overrightarrow{SA'}+\overrightarrow{SB'}+\overrightarrow{SC'}+\overrightarrow{SD'}=4\overrightarrow{SO'}$.\\
 Mà $\left| \overrightarrow{F} \right|=60\Rightarrow \left| \overrightarrow{SO'} \right|=15$ và $\overrightarrow{SO}=(0;0;-40)$ nên $\left| \overrightarrow{SO} \right|=40$.\\
 Vậy $\overrightarrow{SO'}=\dfrac{15}{40}\overrightarrow{SO}=\dfrac{3}{8}\overrightarrow{SO}$.\\
 Suy ra
 $\overrightarrow{SA'}=\dfrac{3}{8}\overrightarrow{SA}$, 
 $\overrightarrow{SB'}=\dfrac{3}{8}\overrightarrow{SB}$, 
 $\overrightarrow{SC'}=\dfrac{3}{8}\overrightarrow{SC}$ và 
 $\overrightarrow{SD'}=\dfrac{3}{8}\overrightarrow{SD}$.\\
 Do đó $\overrightarrow{F}_1=(15;0;-15)$, $\overrightarrow{F}_2=(0;15;-15)$, $\overrightarrow{F}_3=(-15;0;-15)$, $\overrightarrow{F}_4=(0;-15;-15)$.\\
 Suy ra $\overrightarrow{F}_1+\overrightarrow{F}_2+\overrightarrow{F}_3+3\overrightarrow{F}_4=(0;-30;-90)$. \\
 Vậy $\left| \overrightarrow{F}_1+\overrightarrow{F}_2+\overrightarrow{F}_3+3\overrightarrow{F}_4 \right|=\sqrt{0^2+(-30)^2+(-90)^2}=30\sqrt{10}\approx 95$\,(N).
 }
\end{ex}
\Closesolutionfile{ans}
 
\begin{indapan}
	{ans/ans\currfilebase}
\end{indapan}


%C3
% \begin{name}
	{\tenchude}
	{ĐỀ ÔN TẬP CHƯƠNG III}
	{LỚP TOÁN THẦY PHÁT}
	{\thoigian}
\end{name}
\TN
\Opensolutionfile{ans}[ans/ans\currfilebase-Phan-I]
\begin{ex}%[2-D3B3-SO-7-2425]%[VN-MT-7, Nguyễn Kiều Nhã Tú]%[2D3N1-1]
 Cho mẫu số liệu ghép nhóm
 \begin{center}
 \begin{tabular}{|c|c|c|c|c|c|}
 \hline
 Nhóm & $[a_1;a_2)$ & $\ldots$ & $[a_j;a_{j+1})$ & $\ldots$ &$[a_k;a_{k+1})$ \\
 \hline
 Tần số & $m_1$ & $\ldots$ & $m_i$ & $\ldots$ & $m_k$ \\
 \hline
 \end{tabular}
 \end{center}
 trong đó các tần số $m_1 > 0$, $m_k > 0$ và $n = m_1 + \cdots + m_k$ là cỡ mẫu.\\
 Khoảng biến thiên của mẫu số liệu ghép nhóm trên là
 \choice
 {\True $R = a_{k+1} - a_1$}
 {$R = a_k - a_{k+1}$}
 {$R = a_{k+1} + a_1$}
 {$R = a_k + a_{k+1}$}
 \loigiai{
 Khoảng biến thiên của mẫu số liệu ghép nhóm trên là $R = a_{k+1} - a_1$.
 }
\end{ex}

\begin{ex}%[2-D3B3-SO-7-2425]%[VN-MT-7, Nguyễn Kiều Nhã Tú]%[2D3N1-1]
 Cho mẫu số liệu ghép nhóm có tứ phân vị thứ nhất, thứ hai, thứ ba lần lượt là $Q_1$, $Q_2$ và $Q_3$. Khoảng tứ phân vị của mẫu số liệu ghép nhóm đó bằng
 \choice
 {$Q_2 - Q_1$}
 {$Q_1 - Q_3$}
 {\True $Q_3 - Q_1$}
 {$Q_1 - Q_2$}
 \loigiai{
 Khoảng tứ phân vị của mẫu số liệu ghép nhóm là $Q_3 - Q_1$.
 }
\end{ex}

\begin{ex}%[2-D3B3-SO-7-2425]%[VN-MT-7, Nguyễn Kiều Nhã Tú]%[2D3H1-3]
 Một người ghi lại thời gian đàm thoại của một số cuộc gọi cho kết quả như bảng sau:
 \begin{center}
 \begin{tabular}{|c|c|c|c|c|c|}
 \hline
 Thời gian $t$ (phút) & $[0;1)$ & $[1;2)$ & $[2;3)$ & $[3;4)$ & $[4;5)$ \\
 \hline
 Số cuộc gọi & $8$ & $17$ & $25$ & $20$ & $10$ \\
 \hline
 \end{tabular}
 \end{center}
 Khoảng tứ phân vị của mẫu số liệu trên có giá trị bằng
 \choice
 {\True $\dfrac{61}{34}$}
 {$\dfrac{7}{2}$}
 {$\dfrac{29}{17}$}
 {$\dfrac{177}{34}$}
 \loigiai{
 Cỡ mẫu $n = 80$.\\
 Giả sử $x_1$, $x_2$, $\ldots$, $x_{80}$ là thời gian đàm thoại của $80$ cuộc gọi đã được sắp xếp theo thứ tự không giảm.
 \begin{itemize}
 \item Vì $\dfrac{n}{4} = 20$ và $8 < 20 < 8 + 17$ nên tứ phân vị thứ nhất thuộc nhóm $[1;2)$ và có giá trị là
 \[Q_1 = 1 + \dfrac{20-8}{17}\cdot 1 = \dfrac{29}{17}.\]
 \item Lại có $\dfrac{3n}{4} = 60$ và $8+17+25 < 60 < 8 + 17+25+20$ nên tứ phân vị thứ ba thuộc nhóm $[3;4)$ và có giá trị là
 \[Q_3 = 3 + \dfrac{60-(8+17+25)}{20}\cdot 1 = \dfrac{7}{2}.\]
 \end{itemize}
 Vậy khoảng tứ phân vị là $\Delta_{Q}=Q_3-Q_1=\dfrac{7}{2}-\dfrac{29}{17}=\dfrac{61}{34}$.
 }
\end{ex}

\begin{ex}%[2-D3B3-SO-7-2425]%[VN-MT-7, Nguyễn Kiều Nhã Tú]%[1D5H2-3]
 Sau khi kiểm tra sức khoẻ tổng quát, kết quả số cân nặng của học sinh lớp 12A sĩ số $40$ học sinh được thể hiện trong bảng số liệu sau:
 \begin{center}
 \begin{tabular}{|c|c|c|c|c|c|}
 \hline
 Cân nặng (kg)& $[40;50)$ & $[50;60)$ & $[60;70)$ & $[70;80)$ & $[80;90)$ \\
 \hline
 Số học sinh & $7$ & $12$ & $12$ & $7$ & $2$ \\
 \hline
 \end{tabular}
 \end{center}
 Tứ phân vị thứ nhất của mẫu số liệu trên bằng
 \choice
 {$50$}
 {$50{,}5$}
 {\True $52{,}5$}
 {$55{,}5$}
 \loigiai{
 Cỡ mẫu $n = 40$.\\
 Giả sử $x_1$, $x_2$, $\ldots$, $x_{40}$ là cân nặng của $40$ học sinh đã được sắp xếp theo thứ tự không giảm.\\
 Vì $\dfrac{n}{4} = 10$ và $7 < 10 < 7 + 12$ nên tứ phân vị thứ nhất thuộc nhóm $[50;60)$ và có giá trị là \[Q_1 = 50 + \dfrac{10-7}{12}\cdot 10 = 52{,}5.\]
 }
\end{ex}

\begin{ex}%[2-D3B3-SO-7-2425]%[VN-MT-7, Nguyễn Kiều Nhã Tú]%[1D5H2-3]
 Chỉ số ô nhiễm không khí (AQI) tại thủ đô Hà Nội trong tháng $6/2024$ được thống kê vào $10$h$30$ sáng các ngày trong tháng thể hiện trong bảng số liệu sau:
 \begin{center}
 \begin{tabular}{|c|c|c|c|c|c|}
 \hline
 Chỉ số (AQI) & $[130;145)$ & $[145;160)$ & $[160;175)$ &$[175;190)$ & $[190;205)$ \\
 \hline
 Số ngày & $8$ & $7$ & $6$ & $7$ & $2$ \\
 \hline
 \end{tabular}
 \end{center}
 Tứ phân vị thứ ba của mẫu số liệu trên gần nhất với giá trị nào trong các giá trị sau?
 \choice
 {$175$}
 {$176{,}5$}
 {$180{,}2$}
 {\True $178{,}2$}
 \loigiai{
 Cỡ mẫu $n = 30$.\\
 Giả sử $x_1$, $x_2$, $\ldots$, $x_{30}$ là chỉ số (AQI) của $30$ ngày trong tháng $6/2024$ đã được sắp xếp theo thứ tự không giảm.\\
 Vì $\dfrac{3n}{4} =22{,}5 $ và $8+7+6< 22{,}5 < 8+7+6+7$ nên tứ phân vị thứ ba thuộc nhóm $[175;190)$ và có giá trị là
 \[Q_3 = 175 + \dfrac{22,5-(8+7+6)}{7}\cdot 15 \approx 178{,}2.\]
 }
\end{ex}

\begin{ex}%[2-D3B3-SO-7-2425]%[VN-MT-7, Nguyễn Kiều Nhã Tú]%[2D3N1-2]
 Trong kì thi chọn học sinh giỏi ở cụm trường THPT A, môn Toán có $25$ học sinh tham gia kết quả điểm bài thi của học sinh được thể hiện trong bảng sau:
 \begin{center}
 \begin{tabular}{|c|c|c|c|c|c|}
 \hline
 Điểm bài thi & $[10;12)$ & $[12;14)$ & $[14;16)$ & $[16;18)$ & $[18;20)$ \\
 \hline
 Số lần & $4$ & $6$ & $8$ & $4$ & $3$ \\
 \hline
 \end{tabular}
 \end{center}
 Khoảng biến thiên của mẫu số liệu ghép nhóm nhận giá trị nào trong các giá trị dưới đây?
 \choice
 {$18,5$}
 {$10,5$}
 {$8$}
 {\True $10$}
 \loigiai{
 Khoảng biến thiên của mẫu số liệu là $20 - 10 = 10$.
 }
\end{ex}

\begin{ex}%[2-D3B3-SO-7-2425]%[VN-MT-7, Nguyễn Kiều Nhã Tú]%[1D5H2-3]
 Đo cân nặng $40$ học sinh lớp 12A ta được bảng số liệu như sau:
 \begin{center}
 \begin{tabular}{|c|c|c|c|c|c|c|c|}
 \hline
 Khối lượng (kg) & $[40;45)$ & $[45;50)$ & $[50;55)$ & $[55;60)$ & $[60;65)$ & $[65;70)$ & $[70;75)$ \\
 \hline
 Số học sinh & $4$ & $13$ & $7$ & $5$ & $6$ & $2$ & $1$ \\
 \hline
 \end{tabular}
 \end{center}
 Tứ phân vị thứ nhất của mẫu số liệu ghép nhóm thuộc khoảng nào sau đây?
 \choice
 {$[40;45)$}
 {\True $[45;50)$}
 {$[50;55)$}
 {$[55;60)$}
 \loigiai{
 Cỡ mẫu $n = 40$.\\
 Giả sử $x_1$, $x_2$, $\ldots$, $x_{40}$ là cân nặng của $40$ học sinh lớp 12A đã được sắp xếp theo thứ tự không giảm.\\
 Vì $\dfrac{n}{4} =10$ và $4< 10 < 4+13$ nên tứ phân vị thứ nhất thuộc nhóm $[45;50)$.
 }
\end{ex}

\begin{ex}%[2-D3B3-SO-7-2425]%[VN-MT-7, Nguyễn Kiều Nhã Tú]%[1D5H1-3]
 Thống kê điểm thi đánh giá năng lực của một trường THPT qua thang điểm $120$ môn Toán như sau:
 \begin{center}
 \begin{tabular}{|c|c|c|c|c|c|}
 \hline
 Điểm & $[0;20)$ & $[20;40)$ & $[40;60)$ & $[60;80)$ & $[80;100)$ \\
 \hline
 Số học sinh & $25$ & $35$ & $37$ & $15$ & $8$ \\
 \hline
 \end{tabular}
 \end{center}
 Điểm trung bình của tất cả các học sinh tham gia dự thi thuộc khoảng nào sau đây?
 \choice
 {\True $(40;45)$}
 {$(45;50)$}
 {$(50;55)$}
 {$(55;60)$}
 \loigiai{
 Ta có bảng sau:
 \begin{center}
 \begin{tabular}{|c|c|c|c|c|c|}
 \hline
 Điểm & $[0;20)$ & $[20;40)$ & $[40;60)$ & $[60;80)$ & $[80;100)$ \\
 \hline
 Giá trị đại diện & $10$ & $30$ & $50$ & $70$ & $90$ \\
 \hline
 Số học sinh & $25$ & $35$ & $37$ & $15$ & $8$ \\
 \hline
 \end{tabular}
 \end{center}
 Điểm trung bình của các thí sinh dự thi là
 \[\overline{x} = \dfrac{25 \cdot 10 + 35 \cdot 30 + 37 \cdot 50 + 15 \cdot 70 + 8 \cdot 90}{120} = 41.\]
 Do đó, điểm trung bình của tất cả các học sinh tham gia dự thi thuộc khoảng $(40;45)$.
 }
\end{ex}

\begin{ex}%[2-D3B3-SO-7-2425]%[VN-MT-7, Nguyễn Kiều Nhã Tú]%[2D3H2-2]
 Đo chiều cao các em học sinh khối $10$ ta thu được kết quả trong bảng sau:
 \begin{center}
 \begin{tabular}{|c|c|c|c|c|c|c|}
 \hline
 Chiều cao (cm) & $[150;152)$ & $[152;154)$ & $[154;156)$ & $[156;158)$ & $[158;160)$ & $[160;162]$ \\
 \hline
 Số học sinh & $5$ & $18$ & $40$& $26$ & $8$ & $3$ \\
 \hline
 \end{tabular}
 \end{center}
 Tính phương sai của mẫu số liệu ghép nhóm trên (làm tròn kết quả đến hàng phần mười).
 \choice
 {$4{,}5$}
 {$5{,}6$}
 {\True $4{,}7$}
 {$4{,}8$}
 \loigiai{
 Ta có bảng sau:
 \begin{center}
 \begin{tabular}{|c|c|c|c|c|c|c|}
 \hline
 Chiều cao (cm) & $[150;152)$ & $[152;154)$ & $[154;156)$ & $[156;158)$ & $[158;160)$ & $[160;162]$ \\
 \hline
 Giá trị đại diện & $151$ &$153$ & $155$& $157$ & $159$ & $161$ \\
 \hline
 Số học sinh & $5$ & $18$ & $40$& $26$ & $8$ & $3$ \\
 \hline
 \end{tabular}
 \end{center}
 Giá trị trung bình của mẫu số liệu là
 \[\overline{x} = \dfrac{5 \cdot 151 + 18 \cdot 153 + 40 \cdot 155 + 26 \cdot 157 + 8 \cdot 159 + 3 \cdot 161}{100} = 155{,}46.\]
 Khi đó phương sai của mẫu số liệu là
% \[s^2 =\dfrac{5(151-155{,}46)^2 + 18(153-155{,}46)^2 + \ldots + 3(161-155{,}46)^2}{100}=4{,}7084.\]
 \[s^2 = \dfrac{5 \cdot 151^2 + 18 \cdot 153^2 + 40 \cdot 155^2 + 26 \cdot 157^2 + 8 \cdot 159^2 + 3 \cdot 161^2}{100} -(155{,}46)^2 = 4{,}7084.\]
 }
\end{ex}

\begin{ex}%[2-D3B3-SO-7-2425]%[VN-MT-7, Nguyễn Kiều Nhã Tú]%[2D3N2-1]
 Số đặc trưng nào sau đây \textbf{không sử dụng} để đo mức độ phân tán của mẫu số liệu ghép nhóm?
 \choice
 {Khoảng biến thiên}
 {\True Trung vị}
 {Phương sai}
 {Khoảng tứ phân vị}
 \loigiai{
 \begin{itemize}
 \item Khoảng biến thiên của mẫu số liệu ghép nhóm xấp xỉ cho khoảng biến thiên của mẫu số liệu gốc. Khoảng biến thiên được dùng để đo mức độ phân tán của mẫu số liệu ghép nhóm. Khoảng biến thiên càng lớn thì mẫu số liệu càng phân tán.
 \item Trung vị của mẫu số liệu ghép nhóm xấp xỉ cho trung vị của mẫu số liệu gốc, nó chia mẫu số liệu thành hai phần, mỗi phần chứa $50$\% giá trị. Vậy trung vị không thể hiện mức độ phân tán.
 \item Phương sai của mẫu số liệu ghép nhóm xấp xỉ cho phương sai của mẫu số liệu gốc. Phương sai được dùng để đo mức độ phân tán của mẫu số liệu ghép nhóm xung quanh số trung bình của mẫu số liệu đó. Phương sai càng lớn thì mẫu số liệu càng phân tán.
 \item Khoảng tứ phân vị của mẫu số liệu ghép nhóm xấp xỉ cho khoảng tứ phân vị của mẫu số liệu gốc. Khoảng tứ phân vị được dùng để đo mức độ phân tán của mẫu số liệu ghép nhóm. Khoảng tứ phân vị càng lớn thì mẫu số liệu càng phân tán.
 \end{itemize}
 }
\end{ex}

\begin{ex}%[2-D3B3-SO-7-2425]%[VN-MT-7, Nguyễn Kiều Nhã Tú]%[2D3N2-1]
 Ý nghĩa của độ lệch chuẩn đối với mẫu số liệu ghép nhóm là
 \choice
 {\True dùng độ lệch chuẩn của mẫu số liệu để ước lượng độ lệch chuẩn xung quanh số trung bình của mẫu số liệu đó}
 {cho biết về ý nghĩa trung tâm của mẫu số liệu và cả về độ tán xạ dữ liệu của mẫu số liệu}
 {chia mẫu số liệu thành hai phần, mỗi phần chứa $50$\% giá trị}
 {chia mẫu số liệu thành bốn phần, mỗi phần chứa $25$\% giá trị}
 \loigiai{
 Ý nghĩa độ lệch chuẩn của mẫu số liệu ghép nhóm: Độ lệch chuẩn của mẫu số liệu ghép nhóm xấp xỉ cho độ lệch chuẩn của mẫu số liệu gốc. Độ lệch chuẩn được dùng để đo mức độ phân tán của mẫu số liệu ghép nhóm xung quanh số trung bình của mẫu số liệu đó. Độ lệch chuẩn càng lớn thì mẫu số liệu càng phân tán.
 }
\end{ex}

\begin{ex}%[2-D3B3-SO-7-2425]%[VN-MT-7, Nguyễn Kiều Nhã Tú]%[2D3H2-2]
 Quãng đường đi bộ tập thể dục mỗi ngày (đơn vị: km) của bác An trong $20$ ngày được thống kê lại ở bảng sau:
 \begin{center}
 \begin{tabular}{|c|c|c|c|c|c|}
 \hline
 Quãng đường (km) & $[2{,}2;2{,}6)$ & $[2{,}6;3{,}0)$ & $[3{,}0;3{,}4)$ & $[3{,}4;3{,}8)$ & $[3{,}8;4{,}2)$ \\
 \hline
 Tần số & $3$ & $6$ & $5$ & $5$ & $1$ \\
 \hline
 \end{tabular}
 \end{center}
 Độ lệch chuẩn của mẫu số liệu trên có giá trị xấp xỉ bằng
 \choice
 {$3{,}1$}
 {$0{,}042$}
 {$0{,}206$}
 {\True $0{,}45$}
 \loigiai{
 Ta có bảng sau:
 \begin{center}
 \begin{tabular}{|c|c|c|c|c|c|}
 \hline
 Quãng đường (km) & $[2{,}2;2{,}6)$ & $[2{,}6;3{,}0)$ & $[3{,}0;3{,}4)$ & $[3{,}4;3{,}8)$ & $[3{,}8;4{,}2)$ \\
 \hline
 Giá trị đại diện &$2{,}4$&$2{,}8$&$3{,}2$&$3{,}6$&$4{,}0$\\
 \hline
 Tần số & $3$ & $6$ & $5$ & $5$ & $1$ \\
 \hline
 \end{tabular}
 \end{center}
 Số trung bình của mẫu số liệu ghép nhóm là
 \[\overline{x} = \dfrac{3 \cdot 2{,}4 + 6 \cdot 2{,}8 + 5 \cdot 3{,}2 + 5 \cdot 3{,}6 + 1 \cdot 4{,}0}{20} = 3{,}1.\]
 Phương sai của mẫu số liệu ghép nhóm là
 \[s^2 = \dfrac{3 \cdot (2{,}4 - 3{,}1)^2 + 6 \cdot (2{,}8 - 3{,}1)^2 + 5 \cdot (3{,}2 - 3{,}1)^2 + 5 \cdot (3{,}6 - 3{,}1)^2 + 1 \cdot (4{,}0 - 3{,}1)^2}{20} = 0{,}206.\]
 Độ lệch chuẩn của mẫu số liệu ghép nhóm là
 \[s = \sqrt{0{,}206} \approx 0{,}45.\]
 }
\end{ex}
\Closesolutionfile{ans}

\TNTF
\Opensolutionfile{ans}[ans/ans\currfilebase-Phan-II]

\begin{ex}%[2-D3B3-SO-7-2425]%[VN-MT-7, Nguyễn Kiều Nhã Tú]%[2D3H1-4]
 Thành tích chạy $50$\,m của $30$ em học sinh lớp $10$ trường THPT A (đơn vị: giây) được thống kê như bảng sau:
 \begin{center}
 \begin{tabular}{*{6}{p{2cm}}}
 $6{,}3$ & $6{,}2$ & $6{,}5$ & $6{,}8$ & $6{,}9$ & $8{,}2$ \\
 $6{,}6$ & $6{,}7$ & $7{,}0$ & $7{,}1$ & $7{,}2$ & $8{,}3$ \\
 $7{,}4$ & $7{,}3$ & $7{,}2$ & $7{,}1$ & $7{,}0$ & $8{,}4$ \\
 $7{,}1$ & $7{,}3$ & $7{,}5$ & $7{,}5$ & $7{,}6$ & $8{,}7$ \\
 $7{,}6$ & $7{,}7$ & $7{,}8$ & $7{,}5$ & $7{,}7$ & $7{,}8$. \\
 \end{tabular}
 \end{center}
 \choiceTF
 {\True Tần số của nhóm $[7,0;7,5)$ là $10$}
 {Trung bình mỗi em chạy $50$\,m hết số thời gian là $7{,}5$ (giây)}
 {Khoảng biến thiên của mẫu số liệu ghép nhóm trên là $R=3{,}1$}
 {\True Khoảng tứ phân vị của mẫu số liệu ghép nhóm trên là $\Delta_{Q}=0{,}781$}
 \loigiai{
 \begin{itemchoice}
 \itemch \textbf{Đúng}.\\
 Bảng tần số ghép nhóm của mẫu số liệu trên là
 \begin{center}
 \begin{tabular}{|c|c|c|c|c|c|c|}
 \hline
 Thời gian chạy (giây) & $[6{,}0;6{,}5)$ & $[6{,}5;7{,}0)$ & $[7{,}0;7{,}5)$ & $[7{,}5;8{,}0)$ & $[8{,}0;8{,}5)$ & $[8{,}5;9{,}0)$ \\
 \hline
 Tần số & $2$ & $5$ & $10$ & $9$ & $3$ & $1$ \\
 \hline
 \end{tabular}
 \end{center}
 \itemch \textbf{Sai}.\\
 Tổng số học sinh là $n=30$.\\
 Trung bình mỗi em chạy $50$\,m hết số thời gian là
 \begin{center}
 $\overline{x}=\dfrac{6{,}25 \cdot 2+6{,}75 \cdot 5+7{,}25 \cdot 10+7{,}75 \cdot 9+8{,}25 \cdot 3+8{,}75 \cdot 1}{30}=7{,}4$ (giây).
 \end{center}
 \itemch \textbf{Sai}.\\
 Khoảng biến thiên của mẫu số liệu ghép nhóm trên là $R=9{,}0-6{,}0=3{,}0$.
 \itemch \textbf{Đúng}.
 \begin{itemize}
 \item Cỡ mẫu là $n=30$.\\
 Gọi $x_{1}$,$\ldots$, $x_{30}$ là thời gian chạy $50$\,m của $30$ học sinh và giả sử dãy này đã được sắp xếp theo thứ tự không giảm.\\
 Khi đó, trung vị là $\dfrac{x_{15}+x_{16}}{2}$. Do $2$ giá trị $x_{15}$, $x_{16}$ thuộc nhóm $[7{,}0;7{,}5)$ nên nhóm này chứa trung vị. Do đó
 \[
 M_{\text{e}}=7{,}0+\dfrac{\dfrac{30}{2}-(2+5)}{10} \cdot 0{,}5=7{,}4.
 \]
 Tứ phân vị thứ hai $Q_{2}$ chính là trung vị $M_{\text{e}}$.
 \item Tứ phân vị thứ nhất $Q_{1}$\\
 Nhóm chứa $Q_{1}$ là nhóm $[7{,}0;7{,}5)$. Khi đó $Q_{1}=7{,}0+\dfrac{\dfrac{30}{4}-(2+5)}{10} \cdot 0{,}5=7{,}025$.
 \item Tứ phân vị thứ ba $Q_{3}$\\
 Nhóm chứa $Q_{3}$ là nhóm $[7{,}5;8{,}0)$.\\
 Khi đó $Q_{3}=7{,}5+\dfrac{\dfrac{3\cdot30}{4}-(2+5+10)}{9} \cdot 0{,}5=7{,}80(5) \approx 7{,}806$.
 \item Khoảng tứ phân vị của mẫu số liệu ghép nhóm trên là $\Delta_{Q}=Q_{3}-Q_{1}=7{,}806-7{,}025=0{,}781$.
 \end{itemize}
 \end{itemchoice}
 }
\end{ex}

\begin{ex}%[2-D3B3-SO-7-2425]%[VN-MT-7, Nguyễn Kiều Nhã Tú]%[2D3H2-3]
 Khảo sát thời gian xem điện thoại trong một ngày của một số học sinh khối $12$ thu được mẫu số liệu ghép nhóm sau:
 \begin{center}
 \begin{tabular}{|c|c|c|c|c|c|}
 \hline
 Thời gian (phút) & $[0;20)$ & $[20;40)$ & $[40;60)$ & $[60;80)$ & $[80;100)$ \\
 \hline
 Số học sinh & $4$ & $8$ & $12$ & $10$ & $8$ \\
 \hline
 \end{tabular}
 \end{center}
 \choiceTF
 {\True Tổng số học sinh được khảo sát là $42$}
 {Mốt của mẫu số liệu lớn hơn $54$}
 {\True Khoảng tứ phân vị của mẫu số liệu lớn hơn $38$}
 {\True Phương sai của mẫu số liệu nhỏ hơn $610$}
 \loigiai{
 \begin{itemchoice}
 \itemch \textbf{Đúng}.\\
 Tổng số học sinh được khảo sát là $n=4+8+12+10+8=42$.
 \itemch \textbf{Sai}.\\
 Nhóm có tần số lớn nhất là $[40;60)$.\\
 Mốt của mẫu số liệu là
 \[M_{\text{0}}=40+\dfrac{12-8}{(12-8)+(12-10)} \cdot(60-40) \approx 53{,}3.\]
 \itemch \textbf{Đúng}.\\
 Gọi $x_{1}$, $x_{2}$,$\ldots$, $x_{42}$ là thời gian xem điện thoại trong ngày của $42$ học sinh khối $12$ và giả sử dãy này đã sắp xếp theo thứ tự không giảm.\\
 Khi đó tứ phân vị thứ nhất của mẫu gốc là trung vị của dãy $x_{1}$, $x_{2}$,$\ldots$, $x_{21}$, tức là $x_{11}$. Do đó $Q_{1}$ thuộc nhóm $[20;40)$.\\
 Tứ phân vị thứ ba của mẫu gốc là trung vị của dãy $x_{22}$, $x_{2}$,$\ldots$, $x_{42}$, tức là $x_{32}$. Do đó $Q_{3}$ thuộc nhóm $[60;80)$.\\
 Suy ra $Q_{1}=20+\dfrac{\dfrac{42}{4}-4}{8} \cdot(40-20)=36{,}25$ và
 $Q_{3}=60+\dfrac{\dfrac{3 \cdot 42}{4}-24}{10} \cdot(80-60)=75$.\\
 Khoảng tứ phân vị của mẫu số liệu là $\Delta Q=Q_{3}-Q_{1}=75-36{,}25=38{,}75$.
 \itemch \textbf{Đúng}.\\
 Số trung bình của mẫu số liệu là
 \[
 \overline{x}=\dfrac{4 \cdot 10+8 \cdot 30+12 \cdot 50+10 \cdot 70+8 \cdot 90}{42} \approx 54{,}76.
 \]
 Phương sai của mẫu số liệu là
 \begin{eqnarray*}
 s^{2} & = & \dfrac{4 (10-54{,}76)^{2}+8 (30-54{,}76)^{2}+12 (50-54{,}76)^{2}+10 (70-54{,}76)^{2}+8 (90-54{,}76)^{2}}{42}\\
 & \approx & 605{,}9.
 \end{eqnarray*}
 \end{itemchoice}
 }
\end{ex}

\begin{ex}%[2-D3B3-SO-7-2425]%[VN-MT-7, Nguyễn Kiều Nhã Tú]%[2D3H2-3]
 Một trang trại phân $1\ 000$ quả trứng thành $5$ loại, tùy theo khối lượng (đã được làm tròn) của chúng được thống kê bởi bảng dưới đây:
 \begin{center}
 \begin{tabular}{|c|c|c|c|c|c|}
 \hline
 Khối lượng (gam) & $[30;36)$ & $[36;42)$ & $[42;48)$ & $[48;54)$ & $[54;60)$ \\
 \hline
 Số trứng & $45$ & $190$ & $500$ & $250$ & $15$ \\
 \hline
 \end{tabular}
 \end{center}
 \choiceTF
 {\True Tần suất của khối lượng trứng $[30;36)$ là $19 \%$}
 {Số trung vị của mẫu số liệu là $43$}
 {Khoảng biến thiên của mẫu số liệu $39{,}18$}
 {\True Độ lệch chuẩn của mẫu số liệu là $\dfrac{6 \sqrt{17}}{5}$}
 \loigiai{
 \begin{itemchoice}
 \itemch \textbf{Đúng}.\\
 Tần suất của khối lượng trứng $[30;36)$ là $\dfrac{190}{1\ 000} \cdot 100=19 \%$.
 \itemch \textbf{Sai}.\\
 Nhóm chứa trung vị là nhóm $[42;48)$.
 \[M_{\text{e}}=42+\dfrac{\dfrac{1\ 000}{2}-235}{500} \cdot(48-42)=\dfrac{2\ 259}{50}.\]
 \itemch \textbf{Sai}.\\
 Khoảng biến thiên của mẫu số liệu là $60-30=30$.
 \itemch \textbf{Đúng}.\\
 Ta có bảng sau:
 \begin{center}
 \begin{tabular}{|c|c|c|c|c|c|}
 \hline
 Khối lượng (gam) & $[30 ; 36)$ & $[36 ; 42)$ & $[42 ; 48)$ & $[48 ; 54)$ & $[54 ; 60)$ \\
 \hline
 Số trứng & $45$ & $190$ & $500$ & $250$ & $15$ \\
 \hline
 Giá trị đại diện & $33$ & $39$ & $45$ & $51$ & $57$ \\
 \hline
 \end{tabular}
 \end{center}
 Phương sai là
 \[
 s^{2}=\dfrac{33^{2} \cdot 45+39^{2} \cdot 190+45^{2} \cdot 500+51^{2} \cdot 250+57^{2} \cdot 15}{1000}-45^{2}=24{,}48.
 \]
 Vậy độ lệch chuẩn của mẫu số liệu là $s=\sqrt{24{,}48}=\dfrac{6 \sqrt{17}}{5}$.
 \end{itemchoice}
 }
\end{ex}

\begin{ex}%[2-D3B3-SO-7-2425]%[VN-MT-7, Nguyễn Kiều Nhã Tú]%[2D3V2-3]
 Bảng sau thống kê lại tổng số giờ nắng trong tháng $6$ của các năm từ $2002$ đến $2021$ tại hai trạm quan trắc đặt ở Nha Trang và Quy Nhơn.
 \begin{center}
 \begin{tabular}{|c|c|c|c|c|c|c|}
 \hline
 Số giờ nắng & $[130 ; 160)$ & $[160 ; 190)$ & $[190 ; 220)$ & $[220 ; 250)$ & $[250 ; 280)$ & $[280 ; 310)$ \\
 \hline
 Số liệu ở Nha Trang & $1$ & $1$ & $1$ & $8$ & $7$ & $2$ \\
 \hline
 Số liệu ở Quy Nhơn & $0$ & $1$ & $2$ & $4$ & $10$ & $3$ \\
 \hline
 \end{tabular}
 \end{center}
 \begin{flushright}
 (Nguồn: Tổng cục Thống kê)
 \end{flushright}
 \choiceTF
 {Xét số liệu ở Nha Trang thì khoảng tứ phân vị của mẫu số liệu ghép nhóm là $32{,}64$}
 {\True Nếu so sánh theo khoảng tứ phân vị thì số giờ nắng trong tháng $6$ của Quy Nhơn đồng đều hơn}
 {\True Xét số liệu của Quy Nhơn ta có độ lệch chuẩn của mẫu số liệu ghép nhóm (làm tròn kết quả đến hàng phần trăm) là $30{,}59$}
 {Nếu so sánh theo độ lệch chuẩn thì số giờ nắng trong tháng $6$ của Nha Trang đồng đều hơn}
 \loigiai{
 \begin{itemchoice}
 \itemch \textbf{Sai}.\\
 Cỡ mẫu $n=20$.\\
 Gọi $x_{1}$, $x_{2}$,$\ldots$, $x_{20}$ là mẫu số liệu gốc về số giờ nắng trong tháng $6$ trong $20$ năm của Nha Trang được xếp theo thứ tự không giảm.\\
 Ta có $x_{1} \in[130;160)$; $x_{2} \in[160;190)$; $x_{3} \in[190;220)$; $x_{4}$,$\ldots$, $x_{11} \in[220;250)$; $x_{12}$,$\ldots$,\\
 $x_{18} \in[250;280)$; $x_{19}$, $x_{20} \in[280;310)$.\\
 Tứ phân vị thứ nhất của mẫu số liệu gốc là $\dfrac{1}{2}\left(x_{5}+x_{6}\right) \in[220;250)$. Do đó, tứ phân vị thứ nhất của mẫu số liệu ghép nhóm là \[Q_{1}=220+\dfrac{\dfrac{20}{4}-(1+1+1)}{8}\cdot(250-220)=227{,}5.\]
 Tứ phân vị thứ ba của mẫu số liệu gốc là $\dfrac{1}{2}\left(x_{15}+x_{16}\right) \in[250;280)$. Do đó, tứ phân vị thứ ba của mẫu số liệu ghép nhóm là \[Q_{3}=250+\dfrac{\dfrac{3\cdot20}{4}-(1+1+1+8)}{7}\cdot(280-250)=\dfrac{1870}{7}.\]
 Khoảng tứ phân vị của mẫu số liệu ghép nhóm là $\Delta_{Q}=Q_{3}-Q_{1}\approx 39{,}64$.
 \itemch \textbf{Đúng}.\\
 Gọi $y_{1}$, $y_{2}$,$\ldots$, $y_{50}$ là mẫu số liệu gốc về số giờ nắng trong tháng $6$ trong $20$ năm của Quy Nhơn được xếp theo thứ tự không giảm.\\
 Ta có $y_{1} \in[160;190)$; $y_{2}$, $y_{3} \in[190;220)$; $y_{4}$,$\ldots$, $y_{7} \in[220;250)$; $y_{8} $,$\ldots$, $y_{17} \in[250;280)$;
 $y_{18}$,$\ldots$, $y_{20} \in[280;310)$.\\
 Tứ phân vị thứ nhất của mẫu số liệu gốc là $\dfrac{1}{2}\left(y_{5}+y_{6}\right) \in[220;250)$. Do đó, tứ phân vị thứ nhất của mẫu số liệu ghép nhóm là \[Q_{1}'=220+\dfrac{\dfrac{20}{4}-(1+2)}{4}\cdot(250-220)=235.\]
 Tứ phân vị thứ ba của mẫu số liệu gốc là $\dfrac{1}{2}\left(y_{15}+y_{16}\right) \in[250;280)$. Do đó, tứ phân vị thứ ba của mẫu số liệu ghép nhóm là \[Q_{3}'=250+\dfrac{\dfrac{3\cdot 20}{4}-(1+2+4)}{10}\cdot(280-250)=274.\]
 Khoảng tứ phân vị của mẫu số liệu ghép nhóm là $\Delta_{Q}'=Q_{3}'-Q_{1}'=39$.\\
 Vậy nếu so sánh theo khoảng tứ phân vị thì số giờ nắng trong tháng $6$ của Quy Nhơn đồng đều hơn.
 \itemch \textbf{Đúng}.\\
 Xét số liệu của Nha Trang
 \begin{itemize}
 \item Số trung bình 
 \[\overline{x}_{X}=\dfrac{1\cdot 145+1\cdot 175+1\cdot 205+8\cdot 235+7\cdot 265+2\cdot 295}{20}=242{,}5.\]
 \item Độ lệch chuẩn \[s_{X}=\sqrt{\dfrac{1\cdot 145^{2}+1\cdot 175^{2}+1\cdot 205^{2}+8\cdot 235^{2}+7\cdot 265^{2}+2\cdot 295^{2}}{20}-242{,}5^{2}} \approx 35{,}34.\]
 \end{itemize}
 \itemch \textbf{Sai}.\\
 Xét số liệu của Quy Nhơn
 \begin{itemize}
 \item Số trung bình 
 \[\overline{x}_{Y}=\dfrac{1\cdot 175+2\cdot 205+4\cdot 235+10\cdot 265+3\cdot 295}{20}=253.\]
 \item Độ lệch chuẩn 
 \[s_{Y}=\sqrt{\dfrac{1\cdot 175^{2}+2\cdot 205^{2}+4\cdot 235^{2}+10\cdot 265^{2}+3\cdot 295^{2}}{20}-253^{2}} \approx 30{,}59.\]
 \end{itemize}
 Vậy nếu so sánh theo độ lệch chuẩn thì số giờ nắng trong tháng $6$ của Quy Nhơn đồng đều hơn.
 \end{itemchoice}
 }
\end{ex}
\Closesolutionfile{ans}

\TNSA
\Opensolutionfile{ans}[ans/ans\currfilebase-Phan-III]

\begin{ex}%[2-D3B3-SO-7-2425]%[VN-MT-7, Nguyễn Kiều Nhã Tú]%[2D3N1-2]
 Chỉ số AQI là chỉ số thể hiện chất lượng không khí. Có $5$ thông số ảnh hưởng đến chỉ số AQI là Ozone mặt đất, ô nhiễm phân tử (bụi min PM$2.5$ và PM$10$), CO, NO$_2$, SO$_2$ (với NO$_2$, SO$_2$ là tác nhân gây ra mưa axit). Chỉ số AQI từ $0-50$ là mức tốt, từ $51-100$ là trung bình, từ $101-150$ là không tốt cho các nhóm nhạy cảm, từ $151-200$ là không lành mạnh, từ $201-300$ là rất không tốt, và trên $301$ là rất nguy hiểm. Hà Nội của chúng ta là một trong những thành phố ô nhiễm nhất thế giới. Ngày 5/3/2024 chỉ số AQI của Hà Nội đạt mức $241$ và là thành phố ô nhiễm nhất thế giới ngày hôm đó. Chỉ số AQI của một số các thành phố ngày $24/6/2024$ được cho trong bảng sau:
 \begin{center}
 \begin{tabular}{|c|c|c|c|c|}
 \hline
 Chỉ số AQI& $[0;50)$& $[50;100)$& $[100;150)$& $[150;200)$\\\hline
 Số thành phố& $73$& $47$& $7$& $2$\\\hline
 \end{tabular}
 \end{center}
 Khoảng biến thiên của mẫu số liệu trên là bao nhiêu?
 \par
 \shortans{200}
 \loigiai{Khoảng biến thiên của mẫu số liệu trên là $200-0=200$.}
\end{ex}

\begin{ex}%[2-D3B3-SO-7-2425]%[VN-MT-7, Nguyễn Kiều Nhã Tú]%[2D3V1-3]
 Thống kê lượng khách du lịch đến tỉnh Quảng Ninh từ năm $2007$ đến năm $2023$ (đơn vị: triệu người) cho kết quả như sau:
 \begin{center}
 \begin{tabular}{|c|c|c|c|c|c|c|c|c|}
 \hline
 $3{,}4$& $4{,}2$& $5{,}0$& $5{,}4$& $6{,}2$& $7$& $7{,}5$& $7{,}5$& $7{,}8$\\\hline
 $8{,}3$& $9{,}87$& $12{,}2$& $14$& $8{,}8$& $4{,}4$& $9{,}5$& $15{,}5$&\\\hline
 \end{tabular}
 \end{center}
 Ghép nhóm dãy số liệu trên thành các nhóm có độ dài bằng nhau với nhóm đầu tiên là $[1;5)$ rồi cho biết khoảng tứ phân vị của mẫu số liệu ghép nhóm trên.
 \par
 \shortans[]{4{,}44}
 \loigiai{
 Số lượng khách du lịch đến tỉnh Quảng Ninh được cho dưới bảng sau:
 \begin{center}
 \begin{tabular}{|c|c|c|c|c|}
 \hline
 Lượng khách (triệu người)& $[1;5)$& $[5;9)$& $[9;13)$& $[13;17)$\\\hline
 Số năm& $3$& $9$& $3$& $2$\\\hline
 \end{tabular}
 \end{center}
 Cỡ mẫu là $ n=3+9+3+2=17$.\\
 Gọi $x_1$, $x_2$, $\ldots$, $x_{17}$ là số khách đến Quảng Ninh du lịch và giả sử rằng dãy số liệu gốc này đã được sắp xếp theo thứ tự không giảm. \\
 Tứ phân vị thứ nhất của mẫu số liệu gốc này là $\dfrac{1}{2}\left(x_4+x_5\right)$ nên nhóm chứa tứ phân vị thứ nhất là nhóm $[5;9)$ và ta có
 \[Q_1=5+\dfrac{\dfrac{17}{4}-3}{9}\cdot 4\approx 5{,}56.\]
 Tứ phân vị thứ ba của mẫu số liệu gốc là $\dfrac{1}{2}\left(x_{13}+x_{14}\right)$ nên nhóm chứa tứ phân vị thứ ba là nhóm $[9;13)$ và ta có
 \[Q_3=9+\dfrac{\dfrac{3\cdot 17}{4}-12}{3} \cdot 4=10\]
 Vậy khoảng tứ phân vị của mẫu số liệu ghép nhóm là $\Delta Q=Q_3-Q_1=10-5{,}56=4{,}44$.
 }
\end{ex}

\begin{ex}%[2-D3B3-SO-7-2425]%[VN-MT-7, Nguyễn Kiều Nhã Tú]%[2D3H2-2]
 Chiều dài của $40$ bé sơ sinh $12$ ngày tuổi được chọn ngẫu nhiên ở viện nhi trung ương được nghiên cứu thống kê ở bảng dưới đây:
 \begin{center}
 \begin{tabular}{|c|c|c|c|c|c|c|c|}
 \hline
 Chiều dài (cm)& $[44;46)$& $[46;48)$& $[48;50)$&$[50;52)$& $[52;54)$& $[54;56)$& $[56;58)$\\
 \hline
 Số trẻ& $3$& $3$& $10$&$0$& $15$& $7$& $2$\\
 \hline
 \end{tabular}
 \end{center}
 Tìm độ lệch chuẩn (làm tròn đến hàng phần trăm) của $40$ bé sơ sinh ở bảng thống kê trên.
 \par
 \shortans[]{3{,}28}
 \loigiai{
 Ta có bảng phân bố của mẫu ghép nhóm $40$ bé sơ sinh:
 \begin{center}
 \begin{tabular}{|c|c|c|c|c|c|c|c|}
 \hline
 Chiều dài (cm)& $[44;46)$& $[46;48)$& $[48;50)$&$[50;52)$& $[52;54)$& $[54;56)$& $[56;58)$\\\hline
 Số trẻ& $3$& $3$& $10$&$0$& $15$& $7$& $2$\\\hline
 Chiều dài
 đại diện
 (cm)& $45$& $47$& $49$& $51$& $53$& $55$& $57$\\\hline
 \end{tabular}
 \end{center}
 Chiều dài trung bình của $40$ trẻ là
 \[\overline{x}=\dfrac{45\cdot3+47\cdot3+49\cdot10+51\cdot0+53\cdot15+55\cdot7+57\cdot2}{40}=51{,}5 \text{ (cm)}.\]
 Phương sai của $40$ bé sơ sinh ở bảng thống kê trên là
% \[s^2=\frac{3\cdot {{( 51{,}5-45 )}^2}+3\cdot {{( 51{,}5-47 )}^2}+ +2\cdot {{( 57-51{,}5 )}^2}}{40}=10{,}75.\]
 \[s^2=\dfrac{45^2\cdot3+47^2\cdot3+49^2\cdot10+51^2\cdot0+53^2\cdot15+55^2\cdot7+57^2\cdot2}{40}-51{,}5()^2=10{,}75.\]
 Độ lệch chuẩn của $40$ bé sơ sinh ở bảng thống kê trên là $s=\sqrt{s^2}\approx 3{,}28$.
 }
\end{ex}

\begin{ex}%[2-D3B3-SO-7-2425]%[VN-MT-7, Nguyễn Kiều Nhã Tú]%[2D3H2-2]
 Một công ty bất động sản Đất Vàng thực hiện cuộc khảo sát khách hàng xem họ có nhu cầu mua nhà ở mức giá nào để tiến hành dự án xây nhà ở Thăng Long group sắp tới. Kết quả khảo sát $500$ khách hàng được ghi lại ở bảng sau:
 \begin{center}
 \begin{tabular}{|c|c|c|c|c|c|}
 \hline
 Mức giá
 (triệu đồng)& $[10;14)$& $[14;18)$& $[18;22)$& $[22;26)$& $[26;30)$\\\hline
 Số khách hàng& $75$& $105$& $179$& $96$& $45$\\\hline
 \end{tabular}
 \end{center}
 Độ lệch chuẩn (làm tròn đến hàng phần trăm) của mức giá đất là bao nhiêu?
 \par
 \shortans[]{4{,}64}
 \loigiai{
 Bảng phân bố tần số tần suất của bảng số liệu của công ty bất động sản Đất Vàng như sau:
 \begin{center}
 \begin{tabular}{|c|c|c|c|c|c|}
 \hline
 Mức giá
 (triệu đồng)& $[10;14)$& $[14;18)$& $[18;22)$& $[22;26)$& $[26;30)$\\\hline
 Số khách hàng& $75$& $105$& $179$& $96$& $45$\\\hline
 Mức giá đại diện& $12$& $16$& $20$& $24$& $28$\\\hline
 \end{tabular}
 \end{center}
 Mức giá trung bình của công ty là 
 \[\overline{x}=\dfrac{75 \cdot 12 + 105 \cdot 16 + 179 \cdot 20 + 96 \cdot 24 + 45 \cdot 28}{500}\approx 19{,}45\, \text{(triệu đồng)}.\]
 Phương sai của mức giá là 
 \begin{eqnarray*}
 s^2&=&\dfrac{75 (12-19{,}45)^2+105(16-19{,}45)^2+179 (20-19{,}45)^2 + 96(24-19{,}45)^2 + 45 (28-19{,}45)^2}{500}\\
 &\approx& 21{,}49.
 \end{eqnarray*}
 Độ lệch chuẩn của mức giá $\sqrt {s^2} \approx 4{,}64$.
 }
\end{ex}

\begin{ex}%[2-D3B3-SO-7-2425]%[VN-MT-7, Nguyễn Kiều Nhã Tú]%[2D3H2-2]
 Bạn Minh Nhàn sử dụng điện thoại thông minh để chơi game trong một ngày. Số lần bạn sử dụng điện thoại được thống kê như sau:
 \begin{center}
 \begin{tabular}{|c|c|c|c|c|c|}
 \hline
 Thời gian (đơn vị: h)& $[3; 5)$ & $[5; 7)$ & $[7; 9)$ & $[9; 11)$ & $[11; 13)$ \\
 \hline
 Số lần sử dụng & $2$ & $5$ & $13$ & $8$ & $2$ \\
 \hline
 \end{tabular}
 \end{center}
 Hãy tính tỉ số phần trăm (làm tròn 1 chữ số thập phân) giữa độ lệch chuẩn và giá trị trung bình.
 \par
 \shortans[]{23{,}9}
 \loigiai{
 Ta có bảng sau:
 \begin{center}
 \begin{tabular}{|c|c|c|c|c|c|}
 \hline
 Thời gian (đơn vị: h) & $[3; 5)$ & $[5; 7)$ & $[7; 9)$ & $[9; 11)$ & $[11; 13)$ \\
 \hline
 Giá trị đại diện & $4$ & $6$ & $8$ & $10$ & $12$ \\
 \hline
 Số lần sử dụng & $2$ & $5$ & $13$ & $8$ & $2$ \\
 \hline
 \end{tabular}
 \end{center}
 Xét mẫu số liệu của Minh Nhàn $n=2+5+13+8+2=30$.\\
 Số trung bình của mẫu số liệu ghép nhóm là
 \[
 \overline{x}=\dfrac{2\cdot 4+5\cdot 6+13\cdot 8+8\cdot 10+2\cdot 12}{30}=8{,}2.
 \]
 Phương sai của mẫu số liệu ghép nhóm là
 \[
 s^2=\dfrac{1}{30}\left(2\cdot 4^2+5\cdot 6^2+13\cdot 8^2+8\cdot 10^2+2\cdot 12^2\right)-(8{,}2)^2=3{,}83.
 \]
 Độ lệch chuẩn của mẫu số liệu ghép nhóm là $s=\sqrt{s^2} \approx \sqrt{3{,}83} \approx 1{,}96$.\\
 Vậy tỉ số phần trăm giữa độ lệch chuẩn và giá trị trung bình là
 $\dfrac{1{,}96}{8{,}2}\cdot 100\%\approx 23{,}9\%$.
 }
\end{ex}

\begin{ex}%[2-D3B3-SO-7-2425]%[VN-MT-7, Nguyễn Kiều Nhã Tú]%[2D3H2-2]
 Điều tra chi phí thuê nhà ở hằng tháng của một số nhân viên độc thân, công ty $X$ thu được số liệu dưới đây:
 \begin{center}
 \begin{tabular}{|l|c|c|c|c|c|}
 \hline \begin{tabular}{l}
 {Tiền thuê nhà} \\
 {(trăm nghìn đồng)}
 \end{tabular}
 & $[3 ; 6)$ & $[6 ; 9)$ & $[9 ; 12)$ & $[12 ; 15)$ & $[15 ; 18)$ \\
 \hline {Số nhân viên} & $64$ & $40$ & $84$ & $56$ & $16$ \\
 \hline
 \end{tabular}
 \end{center}
 Tính độ lệch chuẩn chi phí thuê nhà hằng tháng của những nhân viên được điều tra (kết quả làm tròn đến hàng phần chục).
 \par
 \shortans[]{3{,}68}
 \loigiai{
 Bổ sung thêm các giá trị đại diện, ta có bảng sau:
 \begin{center}
 \begin{tabular}{|l|c|c|c|c|c|}
 \hline \begin{tabular}{l}
 {Tiền thuê nhà} \\
 {(trăm nghìn đồng)}
 \end{tabular}
 & $[3 ; 6)$ & $[6 ; 9)$ & $[9 ; 12)$ & $[12 ; 15)$ & $[15 ; 18)$ \\
 \hline { Giá trị đại diện} & $4{,}5$ & $7{,}5$ & $10{,5}$ & $13{,}5$ & $16{,}5$ \\
 \hline { Số nhân viên} & $64$ & $40$ & $84$ & $56$ & $16$ \\
 \hline
 \end{tabular}
 \end{center}
 Từ mẫu số liệu đã cho, ta tính được số trung bình là
 \[\overline{x}=\dfrac{4{,}5\cdot 64 + 7{,}5\cdot 40 + 10{,}5\cdot 84 + 13{,}5 \cdot 56 + 16{,}5\cdot 16}{64+40+84+56+16}\approx9{,}58.\]
 Từ đó ta có phương sai chi phí thuê nhà hàng tháng của những nhân viên được điều tra là
 \begin{eqnarray*}
 s^2&=&\dfrac{64 \left(4{,}5-\overline{x}\right)^2+40\left(7{,}5-\overline{x}\right)^2+84\left(10{,}5-\overline{x}\right)^2+56\left(13{,}5-\overline{x}\right)^2+16 \left(16{,}5-\overline{x}\right)^2}{64+40+84+56+16}\\
 &\approx& 13{,}55.
 \end{eqnarray*}
 Vậy độ lệch chuẩn chi phí thuê nhà hàng tháng của những nhân viên được điều tra là
 \[s=\sqrt{s^2}\approx 3{,}68.\]
 }
\end{ex}
\Closesolutionfile{ans}
% \begin{indapan}
% 	{ans/ans\currfilebase}
% \end{indapan}


% \begin{name}
 {Biên soạn: Lại Thị Hảo \\ Phản biện: Nguyễn Tài Tuệ}
 {Đề ôn tập chương III}
\end{name}

\caulc
\Opensolutionfile{ans}[ans/ans\currfilebase-Phan-I]

\begin{ex}%[2-D3B3-SO-8-2425]%[VN-MT-7, Lại Thị Hảo]%[2D3N1-2]
Xét mẫu số liệu ghép nhóm cho bởi bảng sau:
\begin{center}
\begin{tabular}{|c|c|c|c|c|c|c|}
 \hline
 Nhóm & $[40; 45)$ & $[45; 50)$ & $[50; 55)$ & $[55; 60)$ & $[60; 65)$ & \\
 \hline
 Tần số & $4$ & $11$ & $9$ & $8$ & $8$ & $n=40$ \\
 \hline
\end{tabular}
\end{center}
Khoảng biến thiên của mẫu số liệu ghép nhóm đã cho bằng
\choice
{$5$}
{$40$}
{$6$}
{\True $25$}
\loigiai{
 Ta có đầu mút trái của nhóm $1$ là $a_1=40$, đầu mút phải của nhóm $5$ là $a_6=65$.\\
 Vậy khoảng biến thiên của mẫu số liệu ghép nhóm đó là
 \[R=a_6-a_1=65-40=25.\] 
 }
\end{ex}

\begin{ex}%[2-D3B3-SO-8-2425]%[VN-MT-7, Lại Thị Hảo]%[2D3N1-1]
 Xét mẫu số liệu ghép nhóm cho bởi bảng sau:
\begin{center}
\begin{tabular}{|c|c|c|c|c|c|c|}
 \hline
 Nhóm & $[3; 13)$ & $[13; 23)$ & $[23; 33)$ & $[33; 43)$ & $[43; 53)$ & \\
 \hline
 Tần số & $8$ & $7$ & $10$ & $6$ & $9$ & $n=40$ \\
 \hline
\end{tabular}
\end{center}
 Tần số của nhóm $2$ của mẫu số liệu ghép nhóm đã cho bằng
 \choice
 {$6$}
 {\True $7$}
 {$9$}
 {$40$}
 \loigiai{Tần số của nhóm $2$ của mẫu số liệu ghép nhóm đã cho là $n_2=7$.}
\end{ex}

\begin{ex}%[2-D3B3-SO-8-2425]%[VN-MT-7, Lại Thị Hảo]%[2D3N1-4]
Xét mẫu số liệu ghép nhóm cho bởi bảng như hình bên.
 Tần số tích lũy $c f_2$ của nhóm $2$ của mẫu số liệu ghép nhóm đã cho bằng
 \choice
 {$4$}
 {$11$}
 {\True $15$}
 {$40$}
 \begin{center}
 \begin{tabular}{|c|c|c|c|c|c|c|}
 \hline
 Nhóm & $[17; 21)$ & $[21; 25)$ & $[25; 29)$ & $[29; 33)$ & $[33; 37)$ & \\
 \hline
 Tần số & $5$ & $10$ & $6$ & $7$ & $12$ & $n=40$ \\
 \hline
 \end{tabular}
 \end{center}
\loigiai{
 Ta có $c f_2=n_1+n_2=15$.
}
\end{ex}

\begin{ex}%[2-D3B3-SO-8-2425]%[VN-MT-7, Lại Thị Hảo]%[2D3N1-1]
 Khảo sát thời gian tập thể dục trong ngày của một số học sinh khối $11$ thu được mẫu số liệu ghép nhóm sau:
 \begin{center}
 \begin{tabular}{|c|c|c|c|c|c|}
 \hline
 Thời gian (phút) & $[0; 20)$ & $[20; 40)$ & $[40; 60)$ & $[60; 80)$ & $[80; 100)$\\
 \hline
 Số học sinh & $5$ & $9$ & $12$ & $10$ & $6$ \\
 \hline
 \end{tabular}
 \end{center}
 Giá trị đại diện của nhóm $[60; 80)$ là
 \choice
 {$10$}
 {$20$}
 {\True $70$}
 {$40$}
 \loigiai{
 Giá trị đại diện của nhóm $[60; 80)$ là $\dfrac{60+80}{2}=70$.
 }
\end{ex}

\begin{ex}%[2-D3B3-SO-8-2425]%[VN-MT-7, Lại Thị Hảo]%[2D3N1-1]
 Mẫu số liệu dưới đây ghi lại tốc độ của $40$ ô tô khi đi qua một trạm đo tốc độ (đơn vị: km/h):
 \begin{center}
 \begin{tabular}{|l|c|c|c|c|c|c|}
 \hline
 Tốc độ (km/h) & $[40; 45)$ & $[45; 50)$ & $[50; 55)$ & $[55; 60)$ & $[60; 65)$ & $[65; 70)$\\
 \hline
 Số ô tô & $4$ & $11$ & $7$ & $8$ & $8$ & $2$ \\
 \hline
 \end{tabular}
 \end{center}
 Độ dài của nhóm $[55; 60)$ là
 \choice
 {$10$}
 {$55$}
 {\True $5$}
 {$60$}
 \loigiai{
 Độ dài của nhóm $[55; 60)$ là $60-55=5$.
 }
\end{ex}

\begin{ex}%[2-D3B3-SO-8-2425]%[VN-MT-7, Lại Thị Hảo]%[1D5H1-3]
 Người ta đếm số xe ô tô đi qua một trạm thu phí mỗi phút trong khoảng thời gian từ $9$ giờ đến $9$ giờ $30$ phút sáng. Kết quả được ghi lại ở bảng sau:
 \begin{center}
 \begin{tabular}{|c|c|c|c|c|c|}
 \hline
 Số xe & $[6; 10]$ & $[11; 15]$ & $[16; 20]$ & $[21; 25]$ & $[26; 30]$\\
 \hline
 Số lần & $5$ & $9$ & $3$ & $9$ & $4$ \\
 \hline
 Giá trị đại diện & $8$ & $13$ & $18$ & $23$ & $28$ \\
 \hline
 \end{tabular}
 \end{center}
 Tính số trung bình cộng của mẫu số liệu ghép nhóm trên.
 \choice
 {$18{,}4$}
 {$18{,}7$}
 {$17{,}4$}
 {\True $17{,}7$}
 \loigiai{
 Số xe trung bình đi qua trạm trong mỗi phút xấp xỉ bằng
 \[\overline{x}=\dfrac{5 \cdot 8+9 \cdot 13+3 \cdot 18+9 \cdot 23+4 \cdot 28}{30}=17{,}7.\]
 }
\end{ex}

\begin{ex}%[2-D3B3-SO-8-2425]%[VN-MT-7, Lại Thị Hảo]%[1D5H2-3]
 Xét mẫu số liệu ghép nhóm cho bởi bảng sau:
\begin{center}
 \begin{tabular}{|c|c|c|}
 \hline
 Nhóm & Tần số & Tần số tích lũy\\
 \hline
 $[160; 163)$ & $6$ & $6$ \\
 $[163; 166)$ & $11$ & $17$ \\
 $[166; 169)$ & $9$ & $26$ \\
 $[169; 172)$ & $7$ & $33$ \\
 $[172; 175)$ & $3$ & $36$ \\
 \hline
 & $n=36$ & \\
 \hline
\end{tabular}
\end{center}
Tứ phân vị thứ nhất của mẫu số liệu ghép nhóm đã cho bằng
\choice
{\True $\dfrac{1802}{11}$}
{$163$}
{$9$}
{$\dfrac{329}{2}$}
 \loigiai{
Số phần tử của mẫu là $n=36$.\\
Ta có $\dfrac{n}{4}=\dfrac{36}{4}=9$ mà $6< 9< 17$. Suy ra nhóm $2$ là nhóm đầu tiên có tần số tích lũy lớn hơn hoặc bằng $9$.\\
Xét nhóm $2$ là nhóm $[163; 166)$ có $s=163$, $h=3$, $n_2=11$ và nhóm $1$ là nhóm $[160; 163)$ có $c f_1=6$.\\
Tứ phân vị thứ nhất là
\[Q_1=s+\left(\dfrac{9-c f_1}{n_2}\right) \cdot h=163+\left(\dfrac{9-6}{11}\right) \cdot 3=\dfrac{1802}{11}.\]
 }
\end{ex}

\begin{ex}%[2-D3B3-SO-8-2425]%[VN-MT-7, Lại Thị Hảo]%[1D5H2-3]
 Xét mẫu số liệu ghép nhóm cho bởi bảng sau:
\begin{center}
 \begin{tabular}{|c|c|c|}
\hline
Nhóm & Tần số & Tần số tích lũy\\
\hline
$[40; 45)$ & $5$ & $5$ \\
$[45; 50)$ & $10$ & $15$ \\
$[50; 55)$ & $7$ & $22$ \\
$[55; 60)$ & $9$ & $31$ \\
$[60; 65)$ & $7$ & $38$ \\
$[65; 70)$ & $4$ & $42$ \\
\hline
& $n=42$ & \\
\hline
 \end{tabular}
\end{center}
 Tứ phân vị thứ hai của mẫu số liệu ghép nhóm đã cho bằng
 \choice
 {\True $\dfrac{380}{7}$}
 {$50$}
 {$\dfrac{42}{7}$}
 {$\dfrac{105}{2}$}
\loigiai{
 Số phần tử của mẫu là $n=42$.\\
 Ta có $\dfrac{n}{2}=\dfrac{42}{2}=21$ mà $15< 21< 22$. Suy ra nhóm $3$ là nhóm đầu tiên có tần số tích lũy lớn hơn hoặc bằng $21$.\\
 Xét nhóm $3$ là nhóm $[50; 55)$ có $r=50$, $d=5$, $n_3=7$ và nhóm $2$ là nhóm $[45; 50)$ có $c f_2=15$.\\
 Tứ phân vị thứ hai là
 \[Q_2=r+\left(\dfrac{21-c f_2}{n_3}\right) \cdot d=50+\left(\dfrac{21-15}{7}\right) \cdot 5=\dfrac{380}{7}.\]
}
\end{ex}

\begin{ex}%[2-D3B3-SO-8-2425]%[VN-MT-7, Lại Thị Hảo]%[1D5H2-3]
Doanh thu bán hàng trong $20$ ngày được lựa chọn ngẫu nhiên của một của hàng được ghi lại ở bảng sau (đơn vị: triệu đồng):
\begin{center}
 \begin{tabular}{|c|c|c|c|c|c|}
 \hline
 Doanh thu & $[5; 7)$ & $[7; 9)$ & $[9; 11)$ & $[11; 13)$ & $[13; 15)$\\
 \hline
 Số ngày & $2$ & $7$ & $7$ & $3$ & $1$ \\
 \hline
\end{tabular}
\end{center}
Tứ phân vị thứ ba của mẫu số liệu gần nhất với giá trị nào trong các giá trị dưới đây?
\choice
{$10$}
{\True $11$}
{$12$}
{$13$}
 \loigiai{
Gọi $x_1$, $x_2$, $\ldots$, $x_{20}$ là doanh thu bán hàng trong $20$ ngày xếp theo thứ tự không giảm.\\
Khi đó $x_1$, $x_2 \in[5; 7)$; $x_3, \ldots, x_9 \in[7; 9)$; $x_9, \ldots, x_{16} \in[9; 11)$; $x_{17}, \ldots, x_{19} \in[11; 13)$; $x_{20} \in[13; 15)$.\\
Do đó, tứ phân vị thứ ba của mẫu số liệu thuộc nhóm $[9; 11)$.\\
Ta có $n=20$, $n_m=7$, $C=9$, $u_m=9$, $u_{m+1}=11$. Khi đó
\[Q_3=9+\dfrac{\dfrac{3 \cdot 20}{4}-9}{7}\cdot(11-9) =\dfrac{75}{7}\approx 11.\]
}
\end{ex}

\begin{ex}%[2-D3B3-SO-8-2425]%[VN-MT-7, Lại Thị Hảo]%[2D3H1-3]
Mẫu số liệu đây ghi lại tốc độ của $40$ ô tô khi đi qua một trạm đo tốc độ (đơn vị: km/h) được lập bảng tần số ghép nhóm như sau:
\begin{center}
\begin{tabular}{|c|c|c|c|c|c|c|}
 \hline
 Nhóm & $[40; 45)$ & $[45; 50)$ & $[50; 55)$ & $[55; 60)$ & $[60; 65)$ & $[65; 70)$ \\
 \hline
 Giá trị đại diện & $42{,}5$ & $47{,}5$ & $52{,}5$ & $57{,}5$ & $62{,}5$ & $67{,}5$ \\
 \hline
 Tần số & $4$ & $11$ & $7$ & $8$ & $8$ & $2$ \\
 \hline
\end{tabular}
\end{center}
 Khoảng tứ phân vị của mẫu số liệu trên gần bằng số nào dưới đây?
 \choice
 {$11{,}5$}
 {\True $12{,}3$}
 {$14{,}6$}
 {$23$}
\loigiai{
 Số phần tử của mẫu là $n=40$.\\
 Ta có $\dfrac{n}{4}=\dfrac{40}{4}=10$. Suy ra nhóm $2$ là nhóm đầu tiên có tần số tích lũy lớn hơn hoặc bằng $10$.\\
 Xét nhóm $2$ là nhóm $[45; 50)$ có $r=45$; $d=5$; $n_2=11$ và nhóm $1$ là nhóm $[40; 45)$ có $c f_1=4$.\\
 Áp dụng công thức, ta có $Q_1$ của mẫu số liệu là 
 \[Q_1=45+\left(\dfrac{10-4}{11}\right) \cdot 5\approx 47{,}7\, (\mathrm{km/h}).\]
 Ta có $\dfrac{3n}{4}=30$. Suy ra nhóm $4$ là nhóm đầu tiên có tần số tích lũy lớn hơn hoặc bằng $30$.\\ 
 Xét nhóm $4$ là nhóm $[55; 60)$ có $r=55$; $d=5$; $n_4=8$ và nhóm $3$ là nhóm $\left[50; 55\right)$ có $c f_3=22$.\\
 Áp dụng công thức, ta có $Q_3$ của mẫu số liệu là
 \[Q_3=55+\left(\dfrac{30-22}{8}\right) \cdot 5=60\,(\mathrm{km/h}).\]
 Do đó $\Delta_Q=Q_3-Q_1=60-\dfrac{525}{11}=\dfrac{135}{11} \approx 12{,}3$.
}
\end{ex}

\begin{ex}%[2-D3B3-SO-8-2425]%[VN-MT-7, Lại Thị Hảo]%[2D3H2-2]
Mỗi ngày bác An đều đi bộ để rèn luyện sức khỏe. Quãng đường đi bộ mỗi ngày (đơn vị: km) của bác An trong $20$ ngày được thống kê lại ở bảng sau:
\begin{center}
 \begin{tabular}{|c|c|c|c|c|c|}
\hline
Quãng đường (km) & $[2{,}7; 3{,}0)$ & $[3{,}0; 3{,}3)$ & $[3{,}3; 3{,}6)$ & $[3{,}6; 3{,}9)$ & $[3{,}9; 4{,}2)$\\
\hline
Số ngày & $3$ & $6$ & $5$ & $4$ & $2$ \\
\hline
 \end{tabular}
\end{center}
 Phương sai của mẫu số liệu ghép nhóm là
 \choice
 {$3{,}39$}
 {$11{,}62$}
 {\True $0{,}1314$}
 {$0{,}36$}
\loigiai{
 Ta có bảng sau:
\begin{center}
 \begin{tabular}{|c|c|c|c|c|c|}
\hline
Quãng đường (km) & $[2{,}7; 3{,}0)$ & $[3{,}0; 3{,}3)$ & $[3{,}3; 3{,}6)$ & $[3{,}6; 3{,}9)$ & $[3{,}9; 4{,}2)$\\
\hline
Giá trị đại diện & $2{,}85$ & $3{,}15$ & $3{,}45$ & $3{,}75$ & $4{,}05$ \\
\hline
Số ngày & $3$ & $6$ & $5$ & $4$ & $2$ \\
\hline
 \end{tabular}
\end{center}
 Số trung bình của mẫu số liệu ghép nhóm là
 \[\overline{x}=\dfrac{3 \cdot 2{,}85+6 \cdot 3{,}15+5 \cdot 3{,}45+4 \cdot 3{,}75+2 \cdot 4{,}05}{20}=3{,}39.\]
 Phương sai của mẫu số liệu ghép nhóm là 
 \[s^2=\dfrac{3 (2{,}85-3{,}39)^2+6(3{,}15-3{,}39)^2+5(3{,}45-3{,}39)^2+4(3{,}75-3{,}39)^2+2 (4{,}05-3{,}39)^2}{20}=0{,}1314.\]
}
\end{ex}

\begin{ex}%[2-D3B3-SO-8-2425]%[VN-MT-7, Lại Thị Hảo]%[2D3H2-2]
 Một bác tài xế thống kê lại độ dài quãng đường (đơn vị: km) bác đã lái xe mỗi ngày trong một tháng ở bảng sau:
\begin{center}
 \begin{tabular}{|c|c|c|c|c|c|}
 \hline
 Độ dài quãng đường (km) & $[50; 100)$ & $[100; 150)$ & $[150; 200)$ & $[200; 250)$ & $[250; 300)$\\
 \hline
 Số ngày & $5$ & $10$ & $9$ & $4$ & $2$ \\
 \hline
\end{tabular}
\end{center}
Độ lệch chuẩn của mẫu số liệu ghép nhóm gần bằng
\choice
{$33{,}91$}
{$155{,}15$}
{\True $55{,}68$}
{$36{,}54$}
 \loigiai{
Ta có bảng sau:
\begin{center}
 \begin{tabular}{|c|c|c|c|c|c|}
 \hline
Độ dài quãng đường (km)& $[50; 100)$ & $[100; 150)$ & $[150; 200)$ & $[200; 250)$ & $[250; 300)$\\
 \hline
 Giá trị đại diện & $75$ & $125$ & $175$ & $225$ & $275$ \\
 \hline
 Số ngày & $3$ & $6$ & $5$ & $4$ & $2$\\
 \hline
\end{tabular}
\end{center}
Số trung bình của mẫu số liệu ghép nhóm là
\[\overline{x}=\dfrac{5 \cdot 75+10 \cdot 125+9 \cdot 175+4 \cdot 225+2 \cdot 275}{30}=155.\]
Phương sai của mẫu số liệu ghép nhóm là
\[s^2=\dfrac{5 \cdot (75-155)^2+10 \cdot (125-155)^2+9 \cdot (175-155)^2+4 \cdot (225-155)^2+2 \cdot (275-155)^2}{30}=3\,100.\]
Độ lệch chuẩn của mẫu số liệu ghép nhóm là 
\[s=\sqrt{s^2}=\sqrt{3\,100} \approx 55{,}68.\]
 }
\end{ex}
\Closesolutionfile{ans}

\cauds
\Opensolutionfile{ans}[ans/ans\currfilebase-Phan-II]

\begin{ex}%[2-D3B3-SO-8-2425]%[VN-MT-7, Lại Thị Hảo]%[2D3N1-3]
 Cho bảng số liệu sau:
\begin{center}
 \begin{tabular}{|c|c|c|c|c|c|}
 \hline
 Nhóm & $[20; 25)$ & $[25; 30)$ & $[30; 35)$ & $[35; 40)$ & $[40; 45)$\\
 \hline
 Tần số & $6$ & $6$ & $4$ & $1$ & $1$ \\
 \hline
 \end{tabular}
\end{center}
 \choiceTF
 {\True Khoảng biến thiên của mẫu số liệu ghép nhóm là $25$}
 {\True Tần số của nhóm hai là $6$}
 {Tần số tích lũy của nhóm ba là $4$}
 {Khoảng tứ phân vị của mẫu số liệu ghép nhóm là hiệu giữa tứ phân vị thứ ba và tứ phân vị thứ hai của mẫu số liệu ghép nhóm}
 \loigiai{
\begin{itemchoice}
\itemch \textbf{Đúng}.\\
Khoảng biến thiên của mẫu số liệu ghép nhóm là $R=45-20=25$.
\itemch \textbf{Đúng}.\\
Tần số của nhóm hai (nhóm $[25;30)$) là $6$.
\itemch \textbf{Sai}.\\
Tần số tích lũy của nhóm ba là $6+6+4=16$.
\itemch \textbf{Sai}.\\
Khoảng tứ phân vị của mẫu số liệu ghép nhóm là hiệu giữa tứ phân vị thứ ba và tứ phân vị thứ nhất của mẫu số liệu ghép nhóm.
\end{itemchoice}}
\end{ex}

\begin{ex}%[2-D3B3-SO-8-2425]%[VN-MT-7, Lại Thị Hảo]%[2D3H1-3]
 Một vườn thú ghi lại tuổi thọ (đơn vị: năm) của $20$ con hổ và thu được kết quả như sau:
\begin{center}
 \begin{tabular}{|c|c|c|c|c|c|}
\hline
Tuổi thọ & $[14; 15)$ & $[15; 16)$ & $[16; 17)$ & $[17; 18)$ & $[18; 19)$\\
\hline
Số con hổ & $1$ & $3$ & $8$ & $6$ & $2$ \\
\hline
 \end{tabular}
\end{center}
 \choiceTF
 {\True Khoảng biến thiên của mẫu số liệu ghép nhóm này là $5$}
 {\True Nhóm chứa tứ phân vị thứ nhất là $[16; 17)$}
 {Nhóm chứa tứ phân vị thứ ba là $[18; 19)$}
 {\True Tần số tích lũy của nhóm $[17; 18)$ là $18$}
\loigiai{
\begin{itemchoice}
\itemch \textbf{Đúng}.\\
Khoảng biến thiên $R=19-14=5$.
\itemch \textbf{Đúng}.\\
Cỡ mẫu là $1+3+8+6+2=20$.\\
Gọi $x_1, x_2, \ldots, x_{20}$ là tuổi thọ của $20$ con hổ được sắp xếp theo thứ tự không giảm.\\
Tứ phân vị thứ nhất của mẫu số liệu gốc là $\dfrac{x_5+x_6}{2} \in[16; 17)$ nên nhóm chứa tứ phân vị thứ nhất là $[16; 17)$.
\itemch \textbf{Sai}.\\
Tứ phân vị thứ ba của mẫu số liệu gốc là $\dfrac{x_{15}+x_{16}}{2} \in[17; 18)$. Do đó nhóm chứa tứ phân vị thứ ba là $[17; 18)$.
\itemch \textbf{Đúng}.\\
Tần số tích lũy của nhóm $[17; 18)$ là $1+3+8+6=18$.
\end{itemchoice}
}
\end{ex}

\begin{ex}%[2-D3B3-SO-8-2425]%[VN-MT-7, Lại Thị Hảo]%[1D5H2-2]
Cho mẫu số liệu ghép nhóm về lương của nhân viên trong phòng kế toán tổng hợp một công ty X như sau:
\begin{center}
 \begin{tabular}{|l|c|c|c|c|c|}
\hline
 Lương (triệu đồng) & $[6; 9)$ & $[9; 12)$ & $[12; 15)$ & $[15; 18)$ & $[18; 21)$\\
\hline
Số nhân viên & $6$ & $5$ & $3$ & $2$ & $1$ \\
\hline
\end{tabular}
\end{center}
\choiceTF
{\True Giá trị đại diện của nhóm $[6; 9)$ là $7{,}5$}
{\True Trung bình lương các nhân viên là $11{,}2$ triệu đồng}
{Nhóm chứa trung vị là $[12; 15)$}
{\True Độ dài nhóm $[15; 18)$ là $3$}
\loigiai{
 \begin{itemchoice}
 \itemch \textbf{Đúng}.\\
 Giá trị đại diện của nhóm $[6; 9)$ là $\dfrac{6+9}{2}=7{,}5$.
 \itemch \textbf{Đúng}.\\
 Trung bình lương các nhân viên là
 \[\overline{x}=\dfrac{1}{17}(6\cdot 7{,}5+5\cdot 10{,}5+3\cdot 13{,}5+2\cdot 16{,}5+19{,}5)=11{,}2\, \text{(triệu đồng)}.\]
 \itemch \textbf{Sai}.\\
 Phòng kế toán có $17$ nhân viên. Vì $x_9 \in[9; 12)$ nên nhóm này chứa trung vị.
 \itemch \textbf{Đúng}.\\
 Độ dài nhóm $[15; 18)$ là $18-15=3$.
 \end{itemchoice}
 }
\end{ex}

\begin{ex}%[2-D3B3-SO-8-2425]%[VN-MT-7, Lại Thị Hảo]%[2D3V2-2]
 Cho mẫu số liệu ghép nhóm thống kê chiều cao (đơn vị: cm) của $45$ học sinh lớp 9A như sau:
 \begin{center}
 \begin{tabular}{|c|c|c|c|c|c|}
 \hline
 Nhóm & $[145; 150)$ & $[150; 155)$ & $[155; 160)$ & $[160; 165)$ & $[165; 170)$ \\
 \hline
 Tần số & $8$ & $12$ & $15$ & $6$ & $4$ \\
 \hline
 \end{tabular}
 \end{center}
 \choiceTF
 {Giá trị đại diện của nhóm $[150; 155)$ là $152$\,cm}
 {\True Chiều cao trung bình của học sinh là $155{,}94$\,cm}
 {Phương sai của mẫu số liệu (làm tròn đến hàng phần trăm) là $36{,}04$}
 {\True Độ lệch chuẩn của mẫu số liệu (làm tròn đến hàng phần trăm) là $5{,}85$}
\loigiai{
\begin{itemchoice}
 \itemch \textbf{Sai}.\\
 Giá trị đại diện của nhóm $[150; 155)$ là $\dfrac{150+155}{2}=152{,}5$.
 \itemch \textbf{Đúng}.\\
 Ta có bảng giá trị đại diện như sau:
 \begin{center}
 \begin{tabular}{|l|c|c|}
 \hline
 Nhóm & Giá trị đại diện & Tần số \\
 \hline
 $[145; 150)$ & $147{,}5$ & $8$ \\
 \hline
 $[150; 155)$ & $152{,}5$ & $12$ \\
 \hline
 $[155; 160)$ & $157{,}5$ & $15$ \\
 \hline
 $[160; 165)$ & $162{,}5$ & $6$ \\
 \hline
 $[165; 170)$ & $167{,}5$ & $4$ \\
 \hline
 \end{tabular}
 \end{center}
 Chiều cao trung bình của học sinh là
 \[\overline{x}=\dfrac{147{,}5 \cdot 8+152{,}5\cdot 12+157{,}5\cdot 15+162{,}5\cdot 6+167{,}5\cdot 4}{45}=\dfrac{2\,807}{18} \approx 155{,}94.\]
 \itemch \textbf{Sai}.\\
 Phương sai của mẫu số liệu là 
 \[s^2=\dfrac{8(147{,}5-155{,}94)^2+12(152{,}5-155{,}94)^2+\cdots+4(167{,}5-155{,}94)^2}{45}=\dfrac{2\,774}{81}\approx 34{,}25.\]
 \itemch \textbf{Đúng}.\\
 Độ lệch chuẩn $s=\sqrt{s^2}\approx 5{,}85$.
\end{itemchoice}
 }
\end{ex}
\Closesolutionfile{ans}

\caukq
\Opensolutionfile{ans}[ans/ans\currfilebase-Phan-III]

\begin{ex}%[2-D3B3-SO-8-2425]%[VN-MT-7, Lại Thị Hảo]%[2D3N1-2]
 Cho mẫu số liệu ghép nhóm số tiền điện phải trả trong một tháng của các hộ gia đình ở một khu phố (đơn vị: ngàn đồng) như sau:
\begin{center}
 \begin{tabular}{|l|c|c|c|c|c|c|}
 \hline
 Nhóm & $[375; 450)$ & $[450; 525)$ & $[525; 600)$ & $[600; 675)$ & $[675; 750)$ & $[750; 825]$\\
 \hline
 Tần số & $6$ & $15$ & $10$ & $6$ & $9$ & $4$ \\
 \hline
 \end{tabular}
\end{center}
 Tìm khoảng biến thiên của mẫu số liệu ghép nhóm trên.
 
 \shortans[]{450}
 \loigiai{
 Khoảng biến thiên của mẫu số liệu ghép nhóm trên là $R=a_7-a_1=825-375=450$. 
 }
\end{ex}

\begin{ex}%[2-D3B3-SO-8-2425]%[VN-MT-7, Lại Thị Hảo]%[2D3N1-2]
 Cho mẫu số liệu ghép nhóm về tuổi thọ (đơn vị tính là năm) của một loại bóng đèn mới như sau:
 \begin{center}
 \begin{tabular}{|l|c|c|c|c|}
 \hline
 Tuổi thọ & $[2; 3{,}5)$ & $[3{,}5; 5)$ & $[5; 6{,}5)$ & $[6{,}5; 8)$\\
 \hline
 Số bóng đèn & $8$ & $22$ & $35$ & $15$ \\
 \hline
 \end{tabular}
 \end{center}
 Tìm khoảng biến thiên của mẫu số liệu trên.
 
 \shortans[]{6}
 \loigiai{
 Khoảng biến thiên của mẫu số liệu trên là $8-2=6$. 
 }
\end{ex}

\begin{ex}%[2-D3B3-SO-8-2425]%[VN-MT-7, Lại Thị Hảo]%[2D3N1-4]
Cho bảng tần số ghép nhóm số liệu thống kê chiều cao của $38$ mẫu cây ở một vườn thực vật (đơn vị: centimét) như sau:
\begin{center}
 \begin{tabular}{|c|c|c|c|c|c|c|}
 \hline
 Nhóm & $[30; 40)$ & $[40; 50)$ & $[50; 60)$ & $[60; 70)$ & $[70; 80)$ & \\
 \hline
 Tần số & $4$ & $10$ & $14$ & $6$ & $4$ & $n=38$ \\
 \hline
 \end{tabular}
\end{center}
 Tần số tích luỹ của nhóm $4$ bằng bao nhiêu?
 \par
\shortans[]{34}
 \loigiai{
 Ta có bảng số liệu ghép nhóm như sau:
\begin{center}
 \begin{tabular}{|l|c|c|}
 \hline
 Nhóm & Tần số & Tần số tích lũy \\
 \hline
 $[30; 40)$ & $4$ & $4$ \\
 \hline
 $[40; 50)$ & $10$ & $14$ \\
 \hline
 $[50; 60)$ & $14$ & $28$ \\
 \hline
 $[60; 70)$ & $6$ & $34$ \\
 \hline
 $[70; 80)$ & $4$ & $38$ \\
 \hline
 \end{tabular}
\end{center} 
 Vậy tần số tích luỹ của nhóm $4$ là $34$. 
 }
\end{ex}

\begin{ex}%[2-D3B3-SO-8-2425]%[VN-MT-7, Lại Thị Hảo]%[1D5H1-3]
 Cân nặng của một số quả mít trong một khu vườn được thống kê ở bảng sau:
\begin{center}
 \begin{tabular}{|c|c|c|c|c|c|}
 \hline
 Cân nặng (kg) & $[4; 6)$ & $[6; 8)$ & $[8; 10)$ & $[10; 12)$ & $[12; 14)$\\
 \hline
 Số quả mít & $6$ & $12$ & $19$ & $9$ & $4$ \\
 \hline
 \end{tabular}
\end{center} 
 Tính cân nặng trung bình của một quả mít.
 
 \shortans[]{8{,}72}
 \loigiai{
 Số trung bình cộng của mẫu số liệu ghép nhóm là
 \[\overline{x}=\dfrac{6 \cdot 5+12 \cdot 7+19 \cdot 9+9 \cdot 11+4\cdot 13}{50}=8{,}72.\]
 Vậy cân nặng trung bình của một quả mít là $8{,}72$ kg.
 }
\end{ex}

\begin{ex}%[2-D3B3-SO-8-2425]%[VN-MT-7, Lại Thị Hảo]%[1D5H2-3]
 Để đánh giá chất lượng dịch vụ tài xế công nghệ của hãng X, người ta ghi lại thời gian chờ của các khách hàng được thể hiện trong bảng sau:
\begin{center}
 \begin{tabular}{|c|c|c|c|c|c|}
 \hline
 Thời gian chờ (phút)& $[1; 2{,}5)$ & $[2{,}5; 4)$ & $[4; 5{,}5)$ & $[5{,}5; 7)$ & $[7; 8{,}5)$\\
 \hline
 Lượng khách hàng (tần số) & $10$ & $5$ & $23$ & $6$ & $3$ \\
 \hline
 \end{tabular}
\end{center}
 Tìm tứ phân vị thứ nhất của mẫu số liệu trên (kết quả làm tròn đến hàng phần trăm).
 
 \shortans[]{3{,}03}
 \loigiai{
 Cỡ mẫu là $n=10+5+23+6+3=47$.\\
 Gọi $x_1, \ldots, x_{47}$ là thời gian chờ của $47$ khách hàng và giả sử số liệu gốc này đã được sắp xếp theo thứ tự không giảm.\\
 Tứ phân vị thứ nhất của mẫu số liệu gốc là $x_{12}$ nên nhóm chứa $Q_1$ là nhóm $[2{,}5; 4)$.\\
 Khi đó $Q_1=2{,}5+\dfrac{\dfrac{1 \cdot 47}{4}-10}{5} \cdot 1{,}5=3{,}025\approx 3{,}03$.
 
 }
\end{ex}

\begin{ex}%[2-D3B3-SO-8-2425]%[VN-MT-7, Lại Thị Hảo]%[2D3H2-2]
 Tìm hiểu thời gian sử dụng điện thoại trong một ngày của các bạn học sinh lớp 12A được ghi lại trong bảng sau:
\begin{center}
 \begin{tabular}{|l|c|c|c|c|}
 \hline
 Thời gian (giờ) & $[0; 1{,}5)$ & $[1{,}5; 3)$ & $[3; 4{,}5)$ & $[4{,}5; 6)$\\
 \hline
 Số học sinh & $8$ & $12$ & $6$ & $4$ \\
 \hline
 \end{tabular}
\end{center}
 Tìm phương sai của mẫu số liệu trên.
 
 \shortans[]{2{,}16}
 \loigiai{
 Chọn giá trị đại diện cho các nhóm số liệu, ta có:
\begin{center}
 \begin{tabular}{|l|c|c|c|c|}
 \hline
 Thời gian (giờ) & $[0; 1{,}5)$ & $[1{,}5; 3)$ & $[3; 4{,}5)$ & $[4{,}5; 6)$\\
 \hline
 Giá trị đại diện & $0{,}75$ & $2{,}25$ & $3{,}75$ & $5{,}25$ \\
 \hline
 Số học sinh & $8$ & $12$ & $6$ & $4$ \\ 
 \hline
 \end{tabular}
\end{center}
 Thời gian sử dụng điện thoại trung bình của các bạn lớp 12A là
 \[\overline{x}=\dfrac{1}{30}(8 \cdot 0{,}75+12 \cdot 2{,}25+6 \cdot 3{,}75+4 \cdot 5{,}25)=2{,}55.\]
 Phương sai của mẫu số liệu trên là \[s^2=\dfrac{1}{30}\left(8 \cdot 0{,}75^2+12 \cdot 2{,}25^2+6 \cdot 3{,}75^2+4 \cdot 5{,}25^2\right)-2{,}55^2=2{,}16.\]
 }
\end{ex}
\Closesolutionfile{ans}
\begin{indapan}
	{ans/ans\currfilebase}
\end{indapan}


% \begin{name}
 {Biên soạn:Nguyễn Tài Tuệ \\ Phản biện: Bùi Văn Lợi}
 {Đề ôn tập chương III}
\end{name}

\TN
\Opensolutionfile{ans}[ans/ans\currfilebase-Phan-I]
\begin{ex}%[2-D3B3-SO-9-2425]%[VN-MT-7, Nguyễn Tài Tuệ]%[2D3N1-1]
\immini{Cho mẫu số liệu ghép nhóm được cho trong bảng bên. 
Gọi $ Q_1$, $Q_2$, $Q_3$ lần lượt là tứ phân vị thứ nhất, tứ phân vị thứ hai và tứ phân vị thứ ba của mẫu số liệu. Khoảng tứ phân vị của mẫu số liệu trên là
\choice
{\True $\Delta_Q=Q_3-Q_1$}
{$\Delta_Q=Q_3-Q_2$}
{$\Delta_Q=Q_2-Q_1$}
{$\Delta_Q=Q_3-\dfrac{3}{2}{Q_1}$}}{
\begin{tabular}{|c|c|}
 \hline Nhóm & Tần số \\
 \hline $[a_1; a_2)$ & $n_1$ \\
 \hline$[a_2; a_3)$ & $n_2$ \\
 \hline$\ldots$ & $\ldots$ \\
 \hline$[a_m; a_{m+1})$ & $n_m$\\
 \hline & $n$ \\
 \hline
\end{tabular}}
\loigiai{
Theo định nghĩa $\Delta_Q=Q_3-Q_1$.}
\end{ex}

\begin{ex}%[2-D3B3-SO-9-2425]%[VN-MT-7, Nguyễn Tài Tuệ]%[2D3N1-1]
\immini{Cho mẫu số liệu ghép nhóm được cho trong bảng bên. 
Khoảng biến thiên của mẫu số liệu trên là
\choice
{$R=a_m-a_1$}
{$R=a_{m+1}-a_m$}
{$R=a_{m+1}-a_2$}
{\True $R=a_{m+1}-a_1$}}{

\begin{tabular}{|c|c|}
 \hline Nhóm & Tần số \\
 \hline $[a_1 ; a_2)$ & $n_1$ \\
 \hline $[a_2 ; a_3)$ & $n_2$ \\
 \hline $\ldots$ & $\ldots$ \\
 \hline $[a_m ; a_{m+1})$ & $n_m$\\
 \hline & $n$ \\
 \hline
\end{tabular}}
\loigiai{
Ta có $R=a_{m+1}-a_1$.
}
\end{ex}

\begin{ex}%[2-D3B3-SO-9-2425]%[VN-MT-7, Nguyễn Tài Tuệ]%[2D3N1-2]
\immini{ Bảng thống kê chiều cao của $40$ mẫu cây ở một vườn thực vật (đơn vị: centimét) được cho trong bảng như hình bên. 
Khoảng biến thiên của mẫu số liệu trên bằng
\choice
{\True $R=60$}
{$R=50$}
{$R=70$}
{$R=10$}}{\begin{tabular}{|c|c|c|}
 \hline Nhóm & Tần số & Tần số tích lūy \\
 \hline$[30 ; 40)$ & $4$ & $4$ \\
 $[40 ; 50)$ & $10$ & $14$ \\
 $[50 ; 60)$ & $14$ & $28$ \\
 $[60 ; 70)$ & $6$ & $34$ \\
 $[70 ; 80)$ & $4$ & $38$ \\
 $[80 ; 90)$ & $2$ & $40$ \\
 \hline & $n=40$ & \\
 \hline
\end{tabular}}
\loigiai{
Khoảng biến thiên của mẫu số liệu là $R=90-30=60$.}
\end{ex}

\begin{ex}%[2-D3B3-SO-9-2425]%[VN-MT-7, Nguyễn Tài Tuệ]%[1D5H2-3]
Thời gian (phút) truy bài trước mỗi buổi học của một số học sinh trong một tuần được ghi lại ở bảng sau:
 \begin{center}
\begin{tabular}{|c|c|c|c|c|c|}
\hline
Thời gian & $[9{,}5 ; 12{,}5)$ & $[12{,}5 ; 15{,}5)$ & $[15{,}5 ; 18{,}5)$ & $[18{,}5 ; 21{,}5)$ & $[21{,}5 ; 24{,}5)$ \\
\hline
Số học sinh & $3$ & $12$ & $15$ & $24$ & $2$ \\
\hline
\end{tabular}
\end{center}
Nhóm chứa tứ phân vị thứ nhất là
\choice
{$[9{,}5;12{,}5)$}
{\True $[12{,}5;15{,}5)$}
{$[15{,}5;18{,}5)$}
{$[18{,}5;21{,}5)$}
\loigiai{
Cỡ mẫu $n=56$.\\
Gọi $x_1$, $x_2,\ldots,x_{56} $ là mẫu số liệu gốc về thời gian truy bài trước mỗi buổi học của $56$ số học sinh trong một tuần được xếp theo thứ tự không giảm.\\
Ta có $\dfrac{x_{14}+x_{15}}{2}\in [12{,}5;15{,}5)$, nên nhóm chứa tứ phân vị thứ nhất là $[12{,}5;15{,}5) $.}
\end{ex}

\begin{ex}%[2-D3B3-SO-9-2425]%[VN-MT-7, Nguyễn Tài Tuệ]%[1D5H2-3]
Khảo sát thời gian tập thể dục trong ngày của một số học sinh khối $11$ thu được mẫu số liệu ghép nhóm sau:
\begin{center}
\begin{tabular}{|c|c|c|c|c|c|}
\hline
Thời gian (phút) & $[0 ; 20)$ & $[20 ; 40)$ & $[40 ; 60)$ & $[60 ; 80)$ & $[80 ; 100)$ \\
\hline
Số học sinh & $5$ & $9$ & $12$ & $10$ & $6$ \\
\hline
\end{tabular}
\end{center}
Nhóm chứa tứ phân vị thứ ba là
\choice
{$[20 ;40)$}
{$[40 ;60)$}
{\True $[60 ;80)$}
{$[80 ;100)$}
\loigiai{
Cỡ mẫu $n=42$.\\
Gọi $x_1$, $x_2,\ldots, x_{42} $ là mẫu số liệu gốc về thời gian tập thể dục trong ngày của $42$ học sinh khối $11$ được xếp theo thứ tự không giảm.\\
Tứ phân vị thứ ba của mẫu số liệu gốc là $x_{32}$ thuộc nhóm $[60 ;80)$.}
\end{ex}

\begin{ex}%[2-D3B3-SO-9-2425]%[VN-MT-7, Nguyễn Tài Tuệ]%[1D5H2-3]
Cho mẫu số liệu ghép nhóm về thời gian (phút) đi từ nhà đến nơi làm việc của các nhân viên của một công ty như sau:
\begin{center}
\begin{tabular}{|c|c|c|c|c|c|c|c|}
\hline
Thời gian & $[15 ; 20)$ & $[20 ; 25)$ & $[25 ; 30)$ & $[30 ; 35)$ & $[35 ; 40)$ & $[40 ; 45)$ & $[45 ; 50)$ \\
\hline
Số nhân viên & $7 $& $14$ & $25$ & $37$ & $21$ & $14$ & $10$ \\
\hline
\end{tabular}
\end{center}
Tứ phân vị thứ nhất $Q_1$ và tứ phân vị thứ ba $Q_3$ của mẫu số liệu ghép nhóm này là
\choice
{\True $Q_1=\dfrac{136}{5}$, $Q_3=\dfrac{800}{21}$}
{$Q_1=\dfrac{1360}{37}$, $Q_3=\dfrac{800}{21}$}
{$Q_1=\dfrac{1360}{37}$, $Q_3=\dfrac{3280}{83}$}
{$Q_1=\dfrac{136}{5}$, $Q_3=\dfrac{3280}{83}$}
\loigiai{
Cỡ mẫu $n=128$.\\
Gọi $x_1$, $x_2,\ldots, x_{128} $ là mẫu số liệu gốc về thời gian đi từ nhà đến nơi làm việc của các nhân viên của một công ty được xếp theo thứ tự không giảm.\\
Tứ phân vị thứ nhất của mẫu số liệu gốc là $\dfrac{x_{32}+x_{33}}{2} \in [25;30)$.\\
Do đó, tứ phân vị thứ nhất của mẫu số liệu ghép nhóm là
\[Q_1= 25+ \dfrac{\dfrac{128}{4} - (7+14)}{25} \cdot (30-25)
=\dfrac{136}{5}.\]
Tứ phân vị thứ ba của mẫu số liệu gốc là $\dfrac{x_{96}+x_{97}}{2}\in[35 ;40)$.\\
Do đó, tứ phân vị thứ ba của mẫu số liệu ghép nhóm là
\[ Q_3=35+\dfrac{\dfrac{3\cdot 128}{4}-(7+14+25+37)}{21}\cdot (40-35)
=\dfrac{800}{21}.\]
}
\end{ex}

\begin{ex}%[2-D3B3-SO-9-2425]%[VN-MT-7, Nguyễn Tài Tuệ]%[1D5H2-3]
Thời gian (phút) truy cập Internet mỗi buổi tối của một số học sinh được cho trong bảng sau:
\begin{center}
\begin{tabular}{|l|c|c|c|c|c|}
\hline
Thời gian (phút) & $[9{,}5 ; 12{,}5)$ & $[12{,}5 ; 15{,}5)$ & $[15{,}5 ; 18{,}5)$ & $[18{,}5 ; 21{,}5)$ & $[21{,}5 ; 24{,}5)$ \\
\hline Số học sinh & $3$ & $12$ & $15$ & $24$ & $2$ \\
\hline
\end{tabular}
\end{center}
Tìm tứ phân vị thứ nhất $Q_1$.
\choice
{$Q_1=15$}
{$Q_1=15{,}5$}
{$Q_1=15{,}2$}
{\True $Q_1=15{,}25$}
\loigiai{
Cỡ mẫu là $n=56$.\\
Tứ phân vị thứ nhất $Q_1$ là $\dfrac{x_{14}+x_{15}}{2}$. Do $x_{14}$, $x_{15}$ đều thuộc nhóm $[12{,}5;15{,}5)$ nên nhóm này chứa $Q_1$ .\\
Do đó, $p=2$; ${a_2}=12{,}5$; ${m_2}=12$; ${m_1}=3$, $a_3-a_2=3$ và ta có
\[ Q_1=12{,}5+\dfrac{\dfrac{56}{4}-3}{12}\cdot 3=15{,}25.\]
}
\end{ex}

\begin{ex}%[2-D3B3-SO-9-2425]%[VN-MT-7, Nguyễn Tài Tuệ]%[1D5H1-3]
Thống kê cân nặng của học sinh lớp 11A cho trong bảng dưới đây:
\begin{center}
\begin{tabular}{|c|c|c|c|c|c|c|}
\hline
Cân nặng & $[40{,}5 ; 45{,}5)$ & $[45{,}5 ; 50{,}5)$ & $[50{,}5 ; 55{,}5)$ & $[55{,}5 ; 60{,}5)$ & $[60{,}5 ; 65{,}5)$ & $[65{,}5 ; 70{,}5)$\\
\hline
Số học sinh & $10$ & $7$ & $16$ & $4$ & $2$ & $3$ \\
\hline
\end{tabular}
\end{center}
Tính cân nặng trung bình của học sinh lớp 11A (kết quả làm tròn đến hàng phần trăm).
\choice
{$50{,}1$}
{$52{,}83$}
{$50{,}81$}
{\True $51{,}81$}
\loigiai{
Trong mỗi khoảng cân nặng, giá trị đại diện là trung bình cộng của giá trị hai đầu mút nên ta có bảng sau:
\begin{center}
\begin{tabular}{|c|c|c|c|c|c|c|}
\hline
Cân nặng (kg) & $43$ & $48$ & $53$ & $58$ & $63$ & $68$ \\
\hline
Số học sinh & $10$ & $7$ & $16$ & $4$ & $2$ & $3$ \\
\hline
\end{tabular}
\end{center}
Tổng số học sinh là $n=42 $.\\
Cân nặng trung bình của học sinh lớp 11A là \\
\[ \overline{x}=\dfrac{10\cdot 43+7\cdot 48+16\cdot 53+4\cdot 58+2\cdot 63+3\cdot 68}{42}\approx 51{,}81~\text{(kg).}\]
}
\end{ex}

\begin{ex}%[2-D3B3-SO-9-2425]%[VN-MT-7, Nguyễn Tài Tuệ]%[2D3H2-2]
\immini{Tìm phương sai của một mẫu số liệu ghép nhóm cho bởi bảng thống kê như hình bên.
\choice
{$ 13{,}24$}
{$15{,}74$}
{$18{,}84$}
{\True $14{,}84$}}{\begin{tabular}{|c|c|c|}
 \hline Lớp chiều cao & Giá trị đại diện & Tần số \\
 \hline$[150 ; 154)$ & $152$ & $25$ \\
 \hline$[154 ; 158)$ & $156$ & $50$ \\
 \hline$[158 ; 162)$ & $160$ & $200$ \\
 \hline$[162 ; 166)$ & $164$ & $175$ \\
 \hline$[166 ; 170)$ & $168$ & $50$ \\
 \hline
\end{tabular}}
\loigiai{
Ta có chiều cao trung bình \\
\[\overline x=\dfrac{1}{500}(152\cdot 25+156\cdot 50+160\cdot 200+164\cdot 175+168\cdot 50)=161{,}4.\] 
Phương sai của mẫu số liệu ghép nhóm là
\begin{align*}
 s^2=\dfrac{1}{500}& \left[25(152-161{,}4)^2+50(156-161{,}4)^2+200(160-161{,}4)^2+175(164-161{,}4)^2\right. \\
 &\left.+50(168-161{,}4)^2\right] = 14{,}84.
\end{align*}
}
\end{ex}

\begin{ex}%[2-D3B3-SO-9-2425]%[VN-MT-7, Nguyễn Tài Tuệ]%[2D3H2-2]
Kết quả khảo sát thời gian sử dụng liên tục (đơn vị: giờ) từ lúc sạc đầy cho đến khi hết pin của một số máy vi tính cùng loại được thống kê ở bảng sau:
\begin{center}
\begin{tabular}{|c|c|c|c|c|}
\hline
Thời gian sử dụng & $[7{,}2 ; 7{,}4)$ & $[7{,}4 ; 7{,}6)$ & $[7{,}6 ; 7{,}8)$& $[7{,}8 ; 8{,}0)$ \\
\hline
Số máy & $2$ & $4$ & $7$ & $6$ \\
\hline
\end{tabular}
\end{center}
Tính độ lệch chuẩn của mẫu số liệu ghép nhóm (kết quả làm tròn đến hàng phần nghìn).
\choice
{$0{,}192$}
{\True $0{,}194$}
{$0{,}037$}
{$0{,}2$}
\loigiai{
Từ bảng thống kê ta có
\begin{center}
\begin{tabular}{|c|c|c|c|c|}
\hline
Thời gian sử dụng & $[7{,}2 ; 7{,}4)$ & $[7{,}4 ; 7{,}6)$ & $[7{,}6 ; 7{,}8)$ & $[7{,}8 ; 8{,}0)$ \\
\hline
Giá trị đại diện & $7{,}3$ & $7{,}5$ & $7{,}7$ & $7{,}9$ \\
\hline
Số máy & $2$ & $4$ & $7$ & $6$ \\
\hline
\end{tabular}
\end{center}
\noindent
Tổng số máy $n=2+4+7+6=19$.\\
Thời gian sử dụng trung bình của pin là $\overline x=\dfrac{2\cdot 7{,}3+4\cdot 7{,}5+7\cdot 7{,}7+6\cdot 7{,}9}{19}=\dfrac{1459}{190}$.\\
Phương sai của mẫu số liệu là $s^2=\dfrac{1}{19}(2\cdot 7{,}3^2+4\cdot 7{,}5^2+7\cdot 7{,}7^2+6\cdot 7{,}9^2)-\left (\dfrac{1459}{190}\right )^2=\dfrac{338}{9025}$.\\
Độ lệch chuẩn của mẫu số liệu là $s=\sqrt{s^2}=\sqrt{\dfrac{ 338}{9025}} =\dfrac{13 \sqrt{2}}{95}\approx 0,194 $.}
\end{ex}

\begin{ex}%[2-D3B3-SO-9-2425]%[VN-MT-7, Nguyễn Tài Tuệ]%[2D3N1-1]
Đại lượng nào đo độ phân tán của nửa giữa của mẫu số liệu, không bị ảnh hưởng nhiều bởi các giá trị ngoại lệ trong mẫu số liệu?
\choice
{Khoảng biến thiên}
{\True Khoảng tứ phân vị}
{Phương sai}
{Độ lệch chuẩn}
\loigiai{
Khoảng tứ phân vị dùng để đo độ phân tán của nửa giữa của mẫu số liệu, không bị ảnh hưởng nhiều bởi các giá trị ngoại lệ trong mẫu số liệu.}
\end{ex}

\begin{ex}%[2-D3B3-SO-9-2425]%[VN-MT-7, Nguyễn Tài Tuệ]%[2D3N2-1]
Để so sánh mức độ phân tán của các mẫu số liệu ghép nhóm có cùng số trung bình ta dùng đại lượng nào?
\choice
{Khoảng biến thiên}
{Khoảng tứ phân vị}
{Trung vị}
{\True Độ lệch chuẩn}
\loigiai{
Để so sánh mức độ phân tán của các mẫu số liệu ghép nhóm có cùng số trung bình ta dùng phương sai và độ lệch chuẩn.
}
\end{ex}
\Closesolutionfile{ans}

\TNTF
\Opensolutionfile{ans}[ans/ans\currfilebase-Phan-II]
\begin{ex}%[2-D3B3-SO-9-2425]%[VN-MT-7, Nguyễn Tài Tuệ]%[2D3H1-2]
Mẫu số liệu dưới đây ghi lại tốc độ của $40$ ô tô khi đi qua một trạm đo tốc độ (đơn vị: km/h):
\begin{center}
\begin{tabular}{llllllllll}
$48{,}5$ & $43$ & $50$ & $55$ & $45$ & $60$ & $53$ & $55{,}5$ & $44$ & $65$ \\
$51$ & $62{,}5$ & $41$ & $44{,}5$ & $57$ & $57$ & $68$ & $49$ & $46{,}5$ & $53{,}5$ \\
$61$ & $49{,}5$ & $54$ & $62$ & $59$ & $56$ & $47$ & $50$ & $60$ & $61$ \\
$49{,}5$ & $52{,}5$ & $57$ & $47$ & $60$ & $55$ & $45$ & $47{,}5$ & $48$ & $61{,}5$
\end{tabular}
\end{center}
\choiceTF
{\True Bảng tần số ghép nhóm cho mẫu số liệu trên có sáu nhóm ứng với sáu nửa khoảng là: 
\centerline{
\begin{tabular}{|c|c|c|c|c|c|c|c|}
 \hline
 Nhóm & $[40 ; 45)$ & $[45 ; 50)$ & $[50 ; 55)$ & $[55 ; 60)$ & $[60 ; 65)$ & $[65 ; 70)$ & \\
 \hline
 Tần số & $4$ & $11$ & $7$ & $8$ & $8$ & $2$ & $n=40$ \\
 \hline
\end{tabular}
}}
{Mẫu số liệu trên có số trung bình là $54{,}875$}
{\True Tứ phân vị của mẫu số liệu trên là $Q_1=47{,}8~\text{(km/h)}$; ${Q_2}=53{,}6 ~\text{(km/h)}$; ${Q_3}=60~\text{(km/h)}$}
{Khoảng biến thiên của mẫu số liệu trên là $25$}
\loigiai{
\begin{itemchoice}
\itemch {\bf Đúng}.\\
Bảng tần số ghép nhóm: 
\begin{center}
\begin{tabular}{|c|c|c|c|c|c|c|c|}
 \hline
 Nhóm & $[40 ; 45)$ & $[45 ; 50)$ & $[50 ; 55)$ & $[55 ; 60)$ & $[60 ; 65)$ & $[65 ; 70)$ & \\
 \hline
 Tần số & $4$ & $11$ & $7$ & $8$ & $8$ & $2$ & $n=40$ \\
 \hline
\end{tabular}
\end{center}
Vậy bảng tần số đã cho đúng.
\itemch {\bf Sai}.\\
Số trung bình
\[
\overline x= \dfrac{4\cdot 42{,}5+11\cdot 47{,}5+7\cdot 52{,}5+8\cdot 57{,}5+8\cdot 62{,}5+2\cdot 67 \cdot 5}{40}= 53{,}875~\text{(km/h).}
\]
\itemch \textbf{Đúng}.\\
Cỡ mẫu là $n=40$.\\
Ta có $\dfrac{n}{2}=20$ nên nhóm $3$ là nhóm đầu tiên có tần số tích lũy lớn hơn hoặc bằng $20$.
Xét nhóm $3$ là nhóm $[50;55)$ có $r=50$, $d=5$, $n_3=7$ và $c{f_2}=15$.
Số trung vị của mẫu số liệu là
\[
M_e=50+\dfrac{20-15}{7}\cdot 5\approx 53{,}6 ~\text{(km/h)}.
\]
Ta có $\dfrac{n}{4}=10$ nên nhóm $2$ là nhóm đầu tiên có tần số tích lũy lớn hơn hoặc bằng $10$.\\
Xét nhóm $2$ là nhóm $[45;50)$ có $r=45$, $d=5$, $n_2=11$ và $cf_1=4$.\\
Tứ phân vị thứ nhất là 
\[
Q_1=45+\dfrac{10-4}{11}\cdot 5\approx 47{,}8~\text{(km/h).}\]
Ta có $\dfrac{3n}{4}=30$ nên nhóm $4$ là nhóm đầu tiên có tần số tích lũy lớn hơn hoặc bằng $30$.\\
Xét nhóm $4$ là nhóm $[(60)$ có $r=55$, $d=5$, $n_4=8$ và $c{f_3}=22$.\\
Tứ phân vị thứ ba là
\[ Q_3=55+\dfrac{30-22}{8}\cdot 5=60~\text{(km/h).} \]
Vậy các tứ phân vị của mẫu số liệu trên là
$Q_1=47{,}8$ (km/h); ${Q_2}=53{,}6$ (km/h); $ Q_3 =60$ (km/h).
\itemch {\bf Sai}.\\
Khoảng biến thiên của mẫu số liệu là
$R=70-40=30$.
\end{itemchoice}
}
\end{ex}

\begin{ex}%[2-D3B3-SO-9-2425]%[VN-MT-7, Nguyễn Tài Tuệ]%[2D3H2-2]
\immini{Bảng bên cho ta bảng tần số ghép nhóm số liệu thống kê cân nặng của $40$ học sinh lớp 12B trong một trường trung học phổ thông (đơn vị: kilôgam). 
Các mệnh đề sau \textbf{đúng} hay \textbf{sai}?
\choiceTF
{\True Số học sinh nặng dưới $50$ (kg) là $12$}
{\True Mốt của mẫu số liệu ghép nhóm trên xấp xỉ bằng $54{,}29$ (kg)}
{Khoảng tứ phân vị của mẫu số liệu ghép nhóm trên là $\dfrac{39}{2}$}
{Phương sai của mẫu số liệu ghép nhóm là $128$}}{
\begin{tabular}{|c|c|}
 \hline
 Nhóm & Số học sinh\\
 \hline $[30 ;40)$ & $2$\\
 \hline $[40 ;50)$ & $10$\\
 \hline $[50 ;60)$ & $16$\\
 \hline $[60 ;70)$ & $8$\\
 \hline $[70 ;80)$ & $2$\\
 \hline $[80 ;90)$ & $2$\\
 \hline & $n=40$\\
 \hline
\end{tabular}}
\loigiai{
\begin{itemchoice}
\itemch {\bf Đúng}.\\
Số học sinh nặng dưới $50$ kg là $2+10=12$. 
\itemch {\bf Đúng}.\\
Nhóm chứa mốt của mẫu số liệu là $[50 ;60)$.\\
Do đó $u_m=50$, $n_m=16$, $n_{m-1}=10$, $n_{m+1}=8$, $u_{m+1}-u_m=60-50=10$.\\
Mốt của mẫu số liệu ghép nhóm là 
\[
M_\text{o}=50+\dfrac{16-10}{(16-10)+(16-8)}\cdot 10=\dfrac{380}{7}\approx 54{,}29~\text{(kg).}
\]
Mốt của mẫu số liệu ghép nhóm trên xấp xỉ bằng $54{,}29$ (kg). 
\itemch \textbf{Sai}.\\
Cỡ mẫu $n=40$.\\
Gọi $x_1, x_2 \in [30 ;40)$; $x_3,\,\ldots,x_{12} \in [40 ;50)$; $x_{13},\,...,x_{28}\in [50 ;60);$ $x_{29},\,...,x_{36}\in [60 ;70);$ $x_{37},x_{38}\in [70 ;80);$ $x_{39},x_{40}\in [80 ;90) $.\\
Tứ phân vị thứ nhất của mẫu số liệu gốc là
$ \dfrac{1}{2}(x_{10}+x_{11})\in[40 ;50)$. 
Do đó, tứ phân vị thứ nhất của mẫu số liệu ghép nhóm là 
\[
Q_1=40+\dfrac{\dfrac{40}{4}-2}{10} \cdot (50-40)=48.
\]
Tứ phân vị thứ ba của mẫu số liệu gốc là
$\dfrac{1}{2}(x_{30}+x_{31})\in[60 ;70)$.
Do đó, tứ phân vị thứ ba của mẫu số liệu ghép nhóm là 
\[
Q_3=60+\dfrac{\dfrac{3\cdot 40}{4}-(2+10+16)}{8}\cdot (70-60)=\dfrac{125}{2}.
\]
Vậy khoảng tứ phân vị của mẫu số liệu ghép nhóm là 
\[
\Delta_Q=\dfrac{125}{2}-48=\dfrac{29}{2}.
\]
\itemch {\bf Sai}. \\ Ta có bảng cân nặng của các em học sinh theo giá trị đại diện:
\begin{center}
\begin{tabular}{|c|c|c|}
\hline
Nhóm & Giá trị đại diện & Tần số\\
\hline $[30 ;40)$ & $35$ & $2$\\
\hline $[40 ;50)$ & $45$ & $10$\\
\hline $[50 ;60)$ & $55$ & $16$\\
\hline $[60 ;70)$ & $65$ & $8$\\
\hline $[70 ;80)$ & $75$ & $2$\\
\hline $[80 ;90)$ & $85$ & $2$\\
\hline & & $n=40$\\
\hline
\end{tabular}
\end{center}
Cỡ mẫu $n=2+10+16+8+2+2=40 $.\\
Số trung bình của mẫu số liệu ghép nhóm là 
\[
\dfrac{35\cdot 2+45\cdot 10+55\cdot 16+65\cdot 8+75\cdot 2+85\cdot 2}{40}=\dfrac{2240}{40}=56 \text{ (kg).}
\]
Phương sai của mẫu số liệu ghép nhóm là 
\[
s^2=\dfrac{1}{40}\left(2\cdot 35^2+10\cdot 45^2+16\cdot 55^2+8\cdot 65^2+2\cdot 75^2+2\cdot 85^2\right)-56^2=3265-3136=129.
\]
\end{itemchoice}
}
\end{ex}

\begin{ex}%[2-D3B3-SO-9-2425]%[VN-MT-7, Nguyễn Tài Tuệ]%[2D3H2-2]
Trong một hội thao, thời gian chạy $200$\,m của một nhóm các vận động viên được ghi lại ở bảng sau:
\begin{center}
\begin{tabular}{|c|c|c|c|c|c|}
\hline
Thời gian (giây) & $[21 ; 21{,}5)$ & $[21{,}5 ; 22)$ & $[22 ; 22{,}5)$ & $[22{,}5 ; 23)$ & $[23 ; 23{,}5)$\\
\hline
Số vận động viên & $5$ & $10$ & $30$ & $45$ & $30$\\
\hline
\end{tabular}
\end{center}
\choiceTF 
{Tần suất của nhóm vận động viên chạy trong khoảng thời gian từ $22$ giây đến dưới $22{,}5$ giây bằng $30\% $}
{\True Số trung vị của mẫu số liệu (làm tròn đến chữ số thập phân thứ $2$) bằng $22{,}67$}
{Khoảng biến thiên của mẫu số liệu bằng $R=2$}
{Độ lệch chuẩn của mẫu số liệu (làm tròn đến chữ số thập phân thứ $2$) bằng $0{,}28$}
\loigiai{
\begin{itemchoice}
\itemch {\bf Sai}.\\ 
Cỡ mẫu $n=5+10+30+45+30=120$.\\
Tần suất của nhóm vận động viên chạy trong khoảng thời gian từ $22$ giây đến dưới $22{,}5$ giây bằng $f_3=\dfrac{n_3}{n}=\dfrac{30}{120}=25\% $.
\itemch {\bf Đúng}.\\
Gọi $x_1$, $x_2$, $ \ldots$, $x_{120}$ là thời gian chạy của $120$ vận động viên và dãy này là một dãy không giảm.\\
Khi đó trung vị là
$\dfrac{x_{60}+x_{61}}{2}$. Do $x_{60},\,x_{61}\in[22{,}5;\,23)$ nên nhóm này chứa trung vị.\\
Ta có 
$M_\text{e}=22{,}5+\dfrac{\dfrac{120}{2}-(5+10+30)}{45}\cdot(23-22{,}5)\approx 22{,}67$.
\itemch {\bf Sai}. \\
Khoảng biến thiên của mẫu số liệu bằng $R=23{,}5-21=2{,}5$.
\itemch {\bf Sai}. \\
Giá trị trung bình của mẫu số liệu là
\[
\overline x=\dfrac{5\cdot 21{,}25+10\cdot 21{,}75+30\cdot 22{,}25+45\cdot 22{,}75+30\cdot 23{,}25}{120}\approx 22{,}60.
\]
Độ lệch chuẩn của mẫu số liệu (làm tròn đến chữ số thập phân thứ $2$) bằng
\[
s=\sqrt{\dfrac{5(-1{,}35)^2+10(-0{,}85)^2+30(-0{,}35)^2+45(0{,}15)^2+30(0{,}65)^2}{120}}\approx 0{,}53.
\]
\end{itemchoice}
}
\end{ex}

\begin{ex}%[2-D3B3-SO-9-2425]%[VN-MT-7, Nguyễn Tài Tuệ]%[2D3V1-3]
Giả sử kết quả khảo sát hai khu vực A và B về độ tuổi kết hôn của một số phụ nữ vừa lập gia đình được cho ở bảng sau:
\begin{center}
\begin{tabular}{|c|c|c|c|c|c|}
\hline
Tuổi kết hôn & $[19;22)$ & $[22;25)$ & $[25;28)$ & $[28;31)$ & $[31;34)$\\
\hline
Số phụ nữ khu vực A & $10$ & $27$ & $31$ & $25$ & $7$\\
\hline
Số phụ nữ khu vực B & $47$ & $40$ & $11$ & $2$ & $0$\\
\hline
\end{tabular}
\end{center}
\choiceTF
{Khoảng biến thiên của mẫu số liệu ghép nhóm ứng với khu vực A là $15$ (tuổi)}
{Khoảng biến thiên của mẫu số liệu ghép nhóm ứng với khu vực B là $12$ (tuổi)}
{Khoảng tứ phân vị của mẫu số liệu ghép nhóm ứng với khu vực A là $\dfrac{61}{3}$ (tuổi)}
{Nếu so sánh theo khoảng tứ phân vị thì phụ nữ ở khu vực B có độ tuổi kết hôn đồng đều hơn}
\loigiai{
\begin{itemchoice}
\itemch \textbf{Đúng}.\\ 
Khoảng biến thiên của mẫu số liệu ghép nhóm ứng với khu vực A là $34-19=15$ (tuổi). 
\itemch \textbf{Đúng}. \\
Khoảng biến thiên của mẫu số liệu ghép nhóm ứng với khu vực B là $31-19=12$ (tuổi)
\itemch \textbf{Sai}. \\ 
Cỡ mẫu $n=100$.\\
Gọi $x_1$, $x_2$, $\ldots$, $x_{100} $ là mẫu số liệu gốc về độ tuổi kết hôn của phụ nữ ở khu vực A được xếp theo thứ tự không giảm.\\
Ta có $x_1$, $x_2$, $\ldots$, $x_{10}\in[19;22)$; ${x_{11}}$, $\ldots$, $ x_{37} \in[22;25)$; $x_{38},\ldots, x_{68} \in[25;28)$; $x_{69},\ldots,x_{93} \in[28;31)$; $x_{94},\ldots, x_{100}\in[31;34)$.\\
Tứ phân vị thứ nhất của mẫu số liệu gốc là
$\dfrac{1}{2}(x_{25}+x_{26})\in[22;25)$. Do đó, tứ phân vị thứ nhất của mẫu số liệu ghép nhóm là $Q_1=22+\dfrac{\dfrac{100}{4}-10}{27}(25-22)=\dfrac{71}{3}$.\\
Tứ phân vị thứ ba của mẫu số liệu gốc là
$\dfrac{1}{2}(x_{75}+x_{76})\in[28;31)$. Do đó, tứ phân vị thứ ba của mẫu số liệu ghép nhóm là $Q_3=28+\dfrac{\dfrac{3\cdot 100}{4}-(10+27+31)}{25}(31-28)=\dfrac{721}{25}$.\\
Khoảng tứ phân vị của mẫu số liệu ghép nhóm là $\Delta_Q=Q_3-Q_1=\dfrac{388}{75}$.
\itemch \textbf{Đúng}.\\
Gọi $y_1$, $y_2,\ldots,y_{100} $ là mẫu số liệu gốc về độ tuổi kết hôn của phụ nữ ở khu vực B được xếp theo thứ tự không giảm.\\
Ta có $y_1$, $y_2,\ldots,y_{47} \in [19;22)$; $ y_{48},\ldots, y_{87} \in[22;25)$; $y_{88}, \ldots, y_{98} \in[25;30)$;$y_{99}$, ${y_{100}}\in[28;31)$.\\
Tứ phân vị thứ nhất của mẫu số liệu gốc là $\dfrac{1}{2}(y_{25}+y_{26})\in[19;22)$.
Do đó, tứ phân vị thứ nhất của mẫu số liệu ghép nhóm là
\[
Q_1^\prime=19+\dfrac{\dfrac{100}{4}}{47}(22-19)=\dfrac{968}{47}
\]
Tứ phân vị thứ ba của mẫu số liệu gốc là $\dfrac{1}{2}(y_{75}+y_{76})\in[22;25)$. Do đó, tứ phân vị thứ ba của mẫu số liệu ghép nhóm là
\[
Q_3^\prime=22+\dfrac{\dfrac{3\cdot 100}{4}-47}{40}(25-22)=\dfrac{241}{10}
\]
Có $\Delta_Q^\prime < \Delta_Q$ nên phụ nữ ở khu vực B có độ tuổi kết hôn đồng đều hơn.
\end{itemchoice}
}
\end{ex}
\Closesolutionfile{ans}

\TNSA
\Opensolutionfile{ans}[ans/ans\currfilebase-Phan-III]
\begin{ex}%[2-D3B3-SO-9-2425]%[VN-MT-7, Nguyễn Tài Tuệ]%[2D3N1-2]
Thời gian tập luyện trong một ngày (tính theo giờ) của một số vận động viên được ghi lại ở bảng sau:
\begin{center}
\begin{tabular}{|c|c|c|c|c|c|}
\hline
Thời gian tập luyện &$[0; 2)$ &$[2; 4)$ &$[4; 6)$ &$[6; 8)$ &$[8; 10)$\\
\hline
Số vận động viên & $3$ & $8$ & $12$ & $12$ & $4$\\
\hline
\end{tabular}
\end{center}
Hãy tìm khoảng biến thiên cho thời gian tập luyện của các vận động viên.
\shortans[]{10}
\loigiai{
Gọi $R$ là khoảng biến thiên của mẫu số liệu ghép nhóm về thời gian tập luyện trong ngày của các vận động viên. Ta có $R=10-0=10$.}
\end{ex}

\begin{ex}%[2-D3B3-SO-9-2425]%[VN-MT-7, Nguyễn Tài Tuệ]%[2D3H1-3]
Một trang báo điện tử thống kê thời gian người sử dụng đọc thông tin trên trang trong mỗi lần truy cập ở bảng sau:\\
\begin{center}
\begin{tabular}{|c|c|c|c|c|c|}
\hline
Thời gian đọc (phút) & $[0; 2)$ &$[2; 4)$ &$[4; 6)$ &$[6; 8)$ &$[8; 10)$\\
\hline
Số lượt truy cập & $45$ & $34$ & $23$ & $18$ & $5$\\
\hline
\end{tabular}
\end{center}
Hãy tìm khoảng tứ phân vị của mẫu số liệu ghép nhóm trên.
\shortans[]{3{,}89}
\loigiai{
Cỡ mẫu là $n=45+34+23+18+5=125$.\\
Gọi $x_1$, $x_2$,\ldots, $x_{125}$ là thời gian đọc thông tin trên trang báo điện tử của $125$ lượt truy cập và giả sử rằng dãy số liệu gốc này đã được sắp xếp theo thứ tự tăng dần.\\
Tứ phân vị thứ nhất của mẫu số liệu gốc là $\dfrac{1}{2}(x_{31}+x_{32})$ nên nhóm chứa tứ phân vị thứ nhất là nhóm $[0; 2)$.\\
Tứ phân vị thứ nhất của mẫu số liệu ghép nhóm là
\[
Q_1 = 0+\dfrac{\dfrac{1\cdot 125}{4}-0}{45}\cdot (2-0)\approx 1{,}39.
\]
Tứ phân vị thứ ba của mẫu số liệu gốc là $\dfrac{1}{2}(x_{94}+x_{95})$ nên nhóm chứa tứ phân vị thứ nhất là nhóm $[4; 6)$.\\
Tứ phân vị thứ ba của mẫu số liệu ghép nhóm là
\[ Q_3=4+\dfrac{\dfrac{3\cdot 125}{4}-(45+34)}{23}\cdot (6-4) \approx 5{,}28.
\]
Vậy khoảng tứ phân vị của mẫu số liệu ghép nhóm là
\[
\Delta_Q=Q_3-Q_1\approx 5{,}28-1{,}39=3{,}89.
\]
}
\end{ex}

\begin{ex}%[2-D3B3-SO-9-2425]%[VN-MT-7, Nguyễn Tài Tuệ]%[2D3H2-2]
Người ta ghi lại tiền lãi (đơn vị: triệu đồng) của một số nhà đầu tư (với số tiền đầu tư như nhau), khi đầu tư vào hai lĩnh vực A, B cho kết quả như sau:
\begin{center}
\begin{tabular}{|c|c|c|c|c|c|}
\hline
Tiền lãi & $[5;10)$ & $[10;15)$ & $[15;20)$ & $[20;25)$ & $[25;30)$\\
\hline
Số nhà đầu tư vào lĩnh vực A & $2$ & $5$ & $8$ & $6$ & $4$\\
\hline
Số nhà đầu tư vào lĩnh vực B & $8$ & $4$ & $2$ & $5$ & $6$\\
\hline
\end{tabular}
\end{center} 
Tính hiệu phương sai $s_B^2-s_A^2$ cho các mẫu số liệu về tiền lãi của các nhà đầu tư ở hai lĩnh vực này. \par
\shortans[]{47{,}7}
\loigiai{
Ta có mẫu số liệu ghép nhóm với giá trị đại diện là:
\begin{center}
\begin{tabular}{|c|c|c|c|c|c|}
\hline
Tiền lãi & $[5;10)$ & $[10;15)$ & $[15;20)$ & $[20;25)$ & $[25;30)$\\
\hline
Giá trị đại diện & $7{,}5$ & $12{,}5$ & $17{,}5$ & $22{,}5$ & $27{,}5$\\
\hline
Số nhà đầu tư vào lĩnh vực A & $2$ & $5$ & $8$ & $6$ & $4$\\
\hline
Số nhà đầu tư vào lĩnh vực B & $8$ & $4$ & $2$ & $5$ & $6$\\
\hline
\end{tabular}
\end{center}
Tiền lãi trung bình khi đầu tư vào lĩnh vực A là
\[
\overline{x}_A=\dfrac{7{,}5\cdot 2+12{,}5\cdot 5+17{,}5\cdot 8+22{,}5\cdot 6+27{,}5\cdot 4}{2+5+8+6+4}
=18{,}5.
\]
Tiền lãi trung bình khi đầu tư vào lĩnh vực B là
\[
\overline{x}_B=\dfrac{7{,}5\cdot 8+12{,}5\cdot 4+17{,}5\cdot 2+22{,}5\cdot 5+27{,}5\cdot 6}{8+4+2+5+6}
=16{,}9.
\]
Phương sai của mẫu số liệu về tiền lãi khi đầu tư vào lĩnh vực A là
\[
s_A^2=\dfrac{1}{25}\left(7{,}5^2\cdot 2+12{,}5^2\cdot 5+17{,}5^2\cdot 8+22{,}5^2\cdot 6+27{,}5^2\cdot 4\right)-18{,}5^2
=34.
\]
Phương sai của mẫu số liệu về tiền lãi khi đầu tư vào lĩnh vực B là \\
\[
s_B^2=\dfrac{1}{25}\left(7{,}5^2\cdot 8+12{,}5^2\cdot 4+17{,}5^2\cdot 2+22{,}5^2\cdot 5+27{,}5^2\cdot 6\right)-16{,}9^2
=64{,}64.
\]
Do đó $s_B^2-s_A^2 = 47{,}7$.
}
\end{ex}

\begin{ex}%[2-D3B3-SO-9-2425]%[VN-MT-7, Nguyễn Tài Tuệ]%[2D3H2-2]
Thời gian hoàn thành một bài kiểm tra trắc nghiệm của một số học sinh lớp $10$ của hai lớp 10A và 10B được ghi lại ở bảng sau:
\begin{center}
\begin{tabular}{|c|c|c|c|c|c|}
\hline
Thời gian (phút) & $[6;7)$ & $[7;8)$ & $[8;9)$ & $[9;10)$ & $[10;11)$\\
\hline
Học sinh lớp $10A$ & $8$ & $10$ & $13$ & $10$ & $9$\\
\hline
Học sinh lớp $10B$ & $4$ & $12$ & $17$ & $14$ & $3$\\
\hline
\end{tabular}
\end{center} 
Tính hiệu độ lệch chuẩn $s_{10A}-s_{10B}$ (kết quả làm tròn đến hàng phần trăm).
\shortans[]{0{,}29}
\loigiai{
Lập lại mẫu số liệu ghép nhóm theo giá trị đại diện, ta được:
\begin{center}
\begin{tabular}{|c|c|c|c|c|c|}
\hline
Giá trị đại diện & $6{,}5$ & $7{,}5$ & $8{,}5$ & $9{,}5$ & $10{,}5$\\
\hline
Học sinh lớp $10A$ & $8$ & $10$ & $13$ & $10$ & $9$\\
\hline
Học sinh lớp $10B$ & $4$ & $12$ & $17$ & $14$ & $3$\\
\hline
\end{tabular}
\end{center}
Cỡ mẫu $n=50$.
\begin{enumerate}
\item Xét số liệu của lớp $10A$.\\
Số trung bình là
$\overline x_{10A}=\dfrac{8\cdot 6{,}5+10\cdot 7{,}5+13\cdot 8{,}5+10\cdot 9{,}5+9\cdot 10{,}5}{50}=8{,}54$.\\
Độ lệch chuẩn là
$s_{10A}=\sqrt{\dfrac{8\cdot 6{,}5^2+10\cdot 7{,}5^2+13\cdot 8{,}5^2+10\cdot 9{,}5^2+9\cdot 10{,}5^2}{50}-8{,}54^2}\approx 1{,}33$.
\item Xét số liệu của lớp $10B$.\\
Số trung bình là
$\overline x_{10B}=\dfrac{4\cdot 6{,}5+12\cdot 7{,}5+17\cdot 8{,}5+14\cdot 9{,}5+3\cdot 10{,}5}{50}=8{,}5$.\\
Độ lệch chuẩn là $s_{10B}=\sqrt{\dfrac{4\cdot 6{,}5^2+12\cdot 7{,}5^2+17\cdot 8{,}5^2+14\cdot 9{,}5^2+3\cdot 10{,}5^2}{50}-8{,}5^2}\approx 1{,}04$.
\end{enumerate}
Do đó $s_{10A}-s_{10B}\approx 0{,}29$.
}
\end{ex}

\begin{ex}%[2-D3B3-SO-9-2425]%[VN-MT-7, Nguyễn Tài Tuệ]%[2D3H2-2]
Giá đóng cửa của một cổ phiếu là giá của cổ phiếu đó cuối một phiên giao dịch. Bảng sau thống kê giá đóng cửa (đơn vị: nghìn đồng) của hai mã cổ phiếu A và B trong $50$ ngày giao dịch liên tiếp:
\begin{center}
\begin{tabular}{|c|c|c|c|c|c|}
\hline
Giá đóng cửa &$[120;122)$ &$[122;124)$ &$[124;126)$ &$[126;128)$ &$[128;130)$\\
\hline
Số ngày giao dịch của cổ phiếu A & $8$ & $9$ & $12$ & $10$ & $11$\\
\hline
Số ngày giao dịch của cổ phiếu B & $16$ & $4$ & $3$ & $6$ & $21$\\
\hline
\end{tabular} 
\end{center} 
Tính tỉ số $\dfrac{s_B^2}{s_A^2}$ (kết quả làm tròn đến hàng phần trăm).
\shortans[]{1{,}65}
\loigiai{
Ta có bảng thống kê giá đóng cửa theo giá trị đại diện như sau:
\begin{center}
\begin{tabular}{|c|c|c|c|c|c|}
\hline
Giá trị đại diện & $121$ & $123$ & $125$ & $127$ & $129$\\
\hline
Số ngày giao dịch của cổ phiếu A & $8$ & $9$ & $12$ & $10$ & $11$\\
\hline
Số ngày giao dịch của cổ phiếu B & $16$ & $4$ & $3$ & $6$ & $21$\\
\hline
\end{tabular}
\end{center}
\begin{enumerate}
\item Xét mẫu số liệu của cổ phiếu A
\begin{itemize}
\item Số trung bình của mẫu số liệu ghép nhóm là
\[
\overline{x}_A=\dfrac{8\cdot 121+9\cdot 123+12\cdot 125+10\cdot 127+11\cdot 129}{50}=125{,}28.
\]
\item Phương sai của mẫu số liệu ghép nhóm là
\[
s_A^2=\dfrac{1}{50}\left(8\cdot 121^2+9\cdot 123^2+12\cdot 125^2+10\cdot 127^2+11\cdot 129^2)-(125{,}28\right)^2
=7{,}5216.
\]
\end{itemize}
\item Xét mẫu số liệu của cổ phiếu B
\begin{itemize}
\item Số trung bình của mẫu số liệu ghép nhóm là
\[
\overline{x}_B=\dfrac{16\cdot 121+4\cdot 123+3\cdot 125+6\cdot 127+21\cdot 129}{50}
=125{,}48.
\]
\item Phương sai của mẫu số liệu ghép nhóm là\\
\[
s_B^2=\dfrac{1}{50}\left(16\cdot 121^2+4\cdot 123^2+3\cdot 125^2+6\cdot 127^2+21\cdot 129^2\right)-(125{,}48)^2
=12{,}4096.
\]
\end{itemize}
\end{enumerate}
Do đó $\dfrac{s_B^2}{s_A^2}\approx 1{,}65$.
}
\end{ex}

\begin{ex}%[2-D3B3-SO-9-2425]%[VN-MT-7, Nguyễn Tài Tuệ]%[2D3H2-2]
Thầy Niên thống kê lại điểm trung bình cuối năm của các học sinh lớp 10A và 10B ở bảng sau:
\begin{center}
\begin{tabular}{|c|c|c|c|c|c|}
\hline
Điểm trung bình &$[5;6)$ &$[6;7)$ &$[7;8)$ &$[8;9)$ &$[9;10)$\\
\hline
Số học sinh lớp 10A & $1$ & $0$ & $11$ & $22$ & $6$\\
\hline
Số học sinh lớp 10B & $0$ & $6$ & $8$ & $14$ & $12$\\
\hline
\end{tabular}
\end{center}
Tính $s_A-s_B$ (kết quả làm tròn đến hàng phần chục).
\shortans[]{-0{,}2}
\loigiai{
Ta có bảng thống kê điểm trung bình theo giá trị đại diện
\begin{center}
\begin{tabular}{|c|c|c|c|c|c|}
\hline
Giá trị đại diện & $5{,}5$ & $6{,}5$ & $7{,}5$ & $8{,}5$ & $9{,}5$\\
\hline
Số học sinh lớp 10A & $1$ & $0$ & $11$ & $22$ & $6$\\
\hline
Số học sinh lớp 10B & $0$ & $6$ & $8$ & $14$ & $12$\\
\hline
\end{tabular}
\end{center}
\begin{enumerate}
\item Xét mẫu số liệu của lớp 10A
\begin{itemize}
\item Số trung bình của mẫu số liệu ghép nhóm là
\[
\overline{x}_A=\dfrac{1\cdot 5{,}5+0.6{,}5+11\cdot 7{,}5+22\cdot 8{,}5+6\cdot 9{,}5}{40}=8{,}3.
\]
\item Phương sai của mẫu số liệu ghép nhóm là 
\[
s_A^2=\dfrac{1}{40}\left(1\cdot 5{,}5^2+0\cdot 6{,}5^2+11\cdot 7{,}5^2+22\cdot 8{,}5^2+6\cdot 9{,}5^2\right)-(8{,}3)^2
=0{,}61.
\]
\item Độ lệch chuẩn của mẫu số liệu ghép nhóm là $s_A=\sqrt{0{,}61}$.
\end{itemize}
\item Xét mẫu số liệu của lớp 10B
\begin{itemize}
\item Số trung bình của mẫu số liệu ghép nhóm là\\
\[
\overline{x}_B=\dfrac{0\cdot 5{,}5+6\cdot 6{,}5+8\cdot 7{,}5+14\cdot 8{,}5+12\cdot 9{,}5}{40}
=8{,}3.
\]
\item Phương sai của mẫu số liệu ghép nhóm là\\
\[s_B^2=\dfrac{1}{40}\left(0\cdot 5{,}5^2+6\cdot 6{,}5^2+8\cdot 7{,}5^2+14\cdot 8{,}5^2+12\cdot 9{,}5^2\right)-(8{,}3)^2
=1{,}06.\]
\item Độ lệch chuẩn của mẫu số liệu ghép nhóm là $s_B=\sqrt{1{,}06}$.
\end{itemize}
\end{enumerate}
Do đó $s_A-s_B \approx -0{,}2$.
}
\end{ex}
\Closesolutionfile{ans}
\begin{indapan}
	{ans/ans\currfilebase}
\end{indapan}


% \begin{name}
 {Biên soạn: Bùi Văn Lợi \\ Phản biện: HP Minh Nguyen}
 {Đề ôn tập chương III}
\end{name}

\TN
\Opensolutionfile{ans}[ans/ans\currfilebase-Phan-I]

\begin{ex}%[2-D3B3-SO-10-2425]%[VN-MT-7, Bùi Văn Lợi]%[2D3N1-2]
Xét mẫu dữ liệu cho bởi bảng sau:
\begin{center}
\begin{tabular}{|c|c|c|c|c|c|c|}
\hline 
Nhóm & $[14;15)$ & $[15;16)$ & $[16;17)$ & $[17;18)$ & $[18;19)$ & \\ 
\hline 
Tần số & $1$ & $3$ & $8$ & $6$ & $2$ & $n=20$ \\ 
\hline 
\end{tabular} 
\end{center}
Khoảng biến thiên của mẫu số liệu ghép nhóm bằng
\choice
{$3$}
{$4$}
{\True $5$}
{$6$}

\loigiai{
Khoảng biến thiên của mẫu số liệu ghép nhóm là $R=19-14=5$.
}
\end{ex}

\begin{ex}%[2-D3B3-SO-10-2425]%[VN-MT-7, Bùi Văn Lợi]%[2D3N1-1]
Xét mẫu số liệu cho bởi bảng sau:
\begin{center}
\begin{tabular}{|c|c|c|c|c|c|c|}
\hline
Nhóm & $[40;45)$ & $[45;50)$ & $[50;55)$ & $[55;60)$ & $[60;65)$ & \\
\hline
Tần số & $4$ & $11$ & $9$ & $n_4$ & $8$ & $n=40$ \\
\hline
\end{tabular}
\end{center}
Tần số $n_4$ của nhóm $4$ trong mẫu số liệu trên bằng
\choice
{$7$}
{\True $8$}
{$9$}
{$10$}

\loigiai{
Ta có $n_4 = 40-(4+11+9+8)=8$.
}
\end{ex}

\begin{ex}%[2-D3B3-SO-10-2425]%[VN-MT-7, Bùi Văn Lợi]%[2D3H1-4]
Xét mẫu số liệu cho bởi bảng sau:
\begin{center}
\begin{tabular}{|c|c|c|c|c|c|c|c|}
\hline
Nhóm & $[0;2)$ & $[2;4)$ & $[4;6)$ & $[6;8)$ & $[8;10)$ & $[10;12)$ &\\
\hline
Tần số & $3$ & $8$ & $12$ & $10$ & $7$ & $5$ & $n=45$\\
\hline
Tần số tích lũy & $3$ & $11$ & $23$ & $33$ & $40$ & $n_6$ &\\
\hline
\end{tabular}
\end{center}
Tần số tích lũy của nhóm $6$ bằng
\choice
{$40$}
{$42$}
{\True $45$}
{$41$}

\loigiai{
Tần số tích lũy của nhóm $6$ bằng $n_1+n_2+n_3+n_4+n_5+n_6=3+8+12+10+7+5=45$.
}
\end{ex}

\begin{ex}%[2-D3B3-SO-10-2425]%[VN-MT-7, Bùi Văn Lợi]%[1D5H2-3]
Xét mẫu số liệu cho bởi bảng sau:
\begin{center}
\begin{tabular}{|c|c|c|c|c|c|c|c|}
\hline Nhóm & $[40;45)$ & $[45;50)$ & $[50;55)$ & $[55;60)$ & $[60;65)$ & $[65;70)$ & \\
\hline Tần số & $5$ & $10$ & $7$ & $9$ & $7$ & $4$ & $n=42$ \\
\hline Tần số tích lũy & $5$ & $15$ & $22$ & $31$ & $38$ & $42$ & \\
\hline
\end{tabular}
\end{center}
Tứ phân vị thứ nhất của mẫu số liệu trên bằng
\choice
{$47{,}5$}
{\True $47{,}75$}
{$48$}
{$48{,}25$}

\loigiai{
Cỡ mẫu là $n=42$.\\
Gọi $x_1,x_2,\cdots,x_{42}$ là mẫu số liệu gốc được xếp theo thứ tự không giảm.\\
Tứ phân vị thứ nhất của mẫu số liệu gốc là $x_{11} \in [45;50)$.\\
Xét nhóm $[45;50)$ có đầu mút trái $s=45$, độ dài $h=5$, tần số $n_2=10$ và $cf_1=5$.\\
Tứ phân vị thứ nhất là $Q_1 = 45 + \left(\dfrac{10{,}5-5}{10}\right) \cdot 5=47{,}75$.
}
\end{ex}

\begin{ex}%[2-D3B3-SO-10-2425]%[VN-MT-7, Bùi Văn Lợi]%[1D5H2-3]
Xét mẫu số liệu cho bởi bảng sau:
\begin{center}
\begin{tabular}{|c|c|c|c|c|c|c|c|}
\hline
Nhóm & $[0;2)$ & $[2;4)$ & $[4;6)$ & $[6;8)$ & $[8;10)$\\
\hline
Tần số & $3$ & $8$ & $12$ & $12$ & $4$\\
\hline
Tần số tích lũy & $3$ & $11$ & $23$ & $35$ & $39$\\
\hline
\end{tabular}
\end{center}
Tứ phân vị thứ hai của mẫu số liệu trên gần nhất với kết quả nào dưới đây?
\choice
{$5{,}52$}
{\True $5{,}42$}
{$4{,}5$}
{$4{,}75$}

\loigiai{
Cỡ mẫu là $n=39$.\\
Gọi $x_1,x_2,\cdots,x_{39}$ là mẫu số liệu gốc được xếp theo thứ tự không giảm.\\
Tứ phân vị thứ hai của mẫu số liệu gốc là $x_{20} \in [4;6)$.\\
Xét nhóm $[4;6)$ có đầu mút trái $r=11$, độ dài $d=2$, tần số $n_3=12$ và $cf_2=11$.\\
Tứ phân vị thứ hai là $Q_2 =4+\left(\dfrac{19{,}5-11}{12}\right) \cdot 2 = \dfrac{65}{12} \approx 5{,}42$.
}
\end{ex}

\begin{ex}%[2-D3B3-SO-10-2425]%[VN-MT-7, Bùi Văn Lợi]%[1D5H2-3]
Cho mẫu số liệu ghép nhóm về tuổi thọ (đơn vị tính là năm) của một loại bóng đèn mới như sau:
\begin{center}
\begin{tabular}{|c|c|c|c|c|c|c|c|}
\hline
Tuổi thọ & $[2;3{,}5)$ & $[3{,}5;5)$ & $[5;6{,}5)$ & $[6{,}5;8)$\\
\hline
Số bóng đèn & $8$ & $22$ & $35$ & $15$\\
\hline
\end{tabular}
\end{center}
Nhóm chứa tứ phân vị thứ ba của mẫu số liệu là
\choice
{$[2;3{,}5)$}
{$[3{,}5;5)$}
{\True $[5;6{,}5)$}
{$[6{,}5;8)$}

\loigiai{
Lập bảng tần số tích lũy
\begin{center}
\begin{tabular}{|c|c|c|c|c|c|c|c|}
\hline
Nhóm & $[2;3{,}5)$ & $[3{,}5;5)$ & $[5;6{,}5)$ & $[6{,}5;8)$\\
\hline
Tần số & $8$ & $22$ & $35$ & $15$\\
\hline
Tần số tích lũy & $8$ & $30$ & $65$ & $80$\\
\hline
\end{tabular}
\end{center}
Cỡ mẫu là $n=8+22+35+15=80$.\\
Gọi $x_1,x_2,\cdots,x_{80}$ là mẫu số liệu gốc về tuổi thọ của $80$ bóng đèn được xếp theo thứ tự không giảm.\\
Tứ phân vị thứ ba của mẫu số liệu gốc là $\dfrac{x_{60}+x_{61}}{2} \in [5;6{,}5)$.\\
Khi đó nhóm chứa tứ phân vị thứ ba là $[5;6{,}5)$.
}
\end{ex}

\begin{ex}%[2-D3B3-SO-10-2425]%[VN-MT-7, Bùi Văn Lợi]%[2D3H1-3]
Bạn Thu rất thích nhảy hiện đại. Thời gian tập nhảy mỗi ngày trong thời gian gần đây của bạn Thu được thống kê lại ở bảng sau:
\begin{center}
\begin{tabular}{|c|c|c|c|c|c|c|c|}
\hline
Thời gian (phút) & $[20;25)$ & $[25;30)$ & $[30;35)$ & $[35;40)$ & $[40;45)$\\
\hline
Số ngày & $6$ & $6$ & $4$ & $1$ & $1$\\
\hline
\end{tabular}
\end{center}
Khoảng tứ phân vị của mẫu số liệu ghép nhóm là
\choice
{\True $8{,}125$}
{$8{,}5$}
{$13{,}5$}
{$4{,}5$}

\loigiai{
Ta lập bảng tần số tích lũy
\begin{center}
\begin{tabular}{|c|c|c|c|c|c|c|c|}
\hline Nhóm & $[20;25)$ & $[25;30)$ & $[30;35)$ & $[35;40)$ & $[40;45)$\\
\hline Tần số & $6$ & $6$ & $4$ & $1$ & $1$\\
\hline Tần số tích lũy & $6$ & $12$ & $16$ & $17$ & $18$\\
\hline
\end{tabular}
\end{center}
Cỡ mẫu là $n=18$.\\
Gọi $x_1,x_2,\cdots,x_{18}$ là mẫu số liệu gốc được xếp theo thứ tự không giảm.\\
Tứ phân vị thứ nhất của mẫu số liệu gốc là $x_5 \in [20;25)$. Khi đó\\
\[Q_1 = 20+\dfrac{4{,}5}{6} \cdot 5 = 23{,}75.\]
Tứ phân vị thứ nhất của mẫu số liệu gốc là $x_{14} \in [30;35)$. Khi đó\\
\[Q_3 = 30+\dfrac{13{,}5-12}{4} \cdot 5 = 31{,}875.\]
Vậy khoảng tứ phân vị của mẫu số liệu ghép nhóm là $\Delta_Q = Q_3-Q_1 = 8{,}125$.
}
\end{ex}

\begin{ex}%[2-D3B3-SO-10-2425]%[VN-MT-7, Bùi Văn Lợi]%[1D5N1-2]
Điều tra về chiều cao của học sinh khối 12 của một trường THPT, ta có kết quả sau:
\begin{center}
\begin{tabular}{|c|c|c|}
\hline
Nhóm & Chiều cao (cm) & Số học sinh\\
\hline
$1$ & $[160;163)$ & $6$\\
\hline
$2$ & $[163;166)$ & $10$\\
\hline
$3$ & $[166;169)$ & $12$\\
\hline
$4$ & $[169;172)$ & $20$\\
\hline
$5$ & $[172;175)$ & $40$\\
\hline
$6$ & $[175;178)$ & $12$\\
\hline
& & $n=100$\\
\hline
\end{tabular}
\end{center}
Giá trị đại diện của nhóm thứ năm là
\choice
{$173$}
{\True $173{,}5$}
{$174{,}5$}
{$175$}

\loigiai{
Giá trị đại diện của nhóm thứ năm là $\dfrac{172+175}{2}=173{,}5$.
}
\end{ex}

\begin{ex}%[2-D3B3-SO-10-2425]%[VN-MT-7, Bùi Văn Lợi]%[1D5N1-2]
Các bạn học sinh lớp 12A trả lời $40$ câu hỏi trong một bài kiểm tra. Kết quả được thống kê ở bảng sau. Hãy tính độ dài mỗi nhóm.
\begin{center}
\begin{tabular}{|c|c|c|c|c|c|}
\hline
Số câu trả lời đúng & $[20;24)$ & $[24;28)$ & $[28;32)$ & $[32;36)$ & $[36;40)$ \\
\hline
Số học sinh & $4$ & $7$ & $5$ & $10$ & $6$ \\
\hline
\end{tabular}
\end{center}
\choice
{$2$}
{$3$}
{\True $4$}
{$5$}

\loigiai{
Độ dài mỗi nhóm là $4$.
}
\end{ex}

\begin{ex}%[2-D3B3-SO-10-2425]%[VN-MT-7, Bùi Văn Lợi]%[1D5N1-3]
Tìm cân nặng trung bình của học sinh lớp 12A cho trong bảng sau, làm tròn đến hàng phần trăm.
\begin{center}
\begin{tabular}{|c|c|c|c|c|c|c|}
\hline
Cân nặng ($\mathrm{kg}$) & $[40{,}5;45{,}5)$ & $[45{,}5;50{,}5)$ & $[50{,}5;55{,}5)$ & $[55{,}5;60{,}5)$ & $[60{,}5;65{,}5)$ & $[65{,}5;70{,}5)$ \\
\hline
Số học sinh & $8$ & $10$ & $5$ & $12$ & $6$ & $3$ \\
\hline
\end{tabular}
\end{center}
\choice
{$54{,}5$}
{$53{,}31$}
{\True $53{,}79$}
{$54{,}45$}

\loigiai{
Trong mỗi khoảng cân nặng, giá trị đại diện là trung bình cộng của hai giá trị đầu mút nên ta có bảng sau
\begin{center}
\begin{tabular}{|c|c|c|c|c|c|c|}
\hline
Giá trị đại diện & $43$ & $48$ & $53$ & $58$ & $63$ & $68$ \\
\hline
Số học sinh & $8$ & $10$ & $5$ & $12$ & $6$ & $3$ \\
\hline
\end{tabular}
\end{center}
Tổng số học sinh là $n=44$. Cân nặng trung bình của học sinh lớp 12A là
\[\overline{x} = \dfrac{8\cdot 43 +10\cdot 48 +5\cdot 53 +12 \cdot 58+6 \cdot 63+ 3\cdot 68}{44} = 53{,}79.\]
}
\end{ex}

\begin{ex}%[2-D3B3-SO-10-2425]%[VN-MT-7, Bùi Văn Lợi]%[2D3H2-2]
Tìm phương sai của mẫu số liệu ghép nhóm được cho ở bảng sau (làm tròn kết quả đến hàng phần mười).
\begin{center}
\begin{tabular}{|c|c|c|c|c|c|c|}
\hline
Nhóm & $[1{,}5;2)$ & $[2;2{,}5)$ & $[2{,}5;3)$ & $[3;3{,}5)$ & $[3{,}5;4)$\\
\hline
Tần số & $4$ & $9$ & $13$ & $8$ & $6$\\
\hline
\end{tabular}
\end{center}
\choice
{\True $0{,}4$}
{$0{,}7$}
{$1{,}2$}
{$0{,}8$}

\loigiai{
Lập bảng giá trị đại diện cho mẫu số liệu như sau:
\begin{center}
\begin{tabular}{|c|c|c|c|c|c|c|}
\hline
Nhóm & $[1{,}5;2)$ & $[2;2{,}5)$ & $[2{,}5;3)$ & $[3;3{,}5)$ & $[3{,}5;4)$\\
\hline
Giá trị đại diện & $1{,}75$ & $2{,}25$ & $2{,}75$ & $3{,}25$ & $3{,}75$\\
\hline
Tần số & $4$ & $9$ & $13$ & $8$ & $6$\\
\hline
\end{tabular}
\end{center}
Ta có $n=40$, số trung bình cộng của mẫu số liệu ghép nhóm là
\[\overline{x} = \dfrac{4\cdot 1{,}75+9\cdot 2{,}25+13\cdot 2{,}75+8\cdot 3{,}25+6\cdot 3{,}75}{40} = 2{,}7875.\]
Phương sai của mẫu số liệu là
\[s^2 = \dfrac{4\cdot 1{,}75^2+9\cdot 2{,}25^2+13\cdot 2{,}75^2+8\cdot 3{,}25^2+6\cdot 3{,}75^2}{40} - (2{,}7875)^2 \approx 0{,}4.\]
}
\end{ex}

\begin{ex}%[2-D3B3-SO-10-2425]%[VN-MT-7, Bùi Văn Lợi]%[2D3H2-2]
Tìm độ lệch chuẩn của mẫu số liệu ghép nhóm được cho ở bảng sau (làm tròn kết quả đến hàng phần mười).
\begin{center}
\begin{tabular}{|c|c|c|c|c|c|c|}
\hline
Nhóm & $[25;35)$ & $[35;45)$ & $[45;55)$ & $[55;65)$ & $[65;75)$ & \\
\hline
Tần số & $9$ & $7$ & $5$ & $10$ & $9$ & $n=40$ \\
\hline
\end{tabular}
\end{center}
\choice
{$15{,}1$}
{$15{,}0$}
{$14{,}8$}
{\True $14{,}9$}

\loigiai{
Ta có bảng thống kê sau:
\begin{center}
\begin{tabular}{|c|c|c|c|c|c|c|}
\hline Nhóm & $[25;35)$ & $[35;45)$ & $[45;55)$ & $[55;65)$ & $[65;75)$ & \\
\hline Giá trị đại diện & $30$ & $40$ & $50$ & $60$ & $70$ &\\
\hline Tần số & $9$ & $7$ & $5$ & $10$ & $9$ & $n=40$ \\
\hline
\end{tabular}
\end{center}
Số trung bình cộng của mẫu số liệu ghép nhóm là
\[
\overline{x} = \dfrac{30\cdot 9+40\cdot 7+50\cdot 5+60\cdot 10+70\cdot 9}{40} = 50{,}75.
\]
Phương sai của mẫu số liệu là
\[
s^2 = \dfrac{30^2\cdot 9+40^2\cdot 7+50^2\cdot 5+60^2\cdot 10+70^2\cdot 9}{40} - (50{,}75)^2=221{,}9375.
\]
Vậy độ lệch chuẩn của mẫu số liệu trên là $s = \sqrt{221{,}9375} \approx 14{,}9$.
}
\end{ex}
\Closesolutionfile{ans}

\TNTF
\Opensolutionfile{ans}[ans/ans\currfilebase-Phan-II]

\begin{ex}%[2-D3B3-SO-10-2425]%[VN-MT-7, Bùi Văn Lợi]%[2D3H1-4]
Cho bảng số liệu sau:
\begin{center}
\begin{tabular}{|c|c|c|c|c|c|c|}
\hline
Nhóm & $[40;45)$ & $[45;50)$ & $[50;55)$ & $[55;60)$ & $[60;65)$ & \\
\hline
Tần số & $9$ & $5$ & $5$ & $4$ & $7$ & $n=30$ \\
\hline
\end{tabular}
\end{center}
\choiceTF
{Khoảng biến thiên của mẫu số liệu ghép trên là $R = 65$}
{\True Tần số của nhóm $5$ là $7$}
{Tần số tích lũy của nhóm $3$ là $15$}
{\True Tần số tích lũy của nhóm $5$ hơn nhóm $2$ là $16$}

\loigiai{
\begin{itemchoice}
\itemch \textbf{Sai}.\\
Khoảng biến thiên của mẫu số liệu ghép nhóm trên là $R=65-40=25$.
\itemch \textbf{Đúng}.\\
Tần số của nhóm $5$ là $7$.
\itemch \textbf{Sai}.\\
Tần số tích lũy của nhóm $3$ là $cf_3 = n_1+n_2+n_3 = 9+5+5 = 19$.
\itemch \textbf{Đúng}.\\
Tần số tích lũy của nhóm $2$ là $cf_2 = n_1+n_2 = 9+5 = 14$.\\
Tần số tích lũy của nhóm $5$ là $cf_5 = n_1+n_2+n_3+n_4+n_5 = 9+5+5+4+7 = 30$.\\
Vậy tần số tích lũy của nhóm $5$ hơn tần số tích lũy của nhóm $2$ là $16$.
\end{itemchoice}
}
\end{ex}

\begin{ex}%[2-D3B3-SO-10-2425]%[VN-MT-7, Bùi Văn Lợi]%[2D3H1-3]
Cho bảng số liệu sau:
\begin{center}
\begin{tabular}{|c|c|c|c|c|c|c|}
\hline Nhóm & $[6;8)$ & $[8;10)$ & $[10;12)$ & $[12;14)$ & $[14;16)$ & \\
\hline Tần số & $3$ & $5$ & $8$ & $10$ & $4$ & $n=30$ \\
\hline
\end{tabular}
\end{center}
\choiceTF
{Tứ phân vị thứ nhất của mẫu số liệu trên là $Q_1 = 9$}
{\True Tứ phân vị thứ hai của mẫu số liệu có giá trị nhỏ hơn $12$}
{\True Tứ phân vị thứ ba của mẫu số liệu có giá trị nằm trong khoảng $[12;14)$}
{Khoảng tứ phân vị của mẫu số liệu ghép nhóm trên là $\Delta_Q = 3{,}75$}

\loigiai{
Ta có bảng số liệu ghép nhóm:
\begin{center}
\begin{tabular}{|c|c|c|c|c|c|c|}
\hline Nhóm & $[6;8)$ & $[8;10)$ & $[10;12)$ & $[12;14)$ & $[14;16)$ & \\
\hline Tần số & $3$ & $5$ & $8$ & $10$ & $4$ & $n=30$ \\
\hline Tần số tích lũy & $3$ & $8$ & $16$ & $26$ & $30$ & \\
\hline
\end{tabular}
\end{center}
\begin{itemchoice}
\itemch \textbf{Sai}.\\
Ta có $\dfrac{n}{4} = 7{,}5$ mà $3<7{,}5<8$ nên nhóm $2$ là nhóm đầu tiên có tần số tích lũy lớn hơn hoặc bằng $7{,}5$.\\
Xét nhóm $[8;10)$ có đầu mút trái $s=8$, độ dài $h=2$, tần số $n_2 = 5$ và $cf_1 =3$.\\
Vậy, tứ phân vị thứ nhất là
$Q_1 = s+\left(\dfrac{7{,}5-cf_1}{n_2}\right) \cdot h = 8+\left(\dfrac{7{,}5-3}{5}\right) \cdot 2 = 9{,}8$.
\itemch \textbf{Đúng}.\\
Ta có $\dfrac{n}{2} =15$ mà $8<15<16$ nên nhóm $3$ là nhóm đầu tiên có tần số tích lũy lớn hơn hoặc bằng $15$.\\
Xét nhóm $[10;12)$ có đầu mút trái $r=10$, độ dài $d=2$, tần số $n_3 = 8$ và $cf_2 =8$.\\
Vậy, tứ phân vị thứ hai là
$Q_2 = r+\left(\dfrac{15-cf_2}{n_3}\right) \cdot d = 10+\left(\dfrac{15-8}{8}\right) \cdot 2 = 11{,}75 < 12$.
\itemch \textbf{Đúng}.\\
Ta có $\dfrac{3n}{4} =22{,}5$ mà $16<22{,}5<26$ nên nhóm $4$ là nhóm đầu tiên có tần số tích lũy lớn hơn hoặc bằng $22{,}5$.\\
Xét nhóm $[12;14)$ có đầu mút trái $t=12$, độ dài $l=2$, tần số $n_4 = 10$ và $cf_3 =16$.\\
Vậy, tứ phân vị thứ ba là
$Q_3 = t+\left(\dfrac{22{,}5-cf_3}{n_4}\right) \cdot l = 12+\left(\dfrac{22{,}5-16}{10}\right) \cdot 2 = 13{,}3$.
\itemch \textbf{Sai}.\\
Khoảng tứ phân vị của mẫu số liệu ghép nhóm trên là $\Delta_Q = Q_3-Q_1 = 3{,}5$.
\end{itemchoice}
}
\end{ex}

\begin{ex}%[2-D3B3-SO-10-2425]%[VN-MT-7, Bùi Văn Lợi]%[1D5H1-3]
Một mẫu số liệu được cho ở dạng bảng tần số ghép nhóm như sau:
\begin{center}
\begin{tabular}{|c|c|c|c|c|c|}
\hline
Nhóm & $[0{,}5;2{,}5)$ & $[2{,}5;4{,}5)$ & $[4{,}5;6{,}5)$ & $[6{,}5;8{,}5)$ & $[8{,}5;10{,}5)$ \\ 
\hline
Tần số & $4$ & $7$ & $16$ & $8$ & $5$ \\
\hline
\end{tabular}
\end{center}
\choiceTF
{\True Nhóm $[4{,}5;6{,}5)$ có giá trị đại diện là $5{,}5$}
{Nhóm $[8{,}5;10{,}5)$ có giá trị đại diện là $9$}
{\True Các nhóm trong mẫu số liệu đều độ dài bằng nhau}
{Số trung bình của mẫu số liệu trên lớn hơn $5{,}5$ (làm tròn kết quả đến hàng phần trăm)}

\loigiai{
Cỡ mẫu của số liệu là $n = 4+7+16+8+5 = 40$.\\
Bảng sau cho biết giá trị đại diện và độ dài của mỗi nhóm
\begin{center}
\begin{tabular}{|c|c|c|c|c|c|}
\hline
Nhóm & $[0{,}5;2{,}5)$ & $[2{,}5;4{,}5)$ & $[4{,}5;6{,}5)$ & $[6{,}5;8{,}5)$ & $[8{,}5;10{,}5)$ \\
\hline
Giá trị đại diện & $1{,}5$ & $3{,}5$ & $5{,}5$ & $7{,}5$ & $9{,}5$ \\
\hline
Độ dài nhóm & $2$ & $2$ & $2$ & $2$ & $2$ \\
\hline
\end{tabular}
\end{center}
\begin{itemchoice}
\itemch \textbf{Đúng}.\\
Nhóm $[4{,}5;6{,}5)$ có giá trị đại diện là $5{,}5$.
\itemch \textbf{Sai}.\\
Nhóm $[8{,}5;10{,}5)$ có giá trị đại diện là $9{,}5$.
\itemch \textbf{Đúng}.\\
Các nhóm trong mẫu số liệu có độ dài bằng nhau.
\itemch \textbf{Sai}.\\
Số trung bình cộng của mẫu số liệu ghép nhóm là
\[ \overline{x} = \dfrac{4\cdot 1{,}5+7\cdot 3{,}5+16\cdot 5{,}5+8\cdot 7{,}5+5\cdot 9{,}5}{40} = \dfrac{113}{20} = 5{,}65.\]
\end{itemchoice}
}
\end{ex}

\begin{ex}%[2-D3B3-SO-10-2425]%[VN-MT-7, Bùi Văn Lợi]%[2D3H2-2]
Mỗi ngày bác Nam đều đi bộ để rèn luyện sức khỏe. Quãng đường đi bộ mỗi ngày (đơn vị: km) của bác Nam trong $25$ ngày được thống kê ở bảng sau:
\begin{center}
\begin{tabular}{|c|c|c|c|c|c|c|}
\hline
Quãng đường (km) & $[1{,}5;2{,}5)$ & $[2{,}5;3{,}5)$ & $[3{,}5;4{,}5)$ & $[4{,}5;5{,}5)$ & $[5{,}5;6{,}5)$ & \\
\hline
Số ngày & $4$ & $2$ & $10$ & $7$ & $2$ & $n=25$\\
\hline
\end{tabular}
\end{center}
Các mệnh đề sau đúng hay sai? Kết quả làm tròn đến hàng phần trăm.
\choiceTF
{\True Độ dài nhóm của mẫu số liệu trên là $1$}
{\True Số trung bình của mẫu số liệu trên là $4{,}04$}
{Phương sai của mẫu số liệu trên là $1{,}6$}
{\True Độ lệch chuẩn của mẫu số liệu trên là $1{,}15$}

\loigiai{
Ta có bảng sau:
\begin{center}
\begin{tabular}{|c|c|c|c|c|c|}
\hline
Nhóm & $[1{,}5;2{,}5)$ & $[2{,}5;3{,}5)$ & $[3{,}5;4{,}5)$ & $[4{,}5;5{,}5)$ & $[5{,}5;6{,}5)$\\
\hline
Giá trị đại diện & $2$ & $3$ & $4$ & $5$ & $6$\\
\hline
Tần số & $4$ & $2$ & $10$ & $7$ & $2$\\
\hline
\end{tabular}
\end{center}
\begin{itemchoice}
\itemch \textbf{Đúng}.\\
Độ dài nhóm của mẫu số liệu là $1$.
\itemch \textbf{Đúng}.\\
Số trung bình của mẫu số liệu là
\[\overline{x} = \dfrac{2\cdot 4+3\cdot 2 +4\cdot 10+5\cdot 7+6\cdot 2}{25} = 4{,}04.\]
\itemch \textbf{Sai}.\\
Phương sai của mẫu số liệu là
\[
s^2 = \dfrac{(2-4{,}04)^2 \cdot 4+ (3-4{,}04)^2 \cdot 2 + (4-4{,}04)^2\cdot 10+(5-4{,}04)^2\cdot 7+(6-4{,}04)^2\cdot 2}{25} \approx 1{,}32.
\]
\itemch \textbf{Đúng}.\\
Độ lệch chuẩn của mẫu số liệu là $s = \sqrt{s^2} \approx 1{,}15$.
\end{itemchoice}
}
\end{ex}
\Closesolutionfile{ans}

\TNSA
\Opensolutionfile{ans}[ans/ans\currfilebase-Phan-III]

\begin{ex}%[2-D3B3-SO-10-2425]%[VN-MT-7, Bùi Văn Lợi]%[2D3N1-2]
Kết quả đo chiều cao của $250$ cây dừa đột biến $3$ năm tuổi ở một viện nghiên cứu được tổng hợp ở bảng sau:
\begin{center}
\begin{tabular}{|c|c|c|c|c|c|}
\hline 
Chiều cao (m$^2$) & {$[8{,}5 ; 8{,}8)$} & {$[8{,}8 ; 9{,}1)$} & {$[9{,}1 ; 9{,}4)$} & {$[9{,}4 ; 9{,}7)$} & {$[9{,}7 ; 10)$} \\
\hline Số cây & $36$ & $45$ & $83$ & $65$ & $21$ \\
\hline
\end{tabular}
\end{center}
Tìm khoảng biến thiên của mẫu số liệu ghép nhóm trên.

\shortans[]{1{,}5}

\loigiai{
Khoảng biến thiên của mẫu số liệu ghép nhóm là
$R=10-8{,}5=1{,}5$.
}
\end{ex}

\begin{ex}%[2-D3B3-SO-10-2425]%[VN-MT-7, Bùi Văn Lợi]%[2D3H1-4]
Một cộng ty bất động sản Đất Vàng thực hiện cuộc khảo sát khách hàng xem họ có
nhu cầu mua nhà ở mức giá nào đã tiến hành dự án xây nhà ở Thăng Long group sắp tới. Kết quả khảo sát $500$ khách hàng được ghi lại ở bảng sau:
\begin{center}
\begin{tabular}{|l|c|c|c|c|c|}
\hline 
Mức giá (triệu đồng) & $[10 ; 14)$ & $[14 ; 18)$ & $[18 ; 22)$ & $[22 ; 26)$ & $[26 ; 30)$ \\
\hline Số khách hàng & $75$ & $105$ & $179$ & $96$ & $45$ \\
\hline
\end{tabular}
\end{center}
Tìm tần số tích lũy của nhóm $[18;22)$.

\shortans[]{359}

\loigiai{
Tần số tích lũy của nhóm $[18;22)$ là $cf_3 = 75+105+179 = 359$.
}
\end{ex}

\begin{ex}%[2-D3B3-SO-10-2425]%[VN-MT-7, Bùi Văn Lợi]%[1D5H2-3]
Bảng dưới đây biểu diễn mẫu số liệu ghép nhóm về chiều cao (đơn vị: cm) của $43$ học sinh trong một lớp học khối $11$ của một trường phổ thông.
\begin{center}
\begin{tabular}{|c|c|c|c|c|c|c|c|}
\hline 
Nhóm & $[150 ; 155)$ & $[155 ; 160)$ & $[160 ; 165)$ & $[165 ; 170)$ & $[170 ; 175)$ & $[175 ; 180)$ & \\
\hline 
Tần số & $5$ & $10$ & $12$ & $9$ & $4$ & $3$ & $n=43$ \\ 
\hline 
\end{tabular} 
\end{center}
Tứ phân vị thứ hai của mẫu số liệu ghép nhóm trên bằng bao nhiêu (kết quả làm tròn đến hàng đơn vị)?

\shortans[]{163}

\loigiai{
Ta có bảng sau:
\begin{center}
\begin{tabular}{|c|c|c|}
\hline Nhóm & Tần số & Tần số tích lũy\\
\hline$[150 ; 155)$ & $5$ & $5$\\
\hline$[155 ; 160)$ & $10$& $15$\\
\hline$[160 ; 165)$ & $12$& $27$\\
\hline$[165 ; 170)$ & $9$ & $36$\\
\hline$[170 ; 175)$ & $4$ & $40$\\
\hline$[175 ; 180)$ & $3$ & $43$\\
\hline & $n=43$ & \\
\hline
\end{tabular}
\end{center}
Số phần tử của mẫu là $n=43$.\\
Ta có $\dfrac{n}{2}=21{,}5$ mà $15<21{,}5<27$ nên nhóm $3$ là nhóm đầu tiên có tần số tích lũy lớn hơn hoặc bằng $21{,}5$.\\
Xét nhóm $3$ là nhóm $[160;165)$ có đầu mút trái $r=160$, độ dài $d=5$, tần số $n_3=12$ và $cf_2=15$.\\
Từ đó ta có tứ phân vị thứ $2$ là
\[Q_2=160+\left(\dfrac{21{,}5-15}{12}\right)\cdot 5 \approx 163~\text{(cm)}.\]
}
\end{ex}

\begin{ex}%[2-D3B3-SO-10-2425]%[VN-MT-7, Bùi Văn Lợi]%[2D3H1-3] 
Bảng sau thống kê khối lượng (đơn vị: gam) một số quả măng cụt được lựa chọn ngẫu nhiên trong một thùng hàng
\begin{center}
\begin{tabular}{|c|c|c|c|c|c|c|}
\hline 
Nhóm & $[80 ; 82)$ & $[82 ; 84)$ & $[84 ; 86)$ & $[86 ; 88)$ & $[88 ; 90)$ & \\ 
\hline 
Tần số & $17$ & $22$ & $26$ & $19$ & $16$ & $n=100$ \\ 
\hline 
\end{tabular} 
\end{center}
Tính khoảng tứ phân vị của mẫu số liệu ghép nhóm trên (kết quả làm tròn đến hàng phần mười).

\shortans[]{4{,}3}

\loigiai{
Ta có bảng sau:
\begin{center}
\begin{tabular}{|c|c|c|}
\hline Nhóm & Tần số & Tần số tích lũy\\
\hline$[80 ; 82)$ & $17$ & $17$\\
\hline$[82 ; 84)$ & $22$ & $39$\\
\hline$[84 ; 86)$ & $26$ & $65$\\
\hline$[86 ; 88)$ & $19$ & $84$\\
\hline$[88 ; 90)$ & $16$ & $100$\\
\hline & $n=100$ & \\
\hline
\end{tabular}
\end{center}
Số phần tử của mẫu là $n=100$.\\
Ta có $\dfrac{n}{4}=25$ mà $17<25<39$.\\ 
Suy ra nhóm $2$ là nhóm đầu tiên có tần số tích lũy lớn hơn hoặc bằng $25$.\\ 
Xét nhóm $2$ là nhóm $[82 ; 84)$ có đầu mút trái $s=82$, độ dài $h=2$, tần số $n_2=22$ và $cf_1=17$.\\
Từ đó ta có tứ phân vị thứ nhất là 
\[Q_1=82+\left(\dfrac{25-17}{22}\right)\cdot 2 = \dfrac{910}{11}.\]
Tương tự, ta có $\frac{3n}{4}=75$ mà $65<75<84$.
Suy ra nhóm $4$ là nhóm có tần số tích lũy lớn hơn hoặc bằng $75$.\\
Khi đó tứ phân vị thứ ba là 
\[Q_3=86+ \left(\dfrac{75-65}{19}\right) \cdot 2 = \dfrac{1654}{19}.\]
Vậy khoảng tứ phân vị của mẫu số liệu ghép nhóm đã cho là 
\[\Delta_Q=Q_3-Q_1 =\dfrac{1654}{19}-\dfrac{910}{11} \approx 4{,}3~\text{(gam)}.\]
}
\end{ex}

\begin{ex}%[2-D3B3-SO-10-2425]%[VN-MT-7, Bùi Văn Lợi]%[1D5H1-3]
Mẫu số liệu cân nặng (đơn vị: gam) của $30$ trái xoài trong một thùng xoài chuẩn bị đem ra thị trường được biểu thị ở bảng sau:
\begin{center}
\begin{tabular}{|c|c|c|c|c|c|}
\hline
Cân nặng (g) & $[400;480)$ & $[480;560)$ & $[560;640)$ & $[640;720)$ & $[720;800)$ \\
\hline
Số quả xoài & $5$ & $3$ & $13$ & $7$ & $2$ \\
\hline
\end{tabular}
\end{center}
Cân nặng trung bình của mỗi quả xoài có dạng $\dfrac{a}{b}$ (gam) với $\dfrac{a}{b}$ là phân số tối giản ($a$, $b \in \mathbb{N}$). Khi đó giá trị của biểu thức $M = 2a-3b$ là bao nhiêu?

\shortans[]{3559}

\loigiai{
Ta có bảng sau:
\begin{center}
\begin{tabular}{|c|c|c|c|c|c|}
\hline
Nhóm & $[400;480)$ & $[480;560)$ & $[560;640)$ & $[640;720)$ & $[720;800)$ \\
\hline
Giá trị đại diện & $440$ & $520$ & $600$ & $680$ & $760$ \\
\hline
Tần số & $5$ & $3$ & $13$ & $7$ & $2$ \\
\hline
\end{tabular}
\end{center}
Số trung bình cộng của mẫu số liệu ghép nhóm trên là
\[
\overline{x}=\dfrac{5 \cdot 440+3\cdot 520+13 \cdot 600+7\cdot 680+2 \cdot 760}{30} = \dfrac{1784}{3}~\text{(gam)}.
\]
Khi đó $\dfrac{a}{b} = \dfrac{1784}{3}$ nên $M =2a-3b=3559$.
}
\end{ex}

\begin{ex}%[2-D3B3-SO-10-2425]%[VN-MT-7, Bùi Văn Lợi]%[2D3H2-2] 
Kiểm tra khối lượng của $40$ bao xi măng (đơn vị: kg) được chọn ngẫu nhiêm trước khi xuất xưởng, ta được mẫu số liệu ghép nhóm sau:
\begin{center}
\begin{tabular}{|c|c|c|c|c|c|c|}
\hline
Khối lượng (kg) & $[48{,}5;49)$ & $[49;49{,}5)$ & $[49{,}5;50)$ & $[50;50{,}5)$ & $[50{,}5;51)$ & $[51;51{,}5)$ \\
\hline
Số bao xi măng & $7$ & $3$ & $8$ & $6$ & $7$ & $9$ \\
\hline
\end{tabular}
\end{center}
Phương sai của mẫu số liệu ghép nhóm trên bằng bao nhiêu (kết quả làm tròn đến hàng phần trăm)?

\shortans[]{0{,}77}

\loigiai{
Ta có bảng sau:
\begin{center}
\begin{tabular}{|c|c|c|c|c|c|c|}
\hline
Nhóm & $[48{,}5;49)$ & $[49;49{,}5)$ & $[49{,}5;50)$ & $[50;50{,}5)$ & $[50{,}5;51)$ & $[51;51{,}5)$ \\
\hline
Giá trị đại diện & $48{,}75$ & $49{,}25$ & $49{,}75$ & $50{,}25$ & $50{,}75$ & $51{,}25$ \\
\hline
Tần số & $7$ & $3$ & $8$ & $6$ & $7$ & $9$ \\
\hline
\end{tabular}
\end{center}
Số trung bình cộng của mẫu số liệu ghép nhóm trên là
\[
\overline{x}=\dfrac{7\cdot 48{,}75+3\cdot 49{,}25+8\cdot 49{,}75+6\cdot 50{,}25+7\cdot 50{,}75+9\cdot 51{,}25}{40} = 50{,}125~\text{(kg)}.
\]
Phương sai của mẫu số liệu ghép nhóm trên là
\[
s^2=\dfrac{7\cdot 48{,}75^2+3\cdot 49{,}25^2+8\cdot 49{,}75^2+6\cdot 50{,}25^2+7\cdot 50{,}75^2+9\cdot 51{,}25^2}{40} - (50{,}125)^2 \approx 0{,}77.
\]
}
\end{ex}
\Closesolutionfile{ans}

\begin{indapan}
	{ans/ans\currfilebase}
\end{indapan}


% \begin{name}
 {Biên soạn: Nguyen Huynh \\ Phản biện: Nguyễn Kiều Nhã Tú}
 {Đề ôn tập chương III}
\end{name}

\caulc
\Opensolutionfile{ans}[ans/ans\currfilebase-Phan-I]

\begin{ex}%[2-D3B3-SO-11-2425]%[VN-MT-7, Nguyen Huynh]%[1D5H2-3]
 Khảo sát thời gian xem ti vi trong một ngày của một số học sinh khối $11$ thu được mẫu số liệu ghép nhóm sau:
 \begin{center}
 \begin{tabular}{|l|c|c|c|c|c|}
 \hline Thời gian (phút)&$[0 ; 20)$&$[20 ; 40)$&$[40 ; 60)$&$[60 ; 80)$&$[80 ; 100)$\\
 \hline Số học sinh & $5$ & $9$ & $12$ & $10$ & $6$ \\
 \hline
 \end{tabular}
 \end{center}
 Nhóm chứa tứ phân vị thứ nhất là
 \choice
 {$[0 ; 20)$}
 {\True $[20 ; 40)$}
 {$[40 ; 60)$}
 {$[60 ; 80)$}
 \loigiai{
 Cỡ mẫu $n=5+9+12+10+6=42$.\\ 
 Gọi $x_1$, $x_2$, $\ldots$, $x_{42}$ là mẫu số liệu về thời gian xem ti vi trong một ngày của một số học sinh khối $11$ được xếp theo thứ tự không giảm.\\
 Ta có
 $x_1$, $\ldots$, $x_5 \in[0 ; 20)$; $x_6$, $\ldots$, $x_{14} \in[20 ; 40)$; $x_{15}$, $\ldots$, $x_{26} \in[40 ; 60)$; $x_{27}$, $\ldots$, $x_{36} \in[60 ; 80)$; $x_{37}; \ldots$; $x_{42} \in[40 ; 45)$.\\
 Tứ phân vị thứ nhất của mẫu số liệu $x_1$, $x_2$, $\ldots$, $x_{42}$ là $\dfrac{1}{2}(x_{10}+x_{11})$.\\ 
 Do $x_{10}\in [20 ; 40)$ và $x_{11}\in [20 ; 40)$ nên nhóm chứa tứ phân vị thứ nhất là $[20 ; 40)$.}
\end{ex}

\begin{ex}%[2-D3B3-SO-11-2425]%[VN-MT-7, Nguyen Huynh]%[2D3N1-2]
 Cô Hà thống kê lại đường kính thân gỗ của một số cây xoan đào $6$ năm tuổi được trồng ở một lâm trường ở bảng sau:
 \begin{center}
 \begin{tabular}{|c|c|c|c|c|c|}
 \hline Đường kính $(\mathrm{cm})$ &$[40 ; 45)$&$[45 ; 50)$&$[50 ; 55)$&$[55 ; 60)$&$[60 ; 65)$\\
 \hline Tần số & $5$ & $20$ & $18$ & $7$ & $3$ \\
 \hline
 \end{tabular}
 \end{center}
 Hãy tìm khoảng biến thiên của mẫu số liệu ghép nhóm trên.
 \choice
 {\True $25$}
 {$30$}
 {$6$}
 {$69{,}8$}
 \loigiai{Khoảng biến thiên của mẫu số liệu ghép nhóm trên là $65-40=25$.}
\end{ex}

\begin{ex}%[2-D3B3-SO-11-2425]%[VN-MT-7, Nguyen Huynh]%[2D3H1-3]
 Bạn Chi rất thích nhảy hiện đại. Thời gian tập nhảy mỗi ngày trong thời gian gần đây của bạn Chi được thống kê lại ở bảng sau:
 \begin{center}
 \begin{tabular}{|c|c|c|c|c|c|}
 \hline Thời gian(phút)&{$[20 ; 25)$}&{$[25 ; 30)$}&{$[30 ; 35)$}&{$[35 ; 40)$}&{$[40 ; 45)$}\\
 \hline Số ngày & $6$ & $6$ & $4$ & $1$ & $1$ \\
 \hline
 \end{tabular}
 \end{center}
 Khoảng tứ phân vị của mẫu số liệu ghép nhóm là
 \choice
 {$23{,}75$}
 {$27{,}5$}
 {$31{,}88$}
 {\True $8{,}125$}
 \loigiai{Cỡ mẫu $n=18$. \\
 Gọi $x_1$, $x_2$, $\ldots$, $x_{18}$ là mẫu số liệu về thời gian tập nhảy mỗi ngày của bạn Chi được xếp theo thứ tự không giảm.\\
 Ta có
 $x_1$, $\ldots$, $x_6 \in[20 ; 25)$; $x_7$, $\ldots$, $x_{12} \in[25 ; 30)$; $x_{13}$, $\ldots$, $x_{16} \in[30 ; 35)$; $x_{17}$, $\in[35 ; 40)$; $x_{18} \in[40 ; 45)$.\\
 Tứ phân vị thứ nhất của mẫu số liệu gốc là $x_5 \in[20 ; 25)$. Do đó, tứ phân vị thứ nhất của mẫu số liệu ghép nhóm là \[Q_1=20+\dfrac{\dfrac{18}{4}}{6}(25-20)=23{,}75.\]
 Tứ phân vị thứ ba của mẫu số liệu góc là $x_{14} \in[30 ; 35)$. Do đó, tứ phân vị thứ ba của mẫu số liệu ghép nhóm là \[Q_3=30+\dfrac{\dfrac{3\cdot18}{4}-(6+6)}{4}(35-30)=31{,}875.\]
 Khoảng tứ phân vị của mẫu số liệu ghép nhóm là $\Delta_Q=Q_3-Q_1=8{,}125$.}
\end{ex}

\begin{ex}%[2-D3B3-SO-11-2425]%[VN-MT-7, Nguyen Huynh]%[1D5H2-3]
 Trong dịp nghỉ hè bạn Lan rất thích đi bơi. Thời gian đi bơi mỗi ngày trong thời gian gần đây của bạn Lan được thống kê lại ở bảng sau:
 \begin{center}
 \begin{tabular}{|c|c|c|c|c|c|}
 \hline Thời gian (phút) & {$[30 ; 35)$} & {$[35 ; 40)$} & {$[45 ; 50)$} & {$[50 ; 55)$} & {$[55 ; 60)$} \\
 \hline Số ngày & $3$ & $6$ & $4$ & $8$ & $4$ \\
 \hline
 \end{tabular}
 \end{center}
 Nhóm chứa tứ phân vị thứ nhất $Q_1$ là
 \choice
 {$[30 ; 35)$}
 {\True $[35 ; 40)$}
 {$[45 ; 50)$}
 {$[50 ; 55)$}
 \loigiai{Cỡ mẫu là $n=25$.\\
 Gọi $x_1$; $x_2$; $\ldots$; $x_{25}$ là mẫu số liệu về thời gian đi bơi mỗi ngày trong thời gian gần đây của bạn Lan được xếp theo thứ tự không giảm.\\
 Ta có
 $x_1$; $\ldots$; $x_3 \in[30 ; 35)$; $x_4$; $\ldots$; $x_{9} \in[35 ; 40)$; $x_{10}$; $\ldots$; $x_{13} \in[45 ; 50)$; $x_{14}$; $\ldots$; $x_{21}\in[50 ; 55)$; $x_{22}$; $\ldots$; $x_{25} \in[55 ; 60)$.\\ 
 Tứ phân vị thứ nhất của mẫu số liệu gốc là $\frac{x_6+x_7}{2}$. Do $x_6$, $x_7$ đều thuộc nhóm $[35 ; 40)$ nên nhóm này chứa $Q_1$.}
\end{ex}

\begin{ex}%[2-D3B3-SO-11-2425]%[VN-MT-7, Nguyen Huynh]%[1D5H2-3]
 Khảo sát thời gian tập nghe nhạc trong ngày của học sinh lớp 12B thu được mẫu số liệu ghép nhóm sau:
 \begin{center}
 \begin{tabular}{|c|c|c|c|c|c|}
 \hline Thời gian(phút)&{$[0 ; 20)$}&{$[20 ; 40)$}&{$[40 ; 60)$}&{$[60 ; 80)$}&{$[80 ; 100)$}\\
 \hline Số học sinh & $5$ & $10$ & $12$ & $9$ & $4$ \\
 \hline
 \end{tabular}
 \end{center}
 Nhóm chứa tứ phân vị thứ ba $Q_3$ là
 \choice
 {$[20 ; 40)$}
 {$[40 ; 60)$}
 {\True $[60 ; 80)$}
 {$[80 ; 100)$}
 \loigiai{
 Cỡ mẫu là $n=40$.\\
 Gọi $x_1$; $x_2$; $\ldots$; $x_{40}$ là mẫu số liệu về tập nghe nhạc trong ngày của học sinh lớp 12B được xếp theo thứ tự không giảm.\\
 Ta có
 $x_1$; $\ldots$; $x_5 \in[0 ; 20)$; $x_6$; $\ldots$; $x_{15} \in[20 ; 40)$; $x_{16}$; $\ldots$; $x_{27} \in[40 ; 60)$; $x_{28}$; $\ldots$; $x_{36}\in[60 ; 80)$; $x_{37}$; $\ldots$; $x_{40} \in[80 ; 100)$.\\ 
 Tứ phân vị thứ ba của mẫu số liệu gốc là $\dfrac{x_{30}+x_{31}}{2}$. Do $x_{30}$, $x_{31}$ đều thuộc nhóm $[60 ; 80)$ nên nhóm này chứa $Q_3$.}
\end{ex}

\begin{ex}%[2-D3B3-SO-11-2425]%[VN-MT-7, Nguyen Huynh]%[1D5H2-3]
 Một nhóm học sinh thi nhau giải khối rubik $4 \times 4$. Thời gian (đơn vị: giây) hoàn thành của nhóm học sinh được thống kê trong bảng sau:
 \begin{center}
 \begin{tabular}{|c|c|c|c|c|c|}
 \hline Thời gian giải rubik &{$[8 ; 10)$}&{$[10 ; 12)$}&{$[12 ; 14)$}&{$[14 ; 16)$}&{$[16 ; 18)$}\\
 \hline Số học sinh & $4$ & $6$ & $8$ & $4$ & $3$ \\
 \hline
 \end{tabular}
 \end{center}
 Tìm tứ phân vị thứ nhất và tứ phân vị thứ ba của mẫu số liệu.
 \choice
 {\True $Q_1=10{,}75$, $Q_3=14{,}375$}
 {$Q_1=11{,}0625$, $Q_3=14{,}375$}
 {$Q_1=10{,}75$, $Q_3=13{,}83$}
 {$Q_1=10{,}85$, $Q_3=14{,}75$}
 \loigiai{Cỡ mẫu là $n=25$.\\
 Tứ phân vị thứ nhất của mẫu số liệu gốc là $\dfrac{x_6+x_7}{2}\in[10 ; 12)$. Do đó, tứ phân vị thứ nhất của mẫu số liệu ghép nhóm là \[Q_1=10+\dfrac{\dfrac{25}{4}-4}{6}(12-10)=10{,}75.\]
 Tứ phân vị thứ ba của mẫu số liệu là $\dfrac{x_{19}+x_{20}}{2}\in[14 ; 16)$. Do đó, tứ phân vị thứ ba của mẫu số liệu ghép nhóm là \[Q_3=14+\dfrac{\dfrac{3\cdot25}{4}-(4+6+8)}{4}(16-14)=14{,}375.\]}
\end{ex}

\begin{ex}%[2-D3B3-SO-11-2425]%[VN-MT-7, Nguyen Huynh]%[2D3H1-3]
 Mỗi ngày bác Hương đều đi bộ để rèn luyện sức khoẻ. Quãng đường đi bộ mỗi ngày (đơn vị: km) của bác Hương trong $20$ ngày được thống kê lại ở bảng sau:
 \begin{center}
 \begin{tabular}{|c|c|c|c|c|c|}
 \hline
 Quãng đường & $[2{,}7;3{,}0)$ & $[3{,}0;3{,}3)$ & $[3{,}3;3{,}6)$ & $[3{,}6;3{,}9)$ & $[3{,}9;4{,}2)$ \\
 \hline
 Số ngày & $3$ & $6$ & $5$ & $4$ & $2$ \\
 \hline
 \end{tabular}
 \end{center}
 Khoảng tứ phân vị của mẫu số liệu ghép nhóm là
 \choice
 {$0{,}9$}
 {$0{,}975$}
 {$0{,}5$}
 {\True $0{,}575$}
 \loigiai{
 Cỡ mẫu $n=20$.\\
 Gọi $x_1$, $x_2$, $\ldots$, $x_{20}$ là mẫu số liệu về quãng đường đi bộ mỗi ngày của bác Hương trong $20$ ngày được xếp theo thứ tự không giảm.\\
 Ta có $x_1,\ldots,x_3 \in [2{,}7;3{,}0)$; $x_4,\ldots,x_9 \in [3{,}0;3{,}3)$; $x_{10},\ldots,x_{14} \in [3{,}3;3{,}6)$; $x_{15},\ldots,x_{18} \in [3{,}6;3{,}9)$; $x_{19},x_{20} \in [3{,}9;4{,}2)$.\\
 Tứ phân vị thứ nhất của mẫu số liệu gốc là $\dfrac{1}{2} \left(x_5+x_6 \right)\in [3{,}0;3{,}3)$.
 Do đó, tứ phân vị thứ nhất của mẫu số liệu ghép nhóm là \[Q_1=3{,}0+\dfrac{\dfrac{20}{4}-3}{6} (3{,}3-3{,}0)=3{,}1\]
 Tứ phân vị thứ ba của mẫu số liệu là $\dfrac{1}{2} \left(x_{15}+x_{16} \right)\in [3{,}6;3{,}9)$.
 Do đó, tứ phân vị thứ ba của mẫu số liệu ghép nhóm là
 \[Q_3=3{,}6+\dfrac{\dfrac{3\cdot 20}{4}-(3+6+5)}{4} (3{,}9-3{,}6)=3{,}675.
 \]
 Khoảng tứ phân vị của mẫu số liệu ghép nhóm là
 \[\Delta_Q=Q_3-Q_1=0{,}575.
 \]
 }
\end{ex}


\begin{ex}%[2-D3B3-SO-11-2425]%[VN-MT-7, Nguyen Huynh]%[1D5H1-3]
 Doanh thu bán hàng trong $20$ ngày được lựa chọn ngẫu nhiên của một của hàng được ghi lại ở bảng sau (đơn vị: triệu đồng):
 \begin{center}
 \begin{tabular}{|c|c|c|c|c|c|}
 \hline
 Doanh thu & $[5;7)$ & $[7;9)$ & $[9;11)$ & $[11;13)$ & $[13;15)$ \\
 \hline
 Số ngày & $2$ & $7$ & $7$ & $3$ & $1$ \\
 \hline
 \end{tabular}
 \end{center}
 Số trung bình của mẫu số liệu trên thuộc khoảng nào trong các khoảng dưới đây?
 \choice
 {$[7;9)$}
 {\True $[9;11)$}
 {$[11;13)$}
 {$[13;15)$}
 \loigiai{
 Bảng tần số ghép nhóm theo giá trị đại diện là
 \begin{center}
 \begin{tabular}{|c|c|c|c|c|c|}
 \hline
 Doanh thu & $[5;7)$ & $[7;9)$ & $[9;11)$ & $[11;13)$ & $[13;15)$ \\
 \hline
 Giá trị đại diện & $6$ & $8$ & $10$ & $12$ & $14$ \\
 \hline
 Số ngày & $2$ & $7$ & $7$ & $3$ & $1$ \\
 \hline
 \end{tabular}
 \end{center}
 Số trung bình \[\overline{x}=\dfrac{2\cdot6+7\cdot8+7\cdot10+3\cdot12+1\cdot14}{20}=9{,}4.\]
 }
\end{ex}


\begin{ex}%[2-D3B3-SO-11-2425]%[VN-MT-7, Nguyen Huynh]%[2D3H2-2]
 Một siêu thị thống kê số tiền (đơn vị: chục nghìn đồng) mà $44$ khách hàng mua hàng ở siêu thị đó trong một ngày. Số liệu được ghi lại trong bảng sau:
 \begin{center}
 \begin{tabular}{|c|c|c|}
 \hline
 Nhóm & Giá trị đại diện & Tần số \\
 \hline
 $[40;45)$ & $42{,}5$ & $4$ \\
 \hline
 $[45;50)$ & $47{,}5$ & $14$ \\
 \hline
 $[50;55)$ & $52{,}5$ & $8$ \\
 \hline
 $[55;60)$ & $57{,}5$ & $10$ \\
 \hline
 $[60;65)$ & $62{,}5$ & $6$ \\
 \hline
 $[65;70)$ & $67{,}5$ & $2$ \\
 \hline
 & & $n=44$ \\
 \hline
 \end{tabular}
 \end{center}
 Phương sai của mẫu số liệu ghép nhóm trên là
 \choice
 {$53{,}2$}
 {\True $46{,}1$}
 {$30$}
 {$11$} 
 \loigiai{Số trung bình cộng của mẫu số liệu ghép nhóm là
 \[\overline{x}=\dfrac{4\cdot42{,}5+14\cdot47{,}5+8\cdot52{,}5+10\cdot57{,}5+6\cdot62{,}5+2\cdot67{,}5}{44}=\dfrac{585}{11}.\]
 Phương sai của mẫu số liệu ghép nhóm là
 \begin{align*}
 \allowdisplaybreaks
 s^2&=\dfrac{4\left(42{,}5-\dfrac{585}{11} \right)^2+14\left(47{,}5-\dfrac{585}{11} \right)^2+8\left(52{,}5-\dfrac{585}{11} \right)^2+10\left(57{,}5-\dfrac{585}{11} \right)^2}{44} \\
 &+\dfrac{6\left(62{,}5-\dfrac{585}{11} \right)^2+2\cdot \left(67{,}5-\dfrac{585}{11} \right)^2}{44}\approx 46{,}1.
 \end{align*}
 }
\end{ex}


\begin{ex}%[2-D3B3-SO-11-2425]%[VN-MT-7, Nguyen Huynh]%[2D3H2-2]
 Khảo sát chiều cao (đơn vị cm) của học sinh lớp 12A, ta thu được kết quả như sau: 
 \begin{center}
 \begin{tabular}{|c|c|c|c|c|c|}
 \hline
 Kết quả đo & $[150;155)$ & $[155;160)$ & $[160;165)$ & $[165;170)$ & $[170;175)$ \\
 \hline
 Số học sinh & $6$ & $10$ & $14$ & $5$ & $5$ \\
 \hline
 \end{tabular}
 \end{center}
 Độ lệch chuẩn của mẫu số liệu ghép nhóm trên thuộc khoảng nào sau đây
 \choice
 {$\left(5{,}5;6\right)$}
 {\True $\left(6;6{,}5\right)$}
 {$\left(6{,}5;7\right)$}
 {$\left(7;7{,}5\right)$}
 \loigiai{
 Chọn giá trị đại diện cho các nhóm số liệu, ta có
 \begin{center}
 \begin{tabular}{|c|c|c|c|c|c|}
 \hline
 Giá trị đại diện & $152{,}5$ & $157{,}5$ & $162{,}5$ & $167{,}5$ & $172{,}5$ \\
 \hline
 Số học sinh & $6$ & $10$ & $14$ & $5$ & $5$ \\
 \hline
 \end{tabular}
 \end{center}
 Tổng số học sinh tham gia khảo sát là $n=6+10+14+5+5=40$.\\
 Chiều cao trung bình của học sinh trong lớp là \[\overline{x}=\dfrac{152{,}5\cdot 6+157{,}5\cdot10+162{,}5\cdot14+167{,}5\cdot5+172{,}5\cdot5}{40}=161{,}625\approx 161{,}6.\]
 Phương sai của mẫu số liệu trên là
 \begin{align*}
 s^2&=\dfrac{m_1 \left(x_1-\overline{x}\right)^2+\cdots+m_k \left(x_k-\overline{x}\right)^2}{n}\\
 &=\dfrac{6\left(152{,}5-161{,}6\right)^2+10\left(157{,}5-161{,}6\right)^2+14\left(162{,}5-161{,}6\right)^2}{40}\\
 &+\dfrac{5\left(167{,}5-161{,}6\right)^2+6\left(172{,}5-161{,}6\right)^2}{40}\\
 &\approx 36{,}1.
 \end{align*}
 Độ lệch chuẩn của mẫu số liệu trên là $s=\sqrt{s^2}=\sqrt{36{,}1} \approx 6{,}01\in(6;6{,}5)$.
 }
\end{ex}


\begin{ex}%[2-D3B3-SO-11-2425]%[VN-MT-7, Nguyen Huynh]%[2D3N1-1]
 Có bao nhiêu nhận xét đúng trong các nhận xét sau:
 \begin{enumerate}
 \item Khoảng biến thiên của mẫu số liệu ghép nhóm luôn luôn bằng khoảng biến thiên của mẫu số liệu.
 \item Khoảng biến thiên của mẫu số liệu ghép nhóm được dùng để đo mức độ phân tán của mẫu số liệu ghép nhóm.
 \item Khoảng biến thiên của mẫu số liệu ghép nhóm càng lớn thì mẫu số liệu càng phân tán.
 \end{enumerate}
 \choice
 {$0$}
 {$1$}
 {\True $2$}
 {$3$}
 \loigiai{
 \begin{itemize}
 \item Nhận xét ``Khoảng biến thiên của mẫu số liệu ghép nhóm luôn luôn bằng khoảng biến thiên của mẫu số liệu''\, sai vì khoảng biến thiên của mẫu số liệu ghép nhóm xấp xỉ cho khoảng biến thiên của mẫu số liệu. 
 \item Nhận xét ``Khoảng biến thiên của mẫu số liệu ghép nhóm được dùng để đo mức độ phân tán của mẫu số liệu ghép nhóm''\, đúng.
 \item Nhận xét ``Khoảng biến thiên của mẫu số liệu ghép nhóm càng lớn thì mẫu số liệu càng phân tán''\, đúng.
 \end{itemize}
 
 }
\end{ex}


\begin{ex}%[2-D3B3-SO-11-2425]%[VN-MT-7, Nguyen Huynh]%[2D3N1-1]
 Nhận xét nào \textbf{sai} trong các nhận xét sau?
 \begin{enumerate}
 \item Khoảng tứ phân vị của mẫu số liệu ghép nhóm bị ảnh hưởng bởi các giá trị bất thường trong mẫu số liệu.
 \item Khoảng tứ phân vị của mẫu số liệu ghép nhóm xấp xỉ cho khoảng tứ phân vị của mẫu số liệu.
 \item Khoảng tứ phân vị càng lớn thì mẫu số liệu càng phân tán.
 \item Khoảng tứ phân vị được dùng để đo mức độ phân tán của mẫu số liệu ghép nhóm.
 \end{enumerate}
 \choice
 {\True Nhận xét a)}
 {Nhận xét b)}
 {Nhận xét c)}
 {Nhận xét d)}
 \loigiai{\begin{itemize}
 \item Nhận xét ``Khoảng tứ phân vị của mẫu số liệu ghép nhóm bị ảnh hưởng bởi các giá trị bất thường trong mẫu số liệu''\, sai vì khoảng tứ phân vị của mẫu số liệu ghép nhóm chỉ phụ thuộc vào nửa giữa của mẫu số liệu, nên không bị ảnh hưởng bởi các giá trị bất thường và có thể dùng đại lượng này để loại giá trị bất thường.
 \item Nhận xét ``Khoảng tứ phân vị của mẫu số liệu ghép nhóm xấp xỉ cho khoảng tứ phân vị của mẫu số liệu''\, đúng.
 \item Nhận xét ``Khoảng tứ phân vị càng lớn thì mẫu số liệu càng phân tán''\, đúng.
 \item Nhận xét ``Khoảng tứ phân vị được dùng để đo mức độ phân tán của mẫu số liệu ghép nhóm''\, đúng.
 \end{itemize}
 }
\end{ex}

\Closesolutionfile{ans}

\cauds
\Opensolutionfile{ans}[ans/ans\currfilebase-Phan-II]

\begin{ex}%[2-D3B3-SO-11-2425]%[VN-MT-7, Nguyen Huynh]%[2D3H2-2]
 Bảng bảng biểu diễn mẫu số liệu ghép nhóm về nhiệt độ ($^\circ$C) của tỉnh Nghệ An tháng $5$ năm $2024$.
 \begin{center}
 \begin{tabular}{|c|c|c|c|}
 \hline
 Nhóm & Giá trị đại diện & Tần số & Tần số tích lũy \\
 \hline
 $[29;31)$ & $30$ & $1$ & $1$ \\
 \hline
 $[31;33)$ & $32$ & $4$ & $5$ \\
 \hline
 $[33;35)$ & $34$ & $5$ & $10$ \\
 \hline
 $[35;37)$ & $36$ & $13$ & $26$ \\
 \hline
 $[37;39]$ & $38$ & $7$ & $33$ \\
 \hline
 & & $n=30$ & \\
 \hline
 \end{tabular}
 \end{center}
 \choiceTF
 {\True Nhóm $[31;33)$ có tần số bằng $4$}
 {Mốt của mẫu số liệu ghép nhóm đã cho là $13$ (làm tròn đến hàng phần trăm)}
 {\True Khoảng tứ phân vị của mẫu số liệu ghép nhóm trên bằng $2{,}92$ (làm tròn đến hàng phần trăm)}
 {\True Phương sai của mẫu số liệu ghép nhóm trên bằng $4{,}57$ (làm tròn đến hàng phần trăm)} 
 \loigiai{
 \begin{itemchoice}
 \itemch \textbf{Đúng}. 
 \\Nhóm $[31;33)$ có tần số bằng $4$.
 
 \itemch \textbf{Sai}. 
 \\Ta có nhóm $[35;37)$ có tần số lớn nhất nên mốt của mẫu số liệu trên là \[M_{\text{o}}=u+\dfrac{n_i-n_{i-1}}{2n_i-n_{i-1}-n_{i+1}}\cdot g=35+\dfrac{13-5}{2\cdot 13-5-7}\cdot2=36{,}14.\]
 
 \itemch \textbf{Đúng}. 
 \\Ta có số phần tử của mẫu là $n=30$.\\
 Ta có $\dfrac{n}{4}=7{,}5$ nên nhóm $[33;35)$ là nhóm đầu tiên có tần số tích lũy lớn hơn hoặc bằng $7{,}5$.\\
 Nhóm $[33;35)$ có $s=33;h=2; n=5$ và nhóm $2$ là nhóm $[31;33)$ có $cf_1=5$.\\
 Áp dụng công thức ta có tứ phân vị thứ nhất là $Q_1=33+\dfrac{7{,}5-5}{5}\cdot2=34$ ($^\circ$C).\\
 Ta có $\dfrac{3n}{4}=22{,}5$ nên nhóm $[35;37)$ là nhóm đầu tiên có tần số tích lũy lớn hơn hoặc bằng $22{,}5$.\\
 Xét nhóm $4$ là nhóm $[35;37)$ có $t=35$; $l=2$; $n_4=13$ và nhóm $3$ có tần số tích lũy $cf_4=10$.\\
 Áp dụng công thức, ta có tứ phân vị thứ $3$ là $Q_3=35+\dfrac{22{,}5-10}{13}\cdot2=36{,}92$ ($^\circ$C).\\
 Vậy khoảng tứ phân vị của mẫu số liệu ghép nhóm đã cho là \[\Delta Q=Q_3-Q_1=36{,}92-34=2{,}92.\]
 
 \itemch \textbf{Đúng}. 
 \\Ta có số trụng bình cộng của mẫu số liệu ghép nhóm trên là\\
 \[\overline{x}=\dfrac{1}{30} \left(1\cdot30+32\cdot4+34\cdot5+36\cdot13+38\cdot7\right)=35{,}4\,(^\circ \mathrm C).\]
 Vậy phương sai của mẫu số liệu ghép nhóm trên là 
% \begin{eqnarray*}
% s^2&=&\dfrac{1}{30} \left[1\cdot(30-35{,}4)^2+4\cdot(32-35{,}4)^2+5\cdot(34-35{,}4)^2+13\cdot(36-35{,}4)^2+7\cdot(38-35{,}4)^2\right]\\&\approx&4{,}57.
% \end{eqnarray*}
 \[s^2=\dfrac{1}{30} \left(30^2\cdot1+32^2\cdot4+34^2\cdot5+36^2\cdot13+38^2\cdot7\right)-(35{,}4)^2\approx 4{,}57.\]
 \end{itemchoice}
 }
\end{ex}

\begin{ex}%[2-D3B3-SO-11-2425]%[VN-MT-7, Nguyen Huynh]%[2D3H2-2]
 Cho bảng phân bố tần số ghép lớp cân nặng (đơn vị: kg) của các công nhân trong một công ty như sau 
 \begin{center}
 \begin{tabular}{|c|c|c|c|c|c|c|}
 \hline Cân nặng &$[50;52)$&$[52;54)$&$[54;56)$&$[56;58)$&$[58;60)$& Cộng \\
 \hline Tần số & $15$ & $20$ & $45$ & $15$ & $5$ & $100$ \\
 \hline
 \end{tabular}
 \end{center}
 \choiceTF
 {\True Tần suất của nhóm $[52;54)$ là $20$}
 {Số trung vị của mẫu số liệu lớn hơn $54{,}9$}
 {\True Khoảng biến thiên của mẫu số liệu trên là $10$}
 {Độ lệch chuẩn của mẫu số liệu trên là $4{,}35$}
 \loigiai{
 \begin{itemchoice}
 \itemch \textbf{Đúng}. 
 \\Tần số của nhóm $[52;54)$ là $20$.\\
 Tần suất của nhóm $[52;54)$ là $\dfrac{20}{100}\cdot100\%=20\%$.
 
 \itemch \textbf{Sai}. 
 \\Trung vị của mẫu số liệu là $x_3 \in [54;56)$.\\
 Do đó, trung vị của mẫu số liệu ghép nhóm là\\ $M_{\text{e}}=Q_2=54+\dfrac{\dfrac{2\cdot100}{4}-(15+20)}{45} (56-54)=\dfrac{164}{3}\approx54{,}667$.
 \\Do đó trung vị của mẫu số liệu bé hơn $54{,}9$
 \itemch\textbf{Đúng}. 
 \\Khoảng biến thiên của mẫu số liệu là $R=60-50=10$.
 
 \itemch \textbf{Sai}. 
 \\Số trung bình cộng của mẫu số liệu ghép nhóm của công ty là \[\overline{x}=\dfrac{51\cdot15+53\cdot20+55\cdot45+57\cdot15+59\cdot5}{100}=54{,}5.\]
 Phương sai của mẫu số liệu ghép nhóm của công ty là
% \begin{eqnarray*}
% s^2&=&\dfrac{15\cdot \left(51-54{,}5\right)^2+20\cdot \left(53-54{,}5\right)^2+45\cdot \left(55-54{,}5\right)^2+15\cdot \left(57-54{,}5\right)^2+5\cdot(59-54{,}4)^2}{100}\\&=&4{,}35.
% \end{eqnarray*}
 \[s^2=\dfrac{51^2\cdot15+53^2\cdot20+55^2\cdot45+57^2\cdot15+59^2\cdot5}{100}-(54{,}5)^2=4{,}35.\]
 Độ lệch chuẩn của mẫu số liệu ghép nhóm của công ty là $s=\sqrt{s^2}=\sqrt{4{,}35} \approx 2{,}09$.
 \end{itemchoice}
 }
\end{ex}

\begin{ex}%[2-D3B3-SO-11-2425]%[VN-MT-7, Nguyen Huynh]%[2D3V2-3]
 Cho bảng số liệu dưới đây về thời gian (phút) tập thể dục buổi sáng của hai bạn Bình và Chi trong $30$ ngày.
 \begin{center}
 \begin{tabular}{|c|c|c|c|c|c|}
 \hline
 Thời gian & $[15;20)$ & $[20;25)$ & $[25;30)$ & $[30;35)$ & $[35;40)$ \\
 \hline
 Bạn Bình & $5$ & $8$ & $10$ & $4$ & $3$ \\
 \hline
 Bạn Chi & $10$ & $10$ & $5$ & $3$ & $2$ \\
 \hline
 \end{tabular}
 \end{center}
 \choiceTF
 {\True Khoảng biến thiên của mẫu số liệu ghép nhóm về thời gian tập thể dục của Chi là $25$ (phút)}
 {Tứ phân vị thứ nhất của mẫu số liệu ghép nhóm về thời gian tập thể dục buổi sáng của bạn Bình là $Q_1=\dfrac{354}{16}$}
 {\True Khoảng tứ phân vị của mẫu số liệu ghép nhóm về thời gian tập thể dục buổi sáng của bạn Chi là $28{,}75$}
 {\True Phương sai của mẫu số liệu ghép nhóm về thời gian tập thể dục buổi sáng của bạn Bình là $\dfrac{314}{9}$} 
 \loigiai{
 Ta có \begin{center}
 \begin{tabular}{|c|c|c|c|c|c|}
 \hline
 Thời gian & $\left[15;20\right)$ & $\left[20;25\right)$ & $\left[25;30\right)$ & $\left[30;35\right)$ & $\left[35;40\right)$ \\
 \hline
 Giá trị đại diện & $17{,}5$ & $22{,}5$ & $27{,}5$ & $32{,}5$ & $37{,}5$ \\
 \hline
 Bạn Bình & $5$ & $8$ & $10$ & $10$ & $10$ \\
 \hline
 Bạn Chi & $10$ & $10$ & $5$ & $3$ & $2$ \\
 \hline
 \end{tabular}
 \end{center}
 
 \begin{itemchoice}
 \itemch \textbf{Đúng}. 
 \\Khoảng biến thiên của mẫu số liệu ghép nhóm là $40-15=25$ (phút).
 \itemch \textbf{Sai}. 
 \\Xét số liệu của bạn Bình.\\
 Ta có cỡ mẫu $n=30$.\\
 Vì $\dfrac{n}{4}=\dfrac{30}{4}=7{,}5$ và $5< 7{,}5< 5+8$ nên tứ phân vị thứ nhất thuộc nhóm $[20;25)$.\\
 Tứ phân vị thứ nhất của mẫu số liệu về thời gian tập thể dục buổi sáng của bạn Bình là \[Q_1=20+\dfrac{\dfrac{30}{4}-5}{8}\cdot5=\dfrac{345}{16}.\]
 \itemch \textbf{Đúng}. 
 \\Xét số liệu của bạn Chi.\\
 Ta có cỡ mẫu $n=30$.\\
 Vì $\dfrac{n}{4}=\dfrac{30}{4}=7{,}5$ và $7{,}5< 10$ nên tứ phân vị thứ nhất thuộc nhóm $[15;20)$.\\
 Tứ phân vị thứ nhất của mẫu số liệu về thời gian tập thể dục buổi sáng của bạn Chi là\\ \[Q'_1=15+\dfrac{\dfrac{30}{4}-0}{10} \cdot 5=18{,}75.\]
 Vì $\dfrac{3n}{4}=\dfrac{3 \cdot 30}{4}=22{,}5$ và $10+10< 22{,}5< 10+10+5$. Do đó tứ phân vị thứ ba thuộc nhóm $[25;30)$.\\
 Tứ phân vị thứ ba của mẫu số liệu về thời gian tập thể dục buổi sáng của bạn Chi là \[Q'_3=25+\dfrac{\dfrac{3\cdot30}{4}-(10+10)}{5}\cdot5=27{,}5.\]
 Vậy khoảng tứ phân vị là $\Delta_Q'=Q'_3-Q'_1=28{,}75$.
 
 \itemch \textbf{Đúng}. 
 \\Thời gian trung bình bạn Bình tập thể dục buổi sáng là
 \[\overline{x}=\dfrac{17{,}5\cdot5+22{,}5\cdot8+27{,}5\cdot10+32{,}5\cdot4+37{,}5\cdot3}{30}=\dfrac{157}{6} \approx 26{,}17.\]
 Phương sai của mẫu số liệu ghép nhóm về thời gian tập thể dục buổi sáng của bạn Bình là
 % \begin{align*}
 % s_B^2&=\dfrac{5\left(17{,}5-\dfrac{157}{6} \right)^2+8\left(22{,}5-\dfrac{157}{6} \right)^2+10\left(27{,}5-\dfrac{157}{6} \right)^2+4\left(32{,}5-\dfrac{157}{6} \right)^2}{30}\\
 % &+\dfrac{3\left(37,5-\dfrac{157}{6} \right)^2}{30}\\
 % &=\dfrac{314}{9}.
 % \end{align*}
 \[s_B^2=\dfrac{17{,}5^2\cdot5+22{,}5^2\cdot8+27{,}5^2\cdot10+32{,}5^2\cdot4+37{,}5^2\cdot3}{30}-\left(\dfrac{157}{6}\right)^2 =\dfrac{314}{9}.\]
 \end{itemchoice}
 }
\end{ex}

\begin{ex}%[2-D3B3-SO-11-2425]%[VN-MT-7, Nguyen Huynh]%[2D3V2-3]
 Bảng sau đây cho biết chiều cao của các em học sinh lớp 12A và 12B:
 \begin{center}
 \begin{tabular}{|c|c|c|c|c|c|c|}
 \hline
 Chiều cao (cm) & $[145; 150)$ & $[150; 155)$ & $[155; 160)$ & $[160; 165)$ & $[165; 170)$ & $[170; 175)$ \\
 \hline
 Số học sinh của lớp 12A& $2$ & $1$ & $15$ & $11$ & $9$ & $3$ \\
 \hline
 Số học sinh của lớp 12B & $0$ & $1$ & $16$ & $11$ & $10$ & $4$ \\
 \hline 
 \end{tabular}
 \end{center}
 \choiceTF
 {\True Dựa vào khoảng biến thiên của mẫu số liệu ghép nhóm thì chiều cao của học sinh lớp 12A phân tán hơn lớp 12B}
 {Dựa vào khoảng tứ phân vị của mẫu số liệu ghép nhóm thì học sinh lớp 12A có chiều cao phân tán hơn học sinh lớp 12B}
 {Dựa vào phương sai của mẫu số liệu ghép nhóm thì chiều cao của học sinh lớp 12A ít phân tán hơn học sinh lớp 12B}
 {\True Học sinh lớp 12B có chiều cao đồng đều hơn học sinh lớp 12A vì có độ lệch chuẩn nhỏ hơn} 
 \loigiai{
 \begin{itemchoice}
 \itemch \textbf{Đúng}.\\ Với số liệu của học sinh lớp 12A, có khoảng biến thiên $R_A=175-145=30$.\\
 Với số liệu của học sinh lớp 12B, có khoảng biến thiên $R_B=175-150=25$.\\
 Do $R_A > R_B$ nên chiều cao của học sinh lớp 12A phân tán hơn lớp 12B.
 \itemch \textbf{Sai}.\\Với mẫu số liệu ghép nhóm của học sinh lớp 12A.\\
 Cỡ mẫu là $n=2+1+15+11+9+3=41$.\\
 Gọi $x_1$, $x_2$, $x_3$, $\ldots$, $x_{41}$ là mẫu số liệu gồm chiều cao của học sinh lớp 12A được sắp xếp theo thứ tự không giảm.\\
 Tứ phân vị thứ nhất của mẫu số liệu gốc là $\dfrac{1}{2} (x_{10}+x_{11})\in [155; 160)$ nên nhóm chứa tứ phân vị thứ nhất là nhóm $[155; 160)$. Do đó \[Q_1=155+\dfrac{\dfrac{41}{4}-3}{15}\cdot5\approx 157{,}42.\]
 Tứ phân vị thứ ba của mẫu số liệu gốc là $\dfrac{1}{2} (x_{31}+x_{32})\in [165; 170)$ nên nhóm chứa tứ phân vị thứ ba là nhóm $[165; 170)$. Do đó \[Q_3=165+\dfrac{\dfrac{3\cdot41}{4}-29}{9}\cdot5\approx 165{,}97.\]
 Suy ra $\Delta_{Q_A}=Q_3-Q_1 \approx 8{,}55$.\\
 Với mẫu số liệu ghép nhóm của học sinh lớp 12B\\
 Cỡ mẫu là $n=1+16+11+10+4=42$.
 \\Gọi $x_1$, $x_2$, $x_3$, $\ldots$, $x_{42}$ là mẫu số liệu gồm chiều cao của học sinh lớp 12B được sắp xếp theo thứ tự không giảm.\\
 Tứ phân vị thứ nhất của mẫu số liệu gốc là $x_{11} \in [155; 160)$ nên nhóm chứa tứ phân vị thứ nhất là nhóm $[155; 160)$. Do đó \[Q'_1=155+\dfrac{\dfrac{42}{4}-1}{16}\cdot5\approx 157{,}97.\]
 Tứ phân vị thứ ba của mẫu số liệu gốc là $x_{32} \in [165; 170)$ nên nhóm chứa tứ phân vị thứ ba là nhóm $[165; 170)$. Do đó \[Q'_3=165+\dfrac{\dfrac{3\cdot42}{4}-28}{10}\cdot5=166{,}75.\]
 Suy ra $\Delta_{Q_B}=Q'_3-Q'_1 \approx 8{,}78$.\\
 Do $\Delta_{Q_A} < \Delta_{Q_B}$ nên học sinh lớp 12A có chiều cao phân tán ít hơn học sinh lớp 12B.
 
 \itemch \textbf{Sai}.\\ Chọn giá trị đại diện cho các nhóm số liệu ta được bảng
 \begin{center}
 \begin{tabular}{|c|c|c|c|c|c|c|}
 \hline
 Chiều cao (cm) & $[145; 150)$ & $[150; 155)$ & $[155; 160)$ & $[160; 165)$ & $[165; 170)$ & $[170; 175)$ \\
 \hline
 Giá trị đại diện & $147{,}5$ & $152{,}5$ & $157{,}5$ & $162{,}5$ & $167{,}5$ & $172{,}5$ \\
 \hline
 Số học sinh của lớp 12A & $2$ & $1$ & $15$ & $11$ & $9$ & $3$ \\
 \hline
 Số học sinh của lớp 12B & $0$ & $1$ & $16$ & $11$ & $10$ & $4$ \\
 \hline 
 \end{tabular}
 \end{center}
 Chiều cao trung bình của học sinh lớp 12A là \[\overline{x}_A=\dfrac{1}{41} (2\cdot 147{,}5+1\cdot 152{,}5+15\cdot 157{,}5+11\cdot 162{,}5+9\cdot 167{,}5+3\cdot 172{,}5)\approx 161{,}52.\]
 Chiều cao trung bình của học sinh lớp 12B là \[\overline{x}_B=\dfrac{1}{42} (0\cdot 147{,}5+1\cdot 152{,}5+16\cdot 157{,}5+11\cdot 162{,}5+10\cdot 167{,}5+4\cdot 172{,}5)=162{,}5.\]
 Phương sai của mẫu số liệu lớp 12A là
 \begin{eqnarray*}
 s_A^2&=&\dfrac{1}{41} \left(2\cdot 147{,}5^2+1\cdot 152{,}5^2+15\cdot 157{,}5^2+11\cdot 162{,}5^2+9\cdot 167{,}5^2+3\cdot 172{,}5^2\right)-\left(161{,}52\right)^2\\&\approx& 34{,}41.
 \end{eqnarray*}
 
 Phương sai của mẫu số liệu lớp 12B là
 \begin{eqnarray*}
 s_B^2&=&\dfrac{1}{42} \left(0\cdot 147{,}5^2+1\cdot 152{,}5^2+16\cdot 157{,}5^2+11\cdot 162{,}5^2+10\cdot 167{,}5^2+4\cdot 172{,}5^2\right)-\left(162{,}50\right)^2\\&\approx& 27{,}38.
 \end{eqnarray*}
 Vậy dựa vào phương sai của mẫu số liệu ghép nhóm thì chiều cao của học sinh lớp 12A phân tán hơn học sinh lớp 12B.
 
 \itemch \textbf{Đúng}. 
 \\Độ lệch chuẩn của mẫu số liệu lớp 12A là $s_A \approx \sqrt{35{,}83} \approx 5{,}99$.\\
 Độ lệch chuẩn của mẫu số liệu lớp 12B là $s_B \approx \sqrt{27{,}38} \approx 5{,}23$.\\
 Vậy học sinh lớp 12B có chiều cao đồng đều hơn học sinh lớp 12A vì có độ lệch chuẩn nhỏ hơn.
 \end{itemchoice}
 }
\end{ex}
\Closesolutionfile{ans}

\caukq
\Opensolutionfile{ans}[ans/ans\currfilebase-Phan-III]
\begin{ex}%[2-D3B3-SO-11-2425]%[VN-MT-7, Nguyen Huynh]%[2D3N1-2]%[Câu 1]
 Bảng sau đây cho biết chiều cao (đơn vị: cm) của học sinh lớp 5A:
 \begin{center}
 \begin{tabular}{|c|c|c|c|c|c|c|}
 \hline
 Chiều cao & $[85; 90)$ & $[90; 95)$ & $[95; 100)$ & $[100; 105)$ & $[105; 110)$& $[110; 115)$\\
 \hline Tần số & $1$ & $4$ & $8$ & $12$ & $3$ & $2$ \\ \hline
 \end{tabular}
 \end{center}
 Tìm khoảng biến thiên của mẫu số liệu ghép nhóm về chiều cao của học sinh lớp 5A.
 
 \shortans[]{30}
 \loigiai{
 Khoảng biến thiên của mẫu số liệu ghép nhóm trên là $R=115-85=30$.
 }
\end{ex}

\begin{ex}%[2-D3B3-SO-11-2425]%[VN-MT-7, Nguyen Huynh]%[2D3H1-3]
 Bảng sau đây cho biết chiều cao (đơn vị: cm) của học sinh lớp 5A:
 \begin{center}
 \begin{tabular}{|c|c|c|c|c|c|c|}
 \hline
 Chiều cao & $[85; 90)$ & $[90; 95)$ & $[95; 100)$ & $[100; 105)$ & $[105; 110)$& $[110; 115)$\\
 \hline 
 Tần số & $1$ & $4$ & $8$ & $12$ & $3$ & $2$ \\ 
 \hline
 \end{tabular}
 \end{center}
 Tìm khoảng tứ phân vị của mẫu số liệu ghép nhóm về chiều cao của học sinh lớp 5A (làm tròn đến hàng phần mười).
 
 \shortans[]{7{,}4}
 \loigiai{
 Cỡ mẫu là $n=1+4+8+12+3+2=30$.\\
 Gọi $x_1$; $x_2$; $\ldots$; $x_{30}$ là mẫu số liệu gốc về chiều cao của học sinh lớp 5A được sắp xếp theo thứ tự không giảm.\\
 Khi đó 
 $x_1 \in [85;90)$;
 $x_2$, $x_3$, $x_4$, $x_5\in [90;95)$;
 $x_6$, $\ldots$, $x_{13} \in [95;100)$;
 $x_{14}$, $\ldots$, $x_{25} \in [100;105)$;
 $x_{26}$, $x_{27}$, $x_{28} \in [105;110)$;
 $x_{29}$, $x_{30} \in [110;115)$;\\
 Tứ phân vị thứ nhất của mẫu số liệu gốc là $x_{8} \in [95;100)$.\\
 Do đó tứ phân vị thứ nhất của mẫu số liệu ghép nhóm là 
 \[Q_1=95+\dfrac{\dfrac{30}{4}-5}{8} \cdot (100-95)\approx 96{,}56.\]
 Tứ phân vị thứ ba của mẫu số liệu gốc là $x_{23} \in [100;105)$.\\ 
 Do đó tứ phân vị thứ ba của mẫu số liệu ghép nhóm là 
 \[Q_3=100+\dfrac{\dfrac{3\cdot 30}{4}-13}{12} \cdot (105-100)\approx 103{,}96.\]
 Vậy khoảng tứ phân vị của mẫu số liệu trên là $\Delta_Q=Q_3-Q_1\approx 103{,}96-96{,}56=7{,}4$.}
\end{ex}

\begin{ex}%[2-D3B3-SO-11-2425]%[VN-MT-7, Nguyen Huynh]%[2D3H2-2]
 Số người xem trong $60$ buổi chiếu phim của một rạp chiếu phim nhỏ.
 \begin{center}
 \begin{tabular}{|c|c|c|c|c|c|c|c|}
 \hline
 Lớp người xem & $[0; 10)$&$[10; 20)$&$[20; 30)$&$[30; 40)$&$[40; 50)$&$[50; 60]$&Cộng\\
 \hline
 Tần số & $5$&$9$&$11$&$15$&$12$&$8$&$60$\\
 \hline
 \end{tabular}
 \end{center}
 Hãy tính phương sai của mẫu số liệu ghép nhóm trên (kết quả được làm tròn đến hàng đơn vị). 
 \par\shortans[]{220}
 \loigiai{
 \begin{center}
 \begin{tabular}{|c|c|c|c|c|c|c|c|}
 \hline
 Lớp người xem & $[0; 10)$&$[10; 20)$&$[20; 30)$&$[30; 40)$&$[40; 50)$&$[50; 60]$&Cộng\\
 \hline
 Giá trị đại diện & $5$ & $15$ & $25$ & $35$ & $45$ & $55$ & \\
 \hline
 Tần số & $5$&$9$&$11$&$15$&$12$&$8$&$60$\\
 \hline
 \end{tabular}
 \end{center}
 Số trung bình của mẫu số liệu ghép nhóm là
 \begin{eqnarray*}
 \overline{x}&=&\dfrac{n_1c_1+n_2c_2+n_3c_3+n_4c_4+n_5c_5+n_6c_6}{n}\\
 &=&\dfrac{5 \cdot 5+9 \cdot 15+11 \cdot 25+15 \cdot 35+12 \cdot 45+8 \cdot 55}{60}\\
 &=&\dfrac{97}{3}\approx 32{,}3.
 \end{eqnarray*}
 Phương sai của mẫu số liệu ghép nhóm là
% \begin{eqnarray*}
% s_x^2&=&\dfrac{n_1(c_1-\overline{x})^2+n_2(c_2-\overline{x})^2+n_3(c_3-\overline{x})^2+n_4(c_4-\overline{x})^2+n_5(c_5-\overline{x})^2+n_6(c_6-\overline{x})^2}{n}\\
% &=&\dfrac{5(5-32{,}3)^2+9(15-32{,}3)^2+11(25-32{,}3)^2+15(35-32{,}3)^2}{60}+\\
% && +\dfrac{12(45-32{,}3)^2+8(55-32{,}3)^2}{60}\\
% &\approx& 220.
% \end{eqnarray*} 
 \[s_x^2=\dfrac{5 \cdot 5^2+9 \cdot 15^2+11 \cdot 25^2+15 \cdot 35^2+12 \cdot 45^2+8 \cdot 55^2}{60}-\left(\dfrac{97}{3}\right)^2 \approx 220.\] 
 }
\end{ex}

\begin{ex}%[2-D3B3-SO-11-2425]%[VN-MT-7, Nguyen Huynh]%[2D3H2-2]
 Bảng thống kê cự li ném tạ của một vận động viên như sau:
 \begin{center}
 \begin{tabular}{|c|c|c|c|c|c|}
 \hline
 Cự li (m) & $[19; 19{,}5)$&$[19{,}5; 20)$&$[20; 20{,}5)$&$[20{,}5; 21)$&$[21; 21{,}5)$\\
 \hline
 Tần số & $13$ & $45$ & $24$ & $12$ & $6$\\
 \hline
 \end{tabular}
 \end{center} 
 Hãy tính độ lệch chuẩn của mẫu số liệu ghép nhóm trên (kết quả được làm tròn đến hàng phần trăm).
 
 \shortans[]{0{,}53}
 \loigiai{
 \begin{center}
 \begin{tabular}{|c|c|c|c|c|c|}
 \hline
 Cự li (m) & $[19; 19{,}5)$&$[19{,}5; 20)$&$[20; 20{,}5)$&$[20{,}5; 21)$&$[21; 21{,}5)$\\
 \hline
 Giá trị đại diện & $19{,}25$ & $19{,}75$ & $20{,}25$ &$20{,}75$ &$21{,}25$ \\
 \hline
 Tần số & $13$ & $45$ & $24$ & $12$ & $6$\\
 \hline
 \end{tabular}
 \end{center}
 Cỡ mẫu $n=100$.
 \\Số trung bình \[\overline{x}=\dfrac{13 \cdot 19{,}25+45 \cdot 19{,}75+24 \cdot 20{,}25+12 \cdot 20{,}75+6 \cdot 21{,}25}{100}=20{,}015\]
 Phương sai
 \begin{eqnarray*}
 s^2&=&\dfrac{13 \cdot \left(19{,}25-20{,}015\right)^2+45 \cdot \left(19{,}75-20{,}015\right)^2+24 \cdot \left(20{,}25-20{,}015\right)^2}{100}\\
 && +\dfrac{12 \cdot \left(20{,}75-20{,}015\right)^2+6 \cdot \left(21{,}25-20{,}015\right)^2}{100} \\
 &\approx& 0{,}28. 
 \end{eqnarray*}
 Độ lệch chuẩn $s=\sqrt{0{,}28} \approx 0{,}53$.
 }
\end{ex}

\begin{ex}%[2-D3B3-SO-11-2425]%[VN-MT-7, Nguyen Huynh]%[2D3H2-3]
 Thành tích môn nhảy cao của các vận động viên tại một giải điền kinh dành cho học sinh trung học phổ thông như sau:
 \begin{center}
 \begin{tabular}{|c|c|c|c|c|}
 \hline
 Mức xà (cm) & $[170;172)$ & $[172;174)$ & $[174;176)$ & $[176;180)$ \\ \hline
 Số vận động viên & $3$ & $10$ & $6$ & $1$ \\ \hline
 \end{tabular}
 \end{center}
 Tính khoảng tứ phân vị $\Delta_Q$ và độ lệch chuẩn $s$ của mẫu số liệu ghép nhóm trên. Khi đó $\Delta_Q+s$ bằng bao nhiêu (kết quả làm tròn đến hàng phần trăm)?
 \par\shortans[]{3{,}96}
 \loigiai{
 Cỡ mẫu là $n=3+10+6+1=20$.\\
 Gọi $x_1$, $x_2$, $\ldots$, $x_{20}$ là mức xà của $20$ vận động viên được sắp xếp theo thứ tự tăng dần.\\
 Tứ phân vị thứ nhất của mẫu số liệu gốc là $\dfrac{x_5+x_6}{2}\in [172; 174)$. \\
 Do đó, tứ phân vị thứ nhất của mẫu số liệu ghép nhóm là \[Q_1=172+\dfrac{\dfrac{20}{4}-3}{10}\cdot(174-172)=172{,}4.\]
 Tứ phân vị thứ ba của mẫu số liệu gốc là $\dfrac{x_{15}+x_{16}}{2}\in [174; 176)$. \\
 Do đó, tứ phân vị thứ ba của mẫu số liệu ghép nhóm là \[Q_3=174+\dfrac{\dfrac{3\cdot20}{4}-13}{6}\cdot(176-174)\approx 174{,}7.\]
 Do đó khoảng tứ phân vị là $\Delta_Q=Q_3-Q_1\approx 174{,}7-172{,}4\approx 2{,}3$.\\
 Chọn giá trị đại diện cho mẫu số liệu ta có
 \begin{center}
 \begin{tabular}{|c|c|c|c|c|}
 \hline
 Mức xà (cm) &$[170; 172)$ &$[172; 174)$ &$[174; 176)$ &$[176; 180)$\\
 \hline
 Giá trị đại diện & $171$ & $173$ & $175$ & $178$\\
 \hline
 Số vận động viên & $3$ & $10$ & $6$ & $1$\\
 \hline
 \end{tabular}
 \end{center}
 Mức xà trung bình là $\overline{x}=\dfrac{3\cdot171+10\cdot173+6\cdot175+1\cdot178}{20}=173{,}55$.\\
 Phương sai và độ lệch chuẩn
 \[S^2=\dfrac{1}{20}\left(3\cdot171^2+10\cdot173^2+6\cdot175^2+1\cdot178^2\right)-(173{,}55)^2\approx 2{,}75.\]
 Suy ra $s=\sqrt{s^2}=\sqrt{2{,}75}\approx 1{,}66$. 
 Khi đó $\Delta_Q+s\approx 3{,}96$.
 }
\end{ex}

\begin{ex}%[2-D3B3-SO-11-2425]%[VN-MT-7, Nguyen Huynh]%[2D3H1-3]
 Một người ghi lại thời gian đàm thoại của một số cuộc gọi cho kết quả như bảng sau:
 \begin{center}
 \begin{tabular}{|c|c|}
 \hline
 Thời gian (phút) & Số cuộc gọi \\
 \hline
 $0\le t<1$ & $8$ \\
 \hline
 $1\le t<2$ & $17$ \\
 \hline
 $2\le t<3$ & $25$ \\
 \hline
 $3\le t<4$ & $20$ \\
 \hline
 $4\le t<5$ & $10$ \\
 \hline
 \end{tabular}
 \end{center}
 Tìm khoảng tứ phân vị của mẫu số liệu ghép nhóm trên (làm tròn đến hàng phần mười).
 \par\shortans[]{1{,}8}
 \loigiai{ 
 Ta có bảng mẫu số liệu ghép nhóm được viết lại như sau:
 \begin{center}
 \begin{tabular}{|c|c|c|c|c|c|}
 \hline
 Thời gian t (phút) & $[0;1)$ & $[1; 2)$ & $[2; 3)$ & $[3; 4)$ & $[4; 5)$ \\
 \hline
 Số cuộc gọi & $8$ & $17$ & $25$ & $20$ & $10$ \\
 \hline
 \end{tabular}
 \end{center}
 Có cỡ mẫu $n=8+17+25+20+10=80$.\\
 Giả sử $x_1$, $x_2$, $\ldots$, $x_{80}$ là thời gian đàm thoại của $80$ cuộc gọi được sắp xếp theo thứ tự tăng dần.\\
 Tứ phân vị thứ nhất của mẫu số liệu gốc là $\dfrac{x_{20}+x_{21}}{2} \in [1; 2)$ nên nhóm chứa tứ phân vị thứ nhất là $[1; 2)$.
 \[Q_1=1+\dfrac{\dfrac{80}{4}-8}{17} \left(2-1\right)\approx 1{,}7.\]
 Tứ phân vị thứ ba của mẫu số liệu gốc là $\dfrac{x_{60}+x_{61}}{2} \in [3; 4)$ nên nhóm chứa tứ phân vị thứ ba là $[3; 4)$.
 \[Q_3=3+\dfrac{\dfrac{3\cdot 80}{4}-50}{20} \left(4-3\right)=3{,}5.\]
 Khoảng tứ phân vị của mẫu số liệu ghép nhóm là $\Delta_Q=Q_3-Q_1 \approx 1{,}8$. 
 }
\end{ex}
\Closesolutionfile{ans}
\begin{indapan}
	{ans/ans\currfilebase}
\end{indapan}



%%GK2
% \begin{name}
	{\tenchude}
	{TOÁN 12}
	{LỚP TOÁN THẦY PHÁT}
	{Thời gian: 90 phút - Không kể thời gian phát đề}
\end{name}
\TN
\Opensolutionfile{ans}[ans/ansDe1-TN1]
\begin{ex}%[2D4N1-1]%[Dự án EX-TF-TLN lần 3 -Mui Doan]
	Nguyên hàm của hàm số $f(x)=x^n$ (với $n\neq -1$) là
	\choice
	{\True $\dfrac{x^{n+1}}{n+1}+C$}
	{$x^{n+1}+C$}
	{$\dfrac{n+1}{x^{n+1}}+C$}
	{$\dfrac{1}{x^{n+1}}+C$}
	\loigiai{
		Nguyên hàm của hàm số $f(x)=x^n$ (với $n\neq -1$) là	$\dfrac{x^{n+1}}{n+1}+C$.
	}
\end{ex}

\begin{ex}%[2D4N1-1]%[Tổ 19 - Đợt 17 - Chương 4 - - CD - Đề 2]%[Bình]
	Hàm số $y=F(x)$ là một nguyên hàm của hàm số $y=f(x)$. Hãy chọn khẳng định \textbf{đúng}.
	\choice
	{$F(x)=f'(x)$}
	{\True $F'(x)=f(x)$}
	{$F(x)=f'(x)+C$}
	{$F'(x)+C=f(x)$}
	\loigiai
	{
		Khẳng định đúng là: $F'(x)=f(x)$.
	}
\end{ex}

\begin{ex}%[2D4N1-2]
	Họ nguyên hàm của hàm số $f(x)=3x^2+1$ là
	\choice
	{$x^3+C$}
	{$\dfrac{x^3}{3}+x+C$}
	{$6x+C$}
	{\True $x^3+x+C$}
	\loigiai{
		$\displaystyle\int{(3x^2+1)\mathrm{\,d}x=x^3+x+C}$.}
\end{ex}

\begin{ex}%[12-MH-2-MH2025]%[MH-2025,Chu Hà]%[2D4N1-3]
	Họ nguyên hàm của hàm số $f(x)= \dfrac{1}{\sqrt{x}} $ là
	\choice
	{\True $2\sqrt{x}+C$}
	{$\sqrt{x} + C$}
	{$- \sqrt{x} +C$}
	{$-2\sqrt{x}+C$}
	\loigiai
	{
	Ta có $\displaystyle \int\dfrac{1}{\sqrt{x}}\mathrm{\,d}x=\int x^{-\tfrac{1}{2}}\mathrm{\,d}x=2x^{\tfrac{1}{2}}+C=2\sqrt{x}+C$.
	}
\end{ex}

\begin{ex}%[2D4N1-4]
	Phát biểu nào sau đây là đúng?
	\choice
	{$\displaystyle\int e^{-3 x} \mathrm{~d} x=e^{-3 x}+C$}
	{\True $\displaystyle\int e^{-3 x} \mathrm{~d} x=-\frac{1}{3} e^{-3 x}+C$}
	{$\displaystyle\int e^{-3 x} \mathrm{~d} x=\frac{1}{3} e^{-3 x}+C$}
	{$\displaystyle\int e^{-3 x} \mathrm{~d} x=-\frac{1}{3} e^{-3 x}$}
	\loigiai{$\displaystyle\int e^{-3 x} \mathrm{~d} x=-\frac{1}{3} e^{-3 x}+C$.}
\end{ex}

\begin{ex}%[Dự án 2025 - đề cấu trúc mới, Nguyễn Kiều Nhã Tú]%[2D4N1-4]
	Họ nguyên hàm của hàm số $f(x)=\dfrac{1}{x^2}+2^x$ là
	\choice
	{$\ln x^2+2^x \cdot \ln 2+C$}
	{$\ln x^2+\dfrac{2^x}{\ln 2}+C$}
	{\True $-\dfrac{1}{x}+\dfrac{2^x}{\ln 2}+C$}
	{$\dfrac{1}{x}+2^x \cdot \ln 2+C$}
	\loigiai{
		Ta có: $\displaystyle\int\left(\dfrac{1}{x^2}+2^x\right)\mathrm{\,d} x=-\dfrac{1}{x}+\dfrac{2^x}{\ln 2}+C$.
	}
\end{ex}

\begin{ex}%[12-MH-2-MH2025]%[MH-2025, Nguyễn Trần Phong]%[2D4H1-2]
	Tìm một nguyên hàm $F(x)$ của hàm số $ f(x)=3 x^{2} + 5$ biết $F(6) =241$.
	\choice
	{ $ F(x)=x^{3} + 10 x + 15$ }
	{ \True $ F(x)=x^{3} + 5 x - 5$ }
	{ $ F(x)=x^{3} + 5 x - 25$ }
	{ $ F(x)=x^{3} + 8 x^{2} + 5 x - 5$ }
	\loigiai{
	$\displaystyle \int {\left(3 x^{2} + 5\right)\mathrm{\,d}x} = x^{3} + 5 x + C$.\\
	Mà $F(6) =241 \Leftrightarrow 6^3 + 5 \cdot 6 + C =241 \Leftrightarrow C = -5 $.\\
	Vậy $ F(x)=x^{3} + 5 x - 5$.
	}
\end{ex}

\begin{ex}%[2D4H1-3]
	$\displaystyle \int \left(\sin \dfrac{x}{2}+\cos\dfrac{x}{2} \right)^2 \mathrm{d}x$ bằng
	\choice
	{\True $x-\cos x+C$}
	{$\left(-\cos \dfrac{x}{2}+\sin\dfrac{x}{2} \right)^2$}
	{$\dfrac{1}{3} \left(\sin \dfrac{x}{2}+\cos\dfrac{x}{2}\right)^3$}
	{$x+\cos x+C$}
	\loigiai{
		Ta có
		\allowdisplaybreaks
		\begin{eqnarray*}
			\displaystyle \int \left(\sin \dfrac{x}{2}+\cos\dfrac{x}{2} \right)^2 \mathrm{d}x &=& \displaystyle \int \left(\sin^2 \dfrac{x}{2}+\cos^2\dfrac{x}{2}+2\sin \dfrac{x}{2}\cos \dfrac{x}{2} \right) \mathrm{d}x\\
			&=& \displaystyle \int \left(1+\sin x \right) \mathrm{d}x\\
			&=& \displaystyle \int \mathrm{d}x+\displaystyle \int \sin x \mathrm{d}x\\
			&=& x-\cos x+C.
		\end{eqnarray*}
	}
\end{ex}

\begin{ex}%[Cau-2]%[2D4N2-1]
	Cho hàm số $f(x)$ liên tục trên $\mathbb{R}$ và $F(x)$ là nguyên hàm của $f(x)$, biết $\displaystyle \int\limits_{0}^{9} f(x)\mathrm{\,d}x=9$ và $F(0)=3$. Tính $F(9)$.
	\choice
	{$F(9)=-6$}
	{$F(9)=6$}
	{\True $F(9)=12$}
	{$F(9)=-12$}
	\loigiai{
		Ta có $I=\displaystyle\int\limits_{0}^{9} f(x)\mathrm{\,d}x = F(x)\Big|_0^9 = F(9)- F(0)=9\Leftrightarrow F(9)=9 + F(0)=9 + 3= 12$.
	}
\end{ex}

\begin{ex}%[2D4N2-2]
	Tích phân $\displaystyle\int\limits_{1}^{2} x^3\mathrm{\,d}x$ bằng
	\choice{$\dfrac{15}{3}$}{$\dfrac{17}{4}$}{$\dfrac{7}{4}$}{\True$\dfrac{15}{4}$}
	\loigiai{
		Ta có $\displaystyle\int\limits_{1}^{2} x^3\mathrm{\,d}x=\dfrac{x^4}{4}\Big|_1^2=\dfrac{15}{4}$.
	}
\end{ex}

\begin{ex}%[2D4N2-3]
	Tính tích phân $\displaystyle\int\limits_0^\pi \sin 3 x \mathrm{\,d} x$.
	\choice
	{$-\dfrac{1}{3}$}
	{$\dfrac{1}{3}$}
	{$-\dfrac{2}{3}$}
	{\True $\dfrac{2}{3}$}
	\loigiai{Ta có $\displaystyle\int\limits_0^\pi \sin 3 x \mathrm{\,d} x=-\left.\dfrac{1}{3} \cos 3 x\right|_0 ^\pi=-\dfrac{1}{3}(-1-1)=\dfrac{2}{3}$.}
\end{ex}

\begin{ex}%[Tổ 20 - Chương 4 - - CD]%[Nguyễn Văn Sang]%[2D4N2-4]
	Tính giá trị tích phân $\displaystyle\int\limits_1^3 3\cdot 5^x \mathrm{~d}x$.
	\choice
	{$\dfrac{\mathrm{e}^2}{\ln 5}$ }
	{\True $\dfrac{360}{\ln 5}$ }
	{$\dfrac{\mathrm{e}^3-\mathrm{e}}{\ln 5}$ }
	{$\dfrac{320}{\ln 5}$ }
	\loigiai{
		Ta có $\displaystyle\int\limits_1^3 3\cdot 5^x \mathrm{~d}x=3 \displaystyle\int\limits_1^3 5^x \mathrm{~d}x=\dfrac{3\cdot 5^x}{\ln 5}\bigg|_1 ^3=\dfrac{360}{\ln 5}$.
	}
\end{ex}
\Closesolutionfile{ans}

\TNTF
\Opensolutionfile{ans}[ans/ansDe1-TN2]
\begin{ex}%[Dự án EX - TF - TLN 2024]%[Doan Hung]%[2D4H1-2]
	Cho hàm số $F(x)=x^3-2x+1$, $x \in \mathbb{R}$ là một nguyên hàm của hàm số $f(x)$.
	\choiceTF
	{Nếu hàm số $G(x)$ cũng là một nguyên hàm của hàm số $f(x)$ và $G(-1)=3$ thì $G(x)=F(x)-1, x \in \mathbb{R}$}
	{\True Nếu hàm số $H(x)$ cũng là một nguyên hàm của hàm số $f(x)$ và $H(1)=-3$ thì $H(x)=F(x)-3, x \in \mathbb{R}$}
	{Nếu hàm số $K(x)$ cũng là một nguyên hàm của hàm số $f(x)$ và $K(0)=0$ thì $K(x)=F(x)+1, x \in \mathbb{R}$}
	{\True Nếu hàm số $M(x)$ cũng là một nguyên hàm của hàm số $f(x)$ và $M(2)=4$ thì $M(x)=F(x)-1, x \in \mathbb{R}$}
	\loigiai{
	\begin{itemchoice}
	\itemch Vì hàm số $G(x)$ cũng là một nguyên hàm của hàm số $f(x)$ nên $G(x)=F(x)+C$.\\
	Vì $G(-1)=3\Leftrightarrow F(-1)+C=3\Leftrightarrow C=3-F(-1)\Leftrightarrow C=3-2=1$.\\
	Vậy $G(x)=F(x)+1$.
	\itemch Vì hàm số $H(x)$ cũng là một nguyên hàm của hàm số $f(x)$ nên $H(x)=F(x)+C$.\\
	Vì $H(1)=-3\Leftrightarrow F(1)+C=-3\Leftrightarrow C=-3-F(1)\Leftrightarrow C=-3-0=-3$.\\
	Vậy $H(x)=F(x)-3$.
	\itemch Vì hàm số $K(x)$ cũng là một nguyên hàm của hàm số $f(x)$ nên $K(x)=F(x)+C$.\\
	Vì $K(0)=0\Leftrightarrow F(0)+C=0\Leftrightarrow C=-F(0)\Leftrightarrow C=-1$.\\
	Vậy $H(x)=F(x)-1$.
	\itemch Vì hàm số $M(x)$ cũng là một nguyên hàm của hàm số $f(x)$ nên $M(x)=F(x)+C$.\\
	Vì $M(2)=4\Leftrightarrow F(2)+C=4\Leftrightarrow C=4-F(2)\Leftrightarrow C=-1$.\\
	Vậy $H(x)=F(x)-1$.
	\end{itemchoice}
	}
	\end{ex}

\begin{ex}%[2D4N2-2]
	Cho hàm số $f(x)=x^2$.
	\choiceTF
	{\True $\displaystyle\int f(x)\mathrm{\,d}x=\dfrac{x^3}{3}+C$}
	{$\displaystyle\int\limits_0^2 f(x)\mathrm{\,d}x=\dfrac{7}{3}$}
	{Giả sử $F(x)$ là một nguyên hàm của $f(x)$. Khi đó $f'(x)=F(x)$}
	{\True Gọi $F(x)$ là một nguyên hàm của $f(x)$. Nếu đồ thị hàm số của $F(x)$ đi qua điểm $(3;1)$ thì $F(x)=\dfrac{x^3}{3}-8$.}
	\loigiai{
		\begin{itemchoice}
			\itemch  $\displaystyle\int x^2\mathrm{\,d}x=\dfrac{x^3}{3}+C$.\\
			\itemch $\displaystyle\int\limits_0^2 x^2\mathrm{\,d}x=\dfrac{x^3}{3}\Big|_0^2=\dfrac{8}{3}-0=\dfrac{8}{3}$.\\
			\itemch  $F(x)$ là nguyên hàm của $f(x)$ thì $F'(x)=f(x)$.\\
			\itemch Nguyên hàm $F(x)=\dfrac{x^3}{3}+C$. Mà $F(x)$ đi qua  $(3;1)$ nên $C=-8$.
		\end{itemchoice}
	}
\end{ex}
\Closesolutionfile{ans}

\TNSA
\Opensolutionfile{ans}[ans/ansDe1-TN3]
\begin{ex}%[2D4H1-1]%[Đào Trung Kiên]
	Biết $ F(x) $ là một nguyên hàm của hàm số $ f(x) = \mathrm{e}^{2x} $ và $ F(0) = 0$. Tính giá trị của $F(\ln 3)$.
	\shortans[]{$4$}
	\loigiai{
		Ta có $ \heva{& F(0) = 0 \\ & F(x) = \dfrac{1}{2} \cdot \mathrm{e}^{2x} + C } \Rightarrow F(x) = \dfrac{1}{2} \cdot \mathrm{e}^{2x} - \dfrac{1}{2} \Rightarrow F(\ln 3) =  \dfrac{1}{2}  \cdot \left  (  \mathrm{e}^{ 2 \cdot \ln 3 } - 1 \right ) = 4$.
	}
\end{ex}

\begin{ex}%[2D4H2-2]%[Tổ 20 - Đợt 17 - Chương 4 - - CD - Đề 7]%[Lê Thị Thanh Tuyền]
	Biết rằng $\displaystyle\int\limits_{-1}^3[2f(x)-3g(x)] \mathrm{\,d} x=10$ và $\displaystyle\int\limits_{-1}^3[3f(x)+g(x)] \mathrm{\,d} x=4$.
	Tích phân $\displaystyle\int\limits_{-1}^3[10f(x)+7g(x)] \mathrm{\,d} x$ bằng
	\shortans{$6$}

	\loigiai{

	Đặt $a=\displaystyle\int\limits_{-1}^3f(x) \mathrm{\,d} x,\quad b=\displaystyle\int\limits_{-1}^3g(x) \mathrm{\,d} x$.\\
	Ta có $\displaystyle\int\limits_{-1}^3[2f(x)-3g(x)] \mathrm{\,d} x=10$ và $\displaystyle\int\limits_{-1}^3[3f(x)+g(x)]\mathrm{\,d} x=4$.\\
	Suy ra  $\heva{&2a-3b=10\\& 3a+b=4} \Leftrightarrow\heva{&a=2\\&b=-2.}$
	\\
	Vậy $\displaystyle\int\limits_{-1}^3[10f(x)+7g(x)] \mathrm{\,d} x=10a+7b=6$.
	}
\end{ex}

\begin{ex}%[ST12-dot1-Trần xuân Hòa]%[2D4H3-1]
	Tính diện tích hình phẳng giới hạn bởi đồ thị các hàm số $y=-x^2+2x$ và $y=-3x$. (Kết quả làm tròn đến chữ số thập phân thứ nhất).
	\shortans{$20{,}8$}
	\loigiai
	{Phương trình hoành độ giao điểm $-x^2+2x=-3x\Leftrightarrow\hoac{&x=0\\&x=5.}$\\
		Khi đó diện tích $S$ của hình phẳng được xác định bởi
		\begin{eqnarray*}
			&S&=\displaystyle\int \limits_0^5|-x^2+2x+3x|\mathrm{\, d}x\\
			&&=\displaystyle\int\limits_0^5|-x^2+5x|\mathrm{\, d}x\\
			&&=\left| \displaystyle\int\limits_0^5(-x^2+5x)\mathrm{\, d}x\right|\\
			&&=\left| \left(-\dfrac{x^3}{3}+\dfrac{5x^2}{2}\right)\Bigg|_0^5\right| =\dfrac{125}{6}\approx 20{,}8.
		\end{eqnarray*}
	}
\end{ex}

\begin{ex}%[Vovanle]%[2D4H3-3]
	Cho hình phẳng giới hạn bởi các đường $y=\sqrt{x}-2$, $y=0$ và $x=9$ quay xung quanh trục $Ox$. Tính thể tích khối tròn xoay tạo thành (làm tròn kết quả thể tích đến hàng phần trăm).
	\shortans{$5{,}76$}
	\loigiai{
		Phương trình hoành độ giao điểm của đồ thị hàm số $y=\sqrt{x}-2$ và trục hoành
		\[\sqrt{x}-2=0\Leftrightarrow \sqrt{x}=2\Leftrightarrow x=4.\]
		Thể tích của khối tròn xoay tạo thành là
		\allowdisplaybreaks
		\begin{eqnarray*}
			V&=&\pi \displaystyle\int\limits_4^{9}{{{\left(\sqrt{x}-2\right)}^2}\mathrm{\,d}x}\\
			&=&\pi\displaystyle\int\limits_4^{9}{\left(x-4\sqrt{x}+4\right)}\mathrm{\,d}x\\
			&=&\pi\left.\left(\dfrac{x^2}2-\dfrac{8x\sqrt{x}}3+4x\right)\right|_4^{9}\\
			&=&\pi\left(\dfrac{81}{2}-72+36\right)-\pi\left(\dfrac{16}{2}-\dfrac{64}{3}+16\right)\\
			&=&\dfrac{11\pi}{6}\approx 5{,}76.
		\end{eqnarray*}
	}
\end{ex}

\Closesolutionfile{ans}

\TL
\begin{ex}%[2D4H2-2]
	Cho số thực $a>1$, tính tích phân
	$\displaystyle\int\limits_0^a {|x-1|}\mathrm{\,d}x$ theo $ a$.
	\loigiai{
		Ta có hàm số $ f(x)=x-1$ có một nguyên hàm $F(x)=\dfrac{x^2}{2}-x$.\\
		$f(x)=|x-1|=\heva{x-1 & \text{ nếu } x\ge 1\\ -(x-1) & \text{ nếu } x<1.}$\\
		\begin{eqnarray*}
			\text{Ta lại có } \displaystyle\int\limits_0^a {|x-1|}\mathrm{\,d}x
			&=& \displaystyle\int\limits_0^1 {|x-1|}\mathrm{\,d}x+
			\displaystyle\int\limits_1^a {|x-1|}\mathrm{\,d}x\\
			&=& \displaystyle\int\limits_0^1 {-(x-1)}\mathrm{\,d}x+
			\displaystyle\int\limits_1^a {(x-1)}\mathrm{\,d}x\\
			&=& F(0) - F(1) + F(a) - F(1) = \dfrac{a^2}{2}-a+1.
		\end{eqnarray*}
	}
\end{ex}

\begin{ex}%[2D4V1-4]
	$F(x)$ là một nguyên hàm của hàm số $f(x)=2^x$, thỏa mãn $ F(0)=\dfrac{1}{\ln 2}$. Biểu thức $ F(0)+F(1)+F(2)+\ldots+F(2024)=\dfrac{a^b-c}{\ln a}$ $(a$, $b$, $c\in{N^*})$. Tính $ T=a+b-2c$.
	% \shortans{2025}
	\loigiai{
		Ta có $F(x)=\displaystyle\int 2^x \mathrm{\,d}x=\dfrac{2^x}{\ln 2}+C$.\\
		Theo giả thiết $F(0)=\dfrac{1}{\ln 2}\Leftrightarrow\dfrac{2^0}{\ln 2}+C=\dfrac{1}{\ln 2}\Leftrightarrow C=0 \Rightarrow F(x)=\dfrac{2^x}{\ln 2}$.\\
		Khi đó
		\allowdisplaybreaks
		\begin{eqnarray*}
			F(0)+F(1)+F(2)+\ldots+F(2024)&=&\dfrac{2^0}{\ln 2}+\dfrac{2^1}{\ln 2}+\dfrac{2^2}{\ln 2}+\ldots+\dfrac{2^{2024}}{\ln 2}\\
			&=&\dfrac{1}{\ln 2}(2^0+2^1+2^2+\ldots+2^{2024})\\
			&=&\dfrac{1}{\ln 2}\cdot \dfrac{1(1-2^{2025})}{1-2}=\dfrac{2^{2025}-1}{\ln 2}.
		\end{eqnarray*}
		$\Rightarrow a=2$, $b=2025$, $c=1$.\\
		Vậy $ T=a+b-2c=2025$.}
\end{ex}

\begin{ex}%[2D4C3-2]
	\immini
	{Ông An xây dựng một sân bóng đá mini hình chữ nhật có chiều rộng $30$m và chiều dài $50$m. Để giảm bớt chi phí cho việc trồng cỏ nhân tạo, ông An chia sân bóng ra làm hai phần (tô đen và không tô đen) như hình bên. Phần tô đen gồm hai phần diện tích bằng nhau và đường cong $AIB$ là một parabol đỉnh $I$ được trồng cỏ nhân tạo với giá $130\,000$ đồng/m$^2$ và phần còn lại được trồng với giá $90\,000$ đồng/m$^2$.
	}
	{\begin{tikzpicture}[scale=0.9, font=\footnotesize,line join=round, line cap=round, >=stealth]
			\coordinate (I) at (0,0);
			\coordinate (A) at (1.5,2.25);
			\coordinate (B) at ($(A)+(0,-4.5)$);
			\coordinate (C) at ($(B)+(-7.5,0)$);
			\coordinate (D) at ($(A)-(B)+(C)$);
			\coordinate (M) at ($(A)!1/2!(D)$);
			\coordinate (N) at ($(B)!1/2!(C)$);
			\coordinate (P) at ($(A)+(-1.5,0)$);
			\coordinate (Q) at ($(B)+(-1.5,0)$);
			\coordinate (R) at ($(D)+(1.5,0)$);
			\coordinate (S) at ($(C)+(1.5,0)$);
			\coordinate (td) at ($(D)+(0,0.3)$);
			\coordinate (dt) at ($(D)+(-0.3,0)$);
			\coordinate (ct) at ($(C)+(-0.3,0)$);
			\coordinate (tr) at ($(R)+(0,0.3)$);
			\coordinate (rp) at ($(R)+(0.3,0)$);
			\coordinate (I') at ($(R)!1/2!(S)$);
			\coordinate (g) at ($(I')+(0.3,0)$);
			\fill[gray]plot[domain=0:1.5](\x,{sqrt(3.375*(\x))})--(A)--plot[domain=1.5:0](\x,{-sqrt(3.375*(\x))})--cycle;
			\fill[gray](C)--(D)--plot[domain=-6:-4.5](\x,{sqrt(3.375*(-\x-4.5))})--plot[domain=-4.5:-6](\x,{-sqrt(3.375*(-\x-4.5))})--cycle;
			\draw (A)--(B)--(C)--(D)--cycle (M)--(N);
			\draw[dashed] (P)--(Q) (R)--(S);
			\draw[<->](td)--(tr);
			\node at ($(td)!1/2!(tr)$)[above]{$10$ m};
			\draw[<->](dt)--(ct);
			\node at ($(dt)!1/2!(ct)$)[above,rotate=90]{$30$ m};
			\draw[<->](rp)--(g);
			\node at ($(rp)!1/2!(g)$)[above,rotate=-90]{$15$ m};
			\foreach \x/\g in {A/90,B/-90,I/180,I'/-40}\draw[fill=black] (\x) circle (.05) +(\g:.5)node{\footnotesize$\x$};
		\end{tikzpicture}}
	\noindent
	Hỏi ông An phải trả bao nhiêu tiền (triệu đồng) để trồng cỏ nhân tạo cho sân bóng.
	% \shortans{$151$}
	\loigiai{
		\immini{
			Chọn hệ trục tọa độ như hình vẽ ($I$ là gốc tọa độ). Khi đó đường cong $IAB$ là một parabol có phương trình dạng $y=ax^2$.\\
			Parabol đi qua điểm $\left(15;10 \right)$, suy ra
			\[a \cdot 15^2=10 \Rightarrow a=\dfrac{2}{45}.\]
		}
		{
			\begin{tikzpicture}[smooth,samples=300,scale=0.6,>=stealth]
				\fill[gray!30] (-4,2)--(4,2)--plot[domain=4:-4](\x,{0.125*(\x)^2});
				\draw[->] (-5,0)--(5,0) node[below]{$x$};
				\draw[->] (0,-1)--(0,3) node[right]{$y$};
				\draw (0,0) node[below left]{$I$};
				\draw[domain=-4:4] plot(\x,{0.125*(\x)^2});
				\draw[fill=black] (4,2) circle(1.5pt) (-4,2) circle(1.5pt);
				\draw[dashed] (4,0)node[below]{$15$}--(4,2)node[right]{$B$}--(-4,2)node[left]{$A$}--(-4,0)node[below]{$-15$};

				\node[right] at (0,2.4) {$10$};
			\end{tikzpicture}}
		\noindent
		Vậy $y=\dfrac{2}{45}x^2$. Diện tích phần tô đen là $S=2 \cdot \displaystyle\int\limits_{-15}^{15} \left(10-\dfrac{2}{45}x^2 \right) \mathrm{\,d}x=400 \, (\text{m}^2)$.\\
		Diện tích phần còn lại của sân bóng là $S_2=30 \cdot 50-400=1100\,(\text{m}^2).$\\
		Số tiền Ông An phải trả để trồng cỏ nhân tạo cho sân bóng là
		\[130000\times 400+90000\times 1100=151000000\text{ đồng}=151\text{ triệu đồng.}\]}
\end{ex}

% \Closesolutionfile{ansbook}
% \HetDe
% \label{De1}
% %
% \cleardoublepage
% \setcounter{page}{1}
% \rfoot{Trang \thepage/\pageref{DA1} - Đáp án trắc nghiệm Mã đề 1}
% \begin{center}
% 	\bfseries ĐÁP ÁN TRẮC NGHIỆM MÃ ĐỀ 1
% \end{center}

% \inputansbox{10}{ans/ansDe1-TN1}
% \inputansbox[3]{2}{ans/ansDe1-TN2}
% \inputansbox{3}{ans/ansDe1-TN3}
% \label{DA1}
% %

% \begin{name}
	{\tenchude}
	{TOÁN 12}
	{LỚP TOÁN THẦY PHÁT}
	{Thời gian: 90 phút - Không kể thời gian phát đề}
\end{name}
\TN
\Opensolutionfile{ans}[ans/ansDe2-TN1]
\begin{ex}%[EX-Ôn tập TN 2025, Nguyễn Tiến]%[2D4N1-1]
	Phát biểu nào sau đây là đúng?
	\choice
	{\True $\displaystyle\int\limits F'(x)\mathrm{\,d}x=F(x)+C$}
	{$\displaystyle\int\limits F(x)\mathrm{\,d}x=F'(x)+C$}
	{$\displaystyle\int\limits F(x)\mathrm{\,d}x=F(x)+C$}
	{$\displaystyle\int\limits F'(x)\mathrm{\,d}x=F'(x)+C$}
	\loigiai{
		Phát biểu đúng là $\displaystyle\int\limits F'(x)\mathrm{\,d}x=F(x)+C$.
	}
\end{ex}

\begin{ex}%[Đề 12 - 2025 - Nguyen Son]%[2D4H1-1]
	Cho hàm số $ f(x) $ có đạo hàm $ f'(x) $ liên tục và có một nguyên hàm là hàm số $ F(x) $. Tìm nguyên hàm $ I=\displaystyle\int \left[2f(x)+f'(x)+1\right]\mathrm{d}x $.
	\choice
	{\True $ I=2F(x)+f(x)+x+C $}
	{$ I=2F(x)+xf(x)+C $}
	{$ I=2xF(x)+f(x)+x+1 $}
	{$ I=2xF(x)+f(x)+x+C $}
	\loigiai{
		Ta có $ I=\displaystyle\int \left[2f(x)+f'(x)+1\right]\mathrm{d}x=2F(x)+f(x)+x+C $.
	}
\end{ex}

\begin{ex}%[2D4N1-2]
	Nguyên hàm của hàm số $f(x)=x^3+x$ là
	\choice
	{\True $\dfrac{1}{4}x^4+\dfrac{1}{2}x^2+C$}
	{$3x^2+1+C$}
	{$x^3+x+C$}
	{$x^4+x^2+C$}
	\loigiai{
		$\displaystyle\int{(x^3+x^2)\mathrm{\,d}x}=\dfrac{1}{4}x^4+\dfrac{1}{2}x^2+C$.}
\end{ex}

\begin{ex}%[2D4N1-3]%[Dự án EX-TF-TLN lần 3 -Mui Doan]
	Họ tất cả các nguyên hàm của hàm số $f(x)=x+\sin x$ là
	\choice
	{$\dfrac{x^2}{2}+\cos x+C$}
	{$x^2+\cos x+C$}
	{$x^2-\cos x+C$}
	{\True $\dfrac{x^2}{2}-\cos x+C$}
	\loigiai{
		Ta có $\displaystyle\int f(x)\mathrm{\,d}x=\displaystyle\int\left(x+\sin x\right)\mathrm{\,d}x=\dfrac{x^2}{2}-\cos x+C$.}
\end{ex}

\begin{ex}%[Mui Doan, dự án 12EX-OTTN2025]%[2D4N1-4]
	Hàm số nào sau đây là một nguyên hàm của hàm số $y=10^x$?
	\choice
	{$y=10^x \ln 10$}
	{$y=10^x$}
	{$y=\dfrac{10^{x+1}}{x+1}$}
	{\True $y=\dfrac{10^x}{\ln 10}$}
	\loigiai{
		Theo công thức nguyên hàm, một nguyên hàm của $y=10^x$ là $y=\dfrac{10^x}{\ln 10}$.
	}
\end{ex}

\begin{ex}%[2D4N1-4]
	$\displaystyle\int (3^x+4^x)\mathrm{d}\,x$ bằng
	\choice
	{\True  $\dfrac{3^x}{\ln 3}+\dfrac{4^x}{\ln 4}+C$}
	{$\dfrac{3^x}{\ln 4}+\dfrac{4^x}{\ln 3}+C$}
	{$\dfrac{4^x}{\ln 3}-\dfrac{3^x}{\ln 4}+C$}
	{$\dfrac{3^x}{\ln 3}-\dfrac{4^x}{\ln 4}+C$}
	\loigiai{
		Áp dụng công thức $\displaystyle\int a^x\,\mathrm{d}x=\frac{a^x}{\ln a}+C$.\\
		Ta có $\displaystyle\int(3^x+4^x)\mathrm{d}\,x
			=\int 3^x\mathrm{d}\,x+\int 4^x\mathrm{d}\,x=\dfrac{3^x}{\ln 3}+\dfrac{4^x}{\ln 4}+C$.}
\end{ex}

\begin{ex}%[2D4H1-2]
	Tìm nguyên hàm của hàm số $ f(x)=\dfrac{{x^4}+2}{x^2}$.
	\choice
	{$\displaystyle\int{f(x)\mathrm{d}x=}\dfrac{{x^3}}{3}-\dfrac{1}{x}+C$}
	{\True $\displaystyle\int{f(x)\mathrm{d}x=}\dfrac{{x^3}}{3}+\dfrac{2}{x}+C$}
	{$\displaystyle\int{f(x)\mathrm{d}x=}\dfrac{{x^3}}{3}+\dfrac{1}{x}+C$}
	{\True $\displaystyle\int{f(x)\mathrm{d}x=}\dfrac{{x^3}}{3}-\dfrac{2}{x}+C$}
	\loigiai{
		Ta có: $\displaystyle\int{f(x)\mathrm{d}x=}\displaystyle\int{\dfrac{{x^4}+2}{{x^2}}}\mathrm{d}x=\displaystyle\int{( {x^2}+\dfrac{2}{{x^2}} )}\mathrm{d}x=\dfrac{{x^3}}{3}-\dfrac{2}{x}+C$.
	}
\end{ex}

\begin{ex}%[2D4H1-3]
	$\displaystyle \int \left(\cos \dfrac{x}{2} \right)^2 \mathrm{d}x$ bằng
	\choice
	{$x+\sin x+C$}
	{$\dfrac{1}{3}\left(\cos \dfrac{x}{2} \right)^3+C$}
	{$\left(\sin \dfrac{x}{2} \right)^2+C$}
	{\True $\dfrac{1}{2}x+ \dfrac{1}{2}\sin x+C$}
	\loigiai{
		Ta có
		\allowdisplaybreaks
		\begin{eqnarray*}
			\displaystyle \int \left(\cos \dfrac{x}{2} \right)^2 \mathrm{d}x &=& \displaystyle \int \dfrac{1+\cos x}{2} \mathrm{d}x\\
			&=& \dfrac{1}{2} \displaystyle \int \mathrm{d}x+ \dfrac{1}{2}\displaystyle \int \cos x \mathrm{d}x \\
			&=& \dfrac{1}{2}x+ \dfrac{1}{2}\sin x+C.
		\end{eqnarray*}
	}
\end{ex}

\begin{ex}%[Dự án 2025 - đề cấu trúc mới, Nguyễn Kiều Nhã Tú]%[2D4N2-1]
	Cho hai hàm số $f(x)$ và $g(x)$ liên tục trên $K$, $a$, $b\in K$. Khẳng định nào sau đây là \textbf{sai}?
	\choice
	{$\displaystyle\int\limits_a^b [f(x)+g(x)]\mathrm{\,d} x=\displaystyle\int\limits_a^b f(x)\mathrm{\,d} x+\displaystyle\int\limits_a^b g(x)\mathrm{\,d}x$}
	{$\displaystyle\int\limits_a^b 2f(x)\mathrm{\,d}x=2 \displaystyle\int\limits_a^b f(x)\mathrm{\,d}x$}
	{\True $\displaystyle\int\limits_a^b f(x)g(x)\mathrm{\,d} x=\displaystyle\int\limits_a^b f(x)\mathrm{\,d}x \cdot \displaystyle\int\limits_a^b g(x)\mathrm{\,d}x$}
	{$\displaystyle\int\limits_a^b[f(x)-g(x)]\mathrm{\,d} x=\displaystyle\int\limits_a^b f(x)\mathrm{\,d} x-\displaystyle\int\limits_a^b g(x)\mathrm{\,d}x$}
	\loigiai{
		Theo tính chất của tích phân, không có tính chất $\displaystyle\int\limits_a^b f(x)g(x)\mathrm{\,d} x=\displaystyle\int\limits_a^b f(x)\mathrm{\,d}x \cdot \displaystyle\int\limits_a^b g(x)\mathrm{\,d}x$.
	}
\end{ex}

\begin{ex}%[2D4N2-2]%[To 20 - Dot 17 - Chuong 4 - Bai 3 - CD - De 3]%[Đình Nguyên]
	Tính $\displaystyle\int\limits_{3}^{5}\left(x^{2}+2 x\right) \mathrm{\,d}x$.
	\choice
	{\True $\dfrac{146}{3}$}
	{$146$}
	{$3$}
	{$\dfrac{143}{6}$}
	\loigiai
	{
		Ta có
		\begin{align*}
			\int\limits_{3}^{5}\left(x^{2}+2 x\right) \mathrm{\,d}x & =\int\limits_{3}^{5} x^{2} \mathrm{\,d}x+2 \int\limits_{3}^{5} x \mathrm{\,d}x             \\
			                                                        & =\left.\dfrac{1}{3} x^{3}\right|_{3} ^{5}+\left.2 \cdot \dfrac{1}{2} x^{2}\right|_{3} ^{5} \\
			                                                        & =\left.\dfrac{1}{3} x^{3}\right|_{3} ^{5}+\left.x^{2}\right|_{3} ^{5}                      \\
			                                                        & =\dfrac{1}{3}\left(5^{3}-3^{3}\right)+\left(5^{2}-3^{2}\right)=\dfrac{146}{3}.
		\end{align*}
	}
\end{ex}

\begin{ex}%[2D4N2-3]
	Tích phân $I=\displaystyle\int\limits_{\frac{\pi}{4}}^{\frac{\pi}{3}} \dfrac{\mathrm{\,d} x}{\sin ^2 x}$ bằng
	\choice
	{$\cot \dfrac{\pi}{3}-\cot \dfrac{\pi}{4}$}
	{$\cot \dfrac{\pi}{3}+\cot \dfrac{\pi}{4}$}
	{\True $-\cot \dfrac{\pi}{3}+\cot \dfrac{\pi}{4}$}
	{$-\cot \dfrac{\pi}{3}-\cot \dfrac{\pi}{4}$}
	\loigiai{Ta có
	$I=\displaystyle\int\limits_{\frac{\pi}{4}}^{\frac{\pi}{3}} \dfrac{\mathrm{\,d} x}{\sin ^2 x}=-\left.\cot x\right|_{\frac{\pi}{4}} ^{\frac{\pi}{3}}=-\cot \dfrac{\pi}{3}+\cot \dfrac{\pi}{4}$.
	}
\end{ex}

\begin{ex}%[Tổ 20 - Chương 4 - - CD]%[Nguyễn Văn Sang]%[2D4N2-4]
	Tính $I=\displaystyle\int\limits_0^1\left(2 \cdot  \mathrm{e}^{-x}+4^x\right) \mathrm{\,d}x$
	\choice
	{\True $I=\dfrac{-2}{\mathrm{e}}+2+\dfrac{3}{\ln 4}$ }
	{$I=\dfrac{2}{\mathrm{e}}+2+\dfrac{3}{\ln 4}$ }
	{$I=\dfrac{-2}{\mathrm{e}}+2-\dfrac{3}{\ln 4}$ }
	{$I=\dfrac{-2}{\mathrm{e}}+2+\dfrac{4}{\ln 4}$ }
	\loigiai{
		\[I=\displaystyle\int\limits_0^1\left(2 \cdot \mathrm{e}^{-x}+4^x\right) \mathrm{\,d} x=\left(-2 \mathrm{e}^{-x}+\dfrac{4^x}{\ln 4}\right)\bigg|_0 ^1=\dfrac{-2}{\mathrm{e}}+\dfrac{4}{\ln 4}+2-\dfrac{1}{\ln 4}=\dfrac{-2}{\mathrm{e}}+2+\dfrac{3}{\ln 4}.\]
	}
\end{ex}
\Closesolutionfile{ans}

\TNTF
\Opensolutionfile{ans}[ans/ansDe2-TN2]
\begin{ex}%[BG-12-4in1, Phạm Đức]%[2D4N1-1]
	Cho hàm số $f(x)$ có một nguyên hàm là $F(x)$, $k$ là số thực bất kỳ. Mỗi khẳng định sau đây đúng hay sai?
	\choiceTF[t]
	{\True $\displaystyle\int f(x)\mathrm{\,d}x=F(x)+C$ với $C$ là hằng số}
	{$f'(x)=F(x)$}
	{$\displaystyle\int kf(x)\mathrm{\,d}x=k\int f(x)\mathrm{\,d}x$}
	{\True $\displaystyle\int \left[f(x)+F'(x)\right]\mathrm{\,d}x=2\left[F(x)+2025\right]+C$ với $C$ là hằng số}
	\loigiai{
		\begin{itemchoice}
			\itemch Theo định nghĩa nguyên hàm ta có $\displaystyle\int f(x)\mathrm{\,d}x=F(x)+C$ với $C$ là hằng số.
			\itemch Tổng quát ta chỉ có kết quả $F'(x)=f(x)$.
			\itemch Tính chất $\displaystyle\int kf(x)\mathrm{\,d}x=k\int f(x)\mathrm{\,d}x$ chỉ đúng với $k\ne 0$.
			\itemch Ta có $\left(2\left[F(x)+2025\right]+C\right)'=2F'(x)=2f(x)=f(x)+f(x)=f(x)+F'(x)$.

			Vậy $\displaystyle\int \left[f(x)+F'(x)\right]\mathrm{\,d}x=2\left[F(x)+2025\right]+C$ với $C$ là hằng số.
		\end{itemchoice}
	}
\end{ex}

\begin{ex}%[2D4H2-1]
	Cho $\displaystyle\int\limits_0^{\frac{\pi}{2}} \left[\sin x +f(x)\right] \mathrm{\, d}x=6$. Xét tính đúng, sai cho mỗi khẳng định sau.
	\choiceTF[t]
	{\True $\displaystyle\int\limits_0^{\tfrac{\pi}{2}} \left[\sin x +f(x)\right] \mathrm{\, d}x= \displaystyle\int\limits_0^{\tfrac{\pi}{2}} \sin x \mathrm{\, d}x + \displaystyle\int\limits_0^{\tfrac{\pi}{2}} f(x) \mathrm{\, d}x$}
	{\True  $\displaystyle\int\limits_0^{\tfrac{\pi}{2}} \sin x\mathrm{\,d}x=1$}
	{\True $\displaystyle\int\limits_0^{\tfrac{\pi}{2}} f(x)\mathrm{\, d}x=5$}
	{$\displaystyle\int\limits_0^{\tfrac{\pi}{2}}\left[3+2f(x)\right]\mathrm{\, d}x=\dfrac{3\pi}{2}+5$}
	\loigiai{
		\begin{itemchoice}
			\itemch  Ta có $\displaystyle\int\limits_0^{\tfrac{\pi}{2}} \left[\sin x +f(x)\right] \mathrm{\, d}x= \displaystyle\int\limits_0^{\tfrac{\pi}{2}} \sin x \mathrm{\, d}x + \displaystyle\int\limits_0^{\tfrac{\pi}{2}} f(x) \mathrm{\, d}x$.
			\itemch  Ta có $\displaystyle\int\limits_0^{\tfrac{\pi}{2}} \sin x\mathrm{\,d}x=-\cos x\bigg|_0^{\tfrac{\pi}{2}}=-\cos \dfrac{\pi}{2}+\cos 0=1$.
			\itemch  Ta có $\displaystyle\int\limits_0^{\tfrac{\pi}{2}} \left[\sin x +f(x)\right] \mathrm{\, d}x= \displaystyle\int\limits_0^{\tfrac{\pi}{2}} \sin x \mathrm{\, d}x + \displaystyle\int\limits_0^{\tfrac{\pi}{2}} f(x) \mathrm{\, d}x= 1 + \displaystyle\int\limits_0^{\tfrac{\pi}{2}} f(x) \mathrm{\, d}x=6$.\\
			Suy ra $\displaystyle\int\limits_0^{\tfrac{\pi}{2}} f(x)\mathrm{\, d}x=6-1=5$.
			\itemch  Ta có $\displaystyle\int\limits_0^{\tfrac{\pi}{2}}\left[3+2f(x)\right]\mathrm{\, d}x=\displaystyle\int\limits_0^{\tfrac{\pi}{2}}3\mathrm{\, d}x+2\displaystyle\int\limits_0^{\tfrac{\pi}{2}} f(x)\mathrm{\, d}x=3x\bigg|_0^{\tfrac{\pi}{2}}+2\cdot5 =\dfrac{3\pi}{2}+10\neq\dfrac{3\pi}{2}+5$.
		\end{itemchoice}
	}
\end{ex}
\Closesolutionfile{ans}

\TNSA
\Opensolutionfile{ans}[ans/ansDe2-TN3]
\begin{ex}%[2D4H1-1]%[Đào Trung Kiên]
	Biết $F(x)$ là một nguyên hàm của hàm số $f(x)=\sin x$ và đồ thị hàm số $y=F(x)$ đi qua điểm $M\left(0;1\right)$. Tính $F\left(\dfrac{\pi}{2}\right)$ (làm tròn kết quả tới hàng đơn vị).
	\shortans[]{$2$}
	\loigiai{
		Ta có $F(x)=\displaystyle\int f(x)\mathrm{\,d}x=-\cos x+C$.\\
		Mà đồ thị hàm số $y=F(x)$ đi qua $M(0;1)$ nên $F(0)=1\Leftrightarrow -1+C=1\Leftrightarrow C=2$.\\
		Suy ra $F(x)=-\cos x+2$ nên $F\left(\dfrac{\pi}{2}\right)=2$.}
\end{ex}

\begin{ex}%[2D4H2-2]%[Tổ 20 - Đợt 17 - Chương 4 - - CD]%[Phạm Hà Giang]
	Tích phân $\displaystyle\int\limits_2^8 \dfrac{1}{x} \mathrm{\,d}x=a\ln 2$. Giá trị của $a$ bằng bao nhiêu?
	\shortans{$2$}
	\loigiai
	{ $\displaystyle\int\limits_2^8 \dfrac{1}{x} \mathrm{\,d}x =\left. \ln \left| x \right| \right| _2^8=\ln 8 -\ln 2 =2\ln 2.$\\
		Suy ra $a=2$.
	}
\end{ex}

\begin{ex}%[2D4H3-1]
	\immini{Cho hàm số $f(x)$ liên tục trên $\mathbb{R}$. Đồ thị hàm số $y=f'(x)$ được cho như hình bên. Diện tích các hình phẳng $(K)$, $(H)$ lần lượt là $\dfrac{5}{12}$ và $\dfrac{8}{3}$. Biết $f(-1)=\dfrac{19}{12}$, tính $f(2)$(kết quả làm tròn đến hàng phần mười).
	}
	{
		\begin{tikzpicture}[scale=1, font=\footnotesize, line join=round, line cap=round, >=stealth]
			\tikzset{label style/.style={font=\footnotesize}}
			%Nhập giới hạn đồ thị và hàm số cần vẽ
			\def \xmin{-1.5}
			\def \xmax{2.5}
			\def \ymin{-2.6}
			\def \ymax{1}
			\def \hamso{sin((2*(\x))*180/pi)}
			%\def \tiemcanxien{\x+1}
			%Tự động
			\draw[->] (\xmin,0)--(\xmax,0) node[below left] {$x$};
			\draw[->] (0,\ymin)--(0,\ymax) node[below left] {$y$};
			\draw[fill=black] (0,0) circle(1pt) node [below left] {$O$};
			%Vẽ các điểm trên 2 hệ trục
			\foreach \x in {-1,2}
			\fill[black] (\x,0) circle(1pt) node [above left] {$\x$};
			%Tự động
			\begin{scope}
				\clip (\xmin+0.01,\ymin+0.01) rectangle (\xmax-0.01,\ymax-0.01);
				\draw[samples=350,domain=\xmin+0.01:\xmax-0.01,smooth,variable=\x] plot (\x,{(\x+1)*(\x)*(\x-2)});
			\end{scope}
			\draw[pattern = north east lines] (-1,0)--plot[domain=-1:2] (\x,{(\x+1)*(\x)*(\x-2)});
			\draw (-.5,-.5)node{$(K)$};
			\draw (1,.5)node{$(H)$};
		\end{tikzpicture}
	}
	\shortans[]{$-0{,}7$}
	\loigiai{
	Ta có $S_K=\displaystyle \int\limits_{-1}^0 f'(x)\mathrm{\,d}x
		=f(0)-f(-1)	=\dfrac{5}{12}
		\Rightarrow f(0)=\dfrac{5}{12}+f(-1)=\dfrac{5}{12}+\dfrac{19}{12}=2$.\\
	Lại có: $S_H=\displaystyle \int\limits_0^2 \left| f'(x) \right|\mathrm{\,d}x
		=-\displaystyle \int\limits_0^2 f'(x)\mathrm{\,d}x
		=f(0)-f(2)=\dfrac{8}{3}
		\Rightarrow f(2)=f(0)-\dfrac{8}{3}=2-\dfrac{8}{3}=\dfrac{-2}{3}\approx -0{,}7$.
	}
\end{ex}

\begin{ex} %[2D4H3-3]
	Cho hình phẳng $(D)$ được giới hạn bởi các đường $ y={x^2}+2x+2$; $y=6-x$; $y=2$ và $(D)$ nằm ngoài Parabol $y={x^2}+2x+2$. Khi cho $(D)$ quay quanh trục $Ox$, ta nhận được vật thể tròn xoay có thể tích $V=\dfrac{a\pi }{b}$, trong đó $a$, $b$ là các số nguyên dương. Giá trị biểu thức $P=a-2{b^2}$ bằng bao nhiêu.
	\shortans[1]{$73$}
	\loigiai{
	Vẽ các đường $ y={x^2}+2x+2$; $y=6-x$; $y=2$.
	\begin{center}
		\begin{tikzpicture}[scale=0.7,line join=round,line cap=round,font=\footnotesize,>=stealth]
			\def\f(#1){1*((#1)^2)+2*(#1)+2}
			\def\g(#1){6-1*(#1)}
			\draw[->] (-3,0)--(7,0) node[below] { $x$};
			\draw[->] (0,-1)--(0,7) node[left] { $y$};
			\draw (0,0) node [below left] { $O$};
			\foreach \x in {1,4}		\draw (\x,0.1)--(\x,-0.1) node [below] { $\x$};
			\foreach \y in {5}		\draw (0.1,\y)--(-0.1,\y) node [left] { $\y$};
			\node[left] at (0,2.2) {$2$};
			\clip (-4,-1) rectangle (6,6);
			\draw[thick,smooth,samples=200] plot[domain=-3:3] (\x,{\f(\x)});
			\draw[thick,smooth,samples=200] plot[domain=-2:7] (\x,{\g(\x)});
			\draw[thick,smooth,samples=200] plot[domain=-3:3] (-3,2)--(6,2);
			\draw[dashed] (4,0)--(4,2) (1,0)-- (1,5)--(0,5);
			\fill[pattern=north east lines,pattern color=red]plot[domain=0:2](\x,{(\x)^2+2*(\x)+2})-- (1,5)--(1,2) -- cycle;
			\fill[pattern=north east lines,pattern color=black] (1,2)--(1,5)--plot[domain=0:2](\x,{6-1*(\x)})-- (4,2) -- cycle;
		\end{tikzpicture}
	\end{center}
	Dựa vào hình vẽ ta có\\
	$V=\pi \int\limits_0^1{| {{( {x^2}+2x+2 )}^2}-{2^2} |\mathrm{\,d}x}+\pi \int\limits_1^4{| {{( 6-x )}^2}-{2^2} |\mathrm{\,d}x}=\dfrac{523\pi }{15}\\
		\Rightarrow a=523,b=15$.\\
	Vậy $P=523-{{2\cdot 15}^2}=73$.
	}
\end{ex}

\Closesolutionfile{ans}

\TL
\begin{ex}%[2D4H2-2] 
	Tính tích phân $I=\displaystyle\int\limits_{-1}^2{\left|x^3-3x+2\right|}\mathrm{\,d}x$.
	\loigiai{
	$C=\displaystyle\int\limits_{-1}^2 {\left|x^3-3x+2\right|}\mathrm{\,d}x$.\\
	Xét $f(x) = x^3-3x-2$ trên $[-1; 2]$.\\
	Cho $f(x) = 0
		\Leftrightarrow
		x^3-3x-2 = 0
		\Leftrightarrow
		\hoac{& x=2\in [-1; 2] \\ & x=-1\in [-1; 2].}$\\
	Bảng xét dấu $f(x)$ trên đoạn $[-1; 2]$.
	\begin{center}
		\begin{center}
			\begin{tikzpicture}
				\tkzTabInit[nocadre=false,lgt=3,espcl=2.5,deltacl=0]
				{$x$ /0.6,$x^3-3x+2$ /0.6}
				{,$-1$,$2$,}
				\tkzTabLine{,h,$0$,-,$0$,h,}
			\end{tikzpicture}
		\end{center}
	\end{center}
	Do đó
	$C = -\displaystyle\int\limits_{-1}^2 {(x^3-3x-2)}\mathrm{\,d}x	=
		-\left.\left(\dfrac{1}{4} x^4-\dfrac{3}{2} x^2 -2x\right)\right|_{-1}^2=
		6- \left( -\dfrac{3}{4} \right)=\dfrac{27}{4}$.
	}
\end{ex}


\begin{ex}%[2D4V1-4]
	Cho hàm số $f(x)$ thỏa mãn $ f(0)=1-\ln 2$ và $\mathrm{e}^ xf'(x)=2^x\left[f(x)\right]^2$ với mọi $x\in\mathbb{R}$. Giá trị của $f(1)$ bằng bao nhiêu? (\textit{Kết quả làm tròn đến hàng phần trăm}).
	% \shortans{$0{,}42$}
	\loigiai{
		Từ giả thiết ta có $f'(x)=\dfrac{2^x}{\mathrm{e}^ x}{\left[f(x)\right]^2}$ với mọi $ x\in\left(1;2\right]$.\\
		Do đó $ f(x)\ge f(1)=1>0$ với mọi $ x\in\left[1;2\right]$.\\
		Xét với mọi $ x\in [1 ; 2]$ ta có
		\begin{align*}
			\mathrm{e}^ x{f}'(x)=2^x{\left[f(x)\right]^2} & \Rightarrow{f}'(x)=\dfrac{2^x}{\mathrm{e}^ x}{\left[f(x)\right]^2} \\&\Rightarrow\dfrac{f'(x)}{\left[f(x)\right]^2}=\left(\dfrac{2}{\mathrm{e}}\right)^x\\&\Rightarrow-\left(\dfrac{1}{f(x)}\right)'=\left(\dfrac{2}{\mathrm{e}}\right)^x \\&\Rightarrow{\left(\dfrac{1}{f(x)}\right)'}=-\left(\dfrac{2}{\mathrm{e}}\right)^x\\&\Rightarrow\dfrac{1}{f(x)}=-\displaystyle\int\left(\dfrac{2}{\mathrm{e}}\right)^x\mathrm{\,d} x\\&\Rightarrow\dfrac{1}{f(x)}=-\dfrac{\left(\dfrac{2}{\mathrm{e}}\right)^x}{\ln \dfrac{2}{\mathrm{e}}}+C\\&\Rightarrow\dfrac{1}{f(x)}=\dfrac{\left(\dfrac{2}{\mathrm{e}}\right)^x}{1-\ln 2}+C.
		\end{align*}
		Mà $ f(0)=1-\ln 2\Rightarrow C=0$. \\Do đó
		$\dfrac{1}{f(x)}=\dfrac{\left(\dfrac{2}{\mathrm{e}}\right)^x}{1-\ln 2}$
		$\Rightarrow f(x)=\dfrac{1-\ln 2}{\left(\dfrac{2}{\mathrm{e}}\right)^x}=\dfrac{(1-\ln 2)\mathrm{e}^x}{2^x}$.\\
		Vậy $ f(1)=\dfrac{\mathrm{e}-\mathrm{e}\ln 2}{2}\approx 0{,}42$.}
\end{ex}

\begin{ex}%[12-MH-2-MH2025]%[MH-2025, Nguyễn Trần Phong]%[2D4C3-2]
	\immini{Chướng ngại vật \lq\lq  tường cong\rq\rq trong một sân thi đấu X-Game là một khối bê tông có chiều cao từ mặt đất lên là $3$ m. Giao của mặt tường cong và mặt đất là đoạn thẳng $AB = 2$ m. Thiết diện của khối tường cong cắt bởi mặt phẳng vuông góc với $AB$ tại $A$ là một hình tam giác vuông cong $ACE$ với $AC = 4$ m, $CE = 3$ m và cạnh cong $AE$ nằm trên một đường Parabol có trục đối xứng vuông góc với mặt đất. Tại vị trí $M$ là trung điểm của $AC$ thì tường cong có độ cao $1$ m. Thể tích bê tông cần sử dụng để tạo nên khối tường cong đó gần nhất với số nào dưới đây?
		% \shortans{$9{,}3$}
	}{\begin{tikzpicture}[>=stealth,x=0.8cm,y=0.8cm,scale=0.7]
			\coordinate[label=below:$A$] (A) at (0,0);
			\coordinate[label=left:$B$] (B) at (-2,2);
			\coordinate[label=below:$C$] (C) at (6,0);
			\coordinate[label=right:$E$] (E) at (6,6);
			\coordinate (G) at (2,4);
			\coordinate (H) at (4,1.8);
			\coordinate (D) at ($(C)+(B)-(A)$);
			\coordinate (F) at ($(E)+(D)-(C)$);
			\coordinate[label=below:$M$] (M) at ($(A)!0.5!(C)$);
			\coordinate (K) at ($(A)!0.5!(B)$);
			\coordinate (N) at ($(M)+(0,1.1)$);
			\draw (D)--(C)--(A)--(B) (C)--(E)--(F) (M)--(N);
			\draw[dashed] (B)--(D)--(F);
			\foreach \diem in {A,B,C,D,E,F,M,F}	\fill (\diem)circle(1.5pt);
			%\tkzLabelPoints[above left](D)
			%\tkzLabelSegment[right](M,N){\footnotesize$1$ m}
			%\tkzLabelSegment[left](A,B){\footnotesize$2$ m}
			%\tkzLabelSegment[right](C,E){\footnotesize$3{,}5$ m}
			\draw(-1,.8) node[left]{\footnotesize $2$ m} (3,0.8) node[right]{\footnotesize$1$ m} (6,3) node[right]{\footnotesize$3$ m};

			\draw plot[smooth,tension=.65] coordinates{(B) (G) (F)};
			\draw plot[smooth,tension=.65] coordinates{(A) (H) (E)};
			\fill [pattern = north east lines] plot[smooth,tension=.65] coordinates{(A) (H) (E)} (0,0) --(-2,2)--(4,8)--(6,6)--cycle;
			\fill [draw=none, pattern = north east lines, color=white] (0,0) plot[smooth,tension=.65] coordinates{(B) (G) (F)} (-2,2)--(4,2)--cycle;
		\end{tikzpicture}
	}
	\loigiai{
		\immini{Chọn hệ trục tọa độ như hình vẽ.\\
			Gọi $AE \colon y = ax^2 + bx + c$.\\
			Do $AE$ đi qua $A(-4; 0)$ nên ta có $16a - 4b + c = 0$.\\
			Do $E (0; 3)$ thuộc cạnh cong $AE$ nên $c = 3$ (2).\\
			Do $N(-2; 1)$ thuộc cạnh cong $AE$ nên $4a - 2b + c = 1$ (3).\\
			Từ (1), (2), (3) suy ra $a = \dfrac{1}{8}$, $b = \dfrac{5}{4}$, $c = 3 \Rightarrow AE \colon y = \dfrac{1}{8}x^2 + \dfrac{5}{4}x + 3$.\\
			Khi đó $S_{AEC} = \displaystyle\int_{-4}^0\left(\dfrac{1}{8} x^2 + \dfrac{5}{4}x + 3\right) dx = \dfrac{14}{3}\left(m^2\right)$.
		}{\begin{tikzpicture}[>=stealth,x=0.8cm,y=0.8cm,scale=0.7]
				\coordinate[label=below:$A$] (A) at (0,0);
				\coordinate[label=left:$B$] (B) at (-2,2);
				\coordinate[label=below:$C$] (C) at (6,0);
				\coordinate[label=right:$E$] (E) at (6,6);
				\coordinate (G) at (2,4);
				\coordinate (H) at (4,1.8);
				\coordinate[label = above left:$D$] (D) at ($(C)+(B)-(A)$);
				\coordinate[label = above:$F$] (F) at ($(E)+(D)-(C)$);
				\coordinate[label=below:$M$] (M) at ($(A)!0.5!(C)$);
				\coordinate (K) at ($(A)!0.5!(B)$);
				\coordinate[label=above left:$N$] (N) at ($(M)+(0,1.1)$);
				\draw (D)--(C)--(A)--(B) (C)--(E)--(F) (M)--(N);
				\draw[dashed] (B)--(D) (D)--(F);
				\foreach \diem in {A,B,C,D,E,F,M,F,N}	\fill (\diem)circle(1.5pt);
				\coordinate (x) at ($(M)!1.5!(C)$);
				\draw[->](C)--(x); \draw (x) node[right]{$x$};
				\coordinate (y) at ($(C)!1.4!(E)$);
				\draw[->](E)--(y); \draw (y) node[right]{$y$};
				%\tkzLabelPoints[above left](D)
				%\tkzLabelSegment[right](M,N){\footnotesize$1$ m}
				%\tkzLabelSegment[left](A,B){\footnotesize$2$ m}
				%\tkzLabelSegment[right](C,E){\footnotesize$3{,}5$ m}
				\draw plot[smooth,tension=.65] coordinates{(B) (G) (F)};
				\draw plot[smooth,tension=.65] coordinates{(A) (H) (E)};
				\fill [pattern = north east lines] plot[smooth,tension=.65] coordinates{(A) (H) (E)} (0,0) --(-2,2)--(4,8)--(6,6)--cycle;
				\fill [draw=none, pattern = north east lines, color=white] (0,0) plot[smooth,tension=.65] coordinates{(B) (G) (F)} (-2,2)--(4,2)--cycle;
			\end{tikzpicture}
		}
		\noindent Thể tích khối tường cong là $V = S_{AEC} \cdot AB = \frac{14}{3} \cdot 2 = \dfrac{28}{3} = 9{,}3\left(\mathrm{~m}^3\right)$.	}
\end{ex}

% \Closesolutionfile{ansbook}
% \HetDe
% \label{De2}
% %
% \cleardoublepage
% \setcounter{page}{1}
% \rfoot{Trang \thepage/\pageref{DA2} - Đáp án trắc nghiệm Mã đề 2}
% \begin{center}
% 	\bfseries ĐÁP ÁN TRẮC NGHIỆM MÃ ĐỀ 2
% \end{center}

% \inputansbox{10}{ans/ansDe2-TN1}
% \inputansbox[3]{2}{ans/ansDe2-TN2}
% \inputansbox{3}{ans/ansDe2-TN3}
% \label{DA2}
% %

% \begin{name}
	{\tenchude}
	{TOÁN 12}
	{LỚP TOÁN THẦY PHÁT}
	{Thời gian: 90 phút - Không kể thời gian phát đề}
\end{name}
\TN
\Opensolutionfile{ans}[ans/ansDe3-TN1]
\begin{ex}%[EX-Ôn Tập TN 2025,  Đỗ Vũ Minh Thắng]%[2D4N1-1]
	Hàm số nào sau đây là một nguyên hàm của hàm số $f(x) = \sin x$?
	\choice
	{$F_{1}(x) = \sin x$}
	{$F_{2}(x) = -\sin x$}
	{$F_{3}(x) = \cos x$}
	{\True $F_{4}(x) = -\cos x$}
	\loigiai{
		Vì $(-\cos x)'=\sin x$ nên $F_{4}(x)$ là một nguyên hàm của hàm số $f(x)$.
	}
\end{ex}

\begin{ex}%[2D4N1-1]%[To 20 - Dot 17 - Chuong 4 - Bai 3 - CD - De 1 - TN]%[Nguyễn Hữu Duy]
	Nếu $F(x)$ là nguyên hàm của hàm số $f(x)$, thì tích phân của $f(x)$ trên đoạn $[a;b]$ được tính như thế nào?
	\choice
	{\True $F(b)-F(a)$}
	{$F(a)-F(b)$}
	{$\dfrac{F(b)}{F(a)}$}
	{$\dfrac{F(a)}{F(b)}$}
	\loigiai
	{ Ta có
		$\displaystyle\int_{a}^{b} f(x) \mathrm{\,d}x = F(b) - F(a)$.
	}
\end{ex}

\begin{ex}%[2D4N1-2]
	Nguyên hàm của hàm số $f(x)=x^4+x^2$ là
	\choice
	{\True $\dfrac{1}{5}x^5+\dfrac{1}{3}x^3+C$}
	{$x^4+x^2+C$}
	{$x^5+x^3+C$}
	{$4x^3+2x+C$}
	\loigiai{
		$\displaystyle\int{f(x)\mathrm{\,d}x}=\displaystyle\int{(x^4+x^2)\mathrm{\,d}x}$ $=\dfrac{1}{5}x^5+\dfrac{1}{3}x^3+C$.}
\end{ex}

\begin{ex}%[2D4N1-3]%[Dự án EX-TF-TLN lần 3 -Mui Doan]
	Họ tất cả các nguyên hàm của hàm số $f(x)=\sin x$ là
	\choice
	{$-\sin x+C$}
	{$\cos x+C$}
	{$\dfrac{1}{2}\sin^2x+C$}
	{\True $-\cos x+C$}
	\loigiai{
		Ta có $\displaystyle\int f(x)\mathrm{\,d}x=\displaystyle\int\sin x\mathrm{\,d}x=-\cos x+C$.}
\end{ex}

\begin{ex}%[2D4N1-4]
	Hàm số nào dưới đây là một nguyên hàm của hàm số $f(x)=\sqrt{x}-1$ trên $(0;+\infty)$?
	\choice
	{$F(x)=\dfrac{1}{2\sqrt{x}}$}
	{$F(x)=\dfrac{1}{2\sqrt{x}}-x$}
	{$F(x)=\dfrac{2}{3}\sqrt[3]{x^2}-x+1$}
	{\True $F(x)=\dfrac{2}{3}\sqrt{x^3}-x+2$}
	\loigiai{
		Ta có : $\displaystyle\int (\sqrt{x}-1)\mathrm{d}x=\frac{2}{3}\sqrt{x^3}-x+C$.}
\end{ex}

\begin{ex}%[Mức độ 1]%[BG-12-New-4in1, Phạm Đức Thiệu]%[2D4N1-4]
	Tìm họ nguyên hàm của hàm số $f(x)=2^x$.
	\choice
	{$\displaystyle\int f(x)\mathrm{\,d}x=2^x+C$}
	{\True $\displaystyle\int f(x)\mathrm{\,d}x=\dfrac{2^x}{\ln 2}+C$}
	{$\displaystyle\int f(x)\mathrm{\,d}x=2^x\ln 2+C$}
	{$\displaystyle\int f(x)\mathrm{\,d}x=\dfrac{2^{x+1}}{x+1}+C$}
	\loigiai{
		Ta có $\displaystyle\int f(x)\mathrm{\,d}x=\displaystyle\int 2^x\mathrm{\,d}x=\dfrac{2^x}{\ln 2}+C$.
	}
\end{ex}

\begin{ex}%[2D4H1-2]
	Khẳng định nào sau đây \textbf{đúng}?
	\choice
	{\True $\displaystyle\int{\left( x-\dfrac{1}{x} \right)^2\mathrm{d}x} =\dfrac{x^3}{3}-2x-\dfrac{1}{x}+C$ }
	{ $\displaystyle\int{\left( x-\dfrac{1}{x} \right)^2\mathrm{d}x}=\dfrac{x^3}{3}-2x+\dfrac{1}{x}+C$ }
	{ $\displaystyle\int{\left( x-\dfrac{1}{x} \right)^2\mathrm{d}x} =\dfrac{1}{3}\left( x-\dfrac{1}{x} \right)^3+C$ }
	{ $\displaystyle\int{\left( x-\dfrac{1}{x} \right)^2\mathrm{d}x}=\dfrac{1}{3}\left( x-\dfrac{1}{x} \right)^3\left( 1+\dfrac{1}{x^2} \right)+C$ }
	\loigiai{
		Ta có $\displaystyle\int{\left( x-\dfrac{1}{x} \right)^2\mathrm{d}x}=\int{\left( x^2-2+\dfrac{1}{x^2} \right)\mathrm{d}x}=\dfrac{x^3}{3}-2x-\dfrac{1}{x}+C$.
	}
\end{ex}

\begin{ex}%[2D4H1-3]
	Tìm nguyên hàm của hàm số $f(x)=\sin x +6x$ là
	\choice
	{\True $-\cos x+3x^{2}+C$}
	{$\cos x +6x^{2}+C$}
	{$\cos x +C$}
	{$\cos x +3x^{2}+C$}
	\loigiai{ Ta có $\displaystyle\int\limits (\sin x +6x)\mathrm{\,d}x=-\cos x+3x^{2}+C$.
	}
\end{ex}

\begin{ex}%[Dự án 2025 - đề cấu trúc mới, Nguyễn Kiều Nhã Tú]%[2D4N2-1]
	Cho hàm số $y=f(x)$ liên tục trên đoạn $[a;b]$. Mệnh đề nào dưới đây là \textbf{sai}?
	\choice
	{$\displaystyle\int\limits_a^b f(x)\mathrm{\,d} x=-\displaystyle\int\limits_b^a f(x)\mathrm{\,d}x$}
	{\True $\displaystyle\int\limits_a^b f(x)\mathrm{\,d} x=\displaystyle\int\limits_a^c f(x)\mathrm{\,d} x+\displaystyle\int\limits_c^b f(x)\mathrm{\,d}x$, $\forall c \in \mathbb{R}$}
	{$\displaystyle\int\limits_a^b f(x)\mathrm{\,d} x=\displaystyle\int\limits_a^b f(t)\mathrm{\,d}t$}
	{$\displaystyle\int\limits_a^a f(x)\mathrm{\,d}x=0$}
	\loigiai{
		Theo tính chất của tích phân,  $\displaystyle\int\limits_a^b f(x) \mathrm{\,d}x=\displaystyle\int\limits_a^c f(x)\mathrm{\,d} x+\displaystyle\int\limits_c^b f(x) \mathrm{\,d}x$, $\forall c\in \mathbb{R}$ chỉ đúng khi $c\in[a;b]$.
	}
\end{ex}

\begin{ex}%[2D4N2-2]%[To 20 - Dot 17 - Chuong 4 - Bai 3 - CD - De 3]%[Đình Nguyên]
	Tích phân $\displaystyle\int\limits_{0}^{1}\left(2 x^{2}-1\right) \mathrm{\,d}x$ có giá trị bằng
	\choice
	{$1$}
	{$2$}
	{$\dfrac{1}{3}$}
	{\True $\dfrac{-1}{3}$}
	\loigiai
	{
	Ta có $ \displaystyle
		\int\limits_{0}^{1}\left(2 x^{2}-1\right) \mathrm{\,d}x=\left.\left(\dfrac{2 x^{3}}{3}-x\right)\right|_{0} ^{1}=\dfrac{-1}{3}
	$.
	}
\end{ex}

\begin{ex}%[2D4N2-3]
	Giá trị của $\displaystyle\int\limits_0^{\frac{\pi}{2}}{\sin x\mathrm{\,d}x}$ bằng
	\choice
	{0}
	{\True 1}
	{$-1$}
	{$\dfrac{\pi}{2}$}
	\loigiai{
	Tính được $\displaystyle\int\limits_0^{\frac{\pi}{2}}{\sin x\mathrm{\,d}x}=-\cos x\Big|_0^{\frac{\pi}{2}}=1$.}
\end{ex}

\begin{ex}%[Tổ 20 - Chương 4 - - CD]%[Nguyễn Văn Sang]%[2D4N2-4]
	Biết $I=\displaystyle\int\limits_0^1 3^x \cdot 7^{x+1} \cdot \mathrm{\,d} x=\dfrac{a}{\ln b+\ln c}$, trong đó $a, b, c \in \mathbb{Z}$ và $b, c$ là số nguyên tố.  Khi đó $a+b+c$ bằng
	\choice
	{\True $150$}
	{$147$}
	{$157$}
	{$140$}
	\loigiai{
		Ta có \[
			I=\displaystyle\int\limits_0^1 3^x \cdot 7^{x+1} \cdot \mathrm{\,d} x=7 \cdot \displaystyle\int\limits_0^1 21^x \cdot \mathrm{\,d} x=7 \cdot \dfrac{21^x}{\ln 21}\bigg|_0 ^1=\dfrac{140}{\ln 21}=\dfrac{140}{\ln 3+\ln 7}.
		\]
		Suy ra $a=140$, $b=3$, $c=7$ và $a+b+c=150$.
	}
\end{ex}
\Closesolutionfile{ans}

\TNTF
\Opensolutionfile{ans}[ans/ansDe3-TN2]
\begin{ex}%[2D4H1-1]%[Đào Trung Kiên]
	Xét hàm số $f(x)=\left(ax+b\right)^n$ với $a\neq 0$, $n \in \mathbb{R}\setminus \{0,1\}$ thì
	\choiceTF{\True $\displaystyle\int f(x)\mathrm{\,d}x=\dfrac{1}{a(n+1)}(ax+b)^{n+1}+C$}
	{$f'(x)=\dfrac{1}{a(n+1)}(ax+b)^{n+1}+C$}
	{\True Nếu $F(x)$ là một nguyên hàm của $f(x)$ và thỏa mãn $F\left(-\dfrac{b}{a}\right)=0$ thì $F(0)=\dfrac{b^{n+1}}{a(n+1)}$}
	{$\displaystyle\int f(x)\mathrm{\,d}x=\dfrac{1}{(n+1)}(ax+b)^{n+1}+C$}
	\loigiai{Với $f(x)=\left(ax+b\right)^n$ với $a\neq 0$, $n \in \mathbb{R}\setminus \{0,1\}$ thì ta có
		\begin{itemchoice}
			\itemch $\displaystyle\int f(x)\mathrm{\,d}x=\dfrac{1}{a(n+1)}(ax+b)^{n+1}+C$.
			\itemch Ta có $f'(x)=n(ax+b)^{n-1}$ nên khẳng định trong đề bài là sai.
			\itemch Nếu $F(x)$ là một nguyên hàm của $f(x)$ suy ra $F(x)=\dfrac{1}{a(n+1)}(ax+b)^{n+1}+C$ và thỏa mãn $F\left(-\dfrac{b}{a}\right)=0$ thì $C=0$ nên $F(0)=\dfrac{b^{n+1}}{a(n+1)}$.
			\itemch $\displaystyle\int f(x)\mathrm{\,d}x=\dfrac{1}{a(n+1)}(ax+b)^{n+1}+C$ là đúng nên $\displaystyle\int f(x)\mathrm{\,d}x=\dfrac{1}{(n+1)}(ax+b)^{n+1}+C$ là sai.
		\end{itemchoice}

	}
\end{ex}

\begin{ex}%[2D4H2-2]%[Tổ 20 - Đợt 17 - Chương 4 - - CD - Đề 7]%[Bình]
	Xét tính đúng sai của các mệnh đề sau:
	\choiceTF
	{\True $\displaystyle\int\limits_1^2 \dfrac{1}{x}\mathrm{\,d}x=\ln 2$}
	{\True $\displaystyle\int\limits_0^a \left(\dfrac{1}{\cos^2 x}+3\right)\mathrm{\,d}x=\tan a+3a$}
	{$\displaystyle\int\limits_m^n \dfrac{1}{x\sqrt{x}}\mathrm{\,d}x=2\left(\sqrt{m}-\sqrt{n}\right)$}
	{Có $2$ giá trị $a$ để $\displaystyle\int\limits_1^a 3x(x-2)\mathrm{\,d}x=0$}
	\loigiai
	{
		\begin{itemchoice}
			\itemch Đúng. $\displaystyle\int\limits_1^2 \dfrac{1}{x}\mathrm{\,d}x=\ln |x|\bigg\rvert_1^2=\ln 2$.
			\itemch Đúng. $\displaystyle\int\limits_0^a \left(\dfrac{1}{\cos^2 x}+3\right)\mathrm{\,d}x=(\tan x+3x)\bigg\rvert_0^a=\tan a+3a$.
			\itemch Sai. $\displaystyle\int\limits_m^n \dfrac{1}{x\sqrt{x}}\mathrm{\,d}x=\displaystyle\int\limits_m^n x^{\tfrac{-3}{2}}\mathrm{\,d}x=\dfrac{-2}{\sqrt{x}}\bigg\rvert_m^n=-2\left(\dfrac{1}{\sqrt{n}}-\dfrac{1}{\sqrt{m}}\right)$.
			\itemch Sai.
			\begin{eqnarray*}
				\displaystyle\int\limits_1^a 3x(x-2)\mathrm{\,d}x=\displaystyle\int\limits_1^a \left(3x^2-6x\right)\mathrm{\,d}x=\left(x^3-3x^2\right)\bigg\rvert_1^a&=&a^3-3a^2+2=0\\
				&\Leftrightarrow&\hoac{&a=1+\sqrt{3}\\&a=1-\sqrt{3}\\&a=1.}
			\end{eqnarray*}
			Vậy có $3$ giá trị $a$ để $\displaystyle\int\limits_1^a 3x(x-2)\mathrm{\,d}x=0$.
		\end{itemchoice}
	}
\end{ex}
\Closesolutionfile{ans}

\TNSA
\Opensolutionfile{ans}[ans/ansDe3-TN3]
\begin{ex}%[2D4H1-1]%[Đào Trung Kiên]
	Cho $F(x)$ là một nguyên hàm của hàm số $f(x) = ax + \dfrac{b}{x^2}$ $(x \neq 0)$. Biết $F(-1) = 1$, $F(1) = 4$, $f(1) = 0$. Tính giá trị của $M = 2a - b$ (làm tròn tới hàng phần mười).
	\shortans[]{$4,5$}
	\loigiai{
	Ta có $\displaystyle\int f(x)\mathrm{\,d}x = \displaystyle\int \left(ax + \dfrac{b}{x^2}\right)\mathrm{\,d}x = \dfrac{ax^2}{2} - \dfrac{b}{x} + C$.\\
	Theo giả thiết, ta có hệ phương trình $\heva{&F(-1) = 1 \\ &F(1) = 4 \\ &f(1) = 0} \Leftrightarrow \heva{&a + b + C = 1 \\ &a - b + C = 4 \\ &a + b = 0} \Rightarrow \heva{&a = \dfrac{3}{2} \\ &b = -\dfrac{3}{2}\cdot}$\\
	Vậy $M = 2a - b = 3 + \dfrac{3}{2} = \dfrac{9}{2}=4{,}5$
	}
\end{ex}

\begin{ex}%[2D4H2-2]%[Tổ 20 - Đợt 17 - Chương 4 - - CD - Đề 7]%[Lê Thị Thanh Tuyền]
	Cho hàm số $f(x)=a x^2+b x+c$ thỏa mãn $\displaystyle\int\limits_0^1f(x) \mathrm{\,d}x=28, \displaystyle\int\limits_{-1}^1f(x) \mathrm{\,d} x=52$ và $\displaystyle\int\limits_0^2f(x) \mathrm{\,d} x=66$. Giá trị của biểu thức $P=a^b. c$ là
	\shortans{$2025$}

	\loigiai{
		\begin{itemize}
			\item Ta có: $\displaystyle\int f(x) \mathrm{\,d} x=\displaystyle\int\left(a x^2+b x+c\right) d x=\dfrac{a}{3} x^3+\dfrac{b}{2} x^2+c x+C$.
			\item  $\displaystyle\int\limits_0^1f(x)\mathrm{\,d} x=\left.28\Rightarrow\left(\dfrac{a}{3} x^3+\dfrac{b}{2} x^2+c x\right)\right|_0 ^1=28\Leftrightarrow \dfrac{1}{3} a+\dfrac{1}{2} b+c=28 \quad (1)$.
			\item 	$\displaystyle\int\limits_{-1}^1f(x) \mathrm{\,d} x=\left.52\Rightarrow\left(\dfrac{a}{3} x^3+\dfrac{b}{2} x^2+c x\right)\right|_{-1} ^1=52\Leftrightarrow \dfrac{2}{3} a+2c=52 \quad (2)$.
			\item 	$\displaystyle\int\limits_0^2f(x)\mathrm{\,d} x=\left.66\Rightarrow\left(\dfrac{a}{3} x^3+\dfrac{b}{2} x^2+c x\right)\right|_0 ^2=66\Leftrightarrow \dfrac{8}{3} a+2b+2c=66 \quad (3)$.
			\item 	Từ $(1), (2)$ và $(3)$ ta có hệ phương trình: $\heva{&\dfrac{1}{3} a+\dfrac{1}{2} b+c=28\\& \dfrac{2}{3} a+2c=52\\& \dfrac{8}{3} a+2b+2c=66}\Leftrightarrow\heva{&a=3\\&b=4\\&c=25.}$
			\item Vậy $P=3^4.25=2025$.
		\end{itemize}
	}
\end{ex}

\begin{ex}%[2D4H3-1]%[Tổ 21 - Đợt 17 - Chương 4 - - Cánh Diều - Đề 2]
	Cho hàm số $f(x)=\heva{&7-4 x^{3} \text { khi } 0 \leq x \leq 1 \\ &4-x^{2} \text { khi } x>1}$. Tính diện tích hình phẳng giới hạn bởi đồ thị hàm số $f(x)$ và các đường thẳng $x=0$, $x=3$, $y=0$.
	\shortans{$10$}
	\loigiai
	{\immini{Ta có
			\begin{align*}
				S & =\displaystyle\int_{0}^{1} (7-4 x^{3}) \mathrm{d}x+\displaystyle\int_{1}^{2} (4-x^{2}) \mathrm{d}x+\displaystyle\int_{2}^{3}( x^{2}-4) \mathrm{d}x \\
				  & =7 x-\left.x^{4}\right|_{0} ^{1}+\left.\left(4 x-\dfrac{x^{3}}{3}\right)\right|_{1} ^{2}+\left.\left(\dfrac{x^{3}}{3}-4 x\right)\right|_{2} ^{3}   \\
				  & =6+4-\dfrac{7}{3}-3-\dfrac{8}{3}+8=10.
			\end{align*}}{
			\begin{tikzpicture}[scale=.5,font=\footnotesize, line join=round,line cap=round,>=stealth]
				\draw[->] (-1,0)--(4,0) node[below] {$x$};
				\draw[->] (0,-6)--(0,8) node[left] {$y$};
				% Tô vùng giữa x=2 và x=3, trục Ox và đồ thị 4-x^2
				\fill[fill=orange, opacity=0.5]
				(2,0) -- plot[domain=2:3, samples=100] (\x,{4-(\x)^2}) -- (3,0) -- cycle;
				\fill[fill=orange, opacity=0.5]
				(1,0) -- plot[domain=1:2, samples=100] (\x,{4-(\x)^2}) -- (2,0) -- cycle;
				\fill[fill=orange, opacity=0.5]
				(0,0) -- plot[domain=0:1, samples=100] (\x,{7-4*(\x)^3}) -- (1,0) -- cycle;
				% Vẽ các đồ thị hàm số
				\draw[color=blue, samples=100, domain=0:1] plot(\x,{7-4*(\x)^3});
				\draw[color=blue, samples=100, domain=1:3] plot(\x,{4-(\x)^2});
				% Đường nét đứt và chú thích các điểm
				\draw[dashed] (0,3)--(1,3)--(1,0) (3,0)--(3,-5)--(0,-5);
				\node[left] at (0,3) {$3$};
				\node[left] at (0,7) {$7$};
				\node[left] at (0,-5) {$-5$};
				\node[below left] at (0,0) {$O$};
				\node[below] at (1,0) {$1$};
				\node[below left] at (2,0) {$2$};
				\node[below left] at (3,0) {$3$};
			\end{tikzpicture}
		}
	}
\end{ex}

\begin{ex}%[Nguyễn Tuấn, dự án sáng tác đề 12]%[2D4H3-3]
	\immini
	{
		Một vật trang trí có dạng khối tròn xoay tạo thành khi quay miền $(R)$ được giới hạn bởi đường gấp khúc $DABFE$ và cung tròn $ED$ (phần gạch chéo trong hình bên) xung quanh trục $AB$. Biết $ABCD$ là hình chữ nhật cạnh $AB =3$ cm, $AD=2$ cm; $F$ là trung điểm của $BC$; điểm $E$ cách $AD$ một đoạn bằng $1$ cm. Tính thể tích của vật trang trí đó, làm tròn kết quả đến hàng phần mười, đơn vị cm$^3$.
		\shortans{$16{,}5$}
	}
	{
		\begin{tikzpicture}[scale=1.2, line join=round, line cap=round,>=stealth,font=\footnotesize]
			\path (0,0) coordinate (B)	(2,0) coordinate (C) (2,3) coordinate (D) (0,3) coordinate (A) ($(B)!.5!(C)$) coordinate (F) ($(F)+(0,2)$) coordinate (E) ($(A)!.5!(D)$) coordinate (I) ;
			\draw (A)--(B)--(C)--(D)--cycle (E)--(F);
			\draw[dashed] (E)--(I);
			\foreach \d/\g in {A/90,B/-90,C/0,D/90,F/-90,E/-60}
			\draw[fill=black] (\d) circle (1pt) +(\g:0.2) node{$\d$};
			\draw (E) arc (-90:0:1cm);
			\fill[color=gray,opacity=0.5] (A)--(B)--(F)--(I)--cycle (E) arc (-90:0:1cm)--(D)--(I)--cycle;
		\end{tikzpicture}
	}
	\loigiai{
		\immini
		{
			Gắn hệ trục tọa độ như hình vẽ bên.\\
			Quay hình chữ nhật $GEFB$ xung quanh trục $AB$ ta được khối trụ có bán kính đáy bằng $1$ và chiều cao bằng $2$.\\
			Phương trình đường tròn tâm $I$ bán kính bằng $1$ là $x^2+(y-1)^2=1$, suy ra phương trình của cung $DE$ là $y=1+\sqrt{1-x^2}$.\\
			Vây thể tích của vật trang trí là
			\[V=\pi \displaystyle\int\limits_{0}^{1} \left(1+\sqrt{1-x^2}\right)^2 \mathrm{\,d}x+\pi \cdot 1^2\cdot 2\approx 16{,}5\ (\mathrm{cm}^3).\]
		}
		{
			\begin{tikzpicture}[scale=1.2, line join=round, line cap=round,>=stealth,font=\footnotesize]
				\draw[->] (-0.5,0)--(4,0) node[below]{$x$};
				\draw[->] (0,-0.5)--(0,3) node[left]{$y$};
				\path (0,0) coordinate (A)	(3,0) coordinate (B) (3,2) coordinate (C) (0,2) coordinate (D) ($(B)!.5!(C)$) coordinate (F) ($(F)-(2,0)$) coordinate (E) ($(A)!.5!(D)$) coordinate (I) (1,0) coordinate (G) ;
				\draw (A)--(B)--(C)--(D)--cycle (E)--(F);
				\draw[dashed] (E)--(I) (E)--(G);
				\foreach \d/\g in {A/-120,B/-90,C/0,D/120,F/0,E/-60,I/180,G/-90}
				\draw[fill=black] (\d) circle (1pt) +(\g:0.2) node{$\d$};
				\draw (E) arc (0:90:1cm);
				\fill[color=gray,opacity=0.5] (A)--(B)--(F)--(I)--cycle (E) arc (0:90:1cm)--(D)--(I)--cycle;
			\end{tikzpicture}
		}
	}
\end{ex}

\Closesolutionfile{ans}

\TNTF
\begin{ex}%[2D4H2-2]
	Tính tích phân $A=\displaystyle\int\limits_{-3}^5 {\left(|x+2|-|x-2|\right)}\mathrm{\,d}x$.

	\loigiai{
	$A=\displaystyle\int\limits_{-3}^5 {\left(|x+2|-|x-2|\right)}\mathrm{\,d}x$.\\
	Ta có bảng xét dấu để phá trị tuyệt đối:
	\begin{center}
		\begin{tikzpicture}
			\tkzTabInit[nocadre=false,lgt=3,espcl=2.5,deltacl=0.5]
			{$x$/.6,$|x+2|$/.6,$|x-2|$/.6,$|x+2|+|x-2|$/.6}
			{$-3$,$-2$,$2$,$5$}
			\tkzTabLine{,-x-2,0,x+2,|,-x-2,}
			\tkzTabLine{,-x+2,|,-x+2,0,x-2,}
			\tkzTabLine{,-4,|,2x,|,4,}
		\end{tikzpicture}
	\end{center}
	Khi đó
	\begin{eqnarray*}
		A=	\displaystyle\int\limits_{-3}^5 {\left(|x+2|-|x-2|\right)}\mathrm{\,d}x
		&=& \displaystyle\int\limits_{-3}^{-2} {(-4)}\mathrm{\,d}x +
		\displaystyle\int\limits_{-2}^2 {2x}\mathrm{\,d}x +
		\displaystyle\int\limits_2^5 {4}\mathrm{\,d}x\\
		&=& -4x \Big|_{-3}^{-2} + x^2 \Big|_{-2}^2 + \cdot 4x \Big|_2^5=8.
	\end{eqnarray*}
	}
\end{ex}


\begin{ex}%[2D4V1-4]
	Cho hàm số $ y=f(x)$ đồng biến và có đạo hàm liên tục trên $\mathbb{R}$ thỏa mãn $\left(f'(x)\right)^2=f(x)\cdot\mathrm{e}^x$, $\forall x\in\mathbb{R}$ và $f(0)=2$. Tính $ f(2)$. (Kết quả làm tròn đến hàng phần trăm).
	% \shortans{$9{,}81$}
	\loigiai{
	Vì hàm số $ y=f(x)$ đồng biến và có đạo hàm liên tục trên $\mathbb{R}$ đồng thời $ f(0)=2$ nên $f'(x)\ge 0$ và $ f(x)>0$ với mọi $ x\in\left[0;+\infty\right)$.\\
	Từ giả thiết $\left(f'(x)\right)^2=f(x)\cdot \mathrm{e}^x$, $\forall x\in\mathbb{R}$ suy ra $f'(x)=\sqrt{f(x)}\cdot\mathrm{e}^{\tfrac{x}{2}}$, $\forall x\in\left[0;+\infty\right).$\\
	Do đó $\dfrac{f'(x)}{2\sqrt{f(x)}}=\dfrac{1}{2}{\mathrm{e}^{\tfrac{x}{2}}}$, $\forall x\in\left[0;+\infty\right).$\\
	Lấy nguyên hàm hai vế, ta được $\sqrt{f(x)}=e^{\tfrac{x}{2}}+C$, $\forall x\in\left[0;+\infty\right)$ với $C$ là hằng số.\\
	Kết hợp với $ f(0)=2$, ta được $C=\sqrt{2}-1$.\\
	Suy ra $ f(2)=\left(\mathrm{e}+\sqrt{2}-1\right)^2\approx 9{,}81$.}
\end{ex}

\begin{ex}%[EX-Ôn Tập TN 2025, Võ Thanh Phong]%[2D4C3-2]
	Một cổng có đạng hình parabol với chiều cao $8$ m, chiều rộng chân đế $8$ m. Người ta căng hai sợi dây trang trí $AB$,  $CD$ nằm ngang, đồng thời chia cổng thành ba phần sao cho hai phần ở phía trên có diện tích bằng nhau. Tỉ số $\dfrac{CD}{AB}$ bằng bao nhiêu (làm tròn kết quả đến hàng phần trăm)?
	\begin{center}
		\begin{tikzpicture}[scale=0.7, font=\footnotesize, line join=round, line cap=round,>=stealth]
			\begin{scope}
				\clip (-4,-4.5) rectangle (4,0);
				\draw [smooth,domain=-5:4, samples=200] plot (\x, {-0.5*(\x)^2});
			\end{scope}
			\draw [<->] (3.3,0)--(3.3,-4.5);
			\node[right] at (3.3,-2){$8\,\,m$};
			\draw [<->] (-3,-4.5)--(3,-4.5);
			\node[below] at (0,-4.5){$8\,\,m$};
			\node[left] at (-1.8,-1.62){$A$};
			\node[right] at (1.8,-1.62){$B$};
			\node[left] at (-2.5,-3.125){$C$};
			\node[right] at (2.5,0-3.125){$D$};
			\draw [-] (-1.8,-1.62)--(1.8,-1.62);
			\draw [-] (-2.5,-3.125)--(2.5,0-3.125);
			\draw [dashed] (0,0)--(4,0);
			\node[below] at (current bounding box.south){\textit{Hình 5}};
		\end{tikzpicture}
	\end{center}
	% \shortans{ $1{,}26$}
	\loigiai{
	\immini{Gắn hệ trục tọa độ $Oxy$ vào cổng parabol như hình bên với trục $Oy$ trùng với đường đối xứng của parabol, gốc $O$ nằm ở đỉnh của parabol, đơn vị trên mỗi trục tính theo mét. Khi đó, phương trình parabol có dạng $y=ax^2$.\\
		Vì parabol đi qua điểm có toạ độ $(-4 ;-8)$ nên $a=-\dfrac{1}{2}$. Suy ra phương trình parabol là $y=-\dfrac{1}{2} x^2$.\\}{
		\begin{tikzpicture}[scale=0.7, font=\footnotesize, line join=round, line cap=round,>=stealth]
			\draw[->] (-4,0) --(4,0) node[below]{$x$};
			\draw[->] (0,-5) --(0,1) node[left]{$y$};
			\draw (0,0) node[above left=-3pt]{$O$};
			\begin{scope}
				\clip (-4,-4.5) rectangle (4,0);
				\draw [smooth,domain=-5:4, samples=200] plot (\x, {-0.5*(\x)^2});
			\end{scope}
			\draw [<->] (3.3,0)--(3.3,-4.5);
			\node[right] at (3.3,-2){$8\,\,m$};
			\draw [<->] (-3,-4.5)--(3,-4.5);
			\node[below] at (0,-4.5){$8\,\,m$};
			\node[left] at (-1.8,-1.62){$A$};
			\node[right] at (1.8,-1.62){$B$};
			\node[left=-3pt] at (-2.5,-3.125){$C$};
			\node[right] at (2.5,0-3.125){$D$};
			\draw [-] (-1.8,-1.62)--(1.8,-1.62);
			\draw [-] (-2.5,-3.125)--(2.5,0-3.125);
			\draw [dashed] (0,0)--(4,0) (1.8,-1.62)--(1.8,0) (2.5,0-3.125)--(2.5,0) (-3,-4.5)--(-3,0);
			\node[above] at (1.8,0){$x_1$};
			\node[above] at (2.5,0){$x_2$};
			\node[above] at (-3,0){$-4$};
		\end{tikzpicture}}
	Giả sử $B$ có hoành độ $x_1$,  $D$ có hoành độ $x_2$. Khi đó, phương trình đường thẳng $AB$ là $y=-\dfrac{1}{2} x_1^2$, phương trình đường thẳng $CD$ là $y=-\dfrac{1}{2} x_2^2$.\\
	Diện tích hình phẳng giới hạn bởi parabol và đường thẳng $AB$ là
	\[
		S_1=2\displaystyle\int\limits_0^{x_1}\left[-\dfrac 12x^2-\left(-\dfrac 12x_1^2\right)\right]\mathrm{\,d} x=\left.2\left(-\dfrac{x^3}6+\dfrac{x_1^2}2x\right)\right|_0^{x_1}=\dfrac 23x_1^3\,\,\left(\mathrm{m}^2\right).
	\]
	Diện tích hình phẳng giới hạn bởi parabol và đường thẳng $CD$ là
	\[
		S_2=2\displaystyle\int\limits_0^{x_2}\left[-\dfrac 12x^2-\left(-\dfrac 12x_2^2\right)\right]\mathrm{\,d} x=\left.2\left(-\dfrac{x^3}6+\dfrac{x_2^2}2x\right)\right|_0^{x_2}=\dfrac 23x_2^3\,\,\left(\mathrm{m}^2\right).
	\]
	Theo giả thiết, ta có  $S_2=2S_1\Leftrightarrow x_2^3=2x_1^3\Leftrightarrow\dfrac{x_2}{x_1}=\sqrt[3]2\approx 1{,}26$.\\
	Khi đó, $\dfrac{CD}{AB}=\dfrac{2x_2}{2x_1}\approx 1{,}26$.
	}
\end{ex}

% \Closesolutionfile{ansbook}
% \HetDe
% \label{De3}
% %
% \cleardoublepage
% \setcounter{page}{1}
% \rfoot{Trang \thepage/\pageref{DA3} - Đáp án trắc nghiệm Mã đề 3}
% \begin{center}
% 	\bfseries ĐÁP ÁN TRẮC NGHIỆM MÃ ĐỀ 3
% \end{center}

% \inputansbox{10}{ans/ansDe3-TN1}
% \inputansbox[3]{2}{ans/ansDe3-TN2}
% \inputansbox{3}{ans/ansDe3-TN3}
% \label{DA3}
%

% \begin{name}
	{\tenchude}
	{TOÁN 12}
	{LỚP TOÁN THẦY PHÁT}
	{Thời gian: 90 phút - Không kể thời gian phát đề}
\end{name}
\TN
\Opensolutionfile{ans}[ans/ansDe4-TN1]
\begin{ex}%[2D4N1-1]
	Cho hàm số $F(x)$ là một nguyên hàm của hàm số $f(x)$ trên $K$. Các mệnh đề sau, mệnh đề nào \textbf{sai}.
	\choice
	{$\displaystyle\int{f(x)\mathrm{\,d}x=}F(x)+C$}
	{$\displaystyle{\left(\displaystyle\int{f(x)\mathrm{\,d}x}\right)'}=f(x)$}
	{\True $\displaystyle{\left(\displaystyle\int{f(x)\mathrm{\,d}x}\right)'}=f'(x)$}
	{$\displaystyle{\left(\displaystyle\int{f(x)\mathrm{\,d}x}\right)'}=F'(x)$}
	\loigiai{
		Ta có $\displaystyle\int{f(x)\mathrm{\,d}x=}F(x)+C\Leftrightarrow F'(x)=f(x)$ nên phương án $\left(\displaystyle\int{f(x)\mathrm{\,d}x}\right)'=f'(x)$ sai.}
\end{ex}

\begin{ex}%[2D4N1-1]%[To 20 - Dot 17 - Chuong 4 - Bai 3 - CD - De 1 - TN]%[Nguyễn Hữu Duy]
	Nếu hàm số $f(x)$ liên tục trên đoạn $[a;b]$ và $c$ là số thực tùy ý thuộc đoạn $[a;b]$, thì tính chất nào sau đây đúng?
	\choice
	{\True $\displaystyle\int_a^b f(x) \mathrm{\,d}x = \displaystyle\int_a^c f(x) \mathrm{\,d}x + \displaystyle\int_c^b f(x)\mathrm{\,d}x$}
	{$\displaystyle\int_a^b f(x)\mathrm{\,d}x = \displaystyle\int_a^c f(x) \mathrm{\,d}x - \displaystyle\int_c^b f(x) \mathrm{\,d}x$}
	{$\displaystyle\int_a^b f(x)\mathrm{\,d}x = \displaystyle\int_a^c f(x)\mathrm{\,d}y + \displaystyle\int_c^b f(x) \mathrm{\,d}z$}
	{$\displaystyle\int_a^b f(x)\mathrm{\,d}x = \displaystyle\int_a^c f(x)\mathrm{\,d}y - \displaystyle\int_c^b f(x)\mathrm{\,d}z$}
	\loigiai{
		Theo định nghĩa tích phân ta có $\displaystyle\int_a^b f(x) \mathrm{\,d}x = \displaystyle\int_a^c f(x) \mathrm{\,d}x + \displaystyle\int_c^b f(x)\mathrm{\,d}x$.
	}
\end{ex}

\begin{ex}%[2D4N1-2]
	Cho hàm số $f(x)=x^2+4$. Mệnh đề nào sau đây đúng?

	\choice
	{$\displaystyle{\displaystyle\int f(x)\mathrm{\,d}x=2 x+C}$}
	{$\displaystyle{\displaystyle\int f(x)\mathrm{\,d}x=x^2+4 x+C}$}
	{\True $\displaystyle{\displaystyle\int f(x)\mathrm{\,d}x=\dfrac{x^3}{3}+4 x+C}$}
	{$\displaystyle{\displaystyle\int f(x)\mathrm{\,d}x=x^3+4 x+C}$}
	\loigiai{
		Ta có $f(x)=x^2+4 $ nên $ \displaystyle\int f(x)\mathrm{\,d}x=\dfrac{x^3}{3}+4 x+C$.}
\end{ex}

\begin{ex}%[2D4N1-3]
	Tìm nguyên hàm của hàm số $ f(x)=\cos 3x$.
	\choice
	{ $\displaystyle\int{\cos 3x\mathrm{d}x=3\sin 3x+C}$}
	{\True $\displaystyle\int{\cos 3x\mathrm{d}x=\dfrac{\sin 3x}{3}+C}$}
	{ $\displaystyle\int{\cos 3x\mathrm{d}x=\sin 3x+C}$}
	{$\displaystyle\int{\cos 3x\mathrm{d}x=-\dfrac{\sin 3x}{3}+C}$}
	\loigiai{
		Ta có:$\displaystyle\int{\cos 3x\mathrm{d}x=\dfrac{\sin 3x}{3}+C}$.
	}
\end{ex}

\begin{ex}%[2D4N1-4]
	Hàm số $F(x)=\mathrm{e}^{2x}$ là một nguyên hàm của hàm số nào dưới đây?
	\choice
	{$f_4(x)=\dfrac{1}{2}\mathrm{e}^{2x}$}
	{$f_1(x)=\mathrm{e}^{2x}$}
	{$f_2(x)=\mathrm{e}^{x^2}$}
	{\True $f_3(x)=2\mathrm{e}^{2x}$}
	\loigiai{
		Ta có $F'(x)=f(x)$ nên $f(x)=\left(\mathrm{e}^{2x}\right)^\prime=2\mathrm{e}^{2x}$.}
\end{ex}

\begin{ex}%[BG-12-4in1, Phạm Đức]%[2D4N1-4]
	Họ nguyên hàm của hàm số $f(x)=2024^x$ là
	\choice
	{\True $\dfrac{2024^x}{\ln 2024}+C$}
	{$2024^x\ln 2024+C$}
	{$2024^x+C$}
	{$2024^x\ln x+C$}
	\loigiai{

	}
\end{ex}

\begin{ex}%[2D4H1-2]
	Tìm nguyên hàm của hàm số $f(x)=(5x+3)^5$.
	\choice
	{$(5x+3)^6+C$}
	{$(5x+3)^4+C$}
	{\True $\dfrac{(5x+3)^6}{30}+C$}
	{$\dfrac{(5x+3)^4}{30}+C$}
	\loigiai{
		$f(x)=(5x+3)^5$ $\displaystyle \Rightarrow \displaystyle\int{f(x)\mathrm{\,d}x=}\displaystyle\int{(5x+ 3)^5\mathrm{\,d}x=}\dfrac{1}{5}\cdot \dfrac{(5x+3)^6}{6}+C=\dfrac{(5x+3)^6}{30}+C$.}
\end{ex}

\begin{ex}%[2D4H1-3]
	Họ nguyên hàm của hàm số $f(x)=\cos 2x$ là
	\choice
	{\True $\dfrac{1}{2} \sin 2x +C$}
	{$-2 \sin 2x +C$}
	{$-\dfrac{1}{2} \sin 2x +C$}
	{$2 \sin 2x +C$}
	\loigiai{
		Ta có  $\displaystyle\int\limits \cos 2x\mathrm{\,d}x=\dfrac{1}{2} \sin 2x+C$.
	}
\end{ex}

\begin{ex}%[2D4N2-1]
	Cho hàm số $f(t)$ liên tục trên $K$ và $a$, $b\in K$, $F(t)$ là một nguyên hàm của $ f(t)$ trên $K$. Chọn khẳng định \textbf{sai} trong các khẳng định sau.
	\choice
	{\True $F(a)-F(b)=\displaystyle\int\limits_a^b f(t)\mathrm{d}t$}
	{$\displaystyle\int\limits_a^bf(t)\mathrm{d}t=F(t)\big|^b_a$}
	{$\displaystyle\int\limits_a^bf(t)\mathrm{d}t=\left(\displaystyle\int f(t)\mathrm{d}t\right)\bigg|^b_a$}
	{$\displaystyle\int\limits_a^bf(x)\mathrm{d}x=\displaystyle\int\limits_a^bf(t)\mathrm{d}t$}
	\loigiai{
		Theo tính chất của tích phân.
		Ta có  $\displaystyle\int\limits_a^b f(t)\mathrm{d}t=F(b)-F(a)$.}
\end{ex}

\begin{ex}%[2D4N2-2]%[Tổ 20 - Đợt 17 - Chương 4 - - CD]%[Phạm Hà Giang]
	Tính tích phân $\displaystyle\int\limits_1^2 \dfrac{1}{x}\mathrm{\,d}x$
	\choice
	{$0$}
	{\True $\ln 2$}
	{$\dfrac{1}{2}$}
	{$\dfrac{-1}{2}$}
	\loigiai
	{
		$\displaystyle\int\limits_1^2 \dfrac{1}{x}\mathrm{\,d}x=\left. \ln \left| x\right| \right| _1^2 =\ln 2$.
	}
\end{ex}

\begin{ex}%[Nguyễn Tuấn, dự án sáng tác đề 12]%[2D4N2-3]
	Giá trị của $\displaystyle\int\limits_0^{\frac{\pi}{2}} \cos x\mathrm{\,d}x$ bằng
	\choice
	{$0$}
	{\True $1$}
	{$\dfrac{\pi}{2}$}
	{$\pi$}
	\loigiai
	{
	Ta có $\displaystyle\int\limits_0^{\frac{\pi}{2}} \cos x\mathrm{\,d}x = \sin x\Big|_0^{\frac{\pi}{2}} = \sin\dfrac{\pi}{2}-\sin 0 = 1$.
	}
\end{ex}

\begin{ex}%[Tổ 20 - Chương 4 - - CD]%[Nguyễn Văn Sang]%[2D4N2-4]
	Biết $I=\displaystyle\int\limits_0^1 3^x \cdot 4^{2 x} \cdot \mathrm{\,d} x=\dfrac{a}{\ln 48}$.  Khi đó $a+1$ bằng
	\choice
	{\True $48$}
	{$46$}
	{$47$}
	{$49$}
	\loigiai{
		Ta có	\[
			I=\displaystyle\int\limits_0^1 3^x \cdot 4^{2 x}\mathrm{\,d}x=\displaystyle\int\limits_0^1 3^x \cdot 16^x d x=\displaystyle\int\limits_0^1 48^x\mathrm{\,d} x=\dfrac{48^x}{\ln 48}\bigg|_0 ^1=\dfrac{47}{\ln 48}.
		\]
		Suy ra $a=47$ và $a+1=48$.
	}
\end{ex}
\Closesolutionfile{ans}

\TNTF
\Opensolutionfile{ans}[ans/ansDe4-TN2]
\begin{ex}%[2025-TLDH- Huỳnh Xuân Tín]%[2D4N1-4]
	Cho $I_1=\displaystyle\int\left(\mathrm e^x+\dfrac{1}{x^2}\right) \mathrm{d}x$ và $I_2=\displaystyle\int\left( \mathrm e^{2x-1}-\dfrac{1}{x^2}\right) \mathrm{d}x$.
	\choiceTF
	{\True  $I_1=\mathrm{e}^x-\dfrac{1}{x}+C$}
	{$I_2=\dfrac{\mathrm{e}^{2x-1}}{2}+\ln |x|+C$ }
	{\True $I_1+I_2=\mathrm{e}^x+\dfrac{{\mathrm{e}^{2x-1}}}{2}+C$ }
	{Gọi $F(x)$ là nguyên hàm của hàm số $f(x)$, với $f(x)=\mathrm{e}^x+\dfrac{1}{x^2}$. Nếu $F(1)=\mathrm{e}$ thì $F(\ln 2)=1-\dfrac{1}{\ln 2}$}
	\loigiai{
		\begin{itemchoice}
			\itemch \textbf{Đúng.}\\
			Vì $I_1=\displaystyle\int(\mathrm{e}^x+\dfrac{1}{x^2})\mathrm{d}x=\mathrm{e}^x-\dfrac{1}{x}+C$.
			\itemch \textbf{Sai.}\\
			Ta có $I_2=\displaystyle\int\left( \mathrm{e}^{2x-1}-\dfrac{1}{x^2}\right) \mathrm{d}x=\dfrac{\mathrm{e}^{2x-1}}{2}+\dfrac{1}{x}+C$.
			\itemch \textbf{Đúng.}\\
			Ta có
			\begin{eqnarray*}
				I_1+I_2	&= & f(x)=g(x)\displaystyle\int( \mathrm{e}^x+\dfrac{1}{x^2} )\mathrm{\,d}x+\displaystyle\int( {e^{2x-1}}-\dfrac{1}{x^2} )\mathrm{\,d}x\\
				&=& \displaystyle\int\left(\mathrm{e}^x+\mathrm{e}^{2x-1} \right) \mathrm{\,d}x\\
				&= &\mathrm{e}^x+\dfrac{\mathrm{e}^{2x-1}}{2}+C.
			\end{eqnarray*}
			\itemch \textbf{Sai.}\\
			Ta có $I_1=\displaystyle\int(\mathrm{e}^x+\dfrac{1}x^2)\mathrm{d}x=\mathrm{e}^x-\dfrac{1}{x}+C$. Vì $F(1)=\mathrm{e}\Rightarrow \mathrm{e}-1+C=\mathrm{e}\Rightarrow C=1$.\\
			$F(x)=\mathrm{e}^x-\dfrac{1}{x}+1\Rightarrow F(\ln 2)=\mathrm{e}^{\ln 2}-\dfrac{1}{\ln 2}+1=2-\dfrac{1}{\ln 2}+1=3-\dfrac{1}{\ln 2}$.
		\end{itemchoice}
	}
\end{ex}

\begin{ex}%[Dự án 2025 - đề cấu trúc mới, Nguyễn Kiều Nhã Tú]%[2D4H2-3]
	Cho hàm số $y=f(x)$. Biết $f'(x)=2\cos^2 x + 3$, $\forall x\in \mathbb{R}$.
	\choiceTF
	{$f'(x)>0$ với $\forall x\in\mathbb{R}$ nên $f(x)>0$,  $\forall x\in \mathbb{R}$}
	{$f'(x)=\displaystyle\int f(x)\mathrm{\,d}x$}
	{\True $f(x)=\dfrac{1}{2}\sin 2x+4x+C$}
	{\True Biết $f(0)=4$. Khi đó $\displaystyle\int\limits_0^{\frac{\pi}{4}}f(x) \mathrm{\,d}x$ bằng $\dfrac{\pi^2+8\pi+2}{8}$}
	\loigiai{
		\begin{itemchoice}
			\itemch \textbf{Sai}. Vì vì đạo hàm không có tính chất này.
			\itemch \textbf{Sai}. Vì $f(x)=\displaystyle\int f'(x)\mathrm{\,d}x$.
			\itemch \textbf{Đúng}. Vì
			\begin{align*}
				f(x) & =\displaystyle\int f'(x)\mathrm{\,d}x=\displaystyle\int\left(2\cos^2 x + 3\right)\mathrm{\,d}x \\
				     & =\displaystyle\int \left(2\cdot\dfrac{1+\cos 2x}{2}+3\right)  \mathrm{\,d}x
				= \displaystyle\int \left(\cos 2x+4\right)\mathrm{\,d}x                                               \\
				     & = \dfrac{1}{2} \sin{2x} + 4x + C.
			\end{align*}
			\itemch \textbf{Đúng}.
			Ta có $f(x)=\dfrac{1}{2}\sin 2x + 4x + C$.
			Do $f(0)=4 \Rightarrow C=4$ \\
			Vậy $f(x)=\dfrac{1}{2}\sin 2x + 4x + 4$ nên \\
			$\displaystyle\int\limits_0^{\frac{\pi}{4}}f(x)\mathrm{\,d}x = \displaystyle\int\limits_0^{\frac{\pi}{4}}\left(\dfrac{1}{2}\sin 2x+ 4x+4\right)\mathrm{\,d}x
				=\left(-\dfrac{1}{4}\cos 2x+2x^2+4x\right)\Big|_0^{\frac{\pi}{4}}
				= \dfrac{\pi^2+8\pi+2}{8}$.
		\end{itemchoice}
	}
\end{ex}
\Closesolutionfile{ans}

\TNSA
\Opensolutionfile{ans}[ans/ansDe4-TN3]
\begin{ex}%[2D4H1-1]%[Đào Trung Kiên]
	Giả sử $F(x)$ là một nguyên hàm của hàm số $f(x)=\mathrm{e}^x$, biết $F(0)=4$. Tìm $F(1)$ (làm tròn kết quả tới phần mười).
	\shortans[]{$5,7$}
	\loigiai{
		Do $F(x)$ là một nguyên hàm của $f(x)=\mathrm{e}^x$ nên $F(x)=\mathrm{e}^x+C$.\\
		Lại có $F(0)=4$ nên $C=3$ hay $F(x)=\mathrm{e}^x+3$ nên $F(1)=\mathrm{e}+3\approx 5{,}7$.
	}
\end{ex}

\begin{ex}%[2D4H2-2]%[Tổ 20 - Đợt 17 - Chương 4 - - CD - Đề 7]%[Lê Thị Thanh Tuyền]
	Có bao nhiêu giá trị nguyên của $a$ để $\displaystyle\int\limits_1^a(2x-3) \mathrm{\,d} x \leq 6$?
	\shortans{$6$}

	\loigiai{
		\begin{itemize}
			\item Ta có: $\displaystyle\int\limits_1^a(2x-3)\mathrm{\,d} x=\left.\left(x^2-3x\right)\right|_1 ^a=a^2-3a+2$.
			\item Khi đó: $\displaystyle\int\limits_1^a(2x-3) \mathrm{\,d} x \leq 6\Leftrightarrow a^2-3a+2\leq 6\Leftrightarrow-1\leq a \leq 4$
			\item	Mà $a$ là số nguyên nên $a \in\{-1; 0; 1; 2; 3; 4\}$.
			\item	Vậy có $6$ giá trị của $a$ thỏa đề bài.
		\end{itemize}


	}
\end{ex}

\begin{ex}%[2D4H3-1]
	Gọi $S$ là hình phẳng giới hạn bởi đồ thị hàm số $(H)\colon y=\dfrac{x-1}{x+1}$ và các trục tọa độ. Tính diện tích hình phẳng $(S)$ (làm tròn đến chữ số thứ hai sau dấu phẩy).
	\shortans{$0{,}39$}
	\loigiai{
		Điều kiện $x\ne -1$.\\
		Hoành độ giao điểm của đồ thị hàm số và trục $Ox$ là nghiệm của phương trình \[\dfrac{x-1}{x+1}=0\Leftrightarrow x=1.\]
		Vậy diện tích hình phẳng cần tìm là
		\allowdisplaybreaks
		\begin{eqnarray*}
			S&=&\displaystyle\int\limits_0^1 \left|\dfrac{x-1}{x+1}\right| \mathrm{\,d}x=\displaystyle\int\limits_0^1 \dfrac{1-x}{x+1} \mathrm{\,d}x\\
			&=&\displaystyle\int\limits_0^1 \left(-1+\dfrac{2}{x+1}\right) \mathrm{\,d}x\\
			&=&\left(-x+2\ln |x+1|\right)\Bigr\rvert_0^1=2\ln 2-1\approx 0{,}39.
		\end{eqnarray*}
	}
\end{ex}

\begin{ex}%[2D4H3-3]
	Tính thể tích của vật thể tròn xoay được tạo thành khi quay hình $(H)$ quanh $Ox$ với $(H)$ được giới hạn bởi đồ thị hàm số $y=\sqrt{4x-x^2}$ và trục hoành. (kết quả làm tròn đến hàng phần mười)
	\shortans{$33{,}5$}
	\loigiai{
	Điều kiện xác định: $4x-x^2\ge 0\Leftrightarrow 0\le x\le 4$.\\
	Phương trình hoành độ giao điểm của đồ thị hàm số $y=\sqrt{4x-x^2}$ và trục hoành là
	\[\sqrt{4x-x^2}=0\Leftrightarrow 4x-x^2=0\Leftrightarrow \hoac{
			&x=0 \\
			&x=4. \\
		}\]
	Thể tích của vật thể tròn xoay khi quay hình $(H)$ quanh $Ox$ là
	\[V=\pi \displaystyle \int\limits_0^4\left(\sqrt{4x-x^2}\right)^2\mathrm{\,d}x=\pi \displaystyle \int\limits_0^4{(4x-x^2)}\mathrm{\,d}x=\dfrac{32}{3}\pi.\]
	Vậy thể tích của vật thể tròn xoay khi quay hình $(H)$ quanh $Ox$ là $\dfrac{32}{3}\pi\approx33{,}5$.
	}
\end{ex}

\Closesolutionfile{ans}

\TL
\begin{ex}%[Mức độ 2]%[BG12, Nguyễn Kiều Nhã Tú]%[2D4H2-2]
	Cho hàm số $f(x)=\heva{&x^2\,\, \text{khi}\,\, 0\le x \le 1\\&2-x\,\, \text{khi} \,\,1< x \le 2}$. Tính $\displaystyle\int_0^2 f(x) \mathrm{\,d}x$.
	\loigiai{
		Ta có $\displaystyle\int_0^2 f(x) \mathrm{\,d}x=\displaystyle\int_0^1 x^2 \mathrm{\,d}x+\displaystyle\int_1^2 (2-x) \mathrm{\,d}x=\dfrac{x^3}{3}\bigg|_0^1+\left( 2x-\dfrac{x^2}{2} \right)\bigg|_1^2=\dfrac{5}{6}$.
	}
\end{ex}

\begin{ex}%[2D4V1-4]
	Cho hàm số $ f(x)$ nhận giá trị dương và thỏa mãn $ f(0)=1$, $\left(f'(x)\right)^3=\mathrm{\mathrm{e}}^ x{\left(f(x)\right)^2}$, $\forall x\in\mathbb{R}$. Tính $ f(3)$ (\textit{kết quả làm tròn đến hàng phần mười}).
	% \shortans{$20{,}1$}
	\loigiai{
	Ta có

	\begin{align*}
		\left(f'(x)\right)^3=\mathrm{e}^x{\left(f(x)\right)^2},\,\forall x\in\mathbb{R}
		 & \Leftrightarrow{f}'(x)=\sqrt[3]{\mathrm{e}^x}\cdot \sqrt[3]{\left(f(x)\right)^2}\Leftrightarrow\dfrac{f'(x)}{\sqrt[3]{\left(f(x)\right)^2}}=\sqrt[3]{\mathrm{e}^x}     \\
		 & \Leftrightarrow\dfrac{f'(x)}{\sqrt[3]{\left(f(x)\right)^2}}=\sqrt[3]{\mathrm{e}^x}\Leftrightarrow{f}'(x)\cdot \left(f(x)\right)^{-\tfrac{2}{3}}=\sqrt[3]{\mathrm{e}^x} \\&\Leftrightarrow 3\left[\left(f(x)\right)^{\tfrac{1}{3}}\right]'=\sqrt[3]{\mathrm{e}^x}\Leftrightarrow{\left[\left(f(x)\right)^{\tfrac{1}{3}}\right]'}=\dfrac{1}{3}\sqrt[3]{\mathrm{e}^x}\\&\Leftrightarrow{\left(f(x)\right)^{\tfrac{1}{3}}}=\dfrac{1}{3}\displaystyle\int{\sqrt[3]{\mathrm{e}^x}}\mathrm{\,d} x \Leftrightarrow{\left(f(x)\right)^{\tfrac{1}{3}}}=e^{\tfrac{x}{3}}+C.
	\end{align*}
	Vì	$f(0)=1$ nên $1=1+C\Rightarrow C=0\Rightarrow{\left(f(x)\right)^{\tfrac{1}{3}}}=e^{\tfrac{x}{3}}\Rightarrow f(x)=\mathrm{e}^x$.\\
	Vậy	$f(3)=e^3\approx 20{,}1$.
	}
\end{ex}

\begin{ex}%[2D4C3-5]
	Cho một mô hình $3-D$ mô phỏng một đường hầm như hình vẽ bên. Biết rằng đường hầm mô hình có chiều dài $5$ (cm); khi cắt hình này bởi mặt phẳng vuông góc với đáy của nó, ta được mặt cắt là một hình parabol có độ dài đáy gấp đôi chiều cao parabol. Chiều cao của mỗi mặt cắt hình parabol cho bởi công thức $ y=3-\dfrac{2}{5}x$ (cm), với $x$ (cm) là khoảng cách tính từ lối vào lớn hơn của đường hầm mô hình. Tính thể tích (theo đơn vị cm$^3$) không gian bên trong đường hầm mô hình (làm tròn kết quả đến hàng đơn vị).
	% \shortans{$29$}
	\begin{center}
	\begin{tikzpicture}[scale=1,declare function={a=0.8;b=0.6;c=0.4;d=0.2;}]
	\tikzset{
	homothety at/.style args={#1 scaled by #2}{shift={($(#1)!#2!(0,0)$)},scale=#2},
	}
	\def\mypath{(-120:2)..controls +(90:0.6) and +(-180:0.6)..(0,3)}
	\def\mydot{(0,3)..controls +(0:0.25) and +(95:0.05)..(60:2)}
	\draw \mypath;
	\draw[dashed] \mydot;
	\path (7,0) coordinate (c1);
	\begin{scope}[homothety at=c1 scaled by a]
	\draw \mypath;
	\draw[dashed] \mydot;
	\end{scope}
	\begin{scope}[homothety at=c1 scaled by b]
	\draw \mypath;
	\draw[dashed] \mydot;
	\end{scope}
	\begin{scope}[homothety at=c1 scaled by c]
	\draw \mypath;
	\draw[dashed] \mydot;
	\end{scope}
	\begin{scope}[homothety at=c1 scaled by d]
	\draw \mypath;
	\draw \mydot;
	\end{scope}
	\path
	(-120:2) coordinate (A)
	(0,3) coordinate (B)
	(60:2) coordinate (C);
	\foreach \x in {A,B,C}{\path ($(c1)!a!(\x)$) coordinate (\x_1);}
	\foreach \x in {A,B,C}{\path ($(c1)!b!(\x)$) coordinate (\x_2);}
	\foreach \x in {A,B,C}{\path ($(c1)!c!(\x)$) coordinate (\x_3);}
	\foreach \x in {A,B,C}{\path ($(c1)!d!(\x)$) coordinate (\x_4);}
	\path ($(A_4)!0.5!(B_4)$) coordinate (D);
	\draw (A)--(A_4) (B)--(B_4) (A_4)--(C_4)
	;
	\draw[dashed] (B)node[above]{$3$}--(0,0)--(D)node[below right]{$5$} (A)--(C) (A_1)--(C_1) (A_2)--(C_2) (A_3)--(C_3)  (C)--(C_4);
	\end{tikzpicture}
	\end{center}
	\loigiai{
	\begin{center}
	\begin{tikzpicture}[scale=1,font=\footnotesize]
	\path (0,0) coordinate (O)
	(2,0) coordinate (A)
	(0,2) coordinate (B)
	;
	\draw[-stealth] (-3.5,0)--(0,0)--(3,0)node[below]{$x$};
	\draw[-stealth] (0,-1.5)--(0,4)node[left]{$y$};
	\draw[smooth,samples=100] plot[domain=-2:2](\x,{(-1/2)*(\x)^2+2});
	\foreach \x in {O,A,B}{\draw[fill=blue!40] (\x) circle (1pt);}
	\foreach \x in {-3,-2,-1,1}{\draw (\x,0.05)--(\x,-0.05);}
	\foreach \x in {-1,1,3}{\draw (-0.05,\x)--(0.05,\x);}
	\node[above left] at (B) {$h$};
	\path (O)--(A)node[below]{$h$};
	\node at (0,0) [below left]{$O$};
	\end{tikzpicture}
	\end{center}
	Xét một mặt cắt hình parabol có chiều cao là $h$ và độ dài đáy $2h$ và chọn hệ trục $Oxy$ như hình vẽ trên.\\
	Parabol $(P)$ có phương trình $(P)\colon y=ax^2+h$, $(a<0)$.\\
	Có $B(h;0)\in(P)\Leftrightarrow 0=ah^2+h\Leftrightarrow a=-\dfrac{1}{h}$ (do $h>0$).\\
	Diện tích $S$ của mặt cắt là \[S=\displaystyle\int\limits_{-h}^h\left(-\dfrac{1}{h}{x^2}+h\right)\mathrm{\,d}x=\dfrac{4h^2}{3}, h=3-\dfrac{2}{5}x.\]
	$\Rightarrow S(x)=\dfrac{4}{3}{\left(3-\dfrac{2}{5}x\right)^2}.$\\
	Suy ra thể tích không gian bên trong của đường hầm mô hình
	\[ V=\displaystyle\int\limits_0^5S(x)\mathrm{\,d}x=\displaystyle\int\limits_0^5\dfrac{4}{3}\left(3-\dfrac{2}{5}x\right)^2\mathrm{\,d}x=\dfrac{260}{9}\approx 29\,\left(\text{cm}^3\right).\]
	}
	\end{ex}

% \Closesolutionfile{ansbook}
% \HetDe
% \label{De4}
% %
% \cleardoublepage
% \setcounter{page}{1}
% \rfoot{Trang \thepage/\pageref{DA4} - Đáp án trắc nghiệm Mã đề 4}
% \begin{center}
% 	\bfseries ĐÁP ÁN TRẮC NGHIỆM MÃ ĐỀ 4
% \end{center}

% \inputansbox{10}{ans/ansDe4-TN1}
% \inputansbox[3]{2}{ans/ansDe4-TN2}
% \inputansbox{3}{ans/ansDe4-TN3}
% \label{DA4}
%

% \begin{name}
	{\tenchude}
	{TOÁN 12}
	{LỚP TOÁN THẦY PHÁT}
	{Thời gian: 90 phút - Không kể thời gian phát đề}
\end{name}
\Opensolutionfile{ans}[ans/ansDe1-TN1]
\begin{ex}%[2D4N1-1]%[To 20 - Dot 17 - Chuong 4 - Bai 3 - CD - De 1 - TN]%[Nguyễn Hữu Duy]
Nếu hàm số $f(x)$ liên tục trên đoạn $[a;b]$ và $c$ là số thực tùy ý thuộc đoạn $[a;b]$, thì tính chất nào sau đây đúng?
\choice
{\True $\displaystyle\int_a^b f(x) \mathrm{\,d}x = \displaystyle\int_a^c f(x) \mathrm{\,d}x + \displaystyle\int_c^b f(x)\mathrm{\,d}x$}
{$\displaystyle\int_a^b f(x)\mathrm{\,d}x = \displaystyle\int_a^c f(x) \mathrm{\,d}x - \displaystyle\int_c^b f(x) \mathrm{\,d}x$}
{$\displaystyle\int_a^b f(x)\mathrm{\,d}x = \displaystyle\int_a^c f(x)\mathrm{\,d}y + \displaystyle\int_c^b f(x) \mathrm{\,d}z$}
{$\displaystyle\int_a^b f(x)\mathrm{\,d}x = \displaystyle\int_a^c f(x)\mathrm{\,d}y - \displaystyle\int_c^b f(x)\mathrm{\,d}z$}
\loigiai{
Theo định nghĩa tích phân ta có $\displaystyle\int_a^b f(x) \mathrm{\,d}x = \displaystyle\int_a^c f(x) \mathrm{\,d}x + \displaystyle\int_c^b f(x)\mathrm{\,d}x$.
}
\end{ex}

\begin{ex}%[2D4N1-2]
Tìm họ nguyên hàm của hàm số $f(x)=\dfrac{1}{x}+1$. \choice
{$F(x)=-\dfrac{1}{x^2}+x+C$}
{\True $F(x)=\ln |x|+x+C$}
{$F(x)=\ln x+x+C$}
{$F(x)=\ln |x|+C$}
\loigiai{
Họ nguyên hàm của hàm số $f(x)=\dfrac{1}{x}+1$ là $F(x)=\ln |x|+x+C$.}
\end{ex}

\begin{ex}%[2D4H1-3]
Hàm số $F(x)=x\sin x+\cos x+2024$ là một nguyên hàm của hàm số nào trong các hàm số sau?
\choice
{ $f(x)=x\sin x$ }
{ $f(x)=-x\cos x$ }
{ $f(x)=-x\sin x$ }
{\True $f(x)=x\cos x$ }
\loigiai{
$F'(x)=( x\sin x+\cos x+2024 )' = \sin x + x\cos x - \sin x = x\cos x$, $\forall x\in \mathbb{R}$.\\
$\Rightarrow$ Hàm số $F(x)$ là một nguyên hàm của hàm số $f(x)=x\cos x$ trên $\mathbb{R}$.
}
\end{ex}

\begin{ex}%[2D4H2-1]
Cho hàm số $f(x)$ và $F(x)$ liên tục trên $\mathbb{R}$ thoả mãn $F'(x)=f(x)$ với mọi số thực $x$. Tính   $\displaystyle \int\limits_0^1 f(x)\mathrm{\,d}x$. Biết $F(0)=2; F(1)=5$.
\choice
{\True $3$}
{$4$}
{$5$}
{$6$}
\loigiai{
$\displaystyle \int\limits_0^1 f(x)\mathrm{\,d}x = F(1)-F(0)=5-2=3$.
}
\end{ex}

\begin{ex}%[2D4N3-1]
\immini[thm]{Diện tích hình phẳng giới hạn bởi đồ thị hàm số $ y=f(x)$ và trục hoành (phần gạch chéo trong hình vẽ) là
\choice
{\True $ S=\displaystyle\int\limits_{-2}^0f(x)\mathrm{d}x-\displaystyle\int\limits_0^1f(x)\mathrm{d}x$}
{$ S=\displaystyle\int\limits_{-2}^0f(x)\mathrm{d}x+\displaystyle\int\limits_0^1f(x)\mathrm{d}x$}
{$ S=\displaystyle\int\limits_0^1f(x)\mathrm{d}x-\displaystyle\int\limits_{-2}^0f(x)\mathrm{d}x$}
{$\left|\displaystyle\int\limits_{-2}^1f(x)\mathrm{d}x\right|$}
}
{
\begin{tikzpicture}[scale=0.8,>=stealth, font=\footnotesize, line join=round, line cap=round]
\def\a{1} \def\b{-3} \def\c{0} \def\d{2} % Hệ số
\def\xmin{-3} \def\xmax{2}
\def\ymin{-1} \def\ymax{3}
%	\draw[color=gray!50,dashed] (\xmin,\ymin) grid (\xmax,\ymax);
\draw[->] (\xmin,0)--(\xmax,0) node [below]{$x$};
\draw[->] (0,\ymin)--(0,\ymax) node [left]{$y$};
\node at (0,0) [below left]{$O$};
\clip (\xmin+0.1,\ymin+0.1) rectangle (\xmax-0.5,\ymax-0.1);
\draw[smooth,samples=300] plot(\x,{\a*(\x+2)*(\x)*(\x-1)});

\fill[pattern=north east lines,opacity=0.8] plot[domain=-2:1](\x,{\a*(\x+2)*(\x)*(\x-1)})--cycle;
%	\draw[dashed](-1,0)--(-1,-2)
%	(3,0)--(3,2);
\draw[fill=black](-2,0)node[above left]{$-2$}circle(1pt)
(1,0)node[above]{$1$}circle(1pt)
(1.2,1.5)node[above,rotate=80,scale=0.8]{$y=f(x)$}
;
\end{tikzpicture}
}
\loigiai{
Diện tích hình phẳng giới hạn bởi đồ thị hàm số $ y=f(x)$ và trục hoành (phần gạch chéo trong hình vẽ) là $ S=\displaystyle\int\limits_{-2}^1\left| f(x)\right|\mathrm{d}x=\displaystyle\int\limits_{-2}^0f(x)\mathrm{d}x-\displaystyle\int\limits_0^1f(x)\mathrm{d}x$.}
\end{ex}

\begin{ex}%[2D4H3-3]
Tính thể tích vật thể tạo thành khi quay hình phẳng $(H)$ quanh trục $Ox$, biết $(H)$ được giới hạn bởi các đường $y=4x^2-1$, $y=0$.
\choice
{\True $\dfrac{8\pi}{15}$}
{$\dfrac{4\pi}{15}$}
{$\dfrac{16\pi}{15}$}
{$\dfrac{2\pi}{15}$}
\loigiai{
Phương trình hoành độ giao điểm $4x^2-1=0\Leftrightarrow x=\pm\dfrac{1}{2}$.\\
Suy ra $V=\pi\displaystyle\int\limits_{-\frac{1}{2}}^{\frac{1}{2}}(4x^2-1)^2\mathrm{\,d}x=\pi\displaystyle\int\limits_{-\frac{1}{2}}^{\frac{1}{2}}(16x^4-8x^2+1)\mathrm{\,d}x=\pi\left.\left(\dfrac{16}{5}x^5-\dfrac{8}{3}x^3+x\right)\right|_{-\frac{1}{2}}^{\frac{1}{2}}=\dfrac{8\pi}{15}.$
}
\end{ex}

\begin{ex}%[2H5N1-1]
Trong không gian $Oxyz$, phương trình của mặt phẳng $(Oxy)$ là
\choice
{\True $z=0$}
{$x=0$}
{$y=0$}
{$x+y=0$}
\loigiai{
Phương trình của mặt phẳng $(Oxy)$ là $z=0$.
}
\end{ex}

\begin{ex}%[2H5N1-2]
Trong không gian $Oxyz$, cho mặt phẳng $(P)\colon 2x+y-z+3=0$. Véc-tơ nào sau đây là véc-tơ pháp tuyến của mặt phẳng $(P)$?
\choice
{$\overrightarrow{n}_1=(1;-1;3)$}
{$\overrightarrow{n}_2=(2;-1;3)$}
{\True $\overrightarrow{n}_3=(2;1;-1)$}
{ $\overrightarrow{n}_4=(2;1;3)$}
\loigiai{
Mặt phẳng $Ax+By+Cz+D=0$ nhận véc-tơ $\overrightarrow{n}=(A;B;C)$ làm một véc-tơ pháp tuyến.
}
\end{ex}

\begin{ex}%[2H5H1-3]
Cho hai mặt phẳng $(\alpha)\colon  3 x-2 y+2 z+7=0,$ $(\beta)\colon 5 x-4 y+3 z+1=0$. Phương trình mặt phẳng đi qua gốc tọa độ $O$ đồng thời vuông góc với cả $(\alpha)$ và $(\beta)$ là
\choice
{$2 x-y-2 z=0$}
{$2 x-y+2 z=0$}
{\True $2 x+y-2 z=0$}
{$2 x+y-2 z+1=0$}
\loigiai{
Véc-tơ pháp tuyến của hai mặt phẳng lần lượt là $\overrightarrow{n}_\alpha=(3 ;-2 ; 2), \overrightarrow{n}_\beta=(5 ;-4 ; 3)$.\\
Suy ra $\left[\overrightarrow{n}_\alpha ; \overrightarrow{n}_\beta\right]=(2 ; 1 ;-2)$ là véc-tơ pháp tuyến của mặt phẳng cần tìm.\\
Phương trình mặt phẳng đi qua gốc tọa độ $O, $ có véc-tơ pháp tuyến $\vec{n}=(2 ; 1 ;-2)$ là $2 x+y-2 z=0$.
}
\end{ex}

\begin{ex}%[2H5N2-1]%[Dự án 2025 - Đề cấu trúc mới của Bộ theo [Thành Đức Trung]
Trong không gian $Oxyz$, đường thẳng $d\colon \dfrac{x-1}{2}=\dfrac{y-2}{-1}=\dfrac{z-3}{2}$ đi qua điểm nào dưới đây?
\choice
{$M(-1;-2;-3)$}
{\True $P(1;2;3)$}
{$Q(2;-1;2)$}
{$N(-2;1;-2)$}
\loigiai
{
Vì $\dfrac{-1-1}{2}=\dfrac{2-2}{-1}=\dfrac{3-3}{2}=0$ nên đường thẳng $d$ đi qua điểm $P(1;2;3)$.
}
\end{ex}

\begin{ex}%[2H5N2-7]%[Dự án EX-TF-TLN-2024 Đợt 3-  GV. Đỗ Chí Tâm]
Trong không gian $Oxyz$, góc giữa đường thẳng $d:\dfrac{x-3}{2}=\dfrac{y+1}{1}=\dfrac{z-3}{1}$ và mặt phẳng  $(P):x+2y-z+5=0$ là
\choice
{\True $30^{\circ}$}
{ $45^{\circ}$}
{$60^{\circ}$}
{$90^{\circ}$}
\loigiai{
Gọi $\varphi$ là góc giữa $d$ và $(P)$.\\
Đường thẳng $d$ có véc-tơ chỉ phương $\vec{u}=(2;1;1)$, $(P)$ có véc-tơ pháp tuyến $n=(1;2;-1)$.\\
\[\sin \varphi =\dfrac{|\vec{u}\cdot \vec{n}|}{|\vec{u}|\cdot|\vec{n}|}= \dfrac{\left| 2\cdot 1+1\cdot 2+1\cdot (-1)\right|}{\sqrt{2^2+1^2+1^2} \cdot \sqrt{1^2 +2^2+(-1)^2}}=\dfrac{1}{2} \Rightarrow \varphi = 30^{\circ}.\]
}
\end{ex}

\begin{ex}%[2H5H2-4]
Trong không gian $Oxyz$, cho hai đường thẳng $d_{1}  \colon \dfrac{x-3}{-1} =\dfrac{y-3}{-2} =\dfrac{z+2}{1} $; $d_{2}  \colon \dfrac{x-5}{-3} =\dfrac{y+1}{2} =\dfrac{z-2}{1} $ và mặt phẳng $\left(P\right) \colon x+2y+3z-5=0$. Đường thẳng vuông góc với $\left(P\right)$, cắt $d_{1} $ và $d_{2} $ có phương trình là
\choice
{$\dfrac{x-1}{3} =\dfrac{y+1}{2} =\dfrac{z}{1} $}
{$\dfrac{x-2}{1} =\dfrac{y-3}{2} =\dfrac{z-1}{3} $}
{$\dfrac{x-3}{1} =\dfrac{y-3}{2} =\dfrac{z+2}{3} $}
{\True $\dfrac{x-1}{1} =\dfrac{y+1}{2} =\dfrac{z}{3} $}
\loigiai{
Phương trình $d_{1}  \colon \heva{x&=3-t_{1}  \\ y&=3-2t_{1}  \\ z&=-2+t_{1} } $ và $d_{2}  \colon \heva{x&=5-3t_{2}  \\ y&=-1+2t_{2}  \\ z&=2+t_{2}.} $\\
Gọi đường thẳng cần tìm là $\Delta $. \\
Giả sử đường thẳng $\Delta $ cắt đường thẳng $d_{1} $ và $d_{2} $ lần lượt tại $A$, $B$. \\
Gọi $A\left(3-t_{1} ;3-2t_{1} ;-2+t_{1} \right)$, $B\left(5-3t_{2} ;-1+2t_{2} ;2+t_{2} \right)$.\\ $\overrightarrow{AB}=\left(2-3t_{2} +t_{1} ;-4+2t_{2} +2t_{1} ;4+t_{2} -t_{1} \right).$ \\
véc-tơ pháp tuyến của $\left(P\right)$ là $\vec{n}=\left(1;2;3\right)$. \\
Do $\overrightarrow{AB}$ và $\vec{n}$ cùng phương nên $\dfrac{2-3t_{2} +t_{1} }{1} =\dfrac{-4+2t_{2} +2t_{1} }{2} =\dfrac{4+t_{2} -t_{1} }{3} $\\
$\Leftrightarrow \heva{\dfrac{2-3t_{2} +t_{1} }{1} =\dfrac{-4+2t_{2} +2t_{1} }{2}  \\ \dfrac{-4+2t_{2} +2t_{1} }{2} =\dfrac{4+t_{2} -t_{1} }{3} } $$\Leftrightarrow \heva{t_{1} =2. \\ t_{2} =1} $ Do đó $A\left(1;-1;0\right)$, $B\left(2;-1;3\right)$.\\
Phương trình đường thẳng $\Delta $ đi qua $A\left(1;-1;0\right)$ và có véc-tơ chỉ phương $\vec{n}=\left(1;2;3\right)$ là $\dfrac{x-1}{1} =\dfrac{y+1}{2} =\dfrac{z}{3} .$}
\end{ex}
\Closesolutionfile{ans}

\TNTF
\Opensolutionfile{ans}[ans/ansDe1-TN2]
\begin{ex}%[MĐ2]%[2D4H2-4]
Nếu các số hữu tỉ $a,b$ thỏa mãn $\displaystyle\int \limits_0^1(a\mathrm{e}^x+b)\mathrm{d}x=3\mathrm{e}+4$ thì các phát biểu sau đúng hay sai?
\choiceTF[t]
{$a>b$}
{$a=2b$}
{\True $a=3,\, b=7$}
{\True $2a-b=-1$}
\loigiai{
Ta có $\displaystyle\int \limits_0^1(a\mathrm{e}^x+b)\mathrm{d}x=(a\mathrm{e}^x+bx)\bigg|_0^1=a\mathrm{e}+(-a+b)\Rightarrow \heva{&a=3 \\ &-a+b=4} \Rightarrow \heva{&a=3 \\ &b=7.}$\\
\begin{itemchoice}
\itemch Suy ra $a>b$ là sai.
\itemch Do $a=3, \, b=7$ nên $a=2b$ là sai.
\itemch $a=3,\, b=7$ là khẳng định đúng.
\itemch $2a-b= 2\cdot 3-7=-1$ là khẳng định đúng.
\end{itemchoice}
}
\end{ex}

\begin{ex}%[Dat Thai, Dự án Ex-TF-TLN-2024-Dot03]%[2H5H2-5]
Cho đường thẳng $d$ đi qua điểm $A(0; 0; 1)$ có véc-tơ chỉ phương $\overrightarrow{u} = (1; 1; 3)$ và mặt phẳng $(\alpha) \colon 2x + y - z + 5 = 0$.
\choiceTF
{\True $B(2;2;7)\in d$}
{Phương trình đường thẳng $d$ là $\dfrac{x}{1} = \dfrac{y}{1} = \dfrac{z+1}{3}$}
{$\overrightarrow{n} = (2;1;1)$ là một véc-tơ pháp tuyến của mặt phẳng $(\alpha)$}
{\True Đường thẳng $d$ song song với mặt phẳng $(\alpha)$}
\loigiai{
\begin{itemchoice}
\itemch \textbf{Đúng}. Vì $\overrightarrow{AB} = (2;2;6) = 2\overrightarrow{u}$.
\itemch \textbf{Sai}. Vì phương trình đường thẳng $d$ là $\dfrac{x}{1} = \dfrac{y}{1} = \dfrac{z-1}{3}$.
\itemch \textbf{Sai}. Vì họ véc-tơ pháp tuyến của mặt phẳng $(\alpha)$ có dạng $k\cdot(2;1;-1)$ ($k$ khác $0$), nên nếu giả sử $\overrightarrow{n} = (2;1;1)$ là một véc-tơ pháp tuyến của mặt phẳng $(\alpha)$ thì tồn tại số thực $k$ khác $0$ để
\[
\heva{&2 = 2k\\ & 1 = k\\ & 1=-k} \Leftrightarrow\heva{&k=1\\ &k=-1} (\text{vô lí}).
\]
\itemch \textbf{Đúng}. Vì $\vec{n} = (2;1;-1)$ là một véc-tơ pháp tuyến của mặt phẳng $(\alpha)$ thoả mãn
\[
\vec{n} \cdot \vec{u} = 2\cdot 1 + 1\cdot 1 + (-1)\cdot 3 = 0 \text{ hay } \vec{n}\perp \vec{u}
\]
nên $d\parallel (\alpha)$ hoặc $d\subset(\alpha)$.\\
Kết hợp với $A\notin (\alpha)$ ta thu được $d\parallel (\alpha)$.
\end{itemchoice}
}
\end{ex}
\Closesolutionfile{ans}

\TNSA
\Opensolutionfile{ans}[ans/ansDe1-TN3]
\begin{ex}%[2D4H1-1]%[Đào Trung Kiên]
Biết $ F(x) $ là một nguyên hàm của hàm số $ f(x) = \mathrm{e}^{2x} $ và $ F(0) = 0$. Tính giá trị của $F(\ln 3)$.
\shortans[]{$4$}
\loigiai{
Ta có $ \heva{& F(0) = 0 \\ & F(x) = \dfrac{1}{2} \cdot \mathrm{e}^{2x} + C } \Rightarrow F(x) = \dfrac{1}{2} \cdot \mathrm{e}^{2x} - \dfrac{1}{2} \Rightarrow F(\ln 3) =  \dfrac{1}{2}  \cdot \left  (  \mathrm{e}^{ 2 \cdot \ln 3 } - 1 \right ) = 4$.
}
\end{ex}

\begin{ex}%[2D4V3-1]
	\immini{
	Gọi $H$ là hình phẳng giới hạn bởi đồ thị hàm số $y=-x^2+4x$ và trục hoành. Hai đường thẳng $y=m$ và $y=n$ chia $(H)$ thành ba phần có diện tích bằng nhau (tham khảo hình vẽ). Giá trị của biểu thức $T=(4-m)^3+(4-n)^3$ bằng bao nhiêu? (Kết quả làm tròn đến hàng phần mười)
	}{
	\begin{tikzpicture}[thick,>=stealth,x=1cm,y=1cm,scale=.8]
	\draw[->] (-1,0) -- (5,0) node[below] {\small $x$};
	\draw[->] (0,-1) -- (0,5) node[right] {\small $y$};
	\draw [fill=white,draw=black] (0,0) circle (1pt)node[above left] {\footnotesize $O$};
	\clip(-1,-1) rectangle (5,5);
	\draw[thick,smooth,samples=100,domain=-1:5] plot(\x,{-(\x)^2+4*(\x)});
	\draw (-1,1.2)--(5.1,1.3)node[above left]{$y=n$};
	\draw (-1,2.3)--(5.1,2.4)node[above left]{$y=m$};
	\end{tikzpicture}}
	\shortans[]{$35{,}6$}
	\loigiai{
	\immini{
	Gọi $S$ là diện tích hình phẳng giới hạn bởi đồ thị hàm số $y=-x^2+4x$ và trục $Ox$ và hai đường thẳng $x=0$, $x=2$.\\
	Khi đó $S=\displaystyle\int\limits_{0}^{2} (-x^2+4x)\mathrm{\,d}x =\dfrac{16}{3}$.\\
	Đường thẳng $y=m$ và $y=n$ chia $S$ thành ba phần bằng nhau có diện tích theo thứ tự từ trên xuống là $S_1$; $S_2$; $S_3$.\\
	Gọi hoành độ các giao điểm của parabol với hai đường thẳng như hình bên.\\
	Ta có
	\begin{eqnarray*}
	&& S_1=2\displaystyle\int\limits_{a}^{2} (-x^2+4x-m)\mathrm{\,d}x =\dfrac{1}{3}S\\
	&\Leftrightarrow & \left(-\dfrac{x^3}{3}+2x^2-mx\right)\Big|_{a}^{2}=\dfrac{1}{3}\cdot\dfrac{16}{3}\\
	&\Leftrightarrow & \left(\dfrac{16}{3}-2m\right)-\left(-\dfrac{a^3}{3}+2a^2-ma\right)=\dfrac{16}{9}\quad(1).
	\end{eqnarray*}
	Mà $x=a$ là nghiệm của phương trình $-x^2+4x=m$ nên ta có $-a^2+4a=m\quad(2)$.\\
	Thay $(2)$ vào $(1)$ ta được $-\dfrac{2a^3}{3}+4a^2-8a+\dfrac{32}{9}=0\Leftrightarrow a\approx 0{,}613277$.\\
	Suy ra $m=-a^2+4a\approx 2{,}077$.\\
	Tương tự ta có
	\begin{eqnarray*}
	&& S_1+S_2=\dfrac{2}{3}S\\
	&\Rightarrow & 2\displaystyle\int\limits_{b}^{2} (-x^2+4x-n)\mathrm{\,d}x =\dfrac{2}{3}\cdot 2\cdot\displaystyle\int\limits_{0}^{2} (-x^2+4x)\mathrm{\,d}x\\
	&\Leftrightarrow & -\dfrac{2}{3}b^3+4b^2-8b+\dfrac{16}{9}=0\\
	&\Leftrightarrow & b\approx 0{,}252839\Rightarrow n=-b^2+4b=0{,}947428.
	\end{eqnarray*}
	Khi đó $T=(4-m)^3+(4-n)^3=\dfrac{320}{9}\approx35{,}6$.
	}{
	\begin{tikzpicture}[thick,>=stealth,x=1cm,y=1cm,scale=.8]
	\draw[->] (-1,0) -- (5,0) node[below] {\small $x$};
	\draw[->] (0,-1) -- (0,5) node[right] {\small $y$};
	\draw [fill=white,draw=black] (0,0) circle (1pt)node[above left] {\footnotesize $O$};
	\clip(-1,-1) rectangle (5,5);
	\draw[thick,smooth,samples=100,domain=-1:5] plot(\x,{-(\x)^2+4*(\x)});
	\draw (-1,1.2)--(5,1.2)node[above left]{$y=n$};
	\draw (-1,2.3)--(5,2.3)node[above left]{$y=m$};
	\draw[dashed](0.33,0)node[below]{$b$}--(0.33,1.2) (0.7,0)node[below]{$a$}--(0.7,2.3) (2,0)node[below]{$2$}--(2,4);
	\draw[dashed] (0,4) node[below left]{$4$}--(2,4);
	\end{tikzpicture}
	}
	}
	\end{ex}

\begin{ex}%[2H5H2-7]
Gọi $\varphi$ là góc giữa hai đường thẳng $d_1 \colon \dfrac{x-1}{-2}= \dfrac{y+2}{1}= \dfrac{z-3}{2}$ và $d_2 \colon \dfrac{x+3}{1}= \dfrac{y-1}{1}= \dfrac{z+2}{-4}$. Tính $\cos \varphi$ (làm tròn đến hàng phần trăm).
\shortans{$0{,}71$}
\loigiai
{
Đường thẳng $ d_1 $ có một véc-tơ chỉ phương $ \vec u_1 =(-2;1;2)$.\\
Đường thẳng $ d_2 $ có một véc-tơ chỉ phương $ \vec u_2 =(1;1;-4)$.\\
Ta có
\begin{eqnarray*}
\cos \varphi
&=&\left| \cos \left( \vec u_1, \vec u_2 \right)\right| = \dfrac{\left|  \vec u_1 \cdot  \vec u_2 \right|}{\left| { \vec u_1} \right| \cdot \left| { \vec u_2} \right|}\\
&=&\dfrac{|-2 \cdot 1+ 1\cdot 1+ 2\cdot (-4)|}{\sqrt{(-2)^2+1^2+2^2} \cdot \sqrt{1^2+1^2+(-4)^2}}\\
&=& \dfrac{\sqrt {2}}{2} \approx 0{,}71.
\end{eqnarray*}
}
\end{ex}

\begin{ex}%[2H5V1-7]
Một công trình đang xây dựng được gắn hệ trục $Oxyz$ (đơn vị trên mỗi trục tọa độ là mét). Ba bức tường $(P),(Q),(R)$ (như hình vẽ) của tòa nhà lần lượt có phương trình $(P)\colon 2x-y-z+1=0$, $(Q)\colon x+3y-z-2=0,(R)\colon 4x-2y-2z+9=0$. Tính chiều rộng bức tường $(Q)$ của tòa nhà. (Kết quả làm tròn đến hàng phần chục).
\begin{center}
\includegraphics[width=0.7\textwidth]{images/C5B1CD3-H3.png}
\end{center}
\shortans{$2{,}9$}
\loigiai{
\begin{itemize}
\item Kiểm tính song song hoặc vuông góc giữa các bức tường $(P),(Q),(R)$ của tòa nhà.\\
Ta có $(P)$ có vectơ pháp tuyến là $\vec{n}_P=(2;-1;-1)$, $(Q)$ có vectơ pháp tuyến là $\vec{n}_Q=(1; 3;-1)$, $(R)$ có vectơ pháp tuyến là $\vec{n}_R=(4;-2;-2)$.\\
Khi đó $\vec{n}_R=(4;-2;-2)=2(2;-1;-1) \Rightarrow \vec{n}_R=2\vec{n}_P$ nên hai bức tường $(P)$ và $(R)$ song song nhau.\\
Mặt khác $\vec{n}_P \cdot \vec{n}_Q=2\cdot 1+(-1) \cdot 3+(-1) \cdot(-1)=0\Rightarrow \vec{n}_P \perp \vec{n}_Q$ nên bức tường $(Q)$ vuông góc với hai bức tường $(P)$ và $(R)$.
\item Tính chiều rộng bức tường $(Q)$ của tòa nhà.\\
Do hai bức tường $(P)$ và $(R)$ song song nhau nên chiều rộng bức tường $(Q)$ là khoảng cách giữa hai bức tường $(P)$ và $(R)$.\\
Chọn điểm $N(0; 0; 1) \in(P)$. Do hai bức tường $(P)$ và $(R)$ song song nhau nên
\[\mathrm{d}((P),(R))=\mathrm{d}(N,(R))=\dfrac{|4\cdot 0-2\cdot 0-2\cdot 1+9|}{\sqrt{4+1+1}}=\dfrac{7}{\sqrt{6}} \approx 2{,}9.\]
\end{itemize}
}
\end{ex}

\TL
\begin{ex}%[2H5H2-3]%[Dự án 2025 - Đề cấu trúc mới của Bộ theo [Thành Đức Trung]
Trong không gian $Oxyz$, cho tam giác $ABC$ có $A(0;0;1)$, $B(-3;2;0)$, $C(2;-2;3)$. Viết phương trình tham số đường cao kẻ từ $B$ của tam giác $ABC$.
% \shortans{$-2$}
\loigiai
{
Gọi $\Delta$ là đường cao kẻ từ $B$ của tam giác $ABC$.\\
Ta có $\heva{& \overrightarrow{AB}=(-3;2;-1) \\ & \overrightarrow{AC}=(2;-2;2)} \Rightarrow \left[\overrightarrow{AB},\overrightarrow{AC}\right]=(2;4;2)$. \\
Suy ra một véc-tơ pháp tuyến của mặt phẳng $(ABC)$ là $\overrightarrow{n}=(1;2;1)$.\\
Ta có $\heva{ & \Delta \subset(ABC) \\ & \Delta \perp AC}$, suy ra đường thẳng $\Delta$ nhận $\left[\overrightarrow{n},\overrightarrow{AC}\right]$ làm một véc-tơ chỉ phương.\\
Có $\left[\overrightarrow{n},\overrightarrow{AC}\right]=(6;0;-6)=6\overrightarrow{u}$ với $\overrightarrow{u}=(1;0;-1)$. \\
Suy ra đường thẳng $\Delta$ nhận $\overrightarrow{u}=(1;0;-1)$ làm véc-tơ chỉ phương.\\
Do đó phương trình đường thẳng $\Delta$ là $\Delta \colon \heva{ & x=-3+t \\ & y=2 \\ & z=-t}$.
}
\end{ex}

\begin{ex}%[12-MH-2-MH2025]%[MH-2025, Nguyễn Trần Phong]%[2D4C3-2]
	\immini{Chướng ngại vật \lq\lq  tường cong\rq\rq trong một sân thi đấu X-Game là một khối bê tông có chiều cao từ mặt đất lên là $3$ m. Giao của mặt tường cong và mặt đất là đoạn thẳng $AB = 2$ m. Thiết diện của khối tường cong cắt bởi mặt phẳng vuông góc với $AB$ tại $A$ là một hình tam giác vuông cong $ACE$ với $AC = 4$ m, $CE = 3$ m và cạnh cong $AE$ nằm trên một đường Parabol có trục đối xứng vuông góc với mặt đất. Tại vị trí $M$ là trung điểm của $AC$ thì tường cong có độ cao $1$ m. Thể tích bê tông cần sử dụng để tạo nên khối tường cong đó gần nhất với số nào dưới đây?
	\shortans{$9{,}3$}
	}{\begin{tikzpicture}[>=stealth,x=0.8cm,y=0.8cm,scale=0.7]
	\coordinate[label=below:$A$] (A) at (0,0);
	\coordinate[label=left:$B$] (B) at (-2,2);
	\coordinate[label=below:$C$] (C) at (6,0);
	\coordinate[label=right:$E$] (E) at (6,6);
	\coordinate (G) at (2,4);
	\coordinate (H) at (4,1.8);
	\coordinate (D) at ($(C)+(B)-(A)$);
	\coordinate (F) at ($(E)+(D)-(C)$);
	\coordinate[label=below:$M$] (M) at ($(A)!0.5!(C)$);
	\coordinate (K) at ($(A)!0.5!(B)$);
	\coordinate (N) at ($(M)+(0,1.1)$);
	\draw (D)--(C)--(A)--(B) (C)--(E)--(F) (M)--(N);
	\draw[dashed] (B)--(D)--(F);
	\foreach \diem in {A,B,C,D,E,F,M,F}	\fill (\diem)circle(1.5pt);
	%\tkzLabelPoints[above left](D)
	%\tkzLabelSegment[right](M,N){\footnotesize$1$ m}
	%\tkzLabelSegment[left](A,B){\footnotesize$2$ m}
	%\tkzLabelSegment[right](C,E){\footnotesize$3{,}5$ m}
	\draw(-1,.8) node[left]{\footnotesize $2$ m} (3,0.8) node[right]{\footnotesize$1$ m} (6,3) node[right]{\footnotesize$3$ m};
	
	\draw plot[smooth,tension=.65] coordinates{(B) (G) (F)};
	\draw plot[smooth,tension=.65] coordinates{(A) (H) (E)};
	\fill [pattern = north east lines] plot[smooth,tension=.65] coordinates{(A) (H) (E)} (0,0) --(-2,2)--(4,8)--(6,6)--cycle;
	\fill [draw=none, pattern = north east lines, color=white] (0,0) plot[smooth,tension=.65] coordinates{(B) (G) (F)} (-2,2)--(4,2)--cycle;
	\end{tikzpicture}
	}
	\loigiai{
	\immini{Chọn hệ trục tọa độ như hình vẽ.\\
	Gọi $AE \colon y = ax^2 + bx + c$.\\
	Do $AE$ đi qua $A(-4; 0)$ nên ta có $16a - 4b + c = 0$.\\
	Do $E (0; 3)$ thuộc cạnh cong $AE$ nên $c = 3$ (2).\\
	Do $N(-2; 1)$ thuộc cạnh cong $AE$ nên $4a - 2b + c = 1$ (3).\\
	Từ (1), (2), (3) suy ra $a = \dfrac{1}{8}$, $b = \dfrac{5}{4}$, $c = 3 \Rightarrow AE \colon y = \dfrac{1}{8}x^2 + \dfrac{5}{4}x + 3$.\\
	Khi đó $S_{AEC} = \displaystyle\int_{-4}^0\left(\dfrac{1}{8} x^2 + \dfrac{5}{4}x + 3\right) dx = \dfrac{14}{3}\left(m^2\right)$.
	}{\begin{tikzpicture}[>=stealth,x=0.8cm,y=0.8cm,scale=0.7]
	\coordinate[label=below:$A$] (A) at (0,0);
	\coordinate[label=left:$B$] (B) at (-2,2);
	\coordinate[label=below:$C$] (C) at (6,0);
	\coordinate[label=right:$E$] (E) at (6,6);
	\coordinate (G) at (2,4);
	\coordinate (H) at (4,1.8);
	\coordinate[label = above left:$D$] (D) at ($(C)+(B)-(A)$);
	\coordinate[label = above:$F$] (F) at ($(E)+(D)-(C)$);
	\coordinate[label=below:$M$] (M) at ($(A)!0.5!(C)$);
	\coordinate (K) at ($(A)!0.5!(B)$);
	\coordinate[label=above left:$N$] (N) at ($(M)+(0,1.1)$);
	\draw (D)--(C)--(A)--(B) (C)--(E)--(F) (M)--(N);
	\draw[dashed] (B)--(D) (D)--(F);
	\foreach \diem in {A,B,C,D,E,F,M,F,N}	\fill (\diem)circle(1.5pt);
	\coordinate (x) at ($(M)!1.5!(C)$);
	\draw[->](C)--(x); \draw (x) node[right]{$x$};
	\coordinate (y) at ($(C)!1.4!(E)$);
	\draw[->](E)--(y); \draw (y) node[right]{$y$};
	%\tkzLabelPoints[above left](D)
	%\tkzLabelSegment[right](M,N){\footnotesize$1$ m}
	%\tkzLabelSegment[left](A,B){\footnotesize$2$ m}
	%\tkzLabelSegment[right](C,E){\footnotesize$3{,}5$ m}
	\draw plot[smooth,tension=.65] coordinates{(B) (G) (F)};
	\draw plot[smooth,tension=.65] coordinates{(A) (H) (E)};
	\fill [pattern = north east lines] plot[smooth,tension=.65] coordinates{(A) (H) (E)} (0,0) --(-2,2)--(4,8)--(6,6)--cycle;
	\fill [draw=none, pattern = north east lines, color=white] (0,0) plot[smooth,tension=.65] coordinates{(B) (G) (F)} (-2,2)--(4,2)--cycle;
	\end{tikzpicture}
	}
	\noindent Thể tích khối tường cong là $V = S_{AEC} \cdot AB = \frac{14}{3} \cdot 2 = \dfrac{28}{3} = 9{,}3\left(\mathrm{~m}^3\right)$.	}
	\end{ex}

\begin{ex}%[2H5C1-7]
Người ta thiết kế một mái che hình chữ nhật $ ABCD $ phía trên sân khấu. Gắn hệ trục tọa độ $ Oxyz $ (đơn vị trên trục là mét), người ta xác định được toạ dộ của các điểm như sau: $ A(0;0;8)$, $B(0;20;8)$, $D(15;0;14)$, $C(15;20;14) $. Một cổng chào hình chữ nhật $ EFHG $ với tọa độ điểm $ G(8;0;4) $ dựng vuông góc với mặt đất. Người ta muốn làm các đoạn dây nối thanh ngang $ GE $ với mái che để gắn hoa và đèn led. Độ dài ngắn nhất của mỗi đoạn dây này bằng bao nhiêu mét? (làm tròn đến chữ số thập phân thứ nhất)
\begin{center}
\includegraphics[scale=.3]{images/2P5-1-H5-16}
\end{center}
\shortans{$1{,}8$}
\loigiai{
Ta có $ A(0;0;8)$, $B(0;20;8)$, $D(15;0;14)$, $C(15;20;14) $.\\
Ta có $ \vec{AB}=(0;20;0) $, $\vec{AC}=(15;20;6)$ nên $ \vec{n}_1=\left[\vec{AB},\vec{AC}\right]=(80;0;300) $ là vectơ pháp tuyến của $ (ABCD) $.\\
Mà mặt phẳng mái che $ (ABCD) $ qua $ A(0;0;8)$ nên có phương trình
\[ 80(x-0)+0(y-0)+300(z-8)=0\Leftrightarrow 4x+15z-120=0 .\]
Độ dài ngắn nhất của dây nối thanh ngang $ GE $ với mái che là khoảng cách từ $ G $ đến mái che (mặt phẳng $ ABCD $) là \[ \mathrm{d}(G,(ABCD))=\dfrac{|4\cdot8+0+15\cdot4-120|}{\sqrt{4^2+0^2+15^2}}=\dfrac{28}{\sqrt{241}}=1{,}8\ (\text{m}). \]
}
\end{ex}
\Closesolutionfile{ans}


% \Closesolutionfile{ansbook}
% \HetDe
% \label{De1}
% %
% \cleardoublepage
% \setcounter{page}{1}
% \rfoot{Trang \thepage/\pageref{DA1} - Đáp án trắc nghiệm Mã đề 1}
% \begin{center}
% 	\bfseries ĐÁP ÁN TRẮC NGHIỆM MÃ ĐỀ 1
% \end{center}

% \inputansbox{10}{ans/ansDe1-TN1}
% \inputansbox[3]{2}{ans/ansDe1-TN2}
% \inputansbox{3}{ans/ansDe1-TN3}
% \label{DA1}
% %

% \begin{name}
	{\tenchude}
	{TOÁN 12}
	{LỚP TOÁN THẦY PHÁT}
	{Thời gian: 90 phút - Không kể thời gian phát đề}
\end{name}
\Opensolutionfile{ans}[ans/ansDe2-TN1]
\begin{ex}%[2D4N1-1]%[To 20 - Dot 17 - Chuong 4 - Bai 3 - CD - De 1 - TN]%[Nguyễn Hữu Duy]
Cho hàm số $f(x)$ liên tục trên đoạn $[a;c]$ và $b$ là số thực tùy ý thuộc đoạn $[a;c]$. Nếu biết $\displaystyle\int_a^b f(x) \mathrm{\,d}x = 3$ và $\displaystyle\int_b^c f(x) \mathrm{\,d}x = 8$, thì giá trị của $\displaystyle\int_a^c f(x) \mathrm{\,d}x$ là bao nhiêu?
\choice
{\True $11$}
{$-5$}
{$5$}
{$-11$}
\loigiai
{
Ta có $\displaystyle\int_a^c f(x) \mathrm{\,d}x = \displaystyle\int_a^b f(x) \mathrm{\,d}x + \displaystyle\int_b^c f(x) \mathrm{\,d}x = 3 + 8 = 11$.
}
\end{ex}

\begin{ex}%[12-MH-1-MH2025]%[MH-2025, Mã/Tên TV biên soạn]%[2D4N1-2]
Cho hàm số $y = f(x)$ là một nguyên hàm của hàm số $y = x^3$. Phát biểu nào sau đây là đúng?
\choice
{$f(x) = \dfrac{x^4}{4} + C$}
{$f(x) = 3x^2$}
{$f(x) = 4x^3$}
{\True $f(x) = \dfrac{x^4}{4}$}
\loigiai{Hàm số $f(x) = \dfrac{x^4}{4}$ là một nguyên hàm của hàm số $y = x^3$ vì $f'(x) = x^3$.
}
\end{ex}

\begin{ex}%[2D4H1-3]
Tìm nguyên hàm $\displaystyle\int \dfrac{\cos 2x}{\sin^2 x \cos^2 x} \mathrm{d}x$.
\choice
{ $F(x) = -\cos x - \sin x + C$ }
{ $F(x) = \cos x + \sin x + C$ }
{ $F(x) = \cot x - \tan x + C$ }
{\True $F(x) = -\cot x - \tan x + C$ }
\loigiai{
Ta có: $\displaystyle\int \dfrac{\cos 2x}{\sin^2 x \cos^2 x} \mathrm{d}x = \displaystyle\int \left( \dfrac{1}{\sin^2 x} - \dfrac{1}{\cos^2 x} \right) \mathrm{d}x = -\cot x - \tan x + C$.
}
\end{ex}

\begin{ex}%[2D4H2-1]
Tích phân $\displaystyle\int\limits_{1}^{3}\left[2f(x)+1\right]\mathrm{\,d}x=5$ thì $\displaystyle\int\limits_{1}^{3} f(x)\mathrm{\,d}x$ bằng
\choice
{$3$}
{$2$}
{$\dfrac{3}{4}$}
{\True$\dfrac{3}{2}$}
\loigiai{
Ta có
\begin{eqnarray*}
\displaystyle\int\limits_{1}^{3} \left[2f(x)+1\right]\mathrm{\,d}x=5
&\Leftrightarrow&2\displaystyle\int\limits_{1}^{3} f(x)\mathrm{\,d}x+\displaystyle \int_{1}^{3} \mathrm{\,d}x =5\\
&\Leftrightarrow& 2\displaystyle\int\limits_{1}^{3} f(x)\mathrm{\,d}x+\displaystyle 2 =5 \Leftrightarrow \displaystyle\int\limits_{1}^{3} f(x)\mathrm{\,d}x =\dfrac{3}{2}.
\end{eqnarray*}
}
\end{ex}

\begin{ex}%[2D4N3-1]
\immini{Diện tích phần hình phẳng gạch chéo trong hình vẽ bên được tính theo công thức nào?
\choice
{$\displaystyle\int\limits_{-5}^{-3}\left(x+5\right)\mathrm{d}x-\displaystyle\int\limits_{-3}^1\sqrt{1-x}\mathrm{d}x$}
{\True $\displaystyle\int\limits_{-5}^{-3}\left(x+5\right)\mathrm{d}x+\displaystyle\int\limits_{-3}^1\sqrt{1-x}\mathrm{d}x$}
{$\displaystyle\int\limits_{-5}^1\left[\left(x+5\right)-\sqrt{1-x}\right]\mathrm{d}x$}
{$\displaystyle\int\limits_{-5}^1\left[\sqrt{1-x}-\left(x+5\right)\right]\mathrm{d}x$}
}
{
\begin{tikzpicture}[scale=0.8,>=stealth, font=\footnotesize, line join=round, line cap=round]
\def\a{1} \def\b{-3} \def\c{0} \def\d{2} % Hệ số
\def\xmin{-6} \def\xmax{2}
\def\ymin{-1} \def\ymax{5.5}
%	\draw[color=gray!50,dashed] (\xmin,\ymin) grid (\xmax,\ymax);
\draw[->] (\xmin,0)--(\xmax,0) node [below]{$x$};
\draw[->] (0,\ymin)--(0,\ymax) node [left]{$y$};
\node at (0,0) [below left]{$O$};
\clip (\xmin+0.1,\ymin+0.1) rectangle (\xmax-0.5,\ymax-0.1);
\draw[smooth,samples=300,domain=-5.5:1] plot(\x,{sqrt(1-\x)});
\draw[smooth,samples=300] plot(\x,{1*(\x)+5});
\fill[pattern=north east lines,opacity=0.8] (-5,0)--plot[domain=-5:-3](\x,{1*(\x)+5})--plot[domain=-3:1](\x,{sqrt(1-\x)})--(1,0);
\draw[dashed](-3,0)--(-3,2)
;
\draw[fill=black](-3,0)node[below]{$-3$}circle(1pt)
(1,0)node[below]{$1$}circle(1pt)
(-5,0)node[below]{$-5$}circle(1pt)
(-2,3)node[above,rotate=45,scale=0.8]{$y=x+5$}
(-1.5,1.5)node[above,rotate=-20,scale=0.8]{$y=\sqrt{1-x}$}
;
\end{tikzpicture}
}
\loigiai{
Ta chia hình phẳng gạch chéo làm $2$ phần. Nên diện tích hình phẳng là \[ S=\displaystyle\int\limits_{-5}^{-3}\left(x+5\right)\mathrm{d}x+\displaystyle\int\limits_{-3}^1\sqrt{1-x}\mathrm{d}x.\]}
\end{ex}

\begin{ex}%[2D4H3-3]
\immini[thm]{Cho tam giác $OAB$ vuông tại $A$, có cạnh $OA=a$ nằm trên tục $Ox$ và $\widehat{AOB}=\dfrac{\pi}{3}$.
Gọi $\beta$ là khối tròn xoay sinh ra khi quay miền tam giác $OAB$ xung quanh trục $Ox$. Thể tích của khối $\beta$ bằng
\choice
{$3\pi a^3$}
{\True $\pi a^3$}
{$\dfrac{\pi a^3}{3}$}
{$\dfrac{\pi a^3}{9}$}
}{
\begin{tikzpicture}[scale=0.7, font=\footnotesize, line join=round, line cap=round, >=stealth]
\fill[yellow!30]	(0,0)--(4,2)--(4,0)--cycle;
\draw[->] (4,0) -- (5.5,0) node[right] {$x$};
\draw[->] (0,-2.5) -- (0,2.5) node[above] {$y$};
\draw[->] (0,0) -- (-1.5,-1.5) node[below left] {$z$};
\draw (0,0)--(4,2)	(0,0)--(4,-2)	(4,2)--(4,0)	(-0.5,0)--(0,0);
\draw[dashed] (0,0)--(4,0)	;
\draw (4,0) ellipse (0.5 and 2);
\fill (0,0) circle (1pt) node[above left]{$O$};
\fill (4,0) circle (1pt) node[below]{$A$};
\fill (4,2) circle (1pt) node[above]{$B$};
\clip (4,0) -- (0,0) -- (4,2);
\draw (0,0) circle (1cm);
\draw ($(0,0)+(1,0)$) node[above right]{$\alpha$};
%\node[below] at (2.5,-2.5) {Hình $4.31$};
\end{tikzpicture}
}
\loigiai{
Do $OB$ đi qua gốc tọa độ và tạo với $Ox$ một góc $\dfrac{\pi}{3}$ nên $OB\colon y=\tan \dfrac{\pi}{3}x=\sqrt{3} x$.\\
Khi đó, thể tích của khối $\beta$ là
\[
V=\pi \displaystyle\int_0^a\left(\sqrt{3} x\right)^2\mathrm{d}x=\pi \displaystyle\int_0^a3x^2\mathrm{d}x= \pi x^3\bigg|_0^a=\pi a^3.
\]
}
\end{ex}

\begin{ex}%[2H5N1-1]
Trong không gian với hệ toạ độ $Oxyz$, phương trình nào dưới đây là phương trình của mặt phẳng $(Oyz)$?
\choice
{$y=0$}
{\True $x=0$}
{$y-z=0$}
{$z=0$}
\loigiai{
Mặt phẳng $(Oyz)$ đi qua điểm $O(0 ; 0 ; 0)$ và có véc-tơ  pháp tuyến là $\vec{i}=(1 ; 0 ; 0)$ nên ta có phương trình mặt phẳng $(O y z)$ là  $1(x-0)+0(y-0)+0(z-0)=0 \Leftrightarrow x=0$.
}
\end{ex}

\begin{ex}%[2H5N1-2]
Trong không gian $Oxyz$, mặt phẳng nào sau đây nhận véc-tơ $\overrightarrow{n}=(1;2;3)$ làm véc-tơ pháp tuyến?
\choice
{\True $2x+4y+6z=1$}
{$x-2y+3z+1=0$}
{$x+2y-3z-1=0$}
{$2x-4z+6=0$}
\loigiai{
Mặt phẳng $2x+4y+6z=1$ có một véc-tơ pháp tuyến là $\overrightarrow{m}=(2;4;6)$, nên $\overrightarrow{n}=\dfrac{1}{2}\overrightarrow{m}$ cũng là véc-tơ pháp tuyến của mặt phẳng $2x+4y+6z=1$.
}
\end{ex}

\begin{ex}%[2H5H1-3]
Trong không gian với hệ tọa độ $O x y z$, cho điểm $A(2 ; 4 ; 1) ;$ $ B(-1 ; 1 ; 3)$ và mặt phẳng $(P)\colon x-3 y+2 z-5=0$. Một mặt phẳng $(Q)$ đi qua hai điểm $A, B$ và vuông góc với mặt phẳng $(P)$ có dạng $a x+b y+c z-11=0$. Khẳng định nào sau đây là đúng?
\choice
{\True $a+b+c=5$}
{$a+b+c=15$}
{$a+b+c=-5$}
{$a+b+c=-15$}
\loigiai{Vì $(Q)$ vuông góc với $(P)$ nên $(Q)$ nhận véc-tơ pháp tuyến $\vec{n}=(1 ;-3 ; 2)$ của $(P)$ làm véc-tơ chỉ phương.\\
Mặt khác $(Q)$ đi qua $A$ và $B$ nên $(Q)$ nhận $\overrightarrow{A B}=(-3 ;-3 ; 2)$ làm véc-tơ chỉ phương.\\
$(Q)$ nhận $\overrightarrow{n}_Q=[\vec{n}, \overrightarrow{A B}]=(0 ; 8 ; 12)$ làm véc-tơ pháp tuyến.\\
Vậy phương trình mặt phẳng $(Q)\colon  0(x+1)+8(y-1)+12(z-3)=0\Leftrightarrow 2 y+3 z-11=0$.\\
Vậy $a+b+c=5$.}
\end{ex}

\begin{ex}%[2H5N2-1]%[Dự án 2025 - Đề cấu trúc mới của Bộ theo [Thành Đức Trung]
Trong không gian $Oxyz$, đường thẳng $d\colon \heva{ & x=1+2t \\ & y=3-t \\ & z=1-t}$ $(t\in \mathbb{R})$ đi qua điểm nào dưới đây?
\choice
{$M(1;3;-1)$}
{\True $N(-3;5;3)$}
{$P(3;5;3)$}
{$Q(1;2;-3)$}
\loigiai
{
Thay tọa độ các điểm vào phương trình $d$, ta có $\heva{ & -3=1+2t \\ &5=3-t \\ & 3=1-t} \Leftrightarrow t=-2$. \\
Vậy $N(-3;5;3)\in d$.
}
\end{ex}

\begin{ex}%[2H5N2-7]
Trong không gian với hệ trục $Oxyz$, cho hai đường thẳng $d_1:\dfrac{x}{1}=\dfrac{y+1}{-1}=\dfrac{z-1}{2}$ và $d_2:\dfrac{x+1}{-1}=\dfrac{y}{1}=\dfrac{z-3}{1}$. Góc giữa hai đường thẳng đó bằng
\choice
{$45^\circ $}
{\True $90^\circ $}
{$60^\circ $}
{$30^\circ $}
\loigiai{
Đường thẳng $d_1$ có véctơ chỉ phương $\overrightarrow{u}_1=\left( 1;-1;2 \right)$.\\
Đường thẳng $d_2$ có véctơ chỉ phương $\overrightarrow{u}_2=\left( -1;1;1 \right)$.\\
Gọi $\alpha$ là góc giữa hai đường thẳng trên.\\
Khi đó ta có $\cos \alpha =\left| \cos \left( \overrightarrow{u}_1,\overrightarrow{u}_2 \right) \right|=\dfrac{\left| 1\cdot\left( -1 \right)+\left( -1 \right)\cdot1+2\cdot1 \right|}{\sqrt{1^2+{{\left( -1 \right)}^2}+2^2}\cdot\sqrt{{{\left( -1 \right)}^2}+1^2+1^2}}=0$.\\$\Rightarrow \left( \widehat{d_1,d_2} \right)=90^\circ $.}
\end{ex}

\begin{ex}%[2H5H2-4]
Trong không gian $Oxyz$ cho đường thẳng $\Delta  \colon \dfrac{x}{1} =\dfrac{y+1}{2} =\dfrac{z-1}{1} $ và mặt phẳng $\left(P\right) \colon x-2y-z+3=0$. Đường thẳng nằm trong $\left(P\right)$ đồng thời cắt và vuông góc với $\Delta $ có phương trình là
\choice
{$\heva{x&=1+2t \\ y&=1-t \\ z&=2} $}
{$\heva{x&=-3 \\ y&=-t \\ z&=2t} $}
{$\heva{x&=1+t \\ y&=1-2t \\ z&=2+3t} $}
{\True $\heva{x&=1 \\ y&=1-t \\ z&=2+2t} $}
\loigiai{
Ta có $\Delta  \colon \dfrac{x}{1} =\dfrac{y+1}{2} =\dfrac{z-1}{1} $$\Rightarrow \Delta  \colon \heva{x&=t \\ y&=-1+2t \\ z&=1+t.} $ \\
Gọi $M=\Delta \cap \left(P\right)$ $\Rightarrow M\in \Delta \Rightarrow M\left(t;2t-1;t+1\right)$ $M\in \left(P\right)\Rightarrow t-2\left(2t-1\right)-\left(t+1\right)+3=0 \Leftrightarrow 4-4t=0\Leftrightarrow t=1\Rightarrow M\left(1;1;2\right)$. \\
Véc-tơ pháp tuyến của mặt phẳng $\left(P\right)$ là $\overrightarrow{n}=\left(1;-2;-1\right)$. \\
Véc-tơ chỉ phương của đường thẳng $\Delta $ là $\overrightarrow{u}=\left(1;2;1\right)$.\\
Đường thẳng $d$ nằm trong mặt phẳng $\left(P\right)$ đồng thời cắt và vuông góc với $\Delta $. \\
$\Rightarrow $ đường thẳng $d$ nhận $\dfrac{1}{2} \left[\overrightarrow{n},\overrightarrow{u}\right]=\left(0;-1;2\right)$ làm véc-tơ chỉ phương và $M\left(1;1;2\right)\in d$.\\
$\Rightarrow $ Phương trình đường thẳng $d \colon \heva{x&=1 \\ y&=1-t \\ z&=2+2t.}$}
\end{ex}
\Closesolutionfile{ans}

\TNTF
\Opensolutionfile{ans}[ans/ansDe2-TN2]
\begin{ex}%[2D4H2-4]%[Tổ 20 - Đợt 17 - Chương 4 - - CD - Đề 6]%[Nắng Đông]
Cho $A=\displaystyle\int\limits_0^1 \dfrac{3}{2^x} \mathrm{\,d}x$ và  $B=\displaystyle\int\limits_0^1 4\mathrm{e}^{-2x} \mathrm{\,d}x$.
\choiceTF[t]
{\True $A=-\dfrac{3}{2^x\ln 2}\bigg|_0^1$}
{$B=\dfrac{2}{\mathrm{e}^{2x}}\bigg|_0^1$}
{$A=-\dfrac{3}{2\ln 2}$}
{\True $B=a+\dfrac{b}{\mathrm{e}^2}$, với $a$, $b$ là các số nguyên thì $a\cdot b = -4$}
\loigiai
{Ta có $A=\displaystyle\int\limits_0^1 \dfrac{3}{2^x} \mathrm{\,d}x
= \displaystyle\int\limits_0^1 {3\cdot \left(\dfrac{1}{2}\right)^x} \mathrm{\,d}x
= \dfrac{3}{2^x\ln \dfrac{1}{2}}\bigg|_0^1
=-\dfrac{3}{2^x\ln 2}\bigg|_0^1
= \dfrac{3}{2\ln 2}$.\\
Và $B=\displaystyle\int\limits_0^1 4\mathrm{e}^{-2x} \mathrm{\,d}x
=\displaystyle\int\limits_0^1 4\left(\mathrm{e}^{-2}\right)^x \mathrm{\,d}x
= 4\cdot \dfrac{\left(\mathrm{e}^{-2}\right)^x}{\ln \mathrm{e}^{-2}}\bigg|_0^1
= -\dfrac{2}{\mathrm{e}^{2x}}\bigg|_0^1
=2 - \dfrac{2}{\mathrm{e}^2}$.
\begin{itemchoice}
\itemch Đúng.
\itemch Sai.
\itemch Sai.
\itemch Đúng. Vì $a=2$, $b=-2$ suy ra $a\cdot b = -4$.
\end{itemchoice}
}
\end{ex}

\begin{ex}%[Dat Thai, Dự án Ex-TF-TLN-2024-Dot03]%[2H5H2-5]
Cho đường thẳng $d\colon \heva{& x = 1 + t\\& y = 2 - t\\& z = 1 + 2t}, t\in \mathbb{R}$ và mặt phẳng $(P)\colon x + 2y + z - 5 = 0$. Tọa độ giao điểm $A$ của đường thẳng $d$ và mặt phẳng $(P)$ là $(a;b;c)$.
\choiceTF
{\True $a+b+c = 2$}
{\True Có đúng $1$ số dương trong ba số $a$, $b$, $c$}
{$a$ là số lớn nhất trong ba số $a$, $b$, $c$}
{$a$, $b$, $c$ theo thứ tựu lập thành một cấp số cộng}
\loigiai{
Tọa độ giao điểm $A$ của đường thẳng $d$ và mặt phẳng $P$ là nghiệm của hệ phương trình sau
\begin{eqnarray*}
\heva{& x = 1 + t\\& y = 2 - t\\ & z = 1 + 2t \\& z + 2y + z -5 = 0} \Leftrightarrow \heva{& t  =-1\\& x = 0\\& y = 3 \\& z = -1} \Rightarrow A(0; 3; -1).
\end{eqnarray*}
Vậy
\begin{itemchoice}
\itemch \textbf{Đúng}.
\itemch \textbf{Đúng}. Vì chỉ có mỗi $b$ là số dương trong ba số $a$, $b$, $c$.
\itemch \textbf{Sai}. Vì $b$ là số lớn nhất trong ba số $a$, $b$, $c$.
\itemch \textbf{Sai}. Vì $2b \ne a+ c$.
\end{itemchoice}
}
\end{ex}
\Closesolutionfile{ans}

\TNSA
\Opensolutionfile{ans}[ans/ansDe2-TN3]
\begin{ex}%[2D4H1-1]%[Đào Trung Kiên]
Biết $F(x)$ là một nguyên hàm của hàm số $f(x)=\sin x$ và đồ thị hàm số $y=F(x)$ đi qua điểm $M\left(0;1\right)$. Tính $F\left(\dfrac{\pi}{2}\right)$ (làm tròn kết quả tới hàng đơn vị).
\shortans[]{$2$}
\loigiai{
Ta có $F(x)=\displaystyle\int f(x)\mathrm{\,d}x=-\cos x+C$.\\
Mà đồ thị hàm số $y=F(x)$ đi qua $M(0;1)$ nên $F(0)=1\Leftrightarrow -1+C=1\Leftrightarrow C=2$.\\
Suy ra $F(x)=-\cos x+2$ nên $F\left(\dfrac{\pi}{2}\right)=2$.}
\end{ex}

\begin{ex}%[2D4V3-2]
\immini{
Sàn của một viện bảo tàng mỹ thuật được lát bằng những viên gạch hình vuông cạnh $40$ cm  như hình bên. Biết rằng người thiết kế đã sử dụng các đường cong có phương trình $4x^2=y^4$  và  $4y^2 =x^4$ để tạo hoa văn cho viên gạch. Tính diện tích (đơn vị cm$^2$) phần được tô đậm (làm tròn kết quả đến hàng đơn vị).
}
{
\begin{tikzpicture}
\draw (2,2)--(2,-2)--(-2,-2)--(-2,2)--cycle;
\draw[fill=cyan, smooth, samples=200] plot[domain=-2:2,variable=\y]({0.5*(\y)^2},{\y})--plot[domain=2:0,variable=\x]({\x},{0.5*(\x)*(\x)})--plot[domain=0:2,variable=\x]({\x},{(-0.5)*(\x)*(\x)});
\draw[fill=cyan, smooth, samples=200] plot[domain=-2:2,variable=\y]({-0.5*(\y)^2},{\y})--plot[domain=-2:0,variable=\x]({\x},{0.5*(\x)*(\x)})--plot[domain=0:-2,variable=\x]({\x},{(-0.5)*(\x)*(\x)});
\end{tikzpicture}
}
\shortans{$533$}
\loigiai{
\immini{
Do tính đối xứng nên diện tích cần tìm bằng $4$ lần diện tích phần tô đậm của hình vẽ bên. Ta chỉ xét đồ thị trong góc phần tư thứ nhất do đó
\begin{align*}
&4x^2=y^4 \Leftrightarrow y = \sqrt{2x} \\
&4y^2 =x^4 \Leftrightarrow  y = \dfrac{1}{2} x^2.
\end{align*}
}
{
\begin{tikzpicture}[scale=1,>=stealth]
\draw[->] (-1,0)--(3,0) node[below] {$x$};
\draw[->] (0,-1)--(0,3) node[left] {$y$};
%\draw[fill] (1,0) circle (1pt) node[below] {$1$};
\draw[fill] (0,0) circle node[below left=-2pt] {$O$};
\draw[dashed] (2,0)--(2,2)--(0,2);
\draw[fill] (2,0) circle (1pt) node[below] {$2$};
\draw[fill] (0,2) circle (1pt) node[left] {$2$};
\draw[fill=cyan, smooth, samples=200] plot[domain=0:2,variable=\y]({0.5*(\y)^2},{\y})--plot[domain=2:0,variable=\x]({\x},{0.5*(\x)*(\x)});
\end{tikzpicture}
}
\noindent Một đơn vị trong hệ tọa độ $Oxy$ bằng $10$ cm, do đó diện tích của phần tô đậm ban đầu là
\[
S =4\cdot 10^2 \int \limits_0^2 \left ( \sqrt{2x}-\dfrac{1}{2}x ^2\right) \mathrm{d}x = 400 \left ( \dfrac{2\sqrt{2}}{3} \sqrt{x^3}-\dfrac{1}{6}x^3 \right ) \Bigg|_0^2= \dfrac{1600}{3}\approx 533 \ \mathrm{(cm^2)}.
\]
}
\end{ex}

\begin{ex}%[2H5H2-7]
Số các mặt phẳng $(\alpha)$ chứa đường thẳng $d\colon\dfrac{x}{1}=\dfrac{y}{-1}=\dfrac{z}{-3}$ và tạo với mặt phẳng $(P)\colon 2x-z+1=0$ góc $45^\circ $ bằng
\shortans{$2$}
\loigiai{
Đường thẳng $d$ đi qua điểm $O(0;0;0)$ có véc-tơ chỉ phương $\overrightarrow{u}=(1;-1;-3)$.\\
Ta có $(\alpha)$ qua $O$ có véc-tơ pháp tuyến $\overrightarrow{n}=(a;b;c)$ có dạng $ax+by+cz=0$.\\
Vì $\overrightarrow{n}\perp \overrightarrow{u}$ nên $\overrightarrow{n}\cdot \overrightarrow{u}=0$. Do đó $ a-b-3c=0$.\\
Mặt phẳng $(P)\colon 2x-z+1=0$ có véc-tơ pháp tuyến $\overrightarrow{k}=(2;0;-1)$.\\
Ta có
\begin{eqnarray*}
&&\cos 45^\circ =\dfrac{\left| \overrightarrow{n}\cdot \overrightarrow{k}\right|}{\left|\overrightarrow{n}\right|\cdot\left|\overrightarrow{k}\right|}\\
&\Leftrightarrow&\dfrac{\left| 2a-c\right|}{\sqrt{5(a^2+b^2+c^2)}}=\dfrac{\sqrt{2}}{2}\\
&\Leftrightarrow&10(a^2+b^2+c^2)=(4a-2c)^2\\
&\Leftrightarrow&10(b^2+6bc+9c^2+b^2+c^2)=(4b+12c-2c)^2\\
&\Leftrightarrow&10(2b^2+6bc+10c^2)=(4b+10c)^2\\
&\Leftrightarrow&4b^2-20bc=0\\
&\Leftrightarrow&\hoac{&b=0 \\&b=5c.}
\end{eqnarray*}
Xét
\begin{itemize}
\item $b=0\Rightarrow a=3c$ nên $(\alpha)\colon x+3z=0$.
\item $b=5c$, chọn $c=1\Rightarrow b=5$, $a=8$ nên $(\alpha)\colon 8x+5y+z=0$.
\end{itemize}
}
\end{ex}

\begin{ex}%[12-MH-2-MH2025]%[MH-2025, Nguyễn Trần Phong]%[2H5V1-7]
Một phần sân nhà bác An có dạng hình thang $ABCD$ vuông tại $A$ và $B$ với độ dài $AB=9$ m, $AD=5$ m và $BC=6$ m. Theo thiết kế ban đầu thì mặt sân bằng phẳng và $A$, $B$, $C$, $D$ có độ cao như nhau. Sau đó bác An thay đổi thiết kế để nước có thể thoát về phía góc sân ở vị trí $C$ bằng cách giữ nguyên độ cao ở $A$, giảm độ cao của sân ở vị trí $B$ và $D$ xuống thấp hơn độ cao ở $A$ lần lượt là $6$ cm và $3{,}6$ cm. Để mặt sân sau khi lát gạch vẫn là bề mặt phẳng thì bác An cần phải giảm độ cao ở $C$ xuống bao nhiêu cen-ti-mét so với độ cao ở $A$? \textit{Kết quả làm tròn đến hàng phần mười)}
\shortans{$10{,}3$}
\indent\indent\indent\indent\indent\indent\indent\indent\indent\begin{tikzpicture}[scale=0.5, font=\footnotesize,line join=round, line cap=round, >=stealth]
\path
(0,0) coordinate (A)
(9,0) coordinate(B)
(9,-6) coordinate(C)
(0,-5) coordinate(D)
;
\draw[thick] (A)--(B)--(C)--(D)--cycle;
\node [above] at ($(A)!0.5!(B)$) {$9$ m};
\node [right] at ($(B)!0.5!(C)$) {$6$ m};
\node [left] at ($(A)!0.5!(D)$) {$5$ m};
\foreach \i/\g in {A/90,B/90,C/-90,D/-90}{\draw[fill=black](\i) circle (0pt) ($(\i)+(\g:4mm)$) node[scale=1]{$\i$};}
\end{tikzpicture}
\loigiai{
Tại vị trí ban đầu $A$, $B$, $C$, $D$ có độ cao như nhau, chọn hệ trục tọa độ có gốc tọa độ là điểm $A$ và các trục tọa độ lần lượt là $AD$, $AB$ và $Az$, với $Az \perp (ABCD)$.\\
Khi đó $A(0 ; 0 ; 0)$, $D(5; 0 ; 0)$, $B(0; 9 ; 0)$, $C(6; 9 ; 0)$.\\
Sau đó bác An thay đổi thiết kế để nước có thể thoát về phía góc sân ở vị trí $C$ bằng cách giữ nguyên độ cao ở $A$, giảm độ cao của sân ở vị trí $B$ và $D$ xuống thấp hơn độ cao ở $A$ lần lượt là $6$ cm và $3{,}6$ cm.\\
Khi đó, $A(0 ; 0 ; 0)$, $D(5; 0 ; -3{,}6)$, $B(0; 9 ; -6)$.\\
Ta có $\overrightarrow{AB}=(0 ; 9 ; -6)$, $ \overrightarrow{AD}=(5 ; 0 ; -3{,}6)$ là cặp véc-tơ chỉ phương của mặt phẳng $(ABD)$ nên một véc-tơ pháp tuyến của $(ABD)$ là $\left[\overrightarrow{AB}, \overrightarrow{AD}\right]=(-32{,}4 ; -30 ; -45)$.\\
Vậy mặt phẳng $(ABD)$ qua $A(0 ; 0 ; 0)$ và có véc-tơ pháp tuyến $\vec{n}=(-32{,}4 ; -30 ; -45)$ nên có phương trình là
\allowdisplaybreaks
\begin{eqnarray*}
-32{,}4 (x-2)-30(y+1)-45(z-3)=0 \qquad \text{hay } -32{,}4 x -30y -45z=0.
\end{eqnarray*}
Để mặt sân sau khi lát gạch vẫn là bề mặt phẳng thì bác An cần phải giảm độ cao ở $C$ xuống $k$ centimét so với độ cao ở $A$ nên suy ra $C(6; 9 ; -k)$.\\
Ta có $A$, $B$, $C$, $D$ đồng phẳng\\
$\Leftrightarrow C \in (ABD)$\\
$\Leftrightarrow -32{,}4\cdot 6 -30 \cdot 9 -45\cdot (-k)=0$\\
$\Leftrightarrow k=10{,}32$.\\
Vậy bác An cần phải giảm độ cao ở $C$ xuống $10{,}3$ cen-ti-mét so với độ cao ở $A$.
}\end{ex}

\TL
\begin{ex} %[2H5H2-3].
Trong không gian $Oxyz$, cho ba điểm $A(3 ;-2 ;-2), B(3 ; 2 ; 0), C(0 ; 2 ; 1)$. Phương trình mặt phẳng $(ABC)$ có dạng $=ax+by+cz+d=0$. Tính $a+b+c$.
\shortans{$5$}
\loigiai{
Ta có $\overrightarrow{AB}=(0 ; 4 ; 2), \overrightarrow{AC}=(-3 ; 4 ; 3), \overrightarrow{n}=\left[ \overrightarrow{B} ; \overrightarrow{C}\right]=(4 ;-6 ; 12)$.\\
Ta có $\overrightarrow{n}=(4 ;-6 ; 12)$ cùng phương $\overrightarrow{n}_{1}=(2 ;-3 ; 6)$.\\
Mặt phẳng $(ABC)$ đi qua điểm $C(0 ; 2 ; 1)$ và có một véc-tơ pháp tuyến $\overrightarrow{n}_{1}=(2 ;-3 ; 6)$ nên $(ABC)$ có phương trình là
\[2(x-0)-3(y-2)+6(z-1)=0 \Leftrightarrow 2 x-3 y+6 z=0.\]
Vậy phương trình mặt phẳng cần tìm là $2x-3y+6z=0$.\\
Suy ra $a+b+c=5$.
}
\end{ex}

\begin{ex}%[2D4C3-2]
	\immini[thm]{Một họa tiết hình cánh bướm như hình vẽ bên. Phần tô đậm được đính đá với giá thành $500\,000$/$\,\mathrm{m^2}$. Phần còn lại được tô màu với giá thành $250\,000$/$\,\mathrm{m^2}$. Cho $AB=4$\,dm; $BC=8$\,dm. Hỏi để trang trí $1\,000$ họa tiết như vậy cần số tiền là bào nhiêu? (làm tròn đến hàng nghìn)
	% \choice
	% {$105\,660\,667$}
	% {\True $106\,666\,667$}
	% {$ 107\,665\,667$}
	% {$ 108\,665\,667$}
	}{
	\begin{tikzpicture}[line join=round, line cap=round,>=stealth,thick,scale=0.5]
	\tikzset{every node/.style={scale=0.9}}
	\begin{scope}
	\draw[fill=gray!35](-2,0)--plot[samples=200,domain=-2:2,smooth,variable=\x] (\x,{(\x)^2})--(2,0);
	\draw[fill=gray!35](-2,0)--plot[samples=200,domain=-2:2,smooth,variable=\x] (\x,{-1*(\x)^2})--(2,0);
	\draw[fill=black](-2,4) circle (1.5pt) node[left]{$A$} (2,4) circle (1.5pt) node[right]{$B$} (2,-4) circle (1.5pt) node[right]{$C$} (-2,-4) circle (1.5pt) node[left]{$D$};
	\draw (-2,4)--(2,4) (2,-4)--(-2,-4);
	\end{scope}
	\draw[->] (-3,0)--(3,0) node[below left] {$x$};
	\draw[->] (0,-5)--(0,5) node[below left] {$y$};
	\end{tikzpicture}
	}
	\loigiai{
	Vì $AB=4$\,dm; $BC=8$\,dm $\Rightarrow A(-2;4)$, $B(2;4)$, $C(2;-4)$, $D(-2;-4)$.\\
	parabol là $y=x^2$ hoặc $y=-x^2$.\\
	Diện tích phần tô đậm là $S_1=4\displaystyle\int\limits_0^2{x^2}\mathrm{\,d}x=\dfrac{32}{3}\mathrm{\,(dm^2)}$.\\
	Diện tích hình chữ nhật là $S=4\cdot 8=32\mathrm{\,(dm^2)}$.\\
	Diện tích phần trắng là $S_2=S-S_1=32-\dfrac{32}{3}=\dfrac{64}{3}\mathrm{\,(dm^2)}$.\\
	Tổng chi phí trang chí là $T=\left(\dfrac{32}{3}\cdot 5\,000+\dfrac{64}{3}\cdot 2\,500\right)\cdot 1\,000\approx 106\,667\,000$.
	}
	\end{ex}

\begin{ex}%[Mức độ ]giảng 12 New - 4in1, Đoàn Hùng]%[2H5V1-7]
	\immini
	{Trong không gian với hệ tọa độ $Oxyz$ (đơn vị trên mỗi trục toạ độ là km), một máy bay đang ở vị trí $A(3;-2{,}5; 0{,}5)$ và sẽ hạ cánh ở vị trí $B(3; 7{,}5; 0)$ trên đường băng (hình bên). Có một lớp mây được mô phỏng bởi một mặt phẳng $(\alpha)$ đi qua ba điểm $M(9;0;0)$, $N(0;-9;0)$, $P(0;0;0{,}9)$. Tính độ cao của máy bay khi máy bay xuyên qua đám mây để hạ cánh.}
	{\begin{tikzpicture}[line join = round, line cap = round,>=stealth,font=\footnotesize,scale=.5]
	\path
	(0,0) coordinate (O)
	(-9,0) coordinate (N)
	($(O)!1!40:(N)$) coordinate (M)
	(0,0.9) coordinate (P)
	($(O)!1.2!(M)$) coordinate (x)
	(-5,0.8) coordinate (A)
	(4,-1.5) coordinate (B)
	($(A)!2.1cm!(B)$) coordinate (C)
	(intersection of A--B and O--M) coordinate (B1)
	;
	\draw[line width=0.3mm,red] (A)--(C) (B1)--(B);
	\draw[line width=0.3mm,red,dashed] (C)--(B1);
	\draw[->,>=stealth,line width=0.3mm,blue] (0,0)--(x) node[right=0.2cm]{$x$};
	\draw[->,>=stealth,line width=0.3mm,blue] (-10,0)--(-9,0) (0,0) node[above right]{$O$}--(7,0) node[below]{$y$};
	\draw[line width=0.3mm,blue,dashed] (-9,0)--(0,0);
	\draw[->,>=stealth,line width=0.3mm,blue] (0,0)--(0,3) node[left]{$z$};
	\draw[line width=0.3mm,blue] (N)--(P)--(M) (N)--(M);
	\foreach \x/\gm in {N/90,P/140} \fill (\x) circle (1pt) ($(\x)+(\gm:5mm)$)node[blue]{$\x$};
	\filldraw[red] (N)node[below left,blue]{$-9$} (P)node[blue,above right]{$0{,}9$} (M)node[blue,right=0.1cm]{$9$}node[blue,left=0.1cm]{$M$} (C) circle (3pt) node[below=0.2cm,blue]{\scriptsize $C$} (A) circle (3pt)node[above left,red,blue]{$A$} (B)circle (3pt) node[below,blue]{$B$};
	\end{tikzpicture}}
	% \shortans{$0{,}45$}
	\loigiai{
	Giả sử điểm $C\left(x_C;y_C;z_C\right)$ là vị trí mà máy bay xuyên qua đám mây để hạ cánh, suy ra $C\in (\alpha)$. Áp dụng phương trình mặt phẳng theo đoạn chắn, ta thấy mặt phẳng $(\alpha)$ có phương trình là
	\[\dfrac{x}{9}-\dfrac{y}{9}+\dfrac{z}{0{,}9}=1 \Leftrightarrow x-y+10z=9 \Rightarrow x_C-y_C+10z_C=9.\]
	Mặt khác, vì $\vec{AC}$, $\vec{AB}$ là hai véc-tơ cùng hướng nên tồn tại số thực $t>0$ sao cho $\vec{AC}=t\cdot \vec{AB}$.\\
	Do $\vv{AC}=\left(x_C-3;y_C+2{,}5;z_C-0{,}5\right)$; $\vv{AB}=\left(3-3;7{,}5+2{,}5;0-0{,}5\right)=\left(0;10;-0{,}5\right)$\\
	nên $\heva{&x_C-3=0t\\&y_C+2{,}5=10t\\&z_C-0{,}5=-0{,}5t} \Leftrightarrow \heva{&x_C=3\\&y_C=10t-2{,}5\\&z_C=-0{,}5t+0{,}5.}$\\
	Vì $C\in(\alpha)$ nên $3-(10 t-2{,}5)+10(-0{,}5 t+0{,}5)=9 \Leftrightarrow t=0{,}1$. Suy ra $C(3;-1{,}5;0{,}45)$.\\
	Vậy tại vị trí $C$, độ cao của máy bay là $0{,}45$ km.
	}
	\end{ex}
\Closesolutionfile{ans}


% \Closesolutionfile{ansbook}
% \HetDe
% \label{De2}
% %
% \cleardoublepage
% \setcounter{page}{1}
% \rfoot{Trang \thepage/\pageref{DA2} - Đáp án trắc nghiệm Mã đề 2}
% \begin{center}
% 	\bfseries ĐÁP ÁN TRẮC NGHIỆM MÃ ĐỀ 2
% \end{center}

% \inputansbox{10}{ans/ansDe2-TN1}
% \inputansbox[3]{2}{ans/ansDe2-TN2}
% \inputansbox{3}{ans/ansDe2-TN3}
% \label{DA2}
% %

% \begin{name}
	{\tenchude}
	{TOÁN 12}
	{LỚP TOÁN THẦY PHÁT}
	{Thời gian: 90 phút - Không kể thời gian phát đề}
\end{name}
\Opensolutionfile{ans}[ans/ansDe3-TN1]
\begin{ex}%[2D4N1-1]%[To 20 - Dot 17 - Chuong 4 - Bai 3 - CD - De 1 - TN]%[Nguyễn Hữu Duy]
Cho hàm số $f(x)$ liên tục trên đoạn $[a;b]$. Nếu biết $\displaystyle\int_a^b f(x) \mathrm{\,d}x = 2025$, thì giá trị $\displaystyle\int_a^b 2f(x) \mathrm{\,d}x$ là bao nhiêu?
\choice
{\True $4050$}
{$4051$}
{$4052$}
{$4053$}
\loigiai
{ Ta có $\displaystyle\int_a^b 2f(x) \mathrm{\,d}x = 2\displaystyle\int_a^b f(x) \mathrm{\,d}x = 2 \cdot 2025 = 4050$.
}
\end{ex}

\begin{ex}%[2D4N1-2]%[2015_K12 Huyên Nguyễn]
Họ nguyên hàm của hàm số $f(x)=2x+\dfrac{3}{x^2}$ là
\choice
{\True $x^2-\dfrac{3}{x}+C$}
{$x^2+\dfrac{3}{x}+C$}
{$x^2+3\ln {x^2}+C$}
{$x^2+\dfrac{3}{2}\ln |x^2|+C$}
\loigiai{
$\displaystyle\int\left(2x+\dfrac{3}{x^2}\right)\mathrm{\,d}x=x^2-\dfrac{3}{x}+C$.
}
\end{ex}

\begin{ex}%[2D4H1-3]
Nguyên hàm $F(x)$ của hàm số $f(x) = \cos x$ thỏa mãn $F(0) = 1$ là
\choice
{\True $F(x) = \sin x + 1$ }
{ $F(x) = -\sin x + 1$ }
{ $F(x) = \cos x$ }
{ $F(x) = -\cos x + 2$ }
\loigiai{
Ta có $F(x) = \displaystyle\int f(x) \mathrm{d}x = \displaystyle\int \cos x \mathrm{d}x = \sin x + C$. \\
Vì $F(0) = 1$, nên $\sin 0 + C = 1 \Rightarrow C = 1$. \\
Vậy $F(x) = \sin x + 1$.
}
\end{ex}

\begin{ex}%[DỰ ÁN TEX 12 TOÁN TỪ TÂM - TRƯƠNG ĐĂNG KHOA]%[2D4H2-1]
Cho hàm số $y=f(x)$ có đạo hàm $f'(x)$ và $f'(x)$ liên tục trên đoạn $[a;b]$. Gọi $F(x)$ là một nguyên hàm của hàm số $f(x)$ trên đoạn $[a;b]$. Chọn mệnh đề đúng.
\choice
{\True $f(b)-f(a)=\displaystyle\int\limits_{a}^{b}f'(x)\mathrm{\,d}x$}
{$F(b)-F(a)=\displaystyle\int\limits_{a}^{b}f'(x)\mathrm{\,d}x$}
{$f(b)-f(a)=\displaystyle\int\limits_{a}^{b}F(x)\mathrm{\,d}x$}
{$f'(b)-f'(a)=\displaystyle\int\limits_{a}^{b}f'(x)\mathrm{\,d}x$}
\loigiai{Ta có $f(b)-f(a)=\displaystyle\int\limits_{a}^{b}f'(x)\mathrm{\,d}x$.}
\end{ex}

\begin{ex}%[Nguyễn Tuấn, dự án sáng tác đề 12]%[2D4N3-1]
\immini{Diện tích phần hình phẳng gạch chéo trong hình vẽ bên được tính theo công thức nào dưới đây?
\choice
{$\displaystyle\int\limits_{-1}^2\left(2x^2-2x-4\right)\mathrm{\,d}x$}
{$\displaystyle\int\limits_{-1}^2(-2x+2)\mathrm{\,d}x$}
{$\displaystyle\int\limits_{-1}^2(2x-2)\mathrm{\,d}x$}
{\True $\displaystyle\int\limits_{-1}^2\left(-2x^2+2x+4\right)\mathrm{\,d}x$}}
{\begin{tikzpicture}[scale=.7, font=\footnotesize, line join=round, line cap=round, >=stealth]
\draw[->] (-1.5,0) -- (3,0)node[below]{\footnotesize $x$};
\draw (-1,0) circle (.5pt)node[below]{\footnotesize $-1$};
\draw (2,0) circle (.5pt)node[above]{\footnotesize $2$};
\draw[->,color=black] (0,-2.5) -- (0,3.5)node[below left]{\footnotesize $y$};
\fill[pattern=north west lines] plot[smooth,samples=100,domain=-1:2] (\x,{(\x)^2-2*(\x)-1})--plot[smooth,samples=100,domain=2:-1] (\x,{-(\x)^2+3});
\draw[thick,smooth,samples=100,domain=-1.5:2.2] plot(\x,{-(\x)^2+3});
\draw[thick,smooth,samples=100,domain=-1.2:3] plot(\x,{(\x)^2-2*(\x)-1});
\draw[dashed] (2,0) -- (2,-1) (-1,0) -- (-1,2);
\filldraw[fill=white] (0,0) circle (1pt)node[shift={(-45:6pt)}]{\footnotesize $O$};
\draw (2,3) node{\footnotesize $y=-x^2+3$};
\draw (-1,-2.2) node{\footnotesize $y=x^2-2x-1$};
\end{tikzpicture}}
\loigiai{
$S=\displaystyle\int\limits_{-1}^2\left[\left(-x^2+3\right)-\left(x^2-2x-1\right)\right]\mathrm{\,d}x=\displaystyle\int\limits_{-1}^2\left(-2x^2+2x+4\right)\mathrm{\,d}x$.
}
\end{ex}

\begin{ex}%[2D4H3-3]
\immini[thm]{
Cho tam giác vuông $OAB$, có cạnh $OA=a$ nằm trên tục $Ox$ và $\widehat{AOB}=\alpha \left(0< \alpha \le \dfrac{\pi}{4} \right)$. Gọi $\beta$ là khối tròn xoay sinh ra khi quay miền tam giác $OAB$ xung quanh trục $Ox$. Thể tích $V$ của $\beta$ tính theo $a$ và $\alpha$ là
\choice
{\True $V=\dfrac{\pi \tan ^2\alpha\cdot a^3}{3}$}
{$V=\dfrac{3\pi \tan ^2\alpha\cdot a^3}{2}$}
{$V=\dfrac{\pi \sin ^2\alpha\cdot a^3}{3}$}
{$V=\dfrac{2\pi \cos^2\alpha\cdot a^3}{3}$}
}{
\begin{tikzpicture}[scale=0.7, font=\footnotesize, line join=round, line cap=round, >=stealth]
\fill[yellow!30]	(0,0)--(4,2)--(4,0)--cycle;
\draw[->] (4,0) -- (5.5,0) node[right] {$x$};
\draw[->] (0,-2.5) -- (0,2.5) node[above] {$y$};
\draw[->] (0,0) -- (-1.5,-1.5) node[below left] {$z$};
\draw (0,0)--(4,2)	(0,0)--(4,-2)	(4,2)--(4,0)	(-0.5,0)--(0,0);
\draw[dashed] (0,0)--(4,0)	;
\draw (4,0) ellipse (0.5 and 2);
\fill (0,0) circle (1pt) node[above left]{$O$};
\fill (4,0) circle (1pt) node[below]{$A$};
\fill (4,2) circle (1pt) node[above]{$B$};
\clip (4,0) -- (0,0) -- (4,2);
\draw (0,0) circle (1cm);
\draw ($(0,0)+(1,0)$) node[above right]{$\alpha$};
%\node[below] at (2.5,-2.5) {Hình $4.31$};
\end{tikzpicture}
}
\loigiai{
Do $OB$ đi qua gốc tọa độ và tạo với $Ox$ một góc $\alpha$ nên $OB\colon y=x\cdot \tan \alpha$.\\
Khi đó, thể tích của khối $\beta$ là
\[
V=\pi \displaystyle\int_0^a\left(x\cdot \tan \alpha \right)^2\mathrm{d}x=\pi \tan ^2\alpha \displaystyle\int_0^ax^2\mathrm{d}x=\dfrac{\pi \tan ^2\alpha\cdot x^3}{3}\bigg|_0^a=\dfrac{\pi \tan ^2\alpha\cdot a^3}{3}.
\]
}
\end{ex}

\begin{ex}%[2H5N1-1]
Trong không gian với hệ tọa độ $O x y z$, phương trình nào sau đây là phương trình của mặt phẳng $O z x$ ?
\choice
{$x=0$}
{$y-1=0$}
{\True $y=0$}
{$z=0$}
\loigiai{
Ta có mặt phẳng $(Oxz)$ đi qua điểm $O(0 ; 0 ; 0)$ và vuông góc với trục $O y$ nên có VTPT $\vec{n}=(0 ; 1 ; 0)$.\\
Do đó phương trình của mặt phẳng $(Oxz)$ là $y=0$.
}
\end{ex}

\begin{ex}%[2H5N1-2]
Trong không gian $Oxyz$, véc-tơ nào sau đây là một véc-tơ pháp tuyến của mặt phẳng $(Oxy)$?
\choice
{$\overrightarrow{n}=(1;1;0)$}
{$\overrightarrow{j}=(0;1;0)$}
{$\overrightarrow{i}=(1;0;0)$}
{\True $\overrightarrow{k}=(0;0;1)$}
\loigiai{
Mặt phẳng $(Oxy)$ vuông góc với trục $Oz$ nên véc-tơ $\overrightarrow{k}=(0;0;1)$ là một véc-tơ pháp tuyến của mặt phẳng $(Oxy)$.
}
\end{ex}

\begin{ex}%[2H5H1-3]
Trong không gian với hệ trục tọa độ $O x y z$, cho mặt phẳng $(P)\colon  a x+b y+c z-9=0$ chứa hai điểm $A(3 ; 2 ; 1),$ $ B(-3 ; 5 ; 2)$ và vuông góc với mặt phẳng $(Q)\colon  3 x+y+z+4=0$. Tính tổng $S=a+b+c$?
\choice
{$S=-12$}
{$S=2$}
{\True $S=-4$}
{$S=-2$}
\loigiai{
$\overrightarrow{A B}=(-6 ; 3 ; 1)$.\\
$\overrightarrow{n}_{(Q)}=(3 ; 1 ; 1)$ là véc-tơ pháp tuyến  của $(Q)$.\\
Mặt phẳng $(P)$ chứa hai điểm $A(3 ; 2 ; 1),$ $ B(-3 ; 5 ; 2)$ và vuông góc với mặt phẳng $(Q)$.\\
Suy ra $ \overrightarrow{n}_{(P)}=\left[\overrightarrow{A B}, \overrightarrow{n}_{(Q)}\right]=(2 ; 9 ;-15)$ là véc-tơ pháp tuyến  của $(P)$.\\
$A(3 ; 2 ; 1) \in(P)\Rightarrow(P)\colon 2 x+9 y-15 z-9=0$ hoặc $(P)\colon -2 x-9 y+15 z+9=0$.\\
Mặt khác $(P)\colon a x+b y+c z-9=0 \Rightarrow a=2 ; $ $b=9 ;$ $ c=-15$.\\
Vậy $S=a+b+c=2+9+(-15)=-4$.
}
\end{ex}

\begin{ex}%[2H5N2-1]
Đường thẳng $(\Delta)\colon \dfrac{x-1}{2}=\dfrac{y+2}{1}=\dfrac{z}{-1}$ đi qua điểm nào dưới đây?
\choice
{\True $M(1 ;-2 ; 0)$}
{$N(-1 ; 2 ; 0)$}
{$P(3 ; 1 ;-1)$}
{$Q(-1 ;-2 ; 0)$}
\loigiai{
Ta có $\dfrac{1-1}{2}=\dfrac{2-2}{1}=\dfrac{0}{-1}$ nên điểm $M(1 ;-2 ; 0)$ thuộc đường thẳng $(\Delta)$.
}
\end{ex}

\begin{ex}%[2H5N2-7]
Trong không gian $Oxyz$, cho mặt phẳng $(P)\colon -\sqrt{3}x+y+1=0$. Tính góc tạo bởi $(P)$ với trục $Ox$?
\choice
{\True $60^\circ$}
{$30^\circ$}
{$120^\circ$}
{$150^\circ$}
\loigiai{
Mặt phẳng $(P)$ có véc tơ pháp tuyến $\overrightarrow{n}=(-\sqrt{3};1;0)$\\
Trục $Ox$ có có véc tơ pháp tuyến $\overrightarrow{i}=(1;0;0)$.\\
Góc tạo bởi $(P)$ với trục $Ox$\\
$\sin((P),Ox)=\left| \cos((P), Ox) \right|=\dfrac{\left| \overrightarrow{n}\cdot\overrightarrow{i} \right|}{\left| \overrightarrow{n} \right|\cdot\left| \overrightarrow{i} \right|}=\dfrac{\left| -\sqrt{3}\cdot1+1\cdot0+0\cdot0 \right|}{\sqrt{3+1}\cdot\sqrt{1}}=\dfrac{\sqrt{3}}{2}$.\\
Vậy góc tạo bởi $(P)$ với trục $Ox$ bằng $60^\circ$.}
\end{ex}

\begin{ex}%[2H5H2-4]
Trong không gian với hệ tọa độ $Oxyz$ cho $A\left(1; -1; 3\right)$ và hai đường thẳng $d_{1}  \colon \dfrac{x-4}{1} =\dfrac{y+2}{4} =\dfrac{z-1}{-2} ,$ $d_{2}  \colon \dfrac{x-2}{1} =\dfrac{y+1}{-1} =\dfrac{z-1}{1}$. Phương trình đường thẳng qua $A$, vuông góc với $d_{1} $ và cắt $d_{2} $ là
\choice
{$\dfrac{x-1}{2} =\dfrac{y+1}{1} =\dfrac{z-3}{3} $}
{$\dfrac{x-1}{4} =\dfrac{y+1}{1} =\dfrac{z-3}{4} $}
{$\dfrac{x-1}{-1} =\dfrac{y+1}{2} =\dfrac{z-3}{3} $}
{\True $\dfrac{x-1}{2} =\dfrac{y+1}{-1} =\dfrac{z-3}{-1} $}
\loigiai{
Gọi $d$ là đường thẳng qua $A$ và $d$ cắt $d_{2} $ tại $K$. Khi đó $K\left(2+t; -1-t; 1+t\right)$. \\
Ta có $\overrightarrow{AK}=\left(1+t; -t; t-2\right)$. Đường $AK\perp d_{1} $$\Leftrightarrow \overrightarrow{AK}\cdot\overrightarrow{u_{1} }=0$, với $\vec{u}_{1} =\left(1; 4; -2\right)$ là một véc-tơ chỉ phương của $d_{1} $. \\
Do đó $1+t-4t-2t+4=0\Leftrightarrow t=1$, suy ra $\overrightarrow{AK}=\left(2; -1; -1\right)$. \\
Vậy phương trình đường thẳng $d \colon \dfrac{x-1}{2} =\dfrac{y+1}{-1} =\dfrac{z-3}{-1}.$}
\end{ex}
\Closesolutionfile{ans}

\TNTF
\Opensolutionfile{ans}[ans/ansDe3-TN2]
\begin{ex}%[2D4H2-4]
Các mệnh đề sau đây đúng hay sai.
\choiceTF
{\True $\displaystyle\int\limits_0^1\dfrac{\mathrm{e}^{2x}-4}{\mathrm{e}^x+2}\mathrm{\,d}x=\mathrm{e}-3$}
{$\displaystyle\int\limits_0^1\dfrac{\mathrm{e}^x}{2^x}\mathrm{\,d}x=\dfrac{\mathrm{e}}{2}+1$}
{\True $\displaystyle\int\limits_1^2\mathrm{e}^x\left(1-\dfrac{\mathrm{e}^{-x}}{x}\right)\mathrm{\,d}x=\mathrm{e}^2-\mathrm{e}-\ln 2$}
{$\displaystyle\int\limits_0^1\dfrac{\mathrm{e}^{2x-1}-\mathrm{e}^{-3x}+1}{\mathrm{e}^x}\mathrm{\,d}x=\mathrm{e}^4-1$}
\loigiai{\begin{itemchoice}
\itemch Đúng. \allowdisplaybreaks
\begin{eqnarray*} \displaystyle\int\limits_0^1\dfrac{\mathrm{e}^{2x}-4}{\mathrm{e}^x+2}\mathrm{\,d}x&=&\displaystyle\int\limits_0^1\dfrac{\left(\mathrm{e}^x-2\right)\left(\mathrm{e}^x+2\right)}{\mathrm{e}^x+2}\mathrm{\,d}x\\
&=&\displaystyle\int\limits_0^1\left(\mathrm{e}^x-2\right)\mathrm{\,d}x=\left(\mathrm{e}^x-2x\right)\big|_0^1=\mathrm{e}-3.
\end{eqnarray*}
\itemch Sai.  $\displaystyle\int\limits_0^1\dfrac{\mathrm{e}^x}{2^x}\mathrm{\,d}x=\displaystyle\int\limits_0^1\left(\dfrac{\mathrm{e}}{2}\right)^x\mathrm{\,d}x=\left[\left(\dfrac{\mathrm{e}}{2}\right)^x\right]\Big|_0^1=\dfrac{\mathrm{e}}{2}-1$.
\itemch Đúng. $\displaystyle\int\limits_1^2\mathrm{e}^x\left(1-\dfrac{\mathrm{e}^{-x}}{x}\right)\mathrm{\,d}x=\displaystyle\int\limits_1^2\left(\mathrm{e}^x-\dfrac{1}{x}\right)\mathrm{\,d}x=\left(\mathrm{e}^x-\ln \left| x\right|\right)\big|_1^2=\mathrm{e}^2-\mathrm{e}-\ln 2$.
\itemch Sai.\allowdisplaybreaks
\begin{eqnarray*} \displaystyle\int\limits_0^1\dfrac{\mathrm{e}^{2x-1}-\mathrm{e}^{-3x}+1}{\mathrm{e}^x}\mathrm{\,d}x&=&\displaystyle\int\limits_0^1\left(\mathrm{e}^{x-1}-\mathrm{e}^{-4x}+\mathrm{e}^{-x}\right)\mathrm{\,d}x\\
&=&\left(\mathrm{e}^{x-1}-\mathrm{e}^{-4x}+\mathrm{e}^{-x}\right)\big|_0^1=\dfrac{1-\mathrm{e}^4}{\mathrm{e}^4}=\mathrm{e}^{-4}-1.
\end{eqnarray*}
\end{itemchoice}
}
\end{ex}

\begin{ex}%[2H5H2-5]%[Dự án 2025 - Đề cấu trúc mới của Bộ theo [Thành Đức Trung]
Trong không gian $Oxyz$, cho hai điểm $A(-1;1;2)$, $B(2;-2;3)$.
\choiceTF
{\True Đường thẳng $AB$ có một véc-tơ chỉ phương là $\overrightarrow{u}=(3;-3;1)$}
{\True Phương trình tham số của đường thẳng $AB$ là $\heva{ & x=-1+3t \\ & y=1-3t \\ & z=2+t}$}
{\True Mặt phẳng $(P)$ đi qua $A$ và vuông góc với $AB$ là $(P)\colon 3x-3y+z+4=0$}
{Mặt phẳng $(Q)$ là mặt phẳng trung trực của đoạn thẳng $AB$ là $(Q) \colon 3x-3y+z-11=0$}
\loigiai
{
\begin{itemchoice}
\itemch \textbf{Đúng}. Vì đường thẳng $AB$ có một véc-tơ chỉ phương là $\overrightarrow{u}=\overrightarrow{AB}=(3;-3;1)$.
\itemch \textbf{Đúng}. Vì phương trình tham số của đường thẳng $AB$ là $\heva{ & x=-1+3t \\ & y=1-3t \\ & z=2+t.}$
\itemch \textbf{Đúng}. Vì mặt phẳng $(P)$ đi qua $A$ và vuông góc với $AB$ nên có một véc-tơ pháp tuyến là $\overrightarrow{n}=(3;-3;1)$. \\
Vậy $(P)\colon 3(x+1)-3(y-1)+1(z-2)=0 \Leftrightarrow 3x-3y+z+4=0$.
\itemch \textbf{Sai}. Vì gọi $M$ là trung điểm $AB$ là $M\left(\dfrac{1}{2};-\dfrac{1}{2};\dfrac{5}{2}\right)$.\\
Mặt phẳng $(Q)$ đi qua điểm $M\left(\dfrac{1}{2};-\dfrac{1}{2};\dfrac{5}{2}\right)$ và có một véc-tơ pháp tuyến $\overrightarrow{n}=(3;-3;1)$ là
\[(Q)\colon 3\left(x-\dfrac{1}{2}\right)-3\left(y+\dfrac{1}{2}\right)+1\left(z-\dfrac{5}{2}\right)=0 \Leftrightarrow 3x-3y+z-\dfrac{11}{2}=0.\]
\end{itemchoice}
}
\end{ex}
\Closesolutionfile{ans}

\TNSA
\Opensolutionfile{ans}[ans/ansDe3-TN3]
\begin{ex}%[2D4H1-1]%[Đào Trung Kiên]
Cho $F(x)$ là một nguyên hàm của hàm số $f(x) = ax + \dfrac{b}{x^2}$ $(x \neq 0)$. Biết $F(-1) = 1$, $F(1) = 4$, $f(1) = 0$. Tính giá trị của $M = 2a - b$ (làm tròn tới hàng phần mười).
\shortans[]{$4,5$}
\loigiai{
Ta có $\displaystyle\int f(x)\mathrm{\,d}x = \displaystyle\int \left(ax + \dfrac{b}{x^2}\right)\mathrm{\,d}x = \dfrac{ax^2}{2} - \dfrac{b}{x} + C$.\\
Theo giả thiết, ta có hệ phương trình $\heva{&F(-1) = 1 \\ &F(1) = 4 \\ &f(1) = 0} \Leftrightarrow \heva{&a + b + C = 1 \\ &a - b + C = 4 \\ &a + b = 0} \Rightarrow \heva{&a = \dfrac{3}{2} \\ &b = -\dfrac{3}{2}\cdot}$\\
Vậy $M = 2a - b = 3 + \dfrac{3}{2} = \dfrac{9}{2}=4{,}5$
}
\end{ex}

\begin{ex}%[2D4V3-4]%[Dự Án EX-TF-TLN Toán 12 - Đợt 2 - Quan Ón]
	Một vật thể nằm giữa hai mặt phẳng $x = 0$ và $x = \dfrac{\pi}{2}$; biết rằng mặt cắt của vật thể cắt bởi mặt phẳng vuông góc với trục $Ox$ tại điểm có hoành độ $x$ $\left(0 \leq x \leq \dfrac{\pi}{2}\right)$ là tam giác đều có cạnh là $2\sqrt{\cos x + \sin x}$. Thể tích của vật thể trên có dạng $a\sqrt{b}$. Hãy tính $5a + b^2$.
	\shortans{$19$}
	\loigiai{
	Diện tích của mặt cắt là $S(x) = \dfrac{\left(2\sqrt{\cos x + \sin x} \right)^2\cdot \sqrt{3}}{4} = \sqrt{3}\left(\cos x + \sin x\right)$.
	Vậy thể tích của vật thể đã cho là
	\[ V = \int_{0}^{\frac{\pi}{2}}S(x)\,\mathrm{d}x = \int_{0}^{\frac{\pi}{2}}\sqrt{3}\left(\cos x + \sin x\right)\,\mathrm{d}x = 2\sqrt{3}.\]
	Do đó $a = 2$ và $b = 3$.\\
	Vậy $5a + b^2 = 5\cdot 2 + 3^2 = 19$.
	}
	\end{ex}

\begin{ex}%[Nguyễn Tuấn, dự án sáng tác đề 12 theo chủ đề]%[2H5H2-7]
Cho mặt phẳng $(P)$ có véc-tơ pháp tuyến $\overrightarrow{n}=(1 ; 2 ; 2)$ và đường thẳng $\Delta$ có véc-tơ chỉ phương $\overrightarrow{u}=(2 ; 2 ;-1)$. Góc giữa đường thẳng $\Delta$ và mặt phẳng $(P)$ bằng bao nhiêu độ (làm tròn kết quả đến hàng đơn vị)?
\shortans{$26$}
\loigiai{
Ta có $\sin (\Delta,(P))=\dfrac{|1 \cdot 2+2 \cdot 2+2 \cdot(-1)|}{\sqrt{1^2+2^2+2^2} \cdot \sqrt{2^2+2^2+(-1)^2}}=\dfrac{4}{9}$. \\
Suy ra $(\Delta,(P)) \approx 26^{\circ}$.
}
\end{ex}

\begin{ex}%[2H5V1-7]
\immini
{
Để chuẩn bị cho chuyến đi dã ngoại, nhóm bạn Đức thiết kế lều cắm trại dạng hình chóp tứ giác đều có đáy là hình vuông cạnh $4$m. Theo bản vẽ thiết kế thì góc giữa hai mặt bên của lều bằng $60^{\circ}$. Bằng phương pháp tọa độ, hãy tính chiều cao của lều này.
}
{
\begin{tikzpicture}[>=stealth,line join=round,line cap=round,font=\footnotesize,scale=1]
\def \cao{3};
\def \x{3};
\coordinate (D) at (0,0);
\coordinate (C) at (1,.8);
\coordinate (A) at (\x,0);
\coordinate (B) at ($(A)-(D)+(C)$);
\coordinate (O) at ($(C)!.5!(A)$);
\coordinate (S) at ($(O)+(90:\cao)$);
\coordinate (X) at (intersection of S--O and  D--A);
\draw (D)--(A)--(B)--(S)--cycle (S)--(A);
\draw[dashed] (C)--(B) (S)--(C)--(D);
\draw[dashed,red] (A)--($(A)!1.2!(C)$) (D)--(B) (S)--(X);
\draw[->,red] (A)--($(O)!1.6!(A)$) node[right]{$x$};
\draw[->,red] (B)--($(O)!1.4!(B)$) node[above]{$y$};
\draw[->,red] (S)--($(O)!1.3!(S)$) node[left]{$z$};
\draw[red] (X)--++(-90:0.7) (D)--($(O)!1.3!(D)$);
\foreach \diem/\goc in {C/45,D/-90,A/-90,B/-30,S/45,O/-130} \fill[black](\diem) circle (1pt) ($(\diem)+(\goc:3mm)$) node{$\diem$};
\end{tikzpicture}
}
\shortans{$12$}
\loigiai{
Ta có $AC=BD=\sqrt{AD^2+AB^2}=\sqrt{4^2+4^2}=4\sqrt{2}$.\\
Chọn hệ trục tọa độ $Oxyz$ với gốc tọa độ $O$ là giao điểm của hai đường chéo $AC$ và $BD$ như hình vẽ.\\
Gọi $z$ là chiều cao của lều.\\
Ta có $O(0;0;0),A(2\sqrt{2};0;0),B(0;2\sqrt{2};0),C(-2\sqrt{2};0;0),D(0;-2\sqrt{2};0),S(0;0;z)$, với $z>0$.\\
Ta có $\overrightarrow{AD}=(-2\sqrt{2};-2\sqrt{2};0),\overrightarrow{AB}=(-2\sqrt{2};2\sqrt{2};0),\overrightarrow{AS}=(-2\sqrt{2};0;z)$.\\
Véc-tơ pháp tuyến của $(SAD)$ là $\overrightarrow{n_1}=[\overrightarrow{AD},\overrightarrow{AS}]=(-2z\sqrt{2};2z\sqrt{2};-8)$.\\
Véc-tơ pháp tuyến của $(SAB)$ là $\overrightarrow{n_2}=[\overrightarrow{AB},\overrightarrow{AS}]=(2z\sqrt{2};2z\sqrt{2};8)$.\\
Ta có
\begin{eqnarray*}
\cos((SAD),(SAB))&=&\dfrac{|\overrightarrow{n_1}\cdot \overrightarrow{n_2}|}{|\overrightarrow{n_1}|\cdot |\overrightarrow{n_2}|}\\
&=&\dfrac{|-2z\sqrt{2}\cdot 2z\sqrt{2}+2z\sqrt{2}\cdot 2z\sqrt{2}+8\cdot(-8)|}{\sqrt{(-2z\sqrt{2})^2+(2z\sqrt{2})^2+(-8)^2}\cdot \sqrt{(2z\sqrt{2})^2+(2z\sqrt{2})^2+8^2}}\\
&=&\dfrac{64}{16z^2+64}.
\end{eqnarray*}
Vì góc giữa hai mặt bên bằng $60^{\circ}$ nên góc giữa hai mặt phẳng $(SAD)$ và $(SAB)$ bằng $60^{\circ}$.\\
Do đó
\[\cos 60^{\circ}=\dfrac{64}{16z^2+64} \Leftrightarrow \dfrac{1}{2}=\dfrac{64}{16z^2+64}\Leftrightarrow 16z^2+64=128\Leftrightarrow \hoac{&z=2\\&z=-2}\]
Vì $z>0$ nên ta có $S(0;0;2).$\\
Vậy chiều cao của lều là $2$ m.
}

\end{ex}

\TL
\begin{ex}%[2H5H2-3]
Trong không gian $Oxyz$, gọi $M, N, P$ lần lượt là hình chiếu vuông góc của $A(2 ;-3 ; 1)$ lên các mặt phẳng tọa độ. Tính $a+b+c$ của phương trình mặt phẳng $(MNP)\colon ax+by+cz+d=0$.
% \shortans{$7$}
\loigiai{
Không mất tính tổng quát, ta giả sử $M, N, P$ lần lượt là hình chiếu vuông góc của $A(2 ;-3 ; 1)$ lên các mặt phẳng tọa độ $(Oxy),(Oxz),(Oyz)$. \\
Khi đó $M(2 ;-3 ; 0), N(2 ; 0 ; 1)$ và $P(0 ;-3 ; 1).$\\
$\overrightarrow{MN}=(0 ; 3 ; 1)$ và $\overrightarrow{MP}=(-2 ; 0 ; 1)$. \\
Ta có $\overrightarrow{MN}$ và $\overrightarrow{MP}$ là cặp véc-tơ không cùng phương và có giá nằm trong $(MNP)$.\\
Do đó $(MNP)$ có một véc-tơ pháp tuyến là $\overrightarrow{n}=\left[\overrightarrow{M N}, \overrightarrow{MP}\right]=(3 ;-2 ; 6)$.\\
Mặt khác $(MNP)$ đi qua $M(2 ;-3 ; 0)$ nên có phương trình là
\[3(x-2)-2(y+3)+6(z-0)=0 \Leftrightarrow 3x-2y+6z-12=0.\]
Suy ra $a+b+c=7$.
}
\end{ex}

\begin{ex}%[2D4C3-2]
\immini
{Ông An xây dựng một sân bóng đá mini hình chữ nhật có chiều rộng $30$m và chiều dài $50$m. Để giảm bớt chi phí cho việc trồng cỏ nhân tạo, ông An chia sân bóng ra làm hai phần (tô đen và không tô đen) như hình bên. Phần tô đen gồm hai phần diện tích bằng nhau và đường cong $AIB$ là một parabol đỉnh $I$ được trồng cỏ nhân tạo với giá $130\,000$ đồng/m$^2$ và phần còn lại được trồng với giá $90\,000$ đồng/m$^2$.
}
{\begin{tikzpicture}[scale=0.9, font=\footnotesize,line join=round, line cap=round, >=stealth]
\coordinate (I) at (0,0);
\coordinate (A) at (1.5,2.25);
\coordinate (B) at ($(A)+(0,-4.5)$);
\coordinate (C) at ($(B)+(-7.5,0)$);
\coordinate (D) at ($(A)-(B)+(C)$);
\coordinate (M) at ($(A)!1/2!(D)$);
\coordinate (N) at ($(B)!1/2!(C)$);
\coordinate (P) at ($(A)+(-1.5,0)$);
\coordinate (Q) at ($(B)+(-1.5,0)$);
\coordinate (R) at ($(D)+(1.5,0)$);
\coordinate (S) at ($(C)+(1.5,0)$);
\coordinate (td) at ($(D)+(0,0.3)$);
\coordinate (dt) at ($(D)+(-0.3,0)$);
\coordinate (ct) at ($(C)+(-0.3,0)$);
\coordinate (tr) at ($(R)+(0,0.3)$);
\coordinate (rp) at ($(R)+(0.3,0)$);
\coordinate (I') at ($(R)!1/2!(S)$);
\coordinate (g) at ($(I')+(0.3,0)$);
\fill[gray]plot[domain=0:1.5](\x,{sqrt(3.375*(\x))})--(A)--plot[domain=1.5:0](\x,{-sqrt(3.375*(\x))})--cycle;
\fill[gray](C)--(D)--plot[domain=-6:-4.5](\x,{sqrt(3.375*(-\x-4.5))})--plot[domain=-4.5:-6](\x,{-sqrt(3.375*(-\x-4.5))})--cycle;
\draw (A)--(B)--(C)--(D)--cycle (M)--(N);
\draw[dashed] (P)--(Q) (R)--(S);
\draw[<->](td)--(tr);
\node at ($(td)!1/2!(tr)$)[above]{$10$ m};
\draw[<->](dt)--(ct);
\node at ($(dt)!1/2!(ct)$)[above,rotate=90]{$30$ m};
\draw[<->](rp)--(g);
\node at ($(rp)!1/2!(g)$)[above,rotate=-90]{$15$ m};
\foreach \x/\g in {A/90,B/-90,I/180,I'/-40}\draw[fill=black] (\x) circle (.05) +(\g:.5)node{\footnotesize$\x$};
\end{tikzpicture}}
\noindent
Hỏi ông An phải trả bao nhiêu tiền (triệu đồng) để trồng cỏ nhân tạo cho sân bóng.
% \shortans{$151$}
\loigiai{
\immini{
Chọn hệ trục tọa độ như hình vẽ ($I$ là gốc tọa độ). Khi đó đường cong $IAB$ là một parabol có phương trình dạng $y=ax^2$.\\
Parabol đi qua điểm $\left(15;10 \right)$, suy ra
\[a \cdot 15^2=10 \Rightarrow a=\dfrac{2}{45}.\]
}
{
\begin{tikzpicture}[smooth,samples=300,scale=0.6,>=stealth]
\fill[gray!30] (-4,2)--(4,2)--plot[domain=4:-4](\x,{0.125*(\x)^2});
\draw[->] (-5,0)--(5,0) node[below]{$x$};
\draw[->] (0,-1)--(0,3) node[right]{$y$};
\draw (0,0) node[below left]{$I$};
\draw[domain=-4:4] plot(\x,{0.125*(\x)^2});
\draw[fill=black] (4,2) circle(1.5pt) (-4,2) circle(1.5pt);
\draw[dashed] (4,0)node[below]{$15$}--(4,2)node[right]{$B$}--(-4,2)node[left]{$A$}--(-4,0)node[below]{$-15$};

\node[right] at (0,2.4) {$10$};
\end{tikzpicture}}
\noindent
Vậy $y=\dfrac{2}{45}x^2$. Diện tích phần tô đen là $S=2 \cdot \displaystyle\int\limits_{-15}^{15} \left(10-\dfrac{2}{45}x^2 \right) \mathrm{\,d}x=400 \, (\text{m}^2)$.\\
Diện tích phần còn lại của sân bóng là $S_2=30 \cdot 50-400=1100\,(\text{m}^2).$\\
Số tiền Ông An phải trả để trồng cỏ nhân tạo cho sân bóng là
\[130000\times 400+90000\times 1100=151000000\text{ đồng}=151\text{ triệu đồng.}\]}
\end{ex}

\begin{ex}%[2H5C1-7]
Một phần sân nhà bác An có dạng hình thang $ABCD$ vuông tại $A$ và $B$ với độ dài $AB=9$ m, $AD=5$ m và $BC=6$ m như Hình 5.9. Theo thiết kế ban đầu thì mặt sân bằng phẳng và $A$, $B$, $C$, $D$ có độ cao như nhau. Sau đó bác An thay đổi thiết kế để nước có thể thoát về phía góc sân ở vị trí $C$ bằng cách giữ nguyên độ cao ở $A$, giảm độ cao của sân ở vị trí $B$ và $D$ xuống thấp hơn độ cao ở $A$ lần lượt là $6$ cm và $3{,}6$ cm. Để mặt sân sau khi lát gạch vẫn là bề mặt phẳng thì bác An cần phải giảm độ cao ở $C$ xuống bao nhiêu cm so với độ cao ở $A$?
\begin{center}
\includegraphics[scale=.4]{images/2P5-1-H5-9}
\hspace{0.5cm}
\begin{tikzpicture}[scale=0.5, font=\footnotesize,line join=round, line cap=round, >=stealth]
\path
(0,0) coordinate (A)
(9,0) coordinate(B)
(9,-6) coordinate(C)
(0,-5) coordinate(D)
;
\draw[thick] (A)--(B)--(C)--(D)--cycle;
\node [above] at ($(A)!0.5!(B)$) {$9$ m};
\node [right] at ($(B)!0.5!(C)$) {$6$ m};
\node [left] at ($(A)!0.5!(D)$) {$5$ m};
\foreach \i/\g in {A/90,B/90,C/-90,D/-90}{\draw[fill=black](\i) circle (0pt) ($(\i)+(\g:4mm)$) node[scale=1]{$\i$};}
\end{tikzpicture}
\end{center}
% \shortans{$10{,}32$}
\loigiai{
Tại vị trí ban đầu $A$, $B$, $C$, $D$ có độ cao như nhau, chọn hệ trục tọa độ có gốc tọa độ là điểm $A$ và các trục tọa độ lần lượt là $AD$, $AB$ và $Az$, với $Az \perp (ABCD)$.\\
Khi đó $A(0 ; 0 ; 0)$, $D(5; 0 ; 0)$, $B(0; 9 ; 0)$, $C(6; 9 ; 0)$.\\
Sau đó bác An thay đổi thiết kế để nước có thể thoát về phía góc sân ở vị trí $C$ bằng cách giữ nguyên độ cao ở $A$, giảm độ cao của sân ở vị trí $B$ và $D$ xuống thấp hơn độ cao ở $A$ lần lượt là $6$ cm và $3{,}6$ cm.\\
Khi đó, $A(0 ; 0 ; 0)$, $D(5; 0 ; -3{,}6)$, $B(0; 9 ; -6)$.\\
Ta có $\overrightarrow{AB}=(0 ; 9 ; -6)$, $ \overrightarrow{AD}=(5 ; 0 ; -3{,}6)$ là cặp véc-tơ chỉ phương của mặt phẳng $(ABD)$ nên một véc-tơ pháp tuyến của $(ABD)$ là $\left[\overrightarrow{AB}, \overrightarrow{AD}\right]=(-32{,}4 ; -30 ; -45)$.\\
Vậy mặt phẳng $(ABD)$ qua $A(0 ; 0 ; 0)$ và có véc-tơ pháp tuyến $\vec{n}=(-32{,}4 ; -30 ; -45)$ nên có phương trình là
\[-32{,}4 (x-2)-30(y+1)-45(z-3)=0 \qquad \text{hay } -32{,}4 x -30y -45z=0.\]
Để mặt sân sau khi lát gạch vẫn là bề mặt phẳng thì bác An cần phải giảm độ cao ở $C$ xuống $k$ centimét so với độ cao ở $A$ nên suy ra $C(6; 9 ; -k)$.\\
Ta có $A$, $B$, $C$, $D$ đồng phẳng	$\begin{aligned}[t]
&\Leftrightarrow C \in (ABD)\\
&\Leftrightarrow -32{,}4\cdot 6 -30 \cdot 9 -45\cdot (-k)=0\\
&\Leftrightarrow k=10{,}32.
\end{aligned}$\\
Vậy bác An cần phải giảm độ cao ở $C$ xuống $10{,}32$ centimét so với độ cao ở $A$.
}
\end{ex}
\Closesolutionfile{ans}


% \Closesolutionfile{ansbook}
% \HetDe
% \label{De3}
% %
% \cleardoublepage
% \setcounter{page}{1}
% \rfoot{Trang \thepage/\pageref{DA3} - Đáp án trắc nghiệm Mã đề 3}
% \begin{center}
% 	\bfseries ĐÁP ÁN TRẮC NGHIỆM MÃ ĐỀ 3
% \end{center}

% \inputansbox{10}{ans/ansDe3-TN1}
% \inputansbox[3]{2}{ans/ansDe3-TN2}
% \inputansbox{3}{ans/ansDe3-TN3}
% \label{DA3}
% %

% \begin{name}
	{\tenchude}
	{TOÁN 12}
	{LỚP TOÁN THẦY PHÁT}
	{Thời gian: 90 phút - Không kể thời gian phát đề}
\end{name}
\Opensolutionfile{ans}[ans/ansDe4-TN1]
\begin{ex}%[2D4N1-1]%[2015_K12 Huyên Nguyễn]
	Cho $F(x)$ là nguyên hàm của hàm số $f(x)=\sin \left(\dfrac{\pi}{2}-x\right)$. Khi đó $F'(x)$ bằng
	\choice
	{$\cos \left(\dfrac{\pi}{2}-x\right)$}
	{$\sin x$}
	{$-\cos \left(\dfrac{\pi}{2}-x\right)$}
	{\True $\cos x$}
	\loigiai{
		Vì $F(x)$ là một nguyên hàm của hàm số $f(x)$ nên $F'(x)=f(x)=\sin \left(\dfrac{\pi}{2}-x\right)=\cos x$.
	}
\end{ex}

\begin{ex}%[2D4N1-2]%[2015_K12 Huyên Nguyễn]
	Cho $F(x)$ là một nguyên hàm của hàm số $f(x)=\dfrac{1}{x}$ và thỏa mãn $F\left(\mathrm{e}^2\right)=3$. Khi đó $F(x)$ bằng
	\choice
	{$\ln |x|+5$}
	{\True $\ln |x|+1$}
	{$\ln |x|+3$}
	{$\ln |x|-1$}
	\loigiai{
		Ta có $F(x)=\displaystyle\int\dfrac{1}{x}\mathrm{\,d}x=\ln |x|+C$.\\
		Mà $F\left(\mathrm{e}^2\right)=3\Leftrightarrow \ln |\mathrm{e}^2|+C=3\Leftrightarrow C=1$.\\
		Vậy $F(x)=\ln |x|+1$.
	}
\end{ex}

\begin{ex}%[2D4H1-3]
	Cho hàm số $f(x)=3\cos x-\dfrac{2}{x}+\dfrac{4}{{{\sin }^2}x}$. Khẳng định nào dưới đây là đúng?
	\choice
	{\True $\displaystyle\int{f(x)\mathrm{d}x}=3\sin x-2\ln |x|-4\cot x+C$}
	{ $\displaystyle\int{f(x)\mathrm{d}x}=3\sin x-2\ln x-4\cot x+C$}
	{ $\displaystyle\int{f(x)\mathrm{d}x}=3\sin x-2\ln |x|+4\cot x+C$}
	{ $\displaystyle\int{f(x)\mathrm{d}x}=-3\sin x-2\ln |x|-4\cot x+C$}
	\loigiai{
		Ta có $\displaystyle\int{f(x)\mathrm{d}x}=\displaystyle\int{(3\cos x-\dfrac{2}{x}+\dfrac{4}{{\sin^2}x})\mathrm{d}x}$ \\
		$=3\displaystyle\int{\cos x\mathrm{d}x}-2\displaystyle\int{\dfrac{1}{x}\mathrm{d}x}+4\displaystyle\int{\dfrac{1}{{\sin^2}x}\mathrm{d}x}$ \\
		$=3\sin x-2\ln |x|-4\cot x+C$.
	}
\end{ex}

\begin{ex}%[2D4H2-1]
	Biết $F(x)=x^2$ là một nguyên hàm của hàm số $f(x)$ trên $\mathbb{R}$. Giá trị của $\displaystyle  \int\limits_1^2\left[2+f(x)\right]\mathrm{\,d}x$ bằng
	\choice
	{$3$}
	{\True $5$}
	{$\dfrac{13}{3}$}
	{$\dfrac{7}{3}$}
	\loigiai{
		Ta có $\displaystyle  \int\limits_1^2\left[2+f(x)\right]\mathrm{\,d}x$ $=\left.(2x+x^2)\right|_1^2=8-3=5$.
	}
\end{ex}

\begin{ex}%[2D4N3-1]
	Gọi $S$ là diện tích hình phẳng giới hạn bởi các đường $y=3^x$, $y=0$, $x=0$ và $x=2$. Mệnh đề nào dưới đây là đúng?
	\choice
	{$S=\pi \displaystyle\int\limits_{0}^{2}{3^x}\mathrm{\,d}x$}
	{$S=\displaystyle\int\limits_{0}^{2}{3^{2x}}\mathrm{\,d}x$}
	{\True $S=\displaystyle\int\limits_{0}^{2}{3^x}\mathrm{\,d}x$}
	{$S=\pi \displaystyle\int\limits_{0}^{2}{3^{2x}}\mathrm{\,d}x$}
	\loigiai
	{Diện tích hình phẳng giới hạn bởi các đường $y=3^x$, $y=0$, $x=0$ và $x=2$ là $S=\displaystyle\int\limits_{0}^{2}{\left|3^x\right |}\mathrm{\,d}x = \displaystyle\int\limits_{0}^{2}{3^x}\mathrm{\,d}x$.}
\end{ex}

\begin{ex}%[Dự án 2025 - đề cấu trúc mới, Hung Doan]%[2D4H3-3]
	Tính thể tích khối tròn xoay được tạo bởi hình phẳng giới hạn bởi đồ thị hàm số $y=3x-x^2$ và trục hoành khi quay quanh trục hoành.
	\choice
	{$\dfrac{85\pi}{7}$}
	{$\dfrac{8\pi}{7}$}
	{\True $\dfrac{81\pi}{10}$}
	{$\dfrac{41\pi}{7}$}
	\loigiai{
		Phương trình hoành độ giao điểm của đồ thị hàm số $y=3x-x^2$ và trục hoành là \[3x-x^2=0\Leftrightarrow \hoac{&x=0\\&x=3.}\]
		Thể tích của khối tròn xoay là $V=\pi \displaystyle\int \limits_0^3 (3x-x^2)^2\mathrm{\,d}x=\dfrac{81\pi}{10}$.}
\end{ex}

\begin{ex}%[2H5N1-1]%[Dự án Khối 12- Ex-TF-TLN-2024]%[VU Ngoc Hao]
	\immini{
		Cho hình hộp chữ nhật $ABCD.A'B'C'D'$. Bốn véc-tơ pháp tuyến  của mặt phẳng $\left(AA'B'B\right)$ là
		\choice
		{$\overrightarrow{AD}$, $\overrightarrow{A'D'}$, $ \overrightarrow{BD}$, $\overrightarrow{B'C'}$}
		{$\overrightarrow{AD}$, $\overrightarrow{A'D'}$, $ \overrightarrow{BC}$, $\overrightarrow{BC'}$}
		{$\overrightarrow{AC}$, $\overrightarrow{A'D'}$, $ \overrightarrow{BC}$, $\overrightarrow{B'C'}$}
		{\True  $\overrightarrow{AD}$, $\overrightarrow{A'D'}$, $ \overrightarrow{BC}$, $\overrightarrow{B'C'}$}
	}
	{
		\begin{tikzpicture}[scale=0.5, font=\footnotesize,line join=round, line cap=round, >=stealth]
			\coordinate (A) at (0,0);
			\coordinate (B) at (-2,-1.5);
			\coordinate (D) at (5,0);
			\coordinate (C) at ($(B)+(D)-(A)$);
			\foreach \i in {A,B,C,D}{\coordinate (\i') at ($(\i)+(0,4)$);}
			\draw (A')--(B')--(C')--(D')--cycle;
			\draw (B)--(B') (C)--(C') (D)--(D')  (B)--(C)--(D);
			\draw[dashed,thin](B)--(A)--(A') (A)--(D);
			\foreach \i/\g in {A'/90,B'/90,C'/90,D'/90,A/-90,B/-90,C/-90,D/-90}{\draw[fill=black](\i) circle (1pt) ($(\i)+(\g:5mm)$) node[scale=1]{$\i$};}
		\end{tikzpicture}
	}
	\loigiai{
		Bốn véc-tơ pháp tuyến của mặt phẳng $\left(AA'B'B\right)$ là  $\overrightarrow{AD}$, $\overrightarrow{A'D'}$, $ \overrightarrow{BC}$, $\overrightarrow{B'C'}$.
	}
\end{ex}

\begin{ex}%[2H5N1-2]
	Trong không gian với hệ tọa độ $Oxyz$, cho mặt phẳng $(P)\colon x-y+3=0$. Véc-tơ nào sau đây \textbf{không phải} là véc-tơ pháp tuyến của mặt phẳng $(P)$?
	\choice
	{$\overrightarrow{a}=(3;-3;0)$}
	{\True $\overrightarrow{a}=(1;-1;3)$}
	{$\overrightarrow{a}=(1;-1;0)$}
	{$\overrightarrow{a}=(-1;1;0)$}
	\loigiai{
		Véc-tơ pháp tuyến của mặt phẳng $(P)$ là $\overrightarrow{n}=(1;-1;0)$.\\
		Ta có $\overrightarrow{a}=(-1;1;0)=-(1;-1;0)=-\overrightarrow{n}$. Vậy $\overrightarrow{a}=(-1;1;0)$ là một véc-tơ pháp tuyến của mặt phẳng $(P)$.\\
		Tương tự $\overrightarrow{a}=(3;-3;0)=3(1;-1;0)=3\overrightarrow{n}$. Vậy $\overrightarrow{a}=(3;-3;0)$ là một véc-tơ pháp tuyến của mặt phẳng $(P)$.\\
		Do véc-tơ $\overrightarrow{a}=(1;-1;3)$ không cùng phương với véc-tơ $\overrightarrow{n}=(1;-1;0)$. Nên $\overrightarrow{a}=(1;-1;3)$ không là véc-tơ pháp tuyến của mặt phẳng $(P)$.
	}
\end{ex}

\begin{ex}%[2H5H1-3]
	Trong không gian $O x y z$, phương trình của mặt phẳng $(P)$ đi qua điểm $B(2 ; 1 ;-3)$, đồng thời vuông góc với hai mặt phẳng $(Q)\colon x+y+3 z=0,$ $(R)\colon 2 x-y+z=0$ là
	\choice
	{$4 x+5 y-3 z+22=0$}
	{$4 x-5 y-3 z-12=0$}
	{$2 x+y-3 z-14=0$}
	{\True $4 x+5 y-3 z-22=0$}
	\loigiai{
		Mặt phẳng $(Q)\colon x+y+3 z=0,$ $(R)\colon 2 x-y+z=0$ có các véc-tơ pháp tuyến lần lượt là $\overrightarrow{n}_1=(1 ; 1 ; 3)$ và $\overrightarrow{n}_2=(2 ;-1 ; 1)$.\\
		Vì $(P)$ vuông góc với hai mặt phẳng $(Q),$ $(R)$ nên $(P)$ có véc-tơ pháp tuyến là $\vec{n}=\left[\overrightarrow{n}_1, \overrightarrow{n}_2\right]=(4 ; 5 ;-3)$.\\
		Ta lại có $(P)$ đi qua điểm $B(2 ; 1 ;-3)$ nên \[(P)\colon 4(x-2)+5(y-1)-3(z+3)=0\Leftrightarrow 4 x+5 y-3 z-22=0.\]
	}
\end{ex}

\begin{ex}%[Dự án EX-Ôn Tập TN 2025, Đoàn Hùng]%[2H5N2-1]
	Trong không gian toạ độ, phương trình nào sau đây là phương trình tham số của đường thẳng?
	\choice
	{$\heva{&x=2+t^2\\&y=3-t\\&z=4+t}$}
	{$\heva{&x=2+y\\&y=3-t^2\\&z=-4+2t}$}
	{$\heva{&x=2+t\\&y=3-t\\&z =t^2}$}
	{\True $\heva{&x=2+3t\\&y=4+5t\\&z=5+6t}$}
	\loigiai{
		Phương trình tham số của đường thẳng có dạng $\heva{&x=x_0+at\\&y=y_0+bt\\&z=z_0+ct}$ với $a^2+b^2+c^2\neq0$. Do đó đáp án cần chọn là
		\[\heva{&x=2+3t\\&y=4+5t\\&z=5+6t}.\]
	}
\end{ex}

\begin{ex}%[2H5N2-7]
	Trong không gian với hệ tọa độ $Oxyz$, tính góc giữa hai đường thẳng $d_1\colon\dfrac{x}{1}=\dfrac{y+1}{-1}=\dfrac{z-1}{2}$ và $d_2\colon\dfrac{x+1}{-1}=\dfrac{y}{1}=\dfrac{z-3}{1}$.
	\choice
	{$45^\circ $}
	{$30^\circ $}
	{$60^\circ $}
	{\True $90^\circ $}
	\loigiai{
		Ta có $\overrightarrow{u}_{d_1}=\left( 1;-1;2 \right)$ và
		$\overrightarrow{u}_{d_2}=\left( -1;1;1 \right)$ lần lượt là véc tơ chỉ phương của $d_1$ và $d_2$.\\
		$\overrightarrow{u}_{d_1}\cdot\overrightarrow{u}_{d_2}=1\cdot\left( -1 \right)+\left( -1 \right)\cdot1+2\cdot1=0\Rightarrow {d_1}\bot {d_2}\Rightarrow \left( \widehat{d_1,d_2} \right)=90^\circ $.}
\end{ex}

\begin{ex}%[2H5H2-4]
	Trong không gian $Oxyz,$ cho hai đường thẳng $d_1\colon \dfrac{x-1}{2}=\dfrac{y}{1}=\dfrac{z+2}{-2}$, $d_2\colon \dfrac{x+2}{-2}=\dfrac{y-1}{-1}=\dfrac{z}{2}$. Xét vị trí tương đối của hai đường thẳng đã cho.
	\choice
	{Chéo nhau}
	{Trùng nhau}
	{\True Song song}
	{Cắt nhau}
	\loigiai{
		Ta có $\heva{& \overrightarrow{u}_{d_1}=(2;1;-2) \\ & \overrightarrow{u}_{d_2}=(-2;-1;2 )} \Rightarrow \overrightarrow{u}_1=-\overrightarrow{u}_2 $. Do đó $d_1$ song song hoặc trùng với $d_2$.\\
		Gọi điểm $M( 1;0;-2 )\in d_1 $ thay $M$ vào $d_2$ ta được $\dfrac{1+2}{-2}=\dfrac{0-1}{-1}=\dfrac{-2}{2}$ (vô lí).\\
		Vậy $d_1\parallel d_2$.
	}
\end{ex}
\Closesolutionfile{ans}

\TNTF
\Opensolutionfile{ans}[ans/ansDe4-TN2]
\begin{ex}%[DA-EX-TF-TLN2024 - Trần Xuân Hòa]%[2D4H2-4]
	Cho hàm số $f(x)=\heva{&\mathrm{e}^{2x} \text{ khi }x\ge 0\\&x^2+x+2\text{ khi }x<0}$.
	\choiceTF
	{Giá trị $\displaystyle\int\limits_0^1f(x)\mathrm{\,d}x=4$}
	{\True Giá trị $\displaystyle\int\limits_{-1}^0f(x)\mathrm{\,d}x=\dfrac{11}{6}$}
	{Giá trị của $m, m>0$ để $\displaystyle\int\limits_m^{1}f(x)\mathrm{\,d}x=4$ bằng $1$}
	{\True Biết $\displaystyle\int\limits_{-1}^1f(x)\mathrm{\,d}x=\dfrac{a}{b}+\dfrac{\mathrm{e}^2}{c}$ với $\dfrac{a}{b}$ tối giản. Giá trị $a+b+c$ bằng $9$}
	\loigiai{
		\begin{itemchoice}
			\itemch Sai. Vì $\displaystyle\int\limits_0^1f(x)\mathrm{\,d}x=\displaystyle\int\limits_0^1\mathrm{e}^{2x}\mathrm{\,d}x=\dfrac{1}{2}\mathrm{e}^{2x}\bigg|_0^1=\dfrac{1}{2}(\mathrm{e}^2-1)$.
			\itemch Đúng. Vì $\displaystyle\int\limits_{-1}^0f(x)\mathrm{\,d}x=\displaystyle\int\limits_{-1}^0(x^2+x+2)\mathrm{\,d}x=\left(\dfrac{x^3}{3}+\dfrac{x^2}{2}+2x\right)\Bigg|_0^1=\dfrac{11}{6}$
			\itemch Sai. Vì $\displaystyle\int\limits_m^{1}f(x)\mathrm{\,d}x=4\Leftrightarrow \left(\dfrac{1}{2}\mathrm{e}^{2x}\right)\Bigg|_m^1=4\Leftrightarrow \dfrac{1}{2}(\mathrm{e}^2-\mathrm{e}^{2m})=4\Leftrightarrow \mathrm{e}^{2m}=\mathrm{e}^2-8<0$ (vô nghiệm).\\
			Vậy không có $m$ thỏa mãn.
			\itemch Đúng. \begin{eqnarray*}
				\displaystyle\int\limits_{-1}^1f(x)\mathrm{\,d}x&=&\displaystyle\int\limits_{-1}^0f(x)\mathrm{\,d}x+\displaystyle\int\limits_{0}^1f(x)\mathrm{\,d}x\\
				&=&\displaystyle\int\limits_{-1}^0(x^2+x+2)\mathrm{\,d}x+\displaystyle\int\limits_{0}^1\mathrm{e}^{2x}\mathrm{\,d}x\\
				&=&\dfrac{11}{6}+\dfrac{1}{2}(\mathrm{e}^2-1)\\
				&=&\dfrac{\mathrm{e}^2}{2}+\dfrac{4}{3}.
			\end{eqnarray*}
			Do đó $a=4$, $b=3$, $c=2$. Giá trị $a+b+c=9$.
		\end{itemchoice}
	}
\end{ex}

\begin{ex}%[2H5H2-5]
	Trong không gian $O x y z$ cho đường thẳng $d$ có phương trình tham số $\heva{&x=-1+2 t \\& y=1+t \\& z=3-2 t.}$
	\choiceTF
	{\True Phương trình chính tắc của đường thẳng $d$ là $\dfrac{x+1}{2}=\dfrac{y-1}{1}=\dfrac{z-3}{-2}$}
	{\True  Đường thẳng $d$ đi qua điểm $A(-1 ; 1 ; 3)$}
	{Véc-tơ $\overrightarrow{a}=(4 ; 2 ;-3)$ là một véc-tơ chỉ phương   của đường thẳng $d$}
	{Giao điểm của đt $d$ và mặt phẳng $(P)\colon x+2 y-3 z+2=0$  là $I(0 ; 1 ; 2)$}
	\loigiai{
		\begin{itemchoice}
			\itemch \textbf{Đúng.} Vì  đường thẳng $d$ đi qua điểm $M(-1 ; 1 ; 3)$ và có một  véc-tơ chỉ phương   là $\overrightarrow{u}_d=(2 ; 1 ;-2)$ nên có phương trình chính tắc là $\dfrac{x+1}{2}=\dfrac{y-1}{1}=\dfrac{z-3}{-2}$.
			\itemch \textbf{Đúng.} Vì theo phương trình tham số của đường thẳng $d$ thì $d$ đi qua điểm $A(-1 ; 1 ; 3)$.
			\itemch \textbf{Sai.} Vì véc-tơ chỉ phương của đường thẳng $d$ là $\overrightarrow{u}_d=(2 ; 1 ;-2)$. \\Xét hai véc-tơ $\overrightarrow{a}=(4 ; 2 ;-3)$ và $\overrightarrow{u}_d=(2 ; 1 ;-2)$.\\
			Vì $\dfrac{1}{2} \neq \dfrac{-2}{-3}$ nên $\overrightarrow{a}=(4 ; 2 ;-3)$ và $\overrightarrow{u}_d$ không cùng phương. \\Do đó $\overrightarrow{a}$ không là véc-tơ chỉ phương của đường thẳng $d$.
			\itemch \textbf{Sai.} Vì  ta có $-1+2 t+2(1+t)-3(3-2 t)+2=0$ $\Leftrightarrow 10 t-6=0 \Leftrightarrow t=\dfrac{3}{5}$.\\
			Giao điểm của $d$ và $(P)$ là $B\left(\dfrac{1}{5} ; \dfrac{8}{5} ; \dfrac{9}{5}\right)$.\end{itemchoice}
	}
\end{ex}
\Closesolutionfile{ans}

\TNSA
\Opensolutionfile{ans}[ans/ansDe4-TN3]
\begin{ex}%[2D4H1-1]%[Đào Trung Kiên]
	Giả sử $F(x)$ là một nguyên hàm của hàm số $f(x)=\mathrm{e}^x$, biết $F(0)=4$. Tìm $F(1)$ (làm tròn kết quả tới phần mười).
	\shortans[]{$5,7$}
	\loigiai{
		Do $F(x)$ là một nguyên hàm của $f(x)=\mathrm{e}^x$ nên $F(x)=\mathrm{e}^x+C$.\\
		Lại có $F(0)=4$ nên $C=3$ hay $F(x)=\mathrm{e}^x+3$ nên $F(1)=\mathrm{e}+3\approx 5{,}7$.
	}
\end{ex}

\begin{ex} %[2D4V3-3]
	Cho tam giác vuông $OAB$ có cạnh $OA$ nằm trên trục $Ox$ và $\widehat{AOB}=\alpha \, (0<\alpha <\frac{\pi }{2})$ và $B(a;b)$ với $a$, $b$ là các số thực thỏa ${a^2}+{b^2}=1$. Gọi $\beta $ là khối tròn xoay sinh ra khi quay miền tam giác $OAB$ xung quanh trục $Ox$.
	\begin{center}
	\begin{tikzpicture}[scale=1,line join=round, line cap=round,>=stealth,declare function={rt=.03;xmin=-2.5;xmax=8;ymin=-2.5;ymax=3;a=2.5;b=a/3;}]
	\path (0,0) coordinate (O)
	(6,0)coordinate (A)++(90:a) coordinate (B)
	;
	\draw[->] (0,ymin)--(0,ymax) node[right] {$y$};
	\draw[->] (0,0)--(-150:2) node[above left] {$z$};
	\draw[fill=black] (0,0) circle (rt);
	\fill[red!15] (O)--(B)--(A)--cycle;
	\draw[->]  (A)--(xmax,0) node[below] {$x$};
	\draw (A) ellipse ({b} and {a})
	(-1,0)--(0,0) node[above left]{$O$} (O)--(B)--(A) (O)--($(A)+(-90:a)$)
	pic["$\alpha$",angle radius=15mm] {angle = A--O--B}
	pic[draw,angle radius=7mm] {angle = A--O--B}
	;
	\draw[dashed](O)--(A);
	\foreach \p/\g in {A/-90,B/90}
	\fill (\p) circle(1pt)+(\g:.3) node{$\p$};
	\end{tikzpicture}
	\end{center}
	Tính giá trị $\tan \alpha $ khi thể tích của khối $\beta $ đạt giá trị lớn nhất (Làm tròn kết quả đến chữ số thập phân thứ 2).
	\shortans[1]{$1{,}41$}
	\loigiai{
	Do $OB$ đi qua gốc tọa độ nên ta đặt $OB\colon y=kx$ với $ k$ là số thực dương.\\
	Do $OB$ đi qua $B(a;b) \Rightarrow OB\colon y=\frac{b}{a}x$ và $\tan \alpha =\frac{b}{a}$.\\
	Khi đó, $V=\pi \int\limits_0^a{{\left(  \dfrac{b}{a}x\right) ^2}\mathrm{\,d}x}=
	\pi \int\limits_0^a{\dfrac{{b^2}}{{a^2}}{x^2}\mathrm{\,d}x}=
	\dfrac{\pi {b^2}{x^3}}{3{a^2}} \bigg|_0^a=
	\dfrac{\pi }{3}{b^2}a$.\\
	Áp dụng bất đẳng thức Am – Gm:\\
	\[{V^2}=\dfrac{{{\pi }^2}}{9}{b^4}{a^2}=\dfrac{{{\pi }^2}}{18}{b^2}{b^2}2{a^2}\le \dfrac{{{\pi }^2}}{18}\dfrac{{{( {b^2}+{b^2}+2{a^2} )}^3}}{27}=\dfrac{4{{\pi }^2}}{243}\\
	\Rightarrow V\le \frac{2\pi \sqrt {3}}{27}.\]
	Đẳng thức xảy ra khi và chỉ khi ${b^2}=2{a^2} \Rightarrow \dfrac{b}{a}=\tan \alpha =\sqrt {2} \approx 1{,}41$.}
	\end{ex}

\begin{ex}%[2H5H2-7]
	Trong không gian với hệ tọa độ $Oxyz$, cho điểm $A(3;-1;0)$ và đường thẳng $d\colon \dfrac{x-2}{-1}=\dfrac{y+1}{2}=\dfrac{z-1}{1}$. Phương trình mặt phẳng $(\alpha)$ chứa $d$ sao cho khoảng cách từ $A$ đến $(\alpha)$ lớn nhất có dạng $ax+by+cz=0$. Khi đó $\dfrac{a}{b}$ bằng
	\shortans{$1$}
	\loigiai{
		Gọi $H$ là hình chiếu của $A$ lên $d$.\\
		Khi đó $H(2-t;-1+2t;1+t)\Rightarrow \overrightarrow{AH}=(-1-t;2t;1+t)$.\\
		Do $AH\perp d$ nên $ -(-1-t)+2\cdot 2t+1+t=0\Leftrightarrow t=-\dfrac{1}{3}$. Khi đó $\overrightarrow{AH}=\left(-\dfrac{2}{3};-\dfrac{2}{3};\dfrac{2}{3}\right)$.\\
		Mặt phẳng $(\alpha)$ chứa $d$ sao cho khoảng cách từ $A$ đến $(\alpha)$ lớn nhất khi $AH\perp (\alpha)$.\\
		Do đó $(\alpha)$ có véc-tơ pháp tuyến là $\overrightarrow{n}=(1;1;-1)$.\\
		Vậy $(\alpha)\colon 1(x-2)+1(y+1)-1(z-1)=0\Leftrightarrow x+y-z=0$.\\ Do đó $a=1$, $b=1$, $c=-1$ và $\dfrac{a}{b}=1$.}
\end{ex}

\begin{ex}%[2H5V1-7]
	\immini{
		Một ngôi nhà có nền nhà là hình vuông, cạnh là $7$ mét. Các vách tường hình vuông và vị trí cao nhất trên mái nhà cách sàn nhà $8$ mét. Biết rằng hai mái nhà là hai hình chữ nhật bằng nhau. Khi gắn hệ trục tọa độ $Oxyz$ (đơn vị trên mỗi trục tính theo mét) vào một căn nhà sao cho nền nhà thuộc mặt phẳng $\left(Oxy\right)$, người ta coi mỗi mái nhà là một phần của mặt phẳng. Góc giữa mái nhà bên phải và nền nhà bằng bao nhiêu độ (làm tròn kết quả đến hàng đơn vị)?}{\begin{tikzpicture}[scale=0.6, font=\footnotesize, line join=round, line cap=round, >=stealth]
			\def\bc{6} % cạnh BC
			\def\ba{3} % cạnh BA
			\def\gocB{35} % góc B của đáy
			\coordinate[label=below left:$O(0;0;0)$] (O) at (0,0);
			\coordinate[label=above left:$C$] (C) at (\gocB:\ba);
			\coordinate[label=below:$A$] (A) at (\bc,0);
			\coordinate[label=right:$B(7;7;0)$] (B) at ($(A)-(O)+(C)$);
			\coordinate[label=above left:$H$] (H) at ($(C)+(90:\bc)$);
			\coordinate[label=left:$E(0;0;7)$] (E) at ($(O)-(C)+(H)$);
			\coordinate[label=below right:$F$] (F) at ($(A)-(C)+(H)$);
			\coordinate[label=right:$G$] (G) at ($(B)-(C)+(H)$);
			\coordinate (P) at (3,7);
			\coordinate[label=above:$Q\left(\dfrac{7}{2}; 7; 8 \right)$] (Q) at ($(P)-(E)+(H)$);
			\draw (E)--(O)--(A)--(B)--(G) (H)--(E)--(F)--(G) (A)--(F) (E)--(P)--(F) (H)--(Q)--(G) (P)--(Q);
			\draw[dashed] (H)--(C)--(B) (C)--(O)  (Q)--(G)--(H);
			\foreach \diem in {C,O,A,B,H,E,F,G,P, Q}	\fill (\diem)circle(1.5pt);
			\draw[black,->,>=stealth] (O)--($(O)!1.25!(C)$);
			\draw[black,->,>=stealth] (O)--($(O)!1.25!(E)$);
			\draw[black,->,>=stealth] (O)--($(O)!1.25!(A)$);
			\path (O)--($(O)!1.15!(C)$)node[pos=1.05,sloped,black,above]{$y$};
			\path (O)--($(O)!1.15!(E)$)node[pos=1.05,black,right]{$z$};
			\path (O)--($(O)!1.15!(A)$)node[pos=1.05,sloped,black,above]{$x$};
		\end{tikzpicture}}
	\shortans{$16$}
	\loigiai{
		Gắn hệ trục tọa độ $Oxyz$ như hình vẽ. Theo giả thiết, ta có $F(7;0;7)$, $G(7;7;7)$ và $Q\left(\dfrac{7}{2}; 7;8\right)$
		Ta gọi $\alpha$ là góc giữa mái nhà bên phải và nền nhà.\\ Khi đó $\alpha = \left(\left(QGF\right),(Oxy)\right)$.\\
		Với $\left(QGF\right)$ nhận véc-tơ $\overrightarrow{n}=\left[\overrightarrow{FG},\overrightarrow{GQ}\right]$ với $\overrightarrow{FG}=\left(0;7;0\right)$ và $\overrightarrow{GQ}=\left(\dfrac{7}{2};0;1\right)$. Khi đó $\overrightarrow{n}=\left(7;0;-\dfrac{49}{2}\right)= \dfrac{7}{2} \left( 2;0;-7\right)$.\\
		Với $\left(Oxy\right)$  nhận $\overrightarrow{k}=\left(0;0;1\right)$ làm véc-tơ pháp tuyến.\\
		Vậy $\cos \alpha =\dfrac{7}{\sqrt{53}} $.\\ Vậy $\alpha \approx 16^{\circ}$.
	}
\end{ex}

\TL
\begin{ex}%[2H5H2-3]%[Dự án EX-TF-TLN lần 3 - Nguyen Chín Em]
	Trong không gian với hệ tọa độ $Oxyz$. Viết phương trình chính của $d$ biết đường thẳng $d$ đi qua $A(-1; 1; -3)$ và $B(-2; 2; 0)$. Đường thẳng $AB$ đi qua điểm $(x_0;0;z_0)$. Tính $x_0+z_0$.
	% \shortans{$-6$}
	\loigiai{
		Đường thẳng $d$ đi qua $A, B$ nên có véc-tơ chỉ phương $\overrightarrow{AB} = (-1; 1; 3)$.\\
		Phương trình chình tắc của $d\colon \dfrac{x+1}{-1}=\dfrac{y-1}{1}=\dfrac{z+3}{3}$.\\
		Cho $y=0$, suy ra được $x_0=0$, $z_0=-6$. Khi đó $x_0+z_0=-6$.
	}
\end{ex}

\begin{ex}%[EX-Ôn Tập TN 2025, Võ Thanh Phong]%[2D4C3-2]
	Một cổng có đạng hình parabol với chiều cao $8$ m, chiều rộng chân đế $8$ m. Người ta căng hai sợi dây trang trí $AB$,  $CD$ nằm ngang, đồng thời chia cổng thành ba phần sao cho hai phần ở phía trên có diện tích bằng nhau. Tỉ số $\dfrac{CD}{AB}$ bằng bao nhiêu (làm tròn kết quả đến hàng phần trăm)?
	\begin{center}
		\begin{tikzpicture}[scale=0.7, font=\footnotesize, line join=round, line cap=round,>=stealth]
			\begin{scope}
				\clip (-4,-4.5) rectangle (4,0);
				\draw [smooth,domain=-5:4, samples=200] plot (\x, {-0.5*(\x)^2});
			\end{scope}
			\draw [<->] (3.3,0)--(3.3,-4.5);
			\node[right] at (3.3,-2){$8\,\,m$};
			\draw [<->] (-3,-4.5)--(3,-4.5);
			\node[below] at (0,-4.5){$8\,\,m$};
			\node[left] at (-1.8,-1.62){$A$};
			\node[right] at (1.8,-1.62){$B$};
			\node[left] at (-2.5,-3.125){$C$};
			\node[right] at (2.5,0-3.125){$D$};
			\draw [-] (-1.8,-1.62)--(1.8,-1.62);
			\draw [-] (-2.5,-3.125)--(2.5,0-3.125);
			\draw [dashed] (0,0)--(4,0);
			\node[below] at (current bounding box.south){\textit{Hình 5}};
		\end{tikzpicture}
	\end{center}
	% \shortans{ $1{,}26$                 }
	\loigiai{
	\immini{Gắn hệ trục tọa độ $Oxy$ vào cổng parabol như hình bên với trục $Oy$ trùng với đường đối xứng của parabol, gốc $O$ nằm ở đỉnh của parabol, đơn vị trên mỗi trục tính theo mét. Khi đó, phương trình parabol có dạng $y=ax^2$.\\
		Vì parabol đi qua điểm có toạ độ $(-4 ;-8)$ nên $a=-\dfrac{1}{2}$. Suy ra phương trình parabol là $y=-\dfrac{1}{2} x^2$.\\}{
		\begin{tikzpicture}[scale=0.7, font=\footnotesize, line join=round, line cap=round,>=stealth]
			\draw[->] (-4,0) --(4,0) node[below]{$x$};
			\draw[->] (0,-5) --(0,1) node[left]{$y$};
			\draw (0,0) node[above left=-3pt]{$O$};
			\begin{scope}
				\clip (-4,-4.5) rectangle (4,0);
				\draw [smooth,domain=-5:4, samples=200] plot (\x, {-0.5*(\x)^2});
			\end{scope}
			\draw [<->] (3.3,0)--(3.3,-4.5);
			\node[right] at (3.3,-2){$8\,\,m$};
			\draw [<->] (-3,-4.5)--(3,-4.5);
			\node[below] at (0,-4.5){$8\,\,m$};
			\node[left] at (-1.8,-1.62){$A$};
			\node[right] at (1.8,-1.62){$B$};
			\node[left=-3pt] at (-2.5,-3.125){$C$};
			\node[right] at (2.5,0-3.125){$D$};
			\draw [-] (-1.8,-1.62)--(1.8,-1.62);
			\draw [-] (-2.5,-3.125)--(2.5,0-3.125);
			\draw [dashed] (0,0)--(4,0) (1.8,-1.62)--(1.8,0) (2.5,0-3.125)--(2.5,0) (-3,-4.5)--(-3,0);
			\node[above] at (1.8,0){$x_1$};
			\node[above] at (2.5,0){$x_2$};
			\node[above] at (-3,0){$-4$};
		\end{tikzpicture}}
	Giả sử $B$ có hoành độ $x_1$,  $D$ có hoành độ $x_2$. Khi đó, phương trình đường thẳng $AB$ là $y=-\dfrac{1}{2} x_1^2$, phương trình đường thẳng $CD$ là $y=-\dfrac{1}{2} x_2^2$.\\
	Diện tích hình phẳng giới hạn bởi parabol và đường thẳng $AB$ là
	\[
		S_1=2\displaystyle\int\limits_0^{x_1}\left[-\dfrac 12x^2-\left(-\dfrac 12x_1^2\right)\right]\mathrm{\,d} x=\left.2\left(-\dfrac{x^3}6+\dfrac{x_1^2}2x\right)\right|_0^{x_1}=\dfrac 23x_1^3\,\,\left(\mathrm{m}^2\right).
	\]
	Diện tích hình phẳng giới hạn bởi parabol và đường thẳng $CD$ là
	\[
		S_2=2\displaystyle\int\limits_0^{x_2}\left[-\dfrac 12x^2-\left(-\dfrac 12x_2^2\right)\right]\mathrm{\,d} x=\left.2\left(-\dfrac{x^3}6+\dfrac{x_2^2}2x\right)\right|_0^{x_2}=\dfrac 23x_2^3\,\,\left(\mathrm{m}^2\right).
	\]
	Theo giả thiết, ta có  $S_2=2S_1\Leftrightarrow x_2^3=2x_1^3\Leftrightarrow\dfrac{x_2}{x_1}=\sqrt[3]2\approx 1{,}26$.\\
	Khi đó, $\dfrac{CD}{AB}=\dfrac{2x_2}{2x_1}\approx 1{,}26$.
	}
\end{ex}

\begin{ex}%[2H5V1-7]
	Từ mặt nước trong một bể nước, tại ba vị trí đôi một cách nhau $2$ m, người ta lần lượt thả dây dọi để quả dọi chạm đáy bể. Phần dây dọi (thẳng) nằm trong nước tại ba vị trí đó lần lượt có độ dài $4$ m; $4{,}4$ m; $4{,}8$ m. Biết đáy bể là phẳng. Hỏi đáy bể nghiêng so với mặt phẳng nằm ngang một góc bao nhiêu độ (làm tròn đến hàng phần chục)?
	% \shortans{$21{,}8$}
	\loigiai{
	Gọi ba vị trí trên mặt nước là $A$, $B$, $C$ thì tam giác $ABC$ là tam giác đều cạnh bằng $2$ m. Gọi dây dọi lần lượt là $AA'$, $BB'$, $CC'$ có độ dài lần lượt là $4$ m; $4{,}4$ m; $4{,}8$ m.\\
	Chọn hệ trục toạ độ $Oxyz$ sao cho $O$ là trung điểm của $BC$, tia $Ox$ chứa điểm $A$, tia $Oy$ chứa điểm $B$, tia $Oz$ đi qua trung điểm của $B'C'$ và đơn vị trên các trục là mét.\\
	Ta có $OB=OC=1$, $OA=\sqrt{3}$ $\Rightarrow$ $A'\left(\sqrt{3};0;4\right)$, $B'(0;1;4{,}4)$, $C'(0;-1;4{,}8)$.\\
	Khi đó, $\overrightarrow{A'B'}=\left(-\sqrt{3};1;0{,}4\right)$, $\overrightarrow{A'C'}=\left(-\sqrt{3};-1;0{,}8\right)$.\\
	Mặt phẳng $(A'B'C')$ có một véc-tơ pháp tuyến là $\overrightarrow{n}=\left[\overrightarrow{A'B'},\overrightarrow{A'C'}\right]=0{,}4\sqrt{3}\left(\sqrt{3};1;5\right)$.\\
	Mặt phẳng $(ABC)$ có một véc-tơ pháp tuyến là $\overrightarrow{k}=(0;0;1)$.\\
	Do đó, $\cos\big((ABC),(A'B'C')\big)=\left|\cos\left(\overrightarrow{n},\overrightarrow{k}\right)\right|=\dfrac{5}{\sqrt{29}}$. Góc cần tìm gần bằng $21{,}8^\circ$.
	}
\end{ex}
\Closesolutionfile{ans}


% \Closesolutionfile{ansbook}
% \HetDe
% \label{De4}
% %
% \cleardoublepage
% \setcounter{page}{1}
% \rfoot{Trang \thepage/\pageref{DA4} - Đáp án trắc nghiệm Mã đề 4}
% \begin{center}
% 	\bfseries ĐÁP ÁN TRẮC NGHIỆM MÃ ĐỀ 4
% \end{center}

% \inputansbox{10}{ans/ansDe4-TN1}
% \inputansbox[3]{2}{ans/ansDe4-TN2}
% \inputansbox{3}{ans/ansDe4-TN3}
% \label{DA4}
% %


%%CK2
% \begin{name}
	{\tenchude}
	{TOÁN 12}
	{LỚP TOÁN THẦY PHÁT}
	{Thời gian: 90 phút - Không kể thời gian phát đề}
\end{name}
\Opensolutionfile{ansbook}[ans/ansbookDe1]
\TN
\Opensolutionfile{ans}[ans/ansDe1-TN1]
\begin{ex}%[2D4N1-1]
Khẳng định nào sau đây là \textbf{sai}?
\choice
{ Mọi hàm số $f(x)$ liên tục trên đoạn $[a;b]$ đều có nguyên hàm trên đoạn $[a;b]$}
{\True $\displaystyle\int x^\alpha \mathrm{d}x=\dfrac{x^{\alpha +1}}{\alpha +1}+C$ ($C$ là hằng số, $\alpha $ là hằng số)}
{$\displaystyle\int \mathrm{e}^x\mathrm{d}x=\mathrm{e}^x+C$ ($C$ là hằng số)}
{$\displaystyle\int{\dfrac{1}{x}\mathrm{d}x=\ln \left| x \right|+C}$ ($C$ là hằng số) với $x\ne 0$}
\loigiai{
$\displaystyle\int x^{\alpha} \mathrm{d}\,x=\dfrac{x^{\alpha +1}}{\alpha +1}+C$ ($C$ là hằng số, $\alpha $ là hằng số và $\alpha \ne -1$).}
\end{ex}

\begin{ex}%[2D4N1-2]
Tìm họ nguyên hàm $F(x)$ của hàm số $f(x)=\dfrac{1}{x}$.
\choice
{\True  $F(x)=\ln \left| x \right|+C$}
{$F(x)=\ln x+C$}
{$F(x)=\ln \left| x \right|$}
{$F(x)=-\dfrac{1}{x^2}+C$}
\loigiai{
Áp dụng công thức nguyên hàm của hàm số ta có $\displaystyle\int{\frac{1}{x}\mathrm{d}\,x}=\ln \left| x \right|+C$.}
\end{ex}

\begin{ex}%[2D4N2-1]
Nếu $\displaystyle\int\limits_0^2f(x)\mathrm{\,d}x=5$ thì $\displaystyle\int\limits_0^2\left[2f(x)-1\right]\mathrm{\,d}x$ bằng
\choice
{\True $8$}
{$9$}
{$10$}
{$12$}
\loigiai{
Ta có $\displaystyle\int _0^2\left[2f(x)-1\right]\mathrm{\,d}x=2\displaystyle\int _0^2f(x)\mathrm{\,d}x-\displaystyle\int _0^21\mathrm{\,d}x=2\cdot 5-2=8$.}
\end{ex}

\begin{ex}%[2H5N1-1]
Trong không gian $Oxyz$, phương trình của mặt phẳng $(Oxy)$ là
\choice
{\True $z=0$}
{$x=0$}
{$y=0$}
{$x+y=0$}
\loigiai{
Phương trình của mặt phẳng $(Oxy)$ là $z=0$.
}
\end{ex}

\begin{ex}%[Mức độ 1]%[BG-12-New-4in1, Hiệp Hà]%[2H5N1-2]
Cho $(\alpha)$ vuông góc với giá của $\vec{a}=(-4;2;6)$. Vectơ nào dưới đây là một vectơ pháp tuyến của $(\alpha)$?
\choice
{$\vec{n_1}=(2;1;3)$}
{\True $\vec{n_2}=(-2;1;3)$}
{$\vec{n_3}=(4;-2;6)$}
{$\vec{n_4}=(4;2;-6)$}
\loigiai{
$(\alpha)$ vuông góc với giá của $\vec{a}=(-4;2;6)$ nên $\vec{a}$ là một vectơ pháp tuyến của $(\alpha)$.\\
Do đó $\vec{n_2}=\dfrac{1}{2}\vec{a}$ cũng là một vectơ pháp tuyến của $(\alpha)$.
}
\end{ex}

\begin{ex}%[2H5N2-1]
Trong không gian $Oxyz$, điểm nào dưới đây thuộc đường thẳng $d\colon\heva{& x=1+2t\\ & y=-3+t\\ & z=4+5t}$?
\choice
{$Q(4;1;3)$}
{$P(3;-2;-1)$}
{$N(2;1;5)$}
{\True $M(1;-3;4)$}
\loigiai{
Dễ thấy đường thẳng $d$ đi qua điểm $M(1;-3;4)$.
}
\end{ex}

\begin{ex}%[2025-TLOT-2018,Trần Xuân Hòa]%[2H5N2-2]
Trong không gian $O x y z$, vectơ nào sau đây là một vectơ chỉ phương của đường thẳng $\Delta\colon \dfrac{x-5}{8}=\dfrac{y-9}{6}=\dfrac{z-12}{3}$?
\choice
{\True $\overrightarrow{u}_1=(8 ; 6 ; 3)$}
{$\overrightarrow{u}_2=(8 ; 6 ;-3)$}
{$\overrightarrow{u}_3=(-8 ; 6 ;-3)$}
{$\overrightarrow{u}_4=(5 ; 9 ; 12)$}
\loigiai{
Vectơ chỉ phương của đường thẳng $\Delta\colon \dfrac{x-5}{8}=\dfrac{y-9}{6}=\dfrac{z-12}{3}$ là $\overrightarrow{u}_1=(8 ; 6 ; 3)$.
}
\end{ex}

\begin{ex}%[2H5N3-2]
Trong không gian $Oxyz$, mặt cầu $(S) \colon x^2+y^2+z^2+4x-2y+8z-1=0$ có tọa độ tâm là
\choice
{$(4;-2;8)$}
{ $(2;-1;4)$}
{\True $(-2;1;-4)$}
{$(2;-1;-4)$}
\loigiai{

}
\end{ex}

\begin{ex}%[2D6N1-1]
Cho $P(A)$, $P(B)>0$. Chọn khẳng định đúng trong các khẳng định sau.
\choice
{$\mathrm{P}(A\mid B)=\dfrac{\mathrm{P}(A\cup B)}{\mathrm{P}(A)}$}
{$\mathrm{P}(A\mid B)=\dfrac{\mathrm{P}(A\cup B)}{\mathrm{P}(B)}$}
{\True $\mathrm{P}(A\mid B)=\dfrac{\mathrm{P}(A\cap B)}{\mathrm{P}(B)}$}
{$\mathrm{P}(A\mid B)=\dfrac{\mathrm{P}(A\cap B)}{\mathrm{P}(A)}$}
\loigiai{
Nếu $\mathrm{P}(B)>0$ thì $\mathrm{P}(A\mid B)=\dfrac{\mathrm{P}(A\cap B)}{\mathrm{P}(B)}$.
}
\end{ex}

\begin{ex}%[2D6N2-1]
Cho hai biến cố ngẫu nhiên $A$, $B$ thoả mãn $\mathrm{P}(A)>0$ và $0<\mathrm{P}(B)<1$. Chọn khẳng định đúng trong các khẳng định sau.
\choice
{$\mathrm{P}(B\mid A)=\dfrac{\mathrm{P}(A)\cdot \mathrm{P}(A\mid \overline{B})}{\mathrm{P}(B)\cdot \mathrm{P}(A\mid B)+\mathrm{P}(B)\cdot \mathrm{P}(A\mid B)}$}
{$\mathrm{P}(B\mid A)=\dfrac{\mathrm{P}(\overline{B})\cdot \mathrm{P}(A\mid \overline{B})}{\mathrm{P}(B)\cdot \mathrm{P}(A\mid B)+\mathrm{P}(B)\cdot \mathrm{P}(A\mid B)}$}
{$\mathrm{P}(B\mid A)=\dfrac{\mathrm{P}(A)\cdot \mathrm{P}(A\mid B)}{\mathrm{P}(B)\cdot \mathrm{P}(A\mid B)+\mathrm{P}(\overline{B})\cdot \mathrm{P}(A\mid \overline{B})}$}
{\True $\mathrm{P}(B\mid A)=\dfrac{\mathrm{P}(B)\cdot \mathrm{P}(A\mid B)}{\mathrm{P}(B)\cdot \mathrm{P}(A\mid B)+\mathrm{P}(\overline{B})\cdot \mathrm{P}(A\mid \overline{B})}$}
\loigiai{
$\mathrm{P}(B\mid A)=\dfrac{\mathrm{P}(B)\cdot \mathrm{P}(A\mid B)}{\mathrm{P}(B)\cdot \mathrm{P}(A\mid B)+\mathrm{P}(\overline{B})\cdot \mathrm{P}(A\mid \overline{B})}$.
}
\end{ex}

\begin{ex}%[2D6N2-3]
Cho hai biến cố $A$, $B$ thoả mãn $\mathrm{P}(A)=0{,}4$; $\mathrm{P}(B)=0{,}3$; $\mathrm{P}(A | B)=0{,}25$. Khi đó, $\mathrm{P}(B | A)$ bằng:
\choice
{\True $0{,}1875$}
{$0{,}48$}
{$0{,}333$}
{$0{,}95$}
\loigiai{
Theo công thức Bayes, ta có $\mathrm{P}(B|A)=\dfrac{\mathrm{P}(B) \cdot \mathrm{P}(A|B)}{\mathrm{P}(A)}=\dfrac{0{,}3 \cdot 0{,}25}{0{,}4}=0{,}1875$.
}
\end{ex}

\begin{ex}%[2H5N3-3]
Trong không gian $Oxyz$, mặt cầu có tâm $I(2; 1; -3)$ và bán kính $9$ và có phương trình là
\choice
{\True $(x-2)^2+(y-1)^2+(z+3)^2=81$}
{$(x+2)^2+(y+1)^2+(z-3)^2=81$}
{$(x-2)^2+(y-1)^2+(z+3)^2=9$}
{$(x+2)^2+(y+1)^2+(z-3)^2=9$}
\loigiai{
Trong không gian $Oxyz$, mặt cầu có tâm $I(2; 1; -3)$ và bán kính $9$ và có phương trình là $(x-2)^2+(y-1)^2+(z+3)^2=81$.
}
\end{ex}
\Closesolutionfile{ans}

\TNTF
\Opensolutionfile{ans}[ans/ansDe1-TN2]
\begin{ex}
Cho hàm số $f(x)=\dfrac{2x+1}{x+2}$.
\choiceTF
{\True $\displaystyle\int f'(x)\mathrm{\,d}x=\dfrac{2x+1}{x+2}+C$}
{$\displaystyle\int f(x)\mathrm{\,d}x=2\ln \left| x+2 \right|+C$}
{\True $\displaystyle\int_{-1}^2 [f'(x)-2] \mathrm{\,d}x >-3$}
{\True Diện tích hình phẳng giới hạn với đường cong $y=f(x)$, trục hoành và các đường thẳng $x=-1$ và $x=2$ bằng $4+6\ln a$ và $0<a<1$}
\loigiai{
    \begin{itemchoice}
    \itemch $\displaystyle\int f'(x)\mathrm{\,d}x= f(x)+C = \dfrac{2x+1}{x+2}+C$.
    \itemch $\displaystyle\int f(x)\mathrm{\,d}x=\displaystyle\int \left(2-\dfrac{3}{x+2}\right)\mathrm{\,d}x=2x-3\ln \left| x+2 \right|+C$.
    \itemch $\displaystyle\int_{-1}^2 [f'(x)-2] \mathrm{\,d}x=f(2)-f(-1)-2x\vert_{-1}^2=\dfrac54+\dfrac13-2\cdots 3=0{,}5>-3$.
    \itemch Diện tích hình phẳng giới hạn với đường cong $y=f(x)$, trục hoành và các đường thẳng $x=-1$ và $x=2$ là 
    $$S=\displaystyle\int_{-1}^2 |f(x)| \mathrm{\,d}x = -\int_{-1}^{-\frac12} f(x) \mathrm{\,d}x + \int_{-\frac12}^{2} f(x) \mathrm{\,d}x = - \left(2x-3\ln \left| x+2 \right| \right)\vert_{-1}^{-\frac12} + \left(2x-3\ln \left| x+2 \right| \right)\vert_{-\frac12}^{2} = 4+3\ln \dfrac{9}{16} = 4+6\ln \dfrac{3}{4}$$
    Vậy $0<a=\dfrac{3}{4}<1$.
    \end{itemchoice}
}


\end{ex}

\begin{ex}%[Mức độ 2]%[BG-12-New-4in1, Hiệp Hà]%[2H5H1-2]
Cho mặt phẳng $(\alpha)$ đi qua $A(-1;1;2)$ có cặp vectơ chỉ phương là $\vec{a} = (1; -2; 3)$ và $\vec{b}= (2; 1; -1)$.
\choiceTF
{$\vec{a}$ có giá song song với mặt phẳng $(\alpha)$}
{\True Mặt phẳng $(\alpha)$ có phương trình là $x-7y-5z+18=0$}
{Đường thẳng $d \colon \dfrac{x+1}{2}=\dfrac{y-1}{1}=\dfrac{z-2}{-1}$ song song với mặt phẳng $(\alpha)$}
{Mặt phẳng $(\alpha)$ cắt mặt cầu $(S)$ có tâm $I(2; 1; -3)$ và bán kính $1$ theo giao tuyến là một đường tròn}
\loigiai{
\begin{itemchoice}
\itemch $\vec{a}$ là một vectơ trong cặp vectơ chỉ phương nên $\vec{a}$ có giá song song hoặc nằm trong mặt phẳng $(\alpha)$.
\itemch Ta có
\[
\begin{aligned}
\vec{n}=[\vec{a}, \vec{b}] & =\left(\left|\begin{array}{cc}
-2 & 3 \\ 1 & -1
\end{array}\right| ;\left|\begin{array}{cc}
3 & 1 \\ -1 & 2
\end{array}\right| ;\left|\begin{array}{cc}
1 & -2 \\ 2 & 1
\end{array}\right|\right) \\
& =(-1; 7; 5) .
\end{aligned}
\]
Do đó $\vec{n_1}=(1;-7;-5) =-\vec{n}$ là một vectơ pháp tuyến của mặt phẳng $(\alpha)$ nên mặt phẳng $(\alpha)$ đi qua $A(-1;1;2)$ có phương trình là
$$1(x+1)-7(y-1)-5(z-2)=0 \Leftrightarrow x-7y-5z+18=0.$$
\itemch Đường thẳng $d \colon \dfrac{x+1}{2}=\dfrac{y-1}{1}=\dfrac{z-2}{-1}$ có vectơ chỉ phương là $\vec{b}=(2;1;-1)$ và đi qua $A(-1;1;2)$ nên $d$ nằm trên $(\alpha)$.
\itemch Vì $d(I,(\alpha))=\dfrac{\left| 1\cdot 2-7\cdot 1-5\cdot (-3) \right|}{\sqrt{1^2+7^2+5^2}} = \dfrac{2\sqrt{3}}{3}>1$ nên mặt phẳng $(\alpha)$ không cắt mặt cầu $(S)$.
\end{itemchoice}
}
\end{ex}
\Closesolutionfile{ans}

\TNSA
\Opensolutionfile{ans}[ans/ansDe1-TN3]
\begin{ex}%[2D4H1-4]
Cho hàm số $f(x)=2x+\mathrm{e}^x$. Một nguyên hàm $F(x)$ của hàm số $f(x)$ thỏa mãn $F(0)=2024$. Biết $F(x)=ax^2+b\mathrm{e}^x+c$, giá trị của $a+b+c$ là
\shortans{$2025$}
\loigiai{
Ta có $\displaystyle\int f(x)\mathrm{\,d}x=\displaystyle\int (2x+\mathrm{e}^x)\mathrm{\,d}x=x^2+\mathrm{e}^x+C$.\\
Có $F(x)$ là một nguyên hàm của $f(x)$ và $F(0)=2024$.\\
Tìm được $\heva{&F(x)=x^2+\mathrm{e}^x+C\\ &F(0)=2024} \Rightarrow 1+C=2024 \Leftrightarrow C=2023$.\\
Suy ra $F(x)=x^2+\mathrm{e}^x+2023$.\\
Vậy $a+b+c=2025$.
}
\end{ex}

\begin{ex}%[2D4V2-6]
Gọi $h(t)$ cm là mức nước trong bồn chứa sau khi bơm được $t$ giây. Biết rằng tốc độ tăng giảm của mực nước là $h^{\prime}(t)=\dfrac{1}{5} \sqrt[3]{t+8}$ (cm/s) và lúc đầu bồn không có nước. Tìm mức nước ở bồn (đơn vị: cm) sau khi bơm nước được $6$ giây (làm tròn đến chữ số hàng phần trăm).
\shortans{$2{,}66$}
\loigiai{Hàm $h(t)=\displaystyle\int\limits \dfrac{1}{5} \sqrt[3]{t+8} \mathrm{\,d} t=\dfrac{3}{20}(t+8) \sqrt[3]{t+8}+C$.\\
Lúc $t=0$, bồn không chứa nước. Suy ra $h(0)=0 \Rightarrow \dfrac{12}{5}+C=0 \Leftrightarrow C=-\dfrac{12}{5}$.\\
Vậy, hàm $h(t)=\dfrac{3}{20}(t+8) \sqrt[3]{t+8}-\dfrac{12}{5}$.\\
Mức nước trong bồn sau $6$ giây là $h(6) \simeq 2{,}66$ cm.
}
\end{ex}

\begin{ex}%[2H5V2-8]
    Hình vẽ dưới đây là hình ảnh Cầu Cổng Vàng (The Golden Gate Bridge) ở Mỹ. Xét hệ trục toạ độ $O x y z$ với $O$ là bệ của chân cột trụ tại mặt nước, trục $O z$ trùng với cột trụ, mặt phẳng $O x y$ là mặt nước và xem như trục $O y$ cùng phương với cầu như hình vẽ. Dây cáp $A D$ (xem như là một đoạn thẳng) đi qua đỉnh $D$ thuộc trục $O z$ và điểm $A$ thuộc mặt phẳng $O y z$, trong đó điểm $D$ là đỉnh cột trụ cách mặt nước $227$ m, điểm $A$ cách mặt nước $75$ m và cách trục $O z$ khoảng $343$ m. \begin{flushright}
    \textit{(Nguồn: https://www.goldengate.org/assets/1/6/ggb-exhibit-chapter-statistics.pdf)}
    \end{flushright}
    \begin{center}
    \includegraphics[scale=.5]{images/Cau-cong-vang}
    \end{center}
    Giả sử ta dùng một đoạn dây nối điểm $N$ trên dây cáp $A D$ và điểm $M$ trên thành cầu, biết $M$ cách mặt nước $75$ m và $M N$ song song với cột trụ. Tính độ dài $M N$ (đơn vị mét) biết điểm $M$ cách trục $O z$ một khoảng bằng $230$ m (kết quả làm tròn đến hàng phần mười).
    \shortans{$50{,}1$}
    \loigiai{
    Chọn một đơn vị trên các trục bằng $1$ m.\\
    Ta có $D(0 ; 0 ; 227),$ $ A(0 ;-343 ; 75),$ $ M(0 ;-230 ; 75)$, $\overrightarrow{A D}=(0 ; 343 ; 152)$.\\ Phương trình đường thẳng $A D\colon \heva{&x=0 \\& y=343 t \\& z=227+152 t} \Rightarrow N(0 ; 343 t ; 227+152 t)$.\\
    Ta có $\overrightarrow{M N}=(0 ; 343 t+230 ; 152+152 t)$, $M N$ song song với trục $O z$, suy ra \[ 343 t+230=0 \Rightarrow t=-\dfrac{230}{343} \Rightarrow M N=152+152 \cdot \left(-\dfrac{230}{343}\right) \approx 50,1(m).\]
    }
    \end{ex}

\begin{ex}%[2D6V1-3]
Ông An hằng ngày đi làm bằng xe máy hoặc xe buýt. Nếu hôm nay ông đi làm bằng xe buýt thì xác suất để hôm sau ông đi làm bằng xe máy là $0,4$ . Nếu hôm nay ông đi làm bằng xe máy thì xác suất để hôm sau ông đi làm bằng xe buýt là $0,7$ . Xét một tuần mà thứ Hai ông An đi làm bằng xe buýt. Tính xác suất để thứ Tư trong tuần ông An đi làm bằng xe máy.

\shortans{0,36}
\loigiai
{
Gọi $A$ là biến cố \textquotedblleft Thứ Ba, ông An đi làm bằng xe máy\textquotedblright.\\
$B$ là biến cố \textquotedblleft Thứ Tư, ông An đi làm bằng xe máy\textquotedblright.\\
Khi đó $\heva{&P(A) = 0{,}4&\Rightarrow& P(\overline{A}) = 1-0{,}4 = 0{,}6\\ &P(\overline{B}|A) = 0{,}7 &\Rightarrow& P(B|A) = 1-0{,}7 = 0{,}3\\ &P(B|\overline{A}) = 0{,}4 &\Rightarrow& P(\overline{B}|\overline{A}) = 1-0{,}4=0{,}6.}$
\begin{center}
\begin{tikzpicture}[>=stealth]
%Khung 1
\draw (1,3.0) node{\textbf{Thứ Hai}};
\draw (-0,-1) rectangle (2.2,0);
\draw (1.1,-0.5) node{Buýt};
%Mui ten 1,2
\draw [->] (2.2,-0.5)--(3.8,1.6) node[pos=0.5,sloped,above]{$0{,}4$};
\draw [->] (2.2,-0.5)--(3.8,-2.6) node[pos=0.5,sloped,below]{\color{red}$0{,}6$};
%Khung 2.1
\draw (4.5,3.0) node{\textbf{Thứ Ba}};
\draw (3.8,1.1) rectangle (5.1,2.1);
\draw (8.9/2,1.6) node{$A$} ;
%Khung 2.2
\draw (3.8,-2.1) rectangle (5.1,-3.1);
\draw (8.9/2,-2.6) node{$\overline{A}$};
%Mui ten 3,4
\draw [->] (5.1,1.6)--(6.5,2.6) node[pos=0.5,sloped,above]{\color{red}$0{,}3$};
\draw [->] (5.1,1.6)--(6.5,0.6) node[pos=0.5,sloped,below]{$0{,}7$};
%Mui ten 5,6
\draw [->] (5.1,-2.6)--(6.5,-1.6) node[pos=0.5,sloped,above]{$0{,}4$};
\draw [->] (5.1,-2.6)--(6.5,-3.6) node[pos=0.5,sloped,below]{\color{red}$0{,}6$};
%Khung 3.1
\draw (6.5,2.2) rectangle (7.7,3.2);
\draw (7.1,5.4/2) node{$B$} ;
%Khung 3.2
\draw (7.0,3.7) node{\textbf{Thứ Tư}};
\draw (6.5,1.2) rectangle (7.7,0.2);
\draw (7.1,1.4/2) node{$\overline{B}$} ;
%Khung 3.3
\draw (6.5,-1.1) rectangle (7.7,-2.1);
\draw (7.1,-3.2/2) node{$B$} ;
%Khung 3.3
\draw (6.5,-2.9) rectangle (7.7,-3.9);
\draw (7.1,-3.4) node{$\overline{B}$} ;
%Kết quả
\draw (9.5,3.7) node{\textbf{Kết quả}};
\draw (9.5,2.7) node{$AB$};
\draw (9.5,0.7) node{$A \overline{B}$};
\draw (9.5,-1.6) node{$\overline{A}B$};
\draw (9.5,-3.4) node{$\overline{A}~\overline{B}$};
%Xác suất
\draw (12.5,3.7) node{\textbf{Xác suất}};
\draw (12.5,2.7) node{$0{,}12$};
\draw (12.5,0.7) node{$0{,}28$};
\draw (12.5,-1.6) node{$0{,}24$};
\draw (12.5,-3.4) node{$0{,}36$};
\end{tikzpicture}
\end{center}
Áp dụng công thức xác suất toàn phần để tính xác suất thứ Tư ông An đi làm bằng xe máy là
\[P(B) = P(A) \cdot P(B|A) + P(\overline{A}) \cdot P(B|\overline{A}) = 0{,}4 \cdot 0{,}3 + 0{,}6 \cdot 0{,}4 = 0{,}36.\]
}
\end{ex}
\TL
\begin{ex}%[2H5H2-3]%[Dự án 2025 - Đề cấu trúc mới của Bộ theo [Thành Đức Trung]
Trong không gian $Oxyz$, cho tam giác $ABC$ có $A(0;0;1)$, $B(-3;2;0)$, $C(2;-2;3)$. Viết phương trình đường cao kẻ từ $B$ của tam giác $ABC$.% đi qua điểm $K(a;b;-2)$. Tính $ab$.
% \shortans{$-2$}
\loigiai
{
Gọi $\Delta$ là đường cao kẻ từ $B$ của tam giác $ABC$.\\
Ta có $\heva{& \overrightarrow{AB}=(-3;2;-1) \\ & \overrightarrow{AC}=(2;-2;2)} \Rightarrow \left[\overrightarrow{AB},\overrightarrow{AC}\right]=(2;4;2)$. \\
Suy ra một véc-tơ pháp tuyến của mặt phẳng $(ABC)$ là $\overrightarrow{n}=(1;2;1)$.\\
Ta có $\heva{ & \Delta \subset(ABC) \\ & \Delta \perp AC}$, suy ra đường thẳng $\Delta$ nhận $\left[\overrightarrow{n},\overrightarrow{AC}\right]$ làm một véc-tơ chỉ phương.\\
Có $\left[\overrightarrow{n},\overrightarrow{AC}\right]=(6;0;-6)=6\overrightarrow{u}$ với $\overrightarrow{u}=(1;0;-1)$. \\
Suy ra đường thẳng $\Delta$ nhận $\overrightarrow{u}=(1;0;-1)$ làm véc-tơ chỉ phương.\\
Do đó phương trình đường thẳng $\Delta$ là $\Delta \colon \heva{ & x=-3+t \\ & y=2 \\ & z=-t} $%\Rightarrow K(-1;2;-2)$.\\
% Vậy $a=-1$; $b=2$ và $ab=-2$.
}
\end{ex}

\begin{ex}%[BAI-GIANG-12-4IN1, Võ Thị Thùy Trang]%[Cánh diều]%[2D6C2-4]
Giả sử có một loại bệnh mà tỉ lệ người mắc bệnh là $0{,}1\%$. Giả sử có một loại xét nghiệm, mà ai mắc bệnh khi xét nghiệm cũng có phản ứng dương tính, nhưng tỉ lệ phản ứng dương tính giả là $5\%$ (tức là trong số những người không bị bệnh có $5\%$ số người xét nghiệm lại có phản ứng dương tính). Khi một người xét nghiệm có phản ứng dương tính thì khả năng mắc bệnh của người đó là bao nhiêu phần trăm (làm tròn kết quả đến hàng phần trăm)?
% \shortans{$1{,}96$}
\loigiai{
\begin{itemize}
\item Xét hai biến cố\\
$K$ \lq\lq Người được chọn ra không mắc bệnh\rq\rq;\\
$D$ \lq\lq Người được chọn ra có phản ứng dương tính\rq\rq.\\
Do tỉ lệ người mắc bệnh là $0{,}1 \%=0{,}001$ nên $\mathrm{P}(K)=1-0{,}001=0{,}999$.\\
Trong số những người không mắc bệnh có $5 \%$ số người có phản ứng dương tính nên \break$\mathrm{P}(D \mid K)=5 \%=0{,}05$. Vì ai mắc bệnh khi xét nghiệm cũng có phản ứng dương tính nên $\mathrm{P}(D \mid \overline{K})=1$.\\
Sơ đồ hình cây ở bên dưới biểu thị tình huống đã cho.
\begin{center}
\begin{tikzpicture}[teal,grow=right, edge from parent/.style={draw,-latex}, label distance = 0.2cm,
level 1/.style = {level distance=3.5cm, sibling distance=28mm},
level 2/.style = {level distance=4.5cm, sibling distance=22mm},
level 3/.style = {level distance=4.5cm, sibling distance=22mm},
]

\node {}
child {node {$\overline{K}$
}
child {node {$\overline {D}$}
}
child {node[opacity=.75] {$D$}
edge from parent
node[above,sloped] {$\mathrm{P}(D\mid \overline {K})=1$}
}
edge from parent
node[below,sloped] {$\mathrm{P}(\overline {K})=0{,}001$}
}
child {node[] {$K$}
child {node[opacity=.75] {$\overline {D}$}
}
child {node[] {$D$}
edge from parent
node[above,sloped] {$\mathrm{P}(D\mid K)=0{,}05$}
}
edge from parent
node[above,sloped] {$\mathrm{P}(K)=0{,}009$}
};
\end{tikzpicture}
\end{center}
\item Ta thấy, khả năng mắc bệnh của một người xét nghiệm có phản ứng dương tính chính là $\mathrm{P}(\overline{K} \mid D)$. Áp dụng công thức Bayes, ta có
\[
\mathrm{P}(\overline{K}|D)=\dfrac{\mathrm{P}(\overline{K}) \cdot \mathrm{P}(D| \overline{K})}{\mathrm{P}(\overline{K}) \cdot \mathrm{P}(D| \overline{K})+\mathrm{P}(K) \cdot \mathrm{P}(D| K)}=\dfrac{0{,}001}{0{,}001+0{,}999 \cdot 0{,}05} \approx 1{,}96 \% .
\]
Vậy xác suất mắc bệnh của một người xét nghiệm có phản ứng dương tính là $1{,}96\%$.
\end{itemize}
}
\end{ex}

\begin{ex}%[Mức độ 3]giảng 12 New - 4in1, Đoàn Hùng]%[2H5C1-7]
Trong không gian với hệ tọa độ $Oxyz$, cho ba điểm $A(1;4;5)$, $B(3;4;0)$, $C(2;-1;0)$ và mặt phẳng $(P)\colon 3x-3y-2z-12=0$. Gọi $M(a;b;c)$ thuộc $(P)$ sao cho $MA^2+MB^2+3MC^2$ đạt giá trị nhỏ nhất. Tính tổng $a+b+c$.
% \shortans{$3$}
\loigiai{
Gọi $I(x;y;z)$ là điểm thỏa mãn $\overrightarrow{IA}+\overrightarrow{IB}+3\overrightarrow{IC}=\overrightarrow{0}$.\\
Ta có $\overrightarrow{IA}=(1-x;4-y;5-z)$, $\overrightarrow{IB}=(3-x;4-y;-z)$ và $3\overrightarrow{IC}=(6-3x;-3-3y;-3z)$.\\
Từ ta có hệ phương trình: $\heva{&1-x+3-x+6-3x=0\\&4-y+4-y-3-3y=0\\&5-z-z-3z=0} \Leftrightarrow \heva{&x=2\\&y=1\\&z=1} $$\Rightarrow I(2;1;1)$.\\
Ta có
\begin{itemize}
\item $MA^2=\overrightarrow{MA}^2=(\overrightarrow{MI}+\overrightarrow{IA})^2=MI^2+2\overrightarrow{MI}\cdot \overrightarrow{IA}+IA^2$.
\item $MB^2=\overrightarrow{MB}^2=(\overrightarrow{MI}+\overrightarrow{IB})^2=MI^2+2\overrightarrow{MI}\cdot \overrightarrow{IB}+IB^2$.
\item $3MC^2=3\overrightarrow{MC}^2=3(\overrightarrow{MI}+\overrightarrow{IC})^2=3(MI^2+2\overrightarrow{MI}\cdot \overrightarrow{IC}+IC^2)$.
\end{itemize}
Do đó $S=MA^2+MB^2+3MC^2=5MI^2+IA^2+IB^2+3IC^2$.\\
Do $IA^2+IB^2+3IC^2$ không đổi nên $S$ đạt giá trị nhỏ nhất khi và chỉ khi $MI$ đạt giá trị nhỏ nhất. Tức là $M$ là hình chiếu của $I$ lên mặt phẳng $(P)\colon 3x-3y-2z-12=0$ Suy ra $\overrightarrow{IM}$ cùng phương với véc-tơ chỉ phương pháp tuyến $\overrightarrow{n}=(3;-3;-2)$ của $(P)$.\\
Suy ra $M(2+3t;1-3t;1-2t)$.\\
Vì $M\in (P)$ nên $3(2+3t)-3(1-3t)-2(1-2t)-12=0\Leftrightarrow 22t-11=0\Leftrightarrow t=\dfrac{1}{2}$.\\
Suy ra $M\left(\dfrac{7}{2};-\dfrac{1}{2};0\right)$.\\
Vậy $a+b+c=\dfrac{7}{2}-\dfrac{1}{2}=3$.
}
\end{ex}
\Closesolutionfile{ans}


\Closesolutionfile{ansbook}

% \begin{name}
	{\tenchude}
	{TOÁN 12}
	{LỚP TOÁN THẦY PHÁT}
	{Thời gian: 90 phút - Không kể thời gian phát đề}
\end{name}
\Opensolutionfile{ansbook}[ans/ansbookDe2]
\TN
\Opensolutionfile{ans}[ans/ansDe2-TN1]
\begin{ex}%[Dự án 2025 - đề cấu trúc mới, Nguyễn Kiều Nhã Tú]%[2D4N1-1]
Hàm số $F(x)$ là một nguyên hàm của hàm số $f(x)$ trên khoảng $K$ nếu
\choice
{$F'(x)=-f(x)$, $\forall x\in K$}
{$f'(x)=F(x)$, $\forall x\in K$}
{\True $F'(x)=f(x)$, $\forall x \in K$}
{$f'(x)=-F(x)$, $\forall x\in K$}
\loigiai{
Theo tính chất của nguyên hàm có $F'(x)=f(x)$, $\forall x\in K$.
}
\end{ex}

\begin{ex}%[2D4N1-2]
Tìm nguyên hàm của hàm số $f(x)=\sqrt {2x-1}$.
\choice
{ $\displaystyle\int{f(x)\mathrm{d}x=\dfrac{2}{3}( 2x-1 )\sqrt {2x-1}+C}$}
{\True $\displaystyle\int{f(x)\mathrm{d}x=\dfrac{1}{3}( 2x-1 )\sqrt {2x-1}+C}$}
{ $\displaystyle\int{f(x)\mathrm{d}x=-\dfrac{1}{3}\sqrt {2x-1}+C}$}
{$\displaystyle\int{f(x)\mathrm{d}x=\dfrac{1}{2}\sqrt {2x-1}+C}$}
\loigiai{
$\displaystyle\int{f\left( x \right)\mathrm{d}x=\displaystyle\int{\sqrt {2x-1}\mathrm{d}x=\dfrac{1}{2}\displaystyle\int{{\left( 2x-1 \right)}^{\frac{1}{2}}\mathrm{d}( 2x-1 )}}}=\dfrac{1}{3}( 2x-1 )\sqrt {2x-1}+C$.}
\end{ex}

\begin{ex}%[2D4N2-1]
Nếu $\displaystyle\int_0^2 f(x) d x=3$ thì $\displaystyle\int_0^2\left[2f(x)-1\right]\mathrm{\,d}x$ bằng
\choice
{$6$}
{\True $4$}
{$8$}
{$5$}
\loigiai{
Ta có $\displaystyle\int_0^2\left[2f(x)-1\right]\mathrm{\,d}x=2\displaystyle\int_0^2f(x)\mathrm{\,d}x-\displaystyle\int_0^2\mathrm{\,d}x=2\cdot 3-2=4$.}
\end{ex}

\begin{ex}%[2H5N1-1]
Trong không gian với hệ toạ độ $Oxyz$, phương trình nào dưới đây là phương trình của mặt phẳng $(Oyz)$?
\choice
{$y=0$}
{\True $x=0$}
{$y-z=0$}
{$z=0$}
\loigiai{
Mặt phẳng $(Oyz)$ đi qua điểm $O(0 ; 0 ; 0)$ và có véc-tơ  pháp tuyến là $\vec{i}=(1 ; 0 ; 0)$ nên ta có phương trình mặt phẳng $(O y z)$ là  $1(x-0)+0(y-0)+0(z-0)=0 \Leftrightarrow x=0$.
}
\end{ex}

\begin{ex}%[Mức độ 1]%[BG-12-New-4in1, Hiệp Hà]%[2H5N1-2]
Vectơ nào dưới đây là một vectơ pháp tuyến của $(Oxy)$?
\choice
{$\vec{n_1}=(2;0;0)$}
{$\vec{n_2}=(1;1;0)$}
{$\vec{n_3}=(0;3;0)$}
{\True $\vec{n_4}=(0;0;-1)$}
\loigiai{
Ta có $Oz \perp (Oxy)$ nên $\vec{k}=(0;0;1)$ là một vectơ pháp tuyến của $(\alpha)$.\\
Khi đó, $\vec{n_4}=-\vec{k}$ là một vectơ pháp tuyến của $(\alpha)$.
}
\end{ex}

\begin{ex}%[12-MH-2-MH2025]%[MH-2025,Chu Hà]%[2H5N2-1]
Trong không gian tọa độ $Oxyz$, phương trình nào sau đây là phương trình tham số của đường thẳng?
\def\dotEX{}
\choice
{$\heva{&2x+3y-z=0\\&x+y+z=8.}$}
{$\heva{&-3x+z=0\\&x+2y+z-7=0.}$}
{$\heva{&x=2+t\\&y=3-t\\&z =t^2.}$}
{\True  $\heva{&x=-4-3t\\&y=2-5t\\&z=-1+6t.}$}
\loigiai{
Phương trình đường thẳng có dạng $\heva{&x=-4-3t\\&y=2-5t\\&z=-1+6t.}$ với $t$ là tham số.
}
\end{ex}

\begin{ex}%[Ex-Ôn tập 2025, Nguyễn Văn Nay]%[2H5N2-2]
Trong không gian với hệ tọa độ $Oxyz$, vectơ nào sau đây là vectơ chỉ phương của đường thẳng $\Delta\colon\heva{&x=-4+2t \\ &y=7-3t \\&z=8-9t}$?
\choice
{$\vec{u}_1=(4;7;8)$}
{$\vec{u}_2=(-4;7;8)$}
{$\vec{u}_3=(2;3;9)$}
{\True $\vec{u}_4=(2;-3;-9)$}
\loigiai{Một vectơ chỉ phương của đường thẳng là $\vec{u}_4=(2;-3;-9)$.}
\end{ex}

\begin{ex}%[NB]giảng 12 New - 4in1, Nguyễn Vân Trường]%[2H5N3-2]
Trong hệ trục toạ độ $Oxyz$, tìm $m$ để phương trình $(S_m) \colon x^2+y^2+z^2+2x-2my+2mz-5m^2 = 0$ xác định một mặt cầu có bán kính nhỏ nhất.
\choice
{$m=1$}
{$m=7$}
{\True $m=0$}
{$m=\dfrac{1}{7}$}
\loigiai{
Để phương trình $(S_m)$ xác định một mặt cầu thì $1^2+m^2+m^2+5m^2 =1+7m^2>0$ với mọi $m \in \mathbb{R}$.
Khi đó bán kính của mặt cầu $(S_m)$ là
$R=\sqrt{1+7m^2} \ge 1$. \\
Dấu đẳng thức xảy ra khi $m = 0$. Vậy $m=0$.
}
\end{ex}

\begin{ex}%[2D6N1-1]
Cho hai biến cố xung khắc $A$, $B$ thoả mãn $\mathrm{P}(A)=0{,}35$ và $\mathrm{P}(B)=0{,}55$. Khi đó $\mathrm{P}(A\mid B)$ bằng
\choice
{\True $0$}
{$0{,}9$}
{$0{,}1$}
{$0{,}4$}
\loigiai{
Do $A$, $B$ xung khắc nên $\mathrm{P}(A\mid B)=0$.
}
\end{ex}

\begin{ex}%[2D6N2-1]%[Lê Công Trường]
Cho hai biến cố $C$ và $D$ với $0\leq \mathrm{P}(D)\leq 1$. Công thức xác suất toàn phần là
\choice
{\True $\mathrm{P}(C)=\mathrm{P}(D)\cdot\mathrm{P}(C\mid D)+\mathrm{P}(\overline{D})\cdot\mathrm{P}(C\mid \overline{D})$}
{$\mathrm{P}(C)=\mathrm{P}(\overline{D})\cdot\mathrm{P}(C\mid D)+\mathrm{P}(D)\cdot\mathrm{P}(C\mid \overline{D})$}
{$\mathrm{P}(D)=\mathrm{P}(\overline{D})\cdot\mathrm{P}(C\mid D)+\mathrm{P}(D)\cdot\mathrm{P}(C\mid \overline{D})$}
{$\mathrm{P}(D)=\mathrm{P}(\overline{D})\cdot\mathrm{P}(C\mid D)+\mathrm{P}(C)\cdot\mathrm{P}(C\mid \overline{D})$}
\loigiai{Cho hai biến cố $C$ và $D$ với $0\leq \mathrm{P}(D)\leq 1$. Công thức xác suất toàn phần là \[\mathrm{P}(C)=\mathrm{P}(D)\cdot\mathrm{P}(C\mid D)+\mathrm{P}(\overline{D})\cdot\mathrm{P}(C\mid \overline{D})\] }
\end{ex}

\begin{ex}%[2D6N2-3]%[Dự án EX-TF-TLN lần 4 - Quan Ón]
Cho các biến cố $A$ và $B$ thỏa mãn $\mathrm{P}(A) > 0$, $\mathrm{P}(B) > 0$. Khi đó $\mathrm{P}(A\mid B)$ bằng biểu thức nào dưới đây?
\choice
{\True $\dfrac{\mathrm{P}(A)\cdot \mathrm{P}(B\mid A)}{\mathrm{P}(B)}$}
{$\dfrac{\mathrm{P}(B)\cdot \mathrm{P}(B\mid A)}{\mathrm{P}(A)}$}
{$\dfrac{\mathrm{P}(B)}{\mathrm{P}(A)\cdot \mathrm{P}(B\mid A)}$}
{$\dfrac{\mathrm{P}(A)}{\mathrm{P}(B)\cdot \mathrm{P}(B\mid A)}$}
\loigiai{
Theo công thức xác suất có điều kiện, ta có $\mathrm{P}(A\mid B) = \dfrac{\mathrm{P}(AB)}{\mathrm{P}(B)}$.\\
Theo công thức nhân, ta có $\mathrm{P}(AB) = \mathrm{P}(A)\cdot \mathrm{P}(B\mid A)$.\\
Do đó $\mathrm{P}(A\mid B) = \dfrac{\mathrm{P}(AB)}{\mathrm{P}(B)} = \dfrac{\mathrm{P}(A)\cdot \mathrm{P}(B\mid A)}{\mathrm{P}(B)}$.
}
\end{ex}

\begin{ex}%[2H5N3-3]
Trong không gian $Oxyz$, mặt cầu có tâm $I(1;-1; 2)$ và bán kính $R=5$ có phương trình là
\choice
{\True $(x-1)^2+(y+1)^2+(z-2)^2=25$}
{$(x-1)^2+(y+1)^2+(z-2)^2=5$}
{$(x+1)^2+(y-1)^2+(z+2)^2=25$}
{$(x+1)^2+(y-1)^2+(z+2)^2=5$}
\loigiai{
Mặt cầu cần tìm có phương trình $(x-1)^2+(y+1)^2+(z-2)^2=25$.
}
\end{ex}
\Closesolutionfile{ans}

\TNTF
\Opensolutionfile{ans}[ans/ansDe2-TN2]
\begin{ex}%[2D4H2-2]
Cho hàm số $f(x) = \heva{
&2x^2 + 3 \text{ khi } x \geq 1 \\
&2 - x^3 \text{ khi } x < 1
}$.
\choiceTF
{\True Trên $[1;+\infty)$ hàm số $f(x)$ có nguyên hàm là $F_1(x) = \dfrac{2}{3}x^3 + 3x + C_1$}

{\True Trên $(-\infty;1)$ hàm số $f(x)$ có nguyên hàm là $F_2(x) = 2x - \dfrac{1}{4}x^4 + C_2$}
{$\displaystyle\int\limits_{-2025}^{2025} f(x) \mathrm{d}x = \displaystyle\int\limits_{1}^{2025} (2x^2 + 3) \mathrm{d}x + \displaystyle\int\limits_{-2025}^{1} (2 - x^3) \mathrm{d}x$}
{\True Diện tích hình phẳng tạo bởi đồ thị hàm số $f(x)$, trục hoành và hai đường thẳng $x=-1$ và $x=2$ là $S=\displaystyle\int\limits_{-1}^{1} (2 - x^3) \mathrm{d}x + \displaystyle\int\limits_{1}^{2} (2x^2 + 3) \mathrm{d}x$}

\loigiai{
Do $f(x) = \heva{
&2x^2 + 3 \text{ khi } x \geq 1 \\
&2 - x^3 \text{ khi } x < 1
}$ nên
\begin{itemchoice}
\itemch Trên $[1;+\infty)$ hàm số $f(x)$ có nguyên hàm là $\displaystyle\int (2x^2 + 3) \mathrm{d}x = \dfrac{2}{3}x^3 + 3x + C_1$.
\itemch Trên $(-\infty;1)$ hàm số $f(x)$ có nguyên hàm là $\displaystyle\int (2 - x^3) \mathrm{d}x = 2x - \dfrac{1}{4}x^4 + C_2$.
\itemch Ta có $\displaystyle\int\limits_{-2025}^{2025} f(x) \mathrm{d}x = \displaystyle\int\limits_{1}^{2025} (2x^2 + 3) \mathrm{d}x + \displaystyle\int\limits_{-2025}^{1} (2 - x^3) \mathrm{d}x$
\itemch Diện tích hình phẳng tạo bởi đồ thị hàm số $f(x)$, trục hoành và hai đường thẳng $x=-1$ và $x=2$ là $$S=\displaystyle\int\limits_{-1}^{1} |f(x)| \mathrm{d}x + \displaystyle\int\limits_{1}^{2} |f(x)| \mathrm{d}x = \displaystyle\int\limits_{-1}^{1} |2 - x^3| \mathrm{d}x + \displaystyle\int\limits_{1}^{2} |2x^2 + 3| \mathrm{d}x = \displaystyle\int\limits_{-1}^{1} (2 - x^3) \mathrm{d}x + \displaystyle\int\limits_{1}^{2} (2x^2 + 3) \mathrm{d}x$$
\end{itemchoice}
}

\end{ex}

\begin{ex}%[Mức độ 2]%[BG-12-New-4in1, Hiệp Hà]%[2H5H1-2]
Cho mặt phẳng $(\alpha)$ đi qua các điểm $M(1; -2; 3)$, $N(2; 1; -1)$, $P(0;-2;4)$. Mỗi khẳng định dưới đây đúng hay sai?
\choiceTF
{\True $\vec{MN}$, $\vec{PN}$ là cặp vectơ chỉ phương của mặt phẳng $(\alpha)$}
{$\vec{n_1}=(3;-1;2)$ là một vectơ pháp tuyến của mặt phẳng $(\alpha)$}
{\True Đường thẳng $d$ đi qua $M$, $N$ có phương trình $\heva{&x=1+t\\&y=-2+3t\\&z=3-4t}$}
{Mặt cầu $(S)$ có đường kính $NP$ có phương trình là $(x-1)^2+(y+2)^2+(z-4)^2=25$}
\loigiai{
\begin{itemchoice}
\itemch
$\vec{MN}=(1;3;-4)$, $\vec{PN}=(2;3;-5)$ là hai vectơ không cùng phương và có giá nằm trong $(\alpha)$.\\
Do đó, $\vec{MN}$, $\vec{PN}$ là cặp vectơ chỉ phương của mặt phẳng $(\alpha)$.
\itemch Ta có
\[
\begin{aligned}
\vec{n}=[\vec{MN}, \vec{PN}] & =\left(\left|\begin{array}{cc}
3 & -4 \\ 3 & -5
\end{array}\right| ;\left|\begin{array}{cc}
-4 & 1 \\ -5 & 2
\end{array}\right| ;\left|\begin{array}{cc}
1 & 3 \\ 2 & 3
\end{array}\right|\right) \\
& =(-3; -3; -3) .
\end{aligned}
\]
Do đó, $\vec{n_1}=(3;-1;2)$ không phải là vectơ pháp tuyến của mặt phẳng $(\alpha)$.
\itemch Đường thẳng $d$ đi qua $M$, $N$ nên có vectơ chỉ phương là $\vec{MN}=(1;3;-4)$ và có phương trình tham số là $\heva{&x=1+t\\&y=-2+3t\\&z=3-4t}$.
\itemch Mặt cầu $(S)$ có đường kính $NP$ có tâm là trung điểm của đoạn thẳng $NP$ là $I(1; -\dfrac{1}{2}; \dfrac{3}{2})$ và bán kính là $R=\dfrac{1}{2}NP=\dfrac{1}{2}\sqrt{38}$.\\
Vậy phương trình mặt cầu $(S)$ là $(x-1)^2+(y+\dfrac{1}{2})^2+(z-\dfrac{3}{2})^2=\dfrac{19}{2}$.
\end{itemchoice}
}
\end{ex}
\Closesolutionfile{ans}

\TNSA
\Opensolutionfile{ans}[ans/ansDe2-TN3]
\begin{ex}%[2D4H1-4]
Cho $F(x)$ là một nguyên hàm của hàm số $f(x)=x\sqrt{x}+\dfrac{1}{\sqrt{x}}$. Biết $F(1)=-2$. Tính $F(0)$.
\shortans{$-4{,}4$}
\loigiai{
Hàm số $f(x)=x\sqrt{x}+\dfrac{1}{\sqrt{x}}=x^{\tfrac{3}{2}}+x^{-\tfrac{1}{2}}$.\\
Có $F(x)=\displaystyle \int f(x)\mathrm{\,d}x=\displaystyle \int\left(x^{\tfrac{3}{2}}+x^{-\tfrac{1}{2}}\right)\mathrm{\,d}x=\dfrac{2}{5}x^{\tfrac{5}{2}}+2\sqrt{x}+C$.\\
Do $F(1)=-2\Rightarrow -2=\dfrac{2}{5}\cdot 1^{\tfrac{5}{2}}+2\sqrt{1} +C \Rightarrow C=-\dfrac{22}{5}$.\\
Suy ra $F(x)=\dfrac{2}{5}x^{\tfrac{5}{2}}+2\sqrt{x} -\dfrac{22}{5}$.\\
Vậy $F(0)=-4{,}4$.
}
\end{ex}

\begin{ex}%[2D4V2-6]
Một chất điểm $A$ xuất phát từ $O$, chuyển động thẳng với vận tốc biến thiên theo thời gian bởi quy luật $v \left(t\right) = \dfrac{1}{100}t^2 + \dfrac{13}{30}t$ (m/s), trong đó $t$ (giây) là khoảng thời gian tính từ lúc $A$ bắt đầu chuyển động. Từ trạng thái nghỉ, một chất điểm $B$ cũng xuất phát từ $O$, chuyển động thẳng cùng hướng với $A$ nhưng chậm hơn $10$ giây so với $A$ và có gia tốc bằng $a$ (m/s$^2$ ) ( $a$ là hằng số). Sau khi $B$ xuất phát được $15$ giây thì đuổi kịp $A$. Vận tốc của $B$ tại thời điểm đuổi kịp $A$ bằng bao nhiêu m/s?
\shortans{$25$}
\loigiai{
Ta có $v_{B}(t) = \displaystyle\int a \cdot \mathrm{\,d}t = at + C$, $v_{B} (0) = 0 \Rightarrow C = 0 \Rightarrow v_{B} \left(t\right) = at$.\\
Quãng đường chất điểm $A$ đi được trong $25$ giây là
\[S_{A} = \displaystyle\int\limits_0^{25} \left(\dfrac{1}{100}t^2 + \dfrac{13}{30}t \right) \mathrm{\,d}t = \left(\dfrac{1}{300}t^3 + \dfrac{13}{60}t^2 \right) \Big|_0^{25} = \dfrac{375}{2}.\]
Quãng đường chất điểm $B$ đi được trong $15$ giây là
\[S_{B} = \displaystyle\int\limits_0^{15} at \cdot \mathrm{\,d}t = \dfrac{at^2}{2} \Big|_0^{15} = \dfrac{225a}{2}.\]
Ta có $\dfrac{375}{2} = \dfrac{225a}{2} \Leftrightarrow a = \dfrac{5}{3}$.\\
Vận tốc của $B$ tại thời điểm đuổi kịp $A$ là $v_{B} \left(15\right) = \dfrac{5}{3} \cdot 15 = 25$ (m/s).
}
\end{ex}

\begin{ex}%[2H5C2-7]
	Cho biết kim tự tháp Memphis tại bang Tennessee (Mỹ) có dạng hình chóp tứ giác đều $S.ABCD$ với chiều cao $98$ m và cạnh đáy $180$ m. Số đo góc giữa hai mặt bên bằng bao nhiêu độ (làm tròn kết quả đến hàng đơn vị)?
	\shortans{$63$}
	\loigiai{
	\immini{
	Gọi $O=AC \cap BD$. Vì $S.ABCD$ là hình chóp đều nên $S O \perp(A B C D)$.\\
	Ta có $AC=BD=AB \sqrt{2}=180 \sqrt{2}$.\\
	Chọn hệ trục $Oxyz$ như hình vẽ với $O(0;0;0)$, $C(90 \sqrt{2} ; 0 ; 0)$, $D(0 ; 90 \sqrt{2} ; 0)$, $B(0;-900\sqrt{2};0)$ và $S(0 ; 0 ; 98)$.
	}
	{
	\begin{tikzpicture}[scale=0.55, font=\footnotesize, line join=round, line cap=round, >=stealth]
	\def\bc{5} % cạnh BC
	\def\ba{3.5} % cạnh BA
	\def\h{5} % đường cao
	\def\gocB{40} % góc B của đáy
	\coordinate (B) at (0,0);
	\coordinate (A) at (\gocB:\ba);
	\coordinate (C) at (\bc,0);
	\coordinate (D) at ($(C)-(B)+(A)$);
	\coordinate (O) at ($(A)!.5!(C)$);
	\coordinate (S) at ($(O)+(90:\h)$);
	\coordinate (z) at ($(O)+(90:\h+1)$);
	\coordinate (y) at ($(O)!1.25!(D)$);
	\coordinate (x) at ($(O)!1.5!(C)$);
	\draw (B)--(C)--(D)--(S)--cycle (S)--(C);
	\draw[dashed] (C)--(A)--(D)--(B) (O)--(S)--(A)--(B);
	\draw [->] (S)--(z);
	\draw [->] (D)--(y);
	\draw [->] (C)--(x);
	\foreach \p/\i in {S/180, A/180,B/-60,D/40,C/-120,O/170}
	\fill (\p) circle (1.5pt) node[shift={(\i:3mm)}]{$\p$};
	\foreach \p/\i in {z/180,y/-90,x/40}
	\fill (\p)  node[shift={(\i:3mm)}]{$\p$};
	\end{tikzpicture}}
	\noindent
	Phương trình mặt phẳng $(SBC)$ dưới dạng đoạn chắn là
	\[ \dfrac{x}{90\sqrt{2}}+\dfrac{y}{-90 \sqrt{2}}+\dfrac{z}{98}=1\qquad \text{hay } 49x-49y+45\sqrt{2} z-4410 \sqrt{2}=0.\]
	Phương trình mặt phẳng $(SCD)$ dưới dạng đoạn chắn là\\ $ \dfrac{x}{90 \sqrt{2}}+\dfrac{y}{90 \sqrt{2}}+\dfrac{z}{98}=1$ hay $49x+49y+45\sqrt{2} z-a \sqrt{2}=0$.\\
	Khi đó hai mặt phẳng $(SBC)$ và $(SCD)$ có véc-tơ pháp tuyến lần lượt là  $\overrightarrow{n}_1=(49 ; -49 ; 45\sqrt{2})$, $\overrightarrow{n}_2=(49 ; 49 ; 45\sqrt{2})$.\\
	Suy ra \[\cos ((S BC),(SCD))=\dfrac{|\overrightarrow{n}_1 \cdot \overrightarrow{n}_2|}{|\overrightarrow{n}_1| \cdot|\overrightarrow{n}_2|}=\dfrac{\left|49\cdot 49-49\cdot 49+45\sqrt{2}\cdot 45\sqrt{2} \right| }{\sqrt{49^2+(-49)^2+(45\sqrt{2})^2}\cdot \sqrt{49^2+49^2+(45\sqrt{2})^2}}=\dfrac{\sqrt{2025}}{4426}.\]
	Do đó $((SBC),(SCD))\approx 63^{\circ}$.
	}
	\end{ex}

	\begin{ex}%[2D6C1-4]
		Kết quả một cuộc khảo sát các vụ tai nạn giao thông ô tô về mối quan hệ giữa việc thắt dây an toàn của người lái xe khi xảy ra tai nạn giao thông và nguy cơ tử vong của người lái xe khi xảy ra tai nạn giao thông cho thấy:
		\begin{itemize}
		\item Tỉ lệ người lái xe tử vong khi xảy ra tai nạn giao thông là $0{,4}\%$.
		\item Tỉ lệ người lái xe không thắt dây an toàn giao thông khi xảy ra tai nạn giao thông là $28 \%$.
		\item Tỉ lệ người lái xe tử vong khi xảy ra tai nạn giao thông trong trường hợp không thắt dây an toàn là $0{,}3 \%$.
		\end{itemize}
		Hỏi theo kết quả khảo sát trên, việc thắt dây an toàn của người lái xe ô tô sẽ làm giảm khả năng tử vong là bao nhiêu lần? (làm tròn đến hàng phần mười).
		\shortans{$7{,}7$}
		\loigiai{
		Chọn ngẫu nhiên một một vụ tai nạn giao thông của cuộc khảo sát trên. Xét các biến cố:\\
		$A$: "Người lái xe đó tử vong khi xảy ra tai nạn giao thông."\\
		$B$: "Người lái xe đó không thắt dây an toàn khi xảy ra tai nạn giao thông."\\
		Ta có $P(A)= 0{,4}\%$; $P(B)= 28\%$; $P(A\cap B)=0{,}3\%$.\\
		Xác suất người lái xe đó tử vong khi xảy ra tai nạn giao thông trong trường hợp không thắt dây an toàn là
		\[P(A|B)=\dfrac{P(A\cap B)}{P(B)}=\dfrac{3}{280}.\]
		Xác suất người lái xe đó có thắt dây an toàn giao thông là $P(\overline{B})=72\%$.\\
		Xác suất người lái xe đó tử vong khi xảy ra tai nạn giao thông trong trường hợp có thắt dây an toàn là
		\[P(A|\overline{B})=\dfrac{P(A\cap \overline{B})}{P(\overline{B})}=\dfrac{P(A)-P(A\cap B)}{P(\overline{B}}=\dfrac{1}{720}.\]
		Ta có
		\[\dfrac{P(A|B)}{P(A|\overline{B})}=\dfrac{54}{7}\approx7{,}7.\]
		Vậy theo khảo sát trên, việc thắt dây an toàn của người lái xe ô tô sẽ làm giảm khả năng tử vong khoảng $7{,}7$ lần.
		}
		\end{ex}
\TL
		\begin{ex}%[2H5V2-4]
			Trong không gian với hệ trục tọa độ $O x y z$, cho điểm $M(3 ; 3 ;-2)$ và hai đường thẳng $d_1\colon \dfrac{x-1}{1}=\dfrac{y-2}{3}=\dfrac{z}{1} ;$ $ d_2\colon \dfrac{x+1}{-1}=\dfrac{y-1}{2}=\dfrac{z-2}{4}$. Viết phương trình đường thẳng $d$ đi qua $M$ vuông góc $d_1,$ $ d_2$.
			% \shortans{$3$}
			\loigiai{
			Ta có $d_1$ có véc-tơ chỉ phương là $\overrightarrow{u_1}=(1;3;1)$, $d_2$ có véc-tơ chỉ phương là $\overrightarrow{u_2}=(-1;2;4)$.\\
			Đường thẳng $d$ đi qua $M$ vuông góc với $d_1$, $d_2$ có véc-tơ chỉ phương là $\overrightarrow{u}=\dfrac15[\overrightarrow{u_1}, \overrightarrow{u_2}] = \left(2; -1; 1\right)$ nên có phương trình tham số là $\heva{&x=3+2t\\&y=3-t\\&z=-2+t}$.
			}
			\end{ex}

\begin{ex}%[2D6C2-4]
Tỉ lệ người dân đã tiêm vắc xin phòng bệnh A ở một địa phương là $65\%$. Trong số những  người đã tiêm phòng, tỉ lệ mắc bệnh A là $5\%$ còn trong số những người chưa tiêm, tỉ lệ mắc bệnh A là $17\%$. Gặp ngẫu nhiên một người ở địa phương đó. Biết rằng người đó mắc bệnh X. Khi đó xác suất người đó không tiêm vắc xin phòng bệnh X là bao nhiêu?
% \shortans{$65$}
\loigiai{
Gọi $A$ là biến cố \lq\lq người đó mắc bệnh $X$\rq\rq\,và $B$ là biến cố \lq\lq Gặp được người đã tiêm vắc xin phòng bệnh X\rq\rq.\\
Theo công thức xác suất toàn phần, ta có
\begin{eqnarray*}
\mathrm{P}(A) & = &\mathrm{P}(B) \cdot \mathrm{P}(A \mid B)+\mathrm{P}(\overline{B}) \cdot \mathrm{P}(A \mid \overline{B}) \\
& = &0,65 \cdot 0{,}05+0{,}35 \cdot 0{,}17=0{,}092.
\end{eqnarray*}
Suy ra
\begin{eqnarray*}
\mathrm{P}(\overline{B} \mid A) & =& \dfrac{\mathrm{P}(A \overline{B})}{\mathrm{P}(A)}=\dfrac{\mathrm{P}(\overline{B}) \mathrm{P}(A \mid \overline{B})}{\mathrm{P}(A)} \\
& =& \dfrac{0,35 \cdot 0{,}17}{0{,}092}=\dfrac{119}{184}.
\end{eqnarray*}
% Khi đó $a=119$ và $b=184$, suy ra $b-a=65$.
}
\end{ex}

\begin{ex}%[2H5C1-7]
Trong không gian $Oxyz$, cho hai điểm $A(2;1;0)$, $B(1;2;0)$ và điểm $M$ di động trên tia $Oz$. Gọi $H$, $K$ lần lượt là hình chiếu vuông góc của $A$ lên $OB$ và $MB$. Đường thẳng $HK$ cắt trục $Oz$ tại điểm $N$. Khi thể tích khối tứ diện $ABMN$ nhỏ nhất thì mặt phẳng $(AHK)$ có dạng $ax+by+cz-4=0$. Giá trị của $a+b+c$ bằng
\shortans{$1$}
\loigiai{
\immini
{Ta có $A(2;1;0)$, $B(1;2;0)$ $\Rightarrow A,~ B \in (Oxy)$.\\
Có $\heva{&AK\perp MB\\&AH\perp OB;~AH\perp OM \Rightarrow AH \perp (OBM) \Rightarrow AH \perp MB}$ mà $AK\perp MB$ nên
$ MB \perp (AHK)$.\\
Gọi $M(0;0;m)$, $(m>0)$ thuộc tia $Oz$ khi đó $\overrightarrow{MB}=(1;2;-m)$\\
$\Rightarrow (AHK)\colon 1(x-2)+2(y-1)-mz=0$.\\
$\Rightarrow N=HK \cap Oz =(AHK) \cap Oz \Rightarrow N\left(0;0;-\dfrac{4}{m}\right)$.\\
Ta có
\allowdisplaybreaks
$\begin{aligned}[t]
V_{ABMN}&=V_{M.OAB}+V_{N.OAB}=\dfrac{1}{3}S_{OAB}\cdot OM +\dfrac{1}{3}S_{OAB}\cdot ON\\
&=\dfrac{1}{3}S_{OAB}(OM+ON)=\dfrac{1}{3}S_{OAB}\cdot MN
\end{aligned}$\\
Vì $S_{OAB}$ không đổi nên $V_{ABMN}$ nhỏ nhất khi và chỉ khi $MN$ nhỏ nhất.\\
Ta có $MN=m+\dfrac{4}{m}\geq 2\sqrt{m\cdot \dfrac{4}{m}}=4$.\\
Dấu bằng xảy ra khi $m=\dfrac{4}{m}\Rightarrow m=2$.\\
Vậy $(AHK)\colon x+2y-2z-4=0 \Rightarrow a+b+c=1+2-2=1$.}
{\begin{tikzpicture}[scale=0.7, font=\footnotesize, line join=round, line cap=round,>=stealth]
\def\xmin{-2};\def\ymin{-3};\def\xmax{6};\def\ymax{5};
\coordinate (O) at (0,0);
\coordinate (B) at (2,-1);
\coordinate (A) at (3,0);
\coordinate (M) at (0,2);
\coordinate (K) at ($(M)+2/5*(B)-2/5*(M)$);
\coordinate (H) at ($(O)+1/4*(B)-1/4*(O)$);
\coordinate (N) at (intersection of M--O and K--H);
\draw (A)--(B)--(O)--(M)--cycle (M)--(B)--(N) (A)--(K)--(N)--(O);
\draw [dashed] (O)--(A)--(H) (A)--(N);
\draw[black] pic[draw, angle radius=2mm, angle eccentricity=1.5]{right angle=B--K--A};
\draw[black] pic[draw, angle radius=2mm, angle eccentricity=1.5]{right angle=B--H--A};
\foreach \t/\g in {O/180,B/-90,A/0,M/70,K/90,H/200,N/-90}{\draw[fill=white] (\t) circle (1pt) node[shift={(\g:7pt)},font=\scriptsize]{$\t$};}
\end{tikzpicture}}
}
\end{ex}
\Closesolutionfile{ans}


\Closesolutionfile{ansbook}

% \begin{name}
	{\tenchude}
	{TOÁN 12}
	{LỚP TOÁN THẦY PHÁT}
	{Thời gian: 90 phút - Không kể thời gian phát đề}
\end{name}
\Opensolutionfile{ansbook}[ans/ansbookDe3]
\TN
\Opensolutionfile{ans}[ans/ansDe3-TN1]
\begin{ex}%[Dự án 2025 - đề cấu trúc mới, Nguyễn Kiều Nhã Tú]%[2D4N1-1]
Mệnh đề nào dưới đây \textbf{sai}?
\choice
{$\displaystyle\int f'(x)\mathrm{\,d}x=f(x)+C$ với mọi hàm số $f(x)$ có đạo hàm trên $\mathbb{R}$}
{$\displaystyle\int[f(x)+g(x)]\mathrm{\,d}x=\displaystyle\int f(x) \mathrm{\,d}x+\displaystyle\int g(x)\mathrm{\,d}x$ với mọi hàm số $f(x)$, $g(x)$ có đạo hàm trên $\mathbb{R}$}
{\True $\displaystyle\int kf(x)\mathrm{\,d}x=k\displaystyle\int f(x) \mathrm{\,d}x$ với mọi hằng số $k$ và với mọi hàm số $f(x)$ có đạo hàm trên $\mathbb{R}$}
{$\displaystyle\int[f(x)-g(x)]\mathrm{\,d}x=\displaystyle\int f(x) \mathrm{\,d}x-\displaystyle\int g(x)\mathrm{\,d}x$ với mọi hàm số $f(x)$, $g(x)$ có đạo hàm trên $\mathbb{R}$}
\loigiai{
Theo tính chất của nguyên hàm, $\displaystyle\int kf(x)\mathrm{\,d}x=k \displaystyle\int f(x)\mathrm{\,d}x$ sai khi $k=0$.
}
\end{ex}

\begin{ex}%[2D4N1-2]
Họ nguyên hàm của hàm số $f(x)=x^3$ là
\choice
{$4x^4+C$}
{$3x^2+C$}
{$x^4+C$}
{\True $\dfrac{1}{4}x^4+C$}
\loigiai{
Ta có
$\displaystyle\int x^3\mathrm{\,d}x=\dfrac{1}{4}x^4+C$.
}
\end{ex}

\begin{ex}%[2D4N2-1]
Cho $\displaystyle\int\limits_0^1f(x)\mathrm{\,d}x=2$ và $\displaystyle\int\limits_0^1g(x)\mathrm{\,d}x=5$, khi $\displaystyle\int\limits_0^1\left[f(x)-2g(x)\right]\mathrm{\,d}x$ bằng
\choice
{\True $-8$}
{$1$}
{$-3$}
{$12$}
\loigiai{
Ta có $\displaystyle\int\limits_0^1\left[f(x)-2g(x)\right]\mathrm{\,d}x=\displaystyle\int\limits_0^1f(x)\mathrm{\,d}x-2\displaystyle\int\limits_0^1g(x)\mathrm{\,d}x=2-2\cdot 5=-8$.}
\end{ex}

\begin{ex}%[2H5N1-1]
Trong không gian với hệ tọa độ $O x y z$, phương trình nào sau đây là phương trình của mặt phẳng $O z x$ ?
\choice
{$x=0$}
{$y-1=0$}
{\True $y=0$}
{$z=0$}
\loigiai{
Ta có mặt phẳng $(Oxz)$ đi qua điểm $O(0 ; 0 ; 0)$ và vuông góc với trục $O y$ nên có VTPT $\vec{n}=(0 ; 1 ; 0)$.\\
Do đó phương trình của mặt phẳng $(Oxz)$ là $y=0$.
}
\end{ex}

\begin{ex}%[Mức độ 1]%[BG-12-New-4in1, Hiệp Hà]%[2H5N1-2]
Vectơ nào dưới đây là một vectơ pháp tuyến của $(Oyz)$?
\choice
{\True $\vec{n_1}=(2;0;0)$}
{$\vec{n_2}=(1;1;0)$}
{$\vec{n_3}=(0;3;0)$}
{$\vec{n_4}=(0;0;-1)$}
\loigiai{
Ta có $Ox \perp (Oyz)$ nên $\vec{i}=(1;0;0)$ là một vectơ pháp tuyến của $(\alpha)$.\\
Khi đó, $\vec{n_1}=2\vec{i}$ là một vectơ pháp tuyến của $(\alpha)$.
}
\end{ex}

\begin{ex}%[2H5N2-1]%[Dự án EX-TF-TLN lần 3 - Nguyễn Thắng]
Trong không gian $Oxyz$, cho đường thẳng $\Delta$ đi qua điểm $A(x_0;y_0;z_0)$ và có véc-tơ chỉ phương $\vec{u}=(a;b;c)\ne \vec{0}$. Khi đó hệ phương trình nào sau đây là phương trình tham số của đường thẳng $\Delta$?
\choice
{$\heva{&x=x_0-at\\&y=y_0+bt\\&z=z_0+ct}$}
{$\heva{&x=x_0+at\\&y=y_0-bt\\&z=z_0+ct}$}
{$\heva{&x=x_0-at\\&y=y_0+bt\\&z=z_0-ct}$}
{\True $\heva{&x=x_0+at\\&y=y_0+bt\\&z=z_0+ct}$}
\loigiai{

}
\end{ex}

\begin{ex}%[2H5N2-2]%[Dự án 2025 - Đề cấu trúc mới của Bộ theo [Thành Đức Trung]
Trong không gian $Oxyz$, cho đường thẳng $d\colon \heva{&x=1-t\\&y=-2+2t\\&z=1+t}$. Véc-tơ nào dưới đây là véc-tơ chỉ phương của $d$?
\choice
{$\overrightarrow{u}=(1;-2;1)$}
{$\overrightarrow{u}=(1;2;1)$}
{$\overrightarrow{u}=(-1;-2;1)$}
{\True $\overrightarrow{u}=(-1;2;1)$}
\loigiai
{
Ta có véc-tơ chỉ phương của $d$ là $\overrightarrow{u}=(-1;2;1)$.
}
\end{ex}

\begin{ex}%[2H5N3-2]
Khối cầu $(S)$ có bán kính $R$ có thể tích bằng
\choice
{$4\pi{R^2}$}
{\True $\dfrac{4}{3}\pi{R^3}$}
{$\dfrac{1}{3}\pi{R^3}$}
{$\pi{R^3}$}
\loigiai{
Thể tích khối cầu được tính bằng công thức $V=\dfrac{4}{3}\pi{R^3}$.}
\end{ex}

\begin{ex}%[2D6N1-1]
Cho hai biến cố độc lập $A$, $B$. Chọn khẳng định \textbf{sai} trong các khẳng định sau.
\choice
{$\mathrm{P}(A\cap B)=\mathrm{P}(A)\cdot \mathrm{P}(B)$}
{$\mathrm{P}(A\mid B)=\mathrm{P}(A)$}
{\True $\mathrm{P}(A\mid \overline{B})=\mathrm{P}(\overline{B})$}
{ $\mathrm{P}(B\mid A)=\mathrm{P}(B)$}
\loigiai{
$\mathrm{P}(A\mid \overline{B})=\mathrm{P}(A)$.
}
\end{ex}

\begin{ex}%[2D6N2-1]%[Lê Công Trường]
Biến cố $A_0$ và $A_1$ là hai biến cố ngẫu nhiên thoả mãn $\mathrm{P}(A_0)>0$ và $0<\mathrm{P}(A_1)<1$. Khi đó công thức Bayes là
\choice
{$\mathrm{P}(A_0\mid  A_1)=\dfrac{\mathrm{P}(A_1)\cdot\mathrm{P}(A_0\mid A_1)}{\mathrm{P}(A_1)\cdot\mathrm{P}(A_0\mid A_1)+\mathrm{P}(\overline{A_1})\cdot\mathrm{P}(A_0\mid \overline{A_0})}$}
{\True $\mathrm{P}(A_1\mid A_0)=\dfrac{\mathrm{P}(A_1)\cdot\mathrm{P}(A_0\mid A_1)}{\mathrm{P}(A_1)\cdot\mathrm{P}(A_0\mid A_1)+\mathrm{P}(\overline{A_1})\cdot\mathrm{P}(A_0\mid \overline{A_1})}$}
{$\mathrm{P}(A_1\mid A)=\dfrac{\mathrm{P}(A_1)\cdot\mathrm{P}(A_0\mid A_1)}{\mathrm{P}(A_1)\cdot\mathrm{P}(A_0\mid A_1)+\mathrm{P}(\overline{A_1})\cdot\mathrm{P}(A_0\mid \overline{A_0})}$}
{$\mathrm{P}(A_1\mid A_0)=\dfrac{\mathrm{P}(A_0)\cdot\mathrm{P}(A_0\mid A_1)}{\mathrm{P}(A_1)\cdot\mathrm{P}(A_0\mid A_1)+\mathrm{P}(\overline{A_0})\cdot\mathrm{P}(A_0\mid \overline{A_0})}$}
\loigiai{ Biến cố $A_0$ và $A_1$ là hai biến cố ngẫu nhiên thoả mãn $\mathrm{P}(A_0)>0$ và $0<\mathrm{P}(A_1)<1$. Khi đó công thức Bayes là \[\mathrm{P}(A_1\mid A_0)=\dfrac{\mathrm{P}(A_1)\cdot\mathrm{P}(A_0\mid A_1)}{\mathrm{P}(A_1)\cdot\mathrm{P}(A_0\mid A_1)+\mathrm{P}(\overline{A_1})\cdot\mathrm{P}(A_0\mid \overline{A_1})}.\] }
\end{ex}

\begin{ex}%[2D6N2-3]%[Dự án EX-TF-TLN lần 4 - Quan Ón]
Cho hai biến cố $A$, $B$ thoả mãn $\mathrm{P}(A) = 0{,}4$; $\mathrm{P}(B) = 0{,}3$ ; $\mathrm{P}(A\mid B) = 0{,}25$. Khi đó, $\mathrm{P}(B\mid A)$ bằng
\choice
{$0{,}6667$}
{\True $0{,}1875$}
{$0{,}3195$}
{$0{,}5920$}
\loigiai{
Áp dụng công thức Bayes, ta có $\mathrm{P}(B\mid A) = \dfrac{\mathrm{P}(B)\cdot\mathrm{P}(A\mid B)}{\mathrm{P}(A)} = \dfrac{0{,}3\cdot 0{,}25}{0{,}4} = 0{,}1875$.
}
\end{ex}

\begin{ex}%[2H5N3-3]
Trong không gian Oxyz , cho mặt cầu $\left(S\right)$ có tâm $I\left(-1;2;3\right)$ và tiếp xúc với mặt phẳng $(P):2x-y-2z+1=0$. Phương trình của $\left(S\right)$ là
\choice
{$\left(x+1\right)^{2}+\left(y-2\right)^{2}+\left(z-3\right)^{2}=3$}
{$\left(x{-}1\right)^{2}+\left(y+2\right)^{2}+\left(z+3\right)^{2}=9$}
{\True$\left(x+1\right)^{2}+\left(y-2\right)^{2}+\left(z-3\right)^{2}=9$}
{$\left(x{-}1\right)^{2}+\left(y+2\right)^{2}+\left(z+3\right)^{2}=3$}
\loigiai
{
Bán kính mặt cầu $r=\mathrm{d}\left(I,\left(P\right)\right)=\dfrac{\left|2\left(-1\right)-2-2.3+1\right|}{\sqrt{2^{2}+\left(-1\right)^{2}+\left(-2\right)^{2}}}=3.$\\
Phương trình mặt cầu là $\left(x+1\right)^{2}+\left(y-2\right)^{2}+\left(z-3\right)^{2}=9.$
}
\end{ex}
\Closesolutionfile{ans}

\TNTF
\Opensolutionfile{ans}[ans/ansDe3-TN2]
\begin{ex}
    Cho hàm số $f(x)=3x^2-\dfrac{2}{x}$.
    \choiceTF
    {\True $\displaystyle\int f(x) \mathrm{\,d}x= x^3-2\ln |x|+C$}
    {Hàm số $G(x)=x^3-2\ln|2x|$ là một nguyên hàm của $f(x)$}
    {\True $\displaystyle\int_{1}^{2} f(x) \mathrm{\,d}x=a-\ln b$ với $a$, $b \in \mathbb{Z}$ thoả $a+b=7$}
    {\True Thể tích khối tròn xoay tạo thành khi quay hình phẳng giới hạn bởi đồ thị hàm số $y=f(x)$, trục hoành và các đường thẳng $x=1$ và $x=2$ bằng $\dfrac{199\pi}{5}$}
\loigiai{ }
\end{ex}
\begin{ex}%[2H5H1-2]
Trong không gian $O x y z$, cho mặt phẳng $(P)\colon 2 x+3 y+z-5=0$. Các mệnh đề sau đây đúng hay sai?
\choiceTF
{\True Mặt phẳng $(P)$ có một vectơ pháp tuyến là $\overrightarrow{n}=(2; 3; 1)$}
{Phương trình mặt phẳng $(Q)$ đi qua $A(-3;1;2)$ và song song với mặt phẳng $(P)$ là $2x+3y+z=0$}
{Đường thẳng $d$ có phương trình tham số $\heva{&x=1+2t\\&y=3t\\&z=3+t}$ song song với mặt phẳng $(P)$}
{Mặt cầu $(S)$ có phương trình $x^2+y^2+z^2-2x-3y-5=0$ có tâm nằm trên mặt phẳng $(P)$}
\loigiai{
}
\end{ex}
\Closesolutionfile{ans}

\TNSA
\Opensolutionfile{ans}[ans/ansDe3-TN3]
\begin{ex}%[2D4H1-4]
Hàm số $f(x)$ có đạo hàm liên tục trên $\mathbb{R}$ và $f'(x)=1+\mathrm{e}^{2x}$, $\forall x$; $f(0)=2$. Tính giá trị của $f(2)$. (Làm tròn đến số thập phân thứ nhất)
\shortans{$30{,}8$}
\loigiai{
Hàm số $f(x)=\displaystyle \int (1+\mathrm{e}^{2x})\mathrm{\,d}x=x+\dfrac{1}{2}\mathrm{e}^{2x}+C$.\\
Do $f(0)=2\Leftrightarrow 2=\dfrac{1}{2}+C\Leftrightarrow C=\dfrac{3}{2}$.\\
Suy ra $f(x)=x+\dfrac{1}{2}\mathrm{e}^{2x}+\dfrac{3}{2}$.\\
Vậy $f(2)=30{,}8$.
}
\end{ex}

\begin{ex}%[2D4V2-6]
Một vật chuyển động trong $4$ giờ với vận tốc $v$ (km/h) phụ thuộc thời gian $t$ (h) có đồ thị của vận tốc như hình bên. Trong khoảng thời gian $3$ giờ kể từ khi bắt đầu chuyển động, đồ thị đó là một phần của đường parabol có đỉnh $I \left(2;9\right)$ với trục đối xứng song song với trục tung, khoảng thời gian còn lại đồ thị là một đoạn thẳng song song với trục hoành. Tính quãng đường $s$ mà vật di chuyển được trong $4$ giờ đó (đơn vị tính bằng km).
\begin{center}
\begin{tikzpicture}[>=stealth,scale=0.45]
% Vẽ 2 trục, điền các số lên trục
\draw[->] (-0.5,0)--(0,0) node[below left]{$O$}--(5,0) node[above]{$t$};
\foreach \x in {2,3,4}
\draw[shift={(\x,0)},color=black] (0pt,2pt)--(0pt,-2pt)
node[below] { $\x$};
\draw[->,color=black] (0,-0.5)--(0,10) node[right]{$v$};
\foreach \y in {9}
\draw[shift={(0,\y)},color=black] (2pt,0pt) -- (-2pt,0pt)
node[left] {$\y$};
\clip(-1,-1) rectangle (5,10); %vùng đồ thị
%\draw[gray!50,thin,opacity=.5] (-1,-1) grid (4,10); %ô vuông
%Vẽ đồ thị
\draw[smooth,samples=100,domain=0:3, font=\footnotesize, line join=round, line cap=round, thick, smooth]
plot(\x,{(-9/4)*(\x)^2+9*(\x)});
\draw[smooth,samples=100, font=\footnotesize, line join=round, line cap=round, thick, smooth,domain=3:4]
plot(\x,{27/4});
% Vẽ thêm mấy cái râu ria
\draw[dashed] (3,0)--(3,27/4) circle(1.5pt);  \draw[dashed] (2,0)--(2,9) circle(1.5pt) node[above]{$I$}--(0,9) circle(1.5pt); \draw[dashed] (4,0)--(4,27/4) circle(1.5pt);
%Vẽ dấu chấm tròn
\fill (0cm,0cm) circle (1.5pt);
\end{tikzpicture}
\end{center}
\shortans{$27$}
\loigiai{
Gọi $\left(P\right) \colon y = ax^2+bx+c$.\\
Vì $\left(P\right)$ qua $O\left(0;0\right)$ và có đỉnh $I\left(2;9\right)$ nên dễ tìm được phương trình là $y = \dfrac{-9}{4}x^2 + 9x$.\\
Ngoài ra tại $x=3$ ta có $y = \dfrac{27}{4}$.\\
Vậy quãng đường cần tìm là: $S = \displaystyle\int\limits_0^3 \left(\dfrac{-9}{4}x^2 +9x \right) \mathrm{\,d}x + \displaystyle\int\limits_3^4 \dfrac{27}{4} \mathrm{\,d}x = 27$ (km).
}
\end{ex}

\begin{ex}%[EX-Ôn Tập TN 2025,  Lê Hoàng Anh]%[2H5V2-8]
    Trong không gian $Oxyz$, đài kiểm soát không lưu sân bay có toạ độ $O(0;0;0)$, đơn vị trên mỗi trục tính theo kilômét. Một máy bay chuyển động hướng về đài kiểm soát không lưu, bay qua hai vị trí $A(-500;-250;150)$, $B(-200;-200;100)$. Khi máy bay ở gần đài kiểm soát nhất, toạ độ của vị trí máy bay là $(a;b;c)$. Biết rằng $a-b+c=\dfrac{m}{n}$, tính giá trị biểu thức $m-n$?
    \shortans{$4\,881$}
    \loigiai{
    Véc-tơ $\overrightarrow{AB}=(300;50;-50)$ nên $\overrightarrow{u}=(6;1;-1)$ là một véc-tơ chỉ phương của đường thẳng $AB$.\\
    Phương trình đường thẳng $AB$ là
    \[
    \dfrac{x+500}{6}=\dfrac{y+250}{1}=\dfrac{z-150}{-1}.
    \]
    Gọi $H$ là hình chiếu của điểm $O$ trên đường thẳng $AB$ thì $OH$ là khoảng cách ngắn nhất giữa máy bay và đài kiểm soát. Khi đó $H(6t-500;t-250;-t+150)$.\\
    Ta có $\overrightarrow{OH} \cdot \overrightarrow{u}=(6t-500) \cdot 6+t-250-(-t+150)=0 \Leftrightarrow t=\dfrac{1\,700}{19}$.\\
    Suy ra toạ độ của vị trí máy bay khi đó là $\left(\dfrac{700}{19};\dfrac{-3\,050}{19};\dfrac{1\,150}{19}\right)$.\\ Vậy $a-b+c= \dfrac{4\,900}{19}$, suy ra $m-n=4\,881$.
    }
    \end{ex}

\begin{ex}%[2D6V1-3]
Tất cả các học sinh của trường Hạnh Phúc đều  tham gia câu lạc bộ bóng chuyền hoặc bóng rổ, mỗi học sinh chỉ tham gia đúng $1$ câu lạc bộ. Có $60$\% học sinh của trường tham gia câu lạc bộ bóng chuyền và $40$\% học sinh của trường tham gia câu lạc bộ bóng rổ. Số học sinh nữ chiếm $65$\% trong câu lạc bộ bóng chuyền và $25$\% trong câu lạc bộ bóng rổ. Chọn ngẫu nhiên $1$ học sinh. Xác suất chọn được học sinh nữ là bao nhiêu?

\shortans{0,49}
\loigiai
{
Gọi $A$ là biến cố \textquotedblleft Số học sinh thuộc câu lạc bộ bóng chuyền\textquotedblright.\\
$B$ là biến cố \textquotedblleft Số học sinh nữ\textquotedblright.\\
Khi đó $\heva{&P(A) = 60\% = 0{,}6 &\Rightarrow& P(\overline{A}) = 1-0{,}6 = 0{,}4\\ &P(B|A) = 65\% = 0{,}65 &\Rightarrow& P(\overline{B}|A) = 1-0{,}65 = 0{,}35\\ &P(B|\overline{A}) = 25\% = 0{,}25 &\Rightarrow& P(\overline{B}|\overline{A}) = 1-0{,}25=0{,}75.}$
\begin{center}
\begin{tikzpicture}[>=stealth]
%Khung 1
\draw (-0,-1) rectangle (2.2,0);
\draw (1.1,-0.5) node{Gốc};
%Mui ten 1,2
\draw [->] (2.2,-0.5)--(3.8,1.6) node[pos=0.5,sloped,above]{$0{,}6$};
\draw [->] (2.2,-0.5)--(3.8,-2.6) node[pos=0.5,sloped,below]{\color{red}$0{,}4$};
%Khung 2.1
\draw (4.5,3.0) node{\textbf{Thuộc câu lạc bộ}};
\draw (3.8,1.1) rectangle (5.1,2.1);
\draw (8.9/2,1.6) node{$A$} ;
%Khung 2.2
\draw (3.8,-2.1) rectangle (5.1,-3.1);
\draw (8.9/2,-2.6) node{$\overline{A}$};
%Mui ten 3,4
\draw [->] (5.1,1.6)--(6.5,2.6) node[pos=0.5,sloped,above]{$0{,}65$};
\draw [->] (5.1,1.6)--(6.5,0.6) node[pos=0.5,sloped,below]{\color{red}$0{,}35$};
%Mui ten 5,6
\draw [->] (5.1,-2.6)--(6.5,-1.6) node[pos=0.5,sloped,above]{$0{,}25$};
\draw [->] (5.1,-2.6)--(6.5,-3.6) node[pos=0.5,sloped,below]{\color{red}$0{,}75$};
%Khung 3.1
\draw (6.5,2.2) rectangle (7.7,3.2);
\draw (7.1,5.4/2) node{$B$} ;
%Khung 3.2
\draw (7.0,3.7) node{\textbf{Nữ}};
\draw (6.5,1.2) rectangle (7.7,0.2);
\draw (7.1,1.4/2) node{$\overline{B}$} ;
%Khung 3.3
\draw (6.5,-1.1) rectangle (7.7,-2.1);
\draw (7.1,-3.2/2) node{$B$} ;
%Khung 3.3
\draw (6.5,-2.9) rectangle (7.7,-3.9);
\draw (7.1,-3.4) node{$\overline{B}$} ;
%Kết quả
\draw (9.5,3.7) node{\textbf{Kết quả}};
\draw (9.5,2.7) node{$AB$};
\draw (9.5,0.7) node{$A \overline{B}$};
\draw (9.5,-1.6) node{$\overline{A}B$};
\draw (9.5,-3.4) node{$\overline{A}~\overline{B}$};
%Xác suất
\draw (12.5,3.7) node{\textbf{Xác suất}};
\draw (12.5,2.7) node{$0{,}39$};
\draw (12.5,0.7) node{$0{,}21$};
\draw (12.5,-1.6) node{$0{,}1$};
\draw (12.5,-3.4) node{$0{,}3$};
\end{tikzpicture}
\end{center}
Áp dụng công thức xác suất toàn phần để tính xác chọn được học sinh là nữ
\[P(B) = P(A) \cdot P(B|A) + P(\overline{A}) \cdot P(B|\overline{A}) = 0{,}6 \cdot 0{,}65 + 0{,}4 \cdot 0{,}25 = 0{,}49.\]
}
\end{ex}
\TL
\begin{ex}%[2H5H2-3]%[Dự án EX-TF-TLN lần 3 - Nguyen Chín Em]
Trong không gian với hệ tọa độ $Oxyz$. Viết phương trình tham số của $d$ biết $d$ đi qua điểm $M(3; 1; 5)$ và song song với hai mặt phẳng $(P)\colon 2x+3y-2z+1=0$ và $(Q)\colon x-3y+z-2=0$. %Khi đó đường thẳng $d$ đi qua $(6;y_0;z_0)$. Tính $z_0-y_0$.
% \shortans{$9$}
\loigiai{
Ta có  $\overrightarrow{n_P} = (2; 3; -2)$, $\overrightarrow{n_Q} = (1; -3; 1)$ lần lượt là véc-tơ pháp tuyến của hai mặt phẳng $(P)$ và $(Q)$. Do $d \parallel (P)$ và $d \parallel (Q)$ nên véc-tơ chỉ phương của $d$ là $\overrightarrow{u}=\left[\overrightarrow{n_P}, \overrightarrow{n_Q}\right]= (-3; -4; -9)$.\\
Phương trình tham số của $d$ là $\heva{&x = 3 - 3t\\&y = 1 - 4t\\&z = 5 - 9t}, (t\in\mathbb{R})$.\\
Với $t=-1$, suy ra đường thẳng $d$ đi qua $(6;5;14)$. Suy ra được $y_0=5;z_0=14\Rightarrow z_0-y_0=9$.
}
\end{ex}

\begin{ex}%[2D6C2-4]
Ở một khu rừng nọ có $7$ chú lùn, trong đó có $4$ chú luôn nói thật, $3$ chú còn lại nói thật với  xác suất $0{,}5$. Bạn Tuyết gặp ngẫu nhiên một chú lùn. Gọi $A$ là biến cố \lq\lq Chú lùn đó luôn nói thật\rq\rq\,và $B$ là biến cố \lq\lq Chú lùn đó tự nhận mình luôn nói thật\rq\rq. Biết rằng chú lùn mà bạn Tuyết gặp tự nhận mình là người luôn nói thật. Tính xác suất để chú lùn đó luôn nói thật (làm tròn hai chữ số thập phân).
% \shortans{$0{,}73$}
\loigiai{
Ta có $\mathrm{P}(A)=\dfrac{4}{7}$; $\mathrm{P}(\overline{A})=\dfrac{3}{7}$; $\mathrm{P}(B \mid A)=1$; $\mathrm{P}(B \mid \overline{A})=0{,}5$.\\
Theo công thức xác suất toàn phần, ta có
\begin{eqnarray*}
\mathrm{P}(B) & =& \mathrm{P}(A) \cdot \mathrm{P}(B \mid A)+\mathrm{P}(\overline{A}) \cdot \mathrm{P}(B \mid \overline{A}) \\
& = & \dfrac{4}{7} \cdot 1+\dfrac{3}{7} \cdot 0{,}5=\dfrac{11}{14}.
\end{eqnarray*}
Khi đó
\[
\mathrm{P}(A \mid B)=\dfrac{\mathrm{P}(AB)}{\mathrm{P}(B)}=\dfrac{\mathrm{P}(A) \cdot \mathrm{P}(B \mid A)}{\mathrm{P}(B)}=\dfrac{\dfrac{4}{7} \cdot 1}{\dfrac{11}{14}}=\dfrac{8}{11} \approx 0{,}73.
\]
}
\end{ex}

\begin{ex}%[2H5C1-7]
Người ta thiết kế một mái che hình chữ nhật $ ABCD $ phía trên sân khấu. Gắn hệ trục tọa độ $ Oxyz $ (đơn vị trên trục là mét), người ta xác định được toạ dộ của các điểm như sau: $ A(0;0;8)$, $B(0;20;8)$, $D(15;0;14)$, $C(15;20;14) $. Một cổng chào hình chữ nhật $ EFHG $ với tọa độ điểm $ G(8;0;4) $ dựng vuông góc với mặt đất. Người ta muốn làm các đoạn dây nối thanh ngang $ GE $ với mái che để gắn hoa và đèn led. Độ dài ngắn nhất của mỗi đoạn dây này bằng bao nhiêu mét? (làm tròn đến chữ số thập phân thứ nhất)
\begin{center}
\includegraphics[scale=.7]{images/2P5-1-H5-16}
\end{center}
% \shortans{$1{,}8$}
\loigiai{
Ta có $ A(0;0;8)$, $B(0;20;8)$, $D(15;0;14)$, $C(15;20;14) $.\\
Ta có $ \vec{AB}=(0;20;0) $, $\vec{AC}=(15;20;6)$ nên $ \vec{n}_1=\left[\vec{AB},\vec{AC}\right]=(80;0;300) $ là vectơ pháp tuyến của $ (ABCD) $.\\
Mà mặt phẳng mái che $ (ABCD) $ qua $ A(0;0;8)$ nên có phương trình
\[ 80(x-0)+0(y-0)+300(z-8)=0\Leftrightarrow 4x+15z-120=0 .\]
Độ dài ngắn nhất của dây nối thanh ngang $ GE $ với mái che là khoảng cách từ $ G $ đến mái che (mặt phẳng $ ABCD $) là \[ \mathrm{d}(G,(ABCD))=\dfrac{|4\cdot8+0+15\cdot4-120|}{\sqrt{4^2+0^2+15^2}}=\dfrac{28}{\sqrt{241}}=1{,}8\ (\text{m}). \]
}
\end{ex}
\Closesolutionfile{ans}
\Closesolutionfile{ansbook}

% \begin{name}
	{\tenchude}
	{TOÁN 12}
	{LỚP TOÁN THẦY PHÁT}
	{Thời gian: 90 phút - Không kể thời gian phát đề}
\end{name}
\Opensolutionfile{ansbook}[ans/ansbookDe4]
\TN
\Opensolutionfile{ans}[ans/ansDe4-TN1]
\begin{ex}%[2D4N1-1]
Cho hai hàm số $f(x)$, $g(x)$ là hàm số liên tục, có $F(x)$, $G(x)$ lần lượt là nguyên hàm của $f(x)$, $g(x)$. Xét các mệnh đề sau.
\begin{itemize}
\item[(i)] $F(x)+G(x)$ là một nguyên hàm của $f(x)+g(x)$.
\item[(ii)] $k\cdot F(x)$ là một nguyên hàm của $k\cdot f(x)$ với $k\in\mathbb{R}$.
\item[(iii)] $F(x)\cdot G(x)$ là một nguyên hàm của $f(x)\cdot g(x)$.
\end{itemize}
Các mệnh đề đúng là
\choice
{(ii) và (iii)}
{(i), (ii) và (iii)}
{(i) và (iii)}
{\True (i)}
\loigiai{
Theo tính chất của nguyên hàm, khẳng định (ii) sai khi $ k=0$, khẳng định (iii) sai.}
\end{ex}

\begin{ex}%[2D4N1-2]
Họ nguyên hàm của hàm số $y=x^2-3x+\dfrac{1}{x}$ là
\choice
{$\dfrac{x^3}{3}-\dfrac{3x^2}{2}-\ln\left|x\right|+C$}
{$\dfrac{x^3}{3}-\dfrac{3x^2}{2}+\ln x+C$}
{\True $\dfrac{x^3}{3}-\dfrac{3x^2}{2}+\ln\left|x\right|+C$}
{$\dfrac{x^3}{3}-\dfrac{3x^2}{2}+\dfrac{1}{x^2}+C$}
\loigiai{
Ta có
$\displaystyle\int \left(x^2-3x+\dfrac{1}{x}\right) \mathrm{\,d}x=\dfrac{x^3}{3}-\dfrac{3x^2}{2}+\ln\left|x\right|+C$.

}
\end{ex}

\begin{ex}%[2D4N2-1]Biên soạn:Phùng Hoàng Em Phản biện:Nguyễn Đắc Kiên
Cho hàm số $f(x)$ liên tục trên $\mathbb{R}$ và $a$ là số thực dương. Khẳng định nào sau đây là khẳng định đúng?
\choice
{$\displaystyle\int\limits_a^a f(x) \mathrm{\,d}x=1$}
{$\displaystyle\int\limits_a^af(x) \mathrm{\,d}x=a^2$}
{\True $\displaystyle\int\limits_a^af(x) \mathrm{\,d}x=0$}
{$\displaystyle\int\limits_a^a f(x) \mathrm{\,d}x=2a$}
\loigiai{
Gọi $F(x)$ là một nguyên hàm của $f(x)$.
Ta có $\displaystyle\int\limits_a^af(x) \mathrm{\,d}x=F(x)\Bigg|_a^a=F(a)-F(a)=0$.
}
\end{ex}

\begin{ex}%[2H5N1-1]%[Dự án Khối 12- Ex-TF-TLN-2024]%[VU Ngoc Hao]
\immini{
Cho hình hộp chữ nhật $ABCD.A'B'C'D'$. Bốn véc-tơ pháp tuyến  của mặt phẳng $\left(AA'B'B\right)$ là
\choice
{$\overrightarrow{AD}$, $\overrightarrow{A'D'}$, $ \overrightarrow{BD}$, $\overrightarrow{B'C'}$}
{$\overrightarrow{AD}$, $\overrightarrow{A'D'}$, $ \overrightarrow{BC}$, $\overrightarrow{BC'}$}
{$\overrightarrow{AC}$, $\overrightarrow{A'D'}$, $ \overrightarrow{BC}$, $\overrightarrow{B'C'}$}
{\True  $\overrightarrow{AD}$, $\overrightarrow{A'D'}$, $ \overrightarrow{BC}$, $\overrightarrow{B'C'}$}
}
{
\begin{tikzpicture}[scale=0.5, font=\footnotesize,line join=round, line cap=round, >=stealth]
\coordinate (A) at (0,0);
\coordinate (B) at (-2,-1.5);
\coordinate (D) at (5,0);
\coordinate (C) at ($(B)+(D)-(A)$);
\foreach \i in {A,B,C,D}{\coordinate (\i') at ($(\i)+(0,4)$);}
\draw (A')--(B')--(C')--(D')--cycle;
\draw (B)--(B') (C)--(C') (D)--(D')  (B)--(C)--(D);
\draw[dashed,thin](B)--(A)--(A') (A)--(D);
\foreach \i/\g in {A'/90,B'/90,C'/90,D'/90,A/-90,B/-90,C/-90,D/-90}{\draw[fill=black](\i) circle (1pt) ($(\i)+(\g:5mm)$) node[scale=1]{$\i$};}
\end{tikzpicture}
}
\loigiai{
Bốn véc-tơ pháp tuyến của mặt phẳng $\left(AA'B'B\right)$ là  $\overrightarrow{AD}$, $\overrightarrow{A'D'}$, $ \overrightarrow{BC}$, $\overrightarrow{B'C'}$.
}
\end{ex}

\begin{ex}%[Mức độ 1]%[BG-12-New-4in1, Hiệp Hà]%[2H5N1-2]
Cho $(\alpha)$ vuông góc với giá của $\vec{a}=(2;-1;3)$. Vectơ nào dưới đây là vectơ pháp tuyến của $(\alpha)$?
\choice
{$\vec{n_1}=(-2;1;3)$}
{\True $\vec{n_2}=(-2;1;-3)$}
{$\vec{n_3}=(4;2;6)$}
{$\vec{n_4}=(4;-2;-6)$}
\loigiai{
$(\alpha)$ vuông góc với giá của $\vec{a}=(2;-1;3)$ nên $\vec{a}$ là một vectơ pháp tuyến của $(\alpha)$.\\
Do đó $\vec{n_2}=-\vec{a}$ cũng là một vectơ pháp tuyến của $(\alpha)$.
}
\end{ex}

\begin{ex}%[2H5N2-1]%[Dự án EX-TF-TLN lần 3 - Nguyễn Thắng]
Trong không gian $Oxyz$, cho đường thẳng $\Delta$ đi qua điểm $M(x_0;y_0;z_0)$ và có véc-tơ chỉ phương $\vec{u}=(a;b;c)$ và $abc\ne 0$. Khi đó hệ phương trình nào sau đây là phương trình chính tắc của đường thẳng $\Delta$?
\choice
{$\dfrac{x-x_0}{a}=\dfrac{y+y_0}{b}=\dfrac{z+z_0}{c}$}
{$\dfrac{x+x_0}{a}=\dfrac{y+y_0}{b}=\dfrac{z+z_0}{c}$}
{$\dfrac{x+x_0}{a}=\dfrac{y+y_0}{b}=\dfrac{z-z_0}{c}$}
{\True $\dfrac{x-x_0}{a}=\dfrac{y-y_0}{b}=\dfrac{z-z_0}{c}$}
\loigiai{

}
\end{ex}

\begin{ex}%[2H5N2-2]%[Dự án 2025 - Đề cấu trúc mới của Bộ theo [Thành Đức Trung]
Trong không gian $Oxyz$, cho mặt phẳng $(P)\colon x-2y-3z-2=0$. Đường thẳng $d$ vuông góc với mặt phẳng $(P)$ có một véc-tơ chỉ phương là
\choice
{$\overrightarrow{u}_{1}=(1;-2;-2)$}
{\True $\overrightarrow{u}_{2}=(1;-2;-3)$}
{$\overrightarrow{u}_{4}=(1;2;3)$}
{$\overrightarrow{u}_{3}=(1;-3;-2)$}
\loigiai
{
Ta có $(P)\colon x-2y-3z-2=0$, suy ra một véc-tơ pháp tuyến của $(P)$ là $\overrightarrow{u}_{2}=(1;-2;-3)$.
}
\end{ex}

\begin{ex}%[2H5N3-2]
Trong không gian $Oxyz$, cho mặt cầu $(S)\colon x^2+(y-4)^2+(z-1)^2=25$. Tọa độ tâm $I$ và bán kính $R$ của mặt cầu $(S)$ là
\choice
{$I(0;-4;-1)$, $R=25$}
{$I(0;-4;-1)$, $R=5$}
{$I(0;4;1)$, $R=25$}
{\True $I(0;4;1)$, $R=5$}
\loigiai{
Mặt cầu $(S)$ có tâm $I(0;4;1)$ và bán kính $R=5$.
}
\end{ex}

\begin{ex}%[12-PTMH-1-2025]%[Võ Thị Thùy Trang]%[2D6N1-1]
Cho hai biến cố $A$ và $B$ bất kì, với $\mathrm{P}(B)>0$. Công thức tính xác suất nào sau đây là đúng?
\choice
{$\mathrm{P}(A\mid B)= \dfrac{\mathrm{P}(A)}{\mathrm{P}(B)}$}
{\True $\mathrm{P}(A\mid B)= \dfrac{\mathrm{P}(AB)}{\mathrm{P}(B)}$}
{$\mathrm{P}(A\mid B)= \dfrac{\mathrm{P}(AB)}{\mathrm{P}(A)}$}
{$\mathrm{P}(A\mid B)= \dfrac{\mathrm{P}(B)}{\mathrm{P}(A)}$}
\loigiai
{
Theo tính chất, công thức đúng là $\mathrm{P}(A\mid B)= \dfrac{\mathrm{P}(AB)}{\mathrm{P}(B)}$.
}
\end{ex}

\begin{ex}%[2D6N2-1]%[Lê Công Trường]
Giả sử tỉ lệ người dân của tỉnh Khánh Hòa nghiện thuốc lá là $\mathrm{P}(A)$; tỉ lệ người bị bệnh phổi
trong số người nghiện thuốc lá là $\mathrm{P}(B)$, trong số người không nghiện thuốc lá là $\mathrm{P}(B\mid \overline{A})$. Hỏi khi ta gặp ngẫu nhiên một người dân của tỉnh Khánh Hòa thì khả năng mà đó bị bệnh phổi là
\choice
{$\mathrm{P}(A)=\mathrm{P}(B)\cdot\mathrm{P}(A\mid B)+\mathrm{P}(\overline{B})\cdot\mathrm{P}(A\mid \overline{B})$}
{$\mathrm{P}(A)=\mathrm{P}(A)\cdot\mathrm{P}(A\mid B)+\mathrm{P}(\overline{A})\cdot\mathrm{P}(A\mid \overline{B})$}
{\True $\mathrm{P}(B)=\mathrm{P}(A)\cdot\mathrm{P}(B\mid A)+\mathrm{P}(\overline{A})\cdot\mathrm{P}(B\mid \overline{A})$}
{$\mathrm{P}(A\mid B)=\dfrac{\mathrm{P}(A)\cdot\mathrm{P}(B\mid A)}{\mathrm{P}(B)}$}
\loigiai{Khi ta gặp ngẫu nhiên một người dân của tỉnh Khánh Hòa thì khả năng mà đó bị bệnh phổi là \[\mathrm{P}(B)=\mathrm{P}(A)\cdot\mathrm{P}(B\mid A)+\mathrm{P}(\overline{A})\cdot\mathrm{P}(B\mid \overline{A}).\]}
\end{ex}

\begin{ex}%[2D6N2-3]%[Dự án EX-TF-TLN lần 4 - Quan Ón]
Cho hai biến cố $A$, $B$ sao cho $\mathrm{P}(A) = 0{,}6$; $\mathrm{P}(B) = 0{,}4$ ; $\mathrm{P}(B\mid A) = 0{,}2$. Khi đó, $\mathrm{P}(A\mid B)$ bằng
\choice
{$0{,}11$}
{$0{,}57$}
{$0{,}83$}
{\True $0{,}30$}
\loigiai{
Áp dụng công thức Bayes, ta có $\mathrm{P}(A\mid B) = \dfrac{\mathrm{P}(A)\cdot\mathrm{P}(B\mid A)}{\mathrm{P}(B)} = \dfrac{0{,}6\cdot 0{,}2}{0{,}4} = 0{,}3$.
}
\end{ex}

\begin{ex}%[2H5N3-3]
Trong không gian $Oxyz$, phương trình mặt cầu tâm $I(-1;2;0)$ và bán kính bằng $2$ là
\choice
{$(x+1)^{2}+(y-2)^{2}+z^{2}=4$}
{$(x-1)^{2}+(y+2)^{2}+z^{2}=2$}
{$(x+1)^{2}+(y-2)^{2}+z^{2}=2$}
{\True $(x-1)^{2}+(y+2)^{2}+z^{2}=4$}
\loigiai{
Phương trình mặt cầu tâm $I(-1;2-0)$ và bán kính bằng $2$ là \[(x-(-1))^{2}+(y-2)^{2}+z^{2}=4\] hay \[(x+1)^{2}+(y-2)^{2}+z^{2}=4.\]
}
\end{ex}
\Closesolutionfile{ans}

\TNTF
\Opensolutionfile{ans}[ans/ansDe4-TN2]
\begin{ex}%[2025-TLOT-2018,Trần Xuân Hòa]%[2D4H2-2]
Cho hàm số $f(x)=x(x^2+3)$. Xét $I=\displaystyle\int\limits_{-1}^1|f(x)|\mathrm{\; d}x$.
\choiceTF
{\True Đặt $I_1=\displaystyle\int\limits_{0}^1|f(x)|\mathrm{\; d}x$ và $I_2=\displaystyle\int\limits_{-1}^0|f(x)|\mathrm{\; d}x$. Khi đó $I_1=I_2$}
{Giá trị $I=0$}
{\True Số thực dương $m$ để $\displaystyle\int\limits_0^m|f(x)|\mathrm{\;d}x=4$ bằng $\sqrt{2}$}
{Số thực $a$ để $\displaystyle\int\limits_0^1x(x^2+3-a\sqrt{x})\mathrm{\; d}x=0$ bằng $4$}
\loigiai{
\begin{itemchoice}
\itemch Đúng.
\begin{itemize}
\item Ta có $I_1=\displaystyle\int\limits_0^1(x^3+3x)\mathrm{\; d}x=\left(\dfrac{x^4}{4}+\dfrac{3x^2}{2}\right)\bigg |_0^1=\dfrac{7}{4}$.
\item Ta có $I_2=\displaystyle\int\limits_{-1}^0(-x^3-3x)\mathrm{\; d}x=\left(-\dfrac{x^4}{4}-\dfrac{3x^2}{2}\right)\bigg |_{-1}^0=\dfrac{7}{4}$.
\end{itemize}
Do đó $I_1=I_2$.
\itemch Sai. Ta có $I=\displaystyle\int\limits_{-1}^0|f(x)|\mathrm{\; d}x+\displaystyle\int\limits_{0}^1|f(x)|\mathrm{\; d}x=I_1+I_2=\dfrac{7}{4}+\dfrac{7}{4}=\dfrac{7}{2}$.
\itemch Đúng. Ta có
\begin{eqnarray*}
&&\displaystyle\int\limits_0^m|f(x)|\mathrm{\;d}x=4\\
&\Leftrightarrow&\displaystyle\int\limits_0^{m}(x^3+3x)\mathrm{\;d}x=4\\
&\Leftrightarrow&\left(\dfrac{x^4}{4}+\dfrac{3x^2}{2}\right)\bigg |_0^m=4\\
&\Leftrightarrow&\dfrac{m^4}{4}+\dfrac{3m^2}{2}-4=0\\
&\Leftrightarrow&\hoac{&m=\sqrt{2}\\&m=-\sqrt{2} \text{ ( loại )}.}
\end{eqnarray*}
Vậy $m=\sqrt{2}$ thỏa mãn.
\itemch Sai. Ta có
\begin{eqnarray*}
&&\displaystyle\int\limits_0^1x(x^2+3-a\sqrt{x})\mathrm{\; d}x=0\\
&\Leftrightarrow&\displaystyle\int\limits_0^1\left(x^3+3x-ax^{\tfrac{3}{2}}\right)\mathrm{\; d}x=0\\
&\Leftrightarrow&\left(\dfrac{1}{4}x^4+\dfrac{3}{2}x^2-\dfrac{2a}{5}x^{\tfrac{5}{2}}\right)\bigg|_0^1=0\\
&\Leftrightarrow&\dfrac{7}{4}-\dfrac{2a}{5}=0\\
&\Leftrightarrow&a=\dfrac{35}{8}.
\end{eqnarray*}
\end{itemchoice}
}
\end{ex}
\begin{ex}%[2H5H1-4]
	Trong không gian $Oxyz$, cho mặt phẳng $(P)$ đi qua $A(3;-2;5)$ và có vectơ pháp tuyến $\vec{n}=(4;-3;2)$ và mặt phẳng $(R)\colon x+2y-z+6=0$.
	\choiceTF
	{Phương trình mặt phẳng $(P)$ là $3x-2y+5z-28=0$}
	{\True $(P)$ vuông góc với mặt phẳng $(R)$}
	{Mặt phẳng $(P)$ cắt mặt phẳng $(R)$ theo giao tuyến là đường thẳng $d\colon \dfrac{x-3}{4}=\dfrac{y+2}{-3}=\dfrac{z-5}{2}$}
	{\True Mặt cầu tâm $A(3;-2;5)$ và bán kính $R=2$ cắt mặt phẳng $(R)$ theo giao tuyến là đường tròn có bán kính bằng $2$}
	\loigiai{
	\begin{itemchoice}
	\itemch Sai. Phương trình mặt phẳng $(P)$ đi qua $A(3;-2;5)$ có một vectơ pháp tuyến $\vec{n}_{(P)}=(4;-3;2)$ có dạng $4(x-3)-3(y+2)+2(z-5)=0$, hay $(P)\colon 4x-3y+2z-28=0$.
	\itemch Đúng. Vì với $\vec{n}_{(P)}=(4;-3;2)$ và $\vec{n}_{(R)}=(1;2;1)$ lần lượt là vectơ pháp tuyến của $(P)$ và $(R)$, ta thấy $\vec{n}\cdot \vec{n}_2=0$ nên $(P)\perp (R)$.
	
	\end{itemchoice}
	}
	\end{ex}
\Closesolutionfile{ans}

\TNSA
\Opensolutionfile{ans}[ans/ansDe4-TN3]
\begin{ex}%[2D4H1-4]
Hàm số $f(x)$ có đạo hàm liên tục trên $\mathbb{R}$ và $f'(x)=2^x+3^x$, $\forall x$; $f(0)=\dfrac{1}{\ln3}$. Tính giá trị của $f(1)$. (Làm tròn đến số thập phân thứ hai)
\shortans{$4{,}17$}
\loigiai{
Hàm số $f(x)=\displaystyle \int (2^x+3^x)\mathrm{\,d}x= \displaystyle \int 2^x \mathrm{\,d}x+\displaystyle \int 3^x \mathrm{\,d}x=\dfrac{2^x}{\ln2}+\dfrac{3^x}{\ln3}+C$.\\
$f(x)=\dfrac{2^x}{\ln2}+\dfrac{3^x}{\ln3}+C$.\\
Do $f(0)=\dfrac{1}{\ln3} \Leftrightarrow \dfrac{1}{\ln3}=\dfrac{1}{\ln2}+\dfrac{1}{\ln3}+C\Leftrightarrow C=-\dfrac{1}{\ln2}$\\
Suy ra $f(x)=\dfrac{2^x}{\ln2}+\dfrac{3^x}{\ln3}-\dfrac{1}{\ln2}$.\\
Vậy $f(1)=4{,}17$.
}
\end{ex}

\begin{ex}%[2D4V2-6]
Một vật chuyển động trong $6$ giờ với vận tốc $v$ (km/h) phụ thuộc vào thời gian $t$ (h) có đồ thị như hình bên dưới. Trong khoảng thời gian $2$ giờ từ khi bắt đầu chuyển động, đồ thị là một phần đường Parabol có đỉnh $I\left(3;9\right)$ và có trục đối xứng song song với trục tung. Khoảng thời gian còn lại, đồ thị vận tốc là một đường thẳng có hệ số góc bằng $\dfrac{1}{4}$. Tính quảng đường $s$ mà vật di chuyển được trong $6$ giờ? (đơn vị tính bằng km, làm tròn đến chữ số thập phân thứ nhất).
\begin{center}
\begin{tikzpicture}[>=stealth,scale=0.5]
% Vẽ 2 trục, điền các số lên trục
\draw[->] (-0.5,0)--(0,0) node[below left]{$O$}--(7,0) node[above]{$t$}; %định dạng trục Ox
\foreach \x in {2,3,6}
\draw[shift={(\x,0)},color=black] (0pt,2pt)--(0pt,-2pt)
node[below] { $\x$};
\draw[->,color=black] (0,-0.5)--(0,10) node[right]{$v$};  %định dạng trục Oy
\foreach \y in {8,9}
\draw[shift={(0,\y)},color=black] (2pt,0pt) -- (-2pt,0pt)
node[left] {$\y$};
\clip(-1,-1) rectangle (7,10); %vùng đồ thị
%\draw[gray!50,thin,opacity=.5] (-1,-1) grid (4,10); %ô vuông
%Vẽ đồ thị
\draw[smooth,samples=100,domain=0:2,font=\footnotesize, line join=round, line cap=round, thick]
plot(\x,{(-1)*(\x)^2+6*(\x)});
\draw[smooth,domain=2:6, line join=round, line cap=round,dashed]
plot(\x,{(-1)*(\x)^2+6*(\x)});
\draw[smooth,samples=100,domain=2:6,font=\footnotesize, line join=round, line cap=round, thick]
plot(\x,{(1/4)*(\x)+15/2});
% Vẽ thêm mấy cái râu ria
\draw[dashed] (3,0)--(3,9) circle(1.5pt) node[above]{$I$}--(0,9) circle(1.5pt);
\draw[dashed] (2,0)--(2,8) circle(1.5pt) --(0,8);
\draw[dashed] (6,0)--(6,9) circle(1.5pt) --(0,9);
%Vẽ dấu chấm tròn
\fill (0cm,0cm) circle (1.5pt);
\end{tikzpicture}
\end{center}
\shortans{$43{,}3$}
\loigiai{
Vì Parabol đi qua $O\left(0;0\right)$ và có tọa độ đỉnh $I\left(3;9\right)$ nên thiết lập được phương trình Parabol là $\left(P\right) \colon y = v\left(t\right) = -t^2+6t$; $\forall t \in \left[0;2\right]$.\\
Sau $2$ giờ đầu thì hàm vận tốc có dạng là hàm bậc nhất $y = \dfrac{1}{4}t + m$, dựa trên đồ thị ta thấy đi qua điểm có tọa độ $\left(6;9\right)$ nên thế vào hàm số và tìm được $m = \dfrac{15}{2}$.\\
Nên hàm vận tốc từ giờ thứ $2$ đến giờ thứ $6$ là: $y = \dfrac{1}{4}t + \dfrac{15}{2},\forall t \in \left[2;6\right]$.\\
Quảng đường vật đi được bằng tổng đoạn đường $2$ giờ đầu và đoạn đường $4$ giờ sau.
\[S = S_1 +S_2 = \displaystyle\int\limits_0^2 \left(-t^2+6t\right) \mathrm{\,d}t + \displaystyle\int\limits_2^6 \left(\dfrac{1}{4}t+\dfrac{15}{2} \right) \mathrm{\,d}t = \dfrac{130}{3} \approx 43{,}3 \left(\text{km}\right).\]
}
\end{ex}

\begin{ex}%[Nguyễn Tuấn, dự án TLDT-2]%[2H5V2-8]
	Trong không gian $Oxyz$, hai máy bay cùng xuất phát từ hai phi trường, trên màn hình rađa của trạm điều khiển (với đơn vị trên ba trục chính theo đơn vị km), sau khi xuất phát $ t$ giờ $(t\ge 0)$, vị trí của máy bay số một được xác định bởi công thức $\heva{&x=20+2t \\ &y=20+t \\ &z=-10-t }$, vị trí máy bay số hai có tọa độ là $(30+t';20+t';-10-t')$. Hỏi nếu hai máy bay không thay đổi đường bay thì sau bao lâu thì hai máy bay có thể va chạm nhau?
	\shortans{10}
	\loigiai{
		Giả sử đường bay của máy bay số 1 là $(\Delta_1)\colon \heva{&x=20+2t \\ &y=20+t \\ &z=-10-t }$ có $\overrightarrow{u}_1=(2;1;-1)$ và đường bay của máy bay số 2 thỏa $(30+t';20+t';-10-t')\in (\Delta_2)\colon \heva{&x=30+t' \\ &y=20+t' \\ &z=-10-t' }$ có $\overrightarrow{u}_2=(1;1;-1)$.\\
		Kể từ thời điểm xuất phát, để hai may bay gần nhau nhất thì hai máy bay phải gần tọa độ giao điểm của $\Delta_1$ và $\Delta_2$.\\
	Ta có $\heva{&20+2t=30+t' \\ &20+t=20+t' \\ &-10-t=-10-t' }\Leftrightarrow \heva{&2t-t'=10 \\ &t-t'=0 \\ &t-t'=0 }\Leftrightarrow \heva{&t=10 \\ &t'=10.}$\\
	Vậy sau $10$ giờ thì hai máy bay có thể va chạm nhau.
	}
	\end{ex}

\begin{ex}%[2D6V1-3]
Trường Hạnh Phúc có $20$\% học sinh tham gia câu lạc bộ âm nhạc, trong số học sinh đó có $85$\% học sinh biết chơi đàn guitar. Ngoài ra, có $10$\% số học sinh không tham câu lạc bộ âm nhạc cũng biết chơi đàn guitar. Chọn ngẫu nhiên $1$ học sinh của trường. Giả sử học sinh đó biết chơi đàn guitar. Xác suất chọn được học sinh thuộc câu lạc bộ âm nhạc là bao nhiêu?

\shortans{0,68}
\loigiai
{
Gọi $A$ là biến cố \textquotedblleft Số học sinh thuộc câu lạc bộ âm nhạc\textquotedblright.\\
$B$ là biến cố \textquotedblleft Số học sinh không thuộc câu lạc bộ\textquotedblright.\\
Khi đó $\heva{&P(A) = 20\% = 0{,}2 &\Rightarrow& P(\overline{A}) = 1-0{,}2 = 0{,}8\\ &P(B|A) = 85\% = 0{,}85 &\Rightarrow& P(\overline{B}|A) = 1-0{,}85 = 0{,}15\\ &P(B|\overline{A}) = 10\% = 0{,}1 &\Rightarrow& P(\overline{B}|\overline{A}) = 1-0{,}1=0{,}9.}$
\begin{center}
\begin{tikzpicture}[>=stealth]
%Khung 1
\draw (-0,-1) rectangle (2.2,0);
\draw (1.1,-0.5) node{Gốc};
%Mui ten 1,2
\draw [->] (2.2,-0.5)--(3.8,1.6) node[pos=0.5,sloped,above]{$0{,}2$};
\draw [->] (2.2,-0.5)--(3.8,-2.6) node[pos=0.5,sloped,below]{\color{red}$0{,}8$};
%Khung 2.1
\draw (4.5,3.0) node{\textbf{Thuộc câu lạc bộ}};
\draw (3.8,1.1) rectangle (5.1,2.1);
\draw (8.9/2,1.6) node{$A$} ;
%Khung 2.2
\draw (3.8,-2.1) rectangle (5.1,-3.1);
\draw (8.9/2,-2.6) node{$\overline{A}$};
%Mui ten 3,4
\draw [->] (5.1,1.6)--(6.5,2.6) node[pos=0.5,sloped,above]{$0{,}85$};
\draw [->] (5.1,1.6)--(6.5,0.6) node[pos=0.5,sloped,below]{\color{red}$0{,}15$};
%Mui ten 5,6
\draw [->] (5.1,-2.6)--(6.5,-1.6) node[pos=0.5,sloped,above]{$0{,}1$};
\draw [->] (5.1,-2.6)--(6.5,-3.6) node[pos=0.5,sloped,below]{\color{red}$0{,}9$};
%Khung 3.1
\draw (6.5,2.2) rectangle (7.7,3.2);
\draw (7.1,5.4/2) node{$B$} ;
%Khung 3.2
\draw (6.7,3.7) node{\textbf{Biết chơi guitar}};
\draw (6.5,1.2) rectangle (7.7,0.2);
\draw (7.1,1.4/2) node{$\overline{B}$} ;
%Khung 3.3
\draw (6.5,-1.1) rectangle (7.7,-2.1);
\draw (7.1,-3.2/2) node{$B$} ;
%Khung 3.3
\draw (6.5,-2.9) rectangle (7.7,-3.9);
\draw (7.1,-3.4) node{$\overline{B}$} ;
%Kết quả
\draw (9.5,3.7) node{\textbf{Kết quả}};
\draw (9.5,2.7) node{$AB$};
\draw (9.5,0.7) node{$A \overline{B}$};
\draw (9.5,-1.6) node{$\overline{A}B$};
\draw (9.5,-3.4) node{$\overline{A}~\overline{B}$};
%Xác suất
\draw (12.5,3.7) node{\textbf{Xác suất}};
\draw (12.5,2.7) node{$0{,}17$};
\draw (12.5,0.7) node{$0{,}03$};
\draw (12.5,-1.6) node{$0{,}08$};
\draw (12.5,-3.4) node{$0{,}72$};
\end{tikzpicture}
\end{center}
Áp dụng công thức xác suất Bayes để tính xác chọn được học sinh thuộc câu lạc bộ âm nhạc
\[P(A|B) = \dfrac{P(A)\cdot P(B|A)}{P(A) \cdot P(B|A) + P(\overline{A}) \cdot P(B|\overline{A})} = \dfrac{0{,}2 \cdot 0{,}85}{0{,}2 \cdot 0{,}85 + 0{,}8\cdot 0{,}1} = 0{,}68.\]
}
\end{ex}
\TL
\begin{ex}%[2H5H2-3]%[Dự án EX-TF-TLN lần 3 - Nguyen Chín Em]
Trong không gian với hệ tọa độ $Oxyz$. Viết phương trình tham số của đường thẳng $d$ biết $d$ đi qua điểm $M(2; -3; 4)$, vuông góc với $d_1$ và $d_2: \dfrac{x + 1}{2} = \dfrac{y}{5} =\dfrac{z + 3}{3}$. Khi đó đường thẳng $d$ đi qua điểm $(x_0;y_0;-13)$. Tính $x_0^{2}+y_0^{2}$.
% \shortans{$125$}
\loigiai{
Ta có véc-tơ chỉ phương của $d_1$ là $\overrightarrow{u_1} = (-3; 1; 2)$ và véc-tơ chỉ phương của $d_2$ là $\overrightarrow{u_2} = (2; 5; 3)$.\\
Do $d \perp d_1$ và $d \perp d_2$ nên véc-tơ chỉ phương của $d$ là $\overrightarrow{u} =\left[\overrightarrow{u_1}, \overrightarrow{u_2}\right] = (-7; 13; -17)$.\\
Phương trình tham số của đường thẳng $d$ là $\heva{&x = 2 - 7t \\&y = -3 + 13t \\&z = 4 - 17t},(t\in\mathbb{R})$.\\
% Với $t=1$ đường thẳng $d$ đi qua điểm $(-5;10;-13)$, suy ra $x_0^{2}+y_0^{2}=125$.
}
\end{ex}

\begin{ex}%[12-PTMH-1-2025]%[Võ Thị Thùy Trang]%[2D6C2-4]
Ở một khu rừng nọ có $7$ chú lùn, trong đó có $4$ chú luôn nói thật, $3$ chú còn lại luôn tự nhận mình nói nhật nhưng xác suất để mỗi chú này nói thật là $0{,}5$. Bạn Tuyết gặp ngẫu nhiên một chú lùn. Gọi $A$ là biến cố \lq \lq Chú lùn đó luôn nói thật\rq \rq \, và $B$ là biến cố \lq \lq Chú lùn đó tự nhận mình luôn nói thật\rq \rq.
Biết rằng chú lùn mà bạn Tuyết gặp tự nhận mình là người luôn nói thật. Biết xác suất để chú lùn đó luôn nói thật có thể được biểu diễn dưới dạng $\dfrac{a}{b}$ sao cho $\dfrac{a}{b}$ là phân số tối giản. Tính $2a+b$.
% \shortans{$27$}
\loigiai{

Gọi $A$ là biến cố \lq \lq Chú lùn bạn Tuyết gặp luôn nói thật\rq \rq
\,và $B$ là biến cố \lq \lq Chú lùn đó luôn tự nhận mình nói thật\rq \rq.\\
Vì có $4$ chú lùn luôn nói thật nên xác suất để bạn Tuyết gặp chú lùn luôn nói thật là $\mathrm{P}(A)=\dfrac{4}{7}$ và gặp chú lùn tự nhận mình luôn nói thật là $\mathrm{P}(\overline{A})=\dfrac{3}{7}$.\\
Theo đề bài, ta có $\mathrm{P}(B\mid A)=1$, $\mathrm{P}(B\mid \overline{A})=0{,}5$.\\
Theo công thức xác suất toàn phần, ta có
\begin{align*}
\mathrm{P}(B) &=\mathrm{P}(B\mid A)\cdot\mathrm{P}(A)+\mathrm{P}(B\mid\overline{A})\cdot\mathrm{P}(\overline{A})\\
&=1 \cdot \dfrac{4}{7} + 0{,}5 \cdot \dfrac{3}{7}\\
&=\dfrac{11}{14}.
\end{align*}
Theo công thức Bayes, ta có
\[\mathrm{P}(A\mid B)=\dfrac{\mathrm{P}(B\mid A) \cdot \mathrm{P}(A)}{\mathrm{P}(B)}=\dfrac{1 \cdot \dfrac{4}{7}}{\dfrac{11}{14}}=\dfrac{8}{11}.\]
Vậy nếu chú lùn mà bạn Tuyết gặp tự nhận mình là người luôn nói thật, xác suất để chú lùn đó luôn nói thật là $\dfrac{8}{11}$.\\
Do đó $a=8$, $b=11$ nên $2a+b=2 \cdot 8+11=27$.
}
\end{ex}

\begin{ex}%[Mức độ ]giảng 12 New - 4in1, Đoàn Hùng]%[2H5V1-7]
	\immini
	{Trong không gian với hệ tọa độ $Oxyz$ (đơn vị trên mỗi trục toạ độ là km), một máy bay đang ở vị trí $A(3;-2{,}5; 0{,}5)$ và sẽ hạ cánh ở vị trí $B(3; 7{,}5; 0)$ trên đường băng (hình bên). Có một lớp mây được mô phỏng bởi một mặt phẳng $(\alpha)$ đi qua ba điểm $M(9;0;0)$, $N(0;-9;0)$, $P(0;0;0{,}9)$. Tính độ cao của máy bay khi máy bay xuyên qua đám mây để hạ cánh.}
	{\begin{tikzpicture}[line join = round, line cap = round,>=stealth,font=\footnotesize,scale=.5]
	\path
	(0,0) coordinate (O)
	(-9,0) coordinate (N)
	($(O)!1!40:(N)$) coordinate (M)
	(0,0.9) coordinate (P)
	($(O)!1.2!(M)$) coordinate (x)
	(-5,0.8) coordinate (A)
	(4,-1.5) coordinate (B)
	($(A)!2.1cm!(B)$) coordinate (C)
	(intersection of A--B and O--M) coordinate (B1)
	;
	\draw[line width=0.3mm,red] (A)--(C) (B1)--(B);
	\draw[line width=0.3mm,red,dashed] (C)--(B1);
	\draw[->,>=stealth,line width=0.3mm,blue] (0,0)--(x) node[right=0.2cm]{$x$};
	\draw[->,>=stealth,line width=0.3mm,blue] (-10,0)--(-9,0) (0,0) node[above right]{$O$}--(7,0) node[below]{$y$};
	\draw[line width=0.3mm,blue,dashed] (-9,0)--(0,0);
	\draw[->,>=stealth,line width=0.3mm,blue] (0,0)--(0,3) node[left]{$z$};
	\draw[line width=0.3mm,blue] (N)--(P)--(M) (N)--(M);
	\foreach \x/\gm in {N/90,P/140} \fill (\x) circle (1pt) ($(\x)+(\gm:5mm)$)node[blue]{$\x$};
	\filldraw[red] (N)node[below left,blue]{$-9$} (P)node[blue,above right]{$0{,}9$} (M)node[blue,right=0.1cm]{$9$}node[blue,left=0.1cm]{$M$} (C) circle (3pt) node[below=0.2cm,blue]{\scriptsize $C$} (A) circle (3pt)node[above left,red,blue]{$A$} (B)circle (3pt) node[below,blue]{$B$};
	\end{tikzpicture}}
	\shortans{$0{,}45$}
	\loigiai{
	Giả sử điểm $C\left(x_C;y_C;z_C\right)$ là vị trí mà máy bay xuyên qua đám mây để hạ cánh, suy ra $C\in (\alpha)$. Áp dụng phương trình mặt phẳng theo đoạn chắn, ta thấy mặt phẳng $(\alpha)$ có phương trình là
	\[\dfrac{x}{9}-\dfrac{y}{9}+\dfrac{z}{0{,}9}=1 \Leftrightarrow x-y+10z=9 \Rightarrow x_C-y_C+10z_C=9.\]
	Mặt khác, vì $\vec{AC}$, $\vec{AB}$ là hai véc-tơ cùng hướng nên tồn tại số thực $t>0$ sao cho $\vec{AC}=t\cdot \vec{AB}$.\\
	Do $\vv{AC}=\left(x_C-3;y_C+2{,}5;z_C-0{,}5\right)$; $\vv{AB}=\left(3-3;7{,}5+2{,}5;0-0{,}5\right)=\left(0;10;-0{,}5\right)$\\
	nên $\heva{&x_C-3=0t\\&y_C+2{,}5=10t\\&z_C-0{,}5=-0{,}5t} \Leftrightarrow \heva{&x_C=3\\&y_C=10t-2{,}5\\&z_C=-0{,}5t+0{,}5.}$\\
	Vì $C\in(\alpha)$ nên $3-(10 t-2{,}5)+10(-0{,}5 t+0{,}5)=9 \Leftrightarrow t=0{,}1$. Suy ra $C(3;-1{,}5;0{,}45)$.\\
	Vậy tại vị trí $C$, độ cao của máy bay là $0{,}45$ km.
	}
	\end{ex}
\Closesolutionfile{ans}

\Closesolutionfile{ansbook}

%


%%Ôn THI THPTQG
% % \def\tendethi{ĐỀ MINH HOẠ 2025}
\begin{name}
    {\tenchude}
    {ĐỀ MINH HOẠ 2025}
    {\tentruong}
    {\thoigian}
\end{name}

\Opensolutionfile{ansbook}[ans/ansbook-DE-PNL-1-MH]
\TN
\Opensolutionfile{ans}[ans/ans-DE-PNL-01-T]


%%==========Câu 1
\begin{ex}%[Đề Minh Hoạ TNTHPT 2024-2025]%[2PhatTrien-DMH-2024-2025, GV: Hoàng Trọng Tấn]%[2D4N1-4]
    Nguyên hàm của hàm số $f(x) = \mathrm{e}^x$ là:
    \choice
    {$\dfrac{\mathrm{e}^{x+1}}{x+1} + C$}
    {\True $\mathrm{e}^x + C$}
    {$\dfrac{\mathrm{e}^x}{x} + C$}
    {$x \mathrm{e}^{x-1} + C$}
    \loigiai{
        Nguyên hàm của $f(x)=\mathrm{e}^x$ là $F(x)=\mathrm{e}^x+C$.	
    }
\end{ex}

%%==========Câu 2
\begin{ex}%[Đề Minh Hoạ TNTHPT 2024-2025]%[2PhatTrien-DMH-2024-2025, GV: Hoàng Trọng Tấn]%[2D4N3-3]
    Cho hàm số $y = f(x)$ liên tục, nhận giá trị dương trên đoạn $[a;b]$. Xét hình phẳng $(H)$ giới hạn bởi đồ thị hàm số $y = f(x)$, trục hoành và hai đường thẳng $x = a$, $x = b$. Khối tròn xoay được tạo thành khi quay hình phẳng $(H)$ quanh trục $Ox$ có thể tích là:
    \choice
    {$V = \pi \displaystyle \int_a^b [f(x)] \mathrm{\,d}x$}
    {$V = \pi^2 \displaystyle \int_a^b f(x) \mathrm{\,d}x$}
    {$V = \pi^2 \displaystyle \int_a^b [f(x)]^2 \mathrm{\,d}x$}
    {\True $V = \pi \displaystyle \int_a^b [f(x)]^2 \mathrm{\,d}x$}
    \loigiai{
        Khối tròn xoay được tạo thành khi quay hình phẳng $(H)$ quanh trục $Ox$ có thể tích là $V = \pi \displaystyle \int_a^b [f(x)]^2 \mathrm{\,d}x$.
    }
\end{ex}

%%==========Câu 3
\begin{ex}%[Đề Minh Hoạ TNTHPT 2024-2025]%[2PhatTrien-DMH-2024-2025, GV: Hoàng Trọng Tấn]%[2D3H2-1]
    Hai mẫu số liệu ghép nhóm $M_1$, $M_2$ có bảng tần số ghép nhóm như sau:
    \begin{center}
        \begin{tabular}{|c|c|c|c|c|c|}
            \hline
            Nhóm & $[8;10)$ & $[10;12)$ & $[12;14)$ & $[14;16)$ & $[16;18)$ \\
            \hline
            $M_1$ & $3$ & $4$ & $8$ & $6$ & $4$ \\
            \hline
            $M_2$ & $6$ & $8$ & $16$ & $12$ & $8$ \\
            \hline
        \end{tabular}
    \end{center}	
    Gọi $s_1$, $s_2$ lần lượt là độ lệch chuẩn của mẫu số liệu ghép nhóm $M_1$, $M_2$. Phát biểu nào sau đây là đúng?
    \choice
    {\True $s_1 = s_2$}
    {$s_1 = 2s_2$}
    {$2s_1 = s_2$}
    {$4s_1 = s_2$}
    \loigiai{
        Giá trị trung bình của mẫu $M_1$:
        $$
        \overline{x}_1 = \dfrac{9 \cdot 3 + 11 \cdot 4 + 13 \cdot 8 + 15 \cdot 6 + 17 \cdot 4}{3 + 4 + 8 + 6 + 4}
        = \dfrac{333}{25} = 13{,}32
        $$	
        Tính các sai lệch bình phương:
        $$
        (9 - 13{,}32)^2 = 18{,}6624,  (11 - 13{,}32)^2 = 5{,}3824, (13 - 13{,}32)^2 = 0{,}1024, 
        $$
        $$
        (15 - 13{,}32)^2 = 2{,}8224, (17 - 13{,}32)^2 = 13{,}5424
        $$
        Phương sai của mẫu $M_1$:
        $$
        s_1^2 = \dfrac{3\cdot 18{,}6624 + 4\cdot 5{,}3824 + 8\cdot 0{,}1024 + 6\cdot 2{,}8224 + 4\cdot 13{,}5424}{25}=5{,}9776
        $$
        Vậy, độ lệch chuẩn của mẫu $M_1$ là:
        $$
        s_1 = \sqrt{5{,}9776} \approx 2{,}445
        $$
        Tương tự 
        \\
        Giá trị trung bình của mẫu $M_2$:
        $$
        \overline{x}_2 = \dfrac{9 \cdot 6 + 11 \cdot 8 + 13 \cdot 16 + 15 \cdot 12 + 17 \cdot 8}{6 + 8 + 16 + 12 + 8}
        = \dfrac{666}{50} = 13{,}32
        $$	
        Tính các sai lệch bình phương:
        $$
        (9 - 13{,}32)^2 = 18{,}6624, \quad (11 - 13{,}32)^2 = 5{,}3824, \quad (13 - 13{,}32)^2 = 0{,}1024,
        $$
        $$
        (15 - 13{,}32)^2 = 2{,}8224, \quad (17 - 13{,}32)^2 = 13{,}5424
        $$
        Phương sai của mẫu $M_2$:
        $$
        s_2^2 = \dfrac{6\cdot 18{,}6624 + 8\cdot 5{,}3824 + 16\cdot 0{,}1024 + 12\cdot 2{,}8224 + 8\cdot 13{,}5424}{50}= 5{,}9776
        $$
        
        Vậy, độ lệch chuẩn của mẫu $M_2$ là:
        $$
        s_2 = \sqrt{5{,}9776} \approx 2{,}445
        $$
        \\
        Cả hai mẫu $M_1$ và $M_2$ đều có độ lệch chuẩn bằng nhau:
        $$
        s_1 = s_2 \approx 2{,}445
        $$
    }
\end{ex}

%%==========Câu 4
\begin{ex}%[Đề Minh Hoạ TNTHPT 2024-2025]%[2PhatTrien-DMH-2024-2025, GV: Hoàng Trọng Tấn]%[2H5N2-3]
    Trong không gian với hệ trục tọa độ $Oxyz$, phương trình của đường thẳng đi qua điểm $M(1;-3;5)$ và có một vectơ chỉ phương $\overrightarrow{u}(2;-1;1)$ là:
    \choice
    {$\dfrac{x-1}{2} = \dfrac{y+3}{-1} = \dfrac{z-5}{1}$}
    {$\dfrac{x+1}{2} = \dfrac{y+3}{-1} = \dfrac{z+5}{1}$}
    {\True $\dfrac{x-1}{2} = \dfrac{y+3}{1} = \dfrac{z-5}{-1}$}
    {$\dfrac{x+1}{2} = \dfrac{y+3}{-1} = \dfrac{z-5}{-1}$}
    \loigiai{
        Phương trình của đường thẳng đi qua điểm $M(1;-3;5)$ và có một vectơ chỉ phương $\overrightarrow{u}(2;-1;1)$ là $\dfrac{x-1}{2} = \dfrac{y+3}{1} = \dfrac{z-5}{-1}$.
    }
\end{ex}

%%==========Câu 5
\begin{ex}%[Đề Minh Hoạ TNTHPT 2024-2025]%[2PhatTrien-DMH-2024-2025, GV: Hoàng Trọng Tấn]%[2D1N4-1]
    %[id6]
    \immini{
        Cho hàm số $y = \dfrac{ax + b}{cx + d}$ ($c \neq 0$, $ad - bc \neq 0$) có đồ thị như hình vẽ bên. Tiệm cận ngang của đồ thị hàm số là:
        \choice
        {$x = -1$}
        {\True $y = \dfrac{1}{2}$}
        {$y = -1$}
        {$x = \dfrac{1}{2}$}
    }
    {
        \begin{tikzpicture}[line join = round, line cap=round,>=stealth,font=\footnotesize,scale=0.7]
            \clip (-5,-3) rectangle (3,3);
            \draw[smooth,samples=100,domain=-0.7:4] plot(\x,{
                ((\x)-1)/(2*(\x)+2)
            });
            \draw[smooth,samples=100,domain=-5:-1.1] plot(\x,{
                ((\x)-1)/(2*(\x)+2)
            });
            \draw[->] (0,-3)--(0,3) node[below right]{$y$};
            \draw[->] (-5,0)--(3,0) node[below left]{$x$};
            \draw[dashed] 
            (-1,-3)--(-1,3)(-5,0.5)--(5,0.5)
            ;
            \fill 
            (0,0) circle(1.5pt) node[below left]{$O$}
            (0,0.5) circle(1.5pt) node[above right]{$\frac{1}{2}$}
            (-1,0) circle(1.5pt) node[below left]{$-1$}
            (1,0) circle(1.5pt) node[below right]{$1$}
            (0,-0.5) circle(1.5pt) node[right,yshift=-5pt]{$-\frac{1}{2}$}
            ;
        \end{tikzpicture}
    }
    \loigiai{ 
        Dựa vào hình vẽ bên ta thấy đường tiệm cận ngang của đồ thị hàm là $y=\dfrac{1}{2}$.
    }
\end{ex}


%%==========Câu 6
\begin{ex}%[Đề Minh Hoạ TNTHPT 2024-2025]%[2PhatTrien-DMH-2024-2025, GV: Hoàng Trọng Tấn]%[1D6N4-3]
    Tập nghiệm của bất phương trình $\log_2 (x-1) < 3$ là:
    \choice
    {\True $(1;9)$}
    {$(-\infty;9)$}
    {$(9; +\infty)$}
    {$(1;7)$}
    \loigiai{
        Ta có $\log_2 (x-1) < 3\Leftrightarrow \log_2(x-1)< \log_2 8\Leftrightarrow 0<x-1<8\Leftrightarrow 1<x<9$.
    }
\end{ex}

%%==========Câu 7
\begin{ex}%[Đề Minh Hoạ TNTHPT 2024-2025]%[2PhatTrien-DMH-2024-2025, GV: Hoàng Trọng Tấn]%[2H5N1-2]
    Trong không gian với hệ trục tọa độ $Oxyz$, cho mặt phẳng $(P)$ có phương trình $x - 3y - z + 8 = 0$. Véctơ nào sau đây là một vectơ pháp tuyến của mặt phẳng $(P)$?
    \choice
    {$\overrightarrow{n}_1(1;-3;1)$}
    {\True $\overrightarrow{n}_2(1;-3;-1)$}
    {$\overrightarrow{n}_3(1;3;8)$}
    {$\overrightarrow{n}_4(1;3;8)$}
    \loigiai{
        Véctơ $\overrightarrow{n}_2(1;-3;-1)$ là một vectơ pháp tuyến của mặt phẳng $(P)$.
    }
\end{ex}

%%==========Câu 8
\begin{ex}%[Đề Minh Hoạ TNTHPT 2024-2025]%[2PhatTrien-DMH-2024-2025, GV: Hoàng Trọng Tấn]%[1H8N4-2]
    Cho hình chóp $S.ABCD$ có đáy $ABCD$ là hình chữ nhật và $SA \perp (ABCD)$. Mặt phẳng nào sau đây vuông góc với mặt phẳng $(ABCD)$?
    \choice
    {\True $(SAB)$}
    {$(SBC)$}
    {$(SCD)$}
    {$(SBD)$}
    \loigiai{
        Ta có $SA\subset (SAB)$ và $SA \perp (ABCD)\Rightarrow (SAB)\perp (ABCD)$.
    }
\end{ex}

%%==========Câu 9
\begin{ex}%[Đề Minh Hoạ TNTHPT 2024-2025]%[2PhatTrien-DMH-2024-2025, GV: Hoàng Trọng Tấn]%[1D6N4-2]
    Nghiệm của phương trình $2^x = 6$ là:
    \choice
    {$x = \log_2 2$}
    {$x = 3$}
    {$x = 4$}
    {\True $x = \log_2 6$}
    \loigiai{
        Ta có $2^x = 6\Leftrightarrow x=\log_2 6$.	
    }
\end{ex}

%%==========Câu 10
\begin{ex}%[Đề Minh Hoạ TNTHPT 2024-2025]%[2PhatTrien-DMH-2024-2025, GV: Hoàng Trọng Tấn]%[1D2N1-3]
    Cấp số cộng $(u_n)$ có $u_1 = 1$ và $u_2 = 3$. Số hạng $u_5$ của cấp số cộng là:
    \choice
    {$5$}
    {$7$}
    {\True $9$}
    {$11$}
    \loigiai{
        Ta có công sai $d=u_2-u_1=2$.
        \\
        Suy ra $u_5=u_1+4d\Leftrightarrow u_5=1+4\cdot 2=9$.	
    }
\end{ex}

%%==========Câu 11
\begin{ex}%[Đề Minh Hoạ TNTHPT 2024-2025]%[2PhatTrien-DMH-2024-2025, GV: Hoàng Trọng Tấn]%[2H2H1-2]
    
    \immini{
        Cho hình hộp $ABCDA'B'C'D'$ (minh họa như hình bên). Phát biểu nào sau đây là đúng?
        \choice
        {$\overrightarrow{AB} + \overrightarrow{BB'} + \overrightarrow{B'A'} = \overrightarrow{AC'}$}
        {$\overrightarrow{AB} + \overrightarrow{BC} + \overrightarrow{C'D'} = \overrightarrow{AC'}$}
        {$\overrightarrow{AB} + \overrightarrow{AC} + \overrightarrow{AA'} = \overrightarrow{AC'}$}
        {\True $\overrightarrow{AB} + \overrightarrow{AA'} + \overrightarrow{AD} = \overrightarrow{AC'}$}
    }
    {
        \begin{tikzpicture}[line join = round, line cap=round,>=stealth,font=\footnotesize,scale=0.7]
            \def\r{3}
            \path 
            (0,0) coordinate (A)
            (4,0) coordinate (D)
            (-2,-2) coordinate (B)
            ($(B)+(D)-(A)$) coordinate (C)
            ($(A)+(1,\r)$) coordinate (A')
            ($(B)+(1,\r)$) coordinate (B')
            ($(C)+(1,\r)$) coordinate (C')
            ($(D)+(1,\r)$) coordinate (D')
            ;
            
            \draw (B)--(C)--(D)--(D')--(A')--(B')--(B)
            (B')--(C')--(D') (C)--(C')
            ;
            \draw[dashed]
            (B)--(A)--(D) (A)--(A') (A)--(C');
            \foreach \p/\r in {A/-90,B/-90,C/-90,D/-90,A'/90,B'/90,C'/90,D'/90}
            \fill (\p) circle (1.2pt) node[shift={(\r:3mm)}]{$\p$};
        \end{tikzpicture}
    }
    \loigiai{
        Ta có $\overrightarrow{AB}+\overrightarrow{AD}=\overrightarrow{AC}$.
        \\
        Suy ra $\overrightarrow{AB} + \overrightarrow{AA'} + \overrightarrow{AD} = \overrightarrow{AC}+\overrightarrow{AA'}=\overrightarrow{AC'}$.
    }
\end{ex}

%%==========Câu 12
\begin{ex}%[Đề Minh Hoạ TNTHPT 2024-2025]%[2PhatTrien-DMH-2024-2025, GV: Hoàng Trọng Tấn]%[2D1N1-2]
    \immini{
        Cho hàm số có đồ thị như hình vẽ bên. Hàm số đã cho đồng biến trên khoảng nào sau đây?
        \choice
        {$(-\infty;1)$}
        {$(-\infty;-1)$}
        {\True $(-1;1)$}
        {$(1;+ \infty)$}	
    }
    {
        \begin{tikzpicture}[line join = round, line cap=round,>=stealth,font=\footnotesize,scale=0.7]
            \clip (-3,-2.6) rectangle (3,2.6);
            \draw[smooth,samples=100,domain=-3:3] plot(\x,{
                -(\x)^3+3*(\x)
            });
            \draw[->] (-2,0)--(2.7,0) node[above]{$x$};
            \draw[->] (0,-3)--(0,2.5) node[right]{$y$};
            \fill(0,0) circle(1.5pt) node[below left]{$O$};
            \fill 
            (1,0) circle(1.5pt) node[below]{$1$}
            (-1,0) circle(1.5pt) node[above]{$-1$}
            (0,2) circle(1.5pt) node[left]{$2$}
            (0,-2) circle(1.5pt) node[right]{$-2$}
            ;
            \draw[dashed] 
            (-1,0)--(-1,-2)--(0,-2)
            (1,0)--(1,2)--(0,2)
            ;
        \end{tikzpicture}
    }
    \loigiai{
        Quan sát hình vẽ từ trái sang phải ta thấy hình vẽ hướng lên, trên khoảng $(-1;1)$.
        \\
        Suy ra hàm số đã cho đồng biến trên $(-1;1)$.
    }
\end{ex}

\Closesolutionfile{ans}
\TNTF
\Opensolutionfile{ans}[ans/ans-DE-PNL-01-TF]




%%==========Câu 13
\begin{ex}%[Đề Minh Hoạ TNTHPT 2024-2025]%[2PhatTrien-DMH-2024-2025, GV: Hoàng Trọng Tấn]%[2D1H3-1]
    Cho hàm số $f(x) = 2\cos x + x$
    \choiceTF
    {\True $f(0) = 2$; $f\left(\dfrac{\pi}{2}\right) = \dfrac{\pi}{2}$}
    {$f'(x) = 2\sin x + 1$}
    {\True $f'(x) = 0$ có nghiệm trên đoạn $\left[0; \dfrac{\pi}{2}\right]$ là $\dfrac{\pi}{6}$}
    {\True Giá trị lớn nhất của $f(x)$ trên đoạn $\left[0; \dfrac{\pi}{2}\right]$ là $\sqrt{3} + \dfrac{\pi}{6}$}
    \loigiai{
        
        \begin{itemchoice}	
            \itemch \textbf{Đúng.}\\	Ta có $f(0)=2\cos 0 +0 =2$ và $f\left(\dfrac{\pi}{2}\right)=2\cdot 0+\dfrac{\pi}{2}$.
            \itemch \textbf{Sai.} \\	$f'(x)=-2\sin x +1$.
            \itemch  \textbf{Đúng.} \\	$f'(x)=0\Leftrightarrow -2\sin x +1 =0 \Leftrightarrow \sin x=\dfrac{1}{2}$.
            \\
            Mà $x\in \left[0;\dfrac{\pi}{2}\right]\Rightarrow x=\dfrac{\pi}{6}$.
            \itemch \textbf{Đúng.} \\Ta có $f\left(\dfrac{\pi}{6}\right)=\sqrt{3}+\dfrac{\pi}{6}>2$.
            Suy ra $\max\limits_{[0;\tfrac{\pi}{2}]}=\sqrt{3}+\dfrac{\pi}{6}$.
        \end{itemchoice}
    }
\end{ex}



%%==========Câu 14
\begin{ex}%[Đề Minh Hoạ TNTHPT 2024-2025]%[2PhatTrien-DMH-2024-2025, GV: Hoàng Trọng Tấn]%[2D4V2-6]
    Một người điều khiển ô tô đang ở đường dẫn muốn nhập làn vào đường cao tốc. Khi ô tô cách điểm nhập làn $200$ m, tốc độ của ô tô là $36$ km/h. Hai giây sau đó, ô tô bắt đầu tăng tốc với tốc độ $v(t) = at + b$ ($a$, $b \in \mathbb{R}$, $a > 0$), trong đó $t$ là thời gian tính bằng giây kể từ khi bắt đầu tăng tốc. Biết rằng ô tô nhập làn cao tốc sau $12$ giây và duy trì sự tăng tốc trong $24$ giây kể từ khi bắt đầu tăng tốc.
    \choiceTF
    {\True Quãng đường ô tô đi được từ khi bắt đầu tăng tốc đến khi nhập làn là $180$ m}
    {\True Giá trị của $b$ là $10$}
    {Quãng đường $S(t)$ (đơn vị: mét) mà ô tô đi được trong thời gian $t$ giây ($0 \leq t \leq 24$) kể từ khi tăng tốc được tính theo công thức $S(t) =\displaystyle \int_0^{24} v(t) \mathrm{\,d}t$}
    {Sau $24$ giây kể từ khi tăng tốc, tốc độ của ô tô không vượt quá tốc độ tối đa cho phép là $100$ km/h}
    \loigiai{
        \begin{center}
            \begin{tikzpicture}[line join = round, line cap=round,>=stealth,font=\footnotesize,scale=1]
                \draw 
                (0,0)node[above]{$A$} node[yshift=25pt,pos=0.5]{$v_0=36$ km/h}
                -- (3,0)node[above]{$B$} node[yshift=-10pt,pos=0.5]{$2$ giây} 
                node[xshift=20pt,yshift=25pt]{$v(t)=at+b$ (m/s)}--(8,0) node[above]{$C$}--(14,0) node[above]{$D$}
                ;
                \draw[<->] (3,-0.75)--(8,-0.75) node[below,pos=0.5]{$12$ giây}	;
                \draw[<->] (3,-1.5)--(14,-1.5) node[below,pos=0.5]{$24$ giây}	;
                %\draw[<->] (0,-2.5)--(3,-2.5) node[below,pos=0.5]{$S_1$ m}	;
                \draw[<->] (0,-2.5)--(8,-2.5) node[below,pos=0.5]{$200$ m}	;
            \end{tikzpicture}
        \end{center}
        
        \begin{itemchoice}
            \itemch \textbf{Đúng.}	\\ 
            Đổi đơn vị $36$ km/h bằng $10$ m/s.
            \\
            Quãng đường ô tô đi được $2$ giây đầu là $AB=10\cdot 2=20$ m. 
            \\
            Quãng đường ô tô đi được từ khi bắt đầu tăng tốc đến khi nhập làn là $BC=AC-AB=200-20=180$ m.
            
            \itemch \textbf{Đúng.}\\
            Khi $t=0$ thì $v_t=v_o=10\Rightarrow 10=a\cdot 0 +b\Rightarrow b=10$.
            
            \itemch\textbf{Sai.}\\
            Quãng đường đi được trong $t$ giây là $S(t)-S(0)=\displaystyle \int_0^t v(t) \mathrm{\,d}t$.
            
            \itemch \textbf{Sai.}
            \\
            Ta có $v(t)=at+10$.	Lại có 
            \begin{eqnarray*}
                BC=\displaystyle\int_0^{12} (at+10) \mathrm{\,d}t
                &\Leftrightarrow&	\left( \dfrac{at^2}{2}+10t\right)\Bigg|_0^{12}=180
                \\
                &\Leftrightarrow& 72a+120=180
                \\
                &\Leftrightarrow& a=\dfrac{5}{6}
            \end{eqnarray*}
            Vậy $v(t)=\dfrac{5}{6}t +10 \Rightarrow v(24)=30$ m/s $=108$ km/h. 
        \end{itemchoice}
    }
\end{ex}



%%==========Câu 15
\begin{ex}%[Đề Minh Hoạ TNTHPT 2024-2025]%[2PhatTrien-DMH-2024-2025, GV: Hoàng Trọng Tấn]%[2D6V2-3]
    Trước khi đưa một loại sản phẩm ra thị trường, người ta đã phỏng vấn ngẫu nhiên $200$ khách hàng về sản phẩm đó. Kết quả thống kê như sau: có $105$ người trả lời 
    \lq\lq  sẽ mua\rq\rq; có $95$ người trả lời \lq\lq  không mua\rq\rq.
    Kinh nghiệm cho thấy tỉ lệ khách hàng thực sự sẽ mua sản phẩm tương ứng với những câu trả lời \lq\lq  sẽ mua\rq\rq\, và \lq\lq  không mua\rq\rq\, lần lượt là $70\%$ và $30\%$. Gọi $A$ là biến cố \lq\lq  Người được phỏng vấn thực sự sẽ mua sản phẩm\rq\rq. Gọi $B$ là biến cố \lq\lq  Người được phỏng vấn trả lời sẽ mua sản phẩm\rq\rq.
    \choiceTF
    {\True $\mathrm{P}(B) = \dfrac{21}{40}$ và $\mathrm{P}(\overline{B}) = \dfrac{19}{40}$}
    {$\mathrm{P}(A \mid B) = 0{,}3$}
    {\True $\mathrm{P}(A) = 0{,}51$}
    {Trong số những người được phỏng vấn thực sự sẽ mua sản phẩm có $70\%$ người đã trả lời \lq\lq  sẽ mua\rq\rq\, khi được phỏng vấn (kết quả tính theo phần trăm được làm tròn đến hàng đơn vị)}
    \loigiai{			
        \begin{itemchoice}
            \itemch \textbf{Đúng}.\\
            Xác suất người được phỏng vấn thực sự sẽ mua sản phẩm $B$ và không mua sản phẩm $\overline{B} $ được cho như sau:
            $\mathrm{P}(B) = \dfrac{105}{200}=\dfrac{21}{40}$, $\mathrm{P}(\overline{B}) =1-\dfrac{21}{40}=\dfrac{19}{40}$.
            
            \itemch \textbf{Sai.}\\
            $\mathrm{P}(A\mid B)$ là xác suất có điều kiện, mô tả xác suất một người thực sự sẽ mua sản phẩm (biến cố $A$) khi biết rằng người đó đã trả lời \lq\lq  sẽ mua\rq\rq\, trong cuộc phỏng vấn (biến cố $B$).
            \\
            Vậy $\mathrm{P}(A\mid B)=70\%=0{,}7$.
            
            \itemch \textbf{Đúng.}\\
            Từ $\mathrm{P}(A\mid B)=0{,}7\Rightarrow P(A\mid \overline{B})=1-0{,}7=0{,}3$.
            \\
            Áp dụng công thức xác suất toàn phần ta có 
            $$
            \mathrm{P}(A) = \mathrm{P}(A \mid B) \cdot \mathrm{P}(B) + \mathrm{P}(A \mid \overline{B}) \cdot \mathrm{P}(\overline{B})=0{,}7 \cdot \dfrac{21}{40} + 0{,}3 \cdot \dfrac{19}{40}=0{,}51.
            $$
            
            \itemch \textbf{Sai.}\\
            Phát biểu này yêu cầu tính xác suất có điều kiện $\mathrm{P}(B\mid A)$, tức là xác suất một người đã trả lời \lq\lq  sẽ mua\rq\rq\, khi biết rằng người đó thực sự sẽ mua sản phẩm.
            \\
            Để tính $\mathrm{P}(B \mid A)$, ta áp dụng định lý Bayes:
            \begin{eqnarray*}
                && \mathrm{P}(B \mid A) = \dfrac{\mathrm{P}(A \mid B) \cdot \mathrm{P}(B)}{\mathrm{P}(A)}
                \\
                &\Leftrightarrow& \mathrm{P}(B \mid A) = \dfrac{0{,}7 \cdot \dfrac{21}{40}}{0{,}51}\approx 0{,}72=72\%.
            \end{eqnarray*}
            
        \end{itemchoice}
        
    }
\end{ex}

%%==========Câu 16
\begin{ex}%[Đề Minh Hoạ TNTHPT 2024-2025]%[2PhatTrien-DMH-2024-2025, GV: Hoàng Trọng Tấn]%[id6]
    \immini{
        Các thiên thạch có đường kính lớn hơn $140$ m và có thể lại gần Trái Đất ở khoảng cách nhỏ hơn $7\,500\,000$ km được coi là những vật thể có khả năng va chạm gây nguy hiểm cho Trái Đất. Để theo dõi những thiên thạch này, người ta đã thiết lập các trạm quan sát các vật thể bay gần Trái Đất. Giả sử có một hệ thống quan sát có khả năng theo dõi các vật thể ở độ cao không vượt quá $6\,600$ km so với mực nước biển. Coi Trái Đất là khối cầu có bán kính $6\,400$ km. Chọn hệ trục tọa độ $Oxyz$ trong không gian có gốc $O$ tại tâm Trái Đất và đơn vị độ dài trên mỗi trục tọa độ là $1\,000$ km. Một thiên thạch (coi như một hạt) chuyển động với tốc độ không đổi theo một đường thẳng từ điểm $M(6;20;0)$ đến điểm $N(-6;-12;16)$.
        
    }
    {
        \begin{tikzpicture}[line join = round, line cap=round,>=stealth,font=\footnotesize,transform shape,scale=0.8]
            \draw 
            (0,0) circle(3.5cm)
            ;
            \draw[fill=black!30] 
            (0,0) circle(1.6cm);
            \draw[<->] (0:3.5)--(0:1.6) node[pos=0.5,above]{$6\, 600$ km};
            \draw[<->] (0:0)--(0:1.6) node[pos=0.5,above]{$6\, 400$ km};
            \draw (130:3.5)coordinate (A)--(40:3.5)coordinate (B);
            \draw ($(A)!-0.4!(B)$) --(A)
            ($(A)!1.4!(B)$)--(B)
            ;
            
            \fill 
            (A)circle(1.5pt)node[above left]{$A$}
            (B)circle(1.5pt)node[above right]{$B$}
            ($(A)!-0.4!(B)$) circle(1.5pt)node[above]{$M$}
            ($(A)!1.4!(B)$)circle(1.5pt)node[above]{$N$}
            ;
            
        \end{tikzpicture}
    }
    \choiceTF
    {\True Đường thẳng $MN$ có phương trình tham số là $\heva{&x = 6 + 3t \\& y = 20 + 8t \\& z = -4t}$, $(t \in \mathbb{R})$}
    {Vị trí đầu tiên thiên thạch đi chuyển vào phạm vi theo dõi của hệ thống quan sát là điểm $A(-3; -4; 12)$}
    {\True Khoảng cách giữa vị trí đầu tiên và vị trí cuối cùng mà thiên thạch di chuyển trong phạm vi theo dõi của hệ thống quan sát  là $18\,900$ km (kết quả làm tròn đến hàng trăm đơn vị ki-lô-mét)}
    {\True Nếu thời gian di chuyển của thiên thạch trong phạm vi theo dõi của hệ thống quan sát là $3$ phút thì thời gian nó di chuyển từ $M$ đến $N$ là $6$ phút}
    \loigiai{
        \begin{itemchoice}
            \itemch \textbf{Đúng}.\\
            Ta có $\overrightarrow{MN}=(-12;-32;16)=-4\cdot (3;8;-4)$.
            \\
            Suy ra $MN$ có một véctơ chỉ phương là $\overrightarrow{u}=(3;8;-4)$.
            \\
            Phương trình tham số của đường thẳng $MN$ là $\heva{&x = 6 + 3t \\& y = 20 + 8t \\& z = -4t}$, $(t \in \mathbb{R})$.
            
            \itemch \textbf{Sai.}\\
            \begin{center}
                \begin{tikzpicture}[line join = round, line cap=round,>=stealth,font=\footnotesize,transform shape,scale=0.7]
                    \draw 
                    (0,0)coordinate (O) circle(3.5cm)
                    ;
                    \draw[fill=black!30] 
                    (0,0) circle(1.6cm);
                    \draw[<->] (0:3.5)--(0:1.6) node[pos=0.5,above]{$6\, 600$ km};
                    \draw[<->] (0:0)--(0:1.6) node[pos=0.5,above]{$6\, 400$ km};
                    \draw (130:3.5)coordinate (A)--(40:3.5)coordinate (B);
                    \path (0,0) node{\hypersetup{hidelinks}\href{HWKJsE1}{ }};
                    \draw ($(A)!-0.4!(B)$) --(A)
                    ($(A)!1.4!(B)$)--(B)
                    ;
                    
                    \fill 
                    (A)circle(1.5pt)node[above left]{$A$}
                    (B)circle(1.5pt)node[above right]{$B$}
                    ($(A)!-0.4!(B)$)coordinate (M) circle(1.5pt) circle(1.5pt)node[above]{$M$}
                    ($(A)!1.4!(B)$) node[above]{$N$}
                    ;
                    \draw 
                    (O)--($(A)!1/2!(B)$)coordinate(H) 
                    (O)--(A)
                    (O)--(M)
                    ; 
                    
                    
                    \foreach \p/\r in {O/-90,H/90}
                    \fill (\p) circle (1.2pt) node[shift={(\r:3mm)}]{$\p$};			
                    
                    \draw pic[draw,angle radius=3mm] {right angle = O--H--A}; 
                    
                \end{tikzpicture}
            \end{center}
            Do $H\in MN\Rightarrow H(6+3t;20+8t;-4t)$ suy ra $\overrightarrow{OH}=(6+3t;20+8t;-4t)$ 
            \\
            Hơn nữa
            \begin{eqnarray*}
                OH\perp MN &\Leftrightarrow& \overrightarrow{OH}\cdot \overrightarrow{u}=0
                \\
                &\Leftrightarrow& 3(6 + 3t) + 8(20 + 8t) + (-4)\cdot (-4t)=0
                \\
                &\Leftrightarrow&178 + 89t=0 \Leftrightarrow t=-2.
            \end{eqnarray*}
            Suy ra $H(0;4;8)$ và độ dài của $OH=\sqrt{4^2+8^2}=4\sqrt 5$ (nghìn km).
            \\
            Xét tam giác $OMH$
            \\ 
            $OM=\sqrt{6^2+20^2+0}=2\sqrt{109}$ (nghìn km).
            \\
            Suy ra $MH=\sqrt{OM^2-OH^2}=\sqrt{436-80}=2\sqrt{89}$ (nghìn km).
            \\
            Xét tam giác $OAH$, ta có $OA=6\,400+6\,600=13$ (nghìn km).
            \\
            $AH=\sqrt{OA^2-OH^2}=\sqrt{169-80}=\sqrt{89}$ (nghìn km).
            \\
            Suy ra $MH=2AH$ hay $A$ là trung điểm $MH$.
            \\
            Vậy toạ độ $A(3;12;4)$.
            
            
            \itemch \textbf{Đúng.}
            \\
            Độ dài đoạn $AB=2AH=2\sqrt{89}\approx 18{,}87$ nghìn km tức là $AH=18\, 870$ km $\approx 18\, 900$ km.
            
            \itemch \textbf{Đúng.}
            \\
            Ta có $3$ phút bằng $\dfrac{1}{20}$ giờ và $3$ phút bằng $\dfrac{1}{10}$ giờ.
            \\
            Vận tốc của thiên thạch $v=\dfrac{AB}{\dfrac{1}{20}}=378\,000$ (km/h) $=378$ (nghìn km/h).
            \\
            Độ dài của $MN=2MH=4\sqrt{89}$ (nghìn km).
            \\
            Thời gian đi từ $M$ đến $N$ là $\dfrac{4\sqrt{89}}{378}\approx 0{,}1$ h bằng $6$ phút.
        \end{itemchoice}
    }
\end{ex}
\Closesolutionfile{ans}
\TNSA
\Opensolutionfile{ans}[ans/ans-DE-PNL-01-SA]


%%==========Câu 17
\begin{ex}%[Đề Minh Hoạ TNTHPT 2024-2025]%[2PhatTrien-DMH-2024-2025, GV: Hoàng Trọng Tấn]%[id6]
    Cho hình lăng trụ đứng $ABC.A'B'C'$ có $AB = 5$, $BC = 6$, $CA = 7$. Khoảng cách giữa hai đường thẳng $AA'$ và $BC$ bằng bao nhiêu? (làm tròn kết quả đến hàng phần mười)
    \shortans[]{$4{,}9$}
    \loigiai{
        \begin{center}
            \begin{tikzpicture}[line join = round, line cap=round,>=stealth,font=\footnotesize,scale=1]
                \def \dai{4} 
                \def\rong{-1}
                \def\cao{3}
                \path 
                (0,0) coordinate (A) 
                (\dai,0) coordinate (B)
                (\dai/3,\rong) coordinate (C)
                ($(A)+(0,\cao)$) coordinate (A')
                ($(B)+(0,\cao)$) coordinate (B')
                ($(C)+(0,\cao)$) coordinate (C')
                ($(C)!0.3!(B)$) coordinate (H)
                ;
                \draw 
                (A')--(B')--(C')--cycle
                (A')--(A) (B')--(B) (C')--(C)
                (A)--(C)--(B)
                ;
                \draw pic[draw,angle radius=3mm] {right angle = A--H--B}; 
                \draw[dashed] (A)--(B) (A)--(H);
                \foreach \p/\r in {A/-120,B/-40,C/-90,A'/120,B'/40,C'/90,H/-90}
                \fill (\p) circle (1.2pt) node[shift={(\r:3mm)}]{$\p$};
                
                
            \end{tikzpicture}
        \end{center}	
        Kẻ đường cao $AH$ của tam giác $ABC$.
        \\
        Suy ra $AH\perp BC$ hơn nữa $AA'\perp (ABC)$ mà $AH\subset (ABC)\Rightarrow AA'\perp AH$.
        \\
        Suy ra $AH$ là đoạn vuông góc chung của $AA'$ và $BC$ hay $d[AA',BC]=AH$.
        \\
        Nửa chu vi tam giác $ABC$ là $\mathrm{P}=\dfrac{5+6+7}{2}=9$.
        \\
        Áp dụng công thức Hê Rông ta có diện tích tam giác $ABC$ là
        $$
        S=\sqrt{p(p-5)(p-6)(p-7)}=6\sqrt 6.
        $$
        Suy ra độ dài đường cao $AH$ là $\dfrac{2\cdot 6\sqrt 6}{6}=2\sqrt 6\approx 4{,}9$.
    }
\end{ex}

%%==========Câu 18
\begin{ex}%[Đề Minh Hoạ TNTHPT 2024-2025]%[2PhatTrien-DMH-2024-2025, GV: Hoàng Trọng Tấn]%[id6]
    \immini{
        Một trò chơi điện tử quy định như sau: Có $4$ trụ $A$, $B$, $C$, $D$ với số lượng các thử thách trên đường đi giữa các cặp trụ được mô tả trong hình bên. Người chơi xuất phát từ một trụ nào đó, đi qua tất cả các trụ còn lại, mỗi khi đi qua một trụ thì trụ đó sẽ bị phá hủy và không thể quay trở lại trừ khi trò chơi kết thúc. Người chơi vẫn phải trở về trụ ban đầu. Tổng số thử thách của đường đi thỏa mãn điều kiện trên nhận giá trị nhỏ nhất là bao nhiêu?
    }
    {
        \begin{tikzpicture}[line join = round, line cap=round,>=stealth,font=\footnotesize,scale=1]
            \path 
            (0,0) coordinate (B)
            (0.5,3) coordinate (A)
            (4,0) coordinate (C)
            (1,1)coordinate (D)
            ;
            \draw (A)--(B)--(C)--cycle
            (A)--(D) (B)--(D) (C)--(D)
            ;
            \path 
            (A)--(B)node[left,pos=0.5]{$10$}
            (A)--(C)node[right,pos=0.5]{$11$}
            (C)--(B)node[below,pos=0.5]{$12$}
            (A)--(D)node[right,pos=0.5]{$9$}
            (B)--(D)node[below,xshift=7pt,yshift=2pt,pos=0.5]{$11$}
            (C)--(D)node[above,pos=0.5]{$14$}
            ;
            \foreach \p/\r in {A/90,B/-140,C/0,D/40}
            \fill (\p) circle (1.2pt) node[shift={(\r:3mm)}]{$\p$};
            
            
        \end{tikzpicture}
    }
    \shortans{$43$}
    \loigiai{
        Do $AD$ có $9$ thử thách là đường đi có số thử thách ít nhất nên ta sẽ bắt đầu từ $A$ đến $D$.
        \\
        Do $DB$ có $11$ thử thách và $DC$ có $14$ thử thách nên ta sẽ đi từ $D$ đến $B$.
        \\
        Lúc này trụ $A$ và $D$ đã bị phá huỷ nên ta đi từ $B$ đến $C$ với số thử thách là $12$.
        \\
        Cuối cùng đi từ $C$ đến $A$ với số thử thách là $11$.
        \\
        Vậy tổng cộng có $9+11+12+11=43$ thử thách.
    }
\end{ex}

%%==========Câu 19
\begin{ex}%[Đề Minh Hoạ TNTHPT 2024-2025]%[2PhatTrien-DMH-2024-2025, GV: Hoàng Trọng Tấn]%[id6]
    Hệ thống định vị toàn cầu GPS là một hệ thống cho phép xác định vị trí của một vật thể trong không gian. Trong cùng một thời điểm, vị trí của một điểm $M$ trong không gian sẽ được xác định bởi bốn vệ tinh cho trước nhờ các bộ thu phát tín hiệu đặt trên các vệ tinh. Giả sử trong không gian với hệ tọa độ $Oxyz$, có bốn vệ tinh lần lượt đặt tại các điểm $A(3;1;0)$, $B(3;6;6)$, $C(4;6;2)$, $D(6;2;14)$; vị trí $M(a;b;c)$ thỏa mãn $MA = 3$, $MB = 6$, $MC = 5$, $MD = 13$. Khoảng cách từ điểm $M$ đến điểm $O$ bằng bao nhiêu?
    \shortans{3}
    \loigiai{
        Giả sử $M(a;b;c)$. Ta có
        \begin{eqnarray*}
            MA=3&\Leftrightarrow& \sqrt{(a-3)^2+(b-1)^2+c^2}=3
            \\
            MB=6&\Leftrightarrow&\sqrt{(a-3)^2+(b-6)^2+(c-6)^2}=6
            \\
            MC=5&\Leftrightarrow&\sqrt{(a-4)^3+(b-6)^2+(c-2)^2}=5
            \\
            MD=13&\Leftrightarrow&\sqrt{(a-6)^3+(b-2)^2+(c-14)^2}=13
        \end{eqnarray*}
        Ta có hệ phương trình $\heva{
            & a^2 + b^2 + c^2 - 6a - 2b + 1 = 0\\
            & a^2 + b^2 + c^2 - 6a - 12b - 12c + 45 = 0 \\
            & a^2 + b^2 + c^2 - 8a - 12b - 4c + 31 = 0 \\
            & a^2 + b^2 + c^2 - 12a - 4b - 28c + 67 = 0.
        }$
        \\
        Giữ nguyên phương trình thứ nhất, lấy phương trình thứ nhất trừ vế theo vế với các phương trình còn lại ta được hệ phương trình mới như sau 
        \\
        $\heva{
            & a^2 + b^2 + c^2 - 6a - 2b + 1 = 0 \\
            & 10b + 12c = 44 \\
            & 2a + 10b + 4c = 30 \\
            & 6a + 2b + 28c = 66
        }\Leftrightarrow \heva{
            & a^2 + b^2 + c^2 - 6a - 2b + 1 = 0 \\
            & a = 1 \\
            & b = 2 \\
            & c = 2.
        }$
        \\
        Thế $a=1$, $b=2$, $c=2$ vào phương trình thứ nhất ta thấy thoả mãn. 
        \\
        Vậy điểm $M(1;2;2)\Rightarrow OM=\sqrt{1+4+4}=3$.
    }
\end{ex}

%%==========Câu 20
\begin{ex}%[Đề Minh Hoạ TNTHPT 2024-2025]%[2PhatTrien-DMH-2024-2025, GV: Hoàng Trọng Tấn]%[id6]
    \immini{
        Kiến trúc sư thiết kế một khu sinh hoạt cộng đồng có dạng hình chữ nhật với chiều rộng và chiều dài lần lượt là $60$ m và $80$ m. Trong đó, phần được tô màu đậm là sân chơi, phần còn lại để trồng hoa. Mỗi phần trồng hoa có đường biên cong là một phần của parabola với đỉnh thuộc một trục đối xứng của hình chữ nhật và có khoảng cách từ đường biên cong đến trục đối xứng bằng $20$ m (xem hình minh họa). Diện tích của phần sân chơi là bao nhiêu mét vuông?
    }
    {
        \begin{tikzpicture}[line join = round, line cap=round,>=stealth,font=\footnotesize,scale=1]
            \draw (0,0) rectangle (3,5);
            \draw[fill=black!40] (0,0)..controls +(60:2) and +(120:2)..(3,0)--(3,5)
            ..controls +(-120:2) and +(-60:2).. (0,5)--cycle
            ;
            
            \draw[<->] (1.5,0)--(1.5,1.3)node[right,pos=0.5]{$20$ m};
            \draw[<->] (1.5,5)--(1.5,3.7)node[right,pos=0.5]{$20$ m};
            
            %\draw[->,line width=1pt] (1.5,0)--(1.5,6)node[right]{$y$};
            %\draw[->,line width=1pt] (0,0)--(4,0)node[above]{$x$};
            %\fill (1.5,0) circle(1.2pt) node[below]{$O$};
        \end{tikzpicture}
    }
    \shortans{3200}
    \loigiai{
        \immini{
            Dựng hệ trục $Oxy$ như hình vẽ, dễ thấy Parabola $(P)$ có phương trình
            $(P)\colon y=ax^2+b$.
            \\
            Đồng thời $(P)$ đi qua điểm $(30;0)$ và $(0;20)$ nên ta có hệ phương trình $\heva{&0=a\cdot 30^2+b\\&20=a\cdot 0^2+b}\Leftrightarrow \heva{&a=-\dfrac{1}{45}\\&b=20.}$
            \\
            Suy ra $(P)\colon y=-\dfrac{1}{45}x^2+20$.
            \\
            Diện tích một nửa phần trồng hoa là 
            $$
            \displaystyle \int_{-30}^{30} \left(-\dfrac{1}{45}x^2+20\right) \mathrm{\,d}x=800.
            $$
            Vậy diện tích phần sân chơi là $60\times 80 -800 \cdot 2=3\,200$ (m$^2$).
        }
        {
            \begin{tikzpicture}[line join = round, line cap=round,>=stealth,font=\footnotesize,scale=0.8]
                \draw (0,0) rectangle (3,5);
                \draw[fill=black!40] (0,0)..controls +(60:2) and +(120:2)..(3,0)--(3,5)
                ..controls +(-120:2) and +(-60:2).. (0,5)--cycle
                ;
                \draw[blue,line width=1pt] (0,0)..controls +(60:2) and +(120:2)..(3,0);
                \draw (2.5,1.5)node[]{$(P)$};
                \draw[] (1.5,0)--(1.5,1.3)node[right,pos=0.5]{$20$};
                \draw[] (1.5,5)--(1.5,3.7)node[right,pos=0.5]{$20$};
                
                \draw[->,line width=1pt] (1.5,0)--(1.5,6)node[right]{$y$};
                \draw[->,line width=1pt] (0,0)--(4,0)node[above]{$x$};
                \fill (1.5,0) circle(1.2pt) node[below]{$O$};
            \end{tikzpicture}
        }
    }
\end{ex}

%%==========Câu 21
\begin{ex}%[Đề Minh Hoạ TNTHPT 2024-2025]%[2PhatTrien-DMH-2024-2025, GV: Hoàng Trọng Tấn]%[id6]
    Một doanh nghiệp dự định sản xuất không quá $500$ sản phẩm. Nếu doanh nghiệp sản xuất $x$ sản phẩm ($1 \leq x \leq 500$) thì doanh thu nhận được khi bán hết số sản phẩm đó là $F(x) = x^3 - 1\,999x^2 + 1\,001\,000x + 250\,000$ đồng, trong khi chi phí sản xuất bình quân cho một sản phẩm là $G(x) = x + 1\,000 + \dfrac{250\,000}{x}$ (đồng). Doanh nghiệp cần sản xuất bao nhiêu sản phẩm để lợi nhuận thu được là lớn nhất?
    \shortans[]{$333$}
    \loigiai{
        Chi phí sản xuất cho $x$ sản phẩm là $xG(x)=x^2+1\,000x+250\,000$ (đồng).
        \\
        Lợi nhuận thu được $L(x)=F(x)-xG(x)= x^3-2\,000x^2+1\,000\,000x$ (đồng).
        \\
        Đạo hàm $L'(x)=3x^2-4\,000x+1\,000\,000$, giải phương trình $L'(x)=0$
        \\
        $3x^2-4\,000x+1\,000\,000=0\Leftrightarrow x=1000$ hoặc $x=\dfrac{2\,000}{6}$.
        \\
        Do $1\le x \le 500$ nên $x=\dfrac{2\,000}{6}$.
        \\
        Mà $333<x=\dfrac{2\,000}{6}<334$.
        \\
        Ta có $L(333)=148\, 148\, 037$, $L(334)=148\,147\,704$, $L(1)=998\,001$ và $L(500)=125\,000\,000$.
        \\
        Vậy doanh nghiệp cần sản xuất $333$ sản phẩm để lợi nhuận thu được là lớn nhất.
    }
\end{ex}

%%==========Câu 22
\begin{ex}%[Đề Minh Hoạ TNTHPT 2024-2025]%[2PhatTrien-DMH-2024-2025, GV: Hoàng Trọng Tấn]%[id6]
    Có hai chiếc hộp, hộp I có $6$ quả bóng màu đỏ và $4$ quả bóng màu vàng, hộp II có $7$ quả bóng màu đỏ và $3$ quả bóng màu vàng, các quả bóng có cùng kích thước và khối lượng. Lấy ngẫu nhiên một quả bóng từ hộp I bỏ vào hộp II. Sau đó, lấy ra ngẫu nhiên một quả bóng từ hộp II. Tính xác suất để quả bóng được lấy ra từ hộp II là quả bóng được chuyển từ hộp I sang, biết rằng quả bóng đó có màu đỏ (làm tròn kết quả đến hàng phần trăm).
    \shortans[]{0,21}
    \loigiai{
        Gọi $A$ là biến cố \lq\lq  quả lấy ra ở II là quả bóng được đưa từ I vào\rq\rq.
        \\
        Gọi $B$ là biến cố \lq\lq  quả bóng lấy ra ở II là đỏ\rq\rq.
        \\
        $\mathrm{P}(B)$ xảy ra theo $2$ trường hợp:
        \\
        \textbf{TH1:} Chuyển một quả đỏ từ I sang II xác suất trường hợp này là $\dfrac{6}{10}\cdot \dfrac{8}{11}$.
        \\
        \textbf{TH2:} Chuyển một quả vàng từ I sang II xác suất trường hợp này là $\dfrac{4}{10}\cdot \dfrac{7}{11}$.
        \\
        Suy ra $\mathrm{P}(B)=\dfrac{6}{10}\cdot \dfrac{8}{11}+\dfrac{4}{10}\cdot \dfrac{7}{11}=\dfrac{38}{55}$.
        \\
        $A\cap B$ là biến cố \lq\lq  quả bóng lấy ra ở II là đỏ và nó là quả bóng thuộc I\rq\rq.
        \\
        Phép thử gồm $2$ hành động: lấy $1$ quả ở I đưa vào II và từ II lấy $1$ quả.
        \\
        Không gian mẫu có $10\cdot 11=110$ kết quả.
        \\
        $A\cap B$ có số kết quả thuận lợi là $6\cdot 1=6$ kết quả.
        \\
        Suy ra $\mathrm{P}(A\cap B)=\dfrac{6}{110}$.
        \\
        Theo định lý Bayes ta có $\mathrm{P}(A\mid B)=\dfrac{\mathrm{P}(A\cap B)}{\mathrm{P}(B)}=\dfrac{\dfrac{6}{110}}{\dfrac{38}{55}}\approx 0{,}08$.
    }
\end{ex}
\Closesolutionfile{ans}
\Closesolutionfile{ansbook}
\inputansbox{6,2,3}{ans/ans-DE-PNL-01-T,ans/ans-DE-PNL-01-TF,ans/ans-DE-PNL-01-SA}


% %=================================
\begin{name}
	{\tenchude}
	{\tendethi}
	{\tentruong}
	{\thoigian}
\end{name}
\Opensolutionfile{ansbook}[ans/ansbook-DE-PNL-02]
\TN
\Opensolutionfile{ans}[ans/ans-DE-PNL-02-T]
\begin{ex}%Câu 1
	Cho hàm số $y=f(x)$ có bảng biến thiên như sau:
	\begin{center}
		\begin{tikzpicture}[>=stealth]
			\tkzTabInit[espcl=2.5,lgt=1.5,nocadre=false]
			{$x$/0.7,$f'(x)$/0.7,$f(x)$/2.1}
			{$-\infty$,$-1$,$1$,$+\infty$}
			\tkzTabLine{,+,0,-,0,+,}
			\tkzTabVar{-/$-\infty$,+/$2$,-/$-2$,+/$+\infty$}
		\end{tikzpicture}
	\end{center}
	Hàm số đã cho nghịch biến trên khoảng nào dưới đây?
	\choice
	{$\left(-2;2\right)$}
	{\True $\left(-1;1\right)$}
	{$\left(-2;1\right)$}
	{$\left(-1;+\infty\right)$}
	\loigiai{
		Từ bảng biến thiên ta suy ra hàm số nghịch biến trên khoảng $\left(-1;1\right)$.}
\end{ex}

\begin{ex}%Câu 2
	Cho cấp số nhân $\left(u_n\right)$ với $u_1=6$ và $u_2=-12.$ Công bội $q$ của cấp số nhân đã cho là
	\choice
	{$q=-\dfrac{1}{2}$}
	{\True $q=-2$}
	{$q=-18$}
	{$q=-6$}
	\loigiai{
		Ta có $u_2=u_1.q\Leftrightarrow-12=6.q\Leftrightarrow q=-2$.}
\end{ex}

\begin{ex}%Câu 3
	Trong không gian $Oxyz$, cho hai điểm $A\left(1;1;-2\right)$ và $B\left(3;-1;2\right)$. Tọa độ của vectơ $\overrightarrow{BA}$ là
	\choice
	{$\left(2;-2;4\right)$}
	{$\left(2;0;0\right)$}
	{$\left(1;-1;2\right)$}
	{\True $~\left(-2;2;-4\right)$}
	\loigiai{
		Tọa độ của vectơ $\overrightarrow{BA}$ là: $\overrightarrow{BA}=\left(x_A-x_B;y_A-y_B;z_A-z_B\right)=\left(-2;2;-4\right)$ .}
\end{ex}

\begin{ex}%Câu 4
	Tính thể tích vật thể tròn xoay khi quay hình phẳng giới hạn bởi các đường cong $y=\sqrt{e^x-x},$ $y=0$, $x=1$, $x=2$ xung quanh trục $Ox$ là
	\choice
	{\True $\pi\left(e^2-e-\dfrac{3}{2}\right)$}
	{$e^2-e-\dfrac{5}{2}$}
	{$\pi\left(e^2-e-\dfrac{5}{2}\right)$}
	{$e^2-e-\dfrac{3}{2}$}
	\loigiai{
		Hàm số $y=\sqrt{e^x-x}$ liên tục và không âm trên đoạn $\left[1;2\right]$ nên thể tích vật thể tròn xoay khi quay hình phẳng giới hạn bởi các đường $y=\sqrt{e^x-x},$ $y=0,x=1,x=2$ xung quanh trục Ox là:\\
		$V=\pi\displaystyle\int\limits_1^2\left(\sqrt{e^x-x}\right)^2dx=\pi\displaystyle\int\limits_1^2\left(e^x-x\right)dx=\left.\pi\left(e^x-\dfrac{1}{2}{x^2}\right)\right|_1^2$ $=\pi\left[\left(e^2-\dfrac{1}{2}{2^2}\right)-\left(e^1-\dfrac{1}{2}{1^2}\right)\right]=\pi\left(e^2-e-\dfrac{3}{2}\right)$ .}
\end{ex}

\begin{ex}%Câu 5
	Với mọi số thực dương a, $\log_3\left(27a\right)-\log_3a$ bằng
	\choice
	{$\log_3\left(26a\right)$}
	{$9$}
	{\True 3}
	{$3-2\log_3a$}
	\loigiai{
		Ta có: $\log_3\left(27a\right)-\log_3a=\log_327+\log_3a-\log_3a=3$ .}
\end{ex}

\begin{ex}%Câu 6
	Trong không gian $Oxyz$ , cho đường thẳng $d:\dfrac{x-1}{4}=\dfrac{-y}{2}=\dfrac{z+2}{-6}$. Vectơ nào dưới đây là một vectơ chỉ phương của $d$?
	\choice
	{$\overrightarrow{u_2}=\left(2;-1;3\right)$}
	{$\overrightarrow{u_1}=\left(4;2;-6\right)$}
	{\True $\overrightarrow{u_3}=\left(-2;1;3\right)$}
	{$\overrightarrow{u_4}=\left(1;0;2\right)$}
	\loigiai{
		Phương trình đường thẳng $d:\dfrac{x-1}{4}=\dfrac{-y}{2}=\dfrac{z+2}{-6}$ được viết lại $\dfrac{x-1}{4}=\dfrac{y}{-2}=\dfrac{z+2}{-6}$\\
		Suy ra một vectơ chỉ phương của $d$ là $\overrightarrow{u_3}=\left(-2;1;3\right)$.}
\end{ex}

\begin{ex}%Câu 7
	Tiệm cận ngang của đồ thị hàm số $y=\dfrac{2x-3}{x+1}$ là đường thẳng có phương trình:
	\choice
	{$y=-1$}
	{$x=-1$}
	{\True $y=2$}
	{$x=2$}
	\loigiai{
		Tập xác định: $D=\mathbb{R}\setminus\left\{-1\right\}$ thì ta có $\underset{x\to+\infty}{\mathop{\lim y}}=\underset{x\to+\infty}{\lim}\,\dfrac{2x-3}{x+1}=2;\underset{x\to-\infty}{\mathop{\lim y}}\,=\underset{x\to-\infty}{\lim}\,\dfrac{2x-3}{x+1}=2\,$ .\\
		Suy ra $y=2$ là tiệm cận ngang của đồ thị hàm số.}
\end{ex}

\begin{ex}%Câu 8
	Trong không gian $Oxyz,$ cho mặt cầu $(S)$ có tâm $I\left(0;-2;1\right)$ và bán kính $R=5$. Phương trình của $(S)$ là
	\choice
	{\True $x^2+\left(y+2\right)^2+\left(z-1\right)^2=25$}
	{$x^2+\left(y-2\right)^2+\left(z+1\right)^2=25$}
	{$x^2+\left(y+2\right)^2+\left(z-1\right)^2=5$}
	{$x^2+\left(y-2\right)^2+\left(z+1\right)^2=5$}
	\loigiai{
		Mặt cầu tâm $I\left(a;b;c\right)$ và bán kính $R$ có phương trình là $\left(x-a\right)^2+\left(y-b\right)^2+\left(z-c\right)^2=R^2$.}
\end{ex}

\begin{ex}%Câu 9
	Công thức tính thể tích của một khối trụ có bán kính đáy là $R$ và chiều cao $h$ là
	\choice
	{$V=2\pi{R^2}h$}
	{$V=\dfrac{4}{3}\pi{R^2}h$}
	{$V=\dfrac{1}{3}\pi{R^2}h$}
	{\True $V=\pi{R^2}h$}
	\loigiai{
		Công thức tính thể tích khối trụ là $V=\pi{R^2}h$}
\end{ex}

\begin{ex}%Câu 10
	Diện tích hình phẳng gạch sọc trong hình vẽ bên dưới bằng
	\begin{center}
		\begin{tikzpicture}[>=stealth,thick]
			\tikzset{every node/.style={scale=0.9},y=.7cm}
			\draw[->] (-3.1,0)--(5.1,0) node[below left] {$x$};
			\draw[->] (0,-1.1)--(0,9.1) node[below left] {$y$};
			\draw (0,0) node [below left] {$O$};
			\foreach \x/\nx in {1/1,3/3}
			\draw[thin] (\x,1pt)--(\x,-1pt) node [below] {$\nx$};
			\foreach \y/\ny in {2/2,8/8}
			\draw[thin] (1pt,\y)--(-1pt,\y) node [left] {$\ny$};
			\draw[dashed,thin](1,0)--(1,2)--(0,2) (1,2)--(3,2);
			\draw[dashed,thin](3,0)--(3,8)--(0,8);
			\draw[pattern=vertical lines,pattern color=red] (1,2)--(3,2)--(3,8)--plot[samples=200,domain=3:1,smooth](\x,{2^(\x)});
			\draw (2,6) node{$y=2^x$};
			\begin{scope}
				\clip (-3,-1) rectangle (5,9);
				\draw[samples=200,domain=-3:3.1,smooth] plot (\x,{2^(\x)});
			\end{scope}
		\end{tikzpicture}
	\end{center}
	\choice
	{\True $\displaystyle\int\limits_1^3\left(2^x-2\right)\,\mathrm{\,d}x$}
	{$\displaystyle\int\limits_1^3\left(2^x+2\right)\,\mathrm{\,d}x$}
	{$\displaystyle\int\limits_1^3\left(2-2^x\right)\,\mathrm{\,d}x$}
	{$\displaystyle\int\limits_1^32^x\,\mathrm{\,d}x$}
	\loigiai{
		Hình phẳng gạch sọc trong hình vẽ được giới hạn bởi các đường $y=2^x,y=2,x=1$ và $x=3$ .\\
		Do đó diện tích hình phẳng gạch sọc trong hình vẽ bằng $\displaystyle\int\limits_1^3\left|2^x-2\right|\,\mathrm{\,d}x=\displaystyle\int\limits_1^3\left(2^x-2\right)\,\mathrm{\,d}x$.}
\end{ex}

\begin{ex}%Câu 11
	Cho tứ diện $ABCD.$ Gọi $M$ và $P$ lần lượt là trung điểm của các cạnh $AB$ và $CD.$ Đặt $\overrightarrow{BA}=\overrightarrow{b}$,  $\overrightarrow{AC}=\overrightarrow{c}$,  $\overrightarrow{AD}=\overrightarrow{d}$. Khẳng định nào sau đây là đúng?
	\choice
	{\True $\overrightarrow{MP}=\dfrac{1}{2}\left(\overrightarrow{c}+\overrightarrow{d}+\overrightarrow{b}\right)$}
	{$\overrightarrow{MP}=\dfrac{1}{2}\left(\overrightarrow{d}+\overrightarrow{b}-\overrightarrow{c}\right)$}
	{$\overrightarrow{MP}=\dfrac{1}{2}\left(\overrightarrow{c}+\overrightarrow{b}-\overrightarrow{d}\right)$}
	{$\overrightarrow{MP}=\dfrac{1}{2}\left(\overrightarrow{c}+\overrightarrow{d}-\overrightarrow{b}\right)$}
	\loigiai{
		\immini{
			\begin{align*}
				\overrightarrow{MP} & = \dfrac{1}{2}\left(\overrightarrow{MC}+\overrightarrow{MD}\right)                                        \\
				                    & =\dfrac{1}{2}\left(\overrightarrow{AC}-\overrightarrow{AM}+\overrightarrow{AD}-\overrightarrow{AM}\right) \\
				                    & =\dfrac{1}{2}\left(\overrightarrow{AC}+\overrightarrow{AD}-2\overrightarrow{AM}\right)                    \\
				                    & =\dfrac{1}{2}\left(\overrightarrow{AC}+\overrightarrow{AD}-\overrightarrow{AB}\right)                     \\
				                    & =\dfrac{1}{2}\left(\overrightarrow{c}+\overrightarrow{d}+\overrightarrow{b}\right).
			\end{align*}
		}
		{\begin{tikzpicture}[line join = round, line cap = round, thick, font = \small, scale = .7]
				\path
				(0:0) coordinate (B)
				+(0:5) coordinate (C)
				+(-70:3) coordinate (D)
				+(70:4) coordinate (A)
				($(A)!.5!(B)$) coordinate (M)
				($(C)!.5!(D)$) coordinate (P);
				\draw[dashed]
				(B)--(C) (M)--(P)
				;
				\draw
				(A)--(B)--(D)--(C)--cycle
				(A)--(D)
				;
				\foreach \x/\g in {B/-135,C/-45,D/-45,A/135,M/135,P/-45}
				\fill (\x) circle (1.5pt)
				+(\g:3mm) node {$\x$};
			\end{tikzpicture}}
	}
\end{ex}

\begin{ex}%Câu 12
	Thống kê điểm trung bình môn Toán của một số học sinh lớp $12$ được mẫu số liệu sau:\\
	\centerline{\begin{tabular}{|c|c|c|c|c|c|c|c|}
			\hline
			Khoảng điểm & $\left[6,5;7\right)$ & $\left[7;7,5\right)$ & $\left[7,5;8\right)$ & $\left[8;8,5\right)$ & $\left[8,5;9\right)$ & $\left[9;9,5\right)$ & $\left[9,5;10\right)$ \\
			\hline
			Tần số      & $8$                  & $10$                 & $16$                 & $24$                 & $13$                 & $7$                  & $4$                   \\
			\hline
		\end{tabular}}\\
	Phương sai của mẫu số liệu về điểm trung bình môn Toán của các học sinh đó là
	\choice
	{$0{,}616$}
	{$0{,}785$}
	{$0{,}78$}
	{\True $0{,}609$}
	\loigiai{
		Cỡ mẫu $n=8+10+16+24+13+7+4=82$ .\\
		Số trung bình của mẫu số liệu ghép nhóm là\\
		$$\bar{x}=\dfrac{8.6,75+10.7,25+16.7,75+24.8,25+13.8,75+7.9,25+4.9,75}{82}=\dfrac{333}{41}$$
		Phương sai của mẫu số liệu ghép nhóm là:\\
		$$s^2=\dfrac{1}{82}\left(8.6,75^2+10.7,25^2+16.7,75^2+24.8,25^2+13.8,75^2+7.9,25^2+4.9,75^2\right)-\left(\dfrac{333}{41}\right)^2 \approx 0,609$$.}
\end{ex}

\Closesolutionfile{ans}
\TNTF
\Opensolutionfile{ans}[ans/ans-DE-PNL-02-TF]

\begin{ex}%Câu 13
	Vẽ bảng biến thiên hàm số $f(x)=4\sin x\cos x+2x$ trên $\left[-\pi ;\pi\right]$.
	\choiceTF
	{Đạo hàm của hàm số đã cho là $f'(x)=4\sin 2x+2$}
	{\True Hàm số $y=f(x)$ có $4$ điểm cực trị thuộc $\left[-\pi ;\pi\right]$}
	{Hàm số $y=f(x)$ nghịch biến trên khoảng $\left(-2;-1\right)$}
	{\True Giá trị lớn nhất của $f(x)$ trên đoạn $\left[0;\dfrac{\pi}{2}\right]$ là $\dfrac{2\pi}{3}+\sqrt{3}$}
	\loigiai{
		\begin{itemchoice}
			\itemch $f(x)=4\sin x\cos x+2x=2\sin 2x+2x$ $\Rightarrow{f}'(x)=4\cos 2x+2$
			\itemch $f'(x)=0\Leftrightarrow\cos 2x=-\dfrac{1}{2}
				\Leftrightarrow\hoac{ & 2x=\dfrac{2\pi}{3}+k2\pi\\ & 2x=-\dfrac{2\pi}{3}+k2\pi}
				\Leftrightarrow \hoac{ & x=\dfrac{\pi}{3}+k\pi\\ & x=-\dfrac{\pi}{3}+k\pi},k\in\mathbb{Z}.$\\
			Trên đoạn $\left[-\pi ;\pi\right],f'(x)=0\Leftrightarrow x\in\left\{\dfrac{\pi}{3};-\dfrac{\pi}{3};\dfrac{2\pi}{3};-\dfrac{2\pi}{3}\right\}$. Vẽ bảng biến thiên của hàm số $y=f(x)$ trên đoạn $\left[-\pi ;\pi\right]$.
			\begin{center}
				\begin{tikzpicture}[>=stealth]
					\tkzTabInit[espcl=2.5,lgt=1.5,nocadre=false]
					{$x$/0.7,$f'(x)$/0.7,$f(x)$/2.1}
					{$-\pi$,$-\dfrac{2\pi}{3}$,$-\dfrac{\pi}{3}$,$\dfrac{\pi}{3}$,$\dfrac{2\pi}{3}$,$\pi$}
					\tkzTabLine{,+,0,-,0,+,0,-,0,+,}
					\tkzTabVar{-/ , +/ , -/ , +/ , -/ , +/ }
				\end{tikzpicture}
			\end{center}
			Từ bảng biến thiên ta thấy trên đoạn $\left[-\pi ;\pi\right]$ hàm số có 4 điểm cực trị.
			\itemch Nhận thấy $-\dfrac{2\pi}{3}< -2 < -\dfrac{\pi}{3} < -1$ nên từ bảng biến thiên trên ta có hàm số $y=f(x)$ không nghịch biến trên khoảng $\left(-2;-1\right)$.
			\itemch Dựa vào ta có $\max \limits _ {\left[0;\frac{\pi}{2}\right]} = f\left(\dfrac{\pi}{3}\right) = \dfrac{2\pi}{3}+\sqrt{3}$.
		\end{itemchoice}
	}
\end{ex}

\begin{ex}%Câu 14
	Một người điều khiển ô tô đang ở đường dẫn muốn nhập làn vào đường cao tốc. Khi ô tô cách điểm nhập làn $200$ m thì tốc độ của ô tô là $36$ (km/h). Hai giây sau đó, ô tô bắt đầu tăng tốc với tốc độ $v(t)=at+b$ ($a,b\in\mathbb{R},a>0$), trong đó $t$ là thời gian tính bằng giây kể từ khi bắt đầu tăng tốc. Biết rằng ô tô nhập làn cao tốc sau $12$ giây và duy trì sự tăng tốc trong $24$ giây kể từ khi bắt đầu tăng tốc. Sau $24$ giây đó ô tô duy trì tốc độ cao nhất trong thời gian còn lại trên cao tốc.\\
	\centerline{\includegraphics[height=5cm]{images/2.14}}
	\choiceTF
	{\True Quãng đường ô tô đi được từ khi bắt đầu tăng tốc đến khi nhập làn là $180$ m}
	{Vận tốc của ô tô tại thời điểm nhập làn là $72$ (km/h)}
	{Quãng đường mà ô tô đi được trong thời gian $30$ giây kể từ khi ô tô cách điểm nhập làn $200$ m là $620$ m}
	{Sau 24 giây kể từ khi tăng tốc, ô tô duy trì tốc độ cao nhất trong vòng $5$ giây thì phát hiện chướng ngoại vật cách đó $300$ m. Người điều khiển lập tức đạp phanh và ô tô chuyển động chậm dần đều với $a(t)=-3$ (m/s$^2$). Khi đó ô tô dừng lại cách chứng ngại vật $10$ m}
	\loigiai{
		\begin{itemchoice}
			\itemch Ta có $36$ (km/h)$=10$ (m/s).\\
			Vận tốc ô tô lúc đầu là $10$ (m/s) nên trong $2$ giây ô tô đi được quãng đường $2 \cdot 10=20$ m\\
			Quãng đường ô tô đi được từ khi bắt đầu tăng tốc đến khi nhập làn là $200-20=180$ m.
			\itemch Ôtô nhập làn sau $12$ giây tăng tốc nên
			$$s(12)=180 \Leftrightarrow \displaystyle\int\limits_0^{12}{v(t)dt}=180 \Leftrightarrow \dfrac{1}{2}.a.12^2+12b=180 \Leftrightarrow 72a+12b=180.$$
			Mặt khác: $v(0)=10$ nên $b=10$ suy ra $a=\dfrac{5}{6}$\\
			Do đó vận tốc ô tô tại thời điểm nhập làn là $v(12)=20$ (m/s) $=72$ (km/h).
			\itemch Quãng đường ô tô đi được sau $24$ giây tăng tốc là $s(24)=\displaystyle\int\limits_0^{24}{v(t)dt}=\displaystyle\int\limits_0^{24}{\left(at+b\right)dt}=480$ m\\
			Vận tốc tại thời điểm 24 giây từ lúc tăng tốc là $v\left(24\right)=\dfrac{5}{6}.24+10=30$ (m/s)\\
			Quãng đường $4$ giây kể từ lúc vận tốc đạt lớn nhất là $30.4=120$ m\\
			Vậy tổng quãng đường đi được sau $30$ giây kể từ khi ô tô cách điểm nhập làn $200$ m là\\
			$20+480+120=620$ m.
			\itemch Vận tốc tốc tại thời điểm vượt chướng ngại vật là $30$ (m/s)\\
			Vận tốc từ khi bắt đầu phanh là $v=\displaystyle\int a dt=\displaystyle\int -3dt=-3t+C$\\
			Vì vận tốc khi phanh là $30$ (m/s) nên $C=30$ do đó vận tốc là $v=-3t+30$\\
			Khi xe dừng thì vận tốc bằng $0$ nên ta có $-3t+30=0\Leftrightarrow t=10$ (giây)\\
			Quãng đường đi được đến khi xe dừng là $S=\displaystyle\int\limits_0^{10}{\left(-3t+30\right)dt}=150$ m\\
			Vậy ôtô cách chướng ngại vật $300-150=150$ m.
		\end{itemchoice}
	}
\end{ex}

\begin{ex}%Câu 15
	Một loại sản phẩm do hai nhà máy số I, số II cùng sản xuất. Tỷ lệ phế phẩm của các nhà máy I, II lần lượt là $0,04$ ; $0,03$ . Trong một lô sản phẩm để lẫn lộn $80$ sản phẩm của nhà máy số I và $120$ sản phẩm nhà máy số II. Một khách hàng lấy ngẫu nhiên 1 sản phẩm từ lô hàng đó.
	\choiceTF
	{\True Số phần tử của không gian mẫu là $200$}
	{\True Xác suất để lấy được sản phẩm tốt là $\dfrac{483}{500}$}
	{Biết sản phẩm lấy được là phế phẩm, xác suất sản phẩm được sản xuất từ nhà máy I là $\dfrac{8}{19}$}
	{Khả năng lấy được sản phẩm không tốt của nhà máy II là thấp hơn nhà máy I}
	\loigiai{
		\begin{itemchoice}
			\itemch Số phần tử của không gian mẫu là $n\left(\Omega\right)=80+120=200$.
			\itemch Gọi $B$ là biến cố lấy được sản phẩn không tốt.\\
			Xác suất để lấy được sản phẩm tốt là $P(\bar{B})=\dfrac{80}{200}.0,96+\dfrac{120}{200}.0,97=\dfrac{483}{500}$.
			\itemch Gọi $A$ là biến cố lấy được sản phẩm của nhà máy I.\\
			Khi đó $P\left(A|B\right)=\dfrac{P\left(A\cap B\right)}{P(B)}=\dfrac{P\left(B|A\right)P(A)}{P(B)}=\dfrac{0,04 \cdot \dfrac{80}{200}}{1-\dfrac{483}{500}}=\dfrac{8}{17}$.
			\itemch Xác suất để lấy được sản phẩm không tốt ở máy I là $P(AB)=P(B|A) \cdot P(A)=0,04 \cdot \dfrac{80}{200}=\dfrac{8}{100}=0,016$\\
			Xác suất để lấy được sản phẩm không tốt ở máy II là $P\left(\overline{A}B\right)=P(B|\overline{A}) \cdot P(\overline{A})=0,03 \cdot \dfrac{120}{200}=\dfrac{3}{100}=0,018$\\
			Vậy khả năng lấy được sản phẩm không tốt ở máy II cao hơn máy I.
		\end{itemchoice}
	}
\end{ex}

\begin{ex}%Câu 16
	Các thiên thạch có đường kính lớn hơn $140m$ và có thể lại gần Trái Đất ở khoảng cách nhỏ hơn $7500000km$ được coi là những vật thể có khả năng va chạm gây nguy hiểm cho Trái Đất. Để theo dõi những thiên thạch này, các nhà nghiên cứu của trung tâm Vũ Trụ Nasa đã thiết lập các trạm quan sát các vật thể bay gần Trái Đất. Giả sử có một hệ thống quan sát có khả năng theo dõi các vật thể ở độ cao không vượt quá $4600km$ so với mực nước biển. Coi Trái Đất là khối cầu có bán kính $6400km$ . Chọn hệ trục tọa độ $Oxyz$ trong không gian có gốc $O$ tại tâm Trái Đất và đơn vị độ dài trên mỗi trục tọa độ là $1000$ km. Một thiên thạch (coi như một hạt) chuyển động với tốc độ $v_1=2\sqrt{2}.10^3$ (km/h) không đổi theo đường thẳng xuất phát từ điểm $M\left(0;5;12\right)$ đến $N\left(12;5;0\right)$\\
	\centerline{\includegraphics[width=.4\textwidth]{images/2.16.png}}
	\choiceTF
	{\True Khoảng cách thiên thạch gần với trái đất nhất có độ dài bằng $3449km$ (Kết quả làm tròn đến hàng đơn vị)}
	{\True Các nhà nghiên cứu của trung tâm vũ trụ Nasa đưa ra giả thiết nếu lúc thiên thạch đang ở vị trí $M$ bất ngờ đổi hướng và lao xuống trái đất với phương thẳng thì quãng đường dài nhất nó có thể va chạm với trái đất là $11315$ km (Kết quả làm tròn đến hàng đơn vị)}
	{Tại thời điểm thiên thạch đang ở vị trí $M$ thì có 2 vệ tinh đang ở vị trí $A\left(-6;\,-5;\,-6\right)$ và $B\left(7;\,-6;\,7\right)$ có vận tốc khác nhau di chuyển trong mặt phẳng trung trực của $MN$ và luôn cách trái đất với khoảng cố định. Khoảng cách xa nhất của 2 vệ tinh có thể đạt là $18412km$ (Kết quả làm tròn đến hàng đơn vị)}
	{\True Nếu vệ tinh $A$ đi với vận tốc $v_2=\dfrac{\pi\sqrt{97}}{3}.10^3$ (km/h) thì sẽ va chạm với thiên thạch}
	\loigiai{
		\begin{itemchoice}
			\itemch Phương trình đường thẳng $MN$ là: $\heva{& x=12t\\&y=5\\&z=12-12t}$\\
			Khoảng cách thiên thạch gần tâm Trái Đất nhất là $d(O,MN)=\dfrac{|[\vec{OM},\vec{MN}]|}{|\vec{MN}|}=\sqrt{97}$\\
			Suy ra khoảng cách thiên thạch gần Trái Đất nhất là $d(O,MN)-R=\sqrt{97}.1000-6400\approx 3449$ (km)\\
			\itemch Quãng đường dài nhất thiên thạch va chạm trái đất là $MK=\sqrt{OM^2-R^2} \approx \sqrt{13^2-6,4^2}.10^3\approx 11315$ (km)\\
			\itemch Phương trình mặt phẳng trung trực $MN$ là $x-z=0$\\
			Hai vệ tinh $A$, $B$ và tâm $O$ đều nằm trên mặt phẳng trung trực của $MN$ nên khoảng cách xa nhất của hai vệ tinh là $AB=OA+OB=\sqrt{6^2+5^2+6^2}+\sqrt{7^2+6^2+7^2}=\sqrt{97}+\sqrt{134}\approx 21425$ (km)\\
			\itemch Giả sử vệ tinh $A$ có thể va chạm với thiên thạch. Gọi vị trí va chạm là $H$. Khi đó $H$ là giao điểm của $MN$ với mặt phẳng trung trực của nó (vì $A$ nằm trên đó), do đó $H$ là trung điểm $MN$ và có toạ độ $H\left(6;5;6\right)$\\
			Ta có $AH=\sqrt{12^2+10^2+12^2}=\sqrt{388}$ (nghìn km)\\
			Khi đó: $\cos\widehat{AOH}=\dfrac{2OA^2-AH^2}{2OA \cdot OH}=\dfrac{2 \cdot 97-388}{2\sqrt{97} \cdot \sqrt{97}}=-1\Rightarrow\widehat{AOH}=180^\circ$\\
			Độ dài cung trên $AH$ là $l_{AH}=\pi\sqrt{97}.10^3$ (km)\\
			Thời gian vệ tinh di chuyển từ $A$ đến $H$ là $t_2=\dfrac{l_{AH}}{v_2}=\dfrac{\left(\pi \cdot \sqrt{97}.10^3\right)}{\left(\pi \cdot \dfrac{\sqrt{97}}{3}.10^3\right)}=3$ (giờ)\\
			Thời gian thiên thạch đi từ $M$ tới $H$ là $t_{MH}=\dfrac{MH}{v_1}=\dfrac{\sqrt{72}.10^3}{2\sqrt{2}.10^3}=3$ (giờ)\\
			Vậy vì $t_1=t_2$ nên thiên thạch và vệ tinh $A$ sẽ va chạm với nhau.
		\end{itemchoice}
	}
\end{ex}
\Closesolutionfile{ans}
\TNSA
\Opensolutionfile{ans}[ans/ans-DE-PNL-02-SA]
\begin{ex}%Câu 17
	Cho hình lăng trụ đứng $ABC.\,A'{B}'{C}'$ có đáy $ABC$ là tam giác đều cạnh bằng$\sqrt{2}$ và độ dài cạnh$B{A}'=\sqrt{6}$ . Hãy tính khoảng cách giữa hai đường thẳng $A'B$ và $B'C$ (Kết quả làm tròn đến hàng phần trăm)?\\
	\centerline{\begin{tikzpicture}[scale=1, font=\footnotesize, line join=round, line cap=round, >=stealth,thick]
			\def\ac{4} % cạnh AC
			\def\ab{2} % cạnh AB
			\def\h{4} % chiều cao
			\def\gocA{50} % góc A của đáy
			\path
			(0,0) coordinate (A)
			(\ac,0) coordinate (C)
			(-\gocA:\ab) coordinate (B)
			($(A)+(90:\h)$) coordinate (A')
			($(B)-(A)+(A')$) coordinate (B')
			($(C)-(A)+(A')$) coordinate (C');
			\draw (B)--(A')--(A)--(B)--(C)--(C')--(A')--(B')--(C') (B)--(B')--(C);
			\draw[dashed] (A)--(C);
			\foreach \x/\g in {A/180,B/-90,C/0,A'/180,C'/0,B'/-140}\fill[red] (\x) circle (1pt)+(\g:3mm) node[black]{$ \x $};
		\end{tikzpicture}}
	\shortans{0,67}
	\loigiai{
	Gọi $M,\,N$ lần lượt là trung điểm của $AC$ và $A'B$ suy ra $MN\parallel{B}'C$ nên $B'C\parallel\left(A'BM\right)$ .\\
	Khi đó: $d\left(\,B'C,\,A'B\right)=d\left(\,B'C,\,\left(A'BM\right)\right)=d\left(C,\,\left(A'BM\right)\right)=d\left(A,\,\left(A'BM\right)\right)$ .\\
	Kẻ $AH\bot{A}'M$ tại $H$ thì ta có $\left\{\begin{aligned}
			 & BM\perp AC  \\
			 & BM\perp AA' \\
		\end{aligned}\right.\Rightarrow BM\perp\left(AC{C}'{A}'\right)$ .\\
	Khi đó $\left\{\begin{aligned}
			 & AH\bot{A}'M \\
			 & AH\perp BM  \\
		\end{aligned}\right.\Rightarrow AH\perp\left(A'BM\right)$ nên $d\left(A,\,\left(A'BM\right)\right)=AH$ .\\
	Ta có $AM=\dfrac{1}{2}AC=\dfrac{\sqrt{2}}{2}$ ; $A{A}'=\sqrt{A'{B^2}-A{B^2}}=2$ .\\
	Trong $\Delta A{A}'K$ ta có: $\dfrac{1}{A{H^2}}=\dfrac{1}{A{A'^2}}+\dfrac{1}{A{M^2}}\Rightarrow d\left(\,B'C,\,A'B\right)=AH=\dfrac{A{A}'.\,AM}{\sqrt{A{A'^2}+A{M^2}}}=\dfrac{2}{3}\approx 0,67$
	}
\end{ex}

\begin{ex}%Câu 18
	Có năm địa điểm $A,B,C,D,E$. Một số địa điểm có đường đi tới nhau môt tả bằng các cạnh với độ dài quãng đường tính theo kilomet cho bởi số gắn với cạnh đó như hình vẽ. Một người đưa thư xuất phát từ bưu điện ở vị trí $C$ , cần đi qua tất cả các đường (mỗi đường đi qua ít nhất một lần) và sau đó phải trở về vị trí ban đầu $C$. Tổng số kilomet mà người đưa thư phải đi nhỏ nhất là bao nhiêu?\\
	\centerline{
		\begin{tikzpicture}[declare function={r=3;},thick]
			\path
			(0,0) coordinate (B)
			(0:r) coordinate (C)
			(-90:r) coordinate (D)
			($(C)+(D)-(B)$) coordinate (E)
			($(B)!.5!(C) +(90:2)$) coordinate (A)
			;
			\draw (B) rectangle (E) (E)--(B)node[midway,below]{$5$}--(A) node[midway,left]{$1$}--(C)node[midway,above]{$2$}
			;
			\path (B)--(D) node[left,midway]{$3$}
			(B)--(C) node[left,midway]{$3$}
			(E)--(D) node[below,midway]{$6$}
			(E)--(C) node[right,midway]{$10$};
			\foreach \x/\g in {A/90,B/180,C/0,D/-90,E/-90}\draw[fill=white] (\x) circle (1pt)+(\g:3mm) node{$\x$};
		\end{tikzpicture}
	}
	\shortans{38}
	\loigiai{
		Đi từ $C$ đến $E$ theo đường đi Euler $CABCEBDE$ dài: $2+1+3+3+5+6+10=30$ (km)\\
		Đi từ $E$ đến $C$ với quảng đường ngắn nhất dài: $5+3=8$ (km)\\
		Vậy tổng số kilomet mà người đưa thư phải đi nhỏ nhất là $30+8=38$ (km).}
\end{ex}

\begin{ex}%Câu 19
	Hệ thống định vị toàn cầu GPS (Global Positioning System) là một hệ thống cho phép xác định vị trí của một vật thể trong không gian. Trong cùng một thời điểm, vị trí của một điểm $M$ trong không gian sẽ được xác định bởi bốn vệ tinh cho trước nhờ các bộ thu phát tín hiệu đặt trên các vệ tinh. Giả sử trong không gian với hệ tọa độ $Oxyz$ , tỉ lệ độ dài trên các trục là $10$ km tính cho một đơn vị tỉ lệ trên mỗi trục có 4 vệ tinh lần lượt đặt tại các điểm $A(-1;2;-1)$ ,$B(1;4;0)$ ,$C(3;0;9)$ và $D(7;10;-1)$ . Ở một thời điểm cả bốn vệ tinh bắn tín hiệu về điểm $M$ và đo được độ dài $MA=6,\,MB=3,\,MC=10,\,MD=6$ . Ngay sau đó $10$ giây, cả bốn vệ tinh lại bắn tin hiệu về vật $M$ và đo được $MA=3,\,MB=4,\,MC=\sqrt{85},\,MD=\sqrt{137}$. Nếu coi như vật $M$ chuyển động thẳng đều thì tốc độ của vật bằng bao nhiêu (đơn vị: m/s và kết quả làm tròn đến hang đơn vị)? (Bỏ qua khoảng thời gian phát và thu tín hiệu)
	\shortans{6403}
	\loigiai{
		Gọi $M\left(a;\,b;\,c\right)$ .\\
		Tại thời điểm đầu bốn vị tinh bắn tín hiệu về điểm $M$ . Khi đó ta có hệ phương trình sau:\\
		$\Leftrightarrow\left\{\begin{matrix}
				{\left(a+1\right)^2}+\left(b-2\right)^2+\left(c+1\right)^2=36\,\,\,\,\,\,\,\,\,\,\,\,\,\,(1)               \\
				{\left(a-1\right)^2}+\left(b-4\right)^2+c^2=9\,\,\,\,\,\,\,\,\,\,\,\,\,\,\,\,\,\,\,\,\,\,\,\,\,\,\,\,\,(2) \\
				{\left(a-3\right)^2}+b^2+\left(c-9\right)^2=100\,\,\,\,\,\,\,\,\,\,\,\,\,\,\,\,\,\,\,\,\,\,\,(3)           \\
				{\left(a-7\right)^2}+\left(b-10\right)^2+\left(c+1\right)^2=36\,\,\,\,\,\,\,\,\,\,\,(4)                    \\
			\end{matrix}\right.\Leftrightarrow\left\{\begin{aligned}
				 & (2)-(1):-4a-4b-2c=-38  \\
				 & (3)-(1):-8a+4b-20c=-20 \\
				 & (4)-(1):-14a-14b=-144  \\
			\end{aligned}\right.$\\
		Giải hệ phương trình này, ta tìm được $a~=~3,b=6,c=1\Rightarrow M\left(3;\,6;\,1\right)$\\
		Sau đó 10 giây, về tinh lại bán tín hiệu về điểm $M$ . Gọi $M'\left(x;\,y;\,z\right)$ khi đó ta có:\\
		$\Leftrightarrow\left\{\begin{matrix}
				{\left(x+1\right)^2}+\left(y-2\right)^2+\left(z+1\right)^2=9\,\,\,\,\,\,\,\,\,\,\,\,\,\,(1)     \\
				{\left(x-1\right)^2}+\left(y-4\right)^2+z^2=16\,\,\,\,\,\,\,\,\,\,\,\,\,\,\,\,\,\,\,\,\,\,\,(2) \\
				{\left(x-3\right)^2}+y^2+\left(z-9\right)^2=85\,\,\,\,\,\,\,\,\,\,\,\,\,\,\,\,\,\,\,\,\,\,(3)   \\
				{\left(x-7\right)^2}+\left(y-10\right)^2+\left(z+1\right)^2=137\,\,\,\,(4)                      \\
			\end{matrix}\right.\Leftrightarrow\left\{\begin{aligned}
				 & (2)-(1):-4x-4y-2z=-4  \\
				 & (3)-(1):-8x+4y-20z=-8 \\
				 & (4)-(1):-14x-14y=-16  \\
			\end{aligned}\right.$\\
		Giải hệ phương trình này, ta tìm được $x~=~1,y=0,z=0\Rightarrow{M}'\left(1;\,0;\,0\right)$\\
		Khi đó khoảng cách $M{M}'=\sqrt{\left(1-3\right)^2+\left(0-6\right)^2+\left(0-1\right)^2}=\sqrt{41}$\\
		Do vật chuyển động thẳng đều nên vận tốc của vật là: $v_M=\dfrac{M{M}'}{t}=\dfrac{\sqrt{41}{10^4}}{10}=6403$ (m/s)
	}
\end{ex}
\begin{ex}%Câu 20
	Một ly trà sữa dạng hình nón cụt, có đường kính đáy ly 6 cm, đường kính miệng ly 9 cm, chiều cao 13,4 cm, ở miệng ly có sử dụng một nắp đậy có hình dạng nửa mặt cầu và ở đỉnh của nửa mặt cầu này có một hình tròn có đường kính 2 cm để cắm ống hút, mặt phẳng chứa hình tròn này song song với mặt phẳng chứa miệng ly (tham khảo hình vẽ sau).
	\begin{center}
		\includegraphics[width=.3\textwidth]{images/2.20a.png} \includegraphics[width=.22\textwidth]{images/2.20b.png}
		\includegraphics[width=.3\textwidth]{images/2.20c.png}
	\end{center}
	Chọn hệ trục $Oxy$ (đơn vị trên trục là centimet) với trục $Ox$ đi qua tâm của 2 đáy hình nón cụt và gốc tọa độ $O$ trùng với tâm của đáy lớn như hình vẽ trên. Tính thể tích bên trong của ly bao gồm cả thể tích của nắp (Kết quả làm tròn kết quả đến hàng đơn vị).
	\shortans{790}
	\loigiai{
	Thể tích bên trong của ly không bao gồm nắp là: $V_1=\pi{\displaystyle\int\limits_{-13,4}^0\left(\dfrac{1,5x+60,3}{13,4}\right)^2}dx\approx 600$ cm3\\
	Thể tích bên trong của ly bao gồm cả thể tích của nắp là: $V=V_1+\left(V_2-V_3\right)$ trong đó $V_2$ là nửa thể tích của khối cầu và $V_3$ là thể tích của chỏm cầu.\\
	Nửa thể tích khối cầu là: $V_2=\dfrac{1}{2}.\dfrac{4}{3}.\pi .R^3=\dfrac{243\pi}{4}$\\
	Thể tích chỏm cầu: $V_3=\pi\displaystyle\int\limits_{R-h}^R{\left(\sqrt{R^2-x^2}\right)^2dx}$ $\Leftrightarrow{V_3}=\pi\displaystyle\int\limits_{R-h}^R{\left(R^2-x^2\right)dx\Leftrightarrow{V_3}=\,}\pi\left.\left(R^2x-\dfrac{x^3}{3}\right)\right|_{R-h}^R$\\
	$\Leftrightarrow{V_3}=\pi\left[\left(R^3-\dfrac{R^3}{3}\right)-\left(R^2\left(R-h\right)-\dfrac{\left(R-h\right)^3}{3}\right)\right]$ $\Leftrightarrow{V_3}=\pi{h^2}\left(R-\dfrac{h}{3}\right)$ .\\
	Thay số ta có: $h=4,5-\dfrac{\sqrt{77}}{2}$ ; $R=4,5$ $\Rightarrow{V_3}=\dfrac{\left(729-83\sqrt{77}\right)\pi}{12}$\\
	Thể tích bên trong của ly bao gồm cả thể tích của nắp là:\\
	$V=600+\left[\dfrac{243\pi}{4}-\dfrac{\left(729-83\sqrt{77}\right)\pi}{12}\right]\approx 790$ ml.}
\end{ex}

\begin{ex}%Câu 21
	Một doanh nghiệp dự định sản xuất không quá $400$ sản phẩm. Nếu doanh nghiệp sản xuất $x$ sản phẩm $\left(1\le x\le 400\right)$ thì doanh thu nhận được khi bán hết số sản phẩm đó được biểu diễn bởi công thức là $F(x)=x^3-1999x^2+1001000x+250000$ (đồng). Trong đó chi phí vận hành máy móc cho mỗi sản phẩm là $G(x)=\dfrac{100000x}{\dfrac{3}{2}x+1}$ (đồng). Tổng chi phí mua nguyên vật liệu được biểu diễn bởi hàm $H(x)=2x^3+100000x-50000$ (đồng) nhưng do doanh nghiệp đó mua nguyên vật liệu với số lượng lớn nên được giảm $1\%$ cho 200 sản phẩm đầu tiên doanh nghiệp sản xuất và giảm $2\%$ cho sản phẩm tiếp theo. Doanh nghiệp cần sản xuất bao nhiêu sản phẩm để lợi nhuận thu được là lớn nhất?
	\shortans{184}
	\loigiai{
	Lợi nhuận $P(x)$ được tính bằng doanh thu trừ đi tổng chi phí: $P(x)=F(x)-xG(x)-H(x)$ .\\
	Khi $x\le 200$ thì chi phí mua nguyên liệu là:
	$$H_1(x)=0,99H(x)=0,99\left(2x^3+100000x-50000\right) \text{ (đồng).}$$
	Khi $x>200$ thì chi phí mua nguyên liệu là:
	$$H_2(x)=0,99H(200)+0,98[H(x)-H(200)]=0,01H(200)+0,98H(x) \text{ (đồng).}$$
	Xét đồng thời 2 trường hợp:\\
	Trường hợp 1: Với $1\le x\le 200$ thì ta có lợi nhuận thu được là:
	\begin{align*}
		P_1(x) & =F(x)-xG(x)-H_1(x)                                                 \\
		       & =-0,98x^3-1999x^2+902000x+299500-\dfrac{100000x^2}{\frac{3}{2}x+1} \\
	\end{align*}
	Ta có: $P_1'(x)=-2,94{x^2}-3998x-\dfrac{600000x^2+800000x}{\left(3x+2\right)^2}+902000$\\
	Phương trình $P_1'(x)=0$ có nghiệm $x=184,03\in\left(1;200\right)$ .\\
	Ta thấy $\max\limits_{\left[1;200\right]} P_1(x)=80037062,09$ tại $x=184,03$ .\\
	Trường hợp 2: Với $201\le x\le 400$ ta có lợi nhuận thu được là:
	\begin{align*}
		P_2(x) & =F(x)-xG(x)-H_2(x)                                                 \\
		       & =-0,96x^3-1999x^2+903000x+609500-\dfrac{100000x^2}{\frac{3}{2}x+1}
	\end{align*}
	Ta có $P_2'(x)=-2,88{x^2}-3998x+903000-\dfrac{600000x^2+800000x}{\left(3x+2\right)^2}$\\
	Phương trình $P_2'(x)=0$ không có nghiệm thuộc $\left(201;400\right)$ .\\
	Suy ra $\max\limits_{\left[201;400\right]} P_2(x)=P(201)\approx 7,959.10^8$\\
	Vậy doanh nghiệp cần sản xuất 184 sản phẩm thì lợi nhuận thu được là lớn nhất.
	}
\end{ex}

\begin{ex}%Câu 22
	Có 10 học sinh làm bài kiểm tra xác suất thống kê, trong đó có 2 học sinh giỏi (trả lời được 100\% các câu hỏi), 3 học sinh khá (trả lời được 80\% các câu hỏi), 5 học sinh trung bình (trả lời được 50\% các câu hỏi). Bài kiểm tra có 4 câu hỏi được lấy ngẫu nhiên từ 20 câu hỏi. Giáo viên chọn ngẫu nhiên một bài làm của học sinh để chấm điểm. Xác suất bài làm đó trả lời được cả 4 câu hỏi là bao nhiêu (kết quả làm tròn đến hàng phần trăm)?

	\shortans{0,33}
	\loigiai{
		Gọi $A$ là biến cố chọn được bài làm trả lời được cả 4 câu hỏi.\\
		Gọi $B_1,B_2,B_3$ lần lượt là biến cố chọn được bài làm của học sinh giỏi, khá, trung bình.\\
		Khi đó: $P\left(B_1\right)=\dfrac{2}{10},P\left(B_2\right)=\dfrac{3}{10},P\left(B_3\right)=\dfrac{5}{10}$ .\\
		Học sinh giỏi trả lời được 100\% các câu hỏi, nghĩa là trả lời được 20 câu suy ra $P\left(A|B_1\right)=\dfrac{C_{20}^4}{C_{20}^4}.$\\
		Học sinh khá trả lời được 80\% các câu hỏi, nghĩa là trả lời được 16 câu suy ra $P\left(A|B_2\right)=\dfrac{C_{16}^4}{C_{20}^4}.$\\
		Học sinh trung bình trả lời được 50\% các câu hỏi, nghĩa là trả lời được 10 câu suy ra\\
		$P\left(A|B_3\right)=\dfrac{C_{10}^4}{C_{20}^4}$ .\\
		Theo công thức xác suất toàn phần ta có:\\
		$P(A)=P\left(B_1\right).P\left(A|B_1\right)+P\left(B_2\right).P\left(A|B_2\right)+P\left(B_3\right).P\left(A|B_3\right)$\\
		$=\dfrac{2}{10}.\dfrac{C_{20}^4}{C_{20}^4}+\dfrac{3}{10}.\dfrac{C_{16}^4}{C_{20}^4}+\dfrac{5}{10}.\dfrac{C_{10}^4}{C_{20}^4}=\dfrac{108}{323}\approx 0,33$}
\end{ex}
\Closesolutionfile{ans}
\Closesolutionfile{ansbook}
\inputansbox{6,2,3}{ans/ans-DE-PNL-02-T,ans/ans-DE-PNL-02-TF,ans/ans-DE-PNL-02-SA}
% \begin{name}
	{\tenchude}
	{\tendethi}
	{\tentruong}
	{\thoigian}
\end{name}
\Opensolutionfile{ansbook}[ans/ansbook-3]
\TN
\Opensolutionfile{ans}[ans/ans-3-T]
\begin{ex}%Câu 1
	Hàm số $y=f(x)$ có đồ thị như sau. Hàm số $y=f(x)$ đồng biến trên khoảng nào dưới đây?\\
	\centerline{
		\begin{tikzpicture}[line join=round, line cap=round,>=stealth,thick]
			\tikzset{every node/.style={scale=0.9}}
			\draw[->] (-3.1,0)--(3.1,0) node[below left] {$x$};
			\draw[->] (0,-3.1)--(0,3.1) node[below left] {$y$};
			\draw (0,0) node [below left] {$O$};
			\foreach \x/\nx in {-2/-2,-1/-1,1/1}
			\draw[thin] (\x,1pt)--(\x,-1pt) node [below] {$\nx$};
			\foreach \y/\ny in {-2/-2,2/2}
			\draw[thin] (1pt,\y)--(-1pt,\y) node [below left] {$\ny$};
			\draw[dashed,thin](-2,0)--(-2,-2)--(0,-2);
			\draw[dashed,thin](1,0)--(1,-2)--(0,-2);
			\draw[dashed,thin](-1,0)--(-1,2)--(0,2);
			\begin{scope}
				\clip (-3,-3) rectangle (3,3);
				\draw[samples=200,domain=-2.1:2.1,smooth,variable=\x] plot (\x,{1*((\x)^3)+0*((\x)^2)+-3*(\x)+0});
			\end{scope}
		\end{tikzpicture}
	}
	\choice
	{\True $\left(-2;-1\right)$}
	{$\left(-1;1\right)$}
	{$\left(-2;1\right)$}
	{$\left(-1;2\right)$}
	\loigiai{
		Từ đồ thị, hàm số đã cho đồng biến trên khoảng $\left(-2;-1\right)$ .}
\end{ex}
\begin{ex}%Câu 2
	Trong không gian với hện tọa độ $Oxyz$ , cho mặt cầu $(S):{x^2}+y^2+z^2-2x+4y+6z-2=0$ . Bán kính của mặt cầu $(S)$ bằng:
	\choice
	{8}
	{12}
	{\True 4}
	{16}
	\loigiai{
		Mặt cầu $(S)$ có tâm $I\left(1\,;\,-2\,;\,-3\right)$ và có bán kính $R=\sqrt{1+4+9-\left(-2\right)}=\sqrt{16}=4$}
\end{ex}
\begin{ex}%Câu 3
	Tiệm cận ngang của đồ thị hàm số $y=\dfrac{1-2x}{x-2}$ là đường thẳng:
	\choice
	{$y=1$}
	{$x=2$}
	{\True $y=-2$}
	{$x=-2$}
	\loigiai{
		Đường tiệm cận ngang của đồ thị hàm số là: $y=\dfrac{a}{c}=-\dfrac{2}{1}=-2$ .}
\end{ex}
\begin{ex}%Câu 4
	Tìm tập nghiệm $S$ của bất phương trình $\log_{\tfrac{1}{2}}\left(x-3\right)\ge\log_{\tfrac{1}{2}}4$ .
	\choice
	{\True $S=\left(3;7\right]$}
	{$S=\left[3;7\right]$}
	{$S=\left(-\infty ;7\right]$}
			{$S=\left[7;+\infty\right)$}
	\loigiai{
		$\log_{\tfrac{1}{2}}\left(x-3\right)\ge\log_{\tfrac{1}{2}}4\Leftrightarrow 0<x-3\le 4\Leftrightarrow 3<x\le 7$ (vì $0<\dfrac12<1$).\\
		Vậy tập nghiệm của bất phương trình là: $S=\left(3;7\right]$}
\end{ex}
\begin{ex}%Câu 5
	Trong không gian $Oxyz$, một vectơ chỉ phương của đường thẳng $d\colon \heva{&x=-t+2\\&    y=2t\\&    z=3t}$ là
	\choice
	{$\left(2;2;3\right)$}
	{$\left(1;2;3\right)$}
	{\True $\left(1;-2;-3\right)$}
	{$\left(2;-2;-3\right)$}
	\loigiai{
		Một vectơ chỉ phương của đường thẳng $d$ là $\overrightarrow{u_d}=\left(-1\,;\,2\,;\,3\right)$}
\end{ex}
\begin{ex}%Câu 6
	Cho hình chóp $S.ABCD$ có đáy $ABCD$ là hình chữ nhật, cạnh $AB=a,BC=2a$ . Hai mặt bên $\left(SAB\right)$ và ($SAD)$ cùng vuông góc với mặt phẳng $\left(ABCD\right)$ và cạnh $SA=a$ . Tính theo $a$ thể tích $V$ của khối chóp $S.ABCD$
	\choice
	{$V=a^3$}
	{\True $V=\dfrac{2a^3}{3}$}
	{$V=2a^3$}
	{$V=\dfrac{a^3}{3}$}
	\loigiai{
		Ta có: $SA\perp\left(ABCD\right)$ và $SA=a$\\
		Thể tích khối chóp $S.ABCD$ là: $V_{S.ABCD}=\dfrac{1}{3} \cdot a \cdot a \cdot 2a=\dfrac{2a^3}{3}$}
\end{ex}
\begin{ex}%Câu 7
	Hàm số $y=3^{x^2+1}$ có giá trị nhỏ nhất bằng
	\choice
	{$1$}
	{$5$}
	{\True $3$}
	{$0$}
	\loigiai{
		Ta thấy: $x^2+1\ge 1$ và dấu \lq\lq =\rq\rq~ xảy ra khi $x=0$ nên $3^{x^2+1}\ge{3^1}=3$}
\end{ex}
\begin{ex}%Câu 8
	Cho hàm số $f(x)=x^2-\dfrac{4}{x}$. Giá trị của $\displaystyle\int\limits_1^2 f'(x)\mathrm{\,d}x$ bằng
	\choice
	{$\dfrac{7}{3}- \ln 2$}
	{$3$}
	{$\dfrac{7}{3}$}
	{\True $5$}
	\loigiai{
		Ta có: $\displaystyle\int\limits_1^2f'(x)\mathrm{\,d}x= f(x)\vert_1^2=f(2)-f(1)=\left(2^2-\dfrac{4}{2}\right)-\left(1^2-\dfrac{4}{1}\right)=5$
	}
\end{ex}
\begin{ex}%Câu 9
	Cho một cấp số cộng có số hạng đầu là $u_1=2$ và công sai $d=3$ . Số hạng thứ 10 bằng:
	\choice
	{\True 29}
	{32}
	{26}
	{30}
	\loigiai{
		Ta có: $u_{10}=u_1+9d=2+9.3=2+27=29$}
\end{ex}
\begin{ex}%[Nguyễn Tuấn, dự án sáng tác đề 12]%[2D4N3-1]
	\immini{Diện tích phần hình phẳng gạch chéo trong hình vẽ bên được tính theo công thức nào dưới đây?
		\choice
		{$\displaystyle\int\limits_{-1}^2\left(2x^2-2x-4\right)\mathrm{\,d}x$}
		{$\displaystyle\int\limits_{-1}^2(-2x+2)\mathrm{\,d}x$}
		{$\displaystyle\int\limits_{-1}^2(2x-2)\mathrm{\,d}x$}
		{\True $\displaystyle\int\limits_{-1}^2\left(-2x^2+2x+4\right)\mathrm{\,d}x$}}
	{\begin{tikzpicture}[scale=.7, font=\footnotesize, line join=round, line cap=round, >=stealth]
			\draw[->] (-1.5,0) -- (3,0)node[below]{\footnotesize $x$};
			\draw (-1,0) circle (.5pt)node[below]{\footnotesize $-1$};
			\draw (2,0) circle (.5pt)node[above]{\footnotesize $2$};
			\draw[->,color=black] (0,-2.5) -- (0,3.5)node[below left]{\footnotesize $y$};
			\fill[pattern=north west lines] plot[smooth,samples=100,domain=-1:2] (\x,{(\x)^2-2*(\x)-1})--plot[smooth,samples=100,domain=2:-1] (\x,{-(\x)^2+3});
			\draw[thick,smooth,samples=100,domain=-1.5:2.2] plot(\x,{-(\x)^2+3});
			\draw[thick,smooth,samples=100,domain=-1.2:3] plot(\x,{(\x)^2-2*(\x)-1});
			\draw[dashed] (2,0) -- (2,-1) (-1,0) -- (-1,2);
			\filldraw[fill=white] (0,0) circle (1pt)node[shift={(-45:6pt)}]{\footnotesize $O$};
			\draw (2,3) node{\footnotesize $y=-x^2+3$};
			\draw (-1,-2.2) node{\footnotesize $y=x^2-2x-1$};
		\end{tikzpicture}}
	\loigiai{
		$S=\displaystyle\int\limits_{-1}^2\left[\left(-x^2+3\right)-\left(x^2-2x-1\right)\right]\mathrm{\,d}x=\displaystyle\int\limits_{-1}^2\left(-2x^2+2x+4\right)\mathrm{\,d}x$.
	}
\end{ex}
\begin{ex}%Câu 11
	Trong không gian với hệ tọa độ $Oxyz$ , cho điểm $M\left(3;5;-7\right)$ . Tìm tọa độ của điểm $M'$ đối xứng với điểm $M$ qua trục $Oy$ .
	\choice
	{$M'\left(3;-5;-7\right)$}
	{$M'\left(3;5;7\right)$}
	{$M'\left(-3;5;-7\right)$}
	{\True $M'\left(-3;5;7\right)$}
	\loigiai{
		Gọi $H$ là hình chiếu của $M$ lên $Oy$ nên suy ra $H\left(0\,;\,5\,;\,0\right)$\\
		Điểm $M'$ là điểm đối xứng với $M$ qua $H$ nên $M'=2H-M=\left(-3\,;\,5\,;\,7\right)$}
\end{ex}
\begin{ex}%Câu 12
	Một công ty thống kê lương của nhân viên theo tuần (đơn vị: USD) theo bảng sau:
	\begin{center}
		\begin{tabular}{|c|c|c|c|c|c|}
			\hline
			Lương theo tuần (USD) & $[10; 20)$ & $[20; 30)$ & $[30; 40)$ & $[40; 50)$ & $[50; 60]$ \\
			\hline
			Số công nhân          & 4          & 6          & 10         & 20         & 10         \\
			\hline
		\end{tabular}
	\end{center}
	Độ lệch chuẩn của mẫu số liệu này bằng bao nhiêu? (làm tròn tới hàng phần chục)
	\choice
	{\True 11,7}
	{12}
	{11,4}
	{12,5}
	\loigiai{
		Ta có: $n=50$\\
		Khi đó: $\overline{x}=\dfrac{1}{50}\left(4.15+6.25+10.35+20.45+10.55\right)=40,2$\\
		Phương sai của mẫu số liệu: $s^2=\dfrac{1}{50}\left(4.15^2+6.25^2+10.35^2+20.45^2+10.55^2\right)-\overline{x}^2=136,96$\\
		Độ lệch chuẩn của mẫu số liệu là: $s=\sqrt{s^2}=\sqrt{136,96}\approx 11,7$}
\end{ex}
\Closesolutionfile{ans}
\TNTF
\Opensolutionfile{ans}[ans/ans-3-TF]
\begin{ex}%Câu 13
	Cho hàm số $y=f(x)=ax+b-\dfrac{1}{x+c}$ có đồ thị như hình vẽ.\\
	\centerline{
		\begin{tikzpicture}[thick,scale=.7]
			\tikzset{every node/.style={scale=0.9}}
			\draw[->] (-5.1,0)--(5.1,0) node[below left] {$x$};
			\draw[->] (0,-5.1)--(0,5.1) node[below left] {$y$};
			\draw (0,0) node [above right] {$O$};
			\draw[dashed] (-0.99,-5)--(-0.99,5)
			(-1,-2)--(0,-2);
			\path
			(-1,0) node[above right]{$-1$}
			(0,-2)node[below right]{$-2$}
			(0,-1)node[below]{$-1$}
			;
			\begin{scope}
				\clip (-5,-5) rectangle (5,5);
				\draw[samples=200,domain=-4:-1.01,smooth,variable=\x] plot (\x,{(1*((\x)^2)+0*(\x)+-2)/(1*(\x)+1)});
				\draw[samples=200,domain=-0.99:4,smooth,variable=\x] plot (\x,{(1*((\x)^2)+0*(\x)+-2)/(1*(\x)+1)});
				\draw[dashed,thin] (-4.1,-5.1)--(4.1,3.1);
			\end{scope}
		\end{tikzpicture}
	}
	\choiceTF
	{Đồ thị hàm số nhận đường $x=-1$ làm tiệm cận đứng và đường $y=-x-1$ làm tiệm cận xiên}
	{\True Tâm đối xứng của đồ thị hàm số là điểm có tọa độ $\left(-1;-2\right)$}
	{\True $a+b+c=1$}
	{Gọi $I$ là tâm đối xứng của đồ thị hàm số và $M$ là 1 điểm bất kì thuộc đồ thị. Giá trị nhỏ nhất của độ dài đoạn thẳng $IM$ bằng $2\left(\sqrt{2}-1\right)$}
	\loigiai{
		a) Sai: Tiệm cận xiên của đồ thị hàm số có dạng $y=ax+b$ đi qua $\left(0\,;\,-1\right)$ và $\left(-1\,;\,-2\right)$ nên suy ra $\left\{\begin{aligned}
				 & a=1  \\
				 & b=-1 \\
			\end{aligned}\right.$ nên đường tiệm cận xiên là $y=x-1$\\
		Tiệm cận đứng là đường thẳng $x=-1\Leftrightarrow x=-c\Rightarrow-c=-1\Rightarrow c=1$\\
		b) Đúng: Từ đồ thị ta thấy tâm đối xứng của đồ thị hàm số là điểm có tọa độ $\left(-1;-2\right)$\\
		c) Đúng: $a+b+c=1+\left(-1\right)+1=1$.\\
		d) Sai: Dễ thấy $I\left(-1\,;\,-2\right)$ và hàm số ban đầu là $f(x)=x-1-\dfrac{1}{x+1}$ nên $M\left(x;\,x-1-\dfrac{1}{x+1}\right)$ với mọi $x\ne-1$ .\\
		Xét biểu thức $I{M^2}=\left(x+1\right)^2+\left[x-1-\dfrac{1}{x+1}-\left(-2\right)\right]^2=\left(x+1\right)^2+\left(x+1-\dfrac{1}{x+1}\right)^2$\\
		$=2\left(x+1\right)^2+\dfrac{1}{\left(x+1\right)^2}-2\ge 2\sqrt{2\left(x+1\right)^2.\dfrac{1}{\left(x+1\right)^2}}-2=2\sqrt{2}-2$ (Bất đẳng thức Cauchy)\\
		Vậy $IM=\sqrt{2\left(\sqrt{2}-1\right)}$ .}
\end{ex}
\begin{ex}%Câu 14
	Một nghệ nhân muốn làm một bình gốm có dạng mô hình như hình 2 bằng cách quay hình phẳng $(H)$ ở hình 1 quanh trục hoành. Biết đường cong trong hình 1 là phần đồ thị $y=0,125x^3$ trên đoạn $\left[0;2\right]$ và mỗi đơn vị trên đồ thị ở hình 1 có độ dài bằng 10 cm.\\
	\centerline{
		\begin{tikzpicture}[line join=round, line cap=round,>=stealth,thick,declare function={hs(\x)=0.125*((\x)^3);},x=1.75cm]
			\draw[->] (-.5,0)--(2.5,0) node[below] {$x$};
			\draw[->] (0,-.7)--(0,2) node[left] {$y$};
			\draw (0,0) node [below left] {$O$};
			\foreach \x/\nx in {2/2}
			\draw[thin] (\x,1pt)--(\x,-1pt) node [below] {$\nx$};
			\draw[dashed] (2,0)--(2,1);
			\fill (2,1) circle(1.5pt);
			\fill[gray,opacity=.7](0,0)--(2,0)--(2,1) plot[samples=200,domain=2:0,smooth](\x,{hs(\x)});
			\draw[samples=200,domain=0:2,smooth] plot (\x,{hs(\x)});
		\end{tikzpicture}
		\begin{tikzpicture}[line join=round, line cap=round,>=stealth,thick,declare function={hs(\x)=0.125*((\x)^3);},x=1.75cm,y=1.5cm]
			\draw[dashed] (2,-1)--(2,1);
			\fill (2,1) circle(1.5pt) (2,-1) circle(1.5pt);
			\fill[gray,opacity=.7] plot[samples=200,domain=2:0,smooth](\x,{hs(\x)}) plot[samples=200,domain=0:2,smooth](\x,{hs(-\x)})--(2,1);
			\draw[samples=200,domain=0:2,smooth] plot (\x,{hs(\x)});
			\draw[samples=200,domain=0:2,smooth] plot (\x,{hs(-\x)});
			\draw[dashed] (2,0) ellipse (0.14 and 1);
		\end{tikzpicture}
	}
	\choiceTF
	{\True Chiều cao của bình gốm bằng 20 cm}
	{Đường kính đáy của bình gốm bằng 10 cm}
	{Khi cắt bình gốm bởi 1 mặt phẳng qua trục thì thiết diện thu được có diện tích bằng $10\text\!\!~\!\!\text{c}{\text{m}^2}$}
	{\True Thể tích bình gốm bằng 0,90 (đơn vị: lít, kết quả làm tròn tới hàng phần trăm)}
	\loigiai{

		a) Đúng: Chiều cao của bình gốm bằng $2.10=20$ cm.\\
		b) Sai: Ta có $y_A=f\left(x_A\right)=f(2)=0,125.2^3=\dfrac{1}{8}.8=1$ nên đường kính đáy của bình gốm là $20$ cm.\\
		c) Sai: Thiết diện thu được là: $S=2\displaystyle\int\limits_0^2\left(0,125x^3\right)\mathrm{\,d}x=1$ nhưng thực tế mỗi đơn vị trên đồ thị có độ dài bằng 10 cm nên diện tích thiết diện là: $S=1.10^2=100\text\!\!~\!\!\text{c}{\text{m}^2}$ \\
		d) Đúng: Đổi 1 lít $\text{=1d}{\text{m}^3}=1000\,\text{c}{\text{m}^3}$\\
		Thể tích bình gốm là: $S=\pi\displaystyle\int\limits_0^2\left(0,125x^3\right)^2\mathrm{\,d}x=\dfrac{2\pi}{7}$ (đvtt) nhưng thực thế mỗi đơn vị trên đồ thị có độ dài bằng 10 cm nên một đơn vị thể thích là $10^3\text{c}{\text{m}^3}$ .\\
		Thể tích bình gốm thực tế là: $V=\dfrac{2\pi}{7}{10^3}\left(\text{c}{\text{m}^3}\right)=\dfrac{2\pi}{7}\left(\mathrm{\,d}{\text{m}^3}\right)\approx 0,90$ lít.}
\end{ex}
\begin{ex}%Câu 15
	Một hộp có 80 viên bi, trong đó có 50 viên bi màu đỏ và 30 viên bi màu vàng. Các viên bi có kích thước và khối lượng như nhau. Sau khi kiểm tra, người ta thấy có 60\% số viên bi màu đỏ đánh số và 50\% số viên bi màu vàng có đánh số, những viên bi còn lại không đánh số. Lấy ra ngẫu nhiên 1 viên bi trong hộp.
	\choiceTF
	{\True Xác suất chọn được viên bi màu đỏ bằng $62,5\%$}
	{Xác suất chọn được viên bi màu vàng có đánh số bằng 18,57\%}
	{\True Xác suất chọn được viên bi không đánh số bằng 43,75\%}
	{Giả sử viên bi được lấy ra là viên bi chưa được đánh số, xác suất để viên bi đó là bi đỏ thấp hơn xác suất viên bi đó là bi vàng}
	\loigiai{
		Xét phép thử chọn ngẫu nhiên một viên bi.\\
		Gọi $A$ là biến cố: \lq\lq Chọn được viên bi màu đỏ\rq\rq~; $B$ là biến cố: \lq\lq Chọn được viên bi đã được đánh số\rq\rq~.\\
		Theo bài ra ta có $P\left(B\left| A\right.\right)=60\%,\,\,P\left(B\left|\overline{A}\right.\right)=50\%$ \\
		a) Đúng: Xác suất chọn được viên bi màu đỏ là $P(A)=\dfrac{50}{80}=62,5\%$ .\\
		b) Sai: Ta có $P\left(\overline{A}\right)=\dfrac{30}{80}=37,5\%$ , $P\left(B\left|\overline{A}\right.\right)=50\%$ .\\
		Xác suất chọn được viên bi màu vàng đã được đánh số là\\
		$P\left(\overline{A}\cap B\right)=P\left(\overline{A}\right).P\left(B\left|\overline{A}\right.\right)=37,5\%.50\%=18,75\%$ .\\
		c) Đúng: Xác suất chọn được viên bi đã được đánh số là\\
		$P(B)=P(A).P\left(B\left| A\right.\right)+P\left(\overline{A}\right).P\left(B\left|\overline{A}\right.\right)=62,5\%.60\%+37.5\%.50\%=56,25\%$\\
		Suy ra xác suất chọn được viên bi chưa đánh số là $P\left(\overline{B}\right)=1-P(B)=43,75\%$ .\\
		d) Sai: Giả sử viên bi được lấy ra là viên bi chưa được đánh số. Khi đó:\\
		Xác suất để viên bi đó là bi đỏ: $P\left(A\left|\overline{B}\right.\right)=\dfrac{P(A).P\left(\overline{B}\left| A\right.\right)}{P\left(\overline{B}\right)}=\dfrac{62.5\%.\left(1-60\%\right)}{43,75\%}=\dfrac{4}{7}$ .\\
		Xác suất để viên bi đó là bi vàng: $P\left(\overline{A}\left|\overline{B}\right.\right)=\dfrac{P\left(\overline{A}\right).P\left(\overline{B}\left| A\right.\right)}{P\left(\overline{B}\right)}=\dfrac{37,5\%.\left(1-50\%\right)}{43,75\%}=\dfrac{3}{7}$ .\\
		Vây xác suất viên bi đó là bi đỏ cao hơn xác suất viên bi đó là bi xanh.
	}
\end{ex}
\begin{ex}%Câu 16
	Trong không gian với hệ tọa độ $Oxyz$ , một cabin cáp treo xuất phát từ điểm $A\left(10;3;0\right)$ và chuyển động đều theo đường cáp có vectơ chỉ phương là $\vec{u}=\left(2;-2;1\right)$ với tốc độ $4,5$ m/s (đơn vị trên mỗi trục tọa độ là mét).\\
	\centerline{\includegraphics{images/3.16.jpg}}
	\choiceTF
	{\True Phương trình tham số của đường cáp là: $\left\{\begin{aligned}
				 & x=10+2t \\
				 & y=3-2t  \\
				 & z=t     \\
			\end{aligned}\right.\left(t\in\mathbb{R}\right)$}
	{\True Giả sử sau thời gian $t$ (s) kể từ lúc xuất phát $\left(t\ge 0\right)$ thì cabin đến điểm $M$ . Khi đó tọa độ điểm $M$ là $M\left(3t+10;-3t+3;\dfrac{3t}{2}\right)$}
	{Cabin dừng ở điểm $B$ có hoành độ $x_B=550$ , khi đó quãng đường $AB$ dài 800 m}
	{Đường cáp $AB$ tạo với mặt phẳng $\left(Oxy\right)$ một góc $30^\circ $}
	\loigiai{
		a) Đúng: Phương trình tham số của đường cáp là: $\left\{\begin{aligned}
				 & x=10+2t \\
				 & y=3-2t  \\
				 & z=t     \\
			\end{aligned}\right.\left(t\in\mathbb{R}\right)$\\
		b) Đúng: Ta có $AM=v.t=4,5t$ và ta gọi $M\left(10+2m\,;\,\,3-2m\,;\,\,m\right)$ thuộc đường thẳng $d$\\
		Khi đó: $\overrightarrow{AM}=\left(2m;\,-2m;\,m\right)$ và $\overrightarrow{AM}$ cùng hướng với vectơ $\overrightarrow{u}$ nên $m$ dương\\
		Suy ra $AM=\sqrt{4m^2+4m^2+m^2}=3\left| m\right|\xrightarrow{m>0}$ $m=1,5t$ nên $M\left(3t+10;-3t+3;\dfrac{3t}{2}\right)$\\
		c) Sai: Từ câu trên suy ra $M\equiv B\Leftrightarrow 10+3t=550\Leftrightarrow t=180$\\
		Khi đó: $AB=vt=4,5.t=4,5.180=810$ mét\\
		d) Sai: Ta có $\overrightarrow{u_{AB}}=\left(2;\,-2;\,1\right)$ và mặt phẳng $\left(Oxy\right)$ là $z=0$ nên ta có $\overrightarrow{n}=\left(0\,;\,0;\,1\right)$\\
		Từ đó: $\sin\alpha=\left|\dfrac{\overrightarrow{u}.\overrightarrow{n}}{\left|\overrightarrow{u}\right|.\left|\overrightarrow{n}\right|}\right|=\dfrac{1}{3}$ nên $\alpha\ne 30^\circ $ }
\end{ex}
\Closesolutionfile{ans}
\TNSA
\Opensolutionfile{ans}[ans/ans-3-SA]
\begin{ex}%Câu 17
	Cho khối lăng trụ tam giác đều $ABC.A'{B}'{C}'$ . Biết số đo góc nhị diện $\left[A',BC,A\right]$ bằng $30^\circ $ và tam giác $A'BC$ có diện tích bằng 32. Khoảng cách giữa hai đường thẳng $AB$ và $A'{C}'$ bằng bao nhiêu?
	\shortans{4}
	\loigiai{
	\immini{
	Gọi $M$ là trung điểm của $BC$ thì ta có $\left\{\begin{matrix}
			BC\perp AM    \\
			BC\perp A{A}' \\
		\end{matrix}\Rightarrow BC\perp\left(A'AM\right)\right.$\\
	Suy ra $\left[A',BC,A\right]=\widehat{A'MA}=30^\circ $ và đặt $AB=BC=CA=x$ $\Rightarrow\left\{\begin{matrix}
			AM=\dfrac{\sqrt{3}}{2}x                   \\
			{A}'M=\dfrac{2S{A}'.BC}{BC}=\dfrac{64}{x} \\
		\end{matrix}\right.$\\
	$\Rightarrow\text{cos}30^\circ=\dfrac{AM}{A'M}=\dfrac{\sqrt{3}}{2}x:\dfrac{64}{x}=\dfrac{\sqrt{3}{x^2}}{128}=\dfrac{\sqrt{3}}{2}\Rightarrow x=8$\\
	Vậy $A{A}'=AM.\text{tan}30^\circ=\dfrac{\sqrt{3}}{2}x.\dfrac{1}{\sqrt{3}}=4\Rightarrow d=4$\\
	Khoảng cách giữa hai đường thẳng $AB$ và $A'{C}'$ bằng $4$}
	{\begin{tikzpicture}[line cap = round, line join = round, scale = 1, declare function={a=4; b=3; h=4; g=30;}]
		\path
		(0,0) coordinate (A)
		+(90:h) coordinate (A')
		(-g:b) coordinate (B)
		+(90:h) coordinate (B')
		(0:a) coordinate (C)
		+(90:h) coordinate (C')
		($(B)!.5!(C)$) coordinate (M)
		;
		\draw (A)--(B)--(C)--(C')--(A')--cycle (A')--(B')--(C') (A')--(B)--(B');
		\draw [dashed] (A')--(C)--(A)--(M)--(A');
		\foreach \x/\g in {A/180,B/-45,C/0,A'/135,B'/-45,C'/45,M/-45} {\fill (\x)circle(1.5pt) node[shift=(\g:10pt)] {$\x$};}
	\end{tikzpicture}}
	}
\end{ex}
\begin{ex}%Câu 18
	\immini[thm]{
		Một người đưa thư xuất phát từ bưu điện ở vị trí $A$ , các điểm cần phát thư nằm dọc theo các con đường cần phải đi qua. Biết người này phải đi trên mỗi con đường ít nhất một lần (để phát được thư cho tất cả các điểm cần phát nằm dọc theo con đường đó) và cuối cùng quay lại điểm xuất phát. Độ dài các con đường như hình vẽ (đơn vị độ dài). Hỏi tổng quãng đường người đưa thư có thể đi ngắn nhất có thể bằng bao nhiêu?
	}{\begin{tikzpicture}[line cap = round, line join = round,scale=1]
			\tikzstyle{vertex}=[circle,draw,minimum size=12pt,inner sep=2pt]
			\foreach \d/\x/\y in {A/0/0, B/3/0, C/4/-1.5, D/3/-3, E/0/-3} {\node[vertex] (\d) at (\x,\y) {$\d$};}
			\foreach \a/\b/\l in {A/B/8, B/C/5, C/D/2, D/B/4, D/E/9, E/B/10, A/E/6} {\draw (\a)-- node [midway,sloped,above] (L){\phantom{L}} (\b); \node at (L) {$\l$};}
			\draw (A) to[out=-135, in=135] node [midway,left] {$7$} (E);
		\end{tikzpicture}}
	\shortans{39}
	\loigiai{
		Quãng đường người đưa thư đi từ $A$ đến $D$ là: $7+6+10+4+8+5+2+9=51$\\
		Quãng đường ngắn nhất đi từ $D$ về $A$ là qua $B$ : $4+8=12$\\
		Tổng quãng đường người đưa thư có thể đi ngắn nhất có thể là: $51+12=63$ .}
	\loigiai{
		Đi từ $C$ đến $E$ theo đường đi Euler dài: $2+1+4+3+5+6+10=31$ (km)\\
		Đi từ $E$ đến $C$ với quảng đường ngắn nhất dài: $5+1+2=8$ (km)\\
		Vậy tổng số kilomet mà người đưa thư phải đi nhỏ nhất là $31+8=39$ (km)}
\end{ex}
\begin{ex}%Câu 19
	Cho một tấm nhôm hình lục giác đều cạnh $90$ cm. Người ta cắt ở mỗi đỉnh của tấm nhôm hai hình tam giác vuông bằng nhau, biết cạnh góc vuông nhỏ bằng $x$ (cm) (cắt phần tô đậm của tấm nhôm) rồi gập tấm nhôm như hình vẽ để được một hình lăng trụ lục giác đều không có nắp. Tìm x để thể tích của khối lăng trụ lục giác đều trên là lớn nhất.\\
	\centerline{\includegraphics[width=.7\textwidth]{images/3.19}}
	\shortans{15}
	\loigiai{
	% {\color{red}HÌNH Ở ĐÂY}\\
	Điều kiện $0<x<45$\\
	Cạnh đáy của lăng trụ lục giác đều:$AB=HK=90-2x$\\
	Chiều cao của lăng trụ lục giác đều: $HA=MH.\tan 60^\circ=x\sqrt{3}$\\
	Diện tích đáy của lăng trụ lục giác đều: $S_{ABCDEF}=6S_{ABO}=6.\dfrac{\sqrt{3}}{4}{\left(90-2x\right)^2}=9x(45-x)^2$\\
	Thể tích của khối lăng trụ lục giác đều: $V(x)=HA.S_{ABCDEF}=9x\left(45-x\right)^2$\\
	Hay $V(x)=9x\left(45-x\right)^2$\\
	Xét hàm số $V(x)=9x\left(45-x\right)^2$ trên khoảng $\left(0;\,45\right)$.\\
	Bảng biến thiên:
	\begin{center}
		\begin{tikzpicture}
			\tkzTabInit[lgt=1.2,espcl=2]
			{$x$ /.7, $V’(x)$ /.7, $V(x)$ /2}
			{$-\infty$,$15$,$45$,$+\infty$}
			\tkzTabLine{ ,+,z,-,z,+, }
			\tkzTabVar{-/$-\infty$,+/$V(15)$,-/$V(45)$,+/$+\infty$}
		\end{tikzpicture}
	\end{center}
	Từ bảng biến thiên ta có: $\max \limits_{(0;45)}V(x)=V(15)=243000$ (cm$^3$)\\
	Vậy thể tích của khối lăng trụ lục giác đều lớn nhất khi và chỉ khi $x=15$ cm}
\end{ex}
\begin{ex}%Câu 20
	Từ một quả cầu bằng đá trắng sứ bán kính bằng 1 dm, người ta khoan rút lõi ngay \lq\lq chính giữa\rq\rq~ quả cầu (trục đối xứng của lõi và quả cầu trùng nhau) như hình sau với đường kính mũi khoan là 1 dm được một vật thể có thể tích $V$ là bao nhiêu dm3? (Bỏ qua độ dày mũi khoan và kết quả làm tròn đến chữ số thập phân thứ hai sau dấu phẩy)\\
	\centerline{
		\includegraphics[width=.2\textwidth]{images/3.20a.png}
		\includegraphics[width=.25\textwidth]{images/3.20b.jpg}
		\includegraphics[width=.25\textwidth]{images/3.20c.png}
	}
	\shortans{1,47}
	\loigiai{
	Gọi $V_1$ là thể tích của khối trụ và $V_2$ là thể tích của chỏm cầu\\
	Nửa chiều cao của khối trụ là: $l=\sqrt{1^2-\left(0,5\right)^2}=\dfrac{\sqrt{3}}{2}$ nên ta có thể suy ra chiều cao của chỏm cầu là: $h=1-\dfrac{\sqrt{3}}{2}$ . Thể tích khối trụ là: $V_1=\pi{R^2}.H=\pi .R^2.2l=\dfrac{\pi\sqrt{3}}{4}$\\
	Thể tích chỏm cầu: $V_2=\pi\displaystyle\int\limits_{R-h}^R{\left(\sqrt{R^2-x^2}\right)^2dx}$ $\Leftrightarrow{V_2}=\pi\displaystyle\int\limits_{R-h}^R{\left(R^2-x^2\right)dx\Leftrightarrow{V_2}=\,}\pi\left.\left(R^2x-\dfrac{x^3}{3}\right)\right|_{R-h}^R$\\
	$\Leftrightarrow{V_2}=\pi\left[\left(R^3-\dfrac{R^3}{3}\right)-\left(R^2\left(R-h\right)-\dfrac{\left(R-h\right)^3}{3}\right)\right]$ $\Leftrightarrow{V_2}=\pi{h^2}\left(R-\dfrac{h}{3}\right)$ .\\
	Thay số ta suy ra được thể tích của chỏm cầu là $V_2=\pi\left(\dfrac{2}{3}-\dfrac{3\sqrt{3}}{8}\right)$\\
	Khi đó thể tích của khối cần tìm là $V=V_1+2V_2=\dfrac{\pi\sqrt{3}}{4}+\pi\left(\dfrac{2}{3}-\dfrac{3\sqrt{3}}{8}\right)=\dfrac{8-3\sqrt{3}}{6}\pi\approx 1,47$ dm3}
\end{ex}
\begin{ex}%Câu 21
	Sân hiên hình chữ nhật của một ngôi nhà là khoảng đất $ABCD$ được lợp mái bằng kính màu để hạn chế ánh sáng đi qua với mái dốc. Các bề mặt bên $ADHE$ và $CGHD$ nằm ở bức tường bên ngoài ngôi nhà. Đặt vào mô hình hệ trục tọa độ như hình vẽ thì ta có $B\left(5;\dfrac{7}{2};0\right)\,;\,\,E\left(5;0;2\right)$ và $H\left(0;0;3\right)$ . Trên tường nhà có một ngọn đèn đặt tại điểm $L$ cách điểm $D$ một khoảng $6$ m theo phương thẳng đứng. Phần có mái của sân hiên in bóng lên khu vườn bằng phẳng phía trước ngôi nhà dưới ánh đèn tạo thành khoảng đất hạn chế ánh sáng. Tính diện tích khoảng đất đó (Kết quả làm tròn kết quả đến hàng phần chục).
	\shortans{45,9}
	\loigiai{
	{\color{red}HÌNH Ở ĐÂY}\\
	Ta có $H$ là trung điểm của $DL$ nên $GH$ là đường trung bình của $\Delta LD{G}'$ nên ta suy ra $G$ là trung điểm của $L{G}'$ .\\
	Mặt khác: $GC$ là đường trung bình của $\Delta LD{G}'$ nên $G'\left(0;\,7\,;0\right)$\\
	Ta có: $\dfrac{E'A}{ED}=\dfrac{EA}{LD}=\dfrac{2}{6}=\dfrac{1}{3}\Rightarrow{E}'A=\dfrac{1}{3}{E}'D\Rightarrow{E}'A=\dfrac{1}{2}AD=2,5\Rightarrow{E}'\left(7,5;0 ;0\right)$\\
	Mặt khác: $\dfrac{EF}{E'{F}'}=\dfrac{LE}{L{E}'}=\dfrac{2}{3}\Rightarrow{E}'{F}'=1,5.EF=1,5.3,5=5,25$\\
	Diện tích khoảng đất khi đó là: $S_{DG'F'E'}=\dfrac{\left(DG'+E'F'\right).DE'}{2}=\dfrac{(7+5,25).7,5}{2}=45,9$
	}
\end{ex}
\begin{ex}%Câu 22
	Có hai chiếc hộp, hộp I có $5$ viên bi màu trắng và $5$ viên bi màu đen; hộp II có $6$ viên bi màu trắng và $4$ viên bi màu đen. Các viên bi có cùng kích thước và khối lượng. Lấy ngẫu nhiên đồng thời hai viên bi từ hộp I bỏ sang hộp II. Sau đó lấy ngẫu nhiên một viên bi từ hộp II. Lấy ra ngẫu nhiên một viên bi, giả sử viên bi được lấy ra là viên bi màu trắng. Tính xác suất viên bi màu trắng đó thuộc hộp I (Kết quả làm tròn $2$ chữ số thập phân)
	\shortans{0,14}
	\loigiai{
		Xét các biến cố: $A$ : \lq\lq Viên bi lấy ra là viên màu trắng\rq\rq~\\
		$B_1$ : \lq\lq 2 viên bi lấy ra từ hộp I có màu trắng\rq\rq~; $B_2$ : \lq\lq 2 viên bi lấy ra từ hộp I có màu đen\rq\rq~.\\
		$B_3$ : \lq\lq 2 viên bi lấy ra từ hộp I có cả hai màu đen trắng\rq\rq~.\\
		Ta có: $P\left(B_1\right)=\dfrac{\text{C}_5^2}{\text{C}_{10}^2}=\dfrac{2}{9};\,\,P\left(B_2\right)=\dfrac{\text{C}_5^2}{\text{C}_{10}^2}=\dfrac{2}{9};\,P\left(B_3\right)=\dfrac{\text{C}_5^1\cdot\text{C}_5^1}{\text{C}_{10}^2}=\dfrac{5}{9}.$\\
		Áp dụng công thức xác suất toàn phần, ta có\\
		$P(A)=P\left(A|B_1\right).P\left(B_1\right)+P\left(A|B_2\right).P\left(B_2\right)+P\left(A|B_3\right).P\left(B_3\right)$ $=\dfrac{8}{12}.\dfrac{2}{9}+\dfrac{6}{12}.\dfrac{2}{9}+\dfrac{7}{12}.\dfrac{5}{9}=\dfrac{7}{12}$\\
		Gọi $C$ là biến cố: \lq\lq Viên bi lấy ra là bi trắng từ hộp I\rq\rq.\\
		Khi đó ta có:\\
		$P(C)=P(C|B_1).P(B_1)+P(C|B_2).P(B_2)+P(C|B_3).P(B_3) = \dfrac{2}{12}. \dfrac{2}{9}+0.\dfrac{2}{9}+\dfrac{1}{12}.\dfrac{5}{9}=\dfrac{1}{12}$.\\
		Vậy xác suất cần tìm là:\\ 
		$P\left(C|A\right)=\dfrac{P(CA)}{P(A)}=\dfrac{\dfrac{1}{12}}{\dfrac{7}{12}}=\dfrac{1}{7}\approx 0,14 $ (vì $C \subset A$)\\
		}
\end{ex}
\Closesolutionfile{ans}
\Closesolutionfile{ansbook}
\inputansbox{6,2,3}{ans/ans-3-T,ans/ans-3-TF,ans/ans-3-SA}
% \begin{name}
    {\tenchude}
    {\tendethi}
    {\tentruong}
    {\thoigian}
\end{name}
\Opensolutionfile{ansbook}[ans/ansbook-4]
\TN
\Opensolutionfile{ans}[ans/ans-4-T]
\begin{ex}%Câu 1
Cho cấp số cộng $\left(u_n\right)$ với $u_1=-1$ và $u_2=4.$ Giá trị của $u_3$ bằng
\choice
{\True $ 9$}
{$-16$}
{$ 7$}
{$-8$}
\loigiai{
Ta có: $ d=u_2-u_1=4-\left(-1\right)=5$ nên $u_3=u_2+d=4+5=9$.}
\end{ex}

\begin{ex}%Câu 2
Cho hàm số $ y=\dfrac{ax+b}{cx+d}$ với $ a,b,c \in\mathbb{R}$ có đồ thị như hình vẽ. Mệnh đề nào đúng?\\
\centerline{\begin{tikzpicture}[line join=round, line cap=round,>=stealth,thick]
        \tikzset{every node/.style={scale=0.9}}
        \draw[->] (-4.1,0)--(4.1,0) node[below left] {$x$};
        \draw[->] (0,-4.1)--(0,4.1) node[below left] {$y$};
        \draw (0,0) node [below left] {$O$}
        (1,0)node[below right]{$1$};
        \draw[dashed,thin] (1.01,-4)--(1.01,4);
        \begin{scope}
            \clip (-4,-4) rectangle (4,4);
            \draw[samples=200,domain=-4:0.99,smooth,variable=\x] plot (\x,{(1*(\x)+1)/(1*(\x)+-1)});
            \draw[samples=200,domain=1.01:4,smooth,variable=\x] plot (\x,{(1*(\x)+1)/(1*(\x)+-1)});
            \draw[dashed,thin] (-4,1/1)--(4,1/1);
        \end{scope}
\end{tikzpicture}}
\choice
{$y'>0,\forall x\ne 1$}
{$y'>0,\forall x\in\mathbb{R}$}
{$y'<0,\forall x\in\mathbb{R}$}
{\True $y'<0, \forall x\ne 1$}
\loigiai{
Từ đồ thị ta thấy $y'<0,\forall x\ne 1.$}
\end{ex}

\begin{ex}%Câu 3
Trong không gian $ Oxyz,$ cho điểm $ M\left(1;2;-1\right)$ và mặt phẳng $(P):x+2y+z=0.$ Mặt phẳng $(Q)$ qua $ M$ và song song với $(P)$ có phương trình là
\choice
{$ x+2y+z+4=0$}
{$ x+2y+z-1=0$}
{$ x+2y-z-6=0$}
{\True $ x+2y+z-4=0$}
\loigiai{
Mặt phẳng $(Q)$ có phương trình: $ x+2y+z+d=0$ với $ d\ne 0$\\
Điểm $ M\in(Q)$ nên suy ra $ 1+2.2+\left(-1\right)+d=0\Rightarrow d=-4$\\
Vậy phương trình mặt phẳng $(Q)$ là: $ x+2y+z-4=0.$}
\end{ex}

\begin{ex}%Câu 4
Số nghiệm nguyên của bất phương trình $\log_{0,5}\left(2x+6\right)\ge-5$ là
\choice
{\True $ 16$}
{$ 13$}
{$ 15$}
{$ 8$}
\loigiai{
Bất phương trình đã cho $\Leftrightarrow 0<2x+6\le{\left(0,5\right)^{-5}}\Leftrightarrow 0<2x+6\le 32\Leftrightarrow-3<x\le 13$\\
Vì $ x\in\mathbb{Z}$ nên có tất cả $ 16$ nguyện nguyên thỏa mãn.}
\end{ex}

\begin{ex}%Câu 5
Cho hình lập phương $ABCD.A'{B}'{C}'{D}'$ có cạnh $ 2$ (tham khảo hình vẽ bên dưới). Độ dài của vectơ $\vec{u}=\vec{A'{C}'}-\vec{A'A}$ bằng\\
\centerline{
\begin{tikzpicture}[line cap=round,line join=round, >=stealth,scale=.7]
    \def \a{-1.5} \def \b{-1}\def \c{4.5} \def \h{4}
    \path (.5,.5)coordinate(A) 
    +(\a,\b)coordinate(B)
    +(\c,0)coordinate(D)
    ($(B)+(D)-(A)$)coordinate(C)
    +(0,\h) coordinate(C')
    ($(B)+(C')-(C)$)coordinate(B')
    ($(A)+(C')-(C)$)coordinate(A')
    ($(D)+(C')-(C)$)coordinate(D');
    \draw [dashed] (A)--(B)(D)--(A)--(A');
    \draw (B')--(B)--(C)(B')--(C')--(C)--(D)--(D')--(A')--(B')(C')--(D');
    \foreach \x/\g in {A/135,B/-135,C/-45,D/0,A'/135,B'/180,C'/-20,D'/0}\fill[red] (\x) circle (1pt)+(\g:3mm) node[black]{$\x$};
\end{tikzpicture}
}
\choice
{$ 2\sqrt{2}$}
{$\sqrt{3}$}
{$ 2\sqrt{6}$}
{\True $ 2\sqrt{3}$}
\loigiai{
Ta có $\vec{u'}=\vec{A{C}'}-\vec{A'A}=\vec{A{A}'}+\vec{A'{C}'}=\vec{A{C}'}\Rightarrow\left|\vec{u'}\right|=\left|\vec{A{C}'}\right|=A{C}'$\\
Khi đó $A{C}'=\sqrt{A{A^2}+A'{C^{\prime 2}}}=\sqrt{2^2+8}=2\sqrt{3}$}
\end{ex}

\begin{ex}%Câu 6
Biết $\displaystyle\int{f(x)\mathrm{\,d}x=\cos x+C}$ thì $\displaystyle\int{f'(x)\mathrm{\,d}x}$ bằng
\choice
{$\sin x+C'$}
{$\cos x+C'$}
{\True $-\sin x+C'$}
{$-\cos x+C$}
\loigiai{
Ta có $ f(x)=\left(\displaystyle\int{f(x)\mathrm{\,d}x}\right)'=\left(\cos x+C\right)'=-\sin x$ nên $\displaystyle\int{f(x)\mathrm{\,d}x=-\sin x+C}$}
\end{ex}

\begin{ex}%Câu 7
Cho hàm số $f(x)$ xác định trên $\left(-\infty ;0\right)\setminus\left\{-2\right\}$ và có bảng biến thiên bên dưới. Đồ thị hàm số đã cho có tổng số đường tiệm cận ngang và tiệm cận đứng là\\
\centerline{
\begin{tikzpicture}[>=stealth]
    \tkzTabInit[espcl=3,lgt=1.5,nocadre=false]
    {$x$/0.7,$f'(x)$/1,$f(x)$/3}
    {$-\infty$,$-2$,$+\infty$}
    \tkzTabLine{,+,d,-,}
    \tkzTabVar{-/$-\infty$,+D+/$+\infty$/$1$,-/$0$}
\end{tikzpicture}
}
\choice
{\True $ 1$}
{$ 3$}
{$ 0$}
{$ 2$}
\loigiai{
Từ đồ thị hàm số không có tiệm cận ngang và có một tiệm cận đứng là đường thẳng $ x=-2$.}
\end{ex}

\begin{ex}%Câu 8
Khi cắt vật thể bởi mặt phẳng vuông góc với trục $ Ox$ tại điểm có hoành độ là $ x$ $\left(0\le x\le 3\right),$ ta được mặt cắt là một hình vuông có cạnh là $\sqrt{9-x^2}$ (được mô hình hóa bởi hình vẽ bên dưới). Thể tích của vật thể đó bằng\\
\centerline{\includegraphics{images/4.8}}
\choice
{$ 171\pi $}
{$ 171$}
{$ 18\pi $}
{\True $18$}
\loigiai{
Thể tích vật thể là: $V=\displaystyle\int\limits_0^3S(x)\mathrm{\,d}x=\displaystyle\int\limits_0^3\left(9-x^2\right)\mathrm{\,d}x=18$ .}
\end{ex}

\begin{ex}%Câu 9
Cho lăng trụ tam giác $ ABC.A'{B}'{C}'.$ Biết diện tích mặt bên $ AB{B}'{A}'$ bằng $ 15$ và khoảng cách từ $ C$ đến mặt phẳng $\left(AB{B}'{A}'\right)$ bằng $ 6$ (tham khảo hình vẽ bên cạnh). Thể tích của khối lăng trụ $ ABC.A'{B}'{C}'$ bằng bao nhiêu?\\
\centerline{
    \begin{tikzpicture}[scale=1, font=\footnotesize, line join=round, line cap=round, >=stealth]
        \def\ac{4} % cạnh AC
        \def\ab{2} % cạnh AB
        \def\h{4} % chiều cao
        \def\gocA{50} % góc A của đáy
        \path
        (0,0) coordinate (A)
        (\ac,0) coordinate (C)
        (-\gocA:\ab) coordinate (B)
        ($(A)+(80:\h)$) coordinate (A')
        ($(B)-(A)+(A')$) coordinate (B')
        ($(C)-(A)+(A')$) coordinate (C');
        \draw (A')--(A)--(B)--(C)--(C')--(A')--(B')--(C') (B)--(B');
        \draw[dashed] (A)--(C);
        \foreach \x/\g in {A/180,B/-90,C/0,A'/180,C'/0,B'/-40}\fill[red] (\x) circle (1pt)+(\g:3mm) node[black]{$ \x $};
    \end{tikzpicture}
}
\choice
{$60$}
{\True $45$}
{$90$}
{$30$}
\loigiai{
Ta có: $V_{C.AB{B}'{A}'}=\dfrac{1}{3}.S_{AB{B}'{A}'}.d\left(C,\left(AB{B}'{A}'\right)\right)=\dfrac{1}{3}.15.6=30$\\
Mặt khác: $V_{C.A'{B}'{C}'}=\dfrac{1}{3}.d\left(C,\left(A'{B}'{C}'\right)\right).S_{A'{B}'{C}'}=\dfrac{1}{3}$ thể tích khối lăng trụ\\
Suy ra $V_{C.AB{B}'{A}'}=\dfrac{2}{3}$ thể tích khối lăng trụ nên $V_{ABC.A'{B}'{C}'}=\dfrac{3}{2}.30=45$.}
\end{ex}

\begin{ex}%Câu 10
Trong không gian $Oxyz,$ cho hai điểm $A\left(1;3;2\right)$ và $B\left(4;5;6\right).$ Gọi $\alpha $ là góc giữa đường thẳng $AB$ và mặt phẳng $\left(Oxy\right).$ Giá trị của $\text{cos}\alpha $ bằng
\choice
{$\dfrac{4\sqrt{29}}{29}\cdot $}
{$\dfrac{16}{29}\cdot $}
{\True $\dfrac{\sqrt{377}}{29}\cdot $}
{$\dfrac{13}{29}\cdot $}
\loigiai{
Vectơ pháp tuyến của mặt phẳng $\left(Oxy\right)$ là: $\vec{n_{\left(Oxy\right)}}=\left(0;\,0;\,1\right)$\\
Vectơ chỉ phương của đường thẳng $ AB$ là: $\vec{n_{AB}}=\left(3;\,2;\,4\right)$\\
Ta có: $\sin\alpha=\dfrac{\left|\vec{n}.\vec{u}\right|}{\left|\vec{n}\right|.\left|\vec{u}\right|}=\dfrac{4}{1.\sqrt{9+4+16}}=\dfrac{4}{\sqrt{29}}$ suy ra $\cos\alpha=\sqrt{1-\sin^2\alpha}=\dfrac{\sqrt{377}}{29}$.}
\end{ex}

\begin{ex}%Câu 11
Theo thống kê điểm trung bình môn Toán của một số học sinh đã trúng tuyển vào lớp 10 năm học 2024 – 2025 của một trường được kết quả như bảng sau:
\begin{center}
\begin{tabular}{|c|c|c|c|c|c|c|c|}
    \hline
    Khoảng điểm & $[6,5; 7)$ & $[7; 7,5)$ & $[7,5; 8)$ & $[8; 8,5)$ & $[8,5; 9)$ & $[9; 9,5)$ & $[9,5; 10)$ \\
    \hline
    Tần số & 7 & 10 & 17 & 24 & 13 & 8 & 5 \\
    \hline
\end{tabular}
\end{center}

Khoảng tứ phân vị của mẫu số liệu ghép nhóm trên là (Kết quả làm tròn đến hàng phần chục)
\choice
{\True $\Delta_Q=1,1$}
{$\Delta_Q=1$}
{$\Delta_Q=1,2$}
{$\Delta_Q=0,6$}
\loigiai{
Ta có: $ n=84$ và $\Delta Q=Q_3-Q_1$.\\
Mặt khác: $\dfrac{3n}{4}=63\Rightarrow{Q_3}=\dfrac{\dfrac{3n}{4}-\left(7+10+17+24\right)}{13}.0,5+8,5=\dfrac{113}{13}$\\
$\dfrac{n}{4}=21\Rightarrow{Q_1}=\dfrac{\dfrac{n}{4}-7-10}{17}.\left(8-7,5\right)+7,5=\dfrac{259}{34}$\\
Vậy khoảng tứ phân vị là: $\Delta Q=Q_3-Q_1=\dfrac{113}{13}-\dfrac{259}{34}\approx 1,1$}
\end{ex}

\begin{ex}%Câu 12
Một người gửi tiết kiệm $ 10$ triệu đồng vào một ngân hàng với lãi suất $7\%/$ một năm. Biết rằng nếu không rút tiền ra khỏi ngân hàng thì cứ sau mỗi năm, số tiền lãi sẽ được nhập vào vốn ban đầu. Sau $5$ năm mới rút lãi thì người đó thu được số tiền lãi là
\choice
{$14,026$ triệu đồng}
{$50,7$ triệu đồng}
{\True $4,026$ triệu đồng}
{$3,5$ triệu đồng}
\loigiai{
Số tiền cả gốc và lãi sau $ 5$ năm là: $P_5=P_0\left(1+r\right)^5=10\left(1+7\%\right)^5\approx 14,026$ triệu đồng.\\
Vậy số tiền lãi mà người đó nhận là: $ 14,026-10=4,026$ triệu đồng}
\end{ex}
\Closesolutionfile{ans}
\TNTF
\Opensolutionfile{ans}[ans/ans-4-TF]
\begin{ex}%Câu 13
Cho hàm số $y=\dfrac{a{x^2}+bx+c}{mx+n}$ có đồ thị như hình vẽ sau:\\
\centerline{
    \begin{tikzpicture}[line join=round, line cap=round,>=stealth,thick]
        \tikzset{every node/.style={scale=0.9}}
        \draw[->] (-4.1,0)--(4.1,0) node[below left] {$x$};
        \draw[->] (0,-4.1)--(0,4.1) node[below left] {$y$};
        \draw (0,0) node [below left] {$O$};
        \foreach \x/\nx in {-2/-2,-1/-1}
        \draw[thin] (\x,1pt)--(\x,-1pt) node [below] {$\nx$};
        \foreach \y/\ny in {-2/-2,1/1}
        \draw[thin] (1pt,\y)--(-1pt,\y) node [right] {$\ny$};
        \draw[dashed,thin](-2,0)--(-2,-2)--(0,-2);
        \draw[dashed,thin] (-0.99,-4)--(-0.99,4);
        \begin{scope}
            \clip (-4,-4) rectangle (4,4);
            \draw[samples=200,domain=-4:-1.01,smooth,variable=\x] plot (\x,{(1*((\x)^2)+2*(\x)+2)/(1*(\x)+1)});
            \draw[samples=200,domain=-0.99:4,smooth,variable=\x] plot (\x,{(1*((\x)^2)+2*(\x)+2)/(1*(\x)+1)});
            \draw[dashed,thin] (-4.1,-3.1)--(4.1,5.1);
        \end{scope}
    \end{tikzpicture}
}
\choiceTFt
{Hàm số đã cho nghịch biến trên khoảng $(-2;0)$}
{\True Đồ thị của hàm số đã cho có tiệm cận xiên $y=x+1$}
{Gọi $A,B$ là hai điểm cực trị của hàm số đã cho, diện tích của tam giác $OAB$ bằng $8$ (với $O$ là gốc tọa độ)}
{\True Một trục đối xứng của đồ thị đã cho là $d:y=\left(x+1\right)\tan\dfrac{3\pi}{8}\cdot $}
\loigiai{
a) Sai: Đồ thị đi xuống trên các khoảng $\left(-2;-1\right),\left(-1;0\right)$ nên nghịch biến trên các khoảng này.\\
b) Đúng: Tiệm cận xiên qua hai điểm $\left(-1;0\right),\left(0;1\right)$ là $ y=x+1$.\\
c) Sai: Hai điểm cực trị của đồ thị là $ A\left(0;2\right),B\left(-2;-2\right)\Rightarrow{S_{OAB}}=\dfrac{1}{2}\left| 0.\left(-2\right)-\left(-2\right).2\right|=2$.\\
d) Đúng: Tiệm cận đứng $ x=-1$. Trục đối xứng của đồ thị hàm số là đường phân giác của góc tạo bởi tiệm cận đứng và tiệm cận xiên:\\
$\dfrac{x-y+1}{\sqrt{2}}=\pm\left(x+1\right)\Leftrightarrow y=\left(x+1\right)\pm\sqrt{2}\left(x+1\right)=\left(1\pm\sqrt{2}\right)\left(x+1\right)=\tan\left(\dfrac{3\pi}{8}\right)$.}
\end{ex}
\begin{ex}%[Nguyễn Tuấn, dự án sáng tác đề 12 theo chủ đề]%[2H5V2-7] 
    Trong không gian với hệ tọa độ $O x y z$, một cabin cáp treo xuất phát từ điểm $A(10 ; 3 ; 0)$ và chuyển động đều theo đường cáp có véc-tơ chỉ phương là $\vec{u}=(2 ;-2 ; 1)$ với tốc độ là 4,5 m/s (đơn vị trên mỗi trục tọa độ là mét) (Hình bên dưới).
    \definecolor{ecru}{rgb}{0.76, 0.7, 0.5}
    \definecolor{darkolivegreen}{rgb}{0.33, 0.42, 0.18}
    \definecolor{deepskyblue}{rgb}{0.0, 0.75, 1.0}
    \definecolor{antiquebrass}{rgb}{0.8, 0.58, 0.46}
    \definecolor{arsenic}{rgb}{0.23, 0.27, 0.29}
    \definecolor{ashgrey}{rgb}{0.7, 0.75, 0.71}\definecolor{alizarin}{rgb}{0.82, 0.1, 0.26}
    \begin{center}
        \begin{tikzpicture}[line join=round, line cap=round,scale=1,transform shape]
            \clip (-3,-3) rectangle (3,3);
            \fill[bottom color=white,top color=deepskyblue!90, middle color=white] (-3,-3) rectangle (3,3);
            \tikzset{dat/.pic={
                    \def\N{ 
                        (-3,3)
                        ..controls +(20:.7) and +(150:.7) ..(-.8,1.5)
                        ..controls +(-30:.7) and +(150:.6) ..(1.5,-.8)
                        ..controls +(-30:1) and +(90:.4) ..(3,-3)--(-3,-3)--cycle
                        ;
                    }
                    \draw[black]\N;
                    \fill[darkolivegreen!50] \N;
            }}
            \tikzset{cap_treo/.pic={
                    \def\X{ 
                        (-.85,1)--(-.8,1)
                        ..controls +(-90:.9) and +(-180:.6) ..(-.1,0.1)--(-.1,.05)
                        ..controls +(-180:.65) and +(-90:.9) ..cycle
                        (-.15,0.05)--(-.15,-.1)--(-.23,-.22)--(-.6,-.22)--(-.5,-.1)--(-.5,.08)
                        ;
                    }
                    %\fill[black] \X;
                    \draw\X;
                    \draw(-3,2.3)--(3.5,-1.2);
                    \draw(-1.2,1)--(-.5,1);
                    \draw(-.5,1)--(-.9,1.2)--(-.95,1.17)--(-.6,.97)--cycle;
                    \draw[fill=arsenic!80](-.9,-.05)--(.52,-.05)--(.28,-.3)--(-1.05,-.3)--cycle;
                    \def\M{ 
                        (-.85,1)--(-.8,1)
                        ..controls +(-90:.9) and +(-180:.6) ..(-.1,0.1)--(-.1,.05)
                        ..controls +(-180:.65) and +(-90:.9) ..cycle
                        (-.15,0.05)--(-.15,-.1)--(-.23,-.22)--(-.6,-.22)--(-.5,-.1)--(-.5,.08)
                        ;
                    }
                    \fill[arsenic] \M;
                    \draw\M;
                    \def\N{ 
                        (-1.05,-.3)
                        ..controls +(-120:.6) and +(120:.6) ..(-.98,-1.5)--(.4,-1.5)
                        ..controls +(70:.6) and +(-60:.3) ..(.28,-.3)--cycle
                        ;
                    }
                    \fill[arsenic] \N;
                    \draw\N;
                    \def\P{ 
                        (.4,-1.5)
                        ..controls +(70:.6) and +(-60:.3) ..(.28,-.3)--(.52,-.05)
                        ..controls +(-40:.4) and +(50:.4) ..(.6,-1.2)--cycle
                        ;
                    }
                    \fill[arsenic!80] \P;
                    \draw\P;
                    \draw (-1.23,-1)--(.5,-1)--(.77,-.6);
                    \draw (-1.05,-.5)--(-1,-.4)--(.2,-.4)--(.25,-.5)--cycle
                    (.25,-.5)--(.3,-.6)--(.3,-.6)--(-1.1,-.6)--(.-1.05,-.5)--cycle
                    ;
                    \draw[alizarin](.42,-1)..controls +(100:.4) and +(-65:.4) ..(.2,-.3);
                    \draw[alizarin](-1.15,-1)..controls +(100:.3) and +(-120:.3) ..(-.95,-.3);
            }}
            \path
            (0,0)pic[scale=1]{dat}(0,0)pic[scale=1]{cap_treo};
        \end{tikzpicture}
        \begin{tikzpicture}[scale=1, font=\footnotesize, line join=round,xscale=.2, line cap=round,>=stealth]
            \def\a{1/16} 
            \def\xmin{-3} \def\xmax{12}
            \def\ymin{-3} \def\ymax{3} 
            \coordinate (O) at (0,0);
            \coordinate (E) at (-10,-3);
            \coordinate (N) at ($(E)!.7!(O)$);
            \coordinate (P) at ($(E)!.2!(O)$);
            \coordinate (D) at ($(E)!.3!(O)$);
            \coordinate (A) at ($(N)+(3,0)$);
            \coordinate (B) at (-14,1);
            \coordinate (M) at ($(A)!.7!(B)$);
            \draw[->] (\xmin,0)--(\xmax,0) node [above right]{$y$};
            \draw[->] (O)--(0,\ymax) node [left]{$z$};
            \draw[->] (O)--(E) node [below right]{$x$};
            \node at (0,0)[above right]{$O$};
            \draw[dashed] (B)--(P)node [right]{$550$} (N)--(A)node [below right]{$A(10;3;0)$}--(3,0);
            \draw(B)node [above]{$B$}--(A);
            \draw[->,red](O)--(-4,.6)node [above right]{$\vec{u}$};
            \node at (N) [left]{$10$};
            \node at (M) [above]{$M$};
            \node at (P) [left]{$x_B$};
            \draw[fill=black] (3,0) circle (.5pt) node[below]{\footnotesize $3$};
            %	\path (-28,0) node[opacity=.5,scale=.5] {\includegraphics{images/h35}};
            \clip (\xmin+0.1,\ymin+0.1) rectangle (\xmax-0.1,\ymax-0.1);
        \end{tikzpicture}
    \end{center}
    \choiceTF
    {Phương trình chính tắc của đường cáp là $\dfrac{x+10}{2}=\dfrac{y+3}{-2}=\dfrac{z}{1}$}
    {\True Giả sử sau $t$ (s) kể từ lúc xuất phát $(t \geq 0)$, cabin đến điểm $M$. Khi đó điểm $M$ có tọạ độ là $\left(3 t+10 ;-3 t+3 ;\dfrac{3t}{2}\right)$}
    {Cabin dừng ở điểm $B$ có hoành độ $x_B=550$. Khi đó $AB=750$ m}
    {\True Đường cáp $AB$ tạo với mặt phẳng $(Oxy)$ góc $19^\circ$ (kết quả làm tròn đến hàng đơn vị của độ)}
    \loigiai{
        \begin{itemchoice}
            \itemch Phương trình chính tắc của đường cáp là $\dfrac{x-10}{2}=\dfrac{y-3}{-2}=\dfrac{z}{1}$.
            \itemch Do tốc độ chuyển động của cabin là $4{,}5$ m/s nên độ dài $A M$ bằng $4{,}5 t$ m. \\
            Vì vậy $\left|\vec{AM}\right|=4{,}5 t$ $(t \geq 0)$.\\
            Do hai véc-tơ $\vec{A M}$ và $\vec{u}$ là cùng phương và cùng hướng nên $\vec{A M}=k \vec{u}$ với $k$ là số thực dương nào đó. \\
            Suy ra $\left|\vec{A M}\right|=k|\vec{u}|=k \cdot \sqrt{2^2+(-2)^2+1}=3 k$. Do đó $3 k=4{,}5 t$. Suy ra $k=\dfrac{3 t}{2}$. \\
            Vì thế, ta có $\vec{A M}=\dfrac{3 t}{2} \vec{u}=\left(3 t ;-3 t ; \dfrac{3 t}{2}\right)$.\\
            Gọi tọa độ của điểm $M$ là $\left(x_M ; y_M ; z_M\right)$.\\
            Ta có $\vec{A M}=\left(x_M-x_A ; y_M-y_A ; z_M-z_A\right)=\left(3 t ;-3 t ; \dfrac{3 t}{2}\right)$.\\
            Nên $\heva{&x_M=3 t+x_A \\ &y_M=-3 t+y_A \\ &z_M=\dfrac{3 t}{2}+z_A}\Leftrightarrow\heva{&x_M=3 t+10 \\ &y_M=-3 t+3 \\ &z_M=\dfrac{3 t}{2}.}$\\
            Vậy điểm $M$ có tọạ độ là $\left(3 t+10 ;-3 t+3 ;\dfrac{3 t}{2}\right)$.
            \itemch Do $x_B=550$ nên $3 t+10=550$, tức là $t=180$ s. Do đó, ta có điểm $B(550 ;-537 ; 270)$. \\
            Vậy $A B=\sqrt{(550-10)^2+(-537-3)^2+(270-0)^2}=\sqrt{656100}=810$ m.
            \itemch Đường thẳng $AB$ có véc-tơ chỉ phương $\vec{u}=(2 ;-2 ; 1)$ và mặt phẳng $(Oxy)$ có véc-tơ pháp tuyến $\vec{k}=(0 ; 0 ; 1)$. Do đó, ta có
            $$
            \sin (\Delta,(O x y))=|\cos (\vec{u},\vec{k})|=\dfrac{\left|\vec{u} \cdot \vec{k}\right|}{|\vec{u}| \cdot|\vec{k}|}=\dfrac{1}{3 \cdot 1}=\dfrac{1}{3}.$$
            Vậy $(\Delta,(O x y)) \approx 19^{\circ}$.
        \end{itemchoice}
    }
\end{ex}

\begin{ex}%Câu 15
Một đoàn tàu đang đứng yên trong sân ga, ngay trước đầu tàu có một cái cây. Đoàn tàu khởi hành từ trạng thái đứng yên với gia tốc $ a=0,005t\text(\text{m/}{\text{s}^{\text{2}}})$ và đi qua cái cây trong thời gian $60$ giây. Sau $80$ giây đoàn tàu chuyển sang trạng thái chuyển động đều. Xét tính đúng sai của các khẳng định sau:\\
\centerline{\includegraphics[width=.6\textwidth]{images/4.15}}
\choiceTF
{Vận tốc của đoàn tàu là $v=5.10^{-3}{t^2}\text(\text{m/s})$}
{\True Chiều dài của đoàn tàu là $l=180$m}
{\True Sau $80$ giây, đoàn tàu chuyển động với tốc tốc $57,6\text(\text{km/h})$}
{Sau khi chuyển động đều một thời gian, đoàn tàu gặp một cây cầu có chiều dài $480$ m, khi đó đoàn tàu đi qua cây cầu đó trong thời gian $30$ giây}
\loigiai{

a) Sai: Vận tốc của tàu là $ v(t)=\displaystyle\int{a(t)\mathrm{\,d}t}=\displaystyle\int{0,005t\,\mathrm{\,d}t}=2,5.10^{-3}{t^2}\,(\text{m/s})$ với $ v(0)=0$\\
b) Đúng: Chiều dài của đoàn tàu bằng quãng đường tàu đi trong 60 giây đầu tiên và bằng\\
$ l=\displaystyle\int\limits_0^{60}{v(t)\mathrm{\,d}t}=\displaystyle\int\limits_0^{60}{2,5.10^{-3}{t^2}\mathrm{\,d}t}=180$ m\\
c) Đúng: Sau 80 giây, đoàn tàu chuyển động với tốc tốc:\\
$ v\left(80\right)=2,5.10^{-3}{80^2}=16\,\,\left(\text{m}/\text{s}\right)=57,6\left(\text{km}/\text{h}\right)$\\
d) Sai: Sau khi chuyển động đều một thời gian, đoàn tàu gặp một cây cầu có chiều dài 480 (m).\\
Khi đó đoàn tàu đi qua cây cầu đó trong thời gian $ t=\dfrac{480+180}{16}=41,25$ giây.}
\end{ex}

\begin{ex}%Câu 16
Có hai hộp đựng phiếu thi, mỗi phiếu ghi một câu hỏi. Hộp thứ nhất có $15$ phiếu và hộp thứ hai có $9$ phiếu. Sinh viên A đi thi chỉ thuộc $10$ câu ở hộp thứ nhất và $8$ câu ở hộp thứ hai (Kết quả làm tròn đến 2 chữ số sau dấu phẩy)
\choiceTF
{\True Thầy giáo rút ngẫu nhiên từ mỗi hộp ra một phiếu thi, sau đó cho sinh viên A rút ngẫu nhiên ra 1 phiếu từ 2 phiếu mà thầy giáo đã rút. Xác suất để sinh viên A trả lời được câu hỏi trong phiếu là $0,78$}
{Thầy giáo rút ngẫu nhiên ra 1 phiếu từ hộp thứ nhất bỏ vào hộp thứ hai, sau đó cho sinh viên A rút ngẫu nhiên ra 1 phiếu từ hộp thứ hai. Xác suất để sinh viên trả lời được câu hỏi trong phiếu là $0,73$}
{Thầy giáo rút ngẫu nhiên ra 2 phiếu từ hộp thứ nhất bỏ vào hộp thứ hai, sau đó cho sinh viên A rút ngẫu nhiên ra 2 phiếu từ hộp thứ hai, xác suất để sinh viên đó rút được hai câu thuộc là $0,62$}
{Thầy giáo rút ngẫu nhiên ra 1 phiếu từ hộp thứ nhất bỏ vào hộp thứ hai, sau đó cho sinh viên A rút ngẫu nhiên ra 2 phiếu từ hộp thứ hai, xác suất để sinh viên đó rút được hai câu thuộc là $0,83$}
\loigiai{
a) Đúng: Gọi $A_1$ là biến cố sinh viên rút ra được phiếu thuộc hộp 1.\\
$A_2$ là biến cố sinh viên rút ra được phiếu thuộc hộp 2.\\
$A$ là biến cố sinh viên rút ra $1$ câu thuộc, khi đó:\\
$A=\left(A_1\cap A\right)\cup\left(A_2\cap A\right)\Rightarrow P(A)=P\left(A_1\right).P\left(A|A_1\right)+P\left(A_2\right).P\left(A|A_2\right)$ .\\
Ta có: $P\left(A_1\right)=\dfrac{1}{2};P\left(A_2\right)=\dfrac{1}{2}\Rightarrow P\left(A|A_1\right)=\dfrac{C_{10}^1}{C_{15}^1}=\dfrac{2}{3};P\left(A|A_2\right)=\dfrac{C_8^1}{C_9^1}=\dfrac{8}{9}$
Vậy $P(A)=\dfrac{7}{9}\approx 0,78$ .\\
b) Sai: Gọi $B_1$ là biến cố thầy giáo rút 1 câu thuộc từ hộp 1 bỏ vào hộp 2, khi đó hộp 2 có 9 câu thuộc và 1 câu không thuộc.\\
Gọi $B_2$ là biến cố thầy giáo rút 1 câu không thuộc từ hộp 1 bỏ vào hộp 2, khi đó hộp 2 có 8 câu thuộc và 2 câu không thuộc.\\
Gọi $B$ là biến cố sinh viên rút ra $1$ câu thuộc, khi đó:\\
$B=\left(B_1\cap B\right)\cup\left(B_2\cap B\right)\Rightarrow P(B)=P\left(B_1\right).P\left(B|B_1\right)+P\left(B_2\right).P\left(B|B_2\right)$ .\\
Ta có: $P\left(B_1\right)=\dfrac{C_{10}^1}{C_{15}^1}=\dfrac{2}{3};P\left(B_2\right)=\dfrac{C_5^1}{C_{15}^1}=\dfrac{1}{3}\Rightarrow P\left(B|B_1\right)=\dfrac{C_9^1}{C_{10}^1}=\dfrac{9}{10};P\left(B|B_2\right)=\dfrac{C_8^1}{C_{10}^1}=\dfrac{4}{5}$
Vậy $P(B)\approx 0,94$ .\\
c) Sai: Gọi $C_1$ là biến cố thầy giáo rút 2 câu thuộc từ hộp 1 bỏ vào hộp 2, khi đó hộp 2 có 10 câu thuộc và 1 câu không thuộc.\\
Gọi $C_2$ là biến cố thầy giáo rút 1 câu thuộc và 1 câu không thuộc từ hộp 1 bỏ vào hộp 2, khi đó hộp 2 có 9 câu thuộc và 2 câu không thuộc.\\
Gọi $C_3$ là biến cố thầy giáo rút 2 câu không thuộc từ hộp 1 bỏ vào hộp 2, khi đó hộp 2 có 8 câu thuộc và 3 câu không thuộc.\\
Gọi $C$ là biến cố sinh viên rút ra $2$ câu thuộc, khi đó: $C=\left(C_1\cap C\right)\cup\left(C_2\cap C\right)\cup\left(C_3\cap C\right)$\\
$\Rightarrow P(C)=P\left(C_1\right).P\left(C|C_1\right)+P\left(C_2\right).P\left(C|C_2\right)+P\left(C_3\right).P\left(C|C_3\right)$\\
Ta có: $P\left(C_1\right)=\dfrac{C_{10}^2}{C_{15}^2}=\dfrac{3}{7};P\left(C_2\right)=\dfrac{C_5^1.C_{10}^1}{C_{15}^2}=\dfrac{10}{21};P\left(C_3\right)=\dfrac{C_5^2}{C_{15}^2}=\dfrac{2}{21}$\\
$P\left(C|C_1\right)=\dfrac{C_{10}^2}{C_{11}^2}=\dfrac{9}{11};P\left(C|C_2\right)=\dfrac{C_9^2}{C_{11}^2}=\dfrac{12}{35};P\left(C|C_3\right)=\dfrac{C_8^2}{C_{11}^2}=\dfrac{3}{55}$
Vậy $P(C)\approx 0,52$ .\\
d) Sai: Gọi $D_1$ là biến cố thầy giáo rút 1 câu thuộc từ hộp 1 bỏ vào hộp 2, khi đó hộp 2 có 9 câu thuộc và 1 câu không thuộc.\\
Gọi $D_2$ là biến cố thầy giáo rút 1 câu không thuộc từ hộp 1 bỏ vào hộp 2, khi đó hộp 2 có 8 câu thuộc và 2 câu không thuộc.\\
Gọi $D$ là biến cố sinh viên rút ra 2 câu thuộc, khi đó:\\
$D=\left(D_1\cap D\right)\cup\left(D_2\cap D\right)\Rightarrow P(D)=P\left(D_1\right).P\left(D|D_1\right)+P\left(D_2\right).P\left(D|D_2\right)$ .\\
Ta có: $P\left(D_1\right)=\dfrac{C_{10}^1}{C_{15}^1}=\dfrac{2}{3};P\left(D_2\right)=\dfrac{C_5^1}{C_{15}^1}=\dfrac{1}{3}$ $\Rightarrow P\left(D|D_1\right)=\dfrac{C_9^2}{C_{10}^2}=\dfrac{4}{5};P\left(D|D_2\right)=\dfrac{C_8^2}{C_{10}^2}=\dfrac{28}{45}$ .\\
Vậy $P(D)\approx 0,74$ }
\end{ex}
\Closesolutionfile{ans}
\TNSA
\Opensolutionfile{ans}[ans/ans-4-SA]
\begin{ex}%Câu 17
Cho hình lăng trụ đứng $ ABC.A'B'C'$ có đáy $ ABC$ là tam giác vuông tại $ B$ và có độ dài các cạnh$ AB=a\sqrt{3}$, $ BC=2a$, $ AA'=a\sqrt{2}$. Gọi $ M$ là trung điểm của $ BC$. Tính khoảng cách giữa hai đường thẳng $ AM$ và $B'C$ khi $ a=1$ (Kết quả làm tròn đến hàng phần trăm).
\shortans{0,55}
\loigiai{
{\color{red}HÌNH Ở ĐÂY}\\
Gọi $ N$ là trung điểm của $ B{B}'$ thì $ MN\text{//}{B}'C\Rightarrow{B}'C\text{//}\left(AMN\right)$.\\
Ta có: $ d\left(B'C,AM\right)=d\left(B'C,\left(AMN\right)\right)=d\left(B',\left(AMN\right)\right)=d\left(B,\left(AMN\right)\right)$.\\
Dựng $ BI\perp AM, BH\perp NI$$\Rightarrow BH\perp\left(AMN\right)$ nên do đó $ d\left(B'C,AM\right)=d\left(B,\left(AMN\right)\right)=BH$.\\
Vì $\Delta ABM$ vuông tại $ B$, ta có $BM=\dfrac{BC}{2}=a$ ; $ BI=\dfrac{BA.BM}{\sqrt{B{A^2}+B{M^2}}}=\dfrac{a\sqrt{3}.a}{\sqrt{3a^2+a^2}}=\dfrac{a\sqrt{3}}{2}$.\\
Xét $\Delta BIN$ vuông tại $ B$, ta có:\\
$ BN=\dfrac{B{B}'}{2}=\dfrac{a\sqrt{2}}{2}$; $ BH=\dfrac{BN.BI}{\sqrt{B{N^2}+B{I^2}}}=\dfrac{\dfrac{a\sqrt{2}}{2}.\dfrac{a\sqrt{3}}{2}}{\sqrt{\left(\dfrac{a\sqrt{2}}{2}\right)^2+\left(\dfrac{a\sqrt{3}}{2}\right)^2}}=\dfrac{a\sqrt{30}}{10}$.\\
Vậy $ d\left(B'C,AM\right)=BH=\dfrac{\sqrt{30}}{10}\xrightarrow{a=1}d\left(B'C,AM\right)\approx 0,55$.}
\end{ex}

\begin{ex}%Câu 18
Trong một cuộc thi về “bữa ăn dinh dưỡng”, ban tổ chức yêu cầu để đảm bảo lượng dinh dưỡng hằng ngày thì mỗi gia đình có $4$ thành viên cần ít nhất $900$ đơn vị prôtêin và $400$ đơn vị lipít trong thức ăn hằng ngày. Mỗi kg thịt bò chứa $800$ đơn vị prôtêin và $200$ đơn vị lipit, $1\text{(kg)}$ thịt heo chứa $600$ đơn vị prôtêin và $400$ đơn vị lipit. Biết rằng người nội trợ chỉ được chi tối đa $200$ ngàn đồng để mua thịt. Biết rằng $1\text{(kg)}$ thịt bò giá $200$ ngàn đồng, $1\text{(kg)}$ thịt heo giá $100$ ngàn đồng. Người nội trợ nên mua $x\text{(kg)}$ thịt bò và $y\text{(kg)}$ thịt heo để phí thấp nhất cho khẩu phần thức ăn mà vẫn đảm bảo chất dinh dưỡng. Khi đó hãy tìm $x+2y.$
\shortans{3}
\loigiai{
Chi phí mua thịt là $ F\left(x;y\right)=200x+100y$ (ngàn đồng).\\
Hệ điều kiện ràng buộc giữa $ x$ và $ y$ là $\left\{\begin{matrix}
x\ge 0\\
y\ge 0\\
800x+600y\ge 900\\
200x+400y\ge 400\\
\end{matrix}\Leftrightarrow\left\{\begin{matrix}
x\ge 0\\
y\ge 0\\
8x+6y\ge 9\\
x+2y\ge 2\\
\end{matrix}\right.\right.$.\\
Miền nghiệm của hệ bất phương trình này là miền không bị gạch sọc trong hình vẽ dưới đây\\
{\color{red}HÌNH Ở ĐÂY}\\
Các điểm cực biên là $ A\left(0;\,1,5\right),\,\,B\left(0,6;\,0,7\right),\,\,C\left(2;\,0\right)$.\\
Ta có: $ F\left(0;1,5\right)=150,\,\,\,F\left(0,6;0,7\right)=190,\,\,\,F\left(2;0\right)=400$.\\
Vậy chi phí mua thịt thấp nhất khi $ x=0,y=1,5\Rightarrow x+2y=3$.
}
\end{ex}
%
\begin{ex}%Câu 19
Đường đi của một khinh khí cầu được gắn trong hệ trục tọa độ là một đường cong bậc hai trên bậc nhất có đồ thị cắt trục hoành tại hai điểm có tọa độ là $\left(1;0\right)$ và $\left(8;0\right)$ với đơn vị trên hệ trục tọa độ là $ 1$ km. Biết rằng điểm cực đại của đồ thị hàm số là điểm $\left(6;5\right).$ Hỏi khi khí cầu đi qua điểm cực đại và cách mặt đất $3875$ M thì khí cầu cách gốc tọa độ theo phương ngang bao nhiêu? (đơn vị: km)\\
\centerline{\includegraphics[width=.3\textheight]{images/4.19}}
\shortans{7,2}
\loigiai{
Đồ thị hàm số bậc hai trên bậc nhất $ y=\dfrac{a{x^2}+bx+c}{x+m}$ cắt trục hoành tại hai điểm có hoành độ $ x=1,\,\,x=8$ nên ta có: $ y=\dfrac{a\left(x-1\right)\left(x-8\right)}{x+m}=\dfrac{a\left(x^2-9x+8\right)}{x+m}.$\\
Đường thẳng qua hai điểm cực trị của đồ thị hàm số là $(d):y=\dfrac{\left[a\left(x^2-9x+8\right)\right]'}{\left(x+m\right)'}=a\left(2x-9\right)$ qua điểm cực đại $\left(6;5\right)\Leftrightarrow 5=a\left(2.6-9\right)\Leftrightarrow a=\dfrac{5}{3}\Rightarrow y=\dfrac{5\left(x^2-9x+8\right)}{3\left(x+m\right)}$.\\
Ta có $y'(6)=0\Leftrightarrow m=-\dfrac{28}{3}\Rightarrow y=\dfrac{5\left(x^2-9x+8\right)}{3\left(x-\dfrac{28}{3}\right)}$.\\
Xét $ y=3,875\Leftrightarrow\dfrac{5\left(x^2-9x+8\right)}{3\left(x-\dfrac{28}{3}\right)}=3,875\Leftrightarrow\left[\begin{aligned}
& x\approx 4,125\\ 
& x\approx 7,2\\ 
\end{aligned}\right.$\\
Vậy khi khí cầu đi qua điểm cực đại và cách mặt đất $ 3875$(m) thì khí cầu cách gốc tọa độ theo phương ngang $ 7,2$ km.}
\end{ex}

\begin{ex}%Câu 20
Hệ thống lọc nước bể bơi vô cùng quan trọng khi tiến hành xây dựng công trình bơi lội để nguồn nước được làm sạch thường xuyên và giữ vệ sinh cho người bơi. Trong quá trình vận hành lọc nước thì lượng nước trong bể sẽ thay đổi theo thời gian. Lượng nước trong bể giảm nếu hệ thống đang xả nước bẩn ra khỏi bể và tăng nếu hệ thống đang cấp thêm nước sạch cho bể. Biết rằng $1$ gallon gần bằng $3,785$ lít, dung tích của bể là $ 1000$ gallon và thời điểm $6$ giờ sáng bể chứa $ 250$ gallon nước. Hàm số $ f(t)$ biểu thị cho tốc độ thay đổi lượng nước trong bể theo thời gian $ t$ giờ, từ thời điểm $6$ giờ sáng đến $6$ giờ chiều được cho bởi $f(t)=\left\{\begin{aligned}
& 100t, \text{ khi }0\le t\le 3\\ 
& 900-200t, \text{ khi }3\le t\le 6\\ 
& 100t-900, \text{ khi }6\le t\le 12\\ 
\end{aligned}\right.$ với mốc thời gian $ t=0$ tại thời điểm $6$ giờ sáng. Hỏi ở thời điểm $6$ giờ chiều thì trong bể chứa nhiêu gallon nước?
\shortans{700}
\loigiai{
Gọi $ F(t)$ là một nguyên hàm của hàm số $ f(t)$\\
Vì $ f(t)$ biểu thị cho tốc độ thay đổi lượng nước trong bể theo thời gian $ t$ nên $ F(t)$ chính là lượng nước có trong bể theo thời gian $ t$.\\
Lượng nước trong bể lúc 6 giờ sáng (ứng với $ t=0$) là $ F(0)=250$.\\
Lượng nước trong bể lúc 6 giờ chiều (ứng với $ t=12$) là $ F\left(12\right)=F(0)+\displaystyle\int\limits_0^{12}{f(t)\mathrm{\,d}t}$\\
$=250+\displaystyle\int\limits_0^3100t\mathrm{\,d}t+\displaystyle\int\limits_3^6\left(900-200t\right)\mathrm{\,d}t+\displaystyle\int\limits_6^{12}{\left(100t-900\right)\mathrm{\,d}t}=700$ (galon)\\
Đáp án:\\
7 0 0}
\end{ex}

\begin{ex}%Câu 21
Hệ thống định vị toàn cầu GPS (Global Positioning System) là một hệ thống cho phép xác định vị trí của một vật thể trong không gian. Trong cùng một thời điểm vị trí của một điểm $ M$ trong không gian sẽ được xác định bởi bốn vệ tinh cho trước nhờ các bộ thu phát tín hiệu đặt trên các vệ tinh đó. Giả sử trong không gian với hệ trục tọa độ $ Oxyz$, có bốn vệ tinh lần lượt đặt tại các điểm $ A\left(2;4;0\right),\,B\left(0;4;6\right),\,C\left(2;0;6\right),\,D\left(-1;-2;-3\right)$ và vị trí của điểm $ M\left(a;b;c\right)$ thỏa mãn biểu thức $ MA+MB+MC+MD$ nhỏ nhất. Tính độ dài $ MO$ (kết quả làm tròn đến hàng phần chục)
\shortans{3,7}
\loigiai{
Gọi $ G$ là trọng tâm của tứ diện $OABC$. Suy ra $G(1;2;3)$.\\
Từ đó suy ra $ GA=GB=GC=GO=\sqrt{14}$ và $O$ là trung điểm $GD$.
\begin{align*}
MA+MB+MC+MD&=\dfrac{MA \cdot GA + MB \cdot GB + MC \cdot GC + MD \cdot GD}{GA} \\
& \ge \dfrac{\vec{MA} \cdot \vec{GA}+\vec{MB} \cdot \vec{GB}+\vec{MC} \cdot \vec{GC}+\vec{MD} \cdot \vec{GO}}{GA} \\
&=\dfrac{\vec{MG}\left(\vec{GA}+\vec{GB}+\vec{GC}+\vec{GO}\right)+5GA^2}{GA}=5GA
\end{align*}
Dấu bằng xảy ra khi và chỉ khi $ M$ trùng với điểm $ G$.\\
Khi đó $MO=GO=\sqrt{14}\approx 3,7 $.
}
\end{ex}

\begin{ex}%Câu 22
Một hộp chứa $10$ viên bi xanh và $5$ viên bi đỏ. Bạn An lấy ra ngẫu nhiên $1$ viên bi từ hộp, xem màu, rồi bỏ ra ngoài. Nếu viên bi An lấy ra có màu xanh, bạn Bình sẽ lấy ra ngẫu nhiên $2$ viên bi từ hộp; còn nếu viên bi An lấy ra có màu đỏ, bạn Bình sẽ lấy ra ngẫu nhiên $3$ viên bi từ hộp. Tính xác suất để An lấy được viên bi màu xanh, biết rằng tất cả các viên bi được hai bạn chọn ra đều có đủ cả hai màu.
\shortans{0,55}
\loigiai{
Gọi $ A$ là biến cố An lấy được viên bi màu xanh thì $\overline{A}$ là biến cố An lấy được viên bi màu đỏ.
Gọi $ B$ là biến cố tất cả các viên bi được hai bạn chọn ra đều có đủ cả hai màu.\\
Ta có $ P(A)=\dfrac{10}{15},P\left(\overline{A}\right)=\dfrac{5}{15}$. Ta cần tính $ P\left(A\mid B\right)=\dfrac{P\left(AB\right)}{P(B)}$.\\
Khi $ A$ xảy ra thì trong hộp còn 9 viên bi xanh và 5 viên bi đỏ, Bình cần lấy ra 2 viên bi trong đó có ít nhất một viên bi đỏ nên $ P\left(B\mid A\right)=\dfrac{C_{14}^2-C_9^2}{C_{14}^2}$.\\
Khi $\overline{A}$ xảy ra thì trong hộp còn 10 viên bi xanh và 4 viên bi đỏ, Bình cần lấy ra 3 viên bi trong đó có ít nhất một viên bi xanh nên $ P\left(B\mid\overline{A}\right)=\dfrac{C_{14}^3-C_4^3}{C_{14}^3}$.\\
Vậy $ P\left(AB\right)=P(A).P\left(B\mid A\right)=\dfrac{10}{15}\cdot\dfrac{\left(C_{14}^2-C_9^2\right)}{C_{14}^2}=\dfrac{110}{273}$\\
Mặt khác: $ P(B)=P(A).P\left(B\mid A\right)+P\left(\overline{A}\right).P\left(B\mid\overline{A}\right)$ $=\dfrac{10}{15}\cdot\dfrac{\left(C_{14}^2-C_9^2\right)}{C_{14}^2}+\dfrac{5}{15}\cdot\dfrac{\left(C_{14}^3-C_4^3\right)}{C_{14}^3}=\dfrac{200}{273}$.\\
Vậy $ P\left(A\mid B\right)=\dfrac{\dfrac{110}{273}}{\dfrac{200}{273}}=0,55$.}
\end{ex}
\Closesolutionfile{ans}
\Closesolutionfile{ansbook}
\inputansbox{6,2,3}{ans/ans-4-T,ans/ans-4-TF,ans/ans-4-SA}
% \begin{name}
    {\tenchude}
    {\tendethi}
    {\tentruong}
    {\thoigian}
\end{name}
\Opensolutionfile{ansbook}[ans/ansbook-5]
\TN
\Opensolutionfile{ans}[ans/ans-5-T]
\begin{ex}%Câu 1
Họ nguyên hàm của hàm số $ f(x)=\dfrac{1}{2x+3}$ là
\choice
{$ 3\ln \left| 2x+3\right|+C$}
{$\dfrac{1}{3}\ln \left| 2x+3\right|+C$}
{$ 2\ln \left| 2x+3\right|+C$}
{\True $\dfrac{1}{2} \ln \left| 2x+3\right|+C$}
\loigiai{
Ta có $\displaystyle\int{f(x)}\mathrm{\,d}x=\dfrac{1}{2}\ln \left| 2x+3\right|+C$.
}
\end{ex}

\begin{ex}%Câu 2
Diện tích $ S$ của hình phẳng giới hạn bởi đồ thị hàm số $ y=f(x)$, liên tục trên $\left[a;b\right]$ trục hoành và hai đường thẳng $ x=a,\,x=b$ ($ a<b$) cho bởi công thức:
\choice
{\True $ S=\displaystyle\int\limits_a^b{\left| f(x)\right|dx}$}
{$ S=\displaystyle\int\limits_a^b{f(x)dx}$}
{$ S=\pi\displaystyle\int\limits_a^b{\left| f(x)\right|dx}$}
{$ S=\pi\displaystyle\int\limits_a^b{f^2(x)dx}$}
\loigiai{
Diện tích $ S$ của hình phẳng là: $ S=\displaystyle\int\limits_a^b{\left| f(x)\right|dx}$}
\end{ex}

\begin{ex}%Câu 3
Sau khi kiểm tra sức khỏe tổng quát, kết quả số cân nặng của học sinh lớp 12A sĩ số 40 HS được thể hiện trong bảng số liệu sau: (Đơn vị: kg)
\begin{center}
\begin{tabular}{|c|c|c|c|c|c|}
    \hline
    Cân nặng & $[40; 50)$ & $[50; 60)$ & $[60; 70)$ & $[70; 80)$ & $[80; 90)$ \\
    \hline
    Số học sinh & 7 & 12 & 12 & 7 & 2 \\
    \hline
\end{tabular}
\end{center}
Tứ phân vị thứ nhất của mẫu số liệu trên gần nhất với giá trị nào trong các giá trị sau?
\choice
{$ 50$}
{$ 50,5$}
{\True $ 52,5$}
{$ 55,5$}
\loigiai{
Tứ phân vị thứ nhất của dãy số liệu thuộc nhóm $\left[50;60\right)$ nên tứ phân vị thứ nhất của mẫu số liệu là $Q_1=50+\dfrac{\dfrac{40}{4}-7}{12}\left(60-50\right)=52,5$}
\end{ex}

\begin{ex}%Câu 4
Trong không gian $Oxyz$ , phương trình của đường thẳng đi qua điểm $ A\left(1;2;\,-1\right)$ và có vectơ chỉ phương $\vec{u}=\left(1;\,3;\,2\right)\,$ là
\choice
{$\dfrac{x+1}{1}\,=\,\dfrac{y\,+\,3}{2}\,=\,\dfrac{z\,+\,2}{-1}$}
{$\dfrac{x-1}{1}\,=\,\dfrac{y\,-\,3}{2}\,=\,\dfrac{z\,-\,2}{-1}$}
{$\dfrac{x+1}{1}\,=\,\dfrac{y\,+\,2}{3}\,=\,\dfrac{z\,-\,1}{2}$}
{\True $\dfrac{x-1}{1}\,=\,\dfrac{y\,-\,2}{3}\,=\,\dfrac{z\,+\,1}{2}$}
\loigiai{
Đường thẳng đi qua điểm $ A\left(1;2;\,-1\right)$ và có vectơ chỉ phương $\vec{u}=\left(1;\,3;\,2\right)\,$ có phương trình là: $\dfrac{x-1}{1}\,=\,\dfrac{y\,-\,2}{3}\,=\,\dfrac{z\,+\,1}{2}$.}
\end{ex}

\begin{ex}%Câu 5
Hàm số số $ y=\dfrac{ax+b}{cx+d}$ có đồ thị như hình bên dưới:\\
\centerline{
    \begin{tikzpicture}[line join=round, line cap=round,>=stealth,thick]
\tikzset{every node/.style={scale=0.9}}
\draw[->] (-3.1,0)--(3.1,0) node[below left] {$x$};
\draw[->] (0,-2.1)--(0,4.1) node[below left] {$y$};
\draw (0,0) node [below left] {$O$}
(-1,0)node[below left]{$-1$}
(0,2)node[above right]{$2$};
\draw[dashed,thin] (-0.99,-2)--(-0.99,4);
\begin{scope}
\clip (-3,-2) rectangle (3,4);
\draw[samples=200,domain=-4:-1.01,smooth,variable=\x] plot (\x,{(2*(\x)+1)/(1*(\x)+1)});
\draw[samples=200,domain=-0.99:4,smooth,variable=\x] plot (\x,{(2*(\x)+1)/(1*(\x)+1)});
\draw[dashed,thin] (-4,2/1)--(4,2/1);
\end{scope}
\end{tikzpicture}
}
Đường tiệm cận đứng của đồ thị là đường thẳng có phương trình
\choice
{$ x=1$}
{$ x=2$}
{$ x=-2$}
{\True $ x=-1$}
\loigiai{
Dựa vào đồ thị ta có đường tiệm cận đứng của đồ thị là đường thẳng $ x=-1$.}
\end{ex}

\begin{ex}%Câu 6
Tập nghiệm của bất phương trình $\log_{\dfrac{1}{2}}\left(x-2\right)\le 1$ là:
\choice
{\True $\left[\dfrac{5}{2};\,\,+\infty\right)$}
{$\left(\dfrac{5}{2};\,\,+\infty\right)$}
{$\left(-\infty ;{\log_2}5\right)$}
{$\left(-\infty ;\dfrac{5}{2}\right)$}
\loigiai{
Điều kiện: $ x-2>0\Leftrightarrow x>2$.\\
Bất phương trình: $\log_{\dfrac{1}{2}}\left(x-2\right)\le 1\Leftrightarrow x-2\ge\dfrac{1}{2}\Leftrightarrow x\ge\dfrac{5}{2}$.\\
Kết hợp với điều kiện ta có tập nghiệm $ T=\left[\dfrac{5}{2};\,\,+\infty\right)$.}
\end{ex}

\begin{ex}%Câu 7
Trong không gian $Oxyz$ , mặt phẳng nào sau đây nhận $\overrightarrow{n}=\left(1;2;3\right)$ làm vectơ pháp tuyến?
\choice
{$x+2y+3=0$}
{\True $x+2y+3z=0$}
{$y+2z+3=0$}
{$x+2z+3=0$}
\loigiai{
Mặt phẳng $x+2y+3z=0$ có vectơ pháp tuyến là $\overrightarrow{n}=\left(1;2;3\right)$ .}
\end{ex}

\begin{ex}%Câu 8
Cho hình chóp $ S.ABCD$ có đáy $ ABCD$ là hình chữ nhật tâm $ I$ và cạnh bên $ SA$ vuông góc với đáy. Khẳng định nào sau đây đúng?
\choice
{\True $\left(SCD\right)\perp\left(SAD\right)$}
{$\left(SBC\right)\perp\left(SIA\right)$}
{$\left(SDC\right)\perp\left(SAI\right)$}
{$\left(SBD\right)\perp\left(SAC\right)$}
\loigiai{
% {\color{red}HÌNH Ở ĐÂY}\\
Ta có: $CD\perp AD$ (Vì $ ABCD$ là hình chữ nhật) và $ SA\perp\left(ABCD\right)\Rightarrow SA\perp CD$\\
Mặt khác: $ SA\cap AD=A$ và $ SA,AD\subset\left(SAD\right)$$\Rightarrow CD\perp\left(SAD\right)$\\
Mà $CD\subset\left(SCD\right)$ nên $\left(SCD\right)\perp\left(SAD\right)$ .}
\end{ex}

\begin{ex}%Câu 9
Nghiệm của phương trình $\left(\dfrac{1}{5}\right)^{x^2-2x-3}=5^{x+1}$ là
\choice
{\True $ x=-1;\,x=2$}
{Vô nghiệm}
{$ x=1;\,x=2$}
{$ x=1;\,x=-2$}
\loigiai{
Phương trình đã cho tương đương $5^{-x^2+2x+3}=5^{x+1}\Leftrightarrow-x^2+x+2=0\Leftrightarrow\left[\begin{aligned}
& x=-1\\ 
& x=2.\\ 
\end{aligned}\right.$\\
Vậy phương trình có nghiệm $ x=-1;\,x=2$.}
\end{ex}

\begin{ex}%Câu 10
Cho cấp số cộng $\left(u_n\right)$ có $u_1=2$, $u_2=6$. Công sai của cấp số cộng bằng
\choice
{$8$}
{$-4$}
{3}
{\True 4}
\loigiai{
Vì $u_2=6\Leftrightarrow{u_1}+d=6\Leftrightarrow 2+d=6\Leftrightarrow d=4$ .}
\end{ex}

\begin{ex}%Câu 11
Cho hình lăng trụ tam giác $ ABC.A'B'C'$. Đặt $\overrightarrow{AA'}=\overrightarrow{a},\overrightarrow{AB}=\overrightarrow{b},\overrightarrow{AC}=\overrightarrow{c}$. Khi đó biểu diễn $\overrightarrow{BC'}$ theo các vectơ $\overrightarrow{a},\overrightarrow{b},\overrightarrow{c}$\\
\centerline{
\begin{tikzpicture}[scale=1, font=\footnotesize, line join=round, line cap=round, >=stealth]
    \def\ac{4} % cạnh AC
    \def\ab{2} % cạnh AB
    \def\h{4} % chiều cao
    \def\gocA{50} % góc A của đáy
    \path
    (0,0) coordinate (A)
    (\ac,0) coordinate (C)
    (-\gocA:\ab) coordinate (B)
    ($(A)+(80:\h)$) coordinate (A')
    ($(B)-(A)+(A')$) coordinate (B')
    ($(C)-(A)+(A')$) coordinate (C');
    \draw (A')--(A)--(B)--(C)--(C')--(A')--(B')--(C') (B)--(B');
    \draw[dashed] (A)--(C);
    \foreach \x/\g in {A/180,B/-90,C/0,A'/180,C'/0,B'/-40}\fill[red] (\x) circle (1pt)+(\g:3mm) node[black]{$ \x $};
\end{tikzpicture}
}
\choice
{$\overrightarrow{BC'}=-\overrightarrow{a}+\overrightarrow{b}+\overrightarrow{c}$}
{\True $\overrightarrow{BC'}=\overrightarrow{a}-\overrightarrow{b}+\overrightarrow{c}$}
{$\overrightarrow{BC'}=\overrightarrow{a}+\overrightarrow{b}+\overrightarrow{c}$}
{$\overrightarrow{BC'}=\overrightarrow{a}+\overrightarrow{b}-\overrightarrow{c}$}
\loigiai{
Do $ ABC.A'B'C'$ là hình lăng trụ nên $\overrightarrow{A'C'}=\overrightarrow{AC}$ nên ta có:\\
$\overrightarrow{BC'}=\overrightarrow{AC'}-\overrightarrow{AB}=\overrightarrow{AA'}+\overrightarrow{A'C'}-\overrightarrow{AB}=\overrightarrow{a}-\overrightarrow{b}+\overrightarrow{c}$}
\end{ex}

\begin{ex}%Câu 12
Cho hàm số $ y=f(x)$ có bảng biến thiên như sau :\\
\centerline{
    \begin{tikzpicture}
\tkzTabInit[espcl=2.5,lgt=1.5,nocadre=false]
{$x$/0.7,$f'(x)$/0.7,$f(x)$/2.1}
{$-\infty$,$0$,$2$,$+\infty$}
\tkzTabLine{,+,0,-,0,+,}
\tkzTabVar{-/$-\infty$,+/$4$,-/$0$,+/$+\infty$}
\end{tikzpicture}
}
Hàm số đã cho đồng biến trên khoảng nào dưới đây?
\choice
{$\left(-4\,;\,1\right)$}
{$\left(0\,;+\infty\right)$}
{\True $\left(-\infty ;\,0\,\right)$}
{$\left(0\,;2\right)$}
\loigiai{
Hàm số đã cho đồng biến trên khoảng $\left(-\infty ;\,0\,\right)$.}
\end{ex}
\Closesolutionfile{ans}
\TNTF
\Opensolutionfile{ans}[ans/ans-5-TF]

\begin{ex}%Câu 13
Cho hàm số $f(x)=\cos 2x+2x+1$ . 
\choiceTF
{\True $ f\left(\dfrac{\pi}{2}\right)=\pi $}
{Đạo hàm của hàm số đã cho là $f'(x)=2\sin 2x+2$}
{\True Nghiệm của phương trình $f'(x)=0$ trên đoạn $\left[-\dfrac{\pi}{2};\pi\right]$ là $ x=\dfrac{\pi}{4}$}
{Tổng giá trị lớn nhất và giá trị nhỏ nhất của hàm số trên đoạn $\left[-\dfrac{\pi}{2};\pi\right]$ bằng $ 2\pi $}
\loigiai{

a) Đúng: $ f\left(\dfrac{\pi}{2}\right)=\cos\left(2.\dfrac{\pi}{2}\right)+2.\dfrac{\pi}{2}+1=\pi $\\
b) Sai: Đạo hàm $f'(x)=-2\sin 2x+2$\\
c) Đúng: Phương trình $f'(x)=0\Leftrightarrow-2\sin 2x+2=0\Leftrightarrow\sin 2x=1\Leftrightarrow x=\dfrac{\pi}{4}+k\pi ,\,k\in\mathbb{Z}$\\
Vì $x\in\left[-\dfrac{\pi}{2};\pi\right]$ nên $x=\dfrac{\pi}{4}$\\
d) Sai: Ta có $f\left(-\dfrac{\pi}{2}\right)=-\pi $ ; $f\left(\dfrac{\pi}{4}\right)=\dfrac{\pi}{2}+1$ ; $f\left(\pi\right)=2\pi+2$ .\\
Vậy $\underset{\left[-\dfrac{\pi}{2};\,\pi\right]}{\max}\,f(x)=f\left(-\dfrac{\pi}{2}\right)=-\pi $ và $\underset{\left[-\dfrac{\pi}{2};\pi\right]}{\min}\,f(x)=f\left(\pi\right)=2\pi+2$\\
Khi đó tổng giá trị lớn nhất và nhỏ nhất của hàm số trên đoạn $\left[-\dfrac{\pi}{2};\pi\right]$ bằng $\pi+2$.}
\end{ex}

\begin{ex}%Câu 14
Các nhà kinh tế sử dụng đường cong Lorenz để minh họa sự phân phối thu nhập trong một quốc gia. Gọi $x$ là đại diện cho phần trăm số gia đình trong một quốc gia và $y$ là phần trăm tổng thu nhập, mô hình $y=x$ sẽ đại diện cho một quốc gia mà các gia đình có thu nhập như nhau. Đường cong Lorenz $y=f(x)$ , biểu thị sự phân phối thu nhập thực tế. Diện tích giữa hai mô hình này, với $0\le x\le 100$ , biểu thị “sự bất bình đẳng về thu nhập” của một quốc gia. Năm $2005$ , đường cong Lorenz của Hoa Kỳ có thể được mô hình hóa bởi hàm số:
$$y=\left(0,00061x^2+0,0218x+1,723\right)^2,0\le x\le 100$$
Trong đó $x$ được tính từ các gia đình nghèo nhất đến giàu có nhất\\
\textit{(Theo R.Larson, Brief Calculus: An Applied Approach, 8th edition, Cengage Learning, 2009)}
\choiceTF
{\True Tính theo thứ tự từ các gia đình nghèo nhất đến giàu nhất, tổng thu nhập thực tế của $60\%$ các gia đình đầu tiên chiếm chưa đến $30\%$ so với tổng thu nhập của toàn bộ các gia đình}
{Nếu sắp xếp các gia đình theo thứ tự từ nghèo nhất đến giàu nhất, rồi chia thành $10$ nhóm bằng nhau từ $1$ đến $10$ , tổng thu nhập của các gia đình trong nhóm $3$ chiếm khoảng $8,56\%$ tổng thu nhập của toàn bộ các gia đình}
{Sự bất bình đẳng về thu nhập của Hoa Kì năm $2005$ được xác định bởi công thức:\\
$\displaystyle\int\limits_0^{100}{\left[x-\left(0,00061x^2+0,0218x+1,723\right)^2\right]\mathrm{\,d}x}$}
{\True Sự bất bình đẳng về thu nhập của Hoa Kỳ năm $2005$ đã vượt quá $2000$}
\loigiai{

a) Đúng: Tính theo thứ tự từ các gia đình nghèo nhất đến giàu nhất, tổng thu nhập của $60\%$ các gia đình của đầu tiên chiếm tỷ lệ trong tổng thu nhập là: $f\left(60\right)=27,321529(\%)$ .\\
b) Sai: Nếu sắp xếp các gia đình theo thứ tự từ nghèo nhất đến giàu nhất, rồi chia thành $10$ nhóm bằng nhau từ $1$ đến $10$ , tổng thu nhập của các gia đình trong nhóm $3$ chiếm khoảng $8,56\%$ tổng thu nhập của toàn bộ các gia đình.\\
Nếu sắp xếp các gia đình theo thứ tự từ nghèo đến giàu, rồi chia thành $10$ nhóm bằng nhau, mỗi nhóm chiếm $10\%$ số gia đình của Hoa Kỳ.\\
Tổng thu nhập của $30\%$ số gia đình (là các gia đình thuộc nhóm $1,2,3$) chiếm tỷ lệ trong tổng thu nhập của tất cả các gia đình là: $f\left(30\right)=8,561476\,\,\left(\%\right)$ .\\
Tổng thu nhập của $20\%$ số gia đình (là các gia đình thuộc nhóm $1,2$) chiếm tỷ lệ trong tổng thu nhập của tất cả các gia đình là: $f\left(20\right)=5,774409\,\,\,\left(\%\right)$ .\\
Tỷ lệ của tổng thu nhập các gia đình nhóm thứ $3$ so với toàn bộ các gia đình là:\\
$f\left(30\right)-f\left(20\right)=2,787067(\%)$ .\\
c) Sai: Sự bất bình đẳng về thu nhập của Hoa Kì năm $2005$ được xác định bởi công thức:\\
$\displaystyle\int\limits_0^{100}{\left[x-\left(0,00061x^2+0,0218x+1,723\right)^2\right]\mathrm{\,d}x}$ .\\
Sự bất bình đẳng về thu nhập của Hoa Kì vào năm $2005$ là diện tích hình phẳng $S$ giới hạn bởi hai đồ thị:\\
$\left\{\begin{aligned}
& y=x\\ 
& y=\left(0,00061x^2+0,0218+1,723\right)^2\\ 
& x=0;x=100\\ 
\end{aligned}\right.$ $\Rightarrow S=\displaystyle\int\limits_0^{100}{\left|\left(0,00061x^2+0,0218x+1,723\right)^2-x\right|\mathrm{\,d}x}$ .\\
Cách 1:\\
Sử dụng máy tính cầm tay, ta thấy phương trình $\left(0,00061x^2+0,0218x+1,723\right)^2-x=0$ có hai ngiệm $x=a\,;x=b\,\,\left(a<b\right)$ thuộc $\left[0\,;100\right]$ .\\
Xét dấu biểu thức $g(x)=\left(0,00061x^2+0,0218x+1,723\right)^2-x$ ta suy ra:\\ $S=\displaystyle\int\limits_0^{100}{\left| g(x)\right|\mathrm{\,d}x}=\displaystyle\int\limits_0^a{\left| g(x)\right|\mathrm{\,d}x}+\displaystyle\int\limits_a^b{\left| g(x)\right|\mathrm{\,d}x}+\displaystyle\int\limits_b^{100}{\left| g(x)\right|\mathrm{\,d}x}$ .\\
$=\left|\displaystyle\int\limits_0^a{g(x)\mathrm{\,d}x}\right|+\left|\displaystyle\int\limits_a^b{g(x)\mathrm{\,d}x}\right|+\left|\displaystyle\int\limits_b^{100}{g(x)\mathrm{\,d}x}\right|$ $=\displaystyle\int\limits_0^a{g(x)\mathrm{\,d}x}-\displaystyle\int\limits_a^b{g(x)\mathrm{\,d}x}+\displaystyle\int\limits_b^{100}{g(x)\mathrm{\,d}x}$ .\\
Cách 2:\\
Sử dụng máy tính cầm tay ta được: $S=\displaystyle\int\limits_0^{100}{\left|\left(0,00061x^2+0,0218x+1,723\right)^2-x\right|\mathrm{\,d}x}\approx 2068,9$ .\\
Kiểm tra phép tính của đề bài, ta có: $\displaystyle\int\limits_0^{100}{\left[x-\left(0,00061x^2+0,0218x+1,723\right)^2\right]\mathrm{\,d}x}=2059,3131$ .\\
d) Đúng: Sự bất bình đẳng về thu nhập của Hoa Kỳ năm $2005$ đã vượt quá $2000$ .\\
Sự bất bình đẳng thu nhập của Hoa Kỳ năm $2005$ là:\\
$S=\displaystyle\int\limits_0^{100}{\left|\left(0,00061x^2+0,0218x+1,723\right)^2-x\right|\mathrm{\,d}x}\approx 2068,9$ .}
\end{ex}

\begin{ex}%Câu 15
Một công ty tham gia đấu thầu 2 dự án. Khả năng thắng thầu dự án 1 là 60\% và dự án 2 là 50\%. Khả năng thắng thầu cả hai dự án là 40\%. Gọi $ A$ và $ B$ lần lượt là biến cố công ty thắng thầu dự án 1 và dự án 2.
\choiceTF
{$ A$ và $ B$ là hai biến cố độc lập}
{\True Khả năng công ty thắng thầu đúng 1 dự án là 30\%}
{Xác suất công ty thắng thầu dự án 2 biết công ty đã thắng thầu dự án 1 là $\dfrac{1}{2}$}
{Xác suất công ty không thắng thầu dự án 2, biết công ty đã không thắng thầu dự án 1 là $\dfrac{1}{4}$}
\loigiai{
Ta có $P(A)=0,6\Rightarrow P\left(\bar{A}\right)=0,4;\quad P(B)=0,5\Rightarrow P\left(\bar{B}\right)=0,5$ và $P\left(AB\right)=0,4$ 

a) Sai: Vì $P\left(AB\right)\ne P(A).P(B)$ nên $A,B$ không độc lập .\\
b) Đúng: Do $ A\overline{B}$ và $\overline{A}B$ là hai biến cố xung khắc nên xác suất công ty thắng thầu đúng 1 dự án là: $P\left(A\bar{B}\right)+P\left(\bar{A}B\right)=P(A)-P\left(AB\right)+P(B)-P\left(AB\right)=0,6-0,4+0,5-0,4=0,3$\\
c) Sai: Xác suất công ty thắng thầu dự án 2 biết công ty đã thắng thầu dự án 1 là:\\
$P\left(B|A\right)=\dfrac{P\left(BA\right)}{P(A)}=\dfrac{0,4}{0,6}=\dfrac{2}{3}$ .\\
d) Sai: Xác suất công ty thắng thầu dự án 2, biết công ty đã không thắng thầu dự án 1 là:\\
$P\left(B|\bar{A}\right)=\dfrac{P\left(B\bar{A}\right)}{P\left(\bar{A}\right)}=\dfrac{P(B)-P\left(AB\right)}{P\left(\bar{A}\right)}=\dfrac{0,5-0,4}{0,4}=\dfrac{1}{4}$ .\\
Vậy xác suất công ty không thắng thầu dự án 2, biết công ty đã không thắng thầu dự án 1 là\\
$ P\left(\overline{B}\left|\overline{A}\right.\right)=1-P\left(B\left|\overline{A}\right.\right)=1-\dfrac{1}{4}=\dfrac{3}{4}$.}
\end{ex}

\begin{ex}%Câu 16
Trong không gian $Oxyz$ (đơn vị trên mỗi trục tính theo kilômét), một trạm thu phát sóng điện thoại di động được đặt ở vị trí $I\left(1;\,3;\,7\right)$ . Trạm thu phát sóng đó được thiết kế với bán kính phủ sóng là 3 km.\\
\centerline{\includegraphics[width=.4\textwidth]{images/5.16}}
\choiceTF
{Phương trình mặt cầu $(S)$ để mô tả ranh giới bên ngoài của vùng phù sóng trong không gian là $\left(x+1\right)^2+\left(y+3\right)^2+\left(z+7\right)^2=9$}
{\True Nếu người dùng điện thoại ở vị trí điểm $A\left(2;\,2;\,7\right)$ thì có thể sử dụng dịch vụ của trạm thu phát sóng đó}
{\True Nếu người dùng điện thoại ở vị trí có toạ độ $B\left(5;\,6;\,7\right)$ thì không thể sử dụng dịch vụ của trạm thu phát sóng đó}
{\True Tính theo đường chim bay, khoảng cách lớn nhất để một người ở vị trí có toạ độ $B\left(5;\,6;\,7\right)$ di chuyển được tới vùng phủ sóng theo đơn vị km là $8$ km}
\loigiai{
a) Sai: Phương trình mặt cầu $(S)$ tâm $I\left(1;\,3;\,7\right)$ bán kính 3km mô tả ranh giới bên ngoài của vùng phủ sóng trong không gian là $\left(x-1\right)^2+\left(y-3\right)^2+\left(z-7\right)^2=9$ .\\
b) Đúng: Ta có: $IA=\sqrt{\left(2-1\right)^2+\left(2-3\right)^2+\left(7-7\right)^2}=\sqrt{2}<3$ nên điểm $A$ nằm trong mặt cầu. Vì điểm $A$ nằm trong mặt cầu nên người dùng điện thoại ở vị trí có toạ độ $A\left(2;\,2;\,7\right)$ có thể sử dụng dịch vụ của trạm thu phát sóng đó.\\
c) Đúng: Ta có: $IB=\sqrt{\left(5-1\right)^2+\left(6-3\right)^2+\left(7-7\right)^2}=5>3$ nên điểm $B$ nằm ngoài mặt cầu. Vậy người dùng điện thoại ở vị trí có toạ độ $B\left(5;\,6;\,7\right)$ không thể sử dụng dịch vụ của trạm thu phát sóng đó.\\
d) Đúng: Ta có: $\overrightarrow{IB}\left(4;\,3;\,0\right);$ $IB=\sqrt{\left(5-1\right)^2+\left(6-3\right)^2+\left(7-7\right)^2}=5>3$ nên điểm $B$ nằm ngoài mặt cầu.\\
Phương trình đường thẳng $BI$ dạng: $\left\{\begin{aligned}
& x=1+4t\\ 
& y=3+3t\\ 
& z=7\\ 
\end{aligned}\right.$ .\\
Gọi mặt cầu $(S)\cap BI\equiv E$ suy ra tọa độ $E$ là nghiệm của hệ\\
$\left\{\begin{aligned}
& x=1+4t\\ 
& y=3+3t\\ 
& z=7\\ 
&{\left(x-1\right)^2}+\left(y-3\right)^2+\left(z-7\right)^2=9\\ 
\end{aligned}\right.\Leftrightarrow\left[\begin{aligned}
&\left\{\begin{aligned}
& t=\dfrac{3}{5}\\ 
& x=\dfrac{17}{5}\\ 
& y=\dfrac{24}{5}\\ 
& z=7\\ 
\end{aligned}\right.\Rightarrow E\left(\dfrac{17}{5};\,\dfrac{24}{5};7\right)\Rightarrow EB\approx 1,7\\ 
&\left\{\begin{aligned}
& t=-\dfrac{3}{5}\\ 
& x=-\dfrac{7}{5}\\ 
& y=\dfrac{6}{5}\\ 
& z=7\\ 
\end{aligned}\right.\Rightarrow E\left(-\dfrac{7}{5};\,\dfrac{6}{5};7\right)\Rightarrow EB=8\\ 
\end{aligned}\right.$\\
Vậy khoảng cách lớn nhất để một người ở vị trí có toạ độ $B\left(5;\,6;\,7\right)$ di chuyển được tới vùng phủ sóng theo đơn vị kilomet là $8$ km.}
\end{ex}
\Closesolutionfile{ans}
\TNSA
\Opensolutionfile{ans}[ans/ans-5-SA]
\begin{ex}%Câu 17
Cho hình hộp chữ nhật $ ABCD.A'B'C'D'$ có $ ABCD$ là hình vuông cạnh $ 2$ và góc giữa hai mặt phẳng $\left(AC'D'\right)$ và $\left(ABCD\right)$ bằng $30^{\text{o}}$. Tính khoảng cách giữa hai đường thẳng $ AD'$ và $A'B$. (Kết quả làm tròn đến chữ số thập phân thứ hai sau dấu phẩy)
\shortans{0,89}
\loigiai{
% {\color{red}HÌNH Ở ĐÂY}\\
Vì $\left(ABCD\right)\parallel\left(A'B'C'D'\right)$ nên góc giữa $\left(AC'D'\right)$ và mặt phẳng $\left(ABCD\right)$ bằng góc giữa $\left(AC'D'\right)$ và mặt phẳng $\left(A'B'C'D'\right)$.\\
Ta có $\heva{&\left(AC'D'\right)\cap\left(A'B'C'D'\right)=C'D'\\ &AD'\bot C'D' \\& A'D'\bot C'D'}$$\Rightarrow\left(\left(AC'D'\right);\left(A'B'C'D'\right)\right)=\left(AD';\,A'D'\right)=\widehat{AD'A'}=30^\circ $.\\
Trong tam giác vuông $ AA'D'$ có $ AA'=A'D'.\tan 30^\circ=\dfrac{2\sqrt{3}}{3}$.\\
Vì $A'B\parallel CD'$ nên $A'B\parallel\left(ACD'\right)$ suy ra $ d\left(A'B;\,AD'\right)=d\left(A'B;\,\left(ACD'\right)\right)=d\left(A';\,\left(ACD'\right)\right)$\\
Mặt khác $A'B\cap\left(ACD'\right)=I$ là trung điểm của $A'D$ nên $ d\left(A';\,\left(ACD'\right)\right)=d\left(D;\,\left(ACD'\right)\right)$.\\
Tại $ D$ ta có ba tia $ DA,\,DC,\,DD'$ đôi một vuông góc nên:\\
$\dfrac{1}{d^2\left(D;\,\left(ACD'\right)\right)}=\dfrac{1}{D{A^2}}+\dfrac{1}{D{C^2}}+\dfrac{1}{D{D'^2}}$$=\dfrac{1}{4}+\dfrac{1}{4}+\dfrac{3}{4}=\dfrac{5}{4}$\\
Suy ra $ d\left(D;\,\left(ACD'\right)\right)=\dfrac{2\sqrt{5}}{5}$ hay $ d\left(A'B;\,AD'\right)=\dfrac{2\sqrt{5}}{5}\approx 0,89$.}
\end{ex}

\begin{ex}%Câu 18
Công ty sản xuất "SmartTech" đang lên kế hoạch sản xuất hai loại sản phẩm A và B. Mỗi sản phẩm A đem lại lợi nhuận 800.000 đồng và mỗi sản phẩm B đem lại lợi nhuận 500.000 đồng. Tuy nhiên, việc sản xuất mỗi sản phẩm đòi hỏi nguyên vật liệu và công nhân khác nhau:\\
• Để sản xuất một sản phẩm A công ty cần sử dụng 2 kg nguyên vật liệu và 3 giờ lao động.\\
• Để sản xuất một sản phẩm B công ty cần sử dụng 1 kg nguyên vật liệu và 4 giờ lao động.\\
Hiện tại, công ty có sẵn 100 kg nguyên vật liệu và có thể sử dụng tối đa 180 giờ lao động. Công ty cần sản xuất $m$ sản phẩm A và $ n$ sản phẩm B để tối đa hóa lợi nhuận, đồng thời thỏa mãn các điều kiện về nguyên vật liệu và giờ lao động. Khi đó tổng $m+n$ bẳng bao nhiêu?
\shortans{56}
\loigiai{
Gọi $x$ là số lượng sản phẩm A và $y$ là số lượng sản phẩm B (Điều kiện: $x\ge 0;y\ge 0$)\\
Điều kiện về nguyên vật liệu: $2x+y\le 100$ .\\
Điều kiện về giờ lao động: $3x+4y\le 180$ .\\
Ta có hệ bất phương trình: $\left\{\begin{aligned}
& x\ge 0\\ 
& y\ge 0\\ 
& 2x+y\le 100\\ 
& 3x+4y\le 180\\ 
\end{aligned}\right.$\\
Miền nghiệm của hệ bất phương trình là miền không bị gạch sọc như hình vẽ dưới đây:\\
{\color{red}HÌNH Ở ĐÂY}\\
Ta có miền nghiệm của hệ bất phương trình là miền trong của tứ giác $OEFG$\\
Trong đó, $E\left(0;45\right),F(44;12),G\left(50;0\right)$ .\\
Tối ưu hóa lợi nhuận: Ta sẽ tính toán lợi nhuận $P=800.000x+500.000y$ tại các đỉnh của miền nghiệm ta được:\\
Tại $E\left(0;45\right)$ : $P=22.500.000$ ; $F\left(44;12\right)$ : $P=41.200.000$ ; $G\left(50;0\right)$ : $P=40.000.000$\\
Vậy để tối đa hóa lợi nhuận thì nhà máy cần sản xuất 44 sản phẩm A và 12 sản phẩm B.\\
Khi đó tổng số là $44+12=56$ sản phẩm hay $ m+n=56$.}
\end{ex}

\begin{ex}%Câu 19
Hệ thống định vị toàn cầu GPS (Global Positioning System) là một hệ thống cho phép xác định vị trí của một vật thể trong không gian. Trong cùng một thời điểm vị trí của một điểm $ M$ trong không gian sẽ được xác định bởi bốn vệ tinh cho trước nhờ các bộ thu phát tín hiệu đặt trên các vệ tinh. Giả sử trong không gian $ Oxyz$, tỉ lệ dài trên các trục là $10$ km tính cho một đơn vị tỉ lệ trên mỗi trục, cho bốn vệ tinh có tọa độ lần lượt là $ A\left(-1;2;-1\right),B\left(1;4;0\right),C\left(3;0;9\right),D\left(7;10;-1\right)$ đang tiến hành theo dõi vật thể $ M$ (coi là một chất điểm). Các vệ tinh dùng sóng điện từ có tần số 1MHz . Ở một thời điểm cả 4 vệ tinh bắn tín hiệu về $ M$ thì sau nhận được tín hiệu trả về ngay sau đó trong những khoảng thời gian là $t_A=0,4$ m/s, $t_B=0,2 $m/s, $ t_c=\dfrac{2}{3}$ m/s, $t_D=0,4$ m/s. Tính khoảng cách từ $ M$ đến $ O$ (biết rằng vận tốc của sóng điện từ bằng vận tốc ánh sáng $ c=3.10^8$ m/s và kết quả tính được làm tròn đến hàng phần trăm).
\shortans{6,78}
\loigiai{
Từ vệ tinh $ A$ sẽ phát tín hiệu đến điểm $ M$ và sẽ thu tín hiệu quay về $ A$\\
Từ đó ta có thể suy ra được: $ 2AM=c.t_A\Rightarrow AM=\dfrac{c.t_A}{2}$\\
Do đó $ AM=\dfrac{c.t_A}{2}=\dfrac{3.10^8.0,4.10^{-3}}{2}=60\,000$m$=60$km và ta quy đổi trong hệ trục $ Oxyz$ thì ta có $ AM=6$.\\
Tương tự: $ BM=3$, $ CM=10$, $ DM=6$.\\
Tại thời điểm bốn vị tinh bắn tín hiệu về điểm $ M$. Khi đó ta có hệ phương trình sau:\\
$\Leftrightarrow\left\{\begin{matrix}
{\left(a+1\right)^2}+\left(b-2\right)^2+\left(c+1\right)^2=36\,\,\,\,\,\,\,\,\,\,\,\,\,\,(1)\\
{\left(a-1\right)^2}+\left(b-4\right)^2+c^2=9\,\,\,\,\,\,\,\,\,\,\,\,\,\,\,\,\,\,\,\,\,\,\,\,\,\,\,\,\,(2)\\
{\left(a-3\right)^2}+b^2+\left(c-9\right)^2=100\,\,\,\,\,\,\,\,\,\,\,\,\,\,\,\,\,\,\,\,\,\,\,(3)\\
{\left(a-7\right)^2}+\left(b-10\right)^2+\left(c+1\right)^2=36\,\,\,\,\,\,\,\,\,\,\,(4)\\
\end{matrix}\right.\Leftrightarrow\left\{\begin{aligned}
&(2)-(1):-4a-4b-2c=-38\\ 
&(3)-(1):-8a+4b-20c=-20\\ 
&(4)-(1):-14a-14b=-144\\ 
\end{aligned}\right.$\\
Giải hệ phương trình này, ta tìm được $ a~=~3,b=6,c=1\Rightarrow M\left(3;\,6;\,1\right)$\\
Vậy khoảng cách từ $ M$ đến $ O$ là: $\sqrt{\left(3-0\right)^2+\left(6-0\right)^2+\left(1-0\right)^2}=\sqrt{46}\approx 6,78$.}
\end{ex}

\begin{ex}%Câu 20
Một biển quảng cáo có dạng hình vuông $ ABCD$ cạnh $ AB=4\text{m}$. Trên tấm biển đó có các đường tròn tâm $ A$ và đường tròn tâm $ B$ cùng bán kính $ R=4\text{m}$, hai đường tròn cắt nhau như hình vẽ. Chi phí để sơn phần gạch chéo là $ 150\, 000$ đồng/$\text{m}^2$, chi phí sơn phần màu đen là $ 100\,000$ đồng/ $\text{m}^2$ và chi phí để sơn phần còn lại là $ 250\,000$ đồng/$\text{m}^2$\\
\centerline{\includegraphics[width=.25\textwidth]{images/5.20}}\\
Hỏi số tiền để sơn biển quảng cáo theo cách trên (Đơn vị: triệu đồng và kết quả làm tròn đến chữ số thập phân thứ hai sau dấu phẩy)?
\shortans{2,2}
\loigiai{
Gọi $ I$ là giao điểm của 2 cung tròn $\overset\frown{AC};\overset\frown{BD}$ . Chọn gốc toạ độ $ A\left(0;0\right)$$\Rightarrow B\left(4,0\right)$\\
Xét cung tròn có phương trình $ y=\sqrt{16-x^2}$\\
Phần diện tích gạch chéo $ S=2.\displaystyle\int\limits_2^4\sqrt{16-x^2}\mathrm{\,d}x=16\left.\left(x+\dfrac{1}{2}\sin 2x\right)\right|_{\dfrac{\pi}{6}}^{\dfrac{\pi}{2}}=\dfrac{16\pi}{3}-4\sqrt{3}$\\
Phần diện tích màu đen: $ 2.\left(\dfrac{1}{4}\pi{4^2}-\dfrac{16\pi}{3}+4\sqrt{3}\right)=\dfrac{-8\pi}{3}+8\sqrt{3}$\\
Phần diện tích còn lại: $ 16-\left(\dfrac{16\pi}{3}-4\sqrt{3}+\dfrac{-8\pi}{3}+8\sqrt{3}\right)=16-\dfrac{8\pi}{3}-4\sqrt{3}$\\
Số tiền để sơn biển quảng cáo:\\
$\left(\dfrac{16\pi}{3}-4\sqrt{3}\right).150\text{000}\,\text{+}\left(\dfrac{-8\pi}{3}+8\sqrt{3}\right).100\text{000}+\left(16-\dfrac{8\pi}{3}-4\sqrt{3}\right).250\text{000}\,\approx 2,2$ triệu đồng.}
\end{ex}

\begin{ex}%Câu 21
Anh An thành lập một công ty sản xuất in ấn Sách Giáo Khoa chương trình "Chân trời sáng tạo". Nhằm tạo điều kiện cho các nhà sách tiêu thụ giá hợp lí, đơn giá mỗi bộ sách ban đầu được biểu diễn theo hàm $ p(x)=200-3x$ (nghìn đồng) với $ x$ là số lượng từng bộ sách bán ra và tổng chi phí sản xuất được biểu diễn theo hàm $ C(x)=75+\left(80+T\right)x-x^2$ (nghìn đồng) với mọi $ x$ thỏa $ 0\le x\le 40$, trong đó $ T$ (nghìn đồng) là mức thuế giá trị gia tăng VAT phải đóng trên từng số lượng bộ sách sản xuất ra mà công ty anh An phải chi trả. Xem như công ty anh An sản xuất đều đặn trong điều kiện lí tưởng, khi lợi nhuận của công ty đạt giá trị cao nhất thì tổng mức thuế phải chi trả cũng đồng thời cao nhất. Khi đó mức thuế của mỗi bộ sách mà công ty phải trả là bao nhiêu (đơn vị: nghìn đồng)?
\shortans{60}
\loigiai{
Trước hết ta có hàm chi phí sản xuất là:$ C(x)=75+\left(80+T\right)x-x^2$\\
Doanh thu của công ty anh An biểu diễn theo hàm $ R(x)=x.p(x)=x\left(200-3x\right)$\\
Lợi nhuận mà công ty anh An có được là:$ P(x)=R(x)-C(x)=-2x^2+\left(120-T\right)x-75$\\
Do cần xác định số lượng bộ sách bán ra đề lợi nhuận là cao nhất nên ta có:\\
$P'(x)=0\Leftrightarrow-4x^2+\left(120-T\right)=0\Leftrightarrow x=30-\dfrac{T}{4}$\\
Khi đó với thuế mỗi bộ sách là $ T$ thì tồng mức thuế công ty phải trả là $ G(T)=T\left(30-\dfrac{T}{4}\right)$\\
Khi lợi nhuận của công ty đạt giá trị cao nhất thì tổng mức thuế phải chi trả cũng đồng thời cao nhất nên ta suy ra $G'(T)=0\Leftrightarrow\dfrac{T}{2}-30=0\Leftrightarrow T=60$ (nghìn đồng).}
\end{ex}

\begin{ex}%Câu 22
Bạn Tuấn hằng ngày ăn sáng bằng xôi hoặc bún. Nếu hôm nay bạn ăn sáng bằng xôi thì xác suất để hôm sau bạn ăn sáng bằng bún là $ 0,7$. Xét một tuần mà thứ ba bạn ăn sáng bằng xôi. Biết xác suất để thứ năm tuần đó, bạn Tuấn ăn sáng bằng bún là $ 0,63$. Hỏi nếu hôm nay bạn ăn sáng bằng bún thì xác suất để hôm sau bạn ăn sáng bằng xôi là bao nhiêu?
\shortans{0,4}
\loigiai{
Giả sử nếu hôm nay bạn ăn sáng bằng bún thì xác suất để hôm sau bạn ăn sáng bằng xôi là $ x$ $\left(x<1\right)$.\\
Gọi $ A$ là biến cố “Thứ tư, bạn Tuấn ăn sáng bằng bún”,\\
$ B$ là biến cố “Thứ năm, bạn Tuấn ăn sáng bằng bún”, khi đó $ P(B)=0,63$\\
Ta có thứ ba bạn Tuấn ăn sáng bằng xôi nên $ P(A)=0,7$, $ P\left(\overline{A}\right)=1-0,7=0,3$\\
Vì nếu hôm nay bạn ăn sáng bằng bún thì xác suất để hôm sau bạn ăn sáng bằng xôi là $ x$ và ăn sáng bằng bún là $ 1-x$ hay $ P\left(B|A\right)=1-x$.\\
Ta có $ P\left(B|\overline{A}\right)=0,7$\\
Theo công thức xác suất toàn phần: $ P(B)=P(A).P\left(B|A\right)+P\left(\overline{A}\right).P\left(B|\overline{A}\right)$\\
$\Rightarrow 0,63=0,7.\left(1-x\right)+0,3.0,7$$\Rightarrow x=0,4$\\
Vậy nếu hôm nay bạn ăn sáng bằng bún thì xác suất để hôm sau bạn ăn sáng bằng xôi là $ 0,4$.
}
\end{ex}
\Closesolutionfile{ans}
\Closesolutionfile{ansbook}
\inputansbox{6,2,3}{ans/ans-4-T,ans/ans-4-TF,ans/ans-4-SA}
% \setcounter{deso}{5}
% \begin{name}
	{\tenchude}
	{\tendethi}
	{\tentruong}
	{\thoigian}
	\end{name}
\TN
\Opensolutionfile{ans}[ans/de1-phanI]
\begin{ex}%Câu 1
	Cho cấp số nhân $(u_n)$ với $u_1=2$ và $u_4=16$. Công bội của cấp số nhân đã cho bằng
	\choice
	{$q=4$}
	{\True $q=2$}
	{$q=-2$}
	{$q=-4$}
	\loigiai{
		Ta có $\dfrac{u_4}{u_1}=\dfrac{16}{2}=8=q^3$ nên $q=2$.}
\end{ex}
\begin{ex}%Câu 2
	Cho khối lăng trụ đứng có cạnh bên bằng $5$, đáy là hình vuông có cạnh bằng $4$. Hỏi thể tích của khối lăng trụ bằng bao nhiêu?
	\choice
	{$100$}
	{$20$}
	{$64$}
	{\True $80$}
	\loigiai{
		Ta có thể tích khối lăng trụ bằng $4^2\cdot 5=80$.}
\end{ex}
\textbf{\textit{Sử dụng thông tin dưới đây để trả lời câu \ref{câu 3-đề 1} và câu \ref{câu 4-đề 1}}}\\[0.5em]
Cho hàm số đa thức bậc ba $y=f(x)$ có bảng biến thiên như hình vẽ dưới đây
\begin{center}
	\begin{tikzpicture}
		\tkzTabInit[nocadre=false,lgt=1.2,espcl=2.5,deltacl=0.6]
		{$x$ /0.6, $y'$ /0.6, $y$ /2}
		{$-\infty$,$-1$,$1$,$+\infty$}
		\tkzTabLine{,+,$0$,-,$0$,+,}
		\tkzTabVar{-/$-\infty$,+/$2$,-/$-2$,+/$+\infty$}
	\end{tikzpicture}
\end{center}
\begin{ex}%Câu 3
	\label{câu 3-đề 1}
	Giá trị cực đại của hàm số $y=f(x)$ bằng
	\choice
	{$y_{\text{CĐ}}=-2$}
	{\True $y_{\text{CĐ}}=2$}
	{$y_{\text{CĐ}}=-1$}
	{$y_{\text{CĐ}}=1$}
	\loigiai{
		Dựa vào bảng biến thiên, ta có $y_{\text{CĐ}}=2$.}
\end{ex}
\begin{ex}%Câu 4
	\label{câu 4-đề 1}
	Hàm số nào dưới đây có bảng biến thiên như hình vẽ trên?\vspace{3pt}
	\choice
	{$y=\dfrac{x+1}{x-1}$}
	{$y=-x^3+3x$}
	{$y=x^3+3x$}
	{\True $y=x^3-3x$}
	\loigiai{
		Hàm số có bảng biến thiên như trên có tập xác định là $\mathbb{R}$, đạo hàm có hai nghiệm là $x=1$ và $x=-1$ nên ta loại phương án $\circled{A}$ và $\circled{C}$.\\
		Lại có, $f(x)$ tiến về $+\infty$ khi $x\to+\infty$ nên ta chọn đáp án $\circled{D}$.}
\end{ex}
\begin{ex}%Câu 5
	Tập nghiệm của bất phương trình $3^{-x}\ge\dfrac{1}{27}$ là
	\choice
	{$(-\infty;-3)$}
	{$[3;+\infty)$}
	{$[-3;+\infty)$}
	{\True $(-\infty;3]$}
	\loigiai{
		Ta có $3^{-x}\geq\dfrac{1}{27}\Leftrightarrow -x\geq\log_3\left(\dfrac{1}{27}\right)=-3\Leftrightarrow x\leq 3$.}
\end{ex}
\begin{ex}%Câu 6
	Trong không gian $Oxyz$, cho mặt cầu $(S)\colon x^2+y^2+z^2-4x+2y-2z-3=0$. Tìm tọa độ tâm $I$ và bán kính $R$ của $(S)$.
	\choice
	{\True $I(2;-1;1)$ và $R=3$}
	{$I(-2;1;-1)$ và $R=3$}
	{$I(2;-1;1)$ và $R=9$}
	{$I(-2;1;-1)$ và $R=9$}
	\loigiai{
		Ta có $x^2+y^2+z^2-4x+2y-2z-3=0\Leftrightarrow (x-2)^2+(y+1)^2+(z-1)^2=9$.\\
		Do đó, tâm $I(2;-1;1)$ và $R=3$.}
\end{ex}
\begin{ex}%Câu 7
	Mệnh đề nào \textbf{sai} trong các mệnh đề sau?\vspace{3pt}
	\choice[0.5em]
	{$\displaystyle\int \dfrac{1}{\sin^2 x} \mathrm{\,d}x=-\cot x+C$}
	{$\displaystyle\int \cos x \mathrm{\,d}x=\sin x+C$}
	{$\displaystyle\int \dfrac{1}{\cos^2 x} \mathrm{\,d}x=\tan x+C$}
	{\True $\displaystyle\int \sin x \mathrm{\,d}x=\cos x+C$}
	\loigiai{
		Ta có $\displaystyle\int \sin x \mathrm{\,d}x=-\cos x+C$.}
\end{ex}
\begin{ex}%Câu 8
	Với $a$ là số thực dương tùy ý, $\log_{\sqrt{3}}\left(9a^3\right)$ bằng\vspace{3pt}
	\choice
	{\True $4+6\log_3 a$}
	{$1+\dfrac{3}{2}\log_3 a$}
	{$4-6\log_3 a$}
	{$1-\dfrac{3}{2}\log_3 a$}
	\loigiai{
		Ta có $\log_{\sqrt{3}}\left(9a^3\right)=2\cdot\log_3\left(9a^3\right)=2\cdot\left(2+3\log_3 a\right)=4+6\log_3 a$.}
\end{ex}
\renewcommand{\baselinestretch}{1.55}
\begin{ex}%Câu 9
	\immini[thm]
	{
		Cho hình chóp $S.ABCD$ có đáy là hình bình hành (như hình vẽ minh họa). Hãy chọn khẳng định đúng trong các khẳng định sau.\vspace{3pt}
		\choice[0.3em]
		{\True $\overrightarrow{SA}+\overrightarrow{SC}=\overrightarrow{SB}+\overrightarrow{SD}$}
		{$\overrightarrow{SA}+\overrightarrow{AB}=\overrightarrow{SD}+\overrightarrow{DC}$}
		{$\overrightarrow{SA}+\overrightarrow{AD}=\overrightarrow{SB}+\overrightarrow{BC}$}
		{$\overrightarrow{SA}+\overrightarrow{SB}=\overrightarrow{SC}+\overrightarrow{SD}$}
	}
	{
		\begin{tikzpicture}[scale=0.6,>=stealth, font=\footnotesize, line join=round, line cap=round]
			\def\a{4}
			\path 	(0:0) coordinate (A)
			++(0:\a) coordinate (D)
			++(-130:\a/2) coordinate (C)
			($(A)+(C)-(D)$) coordinate (B)
			($(A)+(80:\a)$) coordinate (S)
			(intersection of A--C and B--D) coordinate (O);%giao điểm O
			\draw[dashed,thick] 	(B)--(A)--(D)	(A)--(S);
			\draw[thick] 			(B)-- (C)--(D)
			(B)--(S)	(C)--(S)	(D)--(S);
			\foreach \x/\g in {A/135,B/-135,C/-45,D/45,S/90}
			\fill[black] 	(\x) circle (1pt)
			($(\g:3mm)+(\x)$) node {$\x$};	
		\end{tikzpicture}
	}
	\loigiai{
		Ta có $\overrightarrow{SA}+\overrightarrow{SC}=\overrightarrow{SB}+\overrightarrow{SD}\Leftrightarrow \overrightarrow{SA}-\overrightarrow{SB}=\overrightarrow{SD}-\overrightarrow{SC}\Leftrightarrow \overrightarrow{BA}=\overrightarrow{CD}$.\\
		Do $ABCD$ là hình bình hành nên $\overrightarrow{BA}=\overrightarrow{CD}$ là đẳng thức đúng.}
\end{ex}
\vspace{15pt}
\begin{ex}%Câu 10
	\immini[thm]
	{
		Hình thang cong $ABCD$ ở hình vẽ bên có diện tích bằng\vspace{3pt}
		\choice[0.3em]
		{$\displaystyle\int\limits_{1}^{3} \left(\dfrac{3}{x}-x+2\right) \mathrm{\,d}x$}
		{$\displaystyle\int\limits_{1}^{3} \left(\dfrac{3}{x}-x-2\right) \mathrm{\,d}x$}
		{$\displaystyle\int\limits_{1}^{3} \left(\dfrac{3}{x}+x+2\right) \mathrm{\,d}x$}
		{\True $\displaystyle\int\limits_{1}^{3} \left(\dfrac{3}{x}+x-2\right) \mathrm{\,d}x$}
	}
	{
		\begin{tikzpicture}[scale=0.6,>=stealth, font=\footnotesize, line join=round, line cap=round]
			\def\xmin{-1} \def\xmax{4}
			\def\ymin{-2} \def\ymax{4} 
			\draw[->] (\xmin,0)--(\xmax,0) node [below]{$x$};
			\draw[->] (0,\ymin)--(0,\ymax) node [left]{$y$};
			\node at (0,0) [below left]{$O$};
			\draw (3,-1.8)node[]{$y=-x+2$} (1,4.2)node[]{$y=\dfrac{3}{x}$};
			\clip (\xmin+0.1,\ymin+0.1) rectangle (\xmax-0.5,\ymax-0.1);
			\draw[smooth,samples=300,domain=\xmin:\xmax] plot(\x,{-(\x)+2});
			\draw[smooth,samples=300,domain=0.1:\xmax] plot(\x,{3/(\x)});
			\draw[dashed] (1,0)node[below]{$1$}--(1,3)--(0,3)node[left]{$3$} (0,-1)node[left]{$-1$}--(3,-1)--(3,0)node[below right]{$3$}--(3,1)--(0,1)node[left]{$1$};
			\fill (1,1)node[below left]{$A$}circle(1pt);
			\fill (1,3)node[above right]{$B$}circle(1pt);
			\fill (3,1)node[above right]{$C$}circle(1pt);
			\fill (3,-1)node[below]{$D$}circle(1pt);
			\fill[color=gray,opacity=0.4] plot[domain=1:3](\x,{3/(\x)})--plot[domain=3:1](\x,{-(\x)+2})--cycle;
		\end{tikzpicture}
	}
	\loigiai{
		Hình thang cong bị giới hạn bởi các đường  $y=\dfrac{3}{x}$, $y=-x+2$, $x=1$ và $x=3$.\\
		Trong khoảng $(1;3)$ đồ thị hàm số $y=\dfrac{3}{x}$ nằm trên đồ thị hàm số $y=-x+2$ nên diện tích hình thang cong $ABCD$ bằng $\displaystyle\int\limits_{1}^{3} \left(\dfrac{3}{x}+x-2\right) \mathrm{\,d}x$.}
\end{ex}
\begin{ex}%Câu 11
	Trong không gian $Oxyz$, cho mặt phẳng $(P)\colon 2x-y-2z+3=0$. Đường thẳng $\Delta$ đi qua điểm $M(4;1;-3)$ và vuông góc với $(P)$ có phương trình chính tắc là\vspace{4pt}
	\choice[0.3em]
	{$\dfrac{x+4}{2}=\dfrac{y+1}{-1}=\dfrac{z-3}{-2}$}
	{$\dfrac{x-2}{4}=\dfrac{y+1}{1}=\dfrac{z+2}{-3}$}
	{$\dfrac{x+2}{2}=\dfrac{y+2}{1}=\dfrac{z-3}{-2}$}
	{\True $\dfrac{x-4}{2}=\dfrac{y-1}{-1}=\dfrac{z+3}{-2}$}
	\loigiai{
		Đường thẳng $\Delta$ vuông góc với $(P)$ nên có một vectơ chỉ phương là $(2;-1;-2)$.\\
		Phương trình chính tắc của $\Delta$ là $\dfrac{x-4}{2}=\dfrac{y-1}{-1}=\dfrac{z+3}{-2}$.}
\end{ex}
\begin{ex}%Câu 12
	Kết quả khảo sát năng suất (đơn vị: tấn/ha) của một số thửa ruộng được minh họa ở biểu đồ sau
	\begin{center}
		\begin{tikzpicture}[line join=round, line cap=round,>=stealth,xscale=1.2,yscale=0.7,font=\footnotesize,scale=0.8]
			\def\a{1}
			\def\xmax{9}
			\def\ymax{7}
			\tikzset{label style/.style={font=\footnotesize}}
			\draw[->] (0,0)--(\xmax,0) node[below] {\text{Năng suất (tấn/ha)}};
			\draw[->] (0,0)--(0,\ymax) node[left] {\text{Số thửa ruộng}};
			\draw (0,0) node [below left] {$O$};
			\draw[thin]
			(0,1)--(\xmax,1)
			(0,2)--(\xmax,2)
			(0,3)--(\xmax,3)
			(0,4)--(\xmax,4)
			(0,5)--(\xmax,5)
			(0,6)--(\xmax,6)
			;
			\foreach \i in {1,2,3,4,5,6}{
				\draw (0,\i) node[left]{$\i$};
			}
			%		\foreach \i/\j in {0.5*\a/3,1.5*\a/4,2.5*\a/6,3.5*\a/5,4.5*\a/5,5.5*\a/2}{
				%			\draw (\i,\j) node[above]{$\j$};
				%		}
			\foreach \i/\j in {1.5*\a/{$[5{,}5;5{,}7)$},2.5*\a/{$[5{,}7;5{,}9)$},3.5*\a/{$[5{,}9;6{,}1)$},4.5*\a/{$[6{,}1;6{,}3)$},5.5*\a/{$[6{,}3;6{,}5)$},6.5*\a/{$[6{,}5;6{,}7)$}}{
				\draw (\i-0.1,-0.6) node[below,rotate=45]{$\j$};
			}
			\fill[blue!20]
			(\a,0) rectangle (2*\a,3)
			(2*\a,0) rectangle (3*\a,4)
			(3*\a,0) rectangle (4*\a,6)
			(4*\a,0) rectangle (5*\a,5)
			(5*\a,0) rectangle (6*\a,5)
			(6*\a,0) rectangle (7*\a,2)
			;
			\begin{scope}
				\draw
				(\a,3)--(2*\a,3) 
				(2*\a,4)--(3*\a,4)
				(3*\a,6)--(4*\a,6)
				(4*\a,5)--(5*\a,5)
				(5*\a,5)--(6*\a,5)
				(6*\a,2)--(7*\a,2)
				(\a,0)--(\a,3)
				(2*\a,0)--(2*\a,4)
				(3*\a,0)--(3*\a,6)
				(4*\a,0)--(4*\a,6)
				(5*\a,0)--(5*\a,5)
				(6*\a,0)--(6*\a,5)
				(7*\a,0)--(7*\a,2)
				;
				\draw (3.5,\ymax) node[above]{\textbf{Năng suất lúa của một số thửa ruộng}};
			\end{scope}
		\end{tikzpicture}
	\end{center}
	Lập bảng tần số ghép nhóm ta tính được khoảng tứ phân vị của mẫu số liệu trên \textbf{gần bằng} giá trị nào dưới đây?
	\choice
	{$0{,}3$}
	{$0{,}4$}
	{\True $0{,}5$}
	{$0{,}6$}
	\loigiai{
		Bảng tần số ghép nhóm của mẫu số liệu trên.
		\begin{center}
			\begin{tabular}{|c|c|c|c|c|c|c|}
				\hline
				Nhóm &$[5{,}5;5{,}7]$ &$[5{,}7;5{,}9]$  &$[5{,}9;6{,}1]$  &$[6{,}1;6{,}3]$  &$[6{,}3;6{,}5]$  &$[6{,}5;6{,}7]$  \\
				\hline
				Tần số &$3$ &$4$  &$6$  &$5$  &$5$  &$2$  \\
				\hline
			\end{tabular}
		\end{center}
		Mẫu số liệu có $25$ giá trị nên trung vị là giá trị thứ $13$.\\ 
		Do đó, tứ phân vị thứ nhất là trung bình cộng của giá trị thứ $6$ và thứ $7$ nên $Q_1\in[5{,}7;5{,}9]$, tứ phân vị thứ ba là trung bình cộng của giá trị thứ $19$ và $20$ nên $Q_3\in[6{,}3;6{,}5]$.\\
		Ta có $Q_1=5{,}7+\dfrac{\dfrac{25}{4}-3}{4}\cdot0{,}2=5{,}8625$; $Q_3=6{,}3+\dfrac{\dfrac{25\cdot 3}{4}-18}{5}\cdot0{,}2=6{,}33$.\\
		Vậy khoảng tứ phân vị $\Delta_Q=6{,}33-5{,}8625=0{,}4675\approx 0{,}5$.}
\end{ex}
\Closesolutionfile{ans}
%\renewcommand{\baselinestretch}{1.35}
%{\fontfamily{qtm}\fontsize{13pt}{2pt}\selectfont\textbf{PHẦN II. Câu trắc nghiệm đúng sai}. Thí sinh trả lời từ câu 1 đến câu 4. Trong mỗi ý \textbf{a)}, \textbf{b)}, \textbf{c)}, \textbf{d)} ở mỗi câu, thí sinh chọn đúng hoặc sai.}
%\setcounter{ex}{0}% Reset lại số đếm câu hỏi
\TNTF
\Opensolutionfile{ans}[ans/de1-phanII]
\begin{ex}%Câu 1
	Cho hàm số $f(x)=\left(x^2-3x-3\right)\mathrm{e}^x$. Xét tính đúng sai của các mệnh đề sau
	\choiceTF
	{\True Hàm số đã cho xác định với mọi $x\in\mathbb{R}$}
	{Đạo hàm của hàm số đã cho là $f'(x)=\left(x^2+x-6\right)\mathrm{e}^x$}
	{\True Phương trình $f'(x)=0$ có hai nghiệm thực phân biệt}
	{\True Hàm số $f(x)$ nghịch biến trên khoảng $(-2;3)$}
	\loigiai{
		\begin{itemchoice}
			\itemch \textbf{Đúng.}\\
			Hàm số đã cho là tích của hàm đa thức và hàm mũ nên tập xác định là $\mathbb{R}$.
			\itemch \textbf{Sai.}\\
			Ta có $f'(x)=(2x-3)\cdot\mathrm{e}^x+(x^2-3x-3)\cdot\mathrm{e}^x=(x^2-x-6)\mathrm{e}^x$.
			\itemch \textbf{Đúng.}\\
			Ta có $f'(x)=0\Leftrightarrow (x^2-x-6)\mathrm{e}^x=0\Leftrightarrow x^2-x-6=0\Leftrightarrow\hoac{&x=-2 \\&x=3.}$
			\itemch \textbf{Đúng.}\\
			Ta có $f'(x)\leq 0\Leftrightarrow (x^2-x-6)\mathrm{e}^x\leq 0\Leftrightarrow x^2-x-6\leq 0\Leftrightarrow -2\leq x\leq 3$.\\
			Vậy hàm số nghịch biến trên $[-2;3]$ nên cũng nghịch biến trên $(-2;3)$.
	\end{itemchoice}}
\end{ex}
\begin{ex}%Câu 2
	Một nắp bể nước hình chữ nhật $ABCD$ nằm cạnh bờ tường có kích thước $9\text{ dm}\times 12\text{ dm}$ được kéo ra từ mặt sàn, do tác dụng của trọng lực nên nắp bể không thể mở ra được nếu không có người giữ. Người ta dùng một sợi dây dài $15$ dm và kéo căng nối đỉnh $C$ của hình chữ nhật với điểm $M$ nằm phía trên bờ tường sao cho $AM=9$ dm và $AM$ vuông góc với mặt sàn. Chọn hệ trục $Oxyz$ như hình vẽ, khi đó nắp bể mở ra và tạo với mặt sàn một góc $\alpha$ (đơn vị trên mỗi trục tọa độ tính bằng dm). Bỏ qua độ dày của nắp bể.
	\immini{
	Xét tính đúng sai của các mệnh đề sau
	\choiceTF
	{Điểm $M$ thuộc mặt phẳng có phương trình $z=0$}
	{Tọa độ điểm $C$ là $C\left(9\sin\alpha;12;9\cos\alpha\right)$}
	{Góc giữa nắp bể và mặt sàn sau khi kéo lên là $\alpha=60^\circ$}
	{\True Phương trình mặt phẳng chứa nắp bể sau khi kéo lên là $x-\sqrt{3}z=0$}}
	{\begin{tikzpicture}[scale=0.5,>=stealth, font=\footnotesize, line join=round, line cap=round]
			\draw[fill=gray!90] (-6.4,1.5)--(-6,1.3)--(-5.82,-3.49)--(-2.02,-6.02)--(5.8,-1.33)--(6.2,-1.6)--(-2,-6.52)--(-6.2,-3.72)--cycle;
			\draw[fill=gray!40] (-6.4,1.5)--(1.1,5.7)--(1.4,5.44)--(-6,1.3)--cycle;
			\draw[fill=gray!60] (-6,1.3)--(1.4,5.44)--(2,1.2)--(-5.82,-3.49)--cycle;
			\draw (1.4,5.44)--(2,1.2)--(6.2,-1.6);
			\coordinate (A) at (0,0);
			\coordinate (C) at (-1.42,-2.71);
			\coordinate (D) at (-4.3,-2.58);
			\coordinate (B) at ($(A)+(C)-(D)$);
			\draw[fill=black!60,opacity=0.9] (A)--(B)--(C)--(D)--cycle;
			\draw[fill=black!80] (B)--(3.05,-0.28)--(-1.42,-2.97)--(-3.99,-2.85)--(-4.3,-2.58)--(-1.42,-2.71)--cycle;
			\coordinate (E) at (-1.73,-4.82);
			\coordinate (F) at ($(A)+(E)-(D)$);
			\coordinate (M) at (0,3.72);
			\draw (D)--(E) (A)--(F) (E)--(F) (M)node[above right]{$M$}--(C);
			\fill[gray!40] (-5.82,-3.49)--(-2.02,-6.02)--(5.8,-1.33)--(2,1.2)--(A)--(B)--(3.05,-0.28)--(1.44,-1.25)--(F)--(E)--(D)--cycle;
			\draw[->] (0,0)node[above left,xshift=0.1cm,yshift=-0.1cm]{$O$}--(0,6)node[right]{$z$};
			\draw[->] (2,1.2)--(-7,-4.2)node[above]{$y$};
			\draw[->] (0,0)--(5.9,-3.5)node[above]{$x$};
			\draw (A)node[above right,yshift=0.2cm,xshift=-0.1cm]{$A$} (B)node[above]{$B$} ($(A)!0.5!(B)$)node[above,xshift=0.1cm]{$9$ dm} ($(A)!0.5!(D)$)node[above,xshift=-0.25cm]{$12$ dm} (D)node[above]{$D$} (C)node[above left]{$C$} (D)node[xshift=0.7cm,yshift=-0.4cm]{$\alpha$};
			\draw[dashed] (C) to[bend left] (E);
			\draw[dashed] (B) to[bend left] (F);
			\draw[decorate,decoration={markings,mark=between positions 2mm and \pgfdecoratedpathlength-2mm step 2mm with{\draw[black] (-3.5mm,-1.25mm) to[out=0,in=160] (-2mm,-1.25mm) to[out=-20,in=160] (2mm,1.25mm) to[out=-20,in=180] (3.5mm,1.25mm);}}] (M) -- (C);
			\fill (A)circle(3pt);
			\fill (B)circle(3pt);
			\fill (C)circle(3pt);
			\fill (D)circle(3pt);
			\fill (M)circle(3pt);
	\end{tikzpicture}}
	\loigiai{
		\begin{itemchoice}
			\itemch \textbf{Sai.}\\
			Điểm $M\in (Oxz)$ có phương trình $x=0$.
			\itemch \textbf{Sai.}\\
			Tọa độ điểm $C$ là $C(9\cos\alpha;12;9\sin\alpha)$.
			\itemch \textbf{Sai.}\\
			Ta có 
			\begin{align*}
				CM=15\Leftrightarrow CM^2=225&\Leftrightarrow (9\cos\alpha)^2+12^2+(9\sin\alpha-9)^2=225\\
				&\Leftrightarrow 81+12^2+9^2-162\sin\alpha=225\\
				&\Leftrightarrow162\sin\alpha=81\\
				&\Leftrightarrow\sin\alpha=\dfrac{1}{2}\Leftrightarrow\alpha=30^\circ.
			\end{align*}
			\itemch \textbf{Đúng.}\\
			Mặt phẳng chứa nắp bể sau khi kéo lên đi qua $3$ điểm $A(0;0;0)$, $C\left(\dfrac{9\sqrt{3}}{2};12;\dfrac{9}{2}\right)$ và $D(0;12;0)$.\\
			Do đó, vectơ pháp tuyến của mặt phẳng cùng phương với $\left[\overrightarrow{AC},\overrightarrow{AD}\right]$.\\
			Ta có $\overrightarrow{AC}=\left(\dfrac{9\sqrt{3}}{2};12;\dfrac{9}{2}\right)$, $\overrightarrow{AD}=(0;12;0)$ nên $\left[\overrightarrow{AC},\overrightarrow{AD}\right]=(-54;0;54\sqrt{3})$.\\
			Suy ra một vectơ pháp tuyến của mặt phẳng là $\overrightarrow{n}=(1;0;-\sqrt{3})$.\\
			Vậy phương trình mặt phẳng là $x-\sqrt{3}z=0$.
	\end{itemchoice}}
\end{ex}
\begin{ex}%Câu 3
	Trong một ngôi làng có $500$ người thì $240$ người là nam. Thống kê cho thấy rằng, khả năng mắc bệnh hô hấp ở người nam trong làng là $0{,}6\%$ và ở người nữ trong làng là $0{,}35\%$. Giả sử gặp một người trong làng.
	\begin{itemize}
		\item Gọi $A$ là biến cố \lq\lq gặp người mắc bệnh trong làng\rq\rq.
		\item Gọi $B$ là biến cố \lq\lq gặp được nam trong làng\rq\rq.
	\end{itemize}
	Xét tính đúng sai của các mệnh đề sau\vspace{3pt}
	\choiceTF
	{\True $\mathrm{P}(\overline{B})=\dfrac{13}{25}$}
	{Xác suất có điều kiện $\mathrm{P}(A\mid\overline{B})=0{,}006$}
	{Tỉ lệ mắc bệnh hô hấp chung của cả làng là $0{,}42\%$}
	{Giả sử có một người trong làng không mắc bệnh. Xác suất để người đó là nữ bằng $47{,}94\%$}
	\loigiai{
		\begin{itemchoice}
			\itemch \textbf{Đúng.}\\
			Ta có $\mathrm{P}(B)=\dfrac{240}{500}=\dfrac{12}{25}\Rightarrow\mathrm{P}(\overline{B})=1-\dfrac{12}{25}=\dfrac{13}{25}$.
			\itemch \textbf{Sai.}\\
			Ta có $\mathrm{P}(A\mid\overline{B})=0{,}35\%=0{,}0035$.
			\itemch \textbf{Sai.}\\
			Ta có $\mathrm{P}(A)=\mathrm{P}(B)\cdot\mathrm{P}(A\mid B)+\mathrm{P}(\overline{B})\cdot\mathrm{P}(A\mid\overline{B})=\dfrac{12}{25}\cdot 0{,}6\%+\dfrac{13}{25}\cdot 0{,}35\%=0{,}47\%$.
			\itemch \textbf{Sai.}\\
			Ta cần tính $\mathrm{P}(\overline{B}\mid \overline{A})$. Ta có $\mathrm{P}(\overline{B}\mid \overline{A})=\dfrac{\mathrm{P}(\overline{B})\cdot\mathrm{P}(\overline{A}\mid\overline{B})}{\mathrm{P}(\overline{B})\cdot\mathrm{P}(\overline{A}\mid\overline{B})+\mathrm{P}(B)\cdot\mathrm{P}(\overline{A}\mid B)}$.\\
			Lại có $\mathrm{P}(\overline{A}\mid\overline{B})=99{,}65\%$, $\mathrm{P}(\overline{A}\mid B)=99{,}4\%$. Do đó, $\mathrm{P}(\overline{B}\mid\overline{A})=\dfrac{\dfrac{13}{25}\cdot99{,}65\%}{\dfrac{13}{25}\cdot99{,}65\%+\dfrac{12}{25}\cdot 99{,}4\%}\approx 52{,}06\%$.
	\end{itemchoice}}
\end{ex}
\renewcommand{\baselinestretch}{1.55}
\begin{ex}%Câu 4
	Một quần thể vi khuẩn $(A)$ có số lượng cá thể là $P(t)$ sau $t$ phút quan sát được phát hiện thay đổi với tốc độ là $P'(t)=a\mathrm{e}^{0{,}1t}+150\mathrm{e}^{-0{,}03t}$ (vi khuẩn/phút) $(a\in\mathbb{R})$. Biết rằng lúc bắt đầu quan sát, quần thể có $200\,000$ vi khuẩn và đạt tốc độ tăng trưởng là $350$ vi khuẩn/phút. Xét tính đúng sai của các mệnh đề sau
	\choiceTF
	{\True Giá trị của $a=200$}
	{$P(t)=2\,000\mathrm{e}^{0{,}1t}-5\,000\mathrm{e}^{-0{,}03t}+200\,000$}
	{\True Sau $12$ phút số lượng vi khuẩn trong quần thể là $206\,152$ con (làm tròn kết quả đến hàng đơn vị)}
	{Sau $12$ phút, một quần thể vi khuẩn $(B)$ có tốc độ tăng trưởng là $G'(t)=500\mathrm{e}^{0{,}2t}$ (vi khuẩn/phút) bắt đầu cạnh tranh nguồn thức ăn trực tiếp với quần thể $(A)$, một cá thể tại quần thể $(B)$ triệt tiêu một cá thể tại quần thể $(A)$. Sau $5$ phút cạnh tranh quần thể $(A)$ bị triệt tiêu hoàn toàn. Số lượng vi khuẩn của quần thể $(B)$ ở thời điểm bắt đầu cạnh tranh là $191\,967$ con (làm tròn kết quả đến hàng đơn vị)}
	\loigiai{
		\begin{itemchoice}
			\itemch \textbf{Đúng.}\\
			Tại $t=0$, ta có $P'(0)=350\Leftrightarrow a+150 =350\Leftrightarrow a=200$.
			\itemch \textbf{Sai.}\\
			Ta có 
			\begin{align*}
				P(t)=\displaystyle\int P'(t)\mathrm{\,d}t&=\int\left(200\mathrm{e}^{0{,}1t}+150\mathrm{e}^{-0{,}03t}\right)\mathrm{\,d}t\\
				&=\dfrac{200}{0{,}1}\mathrm{e}^{0{,}1t} - \dfrac{150}{0{,}03}\mathrm{e}^{-0{,}03t}+C\\
				&=2\,000\mathrm{e}^{0{,}1t} - 5\,000\mathrm{e}^{-0{,}03t}+C.
			\end{align*}
			Tại $t=0$, ta có $P(0)=200\,000\Leftrightarrow 2\,000-5\,000+ C= 200\,000\Leftrightarrow C=203\,000$.\\
			Vậy $P(t)=2\,000\mathrm{e}^{0{,}1t}-5\,000\mathrm{e}^{-0{,}03t}+203\,000$.
			\itemch \textbf{Đúng.}\\
			Ta có $P(12)=2\,000\mathrm{e}^{0{,}1\cdot12}-5\,000\mathrm{e}^{-0{,}03\cdot 12}+203\,000\approx 206\,152$.
			\itemch \textbf{Sai.}\\
			Sau $5$ phút từ khi quần thể $(B)$ xuất hiện, số lượng vi khuẩn của quần thể $(A)$ là $P(17)\approx 210\, 945$ con.\\
			Ta có số lượng vi khuẩn của quần thể $(B)$ là $G(t)=\displaystyle\int G'(t)\mathrm{\,d}t=2\,500\mathrm{e}^{0{,}2t}+C$.\\
			Vì một cá thể tại quần thể $(B)$ triệt tiêu một cá thể tại quần thể $(A)$ và sau $5$ phút cạnh tranh quần thể $(A)$ bị triệt tiêu hoàn toàn nên $G(5)=P(17)$, tức là
			\[2\,500\mathrm{e}^{0{,}2\cdot5}+C=210\, 945\Leftrightarrow C\approx 204\, 149.\]
			Vậy số lượng vi khuẩn của quần thể $(B)$ ban đầu là $G(0)\approx 206\,649$ vi khuẩn.
	\end{itemchoice}}
\end{ex}
\Closesolutionfile{ans}
%{\fontfamily{qtm}\fontsize{13pt}{2pt}\selectfont\textbf{PHẦN III. Câu trắc nghiệm trả lời ngắn}. Thí sinh trả lời từ câu 1 đến câu 6 và điền đáp án vào ô trống.}
%\setcounter{ex}{0}% Reset lại số đếm câu hỏi
\TNSA
\Opensolutionfile{ans}[ans/de1-phanIII]
\begin{ex}%Câu 1
	% \immini[thm]
	% {
		Cho hình lập phương $ABCD.A'B'C'D'$ có cạnh $a$. Gọi $I$ là trung điểm của cạnh $BD$. Góc giữa hai đường thẳng $A'D$ và $B'I$ bằng bao nhiêu độ?
	% }
	% {
	% 	\begin{tikzpicture}[scale=0.75,>=stealth, font=\footnotesize, line join=round, line cap=round]
	% 		\def\a{3.5}
	% 		\path 	(0:0) coordinate (A)
	% 		++(0:\a) coordinate (D)
	% 		++(-130:\a/2) coordinate (C)
	% 		($(A)+(C)-(D)$) coordinate (B)
	% 		($(A)+(90:\a)$) coordinate (A')
	% 		($(B)+(90:\a)$) coordinate (B')
	% 		($(C)+(90:\a)$) coordinate (C')
	% 		($(D)+(90:\a)$) coordinate (D')
	% 		($(B)!0.5!(D)$) coordinate (I)
	% 		;
	% 		\draw[dashed,thick] 	(B)--(A)--(D)	(A)--(A');
	% 		\draw[thick] 	(C)--(C') 	(D)--(D') 	(B)--(B') 	(B)--(C)--(D) (A')--(B')--(C')--(D')--cycle;
	% 		\foreach \x/\g in {A/180,B/180,C/0,D/0,A'/180,B'/180,C'/0,D'/0}
	% 		\fill[black] 	(\x) circle (1pt)
	% 		($(\g:4mm)+(\x)$) node {$\x$};	
	% 	\end{tikzpicture}
	% }
	\shortans[0]{$30$}
	\loigiai{
		\begin{center}
			\begin{tikzpicture}[scale=0.75,>=stealth, font=\footnotesize, line join=round, line cap=round]
				\def\a{3.5}
				\path 	(0:0) coordinate (A)
				++(0:\a) coordinate (D)
				++(-130:\a/2) coordinate (C)
				($(A)+(C)-(D)$) coordinate (B)
				($(A)+(90:\a)$) coordinate (A')
				($(B)+(90:\a)$) coordinate (B')
				($(C)+(90:\a)$) coordinate (C')
				($(D)+(90:\a)$) coordinate (D')
				($(B)!0.5!(D)$) coordinate (I)
				;
				\draw[dashed,thick] 	(B)--(A)--(D)	(A)--(A') (B)--(D)--(A') (B')--(I);
				\draw[thick] 	(C)--(C') 	(D)--(D') 	(B)--(B')--(C) 	(B)--(C)--(D) (A')--(B')--(C')--(D')--cycle;
				\foreach \x/\g in {A/180,B/180,C/0,D/0,A'/180,B'/180,C'/0,D'/0,I/-90}
				\fill[black] 	(\x) circle (1pt)
				($(\g:4mm)+(\x)$) node {$\x$};	
			\end{tikzpicture}
		\end{center}
		Ta có $A'D\parallel B'C$ nên $(A'D,B'I)=(B'C,B'I)=\widehat{IB'C}$.\\
		Lại có $B'C=a\sqrt{2}$, $IC=\dfrac{a\sqrt{2}}{2}$, $B'I=\sqrt{BI^2+BB'^2}=\sqrt{\dfrac{a^2}{2}+a^2}=\dfrac{a\sqrt{6}}{2}$.\\
		Do đó, $\cos\widehat{IB'C}=\dfrac{B'C^2+B'I^2-IC^2}{2B'I\cdot B'C}=\dfrac{\sqrt{3}}{2}\Rightarrow\widehat{IB'C}=30^\circ$.}
\end{ex}
\vspace{15pt}
\begin{ex}%Câu 2
	\immini[thm]
	{
		Từ kho $D$ xe bưu chính đến lấy thư từ các hộp thư tại $E$, $F$, $G$ và $H$ rồi quay lại kho. Sơ đồ bên hiển thị thời gian xe bưu chính di chuyển giữa các hộp thư (đơn vị: phút). Thời gian ngắn nhất để xe bưu chính thực hiện điều đó là bao nhiêu phút?
	}
	{
		\begin{tikzpicture}[scale=0.81,>=stealth, font=\footnotesize, line join=round, line cap=round]
			\coordinate (H) at (0,0);
			\coordinate (G) at (4,0);
			\coordinate (E) at (2.5,4);
			\coordinate (D) at (-0.5,2);
			\coordinate (F) at (4.6,2.4);
			\draw (E)--(D)--(H)--(G)--(F)--cycle (E)--(H) (E)--(G) (D)--(F) (D)--(G) (H)--(F);
			\foreach \x/\g in {D/180,E/90,F/0,G/-45,H/-150} 
			\fill[black] (\x) circle(2pt) +(\g:4mm) node {$\x$};
			\draw ($(E)!0.5!(D)$)node[above left]{$11$} ($(E)!0.5!(F)$)node[above right]{$7$} ($(D)!0.5!(H)$)node[below left]{$3$} ($(G)!0.5!(F)$)node[below right]{$13$} ($(H)!0.6!(F)$)node[above left,yshift=-0.15cm]{$10$} ($(D)!0.6!(F)$)node[above]{$9$} ($(E)!0.7!(G)$)node[right]{$10$} ($(D)!0.4!(G)$)node[above]{$7$} ($(H)!0.7!(E)$)node[left]{$9$} ($(H)!0.5!(G)$)node[below]{$6$};
		\end{tikzpicture}
	}
	\shortans[0]{$35$}
	\loigiai{
		Xét một lộ trình di chuyển thỏa đề bài, chẳng hạn $D\to E\to F\to G\to H\to D$. Ta nhận thấy trong lộ trình di chuyển này, tại mỗi điểm, xe bưu chính bao giờ cũng sẽ có một tuyến đường vào và có một tuyến đường ra. Chẳng hạn, tại $D$ tuyến đường vào là $H\to D$, tuyến đường ra là $D\to E$.\\
		Điều này cho ta thấy, để chọn một lộ trình thỏa mãn đề bài, ta phải bỏ đi $2$ trong $4$ tuyến đường xuất phát từ mỗi đỉnh của sơ đồ.\\
		Mặt khác, sơ đồ có tất cả $10$ tuyến đường và một lộ trình thỏa mãn chỉ cần $5$ tuyến nên số đường đi phải bỏ đi là $5$.
		Để thời gian di chuyển của xe bưu chính là ngắn nhất thì tổng thời gian trên $5$ tuyến bị bỏ đi phải là lâu nhất. Tuyến đường nhiều thời gian nhất thỏa mãn lập luận trên là $D\to E\to H\to F\to G\to D$, thời gian thực hiện là $50$.\\
		Mà tổng thời gian trên các tuyến đường là $85$ nên thời gian ngắn nhất để xe bưu chính thực hiện là $85-50=35$ phút.}
\end{ex}
\vspace{15pt}
\begin{ex}%Câu 3
	\immini[thm]
	{
		Kiến trúc sư thiết kế một khu sinh hoạt cộng đồng có dạng hình vuông $ABCD$ có độ dài đường chéo $AC=120$ m. Trong đó, phần được tô màu đậm là sân chơi, phần còn lại để trồng hoa. Mỗi phần trồng hoa có đường biên cong là một phần của đường parabol với đỉnh thuộc một trục đối xứng của hình vuông, khoảng cách từ đỉnh đó đến đỉnh tương ứng của hình vuông bằng $40$ m và\break $AM=MN=NB$ (xem hình minh họa). Diện tích của phần sân chơi là bao nhiêu mét vuông? (làm tròn kết quả đến hàng đơn vị).
	}
	{
		\begin{tikzpicture}[scale=0.7,>=stealth, font=\footnotesize, line join=round, line cap=round]
			\coordinate (A) at (0,3);
			\coordinate (B) at (3,0);
			\coordinate (C) at (0,-3);
			\coordinate (D) at (-3,0);
			\coordinate (M) at (1,2);
			\coordinate (N) at (2,1);
			\draw (A)--(B)--(C)--(D)--cycle;
			\draw[smooth,samples=300,domain=-1:1] plot(\x,{(\x)^2+1});
			\draw[smooth,samples=300,domain=-1:1] plot(\x,{-(\x)^2-1});
			\draw[smooth,samples=300,domain=-1:1] plot({(\x)^2+1},{\x});
			\draw[smooth,samples=300,domain=-1:1] plot({-(\x)^2-1},{\x});
			\foreach \x/\g in {A/90,B/0,C/-90,D/180,M/60,N/60} 
			\fill[black] (\x) circle(2pt) +(\g:4mm) node {$\x$};
			\draw[latex-latex] (A)--(0,1);
			\draw[latex-latex] (B)--(1,0);
			\draw[latex-latex] (C)--(0,-1);
			\draw[latex-latex] (D)--(-1,0);
			\draw (0,2)node[left]{$40$} (0,-2)node[left]{$40$} (2,0)node[above]{$40$} (-2,0)node[above]{$40$};
			\fill[color=gray!80] plot[domain=-1:1](\x,{(\x)^2+1})--plot[domain=1:-1]({(\x)^2+1},{\x})--plot[domain=1:-1](\x,{-(\x)^2-1})--plot[domain=-1:1]({-(\x)^2-1},{\x})--cycle;
		\end{tikzpicture}
	}
	\shortans[0]{$3467$}
	\loigiai{
		\begin{center}
			\begin{tikzpicture}[scale=0.7,>=stealth, font=\footnotesize, line join=round, line cap=round]
				\coordinate (A) at (0,3);
				\coordinate (B) at (3,0);
				\coordinate (C) at (0,-3);
				\coordinate (D) at (-3,0);
				\coordinate (M) at (1,2);
				\coordinate (N) at (2,1);
				\draw (A)--(B)--(C)--(D)--cycle;
				\draw[smooth,samples=300,domain=-1:1] plot(\x,{(\x)^2+1});
				\draw[smooth,samples=300,domain=-1:1] plot(\x,{-(\x)^2-1});
				\draw[smooth,samples=300,domain=-1:1] plot({(\x)^2+1},{\x});
				\draw[smooth,samples=300,domain=-1:1] plot({-(\x)^2-1},{\x});
				\foreach \x/\g in {A/120,B/-40,C/-120,D/130,M/60,N/60} 
				\fill[black] (\x) circle(2pt) +(\g:4mm) node {$\x$};
				\draw[latex-latex] (A)--(0,1);
				\draw[latex-latex] (B)--(1,0);
				\draw[latex-latex] (C)--(0,-1);
				\draw[latex-latex] (D)--(-1,0);
				\draw (0,2)node[left]{$40$} (0,-2)node[left]{$40$} (2,0)node[below]{$40$} (-2,0)node[above]{$40$};
				\fill[color=gray!80] plot[domain=-1:1](\x,{(\x)^2+1})--plot[domain=1:-1]({(\x)^2+1},{\x})--plot[domain=1:-1](\x,{-(\x)^2-1})--plot[domain=-1:1]({-(\x)^2-1},{\x})--cycle;
				\path 	(0,0) coordinate (O);	
				\draw[-stealth] (-5,0)--(5,0)node[above]{$x$};
				\draw[-stealth] (0,-5)--(0,5)node[left]{$y$};
				\node [below left] at (0,0) {$O$};
				\draw[dashed, thick] (2,1)--(2,0) (1,2)--(0,2); 
			\end{tikzpicture}
		\end{center}
		Ta chọn hệ trục tọa độ $Oxy$ như hình vẽ, trong đó $O$ là giao điểm hai đường chéo $AC$ và $BD$, trục $Ox$ trùng với đường thẳng $BD$, trục $Oy$ trùng với đường thẳng $AC$.\\
		Khi đó, tọa độ các điểm là $A(0;60)$, $B(60;0)$, $C(0;-60)$, $D(-60;0)$.\\
		Vì $AM=MN=NB$ nên $\overrightarrow{AM}=\dfrac{1}{3}\overrightarrow{AB}$. Giả sử $M(x;y)$, khi đó $\heva{&x=\dfrac{1}{3}\cdot 60=20 \\&y-60=\dfrac{1}{3}\cdot(-60)=-20}\Leftrightarrow\heva{&x=20 \\&y=40.}$\\
		Tương tự, ta có $\overrightarrow{AN}=\dfrac{2}{3}\overrightarrow{AB}$ nên $N(40;20)$.\\
		Đường parabol đi qua điểm $M$ có đỉnh nằm trên trục $Oy$, cắt trục $Oy$ tại điểm có tung độ $20$ có phương trình là $y=\dfrac{1}{20}x^2+20$.\\
		Đường parabol đi qua điểm $N$ có đỉnh nằm trên trục $Ox$, cắt trục $Ox$ tại điểm có hoành độ $20$ có phương trình là $x=\dfrac{1}{20}y^2+20$.\\
		Ta gọi $(H_1)$ là hình giới hạn bởi parabol $y=\dfrac{1}{20}x^2+20$, trục $Oy$, đường thẳng $AM$ có phương trình $y=-x+60$.\\
		Ta gọi $(H_2)$ là hình giới hạn bởi parabol $x=\dfrac{1}{20}y^2+20$, trục $Ox$, đường thẳng $BN$ có phương trình $y=-x+60$.\\
		Khi đó, \[S_{H_1}=\displaystyle\int\limits_{20}^{40}\sqrt{20\cdot(y-20)}\mathrm{\,d}y+\int\limits_{40}^{60}(60-y)\mathrm{\,d}y=\dfrac{1\,400}{3};\quad S_{H_2}=\displaystyle\int\limits_{20}^{40}\sqrt{20\cdot(x-20)}\mathrm{\,d}x+\int\limits_{40}^{60}(60-x)\mathrm{\,d}x=\dfrac{1\,400}{3}.\]
		Vậy diện tích một phần tư sân chơi bẳng $S_{AOB}-S_{H_1}-S_{H_2}=\dfrac{1}{2}\cdot 60\cdot 60 -2\cdot\dfrac{1\,400}{3}=\dfrac{2\,600}{3}$ nên diện tích sân chơi là $4\cdot\dfrac{2\,600}{3}\approx 3\,467$ m$^2$.}
\end{ex}
\renewcommand{\baselinestretch}{1.5}
\begin{ex}%Câu 4
	Sách Toán của một đơn vị xuất bản được in tại hai phân xưởng $A$ và $B$ và được vận chuyển về kho sau khi in xong. Xưởng $A$ có nhiệm vụ in $60\%$ tổng số lượng sách, xưởng $B$ sẽ in số lượng sách còn lại. Biết rằng số lượng sách Toán xưởng $A$ và $B$ in đạt yêu cầu về chất lượng và chuyển về kho lần lượt là $95\%$ và $90\%$. Nhân viên kiểm kho chọn ra ngẫu nhiên một cuốn sách Toán để kiểm tra thì thấy cuốn sách này không đạt yêu cầu về chất lượng. Xác suất để cuốn sách Toán đó được in ở xưởng $A$ là bao nhiêu phần trăm? (Làm tròn kết quả đến hàng phần chục).
	
	\shortans[0]{$42{,}9$}
	\loigiai{
		Gọi $M$ là biến cố quyển sách được in ở xưởng $A$, $N$ là biến cố quyển sách in không đạt yêu cầu về chất lượng.\\
		Theo đề, ta cần tính $\mathrm{P}(M\mid N)=\dfrac{\mathrm{P}(M)\cdot\mathrm{P}(N\mid M)}{\mathrm{P}(M)\cdot\mathrm{P}(N\mid M)+\mathrm{P}(\overline{M})\cdot\mathrm{P}(N\mid \overline{M})}$.\\
		Lại có $\mathrm{P}(M)=0{,}6$, $\mathrm{P}(\overline{M})=0{,}4$, $\mathrm{P}(N\mid M)=0{,}05$, $\mathrm{P}(N\mid \overline{M})=0{,}1$.\\
		Suy ra $\mathrm{P}(M\mid N)\approx 42{,}9\%$.}
\end{ex}
\begin{ex}%Câu 5
	\immini[thm]{Một phần của đường chạy của tàu lượn siêu tốc khi gắn hệ trục tọa độ $Oxy$ được mô phỏng ở hình dưới. Biết đường chạy của nó có dạng đồ thị hàm số bậc ba $y=ax^3+bx^2+cx+d$ $(0\le x\le 90)$; tàu lượn xuất phát từ điểm $A$ đồng thời đi qua các điểm $B$, $C$, $D$ (như hình vẽ). Đơn vị mỗi trục là mét, dựa vào đồ thị hình dưới, em hãy tính độ cao lớn nhất (theo đơn vị mét) mà tàu lượn siêu tốc đạt được so với mặt đất (xem $Ox$ là mặt đất). Kết quả làm tròn đến hàng phần mười.}
	{\begin{tikzpicture}[scale=0.7,>=stealth, font=\footnotesize, line join=round, line cap=round]
			\def\a{-1/15} \def\b{13/15} \def\c{-8/3} \def\d{3} % Hệ số
			\def\xmin{-1} \def\xmax{10}
			\def\ymin{-1} \def\ymax{5} 
			\draw[->] (\xmin,0)--(\xmax,0) node [below]{$x$};
			\draw[->] (0,\ymin)--(0,\ymax) node [left]{$y$};
			\node at (0,0) [below left]{$O$};
			\clip (\xmin+0.1,\ymin+0.1) rectangle (\xmax-0.5,\ymax-0.1);
			\draw[smooth,samples=300,domain=0:9] plot(\x,{\a*(\x)^3+\b*(\x)^2+\c*(\x)+\d});
			\draw[dashed] (3,0)node[below]{$30$}--(3,1)node[above]{$B$}--(0,1)node[left]{$10$} (5,0)node[below]{$50$}--(5,3)node[above]{$C$} (8,0)node[below]{$80$}--(8,3)node[above right]{$D$} (0,3)node[left]{$30$} (0,3)node[above right]{$A$}--(8,3);
	\end{tikzpicture}}	
	\shortans[0]{$39{,}9$}
	\loigiai{
		Đồ thị hàm số bậc ba đi qua điểm $A(0;30)$ nên $d=30$. Lại có, đồ thị đi qua điểm $B(30;10)$, $C(50;30)$, $D(80;30)$ nên ta có hệ phương trình sau
		\[\heva{&a\cdot 30^3+b\cdot 30^2+c\cdot30 +30=10 \\&a\cdot 50^3+b\cdot 50^2+c\cdot50 +30=30 \\&a\cdot 80^3+b\cdot 80^2+c\cdot80 +30=30}\Leftrightarrow\heva{&a=-\dfrac{1}{1\,500} \\&b=\dfrac{13}{150} \\&c=-\dfrac{8}{3}.}\]
		Ta có $f(x)=-\dfrac{1}{1\,500}x^3+\dfrac{13}{150}x^2-\dfrac{8}{3}x+30$. Khi đó, $f'(x)=-\dfrac{1}{500}x^2+\dfrac{13}{75}x-\dfrac{8}{3}$.\\
		Phương trình $f'(x)=0\Leftrightarrow -\dfrac{1}{500}x^2+\dfrac{13}{75}x-\dfrac{8}{3}=0\Leftrightarrow\hoac{&x=20 \\&x=\dfrac{200}{3}.}$\\
		Dựa vào đồ thị, ta thấy tàu đạt độ cao lớn nhất so với mặt đất khi đi qua điểm cực đại của đồ thị hàm số $f(x)$. Độ cao lớn nhất khi đó bằng $f\left(\dfrac{200}{3}\right)\approx 39{,}9$ m.}
\end{ex}
\begin{ex}%Câu 6
	\immini[thm]{Một quả bóng hình cầu có bán kính $r$ đang được treo trong một góc tường nhà (hai bờ tường vuông góc), một điểm $B$ cố định nằm trên mép của hai bờ tường và cách mặt đất $80$ cm, sợi dây treo bóng có độ dài $AB=30$ cm và đây cũng là độ dài ngắn nhất nối điểm $B$ với mặt xung quanh của quả bóng.\\
	Biết rằng quả bóng tiếp xúc với hai bên bờ tường và điểm thấp nhất của quả bóng cách mặt đáy $20$ cm. Hỏi đường kính của quả bóng là bao nhiêu centimet (làm tròn kết quả đến hàng đơn vị).
	}
	{\begin{tikzpicture}[scale=1,>=stealth, font=\footnotesize, line join=round, line cap=round]
			\fill[gray!60] (0,0)--(-1.8,-0.52)--(-1.8,3.98)--(0,4.5)--cycle;
			\fill[gray!35] (-1.8,-0.52)--(-2.69,0)--(-2.69,4.5)--(-1.8,3.98)--cycle;
			\fill[gray!35] (0,0)--(0,4.5)--(1.45,3.65)--(1.45,-0.85)--cycle;
			\fill[gray!60] (1.45,-0.85)--(2.36,-0.58)--(2.36,3.92)--(1.45,3.65)--cycle;
			\coordinate (I) at (-0.16,1.58);
			\coordinate (B) at (0,4);
			\coordinate (r) at (-0.69,1.03);
			\coordinate (x) at (-0.16,-0.52);
			\coordinate (L) at (0.29,-0.52);
			%			\tkzInterLC[R](I,B)(I,1.06)\tkzGetPoints{D}{A}
			%			\tkzInterLC[R](I,x)(I,1.06)\tkzGetPoints{E}{F}
			\draw[shading=ball,ball color=gray,opacity=0.8,name path=C1] (I) circle(1.06cm);
			\path[name path=IB] (I)--(B);
			\path[name path=Ix] (I)--(x);
			\path  	[name intersections={of=C1 and IB,by={A}}];	
			\path  	[name intersections={of=C1 and Ix,by={D}}];	
			\fill (I)circle(1.5pt);
			\fill (B)circle(1.5pt);
			\draw[->] (0,0)--(-2.76,-0.79)node[below]{$x$};
			\draw[->] (0,0)--(2.25,-1.32)node[below]{$y$};
			\draw[->] (0,0)--(0,5)node[right]{$z$};
			\draw (A)node[above left,xshift=0.1cm,yshift=-0.1cm]{$A$}--(B)node[left,xshift=0.1cm]{$B$} (0,0)node[right,yshift=0.1cm,xshift=-0.1cm]{$O$} ($(A)!0.5!(B)$)node[left]{$30$ cm} ($(I)!0.5!(r)$)node[above left,xshift=0.2cm]{$r$} (r)node[rotate=30]{$\times$} 
			(x)--(D)
			(L)--($(L)!2!(x)$) ($(I)!0.6!(x)$)node[left]{$20$ cm};
			\draw[->] (I)--(r);
			\draw[dashed] (I)--(A) (D)--(I) (I)--(B);
			\draw pic[draw,angle radius=1mm]{right angle=L--x--I};
			
			%	\foreach \x/\g in {A/90,B/180,I/-45,F/0,x/0,L/0,r/0}
			%			\fill[black] 	(\x) circle (1pt)
			%			($(\g:3mm)+(\x)$) node {$\x$};
	\end{tikzpicture}}	
	
	\shortans[0]{$30$}
	\loigiai{
		Vì $AB$ là độ dài ngắn nhất nối điểm $B(0;0;80)$ với mặt xung quanh của quả bóng nên $AB$ đi qua tâm $I$ của quả bóng.\\
		Thật vậy, với $M$ bất kì nằm trên mặt xung quanh của quả bóng, ta luôn có $BM+MI\geq BI$ hay $BM\geq BI-r$. Do đó, độ dài nhỏ nhất của $BM$ bằng $BI-r$. Điều này xảy ra khi $M\equiv A$, tức là $BA=BI-r$.\\
		Giả sử $H$ là hình chiếu của $I$ trên $(Oyz)$, khi đó, ta có $H(0;r;20+r)$. Do đó, $BH=\sqrt{r^2+(r-60)^2}$.\\
		Tam giác $BIH$ vuông tại $H$ nên theo định lý Py-ta-go, ta có
		\allowdisplaybreaks
		\begin{eqnarray*}
		&&BI^2=BH^2+IH^2\\
		&\Leftrightarrow& (30+r)^2=r^2+(r-60)^2+r^2\\
		&\Leftrightarrow& 2r^2-180r+2700=0\\
		&\Leftrightarrow&\hoac{&r=45-15\sqrt{3}\\& r=45+\sqrt{3} \,(\text{loại}).}
		\end{eqnarray*}
		Vậy đường kính quả bóng là $d=2r=2\cdot (45-\sqrt{3})=90-30\sqrt{3}\approx 38$ cm.
		}
\end{ex}
\Closesolutionfile{ans}
% \begin{name}
	{\tenchude}
	{\tendethi}
	{\tentruong}
	{\thoigian}
	\end{name}
\TN
\Opensolutionfile{ans}[ans/de3-phanI]
\begin{ex}%Câu 1
	Cho hàm số $f(x)=3\cos x$. Khẳng định nào dưới đây đúng?
	\choice
	{$\displaystyle\int f(x) \mathrm{\,d}x=3x\cdot\sin x+C$}
	{$\displaystyle\int f(x) \mathrm{\,d}x=-3\sin x+C$}
	{$\displaystyle\int f(x) \mathrm{\,d}x=3x+\sin x+C$}
	{\True $\displaystyle\int f(x) \mathrm{\,d}x=3\sin x+C$}
	\loigiai{
	Ta có $\displaystyle\int f(x) \mathrm{\,d}x=3\sin x+C$.}
\end{ex}
\begin{ex}%Câu 2
	Cho cấp số cộng $(u_n)$ với $u_1=2$ và công sai $d=3$. Giá trị của $u_5$ bằng
	\choice
	{$162$}
	{\True $14$}
	{$30$}
	{$10$}
	\loigiai{
	Ta có $u_5=u_1+(5-1)\cdot 3=14$.}
\end{ex}
\begin{ex}%Câu 3
	Nghiệm của phương trình $2^{2x+1}=\dfrac{1}{8}$ là
	\choice
	{$x=-1$}
	{$x=2$}
	{$x=1$}
	{\True $x=-2$}
	\loigiai{
	Ta có $2^{2x+1}=\dfrac{1}{8}\Leftrightarrow 2^{2x+1}=2^{-3}\Leftrightarrow 2x+1=-3\Leftrightarrow x=-2$.}
\end{ex}
\begin{ex}%Câu 4
	Cho hình phẳng $(H)$ giới hạn bởi đường cong $y=\sqrt{x+1}$, trục hoành và các đường thẳng $x=-1$, $x=2$. Thể tích $V$ của khối tròn xoay tạo thành khi quay $(H)$ quanh trục hoành được tính bởi công thức nào sau đây?
	\choice[0.3em]
	{$V=\pi\displaystyle\int\limits_{-1}^{2} \sqrt{x^2+1} \mathrm{\,d}x$}
	{$V=\pi^2\displaystyle\int\limits_{-1}^{2} (x+1) \mathrm{\,d}x$}
	{\True $V=\pi\displaystyle\int\limits_{-1}^{2} (x+1) \mathrm{\,d}x$}
	{$V=\pi\displaystyle\int\limits_{-1}^{2} \sqrt{x+1} \mathrm{\,d}x$}
	\loigiai{
	Thể tích khối tròn xoay tính bởi công thức $V=\pi\displaystyle\int\limits_{-1}^{2} (x+1) \mathrm{\,d}x$.}
\end{ex}
\begin{ex}%Câu 5
	Tập xác định của hàm số $y=\ln(\ln x)$ là
	\choice
	{\True $(1;+\infty)$}
	{$(0;+\infty)$}
	{$(\mathrm{e};+\infty)$}
	{$\mathbb{R}$}
	\loigiai{
	Hàm số xác định khi và chỉ khi $\heva{&x>0 \\&\ln x>0}\Leftrightarrow\heva{&x>0 \\& x>1}\Leftrightarrow x>1$.\\
	Vậy tập xác định là $\mathscr{D}=(1;+\infty)$.}
\end{ex}
\textbf{\textit{Sử dụng thông tin dưới đây để trả lời câu \ref{câu 6-đề 3} và câu \ref{câu 7-đề 3}}}\\[0.5em]
Trong không gian với hệ tọa độ $Oxyz$, cho ba điểm $A(1;-2;-1)$, $B(1;0;2)$ và $C(0;2;1)$.
\begin{ex}%Câu 6
	\label{câu 6-đề 3}
	Độ dài vectơ $\overrightarrow{AB}$ bằng
	\choice
	{$5$}
	{$\sqrt{5}$}
	{$\sqrt{14}$}
	{\True $\sqrt{13}$}
	\loigiai{
	Độ dài vectơ $\overrightarrow{AB}$ bằng $\sqrt{(1-1)^2+(0+2)^2+(2+1)^2}=\sqrt{13}$.}
\end{ex}
\begin{ex}%Câu 7
	\label{câu 7-đề 3}
	Mặt phẳng đi qua $A$ và vuông góc với đường thẳng $BC$ có phương trình là
	\choice
	{\True $x-2y+z-4=0$}
	{$x-2y+z+4=0$}
	{$x-2y-z-6=0$}
	{$z-2y-z+4=0$}
	\loigiai{
	Mặt phẳng đi qua $A$ và vuông góc với $BC$ có vectơ pháp tuyến cùng phương với $\overrightarrow{BC}$.\\
	Ta có $\overrightarrow{BC}=(-1;2;-1)$ nên một vectơ pháp tuyến của mặt phẳng là $\overrightarrow{n}=(1;-2;1)$.\\
	Phương trình mặt phẳng là $x-2y+z-4=0$.}
\end{ex}
\begin{ex}%Câu 8
	Thống kê thu nhập theo tháng (đơn vị: triệu đồng) của một nhóm người chạy xe máy Xanh SM được cho trong bảng sau
	\begin{center}
		\begin{tabular}{|c|c|c|c|c|}
			\hline
			Thu nhập (triệu đồng) & $[3;5)$ & $[5;7)$ & $[7;9)$ & $[9;11)$\\
			\hline
			Số người & $5$ & $10$ & $5$ & $2$ \\
			\hline
		\end{tabular}
	\end{center}
	Tứ phân vị thứ ba của mẫu số liệu ghép nhóm trên là 
	\choice
	{\True $7{,}6$}
	{$8{,}1$}
	{$7{,}5$}
	{$8{,}2$}
	\loigiai{
	Mẫu số liệu có $22$ giá trị nên trung vị là trung bình cộng của số đứng thứ $11$ và $12$. Do đó, tứ phân vị thứ ba là số đứng thứ $17$.\\
	Dựa vào bảng số liệu, ta thấy $Q_3\in[7;9)$, do đó $Q_3=7+\dfrac{\dfrac{22\cdot3}{4}-15}{5}\cdot 2=7{,}6$.}
\end{ex}
\renewcommand{\baselinestretch}{1.52}
\begin{ex}%Câu 9
	Hàm số $f(x)=\sqrt{x^2-4}$ đồng biến trên khoảng nào dưới đây?
	\choice
	{$(-\infty;-2)$}
	{\True $(2;+\infty)$}
	{$(0;+\infty)$}
	{$(-2;2)$}
	\loigiai{
	Tập xác định của hàm số là $\mathscr{D}=(-\infty;-2)\cup(2;+\infty)$.\\
	Ta có $f'(x)=\dfrac{x}{\sqrt{x^2-4}}$, $\forall x\in \mathscr{D}$.\\
	Khi đó, $f'(x)\geq 0\Leftrightarrow x\geq 0$.\\
	Kết hợp với tập xác định của hàm số, ta kết luận hàm số đồng biến trên $(2;+\infty)$.}
\end{ex}
\begin{ex}%Câu 10
	Cho hình chóp tứ giác $S.ABCD$, gọi $M$ và $N$ lần lượt là trung điểm của $SA$ và $SC$. Mặt phẳng nào sau đây song song với đường thẳng $MN$?
	\choice
	{$(SAB)$}
	{$(SCD)$}
	{$(SBC)$}
	{\True $(ABCD)$}
	\loigiai{
	\begin{center}
		\begin{tikzpicture}[scale=0.8,>=stealth, font=\footnotesize, line join=round, line cap=round]
		\def\a{4}
		\path 	(0:0) coordinate (A)
		++(0:\a) coordinate (D)
		++(-130:\a/2) coordinate (C)
		++(-165:2*\a/3) coordinate (B)
		($(A)+(70:\a)$) coordinate (S)
		(intersection of A--C and B--D) coordinate (O)
		($(S)!0.5!(A)$) coordinate (M)
		($(S)!0.5!(C)$) coordinate (N);%giao điểm O
		\draw[dashed,thick] 	(C)--(A)--(D) (M)--(N);
		\draw[thick] 			(A)--(B)--(C)--(D)
		(A)--(S)	(B)--(S)	(C)--(S)	(D)--(S);
		\foreach \x/\g in {A/180,B/-135,C/-45,D/0,S/90,M/180,N/0}
		\fill[black] 	(\x) circle (1pt)
		($(\g:3mm)+(\x)$) node {$\x$};	
		\end{tikzpicture}
		\end{center}
	Ta có $MN$ là đường trung bình của tam giác $SAC$ nên $MN\parallel AC$, do đó $MN\parallel (ABCD)$.}
\end{ex}
\begin{ex}%Câu 11
	\immini[thm]
	{
		Cho hình lăng trụ tam giác $ABC.A'B'C'$ (minh họa như hình bên). Đặt $\overrightarrow{AA'}=\overrightarrow{a}$, $\overrightarrow{AB}=\overrightarrow{b}$, $\overrightarrow{AC}=\overrightarrow{c}$. Phát biểu nào sau đây đúng?
	\choice
	{$\overrightarrow{BC'}=-\overrightarrow{a}+\overrightarrow{b}+\overrightarrow{c}$}
	{\True $\overrightarrow{BC'}=\overrightarrow{a}-\overrightarrow{b}+\overrightarrow{c}$}
	{$\overrightarrow{BC'}=\overrightarrow{a}+\overrightarrow{b}+\overrightarrow{c}$}
	{$\overrightarrow{BC'}=\overrightarrow{a}+\overrightarrow{b}-\overrightarrow{c}$}
	}
	{
		\begin{tikzpicture}[scale=0.9,>=stealth, font=\footnotesize, line join=round, line cap=round]
		\def\a{3}
	\def\h{3}
	\path 	(0:0) coordinate (A)
			++(0:\a) coordinate (C)
			(-30:\a/2) coordinate (B)
			($(B)!0.5!(C)$) coordinate (M)
			($(A)!2/3!(M)$) coordinate (G)			
			($(G)+(90:\h)$) coordinate (A')
			($(A')+(C)-(A)$) coordinate (C')
			($(C')+(B)-(C)$) coordinate (B'); 
	\draw[dashed,thick] (A)--(C);
	\draw[thick] (C)--(C') 	(B)--(B') 	(A)--(A') 
				(A)--(B)--(C) (A')--(B')--(C')--cycle;
	\foreach \x/\g in {A/180,B/-45,C/0,A'/180,B'/-45,C'/0}
				\fill[black] 	(\x) circle (1pt)
				($(\g:4mm)+(\x)$) node {$\x$};	
%	\draw pic[draw,angle radius=2mm]{right angle=A'--G--C};%Theo chiều dương	
\end{tikzpicture}
	}
	\loigiai{
	Ta có $\overrightarrow{BC'}=\overrightarrow{AC'}-\overrightarrow{AB}=\overrightarrow{AA'}+\overrightarrow{AC}-\overrightarrow{AB}=\overrightarrow{a}+\overrightarrow{c}-\overrightarrow{b}$.}
\end{ex}
\begin{ex}%Câu 12
	Cho hàm số $f(x)$ liên tục trên $\mathbb{R}$, có bảng xét dấu $f'(x)$ như sau
	\begin{center}
		\begin{tikzpicture}
			\tikzset{double style/.append style={double distance=1.5pt}}
			\tkzTabInit[nocadre=false,lgt=1.2,espcl=2.4,deltacl=0.6]
			{$x$ /0.6,$f'(x)$ /0.6}
			{$-\infty$,$-2$,$0$,$1$,$2$,$+\infty$}
			\tkzTabLine{,+,$0$,-,$0$,-,$0$,+,d,-,}
		\end{tikzpicture}
	\end{center}
	Số điểm cực đại của hàm số $y=f(x)+1$ là
	\choice
	{$4$}
	{\True $2$}
	{$1$}
	{$3$}
	\loigiai{
	Đạo hàm đổi dấu khi đi qua $x=-2$ và $x=2$ và hàm số liên tục trên $\mathbb{R}$ nên $f(x)$ có $2$ điểm cực đại.}
\end{ex}
\Closesolutionfile{ans}
\TNTF
% \setcounter{ex}{0}% Reset lại số đếm câu hỏi
\Opensolutionfile{ans}[ans/de3-phanII]
\begin{ex}%Câu 1
	Cho hàm số $f(x)=\dfrac{x^2-x+2}{x+1}$. Xét tính đúng sai của các mệnh đề sau
	\choiceTF
	{\True Tiệm cận đứng của đồ thị hàm số đã cho là $x=-1$}
	{\True Tiệm cận xiên của đồ thị hàm số đã cho có hệ số góc bằng $1$}
	{\True Tiệm cận xiên của đồ thị hàm số đã cho đi qua điểm có tọa độ là $(0;-2)$}
	{Hai đường tiệm cận của đồ thị hàm số đã cho cùng với hai trục tọa độ $Ox$, $Oy$ tạo thành một đa giác có diện tích bằng $3$}
	\loigiai{
	\begin{itemchoice}
		\itemch \textbf{Đúng.}\\
		Ta có $x+1=0\Leftrightarrow x=-1$ và $x=-1$ không là nghiệm của tử thức nên hàm số có một tiệm cận đứng là $x=-1$.
		\itemch \textbf{Đúng.}\\
		Giả sử tiệm cận xiên của đồ thị hàm số là $y=ax+b$. Khi đó, ta có
		\[a=\lim\limits_{x\to+\infty}\dfrac{x^2-x+2}{x(x+1)}=1.\]
		\itemch \textbf{Đúng.}\\
		Ta có $b=\displaystyle\lim_{x\to+\infty}\left(\dfrac{x^2-x+2}{x+1}-x\right)=\lim_{x\to+\infty}\dfrac{-2x+2}{x+1}=-2$.\\
		Vậy tiệm cận xiên của đồ thị hàm số là $y=x-2$. Đường thẳng này đi qua điểm có tọa độ $(0;-2)$.
		\itemch \textbf{Sai.}
		\begin{center}
			\begin{tikzpicture}[line cap=butt,line join=miter,>=stealth,xscale=0.62,yscale=0.62]
				\tikzset{declare function={xmin=-5;xmax=3;Xkxd=-1;
						ymin=-7;ymax=1;
						a=1; b=-1; c=2; d=1;et=1;
						akxd=1;bkxd=-2;
						f(\x)=(a*(\x)^2+b*\x+c)/(d*(\x)+et);
						h(\x)=akxd*\x+bkxd;
					},
					smooth,samples=50
				}
				\draw[->] (xmin-0.25,0)--(xmax+0.5,0)
				node[shift={(-95:7pt)},font=\normalsize]{$ x $};
				\draw[->] (0,ymin-0.25)--(0,ymax+0.5)
				node[shift={(5:7pt)},font=\normalsize]{$ y $};
				\fill (0,0) node[shift={(45:9pt)},font=\normalsize]{$ O $};
				\foreach \x in {-5, -4, -3, -2, -1, 1, 2, 3}{
					\draw (\x,2pt)--(\x,-2pt) +(0,-9pt) node[font=\scriptsize,fill=white,inner sep=1pt]{$\x$};
				}
				\foreach \y in {-7, -6, -5, -4, -3, -2, -1, 1}{
					\draw (2pt,\y)--(-2pt,\y) +(-3pt,0) node[font=\scriptsize,anchor=east,fill=white,inner sep=1pt]{$\y$};
				}
				\begin{scope}[thick]
					\clip (xmin,ymin) rectangle (xmax,ymax);
					\draw[blue!50!black] (Xkxd,ymin)--(Xkxd,ymax);
					\draw[blue!50!black] plot[domain=xmin:xmax] (\x, {h(\x)});
					\draw[cyan!85!blue] plot[domain=xmin:{Xkxd-0.01}] (\x, {f(\x)});
					\draw[cyan!85!blue] plot[domain={Xkxd+0.01}:xmax] (\x, {f(\x)});
				\end{scope}
				\node [left] at (-1,-3) {$B$}; 
				\node [ above left] at (-1,0) {$A$};
				\node [right] at (0,-2) {$C$};  
			\end{tikzpicture}
		\end{center}
		Tứ giác cần tìm là hình thang vuông $OABC$, ta có $S_{OABC}=\dfrac{(OC+AB)\cdot OA}{2}=\dfrac{5\cdot 2}{2}=5$.
	\end{itemchoice}}
\end{ex}
\begin{ex}%Câu 2
	Một con sư tử đang đuổi theo một con ngựa vằn. Con ngựa vằn nhận ra con sư tử khi con sư tử cách xa nó $40$ m. Từ thời điểm này, con sư tử đuổi con ngựa vằn với tốc độ $v_1(t)=15\mathrm{e}^{-0{,}1t}$ m/s và con ngựa vằn chạy trốn với tốc độ $v_2(t)=20-20\mathrm{e}^{-0{,}1t}$ m/s trên cùng một đường thẳng (với $t$ tính theo giây và $0\le t\le 60$). Xét tính đúng sai của các mệnh đề sau
	\choiceTF
	{Tại thời điểm $t=0$ vận tốc của con ngựa vằn là $20$ m/s}
	{\True Tốc độ của sư tử giảm dần theo thời gian trong khi tốc độ của ngựa vằn tăng dần theo thời gian}
	{Sư tử sẽ ở gần với ngựa vằn nhất khi $v'_1(t)=v'_2(t)$}
	{Sư tử sẽ không bắt được con ngựa vằn và khoảng cách ngắn nhất giữa chúng là $1{,}42$ mét (kết quả làm tròn đến hàng phần trăm)}
	\loigiai{
	\begin{itemchoice}
		\itemch \textbf{Sai.}\\
		Tại $t=0$, ta có $v_2(0)=20-20=0$ m/s.
		\itemch \textbf{Đúng.}\\
		Ta có $v'_1(t)=-0{,}1\cdot 15\mathrm{e}^{-0{,}1t}<0$, $\forall 0\le t\le 60$. Do đó, tốc độ của sử tử giảm dần theo thời gian.\\
		Lại có $v'_2(t)=0{,}1\cdot 20\mathrm{e}^{-0{,}1t}>0$, $\forall 0\le t\le 60$. Do đó, tốc độ của ngựa vằn tăng dần theo thời gian.
		\itemch \textbf{Sai.}\\
		Ta có $v'_1(t)=v'_2(t)\Leftrightarrow 15\mathrm{e}^{-0{,}1t}=-20\mathrm{e}^{-0{,}1t}$. Phương trình này vô nghiệm.
		\itemch \textbf{Sai.}\\
		Chọn hệ quy chiếu với gốc của chuyển động ở vị trí con sư tử bắt đầu đuổi con ngựa vằn.\\
		Gọi $s_1(t)$ là quãng đường con sư tử di chuyển được trong $t$ giây, $s_2(t)$ là quãng đường con ngựa vằn di chuyển được trong $t$ giây.\\
		Ta có $s_1(t)=\displaystyle\int v_1(t)\mathrm{\,d}t=-150\mathrm{e}^{-0{,}1t}+C$.\\
		Tại $t=0$, ta có $s_1(0)=0\Leftrightarrow -150+C=0\Leftrightarrow C=150$. Do đó, $s_2(t)=-150\mathrm{e}^{-0{,}1t}+150$.\\
		Ta có $s_2(t)=\displaystyle\int v_2(t)\mathrm{\,d}t=20t+200\mathrm{e}^{-0{,}1t}+C$.\\
		Tại $t=0$, ta có $s_2(0)=40\Leftrightarrow 200+C=40\Leftrightarrow C=-160$. Do đó, $s_1(t)=20t+200\mathrm{e}^{-0{,}1t}-160$.\\
		Sư tử sẽ ở gần ngựa vằn nhất khi $s_2(t)-s_1(t)$ đạt $\min$. Ta có
		\[s_2(t)-s_1(t)=f(t)=20t+350\mathrm{e}^{-0{,}1t}-310,\forall 0\le t\le 60.\]
		Lại có $f'(t)=20-35\mathrm{e}^{-0{,}1t}$, $\forall 0\le t\le 60$. Khi đó, $f'(t)=0\Leftrightarrow \mathrm{e}^{-0{,}1t}=\dfrac{4}{7}\Leftrightarrow t\approx 5{,}6$ s.\\
		Ta có bảng biến thiên
		\begin{center}
			\begin{tikzpicture}
			\tkzTabInit[nocadre=false,lgt=1.2,espcl=2.5,deltacl=0.6]
			{$t$/0.6,$f'(t) $/0.6,$f(t)$/2}
			{$0$,$5{,}6$,$60$}
			\tkzTabLine{,-,0,+,} %
			\tkzTabVar{+/$40$,-/$1{,}92$, +/$890{,}1$} %dấu mũi tên, + trên, -dưới
		\end{tikzpicture}
		\end{center}
		Vậy khoảng cách ngắn nhất là $1{,}92$ m.
	\end{itemchoice}}
\end{ex}
\renewcommand{\baselinestretch}{1.5}
\begin{ex}%Câu 3
	Một vật dụng bằng sắt đang nằm trên mặt sàn có tay cầm dài $58$ cm nối với một ống trụ dày $4$ cm và có đường kính đáy bằng $30$ cm. Nếu không giữ thì sẽ luôn có một lực làm vật rung động, để vật đứng yên thì người ta đã nối một đoạn dây từ điểm $B$ (là một điểm nằm trên đường tròn chính giữa của ống trụ to) đến điểm $C$ nằm trên bờ tường. Trên hệ trục $Oxyz$, xét gốc tọa độ là điểm gắn ống trụ với bờ tường, bờ tường là mặt phẳng $(Oxz)$, trục $Oy$ là trục của hình trụ, điểm $A$ nằm chính giữa ống trụ to, điểm $B$ có hoành độ âm, cao độ dương và $AB$ tạo với trục $Oz$ một góc $30^\circ$, các số liệu được cho như hình vẽ, đơn vị trên các hệ trục tính theo cm. Biết rằng lực căng $\overrightarrow{T}$ trên đoạn dây $BC$ có độ lớn bằng $500$ N. Xét tính đúng sai của các mệnh đề sau
	\begin{center}
		\begin{tikzpicture}[scale=1,>=stealth, font=\footnotesize, line join=round, line cap=round]
			\fill[gray!20] (-0.58,-0.85)--(-0.58,3.38)--(4.01,5.22)--(4.01,0.98)--cycle;
			\coordinate (O) at (0,0);
			\coordinate (y) at (5.27,-2.17);
			\coordinate (x) at (4.48,1.79);
			\coordinate (A) at ($(O)!0.7!(y)$);
			\coordinate (E) at (2.75,-0.9);
			\path 
			
			;
			\tkzDefLine[parallel=through E](O,y) \tkzGetPoint{d}
			\tkzInterLL(O,x)(E,d)\tkzGetPoint{E'}
			\draw (E)--(E');
			\coordinate (F) at (2.6,-1.3);
			\tkzDefLine[parallel=through F](O,y) \tkzGetPoint{d'}
			\tkzInterLL(O,x)(F,d')\tkzGetPoint{F'}
			\draw (F)--(F');
			\fill[gray!40] (E)--(E')--(E')..controls+(150:0.5) and +(50:0.1)..(F')--(F)--cycle;
			\begin{scope}[rotate=-20]
				\draw[fill=gray!80] (A) ellipse (0.8 cm and 1.38 cm);
				\coordinate (M) at ($(A)+(90:0.8 cm and 1.38 cm)$);
				\coordinate (M') at ($(M)+(-0.3,0)$);
				\coordinate (N) at ($(A)+(-90:0.8 cm and 1.38 cm)$);
				\coordinate (N') at ($(N)+(-0.3,0)$);
				\draw (M)--(M');
				\draw (N)--(N');
				\draw (M') arc(90:270:0.8 cm and 1.38 cm);
				\fill[gray!40] (M)--(M') arc(90:270:0.8 cm and 1.38 cm)--(N)--(N) arc(270:90:0.8 cm and 1.38 cm)--cycle;
			\end{scope}
			\coordinate (B) at (3.05,-0.7);
			\coordinate (C) at (2.58,3.7);
			\coordinate (I) at ($(A)+(0,2)$);
			\fill (B)node[above left]{$B$}circle(2pt);
			\draw (1.64,-2.21)--(A)--(B) (0.26,-1.33)node[below left]{$60$ cm} (5,-1.29)node[below right]{$15$ cm} (4.47,-0.31)--(5.56,-0.76) ($(B)!0.7!(C)$)--(C)--($(C)+(0,0.5)$) (1.27,3.64)node[above]{$35$ cm} (C)--(3.84,4.21) (3.39,2.69)node[right]{$30$ cm} ($(B)!0.5!(C)$)node[left]{$\overrightarrow{T}$} (A)--(I);
			\draw[->,dashed] (-1.88,-0.75)--(4.48,1.79)node[below]{$x$};
			\draw[->,dashed] (O)node[below left,xshift=0.2cm,yshift=-0.1cm]{$O$}--(0,4)node[right]{$z$};
			\draw[->,dashed] (O)--(y)node[below]{$y$};
			\fill (O)circle(2pt);
			\fill (A)node[above right]{$A$}circle(2pt);
			\fill (C)node[above right,yshift=0.1cm]{$C$}circle(2pt);
			\draw[<->] (-1.51,-0.6)--(2.02,-2.06);
			\draw[<->] (4.7,-1.94)--(5.3,-0.65);
			\draw[<->] (0,3.12)--(2.58,4.16);
			\draw[<->] (3.39,4.03)--(3.39,1.36);
			\draw[->] (B)--($(B)!0.7!(C)$);
			\draw pic[draw,,angle radius=4mm]{angle=I--A--B};
			\draw pic["$30^\circ$",-stealth,angle radius=11mm]{angle=I--A--B};
		\end{tikzpicture}
	\end{center}
	\choiceTF
	{\True Hình chiếu của $C$ lên mặt phẳng $(Oxy)$ có tọa độ $(35;0;0)$}
	{Góc giữa đường thẳng $AB$ và mặt phẳng $(Oxy)$ bằng $30^\circ$}
	{Vectơ $\overrightarrow{BC}$ có tọa độ $(a;b;c)$. Khi đó $2a-b=40$}
	{Vectơ lực tác dụng lên đoạn dây $BC$ có hoành độ là $120$ (làm tròn kết quả đến hàng đơn vị theo Newton)}
	\loigiai{
	\begin{itemchoice}
		\itemch \textbf{Đúng.}\\
		Điểm $C$ có tọa độ $(35;0;30)$ nên hình chiếu của $C$ lên mặt phẳng $(Oxy)$ có tọa độ $(35;0;0)$.
		\itemch \textbf{Sai.}\\
		Ta có $(AB;Oz)=30^\circ$ mà $Oz\perp (Oxy)$ nên $(AB;(Oxy))=60^\circ$.
		\itemch \textbf{Sai.}\\
		Vì đường kính đáy bằng $30$ nên bán kính đáy bằng $15$.\\
		Ta có $|x_B|=AB\cdot\cos 60^\circ=7{,}5$. Mà $x_B<0$ nên $x_B=-7{,}5$.\\
		Lại có $|z_B|=AB\cdot \cos 30^\circ=\dfrac{15\sqrt{3}}{2}$ mà $z_B>0$ nên $z_B=\dfrac{15\sqrt{3}}{2}$. Do đó, $B=\left(-7{,}5;60;\dfrac{15\sqrt{3}}{2}\right)$.
		Khi đó, $\overrightarrow{BC}=\left(42{,}5;-60;30-\dfrac{15\sqrt{3}}{2}\right)$. Suy ra $2a-b=2\cdot42{,}5+60=145$.
		\itemch \textbf{Sai.}\\
		Ta có $\left|\overrightarrow{BC}\right|=\sqrt{(42{,}5)^2+60^2+\left(30-\dfrac{15\sqrt{3}}{2}\right)}\approx 75$.\\
		Suy ra $\overrightarrow{T}=\dfrac{500}{75}\overrightarrow{BC}$ nên hoành độ của $\overrightarrow{T}$ bằng $\dfrac{500}{75}\cdot 42{,}5\approx 283$.
	\end{itemchoice}}
\end{ex}
\begin{ex}%Câu 4
	Trong một khu dân cư, tỉ lệ người nghiện thuốc lá và và bị ung thư vòm họng là $15\%$. Có $25\%$ người nghiện thuốc lá nhưng không bị ung thư vòm họng, $50\%$ người không nghiện thuốc lá và không bị ung thư họng và có $10\%$ số người không nghiện thuốc lá nhưng mắc ung thư vòm họng.
	\begin{itemize}
		\item Gọi $A$ là biến cố \lq\lq người đó nghiện thuốc lá\rq\rq.
		\item Gọi $B$ là biến cố \lq\lq người đó bị ung thư vòm họng\rq\rq.
	\end{itemize}
	Xét tính đúng sai của các mệnh đề sau
	\choiceTF
	{$\mathrm{P}(AB)=0{,}25$ và $\mathrm{P}(\overline{A}B)=0{,}15$}
	{$\mathrm{P}(A)=0{,}6$}
	{\True $\mathrm{P}(B\mid A)=0{,}375$}
	{\True Với những dữ liệu thống kê như trên có thể thấy nguy cơ của người nghiện thuốc là mắc ung thư vòm họng gấp $2{,}25$ lần người không nghiện thuốc lá}
	\loigiai{
	\begin{itemchoice}
		\itemch \textbf{Sai.}\\
		Ta có $\mathrm{P}(AB)=0{,}15$, $\mathrm{P}(\overline{A}B)=0{,}1$.
		\itemch \textbf{Sai.}\\
		Ta có $\mathrm{P}(A)=\mathrm{P}(AB)+\mathrm{P}(A\overline{B})$.\\
		Theo đề, ta có $\mathrm{P}(A\overline{B})=0{,}25$, $\mathrm{P}(\overline{A}\cap\overline{B})=0{,}5$, $\mathrm{P}(\overline{A}B)=0{,}1$.\\
		Do đó, $\mathrm{P}(A)=0{,}15+0{,}25=0{,}4$.
		\itemch \textbf{Đúng.}\\
		Ta có $\mathrm{P}(B\mid A)=\dfrac{\mathrm{P}(AB)}{\mathrm{P}(A)}=\dfrac{0{,}15}{0{,}4}=0{,}375$.
		\itemch \textbf{Đúng.}\\
		Ta có $\mathrm{P}(B\mid \overline{A})=\dfrac{\mathrm{P}(B\overline{A})}{\mathrm{P}(\overline{A})}=\dfrac{0{,}1}{0{,}6}=\dfrac{1}{6}$.\\
		Do đó $\mathrm{P}(B\mid A)=2{,}25\cdot\mathrm{P}(B\mid\overline{A})$.
	\end{itemchoice}}
\end{ex}
\Closesolutionfile{ans}
\TNSA
% \setcounter{ex}{0}% Reset lại số đếm câu hỏi
\Opensolutionfile{ans}[ans/de3-phanIII]
\begin{ex}%Câu 1
	Cho hình chóp $S.ABC$ có cạnh bên $SA$ vuông góc với mặt phẳng $(ABC)$ và $AB=1$, $AC=2$, $\widehat{BAC}=60^\circ$. Khoảng cách giữa hai đường thẳng $SA$ và $BC$ bằng bao nhiêu?
	
	\shortans[]{$1$}
	\loigiai{
	\begin{center}
		\begin{tikzpicture}
		\def\a{4} %Khai báo cạnh
		\def\h{4}
		\path 	(0:0) coordinate (A)
		++(0:\a) coordinate (C)
		++(-150:4*\a/5) coordinate (B)
		($(A)+(90:\h)$) coordinate (S)
		($(B)!0.3!(C)$) coordinate (H);
		\draw[thick] 	(A)--(B)--(C)
		(A)--(S)--(H)	(C)--(S)	(B)--(S);
		\draw[dashed,thick] 	(H)--(A)--(C);
		\foreach \x /\goc in {A/180,C/0,B/-135,S/90,H/-45}
		\fill[black] (\x) circle (1.5pt)
		($(\x)+(\goc:3mm)$) node {$\x$};
		\draw pic[draw,angle radius=2mm]{right angle=C--A--S};%Theo chiều dương
		\draw pic[draw,angle radius=2mm]{right angle=A--H--C};
		\end{tikzpicture}
		\end{center}
	Kẻ $AH\perp BC$ ($H\in BC$).\\
	Ta có $SA\perp(ABC)$ nên $SA\perp AH$. Do đó, $AH$ là đoạn vuông góc chung của $SA$ và $BC$.\\
	Suy ra $\mathrm{d}(SA,BC)=AH$.\\
	Lại có $BC=\sqrt{AB^2+AC^2-2AB\cdot AC\cdot \cos \widehat{BAC}}=\sqrt{1+4-2\cdot1\cdot2\cdot\cos 60^\circ}=\sqrt{3}$.\\
	Mà $S_{ABC}=\dfrac{1}{2}AH\cdot BC=\dfrac{1}{2}AB\cdot AC\sin\widehat{BAC}$ nên $AH=\dfrac{AB\cdot AC\sin\widehat{BAC}}{BC}=1$.}
\end{ex}
\begin{ex}%Câu 2
	Trong một vườn cây ăn trái, có ba loại cây: cây cam, cây chanh và cây bưởi. Sau $3$ năm, số cây cam tăng gấp ba lần, số cây chanh tăng gấp hai lần và cây bưởi tăng gấp bốn lần số lượng cây ban đầu. Tổng số cây sau $3$ năm là $330$ cây. Biết rằng ban đầu số lượng cây bưởi bằng trung bình cộng của số lượng cây cam và cây chanh. Sau $3$ năm thu hoạch, tổng số cây cam và cây chanh tăng thêm nhiều hơn $15$ cây so với số cây bưởi tăng thêm. Vậy tổng số cây cam và cây bưởi ban đầu là bao nhiêu?
	
	\shortans[]{$85$}
	\loigiai{
	Gọi $x$ (cây) là số cây cam ban đầu, $y$ (cây) là số cây bưởi ban đầu, $z$ (cây) là số cây chanh ban đầu.\\
	Theo đề, ta có $\heva{&3x+4y+2z=330 \\&y=\dfrac{x+z}{2} \\&2x+z-3y=15}\Leftrightarrow\heva{&x=50 \\&y=35 \\&z=20.}$\\
	Vậy tổng số cây cam và cây bưởi ban đầu là $50+35=85$ cây.}
\end{ex}
\begin{ex}%Câu 3
	Có hai bình như sau: Bình $A$ chứa $5$ bi đỏ, $3$ bi trắng và $8$ bi xanh; bình $B$ chứa $3$ bi đỏ và $5$ bi trắng. Gieo một con xúc xắc ngẫu nhiên: Nếu mặt $3$ hoặc mặt $5$ xuất hiện thì chọn ngẫu nhiên một bi từ bình $B$; các trường hợp khác thì chọn ngẫu nhiên một bi từ bình $A$. Nếu viên bi trắng được chọn ra, hãy tính xác suất để mặt $5$ của con xúc xắc xuất hiện nhiều nhất.
	
	\shortans[]{$0{,}31$}
	\loigiai{
	Gọi $A$ là biến cố \lq\lq Bi được chọn là bi trắng\rq\rq, $B$ là biến cố \lq\lq Xúc xắc xuất hiện mặt 5\rq\rq.\\
	Ta cần tính $\mathrm{P}(B\mid A)$. Theo công thức xác suất Bayes, ta có $\mathrm{P}(B\mid A)=\dfrac{\mathrm{P}(B)\cdot\mathrm{P}(A\mid B)}{\mathrm{P}(A)}$.\\
	Theo đề, ta có $\mathrm{P}(B)=\dfrac{1}{6}$, $\mathrm{P}(A\mid B)=\dfrac{5}{8}$.\\
	Lại có $\mathrm{P}(A)=\mathrm{P}(AB)+\mathrm{P}(A\overline{B})$.\\
	Nếu xúc xắc đổ ra số $3$, khi đó xác suất bốc trúng bi trắng là $\dfrac{5}{8}$.\\
	Nếu xúc xắc đổ ra các số còn lại (ngoại trừ số $5$) thì xác suất bốc trúng bi trắng là $\dfrac{3}{16}$.\\
	Do đó, $\mathrm{P}(A\overline{B})=\dfrac{1}{6}\cdot\dfrac{5}{8}+\dfrac{2}{3}\cdot\dfrac{3}{16}=\dfrac{11}{48}$. Suy ra $\mathrm{P}(A)=\dfrac{1}{6}\cdot\dfrac{5}{8}+\dfrac{11}{48}=\dfrac{1}{3}$.\\
	Vậy $\mathrm{P}(B\mid A)=\dfrac{\dfrac{1}{6}\cdot\dfrac{5}{8}}{\dfrac{1}{3}}=\dfrac{5}{16}=0{,}3125$.
	}
\end{ex}
\begin{ex}%Câu 4
	Cần trục chân đế là kiểu cột quay được sử dụng để phục vụ công việc xếp dỡ hàng hóa chủ yếu ngoài các cảng, bến, bãi (như hình minh họa).
	
	{\centering \begin{tikzpicture}[scale=0.7,>=stealth, font=\footnotesize, line join=round, line cap=round]
			\draw (0,0)node[opacity=0.65]{\includegraphics[scale=0.21]{images/de3-1}};
			\coordinate (K) at (1.85,1.3);
			\coordinate (M) at (-3,2.9);
			\coordinate (O) at (2.75,-3.15);
			\fill (K)node[above right,font=\bfseries]{$K$}circle(3pt);
			\fill (M)node[above,font=\bfseries]{$M$}circle(3pt);
			\draw[->,thick] (O)--($(O)!1.4!(K)$)node[right,font=\bfseries]{$z$};
			\draw[->,thick] (O)--(3.7,-2.85)node[above,font=\bfseries]{$x$};
			\draw[->,thick] (O)--($(O)+(-6.5,0)$)node[above,font=\bfseries]{$y$};
	\end{tikzpicture}\par}\noindent
	Ta chọn hệ trục $Oxyz$ thỏa mãn $(Oxy)$ song song với mặt đất, trục $Ox$ trùng với trục chân đế, trục $Oz$ trùng với trục cần cẩu và trục $Oy$ như hình vẽ. Gọi $M$ là vị trí tại đỉnh cần cẩu, $H$ là hình chiếu của $M$ lên $(Oxy)$. Biết tay cần $KM$ của cần trục dài $50$ m, trục cần $OK$ dài $50$ m, $\left(\overrightarrow{k},\overrightarrow{KM}\right)=60^\circ$; $\left(\overrightarrow{i},\overrightarrow{OH}\right)=45^\circ$. Biết điểm $M$ có tọa độ $M(a;b;c)$ trong hệ tọa độ $Oxyz$ trên, giá trị của $a+b+c$ bằng bao nhiêu? (làm tròn kết quả đến hàng đơn vị).
	
	\shortans[]{$136$}
	\loigiai{
	Theo đề, ta có $c=OK+KM\cdot\cos 60^\circ=\dfrac{50}{2}=75$, $a=x_H$, $b=y_H$.\\
	Vì $\left(\overrightarrow{i},\overrightarrow{OH}\right)=45^\circ$ và $OH=MK\cdot\sin 60^\circ=25\sqrt{3}$ nên $x_H=y_H=\dfrac{OH}{\sqrt{2}}=\dfrac{25\sqrt{3}}{\sqrt{2}}$.\\
	Vậy $M\left(\dfrac{25\sqrt{3}}{\sqrt{2}};\dfrac{25\sqrt{3}}{\sqrt{2}};75\right)$ nên $a+b+c=25\sqrt{6}+75\approx 136$.}
\end{ex}
\begin{ex}%Câu 5
	\immini[thm]
	{
		Hai hình chữ nhật bằng nhau, nội tiếp trong đường tròn tâm $O$, bán kính $r=1$ cm tạo thành một hình chữ thập đối xứng (như hình vẽ bên). Diện tích lớn nhất của hình chữ thập là bao nhiêu cm$^2$? (Kết quả làm tròn đến hàng phần trăm).
		
	\shortans[]{$2{,}47$}
	}
	{
		\begin{tikzpicture}[scale=0.8,>=stealth, font=\footnotesize, line join=round, line cap=round]
			\coordinate (O) at (0,0);
			\coordinate (A) at (-1,2);
			\coordinate (B) at (1,2);
			\coordinate (C) at (1,1);
			\coordinate (D) at (2,1);
			\coordinate (E) at (2,-1);
			\coordinate (F) at (1,-1);
			\coordinate (G) at (1,-2);
			\coordinate (H) at (-1,-2);
			\coordinate (I) at (-1,-1);
			\coordinate (J) at (-2,-1);
			\coordinate (K) at (-2,1);
			\coordinate (L) at (-1,1);
			\draw (O) circle(2.236 cm);
			\draw (A)--(B)--(C)--(D)--(E)--(F)--(G)--(H)--(I)--(J)--(K)--(L)--cycle;
			\fill[color=gray!80,opacity=0.5pt] (A)--(B)--(C)--(D)--(E)--(F)--(G)--(H)--(I)--(J)--(K)--(L)--cycle;
			\draw (O)--(D) ($(O)!0.5!(D)$)node[below]{$r$};
			\fill (O)node[below]{$O$}circle(2pt);
			
		\end{tikzpicture}
	}
	\loigiai{
	\begin{center}
		\begin{tikzpicture}[scale=0.8,>=stealth, font=\footnotesize, line join=round, line cap=round]
		\coordinate (O) at (0,0);
		\coordinate (A) at (-1,2);
		\coordinate (B) at (1,2);
		\coordinate (C) at (1,1);
		\coordinate (D) at (2,1);
		\coordinate (E) at (2,-1);
		\coordinate (F) at (1,-1);
		\coordinate (G) at (1,-2);
		\coordinate (H) at (-1,-2);
		\coordinate (I) at (-1,-1);
		\coordinate (J) at (-2,-1);
		\coordinate (K) at (-2,1);
		\coordinate (L) at (-1,1);
		\draw (O) circle(2.236 cm);
		\draw (A)--(B)--(C)--(D)--(E)--(F)--(G)--(H)--(I)--(J)--(K)--(L)--cycle;
		\fill[color=gray!80,opacity=0.5pt] (A)--(B)--(C)--(D)--(E)--(F)--(G)--(H)--(I)--(J)--(K)--(L)--cycle;
		\draw (O)--(D) ($(O)!0.5!(D)$)node[left]{$r$};
		\fill (O)node[below]{$O$}circle(2pt);
		\foreach \x /\goc in {A/160,C/45,B/45,D/0,E/-45,F/-45,I/-135,G/-30,H/-120,J/180,K/180,L/135}
		\fill[black] (\x) circle (1.5pt)
		($(\x)+(\goc:3mm)$) node {$\x$};
		\draw[dashed] (O)--(2,0) (L)--(C)--(F)--(I)--cycle;
	\end{tikzpicture}
	\end{center}
	Đặt $DE=AB=KJ=HG=x$ (cm). Ta có $\dfrac{JE}{2}=\sqrt{1-\dfrac{x^2}{4}}\Rightarrow JE=\sqrt{4-x^2}$.\\
	Khi đó, diện tích hình chữ thập bằng \[S_{KDEJ}+2S_{ABCL}=x\sqrt{4-x^2}+x\left(\sqrt{4-x^2}-x\right)=-x^2+2x\sqrt{4-x^2}.\]
	Xét $f(x)=-x^2+2x\sqrt{4-x^2}$, với $0<x<2$.\\
	Ta có $f'(x)=-2x+2\sqrt{4-x^2}-\dfrac{2x^2}{\sqrt{4-x^2}}$, với $0<x<2$.\\
	Khi đó $f'(x)=0\Leftrightarrow x+\dfrac{x^2}{\sqrt{4-x^2}}=\sqrt{4-x^2}\Leftrightarrow x\sqrt{4-x^2}=4-2x^2\Leftrightarrow x\approx1{,}05$.\\
	Do đó, $\displaystyle\max_{(0;2)}f(x)=f(1{,}05)\approx 2{,}47$.
	}
\end{ex}
\begin{ex}%[1H8V7-9]%[TEX ĐỀ MOON 2025]%[Huỳnh Thanh Chí]
	Người ta cần trang trí một kim tự tháp hình chóp tứ giác đều $S.ABCD$ có cạnh bên bằng $200$ m, góc $\widehat{ASB}=15^\circ$ bằng đường gấp khúc dây đèn led vong quanh kim tự tháp $AEFGHIJKLS$. Trong đó điểm $L$ cố định và $LS=40$ m.
	\begin{center}
		\begin{tikzpicture}[scale=1,>=stealth, font=\footnotesize, line join=round, line cap=round]
			\coordinate (A) at (-1.9,-1.6);
			\coordinate (B) at (0,0);
			\coordinate (D) at (1.6,-1.6);
			\coordinate (C) at ($(B)+(D)-(A)$);
			\coordinate (O) at ($(A)!1/2!(C)$);
			\coordinate (S) at ($(O)+(0,4)$);
			\coordinate (L) at ($(S)!0.2!(A)$);
			\coordinate (K) at ($(S)!0.28!(D)$);
			\coordinate (J) at ($(S)!0.4!(C)$);
			\coordinate (I) at ($(S)!0.65!(B)$);
			\coordinate (H) at ($(S)!0.45!(A)$);
			\coordinate (G) at ($(S)!0.55!(D)$);
			\coordinate (F) at ($(S)!0.7!(C)$);
			\coordinate (E) at ($(S)!0.8!(B)$);
			\draw (S)--(A)--(D)--(C)--cycle (S)--(D) (F)--(G)--(H) (J)--(K)--(L);
			\draw[dashed] (A)--(B)--(C) (S)--(B) (A)--(E)--(F) (H)--(I)--(J);
			\foreach \x/\g in {S/90,A/-150,B/-60,C/0,D/-45,E/170,F/45,G/-30,H/170,I/140,J/30,K/60,L/170}
			\fill[black] (\x) circle (1pt) ($(\g:3mm)+(\x)$) node {$\x$};
		\end{tikzpicture}
	\end{center}
	Hỏi khi đó cần dùng ít nhất bao nhiêu mét dây đèn led để trăng trí? (làm tròn đến hàng đơn vị).
	
	\shortans[]{$263$}
	\loigiai{
		Ta trải hình chóp tứ giác đều thành vẽ như sau
		\begin{center}
			\begin{tikzpicture}[scale=1.5,font=\footnotesize,line join=round,line cap=round,>=stealth]
				\def\a{4}
				\def\r{15}
				\path 
				(0,0) coordinate (S)
				(-135:\a) coordinate (A)
				(-135+\r:\a) coordinate (D)
				($(A)!1!-90:(D)$) coordinate (B_2)
				($(D)!1!90:(A)$) coordinate (C_2)
				;
				\foreach \x/\i in {C/2,B/3,A_1/4,D_1/5,C_1/6,B_1/7,A_2/8}{
					\path 
					(-135+\i*\r:\a) coordinate (\x)
					;
					\draw (S)--(\x);	
				}
				\foreach \x/\y/\i in {A/L/1,D/K/2,C/J/3,B/I/4,A_1/H/5,D_1/G/6,C_1/F/7,B_1/E/8,
					A_1/H_1/7,B/I_1/6,C/J_1/5,D/K_1/4
				}{
					\path 
					($(S)!1/9*\i!(\x)$) coordinate (\y)
					;}
				\draw (S)--(D)--(C_2)--(B_2)--(A)--(S) (A)--(D)--(C)--(B)--(A_1)--(D_1)--(C_1)--(B_1)--(A_2)
				(L)--(K)--(J)--(I)--(H)--(G)--(F)--(E)--(A_2)--(L)
				%	(K1)--(J1)--(I1)--(H1)
				;
				
				\foreach \x/\g in {A/135,D/-65,S/180,C/-90,B/-90,A_1/-90,D_1/-90,C_1/-60,B_1/-45,A_2/0,C_2/-135,B_2/-90}
				\fill 	(\x) circle (1pt)
				($(\g:3mm)+(\x)$) node {$\x$};
				\foreach \x/\g in {L/135,K/-90,J/-90,I/-90,H/-90,G/-90,F/-90,E/-90}
				\fill 	(\x) circle (1pt)
				($(\g:3mm)+(\x)$) node {$\x$};
			\end{tikzpicture}
		\end{center}
		Ta có $T=SL+LK+KJ+\ldots+EA_2\ge SL+LA_2$ (vì $SL$ không đổi).\\
		Để sợi dây trang trí ngắn nhất thì $T=SL+LA_2$.\\
		Ta có $\widehat{LSA_2}=15^\circ \cdot 8=120^\circ$.\\
		Áp dụng định lí cosin vào $\triangle SLA_2$ có
		\allowdisplaybreaks
		\begin{eqnarray*}
			LA_2=\sqrt{SL^2+SA_2^2-2\cdot SL\cdot SA_2\cdot\cos \widehat{SLA_2}}=40\sqrt{31}.
		\end{eqnarray*}
		Vậy $T=40+40\sqrt{31}\approx 263$.
	}
\end{ex}
\Closesolutionfile{ans}
% \begin{name}
	{\tenchude}
	{\tendethi}
	{\tentruong}
	{\thoigian}
	\end{name}
\TN
\Opensolutionfile{ans}[ans/de5-phanI]
\begin{ex}%Câu 1
	Cho hàm số $y=f(x)$ có bảng biến thiên như sau
	\begin{center}
		\begin{tikzpicture}
			\tkzTabInit[nocadre=false,lgt=1.2,espcl=2.5,deltacl=0.6]
			{$x$ /0.6,$y'$ /0.6,$y$ /2}
			{$-\infty$,$-1$,$0$,$1$,$+\infty$}
			\tkzTabLine{,-,$0$,+,$0$,-,$0$,+,}
			\tkzTabVar{+/$+\infty$, -/$1$,+/$2$,-/$1$,+/$+\infty$}
		\end{tikzpicture}
	\end{center}
	Hàm số đã cho nghịch biến trên khoảng nào dưới đây?
	\choice
	{$(0;+\infty)$}
	{$(-\infty;1)$}
	{\True $(0;1)$}
	{$(-1;0)$}
	\loigiai{
		Quan sát bảng biến thiên ta thấy $y'<0$ trên các khoảng $(-\infty;-1)$ và $(0;1)$ nên hàm số đã cho nghịch biến trên các khoảng $(-\infty;-1)$ và $(0;1)$.
	}
\end{ex}
\begin{ex}%Câu 2
	Họ tất cả các nguyên hàm của hàm số $f(x)=4x+\sin x$ là
	\choice
	{\True $2x^2-\cos x+C$}
	{$2x^2+\cos x+C$}
	{$2x^2-\sin x+C$}
	{$2x^2+\sin x+C$}
	\loigiai{
		Ta có 
		\begin{align*}
			\displaystyle \int f(x)\mathrm{\,d}x &= \int \left(4x+\sin x\right) \mathrm{\,d}x \\
			&= \int 4x\mathrm{\,d}x+\int \sin x \mathrm{\,d}x\\
			&=4\cdot \dfrac{x^2}{2}-\cos x+C\\
			&=2x^2-\cos x+C.
		\end{align*}
	}
\end{ex}
\begin{ex}%Câu 3
	Kết quả khảo sát cân nặng số táo ở lô hàng $B$ được cho ở bảng sau
	\begin{center}
		\begin{tabular}{|c|c|c|c|c|c|}
			\hline
			Cân nặng (g) & $[150;155)$ & $[155;160)$ & $[160;165)$ & $[165;170)$ & $[170;175)$ \\
			\hline
			Số quả táo ở lô hàng $B$ & $1$ & $3$ & $7$ & $10$ & $4$ \\
			\hline
		\end{tabular}
	\end{center}
	Số tạo được khảo sát trong bảng số liệu là
	\choice
	{$6$}
	{\True $25$}
	{$7$}
	{$5$}
	\loigiai{
		Số táo 	được khảo sát trong bảng số liệu là $n=1+3+7+10+4=25$.
	}
\end{ex}
\textbf{\textit{Sử dụng thông tin dưới đây để trả lời câu \ref{câu 4-đề 5} và câu \ref{câu 5-đề 5}}}\\[0.5em]
Trong không gian $Oxyz$, cho mặt phẳng $(P)\colon 2x-3y+6z-5=0$ và điểm $A(2;-3;1)$.
\begin{ex}%Câu 4
	\label{câu 4-đề 5}
	Một vectơ pháp tuyến của mặt phẳng $(P)$ là
	\choice
	{$\overrightarrow{n}_1=(2;-3;-5)$}
	{\True $\overrightarrow{n}_2=(2;-3;6)$}
	{$\overrightarrow{n}_3=(2;3;-5)$}
	{$\overrightarrow{n}_4=(2;-3;5)$}
	\loigiai{
		Một vectơ pháp tuyến của mặt phẳng $(P)$ là $\overrightarrow{n}=(2;-3;6)$.
	}
\end{ex}
\begin{ex}%Câu 5
	\label{câu 5-đề 5}
	Đường thẳng $d$ đi qua điểm $A$ và vuông góc với mặt phẳng $(P)$ có phương trình tham số là
	\choice
	{$d\colon\heva{& x=2+2t \\ & y=-3-3t\\ & z=1-5t}$}
	{$d\colon\heva{& x=2+2t \\ & y=-3-3t\\ & z=6+t}$}
	{\True $d\colon\heva{& x=2+2t \\ & y=-3-3t\\ & z=1+6t}$}
	{$d\colon\heva{& x=-2+2t \\ & y=3-3t\\ & z=-1+6t}$}
	\loigiai{
		Ta có đường thẳng $d\perp (P)$  nên có vectơ chỉ phương là $\overrightarrow{u}_d=\overrightarrow{n}_{(P)}=(2;-3;6)$.\\
		Đường thẳng $d$ đi qua $A$ và có vectơ chỉ phương là $\overrightarrow{u}_d==(2;-3;6)$ có phương trình tham số là
		\[\heva{&x=2+2t\\&y=-3-3t\\&z=1+6t.}\]
	}
\end{ex}
\begin{ex}%Câu 6
	$\lim\left(-3n^3+2n^2-5\right)$ bằng
	\choice
	{$-3$}
	{$-6$}
	{\True $-\infty$}
	{$+\infty$}
	\loigiai{
		Ta có $\lim \left(-3n^3+2n^2-5\right)= \lim n^3 \left(-3+\dfrac{2}{n}-\dfrac{5}{n^3}\right)$.\\
		Vì $
		\heva{&\lim n^3=+\infty\\&\lim\left(-3+\dfrac{2}{n}- \dfrac{5}{n^3}\right)=-3<0}$ nên  $\lim n^3 \left(-3+\dfrac{2}{n}-\dfrac{5}{n^3}\right)=-\infty$.\\ Suy ra $\lim \left(-3n^3+2n^2-5\right)=-\infty$ .
	}
\end{ex}
\begin{ex}%Câu 7
	\immini[thm]
	{
		Diện tích hình thang cong ở hình vẽ bên là $S=10$. Tích phân $\displaystyle\int\limits_{0}^{4} \left[4x+f(x)\right] \mathrm{\,d}x$ bằng
	\choice[2]
	{$14$}
	{\True $42$}
	{$32$}
	{$26$}
	}
	{
		\begin{tikzpicture}[scale=0.6,>=stealth, font=\footnotesize, line join=round, line cap=round]
			\def\a{0.25} \def\b{-1.5} \def\c{2.25} \def\d{2} % Hệ số
			\def\xmin{-1} \def\xmax{5}
			\def\ymin{-1} \def\ymax{4} 
			\draw[->] (\xmin,0)--(\xmax,0) node [below]{$x$};
			\draw[->] (0,\ymin)--(0,\ymax) node [left]{$y$};
			\node at (0,0) [below left]{$O$};
			\clip (\xmin+0.1,\ymin+0.1) rectangle (\xmax-0.5,\ymax-0.1);
			\draw[smooth,samples=300] plot(\x,{\a*(\x)^3+\b*(\x)^2+\c*(\x)+\d});
			\fill[pattern=north east lines,opacity=0.8] (0,0)--plot[domain=0:4](\x,{\a*(\x)^3+\b*(\x)^2+\c*(\x)+\d})--(4,0)--cycle;
			\draw[dashed] (4,0)node[below]{$4$}--(4,3) (0,2)node[left]{$2$};
		\end{tikzpicture}
	}
	\loigiai{
		Ta có $S=\displaystyle\int\limits_0^4 \left|f(x)\right|\mathrm{\,d}x=\displaystyle\int\limits_0^4 f(x)	\mathrm{\,d}x=10$.\\
		Do đó,
		\[
		\displaystyle\int\limits_0^4 \left[4x+f(x)\right]\mathrm{\,d}x=\displaystyle\int\limits_0^4 4x \mathrm{\,d}x+\displaystyle\int\limits_0^4 f(x)\mathrm{\,d}x
		=\left( 2x^2\right)\bigg|_0^4+10=32+10=42.
		\]
	}
\end{ex}
%\renewcommand{\baselinestretch}{1.4}
\begin{ex}%Câu 8
	Cho các số thực dương $a$, $b$ thỏa mãn $3\log a+2\log b=1$. Mệnh đề nào sau đây đúng?
	\choice
	{$a^3+b^2=1$}
	{$3a+2b=10$}
	{\True $a^3b^2=10$}
	{$a^3+b^2=10$}
	\loigiai{
		Ta có
		\[
		3\log a+2\log b=1\Leftrightarrow
		\log a^3+\log b^2 =1 \Leftrightarrow \log \left(a^3b^2\right) =1 \Leftrightarrow a^3b^2=10.
		\]
	}
\end{ex}
\begin{ex}%Câu 9
	Cho hình hộp $ABCD.A'B'C'D'$. Tính tổng $\overrightarrow{AB}+\overrightarrow{AD}+\overrightarrow{A'C'}$.
	\choice
	{$2\overrightarrow{AA'}$}
	{$\overrightarrow{0}$}
	{\True $2\overrightarrow{AC}$}
	{$2\overrightarrow{C'A'}$}
	\loigiai{
		Ta có $\overrightarrow{A'C'}=\overrightarrow{AC}$ và $\overrightarrow{AB}+\overrightarrow{AD}=\overrightarrow{AC}$ (quy tắc hình bình hành).\\
		Suy ra $\overrightarrow{AB}+\overrightarrow{AD}+\overrightarrow{A' C'}=\left(\overrightarrow{AB}+\overrightarrow{AD}\right)+\overrightarrow{A' C'}=\overrightarrow{AC}+\overrightarrow{AC}=2\overrightarrow{AC}$.
	}
\end{ex}
\begin{ex}%Câu 10
	Đồ thị hàm số $y=x^3-3x^2+2$ là đường cong nào trong hình sau đây?
	\choice
	{\begin{tikzpicture}[scale=0.53,>=stealth, font=\footnotesize, line join=round, line cap=round]
			\def\a{1} \def\b{-2} \def\c{2} % Hệ số
			\def\xmin{-2.5} \def\xmax{2.5}
			\def\ymin{-1} \def\ymax{4} 
			\draw[->] (\xmin,0)--(\xmax,0) node [below]{$x$};
			\draw[->] (0,\ymin)--(0,\ymax) node [left]{$y$};
			\node at (0,0) [below left]{$O$};
			\clip (\xmin+0.1,\ymin+0.1) rectangle (\xmax-0.5,\ymax-0.1);
			\draw[smooth,samples=300,domain=\xmin:\xmax] plot(\x,{\a*(\x)^4+\b*(\x)^2+\c});
	\end{tikzpicture}}
	{\begin{tikzpicture}[scale=0.44,>=stealth, font=\footnotesize, line join=round, line cap=round]
			\def\a{1} \def\b{0} \def\c{-3} \def\d{2} % Hệ số
			\def\xmin{-3} \def\xmax{3}
			\def\ymin{-1} \def\ymax{5} 
			\draw[->] (\xmin,0)--(\xmax,0) node [below]{$x$};
			\draw[->] (0,\ymin)--(0,\ymax) node [left]{$y$};
			\node at (0,0) [below left]{$O$};
			\clip (\xmin+0.1,\ymin+0.1) rectangle (\xmax-0.5,\ymax-0.1);
			\draw[smooth,samples=300] plot(\x,{\a*(\x)^3+\b*(\x)^2+\c*(\x)+\d});
	\end{tikzpicture}}
	{\True \begin{tikzpicture}[scale=0.44,>=stealth, font=\footnotesize, line join=round, line cap=round]
			\def\a{1} \def\b{-3} \def\c{0} \def\d{2} % Hệ số
			\def\xmin{-2} \def\xmax{4}
			\def\ymin{-3} \def\ymax{3} 
			\draw[->] (\xmin,0)--(\xmax,0) node [below]{$x$};
			\draw[->] (0,\ymin)--(0,\ymax) node [left]{$y$};
			\node at (0,0) [below left,xshift=0.1cm]{$O$};
			\clip (\xmin+0.1,\ymin+0.1) rectangle (\xmax-0.5,\ymax-0.1);
			\draw[smooth,samples=300] plot(\x,{\a*(\x)^3+\b*(\x)^2+\c*(\x)+\d});
	\end{tikzpicture}}
	{\begin{tikzpicture}[scale=0.44,>=stealth, font=\footnotesize, line join=round, line cap=round]
			\def\a{-1} \def\b{3} \def\c{0} \def\d{0} % Hệ số
			\def\xmin{-2} \def\xmax{4}
			\def\ymin{-1} \def\ymax{5} 
			\draw[->] (\xmin,0)--(\xmax,0) node [below]{$x$};
			\draw[->] (0,\ymin)--(0,\ymax) node [left]{$y$};
			\node at (0,0) [below left]{$O$};
			\clip (\xmin+0.1,\ymin+0.1) rectangle (\xmax-0.5,\ymax-0.1);
			\draw[smooth,samples=300] plot(\x,{\a*(\x)^3+\b*(\x)^2+\c*(\x)+\d});
	\end{tikzpicture}}
	\loigiai{
		Đồ thị hàm bậc ba $y=x^3-3x^2+2$ với $a>0$ nên loại 
		\begin{center}
			\begin{tabular}{cc}
				\begin{tikzpicture}[thick,>=stealth,scale=0.6] 
					\clip(-2.5,-1) rectangle (2.5,3.5);
					\draw[->,very thick,blue] (-2.5,0) -- (2.5,0) node[below left] {\small $x$};
					\draw[->,very thick,blue] (0,-1) -- (0,3.5) node[below left] {\small $y$};
					\draw [fill=white,draw=blue] (0,0) circle (1pt)node[below right] {\footnotesize $O$};
					\draw[very thick,black,smooth,samples=100,domain=-2.5:2.5] plot(\x,{(\x)^4-2*(\x)^2+2});
					%	\draw[dashed, thick,blue] 
					%	(1,0) node[below]{1}|-(0,1) node[above left]{1}
					%	(-1,0) node[below]{-1}|-(0,1)node[left]{};
				\end{tikzpicture}
				&
				\begin{tikzpicture}[thick,>=stealth,scale=0.6] 
					\clip(-2,-1.5) rectangle (4.5,4.5);
					\draw[->,very thick,blue] (-2,0) -- (4.5,0) node[below left] {\small $x$};
					\draw[->,very thick,blue] (0,-1.5) -- (0,4.5) node[below left] {\small $y$};
					\draw [fill=white,draw=blue] (0,0) circle (1pt)node[below right] {\footnotesize $O$};
					\draw[very thick,black,smooth,samples=100,domain=-2:4.5] plot(\x,{-(\x)^3+3*(\x)^2});
					%	\draw[dashed, thick,blue] 
					%	(1,0) node[below]{1}|-(0,1) node[above left]{1}
					%	(-1,0) node[below]{-1}|-(0,1)node[left]{};
				\end{tikzpicture}
			\end{tabular}
		\end{center}
		Hàm số $y=x^3-3x^2+2$ có $y'=3x^2-6x$. Cho $y'=0 \Rightarrow \hoac{&x=0\\&x=2.}$\\
		Suy ra $x=0$ và $x=2$ là hai điểm cực trị nên chọn
		\begin{center}
			\begin{tikzpicture}[thick,>=stealth,scale=0.6] 
				\clip(-2,-2.5) rectangle (4.5,3.5);
				\draw[->,very thick,blue] (-2,0) -- (4.5,0) node[below left] {\small $x$};
				\draw[->,very thick,blue] (0,-2.5) -- (0,3.5) node[below left] {\small $y$};
				\draw [fill=white,draw=blue] (0,0) circle (1pt)node[below right] {\footnotesize $O$};
				\draw[very thick,black,smooth,samples=100,domain=-2:4.5] plot(\x,{(\x)^3-3*(\x)^2+2});
			\end{tikzpicture}
		\end{center}
	}
\end{ex}
\begin{ex}%Câu 11
	Cho hình chóp $S.ABCD$ có đáy $ABCD$ là hình thoi tâm $O$. Biết rằng $SA=SC$ và $SB=SD$. Khẳng định nào sau đây là \textbf{sai}?
	\choice
	{$SO\perp(ABCD)$}
	{$AC\perp BD$}
	{$(SBD)\perp(SAC)$}
	{\True $BD\perp SD$}
	\loigiai{
		\immini{
			Ta có $\heva{& SO\perp AC\\& SO\perp BD}\Rightarrow SO\perp (ABCD)$ đúng.\\
			Vì $ABCD$ là hình thoi nên $AC\perp BD$ đúng.\\
			Ta có $\heva{& BD\perp AC\\& BD\perp SO} \Rightarrow BD\perp (SAC)$
			Suy ra $\heva{& BD\subset (SBD)\\& BD\perp (SAC)}\Rightarrow (SBD)\perp (SAC)
			$ đúng.\\
			Ta có $BD\subset (SBD)$ mà tam giác $SBD$ cân tại $S$ nên $\widehat{SDB}=\widehat{SBD}<90^\circ$ do đó $BD\perp SD$ sai.
		}{
			\begin{tikzpicture}[declare function={a=2;b=4;h=4;},line join=round]
				\path (0,0) coordinate (A)
				(-145:a) coordinate (B)
				(b,0) coordinate (D)
				($ (B)!0.5!(D) $) coordinate (O)
				($(O)+(0,h) $) coordinate (S);
				\path ($(D)-(A)+(B)$) coordinate (C);
				\draw[dashed] (S)--(A)--(B) (A)--(D)--(B) (A)--(C) (S)--(O);
				\draw (B)--(C)--(D) (B)--(S)  (D)--(S)--(C);
				\foreach \x/\y/\z in {S/O/B,S/O/C,A/O/B}{
					\path pic[draw,angle radius=5pt]{right angle= \x--\y--\z};
				}
				\foreach \t/\g in {A/150,B/-90,C/-90,D/0,S/90,O/-90}{
					\draw[fill=black] (\t) circle (1pt) node[shift={(\g:7pt)},font=\scriptsize]{$ \t $};
				}
			\end{tikzpicture}
		}
	}
\end{ex}
\begin{ex}%Câu 12
	Tìm tập nghiệm $S$ của bất phương trình $\log_{\tfrac{1}{5}}\left(x^2-1\right)<\log_{\tfrac{1}{5}}\left(3x-3\right)$.
	\choice
	{\True $S=(2;+\infty)$}
	{$S=(-\infty;1)\cup(2;+\infty)$}
	{$S=(-\infty;-1)\cup(2;+\infty)$}
	{$S=(1;2)$}
	\loigiai{
		Phương trình đã cho tương đương
		\[
		\heva{&x^2-1>3x-3\\&3x-3>0}\Leftrightarrow \heva{&x^2-3x+2>0\\&x>1} \Leftrightarrow \heva{&\hoac{&x<1\\&x>2}\\&x>1} \Leftrightarrow x>2.
		\]
		Vậy tập nghiệm của bất phương trình là $S=(2;+\infty)$.
	}
\end{ex}
\Closesolutionfile{ans}
%{\fontfamily{qtm}\fontsize{13pt}{2pt}\selectfont\textbf{PHẦN II. Câu trắc nghiệm đúng sai}. Thí sinh trả lời từ câu 1 đến câu 4. Trong mỗi ý \textbf{a)}, \textbf{b)}, \textbf{c)}, \textbf{d)} ở mỗi câu, thí sinh chọn đúng hoặc sai.}
%\setcounter{ex}{0}% Reset lại số đếm câu hỏi
\TNTF
\Opensolutionfile{ans}[ans/de5-phanII]
\begin{ex}%Câu 1
	Trong không gian $Oxyz$, cho các điểm $A(0;-1;1)$, $B(-2;1;-1)$ và $C$ thỏa mãn điều kiện $\overrightarrow{OC}=-\overrightarrow{i}+3\overrightarrow{j}+2\overrightarrow{k}$. Xét tính đúng sai của các mệnh đề sau
	\choiceTF
	{Tọa độ điểm $C$ là $(-1;2;3)$}
	{\True Tọa độ các vectơ $\overrightarrow{AB}=(-2;2;-2)$ và $\overrightarrow{AC}=(-1;4;1)$}
	{Một vectơ pháp tuyến của mặt phẳng $(ABC)$ là $(5;-2;-3)$}
	{Khoảng cách từ gốc tọa độ $O$ đến mặt phẳng $(ABC)$ bằng $\dfrac{5}{\sqrt{33}}$}
		\loigiai{
		\begin{itemchoice}
			\itemch 	Ta có $\overrightarrow{OC}=-\overrightarrow{i}+3\overrightarrow{j}+2\overrightarrow{k}$ nên tọa độ điểm $C(-1;3;2)$.
			\itemch Tọa độ $\overrightarrow{AB}=(-2;2;-2)$ và $\overrightarrow{AC}=(-1;4;1)$.
			\itemch 	Một vectơ pháp tuyến của mặt phẳng $(ABC)$ là \[\overrightarrow{n}=[\overrightarrow{AB},\overrightarrow{AC}] = (10;4;-6)=2(5;2;-3).\]
			\itemch Phương trình mặt phẳng $(ABC)$ là
			\[
			5(x-0)+2(y+1)-3(z-1)=0 \Leftrightarrow 5x+2y-3z+5=0.
			\]
			Khoảng cách từ $O$ đến $(ABC)$ là
			\[
			\mathrm{d}(O,(ABC))=\dfrac{|5\cdot 0+2\cdot 0-3\cdot 0+5|}{\sqrt{5^2+2^2+(-3)^2}}=\dfrac{5}{\sqrt{38}}=\dfrac{5\sqrt{38}}{38}.
			\]
		\end{itemchoice}	
	}
\end{ex}
\begin{ex}%Câu 2
	Tốc độ giao đổi chất cơ bản của sinh vật có thể tăng hoặc giảm tùy thuộc vòa hoạt động của sinh vật. Cụ thể sau khi hấp thụ chất dinh dưỡng, sinh vật thường trải qua một sự tăng đột biến trong tốc độ trao đổi chất của nó, sau đó dần dần trở lại mức cơ bản. Linh vừa kết thúc bữa ăn tối của mình với năng lượng nạp vào là $5120$ J và tốc độ trao đổi chất của cô đã tăng đột biến từ mức cơ bản $M_0$. Sau đó cô đã tiêu hao hết năng lượng đó trong $12$ giờ tiếp theo. Giả sử $t$ giờ sau bữa ăn Linh tiêu hao được $M(t)$ kJ, tốc độ trao đổi chất của cô được cho bởi hàm số $M'(t)=M_0+t\mathrm{e}^{-0{,}1t^2}$ (kJ/h), $t\in[0;12]$. Xét tính đúng sai của các mệnh đề sau
	\choiceTF
	{\True $M(t)=M_0t-5\mathrm{e}^{-0{,}1t^2}+C$ là họ nguyên hàm của hàm số $M'(t)$}
	{\True $M_0=0{,}01$ (làm tròn đến hàng phần trăm)}
	{\True Năng lượng còn lại sau $6$ giờ đâu là $197$ J (làm tròn đến hàng đơn vị)}
	{Tốc độ tiêu hao năng lượng trung bình trong khoảng thời gian từ $a$ giờ tới $b$ giờ được tính bởi công thức $v_{\text{tb}}=\dfrac{M(b)-M(a)}{b-a}$. Tốc độ tiêu hao năng lượng trung bình từ $6$ giờ đến $12$ giờ của Linh là $32{,}76$ J/h (làm tròn kết quả đến hàng đơn vị)}
	\loigiai{
		\begin{itemchoice}
			\itemch Ta có $M'(t)=\left(M_0 t-5\mathrm{e}^{-0{,}1t^2}+C\right)' = M_0+t\mathrm{e}^{-0{,}1t^2}$.\\
			Suy ra $\displaystyle \int M'(t)\mathrm{\,d}t=M(t)=M_0 t-5\mathrm{e}^{-0{,}1t^2}+C$.
			\itemch Tổng năng lượng tiêu hao trong khoảng thời gian $12$ giờ là $5\,120$ J $ =5{,}12$ kJ tức là
			\[
			M(12)-M(0)=5{,}12\,\,\text{kJ}.
			\]
			Khi $t=12$ ta có $M(12)=M_0\cdot 12-5\mathrm{e}^{-0{,}1\cdot 12^2}+C$.\\
			Khi $t=0$ ta có $M(0)=M_0\cdot 0-5\mathrm{e}^{-0{,}1\cdot 0^2}+C=-5+C$.\\
			Suy ra $M(12)-M(0)=12M_0-5\left( \mathrm{e}^{-0{,}1\cdot 12^2}-1\right)$.\\
			Thay $M(12)-M(0)=5\,120$ ta được
			\[
			12M_0-5\left( \mathrm{e}^{-0{,}1\cdot 12^2}-1\right) = 5{,}12
			\Leftrightarrow M_0=0{,}01\,\, \rm{kJ/h}.
			\]
			\itemch Năng lượng tiêu hao trong $6$ giờ
			\[
			M(6)-M(0)=6M_0-5\left(\mathrm{e}^{-0{,}1\cdot 6^2}-1\right)\approx 4{,}92\,\,\text{kJ}.
			\]
			Năng lượng còn lại sau $6$ giờ
			\[
			5120\,\,\text{J}-4920\,\,\text{J}=200\,\,{J}.
			\]
			\itemch 	Tốc độ tiêu hao trung bình từ $6$ giờ đến $12$ giờ là
			\[
			v_{\text{tb}}=\dfrac{M(12)-M(6)}{12-6}=\dfrac{6M_0-5\left(\mathrm{e}^{-0{,}1\cdot 12^2}-\mathrm{e}^{-0{,}1\cdot 6^2}\right)}{6}\approx 0{,}0327695\,\,\rm{kJ/h}\approx 32{,}77 \,\,\rm{J/h}.
			\]
		\end{itemchoice}	
	}
\end{ex}
%\renewcommand{\baselinestretch}{1.4}
\begin{ex}%Câu 3
	\immini[thm]
	{
		Ta có Trái Đất là hình cầu hoàn hảo với bán kính $R=6370$ km và diện tích toàn phần là $S=4\pi R^2$. Các phi hành gia từ tàu vũ trụ chỉ có thể nhìn thấy một phần bề mặt Trái Đất. Ở độ cao $h$, phần diện tích Trái Đất các phi hành gia có thể nhìn thấy sẽ được tính theo công thức $S_{T}=2\pi R^2\left(1-\dfrac{R}{R+h}\right)$, trong đó $R$ là bán kính Trái Đất. Gọi $K$ là tỷ số diện tích bề mặt Trái Đất nhìn thấy được ở độ cao $h$ với diện tích toàn phần của Trái Đất. Xét tính đúng sai của các mệnh đề sau
	}
	{
		\begin{tikzpicture}[scale=0.7,>=stealth, font=\footnotesize, line join=round, line cap=round]
			\coordinate (O) at (0,0);
			\draw (O)node[opacity=0.85]{\includegraphics[width=3.45cm]{images/de5-2}};
			\coordinate (M) at ($(O)+(-130:1.5cm)$);
			\coordinate (A) at (120:3 cm and 3 cm);
			\draw[dashed] (A) arc (120:-120:3 cm and 3 cm);
			\coordinate (A') at (0:3 cm and 3 cm);
			\draw (A')node[rotate=90,left,xshift=0.1cm]{\includegraphics[width=0.6cm]{images/de5-1}};
			\draw[dashed] (O)--(A') ($(O)!0.7!(A')$)node[above]{$h$};
			\coordinate (B) at (45:4.5 cm and 4.5 cm);
			\draw[dashed] (B) arc (45:-45:4.5 cm and 4.5 cm);
			\coordinate (B') at (0:4.5 cm and 4.5 cm);
			\draw (B')node[rotate=90,left,xshift=0.1cm]{\includegraphics[width=0.6cm]{images/de5-1}};
			\draw[thick] (O)--(M);
			\tkzDefLine[tangent from = A'](O,M)% with 4.25
			\tkzGetPoints{X}{Y} 
			\draw (A')--($(A')!1.5!(X)$) (A')--($(A')!1.5!(Y)$);
			\tkzDefLine[tangent from = B'](O,M)% with 4.25
			\tkzGetPoints{X'}{Y'}
			\draw[dashed] (B')--(X') (B')--(Y');
		\end{tikzpicture}
	}\vspace{3pt}
	\choiceTF
	{Công thức tính $K$ là $K=\dfrac{1}{2}\left(1-\dfrac{h}{R+h}\right)$}
	{Trong một chuyến bay của tàu con thoi, các phi hành gia đã thực hiện một hoạt động ngoài tàu ở độ cao $280$ km. Có $2{,}5\%$ (làm tròn đến hàng phần mười) diện tích bề mặt Trái Đất có thể nhìn thấy ở độ cao đó}
	{Muốn nhìn thấy $\dfrac{1}{4}$ diện tích bề mặt Trái Đất, các phi hành gia cần đưa tàu con thoi đạt đến độ cao $6470$ km}
	{\True Khi độ cao $h$ càng tăng lên thì $K$ càng tăng nhưng không vượt quá $50\%$}
	\loigiai{
		\begin{itemchoice}
			\itemch 	Ta có tỷ số $K$ là diện tích nhìn thấy được chia cho diện tích toàn phần
			\[
			K =\dfrac{S_T}{S}=\dfrac{2\pi R^2\left(1-\dfrac{R}{R+h}\right)}{4\pi R^2}
			= \dfrac{1}{2}\left(1-\dfrac{R}{R+h}\right)=\dfrac{1}{2}\cdot \dfrac{h}{R+h}.
			\]
			\itemch Với bán kính $R=6\,370$ km và chiều cao $h=280$ km thì tỷ số $K$ tại độ cao $h$ là
			\[
			K=\dfrac{1}{2}\cdot \dfrac{h}{R+h}=\dfrac{1}{2}\cdot \dfrac{280}{6\,370+280}=\dfrac{2}{95}\approx0{,}021\approx 2{,}1\%.
			\]		
			\itemch Với $K=\dfrac{1}{4}$ thì
			\[
			K=\dfrac{1}{2}\cdot \dfrac{h}{R+h}\Leftrightarrow \dfrac{1}{4}=\dfrac{1}{2}\cdot \dfrac{h}{6\,370+h}\Leftrightarrow h=6\,370\,\,\rm{(km)}.\]
			\itemch 	Khi $h$ càng lớn ta có
			\[
			\lim\limits_{h\rightarrow +\infty} K =\lim\limits_{h\rightarrow +\infty} \dfrac{1}{2}\dfrac{h}{R+h}=\dfrac{1}{2}\lim\limits_{h\rightarrow +\infty} \dfrac{1}{\dfrac{R}{h}+1}=\dfrac{1}{2}=0{,}5=50\%.
			\]
		\end{itemchoice}
	}
\end{ex}
\begin{ex}%Câu 4
	Ở một khu rừng nọ có $7$ chú lùn, trong đó có $5$ chú luôn nói thật, $2$ chú còn lại nói thật với xác suất $0{,}5$. Nàng Bạch Tuyết lạc vào trong rừng và gặp một chú lùn.
	\begin{itemize}
		\item Gọi $A$ là biến cố \lq\lq Chú lùn gặp được luôn nói thật\rq\rq.
		\item Gọi $B$ là biến cố \lq\lq Chú lùn đó tự nhận mình luôn nói thật\rq\rq.
	\end{itemize}
	Xét tính đúng sai của các mệnh đề sau\vspace{5pt}
	\choiceTF
	{\True $\mathrm{P}(A)=\dfrac{5}{7}$ và $\mathrm{P}(\overline{A})=\dfrac{2}{7}$}
	{Xác suất có điều kiện $\mathrm{P}(B\mid A)=0{,}5$}
	{\True $\mathrm{P}(B)=\dfrac{6}{7}$}
	{\True Nàng Bạch Tuyết gặp ngẫu nhiên một chú lùn. Biết rằng chú lùn mà Bạch Tuyết gặp tự nhận mình là luôn nói thật. Xác suất để chú lùn đó luôn nói thật là $\dfrac{5}{6}$}
	\loigiai{
		\begin{itemchoice}
			\itemch Ta có $\mathrm{P}(A)=\dfrac{5}{7}\Rightarrow \mathrm{P}(\overline{A}) = \dfrac{2}{7}$ 
			\itemch Xác xuất có điều kiện $P(B\mid A)=1$.
			\itemch Theo công thức Bayes, ta có
			\[\mathrm{P}(B)=\mathrm{P}(A) \cdot \mathrm{P}(B|A)+\mathrm{P}(\overline{A})\cdot\mathrm{P}(B|\overline{A}) \Leftrightarrow \mathrm{P}(B)=1\cdot \dfrac{5}{7}+ 0{,}5\cdot \dfrac{2}{7}=\dfrac{6}{7}.\]
			\itemch Ta có
			\[\mathrm{P}(A|B)=\dfrac{\mathrm{P}(AB)}{\mathrm{P(B)}}=\dfrac{\mathrm{P}(A)\cdot \mathrm{P}(B|A)}{\mathrm{P}(B)}= \dfrac{\dfrac{5}{7}\cdot 1}{\dfrac{6}{7}}=\dfrac{5}{6}.\]
		\end{itemchoice}	
	}
\end{ex}
\Closesolutionfile{ans}
%{\fontfamily{qtm}\fontsize{13pt}{2pt}\selectfont\textbf{PHẦN III. Câu trắc nghiệm trả lời ngắn}. Thí sinh trả lời từ câu 1 đến câu 6 và điền đáp án vào ô trống.}
%\setcounter{ex}{0}% Reset lại số đếm câu hỏi
\TNSA
\Opensolutionfile{ans}[ans/de5-phanIII]
\begin{ex}%Câu 1
	Cho hình chóp đều $S.ABCD$ có cạnh đáy bằng $4$, khoảng cách giữa hai đường thẳng $SA$ và $CD$ bằng $2$. Thể tích của khối chóp $S.ABCD$ bằng bao nhiêu? (làm tròn kết quả đến hàng phần mười).
	
	\shortans[0]{$6{,}2$}
	\loigiai{
		\immini{
			Diện tích mặt đáy $S_{ABCD}=4^2=16$.\\
			Chọn $(SAB)$ chứa $SA$, ta có $\heva{
				&CD\parallel AB\\
				&AB\subset (SAB).
			}$\\
			Suy ra $CD\parallel (SAB)$ nên 
			\begin{align*}
				\mathrm{d}(CD,SA)&=\mathrm{d}(CD,(SAB))=\mathrm{d}(D,(SAB))\\
				&=2\mathrm{d}(O,(SAB))=2\\
				\Rightarrow &\mathrm{d}(O,(SAB))=1.
			\end{align*}
			Gọi $H$ là hình chiếu vuông góc của $O$ trên $AB$ và 
		}{
			\begin{tikzpicture}[declare function={a=2;b=4;h=4;},line join=round]
				\path (0,0) coordinate (A)
				(-145:a) coordinate (B)
				(b,0) coordinate (D)
				($ (B)!0.5!(D) $) coordinate (O)
				($(O)+(0,h) $) coordinate (S);
				\path ($(D)-(A)+(B)$) coordinate (C);
				\path ($(A)!0.5!(B)$) coordinate (H);
				\path ($(S)!0.7! (H)$) coordinate (K);
				\draw[dashed] (S)--(A)--(B) (A)--(D)--(B) (A)--(C) (S)--(O)--(H)--(S) (K)--(O)--(H);
				\draw (B)--(C)--(D) (B)--(S)  (D)--(S)--(C);		
				\foreach \x/\y/\z in {S/O/B,S/O/C,A/O/B,O/H/A,O/K/S}{
					\path pic[draw,angle radius=5pt]{right angle= \x--\y--\z};
				}
				\foreach \t/\g in {A/160,B/-90,C/-90,D/0,S/90,O/-90,H/160,K/-140}{
					\draw[fill=black] (\t) circle (1pt) node[shift={(\g:7pt)},font=\scriptsize]{$ \t $};
				}
			\end{tikzpicture}
		}
		$K$ là hình chiếu vuông góc của $O$ trên $SH$, ta có
		\[
		\heva{
			&AB\perp OH\\
			&AB\perp SO
		}\Rightarrow AB\perp (SOH)\Rightarrow AB\perp OK.\quad (1)
		\]
		Mà $OK\perp SH$. \quad (2)\\
		Suy ra $OK\perp (SAB)\Rightarrow \mathrm{d}(O,(SAB))=OK=1$.\\
		Với $OH=\dfrac{1}{2}AD=2$ ta có
		\[
		\dfrac{1}{OK^2}=\dfrac{1}{OH^2}+\dfrac{1}{SO^2}\Leftrightarrow
		\dfrac{1}{SO^2}=\dfrac{1}{1^2}-\dfrac{1}{2^2}=\dfrac{3}{4}\Rightarrow
		SO^2=\dfrac{4}{3}\Rightarrow SO=\dfrac{2\sqrt{3}}{3}.
		\]
		Thể tích khối chóp $S.ABCD$ là
		\[
		V=\dfrac{1}{3}\cdot S_{ABCD}\cdot SO=\dfrac{32\sqrt{3}}{9}\approx 6{,}2.
		\]
	}
\end{ex}
\begin{ex}%Câu 2
	Một xạ thủ bắn hai viên đạn vào một bia. Xác suất bắn trúng viên thứ nhất là $0{,}7$. Nếu bắn trúng viên thứ nhất thì khả năng bắn trung viên thứ hai là $0{,}8$, nhưng nếu bắn trượt viên thứ nhất thì sẽ bị tâm lí dẫn đến khả năng bắn trúng viên thứ hai chỉ còn $0{,}3$. Biết rằng viên thứ hai xạ thủ bắn trúng, xác suất xạ thủ bắn trung viên thứ nhất là bao nhiêu $\%$ (làm tròn kết quả đến hàng phần chục).
	
	\shortans[0]{$86{,}2$}
	\loigiai{
		Gọi $A$ là biến cố: \lq\lq xạ thủ bắn trúng viên thứ nhất\rq\rq.\\
		Gọi $B$ là biến cố: \lq\lq xạ thủ bắn trúng viên thứ hai\rq\rq.\\
		Ta có $\mathrm{P}(A)=0{,}7$; $\mathrm{P}(\overline{A})=0{,}3$; 
		$\mathrm{P}(B\mid A)=0{,}8$; $\mathrm{P}(B\mid \overline{A})=0{,}3$; \\
		Theo công thức Bayes
		\[
		\mathrm{P}(A\mid B)=\dfrac{P(AB)}{P(B)}.
		\]
		Với $\mathrm{P}(AB)=\mathrm{P}(A)\cdot \mathrm{P}(B\mid A)=0{,}7\cdot 0{,}8=0{,}56$.\\
		Đồng thời $\mathrm{P}(B)=\mathrm{P}(AB)+\mathrm{P}(\overline{A}B)$\\
		Với $\mathrm{P}(AB)=0{,}56$ và $\mathrm{P}(\overline{A}B)=\mathrm{P}(\overline{A})\cdot \mathrm{P}(B\mid \overline{A})=0{,}3\cdot 0{,}3=0{,}09$.\\
		Suy ra $\mathrm{P}(B)=0{,}56+0{,}09=0{,}65$.\\
		Vậy
		\[
		\mathrm{P}(A\mid B)=\dfrac{0{,}56}{0{,}65}\approx 0{,}8615 = 86{,} 2\%.
		\]
	}
\end{ex}
\renewcommand{\baselinestretch}{1.4}
\begin{ex}%Câu 3
	Một con bọ di chuyển từ điểm $A$ đến điểm $B$ dọc theo các đoạn thẳng trong mạng lưới lục giác như hình bên dưới.
%	\begin{center}
%		\includegraphics[scale=0.85]{images/de5-3}
%	\end{center}
	\begin{center}
		\begin{tikzpicture}[>=stealth,line join=round,line cap=round,font=\small,scale=.7]
			\def\l{1.2}
			\newcommand{\drawhexagon}[3]{
				\begin{scope}[shift={(#1,#2)}]
					% Vẽ lục giác
					\draw (0:\l) -- (60:\l) -- (120:\l) -- (180:\l) -- (240:\l) -- (300:\l) -- cycle;
				\end{scope}
			}
			% Hàng 1
			\drawhexagon{0}{0}{}
			\draw [->] (0,1.04)--(0.1,1.04) node [above] {$1$} ;
			\draw [->] (0,-1.04)--(0.1,-1.04)node [below] {$2$};
			\draw (0,1.3) node[above] {$(C_1)$};
			% Hàng 2
			\drawhexagon{1.8}{1.04}{}
			\drawhexagon{1.8}{-1.04}{}
			\draw [->] (1.8,2.07)--(1.9,2.07) node [above] {$3$};
			\draw [<-] (1.8,0)--(1.9,0) ;
			\draw [->] (1.8,-2.08)--(1.9,-2.08)node [below] {$4$};
			\draw (1.8,2.4) node[above] {$(C_2)$};
			% Hàng 3
			\drawhexagon{3.6}{2.08}{}
			\drawhexagon{3.6}{0}{}
			\drawhexagon{3.6}{-2.08}{}
			\draw [->] (3.5,3.12)--(3.6,3.12) node [above] {$5$};
			\draw [->] (3.5,1.04)--(3.6,1.04) node [above] {$6$} ;
			\draw [->] (3.5,-1.04)--(3.6,-1.04) node [above] {$7$} ;
			\draw [->] (3.5,-3.12)--(3.6,-3.12) node [above] {$8$} ;
			\draw (3.5,3.4) node[above] {$(C_3)$};
			% Hàng 4
			\drawhexagon{5.4}{1.04}{}
			\drawhexagon{5.4}{-1.04}{}
			\draw [->] (5.2,2.07)--(5.3,2.07) node [above] {$9$};
			\draw [<-] (5.2,0)--(5.3,0) ;
			\draw [->] (5.2,-2.08)--(5.3,-2.08)node [below] {$10$};
			\draw (5.2,2.4) node[above] {$(C_4)$};
			% Hàng 5
			\drawhexagon{7.2}{0}{}
			\draw [->] (7,1.04)--(7.1,1.04) node [above] {$11$} ;
			\draw [->] (7,-1.04)--(7.1,-1.04)node [below] {$12$};
			\draw (7,1.3) node[above] {$(C_5)$};
			% Điểm A và B
			\node[fill=black,circle,inner sep=1pt,label=left:A] at (-1.2,0) {};
			\node[fill=black,circle,inner sep=1pt,label=right:B] at (8.4,0) {};
		\end{tikzpicture}
	\end{center}
	Các đoạn thẳng này có dấu mũi tên chỉ được di chuyển theo hướng của mũi tên và con bọ không bao giờ di chuyển trên cùng một đoạn thẳng quá một lần. Vậy con bọ có bao nhiêu con đường khác nhau từ $A$ đến $B$?
	
	\shortans[0]{$100$}
	\loigiai{
		\begin{itemize}
			\item Từ $A$ có $2$ cách đến $C1$
			\begin{itemize}
				\item Từ $A$ có $1$ cách đến mũi tên số $1$;
				\item Từ $A$ có $1$ cách đến mũi tên số $2$.
			\end{itemize}
			\item Từ $C1$ có $5$ cách đến $C3$\\
			Không mất tính tổng quát giả sử đi từ $C1$ mũi tên số $1$ đến $C3$ mũi tên số $6$ hoặc mũi tên số $7$.
			\begin{itemize}
				\item Từ mũi tên số $1$ có $2$ cách để đi đến mũi tên số $6$.
				\item Từ mũi tên số $1$ có $3$ cách để đi đến mũi tên số $7$.
			\end{itemize}
			Suy ra từ $C1$ có $5$ cách đến $C3$.
			\item Từ $C3$ có $5$ cách đến $C5$\\
			Không mất tính tổng quát giả sử đi từ $C3$ mũi tên số 5 đến $C5$ mũi tên số $11$ hoặc mũi tên số $12$.
			\begin{itemize}
				\item Từ mũi tên số $5$ có $2$ cách để đi đến mũi tên số $11$.
				\item Từ mũi tên số $5$ có $3$ cách để đi đến mũi tên số $11$.
			\end{itemize}
			Suy ra từ $C3$ có $5$ cách đến $C5$.
			\item Từ $C5$ có $2$ cách đến $B$
			\begin{itemize}
				\item Từ mũi tên số $11$ có $1$ cách đến $B$.
				\item Từ mũi tên số $12$ có $1$ cách đến $B$.
			\end{itemize}
		\end{itemize}
		Vậy có $2\cdot 5\cdot 5\cdot 2 = 100$ cách đi từ $A$ đến $B$.
	}
\end{ex}
\begin{ex}%Câu 4
	Một tấm ván gỗ chỉ được hỗ trợ ở hai đầu $O$ và $P$, cách nhau $4$ m. Tấm ván võng xuống dưới do trọng lượng của nó tạo thành một đường cong. Xét trên hệ trục $Oxy$ như hình vẽ dưới, đơn vị mỗi trục là mét, đường cong trong hình vẽ có phương trình $y=f(x)$.
	\begin{center}
		\begin{tikzpicture}[scale=1,>=stealth, font=\footnotesize, line join=round, line cap=round]
			\def\a{0.04} \def\b{-0.4} \def\c{0} % Hệ số
			\def\xmin{-1.5} \def\xmax{12}
			\def\ymin{-1.8} \def\ymax{1.5}
			\draw[->] (\xmin,0)--(\xmax,0) node [below]{$x$};
			\draw[->] (0,\ymin)--(0,\ymax) node [left]{$y$};
			\node at (0,0) [above left,xshift=0.1cm]{$O$};
			\clip (\xmin+0.1,\ymin+0.1) rectangle (\xmax-0.5,\ymax-0.1);
			\draw[smooth,samples=300,domain=-1:11] plot(\x,{\a*(\x)^2+\b*(\x)+\c});
			\draw (5,-1)node[below]{$y=f(x)$} (10,0)node[above]{$P$};
			\fill (0,0)circle(3pt);
			\fill (10,0)circle(3pt);
			\coordinate (A) at ($(0,0)+(-60:1cm)$);
			\coordinate (B) at ($(0,0)+(-120:1cm)$);
			\draw[fill=black] (0,0)--(A)--(B)--cycle;
			\coordinate (C) at ($(10,0)+(-60:1cm)$);
			\coordinate (D) at ($(10,0)+(-120:1cm)$);
			\draw[fill=black] (10,0)--(C)--(D)--cycle;
		\end{tikzpicture}
	\end{center}
	Người ta chứng minh được $f''(x)=\dfrac{1}{100}\left(2x-\dfrac{x^2}{2}\right)$ với $0\le x\le 4$. Tại điểm cách $P$ một khoảng $1$ mét, tấm ván bị võng xuống bao nhiêu cm? (làm tròn kết quả đến hàng phần trăm).
	
	\shortans[0]{$2{,}38$}
	\loigiai{
		Ta có $\displaystyle  f'(x)=\int f''(x)\mathrm{\,d}x=\int \dfrac{1}{100}\left(2x-\dfrac{x^2}{2}\right)\mathrm{\,d}x=\dfrac{1}{100}\left(x^2-\dfrac{x^3}{6}\right)+C_1$.\\
		Suy ra $\displaystyle f(x)=\int f'(x)\mathrm{\,d}x=\dfrac{1}{100}\left(\dfrac{x^3}{3}-\dfrac{x^4}{24}\right)+C_1x+C_2$.\\
		Vì $f(0)=0$ và $f(4)=0$ nên ta có
		\[
		\heva{
			&\dfrac{1}{100}\left(\dfrac{0^3}{3}-\dfrac{0^4}{24}\right)+C_1\cdot 0+C_2=0\\
			&\dfrac{1}{100}\left(\dfrac{4^3}{3}-\dfrac{4^4}{24}\right)+C_1\cdot 4+C_2=0
		}\Leftrightarrow \heva{
			&C_2=0\\
			&C_1=-\dfrac{2}{75}.
		}
		\]
		Vậy $f(x)=\dfrac{1}{100}\left(\dfrac{x^3}{3}-\dfrac{x^4}{24}\right)-\dfrac{2}{75}x$.\\
		Suy ra $f(3)=-\dfrac{19}{800}$ (m).\\
		Tại điểm cách $P$ $1$ (m), tấm ván bị võng xuống khoảng $2{,}38$ (cm).
	}
\end{ex}
\begin{ex}%Câu 5
	Một tấm bìa cứng có kích thước $60\text{ cm}\times 90\text{ cm}$ được gấp đôi thành một hình chữ nhật $60\text{ cm}\times 45\text{ cm}$ như hình vẽ. Sau đó, cắt ra từ các góc của hình chữ nhật vừa gấp bốn hình vuông bằng nhau có cạnh $x$ (cm). Tấm bìa được mở ra và sáu mép được gấp lên để tạo thành một hộp chữ nhật $(H)$ có nắp và đáy (như hình vẽ). Thể tích lớn nhất của khối $(H)$ bằng bao nhiêu lít? Làm tròn đến hàng phần mười.
	\begin{center}
		\begin{tikzpicture}[scale=1,>=stealth, font=\footnotesize, line join=round, line cap=round]
			\draw (0,0)--(4.5,0)--(4.5,-3)--(0,-3)--cycle;
			\draw[dashed] (2.25,0)--(2.25,-3);
			\draw (5,-0.5)--(5.5,-0.5)--(5.5,0)--(6.75,0)--(6.75,-0.5)--(7.25,-0.5)--(7.25,-2.5)--(6.75,-2.5)--(6.75,-3)--(5.5,-3)--(5.5,-2.5)--(5,-2.5)--cycle (5.25,-0.5)node[below]{$x$} (5.25,-2.5)node[above]{$x$} (7,-0.5)node[below]{$x$} (7,-2.5)node[above]{$x$} (5.5,-0.25)node[right]{$x$} (5.5,-2.75)node[right]{$x$} (6.75,-0.25)node[left]{$x$} (6.75,-2.75)node[left]{$x$};
			\draw[xshift=0.5cm] (9.5,-0.5)--(10,-0.5)--(10,0)--(11.25,0)--(11.25,-0.5)--(12.25,-0.5)--(12.25,0)--(13.5,0)--(13.5,-0.5)--(14,-0.5)--(14,-2.5)--(13.5,-2.5)--(13.5,-3)--(12.25,-3)--(12.25,-2.5)--(11.25,-2.5)--(11.25,-3)--(10,-3)--(10,-2.5)--(9.5,-2.5)--cycle;
			\draw[dashed,xshift=0.5cm] (10,-0.5)--(11.25,-0.5)--(11.25,-2.5)--(10,-2.5)--cycle (12.25,-0.5)--(13.5,-0.5)--(13.5,-2.5)--(12.25,-2.5)--cycle;
			\draw[|<->|] (-0.75,0)--(-0.75,-3) node[pos=.5,fill=white]{$60$ cm};
			\draw[|<->|] (8,0)--(8,-3) node[pos=.5,fill=white]{$60$ cm};
			\draw[|<->|] (9.25,0)--(9.25,-3) node[pos=.5,fill=white]{$60$ cm};
			\draw[|<->|] (0,-3.5)--(4.5,-3.5) node[pos=.5,fill=white]{$90$ cm};
			\draw[|<->|] (5,-3.5)--(7.25,-3.5) node[pos=.5,fill=white]{$45$ cm};
			\draw[|<->|] (10,-3.5)--(14.5,-3.5) node[pos=.5,fill=white]{$90$ cm};
		\end{tikzpicture}
	\end{center}
	\shortans[0]{$20{,}5$}
	\loigiai{
		Sau khi cắt bốn hình vuông cạnh $x, (0<x<\dfrac{45}{2})$ cm và gấp tấm bìa, kích thước của hình hộp là
		\begin{itemize}
			\item Chiều dài $60-2x$;
			\item Chiều rộng $45-2x$;
			\item Chiều cao $2x$.
		\end{itemize}    
		Thể tích khối hộp là
		\[
		V(x)=2x(60-2x)(45-2x)=8x^3-420x^2+5\,400x.
		\]
		Ta có $V'(x)=24x^2-840x+5400; V'(x)=0 \Leftrightarrow \hoac{
			&x=\dfrac{35+5\sqrt{13}}{2}\approx 36{,}18\,\,(\text{loại})\\
			&x=\dfrac{35-5\sqrt{13}}{2}\approx 13{,}82\,\,(\text{nhận}).
		}$\\
		Bảng biến thiên
		\begin{center}
			\begin{tikzpicture}
				\tkzTabInit[nocadre,lgt=2.2,espcl=2.5,deltacl=0.5]
				{$x$/1.6,$V'(x)$/0.6,$V(x)$/2}
				{$0$,$\dfrac{35-5\sqrt{13}}{2}$,$\dfrac{45}{2}$}
				\tkzTabLine{,-,0,+,}
				\tkzTabVar{-/$ $,+/$V\left(\dfrac{35-5\sqrt{13}}{2}\right)$,-/$ $}
			\end{tikzpicture}
		\end{center}
		Vậy thể tích lớn nhất của khối hộp là $V\left(\dfrac{35-5\sqrt{13}}{2}\right)\approx 20\,468{,}04 \,\,\rm{cm^3}\approx 20{,}5$ lít.
	}
\end{ex}
\begin{ex}%Câu 6
	Trong không gian $Oxyz$, cho hai điểm $A(5;0;6)$ và $B(3;5;0)$. Điểm $M$ di động trên trục $Oz$, điểm $N$ di động trên trục $Oy$. Độ dài đường gấp khúc $AMNB$ có độ dài nhỏ nhất bằng bao nhiêu? (Kết quả làm tròn đến hàng phần chục).
	
	\shortans[0]{$13{,}5$}
	\loigiai{
		Để tìm độ dài ngắn nhất của đường gấp khúc $AMNB$ ta sẽ \lq\lq trải\rq\rq \, các điểm $A$, $B$ về cùng $1$ mặt phẳng $(Oyz)$ với các điểm $M$, $N$ và thoả mãn đoạn thẳng mới bằng với đoạn thẳng ban đầu (tức
		$AM=A'M; BM=BM'$) và đoạn gấp khúc ngắn nhất khi $4$ điểm trên thẳng hàng.\\
		Ta quay vuông góc mặt phẳng chứa điểm $A(5; 0; 6)$; (tức mặt phẳng màu xanh) xuống mặt phẳng $(Oyz$) ta được điểm $A'(0;-5; 6)$.\\
		Giả sử điểm $M(0; 0; z) \in Oz$. \\
		Suy ra $\heva{&AM=\sqrt{5^2+(z-6)^2}\\& A'M=\sqrt{(-5)^2+(z-6)^2}} \Rightarrow AM=A'M$.\\
		Tương tự, ta quay vuông góc mặt phẳng chứa điểm $B(3; 5; 0)$ (tức mặt phẳng màu hồng) xuống mặt phẳng $(Oyz)$ ta được điểm $B'(0;5;-3)$.\\
		Giả sử điểm $N(0; y; 0) \in Oy$. Suy ra $BN=BN'$.\\
		Ta có độ dài đường gấp khúc $AMNB=A'M+MN+NB'$.\\
		Suy ra $\left(AM'+MN+N'B\right)_{min}$ xảy ra khi $A'$, $M$, $N$, $B'$ thẳng hàng và bằng $A'B'$.
		\[A'B'=\sqrt{(5+5)^2+(-3-6)^2}=\sqrt{181}\approx 13{,}5.\]
	}
\end{ex}
\Closesolutionfile{ans}
% \begin{name}
	{\tenchude}
	{\tendethi}
	{\tentruong}
	{\thoigian}
	\end{name}
\TN
\Opensolutionfile{ans}[ans/de7-ABCD]
\begin{ex}%[1D2H1-2]%Câu 1
	Cho dãy số $(u_n)$ với $u_n=\dfrac{2n}{3n+2}$, $n\in\mathbb{N}^*$. Khẳng định nào sau đây đúng?\vspace{3pt}
	\choice
	{$u_2=1$}
	{\True $u_2=\dfrac{1}{2}$}
	{$u_2=-\dfrac{1}{2}$}
	{$u_2=\dfrac{1}{3}$}
	\loigiai{
		Ta có $(u_n)$ với 
		$\begin{aligned}[t]
			u_n&=\dfrac{2n}{3n+2}, n\in\mathbb{N}^*.
		\end{aligned}$\\
	Suy ra $u_2=\dfrac{2\cdot2}{3\cdot2+2}=\dfrac{4}{8}=\dfrac{1}{2}$.
	}
\end{ex}
\begin{ex}%[2H5H2-3]%Câu 2
	Trong không gian $Oxyz$, cho đường thẳng $d\colon\dfrac{x-2}{2}=\dfrac{y+1}{-1}=\dfrac{z+2}{2}$. Đường thẳng $d$ đi qua điểm nào dưới đây?
	\choice
	{$A(2;-1;2)$}
	{$B(-2;1;2)$}
	{$C(2;1;2)$}
	{\True $D(2;-1;-2)$}
	\loigiai{
		Đường thẳng $d\colon\dfrac{x-2}{2}=\dfrac{y+1}{-1}=\dfrac{z+2}{2}=t$.\\
		Phương trình tham số $d\colon\heva{&x=2+3t\\ &y=-1-t\\ &z=-2+2t.}$\\
		Chọn $t=0$, đường thẳng $d$ đi qua điểm $D(2;-1;-2)$.
	}
\end{ex}
\begin{ex}%[1D6H2-2]%Câu 3
	Với $a$, $b$ là các số thực dương khác $1$ thỏa mãn $\log_a b=\dfrac{3}{2}$. Giá trị của biểu thức $\log_a b^4$ bằng\vspace{3pt}
	\choice
	{$\dfrac{8}{3}$}
	{$3$}
	{$\dfrac{4}{3}$}
	{\True $6$}
	\loigiai{
		Ta có $\log_ab^4=4\log_ab$.\\
		Mặt khác bài cho $\log_ab=\dfrac{3}{2}$.\\
		Suy ra $\log_ab^4=4\cdot\dfrac{3}{2}=6$.
	}
\end{ex}
\begin{ex}%[1H8H7-2]%Câu 4
	Cho khối chóp $S.ABCD$ có chiều cao bằng $5$, đáy $ABCD$ là hình bình hành có diện tích bằng $6$. Thể tích khối chóp $S.ABC$ bằng
	\choice
	{$10$}
	{$15$}
	{\True $5$}
	{$30$}
	\loigiai{
		Ta có công thức thể tích $V_{\text{chóp}}=\dfrac{1}{3}\cdot S_d\cdot h$.\\
		Suy ra $V_{S.ABC}=\dfrac{1}{3}\cdot S_{\triangle ABC}\cdot h$.\\
		Mặt khác ta có $S_{\triangle ABC}=\dfrac{1}{2}S_{ABCD}=\dfrac{1}{2}\cdot6=3$.\\
		Suy ra $V_{S.ABC}=\dfrac{1}{3}\cdot S_{\triangle ABC}\cdot h=\dfrac{1}{3}\cdot3\cdot5=5$.
	}
\end{ex}
\begin{ex}%[2H5H1-3]%Câu 5
	Trong không gian $Oxyz$, mặt phẳng $(\alpha)\colon x+2y+3z-6=0$ cắt trục tung tại điểm có tung độ bằng
	\choice
	{$2$}
	{$6$}
	{\True $3$}
	{$1$}
	\loigiai{
		Trong không gian $Oxyz$, mặt phẳng $(\alpha)\colon x+2y+3z-6=0$ cắt trục tung, suy ra mặt phẳng $(\alpha)$ sẽ cắt trục tung tại điểm $(0;a;0)$
		\begin{eqnarray*}
			 \Rightarrow 1\cdot0+2\cdot a+3\cdot0-6=0 \Leftrightarrow a=3.
		\end{eqnarray*}
	Vậy mặt phẳng $(\alpha)\colon x+2y+3z-6=0$ cắt trục tung tại điểm có tung độ bằng $3$.
	}
\end{ex}
\textbf{\textit{Sử dụng thông tin dưới đây để trả lời câu \ref{câu 6-đề 7} và câu \ref{câu 7-đề 7}}}\\[0.5em]
Cho hàm số $y=f(x)$ có bảng biến thiên như sau
\begin{center}
	\begin{tikzpicture}
		\tikzset{double style/.append style={double distance=1.5pt}}
		\tkzTabInit[nocadre=false,lgt=1.2,espcl=2.5,deltacl=0.6]
		{$x$ /0.6,$y'$ /0.6,$y$ /2}
		{$-\infty$,$-2$,$3$,$+\infty$}
		\tkzTabLine{,+,d,-,$0$,+,}
		\tkzTabVar{-/$5$,+D+/$+\infty$/$4$,-/$0$,+/$+\infty$}
	\end{tikzpicture}
\end{center}
\begin{ex}%[2D1V3-1]%Câu 6
	\label{câu 6-đề 7}
	Giá trị nhỏ nhất của hàm số $y=f(x)$ trên đoạn $[0;6]$ bằng
	\choice
	{\True $0$}
	{$4$}
	{$f(4)$}
	{$f(6)$}
	\loigiai{
		Từ bảng biến thiên, ta thấy giá trị nhỏ nhất của hàm số $y=f(x)$ trên đoạn $\left[0;6\right]$ bằng $0$.
	}
\end{ex}
\begin{ex}%[2D1H4-1]%Câu 7
	\label{câu 7-đề 7}
	Tổng số đường tiệm cận đứng và tiệm cận ngang của đồ thị hàm số $y=f(x)$ là
	\choice
	{\True $2$}
	{$1$}
	{$3$}
	{$0$}
	\loigiai{
		Từ bảng biến thiên, ta có
		$
			\lim\limits_{x\to-\infty}y=5; \lim\limits_{x \to +\infty} y=+\infty
		$
		nên đồ thị hàm số $y=f(x)$ có một tiệm cận ngang.\\
		Ta lại có $\lim\limits_{x \to -2^-} y=+\infty$ nên đồ thị hàm số $y=f(x)$ có một tiệm cận đứng.\\
		Vậy tổng số đường tiệm cận đứng và tiệm cận ngang của đồ thị hàm số $y=f(x)$ là $2$.
	}
\end{ex}
\begin{ex}%[1D6H4-2]%Câu 8
	Tập nghiệm của bất phương trình $3^{2x-3}\le\dfrac{1}{3}$ là
	\choice
	{\True $(-\infty;1]$}
	{$[1;+\infty)$}
	{$(-\infty;2]$}
	{$[2;+\infty)$}
	\loigiai{
		Ta có 
		$\begin{aligned}[t]
			3^{2x-3}\leq\dfrac{1}{3}
			& \Leftrightarrow 3^{2x-3}\leq3^{-1}\\
			& \Leftrightarrow 2x-3\leq-1\\
			& \Leftrightarrow 2x\leq2\\
			& \Leftrightarrow x\leq1.
		\end{aligned}$\\
	Vậy tập nghiệm của bất phương trình là $\left(-\infty;1\right]$.
	}
\end{ex}

\begin{ex}%[2H2H1-2]%Câu 9
	% \immini[thm]{
		Cho hình lập phương $ABCD.A'B'C'D'$ có cạnh bằng $a$. Độ dài của vectơ $\overrightarrow{u}=\overrightarrow{A'C'}-\overrightarrow{A'A}$ bằng
	\choice
	{$\sqrt{2}a$}
	{$\dfrac{\sqrt{3}a}{2}$}
	{$\sqrt{6}a$}
	{\True $\sqrt{3}a$}
	% }{
	% 	\begin{tikzpicture}[line cap=round, line join=round, font=\footnotesize, thick]
	% 		\def\a{2.75}
	% 		\path 	(0:0) coordinate (A)
	% 				++(0:\a) coordinate (D)
	% 				++(-130:\a/2) coordinate (C)
	% 				($(A)+(C)-(D)$) coordinate (B)
	% 				($(A)+(90:\a)$) coordinate (A')
	% 				($(B)+(90:\a)$) coordinate (B')
	% 				($(C)+(90:\a)$) coordinate (C')
	% 				($(D)+(90:\a)$) coordinate (D');
	% 		\draw[dashed,thin] (B)--(A)--(D) (A)--(A');
	% 		\draw[thick] (C)--(C') (D)--(D') (B)--(B') (B)--(C)--(D) (A')--(B')--(C')--(D')--cycle;
	% 		\foreach \x/\y in {A/180,B/180,C/0,D/0,A'/180,B'/180,C'/0,D'/0}
	% 			\fill[black] (\x) circle (1pt) ($(\y:3mm)+(\x)$) node {$\x$};	
	% 	\end{tikzpicture}
	% }
	\loigiai{
		Theo qui tắc ba điểm, ta có $\overrightarrow{u} =\overrightarrow{A'C'}-\overrightarrow{A'A} =\overrightarrow{AC'}$.\\
		Suy ra độ dài $|\overrightarrow{u}| =\left|\overrightarrow{A'C'}-\overrightarrow{A'A}\right| =\left|\overrightarrow{AC'}\right| =AC'$.
		\immini{
			Có $ABCD.A'B'C'D'$ là hình lập phương cạnh $a$ và có $AC'$ là đường chéo của hình lập phương nên ta suy ra
			\begin{align*}
				AC'&=\sqrt{AC^2+CC'^2}=\sqrt{AB^2+AD^2+CC'^2} \\
				&=\sqrt{AB^2+AD^2+AA'^2}=\sqrt{a^2+a^2+a^2}=a\sqrt{3}.
			\end{align*}
		}{
			\begin{tikzpicture}[line cap=round, line join=round, font=\footnotesize, thick, blue, >=stealth]
				\def\a{2.75}
				\path 	(0:0) coordinate (A)
				++(0:\a) coordinate (D)
				++(-130:\a/2) coordinate (C)
				($(A)+(C)-(D)$) coordinate (B)
				($(A)+(90:\a)$) coordinate (A')
				($(B)+(90:\a)$) coordinate (B')
				($(C)+(90:\a)$) coordinate (C')
				($(D)+(90:\a)$) coordinate (D');
				\draw[dashed,thin,->,magenta] (A)--(A');
				\draw[dashed,thin,->,magenta] (A)--(B);
				\draw[dashed,thin,->,magenta] (A)--(D);
				\draw[thick] (C)--(C') (D)--(D') (B)--(B') (B)--(C)--(D) (A')--(B')--(C')--(D')--cycle;
				\foreach \x/\y in {A/180,B/180,C/0,D/0,A'/180,B'/180,C'/0,D'/0}
				\fill[black] (\x) circle (1pt) ($(\y:3mm)+(\x)$) node {$\x$};
				\draw[->,very thick,magenta] (A)--(C');
			\end{tikzpicture}
		}
		
	}
\end{ex}
\begin{ex}%[1D5H2-3]%Câu 10
	Một vườn thú ghi lại tuổi thọ (đơn vị: năm) của $20$ con hổ và thu được kết quả như sau
	\begin{center}
		\begin{tabular}{|c|c|c|c|c|c|}\hline
			Tuổi thọ & $[14;15)$ & $[15;16)$ & $[16;17)$ & $[17;18)$ & $[18;19)$ \\ \hline
			Số con hổ & $1$ & $3$ & $8$ & $6$ & $2$ \\ \hline
		\end{tabular}
	\end{center}
	Nhóm chứa tứ phân vị thứ nhất là
	\choice
	{$[14;15)$}
	{$[15;16)$}
	{\True $[16;17)$}
	{$[17;18)$}
	\loigiai{
		Số con hổ được khảo sát là $n=20$.\\
		Gọi $x_1,x_2,\ldots,x_{20}$ là tuổi thọ của $20$ con hổ được sắp xếp theo thứ tự không giảm.\\
		Ta có 
		$\begin{aligned}[t]
			x_1 & \in \left[14;15\right); \\
			x_2,x_3,x_4 & \in \left[15;16\right); \\
			x_5,x_6,\ldots,x_{12} & \in \left[16;17\right); \\
			x_{13},x_{14},\ldots,x_{18} & \in \left[17;18\right); \\
			x_{19},x_{20} & \in \left[18;19\right).
		\end{aligned}$\\
	Do đó đối với dãy số liệu $x_1, x_2, \ldots, x_{20}$ thì tứ phân vị thứ nhất của dãy số $x_1, x_2, \ldots, x_{20}$ là $\dfrac{1}{2}\left(x_5+x_6\right)$.\\
	Do đó $x_5, x_6$ thuộc nhóm $\left[16;17\right)$ nên tứ phân vị thứ nhất thuộc nhóm $\left[16;17\right)$.
	}
\end{ex}
\begin{ex}%[2D4V3-4]%Câu 11

		Cho hình phẳng $(H)$ giới hạn bởi đồ thị hàm số $y=4-x^2$ và trục hoành. Thể tích của khối tròn xoay được tạo thành khi quay $(H)$ xung quanh trục $Ox$ bằng\vspace{3pt}
	\choice
	{$\dfrac{32\pi}{3}$}
	{$\dfrac{512}{15}$}
	{\True $\dfrac{512\pi}{15}$}
	{$\dfrac{32}{3}$}
	
	\loigiai{
		\immini{
		Ta có phương trình hoành độ giao điểm của $y=4-x^2$ và trục hoành $y=0$
		$$4-x^2=0 \Leftrightarrow x=\pm2.$$
		Thể tích khối tròn xoay được tạo thành khi quay $(H)$ xung quanh trục $Ox$ là
		\begin{align*}
			V=\pi\displaystyle\int\limits_{-2}^2 \left(4-x^2\right)^2 \mathrm{\,d}x =\dfrac{512}{15}\pi.
		\end{align*}
		}{
		\begin{tikzpicture}[scale=0.8,>=stealth, font=\footnotesize, line join=round, line cap=round, thick]
			\def\a{-1} \def\b{0} \def\c{4} % Hệ số
			\def\xmin{-3} \def\xmax{3}
			\def\ymin{-1} \def\ymax{5}
			\fill[gray!60] plot[domain=-2:2](\x,{\a*(\x)^2+\b*(\x)+\c})--cycle;
			\draw[->] (\xmin,0)--(\xmax,0) node [below]{$x$};
			\draw[->] (0,\ymin)--(0,\ymax) node [left]{$y$};
			\node at (0,0) [below left]{$O$};
			\clip (\xmin+0.1,\ymin+0.1) rectangle (\xmax-0.1,\ymax-0.1);
			\draw[smooth,samples=300] plot(\x,{\a*(\x)^2+\b*(\x)+\c});
			\draw (0,4)node[above right]{$y=4-x^2$} (-0.75,1)node[]{$(H)$};
		\end{tikzpicture}
	}
	}
\end{ex}
\begin{ex}%[2D4H1-2]%Câu 12
	Cho hàm số $f(x)$ liên tục trên $\mathbb{R}$. Biết $F(x)$ là nguyên hàm của hàm số $f(x)$ thỏa mãn $F(2)=2$ và $F(x)=\displaystyle\int \left[x-f(x)\right] \mathrm{\,d}x$, $\forall x\in\mathbb{R}$, Giá trị của $F(4)$ bằng
	\choice
	{\True $5$}
	{$6$}
	{$8$}
	{$9$}
	\loigiai{
		Ta có 
		$\begin{aligned}[t]
			F(x)=\displaystyle\int x \mathrm{\,d}x -\displaystyle\int f(x) \mathrm{\,d}x
			=\dfrac{x^2}{2}+C-F(x) \Leftrightarrow 2F(x) =\dfrac{x^2}{2}+C.
		\end{aligned}$\\
	Tại $x=2$ ta có $2F(2)=2+C \Leftrightarrow C=2$.\\
	Suy ra $x=4$ ta được $2F(4)=\dfrac{4^2}{2}+C \Leftrightarrow 2F(4)=\dfrac{4^2}{2}+2 \Leftrightarrow 2F(4)=10 \Leftrightarrow F(4)=5$.
	}
\end{ex}

\Closesolutionfile{ans}

\inputansbox{6}{ans/de7-ABCD}

\TNTF

\Opensolutionfile{ans}[ans/de7-DS]

\begin{ex}%[1D7H2-1]%Câu 1
	Cho hàm số $f(x)=\ln\left(x^2-2x+1\right)-x$. Xét tính đúng sai của các mệnh đề sau
	\choiceTF
	{Tập xác định của hàm số là $\mathscr{D}=\mathbb{R}$}
	{\True Đạo hàm của hàm số $f(x)$ trên tập xác định của nó là $f'(x)=\dfrac{2}{x-1}-1$}
	{\True Số nghiệm của phương trình $f'(x)=0$ trên khoảng $(1;+\infty)$ là $1$}
	{Giá trị cực đại của $f(x)$ trên khoảng $(1;+\infty)$ là $a\ln 2+b$ với $a$, $b\in\mathbb{Z}$ thì $a+b=1$}
	\loigiai{
		\begin{itemchoice}
			\itemch Điều kiện $x^2-2x+1>0 \Leftrightarrow (x-1)^2>0 \Leftrightarrow x\ne1$.\\
			Suy ra tập xác định của hàm số là $\mathscr{D}=\mathbb{R}\setminus\{1\}$.
			\itemch Áp dụng công thức $\left(\ln u\right)'=\dfrac{u'}{u}$.\\
			Ta có $f(x)=\ln\left(x^2-2x+1\right)-x $.\\
			$\Rightarrow f'(x)=\dfrac{(x^2-2x+1)'}{x^2-2x+1}-1 =\dfrac{2x-2}{x^2-2x+1}-1 =\dfrac{2(x-1)}{(x-1)^2}-1 =\dfrac{2}{x-1}-1$.
			\itemch Ta có $f'(x)=0 \Leftrightarrow \dfrac{2}{x-1}-1=0 \Leftrightarrow x-1=2 \Leftrightarrow x=3$.\\
			Suy ra $f'(x)=0$ có nghiệm duy nhất.
			\itemch Lập bảng biến thiên $ \Rightarrow x=3$ là điểm cực đại của hàm số.\\
			Do đó $y_{CD}=f(3)=\ln4-3=2\ln2-3=a\ln2+b$.\\
			Suy ra $a=2, b=-3$ nên $a+b=2+(-3)=-1$.
		\end{itemchoice}
	}
\end{ex}

\begin{ex}%[2D5V1-2]%[2D5V1-3]%Câu 2
	Một nhà máy sản xuất bóng đèn có tỷ lệ bóng đèn đạt tiêu chuẩn là $82\%$. Trước khi xuất ra thị trường, mỗi bóng đèn được sản xuất ra đều phải qua một khâu kiểm tra chất lượng tự động. Vì sự kiểm tra này không chính xác tuyệt đối nên một bóng đèn tốt chỉ có xác suất $92\%$ được công nhận và một bóng đèn hỏng có xác suất $96\%$ được loại bỏ.
	\begin{itemize}
		\item Gọi $A$ là biến cố \lq\lq Bóng được công nhận đạt tiêu chuẩn sau khi qua kiểm tra chất lượng\rq\rq.
		\item Gọi $B$ là biến cố \lq\lq Sản phẩm đạt tiêu chuẩn\rq\rq.
	\end{itemize}
	Xét tính đúng sai của các mệnh đề sau
	\choiceTF[t]
	{$\mathrm{P}(B)=0{,}18$ ; $\mathrm{P}(\overline{B})=0{,}82$}
	{Xác suất có điều kiện $\mathrm{P}(A|\overline{B})=0{,}92$}
	{\True Tỉ lệ bóng được công nhận đạt tiêu chuẩn sau khi qua kiểm tra chất lượng là $76{,}16\%$}
	{Tỷ lệ bóng đèn tốt trong số những bóng đèn được công nhận là $98{,}01\%$ (kết quả làm tròn đến hàng phần trăm)}
	\loigiai{
		\textbf{Chú ý:} Bóng đèn đạt tiêu chuẩn là bóng đèn tốt và bóng đèn không đạt tiêu chuẩn là bóng đèn hỏng.
		\begin{itemchoice}
			\itemch Theo bài ra ta có $\mathrm{P}(B)=0{,}82$; $\mathrm{P}(\overline{B})=1-0{,}82=0{,}18$.
			\itemch Do tỉ lệ công nhận bóng đèn đạt tiêu chuẩn là $0{,}92$ nên $\mathrm{P}(A|B)=0{,92}$.\\
			Tie lệ loại bỏ một bóng đèn hỏng là $0{,}96$ nên $\mathrm{P}(A|\overline{B})=1-0{,}92=0{,}04$.\\
			Theo sơ đồ hình cây
			\itemch Theo công thức xác suất toàn phần ta có $$\mathrm{P}(A)=\mathrm{P}(B)\cdot\mathrm{P}(A|B)+\mathrm{P}(\overline{B})\cdot\mathrm{P}(A|\overline{B}) =0{82\cdot0{,}92}+0{,}18\cdot0{,}04 =0{,}7616.$$
			\itemch Tỉ lệ bóng đèn tốt trong số những bóng đèn được công nhận là $$\dfrac{0{,}82\cdot0{,}92}{0{,}82\cdot0{,}92+0{,}18\cdot0{,}04}=99{,}05.$$
		\end{itemchoice}
	}
\end{ex}

\begin{ex}%[2H2C2-6]%Câu 3
	\immini[thm]{
		Xét hai chiếc khinh khí cầu bay lên từ cùng một điểm trong cùng một ngày. Lúc $9$ giờ sáng, chiếc thứ nhất đang ở vị trí $A$ cách điểm xuất phát $2$ km về phía nam và $1$ km về phía đông, đồng thời cách mặt đất $0{,}5$ km. Chiếc thứ hai đang ở vị trí $B$ nằm cách điểm xuất phát $1$ km về phía bắc và $1{,}5$ km về phía tây đồng thời cách mặt đất $0{,}8$ km. Chọn hệ trục tọa độ $Oxyz$ với gốc $O$ đặt tại điểm xuất phát của hai khinh khí cầu, mặt phẳng $(Oxy)$ trùn với mặt đất, trục $Ox$ hướng về phía nam, trục $Oy$ hướng về phía đông và trục $Oz$ hướng thẳng đứng lên trời (như hình vẽ). Lấy đơn vị đo trên mỗi trục là km.
	}{\begin{tikzpicture}[scale=1,font=\footnotesize,>=stealth, line join=round, line cap=round]
		\def\xmin{-2} \def\xmax{4}
		\def\ymin{-2} \def\ymax{2}
		\def\zmax{3}
		\draw[->] (\xmin,0)--(\xmax,0) node [below]{$x$};
		\draw[->] (\ymax,\ymax)--(\ymin,\ymin)node [above]{$y$};
		\draw[->] (0,0)--(0,\zmax) node [left]{$z$};
		\draw (\xmax,0)node[above]{Nam} (\xmin,0)node[above]{Bắc} (\ymax,\ymax)node[above]{Tây} (\ymin,\ymin)node[below]{Đông};
		\node at (0,0) [below]{$O$};
		\coordinate (O) at (0,0);
		\coordinate (E) at (3,0);
		\coordinate (F) at (-1,-1);
		\coordinate (G) at ($(E)+(F)-(O)$);
		\coordinate (A) at ($(G)+(0,1.7)$);
		\draw[dashed] (F)--(G)--(E) (O)--(G) (O)--(A)node[above,blue,font=\fontsize{25pt}{2pt}\selectfont]{\faFly}--(G);
		\fill (A)circle(2pt);
		\coordinate (M) at (-1.5,0);
		\coordinate (N) at (0.5,0.5);
		\coordinate (P) at ($(M)+(N)-(O)$);
		\coordinate (B) at ($(P)+(0,1.5)$);
		\draw[dashed] (M)--(P)--(N) (O)--(P) (O)--(B)node[above,magenta,font=\fontsize{25pt}{2pt}\selectfont]{\faFly}--(P);
		\fill (B)circle(2pt);
	\end{tikzpicture}}
	\choiceTF
	{\True Tọa độ của khinh khí cầu thứ nhất lúc $9$ giờ sáng là $A(2;1;0{,}5)$}
	{Phương trình chính tắc của đường thẳng $AB$ là $\dfrac{x-2}{30}=\dfrac{y-1}{25}=\dfrac{z-0{,}5}{3}$}
	{\True Lúc $9$ giờ sáng, khoảng cách giữa hai chiếc khinh khí cầu là $3{,}92$ km (làm tròn đến hàng phần trăm)}
	{Từ $9$ giờ sáng đến $9$ giờ $10$ phút sáng, khinh khí cầu thứ nhất đi thẳng về hướng Nam với vận tốc $50$ km/h và độ cao không đổi để đến điểm $M$, khinh khí cầu thứ hai chuyển động thẳng đến điểm $N$ với vận tốc $60$ km/h, biết vectơ $\overrightarrow{BN}$ cùng hướng với vectơ $\overrightarrow{u}=(2;2;1)$. Bỏ qua lực cản của gió, khoảng cách $MN$ là $4{,}66$ km (làm tròn đến hàng phần trăm)}
	\loigiai{
		\begin{itemchoice}
			\itemch Vị trí của khinh khí cầu thứ nhất lúc $9$ giờ sáng là $A(x_A;y_A;z_A)$ biết
			\begin{itemize}
				\item $A$ cách điểm xuất phát $2$ km về phía Nam $ \Rightarrow x_A=2$.
				\item $1$ km về phía Đông $ \Rightarrow y_A=1$.
				\item Cách mặt đất $0{,}5$ km $ \Rightarrow z_A=0{,}5$.
			\end{itemize}
		Vậy $A(2;1;0{,}5)$.
			\itemch Dựa vào giả thiết, ta có $B(-1;-1{,}5;0{,}8)$.\\
			Suy ra $\overrightarrow{AB}=(-3;-2{,}5;0{,3})$.\\
			Đường thẳng $AB$ đi qua điểm $A(2;1;0{,}5)$ và có vec-tơ chỉ phương $\overrightarrow{u}=(30;25;-3)$ có phương trình chính tắc là $\dfrac{x-2}{30}=\dfrac{y-1}{25}=\dfrac{x-0{,}5}{-3}$.
			\itemch Lúc $9$ giờ sáng, khoảng cách giữa hai kinh khí cầu bằng $$AB=\sqrt{(-3)^2+(-2{,}5)^2+(0{,}3)^2}\approx3{,}92\, (\rm km).$$
			\itemch Để tính được khoảng cách $MN$ ta cần tìm được toạ độ của điểm $M$ và $N$ lúc $9$ giờ $10$ phút sáng
			\begin{itemize}
				\item Tìm điểm $M(x_M;y_M;z_M)$.\\
				Ta có khinh khí cầu thứ nhất di chuyển thẳng đều về phía Nam (tức hoành độ của khinh khí cầu không đổi $ \Leftrightarrow y_M=1$) với vận tốc $50\,\rm km/h$ suy ra từ $9$ giờ sáng đến $9$ giờ $10$ phút sáng, khinh khí cầu đi được $50\cdot\dfrac{1}{6}=\dfrac{25}{3}$ (km).\\
				$ \Rightarrow x_M=2+\dfrac{25}{3}=\dfrac{31}{3}$.\\
				Và cao độ không đổi $ \Rightarrow z_M=0{,}5$.\\
				Vậy điểm $M(\dfrac{31}{3};1;0{,}5)$.
				\item Tìm điểm $N(x_N;y_N;z_N)$\\
				Vì vectơ $\overrightarrow{BN}$ cùng hướng với vectơ $\overrightarrow{u}$ nên suy ra $\overrightarrow{BN}=k(2;2;1)\, (k>0)$.\\
				Khinh khí cầu thứ hai chuyển động thẳng đều đến điểm $N$ với vận tốc $60\, \rm km/h$, nên trong khoảng thời gian từ $9$ giờ sáng đến $9$ giờ $10$ phút sáng khinh khí cầu đi được $1$ đoạn $BN=60\cdot\dfrac{1}{6}=10 \Leftrightarrow k|(2;2;1)|=10 \Leftrightarrow k\sqrt{2^2+2^1+1^1}=10 \Leftrightarrow k=\dfrac{10}{3}$.\\
				$\overrightarrow{BN}=\left(x_N+1;y_N+1;z_N-0{,}8\right) =\dfrac{10}{3}(2;2;1) =\left(\dfrac{20}{3};\dfrac{20}{3};\dfrac{10}{3}\right) \Rightarrow \heva{&x_N=\dfrac{17}{3}\\ &y_N=\dfrac{31}{6}\\ &z_N=\dfrac{62}{15}.}$\\
				$ \Rightarrow N\left(\dfrac{17}{3};\dfrac{31}{3};\dfrac{62}{15}\right)$.\\
				Suy ra $MN=\sqrt{\left(\dfrac{17}{3}-\dfrac{31}{3}\right)^2+\left(\dfrac{31}{3}-1\right)^2+\left(\dfrac{62}{15}-0{,}5\right)^2} \approx 7{,}23$.
			\end{itemize}
		\end{itemchoice}
	}
\end{ex}

\begin{ex}%[2D4C3-2]%Câu 4
	\immini[thm]{
		Hình vẽ bên mô tả hiệu suất làm việc của hai công nhân trong một nhà máy trong thời gian $6$ giờ. Công nhân $A$ đang sản xuất với hiệu suất $Q'_1(t)=-2t^2+4t+58$ sản phẩm mỗi giờ, trong khi công nhân $B$ đang sản xuất với hiệu suất $Q'_2(t)=53+at$ sản phẩm mỗi giờ $(a\in\mathbb{R})$. Biết rằng hàm $Q_1(t)$ và $Q_2(t)$ mô phỏng số lượng sản phẩm mới làm được của công nhân $A$ và công nhân $B$ sau $t$ giờ. Xét tính đúng sai của các mệnh đề sau
	}{
		\begin{tikzpicture}[>=stealth, font=\footnotesize, line join=round, line cap=round,xscale=1,yscale=0.1,scale=0.48]
			\def\xmin{-1} \def\xmax{8}
			\def\ymin{-10} \def\ymax{70}
			\draw[->] (\xmin,0)--(\xmax,0) node [below]{$t$ (h)};
			\draw[->] (0,\ymin)--(0,\ymax) node [left]{$Q'(t)$};
			\node at (0,0) [below left]{$O$};
			\draw[dashed] (5,0)node[below]{$5$}--(5,28) (6,0)node[below]{$6$}--(6,10)node[right]{$Q'_1(t)$}--(6,23)node[right]{$Q'_2(t)$};
			\clip (\xmin+0.1,\ymin+0.1) rectangle (\xmax-0.5,\ymax-0.1);
			\draw[smooth,samples=300,domain=0:6] plot(\x,{-2*(\x)^2+4*(\x)+58});
			\draw[smooth,samples=300,domain=0:6] plot(\x,{-5*(\x)+53});
			\fill[gray] plot[domain=0:6](\x,{-2*(\x)^2+4*(\x)+58})--plot[domain=6:0](\x,{-5*(\x)+53})--cycle;
		\end{tikzpicture}
	}
	\choiceTF
	{\True Hiệu suất cực đại của công nhân $A$ là $60$ sản phẩm mỗi giờ}
	{Phần diện tích tô đậm biểu diễn cho tổng số lượng sản phẩm mới mà $2$ công nhân làm được trong $6$ giờ}
	{\True Sau $5$ giờ số lượng sản phẩm mới mà công nhân $A$ hoàn thành nhiều hơn công nhân $B$ là $54$ sản phẩm (kết quả làm tròn đến hàng đơn vị)}
	{Sau $6$ giờ làm việc tổng số lượng sản phẩm mới mà $2$ công nhân hoàn thành là $502$ sản phẩm}
	\loigiai{
		\begin{itemchoice}
			\itemch Ta có hiệu suất của công nhân $A$ là\\ $Q_1'(t)=-2t^2+4t+58 =-2(t^2-2t+1)+2+58 =-2(t-1)^2+60 \leq60$, $\forall t\in\left[0;6\right]$.\\
			Dấu \lq\lq$=$\rq\rq\, xảy ra khi $t=1$.\\
			Vậy hiệu suất cực đại của công nhân $A$ là $60$ sản phẩm mỗi giờ.
			\itemch Diện tích tô đậm bằng $\displaystyle\int\limits_0^6 \big|Q_1'-Q_2'\big| \mathrm{\,d}t$ nên nó không biểu diễn cho tổng số lượng sản phẩm mới mà 2 công nhân làm được trong $6$ giờ.
			\itemch Dựa vào biểu đồ, ta có 
			$$\begin{aligned}[t]
				Q_1'(5)=Q_2'(5) & \Leftrightarrow -2\cdot5^2+4\cdot5+58=53+5a\\
				& \Leftrightarrow a=-5.
			\end{aligned}$$
		Suy ra $Q_2'(5)=53-5t$.\\
		Sau 5 giờ, số lượng sản phẩm mới của công nhân $A$ hoàn thành nhiều hơn số lượng sản phẩm mới của công nhân $B$ bằng 
		\begin{align*}
			\displaystyle\int\limits_0^5 Q_1'(t) \mathrm{\,d}t -\displaystyle\int\limits_0^5 Q_2'(t) \mathrm{\,d}t =\displaystyle\int\limits_0^5 (-2t^2+4t+58) \mathrm{\,d}t -\displaystyle\int\limits_0^5 (53-5t) \mathrm{\,d}t =\dfrac{325}{6} \approx 54.
		\end{align*}
			\itemch Sau $6$ giờ làm việc tổng số lượng sản phẩm mới 2 công nhân hoàn thành bằng 
		\begin{align*}
			\displaystyle\int\limits_0^5 Q_1'(t) \mathrm{\,d}t +\displaystyle\int\limits_0^5 Q_2'(t) \mathrm{\,d}t =\displaystyle\int\limits_0^5 (-2t^2+4t+58) \mathrm{\,d}t +\displaystyle\int\limits_0^5 (53-5t) \mathrm{\,d}t =\dfrac{770}{3}+\dfrac{405}{2}= \dfrac{2755}{6} \approx 459.
		\end{align*}
		\end{itemchoice}
	}
\end{ex}

\Closesolutionfile{ans}

\inputansbox{3}{ans/de7-DS}

\TNSA

\Opensolutionfile{ans}[ans/de7-KQ]

\begin{ex}%[1H8V6-2]%Câu 1
	Cho hình chóp tam giác đều $S.ABC$ có $AB=2$, $SA=3$. Gọi $\alpha$ là số đo của góc nhị diện $[S,BC,A]$. Giá trị $\tan\alpha$ bằng bao nhiêu? (làm tròn kết quả đến hàng phần mười).
	\shortans[oly]{$4{,}8$}
	\loigiai{
		\begin{center}		
			\begin{tikzpicture}[line cap=round, line join=round, font=\footnotesize, thick, blue]
				\def\a{5} \def\h{3.5}
				\path 	(0:0) coordinate (A)
						(-45:2) coordinate (B)
						(0:\a) coordinate (C)
						($(C)!0.5!(B)$) coordinate (H)
						($(A)!2/3!(H)$) coordinate (O)
						($(O)+(90:\h)$) coordinate (S);
				\draw[] (S)--(A)--(B)--(C)--cycle (B)--(S)--(H);
				\draw[dashed,thin] (C)--(A)--(H) (S)--(O);
				\foreach \x/\y in {A/180,B/-90,C/0,H/-60,O/-90,S/90}
				\fill (\x) circle (1pt) ($(\x)+(\y:3mm)$) node {$\x$};
				\pic[draw,angle radius=2.5mm] {right angle=A--H--B};
				\pic[draw,angle radius=2mm] {right angle=S--O--H};
				\path (A)--(B) node[below,midway]{$2$}
				(A)--(S) node[left,midway]{$3$};
			\end{tikzpicture}
		\end{center}
	Kẻ $AH\perp BC$ (1)\\
	Gọi $O$ là tâm của $\triangle ABC$.\\
	Suy ra $SO\perp(ABC)$ (theo tính chất của hình chóp đều).\\
	$ \Rightarrow SO\perp BC$ (2)\\
	Từ (1) và (2) suy ra $BC\perp(SHA)$ $ \Rightarrow \left[S,BC,A\right] =\widehat{SHO}$.\\
	Vì $\triangle ABC$ đều nên ta có $AH=\sqrt{3} \Rightarrow \heva{&OH=\dfrac{\sqrt{3}}{3}\\ &AO=\dfrac{2\sqrt{3}}{3}.}$\\
	Suy ra $SO=\sqrt{3^2-\left(\dfrac{2\sqrt{3}}{3}\right)^2} =\dfrac{\sqrt{69}}{3}$.\\
	$ \Rightarrow \tan\alpha=\tan\widehat{SHO}=\dfrac{SO}{OH}=\dfrac{\sqrt{69}}{3}\div\dfrac{\sqrt{3}}{3} =\sqrt{23} \approx 4{,8}$.
	}
\end{ex}

\begin{ex}%[2D5V1-4]%Câu 2
	Một hộp đựng $12$ bóng đèn, các bóng đèn trong cùng hộp thì cùng màu. Số hộp đựng bóng đèn màu xanh nhiều gấp $9$ lần số hộp đựng bóng đèn màu vàng. Trong mỗi hộp đựng bóng đèn màu xanh có $3$ bóng bị hỏng, mỗi hộp đựng bóng đèn màu vàng có $2$ bóng bị hỏng. Tính xác suất để lấy ra hai bóng đèn màu xanh ở cùng một hộp, biết cả hai bóng đều bị hỏng. Viết kết quả làm tròn đến hàng phần trăm.
	\shortans[oly]{$0{,}96$}
	\loigiai{
		Gọi $A_1$ là biến cố lấy được một hộp đựng bóng đèn màu vàng.\\
		Suy ra $\mathrm{P}(A_1)=\dfrac{1}{1+9}=\dfrac{1}{10}$.\\
		Gọi $A_2$ là biến cố lấy được một hộp đựng bóng đèn màu xanh.\\
		Suy ra $\mathrm{P}(A_2)=\dfrac{9}{1+9}=\dfrac{9}{10}$.\\
		Gọi $B$ là biến cố lấy được hai bóng đèn hỏng ở cùng 1 hộp.\\
		Ta có xác suất lấy được 2 bóng đèn hỏng từ một hộp đựng bóng đèn vàng là $\mathrm{P}(B|A_1)=\dfrac{\mathrm{C}_2^2}{\mathrm{C}_{12}^2}$ (vì trong mỗi hộp đựng bóng đèn vàng có
		2 bóng bị hỏng).\\
		Tương tự, vì trong mỗi hộp đựng bóng đèn màu xanh có 3 bóng bị hỏng nên xác suất lấy được 2 bóng đèn hỏng từ một hộp đựng bóng đèn xanh là $\mathrm{P}(B|A_2)=\dfrac{\mathrm{C}_3^2}{\mathrm{C}_{12}^2}$.\\
		Ta có sơ đồ hình cây sau\\
		Ta có 
		$\mathrm{P}(B)=\mathrm{P}(A_1)\cdot\mathrm{P}(B|A_1)+\mathrm{P}(A_2)\cdot\mathrm{P}(B|A_2) =\dfrac{1}{10}\cdot\dfrac{\mathrm{C}_2^2}{\mathrm{C}_{12}^2} +\dfrac{9}{10}\cdot\dfrac{\mathrm{C}_3^2}{\mathrm{C}_{12}^2} =\dfrac{7}{165}$.\\
		Suy ra $\mathrm{P}(A_2|B)=\dfrac{\mathrm{P}(A_2)\cdot\mathrm{P}(B|A_2)}{\mathrm{P}(B)} =\dfrac{\dfrac{9}{10}\cdot\dfrac{\mathrm{C}_3^2}{\mathrm{C}_{12}^2}}{\dfrac{7}{167}} =\dfrac{27}{28} \approx 0{,}96$.
	}
\end{ex}

\begin{ex}%[1C2C3-1]%Câu 3
	Trên đường Mạnh đi từ nhà $(M)$ đến công ty $(C)$ có điểm $A$ người ta đang thi công sửa chữa đường nên không thể đi qua $A$.
	\begin{center}
		\begin{tikzpicture}[scale=0.9,>=stealth, font=\footnotesize, line join=round, line cap=round]
			\coordinate (M) at (0,0);
			\coordinate (A) at (3,1);
			\coordinate (C) at (6,4);
			\draw (M)node[above right]{$M$}--(6,0)--(C)node[above right]{$C$}--(0,4)--cycle (0,1)--(6,1) (0,2)--(3,2) (3,3)--(4,3) (1,0)--(1,2) (2,0)--(2,2) (A)node[below,red]{$A$}--(3,4) (4,1)--(4,4) (5,0)--(5,1);
			\draw[->] (-0.3,0)--(-0.3,1);
			\draw[->] (0,-0.3)--(1,-0.3);
			\fill[red] (A)circle(2pt);
			\fill (C)circle(2pt);
			\fill (M)circle(2pt);
		\end{tikzpicture}
	\end{center}
	Biết rằng toàn bộ cung đường theo bản đồ từ dưới lên trên và từ trái qua phải là đường một chiều vì vậy Mạnh chỉ được phép đi lên hoặc đi sang phải. Vậy Mạnh có bao nhiêu cách đến công ty?
	\shortans[oly]{$15$}
	\loigiai{
		Số cách Mạnh đến công ty là $15$ cách.\\
		Minh hoạ
		\begin{center}
			\begin{tikzpicture}[scale=0.9,>=stealth, font=\footnotesize, line join=round, line cap=round, thick]
				\coordinate (M) at (0,0);
				\coordinate (A) at (3,1);
				\coordinate (C) at (6,4);
				\draw (M)node[above right]{$M$}--(6,0)--(C)node[ right]{$C$}--(0,4)--cycle (0,1)--(6,1) (0,2)--(3,2) (3,3)--(4,3) (1,0)--(1,2) (2,0)--(2,2) (A)node[below,red]{$A$}--(3,4) (4,1)--(4,4) (5,0)--(5,1);
				\draw[->] (-0.185,0)--(-0.185,1);
				\draw[->] (0,-0.185)--(1,-0.185);
				\fill[red] (A)circle(2pt);
				\fill (C)circle(2pt);
				\fill (M)circle(2pt);
				\path 
				(1,0) node[below]{$1$} 			(2,0) node[below]{$1$} 
				(5,0) node[below]{$1$} 			(6,0) node[below]{$1$}
				(0,1) node[left]{$1$} 			(0,2) node[left]{$1$} 
				(0,4) node[left]{$1$}			(5,1) node[above]{$1$}
				(1,1) node[above right]{$2$} 	(6,1) node[above right]{$2$}
				(1,2) node[above]{$3$}			(2,1) node[above right]{$3$}
				(2,2) node[above]{$6$}			(3,2) node[above left]{$6$}
				(3,3) node[left]{$6$}			(4,3) node[right]{$6$}
				(3,4) node[above]{$7$}
				(4,4) node[above]{$13$}
				(6,4) node[above]{$15$}
				;
			\end{tikzpicture}
		\end{center}
	}
\end{ex}

\begin{ex}%[2D4V3-5]%Câu 4
	\immini[thm]{
		Bên trong hình vuông cạnh $4$, dựng hình sao bốn cạnh đều như hình vẽ bên (các kích thước cần thiết cho như ở trong hình vẽ).
		Tính thể tích $V$ của khối tròn xoay sinh ra khi quay hình sao đó quanh trục $Ox$ (làm tròn kết quả đến hàng phần mười).
	}{
		\begin{tikzpicture}[scale=0.8,>=stealth, font=\footnotesize, line join=round, line cap=round]
			\def\xmin{-3} \def\xmax{3}
			\def\ymin{-3} \def\ymax{3}
			\fill[gray!70] (-2,2)--(0,1)--(2,2)--(1,0)--(2,-2)--(0,-1)--(-2,-2)--(-1,0)--cycle;
			\draw[->] (\xmin,0)--(\xmax,0) node [below]{$x$};
			\draw[->] (0,\ymin)--(0,\ymax) node [left]{$y$};
			\node at (0,0) [below left]{$O$};
			\draw[dashed] (-2,2)--(2,2)--(2,-2)--(-2,-2)--cycle (-2,0)node[below left]{$-2$} (2,0)node[below right]{$2$} (0,2)node[above left]{$2$} (0,-2)node[below left]{$-2$};
			\draw (-2,2)--(0,1)node[above left,yshift=0.1cm]{$1$}--(2,2)--(1,0)node[below right,xshift=0.1cm]{$1$}--(2,-2)--(0,-1)node[below left,yshift=-0.1cm]{$-1$}--(-2,-2)--(-1,0)node[below left,xshift=-0.1cm]{$-1$}--cycle;
		\end{tikzpicture}
	}
	\shortans[0ly]{$20{,}9$}
	\loigiai{
		\begin{center}
			\begin{tikzpicture}[scale=1,>=stealth, font=\footnotesize, line join=round, line cap=round, thick, blue]
				\def\xmin{-3} \def\xmax{3} \def\ymin{-3} \def\ymax{3}
				\path 
					(-2,2) coordinate (A)
					(2,2) coordinate (B)
					(2,-2) coordinate (C)
					(-2,-2) coordinate (D)
					(0,1) coordinate (M)
					(1,0) coordinate (N)
					(0,-1) coordinate (P)
					(-1,0) coordinate (Q);
				\fill[gray!50] (-2,2)--(0,1)--(2,2)--(1,0)--(2,-2)--(0,-1)--(-2,-2)--(-1,0)--cycle;
				\draw[->] (\xmin,0)--(\xmax,0) node [below]{$x$};
				\draw[->] (0,\ymin)--(0,\ymax) node [left]{$y$};
				\node at (0,0) [below left]{$O$};
				\draw[dashed,thin] (-2,2)--(2,2)--(2,-2)--(-2,-2)--cycle (-2,0)node[below left]{$-2$} (2,0)node[below right]{$2$} (0,2)node[above left]{$2$} (0,-2)node[below left]{$-2$};
				\draw (-2,2)--(0,1)node[above left,yshift=0.1cm]{$1$}--(2,2)--(1,0)node[below right,xshift=0.1cm]{$1$}--(2,-2)--(0,-1)node[below left,yshift=-0.1cm]{$-1$}--(-2,-2)--(-1,0)node[below left,xshift=-0.1cm]{$-1$}--cycle;
				\foreach \x/\y in {A/135,B/45,C/-45,D/225,M/-45,N/225,P/45,Q/-45}
				\fill[black] (\x) circle (1pt) ($(\y:3mm)+(\x)$) node {$\x$};
			\end{tikzpicture}
		\end{center}
		Ta kí hiệu các điểm như hình vẽ.\\
		Ta có khối tròn xoay đó được tạo thành khi quay hình phẳng $QAMBN$ quanh trục $Ox$.\\
		Mà $S_{OQAM}=S_{ONBM}$ nên thể tích của khối tròn xoay đó sẽ bằng 2 lần thể tích của khối tròn xoay khi quay hình phẳng $ONBM$ quanh trục $Ox$.\\
		Suy ra ta có thể tích $V=2\left(\pi\displaystyle\int\limits_0^2 MB^2 \mathrm{\,d}x -\pi\displaystyle\int\limits_1^2 NB^2 \mathrm{\,d}x\right)$.
		\begin{itemize}
			\item[+)] Viết phương trình đường thẳng $MB$, với $M(0;1)$, $B(2;2)$.\\
			Có vectơ chỉ phương $\overrightarrow{MB}=(2;1)$ suy ra một vectơ pháp tuyến của đường thẳng là $\overrightarrow{n}_1=(-1;2)$.\\
			Suy ra $MB\colon -1\cdot(x-0)+2\cdot(y-1)=0 \Rightarrow -x+2y-2=0 \Rightarrow y=\dfrac{1}{2}x+1$.
			\item[+)] Tương tự, ta viết được phương trình đường thẳng $NB$ là\\
			$NB\colon -2\cdot(x-1)+1\cdot(y-0)=0 \Leftrightarrow -2x+y+2=0 \Rightarrow y=2x-2$
		\end{itemize}
		Thể tích là $V=2\left(\pi\displaystyle\int\limits_0^2 \left(\dfrac{1}{2}x+1\right)^2 \mathrm{\,d}x -\pi\displaystyle\int\limits_1^2 \left(2x-2\right)^2 \mathrm{\,d}x\right) =\dfrac{20}{3}\pi \approx 20{,}9$.
		}
\end{ex}

\begin{ex}%[2H2C2-6]%Câu 5
	Một ống phun nước có hình dạng như hình vẽ dưới. Để giữ cho ống nước được cân bằng không bị nghiêng kỹ sư sử dụng ba đoạn thép để nối các điểm $C$, $A$, $G$ với mặt đất, các đoạn thép $CD$, $GF$, $AE$ có độ lớn lực căng lần lượt bằng $1200$ N, $800$ N và $600$ N. Trong hệ tọa độ $Oxyz$, coi gốc tọa độ là chân ống nước, trục $Oz$ hướng lên trời, mặt đất là mặt phẳng $(Oxy)$ các thông số được cho như hình vẽ, đơn vị trên các hệ trục tọa độ tính bằng mét. Coi đường kính ống không đáng kể, độ lớn vectơ hợp lực của ba sợi thép tác động lên ông nước là bao nhiêu Newton (làm tròn kết quả đến hàng đơn vị).
	\begin{center}
		\includegraphics[scale=0.45]{images/de7-1}
	\end{center}
	\shortans[oly]{$2311$}
	\loigiai{
		Giả sử lực tác dụng lên 3 đoạn dây $CD$, $GF$, $AE$ lần lượt là $\overrightarrow{T}_1$, $\overrightarrow{T}_2$, $\overrightarrow{T}_3$.\\
		Suy ra hợp lực $\overrightarrow{T} =\overrightarrow{T}_1 +\overrightarrow{T}_2 +\overrightarrow{T}_3$ (*).\\
		(Áp dụng công thức: Cho 2 vec-tơ $\overrightarrow{u}$, $\overrightarrow{v}$ cùng hướng, ta có $\overrightarrow{u} =\dfrac{|\overrightarrow{u}|}{|\overrightarrow{v}|}\cdot\overrightarrow{v}$)
		\begin{enumerate}[+)]
			\item Vì $\overrightarrow{T_1}$ và $\overrightarrow{CD}$ cùng hướng nên ta suy ra $\overrightarrow{T_1}=\dfrac{T_1}{CD}\cdot\overrightarrow{CD}$.
			\item Vì $\overrightarrow{T_2}$ và $\overrightarrow{GF}$ cùng hướng nên ta suy ra $\overrightarrow{T_2}=\dfrac{T_2}{GF}\cdot\overrightarrow{GF}$.
			\item Vì $\overrightarrow{T_3}$ và $\overrightarrow{AE}$ cùng hướng nên ta suy ra $\overrightarrow{T_3}=\dfrac{T_3}{AE}\cdot\overrightarrow{AE}$.
		\end{enumerate}
		Suy ra $\overrightarrow{T} =\dfrac{T_1}{CD}\cdot\overrightarrow{CD} +\dfrac{T_2}{GF}\cdot\overrightarrow{GF} +\dfrac{T_3}{AE}\cdot\overrightarrow{AE}$.
		\begin{enumerate}[+)]
			\item Ta có $C(-1{,}5;0;4{,}5)$, $D(0;3;0)$ $\Rightarrow \overrightarrow{CD}=(1{,}5;3;-4{,}5) \Rightarrow CD=1{,}5\sqrt{14}$.
			\item Ta có $G(0;-1;3)$, $F(2;-1;0)$ $\Rightarrow \overrightarrow{GF}=(2;0;-3) \Rightarrow GF=\sqrt{13}$.
			\item Ta có $A(0;0;3)$, $E(-1{,}5;0;0)$ $\Rightarrow \overrightarrow{AE}=(-1{,}5;0;-3) \Rightarrow AE=1{,}5\sqrt{5}$.
		\end{enumerate}
		Suy ra 
		$\begin{aligned}[t]
			\overrightarrow{T} &=\dfrac{1\,200}{1{,}5\sqrt{14}}\cdot(1{,}5;3;-4{,}5) +\dfrac{800}{\sqrt{13}}\cdot(2;0;-3) +\dfrac{600}{1{,}5\sqrt{5}}\cdot(-1{,}5;0;-3) \\
			&=(496{,}145;641{,}427;-2164{,}437).
		\end{aligned}$\\
		Độ lớn của vec-tơ hợp lực là $|\overrightarrow{T}| =\sqrt{(496{,}145)^2+(641{,}427)^2+(-2164{,}437)^2} \approx 2311$.
	}
\end{ex}

\begin{ex}%[0H9C3-5]%Câu 6
	Hình vẽ sau mô tả một con thuyền đang kéo một người đàn ông trượt ván bằng một đoạn dây dài $9$ mét. Xét trên hệ trục $Oxy$ (đơn vị trên các hệ trục bằng mét), ban đầu con thuyền đang ở gốc tọa độ và di chuyển trên tia $Oy$, người đàn ông xuất phát từ điểm có tọa độ $(9;0)$ bị kéo theo và quãng đường di chuyển tạo thành một đường cong $y=f(x)$ (tham khảo hình vẽ dưới), bờ biển là đường thẳng $x+2y+1=0$. Khi người đàn ông đến gần bờ biển nhất thì khoảng cách giữa người đàn ông và trục $Oy$ bằng bao nhiêu mét (làm tròn kết quả đến hàng phần trăm)? Biết rằng trong quá trình di chuyển, người đàn ông luôn hướng về phía thuyền, đoạn dây luôn căng và nằm trên tiếp tuyến của đường cong $y=f(x)$.
	\begin{center}
		\includegraphics[scale=0.4]{images/de7-2}
	\end{center}
	\shortans[oly]{$8{,}05$}
	\loigiai{
%		\begin{center}
%			\includegraphics[scale=0.4]{images/de7-3}
%		\end{center}
	Giả sử người đó ở vị trí $M$, $M_0$ là điểm mà tiếp tuyến tại $M_0$ song song
	với đường thẳng $d\colon x+2y+1=0$.\\
	Dựng $M_0N\perp d$, $MK\perp d$.\\
	Dựa vào hình vẽ, ta có $MK\geq M_0N$ nên khoảng cách từ người đến bờ biển ngắn nhất khi tiếp tuyến tại $M$ song song với đường thẳng $x+2y+1=0$.\\
	Dựa vào hình, ta có
	\begin{enumerate}[+)]
		\item $\widehat{PM_0Q}=\widehat{PFO}$ (2 góc đồng vị)
		\item $\widehat{PFO}=\widehat{FED}$ (2 góc so le trong)
	\end{enumerate}
	Suy ra $\widehat{PM_0Q}=\widehat{FED}$.\\
	Thay lần lượt $x=0, y=0$ vào đường thẳng $x+2y+1=0$, ta có $E(-1;0)$, $D\left(0;\frac{1}{2}\right)$ $\Rightarrow \heva{&OE=1\\ &OD=\dfrac{1}{2}.}$\\
	Xét tam giác vuông $OED$, ta có $\tan\widehat{OED}=\dfrac{OD}{OE}=\dfrac{\frac{1}{2}}{1} =\dfrac{1}{2} \Rightarrow \widehat{OED}=\widehat{PM_0Q}\approx26{,}565^\circ$.\\
	Xét tam giác vuông $PM_0Q$, ta có $$\cos\widehat{PM_0Q}=\dfrac{QM_0}{PM_0} \Rightarrow QM_0=\cos\widehat{PM_0Q}\cdot PM_0 =9\cdot\cos(26{,}565^\circ)\approx8{,}05\, (\rm m).$$
	}
\end{ex}
\Closesolutionfile{ans}

\inputansbox{3}{ans/de7-KQ}
% \begin{name}
	{\tenchude}
	{\tendethi}
	{\tentruong}
	{\thoigian}
	\end{name}
\TN
\Opensolutionfile{ans}[ans/de11-phanI]
\begin{ex}%[2D1N2-2]
	Cho hàm số $y=f(x)$ có bảng biến thiên như hình sau:
	\begin{center}
		\begin{tikzpicture}[scale=1, font=\footnotesize, line width=1pt]
			\tkzTabInit[nocadre=true, lgt=1.2, espcl=2, deltacl=0.6]
			{$x$/0.8,$f'(x)$/0.6,$f(x)$/2}
			{$-\infty$,$0$,$2$,$+\infty$};
			\tkzTabLine{,-,$0$,+,$0$,-,};
			\tkzTabVar{+/$+\infty$,-/$1$,+/$5$,-/$-\infty$};
		\end{tikzpicture}
	\end{center}
	Giá trị cực đại của hàm số đã cho bằng
	\choice
	{$0$}
	{$2$}
	{\True $5$}
	{$1$}
	\loigiai
	{
		Từ bảng biến thiên ta có giá trị cực đại của hàm số đã cho bằng $5$ tại $x=2$.
	}
\end{ex}

\begin{ex}%[2H5N3-2]
	Trong không gian $Oxyz$, cho mặt cầu $(S)\colon(x-2)^2+(y+1)^2+(z-3)^2=4$. Tâm của $(S)$ có tọa độ là
	\choice
	{$(-4;2;-6)$}
	{$(-2;1;-3)$}
	{\True $(2;-1;3)$}
	{$(4;-2;6)$}
	\loigiai
	{
		Tâm của $(S)$ có tọa độ là $(2,-1,3)$.
	}
\end{ex}

\begin{ex}%[1D6H4-2]
	Nếu $\log_8p=m$ thì $\log_2p$ bằng
	\choice
	{$\dfrac{m}{3}$}
	{$\dfrac{3}{m}$}
	{$m^3$}
	{\True $3m$}
	\loigiai
	{
		$\log_8p=m \Rightarrow p=8^m \Rightarrow \log_2p=\log_28^m=m\log_28=3m$.
	}
\end{ex}

\begin{ex}%[1H8N1-2]
	Cho hình hộp chữ nhật $ABCD.A'B'C'D'$. Hai đường thẳng nào sau đây vuông góc với nhau?
	\choice
	{$BD$ và $C'D'$}
	{\True $AA'$ và $BD$}
	{$A'B$ và $CD$}
	{$BB'$ và $DD'$}
	\loigiai
	{
		\begin{center}
			\begin{tikzpicture}[scale=1, font=\footnotesize,>=stealth, line width=1pt]%<DTools>
				%Gán số liệu.
				\def\canhAD{3};\def\canhBA{2};\def\gocBAD{-130};\def\h{4};\def\xdinhA'{0};
				%Gán tọa độ.
				\coordinate (A) at (0,0);
				\coordinate (B) at ($(A)+(\gocBAD:\canhBA)$);
				\coordinate (C) at ($(B)+(0:\canhAD)$);
				\coordinate (D) at ($(A)+(0:\canhAD)$);
				\coordinate (A') at ($(A)+(\xdinhA',\h)$);
				\coordinate (B') at ($(B)+(\xdinhA',\h)$);
				\coordinate (C') at ($(C)+(\xdinhA',\h)$);
				\coordinate (D') at ($(D)+(\xdinhA',\h)$);
				%Vẽ khối lẳng trụ ABCD.A'B'C'D'.
				\draw (A')--(B')--(B)--(C)--(C')--(D')--cycle (B')--(C') (D')--(D)--(C);
				\draw[dashed] (A)--(D) (A')--(A)--(B);
				%Gán nhãn.
				\foreach \x/\y in {A/180, B/180, C/0, D/0, A'/180, B'/180, C'/0, D'/0}{\fill (\x) circle(1pt) ($(\x)+(\y:0.3cm)$) node{$\x$};}
			\end{tikzpicture}
		\end{center}
		Ta có $AA'\perp (ABCD) \Rightarrow AA'\perp BD$.
	}
\end{ex}

\begin{ex}%[2D4H1-3]
	Hàm số $F(x)=\sin 2x$ là một nguyên hàm của hàm số nào sau đây?
	\choice
	{$f_3(x)=\cos 2x$}
	{\True $f_2(x)=2\cos 2x$}
	{$f_1(x)=\dfrac{1}{2}\cos 2x$}
	{$f_4(x)=-\dfrac{1}{2}\cos 2x$}
	\loigiai
	{
		Ta có $f(x)=F'(x) =(\sin 2x)'=2\cos 2x$.
	}
\end{ex}

\begin{ex}%[2D1H1-1]
	Hàm số $y=x^4-2x^2+5$ đồng biến trên khoảng nào sau đây?
	\choice
	{$(-\infty;-1)$}
	{$(0;1)$}
	{\True $(-1;0)$}
	{$(0;+\infty)$}
	\loigiai
	{
		Hàm số trên có bảng biến thiên
		\begin{center}
			\begin{tikzpicture}[scale=1, font=\footnotesize, line width=1pt]%<DTools>
				\tkzTabInit[nocadre=true, lgt=1.2, espcl=2, deltacl=0.6]
				{$x$/0.8,$f'(x)$/0.6,$f(x)$/2}
				{$-\infty$,$-1$,$0$,$1$,$+\infty$};
				\tkzTabLine{,-,$0$,+,$0$,-,$0$,+,};
				\tkzTabVar{+/$+\infty$,-/$4$,+/$5$,-/$4$,+/$+\infty$};
			\end{tikzpicture}
		\end{center}
		Từ bảng biến thiên ta thấy hàm số đồng biến trên $(-1;0)$ và $(1;+\infty)$.
	}
\end{ex}

\begin{ex}%[0H5V4-1]
	Cho hình chóp $S.ABC$ có đáy $ABC$ là tam giác đều, $AB=1$, cạnh bên $SB$ vuông góc với mặt phẳng đáy và $SB=1$. Tích vô hướng của hai vectơ $\overrightarrow{SA}$ và $\overrightarrow{SB}$ bằng
	\choice
	{\True $1$}
	{$\sqrt{2}$}
	{$2$}
	{$\dfrac{\sqrt{2}}{2}$}
	\loigiai
	{
		\begin{center}
			\begin{tikzpicture}[scale=1, font=\footnotesize,>=stealth, line width=1pt]%<DTools>
				%Gán số liệu.
				\def\canhBC{4};\def\canhAB{2};\def\gocABC{-50};\def\h{3};\def\xdinhS{0};
				%Gán tọa độ.
				\coordinate (B) at (0,0);
				\coordinate (A) at ($(B)+(\gocABC:\canhAB)$);
				\coordinate (C) at ($(B)+(0:\canhBC)$);
				\coordinate (S) at ($(B)+(\xdinhS,\h)$);
				%Vẽ khối chóp S.BAC.
				\draw (S)--(A) (S)--(B)--(A) (S)--(C)--(A);
				\draw[dashed] (B)--(C);
				%Gán nhãn.
				\foreach \x/\y in {S/90,B/180,A/-90,C/0}{\fill (\x) circle (1pt) ($(\x)+(\y:0.3cm)$) node{$\x$};}
			\end{tikzpicture}
		\end{center}
		$SB\perp (ABC) \Rightarrow SB \perp AB$ mà $SB=AB=1$ nên tam giác $ABC$ là tam giác vuông cân.\\
		Suy ra $\widehat{ASB}=45^\circ$.\\
		Ta có $SA=\sqrt{SB^2+AB^2}=\sqrt{1^2+1^2}=\sqrt{2}$.\\
		Vậy $\overrightarrow{SA}\cdot \overrightarrow{SB}=\left|\overrightarrow{SA}\right|\cdot \left|\overrightarrow{SB}\right|\cdot \cos \widehat{ASB}=\sqrt{2} \cdot 1 \cdot \dfrac{\sqrt{2}}{2}=1$.
	}
\end{ex}

\begin{ex}%[1D2H2-4]
	Cho cấp số cộng $6$, $17$, $28$, $\ldots$, số hạng thứ $10$ của cấp số cộng đã cho bằng
	\choice
	{$108$}
	{$106$}
	{$107$}
	{\True $105$}
	\loigiai
	{
		Cấp số cộng có $u_1=6$ và công sai $d=11$.\\
		Số hạng thứ $10$ là $u_{10}=u_1+(10-1)d=6+9\cdot 11=105$.
	}
\end{ex}

\begin{ex}%[2D4H2-1]
	Nếu $\displaystyle\int\limits_0^2f(x)\mathrm{\, d}x=4$ thì $\displaystyle\int\limits_0^2\left[f(x)-3\right]\mathrm{\, d}x$ bằng
	\choice
	{$-4$}
	{$1$}
	{\True $-2$}
	{$3$}
	\loigiai
	{
		$\displaystyle\int\limits_0^2\left[f(x)-3\right]\mathrm{\, d}x=\displaystyle\int\limits_0^2f(x)\mathrm{\, d}x-\displaystyle\int\limits_0^2 3\mathrm{\, d}x=4-6=-2$.
	}
\end{ex}

\begin{ex}%[2D3N1-4]
	Cân nặng của một số quả mít trong khu vườn được thống kê ở bảng sau
	\begin{center}
		\begin{tabular}{|l|c|c|c|c|c|}
			\hline Cân nặng (kg)&{$[4;6)$}&{$[6;8)$}&{$[8;10)$}&{$[10;12)$}&{$[12;14)$}\\
			\hline Số quả mít&$6$&$12$&$19$&$9$&$4$\\
			\hline
		\end{tabular}
	\end{center}
	Số quả mít có cân nặng ít hơn $10$ kg trong bảng trên là
	\choice
	{$19$}
	{$46$}
	{$40$}
	{\True $37$}
	\loigiai
	{
		Từ bảng số liệu ta có số quả mít có cân nặng ít hơn $10$ kg là $$6+12+19=37.$$
	}
\end{ex}

\begin{ex}%[2H2H2-3]
	Trong không gian $Oxyz$, cho điểm $M(2;0;1)$. Gọi $A$, $B$ lần lượt là hình chiếu vuông góc của $M$ trên trục $O x$ và trên mặt phẳng $(Oyz)$. Đường thẳng $AB$ có một vectơ chỉ phương là vectơ nào sau đây?
	\choice
	{$\vec{u}_1=(2;0;1)$}
	{\True $\vec{u}_2=(-2;0;1)$}
	{$\vec{u}_3=(1;0;-2)$}
	{$\vec{u}_4=(1;0;2)$}
	\loigiai
	{
		$A$ là hình chiếu vuông góc của $M$ trên trục $Ox$ nên $A(2;0;1)$.\\
		$B$ là hình chiếu vuông góc của $M$ trên mặt phẳng $Oyz$ nên $B(0;0;1)$.\\
		Véctơ chỉ phương của đường thẳng $AB$ là véctơ $\overrightarrow{AB}=(-2;0;1)$.
	}
\end{ex}

\begin{ex}%[1D6H4-5]
	Tập nghiệm của bất phương trình $5^{x^2}\le 25^x$ chứa bao nhiêu số nguyên?
	\choice
	{\True $3$}
	{$1$}
	{$2$}
	{$4$}
	\loigiai
	{
		$5^{x^2}\le 25^x \Leftrightarrow x^2 \leq 2x \Leftrightarrow x^2-2x\leq 0 \Leftrightarrow 0\leq x \leq 2$.\\
		Suy ra tập nghiệm của bất phương trình chứa các số nguyên là $0$; $1$; $2$.\\
		Vậy tập nghiệm của bất phương trình chứa ba số nguyên.
	}
\end{ex}
\Closesolutionfile{ans}
%{\fontfamily{qtm}\fontsize{13pt}{2pt}\selectfont\textbf{PHẦN II. Câu trắc nghiệm đúng sai}. Thí sinh trả lời từ câu 1 đến câu 4. Trong mỗi ý \textbf{a)}, \textbf{b)}, \textbf{c)}, \textbf{d)} ở mỗi câu, thí sinh chọn đúng hoặc sai.}
%\setcounter{ex}{0}% Reset lại số đếm câu hỏi
\TNTF
\Opensolutionfile{ans}[ans/de11-phanII]
\begin{ex}%[2D4H1-3]
	Cho hàm số $f(x)=2x-3\cos x$. Gọi $F(x)$ là một nguyên hàm của hàm số $f(x)$ thoả mãn điều kiện $F\left(\dfrac{\pi}{2}\right)=3$.
	\choiceTF
	{\True $F'(x)=2x-3\cos x$}
	{$\displaystyle\int f(x)\mathrm{\, d}x=x^2+3\sin x+C$}
	{\True $F(x)=x^2-3\sin x+6-\dfrac{\pi^2}{4}$}
	{$F(0)=3-\dfrac{\pi^2}{4}$}
	\loigiai{
		\begin{itemchoice}
			\itemch {\bf Đúng}.\\
			$F(x)$ là nguyên hàm của $f(x)$ nên $F'(x)=f(x)=2x-3\cos x$.
			\itemch {\bf Sai}.\\
			$F(x)=\displaystyle\int f(x)\mathrm{\, d}x=\displaystyle\int 2x-3\cos x\mathrm{\, d}x=x^2-3\sin x +C$.
			\itemch {\bf Đúng}.\\
			$F\left(\dfrac{\pi}{2}\right)=3 \Rightarrow 3=\left(\dfrac{\pi}{2}\right)^2-3\sin \left(\dfrac{\pi}{2}\right)+C \Rightarrow C=6-\left(\dfrac{\pi^2}{4}\right)$.\\
			Vậy $F(x)=x^2-3\sin x+6-\dfrac{\pi^2}{4}$
			\itemch {\bf Sai}.\\
			$F(0)=6-\dfrac{\pi^2}{4}$.
		\end{itemchoice}
		
		
		
	}
\end{ex}

%%%%==============HetCau_EX1==============%%%
%%%%==============Cau_EX2==============%%%
\begin{ex} Vào lúc $12$ giờ trưa, tàu $B$ đang nằm ở vị trí $O$, tàu $A$ cách tàu $B$ $12$ km. Tàu $A$ đang di chuyển về phía $O$ với vận tốc $12$ km/h và tiếp tục di chuyển như vậy cả ngày. Tàu $B$ có vận tốc $8$ km/h đang di chuyển theo hướng vuông góc với hướng đi của tàu $A$ và tiếp tục di chuyển như vậy cả ngày. Quãng đường tàu $A$ và tàu $B$ di chuyển được sau $t$ (giờ) (tính từ lúc $12$ giờ trưa lần lượt là $S_A$ và $S_B$.%[1D7V1-4]
	\choiceTF
	{\True $S_A=12t$ (km) và $S_B=8t$ (km)}
	{Khoảng cách giữa $2$ tàu được xác định bởi công thức $S=\sqrt{S_A^2+S_B^2}$ (km)}
	{Lúc $13$ giờ, khoảng cách giữa $2$ tàu bằng $8\sqrt{10}$ (km)}
	{Lúc $13$ giờ, tốc độ thay đổi khoảng cách giữa $2$ tàu bằng $\dfrac{22\sqrt{10}}{5}$ km/h}
	\loigiai{
		\begin{itemchoice}
			\itemch {\bf Đúng}.\\
			Quãng đường tàu $A$ đi được sau $t$ giờ là $S_A=12t$ (km).\\
			Quãng đường tàu $B$ đi được sau $t$ giờ là $S_B=8t$ (km).
			\itemch {\bf Sai}.\\
			Gọi $M$ là vị trí ban đầu của tàu $A$, sau $t$ giờ tàu $A$ đến được vị trí $A$ mới và đi được quãng đường $MA=S_A$, và tàu $B$ đến vị trí $B$ như hình vẽ.
			\begin{center}
				\begin{tikzpicture}[scale=1, font=\footnotesize,line join=round, line cap=round, >=stealth]
					\path
					(0,0) coordinate (O)
					+(-90:3) coordinate (x)
					+(-90:2) coordinate (B)
					+(180:5) coordinate (M)
					+(180:4) coordinate (A)
					;
					\draw[-stealth] (M)--(A);
					\draw[-stealth] (O)--(B);
					\draw (A)--(O)--(x)
					(B)--(A)
					;
					\draw[stealth-stealth] ($(M)+(-90:.3)$)--($(A)+(-90:.3)$) node[midway,below] {$S_A$};
					\draw[stealth-stealth] ($(O)+(0:.5)$)--($(B)+(0:.5)$) node[midway,right] {$S_B$};
					\foreach \x/\g in {M/90,A/90,O/45,B/0}\fill (\x) circle (1pt)+(\g:3mm) node{$\x$};
				\end{tikzpicture}
			\end{center}
			Khoảng cách giữa hai tàu tại thời điểm $t$ là $$S(t)=\sqrt{(12-S_A)^2+S_B^2}=\sqrt{(12-12t)^2+(8t)^2}.$$
			\itemch {\bf Sai}.\\
			Lúc $13$ giờ tàu $B$ cách $O$ một khoảng là $OB=8\cdot 1=8$ (km).\\
			Tàu $A$ cách $O$ một đoạn $OA=12-12=0$ (km).\\
			Vậy khoảng cách giữa hai tàu lúc này chính bằng đoạn $OB=8$ (km).
			\itemch {\bf Sai}.\\
			Ta có
			$S(t)=\sqrt{(12-12t)^2+(8t)^2}=\sqrt{208t^2-288t+144}$.\\
			Vậy $S'(t)=\dfrac{416t-288}{2\sqrt{208t^2-288t+144}}$.\\
			Lúc $13$ giờ ứng với $t=1$ nên ta có
			$S'(1)=\dfrac{32}{2\sqrt{16}}=8$ km/h.
		\end{itemchoice}
		
	}
\end{ex}
%%%==============HetCau_EX2==============%%%
%%%==============Cau_EX3==============%%%
\begin{ex} Một tờ tiền giả lần lượt bị hai người $A$ và $B$ kiểm tra. Xác suất để người $A$ phát hiện ra tờ này giả là $0{,}7$. Nếu người $A$ cho rằng tờ này tiền giả, thì xác suất để người $B$ cũng nhận định như thế là $0{,}8$. Ngược lại, nếu người $A$ cho rằng tờ này là tiền thật thì xác suất để người $B$ cũng nhận định như thế là $0{,}4$.%[2D6V2-3]
	\choiceTF
	{Xác suất để $A$ không phát hiện ra tờ tiền đó giả là $0{,}2$}
	{\True Xác suất để hai người này đều không phát hiện đây là tờ tiền giả là $0{,}12$}
	{\True Xác suất để ít nhất một trong hai người này phát hiện ra tờ tiền đó là giả là $0{,}88$}
	{\True Biết tờ tiền đó đã bị ít nhất một trong hai người này phát hiện là giả, xác suất để $A$ phát hiện ra nó giả là $79{,}5\%$ (làm tròn đến hàng phần chục)}
	\loigiai{
		\begin{itemchoice}
			\itemch {\bf Sai}.\\
			Gọi $A$ là biến cố \lq\lq người $A$ phát hiện tờ tiền này là giả\rq\rq.\\
			Vậy $\overline{A}$ là biến cố \lq\lq người $A$ không phát hiện tờ tiền này là giả\rq\rq.\\
			Gọi $B$ là biến cố \lq\lq người $B$ phát hiện tờ tiền này là giả\rq\rq.\\
			Vậy $\overline{B}$ là biến cố \lq\lq người $B$ không phát hiện tờ tiền này là giả\rq\rq.\\
			Theo đề bài $\mathrm{P}(A)=0{,}7 \Rightarrow \mathrm{P}\left(\overline{A}\right)=1-\mathrm{P}(A)=1-0{,}7=0{,}3$.
			\itemch {\bf Đúng}.\\
			Nếu người $A$ cho rằng tờ tiền này là thật, xác suất để người $B$ cũng nhận định như thế là $\mathrm{P}\left(B\mid\overline{A}\right)=0{,}4$.\\
			Xác suất để cả hai người đều không phát hiện ra tờ tiền giả:
			$$\mathrm{P}\left(\overline{A} \cap\overline{B}\right)=\mathrm{P}(\overline{A}) \cdot \mathrm{P}(B \mid\overline{A})=0{,}3 \cdot 0{,}4=0{,}12.$$
			\itemch {\bf Đúng}.\\
			Gọi $C$ là biến cố \lq\lq Ít nhất một trong hai người này phát hiện ra tờ tiền đó là giả\rq\rq.\\
			Xác suất để ít nhất một trong hai người này phát hiện ra tờ tiền đó là giả là
			$$\mathrm{P}(C)=1-\mathrm{P}\left(\overline{A} \cap\overline{B}\right)=1-0{,}12=0{,}88.$$
			\itemch {\bf Đúng}.\\
			Ta có
			$\mathrm{P}(A \mid C)=\dfrac{\mathrm{P}(C\mid A)\cdot \mathrm{P}(A)}{\mathrm{P}(C)}=\dfrac{1\cdot 0{,}7}{0{,}88}=79{,}5\%$.
		\end{itemchoice}
		
	}
\end{ex}
%%%==============HetCau_EX3==============%%%
%%%==============Cau_EX4==============%%%
\begin{ex}%[2H5V2-8]
	\immini[thm]{
		Một tháp kiểm soát không lưu ở sân bay cao $109$ m đặt một đài kiểm soát không lưu ở độ cao $105$ m. Máy bay trong phạm vi cách đài kiểm soát $450$ km sẽ hiển thị trên màn hình ra đa. Chọn hệ trục tọa độ $Oxyz$ có gốc $O$ trùng với vị trí chân tháp, mặt phẳng ($Oxy$) trùng với mặt đất sao cho trục $Ox$ là hướng Tây, trục $Oy$ là hướng Nam và trục $Oz$ là trục thẳng đứng (Hình vẽ), đơn vị trên mỗi trục là kilômét.
	}
	{
		\tikzset{khongluu/.pic={
				\definecolor{cdce1eb}{RGB}{220,225,235}
				\definecolor{cafb4c8}{RGB}{175,180,200}
				\definecolor{c82d1f5}{RGB}{130,209,245}
				\definecolor{c8ab0e0}{RGB}{138,176,224}
				\definecolor{ce9edf5}{RGB}{233,237,245}
				\definecolor{caaccfa}{RGB}{170,204,250}
				\definecolor{cbec3d2}{RGB}{190,195,210}
				\begin{scope}
					\path[fill=cdce1eb,nonzero rule] (5.13, 5.58) -- (13.02, 5.58) -- (13.02, 0.21) -- (5.13, 0.21) --cycle
					(5.13, 5.58);
					\path[fill=cafb4c8,nonzero rule] (5.13, 5.58) -- (13.02, 5.58) -- (13.02, 4.44) -- (5.13, 4.44) --cycle
					(5.13, 5.58);
					\path[fill=c82d1f5,nonzero rule] (5.14, 4.44) -- (13.03, 4.44) -- (13.03, 3.3) -- (5.14, 3.3) --cycle
					(5.14, 4.44);
					\path[fill=cdce1eb,nonzero rule] (5.13, 5.58) -- (13.02, 5.58) -- (13.02, 0.21) -- (5.13, 0.21) --cycle
					(5.13, 5.58);
					\path[fill=cdce1eb,nonzero rule] (5.13, 5.58) -- (13.02, 5.58) -- (13.02, 4.44) -- (5.13, 4.44) --cycle
					(5.13, 5.58);
					\path[fill=c8ab0e0,nonzero rule] (5.14, 4.44) -- (13.03, 4.44) -- (13.03, 3.3) -- (5.14, 3.3) --cycle
					(5.14, 4.44);
					\path[fill=ce9edf5,nonzero rule] (5.5, 3.3) -- (13.02, 3.3) -- (13.02, 0.7) -- (6.19, 0.7) .. controls (6.17, 0.7) and (6.14, 0.7) ..
					(6.12, 0.7) .. controls (6.1, 0.7) and (6.08, 0.7) ..
					(6.05, 0.71) .. controls (6.03, 0.71) and (6.01, 0.72) ..
					(5.99, 0.73) .. controls (5.97, 0.73) and (5.94, 0.74) ..
					(5.92, 0.75) .. controls (5.9, 0.76) and (5.88, 0.77) ..
					(5.86, 0.78) .. controls (5.84, 0.79) and (5.82, 0.8) ..
					(5.8, 0.81) .. controls (5.79, 0.82) and (5.77, 0.84) ..
					(5.75, 0.85) .. controls (5.73, 0.87) and (5.72, 0.88) ..
					(5.7, 0.9) .. controls (5.68, 0.91) and (5.67, 0.93) ..
					(5.65, 0.95) .. controls (5.64, 0.97) and (5.63, 0.98) ..
					(5.61, 1) .. controls (5.6, 1.02) and (5.59, 1.04) ..
					(5.58, 1.06) .. controls (5.57, 1.08) and (5.56, 1.1) ..
					(5.55, 1.12) .. controls (5.54, 1.14) and (5.53, 1.16) ..
					(5.53, 1.19) .. controls (5.52, 1.21) and (5.52, 1.23) ..
					(5.51, 1.25) .. controls (5.51, 1.27) and (5.5, 1.3) ..
					(5.5, 1.32) .. controls (5.5, 1.34) and (5.5, 1.36) ..
					(5.5, 1.39) --cycle
					(5.5, 3.3);
					\path[fill=ce9edf5,nonzero rule] (13.02, 5.58) -- (13.02, 4.8) -- (6.07, 4.8) .. controls (6.06, 4.8) and (6.04, 4.8) ..
					(6.02, 4.8) .. controls (6, 4.8) and (5.98, 4.8) ..
					(5.96, 4.81) .. controls (5.94, 4.81) and (5.93, 4.82) ..
					(5.91, 4.82) .. controls (5.89, 4.83) and (5.87, 4.83) ..
					(5.85, 4.84) .. controls (5.84, 4.85) and (5.82, 4.86) ..
					(5.8, 4.86) .. controls (5.79, 4.87) and (5.77, 4.88) ..
					(5.75, 4.89) .. controls (5.74, 4.9) and (5.72, 4.92) ..
					(5.71, 4.93) .. controls (5.69, 4.94) and (5.68, 4.95) ..
					(5.67, 4.97) .. controls (5.65, 4.98) and (5.64, 4.99) ..
					(5.63, 5.01) .. controls (5.62, 5.02) and (5.61, 5.04) ..
					(5.6, 5.05) .. controls (5.58, 5.07) and (5.58, 5.08) ..
					(5.57, 5.1) .. controls (5.56, 5.12) and (5.55, 5.13) ..
					(5.54, 5.15) .. controls (5.53, 5.17) and (5.53, 5.19) ..
					(5.52, 5.21) .. controls (5.52, 5.22) and (5.51, 5.24) ..
					(5.51, 5.26) .. controls (5.51, 5.28) and (5.5, 5.3) ..
					(5.5, 5.32) .. controls (5.5, 5.34) and (5.5, 5.35) ..
					(5.5, 5.37) -- (5.5, 5.58) --cycle
					(13.02, 5.58);
					\path[fill=caaccfa,nonzero rule] (13.03, 4.44) -- (13.03, 3.65) -- (6.08, 3.65) .. controls (6.07, 3.65) and (6.05, 3.66) ..
					(6.03, 3.66) .. controls (6.01, 3.66) and (5.99, 3.66) ..
					(5.97, 3.67) .. controls (5.95, 3.67) and (5.94, 3.67) ..
					(5.92, 3.68) .. controls (5.9, 3.69) and (5.88, 3.69) ..
					(5.86, 3.7) .. controls (5.85, 3.71) and (5.83, 3.71) ..
					(5.81, 3.72) .. controls (5.8, 3.73) and (5.78, 3.74) ..
					(5.76, 3.75) .. controls (5.75, 3.76) and (5.73, 3.77) ..
					(5.72, 3.79) .. controls (5.7, 3.8) and (5.69, 3.81) ..
					(5.68, 3.82) .. controls (5.66, 3.84) and (5.65, 3.85) ..
					(5.64, 3.87) .. controls (5.63, 3.88) and (5.62, 3.9) ..
					(5.6, 3.91) .. controls (5.59, 3.93) and (5.58, 3.94) ..
					(5.58, 3.96) .. controls (5.57, 3.98) and (5.56, 3.99) ..
					(5.55, 4.01) .. controls (5.54, 4.03) and (5.54, 4.05) ..
					(5.53, 4.06) .. controls (5.53, 4.08) and (5.52, 4.1) ..
					(5.52, 4.12) .. controls (5.52, 4.14) and (5.51, 4.16) ..
					(5.51, 4.18) .. controls (5.51, 4.19) and (5.51, 4.21) ..
					(5.51, 4.23) -- (5.51, 4.44) --cycle
					(13.03, 4.44);
					\path[fill=cafb4c8,nonzero rule] (11.51, 9.57) -- (10.8, 8.77) .. controls (10.73, 8.77) and (10.21, 8.8) ..
					(9.54, 8.85) -- (10.47, 10.1) -- (10, 10.12) .. controls (9.94, 10.12) and (9.87, 10.11) ..
					(9.81, 10.09) .. controls (9.75, 10.07) and (9.7, 10.04) ..
					(9.65, 9.99) -- (8.48, 8.92) .. controls (8.17, 8.94) and (7.83, 8.96) ..
					(7.72, 8.97) .. controls (7.25, 8.99) and (6.92, 8.5) ..
					(6.92, 8.5) .. controls (6.92, 8.5) and (7.09, 8.09) ..
					(7.54, 8.07) .. controls (7.91, 8.06) and (10, 8.06) ..
					(10.79, 8.06) .. controls (10.84, 8.06) and (10.89, 8.06) ..
					(10.93, 8.07) .. controls (10.98, 8.08) and (11.02, 8.1) ..
					(11.06, 8.12) .. controls (11.1, 8.14) and (11.14, 8.17) ..
					(11.17, 8.2) .. controls (11.21, 8.23) and (11.24, 8.27) ..
					(11.26, 8.31) -- (12.09, 9.54) --cycle
					(11.51, 9.57);
					\path[fill=cbec3d2,nonzero rule] (11, 8.26) .. controls (11.1, 8.26) and (11.2, 8.29) ..
					(11.28, 8.34) -- (12.09, 9.54) -- (11.51, 9.57) -- (10.8, 8.77) .. controls (10.73, 8.77) and (10.21, 8.8) ..
					(9.54, 8.85) -- (10.47, 10.1) -- (10, 10.12) .. controls (9.94, 10.12) and (9.87, 10.11) ..
					(9.81, 10.09) .. controls (9.75, 10.07) and (9.7, 10.04) ..
					(9.65, 9.99) -- (8.48, 8.92) .. controls (8.17, 8.94) and (7.83, 8.96) ..
					(7.71, 8.97) .. controls (7.49, 8.98) and (7.31, 8.87) ..
					(7.17, 8.76) .. controls (7.14, 8.73) and (7.12, 8.71) ..
					(7.12, 8.71) .. controls (7.12, 8.71) and (7.29, 8.29) ..
					(7.75, 8.27) .. controls (8.12, 8.26) and (10.21, 8.26) ..
					(11, 8.26) --cycle
					(11, 8.26);
					\path[fill=cdce1eb,nonzero rule] (11.61, 4.44) -- (11.82, 4.44) -- (11.82, 3.3) -- (11.61, 3.3) --cycle
					(11.61, 4.44);
					\path[fill=cdce1eb,nonzero rule] (10.3, 4.44) -- (10.5, 4.44) -- (10.5, 3.3) -- (10.3, 3.3) --cycle
					(10.3, 4.44);
					\path[fill=cdce1eb,nonzero rule] (8.98, 4.44) -- (9.19, 4.44) -- (9.19, 3.3) -- (8.98, 3.3) --cycle
					(8.98, 4.44);
					\path[fill=cdce1eb,nonzero rule] (7.67, 4.44) -- (7.87, 4.44) -- (7.87, 3.3) -- (7.67, 3.3) --cycle
					(7.67, 4.44);
					\path[fill=cdce1eb,nonzero rule] (6.35, 4.44) -- (6.56, 4.44) -- (6.56, 3.3) -- (6.35, 3.3) --cycle
					(6.35, 4.44);
					\path[fill=cafb4c8,nonzero rule] (2.28, 13.23) .. controls (2.27, 13.23) and (2.25, 13.23) ..
					(2.24, 13.23) .. controls (2.23, 13.22) and (2.21, 13.22) ..
					(2.2, 13.21) .. controls (2.19, 13.21) and (2.18, 13.2) ..
					(2.16, 13.19) .. controls (2.15, 13.19) and (2.14, 13.18) ..
					(2.13, 13.17) .. controls (2.12, 13.16) and (2.11, 13.15) ..
					(2.11, 13.14) .. controls (2.1, 13.13) and (2.09, 13.11) ..
					(2.09, 13.1) .. controls (2.08, 13.09) and (2.08, 13.08) ..
					(2.08, 13.06) .. controls (2.07, 13.05) and (2.07, 13.04) ..
					(2.07, 13.02) -- (2.07, 11.74) -- (2.49, 11.74) -- (2.49, 13.02) .. controls (2.49, 13.04) and (2.48, 13.05) ..
					(2.48, 13.06) .. controls (2.48, 13.08) and (2.48, 13.09) ..
					(2.47, 13.1) .. controls (2.46, 13.11) and (2.46, 13.13) ..
					(2.45, 13.14) .. controls (2.44, 13.15) and (2.43, 13.16) ..
					(2.43, 13.17) .. controls (2.42, 13.18) and (2.41, 13.19) ..
					(2.39, 13.19) .. controls (2.38, 13.2) and (2.37, 13.21) ..
					(2.36, 13.21) .. controls (2.35, 13.22) and (2.33, 13.22) ..
					(2.32, 13.23) .. controls (2.31, 13.23) and (2.29, 13.23) ..
					(2.28, 13.23) --cycle
					(2.28, 13.23);
					\path[fill=cdce1eb,nonzero rule] (3.93, 10.95) -- (3.93, 10.9) -- (0.63, 10.9) -- (0.63, 10.96) .. controls (0.63, 10.98) and (0.63, 11.01) ..
					(0.63, 11.04) .. controls (0.63, 11.06) and (0.64, 11.09) ..
					(0.64, 11.12) .. controls (0.65, 11.14) and (0.65, 11.17) ..
					(0.66, 11.2) .. controls (0.67, 11.22) and (0.68, 11.25) ..
					(0.69, 11.27) .. controls (0.7, 11.3) and (0.71, 11.32) ..
					(0.72, 11.34) .. controls (0.74, 11.37) and (0.75, 11.39) ..
					(0.76, 11.41) .. controls (0.78, 11.44) and (0.79, 11.46) ..
					(0.81, 11.48) .. controls (0.83, 11.5) and (0.85, 11.52) ..
					(0.87, 11.54) .. controls (0.89, 11.56) and (0.91, 11.58) ..
					(0.93, 11.59) .. controls (0.95, 11.61) and (0.97, 11.63) ..
					(0.99, 11.64) .. controls (1.01, 11.66) and (1.04, 11.67) ..
					(1.06, 11.68) .. controls (1.08, 11.69) and (1.11, 11.71) ..
					(1.13, 11.72) .. controls (1.16, 11.73) and (1.18, 11.74) ..
					(1.21, 11.74) .. controls (1.23, 11.75) and (1.26, 11.76) ..
					(1.29, 11.76) .. controls (1.31, 11.77) and (1.34, 11.77) ..
					(1.37, 11.78) .. controls (1.39, 11.78) and (1.42, 11.78) ..
					(1.45, 11.78) -- (3.11, 11.78) .. controls (3.14, 11.78) and (3.16, 11.78) ..
					(3.19, 11.78) .. controls (3.22, 11.77) and (3.24, 11.77) ..
					(3.27, 11.76) .. controls (3.3, 11.76) and (3.32, 11.75) ..
					(3.35, 11.74) .. controls (3.37, 11.74) and (3.4, 11.73) ..
					(3.42, 11.72) .. controls (3.45, 11.71) and (3.47, 11.69) ..
					(3.5, 11.68) .. controls (3.52, 11.67) and (3.54, 11.66) ..
					(3.57, 11.64) .. controls (3.59, 11.63) and (3.61, 11.61) ..
					(3.63, 11.59) .. controls (3.65, 11.58) and (3.67, 11.56) ..
					(3.69, 11.54) .. controls (3.71, 11.52) and (3.73, 11.5) ..
					(3.75, 11.48) .. controls (3.76, 11.46) and (3.78, 11.44) ..
					(3.79, 11.41) .. controls (3.81, 11.39) and (3.82, 11.37) ..
					(3.84, 11.34) .. controls (3.85, 11.32) and (3.86, 11.3) ..
					(3.87, 11.27) .. controls (3.88, 11.25) and (3.89, 11.22) ..
					(3.9, 11.2) .. controls (3.9, 11.17) and (3.91, 11.14) ..
					(3.92, 11.12) .. controls (3.92, 11.09) and (3.93, 11.06) ..
					(3.93, 11.04) .. controls (3.93, 11.01) and (3.93, 10.98) ..
					(3.93, 10.96) --cycle
					(3.93, 10.95);
					\path[fill=ce9edf5,nonzero rule] (3.11, 11.78) -- (1.45, 11.78) .. controls (1.39, 11.78) and (1.34, 11.77) ..
					(1.29, 11.76) .. controls (1.24, 11.7) and (1.21, 11.64) ..
					(1.19, 11.56) .. controls (1.19, 11.55) and (1.18, 11.54) ..
					(1.18, 11.53) .. controls (1.18, 11.52) and (1.18, 11.51) ..
					(1.18, 11.5) .. controls (1.18, 11.48) and (1.18, 11.47) ..
					(1.18, 11.46) .. controls (1.18, 11.45) and (1.18, 11.44) ..
					(1.18, 11.43) .. controls (1.18, 11.42) and (1.18, 11.4) ..
					(1.19, 11.39) .. controls (1.19, 11.38) and (1.19, 11.37) ..
					(1.2, 11.36) .. controls (1.2, 11.35) and (1.21, 11.34) ..
					(1.21, 11.33) .. controls (1.22, 11.32) and (1.22, 11.31) ..
					(1.23, 11.3) .. controls (1.24, 11.29) and (1.25, 11.28) ..
					(1.25, 11.27) .. controls (1.26, 11.27) and (1.27, 11.26) ..
					(1.28, 11.25) .. controls (1.29, 11.24) and (1.3, 11.24) ..
					(1.31, 11.23) .. controls (1.31, 11.22) and (1.32, 11.22) ..
					(1.33, 11.21) .. controls (1.35, 11.21) and (1.36, 11.2) ..
					(1.37, 11.2) .. controls (1.38, 11.19) and (1.39, 11.19) ..
					(1.4, 11.19) .. controls (1.41, 11.19) and (1.42, 11.18) ..
					(1.43, 11.18) .. controls (1.44, 11.18) and (1.46, 11.18) ..
					(1.47, 11.18) -- (3.9, 11.18) .. controls (3.89, 11.22) and (3.87, 11.26) ..
					(3.86, 11.31) .. controls (3.84, 11.35) and (3.81, 11.38) ..
					(3.79, 11.42) .. controls (3.76, 11.46) and (3.74, 11.49) ..
					(3.71, 11.52) .. controls (3.67, 11.56) and (3.64, 11.59) ..
					(3.61, 11.61) .. controls (3.57, 11.64) and (3.53, 11.66) ..
					(3.49, 11.68) .. controls (3.45, 11.7) and (3.41, 11.72) ..
					(3.37, 11.74) .. controls (3.33, 11.75) and (3.29, 11.76) ..
					(3.24, 11.77) .. controls (3.2, 11.78) and (3.15, 11.78) ..
					(3.11, 11.78) --cycle
					(3.11, 11.78);
					\path[fill=cdce1eb,nonzero rule] (0.83, 7.63) .. controls (0.83, 7.63) and (1.41, 6.21) ..
					(1.41, 4.53) -- (1.41, 0.21) -- (3.15, 0.21) -- (3.15, 4.53) .. controls (3.15, 6.21) and (3.73, 7.63) ..
					(3.73, 7.63) -- (3.73, 7.69) -- (0.83, 7.69) --cycle
					(0.83, 7.63);
					\path[fill=ce9edf5,nonzero rule] (1.93, 5.02) -- (1.93, 1.54) .. controls (1.93, 1.51) and (1.93, 1.48) ..
					(1.93, 1.46) .. controls (1.93, 1.43) and (1.94, 1.4) ..
					(1.94, 1.38) .. controls (1.95, 1.35) and (1.95, 1.32) ..
					(1.96, 1.29) .. controls (1.97, 1.27) and (1.98, 1.24) ..
					(1.99, 1.22) .. controls (2, 1.19) and (2.01, 1.17) ..
					(2.03, 1.14) .. controls (2.04, 1.12) and (2.05, 1.09) ..
					(2.07, 1.07) .. controls (2.08, 1.05) and (2.1, 1.03) ..
					(2.12, 1) .. controls (2.14, 0.98) and (2.15, 0.96) ..
					(2.17, 0.94) .. controls (2.19, 0.92) and (2.21, 0.9) ..
					(2.23, 0.89) .. controls (2.26, 0.87) and (2.28, 0.85) ..
					(2.3, 0.84) .. controls (2.32, 0.82) and (2.35, 0.81) ..
					(2.37, 0.8) .. controls (2.4, 0.78) and (2.42, 0.77) ..
					(2.45, 0.76) .. controls (2.47, 0.75) and (2.5, 0.74) ..
					(2.53, 0.73) .. controls (2.55, 0.72) and (2.58, 0.72) ..
					(2.61, 0.71) .. controls (2.63, 0.71) and (2.66, 0.7) ..
					(2.69, 0.7) .. controls (2.71, 0.7) and (2.74, 0.7) ..
					(2.77, 0.7) -- (3.15, 0.7) -- (3.15, 4.53) .. controls (3.15, 6.21) and (3.73, 7.63) ..
					(3.73, 7.63) -- (3.73, 7.69) -- (1.5, 7.69) .. controls (1.67, 7.15) and (1.93, 6.14) ..
					(1.93, 5.02) --cycle
					(1.93, 5.02);
					
					\path[fill=cafb4c8,nonzero rule] (13.02, -0) -- (0.21, -0) .. controls (0.2, -0) and (0.18, 0) ..
					(0.17, 0) .. controls (0.16, 0.01) and (0.15, 0.01) ..
					(0.13, 0.02) .. controls (0.12, 0.02) and (0.11, 0.03) ..
					(0.1, 0.03) .. controls (0.09, 0.04) and (0.08, 0.05) ..
					(0.07, 0.06) .. controls (0.06, 0.07) and (0.05, 0.08) ..
					(0.04, 0.09) .. controls (0.03, 0.1) and (0.03, 0.12) ..
					(0.02, 0.13) .. controls (0.02, 0.14) and (0.01, 0.15) ..
					(0.01, 0.17) .. controls (0.01, 0.18) and (0.01, 0.19) ..
					(0.01, 0.21) .. controls (0.01, 0.22) and (0.01, 0.23) ..
					(0.01, 0.25) .. controls (0.01, 0.26) and (0.02, 0.27) ..
					(0.02, 0.29) .. controls (0.03, 0.3) and (0.03, 0.31) ..
					(0.04, 0.32) .. controls (0.05, 0.33) and (0.06, 0.34) ..
					(0.07, 0.35) .. controls (0.08, 0.36) and (0.09, 0.37) ..
					(0.1, 0.38) .. controls (0.11, 0.39) and (0.12, 0.39) ..
					(0.13, 0.4) .. controls (0.15, 0.4) and (0.16, 0.41) ..
					(0.17, 0.41) .. controls (0.18, 0.41) and (0.2, 0.41) ..
					(0.21, 0.41) -- (13.02, 0.41) .. controls (13.03, 0.41) and (13.04, 0.41) ..
					(13.06, 0.41) .. controls (13.07, 0.41) and (13.08, 0.4) ..
					(13.1, 0.4) .. controls (13.11, 0.39) and (13.12, 0.39) ..
					(13.13, 0.38) .. controls (13.14, 0.37) and (13.15, 0.36) ..
					(13.16, 0.35) .. controls (13.17, 0.34) and (13.18, 0.33) ..
					(13.19, 0.32) .. controls (13.2, 0.31) and (13.2, 0.3) ..
					(13.21, 0.29) .. controls (13.21, 0.27) and (13.22, 0.26) ..
					(13.22, 0.25) .. controls (13.22, 0.23) and (13.22, 0.22) ..
					(13.22, 0.21) .. controls (13.22, 0.19) and (13.22, 0.18) ..
					(13.22, 0.17) .. controls (13.22, 0.15) and (13.21, 0.14) ..
					(13.21, 0.13) .. controls (13.2, 0.12) and (13.2, 0.1) ..
					(13.19, 0.09) .. controls (13.18, 0.08) and (13.17, 0.07) ..
					(13.16, 0.06) .. controls (13.15, 0.05) and (13.14, 0.04) ..
					(13.13, 0.03) .. controls (13.12, 0.03) and (13.11, 0.02) ..
					(13.1, 0.02) .. controls (13.08, 0.01) and (13.07, 0.01) ..
					(13.06, 0) .. controls (13.04, 0) and (13.03, -0) ..
					(13.02, -0) --cycle
					(13.02, -0);
					
					\path[fill=c8ab0e0,nonzero rule] (0.63, 10.95) -- (3.93, 10.95) -- (3.93, 10.09) -- (0.63, 10.09) --cycle
					(0.63, 10.95);
					
					\path[fill=caaccfa,nonzero rule] (1.26, 10.38) -- (3.93, 10.38) -- (3.93, 10.95) -- (1.01, 10.95) -- (1.01, 10.64) .. controls (1.01, 10.62) and (1.01, 10.6) ..
					(1.02, 10.59) .. controls (1.02, 10.57) and (1.02, 10.56) ..
					(1.03, 10.54) .. controls (1.04, 10.53) and (1.04, 10.51) ..
					(1.05, 10.5) .. controls (1.06, 10.48) and (1.07, 10.47) ..
					(1.08, 10.46) .. controls (1.1, 10.45) and (1.11, 10.44) ..
					(1.12, 10.43) .. controls (1.14, 10.42) and (1.15, 10.41) ..
					(1.17, 10.4) .. controls (1.18, 10.4) and (1.2, 10.39) ..
					(1.21, 10.39) .. controls (1.23, 10.39) and (1.25, 10.38) ..
					(1.26, 10.38) --cycle
					(1.26, 10.38);
					
					\path[fill=cdce1eb,nonzero rule] (2.18, 10.96) -- (2.38, 10.96) -- (2.38, 10.08) -- (2.18, 10.08) --cycle
					(2.18, 10.96);
					
					\path[fill=cdce1eb,nonzero rule] (1.35, 10.96) -- (1.56, 10.96) -- (1.56, 10.08) -- (1.35, 10.08) --cycle
					(1.35, 10.96);
					
					\path[fill=cdce1eb,nonzero rule] (3, 10.96) -- (3.21, 10.96) -- (3.21, 10.08) -- (3, 10.08) --cycle
					(3, 10.96);
					
					\path[fill=c8ab0e0,nonzero rule] (0.83, 9.34) -- (3.73, 9.34) -- (3.73, 7.63) -- (0.83, 7.63) --cycle
					(0.83, 9.34);
					
					\path[fill=caaccfa,nonzero rule] (1.61, 7.91) -- (3.73, 7.91) -- (3.73, 9.34) -- (1.25, 9.34) -- (1.25, 8.27) .. controls (1.25, 8.24) and (1.25, 8.22) ..
					(1.26, 8.2) .. controls (1.26, 8.17) and (1.27, 8.15) ..
					(1.28, 8.13) .. controls (1.29, 8.11) and (1.3, 8.09) ..
					(1.31, 8.07) .. controls (1.33, 8.05) and (1.34, 8.03) ..
					(1.36, 8.01) .. controls (1.37, 8) and (1.39, 7.98) ..
					(1.41, 7.97) .. controls (1.43, 7.95) and (1.45, 7.94) ..
					(1.48, 7.93) .. controls (1.5, 7.92) and (1.52, 7.92) ..
					(1.54, 7.91) .. controls (1.57, 7.91) and (1.59, 7.91) ..
					(1.61, 7.91) --cycle
					(1.61, 7.91);
					
					\path[fill=cdce1eb,nonzero rule] (3.73, 8.57) -- (2.38, 8.57) -- (2.38, 9.34) -- (2.18, 9.34) -- (2.18, 8.57) -- (0.83, 8.57) -- (0.83, 8.36) -- (2.18, 8.36) -- (2.18, 7.63) -- (2.38, 7.63) -- (2.38, 8.36) -- (3.73, 8.36) --cycle
					(3.73, 8.57);
					
					\path[fill=cdce1eb,nonzero rule] (0.63, 10.12) -- (3.93, 10.12) .. controls (3.96, 10.12) and (3.99, 10.12) ..
					(4.01, 10.11) .. controls (4.04, 10.11) and (4.07, 10.1) ..
					(4.09, 10.09) .. controls (4.12, 10.08) and (4.14, 10.06) ..
					(4.16, 10.05) .. controls (4.18, 10.03) and (4.21, 10.02) ..
					(4.22, 10) .. controls (4.24, 9.98) and (4.26, 9.96) ..
					(4.28, 9.94) .. controls (4.29, 9.91) and (4.3, 9.89) ..
					(4.31, 9.86) .. controls (4.32, 9.84) and (4.33, 9.81) ..
					(4.34, 9.79) .. controls (4.34, 9.76) and (4.35, 9.73) ..
					(4.35, 9.71) .. controls (4.35, 9.68) and (4.34, 9.65) ..
					(4.34, 9.63) .. controls (4.33, 9.6) and (4.32, 9.57) ..
					(4.31, 9.55) .. controls (4.3, 9.52) and (4.29, 9.5) ..
					(4.28, 9.48) .. controls (4.26, 9.45) and (4.24, 9.43) ..
					(4.22, 9.41) .. controls (4.21, 9.39) and (4.18, 9.38) ..
					(4.16, 9.36) .. controls (4.14, 9.35) and (4.12, 9.33) ..
					(4.09, 9.32) .. controls (4.07, 9.31) and (4.04, 9.31) ..
					(4.01, 9.3) .. controls (3.99, 9.3) and (3.96, 9.29) ..
					(3.93, 9.29) -- (0.63, 9.29) .. controls (0.6, 9.29) and (0.57, 9.3) ..
					(0.54, 9.3) .. controls (0.52, 9.31) and (0.49, 9.31) ..
					(0.47, 9.32) .. controls (0.44, 9.33) and (0.42, 9.35) ..
					(0.4, 9.36) .. controls (0.37, 9.38) and (0.35, 9.39) ..
					(0.33, 9.41) .. controls (0.31, 9.43) and (0.3, 9.45) ..
					(0.28, 9.48) .. controls (0.27, 9.5) and (0.25, 9.52) ..
					(0.24, 9.55) .. controls (0.23, 9.57) and (0.23, 9.6) ..
					(0.22, 9.63) .. controls (0.21, 9.65) and (0.21, 9.68) ..
					(0.21, 9.71) .. controls (0.21, 9.73) and (0.21, 9.76) ..
					(0.22, 9.79) .. controls (0.23, 9.81) and (0.23, 9.84) ..
					(0.24, 9.86) .. controls (0.25, 9.89) and (0.27, 9.91) ..
					(0.28, 9.94) .. controls (0.3, 9.96) and (0.31, 9.98) ..
					(0.33, 10) .. controls (0.35, 10.02) and (0.37, 10.03) ..
					(0.4, 10.05) .. controls (0.42, 10.06) and (0.44, 10.08) ..
					(0.47, 10.09) .. controls (0.49, 10.1) and (0.52, 10.11) ..
					(0.54, 10.11) .. controls (0.57, 10.12) and (0.6, 10.12) ..
					(0.63, 10.12) --cycle
					(0.63, 10.12);
					
					\path[fill=ce9edf5,nonzero rule] (3.93, 10.12) -- (0.63, 10.12) .. controls (0.56, 10.12) and (0.49, 10.1) ..
					(0.43, 10.07) .. controls (0.43, 10.06) and (0.43, 10.04) ..
					(0.42, 10.03) .. controls (0.42, 10.01) and (0.42, 9.99) ..
					(0.42, 9.98) .. controls (0.42, 9.96) and (0.42, 9.95) ..
					(0.42, 9.93) .. controls (0.42, 9.91) and (0.42, 9.9) ..
					(0.43, 9.88) .. controls (0.43, 9.87) and (0.43, 9.85) ..
					(0.44, 9.84) .. controls (0.44, 9.82) and (0.45, 9.81) ..
					(0.45, 9.79) .. controls (0.46, 9.78) and (0.47, 9.76) ..
					(0.48, 9.75) .. controls (0.48, 9.74) and (0.49, 9.72) ..
					(0.5, 9.71) .. controls (0.51, 9.7) and (0.52, 9.68) ..
					(0.53, 9.67) .. controls (0.54, 9.66) and (0.56, 9.65) ..
					(0.57, 9.64) .. controls (0.58, 9.63) and (0.59, 9.62) ..
					(0.61, 9.61) .. controls (0.62, 9.6) and (0.63, 9.6) ..
					(0.65, 9.59) .. controls (0.66, 9.58) and (0.68, 9.58) ..
					(0.69, 9.57) .. controls (0.71, 9.56) and (0.72, 9.56) ..
					(0.74, 9.56) .. controls (0.75, 9.55) and (0.77, 9.55) ..
					(0.78, 9.55) .. controls (0.8, 9.55) and (0.82, 9.55) ..
					(0.83, 9.55) -- (4.14, 9.55) .. controls (4.21, 9.55) and (4.27, 9.56) ..
					(4.33, 9.59) .. controls (4.34, 9.63) and (4.35, 9.67) ..
					(4.35, 9.71) .. controls (4.35, 9.73) and (4.34, 9.76) ..
					(4.34, 9.79) .. controls (4.33, 9.81) and (4.32, 9.84) ..
					(4.31, 9.86) .. controls (4.3, 9.89) and (4.29, 9.91) ..
					(4.28, 9.94) .. controls (4.26, 9.96) and (4.24, 9.98) ..
					(4.22, 10) .. controls (4.21, 10.02) and (4.18, 10.03) ..
					(4.16, 10.05) .. controls (4.14, 10.06) and (4.12, 10.08) ..
					(4.09, 10.09) .. controls (4.07, 10.1) and (4.04, 10.11) ..
					(4.01, 10.11) .. controls (3.99, 10.12) and (3.96, 10.12) ..
					(3.93, 10.12) --cycle
					(3.93, 10.12);
				\end{scope}
		}}
		\begin{tikzpicture}[scale=.7,transform shape]
			\path (0,0) pic[scale=.3]{khongluu};
			\path (.685,0) coordinate (O)
			+(0:4) coordinate (y) node[above] {$y$}
			+(90:5) coordinate (z) node[above] {$z$}
			+(-145:3) coordinate (x) node[above] {$x$}
			;
			\draw[-stealth] (O)--(z);
			\draw[-stealth] (O)--(y);
			\draw[-stealth] (O)--(x);
		\end{tikzpicture}
	}
	Một máy bay đang ở vị trí $A$ cách mặt đất $8$ km, cách $268$ km về phía Đông, $185$ km về phía Nam so với tháp kiểm soát không lưu và đang chuyển động theo đường thẳng $d$ có vectơ chỉ phương là $\overrightarrow{u}=(82;76;0)$ hướng về đài kiểm soát không lưu.
	\choiceTF
	{Vị trí $A$ có tọa độ là $(268;-185;-8)$}
	{\True Đài kiểm soát không lưu có phát hiện được máy bay tại vị trí $A$}
	{\True Phương trình tham số của đường thẳng $d$ là $\heva{&x=-268+82t\\&y=185+76t\\&z=8}$ ($t$ là tham số)}
	{Khoảng cách gần nhất giữa máy bay và đài kiểm soát không lưu là $217{,}98$ km (làm tròn kết quả đến hàng phần trăm)}
	\loigiai{
		\begin{center}
			\tikzset{khongluu/.pic={
					\definecolor{cdce1eb}{RGB}{220,225,235}
					\definecolor{cafb4c8}{RGB}{175,180,200}
					\definecolor{c82d1f5}{RGB}{130,209,245}
					\definecolor{c8ab0e0}{RGB}{138,176,224}
					\definecolor{ce9edf5}{RGB}{233,237,245}
					\definecolor{caaccfa}{RGB}{170,204,250}
					\definecolor{cbec3d2}{RGB}{190,195,210}
					\begin{scope}
						\path[fill=cdce1eb,nonzero rule] (5.13, 5.58) -- (13.02, 5.58) -- (13.02, 0.21) -- (5.13, 0.21) --cycle
						(5.13, 5.58);
						\path[fill=cafb4c8,nonzero rule] (5.13, 5.58) -- (13.02, 5.58) -- (13.02, 4.44) -- (5.13, 4.44) --cycle
						(5.13, 5.58);
						\path[fill=c82d1f5,nonzero rule] (5.14, 4.44) -- (13.03, 4.44) -- (13.03, 3.3) -- (5.14, 3.3) --cycle
						(5.14, 4.44);
						\path[fill=cdce1eb,nonzero rule] (5.13, 5.58) -- (13.02, 5.58) -- (13.02, 0.21) -- (5.13, 0.21) --cycle
						(5.13, 5.58);
						\path[fill=cdce1eb,nonzero rule] (5.13, 5.58) -- (13.02, 5.58) -- (13.02, 4.44) -- (5.13, 4.44) --cycle
						(5.13, 5.58);
						\path[fill=c8ab0e0,nonzero rule] (5.14, 4.44) -- (13.03, 4.44) -- (13.03, 3.3) -- (5.14, 3.3) --cycle
						(5.14, 4.44);
						\path[fill=ce9edf5,nonzero rule] (5.5, 3.3) -- (13.02, 3.3) -- (13.02, 0.7) -- (6.19, 0.7) .. controls (6.17, 0.7) and (6.14, 0.7) ..
						(6.12, 0.7) .. controls (6.1, 0.7) and (6.08, 0.7) ..
						(6.05, 0.71) .. controls (6.03, 0.71) and (6.01, 0.72) ..
						(5.99, 0.73) .. controls (5.97, 0.73) and (5.94, 0.74) ..
						(5.92, 0.75) .. controls (5.9, 0.76) and (5.88, 0.77) ..
						(5.86, 0.78) .. controls (5.84, 0.79) and (5.82, 0.8) ..
						(5.8, 0.81) .. controls (5.79, 0.82) and (5.77, 0.84) ..
						(5.75, 0.85) .. controls (5.73, 0.87) and (5.72, 0.88) ..
						(5.7, 0.9) .. controls (5.68, 0.91) and (5.67, 0.93) ..
						(5.65, 0.95) .. controls (5.64, 0.97) and (5.63, 0.98) ..
						(5.61, 1) .. controls (5.6, 1.02) and (5.59, 1.04) ..
						(5.58, 1.06) .. controls (5.57, 1.08) and (5.56, 1.1) ..
						(5.55, 1.12) .. controls (5.54, 1.14) and (5.53, 1.16) ..
						(5.53, 1.19) .. controls (5.52, 1.21) and (5.52, 1.23) ..
						(5.51, 1.25) .. controls (5.51, 1.27) and (5.5, 1.3) ..
						(5.5, 1.32) .. controls (5.5, 1.34) and (5.5, 1.36) ..
						(5.5, 1.39) --cycle
						(5.5, 3.3);
						\path[fill=ce9edf5,nonzero rule] (13.02, 5.58) -- (13.02, 4.8) -- (6.07, 4.8) .. controls (6.06, 4.8) and (6.04, 4.8) ..
						(6.02, 4.8) .. controls (6, 4.8) and (5.98, 4.8) ..
						(5.96, 4.81) .. controls (5.94, 4.81) and (5.93, 4.82) ..
						(5.91, 4.82) .. controls (5.89, 4.83) and (5.87, 4.83) ..
						(5.85, 4.84) .. controls (5.84, 4.85) and (5.82, 4.86) ..
						(5.8, 4.86) .. controls (5.79, 4.87) and (5.77, 4.88) ..
						(5.75, 4.89) .. controls (5.74, 4.9) and (5.72, 4.92) ..
						(5.71, 4.93) .. controls (5.69, 4.94) and (5.68, 4.95) ..
						(5.67, 4.97) .. controls (5.65, 4.98) and (5.64, 4.99) ..
						(5.63, 5.01) .. controls (5.62, 5.02) and (5.61, 5.04) ..
						(5.6, 5.05) .. controls (5.58, 5.07) and (5.58, 5.08) ..
						(5.57, 5.1) .. controls (5.56, 5.12) and (5.55, 5.13) ..
						(5.54, 5.15) .. controls (5.53, 5.17) and (5.53, 5.19) ..
						(5.52, 5.21) .. controls (5.52, 5.22) and (5.51, 5.24) ..
						(5.51, 5.26) .. controls (5.51, 5.28) and (5.5, 5.3) ..
						(5.5, 5.32) .. controls (5.5, 5.34) and (5.5, 5.35) ..
						(5.5, 5.37) -- (5.5, 5.58) --cycle
						(13.02, 5.58);
						\path[fill=caaccfa,nonzero rule] (13.03, 4.44) -- (13.03, 3.65) -- (6.08, 3.65) .. controls (6.07, 3.65) and (6.05, 3.66) ..
						(6.03, 3.66) .. controls (6.01, 3.66) and (5.99, 3.66) ..
						(5.97, 3.67) .. controls (5.95, 3.67) and (5.94, 3.67) ..
						(5.92, 3.68) .. controls (5.9, 3.69) and (5.88, 3.69) ..
						(5.86, 3.7) .. controls (5.85, 3.71) and (5.83, 3.71) ..
						(5.81, 3.72) .. controls (5.8, 3.73) and (5.78, 3.74) ..
						(5.76, 3.75) .. controls (5.75, 3.76) and (5.73, 3.77) ..
						(5.72, 3.79) .. controls (5.7, 3.8) and (5.69, 3.81) ..
						(5.68, 3.82) .. controls (5.66, 3.84) and (5.65, 3.85) ..
						(5.64, 3.87) .. controls (5.63, 3.88) and (5.62, 3.9) ..
						(5.6, 3.91) .. controls (5.59, 3.93) and (5.58, 3.94) ..
						(5.58, 3.96) .. controls (5.57, 3.98) and (5.56, 3.99) ..
						(5.55, 4.01) .. controls (5.54, 4.03) and (5.54, 4.05) ..
						(5.53, 4.06) .. controls (5.53, 4.08) and (5.52, 4.1) ..
						(5.52, 4.12) .. controls (5.52, 4.14) and (5.51, 4.16) ..
						(5.51, 4.18) .. controls (5.51, 4.19) and (5.51, 4.21) ..
						(5.51, 4.23) -- (5.51, 4.44) --cycle
						(13.03, 4.44);
						\path[fill=cafb4c8,nonzero rule] (11.51, 9.57) -- (10.8, 8.77) .. controls (10.73, 8.77) and (10.21, 8.8) ..
						(9.54, 8.85) -- (10.47, 10.1) -- (10, 10.12) .. controls (9.94, 10.12) and (9.87, 10.11) ..
						(9.81, 10.09) .. controls (9.75, 10.07) and (9.7, 10.04) ..
						(9.65, 9.99) -- (8.48, 8.92) .. controls (8.17, 8.94) and (7.83, 8.96) ..
						(7.72, 8.97) .. controls (7.25, 8.99) and (6.92, 8.5) ..
						(6.92, 8.5) .. controls (6.92, 8.5) and (7.09, 8.09) ..
						(7.54, 8.07) .. controls (7.91, 8.06) and (10, 8.06) ..
						(10.79, 8.06) .. controls (10.84, 8.06) and (10.89, 8.06) ..
						(10.93, 8.07) .. controls (10.98, 8.08) and (11.02, 8.1) ..
						(11.06, 8.12) .. controls (11.1, 8.14) and (11.14, 8.17) ..
						(11.17, 8.2) .. controls (11.21, 8.23) and (11.24, 8.27) ..
						(11.26, 8.31) -- (12.09, 9.54) --cycle
						(11.51, 9.57);
						\path[fill=cbec3d2,nonzero rule] (11, 8.26) .. controls (11.1, 8.26) and (11.2, 8.29) ..
						(11.28, 8.34) -- (12.09, 9.54) -- (11.51, 9.57) -- (10.8, 8.77) .. controls (10.73, 8.77) and (10.21, 8.8) ..
						(9.54, 8.85) -- (10.47, 10.1) -- (10, 10.12) .. controls (9.94, 10.12) and (9.87, 10.11) ..
						(9.81, 10.09) .. controls (9.75, 10.07) and (9.7, 10.04) ..
						(9.65, 9.99) -- (8.48, 8.92) .. controls (8.17, 8.94) and (7.83, 8.96) ..
						(7.71, 8.97) .. controls (7.49, 8.98) and (7.31, 8.87) ..
						(7.17, 8.76) .. controls (7.14, 8.73) and (7.12, 8.71) ..
						(7.12, 8.71) .. controls (7.12, 8.71) and (7.29, 8.29) ..
						(7.75, 8.27) .. controls (8.12, 8.26) and (10.21, 8.26) ..
						(11, 8.26) --cycle
						(11, 8.26);
						\path[fill=cdce1eb,nonzero rule] (11.61, 4.44) -- (11.82, 4.44) -- (11.82, 3.3) -- (11.61, 3.3) --cycle
						(11.61, 4.44);
						\path[fill=cdce1eb,nonzero rule] (10.3, 4.44) -- (10.5, 4.44) -- (10.5, 3.3) -- (10.3, 3.3) --cycle
						(10.3, 4.44);
						\path[fill=cdce1eb,nonzero rule] (8.98, 4.44) -- (9.19, 4.44) -- (9.19, 3.3) -- (8.98, 3.3) --cycle
						(8.98, 4.44);
						\path[fill=cdce1eb,nonzero rule] (7.67, 4.44) -- (7.87, 4.44) -- (7.87, 3.3) -- (7.67, 3.3) --cycle
						(7.67, 4.44);
						\path[fill=cdce1eb,nonzero rule] (6.35, 4.44) -- (6.56, 4.44) -- (6.56, 3.3) -- (6.35, 3.3) --cycle
						(6.35, 4.44);
						\path[fill=cafb4c8,nonzero rule] (2.28, 13.23) .. controls (2.27, 13.23) and (2.25, 13.23) ..
						(2.24, 13.23) .. controls (2.23, 13.22) and (2.21, 13.22) ..
						(2.2, 13.21) .. controls (2.19, 13.21) and (2.18, 13.2) ..
						(2.16, 13.19) .. controls (2.15, 13.19) and (2.14, 13.18) ..
						(2.13, 13.17) .. controls (2.12, 13.16) and (2.11, 13.15) ..
						(2.11, 13.14) .. controls (2.1, 13.13) and (2.09, 13.11) ..
						(2.09, 13.1) .. controls (2.08, 13.09) and (2.08, 13.08) ..
						(2.08, 13.06) .. controls (2.07, 13.05) and (2.07, 13.04) ..
						(2.07, 13.02) -- (2.07, 11.74) -- (2.49, 11.74) -- (2.49, 13.02) .. controls (2.49, 13.04) and (2.48, 13.05) ..
						(2.48, 13.06) .. controls (2.48, 13.08) and (2.48, 13.09) ..
						(2.47, 13.1) .. controls (2.46, 13.11) and (2.46, 13.13) ..
						(2.45, 13.14) .. controls (2.44, 13.15) and (2.43, 13.16) ..
						(2.43, 13.17) .. controls (2.42, 13.18) and (2.41, 13.19) ..
						(2.39, 13.19) .. controls (2.38, 13.2) and (2.37, 13.21) ..
						(2.36, 13.21) .. controls (2.35, 13.22) and (2.33, 13.22) ..
						(2.32, 13.23) .. controls (2.31, 13.23) and (2.29, 13.23) ..
						(2.28, 13.23) --cycle
						(2.28, 13.23);
						\path[fill=cdce1eb,nonzero rule] (3.93, 10.95) -- (3.93, 10.9) -- (0.63, 10.9) -- (0.63, 10.96) .. controls (0.63, 10.98) and (0.63, 11.01) ..
						(0.63, 11.04) .. controls (0.63, 11.06) and (0.64, 11.09) ..
						(0.64, 11.12) .. controls (0.65, 11.14) and (0.65, 11.17) ..
						(0.66, 11.2) .. controls (0.67, 11.22) and (0.68, 11.25) ..
						(0.69, 11.27) .. controls (0.7, 11.3) and (0.71, 11.32) ..
						(0.72, 11.34) .. controls (0.74, 11.37) and (0.75, 11.39) ..
						(0.76, 11.41) .. controls (0.78, 11.44) and (0.79, 11.46) ..
						(0.81, 11.48) .. controls (0.83, 11.5) and (0.85, 11.52) ..
						(0.87, 11.54) .. controls (0.89, 11.56) and (0.91, 11.58) ..
						(0.93, 11.59) .. controls (0.95, 11.61) and (0.97, 11.63) ..
						(0.99, 11.64) .. controls (1.01, 11.66) and (1.04, 11.67) ..
						(1.06, 11.68) .. controls (1.08, 11.69) and (1.11, 11.71) ..
						(1.13, 11.72) .. controls (1.16, 11.73) and (1.18, 11.74) ..
						(1.21, 11.74) .. controls (1.23, 11.75) and (1.26, 11.76) ..
						(1.29, 11.76) .. controls (1.31, 11.77) and (1.34, 11.77) ..
						(1.37, 11.78) .. controls (1.39, 11.78) and (1.42, 11.78) ..
						(1.45, 11.78) -- (3.11, 11.78) .. controls (3.14, 11.78) and (3.16, 11.78) ..
						(3.19, 11.78) .. controls (3.22, 11.77) and (3.24, 11.77) ..
						(3.27, 11.76) .. controls (3.3, 11.76) and (3.32, 11.75) ..
						(3.35, 11.74) .. controls (3.37, 11.74) and (3.4, 11.73) ..
						(3.42, 11.72) .. controls (3.45, 11.71) and (3.47, 11.69) ..
						(3.5, 11.68) .. controls (3.52, 11.67) and (3.54, 11.66) ..
						(3.57, 11.64) .. controls (3.59, 11.63) and (3.61, 11.61) ..
						(3.63, 11.59) .. controls (3.65, 11.58) and (3.67, 11.56) ..
						(3.69, 11.54) .. controls (3.71, 11.52) and (3.73, 11.5) ..
						(3.75, 11.48) .. controls (3.76, 11.46) and (3.78, 11.44) ..
						(3.79, 11.41) .. controls (3.81, 11.39) and (3.82, 11.37) ..
						(3.84, 11.34) .. controls (3.85, 11.32) and (3.86, 11.3) ..
						(3.87, 11.27) .. controls (3.88, 11.25) and (3.89, 11.22) ..
						(3.9, 11.2) .. controls (3.9, 11.17) and (3.91, 11.14) ..
						(3.92, 11.12) .. controls (3.92, 11.09) and (3.93, 11.06) ..
						(3.93, 11.04) .. controls (3.93, 11.01) and (3.93, 10.98) ..
						(3.93, 10.96) --cycle
						(3.93, 10.95);
						\path[fill=ce9edf5,nonzero rule] (3.11, 11.78) -- (1.45, 11.78) .. controls (1.39, 11.78) and (1.34, 11.77) ..
						(1.29, 11.76) .. controls (1.24, 11.7) and (1.21, 11.64) ..
						(1.19, 11.56) .. controls (1.19, 11.55) and (1.18, 11.54) ..
						(1.18, 11.53) .. controls (1.18, 11.52) and (1.18, 11.51) ..
						(1.18, 11.5) .. controls (1.18, 11.48) and (1.18, 11.47) ..
						(1.18, 11.46) .. controls (1.18, 11.45) and (1.18, 11.44) ..
						(1.18, 11.43) .. controls (1.18, 11.42) and (1.18, 11.4) ..
						(1.19, 11.39) .. controls (1.19, 11.38) and (1.19, 11.37) ..
						(1.2, 11.36) .. controls (1.2, 11.35) and (1.21, 11.34) ..
						(1.21, 11.33) .. controls (1.22, 11.32) and (1.22, 11.31) ..
						(1.23, 11.3) .. controls (1.24, 11.29) and (1.25, 11.28) ..
						(1.25, 11.27) .. controls (1.26, 11.27) and (1.27, 11.26) ..
						(1.28, 11.25) .. controls (1.29, 11.24) and (1.3, 11.24) ..
						(1.31, 11.23) .. controls (1.31, 11.22) and (1.32, 11.22) ..
						(1.33, 11.21) .. controls (1.35, 11.21) and (1.36, 11.2) ..
						(1.37, 11.2) .. controls (1.38, 11.19) and (1.39, 11.19) ..
						(1.4, 11.19) .. controls (1.41, 11.19) and (1.42, 11.18) ..
						(1.43, 11.18) .. controls (1.44, 11.18) and (1.46, 11.18) ..
						(1.47, 11.18) -- (3.9, 11.18) .. controls (3.89, 11.22) and (3.87, 11.26) ..
						(3.86, 11.31) .. controls (3.84, 11.35) and (3.81, 11.38) ..
						(3.79, 11.42) .. controls (3.76, 11.46) and (3.74, 11.49) ..
						(3.71, 11.52) .. controls (3.67, 11.56) and (3.64, 11.59) ..
						(3.61, 11.61) .. controls (3.57, 11.64) and (3.53, 11.66) ..
						(3.49, 11.68) .. controls (3.45, 11.7) and (3.41, 11.72) ..
						(3.37, 11.74) .. controls (3.33, 11.75) and (3.29, 11.76) ..
						(3.24, 11.77) .. controls (3.2, 11.78) and (3.15, 11.78) ..
						(3.11, 11.78) --cycle
						(3.11, 11.78);
						\path[fill=cdce1eb,nonzero rule] (0.83, 7.63) .. controls (0.83, 7.63) and (1.41, 6.21) ..
						(1.41, 4.53) -- (1.41, 0.21) -- (3.15, 0.21) -- (3.15, 4.53) .. controls (3.15, 6.21) and (3.73, 7.63) ..
						(3.73, 7.63) -- (3.73, 7.69) -- (0.83, 7.69) --cycle
						(0.83, 7.63);
						\path[fill=ce9edf5,nonzero rule] (1.93, 5.02) -- (1.93, 1.54) .. controls (1.93, 1.51) and (1.93, 1.48) ..
						(1.93, 1.46) .. controls (1.93, 1.43) and (1.94, 1.4) ..
						(1.94, 1.38) .. controls (1.95, 1.35) and (1.95, 1.32) ..
						(1.96, 1.29) .. controls (1.97, 1.27) and (1.98, 1.24) ..
						(1.99, 1.22) .. controls (2, 1.19) and (2.01, 1.17) ..
						(2.03, 1.14) .. controls (2.04, 1.12) and (2.05, 1.09) ..
						(2.07, 1.07) .. controls (2.08, 1.05) and (2.1, 1.03) ..
						(2.12, 1) .. controls (2.14, 0.98) and (2.15, 0.96) ..
						(2.17, 0.94) .. controls (2.19, 0.92) and (2.21, 0.9) ..
						(2.23, 0.89) .. controls (2.26, 0.87) and (2.28, 0.85) ..
						(2.3, 0.84) .. controls (2.32, 0.82) and (2.35, 0.81) ..
						(2.37, 0.8) .. controls (2.4, 0.78) and (2.42, 0.77) ..
						(2.45, 0.76) .. controls (2.47, 0.75) and (2.5, 0.74) ..
						(2.53, 0.73) .. controls (2.55, 0.72) and (2.58, 0.72) ..
						(2.61, 0.71) .. controls (2.63, 0.71) and (2.66, 0.7) ..
						(2.69, 0.7) .. controls (2.71, 0.7) and (2.74, 0.7) ..
						(2.77, 0.7) -- (3.15, 0.7) -- (3.15, 4.53) .. controls (3.15, 6.21) and (3.73, 7.63) ..
						(3.73, 7.63) -- (3.73, 7.69) -- (1.5, 7.69) .. controls (1.67, 7.15) and (1.93, 6.14) ..
						(1.93, 5.02) --cycle
						(1.93, 5.02);
						
						\path[fill=cafb4c8,nonzero rule] (13.02, -0) -- (0.21, -0) .. controls (0.2, -0) and (0.18, 0) ..
						(0.17, 0) .. controls (0.16, 0.01) and (0.15, 0.01) ..
						(0.13, 0.02) .. controls (0.12, 0.02) and (0.11, 0.03) ..
						(0.1, 0.03) .. controls (0.09, 0.04) and (0.08, 0.05) ..
						(0.07, 0.06) .. controls (0.06, 0.07) and (0.05, 0.08) ..
						(0.04, 0.09) .. controls (0.03, 0.1) and (0.03, 0.12) ..
						(0.02, 0.13) .. controls (0.02, 0.14) and (0.01, 0.15) ..
						(0.01, 0.17) .. controls (0.01, 0.18) and (0.01, 0.19) ..
						(0.01, 0.21) .. controls (0.01, 0.22) and (0.01, 0.23) ..
						(0.01, 0.25) .. controls (0.01, 0.26) and (0.02, 0.27) ..
						(0.02, 0.29) .. controls (0.03, 0.3) and (0.03, 0.31) ..
						(0.04, 0.32) .. controls (0.05, 0.33) and (0.06, 0.34) ..
						(0.07, 0.35) .. controls (0.08, 0.36) and (0.09, 0.37) ..
						(0.1, 0.38) .. controls (0.11, 0.39) and (0.12, 0.39) ..
						(0.13, 0.4) .. controls (0.15, 0.4) and (0.16, 0.41) ..
						(0.17, 0.41) .. controls (0.18, 0.41) and (0.2, 0.41) ..
						(0.21, 0.41) -- (13.02, 0.41) .. controls (13.03, 0.41) and (13.04, 0.41) ..
						(13.06, 0.41) .. controls (13.07, 0.41) and (13.08, 0.4) ..
						(13.1, 0.4) .. controls (13.11, 0.39) and (13.12, 0.39) ..
						(13.13, 0.38) .. controls (13.14, 0.37) and (13.15, 0.36) ..
						(13.16, 0.35) .. controls (13.17, 0.34) and (13.18, 0.33) ..
						(13.19, 0.32) .. controls (13.2, 0.31) and (13.2, 0.3) ..
						(13.21, 0.29) .. controls (13.21, 0.27) and (13.22, 0.26) ..
						(13.22, 0.25) .. controls (13.22, 0.23) and (13.22, 0.22) ..
						(13.22, 0.21) .. controls (13.22, 0.19) and (13.22, 0.18) ..
						(13.22, 0.17) .. controls (13.22, 0.15) and (13.21, 0.14) ..
						(13.21, 0.13) .. controls (13.2, 0.12) and (13.2, 0.1) ..
						(13.19, 0.09) .. controls (13.18, 0.08) and (13.17, 0.07) ..
						(13.16, 0.06) .. controls (13.15, 0.05) and (13.14, 0.04) ..
						(13.13, 0.03) .. controls (13.12, 0.03) and (13.11, 0.02) ..
						(13.1, 0.02) .. controls (13.08, 0.01) and (13.07, 0.01) ..
						(13.06, 0) .. controls (13.04, 0) and (13.03, -0) ..
						(13.02, -0) --cycle
						(13.02, -0);
						
						\path[fill=c8ab0e0,nonzero rule] (0.63, 10.95) -- (3.93, 10.95) -- (3.93, 10.09) -- (0.63, 10.09) --cycle
						(0.63, 10.95);
						
						\path[fill=caaccfa,nonzero rule] (1.26, 10.38) -- (3.93, 10.38) -- (3.93, 10.95) -- (1.01, 10.95) -- (1.01, 10.64) .. controls (1.01, 10.62) and (1.01, 10.6) ..
						(1.02, 10.59) .. controls (1.02, 10.57) and (1.02, 10.56) ..
						(1.03, 10.54) .. controls (1.04, 10.53) and (1.04, 10.51) ..
						(1.05, 10.5) .. controls (1.06, 10.48) and (1.07, 10.47) ..
						(1.08, 10.46) .. controls (1.1, 10.45) and (1.11, 10.44) ..
						(1.12, 10.43) .. controls (1.14, 10.42) and (1.15, 10.41) ..
						(1.17, 10.4) .. controls (1.18, 10.4) and (1.2, 10.39) ..
						(1.21, 10.39) .. controls (1.23, 10.39) and (1.25, 10.38) ..
						(1.26, 10.38) --cycle
						(1.26, 10.38);
						
						\path[fill=cdce1eb,nonzero rule] (2.18, 10.96) -- (2.38, 10.96) -- (2.38, 10.08) -- (2.18, 10.08) --cycle
						(2.18, 10.96);
						
						\path[fill=cdce1eb,nonzero rule] (1.35, 10.96) -- (1.56, 10.96) -- (1.56, 10.08) -- (1.35, 10.08) --cycle
						(1.35, 10.96);
						
						\path[fill=cdce1eb,nonzero rule] (3, 10.96) -- (3.21, 10.96) -- (3.21, 10.08) -- (3, 10.08) --cycle
						(3, 10.96);
						
						\path[fill=c8ab0e0,nonzero rule] (0.83, 9.34) -- (3.73, 9.34) -- (3.73, 7.63) -- (0.83, 7.63) --cycle
						(0.83, 9.34);
						
						\path[fill=caaccfa,nonzero rule] (1.61, 7.91) -- (3.73, 7.91) -- (3.73, 9.34) -- (1.25, 9.34) -- (1.25, 8.27) .. controls (1.25, 8.24) and (1.25, 8.22) ..
						(1.26, 8.2) .. controls (1.26, 8.17) and (1.27, 8.15) ..
						(1.28, 8.13) .. controls (1.29, 8.11) and (1.3, 8.09) ..
						(1.31, 8.07) .. controls (1.33, 8.05) and (1.34, 8.03) ..
						(1.36, 8.01) .. controls (1.37, 8) and (1.39, 7.98) ..
						(1.41, 7.97) .. controls (1.43, 7.95) and (1.45, 7.94) ..
						(1.48, 7.93) .. controls (1.5, 7.92) and (1.52, 7.92) ..
						(1.54, 7.91) .. controls (1.57, 7.91) and (1.59, 7.91) ..
						(1.61, 7.91) --cycle
						(1.61, 7.91);
						
						\path[fill=cdce1eb,nonzero rule] (3.73, 8.57) -- (2.38, 8.57) -- (2.38, 9.34) -- (2.18, 9.34) -- (2.18, 8.57) -- (0.83, 8.57) -- (0.83, 8.36) -- (2.18, 8.36) -- (2.18, 7.63) -- (2.38, 7.63) -- (2.38, 8.36) -- (3.73, 8.36) --cycle
						(3.73, 8.57);
						
						\path[fill=cdce1eb,nonzero rule] (0.63, 10.12) -- (3.93, 10.12) .. controls (3.96, 10.12) and (3.99, 10.12) ..
						(4.01, 10.11) .. controls (4.04, 10.11) and (4.07, 10.1) ..
						(4.09, 10.09) .. controls (4.12, 10.08) and (4.14, 10.06) ..
						(4.16, 10.05) .. controls (4.18, 10.03) and (4.21, 10.02) ..
						(4.22, 10) .. controls (4.24, 9.98) and (4.26, 9.96) ..
						(4.28, 9.94) .. controls (4.29, 9.91) and (4.3, 9.89) ..
						(4.31, 9.86) .. controls (4.32, 9.84) and (4.33, 9.81) ..
						(4.34, 9.79) .. controls (4.34, 9.76) and (4.35, 9.73) ..
						(4.35, 9.71) .. controls (4.35, 9.68) and (4.34, 9.65) ..
						(4.34, 9.63) .. controls (4.33, 9.6) and (4.32, 9.57) ..
						(4.31, 9.55) .. controls (4.3, 9.52) and (4.29, 9.5) ..
						(4.28, 9.48) .. controls (4.26, 9.45) and (4.24, 9.43) ..
						(4.22, 9.41) .. controls (4.21, 9.39) and (4.18, 9.38) ..
						(4.16, 9.36) .. controls (4.14, 9.35) and (4.12, 9.33) ..
						(4.09, 9.32) .. controls (4.07, 9.31) and (4.04, 9.31) ..
						(4.01, 9.3) .. controls (3.99, 9.3) and (3.96, 9.29) ..
						(3.93, 9.29) -- (0.63, 9.29) .. controls (0.6, 9.29) and (0.57, 9.3) ..
						(0.54, 9.3) .. controls (0.52, 9.31) and (0.49, 9.31) ..
						(0.47, 9.32) .. controls (0.44, 9.33) and (0.42, 9.35) ..
						(0.4, 9.36) .. controls (0.37, 9.38) and (0.35, 9.39) ..
						(0.33, 9.41) .. controls (0.31, 9.43) and (0.3, 9.45) ..
						(0.28, 9.48) .. controls (0.27, 9.5) and (0.25, 9.52) ..
						(0.24, 9.55) .. controls (0.23, 9.57) and (0.23, 9.6) ..
						(0.22, 9.63) .. controls (0.21, 9.65) and (0.21, 9.68) ..
						(0.21, 9.71) .. controls (0.21, 9.73) and (0.21, 9.76) ..
						(0.22, 9.79) .. controls (0.23, 9.81) and (0.23, 9.84) ..
						(0.24, 9.86) .. controls (0.25, 9.89) and (0.27, 9.91) ..
						(0.28, 9.94) .. controls (0.3, 9.96) and (0.31, 9.98) ..
						(0.33, 10) .. controls (0.35, 10.02) and (0.37, 10.03) ..
						(0.4, 10.05) .. controls (0.42, 10.06) and (0.44, 10.08) ..
						(0.47, 10.09) .. controls (0.49, 10.1) and (0.52, 10.11) ..
						(0.54, 10.11) .. controls (0.57, 10.12) and (0.6, 10.12) ..
						(0.63, 10.12) --cycle
						(0.63, 10.12);
						
						\path[fill=ce9edf5,nonzero rule] (3.93, 10.12) -- (0.63, 10.12) .. controls (0.56, 10.12) and (0.49, 10.1) ..
						(0.43, 10.07) .. controls (0.43, 10.06) and (0.43, 10.04) ..
						(0.42, 10.03) .. controls (0.42, 10.01) and (0.42, 9.99) ..
						(0.42, 9.98) .. controls (0.42, 9.96) and (0.42, 9.95) ..
						(0.42, 9.93) .. controls (0.42, 9.91) and (0.42, 9.9) ..
						(0.43, 9.88) .. controls (0.43, 9.87) and (0.43, 9.85) ..
						(0.44, 9.84) .. controls (0.44, 9.82) and (0.45, 9.81) ..
						(0.45, 9.79) .. controls (0.46, 9.78) and (0.47, 9.76) ..
						(0.48, 9.75) .. controls (0.48, 9.74) and (0.49, 9.72) ..
						(0.5, 9.71) .. controls (0.51, 9.7) and (0.52, 9.68) ..
						(0.53, 9.67) .. controls (0.54, 9.66) and (0.56, 9.65) ..
						(0.57, 9.64) .. controls (0.58, 9.63) and (0.59, 9.62) ..
						(0.61, 9.61) .. controls (0.62, 9.6) and (0.63, 9.6) ..
						(0.65, 9.59) .. controls (0.66, 9.58) and (0.68, 9.58) ..
						(0.69, 9.57) .. controls (0.71, 9.56) and (0.72, 9.56) ..
						(0.74, 9.56) .. controls (0.75, 9.55) and (0.77, 9.55) ..
						(0.78, 9.55) .. controls (0.8, 9.55) and (0.82, 9.55) ..
						(0.83, 9.55) -- (4.14, 9.55) .. controls (4.21, 9.55) and (4.27, 9.56) ..
						(4.33, 9.59) .. controls (4.34, 9.63) and (4.35, 9.67) ..
						(4.35, 9.71) .. controls (4.35, 9.73) and (4.34, 9.76) ..
						(4.34, 9.79) .. controls (4.33, 9.81) and (4.32, 9.84) ..
						(4.31, 9.86) .. controls (4.3, 9.89) and (4.29, 9.91) ..
						(4.28, 9.94) .. controls (4.26, 9.96) and (4.24, 9.98) ..
						(4.22, 10) .. controls (4.21, 10.02) and (4.18, 10.03) ..
						(4.16, 10.05) .. controls (4.14, 10.06) and (4.12, 10.08) ..
						(4.09, 10.09) .. controls (4.07, 10.1) and (4.04, 10.11) ..
						(4.01, 10.11) .. controls (3.99, 10.12) and (3.96, 10.12) ..
						(3.93, 10.12) --cycle
						(3.93, 10.12);
					\end{scope}
			}}
			\begin{tikzpicture}[scale=.7,transform shape]
				\path (0,0) pic[scale=.3]{khongluu};
				\path (.685,0) coordinate (O) node[below right] {$O$}
				+(0:4) coordinate (y) node[below] {$y$ (Hướng Nam)}
				+(90:5) coordinate (z) node[above] {$z$}
				+(-145:3) coordinate (x) node[below] {$x$ (Hướng Tây)}
				;
				\draw[-stealth] (O)--(z);
				\draw[-stealth] (O)--(y);
				\draw[-stealth] (O)--(x);
			\end{tikzpicture}
		\end{center}
		\begin{itemchoice}
			\itemch {\bf Sai}.\\
			Vị trí $A$ cách mặt đất $8$ km, cách $268$ km về phía Đông, $185$ km về phía Nam nên ta có $A\left(-268;185;8\right)$.
			\itemch {\bf Đúng}.\\
			Tọa độ của đài kiểm soát là $M(0;0;0{,}105)$.\\
			Khoảng cách từ đài kiểm soát đến máy bay là
			$$
			MA=\sqrt{(-268)^2+185^2+(8-0{,}105)^2}\approx 325{,}7.
			$$
			Vậy $MA<450$ nên đài kiểm soát có phát hiện được máy bay tại vị trí $A$.
			\itemch {\bf Đúng}.\\
			Đường thẳng $d$ đi qua điểm $A(-268;185;8)$ và có véctơ chỉ phương là $\overrightarrow{u}=(86;76;0)$ nên phương trình tham số của đường thẳng $d$ là $\heva{&x=-268+82t\\&y=185+76t\\&z=8}$ ($t$ là tham số)
			\itemch {\bf Sai}.\\
			Khoảng cách gần nhất giữa máy bay và đài kiểm soát không lưu chính là khoảng cách từ đài kiểm soát không lưu đến quỹ đạo chuyển động $d$ của máy bay.\\
			Ta có $\overrightarrow{MA}=(-268;185;7{,}895)$.\\
			Vậy $h=\dfrac{\left|\left[\overrightarrow{MA},\overrightarrow{u}\right]\right|}{|\overrightarrow{u}|}=317{,}96$ (km).
		\end{itemchoice}
	}
\end{ex}
\Closesolutionfile{ans}
%{\fontfamily{qtm}\fontsize{13pt}{2pt}\selectfont\textbf{PHẦN III. Câu trắc nghiệm trả lời ngắn}. Thí sinh trả lời từ câu 1 đến câu 6 và điền đáp án vào ô trống.}
%\setcounter{ex}{0}% Reset lại số đếm câu hỏi
\TNSA
\Opensolutionfile{ans}[ans/de11-phanIII]
\begin{ex}%[1H8V6-2]
	Cho hình chóp $S.ABC$ có $ABC$, $SAB$ là các tam giác đều và mặt bên $(SAB)$ vuông góc với mặt đáy. Gọi $\alpha$ là góc phẳng nhị diện $[S,BC,A]$. Tính $\cos^2\alpha$.
	\shortans{$0{,}2$}
	\loigiai{
		\begin{center}
			\begin{tikzpicture}[scale=1, font=\footnotesize,>=stealth, line width=1pt]%<DTools>
				%Gán số liệu.
				\def\canhAC{4};\def\canhBA{2};\def\gocBAC{-50};\def\h{3};\def\xdinhS{0};
				%Gán tọa độ.
				\coordinate (A) at (0,0);
				\coordinate (B) at ($(A)+(\gocBAC:\canhBA)$);
				\coordinate (C) at ($(A)+(0:\canhAC)$);
				\path
				($(A)!.5!(B)$) coordinate (M)
				+(90:\h) coordinate (S)
				($(B)!.5!(C)$) coordinate (N)
				($(B)!.5!(N)$) coordinate (P)
				;
				
				
				
				%Vẽ khối chóp S.ABC.
				\draw (S)--(B) (S)--(A)--(B) (S)--(C)--(B)
				(S)--(M)
				(S)--(P)
				\foreach \x/\y/\z in {B/M/S,P/M/S,N/P/M,C/N/A}{
					pic[draw, thin, angle radius = 6pt]{right angle = \x--\y--\z}
				}
				pic[draw, thin, angle radius = 12pt, "$\alpha$", angle eccentricity = 1.5]{ angle = S--P--M}
				
				
				;
				\draw[dashed] (A)--(C)
				(M)--(P)
				(A)--(N)
				
				;
				%Gán nhãn.
				\foreach \x/\y in {S/90,A/180,B/-90,C/0,M/180,P/-30,N/-30}{\fill (\x) circle (1pt) ($(\x)+(\y:0.3cm)$) node{$\x$};}
			\end{tikzpicture}
		\end{center}
		Gọi $M$ là trung điểm của $AB$, $N$ là trung điểm của $BC$, $P$ là trung điểm của $BN$.\\
		Ta có $MPallel AN$ mà $AN\perp BC$ (do tam giác $ABC$ đều) nên $MP\perp BC$.\\
		Mặt bên $(SAB)$ vuông góc với đáy mà $SM\perp AB$ ($SAB$ là tam giác đều), suy ra $SM \perp (ABC) \Rightarrow SM \perp BC$.\\
		Vậy góc $\alpha =\widehat{SPM} \Rightarrow \cos \alpha = \dfrac{MP}{SP}$.\\
		Gọi $a$ là độ dài cạnh của tam giác đều $ABC$ và $SAB$.\\
		Ta có $MP=\dfrac{AN}{2}=\dfrac{a\sqrt{3}}{4}$; $AM=\dfrac{a\sqrt{3}}{2}$.\\
		Suy ra $SP=\sqrt{SM^2+MP^2}=\sqrt{\dfrac{3a^2}{4}+\dfrac{3a^2}{16}}=\dfrac{a\sqrt{15}}{4}$.\\
		Vậy $\cos \alpha = \dfrac{MP}{SP}=\dfrac{\dfrac{a\sqrt{3}}{4}}{\dfrac{a\sqrt{15}}{4}}=\dfrac{1}{\sqrt{5}}$.
		Suy ra $\cos^2 \alpha=\dfrac{1}{5}=0{,}2$.
	}
\end{ex}

\begin{ex}%[2D4V1-6]
	Một người bình thường với chiều cao $h$ cm, nặng $w$ kilogram có diện tích bề mặt cơ thể $S$ được mô hình hoá bởi công thức $S=\dfrac{1}{60}\cdot w^{0{,}5}\cdot h^{0{,}5}$ (m$^2$) (công thức Mosteller).\\
	Một đối tượng có chiều cao bằng $168$ cm, nặng $62$ kg tham gia một cuộc nghiên cứu về sức khỏe trong $5$ năm. Người ta nhận thấy cân nặng của đối tượng quan sát thay đổi với tốc độ $w'(t)=0{,}02t^2+0{,}2t$ kg/năm ($0\le t\le 5$) và chiều cao tăng đều mỗi năm $0{,}5$ cm. Sau $5$ năm quan sát, diện tích bề mặt cơ thể của đối tượng trên tăng thêm bao nhiêu centimet vuông so với ban đầu? (làm tròn kết quả đến hàng đơn vị).
	\shortans{$581$}
	\loigiai{
		Chiều cao của người này sau $5$ năm là
		\begin{center}
			$h_5=h_0+0{,}5t=168+5\times 0{,}5=170{,}5$ cm $=1{,}705$ m.
		\end{center}
		Cân nặng của người ngày sau $5$ năm là
		$$w_5=w_0++\displaystyle\int\limits_0^5\left(0{,}02t^2+0{,}2t\right)\mathrm{\, d}t=62+\dfrac{10}{3}=\dfrac{196}{3}~\mathrm{kg}.$$
		Diện tích bề mặt cơ thể sau $5$ năm tăng thêm là
		$$\begin{aligned}
			\Delta S &= S_5-S_0=\dfrac{1}{60}w_5^{0{,}5}\cdot h_5^{0{,}5}-\dfrac{1}{60}w_0^{0{,}5}\cdot h_0^{0{,}5}\\&=\dfrac{1}{60}\cdot\left(\dfrac{196}{3}\right)^{0{,}5}\cdot 1{,}705^{0{,}5}-\dfrac{1}{60}\cdot62^{0{,}5}\cdot 1{,}68^{0{,}5}=58{,}1\cdot 10^{-3}~\mathrm{m^2}\approx 581~\mathrm{cm^2}
		\end{aligned}$$
	}
\end{ex}

\begin{ex}%[0D8V2-4]
	Một lớp học hè có $15$ học sinh. Biết rằng mỗi ngày $3$ học sinh trong lớp có nhiệm vụ trực nhật sau giờ học. Sau khi kết thúc khóa học hè, người ta thấy rằng hai học sinh bất kỳ trực nhật cùng nhau đúng một ngày. Hỏi lớp học hè kéo dài trong bao nhiêu ngày?
	\shortans{$35$}
	\loigiai{
		Tổng số cặp học sinh có thể tạo thành từ $15$ học sinh là một tổ hợp chập $2$ của $15$ phần tử
		$$\mathrm{C}_{15}^2=105~\text{cặp}.$$
		Mỗi nhóm trực có $3$ học sinh nên số cặp trong mỗi nhóm là $3$ cặp.\\
		Để mỗi học sinh trực nhật cùng nhau đúng một ngày thì tổng số ngày tối đa là $105:3=35$ ngày.
		
	}
\end{ex}

\begin{ex}%[2D1V5-8]
	Giả sử cường độ ánh sáng của một nguồn điểm tỉ lệ thuận với cường độ của nguồn sáng đó và tỉ lệ nghịch với bình phương khoảng cách từ điểm đó đến nguồn sáng. Hai nguồn điểm có cường độ lần lượt là $S$ và $8S$, cách nhau $90$ cm. Xét một điểm $M$ nằm trên đoạn thẳng nối hai nguồn, cường độ ánh sáng tại điểm đó nhỏ nhất thì điểm đó cách nguồn có cường độ $S$ bằng bao nhiêu centimet? (cho biết cường độ sáng tại điểm $M$ bằng tổng cường độ sáng mỗi nguồn tại điểm đó).
	\begin{center}
		\begin{tikzpicture}[scale=1, font=\footnotesize,line join=round, line cap=round, >=stealth]
			%			\draw[gray,xstep = 1, ystep = 1] (0,0) grid (5,5);
			\path
			(3,0) coordinate (M) node[above] {$M$}
			;
			\node[circle, line width = .2 mm, draw = black, anchor = center, minimum size = 1cm] (D1) at (0,0) {};
			\node[circle, line width = .2 mm, draw = black, anchor = center, minimum size = 1cm] (D2) at (9,0) {};
			
			\draw (D1.0)--(D2.180)
			(D1.45)--(D1.-135)
			(D1.135)--(D1.-45)
			(D2.45)--(D2.-135)
			(D2.135)--(D2.-45)
			;
			\fill
			(D1.center) circle (2pt)
			(D2.center) circle (2pt)
			(M) circle (2pt)
			;
			\draw[stealth-stealth] (0,1) -- (9,1) node[midway,above] {$90$ cm};
		\end{tikzpicture}
	\end{center}
	\shortans{$30$}
	\loigiai{
		Gọi $I$ là cường độ ánh sáng. Vì $I$ tỉ lệ thuận với cường độ của nguồn sáng và tỉ lệ nghịch với bình phương khoảng cách từ điểm đó đến nguồn sáng nên $I=k\dfrac{S}{r^2}$.\\
		Giả sử điểm $M$ nằm cách nguồn sáng $S$ (bên trái) một khoảng là $x$ thì $M$ cách nguồn sáng $8S$ một khoảng là $90-x$.\\
		Ta có cường độ ánh sáng tại điểm $M$ do nguồn sáng $1$ gây ra là $I_1=k\dfrac{S}{x^2}$.\\
		Tương tự cường độ ánh sáng tại điểm $M$ do nguồn sáng $1$ gây ra là $I_1=k\dfrac{8S}{(90-x)^2}$.\\
		Vậy cường độ sáng tổng hợp lại $M$ là
		$$I(x)=I_1+I_2=k\dfrac{S}{x^2}+k\dfrac{8S}{(90-x)^2}$$
		Ta có $I'(x)=-k\dfrac{2Sx}{x^4}-k\dfrac{-16S(90-x)}{(90-x)^4}=kS\left(\dfrac{16}{(90-x)^3}-\dfrac{2}{x^3}\right)$.\\
		$I'(x)=0 \Rightarrow (90-x)^3=8x^3 \Rightarrow 90-x=2x \Rightarrow x=30$ cm.\\
		
		
	}
\end{ex}

\begin{ex}%[2D6V2-3]
	Ở vùng $A$ có hai nhóm, nhóm $1$ là nhóm người có thu nhập tốt (trên $15$ triệu đồng/tháng) và nhóm $2$ là nhóm có thu nhập không tốt. Ở vùng $A$ có $40\%$ người có thu nhập tốt và $58\%$ người không gửi tiết kiệm. Khảo sát độc lập những người thuộc nhóm $1$ và nhóm $2$ và tính tỉ lệ phần trăm số người gửi tiết kiệm của từng nhóm thì thấy rằng: Tỉ lệ người gửi tiết kiệm của nhóm $1$ gấp đôi tỉ lệ người tiết kiệm của nhóm $2$. Giả sử một người ở vùng $A$ không gửi tiết kiệm. Xác suất để người ấy có thu nhập tốt là bao nhiêu $\%$? (kết quả làm tròn đến hàng phần chục).
	
	\shortans{$28$}
	\loigiai{
		Gọi $A$ là biến cố \lq\lq Thu nhập tốt\rq\rq.\\
		Gọi $B$ là biến cố \lq\lq Không gửi tiết kiệm\rq\rq.\\
		Gọi $x$ là tỉ lệ người gửi tiết kiệm ở nhóm $1$, $y$ là tỷ lệ người gửi tiết kiệm ở nhóm $2$.\\
		Theo đề bài ta có $x=2y\quad (1)$.\\
		Gọi $N$ là tổng số người ở vùng $A$, thì số người không gửi tiết kiệm ở nhóm $1$ là $(1-x)\cdot 0{,}4 N$.\\
		Số người không gửi tiết kiệm ở nhóm $2$ là $(1-y)\cdot 0{,}6 N$.\\
		Vì tổng số người không gửi tiết kiệm ở vùng $A$ là $0{,}58N$ nên ta có
		$$(1-x)\cdot 0{,}4 N + (1-y)\cdot 0{,}6 N = 0{,}58N \Leftrightarrow (1-x)\cdot 0{,}4  + (1-y)\cdot 0{,}6  = 0{,}58.\quad(2)$$
		Giải $(1)$ và $(2)$ ta có $x=0{,}6$; $y=0{,}3$.\\
		Theo đề bài ta có
		$P(A)=0{,}4$ và $P(B)=0{,}58$.\\
		Xác suất người không gửi tiết kiệm, biết người đó thu nhập tốt là
		$\mathrm{P}(B\mid A)= 1-0{,}6=0{,}4$.\\
		Vậy xác suất người có thu nhập tốt khi biết người đó không gửi tiết kiệm là
		$$P(A\mid B)=\dfrac{\mathrm{P}(A)\cdot \mathrm{P}(B\mid A)}{\mathrm{P}(B)}=\dfrac{0{,}4\cdot 0{,}4}{0{,}58}=28\%.$$
	}
\end{ex}

\begin{ex}%[2H5V2-7]
	Một radar có thể quay $180^\circ$ để quan sát máy bay quanh vùng phủ sóng của nó. Một máy bay cất cánh từ điểm $A$ nằm trên mặt đất theo chiều cùng chiều với vectơ $\vec{AB}$. Trong hệ tọa độ $Oxyz$ mặt đất là mặt phẳng $(Oxy)$, trục $O z$ hướng lên trời, điểm $A$ nằm trên trục $O y$ cách gốc tọa độ $0{,}6$ km; điểm $B$ nằm trên trục $O z$ có cao độ bằng $0{,}3$ km; radar đang nằm trên trục $Ox$ có hoành độ bằng $0{,}4$ km. Máy bay đang ở điểm $B$ bay theo hướng bay như cũ đến điểm $C(a;b;c)$ thì radar quay một góc bằng $60^\circ$. Tính $a+b+c$ theo đơn vị mét (làm tròn đến hàng đơn vị).
	\begin{center}
		\tikzset{rada/.pic={
				\definecolor{c191716}{RGB}{25,23,22}
				
				\begin{scope}[line cap=round,line join=round]
					\path[fill=white,nonzero rule] (0, 13.21) -- (10.42, 13.21) -- (10.42, 0.02) -- (0, 0.02) --cycle
					(0, 13.21);
					
					\path[fill=white,nonzero rule] (0, 13.21) -- (10.42, 13.21) -- (10.42, 0.02) -- (0, 0.02) --cycle
					(0, 13.21);
					
					\path[fill=c191716,nonzero rule] (0.03, 1.5) .. controls (0, 1.74) and (0.05, 1.99) ..
					(0.18, 2.21) .. controls (0.18, 2.21) and (0.18, 2.21) ..
					(0.18, 2.22) .. controls (0.2, 2.23) and (0.22, 2.23) ..
					(0.24, 2.22) -- (1.73, 0.74) -- (1.73, 0.74) .. controls (1.73, 0.74) and (1.73, 0.73) ..
					(1.73, 0.73) .. controls (1.75, 0.71) and (1.74, 0.69) ..
					(1.72, 0.67) .. controls (1.58, 0.59) and (1.42, 0.54) ..
					(1.25, 0.53) .. controls (1.09, 0.51) and (0.92, 0.53) ..
					(0.77, 0.59) .. controls (0.69, 0.61) and (0.62, 0.65) ..
					(0.55, 0.69) .. controls (0.48, 0.74) and (0.41, 0.79) ..
					(0.35, 0.85) .. controls (0.17, 1.03) and (0.06, 1.26) ..
					(0.03, 1.5);
					
					\path[fill=c191716,even odd rule] (0.39, 0.71) -- (0.06, 0.03) -- (1.2, 0.03) -- (1.2, 0.03) -- (1.22, 0.03) -- (1.04, 0.44) .. controls (0.94, 0.45) and (0.84, 0.47) ..
					(0.74, 0.51) .. controls (0.66, 0.54) and (0.58, 0.58) ..
					(0.5, 0.62) .. controls (0.46, 0.65) and (0.42, 0.68) ..
					(0.39, 0.71);
					
					\path[fill=c191716,even odd rule] (1.21, 1.76) .. controls (1.24, 1.79) and (1.28, 1.79) ..
					(1.31, 1.76) .. controls (1.34, 1.73) and (1.34, 1.69) ..
					(1.31, 1.66) -- (1.03, 1.38) .. controls (1, 1.35) and (0.95, 1.35) ..
					(0.93, 1.38) .. controls (0.9, 1.41) and (0.9, 1.45) ..
					(0.93, 1.48) --
					(1.21, 1.76);
					
					\path[fill=c191716,nonzero rule] (1.33, 1.94) .. controls (1.37, 1.94) and (1.41, 1.92) ..
					(1.45, 1.89) .. controls (1.48, 1.86) and (1.49, 1.82) ..
					(1.49, 1.78) .. controls (1.49, 1.74) and (1.48, 1.7) ..
					(1.45, 1.67) .. controls (1.41, 1.64) and (1.37, 1.62) ..
					(1.33, 1.62) .. controls (1.29, 1.62) and (1.25, 1.64) ..
					(1.22, 1.67) .. controls (1.19, 1.7) and (1.17, 1.74) ..
					(1.17, 1.78) .. controls (1.17, 1.82) and (1.19, 1.86) ..
					(1.22, 1.89) .. controls (1.25, 1.92) and (1.29, 1.94) ..
					(1.33, 1.94);
					
					\path[fill=c191716,even odd rule] (1.49, 2.47) .. controls (1.56, 2.49) and (1.65, 2.49) ..
					(1.72, 2.47) .. controls (1.8, 2.45) and (1.87, 2.42) ..
					(1.93, 2.36) .. controls (1.99, 2.3) and (2.03, 2.23) ..
					(2.05, 2.15) .. controls (2.06, 2.08) and (2.06, 1.99) ..
					(2.04, 1.92) .. controls (2.03, 1.88) and (2.06, 1.84) ..
					(2.1, 1.83) .. controls (2.13, 1.82) and (2.17, 1.85) ..
					(2.18, 1.88) .. controls (2.2, 1.98) and (2.21, 2.09) ..
					(2.18, 2.19) .. controls (2.16, 2.29) and (2.11, 2.38) ..
					(2.03, 2.46) .. controls (1.96, 2.53) and (1.86, 2.58) ..
					(1.76, 2.61) .. controls (1.66, 2.63) and (1.55, 2.63) ..
					(1.45, 2.61) .. controls (1.42, 2.6) and (1.39, 2.56) ..
					(1.4, 2.52) .. controls (1.41, 2.48) and (1.45, 2.46) ..
					(1.49, 2.47) --cycle
					(1.43, 2.26) .. controls (1.49, 2.28) and (1.55, 2.28) ..
					(1.6, 2.26) .. controls (1.66, 2.25) and (1.71, 2.22) ..
					(1.75, 2.18) .. controls (1.8, 2.14) and (1.82, 2.09) ..
					(1.84, 2.03) .. controls (1.85, 1.98) and (1.85, 1.92) ..
					(1.83, 1.86) .. controls (1.83, 1.82) and (1.85, 1.78) ..
					(1.89, 1.78) .. controls (1.93, 1.77) and (1.96, 1.79) ..
					(1.97, 1.83) .. controls (1.99, 1.91) and (1.99, 1.99) ..
					(1.97, 2.07) .. controls (1.96, 2.15) and (1.91, 2.22) ..
					(1.85, 2.28) .. controls (1.79, 2.34) and (1.72, 2.38) ..
					(1.64, 2.4) .. controls (1.56, 2.42) and (1.48, 2.42) ..
					(1.4, 2.4) .. controls (1.36, 2.39) and (1.34, 2.35) ..
					(1.34, 2.32) .. controls (1.35, 2.28) and (1.39, 2.25) ..
					(1.43, 2.26) --cycle
					(1.47, 2) .. controls (1.43, 2) and (1.39, 2.02) ..
					(1.38, 2.06) .. controls (1.38, 2.09) and (1.4, 2.13) ..
					(1.44, 2.14) .. controls (1.47, 2.15) and (1.51, 2.15) ..
					(1.55, 2.14) .. controls (1.59, 2.13) and (1.63, 2.11) ..
					(1.66, 2.08) .. controls (1.68, 2.06) and (1.7, 2.02) ..
					(1.71, 1.98) .. controls (1.72, 1.94) and (1.72, 1.9) ..
					(1.71, 1.87) .. controls (1.7, 1.83) and (1.67, 1.81) ..
					(1.63, 1.82) .. controls (1.59, 1.82) and (1.57, 1.86) ..
					(1.57, 1.9) .. controls (1.58, 1.92) and (1.58, 1.93) ..
					(1.58, 1.95) .. controls (1.57, 1.96) and (1.57, 1.97) ..
					(1.56, 1.98) .. controls (1.54, 1.99) and (1.53, 2) ..
					(1.52, 2.01) .. controls (1.5, 2.01) and (1.49, 2.01) ..
					(1.47, 2);
					
					\path[fill=c191716,even odd rule] (10.37, 13.11) .. controls (10.4, 13.05) and (10.41, 12.96) ..
					(10, 12.72) .. controls (9.98, 12.72) and (9.97, 12.71) ..
					(9.96, 12.7) -- (9.77, 11.86) -- (9.63, 11.78) -- (9.62, 12.52) .. controls (9.39, 12.41) and (9.29, 12.38) ..
					(9.15, 12.33) -- (9.09, 11.97) -- (8.96, 11.9) -- (8.93, 12.27) -- (8.63, 12.48) -- (8.75, 12.56) -- (9.08, 12.43) .. controls (9.2, 12.53) and (9.27, 12.61) ..
					(9.48, 12.74) -- (8.86, 13.12) -- (9, 13.2) -- (9.81, 12.95) .. controls (9.83, 12.96) and (9.84, 12.96) ..
					(9.85, 12.97) .. controls (10.27, 13.21) and (10.34, 13.16) ..
					(10.37, 13.11) --cycle
					(10.37, 13.11);
					
				\end{scope}
				
				
		}}
		\begin{tikzpicture}[scale=.7,transform shape]
			\path (0,0) pic[scale=.3]{rada};
			\path (2.9,2) coordinate (O)
			+(-30:3) coordinate (y) node[above] {$y$}
			+(90:3) coordinate (z) node[above] {$z$}
			+(90:1.8) coordinate (B)
			(.2,0) coordinate (M)
			($(O)!1.3!(M)$) coordinate (x) node[above] {$x$}
			($(O)!-1.5!(y)$) coordinate (y')
			($(O)!-1!(y)$) coordinate (A)
			($(A)!1.5!(B)$) coordinate (v) node[above] {$\overrightarrow{v}$}
			($(A)!1.9!(B)$) coordinate (v')
			;
			%			\draw[gray,xstep = 1, ystep = 1] (0,0) grid (5,5);
			
			\draw[-stealth,dashed] (O)--(z);
			\draw[-stealth,dashed] (O)--(y);
			\draw[-stealth,dashed] (O)--(x);
			\draw[dashed] (O)--(y')
			(A)--(v')
			;
			\draw (0.4,.55)--(B) node[midway,left] {$R$};
			\draw[-stealth] (B)--(v);
			\draw[-stealth] (.5,1) to[bend left = 30] (1.5,0.1) node[right] {$\theta$};
			
			\foreach \x/\g in {B/-45,M/-90,O/-90,A/-90}\fill (\x) circle (1.5pt)+(\g:3mm) node{$\x$};
			
		\end{tikzpicture}
		
	\end{center}
	\shortans{2630}
	\loigiai{
		\begin{center}
			\tikzset{rada/.pic={
					\definecolor{c191716}{RGB}{25,23,22}
					
					\begin{scope}[line cap=round,line join=round]
						\path[fill=white,nonzero rule] (0, 13.21) -- (10.42, 13.21) -- (10.42, 0.02) -- (0, 0.02) --cycle
						(0, 13.21);
						
						\path[fill=white,nonzero rule] (0, 13.21) -- (10.42, 13.21) -- (10.42, 0.02) -- (0, 0.02) --cycle
						(0, 13.21);
						
						\path[fill=c191716,nonzero rule] (0.03, 1.5) .. controls (0, 1.74) and (0.05, 1.99) ..
						(0.18, 2.21) .. controls (0.18, 2.21) and (0.18, 2.21) ..
						(0.18, 2.22) .. controls (0.2, 2.23) and (0.22, 2.23) ..
						(0.24, 2.22) -- (1.73, 0.74) -- (1.73, 0.74) .. controls (1.73, 0.74) and (1.73, 0.73) ..
						(1.73, 0.73) .. controls (1.75, 0.71) and (1.74, 0.69) ..
						(1.72, 0.67) .. controls (1.58, 0.59) and (1.42, 0.54) ..
						(1.25, 0.53) .. controls (1.09, 0.51) and (0.92, 0.53) ..
						(0.77, 0.59) .. controls (0.69, 0.61) and (0.62, 0.65) ..
						(0.55, 0.69) .. controls (0.48, 0.74) and (0.41, 0.79) ..
						(0.35, 0.85) .. controls (0.17, 1.03) and (0.06, 1.26) ..
						(0.03, 1.5);
						
						\path[fill=c191716,even odd rule] (0.39, 0.71) -- (0.06, 0.03) -- (1.2, 0.03) -- (1.2, 0.03) -- (1.22, 0.03) -- (1.04, 0.44) .. controls (0.94, 0.45) and (0.84, 0.47) ..
						(0.74, 0.51) .. controls (0.66, 0.54) and (0.58, 0.58) ..
						(0.5, 0.62) .. controls (0.46, 0.65) and (0.42, 0.68) ..
						(0.39, 0.71);
						
						\path[fill=c191716,even odd rule] (1.21, 1.76) .. controls (1.24, 1.79) and (1.28, 1.79) ..
						(1.31, 1.76) .. controls (1.34, 1.73) and (1.34, 1.69) ..
						(1.31, 1.66) -- (1.03, 1.38) .. controls (1, 1.35) and (0.95, 1.35) ..
						(0.93, 1.38) .. controls (0.9, 1.41) and (0.9, 1.45) ..
						(0.93, 1.48) --
						(1.21, 1.76);
						
						\path[fill=c191716,nonzero rule] (1.33, 1.94) .. controls (1.37, 1.94) and (1.41, 1.92) ..
						(1.45, 1.89) .. controls (1.48, 1.86) and (1.49, 1.82) ..
						(1.49, 1.78) .. controls (1.49, 1.74) and (1.48, 1.7) ..
						(1.45, 1.67) .. controls (1.41, 1.64) and (1.37, 1.62) ..
						(1.33, 1.62) .. controls (1.29, 1.62) and (1.25, 1.64) ..
						(1.22, 1.67) .. controls (1.19, 1.7) and (1.17, 1.74) ..
						(1.17, 1.78) .. controls (1.17, 1.82) and (1.19, 1.86) ..
						(1.22, 1.89) .. controls (1.25, 1.92) and (1.29, 1.94) ..
						(1.33, 1.94);
						
						\path[fill=c191716,even odd rule] (1.49, 2.47) .. controls (1.56, 2.49) and (1.65, 2.49) ..
						(1.72, 2.47) .. controls (1.8, 2.45) and (1.87, 2.42) ..
						(1.93, 2.36) .. controls (1.99, 2.3) and (2.03, 2.23) ..
						(2.05, 2.15) .. controls (2.06, 2.08) and (2.06, 1.99) ..
						(2.04, 1.92) .. controls (2.03, 1.88) and (2.06, 1.84) ..
						(2.1, 1.83) .. controls (2.13, 1.82) and (2.17, 1.85) ..
						(2.18, 1.88) .. controls (2.2, 1.98) and (2.21, 2.09) ..
						(2.18, 2.19) .. controls (2.16, 2.29) and (2.11, 2.38) ..
						(2.03, 2.46) .. controls (1.96, 2.53) and (1.86, 2.58) ..
						(1.76, 2.61) .. controls (1.66, 2.63) and (1.55, 2.63) ..
						(1.45, 2.61) .. controls (1.42, 2.6) and (1.39, 2.56) ..
						(1.4, 2.52) .. controls (1.41, 2.48) and (1.45, 2.46) ..
						(1.49, 2.47) --cycle
						(1.43, 2.26) .. controls (1.49, 2.28) and (1.55, 2.28) ..
						(1.6, 2.26) .. controls (1.66, 2.25) and (1.71, 2.22) ..
						(1.75, 2.18) .. controls (1.8, 2.14) and (1.82, 2.09) ..
						(1.84, 2.03) .. controls (1.85, 1.98) and (1.85, 1.92) ..
						(1.83, 1.86) .. controls (1.83, 1.82) and (1.85, 1.78) ..
						(1.89, 1.78) .. controls (1.93, 1.77) and (1.96, 1.79) ..
						(1.97, 1.83) .. controls (1.99, 1.91) and (1.99, 1.99) ..
						(1.97, 2.07) .. controls (1.96, 2.15) and (1.91, 2.22) ..
						(1.85, 2.28) .. controls (1.79, 2.34) and (1.72, 2.38) ..
						(1.64, 2.4) .. controls (1.56, 2.42) and (1.48, 2.42) ..
						(1.4, 2.4) .. controls (1.36, 2.39) and (1.34, 2.35) ..
						(1.34, 2.32) .. controls (1.35, 2.28) and (1.39, 2.25) ..
						(1.43, 2.26) --cycle
						(1.47, 2) .. controls (1.43, 2) and (1.39, 2.02) ..
						(1.38, 2.06) .. controls (1.38, 2.09) and (1.4, 2.13) ..
						(1.44, 2.14) .. controls (1.47, 2.15) and (1.51, 2.15) ..
						(1.55, 2.14) .. controls (1.59, 2.13) and (1.63, 2.11) ..
						(1.66, 2.08) .. controls (1.68, 2.06) and (1.7, 2.02) ..
						(1.71, 1.98) .. controls (1.72, 1.94) and (1.72, 1.9) ..
						(1.71, 1.87) .. controls (1.7, 1.83) and (1.67, 1.81) ..
						(1.63, 1.82) .. controls (1.59, 1.82) and (1.57, 1.86) ..
						(1.57, 1.9) .. controls (1.58, 1.92) and (1.58, 1.93) ..
						(1.58, 1.95) .. controls (1.57, 1.96) and (1.57, 1.97) ..
						(1.56, 1.98) .. controls (1.54, 1.99) and (1.53, 2) ..
						(1.52, 2.01) .. controls (1.5, 2.01) and (1.49, 2.01) ..
						(1.47, 2);
						
						\path[fill=c191716,even odd rule] (10.37, 13.11) .. controls (10.4, 13.05) and (10.41, 12.96) ..
						(10, 12.72) .. controls (9.98, 12.72) and (9.97, 12.71) ..
						(9.96, 12.7) -- (9.77, 11.86) -- (9.63, 11.78) -- (9.62, 12.52) .. controls (9.39, 12.41) and (9.29, 12.38) ..
						(9.15, 12.33) -- (9.09, 11.97) -- (8.96, 11.9) -- (8.93, 12.27) -- (8.63, 12.48) -- (8.75, 12.56) -- (9.08, 12.43) .. controls (9.2, 12.53) and (9.27, 12.61) ..
						(9.48, 12.74) -- (8.86, 13.12) -- (9, 13.2) -- (9.81, 12.95) .. controls (9.83, 12.96) and (9.84, 12.96) ..
						(9.85, 12.97) .. controls (10.27, 13.21) and (10.34, 13.16) ..
						(10.37, 13.11) --cycle
						(10.37, 13.11);
						
					\end{scope}
					
					
			}}
			\begin{tikzpicture}[scale=.7,transform shape]
				\path (0,0) pic[scale=.3]{rada};
				\path (2.9,2) coordinate (O)
				+(-30:3) coordinate (y) node[above] {$y$}
				+(90:3) coordinate (z) node[above] {$z$}
				+(90:1.8) coordinate (B)
				(.2,0) coordinate (M)
				($(O)!1.3!(M)$) coordinate (x) node[above] {$x$}
				($(O)!-1.5!(y)$) coordinate (y')
				($(O)!-1!(y)$) coordinate (A)
				($(A)!1.5!(B)$) coordinate (v) node[above] {$\overrightarrow{v}$}
				($(A)!1.9!(B)$) coordinate (v') node[above] {$C$}
				;
				%			\draw[gray,xstep = 1, ystep = 1] (0,0) grid (5,5);
				
				\draw[-stealth,dashed] (O)--(z);
				\draw[-stealth,dashed] (O)--(y);
				\draw[-stealth,dashed] (O)--(x);
				\draw[dashed] (O)--(y')
				(A)--(v')
				(M)--(v')
				;
				\draw (0.4,.55)--(B) node[midway,left] {$R$};
				\draw[-stealth] (B)--(v);
				%			\draw[-stealth] (.5,1) to[bend left = 30] (1.5,0.1) node[right] {$\theta$};
				
				\foreach \x/\g in {B/-45,M/-90,O/-90,A/-90}\fill (\x) circle (1.5pt)+(\g:3mm) node{$\x$}
				;
				\fill (v') circle(1.5pt);
				\draw[thin,fill=gray] pic[angle radius = 40pt, "$60^\circ$", angle eccentricity = 1.3]{ angle = v'--M--B};
			\end{tikzpicture}
		\end{center}
		Ta có điểm $A(0;-0{,}6;0)$; $B(0;0;0{,}3)$, $M(0{,}4;0;0)$.\\
		Vậy $\overrightarrow{AB}=(0;0{,}6;0{,}3)$ nên véctơ $\overrightarrow{u}=(0;2;1)$.\\
		Do đó phương trình đường thẳng $AB$ là $\heva{&x=0\\&y=2t\\&z=0{,}3t.}$\\
		Điểm $C$ thuộc đường thẳng $AB$ nên có tọa độ là $C(0;2t;0{,}3t)$.
		Vì $C$ ứng với góc quay $60^\circ$ của radar nên $\widehat{BMC}=60^\circ$.\\
		$\overrightarrow{MB}=(-0{,}4;0;0{,}3)$; $\overrightarrow{MC}=(-0{,}4;2t;0{,}3+t)$.\\
		Ta có $$\cos \widehat{BMC}=\dfrac{\overrightarrow{MB}\cdot \overrightarrow{MC}}{\left|\overrightarrow{MB}\right|\cdot \left|\overrightarrow{MC}\right|}=\dfrac{0{,}25+0{,}3t}{0{,}5\cdot \sqrt{5t^2+0{,}6t+0{,}25}}=\dfrac{1}{2}.\quad (*)$$
		Rút gọn $(*)$ ta thu được phương trình bậc hai
		$$
		3{,}56t^2-1{,}8t-0{,}75=0.
		$$
		Phương trình trên có hai nghiệm là $t\approx -0{,}27120$ (loại) và $t\approx 0{,}776819$.\\
		Vậy $a+b+c=0+2t+0{,}3+t=0{,}3+3t=2{,630457}$ (km) $\approx 2630$ (m).
	}
\end{ex}
\Closesolutionfile{ans}
% % \setcounter{section}{0}
\section*{TỔNG HỢP BÀI TẬP}
% \subsection{Kiến thức trọng tâm}


% \subsection{Ví dụ minh họa}
% %\hideansEX{vd}	% ví dụ luôn ẩn lời giải
% %\dotlinefull{vd}	%ví dụ luôn ẩn lời giải và hiện dòng kẻ



% \subsection{Bài tập tự luyện}
% \hideansEX{ex}	%bài tập luôn ẩn lời giải
%\dotlinefull{ex}	%bài tập luôn ẩn lời giải và hiện dòng kẻ

% \Opensolutionfile{ans}[ans/ansBTchoice]

% \subsubsection{Trả lời các câu hỏi sau, mỗi câu hỏi chỉ chọn một phương án}
% \paragraph{Mức độ N}
% \paragraph{Mức độ H}
% \paragraph{Mức độ V}
% \paragraph{Mức độ C}

% \Closesolutionfile{ans}
\setcounter{ex}{0}
\Opensolutionfile{ans}[ans/ansBTchoiceTF]

\subsection{Trả lời các câu hỏi sau, trong mỗi ý a), b), c), d), \ldots ở mỗi câu, thí sinh chọn đúng hoặc sai}
% \paragraph{Mức độ N}
\begin{ex}%[50 Đề minh họa tốt nghiệp 2025 - Đề 13]%[Lê Hữu Kiệt - Lê Quân]%[2D4N1-2]
Biết $F(x)$ là một nguyên hàm của hàm số $f(x)=\dfrac{x^2+1}{x}$ trên khoảng $(0;+\infty)$.
\choiceTF
{\True $f(x)=x+\dfrac{1}{x}$}
{$F(x)=f'(x), \forall x \in (0;+\infty)$}
{$F(x)=\dfrac{1}{2}x^2-\dfrac{1}{x^2}+C$, với $C$ là hằng số}
{Biết rằng đồ thị của hàm số $F(x)$ đi qua $M\left(\mathrm{e};\dfrac{\mathrm{e}^2}{2}\right)$. Khi đó $F(1)=\dfrac{1}{2}$}
\loigiai{
\begin{itemchoice}
\itemch Trên khoảng $(0;+\infty)$, ta có $f(x)=\dfrac{x^2+1}{x}=x+\dfrac{1}{x}$.
\itemch Ta có $F(x)=\displaystyle\int f(x)\mathrm{\,d}x$, $\forall x\in(0;+\infty)$.
\itemch Với $x\in(0;+\infty)$, ta có $\left(\dfrac{1}{2}x^2-\dfrac{1}{x^2}+C\right)'=x+\dfrac{2}{x^3}=\dfrac{x^4+2}{x^3}\ne f(x)$.
\itemch Trên khoảng $(0;+\infty)$, ta có $\displaystyle\int f(x)\mathrm{\,d}x = \displaystyle\int \left(x+\dfrac{1}{x}\right) \mathrm{\,d}x = \dfrac{1}{2}x^2+\ln x + C$, với $C$ là hằng số.\\
Do $M\left(\mathrm{e};\dfrac{\mathrm{e}^2}{2}\right)$ thuộc đồ thị hàm số $F(x)$, suy ra $\dfrac{\mathrm{e}^2}{2} = \dfrac{1}{2}\cdot \mathrm{e}^2 + \ln \mathrm{e} + C \Leftrightarrow C =-1$.\\
Suy ra $F(x)=\dfrac{1}{2}x^2+\ln x - 1$.\\
Khi đó $F(1)=-\dfrac{1}{2}.$
\end{itemchoice}
}
\end{ex}

% \paragraph{Mức độ H}
\begin{ex}%[1D6H3-3]
Cho hàm số $f(x)=\log_2\left(x^2-4x+5\right)$ có đồ thị là $(C)$ và điểm cực trị của đồ thị là $M$.
\choiceTF
{\True Tập xác định của hàm số đã cho là $\mathscr{D}=\mathbb{R}$}
{Đạo hàm của hàm số đã cho là $f'(x)=\dfrac{2x-4}{x^2-4x+5}$}
{\True Tọa độ của điểm $M$ là $(2;0)$}
{\True Đường thẳng $y=1$ cắt đồ thị $(C)$ tại hai điểm phân biệt $A$, $B$ thì tam giác $MAB$ có diện tích bằng $1$}
\loigiai{
\begin{itemchoice}
\itemch Điều kiện xác định: $x^2-4x+5>0 \Leftrightarrow (x-2)^2+1>0$. Luôn đúng.\\
Vậy tập xác định của hàm số đã cho là $\mathscr{D}=\mathbb{R}$.
\itemch
Ta có \[f(x)=\log _2\left(x^2-4 x+5\right)  \Rightarrow f(x)=\dfrac{2 x-4}{\left(x^2-4 x+5\right) \ln 2} .\]
\itemch Ta có \begin{eqnarray*}
f(x)=0
& \Leftrightarrow& \dfrac{2 x-4}{\left(x^2-4 x+5\right) \ln 2}=0 \\
& \Leftrightarrow& 2 x-4=0 \Leftrightarrow x=2.
\end{eqnarray*}
Khi đó $f(2)=\log \left(2^2-4.2+5\right)=\log 1=0$. Vậy $M(2;0)$.
\itemch Phương trình hoành độ giao điểm là
\begin{eqnarray*}
&&\log _2\left(x^2-4 x+5\right)=1 \\
&\Leftrightarrow& x^2-4 x+5=2 \\
&\Leftrightarrow& x^2-4 x+3=0 \\
&\Leftrightarrow&\hoac{
&x=1 \Rightarrow A(1 ; 1) \\
&x=3 \Rightarrow B(3 ; 1) .
}
\end{eqnarray*}
Diện tích tam giác $MAB$ là
\begin{eqnarray*}
S_{M A B}&=&\dfrac{1}{2}\left|\left(x_M-x_A\right)\left(y_B-y_A\right)-\left(x_B-x_A\right)\left(y_M-y_A\right)\right| \\
&=&\dfrac{1}{2}|(2-1)(1-1)-(3-1)(0-1)|=1 .
\end{eqnarray*}
\end{itemchoice}
}
\end{ex}

\begin{ex}%[2H5H3-4]%[TEX ĐỀ MOON 2025]%[Nguyễn Cường]
Một máy bay di chuyển từ sân bay $A$ với tọa độ $A(0;0;0)$ đến sân bay $B$ tại tọa độ $B(760;120;10)$ (đơn vị tính là km). Trên hành trình, máy bay sẽ đi qua vùng kiểm soát không lưu trung gian có bán kính $100$ km, với tâm trạm kiểm soát đặt tại tọa độ $O(380;60;0)$. Máy bay bay với vận tốc không đổi, hoàn thành quãng đường trong $1$ giờ $25$ phút.
\choiceTF
{\True Phương trình tham số của đường bay từ $A$ đến $B$ được cho bởi $\heva{& x=760t \\ & y=120t\\ & z=10t}$, $t\in[0;1{,}42]$ ($t$ được tính bằng giờ)}
{\True Máy bay đi vào phạm vi kiểm soát không lưu (bán kính $100$ km, tâm tại $O(380;60;0)$) tại thời điểm $t=0{,}5$}
{Quãng đường từ $A$ đến $B$ theo đường bay là $766$ km (làm tròn đến hàng đơn vị)}
{Nếu máy bay bay trong vùng kiểm soát trong $15$ phút, nó sẽ bay đúng $\dfrac{1}{6}$ quãng đường từ lúc vào đến khi ra khỏi vùng này}
\loigiai{
\begin{itemchoice}
\itemch Đổi đơn vị $1$ giờ $25$ phút tương ứng với $1{,}41\overline{6}$ giờ.\\
Ta có $\overrightarrow{AB}=(760;120;10)$ là véc-tơ chỉ phương của phương trình đường bay từ $A$ đến $B$.\\
Phương trình tham số của đường bay này là $\heva{& x=760t \\ & y=120t\\ & z=10t}$, $t\in[0;1{,}42]$ ($t$ được tính bằng giờ).
\itemch Phương trình mặt cầu mô tả vùng kiểm soát không lưu trung gian là \[(S)\colon (x-380)^2+(y-60)^2+z^2=100^2.\]
Khi $t=0{,}5$ thì máy bay đang ở tọa độ $C(380;60;5)$.\\
Ta có $\overrightarrow{OC}=(0;0;5)$, suy ra $OC=5<R$ nên máy bay đã đi vào phạm vi kiểm soát không lưu.
\itemch Ta có $\overrightarrow{AB}=(760;120;10)$ nên $AB=\sqrt{760^2+120^2+10^2}\approx 770$\,(km).
\itemch Máy bay bay $15$ phút thì chỉ hoàn thành $\dfrac{25}{85}=\dfrac{5}{17}$ quãng đường từ lúc cất cánh đến khi hoàn thành chuyến bay.
\end{itemchoice}
}
\end{ex}

\begin{ex}%[2H5H3-3]%[TEX ĐỀ MOON 2025]%[Lê Hữu Kiệt]
Trong không gian với hệ tọa độ $Oxyz$, cho mặt phẳng $(P)\colon x-2y-2z-1=0$ và hai điểm $A(1;1;2)$, $B(3;2;-3)$.
\def\dotEX{}
\choiceTF
{\True Điểm $A$ không thuộc mặt phẳng $(P)$.}
{Khoảng cách từ điểm $B$ đến mặt phẳng $(P)$ bằng $3$.}
{Phương trình tham số của đường thẳng $AB$ là $\heva{& x=1+3t \\ & y=1+2t\\ & z=2-3t.}$}
{Mặt cầu $(S)$ có tâm $I$ thuộc trục $Oz$ và đi qua hai điểm $A$, $B$ có phương trình là \begin{center}
$x^2+y^2+z^2-8z+2=0$.
\end{center}}
\loigiai{
\begin{itemchoice}
\itemch Thay tọa độ điểm $A$ vào phương trình mặt phẳng $(P)$ ta được $1-2\cdot1-2\cdot2-1=-6\ne0$ nên $A\not\in(P)$.
\itemch Khoảng cách từ điểm $B$ đến mặt phẳng $(P)$ là
\[\mathrm{d}\left(B,(P)\right)=\dfrac{|3-2\cdot2-2\cdot(-3)-1|}{\sqrt{1^2+(-2)^2+(-2)^2}}=\dfrac{4}{3}.\]
\itemch Ta có $\overrightarrow{AB}=(2;1;-5)$.\\
Phương trình tham số của đường thẳng $AB$ đi qua điểm $A$, nhận $\overrightarrow{AB}$ là vectơ chỉ phương là
\[\heva{&x=1+2t\\&y=1+t\\&z=2-5t.}\]
\itemch Gọi $I(0;0;x_I)$ là tâm của mặt cầu $(S)$.\\
Ta có $IA=\sqrt{1^2+1^2+(2-z_I)^2}=\sqrt{6-4z_I+z_I^2}$,\\$IB=\sqrt{3^2+2^2+(-3-z_I)^2}=\sqrt{22+6z_I+z_I^2}$.\\
Do $A$, $B$ thuộc mặt cầu $(S)$ nên
\allowdisplaybreaks
\begin{eqnarray*}
&&IA=IB\\
&\Leftrightarrow&IA^2=IB^2 \\
&\Leftrightarrow&6-4z_I+z_I^2=22+6z_I+z_I^2\\
&\Leftrightarrow&z_I=-\dfrac{8}{5}
\end{eqnarray*}
Suy ra $I\left(0;0;-\dfrac{8}{5}\right)$, $IA=\dfrac{\sqrt{374}}{5}$.\\
Khi đó phương trình mặt cầu $(S)$ là
\begin{eqnarray*}
&& x^2+y^2+\left(z+\dfrac{8}{5}\right)^2=\dfrac{374}{25} \\
&\Leftrightarrow& x^2+y^2+z^2+\dfrac{16}{5}z-\dfrac{62}{5}=0.
\end{eqnarray*}
\end{itemchoice}
}
\end{ex}

\begin{ex}%[2H5H2-3]
Trong không gian $Oxyz$, gọi $d$ là giao tuyến của mặt phẳng $(P)\colon 2x-y-2z-3=0$ và mặt phẳng $(Q)\colon x-2y-z-6=0$. Xét tính đúng sai của các mệnh đề sau
\choiceTF
{Một véc-tơ pháp tuyến của mặt phẳng $(P)$ là $(2;-1;-3)$}
{\True Đường thẳng $d$ có vectơ chỉ phương là $\overrightarrow{u}_d=(2;0;2)$}
{\True Điểm $A(0;-3;0)$ thuộc đường thẳng $d$}
{\True Phương trình tham số của đường thẳng $d$ là $\heva{& x=1+t \\ & y=-3\\ & z=1+t}$}
\loigiai{
\begin{itemchoice}
\itemch
Một véc-tơ pháp tuyến của mặt phẳng $(P)$ là $(2;-1;-2)$.
\itemch
Vì $d$ là giao tuyến của mặt phẳng $(P)$ và $(Q)$ nên $d$ có hai véc-tơ pháp tuyến là\\ $\overrightarrow{n}_{(P)}=(2;-1;-2)$; $\overrightarrow{n}_{(Q)}=(1;-2;-1)$.\\
Suy ra $\overrightarrow{u}_{d}=\left[\overrightarrow{n}_{(P)};\overrightarrow{n}_{(Q)}\right]=(-3;0;-3)=-3(1;0;1)$.\\
Vậy đường thẳng $d$ có một véc-tơ chỉ phương là $(2;0;2)$.
\itemch Nếu $A\in d \Rightarrow \heva{&A\in(P)\\&A\in (Q).}$\\
Cho $z=0$ ta xét hệ phương trình $\heva{&2x-y=3\\&x-2y=6}\Leftrightarrow \heva{&x=0\\&y=-3.}$\\
Vậy $A(0;-3;0)\in d$.
\itemch Đường thẳng  $d$ đi qua điểm $A(0;-3;0)$ và có véc-tơ chỉ phương $\overrightarrow{u}_{d}=(1;0;1)$ có phương trình tham số là $\heva{&x=t\\&y=-3\\&z=t.}$\\
Khi đó đường thẳng $d$ đi qua điểm $(1;-3;1)$ và có véc-tơ chỉ phương $\overrightarrow{u}_{d}=(1;0;1)$ có phương trình tham số là $\heva{&x=1+t\\&y=-3\\&z=1+t.}$
\end{itemchoice}
}
\end{ex}

\begin{ex}%[2H5H1-3]
Trong không gian với hệ tọa độ $Oxyz$ (đơn vị trên mỗi trục tọa độ là km), một máy bay đang ở vị trí $A(3;-2{,}5;0{,}5)$ và sẽ hạ cánh ở vị trí $B(3;8{,}5;0)$ trên đường băng (hình minh họa bên dưới). Có một lớp mây được mô phỏng bởi một mặt phẳng $(\alpha)$ đi qua ba điểm $M(9;0;0)$, $N(0;-9;0)$ và $P(0;0;0{,}9)$.

{\centering \begin{tikzpicture}[scale=0.9,font=\footnotesize,>=stealth, line join=round, line cap=round]
\def\xmin{-2.5}
\def\ymin{-3.5} \def\ymax{1.5}
\def\zmax{2}
\coordinate (O) at (0,0);
\coordinate (M) at (-1.5,-1.5);
\coordinate (N) at (-2.5,0);
\coordinate (P) at (0,1);
\coordinate (A) at (-1.7,1.4);
\coordinate (B) at (1,-0.5);
\coordinate (C) at ($(A)!0.44!(B)$);
\coordinate (X) at (intersection of A--B and O--P);
\draw[->] (0,0)--(\xmin+0.3,\xmin+0.3)node [below]{$x$};
\draw[->] (0,0)--(\ymax,0) node [above]{$y$};
\draw[dashed] (0,0)--(\ymin,0);
\draw[->] (0,0)--(0,\zmax) node [left]{$z$};
\node at (O) [below right,xshift=-0.1cm]{$O$};
\draw (M)node[below right]{$M$}--(N)node[above]{$N$}--(P)node[right]{$P$}--cycle;
\draw (A)--(C) (X)--(B);
\draw[dashed] (C)--(X);
\fill (A)node[above]{$A$} circle(2pt);
\fill (B)node[below right]{$B$} circle(2pt);
\fill (C)node[below left]{$C$} circle(2pt);
\end{tikzpicture}\par}\noindent

\choiceTF
{\True Khoảng cách hai điểm $A$ và $B$ bằng $11$ km (làm tròn kết quả đến hàng đơn vị)}
{\True Biết tốc độ của máy bay là $250$ km/h trên quãng đường $AB$ thì sau $2{,}64$ phút (làm tròn đến hàng phần trăm) máy bay từ vị trí $A$ hạ cánh tại vị trí $B$}
{Phương trình mặt phẳng $(\alpha)$ là $x-y-10z-9=0$}
{Độ cao của máy bay khi xuyên qua đám mây để hạ cánh là $0{,}35$ km}
\loigiai{
\begin{itemchoice}
\itemch Ta có $A B=\sqrt{0^2+11^2+(-0{,}5)^2}=\sqrt{121{,}25} \approx 11 \mathrm{~km}$.
\itemch Thời gian để máy bay từ vị trí $A$ hạ cánh tại vị trí $B$ là\\
$\dfrac{\sqrt{121{,}25}}{250}(\text{h})=\dfrac{\sqrt{121{,}25}}{250} \cdot 60~ (\text{phút}) \approx 2{,}64$ (phút).
\itemch Giả sử điểm $C\left(x_C ; y_C ; z_C\right)$ là vị trí mà máy  bay xuyên qua đám mây để hạ cánh, suy ra $C \in(\alpha)$. Mặt phẳng $(\alpha)$ có phương trình
\begin{eqnarray*}
&&\dfrac{x}{9}-\dfrac{y}{9}+\dfrac{z}{0{,}9}=1 \\
&\Leftrightarrow&x-y+10 z-9=0.
\end{eqnarray*}
\itemch Do hai vectơ $\overrightarrow{AC}$, $\overrightarrow{AB}$ cùng hướng nên tồn tại số thực $t>0$ sao cho
\begin{eqnarray*}
\overrightarrow{AC}=t \overrightarrow{AB}&\Leftrightarrow&\heva{
&x_C-3=0 \\
&y_C+2{,}5=11 t \\
&z_C-0{,}5=-0{,}5 t
} \\
&\Leftrightarrow&\heva{
&x_C=3 \\
&y_C=-2{,}5+11 t \\
&z_C=0{,}5-0{,}5 t.
}
\end{eqnarray*}
Vì $C \in(\alpha)$ nên $3-(-2{,}5+10 t)+10(0{,}5-0{,}5 t)-9=0 \Leftrightarrow t=0{,}1$.\\
Suy ra $C(3 ;-1{,}4 ; 0{,}45)$.\\
Vậy độ cao của máy bay khi xuyên qua đám mây là $0{,}45$ km
\end{itemchoice}
}
\end{ex}

\begin{ex}%[50 Đề minh họa tốt nghiệp 2025 - Đề 13]%[Lê Hữu Kiệt - Lê Quân]%[2D6H2-4]
Một thùng chứa $100$ quả táo trong đó có $80\%$ số quả táo được dán nhãn, số còn lại không được dán nhãn. Bạn Hoàng lấy ra một quả trong thùng, sau đó bạn Hà lấy ra một quả thứ hai.
\begin{itemize}
\item Gọi $A$ là biến cố: \lq\lq Quả táo bạn Hoàng lấy ra có dán nhãn\rq\rq.
\item Gọi $B$ là biến cố: \lq\lq Quả táo bạn Hà lấy ra có dán nhãn\rq\rq.
\end{itemize}
\choiceTF
{\True $\mathrm{P}(A)=\dfrac{4}{5}$}
{Xác suất có điều kiện $\mathrm{P}(B\mid A)=\dfrac{79}{100}$}
{\True Xác xuất bạn Hà lấy ra quả táo có dán nhãn bằng $0{,}8$}
{Biết rằng bạn Hà lấy ra quả táo có dán nhãn. Xác suất để Hoàng cũng lấy ra quả táo có dán nhãn là $20{,}2\%$ (làm tròn kết quả đến hàng phần mười theo đơn vị phần trăm)}
\loigiai{
Số quả táo được dán nhãn là $80\%\cdot 100=80$ quả; số táo không được dán nhãn là $20$ quả.\\
Khi đó $\mathrm{P}(A)=\dfrac{80}{100}=\dfrac{4}{5}$; $\mathrm{P}(\overline{A})=1-\dfrac{4}{5}=\dfrac{1}{5}$; $\mathrm{P}(B\mid A)=\dfrac{79}{99}$; $\mathrm{P}(B\mid\overline{A})=\dfrac{80}{99}$.
\begin{itemchoice}
\itemch Ta có $n(A)=80$, suy ra $\mathrm{P}(A)=\dfrac{80}{100}=\dfrac{4}{5}$.
\itemch Ta có $\mathrm{P}(B\mid A)=\dfrac{79}{99}$.
\itemch Áp dụng công thức xác suất toàn phần, ta có
\[ \mathrm{P}(B)=\mathrm{P}(A)\cdot \mathrm{P}(B\mid A) + \mathrm{P}(\overline{A})\cdot \mathrm{P}(B\mid\overline{A})=\dfrac{4}{5}\cdot\dfrac{79}{99}+\dfrac{1}{5}\cdot\dfrac{80}{99}=0{,}8.\]
\itemch Xác suất để Hoàng cũng lấy ra quả táo có dán nhãn biết bạn Hà lấy ra quả táo có dán nhãn là $\mathrm{P}(A\mid B)$. Áp dụng công thức Bayes ta có
\[ \mathrm{P}(A\mid B) =\dfrac{\mathrm{P}(A)\cdot \mathrm{P}(B\mid A)}{\mathrm{P}(B)} = \dfrac{\dfrac{4}{5}\cdot\dfrac{79}{99}}{0{,}8} \approx 79,{80}\%.\]
\end{itemchoice}
}
\end{ex}

\begin{ex}%[2D6H2-4]%[TEX Đề Moon 2025]%[Võ Nguyên Thạch]
Một xưởng máy sử dụng một loại linh kiện được sản xuất từ hai cơ sở I và II. Số linh kiện do cơ sở I sản xuất chiếm $61\%$, số linh kiện do cơ sở II sản xuất chiếm $39\%$. Tỉ lệ linh kiện đạt tiêu chuẩn của cơ sở I, cơ sở II lần lượt là $93\%$, $82\%$. Kiểm tra ngẫu nhiên một linh kiện ở xưởng máy. Xét các biến cố
\begin{itemize}
\item $A_1\colon$\lq\lq Linh kiện được kiểm tra do cơ sở I sản xuất\rq\rq.
\item $A_2\colon$\lq\lq Linh kiện được kiểm tra do cơ sở II sản xuất\rq\rq.
\item $B\colon$\lq\lq Linh kiện được kiểm tra đạt tiêu chuẩn\rq\rq.
\end{itemize}
Xét tính đúng sai của các mệnh đề sau
\choiceTF
{\True Xác suất $P(A_1)=0{,}61$}
{\True Xác suất có điều kiện $P(B\mid A_2)=0{,}82$}
{\True Xác suất $P(B)=0{,}8871$}
{Xác suất có điều kiện $P(A_1\mid B)=0{,}55$}
\loigiai{
\begin{itemchoice}
\itemch Do số linh kiện do cơ sở I sản xuất chiếm $61\%$ nên $P(A_1)=0{,}61$.
\itemch Do tỉ lệ linh kiện đạt tiêu chuẩn của cơ sở II là $82\%$ nên $P(B\mid A_2)=0{,}82$.
\itemch Áp dụng định lý xác suất toàn phần ta có
\allowdisplaybreaks
\begin{eqnarray*}
\mathrm{P}(B)&=&\mathrm{P}(A_1)\cdot \mathrm{P}(B|A_1)+\mathrm{P}(A_2)\cdot \mathrm{P}(B|A_2)\mathrm{P}(B)\\
&=&0{,}61\cdot 0{,}93+0{,}39\cdot 0{,}82\\
&=&0{,}5673+0{,}3198\\
&=&0,8871.
\end{eqnarray*}
\itemch Áp dụng công thức Bayes ta có
\allowdisplaybreaks
\begin{eqnarray*}
\mathrm{P}(A_1|B)&=&\dfrac{\mathrm{P}(A_1)\cdot \mathrm{P}(B|A_1)}{\mathrm{P}(B)}\\
&=&\dfrac{0{,}61\cdot 0{,}93}{0{,}8871}\\
&=&\dfrac{0{,}5673}{0{,}8871}\approx 0{,}6395.
\end{eqnarray*}
\end{itemchoice}
}
\end{ex}

\begin{ex}%[2D6H1-2]
Cho hai biến cố $A$ và $B$, biết $\mathrm{P}(A)=0{,}6$, $\mathrm{P}\left(\overline{B}\right)=0{,}2$, $\mathrm{P}(AB)=0{,}42$.
\choiceTF
{\True $\mathrm{P}(B)=0{,}8$}
{$A$ và $B$ là hai biến cố độc lập}
{$\mathrm{P}\left(\overline{A}B\right)=0{,}48$}
{\True $\mathrm{P}\left(B\mid \overline{A}\right)=0{,}95$}
\loigiai{
\begin{itemchoice}
\itemch {\bf Đúng.}\\ Ta có $\mathrm{P}(B)=1-\mathrm{P}\left(\overline{B}\right)=0{,}8$.
\itemch  {\bf Sai.}\\ Ta có $\mathrm{P}(A)\cdot \mathrm{P}(B)=0{,}6\cdot 0{,}8 =0{,}48$.\\
Suy ra $\mathrm{P}(AB)\neq \mathrm{P}(A)\cdot \mathrm{P}(B)$ nên hai biến cố $A$ và $B$ không phải là hai biến cố độc lập.
\itemch  {\bf Sai.}\\ Ta có sơ đồ cây như sau
\begin{center}
\begin{tikzpicture}[>=stealth,line cap=round,line join=round]
\path(0,0)node(a){Xác suất }
++(3,2)node(b){$ A $} ++(3,1.5)node(d){$ B $} (b)++(3,-1.5)node(e){$ \overline{B} $}
(a)++(3,-2)node(c){$\overline{A} $}++(3,1.5)node(f){$ B $}(c)++(3,-1.5)node(g){$ \overline{B} $}
;
\draw(a)--(b) (b)--(e) (b)--(d) (a)--(c) --(f) (c)--(g);
\end{tikzpicture}
\end{center}
Do đó
$\mathrm{P}(B)=\mathrm{P}\left(\overline{A}B\right)+ \mathrm{P}\left(AB\right)$.\\
Suy ra $\mathrm{P}\left(\overline{A}B\right)=\mathrm{P}(B)-\mathrm{P}\left(AB\right)=0{,}8-0{,}42 =0{,}38$.
\itemch {\bf Đúng.}\\ Ta có $\mathrm{P}\left(B\mid \overline{A}\right) =\dfrac{\mathrm{P}\left(\overline{A}B\right)}{\mathrm{P}\left(\overline{A}\right)}=\dfrac{0{,}38}{1-0{,}6}=0{,}95$.
\end{itemchoice}

}
\end{ex}

\begin{ex}%[2D6H1-2]%[TEX ĐỀ MOON 2025]%[Nguyễn Văn Hiệp]
Một chiếc hộp có $80$ viên bi, trong đó có $50$ viên bi màu đỏ và $30$ viên bi màu vàng; các viên bi có kích thước và khối lượng như nhau. Sau khi kiểm tra, người ta thấy có $60\%$ số viên bi màu đỏ đánh số và $50\%$ số viên bi màu vàng có đánh số, những viên bi còn lại không đánh số. Xét tính đúng sai của các mệnh đề sau
\choiceTF
{\True Số viên bi màu đỏ có đánh số là $30$}
{\True Số viên bi màu vàng không đánh số là $15$}
{Lấy ra ngẫu nhiên một viên bi trong hộp. Xác suất để viên bi được lấy ra có đánh số là $\dfrac{3}{5}$}
{\True Lấy ra ngẫu nhiên một viên bi trong hộp. Xác suất để viên bi được lấy ra không có đánh số $\dfrac{7}{16}$}
\loigiai
{
\begin{itemchoice}
\itemch $50 \times 60\% = 30$ bi đỏ có đánh số.
\itemch $30 \times 50\% = 15$ bi không đánh số.
\itemch
Đặt $C$ là biến cố \lq\lq Viên bi được lấy ra có đánh số\rq\rq.\\
$n\left(C\right)=30+15= 45$; $n\left(\Omega\right)=80$; $\mathrm{P}(C) = \dfrac{45}{80} = \dfrac{9}{16}$.
\itemch Đúng. Đặt $\overline{C}$ là biến cố \lq\lq Viên bi được lấy ra không có đánh số\rq\rq.\\ $\mathrm{P}\left(\overline{C}\right) =1-\mathrm{P}\left(C\right) =1 - \dfrac{9}{16} = \dfrac{7}{16}$.
\end{itemchoice}
}
\end{ex}

\begin{ex}%[2D4H3-1]
Cho hình phẳng $(H)$ giới hạn bởi các đồ thị hàm số $f(x)=x^2-2x-1$ và $g(x)=2x-4$. Xét hàm số $Q(x)=\dfrac{x^3}{3}-2x^2+3x$.
\choiceTF
{\True Phương trình $f(x)-g(x)=0$ có hai nghiệm phân biệt}
{Hiệu $f(x)-g(x) > 0$ với mọi $x \in(1; 3)$}
{\True Hàm số $Q(x)$ là một nguyên hàm của hàm số $f(x)-g(x)$}
{\True Diện tích hình phẳng $(H)$ bằng $Q(1)-Q(3)$}
\loigiai{
\begin{itemchoice}
\itemch Phương trình $f(x)=g(x)\Leftrightarrow x^2-2x-1=2x-4\Leftrightarrow x^2-4x+3=0\Leftrightarrow \hoac{&x=1\\&x=3}$.
\itemch Ta có bảng xét dấu sau
\begin{center}
\begin{tikzpicture}
\tkzTabInit[nocadre=false,lgt=2.5,espcl=2,deltacl=0.6]
{$x$ /0.6,$f(x)-g(x)$ /1.1}
{$-\infty$,$1$,$3$,$+\infty$}
\tkzTabLine{,+,$0$,-,$0$,+,}
\end{tikzpicture}
\end{center}
Từ bảng xét dấu suy ra $f(x)-(g(x)<0$ với $x\in(1;3)$.
\itemch
Ta có $\displaystyle\int (f(x)-g(x)) \mathrm{\, d}x=\displaystyle\int (x^2-4x+3) \mathrm{\, d}x=\dfrac{x^3}{3}-2x^2+3x+C$.\\
Vậy $Q(x)=\dfrac{x^3}{3}-2x^2+3x$ là một nguyên hàm của $f(x)-g(x)$.
\itemch
Diện tích hình phẳng $(H)$ là $S=\displaystyle\int\limits_1^3 |x^2-4x+3| \mathrm{\, d}x=\displaystyle\int\limits_1^3 -(x^2-4x+3) \mathrm{\, d}x=Q(1)-Q(3)$.
\end{itemchoice}

}
\end{ex}

\begin{ex}%[2D4H2-6]%[TEX ĐỀ MOON 2025]%[Nguyễn Cường]
Để đảm bảo an toàn khi lưu thông trên đường, các xe ô tô khi dừng đèn đỏ phải cách nhau tối thiểu $5$ m. Một ô tô $A$ đang chạy với vận tốc $16$ m/s thì gặp ô tô $B$ đang dừng đèn đỏ nên ô tô $A$ hãm phanh và chuyển động chậm dần đều với vận tốc được biểu thị bởi công thức $v_A(t)=16-4t$ (đơn vị tính bằng m/s, thời gian $t$ tính bằng giây).
\choiceTF
{\True Thời điểm xe ô tô $A$ dừng lại là $4$ s}
{Quãng đường $S(t)$ (đơn vị: mét) mà ô tô $A$ đi được trong thời gian $t$ giây $(0\le t\le 4)$ kể từ khi hãm phanh được tính theo công thức $S(t)=\displaystyle\int\limits_{0}^{4} v(t)\mathrm{\,d}t$}
{\True Từ khi bắt đầu hãm phanh đến khi dừng lại xe ô tô $A$ đi được quãng đường $32$ m}
{\True Khoảng cách an toàn tối thiểu giữa xe ô tô $A$ và ô tô $B$ là $37$ m}
\loigiai{
\begin{itemchoice}
\itemch Khi xe ô tô $A$ dừng lại, tức là $v_A(t)=16-4t=0\Leftrightarrow t=4$ giây.
\itemch Quãng đường $S(t)$ (đơn vị: mét) mà ô tô $A$ đi được trong thời gian $t$ giây $(0\le t\le 4)$ kể từ khi hãm phanh được tính theo công thức $S(t)=\displaystyle\int\limits_{0}^{4} v_A(t)\mathrm{\,d}t$
\itemch Từ khi bắt đầu hãm phanh đến khi dừng lại xe ô tô $A$ đi được quãng đường
\[S(t)=\displaystyle\int\limits_{0}^{4} (16-4t)\mathrm{\,d}t=32~(\mathrm{m}).\]
\itemch Khoảng cách tối thiểu giữa hai xe ô tô $A$ và $B$ là $32+5=37$\,(m).
\end{itemchoice}
}
\end{ex}

\begin{ex}%[2D4H2-6]
Nam đang tham gia một bài học từ mới tiếng Anh trong vòng $60$ phút. Biết rằng $M(t)$ là số từ mới mà Nam có thể ghi nhớ trong $t$ phút. Tốc độ ghi nhớ từ mới của Nam được xác định bởi hàm số $M'(t) = at - bt^2$ (với $a, b \in \mathbb{R}$) (từ/phút) và đạt cao nhất tại thời điểm $40$ phút. Biết rằng Nam ghi nhớ được $18$ từ mới trong $10$ phút đầu tiên của bài học.
\choiceTF
{$a = 0{,}4$}
{Khả năng ghi nhớ của Nam tại thời điểm $20$ phút là $6$ từ/phút}
{Trong cả bài học Nam ghi nhớ được tổng cộng $427$ từ mới}
{Biết rằng tốc độ học trung bình (từ/phút) tại thời điểm $n$ đến thời điểm $m$ được tính bởi công thức
$\dfrac{1}{m-n}\displaystyle\int_n^m M'(t)\mathrm{\,d}t$.
Tốc độ học trung bình của Nam trong cả bài học là $6$ từ/phút}
\loigiai{
\begin{itemchoice}
\itemch Tốc độ ghi nhớ từ mới của Nam đạt cao nhất tại thời điểm $40$ phút nên $\dfrac{a}{2b} = 40$ hay $a = 80b$.\\
Số từ mới mà Nam có thể ghi nhớ trong $t$ phút là
\[M(t) = \displaystyle\int M'(t) \mathrm{\,d}t = \displaystyle\int (at - bt^2) \mathrm{\,d}t = \dfrac{a}{2}t^2 - \dfrac{b}{3}t^3 + C.\]
Mặt khác, $M(0) = 0 \Rightarrow C = 0 \Rightarrow M(t) = \dfrac{a}{2}t^2 - \dfrac{b}{3}t^3$.

Nam có thể ghi nhớ được $18$ từ mới trong $10$ phút đầu tiên của bài học nên $M(10) = 18$ hay
\[\dfrac{a}{2} \cdot 10^2 - \dfrac{b}{3} \cdot 10^3 = 18 \Rightarrow \dfrac{80b}{2} \cdot 10^2 - \dfrac{b}{3} \cdot 10^3 = 18 \Rightarrow \dfrac{11000}{3}b = 18 \Rightarrow b = \dfrac{27}{5500}.\]

Suy ra $a = \dfrac{108}{275} \approx 0,39$.\\
Vậy $M(t) = \dfrac{54}{275}t^2 - \dfrac{9}{5500}t^3$, $M'(t) = \dfrac{108}{275}t - \dfrac{27}{5500}t^2$.
\itemch Khả năng ghi nhớ của Nam tại thời điểm 20 phút là
\[M'(20) = \dfrac{108}{275} \cdot 20 - \dfrac{27}{5500} \cdot 20^2 \approx 5{,}9 \text{ (từ/phút)}.\]
\itemch Trong cả tiết học Nam ghi nhớ được tổng cộng số từ mới là
\[\displaystyle\int_0^{60} \left(\dfrac{108}{275}t - \dfrac{9}{5500}t^2\right) \mathrm{\,d}t \approx 353 \text{ (từ)}.\]
\itemch Tốc độ học trung bình của Nam trong cả tiết học là
\[\dfrac{1}{60-0} \displaystyle\int_0^{60} M'(t) \mathrm{\,d}t = \dfrac{1}{60} \displaystyle\int_0^{60} \left(\dfrac{54}{275}t - \dfrac{9}{5500}t^2\right) \mathrm{\,d}t = \dfrac{324}{55} \approx 5{,}9 \text{ (từ/phút)}.\]

\end{itemchoice}
}
\end{ex}

\begin{ex}%[2D4H2-6]%[TEX ĐỀ MOON 2025]%[Nguyễn Thế Duy]
Một ô tô đang chạy đều với vận tốc $x$ (m/s) thì người lái xe đạp phanh. Từ thời điểm đó, ô tô chuyển động chậm dần đều với vận tốc thay đổi theo hàm số $v=-5t+20$ (m/s), trong đó $t$ là thời gian tính bằng giây kể từ lúc đạp phanh. Xét tính đúng sai của các mệnh đề sau
\choiceTF
{\True Khi xe dừng hẳn thì vận tốc bằng $0$ (m/s)}
{Thời gian từ lúc người lái xe đạp phanh cho đến khi xe dừng hẳn là $5$ giây}
{\True $\displaystyle\int \left(-5t+20\right) \mathrm{\,d}t=-\dfrac{5t^2}{2}+20t+C$}
{Quãng đường từ lúc đạp phanh cho đến khi xe dừng hẳn là $400$ m}
\loigiai{
\begin{itemchoice}
\itemch \textbf{Đúng}.\\
Khi xe dừng hẳn thì vận tốc bằng $0$ (m/s).
\itemch \textbf{Sai}.\\
Xét $v(t) = 0 \Leftrightarrow -5t + 20 = 0 \Leftrightarrow t = 4$.\\
Vậy xe dừng hẳng sau $4$ giây đạp phanh.
\itemch \textbf{Đúng}.\\
Ta có $\displaystyle\int \left(-5t+20\right) \mathrm{\,d}t=-\dfrac{5t^2}{2}+20t+C$.
\itemch \textbf{Sai}.\\
Quãng đường từ lúc đạp phanh đến khi xe dừng hẳng là $s = \displaystyle\int_0^4 \left(-5t+20\right) \mathrm{\,d}t = 40$ m.
\end{itemchoice}
}
\end{ex}

\begin{ex}%[2D4H1-6]
Ở nhiệt độ $37^{\circ}$ C, một phản ứng hóa học từ chất đầu A, chuyển hóa thành chất B theo phương trình A $\longrightarrow$ B. Giả sử $y(x)$ là nồng độ chất $A$ (đơn vị mol $L^{-1}$) tại thời điểm $x$ (giây), $y(x)>0$ với $x \geq 0$, thỏa mãn hệ thức $y'(x)=-7 \cdot 10^{-4} y(x)$ với $x \geq 0$. Biết rằng tại $x=0$, nồng độ (đầu) của A là $0{,}05$ mol $L^{-1}$. Xét hàm số $f(x)=\ln y(x)$ với $x \geq 0$. Các phát biểu sau đây đúng hay sai?
\choiceTF
{\True $f'(x)=-7 \cdot 10^{-4}$}
{\True $f(x)=-7 \cdot 10^{-4} x+\ln (0{,}05)$}
{$y(30)-y(15)=-6\cdot 10^{-4}$}
{\True Nồng độ trung bình của chất A từ thời điểm $15$ giây đến thời điểm $30$ giây gần bằng $0{,}05$}
\loigiai{
\begin{itemchoice}
\itemch {\bf Đúng.}\\
Ta có $f'(x)=\left[\ln y(x)\right]'=\dfrac{y'(x)}{y(x)}=-7\cdot 10^{-4}$.
\itemch {\bf Đúng.}\\
Ta có $f(x)=\displaystyle\int f'(x)\mathrm{\,d}x=\displaystyle\int -7\cdot 10^{-4}\mathrm{\,d}x =-7\cdot 10^{-4}x+C$.\\
Theo giả thiết, $y(0)=0{,}05$ nên $f(0)=\ln y(0)=\ln 0{,}05$.\\
Vì vậy $C=\ln 0{,}05$, suy ra $f(x)=-7\cdot 10^{-4}x+\ln0{,}05$.
\itemch {\bf Sai.}\\
Từ $f(x)=\ln y(x)$, suy ra $y(x)=\mathrm{e}^{f(x)}=\mathrm{e}^{-7\cdot 10^{-4}x+\ln0{,}05}=\dfrac{1}{20}\cdot \mathrm{e}^{-7\cdot 10^{-4}x}$.\\
Do đó $y(30)-y(15)=\dfrac{1}{20}\left(\mathrm{e}^{-7\cdot 10^{-4}\cdot 30}-\mathrm{e}^{-7\cdot 10^{-4}\cdot15}\right)\approx-5{,}2\cdot 10^{-4}$.
\itemch {\bf Đúng.}\\
Nồng độ trung bình của chất A từ thời điểm $15$ giây đến thời điểm $30$ giây là
\[\dfrac{1}{30-15}\displaystyle\int\limits_{15}^{30}y(x)\mathrm{\, d}x=\dfrac{1}{15}\displaystyle\int\limits_{15}^{30}\left(-\dfrac{1}{7\cdot 10^{-4}}\right)y'(x)\mathrm{\, d}x=-\dfrac{10^4}{105}\cdot y(x)\Bigg|_{15}^{30}\approx 0{,}05.\]
\end{itemchoice}
}
\end{ex}

\begin{ex}%[2D3H2-3]%[TEX ĐỀ MOON 2025]%[Lê Hữu Kiệt]
Kết quả khảo sát năng suất (đơn vị: tấn/ha) của một thửa ruộng được minh họa ở biểu đồ sau
\begin{center}
\begin{tikzpicture}[line join=round, line cap=round,>=stealth,font=\footnotesize,scale=1,declare function={xmax=9;}, y=0.8cm]
\draw[<->] (0,7)node[left]{Số thửa ruộng}--(0,0)node[below left]{$O$}--(xmax,0)node[below]{Năng suất (tấn/ha)};
\foreach \y in {1,...,6}{
\draw (0,\y) node[left]{$\y$};
\draw[gray, thin] (0,\y)--(xmax,\y);
}
\foreach \sothuaruong [count=\i from 1] in {3,4,6,5,5,2}{
\draw[fill=blue!30] (\i,0) rectangle (\i+1,\sothuaruong);
}
\foreach \nhom [count=\i from 1] in {{$[5{,}5;5{,}7)$}, {$[5{,}7;5{,}9)$}, {$[5{,}9;6{,}1)$}, {$[6{,}1;6{,}3)$}, {$[6{,}3;6{,}5)$}, {$[6{,}5;6{,}7)$}}{
\draw (\i+0.1,0)node[below=0.4cm, rotate=45]{\nhom};
}
\node at (current bounding box.north) {\textbf{Năng suất lúa của một số thửa ruộng}};
\end{tikzpicture}
\end{center}
\choiceTF
{\True Có $25$ thửa ruộng đã được khảo sát}
{\True Khoảng biến thiên của mẫu số liệu trên là $1{,}2$ (tấn/ha)}
{Khoảng tứ phân vị của mẫu số liệu ghép nhóm trên là $0{,}4675$}
{\True Phương sai của mẫu số liệu ghép nhóm trên xấp xỉ bằng $0{,}086656$}
\loigiai{
Bảng phân bố tần số ghép nhóm
\begin{center}
\begin{tabular}{|c|c|c|c|c|c|c|}
\hline
Khoảng năng suất (tấn/ha) & $[5{,}5;5{,}7)$ & $[5{,}7;5{,}9)$ & $[5{,}9;6{,}1)$ & $[6{,}1;6{,}3)$ & $[6{,}3;6{,}5)$ &$ [6{,}5;6{,}7)$ \\
\hline
Giá trị đại diện & $5{,}6$ & $5{,}8$ & $6{,}0$ & $6{,}2$ & $6{,}4$ & $6{,}6$ \\ \hline
Số thửa ruộng & $3$ & $4$ & $6$ & $5$ & $5$ & $2$ \\
\hline
Tần số tích lũy & $3$ & $7$ & $13$ & $18$ & $23$ & $25$ \\ \hline
\end{tabular}
\end{center}
\begin{itemchoice}
\itemch Mẫu số liệu có $n=25$. Vậy có $25$ thửa ruộng đã được khảo sát.
\itemch Ta có $x_{\min}=5{,}5$ và $x_{\max}=6{,}7$. Khoảng biến thiên là $R=x_{\max}-x_{\min}=1{,}2$.
\itemch Ta có $\dfrac{n}{4}=6{,}25$, suy ra nhóm chứa $Q_1$ là $[5{,}7;5{,}9)$. Khi đó $Q_1=5{,}7+\dfrac{6{,}25-3}{4}\cdot0{,}2=5{,}8625$.\\
Ta có $\dfrac{3n}{4}=18{,}75$, suy ra nhóm chứa $Q_3$ là $[6{,}3;6{,}5)$. Khi đó $Q_3=6{,}3+\dfrac{18{,}75-18}{5}\cdot0{,}2=6{,}31$.\\
Khoảng tứ phân vị $\Delta_Q=Q_3-Q_1=0{,}4475$.
\itemch Giá trị trung bình cộng
\[\overline{x}=\dfrac{5{,}6\cdot 3+5{,}8\cdot 4+6{,}0\cdot 6+6{,}2\cdot 5+6{,}4\cdot 5+6{,}6\cdot 2}{25}=6{,}088.\]
Phương sai của mẫu số liệu
\[s^2=\dfrac{(5{,}6-\overline{x})^2+(5{,}8-\overline{x})^2+(6{,}0-\overline{x})^2+(6{,}2-\overline{x})^2+(6{,}4-\overline{x})^2+(6{,}6-\overline{x})^2}{25}=0{,}086656.\]
\end{itemchoice}
}
\end{ex}

\begin{ex}%[2D1H5-4]
Cho hàm số $y = x^3 - 3x + 1$. Xét tính đúng sai của các mệnh đề sau:
\choiceTF
{\True Hàm số đã cho đồng biến trên $(1; +\infty)$}
{Hàm số đã cho có giá trị cực tiểu bằng $3$}
{\True Đồ thị hàm số đã cho cắt trục tung tại điểm có tung độ bằng $1$}
{\True Giá trị lớn nhất của hàm số đã cho trên $[-2; 1]$ bằng $3$}
\loigiai{
Tập xác định $\mathscr{D} = \mathbb{R}$.\\
Ta có $y' = 3x^2 - 3$.\\
Cho $y' = 0 \Leftrightarrow 3x^2 - 3 = 0 \Leftrightarrow x^2 = 1 \Leftrightarrow \hoac{& x = 1 \\ & x = -1.}$\\
Bảng biến thiên:
\begin{center}
\begin{tikzpicture}
\tkzTabInit[nocadre=false, lgt=1.2, espcl=3, deltacl=0.5]
{$x$/0.7, $y'$/0.7, $y$/2}
{$-\infty$, $-1$, $1$, $+\infty$}
\tkzTabLine{+,0,-,0,+}
\tkzTabVar{-/$-\infty$, +/$3$, -/$-1$, +/$+\infty$}
\end{tikzpicture}
\end{center}
\begin{itemchoice}
\itemch
Dựa vào bảng biến thiên, hàm số đồng biến trên các khoảng $(-\infty; -1)$ và $(1; +\infty)$.\\
Vậy hàm số đồng biến trên $(1; +\infty)$.

\itemch
Dựa vào bảng biến thiên, hàm số đạt cực tiểu tại $x_{\text{CT}} = 1$ và giá trị cực tiểu $y_{\text{CT}} = y(1) = -1$.

\itemch
Đồ thị hàm số cắt trục tung tại điểm có hoành độ $x = 0$.\\
Khi $x = 0$, ta có $y = 0^3 - 3 \cdot 0 + 1 = 1$.\\
Vậy đồ thị cắt trục tung tại điểm $(0; 1)$, có tung độ bằng $1$.

\itemch
Xét hàm số $y = x^3 - 3x + 1$ trên đoạn $[-2; 1]$.\\
Ta có $y(-2) = (-2)^3 - 3(-2) + 1 = -8 + 6 + 1 = -1$.\\
$y(-1) = (-1)^3 - 3(-1) + 1 = -1 + 3 + 1 = 3$.\\
$y(1) = 1^3 - 3(1) + 1 = 1 - 3 + 1 = -1$.\\
Vậy $\max\limits_{x \in [-2; 1]} y = 3$, đạt được tại $x = -1$.
\end{itemchoice}
}
\end{ex}

\begin{ex}%[2D1H3-6]%[TEX ĐỀ MOON 2025]%[Nguyễn Thế Duy]
Một người muốn xây một cái bể chứa nước, dạng một khối hộp chữ nhật không nắp. Xét tính đúng sai của các mệnh đề sau
\choiceTF
{Nếu đáy bể là hình vuông cạnh bằng $50$ m, lượng nước trong bể cao $1{,}5$ m thì thể tích nước trong bể là $1250$ m$^3$}
{\True Nếu thể tích bể bằng $\dfrac{256}{3}$ m$^3$, đáy bể là hình chữ nhật có chiều dài gấp đôi chiều rộng. Gọi chiều rộng của bể là $x$ (m) thì biểu thức xác định chiều cao bể theo $x$ là $h=\dfrac{128}{3x^2}$ (m)}
{\True Nếu thể tích bể bằng $\dfrac{256}{3}$ m$^3$, đáy bể là hình chữ nhật có chiều dài gấp đôi chiều rộng. Gọi chiều rộng bể là $x$ (m) thì công thức xác định diện tích xung quanh của bể là $S=\dfrac{256}{x}$ (m$^2$)}
{Nếu thể tích bể bằng $\dfrac{256}{3}$ m$^3$, đáy bể là hình chữ nhật có chiều dài gấp đôi chiều rộng. Giá thuê nhân công để xây thành bể là $500.000$ đồng/m$^2$, đổ bê tông đáy bể là $250.000$ đồng/m$^2$. Chi phí thấp nhất để thuê nhân công xây dựng bể đó là $24.100.000$ (kết quả làm tròn đến hàng trăm nghìn)}
\loigiai{
\begin{itemchoice}
\itemch \textbf{Sai}.\\
Thể tích của bể nước có đáy là hình vuông cạnh bằng $50$ m, lượng nước trong bể cao $1{,}5$ m là $V_1 = 50^2 \cdot 1{,}5 = 3750$ m$^3$.
\itemch \textbf{Đúng}.\\
Ta có $\dfrac{256}{3} = 2x \cdot x \cdot h \Rightarrow h = \dfrac{128}{3x^2}$ m.
\itemch \textbf{Đúng}.\\
Diện tích xung quanh của bể là $S = 2\left(2x+x \right) \cdot \dfrac{128}{3x^2} = \dfrac{256}{x}$ m$^2$.
\itemch \textbf{Sai}.\\
Số tiền thuê nhân công là
$T(x) = 2x^2 \cdot 2{,}5 + \dfrac{256}{x} \cdot 5 = 5x^2 + \dfrac{1280}{x}$ (trăm nghìn đồng).\\
Ta có $T'(x) = 10x - \dfrac{1280}{x^2}$.\\
Xét $T'(x) = 0 \Leftrightarrow 10x - \dfrac{1280}{x^2} = 0 \Leftrightarrow x = \sqrt[3]{1280}$.\\
Ta có bảng biến thiên
\begin{center}
\begin{tikzpicture}
\tkzTabInit[nocadre=false,lgt=1.2,espcl=2.7,deltacl=0.7]
{$x$ /0.6, $T'(x)$ /0.6, $T(x)$ /2}
{$0$,$\sqrt[3]{1280}$,$+\infty$}
\tkzTabLine{,-,$0$,+,}
\tkzTabVar{+/$0$,-/$T\left(\sqrt[3]{1280} \right)$,+/$+\infty$}
\end{tikzpicture}
\end{center}
Vậy số tiền tối thiểu cần để thuê nhân công là \\
$T = T\left(\sqrt[3]{1280} \right) \approx 707$ (trăm nghìn đồng) $\approx 70\,700\,000$ (đồng).
\end{itemchoice}
}
\end{ex}

\begin{ex}%[2D1H3-1]%[TEX ĐỀ MOON 2025]%[Nguyễn Văn Hiệp]
Cho hàm số $f(x)=-2\sin x-x$. Xét tính đúng sai của các mệnh đề sau
\choiceTF[0.3em]
{\True $f(0)=0$ và $f(\pi)=-\pi$}
{Đạo hàm của hàm số đã cho là $f'(x)=2\cos x-1$}
{\True Nghiệm của phương trình $f'(x)=0$ trên đoạn $[0;\pi]$ là $\dfrac{2\pi}{3}$}
{\True Giá trị nhỏ nhất của hàm số đã cho trên đoạn $[0;\pi]$ là $-\dfrac{2\pi}{3}-\sqrt{3}$}
\loigiai
{
\begin{itemchoice}
\itemch  $f(0) = -2\sin 0 - 0 = 0$ và $f(\pi) = -2\sin \pi - \pi = -\pi$.
\itemch  $f'(x) = -2\cos x - 1$.
\itemch Giải $-2\cos x - 1 = 0 \Rightarrow x = \dfrac{2\pi}{3} \in [0; \pi]$.
\itemch
$f(0) = -2\sin 0 - 0 = 0$ và $f(\pi) = -2\sin \pi - \pi = -\pi$, $f\left(\dfrac{2\pi}{3}\right) = -2\sin\left(\dfrac{2\pi}{3}\right) - \dfrac{2\pi}{3} = -\sqrt{3} - \dfrac{2\pi}{3}$.\\
Suy ra $\min\limits_{[0;\pi]}=f\left(\dfrac{2\pi}{3}\right)= -\sqrt{3} - \dfrac{2\pi}{3}$.
\end{itemchoice}
}
\end{ex}

\begin{ex}%[2D1H3-1]%[TEX ĐỀ MOON 2025]%[Nguyễn Cường]
Cho hàm số $f(x)=2\sin x-x$.
\choiceTF
{\True $f(0)=0$ và $f\left(\dfrac{\pi}{2}\right)=2-\dfrac{\pi}{2}$}
{Đạo hàm của hàm số đã cho là $f'(x)=-2\cos x-1$}
{\True Nghiệm của phương trình $f'(x)=0$ trên đoạn $\left[0;\dfrac{\pi}{2}\right]$ là $\dfrac{\pi}{3}$}
{\True Giá trị lớn nhất của $f(x)$ trên đoạn $\left[0;\dfrac{\pi}{2}\right]$ là $\sqrt{3}-\dfrac{\pi}{3}$}
\loigiai{
\begin{itemchoice}
\itemch Ta có $f(0)=2\sin 0-0=0$ và $f\left(\dfrac{\pi}{2}\right)=2\sin\dfrac{\pi}{2}-\dfrac{\pi}{2}=2-\dfrac{\pi}{2}$.
\itemch Đạo hàm $f'(x)=2\cos x-1$.
\itemch Ta có
\allowdisplaybreaks
\begin{eqnarray*}
f'(x)=0&\Leftrightarrow& 2\cos x-1=0\\
&\Leftrightarrow& \cos x=\dfrac{1}{2}\\
&\Leftrightarrow& x=\dfrac{\pi}{3}\in \left[0;\dfrac{\pi}{2}\right]
\end{eqnarray*}
\itemch Hàm số liên tục trên đoạn $\left[0;\dfrac{\pi}{2}\right]$.\\
Ta có $f(0)=0$, $f\left(\dfrac{\pi}{2}\right)=2-\dfrac{\pi}{2}$ và $f\left(\dfrac{\pi}{3}\right)=\sqrt{3}-\dfrac{\pi}{3}$.\\
Vậy $\max\limits_{\left[0;\dfrac{\pi}{2}\right]}f(x)=\sqrt{3}-\dfrac{\pi}{3}$.
\end{itemchoice}
}
\end{ex}

\begin{ex}%[2D1H3-1]%[TEX ĐỀ MOON 2025]%[Huỳnh Thanh Chí]
Cho hàm số $f(x)=\sin 2x-x$. Xét tính đúng sai của các mệnh đề sau
\choiceTF[0.4em]
{\True $f\left(-\dfrac{\pi}{2}\right)=\dfrac{\pi}{2}$ và $f\left(\dfrac{\pi}{2}\right)=-\dfrac{\pi}{2}$}
{Đạo hàm của hàm số đã cho là $f'(x)=\cos 2x-1$}
{\True Nghiệm của phương trình $f'(x)=0$ trên đoạn $\left[-\dfrac{\pi}{2};\dfrac{\pi}{2}\right]$ là $x=\pm\dfrac{\pi}{6}$}
{\True Giá trị nhỏ nhất của hàm số đã cho trên đoạn $\left[-\dfrac{\pi}{2};\dfrac{\pi}{2}\right]$ là $-\dfrac{\pi}{2}$}
\loigiai{
\begin{itemchoice}
\itemch Ta có $f\left(-\dfrac{\pi}{2}\right)=\dfrac{\pi}{2}$ và $f\left(\dfrac{\pi}{2}\right)=-\dfrac{\pi}{2}$.
\itemch Ta có $f'(x)=2\cos 2x-1$.
\itemch Ta có \allowdisplaybreaks
\begin{eqnarray*}
f'(x)=0\Leftrightarrow 2\cos 2x-1\Leftrightarrow \cos 2x=\dfrac{1}{2}\Leftrightarrow \hoac{& 2x=\dfrac{\pi}{3}+k2\pi\\ & 2x=-\dfrac{\pi}{3}+k2\pi}
\Leftrightarrow \hoac{& x=\dfrac{\pi}{6}+k\pi\\ & x=-\dfrac{\pi}{6}+k\pi}, k\in\mathbb{Z}.
\end{eqnarray*}
Vì $x\in \left[-\dfrac{\pi}{2};\dfrac{\pi}{2}\right]$ nên tập nghiệm của phương trình là $S=\left\{-\dfrac{\pi}{6};\dfrac{\pi}{6}\right\}$.
\itemch Ta có $f\left(-\dfrac{\pi}{2}\right)=\dfrac{\pi}{2}$, $f\left(\dfrac{\pi}{2}\right)=-\dfrac{\pi}{2}$, $f\left(\dfrac{\pi}{6}\right)=\dfrac{\sqrt{3}}{2}-\dfrac{\pi}{6}$ và $f\left(-\dfrac{\pi}{6}\right)=-\dfrac{\sqrt{3}}{2}+\dfrac{\pi}{6}$.\\
Vậy $\min\limits_{\mathscr{D}} f(x)=f\left(\dfrac{\pi}{2}\right)=-\dfrac{\pi}{2}$, với $\mathscr{D}=\left[-\dfrac{\pi}{2};\dfrac{\pi}{2}\right]$.
\end{itemchoice}
}
\end{ex}

\begin{ex}%[2D1H2-1]
Cho hàm số $f(x)=\dfrac{2x-3}{x^2+4}$.
\choiceTF
{\True Hàm số đã cho xác định với mọi $x\in\mathbb{R}$}
{\True Nghiệm của phương trình $f'(x)=0$ là $x=-1$; $x=4$}
{\True Giá trị cực đại của hàm số $f(x)$ là $\dfrac{1}{4}$}
{Tập giá trị của hàm số $f(x)$ là đoạn $[a;b]$ thì $3a+2b=-2$}
\loigiai{
\begin{itemchoice}
\itemch Ta có $x^2+4>0$ với mọi giá trị $x$.\\
Hàm số đã cho xác định với mọi $x\in\mathbb{R}$.
\itemch Ta có $f'(x)=\dfrac{-2x^2+6x+8}{(x^2+4)^2}$.\\
Xét $f'(x)=0\Leftrightarrow -2x^2+6x+8=0\Leftrightarrow\hoac{&x=-1\\&x=4.}$
\itemch Bảng biến thiên của $f(x)$
\begin{center}
\begin{tikzpicture}
\tkzTabInit[nocadre=false,lgt=1.5,espcl=2.5,deltacl=0.6]
{$x$/0.7,$f'(x)$/0.7,$f(x)$/1.5}{$-\infty$,$-1$,$4$,$+\infty$}
\tkzTabLine{,-,0,+,0,-,}
\tkzTabVar{+/$0$,-/$-1$,+/$\dfrac{1}{4}$,-/$0$}
\end{tikzpicture}
\end{center}
\itemch Dựa vào bảng biến thiên ta có tập giá trị của hàm số là $\left[-1;\dfrac{1}{4} \right]$.\\
Vậy $3a+2b=3\cdot(-1)+2\cdot\dfrac{1}{4}=\dfrac{5}{2}$.

\end{itemchoice}
}
\end{ex}

\begin{ex}%[2D1H1-2]%[TEX ĐỀ MOON 2025]%[Nguyễn Thế Duy]
Cho hàm số $y=f(x)$ có bảng biến thiên như sau
\begin{center}
\begin{tikzpicture}
\tkzTabInit[nocadre=false,lgt=1.2,espcl=2.5,deltacl=0.6]
{$x$ /0.6, $y'$ /0.6, $y$ /2}
{$-\infty$,$-1$,$1$,$+\infty$}
\tkzTabLine{,+,$0$,-,$0$,+,}
\tkzTabVar{-/$-\infty$,+/$3$,-/$-1$,+/$+\infty$}
\end{tikzpicture}
\end{center}
Xét tính đúng sai của các mệnh đề sau
\choiceTF
{\True $y'=0\Leftrightarrow\hoac{& x=-1 \\ & x=1}$}
{Điểm cực tiểu của hàm số là $x=-1$}
{\True Đồ thị hàm số $y=f(x)$ cắt trục $Ox$ tại ba điểm phân biệt}
{\True Hàm số $y=f(2-x)$ đồng biến trên khoảng $(1;3)$}
\loigiai{
\begin{itemchoice}
\itemch \textbf{Đúng}.\\
Từ bảng biến thiên ta có $y'=0\Leftrightarrow\hoac{&x=-1 \\ &x=1.}$
\itemch \textbf{Sai}.\\
Điểm cực tiểu của hàm số là $\left(1; -1 \right)$.
\itemch \textbf{Đúng}.\\
Từ bảng biến thiên ta có $f(x) = 0$ có ba nghiệm phân biệt.
\itemch \textbf{Đúng}.\\
Xét hàm số $y = f(2-x)$ ta có $y' = -f(2-x)$.\\
Xét $y' = 0 \Leftrightarrow \hoac{&2-x = -1\\&2-x=1} \Leftrightarrow \hoac{&x = 3\\&x = 1.}$\\
Ta có bảng xét dấu
\begin{center}
\begin{tikzpicture}
\tkzTabInit[nocadre=false,lgt=1.2,espcl=2.5,deltacl=0.6]
{$x$ /0.6, $y'$ /0.6}
{$-\infty$,$1$,$3$,$+\infty$}
\tkzTabLine{,-,$0$,+,$0$,-,}
\end{tikzpicture}
\end{center}
Vậy hàm số $y = f(2-x)$ đồng biến trên khoảng $\left(1; 3\right)$.
\end{itemchoice}
}
\end{ex}

\begin{ex}%[2D1H3-1]%[TEX Đề Moon 2025]%[Võ Nguyên Thạch]
Cho hàm số $f(x)=x-\sin 2x$.
\choiceTF
{\True $f(0)=0$ và $f(\pi)=\pi$}
{Đạo hàm của hàm số đã cho là $f'(x)=1+2\cos 2x$}
{\True Nghiệm của phương trình $f'(x)=0$ trên đoạn $[0;\pi]$ là $\dfrac{\pi}{6}$ và $\dfrac{5\pi}{6}$}
{\True Giá trị nhỏ nhất của hàm số đã cho trên đoạn $[0;\pi]$ là $\dfrac{\pi}{6}-\dfrac{\sqrt{3}}{2}$}
\loigiai{
\begin{itemchoice}
\itemch Ta có $f(0)=0-\sin(2\cdot 0)=0-0=0$; $f(\pi)=\pi-\sin(2\cdot \pi)=\pi-0=\pi$.
\itemch Ta có $f'(x)=(x-\sin 2x)'=1-(2x)'\cos 2x=1-2\cos 2x$.
\itemch Ta có
\allowdisplaybreaks
\begin{eqnarray*}
&&f'(x)=0\\
&\Leftrightarrow&1-2\cos 2x=0\\
&\Leftrightarrow&\cos 2x=\dfrac{1}{2}\\
&\Leftrightarrow&\hoac{&2x=\dfrac{\pi}{3}+k2\pi\\&2x=-\dfrac{\pi}{3}+k2\pi},\, k\in \mathbb{Z}\\
&\Leftrightarrow&\hoac{&x=\dfrac{\pi}{6}+k\pi\\&x=-\dfrac{\pi}{6}+k\pi},\, k\in \mathbb{Z}.
\end{eqnarray*}
Do $x\in [0;\pi]$ nên $x=\dfrac{\pi}{6}$ và $x=\dfrac{5\pi}{6}$.
\itemch Ta có
\begin{itemize}
\item $f(0)=0-2\sin (2\cdot 0)=0$.
\item $f\left(\dfrac{\pi}{6}\right)=\dfrac{\pi}{6}+\sin \left(2\cdot \dfrac{\pi}{6}\right)=\dfrac{\pi}{6}+\dfrac{\sqrt 3}{2}$.
\item $f\left(\dfrac{5\pi}{6}\right)=\dfrac{5\pi}{6}+\sin \left(2\cdot \dfrac{5\pi}{6}\right)=\dfrac{5\pi}{6}-\dfrac{\sqrt 3}{2}$.
\item $f(\pi)=\pi+\sin(2\pi)=\pi$.
\end{itemize}
Vậy giá trị nhỏ nhất của hàm số đã cho trên đoạn $[0;\pi]$ là $\dfrac{\pi}{6}-\dfrac{\sqrt{3}}{2}$.
\end{itemchoice}
}
\end{ex}

% \paragraph{Mức độ V}
\begin{ex}%[2H5V3-4]%[TEX Đề Moon 2025]%[Vũ Hồng Toàn]
Trong không gian $Oxyz$, cho đường thẳng $\Delta\colon\heva{& x=3+t \\ & y=-1-t\\ & z=-2+t}$, điểm $M(1;2;-1)$ và mặt cầu $(S)\colon x^2+y^2+z^2-4x+10y+14z+64=0$. Xét tính đúng sai của các mệnh đề sau
\choiceTF
{\True Đường thẳng $\Delta $ có một vectơ chỉ phương là $\overrightarrow{u}=(1;-1;1)$}
{\True Mặt cầu $(S)$ có tâm $I(2;-5;-7)$ và bán kính $R=\sqrt{14}$}
{Mặt phẳng đi qua điểm $M$ và vuông góc với đường thẳng $\Delta$ là $x-y+z-2=0$}
{\True Gọi $\Delta'$ là đường thẳng đi qua $M$ cắt đường thẳng $\Delta$ tại $A$, cắt mặt cầu tại $B$ sao cho $\dfrac{AM}{AB}=\dfrac{1}{3}$ và điểm $B$ có hoành độ là số nguyên. Mặt phẳng trung trực của đoạn $AB$ có phương trình là $2x-4y-4z-43=0$}
\loigiai{
\begin{itemchoice}
\itemch Đường thẳng $\Delta $ có một vectơ chỉ phương là $\overrightarrow{u}=(1;-1;1)$.
\itemch Ta có $x^2+y^2+z^2-4x+10y+14z+64=0\Rightarrow (x-2)^2+(y+5)^2+(z+7)^2=14$.\\
Do đó $(S)$ có tâm $I(2;-5;-7)$ và bán kính $R=\sqrt{14}$.
\itemch Mặt phẳng đi qua điểm $M$ và vuông góc với đường thẳng $\Delta$ nên có một vectơ pháp tuyến là $\overrightarrow{n}=\overrightarrow{u}=(1;-1;1)$ nên có phương trình
\[(x-1)-(y-2)+(z+1)=0\Rightarrow x-y+z+2=0.\]
\itemch Do $A\in \Delta\Rightarrow A(3+t;-1-t;-2+t)$, $B(x;y;z)\in (S)$;\\ $\overrightarrow{MB}=(x-1;y-2;z+1)$, $\overrightarrow{MA}=(2+t;-3-t;-1+t)$.\\
Vì $\dfrac{AM}{AB}=\dfrac{1}{3}$. Xét hai trường hợp
\begin{itemize}
\item Trường hợp $M$ nằm giữa $A$ và $B$. Suy ra $\overrightarrow{MB}=-2\overrightarrow{AM}$.\\ Từ đó ta có hệ phương trình
\[\heva{&x-1=-2(2+t)\\&y-2=-2(-3-t)\\&z+1=-2(-1+t)}\Rightarrow \heva{&x=-3-2t\\&y=8+2t\\&z=1-2t.}\]
Mà $B\in(S)$ nên ta có
\allowdisplaybreaks
\begin{eqnarray*}
&&(-3 - 2t - 2)^2 + (8 + 2t + 5)^2 + (1 - 2t + 7)^2 = 14\\
&\Leftrightarrow& 3t^2+10t+61=0\\
&\Rightarrow& \text{phương trình vô nghiệm.}
\end{eqnarray*}
Do đó không thoả mãn.
\item  Trường hợp $M$ nằm ngài $A$ và $B$. Suy ra $\overrightarrow{MB}=4\overrightarrow{AM}$.\\
Từ đó ta có hệ phương trình
\[\heva{&x-1=4(2+t)\\&y-2=4(-3-t)\\&z+1=4(-1+t)}\Rightarrow \heva{&x=9+4t\\&y=-10-4t\\&z=-5+4t.}\]
Mà $B\in(S)$ nên ta có
\allowdisplaybreaks
\begin{eqnarray*}
&&(9 + 4t - 2)^2 + (-10 - 4t + 5)^2 + (-5 + 4t + 7)^2  = 14\\
&\Leftrightarrow& 3t^2+7t+4=0
\Leftrightarrow\hoac{&t=-1&\text{ (nhận)}\\&t=-\dfrac{4}{3}&\text{ (loại)}.}
\end{eqnarray*}
\end{itemize}
Khi đó $A(2;0;-3)$ và $B(5;-6;-9)$.\\
Gọi $C$ là trung điểm của đoạn $AB$ suy ra $C\left(\dfrac{7}{2};-3;-6\right)$. Mặt phẳng trung trực của đoạn $AB$ đi qua $C$ và có một vectơ pháp tuyến là $\dfrac{1}{3}\cdot\overrightarrow{AB}=(1;-2;-2)$ nên có phương trình
\[\left(x-\dfrac{7}{2}\right)-2(y+3)-2(z+6)=0\Rightarrow 2x-4y-4z-43=0.\]
\end{itemchoice}
}
\end{ex}

\begin{ex}%[2H5V3-4]
\immini{Vệ tinh hoạt động dựa trên nguyên lý của vật lý Newton - một vật thể bị kéo bởi một lực hấp dẫn từ một vật thể khác sẽ chuyển động theo một quỹ đạo elip xung quanh vật thể đó. Để đưa vệ tinh lên quỹ đạo, người ta sử dụng các loại tên lửa đẩy khác nhau để cung cấp cho vệ tinh động lượng cần thiết để thoát khỏi trọng lực của Trái Đất và duy trì quỹ đạo ổn định. Để thuận tiện ta quy ước một quỹ đạo gần tròn thành một đường tròn. }{\begin{tikzpicture}[line join = round, line cap=round,>=stealth,font=\footnotesize,transform shape,scale=0.8]
\draw
(0,0) circle(3.5cm)
;
\draw[fill=black!30]
(0,0) circle(1.6cm);
\draw[<->] (0:3.5)--(0:1.6) node[pos=0.5,above]{$h$ km};
\draw[<->] (0:0)--(0:1.6) node[pos=0.5,above]{$6\, 400$ km};
\fill
(3.5,0)circle(1.5pt)node[above right]{$B$};
\end{tikzpicture}}
Trong hệ tọa độ $Oxyz$, gốc tọa độ là tâm trái đất, một vệ tinh nhân tạo tạo quỹ đạo được coi như một đường tròn có bán kính $13440$ km có điểm xuất phát là điểm $B (4032;0;-5376)$ và đây cũng là điểm gần Trái Đất nhất của vệ tinh. Quỹ đạo của vệ tinh này nằm trên mặt phẳng vuông góc với trục tung và có tâm nằm trên đường thẳng $OB$. Coi Trái Đất là hình cầu hoàn hảo có bán kính bằng $6400$ km.
\choiceTF
{Phương trình mặt phẳng chứa quỹ đạo của vệ tinh là $x+z=0$}
{\True Khi xuất phát tại điểm $B$ vệ tinh đang ở độ cao $320$ km so với mặt đất}
{Quỹ đạo của tên lửa là đường tròn có tâm $I(-4032; 0;5120)$}
{Khi Trái Đất quay, điểm cực Nam và cực Bắc của Trái Đất không thay đổi vị trí. Biết rằng điểm cực Nam của Trái Đất có tọa độ là $(0; 3840;5120)$. Khoảng cách gần nhất giữa vệ tinh và điểm cực Nam bằng $10112$ km (làm tròn kết quả đến hàng đơn vị)}
\loigiai{
\begin{itemchoice}
\itemch Quỹ đạo của vệ tinh này nằm trên mặt phẳng vuông góc với trục tung nên một vectơ chỉ phương của
mặt phẳng chứa quỹ đạo của vệ tinh là $(0; 1; 0)$.\\
Khi đó, phương trình mặt phẳng chứa quỹ đạo của vệ tinh có dạng $y+a=0$.\\
Quỹ đạo đi qua $B(4032;0;-5376)$ nên $0+a=0$ hay $a=0$.\\
Vậy phương trình mặt phẳng chứa quỹ đạo của vệ tinh là $y=0$.
\itemch Khoảng cách ngắn nhất từ Trái Đất đến vệ tinh bằng
\[OB-R=\sqrt{4032^{2}+0^{2}+(-5376)^{2}}-6400=320(km).\]
Vậy khi xuất phát tại điểm $B$ vệ tình đang ở độ cao $320$ km so với mặt đất.
\itemch
Quỹ đạo của vệ tinh có tâm nằm trên đường thẳng $OB$ nên $I$ nằm trên đường thẳng $OB$.\\
Mặt khác $IB=R_{qd}=13440=2\cdot OB$ nên $O$ là trung điểm của $IB$.\\
Khi đó
\[\begin{cases}
x_I=2x_O-x_B \\
y_I=2y_O-y_B \\
z_I=2z_O-z_B
\end{cases}
\Leftrightarrow
\begin{cases}
x_I=-4032\\
y_I=0\\
z_I=5376\end{cases}
\Rightarrow I(-4032; 0; 5376).\]
\itemch
Gọi $H$ là hình chiếu của $K$ trên mặt phẳng chứa quỹ đạo $(\alpha)\colon y=0\Rightarrow H(0; 0; 5120)$.\\
Ta có
\begin{itemize}
\item $KH=\mathrm{d}(K, (\alpha))=3840$.
\item $IH=\sqrt{4032^2+(5376-5120)^2}=64\sqrt{3985}$.
\item $NH=IN-IH=13440-64\sqrt{3985}$.
\end{itemize}
Nối $I$ và $H$ cắt vệ tinh tại $N$. Khi đó:
\[KN_{\text{min}}=\sqrt{KH^2+NH^2}=\sqrt{3840^2+(13440-64\sqrt{3985})^2} \approx 10154 \text{ (km)}.\]

\end{itemchoice}
}
\end{ex}

\begin{ex}%[2H5V3-4]%[TEX Đề Moon 2025]%[Võ Nguyên Thạch]
Trong không gian $Oxyz$ (đơn vị trên mỗi trục tính theo mét), một ngọn hải đăng được đặt ở vị trí $I(17;20;45)$. Biết rằng ngọn hải đăng đó được thiết kế với bán kính phủ sáng là $4$ km.
\choiceTF
{\True Phương trình mặt cầu để mô tả ranh giới bên ngoài của vùng phủ sáng trên biển của hải đăng là $(x-17)^2+(y-20)^2+(z-45)^2=16\,000\,000$}
{Nếu người đi biển ở vị trí $M(18;21;50)$ thì không thể nhìn thấy được ánh sáng từ ngọn hải đăng}
{Nếu người đi biển ở vị trí $N(4\,019;21;44)$ thì có thể nhìn thấy được ánh sáng từ ngọn hải đăng}
{\True Nếu hai người đi biển ở vị trí có thể nhìn thấy được ánh sáng từ ngọn hải đăng thì khoảng cách giữa hai người đó không quá $8$ km}
\loigiai{
\begin{itemchoice}
\itemch Ta có $4$ km $=4\,000$ m.\\
Phương trình mặt cầu mô tả ranh giới bên ngoài vùng phủ sáng trên biển của hải đăng là phương trình mặt cầu tâm $I(17;20;45)$, bán kính $4\,000$ m là
\[(x-17)^2+(y-20)^2+(z-45)^2=16\,000\,000.\]
\itemch Ta có $IM=\sqrt{(18-17)^2+(21-20)^2+(50-45)^2}=\sqrt 27<4\,000$.\\
Khi đó, người ở vị trí điểm $M$ có thể nhìn thấy ánh sáng từ ngọn hải đăng.
\itemch Ta có $IN=\sqrt{(4\,019-17)^2+(21-20)^2+(44-45)^2}=\sqrt{16\,016\,006}>4\,000$.\\
Khi đó, người ở vị trí điểm $N$ không thể nhìn thấy ánh sáng từ ngọn hải đăng.
\itemch Vì đường kính của mặt cầu trên bằng $8\,000$ m hay $8$ km nên khoảng cách giữa hai người đi biển ở vị trí có thể nhìn thấy ánh sáng từ ngọn hải đăng không quá $8$ km.
\end{itemchoice}
}
\end{ex}

\begin{ex}%[2H5V3-3]
Trong không gian với tọa độ $Oxyz$, cho mặt cầu $(S)\colon  x^2+y^2+z^2-2x+4y+1=0$ và mặt phẳng $(P)\colon x+y-z-2=0$.
\choiceTF
{Mặt cầu $(S)$ có tâm $I(-1;2;0)$}
{Bán kính của mặt cầu $(S)$ là $R=4$}
{\True Khoảng cách từ tâm $I$ của mặt cầu $(S)$ đến mặt phẳng $(P)$ bằng $\sqrt{3}$}
{\True Mặt phẳng $(P)$ cắt mặt cầu $(S)$ theo giao tuyến là một đường tròn có bán kính $r=1$}
\loigiai{
\begin{itemchoice}
\itemch Mặt cầu $(S)\colon x^2+y^2+z^2-2x+4y+1=0$ có tâm $I(1;-2;0)$.
\itemch Bán kính mặt cầu $(S)\colon  x^2+y^2+z^2-2x+4y+1=0$ là $R=\sqrt{1^2+(-2)^2+0^2-1}=2$.
\itemch Khoảng cách từ tâm $I(1;-2;0)$ của mặt cầu $(S)$ đến mặt phẳng $(P)\colon x+y-z-2=0$ là \[\mathrm{d}(I,(P))=\dfrac{|1+(-2)-0-2|}{\sqrt{1^2+1^2+(-1)^2}}=\sqrt{3}.\]
\itemch Ta thấy $\mathrm{d}(I,(\mathrm{P}))<R$ nên $(P)$ cắt mặt cầu $(S)$ theo giao tuyến là một đường tròn với bán kính
\[R=\sqrt{R^2-\mathrm{d}(I, (P))^2}=\sqrt{2^2-\left(\sqrt{3}\right)^2}=1.\]
\end{itemchoice}
}
\end{ex}

\begin{ex}%[2H5V2-8]%[TEX ĐỀ MOON 2025]%[Huỳnh Thanh Chí]
Trong không gian với hệ tọa độ $Oxyz$, một cabin cáp treo xuất phát từ điểm $A(10;3;0)$ và chuyển động đều theo đường cáp có vectơ chỉ phương $\overrightarrow{u}=(2;-2;1)$ (hướng chuyển động cùng chiều với hướng vectơ $\overrightarrow{u}$ với tốc độ là $4{,}5$ (m/s)) (đơn vị trên mỗi trục là mét). Xét tính đúng sai của các mệnh đề sau
\choiceTF
{\True Phương trình tham số của đường cáp là $\heva{& x=10+2t' \\ & y=3-2t'\\ & z=t'}$ $(t\in\mathbb{R})$}
{\True Giả sử sau thời gian $t$ (s) kể từ khi xuất phát $(t\ge 0)$, cabin đến điểm $M$. Khi đó tọa độ điểm $M$ là $M\left(3t+10;-3t+3;\dfrac{3t}{2}\right)$}
{Cabin dừng ở điểm $B$ có hoành độ $x_B=550$, khi đó quãng đường $AB$ dài $800$ m}
{Đường cáp $AB$ tạo với mặt phẳng $(Oxy)$ một góc $30^\circ$}
\definecolor{ecru}{rgb}{0.76, 0.7, 0.5}
\definecolor{darkolivegreen}{rgb}{0.33, 0.42, 0.18}
\definecolor{deepskyblue}{rgb}{0.0, 0.75, 1.0}
\definecolor{antiquebrass}{rgb}{0.8, 0.58, 0.46}
\definecolor{arsenic}{rgb}{0.23, 0.27, 0.29}
\definecolor{ashgrey}{rgb}{0.7, 0.75, 0.71}\definecolor{alizarin}{rgb}{0.82, 0.1, 0.26}
\begin{center}
\begin{tikzpicture}[line join=round, line cap=round,scale=1,transform shape]
\clip (-3,-3) rectangle (3,3);
\fill[bottom color=white,top color=deepskyblue!90, middle color=white] (-3,-3) rectangle (3,3);
\tikzset{dat/.pic={
\def\N{
(-3,3)
..controls +(20:.7) and +(150:.7) ..(-.8,1.5)
..controls +(-30:.7) and +(150:.6) ..(1.5,-.8)
..controls +(-30:1) and +(90:.4) ..(3,-3)--(-3,-3)--cycle
;
}
\draw[black]\N;
\fill[darkolivegreen!50] \N;
}}
\tikzset{cap_treo/.pic={
\def\X{
(-.85,1)--(-.8,1)
..controls +(-90:.9) and +(-180:.6) ..(-.1,0.1)--(-.1,.05)
..controls +(-180:.65) and +(-90:.9) ..cycle
(-.15,0.05)--(-.15,-.1)--(-.23,-.22)--(-.6,-.22)--(-.5,-.1)--(-.5,.08)
;
}
%\fill[black] \X;
\draw\X;
\draw(-3,2.3)--(3.5,-1.2);
\draw(-1.2,1)--(-.5,1);
\draw(-.5,1)--(-.9,1.2)--(-.95,1.17)--(-.6,.97)--cycle;
\draw[fill=arsenic!80](-.9,-.05)--(.52,-.05)--(.28,-.3)--(-1.05,-.3)--cycle;
\def\M{
(-.85,1)--(-.8,1)
..controls +(-90:.9) and +(-180:.6) ..(-.1,0.1)--(-.1,.05)
..controls +(-180:.65) and +(-90:.9) ..cycle
(-.15,0.05)--(-.15,-.1)--(-.23,-.22)--(-.6,-.22)--(-.5,-.1)--(-.5,.08)
;
}
\fill[arsenic] \M;
\draw\M;
\def\N{
(-1.05,-.3)
..controls +(-120:.6) and +(120:.6) ..(-.98,-1.5)--(.4,-1.5)
..controls +(70:.6) and +(-60:.3) ..(.28,-.3)--cycle
;
}
\fill[arsenic] \N;
\draw\N;
\def\P{
(.4,-1.5)
..controls +(70:.6) and +(-60:.3) ..(.28,-.3)--(.52,-.05)
..controls +(-40:.4) and +(50:.4) ..(.6,-1.2)--cycle
;
}
\fill[arsenic!80] \P;
\draw\P;
\draw (-1.23,-1)--(.5,-1)--(.77,-.6);
\draw (-1.05,-.5)--(-1,-.4)--(.2,-.4)--(.25,-.5)--cycle
(.25,-.5)--(.3,-.6)--(.3,-.6)--(-1.1,-.6)--(.-1.05,-.5)--cycle
;
\draw[alizarin](.42,-1)..controls +(100:.4) and +(-65:.4) ..(.2,-.3);
\draw[alizarin](-1.15,-1)..controls +(100:.3) and +(-120:.3) ..(-.95,-.3);
}}
\path
(0,0)pic[scale=1]{dat}(0,0)pic[scale=1]{cap_treo};
\end{tikzpicture}
\begin{tikzpicture}[scale=1, font=\footnotesize, line join=round,xscale=.2, line cap=round,>=stealth]
\def\a{1/16}
\def\xmin{-3} \def\xmax{12}
\def\ymin{-3} \def\ymax{3}
\coordinate (O) at (0,0);
\coordinate (E) at (-10,-3);
\coordinate (N) at ($(E)!.7!(O)$);
\coordinate (P) at ($(E)!.2!(O)$);
\coordinate (D) at ($(E)!.3!(O)$);
\coordinate (A) at ($(N)+(3,0)$);
\coordinate (B) at (-14,1);
\coordinate (M) at ($(A)!.7!(B)$);
\draw[->] (\xmin,0)--(\xmax,0) node [above right]{$y$};
\draw[->] (O)--(0,\ymax) node [left]{$z$};
\draw[->] (O)--(E) node [below right]{$x$};
\node at (0,0)[above right]{$O$};
\draw[dashed] (B)--(P)node [right]{$550$} (N)--(A)node [below right]{$A(10;3;0)$}--(3,0);
\draw(B)node [above]{$B$}--(A);
\draw[->,red](O)--(-4,.6)node [above right]{$\overrightarrow{u}$};
\node at (N) [left]{$10$};
\node at (M) [above]{$M$};
\node at (P) [left]{$x_B$};
\draw[fill=black] (3,0) circle (.5pt) node[below]{\footnotesize $3$};
%	\path (-28,0) node[opacity=.5,scale=.5] {\includegraphics{image/h35}};
\clip (\xmin+0.1,\ymin+0.1) rectangle (\xmax-0.1,\ymax-0.1);
\end{tikzpicture}
\end{center}
\loigiai{
\begin{itemchoice}
\itemch Phương trình chính tắc của đường cáp là $\dfrac{x-10}{2}=\dfrac{y-3}{-2}=\dfrac{z}{1}$.\\
Phương trình tham số của đường cáp là $\heva{& x=10+2t' \\ & y=3-2t'\\ & z=t'}$ $(t'\in\mathbb{R})$.
\itemch Do tốc độ chuyển động của cabin là $4{,}5$ m/s nên độ dài $A M$ bằng $4{,}5 t$ m. \\
Vì vậy $\left|\overrightarrow{AM}\right|=4{,}5 t$ $(t \geq 0)$.\\
Do hai vectơ $\overrightarrow{A M}$ và $\overrightarrow{u}$ là cùng phương và cùng hướng nên $\overrightarrow{A M}=k \overrightarrow{u}$ với $k$ là số thực dương nào đó. \\
Suy ra $\left|\overrightarrow{A M}\right|=k|\overrightarrow{u}|=k \cdot \sqrt{2^2+(-2)^2+1}=3 k$. Do đó $3 k=4{,}5 t$. Suy ra $k=\dfrac{3 t}{2}$. \\
Vì thế, ta có $\overrightarrow{A M}=\dfrac{3 t}{2} \overrightarrow{u}=\left(3 t ;-3 t ; \dfrac{3 t}{2}\right)$.\\
Gọi tọa độ của điểm $M$ là $\left(x_M ; y_M ; z_M\right)$.\\
Ta có $\overrightarrow{A M}=\left(x_M-x_A ; y_M-y_A ; z_M-z_A\right)=\left(3 t ;-3 t ; \dfrac{3 t}{2}\right)$.\\
Nên $\heva{&x_M=3 t+x_A \\ &y_M=-3 t+y_A \\ &z_M=\dfrac{3 t}{2}+z_A}\Leftrightarrow\heva{&x_M=3 t+10 \\ &y_M=-3 t+3 \\ &z_M=\dfrac{3 t}{2}.}$
Vậy điểm $M$ có tọạ độ là $\left(3 t+10 ;-3 t+3 ;\dfrac{3 t}{2}\right)$.
\itemch Do $x_B=550$ nên $3 t+10=550$, tức là $t=180$ s. Do đó, ta có điểm $B(550 ;-537 ; 270)$. \\
Vậy $A B=\sqrt{(550-10)^2+(-537-3)^2+(270-0)^2}=\sqrt{656100}=810$ m.
\itemch Đường thẳng $AB$ có vectơ chỉ phương $\overrightarrow{u}=(2 ;-2 ; 1)$ và mặt phẳng $(Oxy)$ có vectơ pháp tuyến $\overrightarrow{k}=(0 ; 0 ; 1)$. Do đó, ta có
\[
\sin (\Delta,(O x y))=|\cos (\overrightarrow{u},\overrightarrow{k})|=\dfrac{\left|\overrightarrow{u} \cdot \overrightarrow{k}\right|}{|\overrightarrow{u}| \cdot|\overrightarrow{k}|}=\dfrac{1}{3 \cdot 1}=\dfrac{1}{3}.\]
Vậy $(\Delta,(O x y)) \approx 19^{\circ}$.
\end{itemchoice}
}
\end{ex}

\begin{ex}%[2H5V2-3]%[TEX ĐỀ MOON 2025]%[Nguyễn Thế Duy]
Trong không gian $Oxyz$, cho mặt phẳng $(P)\colon x-z=0$, đường thẳng $d\colon\heva{& x=1+2t \\ & y=t\\ & z=t}$ và hai điểm $A(1;2;1)$, $B(2;1;4)$. Xét tính đúng sai của các mệnh đề sau
\choiceTF
{\True Điểm $A$ thuộc mặt phẳng $(P)$}
{Hoành độ giao điểm của đường thẳng $d$ và mặt phẳng $(P)$ bằng $1$}
{Biết điểm $I(a;b;c)\in d$, $a>0$ sao cho mặt cầu $(S)$ có tâm $I$ bán kính $R=2\sqrt{2}$ tiếp xúc với $(P)$. Khi đó $a+b+c=9$}
{\True Gọi $\Delta$ là đường thẳng vuông góc với mặt phẳng $(P)$ sao cho khoảng cách từ điểm $A$ đến $\Delta$ bằng $1$. Khi khoảng cách từ $B$ đến $\Delta$ đạt giá trị nhỏ nhất thì $\Delta$ đi qua điểm $M\left(\dfrac{5}{3};\dfrac{5}{3};\dfrac{5}{3}\right)$}
\loigiai{
\begin{itemchoice}
\itemch \textbf{Đúng}.\\
Thay tọa độ điểm $A\left(1; 2; 1\right)$ vào $(P)$ ta được $1 - 1 = 0$.\\
Suy ra $A$ thuộc mặt phẳng $(P)$.
\itemch \textbf{Sai}.\\
Gọi $M$ là giao điểm của đường thẳng $d$ và mặt phẳng $(P)$ suy ra $M\left(1+2t; t; t \right)$ (do $M \in d$).\\
Mà $M \in (P)$ suy ra $1 + 2t - t = 0 \Leftrightarrow t = -1$.\\
Suy ra $M\left(-1; -1; -1 \right)$.\\
Vậy hoành độ giao điểm của đường thẳng $d$ và mặt phẳng $(P)$ bằng $-1$.
\itemch \textbf{Sai}.\\
Do mặt cầu $(S)$ tiếp xúc với $(P)$ nên ta có $\mathrm{d} \left(I, (P) \right) = R = 2\sqrt{2}$.\\
Mà $I \in d$ nên $I\left(1+2t; t; t \right)$.\\
Từ đó ta được $\dfrac{\left|1+2t - t \right|}{\sqrt{1^2 + (-1)^2}} = 2\sqrt{2} \Leftrightarrow \hoac{&t=3\\&t = -5.}$\\
Với $t = -5$ ta được $I\left(-9; -5; -5 \right)$ không thỏa mãn.\\
Với $t = 3$ ta được $I\left(7; 3; 3 \right)$ thỏa mãn.\\
Suy ra $a + b + c = 7 + 3 + 3 = 13$.
\itemch \textbf{Đúng}.\\
Gọi $d_1$ là đường thẳng đi qua $A$ và vuông góc với $P$, ta được $d_1 \colon \heva{&x = 1+t\\&y=2\\&z=1-t.}$\\
Gọi $N$ là hình chiếu vuông góc của $B$ lên $d_1$.\\
Ta có $N \left(1 + t; 2; 1-t \right)$ và $\overrightarrow{BN} = \left(t-1; 1; -t-3 \right)$.\\
Mặt khác $BN \perp d_1$ suy ra $\overrightarrow{BN} \cdot \overrightarrow{AN} = 0 \Leftrightarrow\left(t - 1 \right) \cdot 1 + 1 \cdot 0 + \left(-t - 3 \right) \cdot 1 = 0 \Leftrightarrow t = -1$.\\
Khi đó $N \left(0;2;2 \right)$ và $\overrightarrow{BN} = \left(-2;1;-2 \right)$.\\
Gọi $K$ là hình chiếu vuông góc của $B$ lên $\Delta$. \\
Ta có $BK + KN \geq NB$ suy ra $BK \geq NB - NK$.\\
Dấu \lq\lq $=$\rq\rq \, xảy ra khi $K$, $N$, $B$ thẳng hàng ($K$ nằm giữa $B$, $N$), suy ra $K = \Delta \cap BN$.\\
Ta có $BN \colon \heva{&x = 2 + 2t\\&y=1-t\\&z=4+2t}$ suy ra $K\left(2+2t; 1-t; 4+2t \right)$.\\
Suy ra $\overrightarrow{BK} = \left(2t; -t; 2t\right)$ và $\overrightarrow{KN} = \left(-2-2t;1+t;-2-2t \right)$.\\
Mặt khác $NK = 1$ và $BN = \sqrt{(-2)^2 + 1^2 + (-2)^2} = 3$.\\
Ta có $\overrightarrow{BN} = k\overrightarrow{KN} \Rightarrow \dfrac{2t}{-2-2t} = \dfrac{-t}{1+t} = \dfrac{2t}{-2-2t} = \dfrac{2}{1}$.\\
Từ đó ta được $t = \dfrac{-2}{3}$ suy ra $K\left(\dfrac{2}{3}; \dfrac{5}{3}; \dfrac{8}{3} \right)$.\\
Suy ra $\Delta \colon \heva{&x = \dfrac{2}{3} + t\\&y = \dfrac{5}{3}\\&z = \dfrac{8}{3} - t.}$\\
Dễ thấy với $t = 1$ ta được điểm $\left(\dfrac{5}{3}; \dfrac{5}{3}; \dfrac{5}{3} \right)$ thuộc đường thẳng $\Delta$.
\end{itemchoice}
}
\end{ex}

\begin{ex}%[2H5V1-7]
\immini{Một mái nhà hình tròn được đặt trên 3 cây cột trụ. Các cây cột trụ vuông góc với mặt sàn nhà phẳng và có độ cao lần lượt là $8$m, $9$m, $10$m. Ba chân cột là ba đỉnh của một tam giác đều trên mặt sàn nhà với cạnh dài $8$m. Chọn hệ trục tọa độ như hình vẽ, với $B$ thuộc tia $Ox$, $C$ thuộc tia $Oy$, tia $Oz$ cùng hướng với véc-tơ $\vv{AA'}$; gốc tọa độ $O$ trùng với trung điểm của $AC$ và mỗi đơn vị trên trục có độ dài 1 mét (xem hình vẽ).}{\begin{tikzpicture}[>=stealth,line join=round,line cap=round,font=\footnotesize,scale=1]
\coordinate[label=center:$I$] (I)at(0,2.3);
\begin{scope}[rotate=-18]
\draw[red,fill=blue!30] (I) ellipse (2cm and 1.2cm);
\end{scope}
\coordinate[label=below right:$O$] (O)at(0,0);
\draw[dashed,black]  (-1.4,3.2)coordinate(A')--(1.5,1.4)coordinate(C')--(-0.8,1.8)coordinate(B')--(A') (-1.4,0)coordinate(A)--(1.5,0)coordinate(C) (0,2.33)coordinate(T)--(O)--(-0.8,-0.8)coordinate(B) ($(B')!2/3!(T)$)coordinate(I)--($(B)!2/3!(O)$)coordinate(J);
\draw (-1.8,0)--(A)--(B)--(C);
\draw[->] (1.5,0)--(2,0)node[below]{$y$};
\draw[->] (0,2.33)--(0,4)node[right]{$z$};
\draw[->] (-0.8,-0.8)--(-1.2,-1.2)node[right]{$x$};
\draw[line width=3pt,orange!70!black] (-1.4,0.05)--(-1.4,3.15) (1.5,0.05)--(1.5,1.35) (-0.8,-0.75)--(-0.8,1.75);
\foreach \diem in {T,I,J,O} \fill (\diem)circle(1pt);
\foreach \diem/\vitri in {A/below left,B/left,C/below,A'/above,B'/left,C'/right} \fill (\diem)circle(1pt)node[\vitri]{$\diem$};
\end{tikzpicture}}
\choiceTF
{\True Tọa độ các điểm $A'(0;-4; 10)$, $B'(4\sqrt{3}; 0; 9)$, $C'(0; 4; 8)$}
{\True Phương trình mặt phẳng $\left(A' B' C'\right)$ là $y+4z-36=0$}
{Tọa độ các điểm $A'(0;-4; 10)$, $B'(4\sqrt{3}; 0; 9)$, $C'(0; 4; 8)$}
{\True Phương trình mặt phẳng $\left(A' B' C'\right)$ là $y+4z-36=0$}
\loigiai{
\begin{itemchoice}
\itemch Tọa độ các điểm $A(0;-4; 0)$, $B(4\sqrt{3};0; 0)$, $C(0;4; 0)$, $A'(0;-4; 10)$, $B'(4\sqrt{3}; 0; 9)$, $C'(0; 4; 8)$
\itemch Ta có $\vv{A'B'}=\left(4\sqrt{3};4;-1\right)$; $\vv{A'C'}=\left(-4\sqrt{3};4;-1\right)$.\\
Véc-tơ pháp tuyến của mặt phẳng $(A'B'C')$ là $\vv{n}=\left[\vv{A'B'},\vv{A'C'}\right]=\left(0;8\sqrt{3};32\sqrt{3}\right)=8\sqrt{3}\left(0;1;4\right)$.\\
Phương trình mặt phẳng $(A'B'C')$ là $y+z-36=0$.
\itemch Vec-tơ pháp tuyến của mặt phẳng $(ABC)$ là $\vv{k}=(0;0;1)$.\\
Khi đó $\cos\left((ABC),(A'B'C')\right)=\dfrac{|5|}{\sqrt{4^2+1^2}}=\dfrac{4}{\sqrt{17}}\Rightarrow \left((ABC),(A'B'C')\right)\approx 14^\circ$.\\
Vậy độ dốc của mái khoảng $14^\circ$, mái nhà trên không đạt tiêu chuẩn.
\itemch Gọi $I(a;b;c)$. Suy ra
$\heva{&b+c=36\\&a^2+(b+4)^2+(c-10)^2=\left(a-4\sqrt{3}\right)^2+b^2+(c-9)^2\\&a^2+(b+4)^2+(c-10)^2=a^2+(b-4)^2+(c-8)^2}\Leftrightarrow \heva{&a=\sqrt{5}\\&b=0\\&c=9.}$\\
Vậy $I(\sqrt{5};0;9)$, điểm $I$ cách mặt sàn khoảng $9$ mét.
\end{itemchoice}

}
\end{ex}

\begin{ex}%[2H5V1-7]%[TEX ĐỀ MOON 2025]%[Nguyễn Văn Hiệp]
\immini[thm]
{
Hình minh hoạ sơ đồ một ngôi nhà trong hệ trục tọa độ $Oxyz$, trong đó nền nhà, bốn bức tường và hai mái nhà đều là hình chữ nhật. Xét tính đúng sai của các mệnh đề sau
\choiceTF
{\True Tọa độ của điểm $A$ là $(4;0;0)$}
{Tọa độ của vectơ $\overrightarrow{AH}$ là $(4;5;3)$}
{Tích vô hướng của $\overrightarrow{AH}$ và $\overrightarrow{AF}$ bằng $3$}
{\True Góc dốc của mái nhà, tức là số đo của góc nhị diện có cạnh là đường thẳng $FG$, hai mặt lần lượt là $(FGQP)$ và $(FGHE)$ bằng $26{,}6^{\circ}$ (làm tròn kết quả đến hàng phần mười của độ)}
}
{
\begin{tikzpicture}[font=\footnotesize, line join=round, line cap=round, >=stealth, scale=0.7]
\def\a{3}
\def\b{5}
\def\h{3}
\path (0:0) coordinate (C)
++(0:\a) coordinate (B)
++(-160:\b) coordinate (O)
($(O)+(B)-(C)$) coordinate (A)
($(O)+(90:\h)$) coordinate (E)
($(B)+(90:\h)$) coordinate (G)
($(C)+(90:\h)$) coordinate (H)
($(A)+(90:\h)$) coordinate (F)
($(A)+(0:1)$) coordinate (x)
($(H)+(35:2)$) coordinate (Q)
($(E)+(35:2)$) coordinate (P)
($(E)+(90:1)$) coordinate (z)
($(O)!1.3!(C)$) coordinate (y);
\draw[dashed] (G)--(H)--(C)--(B) (C)--(O);
\draw[] (G)--(Q)--(H)--(E)--(F)--(G)--(B)--(A)--(O)--(E) (F)--(A) (F)--(P)--(E) (P)--(Q);
\draw [->] (A)--(x);
\draw [->] (E)--(z);
\draw [->,dashed] (C)--(y);
\draw [] (Q)node[above]{$Q(2; 5; 4)$} (G)node[right]{$G(4; 5; 3)$} (B)node[right]{$B(4; 5; 0)$} (P)node[right]{$P(2; 0; 4)$} (O)node[below]{$O(0; 0; 0)$} (E)node[left]{$E(0; 0; 3)$} (x)node[below]{$x$} (y)node[above]{$y$} (z)node[left]{$z$};
\foreach \x/\g in {A/-90,C/180,F/0,H/90}
\fill[black] (\x) circle (1pt)
($(\g:4mm)+(\x)$) node {$\x$};
\end{tikzpicture}
}
\loigiai
{
\begin{itemchoice}
\itemch Theo hình vẽ tọa độ điểm $A(4;0;0)$.
\itemch Tọa độ $H(0;5;3)$ suy ra $\overrightarrow{AH} = (-4;5;3)$.
\itemch
Tọa độ $F(4;0;3)$ suy ra $\overrightarrow{AF} = (4;0;3)$ và $\overrightarrow{AH}\cdot \overrightarrow{AF}=-4\cdot 4 + 5\cdot 0 + 3\cdot 3 = -7 \neq 3$.
\itemch
Một vectơ chỉ phương của mặt phẳng $(FGQP)$ là\\ $\overrightarrow{n_1}=\left[\overrightarrow{FP},\overrightarrow{FG}\right]=\left[(-2;0;1),(0;5;0)\right]=(-5;0;10)$.\\
Một vectơ chỉ phương của mặt phẳng $(FGHE)$ là\\ $\overrightarrow{n_2}=\left[\overrightarrow{FG},\overrightarrow{FE}\right]=\left[(0;5;0),(-4;0;0)\right]=(0;0;20)$.\\
Đặt $\theta$ góc nhị diện cạnh $FG$,  hai mặt lần lượt là $(FGQP)$ và $(FGHE)$.\\
Ta có $\cos \theta = \dfrac{\overrightarrow{n}_1 \cdot \overrightarrow{n}_2}{\left|\overrightarrow{n}_1\right|\left|\overrightarrow{n}_2\right|}=\dfrac{2\sqrt{5}}{5}\Rightarrow \theta \approx 26{,}6^\circ$.
\end{itemchoice}
}
\end{ex}

\begin{ex}%[2H5V1-5]%[Tex đề Moon 2025]%[Nguyễn Hồng Thạch]
Trong không gian với hệ tọa độ $Oxyz$, cho hai điểm $A(1;2;-1)$, $B(2;1;0)$ và mặt phẳng $(P)\colon 2x+y-3z+1=0$. Gọi $(Q)$ là mặt phẳng chứa $A$, $B$ và vuông góc với mặt phẳng $(P)$.
\choiceTF
{\True Một vec-tơ pháp tuyến của mặt phẳng $(P)$ là $(2;1;-3)$}
{Một vec-tơ pháp tuyến của mặt phẳng $(Q)$ là $(2;1;-3)$}
{\True Phương trình mặt phẳng $(Q)$ có dạng $ax+by+cz+9=0$. Khi đó $a+b+c=-10$}
{Khoảng cách từ điểm $O$ đến mặt phẳng $(Q)$ bằng $\dfrac{15\sqrt{38}}{38}$}
\loigiai{\begin{itemchoice}
\itemch Dựa vào phương trình mặt phẳng $(P)$ ta thấy $(P)$ có một vec-tơ pháp tuyến là\\ $\overrightarrow{n}_P=(2;1;-3)$.
\itemch Ta có $\overrightarrow{AB}= (1;-1;1)$.\\
Gọi $\overrightarrow{n}_Q$ là vec-tơ pháp tuyến của mặt phẳng $(Q)$.\\
Ta có $\heva{&(P)\perp (Q)\\&(P)\ \text{chứa}\ A,\ B}\Rightarrow \heva{&\overrightarrow{n}_P\cdot \overrightarrow{n}_Q=0\\&\overrightarrow{n}_P\cdot\overrightarrow{AB}=0}\Rightarrow \overrightarrow{n}_Q=\left[\overrightarrow{n}_P,\overrightarrow{AB}\right]=(-2;-5;-3)$.\\
Ta thấy $\overrightarrow{n}_Q=(-2;-5;3)$ khồng cùng phương với $\overrightarrow{n}=(2;1;-3)$ nên $\overrightarrow{n}=(2;1;-3)$ không phải là vec-tơ pháp tuyến của mặt phẳng $(Q)$.
\itemch Mặt phẳng $(Q)$ có vec-tơ pháp tuyến là $\overrightarrow{u}_Q=(-2;-5;-3)$ và đi qua điểm $B(2;1;0)$ nên có phương trình là \[-2\cdot(x-2)-5\cdot(y-1)-3\cdot(z-0)=0\Leftrightarrow -2x-5y-3z+9=0.\]
Suy ra $a=-2$, $b=-5$, $c=-3$.\\
Vậy $a+b+c=-10$.
\itemch Khoảng cách từ  điểm $O$ đến mặt phẳng $(Q)$ là
\[\mathrm{d}(O,(Q))=\dfrac{\left|-2\cdot0-5\cdot0-2\cdot0+9\right|}{\sqrt{(-2)^2+(-5)^2+(-3)^2}}=\dfrac{9\sqrt{38}}{38}.\]
\end{itemchoice}}
\end{ex}

\begin{ex}%[2H2V2-6]
Hình vẽ sau mô tả vị trí của máy bay vào thời điểm $9$h$30$ phút. Biết các đơn vị trên hình tính theo đơn vị km.
\begin{center}
\begin{tikzpicture}[line join = round, line cap=round,>=stealth,font=\footnotesize,scale=1]
\path
(0,0) coordinate (O)
(5,0) coordinate (B)
(-3,-2) coordinate (A)
(0,4) coordinate (C)
($(A)+(B)-(O)$) coordinate (N)
($(N)+(0,4)$) coordinate (M)
;
\draw 	(O)--(B)(O)--(A) (O)--(C);

\foreach \x/\r/\p in{A/180/x,B/90/y,C/90/z}
\draw[->,line width=2pt] (O)--($(O)!1.2!(\x)$)node[scale=1.5,shift={(\r:3mm)}]{$\p$};
;

%\draw pic[draw,angle radius=7mm] {angle = M--O--C};
\draw (A) node[shift={(150:4mm)}]{$150$};
\draw (B) node[shift={(90:4mm)}]{$300$};
\draw (C) node[shift={(180:4mm)}]{$9$};
\draw (B) node[shift={(-70:4mm)}]{(Đông)};
\draw (A) node[shift={(-30:4mm)}]{(Nam)};


\draw[dashed] (O)--(N)--(B) (A)--(N)--(M)--(C);
\draw[->,line width=2pt] (O)--(M);


\draw (M) node[yshift=.4cm]{	\begin{tikzpicture}[line join = round, line cap=round,>=stealth,font=\footnotesize,scale=0.15]

\draw[cyan,line width=3pt]
(0,0)--(0.2,-0.5)coordinate (A)--(5,-1.2)coordinate (O)
(10,0)--(9.8,-0.5)coordinate (A')--(5,-1.2)
(5,-0.8) circle(0.7cm)
($(A)!0.6!(O)+(0,-0.3)$) circle(0.3cm)
($(A)!0.6!(O)+(0,-0.3)$) circle(0.2cm)
($(A')!0.6!(O)+(0,-0.3)$) circle(0.3cm)
($(A')!0.6!(O)+(0,-0.3)$) circle(0.2cm)
(7,0)--(5,-0.3) -- (3,0)
(5,-0.3)--(5,1.2)
;
\fill[cyan] (5,-0.8) circle(0.7cm);

\fill[black,xshift=-0.05cm] (4.5,-0.7) rectangle (4.6,-0.5)
(4.7,-0.7) rectangle (4.8,-0.5)
(4.9,-0.7) rectangle (5,-0.5)
(5.1,-0.7) rectangle (5.2,-0.5)
(5.3,-0.7) rectangle (5.4,-0.5)
(5.5,-0.7) rectangle (5.6,-0.5)
;


\draw[line width=2pt]
($(A')!0.75!(O)$)--($(A')!0.75!(O)+(0,-0.65)$)
($(A')!0.75!(O)+(0,-0.65)$)--++(0:0.2)
($(A')!0.75!(O)+(0,-0.65)$)--++(180:0.2)--++(-90:0.1)
($(A')!0.75!(O)+(0,-0.65)$)--++(180:0.2)--++(90:0.1)
($(A')!0.75!(O)+(0,-0.65)$)--++(0:0.2)--++(-90:0.1)
($(A')!0.75!(O)+(0,-0.65)$)--++(0:0.2)--++(90:0.1)
($(A)!0.75!(O)$)--($(A)!0.75!(O)+(0,-0.65)$)
($(A)!0.75!(O)+(0,-0.65)$)--++(0:0.2)
($(A)!0.75!(O)+(0,-0.65)$)--++(180:0.2)--++(-90:0.1)
($(A)!0.75!(O)+(0,-0.65)$)--++(180:0.2)--++(90:0.1)
($(A)!0.75!(O)+(0,-0.65)$)--++(0:0.2)--++(-90:0.1)
($(A)!0.75!(O)+(0,-0.65)$)--++(0:0.2)--++(90:0.1)
;
\end{tikzpicture}	}
;
\end{tikzpicture}
\end{center}
\choiceTF
{\True Máy bay đang ở độ cao $9$ km}
{ Tọa độ của máy bay lúc này là $(300; 150; 9)$}
{\True Phi công để máy bay ở chế độ tự động với vận tốc theo hướng đông là $750$ km/h, độ cao không đổi. Biết rằng gió thổi theo hướng đông với vận tốc $10$ m/s. Giả sử vận tốc và hướng gió không đổi thì lúc $10$h$30$ phút máy bay ở tọa độ là $(150; 1086; 9)$}
{Sau khi bay đến vị trí lúc $10$h$30$ thì máy bay bay ngược lại (hướng tây) với vận tốc $800$ km/h với độ cao không đổi, biết lúc đó trời lặng gió thì lúc $11$h máy bay ở tọa độ $(686; 150; 9)$}
\loigiai{
\begin{itemchoice}
\itemch  {\bf Đúng.}\\
Dựa vào hình vẽ ta thấy máy bay đang ở độ cao $9$ km.
\itemch  {\bf Sai.}\\
Máy bay đang ở tọa độ $(150; 300; 9)$.
\itemch  {\bf Đúng.}\\
Vận tốc gió $10$ m/s $= 36$ km/h.\\
Máy bay bay tự động trong khoảng thời gian từ $9$h$30$ đến $10$h$30$ với quãng đường $750$ km.\\
Quãng đường thực tế máy bay bay được là $750+36=786$ km.\\
Do đó tọa độ máy bay là $(150; 1\,086; 9)$.
\itemch  {\bf Sai.}\\
Quãng đường máy bay bay được trong khoảng thời gian từ $10$h$30$ đến $11$h là \[800\cdot \dfrac{1}{2}=400\,\, \text{km}.\]
Do đó tọa độ máy bay là $(150; 868; 9)$.
\end{itemchoice}
}
\end{ex}

\begin{ex}%[50 Đề minh họa tốt nghiệp 2025 - Đề 13]%[Lê Hữu Kiệt - Lê Quân]%[2H2V2-6]
Một chiếc điện thoại được đặt trên một giá đỡ có ba chân với điểm đặt $S(0;0;20)$ và các điểm chạm mặt đất của ba chân lần lượt là $A(0;-6;0)$, $B(3\sqrt{3};3;0)$, $C(-3\sqrt{3};3;0)$ (đơn vị cm). Cho biết điện thoại có trọng lượng là $2$ N và ba lực tác dụng lên giá đỡ được phân bố như hình vẽ là ba lực $\overrightarrow{F}_1$, $\overrightarrow{F}_2$, $\overrightarrow{F}_3$ có độ lớn bằng nhau và đo bằng đơn vị N.
\begin{center}
\tdplotsetmaincoords{75}{115}
\begin{tikzpicture}[font=\footnotesize, line join=round, line cap=round, >=stealth, scale=0.4, tdplot_main_coords]
\pgfmathsetmacro\bancanba{3*sqrt(3)}
\draw[->] (-7,0,0) -- (9,0,0) node[anchor=north east] {$x$};
\draw[->] (0,-7,0) -- (0,7,0) node[anchor=north west] {$y$};
\draw[->] (0,0,10.8) -- (0,0,12) node[anchor=south] {$z$};
\path
(0,0,0) coordinate (O)
(0,-6,0) coordinate (A)
(\bancanba,3,0) coordinate (B)
(-\bancanba,3,0) coordinate (C)
(0,0,9) coordinate (S);
\draw[dashed] (0,0,0) circle [radius=6] (A)--(B)--(C)--cycle (S)--(O);
\draw (S)--(A) (S)--(B) (S)--(C);
\foreach \x [count=\i from 1] in {A,B,C}{
\path ($(S)!1/3!(\x)$) coordinate (f\i);
\draw[->] (S)--(f\i)node[right=-1mm]{$\overrightarrow{F}_\i$};
}
\draw[rounded corners=1] (0,-2,8.9) rectangle (0,2,11);
\fill (0,-2,9.2) rectangle (0,-1.8,10.2);
\foreach \x/\g in {C/above right, A/above left, B/below, S/above right, O/below}{
\fill (\x) circle (3.3pt)node[\g]{$\x$};
}
\end{tikzpicture}
\end{center}
\choiceTF
{\True $\overrightarrow{SA}=(0;-6;-20)$}
{$\overrightarrow{F}_1+\overrightarrow{F}_2+\overrightarrow{F}_3=\overrightarrow{F}(0;0;2)$}
{$\left|\overrightarrow{F}_1\right|=\dfrac{1}{20}\left|\overrightarrow{SA}\right|$}
{\True Biết tọa độ của lực $\overrightarrow{F}_1=(a;b;c)$, khi đó $T=2a+5b+6c=-5$}
\loigiai{
\begin{itemchoice}
\itemch Ta có $\overrightarrow{SA}=(0;-6;-20)$.
\itemch Ta có $\overrightarrow{F}_1+\overrightarrow{F}_2+\overrightarrow{F}_3=\overrightarrow{F}$, do điện thoại có trọng lượng là $2$ N nên $\left|\overrightarrow{F}\right|=2$.\\
Lại có $\left|\overrightarrow{SO}\right|=20$, $\overrightarrow{F}$ và $\overrightarrow{SO}=(0;0;-20)$ cùng hướng nên $\overrightarrow{F}=\dfrac{1}{10}\overrightarrow{SO}$.\\
Suy ra $\overrightarrow{F}=(0;0;-2)$.
\itemch Do ba lực $\overrightarrow{F}_1$, $\overrightarrow{F}_2$, $\overrightarrow{F}_3$ có độ lớn bằng nhau nên với cùng số $k$, ta có $\overrightarrow{F}_1=k\overrightarrow{SA}$, $\overrightarrow{F}_2=k\overrightarrow{SB}$ và $\overrightarrow{F}_3=k\overrightarrow{SC}$.\\
Ta có $\overrightarrow{SB}=(3\sqrt3;3;-20)$, $\overrightarrow{SC}=(-3\sqrt3;3;-20)$. Khi đó
\[ \overrightarrow{F}_1+\overrightarrow{F}_2+\overrightarrow{F}_3=\overrightarrow{F} \Leftrightarrow \heva{& k\cdot0+k\cdot3\sqrt3+k\cdot(-3\sqrt3) =0 \\& k\cdot(-6)+k\cdot3+k\cdot3=0 \\& k\cdot(-20)+k\cdot(-20)+k\cdot(-20)=-2} \Leftrightarrow k=\dfrac{1}{30}. \]
Suy ra $\overrightarrow{F}_1=\dfrac{1}{30}\overrightarrow{SA}$.\\
Vậy $\left|\overrightarrow{F}_1\right|=\dfrac{1}{30}\left|\overrightarrow{SA}\right|$.
\itemch Ta có $\overrightarrow{F}_1=\dfrac{1}{30}\overrightarrow{SA}$ suy ra $\overrightarrow{F}_1=\left(0;-\dfrac{1}{5};-\dfrac{2}{3}\right)$. Do đó $a=0$, $b=-\dfrac{1}{5}$, $c=-\dfrac{2}{3}$.\\
Vậy $T=2a+5b+6c=-5$.
\end{itemchoice}
}
\end{ex}

\begin{ex}%[2H2V2-6]
Một máy bay đang di chuyển về phía sân bay. Tại thời điểm hiện tại, vị trí của máy bay là $B(150;150;5\,000)$ (trong đó $5\,000$ m là độ cao của máy bay so với mặt đất). Máy bay đang di chuyển thẳng tới sân bay $C(0;0;0)$ với vận tốc $700$ km/h. Xét tính đúng sai của các mệnh đề sau:
\choiceTF
{\True Phương trình tham số của đường thẳng mà máy bay di chuyển theo là $\heva{& x=150-150t \\ & y=150-150t\\ & z=5\,000-5\,000t}$}
{Khoảng cách từ vị trí hiện tại của máy bay $B$ đến sân bay $C$ xấp xỉ bằng $3\,905{,}6$ km}
{Với vận tốc trung bình của máy bay là $700$ km/h, thời gian để máy bay hạ cánh là khoảng $5{,}5$ giờ}
{Nếu hệ thống kiểm soát không lưu yêu cầu liên lạc với máy bay khi nó còn cách sân bay $40$ km thì khi máy bay ở vị trí $(6;6;200)$, nó còn cách sân bay là $40$ km}
\loigiai{
\begin{itemchoice}
\itemch
Véc-tơ chỉ phương của đường thẳng $BC$ là $\overrightarrow{BC}=(-150;-150;-5\,000)$.\\
Phương trình tham số của đường thẳng đi qua $B(150;150;5\,000)$ và nhận $\overrightarrow{BC}$ làm véc-tơ chỉ phương  là
$\heva{& x=150-150t \\ & y=150-150t \\ & z=5\,000-5\,000t}$.
\itemch
Khoảng cách từ $B$ đến $C$ là độ dài đoạn thẳng \allowdisplaybreaks
\begin{eqnarray*}
BC=\left|\overrightarrow{BC}\right|=\sqrt{(-150)^2+(-150)^2+(-5\,000)^2}\approx 5\,004{,}5\,\, (\text{m})=5{,}0\,045\,\, (\text{km}).
\end{eqnarray*}
\itemch
Vận tốc $v=700$ km/h. Quãng đường $d=BC \approx 5{,}0\,045$ km.\\
Thời gian để máy bay hạ cánh (đi hết quãng đường $BC$) là $\dfrac{5{,}0\,045}{700} \approx 0{,}00\,715=$ (giờ).\\
\itemch
Xét vị trí $\mathrm{P}\left(6;6;200\right)$. \\	Khoảng cách từ $\mathrm{P}\left(6;6;200\right)$ đến sân bay $C(0;0;0)$ là \\
$PC=\sqrt{(6-0)^2+(6-0)^2+(200-0)^2}\approx200{,}18\,\, (\text{m})= 0{,}200$ km.
\end{itemchoice}
}
\end{ex}

\begin{ex}%[2H2V2-2]%[TEX Đề Moon 2025]%[Vũ Hồng Toàn]
Trong không gian $Oxyz$, cho tam giác $ABC$ có $A(-1;2;4)$, $B(3;0;-2)$, $C(1;3;7)$. Gọi $D$ là chân đường phân giác trong của góc $A$. Xét tính đúng sai của các mệnh đề sau
\choiceTF
{\True Độ dài cạnh $AB$ là $2\sqrt{14}$}
{Trọng tâm của tam giác $ABC$ là điểm $G(1;2;3)$}
{\True Tích vô hướng $\overrightarrow{AB}\cdot \overrightarrow{AC}$ bằng $-12$}
{\True Độ dài vectơ $\overrightarrow{OD}$ bằng $\dfrac{\sqrt{205}}{3}$}
\loigiai{
\begin{itemchoice}
\itemch Ta có $AB = \sqrt{4^2 + (-2)^2 + (-6)^2} = \sqrt{16 + 4 + 36} = 2\sqrt{14}$.
\itemch Ta có $G\left(\dfrac{-1+3+1}{3};\dfrac{2+0+3}{3};\dfrac{4-2+7}{3}\right)\Rightarrow G\left(1;\dfrac{5}{3};3\right)$.
\itemch $\overrightarrow{AB}=(4;-2;-6)$ và $\overrightarrow{AC}=(2;1;3)$.\\
Vậy $\overrightarrow{AB}\cdot \overrightarrow{AC}=4\cdot 2-2\cdot 1-6\cdot 3=-12$.
\itemch
\immini{
Ta có $AB = 2\sqrt{14}$ và $AC = \sqrt{14}$. Gọi $D(x;y;z)$\\
Khi đó $\dfrac{BD}{DC}=\dfrac{AB}{AC}=\dfrac{2\sqrt{14}}{\sqrt{14}}=2$.\\
Suy ra $\overrightarrow{DB}=-2\overrightarrow{DC}$
}
{
\begin{tikzpicture}[declare function={gocc=35;a=4;b=3;}]
\path (0,0) coordinate (B)++(a,0) coordinate (C) ++(180-gocc:b) coordinate (A)
($(A)!1cm!(B)$) coordinate (AB)
($(A)!1cm!(C)$) coordinate (AC)
($(AB)!0.5!(AC)$) coordinate (At)
(intersection of A--At and B--C) coordinate (D);
\foreach\x/\y/\z in {B/A/D}{
\path pic[draw,angle radius=7pt]{ angle=\x--\y--\z};}
\foreach\x/\y/\z in {D/A/C}{
\path pic[draw,angle radius=9pt]{ angle=\x--\y--\z};}
\draw (A)--(B)--(C)--cycle --(D);
\foreach \x/\goc in {A/90,B/180,C/-90,D/-90}{
\draw[fill] (\x) circle (1pt) node[shift={(\goc:7pt)},font=\small]{$\x$};
}
\end{tikzpicture}
}
$\Rightarrow\heva{&3-x=-2(1-x)\\&-y=-2(3-y)\\&-2-x=-2(7-x)}\Rightarrow\heva{&x=\dfrac{5}{3}\\&y=2\\&z=4.}$\\
Vậy $OD=\sqrt{\left(\dfrac{5}{3}\right)^2+2^2+4^2}=\dfrac{\sqrt{205}}{3}$.
\end{itemchoice}
}
\end{ex}

\begin{ex}%[2D6V2-4]%[TEX ĐỀ MOON 2025]%[Nguyễn Cường]
Lớp 12A có $30$ học sinh, trong đó có $17$ bạn nữ còn lại là nam. Có $3$ bạn tên Hiền, trong đó có $1$ bạn nữ và $2$ bạn nam. Thầy giáo gọi ngẫu nhiên $1$ bạn lên bảng. Các mệnh đề sau đúng hay sai?
\choiceTF
{\True Xác suất để có tên Hiền là $\dfrac{1}{10}$}
{Xác suất để có tên Hiền, nhưng với điều kiện bạn đó nữ là $\dfrac{3}{17}$}
{\True Xác suất để có tên Hiền, nhưng với điều kiện bạn đó nam là $\dfrac{2}{13}$}
{Nếu thầy giáo gọi $1$ bạn có tên là Hiền lên bảng thì xác xuất để bạn đó là bạn nữ là $\dfrac{3}{17}$}
\loigiai{
\begin{itemchoice}
\itemch Gọi $A$ là biến cố \lq\lq bạn Hiền lên bảng\rq\rq.\\
Suy ra $n(A)=3$.\\
Xác suất $\mathrm{P}(A)=\dfrac{n(A)}{n(\Omega)}=\dfrac{3}{30}=\dfrac{1}{10}$.
\itemch Gọi $B$ là biến cố \lq\lq bạn nữ lên bảng\rq\rq, suy ra $\mathrm{P}(B)=\dfrac{17}{30}$.\\
$AB$ là biến cố \lq\lq bạn nữ tên Hiền lên bảng\rq\rq, suy ra $\mathrm{P}(AB)=\dfrac{1}{30}$.\\
Khi đó, xác suất để có tên Hiền, nhưng với điều kiện bạn đó nữ là $\mathrm{P}(A\mid B)=\dfrac{\mathrm{P}(AB)}{\mathrm{P}(B)}=\dfrac{1}{17}$.
\itemch Xác suất để có tên Hiền, nhưng với điều kiện bạn đó nam là $\mathrm{P}(A\mid \overline{B})=\dfrac{2}{13}$.
\itemch Áp dụng công thức Bayes ta có
\allowdisplaybreaks
\begin{eqnarray*}
\mathrm{P}(B\mid A)&=&\dfrac{\mathrm{P}(A\mid B)\cdot \mathrm{P}(B)}{\mathrm{P}(A)}\\
&=&\dfrac{\dfrac{1}{17}\cdot\dfrac{1}{30}}{\dfrac{1}{10}}\\
&=&\dfrac{1}{51}.
\end{eqnarray*}
\end{itemchoice}
}
\end{ex}

\begin{ex} %[2D6V2-3]
Nobita và Shizuka chuẩn bị đi tham quan hòn đảo Honshu trong hai ngày thứ Bảy và Chủ nhật tuần này. Ở hòn đảo Honshu này, mỗi ngày chi có nắng hoặc mưa, nếu một ngày là nắng thì khả năng xảy ra mưa ở ngày ngày tiếp theo là $20\%$, còn nếu một ngày là mưa thì khả năng ngày hôm sau vẫn mưa là $30\%$. Theo dự báo thời tiết, xác suất trời sẽ nắng vào thứ Bảy tuần này là $0{,}7$. Gọi $A$ là biến cố \lq\lq Ngày thứ Bảy tuần này trời nắng\rq\rq\, và $B$ là biến cố \lq\lq Ngày Chủ nhật tuần này trời mưa\rq\rq.
\choiceTF
{\True $P(A)=0{,}7$}
{Xác suất có điều kiện $P(\overline{B} \mid A)=0{,}77$}
{Xác suất ngày chủ nhật tuần này trời nắng là $80\%$}
{Bạn mèo máy Doraemon có thể đến được tương lai nhưng lại chỉ đến hòn đảo vào ngày Chủ nhật và báo cho Nobita biết rằng Chủ nhật tuần này trời mưa, khi đó xác suất ngày thứ 7 trời nắng là $62\%$ (làm tròn đến hàng đơn vị theo đơn vị phần trăm)}
\loigiai{
\begin{center}
\begin{tikzpicture}
\def\gocm{20}
\def\gocn{10}
\def\r{4}
\tikzset{s/.style={outer sep=0.5 mm,draw=magenta,rectangle,minimum width=2.75cm,rounded corners=1mm}}
\path(0,0)node(O){}++(\gocm:\r)node[s](A1){Nắng (A)}++(\gocn:\r)node[s](A2){Mưa (B)};
\path(A1)++({-\gocn}:\r)node[s](a2){Nắng $(\overline{B})$};
\path(O)++(-\gocm:\r)node[s](B1){Mưa $(\overline{A})$}++(\gocn:\r)node[s](B2){Mưa (B)};
\path(B1)++({-\gocn}:\r)node[s](b2){Nắng $(\overline{B})$};
\foreach \x/\y in {
O/A1,A1/A2,
O/B1,B1/B2,
A1/a2,
B1/b2}
\draw[-stealth](\x.east)--(\y.west);
\path(O)--(A1.west)node[pos=0.5,above,sloped]{$\mbox{0{,}7}$}(O)--(B1.west)node[pos=0.5,below]{$\mbox{0{,}3}$}(B1.east)--(B2.west)node[pos=0.5,above]{$\mbox{0{,}3}$}(A1.east)--(A2.west)node[pos=0.5,above]{$\mbox{0{,}2}$}
(A1.east)--(a2.west)node[pos=0.5,below,sloped]{$\mbox{0{,}8}$}
(B1.east)--(b2.west)node[pos=0.5,below,sloped]{$\mbox{0{,}7}$};
%%Node dòng trên
\path(A2)++(0,1)node{\textbf{Chủ nhật}}++(180:4)node{\textbf{Thứ bảy}};
\end{tikzpicture}
\end{center}
Ta có sơ đồ hình cây như hình vẽ.\\
Ta có $A$ là biến cố \lq\lq Ngày thứ Bảy tuần này trời nắng\rq\rq\, và $B$ là biến cố \lq\lq Ngày Chủ nhật tuần này trời mưa\lq\lq.
\begin{itemchoice}
\itemch Theo giả thiết ta có $P(A)=0{,}7$.
\itemch Ta có $ \mathrm{P}(\overline{B} \mid A)=\dfrac{ \mathrm{P}(\overline{B} \cap A)}{ \mathrm{P}(A)}=\dfrac{ \mathrm{P}(\overline{B})\cdot  \mathrm{P}(A \mid \overline{B})}{ \mathrm{P}(A)}=0{,}8$.
\itemch Theo công thức xác suất toàn phần ta có
\begin{eqnarray*}
\mathrm{P}(\overline{B}) &=& \mathrm{P}(A) \cdot \mathrm{P}(\overline{B}|A) +\mathrm{P}\left(\overline{A}\right) \cdot  \mathrm{P}\left(\overline{B}|\overline{A}\right)\\
&=& 0{,}7 \cdot 0{,}8+ 0{,}3\cdot 0{,}7= 0{,}77.
\end{eqnarray*}
\itemch Ta có $ \mathrm{P}(A\mid B)=\dfrac{ \mathrm{P}(A \cap B)}{ \mathrm{P}(B)}=\dfrac{ \mathrm{P}(A)\cdot  \mathrm{P}(B \mid A)}{ \mathrm{P}(B)}$.\\
$ \mathrm{P}(B)=\mathrm{P}(A \cap B)+\mathrm{P}(\overline{A} \cap B)=\mathrm{P}(A)\cdot\mathrm{P}(B\mid A)+\mathrm{P}(\overline{A})\cdot\mathrm{P}(B\mid \overline{A})=0{,}23$.\\
Do đó $ \mathrm{P}(A\mid B)=\dfrac{ \mathrm{P}(A \cap B)}{ \mathrm{P}(B)}=\dfrac{ \mathrm{P}(A)\cdot  \mathrm{P}(B \mid A)}{ \mathrm{P}(B)}=\dfrac{0{,}7 \cdot 0{,}2}{0{,}23}=0{,}6087\approx 61\%$.
\end{itemchoice}
}
\end{ex}

\begin{ex}%[2D6V2-2]
Chuồng I có $3$ con gà trống và $7$ con gà mái, chuồng II có $4$ con gà trống và $5$ con gà mái. Có $1$ con gà từ chuồng I sang chuồng II. Sau đó, có $1$ con gà từ chuồng II chạy ra ngoài.\\
Gọi $A$ là biến cố có $1$ con gà mái từ chuồng I sang chuồng II.\\
Gọi $B$ là biến cố một con gà từ chuồng II chạy ra ngoài là gà trống.
\choiceTF
{\True $\mathrm{P}(A)=0{,}7$}
{$\mathrm{P}(B\mid A)=0{,}5$}
{\True Xác suất để con gà từ chuồng II chạy ra ngoài là gà trống là $43\%$}
{\True Biết con gà từ chuồng II chạy ra ngoài là gà mái, xác suất để con gà từ chuồng I sang chuồng II là gà trống là $\dfrac{5}{19}$}
\loigiai{
\begin{itemchoice}
\itemch Ta có $\mathrm{P}(A)=\dfrac{\mathrm{C}_7^1}{\mathrm{C}_{10}^1}=0{,}7$.\\
Gọi $A$ là biến cố có $1$ con gà mái từ chuồng I sang chuồng II.\\
Suy ra $\mathrm{P}(A)=\dfrac{\mathrm{C}_3^1}{\mathrm{C}_{10}^1}=0{,}3$.
\itemch Ta có $\mathrm{P}(B\mid A)=\dfrac{\mathrm{C}_4^1}{\mathrm{C}_{10}^1}=0{,}4$.
\itemch Ta có $P\left(B \mid \overline{A}\right)=\dfrac{\mathrm{C}_5^1}{\mathrm{C}_{10}^1}=0{,}5$.\\
Ta có sơ đồ cây sau
\begin{center}
\begin{tikzpicture}[declare function={dai=2.5;cao=0.65;},>=stealth,font=\scriptsize]
\tikzset{nhan/.style={minimum size=19pt,font=\small,inner sep=0pt}}
\path (0,0) node[nhan] (G){\text{Gốc}}
(dai,{1.5*cao}) node[nhan] (B) {$A$}
(dai,{-1.5*cao}) node[nhan] (nB) {$\overline{A}$}
({2*dai},{3*cao}) node[nhan] (BA) {$B$}
({2*dai},{cao}) node[nhan] (BnA) {$\overline{B}$}
({2*dai},{-cao}) node[nhan] (nBA) {$B$}
({2*dai},{-3*cao}) node[nhan] (nBnA) {$\overline{B}$};

%Phần mũi tên
\draw[->] (G.0)--(B.200) node[sloped,pos=0.5,above]{$0{,}7$};
\draw[->] (G.0)--(nB.160) node[sloped,pos=0.5,below]{$0{,}3$};
\draw[->] (B.10)--(BA.190) node[sloped,pos=0.5,above]{$0{,}4$};
\draw[->] (B.10)--(BnA.170) ;
\draw[->] (nB.-10)--(nBA.190) node[sloped,pos=0.5,above]{$0{,}5$};
\draw[->] (nB.-10)--(nBnA.170) ;
\end{tikzpicture}
\end{center}
Áp dụng công thức xác suất toàn phần, ta có \[\mathrm{P}(B)=\mathrm{P}(A)\cdot \mathrm{P}(B\mid A)+\mathrm{P}\left(\overline{A}\right)\cdot \mathrm{P}\left(B\mid\overline{A}\right)=0{,}7 \cdot 0{,}4+0{,}3 \cdot 0{,}5=0{,}43=43\%.\]
\itemch
Ta có $\mathrm{P}\left(\overline{B}\right)=1-\mathrm{P}(B)=0{,}57$.\\
Suy ra \[\mathrm{P}\left(\overline{A} \mid\overline{B}\right)=\dfrac{\mathrm{P}\left(\overline{A B}\right)}{\mathrm{P}\left(\overline{B}\right)}=\dfrac{\dfrac{3}{10} \cdot\dfrac{ \mathrm{C}_5^1}{ \mathrm{C}_{10}^1}}{0,57}=\dfrac{5}{19}.\]
\end{itemchoice}
}
\end{ex}

\begin{ex}%[2D6V1-4]
Một công ty đấu thầu hai dự án. Khả năng thắng thầu của các dự án lần lượt là $0{,}4$ và $0{,}5$. Khả năng thắng thầu cả hai dự án là $0{,}3$. Gọi $A$, $B$ lần lượt là biến cố thắng thầu dự án $1$ và dự án $2$. Xét tính đúng sai của các mệnh đề sau
\choiceTF
{Hai biến cố $A$ và $B$ độc lập}
{\True Biết công ty thắng thầu dự án $1$, thì xác suất công ty thắng thầu dự án $2$ là $0{,}75$}
{Biết công ty không thắng thầu dự án $1$, thì xác suất công ty thắng thầu dự án $2$ là $\dfrac{2}{3}$}
{\True Xác suất công ty thắng thầu đúng $1$ dự án là $0{,}3$}
\loigiai{
\begin{itemchoice}
\itemch
Ta có $P(A)\cdot B(B)=0{,}4\cdot 0{,}5=0{,}2\ne 0{,}3=P(AB)$.
\itemch
Xác suất để công ty thắng thầu dự án $2$ khi đã biết thắng thầu dự án $1$ là $P(B|A)$.\\
Ta có $P(B| A)=\dfrac{P(AB)}{P(A)}=\dfrac{0{,}3}{0{,}4}=0{,}75$.
\itemch
Xác suất để công ty thắng thầu dự án $2$ khi đã biết điều kiện không thắng thầu dự án $1$ là $P(B\setminus \overline{A})=\dfrac{P(\overline{A}B)}{P(\overline{A})}$.\\
Vì hai biến cố $\overline{A}B$ và $AB$ xung khắc và $\overline{A}B\cap AB=B$ nên theo tính chất của xác suất ta có
$P(\overline{A}B)=P(B)-P(AB)$. Suy ra
\begin{eqnarray*}
P(B|\overline{A})&=&\dfrac{P(\overline{A}B)}{P(\overline{A})}=\dfrac{P(B)-P(AB)}{1-P(A)}\\
&=&\dfrac{0{,}5-0{,}3}{1-0{,}4}=\dfrac{1}{3}.
\end{eqnarray*}
\itemch
Xác suất để công ty thắng thầu đúng $1$ dự án là $P(A\overline{B})+P(\overline{A}B)$.\\
Vì hai biến cố $\overline{A}B$ và $AB$ xung khắc và $\overline{A}B\cap AB=B$ nên theo tính chất của xác suất ta có
\[P(\overline{A}B)=P(B)-P(AB)\quad(1).\]
Vì hai biến cố $A\overline{B}$ và $AB$ xung khắc và $A\overline{B}\cap AB=A$ nên theo tính chất của xác suất ta có \[P(A\overline{B})=P(A)-P(AB)\quad(2).\]
Từ $(1)$ và $(2)$ ta có
\begin{eqnarray*}
P(A\overline{B})+P(\overline{A}B)&=&P(A)-P(AB)+P(B)-P(AB)\\
&=& P(A)+P(B)-2P(AB)\\
&=& 0{,}4+0{,}5-2\cdot 0{,}3=0{,}3.
\end{eqnarray*}
\end{itemchoice}
}
\end{ex}

\begin{ex}%[2D6V1-2]
Một công ty truyền thông đấu thầu $2$ dự án.Khả năng thắng thầu của dự án $1$ là $0{,}5$ và dự án $2$ là $0{,}6$.Khả năng thắng thầu của cả $2$ dự án là $0{,}4$.Gọi $A$,$B$ lần lượt là biến cố thắng thầu dự án $1$ và dự án $2$.Xét tính đúng sai của các mệnh đề sau
\choiceTF
{\True Xác suất $\mathrm{P}\left(\overline{A}\right)=0{,}5$ và $\mathrm{P}\left(\overline{B}\right)=0{,}4$}
{\True Xác suất công ty thắng thầu đúng $1$ dự án là $0{,}3$}
{Biết công ty thắng thầu dự án $1$,xác suất công ty thắng thầu dự án $2$ là $0{,}4$}
{Biết công ty không thắng thầu dự án $1$,xác suất công ty thắng thầu dự án $2$ là $0{,}8$}
\loigiai{
Ta có $\mathrm{P}(A)=0{,}5$,$\mathrm{P}(B)=0{,}6$,$\mathrm{P}\left(A \cap B\right)=0{,}4$.
\begin{itemchoice}
\itemch
Xác suất không thắng thầu dự án $1$ là $\mathrm{P}\left(\overline{A}\right)=1-\mathrm{P}(A)=1-0{,}5=0{,}5$.\\
Xác suất không thắng thầu dự án $2$ là $\mathrm{P}\left(\overline{B}\right)=1-\mathrm{P}(B)=1-0{,}6=0{,}4$.
\itemch
Biến cố công ty thắng thầu đúng $1$ dự án là $\left(A \cap \overline{B}\right) \cup \left(\overline{A} \cap B\right)$.\\
Xác suất thắng dự án $1$ mà không thắng dự án $2$ là \allowdisplaybreaks
\begin{eqnarray*}
\mathrm{P}\left(A \cap \overline{B}\right)=\mathrm{P}(A)-\mathrm{P}\left(A \cap B\right)=0{,}5-0{,}4=0{,}1.
\end{eqnarray*}
Xác suất không thắng dự án 1 mà thắng dự án $2$ là \allowdisplaybreaks
\begin{eqnarray*}
\mathrm{P}\left(\overline{A} \cap B\right)=\mathrm{P}(B)-\mathrm{P}\left(A \cap B\right)=0{,}6-0{,}4=0{,}2.
\end{eqnarray*}
Vì hai biến cố $(A \cap \overline{B})$ và $(\overline{A} \cap B)$ xung khắc nên xác suất thắng đúng $1$ dự án là  \allowdisplaybreaks
\begin{eqnarray*}
\mathrm{P}\left((A \cap \overline{B}\right) \cup (\overline{A} \cap B))=\mathrm{P}\left(A \cap \overline{B}\right)+\mathrm{P}\left(\overline{A} \cap B\right)=0{,}1+0{,}2=0{,}3.
\end{eqnarray*}

\itemch
Xác suất công ty thắng thầu dự án $2$ biết đã thắng thầu dự án $1$ là xác suất có điều kiện
\allowdisplaybreaks
\begin{eqnarray*}
\mathrm{P}\left(B \mid A\right)=\dfrac{\mathrm{P}\left(A \cap B\right)}{\mathrm{P}(A)}=\dfrac{0{,}4}{0{,}5}=\dfrac{4}{5}=0{,}8.
\end{eqnarray*}
\itemch
Xác suất công ty thắng thầu dự án $2$ biết đã không thắng thầu dự án $1$ là xác suất có điều kiện \allowdisplaybreaks
\begin{eqnarray*}
\mathrm{P}\left(B \mid \overline{A}\right)=\dfrac{\mathrm{P}\left(\overline{A} \cap B\right)}{\mathrm{P}\left(\overline{A}\right)}=\dfrac{0{,}2}{0{,}5}=\dfrac{2}{5}=0{,}4.
\end{eqnarray*}
\end{itemchoice}
}
\end{ex}

\begin{ex}%[2D4V3-5]%[Tex đề Moon 2025]%[Nguyễn Hồng Thạch]
Cho hai hình trụ có cùng bán kính bằng $3$ được đặt lồng vào nhau sao cho trục của hai hình trụ vuông góc với nhau và cắt nhau tại $O$ (hình 1). Gọi $(H)$ là phần giao của hai hình trụ (hình 2). Chọn trục $Ox$ vuông góc với hai trục của hình trụ như hình vẽ. Cắt khối $(H)$ bởi mặt phẳng vuông góc với trục $Ox$ tại điểm có hoành độ $x$ $(-3\le x\le 3)$, ta được thiết diện có diện tích là $S(x)$.
\begin{center}
\begin{tikzpicture}[scale=1,>=stealth, font=\footnotesize, line join=round, line cap=round]
\fill (0,0)node[above left]{$O$}circle(2pt);
\begin{scope}[rotate=-10]
\fill (4,0)circle(2pt);
\fill (-4,0)circle(2pt);
\draw[dashed] (-4,0)--(4,0);
\draw (-4,1) arc(90:270:0.4 cm and 1 cm);
\draw[dashed] (-4,1) arc(90:-90:0.4 cm and 1 cm);
\draw[dashed] (4,1) arc(90:270:0.4 cm and 1 cm);
\draw (4,1) arc(90:-90:0.4 cm and 1 cm);
\draw (-4,1)--(4,1) (-4,-1)--(4,-1);
\end{scope}
\begin{scope}[rotate=-30]
\fill (0,-3)circle(2pt);
\fill (0,3)circle(2pt);
\draw[dashed] (0,3)--(0,-3);
\draw (1,3) arc(0:180:1 cm and 0.4 cm);
\draw[dashed] (1,3) arc(360:180:1 cm and 0.4 cm);
\draw[dashed] (1,-3) arc(0:180:1 cm and 0.4 cm);
\draw (1,-3) arc(360:180:1 cm and 0.4 cm);
\draw (1,-3)--(1,3) (-1,-3)--(-1,3);
\end{scope}
\begin{scope}[rotate=25]
\draw[dashed] (0,0) ellipse (1.4 cm and 0.72 cm);
\draw[dashed] (0,0) ellipse (0.2 cm and 1.21 cm);
\end{scope}
\draw (0,-3.5)node[below]{Hình 1};
% \begin{scope}[xshift=8cm,scale=2]
% \draw[->] (0,-1)--(0,1.5)node[left]{$x$};
% \fill (0,0)node[left]{$O$}circle(1pt);
% \draw[rotate=25] (-15:1.4 cm and 0.72 cm) arc(-15:195:1.4 cm and 0.72 cm);
% \draw[rotate=25,dashed] (-15:1.4 cm and 0.72 cm) arc(-15:-165:1.4 cm and 0.72 cm);
% \draw[rotate=25] (0,-1.21) arc(-90:90:0.2 cm and 1.21 cm);
% \draw[dashed,rotate=25] (0,1.21) arc(90:270:0.2 cm and 1.21 cm);
% \draw[rotate=25] (0,1.21)--(32:1.4 cm and 0.72 cm) (0,1.21)--(148:1.4 cm and 0.72 cm);
% \draw[rotate=25] (-15:0.2 cm and 1.21 cm)--(22:1.4 cm and 0.72 cm) (-15:0.2 cm and 1.21 cm)--(158:1.4 cm and 0.72 cm);
% \draw[rotate=25] (-165:1.4 cm and 0.72 cm) parabola bend (0,-1.21) (-15:1.4 cm and 0.72 cm);
% \draw (0,-1.75)node[below]{Hình 2};
% \end{scope}
\end{tikzpicture}
\end{center}
Xét tính đúng sai của các mệnh đề sau
\choiceTF
{Hình khối $(H)$ là một khối tròn xoay}
{Công thức tính thể tích khối $(H)$ là $V=\pi\displaystyle\int\limits_{-3}^{3} S^2(x) \mathrm{\,d}x$}
{Diện tích $S(x)$ được xác định bởi công thức $S(x)=2\left(9-x^2\right)$}
{\True Thể tích của khối $(H)$ bằng $144$ (đvtt)}
\loigiai{
\begin{itemchoice}
\itemch  Khối $(H)$ là phần giao của hai hình trụ, không phải khối tròn xoay.
\itemch  Thể tích là $V = \displaystyle \int_{-3}^{3} S(x)\mathrm{\,d}x$.
\itemch  Thiết diện vuông góc trục $Ox$ là hình vuông cạnh $2\sqrt{9 - x^2}$ nên diện tích là
\[
S(x) = \left(2\sqrt{9 - x^2}\right)^2 = 4(9 - x^2).
\]
\itemch  Ta có
\[
S(x) = 4(9 - x^2) \Rightarrow V = \displaystyle\int_{-3}^{3} 4(9 - x^2)\mathrm{\,d}x = 4\displaystyle\int_{-3}^{3} (9 - x^2)\mathrm{\,d}x.
\]
Do hàm chẵn nên
\[
V = 4 \cdot 2\displaystyle\int_{0}^{3} (9 - x^2)\mathrm{\,d}x = 8 \left[ 9x - \dfrac{x^3}{3} \right]\Big|_0^3 = 8(27 - 9) = 144.
\]
\end{itemchoice}
}
\end{ex}

\begin{ex}%[2D4V3-5]
Một bình nhiên liệu trên cánh máy bay phản lực được mô hình hóa bằng cách quay hình phẳng giới hạn bởi đồ thị $y=f(x)=\dfrac{3}{5}x^2\sqrt{2-ax}$ $(a\in\mathbb{R})$ và trục $Ox$ quanh trục hoành, trong đó $x$ và $y$ được đo bằng mét (như hình vẽ). Biết rằng chiếc máy bay đó có $4$ bình chứa nhiên liệu như nhau và được đổ đầy trước khi bay. Giả sử tốc độ tiêu hao nhiên liệu trên máy bay được mô phỏng bằng hàm số $h'(t)=-3t^2+120t+2000$ lít/giờ ($t$ tính theo giờ, $0\le t\le 6$).
\begin{center}
\begin{tikzpicture}[scale=1,>=stealth, font=\footnotesize, line join=round, line cap=round]
\def\xmin{-1} \def\xmax{3}
\def\ymin{-1} \def\ymax{2}
\draw[->] (\xmin,0)--(\xmax,0) node [below]{$x$};
\draw[->] (0,\ymin)--(0,\ymax) node [left]{$y$};
\node at (0,0) [below left]{$O$};
\draw (1.5,1.3)node[]{$y=\dfrac{3}{5}x^2\sqrt{2-ax}$};
\clip (\xmin+0.1,\ymin+0.1) rectangle (\xmax-0.5,\ymax-0.1);
\draw[smooth,samples=300,domain=0:2] plot(\x,{3/5*(\x)^2*(2-\x)});
\end{tikzpicture}
\end{center}

\choiceTF
{Giá trị $a=2$}
{Thể tích của nhiên liệu (lít) trên mỗi cánh máy bay được xác định bởi công thức $V=\pi\displaystyle\int\limits_{0}^{2} f^2(x) \mathrm{\,d}x$}
{\True Máy bay đó có thể chứa tối đa $9650$ lít nhiên liệu (làm tròn đến hàng đơn vị)}
{\True Máy bay đó tiêu hao hết $90\%$ năng lượng sau $3{,}91$ giờ (làm tròn đến hàng phần trăm)}
\loigiai{
\begin{itemchoice}
\itemch Ta có \begin{eqnarray*}
y(2)=0 & \Leftrightarrow& \dfrac{3}{5} \cdot 2^2 \sqrt{2-2a}=0 \\
& \Leftrightarrow& \sqrt{2-2a}=0 \Leftrightarrow 2-2 a=0 \\
& \Leftrightarrow& a=1 .
\end{eqnarray*}
\itemch Thể tích của nhiên liệu trên mỗi bình nhiên liệu được xác định bởi công thức $V=\pi \displaystyle\int\limits_0^2 f^2(x) \mathrm{d} x ~\left(\mathrm{m}^3\right)$.
\itemch Thể tích của nhiên liệu trên mỗi bình nhiêu liệu bằng
\begin{eqnarray*}
V_{1 b}&=&\pi \displaystyle\int\limits_0^2 f^2(x) \mathrm{d} x \\
& =&\pi \displaystyle\int\limits_0^2\left(\dfrac{3}{5} x^2 \sqrt{2-x}\right)^2 \mathrm{d} x \\
& =&\dfrac{96 \pi}{125}\left(\mathrm{m}^3\right) .
\end{eqnarray*}
Suy ra thể tích của 4 bình nhiên liệu bằng
\[V_{4 b}=4 \cdot \dfrac{96 \pi}{125}=\dfrac{384 \pi}{125}\left(\mathrm{m}^3\right)=3072 \pi~(\mathrm{l}) \approx 9651~(\mathrm{l}) .\]
\itemch Tốc độ tiêu hao nhiên liệu trên máy bay được mô phỏng bởi hàm số\\
$h'(t)=-3 t^2+120 t+2000$ (lít/giờ).\\
$V_{t t}=0{,}9 \cdot V_{4 b}=0{,}9 \cdot 3072 \pi=\dfrac{13824}{5} \pi$ (l) .\\
Gọi $m$ là thời gian máy bay đó sẽ tiêu hao hết $90$ năng lượng, khi đó
\begin{eqnarray*}
\displaystyle\int\limits_0^m h'(t) \mathrm{d}t=\dfrac{13824 \pi}{5}
& \Leftrightarrow& \displaystyle\int\limits_0^m\left(-3 t^2+120 t+2000\right) \mathrm{d}t=\dfrac{13824 \pi}{5} \\
& \Leftrightarrow&-t^3+60 t^2+\left.2000 t\right|_0 ^m=\dfrac{13824 \pi}{5} \\
& \Leftrightarrow&-m^3+60 m^2+2000 m=\dfrac{13824 \pi}{5} \\
& \Leftrightarrow&\hoac{
&m \approx 82{,}87 \\
&m \approx 3{,}91 \\
&m \approx-26{,}78
} \\
& \Leftrightarrow& m \approx 3{,}91 .
\end{eqnarray*}
Vậy máy bay đó sẽ tiêu hao hết $90$ năng lượng sau $3{,}91$ giờ bay.
\end{itemchoice}
}
\end{ex}

\begin{ex}%[2D4V3-3]%[TEX ĐỀ MOON 2025]%[Lê Hữu Kiệt]
Trong mặt phẳng tọa độ $Oxy$, cho hàm số $f(x)=x^2-x-6$ có đồ thị $(C)$.
\choiceTF
{\True Thể tích của vật thể tròn xoay được sinh ra khi hình phẳng giới hạn bởi đồ thị $(C)$ và trục hoành quay quanh $Ox$ là $V=\pi\displaystyle\int\limits_{-2}^{3} \left(x^2-x-6\right)^2 \mathrm{d}x$}
{Diện tích hình phẳng giới hạn bởi đồ thị $(C)$ và trục hoành là $S=\displaystyle\int\limits_{-2}^{3} \left(x^2-x-6\right) \mathrm{d}x$}
{Giả sử một vật $M$ chuyển động dọc theo một đường thẳng sao cho vận tốc của nó tại thời điểm $x$ (giây) là $f(x)=x^2-x-6$ (m/s). Khi đó độ dịch chuyển của vật $M$ trong khoảng thời gian $x\in[1;4]$ là $\dfrac{9}{2}$}
{\True Tổng quãng đường của vật $M$ ở trên đi được trong khoảng thời gian $x\in[1;4]$ là $\dfrac{61}{6}$ (m)}
\loigiai{
\begin{itemchoice}
\itemch Ta có $f(x)=0\Leftrightarrow x^2-x-6=0 \Leftrightarrow \hoac{&x=3\\&x=-2.}$\\
Khi đó thể tích của vật thể tròn xoay được sinh ra khi hình phẳng giới hạn bởi đồ thị $(C)$ và trục hoành quay quanh $Ox$ là $V=\pi\displaystyle\int\limits_{-2}^{3} \left(x^2-x-6\right)^2 \mathrm{d}x$.
\itemch Diện tích hình phẳng giới hạn bởi đồ thị $(C)$ và trục hoành là
\[S=\displaystyle\int\limits_{-2}^{3} \left|x^2-x-6\right|\mathrm{d}x=\displaystyle\int\limits_{-2}^{3}\left(-x^2+x+6\right)\mathrm{d}x.\]
\itemch Với $x=1$ ta có $f(1)=-6$, ta được điểm $A(1;-6)$.\\
Với $x=4$ ta có $f(4)=6$, ta được điểm $B(4;6)$.\\
Khi đó độ dịch chuyển của chất điểm $M$ trong khoảng thời gian $x\in[1;4]$ là
\[AB=\sqrt{(4-1)^2+(6+6)^2}=3\sqrt{17}.\]
\itemch Tổng quãng đường của một vật $M$ trong khoảng thời gian $x\in[1;4]$ là
\[\int\limits_1^4 \left|x^2-x-6\right|\mathrm{d}x=-\int\limits_1^3\left(x^2-x-6\right)\mathrm{d}x+\int\limits_3^4\left(x^2-x-6\right)\mathrm{d}x=\dfrac{61}{6}.\]
\end{itemchoice}
}
\end{ex}

\begin{ex}%[2D4V3-2]%[TEX Đề Moon 2025]%[Vũ Hồng Toàn]
Cho hàm số $y=f(x)$ có $f'(x)=2\cos^2\dfrac{x}{2}+3$, $\forall x\in \mathbb{R}$. Các mệnh đề sau đúng hay \textbf{sai}?
\choiceTF
{\True Hàm số $y=f(x)$ có dạng $f(x)=\sin x+4x+C$ với $C$ là hằng số}
{\True $\displaystyle\int\limits_{0}^{\tfrac{\pi}{2}} f(x)\mathrm{\,d}x=F\left(\dfrac{\pi}{2}\right)-F(0)$ với $F(x)$ là một nguyên hàm của $f(x)$}
{\True Nếu $f(0)=4$ thì $f\left(\dfrac{\pi}{2}\right)=2\pi+5$}
{Diện tích hình phẳng giới hạn bởi đồ thị của hai hàm số $y=f'(x)$; $y=6$ và hai đường thẳng $x=0$, $x=\dfrac{\pi}{2}$ có dạng $S=a+b\pi$ thì $a+2b=-1$}
\loigiai{
\begin{itemchoice}
\itemch $\forall x\in \mathbb{R}$ ta có $f'(x)=2\cos^2\dfrac{x}{2}+3=\left(2\cos^2\dfrac{x}{2}-1\right)+4=\cos x+4$.\\
Do đó $f(x)=\sin x+4x+C$.
\itemch Với $F(x)$ là một nguyên hàm của $f(x)$ khi đó $\displaystyle\int\limits_{0}^{\tfrac{\pi}{2}} f(x)\mathrm{\,d}x=F(x)\bigg |_0^{\tfrac{\pi}{2}}=F\left(\dfrac{\pi}{2}\right)-F(0)$.
\itemch Do $f(x)=\sin x+4x+C$. Nếu $f(0)=4\Rightarrow C=4$ thì $f(x)=\sin x+4x+4$.\\
Vậy $f\left(\dfrac{\pi}{2}\right)=\sin\dfrac{\pi}{2}+4\cdot \dfrac{\pi}{2}+4=2\pi+5$.
\itemch Ta có\\ $S=\displaystyle\int\limits_{0}^{\tfrac{\pi}{2}} \big|\cos x+4-6\big|\mathrm{\,d}x=\displaystyle\int\limits_{0}^{\tfrac{\pi}{2}} \big(2-\cos x\big)\mathrm{\,d}x=\left(2x-\sin x\right)\bigg|_0^{\tfrac{\pi}{2}}=\pi -1$.\\
Suy ra $a=-1$ và $b=1$. Vậy $a+2b=-1+2\cdot 1=1$.
\end{itemchoice}
}
\end{ex}

\begin{ex}%[2D4V2-6]
Hệ thống lọc nước bể bơi vô cùng quan trọng để nguồn nước được làm sạch thường xuyên và giữ vệ sinh cho người bơi. Trong quá trình vận hành lọc nước thì lượng nước trong bể sẽ thay đổi theo thời gian. Lượng nước trong bể giảm nếu hệ thống đang xả nước bẩn ra khỏi bể và tăng nếu hệ thống đang cấp thêm nước sạch cho bể. Biết rằng $1$ gallon gần bằng $3{,}785$ lít, dung tích của bể là $1000$ gallon và thời điểm $6$ giờ sáng bể chứa $250$ gallon nước. Hàm số $f(t)$ liên tục trên đoạn [$0;12$] biểu thị cho tốc độ thay đổi lượng nước trong bể theo thời gian $t$ giờ, từ thời điểm $6$ giờ sáng đến $6$ giờ chiều được cho bởi hàm số $
f (t)=\heva{&100 t,&&0 \le t \le 3\\
&-200 t+a,&&3 \le t \le 6\\
&100 t-900, && \le t \le 12
},(a \in\mathbb{R})$.\\
Với mốc thời gian $t=0$ tại thời điểm $6$ giờ sáng.
\choiceTF
{ Tại thời điểm $9$ giờ sáng, nước trong bể đang tăng với tốc độ $600$ gallon/giờ}
{\True Giá trị của $a=900$}
{Tốc độ thay đổi Iượng nước trong bể bằng $0$ vào lúc $11$ giờ trưa và $15$ giờ chiều}
{\True Lượng nước trong bể lớn nhất trong khoảng thời gian từ $9$ giờ sáng đến $18$ giờ chiều là $700$ gallon nước}
\loigiai{
\begin{itemchoice}
\itemch Do $t=0$ tại thời điểm $6$ giờ sáng nên tại thời điểm $9$ giờ sáng thì $t=3$.\\
Ta có $f(3)=100\cdot 3=300$ gallon/giờ.
\itemch Với $t=3$, $100t=-200t+a\Leftrightarrow-600+a=300\Leftrightarrow a=900$.
\itemch Tại $11$ giờ trưa (tương đương $t=5$), tốc độ thay đổi lượng nước trong bể là $f(5)=-200\cdot 5+900=-100$ gallon/giờ.\\
Tại $15$ giờ chiều (tương đương $t=9$), tốc độ thay đổi lượng nước trong bể là $f(9)=100\cdot 9-900=0$ gallon/giờ.
\itemch Từ $9$ giờ sáng đến $18$ giờ chiều tức là $3\le t\le 12$.\\
Từ $3\le t\le 9$, lượng nước đang giảm về $0$.\\
Lượng nước trong bể bắt đầu tăng trở lại từ $9\le t\le 12$.\\
Lượng nước trong bể từ $15$ giờ chiều đến $18$ giờ chiều là
\[250+\displaystyle\int\limits_9^{12}(100 t-900)\mathrm{\, d}t=700 .\]
\end{itemchoice}
}
\end{ex}

\begin{ex}%[2D4V2-6]%[TEX ĐỀ MOON 2025]%[Nguyễn Văn Hiệp]
Một vật được ném lên từ độ cao $300$ m với vận tốc cho bởi công thức $v(t)=-9{,}81t+29{,}43$ (m/s). Gọi $h(t)$ (m) là độ cao của vât so với mặt đất tại thời điểm $t$ (s) tính từ lúc bắt đầu ném vật. Xét tính đúng sai của các mệnh đề sau
\choiceTF
{\True Vận tốc của vật triệt tiêu tại thời điểm $t=3$ giây}
{Hàm số $h(t)=-4{,}985t^2+29{,}43t$}
{\True Vật đạt độ cao lớn nhất là $344$ (m) (làm tròn đến hàng đơn vị)}
{\True Sau $11$ giây tính từ lúc ném thì vật đó chạm đất (làm tròn đến hàng đơn vị)}
\loigiai
{
\begin{itemchoice}
\itemch  $v(t)=-9{,}81t+29{,}43=0\Rightarrow t=3$ m/s
\itemch  $h(t) = 300+\displaystyle \int\limits_{0}^t\left(-9{,}81z+29{,}43\right)\mathrm{\,d}z=-4,905t^2 + 29,43t + 300$.
\itemch Vật đạt độ cao lớn nhất khi $t =-\dfrac{b}{2a}=-\dfrac{29{,}43}{2\cdot \left(-4{,}905\right)} = 3$ s; $h_{\text{max}}=h(3) \approx 344$ m.
\itemch Vật chạm đất khi $h(t)=0\Leftrightarrow -4,905t^2 + 29,43t + 300 = 0$ giải được $t \approx 11$ s.
\end{itemchoice}
}
\end{ex}

\begin{ex}%[2D4V2-6]
Một ô tô đang chạy với tốc độ $108$ km/h thì người lái xe bất ngờ phát hiện chướng ngại vật trên đường. Người lái xe phản ứng một giây sau đó bằng cách đạp phanh khẩn cấp. Kể từ thời điểm này, ô tô chuyển động chậm dần đều với tốc độ $v(t)=-10t+30$ (m/s), trong đó $t$ là thời gian tính bằng giây kể từ lúc đạp phanh. Gọi $s(t)$ là quãng đường xe ô tô đi được trong $t$ (s) kể từ lúc đạp phanh. Xét tính đúng sai của các mệnh đề sau:
\choiceTF
{\True Công thức biểu diễn hàm số $s(t)=-5t^2+30t$ (m)}
{Thời gian kể từ lúc đạp phanh đến khi xe ô tô dừng hẳn là $6$ giây}
{\True Sau $3$ giây kể từ lúc đạp phanh,quãng đường xe ô tô di chuyển được là $45$ (m)}
{Quãng đường xe ô tô đã di chuyển kể từ lúc người lái xe phát hiện chướng ngại vật trên đường đến khi xe ô tô dừng hẳn là $120$ (m)}
\loigiai{
Đổi đơn vị: $108$ km/h $= \dfrac{108 \cdot 1000}{3600}$ m/s $= 30$ m/s.\\
Vận tốc của xe tại thời điểm bắt đầu đạp phanh ($t=0$) là $v(0) = -10(0) + 30 = 30$ (m/s).
\begin{itemchoice}
\itemch
Quãng đường $s(t)$ xe đi được kể từ lúc đạp phanh ($t=0$) là nguyên hàm của vận tốc $v(t)$.\\
$s(t) = \displaystyle\int v(t) \, \mathrm{d}t = \displaystyle\int (-10t+30) \, \mathrm{d}t = -10 \dfrac{t^2}{2} + 30t + C = -5t^2 + 30t + C$.\\
Tại $t=0$ (lúc bắt đầu đạp phanh), quãng đường đi được kể từ lúc đó là $s(0)=0$. \\
Thay $t=0$ vào biểu thức $s(t)$, ta có $s(0) = -5(0)^2 + 30(0) + C = 0 \Rightarrow C = 0$. \\
Vậy $s(t) = -5t^2 + 30t$ (m).
\itemch
Xe dừng hẳn khi vận tốc $v(t) = 0$. \\
$-10t + 30 = 0 \Leftrightarrow 10t = 30 \Leftrightarrow t = 3$ (s). \\
Vậy thời gian kể từ lúc đạp phanh đến khi xe dừng hẳn là $3$ giây.
\itemch
Quãng đường xe ô tô di chuyển được sau $3$ giây kể từ lúc đạp phanh là $s(3)$.\\
$s(3) = -5(3)^2 + 30(3) = -5 \cdot 9 + 90 = -45 + 90 = 45$ (m).
\itemch
Quá trình di chuyển gồm $2$ giai đoạn:
\begin{itemize}
\item Giai đoạn 1: Phản ứng (1 giây). Xe chạy với tốc độ không đổi $30$ m/s.
Quãng đường $s_1 = 30 \cdot 1 = 30$ (m).
\item Giai đoạn 2: Đạp phanh đến khi dừng hẳn (từ $t=0$ đến $t=3$ giây).
Quãng đường $s_2 = s(3) = 45$ (m) (tính ở câu trên).
\end{itemize}
Tổng quãng đường từ lúc phát hiện chướng ngại vật đến khi dừng hẳn là: \\
$S = s_1 + s_2 = 30 + 45 = 75$ (m).
\end{itemchoice}
}
\end{ex}

\begin{ex}%[2D4V1-6]%[TEX ĐỀ MOON 2025]%[Huỳnh Thanh Chí]
Một ô tô bắt đầu chuyển động thẳng nhanh dần đều với tốc độ $v(t)=5t$ (m/s); trong đó $t$ là thời gian tính bằng giây kể từ khi ô tô bắt đầu chuyển động. Đi được $6$ (s) người lái xe phát hiện chướng ngại vật và phanh gấp, ô tô tiếp tục chuyển động chậm dần đều với gia tốc $a=-5$ (m/s$^2$). Xét tính đúng sai của các mệnh đề sau
\choiceTF
{\True Tốc độ của ô tô tại thời điểm $10$ (s) tính từ lúc xuất phát là $10$ (m/s)}
{Quãng đường ô tô chuyển động được trong $6$ giây đầu tiên là $80$ m}
{\True Quãng đường $s$ (đơn vị: mét) mà ô tô chuyển động được kể từ lúc bắt đầu đạp phanh đến khi dừng lại được tính theo công thức $s=\displaystyle\int\limits_{0}^{6} \left(30-5t\right) \mathrm{\,d}t$}
{Quãng đường ô tô chuyển động được kể từ lúc bắt đầu chuyển động cho đến khi dừng lại là $170$ m}
\loigiai{
Gọi $v_1(t)$ là tốc độ chuyển động chậm dần đều của ô tô khi đạp phanh.\\
Ta có $v_1(t)=\int a\mathrm{\,d}t=\int -5\mathrm{\,d}t=-5t+C$.\\
Ta có vận tốc lúc ô tô đạp phanh là $v_1(0)=v(6)=5\cdot 6=30$ m/s.\\
Suy ra tốc độ chuyển động chậm dần đều khi đạp phanh là $v_1(t)=30-5t$ (m/s).
\begin{itemchoice}
\itemch Khi ô tô di chuyển được $6$ (s) thì bắt đầu đạp phanh và tiếp tục chuyển động với tốc độ $v_1(t)$ (m/s).\\
Do đó, tốc độ của ô tô tại thời điểm $10$ (s) tính từ lúc xuất phát bằng tốc độ ô tô chuyển động tại thời điểm $4$ (s) với tốc độ $v_1(t)$ (m/s).\\
Vậy $v_1(4)=30-5\cdot 4=10$ m/s.
\itemch Quãng đường ô tô chuyển động được trong $6$ giây đầu tiên là $S_1=\displaystyle\int\limits_{0}^{6} v(t)\mathrm{\,d}t=90$ m.
\itemch Quãng đường $s$ (đơn vị: mét) mà ô tô chuyển động được kể từ lúc bắt đầu đạp phanh đến khi dừng lại được tính theo công thức $s=\displaystyle\int\limits_{0}^{6} \left(30-5t\right) \mathrm{\,d}t$.
\itemch Khi ô tô dừng lại thì ta có phương trình $v_1(t)=0\Leftrightarrow t=6$ (s).\\
Quãng đường ô tô chuyển động được kể từ lúc bắt đầu chuyển động cho đến khi dừng lại là $S=S_1+S_2=30+\int\limits_{0}^{6} v_1(t)\mathrm{\,d}t=90+90=180$ m.
\end{itemchoice}
}
\end{ex}

\begin{ex}%[2D4V1-6]
Một viên muối hình cầu có đường kính $8$ cm đang tan trong nước với tốc độ giảm thể tích tại bất kỳ thời điểm nào tỷ lệ thuận với diện tích bề mặt quả cầu tại thời điểm đó. Sau $30$ giây thì viên muối tan được một nửa. Gọi $V(t)$ và $r(t)$ lần lượt là thể tích và bán kính của viên muối sau $t$ phút. Xét tính đúng sai của các mệnh đề sau
\choiceTF
{\True Thể tích của viên muối sau $t$ phút được xác định bởi công thức $V(t)=\dfrac{4}{3}\pi r^3(t)$}
{\True Tốc độ giảm thể tích của viên muối là $V'(t)=k\pi r^2(t)$ với $k$ là hằng số}
{Giá trị của $k=-4$}
{Sau $45$ giây thì thể tích của viên muối còn lại là $4{,}2$ cm$^3$ (làm tròn đến hàng phần mười)}
\loigiai{
\begin{itemchoice}
\itemch
Thể tích của viên muối sau thời gian $t$ phút là $V(t)=\dfrac{4}{3}\pi r^3(t)$.
\itemch
Ta có $r(0)=4$ cm.\\
$V(0)=\dfrac{4}{3}\pi r^3(0)=\dfrac{4}{3}\pi\cdot4^3=\dfrac{256\pi}{3}$.
\itemch
Ta có $V(t)=\dfrac{4}{3} \pi r^3(t) \Rightarrow V^{\prime}(t)=\dfrac{4}{3} \pi \cdot 3 r^2(t) \cdot r^{\prime}(t)=4 \pi \cdot r^2(t) \cdot r^{\prime}(t)$.\\
Ta lại có $V'(t)=k\pi\cdot r^2(t)$. Suy ra $r'(t)=\dfrac{k}{4}\Rightarrow r(t)=\dfrac{kt}{4}+C$.\\
Mặt khác $r(0)=4\Rightarrow C=4\Rightarrow \dfrac{kt}{4}+4$.
\allowdisplaybreaks
\begin{eqnarray*}
&&V\left(\dfrac{1}{2}\right)=\dfrac{228\pi}{3}\\
&\Leftrightarrow&
\dfrac{4}{3} \pi r^3\left(\dfrac{1}{2}\right)=\dfrac{128 \pi}{3} \\
& \Leftrightarrow& r^3\left(\dfrac{1}{2}\right)=32 \\
& \Leftrightarrow& r\left(\dfrac{1}{2}\right)=\sqrt[3]{32}\\
&\Leftrightarrow& r\left(\dfrac{1}{2}\right)=\dfrac{k \cdot \frac{1}{2}}{4}+4=\sqrt[3]{32}.
\end{eqnarray*}
Suy ra $\dfrac{k}{8}=\sqrt[3]{32}-4
\Rightarrow k=8 \sqrt[3]{32}-32 \approx-6{,}6$.
\itemch
Ta có $r\left(\dfrac{3}{4}\right)=\dfrac{\left(8\sqrt[3]{32}-32\right)\cdot\frac{3}{4}}{4}+4=\dfrac{\left(24\sqrt[3]{32}-96\right)+64}{16}$.\\
$V\left(\dfrac{3}{4}\right)=\dfrac{4}{3}\pi r^3 \left(\dfrac{3}{4}\right)=\dfrac{4}{3}\left(\dfrac{\left(24\sqrt[3]{32}-96\right)+64}{16}\right)^3\approx88{,}28$ (cm$^3$).
\end{itemchoice}
}
\end{ex}

\begin{ex}%[2D4V1-6]
Một miếng thịt sống được lấy ra khỏi ngăn đá của tủ lạnh và để trên bàn để rã đông. Nhiệt độ của miếng thị khi nó được lấy ra khỏi ngăn đá là $-4^\circ$C và sau $t$ giờ thì nhiệt độ của miếng thịt tăng với tốc độ $T'(t)=7{\mathrm{e}^{-0{,}35t}}^\circ$C/giờ. Miếng thịt này được rã đông khi nhiệt độ của nó đạt đến $10^\circ$C. Xét tính đúng sai của các mệnh đề sau
\choiceTF
{\True Sau $2$ giờ tốc độ thay đổi nhiệt độ của miếng thịt bằng $3{,}48^\circ$C/giờ (làm tròn kết quả đến hàng phần trăm)}
{Nhiệt độ của miếng thị bằng $0^\circ$C sau $43$ phút (làm tròn kết quả đến hàng đơn vị của phút)}
{Cần mất $2{,}44$ giờ để miếng thịt được rã đông (làm tròn kết quả đến hàng phần trăm của giờ)}
{Sau khi rã đông được $2$ tiếng, miếng thịt được đem đi nướng trong lò nướng. Tốc độ thay đổi nhiệt độ của miếng thịt trong lò nướng sau $t$ giờ được xác định bởi hàm số\break $L'(t)=80{\mathrm{e}^{0{,}2t}}^\circ$C/giờ. Miếng thịt được coi là chín đều nếu nhiệt độ của nó là $77^\circ$C. Thời gian để nướng chín đều miếng thị là $48$ phút (làm tròn kết quả đến hàng đơn vị của phút)}
\loigiai{
\begin{itemchoice}
\itemch
Tốc độ thay đổi nhiệt sau $2$ giờ là $T'(2)=7{\mathrm{e}^{-0{,}35\cdot2}}\approx 3{,}48^\circ$ C/h.
\itemch Nhiệt độ của miếng thịt sau $t$ giờ là
\allowdisplaybreaks
\begin{eqnarray*}
T(t)&=&\displaystyle\int T^{\prime}(t) \mathrm{\,d}t\\
&=&\displaystyle\int 7{\mathrm{e}^{-0{,}35t}} \mathrm{\,d}t\\
&=&\dfrac{-7}{0{,}35} e^{-0{,}35 t}+C \\
&=&-20\mathrm{e}^{-0{,}35 t}+C.
\end{eqnarray*}
Nhiệt độ khi lấy miếng thịt ra khỏi ngăn đá	\\
$T(0)=-4\Leftrightarrow-20 \mathrm{e}^{-0{,}35.0}+C=-4\Leftrightarrow C=16\Rightarrow T(t)=-20 \mathrm{e}^{-0{,}35 t}+16.$\\
Ta có
\allowdisplaybreaks
\begin{eqnarray*}
T=0&\Leftrightarrow&-20 \mathrm{e}^{-0{,}35 t}+16=0\\
&\Leftrightarrow&\mathrm{e}^{-0{,}35 t}=\dfrac{16}{20}=\dfrac{4}{5}\\
&\Leftrightarrow&-0{,}35 t=\ln \dfrac{4}{5} \\
&\Leftrightarrow&t=-\dfrac{1}{0{,}35} \ln \dfrac{4}{5} \approx 0,64 \text{h} \approx 38\text{p}.
\end{eqnarray*}
Vậy nhiệt độ miếng thịt bằng $0^\circ$C sau $38$ phút.
\itemch Ta có
\allowdisplaybreaks
\begin{eqnarray*}
T=10&\Leftrightarrow&-20 \mathrm{e}^{-0{,}35 t}+16=10\\
&\Leftrightarrow&\mathrm{e}^{-0{,}35 t}=\dfrac{6}{20}=\dfrac{3}{10}\\
&\Leftrightarrow&-0{,}35 t=\ln \dfrac{3}{10} \\
&\Leftrightarrow&t=-\dfrac{1}{0{,}35} \ln \dfrac{3}{10} \approx 3{,}44 \text{h}.
\end{eqnarray*}
Vậy cần $3{,}44$ giờ thì miếng thịt được rã đông.
\itemch Nhiệt độ miếng thịt sau $t$ giờ đưa vào lò vi sóng.\\
$L(t)=\displaystyle\int L^{\prime}(t) \mathrm{\,d}t=\displaystyle\int 80\mathrm{e}^{0{,}2t} \mathrm{\,d}t =400 \mathrm{e}^{0{,}2t}+C$.\\
Thời điểm miếng thịt được đưa vào lò vi sóng là
\[t_1=t+2=3{,}44+2=5{,}44\,(h).\]
Nhiệt độ miếng thịt lúc đưa vào lò vi sóng là
\[L(0)-T(5{,}44)=-20\mathrm{e}^{-0{,}35\cdot5{,}44}+16\approx13{,}0205.\]
Ta có $L(0)=13{,}0205\Leftrightarrow 400\mathrm{e}^{0{,}2\cdot0}+C=13{,}0205\Leftrightarrow C=-386{,}9795\Rightarrow L(t)=400 \mathrm{e}^{0{,}2t}-386{,}9795$.
\allowdisplaybreaks
\begin{eqnarray*}
L(t)=77&\Leftrightarrow&400 \mathrm{e}^{0{,}2t}-386{,}9795=77\\
&\Leftrightarrow&400 \mathrm{e}^{0{,}2t}=463{,}9795\\
&\Leftrightarrow& \mathrm{e}^{0{,}2t}=\dfrac{463{,}9795}{400}\\
&\Leftrightarrow& t=\dfrac{1}{0{,}2}\ln\dfrac{463{,}9795}{400}\approx0{,}74h\approx45(p).
\end{eqnarray*}
\end{itemchoice}
}
\end{ex}

\begin{ex}%[2D1V5-8]
Hai nhà máy được đặt tại các vị trí $A$ và $B$ cách nhau $8$ km. Nhà máy xử lí nước thải được đặt ở vị trí $C$ trên đường trung trực của đoạn thẳng $AB$, cách trung điểm $M$ của đoạn thẳng $AB$ một khoảng là $3$ km. Người ta muốn làm đường ống dẫn nước thải từ hai nhà máy $A$, $B$ đến nhà máy xử lí nước thải $C$ gồm các đoạn thẳng $AI$, $BI$ và $IC$, với $I$ là vị trí nằm giữa $M$ và $C$. Đặt $I M=x$ km (với $0<x<3$).
\begin{center}
\begin{tikzpicture}[>=stealth,line cap=round,line join=round]
\path(-4,0)node[left](A){$A$}
++(4,0)node[below](M){$ M$} ++(4,0)node[right](B){$ B $}
(M)++(0,2.2)node[above left](I){$I$}++(0,2.2)node[above](C){$ C $}
;
\draw(-4,0)--(4,0) (0,0)--(0,4) (-4,0)--(0,2)(4,0)--(0,2);
\end{tikzpicture}
\end{center}
\choiceTF
{$I A=I B=\sqrt{x^2+9}$ (km)}
{ Tổng độ dài đường ống được biểu diễn qua hàm số $f(x)=2 \sqrt{x^2+9}+3-x$ (km)}
{\True Tổng độ dài đường ống nhỏ nhất bằng $9,9$ (km) (làm tròn kết quả đến hàng phần chục)}
{\True Khi tổng độ dài đường ống nhỏ nhất thì góc $\widehat{AIB}=120^{\circ}$}
\loigiai{
\begin{itemchoice}
\itemch {\bf Sai.}\\
Ta có $AM=MB=\dfrac{AB}{2}=4$ (km).\\
Suy ra $IA=IB=\sqrt{x^2+16}$ (km).
\itemch {\bf Sai.}\\
Độ dài đường ống nước là $IA+IB+IC=2\sqrt{x^2+16}+3-x$.\\
Vì vậy $f(x)=2\sqrt{x^2+16}+3-x$ (km).
\itemch {\bf Đúng.}\\
Xét hàm số $f(x)=2\sqrt{x^2+16}+3-x$ với $0<x<3$.\\
Ta có $f'(x)=\dfrac{2x}{\sqrt{x^2+16}}-1=0\Leftrightarrow \sqrt{x^2+16}=2x\Rightarrow x=\dfrac{4\sqrt{3}}{3}$.\\
Bảng biến thiên
\begin{center}
\begin{tikzpicture}
\tkzTabInit[lgt=1.2,espcl=4]
{$x$/1,$f’(x)$/1,$f(x)$/2.5}
{$0$,$\dfrac{4\sqrt{3}}{3}$,$3$}
\tkzTabLine{ ,-,z,+, }
\tkzTabVar{+/$11$,-/$3+4\sqrt{3}$,+/$10$}
\end{tikzpicture}
\end{center}
Suy ra $\min\limits_{(0; 3)}f(x) =f\left(\dfrac{4\sqrt{3}}{3}\right)=3+4\sqrt{3}\approx 9{,}9$ (km).
\itemch {\bf Đúng.}\\
Ta có $\tan \widehat{AIM}=\dfrac{AM}{IM}=\dfrac{4}{\tfrac{4\sqrt{3}}{3}}=\sqrt{3}\Rightarrow \widehat{AIM}=60^\circ\Rightarrow \widehat{AIB}=120^\circ$.
\end{itemchoice}
}
\end{ex}

\begin{ex}%[2D1V5-4]%[TEX Đề Moon 2025]%[Vũ Hồng Toàn]
\immini[thm]
{
Cho hàm số $f(x)=\dfrac{ax^2+bx+c}{x+n}$ có đồ thị $(C)$ như hình vẽ bên. Xét tính đúng sai của các mệnh đề sau
\choiceTF
{\True Hàm số đã cho đồng biến trên khoảng $(0;1)$}
{\True Hàm số đã cho có hai điểm cực trị}
{Đồ thị $(C)$ có tiệm cận xiên đi qua điểm $A(-1;2)$}
{\True Phương trình $x\cdot\left|f(x)\right|=x-4$ có đúng $3$ nghiệm thực phân biệt}
}
{
\begin{tikzpicture}[scale=0.55,>=stealth, font=\footnotesize, line join=round, line cap=round]
\def\a{1} \def\b{2} \def\c{2} \def\d{1}\def\e{1} % Hệ số
\def\xmin{-6} \def\xmax{4}
\def\ymin{-6} \def\ymax{6}
\draw[->] (\xmin,0)--(\xmax,0) node [below]{$x$};
\draw[->] (0,\ymin)--(0,\ymax) node [right]{$y$};
\node at (0,0) [below right]{$O$};
\clip (\xmin+0.1,\ymin+0.1) rectangle (\xmax-0.1,\ymax-0.1);
\draw[smooth,samples=300,domain=\xmin:(-\e/\d-0.1)] plot(\x,{(\a*(\x)^2+\b*(\x)+\c)/(\d*(\x)+\e)});
\draw[smooth,samples=300,domain=(-\e/\d+0.1:\xmax)] plot(\x,{(\a*(\x)^2+\b*(\x)+\c)/(\d*(\x)+\e)});
\draw (-\e/\d,\ymin)--(-\e/\d,\ymax);
\draw[smooth,samples=300,domain=\xmin:\xmax] plot(\x,{(\a/\d)*(\x)+((\b)*(\d)-(\a)*(\e))/((\d)^2)});
\draw[dashed] (0,2)node[below left]{$2$} (-2,0)node[above,xshift=-0.15cm]{$-2$}--(-2,-2)--(0,-2)node[right]{$-2$} (-1,0)node[below right,xshift=-0.15cm]{$-1$};
\end{tikzpicture}
}
\loigiai{
Quan sát đồ thị hàm số $f(x)$ ta thấy
\begin{itemchoice}
\itemch $f(x)$ đồng biến trên các khoảng $(-\infty;-2)$ và $(0;+\infty)$. Do đó $f(x)$ đồng biến trên khoảng $(0;1)$.
\itemch Hàm số đã cho có hai điểm cực trị là $x=-2$ và $x=0$.
\itemch Đồ thị $(C)$ có tiệm cận xiên đi qua điểm $I(-1;0)$.
\itemch Dễ thấy $x=0$ không là nghiệm của phương trình đã cho.\\ Chia hai vế của phương trình cho $x\ne0$, ta được
\[\left|f(x)\right|=\dfrac{x-4}{x}.\]
Vẽ đồ thị hàm số $y=|f(x)|$ và đồ thị hàm số $y=\dfrac{x-4}{x}$ trên cùng một hệ trục tọa độ
\begin{center}
\begin{tikzpicture}[scale=0.55,>=stealth, font=\footnotesize, line join=round, line cap=round,declare function={a=1;b=2;c=2;d1=1;e1=1;xmin=-7;xmax=6;ymin=-5;ymax=7.5;f(\x)=(a*(\x)^2+b*(\x)+c)/(d1*(\x)+e1);g(\x)=((\x)-4)/(\x);} ]
\draw[->] (xmin,0)--(xmax,0) node [below]{$x$};
\draw[->] (0,ymin)--(0,ymax) node [right]{$y$};
\node at (0,0) [below right]{$O$};
\clip (xmin+0.1,ymin+0.1) rectangle (xmax-0.1,ymax-0.1);
\draw[dashed,smooth,samples=300,domain=xmin:(-e1/d1-0.1)] plot(\x,{f(\x)});
\draw[blue,smooth,samples=300,domain=xmin:-0.1] plot(\x,{g(\x)});
\draw[blue,smooth,samples=300,domain=0.1:xmax] plot(\x,{g(\x)});
\draw[red,smooth,samples=300,domain=(-e1/d1+0.1:xmax)] plot(\x,{f(\x)});
\draw (-e1/d1,ymin)--(-e1/d1,ymax) (xmin,1)--(xmax,1);
\draw[red,smooth,samples=300,domain=xmin:(-e1/d1-0.1)] plot(\x,{-f(\x)});
\draw[dashed] (0,2)node[below left]{$2$} (-2,0)node[above,xshift=-0.15cm]{$-2$}--(-2,-2)--(0,-2)node[right]{$-2$} (-1,0)node[below right,xshift=-0.15cm]{$-1$};
\end{tikzpicture}
\end{center}
Quan sát đồ thị ta thấy đồ thị hàm số $y=|f(x)|$ và đồ thị hàm số $y=\dfrac{x-4}{x}$ cắt nhau tại ba điểm phân biệt.\\ Do đó phương trình $x\cdot\left|f(x)\right|=x-4$ có đúng $3$ nghiệm thực phân biệt
\end{itemchoice}
}
\end{ex}

\begin{ex} %[2D1V3-6]
Khi thả một quả bóng từ đỉnh một toà tháp xuống, nó chạm đất sau 3 giây. Sau đó, quả bóng nảy lên trước khi chạm đất lần nữa 4 giây sau đó. Chiều cao tinh bằng mét của quả bóng so với mặt đất sau $t$ giây tuân theo một hàm số liên tục trên $[0; 7]$ như sau:
\[
H(t)=\left\{\begin{array}{lll}-5 t^2+c & \text{khi} & 0 \leq t < 3 \\-5 t^2+d t+e & \text{khi} & 3 \leq t \leq 7
\end{array}(c, d, e \in \mathbb{R}).\right.
\]
\begin{center}
\begin{tikzpicture}[scale=0.8,>=stealth, font=\footnotesize, line join=round, line cap=round]
\def\a{-0.5} \def\b{0} \def\c{4.5} % Hệ số
\def\xmin{-1} \def\xmax{9}
\def\ymin{-1} \def\ymax{6}
\draw[color=gray!50,dashed] (\xmin,\ymin) grid (\xmax,\ymax);
\draw[->] (\xmin,0)--(\xmax,0) node [below]{$t(s)$};
\draw[->] (0,\ymin)--(0,\ymax) node [left]{$H(m)$};
\node at (0,0) [below left]{$O$};
\clip (\xmin+0.1,\ymin+0.1) rectangle (\xmax-0.5,\ymax-0.1);
\draw[smooth,samples=300,domain=0:3] plot(\x,{\a*(\x)^2+\b*(\x)+\c});
\draw[smooth,samples=300,domain=3:7] plot(\x,{-0.5*(\x)^2+5*(\x)-10.5});
\fill[blue] (1,4) circle (5pt);
\fill[blue] (5,2) circle (5pt);
\draw (3,0) circle (1pt);
\draw (6,1.5)  node [above]{$H(t)$};
\draw (3,0)  node [below]{$3$};
\draw (7,0)  node [below]{$7$};
\end{tikzpicture}
\end{center}
\choiceTF
{\True $H(3)=H(7)=0$}
{Quả bóng được thả từ độ cao $40$ m}
{Giá trị của $d$ là $d=100$}
{\True Độ cao lớn nhắt mà quả bóng đạt được sau lần nảy đầu tiên là $20$ m}
\loigiai{
\begin{itemchoice}
\itemch Dựa vào đôg thị ta có $H(3)=H(7)=0$.
\itemch Quả bóng được thả tại thời điểm $t=0$, nên để tìm độ cao của quả bóng ta tìm $H(0)$.\\
Vì hàm số liên tục tại $x=3$ nên $\lim\limits_{x\to3}H(t)=H(3)\Leftrightarrow -5\cdot3^2+c=0\Leftrightarrow c=45
$.\\
Vậy $H(0)=45$. Do đó quả bóng được thả từ độ cao $45$ m.
\itemch
Ta có $\heva{&H(3)=0\\&H(7)=0}\Leftrightarrow \heva{&-5\cdot3^2+3d+e=0\\&-5\cdot7^2+7d+e=0}\Leftrightarrow \heva{&d=50\\&e=-105.}$\\
Vậy $H(t)=-5t^2+50t-105$.
\itemch
Độ cao của quả bóng sau lần nảy đầu tiên trong khoảng thời gian $t\in[3;7]$ nên được mô tả bởi $H(t)=-5t^2+50t-105$.\\
$H'(t)=-10t+50=0\Leftrightarrow t=5$ (tm).\\
Ta có $H(3)=H(7)=0$; $H(5)=20$.\\
Vậy độ cao lớn nhất của quả bóng sau lần nảy đầu tiên là $20$ m.
\end{itemchoice}

}
\end{ex}

\begin{ex}%[2D1V3-6]%[Tex đề Moon 2025]%[Nguyễn Hồng Thạch]
Trong một số trường hợp, tin đồn lan truyền và được mô hình hóa bằng hàm số\break $p(t)=\dfrac{1}{1+a\cdot\mathrm{e}^{-kt}}$, trong đó $p(t)$ là tỉ lệ dân số biết tin đồn tại thời điểm $t$ (giờ) và $a$, $k$ là hằng số dương. Giả sử $a=10$ và $k=0{,}5$. Khi đó
\choiceTF
{\True $\lim\limits_{t\to+\infty}p(t)=1$}
{Tốc độ lan truyền tin đồn là $p'(t)=\dfrac{10\mathrm{e}^{-0{,}5t}}{\left(1+10\mathrm{e}^{-0{,}5t}\right)^2}$}
{Tốc độ lan truyền tin đồn lớn nhất sau $9{,}2$ giờ (kết quả làm tròn đến hàng phần mười)}
{\True Tại thời điểm tin đồn lan truyền với tốc độ lớn nhất thì có $50\%$ dân số biết tin đồn (làm tròn kết quả đến hàng đơn vị)}
\loigiai{
\begin{itemchoice}
\itemch Vì $\lim\limits_{t \to +\infty} \mathrm{e}^{-0{,}5t} = 0 \Rightarrow \lim\limits_{t \to +\infty} p(t) = \dfrac{1}{1 + 10 \cdot 0} = 1$.
\itemch Ta có
\[
p(t) = \dfrac{1}{1 + 10\mathrm{e}^{-0{,}5t}} \Rightarrow p'(t) = \dfrac{10 \cdot 0{,}5 \mathrm{e}^{-0{,}5t}}{(1 + 10\mathrm{e}^{-0{,}5t})^2} = \dfrac{5\mathrm{e}^{-0{,}5t}}{(1 + 10\mathrm{e}^{-0{,}5t})^2}.
\]
\itemch Ta có \begin{eqnarray*}
p''(t)&=&\dfrac{-2{,}5\mathrm{e}^{-0{,}5t}\left(1+10\mathrm{e}^{-0{,}5t}\right)^2-2\left(1+10\mathrm{e}^{-0{,}5t}\right)\cdot\left(-5\mathrm{e}^{-0{,}5t}\right)\cdot 5\mathrm{e}^{-0{,}5t}}{\left(1+10\mathrm{e}^{-0{,}5t}\right)^4}\\
&=&\dfrac{-2{,5}\mathrm{e}^{-0{,}5t}+25\mathrm{e}^{-t}}{\left(1+10\mathrm{e}^{-0{,}5t}\right)^3}
\end{eqnarray*}
Suy ra \[p''(t)=0\Leftrightarrow-2{,5}\mathrm{e}^{-0{,}5t}+25\mathrm{e}^{-t}=0\Leftrightarrow \mathrm{e}^{-0{,}5t}=\dfrac{1}{10}\Leftrightarrow t=\dfrac{\ln 10}{0{,}5}\approx4{,}6.\]
Ta có 	\begin{center}
\begin{tikzpicture}[scale=0.8]
\tkzTabInit
[lgt=1.2,espcl=2.5] % tùy chọn
{$t$/1,$p''(t) $/1,$p'(t)$/2.5}
{$0$,$\tfrac{\ln10}{0{,}5}$,$+\infty$}
\tkzTabLine{,-,0,+,} %
\tkzTabVar{-/, +/,-/} %dấu mũi tên, + trên, -dưới
\end{tikzpicture}
\end{center}
Dựa vào bảng biến thiên ta thấy tốc độ lan truyền lớn nhất tại thời điểm $t\approx 4{,}6$ giờ.
\itemch Thay $t = \dfrac{\ln 10}{0{,}5}$ vào $p(t)$:
\[
p\left(\dfrac{\ln 10}{0{,}5}\right) = \dfrac{1}{1 + 10e^{-0{,}5 \cdot \tfrac{\ln 10}{0{,}5}}} = \dfrac{1}{1 + 10 \cdot \dfrac{1}{10}} = \dfrac{1}{2} = 50\%.
\]
\end{itemchoice}
}
\end{ex}

\begin{ex}%[2D1V3-6]%[TEX Đề Moon 2025]%[Võ Nguyên Thạch]
Một nhà sản xuất trung bình bán được $1\,000$ ti vi màn hình phẳng mỗi tuần với giá $14$ triệu đồng một chiếc. Một cuộc khảo sát thị trường chỉ ra rằng nếu cứ giảm giá bán $500$ nghìn đồng, số lượng ti vi bán ra sẽ tăng thêm khoảng $100$ ti vi mỗi tuần. Gọi $x$ là số ti vi bán được mỗi tuần, $p$ (triệu đồng) là giá bán của mỗi ti vi. Khi đó $p=p(x)$ được gọi là hàm cầu.
\choiceTF
{\True Hàm cầu là $p=-\dfrac{1}{200}x+19$ (triệu đồng)}
{Tổng doanh thu từ tiền bán ti vi là $200p^2+3\,800p$ (triệu đồng)}
{\True Công ty giảm giá $4{,}5$ triệu đồng cho người mua thì doanh thu của công ty sẽ lớn nhất}
{\True Nếu hàm chi phí hằng tuần là $C(x)=12\,000-3x$ (triệu đồng), trong đó $x$ là số ti vi bán ra trong tuần, nhà sản xuất nên đặt giá bán $8$ triệu đồng thì lợi nhuận là lớn nhất}
\loigiai{
\begin{itemchoice}
\itemch Theo giả thiết, tốc độ thay đổi của $x$ tỉ lệ với tốc độ thay đổi của $p$ nên hàm số $p=p(x)$ là hàm số bậc nhất.\\
Khi đó, $p(x)=ax+b$, ($a$ khác $0$).\\
Giá tiền $p_1=14$ ứng với $x_1=1\,000$; giá tiền $p_2=13{,}5$ ứng với $x_2=1\,000+100=1\,100$.\\
Do đó phương trình đường thẳng $p(x)=ax+b$ đi qua hai điểm $(1\,000;14)$ và $(1\,100;13{,}5)$.\\
Ta có hệ phương trình $\heva{&14=1\,000a+b\\&13{,}5=1\,100a+b}\Leftrightarrow \heva{&a=-\dfrac{1}{200}\\&b=19}$ (thỏa mãn).\\
Vậy hàm cầu là
\[p(x)=-\dfrac{1}{200}x+19.\]
\itemch Ta có $p=-\dfrac{1}{200}x+19\Rightarrow x=-200p+3\,00$.\\
Suy ra Tổng doanh thu từ tiền bán ti vi là
\[R(x)=px=p(-200p+3\,800)=-200p^2+3\,800p \text{ (triệu đồng).}\]
\itemch Để doanh thu là lớn nhất thì ta cần tìm $p$ sao cho $R$ đạt giá trị lớn nhất.\\
Ta có $R'=-400p+3\,800=0\Rightarrow p=\dfrac{19}{2}$.\\
Bảng biến thiên
\begin{center}
\begin{tikzpicture}
\tkzTabInit[espcl=2.5,lgt=1.5,nocadre]
{$p$/0.9,$R'(p)$/0.7,$R(x)$/2.1}
{$0$,$\dfrac{19}{2}$,$+\infty$}
\tkzTabLine{,+,0,-,}
\tkzTabVar{-/$0$,+/$18\,050$,-/$-\infty$}
\end{tikzpicture}
\end{center}
Vậy công ty nên giảm giá số tiền một chiếc ti vi là $14-\dfrac{19}{2}=4{,}5$ (triệu đồng) thì doanh thu là lớn nhất.
\itemch Doanh thu bán hàng của x sản phẩm là
\[R(x)=x\cdot p(x)=x\cdot \left(-\dfrac{1}{200}x+19\right)=-\dfrac{x^2}{200}+19x \text{ (triệu đồng).}\]
Do đó hàm số thể hiện lợi nhuận thu được khi bán $x$ sản phẩm là
\[P(x)=R(x)-C(x)=-\dfrac{x^2}{200}+19x-12\,000+3x=-\dfrac{x^2}{200}+22x-12\,000 \text{ (triệu đồng).}\]
Để lợi nhuận là lớn nhất thì $P(x)$ là lớn nhất.\\
Ta có $P'(x)=-\dfrac{x}{100}+22=0\Leftrightarrow x=2\,200$.\\
Bảng biến thiên:
\begin{center}
\begin{tikzpicture}
\tkzTabInit[espcl=2.5,lgt=1.5,nocadre]
{$x$/0.7,$P'(x)$/0.7,$P(x)$/2.1}
{$0$,$22\,$,$+\infty$}
\tkzTabLine{,+,0,-,}
\tkzTabVar{-/$0$,+/$12\,000$,-/$-\infty$}
\end{tikzpicture}
\end{center}
Vậy có $2\,200$ ti vi được bán ra thì lợi nhuận là cao nhất. Số ti vi mua tăng lên là $2\,200-1\,000=1\,200$ (chiếc).\\
Vậy cửa hàng nên đặt giá bán là $14-0{,}5\cdot \dfrac{1\,200}{100}=8 \text{ (triệu đồng)}$.
\end{itemchoice}
}
\end{ex}

\begin{ex}%[2D1V2-2]%[TEX ĐỀ MOON 2025]%[Lê Hữu Kiệt]
\immini[thm]
{Cho hàm số $y=f(x)$ có đạo hàm trên $\mathbb{R}$ và hàm số $y=f'(x)$ là hàm số bậc ba có đồ thị là đường cong trong hình vẽ.
\choiceTF
{Hàm số $y=f(x)$ đồng biến trên khoảng $(-\infty;-2)$}
{Hàm số $y=f(x)$ có hai điểm cực trị}
{$f'(2)=4$}
{\True Hàm số $g(x)=f(x)-\dfrac{1}{2}x^2+x+2024$ đồng biến trên khoảng $\left(-\dfrac{5}{2};-\dfrac{3}{2}\right)$}
}
{\begin{tikzpicture}[scale=0.8,>=stealth, font=\footnotesize, line join=round, line cap=round]
\def\a{1} \def\b{3} \def\c{0} \def\d{-4} % Hệ số
\def\xmin{-4} \def\xmax{2}
\def\ymin{-5} \def\ymax{1}
\draw[->] (\xmin,0)--(\xmax,0) node [below]{$x$};
\draw[->] (0,\ymin)--(0,\ymax) node [left]{$y$};
\node at (0,0) [above right]{$O$};
\draw (1.3,1.2)node[]{$y=f'(x)$};
\clip (\xmin+0.1,\ymin+0.1) rectangle (\xmax-0.5,\ymax-0.1);
\draw[smooth,samples=300] plot(\x,{\a*(\x)^3+\b*(\x)^2+\c*(\x)+\d});
\draw[dashed] (-3,0)node[above,xshift=-0.15cm]{$-3$}--(-3,-4)--(0,-4)node[below right,xshift=-0.1cm]{$-4$} (-2,0)node[above,xshift=-0.15cm]{$-2$} (-1,0)node[above,xshift=-0.15cm]{$-1$}--(-1,-2)--(0,-2)node[right,xshift=-0.1cm]{$-2$} (1,0)node[above right]{$1$};
\foreach \x/\y in {-3/-4, -2/0, -1/-2, 0/-4, 1/0}{\fill (\x,\y) circle (1.25pt);}
\end{tikzpicture}}
\loigiai{
\begin{itemchoice}
\itemch Ta có $f'(x)<0$, $\forall x\in (-\infty;-2)$ nên $y=f(x)$ nghịch biến trên khoảng $(-\infty;-2)$.
\itemch Ta có $y=f'(x)$ chỉ đổi dấu khi qua điểm $x=1$ nên hàm số có $1$ điểm cực trị.
\itemch Gọi $y=f'(x)=ax^3+bx^2+cx+d$ ($a\ne0$). Ta có đồ thị hàm số $y=f'(x)$ đi qua các điểm có tọa độ $(-3;-2)$, $(-2;0)$, $(-1;-2)$ và $(0;-4)$ nên ta có hệ phương trình
\[\heva{&-81a+27b-9c+d=-4\\&-8a+4b-2c+d=0\\&-a+b-c+d=-2\\&d=-4} \Leftrightarrow \heva{&a=1\\&b=3\\&c=0\\&d=-4.}\]
Suy ra $y=f'(x)=x^3+3x^2-4$.\\
Khi đó $f'(2)=16$.
\itemch Ta có $g'(x)=f'(x)-x+1=x^3+3x^2-x-3$.\\
Khi đó $g'(x)=0\Leftrightarrow x^3+3x^2-x-3=0 \Leftrightarrow \hoac{&x=-3\\&x=-1\\&x=1.}$\\
Bảng biến thiên
\begin{center}
\begin{tikzpicture}[font=\footnotesize, line join=round, line cap=round, >=stealth, scale=1]
\tkzTabInit[espcl=2.5,lgt=1.5]
{$x$/0.7,$g'(x)$/0.7,$g(x)$/2}
{$-\infty$, $-3$, $-1$, $1$, $+\infty$}
\tkzTabLine{,-,$0$,+,$0$,-,$0$,+,}
\tkzTabVar{+/, -/, +/, -/, +/}
\end{tikzpicture}
\end{center}
Suy ra hàm số $y=g(x)$ đồng biến trên các khoảng $(-3;-1)$, $(1;+\infty)$.\\
Mà $\left(-\dfrac{5}{2};-\dfrac{3}{2}\right)\subset(-3;-1)$ nên hàm số $y=g(x)$ đồng biến trên khoảng $\left(-\dfrac{5}{2};-\dfrac{3}{2}\right)$.
\end{itemchoice}
}
\end{ex}

\begin{ex}%[2D1V3-6]
\immini{Người ta muốn thiết kế một lồng nuôi cá có bề mặt hình chữ nhật bao gồm phần mặt nước có diện tích bằng $54$ m$^2$ và phần đường đi xung quanh với kích thước (đơn vị: m) như hình vẽ
\choiceTF
{\True Kích thước hình chữ nhật phần mặt nước là $(a-3)$ (m) và $(b-2)$ (m), với $a>3$, $b>2$}
{ Biểu diễn $b$ theo $a$ là $b=\dfrac{54}{a-2}+3$}
{Diện tích phần đường đi theo $a$ là $S(a)=\dfrac{54a}{a-3}+3a-5$, $(a>3)$}
{\True Diện tích phần đường đi là bé nhất bằng $42$ (m$^2$)}}{
\begin{tikzpicture}
\coordinate (A) at (0,0);
\coordinate (B) at (4,0);
\coordinate (C) at (4,3);
\coordinate (D) at (0,3);
\fill[draw=black, fill=gray!50] (A) rectangle (C) ;
\fill[draw=black, fill=gray!20] (1,0.5)--(3.5,0.5)--(3.5,2.5)--(1,2.5)--cycle;
\draw[<->] (0,3.5)--node[above]{$a$}(4,3.5);
\draw[<->] (-0.5,0)--node[left]{$b$}(-0.5,3);
\draw[<->] (0,1)--node[below]{$2$}(1,1);
\draw[<->] (3.5,1)--node[below]{$1$}(4,1);
\draw[<->] (2,0)--node[right]{$1$}(2,0.5);
\draw[<->] (2,2.5)--node[right]{$1$}(2,3);
\end{tikzpicture}
}
\loigiai{
\begin{itemchoice}
\itemch Kích thước hình chữ nhật phần mặt nước là $(a-3)$ (m) và $(b-2)$ (m), với $a>3$, $b>2$.
\itemch Diện tích phần mặt nước là $54$ (m$^2$) nên \[(a-3)(b-2)=54\Leftrightarrow b-2=\dfrac{54}{a-3}\Leftrightarrow b=\dfrac{54}{a-3}+2.\]
\itemch Diện tích cá lồng nuôi cá là \[a\cdot b=a\cdot\left(\dfrac{54}{a-3}+2\right)=\dfrac{54a}{a-3}+2a.\]
Diện tích phần đường đi là $\dfrac{54a}{a-3}+2a-54$.
\itemch Xét hàm số $f(a)=\dfrac{54a}{a-3}+2a-54\Rightarrow f'(a)=\dfrac{-162}{(a-3)^2}+2$.\\
Ta có $f'(a)=0\Leftrightarrow a=-6$ (loại) hoặc $a=12$ (thoả mãn).\\
Tính giá trị của $f(a)$ tại điểm cực trị, ta có diện tích phần đường đi nhó nhất khi $a=12$ (m) và có diện tích $f(12)=42$ (m$^2$).
\end{itemchoice}
}
\end{ex}

% \paragraph{Mức độ C}
\begin{ex}%[50 Đề minh họa tốt nghiệp 2025 - Đề 13]%[Lê Hữu Kiệt - Lê Quân]%[2D1C2-7]
Trên trục $Os$, cho hai chất điểm chuyển động có toạ độ theo thời gian $t$ (giây) lần lượt là $s_1=\sin t$ và $s_2=\sin\left(t+\dfrac{\pi}{3}\right)$ (đơn vị: mét).
\begin{center}
\begin{tikzpicture}[line join=round, line cap=round, >=stealth, scale=1]
\draw[->] (-3.2,0)--(4.5,0)node[below]{$s$};
\draw (0,0)node[above]{$O$};
\fill (-0.8,0) circle (2pt) node[above]{$s_1$} (1.5,0) circle (2pt) node[above]{$s_2$};
\foreach \x in {-3,...,4}{
\draw (\x,0.1)--(\x,-0.1)node[below]{$\x$};
}
\end{tikzpicture}
\end{center}
\choiceTF
{Tại thời điểm ban đầu hai chất điểm cách nhau một khoảng bằng $50$ cm}
{Khoảng cách giữa hai chất điểm được xác định bởi hàm số $d=s_1-s_2$ (mét)}
{\True Trong $6$ giây đầu tiên, có hai thời điểm mà vận tốc của hai chất điểm bằng nhau}
{\True Trong $6$ giây đầu tiên, khoảng cách xa nhất của hai chất điểm là $100$ cm}
\loigiai{
\begin{itemchoice}
\itemch Tại thời điểm bắt đầu thì $t=0$. Khi đó $s_1=\sin 0 = 0$; $s_2=\sin\left(0+\dfrac{\pi}{3}\right)=\dfrac{\sqrt3}{2}$.\\
Khoảng cách giữa hai chất điểm là $s_2-s_1=\dfrac{\sqrt3}{2}$ m.
\itemch Khoảng cách giữa hai chất điểm được xác định bởi hàm số $d=|s_1-s_2|$ (mét).
\itemch Vận tốc của chất điểm thứ nhất và thứ hai lần lượt là $v_1=s_1'=\cos t$ và $v_2=s_2'=\cos\left(t+\dfrac{\pi}{3}\right)$.\\
Khi hai chất điểm có vận tốc bằng nhau
\begin{eqnarray*}
&& v_1=v_2 \\
&\Leftrightarrow& \cos t = \cos\left(t+\dfrac{\pi}{3}\right) \\
&\Leftrightarrow& \hoac{&t=t+\dfrac{\pi}{3}+k2\pi \\& t=-t-\dfrac{\pi}{3}+k2\pi} \\
&\Leftrightarrow& t=-\dfrac{\pi}{6}+k\pi,\, (k\in\mathbb{Z}).
\end{eqnarray*}
Trong $6$ giây đầu tiên, tức $0\leq t \leq 6 \Leftrightarrow 0\leq -\dfrac{\pi}{6}+k\pi \leq 6 \Leftrightarrow \dfrac{1}{6} \leq k < 2{,}08$.\\
Do $k\in\mathbb{Z}$ nên $k\in\{1;2\}$.\\
Vậy trong $6$ giây đầu tiên, có hai thời điểm mà vận tốc của hai chất điểm bằng nhau.
\itemch Xét $y=s_1-s_2=\sin t - \sin\left(t+\dfrac{\pi}{3}\right)$.\\
Tập xác định $\mathscr{D}=\mathbb{R}$.\\
Ta có $y'=\cos t - \cos\left(t+\dfrac{\pi}{3}\right)$.\\
Cho $y'=0 \Leftrightarrow \cos t- \cos\left(t+\dfrac{\pi}{3}\right)=0 \Leftrightarrow t=-\dfrac{\pi}{6}+k\pi$, $k\in\mathbb{Z}$.\\
Bảng biến thiên của $y$ trên đoạn $[0;6]$
\begin{center}
\begin{tikzpicture}[font=\footnotesize, line join=round, line cap=round, >=stealth, scale=1]
\tkzTabInit[lgt=1.2,espcl=2.5,deltacl=0.6]
{$x$/1, $y'$/0.7, $y$/2}
{$0$, $\dfrac{5\pi}{6}$, $\dfrac{11\pi}{6}$, $6$}
\tkzTabLine
{, + , $0$ , - , $0$ , + , }
\tkzTabVar
{-/$-\dfrac{\sqrt3}{2}$ , +/$1$, -/$-1$, +/$-0{,}97$}
\end{tikzpicture}
\end{center}
Suy ra, khoảng cách giữa hai chất điểm là $d=|y|$ có giá trị lớn nhất là $1$ m, hay $100$ cm khi $t=\dfrac{5\pi}{6}$ và $t=\dfrac{11\pi}{6}$.\\
\end{itemchoice}
}
\end{ex}


\Closesolutionfile{ans}

\Opensolutionfile{ans}[ans/ansBTshortans]

\subsection{Câu trắc nghiệm trả lời ngắn}
% \paragraph{Mức độ N}
% \paragraph{Mức độ H}
\setcounter{ex}{0}
\begin{ex}%[1H8H5-4]%[TEX ĐỀ MOON 2025]%[Nguyễn Văn Hiệp]
Cho hình lăng trụ đứng $ABC.A'B'C'$ có đáy là tam giác đều độ dài cạnh bằng $6\sqrt{3}$. Khoảng cách giữa hai đường thẳng $AA'$ và $BC$ bằng bao nhiêu?
\shortans{$9$}
\loigiai{\immini{ \textbf{Bước 1: Xác định hình chiếu vuông góc} \\
Vì lăng trụ đứng nên $AA' \perp (ABC)$. Gọi $M$ là trung điểm $BC$, ta có $AM \perp BC$.\\
\textbf{Bước 2: Tính độ dài đường cao} \\
Tam giác $ABC$ đều cạnh $6\sqrt{3}$ có
\[
AM = \dfrac{6\sqrt{3} \cdot \sqrt{3}}{2} = 9.
\]
\textbf{Bước 3: Kết luận khoảng cách} \\
Vì $AA' \perp (ABC)$ và $AM \perp BC$ nên
\[
\mathrm{d}(AA', BC) = AM = 9.
\]}{\begin{tikzpicture}[line join=round, line cap=round,>=stealth,thick,scale=0.8,font=\scriptsize]
\def\a{4}
\def\h{4.5}
\path 	(0:0) coordinate (A)
++(0:\a) coordinate (C)
++(-150:3*\a/4) coordinate (B)
($(A)+(90:\h)$) coordinate (A')
($(B)+(90:\h)$) coordinate (B')
($(C)+(90:\h)$) coordinate (C')
($(B)!0.5!(C)$) coordinate (M)
;
\draw[dashed,thick] 	(A)--(C) (A)--(M);
\draw[thick]	(C)--(C') 	(B)--(B')	(A)--(A') (A)--(B)--(C) (A')--(B')--(C')--cycle;
\foreach \x/\g in {A/180,B/-45,C/0,A'/180,B'/-45,C'/0,M/-45}
\fill[black] 	(\x) circle (1pt)
($(\g:4mm)+(\x)$) node {$\x$};
\end{tikzpicture}
}

}
\end{ex}

\begin{ex}%[1H8H5-4]%[TEX ĐỀ MOON 2025]%[Nguyễn Cường]
Cho hình lăng trụ đứng $ABC.A'B'C'$ có $AB=5$, $AC=6$, $\widehat{A}=60^\circ$. Khoảng cách giữa hai đường thẳng $AA'$ và $BC$ (làm tròn kết quả đến hàng phần mười) bằng bao nhiêu?
\shortans{$4{,}7$}
\loigiai
{
\begin{center}
\begin{tikzpicture}[scale=1, font=\footnotesize, line join=round, line cap=round, >=stealth]
\path
(0,0)coordinate(A)++(0:4)coordinate(C)++(210:3)coordinate(B)
(A)++(90:3)coordinate(A')++(0:4)coordinate(C')++(210:3)coordinate(B')
($(B)!.4!(C)$)coordinate(H)
;
\draw (A')--(A)--(B)--(C)--(C')--(B')--(A')--(C')
(B)--(B')
;
\draw[dashed] (H)--(A)--(C);
\foreach \i/\g in {A/180,B/-90,C/0,A'/90,B'/90,C'/90,H/-90}{\draw[fill=black](\i) circle (1pt) ($(\i)+(\g:3mm)$) node[scale=1]{$\i$};}
\end{tikzpicture}
\end{center}
Gọi $AH$ là đường cao của $\triangle ABC$.\\
Ta có $\heva{&AA'\perp AH\\&BC\perp AH}\Rightarrow \mathrm{d}\big(AA',BC\big)=AH$.\\
Xét $\triangle ABC$ có
\begin{itemize}
\item $BC=\sqrt{AB^2+AC^2-2AB\cdot AC\cos\widehat{A}}=\sqrt{25+36-2\cdot 5\cdot 6\cdot \cos 60^\circ}=\sqrt{31}$.
\item $AH=\dfrac{AB\cdot AC\cdot\sin 60^\circ}{BC}=\dfrac{5\cdot 6\cdot \sin 60^\circ}{\sqrt{31}}=\dfrac{15\sqrt{93}}{31}\approx 4{,}67$.
\end{itemize}
Vậy $\mathrm{d}\big(AA',BC\big)=AH\approx 4{,}67$.
}
\end{ex}

\begin{ex}%[1H8H5-3]
Cho hình chóp $S.ABCD$ có đáy là hình vuông cạnh bằng $1$, $SA$ vuông góc với mặt phẳng $(ABCD)$ và $SA=\dfrac{\sqrt{3}}{3}$. Khoảng cách từ điểm $A$ đến mặt phẳng $(SCD)$ bằng bao nhiêu? (làm tròn kết quả đến hàng phần mười).
\shortans{0{,}5}
\loigiai{
\immini{
Ta có $\heva{& CD \perp AD \\ & CD \perp SA \\ & AD, SA \subset (SAD) \\ & AD \cap SA = \{A\}} \Rightarrow CD \perp (SAD)$. \\
Trong mặt phẳng $(SAD)$, kẻ $AH \perp SD$ tại $H$. \\
Vì $CD \perp (SAD)$ và $AH \subset (SAD)$ nên $CD \perp AH$. \\
Ta có $\heva{& AH \perp SD \\ & AH \perp CD \\ & SD, CD \subset (SCD) \\ & SD \cap CD = \{D\}} \Rightarrow AH \perp (SCD)$. \\
Do đó, khoảng cách từ điểm $A$ đến mặt phẳng $(SCD)$ là $\mathrm{d}(A, (SCD)) = AH$. \\
Xét tam giác $\triangle SAD$ vuông tại $A$, có $AD=1$ và $SA=\dfrac{\sqrt{3}}{3}$. \\
Áp dụng hệ thức lượng trong tam giác vuông $SAD$, ta có \\
$\dfrac{1}{AH^2} = \dfrac{1}{SA^2} + \dfrac{1}{AD^2} = \dfrac{1}{\left(\dfrac{\sqrt{3}}{3}\right)^2} + \dfrac{1}{1^2} = \dfrac{1}{\frac{3}{9}} + 1 = \dfrac{1}{\frac{1}{3}} + 1 = 3 + 1 = 4$. \\
$\Rightarrow AH^2 = \dfrac{1}{4} \Rightarrow AH = \dfrac{1}{2} = 0{,}5$. \\
Vậy khoảng cách từ điểm $A$ đến mặt phẳng $(SCD)$ bằng $0{,}5$.
}{
\begin{tikzpicture}[>=stealth,line join=round,line cap=round,font=\footnotesize,scale=1]
\tikzset{
pics/hinhChopTuGiac/.style  n args={5}{
code={
\tikzset{
declare function={a=4;b=2;h=3;goc=-120;}
}
\path
(0,0)coordinate (#1)+(0:a)coordinate (#2)+(goc:b)coordinate (#4)+(90:h)coordinate (#5)
($(#2)+(#4)-(#1)$)coordinate (#3)
;
}
}}
\path
(0,0)pic {hinhChopTuGiac={A}{D}{C}{B}{S}}
($(S)!.4!(D)$)coordinate (H)
pic[draw,angle radius=2mm,angle eccentricity=1.5]{right angle=A--H--D}
;
\foreach \pointo/\pointt in {S/B,S/C,S/D,B/C,C/D}{
\draw[fill=black](\pointo)--(\pointt);
}
\foreach \pointo/\pointt in {S/A,A/B,A/D,A/H}{
\draw[fill=black,dashed](\pointo)--(\pointt);
}
\foreach \point/\goc in {A/160,S/90,B/190,D/10,C/-45,H/45}{
\draw[fill=black](\point)circle(.8pt)+(\goc:2mm)node[scale=.8]{$\point$};
}
\end{tikzpicture}
}

}
\end{ex}

\begin{ex}%[1D6H4-6]%[TEX ĐỀ MOON 2025]%[Lê Hữu Kiệt]
Các khí thải gây hiệu ứng nhà kính là nguyên nhân chủ yếu làm Trái Đất nóng lên. Theo OECD (Tổ chức Hợp tác và Phát triển kinh tế Thế giới), khi nhiệt độ Trái Đất tăng lên thì tổng giá trị kinh tế toàn cầu giảm. Người ta ước tính rằng, khi nhiệt độ Trái Đất tăng thêm $2^\circ$C thì tổng giá trị kinh tế toàn cầu giảm $3\%$; còn khi nhiệt độ Trái Đất tăng thêm $5^{\circ}$C thì tổng giá trị kinh tế toàn cầu giảm $10\%$. Biết rằng, nếu nhiệt độ Trái Đất tăng thêm $t^\circ$C, tổng giá trị kinh tế toàn cầu giảm $f(t)\%$ thì $f(t)=k\cdot a^t$, trong đó $k$, $a$ là các hằng số dương. Khi nhiệt độ Trái Đất tăng thêm bao nhiêu độ C thì tổng giá trị kinh tế toàn cầu giảm đến $20\%$ (Làm tròn đến hàng phần chục)?
\shortans{$6{,}7$}
\loigiai{
Với $t=2$ thì $f(t)=3$, suy ra $3=ka^2\Leftrightarrow k=\dfrac{3}{a^2}$.\\
Với $t=5$ thì $f(t)=10$, suy ra
\[10=ka^5\Leftrightarrow 10=3a^3 \Leftrightarrow a=\left(\dfrac{10}{3}\right)^{\tfrac{1}{3}}.\]
Suy ra $k=\dfrac{3}{\left(\dfrac{10}{3}\right)^{\tfrac{2}{3}}}.$ Do đó $f(t)=\dfrac{3}{\left(\dfrac{10}{3}\right)^{\tfrac{2}{3}}}\cdot\left(\dfrac{10}{3}\right)^{\tfrac{t}{3}}=3\cdot\left(\dfrac{10}{3}\right)^{\tfrac{t-2}{3}}$.\\
Khi kinh tế toàn cầu giảm đến $20\%$, tức $f(t)=20$, ta có
\begin{eqnarray*}
&&3\cdot\left(\dfrac{10}{3}\right)^{\tfrac{t-2}{3}}=20 \\
&\Leftrightarrow& \left(\dfrac{10}{3}\right)^{\tfrac{t-2}{3}}=\dfrac{20}{3} \\
&\Leftrightarrow& \dfrac{t-2}{3}=\log_{\tfrac{10}{3}}\dfrac{20}{3} \\
&\Leftrightarrow&t=3\log_{\tfrac{10}{3}}\dfrac{20}{3}+2 \\
&\Leftrightarrow& t\approx 6{,}7.
\end{eqnarray*}
Vậy khi nhiệt độ Trái Đất tăng thêm khoảng $6{,}7$ độ C thì tổng giá trị kinh tế toàn cầu giảm đến $20\%$.
}
\end{ex}

\begin{ex}%[1C2H3-2]%[TEX ĐỀ MOON 2025]%[Huỳnh Thanh Chí]
Một người đưa thư xuất phát từ bưu điện ở vị trí $A$, các điểm cần phát thư nằm dọc các con đường cần đi qua. Biết rằng người này phải đi trên mỗi con đường ít nhất một lần (để phát được thư cho tất cả các điểm cần phát nằm dọc theo con đường đó) và cuối cùng quay lại điểm xuất phát. Độ dài các con đường như hình vẽ (đơn vị độ dài).
\begin{center}
\begin{tikzpicture}[scale=0.75,>=stealth, font=\footnotesize, line join=round, line cap=round]
\coordinate (A) at (0,0);
\coordinate (B) at (4,0);
\coordinate (C) at (6,-1.6);
\coordinate (D) at (4,-2.6);
\coordinate (E) at (0,-3);
\draw (A)node[above]{$A$}--(B)node[above]{$B$}--(C)node[right]{$C$}--(D)node[below]{$D$}--(E)node[below]{$E$}--(A) (E)--(B)--(D) ($(A)!0.5!(B)$)node[above]{$8$} ($(B)!0.5!(C)$)node[above right]{$5$} ($(D)!0.5!(C)$)node[below right]{$2$} ($(B)!0.5!(D)$)node[right]{$4$} ($(B)!0.5!(E)$)node[above left]{$10$} ($(A)!0.5!(E)$)node[right]{$6$} ($(E)!0.5!(D)$)node[below]{$9$};
\draw (A)..controls (-0.5,-1.5) and (-0.5,-1.5)..(E) (-0.5,-1.5)node[left]{$7$};
\end{tikzpicture}
\end{center}
Hỏi tổng quãng đường người đưa thư có thể đi ngắn nhất có thể là bao nhiêu?

\shortans[]{$63$}
\loigiai{
Bài toán yêu cầu tìm chu trình Euler có độ dài nhỏ nhất.\\
Tổng độ dài các cạnh của đồ thị là: $8+5+2+4+10+6+9+7 = 51$.\\
Các đỉnh có bậc lẻ là $A$ (bậc 3), $C$ (bậc 3), $D$ (bậc 3).\\
Cần tìm đường đi ngắn nhất để nối các đỉnh bậc lẻ này.\\
Ta có các cặp đường đi:
\begin{itemize}
\item $AC = AB+BC = 8+5 = 13$.
\item $AD = AB+BD = 8+4 = 12$.
\item $CD = 2$.
\end{itemize}
Chọn $CD=2$, khi đó:
\begin{itemize}
\item $AC=13$.
\item $AD=12$.
\end{itemize}
Ta nối $CD=2$ và chọn $AC,AD$. Ta thấy chỉ cần chọn cặp $CD$ thôi, như vậy đường đi là $AC=13$ hoặc $AD=12$, nhưng tổng $CD=2$. Ta ghép $A,C$ thì tạo đường đi $AC = 13$, và chọn ghép $AD$, có $AD=12$, hoặc $AD=12$, $CD = 2$ chọn $CD$ ngắn hơn, với các cạnh là: CD nối 2 cạnh lại: $CD=2$. Và $A,C,D$ bậc lẻ cần thêm cạnh $AC$ hoăc $AD$ cho chẵn cạnh. Ta có:
\begin{itemize}
\item $AC=13$.
\item $AD=12$.
\end{itemize}
Ta ưu tiên ghép $CD$ nên là $CD=2$. Ghép $A$ vào đỉnh còn lại: chọn đỉnh cách xa 1 đoạn: chọn $AD$ hay $AC$. Theo đó ta chọn cạnh $CD$ với $AD$ hoặc $AC$. Vì $AC,AD >CD$ ta loại $AD$, chọn CD. Sau đó chọn từ $A$ đi $C$ hoặc $D$. Chọn $A,D$.

Vậy, cặp cạnh cần lặp lại có tổng độ dài nhỏ nhất là: $CD+AD= 2+12 = 14$.
Tổng quãng đường người đưa thư phải đi là: $51 + (CD+AD) =51+14 = 63$.

Vậy tổng độ dài quãng đường ngắn nhất là $63$.
}
\end{ex}

\begin{ex}%[1C2H3-1]%[TEX ĐỀ MOON 2025]%[Nguyễn Văn Hiệp]
Giả sử $4$ thành phố $A$, $B$, $C$, $D$ với khoảng cách (đơn vị: km) giữa các thành phố được cho bởi bảng sau
\begin{center}
\begin{tblr}{hlines={0.6pt},vlines={0.6pt},width=0.7\linewidth,rows={abovesep=1pt,belowsep=1pt},colspec={X[1,c]X[1,c]X[1,c]X[1,c]X[1,c]}}
& $A$ & $B$ & $C$ & $D$ \\
$A$ & $0$ & $10$ & $15$ & $20$ \\
$B$ & $10$ & $0$ & $25$ & $35$ \\
$C$ & $15$ & $25$ & $0$ & $30$ \\
$D$ & $20$ & $35$ & $30$ & $0$ \\
\end{tblr}
\end{center}
Hãy tính quãng đường ngắn nhất để đi qua tất cả các thành phố đúng một lần rồi quay lại thành phố xuất phát?
\shortans{$85$}
\loigiai{
\textbf{Bước 1: Liệt kê các chu trình Hamilton} \\
Xét tất cả các hoán vị
\begin{itemize}
\item $A \rightarrow B \rightarrow C \rightarrow D \rightarrow A$: $10 + 25 + 30 + 20 = 85$ km.
\item $A \rightarrow B \rightarrow D \rightarrow C \rightarrow A$: $10 + 35 + 30 + 15 = 90$ km.
\item $A \rightarrow C \rightarrow B \rightarrow D \rightarrow A$: $15 + 25 + 35 + 20 = 95$ km.
\item $A \rightarrow C \rightarrow D \rightarrow B \rightarrow A$: $15 + 30 + 35 + 10 = 90$ km.
\item $A \rightarrow D \rightarrow B \rightarrow C \rightarrow A$: $20 + 35 + 25 + 15 = 95$ km.
\item $A \rightarrow D \rightarrow C \rightarrow B \rightarrow A$: $20 + 30 + 25 + 10 = 85$ km.
\end{itemize}
\textbf{Bước 2: Chọn chu trình tối ưu} \\
Quãng đường ngắn nhất là $85$ km.
}
\end{ex}

\begin{ex}%[50 Đề minh họa tốt nghiệp 2025 - Đề 13]%[Lê Hữu Kiệt - Lê Quân]%[1C2H3-1]
Biểu đồ thể hiện các con đường nối giữa các thị trấn (đơn vị: km). Cán bộ thanh tra xuất phát từ thị trấn $L$ đi kiểm tra tất cả các tuyến đường nối giữa các thị trấn $M$, $N$, $O$ và quay lại $L$. Chiều dài quãng đường tối thiểu thanh tra cần phải đi là bao nhiêu km?
\begin{center}
\begin{tikzpicture}[font=\footnotesize, line join=round, line cap=round, >=stealth, scale=1]
\def\bankinh{2}
\path (0,0) coordinate (O) (-\bankinh,0) coordinate (L) (\bankinh,0) coordinate (N) (60:\bankinh) coordinate (M);
\draw
(L) arc(180:60:\bankinh) node[pos=0.5, above]{$16$}
(M) arc(60:0:\bankinh) node[pos=0.5, above]{$8$}
(L) to[bend right=60] node[pos=0.5, below]{$15$} (N)
(L) to[bend right=40] node[pos=0.5, below]{$9$} (O)
(L) to[bend left=40] node[pos=0.5, above]{$7$} (O)
(O) to[bend right=40] node[pos=0.5, below]{$8$} (N)
(O) to[bend left=40] node[pos=0.5, above]{$6$} (N)
;
\foreach \x/\g in {O/90, L/180, N/0, M/60}{
\fill (\x) circle (2pt)+(\g:0.3)node{$\x$};
}
\end{tikzpicture}
\end{center}
\par\shortans{$37$}
\loigiai{
Để chiều dài quãng đường là tối thiểu, cán bộ thanh tra cần xuất phát từ $L$, đi qua mỗi thị trấn $M$, $N$, $O$ đúng một lần trước khi qua lại $L$ và ưu tiên chọn con đường ngắn hơn trong hai con đường nối hai thị trấn.\\
Ta có quãng đường tối ưu là $L\rightarrow M \rightarrow N \rightarrow O \rightarrow L$ với độ dài là $16+8+6+7=37$ km.
}
\end{ex}

\begin{ex}%[1C2H3-1]%[TEX ĐỀ MOON 2025]%[Nguyễn Cường]
\immini[thm]
{
Công ty giao hàng nhanh có $4$ kho hàng $A$, $B$, $C$ và $D$. Quản lý muốn lên kế hoạch cho xe giao hàng đi qua tất cả các kho hàng để lấy hàng và quay lại kho hàng ban đầu, với điều kiện là mỗi kho hàng chỉ ghé qua một lần. Khoảng cách giữa các kho hàng (km) được mô tả trong hình bên. Quãng đường ngắn nhất để xe giao hàng hoàn thành việc lấy hàng ở các kho và quay trở lại kho hàng ban đầu là bao nhiêu?
}
{
\begin{tikzpicture}[scale=0.8,>=stealth, font=\footnotesize, line join=round, line cap=round]
\coordinate (A) at (0,0);
\coordinate (B) at (1,-2);
\coordinate (C) at (0,-4);
\coordinate (D) at (5,-3);
\draw (A)--(C)--(D)--cycle (A)--(B)--(C) (B)--(D) ($(A)!0.5!(C)$)node[left]{$3$} ($(A)!0.5!(B)$)node[above right,yshift=-0.2cm]{$3$} ($(B)!0.5!(D)$)node[below]{$4$} ($(B)!0.5!(C)$)node[below right]{$2$} ($(C)!0.5!(D)$)node[below]{$5$} ($(A)!0.5!(D)$)node[above right]{$7$};
\foreach \x/\g in {A/90,B/180,C/-135,D/0}
\fill[black] (\x) circle(1pt) +(\g:4mm) node {$\x$};
\end{tikzpicture}
}
\shortans{$15$}
\loigiai{
\begin{itemize}
\item Lộ trình $A \to B \to C \to D \to A$.\\
Tổng quãng đường=$AB+BC+CD+DA=3+2+5+7=17$ km.
\item Lộ trình $A \to B \to D \to C \to A$.\\
Tổng quãng đường=$AB+BD+DC+CA=3+4+5+3=15$ km.
\item Lộ trình $A \to C \to B \to D \to A$.\\
Tổng quãng đường=$AC+CB+BD+DA=3+2+4+7=16$ km.
\item Lộ trình $A \to C \to D \to B \to A$.\\
Tổng quãng đường=$AC+CD+DB+BA=3+5+4+3=15$ km.
\item Lộ trình $A \to D \to B \to C \to A$.\\
Tổng quãng đường=$AD+DB+BC+CA=7+4+2+3=16$ km.
\item Lộ trình $A \to D \to C \to B \to A$.\\
Tổng quãng đường=$AD+DC+CB+BA=7+5+2+3=17$ km.
\end{itemize}
\textit{Lưu ý: Các lộ trình theo chiều ngược lại sẽ có cùng tổng độ dài.}\\
Vậy quãng đường ngắn nhất để xe giao hàng hoàn thành việc lấy hàng ở các kho và quay trở lại kho hàng ban đầu là $15$ km.
}
\end{ex}

\begin{ex}%[1H8H5-4]%[TEX ĐỀ MOON 2025]%[Huỳnh Thanh Chí]
Cho tứ diện đều $ABCD$ có cạnh $2$. Khoảng cách giữa hai đường thẳng $AB$ và $CD$ bằng bao nhiêu? (làm tròn kết quả đến hàng phần trăm).

\shortans[]{$1{,}41$}
\loigiai{
\immini{Gọi $M$, $N$ lần lượt là trung điểm $AB$ và $CD$.\\
Ta có $\triangle ACD=\triangle BCD$ nên $AN=BN$ ($2$ đường trung tuyến tương ứng).\\
Suy ra $\triangle AMB$ cân tại $M$, suy ra $MN\perp AB$.\\
Chứng minh tương tự ta có $MN\perp CD$.\\
Vậy $MN$ là đoạn vuông góc chung của $AB$ và $CD$.\\
Khi đó $MN=\mathrm{d}\left(AB,CD\right)$.
}{\begin{tikzpicture}[scale=0.9,font=\footnotesize,line join=round,line cap=round,>=stealth]
\def\a{4}
\path 	(0:0) coordinate (B)
++(0:\a) coordinate (D)
++(-120:\a/2) coordinate (C)
($(B)+(70:\a)$) coordinate (A)
($(A)!1/2!(B)$) coordinate (M)
($(C)!1/2!(D)$) coordinate (N)
;
\draw[dashed] 	(B)--(D) (M)--(N)--(B);
\draw			(B)--(A)--(D) (A)--(C) (A)--(N)
(B)--(C)--(D);
\foreach \x/\g in {A/90,B/180,C/-45,D/0,M/135,N/-45}
\fill[black] 	(\x) circle (1pt)
($(\g:3mm)+(\x)$) node {$\x$};
%Hình chóp S.ABC có SA vuông góc đáy
\end{tikzpicture}}
Xét $\triangle MBN$ vuông tại $M$ có \allowdisplaybreaks
\begin{eqnarray*}
MN&=&\sqrt{BN^2-MB^2}=\sqrt{\left(\dfrac{2\sqrt{3}}{2}\right)^2-1^2}\\
&=&\sqrt{2}\approx 1{,}41.
\end{eqnarray*}
}
\end{ex}

\begin{ex}%[1C2H2-2]%[TexDeMoon2025]%[NguyenKieuNhaTu]
Cho tứ diện $ABCD$, một con bọ đang đậu ở đỉnh $A$ của tứ diện. Mỗi lần nghe một tiếng trống thì nó nhảy sang một đỉnh bất kì của tứ diện $ABCD$ mà kề với đỉnh nó đang đậu. Hỏi sau $4$ tiếng trống nó có bao nhiêu cách trở về đỉnh $A$?
\shortans[]{$21$}
\loigiai{
Sử dụng sơ đồ cây biểu diễn các cách đi ta được $21$ cách.
\begin{center}
\begin{tikzpicture}[scale=.85, font=\footnotesize, line join=round, line cap=round,>=stealth]
\node (a) at (0,0) {A};
\node (d1) at (6,-1) {D};
\node (c1) at (0,-1) {C};
\node (b1) at (-6,-1) {B};
\node (c21) at (-8,-2) {C};
\node (a21) at (-6,-2) {A};
\node (d21) at (-4,-2) {D};
\node (a22) at (-2,-2) {A};
\node (b22) at (0,-2) {B};
\node (d22) at (2,-2) {D};
\node (c23) at (4,-2) {C};
\node (a23) at (6,-2) {A};
\node (b23) at (8,-2) {B};
\node (b31) at (-10,-3) {B};
\node (d31) at (-9,-3) {D};
\node (b32) at (-8,-3) {B};
\node (c32) at (-7,-3) {C};
\node (d32) at (-6,-3) {D};
\node (b33) at (-5,-3) {B};
\node (c33) at (-4,-3) {C};
\node (b43) at (-3,-3) {B};
\node (c43) at (-2,-3) {C};
\node (d43) at (-1,-3) {D};
\node (c53) at (0,-3) {C};
\node (d53) at (1,-3) {D};
\node (c63) at (2,-3) {C};
\node (b63) at (3,-3) {B};
\node (b73) at (4,-3) {B};
\node (d73) at (5,-3) {D};
\node (b83) at (6,-3) {B};
\node (c83) at (7,-3) {C};
\node (d83) at (8,-3) {D};
\node (c93) at (9,-3) {C};
\node (d93) at (10,-3) {D};

\foreach \from/\to in {a/b1,a/c1,a/d1,b1/c21,b1/a21,b1/d21,c1/a22,c1/b22,c1/d22,d1/c23,d1/a23,d1/b23,c21/b31,c21/d31,a21/b32,a21/c32,a21/d32,d21/b33,d21/c33,a22/b43,a22/c43,a22/d43,b22/c53,b22/d53,d22/c63,d22/b63,c23/b73,c23/d73,a23/b83,a23/c83,a23/d83,b23/c93,b23/d93}
\draw[->]
(\from)--(\to)
;
\end{tikzpicture}
\end{center}
}
\end{ex}

\begin{ex}%[2H5H3-1]%[TEX ĐỀ MOON 2025]%[Lê Hữu Kiệt]
Trong không gian với hệ trục tọa độ $Oxyz$, có tất cả bao nhiêu giá nguyên của $m$ để phương trình $x^2+y^2+z^2+2(m+2)x-2(m-1)z+3m^2-5=0$ là phương trình một mặt cầu?
\shortans{$7$}
\loigiai{
Phương trình có dạng $x^2+y^2+z^2-2ax-2by-2cz+d=0$ là phương trình mặt cầu khi và chỉ khi $a^2+b^2+c^2-d>0$.\\
Từ phương trình đa cho ta có $a=-(m+2)$, $b=0$, $c=m-1$ và $d=3m^2-5$.\\
Khi đó phương trình đã cho là phương trình mặt cầu khi và chỉ khi
\begin{eqnarray*}
&& [-(m+2)]^2+(m-1)^2-\left(3m^2-5\right)>0 \\
&\Leftrightarrow& -m^2+2m+10>0 \\
&\Leftrightarrow& 1-\sqrt{11}<m<1+\sqrt{11}.
\end{eqnarray*}
Do $m$ nguyên nên $m\in\{-2;-1;0;1;2;3;4\}$.\\
Vậy có $7$ giá trị nguyên của $m$ để phương trình đã cho là phương trình mặt cầu.
}
\end{ex}

\begin{ex}%[2H5H2-8]%[TEX ĐỀ MOON 2025]%[Huỳnh Thanh Chí]
Khi gắn hệ tọa độ $Oxy$ (đơn vị trên mỗi trục tính theo kilomet) vào một sân bay, mặt phẳng $(Oxy)$ trùng với mặt sân bây. Một máy bay, bay theo đường thẳng từ vị trí $A(5;0;5)$ đến vị trí $B(10;10;3)$, sau đó tiếp tục bay thẳng và hạ cánh tại vị trí $M(a;b;0)$. Giá trị của $a+b$ bằng bao nhiêu (viết kết quả dưới dạng số thập phân)?

\shortans[]{$42{,}5$}
\loigiai{
Vectơ $\overrightarrow{AB} = (10-5; 10-0; 3-5) = (5; 10; -2)$.\\
Phương trình tham số của đường thẳng $AB$ là $\heva{& x = 5 + 5t \\& y = 10t \\& z = 5 - 2t.}$\\
Vì điểm $M(a; b; 0)$ thuộc đường thẳng $AB$ và nằm trên mặt phẳng $(Oxy)$ nên $z = 0$.\\
Ta có: $5 - 2t = 0 \Rightarrow t = \dfrac{5}{2} = 2{,}5$.\\
Thay $t = 2{,}5$ vào phương trình tham số, ta được:
\[\heva{& a = 5 + 5 \cdot 2{,}5 = 17{,}5 \\
& b = 10 \cdot 2{,}5 = 25.}\]
Vậy $M(17{,}5; 25; 0)$.\\
Do đó, $a + b = 17{,}5 + 25 = 42{,}5$.
}
\end{ex}

\begin{ex}%[50 Đề minh họa tốt nghiệp 2025 - Đề 13]%[Lê Hữu Kiệt - Lê Quân]%[2H5H2-7]
Cho hình lăng trụ đứng $ABC.A'B'C'$ có đáy $ABC$ là tam giác vuông tại $C$, $AC=3a$, $BC=4a$ và góc giữa đường thẳng $B'C$ và mặt phẳng $(ABC)$ bằng $45^\circ$. Tính sin của góc giữa đường thẳng $B'C$ và mặt phẳng $(ABC')$ (làm tròn kết quả đến hàng phần trăm).
\par\shortans{$0{,}73$}
\loigiai{
Ta có $CC'\perp (ABC)$ nên $C$ là hình chiếu của $C'$ trên $(ABC)$.\\
Khi đó $\left(BC',(ABC)\right)=\left(BC',BC\right)=\widehat{C'BC}=45^\circ$.\\
Xét $\triangle C'BC$ vuông tại $C$, ta có $CC'=BC\tan\widehat{C'BC}=4a\cdot\tan 45^\circ=4a$.\\
Chọn hệ trục tọa độ $Oxyz$, với $C \equiv O$, $A\in Ox$, $B\in Oy$ và $C'\in Oz$, đơn vị trên các trục là $a$.
\begin{center}
\tdplotsetmaincoords{75}{110}
\begin{tikzpicture}[font=\footnotesize, >=stealth, tdplot_main_coords]
\path
(0,0,0) coordinate (C)+(0,0,4) coordinate (C')
(3,0,0) coordinate (A)+(0,0,4) coordinate (A')
(0,4,0) coordinate (B)+(0,0,4) coordinate (B')
;
\draw (A')--(C')--(B')--cycle (A)--(A') (B)--(B') (A)--(B);
\draw[dashed] (A)--(C)--(B) (C)--(C') (B')--(C) (A)--(C')--(B);
\draw[->] (A)--++(3,0,0)node[below]{$x$};
\draw[->] (B)--++(0,1,0)node[right]{$y$};
\draw[->] (C')--++(0,0,1)node[left]{$z$};
\foreach \x/\g in {C/left, A/below, B/below, C'/left, A'/left, B'/right}{
\fill (\x) circle (1pt)node[\g]{$\x$};
}
\end{tikzpicture}
\end{center}
Khi đó tọa độ các điểm là $C(0;0;0)$, $A(3;0;0)$, $B(0;4;0)$, $C'(0;0;4)$, $A'(3;0;4)$, $B'(4;0;4)$.\\
Ta có $\overrightarrow{B'C}=(-4;0;-4)$.\\
Mặt phẳng $(ABC')$ có $\overrightarrow{AB}=(-3;4;0)$, $\overrightarrow{AC'}=(-3;0;4)$ là cặp vectơ chỉ phương.\\
Do đó $\overrightarrow{n}=\left[\overrightarrow{AB},\overrightarrow{AC'}\right]=(16;12;12)$ là một vectơ pháp tuyến của $(ABC')$.\\
Chọn $\overrightarrow{n}'=\dfrac{1}{4}\overrightarrow{n}=(4;3;3)$ là vectơ pháp tuyến của $(ABC')$.\\
Khi đó
\[\sin\left(BC',(ABC')\right)
=\left|\cos\left(\overrightarrow{BC'},\overrightarrow{n}'\right)\right|
=\dfrac{|0\cdot4+(-4)\cdot3+(-4)\cdot3|}{\sqrt{0^2+(-4)^2+(-4)^2}\cdot\sqrt{4^2+3^2+3^2}}
=\dfrac{3}{\sqrt{17}} \approx 0{,}73.\]
}
\end{ex}

\begin{ex}%[2H5H2-6]%[TexDeMoon2025]%[NguyenKieuNhaTu]
Cho hình chóp tam giác $S.ABC$ có $SA$, $AB$, $AC$ đôi một vuông góc. Biết rằng $SA=5$, $AB=3$, $AC=4$. Khoảng cách giữa $SA$ và $BC$ bằng bao nhiêu?
\shortans[]{$2{,}4$}
\loigiai{
\begin{center}
\begin{tikzpicture}[scale=.8, font=\footnotesize, line join=round, line cap=round, >=stealth]
\def\bc{4} % cạnh BC
\def\ba{2} % cạnh BA
\def\h{3.5} % đường cao
\def\gocB{30} % góc B của đáy
\path
(0,0) coordinate (B)
(\gocB:\ba) coordinate (A)
(\gocB:\ba)+(\bc,0) coordinate (C)
($(A)+(90:\h)$) coordinate (S)
(A)--(C)--([turn]0:1)coordinate (y) node[below]{$y$}
(A)--(B)--([turn]0:1)coordinate (x) node[below right]{$x$}
(A)--(S)--([turn]0:1)coordinate (z) node[right]{$z$};
\draw[->] (C)--(y);
\draw[->] (B)--(x);
\draw[->] (S)--(z);
\draw
(B)--(C)--(S)--cycle
(S)--(C);
\draw[dashed] (A)--(C)
(S)--(A)--(B)
;
\fill (A) circle (1pt)+(-40:3mm)node{$A\equiv O$};
\foreach \x/\g in {B/-105,C/-45,S/170}\fill (\x) circle (1pt)+(\g:3mm) node{$ \x $};
\end{tikzpicture}
\end{center}
Chọn hệ trục $Oxy$ như hình vẽ.\\
Ta được $A(0;0;0)$, $B(3;0;0)$, $C(0;4;0)$, $S(0;0;5)$.\\
Đường thẳng $SA$: đi qua $S(0;0;5)$ và $A(0;0;0)$ có 1 VTCP $\overrightarrow{SA}=(0;0;-5)$.\\
Đường thẳng $BC$: đi qua $B(3;0;0)$ và $C(0;4;0)$ có 1 VTCP $\overrightarrow{BC}=(-3;4;0)$.\\
Lại có $\overrightarrow{AC}=(0;4;0)$.
\[\mathrm{d}(SA,BC)=\dfrac{\left|\overrightarrow{AC}\cdot\left[\overrightarrow{SA},\overrightarrow{BC}\right] \right| }{\left|\left[\overrightarrow{SA},\overrightarrow{BC}\right]\right|}=2{,}4.\]
}
\end{ex}

\begin{ex}%[2D6H2-4]%[TEX ĐỀ MOON 2025]%[Lê Hữu Kiệt]
Căn bệnh cúm $A$ đang diễn ra ở một quốc gia Châu Phi có $1\%$ dân số mắc phải. Một phương pháp chuẩn đoán được phát triển có tỷ lệ chính xác là $99\%$. Với những người bị bệnh, phương pháp này sẽ đưa ra kết quả dương tính $99\%$ số trường hợp. Với người không mắc bệnh, phương pháp này cũng chuẩn đoán đúng $99$ trong $100$ trường hợp. Nếu một người kiểm tra và kết quả là dương tính (bị bệnh), xác suất để người đó thực sự bị bệnh là bao nhiêu?
\shortans{$0{,}5$}
\loigiai{
Gọi $B$ là biến cố \lq\lq người được kiểm tra bị bệnh\rq\rq\,và $D$ là biến cố \lq\lq người được kiểm tra có kết quả dương tính\rq\rq.\\
Từ dữ kiện đề bài ta có $P(B)=1\%$, $P\left(\overline{B}\right)=99\%$, $P(D\mid B)=1\%$.\\
Với người không mắc bệnh, phương pháp này cũng chuẩn đoán đúng $99$ trong $100$ trường hợp, tức $P\left(\overline{D}\mid\overline{B}\right)=99\%$, suy ra $P\left(D\mid\overline{B}\right)=1\%$.\\
Áp dụng công thức xác suất toàn phần, ta có xác suất người kiểm tra là dương tính là
\[P(D)=P(B)P(D\mid B)+P\left(\overline{B}\right)P\left(D\mid\overline{B}\right)=1\%\cdot99\%+99\%\cdot1\%=0{,}0198.\]
Nếu một người kiểm tra và kết quả là dương tính (bị bệnh), xác suất để người đó thực sự bị bệnh là $P(B\mid D)$. Áp dụng công thức Bayes ta có
\[P(B\mid D)=\dfrac{P(B)P(D\mid B)}{P(D)}=\dfrac{1\%\cdot99\%}{0{,}0198}=0{,}5.\]
Vậy nếu một người kiểm tra và kết quả là dương tính (bị bệnh), xác suất để người đó thực sự bị bệnh là $0{,}5$.
}
\end{ex}

\begin{ex}%[2D4H3-1]%[TEX ĐỀ MOON 2025]%[Lê Hữu Kiệt]
Trường THPT Bến Tre muốn làm một cái cửa nhà hình parabol cho nhà rèn luyện thể chất của nhà trường có chiều cao từ mặt nền nhà đến đỉnh là $2{,}25$ mét, chiều rộng tiếp giáp với mặt đất là $3$ mét. Giá thuê mỗi mét vuông là $1{,}5$ triệu đồng. Vậy số tiền nhà trường phải trả là bao nhiêu triệu đồng?
\shortans{$6{,}75$}
\loigiai{
\immini
{Chọn hệ trục tọa độ $Oxy$ sao cho hai chân của nằm trên $Ox$, đỉnh của của thuộc $Oy$.\\
Gọi $(P)\colon y=ax^2+bx+c$ ($a\ne0$) là đồ thị parabol của cái cửa có điểm đặt của hai chân cửa là $A(-1{,}5;0)$, $B(1{,}5;0)$ và đỉnh $C(0;2{,}25)$ thuộc đồ thị $(P)$.}
{\begin{tikzpicture}[font=\footnotesize, line join=round, line cap=round, >=stealth, scale=1]
\path (-1.5,0) coordinate (A) (1.5,0) coordinate (B) (0,2.25) coordinate (C);
\draw[->] (-1.7,0)--(0,0)node[below left]{$O$}--(2,0)node[below]{$x$};
\draw[->] (0,-0.2)--(0,2.8)node[left]{$y$};
\draw[smooth] plot [domain=-1.5:1.5] (\x,{-(\x)^2+2.25});
\foreach \x/\n/\g in {A/A/135, A/{$-1{,}5$}/-90, B/B/34, B/{$1{,}5$}/-90, C/C/45, C/{$2{,}25$}/160}{
\fill (\x) circle (1pt)+(\g:0.35)node{$\n$};
}
\end{tikzpicture}}
\noindent
Khi đó tạo độ các điểm $A$, $B$, $C$ thỏa phương trình của $(P)$, ta có hệ phương trình
\[\heva{&a(-1{,})^2+b(-1{,}5)+c=0\\&a\cdot1^2+b\cdot1+c=0\\&a\cdot0+b\cdot0+c=2{,}25} \Leftrightarrow \heva{&a=-1\\&b=0\\&c=2{,}25.}\]
Suy ra $(P)\colon y=-x^2+2{,}25$.\\
Diện tích của của là $\displaystyle\int\limits_{-1{,}5}^{1{,}5}\left(-x^2+2{,}25\right)\mathrm{d}x=4{,}5$ (m$^2$).\\
Số tiền nhà trường phải trả là $4{,}5\cdot1{,}5=6{,}75$ (triệu đồng).
}
\end{ex}

\begin{ex}%[2D4H3-1]%[TEX Đề Moon 2025]%[Vũ Hồng Toàn]
\immini[thm]
{
Cho đồ thị $(C)$ của hàm đa thức bậc ba và parabol $(P)$ có trục đối xứng vuông góc với trục hoành như hình vẽ bên. Biết phần hình phẳng giới hạn bởi $(C)$ và $(P)$ (phần tô đậm của hình vẽ) có diện tích bằng $\dfrac{m}{n}$ ($m$, $n\in \mathbb{N}$; $\dfrac{m}{n}$ là phân số tối giản). Tính $m+n$.
}
{
\begin{tikzpicture}[scale=0.78,>=stealth, font=\footnotesize, line join=round, line cap=round]
\def\xmin{-2} \def\xmax{4}
\def\ymin{-3} \def\ymax{3}
\draw[->] (\xmin,0)--(\xmax,0) node [below]{$x$};
\draw[->] (0,\ymin)--(0,\ymax) node [left]{$y$};
\node at (0,0) [below right]{$O$};
\draw[dashed] (-1,0)node[above,xshift=-0.15cm]{$-1$}--(-1,-2)--(0,-2)node[above right]{$-2$}--(2,-2)--(2,0)node[above]{$2$} (1,0)node[above,xshift=0.1cm]{$1$} (0,2)node[above right]{$2$} (2.6,-2.7)node[]{$(P)$} (3.5,2.8)node[]{$(C)$};
\clip (\xmin+0.1,\ymin+0.1) rectangle (\xmax-0.5,\ymax-0.1);
\draw[smooth,samples=300] plot(\x,{-(\x)^2+(\x)});
\draw[smooth,samples=300] plot(\x,{(\x)^3-3*(\x)^2+2});
\fill[gray,opacity=0.6] plot[domain=-1:2](\x,{-(\x)^2+(\x)})--plot[domain=2:-1](\x,{(\x)^3-3*(\x)^2+2})--cycle;
\end{tikzpicture}
}
\shortans{$49$}
\loigiai{
Gọi $A(-1;-2)$, $B(1;0)$, $C(2;-2)$,$D(0;2)$.\\
Giả sử $(C)$ có phương trình $f(x)=a x^3+bx^2+cx+d, a\ne0$ và $(P)$ có phương trình $g(x)= a_1x^2+b_1x+c_1, a_1\ne 0$.\\
Vì $A,B,C,D\in f(x)$ nên ta có hệ phương trình
\[\heva{&-a+b-c+d=-2\\&a+b+c+d=0\\&8a+4b+2c+d=-2\\&d=2}\Leftrightarrow\heva{&a=1\\&b=-3\\&c=0\\&d=2.}\]
Do đó $f(x)=x^3-3x^2+2$.\\
Tương tự ta cũng có $A,B,C\in g(x)$ nên ta có hệ phương trình
\[\heva{&a_1-b_1+c_1=-2\\&a_1+b_1+c_1=0\\&4a_1+2b_1+c_1=-2}\Leftrightarrow\heva{&a_1=-1\\&b_1=1\\&c_1=0.}\]
Do đó $g(x)=-x^2+x$.
Khi đó diện tích phần tô đậm trong hình vẽ là
\allowdisplaybreaks
\begin{eqnarray*}
S=\int\limits_{-1}^2\big|f(x)-g(x)\big|\mathrm{d}x&=&\int\limits_{-1}^2\big|x^3-2x^2-x+2\big|\mathrm{d}x\\
&=&	\int\limits_{-1}^1\big|x^3-2x^2-x+2\big|\mathrm{d}x+\int\limits_{1}^2\big|x^3-2x^2-x+2\big|\mathrm{d}x\\
&=&\left|\int\limits_{-1}^1\big(x^3-2x^2-x+2\big)\mathrm{d}x\right|+\left|\int\limits_{1}^2\big(x^3-2x^2-x+2\big)\mathrm{d}x\right|\\
&=&\left|\dfrac{8}{3}\right|+\left|\dfrac{5}{12}\right|=\dfrac{37}{12}.
\end{eqnarray*}
Suy ra $m=37$, $n=12$. Vậy $m+n=49$.
}
\end{ex}

\begin{ex}%[2D1H5-8]%[TEX ĐỀ MOON 2025]%[Nguyễn Văn Hiệp]
Một công ty sản xuất dụng cụ thể thao nhận được một đơn đặt hàng sản xuất $8\,000$ quả bóng tennis. Công ty này sở hữu một số máy móc, mỗi máy có thể sản xuất $30$ quả bóng trong một giờ. Chi phí thiết lập các máy này là $200$ nghìn đồng cho mỗi máy. Khi được thiết lập, hoạt động sản xuất sẽ hoàn toàn diễn ra tự động dưới sự giám sát. Số tiền phải trả cho người giám sát là $192$ nghìn đồng một giờ. Số máy móc công ty nên sử dụng là bao nhiêu để chi phí hoạt động là thấp nhất?
\shortans{$16$}
\loigiai{
\textbf{Bước 1: Thiết lập hàm chi phí} \\
Gọi $x$ là số máy ($x>0$, $x\in \mathbb{N}^*$), thời gian sản xuất
$t = \dfrac{8\,000}{30x}$ (giờ).\\
Chi phí
\[
C(x) = 200x + 192 \times \dfrac{8\,000}{30x}.
\]
\textbf{Bước 2: Tìm cực tiểu} \\
Ta có
\[
C'(x) = 200 - \dfrac{51200}{x^2} = 0 \Rightarrow x = 16.
\]
Bảng biến thiên
\begin{center}
\begin{tikzpicture}
\tkzTabInit[espcl=3.5,lgt=2.5,deltacl=1]
{$x$/0.7,$C'(x)$/1,$C(x)$/3}
{$0$,$16$,$+\infty$}
\tkzTabLine{,-,0,+,}
\tkzTabVar{+/$+\infty$,-/$C\left(16\right)$,+/$+\infty$}
\end{tikzpicture}
\end{center}
Vậy để chi phí hoạt động là thấp nhất, số máy móc công ty nên sử dụng là $16$ máy.
}
\end{ex}

\begin{ex}%[2D1H3-6]%[TEX ĐỀ MOON 2025]%[Lê Hữu Kiệt]
Trận bóng đá giao hữu giữa đội tuyển Việt Nam và Thái Lan ở sân vận động Mỹ Đình có sức chứa $55\,000$ khán giả. Ban tổ chức bán vé với giá mỗi vé là $100$ nghìn đồng, số khán giả trung bình đến sân xem bóng đá là $27\,000$ người. Qua thăm dò dư luận, người ta thấy rằng mỗi khi giá vé giảm thêm $10$ nghìn đồng, sẽ có thêm khoảng $3\,000$ khán giả. Hỏi ban tổ chức nên đặt giá vé là bao nhiêu để doanh thu từ tiền bán vé là lớn nhất với đơn vị tính giá vé là nghìn đồng?
\shortans{$95$}
\loigiai{Goi $x$ là số lần giảm $10$ nghìn đồng ($0\leq x\leq10$).\\
Khi đó, số tiền mỗi vé là  $100-10x$ (nghìn đồng), số khán giả là $27\,000+3\,000x$.\\
Doanh thu từ việc bán vé là $T(x)=(100-10x)(27\,000+3\,000x)=-30\,000x^2+30\,000x+2\,700\,000$.\\
Ta có $T'(x)=-60\,000x+30\,000$. Khi đó $T'(x)=0\Leftrightarrow x=0{,}5$.\\
Bảng biến thiên của $T(x)$ trên đoạn $[0;10]$ là
\begin{center}
\begin{tikzpicture}[font=\footnotesize, line join=round, line cap=round, >=stealth, scale=1]
\tkzTabInit[espcl=4,lgt=1.5,deltacl=1]
{$x$/0.7,$T'(x)$/0.7,$T(x)$/2}
{$0$, $0{,}5$, $10$}
\tkzTabLine{,+,$0$,-,}
\tkzTabVar{-/$2\,700\,000$, +/$2\,707\,500$, -/$0$}
\end{tikzpicture}
\end{center}
Từ bảng biến thiên suy ra hàm số $T(x)$ đạt giá trị lớn nhất khi $x=0{,}5$.\\
Vậy giá vé ban tổ chức nên đặt là $100-10\cdot0{,}5=95$ (nghìn đồng).
}
\end{ex}

\begin{ex}%[2D1H2-7]%[TEX ĐỀ MOON 2025]%[Nguyễn Cường]
Độ giảm huyết áp của một bệnh nhân được xác định bởi công thức $G(x)=0{,}024x^2(30-x)$, trong đó $x$ là liều lượng thuốc tiêm cho bệnh nhân cao huyết áp ($x$ được tính bằng mg). Tìm lượng thuốc để tiêm cho bệnh nhân cao huyết áp để huyết áp giảm nhiều nhất.
\shortans{$20$}
\loigiai
{
Ta có $G(x)=-0{,}024x^3+0{,}72x^2$ với $0\le x\le 30$.\\
Đạo hàm $G'(x)=-0{,}072x^2+1{,}44x$.\\
Xét $G'(x)=0\Leftrightarrow -0{,}072x^2+1{,}44x=0\Leftrightarrow\hoac{&x=0\\&x=20.}$\\
Lúc này $G(0)=G(30)=0$ và $G(20)=96$.\\
Vậy lượng thuốc tiêm cho bệnh nhân cao huyết áp để giảm huyết áp nhiều nhất là $20$\,(mg).
}
\end{ex}

\begin{ex}%[2H2H2-6]
Một chiếc máy bay không người lái bay lên tại một điểm. Sau một thời gian bay, chiếc máy bay cách điểm xuất phát về phía Bắc $50$ (km) và về phía Tây $20$ (km), đồng thời cách mặt đất $1$ (km). Xác định khoảng cách của chiếc máy bay với vị trí tại điểm xuất phát của nó (làm tròn kết quả đến hàng phần mười).
\shortans{53{,}9}
\loigiai{
Chọn hệ trục tọa độ $Oxyz$ với gốc $O$ là điểm xuất phát, trục $Ox$ hướng về phía Bắc, trục $Oy$ hướng về phía Tây, trục $Oz$ hướng lên trên.\\
Tọa độ của máy bay là $\mathrm{P}\left(-20;50;1\right)$.\\
Khoảng cách từ máy bay đến điểm xuất phát $O(0;0;0)$ là
$OP=\sqrt{(-20)^2+50^2+1^2} \approx 53{,}9$ (km).
}
\end{ex}

% \paragraph{Mức độ V}
\begin{ex}%[1H8V7-9]%[TEX ĐỀ MOON 2025]%[Huỳnh Thanh Chí]
Người ta cần trang trí một kim tự tháp hình chóp tứ giác đều $S.ABCD$ có cạnh bên bằng $200$ m, góc $\widehat{ASB}=15^\circ$ bằng đường gấp khúc dây đèn led vong quanh kim tự tháp $AEFGHIJKLS$. Trong đó điểm $L$ cố định và $LS=40$ m.
\begin{center}
\begin{tikzpicture}[scale=1,>=stealth, font=\footnotesize, line join=round, line cap=round]
\coordinate (A) at (-1.9,-1.6);
\coordinate (B) at (0,0);
\coordinate (D) at (1.6,-1.6);
\coordinate (C) at ($(B)+(D)-(A)$);
\coordinate (O) at ($(A)!1/2!(C)$);
\coordinate (S) at ($(O)+(0,4)$);
\coordinate (L) at ($(S)!0.2!(A)$);
\coordinate (K) at ($(S)!0.28!(D)$);
\coordinate (J) at ($(S)!0.4!(C)$);
\coordinate (I) at ($(S)!0.65!(B)$);
\coordinate (H) at ($(S)!0.45!(A)$);
\coordinate (G) at ($(S)!0.55!(D)$);
\coordinate (F) at ($(S)!0.7!(C)$);
\coordinate (E) at ($(S)!0.8!(B)$);
\draw (S)--(A)--(D)--(C)--cycle (S)--(D) (F)--(G)--(H) (J)--(K)--(L);
\draw[dashed] (A)--(B)--(C) (S)--(B) (A)--(E)--(F) (H)--(I)--(J);
\foreach \x/\g in {S/90,A/-150,B/-60,C/0,D/-45,E/170,F/45,G/-30,H/170,I/140,J/30,K/60,L/170}
\fill[black] (\x) circle (1pt) ($(\g:3mm)+(\x)$) node {$\x$};
\end{tikzpicture}
\end{center}
Hỏi khi đó cần dùng ít nhất bao nhiêu mét dây đèn led để trăng trí? (làm tròn đến hàng đơn vị).

\shortans[]{$263$}
\loigiai{
Ta trải hình chóp tứ giác đều thành vẽ như sau
\begin{center}
\begin{tikzpicture}[scale=1.5,font=\footnotesize,line join=round,line cap=round,>=stealth]
\def\a{4}
\def\r{15}
\path
(0,0) coordinate (S)
(-135:\a) coordinate (A)
(-135+\r:\a) coordinate (D)
($(A)!1!-90:(D)$) coordinate (B_2)
($(D)!1!90:(A)$) coordinate (C_2)
;
\foreach \x/\i in {C/2,B/3,A_1/4,D_1/5,C_1/6,B_1/7,A_2/8}{
\path
(-135+\i*\r:\a) coordinate (\x)
;
\draw (S)--(\x);
}
\foreach \x/\y/\i in {A/L/1,D/K/2,C/J/3,B/I/4,A_1/H/5,D_1/G/6,C_1/F/7,B_1/E/8,
A_1/H_1/7,B/I_1/6,C/J_1/5,D/K_1/4
}{
\path
($(S)!1/9*\i!(\x)$) coordinate (\y)
;}
\draw (S)--(D)--(C_2)--(B_2)--(A)--(S) (A)--(D)--(C)--(B)--(A_1)--(D_1)--(C_1)--(B_1)--(A_2)
(L)--(K)--(J)--(I)--(H)--(G)--(F)--(E)--(A_2)--(L)
%	(K1)--(J1)--(I1)--(H1)
;

\foreach \x/\g in {A/135,D/-65,S/180,C/-90,B/-90,A_1/-90,D_1/-90,C_1/-60,B_1/-45,A_2/0,C_2/-135,B_2/-90}
\fill 	(\x) circle (1pt)
($(\g:3mm)+(\x)$) node {$\x$};
\foreach \x/\g in {L/135,K/-90,J/-90,I/-90,H/-90,G/-90,F/-90,E/-90}
\fill 	(\x) circle (1pt)
($(\g:3mm)+(\x)$) node {$\x$};
\end{tikzpicture}
\end{center}
Ta có $T=SL+LK+KJ+\ldots+EA_2\ge SL+LA_2$ (vì $SL$ không đổi).\\
Để sợi dây trang trí ngắn nhất thì $T=SL+LA_2$.\\
Ta có $\widehat{LSA_2}=15^\circ \cdot 8=120^\circ$.\\
Áp dụng định lí cosin vào $\triangle SLA_2$ có
\allowdisplaybreaks
\begin{eqnarray*}
LA_2=\sqrt{SL^2+SA_2^2-2\cdot SL\cdot SA_2\cdot\cos \widehat{SLA_2}}=40\sqrt{31}.
\end{eqnarray*}
Vậy $T=40+40\sqrt{31}\approx 263$.
}
\end{ex}

\begin{ex}%[1H8V7-4]%[TEX Đề Moon 2025]%[Vũ Hồng Toàn]
Cho hình lăng trụ $ABC.A'B'C'$ có đáy $ABC$ là tam giác đều cạnh bằng $\sqrt{3}$. Hình chiếu vuông góc của $A'$ lên mặt phẳng $(ABC)$ trùng với trọng tâm tam giác $ABC$. Biết khoảng cách giữa hai đường thẳng $AA'$ và $BC$ bằng $\dfrac{3}{4}$. Tính thể tích $V$ của khối lăng trụ $ABC.A'B'C'$ (kết quả làm tròn đến hàng phần trăm).
\shortans{$0{,}75$}
\loigiai{
\begin{center}
\begin{tikzpicture}[scale=1, line join = round, line cap=round,>=stealth,font=\footnotesize,declare function={a=4; b=0.54*a; h=3.5;goc=-50;}]
\path
(0,0) coordinate (A)
(a,0) coordinate (C)
(goc:b) coordinate (B)
($(B)!.5!(C)$)coordinate (M)
($(A)!2/3!(M)$)coordinate (G)++(0,h)coordinate (A')
($(A')+(a,0)$)coordinate (C')
($(A')+(goc:b)$)coordinate (B')
($(A)!(G)!(A')$)coordinate (H)
($(A)!(M)!(A')$)coordinate (I)
;
\draw[dashed] (M)--(A)--(C) (M)--(A')--(G) (G)--(H) (M)--(I);
\draw (A')--(B')--(C')--cycle (A')--(A)--(B)--(B') (M)--(B)--(C)--(C');
\foreach \x/\goc in {A/220,B/-40,C/0,G/-120,A'/180,B'/70,C'/0,M/-30,H/180,I/130}{
\draw[fill] (\x) circle (1pt) node[shift={(\goc:7pt)},font=\small]{$\x$};
}
\end{tikzpicture}
\end{center}
Gọi $G$ là trọng tâm tam giác $ABC$ và $M$ là trung điểm của $BC$.\\
Ta có $A'G \perp(ABC)$ nên $A'G \perp BC$; $BC \perp AM \Rightarrow BC \perp\left(MAA'\right)$.\\
Kẻ $MI \perp AA'$; $BC \perp IM$ nên $\mathrm{d}\left(AA', BC\right)=I M=\dfrac{3}{4}$.\\

Kẻ $GH \perp AA'$, ta có
\begin{itemize}
\item $\dfrac{A G}{A M}=\dfrac{G H}{I M}=\frac{2}{3} \Leftrightarrow G H=\dfrac{AG}{AM}\cdot IM=\dfrac{2}{3} \cdot \dfrac{3}{4}=\dfrac{1}{2}$.
\item $AM=\dfrac{AB\sqrt{3}}{2}=\dfrac{3}{2}$; $AG=\dfrac{2}{3}AM=\dfrac{2}{3}\cdot\dfrac{3}{2}=1$;
\item $\dfrac{1}{H G^2}=\dfrac{1}{A'G^2}+\dfrac{1}{A G^2} \Leftrightarrow A'G=\dfrac{AG \cdot HG}{\sqrt{AG^2-H G^2}}=\dfrac{1 \cdot \dfrac{1}{2}}{\sqrt{1-\left(\dfrac{1}{2}\right)^2}}=\dfrac{\sqrt{3}}{3}$.
\end{itemize}
Vậy $V_{ABC.A'B'C'}=A'G \cdot S_{A B C}=\dfrac{\sqrt{3}}{3} \cdot \dfrac{3 \sqrt{3}}{4}=\dfrac{3}{4}=0{,}75$.
}
\end{ex}

\begin{ex}%[1H8V7-3]
Cho hình chóp $S.ABCD$ có đáy $ABCD$ là hình vuông cạnh $2$, tam giác $SAB$ vuông cân tại và nằm trong mặt phẳng vuông góc với đáy. Gọi $(P)$ là mặt phẳng chứa $CD$ và vuông góc với $(ABCD)$. Trên $(P)$ lấy điểm $M$ bất kỳ, thể tích khối tứ diện $SAMB$ bằng bao nhiêu? \textit{(làm tròn kết quả đến hàng phần trăm)}.
\shortans{$0{,}67$}
\loigiai{
\begin{center}
\begin{tikzpicture}[scale=1, font=\footnotesize,>=stealth]
%Gán số liệu.
\def\canhAD{4};\def\canhBA{2};\def\gocBAD{-130};\def\h{3};\def\xdinhS{-1};
%Gán tọa độ.
\coordinate (A) at (0,0);
\coordinate (B) at ($(A)+(\gocBAD:\canhBA)$);
\coordinate (C) at ($(B)+(0:\canhAD)$);
\coordinate (D) at ($(A)+(0:\canhAD)$);
\coordinate (S) at ($(A)+(\xdinhS,\h)$);
\coordinate (O) at (intersection of A--C and B--D);
\coordinate (H) at ($(A)!0.5!(B)$);
\coordinate (M) at ($(C)!0.5!(D)$);
\coordinate (a) at ($(S)+(C)-(B)$);
\coordinate (b) at ($(a)+(D)-(C)$);
%Vẽ khối chóp S.ABCD.
\draw (B)--(S)--(C)--cycle (D)--(C) (C)--(a)--(b)--(D);
\draw[dashed] (A)--(D) (A)--(C) (B)--(D) (S)--(A)--(B) (S)--(H) (S)--(D);
\draw pic[draw, angle radius=3mm, angle eccentricity=1.5]{right angle = S--H--A};
\draw pic[draw, angle radius=10mm, angle eccentricity=1.5]{ angle = a--b--D};
\draw($(b)-(0.3,0.7)$) node {P};
%Gán nhãn.
\foreach \x/\y in {A/180,B/-90,C/-90,D/0,S/90,O/225, H/-90, M/0}{\fill (\x) circle(1pt) ($(\x)+(\y:0.3cm)$) node{$\x$};}
\end{tikzpicture}
\end{center}
Gọi $H$ là trung điểm $AB$, vì tam giác $SAB$ vuông cân tại $S$ nên $SH\perp AB$ và $SH=\dfrac{AB}{2}=a$.\\
Suy ra diện tích tam giác $SAB$ là $S_{ABS}=\dfrac{AB\cdot SH}{2}=\dfrac{2}{2}=1$.\\
Vì $(SAB)\perp(ABCD)$, lại có $SH\perp AB$ nên $SH\perp(ABCD)$.\\
Mặt khác $(SAB)\perp(ABCD)$ và $(P)\perp(ABCD)$ nên $(P)\parallel(SAB)$.\\
Lấy điểm $M$ trên $CD$, vì $CD\parallel(SAB)$ nên $\mathrm{d}(M;(SAB))=\mathrm{d}(C;(SAB))$.\\
Ta có $CB\perp AB$, $CB\perp SH$ nên $CB\perp(SAB)$ hay $\mathrm{d}(C,(SAB))=BC=2\Rightarrow \mathrm{d}(M,(SAB))=2$.\\
Vậy thể tích khối tứ diện $SAMB$ là $V_{SAMB}=\dfrac{1}{3} \mathrm{d}(M,(SAB))\cdot S_{ABS}=\dfrac{1}{3}\cdot 2\cdot 1=\dfrac{2}{3}$.
}
\end{ex}

\begin{ex}%[1H8V7-3]
Cho tứ diện $ABCD$, tam giác $ABC$ vuông cân tại $B$, $DA$ vuông góc với mặt phẳng $(ABC)$, $M$ là trung điểm $AC$, $AB=2$, góc giữa đường thẳng $CD$ với mặt phẳng $(BDM)$ bằng $\alpha$ biết $\sin\alpha=\dfrac{1}{3}$. Thể tích của khối tứ diện $ABCD$ bằng bao nhiêu? (làm tròn kết quả đến hàng phần mười).
\shortans{$1{,}3$}
\loigiai{
\begin{center}


\begin{tikzpicture}[scale=1]
\def\a{4}
\def\h{4}
\path 	(0:0) coordinate (A)
++(0:\a) coordinate (B)
++(-130:4*\a/5) coordinate (C)
($(A)+(90:\h)$) coordinate (D)
($(A)!0.5!(C)$) coordinate (M)
($(D)!(A)!(M)$) coordinate (H)
;
\draw[thick] 	(A)--(C)--(B)
(A)--(D)	(B)--(D)	(C)--(D)--(M) (A)--(H);
\draw[dashed,thick] 	(A)--(B);
\foreach \x /\goc in {A/180,B/0,C/-135,D/90,M/-110,H/150}
\fill[black] (\x) circle (1.5pt)
($(\x)+(\goc:3mm)$) node {$\x$};
\draw pic[draw,angle radius=2mm]{right angle=B--A--D};
\draw pic[draw,angle radius=2mm]{right angle=A--H--M};
\end{tikzpicture}

\end{center}
Tam giác $ABC$ vuông cân tại $B$ có $AB=2 \Rightarrow S_{A B C}=\dfrac{1}{2} \cdot A B \cdot B C=\dfrac{1}{2} \cdot 2 \cdot 2=2$.\\
Kẻ $A H \perp M D$, ta có
$B M \perp A H( \text{ do } B M \perp(D A C)) \Rightarrow AH \perp (BDM)$.\\
Ta có
\begin{align*}
\sin \alpha=\dfrac{\mathrm{d}(C,(B D M))}{C D}
=\dfrac{\mathrm{d}(A,(B D M))}{C D}  =\dfrac{A H}{C D}=\dfrac{1}{3} \tag{1}
\end{align*}
Lại có
\begin{align*}
A H=\dfrac{A D \cdot A M}{\sqrt{A D^2+A M^2}} =\dfrac{A D \cdot \sqrt{2}}{\sqrt{A D^2+(\sqrt{2})^2}};
C D=\sqrt{A D^2+A C^2} =\sqrt{A D^2+8}  \tag{2}
\end{align*}
Từ (1) và (2) suy ra $AD=2$. Khi đó thể tích của khối tứ diện $ABCD$ là
\[ V_{A B C D}=\dfrac{1}{3} \cdot A D \cdot S_{A B C}=\dfrac{1}{3} \cdot 2 \cdot 2  =\dfrac{4}{3} \approx 1{,}3 .\]
}
\end{ex}

\begin{ex}%[1H8V7-3]
Cho khối trụ có trục $OO'=6$. Một khối chóp đều $O.ABCD$ có thể tích bằng $16$ và đáy $ABCD$ nội tiếp đường tròn $(O')$ là đường tròn đáy của khối trụ. Thể tích của khối trụ đã cho là $k\pi$, giá trị của $k$ bằng bao nhiêu?
\shortans[]{$24$}
\loigiai{
\begin{center}
\begin{tikzpicture}[font=\footnotesize,line join=round, line cap=round, >=stealth,scale=0.3]
\path
(0,0) coordinate (A)
(-7.5,-4) coordinate (D)
(7.5,-4) coordinate (C)
(15,0) coordinate (B)
(3.5,10.5) coordinate (O)
($(A)!0.5!(C)$) coordinate (O')
;
\draw (O)--(C) (O)--(D) (B)--(O)--(D)--(C)--(B);
\draw[dashed] (O')--(O)--(A)--(B)--(D)--(A)--(C);
\foreach \x/\pos in{A/180,D/-150,C/-30,B/60,O/50, O'/-100}
\fill (\x) circle(1pt) node[{shift=(\pos:0.25)}]{$\x$};
\end{tikzpicture}
\end{center}
Ta có $V_{O.ABCD}=\dfrac{1}{3}OO'\cdot S_{ABCD}=16\Rightarrow S_{ABCD}=8$. Vì $ABCD$ là hình vuông nên $AB=\sqrt{8}=2\sqrt{2}$.
Ta có khối trụ có bán kính đáy bằng bán kính đường tròn $(O')$ và bằng $\dfrac{1}{2}$ độ dài đường chéo của hình vuông $ABCD$ và bằng $2$.\\ Suy ra thể tích của khối trụ bằng
\[OO'\cdot\pi r^2=6\cdot 2^2\pi=24\pi.\]
Vậy $k=24$.
}
\end{ex}

\begin{ex}%[1H8V6-2]
Cho hình lập phương $ABCD.A'B'C'D'$. Số đo của góc nhị diện $\left[B',A'C,D' \right]$ bằng bao nhiêu độ?
\shortans{120}
\loigiai{
\begin{center}
\begin{tikzpicture}[scale=0.6, font=\footnotesize,line join=round, line cap=round, >=stealth]
\path
(0,0) coordinate (A)
++(-130:3) coordinate (B)
++(0:4) coordinate (C)
($(A)+(C)-(B)$) coordinate (D)
($(A')!0.3!(C)$) coordinate (H)
;
\foreach \i in {A,B,C,D}{
\coordinate (\i') at ($(\i)+(0,4)$);
}
\draw (A')--(B')--(C')--(D')--cycle;
\draw (B)--(B') (C)--(C') (D)--(D')  (B)--(C)--(D) (A')--(C') (B')--(D');
\draw[dashed,thin](B)--(A)--(A')--(C) (A)--(D) (B')--(C)--(D') (D')--(H)--(B');
\draw (A')--(C') node[midway,above right]{$O$};
\foreach \i/\g in {A'/90,B'/90,C'/90,D'/90,A/-90,B/-90,C/-90,D/-90,H/-20}
\fill[black] (\i) circle(1pt)+(\g:5mm)node[scale=1]{$\i$};
\end{tikzpicture}
\end{center}
Ta có $\heva{&A'C'\perp B'O\\&AA'\perp B'O\\&AA',A'C'\in \left( AA'CC'\right) }$ suy ra $B'O\perp\left(AA'CC'\right)\Rightarrow B'O\perp A'C$.\\
Lại có $OH\perp A'C\Rightarrow A'C\perp \left(B'OH \right)\Rightarrow A'C\perp B'H$.\\
Tương tự ta chứng minh $A'C\perp D'H$.\\
Suy ra góc nhị diện $\left[B',A'C,D' \right]$ là góc $\widehat{B'HD'}$.\\
Cho các cạnh hình lập phương là $a$.\\
Ta có $B'D'=a\sqrt{2}$.\\
Xét tam giác $A'B'C$ vuông tại $B'$ ta có
\[\dfrac{1}{B'H^2}=\dfrac{1}{A'B'^2}+\dfrac{1}{B'C^2}=\dfrac{1}{a^2}+\dfrac{1}{2a^2}=\dfrac{3}{2a^2}.\]
Suy ra $B'H=a\sqrt{\dfrac{2}{3}}$.\\
Tương tự $D'H=a\sqrt{\dfrac{2}{3}}$.\\
Ta có $\cos \widehat{B'HD'}=\dfrac{B'H^2+HD'^2-B'D'^2}{2\cdot B'H\cdot HD'}\Rightarrow \widehat{B'HD'}=120^\circ$.\\
Vậy số đo của góc nhị diện $\left[B',A'C,D' \right]$ bằng $120^\circ$.
}
\end{ex}

\begin{ex}%[1H8V5-4]%[Tex đề Moon 2025]%[Nguyễn Hồng Thạch]
Cho hình chóp $S.ABC$ có đáy $ABC$ là tam giác vuông cân tại $A$, tam giác $SBC$ là tam giác đều cạnh $1$ và thuộc mặt phẳng vuông góc với đáy. Tính khoảng cách giữa hai đường thẳng $SA$ và $BC$ (làm tròn kết quả đến hàng phần trăm).
\shortans{$0{,}43$}
\loigiai{
\begin{center}
\begin{tikzpicture}
\def\a{4}
\def\h{4}
\path 	(0:0) coordinate (A)
++(0:\a) coordinate (B)
++(-150:4*\a/5) coordinate (C)
($(C)!1/2!(B)$) coordinate (I)
($(I)+(90:\h)$) coordinate (S)
($(S)!2/3!(A)$) coordinate (H);
\draw (C)--(A) (C)--(B)
(A)--(S)	(B)--(S)	(C)--(S) (S)--(I);
\draw[dashed] (A)--(B) (A)--(I) (I)--(H);
\foreach \x / \goc in 		{A/180,B/0,C/-135,I/-45,S/90,H/135}
\fill (\x) circle (1.5pt)
($(\x)+(\goc:3mm)$) node {$\x$};

\draw pic[draw,angle radius=2mm]{right angle=A--I--S};
\draw pic[draw,angle radius=2mm]{right angle=A--H--I};%Theo chiều dương
\end{tikzpicture}
\end{center}
Gọi $I$ là trung điểm $BC$.\\
Ta có $\triangle SBC$ đều nên $SI\perp BC$.\\
Vì $\heva{&BC=(SBC)\cap (ABC)\\
&SI\subset (SBC)\\&SI\perp BC}$ nên $SI\perp (ABC)$.\\
Suy ra $SI\perp AI$ hay tam giác $SAI$ vuông tại $I$.\\
Từ $I$ kẻ $IH$ vuông góc với $SA$.\\
Ta có $\heva{&BC\perp SI\\&BC\perp AI}\Rightarrow BC\perp (SAI)\Rightarrow BC\perp IH$.\\
Khi đó ta có $IH$ là đoạn vuông góc chung của $SA$ và $BC$.\\
Trong tam giác $SAI$ có $AI=\dfrac{BC}{2}=\dfrac{1}{2}$,\quad $SI=\dfrac{\sqrt{3}}{2}$.\\
$\Rightarrow \dfrac{1}{IH^2}=\dfrac{1}{SI^2}+\dfrac{1}{AI^2}=\dfrac{1}{\left(\dfrac{\sqrt{3}}{2}\right)^2}+\dfrac{1}{\left(\dfrac{1}{2}\right)^2}=\dfrac{16}{3}$.\\
Vậy $\mathrm{d}(SA,BC)=IH=\dfrac{\sqrt{3}}{4}\approx 0{,}43$.
}
\end{ex}

\begin{ex}%[1H8V5-4]%[TEX Đề Moon 2025]%[Võ Nguyên Thạch]
Cho hình chóp $S.ABCD$ có $SA\perp(ABCD)$, đáy $ABCD$ là hình chữ nhật và $AD=6$. Góc giữa cạnh bên $SD$ và mặt đáy bằng $30^\circ$. Khoảng cách giữa hai đường thẳng $AB$ và $SD$ bằng bao nhiêu?
\shortans{3}
\loigiai{
\begin{center}
\begin{tikzpicture}
\def\a{4}
\def\h{4}
\path 	(0:0) coordinate (A)
++(0:\a) coordinate (D)
++(-130:\a/2) coordinate (C)
($(A)+(C)-(D)$) coordinate (B)
($(A)+(90:\h)$) coordinate (S)
(intersection of A--C and B--D) coordinate (O)%giao điểm O
($(S)!(A)!(D)$) coordinate (H);
\draw[dashed,thick] 	(B)--(A)--(D)	(A)--(S) (A)--(H);
\draw[thick] 			(B)-- (C)--(D)
(B)--(S)	(C)--(S)	(D)--(S);
\foreach \x/\g in {A/135,B/-135,C/-45,D/45,S/90,H/45}
\fill[black] 	(\x) circle (1.5pt)
($(\g:3mm)+(\x)$) node {$\x$};
\draw pic["$30^\circ$",draw,angle eccentricity=1.8,angle radius=0.5cm]{angle=S--D--A};
\draw pic[draw,angle radius=3mm]{right angle=D--A--S};%Theo chiều dương
\end{tikzpicture}
\end{center}
Trong mặt phẳng $(SAD)$, kẻ $AH\perp SD=H$.\\
Ta có $\heva{&SA\perp (ABCD)\Rightarrow AH\perp AB\\&AD\perp AB.}$\\
Suy ra $AB\perp (SAD)$, mà $AH\subset (SAD)$ nên $AB\perp AH$.\\
Khi đó, $AH$ là đoạn vuông góc chung của $AB$ và $SD$.\\
Xét tam giác $SAD$ vuông tại $A$, có
\[\tan \widehat{SDA}=\dfrac{SA}{AD}\Leftrightarrow SA=AD\tan \widehat{SDA}=6\tan 30^\circ = 2\sqrt 3.\]
Xét tam giác $SAD$ vuông tại $A$ có đường cao $AH$
\[\dfrac{1}{AH^2}+\dfrac{1}{SA^2}+\dfrac{1}{AD^2}\Leftrightarrow AH=\dfrac{SA\cdot AD}{\sqrt{SA^2+AD^2}}=\dfrac{2\sqrt 3\cdot 6}{\sqrt{(2\sqrt 3)^2+6^2}}=3.\]
}
\end{ex}

\begin{ex}%[1D6V4-6]
Chị Lan vừa mua một chiếc máy tính xách tay mới với giá $25$ triệu đồng vào ngày $20/9/2022$ bằng thẻ tín dụng của ngân hàng Y. Thẻ tín dụng này được phát hành vào ngày $10/9/2022$. Ngân hàng Y có chế độ không tính lãi trong $45$ ngày đầu và cộng thêm khuyến mãi $15$ ngày tiếp theo không tính lãi. Sau thời gian này, ngân hàng sẽ tính lãi với lāi suất $18\%$ /năm (tính lāi kép theo ngày). Chị Lan dự định sē hoàn tiền cho ngân hàng vào ngày $10/12/2022$. Chị Lan phải trả thêm bao nhiêu nghìn đồng tiền lāi so với giá gốc cho ngân hàng vào ngày $10/12/2022$? \textit{(kết quả làm tròn đến hàng đơn vị)}.
\shortans{$260$}
\loigiai{
Ngân hàng miễn lãi trong $45$ ngày đầu và $15$ ngày khuyến mãi $60$ ngày miễn lãi.\\
Thời gian tính lãi là từ ngày $20/11/2022$ đến ngày $10/12/2022$ nên số ngày có lãi là $21$ ngày.\\
Lãi suất hàng năm là $18\%/$ năm.\\
Suy ra lãi suất hàng ngày là $\dfrac{18}{365}$.\\
Số tiền chị Lan phải trả cho ngân hàng vào ngày $10/12/2022$ là
\[T=25\cdot \left(1+\dfrac{18}{365} \%\right)^{21} \approx 25{,}260 \text{ (triệu đồng)}.\]
Số tiền lãi chị Lan phải trả cho ngân hàng là $25 \,260\,000-25\,000\,000=260\,000$ (đồng).
}
\end{ex}

\begin{ex}%[1D6V3-5]%[TEX Đề Moon 2025]%[Vũ Hồng Toàn]
Một người vay ngân hàng số tiền $350$ triệu đồng, hàng tháng (tính từ ngày gửi) người đó trả góp $8$ triệu đồng. Lãi suất cho số tiền chưa trả là $0{,}79\%$ một tháng và kỳ trả đầu tiên là cuối tháng thứ nhất. Biết số tiền phải trả ở kỳ cuối là $m$ triệu đồng thì người đó trả hết nợ ngân hàng. Tính giá trị $m$ ($m$ làm tròn đến hàng phần trăm).
\shortans{$7{,}14$}
\loigiai{
Gọi $A$ là số tiền vay ngân hàng, $B$ là số tiền trả trong mỗi chu kì, $r=79 \%=0{,}0079$ là lãi suất cho số tiền chưa trả trên một chu kì, $n$ là số kì trả nợ.\\
Số tiền còn nợ ngàn hàng (tính cả lãi) trong từng chu kì như sau
\begin{itemize}
\item Đầu kì thứ nhất là $A$.
\item Cuối kì thứ nhất là $A(1+r)- B$.
\item Cuối kì thứ hai là
$[A(1+r)-n](1+r)-A=A(1+r)-A[(1+r)+1]$.
\item Cuối kì thứ ba là
\[\left[A(1+r)^2-B[(1+r)+1]\right]\left(1+r\right)-B=A(1+r)^3-B\left[(1+r)^2+(1+r)+1\right].\]
$\ldots$
\item Theo giả thiết quy nạp, cuối kì thứ $n$ là
\[A(1+r)^n-B\left[(1+r)^{n-1}+\cdots+(1+r)+1\right]=A(1+r)^n-B\cdot  \dfrac{(1+r)^n-1}{r}.\]
\end{itemize}
Vậy số tiền còn nợ (tính cả lãi) sau $n$ chu kì là
$A(1+r)^n-B\cdot \dfrac{(1+r)^n-1}{r}$.\\
Người đó trả hết nợ ngân hàng khi
\allowdisplaybreaks
\begin{eqnarray*}
&&A(1+r)^n-B\cdot \dfrac{(1+r)^n-1}{r}=0\\
&\Leftrightarrow&(1+r)^n=\dfrac{B}{B-Ar}\\
&\Leftrightarrow& n =\log_{(1+r)}\dfrac{B}{Ar-B}=\log_{1{,}0079}\dfrac{8}{350\cdot 0{,}0079-8}\approx 53{,}9.
\end{eqnarray*}
Tức là phải mất $54$ tháng người này mới trả hết nợ.\\
Cuối tháng thứ $53$, số tiền còn nợ (tính cả lãi) là
\[S_{53}=350\cdot 1{,}0079^{53}-8 \cdot \dfrac{1{,}0079^{53}-1}{0{,}0079} \text{(triệu đồng)}.\]
Kì trả nợ tiếp theo là cuối tháng thứ $54$, khi đó người vay phải trả số tiền
$S_{53}$ và lãi của số tiền này nữa là $S_{53}+0{,}0079 \cdot S_{53}=S_{53} \cdot 1{,}0079 \approx 7{,}14$ (triệu đồng).
}
\end{ex}

\begin{ex}%[1D2V2-7]
\immini[thm]
{
An đã tạo ra một cầu thang $3$ bậc bằng $18$ que tăm như hình minh họa. Vậy An cần thêm bao nhiêu que tăm để hoàn thành một cầu thang $5$ bậc?
}
{
\includegraphics[scale=0.16]{images/de15-1}
}
\shortans[]{$22$}
\loigiai{
\begin{itemize}
\item \textbf{Cách 1}\\
Ta thấy rằng cầu thang $1$ bậc cần $4$ tăm và cầu thang $2$ bậc cần $10$ tăm. Do đó, để đi từ cầu thang $1$ bậc đến $2$ bậc cần thêm $6$ tăm và để đi từ cầu thang $2$ bậc đến $3$ bậc cần thêm $8$ tăm.\\
Áp dụng mô hình này, để đi từ cầu thang $3$ bậc đến $4$ bậc cần thêm $10$ tăm và để đi từ cầu thang $4$ bậc đến $5$ bậc cần thêm $12$ tăm.\\
Vậy bạn An cần thêm $10+12=22$ tăm.
\item \textbf{Cách 2}\\
Với cầu thang $3$ bậc có $2\left[2\cdot 3+2+1\right]=18$ tăm.\\
Tổng quát, ta thấy rằng cầu thang có $x$ bậc có
\[2\left[2x+(x-1)+(x-2)+\ldots+1\right]\,\text{tăm}.\]
Vì vậy đối với $x=5$ bậc ta có
\[2\left[2\cdot 5+4+3+2+1\right]=40\,\text{tăm}.\]
Vậy bạn An cần thêm $40-18=22$ tăm.
\item \textbf{Cách 3}\\
Ta thấy rằng để đến được $4$ các bước, ta thêm hai khối có ba que (phía trên và bên phải) va fhai khối nữa có hai khối toạ thành các bước. Điều này sẽ có thêm $2\cdot 3+2\cdot 2=10$ quê. \\
Sau đó,để đếm được $5$ các bước, ta thêm hai khối cạnh nữa có $3$ que và $3$ nữa có hai que. Ta thêm $2\cdot 3+3\cdot 2=12$ nhiều hơn nữa để tăng tổng cộng $10+12=22$ que.
\end{itemize}
}
\end{ex}

\begin{ex}%[1D2V1-6]
\immini{Một chai soda có giá $1$ đô. Sau khi uống, hai chai rỗng sẽ được đổi lấy một chai soda. Bạn có thể uống nhiều nhất bao nhiêu chai soda nếu bạn có $100$ đô?}{\begin{tikzpicture}[very thick,>=stealth',scale=0.7]
\filldraw[black] (-0.25,3.8)--(-0.25,3.5) arc (180:360:0.25 and 0.1)--(0.25,3.8) arc (0:-180:0.25 and 0.1);
\filldraw[blue!70!green!30] (-0.6,0.5)--(-0.6,-2) arc (180:0:0.6 and 0.3)--(0.6,0.5);
\filldraw[blue!50!green!50,,] (0,0.5) ellipse (0.6 and 0.3);
\filldraw[blue!50!green!50] (0,-2) ellipse (0.6 and 0.3);
\draw (-0.6,1) -- (-0.6,-2) (0.6,1) -- (0.6,-2);
\draw (0,-2) ellipse (0.6 and 0.3);
\draw (-0.6,1)--(-0.6,1.5)--(-0.25,2.8)--(-0.25,3.8);
\draw (0.6,1)--(0.6,1.5)--(0.25,2.8)--(0.25,3.8);
\draw (0,3.8) ellipse (0.25 and 0.1);
\end{tikzpicture}}
\shortans{199}
\loigiai{
\textbf{Mua số lượng chai soda ban đầu:}\\
Với $100$ đô ta sẽ mua được $100$ chai soda.\\
Đổi chai rỗng lấy soda: Sau khi uống $100$ chai, ta sẽ có $100$ chai rỗng, $100$ chai rỗng ta sẽ đổi được 50
chai soda mới.\\
\textbf{	Tiếp tục quy trình:}\\
Sau khi uống $50$ chai soda mới, ta được $50$ chai rỗng.
Đổi $50$ chai rỗng lấy $25$ chai soda mới.\\
\textbf{Lặp lại quy trình:}\\
Uống $25$ chai so da mới, ta được $25$ chai rỗng.\\
Đổi $25$ chai rỗng ta được $12$ chai soda mới và dư $1$ chai rỗng.\\
\textbf{	Tiếp tục đổi:}\\
Uống $12$ chai soda mới ta sẽ có $12$ chai rỗng.\\
Đổi $12$ chai rỗng ta được $6$ chai so đa mới.\\
\textbf{Lặp lại quy trình:}\\
Uống $6$ chai so da mới, ta được $6$ chai rỗng.\\
Đổi $6$ chai rỗng ta được $3$ chai soda mới.\\
\textbf{Tiếp tục đổi:}\\
Uống chai $3$ soda mới vừa đổi ta được $3$ chai rỗng.\\
Cộng với $1$ chai rỗng còn dư ở phía trên ta được $4$ chai rỗng và đổi thêm được $2$ chai soda mới.\\
\textbf{Kết thúc:}
Uống $2$ chai so da mới, ta được $2$ chai rỗng.\\
Đổi $2$ chai rỗng ta được 1 chai soda mới.\\
Uống chai $1$ soda mới vừa đổi ta được 1 chai rỗng.\\
Vậy ta có thể uống được nhiều nhất là $100 + 50 + 25 + 12 + 6 + 3 + 2 + 1 = 199$ chai.
}
\end{ex}

\begin{ex}%[1C2V3-1]
Một nhân viên của bảo tàng nghệ thuật đang có kế hoạch giới thiệu nội dung cuộc triển lãm của bảo tàng đến ba trường học trong khu vực. Người đó muốn đến từng trường và quay trở lại bảo tàng sau khi thăm cả ba trường. Thời gian di chuyển (đơn vị: phút) giữa các trường học và giữa bảo tàng với mỗi trường học được mô tả trong hình vẽ. Tìm thời gian đi ít nhất để thực hiện chu trình trên.
\begin{center}
\begin{tikzpicture}[scale=1,>=stealth, font=\footnotesize, line join=round, line cap=round]
\coordinate (A) at (0,0);
\coordinate (B) at (2,3);
\coordinate (C) at (2,-1);
\coordinate (D) at (3.5,2.5);
\draw (A)--(B)--(D)--(C)--(A) (A)--(D) (B)--(C) ($(A)!0.5!(B)$)node[above left]{$38$} ($(B)!0.6!(D)$)node[above]{$19$} ($(D)!0.5!(C)$)node[right]{$51$} ($(A)!0.5!(C)$)node[below left]{$32$} ($(A)!0.5!(D)$)node[above left,yshift=-0.1cm]{$46$} ($(B)!0.7!(C)$)node[left]{$50$};
\fill (A)node[left]{Trường $A$}circle(2pt);
\fill (B)node[above]{Trường $B$}circle(2pt);
\fill (C)node[below]{Trường $C$}circle(2pt);
\fill (D)node[right]{Bảo tàng}circle(2pt);
\end{tikzpicture}
\end{center}

\shortans{140}
\loigiai{
Gọi $A, B, C$ là ba trường học và $D$ là bảo tàng.
Nhân viên cần thực hiện một chu trình xuất phát từ $D$, đi qua $A, B, C$ (mỗi nơi đúng một lần) và quay trở về $D$.
Các chu trình có thể thực hiện và tổng thời gian tương ứng (tính bằng phút) là
\begin{itemize}
\item $D \to A \to B \to C \to D$: $46+38+50+51=185$.
\item $D \to A \to C \to B \to D$: $46+32+50+19=147$.
\item $D \to B \to A \to C \to D$: $19+38+32+51=140$.
\item $D \to B \to C \to A \to D$: $19+50+32+46=147$.
\item $D \to C \to A \to B \to D$: $51+32+38+19=140$.
\item $D \to C \to B \to A \to D$: $51+50+38+46=185$.
\end{itemize}
So sánh tổng thời gian của các chu trình, ta thấy thời gian đi ít nhất là $140$ phút.
}
\end{ex}

\begin{ex}%[1C2V3-1]%[TEX Đề Moon 2025]%[Võ Nguyên Thạch]
Một người đưa thư xuất phát từ bưu điện (vị trí $A$) và phải đi qua các con đường để phát thư rồi quay lại bưu điện. Sơ đồ các con đường cần đi qua và độ dài của chúng (tính theo mét) được biểu diễn ở hình vẽ dưới.
\begin{center}
\begin{tikzpicture}[scale=1,>=stealth, font=\footnotesize, line join=round, line cap=round]
\coordinate (F) at (0,0);
\coordinate (A) at (2,2);
\coordinate (B) at (5,2);
\coordinate (E) at (2,-2);
\coordinate (D) at (5,-2);
\coordinate (C) at (7,0);
\draw (F)--(A)--(B)--(C)--(D)--(E)--cycle (A)--(E) (E)--(B) (B)--(D) ($(F)!0.5!(A)$)node[above left]{$1\,000$} ($(A)!0.5!(B)$)node[above]{$200$} ($(B)!0.5!(C)$)node[above right]{$300$} ($(C)!0.5!(D)$)node[below right]{$400$} ($(B)!0.5!(D)$)node[right]{$1\,500$} ($(E)!0.5!(D)$)node[below]{$1\,600$} ($(E)!0.5!(B)$)node[above left]{$800$} ($(A)!0.5!(E)$)node[left]{$700$} ($(F)!0.5!(E)$)node[below left]{$900$};
\foreach \x/\g in {F/180,A/90,B/90,C/0,D/-90,E/-90}
\fill[black] (\x) circle(2pt) +(\g:4mm) node {$\x$};
\end{tikzpicture}
\end{center}
Hỏi người đó phải đi như thế nào để đường đi là ngắn nhất?
\shortans{8\,300}
\loigiai{
Đồ thị trên hình chỉ có hai đỉnh bậc lẻ là $A$ và $D$ nên ta có thể tìm được một đường đi Euler từ $A$ đến $D$ (đường đi này đi qua mỗi cạnh đúng một lần).\\
Một đường đi Euler từ $A$ đến $D$ là $AFEABEDBCD$ và tổng độ dài của nó là
\[1\,000+900+700+200+800+1\,600+1\,500+300+400=7\,400.\]
Để quay trở lại điểm xuất phát và có đường đi ngắn nhất, ta cần tìm một đường đi ngắn nhất từ $D$ đến $A$ theo thuật toán gắn nhãn vĩnh viễn.\\
Đường đi ngắn nhất từ $D$ đến $A$ là $DCBA$ và có độ dài là $400+300+200=900$.\\
Vậy một chu trình cần tìm là $AFEABEDBCDCBA$ và có độ dài là $7\,400+900=8\,300$.
}
\end{ex}

\begin{ex}%[1H8V5-4]%[TEX ĐỀ MOON 2025]%[Lê Hữu Kiệt]
Cho hình chóp $S.ABCD$ có đáy $ABCD$ là hình vuông và tam giác $SAB$ đều nằm trong mặt phẳng vuông góc với đáy. Biết khoảng cách giữa hai đường thẳng $SA$ và $BD$ bằng $\sqrt{21}$. Hỏi cạnh đáy của hình chóp đã cho bằng bao nhiêu?
\shortans{$7$}
\loigiai{
\immini
{Đặt $a$ là độ dài cạnh đáy của hình chóp ($a>0$).\\
Gọi $H$ là chân đường cao kẻ từ $S$ trong $\triangle SAB$ đều, khi đó $H$ là trung điểm của $AB$ và $SH=\dfrac{a\sqrt3}{2}$.\\
Gọi $E\in(ABCD)$ sao cho $ADBE$ là hình bình hành. Khi đó $BDallel AE$.\\
Mà $AE\subset (SAE)$ nên $BDallel (SAE)$.\\
Suy ra $\mathrm{d}(BD,SA)=\mathrm{d}\left(BD,(SAE)\right)=\mathrm{d}\left(B,(SAE)\right)$.\\
Ta có $\dfrac{AH}{AB}=\dfrac{1}{2}$ nên $\mathrm{d}\left(B,(SAE)\right)=2\mathrm{d}\left(H,(SAE)\right)$.}
{\begin{tikzpicture}[font=\footnotesize, line join=round, line cap=round, >=stealth, scale=1, declare function={a=3;cao=a*sqrt(3)/2;}]
\path
(0,0) coordinate (B) (-135:a/2) coordinate (A) (0:a) coordinate (C) ($(C)-(B)+(A)$) coordinate (D)
(barycentric cs:A=1,B=1) coordinate (H)+(0,cao) coordinate (S)
($(A)-(D)+(B)$) coordinate (E)
($(E)!3/4!(A)$) coordinate (F)
($(S)!(H)!(F)$) coordinate (K);
\draw (S)--(A)--(D)--(C)--cycle (S)--(D) (S)--(E)--(A) (S)--(F);
\draw[dashed] (A)--(B)--(C) (S)--(B)--(D) (S)--(H)--(F) (B)--(E) (H)--(K);
\foreach \x/\g in {S/90, A/-90, B/45, C/0, D/0, H/-45, E/180, F/-135, K/180}{\fill (\x) circle (1pt)+(\g:0.3)node{$\x$};}
\end{tikzpicture}}
\noindent
Kẻ $HF\perp AE$, khi đó $\heva{&HF\perp AE\\& SH\perp AE\\&HF\cap SH=H}$ suy ra $AE\perp(SHF)$.\\
Suy ra $(SAE)\perp(SHF)$ và $(SAE)\cap(SHF)=SF$.\\
Khi đó $\mathrm{d}\left(H,(SAE\right)=\mathrm{d}(H,SF)$.\\
Kẻ $HK\perp SF$, suy ra $\mathrm{d}(H,SF)=HK$.\\
Xét $\triangle ABE$ ta có $AB=AE=a$, $\widehat{ABE}=90^\circ$ nên $\triangle ADE$ vuông cân tại $B$. Suy ra $\widehat{HAF}=45^\circ$.\\
Xét $\triangle HAF$ vuông tại $F$ ta có $HF=HA\sin\widehat{HAF}=\dfrac{a}{2}\sin45^\circ=\dfrac{a\sqrt2}{4}$.\\
Xét $\triangle SHF$ vuông tại $H$ ta có
\begin{eqnarray*}
&&\dfrac{1}{HK^2}=\dfrac{1}{SH^2}+\dfrac{1}{HF^2} \\
&\Leftrightarrow&\dfrac{1}{HK^2}=\dfrac{1}{\left(\dfrac{a\sqrt3}{2}\right)^2}+\dfrac{1}{\left(\dfrac{a\sqrt2}{4}\right)^2}\\
&\Leftrightarrow&\dfrac{1}{HK^2}=\dfrac{28}{3a^2} \\
&\Leftrightarrow&HK=\dfrac{a\sqrt{21}}{14}.
\end{eqnarray*}
Suy ra $\mathrm{d}(BD,SA)=2HK=\dfrac{a\sqrt{21}}{7}$.\\
Theo giả thiết ta có $\mathrm{d}(BD,SA)=\sqrt{21}$, suy ra $a=7$.
}
\end{ex}

\begin{ex}%[50 Đề minh họa tốt nghiệp 2025 - Đề 13]%[Lê Hữu Kiệt - Lê Quân]%[2H5V3-4]
Trong hệ toạ độ $Oxyz$, có một mặt cầu $(S)\colon (x-1)^2+(y-2)^2+(z+1)^2=3$ và đường thẳng $\Delta\colon \dfrac{x+4}{6}=\dfrac{y-6}{-2}=\dfrac{z-2}{-1}$. Từ điểm $M\in \Delta$ kẻ các tiếp tuyến đến mặt cầu $(S)$ và gọi $(C)$ là tập hợp các tiếp điểm. Biết diện tích hình phẳng giới hạn bởi $(C)$ đạt giá trị nhỏ nhất thì $(C)$ nằm trên mặt phẳng $x+by+cz+d=0$. Tìm $b+c+d$.
\par\shortans{$-2$}
\loigiai{
Ta có mặt cầu $(S)$ có tâm $I(1;2;-1)$, bán kính $R=\sqrt3$; đường thẳng $\Delta$ có $\overrightarrow{u}=(6;-2;-1)$ là một vectơ chỉ phương.\\
Hình phẳng được giới hạn bởi $(C)$ là một hình tròn. Gọi $AB$ là đường kính đường tròn $(C)$.\\
Với $M\in\Delta$, gọi $H=AB\cap IM$, khi đó $H$ là tâm đường tròn $(C)$.
\begin{center}
\begin{tikzpicture}[font=\footnotesize, line join=round, line cap=round, >=stealth, scale=1]
\pgfmathsetmacro\bankinh{sqrt(3)}
\pgfmathsetmacro\goc{acos(\bankinh/3)}
\path (0,0) coordinate (I) (\goc:\bankinh) coordinate (A) (-\goc:\bankinh) coordinate (B) (3,0) coordinate (M)
(intersection of A--B and I--M) coordinate (H)
pic[draw, angle radius=2mm]{right angle=A--H--I}
pic[draw, angle radius=2mm]{right angle=I--A--M}
;
\draw (I) circle (\bankinh) (I)--(A)--(M)--(B)--cycle (A)--(H) (I)--(M) (A)--(B);
\foreach \x/\g in {I/180, M/0, A/60, B/-60, H/45}{
\fill (\x) circle (1pt)+(\g:0.3)node{$\x$};
}
\end{tikzpicture}
\end{center}
Diện tích hình tròn $(C)$ là $S_{(C)}=\pi\cdot AH^2$. Do đó $S_{(C)}$ đạt giá trị nhỏ nhất khi và chỉ khi $AH$ đạt giá trị nhỏ nhất.\\
Ta có $AH$ đạt giá trị nhỏ nhất $\Leftrightarrow IM$ đạt giá trị nhỏ nhất.\\
Suy ra $M$ là hình chiếu của $I$ trên $\Delta$.\\
Ta có $M\in\Delta$ nên $M(-4+6t;6-2t;2-t)$. Khi đó $\overrightarrow{IM}=(-5+6t;4-2t;3-t)$.\\
Do $\overrightarrow{IM}\perp\overrightarrow{u} \Leftrightarrow 6(-5+6t)-2(4-2t)-1(3-t)=0 \Leftrightarrow t=1$.\\
Do đó $M(2;4;1)$ và $\overrightarrow{IM}=(1;2;2)$, suy ra $IM=3$.\\
Xét $\triangle IAM$ vuông tại $A$, ta có $IH=\dfrac{IA^2}{IM}=1$.\\
Do $\overrightarrow{IH}$ và $\overrightarrow{IM}$ cùng phương, $IH=\dfrac{1}{3}IM$, suy ra $\overrightarrow{IH}=\dfrac{1}{3}\overrightarrow{IM}$.\\
Suy ra $H\left(\dfrac{4}{3};\dfrac{8}{3};-\dfrac{1}{3}\right)$.\\
Mặt phẳng chứa $(C)$ đi qua điểm $H$ và nhận $\overrightarrow{IM}$ là vectơ pháp tuyến có dạng
\begin{eqnarray*}
&& 1\left(x-\dfrac{4}{3}\right)+2\left(y-\dfrac{8}{3}\right)+2\left(z+\dfrac{1}{3}\right) = 0 \\
&\Leftrightarrow& x+2y+2z-6=0.
\end{eqnarray*}
Suy ra $b=2$, $c=2$, $d=-6$. Vậy $T=b+c+d=-2$.
}
\end{ex}

\begin{ex}%[2H5V3-4]%[TEX Đề Moon 2025]%[Võ Nguyên Thạch]
Trong không gian với hệ tọa độ $Oxyz$, đài kiểm soát không lưu sân bay có tọa độ $O(0;0;0)$, mỗi đơn vị trên một trục ứng với $1$ km. Máy bay bay trong phạm vi cách đài kiểm soát $417$ km sẽ hiển thị trên màn hình ra đa. Một máy bay đang ở vị trí $A(-688;-185;8)$, chuyển động theo đường thẳng $d$ có véc-tơ chỉ phương là $\overrightarrow{u}=(91;75;0)$ và theo hướng về đài không lưu. Biết $E(a;b;c)$ là vị trí sớm nhất mà máy bay xuất hiện trên màn hình. Tính $T=a+b+c$.

\shortans{-367}
\loigiai{
Đường thẳng $d$ đi qua điểm $A(-688;-185;8)$, có một véc-tơ chỉ phương $\overrightarrow{u}=(91;75;0)$ có phương tình tham số là
\[\heva{&x=-688+91t\\&y=-185+75t\\&z=8}\quad (t \text{ là tham số.})\]
Gọi $B$ là vị trí sớm nhất mà máy bay xuất hiện trên màn hình ra đa.\\
Vì $B$ thuộc $d$ nên $B(-688+91t;-185+75t;8)$.\\
Để $B$ là vị trí sớm nhất mà máy bay xuất hiện trên màn hình ra đa thì $OB=417$.
\allowdisplaybreaks
\begin{eqnarray*}
\text{Do đó }\sqrt{(-688+91t)^2+(-185+75t)^2+8^2}=417&\Leftrightarrow& 13\,906t^2-152\,966t+333\,744=0\\
&\Leftrightarrow&t=3 \text{ hoặc }t=8.
\end{eqnarray*}
Với $t=3$, ta có $B(-415;40;8)$ và $AB=\sqrt{(-415+688)^2+(40+185)^2}=\sqrt{125\,154}$.\\
Với $t=8$ ta có $B(40;415;8)$ và $AB=\sqrt{(40+688)^2+(415+185)^2}=\sqrt{889\,984}$.\\
Vì $\sqrt{125\,154}<\sqrt{889\,984}$ nên tọa độ vị trí sớm nhât mà máy bay xuất hiện trên màn hình ra đa là $(-415;40;8)$.\\
Khi đó $a=-415$; $b=40$; $c=8$.\\
Suy ra $T=a+b+c=-415+40+8=-367$.
}
\end{ex}

\begin{ex}%[2H5V2-8]%[TEX Đề Moon 2025]%[Vũ Hồng Toàn]
Trong không gian với một hệ trục tọa độ $Oxyz$ cho trước (đơn vị trên các trục tính bằng kilomet), ra đa phát hiện một chiếc máy bay di chuyển với vận tốc và hướng không đổi từ điểm $A(800;500;7)$ bay thẳng đến điểm $B(940;550;8)$ trong $10$ phút. Nếu máy bay tiếp tục giữ nguyên vận tốc và hướng bay thì sau $5$ phút tiếp theo, khoảng cách từ máy bay đến gốc tọa độ $O$ bằng bao nhiêu kilomet? (kết quả làm tròn đến hàng đơn vị).
\shortans{$1162$}
\loigiai{
Vị trí của máy bay sau $5$ phút tiếp theo là $C(x ; y ; z)$.\\
Vì hướng của máy bay không đổi nên $\overrightarrow{AB}$ và $\overrightarrow{BC}$ cùng hướng.\\
Do vận tốc của máy bay không đổi và thời gian bay từ $A$ đến $B$ gấp đôi thời gian bay từ $B$ đến $C$ nên $AB=2 BC$.\\
Do đó $\overrightarrow{BC}=\frac{1}{2} \overrightarrow{A B}=\left(\dfrac{940-800}{2} ; \dfrac{550-500}{2} ; \frac{9-7}{2}\right)=(70 ; 25 ; 1)$.\\
Mặt khác, $\overrightarrow{BC}=(x-940 ; y-550 ; z-9)$ nên $\heva{&x-940=70 \\& y-550=25 \\& z-9=1}\Rightarrow\heva{&x=1\,010 \\& y=575 \\& z=10.}$\\
Vậy $OC=\sqrt{1\,101^2+575^2+10^2}\approx 1162$\,(km).
}
\end{ex}

\begin{ex}%[2H5V2-7]
Một lều trại có mặt trước và mặt sau rộng $4$ m, hai mặt bên rộng $3$ m gồm sáu thanh cọc tre, vải bạt chống thấm nước, dây dù hoặc dây thừng để cố định lều tại sáu cọc sắt cắm sát đất như hình vẽ. Biết rằng, hai thanh $AF$, $OC$ có chiều dài $2{,}2$ m, bốn thanh còn lại có chiều dài $1{,}7$ m và đoạn dây thừng $IF=2{,}75$ m.  Chọn hệ trục tọa độ $Oxyz$ như hình vẽ và cho biết góc giữa đường thẳng chứa dây thừng $IF$ và mặt phẳng chứa tấm bạt $(CDEF)$ là $\alpha$.
\begin{center}
\begin{tikzpicture}[line join = round, line cap=round,>=stealth,font=\footnotesize,scale=1]
\path
(0,0) coordinate (O)
(-5,-1) coordinate (x)
(2,-3) coordinate (y)
(0,5) coordinate (z)
%	($(A)+(B)-(O)$) coordinate (N)
%	($(N)+(0,4)$) coordinate (M)
;
\coordinate (A) at ($(O)!5/7!(x)$);
\coordinate (I) at ($(O)!6.3/7!(x)$);
\coordinate (B) at ($(O)!3/5!(y)$);
\coordinate (a1) at	($(A)+(B)-(O)$);
\coordinate (F) at ($(A)+(0,3)$);
\coordinate (C) at ($(O)+(0,3)$);
\coordinate (D) at ($(B)+(0,2.5)$);
\coordinate (E) at ($(a1)+(0,2.5)$);
\coordinate (b1) at ($(A)!-1!(a1)$);
\coordinate (b2) at ($(b1)+(0,2)$);
\coordinate (c1) at	($(b1)+(O)-(A)$);
\coordinate (c2) at ($(c1)+(0,2)$);
\draw[->] (O)--(x)node[below left]{$x$};
\draw[->] (O)--(y)node[below left]{$y$};
\draw[->] (O)--(z)node[above left]{$z$};
\draw(O)--(c1)-- (b1)--(A)--(a1)--(B) (c2)--(b2)--(F)--(E)--(D)--(C)--(c2) (F)--(C) (F)--(I);
\draw[line width=2pt] (A)--(F) (a1)--(E) (O)--(C) (B)--(D) (b1)--(b2) (c1)--(c2);
\draw(E)--(-3.5,-3)  (b2)--(-5.5,2) (D)--(1.8,-1.7) (C)--(1.5,1.2) (c2)--(-.5,2);
\path (A)--(b1) node [below left ,sloped,pos=0.1] {$4$m};
\path (a1)--(B) node [below  ,sloped,pos=0.5] {$3$m};
\foreach \i/\g in {A/60,B/-90,C/30,D/30,E/70,F/90,I/-90}{\draw[fill=black](\i) circle (1.5pt) ($(\i)+(\g:3mm)$) node[scale=1]{$\i$};}
\end{tikzpicture}
\end{center}
Tính giá trị của $\alpha$ (tính theo đơn vị độ và làm tròn kết quả đến hàng đơn vị của độ). \shortans{$51^\circ$}
\loigiai{
Ta có $AI=\sqrt{IF^2-AF^2}=\sqrt{2{,}75^2-2{,}2^2}=1{,}65$ (m).\\
Do đó $A(3; 0; 0)$, $I(4{,}65; 0; 0)$, $B(0; 2; 0)$, $E(3; 2; 1{,}7)$, $F(3; 0; 2{,}2)$ và $C(0; 0; 2{,}2)$.\\
Suy ra $\overrightarrow{IF}=(-1{,}65; 0; 2{,}2)$, $\overrightarrow{EF}=(0; -2; 0{,}5)$ và $\overrightarrow{EC}=(-3; -2; 0{,}5)$.\\
Lại có $\left[\overrightarrow{EF}, \overrightarrow{EC}\right]=(0; -1{,}5; -6)$ nên $\overrightarrow{n}=(0; 1; 4)$ là một vectơ pháp tuyến của mặt phẳng $(CDEF)$.\\
Suy ra
\[\sin\left(IF, (CDEF)\right)=\dfrac{\left|\overrightarrow{IF}\cdot \overrightarrow{n}\right|}{\left|\overrightarrow{IF}\right|\cdot \left|\overrightarrow{n}\right|}=\dfrac{\left|-1{,}65\cdot 0+0\cdot 1+2{,}2\cdot4\right|}{\sqrt{(-1{,}65)^2+0^2+2{,}2^2}\cdot\sqrt{0^2+1^2+4^2}}=\dfrac{16\sqrt{17}}{85}.\]
Do đó $\left(IF, (CDEF)\right)\approx 51^\circ\Rightarrow \alpha\approx 51^\circ$.
}
\end{ex}

\begin{ex}%[2H5V2-3]
Trong không gian $O x y z$, cho tam giác $ABC$ vuông tại $A$, $ABC=30^\circ,BC=3\sqrt{2}$, đường thẳng $BC$ có phương trình $\dfrac{x-4}{1}=\dfrac{y-5}{1}=\dfrac{z+7}{-4}$, đường thẳng $AB$ nằm trong mặt phẳng $(\alpha)\colon  x+z-3=0$. Biết đỉnh $C$ có cao độ âm. Tính hoành độ đỉnh $A$.
\shortans{$4{,}5$}
\loigiai{
Tọa độ điểm $B$ là nghiệm của hệ phương trình $\heva{&\dfrac{x-4}{1}=\dfrac{y-5}{1}=\dfrac{z+7}{-4}\\&x+z-3=0}\Rightarrow B(2;3;1)$.\\
Do $ABC=30^\circ$ nên
\begin{eqnarray*}
\heva{&AB=\dfrac{3\sqrt{6}}{2}\\&AC=\dfrac{3\sqrt{2}}{2}}&\Leftrightarrow&\heva{&(x-2)^2+(y-3)^2+(2-x)^2=\dfrac{27}{2}\\&(x-3)^2+(y-4)^2+(6-x)^2=\dfrac{9}{2}}\\
&\Leftrightarrow&\heva{&2x^2-8y+y^2-6y+\dfrac{7}{2}=0\\&2x^2-18x+y^2-8y+\dfrac{113}{2}=0}\Leftrightarrow\heva{&10x+2y-53=0\quad (1)\\&2x^2-8y+y^2-6y+\dfrac{7}{2}=0.\quad (2)}
\end{eqnarray*}
Từ $(1)$ ta có $y=\dfrac{53-10x}{2}$, thay vào $(2)$
ta có
\begin{eqnarray*}
&&2x^2-8x+\left(\dfrac{53-10x}{2}\right)^2-6\cdot\dfrac{53-10x}{x}+\dfrac{7}{2}=0\\
&\Leftrightarrow& 108 x^2-972 x+2187=0\\
&\Leftrightarrow&(2 x-9)^2=0\\
&\Leftrightarrow& x=\dfrac{9}{2}.
\end{eqnarray*}
Do đó $A\left(\dfrac{9}{2};4;-\dfrac{3}{2}\right)$.\\
Vậy hoành độ đỉnh $A$ là $\dfrac{9}{2}=4{,}5$.
}
\end{ex}

\begin{ex}%[2H5V1-2]
Trong không gian $Oxyz$, cho hình lăng trụ tam giác đều $A_1B_1C_1$ có $A_1(\sqrt{3};-1;1)$, hai đỉnh $B$, $C$ thuộc trục $Oz$ và $AA_1=1$ ($C$ không trùng với $O$). Biết $\vec{u}=(a;b;10)$ là một vectơ chỉ phương của đường thẳng $A_1C$. Giá trị của $a^2 + b^2$ bằng bao nhiêu?

\shortans{400}
\loigiai{
\begin{center}
\begin{tikzpicture}[scale=0.8, font=\footnotesize,line join=round, line cap=round, >=stealth]
\path
(0,0) coordinate (A)
++(-120:2) coordinate (B)
(3,0) coordinate (C)
($(B)!.5!(C)$)coordinate (M)
;
\foreach \i in{A,B,C}{
\coordinate (\i_1) at ($(\i)+(0,3)$);
};
\draw (B)--(B_1) (A)--(B)--(C)--(C_1) (A_1)--(B_1)--(C_1)--(A_1);
\draw[dashed] (A)--(C) (A)--(M)--(A_1)--(A);
\foreach \i/\g in {A/-180,B/-90,C/-90,A_1/90,B_1/120,C_1/90,M/-90}
\fill[black] (\i) circle(1pt)+(\g:4mm)node[scale=1]{$\i$};
\end{tikzpicture}
\end{center}
Gọi $M$ là trung điểm $BC$ khi đó $AM$ vuông góc với $BC$.\\
Ta có $\heva{&AA_1\perp BC\\&AM\perp BC}\Rightarrow BC\perp (AA_1M)$.\\
Mặt phẳng $(A_1AM)$ qua $A_1$ và nhận $\overrightarrow{k}=(0;0;1)$ làm VTPT nên $(A_1AM)\colon z-1=0$.\\
Mà $M=(A_1AM)\cap Oz$ nên $M(0;0;1)$ suy ra $A_1M=2$.\\
Trong tam giác $A_1AM$ có $AM=\sqrt{A_1M^2-AA_1^2}=\sqrt{3}$.\\
Ta có tam giác $ABC$ đều nên $AM=\dfrac{BC\sqrt{3}}{2}\Rightarrow BC=\dfrac{2AM}{\sqrt{3}}=2$.\\
Gọi $B(0;0;m)$ mà $M$ là trung điểm $BC$ nên $C(0;0;2-m)$.\\
Ta có $BC=|2-2m|=2\Leftrightarrow\hoac{&m=0\\&m=2}\Rightarrow B(0;0;0)$, $C(0;0;2)$ vì $C$ không trùng với $O$.\\
Do đó $\overrightarrow{A_1C}=\left(-\sqrt{3};1;1 \right)=\dfrac{1}{10}\left( -10\sqrt{3};10;10\right)\Rightarrow\heva{&a=-10\sqrt{3}\\&b=10}$.\\
Vậy $a^+b^2=400$.
}
\end{ex}

\begin{ex}%[2H2V2-6]%[TEX ĐỀ MOON 2025]%[Nguyễn Văn Hiệp]
Trong không gian với một hệ trục tọa độ cho trước (đơn vị tính bằng mét), một con chim đang bay với tốc độ và hướng không đổi từ điểm $A(20;40;30)$ đến điểm $B(40;50;50)$ trong vòng $4$ phút. Nếu con chim bay tiếp tục giữ nguyên vận tốc và hướng bay thì sau $2$ phút con chim ở vị trí $C(a;b;c)$. Tổng $a+b+c$ bằng bao nhiêu?
\shortans{$165$}
\loigiai{
\textbf{Bước 1: Tính vectơ vận tốc} \\
$\overrightarrow{AB} = (20; 10; 20)$ \\
Vận tốc trung bình:
\[
\overrightarrow{v} = \dfrac{\overrightarrow{AB}}{4} = (5; 2{,}5; 5) \text{ (m/phút)}
\]
\textbf{Bước 2: Tính vị trí sau 6 phút} \\
Tọa độ điểm $C$
\[
\heva{
&a = 20 + 6 \times 5 = 50 \\
&b = 40 + 6 \times 2{,}5 = 55 \\
&c = 30 + 6 \times 5 = 60.
}
\]
Vậy $C(50;55;60)$. Suy ra $a + b + c = 165$.
}
\end{ex}

\begin{ex}%[2H2V2-6]%[TEX ĐỀ MOON 2025]%[Nguyễn Cường]
Hệ thống định vị toàn cầu GPS là một hệ thống cho phép xác định vị trí của một vật thể trong không gian. Trong cùng một thời điểm, vị trí của một điểm $M$ trong không gian sẽ được xác định bởi bốn vệ tinh cho trước nhờ các bộ thu phát tín hiệu đặt trên các vệ tinh. Giả sử trong không gian với hệ tọa độ $Oxyz$, có bốn vệ tinh lần lượt đặt tại các điểm $A(3;1;0)$, $B(3;6;6)$, $C(4;6;2)$, $D(6;2;14)$; vị trí $M(a;b;c)$ thỏa mãn $MA=3$, $MB=6$, $MC=5$, $MD=13$. Khoảng cách từ điểm $M$ đến điểm $O$ bằng bao nhiêu?
\shortans{$3$}
\loigiai{
Giả sử $M(a;b;c)$. Ta có hệ phương trình
\allowdisplaybreaks
\begin{eqnarray*}
\heva{&MA=3\\&MB=6\\&MB=5\\&MD=13}&\Leftrightarrow&\heva{& \sqrt{(a-3)^2+(b-1)^2+c^2}=3\\&\sqrt{(a-3)^2+(b-6)^2+(c-6)^2}=6\\&\sqrt{(a-4)^3+(b-6)^2+(c-2)^2}=5\\&\sqrt{(a-6)^3+(b-2)^2+(c-14)^2}=13}\\
&\Leftrightarrow&\heva{&a^2+b^2+c^2-6a-2b+1=0\\&a^2+b^2+c^2-6a-12b-12c+45=0\\&a^2+b^2+c^2-8a-12b-4c+31=0\\&a^2+b^2+c^2-12a-4b-28c+67=0}
\end{eqnarray*}
Giữ nguyên phương trình thứ nhất, lấy phương trình thứ nhất trừ vế theo vế với các phương trình còn lại ta được hệ phương trình mới như sau
\[\heva{
& a^2+b^2+c^2-6a-2b+1=0 \\
& 10b+12c=44 \\
& 2a+10b+4c=30 \\
& 6a+2b+28c=66
}\Leftrightarrow \heva{
& a^2+b^2+c^2-6a-2b+1=0 \\
& a=1 \\
& b=2 \\
& c=2.
}\]
Thế $a=1$, $b=2$, $c=2$ vào phương trình thứ nhất ta thấy thoả mãn.
\\
Vậy điểm $M(1;2;2)\Rightarrow OM=\sqrt{1+4+4}=3$.
}
\end{ex}

\begin{ex}%[2H2V2-6]%[TEX ĐỀ MOON 2025]%[Nguyễn Thế Duy]
Hai chiếc máy bay không người lái cùng bay lên tại một địa điểm. Sau một thời gian bay, chiếc máy bay thứ nhất cách điểm xuất phát về phía Bắc $20$ km và về phía Tây $10$ km, đồng thời cách mặt đất $0{,}7$ km. Chiếc máy bay thứ hai cách điểm xuất phát về phía Đông $30$ km và về phía Nam $25$ km, đồng thời cách mặt đất $1$ km. Hỏi hai chiếc máy bay cách nhau bao nhiêu km? (Làm tròn kết quả đến hàng đơn vị).
\shortans{$60$}
\loigiai{
\immini{Xét hệ trục toạ độ $Oxyz$ có $O$ trùng với điểm hai máy bay xuất phát; chiều dương của $Ox$ chỉ hướng Nam; chiều dương của $Oy$ chỉ hướng Đông; chiều dương $Oz$ chỉ cao độ.\\
Khi đó máy bay thứ nhất tại thời điểm đang xét có toạ độ là điểm $A\left(-20; -10; 0{,}7 \right)$. toạ độ của máy bay thứ hai là $B\left(25; 30; 1 \right)$.\\
Khoảng cách của hai chiếc máy bay là\\
$AB =\sqrt{45^2 + 40^2 + 0{,}3^2} \approx 60$ km.}
{\begin{tikzpicture}[scale=0.9,>=stealth, font=\footnotesize, line join=round, line cap=round]
\draw[->]
(0,0) -- (4,0) node[below]{$y$};
\draw[->]
(0,0) -- (-2.5,-2.5) node[below]{$x$};
\draw[->] (0,0) -- (0,4) node[above]{$z$};
\path
(-2,2.5) coordinate (A)
(3,2) coordinate (B)
;
\draw[dashed]
(-3,0) -- (0,0) -- (2,2)
;
\fill
(A) circle(1pt) node[above]{$A$}
(B) circle(1pt) node[above]{$B$}
;
\end{tikzpicture}}
}
\end{ex}

\begin{ex}%[2H2V2-2]
Chiếc nón lá có dạng hình nón $(N)$ được đặt trong không gian với hệ trục tọa độ $Oxyz$, biết đỉnh của chiếc nón là điểm $S(1;2;3)$, $A(2;2;3)$ và $B(1;4;3)$ là các điểm nằm trên mặt xung quanh của chiếc nón, điểm $C(1;2;6)$ nằm trên đường tròn đáy. Diện tích xung quanh của chiếc nón bằng bao nhiêu? (Làm tròn kết quả đến hàng phần mười).
\shortans[]{$23{,}1$}
\loigiai{
\begin{center}
\begin{tikzpicture}[scale=0.8,>=stealth, font=\footnotesize, line join=round, line cap=round]
\def\a{3}
\def\b{1}
\def\h{5}
\pgfmathsetmacro\gtt{asin(\b/\h)};
\pgfmathsetmacro\xtt{\a*cos(\gtt)};
\pgfmathsetmacro\ytt{\b*sin(\gtt)};
\path (0,0) coordinate (O)
(0,\h)coordinate (S)
(-60:{\a} and {\b}) coordinate (B')
(240:{\a} and {\b})coordinate (C)
(-\a,0) coordinate (A')
(\a,0) coordinate (N)
($(S)!0.3!(A')$) coordinate (A)
($(S)!0.5!(B')$) coordinate (B)
;
\clip (-\a-0.5,-\b-0.5) rectangle (\a+0.5,\h+0.6);
\draw (-\xtt,\ytt) arc (-180-\gtt:\gtt:\a cm and \b cm);
\draw[dashed] (\xtt,\ytt) arc (\gtt:180-\gtt:\a cm and \b cm);
\draw (\xtt,\ytt)--(S)--(-\xtt,\ytt) (S)--(B') (S)--(C);
\draw[dashed] (O)--(S);
\foreach \x/\g in {S/90,A/180,B'/-90,O/-90,A'/180,C/-90,B/0} \fill[black] (\x) circle (1pt) ($(\x)+(\g:3mm)$)node{$\x$};
\end{tikzpicture}
\end{center}
Ta có
\begin{itemize}
\item $\vec{SC}=(0;0;3)\Rightarrow \ell=SC=3$.
\item $\vec{SA}=(1;0;0)\Rightarrow SA=1$.
\item $\vec{SB}=(0;2;0)\Rightarrow SB=2$.
\end{itemize}
Dễ thấy $SA$, $SB$, $SC$ đôi một vuông góc tại $S$. Lấy $A'$, $B'$ thoả mãn $\vec{SA'}=3\vec{SA}$, $\vec{SB'}=\dfrac{3}{2}\vec{SB}$ suy ra $A'$, $B'$ nằm trên đường tròn đáy hình nón. \\
Vậy đáy hình nón là đường tròn ngoại tiếp tam giác $CA'B'$. Các tam giác $CSA'$, $CSB'$, $A'SB'$ là các tam giác bằng nhau và đều vuông cân tại đỉnh $S$ nên tam giác $CA'B'$ là tam giác đều cạnh bằng $3\sqrt{2}$.\\
Từ đó ta tính được bán kính đường tròn ngoại tiếp tam giác $CA'B'$ bằng $r=\dfrac{2}{3}\cdot \dfrac{3\sqrt{2}\cdot \sqrt{3}}{2}=\sqrt{6}$.\\
Diện tích xung quanh hình nón $(N)$ là $S=\pi r\ell=3\pi\sqrt{6}\approx 23{,}1$.
}
\end{ex}

\begin{ex}%[2H2V1-4]%[TexDeMoon2025]%[NguyenKieuNhaTu]
Người ta cần lắp một camera phía trên sân bóng để phát sóng truyền hình một trận bóng đá, camera có thể di động để luôn thu được hình ảnh rõ nét về diễn biến trên sân. Các kĩ sư dự định trồng bốn chiếc cột cao $30$ m và sử dụng hệ thống cáp gắn vào bốn đầu cột để giữ camera ở vị trí mong muốn. Mô hình thiết kế được xây dựng như sau: Trong hệ trục toạ độ $Oxyz$ (đơn vị độ dài trên mỗi trục là $1$ m), các đỉnh của bốn chiếc cột lần lượt là các điểm $M(90;0;30)$, $N(90;120;30)$, $P(0;120;30)$, $Q(0;0;30)$ (như hình vẽ). Giả sử $K_0$ là vị trí ban đầu của camera có cao độ bằng $25$ và $K_0M=K_0N=K_0P=K_0Q$. Để theo dõi quả bóng đến vị trí $A$, camera được hạ thấp theo phương thẳng đứng xuống điểm $K_1$ có cao độ bằng $19$.

{\centering \begin{tikzpicture}[scale=0.6,>=stealth, font=\footnotesize, line join=round, line cap=round]
\def\h{3};
\draw[->] (0,0)node[above right]{$O$}--(-3,-3) node [left]{$x$};
\draw[->] (0,0)--(9,0) node [above]{$y$};
\draw[->] (0,0)--(0,4) node [left]{$z$};
\fill[green!70,opacity=0.6] (0.25,-0.5)--(6.25,-0.5)--(4.75,-2.5)--(-1.75,-2.5)--cycle;
\coordinate (O) at (0,0);
\coordinate (M') at (-2.5,-2.5);
\coordinate (P') at (7.5,0);
\coordinate (N') at ($(M')+(P')-(O)$);
\coordinate (M) at ($(M')+(0,\h)$);
\coordinate (N) at ($(N')+(0,\h)$);
\coordinate (P) at ($(P')+(0,\h)$);
\coordinate (Q) at ($(O)+(0,\h)$);
\coordinate (I) at ($(Q)!0.5!(N)$);
\coordinate (K0) at ($(I)+(0,-1)$);
\coordinate (K1) at ($(I)+(0,-2.5)$);
\foreach \x in {M,N,P}
\draw[line width=3pt,red] (\x)--(\x');
\draw[line width=3pt,red] (O)--(Q);
\foreach \x/\g in {M/90,N/90,P/90,Q/180}
\fill[black] (\x) circle(3pt) +(\g:4mm) node {$\x$};
\draw (M)--(K0) (P)--(K0) (Q)--(K0) (N)--(K0);
\draw[dashed] (M)--(K1) (P)--(K1) (Q)--(K1) (N)--(K1);
\fill[black] (K0) circle(3pt) +(90:4mm) node {$K_0$};
\fill[black] (K1) circle(3pt) +(-90:4mm) node {$K_1$};
\fill[black] ($(K1)+(3,0)$) circle(3pt) +(-90:4mm) node {$A$};
\end{tikzpicture}\par}\vspace{-5pt}\noindent
Biết trung điểm đoạn $K_0K_1$ có tọa độ là $(a;b;c)$. Khi đó hãy tính giá trị biểu thức $T=5a+7b+9c$.
\shortans[]{$843$}
\loigiai{
\begin{center}
\begin{tikzpicture}[scale=0.7, font=\footnotesize,line join=round, line cap=round, >=stealth]
\path
(0,0) coordinate (Q)
++(-130:4) coordinate (M)
++(0:6) coordinate (N)
($(Q)+(N)-(M)$) coordinate (P)
($(Q)!1/2!(N)$) coordinate (K')
(0,-6) coordinate (O)
;
\foreach \i in {M,N,P}{
\coordinate (\i_1) at ($(\i)-(0,6)$);
}
\path
($(O)!.5!(M_1)$) coordinate (A)
($(O)!.5!(P_1)$) coordinate (B)
($(O)!.5!(N_1)$) coordinate (K)
($(K)!.3!(K')$) coordinate (K_1)
($(K)!2/3!(K')$) coordinate (K_0)
;
\draw (Q)--(M)--(N)--(P)--cycle;
\draw (Q)--(N) (P)--(M) (M)--(M_1) (N)--(N_1)
(P)--(P_1)
(M_1)--(N_1)--(P_1);
\foreach \i in{M,N,P,Q}{\draw[dashed,thin] (K_0)--(\i) (K_1)--(\i);};
\draw[dashed,thin]
(M_1)--(O)--(Q) (O)--(P_1)
(A)--(K)--(B)
(K)--(K');
\pic[draw,angle eccentricity=1.8,angle radius=2mm]{right angle=Q--O--M_1};
\foreach \i/\g in {Q/90,M/90,N/90,P/90,K'/90,K_0/-45,K_1/-180,K/-60,O/-60,M_1/-90,N_1/-90,P_1/-90}
\fill[red] (\i) circle(2pt)+(\g:5mm)node[black,scale=1]{$\i$};
\end{tikzpicture}
\end{center}
Gọi $M_1$, $N_1$, $P_1$, $K$ lần lượt là hình chiếu của $M$, $N$, $P$, $K_0$ lên mặt phẳng $(Oxy)$.\\
Ta thấy $MNPQ.M_1N_1P_1O$ là hình hộp chữ nhật. Gọi $K$ là giao hai đường chéo $MP$ và $NQ$. Khi đó $K'Q = K'P = K'N = K'M$. Vì $K_0M = K_0N = K_0P = K_0Q$. và camera được hạ thấp theo phương thẳng đứng từ điểm $K_0$ xuống điểm $K_1$ nên các điểm $K'$, $K_0$, $K_1$, $K$ thẳng hàng.\\
Do đó các điểm $K'$, $K_0$, $K_1$, $K$ có hoành độ và tung độ bằng nhau.\\
Theo bài ra, cao độ của $K_0$ và $K_1$ lần lượt là $25$ và $19$. Giả sử $K_0(x;y;25)$ và $K_1(x;y;19)$.\\
Ta có $MNPQ.M_1N_1P_1O$ là hình hộp chữ nhật nên $K'K = OQ$, suy ra cao độ của $K'$ bằng $30$. Do đó $K'(x;y;30)$.\\
Ta có $\overrightarrow{K'Q} = \overrightarrow{OQ} - \overrightarrow{OK'} = (-x;-y;0)$, $\overrightarrow{NK'} - \overrightarrow{OK'} - \overrightarrow{ON} = (x - 90; y - 120; 0)$.\\
Vì $K'$ là giao hai đường chéo của hình chữ nhật $MNPQ$ nên $K'$ là trung điểm của $NQ$.\\
Suy ra $\overrightarrow{K'Q} - \overrightarrow{NK'} \Leftrightarrow \heva{&-x = x - 90 \\&-y = y - 120 \\ &0 = 0} \Leftrightarrow \heva{&x = 45 \\ &y = 60.}$\\
Do vậy $K_0(45;60;25)$, $K_1(45;60;19)$, nên ta có tọa độ trung điểm của $K_0 K_1$ là $K_2=(45;60;22)$.\\
Vậy $5a + 7b + 9c = 843$.
}
\end{ex}

\begin{ex}%[2D6V2-4]%[TEX ĐỀ MOON 2025]%[Nguyễn Văn Hiệp]
Một công ty dược phẩm giới thiệu một dụng cụ để kiểm tra sớm bệnh sốt xuất huyết. Về báo cáo kiểm định chất lượng của sản phẩm, họ cho biết như sau: Số người được thử là $8\,000$, trong số đó có $1\,200$ người đã bị nhiễm bệnh sốt xuất huyết và có $6\,800$ người không bị nhiễm bệnh sốt xuất huyết. Nhưng khi kiểm tra lại bằng dụng cụ của công ty, trong $1\,200$ người đã bị nhiễm bệnh sốt xuất huyết, có $70\%$ số người đó cho kết quả dương tính, còn lại cho kết quả âm tính. Trong $6\,800$ người không bị nhiễm bệnh sốt xuất huyết, có $5\%$ số người đó cho kết quả dương tính, còn lại cho kết quả âm tính. Xác suất mà một bệnh nhân với kết quả kiểm tra dương tính là bị nhiễm bệnh sốt xuất huyết bằng bao nhiêu? (viết kết quả dưới dạng số thập phân và làm tròn đến hàng phần trăm).
\shortans{$0{,}71$}
\loigiai{
\textbf{Bước 1: Tính các xác suất}
\begin{itemize}
\item $A$: \lq\lq Người đã bị nhiễm sốt xuất huyết\rq\rq. $\mathrm{P}(A)=\dfrac{1\,200}{8\,000}=0{,}15$.
\item $\overline{A}$: \lq\lq Người không bị nhiễm sốt xuất huyết\rq\rq. $\mathrm{P}\left(\overline{A}\right)=\dfrac{6\,800}{8\,000}=0{,}85$.
\item $D$: \lq\lq Người được kiểm tra cho kết quả dương tính\rq\rq. Ta có $\mathrm{P}\left(D\mid A\right)=0{,}7$; $\mathrm{P}\left(D\mid \overline{A}\right)=0{,}05$.
\item Theo công thức xác suất đầy đủ \[\mathrm{P}(D)=\mathrm{P}(A)\cdot \mathrm{P}\left(D\mid A\right)+ \mathrm{P}\left(\overline{A}\right)\cdot \mathrm{P}\left(D\mid \overline{A}\right)=0{,}15\cdot 0{,}7+0{,}85\cdot 0{,}05=0{,}1475.\]
\end{itemize}
\textbf{Bước 2: Áp dụng công thức Bayes}
\[
\mathrm{P}\left(A\mid D\right) =\dfrac{\mathrm{P}\left(D\mid A\right)\cdot \mathrm{P}(A)} {\mathrm{P}(D)}= \dfrac{0{,}7\cdot 0{,}15}{0{,}1475} \approx 0{,}71.
\]
}
\end{ex}

\begin{ex}%[2D6V2-4]%[TEX ĐỀ MOON 2025]%[Nguyễn Cường]
Có hai chiếc hộp, hộp I có $6$ quả bóng màu đỏ và $4$ quả bóng màu vàng, hộp II có $7$ quả bóng màu đỏ và $3$ quả bóng màu vàng, các quả bóng có cùng kích thước và khối lượng. Lấy ngẫu nhiên một quả bóng từ hộp I bỏ vào hộp II. Sau đó, lấy ra ngẫu nhiên một quả bóng từ hộp II. Tính xác suất để quả bóng được lấy ra từ hộp II là quả bóng được chuyển từ hộp I sang, biết rằng quả bóng đó có màu đỏ (làm tròn kết quả đến hàng phần trăm).
\shortans{$0{,}08$}
\loigiai{
Gọi $A$ là biến cố \lq\lq quả lấy ra ở II là quả bóng được đưa từ I vào\rq\rq.
\\
Gọi $B$ là biến cố \lq\lq quả bóng lấy ra ở II là đỏ\rq\rq.
\\
$\mathrm{P}(B)$ xảy ra theo $2$ trường hợp:
\\
\textbf{TH1:} Chuyển một quả đỏ từ I sang II xác suất trường hợp này là $\dfrac{6}{10}\cdot \dfrac{8}{11}$.
\\
\textbf{TH2:} Chuyển một quả vàng từ I sang II xác suất trường hợp này là $\dfrac{4}{10}\cdot \dfrac{7}{11}$.
\\
Suy ra $\mathrm{P}(B)=\dfrac{6}{10}\cdot \dfrac{8}{11}+\dfrac{4}{10}\cdot \dfrac{7}{11}=\dfrac{38}{55}$.
\\
$A\cap B$ là biến cố \lq\lq quả bóng lấy ra ở II là đỏ và nó là quả bóng thuộc I\rq\rq.
\\
Phép thử gồm $2$ hành động: lấy $1$ quả ở I đưa vào II và từ II lấy $1$ quả.
\\
Không gian mẫu có $10\cdot 11=110$ kết quả.
\\
$A\cap B$ có số kết quả thuận lợi là $6\cdot 1=6$ kết quả.
\\
Suy ra $\mathrm{P}(A\cap B)=\dfrac{6}{110}$.
\\
Theo định lý Bayes ta có $\mathrm{P}(A\mid B)=\dfrac{\mathrm{P}(A\cap B)}{\mathrm{P}(B)}=\dfrac{\dfrac{6}{110}}{\dfrac{38}{55}}\approx 0{,}08$.
}
\end{ex}

\begin{ex}%[2D6V2-3]
Trong quân sự, một máy bay chiến đấu của đối phương có thể xuất hiện ở vị trí X với xác suất $0{,}55$. Nếu máy bay đó không xuất hiện ở vị trí X thì nó xuất hiện ở vị trí Y. Để phòng thủ, các bệ
phóng tên lửa được bố trí tại các vị trí X và Y. Khi máy bay đối phương xuất hiện ở vị trí X hoặc Y thì tên lửa sẽ được phóng để hạ máy bay đó. Xét phương án tác chiến sau: Nếu máy bay xuất hiện tại X thì bắn $2$ quả tên lửa và nếu máy bay xuất hiện tại Y thì bắn một quả tên lửa. Biết rằng, xác suất bắn trúng máy bay của mỗi quả tên lửa là $0{,}8$ và các bệ phóng tên lửa hoạt động độc lập. Máy bay bị bắn hạ nếu nó trúng ít nhất $1$ quả tên lửa. Biết máy bay bị bắn hạ trong phương án tác chiến trên. Tính xác suất máy bay bị bắn hạ ở vị trí X. \textit{(kết quả làm tròn đến hàng phần trăm)}.
\shortans{$0{,}59$}
\loigiai{
Gọi $A$ là biến cố \lq\lq Máy bay xuất hiện tại vị trí X\rq\rq.\\
$\overline{A}$ là biến cố \lq\lq Máy bay xuất hiện tại vị trí Y\rq\rq.\\
$B$ là biến cố \lq\lq Máy bay bị bắn hạ\rq\rq.\\
$\overline{B}$ là biến cố \lq\lq Máy bay không bị bắn hạ\rq\rq.\\
Tính $\mathrm{P}(A\mid B)$.\\
Từ giả thiết, ta có $\mathrm{P}(A)=0{,}55\Rightarrow \mathrm{P}(A)=1-\mathrm{P}(A)=0{,}45$.
\begin{itemize}
\item Xác suất máy bay bị bắn hạ tại vị trí Y là $\mathrm{P}\left(B\mid\overline{A}\right)=0{,}8$.
\item Xác suất máy bay không bị bắn hạ tại vị trí X là $\mathrm{P}(B\mid A)$.\\
Vì máy bay bị bắn hạ nếu bị trúng ít nhất $1$ quả tên lửa do
đó bay không bị bắn hạ khi và chỉ khi cả $2$ quả tên lửa đều không bắn trúng (và xác suất không bắn trúng là $1-0{,}8=0{,}2)$.\\
Nên $\mathrm{P}\left(\overline{B}\mid A\right)=0{,}2\cdot 0{,}2=0{,}04$.
\end{itemize}
Suy ra xác suất máy bay bị bắn hạ tại vị trí A là $\mathrm{P}(B\mid A)=1-\mathrm{P}\left(\overline{B}\mid A\right)=0{,}96$.\\
Từ đó, ta có sơ đồ cây sau
\begin{center}
\begin{tikzpicture}[declare function={dai=2.5;cao=0.65;},>=stealth,font=\scriptsize]
\tikzset{nhan/.style={minimum size=19pt,font=\small,inner sep=0pt}}
\path (0,0) node[nhan] (G){\text{Gốc}}
(dai,{1.5*cao}) node[nhan] (B) {$A$}
(dai,{-1.5*cao}) node[nhan] (nB) {$\overline{A}$}
({2*dai},{3*cao}) node[nhan] (BA) {$B$}
({2*dai},{cao}) node[nhan] (BnA) {$\overline{B}$}
({2*dai},{-cao}) node[nhan] (nBA) {$B$}
({2*dai},{-3*cao}) node[nhan] (nBnA) {$\overline{B}$};

%Phần mũi tên
\draw[->] (G.0)--(B.200) node[sloped,pos=0.5,above]{$0{,}55$};
\draw[->] (G.0)--(nB.160) node[sloped,pos=0.5,below]{$0{,}45$};
\draw[->] (B.10)--(BA.190) node[sloped,pos=0.5,above]{$0{,}96$};
\draw[->] (B.10)--(BnA.170) ;
\draw[->] (nB.-10)--(nBA.190) node[sloped,pos=0.5,above]{$0{,}8$};
\draw[->] (nB.-10)--(nBnA.170) ;
\end{tikzpicture}
\end{center}
Áp dụng công thức xác suất toàn phần, ta có \[\mathrm{P}(B)=\mathrm{P}(A)\cdot \mathrm{P}(B\mid A)+\mathrm{P}\left(\overline{A}\right)\cdot \mathrm{P}\left(B\mid\overline{A}\right)=0{,}55\cdot 0{,}96+0{,}45\cdot 0{,}8=0{,}888.\]
Áp dụng công thức Bayes, ta có $\mathrm{P}(A\mid B)=\dfrac{\mathrm{P}(A)\cdot \mathrm{P}(B\mid A)}{\mathrm{P}(B)}=\dfrac{0{,}55\cdot 0{,}96}{0{,}888}\approx 0{,}59$.
}
\end{ex}

\begin{ex}%[50 Đề minh họa tốt nghiệp 2025 - Đề 13]%[Lê Hữu Kiệt - Lê Quân]%[2D6V2-3]
Vắc xin AstraZeneca (AZD1222) được Tổ chức Y tế Thế giới (WHO) cấp phép sử dụng khẩn cấp giúp ngăn ngừa các triệu chứng nghiêm trọng và giảm tử vong do COVID-19. Vắc xin này được tiêm ở tỉnh X, thống kê cho thấy rằng: Với người có bệnh nền thì xác suất xảy ra phản ứng phụ sau tiêm là $28\%$, với người không có bệnh nền thì xác suất xảy ra phản ứng phụ sau tiêm là $17\%$. Chọn ngẫu nhiên một người được tiêm và thấy người này có phản ứng phụ. Tính xác suất để người này bị bệnh nền. Biết tỷ lệ người có bệnh nền ở tỉnh X là $12\%$. (làm tròn kết quả đến hàng phần trăm).
\par\shortans{$0{,}18$}
\loigiai{
Gọi
\begin{itemize}
\item $B$ là biến cố \lq\lq Người được chọn có bệnh nền\rq\rq.
\item $A$ là biến cố \lq\lq Người được chọn có phản ứng phụ\rq\rq.
\end{itemize}
Ta có sơ đồ cây
\begin{center}
\begin{tikzpicture}[font=\footnotesize, line join=round, line cap=round, >=stealth, scale=1]
\draw[->] (0,0)--++(10:3) coordinate (B) node[sloped, above, pos=0.5]{$0{,}12$}node[sloped, right]{$B$};
\draw[->] (B)+(0:0.5)--+(0:3) coordinate (A1) node[sloped, above=-1mm, pos=0.5]{$0{,}72$} node[sloped, right]{$\overline{A}$};
\draw[->] (B)+(0:0.5)--+(20:3) coordinate (A2) node[sloped, above, pos=0.5]{$0{,}28$} node[sloped, right]{$A$};
\draw[->] (0,0)--++(-10:3) coordinate (B') node[sloped, below, pos=0.5]{$0{,}88$}node[sloped, right]{$\overline{B}$};
\draw[->] (B')+(0:0.5)--+(0:3) coordinate (A3) node[sloped, above=-1mm, pos=0.5]{$0{,}17$} node[sloped, right]{$A$};
\draw[->] (B')+(0:0.5)--+(-20:3) coordinate (A4) node[sloped, below, pos=0.5]{$0{,}83$} node[sloped, right]{$\overline{A}$};
\end{tikzpicture}
\end{center}
Biến cố \lq\lq Chọn một người bị bệnh nền biết người này có phản ưng phụ\rq\rq\, là $B\mid A$. Áp dụng công thức Bayes ta có
\[ \mathrm{P}(B\mid A)
=\dfrac{\mathrm{P}(B) \cdot \mathrm{P}(A\mid B)}{\mathrm{P}(B) \cdot \mathrm{P}(A\mid B) + \mathrm{P}(\overline{B}) \cdot \mathrm{P}(A\mid \overline{B})}
=\dfrac{0{,}12\cdot0{,}28}{0{,}12\cdot0{,}28 + 0{,}88\cdot0{,}17}
=\dfrac{42}{229}
\approx 0{,}18.
\]
}
\end{ex}

\begin{ex}%[2D6V2-2]
Có hai hộp bóng bàn, các quả bóng bàn có kích thước và hình dạng như nhau. Hộp I chứa $3$ bóng bàn màu trắng và $2$ bóng bàn màu vàng, tổng cộng $5$ quả. Hộp II ban đầu chứa $6$ bóng bàn màu trắng và $4$ bóng bàn màu vàng, tổng cộng $10$ quả. Lấy ngẫu nhiên $4$ quả bóng bàn ở hộp I bỏ vào hộp II rồi lấy ngẫu nhiên $1$ quả bóng bàn từ hộp II ra. Tính xác suất để quả bóng bàn lấy từ hộp II có màu vàng.
\shortans{0{,}4}
\loigiai{
Gọi $A$ là biến cố \lq\lq Có đúng $1$ bóng vàng được chuyển từ hộp I sang hộp II\rq\rq.\\
Gọi $B$ là biến cố \lq\lq Lấy được bóng vàng từ hộp II\rq\rq.
Khi đó
\begin{align*}
\mathrm{P}(A) &=\dfrac{\mathrm{C}2^1 \cdot \mathrm{C}3^3}{\mathrm{C}4^5}=\dfrac{2 \cdot 1}{5}=\dfrac{2}{5}, \\
\mathrm{P}(\overline{A}) &=\dfrac{\mathrm{C}2^2 \cdot \mathrm{C}3^2}{\mathrm{C}4^5}=\dfrac{1 \cdot 3}{5}=\dfrac{3}{5}.
\end{align*}
Sau khi chuyển, tổng số bóng ở hộp II là $14$ quả.
\begin{itemize}
\item Nếu xảy ra $A$, số bóng vàng trong hộp II là $4+1=5 \Rightarrow \mathrm{P}(B\mid A)=\dfrac{5}{14}$.
\item Nếu xảy ra $\overline{A}$, số bóng vàng trong hộp II là $4+2=6 \Rightarrow \mathrm{P}(B\mid\overline{A})=\dfrac{6}{14}$.
\end{itemize}
Áp dụng công thức xác suất toàn phần, ta được
\begin{align*}
\mathrm{P}(B) &=\mathrm{P}(A) \cdot \mathrm{P}(B\mid A)+\mathrm{P}(\overline{A}) \cdot \mathrm{P}(B\mid\overline{A}) \\
&=\dfrac{2}{5} \cdot \dfrac{5}{14}+\dfrac{3}{5} \cdot \dfrac{6}{14}=\dfrac{10}{70}+\dfrac{18}{70}=\dfrac{28}{70}=\dfrac{2}{5}=0{,}4.
\end{align*}
}
\end{ex}

\begin{ex}%[2D6V2-2]%[TEX Đề Moon 2025]%[Võ Nguyên Thạch]
Có hai thùng I và II chứa các sản phẩm có khối lượng và hình dạng như nhau. Thùng I có $5$ chính phẩm và $4$ phế phẩm, thùng $2$ có $6$ chính phẩm và $8$ phế phẩm. Lấy ngẫu nhiên $1$ sản phẩm từ thùng I sang thùng II. Sau đó, lấy ngẫu nhiên $1$ sản phẩm từ thùng II để sử dụng. Xác suất lấy được chính phẩm từ thùng II là bao nhiêu (làm tròn kết quả đến hàng phần trăm)?
\shortans{0{,}44}
\loigiai{
Xét các biến cố $A$: \lq\lq Lấy được $1$ chính phẩm từ thùng I sang thùng II\rq\rq.\\
$B$: \lq\lq Lấy được $1$ chính phẩm từ thùng II\rq\rq.\\
Khi đó $\mathrm{P}(A)=\dfrac{5}{9}$; $\mathrm{P}(\overline{A})=\dfrac{4}{9}$; $\mathrm{P}(B|A)=\dfrac{7}{15}$; $\mathrm{P}(B|\overline{A})=\dfrac{6}{15}=\dfrac{2}{5}$.\\
Theo công thức xác suất toàn phần, xác suất của biến cố $B$ là
\[\mathrm{P}(B)=\mathrm{P}(A)\cdot \mathrm{P}(B|A)+\mathrm{P}(\overline{A})\cdot \mathrm{P}(B|\overline{A})=\dfrac{5}{9}\cdot \dfrac{7}{15}+\dfrac{4}{9}\cdot \dfrac{2}{5}\approx0{,}44.\]
}
\end{ex}

\begin{ex}%[2D6V2-2]%[2D6V2-2]%[TEX ĐỀ MOON 2025]%[Nguyễn Thế Duy]
Tất cả các học sinh của trường Hạnh Phúc đều tham gia câu lạc bộ bóng chuyền hoặc bóng rổ, mỗi học sinh chỉ tham gia đúng một câu lạc bộ. Có $60\%$ học sinh của trường tham gia câu lạc bộ bóng chuyền và $40\%$ học sinh của trường tham gia câu lạc bộ bóng rổ. Số học sinh nữ chiếm $65\%$ trong câu lạc bộ bóng chuyền và $25\%$ trong câu lạc bộ bóng rổ. Chọn ngẫu nhiên một học sinh. Xác suất chọn được học sinh nữ là bao nhiêu?
\shortans{$0{,}49$}
\loigiai{
Gọi $A$ là sự kiện \lq\lq Tham gia câu lạc bộ bóng rổ\rq\rq.\\
Suy ra $\overline{A}$ là sự kiện \lq\lq Tham gia câu lạc bộ bóng chuyền\rq\rq.\\
Theo đề bài ta có $P(A) - 0{,}4$ và $P \left(\overline{A} \right) = 0{,}6$.\\
Gọi $B$ là sự kiện \lq\lq chọn được học sinh nữ\rq\rq.\\
Theo đề bài ta có $P\left(B \mid A \right) = 0{,}25$ và $P \left(B \mid \overline{A} \right) = 0{,}65$.\\
Xác suất chọn được học sinh nữ là
\begin{align*}
P(B) &= P(A) \cdot P\left(B \mid \overline{A} \right) +  P\left(\overline{A} \right) \cdot P \left(B \mid \overline{A} \right)\\
&= 0{,}4 \cdot 0{,}25 + 0{,}6 \cdot 0{,}65 = 0{,}49.
\end{align*}
}
\end{ex}

\begin{ex}%[2D6V1-4]%[TEX Đề Moon 2025]%[Vũ Hồng Toàn]
Một xí nghiệp mỗi ngày sản xuất ra $2000$ sản phẩm trong đó có $39$ sản phẩm lỗi. Lần lượt lấy ra ngẫu nhiên hai sản phẩm không hoàn lại để kiểm tra. Tính xác suất của biến cố \lq\lq Sản phẩm lấy ra lần thứ hai bị lỗi\rq\rq\ (kết quả làm tròn đến hàng phần trăm).
\shortans{$0{,}02$}
\loigiai{
Xét các biến cố:\\
$A$ : Sản phẩm lấy ra lần thứ nhất bị lỗi.\\ Khi đó, ta có $P\left(A\right)=\frac{39}{2000}$ ; $P\left(\overline{A}\right)=\dfrac{1961}{2000}$.\\
$B$ : Sản phẩm lấy ra lần thứ hai bị lỗi.\\
Khi sản phẩm lấy ra lần thứ nhất bị lỗi thì còn $1999$ sản phẩm và trong đó có $38$ sản phẩm lỗi nên ta có $P\left(B \mid A\right)=\frac{38}{1999}$, suy ra $P\left(\overline{B} \mid A\right)=\dfrac{1961}{1999}$.\\
Khi sản phẩm lấy ra lần thứ nhất không bị lỗi thì còn $1999$ sản phẩm trong đó có $39$ sản phẩm lỗi nên ta có $P\left(B \mid \overline{A}\right)=\dfrac{39}{1999}$, suy ra $P\left(\overline{B} \mid \overline{A}\right)=\dfrac{1960}{1999}$.\\
Khi đó, xác suất để sản phẩm lấy ra lần thứ hai bị lỗi là
\[
P\left(B\right)=P\left(B \mid A\right) \cdot P\left(A\right)+P\left(B \mid \overline{A}\right) \cdot P\left(\overline{A}\right)=\dfrac{38}{1999} \cdot \dfrac{39}{2000}+\dfrac{39}{1999} \cdot \dfrac{1961}{2000} \approx 0{,}02.\]
}
\end{ex}

\begin{ex}%[2D6V1-2]%[TexDeMoon2025]%[NguyenKieuNhaTu]
Có hai chiếc hộp, hộp I có $6$ bi đỏ và $4$ bi trắng, hộp II có $7$ bi đỏ và $3$ bi trắng, các bi có cùng kích thước và khối lượng. Lấy ngẫu nhiên từ mỗi hộp ra hai bi. Tính xác suất để lấy được ít nhất một bi đỏ từ hộp I, biết rằng trong bốn bi lấy ra số bi đỏ bằng số bi trắng. (Làm tròn kết quả đến hàng phần trăm).
\shortans[]{$0{,}81$}
\loigiai{
Gọi
\begin{itemize}
\item $A\colon$ \lq\lq Lấy được ít nhất một bi đỏ từ hộp I\rq\rq. Suy ra $\overline{A}\colon$ \lq\lq Không lấy được bi đỏ từ hộp I\rq\rq.
\item $B\colon$ \lq\lq Trong bốn bi số bi đỏ bằng số bi trắng\rq\rq.
\item $\overline{A}B\colon$ \lq\lq Lấy $2$ bi trắng từ hộp I và lấy $2$ bi đỏ từ hộp II\rq\rq.
\end{itemize}
Xét biến cố $B$.
\begin{enumerate}[TH 1: ]
\item Hộp I 2 bi đỏ, Hộp II 2 bi trắng: $\mathrm{C}_6^2\cdot\mathrm{C}_3^2=45$.
\item Hộp I 2 bi trắng, Hộp II 2 bi đỏ: $\mathrm{C}_4^2\cdot\mathrm{C}_7^2=126$.
\item Hộp I 1 bi đỏ và 1 bi trắng, Hộp II 1 bi đỏ và 1 bi trắng: $6\cdot4\cdot7\cdot3=504$.
\end{enumerate}
$\Rightarrow n(B)=45+126+504=675$.\\
Lại có $n(\overline{A}B)=126$.\\
$\Rightarrow\mathrm{P}(\overline{A}\mid B)=\dfrac{n(\overline{A}B)}{n(B)}=\dfrac{126}{675}$.\\
Vậy $\mathrm{P}(A\mid B)=1-\mathrm{P}(\overline{A}\mid B)=1-\dfrac{126}{675}=\dfrac{61}{75}\approx0{,}81$.
}
\end{ex}

\begin{ex}%[2D4V3-5]
\immini{Ông Duy có một mảnh vườn hình vuông cạnh bằng $8$ m. Ông dự định xây một cái bể bơi đặc biệt (phần kẻ sọc trong hình vẽ bên). Biết $AM=\dfrac{AB}{4}$, phần đường cong đi qua các điểm $C$, $M$, $N$ là một phần của đường Parabol có trục đối xứng là $MP(MP\parallel AD)$ và chi phí để làm bể bơi là $5$ triệu đồng $/ $1$\mathrm{~m}^2$. Số tiền ông Duy phải trả để xây cái bể bơi đó là bao nhiêu triệu đồng? (làm tròn kết quả đến hàng đơn vị).}{\begin{tikzpicture}[line join=round, line cap=round,>=stealth,thick,scale=0.6]
\path
(0,8)coordinate (A)
(8,8)coordinate (B)
(8,0)coordinate (C)
(0,0)coordinate (D)
(2,0)coordinate (P)
(2,8)coordinate (M)
(0,64/9)coordinate (N)
;
\begin{scope}
\clip (-2,-2) rectangle (8,8);
\fill[pattern=north west lines]plot[samples=200,domain=0:8,smooth,variable=\x] (\x,{-2/9*(\x)^2+8/9*(\x)+64/9})--plot[samples=200,domain=8:0,smooth,variable=\x] (\x,{-8/9*(\x)+64/9})--cycle;
\draw[dashed] plot[samples=200,domain=0:8,smooth,variable=\x] (\x,{-2/9*(\x)^2+8/9*(\x)+64/9});
\draw plot[samples=200,domain=0:8.1,smooth,variable=\x] (\x,{-2/9*(\x)^2+8/9*(\x)+64/9});
\draw plot[samples=200,domain=0:8.1,smooth,variable=\x] (\x,{-8/9*(\x)+64/9});
\draw[samples=200,domain=0:8,smooth,variable=\x] plot (\x,{-2/9*(\x)^2+8/9*(\x)+64/9});
\draw[samples=200,domain=0:8,smooth,variable=\x] plot (\x,{-8/9*(\x)+64/9});
\end{scope}
\draw (A)--(B)--(C)--(D)--cycle;
\draw [dashed](M)--(P);
\foreach \x/\g in {A/180,B/0,C/0,P/-90,M/90,N/180,D/180} \fill[black] (\x) circle (1pt) ($(\x)+(\g:3mm)$)node{$\x$};
\end{tikzpicture}}
\shortans[]{$95$}
\loigiai{
\begin{center}
\begin{tikzpicture}[line join=round, line cap=round,>=stealth,thick,scale=0.6]
\draw[->] (-2.1,0)--(9.1,0) node[below left] {$x$};
\draw[->] (0,-2.1)--(0,9.1) node[below left] {$y$};
\draw (0,0) node [below left] {$O$};
\foreach \x/\nx in {8/8}
\draw[thin] (\x,1pt)--(\x,-1pt) node [below] {$\nx$};
\foreach \y/\ny in {8/8}
\draw[thin] (1pt,\y)--(-1pt,\y) node [left] {$\ny$};
\draw[dashed,thin](2,0)--(2,8)--(0,8);
\path
(0,8)coordinate (A)
(8,8)coordinate (B)
(8,0)coordinate (C)
(0,0)coordinate (D)
(2,0)coordinate (P)
(2,8)coordinate (M)
(0,64/9)coordinate (N)
;
\begin{scope}
\clip (-2,-2) rectangle (8,8);
\fill[pattern=north west lines]plot[samples=200,domain=0:8,smooth,variable=\x] (\x,{-2/9*(\x)^2+8/9*(\x)+64/9})--plot[samples=200,domain=8:0,smooth,variable=\x] (\x,{-8/9*(\x)+64/9})--cycle;
\draw[dashed] plot[samples=200,domain=0:8,smooth,variable=\x] (\x,{-2/9*(\x)^2+8/9*(\x)+64/9});
\draw plot[samples=200,domain=0:8.1,smooth,variable=\x] (\x,{-2/9*(\x)^2+8/9*(\x)+64/9});
\draw plot[samples=200,domain=0:8.1,smooth,variable=\x] (\x,{-8/9*(\x)+64/9});
\draw[samples=200,domain=0:8,smooth,variable=\x] plot (\x,{-2/9*(\x)^2+8/9*(\x)+64/9});
\draw[samples=200,domain=0:8,smooth,variable=\x] plot (\x,{-8/9*(\x)+64/9});
\end{scope}
\draw (A)--(B)--(C);
\foreach \x/\g in {A/40,B/0,C/40,P/-90,M/90,N/180,D/45} \fill[black] (\x) circle (1pt) ($(\x)+(\g:3mm)$)node{$\x$};
\end{tikzpicture}
\end{center}
Gắn trục tọa độ như hình vẽ.
Gọi phương trinh trình của Parabol là $(P)\colon y=ax^2+bx+c$.\\
Ta có $(P)$ đi qua điểm $C(8;0)$, $M(2;8)$ và có hoành độ đỉnh $x=2$ nên ta có hệ phương trình sau
\[\heva{&64a+8b+c=0\\&4a+2b+c=8\\&\dfrac{-b}{2a}=2}\Leftrightarrow \heva{&a=-\dfrac{2}{9}\\&b=\dfrac{8}{9}\\&c=\dfrac{64}{9}}\Rightarrow (P)\colon y=-\dfrac{2}{9}x^2+\dfrac{8}{9}x+\dfrac{64}{9}.\]
Giao điểm của $(P)$ với trục $Oy$ là điểm $N\left(0;\dfrac{64}{9}\right)$.\\
Gọi $d\colon y=ax+b$ là đường thẳng đi qua $N$ và $C$. Khi đó phương trình của $d$ là $y=-\dfrac{8}{9}x+\dfrac{64}{9}$.\\
Diện tích hình phẳng giới hạn bởi đồ thị $(P)$ và đường thẳng $d$ là
\[S=\displaystyle\int\limits_0^8 \left(-\dfrac{2}{9}x^2+\dfrac{8}{9}x+\dfrac{64}{9}+\dfrac{8}{9}x-\dfrac{64}{9}\right) \mathrm{\, d}x=\displaystyle\int\limits_0^8 \left(-\dfrac{2}{9}x^2+\dfrac{16}{9}x\right) \mathrm{\, d}x=\dfrac{512}{27}.\]
Vậy số tiền ông Duy phải trả để xây bể bơi là $\dfrac{512}{27}\cdot 5\approx 95$ triệu đồng.
}
\end{ex}

\begin{ex}%[2D4V3-5]
\immini{Một vật trang trí có dạng khối tròn xoay tạo thành khi quay miền $(R)$ (phần tô đậm trong hình vẽ) quay xung quanh trục $AB$. Miền $(R)$ được giới hạn bởi các cạnh $AD$, $A B$, $B C$, $EF$ và các cung phần tư của các đường tròn bán kính bằng $2$ cm với tâm lần lượt là trung điểm của $AD$ và $BC$. Biết $ABCD$ là hình chữ nhật có cạnh $AB=8$ cm, $AD=4$ cm; điểm $E$ cách $AD$ một đoạn bằng $2$ cm; điểm $F$ cách $BC$ một đoạn bằng $2$ cm. Thể tích của vật thể trang trí trên là bao nhiêu centimetkhối? \textit{(quy tròn đến hàng phần mười)}.}{
\begin{tikzpicture}
% Define coordinates for the points
\coordinate (A) at (0, 4);
\coordinate (B) at (0, 0);
\coordinate (C) at (2, 0);
\coordinate (D) at (2, 4);
\coordinate (E) at (1, 3);
\coordinate (F) at (1, 1);

\fill[gray, draw=black] (A) rectangle (C);
\fill[white, draw=black] (E)--(F) to [out=0, in=90] (C)--(D) to [out=-90, in=0] (E) ;
% Label points using \foreach \x/\y
\foreach \x/\y in {A/180, B/180, C/0, D/0, E/180, F/180} {\fill (\x) circle(1pt) ($(\x)+(\y:0.3cm)$) node{$\x$};}
\end{tikzpicture}
}
\shortans{$213$}
\loigiai{
Gắn hệ trục tọa độ $A(0;0)$, $D(0;4)$, $B(8;0)$, $C(8;4)$, $E(2;2)$, $F(6;2)$.
\begin{center}
\begin{tikzpicture}[rotate=90]
% Define coordinates for the points
\coordinate (A) at (0, 4);
\coordinate (B) at (0, 0);
\coordinate (C) at (2, 0);
\coordinate (D) at (2, 4);
\coordinate (E) at (1, 3);
\coordinate (F) at (1, 1);

\fill[gray, draw=black] (A) rectangle (C);
\fill[white, draw=black] (E)--(F) to [out=0, in=90] (C)--(D) to [out=-90, in=0] (E) ;
% Label points using \foreach \x/\y
\foreach \x/\y in {A/45, B/180, C/0, D/90, E/180, F/180} {\fill (\x) circle(1pt) ($(\x)+(\y:0.3cm)$) node{$\x$};}
\draw[->] (0,5)--(0,-1)node [above]{$x$};
\draw[->] (-1,4)--(3,4) node[above]{$y$};
\end{tikzpicture}
\end{center}
Phương trình đường tròn đi qua hai điểm $D$ và $E$ là
\begin{eqnarray*}
&&x^2+(y-2)^2=2^2\\
&\Rightarrow&(y-2)^2=4-x^2\\
&\Rightarrow& y-2=\sqrt{4-x^2}\\
&\Rightarrow& y=\sqrt{4-x^2}+2.
\end{eqnarray*}
Thể tích của vật thể trang trí là
\[V=2 \pi\displaystyle\int\limits_0^2\left[\left(\sqrt{4-x^2}+2\right)^2-2^2 \right]\mathrm{\, d} x+\pi\displaystyle\int\limits_0^8 2^2 \mathrm{\, d} x \approx 213{,}0.\]
}
\end{ex}

\begin{ex}%[2D4V3-4]%[TEX ĐỀ MOON 2025]%[Nguyễn Thế Duy]
Trong chương trình nông thôn mới, tại một xã $Y$ có xây một cây cầu bằng bê tông như hình vẽ. Đường cong trong hình vẽ là các đường Parabol, chọn hệ trục $Oxy$ như hình vẽ.
\begin{center}
\begin{tikzpicture}[scale=0.9,>=stealth, font=\footnotesize, line join=round, line cap=round]
\draw[smooth,samples=300,domain=-2:2] plot(\x,{-1/2*(\x)^2+2});
\draw[smooth,samples=300,domain=-1:1] plot(\x,{-(\x)^2+1});
\draw (-2,0)--(2,0) (-1.5,0)node[below]{$0{,}5$ m} (1.5,0)node[below]{$0{,}5$ m} (0,0)node[below]{$19$ m} (0,2)--(8,5) (2,0)--(10,3);
\draw[dashed] (0,0)--(0,2) (-2,0)--(6,3) (-1,0)--(7,3) (1,0)--(9,3) (2,0)--(10,3) (0,1)--(8,4) (8,4.5)node[right,xshift=-0.1cm]{$0{,}5$ m} (8,3.5)node[right,xshift=-0.1cm]{$2$ m} (6,1.5)node[below right]{$5$ m};
\draw[smooth,samples=300,domain=6:10,dashed] plot(\x,{-1/2*(\x-8)^2+5});
\draw[smooth,samples=300,domain=7:9,dashed] plot(\x,{-(\x-8)^2+4});
\draw[->,dashed] (6,3)--(10.5,3)node[below]{$x$};
\draw[->,dashed] (8,3)--(8,5.5)node[left]{$y$};
\fill[gray,opacity=0.3] plot[domain=-2:2](\x,{-1/2*(\x)^2+2})--plot[domain=1:-1](\x,{-(\x)^2+1})--cycle;
\fill[gray,opacity=0.5] plot[domain=0:2](\x,{-1/2*(\x)^2+2})--(10,3)--plot[domain=10:8](\x,{-1/2*(\x-8)^2+5})--(8,5)--cycle;
\end{tikzpicture}
\end{center}
Tính lượng bê tông để đổ cây cầu.
\shortans{$40$}
\loigiai{
Gọi $\left(P_1 \right) \colon y = a_1x^2 + b_1x + c_1$ là Parabol phía phần phía trong của cây cầu.\\
$\left(P_2 \right) \colon y = a_2x^2 + b_2x + c_2$ là Parabol phía phần phía ngoài của cây cầu.\\
Theo đề bài ta có $\left(P_1 \right)$ đi qua các điểm  $\left(9{,}5; 0 \right)$, $\left(-9{,}5; 0 \right)$ và $\left(0; 2 \right)$.\\
Từ đó ta được $\left(P_1 \right) \colon y = -\dfrac{8}{361}x^2 + 2$.\\
Lại có $\left(P_2 \right)$ đi qua các điểm $\left(-10;0 \right)$, $\left(0; 2{,}5\right)$ và $\left(10;0 \right)$.\\
Suy ra $\left(P_2 \right) \colon y = \dfrac{-1}{40}x^2 + 2{,}5$.\\
Diện tích phần giới hạn bởi hai Parabol là
\begin{align*}
S = \displaystyle\int_{-10}^{10} \left(\dfrac{-1}{40}x^2+2{,}5\right) \mathrm{\,d}x - \displaystyle\int_{-9{,}5}^{9{,}5} \left(\dfrac{-8}{361}x^2+2\right) \mathrm{\,d}x= 8 \, \text{m}^2.
\end{align*}
Thể tích lượng bê tông cần dùng là $V = 8 \cdot 5 = 40$ m$^3$.
}
\end{ex}

\begin{ex}%[2D4V3-3]%[Tex đề Moon 2025]%[Nguyễn Hồng Thạch]
\immini[thm]
{
Một vật trang trí có dạng một khối tròn xoay được tạo thành khi quay miền $(R)$ (phần được tô màu trong hình vẽ bên) quanh trục $AB$. Miền $(R)$ được giới hạn bởi các cạnh $AB$, $AD$ của hình vuông $ABCD$ và các cung phần tư của các đường tròn bán kính bằng $1$ cm với tâm lần lượt là trung điểm của các cạnh $AD$, $AB$. Tính thể tích của vật trang trí đó, làm tròn kết quả đến hàng phần mười của centimet khối.
}
{
\begin{tikzpicture}[scale=1.5,>=stealth, font=\footnotesize, line join=round, line cap=round]
\fill[gray,opacity=0.6] (0,2) arc(90:0:1 cm and 1 cm)--(1,1) arc(90:0:1 cm and 1 cm)--(0,0)--cycle;
\draw (0,0)node[below left]{$A$}--(2,0)node[below right]{$B$}--(2,2)node[above right]{$C$}--(0,2)node[above left]{$D$}--cycle (0,2) arc(90:0:1 cm and 1 cm) (1,1) arc(90:0:1 cm and 1 cm);
\end{tikzpicture}
}
\shortans{$12{,}3$}
\loigiai{
\begin{center}
\begin{tikzpicture}[scale=1.5,>=stealth, font=\footnotesize, line join=round, line cap=round]
\fill[gray,opacity=0.6] (0,2) arc(90:0:1 cm and 1 cm)--(1,1) arc(90:0:1 cm and 1 cm)--(0,0)--cycle;
\draw (0,0)node[below left]{$A\equiv O$}--(2,0)node[above right]{$B$}--(2,2)node[above right]{$C$}--(0,2)node[above right]{$D$}--cycle (0,2) arc(90:0:1 cm and 1 cm) (1,1) arc(90:0:1 cm and 1 cm);
\draw[->] (-1,0)--(3,0);
\draw[->] (0,-1)--(0,3);
\fill (3,0)node[above]{$x$}
(2,0)node[below]{$2$}circle(0.8pt)
(0,3)node[left]{$y$}
(0,2)node[left]{$2$}circle(0.8pt)
(1,0)node[below]{$1$}circle(0.8pt)
(0,1)node[left]{$1$}circle(0.8pt)
(1,1)node[above right]{$E$}circle(0.8pt);
\end{tikzpicture}
\end{center}
Chọn hệ trục như hình vẽ.\\
Ta có phương trình đường tròn đường kính $AD$ là $x^2+(y-1)^2=1$.\\
$\Rightarrow$ phương trình cung tròn $DE$ là $y=1+\sqrt{1-x^2}$.\\
Ta có phương trình đường tròn đường kính $AB$ là $(x-1)^2+y^2=1$.\\
$\Rightarrow$ phương trình cung tròn $BE$ là $y=\sqrt{1-(x-1)^2}$.\\
Thể tích khối tròn xoay là
\[V=\pi\cdot\displaystyle\int_{0}^{1}(1+\sqrt{1-x^2})^2\mathrm{\,d}x+\pi\cdot\displaystyle\int_{1}^{2} \left(\sqrt{1-(x-1)^2}\right)^2\mathrm{\,d}x\approx 12{,}3.\]
}
\end{ex}

\begin{ex}%[2D4V3-2]%[TEX ĐỀ MOON 2025]%[Nguyễn Văn Hiệp]
Bác Năm làm một cái cửa nhà hình parabol có chiều cao từ mặt đất đến đỉnh là $2{,}25$ mét, chiều rộng tiếp giáp với mặt đất là $3$ mét. Giá thuê mỗi mét vuông là $150\,000$ đồng. Vậy số tiền bác Năm phải trả là bao nhiêu triệu đồng?
\shortans{$6{,}75$}
\loigiai{
\textbf{Bước 1: Thiết lập phương trình parabol}
\immini{Chọn hệ trục tọa độ $Oxy$ như hình vẽ. \\
Phương trình parabol $(P)\colon y=ax^2+bx+c$, ($a$, $b$, $c\in \mathbb{R}$).\\
Các điểm $(0;2{,}25)$, $(-1{,}5;0)$, $(1{,}5;0)$ thuộc đồ thị hàm số nên
\[\heva{&a\cdot 0^2+b\cdot 0+c=2{,}25\\
&a\cdot (-1{,}5)^2+b\cdot (-1{,}5)+c=0\\
&a\cdot (1{,}5)^2+b\cdot (1{,}5)+c=0
}\Leftrightarrow \heva{&a=-1\\&b=0\\&c=2{,}25.}\]
Vậy $(P)\colon y=-x^2 +2{,}25$.
}{\begin{tikzpicture}[line join=round, line cap=round,>=stealth,thick]
\tikzset{every node/.style={scale=0.9}}
\draw[->] (-2.1,0)--(2.1,0) node[below left] {$x$};
\draw[->] (0,-0.6)--(0,3.2) node[below left] {$y$};
\draw (0,0) node [below left] {$O$};
\foreach \x/\nx in {-1.5/-1.5,1.5/1.5}
\draw[thin] (\x,1pt)--(\x,-1pt) node [below] {$\nx$};
\foreach \y/\ny in {2.25/2.25}
\draw[thin] (1pt,\y)--(-1pt,\y) node [above left] {$\ny$};
\begin{scope}
\clip (-2,-0.5) rectangle (2,2.8);
\draw[samples=200,domain=-1.5:1.5,smooth,variable=\x] plot (\x,{-1*(\x)^2+0*(\x)+2.25});
\end{scope}
\end{tikzpicture}}
\noindent \textbf{Bước 2: Tính diện tích} \\
Do $(P)$ đối xứng qua trục $Oy$ nên ta có
\[
S = 2\displaystyle\int\limits_0^{1{,}5} \left(-x^2 + 2{,}25\right) \mathrm{\,d}x = 4{,}5 \,\text{ m}^2.
\]
\textbf{Bước 3: Tính chi phí}
\[
4{,}5 \times 150\,000 = 675\,000 \,\text{ đồng} = 6{,}75 \,\text{ triệu đồng}.
\]
}
\end{ex}

\begin{ex}%[2D4V3-2]%[TEX ĐỀ MOON 2025]%[Nguyễn Cường]
\immini{
Một kiến trúc sư thiết kế một khu sinh hoạt cộng đồng có dạng hình vuông với mỗi cạnh dài $120$ m. Phần sân chơi nằm ở giữa, và phần còn lại để trồng cây xanh. Các đường biên của khu vực trồng cây xanh là các đoạn parabol, với đỉnh của parabol nằm cách trung điểm của mỗi cạnh hình vuông $25$ m. Tính diện tích phần trồng cây xanh.
}
{
\begin{tikzpicture}[scale=0.7,>=stealth, font=\footnotesize, line join=round, line cap=round]
\fill[gray,opacity=0.5] plot[domain=-3:3](\x,{1/6*(\x)^2+1.5})--plot[domain=3:-3](\x,{-1/6*(\x)^2-1.5})--cycle;
\draw (-3,3)--(-3,-3)--(3,-3)--(3,3)--cycle;
\draw[smooth,samples=300,domain=-3:3] plot(\x,{1/6*(\x)^2+1.5});
\draw[smooth,samples=300,domain=-3:3] plot(\x,{-1/6*(\x)^2-1.5});
\draw[dashed] (0,1.5)--(0,-1.5);
\draw[<->] (0,1.5)--(0,3);
\draw[<->] (0,-1.5)--(0,-3);
\draw (0,2.25)node[right]{$25$ m} (0,-2.25)node[right]{$25$ m} (0,-3)node[below]{$120$ m} (3,0)node[right]{$120$ m};
\end{tikzpicture}
}
\shortans{$4000$}
\loigiai{
\immini{
Dựng hệ trục $Oxy$ như hình vẽ, dễ thấy Parabola $(P)$ có phương trình
$(P)\colon y=ax^2+b$.
\\
Đồng thời $(P)$ đi qua điểm $(60;0)$ và $(0;25)$ nên ta có hệ phương trình $\heva{&3600a+b=0\\&b=25}\Leftrightarrow \heva{&a=-\dfrac{1}{144}\\&b=25.}$
\\
Suy ra $(P)\colon y=-\dfrac{1}{144}x^2+25$.
\\
Diện tích một nửa phần trồng hoa là
\[
\displaystyle \int\limits_{-60}^{60} \left(-\dfrac{1}{144}x^2+25\right) \mathrm{\,d}x=2000.
\]
Vậy diện tích phần trồng cây xanh là $4000$ (m$^2$).
}
{
\begin{tikzpicture}[scale=0.7,>=stealth, font=\footnotesize, line join=round, line cap=round]
\fill[gray,opacity=0.5] plot[domain=-3:3](\x,{1/6*(\x)^2+1.5})--plot[domain=3:-3](\x,{-1/6*(\x)^2-1.5})--cycle;
\draw (-3,3)--(-3,-3)--(3,-3)--(3,3)--cycle;
\draw[smooth,samples=300,domain=-3:3] plot(\x,{1/6*(\x)^2+1.5});
\draw[smooth,samples=300,domain=-3:3] plot(\x,{-1/6*(\x)^2-1.5});
\draw[dashed] (0,1.5)--(0,-1.5);
\draw[<->] (0,1.5)--(0,3);
\draw[<->] (0,-1.5)--(0,-3);
\draw (0,2.25)node[right]{$25$ m} (0,-2.25)node[right]{$25$ m} (3,0)node[right]{$120$ m};
\draw[->](-4,-3)--(4,-3)node[below]{$x$} ;
\draw[->](0,-4)--(0,-3)node[below right]{$O$}--(0,4)node[right]{$y$} ;
\end{tikzpicture}
}
}
\end{ex}

\begin{ex}%[2D4V3-2]%[TEX ĐỀ MOON 2025]%[Huỳnh Thanh Chí]
Một bể chứa nhiên liệu hình trụ đặt nằm ngang, có chiều dài $5$ m, có bán kính đáy $1$ m. Chiều cao của mực nhiên liệu là $1{,}5$ m.
\begin{center}
\begin{tikzpicture}[declare function={r=2;h=6;gM=70;gN=40;d=1.5;a=0.5;},scale=0.8]
\path (0:0) coordinate (O)
(90:r) coordinate (A)
(-90:r) coordinate (B)
(A) arc (90:90+gM:{r/2} and {r}) coordinate (M)
(A) arc (90:90-gN:{r/2} and {r}) coordinate (N)
\foreach \x in {O,A,B,M,N}{(\x)++(0:h) coordinate (\x_1)}
(intersection of A_1--B_1 and M_1--N_1) coordinate (I)
($(A_1)+(0:d)$) coordinate (H)
($(B_1)+(0:d)$) coordinate (K)
($(B)+(-90:a)$) coordinate (E)
($(B_1)+(-90:a)$) coordinate (F)
;
\fill[gray!60] (M) arc (90+gM:270:{r/2} and {r})--(B) arc (-90:90-gN:{r/2} and {r})--cycle
(M_1) arc (90+gM:270:{r/2} and {r})--(B_1) arc (-90:90-gN:{r/2} and {r})--cycle
(M)--(B)--(B_1)--(M_1)--(N_1)--(N)--cycle;
\draw (A) arc (90:270:{r/2} and {r})
(A_1) arc (90:270:{r/2} and {r})
(A_1) arc (90:-90:{r/2} and {r})
(A)--(A_1) (B)--(B_1) (M)--(M_1)
(M_1)--(N_1)
;
\draw[dashed] (A) arc (90:-90:{r/2} and {r})
(M)--(N) (N)--(N_1)
;
\draw[>=stealth,<->] (A_1)--(I);
\node at ($(I)+(-90:0.6)$) {$0{,}5$ m};
\draw[>=stealth,<->] (H)--(K)node[midway,right]{$2$ m};
\draw[>=stealth,<->] (E)--(F)node[midway,below]{$5$ m};
\end{tikzpicture}
\end{center}
Tính thể tích phần nhiên liệu trong bể (theo đơn vị m$^3$, làm tròn đến hàng phần chục).

\shortans[]{$12{,}6$}
\loigiai{
\immini{Thể tích phần nhiên liệu sẽ bằng diện tích hình phẳng gạch sọc trong hình nhân với chiều dài của bồn (chiều cao của trụ).\\
Đường tròn có tâm $O(0;0)$, $R=1$ có phương trình là $x^2+y^2=1\Leftrightarrow y=\pm\sqrt{1-x^2}$. \\
Diện tích hình gạch sọc chính là diện tích hình phẳng giới hạn bởi các đường $y=\sqrt{1-x^2}$; $y=-\sqrt{1-x^2}$; $x=-1$; $x=0{,}5$.}
{\begin{tikzpicture}[scale=1,font=\footnotesize,line join=round,line cap=round,>=stealth,declare function={r=2;}]
\path (0,0) coordinate (O)
(60:r) coordinate (A)
(-60:r) coordinate (B)
(90:r) coordinate (C)
(-90:r) coordinate (D)
(180:r) coordinate (E)
(0:r) coordinate (F)
;
\fill[pattern=north east lines,pattern color=green] (A)--(A) arc (60:300:r)--(B)--cycle;
\draw[-stealth] (-r-0.5,0)--(0,0)node[below left]{$O$}--(r+0.5,0)node[below]{$x$};
\draw[-stealth] (0,-r-0.5)--(0,r+0.5)node[left]{$y$};
\draw (O) circle (r) (A)--(B);
\foreach \x in {O,A,B,C,D,E,F}{\draw[fill=green] (\x) circle (1pt);}
\node[above right] at (C) {$1$};
\node[below right] at (D) {$-1$};
\node[below left] at (E) {$-1$};
\node[below left] at (F) {$1$};
\node[above right] at (r/2,0) {$0{,}5$};
\end{tikzpicture}}
Do đó $V=S\cdot h=5\displaystyle\int\limits_{-1}^{0{,}5}\left|\sqrt{1-x^2}-\left(-\sqrt{1-x^2}\right)\right| \mathrm{d}x\approx 12{,}6$ m$^3$.
}
\end{ex}

\begin{ex}%[2D4V3-2]
\immini{Từ một tấm tôn phẳng hình chữ nhật có chiều dài $8$ cm, chiều rộng $5$ cm có gắn hệ toạ độ $Oxy$ như hình vẽ bên. Thầy Tuấn cắt miếng tôn theo ba đường: Đường cong $AIB$ là một phần của Parabol, các đường cong $AE$, $EB$ là một phần đồ thị hàm số bậc ba. Trang trí phần còn lại để tạo thành một chiếc mặt nạ đồ chơi có trục đối xứng là trục $Oy$. Biết đường cong $EB$ đi qua các điểm $(1;-2)$ và $(3;-3)$.}{\begin{tikzpicture}[very thick,>=stealth',scale=0.9]
\tikzset{declare function={xmin=-4;xmax=4;
ymin=-4;ymax=1;
f(\x)=1 - (\x)^2/16;
g(\x)=1/2*(\x)^3-17*(\x)^2/6+13*(\x)/3-4;
},
smooth,samples=50
}
\draw[black,thick] (-4,-4) rectangle (4,1);
\draw (-2,-1) circle (0.5 cm);
\draw (2,-1) circle (0.5 cm);
\draw[->] (xmin-0.25,0)--(xmax+0.5,0)
node[shift={(-100:7pt)},font=\normalsize]{$x$};
\draw[->] (0,ymin-0.25)--(0,ymax+0.5)
node[shift={(170:7pt)},font=\normalsize]{$y$};
\fill (0,0) node[shift={(135:9pt)},font=\normalsize]{$O$};
\foreach \x in {-4, 4}{
\draw (\x,2pt)--(\x,-2pt) +(0,-9pt) node[shift={(-10:5pt)},font=\footnotesize,fill=white,inner sep=1pt]{$\x$};
}
\foreach \y in {-4, 1}{
\draw (2pt,\y)--(-2pt,\y) +(-3pt,0) node[shift={(135:9pt)},font=\footnotesize,fill=white,inner sep=1pt]{$\y$};
}
\begin{scope}
\clip (xmin,ymin) rectangle (xmax,ymax);
\draw[black,thick] plot[domain=xmin:xmax] (\x, {f(\x)});
\draw[black,thick] plot[domain=xmin:xmax] (\x, {g(\x)});
\draw[black,thick] plot[domain=xmin:xmax] (\x, {g(-\x)});
\end{scope}
\fill
(4,0)circle(1.5pt)node[above right]{$B$}
(-4,0)circle(1.5pt)node[above left]{$A$}
(0,1)circle(1.5pt)node[above right]{$I$}
(0,-4)circle(1.5pt)node[above right]{$E$}
;
\end{tikzpicture}}
\noindent Tính diện tích chiếc mặt nạ đồ chơi của thầy Tuấn (làm tròn đến hàng phần mười theo đơn vị cm$^2$).

\shortans{24{,}9}
\loigiai{
Giả sử đường cong $AIB$ có phương trình là $y = f(x) = mx^2 + nx + p$.\\
Đường cong $EB$ có phương trình là $y = g(x) = ax^3 + bx^2 + cx + d$. \\
Vì mặt nạ đối xứng qua trục $Oy$ nên diện tích của mặt nạ bằng $ 2\displaystyle\int_{0}^{4} |f(x) - g(x)|\mathrm{\,d}x$. \\
\textbf{Viết phương trình của $f(x)$: }\\
Ta có đường cong $AIB$ đi qua các điểm $A(-4; 0)$; $I(0; 1)$ và $B(4; 0)$. Từ đó ta có hệ phương trình
\[\heva{&m\cdot(-4)^2-4n+p=0\\&p=1\\&m\cdot4^2+4n+p=0}\Leftrightarrow\heva{&m=-\dfrac{1}{16}&\\&n=0\\&p=1}\Rightarrow f(x)=-\dfrac{1}{16}x^2+1.\]
\textbf{Viết phương trình của $g(x)$: }\\
Vì đồ thị $g(x)$ đi qua $E(0;-4)$ nên $d=-4$. Suy ra $g(x)=ax^3+bx^2+cx-4$.\\
Lại có đường cong $EB$ qua $B(4;0)$ và các điểm $(1;-2)$ và $(3;-3)$. Từ đó ta có hệ phương trình
\[\heva{&a\cdot4^3+b\cdot4^2+4c-4=0\\&a+b+c-4=-2\\&a\cdot3^3+b\cdot3^2+3c-4=-3}\Leftrightarrow\heva{&64a+16b+4c=4\\&a+b+c=2\\&27a+9b+3c=1}\Leftrightarrow \heva{&a=\dfrac{1}{2}\\&b=-\dfrac{17}{6}\\&c=\dfrac{13}{3}.}\]
Suy ra $g(x)=\dfrac{1}{2}x^3-\dfrac{17}{6}x^2+\dfrac{13}{3}x-4$.\\
Suy ra diện tích mặt nạ là
\[2\displaystyle\int_{0}^{4} \left| -\dfrac{1}{16}x^2+1 - \dfrac{1}{2}x^3+\dfrac{17}{6}x^2-\dfrac{13}{3}x+4\right| \mathrm{\,d}x\approx24{,}9.\]
}
\end{ex}

\begin{ex}%[2D4V3-2]%[TexDeMoon2025]%[NguyenKieuNhaTu]
Một biển quảng cáo có dạng hình elip với bốn đỉnh $A_1$, $A_2$, $B_1$, $B_2$ như hình vẽ bên dưới. Biết chi phí để sơn phần tô đậm là $200.000$ (đồng) và phần còn lại $100.000$ (đồng). Biết $A_1A_2=8$ m, $B_1B_2=6$ m và tứ giác $MNPQ$ là hình chữ nhật có $MQ=3$ m.

{\centering\begin{tikzpicture}[scale=0.5,>=stealth, font=\footnotesize, line join=round, line cap=round]
\draw (0,0) ellipse (4 cm and 3 cm);
\draw (-3,2)node[above left]{$M$}--(3,2)node[above right]{$N$}--(3,-2)node[below right]{$P$}--(-3,-2)node[below left]{$Q$}--cycle;
\fill (4,0)node[right]{$A_2$} circle(3pt);
\fill (-4,0)node[left]{$A_1$} circle(3pt);
\fill (0,3)node[above]{$B_2$} circle(3pt);
\fill (0,-3)node[below]{$B_1$} circle(3pt);
\fill[gray,opacity=0.4] plot[domain=-3:3](\x,{3*sqrt(1-(\x)^2/16)})--plot[domain=3:-3](\x,{-3*sqrt(1-(\x)^2/16)})--cycle;
\end{tikzpicture}\par}\vspace{-5pt}\noindent
Hỏi số tiền để sơn theo cách trên (làm tròn đến hàng phần chục, đơn vị triệu đồng) bằng bao nhiêu?
\shortans[]{$7{,}3$}
\loigiai{
\begin{center}
\begin{tikzpicture}[scale=0.5,>=stealth, font=\footnotesize, line join=round, line cap=round]
\draw[->] (-6,0)--(0,0)node[above right]{$O$}--(6,0) node [above]{$x$};
\draw[->] (0,-5)--(0,-1.5)node[right]{$-1{,}5$}--(0,1.5)node[right]{$1{,}5$}--(0,5) node [left]{$y$};
\draw (0,0) ellipse (4 cm and 3 cm);
\draw (-3,2)node[above left]{$M$}--(3,2)node[above right]{$N$}--(3,-2)node[below right]{$P$}--(-3,-2)node[below left]{$Q$}--cycle;
\fill (4,0)node[below right]{$4$} circle(3pt);
\fill (-4,0)node[below left]{$-4$} circle(3pt);
\fill (0,3)node[above right]{$3$} circle(3pt);
\fill (0,-3)node[below right]{$-3$} circle(3pt);
\fill[gray,opacity=0.4] plot[domain=-3:3](\x,{3*sqrt(1-(\x)^2/16)})--plot[domain=3:-3](\x,{-3*sqrt(1-(\x)^2/16)})--cycle;
\end{tikzpicture}
\end{center}
Phương trình của elip có trục lớn $A_1A_2=8$ m và trục nhỏ $B_1B_2=6$ m là
\[\dfrac{x^2}{4^2}+\dfrac{y^2}{3^2}=1\Leftrightarrow\dfrac{x^2}{16}+\dfrac{y^2}{9}=1\Leftrightarrow y=\pm \dfrac{3}{4}\sqrt{16-x^2}.\]
Ta có $MQ=3$ nên $y_N=1{,}5$. Suy ra $x_N=2\sqrt{3}$.\\
Suy ra hình chữ nhật $MNPQ$ có $MN=4\sqrt{3}$.\\
Diện tích phần tô đậm là
\[S_1=2\displaystyle\int_{-2\sqrt{3}}^{2\sqrt{3}}\dfrac{3}{4} \sqrt{16-x^2} \mathrm{\,d}x.\]
Diện tích phần không tô là \[S_2=4\displaystyle\int_{2\sqrt{3}}^{4}\dfrac{3}{4} \sqrt{16-x^2} \mathrm{\,d}x.\]
Vậy tổng số tiền là $T=200\,000\cdot10^{-6}\cdot S_1+100\,000\cdot10^{-6}\cdot S_2\approx 7{,}3$ (triệu đồng).
}
\end{ex}

\begin{ex}%[2D4V2-6]%[TEX Đề Moon 2025]%[Võ Nguyên Thạch]
\immini[thm]
{
Một người có miếng tôn hình tròn có bán kính bằng $5$ (m). Người này tính trang trí sơn vẽ trên tấm tôn đó, biết mỗi mét vuông sơn hết 100 nghìn đồng. Tuy nhiên cần có một khoảng trống để treo tấm tôn nên người này bớt lại một phần tấm tôn nhỏ không trang trí (phần màu trắng như hình vẽ), trong đó $AB=6$(m). Hỏi khi trang trí xong người này hết bao nhiêu tiền chi phí (đơn vị nghìn đồng)?
}
{
\begin{tikzpicture}[scale=0.96,>=stealth, font=\footnotesize, line join=round, line cap=round]
\draw (0,0) circle(2cm);
\fill[blue,opacity=0.4] (1.41,1.41)--plot[domain=1.41:-2,smooth,samples=300](\x,{sqrt(4-(\x)^2)})--plot[domain=-2:1.41,smooth,samples=300](\x,{-sqrt(4-(\x)^2)})--cycle;
\draw (1.41,1.41)circle (1pt)node[above right]{$A$} (1.41,-1.41)circle (1pt)node[below right]{$B$};
\end{tikzpicture}
}
\shortans{7\,445}
\loigiai{
\begin{center}
\begin{tikzpicture}[scale=0.96,>=stealth, font=\footnotesize, line join=round, line cap=round]
\draw [->] (-2.5,0)--(2.5,0) node[below]{$x$};
\draw [->] (0,-2.5)--(0,2.5) node[left]{$y$};
\draw (0,0) node[below left]{$O$};
\draw (0,0) circle(2cm);
\fill[blue,opacity=0.4] (1.41,1.41)--plot[domain=1.41:-2,smooth,samples=300](\x,{sqrt(4-(\x)^2)})--plot[domain=-2:1.41,smooth,samples=300](\x,{-sqrt(4-(\x)^2)})--cycle;
\draw[fill=black] (1.41,1.41)circle (1pt)node[above right]{$A$} (1.41,-1.41)circle (1pt)node[below right]{$B$} (1.41,0)circle (1pt)node[above right]{$H$};
\end{tikzpicture}
\end{center}
Diện tích miếng tôn là $S_1=\pi R^2=25\pi$ (m$^2$).\\
Chọn hệ trục tọa độ $Oxy$ như hình vẽ.\\
Ta có phương trình của đường tròn biên là $x^2+y^2=25$ nên $R=5$; $AH=3\Rightarrow OH=4$.\\
Phương trình của cung tròn nhỏ là $y=\sqrt{25-x^2}$, với $0\le x\le 5$.\\
Diện tích phần không tô màu là $S_2=2\displaystyle\int\limits_4^5{\sqrt{25-x^2}\mathrm{\,d}x}$ (m$^2$).\\
Diện tích phần tô màu là
\[S=S_1-S_2=25\pi-2\displaystyle\int\limits_4^5{\sqrt{25-x^2}\mathrm{\,d}x}~(\text{m}^2).\]
Số tiền thu được là
\[T=100S=10\left(25\pi-2\displaystyle\int\limits_4^5{\sqrt{25-x^2}\mathrm{\,d}x}\right)\approx 7\,445~\text{(nghìn đồng)}.\]
}
\end{ex}

\begin{ex}%[2D1V3-6]%[9D0G1-Ứng dụng cực trị trong thực tế]
Trong một bài thực hành huấn luyện quân sự có một tình huống chiến sĩ phải bơi qua sông để tấn công mục tiêu ở ngay phía bờ bên kia sông. Biết rằng lòng sông rộng $100$ m và vận tốc bơi của chiến sĩ ($v_b$) bằng một phần ba vận tốc chạy trên bộ ($v_c$), tức là $v_c=3v_b$. Biết dòng sông là thẳng, mục tiêu cách chiến sĩ $1$ km theo đường chim bay và chiến sĩ đang ở bờ bên này. Hỏi chiến sĩ phải bơi bao nhiêu mét để đến được mục tiêu nhanh nhất (làm tròn kết quả đến hàng đơn vị)?
\shortans{106}
\loigiai{
\begin{center}
\begin{tikzpicture}[>=stealth,line join=round,line cap=round,font=\footnotesize,scale=1,ultra thick]
\path
(0,0)coordinate (x)+(-90:5)coordinate (x')+(0:3)coordinate(y)
($(x')+(y)-(x)$)coordinate (y')
($(x)!.8!(x')$)coordinate (A)
($(y)!.1!(y')$)coordinate (C)
($(y)!(A)!(y')$)coordinate (B)
($(B)!.3!(C)$)coordinate (D)
;
\path
(A)--(B)node[below,pos=.5]{$100$ m}
(B)--(D)node[right,pos=.5]{$x$}
(A)--(C)node[pos=.5,sloped,above]{$1$ km}
;
\foreach \pointo/\pointt in{x/x',y/y',A/B,A/D,A/C}{
\draw[fill=black](\pointo)--(\pointt);
}
\foreach \point/\goc in{A/180,B/0,D/0,C/0}{
\draw[fill=black](\point)circle(.8pt)+(\goc:2mm)node[scale=.8]{$\point$};
}
\end{tikzpicture}
\end{center}

Gọi vận tốc của chiến sĩ khi bơi là $a$ (m/s), với $a > 0$.\\
$\Rightarrow$ Vận tốc của chiến sĩ khi chạy bộ là $3a$ (m/s).\\
Ta có hình vẽ, khi đó chiến sĩ ở vị trí $A$, mục tiêu ở vị trí $C$.\\
Quãng đường chiến sĩ phải bơi là $AD$, quãng đường chiến sĩ phải chạy bộ là $DC$.\\
Ta có
\[
BC=\sqrt{A\mathrm{C}2-AB^2}=\sqrt{1000^2-100^2}=300\sqrt{11} \text{(m)}.
\]
Đặt $BD=x$ (m), với $0 < x < 300\sqrt{11}$.\\
$\Rightarrow$ Quãng đường chiến sĩ phải bơi là
\[
AD=\sqrt{AB^2+BD^2}=\sqrt{x^2+100^2} \text{(m)}.
\]
Quãng đường chiến sĩ phải chạy bộ là
\[
CD=BC-BD=300\sqrt{11}-x \text{(m)}.
\]
$\Rightarrow$ Thời gian chiến sĩ đến được mục tiêu là
\[
t=\dfrac{AD}{a}+\dfrac{DC}{3a}=\dfrac{\sqrt{x^2+100^2}}{a}+\dfrac{300\sqrt{11}-x}{3a}
=\dfrac{1}{3a} \left( 3\sqrt{x^2+100^2}+300\sqrt{11}-x \right).
\]
Xét hàm số:
\[
f(x)=3\sqrt{x^2+100^2}-x+300\sqrt{11} \text{trên} \left(0;\, 300\sqrt{11}\right),
\]
ta có
\[
f'(x)=\dfrac{3x}{\sqrt{x^2+100^2}}-1.
\]
Khi đó \allowdisplaybreaks
\begin{eqnarray*}
f'(x)=0 &\Rightarrow& \dfrac{3x}{\sqrt{x^2+100^2}}=1\\
&\Leftrightarrow& 3x=\sqrt{x^2+100^2}\\
&\Leftrightarrow&  9x^2=x^2+100^2\\
&\Leftrightarrow&  8x^2=10\,000\\
&\Leftrightarrow&  x^2=1\,250\\
&\Leftrightarrow&  x=\sqrt{1250}.
\end{eqnarray*}
$\Rightarrow$ Quãng đường bơi ngắn nhất là
\[
AD=\sqrt{x^2+100^2}=\sqrt{1250+10\,000}=\sqrt{11\,250} \approx 106 \text{(m)}.
\]
}
\end{ex}

\begin{ex}%[2D1V3-6]
Một gia đình thiết kế chiếc cổng có dạng là một parabol $(P)$ có kích thước như hình vẽ, biết chiều cao cổng bằng chiều rộng của cổng và bằng $4$ m. Người ta thiết kế cửa đi là một hình chữ nhật $CDEF$ sao cho chiều cao cửa đi là $CD=2$ m, phần còn lại dùng để trang trí. Biết chi phí phần tô đậm là $1{,}5$ triệu đồng/m$^2$. Tính số tiền (triệu đồng) gia đình đó phải trả để trang trí phần tô đậm (làm tròn kết quả đến hàng phần mười).
\begin{center}
\begin{tikzpicture}[scale=0.8,>=stealth, font=\footnotesize, line join=round, line cap=round]
\draw[smooth,samples=300,domain=-2:2] plot(\x,{-(\x)^2+4});
\draw (-2,0)--(2,0) (-1.4,0)node[below]{$F$}--(-1.4,2.04)node[above left]{$E$}--(1.4,2.04)node[above right]{$D$}--(1.4,0)node[below]{$C$} (1,3.7)node[]{$(P)$} (2,2)node[right]{$4$ m} (0,-0.8)node[below]{$4$ m};
\fill[gray,opacity=0.7] plot[domain=-2:-1.4](\x,{-(\x)^2+4})--(-1.4,0)--cycle plot[domain=1.4:2](\x,{-(\x)^2+4})--(1.4,0)--cycle plot[domain=-1.4:1.4](\x,{-(\x)^2+4})--(1.4,2.04)--(-1.4,2.04)--cycle;
\draw[<->] (2,0)--(2,4);
\draw[<->] (-2,-0.8)--(2,-0.8);
\end{tikzpicture}
\end{center}
\shortans{$7{,}5$}
\loigiai{
\begin{center}
\begin{tikzpicture}[scale=0.8,>=stealth, font=\footnotesize, line join=round, line cap=round]
\draw[smooth,samples=300,domain=-2:2] plot(\x,{-(\x)^2+4});
\draw (-2,0)--(2,0) (-1.4,0)node[below]{$F$}--(-1.4,2.04)node[above left]{$E$}--(1.4,2.04)node[above right]{$D$}--(1.4,0)node[below]{$C$} (1,3.7)node[]{$(P)$} (2,2)node[right]{$4$ m} (0,-0.8)node[below]{$4$ m};
\fill[gray,opacity=0.7] plot[domain=-2:-1.4](\x,{-(\x)^2+4})--(-1.4,0)--cycle plot[domain=1.4:2](\x,{-(\x)^2+4})--(1.4,0)--cycle plot[domain=-1.4:1.4](\x,{-(\x)^2+4})--(1.4,2.04)--(-1.4,2.04)--cycle;
\draw[->] (-3,0)--(3,0)node[above]{$x$} ;
\draw[->] (0,-1)--(0,5)node[left]{$y$} ;
\draw[<->] (2,0)--(2,4);
\draw[<->] (-2,-0.8)--(2,-0.8);
\foreach \x/\goc in {-2/-110,2/-110}{
\draw[fill=black] (\x,0)circle(1.2pt) node[shift={(\goc:2.8mm)},scale=.8]{$\x$};
}
\foreach \y/\goc in {4/135}{
\draw[fill=black] (0,\y)circle(1.2pt)node[shift={(\goc:2.8mm)},scale=.8]{$\y$};
}
\end{tikzpicture}
\end{center}
Giả sử parabol $(P)$ có phương trình là $y=ax^2+bx+c$ ($a \neq 0$). $(P)$ đi qua ba điểm $(0;4)$, $(-2;0)$ và $(2;0)$. \\
Khi đó, ta có
$\heva{
& 4=a \cdot 0^2+b \cdot 0+c \\
& 0=a \cdot 2^2+b \cdot 2+c \\
& 0=a \cdot (-2)^2+b \cdot (-2)+c
}
\Leftrightarrow
\heva{
& a=-1 \\
& b=0 \\
& c=4.
}$\\
Vậy $(P)\colon y=-x^2+4$.\\
Điểm $D$ và $E$ thuộc đồ thị của tiếp tuyến đường thẳng có phương trình $-x^2+4=2 \Leftrightarrow x=\pm \sqrt{2}$.\\ Theo đó thì, $D(\sqrt{2};2)$ và $E(-\sqrt{2};2)$.\\
Chiều dài cạnh của $DE$ là $2\sqrt{2}$ (m).\\
Diện tích của $S_{CDEF}$ là $2 \cdot 2\sqrt{2}=4\sqrt{2}$ (m$^2$).\\
Diện tích phần đồ thị $(P)$ tạo với trục hoành là $S=\displaystyle\int\limits_{-2}^2 (-x^2+4) \, \mathrm{\,d}x=\dfrac{32}{3}$ (m$^2$).\\
Diện tích cần trang trí là $S_1=S-S_{CDEF}=\dfrac{32}{3}-4\sqrt{2}=\dfrac{32-12\sqrt{2}}{3}$ (m$^2$).\\
Chi phí để trang trí là $\dfrac{32-12\sqrt{2}}{3} \cdot 1{,}5 \approx 7{,}5$ (triệu đồng).
}
\end{ex}

\begin{ex}%[2D1V3-6]%[TEX ĐỀ MOON 2025]%[Huỳnh Thanh Chí]
Có hai xã $A$, $B$ cùng ở một bên phía bờ sông. Khoảng cách từ hai xã đó đến bờ sông lần lượt là $AA'=500$ m, $BB'=600$. Người ta đo được $A'B'=2200$ m như hình vẽ bên. Các kỹ sư muốn xây dựng một trạm cung cấp nước sạch bên bờ sông cho người dân của hai xã đã sử dụng. Để tiết kiệm chi phí, các kỹ sư phải chọn một vị trí $M$ của trạm cung cấp nước sạch đó trên đoạn $A'B'$ sao cho tổng khoảng cách từ hai xã đến vị trị $M$ là nhỏ nhất.
\begin{center}
\begin{tikzpicture}[scale=0.48,>=stealth, font=\footnotesize, line join=round, line cap=round]
\coordinate (A') at (0,0);
\coordinate (B') at (8,0);
\coordinate (M) at (3.5,0);
\coordinate (A) at (0,3);
\coordinate (B) at (8,6);
\draw (A)--(A')--(B')--(B);
\draw ($(A)!0.5!(A')$)node[left]{$500$ m} ($(B)!0.5!(B')$)node[right]{$600$ m} (A)--(M)node[below]{$M$}--(B) (4,-1.5)node[below]{$2\,200$ m};
\foreach \x/\g in {A/135,B/90,A'/-135,B'/-45}
\fill[black] (\x) circle (3pt) ($(\g:8mm)+(\x)$) node {$\x$};
\draw[<->, dashed] (0,-1.5)--(8,-1.5);
\end{tikzpicture}
\end{center}
Giá trị nhỏ nhất của tổng khoảng cách đó bằng bao nhiêu mét? (Làm tròn kết quả đến hàng đơn vị).

\shortans[]{$2460$}
\loigiai{
Đặt $AA'=5$, $BB'=6$, $A'B'=22$.\\
Đặt $A'M=x$, khi đó $MB'=22-x$ với $0<x<22$.\\
Ta có $AM=\sqrt{AA'^2+A'M^2}=\sqrt{25+x^2}$ và $MB=\sqrt{BB'^2+MB'^2}=\sqrt{x^2-44x+520}$.\\
Khi đó, yêu cầu bài toán là tìm $T=AM+MB$ đạt giá trị nhỏ nhất.\\
Đặt $f(x)=\sqrt{25+x^2}+\sqrt{x^2-44x+520}$ với $0<x<22$.\\
Ta có $f'(x) = \dfrac{x}{\sqrt{25+x^2}} + \dfrac{x-22}{\sqrt{x^2-44x+520}}$. \\
Xét phương trình
\allowdisplaybreaks
\begin{eqnarray*}
&&f'(x) = 0 \\
&\Leftrightarrow& \dfrac{x}{\sqrt{25+x^2}} = \dfrac{22-x}{\sqrt{x^2-44x+520}} \\
&\Leftrightarrow& \dfrac{x^2}{25+x^2} = \dfrac{(22-x)^2}{x^2-44x+520} \\
&\Leftrightarrow& x^2(x^2-44x+520) = (25+x^2)(x^2-44x+484) \\
&\Leftrightarrow& x^4 - 44x^3 + 520x^2 = 25x^2 - 1100x + 12100 + x^4 - 44x^3 + 484x^2\\
&\Leftrightarrow& -11x^2 - 1100x + 12100 = 0\\
&\Leftrightarrow& \hoac{& x=10\\ & x= -110 \,(\text{loại}).}
\end{eqnarray*}
Bảng biến thiên
\begin{center}
\begin{tikzpicture}
\tkzTabInit[nocadre]
{$x$/1,$f'(x) $/1,$f(x)$/3}
{$0$,$10$,$22$}
\tkzTabLine{,-,0,+,} %
\tkzTabVar{+/,-/$11\sqrt{5}$, +/} %dấu mũi tên, + trên, -dưới
\end{tikzpicture}
\end{center}
Dựa vào bảng biến thiên thì $\min f(x)=f(10)=11\sqrt{5}$.\\
Vậy giá trị nhỏ nhất của tổng khoảng cách đó bằng $100\cdot 11\sqrt{5}\approx 2460$ m.
}
\end{ex}

\begin{ex}%[2D1V3-6]
Một chiếc phà chạy giữa đất liền và đảo Dedlos. Phà có công suất tối đa là $1\,000$ xe hơi mỗi chuyến, nhưng việc tải gần hết công suất rất tốn thời gian. Biết rằng số lượng xe hơi đưa lên phà mỗi chuyến là $f(t)=\dfrac{2000t}{2t+1}$ và mất một khoảng thời gian là $1$ giờ. Mỗi xe cần trung bình $3{,}6$ giây để dỡ xuống khi đến điểm đích. Thời gian di chuyển đến đảo và thời gian vòng về đều mất $1{,}28$ giờ. Nên tải bao nhiêu xe lên phà cho mỗi chuyến đi để lượng xe trung bình di chuyển qua lại đảo mỗi giờ đạt lớn nhất? (làm tròn kết quả đến hàng đơn vị).
\shortans{615}
\loigiai{
Để đưa được $f(t)$ xe lên phà cần $t$ giờ.\\
Tổng thời gian đưa xe qua đảo hoặc từ đảo về là $t+\dfrac{3{,}6}{3\,600}f(t)+1{,}28$ giờ.\\
Số xe di chuyển trung bình mỗi giờ là $g(t)$, với
\[g(t)=\dfrac{f(t)}{t+\dfrac{3{,}6}{3\,600}f(t)+1{,}28}=\dfrac{\dfrac{2000t}{2t+1}}{t+\dfrac{3{,}6}{3\,600}\cdot\dfrac{2000t}{2t+1}+1{,}28}=\dfrac{2\,000t}{2t^2+5{,}56t+1{,}28}=\dfrac{2\,000}{2t+\dfrac{1{,}28}{t}+5{,}56}.\]
Áp dụng bất đẳng thức Côsi cho $2$ số dương $2t$ và $\dfrac{1{,}28}{t}$, ta có $2t + \dfrac{1{,}28}{t} \ge 2\sqrt{2t\cdot\dfrac{1{,}28}{t}}$.\\
Suy ra $\dfrac{2\,000}{2t + \dfrac{1{,}28}{t} + 5{,}56} \le \dfrac{2\,000}{2\sqrt{2t\cdot\dfrac{1{,}28}{t}} + 5{,}56} = \dfrac{50\,000}{219}$.\\
Dấu bằng xảy ra khi $2t = \dfrac{1{,}28}{t} \Leftrightarrow t^2 = 0{,}64 \Leftrightarrow t = 0{,}8$.\\
Vậy để lượng xe trung bình di chuyển qua lại đảo mỗi giờ đạt lớn nhất cần tải lên phà mỗi chuyến $ f(0,8) \approx 615$ xe.
}
\end{ex}

\begin{ex}%[2D1V3-6]%[TEX ĐỀ MOON 2025]%[Nguyễn Thế Duy]
Một khách sạn có $50$ phòng. Hiện tại mỗi phòng cho thuê với giá $400$ nghìn đồng một ngày thì toàn bộ phòng được thuê hết. Biết rằng cứ mỗi lần tăng giá thêm $20$ nghìn đồng một phòng thì có thêm $2$ phòng trống. Giám đốc phải chọn giá phòng mới là bao nhiêu để thu nhập của khách sạn trong ngày là lớn nhất?
\shortans{$450$}
\loigiai{
Gọi $x$ là số phòng trống của khách sạn $\left(x \in \mathbb{N} \right)$.\\
Khách sạn có $x$ phòng trống khi tăng giá thuê $1$ phòng thêm $\dfrac{20}{2} \cdot x = 10x$ nghìn đồng.\\
Thu nhập của khách sạn là $T(x) = \left(50 - x \right) \left(400 + 10x \right) = -10x^2 + 100x + 200\,000$.\\
Dễ thấy $T(x)$ đạt giá trị lớn nhất khi $x = \dfrac{-100}{2 \cdot (-10)} = 5$.\\
Khi đó giá cho thuê mỗi phòng là $450$ nghìn đồng.
}
\end{ex}

\begin{ex}%[2D1V3-6]%[TEX ĐỀ MOON 2025]%[Nguyễn Thế Duy]
Một nhà sản xuất muốn thiết kế một chiếc hộp có dạng hình hộp chữ nhật không có nắp, có đáy là hình vuông và diện tích bề mặt bằng $108$ cm$^2$. Tìm tích của các kích thước của chiếc hộp sao cho thể tích của hộp là lớn nhất?
\shortans{$108$}
\loigiai{
\immini{
Gọi $x$ là chiều dài cạnh hình vuông đáy $\left(x > 0 \right)$.\\
Gọi $y$ là chiều cao của chiếc hộp.\\
Theo bài ta có, diện tích bề mặt của chiếc hộp \\
$108 = x^2 + 4xy \Rightarrow y = \dfrac{108 - x^2}{4x}$.\\
Khi đó, thể tích của chiếc hộp là $V(x) = x^2 \cdot y = 27x - \dfrac{x^3}{4}$.\\
Ta cần tìm $\underset{\left(0;+108 \right)}{\max}V(x)$.\\
Ta có $V'(x) = 27 - \dfrac{3x^2}{4}$. \\
Xét $V'(x) = 0 \Leftrightarrow \hoac{&x = 6 \quad (\text{thoả mãn})\\&x = -6 \quad (\text{không thoả mãn}).}$
}
{\begin{tikzpicture}[scale=0.9,>=stealth, font=\footnotesize, line join=round, line cap=round]
\path
(0,0) coordinate (A)
(1.5,1.5) coordinate (B)
(5.5,1.5) coordinate (C)
(4,0) coordinate (D)
;
\foreach \x in {A,B,C,D}{
\path
($(\x)+(0,3)$) coordinate (\x');
}
\draw
(A) -- (D) -- (C) -- (C') -- (B') -- (A') -- (D') -- (C')
(A) -- (A') node[pos=0.5,left]{$y$}
(D) -- (D')
;
\draw[->] (3.5,5) node[above]{không có nắp} arc(160:200:2.2)
;
\path
(A) -- (D) node[pos=0.5,below]{$x$}
(C) -- (D) node[pos=0.5,right]{$x$}
;
\draw[dashed]
(A) -- (B) -- (C)
(B) -- (B')
;

\end{tikzpicture}
}
\noindent Ta có bảng biến thiên
\begin{center}
\begin{tikzpicture}
\tkzTabInit[nocadre=false,lgt=1.2,espcl=2.7,deltacl=0.6]
{$x$ /0.6, $y'$ /0.6, $y$ /2}
{$0$,$6$,$108$}
\tkzTabLine{,+,$0$,-,}
\tkzTabVar{-/$0$,+/$108$,-/$0$}
\end{tikzpicture}
\end{center}
Suy ra thể tích của chiếc hộp lớn nhất khi $x = 6$ khi đó $y = 3$.\\
Tích các kích thước của hộp là $3 \cdot 3 \cdot 4 = 108$.
}
\end{ex}

\begin{ex}%[2D1V3-6]%[TexDeMoon2025]%[NguyenKieuNhaTu]
Một cơ sở sản xuất khăn mặt đang bán mỗi chiếc khăn với giá $30000$ đồng một chiếc và mỗi tháng cơ sở bán được trung bình $3000$ chiếc khăn. Cơ sở sản xuất đang có kế hoạch tăng giá bán để có lợi nhận tốt hơn. Sau khi tham khảo thị trường, người quản lý thấy rằng nếu từ mức giá $30000$ đồng mà cứ tăng giá thêm $1000$ đồng thì mỗi tháng sẽ bán ít hơn $100$ chiếc. Biết vốn sản xuất một chiếc khăn không thay đổi là $18000$. Để đạt lợi nhuận lớn nhất thì mỗi chiếc khăn cần bán với giá bao nhiêu nghìn đồng?
\shortans[]{$39$}
\loigiai{
Gọi $x$ là số lần tăng giá 1.000 đồng, khi đó:

\begin{itemize}
\item Giá bán mỗi khăn sẽ là: $30+x$ (nghìn đồng).
\item Số khăn bán ra mỗi tháng là: $3000-100x$ (chiếc).
\item Lợi nhuận trên mỗi khăn là: $(30+x)-18=12+x$ (nghìn đồng).
\end{itemize}
Tổng lợi nhuận mỗi tháng là: $L(x)=(3000-100x)(12+x)$.\\
Khai triển biểu thức:
\begin{eqnarray*}
L(x) &=&3000\cdot (12+x)-100x(12+x) \\
&=&36000+3000x-1200x-100x^2\\
&=&-100x^2+1800x+36000
\end{eqnarray*}
Đây là hàm bậc hai có hệ số $a=-100< 0$, nên đạt cực đại tại: $x=\dfrac{-b}{2a}=\dfrac{-1800}{2\cdot (-100)}=9$.\\
Vậy để đạt lợi nhuận lớn nhất, mỗi chiếc khăn cần bán với giá $30+9=39$ (nghìn đồng).
}
\end{ex}

\begin{ex}%[2D1V3-2]%[TEX Đề Moon 2025]%[Vũ Hồng Toàn]
Anh Vinh đang cắm trại dưới tán cây thông ở điểm $X$ cách điểm $A$ một khoảng $3$ km. Điểm $A$ nằm trên đường bờ biển (đường bờ biển là đường thẳng). Ô tô của anh Vinh đỗ ở vị trí $Y$ cách điểm $B$ một khoảng $3$ km. Điểm $B$ cũng thuộc đường bờ biển. Biết rằng $AB=18$ km, $AM=NB=x$ km và $AX=BY=3$ km (minh hoạ như hình vẽ).
\begin{center}
\begin{tikzpicture}[scale=1,>=stealth, font=\footnotesize, line join=round, line cap=round]
\coordinate (A) at (0,0);
\coordinate (M) at (2,0);
\coordinate (N) at (5,0);
\coordinate (B) at (7,0);
\coordinate (X) at (0,3);
\coordinate (Y) at (7,3);
\draw (X)--(A)node[below]{$A$} (Y)--(B)node[below]{$B$} ($(A)!1.3!(B)$)--($(B)!1.3!(A)$);
\draw[dashed] (X)node[above]{$X$}--(M)node[below]{$M$} (Y)node[above]{$Y$}--(N)node[below]{$N$} (0,1.5)node[left]{$3$ km} (7,1.5)node[right]{$3$ km} ($(M)!0.5!(N)$)node[below]{Bờ biển} (3.5,-0.8)node[below]{$18$ km};
\draw[<->] (0,-0.8)--(7,-0.8);
\fill (A)circle(2pt);
\fill (B)circle(2pt);
\fill (M)circle(2pt);
\fill (N)circle(2pt);
\fill (X)circle(2pt);
\fill (Y)circle(2pt);
\draw pic[draw,angle radius=4mm]{right angle=M--A--X};
\draw pic[draw,angle radius=4mm]{right angle=Y--B--N};
\end{tikzpicture}
\end{center}
Khi đang dựng trại tại vị trí $X$, anh Vinh không may bị rắn cắn, chất độc lan vào máu. Sau khi bị rắn cắn, nồng độ chất độc trong máu tăng theo thời gian được tính theo phương trình $y=50\log(t+2)$. Trong đó, $y$ là nồng độ, $t$ là thời gian tính bằng giờ sau khi bị rắn cắn. Anh Vinh cần quay trở lại ô tô ở vị trí $Y$ để lấy thuốc giải độc. Anh chạy từ chỗ cây thông ở điểm $X$ ra thẳng vị trí $M$  với vận tốc là $5$\,km/h và chạy trên bãi biển từ $M$ tới điểm $N$ với vận tốc là $13$\,km/h sau đó chạy thẳng đến chỗ ô tô với vận tốc $5$\,km/h. Tính nồng độ chất độc trong máu thấp nhất khi anh Vinh về đến ô tô (kết quả làm tròn đến hàng phần chục).
\shortans{$32{,}6$}
\loigiai{
Khoảng cách $XM$ là $XM=\sqrt{A X^2+A M^2}=\sqrt{3^2+x^2}=\sqrt{9+x^2}$.\\
Khoảng cách $MN$ là $MN=18-2x$.\\
Khoảng cách $NY$ là $NY=\sqrt{N B^2+B Y^2}=\sqrt{x^2+3^2}=\sqrt{x^2+9}$.\\
Tổng thời gian $f(x)=\dfrac{\sqrt{9+x^2}}{5}+\dfrac{18-2 x}{13}+\dfrac{\sqrt{x^2+9}}{5}=\dfrac{2\sqrt{x^2+9}}{5}+\dfrac{18-2 x}{13}$.\\
Ta có $f'(x)=\dfrac{2x}{5\sqrt{x^2+9}}-\dfrac{2}{13}=\dfrac{26x-10\sqrt{x^2+9}}{65\sqrt{x^2+9}}$.
\allowdisplaybreaks
\begin{eqnarray*}
&&f'(x)=0\\
&\Rightarrow& 26x-10\sqrt{x^2+9}=0\\
&\Rightarrow& 5\sqrt{x^2+9}=13x\\
&\Rightarrow& 25(x^2+9)=169x^2\\
&\Rightarrow& 144x^2-225=0\Rightarrow\hoac{&x=\dfrac{5}{4}&\text{(nhận)}\\&x=-\dfrac{5}{4}&\text{(loại)}.}
\end{eqnarray*}
Bảng biến thiên\\
\centerline{
\begin{tikzpicture}
\tkzTabInit
{$x$/0.7,$f'(x)$/0.7,$f(x)$/2.1}
{$0$,$\tfrac{5}{4}$,$+\infty$}
\tkzTabLine{,-,$0$,+}
\tkzTabVar{+/,-/$\dfrac{162}{65}$,+/}
\end{tikzpicture}
}
Nồng độ chất độc trong máu thấp nhất khi thời gian về đến ô tô là nhỏ nhất là $t=\dfrac{162}{65}$.\\
Nồng độ chất độc
\[
y=50 \log (t+2)=50 \log\left (\dfrac{162}{65}+2\right)\approx 32{,}6.
\]
}
\end{ex}

\begin{ex}%[2D1V3-6]%[TEX Đề Moon 2025]%[Võ Nguyên Thạch]
Nhà máy $A$ chuyên sản xuất một loại sản phẩm cung cấp cho nhà máy $B$. Hai nhà máy thoả thuận rằng, hàng tháng nhà máy $A$ cung cấp cho nhà máy $B$ số lượng sản phẩm theo đơn đặt hàng của $B$ (tối đa $100$ tấn sản phẩm). Nếu số lượng đặt hàng là $x$ tấn sản phẩm thì giá bán cho mỗi tấn sản phẩm là $P(x)=45-0{,}001x^2$ (triệu đồng). Chi phí để $A$ sản xuất $x$ tấn sản phẩm trong một tháng gồm $100$ triệu đồng chi phí cố định và $30$ triệu đồng cho mỗi tấn sản phẩm. Nhà máy $A$ cần bán cho nhà máy $B$ bao nhiêu tấn sản phẩm mỗi tháng để lợi nhuận thu được lớn nhất? (Làm tròn kết quả đến hàng phần mười).
\shortans{70{,}7}
\loigiai{
Doanh thu của nhà máy khi sản xuất $1$ tấn sản phẩm là $P(x)$ triệu đồng.\\
Doanh thu của nhà máy khi sản xuất $x$ tấn sản phẩm là $xP(x)$ triệu đồng.\\
Chi phí của nhà máy khi sản xuất $x$ tấn sản phẩm là $C(x)$ triệu đồng.\\
Vì Lợi nhuận = Doanh thu – Chi phí nên ta có lợi nhuận của nhà máy $A$ khi sản xuất $x$ tấn sản phẩm là
\[H(x)=xP(x)-C(x)=x(45-0{,}001x^2)-(100+30x)=-0{,}001x^2+15x-100, \text{ với } 0\le x\le 100.\]
Ta có $H'(x)=-0{,}003x^2+15=0\Leftrightarrow \hoac{&x=50\sqrt 2\\&x=-50\sqrt 2.}$\\
Chỉ có $x=50\sqrt 2$ thỏa điều kiện.\\
Ta có $H(0)=-100$; $H(50\sqrt 2)=500\sqrt 2$; $H(100)=400$.\\
Vậy lợi nhuận lớn nhất khi $A$ sản xuất $50\sqrt 2\approx 70{,}7$ tấn sản phẩm.
}
\end{ex}

% \paragraph{Mức độ C}
\begin{ex}%[1D6C4-6]%[TEX ĐỀ MOON 2025]%[Nguyễn Thế Duy]
Bác An vay ngân hàng $900$ triệu đồng theo hình thức lãi kép và trả góp hàng tháng. Cuối mỗi tháng bắt đầu từ tháng thứ nhất Bác An trả $12$ triệu đồng và chịu lãi suất $0{,}95\%$ trên tháng cho số tiền chưa trả. Với hình thức hoàn nợ như vậy thì sau bao nhiêu tháng Bác An sẽ trả hết số nợ ngân hàng, biết rằng lãi suất không đổi trong suốt quá trình vay.
\shortans{$132$}
\loigiai{
Số tiền bác An phải trả sau $n$ tháng là $S_n = 900 \left(1 + \dfrac{0{,}95}{100} \right)^n$.\\
Số tiền bác An đã trả tính đến tháng thứ $n$ là $P_n = 12 \cdot \dfrac{\left(1+0{,}0095 \right)^n - 1}{0{,}0095}$.\\
Bác An trả xong nợ khi $S_n = P_n$ hay
\begin{align*}
900 \left(1 + 0{,}0095 \right)^n = 12 \cdot \dfrac{\left(1+0{,}0095 \right)^n - 1}{0{,}0095} &\Leftrightarrow 3{,}45 \cdot 1{,}0095^n = 12\\
&\Leftrightarrow n = \log_{1{,}0095}{\dfrac{12}{3{,45}}} \approx 131{,}83.
\end{align*}
Vậy với hình thức trên bác An sẽ trả hết nợ sau $132$ tháng.
}
\end{ex}

\begin{ex}%[2H5C2-6]%[Tex đề Moon 2025]%[Nguyễn Hồng Thạch]
Trong một phần mềm 3D mô phỏng một trò chơi điện tử, có hai chất điểm $A$, $B$ luôn chuyển động trên một mặt cầu $(S)$ và cách nhau một khoảng không đổi bằng $1$. Nếu đặt trong không gian tọa độ $Oxyz$, mặt cầu $(S)$ có phương trình là $(x-3)^2+(y+4)^2+z^2=4$. Tìm giá trị nhỏ nhất của biểu thức $OA^2-OB^2$?
\shortans{$-10$}
\loigiai{
Mặt cầu $(S)$ có tâm $I(3; -4; 0)$ và bán kính $R = 2$.\\
Ta có $\heva{&OA^2 = x_A^2 + y_A^2 + z_A^2\\
&OB^2 = x_B^2 + y_B^2 + z_B^2.}$\\
Mặt khác
$\heva{&IA^2 = (x_A - 3)^2 + (y_A + 4)^2 + (z_A - 0)^2 = x_A^2 - 6x_A + 9 + y_A^2 + 8y_A + 16 + z_A^2= 4\\
&IB^2 = (x_B - 3)^2 + (y_B + 4)^2 + (z_B - 0)^2 = x_B^2 - 6x_B + 9 + y_B^2 + 8y_B + 16 + z_B^2 = 4.}$\\
Ta có
\[OA^2 = IA^2 - 9 - 16 + 6x_A - 8y_A = 4 - 25 + 6x_A - 8y_A = 6x_A - 8y_A - 21.\]
\[OB^2 = IB^2 - 9 - 16 + 6x_B - 8y_B = 4 - 25 + 6x_B - 8y_B = 6x_B - 8y_B - 21.\]
Suy ra \[OA^2 - OB^2 = (6x_A - 8y_A - 21) - (6x_B - 8y_B - 21) = 6(x_A - x_B) - 8(y_A - y_B).\]
Ta có $\overrightarrow{AB} = (x_B - x_A; y_B - y_A; z_B - z_A)$, $\overrightarrow{OI} = (3; -4; 0)$.\\
Suy ra $\overrightarrow{OI} \cdot \overrightarrow{AB} = 3\cdot(x_B - x_A) - 4\cdot(y_B - y_A) + 0\cdot(z_B - z_A) = 3\cdot(x_B - x_A) - 4\cdot(y_B - y_A)$.\\
Khi đó $OA^2 - OB^2 = 2\left[3\cdot(x_A - x_B) - 4\cdot(y_A - y_B)\right] =-2(\overrightarrow{OI} \cdot \overrightarrow{AB})$.\\
Mặt khác $\left|\overrightarrow{OI} \cdot \overrightarrow{AB}\right| \le \left|\overrightarrow{OI}\right| \cdot \left|\overrightarrow{AB}\right|$.\\
Với $\left|\overrightarrow{OI}\right| = \sqrt{3^2 + (-4)^2 + 0^2} = 5$ và $\left|\overrightarrow{AB}\right| = 1$.\\
Suy ra $|\overrightarrow{OI} \cdot \overrightarrow{AB}| \le 5 \cdot 1 = 5$\\
Hay $-5 \le \overrightarrow{OI} \cdot \overrightarrow{AB} \le 5$.\\
Khi đó $-2\cdot(5) \le OA^2 - OB^2 \le -2\cdot(-5)$ hay $-10 \le OA^2 - OB^2 \le 10$.\\
Vậy giá trị nhỏ nhất của $OA^2 - OB^2$ là $-10$.
}
\end{ex}

\begin{ex}%[2H5C2-3]
Trong không gian $Oxyz$, cho tam giác $ABC$, đường phân giác $AM$ với $M\in BC$, $M(2;0;4)$. Biết điểm $B$ thuộc đường thẳng $\dfrac{x}{1}=\dfrac{y}{1}=\dfrac{z}{1}$, điểm $C$ thuộc mặt phẳng $2x+y-z-2=0$ và $AB=2AC$. Đường thẳng $BC$ có một vectơ chỉ phương là $(a;1;b)$. Tính $a+b$.
\shortans{$5$}
\loigiai{ Gọi  $B(2 t ; 2 t ; 2 t) \in d\colon \dfrac{x}{1}=\dfrac{y}{1}=\dfrac{z}{1}$, $
C\left(x_C ; y_C ; z_C\right) \in(P)\colon 2 x+y-z-2=0$.\\ Ta có $\overrightarrow{B M}=(2-2 t ;-2 t ; 4-2 t)$, $\overrightarrow{M C}=\left(x_C-2 ; y_C ; z_C-4\right)$.\\
Tam giác $ABC$ có đường phân giác trong $AM$ với
\begin{eqnarray*}
M \in B C &\Rightarrow& \dfrac{M C}{M B}=\dfrac{A C}{A B}=\dfrac{1}{2} \\
&\Rightarrow& \overrightarrow{M C}=\dfrac{1}{2} \overrightarrow{B M}\\
&\Rightarrow& \heva{&x_C-2=\dfrac{1}{2}(2-2 t)=1-t \\
&y_C=\dfrac{1}{2}(-2 t)=-t \\
&z_c-4=\dfrac{1}{2}(4-2 t)=2-t
} \\
&\Rightarrow&\heva{
&x_C=3-t \\
&y_C=-t \\
&z_c=6-t
}\\
&\Rightarrow& C(3-t ;-t ; 6-t) .
\end{eqnarray*}
Vì $C\in (P) \Rightarrow 2(3-t)-t-(6-t)-2=0 \Rightarrow t=-1$. \\Khi đó $B(-2 ;-2 ;-2)$, $C(4 ; 1 ; 7) \Rightarrow \overrightarrow{B C}(6 ; 3 ; 9)$.\\
Suy ra $\overrightarrow{n_{B C}}=(2 ; 1 ; 3) \Rightarrow a=2,b=3\Rightarrow a+b=5$.
}
\end{ex}

\begin{ex}%[2D6C2-3]
Một công nhân đi làm ở thành phố khi trở về nhà chỉ có $2$ cách hoặc đi theo đường ngầm hoặc đi qua cầu. Nếu đi lối đường ngầm $75\%$ trường hợp ông ta về đến nhà trước $6$ giờ tối; còn nếu đi lối cầu chỉ có $70\%$ trường hợp (nhưng đi lối cầu thích hơn). Vợ ông ta nhận thấy rằng: Bình quân cứ $100$ lần về nhà thì $71$ lần ông ta về nhà trước $6$ giờ tối. Tìm xác suất để công nhân đó đã đi lối cầu biết rằng ông ta về đến nhà sau $6$ giờ tối (kết quả làm tròn đến hàng phần trăm).
\shortans[]{$0{,}83$}
\loigiai{
\begin{center}
\begin{tikzpicture}[
font=\small,
node distance=4cm and 3.5cm,
>=Stealth,
every node/.style={align=center},
state/.style={draw, minimum height=1.2cm, minimum width=2.5cm, rounded corners},
box/.style={draw, minimum height=1cm, minimum width=2.5cm, rounded corners},
good/.style={box, fill=red!10, draw=red},
mid/.style={box, fill=blue!15},
mid2/.style={box, fill=green!15}
]

% Nodes
\node[state, fill=purple!10] (start) {Gốc};
\node[mid, right=of start, yshift=1.5cm] (underground) {Ông ta đi đường ngầm};
\node[mid2, right=of start, yshift=-1.5cm] (bridge) {Ông ta đi qua cầu};

\node[good, right=of underground, yshift=0.8cm] (early1) {Về nhà trước 6 giờ tối};
\node[box, right=of underground, yshift=-0.8cm] (late1) {Về nhà sau 6 giờ tối};

\node[good, right=of bridge, yshift=0.8cm] (early2) {Về nhà trước 6 giờ tối};
\node[box, right=of bridge, yshift=-0.8cm] (late2) {Về nhà sau 6 giờ tối};

% Edges from Gốc
\draw[->] (start) -- node[above left] {$x$} node[below left] {$B$} (underground);
\draw[->] (start) -- node[below left] {$\overline{B}$} node[above left] {$1-x$} (bridge);

% Edges from underground
\draw[->] (underground) -- node[above] {$A|B$} node[below] {0.75} (early1);
\draw[->] (underground) -- node[below] {$\overline{A}|B$} (late1);

% Edges from bridge
\draw[->] (bridge) -- node[above] {$A|\overline{B}$} node[below] {0.7} (early2);
\draw[->] (bridge) -- node[below] {$\overline{A}|\overline{B}$} (late2);

\end{tikzpicture}
\end{center}
Gọi $B$ là biến cố ông ta đi đường ngầm.\\
Gọi $A$ là biến cố về nhà trước $6$ giờ tối.\\
Ta có dữ kiện: Bình quân cứ $100$ lần về nhà thì $71$ lần ông ta về nhà trước $6$ giờ tối nên\\
$P(A)=0{,}71$.
$\Leftrightarrow 0{,}75x +0{,}7(1-x)=0{,}71\Leftrightarrow x=0{,}2$.\\
Suy ra $P(B)=0{,}2$; $P(\overline{B})=0{,}8$.\\
Do đó $P(\overline{B}|\overline{A})=\dfrac{P(\overline{B})\cdot P(\overline{A}|\overline{B})}{P(\overline{A})}=\dfrac{1-0{,7}\cdot0{,8}}{1-0{,}71}\approx 0{,}83$.
}

\end{ex}

\begin{ex}%[2D4C3-5]
\immini[thm]
{
Một người thợ gốm sứ muốn thiết kế một cái bình hoa bằng cách quay hình $(H)$ (phần gạch chéo trong hình vẽ bên) quanh trục $AB$. Hình phẳng $(H)$ nằm trong hình chữ nhật $ABCD$, giới hạn bởi các đoạn thẳng $AM$, $BP$ ($M$, $P$ là hai điểm lần lượt thuộc các cạnh $AD$, $BC$, $MP\parallel CD$), cung tròn $MN$ (có tâm là trung điểm của đoạn thẳng $AE$) và cung parabol $NP$. Biết $AB=5$ dm, $AM=BE=1$ dm. Tiếp tuyến của cung tròn và cung parabol tại điểm $N$ là trùng nhau. Bình hoa đó có thể tích bằng bao nhiêu lít? Kết quả làm tròn đến hàng phần mười.
}
{
\begin{tikzpicture}[scale=1,>=stealth, font=\footnotesize, line join=round, line cap=round,rotate=90]
\fill[pattern=north east lines,opacity=0.6] plot[domain=0:4](\x,{sqrt(5-(\x-2)^2)})--plot[domain=4:5](\x,{2*(\x)^2-18*(\x)+41})--(5,0)--(0,0)--cycle;
\draw (0,0)node[below right]{$A$}--(0,2.5)node[below left]{$D$}--(5,2.5)node[above left]{$C$}--(5,0)node[above right]{$B$}--cycle;
\draw[dashed] (0,1)node[below]{$M$}--(5,1)node[above]{$P$} (4,1)node[above left]{$N$}--(4,0)node[right]{$E$};
\draw[smooth,samples=300,domain=0:4] plot(\x,{sqrt(5-(\x-2)^2)});
\draw[smooth,samples=300,domain=4:5] plot(\x,{2*(\x)^2-18*(\x)+41});
\end{tikzpicture}
}
\shortans{$47{,}5$}
\loigiai{
\begin{center}
\begin{tikzpicture}[scale=1.2, >=stealth, font=\footnotesize, line join=round, line cap=round]
\coordinate (A) at (0,0);
\coordinate (B) at (5,0);
\coordinate (C) at (5,2.5);
\coordinate (D) at (0,2.5);
\coordinate (M) at (0,1);
\coordinate (P) at (5,1);
\coordinate (N) at (4,1);
\coordinate (E) at (4,0);
\fill[pattern=north east lines,opacity=0.6]
plot[domain=0:4] ({\x}, {sqrt(5-(\x-2)^2)})
-- plot[domain=4:5] ({\x}, {2*(\x)^2 - 18*(\x) + 41})
-- (5,0) -- (0,0) -- cycle;
\draw (A) node[below left]{$O$}
-- (D) node[above left]{$D$}
-- (C) node[above right]{$C$}
-- (B) node[below right]{$B$}
-- cycle;
\draw[dashed] (M) node[left]{$M$} -- (P) node[right]{$P$};
\draw[dashed] (N) node[above right]{$N$} -- (E) node[below]{$E$};
\draw[smooth,samples=200,domain=0:4] plot(\x,{sqrt(5-(\x-2)^2)});
\draw[smooth,samples=200,domain=4:5] plot(\x,{2*(\x)^2 - 18*(\x) + 41});
\draw[->] (A) -- ($(A)+(6,0)$) node[right]{$x$};
\draw[->] (A) -- ($(A)+(0,3)$) node[above]{$y$};
\node at (2,1.9) {$y = f(x)$};
\node at (4.6,0.4) {$y = g(x)$};

\end{tikzpicture}
\end{center}
Ta có $I M=\sqrt{O M^2+O I^2}=\sqrt{2^2+1^2}=\sqrt{5}\Rightarrow R=\sqrt{5}$.\\
Phương trình đường tròn $(C)$ tâm $I(2;0)$ bán kính $R=\sqrt{5}$ là
\[(x-2)^2+y^2=5\Leftrightarrow y^2=5-(x-2)^2.\]
Gọi $y=f(x)$ là hàm số của cung $MN$, khi đó \[f^2(x)=5-(x-2)^2 \Rightarrow f(x)=\sqrt{5-(x-2)^2}.\]
Thể tích khi quay hình phẳng giới hạn bởi $y=f(x)$, $y=0$, $x=0$, $x=4$ bằng
\[V_1=\pi \displaystyle\int\limits_0^4 f^2(x) \mathrm{d}x=\pi \displaystyle\int\limits_0^4\left(5-(x-2)^2\right) \mathrm{d}x=\dfrac{44 \pi}{3}\left(\mathrm{dm}^3\right).\]
Gọi $y=g(x)$ là hàm số cung parabol đi qua $N(4;1)$ và $P(5 ; 1)$. Khi đó ta có
\[g(x)-1=a(x-4)(x-5) \Leftrightarrow  g(x)=a(x-4)(x-5)+1\]
Tiếp tuyến của cung tròn và cung parabol tại điểm $N(4 ; 1)$ là trùng nhau nên
\[f'(4)=g'(4) \Leftrightarrow-a=-2\Leftrightarrow a=2.\]
Do đó $g(x)=2(x-4)(x-5)+1=2 x^2-18 x+41.$\\
Thể tích khi quay hình phẳng giới hạn bởi $y=g(x)$, $y=0$,
$x=4$, $x=5$ bằng
\[V_2=\pi \displaystyle\int\limits_4^5 g^2(x) \mathrm{d}x=\pi \displaystyle\int\limits_4^5\left(2 x^2-18 x+41\right)^2 \mathrm{d}x=\dfrac{7 \pi}{15}\left(\mathrm{dm}^3\right).\]
Thể tích của bình hoa bằng \[V=V_1+V_2=\dfrac{44 \pi}{3}+\dfrac{7 \pi}{15} =\dfrac{227 \pi}{15} \approx 47{,}5\left(\mathrm{d m}^3\right) =47{,}5(\mathrm{l}) .
\]
}
\end{ex}

\begin{ex}%[2D4C3-4]
Một chiếc dàn Ghita có chiều cao $5$ cm, khi cắt một mặt cắt ngang của cây đàn ta thu được một mặt phẳng như hình vẽ.
\begin{center}
\begin{tikzpicture}[line join=round, line cap=round,>=stealth,thick, xscale=.4, yscale=.2]
\tikzset{every node/.style={scale=0.9}}
\draw (0,0) node [below left] {$O$};
\begin{scope}[rotate=-10, yscale=1.4]
\clip (0,-15) rectangle (11,15);
\draw[fill =gray, line width = 1.2pt,draw=none] (0,0) plot[domain=0:10.08](\x,{1/24*((\x)^4)-7/9*((\x)^3)+14/3*((\x)^2)-32/3*(\x)+0})--(10.08,0)--cycle;
\draw[samples=200,domain=0:10.08,smooth,variable=\x] plot (\x,{1/24*((\x)^4)-7/9*((\x)^3)+14/3*((\x)^2)-32/3*(\x)+0});
\end{scope}
\begin{scope}[rotate=-10]
\draw[->] (-1.1,0)--(12,0) ;
\draw[->] (0,-8.1)--(0,15.1);
\clip (0,-15) rectangle (11,15);
\draw[fill =white, line width = 1.2pt,draw=none] (0,0) plot[domain=0:10.08](\x,{1/24*((\x)^4)-7/9*((\x)^3)+14/3*((\x)^2)-32/3*(\x)+0})--(10.08,0)--cycle;
\draw[samples=200,domain=0:10.08,smooth,variable=\x] plot (\x,{-1/24*((\x)^4)+7/9*((\x)^3)+-14/3*((\x)^2)+32/3*(\x)+0});
\draw[samples=200,domain=0:10.08,smooth,variable=\x] plot (\x,{1/24*((\x)^4)-7/9*((\x)^3)+14/3*((\x)^2)-32/3*(\x)+0});
\draw[samples=200,domain=0:10.08,smooth,variable=\x] plot (\x,{1/24*((\x)^4)-7/9*((\x)^3)+14/3*((\x)^2)-32/3*(\x)+0});

\end{scope}

\end{tikzpicture}
\end{center}
Đối với mỗi vị trí, người ta đo được chiều rộng $h$ của cái đàn và ghi lại qua bảng sau ($x$, $h$ có đơn vị cm)
\begin{center}
\begin{tabular}{|c|c|c|c|c|c|c|c|c|c|c|}
\hline$x$ & $0$ & $1$ & $2$ & $3$ & $4$ & $5$ & $6$ & $7$ & $8$ & $9$ \\
\hline$h$ & $0$ & $13{,}4$ & $16{,}4$ & $15{,}3$ & $14$ & $15{,}6$ & $20$ & $25{,}4$ & $28{,}4$ & $23{,}2$ \\
\hline
\end{tabular}
\end{center}
Bạn Nhâm nhận thấy rằng số liệu được để cập trên bảng gần giống với một hàm bậc bốn. Bằng cách mô phỏng hàm bậc bốn $y=f(x)=ax^4+bx^3+cx^2+dx+e$ trên hệ trục $Oxy$, đồ thị hàm số đi qua các điểm $O, A(6; 10)$ và có $3$ điểm cực trị có hoành độ là $2 ; 4 ; 8$. Dựa vào hàm số $f(x)$ tìm được, tính thể tích của cái đàn ghita (đơn vị $\text{cm}^3$, làm tròn đến hàng đơn vị). \shortans{$889$}
%plot(\x,{1.0/3.0*((-(\x)^(4.0))/24.0+14.0*(\x)^(3.0)/18.0-14.0*(\x)^(2.0)/3.0+32.0*(\x)/3.0)});
\loigiai{
Đồ thị hàm số $y=f(x)$ có $3$ điểm cực trị có hoành độ là $2; 4; 8$ nên
\[f'(x)=k(x-2)(x-4)(x-8),\,\, \text{với}\, k\neq 0,\]
hay
\[f'(x)=k\left(x^3-14x^2+56x-64\right).\]
Suy ra
\[f(x)=k\left(\dfrac{1}{4}x^4-\dfrac{14}{3}x^3+28x^2-64x\right)+e.\]
Do $f(0)=0$ nên $e=0$.\\
Vì vậy $f(x)=k\left(\dfrac{1}{4}x^4-\dfrac{14}{3}x^3+28x^2-64x\right)$.\\
Lại có $f(6)=10\Rightarrow -60k=10\Leftrightarrow k=-\dfrac{1}{6}$.\\
Suy ra $f(x)=-\dfrac{1}{6}\left(\dfrac{1}{4}x^4-\dfrac{14}{3}x^3+28x^2-64x\right)$.\\
Ta có
\begin{eqnarray*}
f(x)=0&\Leftrightarrow &-\dfrac{1}{6}\left(\dfrac{1}{4}x^4-\dfrac{14}{3}x^3+28x^2-64x\right)\\
&\Leftrightarrow& \hoac{&x=0	\\&\dfrac{1}{4}x^3-\dfrac{14}{3}x^2+28x-64=0. }
\end{eqnarray*}
Phương trình $\dfrac{1}{4}x^3-\dfrac{14}{3}x^2+28x-64=0$ có nghiệm duy nhất $x\approx 10{,}07$.\\
Diện tích mặt cắt ngang của cây đàn là
\[S=2\displaystyle\int\limits_{0}^{10{,}07}f(x)\mathrm{\, d}x =2\displaystyle\int\limits_{0}^{10{,}07}-\dfrac{1}{6}\left(\dfrac{1}{4}x^4-\dfrac{14}{3}x^3+28x^2-64x\right)\mathrm{\, d}x\approx 177{,}85\,\, \text{cm}^2.\]
Vậy thể tích của cây đàn là
\[V=\displaystyle\int\limits_{0}^{5}S\mathrm{\, d}x\approx 5\cdot 177{,}85\approx 889\,\, \text{cm}^3.\]
}
\end{ex}

\begin{ex}%[50 Đề minh họa tốt nghiệp 2025 - Đề 13]%[Lê Hữu Kiệt - Lê Quân]%[2D4C2-6]
Một chiếc xe đua Bugatti đang chuyển động trên đường đua. Đồ thị trên hình vẽ bên dưới biểu thị vận tốc $v$ (m/s) của chiếc xe đó trong $5$ giây đầu tiên.
\begin{center}
\begin{tikzpicture}[font=\footnotesize, line join=round, line cap=round, >=stealth, scale=1, y=0.166666666cm]
\draw[smooth] plot[domain=0:2] (\x,{(\x)^3/2+(\x)});
\draw (2,6)--(3,30)--(5,30);
\draw[dashed] (2,0)|- (0,6) (3,0)|-(0,30) (5,0)--(5,30);
\draw[->] (-0.5,0)--(6,0)node[below]{$t$ (s)};
\draw (0,0)node[below left]{$O$};
\draw[->] (0,-1)--(0,35)node[left]{$v$ (m/s)};
\foreach \x/\g/\n in {(2,0)/below/2, (3,0)/below/3, (5,0)/below/5, (0,6)/left/6, (0,30)/left/30, (2,6)/above/\, , (3,30)/above/\, , (5,30)/above/\,}{
\fill \x circle (1pt)node[\g]{$\n$};
}
\end{tikzpicture}
\end{center}
Đồ thị trong $2$ giây đầu tiên là một nhánh của hàm bậc ba nhận $O$ làm tâm đối xứng, trong $1$ giây tiếp theo xe tăng tốc với gia tốc $a$ (m/s$^2$) và đạt vận tốc $30$ m/s tại giây thứ $3$, sau đó duy trì vận tốc này đến giây thứ $5$. Biết quãng đường xe đi được trong $5$ giây đầu bằng $82$ m. Vận tốc của xe tại giây đầu tiên bằng bao nhiêu? (tính theo đơn vị km/h).
\par\shortans{$5{,}4$}
\loigiai{
Gọi $v_{1}(t)=at^3+bt^2+ct+d$ ($a\ne 0$) là hàm số bậc ba biểu diễn vận tốc trong $2$ giây đầu.\\
Ta có $v_{1}'(t)=3at^2+2bt+c$, $v_{1}''(t)=6at+2b$.\\
Do $O$ thuộc đồ thị hàm số $v_{1}(t)$ nên ta có $d=0$.\\
Điểm $O$ là điểm uốn nên $v''(0)=0 \Leftrightarrow b=0$.\\
Do $(2;6)$ thuộc đồ thị hàm số $v_{1}(t)$ nên ta có $8a+2c=6$. \quad$(1)$.\\
Trong thời gian từ giây thứ $2$ đến thứ $3$ vật tăng tốc với gia tốc $a$ m/s$^2$, suy ra $v_{2}(t)=\displaystyle\int a\mathrm{\,d}t=at+C$, với $C$ là hằng số.\\
Để tránh nhầm lẫn gia tốc $a$ m/s$^2$ và hệ số $a$ của $v_{1}(t)$.\\
Ta gọi $v_{2}(t)=et+g$ ($e\ne 0$) là hàm số biểu diễn vận tốc từ giây thứ $2$ đến thứ $3$.\\
Do $(2;6)$ và $(3;30)$ thuộc đồ thị hàm số $v_{2}(t)$ nên ta có hệ phương trình
\[ \heva{& 2e+g=6 \\& 3e+g=30} \Leftrightarrow \heva{& e=24 \\& g=-42.} \]
Suy ra $v_{2}(t)=24t-42$.\\
Từ giây thứ $3$ đến giây thứ $5$ vận tốc giữ ở mức $30$ m/s, suy ra hàm số biểu diễn vận tốc của xe trong thời gian này là $v_{3}(t)=30$.\\
Gọi $v(t)$ là hàm số biểu diễn vận tốc của xe trong $5$ giây đầu tiên, khi đó
\[ v(t)=\heva{
& at^3+ct & \text{\,khi\,} & 0 \leq t < 2 \\
& 24t-42 & \text{\,khi\,} & 2 \leq t < 3 \\
& 30 & \text{\,khi\,} & 3 \leq t \leq 5.
} \]
Quãng đường xe đi được trong $5$ giây đầu bằng $82$ m, ta có
\begin{eqnarray*}
&& \displaystyle\int\limits_0^5 v(t) \mathrm{\,d}x= 82 \\
&\Leftrightarrow& \displaystyle\int\limits_0^2 (at^3+ct) \mathrm{\,d}x + \displaystyle\int\limits_2^3 (24t-42) \mathrm{\,d}x + \displaystyle\int\limits_3^5 30 \mathrm{\,d}x = 82 \\
&\Leftrightarrow& \left(\dfrac{at^4}{4}+\dfrac{ct^2}{2}\right)\Bigg|_0^2 + 18 + 60 = 82 \\
&\Leftrightarrow& 4a+2c=4. \quad(2)
\end{eqnarray*}
Từ $(1)$ và $(2)$ ta có hệ phương trình $\heva{& 8a+2c=6 \\& 4a+2c=4} \Leftrightarrow \heva{&a=\dfrac{1}{2} \\& c=1.}$\\
Suy ra $v_{1}(t)=\dfrac{1}{2}t^3+t$.\\
Vậy trong giây đầu tiên, vận tốc của xe là $v_{1}(1)=1{,}5$ m/s $=5{,}4$ km/h.
}
\end{ex}

\begin{ex}%[2D1C5-8]
\immini{Cấu trúc tổ ong là một cấu trúc đặc biệt, mỗi lỗ ong là một lăng kính hình lục giác, một đầu hở còn một đầu tạo thành một góc tam diện. Ong đã xây các lỗ này với một cách làm tối ưu về diện tích bề mặt (đã sử dụng lượng sáp ong ít nhất để xây tổ). Người ta đã quan sát, nghiên cứu thì thấy rằng góc $\theta(\mathrm{rad})$ ở đỉnh nhất quán một cách đáng kinh ngạc, dựa trên cấu trúc hình học của lỗ ong người ta đã chứng minh được diện tích bề mặt $S$ của lỗ ong là $S=6 s \cdot h-\dfrac{3}{2} s^2 \cdot \cot \theta+\dfrac{3 \sqrt{3}}{2} s^2 \cdot \dfrac{1}{\sin \theta}$ ( $s$ là chiều dài các cạnh của lỗ ong, $h$ là chiều cao, $s$ và $h$ đều là hằng số). Vậy để tối thiểu hoá diện tích bề mặt con ong đã xây một góc $\theta$ bằng bao nhiêu? (làm tròn đến hàng phần trăm).}{
\begin{tikzpicture}[scale=1, line cap=round, line join=round, >=stealth,font=\footnotesize]
\def\a{2} % Độ dài cạnh lục giác
\def\theta{-10} % Góc nghiêng lục giác
\def\scaleY{0.4} % Tỷ lệ nén theo trục y
\def\h{5} % chiều cao
% Định nghĩa các đỉnh của hình lục giác đã nghiêng và nén dọc theo trục y
\foreach \i in {0,60,120,180,240,300} {
\path ({\a*cos(\i + \theta)}, {\scaleY*\a*sin(\i + \theta)}) coordinate (P\i);
}
% Vẽ lục giác
\draw  (P180)--(P240) -- (P300)--(P0) ;
% Vẽ các đường chéo để nhấn mạnh hình dạng
\draw[dashed] (P0) -- (P60) -- (P120) -- (P180) (P0) -- (P180) (P60) -- (P240)(P120) -- (P300);
\coordinate (P0') at ($(P0) + (0,\h-0.2)$) ;
\coordinate (P60') at ($(P60) + (0,\h+0.15)$);
\coordinate (P120') at ($(P120) + (0,\h-1)$);
\coordinate (P180') at ($(P180) + (0,\h)$);
\coordinate (P240') at ($(P240) + (0,\h-0.2)$);
\coordinate (P300') at ($(P300) + (0,\h+0.5)$);

\coordinate (O) at ($(P0)!0.5!(P180)$);
\coordinate (O') at ($(O) + (0,\h+1)$);
\draw[dashed] (P60) -- (P60')  (P120) -- (P120') (O) -- (O') ;
\draw  (P0)--(P0')  (P180)--(P180') (P240)--(P240') (P300)--(P300') (P0')--(P300')--(P240')--(P180')--(O') (O')--(P180') (O')--(P300') (P0')--(P60')--(O') ;
\draw[dashed]  (P60')--(P120')--(P180') ;

\draw ($(O)!0.4!(O')$) node[right]{$h$};
\draw ($(P300)!0.5!(P240)$) node[below]{$s$};
\end{tikzpicture}
}
\shortans[]{$0{,}96$}
\loigiai{
Ta có
\begin{eqnarray*}
S^{\prime}(\theta)&=&\dfrac{3}{2} s^2 \cdot \dfrac{1}{\sin ^2 \theta}-3 s^2 \dfrac{\sqrt{3}}{2} \cdot \dfrac{\cos \theta}{\sin ^2 \theta} \\
& =&\dfrac{3}{2} s^2 \cdot \dfrac{1}{\sin ^2 \theta}(1-\sqrt{3} \cos \theta)=0 \\
&\Leftrightarrow& \cos \theta=\dfrac{1}{\sqrt{3}} \Rightarrow \theta \approx 0{,}96 .
\end{eqnarray*}
Bảng biến thiên
\begin{center}
\begin{tikzpicture}[scale=1, font=\footnotesize, line join=round, line cap=round, >=stealth]
\tkzTabInit[lgt=1]	{$x$/1,$y'$/1,$y$/2}
{$0$,$0{,}96$,$2\pi$}
\tkzTabLine{,-,0,+,}
\tkzTabVar{+/ $+\infty$,-/ $6 s\left(h+\dfrac{1}{2 \sqrt{2}} s\right)$, +/ $+\infty$}
\end{tikzpicture}
\end{center}
Vậy $S_{\min }=6 s\left(h+\dfrac{1}{2 \sqrt{2}} s\right)$ khi
$\theta=0{,}96$
}
\end{ex}

\begin{ex}%[2D1C5-8]
Trong quang học, chúng ta đã biết đến định luật quang học về độ chiếu sáng. Nó được phát biểu như sau: Độ chiếu sáng từ một nguồn sáng $A$ đến một điểm $B$ cho bởi công thức $T=\dfrac{i\cos\alpha}{AB^2}$, trong đó $i$ là độ phát sáng của nguồn $A$; $\alpha$ là góc phản xạ của ánh sáng lên người quan sát (coi rằng người quan sát nhìn thẳng xuống mặt bàn, xem hình vẽ).
\begin{center}
\includegraphics[scale=0.18]{images/de15-2}
\end{center}
Một đồng xu được đặt cách ngọn nến một khoảng $BC=20$ cm. Hỏi ngọn lửa của cây nến nên đặt ở độ cao $h$ bằng bao nhiêu cm để chiếu sáng rõ nhất đồng tiền xu nằm trên bàn (kết quả làm tròn đến hàng phần chục).
\shortans[]{$14{,}1$}
\loigiai{
Trong trường hợp này, chúng ta có thể thay đổi chiều cao của cây nến bằng cách tăng chiều cao của đế nên đặt $AC=h=$ khoảng cách từ chỗ nến cháy xuống mặt bàn.\\
Suy ra $AB=\sqrt{h^2+20^2}=\sqrt{h^2+400}$.\\
Lại có $\cos \alpha=\dfrac{h}{AB}=\dfrac{h}{\sqrt{h^2+400}}$. Khi đó độ chiếu sáng $T=\dfrac{i\dfrac{h}{\sqrt{h^2+400}}}{h^2+400}=i\cdot \dfrac{h}{\sqrt{(h^2+400)^3}}=f(h)$.
\begin{itemize}
\item \textbf{Cách 1}\\
Khảo sát hàm số $f(h)$ trong khoảng $(0;+\infty)$. Ta có
\begin{eqnarray*}
f'(h)&=& i\cdot \dfrac{\sqrt{(h^2+400)^3}-h\cdot \dfrac{3\cdot 2h\cdot (h^2+400)^2}{2\sqrt{(h^2+400)^2}}}{(h^2+400)^3}\\
&=& i \cdot \dfrac{\sqrt{\left(h^2+400\right)^3}-\dfrac{3 h^2 \cdot\left(h^2+400\right)^2}{\sqrt{\left(h^2+400\right)^3}}}{\left(h^2+400\right)^3}\\
&=&  i \cdot \dfrac{\left(h^2+400\right)^3-3 h^2 \cdot\left(h^2+400\right)^2}{\left(h^2+400\right)^3 \sqrt{\left(h^2+400\right)^3}}\\
&=& i\cdot \dfrac{400-2 h^2}{\sqrt{\left(h^2+400\right)^5}}.
\end{eqnarray*}
Khi đó $f'(h)=0\Leftrightarrow h=10\sqrt{2}$. Ta có bảng biến thiên
\begin{center}
\begin{tikzpicture}
\tkzTabInit[nocadre=false,lgt=1.2,espcl=2.5,deltacl=0.7]
{$h$ /1,$f'(h)$ /0.6,$f(h)$ /2}
{$0$,$10\sqrt{2}$,$+\infty$}
\tkzTabLine{,+,$0$,-,}
\tkzTabVar{-/, +/,-/}
\end{tikzpicture}
\end{center}
Từ bảng biến thiên suy ra $\max f(h)$ khi $h=10\sqrt{2}\approx 14{,}1$.
\item \textbf{Cách 2}\\
Dùng Côsi. Ta có
$f(h)=i\cdot \dfrac{h}{\sqrt{\left(h^2+400\right)^3}}=i\sqrt{\dfrac{h^2}{\left(h^2+200+200\right)^3}}$.\\
Áp dụng bất đẳng thức Côsi cho $3$ số dương $h^2$, $200$, $200$ ta có
\begin{eqnarray*}
& &	h^2+200+200\geq 3\sqrt[3]{h^2\cdot 200\cdot 200}\\
&\Rightarrow& \left(h^2+200+200\right)^3\geq \left(3\sqrt[3]{h^2\cdot 200\cdot 200}\right)^3\\
&\Leftrightarrow & \left(h^2+200+200\right)^3\geq 27h^2\cdot 40\, 000.
\end{eqnarray*}
Suy ra $f(h)\leq i\sqrt{\dfrac{h^2}{27h^2\cdot 40\, 000}}=i\dfrac{1}{200\cdot 3\sqrt{3}}=\dfrac{i}{600\sqrt{3}}$.\\
Dấu \lq \lq =\rq \rq xảy ra khi và chỉ khi $h^2=200\Leftrightarrow h=\sqrt{200}\approx 14{,}1$.
\end{itemize}
}
\end{ex}

\begin{ex}%[2D1C5-6]
Hình vẽ sau mô tả một đường cong Agnesi và được xây dựng trong hệ tọa độ $Oxy$ như sau: vẽ một đường tròn có tâm $I(0;1)$ và bán kính bằng $1$, từ điểm $O$ kẻ một đường thẳng cắt đường tròn tại điểm thứ $2$ là điểm $B$ và cắt đường thẳng $y=2$ tại điểm $A$. Gọi $P$ là giao điểm của đường thẳng qua $A$ và vuông góc với $Ox$ và đường thẳng qua $B$ vuông góc với $Oy$. Tập hợp các điểm $P$ tạo thành một đường cong $y=f(x)$ gọi là đường cong Agnesi. Tiếp tuyến của đồ thị hàm số $y=f(x)$ có hệ số góc lớn nhất bằng bao nhiêu? (làm tròn kết quả đến hàng phần mười).
\begin{center}
\begin{tikzpicture}[scale=1,>=stealth, font=\footnotesize, line join=round, line cap=round]
\def\xmin{-4} \def\xmax{4}
\def\ymin{-0.7} \def\ymax{3}
\draw[->] (\xmin,0)--(\xmax,0) node [below]{$x$};
\draw[->] (0,\ymin)--(0,\ymax) node [left]{$y$};
\node at (0,0) [below left]{$O$};
\clip (\xmin+0.1,\ymin+0.1) rectangle (\xmax-0.5,\ymax-0.1);
\draw[smooth,samples=300] plot(\x,{8/(4+(\x)^2)});
\draw (0,1)node[left]{$(0;1)$} circle(1cm) (\xmin,2)node[above,xshift=0.9cm]{$y=2$}--(\xmax,2) (0,0)--(2,2) (0.4,0.1)node[above right]{$t$};
\fill (0,1)circle(2pt) (0,2)node[above left]{$Q$};
\draw[dashed] (2,2)node[above]{$A$}--(2,1)node[above right]{$P(x;y)$}--(1,1)node[left]{$B$};
\fill (2,1)circle(2pt);
\draw[->] (0.5,0) arc(0:45:0.5 cm and 0.5 cm);
\end{tikzpicture}
\end{center}
\shortans{$0{,}6$}
\loigiai{
\begin{center}
\begin{tikzpicture}[scale=1,>=stealth, font=\footnotesize, line join=round, line cap=round]
\def\xmin{-4} \def\xmax{4}
\def\ymin{-0.7} \def\ymax{3}
\draw[->] (\xmin,0)--(\xmax,0) node [below]{$x$};
\draw[->] (0,\ymin)--(0,\ymax) node [left]{$y$};
\node at (0,0) [below left]{$O$};
\clip (\xmin+0.1,\ymin+0.1) rectangle (\xmax-0.5,\ymax-0.1);
\draw[smooth,samples=300] plot(\x,{8/(4+(\x)^2)});
\draw (0,1)node[left]{$(0;1)$} circle(1cm) (\xmin,2)node[above,xshift=0.9cm]{$y=2$}--(\xmax,2) (0,0)--(2,2) (0.4,0.1)node[above right]{$t$};
\fill (0,1)circle(2pt) (0,2)node[above left]{$Q$};
\draw[dashed] (2,2)node[above]{$A$}--(2,1)node[above right]{$P(x;y)$}--(1,1)node[left]{$B$};
\fill (2,1)circle(2pt);
\draw[->] (0.5,0) arc(0:45:0.5 cm and 0.5 cm);
\draw[dashed] (2,2) -- (2,0) node[below]{$K$};
\draw[dashed] (1,1) -- (1,0) node[below]{$H$};
\end{tikzpicture}
\end{center}
Gọi $K$, $H$ lần lượt là hình chiếu của $A$ và $B$ lên $Ox$.\\ Tam giác $AOQ$ vuông tại $Q$ có $QB$ là đường cao, $O A=\sqrt{O K^2+A K^2}=\sqrt{x^2+4}$\\
Suy ra $ O B \cdot O A=O Q^2  \Rightarrow O B=\dfrac{O Q^2}{O A}=\dfrac{4}{\sqrt{x^2+4}}$.\\
Ta có  $\dfrac{B H}{A K}=\dfrac{O B}{O A}  \Rightarrow B H=\dfrac{A K \cdot O B}{O A}$.\\
Xét hàm số $y=\dfrac{2 \cdot \dfrac{4}{\sqrt{x^2+4}}}{\sqrt{x^2+4}}=\dfrac{8}{x^2+4}\Rightarrow y'=\dfrac{-8 \cdot 2 x}{\left(x^2+4\right)^2}=-\dfrac{16 x}{\left(x^2+4\right)^2} . $
\begin{eqnarray*}
y''&=&-16 \cdot \dfrac{\left(x^2+4\right)^2-2\left(x^2+4\right) \cdot 2 x \cdot x}{\left(x^2+4\right)^4}\\
& =&-16 \cdot \dfrac{\left(x^2+4\right)\left(x^2+4-4 x^2\right)}{\left(x^2+4\right)^4} \\
& =&-16 \cdot \dfrac{\left(x^2+4\right)\left(-3 x^2+4\right)}{\left(x^2+4\right)^4} \\
& =&-16 \cdot \dfrac{\left(-3 x^2+4\right)}{\left(x^2+4\right)^3} \cdot \\
\end{eqnarray*}
Ta có \begin{eqnarray*}
y''=0 &\Leftrightarrow&-16 \cdot \dfrac{-3 x^2+4}{\left(x^2+4\right)^3}=0 \\
& \Leftrightarrow&-3 x^2+4=0 \Leftrightarrow x^2=\dfrac{4}{3} \\
& \Leftrightarrow& x= \pm \dfrac{2 \sqrt{3}}{3} .
\end{eqnarray*}
Bảng biến thiên
\begin{center}
\begin{tikzpicture}
% Điều chỉnh tham số để bảng vừa vặn
\tkzTabInit[lgt=1.2,espcl=2.5,deltacl=0.5]%
{$x$/0.8, $y'$/0.8, $y$/3}  % Giảm độ rộng các cột
{$-\infty$, $\frac{-2\sqrt{3}}{3}$, $\frac{2\sqrt{3}}{3}$, $+\infty$}

\tkzTabLine{, +, z, -, z, +}  % z thay vì 0 để căn chỉnh đẹp hơn

% Định vị lại các nút với vị trí chính xác
\path
(N13) node[above=25pt] (A) {$0$}
(N22) node[below=2pt] (B) {\scriptsize $\dfrac{3\sqrt{3}}{8}$}
(N33) node[above=1pt] (C) {\scriptsize $\dfrac{-3\sqrt{3}}{8}$}
(N43) node[above=25pt] (D) {$0$};

% Vẽ mũi tên thẳng với độ dày vừa phải
\draw[-stealth,line width=0.6pt] (A) -- (B);
\draw[-stealth,line width=0.6pt] (B) -- (C);
\draw[-stealth,line width=0.6pt] (C) -- (D);
\end{tikzpicture}
\end{center}
Dựa vào bảng biến thiên suy ra $y_{\max }'=\dfrac{3 \sqrt{3}}{8} \approx 0{,}6$.
}
\end{ex}

\begin{ex}%[50 Đề minh họa tốt nghiệp 2025 - Đề 13]%[Lê Hữu Kiệt - Lê Quân]%[2D1C3-6]
Hệ thống mạch máu chứa các mạch máu gồm động mạch chính, động mạch con, mao mạch và tĩnh mạch để giúp đưa máu từ tim đến các cơ quan và ngược lại. Hệ thống hoạt động để tối ưu hoá (tối thiểu) năng lượng mà tim sử dụng trong quá trình bơm máu. Đặc biệt năng lượng này giảm khi sức cản của máu giảm. Hình vẽ dưới đây minh hoạ một mạch máu chính có bán kính $r_1$ phân nhánh với một góc $\alpha^\circ$ tạo thành một mạch máu nhỏ hơn với bán kính $r_2$.
\begin{center}
\begin{tikzpicture}[font=\footnotesize, line join=round, line cap=round, >=stealth, scale=1]
\draw
(1,1) ellipse ({0.3} and {0.5})
(8,1) ellipse ({0.3} and {0.5})
(8,4) ellipse ({0.15} and {0.4})
;
\draw (1,0.5)--(8,0.5) (1,1.5)--(3,1.5) (4.5,1.5)--(8,1.5);
\draw (7.95,4.38)--(3,1.5) (8,3.6)--(4.5,1.5);
\draw[dashed] (1,1)--(8,1) (8,4)--(3,1)node[below]{$B$};
\fill
(1,1) circle (1pt)node[left=-1mm]{$A$}
(3,1) circle (1pt)node[below]{$B$}
(8,4) circle (1pt)node[right=1mm]{$C$}
;
\draw[|<->|] (8.75,1)--(8.75,4)node[pos=0.5, right]{$b$};
\draw[|<->|] (1,0.25)--(8,0.25)node[pos=0.5, below]{$a$};
\draw[<->] (2,1)--(2,1.5)node[pos=0.5, right]{$r_1$};
\draw[<->] (6.75,3.25)--(6.5,3.53)node[pos=0.5, right]{$r_2$};
\path (8,1) coordinate (D) (3,1) coordinate (B) (8,4) coordinate (C)
pic[angle eccentricity=1.2,"$\alpha$"]{angle=D--B--C};
\draw (2,2)node[align=left]{Phân nhánh\\mạch máu};
\end{tikzpicture}
\end{center}
Sử dụng mô tả Định luật Poiseuille, người ta đã chứng minh được sức cản của máu theo con đường $ABC$ là
\[R(\alpha)=C\left(\dfrac{a-b\cot\alpha}{r_1^4}+\dfrac{b}{r_2^4\sin\alpha}\right)\]
với $C, a, b$ là các hằng số. Khi bán kính mạch máu nhỏ bằng $\dfrac{2}{3}$ bán kính mạch máu chính. Xác định $\alpha$ để sức cản này là nhỏ nhất. (Làm tròn kết quả đến hàng đơn vị).
\par\shortans{$79$}
\loigiai{
Các hằng số $C$, $a$, $b$ là các hằng số dương.\\
Ta có $r_2=\dfrac{2}{3}r_1 \Leftrightarrow r_2^4=\dfrac{16}{81}r_1^4$.\\
Khi đó
\[ R(\alpha)=C\left(\dfrac{a-b\cot\alpha}{r_1^4}+\dfrac{81b}{16r_1^4\sin\alpha}\right) =\dfrac{C}{r_1^4}\left(a-b\cot\alpha + \dfrac{81b}{16\sin\alpha}\right).\]
Điều kiện xác định $0^\circ<\alpha<180^\circ$.\\
Ta có $R'(\alpha)=\dfrac{Cb}{r_1^4}\cdot\dfrac{16-81\cos\alpha}{16\sin^2\alpha}$.\\
Ta có
\begin{eqnarray*}
&& R'(\alpha)=0 \\
&\Leftrightarrow& 16-81\cos\alpha=0 \\
&\Leftrightarrow& \cos\alpha=\dfrac{16}{81} \\
&\Leftrightarrow& \hoac{&\alpha\approx 79^\circ + k360^\circ \\& \alpha\approx -79^\circ + k360^\circ.}
\end{eqnarray*}
Do $0^\circ < \alpha < 180^\circ$ suy ra $\alpha\approx 79^\circ$.\\
Bảng biến thiên
\begin{center}
\begin{tikzpicture}[font=\footnotesize, line join=round, line cap=round, >=stealth, scale=1]
\tkzTabInit[lgt=1.2,espcl=2.5,deltacl=0.6]
{$\alpha$/1, $R'(\alpha)$/0.7, $R(\alpha)$/2}
{$0^\circ$, $79^\circ$ , $180^\circ$}
\tkzTabLine
{, - , $0$ , + ,}
\tkzTabVar
{+/, -/, +/}
\end{tikzpicture}
\end{center}
Từ bảng biến thiên, sức cản của mạch máu là nhỏ nhất đạt được khi $\alpha\approx 79^\circ$.
}
\end{ex}

\begin{ex}%[2D1C3-6]%[Tex đề Moon 2025]%[Nguyễn Hồng Thạch]
\immini[thm]
{
Đường đi của một khinh khí cầu được gắn trong hệ trục tọa độ là một phần của đường cong bậc hai trên bậc nhất có đồ thị cắt trục hoành tại điểm có tọa độ là $(1;0)$ và $(8;0)$ với đơn vị trên hệ trục tọa độ là $1$ km. Biết rằng điểm cực đại của đồ thị hàm số là điểm $(6;5)$. Hỏi khi khinh khí cầu đi qua điểm cực đại và cách mặt đất $3875$ m thì khinh khí cầu cách gốc tọa độ theo phương ngang bao nhiêu km?
}
{
\begin{tikzpicture}[scale=0.4,>=stealth, font=\footnotesize, line join=round, line cap=round]
\def\a{5} \def\b{-45} \def\c{40} \def\d{3}\def\e{-28} % Hệ số
\def\xmin{-1} \def\xmax{10}
\def\ymin{-1} \def\ymax{6}
\draw[->] (\xmin,0)--(\xmax,0) node [below]{$x$};
\draw[->] (0,\ymin)--(0,\ymax) node [left]{$y$};
\node at (0,0) [below left]{$O$};
\draw[smooth,samples=300,domain=1:8] plot(\x,{(\a*(\x)^2+\b*(\x)+\c)/(\d*(\x)+\e)});
%\draw (3,50/19)node[above,font=\fontsize{15pt}{2}\selectfont,yshift=-0.15cm]{\faFly};
\end{tikzpicture}
}
\shortans{$7{,}2$}
\loigiai{
Giả sử đường đi có phương trình là $y=\dfrac{a(x-1)(x-8)}{x+b}$ vì đồ thị hàm số cắt trục hoành tại hai điểm $A(1;0)$ và $B(8;0)$.\\
Ta có $y=\dfrac{ax^2-9ax+8a}{x+b}\ (a\neq 0)\Rightarrow y'=\dfrac{(2ax-9a)(x+b)-(ax^2-9ax+8a)}{(x+b)^2}=\dfrac{ax^2+2abx-9ab-8a}{(x+b)^2}$.\\
Vì hàm số có điểm cực đại $x=6$ nên $y'(6)=0\Rightarrow 28a+3ab=0\Leftrightarrow b=-\dfrac{28}{3}$.\\
Ta có đồ thị đi qua điểm $(6;5)$ nên $5=\dfrac{a(6-1)(6-8)}{6-\dfrac{28}{3}}\Leftrightarrow a=\dfrac{5}{3}$.\\
Suy ra hàm số là $y=\dfrac{\dfrac{5}{3}(x-1)(x-8)}{x-\dfrac{28}{3}}$ hay $y=\dfrac{5x^2-45x+40}{3x-28}$.\\
Vì khinh khí cầu đi qua điểm cực đại nên $x>6$ và cách mặt đất $3\,875$ m nên $y=3{,}875$.\\
Suy ra \begin{eqnarray*}
&&\dfrac{5x^2-45x+40}{3x-28}=3{,}875\\
&\Leftrightarrow& 5x^2-45x+40=11{,}625x-108{,}5\\
&\Leftrightarrow& 5x^2-56{,}625x+148{,}5=0\\
&\Leftrightarrow& \hoac{&x= 4{,}125\\&x=7{,}2.}
\end{eqnarray*}
Kết hợp điều kiện ta có $x=7{,}2$.\\
Vậy khoảng cách từ khinh khí cầu sau khi bay qua điểm cực đại và cách mặt đất $3875$ m đến gốc tọa độ tính theo phương ngang là $7{,}2$ km.
}
\end{ex}

\begin{ex}%[2D4C3-2]
\immini[thm]
{
Một nhà sản xuất dự kiến xây dựng sân khấu cho một concept âm nhạc trên một mảnh đất hình chữ nhật có kích thước $20\text{ m}\times10\text{ m}$. Nhà sản xuất mô phỏng sân khấu thông qua bản vẽ trên hệ trục $Oxy$ như sau: vẽ hai parabol có đỉnh $I_1$, $I_2$ có cùng hoành độ, trong đó parabol đỉnh $I_1$ tiếp xúc với cạnh ngắn của hình chữ nhật. Vị trí giao nhau của hai parabol là $A$ và $B$ cùng với hai đỉnh $I_1$, $I_2$ tạo thành hình thoi có độ dài hai đường chéo là $I_1I_2=16$ (m) và $AB=8$ (m) (tham khảo hình vẽ). Trên thực tế, khu vực màu đen là khu vực thiết kế dành cho khán giả, màu xám là khu vực sân khấu và màu trắng là khu vực hậu trường. Chi phí để xây dựng khu vực sân khấu, hậu trường, khán đài lần lượt là $2$ triệu đồng, $200$ nghìn đồng và $400$ nghìn đồng mỗi mét vuông. Tổng chi phí xây dựng bằng bao nhiêu triệu đồng? Làm tròn đến hàng đơn vị.
}
{
\begin{tikzpicture}[scale=0.5,>=stealth, font=\footnotesize, line join=round, line cap=round]
\fill[gray,opacity=0.6] plot[domain=-4:4](\x,{1/2*(\x)^2-8})--plot[domain=4:-4](\x,{-1/2*(\x)^2+8})--cycle;
\fill[black,opacity=0.8] plot[domain=-5:5](\x,{1/2*(\x)^2-8})--(5,-12)--(-5,-12)--cycle;
\draw (-5,8)--(5,8)--(5,-12)--(-5,-12)--cycle;
\draw[smooth,samples=300,domain=-5:5] plot(\x,{1/2*(\x)^2-8});
\draw[smooth,samples=300,domain=-4:4] plot(\x,{-1/2*(\x)^2+8});
\draw[<->] (-4,0)node[below right]{$A$}--(4,0)node[below left]{$B$};
\draw[<->] (0,8)node[above]{$I_1$}--(0,-8)node[above left]{$I_2$};
\end{tikzpicture}
}
\shortans[]{$210$}
\loigiai{
\begin{center}
\begin{tikzpicture}[scale=0.5,>=stealth, font=\footnotesize, line join=round, line cap=round]
\fill[gray,opacity=0.6] plot[domain=-4:4](\x,{1/2*(\x)^2-8})--plot[domain=4:-4](\x,{-1/2*(\x)^2+8})--cycle;
\fill[black,opacity=0.8] plot[domain=-5:5](\x,{1/2*(\x)^2-8})--(5,-12)--(-5,-12)--cycle;
\draw (-5,8)--(5,8)--(5,-12)--(-5,-12)--cycle;
\draw[smooth,samples=300,domain=-5:5] plot(\x,{1/2*(\x)^2-8});
\draw[smooth,samples=300,domain=-4:4] plot(\x,{-1/2*(\x)^2+8});
\draw[<->] (-4,0)node[below right]{$A$}--(4,0)node[below left]{$B$};
\draw[<->] (0,8)node[above right]{$I_1$}--(0,-8)node[above left]{$I_2$};
\draw[->] (-7.5,0)--(7.5,0) node [below]{$x$};
\draw[->] (0,-8)--(0,9) node [left]{$y$};
\node at (0,0) [below left]{$O$};
\end{tikzpicture}
\end{center}
Gắn hệ trục toạ độ như hình vẽ. Gọi $f(x)$ là hàm số của parabol phía trên khi đó $y=f(x)=a(x-4)(x+4)a(x^2-16)$.\\
Parabol phía trên đi qua điểm $(0;8)$ nên $8=a(0^2-16)\Leftrightarrow a=-\dfrac{1}{2}$. Vậy $f(x)=-\dfrac{1}{2}(x^2-16)$.\\
Gọi $g(x)$ là hàm số của parabol phía dưới khi đó $y=g(x)=b(x-4)(x+4)=b(x^2-16)$.\\
Parabol phía dưới đi qua điểm $(0;-8)$ nên $-8=b(0^2-16)\Leftrightarrow b=\dfrac{1}{2}$. Vậy $g(x)=\dfrac{1}{2}(x^2-16)$.\\
Diện tích sân khấu là
\begin{eqnarray*}
S_{sk}&=&\displaystyle \int_{-4}^{4}\left(f(x)-g(x)\right)\mathrm{\,d}x\\
&=& \displaystyle \int_{-4}^{4}\left(-\dfrac{1}{2}(x^2-16)-\dfrac{1}{2}(x^2-16)\right)\mathrm{\,d}x\\
&=& \dfrac{256}{3}\, (\text{m}^2).
\end{eqnarray*}
Diện tích khán đài là
\begin{eqnarray*}
S{kd}&=&\displaystyle\int_{-5}^{5}\left(g(x)+12\right)\mathrm{\,d}x\\
&=& \displaystyle\int_{-5}^{5}\left(\dfrac{1}{2}(x^2-16)+12\right)\mathrm{\,d}x\\
&=& \dfrac{245}{3}\,(\text{m}^2).
\end{eqnarray*}
Diện tích hậu trường là
\[S_{ht}=S-S_{sk}-S_{kd}=20\cdot 10-\dfrac{256}{3}-\dfrac{245}{3}=33\, (\text{m}^2).\]
Tổng chi phí xây dựng là $=S_{sk}\cdot 2+S_{ht}\cdot 0{,}2+S_{kd}\cdot 0{,}4\approx 210$ (triệu đồng).
}
\end{ex}


\Closesolutionfile{ans}
\newpage

\begin{center}
    % \color{mycolor1}
    \bfseries\faGg~\faGg~\faGg~BẢNG ĐÁP ÁN TRẮC NGHIỆM~\faGg~\faGg~\faGg
\end{center}
% \inputansbox{10}{ans/ansBTchoice}
\inputansbox{3}{ans/ansBTchoiceTF}
\inputansbox{3}{ans/ansBTshortans}
\newpage

% \setcounter{deso}{10}
% \def\sode{1}
\begin{name}
	{\tenchude}
	{\tendethi}
	{\tentruong}
	{\thoigian}
\end{name}
\caulc
\Opensolutionfile{ans}[ans/ans-HXN-\sode-T]
%Câu hỏi
\begin{ex}%Câu 1
\immini
{
        Cho hàm số có đồ thị là đường cong trong hình bên. Hàm số đã cho đồng biến trên khoảng nào dưới đây?
    \choice
    {\True $\left(0;1\right)$}
    {$\left(-\infty;0\right)$}
    {$\left(1;+\infty\right)$}
    {$\left(-1;0\right)$}
}
{
     \begin{tikzpicture}[font=\footnotesize, line join=round, line cap=round, >=stealth]
        \draw[->] (-2,0)--(2,0) node[below]{$x$};
        \draw[->] (0,-1.25)--(0,3) node[left]{$y$};
        \draw[dashed] (-1,0)node[below]{$-1$}|-(0,2)node[above left]{$2$}-|(1,0)node[below]{$1$}
        (0,0)node[below left]{$O$};
        \draw[domain=-1.65:1.65,samples=150] plot(\x,{-(\x)^4+2*(\x)^2+1});
    \end{tikzpicture}
}
\end{ex}

\begin{ex}%Câu 2
    Thống kê điểm kiểm tra giữa kỳ 1 môn Toán của 30 học sinh lớp 12C1 của một trường THPT được ghi lại ở bảng sau:\\
    \centerline{\begin{tabular}{|c|c|c|c|c|}
            \hline
            Điểm & $\left[2;4\right)$ & $\left[4;6\right)$ & $\left[6;8\right)$ & $\left[8;10\right)$\\
            \hline
            Số học sinh & $ 4$ & $ 8$ & $ 11$ & $ 7$\\
            \hline
    \end{tabular}}\\
    Trung vị của mẫu số liệu gốc thuộc khoảng nào trong các khoảng dưới đây?
    \choice
    {$\left[2;4\right)$}
    {$\left[4;6\right)$}
    {\True $\left[6;8\right)$}
    {$\left[8;10\right)$}
\end{ex}

\begin{ex}%Câu 3
    Trong không gian $Oxyz$ , một vectơ pháp tuyến của mặt phẳng $\dfrac{x}{-2}+\dfrac{y}{-1}+\dfrac{z}{3}=1$ là
    \choice
    {\True $\overrightarrow{n}=(3;6;-2)$}
    {$\overrightarrow{n}=(2;-1;3)$}
    {$\overrightarrow{n}=(-3;-6;-2)$}
    {$\overrightarrow{n}=(-2;-1;3)$}
\end{ex}

\begin{ex}%Câu 4
    Cho cấp số cộng $\left(u_n\right)$ với số hạng đầu $u_1=-6$ và công sai $d=4$. Tính tổng $S$ của $14$ số hạng đầu tiên của cấp số cộng đó.
    \choice
    {$S=46$}
    {$S=308$}
    {$S=644$}
    {\True $S=280$}
\end{ex}

\begin{ex}%Câu 5
    Cho tứ diện đều $ABCD$ có cạnh bằng $a$. Tích vô hướng $\overrightarrow{AB}\cdot\overrightarrow{AC}$ bằng
    \choice
    {$a^2$}
    {$-a^2$}
    {\True $\dfrac{1}{2}{a^2}$}
    {$\dfrac{\sqrt{3}}{2}{a^2}$}
\end{ex}

\begin{ex}%Câu 6
    Giá trị lớn nhất của hàm số $f(x)=x^3-8x^2+16x-9$ trên đoạn $\left[1;3\right]$ là 
    \choice
    {$\max\limits_{\left[1;3\right]}f(x)=0$}
    {\True $\max\limits_{\left[1;3\right]}f(x)=\dfrac{13}{27}$}
    {$\max\limits_{\left[1;3\right]}f(x)=-6$}
    {$\max\limits_{\left[1;3\right]}f(x)=5$}
\end{ex}

\begin{ex}%Câu 7
    Trong một phép thử với $ A$, $B$ là hai biến cố bất kì, biết rằng $ P(A)=0{,}5$; $ P\left(AB\right)=0{,}3$. Khi đó $ P(B|A)$ bằng
    \choice
    {\True $ 0{,}6$}
    {$ 0{,}15$}
    {$ 0{,}7$}
    {$ 0{,}35$}
\end{ex}

\begin{ex}%Câu 8
    Cho biết $\int\limits_1^3f(x)\mathrm{\,d}x=3$, giá trị của $\int\limits_1^3\dfrac{1}{3}f(x)\mathrm{\,d}x$ bằng
    \choice
    {$ 2$}
    {\True $ 1$}
    {$\dfrac{1}{3}$}
    {$ 3$}
\end{ex}

\begin{ex}%Câu 9
    Tập nghiệm của bất phương trình $2^x\le 4$ là
    \choice
    {\True $\left(-\infty;2\right]$}
    {$\left[0;2\right]$}
    {$\left(-\infty;2\right)$}
    {$\left(0;2\right)$}
\end{ex}

\begin{ex}%Câu 10
    Phát biểu nào sau đây là đúng?
    \choice
    {$\int\dfrac{1}{x}\mathrm{\,d}x=\left| x\right|+C$}
    {\True $\int\dfrac{1}{x}\mathrm{\,d}x=\ln \left| x\right|+C$}
    {$\int\ln x\mathrm{\,d}x=x+C$}
    {$\int\ln\left| x\right|\mathrm{\,d}x=\ln x+C$}
\end{ex}

\begin{ex}%Câu 11
    Bạn An rất thích nhảy hiện đại. Thời gian tập nhảy mỗi ngày của bạn An được thống kê lại ở bảng sau:\\
    \centerline{\begin{tblr}{
                colspec={|c|c|c|c|c|c|},
                hlines,
                vlines,
            }
            Thời gian (phút) & [20;25) & [25;30) & [30;35) & [35;40) & [40;45) \\
            Số ngày & 6 & 6 & 4 & 1 & 1 \\
    \end{tblr}}
    Độ lệch chuẩn của mẫu số liệu ghép nhóm có giá trị gần nhất với giá trị nào dưới đây?
    \choice
    {$ 31{,}25$}
    {$ 31{,}26$}
    {$ 5{,}4$}
    {\True $ 5{,}6$}
\end{ex}

\begin{ex}%Câu 12
    Trong không gian với hệ tọa độ $Oxyz$ , cho đường thẳng $\Delta :\dfrac{x-2}{-3}=\dfrac{y}{1}=\dfrac{z+1}{2}$. Gọi $M$ là giao điểm của $\Delta $ với mặt phẳng $(P):x+2y-3z+2=0$ . Tọa độ điểm $M$ là
    \choice
    {$ M\left(2;0;-1\right)$}
    {$ M\left(5;-1;-3\right)$}
    {$ M\left(1;0;1\right)$}
    {\True $ M\left(-1;1;1\right)$}
        \end{ex}
\Closesolutionfile{ans}
\cauds
\Opensolutionfile{ans}[ans/ans-HXN-\sode-TF]
%Câu hỏi

\begin{ex}%Câu 13
% \immini
% {
    Năm $2025$, báo Giáo dục đã có cuộc khảo sát tại một trường đại học và thấy rằng có $40\%$ sinh viên quan tâm đến chương trình học bổng A; có $17\%$ trong số những sinh viên quan tâm đến học bổng A cũng đã quan tâm đến học bổng B. Qua khảo sát họ cũng thấy rằng có $20\%$ sinh viên quan tâm đến chương trình học bổng B. Người ta chọn ngẫu nhiên một sinh viên từ trường đại học này để thăm dò ý kiến.
% }
% {
%     \includegraphics[width=6cm]{img/HXN-1-13}
% }
    \choiceTF
    {Xác suất để sinh viên được được chọn quan tâm cả hai chương trình học bổng bằng $0{,}062$}
    {Xác suất để sinh viên quan tâm học bổng A nếu biết rằng họ đã quan tâm học bổng B bằng $0{,}4$}
    {Xác suất để sinh viên không quan tâm đến cả chương trình A lẫn học chương trình B bằng $0{,}41$}
    {\True Sinh viên được chọn cho rằng mình có quan tâm đến học bổng $B$; hai hôm sau một nhà báo khác quay lại trường và tiếp tục chọn ngẫu nhiên một sinh viên để thăm dò ý kiến thì gặp được một sinh viên quan tâm đến học bổng $B$, xác suất để người này không quan tâm đến học bổng A bằng $0{,}66$}
   \loigiai{
       Gọi $A$ là biến cố: \lq\lq Sinh viên quan tâm đến học bổng $A$\rq\rq và $B$ là biến cố: \lq\lq Sinh viên quan tâm đến học bổng $B$\rq\rq.\\
       Theo giả thiết ta có $P(A)=0{,}4$; $P\left(B|A\right)=0{,}17$; $P(B)=0{,}2$.\\
       Từ đây ta có sơ đồ hình cây như sau:\\
       \centerline{\includegraphics[width=6cm]{img/HXN-1-13-LG}}
       \begin{itemchoice}
           \itemch 
           Ta có $P(AB)=P(A)\cdot P\left(B|A\right)=0{,}4\cdot 0{,}17=0{,}068$.
           \itemch Ta có $P\left(A|B\right)=\dfrac{P(AB)}{P(B)}=\dfrac{0{,}068}{0{,}2}=0{,}34$.
           \itemch Ta có $P\left(A\cup B\right)=P(A)+P(B)-P(AB)=0{,}4+0{,}2-0{,}068=0{,}532$.\\
           Do đó $P\left(\bar{A}\bar{B}\right)=1-P\left(A\cup B\right)=1-0{,}532=0{,}468$.
           \itemch Ta có $P\left(\bar{A}|B\right)=\dfrac{P\left(\bar{A}B\right)}{P(B)}=\dfrac{P(B)-P(AB)}{P(B)}=\dfrac{0{,}2-0{,}068}{0{,}2}=0{,}66$.\\
           Vì hai cuộc khảo sát là độc lập nên lần chọn đầu không ảnh hưởng đến lần chọn sau, xác suất cần tính là $P\left(\bar{A}|B\right)=0{,}66$.
       \end{itemchoice}
   }
\end{ex}

\begin{ex}%Câu 14
\immini
{
    Cho hàm số $y=\mathrm{e}^x$ có đồ thị $(C)$. Hình phẳng $(\mathscr{D})$ giới hạn bởi các đồ thị $(C)$, tiếp tuyến của $(C)$ tại điểm $ M\left(1;e\right)$ và đường thẳng $ y=-\dfrac{1}{e}x$ được tô đậm như hình vẽ.
    \choiceTF
    {Phương trình tiếp tuyến của $(C)$ tại điểm $ M\left(1;e\right)$ là $y=e\cdot x+e$}
    {\True Đường thẳng $ y=-\dfrac{1}{e}x$ cắt đồ thị $(\mathscr{D})$ tại điểm $\left(-1;\dfrac{1}{e}\right)$}
    {\True Diện tích hình phẳng $(H)$ bằng $0{,}81$ (làm tròn đến hàng phần trăm)}
    {Khi quay hình $(\mathscr{H})$ quanh trục hoành thì được khối tròn xoay có thể tích bằng $3{,}03$ (làm tròn đến hàng phân trăm)}
}
{
   \includegraphics[width=6cm]{img/HXN-1-14}
}
    
    \loigiai{
        \begin{itemchoice}
            \itemch Phương trình tiếp tuyến của $(C)$ tại điểm $M(1;e)$ là $y=y'(1)(x-1)+e$ hay $y=ex$.
            \itemch Toạ độ giao điểm của $(C)$ và đường thẳng $y=-\dfrac{1}{e}x$ thỏa mãn $\heva{& e^x=-\dfrac{1}{e}x \\& y=e^x } \Leftrightarrow \heva{& x=-1 \\& y={e^{-1}}=\dfrac{1}{e}.} $
            \itemch Dễ thấy đường thẳng $y=ex$ cắt đường thẳng $y=-\dfrac{1}{e}x$ tại điểm có hoành độ $x=0$ và cắt đồ thị $(C)$ tại điểm có hoành độ $x=1$.\\
            Do đó diện tích hình phẳng (H) là $S=\int\limits_{-1}^0{\left(\mathrm{e}^x+\dfrac{1}{e}x\right)\mathrm{\,d}x}+\int\limits_0^1{\left(\mathrm{e}^x-ex\right)\mathrm{\,d}x}\approx0{,}81$.
            \itemch Thể tích khối tròn xoay là $V=\pi \int\limits_{-1}^1{\left(e^x\right)^2\mathrm{\,d}x}-\pi \int\limits_{-1}^0{\left(-\dfrac{1}{e}x\right)^2\mathrm{\,d}x}-\pi\int\limits_0^1{(ex)^2\mathrm{\,d}x}\approx 3{,}51$.
        \end{itemchoice}
    }
\end{ex}

\begin{ex}%Câu 15
\immini
{
Hai thành phố cách nhau một con sông. Lấy $A$ và $B$ lần lượt là hai điểm mốc của hai thành phố trong việc đo đạc, đơn vị là km. Người ta xây dựng một cây cầu $EF$ bắc qua sông biết rằng vị trí $A$ cách con sông một khoảng $AH=5$km và vị trí B cách con sông một khoảng là $BK=7$km (xem hình vẽ), biết $HE+KF=24km$ và độ dài $EF$ không đổi.
Đặt $HE=x$ (km), với $x\in\left(0;24\right)$.
}
{
    \includegraphics[width=6cm]{img/HXN-1-15}
}
    \choiceTF
    {\True $AE=\sqrt{25+x^2}$(km), $BF=\sqrt{49+\left(24-x\right)^2}$(km)}
    {Tổng quãng đường đi từ $A$ đến $B$ bằng $\sqrt{25+x^2}+\sqrt{49+x^2}+EF$ (km)}
    {\True Nếu đặt $f(x)=AE+BF$ (km) thì $f'(x)=\dfrac{x}{\sqrt{x^2+25}}+\dfrac{x-24}{\sqrt{x^2-48x+625}}$, $\forall x\in\left(0;24\right)$}
    {Người ta muốn đi từ $A$ đến $B$ theo quãng đường ngắn nhất thì họ phải xây cầu sao cho khoảng cách hai điểm $E$, $H$ bằng $9$ km}
    \loigiai{
        \begin{itemchoice}
            \itemch Với $HE=x$ thì $FK=24-x$ ($0<x<24$).\\
            Ta có $\heva{& AE=\sqrt{25+x^2} \\& BF=\sqrt{49+(24-x)^2}.} $
            \itemch Tổng quãng đường đi từ $A$ đến $B$ là $AE+EF+BF=\sqrt{25+x^2}+\sqrt{49+(24-x)^2}+EF$ (km).
            \itemch Xét hàm số $f(x)=\sqrt{x^2+25}+\sqrt{x^2-48x+625}$;
            $f'(x)=\dfrac{x}{\sqrt{x^2+25}}+\dfrac{x-24}{\sqrt{x^2-48x+625}}$, $\forall x\in (0;24)$.
            \itemch Ta cần tổng quãng đường $AE+EF+FB$ ngắn nhất, mà $EF$ không đổi nên $AE+FB$ bé nhất. \\
            Ta có $f'(x)=\dfrac{x}{\sqrt{x^2+25}}+\dfrac{x-24}{\sqrt{x^2-48x+625}}$, $\forall x\in (0;24)$; $f'(x)=0\Rightarrow x=10$.\\
            Bảng biến thiên \\
            \centerline{\begin{tikzpicture}[>=stealth]
                    \tkzTabInit[nocadre=false,lgt=1.2,espcl=2.5,deltacl=0.5]{$x$/.7 ,$f'(x)$/.7,$f(x)$/2}
                    {$0$ , $10$ , $24$}
                    \tkzTabLine{ ,-,$0$,+,}
                    \tkzTabVar{+/,-/$12\sqrt{5}$,+/}
                \end{tikzpicture}}
            Ta có $\min\limits_{(0;24)} f(x)=12\sqrt{5}$; khi đó $x=10$km  và $BF=7\sqrt{5}$km $\approx 15{,}65$km.
        \end{itemchoice}
    }
\end{ex}

\begin{ex}%Câu 16
\immini
{
    Trong Dragon Ball, quả cầu Genki là chiêu thức lợi hại mà Son Goku thường sử dụng khi gặp những đối thủ lớn. Được biết trong trận đánh với Frieza đại đế, cuộc chiến có liên quan đến vận mệnh vũ trụ, Goku đã dùng quả cầu này để tung đòn tuyệt sát với Frieza.\\
Chọn hệ trục tọa độ $Oxyz$ thích hợp, đơn vị trên mỗi trục là mét, mặt phẳng $Oxy$ là mặt đất và tia $Oz$ hướng lên trời, Son Goku đứng ở vị trí $ A\left(5;0;40\right)$, Frieza đại đế đứng ở vị trí $ B\left(85;60;40\right)$. Trước khi Goku tạo ra quả cầu Genki thì Frieza đã tấn công phủ đầu, hắn lao về phía Goku với vận tốc $50$ m/s.
}
{
    \includegraphics[width=5cm]{img/HXN-1-16}
}
\choiceTF
    {\True Frieza sẽ mất $2$ giây để đến được vị trí Goku đang đứng}
    {Vectơ vận tốc của Frieza là $\vec{v}=\left(400;300;0\right)$, đơn vị: m/s}
    {\True Sau khi tránh được đòn hiểm từ Frieza, Goku đứng ở vị trí $ C\left(8;-1;46\right)$ đã tạo ra quả cầu Genki được mô hình hóa với phương trình $\left(x-8\right)^2+\left(y+1\right)^2+\left(z-58\right)^2=100$. Khoảng cách bé nhất từ vị trí $ D\left(-182;159;45\right)$ mà Frieza đang đứng đến quả cầu bằng $ 238{,}7$m (kết quả làm tròn đến hàng phần chục)}
    {\True Quả cầu được Goku ném về phía Fide với vận tốc lên đến $ 64$ m/s. Cứ sau mỗi giây thì bán kính nó tăng lên $1$ mét. Nếu Frieza không di chuyển thì sau 3,67 giây (làm tròn đến hàng phần trăm của giây) quả cầu Genki đến được vị trí của Frieza}
    \loigiai{
        \begin{itemchoice}
            \itemch Ta có $\vec{BA}=(-80;-60;0)$ và $AB=\sqrt{(-80)^2+(-60)^2}=100$ m.\\
            Thời gian để Frieza bay từ $B$ đến $A$ để tấn công Goku là $\dfrac{100}{50}=2$s.
            \itemch Vectơ vận tốc của Frieza có dạng $\vec{v}=k\vec{BA}=(-80k;-60k;0)$, với tham số $k>0$.\\
            Ta có $\left| {\vec{v}} \right|=50\Rightarrow \sqrt{(-80k)^2+(-60k)^2}=50\Rightarrow 100|k|=50\Rightarrow k=\dfrac{1}{2}>0$.\\
            Do đó Frieza bay đến chỗ Goku với vectơ vận tốc $\vec{v}=(-40;-30;0)$.
            \itemch Quả cầu Genki có tâm $I(8;-1;58)$, bán kính $R=10\,m$.\\
            $ID=\sqrt{\left(-182-8\right)^2+(159+1)^2+(45-58)^2}=\sqrt{61\,869}\,m\approx 248{,}7\,m$.\\
            Khoảng cách ngắn cần tính là $ID-R=\sqrt{61\,869}-10\approx 238{,}7\,m$.
            \itemch Sau $t$ giây, điểm $M$ (thuộc mặt cầu gần Frieza nhất) di chuyển đoạn đường: $64t+t=65t$ (m).\\
            Khi $M$ chạm vào Frieza (nếu hắn đứng yên) thì $ID-R=65t\Rightarrow t=\dfrac{ID-R}{65}\approx 3{,}67$ (giây).
        \end{itemchoice}
    }
\end{ex}
\Closesolutionfile{ans}
\caukq
\Opensolutionfile{ans}[ans/ans-HXN-\sode-SA]
%Câu hỏi

\begin{ex}%Câu 17
\immini
{
        Một cái ly nước hình hình trụ có chiều cao 9 cm. Lượng nước trong ly chiếm $\dfrac{2}{3}$ thể tích ly nước. Hoa đặt một viên kim cương hình lập phương vào miệng ly nước thì thấy một đỉnh của viên kim cương chạm vào mặt nước, đồng thời mô hình ly nước và kim cương cùng lấy trục ly nước làm trục đối xứng. Nếu ban đầu Hoa đổ nước đầy ly thì sau khi đặt khối lập phương như trên, lượng nước tràn ra là bao nhiêu cm khối (làm tròn đến hàng phần chục và bỏ qua độ dày của ly)?
    \shortans{23,4}
}
{
    \includegraphics[width=3cm]{img/HXN-1-17}
}
\loigiai{
\immini
{
    Xét hình chóp tam giác đều $SABC$ trong đó $S$ là đỉnh của hình lập phương nằm bên trong ly nước và $A$, $B$, $C$ là các điểm chung của kim cương với miệng ly; $O$ là trọng tâm tam giác $ABC$ và $H$ là trung điểm $BC$. \\
Đặt $x$ (cm) là cạnh đáy hình chóp thì\\
\centerline{
    $AO=\dfrac{2}{3}AH=\dfrac{2}{3}\cdot \dfrac{x\sqrt{3}}{2}=\dfrac{x\sqrt{3}}{3}$.
}
Vì hình chóp $S.ABC$ có $SA$, $SB$, $SC$ bằng nhau và đôi một vuông góc (tại $S$) nên $SA=SB=SC=\dfrac{x}{\sqrt{2}}$.\\
Từ đó suy ra $SO=\sqrt{SA^2-OA^2}=\sqrt{\dfrac{x^2}{2}-\dfrac{x^2}{3}}=\dfrac{x\sqrt{6}}{6}$.
}
{
    \includegraphics[width=6cm]{img/HXN-1-17-LG}
}
Theo giả thiết thì chiều cao hình chóp $S.ABC$ bằng $\dfrac{1}{3}$ chiều cao ly nước, tức là $SO=\dfrac{1}{3}\cdot 9=3$.\\
Khi đó $ \dfrac{x\sqrt{6}}{6}=3\Rightarrow x=3\sqrt{6}$ cm.\\
Ta biết rằng thể tích nước tràn ra bằng với thể tích khối chóp $S.ABC$.\\
Thể tích đó là 
$V=\dfrac{1}{3}SO\cdot S_{ABC}=\dfrac{1}{3}\cdot 3\cdot \dfrac{\left(3\sqrt{6}\right)^2\sqrt{3}}{4}=\dfrac{27\sqrt{3}}{2}\approx 23{,}4$ cm$^3$.
}
    \end{ex}
    
    \begin{ex}%Câu 18
% \immini
% {
    Một người công nhân có thể sản xuất với tốc độ là\\
    $q(t)=100+\mathrm{e}^{-0{,}5t}$ đơn vị sản phẩm trong $1$ giờ, với $t$ (giờ) là thời gian tính từ khi bắt đầu làm việc. Biết rằng người công nhân bắt đầu làm việc từ lúc $8$ giờ sáng, hỏi người đó sẽ sản xuất được bao nhiêu đơn vị sản phẩm trong khoảng thời gian từ $9$ giờ sáng đến $11$ giờ trưa (làm tròn đến hàng đơn vị)?
\shortans{201}
% }
% {
%     \includegraphics[width=6cm]{img/HXN-1-18}
% }
\loigiai{
Gọi $Q(t)$ là số đơn vị sản phẩm mà công nhân sản xuất được sau $t$ giờ tính từ lúc $8$ giờ sáng.\\
Ta có $Q'(t)=q(t)=100+{\mathrm{e}^{-0{,}5t}}$.
Số đơn vị sản phẩm người đó sản xuất được từ $9$ giờ sáng ($t=1$) đến $11$ giờ trưa ($t=3$) là\\
$Q(3)-Q(1)=\int\limits_1^3{q(t)\mathrm{\,d}t}=\int\limits_1^3\left(100+{e^{-0.5t}}\right)\mathrm{\,d}t \approx 201$ (đơn vị sản phẩm).
}
\end{ex}

\begin{ex}%Câu 19
\immini
{
    Mảnh đất vườn của nhà anh Điệp có một phần ranh giới cũng là một phần đường cong $(C)\colon y=\dfrac{x+a}{x+b}$, bao quanh nó là sông nước. Với hệ trục tọa độ $Oxy$ thích hợp, đơn vị trên mỗi trục là $10$ mét thì đường cong $(C)$ đi qua điểm $\left(2;3\right)$ và có đường tiệm cận đứng $x=1$. Hàng ngày anh Điệp phải dùng thuyền máy để vận chuyển trái cây từ khu vườn của mình đến hai tuyến đường $\Delta_1\colon 2x+y-4=0$ và $\Delta_2\colon x+2y-2=0$ cho những người lái buôn từ nơi khác đến.
}
{
    \includegraphics[width=6cm]{img/HXN-1-19}
}
Anh Điệp cần xác định một vị trí $ M\left(x_0;y_0\right)$ thuộc khu vườn của mình để tổng các khoảng cách từ vị trí M đó đến hai tuyến đường $\Delta_1,\Delta_2$ là bé nhất. Hỏi khoảng cách từ vị trí được chọn làm gốc tọa độ đến điểm M là bao nhiêu mét (làm tròn đến hàng phần chục)?
\shortans{34,1}
\loigiai{
Đồ thị hàm số có tiệm cận đứng $x=-b=1\Rightarrow b=-1$.\\
Khi đó đồ thị hàm số $y=\dfrac{x+a}{x-1}$ qua $(2;3)\Rightarrow 3=\dfrac{2+a}{2-1}\Rightarrow a=1$; hàm số là $y=\dfrac{x+1}{x-1}$ $(C)$.\\
Gọi $M\left(x_0;\dfrac{x_0+1}{x_0-1}\right)\in (C)$, $x_0>1$. Tổng khoảng cách từ $M$ đến hai đường thẳng $\triangle _1,\triangle _2$ là
\begin{eqnarray*}
    &&d=d\left(M,\triangle_1\right)+d\left(M,\triangle _2\right)=\dfrac{\left| 2x_0+\dfrac{x_0+1}{x_0-1}-4 \right|}{\sqrt{5}}+\dfrac{\left| x_0+2\cdot \dfrac{x_0+1}{x_0-1}-2 \right|}{\sqrt{5}}\\
    &\Leftrightarrow&\sqrt{5}d=\left| \dfrac{2x_0^2-5x_0+5}{x_0-1} \right|+\left| \dfrac{x_0^2-x_0+4}{x_0-1} \right|=\dfrac{2x_0^2-5x_0+5}{x_0-1}+\dfrac{x_0^2-x_0+4}{x_0-1}\\
    &&\left(\text{vì } \heva{& 2x_0^2-5x_0+5>0 \\& x_0-1>0 \\& x_0^2-x_0+4>0 },\,\forall x_0>1\right)
\end{eqnarray*}
Đặt $\sqrt{5}d=\dfrac{3x_0^2-6x_0+9}{x_0-1}=g(x)$ với $x>1$.\\
Ta có: $g'(x)=\dfrac{3x_0^2-6x_0-3}{\left(x_0-1\right)^2}$; $g'(x)=0\Rightarrow 3x_0^2-6x_0-3=0\Rightarrow x_0=1+\sqrt{2}>1$.\\
Ta có $\min \limits_{\left(1;+\infty \right)}\,g(x)=g\left(1+\sqrt{2}\right)=6\sqrt{2}\Rightarrow \sqrt{5}d\ge 6\sqrt{2}\Rightarrow d\ge \dfrac{6\sqrt{10}}{5}$.\\
Dấu đẳng thức xảy ra khi $x_0=1+\sqrt{2} \Rightarrow M\left(1+\sqrt{2};1+\sqrt{2}\right)$.\\
Khoảng cách OM trên thực tế là $10\times \sqrt{\left(1+\sqrt{2}\right)^2+\left(1+\sqrt{2}\right)^2}=10\times \left(1+\sqrt{2}\right)\sqrt{2}\approx 34{,}1$ mét.
}
\end{ex}

\begin{ex}%Câu 20
\immini
{
    Một tên trộm đang cố gắng kéo thùng nữ trang qua một bức tường có độ dày $ BC=1m$; biết rằng tường cao $4$ m và sợi dây được kéo theo đường gấp khúc $ABCD$ có độ dài không đổi bằng $20$ m, đoạn $ BF=0{,}5m$. Trong khi kéo thì tên trộm luôn ghì đầu dây theo một thanh vịn của cầu thang (đầu dây dịch chuyển theo phương $AF$). Biết rằng thanh vịn cầu thang hợp với phương ngang một góc bằng $30^{\circ}$.
}
{
    \includegraphics[width=10cm]{img/HXN-1-20}
}
Khi hai chú cảnh sát xuất hiện thì vị trí $A$ cách F khoảng $6$ m và thùng $D$ tiến về phía $E$ với tốc độ $1$ m/s. Hỏi đầu dây $A$ rời xa điểm $F$ với tốc độ bao nhiêu m/s? (Làm tròn kết quả đến hàng phần trăm).
\shortans{0,95}
\loigiai{
Đặt $DE=x$(m), $AF=y$(m).\\
Ta có $CD=\sqrt{x^2+16}$ và $\widehat{AFB}=180^{\circ }-60^{\circ }=120^{\circ }$.\\
Suy ra $AB=\sqrt{y^2+0{,}5^2-2.0{,}5\cdot y\cos 120^{\circ }}=\sqrt{y^2+0{,}5y+0{,}25}$.\\
Ta có $AB+CD+1=20\Leftrightarrow \sqrt{x^2+16}+\sqrt{y^2+0{,}5y+0{,}25}=19$ \tagEX{*}
Thay $y=6$ vào $(*)$ ta được $\sqrt{x^2+16}+\sqrt{6^2+0{,}5.6+0{,}25}=19\Rightarrow x\approx 12{,}1$  (Lưu vào A).\\
Đạo hàm hai vế của $(*)$ theo biến $t$ ta được:\\
$\dfrac{x}{\sqrt{x^2+16}}\cdot \dfrac{\mathrm{\,d}x}{\mathrm{\,d}t}+\dfrac{2y+0{,}5}{2\sqrt{y^2+0{,}5y+0{,}25}}\cdot \dfrac{dy}{\mathrm{\,d}t}=0$ \tagEX{**}
Thay $y=6$ m; $x=A\approx 12{,}1$ m; $\dfrac{\mathrm{\,d}x}{\mathrm{\,d}t}=-1$ m/s (do $x$ ngày càng giảm theo thời gian $t$) vào $(**)$ ta tính được $\dfrac{dy}{\mathrm{\,d}t}\approx 0{,}95$ m/s hay đầu dây $A$ rời xa điểm $F$ với tốc độ khoảng $0{,}95$ m/s.
}
\end{ex}

\begin{ex}%Câu 21
\immini
{
Trong công trường xây dựng, có một bộ khung sắt hình lập phương như hình vẽ (ta xem nó là hình lập phương dạng $ 2\times 2\times 2$). Người ta nhìn thấy một con kiến và một con gián xuất phát cùng lúc trên hai đỉnh thuộc đường chéo lớn của khung sắt hình lập phương và di chuyển trên các cạnh của mỗi hình vuông nhỏ. Con kiến cần đến vị trí mà con gián xuất phát và ngược lại, mỗi con ngày càng di chuyển xa vị trí mà nó xuất phát. Tính xác suất để hai con côn trùng này gặp nhau biết rằng vận tốc của gián bằng 4 cm/s, vận tốc của kiến là 2 cm/s. Kết quả được làm tròn đến hàng phần trăm.
\shortans{0,27}
}
{
    \includegraphics[width=6cm]{img/HXN-1-21}
}
\loigiai{
\immini
{
    Ta xem mỗi bước di chuyển của mỗi con là 1 đơn vị (ứng với cạnh hình vuông nhỏ).\\
Để đi hết hành trình của mình thì gián cần đi xuống $2$ đơn vị, sang trái $2$ đơn vị và đi dọc $2$ đơn vị (có tất cả là 6 bước di chuyển) nên số cách đi của gián là $\mathrm{C}_6^2\mathrm{C}_4^2$; hoàn toàn tương tự kiến cũng có số cách đi là $\mathrm{C}_6^2\mathrm{C}_4^2$.\\
Gọi $\Omega $ là không gian mẫu thì $n\left(\Omega \right)=\left(C_6^2C_4^2\right)^2$.\\
Vận tốc của gián gấp đôi vận tốc của kiến nên nếu hai con gặp nhau thì tại vị trí chúng gặp gián đã di chuyển $4$ bước, kiến di chuyển 2 bước. Vị trí hai con gặp nhau (nếu có) được đánh dấu ở $6$ vị trí trên hình vẽ.
}
{
    \includegraphics[width=6cm]{img/HXN-1-21-LG}
}
\begin{itemize}
    \item Tại vị trí $A$: Gián có $2$ lần di chuyển sang trái, $2$ lần di chuyển dọc; sau đó đi từ $A$ đến đích thì nó cần $2$ lần đi xuống. Số cách đi của gián là $\mathrm{C}_4^2\mathrm{C}_2^2\mathrm{C}_2^2$.  Hành trình của kiến cũng tương tự mà theo chiều ngược lại nên kiến có $\mathrm{C}_4^2\mathrm{C}_2^2\mathrm{C}_2^2$ cách đi. Số cách đi hai con là $\left(C_4^2C_2^2C_2^2\right)^2$.\\
    Tại các vị trí $A$, $C$, $E$ thì số cách đi mỗi con là như nhau. 
    \item Tại vị trí $B$: Số cách đi của hai con là  $\left(\mathrm{C}_4^1\mathrm{C}_3^1\mathrm{C}_2^2\mathrm{C}_2^1\right)^2$.\\
    Tại các vị trí $B$, $D$, $F$ thì số cách đi mỗi con là như nhau. 
\end{itemize}
Gọi $X$ là biến cố hai con côn trùng gặp nhau trên đường đi, ta có\\ $P(X)=\dfrac{3\left(\mathrm{C}_4^2\mathrm{C}_2^2\mathrm{C}_2^2\right)^2+3\left(\mathrm{C}_4^1\mathrm{C}_3^1\mathrm{C}_2^2\mathrm{C}_2^1\right)^2}{\left(\mathrm{C}_6^2\mathrm{C}_4^2\right)^2}=\dfrac{17}{75}\approx0{,}27$.
}
\end{ex}

\begin{ex}%Câu 22
Trong không gian $Oxyz$, cho mặt cầu $\left(S_1\right)$ có tâm $ I\left(2;1;1\right)$, bán kính bằng $ 4$ và mặt cầu $\left(S_2\right)$ có tâm $J\left(2;1;5\right)$, bán kính bằng $ 2$. Gọi $(P)$ là mặt phẳng thay đổi tiếp xúc với hai mặt cầu $\left(S_1\right),\left(S_2\right)$ và đặt $T_1,T_2$ lần lượt là giá trị nhỏ nhất, giá trị lớn nhất của khoảng cách từ điểm $ O$ đến $(P)$. Tìm giá trị $ T_1^2+T_2^2$.
\shortans{48}
\loigiai{
\immini
{
    Ta có $IJ=4<R_1+R_2$ (với $R_1=4$, $R_2=2$)  nên hai mặt cầu $\left(S_1\right)$ và $\left(S_2\right)$ cắt nhau.\\
Gọi $M$ là giao điểm của $IJ$ và $(P)$.\\
Ta có $\dfrac{MJ}{MI}=\dfrac{R_2}{R_1}=\dfrac{1}{2}\Rightarrow J$ là trung điểm của $MI$.\\
Suy ra $M(2;1;9)$.
}
{
\includegraphics[width=6cm]{img/HXN-1-22-LG}
}
Gọi $\vec{n}=(a;b;c)$ là vectơ pháp tuyến của $(P)$ với $a^2+b^2+c^2>0$.\\
Phương trình $(P)\colon a(x-2)+b(y-1)+c(z-9)=0$ hay $ax+by+cz-2a-b-9c=0$.\\
Ta có $(P)$ tiếp xúc $\left(S_1\right) \Leftrightarrow d\left(I,(P)\right)=4\Leftrightarrow \dfrac{|8c|}{\sqrt{a^2+b^2+c^2}}=4\Leftrightarrow \dfrac{|2c|}{\sqrt{a^2+b^2+c^2}}=1$.\\
Dễ thấy $c\ne 0$ nên ta có thể chọn $c=1\Rightarrow a^2+b^2=3$.\\
Khi đó $d\left(O,(P)\right)=\dfrac{\left| -2a-b-9c \right|}{\sqrt{a^2+b^2+c^2}}=\dfrac{|2a+b+9|}{2}$\tagEX{1}
Theo bất đẳng thức Cauchy Schwarz thì $|2a+b|\le \sqrt{\left(2^2+1^2\right)\left(a^2+b^2\right)}=\sqrt{5.3}=\sqrt{15}$.\\
Dấu đẳng thức xảy ra khi và chỉ khi $\dfrac{a}{2}=\dfrac{b}{1}$.\\
Do đó 
\begin{align}
    &-\sqrt{15}\le 2a+b\le \sqrt{15} \notag\\
    \Rightarrow& \underbrace{9-\sqrt{15}}_{+}\le 2a+b+9\le 9+\sqrt{15}\notag\\
    \Rightarrow& \underbrace{9-\sqrt{15}}_{+}\le |2a+b+9|\le 9+\sqrt{15} \tag{2}
\end{align}
Từ $(1)$ và $(2)$ suy ra $\dfrac{9-\sqrt{15}}{2}\le d\left(O,(P)\right)=\dfrac{|2a+b+9|}{2}\le \dfrac{9+\sqrt{15}}{2}$.\\
Do đó $T_1=\dfrac{9-\sqrt{15}}{2}$; $T_2=\dfrac{9+\sqrt{15}}{2}$ và $T_1^2+T_2^2=48$.
}
\end{ex}
\Closesolutionfile{ans}
\inputansbox{6,4,3}{ans/ans-HXN-\sode-T,ans/ans-HXN-\sode-TF,ans/ans-HXN-\sode-SA}
% %%%%%%%%%%%%%%%%%%%- HXN
\def\sode{2}
\begin{name}
	{\tenchude}
	{\tendethi}
	{\tentruong}
	{\thoigian}
\end{name}
\caulc
\Opensolutionfile{ans}[ans/ans-HXN-\sode-T]
%Câu hỏi
\begin{ex}%Câu 1
    Cho hàm số $y=f(x)$ có bảng biến thiên như sau:\\
   \centerline{\begin{tikzpicture}[>=stealth]
           \tkzTabInit[nocadre=false,lgt=1.2,espcl=2.5,deltacl=0.5]{$x$/.7 ,$f'(x)$/.7,$f(x)$/2}
           {$-\infty$ , $-3$ , $0$ , $3$ , $+\infty$}
           \tkzTabLine{ , - , $0$ , + , $0$ , - , $0$ , + , }
           \tkzTabVar{+/$+\infty$ , -/$-1$, +/$1$ , -/$-1$ , +/$+\infty$}
   \end{tikzpicture}}
    Hàm số đã cho đồng biến trên khoảng nào dưới đây?
    \choice
    {$\left(-3;3\right)$}
    {\True $\left(-3;0\right)$}
    {$\left(0;3\right)$}
    {$\left(-\infty;-3\right)$}
\end{ex}

\begin{ex}%Câu 2
    Trong không gian $Oxyz$ , cho hai điểm $A\left(1;1;-2\right)$ và $B\left(2;2;1\right)$. Vectơ $\overrightarrow{AB}$ có tọa độ là
    \choice
    {$\left(-1;-1;-3\right)$}
    {$\left(3;1;1\right)$}
    {\True $\left(1;1;3\right)$}
    {$\left(3;3;-1\right)$}
\end{ex}

\begin{ex}%Câu 3
    Tìm tập xác định của hàm số $y=\log_2\left(x-3\right)$.
    \choice
    {$\mathscr{D}=\left(-\infty;3\right)$}
    {$\mathscr{D}=\mathbb{R}$}
    {\True $\mathscr{D}=\left(3;+\infty\right)$}
    {$\mathscr{D}=\left[3;+\infty\right)$}
\end{ex}

\begin{ex}%Câu 4
    Xác định số hạng đầu $u_1$ và công sai $d$ của cấp số cộng $\left(u_n\right)$ có $u_9=5u_2$ và $u_{13}=2u_6+5$.
    \choice
    {\True $u_1=3$ và $d=4$}
    {$u_1=3$ và $d=5$}
    {$u_1=4$ và $d=5$}
    {$u_1=4$ và $d=3$}
\end{ex}

\begin{ex}%Câu 5
    Họ nguyên hàm của hàm số $f(x)=3x^2+\sin x$ là 
    \choice
    {$x^3+\cos x+C$}
    {$x^3+\sin x+C$}
    {\True $x^3-\cos x+C$}
    {$3x^3-\sin x+C$}
\end{ex}

\begin{ex}%Câu 6
    Cho hình hộp $ABCD.A'B'C'D'$. Vectơ $\vec{v}=\overrightarrow{B'A'}+\overrightarrow{B'C'}+\overrightarrow{B'B}$ bằng vectơ nào dưới đây?
    \choice
    {$\overrightarrow{DB'}$}
    {$\overrightarrow{B'D'}$}
    {$\overrightarrow{BD'}$}
    {\True $\overrightarrow{B'D}$}
\end{ex}

\begin{ex}%Câu 7
    Người ta thống kê khối lượng của $80$ quả măng cụt (đơn vị: gam) và thu được mẫu số liệu sau:\\
    \centerline{\begin{tabular}{|c|c|c|c|c|c|}
            \hline
            Khối lượng (gam) & $\left[80;82\right)$ & $\left[82;84\right)$ & $\left[84;86\right)$ & $\left[86;88\right)$ & $\left[88;90\right)$\\
            \hline
            Số quả & $17$ & $20$ & $25$ & $16$ & $12$\\
            \hline
    \end{tabular}}\\
    Khoảng biến thiên của mẫu số liệu ghép nhóm trên là
    \choice
    {$11$ gam}
    {$12$ gam}
    {\True $10$ gam}
    {$20$ gam}
\end{ex}

\begin{ex}%[2D4N1-2]%[Tổ 19 - Đợt 17 - Chương 4 - Bài 1 - CD - Đề 2]%[Bình]
    Tìm nguyên hàm $F(x)$ của hàm số $ f(x)=3{x^2}+1$, biết $F(1)=3$ là
    \choice
    { ${x^3}+3$}
    { $\dfrac{{x^3}}{3}+x+3$}
    { ${x^3}+x$}
    {\True ${x^3}+x+1$}
    \loigiai
    {
        $\int (3{x^2}+1)\mathrm{\,d}x={x^3}+x+C=F(x)$.\\
        Ta có $F(1)=3\Rightarrow 1+1+C=3\Leftrightarrow C=1$.\\
        Vậy $F(x)={x^3}+x+1$.
    }
\end{ex}

\begin{ex}%Câu 9
    Mỗi ngày bác Mạnh đều đi bộ để rèn luyện sức khỏe. Quãng đường đi bộ mỗi ngày của bác trong 20 ngày được thống kê lại ở bảng sau:\\
    \centerline{\begin{tabular}{|c|c|c|c|c|c|}
            \hline
            Quãng đường & $\left[2,7;3,0\right)$ & $\left[3,0;3,3\right)$ & $\left[3,3;3,6\right)$ & $\left[3,6;3,9\right)$ & $\left[3,9;4,2\right)$\\
            \hline
            Số ngày & 3 & 6 & 5 & 4 & 2\\
            \hline
    \end{tabular}}\\
    Phương sai của mẫu số liệu ghép nhóm gần nhất với giá trị nào sau đây?
    \choice
    {$0{,}19$}
    {$1{,}26$}
    {\True $0{,}13$}
    {$0{,}26$}
\end{ex}

\begin{ex}%Câu 10
    Cho hàm số $y=f(x)$ có đạo hàm $f'(x)=(x-2)(x+1)$, $\forall x\in\mathbb{R}$. Mệnh đề nào dưới đây đúng?
    \choice
    {Hàm số đã cho đồng biến trên $(-1;+\infty)$}
    {Hàm số đã cho nghịch biến trên $(2;+\infty)$}
    {\True Hàm số đã cho nghịch biến trên $(-1;2)$}
    {Hàm số đã cho đồng biến trên $(-1;2)$}
\end{ex}

\begin{ex}%Câu 11
    Trong không gian $Oxyz$ , mặt phẳng đi qua tâm của mặt cầu $\left(x-1\right)^2+\left(y+2\right)^2+z^2=12$ và song song với mặt phẳng $\left(Ox\text{z}\right)$ có phương trình là:
    \choice
    {$y+1=0$}
    {$y-2=0$}
    {\True $y+2=0$}
    {$x+z-1=0$}
\end{ex}

\begin{ex}%Câu 12
    Biết đồ thị $(C)$ của hàm số $y=\dfrac{x^2-4x+5}{x-1}$ có hai điểm cực trị. Đường thẳng đi qua hai điểm cực trị của đồ thị $(C)$ cắt trục hoành tại điểm $M$ có hoành độ $x_M$ bằng
    \choice
    {\True $x_M=2$}
    {$x_M=1-\sqrt{2}$}
    {$x_M=1$}
    {$x_M=1+\sqrt{2}$}
        \end{ex}
\Closesolutionfile{ans}
\cauds
\Opensolutionfile{ans}[ans/ans-HXN-\sode-TF]
%Câu hỏi

\begin{ex}%Câu 13
\immini
{
    Một cái bể nước có dạng khối chóp tứ giác đều ngược với cạnh đáy bằng $3\sqrt{2}$dm và chiều cao bằng $6$dm (tham khảo hình vẽ bên – các kích thước được nêu ra là phần bên trong hình). Nước được bơm vào bể với tốc độ không đổi là $2$ lít/phút và ban đầu bể không chứa nước (các kết quả bên dưới được làm tròn đến hai chữ số thập phân sau dấu phẩy).
    \choiceTF
    {\True Bể nước được bơm đầy sau $18$ phút}
    {\True Tốc độ dâng lên của nước là $0{,}23$ dm/phút khi thể tích nước trong bể bằng $\dfrac{1}{3}$ thể tích của bể}
    {\True Khi mực nước cách miệng bể $0{,}5$ dm, người ta ngừng bơm và bắt đầu xả ra với ước lượng tốc độ giảm chiều cao của mực nước trong bể theo thời gian $t$ (phút) được mô hình hóa bởi hàm số: $h'(t)=\dfrac{1}{350}t-\dfrac{193}{700}$ (dm/phút). Sau $5$ phút, thể tích nước trong bể là $11,97$ dm$^3$}
    {Cùng với dữ kiện của  thì sau $23{,}59$ phút nước trong bể vừa được xả hết}
}
{
    \includegraphics[width=7cm]{img/HXN-2-13}
}
\loigiai{
    \begin{itemchoice}
        \itemch 
        Thể tích chậu nước hình chóp tứ giác đều là $V_{\text{chậu}}=\dfrac{1}{3}\cdot \left(3\sqrt{2}\right)^2\cdot 6=36\,dm^3=36$ lít.\\
        Thời gian bơm nước đầy bể là $\dfrac{36}{2}=18$ phút.
        \itemch Gọi $V$(t), $h$(t) lần lượt là thể tích và chiều cao của nước sau $t$ phút.\\
        Ta có $\dfrac{V(t)}{V_{\text{chậu}}}=\left(\dfrac{h(t)}{6}\right)^3\Rightarrow V(t)=\left(\dfrac{h(t)}{6}\right)^3\cdot 36$ hay $V(t)=\dfrac{(h(t))^3}{6}$\tagEX{*}
        Đạo hàm hai vế của $(*)$ theo biến $t$ ta được: $\dfrac{dV(t)}{\mathrm{\,d}t}=\dfrac{(h(t))^2}{2}\cdot \dfrac{dh(t)}{\mathrm{\,d}t}$\tagEX{**}
        Thời điểm thể tích nước bằng $\dfrac{1}{3}$ thể tích chậu thì \\
        \centerline{
        $\left(\dfrac{h(t)}{6}\right)^3=\dfrac{1}{3}\Rightarrow \dfrac{h(t)}{6}=\sqrt[3]{\dfrac{1}{3}}\Rightarrow h(t)=6\cdot \sqrt[3]{\dfrac{1}{3}}\approx 4{,}16\,dm$.
        }
        Thay $\dfrac{dV(t)}{\mathrm{\,d}t}=2$ lít/phút $=2$ dm/phút; $h(t)\approx 4{,}16$dm vào (**), thì:$\dfrac{dh(t)}{\mathrm{\,d}t}\approx 0{,}23$ dm/phút.
        \itemch Mực nước cách miệng bể $0{,}5$ dm nên chiều cao ban đầu bằng $5{,}5\,dm$.\\
        Chiều cao của nước trong chậu sau $5$ phút là\\
        \centerline{$h(5)=5{,}5+\int\limits_0^5{\left(\dfrac{1}{350}t-\dfrac{193}{700}\right)\mathrm{\,d}t}=\dfrac{291}{70}\approx 4{,}16$dm.}
        Thể tích nước còn lại trong bể là $V(5)=\left(\dfrac{h(5)}{6}\right)^3.36\approx 11{,}97\,dm^3$.
        \itemch Thời gian để mực nước trong chậu đang là $5{,}5$ dm trở về $0$ dm thỏa mãn phương trình\\
        \centerline{$5{,}5+\displaystyle\int\limits_0^t{\left(\dfrac{1}{350}t-\dfrac{193}{700}\right)\mathrm{\,d}t}=0\Rightarrow t\approx 22{,}59$ phút.}
        Với khoảng thời gian $22{,}59$ phút thì nước trong bể vừa được xả hết.
    \end{itemchoice}
}
\end{ex}

\begin{ex}%Câu 14
Có hai tên cướp vừa lấy được một chiếc ca nô ở vị trí $A$ thuộc bờ sông, chúng liền cho ca nô chạy theo phương hợp với bờ sông một góc $60^\circ$ với vận tốc $v=2t$ (mét/giây), trong đó $t$ (giây) là thời gian kể từ khi xuất phát. Sau $21$ giây, ca nô đến vị trí $B$ và chúng quyết định chuyển hướng cho ca nô chuyển động thẳng đều theo phương song song với bờ sông, tầm nửa phút sau thì ca nô đến $C$ (tham khảo hình vẽ).\\
\centerline{
    \includegraphics[width=8cm]{img/HXN-9-14}
}
    \choiceTF
    {\True Vị trí $B$ mà ca nô bọn cướp chuyển hướng cách bờ sông khoảng $382$ m (làm tròn đến hàng đơn vị)}
    {Khoảng cách $A$, $C$ tính theo đường chim bay bằng $1522$ m (làm tròn đến hàng đơn vị)}
    {\True Nếu các chiến sĩ công an khởi động ca nô và đi thẳng từ $D$ đến $C$ với vận tốc được tăng thêm $3$ m sau mỗi giây thì sau $21$ giây sẽ đến vị trí $D$ (làm tròn đến hàng đơn vị của giây)}
    {Trên thực tế các chiến sĩ đã chọn giải pháp cho ca nô khởi động và di chuyển vuông góc với bờ với gia tốc $a$ dương, cùng lúc đó bọn cướp từ vị trí $D$ tiến thẳng về phía trước (giữ nguyên hướng đi và tốc độ), hai bên giáp mặt nhau khi $a=4{,}78$ m/s$^2$ (làm tròn đến hàng phần trăm)}
    \loigiai{
        \begin{center}
            \begin{tikzpicture}[declare function={d=6;b=2;c=3;},thick]
                \path
                (0,0) coordinate (A)
                (-60:b) coordinate (B)
                ($(B)+(0:c)$) coordinate (C)
                (0:d) coordinate (D)
                ($(A)!(B)!(D)$) coordinate (H)
                ($(A)!(C)!(D)$) coordinate (K)
                ($(B)!(D)!(C)$) coordinate (E)
                pic[draw,angle radius=5mm,"$60^\circ$",angle eccentricity=1.7]{angle=B--A--H}
                ;
                \foreach \x/\y/\z in {B/H/K,C/K/D,C/E/D}\draw pic[draw,angle radius =2mm] {right angle = \x--\y--\z};
                \draw[dashed] (B)--(H) (C)--(K) (C)--(E)--(D);
                \draw (A)--(B)--(C) (D)--(A);
                \draw[red] (A)--(C)--(D);
                \foreach \x/\g in {A/180,B/-90,C/-90,D/0,E/-90,H/90,K/90}\draw[fill=white] (\x) circle (1pt)+(\g:3mm) node{$\x$};
            \end{tikzpicture}
        \end{center}
    \begin{itemchoice}
        \itemch Sau $21$ giây, ca nô bọn cướp đi được $AB=\int\limits_0^{21}{2t\mathrm{d}t}=441m$.\\
        Gọi $H$ là hình chiếu vuông góc của $B$ trên bờ sông, khi đó $BH=AB\sin 60^\circ=\dfrac{441\sqrt{3}}{2}\approx 382m$.
        \itemch Vận tốc của ca nô bọn cướp tại $B$ là $v_B=2\times 21=42$ m/s.\\
        Khoảng cách hai vị trí $B$, $C$ là $BC=42\times 30=1260m$.\\
        Khoảng cách hai vị trí $A$, $H$ là $AH=AB\cos 60^\circ=220{,}5m$.\\
        Gọi $K$ là hình chiếu vuông góc của $C$ trên bờ sông thì $HK=BC=1260m$.\\
        Do đó $AK=AH+HK=1480{,}5m$; $CK=BH=\dfrac{441\sqrt{3}}{2}m$ và $AC=\sqrt{AK^2+CK^2}\approx 1529 m$.
        \itemch Ta có $DK=2000-\left(220{,}5+1260\right)=519{,}5m$; suy ra $CD=\sqrt{CK^2+DK^2}\approx 644{,}8m$.\\
        Thời gian để các chiến sĩ đi được từ $D$ đến $C$ thỏa $\displaystyle\int\limits_0^t{3t\mathrm{d}t}=\sqrt{415741}\Rightarrow t\approx 21s$.
        \itemch Gọi $E$ là vị trí hai bên giáp mặt nhau (nếu có) thì tam giác $CDE$ vuông tại $E$.\\
        Khi đó $CE=DK=519{,}5m$ và $DE=\dfrac{441\sqrt{3}}{2}m$.\\
        Thời gian để ca nô bọn cướp đi từ $C$ đến $E$ là $\dfrac{CE}{42}=\dfrac{1039}{84}\approx 12{,}37s$.\\
        Gia tốc $a$ thỏa mãn $\displaystyle\int\limits_0^{\tfrac{1039}{84}}{at \mathrm{d}t}=\dfrac{441\sqrt{3}}{2}\Rightarrow 4{,}99m/s^2$.
    \end{itemchoice}
    }
\end{ex}

\begin{ex}%Câu 15
\immini
{
    Sao Thủy gần như không có khí quyển thật sự như Trái Đất hay sao Kim. Tuy nhiên, nó có một lớp khí rất mỏng gọi là exosphere – tức là thượng quyển loãng, gồm các hạt khí cực kỳ thưa thớt như hydro, heli, oxy, natri...Trong không gian Oxyz, đơn vị trên mỗi trục là nghìn km, vùng thượng quyển loãng của sao Thủy được mô hình hóa bởi phương trình mặt cầu $x^2+y^2+z^2-2x-4y-4=0$. Các nhà khoa học không gian đang quan sát các tiểu hành tinh ở các vị trí có tọa độ $A\left(4;2;4\right)$, $B\left(1;4;2\right)$ và xem xét sự di chuyển của chúng. Nếu tiểu hành tinh nằm trong vùng thượng quyển loãng thì nó sẽ bị hút xuống bề mặt sao Thủy.
}
{
    \includegraphics[width=5cm]{img/HXN-2-15}
}
\choiceTF
    {\True Vùng thượng quyển loãng sao Thủy có tâm $\left(1;2;0\right)$, bán kính bằng $3$}
    {Hai tiểu hành tinh ở các vị trí $A$, $B$ sẽ bị hút xuống bề mặt sao Thủy}
    {Các nhà quan sát cho rằng có một sao chổi mang tên Haxen di chuyển theo quỹ đạo đường thẳng với vận tốc $51{,}5$ km/s; khoảng cách ngắn nhất từ tâm sao Thủy đến sao chổi bằng $\dfrac{\sqrt{871}}{10}$ nghìn km. Thời gian sao chổi đi trong vùng thượng quyển loãng của sao Thủy bằng $20$ giây (làm tròn đến hàng đơn vị của giây)}
    {\True Sao chổi Haxen di chuyển theo phương vectơ $\vec{u}=\left(0;5;2\right)$ . Giả sử $M$, $N$ là điểm đầu và điểm cuối mà sao chổi này đi qua thuộc vùng thượng quyển loãng của sao Thủy. Giá trị nhỏ nhất của tổng $AM+BN$ bằng $3970$ km (làm tròn đến hàng đơn vị của km)}
    \loigiai{
        \begin{itemchoice}
            \itemch Vùng thượng quyển loãng sao Thủy có tâm $I(1;2;0)$, bán kính $R=\sqrt{1^2+2^2+0^2+4}=3$.
            \itemch Ta có $IA=\sqrt{(4-1)^2+(2-2)^2+(4-0)^2}=5$; $IB=\sqrt{(1-1)^2+(4-2)^2+(2-0)^2}=2\sqrt{2}$.\\
            Vì $IA>R$; $IB<R$ nên tiểu hành tinh $A$ nằm ngoài vùng thượng quyển loãng, còn tiểu hành tinh $B$ thì nằm tròn vùng thượng quyển loãng của sao Thủy và nó sẽ bị hút xuống bề mặt sao Thủy.
            \itemch Gọi $d$ là quỹ đạo đường thẳng của sao chổi và $H$ là hình chiếu vuông góc của tâm $I$ trên $d$.\\
            Gọi $M$, $N$ là điểm đầu và điểm cuối của sao chổi trong vùng thượng quyển loãng của sao Thủy.\\
            Ta có $IH=\dfrac{\sqrt{871}}{10}$; suy ra $MN=2MH=2\sqrt{R^2-IH^2}=2\sqrt{9-\dfrac{871}{100}}=\dfrac{\sqrt{29}}{5}$ (nghìn km).
            Thời gian sao chổi di chuyển trong vùng thượng quyển loãng: $\dfrac{\sqrt{29}}{5}\times 1000\colon 51{,}5\approx 21$ (giây).
            \immini
            {
                \itemch Đặt $\vec{MN}=(0;5k;2k)$;\\
                ta có $MN^2=25k^2+4k^2=\dfrac{29}{25}\Rightarrow k=\dfrac{1}{5}$;\\
                suy ra $\vec{MN}=\left(0;1;\dfrac{2}{5}\right)$.\\
            Chọn $C$ sao cho $AMNC$ là hình bình hành;\\
            suy ra $C\left(4;3;\dfrac{22}{5}\right)$.\\
            Khi đó $AM=CN$ và $AM+BN=CN+BN\ge BC=\dfrac{\sqrt{394}}{5}\approx 3{,}97$.\\
            Vậy giá trị nhỏ nhất của biểu thức $AM+BN$ là khoảng $3970$ km.
            }
            {
                \includegraphics[width=4cm]{img/HXN-2-15-LG}
            }
        \end{itemchoice}
    }
\end{ex}

\begin{ex}%Câu 16
\immini
{
    Có hai hộp bóng $A$ và $B$ chỉ đựng các quả bóng đỏ và trắng, trong đó hộp $B$ đựng $4$ quả bóng đỏ và $5$ quả bóng trắng; tổng số bóng hai hộp không qua $20$. Xét hai phép thử ngẫu nhiên sau:\\
Phép thử thứ nhất: Lấy ngẫu nhiên $1$ quả bóng từ hộp $A$ bỏ vào hộp $B$ rồi lấy ngẫu nhiên $1$ quả bóng từ hộp $B$. Bằng thực nghiệm người ta biết được rằng khả năng lấy được quả bóng đỏ từ hộp thứ hai bằng $\dfrac{33}{70}$.\\
Phép thử thứ hai: Lấy ngẫu nhiên $2$ quả bóng từ hộp $A$ bỏ vào hộp B. Sau đó tiếp tục lấy ngẫu nhiên $2$ quả bóng từ hộp $B$.
}
{
\includegraphics[width=5cm]{img/HXN-2-16}
}
\choiceTF
    {Trong phép thử thứ nhất, nếu lấy được quả bóng đỏ từ hộp $A$ bỏ sang hộp $B$ thì xác suất lấy được quả bóng trắng từ hộp $B$ bằng $0{,}4$}
    {Hộp thứ nhất đựng $4$ quả bóng đỏ và $3$ quả bóng trắng
    }
    {Xác suất để lấy được $2$ quả bóng đỏ từ hộp $B$ bằng $\dfrac{166}{1155}$}
    {Nếu biết rằng hai quả bóng lấy được từ hộp $B$ cùng có màu đỏ, xác suất để có ít nhất $1$ quả là từ hộp $A$ chuyển sang bằng $\dfrac{67}{128}$}
    \loigiai{
        \begin{itemchoice}
            \itemch Trong phép thử thứ nhất, nếu lấy được quả bóng đỏ từ hộp $A$ bỏ sang hộp $B$ thì khi đó hộp $B$ có $5$ quả bóng đỏ, $5$ quả bóng trắng. Do đó xác suất lấy được quả bóng trắng từ hộp $B$ bằng $0{,}5$.
            \itemch Trong phép thử thứ nhất, gọi các biến cố $X$: \lq\lq Lấy được quả bóng đỏ từ hộp $A$\rq\rq và $Y$: \lq\lq Lấy được quả bóng đỏ từ hộp thứ hai\rq\rq. Đặt $P(X)=x\in (0;1)$; suy ra $P\left({\bar{X}}\right)=1-x$.
            Ta có sơ đồ hình cây sau đây:\\
            \centerline{
                \includegraphics[width=5cm]{img/HXN-2-16-LG-a}
            }
            Theo giả thiết ta có $P(Y)=0{,}5x+0{,}6(1-x)=\dfrac{33}{70}\Rightarrow x=\dfrac{5}{7}$.\\
            Vì tổng số bóng hai hộp bóng không quá $20$ mà hộp $B$ có $9$ quả bóng nên hộp $A$ có không quá $11$ quả bóng, mà xác suất để lấy được quả bóng đỏ bằng $\dfrac{5}{7}$ nên hộp A có 5 quả bóng đỏ và 2 quả bóng trắng.
            \itemch Trong phép thử thứ hai, ta gọi 2Đ là biến cố \lq\lq Lấy đúng $2$ quả bóng đỏ từ hộp $B$\rq\rq.\\
            Với thông tin có được, ta có sơ đồ hình cây cho phép thử ngẫu nhiên thứ hai:\\
            \centerline{\includegraphics[width=5cm]{img/HXN-2-16-LG-b}}
            Từ đó suy ra $P(2\text{Đ})=\dfrac{\mathrm{C}_5^2}{\mathrm{C}_7^2}\cdot \dfrac{\mathrm{C}_6^2}{\mathrm{C}_{11}^2}+\dfrac{\mathrm{C}_5^1\mathrm{C}_2^1}{\mathrm{C}_7^2}\cdot \dfrac{\mathrm{C}_5^2}{\mathrm{C}_{11}^2}+\dfrac{\mathrm{C}_2^2}{\mathrm{C}_7^2}\cdot \dfrac{\mathrm{C}_4^2}{\mathrm{C}_{11}^2}=\dfrac{256}{1\,155}$.
            \itemch Gọi $X$ là biến cố lấy được ít nhất 1 quả là từ hộp $A$ chuyển sang, ta có\\
            $P\left(X|2\text{Đ}\right)=\dfrac{\dfrac{\mathrm{C}_5^2}{\mathrm{C}_7^2}\cdot \left(\dfrac{2{,}4}{\mathrm{C}_{11}^2}+\dfrac{1}{\mathrm{C}_{11}^2}\right)+\dfrac{5{,}2}{\mathrm{C}_7^2}\cdot \dfrac{1{,}4}{\mathrm{C}_{11}^2}}{P(2\text{Đ})}=\dfrac{65}{128}$.
        \end{itemchoice}
    }
\end{ex}
\Closesolutionfile{ans}
\caukq
\Opensolutionfile{ans}[ans/ans-HXN-\sode-SA]
%Câu hỏi

\begin{ex}%Câu 17
\immini
{
    Từ một khối gỗ hình lập phương có cạnh bằng $5$ dm, người thợ mộc chỉ cần đến hai lát cắt là có thể tạo ra một khối gỗ có dạng hình chóp $S.ABCD$ với đáy $ABCD$ là hình vuông và $SA=AB=5$ dm. Người thợ cần tạo ra một vật để trang trí theo yêu cầu của khách hàng, anh đã chọn $M$ là trung điểm $SB$, $N$ thuộc cạnh $SD$ sao cho $SN=2ND$; sau đó anh ta tiếp tục thực hiện các lát cắt để có được vật thể hình tứ diện $ACMN$, thể tích vật thể sau cùng mà người thợ mộc làm ra là bao nhiêu dm$^3$ (làm tròn đến hàng phần chục)?
\shortans{10,4}
}
{
    \includegraphics[width=5cm]{img/HXN-2-17}
}
\loigiai{
    Gọi $O$ là giao điểm của $AC$ và $BD$ trong mặt phẳng đáy.\\
    \centerline{
        \includegraphics[width=5cm]{img/HXN-2-17-LG}
    }
    Ta có $V_{S.ABCD}=\dfrac{1}{3}SA\cdot S_{ABCD}=\dfrac{5^3}{3}=\dfrac{125}{3}$.\\
    Vì $OM$ là đường trung bình tam giác $SBD$ nên $OM\parallel SD \Rightarrow SD\parallel(AMC)$.\\
    Do đó $d\left(N,(AMC)\right)=d\left(D,(AMC)\right)=d\left(B,(AMC)\right)$.\\
    Suy ra $V_{ACMN}=V_{N.MAC}=V_{D.MAC}=V_{B.MAC}=V_{M.ABC}$.\\
    Ta lại có $\dfrac{V_{M.ABC}}{V_{S.ABCD}}=\dfrac{d\left(M,(ABCD)\right)\cdot S_{ABC}}{d\left(S,(ABCD)\right)\cdot S_{ABCD}}=\dfrac{\dfrac{1}{2}d\left(S,(ABCD)\right)\cdot \dfrac{1}{2}\cdot S_{ABCD}}{d\left(S,(ABCD)\right)\cdot S_{ABCD}}=\dfrac{1}{4}$.\\
    $\Rightarrow V_{M.ABC}=\dfrac{1}{4}V_{S.ABCD}=\dfrac{1}{4}\cdot \dfrac{125}{3}=\dfrac{125}{12}\approx10{,}4$ dm$^3$.
}
    \end{ex}
    
    \begin{ex}%Câu 18
\immini
{
    Jack có một chiếc điện thoại thông minh đã được sạc đầy pin. Nếu Jack không sử dụng điện thoại một phút nào thì máy sẽ hết pin sau $96$ tiếng; còn nếu anh ấy sử dụng điện thoại liên tục thì máy sẽ hết pin sau $8$ tiếng. Biết Jack đã không sử dụng chiếc smartphone trong suốt $36$ tiếng, sau đó lại dùng nó $90$ phút liên tục. Hỏi Jack còn dùng điện thoại được bao nhiêu phút nữa trước khi máy hết pin?
\shortans{210}
}
{
    \includegraphics[width=5cm]{img/HXN-2-18}
}
\loigiai{
    Ta chuẩn hóa tổng thời lượng pin điện thoại ban đầu là $1$.
    \begin{itemize}
        \item Nếu Jack không sử dụng điện thoại thì sau $96$ tiếng smartphone mới hết pin.\\
        Suy ra sau mỗi tiếng không sử dụng, thời lượng pin sẽ giảm đi $\dfrac{1}{96}$.
        \item Nếu Jack dùng điện thoại liên tục thì sau $8$ tiếng, smartphone sẽ hết pin.\\
        Suy ra sau mỗi tiếng sử dụng, thời lượng pin sẽ giảm đi $\dfrac{1}{8}$.
        \item Sau khi Jack không sử dụng điện thoại trong suốt 36 tiếng, thời lượng pin giảm $36\times \dfrac{1}{96}=\dfrac{3}{8}$.\\
        Thời lượng pin còn lại là: $1-\dfrac{3}{8}=\dfrac{5}{8}$.
        \item Sau khi Jack sử dụng điện thoại liên tục trong $90$ phút, tức $\dfrac{3}{2}$ tiếng, thời lượng pin tiếp tục giảm đi: $\dfrac{3}{2}\times \dfrac{1}{8}=\dfrac{3}{16}$.\\
        Thời lượng pin còn lại là $\dfrac{5}{8}-\dfrac{3}{16}=\dfrac{7}{16}$.
    \end{itemize}
    Vậy trước khi điện thoại hết pin, Jack còn có thể sử dụng $\dfrac{7}{16}: \dfrac{1}{8}=3{,}5$ giờ = $210$ phút. 
}
\end{ex}

\begin{ex}%Câu 19
\immini
{
    Một quả trứng khủng long đồ chơi bằng nhựa có thiết diện qua trục lớn là một đường elip. Biết độ dài mỗi trục là $12$ cm và $8$ cm.\\
Bên trong quả trứng người ta cần thiết kế một chiếc hộp hình trụ để đựng các đồ chơi trẻ con như bóng đèn xanh đỏ, kẹo v.v...\\
Hỏi khối trụ như thế có thể tích tối đa bao nhiêu cm$^3$ (làm tròn đến hàng đơn vị).
\shortans{232}
}
{
    \includegraphics[height=5cm]{img/HXN-2-19-a}\includegraphics[width=5cm]{img/HXN-2-19-b}
}
\loigiai{
    Gắn elip lên hệ trục tọa độ $Oxy$ như hình vẽ, elip có $2a=12\Rightarrow a=6$; $2b=8\Rightarrow b=4$.\\
    \centerline{
    \includegraphics[width=5cm]{img/HXN-2-19-LG}
    }
    Phương trình chính tắc elip $(E)$: $\dfrac{x^2}{36}+\dfrac{y^2}{16}=1$.\\
    Đặt chiều cao và bán kính đáy hình trụ nội tiếp elip là $h$, $r$ thì điểm tiếp xúc  $M\left(\dfrac{h}{2};r\right)$ với $0<h<12;0<r<4$.\\
    Điểm $M$ thuộc $(E)$: $\dfrac{x^2}{36}+\dfrac{y^2}{16}=1\Rightarrow \dfrac{\left(\dfrac{h}{2}\right)^2}{36}+\dfrac{r^2}{16}=1 \Rightarrow \dfrac{h^2}{144}+\dfrac{r^2}{16}=1\Rightarrow r^2=16\left(1-\dfrac{h^2}{144}\right)$.\\
    Thể tích khối trụ là $V=\pi r^2h=\pi 16\left(1-\dfrac{h^2}{144}\right)\cdot h=\pi \left(16h-\dfrac{h^3}{9}\right)$ hay $V=\pi \left(16h-\dfrac{h^3}{9}\right)$.\\
    Ta có $V'=\pi \left(16-\dfrac{h^2}{3}\right)$; $V'=0\Rightarrow 16-\dfrac{h^2}{3}=0\Rightarrow h^2=48\Rightarrow h=4\sqrt{3}\in (0;12)$.\\
    Giá trị lớn nhất của thể tích khối trụ là $V_{\max }=V\left(4\sqrt{3}\right)=\pi \left(16.4\sqrt{3}-\dfrac{\left(4\sqrt{3}\right)^3}{9}\right)\approx 232$ cm$^3$.
}
\end{ex}

\begin{ex}%Câu 20
Một nhóm học sinh lớp $12$ đã lên bản thiết kế mẫu hoa văn cho một loại gạch men lát nền nhà. Các em đã vẽ $4$ đường cong như hình, từ đó tạo thành một hình $(\mathscr{H})$ khép kín ở giữa viên gạch để tạo điểm nhấn. Cụ thể cách dựng hình được thực hiện như sau:
\begin{itemize}
    \item Dựng hệ trục $Oxy$ với điểm $O$ là tâm của viên gạch, tia $Ox$ hướng sang phải và tia $Oy$ hướng lên trên, đơn vị trên mỗi trục là $5$ cm
    \item Các em lấy $O$ là tâm viên gạch và $A$ là trung điểm một cạnh viên gạch, xác định được điểm $B$ thỏa mãn $\overrightarrow{OB}=\dfrac{5}{6}\overrightarrow{OA}$.
    \item Dựng đường thẳng $\Delta \colon 5x-9=0$. Đường cong $\left(L_1\right)$ là tập hợp các điểm $M$ thỏa mãn $3MB=5d\left(M,\Delta\right)$.
    \item Lấy đối xứng đường cong $\left(L_1\right)$ qua tâm $O$ và qua các đường chéo của viên gạch thì được các đường cong còn lại.
\end{itemize}
\begin{center}
    \includegraphics[width=5cm]{img/HXN-2-20-a}\qquad  \includegraphics[width=5cm]{img/HXN-2-20-b}
\end{center}
Biết viên gạch là hình vuông có kích thước $60$cm; hỏi diện tích hình $(\mathscr{H})$ là bao nhiêu centimét vuông (làm tròn đến hàng đơn vị)?
\shortans{1168}
\loigiai{
    Ta chọn hệ trục tọa độ như hình vẽ với mỗi đơn vị trên trục bằng $5$ cm.\\
    \centerline{
        \includegraphics[width=5cm]{img/HXN-2-20-LG}
    }
    Đường cong $\left(L_1\right)$ là tập hợp điểm $M$ thỏa $3MB=5d\left(M,\triangle \right)$ hay $\dfrac{MB}{d\left(M,\triangle \right)}=\dfrac{5}{3}$\tagEX{1}
    Phương trình $\triangle \colon x=\dfrac{9}{5}$ \tagEX{2}
    Từ $(1)$ và $(2)$ ta thấy $M$ thuộc một nhánh của hyperbol $\left(L_1\right)$: $\dfrac{x^2}{a^2}-\dfrac{y^2}{b^2}=1$; trong đó điểm $B(5;0)$ là một trong hai tiểu điểm của $\left(L_1\right)$ nên $c=5$.\\
    Đường chuẩn $x=\dfrac{a}{e}=\dfrac{a^2}{c}=\dfrac{9}{5}\Rightarrow a=3$ (thử lại ta thấy $\dfrac{MB}{d\left(M,\triangle \right)}=\dfrac{5}{3}=e$ (hợp lí)).\\
    Do $c^2=a^2+b^2\Rightarrow b=\sqrt{c^2-a^2}=4$.\\
    Phương trình $\left(L_1\right)\colon \dfrac{x^2}{9}-\dfrac{y^2}{16}=1$ $(x>3)$.\\
    Xét giao điểm của $\left(L_1\right)$ với đương thẳng $y=x$, ta có $\dfrac{x^2}{9}-\dfrac{x^2}{16}=1\,\,\left(x>3\right)\Rightarrow x=\dfrac{12\sqrt{7}}{7}\approx 4{,}54$.\\
    Từ phương trình $\dfrac{x^2}{9}-\dfrac{y^2}{16}=1\Rightarrow \dfrac{y^2}{16}=\dfrac{x^2}{9}-1\Rightarrow y=4\sqrt{\dfrac{x^2}{9}-1}$ $(x>3)$.\\
    Diện tích cần tính là $S=8\left(\int\limits_0^3{x\mathrm{\,d}x}+\int\limits_3^{\tfrac{12\sqrt{7}}{7}}{\left(x-4\sqrt{\dfrac{x^2}{9}-1}\right)\mathrm{\,d}x}\right)\times 25\approx 1168$ cm$^2$.
}
\end{ex}

\begin{ex}%Câu 21
Trong không gian $Oxyz$, cho ba điểm $A\left(-8;-1;6\right)$, $B\left(1;2;3\right)$, $C\left(-4;14;\sqrt{11}\right)$. Điểm $M$ di động trên mặt cầu $\left(S_1\right)\colon {(x-4)^2}+(y-3)^2+(z+3)^2=49$ sao cho tam giác $MAB$ có $2\sin\widehat{MAB}=\sin\widehat{MBA}$. Tính giá trị nhỏ nhất của đoạn thẳng $CM^2$ (làm tròn đến hàng đơn vị).
\shortans{64}
\loigiai{
    Xét $\triangle MAB$, ta có $\dfrac{BM}{\sin \widehat{MAB}}=\dfrac{AM}{\sin \widehat{MBA}}=2R\Rightarrow \sin \widehat{MAB}=\dfrac{BM}{2R};\sin \widehat{MBA}=\dfrac{AM}{2R}$.\\
    Theo giả thiết $2\sin\widehat{MAB}=\sin\widehat{MBA} \Rightarrow 2\cdot \dfrac{BM}{2R}=\dfrac{AM}{2R}\Rightarrow AM=2BM$.\\
    Gọi $M(x;y;z)$ thì ta có 
    \allowdisplaybreaks
    \begin{eqnarray*}
        &&AM^2=4BM^2\\
        &\Leftrightarrow& (x+8)^2+(y+1)^2+(z-6)^2=4\left[(x-1)^2+(y-2)^2+(z-3)^2\right]\\ &\Leftrightarrow& 3x^2+3y^2+3z^2-24x-18y-12z-45=0\\
        &\Leftrightarrow& x^2+y^2+z^2-8x-6y-4z-15=0
    \end{eqnarray*}
    Do đó $M$ di động trên mặt cầu $(S_2)$ có tâm $I_2(4;3;2)$, bán kính $R_2=2\sqrt{11}$.\\
    Mặt khác ta cũng có $M\in (S_1)$ có tâm $I_1(4;3;-3)$, bán kính $R_1=7$.\\
    Ta có $I_1I_2=5<R_1+R_2$ nên $M$ thuộc đường tròn $(C)=(S_1)\cap (S_2)$.
    \immini
    {
        Tập hợp điểm $M$ thuộc $(C)$ thỏa hệ phương trình\\
    $\heva{& (x-4)^2+(y-3)^2+(z+3)^2=49 \\& x^2+y^2+z^2-8x-6y-4z-15=0} $\\
    $\Leftrightarrow \heva{& x^2+y^2+z^2-8x-6y+6z-15=0 \\& x^2+y^2+z^2-8x-6y-4z-15=0 }\\ \Leftrightarrow \heva{& z=0\,\,(Oxy) \\& x^2+y^2+z^2-8x-6y-4z-15=0 } $.
    }
    {
        \includegraphics[width=5cm]{img/HXN-2-21-LG}
    }
    Vậy $(C)$ thuộc mặt phẳng $(Oxy)$; hình chiếu của $I_1(4;3;-3)$ trên $(Oxy)$ là điểm $E(4;3;0)$ cũng là tâm của đường tròn $(C)$; bán kính $(C)$ là $r=\sqrt{R_1^2-I_1E^2}=\sqrt{7^2-9}=2\sqrt{10}$.\\
    Gọi $C'(-4;14;0)$ là hình chiếu của $C$ lên mặt phẳng $(Oxy)$, ta có $EC'=\sqrt{185}>r$ nên $C'$ nằm ngoài đường tròn $(E; r)$.\\
    Ta có $CM^2\ge CM_0^2=CC'^2+C'M_0^2=CC'^2+\left(EC'-r\right)^2 =11+\left(\sqrt{185}-2\sqrt{10}\right)^2\approx 64$.\\
    Vậy giá trị nhỏ nhất của $CM^2$ xấp xỉ $64$.
}
\end{ex}

\begin{ex}%Câu 22
\immini
{
    Vào một hội thi thiết kế đèn lồng Trung thu, ban tổ chức nhận được một chiếc đèn lồng đặc biệt có hình một tứ diện đều. Trên mỗi cạnh tứ diện thí sinh thiết kế $3$ bóng đèn nằm ở $3$ vị trí chia cạnh tứ diện thành $4$ đoạn bằng nhau. Cứ mỗi phút trôi qua, sẽ có ngẫu nhiên $3$ bóng đèn phát sáng, các bóng còn lại thì tắt. Tính xác suất để ngay phút đầu tiên được ban giám khảo chấm điểm, có $3$ bóng đèn phát sáng ứng với $3$ điểm tạo nên mặt phẳng song song với đúng một cạnh của tứ diện, biết rằng $3$ bóng đèn không hoàn toàn thuộc về một cạnh tứ diện. (Kết quả được làm tròn đến hàng phần trăm).
\shortans{0,27}
}
{
     \includegraphics[width=5cm]{img/HXN-2-22}
}
\loigiai{
    Tổng số cách chọn ra $3$ trong $18$ điểm là $\mathrm{C}_{18}^3=816$.\\
    Tuy nhiên, sẽ có các trường hợp ba điểm thẳng hàng, đó là khi ta lấy ba điểm thuộc cùng một cạnh, tổng số cách là $6\times \mathrm{C}_3^3=6$.\\
    Do đó $n\left(\Omega \right)=816-6=810$.\\
    Xét các mặt phẳng qua $3$ điểm ($3$ bóng đèn) và song song với đoạn $AB$.
    \immini
    {
        Trường hợp $1$: Chọn $1$ cặp điểm thuộc mặt phẳng $(ABC)$ và $1$ điểm không thuộc $(ABC)$.
    \begin{itemize}
        \item Bước $1$: Có $3$ cách chọn $1$ cặp điểm thuộc $(ABC)$.
        \item Bước $2$: Với mỗi cách chọn trong bước $1$ thì lẽ ra sẽ có $9$ cách chọn $1$ điểm không thuộc mặt phẳng $(ABC)$; tuy nhiên vì điều kiện mặt phẳng qua $3$ điểm chỉ song song đúng $1$ cạnh tứ diện (ở đây là $AB$) nên ta loại $3$ điểm trong số $9$ điểm không thuộc $(ABC)$ (ví dụ khi chọn cặp điểm $M$, $N$ thuộc $(ABC)$ thì ta loại $P$, $Q$, $R$  thuộc $AD$, $BD$, $CD$).\\
        Do vậy ta có $3\times 6=18$ cách chọn bộ ba điểm trong trường hợp này.
    \end{itemize}
    Trường hợp $2$: Chọn $1$ cặp điểm thuộc mặt phẳng $(ABD)$ và $1$ điểm không thuộc $(ABD)$.
    }
    {
        \includegraphics[width=5cm]{img/HXN-2-22-LG}
    }
    Ta cũng có $3\times 6=18$ cách chọn bộ ba điểm thỏa mãn.\\
    Vì tính chất bình đẳng của $6$ cạnh trong tứ diện đều, ta có tất cả $6\cdot (18+18)=216$ bộ ba điểm thỏa mãn.\\
    Vì vậy xác suất cần tính là $P=\dfrac{216}{810}=\dfrac{4}{15}\approx 0{,}27$.
}
\end{ex}
\Closesolutionfile{ans}
\inputansbox{6,4,3}{ans/ans-HXN-\sode-T,ans/ans-HXN-\sode-TF,ans/ans-HXN-\sode-SA}
% \def\sode{3}
\begin{name}
	{\tenchude}
	{\tendethi}
	{\tentruong}
	{\thoigian}
\end{name}
\caulc
\Opensolutionfile{ans}[ans/ans-HXN-\sode-T]
%Câu hỏi
\begin{ex}%Câu 1
 Nguyên hàm của hàm số $f(x)=2^x$ là
 \choice
 {$\dfrac{2^{x+1}}{x+1}+C$}
 {\True $\dfrac{2^x}{\ln 2}+C$}
 {$\dfrac{2^x}{x}+C$}
 {$x{2^{x-1}}+C$}
 \loigiai{
 Chọn B.\\
 Ta có $\displaystyle\int\limits_{}^{}{2^x\text{d}x}=\dfrac{2^x}{\ln 2}+C$ .}
\end{ex}
\begin{ex}%Câu 2
 Cho hàm số $y=f(x)$ có bảng biến thiên như sau.
 \begin{center}
 \begin{tikzpicture}[>=stealth]
 \tkzTabInit[nocadre=false,lgt=1,espcl=2.5,deltacl=0.5]{$x$/.7 ,$y'$/.7,$y$/2}
 {$-\infty$ , $-2$ , $3$ , $+\infty$}
 \tkzTabLine{ , - , $0$ , + , $0$ , - , }
 \tkzTabVar{+/$+\infty$ , -/$-3$ , +/$2$ , -/$-\infty$}
 \end{tikzpicture}
 \end{center}
 Giá trị cực đại của hàm số đã cho bằng
 \choice
 {$3$}
 {\True $2$}
 {$-2$}
 {$-3$}
 \loigiai{
 Chọn B.\\
 Dựa vào bảng biến thiên, giá trị cực đại của hàm số là $y_{CT}=2$ .}
\end{ex}
\begin{ex}%Câu 3
 Cho cấp số cộng $\left(u_n\right)$ có $u_2=2;\,\,u_5=11$ . Công sai $d$ của cấp số cộng là
 \choice
 {1}
 {2}
 {4}
 {\True 3}
 \loigiai{
 Chọn D.\\
 Ta có: $\left\{\begin{aligned}
 &{u_2}=2\\ 
 &{u_5}=11\\ 
 \end{aligned}\right.\Leftrightarrow\left\{\begin{aligned}
 &{u_1}+d=2\\ 
 &{u_1}+4d=11\\ 
 \end{aligned}\right.\Leftrightarrow\left\{\begin{aligned}
 &{u_1}=-1\\ 
 & d=3\\ 
 \end{aligned}\right.$ . Vậy công sai của cấp số cộng là $d=3$ .}
\end{ex}
\begin{ex}%Câu 4
 Cho hàm số $y=f(x)$ có đạo hàm $f'(x)=x+1$ với mọi $x\in\mathbb{R}$ . Hàm số đã cho nghịch biến trên khoảng nào dưới đây?
 \choice
 {$\left(-1\,;\,\,+\infty\right)$}
 {$\left(1\,;\,\,+\infty\right)$}
 {\True $\left(-\infty\,;\,\,-1\right)$}
 {$\left(-\infty\,;\,\,1\right)$}
 \loigiai{
 Chọn C.\\
 }
\end{ex}
\begin{ex}%Câu 5
 Trong không gian $Oxyz$ , đường thẳng $d$ đi qua điểm $M\left(1\,;\,\,-1\,;\,\,3\right)$ và song song với đường thẳng$\Delta :\dfrac{x-2}{2}=\dfrac{y+1}{1}=\dfrac{z+3}{-1}$ có phương trình là
 \choice
 {\True $\left\{\begin{aligned}
 & x=1+2t\\ 
 & y=-1+t\\ 
 & z=3-t\\ 
 \end{aligned}\right.$}
 {$\left\{\begin{aligned}
 & x=1+2t\\ 
 & y=-1+t\\ 
 & z=3+t\\ 
 \end{aligned}\right.$}
 {$\left\{\begin{aligned}
 & x=2+t\\ 
 & y=1-t\\ 
 & z=-1+3t\\ 
 \end{aligned}\right.$}
 {$\left\{\begin{aligned}
 & x=1+2t\\ 
 & y=1+t\\ 
 & z=3-t\\ 
 \end{aligned}\right.$}
 \loigiai{
 Chọn A.\\
 Đường thẳng d song song $\Delta :\dfrac{x-2}{2}=\dfrac{y+1}{1}=\dfrac{z+3}{-1}$ nên có vectơ chỉ phương $\vec{u}=\left(2\,;\,\,1\,;\,\,-1\right)$ ; mà d qua $M\left(1\,;\,\,-1\,;\,\,3\right)$ nên có phương trình tham số $\left\{\begin{aligned}
 & x=1+2t\\ 
 & y=-1+t\\ 
 & z=3-t\\ 
 \end{aligned}\right.$ .}
\end{ex}
\begin{ex}%Câu 6
 Tập nghiệm của bất phương trình $\log_{\dfrac{1}{2}}\left(9-x^2\right)<0$ chứa bao nhiêu số nguyên?
 \choice
 {1}
 {\True $5$}
 {$3$}
 {4}
 \loigiai{
 Chọn B.\\
 Ta có: $\log_{\dfrac{1}{2}}\left(9-x^2\right)<0\Leftrightarrow 9-x^2>\left(\dfrac{1}{2}\right)^0\Leftrightarrow 9-x^2>1\Leftrightarrow{x^2}<8\Leftrightarrow-2\sqrt{2}<x<2\sqrt{2}$ .\\
 Tập nghiệm bất phương trình chứa 5 số nguyên là: $-2\,;\,\,1\,;\,\,0\,;\,\,1\,;\,\,2$ .}
\end{ex}
\begin{ex}%Câu 7
 Trong không gian $Oxyz$ , cho hai vectơ $\vec{u}=\left(1\,;\,\,-4\,;\,\,0\right)$ và $\vec{v}=\left(-1\,;\,\,-2\,;\,\,1\right)$ . Vectơ $\vec{u}+3\vec{v}$ có tọa độ là
 \choice
 {\True $\left(-2\,;\,\,-10\,;\,\,3\right)$}
 {$\left(-2\,;\,\,-6\,;\,\,3\right)$}
 {$\left(-4\,;\,\,-8\,;\,\,4\right)$}
 {$\left(-2\,;\,\,-10\,;\,\,-3\right)$}
 \loigiai{
 Chọn A.\\
 Ta có: $\vec{u}+3\vec{v}=\left(1\,;\,\,-4\,;\,\,0\right)+3\left(-1\,;\,\,-2\,;\,\,1\right)=\left(-2\,;\,\,-10\,;\,\,3\right)$ .}
\end{ex}
\begin{ex}%Câu 8
 Tâm đối xứng của đồ thị hàm số $y=\dfrac{3x+1}{x-2}$ có tọa độ là
 \choice
 {$\left(3\,;\,\,-2\right)$}
 {$\left(3\,;\,\,2\right)$}
 {$\left(-2\,;\,\,3\right)$}
 {\True $\left(2\,;\,\,3\right)$}
 \loigiai{
 Chọn D.\\
 Đồ thị hàm số có tiệm cận đứng $x\,=2$ và tiệm cận ngang $y\,=\,3$ .\\
 Do đó tâm đối xứng của đồ thị hàm số là $\left(2\,;\,\,3\right)$ .}
\end{ex}
\begin{ex}%Câu 9
 Cho mẫu số liệu ghép nhóm ở bảng sau. Khoảng tứ phân vị của mẫu số liệu ghép nhóm (làm tròn đến hàng phần trăm) là
 \choice
 {$19,15$}
 {$21,32$}
 {$20,07$}
 {\True $22,23$}
 \loigiai{
 Chọn D.\\
 Tứ phân vị thứ nhất của mẫu số liệu gốc là $\dfrac{x_6+x_7}{2}\in\left[30\,;\,\,40\right)$ nên tứ phân vị thứ nhất của mẫu số liệu ghép nhóm là $Q_1=30+\dfrac{\dfrac{25}{4}-3}{7}\cdot 10=\dfrac{485}{14}$ .\\
 Tứ phân vị thứ ba của mẫu số liệu gốc là $\dfrac{x_{19}+x_{20}}{2}\in\left[50\,;\,\,60\right)$ nên tứ phân vị thứ ba của mẫu số liệu ghép nhóm là $Q_3=50+\dfrac{3\cdot\dfrac{25}{4}-16}{4}\cdot 10=\dfrac{455}{8}$ .\\
 Khoảng tứ phân vị của mẫu số liệu ghép nhóm trên là $Q_3\,-\,Q_1\,=\,\dfrac{1245}{56}\approx\,22,23$ .}
\end{ex}
\begin{ex}%Câu 10
 Cho hàm số $f(x)=x^2+\sin x+1$ . Biết rằng $F(x)$ là một nguyên hàm của hàm số $f(x)$ và thỏa mãn $F(0)=1$ . Khi đó $F(x)$ bằng
 \choice
 {$F(x)=x^3-\cos x+x+2$}
 {\True $F(x)=\dfrac{x^3}{3}-\cos x+x+2$}
 {$F(x)=\dfrac{x^3}{3}+\cos x+x$}
 {$F(x)=\dfrac{x^3}{3}+\cos x+2$}
 \loigiai{
 Chọn B.\\
 Ta có $F(x)=\displaystyle\int{\left(x^2+\sin x+1\right)}\text{d}x=\dfrac{x^3}{3}-\cos x+x+C$ .\\
 Theo giả thiết: $F(0)=1\Rightarrow\dfrac{0^3}{3}-\cos 0+0+C=1\Rightarrow C=2$ .\\
 Do đó $F(x)=\dfrac{x^3}{3}-\cos x+x+2$ .}
\end{ex}
\begin{ex}%Câu 11
 Người ta thống kê lại đường kính thân gỗ của một số cây xoan đào 6 năm tuổi được trồng ở một lâm trường ở bảng sau:\\
 \centerline{\begin{tabular}{|c|c|c|c|c|c|}
 \hline
 Đường kính $(cm)$ & $[40\,;\,\,45)$ & $[45\,;\,\,50)$ & $[50\,;\,\,55)$ & $[55\,;\,\,60)$ & $[60\,;\,\,65)$\\
 \hline
 Tần số & 5 & 20 & 18 & 7 & 3\\
 \hline
 \end{tabular}}\\
 Khoảng biến thiên của mẫu số liệu ghép nhóm trên là
 \choice
 {\True 25}
 {30}
 {6}
 {69,8}
 \loigiai{
 Chọn A.\\
 Khoảng biến thiên của mẫu số liệu ghép nhóm trên là $65-40=25\,\,\,cm.$}
\end{ex}
\begin{ex}%Câu 12
 Cho hình hộp $ABCD.A'{B}'{C}'{D}'$ (xem hình vẽ). Phát biểu nào sau đây là đúng? 
 \choice
 {$\overrightarrow{A{C}'}=\overrightarrow{AB}+\overrightarrow{A{B}'}+\overrightarrow{AD}$}
 {\True $\overrightarrow{D{B}'}=\overrightarrow{DA}+\overrightarrow{D{D}'}+\overrightarrow{DC}$}
 {$\overrightarrow{A{C}'}=\overrightarrow{AC}+\overrightarrow{AB}+\overrightarrow{AD}$}
 {$\overrightarrow{DB}=\overrightarrow{DA}+\overrightarrow{D{D}'}+\overrightarrow{DC}$}
 \loigiai{
 Chọn B.\\
 Theo quy tắc hình hộp ta có$\overrightarrow{D{B}'}=\overrightarrow{DA}+\overrightarrow{D{D}'}+\overrightarrow{DC}$ .
 }
 \end{ex}
\Closesolutionfile{ans}
\cauds
\Opensolutionfile{ans}[ans/ans-HXN-\sode-TF]
%Câu hỏi
\begin{ex}
    \immini[thm]{Xét tam giác $ABC$ có $AC = 2AB$ và $BC = 10\ cm$. Trên cạnh $AC$ lấy điểm $D$ sao cho $AD = \dfrac{1}{4}AC$, trên cạnh $AB$ lấy điểm $E$ sao cho $AE = \dfrac{1}{4}AB$, trên cạnh $AD$ lấy điểm $F$ sao cho $AF = \dfrac{1}{4}AD$ và tiếp tục lấy các điểm $G, H, I, J, \dots$ (vô hạn lần) theo quy luật đó.
        
        \choiceTF
        {\True $\dfrac{AB}{AC} = \dfrac{AD}{AB}$}
        {Tam giác $ABD$ đồng dạng với tam giác $ABC$}
        {$BD = 5\ cm$; $DE = 3\ cm$}
        {\True Độ dài đường gấp khúc $CBDEFGH\dots$ bằng $20\ cm$}}{\includegraphics[scale=1.5]{img/HXN-3.13}}
    \loigiai{
    \begin{itemchoice}
        \itemch Ta có: $AD = \dfrac{1}{4}AC = \dfrac{1}{4} \cdot 2AB = \dfrac{1}{2}AB$; suy ra $\dfrac{AD}{AB} = \dfrac{1}{2}$. $\dfrac{AB}{AC} = \dfrac{AB}{2AB} = \dfrac{1}{2}$. Suy ra $\dfrac{AD}{AB} = \dfrac{AB}{AC}$.
        \itemch Hai tam giác $ABD$ và $ACB$ đồng dạng vì có góc $\hat{A}$ chung và $\dfrac{AB}{AC} = \dfrac{AD}{AB}$.
        \itemch Từ câu b) ta suy ra $\dfrac{AB}{AC} = \dfrac{AD}{AB} = \dfrac{BD}{CB} = \dfrac{1}{2} \Rightarrow BD = \dfrac{1}{2}BC = \dfrac{1}{2} \cdot 10 = 5\ cm$. Hoàn toàn tương tự, ta chứng minh được hai tam giác $ADB$ và $AED$ đồng dạng, suy ra $DE = \dfrac{1}{2}BD = \dfrac{1}{2} \cdot 5 = 2,5\ cm$.
        \itemch Độ dài đường gấp khúc $CBDEFGH\dots$ bằng $l = CB + BD + DE + EF + FG + \dots = 10 + 5 + 2,5 + \dots$ Đây là tổng của một cấp số nhân lùi vô hạn có số hạng đầu $u_1 = 10$, công bội $q = \dfrac{1}{2}$. Do đó $l = \dfrac{u_1}{1-q} = \dfrac{10}{1-\dfrac{1}{2}} = \dfrac{10}{\dfrac{1}{2}} = 20\ cm$.
        \end{itemchoice}}
\end{ex}

\begin{ex}
    \immini
    {
        Xét một hệ trục tọa độ $Oxyz$ được cho sẵn, đơn vị trên mỗi trục là dm, mặt ngoài của một quả bóng được mô hình hóa bởi phương trình mặt cầu $(x-2)^2+(y+1)^2+(z+1)^2=6$, quả bóng nằm yên trên sàn nhà. Người ta nhìn thấy một tấm ván ngã xuống đè lên quả bóng, phần giao của tấm ván và sàn nhà là đường thẳng $d$ có phương trình $\dfrac{x+2}{2}=\dfrac{y+1}{-3}=\dfrac{z}{1}$.  Gọi $A$, $B$ lần lượt là hai tiếp điểm của tấm ván, sàn nhà với quả bóng và $I$ là tâm quả bóng.
        \choiceTF
    {\True Quả bóng có tâm $I(2;-1;-1)$ và bán kính $R=\sqrt{6}$}
    {Khoảng cách từ tâm quả bóng đến đường thẳng d bằng $2\sqrt{6}$}
    {\True Nếu $\cos\widehat{AIB}$ bằng $\dfrac{a}{b}$ (phân số tối giản) thì giá trị $a^2+b^2=82$}
    {\True Một con kiến bò từ vị trí $A$ đến vị trí $B$ trên quả bóng với tốc độ $2$ cm/s; thời gian ngắn nhất cho chuyến đi này là $21$ giây (làm tròn đến hàng đơn vị)}
    }
    {
        \includegraphics[width=5cm]{img/HXN-3-14}
    }
    \loigiai{
        \begin{itemchoice}
            \itemch Mặt ngoài quả bóng là mặt cầu $(S)$ có tâm $I(2;-1;-1)$ và bán kính $R=\sqrt{6}$.
            \itemch Đường thẳng  $d\colon \dfrac{x+2}{2}=\dfrac{y+1}{-3}=\dfrac{z}{1}$ qua $A(-2;-1;0)$ và có VTCP $\vec{u}_d=(2;-3;1)$.\\
            Ta có $\vec{AI}=(4;0;-1);\left[{\vec{u}}_d,\vec{AI}\right]=(3;6;12)$.\\
            Do đó $d\left(I,d\right)=\dfrac{\left| \left[{\vec{u}}_d,\vec{AI}\right] \right|}{\left| {\vec{u}}_d \right|}=\dfrac{\sqrt{3^2+6^2+12^2}}{\sqrt{2^2+(-3)^2+1^2}}=\dfrac{3\sqrt{6}}{2}$.
            \immini
            {
            \itemch Gọi $K$ là hình chiếu của $I$ trên $d$ thì $KI=\dfrac{3\sqrt{6}}{2}$ và $KA\perp IA$;\\
            suy ra $\cos\widehat{AIK}=\dfrac{IA}{IK}=\dfrac{2}{3}$\\
            Do vậy $\cos\widehat{AIB}=2\cos^2\widehat{AIK}-1 =-\dfrac{1}{9}=\dfrac{a}{b}$\\
            $\Rightarrow a^2+b^2=82$.
            \itemch Độ dài cung tròn bé nhất mà con kiến có thể đi: $$l_{\wideparen{AB}}=R\times \widehat{AIB}=\sqrt{6}\times \arccos \left(-\dfrac{1}{9}\right)\approx 4{,}12\,dm$$
            Thời gian tối thiểu để kiến đến nơi là $\dfrac{l_{\wideparen{AB}}\times 10}{2}\approx 21$ giây.
            }
            {
                \includegraphics[width=5cm]{img/HXN-3-14-LG}
            }
        \end{itemchoice}
    }
\end{ex}
\begin{ex}
Thám tử lừng danh Sherlock Holmes đang điều tra một vụ án được thực hiện độc lập bởi một trong hai nghi phạm là McFarlane và Oldacre. Ban đầu thám tử đã có bằng chứng ngang nhau chống lại cả hai người.\\
Trong quá trình điều tra thêm tại hiện trường vụ án, Sherlock Holmes phát hiện rằng thủ phạm có nhóm máu mà chỉ $10\%$ dân số có; và Oldacre có nhóm máu này, còn nhóm máu của McFarlane thì chưa biết.\\
Gọi $A$ là biến cố: \lq\lq McFarlane là thủ phạm\rq\rq; $B$ là biến cố: \rq\rq Oldacre là thủ phạm\rq\rq; $C$ là biến cố: \lq\lq Nhóm máu của nghi phạm trùng với nhóm máu thủ phạm thực sự\rq\rq.
\choiceTF
{Trong quá trình điều tra, nếu Sherlock Holmes biết chắc chắn McFarlane không là  thủ phạm, khi đó xác suất để Oldacre là thủ phạm bằng $0{,}98$}
{$P\left(C|A\right)=0{,}1;P\left(C|B\right)=0{,}5$}
{Dựa trên thông tin về nhóm máu, xác suất để Oldacre là thủ phạm bằng $\dfrac{9}{11}$}
{\True Dựa trên thông tin về nhóm máu, xác suất để McFarlane cũng có nhóm máu trùng với nhóm máu thủ phạm bằng $\dfrac{2}{11}$}
\loigiai{
    \begin{itemchoice}
        \itemch Ta có: $P\left(B|\bar{A}\right)=\dfrac{P\left(B\bar{A}\right)}{P\left({\bar{A}}\right)}=\dfrac{P(B)}{P\left({\bar{A}}\right)}=\dfrac{\dfrac{1}{2}}{\dfrac{1}{2}}=1$. \\
        Như vậy nếu McFarlane không là  thủ phạm thì chắc chắn Oldacre là thủ phạm.
        \itemch Ta có: $P\left(C|A\right)=0{,}1;P\left(C|B\right)=1$.
        \itemch Áp dụng công thức Bayes:$P\left(B\mid C\right)=\dfrac{P(B)\cdot P\left(C\mid B\right)}{P(B)\cdot P\left(C\mid B\right)+P(A)\cdot P\left(C\mid A\right)}=\dfrac{\dfrac{1}{2}\cdot 1}{\dfrac{1}{2}\cdot 1+\dfrac{1}{2}\cdot \dfrac{1}{10}}=\dfrac{10}{11}$.
        \itemch Từ câu c) ta có $P\left(\bar{B}|C\right)=1-P\left(B\mid C\right)=\dfrac{1}{11}$.
        Gọi $D$ là biến cố \lq\lq McFarlane có cùng nhóm máu với thủ phạm biết rằng Oldacre có cùng nhóm máu với thủ phạ\rq\rq.
        Ta có $P(D)=\dfrac{10}{11}\cdot \dfrac{1}{10}+\dfrac{1}{11}\cdot 1=\dfrac{2}{11}$.
        (Nếu Oldacre là thủ phạm, xác suất để McFarlane có cùng nhóm máu với thủ phạm bằng $\dfrac{1}{10}$; nếu Oldacre không là thủ phạm thì xác suất để McFarlane có cùng nhóm máu với thủ phạm bằng $1$).
    \end{itemchoice}
}
\end{ex}
\begin{ex}
\immini
{
    Cho $y=f(x)$, $y=g(x)$ lần lượt là các hàm đa thức bậc ba và bậc nhất có đồ thị như hình vẽ.
    Biết diện tích hình $S$ (được tô màu) bằng $\dfrac{250}{81}$. 
    \choiceTF
    {Hàm số $g(x)=\dfrac{3}{5}x-\dfrac{1}{5}$}
    {\True Hàm số $\displaystyle\int\limits_{-2}^{\tfrac{4}{3}}{\left[f(x)-g(x)\right]\mathrm{\,d}x}=\dfrac{250}{81}$}
    {Hàm số $f(x)=\dfrac{3}{10}(x+2)\left(x-\dfrac{4}{3}\right)(x-3)+\dfrac{3}{5}x+\dfrac{1}{5}$}
    {$\int\limits_0^2{f(x)\mathrm{\,d}x}=\dfrac{37}{15}$}
}
{
    \includegraphics[width=5cm]{img/HXN-3-16}
}
    \loigiai{
        \begin{itemchoice}
            \itemch Ta có $g(x)$ là hàm số bậc nhất đi qua các điểm $A\left(\dfrac{4}{3};1\right)$, $B(3;2)$ nên $g(x)=\dfrac{3}{5}x+\dfrac{1}{5}$.
            \itemch Ta thấy hai đồ thị hàm số $y=f(x),y=g(x)$ cắt nhau tại điểm có $y=-1$; thay vào đường thẳng $y=\dfrac{3}{5}x+\dfrac{1}{5}$  thì $x=-2$.
            Do đó $S=\displaystyle\int\limits_{-2}^{\tfrac{4}{3}}{\left[f(x)-g(x)\right]\mathrm{\,d}x}=\dfrac{250}{81}$.
            \itemch Đặt $f(x)-g(x)=a(x+2)\left(x-\dfrac{4}{3}\right)(x-3)$ với $a>0$.\\
            Ta có: $S=\int\limits_{-2}^{\tfrac{4}{3}}{\left[f(x)-g(x)\right]\mathrm{\,d}x}\Leftrightarrow \displaystyle\int\limits_{-2}^{\tfrac{4}{3}}{\left[a(x+2)\left(x-\dfrac{4}{3}\right)(x-3)\right]\mathrm{\,d}x}=\dfrac{250}{81}\Leftrightarrow a=\dfrac{3}{20}$.\\
            Khi đó $f(x)-g(x)=\dfrac{3}{20}(x+2)\left(x-\dfrac{4}{3}\right)(x-3)\Leftrightarrow f(x)=\dfrac{3}{20}(x+2)\left(x-\dfrac{4}{3}\right)(x-3)+\dfrac{3}{5}x+\dfrac{1}{5}$.
            \itemch $\displaystyle\int\limits_0^2{f(x)\mathrm{\,d}x}=\displaystyle\int\limits_0^2{\left[\dfrac{3}{20}(x+2)\left(x-\dfrac{4}{3}\right)(x-3)+\dfrac{3}{5}x+\dfrac{1}{5}\right]\mathrm{\,d}x}=\dfrac{34}{15}$.
        \end{itemchoice}
    }
\end{ex}
\Closesolutionfile{ans}
\caukq
\Opensolutionfile{ans}[ans/ans-HXN-\sode-SA]
%Câu hỏi
\begin{ex}%Câu 17
 Cho tứ diện đều ABCD có tất cả cạnh bằng 2. Tính khoảng cách của hai đường thẳng chéo nhau AB và CD (làm tròn đến hàng phần trăm).
 \shortans{1,41}
\loigiai{
 \begin{center}
 \includegraphics[scale=1]{img/HXN-3.17a}
 \end{center}
 Gọi I, J theo thứ tự là trung điểm của AB, CD.\\
 Các tam giác ABC, ABD đều có I là trung điểm AB nên\\ $\left\{\begin{aligned}
 & AB\perp CI\\ 
 & AB\perp DI\\ 
 \end{aligned}\right.\Rightarrow AB\perp\left(ICD\right)$ mà $IJ\subset\left(ICD\right)\Rightarrow AB\perp IJ$ (1).\\
 Tương tự, các tam giác ACD, BCD đều có J là trung điểm CD nên\\ $\left\{\begin{aligned}
 & CD\perp AJ\\ 
 & CD\perp BJ\\ 
 \end{aligned}\right.\Rightarrow CD\perp\left(ABJ\right)$ , mà $IJ\subset\left(JAB\right)\Rightarrow CD\perp IJ$ (2).\\
 Từ (1) và (2) suy ra IJ là đoạn vuông góc chung của hai đường thẳng AB, CD.\\
 Ta có: $CI=\dfrac{2\sqrt{3}}{2}=\sqrt{3}$ ; $IJ=\sqrt{C{I^2}-C{J^2}}=\sqrt{3-1}=\sqrt{2}\approx 1,41$ .\\
 Vậy khoảng cách hai đường thẳng AB, CD xấp xỉ 1,41.}
 \end{ex}
 \begin{ex}%Câu 18
Buổi họp mặt cuối năm của VFF diễn ra trong không khí hân hoan phấn khởi sau khi ĐTQG Việt Nam vô địch AFF cup 2024. Vào cuối buổi họp thì HLV Kim Sang-sik chỉ bắt tay với một số người, còn lại tất cả thành viên đều bắt tay với nhau, hai người bất kì thì bắt tay không quá một lần. Hỏi cuộc họp này có bao nhiêu người tham dự biết rằng đã có tổng cộng 2014 cái bắt tay được thực hiện?
\shortans{64}
\loigiai{
Gọi n là người có mặt trong cuộc họp $\left(n\in\mathbb{N}\right)$ .\\
Số cái bắt tay tối đa trong cuộc họp là $C_n^2$ .\\
Trong thực tế thì tổng cộng số cái bắt tay là 2014; vì vậy $C_n^2>2014$\\
$\Rightarrow\dfrac{n!}{2\left(n-2\right)!}>2014\Rightarrow n\left(n-1\right)>4028\Rightarrow{n^2}-n-4028>0\Rightarrow n>63,97$ .
\begin{itemize}
\item Với $n=64$ thì số cái bắt tay tối đa là $C_{64}^2=2\,016$ ; số người mà ông Kim không bắt tay là $2016-2014=2$ (thỏa mãn).\\
\item Với $n=65$ thì số cái bắt tay tối đa là $C_{65}^2=2\,080$ ; số người mà ông Kim không bắt tay là $2080-2014=66$ (vô lí vì trong phòng họp đang có 65 người).\\
\item Ta không cần thử lại với $n>65$ vì luôn xảy ra điều vô lí như trên.
\end{itemize}
Vậy số người tham dự cuộc họp là $64$.}
\end{ex}
\begin{ex}%Câu 19
 \immini[thm]{Hình dáng phần đất liền của hai xã thuộc tỉnh Đồng Tháp được mô hình hóa bởi đồ thị hàm số $y=\dfrac{x^2+ax+b}{x-2}$ ; biết đồ thị có một điểm cực trị là $\left(1\,;\,\,1\right)$ , với hệ trục tọa độ Oxy như hình vẽ, đơn vị trên mỗi trục là 10 mét. Để thuận tiện cho giao thông hai xã, lãnh đạo tỉnh đã phê duyệt dự án xây một chiếc cầu nối phần đất liền của hai xã này. Nhằm tiết kiệm chi phí cho công trình, người kỹ sư trưởng thiết kế có nhiệm vụ nghiên cứu để chọn được hai vị trí A, B trên phần đất liền hai xã sao cho độ dài chiếc cầu (đoạn AB) là ngắn nhất có thể. Hỏi độ dài ngắn nhất của chiếc cầu đó (tính theo đường chim bay) là bao nhiêu mét (làm tròn đến hàng phần chục)?
 \shortans{43,9}}{\includegraphics[scale=1]{img/HXN-3.19}}
\loigiai{
Ta có $y'=\dfrac{x^2-4x-2a-b}{\left(x-2\right)^2}$ .\\
 Vì $\left(1\,;\,\,1\right)$ là điểm cực trị của đồ thị hàm số nên $\left\{\begin{aligned}
& y(1)=1\\ 
&{y}'(1)=0\\ 
\end{aligned}\right.\Rightarrow\left\{\begin{aligned}
& a+b=-2\\ 
& 2a+b=-3\\ 
\end{aligned}\right.\Rightarrow\left\{\begin{aligned}
& a=-1\\ 
& b=-1\\ 
\end{aligned}\right.$ .\\
Hàm số trở thành $y=\dfrac{x^2-x-1}{x-2}=x+1+\dfrac{1}{x-2},\,\,\,x\ne 2$ .\\
Gọi $A\left(2+a\,;\,\,3+a+\dfrac{1}{a}\right)\,,\,\,B\left(2-b\,;\,\,3-b-\dfrac{1}{b}\right)$ là hai điểm thuộc hai nhánh đồ thị với $a\,,\,\,b>0$ .\\
 Ta có: $A{B^2}=\left(a+b\right)^2+\left(a+b+\dfrac{1}{a}+\dfrac{1}{b}\right)^2=\left(a+b\right)^2+\left(a+b+\dfrac{a+b}{ab}\right)^2=\left(a+b\right)^2\left[1+\left(1+\dfrac{1}{ab}\right)^2\right]$\\
$=\left(a+b\right)^2\left(2+\dfrac{2}{ab}+\dfrac{1}{a^2b^2}\right)\overset{AM-GM}{\mathop{\ge}}\,4ab\left(2+\dfrac{2}{ab}+\dfrac{1}{a^2b^2}\right)$ $=8+8ab+\dfrac{4}{ab}\overset{AM-GM}{\mathop{\ge}}\,8+8\sqrt{2}$ .\\
Độ dài ngắn nhất của cây cầu (theo đường chim bay) là $AB\times 10=\sqrt{8+8\sqrt{2}}\times 10\approx\, 43,9 \,m$ .\\
Dấu “=” xảy ra khi và chỉ khi $a=b$ và $8ab=\dfrac{4}{ab}\Leftrightarrow a=b=\dfrac{1}{\sqrt[4]{2}}$ .}
\end{ex}
\begin{ex}%Câu 20
 \immini[thm]{Một cái chậu đựng nước có dạng hình chóp cụt đều đáy là các tam giác cạnh bằng 1 dm và 3 dm. Chiều cao chậu nước bằng 4 dm.
 Người ta bơm nước vào chậu với lưu lượng không đổi 0,5 lít/phút. Đến phút thứ 10 thì tốc độ dâng lên của nước trong chậu là bao nhiêu dm/phút? (Kết quả được làm tròn đến hàng phần trăm).
 \shortans{0,17}}{\includegraphics[scale=.8]{img/HXN-3.20}}
\loigiai{
\begin{center}
 \includegraphics[scale=1]{img/HXN-3.20a}
\end{center}
Gọi MN là độ dài cạnh tam giác đều theo mực nước tức thời (MN thay đổi).\\
Đặt $MN=a\times h+b$ (hàm số bậc nhất theo h).\\
Khi $h=0$ thì $MN=1$ ; khi $h=4$ thì $MN=3$ .\\
Do đó $\left\{\begin{aligned}
& b=1\\ 
& 4a+b=3\\ 
\end{aligned}\right.\Rightarrow\left\{\begin{aligned}
& b=1\\ 
& a=\dfrac{1}{2}\\ 
\end{aligned}\right.$ ; suy ra $MN=\dfrac{1}{2}h+1$ .\\
Diện tích mặt nước tức thời là $S=\dfrac{M{N^2}\sqrt{3}}{4}=\dfrac{\left(0,5h+1\right)^2\sqrt{3}}{4}$ .\\
Thể tích nước tức thời là $V=\dfrac{1}{3}h\left(S_0+\sqrt{S_0S}+S\right)$ ; $S_0$ là diện tích mặt nước ban đầu (đáy nhỏ).\\
$V=\dfrac{1}{3}h\left(\dfrac{\sqrt{3}}{4}+\sqrt{\dfrac{\sqrt{3}}{4}\cdot\dfrac{\left(0,5h+1\right)^2\sqrt{3}}{4}}+\dfrac{\left(0,5h+1\right)^2\sqrt{3}}{4}\right)$ $=\dfrac{h\sqrt{3}}{12}\left[1+\left(1+0,5h\right)+\left(1+0,5h\right)^2\right]$ .\\
$V=\dfrac{h\sqrt{3}}{12}\left(0,25h^2+1,5h+3\right)=\dfrac{\sqrt{3}}{12}\left(0,25h^3+1,5h^2+3h\right)$ (*).\\
Sau 10 phút thì lượng nước trong chậu là $V=0,5\times 10=5\,\,d{m^3}$ .\\
Do đó $\dfrac{\sqrt{3}}{12}\left(0,25h^3+1,5h^2+3h\right)=5\Rightarrow h\approx 3,27\,\,dm$ (Lưu vào A).\\
Từ (*) đạo hàm hai vế theo $t$ ta được:
$$ \dfrac{dV}{dt} = \dfrac{\sqrt{3}}{12} (0,75h^2 + 3h + 3) \cdot \dfrac{dh}{dt} \quad (**) $$
Thay các giá trị $h = A; V=5; \dfrac{dV}{dt}=0,5$ vào (**) ta được:
$$ \dfrac{dh}{dt} \approx 0,17 \ m/\text{phút} $$
Vậy tốc độ dâng lên của nước trong chậu xấp xỉ $0,17 \ m/\text{phút}$.
}
\end{ex}
\begin{ex}%Câu 21
Trong một trận đấu cờ vua của hai kỳ thủ là Lê Quang Liêm và vua cờ Carlsen; mỗi ván cờ luôn có kẻ thắng người thua (vì nếu hai kỳ thủ hòa thì sẽ bốc thăm để chọn người thắng ván đó). Biết rằng trong mỗi ván đấu, xác suất để anh Liêm dành chiến thắng bằng 0,4; xác suất để Carlsen dành chiến thắng bằng 0,6. Mỗi trận thắng được tính 1 điểm cho kỳ thủ, người thua không được điểm nào. Nếu người nào tạo được cách biệt 2 điểm thì sẽ dành chiến thắng chung cuộc. Tính xác suất để Lê Quang Liêm là người chiến thắng sau cùng (làm tròn đến hàng phần trăm).\shortans{0,31}
\loigiai{
Gọi $P(n)$ là xác suất để Lê Quang Liêm dành chiến thắng khi hiệu số điểm của anh so với Carlsen là $n$ điểm. Ta có $n \in \{-2; -1; 0; 1; 2\}$.
Ta cần tính xác suất chiến thắng của anh Liêm từ trạng thái $n=0$.
\\Theo công thức xác suất toàn phần: $P(0) = 0,4 \times P(1) + 0,6 \times P(-1)$ (1).
\\Ta có $P(2)=1$ và $P(-2)=0$; $P(1)=0,4 \times P(2) + 0,6 \times P(0)$ hay $P(1)=0,4+0,6 \times P(0)$ (2);
$P(-1)=0,4 \times P(0) + 0,6 \times P(-2)$ hay $P(-1)=0,4 \times P(0)$ (3).
\\Thay (2) và (3) vào (1): $P(0) = 0,4 [0,4 + 0,6 \times P(0)] + 0,6 \times 0,4 \times P(0) \Rightarrow P(0) = \dfrac{4}{13} \approx 0,31$.
\\Vậy xác suất để Lê Quang Liêm chiến thắng Carlsen là xấp xỉ $0,31$}
\end{ex}
\begin{ex}%Câu 22
Trong không gian Oxyz, cho ba điểm $A\left(0\,;\,3\,;\,-5\right),\,\,B\left(1\,;\,1\,;\,-5\right),\,\,C\left(4\,;\,3\,;\,-1\right)$ và mặt cầu $\left(S_m\right):{x^2}+y^2+z^2+\left(m-2\right)x+4y+\left(m-2\right)z-3=0$ (m là tham số thực). Gọi (T) là tập hợp tất cả điểm cố định mà mặt cầu $\left(S_m\right)$ luôn đi qua với mọi số thực m và M là một điểm di động trên (T) sao cho thể tích tứ diện MABC đạt giá trị lớn nhất $V_{\max}$ . Tính giá trị lớn nhất $V_{\max}$ đó (làm tròn đến hàng phần chục).
\shortans{15,3}
\loigiai{
 \begin{center}
 \includegraphics[scale=1]{img/HXN-3.22}
 \end{center}
Xét $M(x;y;z)$ là điểm mà $(S_m)$ luôn đi qua với mọi $m$.
\\Ta có: $x^2 + y^2 + z^2 + (m-2)x + 4y + (m-2)z - 3 = 0, \forall m \in \mathbb{R}$\\
$\Leftrightarrow m(x+z) + x^2 + y^2 + z^2 - 2x + 4y - 2z - 3 = 0, \forall m \in \mathbb{R}$\\
$\Leftrightarrow \begin{cases} x+z = 0 \\ x^2 + y^2 + z^2 - 2x + 4y - 2z - 3 = 0 \end{cases}$
\\Tập hợp điểm $M$ là đường tròn $(C)$ là giao tuyến của mặt phẳng $(P): x+z = 0$ và mặt cầu có tâm $I(1;-2;1)$, bán kính $R=3$;
\\Ta có: $d(I,(P)) = \dfrac{|1 \cdot 1 + 0 \cdot (-2) + 1 \cdot 1|}{\sqrt{1^2+0^2+1^2}} = \dfrac{|1+1|}{\sqrt{2}} = \dfrac{2}{\sqrt{2}} = \sqrt{2}$.
\\Vì vậy $(C)$ có bán kính là $r = \sqrt{R^2 - d^2(I,(P))} = \sqrt{7}$ và tâm $J(0;-2;0)$.
\\Mặt phẳng $(ABC)$ có phương trình $2x+y-2z-13=0$; ta thấy $(ABC)$ vuông góc với $(P)$.\\
$ \Rightarrow [\vec{AB},\vec{AC}] = (-8;-4;8) \Rightarrow S_{ABC} = \dfrac{1}{2}\sqrt{(-8)^2+(-4)^2+8^2} = 6$.
\\Thể tích tứ diện $MABC$ là $V_{MABC} = \dfrac{1}{3}d(M,(ABC)) \cdot S_{ABC} = \dfrac{1}{3}d(M,(ABC)) \cdot 6 = 2d(M,(ABC))$.
\\Thể tích này lớn nhất khi và chỉ khi $d(M,(ABC))$ đạt giá trị lớn nhất.
\\Ta có: $d(M,(ABC))_{\max} = r + d(J,(ABC)) = \sqrt{7} + \dfrac{|2(0) + 1(-2) - 2(0) - 13|}{\sqrt{2^2+1^2+(-2)^2}} = \sqrt{7} + 5$.
\\Vì vậy $V_{\max} = 2\sqrt{7} + 10 \approx 15,3$.}
\end{ex}
\Closesolutionfile{ans}
\inputansbox{6,4,3}{ans/ans-HXN-\sode-T,ans/ans-HXN-\sode-TF,ans/ans-HXN-\sode-SA}
% \def\sode{4}
\begin{name}
	{\tenchude}
	{\tendethi}
	{\tentruong}
	{\thoigian}
\end{name}
\Opensolutionfile{ans}[ans/ans-HXN-\sode-T]
\caulc
\begin{ex}%Câu 1
	Cho hàm số $y=f(x)$ xác định trên $\mathbb{R}$ và có bảng biến thiên như hình vẽ sau:
	\begin{center}
		\begin{tikzpicture}[>=stealth]
			\tkzTabInit[nocadre=false,lgt=1,espcl=2.5,deltacl=0.5]{$x$/.7 ,$y'$/.7,$y$/2}
			{$-\infty$ , $-1$ , $2$ , $+\infty$}
			\tkzTabLine{ , - , $0$ , + , $0$ , - , }
			\tkzTabVar{+/$+\infty$ , -/$-2$ , +/$4$ , -/$-\infty$}
		\end{tikzpicture}
	\end{center}
	Hàm số $y=f(x)$ đồng biến trên khoảng nào?
	\choice
	{$\left(-\infty\,;\,\,-1\right)$}
	{$\left(-2\,;\,\,4\right)$}
	{$\left(2\,;\,\,+\infty\right)$}
	{\True $\left(-1\,;\,\,2\right)$}
	\loigiai{
		Chọn D.}
\end{ex}
\begin{ex}%Câu 2
	Trong không gian $Oxyz$ , cho điểm $ M\left(2\,;\,\,-1\,;\,\,4\right)$ và mặt phẳng $(P):\,3x-2y+z+1=0$. Phương trình của mặt phẳng đi qua $ M$ và song song với mặt phẳng $(P)$ là
	\choice
	{$ 2x-2y+4z-21=0$}
	{$ x-2z+1=0$}
	{$10x+9y+5z-74=0$}
	{\True $ 3x-2y+z-12=0$}
	\loigiai{
		Chọn D.\\
		Phương trình của mặt phẳng đi qua $ M$ và song song với mặt phẳng $(P)$ là\\
		$ 3\left(x-2\right)-2\left(y+1\right)+\left(z-4\right)=0$$\Leftrightarrow 3x-2y+z-12=0$.}
\end{ex}
\begin{ex}%Câu 3
	Điều tra về mức lương khởi điểm (đơn vị: triệu đồng) của $ 20$ công nhân, ta có bảng số liệu sau\\
	\centerline{\begin{tabular}{|c|c|c|c|c|c|}
			\hline
			Mức lương & $\left[5\,;\,\,6\right)$ & $\left[6\,;\,\,7\right)$ & $\left[7\,;\,\,8\right)$ & $\left[8\,;\,\,9\right)$ & $\left[9\,;\,\,10\right)$ \\
			\hline
			Tần số    & $ 4$                     & $ 5$                     & $ 5$                     & $ 4$                     & $ 2$                      \\
			\hline
		\end{tabular}}\\
	Phương sai của mẫu số liệu ghép nhóm là (làm tròn đến hàng phần trăm):
	\choice
	{$s^2=0,63$}
	{$s^2=2,52$}
	{$s^2=1,26$}
	{\True $s^2=1,59$}
	\loigiai{
		Chọn D.\\
		Ta có bảng thống kê về mức lương theo giá trị đại diện như sau:\\
		\centerline{\begin{tabular}{|c|c|c|c|c|c|}
				\hline
				Mức lương        & $\left[5\,;\,\,6\right)$ & $\left[6\,;\,\,7\right)$ & $\left[7\,;\,\,8\right)$ & $\left[8\,;\,\,9\right)$ & $\left[9\,;\,\,10\right)$ \\
				\hline
				Giá trị đại diện & $ 5,5$                   & $ 6,5$                   & $ 7,5$                   & $ 8,5$                   & $ 9,5$                    \\
				\hline
				Tần số           & $ 4$                     & $ 5$                     & $ 5$                     & $ 4$                     & $ 2$                      \\
				\hline
			\end{tabular}}\\
		Số trung bình của mẫu số liệu ghép nhóm là\\
		$\bar{x}=\dfrac{4\cdot 5,5+5\cdot 6,5+5\cdot 7,5+4\cdot 8,5+2\cdot 9,5}{20}=7,25$.\\
		Phương sai của mẫu số liệu ghép nhóm là $$s^2=\dfrac{4\cdot{\left(5,5-7,25\right)^2}+5\cdot{\left(6,5-7,25\right)^2}+5\cdot{\left(7,5-7,25\right)^2}+4\cdot{\left(8,5-7,25\right)^2}+2\cdot{\left(9,5-7,25\right)^2}}{20}\approx 1,59$$.}
\end{ex}
\begin{ex}%Câu 4
	Hàm số nào sau đây không là một nguyên hàm của hàm số $f(x)=3x^2+2x-1$ .
	\choice
	{\True $F(x)=x^3+x^2-1$}
	{$F(x)=x^3+x^2-x$}
	{$F(x)=x^3+x^2-x+2025$}
	{$F(x)=x^3+x^2-x-1$}
	\loigiai{
		Chọn A.\\
		$F(x)=\displaystyle\int{f(x)\text{d}x}=\displaystyle\int{\left(3x^2+2x-1\right)\text{d}x}=x^3+x^2-x+C$ (Với $C$ là một hằng số).}
\end{ex}
\begin{ex}%Câu 5
	Trong không gian $ Oxyz$cho ba điểm $ M\left(1\,;\,\,1\,;\,\,1\right),\,\,N\left(2\,;\,\,3\,;\,\,4\right),\,\,P\left(7\,;\,\,7\,;\,\,5\right)$. Tìm tọa độ điểm $ Q$ để tứ giác $ MNPQ$ là hình bình hành
	\choice
	{\True $ Q\left(6\,;\,\,5\,;\,\,2\right)$}
	{$ Q\left(-6\,;\,\,-5\,;\,\,-2\right)$}
	{$ Q\left(-2\,;\,\,-3\,;\,\,-4\right)$}
	{$Q\left(-4\,;\,\,-3\,;\,\,0\right)$}
	\loigiai{
		Chọn A.\\
		Ta có $\overrightarrow{MN}=\left(1\,;\,\,2\,;\,\,3\right)\,,\,\,\overrightarrow{QP}=\left(7-x_Q\,;\,\,7-y_Q\,;\,\,5-z_Q\right)$.\\
		$ MNPQ$ là hình bình hành $\Leftrightarrow\overrightarrow{MN}=\overrightarrow{QP}$$\Leftrightarrow\left\{\begin{aligned}
			 & 1=7-x_Q \\
			 & 2=7-y_Q \\
			 & 3=5-z_Q \\
		\end{aligned}\right.\Leftrightarrow\left\{\begin{aligned}
			 & {x_Q}=6 \\
			 & {y_Q}=5 \\
			 & {z_Q}=2 \\
		\end{aligned}\right.$. Vậy $ Q(6;5;2)$.}
\end{ex}
\begin{ex}%Câu 6
	Cho cấp số cộng $\left(u_n\right)$ có số hạng đầu $u_1=\dfrac{1}{4}$ và công sai $d=-\dfrac{1}{4}$ . Tổng 5 số hạng đầu tiên của cấp số cộng là
	\choice
	{$S_5=\dfrac{5}{4}$}
	{$S_5=\dfrac{4}{5}$}
	{\True $S_5=-\dfrac{5}{4}$}
	{$S_5=-\dfrac{4}{5}$}
	\loigiai{
		Chọn C.\\
		Sử dụng công thức $S_n=\dfrac{n\left[2u_1+\left(n-1\right)d\right]}{2}$ , ta có: $S_5=\dfrac{5\left[2\cdot\dfrac{1}{4}+\left(5-1\right)\cdot\left(-\dfrac{1}{4}\right)\right]}{2}=-\dfrac{5}{4}$ .}
\end{ex}
\begin{ex}%Câu 7
	Nghiệm của phương trình $3^x=10$ là
	\choice
	{$ x=\dfrac{10}{3}$}
	{\True $ x=\log_310$}
	{$ x=\log_{10}3$}
	{$ x=\dfrac{10}{3}$}
	\loigiai{
		Chọn B.\\
		Ta có $3^x=10\Leftrightarrow x=\log_310$.}
\end{ex}
\begin{ex}%Câu 8
	Cho hình chóp $ S.ABCD$ có đáy $ ABCD$ là hình vuông và $ SA\perp\left(ABCD\right)$. Mặt phẳng $\left(SBC\right)$ vuông góc với mặt phẳng nào sau đây?
	\choice
	{$\left(SCD\right)$}
	{$\left(ABCD\right)$}
	{\True $\left(SAB\right)$}
	{$\left(SBD\right)$}
	\loigiai{
		Chọn C.\\
		Ta có $\left\{\begin{aligned}
				 & BC\perp AB \\
				 & BC\perp SA \\
			\end{aligned}\right.$nên $ BC\perp\left(SAB\right)$;\\
		mà $ BC\subset\left(SBC\right)\Rightarrow (SAB) \perp (SBC) $.}
\end{ex}
\begin{ex}%Câu 9
	Đường tiệm cận xiên của đồ thị hàm số $ y=\dfrac{x^2-3x+4}{x+2}$ là đường thẳng
	\choice
	{$ y=-x+1$}
	{$ y=x-1$}
	{\True $ y=x-5$}
	{$ y=-x-5$}
	\loigiai{
		Chọn C.\\
		Ta có $ y=\dfrac{x^2-3x+4}{x+2}$$=x-5+\dfrac{14}{x+2}$; mặt khác $\underset{x\to+\infty}{\lim}\,\left[y-\left(x-5\right)\right]=\underset{x\to+\infty}{\lim}\,\dfrac{14}{x+2}=0$.\\
			Do đó tiệm cận xiên của đồ thị hàm số là đường thẳng $y=x-5 $.}
\end{ex}
\begin{ex}%Câu 10
	Cho tứ diện $ABCD$ . Gọi $M\,,\,\,N$ lần lượt là trung điểm của $AD$ và $BC$ . Tổng $\overrightarrow{AB}+\overrightarrow{DC}$ bằng
	\choice
	{$\vec{0}$}
	{$ 2\overrightarrow{AD}$}
	{$ 2\overrightarrow{NM}$}
	{\True $ 2\overrightarrow{MN}$}
	\loigiai{
		Chọn D.\\
		Ta có: $\overrightarrow{AB}+\overrightarrow{DC}=\overrightarrow{AM}+\overrightarrow{MN}+\overrightarrow{NB}+\overrightarrow{DM}+\overrightarrow{MN}+\overrightarrow{NC}$ $=\left(\overrightarrow{AM}+\overrightarrow{DM}\right)+2\overrightarrow{MN}+\left(\overrightarrow{NB}+\overrightarrow{NC}\right)=2\vec{MN}.$\\
		(Vì $M,\,N$ lần lượt là trung điểm của $AD$ và $BC$ nên $\overrightarrow{AM}+\overrightarrow{DM}=\overrightarrow{0}\,,\,\,\overrightarrow{NB}+\overrightarrow{NC}=\overrightarrow{0}$).}
\end{ex}
\begin{ex}%Câu 11
	Tiếp tuyến của đồ thị hàm số$ y=x^3-3x^2-2$ có hệ số góc $ k=-3$ có phương trình là
	\choice
	{$ y=-3x-7$}
	{$ y=-3x+7$}
	{$ y=-3x+1$}
	{\True $ y=-3x-1$}
	\loigiai{
		Chọn D\\
		Đạo hàm $y'=3x^2-6x$. Gọi $\left(x_0\,;\,\,y_0\right)$ là tiếp điểm của tiếp tuyến với đồ thị hàm số.\\
		Hệ số góc tiếp tuyến $ k=-3$ nên $ 3x_0^2-6x_0=-3\Rightarrow x_0^2-2x_0+1=0\Rightarrow{x_0}=1\Rightarrow{y_0}=-4$.\\
		Phương trình tiếp tuyến của đồ thị hàm số là: $ y=-3\left(x-1\right)-4\,\,\,\text{hay} y=-3x-1$$ $.}
\end{ex}
\begin{ex}%Câu 12
	\immini[thm]{ Đường gấp khúc trong hình vẽ là đồ thì hàm số $ y=f(x)$ trên đoạn $\left[-2\,;\,\,3\right]$. Tích phân $\displaystyle\int\limits_{-2}^3f(x)\text{d}x$ bằng
		\choice
		{\True $\dfrac{13}{2}$}
		{$\dfrac{17}{2}$}
		{$\dfrac{15}{2}$}
		{$\dfrac{5}{2}$}}{\includegraphics[scale=.8]{img/HXN-4.12}}
	\loigiai{
	Chọn A.\\
	Giá trị tích phân $\displaystyle\int\limits_{-2}^3f(x)\text{d}x$ chính là tổng diện tích hai tam giác ABC và CDE như hình vẽ.\\
	Ta có: $\displaystyle\int\limits_{-2}^3f(x)\text{d}x=S_{ABK}+S_{BCL}=\dfrac{1}{2}\cdot 3\cdot 3+\dfrac{1}{2}\cdot 2\cdot 2=\dfrac{13}{2}$.}
\end{ex}

\Closesolutionfile{ans}
\cauds
\Opensolutionfile{ans}[ans/ans-HXN-\sode-TF]


\begin{ex}
	\immini[thm]{Một mô hình trò chơi vòng quay ở công viên có chiều cao tối đa 23 $m$ so với mặt đất, bán kính vòng quay là 20 $m$. Hai bạn Hoa và Mai cùng chơi chung lượt quay và ngồi trong hai cabin B, C mà góc $BOC = 90^\circ$ (hình vẽ); $\alpha$ là góc lượng giác hợp bởi tia đầu OA, tia cuối OB.
		Xét tính đúng sai các mệnh đề sau:}{\includegraphics[scale=.5]{img/HXN-4.13}}
	\choiceTF
	{\True Chiều cao của B so với mặt đất là $h_B = 23 + 20\sin \alpha$ ($mét$)}
	{\True Khi $\alpha = 45^\circ$ thì chiều cao của B so với mặt đất là $37,14$ $m$ (làm tròn kết quả đến hàng phần trăm)}
	{\True Chiều cao của C so với mặt đất là $h_C = 23 - 20\cos \alpha$ ($mét$)}
	{ Khi B ở vị trí có độ cao 33 $m$ thì C ở độ cao 13 $m$ so với mặt đất?}
	\loigiai{
	\begin{itemchoice}
		\itemch Chiều cao của B là $h_B = 23 + 20\sin \alpha$ ($mét$).
		\itemch Với $\alpha = 45^\circ$ thì $h_B = 23 + 20\sin 45^\circ \approx 37,14$ $m$.
		\itemch Chiều cao của C là $h_C = 23 + 20\sin (OA, OC) = 23 + 20\sin (\alpha - 90^\circ) = 23 - 20\cos \alpha$ ($mét$).
		\itemch Khi B ở vị trí có độ cao 33 $m$ thì $h_B = 23 + 20\sin \alpha = 33 \Rightarrow \sin \alpha = \dfrac{1}{2}$.\\
		Khi đó $\cos \alpha = \pm \sqrt{1-\sin^2 \alpha} = \pm \dfrac{\sqrt{3}}{2}$.\\
		Do đó $h_C = 23 - 20\cos \alpha =23\pm 10\sqrt{3}$.
	\end{itemchoice}
	}
\end{ex}
\begin{ex}
	Song sinh có thể là cùng trứng (identical) hoặc khác trứng (fraternal). Biết rằng $\dfrac{1}{3}$ số cặp song sinh là cùng trứng. Hiển nhiên, song sinh cùng trứng phải cùng giới tính; song sinh khác trứng có thể cùng hoặc khác giới tính. Giả sử song sinh cùng trứng có xác suất là hai bé trai hoặc hai bé gái như nhau, trong khi với song sinh khác trứng thì tất cả bốn khả năng đều có xác suất như nhau. Một nhà khảo sát tìm gặp ngẫu nhiên một người phụ nữ đang mang thai đôi.
	Xét tính đúng sai các mệnh đề sau:
	\choiceTF
	{\True Xác suất để người phụ nữ mang thai đôi là bé gái bằng $0,5$ biết rằng đây là cặp song sinh cùng trứng}
	{ Xác suất để thai đôi của người phụ nữ là một cặp trai gái bằng $0,3$}
	{Xác suất để thai đôi không cùng trứng và cũng không phải con trai bằng $\dfrac{1}{3}$}
	{\True Xác suất để người phụ nữ mang thai đôi là cùng trứng bằng $0,5$ biết rằng cô ấy hạ sinh được hai bé gái}
	\loigiai{
		Gọi A là biến cố: “Người phụ nữ mang thai đôi cùng trứng”, ký hiệu BB, GG, BG, GB lần lượt chỉ các biến cố thai đôi là trai-trai, gái-gái, trai-gái, gái trai. Ta có sơ đồ hình cây sau:
        \begin{center}
			\includegraphics[scale=.8]{img/HXN-4.14a}
		\end{center}
        \begin{itemchoice}
            \itemch Mệnh đề đúng. Ta có $P(GG|A) = \dfrac{1}{2}$.
			\itemch Mệnh đề sai. Ta có $P(BG \cup GB | \bar{A}) = \dfrac{2}{3} (\dfrac{1}{4} + \dfrac{1}{4}) = \dfrac{1}{3}$.
			\itemch Mệnh đề sai. Ta có $P(\bar{A} \cap GG) = \dfrac{2}{3} \cdot \dfrac{1}{4} = \dfrac{1}{6}$.
			\itemch Mệnh đề đúng. Ta có $P(GG) = \dfrac{1}{3} \cdot \dfrac{1}{2} + \dfrac{2}{3} \cdot \dfrac{1}{4}= \dfrac{1}{3}$.\\
			Do đó $P(A|GG) = \dfrac{P(A) \cdot P(GG|A)}{P(GG)} = \dfrac{\dfrac{1}{3} \cdot \dfrac{1}{2}}{\dfrac{1}{3}}= 0,5$.
        \end{itemchoice}
	}
\end{ex}
\begin{ex}
	\immini[thm]{ Một bể bơi hình trụ có đường kính $5$m và chiều cao $1$ m; bể được bơm nước vào với tốc độ không đổi $v_0$. Sau khi nước được bơm đầy, bể bị thủng một lỗ ở đáy và nước chảy ra ngoài; bể chảy hết nước trong $8$ giờ. Biết tốc độ giảm chiều cao của bể khi nước chảy ra ngoài vào thời điểm $t$ giờ (tính từ lúc nước đầy bể và ngừng bơm) được cho bởi hàm số $h'(t) = at + b$, với $a, b \in \mathbb{R}$. Lúc nước chảy hết ra ngoài thì tốc độ giảm chiều cao bằng $0$.
		Xét tính đúng sai các mệnh đề sau:}{\includegraphics[scale=.8]{img/HXN-4.15}}
	\choiceTF
	{\True Thể tích của bể bơi sau khi nước được làm đầy là $6,25\pi$ m$^3$}
	{$32a + 1 = 0$ và $4b - 1 = 0$}
	{\True Sau $4$ giờ kể từ lúc bể bị rò, lượng nước bị mất đi bằng $\dfrac{75\pi}{16}$ m$^3$}
	{\True Lượng nước bị rò rỉ ra ngoài một nửa sau $8 - 4\sqrt{2}$ giờ}
	\loigiai{
		Bể nước hình trụ có bán kính đáy $r = 2,5\ m$, chiều cao $h=1\ m$.
		\begin{itemchoice}
			\itemch Thể tích khi đầy là $V = \pi r^2 h = \pi \cdot (2,5)^2 \cdot 1 = 6,25\pi\ m^3$.
			\itemch Ta có $h'(8)=0 \Rightarrow 8a+b=0\ (1)$.\\ Chiều cao của nước thời điểm $t$ là $h(t) = \int\limits_{ }^{ }(at+b)\mathrm{d} t = \dfrac{at^2}{2} + bt + c$. \\Vì $h(0)=1 \Rightarrow c=1$. $h(8)=0 \Rightarrow 32a+8b+1=0\ (2)$. \\Từ (1) và (2) suy ra $a = \dfrac{1}{32}$, $b = -\dfrac{1}{4}$. Khi đó $32a-1=0$; $4b+1=0$ và $h(t) = \dfrac{1}{64}t^2 - \dfrac{1}{4}t + 1$.
			\itemch Chiều cao mực nước trong bể sau 4 giờ là: $h(4) = \dfrac{1}{64} \cdot 4^2 - \dfrac{1}{4} \cdot 4 + 1 = \dfrac{16}{64} - 1 + 1 = \dfrac{1}{4} = 0,25\ m$. Lượng nước còn lại trong bể sau 4 giờ là $\pi r^2 h(4) = \pi \cdot (2,5)^2 \cdot 0,25 = 6,25\pi \cdot 0,25 = 6,25\pi \cdot \dfrac{1}{4} = \dfrac{25\pi}{16}\ m^3$. Lượng nước đã thoát ra sau 4 giờ là $6,25\pi - \dfrac{25\pi}{16} = \dfrac{100\pi}{16} - \dfrac{25\pi}{16} = \dfrac{75\pi}{16}\ m^3$.
			\itemch Lượng nước còn lại khi bể mất một nửa nước là $\dfrac{6,25\pi}{2} = \dfrac{25\pi}{8}\ m^3$. \\Chiều cao tương ứng $h(t_1)$ của bể thỏa mãn $\pi r^2 h(t_1) = \dfrac{25\pi}{8} \Rightarrow h(t_1) = 0,5\ m$.\\ Ta có $h(t_1) = \dfrac{1}{64}t_1^2 - \dfrac{1}{4}t_1 + 1 = 0,5 \Leftrightarrow \dfrac{1}{64}t_1^2 - \dfrac{1}{4}t_1 + 0,5 = 0$. $\left[ \begin{array}{l} t_1 = 8 + 4\sqrt{2} \approx 13,7 > 8 \\ t_1 = 8 - 4\sqrt{2} \approx 2,3 \in (0; 8) \end{array} \right.$. \\Ta thấy $t_1 = 8 - 4\sqrt{2}$ (giờ) thỏa mãn đề bài.
		\end{itemchoice}
	}
\end{ex}
\begin{ex}
	Trong một mô hình game 3D, với hệ trục tọa độ thích hợp, người chơi đứng cùng với khẩu súng của anh ta được mô phỏng như một chất điểm di chuyển trên mặt phẳng $(P): x - 2y + 2z - 3 = 0$ và nhắm bắn các mục tiêu di động trên mặt cầu $(S)$ có phương trình $x^2 + y^2 + z^2 +2x - 4y - 2z + 5 = 0$. Người chơi vẫn có thể bắn trúng mục tiêu nếu nó di chuyển trên bán cầu khuất phía sau tầm nhìn. Sau khi trò chơi bắt đầu, anh ta quyết định nhắm bắn theo phương vector $\vec{u} = (1; 0; 1)$.
	Xét tính đúng sai các mệnh đề sau:
	\choiceTF
	{\True Mặt cầu $(S)$ có tâm $I(-1; 2; 1)$ và bán kính $R = 1$}
	{\True Mặt phẳng $(P)$ và mặt cầu $(S)$ không có điểm chung}
	{\True Người chơi đứng ở vị trí giao điểm của $(P)$ và $Ox$, khoảng cách từ tâm quả cầu đến đường bay viên đạn bằng $\dfrac{\sqrt{66}}{2}$}
	{\True Khoảng cách lớn nhất từ vị trí người bắn đến mục tiêu bằng $3\sqrt{2}$}
	\loigiai{
		\begin{enumerate}
			\itemch Mệnh đề đúng. Mặt cầu $(S)$ có tâm $I(-1; 2; 1)$ và bán kính $R = 1$.
			\itemch Mệnh đề đúng. Mặt phẳng $(P)$ và mặt cầu $(S)$ không có điểm chung.\\
			      Ta có $d(I, (P)) = \dfrac{|-1-4+2-3|}{\sqrt{1+4+4}} = 2$. $2 \ge R$. Do đó $(P)$ và mặt cầu $(S)$ không có điểm chung.
			\itemch Mệnh đề đúng. \\
			      Giao điểm của $(P)$ và $Ox$ là điểm $M(0; 3; 0) \Rightarrow \vec{IM} = (4; -2; -1)$; $[\vec{IM}, \vec{u}] = (-2; -5; 2)$.\\
			      Khoảng cách từ $I$ đến đường bay viên đạn là $d = \dfrac{|[\vec{IM}, \vec{u}]|}{|\vec{u}|} = \dfrac{\sqrt{(-2)^2 + (-5)^2 + 2^2}}{\sqrt{1^2 + 0^2 + 1^2}} =\dfrac{\sqrt{66}}{2}$.
			\itemch Mệnh đề đúng. Gọi $M$ thuộc $(P)$, $N$ thuộc $(S)$ theo thứ tự là vị trí người chơi và vị trí mục tiêu đang bắn; $H$ là hình chiếu của điểm $N$ trên $(P)$.\\
			      \begin{center}
				      \includegraphics[scale=1]{img/HXN-4.16a}
			      \end{center}
			      $MN$ hợp với $(P)$ một góc $\phi$ thỏa mãn $\sin \phi = \dfrac{|\vec{u}\cdot\vec{n_P}|}{|\vec{u}|\cdot|\vec{n_P}|} = \dfrac{|1+0+2|}{\sqrt{1+0+1}\cdot\sqrt{1+4+4}} = \dfrac{\sqrt{2}}{2}$.\\
			      Xét tam giác $MNH$ vuông tại $H$, ta có
			      $\sin \varphi = \dfrac{NH}{MN} \Rightarrow MN = \dfrac{NH}{\sin \varphi}$ hay $\boxed{MN = \sqrt{2}NH}$.
			      Dễ thấy $MN$ lớn nhất khi và chỉ khi $NH$ lớn nhất; mà
			      $NH \le d(I, (P)) + R = 2+1=3$.
			      Do đó $MN$ lớn nhất bằng $\boxed{3\sqrt{2}}$; khi đó $N, I, H$ nằm trên đường thẳng vuông góc với $(P)$.
		\end{enumerate}
	}
\end{ex}

\Closesolutionfile{ans}
\caukq
\Opensolutionfile{ans}[ans/ans-HXN-\sode-SA]


\begin{ex}
	\immini[thm]{ Một tấm cầu dốc kê bậc thềm được làm bằng kim loại như hình vẽ. Biết chiều cao tối đa của cầu dốc là $0,3 \, m$ và bề mặt cầu là hình vuông có cạnh bằng $1 \, m$. Hãy tính góc tạo bởi đường chéo bề mặt cầu dốc với mặt phẳng sàn nhà theo đơn vị độ (làm tròn kết quả đến hàng phần chục).
		\shortans{12,2}}{\includegraphics[scale=.8]{img/HXN-4.17}}
	\loigiai{
		Xét mô hình cầu dốc với các kí hiệu như hình vẽ.
		Vì $AF$ là hình chiếu của $BF$ trên mặt phẳng $(ACFD)$ nên góc giữa $(BF, (ACFD))$ là góc $\angle BFA$.
		Hình vuông $BCFE$ có cạnh bằng $1 \, m$ nên đường chéo $BF = \sqrt{1^2 + 1^2} = \sqrt{2} \, m$; $AB = 0,3 \, m$.
		Tam giác $ABF$ vuông tại $A$ có:
		$$ \sin \angle BFA = \frac{AB}{BF} = \frac{0,3}{\sqrt{2}} = \frac{0,3 \sqrt{2}}{2} = \frac{3 \sqrt{2}}{20} $$
		$$ \Rightarrow \angle BFA \approx 12,2^\circ $$
		Vậy góc giữa $(BF, (ACFD)) = \angle BFA \approx 12,2^\circ$.
	}
\end{ex}
\begin{ex}%Câu 1
	Có hai vợ chồng đã nghĩ ra một trò chơi đầy trí tuệ như sau: Họ sử dụng hai ly nước giống hệt nhau, mỗi ly chứa tối đa 240 ml nước. Ban đầu, người vợ có một ly nước đầy và người chồng có một cái ly rỗng. Bước thứ nhất người vợ rót 1/2 lượng nước trong ly của mình sang ly của người chồng; bước tiếp theo người chồng lại rót 1/3 lượng nước trong ly của mình sang cho ly người vợ. Quá trình này cứ tiếp tục mà mỗi lần rót thì mẫu số được cộng thêm 1; trò chơi này hấp dẫn đến mức cả hai người thực hiện đến bước thứ 100 thì dừng lại, hỏi lượng nước trong ly người chồng khi đó là bao nhiêu ml? (Làm tròn kết quả đến hàng đơn vị, giả sử trong quá trình rót nước không có giọt nước nào tràn ra ngoài).
	\shortans{119}
	\loigiai{
		Ta có thể thực hiện việc rót theo sơ đồ sau:
		\begin{center}
			\includegraphics[scale=.5]{img/HXN-4.18a}
		\end{center}
		Quá trình này được lặp đi lặp lại và ta thấy rằng trong các bước lẻ (người vợ rót nước cho người chồng) thì lượng nước hai ly bằng nhau.\\
		• Bước thứ 99 thì lượng nước hai ly bằng nhau.\\
		• Bước thứ 100 (người chồng rót 1/101 nước trong ly cho vợ), lượng nước trong ly người chồng là $\dfrac{1}{2}V-\dfrac{1}{101}\cdot\dfrac{1}{2}V=\dfrac{50V}{101}=\dfrac{50\cdot 240}{101}\approx\,119\,ml$ .}
\end{ex}
\begin{ex}%Câu 2
	Cho tập hợp $X=\left\{ 3\,;\,\,4\,;\,\,5\,;\,\,6\right\}$ và Y là tập hợp tất cả số tự nhiên có 2025 chữ số lấy từ X. Chọn ngẫu nhiên một số trong tập Y, biết rằng xác suất để số đó chia hết cho 3 bằng $\dfrac{1}{3}\left(\dfrac{1}{2^a}+b\right)$ , trong đó $a\,,\,\,b$ là các số nguyên dương. Tính giá trị $a-18b$ .
	\shortans{4031}
	\loigiai{
	Gọi $A_n, B_n$ là tập con của $Y$ gồm các số có $n$ chữ số với $A_n$ là tập các số chia hết cho 3 và $B_n$ là tập các số không chia hết cho 3.
	\begin{itemize}
		\item Với mỗi số thuộc $A_n$, có hai cách thêm vào cuối một chữ số 3 hoặc 6 để được $A_{n+1}$ và hai cách thêm vào cuối chữ số 4 hoặc chữ số 5 để được $B_{n+1}$.
		      \begin{itemize}
			      \item Ví dụ: $36 \in A_2$, nếu ta thêm 3 hoặc 6 vào sau nó thì được 363 hoặc 366 đều thuộc $A_3$.
			      \item Ví dụ: $36 \in A_2$, nếu ta thêm 4 hoặc 5 vào sau nó thì được 364 hoặc 365 đều thuộc $B_3$.
		      \end{itemize}
		\item Với mỗi số thuộc $B_n$, có một cách thêm vào cuối một chữ số 4 (hoặc chữ số 5) để được $A_{n+1}$ và có ba cách thêm một số để được $B_{n+1}$.
		      \begin{itemize}
			      \item Ví dụ: $34 \in B_2$, nếu ta thêm 5 vào sau nó thì được 345 thuộc $A_3$.
			      \item Ví dụ: $34 \in B_2$, nếu ta thêm 3 hoặc 4 hoặc 6 vào sau nó thì được 343 hoặc 344 hoặc 346 đều thuộc $B_3$.
		      \end{itemize}
	\end{itemize}
	Do đó ta có:
	\begin{align} |A_{n+1}| &= 2|A_n| + |B_n| & \\ |B_{n+1}| &= 2|A_n| + 3|B_n| & \end{align}
	Thay (1) vào (2), ta được: $|B_{n+1}| = 2|A_n| + 3(|A_{n+1}| - 2|A_n|) = 3|A_{n+1}| - 4|A_n| \quad (3)$.
	\\ Từ (3) suy ra $|B_n| = 3|A_n| - 4|A_{n-1}| \quad (4)$.
	\\Thay (4) vào (1) suy ra $|A_{n+1}| = 2|A_n| + 3|A_n| - 4|A_{n-1}| = 5|A_n| - 4|A_{n-1}|$.
	\\ Do đó $|A_{n+1}| = 5|A_n| - 4|A_{n-1}|$. Xét phương trình đặc trưng $t^2 = 5t - 4 \Rightarrow t=1 \lor t=4$.
	\\Phương trình $(*)$ có nghiệm dạng $|A_n| = a \cdot 1^n + b \cdot 4^n$. $|A_n| = a + b \cdot 4^n$ (a, b là các tham số).
	$A_1 = \{3; 6\} \Rightarrow |A_1| = 2 \Rightarrow a+4b = 2 \quad (5)$.
	\\$A_2 = \{36; 63; 33; 66; 45; 54\} \Rightarrow |A_2| = 6 \Rightarrow a+16b = 6 \quad (6)$.
		\\Từ (5) và (6) ta giải ra được $a = \dfrac{2}{3}$, $b = \dfrac{1}{3}$.
		\\Do đó $|A_{2025}| = \dfrac{2+4^{2025}}{3}$.
		\\Xác suất số đó chia hết cho 3 từ tập $Y$ là:
	$\dfrac{2+4^{2025}}{3} \times \dfrac{1}{4^{2025}} = \dfrac{1}{3} (\dfrac{2}{3 \cdot 4^{2050}} + \dfrac{1}{3 \cdot 4^{2051}})$.
		\\ Do đó $a = 4049, b = 1 \Rightarrow |A_{n}| = a - 18b = 4031$.}
\end{ex}
\begin{ex}
	\immini[thm]{ Cho hàm số $y = -\dfrac{1}{3}x^3 + \dfrac{3}{4}x^2 + 3x$ có đồ thị $(C)$ và đường thẳng $d$ đi qua gốc tọa độ tạo thành hai miền phẳng có diện tích $S_1$ và $S_2$ như hình vẽ. Biết $S_1 = \dfrac{27}{4}$ và $S_2 = \dfrac{m}{n}$ (hai số $m, n$ là nguyên tố cùng nhau), tính giá trị $2m-n$.\shortans{142}}{\includegraphics[scale=.8]{img/HXN-4.20}}
	\loigiai{
		\begin{center}
			\includegraphics[scale=.7]{img/HXN-4.20a}
		\end{center}
		Gọi $a>0$ là hoành độ giao điểm của $(C)$ và $d$.\\
		Đường thẳng $d$ có hệ số góc $k =\dfrac{-\dfrac{1}{3}a^3 + \dfrac{3}{4}a + 3a}{a} = -\dfrac{1}{2}a^2+\dfrac{3}{4}a + 3$.\\
		Mặt khác $d$ đi qua gốc tọa độ nên phương trình là $y = \left(-\dfrac{1}{2}a^2+\dfrac{3}{4}a + 3\right)x$.\\
		Ta có: $S_1 = \int\limits_{0}^{a} \left[-\dfrac{1}{2}x^3 +\dfrac{3}{4}x^2 + 3x - \left(-\dfrac{1}{2}a^2 + \dfrac{3}{4}a + 3\right)x\right] \mathrm{d} x$
		$\Leftrightarrow \dfrac{27}{4} = \dfrac{1}{8}a^4-\dfrac{1}{8}a^3 \Rightarrow a=3 > 0$.
		Ta có phương trình $d: y = \dfrac{3}{4}x$.
		Khi đó phương trình hoành độ giao điểm của $(C)$ và $d$ là:
		$\dfrac{3}{4}x - \left(-\dfrac{1}{2}x^3 + \dfrac{3}{4}x^2 + 3x\right) = 0 \Leftrightarrow x= 3 ; x = 0 ; x = -\dfrac{3}{2}$.\\
		Do đó $S_2 = \int\limits_{-\frac{3}{2}}^{0} \left|\dfrac{1}{2}x^3 - \dfrac{3}{4}x^2 - \dfrac{9}{4}x\right| \mathrm{d} x = \dfrac{135}{128} = \dfrac{m}{n}$. Từ đó suy ra $2m-n=142$.
	}
\end{ex}
\begin{ex}%Câu 4
	\immini[thm]{Một mảnh đất hình chữ nhật có kích thước $40\,m\times 50\,m$ đang được người chủ trồng cỏ tự nhiên. Vào buổi sáng, khi mặt trời vừa lên, mảnh đất này bị một mái nhà xưởng gần đó chắn ánh sáng. Khi mặt trời lên cao hơn, ánh sáng đã chiếu từ từ lên mảnh đất. Ta xem ranh giới giữa phần được chiếu sáng và phần tối là các đường thẳng song song thay đổi.
		Có thời điểm đường ranh giới này đi qua hai điểm A, M như hình vẽ (M là trung điểm một cạnh hình chữ nhật).
		Khi diện tích phần tối của mảnh đất bằng $75\,\,m^2$ , người ta đo được tốc độ giảm cạnh theo phương AD bằng 2 cm/s; hỏi tốc độ giảm diện tích phần tối của mảnh đất là bao nhiêu $c{m^2}$ /s? Kết quả được làm tròn đến hàng phần chục.\shortans{3873}}{\includegraphics[scale=1.1]{img/HXN-4.21}}

	\loigiai{
	\begin{center}
		\includegraphics[scale=1]{img/HXN-4.21a}
	\end{center}
	Xét tam giác ABM vuông tại B có $\tan\widehat{BAM}=\dfrac{BM}{AB}=\dfrac{2}{5}$ .\\
	Đặt $x=DF\in\left[0\,;\,\,40\right]\,,\,\,y=DE\in\left[0\,;\,\,50\right]$ (x, y thay đổi (giảm) vì ánh sáng ngày càng lan rộng).\\
	Vì $EF\text{//}AM$ nên $\tan\widehat{FED}=\tan\widehat{BAM}\Leftrightarrow\dfrac{x}{y}=\dfrac{2}{5}\Leftrightarrow y=\dfrac{5x}{2}$ .\\
	Diện tích phần tối tức thời là $S=\dfrac{1}{2}xy=\dfrac{1}{2}x\cdot\dfrac{5x}{2}=\dfrac{5}{4}x^2$ (1).\\
	Khi diện tích phần tối bằng $75\,\,m^2$ thì $\dfrac{5}{4}{x^2}=75\Rightarrow{x^2}=60\Rightarrow x=2\sqrt{15}$ m.\\
	Đạo hàm hai vế của (1) theo biến t ta được: $ \dfrac{dS}{dt}=\dfrac{5}{2}x .\dfrac{dx}{dt}\quad (2)$ .\\
	Thay $x=2\sqrt{15}\,\,m=200\sqrt{15}\,\,cm$ và $\dfrac{dx}{dt}=2$ cm/s vào (2), ta được $\dfrac{dS}{dt}\approx 3873\,\,c{m^2}$ /s.}
\end{ex}
\begin{ex}%Câu 5
	Trong không gian $Oxyz$ , cho điểm $A\left(-2\,;\,\,6\,;\,\,0\right)$ và mặt phẳng $\left(\alpha\right):3x+4y+89=0$ .
	Đường thẳng $d$ thay đổi nằm trên mặt phẳng $\left(Oxy\right)$ và luôn đi qua điểm $A$ .
	Gọi $H$ là hình chiếu vuông góc của $M\left(4\,;\,\,-2\,;\,\,3\right)$ trên đường thẳng d.
	Khoảng cách nhỏ nhất từ $H$ đến mặt phẳng $\left(\alpha\right)$ bằng bao nhiêu?
	\shortans{15}
	\loigiai{
	\begin{center}
		\includegraphics[scale=1]{img/HXN-4.22}
	\end{center}
	Gọi $K$ là hình chiếu vuông góc của $M$ lên $\left(Oxy\right)$ , suy ra $K\left(4\,;\,\,-2\,;\,\,0\right)$ .\\
	Vì $\left\{\begin{aligned}
			 & AH\perp MK\,\,\,\left(\text{do}\,\,\,MK\perp\left(Oxy\right)\right) \\
			 & AH\perp MH                                                          \\
		\end{aligned}\right.\Rightarrow AH\perp\left(MKH\right)\Rightarrow AH \perp KH $ .\\
	Khi đó $H$ luôn thuộc đường tròn $(C)$ có tâm là trung điểm $I\left(1\,;\,\,2\,;\,\,0\right)$ của đoạn $AK$ , bán kính $R=\dfrac{AK}{2}=5$ .\\
	Gọi $\Delta=\left(\alpha\right)\cap\left(Oxy\right)$ , ta thấy $\left(\alpha\right)\perp\left(Oxy\right)$ (vì $\vec{n}_{\left(\alpha\right)}\cdot\vec{k}=0$) . Khi đó: $d\left(I\,,\,\,\Delta\right)=d\left(I\,,\,\,\left(\alpha\right)\right)=20$ .\\
	Khoảng cách nhỏ nhất từ $H$ đến mặt phẳng $\left(\alpha\right)$ là $d{\left(H\,,\,\,\left(\alpha\right)\right)_{\text{min}}}=d{\left(H\,,\,\,\Delta\right)_{\text{min}}}=d\left(H_0\,,\,\,\Delta\right)$ $=d\left(I\,,\,\,\Delta\right)-R=20-5=15$ .}
\end{ex}
\Closesolutionfile{ans}
\inputansbox{6,4,3}{ans/ans-HXN-\sode-T,ans/ans-HXN-\sode-TF,ans/ans-HXN-\sode-SA}
% \def\sode{5}
\begin{name}
	{\tenchude}
	{\tendethi}
	{\tentruong}
	{\thoigian}
\end{name}
\Opensolutionfile{ans}[ans/ans-HXN-\sode-T]
\caulc
\begin{ex}%Câu 1
	Cho cấp số cộng $\left(u_n\right)$ có $u_2=3,\,\,u_3=5$. Công sai $d$ của cấp số cộng là:
	\choice
	{1}
	{\True 2}
	{8}
	{4}
	\loigiai{
		Chọn B.\\
		Ta có: $\,u_3=u_2+d\Leftrightarrow 5=3+d\Leftrightarrow d=2$ .}
\end{ex}
\begin{ex}%Câu 2
	\immini[thm]{ Cho hàm số có đồ thị như hình vẽ bên. Hàm số đã cho đồng biến trên khoảng nào sau đây?
		\choice
		{$\left(-\infty\,;\,\,-1\right)$}
		{$\left(-1\,;\,\,1\right)$}
		{$\left(-2\,;\,\,1\right)$}
		{$\left(1\,;\,\,+\infty\right)$}}{\includegraphics[scale=.8]{img/HXN-5.2}}
	\loigiai{
		Chọn B.\\
		Từ đồ thị hàm số, ta thấy hàm số đồng biến trên khoảng $\left(-1\,;\,\,1\right)$.}
\end{ex}
\begin{ex}%Câu 3
	Cho hình lăng trụ đứng $ ABC.A'{B}'{C}'$ có đáy là tam giác vuông cân tại $ B$ với $ AB=a$ và $A'B=a\sqrt{3}$. Thể tích khối lăng trụ $ ABC.A'{B}'{C}'$ là
	\choice
	{$\dfrac{a^3\sqrt{3}}{2}$}
	{$\dfrac{a^3}{6}$}
	{$\dfrac{a^3}{2}$}
	{\True $\dfrac{a^3\sqrt{2}}{2}$}
	\loigiai{
	Chọn D.\\
	Ta có $ A{A}'=\sqrt{A'{B^2}-A{B^2}}=a\sqrt{2}$, $S_{ABC}=\dfrac{1}{2}A{B^2}=\dfrac{a^2}{2}$.\\
	Thể tích khối lăng trụ là $ V=A{A}'\cdot{S_{ABC}}=\dfrac{a^3\sqrt{2}}{2}$.}
\end{ex}
\begin{ex}%Câu 4
	Gọi $ S$ là diện tích hình phẳng giới hạn bởi các đường $ y=\text{e}^x$, $ y=0$, $ x=0$, $ x=2$. Mệnh đề nào dưới đây đúng?
	\choice
	{$ S=\pi\displaystyle\int\limits_0^2\text{e}^{2x}\text{d}x$}
	{\True $ S=\displaystyle\int\limits_0^2\text{e}^x\text{d}x$}
	{$ S=\pi\displaystyle\int\limits_0^2\text{e}^x\text{d}x$}
	{$ S=\pi\displaystyle\int\limits_0^2\text{e}^x\text{d}x$}
	\loigiai{
		Chọn B.\\
		Diện tích hình phẳng giới cần tính là $ S=\displaystyle\int\limits_0^2e^x\text{d}x$.}
\end{ex}
\begin{ex}%Câu 5
	Mặt phẳng đi qua ba điểm $ A\left(0\,;\,\,0\,;\,\,2\right)$, $ B\left(1\,;\,\,0\,;\,\,0\right)$ và $C\left(0\,;\,\,3\,;\,\,0\right)$ có phương trình là
	\choice
	{\True $\dfrac{x}{1}+\dfrac{y}{3}+\dfrac{z}{2}=1$}
	{$\dfrac{x}{1}+\dfrac{y}{3}+\dfrac{z}{2}=-1$}
	{$\dfrac{x}{2}+\dfrac{y}{1}+\dfrac{z}{3}=1$}
	{$\dfrac{x}{2}+\dfrac{y}{1}+\dfrac{z}{3}=-1$}
	\loigiai{
		Chọn A.\\
		Mặt phẳng (ABC) chắn các trục tọa độ Ox, Oy, Oz lần lượt tại $ A\left(0\,;\,\,0\,;\,\,2\right)$, $ B\left(1\,;\,\,0\,;\,\,0\right)$ và $C\left(0\,;\,\,3\,;\,\,0\right)$ nên có phương trình $\dfrac{x}{1}+\dfrac{y}{3}+\dfrac{z}{2}=1$.}
\end{ex}
\begin{ex}%Câu 6
	Nếu $\displaystyle\int\limits_{-1}^2f(x)\text{d}x=5$ thì $\displaystyle\int\limits_{-1}^24f(x)\text{d}x$ bằng:
	\choice
	{\True $ 20$}
	{$ 10$}
	{$\dfrac{5}{2}$}
	{$\dfrac{5}{4}$}
	\loigiai{
		Chọn A.\\
		Ta có: $\displaystyle\int\limits_{-1}^24f(x)\text{d}x=4\displaystyle\int\limits_{-1}^2f(x)\text{d}x=4.5=20$.}
\end{ex}
\begin{ex}%Câu 7
	Trong không gian tọa độ $ Oxyz$, mặt cầu $(S)$ có tâm $ I\left(2\,;\,1;\,-1\right)$ và đường kính 6 có phương trình là
	\choice
	{$(x-2)^2+(y-1)^2+(z+1)^2=36$}
	{\True $(x-2)^2+(y-1)^2+(z+1)^2=9$}
	{$(x+2)^2+(y+1)^2+(z-1)^2=9$}
	{$(x+2)^2+(y+1)^2+(z-1)^2=36$}
	\loigiai{
		Chọn B.\\
		Mặt cầu $ (S)$ có tâm $ I(2;1;-1)$, bán kính $ R=3$ nên có phương trình là\\
		$\left(x-2\right)^2+\left(y-1\right)^2+\left(z+1\right)^2=9$.}
\end{ex}
\begin{ex}%Câu 8
	Một mẫu số liệu ghép nhóm về chiều cao của một lớp (đơn vị là centimét) có phương sai là $ 6,25$. Độ lệch chuẩn của mẫu số liệu đó bằng bao nhiêu cm:
	\choice
	{\True $ 2,5$ }
	{$ 12,5$}
	{$ 3,125$}
	{$ 42,25$}
	\loigiai{
		Chọn A.\\
		Độ lệch chuẩn của mẫu số liệu là: $\sqrt{6,25}=2,5$.}
\end{ex}
\begin{ex}%Câu 9
	Tìm giá trị lớn nhất $ M$ của hàm số $ y=\dfrac{3x-1}{x-3}$ trên đoạn $\left[0\,;\,2\right]$.
	\choice
	{$ M=5$}
	{$ M=-5$}
	{\True $ M=\dfrac{1}{3}$}
	{$ M=-\dfrac{1}{3}$}
	\loigiai{
		Chọn C.\\
		Ta có: $y'=\dfrac{-8}{\left(x-3\right)^2}<\,0\,,\,\,\forall x\in\left[0\,;\,\,2\right]$. Hàm số luôn nghịch biến trên $\left[0\,;\,2\right]$.\\
		Ta tính được: $ y(0)=\dfrac{1}{3}$, $ y(2)=\,-5$.\\
		Do đó giá trị lớn nhất của hàm số trên $\left[0\,;\,2\right]$ là $ M=y(0)=\dfrac{1}{3}$.}
\end{ex}
\begin{ex}%Câu 10
	Cho hai biến cố $ A\,,\,\,B$ với $ 0<P(B)<1.$ Phát biểu nào sau đây là đúng?
	\choice
	{$ P(A)=P\left(\overline{B}\right).P\left(A|B\right)+P(B).P\left(A|\overline{B}\right)$}
	{$ P(A)=P(B).P\left(A|B\right)-P\left(\overline{B}\right).P\left(A|\overline{B}\right)$}
	{$ P(A)=P\left(\overline{B}\right).P\left(A|\overline{B}\right)-P(B).P\left(A|B\right)$}
	{\True $ P(A)=P(B).P\left(A|B\right)+P\left(\overline{B}\right).P\left(A|\overline{B}\right)$}
	\loigiai{
		Chọn D.\\
		Theo công thức xác suất toàn phần ta có: $ P(A)=P(B).P\left(A|B\right)+P\left(\overline{B}\right).P\left(A|\overline{B}\right)$.}
\end{ex}
\begin{ex}%Câu 11
	Một thư viện ghi lại số giờ đọc sách của 50 sinh viên trong một ngày và thu được mẫu số liệu ghép nhóm sau:\\
	\centerline{\begin{tabular}{|c|c|c|c|c|c|}
			\hline
			Nhóm giờ     & $\left[0\,;\,\,1\right)$ & $\left[1\,;\,\,2\right)$ & $\left[2\,;\,\,3\right)$ & $\left[3\,;\,\,4\right)$ & $\left[4\,;\,\,5\right)$ \\
			\hline
			Số sinh viên & 8                        & 11                       & 15                       & 9                        & 7                        \\
			\hline
		\end{tabular}}\\
	Khoảng tứ phân vị của mẫu số liệu ghép nhóm gần nhất với giá trị nào sau đây?
	\choice
	{$1,69$}
	{$1,85$}
	{$2,02$}
	{\True $1,98$}
	\loigiai{
	Chọn D.\\
	Giả sử mẫu số liệu gốc là $x_1;\,\,x_2;\,\,...;\,\,x_{50}$ được xếp theo thứ tự không giảm.\\
	Xét nửa bên trái mẫu số liệu gốc là $x_1;\,\,x_2;\,\,...;\,\,x_{25}$. Tứ phân vị thứ nhất của mẫu số liệu gốc là $x_{13}\in\left[1\,;\,\,2\right)$ nên tứ phân vị thứ nhất của mẫu số liệu ghép nhóm là $Q_1=1+\dfrac{\dfrac{50}{4}-8}{11}.1=\dfrac{31}{22}\approx 1,41$ (giờ).\\
	Xét nửa bên phải mẫu số liệu gốc là $x_{26};\,\,x_2;\,\,...;\,\,x_{50}$.\\
	Tứ phân vị thứ ba của mẫu số liệu gốc là $x_{38}\in\left[3\,;\,\,4\right)$ nên tứ phân vị thứ ba của mẫu số liệu ghép nhóm là $Q_3=3+\dfrac{3.\dfrac{50}{4}-34}{9}.1=\dfrac{61}{18}\approx 3,39$ (giờ).\\
	Khoảng tứ phân vị của mẫu số liệu ghép nhóm: $\Delta Q=Q_3-Q_1\approx 1,98$ (giờ).}
\end{ex}
\begin{ex}%Câu 12
	Tập nghiệm của bất phương trình $\log_5\left(2x-1\right)<\log_5\left(x+2\right)$ là
	\choice
	{$S=\left(3\,;\,\,+\infty\right)$}
	{$S=\left(-\infty\,;\,\,3\right)$}
	{\True $S=\left(\dfrac{1}{2}\,;\,\,3\right)$}
	{$S=\left(-2\,;\,\,3\right)$}
	\loigiai{
		Chọn C.\\
		Ta có: $\log_5\left(2x-1\right)<\log_5\left(x+2\right)\Leftrightarrow\left\{\begin{aligned}
				 & 2x-1>0   \\
				 & 2x-1<x+2 \\
			\end{aligned}\right.\Leftrightarrow\left\{\begin{aligned}
				 & x>\dfrac{1}{2} \\
				 & x<3            \\
			\end{aligned}\right.$ .\\
		Vậy tập nghiệm phương trình $S=\left(\dfrac{1}{2}\,;\,\,3\right)$ .}
\end{ex}
\Closesolutionfile{ans}
\cauds
\Opensolutionfile{ans}[ans/ans-HXN-\sode-TF]
\begin{ex}
	Một người đang bơm khí vào một quả bóng bay với tốc độ $100 $cm$^3/s$. Quả bóng ngày càng to dần nhưng luôn có dạng hình cầu. Đây là loại bóng bóng mà nếu người bơm để bán kính vượt quá $30$cm thì bóng sẽ bể.
	Xét tính đúng sai các mệnh đề sau:
	\choiceTF
	{Sau 10 giây, bán kính quả bóng bóng bằng $6,4 cm$ (làm tròn đến hàng phần chục của $cm$)}
	{\True Người bơm không thể để cho thể tích quả bóng bóng vượt quá $113$ lít (làm tròn đến hàng phần chục của lít)}
	{\True Khi đường kính của quả bóng bóng là $50 cm$ thì bán kính của quả bóng đang tăng với tốc độ $0,01 cm/s$ (làm tròn đến hàng phần trăm của $cm/s$)}
	{Nếu sau khi bơm được 4 giây, người bơm tăng tốc độ bơm thêm $5 cm^3$ trên một giây thì sau 189 giây (làm tròn đến hàng đơn vị của giây), bóng bóng sẽ bể}
	\loigiai{
		\begin{itemchoice}
			\itemch Gọi $V(t), R(t)$, là thể tích và bán kính quả bóng bóng sau $t$ giây, ta có $V(t) = \dfrac{4}{3}\pi R^3(t)$.
			Sau 10 giây, thể tích quả bóng là $V(10) = 100 \times 10 = 1000 cm^3$.
			Ta có $V(10) = \dfrac{4}{3}\pi R_{10}^3 = 1000 \Rightarrow R_{10} \approx 6.2 cm$.
			\itemch Bán kính tối đa của quả bóng bóng là $30 cm$; thể tích tối đa của quả bóng bóng là $\dfrac{4}{3}\pi \cdot 30^3 \approx 113097 cm^3 \approx 113$ lít.
			\itemch Khi bán kính bong bóng bóng bằng $\dfrac{50}{2} = 25 cm$ thì thể tích bong bóng là $\dfrac{4\pi \cdot 25^3}{3} = \dfrac{62500\pi}{3} cm^3$.
			Đạo hàm hai vế của $V(t) = \dfrac{4}{3}\pi R^3(t)$ theo t, ta được: $\dfrac{dV(t)}{dt} = 4\pi R^2\cdot \dfrac{dR}{dt}$.
			Thay $R_t = 25 cm$; $\dfrac{dV(t)}{dt} = 100 cm^3/s$, ta có: $\dfrac{dR}{dt} \approx 0.01 cm/s$.
			\itemch Thể tích bong bóng sau $t+4$ giây ($t \ge 0$) là $V(t) = 100\cdot 4 + \int\limits_{0}^{t}(5t+100)dt$.
			Thể tích tối đa của quả bóng bóng là $\dfrac{4}{3}\pi \cdot 30^3 cm^3$.
			Xét $V(t) = 100\cdot 4 + \int\limits_{0}^{t}(5t+100)dt = \dfrac{4}{3}\pi \cdot 30^3 \Rightarrow t \approx 193$ giây.
		\end{itemchoice}
	}
\end{ex}
\begin{ex}
	\immini[thm]{ Vịnh Hạ Long là một địa danh du lịch được nhiều người biết đến trên thế giới, nơi đây vẫn còn nhiều quần thể đảo lớn nhỏ chưa được khám phá. Một công ty du lịch quyết định khai thác khu vực có một số đảo nhỏ với hình dáng đặc biệt nếu nhìn từ trên xuống; trong số đó có hai hòn đảo mà phần giới hạn lát cắt của nó được mô phỏng như hai đồ thị hàm số trên hình. Với hệ trục tọa độ $Oxy$ thích hợp, đơn vị trên mỗi trục là $100$ mét, đường cong mô tả cho hòn đảo thứ nhất có dạng $y = \log_{a}{x}$ đi qua điểm có tọa độ $(3; 1)$.
	}{\includegraphics[scale=1]{img/HXN-5.14}}
	\choiceTF
	{ Điểm có tọa độ $(9; 3)$ thuộc đường cong $y = \log_{a}{x}$}
	{ Chủ dự án muốn xây dựng một nơi trực tiếp nhìn ra biển để du khách tham quan, ăn uống... Họ đã lựa chọn khu vực tam giác cong $ABC$ như trong hình (đường cong $AC$ tiếp giáp biển); diện tích khu vực này là $536$ $m^2$ (làm tròn đến hàng đơn vị)}
	{\True Chủ dự án đã thuê một số kỹ sư rất giỏi toán (đặc biệt giỏi về hàm số mũ-lôgarit) đi khảo sát khu vực này và họ nhận thấy có thể bồi đắp thêm cho hòn đảo thứ hai để đường cong giáp biển $y = g(x)$ của nó đối xứng với đường cong $y = \log_{a}{x}$ qua đường thẳng $y = x+1$. Khi đó đường cong $g(x) = 1+3 \cdot 3^x$}
	{ Chủ dự án định xây một cây cầu nối liền hai hòn đảo, khoảng cách ngắn nhất theo đường chim bay của cây cầu bằng $285$ $m$ (làm tròn đến hàng đơn vị mét)}
	\loigiai{
		\begin{itemchoice}
			\itemch Đường cong $y = \log_{a}{x}$ đi qua điểm $(3; 1)$ nên $1 = \log_{a}{3} \Rightarrow a = 3$.\\
			Khi đó hàm số trở thành $y = \log_{3}{x}$; đường cong này không đi qua điểm $(9; 3)$.
			\itemch Điểm $A(x_A; -1)$ thuộc đồ thị hàm số $y = \log_3 x \Rightarrow \log_3 x_A = -1 \Rightarrow x_A = 3^{-1} = \dfrac{1}{3}$.\\
			Diện tích tam giác cong $ABC$ là phần hình phẳng được giới hạn bởi hai đồ thị $y=\log_3 x$; $y = -1$ cùng các đường thẳng $x = \dfrac{1}{3}$; $x = 4$.\\
			Do đó diện tích cần tính là $S = \int\limits_{\frac{1}{3}}^{4} |\log_3 x - (-1)| \mathrm{d} x \approx 571$ $m^2$.
			\itemch Gọi $M(x_M; y_M) \in (C_1): y = \log_3 x$ và $N(x; y) \in (C_2): y = g(x)$.
			\begin{center}
				\includegraphics[scale=.8]{img/HXN-5.14a}
			\end{center}
			$M, N$ đối xứng qua $x - y + 1 = 0$ nên ta có:\\
			$\begin{cases} \dfrac{x + x_M}{2} - \dfrac{y + y_M}{2} + 1 = 0 \\ 1 \cdot (x - x_M) + (-1) \cdot (y - y_M) = 0 \end{cases} \Leftrightarrow \begin{cases} x + x_M - y - y_M + 2 = 0 \\ x - x_M - y + y_M = 0 \end{cases} \Rightarrow \begin{cases} y_M = x+1 \\ x_M = y-1 \end{cases}$ hay $M(y-1; x+1)$.\\
			Vì $M \in (C_1)$ nên $x+1 = \log_3 (y-1) \Rightarrow y-1 = 3^{x+1} \Rightarrow y = 3^{x+1} + 1$ hay $y = g(x) = 1 + 3 \cdot 3^x$.\\
			Cách giải khác (nhấn vào link) \hyperlink{Cách giải khác(nhấn vào link)}{https://www.tiktok.com/@tp1.phatvn.68/photo/7517274797793955079}
			\itemch Xét tiếp tuyến của đường cong $y = \log_3 x$ biết tiếp tuyến song song với đường thẳng $y = x+1$.\\
			Hệ số góc tiếp tuyến là $k = 1$; gọi $M(x_0; y_0)$ là tiếp điểm.\\
			Thì $f'(x_0) = \dfrac{1}{x_0 \ln 3} = 1 \Rightarrow x_0 = \dfrac{1}{\ln 3}$; $y_0 = \log_3 \dfrac{1}{\ln 3}$.\\
			Độ dài ngắn nhất cây cầu (theo đường chim bay) bằng hai lần khoảng cách từ $M\left(\dfrac{1}{\ln 3}; \log_3 \dfrac{1}{\ln 3}\right)$ đến đường thẳng $y = x+1$.\\
			Ta có: $d_{\min} = 2 \cdot \dfrac{\left| \dfrac{1}{\ln 3} - \log_3 \dfrac{1}{\ln 3} + 1 \right|}{\sqrt{1^2 + (-1)^2}} \times 100 \approx 282$ $m$.
		\end{itemchoice}
	}
\end{ex}
\begin{ex}
	Hộp A đựng 4 bi xanh và 4 bi trắng, hộp B đựng 6 bi xanh và 3 bi trắng, hộp C không có viên bi nào. Người ta thực hiện liên tiếp ba hành động sau đây hoàn toàn ngẫu nhiên:
	\begin{itemize}
		\item Lấy 1 viên bi từ hộp A bỏ sang hộp B.
		\item Lấy 1 viên bi từ hộp B bỏ sang hộp C.
		\item Lấy 1 viên bi từ hộp A bỏ sang hộp C.
	\end{itemize}
	Xét tính đúng sai các mệnh đề sau:
	\choiceTF
	{Nếu từ hộp A đã lấy 1 bi trắng bỏ sang hộp B thì xác suất để lấy bi trắng từ hộp B bỏ sang hộp C bằng $\dfrac{2}{5}$}
	{\True Xác suất để lấy được bi trắng từ hộp B bỏ sang hộp C bằng $\dfrac{7}{20}$}
	{\True Xác suất để lấy từ C được 2 bi xanh bằng $\dfrac{9}{28}$}
	{\True Xác suất để 2 bi lấy từ hộp C đều là các bi từ hộp A chuyển sang bằng $\dfrac{1}{15}$ biết rằng đó là 2 bi xanh}
	\loigiai{
		\begin{itemchoice}
			\itemch Nếu từ hộp A đã lấy 1 bi trắng bỏ sang hộp B thì khi đó hộp B có 6 bi xanh và 4 bi trắng; xác suất để lấy 1 bi trắng từ hộp B là $\dfrac{4}{10} = \dfrac{2}{5}$.
			\itemch Ta mô phỏng bài toán bởi sơ đồ sau:
			\begin{center}
				\includegraphics[scale=1]{img/HXN-5.15}
			\end{center}
			Ta có: $P(\text{Trắng}_{[B]\to[C]}) = \dfrac{1}{2} \cdot \dfrac{3}{10} + \dfrac{1}{2} \cdot \dfrac{4}{10} = \dfrac{7}{20}$.
			\itemch Ta có: $P(2\text{Xanh}_{[C]}) = \dfrac{1}{2} \cdot \dfrac{7}{10} \cdot \dfrac{3}{7} + \dfrac{1}{2} \cdot \dfrac{6}{10} \cdot \dfrac{4}{7} = \dfrac{9}{28}$.\\
			(Trong đó ta xem kí hiệu $2\text{Xanh}_{[C]}$ là lấy được 2 viên bi xanh từ hộp C).
			\itemch Ta có: $P(2\text{ bi}_{[A]\to[C]} | 2\text{Xanh}_{[C]}) = \dfrac{\dfrac{1}{2} \cdot \dfrac{1}{10} \cdot \dfrac{3}{7}}{\dfrac{9}{28}} = \dfrac{1}{15}$.\\
			(Trong đó ta xem kí hiệu $2\text{ bi}_{[A]\to[C]}$ là lấy từ hộp C đúng 2 viên bi từ hộp A chuyển qua).
		\end{itemchoice}
	}
\end{ex}
\begin{ex}
	Trong không gian $Oxyz$ cho trước, đơn vị trên mỗi trục là mét, có hai chiếc chiến đấu cơ từ hai vị trí $A(40;-15;15)$ và $B(55;-10;65)$ cần đáp xuống hai vị trí thuộc tàu sân bay hải quân để nạp nhiên liệu. Bề mặt chứa các đường băng trên tàu là mặt phẳng $(P)$ có phương trình $3x - y + 2z - 25 = 0$. Xét tính đúng sai các mệnh đề sau:
	\choiceTF
	{\True Đường thẳng qua $A$ và vuông góc với mặt phẳng $(P)$ có phương trình chính tắc là $\dfrac{x - 40}{3} = \dfrac{y + 15}{-1} = \dfrac{z - 15}{2}$}
	{ Tổng khoảng cách từ hai vị trí chiến đấu cơ đến mặt phẳng chứa đường băng là $110$ $m$ (làm tròn đến hàng đơn vị của mét)}
	{\True Tọa độ $A'$ đối xứng với $A$ qua $(P)$ là $A'(-20; 5; -25)$}
	{ Người chỉ huy ở tàu sân bay phát tín hiệu để hai chiến đấu cơ đáp xuống các vị trí $M, N$ cách nhau $5\sqrt{6}$ $m$. Tổng đường bay ngắn nhất $AM + BN$ bằng $115$ $m$ (làm tròn đến hàng đơn vị)}
	\loigiai{
		\begin{itemchoice}
			\itemch Đường thẳng qua $A$ và vuông góc với mặt phẳng $(P)$ có phương trình chính tắc là $\dfrac{x - 40}{3} = \dfrac{y + 15}{-1} = \dfrac{z - 15}{2}$.
			\itemch Ta có: $d(A, (P)) + d(B, (P)) = \dfrac{|3 \cdot 40 - (-15) + 2 \cdot 15 - 0|}{\sqrt{3^2 + (-1)^2 + 2^2}} + \dfrac{|3 \cdot 55 - (-10) + 2 \cdot 65 - 0|}{\sqrt{3^2 + (-1)^2 + 2^2}} = 30\sqrt{14} \approx 112$ $m$.
			\begin{center}
				\includegraphics[scale=.8]{img/HXN-5.16}
			\end{center}
			\itemch Gọi $H$ là hình chiếu vuông góc của $A$ trên $(P)$ thì tọa độ $H$ thỏa hệ phương trình\\ $\begin{cases} \dfrac{x - 40}{3} = \dfrac{y + 15}{-1} = \dfrac{z - 15}{2} \\ 3x - y + 2z - 0 = 0 \end{cases} \Leftrightarrow \begin{cases} x+3y+5=0 \\ 2y+z+15=0 \\ 3x-y+2z=25 \end{cases}$. Giải hệ này ta được $\begin{cases} x=10 \\ y=-5 \\ z=-5 \end{cases}$ hay $H(10; -5; -5)$.\\
			$A'$ đối xứng với $A$ qua $(P)$ nên $H$ là trung điểm của $AA'$. Suy ra $A'(-20; 5; -25)$.
			\itemch Lấy điểm $E$ thỏa mãn $\overrightarrow{AE} = \overrightarrow{MN}$, suy ra $A'M = EN$.\\
			Vì $A'$ cố định mà $A'E = 5\sqrt{6}$ nên $E$ thuộc đường tròn tâm $A'$, bán kính $r = 5\sqrt{6}$; đường tròn này thuộc mặt phẳng $(Q)$ qua $A'$ và song song với $(P)$.\\
			Gọi $K, F$ theo thứ tự là hình chiếu vuông góc của $B$ trên $(P)$, $(Q)$ suy ra $K(-5; 10; 25)$. $HK = 15\sqrt{6}$; $KF = HA' = AH = 10\sqrt{33}$.\\
			Ta có $AM + BN = A'M + BN = EN + BN \ge BE$.\\
			Đẳng thức xảy ra khi $E, N, B$ thẳng hàng theo thứ tự đó ($H, M, N, K$ thẳng hàng).\\
			Ta có: $BE = \sqrt{(\sqrt{20\sqrt{14}+10\sqrt{14}})^2 + (15\sqrt{6} - 5\sqrt{6})^2} = \sqrt{(30\sqrt{14})^2 + (10\sqrt{6})^2} = \sqrt{12600 + 600} = \sqrt{13200} = 20\sqrt{33}$.\\
			Vậy tổng độ dài bé nhất $AM + BN$ là $20\sqrt{33} \approx 115$ $m$.
		\end{itemchoice}
	}
\end{ex}

\Closesolutionfile{ans}
\caukq
\Opensolutionfile{ans}[ans/ans-HXN-\sode-SA]
\begin{ex}%Câu 13
	\immini[thm]{ Một trò chơi điện tử có luật chơi như sau:
		\begin{itemize}
			\item Người chơi xuất phát từ A và đi qua tất cả vị trí B, C, D, E trước khi về lại A để kết thúc lượt chơi của mình. Mỗi vị trí người chơi đi qua đúng 1 lần (trừ điểm A).
			\item Thông số trên mỗi đoạn đường đi gồm: x (huy chương) liên quan đến phần thưởng và y (quái vật) liên quan đến chướng ngại vật; điểm số người chơi đạt được trên mỗi đoạn đường có dạng $3x-2y$ .
		\end{itemize}
		Hỏi tổng số điểm tối đa mà người chơi đạt được là bao nhiêu?
		\shortans{ 44}}{\includegraphics[scale=1]{img/HXN-5.17}}
	\loigiai{
		Người chơi đi qua các con đường hợp lệ cùng với số điểm tương ứng như sau:
		\begin{itemize}
			\item $A \to B \to C \to E \to D \to A$; số điểm là $3(5+6+6+3+4)-2(3+4+6+0+1)=44$.
			\item $A \to B \to C \to D \to E \to A$; số điểm là $3(5+6+4+3+8)-2(3+4+5+0+6)=42$.
			\item $A \to E \to B \to C \to D \to A$; số điểm là $3(8+0+6+4+4)-2(6+1+4+5+1)=32$.
			\item $A \to E \to D \to C \to B \to A$; số điểm là $3(8+3+4+6+5)-2(6+0+5+4+3)=42$.
			\item $A \to D \to C \to B \to E \to A$; số điểm là $3(4+4+6+0+8)-2(1+5+4+1+6)=32$.
			\item $A \to D \to E \to C \to B \to A$; số điểm là $3(4+3+6+6+5)-2(1+0+6+4+3)=44$.
		\end{itemize}
		Số điểm tối đa mà người chơi đạt được là 44.}
\end{ex}
\begin{ex}%Câu 14
	Một hộp phấn không bụi có dạng hình hộp chữ nhật, chiều cao hộp phấn bằng 8,2 cm và đáy của nó có hai kích thước là 8,5 cm; 10,5 cm (xem hình vẽ). Tìm số đo góc phẳng nhị diện $\left[A,\,\,B'{D}',\,\,A'\right]$ (tính theo độ, làm tròn kết quả đến hàng phần chục).
	\begin{center}
		\includegraphics[scale=.5]{img/HXN-5.18}
	\end{center}
	\shortans{ 51,1}
	\loigiai{
	\begin{center}
		\includegraphics[scale=1]{img/HXN-5.18a}
	\end{center}
	Trong mặt phẳng $\left(A'{B}'{C}'{D}'\right)$, kẻ $A'H\bot{B}'{D}'$ tại H.\\
	Ta có: $\left\{\begin{aligned}
		 & {B}'{D}'\bot{A}'H                                                                      \\
		 & {B}'{D}'\perp A{A}'\,\,\left(\text{do}\,\,A{A}'\perp\left(A'{B}'{C}'{D}'\right)\right) \\
	\end{aligned}\right.$$\Rightarrow{B}'{D}'\perp\left(A{A}'H\right)\Rightarrow{B}'{D}'\perp AH$.\\
		Do đó $\widehat{AH{A}'}$ là góc phẳng nhị diện $\left[A,\,\,B'{D}',\,\,A'\right]$.\\
		Tam giác $A'{B}'{C}'$ vuông tại $A'$ có đường cao $A'H$ nên $\dfrac{1}{A'{H^2}}=\dfrac{1}{A'{B'^2}}+\dfrac{1}{A'{D'^2}}\Rightarrow{A}'H=\dfrac{A'{B}'.A'{D}'}{\sqrt{A'{B'^2}+A'{D'^2}}}=\dfrac{357}{2\sqrt{730}}$.\\
		Tam giác $ AH{A}'$ vuông tại $A'$ có $\tan\widehat{AH{A}'}=\dfrac{A{A}'}{A'H}=\dfrac{8,2}{\dfrac{357}{2\sqrt{730}}}\Rightarrow\widehat{AH{A}'}\approx 51,1^\circ$.}
\end{ex}
\begin{ex}%Câu 15
	Lan đang dự tính ghi danh học các lớp kỹ năng Anh ngữ, kỹ năng giao tiếp, kỹ năng quản lí v.v... tại một Hệ thống giáo dục trong thành phố, nơi mỗi lớp học chỉ học một lần mỗi tuần. Cô ấy đang chọn giữa 30 lớp học không trùng nhau. Có 6 lớp để lựa chọn cho mỗi ngày trong tuần, từ thứ Hai đến thứ Sáu. Sau nhiều ngày cân nhắc và tìm kiếm lời khuyên, Lan vẫn chưa thể đưa ra lựa chọn phù hợp. Sau cùng cô quyết định đăng ký 7 lớp được chọn ngẫu nhiên trong số 30 lớp đó, với mọi lựa chọn là đồng xác suất. Xác suất để Lan có lớp học vào tất cả các ngày từ thứ Hai đến thứ Sáu bằng $\dfrac{m}{n}$ (trong đó hai số m, n là nguyên tố cùng nhau). Tính $ m+n$.\\
	\shortans{ 491}
	\loigiai{
		Có hai khả năng chính để Lan có lớp học mỗi ngày trong tuần:\\
		• Trường hợp 1: Có 2 ngày có 2 lớp học, và 3 ngày còn lại có 1 lớp học.\\
		Số khả năng cho trường hợp 1 là $ C_5^2\cdot{\left(C_6^2\right)^2}\cdot{\left(C_6^1\right)^3}$.\\
		(Chọn 2 ngày trong 5 ngày có 2 lớp học, mỗi ngày đó chọn 2 lớp trong số 6 lớp; 3 ngày còn lại mỗi ngày chọn 1 lớp trong 6 lớp → có $\left(C_6^1\right)^3$ cách).\\
		• Trường hợp 2: Có 1 ngày có 3 lớp học, và 4 ngày còn lại mỗi ngày có 1 lớp học.\\
		Số khả năng cho trường hợp 2 là $ C_5^1C_6^3\cdot{\left(C_6^1\right)^4}$.\\
		(Chọn 1 ngày có 3 lớp học trong 5 ngày, chọn 3 lớp trong 6 lớp cho ngày đó; 4 ngày còn lại mỗi ngày chọn 1 lớp → $\left(C_6^1\right)^4$ cách).\\
		Vậy xác suất cần tính là $\dfrac{C_5^2\cdot{\left(C_6^2\right)^2}\cdot{\left(C_6^1\right)^3}+C_5^1C_6^3\cdot{\left(C_6^1\right)^4}}{C_{30}^7}=\dfrac{114}{377}=\dfrac{m}{n}$. Suy ra $m+n=491 $.}
\end{ex}
\begin{ex}%Câu 16
	Một chiến sĩ đặc công đang nấp ở bờ sông, cần phải bơi qua bờ bên kia để tấn công mục tiêu. Có thể xem con sông này là thẳng và có độ rộng 100 m; vận tốc bơi của chiến sĩ bằng một phần ba vận tốc chạy bộ. Biết rằng mục tiêu tấn công cách chiến sĩ 1 km theo đường chim bay; hỏi chiến sĩ phải bơi bao nhiêu mét để đến được mục tiêu nhanh nhất (làm tròn kết quả đến hàng đơn vị)?
	\shortans{106 }
	\begin{center}
		\includegraphics[scale=1.2]{img/HXN-5.20}
	\end{center}
	\loigiai{
		\begin{center}
			\includegraphics[scale=1.2]{img/HXN-5.20a}
		\end{center}
		Gọi C là hình chiếu vuông góc của A (vị trí chiến sĩ xuất phát) đối với bờ bên kia và D thuộc đoạn BC là vị trí mà chiến sĩ sẽ bơi đến trước khi chạy bộ tấn công mục tiêu tại A.
		Ta chuẩn hóa bài toán như sau:
		\begin{itemize}
			\item 1 đơn vị độ dài = $100 m$; khi đó $AC = 1$, $AB = 10$.
			\item Vận tốc bơi trên sông của chiến sĩ là 1 (đơn vị vận tốc); vận tốc chạy của chiến sĩ là 3 (đơn vị vận tốc).
		\end{itemize}
		Đặt $AD = x \in (1; 10) \Rightarrow CD = \sqrt{x^2-1}$; $BC = \sqrt{AB^2 - AC^2} = 3\sqrt{11}$.\\
		$BD = BC - CD = 3\sqrt{11} - \sqrt{x^2-1}$.\\
		Tổng thời gian từ khi chiến sĩ xuất phát đến khi tiếp cận mục tiêu là:
		$$t = \dfrac{AD}{1} + \dfrac{BD}{3} = \dfrac{x}{1} + \dfrac{3\sqrt{11} - \sqrt{x^2-1}}{3} = x + \sqrt{11} - \dfrac{\sqrt{x^2-1}}{3}$$
		Xét hàm $f(x) = x + \sqrt{11} - \dfrac{\sqrt{x^2-1}}{3}$; $x \in (1; 10)$; $f'(x) = 1 - \dfrac{1}{3}\dfrac{2x}{2\sqrt{x^2-1}} = 1 - \dfrac{x}{3\sqrt{x^2-1}}$.\\
		$f'(x) = 0 \Rightarrow 1 - \dfrac{x}{3\sqrt{x^2-1}} = 0 \Rightarrow \dfrac{x}{3\sqrt{x^2-1}} = 1 \Rightarrow x = 3\sqrt{x^2-1} \Rightarrow x^2 = 9(x^2-1) \Rightarrow x^2 = 9x^2 - 9 \Rightarrow 8x^2 = 9 \Rightarrow x^2 = \dfrac{9}{8} \Rightarrow x = \dfrac{3\sqrt{2}}{4}$.\\
		Bảng biến thiên:
		\begin{tabular}{|c|ccccc|}
			\hline
			$x$                           & 1            &            & $\dfrac{3\sqrt{2}}{4}$ &            & 10           \\
			\hline
			$f'(x)$                       & $\vert\vert$ & $-$        & 0                      & $+$        & $\vert\vert$ \\
			\hline
			\rule[-.1in]{0in}{.3in}$f(x)$ &              & $\searrow$ & Min                    & $\nearrow$ &              \\
			\hline
		\end{tabular}
		\\Chiến sĩ tiếp cận mục tiêu nhanh nhất khi $BD = x = \dfrac{3\sqrt{2}}{4}$.\\
		Do đó chiến sĩ sẽ bơi một đoạn $AD = 100 \times \dfrac{3\sqrt{2}}{4} \approx 106 m$.
	}
\end{ex}
\begin{ex}%Câu 17
	\immini[thm]{Một người nghệ sĩ đã vẽ hình chiếc nơ theo một cách khác lạ so với các nhà thiết kế. Anh ta vẽ hình chữ nhật ABCD tâm O có chiều dài bằng 4 dm, chiều rộng bằng 2 dm. Chiếc nơ chính là hình (H) nằm bên trong hình chữ nhật sao cho khi kẻ tia Ot bất kì cắt (H) và cạnh hình chữ nhật lần lượt tại M và N thì $ MN=1$ dm. Tính diện tích chiếc nơ hình (H) đó theo $ d{m^2}$ (làm tròn đến hàng phần chục).
		\shortans{1,52 }}{\includegraphics[scale=1]{img/HXN-5.21}}
	\loigiai{
	\begin{center}
		\includegraphics[scale=1]{img/HXN-5.21a}
	\end{center}
	Xét hình vẽ và các kí hiệu như sau.\\
	Gọi $\varphi=\left(Ox\,,\,\,Ot\right)$ thì $\cos\varphi=\dfrac{OH}{ON}=\dfrac{2}{r_{\varphi}+1}\Rightarrow r_{\varphi} =\dfrac{2}{\cos \varphi} -1$ ; với $r_{\varphi}=OM$ quay quanh gốc O khi $0<\varphi <\widehat{HOA}$ .\\
	Gọi $\theta=\left(Oy\,,\,\,O{t}'\right)$ thì $\cos\theta=\dfrac{OK}{O{N}'}=\dfrac{1}{r_{\theta}+1}$ $\Rightarrow r_{\theta}=\dfrac{1}{\cos \theta}-1 $ ; với $r_{\theta}=O{M}'$ quay quanh gốc O khi $0<\theta <\widehat{AOK}$ .\\
	Ta có: $\tan\widehat{AOK}=\dfrac{AK}{OK}=2\Rightarrow\widehat{AOK}=\arctan 2$ .\\
	Do đó diện tích cần tính là $S=4\cdot\dfrac{1}{2}\cdot\left[\displaystyle\int\limits_0^{\arctan 0,5}{\left(\dfrac{2}{\cos\varphi}-1\right)^2\text{d}\varphi}+\displaystyle\int\limits_0^{act\tan 2}{\left(\dfrac{1}{\cos\theta}-1\right)^2\text{d}\theta}\right]\approx 1,52\,\,d{m^2}$ .}
\end{ex}
\begin{ex}%Câu 18
	Trong không gian với hệ trục tọa độ $Oxyz$ cho ba mặt phẳng: $(P):x-2y+z-1=0$, $(Q):x-2y+z+8=0$, $(R):x-2y+z-4=0$. Một đường thẳng $d$ thay đổi cắt ba mặt phẳng $(P)$, $(Q)$, $(R)$ lần lượt tại $ A$, $ B$, $ C$. Tìm giá trị nhỏ nhất của $ T=A{B^2}+\dfrac{144}{AC}$.\\
	\shortans{108 }
	\loigiai{
	\begin{center}
		\includegraphics[scale=.7]{img/HXN-5.22a}
	\end{center}
	Dựa vào phương trình ba mặt phẳng $(P),\,\,(Q),\,\,(R)$ đã cho, ta thấy chúng song song nhau; so sánh hệ số tự do trong phương trình ba mặt phẳng thì: $-4<-1<8$, do vậy mặt phẳng $(P)$ nằm giữa hai mặt phẳng $(Q),\,\,(R)$.\\
	Ta tính khoảng cách giữa $(P)$ với hai mặt phẳng còn lại:\\ $d\left((P),(Q)\right)=\dfrac{\left| 8-\left(-1\right)\right|}{\sqrt{1^2+\left(-2\right)^2+1^2}}=\dfrac{9}{\sqrt{6}}$ ; $ d\left((P),(R)\right)=\dfrac{\left|-4-\left(-1\right)\right|}{\sqrt{1^2+\left(-2\right)^2+1^2}}=\dfrac{3}{\sqrt{6}}.$\\
	Do vậy $d\left((P),(Q)\right)=3 d\left((P),(R)\right)$.\\
	Gọi $A',\,\,B'$ lần lượt là hình chiếu của $ C$ trên các mặt phẳng $(P),\,\,(Q)$$\Rightarrow C{A}'=\dfrac{3}{\sqrt{6}},\,\,A'{B}'=\dfrac{9}{\sqrt{6}}$. Vì $ A{A}'\text{//}B{B}'$ nên $\dfrac{AC}{AB}=\dfrac{C{A}'}{A'{B}'}=\dfrac{\dfrac{3}{\sqrt{6}}}{\dfrac{9}{\sqrt{6}}}=\dfrac{1}{3}$ hay $AC=\dfrac{1}{3}AC $.\\
		Ta có: $ T=A{B^2}+\dfrac{144}{AC}=A{B^2}+\dfrac{144}{\dfrac{1}{3}AB}=A{B^2}+\dfrac{432}{AB}=A{B^2}+\dfrac{216}{AB}+\dfrac{216}{AB}\overset{AM-GM}{\mathop{\ge}}\,3\sqrt[3]{A{B^2}.\dfrac{216}{AB}.\dfrac{216}{AB}}=108$.\\
		Dấu $ ''=''$ xảy ra khi và chỉ khi $ A{B^2}=\dfrac{216}{AB}\Leftrightarrow A{B^3}=216\Leftrightarrow AB=6$, suy ra $ AC=2$.\\
		Vậy $T_{\min}=108$.}
\end{ex}
\Closesolutionfile{ans}
\inputansbox{6,4,3}{ans/ans-HXN-\sode-T,ans/ans-HXN-\sode-TF,ans/ans-HXN-\sode-SA}
% \setcounter{deso}{15}
% \def\sode{6}
\begin{name}
	{\tenchude}
	{\tendethi}
	{\tentruong}
	{\thoigian}
\end{name}
\caulc
\Opensolutionfile{ans}[ans/ans-HXN-\sode-T]
\begin{ex}%Câu 1
 Trong không gian $ Oxyz$, mặt phẳng đi qua điểm $ K(1\,;\,\,1\,;\,\,1)$ nhận $\vec{u}=(1\,;\,\,0\,;\,\,1)$, $\vec{v}=(1\,;\,\,1\,;\,\,0)$ làm căp vectơ chỉ phương có phương trình tổng quát là
 \choice
 {$ x+y+z-3=0$}
 {$ x-y+z-1=0$}
 {$ x+y-z-1=0$}
 {\True $-x+y+z-1=0$}
 \loigiai{
 Chọn D.\\
 Mặt phẳng có vectơ pháp tuyến $\vec{n}=\left[\vec{u},\vec{v}\right]=\left(-1\,;\\,1\ ;\\,1\right)$.\\
 Phương trình mặt phẳng là $-\left(x-1\right)+1\left(y-1\right)+1\left(z-1\right)=0\Leftrightarrow-x+y+z-1=0$.}
\end{ex}
\begin{ex}%Câu 2
 Cho bảng phân bố tần số ghép nhóm về độ dài của 60 lá dương xỉ trưởng thành như sau:\\
 \centerline{\begin{tabular}{|c|c|c|c|c|}
 \hline
 Độ dài (cm) & $\left[10\,;20\right)$ & $\left[20\,;30\right)$ & $\left[30\,;40\right)$ & $\left[40\,;50\right]$\\
 \hline
 Tần số & $ 8$ & $ 18$ & $ 24$ & $ 10$\\
 \hline
 \end{tabular}}\\
 Tính phương sai của mẫu số liệu ghép nhóm đã cho.
 \choice
 {$s_{}^2=83$}
 {\True $s_{}^2=84$}
 {$s_{}^2=85$}
 {$s_{}^2=86$}
 \loigiai{
 Chọn B.\\
 Ta viết lại bảng trên có bổ sung giá trị đại diện:\\
 \centerline{\begin{tabular}{|c|c|c|c|c|}
 \hline
 Độ dài (cm) & $\left[10\,;20\right)$ & $\left[20\,;30\right)$ & $\left[30\,;40\right)$ & $\left[40\,;50\right]$\\
 \hline
 Giá trị đại diện & $15$ & $25$ & $35$ & $45$\\
 \hline
 Tần số & $ 8$ & $ 18$ & $ 24$ & $ 10$\\
 \hline
 \end{tabular}}\\
 Giá trị trung bình của mẫu số liệu ghép nhóm: $\bar{x}=\dfrac{15\times 8+25\times 18+35\times 24+45\times 10}{60}=31$.\\
 Phương sai mẫu số liệu ghép nhóm là:\\
 $ s_{}^2=\dfrac{8\times{(15-31)^2}+18\times{(25-31)^2}+24\times{(35-31)^2}+10\times{(45-31)^2}}{60}=84$.}
\end{ex}
\begin{ex}%Câu 3
 Cho lăng trụ đều $ABC.A'{B}'{C}'$ . Góc giữa hai vectơ $\overrightarrow{BA}$ và $\overrightarrow{C'{B}'}$ bằng bao nhiêu?
 \begin{center}
\begin{tikzpicture}[scale=.7, font=\footnotesize, line join=round, line cap=round, >=stealth]
    \def\ac{4} % cạnh AC
    \def\ab{2} % cạnh AB
    \def\ben{4} % cạnh bên
    \def\gocnghieng{90} % góc nghiêng cạnh bên
    \def\gocA{50} % góc A của đáy
    \coordinate[label=left:$A$] (A) at (0,0);
    \coordinate[label=right:$C$] (C) at (\ac,0);
    \coordinate[label=below left:$B$] (B) at (-\gocA:\ab);
    \coordinate[label=left:$A'$] (A') at ($(A)+(\gocnghieng:\ben)$);
    \coordinate[label=below left:$B'$] (B') at ($(B)-(A)+(A')$);
    \coordinate[label=right:$C'$] (C') at ($(C)-(A)+(A')$);
    \draw (A')--(A)--(B)--(C)--(C')--(A')--(B')--(C') (B)--(B');
    \draw[dashed] (A)--(C);
    \foreach \diem in {A,B,C,A',B',C'} \fill (\diem)circle(1.5pt);
\end{tikzpicture}

 \end{center}
 \choice
 {$30^\circ $}
 {$60^\circ $}
 {\True $120^\circ $}
 {$90^\circ $}
 \loigiai{
 Chọn C.\\
 Ta có $\left(\overrightarrow{BA}\,,\,\,\,\overrightarrow{C'{B}'}\right)=\left(\overrightarrow{BA}\,,\,\,\overrightarrow{CB}\right)=180^\circ-\widehat{ABC}=180^\circ-60^\circ=120^\circ $ .}
\end{ex}
\begin{ex}%Câu 4
 Thống kê điểm thi đánh giá năng lực của một trường THPT qua thang điểm 100 được cho ở bảng sau:
 \begin{center}
     \begin{tabular}{|l|c|c|c|c|c|}
         \hline
         Điểm & $[0; 20)$ & $[20; 40)$ & $[40; 60)$ & $[60; 80)$ & $[80; 100]$ \\
         \hline
         Số học sinh & 25 & 35 & 37 & 15 & 8 \\
         \hline
     \end{tabular}
 \end{center}
 Trung vị của mẫu số liệu ghép nhóm là giá trị nào sau đây?
 \choice
 {$38,2$}
 {\True $40$}
 {$39,6$}
 {$42$}
 \loigiai{
 Chọn B.\\
 Kích thước mẫu số liệu $ n=25+35+37+15+8=120$.\\
 Trung vị của mẫu số liệu gốc là $\dfrac{x_{60}+x_{61}}{2}$; mà $x_{60}\in\left[20\,;\,\,40\right)\,,\,\,x_{61}\in\left[40\,;\,\,60\right)$ nên trung vị của mẫu số liệu ghép nhóm bằng $M_e$=40.}
\end{ex}
\begin{ex}%Câu 5
 Cho cấp số nhân $\left(u_n\right)$ có tổng $n$ số hạng đầu tiên là $S_n=5^n-1$ với $n\in{\mathbb{N}^*}$ . Tìm số hạng đầu $u_1$ và công bội $q$ của cấp số nhân đó.
 \choice
 {$u_1=5$, $q=4$}
 {$u_1=4$, $q=6$}
 {\True $u_1=4$, $q=5$}
 {$u_1=6$, $q=5$}
 \loigiai{
 Chọn C.\\
 Ta có $u_1=S_1=5^1-1=4$; $u_2=S_2-S_1=(5^2-1)-(5^1-1)=20$.\\
 Công bội cấp số nhân là $ q=\dfrac{u_2}{u_1}=\dfrac{20}{4}=5$.}
\end{ex}
\begin{ex}%Câu 6
 Trong không gian với hệ trục tọa độ $Oxyz$ , cho hai điểm $ A(2;-2;1)$, $B(0;1;2)$. Tọa độ điểm $ M$ thuộc mặt phẳng $\left(Oxy\right)$ sao cho ba điểm $ A,B, M$ thẳng hàng là
 \choice
 {\True $M\left(4\,;\,\,-5\,;\,\,0\right)$}
 {$M\left(2\,;\,\,-3\,;\,\,0\right)$}
 {$M\left(0\,;\,\,0\,;\,\,1\right)$}
 {$M\left(4\,;\,\,5\,;\,\,0\right)$}
 \loigiai{
 Chọn A.\\
 Gọi $ M\left(x\,;\,\,y\,;\,\,0\right)\in\left(Oxy\right)$; $\overrightarrow{AB}=\left(-2\,;\,\,3\,;\,\,1\right);\overrightarrow{AM}=\left(x-2\,;\,\,y+2\,;\,\,-1\right)$.\\
 Ba điểm $ A,B, M$ thẳng hàng $\Leftrightarrow $$\overrightarrow{AB}$ và $\overrightarrow{AM}$ cùng phương $\Leftrightarrow\dfrac{x-2}{-2}=\dfrac{y+2}{3}=\dfrac{-1}{1}$$\Leftrightarrow\left\{\begin{aligned}
 &\dfrac{x-2}{-2}=-1\\ 
 &\dfrac{y+2}{3}=-1\\ 
 \end{aligned}\right.\Leftrightarrow\left\{\begin{aligned}
 & x=4\\ 
 & y=-5\\ 
 \end{aligned}\right.$. Vậy $M\left(4\,;\,\,-5\,;\,\,0\right)$ .}
\end{ex}

\begin{ex}%Câu 7
 Giá trị nhỏ nhất của hàm số $ y=\dfrac{x^2+3}{x-1}$ trên đoạn $\left[2\,;\,\,4\right]$là
 \choice
 {\True $\underset{\left[2\,;\,\,4\right]}{\min}\,y=6$}
 {$\underset{\left[2\,;\,\,4\right]}{\min}\,y=-2$}
 {$\underset{\left[2\,;\,\,4\right]}{\min}\,y=-3$}
 {$\underset{\left[2\,;\,\,4\right]}{\min}\,y=\dfrac{19}{3}$}
 \loigiai{
 Chọn A.\\
 Ta có $y'=\dfrac{2x\left(x-1\right)-\left(x^2+3\right)}{x-1}=\dfrac{x^2-2x-3}{\left(x-1\right)^2}$; $y'=0\Rightarrow{x^2}-2x-3=0\Rightarrow\left[\begin{aligned}
 & x=-1\notin\left(2\,;\,\,4\right)\\ 
 & x=3\in\left(2\,;\,\,4\right)\\ 
 \end{aligned}\right.$.\\
 Ta có:$ y(2)=7$, $ y(3)=6$, $ y(4)=\dfrac{19}{3}$. Vậy $\underset{\left[2\,;\,\,4\right]}{\min}\,y=y(3)=6$.}
\end{ex}

\begin{ex}%Câu 8
 Tập nghiệm của bất phương trình $5^{x-1}\ge{5^{x^2-x-9}}$ là
 \choice
 {\True $\left[-2\,;\,\,4\right]$}
 {$\left[-4\,;\,\,2\right]$}
 {$\left(-\infty\,;\,\,-2\right]\cup\left[4\,;\,\,+\infty\right)$}
 {$\left(-\infty\,;\,\,-4\right]\cup\left[2\,;\,\,+\infty\right)$}
 \loigiai{
 Chọn A.\\
 Ta có: $5^{x-1}\ge{5^{x^2-x-9}}\Leftrightarrow x-1\ge{x^2}-x-9\Leftrightarrow{x^2}-2x-8\le 0\Leftrightarrow-2\le x\le 4$.\\
 Tập nghiệm của bất phương trình là $ S=\left[-2\,;\,\,4\right]$.}
\end{ex}

\begin{ex}%Câu 9
    Diện tích hình phẳng giới hạn bởi hai đường $y = x^2 - 1$ và $y = x - 1$ bằng
    \choice
    {$\dfrac{\pi}{6}$}
    {$\dfrac{13}{6}$}
    {$\dfrac{13\pi}{6}$}
    {\True $\dfrac{1}{6}$}
  \loigiai{
 Chọn D.}
\end{ex}
%
\begin{ex}%Câu 10
 Cho hàm số $y=\dfrac{x-2}{x+3}$ . Mệnh đề nào sau đây đúng?
 \choice
 {Hàm số nghịch biến trên khoảng $\left(-\infty\,;\,\,+\infty\right)$}
 {Hàm số nghịch biến trên từng khoảng $\left(-\infty\,;\,\,-3\right)$ và $\left(-3\,;\,\,+\infty\right)$}
 {\True Hàm số đồng biến trên từng khoảng $\left(-\infty\,;\,\,-3\right)$ và $\left(-3\,;\,\,+\infty\right)$}
 {Hàm số đồng biến trên khoảng $\left(-\infty\,;\,\,+\infty\right)$}
 \loigiai{
 Chọn C.\\
 Tập xác định hàm số $D=\mathbb{R}\setminus\left\{-3\right\}$ . Ta có $y'=\dfrac{5}{\left(x+3\right)^2}>0$ , $\forall x\in D$ .\\
 Vậy hàm số đồng biến trên các khoảng $\left(-\infty\,;\,\,-3\right)$ và $\left(-3\,;\,\,+\infty\right)$ .}
\end{ex}

\begin{ex}%Câu 11
 Cho hai biến cố A và B với $ P(A)=0,3;\,\,P(B)=0,4$ và $ P\left(AB\right)=0,2.$ Xác suất để A hoặc B xảy ra bằng
 \choice
 {$ 0,3$}
 {$ 0,4$}
 {$ 0,6$}
 {\True $ 0,5$}
 \loigiai{
 Chọn D.\\
 Ta có: $ P\left(A\cup B\right)=P(A)+P(B)-P\left(AB\right)=0,3+0,4-0,2=0,5$.}
\end{ex}

\begin{ex}%Câu 12
 Biết $ F(x)=x^2$ là một nguyên hàm của hàm số $ f(x)$ trên $\mathbb{R}$. Giá trị của $\displaystyle\int\limits_1^3\left[1+f(x)\right]\text{d}x$ bằng
 \choice
 {\True $ 10$}
 {$ 8$}
 {$\dfrac{26}{3}$}
 {$\dfrac{32}{3}$}
 \loigiai{
 Chọn A.\\
 Ta có $\displaystyle\int\limits_1^3\left[1+f(x)\right]\text{d}x=\left.\left[x+F(x)\right]\right|_1^3=\left.\left(x+x^2\right)\right|_1^3=12-2=10.$}
 \end{ex}
 
\Closesolutionfile{ans}
\cauds
\Opensolutionfile{ans}[ans/ans-HXN-\sode-TF]
% 
 \begin{ex}%Câu 13
 Khi Mặt Trăng quay quanh Trái Đất, mặt đối diện với Trái Đất thường chỉ được Mặt Trời chiếu sáng một phần. Các pha của Mặt Trăng mô tả mức độ phần bề mặt của nó được Mặt Trời chiếu sáng. Khi góc giữa Mặt Trời, Trái Đất và Mặt Trăng là $\alpha$ ($0^\circ \le \alpha \le 360^\circ$) thì tỉ lệ $F$ của phần Mặt Trăng được chiếu sáng cho bởi công thức $F = \dfrac{1}{2}(1-\cos\alpha)$.
 Biết rằng $F = 0$ khi có trăng mới; $F = 0,25$ khi có trăng lưỡi liềm; $F = 0,5$ khi có trăng bán nguyệt đầu tháng và cuối tháng; $F = 1$ khi trăng tròn. 
 \choiceTF
 {Khi có trăng mới thì $\alpha = 90^\circ$}
 {\True Khi có trăng lưỡi liềm thì $\alpha = 60^\circ$ hoặc $\alpha = 300^\circ$}
 {\True Khi có trăng bán nguyệt đầu tháng hoặc cuối tháng thì $\alpha = 90^\circ$ hoặc $\alpha = 270^\circ$}
 {\True Khi có trăng tròn thì $\alpha = 180^\circ$}
 \loigiai{
 \begin{listEX}
     \item Khi có trăng mới thì $F = \dfrac{1}{2}(1-\cos\alpha) = 0 \Rightarrow \cos\alpha = 1 \Rightarrow \alpha = k360^\circ$, $k \in \mathbb{Z}$; mà $0^\circ \le \alpha \le 360^\circ$ nên $\alpha \in \{0^\circ; 360^\circ\}$.
     \item Khi có trăng lưỡi liềm thì $F = \dfrac{1}{2}(1-\cos\alpha) = 0,25 \Rightarrow \cos\alpha = \dfrac{1}{2}$
     $\Rightarrow \begin{cases} \alpha = 60^\circ + k360^\circ \\ \alpha = -60^\circ + k360^\circ \end{cases}$ ($k \in \mathbb{Z}$); mà $0^\circ \le \alpha \le 360^\circ$ nên $\alpha \in \{60^\circ; 300^\circ\}$.
     \item Khi có trăng bán nguyệt đầu tháng hoặc cuối tháng thì
     $F = \dfrac{1}{2}(1-\cos\alpha) = \dfrac{1}{2} \Rightarrow \cos\alpha = 0 \Rightarrow \alpha = 90^\circ + k180^\circ$, $k \in \mathbb{Z}$;
     mà $0^\circ \le \alpha \le 360^\circ$ nên $\alpha \in \{90^\circ; 270^\circ\}$.
     \item Khi có trăng tròn thì $F = \dfrac{1}{2}(1-\cos\alpha) = 1 \Rightarrow \cos\alpha = -1 \Rightarrow \alpha = 180^\circ + k360^\circ$, $k \in \mathbb{Z}$; mà $0^\circ \le \alpha \le 360^\circ$ nên $\alpha = 180^\circ$.
 \end{listEX}}
 \end{ex}
% 
 \begin{ex}%Câu 14
     Hai nguồn sáng giống hệt nhau được đặt cách nhau 10 mét. Một vật sẽ được đặt tại một điểm $P$ nằm trên một đường thẳng $l$, song song với đường nối hai nguồn sáng và cách đường đó một khoảng $d$ mét (xem hình vẽ).
     Ta muốn đặt điểm $P$ trên đường $l$ sao cho cường độ chiếu sáng tại $P$ là nhỏ nhất. Ta cần biết rằng cường độ chiếu sáng tại điểm đơn lẻ tỉ lệ thuận với cường độ của nguồn và tỉ lệ nghịch với bình phương khoảng cách đến nguồn đó.
     Đặt hệ trục tọa độ $Oxy$ sao cho tâm $O$ trùng với nguồn sáng bên trái, tia $Ox$ chứa đoạn nối hai nguồn sáng, tia $Oy$ hướng lên trên, đơn vị trên mỗi trục là mét.
     \begin{center}
         \includegraphics[scale=1]{img/HXN-6.14}
     \end{center}
     Xét tính đúng sai các mệnh đề sau:
     \choiceTF
     {Khoảng cách từ $P$ đến các nguồn sáng là $r_1 = \sqrt{x^2 + d^2}$; $r_2 = \sqrt{(x+10)^2 + d^2}$}
     {\True Tổng cường độ chiếu sáng tại $P$ là $I(x) = k\left( \dfrac{1}{x^2+d^2} + \dfrac{1}{(x-10)^2+d^2} \right)$; với $k > 0$ là hằng số tỉ lệ}
     {\True Khi $d=5$ mét, cường độ ánh sáng tại $P$ đạt cực tiểu khi $x=5$ mét}
     {\True Khi $d=10$ mét, cường độ ánh sáng không đạt cực tiểu khi $P$ ở vị trí chính giữa của thanh $l$}
 \loigiai{
 \begin{listEX}
     \item Mệnh đề sai.
     Khoảng cách từ $P$ đến các nguồn sáng là $r_1 = \sqrt{x^2+d^2}$; $r_2 = \sqrt{(x-10)^2+d^2}$.
     \item Mệnh đề đúng.
     Cường độ chiếu sáng tại $P$ từ mỗi nguồn là:
     \begin{itemize}
         \item Từ nguồn $O$: $I_1 = \dfrac{k}{r_1^2} = \dfrac{k}{x^2+d^2}$.
         \item Từ nguồn bên phải: $I_2 = \dfrac{k}{r_2^2} = \dfrac{k}{(x-10)^2+d^2}$.
     \end{itemize}
     Tổng cường độ sáng tại $P$ là $I(x) = I_1+I_2 = k\left[\dfrac{1}{x^2+d^2} + \dfrac{1}{(x-10)^2+d^2}\right]$.
     \item Mệnh đề đúng.
     Khi $d=5$ thì $I(x) = k\left[\dfrac{1}{x^2+25} + \dfrac{1}{(x-10)^2+25}\right]$; $I'(x) = k\left[\dfrac{-2x}{(x^2+25)^2} + \dfrac{-2(x-10)}{((x-10)^2+25)^2}\right]$.
     Ta có $I'(x)=0 \Leftrightarrow k\left[\dfrac{-2x}{(x^2+25)^2} + \dfrac{-2(x-10)}{((x-10)^2+25)^2}\right] = 0 \Leftrightarrow x=5$.
     
     Bảng biến thiên:
     \begin{center}
         \begin{tabular}{|c|ccccc|}
             \hline
             $x$ & $0$ & & $5$ & & $10$ \\
             \hline
             $I'(x)$ & & $-$ & $0$ & $+$ & \\
             \hline
         \end{tabular}
     \end{center}
     Ta thấy cường độ ánh sáng tại $P$ đạt cực tiểu khi $x=5$ mét.
     \item Mệnh đề đúng.
     Khi $d=10$ thì $I(x) = k\left[\dfrac{1}{x^2+100} + \dfrac{1}{(x-10)^2+100}\right]$.
     \begin{itemize}
         \item Cách giải 1: Học sinh có thể làm như câu c): lập bảng xét dấu đạo hàm và kết luận.
         \item Cách giải 2: So sánh cường độ sáng tại các điểm đặc biệt.
         \begin{itemize}
             \item Tại $x=0$ thì $I(0) = k\left(\dfrac{1}{100} + \dfrac{1}{200}\right) = \dfrac{3k}{200}$.
             \item Tại $x=5$ thì $I(5) = k\left(\dfrac{1}{200} + \dfrac{1}{100}\right) = \dfrac{2k}{125}$. (Chỗ này trong ảnh là $I(5) = k(\dfrac{1}{200} + \dfrac{1}{100}) = \dfrac{2k}{125}$, nhưng $k(\dfrac{1}{200} + \dfrac{1}{100}) = k(\dfrac{1+2}{200}) = \dfrac{3k}{200}$. Nếu $I(5) = k(\dfrac{1}{125} + \dfrac{1}{25+100-100+25}) = k(\dfrac{1}{125}+\dfrac{1}{50})$? Kiểm tra lại đề gốc của $I(5)$ khi $d=10$. $I(5) = k\left[\dfrac{1}{5^2+100} + \dfrac{1}{(5-10)^2+100}\right] = k\left[\dfrac{1}{25+100} + \dfrac{1}{(-5)^2+100}\right] = k\left[\dfrac{1}{125} + \dfrac{1}{25+100}\right] = k\left[\dfrac{1}{125} + \dfrac{1}{125}\right] = \dfrac{2k}{125}$. Phần tính toán trong ảnh là đúng.)
         \end{itemize}
     \end{itemize}
     Vì $\dfrac{3k}{200} < \dfrac{2k}{125}$ nên cường độ sáng không đạt cực tiểu tại $x=5$ ($P$ không ở chính giữa $l$).
 \end{listEX}}
 \end{ex}
% 
 \begin{ex}%Câu 15
  Hộp A có 5 bi đỏ và 3 bi vàng, hộp B có 2 bi đỏ và 2 bi vàng, hộp C có 2 bi đỏ và 2 bi vàng. Lấy ngẫu nhiên 1 bi từ hộp A bỏ sang hộp B, rồi lấy ngẫu nhiên 2 bi từ hộp B bỏ sang hộp C, sau cùng lấy ngẫu nhiên 3 bi từ hộp C.
 Xét tính đúng sai các mệnh đề sau: 

 \choiceTF
 {Xác suất để lấy được 1 bi vàng từ hộp A bằng $\dfrac{5}{8}$}
 {Xác suất để lấy được 2 bi khác màu từ hộp B bằng $\dfrac{2}{5}$}
 {Xác suất để lấy được cả 3 bi màu đỏ từ hộp C bằng $\dfrac{1}{20}$}
 {\True Biết rằng 3 bi được lấy từ hộp C màu đỏ, xác suất để 3 bi đó vốn thuộc về 3 hộp khác nhau bằng $\dfrac{1}{6}$}
 \loigiai{
 a) Mệnh đề sai.
 Xác suất để lấy được 1 bi vàng từ hộp A bằng $\dfrac{3}{8}$.\\
 b) Mệnh đề sai.
 Ta có sơ đồ bài toán như hình bên. Gọi X là biến cố lấy được hai bi khác màu từ hộp B: $ P(X)=\dfrac{5}{8}\times\dfrac{3}{5}+\dfrac{3}{8}\times\dfrac{3}{5}=\dfrac{3}{5}$.\\
 c) Mệnh đề sai.
 Gọi Y là biến cố lấy được cả 3 bi đỏ từ hộp C, ta có: $ P(Y)=\dfrac{5}{8}\times\left(\dfrac{3}{10}\times\dfrac{C_4^3}{C_6^3}+\dfrac{3}{5}\times\dfrac{C_3^3}{C_6^3}\right)+\dfrac{3}{8}\times\left(\dfrac{1}{10}\times\dfrac{C_4^3}{C_6^3}+\dfrac{3}{5}\times\dfrac{C_3^3}{C_6^3}\right)=\dfrac{3}{40}$.\\
 d) Mệnh đề đúng.
 Gọi Z là biến cố: “Lấy từ hộp C 3 viên bi mà mỗi bi thuộc về mỗi hộp A, B, C trước đó”.\\
 Ta có: $ P\left(Z|Y\right)=\dfrac{P\left(ZY\right)}{P(Y)}=\dfrac{\dfrac{5}{8}\times\dfrac{1\times 2}{C_5^2}\times\dfrac{1\times 1\times 2}{C_6^3}}{\dfrac{3}{40}}=\dfrac{1}{6}$.}
 \end{ex}
%
 \begin{ex}
     \immini[thm]{ Trong một mô hình game 3D, với hệ trục tọa độ $Oxyz$ cho trước, đơn vị trên mỗi trục là mét, mặt đất là mặt phẳng $(Oxy)$. Người chơi cùng với khẩu súng của anh ta được xem như một chất điểm, xuất phát từ vị trí gốc tọa độ $O$ và di chuyển trên mặt đất. Hai mục tiêu nhắm bắn là các vị trí cố định $(5;5;5)$, $(5;10;10)$.
         Để tăng thêm độ hấp dẫn trò chơi, người ta dùng công nghệ hologram tạo ra một quả cầu giả lập bán kính thay đổi, có ánh sáng lấp lánh luôn đi qua hai điểm mục tiêu và lăn trên mặt đất, làm người chơi có phần hoa mắt và khó nhắm bắn trúng các mục tiêu này.
         Xét tính đúng sai các mệnh đề sau:
         }{\includegraphics[scale=1]{img/HXN-6.16}}
   \choiceTF
   {Mục tiêu gần nhất cách người chơi khoảng $8,5~m$ (làm tròn đến hàng phần chục)}
   {\True Biết viên đạn có thể bắn xuyên quả cầu, người chơi cần đứng vị trí $(5;0;0)$ để bắn trúng cùng lúc hai mục tiêu}
   {Gọi $M$ là điểm tiếp xúc của quả cầu với mặt đất thì $M$ luôn thuộc một đường tròn cố định có bán kính bằng $8~m$}
   {\True Khoảng cách ngắn nhất từ vị trí người chơi (gốc tọa độ $O$) đến điểm $M$ bằng $5~m$}
    \loigiai{
    \begin{listEX}
        \item Mệnh đề sai.
        Gọi $A(5;5;5)$, $B(5;10;10) \Rightarrow \overrightarrow{AB}=(0;5;5)=5(0;1;1)$.
        $OA = \sqrt{5^2+5^2+5^2} = 5\sqrt{3}$; $OB = \sqrt{5^2+10^2+10^2} = 15 > OA$.
        Do vậy mục tiêu $A$ gần người chơi nhất, có khoảng cách $OA = 5\sqrt{3} \approx 8,7~m$.
        
        \item Mệnh đề đúng.
        Người chơi cần đứng vị trí $I$ sao cho $I, A, B$ thẳng hàng để bắn trúng cùng lúc hai mục tiêu.
        $AB$ qua $A(5;5;5)$, vectơ chỉ phương $\vec{u}=(0;1;1)$ nên có phương trình $AB: \begin{cases} x=5 \\ y=5+t \\ z=5+t \end{cases}$.
        Gọi $I(5;5+t;5+t) \in AB$; $I \in (Oxy): z=0 \Rightarrow 5+t=0 \Rightarrow t=-5 \Rightarrow I(5;0;0)$.
        
        \item Mệnh đề sai.
        Vì $IM$ là tiếp tuyến của mặt cầu nên $IM^2 = IA \cdot IB$
        $= \sqrt{0^2+5^2+5^2} \cdot \sqrt{0^2+10^2+10^2} = 100$
        $\Rightarrow IM=10$.
        Do vậy $M$ luôn thuộc một đường tròn tâm $I$, bán kính $r=10~m$.
        (Hình vẽ minh họa một mặt cầu tiếp xúc với mặt phẳng $(Oxy)$ tại $M$, với $A, B$ là hai điểm trên mặt cầu, $I$ là tâm đường tròn ngoại tiếp tam giác $ABM'$ với $M'$ là hình chiếu của $M$ trên $AB$, hoặc $I$ là điểm trên $(Oxy)$ sao cho $IA, IB$ là tiếp tuyến tới đường tròn $(M, R)$.)
        
        \item Mệnh đề đúng.
        Ta có $OI=5 < 10 = R$. Vì vậy khoảng cách từ $O$ đến tâm quả cầu ngắn nhất khi $I, O, M$ thẳng hàng theo thứ tự đó.
        Ta có $OM_{\min} = R - OI = 10-5=5$.
        Vậy khoảng cách ngắn nhất từ vị trí người chơi đến điểm tiếp xúc quả cầu, mặt đất bằng $5~m$.
        (Hình vẽ minh họa một đường tròn trên mặt phẳng $(Oxy)$ tâm $I$ bán kính $R$. Điểm $O$ nằm trong đường tròn. $M$ là điểm trên đường tròn sao cho $O, M, I$ thẳng hàng và $OM$ ngắn nhất.)
    \end{listEX}
    \begin{center}
        \includegraphics[scale=.7]{img/HXN-6.16a}
    \end{center}
    }
\end{ex}
\Closesolutionfile{ans}
\caukq
\Opensolutionfile{ans}[ans/ans-HXN-\sode-SA]
% 
 \begin{ex}%Câu 17
 Mai là một cô gái dễ mến, tuy nhiên trong chuyện tình cảm thì cô không phải là người đơn giản. Hôm ấy có người bạn trai hẹn đi ăn trưa, Mai cho biết cô sẽ gặp người đó vào thời điểm kim giờ và kim phút gặp nhau lần đầu tiên kể từ sau 12h trưa. Nếu người bạn trai ấy đến địa điểm hẹn vào lúc 12h30 trưa thì anh ấy sẽ phải chờ Mai bao nhiêu phút (làm tròn đến hàng phần chục)?
 \shortans{35,5}
 \loigiai{
 Mỗi phút kim giờ quét được một góc $\dfrac{2\pi}{60.12}=\dfrac{\pi}{360}$ (rad).\\
 Mỗi phút kim phút quét được một góc $\dfrac{2\pi}{60}=\dfrac{\pi}{30}$ (rad).\\
 Gọi t là thời gian (phút) mà kim giờ và kim phút gặp nhau kể từ 12h trưa.\\
 Ta có $\dfrac{\pi}{360}t+2\pi=\dfrac{\pi}{30}t\Rightarrow t\approx 65,5$ (phút).\\
 Người bạn trai đến điểm hẹn lúc 12h30 nên phải chờ Mai khoảng 35,5 phút.}
 \end{ex}
 
 \begin{ex}%Câu 18
     \immini[thm] {Một vật đựng đồ chơi lắp ghép của trẻ con có dạng hình trụ với chiều cao bằng 15 cm. Người ta nhìn vào bên trong hình trụ này thì thấy có các mảnh ghép hình vuông được đặt vừa khít như hình vẽ. Biết mỗi mảnh ghép hình vuông có cạnh 2 cm. Thể tích vật đựng hình trụ này là bao nhiêu $c{m^3}$ (làm tròn đến hàng đơn vị, bỏ qua độ dày của vỏ hộp).  \shortans{589}}{  \includegraphics[scale=.8]{img/HXN-6.18}}

 \loigiai{
 Dựng hệ trục Oxy như hình vẽ, mỗi hình vuông cạnh 1 đơn vị tương ứng với 2 cm). Gọi phương trình đường tròn đáy hình trụ là $x^2+y^2-2ax-2by+c=0$ với $a^2+b^2-c>0$ .\\
 Các điểm $\left(-4\,;\,\,0\right)\,,\,\,\left(-4\,;\,\,1\right)\,,\,\,\left(2\,;\,\,-2\right)$ thuộc đường tròn đáy nên $\left\{\begin{aligned}
 & 16+0+8a-0+c=0\\ 
 & 16+1+8a-2b+c=0\\ 
 & 4+4-4a+4b+c=0\\ 
 \end{aligned}\right.\Leftrightarrow\left\{\begin{aligned}
 & a=-\dfrac{1}{2}\\ 
 & b=\dfrac{1}{2}\\ 
 & c=-12\\ 
 \end{aligned}\right.$ .\\
 Bán kính đáy hình trụ là: $\sqrt{a^2+b^2-c}=\dfrac{5\sqrt{2}}{2}$ ; bán kính thực tế: $R=\dfrac{5\sqrt{2}}{2}\times 2=5\sqrt{2}$ cm.\\
 Thể tích vật đựng hình trụ là $V=\pi{R^2}h=\pi{\left(5\sqrt{2}\right)^2}\cdot 15=750\pi\approx\,\,c{m^3}$ .}
 \end{ex}
 
 \begin{ex}%Câu 19
     \immini[thm]{ Khi ca sĩ ST bước ra sân khấu, có một đèn pha luôn chiếu sáng vào anh. Đèn pha được đặt ở vị trí B, cách đoạn đường mà ca sĩ đang đi một khoảng BH bằng 10 m. Biết rằng ST Cát đi ra với tốc độ 1 m/s, khi ca sĩ cách điểm H trên con đường khoảng 6 m thì đèn pha đang quay với tốc độ bao nhiêu rad/s (làm tròn đến hàng phần trăm)?
         \shortans{0,07}}{\includegraphics[scale=1]{img/HXN-6.19}}

 \loigiai{
 Gọi $\varphi=\widehat{ABH}$ với $0\le\varphi <\dfrac{\pi}{2}\,\,\,(rad)$ ; đăt $x=HA$ (thay đổi).\\
 Tam giác ABH vuông tại H có $\tan\varphi=\dfrac{AH}{BH}=\dfrac{x}{10}\Rightarrow\,\,\,(1)$ .\\
 Đạo hàm hai vế của (1) theo t, ta được: $\,\,\,(2)$ .\\
 Khi $x=6\,\,m$ thì (1) trở thành $6=10\tan\varphi\Rightarrow\varphi\approx 0,54\,\,rad$ (lưu vào A).\\
 Thay lần lượt $\varphi\approx 0,54\,\,rad$ và $\dfrac{\text{d}x}{\text{d}t}=1$ m/s vào (2) ta được $\dfrac{\text{d}\varphi}{\text{d}t}=\dfrac{5}{68}\approx $ rad/s.}
\end{ex}

\begin{ex}%Câu 20
    \immini[thm]
{ Một con châu chấu nhảy lên cầu thang có 18 bậc. Mỗi lần nhảy con châu chấu có thể nhảy 1 bậc hoặc 2 bậc. Tính xác suất để con châu chấu hoàn thành 18 bậc thang với số lần nhảy 2 bậc không bé hơn số lần nhảy 1 bậc (làm tròn kết quả đến hàng phần trăm).
 \shortans{0,31}}{\includegraphics[scale=1]{img/HXN-6.20}}
\loigiai{
 Gọi x là số bước nhảy 1 bậc, y là số bước nhảy 2 bậc; suy ra $\left\{\begin{aligned}
 & x+2y=18\\ 
 & x\,,\,\,y\in\mathbb{N}\\ 
 \end{aligned}\right.$ .\\
 Các cặp (x ; y) thỏa mãn là $\left(18\,;\,\,0\right)\,,\,\,\left(16\,;\,\,1\right)\,,\,\,\left(14\,;\,\,2\right)\,,\,\,\left(12\,;\,\,3\right)\,,\,\,\left(10\,;\,\,4\right)$ , $\left(8\,;\,\,5\right)\,,\,\,\left(6\,;\,\,6\right)\,,\,\,\left(4\,;\,\,7\right)\,,\,\,\left(2\,;\,\,8\right)\,,\,\,\left(0\,;\,\,9\right)$ .\\
 • Nếu $\left(x\,;\,\,y\right)\in\left\{\left(18\,;\,\,0\right)\,,\,\,\left(0\,;\,\,9\right)\right\}$ thì con châu chấu có 2 cách đi.\\
 • Nếu $\left(x\,;\,\,y\right)=\left(16\,;\,\,1\right)$ thì số cách đi của châu chấu là $C_{17}^1$ .\\
 • Tương tự như vậy các trường hợp còn lại sẽ có số cách là $C_{16}^2+C_{15}^3+C_{14}^4+C_{13}^5+C_{12}^6+C_{11}^7+C_{10}^8=4\,162$ (cách).\\
 Vậy tổng số cách nhảy của châu chấu để thoàn thành 18 bậc cầu thang là $2+C_{17}^1+4162=$ (cách). Số phần tử không gian mẫu là $n\left(\Omega\right)=$ .\\
 Số lần nhảy 2 bậc không bé hơn số lần nhảy 1 bậc nên $\left(x\,;\,\,y\right)\in\left\{\left(6\,;\,\,6\right)\,,\,\,\left(4\,;\,\,7\right)\,,\,\,\left(2\,;\,\,8\right)\,,\,\,\left(0\,;\,\,9\right)\right\}$ .\\
 Gọi A là biến cố thỏa đề bài thì $n(A)=C_{12}^6+C_{11}^7+C_{10}^8+C_9^9=$ .\\
 Do vậy $P(A)=\dfrac{n(A)}{n\left(\Omega\right)}=\dfrac{1\,300}{4\,180}=\dfrac{65}{209}\approx $ .}
 \end{ex}
 
 \begin{ex}%Câu 21
Ông Vượng mới khai phá được một mảnh đất hình chữ nhật, nhà nước chưa cấp sổ nên ông cũng chưa biết rõ diện tích mảnh đất là bao nhiêu, chỉ nhớ rằng bản thân là học sinh giỏi toán 12 năm liền thời phổ thông mà thôi. Mảnh đất của ông Vượng nằm ở một vị trí thuận lợi để trồng trọt vì có một dòng suối nhỏ chảy qua với hình dáng một parabol, dòng suối nhỏ này chia mảnh đất ra làm hai phần có diện tích $S_1\,,\,\,S_2\,\,\,\left(S_1>S_2\right)$ . Riêng mảnh đất có diện tích $S_2$ được xem như hình phẳng giới hạn bởi parabol cùng hai tiếp tuyến vuông góc của parabol đó.
Vào vụ Hè thu, ông Vượng quyết định trồng lúa trên phần đất có diện tích $S_1$ và trồng ớt trên phần đất có diện tích $S_2$ . Dự kiến lợi nhuận mang lại từ việc trồng lúa là $30$ nghìn/ $m^2$ và lợi nhuận từ việc trồng ớt là $40$ nghìn/ $m^2$ (trong một vụ mùa).
Ông quyết định dựng hệ trục Oxy như hình vẽ với gốc O trùng với điểm cực trị của dòng suối dạng parabol, đơn vị trên mỗi trục là 100 mét.
\begin{center}
    \includegraphics[scale=.5]{img/HXN-6.21}
\end{center}
 Tính tổng lợi nhuận theo dự kiến của ông Vượng sau vụ Hè thu này (làm tròn đến hàng đơn vị của triệu đồng), biết rằng diện tích con suối không đáng kể.
\shortans{309}

\loigiai{
Đầu tiên ta đặt $ A\left(a\,;\,\,a^2\right)\,,\,\,B\left(b\,;\,\,b^2\right)\,,\,\,a>0\,,\,\,b<0$ là hai tiếp điểm ứng với hai tiếp tuyến vuông góc của parabol (P).\\
Gọi $d_1\,,\,\,d_2$ lần lượt là các tiếp tuyến của đồ thị $(C)$ tại $ A\,,\,\,B$, khi đó: $\left\{\begin{matrix}
 {d_1}:y=2ax-a^2\\
 {d_2}:y=2bx-b^2\\
\end{matrix}\right.$.\\
Do $d_1\bot{d_2}$ nên $ 2a\cdot 2b=-1\Rightarrow b=-\dfrac{1}{4a}\Rightarrow B\left(-\dfrac{1}{4a}\,;\,\,\dfrac{1}{16a^2}\right)$, khi đó $d_2:y=-\dfrac{x}{2a}-\dfrac{1}{16a^2}$.\\
Gọi $ E=d_2\cap{d_1}$, suy ra $ E\left(\dfrac{4a^2-1}{8a}\,;\,\,-\dfrac{1}{4}\right)$; $ EA=\dfrac{\sqrt{\left(4a+1\right)^3}}{8a}\,;\,\,EB=\dfrac{\sqrt{\left(4a+1\right)^3}}{16a^2}$.\\
Ta có $ EA=2EB$; suy ra $ a=1$. Do đó diện tích mảnh đất $ S=EA\cdot EB=\dfrac{\left(4a+1\right)^3}{128a^3}\left|\begin{aligned}
 &\\ 
 & a=1\\ 
\end{aligned}\right.=$.\\
Khi đó phương trình $d_1:y=2x-1\,;\,\,d_2:y=-\dfrac{x}{2}-\dfrac{1}{16}$ và $ A\left(1\,;\,\,1\right)\,,\,\,B\left(-\dfrac{1}{4}\,;\,\,\dfrac{1}{16}\right)\,,\,\,E\left(\dfrac{3}{8}\,;\,\,-\dfrac{1}{4}\right)$.\\
Diện tích $S_2=\displaystyle\int\limits_{-1/4}^{3/8}{\left[x^2-\left(-\dfrac{x}{2}-\dfrac{1}{16}\right)\right]\text{d}x}+\displaystyle\int\limits_{3/8}^1\left[x^2-\left(2x-1\right)\right]\text{d}x=$ và $S_1=S-S_2=$.\\
Tổng số tiền thu được của ông Vượng sau vụ hè thu bằng $\dfrac{625}{768}\times{10^2}\times 30+\dfrac{125}{768}\times{10^2}\times 40\approx 309\,244$ nghìn đồng $\approx $ triệu đồng.}
\end{ex}

\begin{ex}%Câu 22
    \immini[thm]{Trong không gian $ Oxyz$, cho mặt cầu $(S)$ có tâm $ I\left(-1\,;\,\,0\,;\,\,2\right)$ và đi qua điểm $ A\left(0\,;\,\,1\,;\,\,1\right)$. Xét các điểm $ B\,,\,\,C\,,\,\,D$ thuộc $(S)$ sao cho $ AB,\,\,AC,\,\,AD$ đôi một vuông góc với nhau. Thể tích của khối tứ diện $ ABCD$ có giá trị lớn nhất bằng bao nhiêu (làm tròn đền hàng phần trăm)?
        \shortans{1,33}}{\includegraphics[scale=.7]{img/HXN-6.22}}
\loigiai{
Ta nhận diện được đây là bài toán mặt cầu ngoại tiếp tứ diện có ba cạnh đôi một vuông góc nhau. Bán kính mặt cầu là $ R=IA=\sqrt{3}$.\\
Do $ AB,AC,AD$ đôi một vuông góc với nhau nên $ R=\dfrac{\sqrt{A{B^2}+A{C^2}+A{D^2}}}{2}$.\\
Suy ra $ A{B^2}+A{C^2}+A{D^2}=4R^2=12$.\\
Thể tích tứ diện: $$ V_{ABCD}=\dfrac{1}{6}AB.AC.AD=\dfrac{1}{6}\sqrt{A{B^2}.A{C^2}.A{D^2}}\overset{AM-GM}{\mathop{\le}}\,\dfrac{1}{6}\sqrt{\left(\dfrac{A{B^2}+A{C^2}+A{D^2}}{3}\right)^3}=\dfrac{1}{6}\sqrt{\left(\dfrac{12}{3}\right)^3}=\dfrac{4}{3}$$
Do đó $\left(V_{ABCD}\right)_{Max}=\dfrac{4}{3} $. Dấu đẳng thức xảy ra khi và chỉ khi $ AB=AC=AD=2$.}
\end{ex}
\Closesolutionfile{ans}
\inputansbox{6,4,3}{ans/ans-HXN-\sode-T,ans/ans-HXN-\sode-TF,ans/ans-HXN-\sode-SA}
% \def\sode{7}
\begin{name}
	{\tenchude}
	{\tendethi}
	{\tentruong}
	{\thoigian}
\end{name}
\caulc
\Opensolutionfile{ans}[ans/ans-HXN-\sode-T]
\begin{ex}%Câu 1
 \immini[thm]{ Cho hàm số $y=f(x)$ liên tục trên $\left[-1\,;+\infty\right)$ và có đồ thị như hình vẽ. Tìm giá trị lớn nhất của hàm số $y=f(x)$ trên $\left[1\,;\,\,4\right]$ .
 \choice
 {0}
 {1}
 {4}
 {\True 3}}{\includegraphics[scale=.8]{img/HXN-7.1}}
 \loigiai{
 Chọn D.}
\end{ex}
\begin{ex}%Câu 2
 Cho $\log_ab=2$ (với $a>0\,,\,\,b>0\,,\,\,a\ne 1$). Tính $\log_a\left(a\cdot b\right)$ .
 \choice
 {$2$}
 {$4$}
 {$5$}
 {\True $3$}
 \loigiai{
 Chọn D.\\
 Ta có: $\log_a\left(a\cdot b\right)=\log_aa+\log_ab=1+2=3$ .}
\end{ex}
\begin{ex}%Câu 3
 Trong không gian $Oxyz$ , cho mặt cầu $(S):{(x-1)^2}+(y+2)^2+(z-3)^2=16$ . Tâm của $(S)$ có tọa độ là
 \choice
 {$\left(-1\,;\,\,-2\,;\,\,-3\right)$}
 {$\left(1\,;\,\,2\,;\,\,3\right)$}
 {$\left(-1\,;\,\,2\,;\,\,-3\right)$}
 {\True $\left(1\,;\,\,-2\,;\,\,3\right)$}
 \loigiai{
 Chọn D.\\
 Mặt cầu (S) có tâm $I\left(1\,;\,\,-2\,;\,\,3\right)$ , bán kính $R=4$ .}
\end{ex}
\begin{ex}%Câu 4
 Hai đường tiệm cận của đồ thị hàm số $y=\dfrac{2x+1}{x-1}$ tạo với hai trục tọa độ một hình chữ nhật có diện tích bằng bao nhiêu?
 \choice
 {\True $2$}
 {$1$}
 {$3$}
 {$4$}
 \loigiai{
 Chọn A.\\
 Đồ thị hàm số $y=\dfrac{2x+1}{x-1}$ có đường tiệm cận đứng $x=1$ và đường tiệm cận ngang $y=2$ .\\
 Gọi A là giao điểm của tiệm cận đứng với Ox, suy ra $OA=1$ .\\
 Gọi B là giao điểm của tiệm cận ngang với Oy, suy ra $OB=2$ .\\
 Hình chữ nhật cần tính diện tích là hình chữ nhật OAIB với $I\left(1\,;\,\,2\right)$ là tâm đối xứng đồ thị.\\
 Diện tích hình chữ nhật ABCD là $S=OA\cdot OB=2$ .}
\end{ex}
\begin{ex}
 Doanh thu bán hàng trong 20 ngày được lựa chọn ngẫu nhiên của một cửa hàng được ghi lại ở bảng sau (đơn vị: triệu đồng):
 \begin{center}
\begin{tabular}{|c|c|c|c|c|c|}
    \hline
    Doanh thu & $[5;7)$ & $[7;9)$ & $[9;11)$ & $[11;13)$ & $[13;15)$ \\
    \hline
    Số ngày & 2 & 7 & 7 & 3 & 1 \\
    \hline
\end{tabular}
 \end{center}
 Giá trị trung bình của mẫu số liệu ghép nhóm trên thuộc khoảng nào sau đây?
 \choice
 {$[7; 9)$}
 {\True $[9; 11)$}
 {$[11; 13)$}
 {$[13; 15)$}
 \loigiai{
 Chọn B.
 Ta viết lại mẫu số liệu ghép nhóm có thêm giá trị đại diện như sau:
 \begin{center}
 \begin{tabular}{|c|c|c|c|c|c|}
 \hline
 Doanh thu & $[5;7)$ & $[7;9)$ & $[9;11)$ & $[11;13)$ & $[13;15)$ \\
 \hline
 Giá trị đại diện & 6 & 8 & 10 & 12 & 14 \\
 \hline
 Số ngày & 2 & 7 & 7 & 3 & 1 \\
 \hline
 \end{tabular}
 \end{center}
 Giá trị trung bình của mẫu số liệu ghép nhóm là
 $ \bar{x} = \dfrac{2 \cdot 6 + 7 \cdot 8 + 7 \cdot 10 + 3 \cdot 12 + 1 \cdot 14}{20} = \dfrac{12 + 56 + 70 + 36 + 14}{20} = \dfrac{188}{20} = 9,4 $.
 Vì $9,4 \in [9;11)$ nên đáp án đúng là B.
 }
\end{ex}
\begin{ex}%Câu 6
 Cho cấp số nhân $\left(u_n\right)$ với $u_1=2$ và công bội $q=3$ . Tìm số hạng thứ $4$ của cấp số nhân?
 \choice
 {$24$}
 {\True $54$}
 {$162$}
 {$48$}
 \loigiai{
 Chọn B.\\
 Ta có: $u_4=u_1\cdot{q^3}=2\cdot{3^3}=54.$}
\end{ex}
\begin{ex}%Câu 7
 Cho hai hàm số $y=f(x)$ và $y=g(x)$ liên tục trên $\left[a\,;\,\,b\right]$ . Diện tích hình phẳng giới hạn bởi đồ thị của các hàm số $y=f(x)$ , $y=g(x)$ và các đường thẳng $x=a$ , $x=b$ bằng
 \choice
 {$\left|\displaystyle\int\limits_a^b{\left[f(x)-g(x)\right]\text{d}x}\right|$}
 {$\displaystyle\int\limits_a^b{\left| f(x)+g(x)\right|\text{d}x}$}
 {\True $\displaystyle\int\limits_a^b{\left| f(x)-g(x)\right|\text{d}x}$}
 {$\displaystyle\int\limits_a^b{\left[f(x)-g(x)\right]\text{d}x}$}
 \loigiai{
 Chọn C.}
\end{ex}
\begin{ex}%Câu 8
 Điểm kiểm tra 15 phút của lớp 12A được cho bởi bảng sau:\\
 \centerline{\begin{tabular}{|c|c|c|c|c|c|c|c|}
 \hline
 Điểm &[3; 4) &[4; 5) &[5; 6) &[6; 7) &[7; 8) &[8; 9) &[9; 10)\\
 \hline
 Số học sinh & 3 & 8 & 7 & 12 & 7 & 1 & 1\\
 \hline
 \end{tabular}}\\
 Tứ phân vị thứ nhất của mẫu số liệu ghép nhóm trên (làm tròn đến hàng phần trăm) là
 \choice
 {\True $4,84$}
 {$2,10$}
 {$2,09$}
 {$6,94$}
 \loigiai{
 Chọn A.\\
 Mẫu số liệu ghép nhóm có cỡ mẫu là $n=3+8+7+12+7+1+1=39$ .\\
 Tứ phân vị thứ nhất của mẫu số liệu gốc là $x_{10}\in\left[4\,;\,\,5\right)$ ; do đó tứ phân vị thứ nhất của mẫu số liệu ghép nhóm là $Q_1=4+\dfrac{\dfrac{39}{4}-3}{8}.1=\dfrac{155}{32}\approx 4,84$ .}
\end{ex}
\begin{ex}%Câu 9
 Trong không gian $Oxyz$ , cho đường thẳng $\Delta :\left\{\begin{aligned}
 & x=1-2t\\ 
 & y=-1\\ 
 & z=3+t\\ 
 \end{aligned}\right.$ . Vectơ nào sau đây là một vectơ chỉ phương của đường thẳng $\Delta $ ?
 \choice
 {$(-2\,;\,\,-1\,;\,\,1)$}
 {$(1\,;\,\,-1\,;\,\,3)$}
 {\True $(-2\,;\,\,0\,;\,\,1)$}
 {$(2\,;\,\,0\,;\,\,1)$}
 \loigiai{
 Chọn C.\\
 $\Delta $ có vectơ chỉ phương $\vec{u}=\left(-2\,;\,\,0\,;\,\,1\right)$ .}
\end{ex}
\begin{ex}%Câu 10
 Cho tích phân $\displaystyle\int\limits_0^1\left[f(x)+2x\right]\text{d}x=2$ . Khi đó tích phân $\displaystyle\int\limits_0^1f(x)\text{d}x$ bằng ?
 \choice
 {\True $1$}
 {$4$}
 {$2$}
 {$0$}
 \loigiai{
 Chọn A.\\
 Ta có: $\displaystyle\int\limits_0^1\left[f(x)+2x\right]\text{d}x=2\Leftrightarrow\displaystyle\int\limits_0^1f(x)\text{d}x+\left.x^2\right|_0^1=2\Leftrightarrow\displaystyle\int\limits_0^1f(x)\text{d}x+1=2\Leftrightarrow\displaystyle\int\limits_0^1f(x)\text{d}x=1$ .}
\end{ex}
\begin{ex}%Câu 11
 Trong không gian $Oxyz$ , cho ba điểm $A\left(1\,;\,\,1\,;\,\,1\right)$ , $B\left(0\,;\,\,2\,;\,\,1\right)$ và $C\left(1\,;\,\,-1\,;\,\,2\right)$ . Mặt phẳng đi qua $A$ và vuông góc với $BC$ có phương trình là
 \choice
 {$\dfrac{x+1}{1}=\dfrac{y+1}{-3}=\dfrac{z+1}{1}$}
 {$x-3y+z-1=0$}
 {\True $x-3y+z+1=0$}
 {$\dfrac{x-1}{1}=\dfrac{y-1}{-3}=\dfrac{z-1}{1}$}
 \loigiai{
 Chọn C.\\
 Mặt phẳng qua $A\left(1\,;\,\,1\,;\,\,1\right)$ , có vectơ pháp tuyến $\overrightarrow{BC}=\left(1\,;\,\,-3\,;\,\,1\right)$ nên có phương trình\\
 $1(x-1)-3(y-1)+1(z-1)=0\Leftrightarrow x-3y+z+1=0$ .}
\end{ex}
\begin{ex}%Câu 12
 Họ nguyên hàm của hàm số $f(x)=e^{2x}+\dfrac{3}{x}$ là
 \choice
 {$\mathop{\displaystyle\int}f(x)\text{d}x=e^{2x}+3\text{ln}x+C$}
 {\True $\mathop{\displaystyle\int}f(x)\text{d}x=\dfrac{e^{2x}}{2}+3\text{ln}\left| x\right|+C$}
 {$\mathop{\displaystyle\int}f(x)\text{d}x=\dfrac{e^{2x}}{2}+3\text{ln}x+C$}
 {$\mathop{\displaystyle\int}f(x)\text{d}x=e^{2x}+3\text{ln}\left| x\right|+C$}
 \loigiai{
 Chọn B.\\
 Ta có: $\mathop{\displaystyle\int}f(x)\text{d}x=\mathop{\displaystyle\int}\left(e^{2x}+\dfrac{3}{x}\right)\text{d}x=\dfrac{e^{2x}}{2}+3\text{ln}\left| x\right|+C$ .}
 \end{ex}
 \Closesolutionfile{ans}
 \cauds
 \Opensolutionfile{ans}[ans/ans-HXN-\sode-TF]
\begin{ex}
 Cho hàm số $f(x) = \begin{cases} 3 & \text{khi } x \le 1 \\ ax+b & \text{khi } 1 < x < 2 \\ 5 & \text{khi } x \ge 2 \end{cases}$.
 Xét tính đúng sai các mệnh đề sau:
 \choiceTF
 {Hàm số liên tục trên khoảng $(-\infty; 1)$}
 {Hàm số không liên tục trên khoảng $(1; 2)$}
 {Hàm số liên tục tại $x=1$ khi $a+b=5$}
 {\True Hàm số liên tục trên $\mathbb{R}$ khi và chỉ khi $a=2, b=1$}
 \loigiai{
 \begin{listEX}
 \item Mệnh đề đúng.
 Khi $x<1$ thì $f(x)=3$ là hàm hằng số nên $f(x)$ liên tục trên $(-\infty; 1)$.
 \item Mệnh đề sai.
 Khi $x \in (1;2)$ thì $f(x)=ax+b$ là hàm số bậc nhất (nếu $a$ khác $0$) hoặc là hàm số không đổi (nếu $a=0$), do đó $f(x)$ liên tục trên $(1;2)$.
 \item Mệnh đề sai.
 Ta có: $\lim_{x \to 1^+} f(x) = 3$; $f(1)=3$; $\lim_{x \to 1^-} f(x) = \lim_{x \to 1^-} (ax+b) = a+b$.
 Hàm số liên tục tại $x=1$ suy ra $\lim_{x \to 1^+} f(x) = \lim_{x \to 1^-} f(x) = f(1) \Rightarrow a+b=3$.
 \item Mệnh đề đúng.
 Dễ thấy hàm số $f(x)$ liên tục trên các khoảng $(-\infty;1)$, $(1;2)$ và $(2;+\infty)$. Vì vậy hàm số liên tục trên $\mathbb{R}$ khi và chỉ khi hàm số liên tục tại các điểm $x=1; x=2$.
 Ta có: $\lim_{x \to 2^-} f(x) = \lim_{x \to 2^-} (ax+b) = 2a+b$; $\lim_{x \to 2^+} f(x)=5$; $f(2)=5$.
 Hàm số liên tục tại $x=2$ suy ra $\lim_{x \to 2^-} f(x) = \lim_{x \to 2^+} f(x) = f(2) \Rightarrow 2a+b=5$.
 Kết hợp với câu c) ta có hệ phương trình $\begin{cases} a+b=3 \\ 2a+b=5 \end{cases} \Leftrightarrow \begin{cases} a=2 \\ b=1 \end{cases}$.
 \end{listEX}
 }
\end{ex}
\begin{ex}
 Trên một vùng cao nguyên rộng lớn, với hệ tọa độ $Oxyz$ thích hợp, đơn vị trên mỗi trục tọa độ là 5 mét, một con đại bàng đang đậu trên vách đá phẳng được mô hình hóa bởi phương trình $(P): 2x+2y-z+9=0$. Con đại bàng này đang ngắm các mục tiêu là hai con dê núi đang ở các vị trí $A(1;2;-3)$ và $B(-2;-2;1)$.
 \choiceTF
 {\True Con dê ở vị trí $B$ thuộc vách núi đá nơi đại bàng đang đậu}
 {Khoảng cách giữa hai con dê núi là $\sqrt{41}$ mét}
 {Khoảng cách ngắn nhất từ đại bàng đến con dê ở vị trí $A$ bằng 32 mét}
 {\True Đại bàng luôn quan sát hai con dê với một góc $90^\circ$ và con dê ở vị trí $B$ cũng đã biết được sự nguy hiểm sau lưng nó; khoảng cách xa nhất giữa nó với đại bàng bằng 11,2 mét (làm tròn đến hàng phần chục)}
 \loigiai{
 \begin{listEX}
 \item Mệnh đề đúng.
 Thay tọa độ $B$ vào phương trình $(P): 2x+2y-z+9=0$ thì $2(-2)+2(-2)-(1)+9 = -4-4-1+9=0$ (thỏa mãn).
 Do đó con dê ở vị trí $B$ thuộc vách núi đá nơi đại bàng đang đậu.
 \item Mệnh đề sai.
 Ta có: $\overrightarrow{AB}=(-3;-4;4) \Rightarrow AB = \sqrt{(-3)^2+(-4)^2+4^2} = \sqrt{9+16+16} = \sqrt{41}$.
 Khoảng cách thực tế hai con dê là $5 \cdot AB = 5\sqrt{41}$ mét.
 \item Mệnh đề sai.
 Gọi $H$ là hình chiếu của $A$ trên $(P)$. Đường thẳng $AH$ đi qua $A(1;2;-3)$ và nhận $\vec{n_P}=(2;2;-1)$ làm vectơ chỉ phương.
 Phương trình $AH: \begin{cases} x=1+2t \\ y=2+2t \\ z=-3-t \end{cases}$.\\
 Vì $H \in AH$, tọa độ $H$ có dạng $(1+2t; 2+2t; -3-t)$.\\
 Mà $H \in (P)$ nên $2(1+2t)+2(2+2t)-(-3-t)+9=0 \Rightarrow 2+4t+4+4t+3+t+9=0 \Rightarrow 9t+18=0 \Rightarrow t=-2$.
$ \Rightarrow H(-3;-2;-1)$.
 Khoảng cách ngắn nhất từ đại bàng (trên vách đá $P$) đến con dê $A$ chính là khoảng cách từ $A$ đến mặt phẳng $(P)$, tức là $5 \cdot AH = 5 \cdot 6 = 30$ mét.
 \item Mệnh đề đúng.
 Gọi $M$ là vị trí đại bàng trên vách đá (mặt phẳng $(P)$).
 $\begin{cases} BM \perp AH \\ BM \perp AM \end{cases} \Rightarrow BM \perp (AMH) \Rightarrow BM \perp MH$. Do đó $BM \le BH = \sqrt{5}$\\
Vậy khoảng cách lớn nhất là $5 \cdot BH = 5\sqrt{5} \approx 11,2$ mét.
  \end{listEX}
 }
\end{ex}
\begin{ex}
 Cho ba biến cố $A, B, C$, trong đó các cặp biến cố $A$ và $C$ là độc lập, $B$ và $C$ là độc lập, $A$ và $B$ là xung khắc.
 Biết rằng $P(A \cup C) = \dfrac{2}{3}$, $P(B \cup C) = \dfrac{3}{4}$, $P(A \cup B \cup C) = \dfrac{11}{12}$; đặt $a=P(A), b=P(B), c=P(C)$.
 \choiceTF
 {\True $P(A \cap C) = P(A) \cdot P(C)$; $P(A \cap B) = P(A)+P(B)$}
 {$a+c = \dfrac{2}{3}$; $b+c = \dfrac{3}{4}$}
 {$a+b+c-ac = \dfrac{11}{12}$}
 {Xác suất để $A$ xảy ra nếu $B$ hay $C$ xảy ra bằng $\dfrac{1}{9}$}
 \loigiai{
 \includegraphics[scale=.8]{img/HXN-7.15}
 \begin{listEX}
     \item Mệnh đề đúng.\\
     Vì $A, C$ độc lập nên $P(A \cap C) = P(A) \cdot P(C)$; tương tự $P(B \cap C) = P(B) \cdot P(C)$.\\
     Vì $A$ và $B$ xung khắc nên $P(A \cup B) = P(A)+P(B)$ và $P(A \cap B)=0$.
     \item Mệnh đề sai.
     Ta có: $P(A \cup C) = P(A)+P(C)-P(A \cap C)$
     $= P(A)+P(C)-P(A)P(C) = a+c-ac = \dfrac{2}{3}$ (1);\\
     $P(B \cup C) = P(B)+P(C)-P(B \cap C)$
     $= P(B)+P(C)-P(B)P(C) = b+c-bc = \dfrac{3}{4}$ (2).
     \item Mệnh đề sai.\\
     Ta có: $P(A \cup B \cup C) = P(A)+P(B)+P(C)-P(A \cap B)-P(A \cap C)-P(B \cap C)+P(A \cap B \cap C)$
     $= P(A)+P(B)+P(C)-P(A)P(C)-P(B)P(C) = a+b+c-ac-bc = \dfrac{11}{12}$ (3).\\
     (Dễ thấy vì $A$ và $B$ xung khắc nên $P(A \cap B)=0$ và $P(A \cap B \cap C)=0$).
     \item Mệnh đề sai.
     Lấy (3) trừ (1) và (2) ta được $-c = -\dfrac{1}{2} \Rightarrow c=\dfrac{1}{2}$; (2) suy ra $b=\dfrac{1}{2}$; (1) suy ra $a=\dfrac{1}{3}$.\\
     Do đó: $P(A|B \cup C) = \dfrac{P(A \cap (B \cup C))}{P(B \cup C)}$
     $= \dfrac{P((A \cap B) \cup (A \cap C))}{P(B \cup C)}$
     $= \dfrac{P(A \cap C)}{P(B \cup C)} = \dfrac{P(A)P(C)}{P(B \cup C)} = \dfrac{\dfrac{1}{3}\cdot\dfrac{1}{2}}{\dfrac{3}{4}} = \dfrac{\dfrac{1}{6}}{\dfrac{3}{4}} = \dfrac{2}{9}$.
 \end{listEX}
 }
\end{ex}
\begin{ex}
 \immini[thm]{ Tháp giải nhiệt tại nhà máy Nhiệt điện Phả Lại (Tỉnh Hải Dương, Việt Nam) có mặt cắt qua trục theo phương thẳng đứng là một hình hyperbol (H). Tháp có chiều cao là 120 mét, bán kính đáy dưới bằng 40 mét. Một nhóm kỹ sư đã thiết lập hệ trục tọa độ $Oxy$ như hình vẽ sao cho mặt cắt dạng hypebol của tháp nhận $Ox, Oy$ làm các trục đối xứng; lấy đơn vị trên mỗi trục là mét. Biết rằng đoạn giao nhau giữa trục $Ox$ với tháp bằng 30 mét và gốc $O$ ở vị trí có độ cao 80 mét so với mặt đất.
 }{\includegraphics[scale=1]{img/HXN-7.16}}
 
 \choiceTF
 {\True Diện tích đáy dưới của tháp bằng $5027~m^2$ (làm tròn đến hàng đơn vị)}
 {Các điểm $(-20;0), (20;0)$ thuộc hyperbol $(H)$}
 {Phương trình $(H)$ là $\dfrac{x^2}{15^2} - \dfrac{y^2}{11520} = 1$}
 {\True Thể tích của tháp giải nhiệt này bằng $214414~m^3$ (làm tròn đến hàng đơn vị)}
 \loigiai{
 \includegraphics[scale=1]{img/HXN-7.16a}
 \begin{listEX}
     \item Mệnh đề đúng.
     Bán kính đáy dưới của tháp là $R=40~m$.
     Diện tích đáy dưới tháp $S=\pi R^2 = 1600\pi \approx 5027~m^2$.
     \item Mệnh đề sai.
     Hypebol $(H)$ cắt $Ox$ tại các điểm $(-15;0), (15;0)$.
     \item Mệnh đề sai.
     Gọi phương trình chính tắc của $(H)$ là $\dfrac{x^2}{a^2} - \dfrac{y^2}{b^2} = 1$ ($a>0, b>0$).\\
     Ta có $2a=30 \Rightarrow a=15$; do đó $(H): \dfrac{x^2}{15^2} - \dfrac{y^2}{b^2} = 1$.
     $(H)$ qua điểm $A(40;-80)$ nên $\dfrac{40^2}{15^2} - \dfrac{(-80)^2}{b^2} = 1 \Rightarrow b^2 = \dfrac{11520}{11}$.\\
     Phương trình $(H)$ là $\dfrac{x^2}{15^2} - \dfrac{y^2}{\dfrac{11520}{11}} = 1$.
     \item Mệnh đề đúng.
     Khoảng cách từ $O$ đến nóc bằng $120-80=40$ mét.\\
     Từ câu c) ta có $x^2 = 15^2 \left(1+\dfrac{11y^2}{11520}\right)$; với $x=f(y)$.\\
     Thể tích tháp là $V = \pi \int\limits_{-80}^{40} (f(y))^2 \mathrm{d}y = \pi \int\limits_{-80}^{40} 15^2 \left(1+\dfrac{11y^2}{11520}\right) \mathrm{d}y \approx 214414~m^3$.
 \end{listEX}
 }
\end{ex}
\Closesolutionfile{ans}
\caukq
\Opensolutionfile{ans}[ans/ans-HXN-\sode-SA]
% 
 \begin{ex}%Câu 17
     Trong một lễ hội mùa hè, ba người bạn An, Bình và Cường tham gia cuộc thi xếp tháp ly. Luật chơi như sau: người chơi lần lượt xếp ly vào các tầng của một kim tự tháp chung. An bắt đầu, xếp 1 chiếc ly. Đến lượt Bình, cậu xếp 2 chiếc ly. Cường xếp tiếp 3 chiếc ly. Trở lại lượt An, cậu xếp 4 chiếc ly, rồi Bình xếp 5 chiếc, Cường xếp 6 chiếc... Cuộc thi diễn ra sôi nổi cho đến khi số ly không còn đủ để xếp theo quy luật tăng dần, người đến lượt ở vòng cuối sẽ dùng hết số ly còn lại để hoàn thành tầng của mình (hoặc bắt đầu tầng mới nếu có thể). Sau khi cuộc thi kết thúc, An tự hào khoe rằng mình đã góp tay xếp được khoảng 317 chiếc ly vào ngọn tháp. Hỏi tổng cộng cả ba người bạn đã sử dụng bao nhiêu chiếc ly để xây ngọn tháp đó?
 \shortans{ 933}
 \loigiai{
     Số ly mà An đã xếp là $1; 4;7;\ldots$ tạo thành một cấp số cộng với $u_1=1; d=3$\\
     Sau $n$ lượt, tổng số ly mà An đã xếp là: 
     $$S_n=\dfrac{(u_1+u_n)n}{2}=\dfrac{(3n-1)n}{2}$$ 
     Xét $S_n=317\Rightarrow\dfrac{n(3n-1)}{2}=317\Rightarrow 3n^2-n-634=0\Rightarrow n\approx 14,7$ (không thỏa mãn).\\
     Do đó An là người xếp ly cuối cùng. Sau 14 lượt thì An xếp được $S_{14}=\dfrac{14(3\cdot 14-1)}{2}=287$ ; lượt cuối An xếp thêm $317-287=30$ (ly).\\
     Số ly mà Bình đã xếp được là tổng cấp số cộng có số hạng đầu bằng 2, công sai bằng 3.\\
      Số ly mà Cường đã xếp được là tổng cấp số cộng có số hạng đầu bằng 3, công sai bằng 3.\\
     Tổng số ly cả 3 bạn xếp được là $$317+\dfrac{14\left(2\cdot 2+13\cdot 3\right)}{2}+\dfrac{14\left(2\cdot 3+13\cdot 3\right)}{2}=933$$ .
 }
 \end{ex}
 
 \begin{ex}%Câu 18
 Một khối gỗ có hình dạng của một lăng trụ đứng $ABC.A'{B}'{C}'$ , trong đó $AC=1\,\,m,\,\,BC=2\,\,m,$ $\widehat{ACB}=120^\circ $ . Người thợ mộc đánh dấu điểm $M$ nằm chính giữa đoạn $B{B}'$ . Tính khoảng cách giữa hai đường $AM$ và $C{C}'$ và làm tròn đến hàng phần trăm theo đơn vị mét.
  \shortans{0,65 }
 \loigiai{
     \begin{center}
         \includegraphics[scale=.8]{img/HXN-7.18}
     \end{center}
 Ta có: $C{C}'\text{//}B{B}'\Rightarrow C{C}'\text{//}\left(AB{B}'{A}'\right)$ nên $d\left(C{C}',\,\,\left(AB{B}'{A}'\right)\right)=d\left(C,\left(AB{B}'{A}'\right)\right)$ .\\
 Trong mặt phẳng (ABC), kẻ $CH\perp AB$ tại H (1).\\
 $ABC.A'{B}'{C}'$ là hình lăng trụ đứng nên $A{A}'\perp\left(ABC\right)\Rightarrow CH\perp A{A}'$ (2).\\
 Từ (1) và (2) suy ra $CH\perp\left(AB{B}'{A}'\right)$ $\Rightarrow d\left(C,\,\,\left(AB{B}'{A}'\right)\right)=CH$ .\\
 Xét tam giác $ABC$ có $A{B^2}=C{A^2}+C{B^2}-2.CA.CB.\cos 120^\circ=7$ $\Rightarrow AB=\sqrt{7}$ m.\\
 Diện tích tam giác ABC: $S_{\Delta ABC}=\dfrac{1}{2}CA.CB.\sin C=\dfrac{1}{2}AB.CH$\\ $\Rightarrow CH=\dfrac{CA.CB.\sin{120^0}}{AB}=\dfrac{2.\dfrac{\sqrt{3}}{2}}{\sqrt{7}}=\dfrac{\sqrt{21}}{7}$ m.
 Vậy $d\left(C{C}',\,\,\left(AB{B}'{A}'\right)\right)=CH=\dfrac{\sqrt{21}}{7}\,\,m$ .\\
 Ta có AM và $C{C}'$ là hai đường thẳng chéo nhau mà $\left\{\begin{aligned}
 & C{C}'\text{//}\left(AB{B}'{A}'\right)\\ 
 & AM\subset\left(AB{B}'{A}'\right)\\ 
 \end{aligned}\right.$ nên $d\left(C{C}',\,\,AM\right)=d\left(C{C}',\,\,\left(AB{B}'{A}'\right)\right)=\dfrac{\sqrt{21}}{7}\approx\,0,65\,m$ .}
 \end{ex}
 
 \begin{ex}%Câu 19
 Nhà máy $A$ chuyên sản xuất một loại sản phẩm cung cấp cho nhà máy $B$ . Hai nhà máy thoả thuận rằng: Hàng tháng nhà máy $A$ cung cấp cho nhà máy $B$ số lượng sản phẩm theo đơn đặt hàng của $B$ (tối đa $100$ tấn sản phẩm). Nếu số lượng đặt hàng là $x$ tấn sản phẩm thì giá bán cho mỗi tấn sản phẩm là $P(x)=45-0,001x^2$ (triệu đồng).\\
 Chi phí để $A$ sản xuất $x$ tấn sản phẩm trong một tháng bao gồm:
 \begin{itemize}
    \item Chi phí cố định: $100$ triệu đồng.
   \item  Cho phí cho mỗi tấn sản phẩm làm ra: $30$ triệu đồng.
 \end{itemize}
 Hỏi nhà máy $A$ cần bán cho nhà máy $B$ bao nhiêu tấn sản phẩm mỗi tháng để lợi nhuận thu được là lớn nhất? (Làm tròn kết quả đến hàng phần chục).
  \shortans{ 70,7}
 \loigiai{
 Số tiền mà nhà máy $A$ thu được từ việc bán $x$ tấn sản phẩm $\left(0\le x\le 100\right)$ cho nhà máy $B$ là: $R(x)=x.P(x)=x\left(45-0,001x^2\right)=45x-0,001x^3$ (triệu đồng).\\
 Chi phí để $A$ sản xuất $x$ tấn sản phẩm trong một tháng là $C(x)=100+30x$ (triệu đồng).\\
 Lợi nhuận (triệu đồng) mà nhà máy$A$ thu được là:\\
 $P(x)=R(x)-C(x)=45x-0,001x^3-\left(100+30x\right)=-0,001x^3+15x-100$\\
 Xét hàm số $P(x)=-0,001x^3+15x-100$ với $\left(0\le x\le 100\right)$ ta có:\\
 $P'(x)=-0,003x^2+15\,;\,\,P'(x)=0\Rightarrow{x^2}=5000\Rightarrow x=50\sqrt{2}$ .\\
 Ta có $P(0)=-100;\,\,P\left(50\sqrt{2}\right)=500\sqrt{2}-100\approx 607;\,\,P\left(100\right)=400$ .\\
 Vậy nhà máy$A$ thu được lợi nhuận lớn nhất khi bán khoảng $50\sqrt{2}\approx 70,7$ tấn sản phẩm cho nhà máy $B$ mỗi tháng.}
 \end{ex}
 
 \begin{ex}%Câu 20
     \immini[thm]{ Một cái chậu đựng nước hình bán cầu có bán kính bằng 2 dm. Người ta đặt một ống nhựa và cho nước vào chậu với lưu lượng nước không đổi bằng $0,3$ lít/phút. Đến phút thứ 6, tốc độ dâng lên của nước trong chậu bằng bao nhiêu dm/phút (làm tròn đến hàng phần trăm)?
         \shortans{0,05 }}{\includegraphics[scale=.8]{img/HXN-7.20}}

 \loigiai{
     \begin{center}
         \includegraphics[scale=1.5]{img/HXN-7.20a}
     \end{center}
 Sau 6 phút bơm nước thì thể tích trong bát bằng $6 \times 0,3 = 1,8$ lít.
 
 Gọi $h$ là chiều cao tức thời của mực nước trong chậu, thể tích nước tương ứng chiều cao $h$ được tính theo công thức thể tích chỏm cầu $V=\dfrac{1}{3}\pi h^2 (3R-h)$; trong đó $R=2~dm$ nên
 $$ \boxed{V=\dfrac{1}{3}\pi h^2 (6-h) = 2\pi h^2 - \dfrac{1}{3}\pi h^3 \quad (1)} $$
 Xét $V=1,8 \Rightarrow \dfrac{1}{3}\pi h^2 (3 \cdot 2 - h) = 1,8 \Leftrightarrow h \approx 0,56~dm$ (lưu vào A).
 
 Đạo hàm hai vế của (1) theo $t$, ta được:
 $$ \dfrac{dV}{dt} = (4\pi h - \pi h^2)\dfrac{dh}{dt} \quad (2) $$
 Thay $\dfrac{dV}{dt} = 0,3~dm^3/\text{phút}$; $h=A \approx 0,56~dm$ vào (2), ta được: $\dfrac{dh}{dt} \approx 0,05~dm/\text{phút}$.
 
 Vậy tốc độ dâng lên của nước là khoảng $\boxed{0,05}~dm/\text{phút}$.}
\end{ex}

\begin{ex}%Câu 21
 Vào ngày lễ Tổng kết năm học 2024-2025, tại một trường Tiểu học nghèo ở miền núi, có 10 em học sinh hiếu học được vinh dự nhận 20 phần quà từ các anh chị cựu học sinh của trường nay đã thành đạt. Các phần quà này là đồng giá, gồm có: 9 đôi giày, 7 cái áo và 4 cái cặp; những món quà cùng loại thì giống hệt nhau.
 Trong số 10 em học sinh được nhận quà thì có Bình và Minh là đôi bạn rất thân thiết, tính xác suất để đôi bạn này cùng nhận các món quà như nhau.
  \shortans{0,4 }
\loigiai{
 Gọi x là số cặp quà (giày, áo); gọi y là số cặp quà (giày, cặp); gọi z là số cặp quà (áo, cặp).
 
 Ta có: $\begin{cases} x+y=9 \\ x+z=7 \\ y+z=4 \end{cases} \Leftrightarrow \begin{cases} x=6 \\ y=3 \\ z=1 \end{cases}$.
 
 Số cách tặng quà cho 10 học sinh, mỗi người hai phần khác nhau là: $n(\Omega) = C_{10}^6 \times C_4^3 \times C_1^1$.
 (Tức là chọn 6 học sinh trong 10 học sinh để trao (giày, áo); chọn 3 trong 4 học sinh tiếp theo để trao (giày, cặp); 1 học sinh cuối cùng buộc phải nhận món quà còn lại).
 
 Có hai trường hợp để trao quà cho 10 học sinh mà Bình và Minh được nhận quà như nhau:
 \begin{itemize}
     \item \textbf{Trường hợp 1:} Bình và Minh nhận quà (giày, áo).
     Số cách trao quà là $1 \times 1 \times C_8^4 \times C_4^3 \times C_1^1 = \boxed{280}$ (cách).\\
     (Tức là có 1 cách để Bình và Minh nhận quà; chọn 4 học sinh trong 8 học sinh còn lại tiếp theo nhận (giày, áo) $\rightarrow C_8^4$ (cách); chọn 3 học sinh trong 4 học sinh còn lại nhận (giày, cặp) $\rightarrow C_4^3$ (cách)).
     \item \textbf{Trường hợp 2:} Bình và Minh nhận quà (giày, cặp).
     Số cách trao quà là $1 \times 1 \times C_8^1 \times C_7^6 \times C_1^1 = \boxed{56}$ (cách).\\
     (Tức là có 1 cách để Bình và Minh nhận quà; chọn 1 học sinh trong 8 học sinh tiếp theo nhận (giày, cặp) còn lại $\rightarrow C_8^1$ (cách); chọn 6 học sinh trong 7 học sinh còn lại nhận (giày, áo) $\rightarrow C_7^6$ (cách)).
 \end{itemize}
 Số cách trao quà mà Bình và Minh được nhận quà như nhau là $n(A) = 280+56 = \boxed{336}$.\\
 Vậy xác suất cần tính là $n(A) = 280+56 = \boxed{336}$. $P(A) = \dfrac{n(A)}{n(\Omega)} = \dfrac{336}{C_{10}^6 \times C_4^3 \times C_1^1} = \boxed{0.4}$.}
 \end{ex}
 
 \begin{ex}%Câu 22
Trong không gian với trục tọa độ $Oxyz$ , cho ba điểm $A\left(-1\,;\,-4\,;\,\,4\right)$ , $B\left(1\,;\,\,7\,;\,\,-2\right)$ ; $C\left(1\,;\,\,4\,;\,\,-2\right)$ . Mặt phẳng $(P)$ : $2x+by+cz+d=0$ đi qua điểm $A$ sao cho B và C cùng phía so với (P). Đặt $h_1=d\left(B\,,\,\,(P)\right)$ và $h_2=2d\left(C\,,\,\,(P)\right)$ . Khi đó $h_1+h_2$ đạt giá trị lớn nhất. Tính $T=b+c+d$ .
 \shortans{65}
\loigiai{
    \begin{center}
       \includegraphics[scale=1]{img/HXN-7.22}
    \end{center}
Gọi $D$ là điểm đối xứng với $A$ qua $C$ và $I$ là trung điểm $BD$.\\
Suy ra $D(3;12;-8)$, $I\left(2;\dfrac{19}{2};-5\right)$.
Khi đó $h_1+h_2 = d(B,(P)) + d(D,(P)) = 2d(I,(P)) \le 2IA$.\\
Do vậy $h_1+h_2$ đạt giá trị lớn nhất khi $(P)$ qua $A$ và vuông góc với $IA$.\\
$\overrightarrow{IA}=\left(-3;-\dfrac{27}{2};9\right)=-\dfrac{3}{2}(2;9;-6) \Rightarrow (P)$ nhận $\vec{n}=(2;9;-6)$ làm vec tơ pháp tuyến.\\
Phương trình mặt phẳng $(P): 2x+9y-6z+62=0$.\\
Vậy $b=9; c=-6; d=62 \Rightarrow b+c+d=65$. }
\end{ex}
\Closesolutionfile{ans}
\inputansbox{6,4,3}{ans/ans-HXN-\sode-T,ans/ans-HXN-\sode-TF,ans/ans-HXN-\sode-SA}
% \def\sode{8}
\begin{name}
	{\tenchude}
	{\tendethi}
	{\tentruong}
	{\thoigian}
\end{name}
\caulc
\Opensolutionfile{ans}[ans/ans-HXN-\sode-T]
\begin{ex}%Câu 1
\immini
{
     Hình vẽ bên là đồ thị của hàm số $y=\dfrac{ax+b}{cx+d}$ . Đường tiệm cận đứng của đồ thị hàm số có phương trình là 
 \choice
 {\True $x=1$}
 {$x=2$}
 {$y=1$}
 {$y=2$}
}
{
    \begin{tikzpicture}[>=stealth, line join=round, line cap=round, font=\footnotesize, scale=1, declare function={a=2; b=-1; c=1; d=-1; hsf(\x)=(a*\x+b)/(c*\x+d);},x=.4cm,y=.3cm,thick]
        \draw[->] (-5,0) -- (5,0)node[below]{$x$};
        \draw[->] (0,-5) -- (0,5)node[left]{$y$};
        \draw (0,0) node[below left]{$O$}
        (1,0)node[below right]{$1$}
        (0,2)node[above left]{$2$};
        \draw[dashed] ({-d/c},-5)--({-d/c},5) (-5,{a/c})--(5,{a/c});
        \clip (-5,-5) rectangle (5,5);
        \pgfmathsetmacro{\can}{-d/c}
        \draw[,samples=150,smooth,domain=-5:{\can-.1}] plot(\x,{hsf(\x)});
        \draw[,samples=150,smooth,domain={\can+.1}:5] plot(\x,{hsf(\x)});
    \end{tikzpicture}
}
\end{ex}
\begin{ex}%Câu 2
 Gọi $S$ là diện tích của hình phẳng giới hạn bởi các đường $y=5^x$, $y=0$, $x=0$, $x=2$. Mệnh đề nào dưới đây đúng?
 \choice
 {\True $S=\int\limits_0^25^x\text{d}x$}
 {$S=\pi\int\limits_0^25^{2x}\text{d}x$}
 {$S=\dfrac{1}{\ln 5}\int\limits_0^25^x\text{d}x$}
 {$S=\ln 5\int\limits_0^25^x\text{d}x$}
\end{ex}
\begin{ex}%Câu 3
 Giá trị lớn nhất của hàm số $f(x)=x^3-3x^2-9x+10$ trên đoạn $\left[-2;2\right]$ bằng
 \choice
 {$-12$}
 {$10$}
 {\True $15$}
 {$-2$}
\end{ex}
\begin{ex}%Câu 4
 Cho hàm số $y=f(x)$ có bảng xét dấu đạo hàm như sau:\\
 \centerline{
 \begin{tikzpicture}
     \tkzTabInit[nocadre=false,lgt=1.4,espcl=3,deltacl=0.5]{$x$/.7,$f'(x)$/.7}
     {$-\infty$ , $-1$ , $0$ , $1$ , $+\infty$}
     \tkzTabLine{, - , $0$ , + , $0$ , - , $0$ , + }
 \end{tikzpicture}
 }
 Hàm số$f(x)$ đồng biến trên khoảng nào sau đây?
 \choice
 {$\left(0;1\right)$}
 {$\left(-1;0\right)$}
 {\True $\left(-\infty;-1\right)$}
 {$\left(-1;+\infty\right)$}
\end{ex}
\begin{ex}%Câu 5
 Tập nghiệm của bất phương trình $3^{2x-1}>27$ là
 \choice
 {$\left(\dfrac{1}{2};+\infty\right)$}
 {$\left(3;+\infty\right)$}
 {\True $\left(2;+\infty\right)$}
 {$\left(\dfrac{1}{3};+\infty\right)$}
\end{ex}
\begin{ex}%Câu 6
 Tìm nguyên hàm $F(x)$ của hàm số $f(x)=\sin x+\cos x$ thoả mãn $F\left(\dfrac{\pi}{2}\right)=2$.
 \choice
 {$F(x)=-\cos x+\sin x+3$}
 {$F(x)=-\cos x+\sin x-1$}
 {\True $F(x)=-\cos x+\sin x+1$}
 {$F(x)=\cos x-\sin x+3$}
\end{ex}
\begin{ex}%Câu 7
 Cho cấp số cộng $\left(u_n\right)$ với năm số hạng đầu là $2$; $7$; $12$; $17$; $22$. Số hạng tổng quát của cấp số cộng là
 \choice
 {$u_n=3n+5$}
 {$u_n=3n-5$}
 {$u_n=5n+3$}
 {\True $u_n=5n-3$}
\end{ex}
\begin{ex}%Câu 8
 Bảng dưới đây thống kê cự li ném tạ trong quá trình luyện tập của một vận động viên trong một tuần (đơn vị: mét).\\
 \centerline{\begin{tabular}{|c|c|c|c|c|c|}
 \hline
 Cự li (m) & $\left[19;19,5\right)$ & $\left[19,5;20\right)$ & $\left[20;20,5\right)$ & $\left[20,5;21\right)$ & $\left[21;21,5\right)$\\
 \hline
 Tần số & 13 & 45 & 24 & 12 & 6\\
 \hline
 \end{tabular}}\\
 Hãy tính phương sai của mẫu số liệu ghép nhóm trên (làm tròn đến hàng phần nghìn).
 \choice
 {\True $0{,}277$}
 {$0{,}526$}
 {$0{,}326$}
 {$0{,}211$}
 \end{ex}
\begin{ex}%Câu 9
 Cho hai mặt phẳng $(P)\colon 2x-y-z-3=0$ và $(Q)\colon x-z-2=0$. Góc giữa hai mặt phẳng $(P)$ và $(Q)$ bằng
 \choice
 {\True $30^{\circ}$}
 {$45^{\circ}$}
 {$60^{\circ}$}
 {$90^{\circ}$}
\end{ex}
\begin{ex}%Câu 10
 Khảo sát thời gian tập thể dục của một số học sinh khối 11 thu được mẫu số liệu ghép nhóm sau:\\
\centerline{
\begin{tblr}{colspec={|c|c|c|c|c|c|}, hlines, vlines}
    Thời gian (phút) & [0;20) & [20;40) & [40;60) & [60;80) & [80;100) \\
    Số học sinh & 5 & 9 & 12 & 10 & 6 \\
\end{tblr}
}
 Mốt của mẫu số liệu ghép nhóm trên là
 \choice
 {\True $52$}
 {$42$}
 {$53$}
 {$54$}
\end{ex}
\begin{ex}%Câu 11
 Một vật chuyển động có phương trình $s(t)=3\cos t$ . Khi đó, vận tốc tức thời tại thời điểm $t$ của vật là:
 \choice
 {\True $v(t)=-3\sin t$}
 {$v(t)=-3\cos t$}
 {$v(t)=3\cos t$}
 {$v(t)=3\sin t$}
\end{ex}
\begin{ex}%Câu 12
 Trong không gian $Oxyz$ , một vectơ pháp tuyến của mặt phẳng $\dfrac{x}{-2}+\dfrac{y}{-1}+\dfrac{z}{3}=1$ là
 \choice
 {\True $\vec{n}=(3;6;-2)$}
 {$\vec{n}=(2;-1;3)$}
 {$\vec{n}=(-3;-6;-2)$}
 {$\overrightarrow{n}=(-2;-1;3)$}
 \end{ex}
\Closesolutionfile{ans}
\cauds
\Opensolutionfile{ans}[ans/ans-HXN-\sode-TF]
\begin{ex}%Câu 13
 Cho hàm số $y=f(x)=\dfrac{x-m^2+m}{x+1}$ , $m$ là tham số.
 \choiceTF
 {Với $m=1$ thì hàm số $y=f(x)$ luôn nghịch biến trên các khoảng $\left(-\infty;-1\right)$ và $\left(-1;+\infty\right)$}
 {\True Với mọi số thực $m$ thì hàm số $y=f(x)$ luôn đồng biến trên các khoảng $\left(-\infty;-1\right)$ và $\left(-1;+\infty\right)$}
 {\True $\underset{\left[1;2\right]}{\max}f(x)=f(2)$}
 {\True Có hai giá trị nguyên $m$ để $\underset{\left[0;1\right]}{\min}f(x)=-2$}
\end{ex}
\begin{ex}%Câu 14
\immini
{
    Một vật chuyển động trong $3$ giờ với vận tốc $v$ (km/h) phụ thuộc vào thời gian $t$ (h), đồ thị của hàm vận tốc được cho như hình vẽ. Trong thời gian $1$ giờ kể từ khi vật bắt đầu chuyển động, đồ thị hàm vận tốc của nó là một phần của parabol có đỉnh $S(2;9)$ , khoảng thời gian còn lại đồ thị là một đoạn thẳng song song với trục hoành.
 \choiceTF
 {\True Tại thời điểm bắt đầu chuyển động, vật có vận tốc bằng $4$ km/h}
 {Trong thời gian $1$ giờ kể từ khi bắt đầu chuyển động, phương trình vận tốc của vật là $v(t)=-\dfrac{5}{4}{t^2}-5t+4$}
 {\True Sau $30$ phút kể từ khi bắt đầu chuyển động, gia tốc của vật bằng $3,75$ km/h$^2$}
 {\True Quãng đường $S$ mà vật đi được được trong $3$ giờ (làm tròn đến hàng phần trăm) là $21{,}58$ km}
}
{
    \begin{tikzpicture}[>=stealth, thick, scale=1, declare function={a=-5/4; b=5; c=4; hsf(\x)=a*(\x)^2+b*\x+c;},y=.5cm]
        \draw[->] (-.5,0) -- (3.5,0)node[below]{$x$};
        \draw[->] (0,-.5) -- (0,10)node[left]{$y$};
        \draw (0,0) node[below left]{$O$};
        \draw[dashed]  (2,0) node[below]{$2$} |-(0,9) node[left]{$9$}
        (1,0)node[below]{$1$}--(1,{hsf(1)})
        (3,0)node[below]{$3$}--(3,{hsf(3)});
        \fill (2,{hsf(2)}) circle (1.5pt);
        \draw[samples=150, smooth,line width=1.5pt,blue , domain=0:1] plot(\x,{hsf(\x)})--(3,{hsf(3)});
        \draw[samples=150, smooth, dashed, domain=3:1] plot(\x,{hsf(\x)});
    \end{tikzpicture}
}
\end{ex}
\begin{ex}%Câu 15
Trong một căn phòng có chiều ngang $4$ m, chiều rộng $8$ m và chiều cao $4$ m, người chủ đã thiết kế $4$ dãy đèn led chạy dọc theo các đường chéo của hình chữ nhật tương ứng với các bức tường căn phòng sao cho chúng có tính liên tục. Thiết lập hệ trục tọa độ Oxyz như hình vẽ với căn phòng là hình hộp chữ nhật $ABCD.A'B'C'D'$ , trong đó điểm $A$ là gốc tọa độ, đơn vị trên mỗi trục là mét.
\begin{center}
    \includegraphics[width=5cm]{img/HXN-8-15a}\qquad \includegraphics[width=5cm]{img/HXN-8-15b}
\end{center}
 Chủ căn phòng quyết định sử dụng loại đèn LED neon Flex (không chói mắt) với giá thị trường khoảng 85 nghìn đồng/mét.\\
 \choiceTF
 {Phương trình cạnh BD là $\heva{& x=0\\& y=4+t\\& z=t}$ ($t$ là tham số)}
 {\True Số tiền để mua đèn led trang trí trong căn phòng là $2482$ (nghìn đồng), làm tròn đến hàng đơn vị của nghìn đồng}
 {Khoảng cách từ $D$ đến mặt phẳng $\left(A'B'C'\right)$ bằng $5{,}2$m (làm tròn đến hàng phần chục)}
 {Biết đèn LED có điểm sáng $M$ chạy từ $B$ đến $D$ với tốc độ $0{,}2$ m/s , đèn LED có điểm sáng $N$ chạy từ $A'$ đến $C'$ với tốc độ $0{,}3$ m/s. Sau $9{,}1$ giây (làm tròn đến hàng phần chục) kể từ khi mở nguồn thì hai điểm sáng $M$, $N$ có khoảng cách ngắn nhất trước khi có ít nhất một điểm sáng về đích}
 \loigiai{
 \begin{itemchoice}
     \itemch 
     \itemch 
     \itemch 
     \itemch 
     Ta có $\vec{BD}=(0;-4;4)\Rightarrow \vec{u}=0{,}2\times \dfrac{1}{BD}\vec{BD}=\left(0;-\dfrac{\sqrt{2}}{10};\dfrac{\sqrt{2}}{10}\right)$ là vectơ chỉ phương của $BD$.\\
     Điểm $M\in BD$ nên tại thời điểm $t$, điểm $M$ ở vị trí $M\left( 0;4-\dfrac{\sqrt{2}}{10}t;\dfrac{\sqrt{2}}{10}t \right)$\\
     Ta có $\vec{A'C'}=(0;4;4)\Rightarrow \vec{v}=0{,}3\times \dfrac{1}{A'C'}\vec{A'C'}=\left(0;\dfrac{3\sqrt{2}}{20};\dfrac{3\sqrt{2}}{20}\right)$ là vectơ chỉ phương của $A'C'$.\\
     Điểm $N\in A'C'$ nên tại thời điểm $t$, điểm $N$ ở vị trí $N\left(8;\dfrac{3\sqrt{2}}{20}t;\dfrac{3\sqrt{2}}{20}t\right)$.\\
     Do đó $\vec{MN}=\left(8;\dfrac{\sqrt{2}}{4}t-4;\dfrac{\sqrt{2}}{20}t\right)$\\
     $\Rightarrow MN^2=8^2+\left(\dfrac{\sqrt{2}}{4}t-4\right)^2+\left(\dfrac{\sqrt{2}}{20}t\right)^2=\dfrac{13}{100}t^2-2\sqrt{2}t+80$.\\
     Dễ thấy $MN=\sqrt{\dfrac{13}{100}t^2-2\sqrt{2}t+80}$ đạt giá trị nhỏ nhất $MN_{\min }\approx 8{,}04$; khi đó $t\approx 10{,}9$  giây.
     \end{itemchoice}
 }
\end{ex}
\begin{ex}%Câu 16
Hộp I đựng $3$ bi xanh và $2$ bi vàng; hộp II có $3$ bi xanh, $1$ bi đen và $1$ bi vàng; hộp III có $1$ bi xanh, $1$ bi đen và $3$ bi vàng. Lấy ngẫu nhiên $1$ viên bi từ hộp I bỏ sang hộp II; đồng thời lấy ngẫu nhiên 1 viên bi từ hộp III bỏ sang hộp II; sau đó lấy ngẫu nhiên $2$ viên bi từ hộp II.\\
\centerline{
\includegraphics[width=6cm]{img/HXN-8-16}
}
 \choiceTF
 {Xác suất để hộp thứ II nhận được 2 bi cùng màu bằng $\dfrac{8}{25}$}
 {\True Xác suất để lấy được 2 bi đen từ hộp II bằng $\dfrac{1}{105}$}
 {\True Xác suất để lấy được 2 bi vàng từ hộp II bằng $\dfrac{31}{525}$}
 {Xác suất để lấy được 2 bi từ hộp II cũng là 2 bi được chuyển sang từ hai hộp I, III bằng $\dfrac{5}{31}$ , biết rằng đó là 2 viên bi vàng}
\end{ex}
\Closesolutionfile{ans}
\caukq
\Opensolutionfile{ans}[ans/ans-HXN-\sode-SA]
\begin{ex}%Câu 17
 \immini
 {
     Một người đưa thư xuất phát từ bưu điện (vị trí A) và phải đi qua các con đường để phát thư trước khi quay trở lại bưu điện. Sơ đồ các con đường cần đi qua và độ dài của chúng (tính theo mét) được biểu diễn ở hình vẽ dưới. Hỏi người đó phải đi như thế nào để đường đi là ngắn nhất?
 \shortans{8300}
 }
 {
     \includegraphics[width=6cm]{img/HXN-8-17}
 }
 \end{ex}
 \begin{ex}%Câu 18
Cho đồ thị hàm số $y=\sqrt{x^2-4x+3}$ có các đường tiệm cận xiên $d_1$ và $d_2$ . Tìm tổng khoảng cách từ gốc tọa độ đến hai đường tiệm cận xiên $d_1,d_2$ (kết quả được làm tròn đến hàng phần trăm).
\shortans{2,83}
\end{ex}
\begin{ex}%Câu 19
\immini
{
    Một chiếc bánh kem mừng sinh nhật có dạng hình chóp cụt đều $ABC.A'{B}'{C}'$ với cạnh đáy lớn bằng $4$dm, cạnh đáy nhỏ bằng $2$dm và chiều cao của nó bằng $1{,}5$dm . Tìm thể tích của chiếc bánh kem đó theo đơn vị $d$ m$^3$ (làm tròn đến hàng phần trăm, bỏ qua những thứ trang trí quanh chiếc bánh).
\shortans{6,06}
}
{
    \includegraphics[width=6cm]{img/HXN-8-19-LG}
}
\loigiai{
\immini
{
    Xét hình chóp cụt đều $ABC.A'B'C'$ như hình vẽ; trong đó chiều cao $h=IO=1{,}5$dm.\\
Diện tích hai đáy hình chóp cụt đều $S_1=S_{\triangle ABC}=\dfrac{4^2\sqrt{3}}{4}=4\sqrt{3}\,dm^2$; $S_2=S_{\triangle A'B'C'}=\dfrac{2^2\sqrt{3}}{4}=\sqrt{3}\,dm^2$.\\
Thể tích khối chóp cụt đều: $V=\dfrac{1}{3}h\left(S_1+\sqrt{S_1S_2}+S_2\right)=\dfrac{1}{3}\times 1{,}5\left(4\sqrt{3}+\sqrt{4\sqrt{3}\times \sqrt{3}}+\sqrt{3}\right)=\dfrac{7\sqrt{3}}{2}\approx 6{,}06dm^3$
}
{
    \includegraphics[width=6cm]{img/HXN-8-19-lg}
}
}
\end{ex}
\begin{ex}%Câu 20
\immini
{
    Cho hai nửa đường tròn như hình vẽ bên, trong đó đường kính của nửa đường tròn lớn gấp đôi đường kính của nửa đường tròn nhỏ. Biết rằng nửa hình tròn đường kính AB có diện tích $8\pi $ và $\widehat{ABC}=60^{\circ}$ . 
Tính thể tích vật thể tròn xoay tạo thành khi quay hình phẳng $\mathscr{H}$ (phần được tô đậm) quanh đường thẳng $AB$? Kết quả được làm tròn đến hàng phần trăm.
\shortans{85,9}
}
{
\begin{tikzpicture}[>=stealth, line join=round, line cap=round, font=\footnotesize, scale=1,declare function={R=4;r=2;goc=60;},thick]
    \path
    (0,0) coordinate (A)
    (0:R) coordinate (I)
    ($(I)+(0:R)$) coordinate (B)
    (0:r) coordinate (K)
    ($(I)+(goc:R)$) coordinate (C)
    ;
    \fill[fill=gray] (C) --(B)--(I) arc(0:goc:r)--(C);
    \draw[->] (-1,0)--(9,0)node[below]{$x$};
    \draw[->] (0,-1)--(0,5)node[left]{$y$};
    \draw[red] (A) arc (180:0:r) ;
    \draw[blue] (A) arc(180:0:R);
    \draw (A)--(C)--(B);
    \foreach \x/\g in {A/-135,B/-90,C/70}\draw[fill=white] (\x) circle (1pt)+(\g:3mm) node{$\x$};
    \draw[fill=white] (K) circle (1pt)node[below]{$2$} (I)circle (1pt)node[below]{$4$};
\end{tikzpicture}
}
\loigiai{
    Diện tích nửa đường tròn đường kính AB là $\dfrac{1}{2}\cdot \pi \left(\dfrac{AB}{2}\right)^2=8\pi \Rightarrow AB=8$.\\
    Xét hệ trục tọa độ $Oxy$ như hình vẽ với $O\equiv A$ và tia AB trùng với tia $Ox$.\\
    \centerline{
     \begin{tikzpicture}[>=stealth, line join=round, line cap=round, font=\footnotesize, scale=1,declare function={R=4;r=2;goc=60;},thick]
         \path
         (0,0) coordinate (A)
         (0:R) coordinate (I)
         ($(I)+(0:R)$) coordinate (B)
         (0:r) coordinate (K)
         ($(I)+(goc:R)$) coordinate (C)
         ($(I)+(90:{4/3*sqrt(3)})$)coordinate (M)
         ;
         \fill[fill=none] (C) --(B)--(I) arc(0:goc:r) coordinate (D)--(C);
         \fill[fill=gray] (M)--(I) arc(0:goc:r)--(M);
         \fill[fill=violet] (M)--(C)--(6,0)--(I)--(M);
         \fill[fill=blue!30] (C)--(6,0)--(B)--(C);
         \draw[->] (-1,0)--(9,0)node[below]{$x$};
         \draw[->] (0,-1)--(0,5)node[left]{$y$};
         \draw[red] (A) arc (180:0:r) ;
         \draw[blue] (A) arc(180:0:R);
         \draw (A)--(C)--(B)
         (D)--(C)node[midway,sloped,above]{$y=\dfrac{\sqrt{3}}{3}x$}
         ;
         \draw[dashed] (3,0)node[below]{$3$}--(D)
         (6,0)node[below]{$6$}--(C)
         (4,0)--(M);
         
         \foreach \x/\g in {A/-135,B/-90,C/70}\draw[fill=white] (\x) circle (1pt)+(\g:3mm) node{$\x$};
         \draw[fill=white] (K) circle (1pt)node[below]{$2$} (I)circle (1pt)node[below]{$4$};
     \end{tikzpicture}
    }
    Đường thẳng $AC$ có phương trình là $d\colon y=\dfrac{\sqrt{3}}{3}x$ (vì $AC$ đi qua $A(0;0)$ và có hệ số góc $k=\tan \widehat{BAC}=\tan 30^{\circ }=\dfrac{\sqrt{3}}{3}$).\\
    Gọi $(C)$ là đường tròn tâm $(2;0)$, bán kính $R=2$; khi đó phương trình $(C)\colon (x-2)^2+y^2=4$.\\
    Hoành độ giao điểm giữa $d$ và đường tròn $(C)$ thỏa mãn phương trình $$(x-2)^2+\left(\dfrac{\sqrt{3}}{3}x\right)^2=4\Leftrightarrow \dfrac{4}{3}x^2-4x=0\Leftrightarrow \hoac{& x=0 \\& x=3.} $$
    Đường thẳng $BC$ qua $B(8;0)$, vuông góc $d\colon y=\dfrac{\sqrt{3}}{3}x$ nên có phương trình $y=-\sqrt{3}(x-8)$.\\
    Hai đường thẳng $AC$, $BC$ cắt nhau tại $C$ thỏa hệ $\heva{& y=\dfrac{\sqrt{3}}{3}x \\& y=-\sqrt{3}(x-8) } \Rightarrow C\left(6;2\sqrt{3}\right)$.\\
    Thể tích khối tròn xoay khi quay hình $\mathscr{H}$ quanh $AB$ là 
    $$V=\pi \int\limits_3^4{\left[\dfrac{1}{3}x^2-\left(4-(x-2)^2\right)\right]\mathrm{\,d}x}+\pi \int\limits_4^6{\dfrac{1}{3}x^2\mathrm{\,d}x}+\pi \displaystyle\int\limits_6^8{3(x-8)^2\mathrm{\,d}x}=\dfrac{82}{3}\pi \approx 85{,}9$$
}
\end{ex}
\begin{ex}%Câu 21
\immini
{
    Trên một banner quảng cáo, người ta gắn $17$ chiếc bóng đèn vào một khung hình vuông cũng như hai đường chéo của hình vuông đó. Biết rằng các bóng đèn trên một cạnh hoặc đường chéo thì chia cạnh hoặc đường chéo đó làm các đoạn bằng nhau (xem hình vẽ). Các bóng đèn sẽ sáng lên theo quy luật sau:
\begin{itemize}
    \item Vào phút thứ nhất sẽ có ngẫu nhiên 1 bóng đèn sáng lên, đến cuối phút thứ nhất nó sẽ tắt.
    \item Vào phút thứ 2 sẽ có ngẫu nhiên 2 bóng đèn sáng lên, đến cuối phút thứ hai chúng sẽ tắt.
    \item Vào phút thứ 3 sẽ có ngẫu nhiên 3 bóng đèn sáng lên, đến cuối phút thứ ba chúng sẽ tắt.
\end{itemize}
}
{
    \includegraphics[width=6cm]{img/HXN-8-21}
}
Quy luật này cứ tiếp diễn cho đến phút thứ $17$ và một chu trình mới sẽ được lặp lại. Tính xác suất để từ phút thứ $3$ cho đến phút thứ $17$, luôn có ít nhất $3$ bóng đèn sáng lên ở $3$ đỉnh của một tam giác (làm tròn đến hàng phần trăm).
\shortans{0,84}
\loigiai
{
    \centerline{\includegraphics[width=6cm]{img/HXN-8-21-LG}}
Gọi $A_i$ là biến cố: \lq\lq Tại phút thứ $i$ thì có ít nhất $3$ bóng đèn sáng lên ở $3$ đỉnh của tam giác\rq\rq; khi đó $i\in \left\{ 3;4;\ldots;17 \right\}$.
\begin{itemize}
    \item Phút thứ 3:\\
    Số phần tử không gian mẫu là $n\left(\Omega _3\right)=\mathrm{C}_{17}^3$.\\
    Số khả năng để $3$ bóng đèn sáng lên là $3$ đỉnh tam giác: $n\left(\mathrm{A}_3\right)=\mathrm{C}_{17}^3-2C_7^3-4-10=596$.\\
    (Ta loại trừ các trường hợp $3$ điểm thẳng hàng gồm: $2$ trường hợp $3$ điểm thuộc các đường chéo, $4$ trường hợp $3$ điểm thuộc các cạnh, $10$ trường hợp $3$ điểm thẳng hàng khi vẽ thêm hình).\\
    Xác suất tương ứng là $P\left(\mathrm{A}_3\right)=\dfrac{n\left(\mathrm{A}_3\right)}{n\left(\Omega _3\right)}=\dfrac{598}{\mathrm{C}_{17}^3}=\dfrac{299}{340}$.
    \item Phút thứ 4: Xác suất tương ứng là $P\left(\mathrm{A}_4\right)=\dfrac{\mathrm{C}_{17}^4-2\mathrm{C}_7^4}{\mathrm{C}_{17}^4}=\dfrac{33}{34}$.
    \item Phút thứ 5: Xác suất tương ứng là $P\left(\mathrm{A}_5\right)=\dfrac{\mathrm{C}_{17}^5-2\mathrm{C}_7^5}{\mathrm{C}_{17}^5}=\dfrac{439}{442}$.
    \item Phút thứ 6: Xác suất tương ứng là $P\left(\mathrm{A}_6\right)=\dfrac{\mathrm{C}_{17}^6-2\mathrm{C}_7^6}{\mathrm{C}_{17}^6}=\dfrac{883}{884}$.
    \item Phút thứ 7: Xác suất tương ứng là $P\left(\mathrm{A}_7\right)=\dfrac{\mathrm{C}_{17}^7-2}{\mathrm{C}_{17}^7}=\dfrac{9\,723}{9\,724}$.
    \item Từ phút thứ 8 trở đi thì chắc chắn luôn có ít nhất $3$ bóng sáng lên ở $3$ đỉnh của tam giác.\\
    Xác suất cần tìm là $P(A)=P\left(\mathrm{A}_1\right)\times P\left(\mathrm{A}_2\right)\times \cdot \cdot \cdot \times P\left(\mathrm{A}_{17}\right)\approx0{,}84$.
\end{itemize}
}
\end{ex}
\begin{ex}%Câu 22
Trong không gian với hệ trục tọa độ $Oxyz$, cho ba điểm $A(3;0;0)$, $B(-3;0;0)$ và $C(0;5;1)$. Gọi $M$ là một điểm nằm trên mặt phẳng tọa độ $(Oxy)$ sao cho $MA+MB=10$, giá trị nhỏ nhất của $MC$ bằng bao nhiêu (làm tròn đến hàng phần trăm).
\shortans{1,41}
\loigiai{
    \centerline{\includegraphics[width=6cm]{img/HXN-8-22-LG}}
Hai điểm $A$, $B$ cùng thuộc mặt phẳng $(Oxy)$ và $MA+MB=10>6=AB$.\\
Do vậy, tập hợp điểm $M$ là một elip thuộc mặt phẳng $(Oxy)$ với hai tiểu điểm là $A$ và $B$.\\
Đặt $MA+MB=2a=10\Rightarrow a=5$, $AB=2c=6\Rightarrow c=3$, $b=\sqrt{a^2-c^2}=\sqrt{5^2-3^2}=4$.\\
Do vậy $M\in (E)\colon \dfrac{x^2}{a^2}+\dfrac{y^2}{b^2}=1$ hay $M\in (E)\colon \dfrac{x^2}{25}+\dfrac{y^2}{16}=1$.\\
Gọi $D(0;5;0)$ là hình chiếu của $C$ trên mặt phẳng $(Oxy)$.\\
Khi đó ta có: $CD=\sqrt{0^2+0^2+1^2}=1$ và 
$MC=\sqrt{CD^2+DM^2}=\sqrt{1+DM^2}\,\,\left(*\right)$.\\
Do vậy $MC$ bé nhất khi và chỉ khi $DM$ bé nhất.
Theo hình vẽ, ta thấy khi $M$ trùng với đỉnh elip $(E)$ thuộc tia $Oy$ thì $DM$ bé nhất, hay $M(0;4;0)$.
Suy ra $DM=1$, khi đó $MC=\sqrt{1+1}=\sqrt{2}\approx 1{,}41$.
}
\end{ex}
\Closesolutionfile{ans}
\inputansbox{6,4,3}{ans/ans-HXN-\sode-T,ans/ans-HXN-\sode-TF,ans/ans-HXN-\sode-SA}
% %%%%%%%%%%%%%%%%%%%- HXN
\def\sode{9}
\begin{name}
	{\tenchude}
	{\tendethi}
	{\tentruong}
	{\thoigian}
\end{name}

\caulc
\Opensolutionfile{ans}[ans/ans-HXN-\sode-T]
\begin{ex}%Câu 1
    Cho hàm số $y=f(x)$ có bảng biến thiên như sau:\\
    \centerline{\begin{tikzpicture}[>=stealth]
            \tkzTabInit[nocadre=false,lgt=1,espcl=2.5,deltacl=0.5]{$x$/.7 ,$y'$/.7,$y$/2}
            {$-\infty$ , $-2$ , $3$ , $+\infty$}
            \tkzTabLine{ , + , $0$ , - , $0$ , + , }
            \tkzTabVar{-/$-\infty$ , +/$4$ , -/$-3$ , +/$+\infty$}
    \end{tikzpicture}}
    Tổng giá trị cực đại và cực tiểu của hàm số $y=f(x)$ bằng
    \choice
    {\True $1$}
    {$-3$}
    {$4$}
    {$2$}
\end{ex}

\begin{ex}%Câu 2
    Trong không gian $Oxyz$ , cho mặt cầu $(S)$ có tâm $I(-1;2;1)$ và đi qua điểm $M(3;-1;4)$ . Phương trình của mặt cầu $(S)$ là:
    \choice
    {\True $(x+1)^2+(y-2)^2+(z-1)^2=34$}
    {$(x-1)^2+(y-2)^2+(z+1)^2=16$}
    {$(x+1)^2+(y-2)^2+(z-1)^2=16$}
    {$(x-1)^2+(y+2)^2+(z+1)^2=34$}
\end{ex}

\begin{ex}%Câu 3
    Nếu $\mathop{\int}_0^1f(x)\text{d}x=2$ và $\mathop{\int}_1^0g(x)\text{d}x=-3$ thì $\mathop{\int}_0^1[3f(x)-4g(x)]\text{d}x$ là:
    \choice
    {$6$}
    {\True $-6$}
    {$18$}
    {$12$}
\end{ex}

\begin{ex}%Câu 4
    Cho hình lăng trụ đứng $ABC.A' B' C'$ có đáy là tam giác vuông cân tại $B$ và $AB=6$ . Tính khoảng cách từ điểm $C$ đến mặt phẳng $(ABB'A')$
    \choice
    {$3$}
    {$6\sqrt{2}$}
    {$3\sqrt{2}$}
    {\True $6$}
\end{ex}

\begin{ex}%Câu 5
Đường tiệm cận xiên của đồ thị hàm số $ f(x)=\dfrac{2x^2-3x+1}{x+1}$ có phương trình là
\choice
{\True $y=2x-5$}
{$y=2x+5$}
{$y=2x-3$}
{$y=2x+3$}
\end{ex}

\begin{ex}%Câu 6
Trong không gian $ Oxyz$, cho vectơ $\vec{a}=(-1;0;2)$. Mệnh đề nào dưới đây đúng?
\choice
{$2\vec{a}=(2;0;-4)$}
{$2\vec{a}=(-2;0;-4)$}
{\True $2\vec{a}=(-2;0;4)$}
{$2\vec{a}=(2;0;4)$}
\end{ex}

\begin{ex}%Câu 7
Cho hàm số $ y=f(x)$ có đạo hàm $f'(x)=(x-2)(x+1)$, $\forall x\in\mathbb{R}$. Mệnh đề nào dưới đây đúng?
\choice
{Hàm số đồng biến trên $(-1;+\infty)$}
{Hàm số đồng biến trên $(-\infty;2)$}
{Hàm số nghịch biến trên $(-1;2)$}
{\True Hàm số nghịch biến trên $(-1;2)$}
\end{ex}

\begin{ex}%Câu 8
Trong không gian $ Oxyz,$ cho đường thẳng $ d\colon \dfrac{x-1}{2}=\dfrac{y+1}{1}=\dfrac{z}{-3}$. Mặt phẳng $(P)$ đi qua điểm $ A(1;0;1)$ và vuông góc với đường thẳng $d$ có phương trình là:
\choice
{\True $2x+y-3z+1=0$}
{$2x+y-3z-1=0$}
{$x+z+1=0$}
{$x+z-1=0$}
\end{ex}

\begin{ex}%Câu 9
Một người thống kê lại thời gian thực hiện các cuộc gọi điện thoại của người đó trong một tuần ở bảng sau: (đơn vị: giây)\\
\centerline{\begin{tblr}{|c|c|c|c|c|c|c|}
        \hline
        {Thời gian} & $[0;60)$ & $[60;120)$ & $[120;180)$ & $[180;240)$ & $[240;300)$ & $[300;360)$\\
        \hline
        Số cuộc gọi & $ 9$ & $ 9$ & $ 5$ & $ 7$ & $ 2$ & $ 1$\\
        \hline
\end{tblr}}\\
Khoảng tứ phân vị của mẫu số liệu ghép nhóm trên bằng
\choice
{$180$}
{\True $40$}
{$60$}
{$169$}
\end{ex}

\begin{ex}%Câu 10
Tập nghiệm $S$ của bất phương trình $\log_{\tfrac{1}{2}}\left(x-3\right)\ge\log_{\tfrac{1}{2}}4$ là
\choice
{\True $S=\left(3;7\right]$}
{$S=\left[3;7\right]$}
{$S=\left(-\infty;7\right]$}
{$S=\left[7;+\infty\right)$}
\end{ex}

\begin{ex}%Câu 11
Một công ty thống kê lương của nhân viên theo tuần (đơn vị: USD) theo bảng sau:\\
\centerline{
\begin{tblr}{
        colspec = {|c|c|c|c|c|c|},
    }
    \hline
    Lương theo tuần (USD) & [10; 20) & [20; 30) & [30; 40) & [40; 50) & [50; 60] \\\hline
    Số công nhân & 4 & 6 & 10 & 20 & 10 \\\hline
\end{tblr}
}
Độ lệch chuẩn của mẫu số liệu này bằng bao nhiêu (làm tròn tới hàng phần chục)?
\choice
{\True $11{,}7$}
{$12$}
{$11{,}4$}
{$12{,}5$}
\end{ex}

\begin{ex}%Câu 12
Cho hàm số $f(x)=x^2-\dfrac{4}{x}$. Giá trị của $\int\limits_1^2f'(x)\mathrm{d}x$ bằng
\choice
{$\dfrac{7}{3}-\text{ln}2$}
{$3$}
{$\dfrac{7}{3}$}
{\True $5$}
\end{ex}

\Closesolutionfile{ans}
\cauds
\Opensolutionfile{ans}[ans/ans-HXN-\sode-TF]

\begin{ex}%Câu 13
\immini
{
    Cho hàm số $y=f(x)$ xác định trên $\mathbb{R}\setminus\left[-1;1\right]$ liên tục trên mỗi khoảng xác định và có bảng biến thiên như sau
}
{
    \begin{tikzpicture}[>=stealth]
        \tkzTabInit[nocadre=false,lgt=1,espcl=2.5,deltacl=0.5]{$x$/.7 ,$y'$/.7,$y$/2}
        {$-\infty$ , $-1$,$1$ , $+\infty$}
        \tkzTabLine{ , +,t,h,t , - , }
        \tkzTabVar{-/$2$,+DH/$+\infty$, D+/$0$,-/$-2$}
    \end{tikzpicture}
}
\choiceTF
{Đồ thị hàm số $y=f(x)$ có đường tiệm cận đứng $x=1$}
{\True Đồ thị hàm số $y=f(x)$ có đúng hai đường tiệm cận ngang}
{\True Đồ thị hàm số $y=f(x)$ không có đường tiệm cận xiên}
{Đồ thị hàm số $y=\dfrac{1}{f(x)+1}$ có tất cả bốn đường tiệm cận}
\loigiai{
\begin{itemchoice}
    \itemch 
    \itemch 
    \itemch 
    \itemch Đặt $g(x)=\dfrac{1}{f(x)+1}$. \\
    Ta có $\lim\limits_{x\to -\infty }g(x)=\dfrac{1}{2+1}=\dfrac{1}{3}$ (vì $f(x)\to 2$); $\lim\limits_{x\to +\infty }g(x)=\dfrac{1}{-2+1}=-1$ (vì $f(x)\to -2$).\\
    Vì vậy đồ thị hàm số $y=g(x)$ có hai tiệm cận ngang $y=\dfrac{1}{3}$; $y=-1$.\\
    Xét $f(x)+1=0\Leftrightarrow f(x)=-1$. Phương trình này có một nghiệm thuộc khoảng $\left(1;+\infty \right)$.\\
    Do đó đồ thị hàm số $y=g(x)$ có một tiệm cận đứng.\\
    Vậy đồ thị hàm số $y=g(x)$ có tất cả ba đường tiệm cận.
\end{itemchoice}
}
\end{ex}

\begin{ex}%Câu 14
Có hai tên cướp vừa lấy được một chiếc ca nô ở vị trí $A$ thuộc bờ sông, chúng liền cho ca nô chạy theo phương hợp với bờ sông một góc $60^\circ$ với vận tốc $v=2t$ (mét/giây), trong đó $t$ (giây) là thời gian kể từ khi xuất phát. Sau $21$ giây, ca nô đến vị trí $B$ và chúng quyết định chuyển hướng cho ca nô chuyển động thẳng đều theo phương song song với bờ sông, tầm nửa phút sau thì ca nô đến $C$ (tham khảo hình vẽ).\\
\centerline{
    \includegraphics[width=8cm]{img/HXN-9-14}
}
    \choiceTF
    {\True Vị trí $B$ mà ca nô bọn cướp chuyển hướng cách bờ sông khoảng $382$ m (làm tròn đến hàng đơn vị)}
    {Khoảng cách $A$, $C$ tính theo đường chim bay bằng $1522$ m (làm tròn đến hàng đơn vị)}
    {\True Nếu các chiến sĩ công an khởi động ca nô và đi thẳng từ $D$ đến $C$ với vận tốc được tăng thêm $3$ m sau mỗi giây thì sau $21$ giây sẽ đến vị trí $D$ (làm tròn đến hàng đơn vị của giây)}
    {Trên thực tế các chiến sĩ đã chọn giải pháp cho ca nô khởi động và di chuyển vuông góc với bờ với gia tốc $a$ dương, cùng lúc đó bọn cướp từ vị trí $D$ tiến thẳng về phía trước (giữ nguyên hướng đi và tốc độ), hai bên giáp mặt nhau khi $a=4{,}78$ m/s$^2$ (làm tròn đến hàng phần trăm)}
    \loigiai{
        \begin{center}
            \begin{tikzpicture}[declare function={d=6;b=2;c=3;},thick]
                \path
                (0,0) coordinate (A)
                (-60:b) coordinate (B)
                ($(B)+(0:c)$) coordinate (C)
                (0:d) coordinate (D)
                ($(A)!(B)!(D)$) coordinate (H)
                ($(A)!(C)!(D)$) coordinate (K)
                ($(B)!(D)!(C)$) coordinate (E)
                pic[draw,angle radius=5mm,"$60^\circ$",angle eccentricity=1.7]{angle=B--A--H}
                ;
                \foreach \x/\y/\z in {B/H/K,C/K/D,C/E/D}\draw pic[draw,angle radius =2mm] {right angle = \x--\y--\z};
                \draw[dashed] (B)--(H) (C)--(K) (C)--(E)--(D);
                \draw (A)--(B)--(C) (D)--(A);
                \draw[red] (A)--(C)--(D);
                \foreach \x/\g in {A/180,B/-90,C/-90,D/0,E/-90,H/90,K/90}\draw[fill=white] (\x) circle (1pt)+(\g:3mm) node{$\x$};
            \end{tikzpicture}
        \end{center}
    \begin{itemchoice}
        \itemch Sau $21$ giây, ca nô bọn cướp đi được $AB=\int\limits_0^{21}{2t\mathrm{d}t}=441m$.\\
        Gọi $H$ là hình chiếu vuông góc của $B$ trên bờ sông, khi đó $BH=AB\sin 60^\circ=\dfrac{441\sqrt{3}}{2}\approx 382m$.
        \itemch Vận tốc của ca nô bọn cướp tại $B$ là $v_B=2\times 21=42$ m/s.\\
        Khoảng cách hai vị trí $B$, $C$ là $BC=42\times 30=1260m$.\\
        Khoảng cách hai vị trí $A$, $H$ là $AH=AB\cos 60^\circ=220{,}5m$.\\
        Gọi $K$ là hình chiếu vuông góc của $C$ trên bờ sông thì $HK=BC=1260m$.\\
        Do đó $AK=AH+HK=1480{,}5m$; $CK=BH=\dfrac{441\sqrt{3}}{2}m$ và $AC=\sqrt{AK^2+CK^2}\approx 1529 m$.
        \itemch Ta có $DK=2000-\left(220{,}5+1260\right)=519{,}5m$; suy ra $CD=\sqrt{CK^2+DK^2}\approx 644{,}8m$.\\
        Thời gian để các chiến sĩ đi được từ $D$ đến $C$ thỏa $\displaystyle\int\limits_0^t{3t\mathrm{d}t}=\sqrt{415741}\Rightarrow t\approx 21s$.
        \itemch Gọi $E$ là vị trí hai bên giáp mặt nhau (nếu có) thì tam giác $CDE$ vuông tại $E$.\\
        Khi đó $CE=DK=519{,}5m$ và $DE=\dfrac{441\sqrt{3}}{2}m$.\\
        Thời gian để ca nô bọn cướp đi từ $C$ đến $E$ là $\dfrac{CE}{42}=\dfrac{1039}{84}\approx 12{,}37s$.\\
        Gia tốc $a$ thỏa mãn $\displaystyle\int\limits_0^{\tfrac{1039}{84}}{at \mathrm{d}t}=\dfrac{441\sqrt{3}}{2}\Rightarrow 4{,}99m/s^2$.
    \end{itemchoice}
    }
\end{ex}

\begin{ex}%Câu 15
    \immini
    {
        Một chiếc máy bay thương mại Comac C919 đang bay trên bầu trời theo một đường thẳng từ $D$ đến $E$ có hình chiếu trên mặt đất là đoạn $CB$. Tại vị trí $D$ thì máy bay bay cách mặt đất $9000$m, tại vị trí $E$ thì máy bay cách mặt đất $12000$ m. Một ra đa được đặt trên mặt đất tại vị trí $O$ cách $C$ khoảng $20000$m, cách $B$ khoảng $16000$m và $\widehat{BOC}=90^\circ$; phạm vi theo dõi của ra đa là $20$ km.\\
        Xét hệ trục tọa độ $ Oxyz$ (đơn vị trên mỗi trục là $ 1000$ m) với $ O$ là vị trí đặt ra đa, $ B$ thuộc tia $Oy$, $C$ thuộc tia $ Ox$.
    }
    {
        \includegraphics[width=5cm]{img/HXN-9-15}
    }
    \choiceTF
    {Tại $ D$, máy bay cách ra đa $23000$m (làm tròn đến hàng nghìn theo đơn vị mét)}
    {\True Khi máy bay bay đến điểm $ I$ (m là trung điểm của $AB$), máy bay cách mặt đất $10500$ m}
    {\True Trên hành trình bay từ $ D$ đến $E$, máy bay sẽ đi qua điểm có tọa độ$ P(16;3,2;9,6)$}
    {Khoảng cách giữa vị trí đầu tiên và vị trí cuối cùng mà máy bay bay trong phạm vi theo dõi của ra đa là $22000$m (làm tròn đến hàng trăm theo đơn vị mét)}
    \loigiai{
        Tọa độ các điểm là$O(0;0;0)\,,B(0;16;0)\,,C(20;0;0)\,,D(20;0;9)\,,E(0;16;12)$.
        \begin{itemchoice}
            \itemch Ta có $\vec{OD}=(20;0;9) \Rightarrow OD=\sqrt{20^2+9^2}=\sqrt{481}$.\\
            Khi ở $D$, khoảng cách giữa máy bay và ra đa là $1000\times \sqrt{481}\approx 22\,000$ m.
            \itemch Tọa độ trung điểm của $DE$ là $I\left(10;8;\dfrac{21}{2}\right)$; cao độ điểm I là $\dfrac{21}{2}=10{,}5$.\\
            Vậy khi bay đến điểm $I$, máy bay cách mặt đất $10{,}5$ km hay $10\,500$m.
            \itemch Ta có $\vec{DE}=(-20;16;3)$. Phương trình đường thẳng $DE$ là $\heva{& x=20-20t \\& y=16t \\& z=9+3t } $ ($t$ là tham số thực).\\
            Thay tọa độ $P(16;3{,}2;9{,}6)$ vào phương trình $DE$ ta được $\heva{& 16=20-20t \\& 3{,}2=16t \\& 9{,}6=9+3t } \Leftrightarrow t=0{,}2\in (0;1)$.\\
            Do đó trên hành trình bay từ $D$ đến $E$, máy bay sẽ đi qua điểm $P(16;3{,}2;9{,}6)$.
            \itemch Gọi $H(20-20t;16t;9+3t)\in DE$ là hình chiếu vuông góc của $O$ trên $DE$.\\
            \centerline{\includegraphics[width=5cm]{img/HXN-9-15-LG}}
            Hai vectơ $\heva{& \vec{OH}=(20-20t;16t;9+3t) \\& \vec{DE}=(-20;16;3) } $ vuông góc với nhau nên\\
            $\vec{OH}\cdot \vec{DE}=0 \Leftrightarrow -20(20-20t)+16\cdot 16t+3(9+3t)=0\Leftrightarrow t=\dfrac{373}{665}$.\\
            Khi đó $\vec{OH}=\left(\dfrac{1\,168}{133};\dfrac{5\,968}{665};\dfrac{7\,104}{665}\right)$ và $OH=\dfrac{16\sqrt{469490}}{665}$.\\
            Gọi $M$, $N$ lần lượt là điểm đầu tiên và điểm cuối cùng mà máy bay xuất hiện trong phạm vi theo dõi của ra đa, ta có $MN=2\sqrt{R^2-OH^2}=2\sqrt{20^2-\dfrac{180736}{665}}=\dfrac{584\sqrt{665}}{665}\approx 22\,600$m.
        \end{itemchoice}
    }
\end{ex}


\begin{ex}%Câu 16
\immini
{
    Biết rằng $5\%$ cá heo trong một khu bảo tồn mắc một bệnh nhất định. Một xét nghiệm chẩn đoán có thể được áp dụng để xác định cá heo có mắc bệnh hay không:
\begin{itemize}
    \item Nếu cá heo mắc bệnh, xét nghiệm cho kết quả dương tính với xác suất $0{,}96$.
    \item Nếu cá heo không mắc bệnh, xét nghiệm cho kết quả dương tính với xác suất $0{,}02$.
\end{itemize}
Xét nghiệm được áp dụng cho một pé cá heo được chọn ngẫu nhiên.
}
{
    \includegraphics[width=5cm]{img/HXN-9-16}
}
    \choiceTF
    {Xác suất để thu được kết quả dương tính bằng $0{,}07$}
    {Biết rằng kết quả dương tính thu được, xác suất để cá heo này thực sự mắc bệnh bằng $\dfrac{45}{67}$}
    {\True Cá heo khi xét nghiệm đã cho kết quả dương tính lần đầu, xác suất để khi xét nghiệm lần tiếp theo vẫn cho kết quả dương tính bằng $ 0{,}69$ (làm tròn đến hàng phần trăm)}
    {Biết rằng kết quả xét nghiệm lần thứ hai là dương tính, xác suất để cá heo này thực sự mắc bệnh bằng $0{,}97$ (làm tròn đến hàng phần trăm)}
    \loigiai{
        Gọi $A$ là biến cố: \lq\lq Cá heo thực sự mắc một bệnh nhất định\rq\rq.\\
        $B_i$ là biến cố: \lq\lq Xét nghiệm cá heo cho kết quả dương tính lần thứ $i$\rq\rq; với $i\in \mathbb{N}^*$.
    \begin{itemchoice}
        \itemch Xác suất để có kết quả dương tính là $P\left(B_1\right)=P(A)\cdot P\left(B_1\mid A\right)+P\left({\bar{A}}\right)\cdot P\left(B_1\mid \bar{A}\right)$;\\
        $P\left(B_1\right)=0{,}05\times 0{,}96+0{,}95\times 0{,}02=0{,}067$.
        \itemch Ta có $P\left(A|B_1\right)=\dfrac{P(AB)}{P(B)}=\dfrac{0{,}05\times 0{,}96}{0{,}067}=\dfrac{48}{67}$.
        \itemch Ta có $P\left(B_2|B_1\right)=\dfrac{P\left(B_1B_2\right)}{P\left(B_1\right)}=\dfrac{0{,}05\times 0{,}96^2+0{,}95\times 0{,}02^2}{0{,}067}\approx 0{,}69$.
        \itemch Ta có: $P\left(A|B_2\right)=\dfrac{P\left(AB_2\right)}{P\left(B_2\right)}=\dfrac{0{,}05\times 0{,}96^2}{0{,}05\times 0{,}96^2+0{,}95\times 0{,}02^2}=\dfrac{2304}{2323}\approx 0{,}99$.
    \end{itemchoice}
    }
\end{ex}

\Closesolutionfile{ans}
\caukq
\Opensolutionfile{ans}[ans/ans-HXN-\sode-SA]

\begin{ex}%Câu 17
    \immini
    {
        Trong một trò chơi, người chơi muốn tìm đường đi ngắn nhất để đi từ A đến P, biết rằng từ A đến P có những đường đi được cho như hình vẽ; thông số trên các đoạn thẳng chính là khoảng cách giữa hai vị trí tương ứng. Đường đi thoả mãn điều kiện trên nhận giá trị nhỏ nhất là bao nhiêu?
    \shortans{21}
    }
    {
        \includegraphics[width=7cm]{img/HXN-9-17}
    }
    \end{ex}
    
    \begin{ex}%Câu 18
\immini
{
    Một vật chuyển động theo quy luật $ s=s(t)=\dfrac{1}{3}{t^3}-\dfrac{3}{2}{t^2}+10t+2$; với $ t$(giây) là khoảng thời gian tính từ lúc vật bắt đầu chuyển động và $ s$(mét) là quãng đường vật đi được trong thời gian đó). Tính quãng đường mà vật đi được khi vận tốc nó đạt $ 20$m/s (kết quả làm tròn đến hàng phần chục).
\shortans{54,2}
}
{
    \includegraphics[width=5cm]{img/HXN-9-18}
}
\end{ex}

\begin{ex}%Câu 19
\immini
{
    Một cái hộp đựng đồ chơi có dạng hình lập phương $ ABCD.A'B'C'D'$. Gọi $\varphi$ là góc giữa hai mặt phẳng $\left(AB'C'\right)$ và $\left(AC'D'\right)$, tính giá trị $\cot\varphi$ và làm tròn đến hàng phần trăm.
\shortans{0,58}
}
{
    \includegraphics[width=5cm]{img/HXN-9-19}
}
\loigiai{
\immini
{
    \textbf{Nhận xét:} $\left(AB'C'\right)\equiv \left(ADC'B'\right)$, $\left(AC'D'\right)\equiv \left(ABC'D'\right)$.\\
    Ta có: $\heva{& CD'\perp C'D \\& CD'\perp AD\left(doAD\perp \left(CDD'C'\right)\right) } $\\
    $\Rightarrow CD'\perp \left(ADC'B'\right)$\tagEX{1}
    Tương tự: $\heva{& B'C\perp BC' \\& B'C\perp AB\left(doAB\perp \left(BCC'B'\right)\right) } $\\
    $\Rightarrow B'C\perp \left(ABC'D'\right)$\tagEX{2}
    Từ $(1)$ và $(2)$ suy ra $\left(\left(AB'C'\right),\left(AC'D'\right)\right)=\left(CD',CB'\right)$.\\
    Giả sử cạnh hình lập phương bằng $a$.\\
    Ta có $CB'=CD'=B'D'=a\sqrt{2}$ (đường chéo trong hình vuông). Suy ra tam giác $CB'D'$ đều.\\
    Do vậy $\varphi =\left(\left(AB'C'\right),\left(AC'D'\right)\right)=\left(CD',CB'\right)=\widehat{B'CD'}=60^\circ$\\
    Vậy $\cot \varphi \approx 0{,}58$.
}
{
    \includegraphics[width=5cm]{img/HXN-9-19-LG}
}
}
\end{ex}

\begin{ex}%Câu 20
\immini
{
    Trong một lễ hội mùa hè, có một trò chơi mà mỗi lần chơi, người chơi sẽ tung đồng thời bốn đồng xu cân đối một cách ngẫu nhiên. Người chơi chỉ thắng cuộc nếu nhận được ít nhất ba mặt ngửa từ bốn đồng xu đã tung.
Xác suất để trong $5$ lần chơi, người chơi thắng được ít nhất ba lần xấp xỉ bằng $ m\cdot 10^{-2}$ với $ m$ là số tự nhiên có hai chữ số (đã được làm tròn đến hàng đơn vị); hỏi giá trị của $m$ bằng bao nhiêu?
\shortans{18}
}
{
\includegraphics[width=5cm]{img/HXN-9-20}
}
\loigiai{
\textbf{Bước 1:} Tính xác suất thắng trong một lần chơi\\
Trong một lần chơi, người chơi tung $4$ đồng xu cân đối:
\begin{itemize}
    \item Xác suất nhận được đúng 3 mặt ngửa là $\mathrm{C}_4^3\cdot \left(\dfrac{1}{2}\right)^3\cdot \left(\dfrac{1}{2}\right)^1=4\cdot \dfrac{1}{8}\cdot \dfrac{1}{2}=4\cdot \dfrac{1}{16}=\dfrac{1}{4}$.
    \item Xác suất nhận được cả 4 mặt ngửa là $\left(\dfrac{1}{2}\right)^4=\dfrac{1}{16}$.\\
    Vậy xác suất thắng trong một lần chơi là: $\dfrac{1}{4}+\dfrac{1}{16}=\dfrac{5}{16}$;\\
    xác suất thua trong một lần chơi là $1-\dfrac{5}{16}=\dfrac{11}{16}$.
\end{itemize}
\textbf{Bước 2:} Tính xác suất thắng ít nhất $3$ lần trong $5$ lần chơi.\\
Xác suất này bao gồm các trường hợp:
\begin{itemize}
    \item Thắng đúng 3 lần, thua 2 lần: $\mathrm{C}_5^3\cdot \left(\dfrac{5}{16}\right)^3\cdot \left(\dfrac{11}{16}\right)^2=10\cdot \dfrac{125}{4096}\cdot \dfrac{121}{256}=\dfrac{151250}{1048576}$.
    \item Thắng đúng 4 lần, thua 1 lần: $\mathrm{C}_5^4\cdot \left(\dfrac{5}{16}\right)^4\cdot \left(\dfrac{11}{16}\right)^1=5\cdot \dfrac{625}{65536}\cdot \dfrac{11}{16}=\dfrac{34375}{1048576}$.
    \item Thắng cả 5 lần: $\left(\dfrac{5}{16}\right)^5=\dfrac{3125}{1048576}$.\\
    Xác suất cần tính là: $\dfrac{151250}{1048576}+\dfrac{34375}{1048576}+\dfrac{3125}{1048576}\approx 18\cdot 10^{-2}$.
\end{itemize}
}
\end{ex}

\begin{ex}%Câu 21
\immini
{
    Một nghệ nhân muốn thiết kế một chiếc ly hình nón có thể chứa được tối đa $500$ ml nước với bán kính miệng ly $ R<8cm$. Sau đó, anh ta đặt một khối trụ đặc có bán kính đáy $2$ cm vào bên trong ly sao cho trục của hình trụ trùng với trục của hình nón và đáy trên của hình trụ nằm cùng một mặt phẳng với đáy của hình nón. Nghệ nhân muốn đặt các thanh thủy tinh phát sáng có dạng đoạn thẳng vào ly chứa đầy dung dịch màu. Biết rằng $AB$ là thanh thủy tinh có độ dài lớn nhất có thể đặt vào ly (không có điểm nào nhô ra khỏi mặt nước), tìm giá trị lớn nhất đó theo đơn vị cm (làm tròn đến hàng phần mười).
\shortans{9,5}
}
{
\begin{tikzpicture}[scale=1, font=\footnotesize, thick,declare function={
        H=-7;R=5;
        k=.8/2;
        h=k*H;r=k*R;
        c=1/6;
        G=asin(R*c/H);
        g=asin(r*c/h);
        X=R*cos(G);Y=R*sin(G);
        xd=r*cos(g);yd=r*sin(g);
    },line join=round, line cap=round,x=.8cm]
    \path
    (X,Y) coordinate (M)
    (xd,yd) coordinate (m)
    ($(m)+(90:{H-h})$) coordinate (m1)
    (90:H)coordinate (S)
    (m) arc (g:180-g:{r} and {r*c}) coordinate (ma)
    ($(ma)+(90:{H-h})$) coordinate (m1a)
    ;
    \draw[dashed]
    (ma)--(m1a)
    (m)--(m1) arc (g:180-g:{r} and {r*c})
    (m1) arc (360+g:180-g:{r} and {r*c})
    ;
    \draw 
    (m) arc (g:360-g:{r} and {r*c})
    (M) arc (G:180-G:{R} and {R*c})
    (S)--(M) arc(360+G:180-G:{R} and {R*c})--cycle;
    \draw[<->] ([shift={(0:R+.2)}]S)--++(-90:H)node[midway,fill=white]{$5$ cm};
\end{tikzpicture}
}
\loigiai{
    \begin{center}
        \begin{tikzpicture}[scale=1, font=\footnotesize, thick,>=stealth, line join=round, line cap=round, declare function={
                H=-7;R=5;
                k=2/5;
                h=k*H;r=k*R;
                c=1/5;
                G=asin(R*c/H);
                g=asin(r*c/h);
                X=R*cos(G);Y=R*sin(G);
                xd=r*cos(g);yd=r*sin(g);
            },x=.8cm]
            \path
            (0,0) coordinate (O)
            (X,Y) coordinate (M)
            (xd,yd) coordinate (m)
            ($(m)+(90:{H-h})$) coordinate (m1)
            (90:H)coordinate (S)
            (m) arc (g:180-g:{r} and {r*c}) coordinate (ma)
            ($(ma)+(90:{H-h})$) coordinate (m1a)
            ;
            \draw[dashed]
            (ma)--(m1a)
            (m)--(m1) arc (g:360-g:{r} and {r*c})
            ;
            \path
            (M)arc (G:20:{R} and {R*c}) coordinate (B)
            (m) arc (g:-70:{r} and {r*c}) coordinate (C)
            (m1)arc (g:-70:{r} and {r*c}) coordinate (A)
            pic[draw,angle radius=3mm]{right angle=O--C--B};
            \draw 
            (m) arc (g:360-g:{r} and {r*c})
            (M) arc (G:360-G:{R} and {R*c})
            (S)--(M) arc(360+G:180-G:{R} and {R*c})--cycle;
            \draw[<->] ([shift={(180:R+.2)}]S)--++(-90:H)node[midway,fill=white]{$5$ cm};
            \draw[dashed] (C)--(A)--(B) (O)--(90:{H-h})node[midway,fill=white]{$h_t$};
            \draw[red] (O)--(C)node[midway,left,black]{$r$}--(B)--(O)node[midway,above,black]{$R$};
            \foreach \x/\g in {A/-90,B/30,C/-135,O/-170}\draw[fill=white] (\x) circle (1pt)+(\g:3mm) node{$\x$};
        \end{tikzpicture}
    \end{center}
Thể tích khối nón: $V=\dfrac{1}{3}\pi R^2h=500ml=500cm^3 \Rightarrow h=\dfrac{1500}{\pi R^2}$. \tagEX{1}
Gọi $h_t$ là chiều cao hình trụ, $r=2cm$ là bán kính đáy hình trụ.\\
Do hình trụ nằm trong hình nón và có cùng trục, ta có tỉ lệ: $\dfrac{r}{R}=\dfrac{h-h_t}{h}\Rightarrow h_t=h\left(1-\dfrac{r}{R}\right)$. \tagEX{2}
Thay $(1)$ vào $(2)$: $h_t=\dfrac{1500}{\pi R^2}\left(1-\dfrac{r}{R}\right)$. \tagEX{3}
Thanh $AB$ nằm chéo trong không gian giữa hình nón và hình trụ.\\
Gọi $A$ là điểm nằm dưới (thuộc đường tròn đáy dưới hình trụ), $B$ nằm trên mặt nước. Khi đó, độ dài $AB$ được tính bằng công thức: $AB=\sqrt{R^2-4+h_t^2}$. \tagEX{4}
Thay $(3)$ vào $(4)$: $AB=\sqrt{R^2-4+\left(\dfrac{1500}{\pi R^2}\left(1-\dfrac{2}{R}\right)\right)^2}=f(R)$.\\
Khảo sát hàm $f(R)$, ta có $f(R)=AB$ đạt giá trị lớn nhất xấp xỉ $9{,}5cm$; khi đó $R\approx 7{,}7$.
}
\end{ex}

\begin{ex}%Câu 22
Trong không gian $ Oxyz$, cho hai điểm $ A(10;6;-2)$, $B(5;10;-9)$ và mặt phẳng $(\alpha)\colon 2x+2y+z-12=0$. Điểm $ M$ di động trên $\left(\alpha\right)$ sao cho $ MA$, $ MB$ luôn tạo với $(\alpha)$ các góc bằng nhau. Biết rằng $ M$ luôn thuộc một đường tròn $(C)$ cố định có tâm $ H(a;b;c)$. Tính tổng $a^2+b^2+c^2$.
\shortans{248}
\loigiai
{
\immini
{
    Gọi $H$, $K$ lần lượt là hình chiếu vuông góc của $A$, $B$ trên mặt phẳng $\left(\alpha \right)$.\\
Khi đó $AH=d\left(A,\left(\alpha \right)\right)=6$; $BK=d\left(B,(\alpha )\right)=3$.\\
Vì $MA$, $MB$ tạo với $\left(\alpha \right)$ các góc bằng nhau nên $\widehat{AMH}=\widehat{BMK}$ mà $AH=2BK$ suy ra $MA=2MB$.\\
Gọi $M(x;y;z)$, ta có $MA=2MB\Leftrightarrow MA^2=4MB^2$\\
$ \Leftrightarrow x^2+y^2+z^2-\dfrac{20}{3}x-\dfrac{68}{3}y+\dfrac{68}{3}z+228=0$.
}
{
    \includegraphics[width=5cm]{img/HXN-9-22-LG-a}
}
\immini
{
    Như vậy, điểm $M$ nằm trên mặt cầu $(S)$ có tâm $I\left(\dfrac{10}{3};\dfrac{34}{3};-\dfrac{34}{3}\right)$ và bán kính $R=2\sqrt{10}$.\\
Mặt khác ta có $M$ di động trên $\left(\alpha \right)$, vì vậy tập hợp điểm $M$ chính là đường tròn giao tuyến $(C)$ được tạo bởi mặt cầu $(S)$ và mặt phẳng $\left(\alpha \right)$.\\
Gọi $H$ là tâm của đường tròn $(C)$, khi đó $H$ là hình chiếu vuông góc của $I$ trên mặt phẳng $\left(\alpha \right)$.\\
Đường thẳng $d$ qua $I$ và vuông góc  $\left(\alpha \right)$ có phương trình  $d\colon \heva{& x=\dfrac{10}{3}+2t \\& y=\dfrac{34}{3}+2t \\& z=-\dfrac{34}{3}+t }$.
}
{
    \includegraphics[width=5cm]{img/HXN-9-22-LG-b}
}
Thay phương trình  $d$ vào $\left(\alpha \right)\colon 2\left(\dfrac{10}{3}+2t\right)+2\left(\dfrac{34}{3}+2t\right)+\left(-\dfrac{34}{3}+t\right)-12=0\Leftrightarrow t=-\dfrac{2}{3}$.\\
Từ đó suy ra $H(2;10;-12)$ và $a=2$, $b=10$, $c=-12\Rightarrow a^2+b^2+c^2=248$.
}
\end{ex}

\Closesolutionfile{ans}
\inputansbox{6,4,3}{ans/ans-HXN-\sode-T,ans/ans-HXN-\sode-TF,ans/ans-HXN-\sode-SA}

% %%%%%%%%%%%%%%%%%%%- HXN
\def\sode{10}

\begin{name}
	{\tenchude}
	{\tendethi}
	{\tentruong}
	{\thoigian}
\end{name}

\caulc
\Opensolutionfile{ans}[ans/ans-HXN-\sode-T]
\begin{ex}%Câu 1
    Trong không gian $Oxyz$, phương trình đường thẳng đi qua điểm $ M\left(1;-3;5\right)$ và có vectơ chỉ phương $\vec{u}=\left(2;-1;1\right)$ là
    \choice
    {$\dfrac{x-1}{2}=\dfrac{y-3}{-1}=\dfrac{z-5}{1}$}
    {$\dfrac{x-1}{2}=\dfrac{y-3}{-1}=\dfrac{z+5}{1}$}
    {\True $\dfrac{x-1}{2}=\dfrac{y+3}{-1}=\dfrac{z-5}{1}$}
    {$\dfrac{x+1}{2}=\dfrac{y+3}{-1}=\dfrac{z-5}{1}$}
\end{ex}

\begin{ex}%Câu 2
\immini
{
        Cho hàm số $ y=\dfrac{ax+b}{cx+d}$ $\left(c\ne 0,ad-bc\ne 0\right)$ có đồ thị như hình vẽ bên. Tiệm cận ngang của đồ thị hàm số là
    \choice
    {$x=-1$}
    {\True $y=\dfrac{1}{2}$}
    {$y=-1$}
    {$x=\dfrac{1}{2}$}
}
{
    \begin{tikzpicture}[line join=round, line cap=round,>=stealth,thick,x=.8cm]
        \tikzset{every node/.style={scale=0.9}}
        \draw[->] (-3,0)--(3,0) node[below left] {$x$};
        \draw[->] (0,-3)--(0,3) node[below left] {$y$};
        \draw (0,0) node [below left] {$O$}
        (0,-0.25)node[below right]{$-\tfrac{1}{4}$}
        (0,0.5)node[above right]{$\frac{1}{2}$};
        \foreach \x/\nx in {-1/-1,1/1}
        \draw[thin] (\x,1pt)--(\x,-1pt) node [below] {$\nx$};
        \draw[dashed,thin] (-0.99,-3)--(-0.99,3);
        \begin{scope}
            \clip (-3,-3) rectangle (3,3);
            \draw[samples=200,domain=-4:-1.01,smooth,variable=\x] plot (\x,{(-1*(\x)+0.5)/(-2*(\x)-2)});
            \draw[samples=200,domain=-0.99:4,smooth,variable=\x] plot (\x,{(-1*(\x)+0.5)/(-2*(\x)-2)});
            \draw[dashed,thin] (-4,1/2)--(4,1/2);
        \end{scope}
    \end{tikzpicture}
}
\end{ex}

\begin{ex}%Câu 3
    Tập nghiệm của phương trình $\log_3\left(18-x^2\right)=2$ là
    \choice
    {$S=\left\{ 3\right\}$}
    {$S=\left\{-3\right\}$}
    {\True $S=\left\{\pm 3\right\}$}
    {$S=\left\{-4;3\right\}$}
\end{ex}

\begin{ex}%Câu 4
    Trong không gian $Oxyz$, mặt phẳng $(P)$ đi qua điểm $ M\left(1;2;3\right)$ và song song với $(Q)\colon x-2y+3z+1=0$ có phương trình là
    \choice
    {$x-2y+3z+6=0$}
    {$x-2y+3z+16=0$}
    {\True $x-2y+3z-6=0$}
    {$x-2y+3z-16=0$}
\end{ex}

\begin{ex}%Câu 5
    Nếu $\int\limits_1^2f(x)\mathrm{\,d}x=-2$ và $\int\limits_2^3f(x)\mathrm{\,d}x=1$ thì $\int\limits_1^3f(x)\mathrm{\,d}x$ bằng
    \choice
    {$-3$}
    {\True $-1$}
    {$1$}
    {$3$}
\end{ex}

\begin{ex}%Câu 6
    Thống kê điểm kiểm tra giữa kỳ 1 môn Toán của $30$ học sinh lớp 12C1 của một trường THPT được ghi lại ở bảng sau:\\
    \centerline{\begin{tblr}{|c|c|c|c|c|}
            \hline
            Điểm & $\left[2;4\right)$ & $\left[4;6\right)$ & $\left[6;8\right)$ & $\left[8;10\right)$\\
            \hline
            Số học sinh & $ 4$ & $ 8$ & $ 11$ & $ 7$\\
            \hline
    \end{tblr}}\\
    Trung vị của mẫu số liệu gốc thuộc khoảng nào trong các khoảng dưới đây?
    \choice
    {$\left[2;4\right)$}
    {$\left[4;6\right)$}
    {\True $\left[6;8\right)$}
    {$\left[8;10\right)$}
\end{ex}

\begin{ex}%Câu 7
    Cho cấp số cộng $\left(u_n\right)$ với $u_{10}=25$ và công sai $d=3$. Khi đó $u_1$ bằng
    \choice
    {$u_1=2$}
    {$u_1=3$}
    {$u_1=-3$}
    {\True $u_1=-2$}
\end{ex}

\begin{ex}%Câu 8
    Cho hình chóp $S.ABC$ có đáy $ABC$ là tam giác vuông tại $B$ và $SA\perp\left(ABC\right)$. Khẳng định nào sau đây đúng?
    \choice
    {$ AB\perp\left(SBC\right)$}
    {$ AC\perp\left(SBC\right)$}
    {$ BC\perp\left(SAC\right)$}
    {\True $ BC\perp\left(SAB\right)$}
\end{ex}

\begin{ex}%Câu 9
    Tính thể tích vật thể tròn xoay khi quay hình phẳng giới hạn bởi các đường cong $ y=\sqrt{e^x-x}$, $y=0$, $x=1$, $x=2$ xung quanh trục Ox là
    \choice
    {\True $\pi\left(e^2-e-\dfrac{3}{2}\right)$}
    {$e^2-e-\dfrac{5}{2}$}
    {$\pi\left(e^2-e-\dfrac{5}{2}\right)$}
    {$e^2-e-\dfrac{3}{2}$}
\end{ex}

\begin{ex}%Câu 10
    Cho hàm số có bảng biến thiên như sau\\
    \centerline{
    \begin{tikzpicture}[>=stealth]
        \tkzTabInit[nocadre=false,lgt=1.2,espcl=2.5,deltacl=0.5]{$x$/.7 ,$f'(x)$/.7,$f(x)$/2.5}
        {$-\infty$ , $-1$ , $3$ , $+\infty$}
        \tkzTabLine{ ,-,d,-,$0$,+, }
        \tkzTabVar{-/$-\infty$,1+D+/$2$/$+\infty$,1-/$-4$,+/$+\infty$}
    \end{tikzpicture}
    }
    Tổng các giá trị nguyên của $ m$ để đường thẳng $ y=m$ cắt đồ thị hàm số tại ba điểm phân biệt bẳng
    \choice
    {$-3$}
    {\True $-5$}
    {$ 0$}
    {$-1$}
\end{ex}

\begin{ex}%Câu 11
    Bảng số liệu ghép nhóm về chiều cao đo được của 30 học sinh nam lớp 12A2 đầu năm học $ 2024-2025$ của một trường THPT được cho như sau:\\
    \centerline{\begin{tabular}{|c|c|c|c|c|c|}
            \hline
            Chiều cao & $\left[150;\ 155\right)$ & $\left[155;\ 160\right)$ & $\left[160;\ 165\right)$ & $\left[165;\ 170\right)$ & $\left[170;\ 175\right)$\\
            \hline
            Tần số & $ 3$ & $ 7$ & $ 10$ & $ 7$ & $ 3$\\
            \hline
    \end{tabular}}\\
    Tính độ lệch chuẩn của mẫu số liệu ghép nhóm trên.
    \choice
    {\True $\dfrac{\sqrt{285}}{3}$}
    {$\dfrac{\sqrt{287}}{3}$}
    {$ 4\sqrt{2}$}
    {$\sqrt{71}$}
\end{ex}

\begin{ex}%Câu 12
    Đồ thị hàm số $ y=\dfrac{3x^2-x+5}{x-2}$ có hai điểm cực trị $ A,B$ nằm trên đường thẳng $ d$ có phương trình $ y=ax+b.$ Tính $ a+b.$ 
    \choice
    {$ a+b=-1.$}
    {$ a+b=1.$}
    {$ a+b=3.$}
    {\True $ a+b=5$}
        \end{ex}


\Closesolutionfile{ans}
\cauds
\Opensolutionfile{ans}[ans/ans-HXN-\sode-TF]

\begin{ex}%Câu 13
Cho hàm số $ y=\dfrac{-x^2+x+1}{x+1}$.
    \choiceTF
    {Hàm số đồng biến trên khoảng $\left(-2;-1\right)$ và $\left(-1;0\right)$}
    {Đồ thị hàm số có hai điểm cực trị nằm khác phía so với $Ox$}
    {Đồ thị hàm số không cắt trục $ Ox$}
    {Đồ thị hàm số có tiệm cận xiên đi qua điểm $ M\left(1;2\right)$}
\end{ex}

\begin{ex}%Câu 14
\immini
{
    Một cái chậu nước có dạng hình chóp cụt đều với các cạnh đáy lần lượt bằng 6 dm và 3 dm, chiều cao chậu nước bằng 4 dm. Người ta bơm nước vào chậu với tốc độ $0{,}4$ lít/phút.
}
{
    \includegraphics[width=5cm]{img/HXN-10-14}
}
    \choiceTF
    {\True Dung tích của chậu nước bằng $ 21\sqrt{3}d{m^3}$}
    {Nếu người ta giữ nguyên tốc độ bơm nước thì sau $91$ phút (làm tròn đến hàng đơn vị) bể sẽ đầy}
    {Khi mực nước trong chậu có chiều cao h thì thể tích nước trong chậu được tính theo công thức $ V=\dfrac{\sqrt{3}}{12}\left(\dfrac{9}{8}{h^3}+\dfrac{9}{4}{h^2}+27h\right)$ lít}
    {Khi nước được bơm đến phút thứ 8 thì tốc độ dâng lên của nước trong chậu bằng $0{,}05$ dm/phút (làm tròn đến hàng phần trăm)}
    \loigiai{
    \begin{itemchoice}
        \itemch Thể tích chậu nước $V=\dfrac{1}{3}h\left(S_1+\sqrt{S_1S_2}+S_2\right)$; trong đó $h=4\,dm$ và diện tích hai đáy lần lượt là $S_1=\dfrac{3^2\sqrt{3}}{4}=\dfrac{9\sqrt{3}}{4}\,dm^2$; $S_2=\dfrac{6^2\sqrt{3}}{4}=9\sqrt{3}\,dm^2$.\\
        Do đó $V=\dfrac{1}{3}\cdot 4\cdot \left(\dfrac{9\sqrt{3}}{4}+\sqrt{\dfrac{9\sqrt{3}}{4}\cdot 9\sqrt{3}}+9\sqrt{3}\right)=21\sqrt{3}\,dm^3$.
        \itemch Bể đầy nước sau khoảng thời gian $t=\dfrac{21\sqrt{3}}{0{,}4}\approx 91$ phút.
        \itemch Thể tích nước tương ứng chiều cao $h$ là\\ $V=\dfrac{1}{3}h\left(\dfrac{9\sqrt{3}}{4}+\dfrac{x^2\sqrt{3}}{4}+\sqrt{\dfrac{9\sqrt{3}}{4}\cdot \dfrac{x^2\sqrt{3}}{4}}\right)=\dfrac{1}{3}h\left(\dfrac{9\sqrt{3}}{4}+\dfrac{\sqrt{3}x^2}{4}+\dfrac{3\sqrt{3}x}{4}\right)$ \tagEX{ 1}
        \immini
        {
            Gọi $x=MN$ là đường mép nước ứng với một mặt bên chậu, chiều cao mực nước là $h$.\\
        Ta có: $x=ah+b$ (hàm số bậc nhất).
        Vì $x=3$; $h=0$ và $x=6;h=4$ suy ra $\heva{& b=3 \\& 4a+b=6 } \Rightarrow \heva{& a=\dfrac{3}{4} \\& b=3 } $ hay $x=\dfrac{3}{4}h+3$ \tagEX{2}
        }
        {
            \includegraphics{img/HXN-10-14-LG}
        }
        Thay $(2)$ vào $(1)$ ta được: $V=\dfrac{1}{3}h\left(\dfrac{9\sqrt{3}}{4}+\dfrac{\sqrt{3}\left(\dfrac{3}{4}h+3\right)^2}{4}+\dfrac{3\sqrt{3}\left(\dfrac{3}{4}h+3\right)}{4}\right)$.\\
        Thu gọn ta được: $V=\dfrac{\sqrt{3}}{12}\left(\dfrac{9}{16}h^3+\dfrac{27}{4}h^2+27h\right)$ \tagEX{3}
        \itemch Đến phút thứ $8$, mực nước trong chậu là $V(8)=0{,}4\times 8=3{,}2\,dm^3$.\\
        Thay vào $(3)$ ta được $\dfrac{\sqrt{3}}{12}\left(\dfrac{9}{16}h^3+\dfrac{27}{4}h^2+27h\right)=3{,}2\Rightarrow h\approx 0{,}69\,dm$\\
        Lấy đạo hàm hai vế $(3)$ theo $t$, ta được: $\dfrac{dV}{\mathrm{\,d}t}=\dfrac{\sqrt{3}}{12}\left(\dfrac{27}{16}h^2+\dfrac{27}{2}h+27\right)\dfrac{dh}{\mathrm{\,d}t}$ \tagEX{4}
        Thay $\dfrac{dV}{\mathrm{\,d}t}=0{,}4$ dm/phút; $h\approx 0{,}69\,dm$ vào $(4)$ ta được $\dfrac{dh}{\mathrm{\,d}t}\approx 0{,}07$ dm/phút.
    \end{itemchoice}
    
    }
\end{ex}

\begin{ex}%Câu 15
Trong không gian $Oxyz$ cho trước với mặt nước phẳng lặng trùng với mặt phẳng $(Oxy)$, đơn vị trên mỗi trục là mét; có hai con chim bói cá ở các vị trí$ A\left(90;0;25\right)$, $ B\left(80;30;15\right)$ trên các cành cây đang cùng ngắm mục tiêu là một chú cá đang bơi trên mặt hồ. Khi cá nằm im ở vị trí $ C\left(20;10;0\right)$ thì hai con chim quyết định tấn công mục tiêu của mình. Chim bói cá ở vị trí $A$ xuất phát trước con còn lại $1$ giây và bay về phía con cá với vận tốc $12$ m/s; chim bói cá còn lại cũng tấn công mục tiêu với vận tốc $15$ m/s.\\
\centerline{\includegraphics[width=8cm]{img/HXN-10-15}}
    \choiceTF
    {\True Khoảng cách của chim bói cá ở A đến mục tiêu ngắn hơn khoảng cách từ chim bói cá ở B đến mục tiêu}
    {\True Chim bói cá ở vị trí A sẽ đến mục tiêu trước con chim ở vị trí B}
    {Trong thực tế, sau khi bay được $5$ giây, chim bói cá từ vị trí A thấy không tranh được con mồi với đối thủ nên nó chuyển hướng để bay đi và đậu trên một nhành cây khác, vị trí chuyển hướng có tọa độ $\left(34;8;4,5\right)$}
    {Từ khi chuyển hướng, chim bó cá bay với vectơ vận tốc $\vec{u}=\left(3;6;6\right)$ (m/s) và sau 6 giây tiếp theo, nó đã đậu trên một cành cây khác. Khoảng cách từ vị trí mới so với vị trí nó đậu ban đầu bằng $63{,}2$ m (làm tròn đến hàng phần chục của mét)}
    \loigiai{
    \begin{itemchoice}
        \itemch 
        Ta có $\vec{AC}=(-70;10;-25)\Rightarrow AC=75\,m$; $\vec{BC}=(-60;-20;-15)\Rightarrow BC=65\,m\,\left(AC>BC\right)$.
        \itemch Thời gian để con chim từ $A$ đến mục tiêu: $\dfrac{75}{12}=6{,}25$ giây; thời gian để con chim từ B đến mục tiêu kể từ thời điểm con chim từ A xuất phát: $\dfrac{65}{15}+1\approx 5{,}33<6{,}25$.
        Do đó chim bói cá từ B đến mục tiêu trước so với con chim từ A.
        \itemch Gọi $D$ là vị trí chuyển hướng của chim bói cá từ A, ta có\\
        $\vec{AD}=\dfrac{5}{6{,}25}\vec{AC}=\dfrac{5}{6{,}25}(-70;10;-25)=(-56;8;-20)\Rightarrow \heva{& x_D-90=-56 \\& y_D=8 \\& z_D-25=-20 } \Rightarrow D(34;8;5)$.
        \itemch  Sau $6$ giây, con chim bói cá chuyển hướng từ $D$ sẽ tịnh tiến theo vectơ $\vec{v}=6\vec{u}=(18;36;36)$; vị trí mới của nó là $E$ có tọa độ $\heva{& x_E=34+18 \\& y_E=8+36 \\& z_E=5+36 }$ hay $E(52;44;41)$.\\
        Khoảng cách $AE=\sqrt{(52-90)^2+(44-0)^2+(41-25)^2}=6\sqrt{101}\approx 60{,}3\,m$.
    \end{itemchoice}
    }
\end{ex}

\begin{ex}%Câu 16
\immini
{
    Trong vụ án kinh điển Bao Công xử Quách Hòe, khi ấy Phủ Khai Phong chắc chắn rằng Quách Hòe có đến $85\% $ khả năng gây án.
}
{
\includegraphics[width=8cm]{img/HXN-10-16}
}
\begin{itemize}
    \item Nếu Quách Hòe gây án thì người hầu của hắn có $65\%$ phạm tội và quan tri huyện có $45\%$ phạm tội, ngoài ra hai người này (người hầu và tri huyện) cũng có $30\%$ khả năng cùng phạm tội.
    \item Nếu Quách Hòe không gây án thì người hầu của hắn chắc chắn không phạm tội, nhưng tri huyện thì có đến $35\%$ khả năng phạm tội.
\end{itemize}
    Gọi $A$ là biến cố: \lq\lq Quách Hòe gây án\rq\rq; $B$ là biến cố: \lq\lq Người hầu Quách Hòe có phạm tội\rq\rq và $C$ là biến cố: \rq\rq Tri huyện có phạm tội\rq\rq.
    \choiceTF
    {$ P(B|A)=0,65;P\left(C|A\right)=0{,}55$}
    {\True Xác suất để cả người hầu và tri huyện không phạm tội bằng $0{,}2 $nếu biết Quách Hòe đã gây án}
    {Xác suất để quan tri huyện có phạm tội bằng $0{,}44$}
    {\True Nếu quan tri huyện có phạm tội, khả năng để Quách Hòe gây án bằng 0,96 (làm tròn đến hàng phần trăm)}
    \loigiai{
    \immini
    {
        \begin{itemchoice}
        \itemch Ta có $P(B|A)=0{,}6$; $P\left(C|A\right)=0{,}45$.
        \itemch Sử dụng biểu đồ Ven, ta có $P\left(B\cup C|A\right)=0{,}15+0{,}3+0{,}35=0{,}8$.
        Do đó $P\left(\bar{B}\bar{C}|A\right)=1-0{,}8=0{,}2$.
        \itemch Ta có $P(C)=P(A)\cdot P\left(C|A\right)+P\left({\bar{A}}\right)\cdot P\left(C|\bar{A}\right) =0{,}95.0{,}45+0{,}05.0{,}35=0{,}445$.
        \itemch $P\left(A|C\right)=\dfrac{P(AC)}{P(C)}=\dfrac{0{,}95.0{,}45}{0{,}445}\approx 0{,}96$.
    \end{itemchoice}
    }
    {
        \includegraphics[width=8cm]{img/HXN-10-16-LG}
    }
    }
\end{ex}

\Closesolutionfile{ans}
\caukq
\Opensolutionfile{ans}[ans/ans-HXN-\sode-SA]
\begin{ex}%Câu 17
    Một khối gỗ có dạng hình hộp chữ nhật $ ABCD.A'B'C'D'$. Biết rằng $AB=10$cm, $BC=15$cm và góc hai mặt phẳng $\left(BCD'A''\right),\left(ABCD\right)$ bằng $30^\circ$. Tính tổng diện tích tất cả các mặt của khối gỗ đó theo đơn vị $ $ m$^2$ và làm tròn đến hàng đơn vị.\\
    \centerline{
    \includegraphics[width=5cm]{img/HXN-10-17-a}\qquad \includegraphics[width=5cm]{img/HXN-10-17-b}
    }
    \shortans{589}
    \end{ex}
    
    \begin{ex}%Câu 18
\immini
{
    Giả sử chi phí cho việc xuất bản $x$ cuốn tạp chí (gồm: lương cán bộ, công nhân viên, giấy in) được cho bởi công thức $C(x)=0{,}0001x^2-0,2x+10000$, trong đó $C(x)$ được tính theo đơn vị là vạn đồng. Chi phí phát hành cho mỗi cuốn là $4$ nghìn đồng. Gọi $T(x)$ là tổng chi phí (gồm cả chi phí xuất bản và phát hành) cho $ x$ cuốn tạp chí; khi đó tỉ số $ M(x)=\dfrac{T(x)}{x}$ được gọi là chi phí trung bình cho một cuốn tạp chí khi xuất bản $x$ cuốn. Tìm số lượng tạp chí cần xuất bản (đơn vị: nghìn cuốn) sao cho chi phí trung bình là thấp nhất, biết rằng nhu cầu hiện tại xuất bản không quá $30\,000$ cuốn.
\shortans{10}
}
{
\includegraphics[width=5cm]{img/HXN-10-18}
}
\loigiai{
Chi phí phát hành cho mỗi cuốn là $4$ nghìn đồng, tức là $0{,}4$ vạn đồng.\\
Suy ra chi phí phát hành cho $x$ cuốn là $0{,}4x$ (vạn đồng).\\
Tổng chi phí xuất bản và phát hành cho $x$ cuốn tạp chí là:
$T(x)=C(x)+0{,}4x=0{,}0001x^2+0{,}2x+10\,000$; với $x>0$.\\
Đặt $f(x)=\dfrac{T(x)}{x}=0{,}0001x+0{,}2+\dfrac{10\,000}{x}$ với $0<x\le 30\,000$.\\
Ta có $f'(x)=0{,}0001-\dfrac{10\,000}{x^2}=\dfrac{0{,}0001x^2-10\,000}{x^2}$; $f'(x)=0\Leftrightarrow x=10\,000>0$. \\
Dựa vào bảng biến thiên, ta thấy giá trị của $M(x)$ nhỏ nhất khi $x=10\,000$.\\
Vậy số lượng tạp chí cần xuất bản sao cho chi phí trung bình thấp nhất là $10$ nghìn cuốn.
}
\end{ex}

\begin{ex}%Câu 19
\immini
{
    Có một con quạ giỏi toán đang khát nước, nó tìm thấy một ly nước hình trụ có bán kính đáy bằng 4 cm, chiều cao $20$ cm, bên trong ly chỉ chứa ít nước đến nỗi nó không thể thò mỏ vào uống được. Quạ ta liền nhanh trí gắp viên bi gần đó bỏ vào ly để mực nước trong ly dâng lên. Biết rằng viên bi có bán kính 2 cm và ban đầu mực nước trong ly chỉ cao 3 cm; hỏi sau khi quạ bỏ viên bi vào ly thì mực nước trong ly dâng lên thêm được bao nhiêu cm? Làm tròn kết quả đến hàng phần trăm.
\shortans{0{,}65}
}
{
    \includegraphics[width=5cm]{img/HXN-10-19}
}
\loigiai{
Gọi $h$ là mực nước dâng lên thêm sau khi bỏ viên bi vào ly.\\
Thể tích nước ban đầu là $V_1=\pi R^2h_1=\pi \cdot 4^2\cdot 3=48\pi \,cm^3$; với $R=4\,cm$; $h_1=3\,cm$.\\
Thể tích phần khối cầu chìm trong nước (sau khi nước dâng lên) là \\
$V_2=\dfrac{1}{3}\pi (h+3)^2\left[3r-(h+3)\right]=\dfrac{1}{3}\pi (h+3)^2(3-h)\,\,cm^3$; với $r=2\,cm$.\\
Thể tích nước sau khi thả viên bi vào (tính cả phần khối cầu chìm trong nước) là $$V=\pi R^2(h+3)=16\pi (h+3)$$
Ta có $V=V_1+V_2\Leftrightarrow 16\pi (h+3)=48\pi +\dfrac{1}{3}\pi (h+3)^2(3-h)$\\
$\Leftrightarrow 16h+48=48-\dfrac{1}{3}h^3-h^2+3h+9\Leftrightarrow h\approx 0{,}65 $.
}
\end{ex}

\begin{ex}%Câu 20
\immini
{
    Tuấn là môt học sinh giỏi lớp $12$, em rất thích học môn toán. Hôm ấy sau khi đã học xong phần Ứng dụng tích phân, Tuấn quyết định cắt chiếc nón mà người bố hay đội đi làm ruộng để nghiên cứu. Biết rằng hình nón này có bán kính đáy bằng $20$cm, thiết diện qua trục là một tam giác đều. Dù người bố hết sức ngăn cản nhưng Tuấn đã ra tay một cách dứt khoát, cắt hình nón bởi một mặt phẳng đi qua đường kính đáy và vuông góc với đường sinh của hình nón để chia nó ra làm hai phần, phần nhỏ có dạng một hình nêm (H), tính thể tích của khối (H) theo đơn vị centimét khối, làm tròn đến hàng đơn vị.
\shortans{2309}
}
{
    \includegraphics[width=5cm]{img/HXN-10-20}
}
\loigiai{
\immini
{
    Chọn hệ trục tọa độ như hình vẽ.\\
Cắt hình nêm $(H)$ bởi một mặt phẳng vuông góc với trục $Ox$ tại điểm có hoành độ $x$, ta được thiết diện là một tam giác vuông ABC thay đổi như hình vẽ.\\
Thể tích khối $(H)$ được tính theo công thức:\\ $V=\int\limits_{-20}^{20}S(x)\mathrm{\,d}x$ với $S(x)=S_{\triangle ABC}$.
}
{
    \includegraphics[width=5cm]{img/HXN-10-20-LG}
}
Tam giác $ABC$ vuông tại $B$ nên $S_{\triangle ABC}=\dfrac{1}{2}AB\cdot BC$.\\
Tam giác $OAC$ vuông tại $A$ nên $AC=\sqrt{20^2-x^2}$.\\
Ta có $\heva{& BC=AC\cdot \cos 60^\circ=\dfrac{1}{2}\cdot \sqrt{20^2-x^2}  \\&  AB=AC\cdot sin60^\circ=\dfrac{\sqrt{3}}{2}\cdot \sqrt{20^2-x^2} }\Rightarrow S(x)=S_{\triangle ABC}=\dfrac{\sqrt{3}}{8}\cdot \left(r^2-x^2\right)$.\\
Do đó $V=\int\limits _{-20}^{20}S(x)\mathrm{\,d}x=\dfrac{\sqrt{3}}{8} \int\limits_{-20}^{20}\left(20^2-x^2\right)\mathrm{\,d}x=\dfrac{20^3}{2\sqrt{3}}\approx 2309\,cm^3$.
}
\end{ex}

\begin{ex}%Câu 21
\immini
{
    Có 8 bạn cùng ngồi xung quanh một cái bàn tròn, mỗi bạn cầm một đồng xu như nhau. Tất cả 8 bạn cùng tung đồng xu của mình, bạn có đồng xu ngửa thì đứng, bạn có đồng xu sấp thì ngồi.\\
Biết xác suất để không có hai bạn liền kề cùng đứng bằng $\dfrac{m}{n}$ (trong đó $m$, $n$ là các số tự nhiên và phân số $\dfrac{m}{n}$ tối giản); tính $ m+n$.
\shortans{303}
}
{
    \includegraphics[width=5cm]{img/HXN-10-21}
}
\loigiai{
Gọi $A$ là biến cố \lq\lq không có hai người liền kề cùng đứng\rq\rq.\\
Số phần tử của không gian mẫu là $n\left(\Omega \right)=2^8=$.\\
Nếu có nhiều hơn $4$ đồng xu ngửa thì biến cố $A$ không xảy ra. Ta xét các trường hợp sau:\\
\textbf{Trường hợp 1:} Có nhiều nhất 1 đồng xu ngửa; số kết quả là $1+8=$.\\
\textbf{Trường hợp 2:} Có 2 đồng xu ngửa; số kết quả là $C_8^2-8=$.\\
(Loại trừ 8 khả năng 2 đồng xu ngửa đó kề nhau).\\
\textbf{Trường hợp 3:} Có 3 đồng xu ngửa trong $8$ đồng xu; các khả năng để loại trừ là
\begin{itemize}
    \item Cả 3 đồng xu ngửa kề nhau: có 8 kết quả.
    \item Có 2 đồng xu ngửa kề nhau trong 3 đồng xu ngửa: có $8.4=32$ kết quả.\\
    Suy ra, số kết quả của trường hợp này là $C_8^3-8-32=$.
\end{itemize}
\textbf{Trường hợp 4:} Có 4 đồng xu ngửa; có 2 kết quả như thế.\\
(Kết quả của trường hợp này là: S-N-S-N-S-N-S-N và N-S-N-S-N-S-N-S; với kí hiệu N là người nhận được đồng xu mặt ngửa và S là người nhận mặt sấp tương ứng vị trí).\\
Số kết quả thuận lợi là $n(A)=9+20+16+2=10 $.\\
Xác suất để không có hai bạn liền kề cùng đứng là $P(A)=\dfrac{n(A)}{n\left(\Omega \right)}=\dfrac{47}{256}=\dfrac{m}{n}\Rightarrow m+n=303$.
}
\end{ex}

\begin{ex}%Câu 22
\immini
{
    Trong không gian $Oxyz$, đơn vị trên mỗi trục là $2\,000$ km, người ta mô phỏng bề mặt Hỏa tinh dưới dạng mặt cầu $ (S)\colon x^2+y^2+z^2-2x-2y-2z=0$; một robot do thám được gởi đến bởi các nhà khoa học từ trái đất đang ở vị trí $A\left(2;2;0\right)$ . Người ta cần đặt một thiết bị nhận tín hiệu từ robot ở vị trí $B$ thuộc bề mặt sao Hỏa sao cho $B$ có hoành độ dương và tam giác $OAB$ đều. Tìm khoảng cách thực tế từ vị trí $B$ đến vị trí $C\left(0;2;0\right)$ , nơi đáp xuống của tàu vũ trụ (làm tròn đến hàng phần trăm của nghìn km)
\shortans{6{,}93}
}
{
    \includegraphics[width=5cm]{img/HXN-10-22}
}
\loigiai{
Gọi $B\left(x\,;y\,;z\right)$ thuộc (S) với $x>0$ và $H$ trung điểm $OA\Rightarrow H(1;1;0)$.\\
Gọi $(P)$ là mặt phẳng trung trực đoạn $OA$, do đó $(P)$ đi qua trung điểm $H(1;1;0)$ của đoạn $OA$ và nhận $\vec{OA}=(2;2;0)$ làm vectơ pháp tuyến. Suy ra  $(P)\colon 2(x-1)+2(y-1)=0$ $\Leftrightarrow x+y-2=0$.\\
Theo giả thiết: $\heva{& OB=AB \\& OB=OA \\& B\in (S) } \Leftrightarrow \heva{& B\in (P) \\& OB^2=OA^2 \\& B\in (S) }  \Leftrightarrow \heva{& x+y-2=0 \\& x^2+y^2+z^2=8\, \\& x^2+y^2+z^2-2x-2y-2z=0} $\\
$\Leftrightarrow \heva{& x+y=2 \\& x^2+y^2+z^2=8\, \\& 2x+2y+2z=8\, } \Leftrightarrow \heva{& x+y=2 \\& x^2+y^2=4\, \\& z=2\, }  \Leftrightarrow \heva{& x+y=2 \\& (x+y)^2-2xy=4\, \\& z=2\, } \Leftrightarrow \heva{& x+y=2 \\& xy=0\, \\& z=2\, } $.\\
Ta tìm được $\heva{& x=2 \\& y=0\, \\& z=2\, } \Rightarrow B(2;0;2)$, (do $x>0$). Do đó $BC=\sqrt{2^2+2^2+2^2}=2\sqrt{3}$.\\
Khoảng cách thực tế là $2\sqrt{3}\times 2\approx 6{,}93$ (nghìn km).
}
\end{ex}

\Closesolutionfile{ans}
\inputansbox{6,4,3}{ans/ans-HXN-\sode-T,ans/ans-HXN-\sode-TF,ans/ans-HXN-\sode-SA}
% %%%%%%%%%%%%%%%%%%%- HXN
\def\sode{11}
\def\tendethi{ĐỀ PHÁT TRIỂN MINH HOẠ 2025}
\begin{dethi}
 {\tendethi}
\end{dethi}
\caulc
\Opensolutionfile{ans}[ans/ans-HXN-\sode-T]
\begin{ex}%Câu 1
 Cho hàm số $ y=f(x)$ liên tục trên $\mathbb{R}$ và có bảng xét dấu của $f'(x)$ như sau:\\
\centerline{
\begin{tikzpicture}[>=stealth]
    \tkzTabInit[nocadre=true,lgt=1.2,espcl=2.5,deltacl=.5]
    {$x$/.7, $f'(x)$/1}
    {$-\infty$,$-2$,$0$,$2$,$+\infty$}
    \tkzTabLine{ , + , $0$ , - ,$0$,+,$0$,+, }
\end{tikzpicture}
}
 Hàm số $ y=f(x)$ có bao nhiêu điểm cực trị?
 \choice
 {$4$}
 {$3$}
 {\True $2$}
 {$1$}
\end{ex}
\begin{ex}%Câu 2
 Biết $\int\limits_1^3f(x)\mathrm{d}x=3$. Giá trị của $\int\limits_1^32f(x)\mathrm{d}x$ bằng
 \choice
 {$5$}
 {\True $6$}
 {$9$}
 {$\dfrac{3}{2}$}
\end{ex}
\begin{ex}%Câu 3
 Trong không gian $ Oxyz$, cho mặt phẳng $(P)$ đi qua điểm $ M\left(2;2;1\right)$ và có một vectơ pháp tuyến $\vec{n}=\left(5;2;-3\right)$. Phương trình mặt phẳng $(P)$là
 \choice
 {$5x+2y-3z-17=0$}
 {$2x+2y+z-11=0$}
 {\True $5x+2y-3z-11=0$}
 {$2x+2y+z-17=0$}
\end{ex}
\begin{ex}%Câu 4
 Nếu cấp số nhân $\left(u_n\right)$ có số hạng đầu $u_1=3$ và công bội $ q=3$ thì số hạng tổng quát $u_n$ của cấp số nhân đó bằng
 \choice
 {\True $3^{n}$}
 {$3^{n-1}$}
 {$3^{n+1}$}
 {$3+\left(n-1\right)\cdot 3$}
\end{ex}
\begin{ex}%Câu 5
 Tập nghiệm của bất phương trình $2^x\le 4$ là
 \choice
 {\True $\left(-\infty;2\right]$}
 {$\left[0;2\right]$}
 {$\left(-\infty;2\right)$}
 {$\left(0;2\right)$}
\end{ex}
\begin{ex}%Câu 6
 Cho hình chóp $S.ABCD$ có tất cả các cạnh bên và cạnh đáy bằng nhau và $ABCD$ là hình vuông tâm $O$. Khẳng định nào sau đây là khẳng định đúng?
 \choice
 {$SA\perp\left(ABCD\right)$}
 {\True $SO\perp\left(ABCD\right)$}
 {$AB\perp\left(SBC\right)$}
 {$AC\perp\left(SBC\right)$}
\end{ex}
\begin{ex}%Câu 7
 Phát biểu nào sau đây là đúng?
 \choice
 {$\int\dfrac{1}{x}\mathrm{\,d}x=\left| x\right|+\text{C}$}
 {\True $\int\dfrac{1}{x}\mathrm{\,d}x=\text{ln}\left| x\right|+C$}
 {$\int\text{ln}x\mathrm{\,d}x=x+C$}
 {$\int\text{ln}\left| x\right|\mathrm{\,d}x=\text{ln}x+C$}
\end{ex}
\begin{ex}%Câu 8
\immini
{
     Cho hàm số $y=f(x)$ có đồ thị như hình dưới đây.
 Đường tiệm cận xiên của đồ thị hàm số đã cho là đường thẳng 
 \choice
 {$y=x-1$}
 {\True $y=-x-1$}
 {$y=x+1$}
 {$y=-x+1$}
}
{
    \begin{tikzpicture}[line join=round, line cap=round,>=stealth,thick]
        \tikzset{every node/.style={scale=0.9}}
        \draw[->] (-3.1,0)--(2.1,0) node[below left] {$x$};
        \draw[->] (0,-3.1)--(0,3.1) node[below left] {$y$};
        \draw (0,0) node [below left] {$O$};
        \foreach \x/\nx in {-1/-1,1/1}
        \draw[thin] (\x,1pt)--(\x,-1pt) node [below] {$\nx$};
        \foreach \y/\ny in {-2/-2,-1/-1}
        \draw[thin] (1pt,\y)--(-1pt,\y) node [left] {$\ny$};
        \draw[dashed,thin] (-0.99,-3)--(-0.99,3);
        \begin{scope}
            \clip (-3,-3) rectangle (2,3);
            \draw[samples=200,domain=-3:-1.01,smooth,variable=\x] plot (\x,{(-1*((\x)^2)+-2*(\x)+-2)/(1*(\x)+1)});
            \draw[samples=200,domain=-0.99:3,smooth,variable=\x] plot (\x,{(-1*((\x)^2)+-2*(\x)+-2)/(1*(\x)+1)});
            \draw[dashed,thin] (-3.1,2.1)--(3.1,-4.1);
        \end{scope}
    \end{tikzpicture}
}
\end{ex}
\begin{ex}%Câu 9
\immini
{
     Cho hàm số $f(x)$ có đồ thị như hình vẽ bên. Giá trị nhỏ nhất của hàm số $f(x)$ trên đoạn $\left[-2;2\right]$ bằng
 \choice
 {$-2$}
 {$-1$}
 {\True $ 0$}
 {$ 1$}
}
{
    \begin{tikzpicture}[line join=round, line cap=round,>=stealth,thick]
        \tikzset{every node/.style={scale=0.9}}
        \draw[->] (-2.3,0)--(2.5,0) node[below] {$x$};
        \draw[->] (0,-1.5)--(0,3.1) node[below left] {$y$};
        \draw (0,0) node [below left] {$O$};
        \foreach \x/\nx in {-2/-2,-1/-1,1/1,2/2}
        \draw[thin] (\x,1pt)--(\x,-1pt) node [below] {$\nx$};
        \foreach \y/\ny in {2/2}
        \draw[thin] (1pt,\y)--(-1pt,\y) node [above left] {$\ny$};
        \draw[dashed,thin](-1,0)--(-1,2)--(0,2);
        \draw[dashed,thin](2,0)--(2,2)--(0,2);
        \begin{scope}
            \clip (-2.2,-1.5) rectangle (2.2,3);
            \draw[samples=200,domain=-2.2:2.2,smooth,variable=\x] plot (\x,{1/2*((\x)^3)+0*((\x)^2)+-3/2*(\x)+1});
        \end{scope}
    \end{tikzpicture}
}
\end{ex}
\begin{ex}%Câu 10
 Bạn An rất thích nhảy hiện đại. Thời gian tập nhảy mỗi ngày của bạn An được thống kê lại ở bảng sau:\\
 \centerline{
 \begin{tblr}{
         colspec = {Q[l] *{5}{Q[c]}},hlines,vlines
     }
     Thời gian (phút) & [20;25) & [25;30) & [30;35) & [35;40) & [40;45) \\
     Số ngày          & 6       & 6       & 4       & 1       & 1       \\
 \end{tblr}
 }
 Độ lệch chuẩn của mẫu số liệu ghép nhóm có giá trị gần nhất với giá trị nào dưới đây?
 \choice
 {$31{,}25$}
 {$31{,}26$}
 {$5{,}4$}
 {\True $5{,}6$}
\end{ex}
\begin{ex}%Câu 11
 Trong không gian $Oxyz$, điểm nào dưới đây thuộc đường thẳng $ d:\heva{& x=1-t\\& y=5+t\\& z=2+3t}$ ?
 \choice
 {$Q\left(-1;1;3\right)$}
 {$P\left(1;2;5\right)$}
 {$M\left(1;1;3\right)$}
 {\True $N\left(1;5;2\right)$}
\end{ex}
\begin{ex}%Câu 12
\immini
{
     Cho hình lập phương $ABCD.A'B'C'D'$(tham khảo hình bên). Giá trị $\sin$ của góc giữa đường thẳng $ AC'$ và mặt phẳng $\left(ABCD\right)$ bằng
 \choice
 {\True $\dfrac{\sqrt{3}}{3}$}
 {$\dfrac{\sqrt{2}}{2}$}
 {$\dfrac{\sqrt{3}}{2}$}
 {$\dfrac{\sqrt{6}}{3}$}
}
{
    \begin{tikzpicture}[line cap=round,line join=round, >=stealth,scale=1]
        \def \a{-1.5} \def \b{-1}\def \c{4} \def \h{3}
        \path (0,0)coordinate(A) 
        +(\a,\b)coordinate(B)
        +(\c,0)coordinate(D)
        ($(B)+(D)-(A)$)coordinate(C)
        +(0,\h) coordinate(C')
        ($(B)+(C')-(C)$)coordinate(B')
        ($(A)+(C')-(C)$)coordinate(A')
        ($(D)+(C')-(C)$)coordinate(D');
        \draw [dashed] (A)--(B)(D)--(A)--(A');
        \draw (B')--(B)--(C)(B')--(C')--(C)--(D)--(D')--(A')--(B')(C')--(D');
        \foreach \x/\g in {A/135,B/-135,C/-45,D/0,A'/135,B'/180,C'/-20,D'/0}\fill[red] (\x) circle (1pt)+(\g:3mm) node[black]{$\x$};
    \end{tikzpicture}
}
 \end{ex}
\Closesolutionfile{ans}
\cauds
\Opensolutionfile{ans}[ans/ans-HXN-\sode-TF]
\begin{ex}%Câu 13
\immini
{
    Công thức $\log x=11{,}8+1{,}5M$ cho biết mối liên hệ giữa năng lượng $x$ tạo ra (tính theo erg, $1$ erg tương đương $10^{-7}$ jun) với độ lớn $M$ theo thang Richter của một trận động đất
     \choiceTF
    {\True Nếu năng lượng được tạo ra là $ 6{,}3\cdot 10^{14}$ erg thì trận động đất phải có độ lớn bằng 2 độ Richter (làm tròn kết quả đến hàng đơn vị)}
    {Trận động đất có độ lớn $3$ độ Richter tạo ra năng lượng bằng $\left(10^{163}-10^{10}\right)$ erg}
    {\True Trận động đất có độ lớn $5$ độ Richter tạo ra năng lượng gấp $1000$ lần so với trận động đất có độ lớn $3$ độ Richter}
    {\True Người ta ước lượng rằng một trận động đất có độ lớn khoảng từ $4$ đến $6$ độ Richter thì năng lượng do trận động đất đó tạo ra nằm trong khoảng từ $10^{17,8}$erg đến $10^{20{,}8}$ erg}
}
{
    \includegraphics[width=5cm]{img/HXN-11-13}
}
\loigiai{
    \begin{itemchoice}
        \itemch 
        Ta có $x=6{,}3\cdot 10^{14} \Rightarrow M=\dfrac{\log \left(6{,}3\cdot 10^{14}\right)-11{,}8}{1{,}5}\approx 2$ độ Richter.
        \itemch Ta có $M=3 \Rightarrow \log x=11{,}8+1{,}5\cdot 3\Rightarrow x=10^{16{,}3}$ erg. 
        \itemch Gọi $x_3$, $x_5$ lần lượt là năng lượng tạo ra bởi các trận động đất có độ lớn 3 và 5 độ Richter.
        Ta có hệ phương trình $\heva{& \log x_3=11{,}8+1{,}5\cdot 3&&(1) \\& \log x_5=11{,}8+1{,}5\cdot 5&&(2)} $\\
        Lấy $(1)-(2)$ ta được $\log x_3-\log x_5=-3\Rightarrow \log \dfrac{x_3}{x_5}=-3 \Rightarrow \dfrac{x_3}{x_5}=\dfrac{1}{1\,000}\Rightarrow x_5=1\,000x_3$.\\
        Vậy trận động đất có độ lớn $5$ độ Richter tạo ra năng lượng gấp $1\,000$ lần so với trận động đất có độ lớn 3 độ Richter.
        \itemch Gọi $x_4$, $x_6$ lần lượt là năng lượng tạo ra bởi các trận động đất có độ lớn 4 và 6 độ Richter. \\
        Ta có $\heva{& \log x_4=11{,}8+1{,}5\cdot 4\\& \log x_6=11{,}8+1{,}5\cdot 6} \Leftrightarrow \heva{& \log x_4=17{,}8 \\& \log x_6=20{,}8 } \Leftrightarrow \heva{& x_4=10^{17{,}8} \\& x_6=10^{20{,}8}.}$ \\
        Vậy một trận động đất có độ lớn khoảng từ $4$ đến $6$ độ Richter thì năng lượng mà nó tạo ra nằm trong khoảng từ $10^{17{,}8}$erg đến $10^{20{,}8}$erg.
    \end{itemchoice}
}
\end{ex}
\begin{ex}%Câu 14
\immini
{
    Tại một thành phố du lịch vào những ngày tháng $6$, người ta luôn chứng kiến trời nắng hoặc trời mưa (mỗi ngày có thể trời nắng xong đến mưa và ngược lại). Người ta biết được có $\dfrac{2}{3}$ số ngày trong tháng là có nắng, có $\dfrac{5}{6}$ số ngày trong tháng là có mưa. Nếu hôm nào bầu trời tại thành phố chỉ có mưa thì khả năng kẹt xe gấp đôi khả năng không kẹt xe; nếu hôm nào bầu trời chỉ có nắng thì chắc chắn thành phố không xảy ra kẹt xe; những ngày trời vừa có nắng vừa có mưa thì khả năng kẹt xe là $30\%$.
}
{
    \includegraphics[width=5cm]{img/HXN-11-14}
}
 \choiceTF
 {\True Nếu du khách đến thành phố vào ngày chỉ có mưa thì khả năng kẹt xe bằng $\dfrac{2}{3}$}
 {Trong tháng 6, có $10$ ngày mà thành phố vừa có mưa vừa có nắng}
 {Một du khách đang cân nhắc sẽ đến thành phố vào một ngày tháng 6, xác suất du khách gặp cảnh kẹt xe bằng $ 0{,}32$ (làm tròn đến hàng phần trăm)}
 {Sau khi du khách đến nơi thì hôm ấy xảy ra kẹt xe thật, xác suất để thành phố vừa có nắng, vừa có mưa bằng $0{,}3$ (làm tròn đến hàng phần chục)}
 \loigiai{
     \begin{itemchoice}
         \itemch Vào ngày chỉ có mưa, khả năng kẹt xe gấp đôi khả năng không kẹt xe nên khả năng kẹt xe tại thành phố là $\dfrac{2}{3}$ và khả năng không kẹt xe là $\dfrac{1}{3}$.
         \itemch Trong tháng sẽ có $\dfrac{2}{3}+\dfrac{5}{6}-1=\dfrac{1}{2}$ số ngày vừa có nắng vừa có mưa, có $\dfrac{2}{3}-\dfrac{1}{2}=\dfrac{1}{6}$ số ngày chỉ có nắng và có $\dfrac{5}{6}-\dfrac{1}{2}=\dfrac{1}{3}$ số ngày chỉ có mưa.\\
         Số ngày trong tháng vừa có mưa vừa có nắng tại thành phố là $\left(\dfrac{2}{3}+\dfrac{5}{6}-1\right)\cdot 30=15$ (ngày).
         \itemch Gọi A là biến cố: \lq\lq Ngày có nắng tại thành phố\rq\rq, B là biến cố: \lq\lq Ngày có mưa tại thành phố\rq\rq và C là biến cố \lq\lq Ngày có sự cố kẹt xe xảy ra tại thành phố\rq\rq. Ta tham khảo sơ đồ bên cạnh.\\
         \centerline{
             \includegraphics[width=6cm]{img/HXN-11-14-LG}
         }
         Xác suất kẹt xe xảy ra là $P(C)=\dfrac{1}{6}\cdot 0+\dfrac{1}{2}\cdot \dfrac{3}{10}+\dfrac{1}{3}\cdot \dfrac{2}{3}=\dfrac{67}{180}\approx 0{,}37$.
         \itemch Ta có: $P\left(AB|C\right)=\dfrac{P(ABC)}{P(C)}=\dfrac{\dfrac{1}{2}\cdot \dfrac{3}{10}}{\dfrac{67}{180}}=\dfrac{27}{67}\approx 0{,}4$.
     \end{itemchoice}
 }
\end{ex}
\begin{ex}%Câu 15
Một khối gỗ có dạng khối nón có bán kính đáy $r=30$ cm, chiều cao $h=120$ cm. Người thợ mộc tìm cách chế tác khối gỗ đó thành một khúc gỗ có dạng khối trụ như hình vẽ.\\
\centerline{
\includegraphics[width=.5\textwidth]{img/HXN-11-15}
}
 \choiceTF
 {\True Thể tích khối gỗ ban đầu bằng $36\,000\pi {cm^3}$}
 {\True Nếu người thợ mộc muốn tạo ra khối gỗ hình trụ có chiều cao bằng nửa khối gỗ ban đầu thì khối gỗ mới tạo ra có thể tích $ 13500\pi$}
 {\True Nếu người thợ mộc muốn tạo ra khối gỗ hình trụ có bán kính đáy bằng $\dfrac{2}{3}$ bán kính khối gỗ hình nón thì phần gỗ phải bỏ đi có thể tích bằng $ 20000\pi {cm^3}$}
 {\True Thể tích lớn nhất của khối gỗ hình trụ bằng $ 0,016\pi \,m^3$}
 \loigiai{
     \begin{itemchoice}
         \itemch Thể tích khối gỗ hình nón là $V=\dfrac{1}{3}\pi r^2h=\dfrac{1}{3}\pi \cdot 30^2\cdot 120=36\,000\pi \,cm^3$.
         \itemch Xét một nửa thiết diện qua trục của hình nón là tam giác SAB (tham khảo hình vẽ).\\
         \centerline{
             \includegraphics[width=5cm]{img/HXN-11-15-LG}
         }
         Các tam giác $SMN$ và $SAB$ đồng dạng (có $\widehat{S}$ chung, $\widehat{SMN}=\widehat{SAB}=90^0$),\\
         ta có $\dfrac{MN}{AB}=\dfrac{SM}{SA}\Leftrightarrow \dfrac{MN}{30}=\dfrac{1}{2}\Rightarrow MN=15\,cm$.\\
         Khối gỗ hình trụ mới được tạo thành có bán kính đáy $15\,cm$, chiều cao $60\,cm$ nên có thể tích $\pi \cdot 15^2\cdot 60=13\,500\pi \,cm^3$.
         \itemch Bán kính đáy khối gỗ hình trụ là $\dfrac{2}{3}\cdot 30=20\,cm$.\\
         Các tam giác $SMN$ và $SAB$ đồng dạng (có $\widehat{S}$ chung, $\widehat{SMN}=\widehat{SAB}=90^0$),\\
         ta có $\dfrac{MN}{AB}=\dfrac{SM}{SA}\Leftrightarrow \dfrac{2}{3}=\dfrac{SM}{120}\Rightarrow SM=80\,cm\Rightarrow AM=40\,cm$.\\
         Thể tích khối gỗ phải bỏ đi là $36\,000\pi -\pi \cdot 20^2\cdot 40=20\,000\pi \,cm^3$.
         \itemch Đặt $AM=x\,\,(cm)$ $(0<x<120)$. \\
         Ta có $\dfrac{MN}{AB}=\dfrac{SM}{SA}\Leftrightarrow \dfrac{MN}{30}=\dfrac{120-x}{120}\Rightarrow MN=30-\dfrac{x}{4}$.\\
         Vậy khối trụ mới được tạo thành có bán kính $MN=30-\dfrac{x}{4}$, đường cao $AM=x$ nên có thể tích: 
         $$V=\pi \left(30-\dfrac{x}{4}\right)^2x=\pi \cdot \left(30-\dfrac{x}{4}\right)\cdot \left(30-\dfrac{x}{4}\right)\cdot \dfrac{x}{2}\cdot 2$$
         Xét hàm số $V(x)=\pi \left(30-\dfrac{x}{4}\right)^2x=\pi \left(900x-15x^2+\dfrac{1}{16}x^3\right)$  với $0<x<120$.\\
         Ta có $V'(x)=\pi \left(900-30x+\dfrac{3x^2}{16}\right)=0 \Leftrightarrow \hoac{& x=120 &&\text{(loại)} \\& x=40 &&\text{(nhận)}}$.\\
         Bảng biến thiên:\\
         \begin{tikzpicture}[>=stealth]
             \tkzTabInit[nocadre=false,lgt=1.2,espcl=2.5,deltacl=0.5]{$x$/.7 ,$V'(x)$/.7,$V(x)$/2}
             {$0$ , $40$ , $120$}
             \tkzTabLine{ , + , $0$ , - , }
             \tkzTabVar{-/$1$ , +/$16\,000\pi$ , -/$5$}
         \end{tikzpicture}\\
         Dễ thấy $V_{\max}=16\,000\pi \,cm^3=0{,}016\pi \,m^3$.
     \end{itemchoice}
 }
\end{ex}
\begin{ex}%Câu 16
\immini
{
    Trong không gian với hệ tọa độ $Oxyz$, đơn vị trên mỗi trục tọa độ là $km$, NASA đang thiết kế một trạm vũ trụ hình cầu có tâm đặt tại điểm $ O(0;0;0)$ và bán kính $R=10$ km. Một con tàu vũ trụ di chuyển với tốc độ $ 27000$ km/h theo quỹ đạo là một đường thẳng qua điểm $ A(6;8;0)$ và có vectơ chỉ phương $\vec{u}=(2;2;-1)$
}
{
    \includegraphics[width=5cm]{img/HXN-11-16}
}
 \choiceTF
 {Điểm $ A(6;8;0)$ nằm bên trong trạm vũ trụ hình cầu}
 {\True Phương trình quỹ đạo di chuyển của tàu vũ trụ là $\Delta \colon \heva{& x=6+2t\\& y=8+2t\\& z=-t}$ ($t$ là tham số)}
 {Khoảng cách từ tâm của trạm vũ trụ đến đường thẳng quỹ đạo tàu vũ trụ là $3{,}8$ km (làm tròn đến hàng phần chục, đơn vị km)}
 {Tàu vũ trụ sẽ tốn $2{,}3$ giây (làm tròn đến hàng phần chục của giây) để vượt qua phạm vi hình cầu của trạm vũ trụ}
 \loigiai{
     \begin{itemchoice}
         \immini
         {
             \itemch Ta có $OA=\sqrt{(6-0)^2+(8-0)^2+(0-0)^2}=10=R$.
         Do vậy điểm $A$ nằm trên bề mặt hình cầu.
         \itemch  Đường thẳng $\triangle $ qua $A(6;8;0)$,\\
         có vectơ chỉ phương $\vec{u}=(2;2;-1)$ nên có phương trình tham số là $\triangle \colon \heva{& x=6+2t \\& y=8+2t \\& z=-t} $ ($t$ là tham số).
         }
         {
             \includegraphics[width=5cm]{img/HXN-11-16-LG}
         }
         \itemch Khoảng cách từ tâm $O$ đến đường thẳng quỹ đạo là $4$ km.\\
         Ta có $\heva{& \vec{OA}=(6;8;0) \\& \vec{u}=(2;2;-1)} \Rightarrow \left[\vec{OA},\vec{u}\right]=(-8;6;-4)$.\\
         Ta có $d\left(O,\triangle \right)=\dfrac{\left| \left[\vec{OA},\vec{u}\right] \right|}{\left| {\vec{u}} \right|}=\dfrac{\sqrt{(-8)^2+6^2+(-4)^2}}{\sqrt{2^2+2^2+(-1)^2}}=\dfrac{2\sqrt{29}}{3}\approx 3{,}6\,km$.\\
         \itemch Vì $d\left(O,\triangle \right)<R$ nên tàu vũ trụ sẽ đi qua phạm vi hình cầu của trạm vũ trụ, nó cắt mặt cầu tại hai điểm $A$ và $B$; gọi $H$ là trung điểm $AB$.\\
         Ta có $AB=2AH=2\sqrt{OA^2-OH^2}=2\sqrt{10^2-\left(\dfrac{2\sqrt{29}}{3}\right)^2}=\dfrac{56}{3}$ km.\\
         Thời gian khi tàu vũ trụ đi qua phạm vi hình cầu là $\left(\dfrac{56}{3}\colon 27\,000\right)\cdot 3\,600\approx 2{,}5$ (giây).
     \end{itemchoice}
 }
\end{ex}
\Closesolutionfile{ans}
\caukq
\Opensolutionfile{ans}[ans/ans-HXN-\sode-SA]
\begin{ex}%Câu 17
\immini
{
     Một khối đá có dạng hình lăng trụ tam giác đều $ ABC.A'B'C'$ với cạnh đáy bằng $2$ dm, khoảng cách tứ điểm $A'$ đến mặt phẳng $\left(AB'C"\right)$ bằng $\dfrac{\sqrt{3}}{2}$dm. Tìm khoảng cách giữa hai mặt phẳng đáy của khối đá hình lăng trụ đã cho theo đơn vị dm
 \shortans{1}
}
{
    \includegraphics[width=5cm]{img/HXN-11-17}
}
\loigiai{
    \immini
    {
        Trong mặt phẳng ($A'B'C'$), kẻ $A'H\perp B'C'$ tại $H$.\\
        Trong mặt phẳng $\left(AA'H\right)$, kẻ $A'K\perp AH$ tại $K$. \tagEX{1}
    Ta có $\heva{& B'C'\perp A'H \\& B'C'\perp AA'\,\left(do\,AA'\perp \left(A'B'C'\right)\right) }$\\
    $ \Rightarrow B'C'\perp \left(AA'H\right)\Rightarrow A'K\perp B'C'$. \tagEX{2}
    Từ $(1)$ và $(2)$ suy ra $A'K\perp \left(AB'C'\right)$ hay\\
    \centerline{
        $d\left(A',\left(AB'C'\right)\right)=A'K=\dfrac{\sqrt{3}}{2}$ dm.
    }
    Tam giác $A'B'C'$ đều có đường cao $A'H=\dfrac{2\cdot \sqrt{3}}{2}=\sqrt{3}$ dm.\\
    Tam giác $AA'H$ vuông tại $A'$ có đường cao $A'K$ nên\\
    \centerline{
        $\dfrac{1}{A'K^2}=\dfrac{1}{A'H^2}+\dfrac{1}{A'A^2} \Rightarrow \dfrac{1}{\dfrac{3}{4}}=\dfrac{1}{3}+\dfrac{1}{A'A^2}\Rightarrow A'A=1$ dm.
    }
    }
    {
        \includegraphics[width=5cm]{img/HXN-11-17-LG}
    }
     Hai mặt đáy song song với nhau và có khoảng cách là $d\left((ABC),\left(A'B'C'\right)\right)=AA'=1$ dm.
}
 \end{ex}
 \begin{ex}%Câu 18
\immini
{
    Giá bán $P$ (đồng) của một sản phẩm y tế thay đổi theo số lượng $ Q$ (sản phẩm) với $ 0\le Q\le 1\,500$, được cung cấp ra thị trường theo công thức $ P=\sqrt{1500-Q}$. Tính số lượng sản phẩm y tế đó nên được cung cấp ra thị trường để doanh thu của công ty lớn nhất.
\shortans{1000}
}
{
    \includegraphics[width=5cm]{img/HXN-11-18}
}
\loigiai{
    Doanh thu của sản phẩm được tính theo công thức $R=PQ=Q\sqrt{1\,500-Q}$.\\
    Ta có $R'=\dfrac{-3Q+3\,000}{2\sqrt{1\,500-Q}};R'=0\Leftrightarrow Q=1\,000$.\\
    So sánh $R(0)\,,R\left(1\,000\right)$ và $R\left(1\,500\right)$ ta có $R$ lớn nhất khi $Q=1\,000$.
}
\end{ex}
\begin{ex}%Câu 19
\immini
{
    Một tấm đề can hình chữ nhật được cuộn tròn lại theo chiều dài tạo thành một khối trụ có đường kính $50\text{(cm)}$ . Người ta trải ra $250$ vòng để cắt chữ và in tranh cổ động, phần còn lại là một khối trụ có đường kính $45\text{(cm)}$ . Hỏi phần đã trải ra dài bao nhiêu mét (làm tròn đến hàng đơn vị)?
\shortans{373}
}
{
    \includegraphics[width=5cm]{img/HXN-11-19}
}
\loigiai{
    \textbf{Cách giải 1:} Gọi $a$ là bề dày của tấm đề can, sau mỗi vòng được quấn thì đường kính của vòng mới sẽ được tăng lên $2a$.\\
    Vì vậy: $2a\times 250=50-45\Rightarrow a=\dfrac{50-45}{2\times 250}=0{,}01\,cm$.\\
    Gọi $l$ là chiều dài đã trải ra và $h$ là chiều rộng của tấm đề can (tức chiều cao hình trụ).\\
    Khi đó ta có: $lha=\pi \left(\dfrac{50}{2}\right)^2h-\pi \left(\dfrac{45}{2}\right)^2h \Rightarrow l=\dfrac{\pi \left(50^2-45^2\right)}{4a} \approx 37\,306cm \approx 373m$.\\
    \textbf{Cách giải 2:} Gọi $a$ là bề dày của tấm đề can, sau mỗi vòng được quấn thì đường kính của vòng mới sẽ được tăng lên $2a$.\\
    Vì vậy: $2a\times 250=50-45\Rightarrow a=\dfrac{50-45}{2\times 250}=0{,}01\,cm$.\\
    Chiều dài của phần trải ra là tổng chu vi của $250$ đường tròn có bán kính là một cấp số cộng có số hạng đầu bằng $r_1=25$, công sai là $d=-0{,}01$ (do khi trải ra thì bán kính các vòng tròn ngày càng giảm với độ giảm bằng bề dày của tấm đề can).\\
    Do đó chiều dài của phần đề can đã trải ra là: $l=2\pi \left(\underbrace{r_1+r_2+\cdot \cdot \cdot +r_{250}}_{S_{250}}\right)$
    $=2\pi \cdot \dfrac{(2r_1+249d)\cdot 250}{2} =2\pi (2\cdot 25-249\cdot 0{,}01)\dfrac{250}{2}\approx 37314cm \approx 373m$.
}
\end{ex}
\begin{ex}%Câu 20
\immini
{
    Một viên gạch lát nền nhà có dạng hình vuông với hoa văn được thiết kế bởi một học sinh lớp $12$. Xét hình phẳng có diện tích $S_1$ được tạo thành bởi các đường cong $\left(L_1\right),\left(L_2\right)$ và một cạnh viên gạch, trong đó đường $\left(L_1\right)$ là tập hợp các điểm M thỏa mãn $ MA=\sqrt{2}d\left(M,\Delta\right)$ (A là trung điểm một cạnh viên gạch; $\Delta $ là đường phân giác góc phần tư thứ nhất theo hình vẽ); $\left(L_2\right)$ đối xứng với $\left(L_1\right)$ qua trục Ox. Biết viên gạch này có kích thước cạnh bằng 40 cm, tính tổng diện tích $S_1+S_2+S_3+S_4$ và làm tròn đến hàng đơn vị của $ c{m^2}$
}
{
    \includegraphics[width=5cm]{img/HXN-11-20}
}
\loigiai{
    \immini
    {Tọa độ $A(20;0)$ và phương trình $\triangle \colon x-y=0$.\\
    Gọi $M(x;y)$ là tập hợp các điểm thuộc đường cong $\left(L_1\right)$, khi đó
    \begin{eqnarray*}
        &&MA=\sqrt{2}d\left(M,\triangle \right)\\
        &\Leftrightarrow& \sqrt{(x-20)^2+y^2}=\sqrt{2}\cdot \dfrac{|x-y|}{\sqrt{2}} \\
        &\Leftrightarrow& x^2+y^2-40x+400=x^2+y^2-2xy\\
        &\Leftrightarrow& xy=20x-200\\
        &\Leftrightarrow & y=\dfrac{20x-200}{x}
    \end{eqnarray*}
    }
    {
        \includegraphics[width=5cm]{img/HXN-11-20-LG}
    }
    Giao điểm giữa $\left(L_1\right)$ với $Ox$ thỏa hệ phương trình $\heva{& y=\dfrac{20x-200}{x} \\& y=0 } \Rightarrow x=10$.\\
    Ta có: $S_1+S_2+S_3+S_4=4S_1=4\cdot 2\int\limits_{10}^{20}{\dfrac{20x-200}{x}\mathrm{\,d}x}\approx 491\,cm^2$.
    
}
\shortans{491}
\end{ex}
\begin{ex}%Câu 21
\immini
{
    Hộp thứ nhất đựng $6$ viên bi xanh và $4$ viên bi vàng, hộp thứ hai đựng $5$ viên bi xanh và một số bi vàng. Người ta thực hiện ngẫu nhiên ba hành động sau:
\begin{itemize}
    \item Lấy ngẫu nhiên $2$ viên bi từ hộp thứ nhất bỏ sang hộp thứ hai.
    \item Chọn ngẫu nhiên $1$ viên bi từ hộp thứ hai rồi hoàn lại hộp này.
    \item Chọn ngẫu nhiên $1$ viên bi từ hộp thứ hai lần nữa.
\end{itemize}
}
{
    \includegraphics[width=5cm]{img/HXN-11-21}
}
Biết hai lần lấy bi từ hộp thứ hai đều được bi xanh, tính xác suất để hai lần lấy chọn trúng 1 viên bi, và đó cũng là viên bi từ hộp thứ nhất chuyển sang hộp thứ hai (làm tròn kết quả đến hàng phần trăm).
\shortans{0,03}
\loigiai{
    Gọi $A$ là biến cố: \lq\lq Hai lần lấy bi từ hộp thứ hai đều được bi có màu xanh\rq\rq\, và $B$ là biến cố: \lq\lq Hai lần lấy đúng 1 bi và cũng là bi từ hộp thứ nhất chuyển qua\rq\rq.\\
    Giả sử ban đầu hộp thứ hai có $5$ viên bi xanh và $x$ viên bi vàng $\left(x\in \mathbb{N}^*\right)$.\\
    Xác suất để lấy được $2$ bi xanh từ hộp thứ nhất chuyển sang hộp thứ hai là $\dfrac{\mathrm{C}_6^2}{\mathrm{C}_{10}^2}=\dfrac{1}{3}$.\\
    Xác suất để lấy được $1$ bi xanh, $1$ bi vàng từ hộp thứ nhất chuyển sang hộp thứ hai là $\dfrac{\mathrm{C}_6^1\cdot \mathrm{C}_4^1}{\mathrm{C}_{10}^2}=\dfrac{8}{15}$.\\
    Xác suất để lấy được $2$ bi vàng từ hộp thứ nhất chuyển sang hộp thứ hai là $\dfrac{\mathrm{C}_4^2}{\mathrm{C}_{10}^2}=\dfrac{2}{15}$.\\
    Ta biểu diễn bài toán bằng sơ đồ sau:\\
    \centerline{
        \includegraphics[width=5cm]{img/HXN-11-21-LG}
    }
    Ta có $P(A)=\dfrac{1}{3}\cdot \left(\dfrac{7}{x+7}\right)^2+\dfrac{8}{15}\cdot \left(\dfrac{6}{x+7}\right)^2+\dfrac{2}{15}\cdot \left(\dfrac{5}{x+7}\right)^2$.\\
    Do đó 
    \begin{eqnarray*}
        P\left(B|A\right)&=&\dfrac{\dfrac{1}{3}\cdot \dfrac{2}{x+7}\cdot \dfrac{1}{x+7}+\dfrac{8}{15}\cdot \dfrac{1}{x+7}\cdot \dfrac{1}{x+7}}{\dfrac{1}{3}\cdot \left(\dfrac{7}{x+7}\right)^2+\dfrac{8}{15}\cdot \left(\dfrac{6}{x+7}\right)^2+\dfrac{2}{15}\cdot \left(\dfrac{5}{x+7}\right)^2}\\
        &=&\dfrac{\dfrac{1}{(x+7)^2}\left(\dfrac{2}{3}+\dfrac{8}{15}\right)}{\dfrac{1}{(x+7)^2}\left(\dfrac{7^2}{3}+\dfrac{8\cdot 6^2}{15}+\dfrac{2\cdot 5^2}{15}\right)}\\
        &=&\dfrac{18}{583}\approx 0{,}03.
    \end{eqnarray*}
}
\end{ex}
\begin{ex}%Câu 22
\immini
{
Trong không gian với hệ trục tọa độ $Oxyz$ thích hợp, đơn vị trên mỗi trục là mét. Một nhà sinh vật học muốn theo dõi hai tổ chim ở các vị trí $ A\left(2;2;0\right)$, $B\left(2;0;-2\right)$, anh ta đã leo lên mái nhà thuộc mặt phẳng $(P)\colon x+2y-z-1=0$. Nhà sinh vật học muốn đặt một thiết bị theo dõi ở vị trí $ M\left(a;b;c\right)$ thuộc mái nhà cách đều các tổ chim, đồng thời vị trí đó cho anh một góc quan sát là lớn nhất đối với hai tổ chim nói trên (góc $\widehat{AMB}$ lớn nhất). Khi đó tổng $ a+b+c$ bằng bao nhiêu (làm tròn đến hàng phần trăm)?
\shortans{1,27}
}
{
    \includegraphics[width=5cm]{img/HXN-11-22}
}
\loigiai{
    Ta có $M$ thuộc mặt phẳng $(P)$ và $MA=MB$ nên $M$ thuộc giao tuyến của hai mặt phẳng $(P)$ và $(Q)$, trong đó $(Q)$ là mặt phẳng trung trực của đoạn thẳng $AB$.\\
    Mặt phẳng $(Q)$ qua trung điểm $I(2;1;-1)$ của $AB$, vectơ pháp tuyến $\vec{AB}=(0;-2;-2)$ nên có phương trình  $0(x-2)-2(y-1)-2(z+1)=0$ hay $y+z=0$.\\
    Gọi $d=(P)\cap (Q)$; từ hệ phương trình $\heva{& x+2y-z-1=0 \\& y+z=0 }$, đặt $z=t$ ta có phương trình tham số đường thẳng $d$ là $\heva{& x=1+3t \\& y=-t \\& z=t}$.\\
    Với $M\left(1+3t;-t;t\right)\in d$; suy ra $\vec{AM}=\left(3t-1;-t-2;t\right)$, $\vec{BM}=\left(3t-1;-t;t+2\right)$.\\
    Ta có $\cos \widehat{AMB}=\cos \left(\vec{AM},\vec{BM}\right)=\dfrac{(3t-1)^2+2\left(t^2+2t\right)}{(3t-1)^2+t^2+(t+2)^2}=\dfrac{11t^2-2t+1}{11t^2-2t+5}=1-\dfrac{4}{11t^2-2t+5}$.\\
    Ta thấy $\widehat{AMB}$ lớn nhất khi $\cos \widehat{AMB}$ bé nhất; suy ra $1-\dfrac{4}{11t^2-2t+5}$ bé nhất.\\
    Khi đó $\dfrac{4}{11t^2-2t+5}$ lớn nhất nên $11t^2-2t+5=11\left(t-\dfrac{1}{11}\right)^2+\dfrac{54}{11}$ bé nhất; suy ra $t=\dfrac{1}{11}$.\\
    Ta tìm được điểm $M\left(\dfrac{14}{11};-\dfrac{1}{11};\dfrac{1}{11}\right)$ với $a=\dfrac{14}{11}$; $b=-\dfrac{1}{11}$; $c=\dfrac{1}{11}\Rightarrow S=a+b+c=\dfrac{14}{11}\approx 1{,}27$.
}
\end{ex}
\Closesolutionfile{ans}
\inputansbox{6,4,3}{ans/ans-HXN-\sode-T,ans/ans-HXN-\sode-TF,ans/ans-HXN-\sode-SA}
% %%%%%%%%%%%%%%%%%%%- HXN
\def\sode{12}
\def\tendethi{ĐỀ PHÁT TRIỂN MINH HOẠ 2025}
\begin{dethi}
 {\tendethi}
\end{dethi}
\caulc
\Opensolutionfile{ans}[ans/ans-HXN-\sode-T]
\begin{ex}%Câu 1
 Tìm tập xác định của hàm số $ y=\log_2\left(x-3\right)$.
 \choice
 {$\mathscr{D}=\left(-\infty;3\right)$}
 {$\mathscr{D}=\mathbb{R}$}
 {\True $\mathscr{D}=\left(3;+\infty\right)$}
 {$\mathscr{D}=\left[3;+\infty\right)$}
\end{ex}
\begin{ex}%Câu 2
 Số đường tiệm cận của đồ thị hàm số $y=\dfrac{3x-4}{x-1}$ bằng
 \choice
 {\True $2$}
 {$3$}
 {$1$}
 {$0$}
\end{ex}
\begin{ex}%Câu 3
 Độ $pH$ của một dung dịch được tính theo công thức $ pH=-\log\left[H^+\right]$ với $\left[H^+\right]$ là nồng độ ion hydrogen của dung dịch đó. Độ $ pH$ của một loại sữa có $\left[H^+\right]=10^{-6{,}8}$ là bao nhiêu?
 \choice
 {$-6{,}8$}
 {$68$}
 {\True $6{,}8$}
 {$0{,}68$}
\end{ex}
\begin{ex}%Câu 4
 Cho hình lăng trụ đứng $ABC.A'B'C'$ có $BB'=a$ , đáy $ABC$ là tam giác vuông cân tại $ B$ và $BA=BC=a$ . Tính thể tích $ V$ của khối lăng trụ đã cho.
 \choice
 {$V=a^3$}
 {$V=\dfrac{a^3}{3}$}
 {$V=\dfrac{a^3}{6}$}
 {\True $V=\dfrac{a^3}{2}$}
\end{ex}
\begin{ex}%Câu 5
 Trong không gian với hệ tọa độ $ Oxyz$, cho mặt phẳng $\left(\alpha\right)\colon x-2y+2z-3=0.$ Điểm nào sau đây thuộc mặt phẳng $\left(\alpha\right)$?
 \choice
 {$M\left(2;0;1\right)$}
 {$Q\left(2;1;1\right)$}
 {$P\left(2;-1;1\right)$}
 {\True $N\left(1;0;1\right)$}
\end{ex}
\begin{ex}%Câu 6
 Cho hình hộp $ ABCD.A'B'C'D'$. Vectơ $\overrightarrow{v}=\overrightarrow{B'A'}+\overrightarrow{B'C'}+\overrightarrow{B'B}$ bằng vectơ nào dưới đây?
 \choice
 {$\overrightarrow{DB'}$}
{$\overrightarrow{B'D'}$}
{$\overrightarrow{BD'}$}
{\True $\overrightarrow{B'D}$}
\end{ex}
\begin{ex}%Câu 7
Họ nguyên hàm của hàm số $ f(x)=3x^2+\sin x$ là 
\choice
{$x^3+\cos x+C$}
{$x^3+\sin x+C$}
{\True $x^3-\cos x+C$}
{$3x^3-\sin x+C$}
\end{ex}
\begin{ex}%Câu 8
Trong không gian $ Oxyz$, phương trình nào sau đây là phương trình của mặt cầu có tâm $ I\left(7;6;-5\right)$ và bán kính $ 9$?
\choice
{$\left(x+7\right)^2+\left(y+6\right)^2+\left(z-5\right)^2=81$}
{$\left(x+7\right)^2+\left(y+6\right)^2+\left(z-5\right)^2=9$}
{\True $\left(x-7\right)^2+\left(y-6\right)^2+\left(z+5\right)^2=81$}
{$\left(x-7\right)^2+\left(y-6\right)^2+\left(z+5\right)^2=9$}
\end{ex}
\begin{ex}%Câu 9
Bảng số liệu ghép nhóm về chiều cao đo được (đơn vị: cm) của $30$ học sinh nam lớp 12A2 đầu năm học $2024-2025$ của một trường THPT được cho như sau:\\
\centerline{\begin{tblr}{|c|c|c|c|c|c|}
 \hline
 Chiều cao & $\left[150;\ 155\right)$ & $\left[155;\ 160\right)$ & $\left[160;\ 165\right)$ & $\left[165;\ 170\right)$ & $\left[170;\ 175\right)$\\
 \hline
 Tần số & $ 3$ & $ 7$ & $ 10$ & $ 7$ & $ 3$\\
 \hline
\end{tblr}}\\
Tính độ lệch chuẩn của mẫu số liệu ghép nhóm trên.
\choice
{\True $\dfrac{\sqrt{285}}{3}$}
{$\dfrac{\sqrt{287}}{3}$}
{$ 4\sqrt{2}$}
{$\sqrt{71}$}
\end{ex}
\begin{ex}%Câu 10
Cho biết $\lim\limits_{x\to\sqrt{3}}\dfrac{2x^2-6}{x-\sqrt{3}}=a\sqrt{b}$ ($a$, $b$ nguyên và $b<10$). Khi đó giá trị của $P=a+b$ bằng
\choice
{\True $7$}
{$10$}
{$5$}
{$6$}
\end{ex}
\begin{ex}%Câu 11
Cho $\int\dfrac{1}{x{\ln^2}x}\mathrm{\,d}x=F(x)+C$. Khẳng định nào dưới đây đúng?
\choice
{$F'(x)=-\dfrac{1}{\ln x}$}
{$F'(x)=-\dfrac{1}{\ln x}+C$}
{\True $F'(x)=\dfrac{1}{x{\ln^2}x}$}
{$F'(x)=-\dfrac{1}{\ln^2x}$}
\end{ex}
\begin{ex}%Câu 12
\immini
{
 Diện tích phần gạch sọc trong hình vẽ bằng
\choice
{$\int\limits_{-3}^1\left|-x^2-2x-3\right|\mathrm{\,d}x$}
{$\int\limits_{-3}^1\left(x^2-2x-3\right)\mathrm{\,d}x$}
{$\int\limits_{-3}^1\left(x^2+2x-3\right)\mathrm{\,d}x$}
{\True $\int\limits_{-3}^1\left(-x^2-2x+3\right)\mathrm{\,d}x$}
}
{
 \begin{tikzpicture}[>=stealth, line join=round, line cap=round, font=\footnotesize, scale=1,thick,x=.5cm,y=.5cm,
 declare function={f(\x)=-1*(\x)+1;g(\x)=1*(\x)^2+1*(\x)-2;}]
 \draw[->] (-3.5,0)--(2.5,0) node[below left] {$x$};
 \draw[->] (0,-3)--(0,4.5) node[below left] {$y$};
 \draw (0,0) node [below left] {$O$};
 \foreach \x/\nx in {-3/-3,1/1}
 \draw (\x,1pt)--(\x,-1pt) node [below] {$\nx$};
 \draw[pattern = north east lines, pattern color=green, line width = 1.2pt,draw=none] (-3,{f(3)}) plot[domain=1:-3] (\x, {g(\x)});
 \draw[dashed] (-3,0)--(-3,{f(-3)});
 \draw[samples=200,domain=-3.5:1.5,smooth] plot (\x,{f(\x)});
 \draw[samples=200,domain=-3.1:1.2,smooth,variable=\x] plot (\x,{g(\x)});
 \end{tikzpicture}
}
\end{ex}
\Closesolutionfile{ans}
\cauds
\Opensolutionfile{ans}[ans/ans-HXN-\sode-TF]
\begin{ex}%Câu 13
    Cho hàm số $f(x)=\cos 2x+2x+1$
    \choiceTF
    {\True $ f\left(\dfrac{\pi}{2}\right)=\pi $}
    {Đạo hàm của hàm số là $f'(x)=2\sin 2x+2$}
    {\True Nghiệm của phương trình $f'(x)=0$ trên đoạn $\left[-\dfrac{\pi}{2};\pi\right]$ là $ x=\dfrac{\pi}{4}$}
    {Tổng giá trị lớn nhất và giá trị nhỏ nhất của hàm số trên đoạn $\left[-\dfrac{\pi}{2};\pi\right]$ bằng $ 2\pi $}
    \loigiai{
        \begin{itemchoice}
            \itemch Ta có: $f\left(\dfrac{\pi }{2}\right)=\cos \left(2\cdot \dfrac{\pi }{2}\right)+2\cdot \dfrac{\pi }{2}+1=\pi $.
            \itemch Ta có $f'(x)=-2\sin 2x+2$.
            \itemch $f'(x)=0\Leftrightarrow -2\sin 2x+2=0\Leftrightarrow \sin 2x=1\Leftrightarrow 2x=\dfrac{\pi }{2}+k2\pi \,,k\in \mathbb{Z}\Leftrightarrow x=\dfrac{\pi }{4}+k\pi ,k\in \mathbb{Z}$.
            Vì $x\in \left[-\dfrac{\pi }{2};\pi \right]$ nên $x=\dfrac{\pi }{4}$ (ứng với $k=0$).
            \itemch Ta có: $f\left(-\dfrac{\pi }{2}\right)=-\pi $; $f\left(\dfrac{\pi }{4}\right)=\dfrac{\pi }{2}+1$; $f\left(\pi \right)=2\pi +2$.\\
            Vì vậy $\max\limits_{\left[-\dfrac{\pi}{2};\pi \right]} \,f(x)=f\left(\pi \right)=2\pi +2$ và $\min\limits_{\left[-\dfrac{\pi}{2};\pi \right]}\,f(x)=f\left(-\dfrac{\pi }{2}\right)=-\pi $.\\
            Tổng giá trị của chúng là $2\pi +2+\left(-\pi \right)=\pi +2$.
        \end{itemchoice}
    }
\end{ex}
\begin{ex}%Câu 14
\immini
{
 Trong một hệ trục $Oxyz$ cho trước, đơn vị trên mỗi trục là mét, có hai cậu bé đang chơi bắn bi trên mặt đất phẳng (cũng là mặt phẳng $Oxy$). Gần vị trí các cậu bé có một bức tường được mô hình hóa bởi mặt phẳng có phương trình $x+y-10=0$.\\
Cậu bé An bắn viên bi từ vị trí $ A\left(0;0;0\right)$ và theo hướng vectơ $\vec{u}=\left(4;-1;0\right)$, bi có thể di chuyển $10{,}5$ mét nếu không có vật cản.
}
{
 \includegraphics[width=5.5cm]{img/HXN-12-14}
}
Cậu bé Cò bắn viên bi từ vị trí $B\left(2;-2;0\right)$ theo hướng $\vec{v}=\left(1;2;0\right)$, bi có thể di chuyển $8{,}2$ mét nếu không có vật cản.
\choiceTF
 {\True Đường đi viên bi là của An có phương trình $\heva{& x=4t\\& y=-t\\& z=0}$ ($t$ là tham số)}
{\True Đường đi của hai viên bi là các đường thẳng cắt nhau tại điểm có tung độ âm}
 {Đường đi của viên bi do bạn Cò bắn ra tạo với bức tường một góc thuộc khoảng $\left(69^{\circ};71^{\circ}\right)$}
 {\True Một trong hai viên bi chạm vào bức tường}
 \loigiai{
 \begin{itemchoice}
 \itemch Đường đi viên bi là của An có phương trình $d_1\colon \heva{& x=4t \\& y=-t \\& z=0 } $ ($t$ là tham số).
 \itemch Đường đi viên bi là của Cò có phương trình $d_2\colon \heva{& x=2+t' \\& y=-2+2t' \\& z=0 } $ ($t$ là tham số).\\
 Xét hệ phương trình $\heva{& 4t=2+t' \\& -t=-2+2t' \\& 0=0 } \Rightarrow \heva{& t=\dfrac{2}{3} \\& t'=\dfrac{2}{3} } $; đường đi $2$ viên bi cắt nhau tại $M\left(\dfrac{8}{3};-\dfrac{2}{3};0\right)$; trong đó $y_M=-\dfrac{2}{3}<0$ (thỏa mãn).
 \itemch Đường thẳng $d_2$ có vectơ chỉ phương $\vec{v}=(1;2;0)$; mặt phẳng chứa bức tường có vectơ pháp tuyến $\vec{n}=(1;1;0)$ nên $\sin \left(d_2\,,(P)\right)=\dfrac{\left| \vec{v}\cdot \vec{n} \right|}{\left| {\vec{v}} \right|\cdot \left| {\vec{n}} \right|}=\dfrac{|1+2|}{\sqrt{5}\cdot \sqrt{2}}=\dfrac{3\sqrt{10}}{10}$.\\
 $\Rightarrow \left(d_2,(P)\right)\approx 71,6^{\circ }\notin \left(69^{\circ };71^{\circ }\right)$.\\
 \itemch Xét $d_1$ và $(P)\colon x+y-10=0$;\\
 thay phương trình $d_1$ vào $(P)$, ta có $4t+(-t)-10=0\Rightarrow t=\dfrac{10}{3}$.\\
 Do đó $d_1\cap (P)=C\left(\dfrac{40}{3};-\dfrac{10}{3};0\right)$ với $AC=\dfrac{10\sqrt{17}}{3}\approx 13{,}7\,m>10{,}5\,m$.\\
 Vậy viên bi của An không chạm tường.\\
 Xét $d_2$ và $(P)\colon x+y-10=0$;\\
 thay phương trình $d_2$ vào $(P)$, ta có $(2+t)+\left(-2+2t'\right)-10=0\Rightarrow t'=\dfrac{10}{3}$.\\
 Do đó $d_2\cap (P)=D\left(\dfrac{16}{3};\dfrac{14}{3};0\right)$ với $BD=\dfrac{10\sqrt{5}}{3}\approx 7{,}45\,m<8{,}2\,m$.\\
 Vậy viên bi của Cò chạm vào tường.
 \end{itemchoice}
 }
\end{ex}
\begin{ex}%Câu 15
\immini
{
 Trong một live show âm nhạc có ca sĩ Mỹ Tâm tham gia, nhiều fan hâm mộ đã tỏ ra lo ngại rằng ban tổ chức có thể hủy show với một vài lý do khác nhau. Những lo ngại này là đúng vì có đến $0{,}302$ khả năng show diễn sẽ bị hủy.\\
Nếu vé bán hết thì chắc chắn live show sẽ diễn ra.\\
Nếu trời mưa thì ban tổ chức không thể bán hết vé, khi đó khả năng hủy show là $50\%$.\\
Nếu trời không mưa thì khả năng vé được bán hết là $90\%$; trong trường hợp còn vé thì khả năng hủy show là $5\%$.
}
{
 \includegraphics[width=5.5cm]{img/HXN-12-15}
}
\choiceTF
 {\True Nếu vé chưa được bán hết khi trời không mưa thì xác suất để show vẫn diễn ra bằng $0{,}95$}
 {Xác suất để trời mưa bằng $0{,}55$}
 {\True Xác suất để ban tổ chức không bán hết vé bằng $0{,}64$}
 {\True Sau cùng thì show của Mỹ Tâm cũng đã diễn ra, xác suất hôm đó trời mưa bằng $0{,}43$ (làm tròn kết quả đến hàng phần trăm)}
 \loigiai{
 \begin{itemchoice}
 \immini
 {
 \itemch Nếu vé chưa được bán hết khi trời không mưa thì xác suất để show vẫn diễn ra bằng $1-0{,}05=0{,}95$.
 \itemch Gọi A là biến cố \lq\lq Trời mưa\rq\rq, B là biến cố: \lq\lq Vé được bán hết\rq\rq, C là biến cố: \lq\lq Show bị hủy\rq\rq.\\
 Đặt $P(A)=x\in (0;1)$ là xác suất để trời mưa, ta có sơ đồ hình cây bên cạnh.\\
 Khi đó $P(C)=x\cdot 1.0{,}5+(1-x)\cdot 0{,}9.0+(1-x)\cdot 0{,}1.0{,}05=0{,}302\Rightarrow x=0{,}6$ hay $P(A)=0{,}6$.
 }
 {
 \includegraphics[width=5.5cm]{img/HXN-12-15-LG}
 }
 \itemch Ta có $P\left({\bar{B}}\right)=\underbrace{x\cdot 1+(1-x)\cdot 0{,}1}_{x=0{,}6}=0{,}64$.
 \itemch Ta có $P\left({\bar{C}}\right)=1-0{,}302=0{,}698$; $P\left(A|\bar{C}\right)=\dfrac{P\left(A\bar{C}\right)}{P\left({\bar{C}}\right)}=\dfrac{0{,}6.1\cdot 0{,}5}{0{,}698}=\dfrac{150}{349}\approx 0{,}43$.
 \end{itemchoice}
 }
\end{ex}
\begin{ex}%Câu 16
    \immini
    {
        Cho đường cong $(C)\colon y=8x-27x^3$ và đường thẳng $d\colon y=m$ cắt $(C)$ tại ba điểm phân biệt, trong đó có hai điểm nằm ở góc phần tư thứ nhất của hệ tọa độ $ Oxy$. Các phần diện tích tô đậm được kí hiệu là $S_1$, $S_2$ như hình vẽ.
        \choiceTF
        {Diện tích hình phẳng giới hạn vởi $(C)$ và $Ox$ bằng $\dfrac{34}{27}$}
        {Có ba giá trị nguyên của $m$ thỏa mãn đề bài}
        {\True Nếu $ m=\dfrac{5}{8}$ thì $S_1+S_2=0,31$ (làm tròn đến hàng phần trăm)}
        {Nếu $S_1=S_2$ thì có một giá trị $ m=m_0$ thỏa mãn với $ 1<m_0<\dfrac{3}{2}$}
    }
    {
        \begin{tikzpicture}[>=stealth, line join=round, line cap=round, thick,declare function={f(\x)=-27*((\x)^3)+0*((\x)^2)+8*(\x)+0;m=7/8;g(\x)=m;},xscale=3]
            \draw[->] (-1,0)--(1,0) node[below left] {$x$};
            \draw[->] (0,-2.5)--(0,2.5) node[below left] {$y$};
            \draw (0,0) node [below left] {$O$};
            \begin{scope}
                \clip (-1,-2.2) rectangle (1,2.2);
                \draw[samples=200,domain=-2:2,smooth,name path=df] plot (\x,{f(\x)});
                \draw[name path=dg] (-2,m)--(2,m)node[pos=.72,above]{$d$};
                \path [name intersections= {of = df and dg, by={a,b,c}}];
            \end{scope}
            \fill[pattern=north east lines] let \p2=(b),\n2={\x2/1cm} in plot[domain=0:\n2](\x,{f(\x)})--(0,m)--cycle;
            \fill[pattern=north west lines,pattern color=red] let \p2=(b),\n2={\x2/1cm},\p3=(c),\n3={\x3/1cm} in plot[domain=\n2:\n3](\x,{f(\x)})--cycle;
            \draw[->] (.07,.7)--++(-155:.3)node[below left]{$S_1$};
            \draw[->] (.3,1.3)--++(55:.5)node[above right]{$S_2$};
        \end{tikzpicture}
    }
    \loigiai{
        \begin{itemchoice}
            \itemch Hoành độ giao điểm của $(C)$ và $Ox$ thỏa mãn $8x-27x^3=0\Leftrightarrow \hoac{& x=0 \\& x=\pm \dfrac{2\sqrt{6}}{9} } $.\\
            Diện tích cần tính là $S=\displaystyle\int\limits_{-\tfrac{2\sqrt{6}}{9}}^{\tfrac{2\sqrt{6}}{9}}{\left| 8x-27x^3 \right|\mathrm{\,d}x}=\dfrac{32}{27}$.
            \itemch Xét $(C):y=8x-27x^3$ có đạo hàm $y'=8-81x^2;y'=0\Rightarrow x=\pm \dfrac{2\sqrt{2}}{9}$.\\
            Bảng biến thiên:
            \begin{center}
                \begin{tikzpicture}[>=stealth]
                    \tkzTabInit[nocadre=false,lgt=1,espcl=2.5,deltacl=0.5]{$x$/1,$y'$/.7,$y$/2}
                    {$-\infty$ , $-\dfrac{2\sqrt{2}}{9}$ , $\dfrac{2\sqrt{2}}{9}$ , $+\infty$}
                    \tkzTabLine{ ,-,$0$,+,$0$,-, }
                    \tkzTabVar{+/$+\infty$,-/$\approx -1{,}7$,+/$\approx 1{,}7$,-/$-\infty$}
                \end{tikzpicture}
            \end{center}
            Xét đường thẳng d:$y=m$ cắt đồ thị hàm số tại ba điểm phân biệt (giả thiết); mà $m$ nguyên nên $m\in \left\{ -1;0;1 \right\}$. Tuy nhiên chỉ có $m=1$ thỏa mãn vì khi đó d và (C) có hai giao điểm thuộc góc phần tư thứ nhất. 
            \itemch 
            Nếu $m=\dfrac{5}{8}$ thì ta có phương trình hoành độ giao điểm: $8x-27x^3=\dfrac{5}{8}\Leftrightarrow \hoac{& x=\dfrac{1}{2} \\& x=\dfrac{-9\pm \sqrt{141}}{36} } $.\\
            Ta chỉ xét góc phần tư thứ nhất nên nhận $x=\dfrac{1}{2};x=\dfrac{-9+\sqrt{141}}{36}$.\\
            Khi đó $S_1+S_2=\displaystyle\int\limits_0^{\tfrac{1}{2}}{\left| \left(8x-27x^3\right)-\dfrac{5}{8} \right|\mathrm{\,d}x}\approx 0{,}31$.
            \itemch 
            Giả sử đường thẳng $y=m$ cắt đồ thị $(C)$ tại điểm có hoành độ $a,b\,\left(0<a<b\right)$.\\
            Ta có: $8a-27a^3=m$ \tagEX{1}
            Xét hàm số $f(x)=8x-27x^3-m$ có $F(x)=\displaystyle\int{f(x)}\mathrm{\,d}x=4x^2-\dfrac{27}{4}x^4-mx+C$.\\
            Ta có 
            $
            \begin{aligned}[t]
                &S_1=\displaystyle\int\limits_0^a{\left| f(x) \right|}\mathrm{\,d}x=-\displaystyle\int\limits_0^a{f(x)}\mathrm{\,d}x=F(0)-F(a);\\ &S_2=\displaystyle\int\limits_a^b{\left| f(x) \right|}\mathrm{\,d}x=\displaystyle\int\limits_a^b{f(x)}\mathrm{\,d}x=F(b)-F(a).
            \end{aligned}
            $\\
            Theo giả thiết: $S_1=S_2\Leftrightarrow F(0)-F(a)=F(b)-F(a)\Leftrightarrow F(b)=F(0)$.\\
            Vì vậy $4b^2-\dfrac{27}{4}b^4-mb=0$ \tagEX{2}
            Từ $(1)$ và $(2)$ ta có $\heva{& 4b^2-\dfrac{27b^4}{4}-mb=0 \\& 8b-27b^3=m } \Rightarrow \heva{& 4b^2-\dfrac{27b^4}{4}-\left(8b-27b^3\right)b=0 \\& 8b-27b^3=m } \Rightarrow \heva{& b=\dfrac{4}{9}>0 \\& m=\dfrac{32}{27}.} $\\
            Vậy $m=\dfrac{32}{27}\approx 1{,}19\in \left(1;\dfrac{3}{2}\right)$.
        \end{itemchoice}
    }
\end{ex}
\Closesolutionfile{ans}
\caukq
\Opensolutionfile{ans}[ans/ans-HXN-\sode-SA]
\begin{ex}%Câu 17
 Cho hình chóp $S.ABCD$ có đáy là hình vuông cạnh $ 1$, tam giác $SAB$ đều và nằm trong mặt phẳng vuông góc với đáy. Tìm khoảng cách giữa hai đường thẳng SA, BC và làm tròn đến hàng phần trăm.
 \shortans{0,87}
 \loigiai{
\immini
{
         Gọi $H$ là trung điểm $AB$, suy ra $SH\perp AB$ (do tam giác $SAB$ đều).\\
     Mặt khác $(SAB)\perp (ABCD)$ nên $SH\perp (ABCD)$.\\
     Ta có $\heva{& (SAB)\perp (ABCD) \\& AB=(SAB)\cap (ABCD) \\& BC\perp AB } \Rightarrow BC\perp (SAB)$ \tagEX{1}
     Trong mặt phẳng $(SAB)$, dựng $BK\perp SA$ tại $K$ \tagEX{2}
     Từ $(1)$, $(2)$ suy ra $BK$ là đoạn vuông góc chung của $SA$ và $BC$.
     Vậy $d\left(SA, BC\right)=BK=\dfrac{\sqrt{3}}{2}\approx 0{,}87$.
}
{
\includegraphics[width=5cm]{img/HXN-12-17-LG}
}
 }
 \end{ex}
 \begin{ex}%Câu 18
\immini
{
    Một trò chơi điện tử quy định như sau: Có 5 trụ $ A$, $B$, $C$, $D$, $E$ với số lượng các thử thách trên đường đi giữa các cặp trụ được mô tả trong hình bên. Người chơi xuất phát từ một trụ nào đó, đi qua tất cả các trụ còn lại, mỗi khi đi qua trụ nào thì trụ đó sẽ bị phá hủy và không thể quay trở lại trụ đó được nữa, nhưng người chơi vẫn phải trở về trụ ban đầu. Tổng số thử thách của đường đi thoả mãn điều kiện trên nhận giá trị nhỏ nhất là bao nhiêu?
\shortans{45}
}
{
    \includegraphics[width=5cm]{img/HXN-12-18}
}
\end{ex}
\begin{ex}%Câu 19
\immini
{
    Giả sử số lượng của một quần thể nấm men tại môi trường nuôi cấy trong phòng thí nghiệm được mô hình hóa bằng hàm số $ P(t)=\dfrac{a}{b+e^{-0,75t}}$. Trong đó, thời gian $ t$ được tính bằng giờ. Tại thời điểm ban đầu $t=0$, quần thể có 20 tế bào và tăng với tốc độ $12$ tế bào/giờ. Số lượng của quần thể nấm này tại thời điểm $ t=8$ giờ là bao nhiêu tế bào (làm tròn đến hàng đơn vị)?
\shortans{99}
}
{
    \includegraphics[width=5cm]{img/HXN-12-19}
}
\loigiai{
    Ta có $P'(t)=\dfrac{0{,}75a{e^{-0{,}75t}}}{\left(b+{e^{-0{,}75t}}\right)^2}$, $t\ge 0$.\\
    Theo giả thiết: $\heva{& P(0)=20 \\& P'(0)=12 }\Leftrightarrow \heva{& \dfrac{a}{b+1}=20 \\& \dfrac{0{,}75a}{(b+1)^2}=12 } \Leftrightarrow \heva{& a=20(b+1) \\& \dfrac{0{,}75\times 20(b+1)}{(b+1)^2}=12 } \Leftrightarrow \heva{& a=25 \\& b=\dfrac{1}{4}.}$\\
    Do vậy $P(t)=\dfrac{25}{\dfrac{1}{4}+{e^{-0.75t}}}\Rightarrow P(8)\approx 99$ (tế bào).
}
\end{ex}
\begin{ex}%Câu 20
\immini
{
    Trong không gian với hệ trục $Oxyz$ thích hợp, mặt đất là mặt phẳng Oxy, đơn vị trên mỗi trục là mét, một con chim bói cá bay xuống đớp con mồi rồi bay lên khỏi mặt nước với một đường cong parabol đẹp mắt trước khi đậu vào một nhành cây. Biết rằng đường parabol này đi qua các điểm $ A\left(1;2;2\right)$, $B\left(1;3;8\right)$, $C\left(1;-1;10\right)$, vị trí chim bói cá đậu sau khi bắt được mồi có tung độ bằng 4,1 mét. Tính chiều cao của chim đang đậu so với mặt đất theo đơn vị mét và làm tròn đến hàng phần chục.
\shortans{19,6}
}
{
    \includegraphics[width=5cm]{img/HXN-12-20}
}
\loigiai{
    Ta nhận thấy cả ba điểm $A(1;2;2)$, $B(1;3;8)$, $C(1;-1;10)$ cùng thuộc mặt phẳng $x=1$.\\
    Do vậy đường parabol mà con chim bói cá này vẽ nên thuộc mặt phẳng $x=1$.\\
    Đặt $z=ay^2+by+c$, $\left(a\ne 0\right)$ là hàm số bậc hai mô phỏng parabol $(P)$ thuộc mặt phẳng $x=1$.\\
    Thay các cặp $(y;z)$ gồm $(2;2)$, $(3;8)$, $C(-1;10)$ vào hàm số trên ta được hệ phương trình \\
    \centerline{
        $\heva{& 4a+2b+c=2 \\& 9a+3b+c=8 \\& a-b+c=10 } \Leftrightarrow \heva{& a=\dfrac{13}{6} \\& b=-\dfrac{29}{6} \\& c=3 } $ hay $z=\dfrac{13}{6}y^2-\dfrac{29}{6}y+3$.
    }
    Vị trí chim bói cá đậu có $y=4{,}1\Rightarrow z=\dfrac{3\,921}{200}\approx 19{,}6\,m$.\\
    Vậy vị trí chim bói cá đậu cách mặt đất khoảng $19{,}6$ m.
}
\end{ex}
\begin{ex}%Câu 21
\immini
{
    Một cái bục bằng gỗ dùng để đặt đồ trang trí có mặt đáy trên và mặt đáy dưới đều là hình vuông, người thợ thiết kế cái bục này theo ba phần:\\
Phần trên cùng là một hình hộp chữ nhật có các kích thước là $1$ m; $1$ m; $0{,}05$ m.\\
Phần đế của bục cũng là hình hộp chữ nhật có các kích thước là $\sqrt{2}m$; $\sqrt{2}m$; $0{,}2m$.
}
{
    \includegraphics[width=5cm]{img/HXN-12-21}
}
Phần thân (giữ của bục có mặt cắt theo hai đường chéo của đáy trên và đáy dưới là đường hypebol mà các đường tiệm cận của hypebol này tạo với trục đứng một góc bằng $30^{\circ}$.\\
Biết rằng mặt cắt của bục song song với hai đáy tại vị trí có kích thước hình vuông bé nhất bằng $0{,}5$ m. Tìm thể tích của cái bục đã cho theo đơn vị mét khối và làm tròn đến hàng phần trăm.
\shortans{2,33}
\loigiai{
\immini
{
Đặt hệ trục tọa độ như hình vẽ, gọi phương trình chính tắc hypebol $(H)$ là $\dfrac{x^2}{a^2}-\dfrac{y^2}{b^2}=1$, $\left(a>0,\,b>0\right)$.\\
Đường chéo hình vuông nhỏ nhất là $2a=0{,}5\sqrt{2}\Rightarrow a=\dfrac{\sqrt{2}}{4}$.\\
Ta có $\tan 30^{\circ }=\dfrac{a}{b}=\dfrac{\dfrac{\sqrt{2}}{4}}{b}\Rightarrow b=\dfrac{\sqrt{6}}{4}$.\\
Do vậy $(H)\colon \dfrac{x^2}{1/8}-\dfrac{y^2}{3/8}=1$ hay $8x^2=1+\dfrac{8y^2}{3}\Rightarrow x=\pm \sqrt{\dfrac{1}{8}+\dfrac{y^2}{3}}$.\\
Xét mặt cắt vuông góc với $Oy$ của cái bục tại vị trí có hoành độ $x>0$ thì ta thu được hình vuông có cạnh $\dfrac{2x}{\sqrt{2}}=x\sqrt{2}=\sqrt{\dfrac{1}{4}+\dfrac{2y^2}{3}}$.
}
{
    \includegraphics[width=4.6cm]{img/HXN-12-21-LG}
}
Với $x=\dfrac{1\cdot \sqrt{2}}{2}$ (xét hình vuông mặt trên cùng của cái bục); ta có $y=\dfrac{3\sqrt{2}}{4}>0$.\\
Với $x=\dfrac{\sqrt{2}\cdot \sqrt{2}}{2}=1$ (xét hình vuông mặt đáy dưới của cái bục); ta có $y=-\dfrac{\sqrt{42}}{4}<0$.\\
Thể tích cái bục là $V=1\cdot 1\cdot 0{,}05+\sqrt{2}\cdot \sqrt{2}\cdot 0{,}2+\int\limits_{\tfrac{-\sqrt{42}}{4}}^{\tfrac{3\sqrt{2}}{4}}{\left(\dfrac{1}{4}+\dfrac{2y^2}{3}\right)dy}\approx 2{,}33\,m^3$.
}
\end{ex}
\begin{ex}%Câu 22
\immini
{
Lớp 12A có tất cả $30$ học sinh, trong đó bạn An muốn làm lớp trưởng, bạn Bảo muốn làm lớp phó học tập và bạn Châu muốn làm bí thư (các học sinh của lớp không trùng tên). Tính xác suất để cô giáo chọn được học sinh của lớp cho ba chức vụ như trên, mỗi học sinh giữ một chức vụ, đồng thời không có ai trong các bạn An, Bảo, Châu được giữ chức vụ mình thích (làm tròn đến hàng phần chục)?
\shortans{0,9}
}
{
    \includegraphics[width=5cm]{img/HXN-12-22}
}
\loigiai{
    Gọi $A$ là tập hợp các cách chọn chức vụ sao cho An được làm lớp trưởng.\\
    Gọi $B$ là tập hợp các cách chọn chức vụ sao cho Bảo được làm lớp phó học tập.\\
    Gọi $C$ là tập hợp các cách chọn chức vụ sao cho Châu được làm bí thư.\\
    Số cách chọn chức vụ tùy ý: $n\left(\Omega \right)=30\cdot 29\cdot 28=24\,360$.\\
    Ta có $\mid A\cup B\cup C\mid =\mid A\mid +\mid B\mid +\mid C\mid -\mid A\cap B\mid -\mid B\cap C\mid -\mid C\cap A\mid +\mid A\cap B\cap C\mid $.\\
    Trong đó $\mid A\mid =\mid B\mid =\mid C\mid =1\cdot 29\cdot 28$ (chọn một học sinh tương ứng cho chức vụ mình thích, hai chức còn lại thì chọn tùy ý).\\
    Tương tự $\mid A\cap B\mid =\mid B\cap C\mid =\mid C\cap A\mid =1\cdot 1\cdot 28$ (chọn hai học sinh giữ chức vụ minh thích, chức còn lại thì chọn tùy ý).\\ 
    Ta có $\mid A\cap B\cap C\mid =1\cdot 1\cdot 1$ (cả ba học sinh giữ chức vụ mình thích).\\
    Vậy tổng số cách chọn thỏa mãn đề bài là: $n(X)=30\cdot 29\cdot 28-\left(3\cdot 29\cdot 28-3\cdot 28+1\right)=22\,007$.\\
    Xác suất cần tính là $P(X)=\dfrac{n(X)}{n\left(\Omega \right)}=\dfrac{22\,007}{24\,360}\approx 0{,}9$.
}
\end{ex}
\Closesolutionfile{ans}
\inputansbox{6,4,3}{ans/ans-HXN-\sode-T,ans/ans-HXN-\sode-TF,ans/ans-HXN-\sode-SA}
% %%%%%%%%%%%%%%%%%%%- HXN
\def\sode{13}
\def\tendethi{ĐỀ PHÁT TRIỂN MINH HOẠ 2025}
\begin{dethi}
 {\tendethi}
\end{dethi}
\caulc
\Opensolutionfile{ans}[ans/ans-HXN-\sode-T]
\begin{ex}%Câu 1
 Cho hàm số $ f(x)$ có bảng xét dấu của đạo hàm như sau\\
 \centerline{
 \begin{tikzpicture}[>=stealth]
 \tkzTabInit[nocadre=false,lgt=1.2,espcl=2.5,deltacl=.5]
 {$x$/.7, $f'(x)$/1}
 {$-\infty$,$-1$,$0$,$3$,$+\infty$}
 \tkzTabLine{ , + , $0$ , - ,$0$,+,$0$,-, }
 \end{tikzpicture}
 }
 Hàm số đã cho nghịch biến trên khoảng nào dưới đây?
 \choice
 {$\left(-1;3\right)$}
 {$\left(-\infty;-1\right)$}
 {\True $\left(-1;0\right)$}
 {$\left(0;+\infty\right)$}
\end{ex}
\begin{ex}%Câu 2
 Chỉ số ô nhiễm không khí (AQI) tại thủ đô Hà Nội trong tháng 6/2024 được thống kê vào lúc 10h30 sáng các ngày trong tháng thể hiện theo bảng số liệu sau:\\
 \centerline{\begin{tblr}{|c|c|c|c|c|c|}
 \hline
 Chỉ số (AQI) & $[130;145)$ & $[145;160)$ & $[160;175)$ & $[175;190)$ & $[190;205)$\\
 \hline
 Số ngày & 8 & 7 & 6 & 7 & 2\\
 \hline
 \end{tblr}}\\
 Tứ phân vị thứ ba của mẫu số liệu trên gần nhất với giá trị nào sau đây?
 \choice
 {$ 175$}
 {$ 176{,}5$}
 {$ 180{,}2$}
 {\True $ 178{,}2$}
\end{ex}
\begin{ex}%Câu 3
 Trong không gian $ Oxyz$, đường thẳng đi qua điểm $ M\left(1;3;-2\right)$ và nhận vectơ $\vec{u}=\left(1;-1;5\right)$ làm vectơ chỉ phương có phương trình tham số là
 \choice
 {$\heva{& x=1+t\\& y=-1+3t\\& z=5-2t}$}
 {$\heva{& x=1+t\\& y=-1+3t\\& z=5+2t}$}
 {\True $\heva{& x=1+t\\& y=3-t\\& z=-2+5t}$}
 {$\heva{& x=1+t\\& y=3+t\\& z=-2+5t}$}
\end{ex}
\begin{ex}%Câu 4
 Khẳng định nào dưới đây là đúng?
\choice
{$\int{x^3\mathrm{\,d}x=x^4}+C$}
{$\int{x^3\mathrm{\,d}x=3x^2}+C$}
{$\int{x^3\mathrm{\,d}x=\dfrac{x^3}{\ln3}}+C$}
{\True $\int{x^3\mathrm{\,d}x=\dfrac{x^4}{4}}+C$}
\end{ex}
\begin{ex}%Câu 5
 Quãng đường đi bộ mỗi ngày (đơn vị: $km$) của Tom trong $20$ ngày gần nhất được thống kê lại ở bảng sau:\\
 \centerline{\begin{tblr}{|c|c|c|c|c|c|}
 \hline
 Quãng đường (km) & $[2,7 ; 3,0)$ & $[3,0 ; 3,3)$ & $[3,3 ; 3,6)$ & $[3,6 ; 3,9)$ & $[3,9 ; 4,2)$\\
 \hline
 Số ngày & 3 & 6 & 5 & 4 & 2\\
 \hline
 \end{tblr}}\\
 Khoảng biến thiên của mẫu số liệu ghép nhóm là
 \choice
 {\True $ 1{,}5$km}
 {$ 0{,}9$km}
 {$ 0{,}6$km}
 {$ 0{,}3$km}
\end{ex}
\begin{ex}%Câu 6
 Tập nghiệm của bất phương trình $\log_{\frac{2}{3}}x>2$ là
 \choice
 {\True $\left(0;\dfrac{4}{9}\right)$}
 {$\left(-\infty;\sqrt[3]{4}\right)$}
 {$\left(\sqrt[3]{4};+\infty\right)$}
 {$\left(-\infty;\dfrac{4}{9}\right)$}
\end{ex}
\begin{ex}%Câu 7
 Trong không gian $ Oxyz$, hình chiếu vuông góc của điểm $ M\left(3;1;-1\right)$ trên trục $ Oy$ có tọa độ là 
 \choice
 {$\left(3;0;-1\right)$}
 {\True $\left(0;1;0\right)$}
 {$\left(3;0;0\right)$}
 {$\left(0;0;-1\right)$}
\end{ex}
\begin{ex}%Câu 8
 Tiệm cận ngang của đồ thị hàm số $ y=\dfrac{2}{x-1}$ là đường thẳng:
 \choice
 {$y=2$}
 {$x=1$}
 {$y=1$}
 {\True $y=0$}
\end{ex}
\begin{ex}%Câu 9
 Cho cấp số cộng $\left(u_n\right)$ có $u_2=2$, $u_5=11$. Công sai $d$ của cấp số cộng là
 \choice
 {1}
 {2}
 {\True 3}
 {4}
\end{ex}
\begin{ex}%Câu 10
\immini
{
 Cho hàm số $ f(x)$ liên tục trên $\left[-1;5\right]$ và có đồ thị trên đoạn $\left[-1;5\right]$ như hình vẽ bên dưới. Tổng giá trị lớn nhất và giá trị nhỏ nhất của hàm số $ f(x)$ trên đoạn $\left[-1;5\right]$ bằng
 \choice
 {$-1$}
 {$ 4$}
 {\True $ 1$}
 {$ 2$}
}
{
 \begin{tikzpicture}[line cap=round, line join=round, scale=0.5,font=\scriptsize,>=stealth,thick]
 \draw[->](-2,0)--(6,0)node[below]{$x$};
 \draw[->](0,-3.5)--(0,3)node[left]{$y$};
 \draw
 (-1,-3)..controls++(80:1) and++(180:0.5)..(0,1)..controls++(0:0.5) and++(180:0.75)..(2,-3)..controls++(0:0.5) and++(-100:1)..(3,0)..controls++(80:0.35) and++(180:0.5)..(4,2)..controls++(0:0.5) and++(125:0.25)..(5,1);
 \foreach \x/\y/\m/\g in{-1/0/-1/90,0/-3/-3/-145,2/0/2/90,3/0/3/-70,4/0/4/-90,5/0/5/-90,0/2/2/180}
 \draw[fill=black](\x,\y)circle(1pt)node[shift={(\g:0.25)}]{$\m$};
 \draw[dashed](-1,0)--(-1,-3)--(0,-3)--(2,-3)--(2,0)(0,1)--(5,1)--(5,0)(3,0)--(3,1)(4,0)--(4,2)--(0,2);
 \end{tikzpicture}
}
\end{ex}
\begin{ex}%Câu 11
 Cho hình hộp $ABCD.A'B'C'D'$. Gọi $O$ là tâm của hình hộp, khẳng định nào dưới đây đúng?
 \choice
 {$\overrightarrow{OA}+\overrightarrow{O{A}'}=\overrightarrow{0}$}
 {\True $\overrightarrow{OA}+\overrightarrow{O{C}'}=\overrightarrow{0}$}
 {$\overrightarrow{OA}+\overrightarrow{OB}=\overrightarrow{0}$}
 {$\overrightarrow{OA}+\overrightarrow{OD}=\overrightarrow{0}$}
\end{ex}
\begin{ex}%Câu 12
 Với $ a,b$ là các tham số thực thì giá trị tích phân $ I=\int\limits_0^b{\left(3x^2-2ax-1\right)\text{d}x}$ bằng
 \choice
 {$b^3-b{a^2}-b$}
 {$b^3+b^2a+b$}
 {\True $b^3-b^2a-b$}
 {$ 3b^2-2ab-1$}
 \end{ex}
\Closesolutionfile{ans}
\cauds
\Opensolutionfile{ans}[ans/ans-HXN-\sode-TF]
\begin{ex}%Câu 13
Nếu đứng trước biển và nhìn ra xa, người ta sẽ thấy một đường giao giữa mặt biển và bầu trời, đó là đường chân trời đối với người quan sát. Ta có thể hình dung rằng nếu người quan sát ở tại đỉnh của một chiếc nón và trái đất được \lq\lq thả\rq\rq vào trong chiếc nón ấy thì đường chân trời là đường \lq\lq chạm\rq\rq giữa trái đất và chiếc nón.\\
\centerline{
 \includegraphics[width=7cm]{img/HXN-13-13}
}
Trong không gian $Oxyz$, giả sử bề mặt trái đất $(S)$ có phương trình $x^2+y^2+z^2=1$ và người quan sát ở vị trí $ B\left(1;1;-1\right)$; $A$ là một vị trí bất kì trên đường chân trời đối với người quan sát ở vị trí $B$.
 \choiceTF
 {\True Khoảng cách từ vị trí $B$ đến tâm của trái đất là $\sqrt{3}$}
 {\True Khoảng cách hai điểm $A$, $B$ là $\sqrt{2}$}
 {Phương trình mặt cầu đường kính $OB$ là $\left(x-\dfrac{1}{2}\right)^2+\left(y-\dfrac{1}{2}\right)^2+\left(z-\dfrac{1}{2}\right)^2=\dfrac{3}{4}$}
 {\True Điểm $A$ luôn thuộc mặt phẳng cố định $ x+y-z-1=0$}
 \loigiai{
     \begin{itemchoice}
         \itemch Tâm của trái đất là điểm $O(0;0;0)$, bán kính $R=1$.\\
         Ta có $OB=\sqrt{(1-0)^2+(1-0)^2+\left(-1-0\right)^2}=\sqrt{3}$.
         \itemch Vì $\triangle OAB$ vuông tại $A$ nên $AB=\sqrt{OB^2-R^2}=\sqrt{3-1}=\sqrt{2}$.
         \itemch Gọi $I$ là trung điểm OB thì $I\left(\dfrac{1}{2};\dfrac{1}{2};-\dfrac{1}{2}\right)$; mặt cầu đường kính OB có tâm $I$,\\
         bán kính $\dfrac{OB}{2}=\dfrac{\sqrt{3}}{2}$ nên có phương trình $\left(S'\right)\colon \left(x-\dfrac{1}{2}\right)^2+\left(y-\dfrac{1}{2}\right)^2+\left(z+\dfrac{1}{2}\right)^2=\dfrac{3}{4}$.
         \itemch Ta thấy $A$ luôn thuộc đường tròn giao tuyến của hai mặt cầu $(S)\,,\left({S'}\right)$.
         Xét hệ phương trình $\heva{& x^2+y^2+z^2=1 \\& \left(x-\dfrac{1}{2}\right)^2+\left(y-\dfrac{1}{2}\right)^2+\left(z+\dfrac{1}{2}\right)^2=\dfrac{3}{4} }  \Leftrightarrow \heva{& x^2+y^2+z^2=1 \\& x^2+y^2+z^2-x-y+z=0 }  \Leftrightarrow \heva{& x^2+y^2+z^2=1 \\& 1-x-y+z=0 } \Leftrightarrow \heva{& x^2+y^2+z^2=1 \\& x+y-z-1=0 }$.\\
         Vậy điểm $A$ luôn thuộc mặt phẳng cố định $x+y-z-1=0$.
     \end{itemchoice}
 }
\end{ex}
\begin{ex}%Câu 14
Một màn ảnh hình chữ nhật cao $1{,}4$ m được đặt ở độ cao $1{,}8$ m so với tầm mắt (tính từ đầu mép dưới của màn hình). Một người đang xem phim có mắt đặt ở vị trí O và quan sát màn ảnh với góc nhìn $\widehat{BOC}$. Với các điểm như trong hình vẽ, đặt $x=OA$, $ x>0$.
\choiceTF
 {\True $\tan\widehat{BOC}=\dfrac{\tan\widehat{AOC}-\tan\widehat{AOB}}{1+\tan\widehat{AOC}\cdot\tan\widehat{AOB}}$}
 {\True $\tan\widehat{AOC}=\dfrac{3{,}2}{x}$, $\tan\widehat{AOB}=\dfrac{1{,}8}{x}$}
 {Nếu góc nhìn màn hình của mắt người là $15^{\circ}$ thì người đó đang ngồi cách bức tường (nơi gắn màn hình) một khoảng gần nhất bằng $3{,}6$ mét (làm tròn đến hàng phần chụ}
 {Người xem muốn nhìn rõ màn ảnh nhất (góc nhìn lớn nhất) thì người đó phải đứng cách mặt phẳng chứa màn ảnh một khoảng $2{,}2$ mét}
 \loigiai{
     \begin{itemchoice}
         \itemch Ta có $\widehat{BOC}=\widehat{AOC}-\widehat{AOB}\Rightarrow \tan\widehat{BOC}=\tan\left(\widehat{AOC}-\widehat{AOB}\right)=\dfrac{\tan\widehat{AOC}-\tan\widehat{AOB}}{1+\tan\widehat{AOC}\cdot \tan\widehat{AOB}}$.
         \itemch Xét lần lượt các tam giác vuông $OAB$, $OAC$ (vuông tại $O$), ta có
         $\tan\widehat{AOC}=\dfrac{AC}{AO}=\dfrac{3{,}2}{x}$, $\tan\widehat{AOB}=\dfrac{AB}{AO}=\dfrac{1{,}8}{x}$.
         \itemch Ta có: $\tan\widehat{BOC}=\dfrac{\dfrac{3{,}2}{x}-\dfrac{1{,}8}{x}}{1+\dfrac{3{,}2}{x}\cdot \dfrac{1{,}8}{x}}=\dfrac{\dfrac{1{,}4}{x}}{\dfrac{x^2+5{,}76}{x^2}}=\dfrac{1{,}4x}{x^2+5{,}76}$.\\
         Khi $\widehat{BOC}=15^{\circ }$ thì $$\tan 15^{\circ}=\dfrac{1{,}4x}{x^2+5{,}76}\Rightarrow \tan 15^{\circ}\cdot x^2-1{,}4x+5{,}76\cdot \tan 15^{\circ}=0\Rightarrow \hoac{& x\approx 3{,}6\,m \\& x\approx 1{,}6\,m.} $$
         Ta chọn $x\approx 1{,}6<3{,}6$.\\
         Vì vậy người xem ngồi cách tường một khoảng gần nhất xấp xỉ $1{,}6$ mét.
         \itemch Ta cần tìm giá trị lớn nhất của hàm $f(x)=\dfrac{1{,}4x}{x^2+5{,}76}$, $x>0$.\\
         Đạo hàm $f'(x)=\dfrac{-1{,}4x^2+8{,}064}{\left(x^2+5{,}76\right)^2}$; $f'(x)=0\Leftrightarrow x^2=5{,}76\Rightarrow x=2{,}4$.\\
         Bảng biến thiên:\\
         \begin{tikzpicture}[>=stealth]
             \tkzTabInit[nocadre=false,lgt=1.2,espcl=2.5,deltacl=0.5]{$x$/.7 ,$f'(x)$/.7,$f(x)$/2}
             {$-\infty$ , $2{,}4$ , $+\infty$}
             \tkzTabLine{ , + , $0$ , - , }
             \tkzTabVar{-/, +/$\dfrac{7}{24}$ , -/}
         \end{tikzpicture}
         
         Vậy để góc nhìn lớn nhất $\left(\widehat{BOC}_{\max}\right)$ thì $\tan\widehat{BOC}$ đạt giá trị lớn nhất
         $\Leftrightarrow f(x)$ đạt giá trị lớn nhất.\\
         Dựa vào bảng biến thiên, ta thấy $\max\limits_{\left(0;+\infty \right)} f(x)=\dfrac{7}{24}$, khi đó $x=2{,}4$ (mét).
     \end{itemchoice}
 }
\end{ex}
\begin{ex}%Câu 15
Một chậu nước có dạng một khối tròn xoay với thiết diện qua trục của chậu (mặt cắt đi qua hai tâm của hai đường tròn đáy) là hai đường parabol đối xứng nhau qua trục đó. Biết hai đường tròn đáy chậu cùng có bán kính $0{,}5$ m; thiết diện nhỏ nhất vuông góc với trục của chậu có bán kính $0{,}2$ m; chiều cao của chậu nước bằng $1{,}5$ m. Người ta bơm nước vào chậu với tốc độ $5$ lít/phút.\\
Xét hệ trục tọa độ $Oxy$ với gốc $O$ trùng với tâm đường tròn đáy của chậu nước, tia $Ox$ chứa trục của chậu nước (đơn vị trên mỗi trục là mét). Mặt cắt qua trục của chậu nước cho ta hai nhánh parabol như hình vẽ, gọi $ y=f(x)$ là parabol nằm trên trục hoành.
\begin{center}
    \includegraphics[width=5cm]{img/HXN-13-15-a}\includegraphics[height=5cm]{img/HXN-13-15-b}
\end{center}
\choiceTF
 {\True $ f(x)=\dfrac{8}{15}{x^2}-\dfrac{4}{5}x+\dfrac{1}{2}$}
 {\True Sức chứa tối đa của chậu nước bằng $0{,}5$ m$^3$ (làm tròn đến hàng phần chục của mét khối)}
 {\True Sau $1{,}5$ giờ bơm nước (làm tròn đến hàng phần chục của giờ) thì chậu đầy nước}
 {\True Nếu bơm từ đầu như thế thì đến phút thứ $20$, tốc độ dâng lên của nước bằng $0{,}01$ m/phút}
 \loigiai{
     \begin{center}
         \includegraphics[height=5cm]{img/HXN-13-15-LG}
     \end{center}
     \begin{itemchoice}
         \itemch  Parabol $f(x)=ax^2+bx+c$, $\left(a\ne 0\right)$ đi qua các điểm $\left(0;0{,}5\right)$, $\left(1{,}5;0{,}5\right)$, $\left(0{,}75;0{,}2\right)$ nên ta có hệ phương trình $\heva{& c=0{,}5 \\& 1{,}5^2a+1{,}5b+0{,}5=0{,}5 \\& 0{,}75^2a+0{,}75b+0{,}5=0{,}2 } \Rightarrow \heva{& a=\dfrac{8}{15} \\& b=-\dfrac{4}{5} \\& c=\dfrac{1}{2}.} $\\
         Do vậy $f(x)=\dfrac{8}{15}x^2-\dfrac{4}{5}x+\dfrac{1}{2}$.
         \itemch Thiết diện của chậu nước vuông góc với trục $Ox$ tại vị trí có hoành độ $x$ chính là đường tròn có bán kính $r=\dfrac{8}{15}x^2-\dfrac{4}{5}x+\dfrac{1}{2}$ nên có diện tích $S(x)=\pi r^2=\pi \left(\dfrac{8}{15}x^2-\dfrac{4}{5}x+\dfrac{1}{2}\right)^2$.\\
         Thể tích chậu nước là $V_{full}=\int\limits_0^{1{,}5}{S(x)\mathrm{\,d}x}=\int\limits_0^{1{,}5}\pi {\left(\dfrac{8}{15}x^2-\dfrac{4}{5}x+\dfrac{1}{2}\right)^2\mathrm{\,d}x}\approx 0{,}5\,m^3$ (lưu vào A).
         \itemch Đổi đơn vị $5$ lít/phút = $\dfrac{5\colon 1000}{1\colon 60}=\dfrac{3}{10}$ $m^3$/giờ.\\
         Thời gian bơm nước đầy chậu là $\dfrac{V_{full}}{3/10}=\dfrac{49}{100}\pi \approx 1{,}5$ giờ.   [Nhấn máy: A chia $3/10$].
         \itemch Từ câu b) ta thấy hoành độ $x$ $\left(x>0\right)$ cũng chính là chiều cao mực nước trong chậu.\\
         Thể tích chậu nước ứng với chiều cao $x$ là $V=\int\limits_0^x{S(x)\mathrm{\,d}x}$ \tagEX{1}
         Sau $20$ phút bơm nước thì thể tích nước trong chậu là $V_{20}=5\times 20=100$ lít = $0{,}1$ $m^3$/phút.\\
         Xét $V=\int\limits_0^x{S(x)\mathrm{\,d}x}=0{,}1\Rightarrow \int\limits_0^x{\pi \left(\dfrac{8}{15}x^2-\dfrac{4}{5}x+\dfrac{1}{2}\right)^2\mathrm{\,d}x}=0{,}1\Rightarrow x\approx 0{,}164\,m$ (lưu vào B).\\
         Đạo hàm hai vế của $(1)$ theo $t$, ta được\\
         \centerline{$\dfrac{dV}{\mathrm{\,d}t}=S(x)\cdot \dfrac{\mathrm{\,d}x}{\mathrm{\,d}t}\Leftrightarrow \dfrac{dV}{\mathrm{\,d}t}=\pi \left(\dfrac{8}{15}x^2-\dfrac{4}{5}x+\dfrac{1}{2}\right)^2\cdot \dfrac{\mathrm{\,d}x}{\mathrm{\,d}t}$}
         Thay $\dfrac{dV}{\mathrm{\,d}t}=\dfrac{1}{200}\,\,m^3$/phút; $x=B\approx 0{,}164\,m$, ta được $\dfrac{\mathrm{\,d}x}{\mathrm{\,d}t}\approx 0{,}01$ m/phút.
     \end{itemchoice}
 }
\end{ex}
\begin{ex}%Câu 16
\immini
{
    Trong một buổi triển lãm công nghệ quốc tế, công ty A dự định ra mắt sản phẩm mới với sự tham gia của CEO nổi tiếng. Tuy nhiên, có nhiều yếu tố rủi ro có thể khiến sự kiện bị hủy
\begin{itemize}
 \item Nếu lượng khách đăng ký trước vượt 1000 người, sự kiện chắc chắn diễn ra.
 \item Nếu hệ thống máy chủ gặp sự cố (do tấn công mạng hoặc quá tải), công ty không thể xử lý hết số lượng khách đăng ký, khi đó xác suất hủy sự kiện là $60\%$.
 \item Nếu hệ thống hoạt động bình thường, xác suất đủ 1000 đăng ký là $80\%$; nếu không đủ, xác suất hủy sự kiện là $10\%$.
\end{itemize}
}
{
\includegraphics[width=6cm]{img/HXN-13-16}
}
Biết rằng xác suất để sự kiện bị hủy với bất kì lý do gì là $0{,}35$.\\
Gọi các biến cố $A$: \lq\lq Hệ thống gặp sự cố\rq\rq; $B$: \lq\lq Đủ $1000$ đăng ký\rq\rq; $C$: \lq\lq Sự kiện bị hủy\rq\rq.
\choiceTF
 {\True $ P\left(B|A\right)=0$ và $ P\left(\bar{B}|A\right)=1$}
 {\True $ P\left(C|\bar{A}\bar{B}\right)=0{,}1$}
 {Xác suất để hệ thống máy chủ gặp sự cố bằng $0{,}57$ (làm tròn đến hàng phần trăm)}
 {\True Cuối cũng thì những lo lắng cũng đã thành hiện thực, sự kiện quan trọng đã bị hủy, xác suất để hệ thống máy chủ hoạt động bình thường băng $0{,}02$ (làm tròn đến hàng phần trăm)}
 \loigiai{
     \begin{center}
         \includegraphics[height=6cm]{img/HXN-13-16-LG}
     \end{center}
     \begin{itemchoice}
         \itemch Nếu hệ thống gặp sự cố thì số lượng khách hàng đăng ký trước không thể đủ 1000, do đó  $P\left(B|A\right)=0$ và $P\left(\bar{B}|A\right)=1$.
         \itemch Nếu hệ thống máy chủ không gặp sự cố và số lượng đăng ký trước của khách hàng dưới 1000 thì xác suất sự kiện bị hủy bằng $10\%$, do đó $P\left(C|\bar{A}\bar{B}\right)=0{,}1$.
         \itemch Đặt $x=P(A)\in (0;1)$; suy ra $P\left({\bar{A}}\right)=1-x$.\\
         Ta có $P(C)=x\cdot 1\cdot 0{,}6+(1-x)\cdot 0{,}2\cdot 0{,}1=0{,}35\Rightarrow x=\dfrac{33}{58}$ hay $P(A)=\dfrac{33}{58}\approx 0{,}57$.
         \itemch Ta có $P\left(\bar{A}|C\right)=\dfrac{P\left(\bar{A}C\right)}{P(C)}=\dfrac{\left(1-\dfrac{33}{58}\right)\cdot 0{,}2\cdot 0{,}1}{0{,}35}=\dfrac{5}{203}\approx 0{,}02$.
     \end{itemchoice}
 }
\end{ex}
\Closesolutionfile{ans}
\caukq
\Opensolutionfile{ans}[ans/ans-HXN-\sode-SA]
\begin{ex}%Câu 17
\immini
{
     Sau cơn mưa, có 4 cậu bé muốn đi qua một con đê trơn trợt nhưng họ chỉ có hai đôi dép.
 \begin{itemize}
 \item Cậu bé Văn có thể đi qua con đê trong 5 phút.
 \item Cậu bé Võ có thể đi qua con đê trong 9 phút.
 \item Cậu bé Song có thể đi qua con đê trong 13 phút.
 \item Cậu bé Toàn có thể đi qua con đê trong 3 phút.
 \end{itemize}
}
{
    \includegraphics[width=6cm]{img/HXN-13-17}
}
Hỏi thời gian tối thiểu để cả $4$ cậu bé cùng qua được con đê là bao nhiêu phút? Biết rằng mỗi cậu bé muốn đi qua con đê này thì phải mang dép và thời gian để mỗi người đi qua hoặc đi về lại trên con đê là như nhau.
\shortans{31}
\loigiai{
Để tối ưu hóa thời gian qua lại trên con đê thì $2$ cậu bé nhanh nhất (Văn và Toàn) nên đi với nhau, $2$ cậu bé chậm hơn (Võ và Song) nên đi với nhau.\\
Họ phải đi như sau để qua được con đê một cách nhanh nhất:
\begin{itemize}
    \item Văn và Toàn đi cùng nhau $\longrightarrow$ mất 5 phút.
    \item Toàn cầm đôi dép của Văn quay về lại đón bạn $\longrightarrow$ mất 3 phút.
    \item Võ và Song đi cùng nhau $\longrightarrow$ mất 13 phút.
    \item Văn cầm 1 đôi dép quay lại đón bạn $\longrightarrow$ mất 5 phút.
    \item Văn và Toàn đi cùng nhau lần nữa $\longrightarrow$ mất 5 phút.
\end{itemize}
Vậy tổng thời gian ngắn nhất để cả 4 cậu bé qua được con đê là là 31 phút.
}
 \end{ex}
 \begin{ex}%Câu 18
\immini
{
    Một cửa hàng bán lẻ bán được $ 2500$ cái tivi mỗi năm. Để bán được số tivi đó, họ phải đặt hàng từ nhà máy sản xuất tivi nhiều lần trong năm, mỗi lần đặt hàng với số lượng tivi như nhau. Mỗi lần lấy hàng từ nhà máy về thì cửa hàng chỉ trưng bày được một nửa số tivi đó, một nửa số hàng còn lại phải lưu vào kho; chi phí gửi trong kho là $ 10\$$ cho một cái tivi. Chi phí cố định cho mỗi lần đặt hàng là $ 20\$$, ngoài ra cửa hàng phải trả thêm $ 9\$$ cho mỗi tivi. Hỏi mỗi lần đặt hàng trong năm thì cửa hàng cần đặt bao nhiêu tivi để chi phí mà cửa hàng phải trả là nhỏ nhất?
}
{
    \includegraphics[width=6cm]{img/HXN-13-18}
}
\shortans{100}
\loigiai{
Gọi $x$ là số tivi mà cửa hàng đặt mỗi lần $\left(x\in \mathbb{N},\, 1\le x\le 2\,500\right)$.\\
Số tivi lưu vào kho mỗi lần là $\dfrac{x}{2}$; do đó chi phí lưu vào kho là $10\cdot \dfrac{x}{2}=5x$.\\
Số lần đặt hàng trong năm là $\dfrac{2\,500}{x}$ và chi phí đặt hàng là  $\dfrac{2\,500}{x}\left( 20+9x \right)$.\\
Tổng số chi phí mà cửa hàng phải trả là: $\dfrac{2\,500}{x}(20+9x)+5x=5x+\dfrac{50\,000}{x}+22\,500$.\\ 
Áp dụng bất đẳng thức Cauchy ta có  $5x+\dfrac{50\,000}{x}\ge 1\,000$.\\ 
Dấu bằng xảy ra khi $5x=\dfrac{50\,000}{x}\Rightarrow  x=100$.\\  
Vậy mỗi năm cửa hàng cần đặt hàng $\dfrac{2\,500}{100}=25$ lần, mỗi lần $100$ cái.
}
\end{ex}
\begin{ex}%Câu 19
Trong không gian với hệ tọa độ $Oxyz$ cho điểm $A\left(3;2;-1\right)$ và đường thẳng $d\colon \heva{& x=t\\& y=t\\& z=1+t.}$ Tính khoảng cách từ gốc tọa độ đến mặt phẳng $(P)$ biết rằng $(P)$ chứa $d$ sao cho khoảng cách từ $A$ đến $(P)$ là lớn nhất. Làm tròn kết quả đến hàng phần chục.
\shortans{0,8}
\loigiai{
\immini
{
Đường thẳng $d$ qua $M_0(0;0;1)$ có VTCP $\vec{u}=(1;1;1)$.\\
Gọi $H$, $K$ lần lượt là hình chiếu của $A$ lên $(P)$ và $d$ (khi đó $AK$ cố định). Ta có:
$d\left(A,(P)\right)=AH\le AK$.\\
Đẳng thức xảy ra khi và chỉ khi $H\equiv K$.\\
Do đó $d\left(A,(P)\right)_{\max }=AK$.\\
Khi đó $(P)$ đi $M_0(0;0;1)$ nhận $\vec{AK}$ làm vectơ pháp tuyến.
}
{
    \includegraphics[width=6cm]{img/HXN-13-19-LG}
}
Gọi $K\left(t;t;1+t\right)\in d \Rightarrow \vec{AK}=\left(t-3;t-2;t+2\right)$.\\
Ta có $\vec{AK}\perp \vec{u}\Leftrightarrow \vec{AK}\cdot \vec{u}=0 \Leftrightarrow 1\cdot (t-3)+1\cdot (t-2)+1\cdot (t+2)=0\Leftrightarrow t=1$.\\
Suy ra $\vec{AK}=(-2;-1;3)$.\\
Vậy phương trình $(P)\colon -2(x-0)-1\cdot (y-0)+3\cdot (z-1)=0 \Leftrightarrow 2x+y-3z+3=0$.\\
Khi đó: $d\left(O,(P)\right)=\dfrac{3}{\sqrt{4+1+9}}=\dfrac{3\sqrt{14}}{14}\approx 0{,}8$.
}
\end{ex}
\begin{ex}%Câu 20
\immini
{
    Hộp I có 5 quả bóng đỏ và 3 quả bóng trắng, hộp II có 2 quả bóng đỏ và 2 quả bóng trắng, hộp III có 3 quả bóng đỏ và 1 quả bóng trắng. Người ta thực hiện các hành động sau một cách ngẫu nhiên:
}
{
    \includegraphics[width=6cm]{img/HXN-13-20}
}
\begin{itemize}
    \item Lấy 1 quả bóng từ hộp I bỏ sang hộp II.
    \item Lấy 1 quả bóng từ hộp III bỏ sang hộp II.
    \item Lấy ra mỗi hộp 1 quả bóng thì nhận thấy đó là 3 quả bóng trắng.
\end{itemize}
Tính xác suất để cả 3 quả bóng được lấy từ ba hộp vốn là của hộp I và hộp III (ban đầu).\\
Kết quả được làm tròn đến hàng phần trăm.
\shortans{0,13}
\loigiai{
    \begin{center}
        \includegraphics[width=10cm]{img/HXN-13-20-LG}
    \end{center}
    Gọi A là biến cố: \lq\lq Lấy được $3$ quả bóng trắng từ ba hộp\rq\rq và B là biến cố: \lq\lq Cả $3$ quả bóng được lấy từ ba hộp vốn là của hộp I và hộp II\rq\rq.\\
    Để biến cố A xảy ra thì quả bóng chuyển từ hộp III qua hộp II phải là bóng đỏ (vì khi đó hộp III còn 1 quả bóng trắng để lấy ra).\\
    Ta tính xác suất của A trong hai trường hợp quả bóng được chuyển từ hộp I sang hộp II là đỏ hoặc trắng.\\
    Ta có $P(A)=\dfrac{3}{8}\cdot \dfrac{2}{7}\cdot \dfrac{3}{4}\cdot \dfrac{1}{3}\cdot \dfrac{1}{2}+\dfrac{5}{8}\cdot \dfrac{3}{7}\cdot \dfrac{3}{4}\cdot \dfrac{1}{3}\cdot \dfrac{2}{6}=\dfrac{1}{28}$.\\
    Do đó: $P\left(B|A\right)=\dfrac{P(BA)}{P(A)}=\dfrac{\dfrac{3}{8}\cdot \dfrac{2}{7}\cdot \dfrac{3}{4}\cdot \dfrac{1}{3}\cdot \dfrac{1}{6}}{\dfrac{1}{28}}=\dfrac{1}{8}\approx 0{,}13$.
}
\end{ex}
\begin{ex}%Câu 21
\immini
{
    Trong phòng thí nghiệm vật lý, một chất điểm đặt ở vị trí $A$ của hình lập phương được tác động bởi ba lực $\overrightarrow{F_1}$, $\overrightarrow{F_2}$, $\overrightarrow{F_3}$ dọc theo hai cạnh và đường chéo lớn của hình lập phương đó (tham khảo hình vẽ). Biết độ lớn các lực trên hai cạnh bằng $2$ N và $3$ N, độ lớn lực dọc theo đường chéo lớn lập phương bằng $4$ N. Tính độ lớn hợp lực $\left|\overrightarrow{F_1}+\overrightarrow{F_2}+\overrightarrow{F_3}\right|$ theo đơn vị N, làm tròn đến hàng phần trăm.
\shortans{7,22}
}
{
    \includegraphics[width=6cm]{img/HXN-13-21}
}
\loigiai{
Xét $\left(\vec{F_1},\vec{F_2}\right)=90^\circ$; $\left(\vec{F_1},\vec{F_3}\right)=\widehat{BAC'}\Rightarrow \tan \left(\vec{F_1},\vec{F_3}\right)=\tan \widehat{BAC'}=\dfrac{BC'}{AB}=\dfrac{a\sqrt{2}}{a}=\sqrt{2}$\\
$ \Rightarrow \cos \widehat{BAC'}=\sqrt{\dfrac{1}{1+\tan ^2\alpha }}=\dfrac{\sqrt{3}}{3}$; $\left(\vec{F_2},\vec{F_3}\right)=\widehat{DAC'}=\widehat{BAC'}\Rightarrow \cos \left(\vec{F_2},\vec{F_3}\right)=\dfrac{\sqrt{3}}{3}$.\\
Ta có $\left| \vec{F_1}+\vec{F_2}+\vec{F_3} \right|^2=\left| \vec{F_1} \right|^2+\left| \vec{F_2} \right|^2+\left| \vec{F_3} \right|^2+2\vec{F_1}\cdot \vec{F_2}+2\vec{F_1}\cdot \vec{F_3}+2\vec{F_2}\cdot \vec{F_3}$; trong đó:
\begin{itemize}
    \item $2\vec{F_1}\cdot \vec{F_2}=0$ vì $\vec{F_1}\perp \vec{F_2}$.
    \item $2\vec{F_1}\cdot \vec{F_3}=2\left| \vec{F_1} \right|\cdot \left| \vec{F_3} \right|\cos \left(\vec{F_1},\vec{F_3}\right)=2.2\cdot 4\cdot \dfrac{\sqrt{3}}{3}=\dfrac{16\sqrt{3}}{3}$.
    \item $2\vec{F_2}\cdot \vec{F_3}=2\left| \vec{F_2} \right|\cdot \left| \vec{F_3} \right|\cos \left(\vec{F_2},\vec{F_3}\right)=2.3\cdot 4\cdot \dfrac{\sqrt{3}}{3}=8\sqrt{3}$.
\end{itemize}
Do vậy $\left| \vec{F_1}+\vec{F_2}+\vec{F_3} \right|^2=4+9+16+\dfrac{16\sqrt{3}}{3}+8\sqrt{3}=\dfrac{87+40\sqrt{3}}{3}$\\
$\Rightarrow \left| \vec{F_1}+\vec{F_2}+\vec{F_3} \right|=\sqrt{\dfrac{87+40\sqrt{3}}{3}}\approx 7{,}22$N.
}
\end{ex}
\begin{ex}%Câu 22
\immini
{
    Một chiếc trống có đường sinh là các nửa elip tương ứng với độ dài trục lớn $80$ cm, còn độ dài trục bé bằng $60$ cm ; hai đáy trống là các hình tròn có bán kính $60$ cm. Hỏi thể tích của chiếc trống là bao nhiêu dm$^3$ (làm tròn đến hàng đơn vị).
\shortans{345}
}
{
    \includegraphics[width=6cm]{img/HXN-13-22}
}
\loigiai{
    \begin{center}
        \includegraphics[width=6cm]{img/HXN-13-22-LG}
    \end{center}
Đặt hệ trục tọa độ $Oxy$ như hình vẽ, đơn vị trên mỗi trục là dm.
Giả sử elip $(E)$ có độ dài trục lớn $2a=8\Rightarrow a=4$; độ dài trục bé $2b=6\Rightarrow b=3$.\\
Do đó phương trìn $(E)\colon \dfrac{x^2}{16}+\dfrac{y^2}{9}=1\Rightarrow y^2=9\left(1-\dfrac{x^2}{16}\right)$.\\
Ta chọn nửa elip $(E)$ nằm dưới trục hoành là $y=-3\sqrt{1-\dfrac{x^2}{16}}$ $\left(y\le 0\right)$.\\
Để có được hàm số $y=f(x)$ (đường sinh của chiếc trống), ta cần tịnh tiến đồ thị hàm số $y=-3\sqrt{1-\dfrac{x^2}{16}}$ lên trên 6 đơn vị; khi đó $f(x)=6-3\sqrt{1-\dfrac{x^2}{16}}$.\\
Thể tích cái trống là $V=\pi \displaystyle\int\limits_{-4}^4{\left(f(x)\right)^2\mathrm{\,d}x}=\pi \displaystyle\int\limits_{-4}^4{\left(6-3\sqrt{1-\dfrac{x^2}{16}}\right)^2\mathrm{\,d}x}\approx  345\,dm^3$.
}
\end{ex}
\Closesolutionfile{ans}
\inputansbox{6,4,3}{ans/ans-HXN-\sode-T,ans/ans-HXN-\sode-TF,ans/ans-HXN-\sode-SA}
% \begin{name}
	{\tenchude}
	{\tendethi}
	{TRƯỜNG THPT CHUYÊN VĨNH PHÚC - VĨNH PHÚC}
	{\thoigian}
\end{name}

\caulc
% \Opensolutionfile{ans}[Ans/KSCL-THPT-ChuyenVinhPhuc-VinhPhuc-L1-NH24-25-TN]
\Opensolutionfile{ans}[Ans/KSCL-THPT-ChuyenVinhPhuc-VinhPhuc-L1-NH24-25]

%   \Opensolutionfile{ansbook}[Ansbook/KSCL-THPT-ChuyenVinhPhuc-VinhPhuc-L1-NH24-25-TN]%---Nên đặt tên theo bài
  \setcounter{ex}{0}
 %%%==============Cau_EX1==============%%%
\begin{ex}%[Dự án C THPTQG 2025]%[Vương Quốc Phong]%[2D1N1-5]
	\immini[thm]{
	Một doanh nghiệp sản xuất với số lượng là $x$ sản phẩm, $x\in \mathbb{N}$ và thu được lợi nhuận $f(x)$ được biểu thị bởi bảng biến thiên như sau. Hỏi doanh nghiệp sản xuất bao nhiêu sản phẩm trở đi thì lợi nhuận bắt đầu giảm?
	\choice
	{\True $201$}
	{$200$}
	{$101$}
	{$100$}
	}{\begin{tikzpicture}
			\tkzTabInit[lgt=1.2,espcl=2.5]
			{$x$/.7,$f'(x)$/.7,$f(x)$/1.8}
			{$0$,$200$,$+\infty$}
			\tkzTabLine{ ,+,z,-, }
			\tkzTabVar{-/, +/$100$,-/}
		\end{tikzpicture}}
	\loigiai{
		Dựa vào bảng biến thiên ta thấy lợi nhuận bắt đầu giảm từ sản phẩm $201$.
	}
\end{ex}
%%%==============HetCau_EX1==============%%%

%%%==============Cau_EX2==============%%%
\begin{ex}%[Dự án C THPTQG 2025]%[Vương Quốc Phong]%[2D1N3-1]
	\immini[thm]{
	Cho hàm số $y=f(x)$ có bảng biến thiên trên đoạn $[0;3]$ như sau.
	Giá trị nhỏ nhất của hàm số $y=f(x)$ trên đoạn $[0;3]$ là
	\choice
	{\True $-4$}
	{$1$}
	{$4$}
	{$0$}
	}{
		\begin{tikzpicture}
			\tkzTabInit[lgt=1.2,espcl=2.5]
			{$x$/.7,$y$/1.8}
			{$0$, $1$, $3$}
			\tkzTabVar{+/$-3$,-/$-4$,+/$0$}
		\end{tikzpicture}
	}
	\loigiai{
		Dựa vào bảng biến thiên ta thấy giá trị nhỏ nhất của hàm số $y=f(x)$ trên đoạn $\left[0;3\right]$ là $-4$.
	}
\end{ex}
%%%==============HetCau_EX2==============%%%

%%%==============Cau_EX3==============%%%
\begin{ex}%[Dự án C THPTQG 2025]%[Vương Quốc Phong]%[1D6H3-2]
	Tập xác định của hàm số $y=\log_{2024} (3-x)$ là
	\choice
	{\True $\mathscr{D} = (-\infty;3)$}
	{$\mathscr{D} = (3;+\infty)$}
	{$\mathscr{D} = (0;+\infty)$}
	{$\mathscr{D} = \mathbb{R}$}
	\loigiai{
		Ta có $3-x > 0 \Leftrightarrow x < 3$. \\
		Vậy $\mathscr{D} = (-\infty;3)$.
	}
\end{ex}
%%%==============HetCau_EX3==============%%%

%%%==============Cau_EX4==============%%%
\begin{ex}%[Dự án C THPTQG 2025]%[Vương Quốc Phong]%[2D1H2-1]
	Cho hàm số $y=f(x)$ xác định trên $\mathbb{R}$ và có đạo hàm $f'(x)=x^{2024} (3-x)$, $\forall x\in \mathbb{R}$. Hàm số đã cho có mấy điểm cực trị?
	\choice
	{$3$}
	{$0$}
	{$2$}
	{\True $1$}
	\loigiai{
	Ta có $f'(x)=x^{2024} (3-x)=0 \Leftrightarrow \hoac{&{x=0} \\ &{x=3}}$, trong đó $x=0$ (nghiệm bội chẵn). \\
	Bảng biến thiên
	\begin{center}
		\begin{tikzpicture}[>=stealth, thick, x=1.2cm, y=1.0cm]
			\def\sohang{5}
			\def\socot{8}
			%ĐN các điểm
			\foreach \x in {1,...,\sohang}
			\foreach \y in {1,...,\socot}
			\path (\y,-\x) coordinate (r{\x}c{\y}) node (r{\x}c{\y}) {};
			%Đường kẻ ngang, dọc
			\draw
			($(r{1}c{1})!.5!(r{2}c{1})+(-.5,0)$)--($(r{1}c{\socot})!.5!(r{2}c{\socot})+(.5,0)$)
			($(r{2}c{1})!.5!(r{3}c{1})+(-.5,0)$)--($(r{2}c{\socot})!.5!(r{3}c{\socot})+(.5,0)$)
			($(r{1}c{1})!.5!(r{1}c{2})+(0,.5)$)--($(r{\sohang}c{1})!.5!(r{\sohang}c{2})+(0,-.5)$)
			;
			%Khung viền
			\draw ($(r{1}c{1})+(-.5,.5)$) rectangle ($(r{\sohang}c{\socot})+(.5,-.5)$);
			%Node các giá trị
			\foreach \diem/\nhan in {
			%Dòng x
			r{1}c{1}/x,r{1}c{2}/\infty,r{1}c{4}/0,r{1}c{6}/3,r{1}c{8}/+\infty,
			%Dòng f'(x)
			r{2}c{1}/f'(x),r{2}c{3}/+,r{2}c{4}/0,r{2}c{5}/+,r{2}c{6}/0,r{2}c{7}/-
			} \path (\diem) node{$\nhan$};
			\path ($(r{3}c{1})!.5!(r{\sohang}c{1})$) node{$f(x)$};
			\foreach \diem/\nhan in {
			%Các dòng của f(x)
			r{3}c{6}/,r{5}c{2}/,r{5}c{8}/
			} \path (\diem) node[shape=rectangle, fill=white, inner sep=2pt] (\diem) {$\nhan$};
			%Vẽ các dấu mũi tên
			\foreach \dau/\cuoi in {r{5}c{2}/r{3}c{6},r{3}c{6}/r{5}c{8}} \draw[->] (\dau)--(\cuoi);
		\end{tikzpicture}
	\end{center}
	Vậy hàm số đã cho có $1$ điểm cực trị.
	}
\end{ex}
%%%==============HetCau_EX4==============%%%

%%%==============Cau_EX5==============%%%
\begin{ex}%[Dự án C THPTQG 2025]%[Vương Quốc Phong]%[1D9H2-5]
	Cho hai biến cố độc lập $A$ và $B$. Biết $P(A)=\dfrac{1}{4}$, $P(A\cup B)=\dfrac{1}{2}$. Tính $P(B)$.
	\choice
	{$\dfrac{3}{4}$}
	{$\dfrac{1}{4}$}
	{$\dfrac{1}{8}$}
	{\True $\dfrac{1}{3}$}
	\loigiai{
		Ta có $P(A\cup B)=P(A)+P(B)-P(AB)=P(A)+P(B)-P(A)\cdot P(B)$ \\
		$\Rightarrow P(B) = \dfrac{P(A \cup B) - P(A)}{1-P(A)} = \dfrac{1}{3}$
	}
\end{ex}
%%%==============HetCau_EX5==============%%%

%%%==============Cau_EX6==============%%%
\begin{ex}%[Dự án C THPTQG 2025]%[Vương Quốc Phong]%[1H8H5-3]
	% \immini{
		Cho hình chóp $S.ABC$ có $SA \perp (ABC)$, $SA=AB=2a$, tam giác $ABC$ vuông tại $B$. Khoảng cách từ $A$ đến mặt phẳng $(SBC)$ bằng
	% }
	% {
	% 	\begin{tikzpicture}[line join = round, line cap = round, thick, font = \small, scale = 0.8]
	% 		\path
	% 		(0:0) coordinate (A)
	% 		+(0:5) coordinate (C)
	% 		+(-50:3) coordinate (B)
	% 		+(90:4) coordinate (S)
	% 		;
	% 		\draw[dashed]
	% 		(A)--(C)
	% 		;
	% 		\draw
	% 		(S)--(A)--(B)--(C)--cycle
	% 		(S)--(B)
	% 		\foreach \x/\y/\z in {S/A/C, S/A/B}{
	% 				pic[draw, angle radius = 8pt]{right angle = \x--\y--\z}
	% 			}
	% 		;
	% 		\foreach \x/\g in {A/180,C/0,B/-90,S/90}
	% 		\fill (\x) circle (1.5pt)
	% 		+(\g:3mm) node {$\x$};
	% 	\end{tikzpicture}
	% }
	\choice
	{$a\sqrt{3}$}
	{\True $a\sqrt{2}$}
	{$a$}
	{$2a$}
	\loigiai{
		\begin{center}
			\begin{tikzpicture}[line join = round, line cap = round, thick, font = \small, scale = 1]
				\path
				(0:0) coordinate (A)
				+(0:5) coordinate (C)
				+(-50:3) coordinate (B)
				+(90:4) coordinate (S)
				($(B)!(A)!(S)$) coordinate (H)
				;
				\draw[dashed]
				(A)--(C)
				;
				\draw
				(S)--(A)--(B)--(C)--cycle
				(S)--(B)
				(A) -- (H)
				;
				\foreach \x/\g in {A/180,C/0,B/-90,S/90, H/45}
				\fill (\x) circle (1.5pt)
				+(\g:3mm) node {$\x$};
			\end{tikzpicture}
		\end{center}
		Ta có $SA \perp (ABC)\Rightarrow SA \perp BC$, $BC \perp AB \Rightarrow BC \perp (SAB) \Rightarrow BC\perp AH$. \\
		Kẻ $AH \perp SB$ và $AH \perp BC \Rightarrow AH \perp (SBC)$, suy ra khoảng cách từ $A$ đến mặt phẳng $(SBC)$ bằng $AH$. \\
		Xét tam giác $SAH$, ta có $\dfrac{1}{AH^2} = \dfrac{1}{SA^2} + \dfrac{1}{AB^2} = \dfrac{1}{2a^2} \Rightarrow AH=a\sqrt{2}$.
	}
\end{ex}
%%%==============HetCau_EX6==============%%%

%%%==============Cau_EX7==============%%%
\begin{ex}%[Dự án C THPTQG 2025]%[Vương Quốc Phong]%[1D1H5-1]
	Biết phương trình $\sin x=m$ có một họ nghiệm là $x=\dfrac{\pi}{5}+k2\pi$, $k\in \mathbb{Z}$. Họ nghiệm còn lại của phương trình đã cho là biểu thức nào sau đây?
	\choice
	{$x=\dfrac{4\pi}{5}+k\pi, k\in \mathbb{Z}$}
	{$x=\dfrac{\pi}{5}+k\pi, k\in \mathbb{Z}$}
	{\True $x=\dfrac{4\pi}{5}+k2\pi, k\in \mathbb{Z}$}
	{$x=-\dfrac{\pi}{5}+k2\pi, k\in \mathbb{Z}$}
	\loigiai{
		Ta có phương trình $\sin x=m$ có một họ nghiệm là $x=\dfrac{\pi}{5}+k\pi, k\in \mathbb{Z}$, họ nghiệm còn lại là $x=\pi-\dfrac{\pi}{5}+k2\pi=\dfrac{4\pi}{5}+k2\pi, k\in \mathbb{Z}$.
	}
\end{ex}
%%%==============HetCau_EX7==============%%%

%%%==============Cau_EX8==============%%%
\begin{ex}%[Dự án C THPTQG 2025]%[Vương Quốc Phong]%[2D1H1-2]
	\immini[thm]{Cho hàm số $y=f(x)$ xác định, có đạo hàm trên $\mathbb{R}$ và $f'(x)$ có đồ thị như hình vẽ. Hàm số $y=f(x)$ đồng biến trên khoảng nào dưới đây?
	\choice
	{$(1;4)$}
	{\True $(-1;1)$}
	{$(1;+\infty)$}
	{$(-\infty;-1)$}
	}{\begin{tikzpicture}[line join = round, line cap = round, >=stealth, scale = .7]
			%Hệ trục Oxy và hàm số cần vẽ
			\def\xmin{-1.5}     \def\xmax{4.5}
			\def\ymin{-2.5}       \def\ymax{2.5}
			\def\f(#1){0.25*(#1)^3-(#1)^2-0.25*(#1)+1}
			%Vẽ hệ trục
			\draw[->] (\xmin,0)--(0,0) node[below right]{$O$}--(\xmax,0) node[below]{$x$};
			\draw[->] (0,\ymin)--(0,\ymax) node[right]{$y$};
			%Vẽ hàm số
			\begin{scope}
				\clip (\xmin,\ymin) rectangle (\xmax,\ymax);
				\draw[smooth, thick, blue!50!black] plot[domain = \xmin:\xmax, samples = 200, variable=\x]({\x},{\f(\x)});
			\end{scope}
			%Vẽ các điểm gióng
			\foreach \x in {}{
					\pgfmathsetmacro\fx{\f(\x)}
					\draw[dashed,thin] (\x,0) |- (0,{\fx});
				}
			\foreach \x/\g in {-1/135, 1/90, 4/120}
			\fill (\x, 0) circle (1.5pt)
			+(\g:3mm) node {$\x$};
		\end{tikzpicture}}
	\loigiai{
		Dựa vào đồ thị ta thấy $f'(x) > 0\Leftrightarrow \hoac{&{-1< x < 1} \\ &{x > 4}}$, suy ra hàm số đồng biến trên $(-1;1)$, $(4;+\infty)$.
	}
\end{ex}
%%%==============HetCau_EX8==============%%%

%%%==============Cau_EX9==============%%%
\begin{ex}%[Dự án C THPTQG 2025]%[Vương Quốc Phong]%[2H2N1-2]
	Cho hình hộp $ABCD.A'B'C'D'$ có đáy $ABCD$ là hình bình hành tâm $O$. Khi đó $2 \cdot \overrightarrow{AO}$ bằng véc-tơ nào sau đây?
	% \begin{center}
	% 	\begin{tikzpicture}[line join = round, line cap = round, thick, font = \small, scale = 1]
	% 		\def \canh{4}
	% 		\path
	% 		(0:0) coordinate (D')
	% 		+(90:\canh) coordinate (D)
	% 		+(0:\canh) coordinate (C')
	% 		+(40:.6*\canh) coordinate (A')
	% 		($(C')+(D)-(D')$) coordinate (C)
	% 		($(D)+(A')-(D')$) coordinate (A)
	% 		($(C')+(A')-(D')$) coordinate (B')
	% 		($(C)+(A)-(D)$) coordinate (B)
	% 		($(A)!.5!(C)$) coordinate (O)
	% 		;
	% 		\draw[dashed]
	% 		(A')--(A) (A')--(B') (A')--(D')
	% 		;
	% 		\draw
	% 		(A)--(B)--(B')--(C')--(D')--(D)--cycle
	% 		(C)--(B) (C)--(D) (C)--(C') (A)--(C)
	% 		;
	% 		\foreach \x/\g in {D'/-90,C'/-90,D/180,A'/135,C/-45,A/90,B'/0,B/90, O/30}
	% 		\fill (\x) circle (1.5pt)
	% 		+(\g:3mm) node {$\x$};
	% 	\end{tikzpicture}
	% \end{center}
	\choice
	{$\overrightarrow{A'C}$}
	{$\overrightarrow{AB}$}
	{$\overrightarrow{AD}$}
	{\True $\overrightarrow{AC}$}
	\loigiai{
		Ta có $2 \cdot \overrightarrow{AO}=\overrightarrow{AC}$.
	}
\end{ex}
%%%==============HetCau_EX9==============%%%

%%%%==============Cau_EX10==============%%%
\begin{ex}%[Dự án C THPTQG 2025]%[Vương Quốc Phong]%[1H8N2-1]
	Trong không gian, qua một điểm $O$ cho trước có bao nhiêu đường thẳng vuông góc với  mặt phẳng $(\alpha)$ cho trước.
	\choice
	{\True $1$}
	{Vô số}
	{$2$}
	{$0$}
	\loigiai{
		Qua một điểm $O$ cho trước có một đường thẳng vuông góc với mặt phẳng $(\alpha)$ cho trước.
	}
\end{ex}
%%%%==============HetCau_EX10==============%%%

%%%%==============Cau_EX11==============%%%
\begin{ex}%[Dự án C THPTQG 2025]%[Vương Quốc Phong]%[2D1N4-1]
	Đường tiệm cận ngang của đồ thị hàm số $y=\dfrac{2x+1}{x+1}$ là
	\choice
	{\True $y=2$}
	{$x=-1$}
	{$y=-1$}
	{$x=2$}
	\loigiai{
		Đường tiệm cận ngang của đồ thị hàm số $y=\dfrac{2x+1}{x+1}$ là $y=2$.
	}
\end{ex}
%%%%==============HetCau_EX11==============%%%

%%%==============Cau_EX12==============%%%
\begin{ex}%[Dự án C THPTQG 2025]%[Vương Quốc Phong]%[1D7H2-1]
	Đạo hàm của hàm số $y=3^x$ là
	\choice
	{\True $y'=3^x \cdot \ln x$}
	{$y'=3^x$}
	{$y'=x\cdot 3^{x-1}$}
	{\True $y'=3^x\cdot \ln 3$}
	\loigiai{
		Đạo hàm của hàm số $y=3^x$ là $y'=3^x \cdot \ln 3$.
	}
\end{ex}
%%%==============HetCau_EX12==============%%%

%  \Closesolutionfile{ans}
%  \Closesolutionfile{ansbook}
 
\cauds
%   \Opensolutionfile{ansbook}[Ansbook/KSCL-THPT-ChuyenVinhPhuc-VinhPhuc-L1-NH24-25-TF]%---Nên đặt tên theo bài
%   \setcounter{ex}{0}
 %%%==============Cau_EX1==============%%%
\begin{ex}%[Dự án C THPTQG 2025]%[Vương Quốc Phong]%[2D1V5-8]
	Theo báo cáo của một cơ sở sản xuất nước tinh khiết, nếu mỗi ngày cơ sở này sản xuất $x$ (m$^3$) nước tinh khiết thì phải chi phí các khoản sau: $3$ triệu đồng chi phí cố định; $0{,}15$ triệu đồng cho mỗi mét khối sản phẩm; $0{,}0003x^2$ chi phí bảo dưỡng máy móc. Biết công suất tối đa mỗi ngày của cơ sở này là $200$ m$^3$. Gọi $C(x)$ là chi phí sản xuất $x$ (m$^3$) sản phẩm mỗi ngày và $\overline{c}(x)$ là chi phí trung bình mỗi mét khối sản phẩm. Khi đó, mệnh đề sau đây đúng hay sai?
	\choiceTF
	{Chi phí sản xuất $100$ m$^3$ nước tinh khiết là $20$ triệu đồng}
	{\True $\overline{c}(x) = 0{,}0003x + 0{,}15 + \dfrac{3}{x}$}
	{\True Chi phí trung bình mỗi mét khối sản phẩm thấp nhất khi sản lượng nước tinh khiết trong ngày là $100$ m$^3$}
	{$C(x) = 0{,}0003x^2 + 0{,}15x + 5$}
	\loigiai{
	Để sản xuất $x$ (m$^3$) nước tinh khiết thì phải chi phí các khoản sau: $3$ triệu đồng chi phí cố định; $0{,}15$ triệu đồng cho mỗi mét khối sản phẩm; $0{,}0003x^2$ chi phí bảo dưỡng máy móc. \\
	Suy ra để sản xuất $1$ (m$^3$) nước tinh khiết thì cần $\dfrac{3}{x}$ triệu đồng chi phí cố định; $0{,}15$ triệu đồng cho mỗi mét khối sản phẩm; $0{,}0003x$ chi phí bảo dưỡng máy móc. \\
	$\Rightarrow \overline{c}(x)=\dfrac{3}{x}+0{,}15+0{,}0003x$. \\
	$\Rightarrow C(x)=\overline{c}(x)\cdot x=3+0{,}15x+0{,}0003x^2$.
	\begin{itemchoice}
		\itemch Chi phí sản xuất $100$ m$^3$ là $C(100)=3+0{,}15 \cdot 100 + 0{,}0003 \cdot 100^2 =21$ (triệu đồng).
		\itemch Ta tìm được $\overline{c}(x)=\dfrac{3}{x}+0{,}15+0{,}0003x$.
		\itemch Hàm chi phí trung bình mỗi mét khối sản phẩm là $\overline{c}(x)=\dfrac{3}{x}+0{,}15+0{,}0003x$, $0< x\le 200$. \\
		Đặt $f(x)=\overline{c}(x)=\dfrac{3}{x}+0{,}15+0{,}0003x$, $0< x\le 200$. \\
		$f'(x)=-\dfrac{3}{x^2}+0{,}0003$. \\
		$f'(x)=0\Rightarrow-3+0{,}0003x^2=0\Rightarrow x=100$. \\
		Bảng biến thiên của hàm $f(x)$.
		\begin{center}
			\begin{tikzpicture}
				\tkzTabInit[lgt=1.2,espcl=4]
				{$x$/1,$f’(x)$/1,$f(x)$/2}
				{$0$, $100$, $200$}
				\tkzTabLine{ ,-,z,+, }
				\tkzTabVar{+/, -/$0{,}21$, +/}
			\end{tikzpicture}
		\end{center}
		Dựa vào BBT thì chi phí trung bình mỗi mét khối sản phẩm thấp nhất khi sản lượng nước tinh khiết trong ngày là $100$ m$^3$.
		\itemch Ta có: $C(x) = 3 + 0{,}15x+0{,}0003x^2$.
	\end{itemchoice}
	}
\end{ex}
%%%==============HetCau_EX1==============%%%

%%%==============Cau_EX2==============%%%
\begin{ex}%[Dự án C THPTQG 2025]%[Vương Quốc Phong]%[1H8V7-2]
	Cho hình chóp $S.ABC$ có mặt bên $(SAB)$ vuông góc với mặt phẳng đáy và tam giác $SAB$ đều cạnh $2a$. Biết tam giác $ABC$ vuông tại $C$ và cạnh $AC=a\sqrt{3}$. Gọi $H$ là trung điểm $AB$.
	\choiceTF
	{\True Mặt phẳng $(SHC)$ và $(ABC)$ vuông góc với nhau}
	{Thể tích của khối chóp $S.ABC$ bằng $\dfrac{a^3}{6}$}
	{$d(C,(SAB)) = \dfrac{a\sqrt{3}}{3}$}
	{\True $SH \perp (ABC)$}
	\loigiai{
		\begin{center}
			\begin{tikzpicture}[line join = round, line cap = round, thick, font = \small, scale = 1]
				\path
				(0:0) coordinate (B)
				+(0:5) coordinate (C)
				+(-60:3) coordinate (A)
				($(A)!.5!(B)$) coordinate (H)
				++(90:4) coordinate (S)
				;
				\draw[dashed]
				(B)--(C)--(H)
				;
				\draw
				(A)--(B)--(S)--(C)--cycle
				(A)--(S)--(H)
				;
				\foreach \x/\g in {B/180, C/0, S/90, H/210, A/-90}
				\fill (\x) circle (1.5pt)
				+(\g:3mm) node {$\x$};
			\end{tikzpicture}
		\end{center}
		\begin{itemchoice}
			\itemch
			\begin{itemize}
				\item Hình chóp $S.ABC$ có mặt bên $(SAB)$ vuông góc với mặt phẳng đáy và tam giác $SAB$ đều nên $SH \perp AB$, mà $AB$ là giao tuyến của $(SAB)$ và mp đáy nên $SH \perp (ABC)$.
				\item $SH \perp (ABC)$, mà $SH \subset (SHC)$ nên $(SHC)\perp (ABC)$.
			\end{itemize}
			\itemch
			\begin{itemize}
				\item $SH \perp (ABC)$, $SH$ là đường cao của hình chóp, $SH = 2a \dfrac{\sqrt{3}}{2}=a\sqrt{3}$.
				\item Tam giác $ABC$ vuông tại $C$ và cạnh $AC=a\sqrt{3}$, $AB=2a$ nên $BC=\sqrt{(2a)^2-\left(a\sqrt{3} \right)^2}=a$
				\item $S_{ABC}=\dfrac{1}{2} a \cdot a\sqrt{3}=\dfrac{a^2\sqrt{3}}{2}$
				\item $V_{S.ABC}=\dfrac{1}{3} SH \cdot S_{ABC}=\dfrac{1}{3} a\sqrt{3} \cdot \dfrac{a^2\sqrt{3}}{2}=\dfrac{a^3}{2}$
			\end{itemize}
			\itemch $d\left(C, (SAB)\right) = \dfrac{3 \cdot V_{S.ABC}}{_{S_{SAB}}} = \dfrac{3 \cdot a^3}{2 \cdot \dfrac{4a^2\sqrt{3}}{4}}=\dfrac{a\sqrt{3}}{2}$
			\itemch  Theo lập luận trong các câu trên ta có $SH \perp (ABC)$.
		\end{itemchoice}
	}
\end{ex}
%%%==============HetCau_EX2==============%%%

%%%==============Cau_EX3==============%%%
\begin{ex}%[Dự án C THPTQG 2025]%[Vương Quốc Phong]%[2D3H1-3]
	Phòng quản lí đào tạo trường Đại học Kinh tế quốc dân thống kê số giờ làm thêm của một nhóm sinh viên năm thứ tư của trường thu được kết quả như bảng sau:
	\begin{center}
		\begin{tabular}{|c|c|c|c|c|c|}
			\hline
			\thead{\textbf{Số giờ làm thêm (giờ/tuần)}} & $[9; 12)$ & $[12; 15)$ & $[15; 18)$ & $[18; 21)$ & [$21; 24)$ \\
			\hline
			\thead{\textbf{Số sinh viên}}               & $6$       & $12$       & $4$        & $2$        & $1$        \\
			\hline
		\end{tabular}
	\end{center}
	\choiceTF
	{Số giờ làm thêm trung bình của nhóm sinh viên trên trong một tuần là $16{,}5$ giờ}
	{\True Giá trị đại diện của nhóm $[9;12)$ là $10{,}5$}
	{Tứ phân vị thứ ba là $15{,}65$}
	{Nhóm chứa trung vị là $[15;18)$}
	\loigiai{
		\begin{itemchoice}
			\itemch Cỡ mẫu: $n=6+12+4+2+1=25$. \\
			Ta có bảng sau:
			\begin{center}
				\begin{tabular}{|c|c|c|c|c|c|}
					\hline
					\thead{\textbf{Số giờ làm thêm (giờ/tuần)}} & $[9; 12)$ & $[12; 15)$ & $[15; 18)$ & $[18; 21)$ & $[21; 24)$ \\
					\hline
					\thead{\textbf{Giá trị đại diện}}           & $10{,}5$  & $13{,}5$   & $16{,}5$   & $19{,}5$   & $22{,}5$   \\
					\hline
					\thead{\textbf{Số sinh viên}}               & $6$       & $12$       & $4$        & $2$        & $1$        \\
					\hline
				\end{tabular}
			\end{center}
			Số giờ làm thêm trung bình của nhóm sinh viên trên trong một tuần là
			$$\overline{x}=\dfrac{6\cdot 10{,}5+12\cdot 13{,}5+4\cdot 16{,}5+2\cdot 19{,}5+1\cdot 22{,}5}{25}=14{,}1 \text{ (giờ).}$$
			\itemch Giá trị đại diện của nhóm $\left[9;12\right)$ là $10{,}5$.
			\itemch Giả sử $x_1, x_2,\ldots, x_{25}$ số giờ làm thêm của các sinh viên trong mẫu số liệu trên và dãy này đã được sắp xếp theo thứ tự không giảm. \\
			Khi đó, trung vị của mẫu số liệu là $x_{13}$ và tứ phân vị thứ ba là $\dfrac{1}{2} \left(x_{19}+x_{20} \right)$. Vì $x_{19}, x_{20}$ đều thuộc nhóm $\left[15;18\right)$ nên nhóm này chứa tứ phân vị thứ ba. Do đó, tứ phân vị thứ ba là:
			\[Q_3=15+\dfrac{\dfrac{3\cdot 25}{4}-12-6}{4} \cdot \left(18-15\right)=15{,}5625.
			\]
			\itemch  Vì $x_{13}$ thuộc nhóm $\left[12;15\right)$ nên nhóm chứa trung vị là nhóm $\left[12;15\right)$.
		\end{itemchoice}
	}
\end{ex}
%%%==============HetCau_EX3==============%%%

%%%==============Cau_EX4==============%%%
\begin{ex}%[Dự án C THPTQG 2025]%[Vương Quốc Phong]%[2D1V4-3]
	Cho hàm số $f(x)=\dfrac{ax+b}{cx+d}$ với $a,b,c,d\in \mathbb{R}$ và $c \ne 0$ có đồ thị hàm số $y=f'(x)$ nhận đường thẳng $x=-1$ làm tiệm cận đứng như hình vẽ dưới. Biết rằng giá trị lớn nhất của hàm số $y=f(x)$ trên đoạn $\left[-3;-2\right]$ bằng $8$.
	\begin{center}
		\begin{tikzpicture}[line join = round, line cap = round, >=stealth,  scale = .7]
			%Hệ trục Oxy và hàm số cần vẽ
			\def\xmin{-4}     \def\xmax{4}
			\def\ymin{-1}       \def\ymax{5}
			\def\f(#1){3/((#1)+1)^2}
			%Vẽ hệ trục
			\draw[->] (\xmin-0.2,0)--(0,0) node[below right]{$O$}--(\xmax + 0.5,0) node[below]{$x$};
			\draw[->] (0,\ymin)--(0,\ymax + 0.5) node[right]{$y$};
			%Vẽ hàm số
			\begin{scope}
				\clip (\xmin,\ymin) rectangle (\xmax,\ymax);
				\draw[smooth, thick, blue!50!black] plot[domain = \xmin: -1.4, samples = 200, variable=\x]({\x},{\f(\x)});
				\draw[smooth, thick, blue!50!black] plot[domain = -0.7: \xmax, samples = 200, variable=\x]({\x},{\f(\x)});
			\end{scope}
			\draw[dashed] (-1,-1)--(-1,5.5);
			%Vẽ các điểm gióng
			\foreach \x in {}{
					\pgfmathsetmacro\fx{\f(\x)}
					\draw[dashed,thin] (\x,0) |- (0,{\fx});
				}
			%Vẽ các điểm trên trục Ox
			\foreach \x/\g in {-4/-90,-3/-90,-2/-90,-1/-90,1/-90,2/-90,3/-90,4/-90}
			\draw[thin] (\x,2pt)--(\x,-2pt) + (\g:3mm) node {$\x$};
			%Vẽ các điểm trên trục Oy
			\foreach \y/\g in {1/180,2/180,3/180,4/180,5/180}
			\draw[thin] (2pt,\y)--(-2pt,\y) + (\g:3mm) node {$\y$};
		\end{tikzpicture}
	\end{center}
	\choiceTF
	{\True Giá trị nhỏ nhất của hàm số $y=f(x)$ trên đoạn $\left[2;4\right]$ bằng $4$}
	{$f(-3)=8$}
	{\True Hàm số $y=f(x)$ nghịch biến trên khoảng $(-1;+\infty)$}
	{\True Đồ thị hàm số $y=f'(x)$ nhận đường thẳng $y=0$ làm tiệm cận ngang}
	\loigiai{
		Tập xác định: $\mathscr{D} = \mathbb{R} \setminus \left\{-\dfrac{d}{c} \right\}$. \\
		Ta có $f'(x)=\dfrac{ad-bc}{(cx+d)^2}$.\\
		Đồ thị hàm số $y=f'(x)$ nhận đường thẳng $x=-1$ làm tiệm cận đứng nên $c=d\ne 0$. \\
		Hơn nữa, $f'(x)$ nằm hoàn toàn trên trục hoành nên hàm số $y=f(x)$ đồng biến trên các khoảng xác định và $f'(0)=3$ nên
		\[\heva{&\max\limits_{\left[-3;-2\right]} f(x) = f(-2) = 8 \\ &{\dfrac{ad-bc}{d^2}=3}} \Leftrightarrow \heva{&{\dfrac{2a-b}{d}=8} \\ &{(a-b)d = 3d^2}} \xrightarrow{c=d} \heva{&2a-b=8d \\ & \hoac{&d=0 \text{ (loại)} \\ &a-b=3d}} \Leftrightarrow \heva{&{a=5d} \\&{b=2d}}\]
		Chọn $d=1 \Rightarrow a=5$, $b=2$. \\
		Khi đó $f(x)=\dfrac{5x+2}{x+1} \Rightarrow f'(x)=\dfrac{3}{\left(x+1\right)^2}$.
		\begin{itemchoice}
			\itemch Giá trị nhỏ nhất của hàm số $y=f(x)$ trên đoạn $\left[2;4\right]$ bằng $4$. \\
			Hàm số $y=f(x)$ đồng biến trên $\left[2;4\right]$ nên ${\min \limits_{\left[2;4\right]}} f(x)=f(2)=4$.
			\itemch $f(-3) = \dfrac{5 \cdot (-3) + 2}{-3+2}=\dfrac{13}{2}$.
			\itemch Hàm số $y=f(x)$ đồng biến trên khoảng $\left(-1;+\infty \right)$.
			\itemch Đồ thị hàm số $y=f'(x)$ nhận đường thẳng $y=0$ làm tiệm cận ngang. \\
			Ta có $\lim \limits_{x\to+\infty} f'(x) = \lim \limits_{x\to+\infty} \dfrac{3}{(x+1)^2}=0\Rightarrow y=0$ là TCN của đồ thị hàm số $y=f'(x)$.
		\end{itemchoice}
	}
\end{ex}
%%%==============HetCau_EX4==============%%% \Closesolutionfile{ansbook}
 

\caukq
% \Opensolutionfile{ansbt}[Ansbook/KSCL-THPT-ChuyenVinhPhuc-VinhPhuc-L1-NH24-25-TLN]%---Nên đặt tên theo bài
% \setcounter{ex}{0}
%%%==============Bai_BT1==============%%%
\begin{ex}%[Dự án C THPTQG 2025]%[Vương Quốc Phong]%[2D1V5-8]
	Hai con tàu $A$ và $B$ đang ở cùng một vĩ tuyến và cách nhau $6$ hải lí. Cả hai tàu đồng thời cùng khởi hành. Tàu $A$ chạy về hướng Nam với vận tốc $5$ hải lí/giờ, còn tàu $B$ chạy về vị trí hiện tại của tàu $A$ với vận tốc $7$ hải lí/ giờ. Hỏi sau bao nhiêu giờ thì khoảng cách giữa hai tàu là bé nhất?
	\begin{center}
		\begin{tikzpicture}
			\path
			(0,0) coordinate (A)
			(3,0) coordinate (B1)
			(7,0) coordinate (B)
			(0,-2) coordinate (A1)
			(0,-4) coordinate (A2)
			;
			\draw
			(A2)--(A)node[above]{\Huge{\faShip}}--(B)node[above]{\Huge{\faShip}}
			(A1) -- (B1)
			;
			\draw[-stealth, transform canvas = {xshift = -1 cm}] (A)--(A1);
			\draw[-stealth, yshift = 0.5 cm] (6,0)--(4,0);
			\foreach \x/\g in {A/180,B/-90,A1/180, B1/-90}
			\fill (\x) circle (1.5pt)
			+(\g:3mm) node{$\x$};
		\end{tikzpicture}
	\end{center}
	\shortans{0,57}
	\loigiai{
		Giả sử ban đầu tàu $A$ ở vị trí $A$ và tàu $B$ ở vị trí $B$. Sau khoảng thời gian $t$:
		\begin{itemize}
			\item Tàu $A$ di chuyển được quãng đường $5t$ về phía Nam đến vị trí $A_1$.
			\item Tàu $B$ di chuyển được quãng đường $7t$ đến vị trí $B_1$.
		\end{itemize}
		Khoảng cách từ vị trí $B_1$ đến vị trí $A$ là $6-7t$. \\
		Áp dụng định lý Pytago ta có:  $d=A_1 B_1=f(t)=\sqrt{(6-7t)^2+(5t)^2}=\sqrt{74t^2-84t+36}$. \\
		Để khoảng cách giữa hai tàu nhỏ nhất, thì hàm số $g(t)=74t^2-84t+36$ đạt giá trị nhỏ nhất. \\
		Hàm số $g(t)$ đạt giá trị nhỏ nhất tại $t=\dfrac{-(-84)}{2\cdot 74} = \dfrac{21}{37}$, vậy thời điểm khoảng cách giữa hai tàu bé nhất là khi $t=\dfrac{21}{37} \approx 0{,}57$ (giờ).
	}
\end{ex}
%%%==============HetBai_BT1==============%%%

%%%==============Bai_BT2==============%%%
\begin{ex}%[Dự án C THPTQG 2025]%[Vương Quốc Phong]%[2H2V2-6]
	Có ba lực cùng tác động vào một vật. Hai trong ba lực này hợp với nhau một góc $100^{\circ}$ và có độ lớn lần lượt là $25$ N và $12$ N. Lực thứ ba vuông góc với mặt phẳng tạo bởi hai lực đã cho và có độ lớn $4$ N. Tính độ lớn của hợp lực của ba lực trên (làm tròn đến hàng phần chục).
	\shortans{26,1}
	\loigiai{
		\begin{center}
			\begin{tikzpicture}[line join = round, line cap = round, thick, font = \small, scale = 1]
				\path
				(0:0) coordinate (B)
				+(0:5) coordinate (D)
				+(65:3) coordinate (O)
				($(O)+(D)-(B)$) coordinate (A)
				($(O)+(0,3)$) coordinate (C)
				($(C)+(D)-(O)$) coordinate (E)
				;
				\foreach \x/\y in {O/B, O/A, O/D, O/C}
				\draw[-stealth] (\x)--(\y);
				\draw[dashed]
				(C)--(E)--(D)
				(B)--(D)--(A)
				;
				\foreach \x/\g in {D/-90,C/90,A/90,B/180, O/180, E/90}
				\path (\x)
				+(\g:3mm) node{$\x$};
				\draw[thin] pic[draw, angle radius = 7mm, "$100^{\circ}$", angle eccentricity = 1.5]{ angle = B--O--A};
			\end{tikzpicture}
		\end{center}
		Gọi $\overrightarrow{F_1}$, $\overrightarrow{F_2}$, $\overrightarrow{F_3}$ là ba lực tác động vào vật tại điểm $O$ lần lượt có độ lớn $25$ N, $12$ N, $4$ N. \\
		Vẽ $\overrightarrow{OA}=\overrightarrow{F_1}$, $ \overrightarrow{OB}=\overrightarrow{F_2}$, $ \overrightarrow{OC}=\overrightarrow{F_3}$, dựng hình bình hành $OADB$ và $ODEC$. \\
		Khi đó hợp lực tác động vào vật là: $\overrightarrow{F}=\overrightarrow{OA}+\overrightarrow{OB}+\overrightarrow{OC}=\overrightarrow{OD}+\overrightarrow{OC}=\overrightarrow{OE}$. \\
		Áp dụng định lý cô sin trong tam giác $OBD$, ta có:
		\[OD^2 = OB^2 + BD^2 - 2OB \cdot BD \cos \widehat{OBD}=12^2+25^2-2 \cdot 12 \cdot 25 \cdot \cos 80^{\circ}=769-600 \cdot \cos 80^\circ
		\]
		Vì $OC \perp (OADB)$ nên $OC\perp OD$, suy ra $ODEC$ là hình chữ nhật. Do đó tam giác $ODE$ vuông tại $D$. Ta có $OE=\sqrt{OD^2+ED^2} \approx 26{,}1$
	}
\end{ex}
%%%==============HetBai_BT2==============%%%

%%%==============Bai_BT3==============%%%
\begin{ex}%[Dự án C THPTQG 2025]%[Vương Quốc Phong]%[1D6V4-6]
	Dân số trung bình sơ bộ năm $2021$ của tỉnh Vĩnh Phúc là $1.191.782$ người, tăng $1{,}75\%$ so với năm $2020$. Hỏi với tốc độ tăng dân số được duy trì mức $1{,}75\%$ một năm thì đến năm bao nhiêu dân số tỉnh Vĩnh Phúc lần đầu vượt $1.880.000$ người.
	\shortans{2048}
	\loigiai{
	Áp dụng công thức tăng trưởng dân số thì dân số của tỉnh Vĩnh Phúc sau $n$ năm (tính từ năm $2021$) được tính theo công thức:
	\[S_n=S_0 \cdot e^{rn}=1191782 \cdot \mathrm{e}^{0{,}0175n}
	\]
	Để dân số tỉnh Vĩnh Phúc sau $n$ năm vượt $1.880.000$ người điều kiện là:
	\begin{eqnarray*}
		&& S_n > 1880000 \\ &\Leftrightarrow& 1191782 \cdot \mathrm{e}^{0{,}0175n} > 1880000 \\ &\Leftrightarrow& \mathrm{e}^{0{,}0175n} > \dfrac{1880000}{1191782} \Leftrightarrow 0{,}0175n > \ln \left(\dfrac{1880000}{1191782} \right) \\ &\Leftrightarrow& n > \dfrac{\ln \left(\dfrac{1880000}{1191782} \right)}{0{,}0175} \approx 26{,}047
	\end{eqnarray*}
	Mà $n \in \mathbb{N}$ nên $n\ge 27$.
	Vậy năm $2048$ là năm đầu tiên dân số tỉnh Vĩnh Phúc vượt $1.880.000$ người.
	}
\end{ex}
%%%==============HetBai_BT3==============%%%

%%%==============Bai_BT4==============%%%
\begin{ex}%[Dự án C THPTQG 2025]%[Vương Quốc Phong]%[0D0C2-9]
	Hai bạn Nga và Nhung chơi trò tung xúc xắc. Mỗi bạn tung $1$ con xúc xắc $3$ lần, ai có tổng số chấm $3$ lần gieo lớn hơn thì thắng. Nga chơi trước và được $14$ chấm. Khi đó, xác suất để Nhung thắng Nga là $\dfrac{a}{b}$ (với $a,b$ là số nguyên dương và $\dfrac{a}{b}$ là phân số tối giản). Tính $a+b$. \\
	\shortans{59}
	\loigiai{
		Gọi $A$ là biến cố: \lq\lq Nhung thắng Nga sau ba lần tung\rq\rq.\\
		Khi đó $P(A)$ là xác suất tổng số chấm Nga tung được sau ba lần tung lớn hơn $14$. \\
		Ta có: $\Omega=\left\{(a_1, a_2, a_3)|a_i \in \{1,2,\ldots, 6\}, i=\overline{1,3}\right\}\Rightarrow n(\Omega)=6^3=216$
		\[A=\left\{(a_1, a_2, a_3)|a_1+a_2+a_3 \ge 15, a_i \in \{1,2,\ldots, 6\}, i=\overline{1,3}\right\}
		\]
		Để đếm số phần tử của $A$, ta chia thành các trường hợp:
		\begin{itemize}
			\item Trường hợp 1: $a_1+a_2+a_3=15$, gồm bộ $(5,5,5)$ và các bộ là hoán vị của $(4;5;6)$ và $(3;6;6)$. Trường hợp này có $1+6+3=10$ (bộ)
			\item  Trường hợp 2: $a_1+a_2+a_3=16$, gồm các hoán vị của $(5,5,6)$ và $(4,6,6)$. Trường hợp này có $3+3=6$ (bộ)
			\item  Trường hợp 3: $a_1+a_2+a_3=17$, gồm các hoán vị của $(5,6,6)$. Trường hợp này có $3$ (bộ).
			\item Trường hợp 4: $a_1+a_2+a_3=18$, gồm $(6,6,6)$. Trường hợp này có $1$ (bộ).
		\end{itemize}
		Vậy $n(A)=20$. \\
		Xác suất cần tìm: $P(A)=\dfrac{n(A)}{n(\Omega)}=\dfrac{20}{216}=\dfrac{5}{54} \Rightarrow a+b=59$.
	}
\end{ex}
%%%==============HetBai_BT4==============%%%

%%%==============Bai_BT5==============%%%
\begin{ex}%[Dự án C THPTQG 2025]%[Vương Quốc Phong]%[1H8V5-3]
	Cho hình chóp $S.ABC$ có đáy $ABC$ là tam giác đều cạnh bằng $2$, $SA$ vuông góc với mặt phẳng $(ABC)$; Góc giữa đường thẳng $SB$ và mặt phẳng $(ABC)$ bằng $60^\circ$. Gọi $M$ là trung điểm của cạnh $AB$. Tính khoảng cách từ điểm $B$ đến mặt phẳng $(SCM)$, kết quả làm tròn đến phần trăm.
	\shortans{0,96}
	\loigiai{
		\begin{center}
			\begin{tikzpicture}[line join = round, line cap = round, thick, font = \small, scale = 1]
				\path
				(0:0) coordinate (A)
				+(0:5) coordinate (C)
				+(-50:3) coordinate (B)
				+(90:4) coordinate (S)
				($(A)!.5!(B)$) coordinate (M)
				($(S)!0.65!(M)$) coordinate (H)
				;
				\draw[dashed]
				(A)--(C)--(M)
				;
				\draw
				(S)--(A)--(B)--(C)--cycle
				(S)--(B) (S)--(M) (A)--(H)
				\foreach \x/\y/\z in {C/M/B, A/H/M}{
						pic[draw, angle radius = 6pt]{right angle = \x--\y--\z}
					}
				;
				\foreach \x/\g in {B/-90, C/0, A/180, S/90, M/180, H/145}
				\fill (\x) circle (1.5pt)
				+(\g:3mm) node {$\x$};
			\end{tikzpicture}
		\end{center}
		Vì $AB$ là hình chiếu của $SB$ trên mặt phẳng $(ABC)$, nên góc giữa đường thẳng $SB$ và mặt phẳng $(ABC)$ bằng góc $\widehat{SBA}=60^\circ \Rightarrow SA = AB \cdot \tan 60^\circ=2\sqrt{3}$. \\
		Do $M=AB\cap (SCM)$, $M$ là trung điểm của $AB \Rightarrow d(A,(SCM))= d(B,(SCM))$. \\
		Vì $\heva{&{CM\perp AB} \\ &{CM\perp SA}} \Rightarrow CM\perp (SAB)$. \\
		Mặt khác $CM \subset (SCM)\Rightarrow (SCM)\perp (SAB)$, và $(SCM)\cap (SAB)=SM$, nên kẻ $AH\perp SM$ tại $H$ \\
		$\Rightarrow AH\perp (SMB)\Rightarrow AH = d(A,(SMC)) = d(B,(SMC))$. \\
		Xét tam giác $SAM$ vuông tại $A$, ta có $\dfrac{1}{AH^2}=\dfrac{1}{SA^2}+\dfrac{1}{AM^2}=\dfrac{1}{(2\sqrt{3})^2}+\dfrac{1}{1^2}=\dfrac{13}{12} \Rightarrow AH^2=\dfrac{12}{13} \Rightarrow AH=\sqrt{\dfrac{12}{13}} \approx 0{,}96$. \\
		Vậy khoảng cách từ điểm $B$ đến mặt phẳng $(BCM)$ bằng $0{,}96$
	}
\end{ex}
%%%==============HetBai_BT5==============%%%

%%%==============Bai_BT6==============%%%
\begin{ex}%[Dự án C THPTQG 2025]%[Vương Quốc Phong]%[2D1C5-4]
	Cho hàm số $f(x)=x(x-3)^2$. Tính số nghiệm thực của phương trình $\underbrace{f(f \cdots f(x))}_{8 \text{ lần } f}=0$.
	\shortans{3281}
	\loigiai{
		Ta có $f(x)=x(x-3)^2=x^3-6x^2+9x$. \\
		Suy ra $f'(x)=3x^2-12x+9$. \\
		$f'(x)=0\Leftrightarrow \hoac{&{x=0} \\ &{x=3}}$ \\
		Bảng biến thiên
		\begin{center}
			\begin{tikzpicture}[font=\normalsize,t style/.style={style=solid}]
				%dòng khai báo
				\tkzTabInit[lgt=1.2,espcl=1.75,deltacl=0.5]
				{$x$/0.75,$f'(x)$/0.75, $f(x)$/2}
				{$-\infty$, $0$, $1$, $3$, $4$, $+\infty$}
				%dòng xét dấu của đạo hàm
				\tkzTabLine{,, ,+, 0,-, 0,+ , ,,} % z, t, d, h (h: tô miền);
				%Khai báo vị trị các điểm của dòng f(x)
				\path (N13) node[above] (A1){$ -\infty $}
				($(N32)!0.2!(N33)$) node[below] (A2){$ 4 $}
				($(N42)!0.6!(N43)$) node[left] (A3){$ 0 $}
				(N62) node[below] (A4){$ +\infty $}
				($(N22)!0.6!(N23)$) coordinate (B) node[below] {$0$}
				($(N52)!0.3!(N53)$) coordinate (C) node[below] {$4$}
				;
				\draw[dashed]
				(N21)--(B)--(A3)
				(N51)--(C)--(A2)
				;
				\foreach \x/\y in {A1/A2,A2/A3,A3/A4}{
						\draw[-stealth] (\x)--(\y);
					}
			\end{tikzpicture}
		\end{center}
		Ta có
		$f(x)=0$ có $2$ nghiệm. \\
		$f(x)=3$ có $3$ nghiệm. \\
		$\Rightarrow f\left(f(x)\right)=0\Leftrightarrow \hoac{&{f(x)=0} \\ &{f(x)=3}}$ có $2+3^1$ nghiệm. \\
		$f\left(f\left(f(x)\right)\right)=0\Leftrightarrow \hoac{&{f\left(f(x)\right)=0} \\ &{f\left(f(x)\right)=3}}$ có $2+3^1+3^2$ nghiệm. \\
		$\dots$
		$f\left(f\left(\cdots f(x)\right)\right)=0$ có $2+3^1+3^2+\cdots + 3^7 = 3281$ nghiệm.
	}
\end{ex}
%%%==============HetBai_BT6==============%%%

 \Closesolutionfile{ans}
\inputansbox{6,4,3}{ans/KSCL-THPT-ChuyenVinhPhuc-VinhPhuc-L1-NH24-25}%---Nên đặt tên theo bài
 
% \begin{name}
	{\tenchude}
	{\tendethi}
	{TRƯỜNG THPT LÊ THÁNH TÔNG - TP.HCM}
	{\thoigian}
\end{name}

\caulc
\Opensolutionfile{ans}[ans/LeThanhTong]
% \Opensolutionfile{ansbook}[Ansbook/TenFile-TN]%---Nên đặt tên theo bài
\setcounter{ex}{0}
\begin{ex}%[2D1N1-1]%[Dự án C đợt 3 - KSCL LeThanhTong-Võ Hoàng Nghĩa]
	Cho hàm số $y = \log_3(x^2 - 2x + 3)$. Hàm số đồng biến trên khoảng nào sau đây?
	\choice
	{$(0;1)$}
	{$(-1;+\infty)$}
	{$(-\infty;-1)$}
	{\True $(1;+\infty)$}
	\loigiai{Tập xác định $D=\mathbb{R}$.\\
		Ta có $y'=\dfrac{2x-2}{(x^2-2x+3)\ln 3}$. \\
		$y'>0\Leftrightarrow \dfrac{2x-2}{(x^2-2x+3)\ln 3}>0\Leftrightarrow 2x-2>0\Leftrightarrow x>1$.\\
		Vậy hàm số đồng biến trên khoảng $(1;+\infty)$.}
\end{ex}

\begin{ex}%[2D1H3-6]%[Dự án C đợt 3 - KSCL LeThanhTong-Võ Hoàng Nghĩa]
	Một nhà phân tích thị trường làm việc cho một công ty sản xuất thiết bị gia dụng nhận thấy rằng nếu công ty sản xuất và bán $x$ chiếc máy xay sinh tố hàng tháng thì lợi nhuận thu được (nghìn đồng) có thể được tính bằng công thức $P(x) = -0{,}3x^3 + 36x^2 + 1\,800x - 48\,000$. Để có lợi nhuận lớn nhất công ty cần sản xuất đúng bao nhiêu chiếc máy sinh tố mỗi tháng?
	\choice
	{$90$}
	{\True $100$}
	{$110$}
	{$120$}
	\loigiai{Để tìm lợi nhuận lớn nhất của công ty ta tìm giá trị lớn nhất của hàm số $P(x) = -0.3x^3 + 36x^2 + 1800x - 48\,000$ trên $(0;+\infty)$.\\
	Ta có $P'(x)=-0{,}9x^2+72x+1\,800=0\Leftrightarrow\hoac{&x=100 \text{ (Nhận)}\\&x=-20 \text{ (Loại)}.}$\\
	Ta có bảng biến thiên
	\begin{center}
		\begin{tikzpicture}
			\tkzTabInit[nocadre=false,lgt=2,espcl=3,deltacl=0.9]
			{$x$/0.7,$P'(x)$/0.7,$P(x)$/1.5}{$0$,$100$,$+\infty$}
			\tkzTabLine{,+,0,-,}
			\tkzTabVar{-/$-48\,000$,+/$19\,200$,-/$-\infty$}
		\end{tikzpicture}
	\end{center}
	Từ bảng biến thiên suy ra $\underset{\left( 0;+\infty  \right)}{\mathop{\max }}\,P\left( x \right)=19\,200$ khi $x=100$.\\
	Vậy để có lợi nhuận lớn nhất công ty cần sản xuất đúng $100$ chiếc máy sinh tố mỗi tháng.

	}
\end{ex}

\begin{ex}%[2H2N1-2]%[Dự án C đợt 3 - KSCL LeThanhTong-Võ Hoàng Nghĩa]
	\immini{Cho hình hộp $ABCD.A'B'C'D'$ có tâm $O$. Khi đó, $\overrightarrow{AB} + \overrightarrow{AD} + \overrightarrow{AA'} + \overrightarrow{AC'}$ bằng
		\choice
		{$\overrightarrow{BD}$}
		{$2\overrightarrow{OC'}$}
		{\True $4\overrightarrow{AO}$}
		{$2\overrightarrow{AC}$}}{\begin{tikzpicture}[scale=0.6, font=\footnotesize,line join=round, line cap=round, >=stealth]
			\path
			(0,0) coordinate (A)
			++(-130:3) coordinate (B)
			++(0:4) coordinate (C)
			($(A)+(C)-(B)$) coordinate (D)
			($(A)!1/2!(C)$) coordinate (O)
			;
			\foreach \i in {A,B,C,D}{
					\coordinate (\i') at ($(\i)+(1,4)$);
				}
			\draw (A')--(B')--(C')--(D')--cycle;
			\draw (B)--(B') (C)--(C') (D)--(D')  (B)--(C)--(D) (A')--(C');
			\draw[dashed,thin](B)--(A)--(A') (C)--(A)--(D)--(B);
			\foreach \i/\g in {A'/90,B'/90,C'/90,D'/90,A/-90,B/-90,C/-90,D/-90,O/-90}
			\fill[black] (\i) circle(1pt)+(\g:5mm)node[scale=1]{$\i$};
		\end{tikzpicture}}
	\loigiai{Ta có $\overrightarrow{AB}+\overrightarrow{AD}+\overrightarrow{A{A}'}+\overrightarrow{A{C}'}=\overrightarrow{A{C}'}+\overrightarrow{A{C}'}=2\overrightarrow{A{C}'}=4\overrightarrow{AO}$.}
\end{ex}

\begin{ex}%[2H2N2-3]%[Dự án C đợt 3 - KSCL LeThanhTong-Võ Hoàng Nghĩa]
	Trong không gian với hệ tọa độ $Oxyz$, cho các vectơ $\overrightarrow{a} = (1;-1;2)$, $\overrightarrow{b} = (2;1;-3)$, $\overrightarrow{c} = (0;3;-2)$. Điểm $M(x;y;z)$ thỏa mãn $\overrightarrow{OM} + \overrightarrow{a} = 2\overrightarrow{b} - \overrightarrow{c}$. Tổng $x+y+z$ bằng
	\choice
	{$3$}
	{\True $-3$}
	{$4$}
	{$-2$}
	\loigiai{
		Ta có $\overrightarrow{OM}=(x;y;z)\Rightarrow \overrightarrow{OM}+\overrightarrow{a}=(x+1;y-1;z+2)$,
		$2\overrightarrow{b}-\overrightarrow{c}=(4;-1;-4)$.\\
		Mà $\overrightarrow{OM}+\overrightarrow{a}=2\overrightarrow{b}-\overrightarrow{c}\Rightarrow \heva{& x+1=4\\
				& y-1=-1\\
				& z+2=-4\\
			}\Rightarrow \heva{& x=3\\
				& y=0\\
				& z=-6\\
			}$.\\
		Vậy $x+y+z=3+0+(-6)=-3$.
	}
\end{ex}

\begin{ex}%[2D3H1-3]%[Dự án C đợt 3 - KSCL LeThanhTong-Võ Hoàng Nghĩa]
	\immini{Thời gian (phút) truy cập Internet mỗi buổi tối của một số học sinh được cho trong bảng sau.
		Khoảng tứ phân vị của mẫu số liệu trên là
		\choice
		{$10{,}75$}
		{\True $4{,}75$}
		{$4{,}63$}
		{$4{,}38$}}{\begin{tabular}{|c|c|}
			\hline
			Thời gian (phút) & Số học sinh \\
			\hline
			$[9.5; 12.5)$    & 3           \\
			\hline
			$[12.5; 15.5)$   & 12          \\
			\hline
			$[15.5; 18.5)$   & 15          \\
			\hline
			$[18.5; 21.5)$   & 24          \\
			\hline
			$[21.5; 24.5)$   & 2           \\
			\hline
		\end{tabular}}
	\loigiai{
	Ta có $n=3+12+15+24+2=56$.\\
	Tính tứ phân vị thứ nhất $Q_1$.
	\[\dfrac{n}{4}=14 \Rightarrow Q_1\in \left[12{,}5;15{,}5\right)\Rightarrow Q_1=12{,}5+\dfrac{14-3}{12}\cdot3=\dfrac{61}{4}\]
	Tính tứ phân vị thứ ba $Q_3$.
	$$\dfrac{3n}{4}=42 \Rightarrow Q_3\in\left[18{,}5;21{,}5\right) \Rightarrow Q_3=18{,}5+\dfrac{42-30}{24}\cdot3=20$$
	Khoảng tứ phân vị của mẫu số liệu là $\Delta Q=Q_3-Q_1=4{,}75$.
	}
\end{ex}

\begin{ex}%[1D1V1-6]%[Dự án C đợt 3 - KSCL LeThanhTong-Võ Hoàng Nghĩa]
	\immini{Trên đồng hồ tại thời điểm đang xét kim giờ $OG$ chỉ đúng số $3$, kim phút $OP$ chỉ đúng số $12$. Số đo góc lượng giác mà kim giờ quét được từ lúc xét đến khi kim phút và kim giờ gặp nhau lần đầu tiên bằng
		\choice
		{$\alpha = \dfrac{\pi}{22}$}
		{$\alpha = -\dfrac{2\pi}{45}$}
		{$\alpha=-\dfrac{\pi}{21}$}
		{\True $\alpha=-\dfrac{\pi}{22}$}}{\begin{tikzpicture}[scale=0.3]
			\def\hours{0}
			\def\minutes{0}
			\def\seconds{0}
			\draw[line width=0.2cm] (0,0) circle (5.1cm);
			% Minutes
			\foreach \i in {1,2,...,60}{
					\def\angle{\i*6}
					\draw[thin] (\angle:5cm) -- (\angle:4.9cm);
				}

			% 5 minutes
			\foreach \i in {1,2,...,12}{
					\def\angle{\i*-30+90}
					\draw[thin] (\angle:5cm) -- (\angle:4.5cm);
					\node at (\angle:4cm) {\i};
				};

			% Hour hand
			\def\angle{\hours*-30 + \minutes*-0.5 + \seconds*-0.008333 -180}
			\draw[line width=0.1cm] (0,0) -- (0:2.5cm);

			% Minute hand
			\def\angle{\minutes*-6 + \seconds*-0.1 +90}
			\draw[line width=0.05cm] (0,0) -- (\angle:3.5cm);

			%% Second hand
			% \def\angle{\seconds*-6+90}
			% \draw[very thick,color=red] (\angle:-1cm) -- (\angle:4.5cm);
			% \draw[line width=0.1cm,color=red] (\angle:-1cm) -- (\angle:-0.25cm);

			% Center dot
			\draw[fill=black] (0,0) circle (0.1cm);
		\end{tikzpicture}}
	\loigiai{
		Tốc độ quay của kim phút là $2\pi$ (rad/$1$ giờ).\\
		Tốc độ quay của kim giờ là $\dfrac{1}{12}\cdot2\pi=\dfrac{\pi}{6}$  (rad/$1$ giờ).\\
		Khoảng cách ban đầu giữa hai kim giờ và kim phút là  $\dfrac{1}{4}\cdot2\pi=\dfrac{\pi}{2}$.\\
		Gọi $t$ (giờ) là thời gian hai kim phút và giờ gặp nhau lần đầu tiên.\\
		Ta có phương trình $2\pi\cdot t=\dfrac{\pi}{2}+\dfrac{\pi}{6}\cdot t\Leftrightarrow t=\dfrac{3}{11}$ (giờ).\\
		Vậy góc mà kim giờ đã quét được là $\dfrac{\pi}{6}\cdot\dfrac{3}{11}=\dfrac{\pi}{22}$ (rad).\\
		Do góc lượng giác có chiều dương ngược chiều quay kim đồng hồ. Nên góc lượng giác mà kim giờ quay được là $-\dfrac{\pi}{22}$ (rad).
	}
\end{ex}

\begin{ex}%[1D2N1-3]%[Dự án C đợt 3 - KSCL LeThanhTong-Võ Hoàng Nghĩa]
	Cho dãy số $(u_n)$ được cho bởi hệ thức truy hồi $\begin{cases} u_1 = 5 \\ u_{n+1} = u_n + n, n \ge 2 \end{cases}$. Giá trị của $u_3$ là
	\choice
	{\True $8$}
	{$10$}
	{$7$}
	{$9$}
	\loigiai{
		Ta có $u_1=5$, $u_2=u_1+1=6$, $u_3=u_2+2=8$.
	}
\end{ex}

\begin{ex}%[2H5N2-2]%[Dự án C đợt 3 - KSCL LeThanhTong-Võ Hoàng Nghĩa]
	Trong không gian $Oxy$, cho đường thẳng $d\colon\dfrac{x-1}{4} = \dfrac{-y}{2} = \dfrac{z+2}{-6}$. Vectơ nào dưới đây là một vectơ chỉ phương của $d$?
	\choice
	{$\overrightarrow{u_2} = (2;-1;3)$}
	{$\overrightarrow{u_1} = (4;2;-6)$}
	{$\overrightarrow{u_3} = (-2;1;3)$}
	{$\overrightarrow{u_4} = (1;0;2)$}
	\loigiai{
		Viết lại phương trình đường thẳng $d\colon\dfrac{x-1}{4}=\dfrac{y}{-2}=\dfrac{z+2}{-6}$.\\
		Suy ra một vectơ chỉ phương của đường thẳng $d$ là $\overrightarrow{u}=(4;-2;-6)=-2\cdot(-2;1;3)$.\\
		Suy ra vectơ $\overrightarrow{u_3}=(-2;1;3)$ cũng là một vectơ chỉ phương của đường thẳng $d$.
	}
\end{ex}

\begin{ex}%[2H2H1-2]%[Dự án C đợt 3 - KSCL LeThanhTong-Võ Hoàng Nghĩa]
	Cho tứ diện đều $ABCD$ có cạnh bằng $1$. Giá trị của biểu thức $S=\left|\overrightarrow{AB}+\overrightarrow{AD}+\overrightarrow{AC}\right|$ bằng
	\choice
	{$\dfrac{\sqrt{6}}{2}$}
	{$\sqrt{3}$}
	{$2\sqrt{3}$}
	{\True $\sqrt{6}$}
	\loigiai{
		Ta có $$\left( \overrightarrow{AB}+\overrightarrow{AC}+\overrightarrow{AD} \right)^2=AB^2+AC^2+AD^2+2\cdot\left(\overrightarrow{AB}\cdot\overrightarrow{AC}+\overrightarrow{AB}\cdot\overrightarrow{AD}+\overrightarrow{AD}\cdot\overrightarrow{AC} \right)=1^2+1^2+1^2+6\cdot1\cdot1\cdot\cos 60^\circ.$$
		Vậy $S=\sqrt{6}$.
	}
\end{ex}

\begin{ex}%[1D1H5-6]%[Dự án C đợt 3 - KSCL LeThanhTong-Võ Hoàng Nghĩa]
	Giả sử một vật giao động điều hòa xung quanh vị trí cân bằng theo phương trình
	$x(t)=3\cos\left(4t-\dfrac{\pi}{3}\right)$.
	Ở đây, thời gian $t$ tính bằng giây và $x(t)$ là li độ của vật tại thời điểm $t$ tính bằng centimet. Hãy cho biết trong khoảng thời gian từ $0$ đến $4$ giây, vật đạt li độ bằng $\dfrac{3}{2}$ cm bao nhiêu lần?
	\choice
	{\True $6$}
	{$5$}
	{$3$}
	{$4$}
	\loigiai{
		Theo giả thiết ta có phương trình
		$$3\cos \left( 4t-\dfrac{\pi }{3} \right)=\dfrac{3}{2}\Leftrightarrow \cos \left( 4t-\dfrac{\pi }{3} \right)=\dfrac{1}{2}\Leftrightarrow\cos \left( 4t-\dfrac{\pi }{3} \right)=\cos \dfrac{\pi }{3}\Leftrightarrow\hoac{&4t-\dfrac{\pi}{3}=\dfrac{\pi}{3}+k2\pi\\&4t-\dfrac{\pi}{3}=-\dfrac{\pi}{3}+k2\pi.}$$
		Thu gọn, ta được $t=\dfrac{\pi}{6}+\dfrac{k\pi}{2}$ hoặc $t=\dfrac{k\pi}{2}$, $k\in\mathbb{Z}$.\\
		Xét $0\le \dfrac{\pi}{6}+\dfrac{k\pi}{2}\le 4\Leftrightarrow-\dfrac{1}{3}\le k\le \dfrac{24-\pi}{3\pi}\Rightarrow k\in\{0;1;2\}$.\\
		Xét $0\le \dfrac{k\pi}{2}\le 4\Leftrightarrow-\dfrac{1}{3}\le k\le \dfrac{8}{\pi}\Rightarrow k\in\{0;1;2\}$.\\
		Vậy có $6$ lần thỏa mãn.
	}
\end{ex}

\begin{ex}%[1H8H6-2]%[Dự án C đợt 3 - KSCL LeThanhTong-Võ Hoàng Nghĩa]
	\immini{Cho hình chóp $S.ABCD$ có đáy $ABCD$ là hình vuông tâm $O$, đường thẳng $SA$ vuông góc với mặt phẳng đáy và $OC=\sqrt{3}SA$ (tham khảo hình vẽ). Số đo góc phẳng nhị diện $[S,BD,C]$ bằng
		\choice
		{$120^{\circ}$}
		{\True $150^{\circ}$}
		{$30^{\circ}$}
		{$60^{\circ}$}}{\begin{tikzpicture}[scale=0.7, font=\footnotesize,line join=round, line cap=round, >=stealth]
			\path
			(0,0) coordinate (A)
			++(-140:2) coordinate (B)
			++(0:3.5) coordinate (C)
			($(A)+(C)-(B)$) coordinate (D)
			($(A)+(0,3)$) coordinate (S)
			($(A)!1/2!(C)$) coordinate (O)
			;
			\foreach \i in{B,C,D}{\draw (S)--(\i);};
			\draw (B)--(C)--(D);
			\draw[dashed] (S)--(A)--(B) (C)--(A)--(D)--(B);
			\pic[draw,angle eccentricity=1.8,angle radius=2mm]{right angle=S--A--D};
			\foreach \i/\g in {A/-90,B/-90,C/-90,D/-90,S/90,O/90}
			\fill[black] (\i) circle(1pt)+(\g:3mm)node[scale=1]{$\i$};
		\end{tikzpicture}}
	\loigiai{
		Theo giả thiết ta có $\heva{& SA\perp BD\\
				& AO\perp BD\\
			}\Rightarrow SO\perp BD$.\\
		Mà $OC\perp BD$ suy ra $\left[S,BD,C\right]=\widehat{SOC}$.\\
		Xét $\triangle SAO$ vuông tại $A$, có $OA=OC=\sqrt{3}SA$.\\
		Khi đó $\tan \widehat{SOA}=\dfrac{SA}{AO}=\dfrac{SA}{\sqrt{3}SA}=\dfrac{1}{\sqrt{3}}\Rightarrow \widehat{SOA}=30^\circ$.\\
		Vậy $\left[S,BD,C\right]=\widehat{SOC}=150^\circ$.
	}
\end{ex}

\begin{ex}%[2D4V3-1]%[Dự án C đợt 3 - KSCL LeThanhTong-Võ Hoàng Nghĩa]
	\immini{Cho hàm số $y=f(x)$ có đạo hàm $f'(x)$ liên tục trên đoạn $[0;5]$ và đồ thị hàm số $f'(x)$ trên đoạn $[0;5]$ được cho như hình bên. Mệnh đề nào sau đây đúng?
		\choice
		{$f(0)=f(5)<f(3)$}
		{$f(3)<f(0)=f(5)$}
		{$f(3)<f(0)<f(5)$}
		{\True $f(3)<f(5)<f(0)$}}{\begin{tikzpicture}[scale=0.45,line join=round, line cap=round,>=stealth,thick]
			\tikzset{every node/.style={scale=1}}
			\draw[->] (-2.1,0)--(7.1,0) node[below left] {$x$};
			\draw[->] (0,-6.1)--(0,3.1) node[below left] {$y$};
			\draw (0,0) node [below left] {$O$};
			\foreach \x/\nx in {4.05/3,5/5}
			\draw[thin] (\x,1pt)--(\x,-1pt) node [below] {$\nx$};
			\foreach \y/\ny in {-5/-5,1.05/1}
			\draw[thin] (1pt,\y)--(-1pt,\y) node [left] {$\ny$};
			\draw[dashed,thin](4.05,0)--(4.05,1.05)--(0,1.05);
			\begin{scope}
				\clip (-2,-6) rectangle (7,3);
				\draw[samples=200,domain=0:5,smooth,variable=\x] plot (\x,{-8/45*((\x)^3)+49/45*((\x)^2)+0*(\x)+-5});
			\end{scope}
		\end{tikzpicture}}
	\loigiai{
		Xét đồ thị hàm số $f'(x)$ trên đoạn $[0;5]$
		ta có $f'(x)=0\Leftrightarrow \hoac{& x=a\in (0;3) \\
				& x=5.}$\\
		Khi đó ta có bảng biến thiên của hàm số $y=f(x)$ như sau
		\begin{center}
			\begin{tikzpicture}
				\tkzTabInit[nocadre=false,lgt=2,espcl=3,deltacl=0.9]
				{$x$/0.7,$P'(x)$/0.7,$P(x)$/1.5}{$0$,$a$,$5$}
				\tkzTabLine{,-,0,+,}
				\tkzTabVar{+/$f(0)$,-/$f(a)$,+/$f(5)$}
			\end{tikzpicture}
		\end{center}
		Ta có $a<3<5$ mà hàm số đồng biến trên $(a;5)$ nên $f(a)<f(3)<f(5)$.\\
		Gọi $S_1$ là phần hình phẳng giới hạn bởi các đường $y=f'(x)$, $Ox$, $x=0$, $x=a$.\\
		Gọi $S_2$ là phần hình phẳng giới hạn bởi các đường $y=f'(x)$, $Ox$, $x=0$, $x=5$.\\
		Ta có $\heva{&S_1=\displaystyle\int_0^a |f'(x)|\mathrm{\,d}x= f(0)-f(a)\\&S_2=\displaystyle\int_a^5 |f'(x)|\mathrm{\,d}x=f(5) -f(a).}$\\
		Mà $S_1>S_2\Leftrightarrow f(0)-f(a)>f(5)-f(a)\Leftrightarrow f(0)>f(5)$.\\
		Vậy $f(3)<f(5)<f(0)$.
	}
\end{ex} 
% \Closesolutionfile{ans}
% \Closesolutionfile{ansbook}

\cauds
% \Opensolutionfile{ansbook}[Ansbook/TenFile-TF]%---Nên đặt tên theo bài
% \setcounter{ex}{0}
\begin{ex}%[2D4C2-6]%[Dự án C đợt 3 - KSCL LeThanhTong-Võ Hoàng Nghĩa]
	Trong một cuộc thử tên lửa, Triều Tiên đã cho phóng một quả tên lửa có gắn đầu đạn hạt nhân với vận tốc $v(t) = \dfrac{1}{90\,000\,000}t^3 + \dfrac{1}{500}t + 1$ (m/s) trong đó $t$ đơn vị giây tính từ lúc tên lửa Triều Tiên bắt đầu phóng và dự định sẽ rơi xuống một vùng biển. Đi được $1$ giờ thì bay ngang vùng biển thuộc chủ quyền của Nhật Bản ngay lập tức Rada nhận được tín hiệu và gửi tín hiệu về căn cứ quân đội. Khi nhận được tín hiệu của Rada sau $30$ phút quân đội Nhật Bản đã cho phóng một quả tên lửa tầm trung đã xác định sẵn mục tiêu đi với gia tốc $a(t_1) = \dfrac{1}{4\,500}t_1 + \dfrac{n}{100}$ (m/s$^2$), $n > 0$ trong đó $t_1$ đơn vị giây tính từ lúc tên lửa tầm trung bắt đầu phóng.
	\choiceTF
	{Vận tốc của tên lửa tầm trung được biểu thị dưới hàm $v(t_1) = \dfrac{1}{9\,000}t_1^2 + \dfrac{n}{100}t_1$ (m/s$^2$), $n > 0$}
	{\True Kể từ khi bị Rada phát hiện đến lúc Nhật Bản phóng tên lửa thì quả tên lửa gắn đầu đạn hạt nhân đi được $1913{,}4$ km}
	{\True Sau $15$ phút phóng lên thì tên lửa tầm trung hạ được mục tiêu biết quãng đường nó đi được bằng $\dfrac{1}{2}$ quãng đường tên lửa Triều Tiên đi được trong $15$ phút đó, khi đó giá trị $n > 100$}
	{\True Giả sử hàm $h(t) = \dfrac{-5m}{648}t^2 + \dfrac{500m}{9}t + a$ $(m > 0, a \in \mathbb{R})$ (đơn vị: mét) thể hiện độ cao của quả tên lửa gắn đầu đạn hạt nhân so với mực nước biển. Khi quả tên lửa của Triều Tiên đạt độ cao lớn nhất thì quãng đường nó đi được là $483{,}12$ km}
	\loigiai{
		\begin{itemchoice}
			\itemch Vận tốc của tên lửa tầm trung được biểu thị dưới dạng hàm số như sau $$v(t_1)=\displaystyle\int{a(t_1)\mathrm{\,d}t_1}=\displaystyle\int{\left(\dfrac{1}{4500}t_1+\dfrac{n}{100} \right)\mathrm{\,d}t_1}=\dfrac{1}{9000}t_1^2+\dfrac{n}{100}t_1. \mathrm{(m/s)}$$
			\itemch Đổi $1$ giờ $= 3600$ giây và $1$ giờ $30$ phút $= 5400$ giây.\\
			Kể từ khi bị Rada phát hiện đến lúc Nhật Bản phóng tên lửa thì quả tên lửa gắn đầu đạn hạt nhân đi được là
			$$\displaystyle\int\limits_{3\,600}^{5\,400}{v(t)\mathrm{\,d}t}=\displaystyle\int\limits_{3\,600}^{5\,400}{\left(\dfrac{1}{90\,000\,000}t^3+\dfrac{1}{500}t+1 \right)\mathrm{\,d}t}=1\,913\,400\text{ (m)}=1\,913{,}4\text{ (km)}.$$
			\itemch Quãng đường tên lửa Triều Tiên đi được trong $15$ phút trước khi bị hạ là
			\[{s_{TT}}=\int\limits_{5\,400}^{6\,300}{\left(\dfrac{1}{90\,000\,000}t^3+\dfrac{1}{500}t+1 \right)\mathrm{\,d}t}=2\,025\,292{,}5 \text{ (m)}.\]
			Vì trong 15 phút đó, quãng đường tên lửa tầm trung đi được bằng $\dfrac{1}{2}$ quãng đường tên lửa Triều Tiên đi được, ta có
			\allowdisplaybreaks
			\begin{eqnarray*}
				&& s_{NB}=\dfrac{1}{2}s_{TT} \\
				&\Rightarrow& s_{NB}=1\,012\,646{,}25 \text{ (m)}\\
				&\Rightarrow& \displaystyle\int_0^{900} v(t_1)\mathrm{\,d}t_1 =1\,012\,646{,}25\\
				&\Rightarrow& \displaystyle\int_0^{900} \left( \dfrac{1}{9\,000}t_1^2+\dfrac{n}{100}t_1\right) \mathrm{\,d}t_1 =1\,012\,646{,}25\\
				&\Rightarrow& 27\,000+40\,500\cdot\dfrac{n}{100} =1\,012\,646{,}25\\
				&\Rightarrow& n \approx234{,}4>100.
			\end{eqnarray*}
			\itemch Khi quả tên lửa của Triều Tiên đạt độ cao lớn nhất thì
			\begin{eqnarray*}
				&& h'(t)=0 \\
				&\Leftrightarrow& -\dfrac{5m}{324}t+\dfrac{500m}{9}=0\\
				&\Leftrightarrow& t=3\,600.
			\end{eqnarray*}
			Khi quả tên lửa của Triều Tiên đạt độ cao lớn nhất thì quãng đường nó đi được là
			\[{s_{TT}}=\displaystyle\int\limits_0^{3600}{\left( \dfrac{1}{90\,000\,000}t^3+\dfrac{1}{500}t+1 \right)\mathrm{\,d}t}=483\,120\text{ (m)}=483{,}12 \text{ (km).}\]
		\end{itemchoice}
	}
\end{ex}

\begin{ex}%[2D1V4-3]%[Dự án C đợt 3 - KSCL LeThanhTong-Võ Hoàng Nghĩa]
	Cho hàm số $f(x)=\dfrac{x^{3}}{3}-3x-6\ln(2-x)+1$.
	\choiceTF
	{\True Đạo hàm của hàm số đã cho là $f'(x)=\dfrac{x^{3}-2x^{2}-3x}{x-2}$}
	{\True Hàm số đã cho đồng biến trong khoảng $(-\infty;-1)$}
	{\True Tổng các giá trị cực đại và cực tiểu của đồ thị hàm số bằng $\dfrac{14}{3}-6\ln(6)$}
	{Hàm số $g(x)=\dfrac{f(x)}{x^{2}+2x+2}$ có đường tiệm cận xiên có dạng $y=ax+b$. Khi đó $a+b=\dfrac{1}{3}$}
	\loigiai{
		\begin{itemchoice}
			\itemch Hàm số xác định trên khoảng $(-\infty;2)$ và
			$f'(x)=x^2-3+\dfrac{6}{2-x}=\dfrac{x^3-2x^2-3x}{x-2}$.
			\itemch Giải $f'(x)=0\Leftrightarrow x^3-2x^2-3x=0\Leftrightarrow\hoac{&x=3\\&x=-1\\&x=0.}$\\
			Bảng biến thiên
			\begin{center}
				\begin{tikzpicture}
					\tkzTabInit
					[lgt=2,espcl=3.5] % tùy chọn lgt độ dọc/ espcl độ dài 
					{$x$/1.2, $f’(x)$/1, $f(x)$/2.5} % cột đầu tiên
					{$-\infty$, $-1$,$0$, $2$,$ $} % hàng 1 cột 2
					\tkzTabLine{,+,0,-,0,+,d,} % hàng 2 cột 2
					\tkzTabVar{-/$-\infty$,+/$\frac{11}{3}-6\ln 3$,-/$1-6\ln2$,+D/$+\infty$} % hàng 3 cột 2
				\end{tikzpicture}
			\end{center}
			Hàm số đã cho đồng biến trong khoảng $(-\infty;-1)$.
			\itemch Ta có $y_\text{CĐ}=y(-1)=\dfrac{11}{3}-6\ln 3$; $y_\text{CT}=y(0)=1-6\ln 2$. \\
			Suy ra $y_\text{CĐ}+y_\text{CT}=\dfrac{11}{3}-6\ln 3+1-6\ln 2=\dfrac{14}{3}-6\ln 6$.
			\itemch Xét $g(x)=\dfrac{f(x)}{x^{2}+2x+2}$ có đường tiệm cận xiên có dạng $y=ax+b$.\\
			Ta có $a=\lim\limits_{x\to-\infty}\dfrac{g(x)}{x}=\lim\limits_{x\to-\infty}\dfrac{\dfrac{x^3}{3}-3x-6\ln (2-x)+1}{x^3+2x^2+2x}=\dfrac{1}{3}$.\\ $b=\lim\limits_{x\to-\infty}\left(g(x)-\dfrac{1}{3}x \right)=\lim\limits_{x\to-\infty}\left(\dfrac{\dfrac{x^3}{3}-3x-6\ln (2-x)+1}{x^2+2x+2}-\frac{1}{3}x \right)=-\dfrac{2}{3}$. \\
			Vậy $a+b=\dfrac{1}{3}-\dfrac{2}{3}=-\dfrac{1}{3}$.
		\end{itemchoice}
	}
\end{ex}

\begin{ex}%[2H5C3-4]%[Dự án C đợt 3 - KSCL LeThanhTong-Võ Hoàng Nghĩa]
	Trong một cuộc thi thể thao về môn bắn súng. Các vận động viên phải thực hiện bắn hạ mục tiêu đang di động trên mặt của khối cầu đặc có bán kính bằng $1$ m. Chọn hệ trục tọa độ $(Oxyz)$ trong không gian có gốc $O$ đặt tại vị trí xạ thủ $A$ ngắm bắn, xem mặt phẳng $(Oxy)$ là mặt đất, đơn vị độ dài trên mỗi trục tọa độ là $1$ m. Biết khối cầu có tâm $I(7;24;3)$ và xem đường đi của viên đạn là một đường thẳng.
	\choiceTF
	{\True Vị trí xa nhất để xạ thủ $A$ nhìn thấy và ngắm bắn mục tiêu là $25,2$ m (làm tròn đến hàng phần mười)}
	{\True Biết vận tốc viên đạn là $\dfrac{54}{5}\sqrt{65}$ km/h thì khoảng thời gian ngắn nhất để xạ thủ $A$ bắn trúng mục tiêu chưa tới $1$ giây}
	{\True Để các xạ thủ có thể dễ dàng bắn trúng mục tiêu hơn, ban tổ chức đã quyết định cho mục tiêu di chuyển trên đường tròn lớn nhất của mặt cầu và song song với mặt đất. Khi đó khoảng cách ngắn nhất từ vị trí xạ thủ $A$ ngắm bắn đến mục tiêu là $3\sqrt{65}$ m}
	{\True Xạ thủ $A$ đang ngắm ở vị trí gần mục tiêu nhất. Tại thời điểm tuyển thủ $A$ nổ súng thì mục tiêu đang ở vị trí $M(6;24;3)$ di chuyển với vận tốc $v=\arctan\left(\dfrac{24}{7}\right)$ (m/s) và đi ngược chiều kim đồng hồ. Khi đó xạ thủ $A$ bắn trúng mục tiêu}
	\loigiai{\begin{center}
			\begin{tikzpicture}[line join=round, line cap=round,thick]
				%Gọi điểm
				\path
				(3,10) coordinate (I)
				($(I)-(3,0)$) coordinate (M)
				($(I)-(0.3,0.8)$) coordinate (N)
				($(I)+(2,2.25)$) coordinate (E)
				($(I)-(2.12,2.12)$) coordinate (F)
				($(I)-(-1.3,2.7)$) coordinate (T)
				(3,1) coordinate (H)
				(-5,9) coordinate (K)
				(-6,3) coordinate (A)
				(11,3) coordinate (B)
				(-11,0) coordinate (D)
				($ (B)+(D)-(A) $) coordinate (C)
				(-5,2) coordinate (O)
				;
				%Vẽ elip
				\draw[dashed,thin]
				(M) arc (180:0:3 cm and 0.8 cm)
				;
				%Vẽ đường tròn
				\draw
				(I) circle (3 cm)
				(M) arc (-180:0:3 cm and 0.8 cm)
				;
				%Vẽ mặt phẳng
				\draw (A)--(B)--(C)--(D)--cycle
				;
				%Vẽ hệ trục Oxyz
				\draw[->] (O)--(-8,0.3)node[left]{$ x $};
				\draw[->] (O)--(-1,2)node[below]{$ y $};
				%Nối hình chiếu
				\draw
				(I)--(O)
				(N)--(O)
				(T)--(O)
				(K)--(O)
				;
				\draw[dashed]
				(I)--(K)
				(I)--(H)
				(I)--(T)
				(I)--(E)
				;
				\node at ($ (I)+(1.1,0.05) $) {$ (7,24,3) $};
				\node at ($ (H)+(0.8,-0.35) $) {$ (7,24,0) $};
				\node at ($ (K)+(0.7,0.3) $) {$ (0,0,3) $};
				\foreach \i/\g in {O/-90,I/0,N/-90,F/180,T/0,K/90,H/-90}{\draw[fill=white](\i) circle (1.5pt) ($(\i)+(\g:3.3mm)$) node[scale=1]{$\i$};}
			\end{tikzpicture}
		\end{center}
		\begin{itemchoice}
			\itemch  Điểm xa nhất mà xạ thủ $A$ thấy được là tiếp điểm $B$ của tiếp tuyến kẻ từ $O$ đến mặt cầu.\\
			Ta có $OB^2=OI^2-IB^2=7^2+{{24}^2}+3^2-1^2\Rightarrow OB=\sqrt{633}\approx 25{,}2$ (m).
			\itemch Vì vận tốc không đổi nên khoảng thời gian ngắn nhất để xạ thủ $A$ bắn trúng mục tiêu là khoảng thời gian cho quãng đường từ xạ thủ đến vị trí gần xạ thủ nhất.\\
			Ta có $OC=OI-R=\sqrt{7^2+{{24}^2}+3^2}-1=\sqrt{634}-1$ (m).\\
			$v=\dfrac{54}{5}\sqrt{65}$ (km/h) $=3\sqrt{65}$ (m/s).\\
			Mặt khác, $t=\dfrac{s}{v}=\dfrac{OC}{v}=\dfrac{\sqrt{634}-1}{3\sqrt{65}}\approx 0{,}99969<1$ giây.
			\itemch
			Gọi $H(7;24;0)$ là hình chiếu của $I$ lên mặt phẳng $(Oxy)$. Vì đường tròn lớn nhất của mặt cầu nằm trong mặt phẳng song song với mặt đất nên khoảng cách ngắn nhất là $OD$ với $D$ là một trong hai giao điểm của mặt cầu, mặt phẳng $z=3$ và mặt phẳng $(OIH)$.\\
			Ta có $ID=1$; $OI=\sqrt{634}-1$; $\cos \widehat{DIO}=\cos \widehat{IOH}=\cos (\overrightarrow{OH},\overrightarrow{OI})=\dfrac{\sqrt{7^2+{{24}^2}}}{\sqrt{7^2+{{24}^2}+3^2}}=\dfrac{25}{\sqrt{634}}$.\\
			Suy ra $OD=\sqrt{OI^2+ID^2-2OI\cdot ID\cdot\cos \widehat{DOI}}=\sqrt{634+1-2\sqrt{634}.1.\dfrac{25}{\sqrt{634}}}=3\sqrt{65}$ (m).

			\itemch
			Ta có $\overrightarrow{IM}=(-1;0;0)$, $\overrightarrow{IO}=(-7;-24;-3)\Rightarrow \widehat{MIC}=(\overrightarrow{IM},\,\overrightarrow{IO})=\arccos \left(\dfrac{-1\cdot(-7)}{1\cdot\sqrt{634}} \right)=\arccos \left(\dfrac{7}{\sqrt{634}} \right)$.\\
			Khi đó, thời gian mục tiêu di chuyển từ $M$ đến điểm $C$ là \[{t_{mt}}=\dfrac{1\cdot\arccos \left(\dfrac{7}{\sqrt{634}} \right)}{\arctan \left( \dfrac{24}{7} \right) }\approx 1{,}0016 \text{ (giây)}.\]
			Thời gian viên đạn bay đến $C$ là ${t_{vd}}=\dfrac{OC}{v}=\dfrac{\sqrt{634}-1}{3\sqrt{65}}\approx 0{,}99969$ giây.
		\end{itemchoice}
	}
\end{ex}

\begin{ex}%[0D0V2-9]%[Dự án C đợt 3 - KSCL LeThanhTong-Võ Hoàng Nghĩa]
	Sau khi học kì I năm học 2024-2025, thầy Nghĩa chủ nhiệm lớp 12B5 nhận thấy rằng lớp mình có $60\%$ học sinh có kết quả xuất sắc, $40\%$ học sinh có kết quả loại giỏi, không có học sinh khá và trung bình. Nhưng để nắm bắt chính xác hơn về năng lực tư duy môn toán của từng học sinh nên thầy Nghĩa đã cho học sinh làm bài kiểm tra toán trong $90$ phút. Sau khi chấm bài xong, thầy Nghĩa thấy rằng trong số học sinh loại giỏi có $8$ học sinh từ $9$ điểm toán trở lên và có $75\%$ học sinh xuất sắc trong các học sinh được điểm toán từ 9 trở lên. Biết lớp 12B5 có $40$ học sinh.
	\choiceTF
	{\True Tỉ lệ học sinh có điểm toán từ $9$ trở lên của lớp 12B5 là $80\%$}
	{\True Học sinh xuất sắc kiểm tra môn toán đều lớn hơn hoặc bằng $9$ điểm}
	{\True Những học sinh có điểm toán dưới $9$ điểm đều là học sinh loại giỏi}
	{\True Có $22$ học sinh kết quả xuất sắc có điểm trên $9$ biết rằng tỉ lệ học sinh có điểm toán trên $9$ điểm của học sinh giỏi bằng $37{,}5\%$ và trong số học sinh có điểm bằng $9$ có $50\%$ học sinh xuất sắc}
	\loigiai{Số học sinh xuất sắc là $60\%\cdot40=24$ (học sinh).\\
		Số học sinh giỏi là $40\%\cdot40=16$ (học sinh).
		\begin{itemchoice}
			\itemch Gọi $x$ (học sinh) là số học sinh đạt từ $9$ điểm trở lên trong các học sinh xuất sắc $(0\le x\le 24)$.\\
			Số học sinh đạt từ $9$ điểm trở lên là $x+8$ (học sinh).\\
			Theo đề bài, ta có phương trình $\dfrac{x}{x+8}\cdot100\%=75\%\Leftrightarrow x=24$.\\
			Tỉ lệ học sinh có điểm toán từ $9$ điểm trở lên của lớp 12B5 là $\dfrac{24+8}{40}\cdot100\%=80\%$.
			\itemch Theo câu a ta có số học sinh xuất sắc từ $9$ điểm trở lên là $24$ học sinh và bằng tổng số học sinh xuất sắc.
			\itemch Từ câu a ta có số học sinh dưới $9$ điểm đều là học sinh giỏi và bằng $40-(24+8)=8$ (học sinh).
			\itemch Số học sinh giỏi có điểm trên $9$ là $16\cdot37{,}5\%=6$ (học sinh).\\
			Số học sinh giỏi có điểm bằng $9$ là $8-6=2$ (học sinh).\\
			Do trong số học sinh có điểm bằng $9$ có $50\%$ học sinh xuất sắc nên số học sinh xuất sắc có điểm bằng $9$ là $2$ (bằng số học sinh giỏi có điểm bằng $9$).
		\end{itemchoice}
	}
\end{ex}
% \Closesolutionfile{ansbook}


\caukq
% \Opensolutionfile{ansbt}[Ansbook/TenFile-TLN]%---Nên đặt tên theo bài
% \setcounter{ex}{0}
\begin{ex}%[1H8V7-3]
	Cho khối chóp $S.ABCD$ có đáy là hình thoi cạnh $2$, $\widehat{ABC}=120^\circ$, $SB=2$. Mặt phẳng $(SAD)$ vuông góc với mặt đáy và cạnh bên $SA$ tạo với mặt đáy một góc $60^\circ$. Tính thể tích khối chóp $S.ABCD$.
	\shortans[]{$1$}
	\loigiai{
		\immini{
			Gọi $SH$ là đường cao của $\triangle SAD$. \\
			Ta có $\heva{&(SAD)\perp(ABCD)\\&(SAD)\cap (ABCD)=AD\\&SH\perp AD, SH \subset (SAD)} \Rightarrow SH \perp (ABCD)$.\\
			Khi đó $(SA, (ABCD))=\widehat{SAH}=60^\circ$. \\
			Đặt $x=SH$ thì $AH=\dfrac{x}{\sqrt 3}$. \\
			Xét $\triangle SHB$ vuông tại $H$, ta có
			$$BH=\sqrt{SB^2-SH^2}=\sqrt{4-x^2}.$$
			Xét $\triangle ABH$, có $\widehat{HAB}=60^\circ$ (vì $\widehat{ABC}=120^\circ$).
		}{
			\begin{tikzpicture}[>=stealth,line join=round,line cap=round,font=\footnotesize,scale=.8]
				\path
				(0,0) coordinate (D)
				(-3,-2) coordinate (A)
				(6,0) coordinate (C)
				($(C)+(A)-(D)$) coordinate (B)
				($(D)!.75!(A)$) coordinate (H)
				($(H)+(90:5)$) coordinate (S)
				($(D)!.5!(B)$) coordinate (O)
				%		($(D)!.5!(O)$) coordinate (I)
				%		($(D)!.5!(C)$) coordinate (K)
				%		($(C)!.5!(B)$) coordinate (P)
				;
				\draw (S)--(A)--(B)--(C)--(S)--(B);
				\draw[dashed]
				(S)--(D)--(A)--(C)--(D)--(B) (S)--(H)--(B)
				;
				\foreach \p/\g in {S/90, D/170, A/-90, C/0, B/-90,  O/-90, H/180}
				\draw[fill=black] (\p) circle (1pt) node[shift=(\g:2.5mm)] {$\p$};
			\end{tikzpicture}
		}
		Áp dụng định lí cô sin, ta được:
		\allowdisplaybreaks
		\begin{eqnarray*}
			&& BH^2=AH^2+AB^2-2\cdot AH\cdot AB\cdot \cos \widehat{HAB} \\
			&\Leftrightarrow& 4-x^2=\dfrac{x^2}{3}+4-2\cdot2\cdot\dfrac{x}{\sqrt 3}\cdot\cos 60^\circ \\
			&\Rightarrow& x=\dfrac{\sqrt 3}{2}
		\end{eqnarray*}
		Vậy $V_{S.ABCD}=\dfrac{1}{3}\cdot SH\cdot S_{ABCD}=\dfrac{1}{3}\cdot SH\cdot AB\cdot AD \cdot \sin 60^\circ = \dfrac{1}{3}\cdot\dfrac{\sqrt 3}{2}\cdot 2\cdot 2 \cdot \sin 60^\circ=1$.
	}
\end{ex}

\begin{ex}%[2D4C3-2]%[Dự án C đợt 3 - KSCL LeThanhTong-Võ Hoàng Nghĩa]
	Một nhóm các kĩ sư muốn xây dựng một cây cầu vòm dàn thép với giá đỡ dưới bằng thép cao cấp có hình dáng là một parabol nối từ 2 cột trụ $A$ và $B$ nằm bên dưới cây cầu. Biết hai cột trụ cách nhau $400$ m, khoảng cách từ trụ $A$ đến cây cầu là $50$ m và $AB$ song song với mặt đường.
	% \begin{center}
	% 	\includegraphics[scale=1]{images/KSCL-THPT-LeThanhTong-HCM-NH24-25}
	% \end{center}
	Gắn hệ trục $Oxy$ vào cây cầu với đơn vị trục tọa độ là $10$ m. Giá đỡ dưới bằng thép là đường cong parabol tạo với hai trục tọa độ các hình phẳng có diện tích $S_1$, $S_2$ như hình vẽ bên. Biết rằng $S_2-2S_1=\dfrac{2200}{21}$. Điểm cao nhất của giá đỡ dưới bằng thép cao cấp cách mặt đường cây cầu bao nhiêu mét? (làm tròn đến hàng phần mười)
	\begin{center}
		\begin{tikzpicture}[scale=1, font=\footnotesize, line join=round, line cap=round,>=stealth, xscale=0.2, yscale=0.2]
			%Gán số liệu.
			\def\xmin{-5};\def\ymin{-10};\def\xmax{50};\def\ymax{10};
			%Gán tọa độ.
			\coordinate (O) at (0,0);
			%Trục Oxy.
			\draw[->] (\xmin,0)--(\xmax,0) node[below]{$x$};
			\draw[->] (0,\ymin)--(0,\ymax) node[left]{$y$};
			\fill (O) node[below left]{$O$} circle(1pt);
			%Giới hạn đồ thị.
			\clip ({\xmin-0.1},{\ymin-0.1}) rectangle ({\xmax+0.1},{\ymax+0.1});
			%Vẽ đồ thị.
			\draw[thick,samples=100] plot[domain=-5:50](\x,{-1/35*(\x)^2+8/7*\x-5});
			\draw(40,-5) node[right]{$y=f(x)$};
			\draw [dashed] (40,-5)--(40,0);
			\draw [dashed] (0,-5)--(40,-5) node[midway,sloped,below]{$40$};
			\fill[fill=black](0,-5) node[left]{$-5$} circle(5pt);
			\fill[fill=black](40,-5) node[below left]{$B$} circle(5pt);
			\draw(0,-5) node [below right]{$A$};
			\draw(1.8,-1.5) node{$S_1$};
			\draw(20,2.5) node{$S_2$};
		\end{tikzpicture}
	\end{center}
	\shortans[]{$64{,}3$}
	\loigiai{
		\begin{center}
			\begin{tikzpicture}[scale=1, font=\footnotesize, line join=round, line cap=round,>=stealth, xscale=0.2, yscale=0.2]
				%Gán số liệu.
				\def\xmin{-5};\def\ymin{-10};\def\xmax{50};\def\ymax{10};
				%Gán tọa độ.
				\coordinate (O) at (0,0);
				%Trục Oxy.
				\draw[->] (\xmin,0)--(\xmax,0) node[below]{$x$};
				\draw[->] (0,\ymin)--(0,\ymax) node[left]{$y$};
				\fill (O) node[below left]{$O$} circle(1pt);
				%Giới hạn đồ thị.
				\clip ({\xmin-0.1},{\ymin-0.1}) rectangle ({\xmax+0.1},{\ymax+0.1});
				%Vẽ đồ thị.
				\draw[thick,samples=100] plot[domain=-5:50](\x,{-1/35*(\x)^2+8/7*\x-5});
				\draw(40,-5) node[right]{$y=f(x)$};
				\draw [dashed] (40,-5)--(40,0);
				\draw [dashed] (0,-5)--(40,-5) node[midway,sloped,below]{$40$};
				\fill[fill=black](0,-5) node[left]{$-5$} circle(5pt);
				\fill[fill=black](40,-5) node[below left]{$B$} circle(5pt);
				\fill[fill=black](40,0) node[above]{$C$} circle(5pt);
				\draw(0,-5) node [below right]{$A$};
				\draw(1.8,-1.5) node{$S_1$};
				\draw(38.2,-1.5) node{$S_1$};
				\draw(20,2.5) node{$S_2$};
				\draw(20,-2.5) node{$S_3$};
			\end{tikzpicture}
		\end{center}
		Parabol có dạng $(P)\colon y=ax^2+bx+c$.\\
		Vì $(P)$ đi qua $(0;-5)$ nên $f(0)=-5 \Rightarrow c=-5$.\\
		Và $(P)$ đi qua $(40;-5)$ nên $f(40)=-5 \Rightarrow 1600a+40b+c=-5 \Rightarrow 1600a+40b=0 $. \hfill (1)\\
		Ta có
		\allowdisplaybreaks
		\begin{eqnarray*}
			S_2-2S_1 &=& S_2-(S_{OABC}-S_3)=(S_2+S_3)-S_{OABC}\\
			&=&\displaystyle\int_0^{40} \left( ax^2+bx+c+5\right) \mathrm{\,d}x-5\cdot40\\
			&=& \dfrac{64000}{3}a+800b+40c.
		\end{eqnarray*}
		Suy ra $\dfrac{64000}{3}a+800b +40c= \dfrac{6400}{21}$. \hfill $(2)$\\
		Từ $(1)$ và $(2)$ ta giải được $a=-\dfrac{1}{35}$, $b=\dfrac{8}{7}$. \\
		Suy ra $(P)\colon y=-\dfrac{1}{35}x^2+\dfrac{8}{7}x-5$.\\
		Do đó $(P)$ có đỉnh $I\left( 20;\dfrac{45}{7}\right) $.\\
		Vậy điểm cao nhất của giá đỡ dưới bằng thép cao cấp cách mặt đường cây cầu $10\cdot \dfrac{45}{7} \approx 64{,}3$ m.

	}
\end{ex}

\begin{ex}%[1H8V7-9]
	Cho khối trụ có bán kính là $R$ chiều cao $h$, hai đường tròn đáy có tâm là $O$ và $O'$. Một khối nón có đỉnh trùng với $O'$ và đáy có tâm $(O; 2R)$. Gọi $V_1$ là thể tích phần khối nón nằm bên ngoài khối trụ, $V_2$ là thể tích phần khối trụ nằm bên ngoài khối nón. Tính $\dfrac{V_1}{V_2}$?
	\shortans[]{2}
	\loigiai{
		\immini{
			\begin{itemize}
				\item Thể tích khối trụ chiều cao $h$ bán kính $R$ là
				      $V=\pi R^2 h.$
				\item Thể tích khối nón chiều cao $h$, bán kính $2R$ là % đỉnh $O'$, đường tròn đáy $(O;2R)$ là
				      $V_3=\dfrac{1}{3}\pi (2R)^2h=\dfrac{4}{3}V.$
				\item Thể tích khối nón chiều cao $\dfrac{h}{2}$, bán kính $R$ là
				      $V_4=\dfrac{1}{3}\pi R^2 \dfrac{h}{2}=\dfrac{1}{6}V.$
				\item Thể tích khối trụ chiều cao $\dfrac{h}{2}$, bán kính $R$ là
				      $V_5=\pi R^2 \dfrac{h}{2}=\dfrac{1}{2}V.$
				\item Thể tích hình nón cụt chiều cao $\dfrac{h}{2}$ và bán kính hai đáy lần lượt là $R$ và $2R$ là
				      $V_6=V_3-V_4=\dfrac{4}{3}V-\dfrac{1}{6}V=\dfrac{7}{6}V.$
			\end{itemize}
		}{
			\begin{tikzpicture}[scale=.7, font=\footnotesize, line join=round, line cap=round, >=stealth]
				\def\a{2} \def\b{0.5} \def\h{6}
				\path
				(0,0) coordinate (O)			(O)+(0,\h) coordinate (O')			(180:\a cm and \b cm) coordinate (A)			(A) + (0,\h) coordinate (B)
				(A) + (2*\a,0) coordinate (C)			(C) + (0,\h) coordinate (D)
				($(O)!2!(A)$) coordinate (A') ($(O)!2!(C)$) coordinate (C')
				($(B)!.5!(A)$) coordinate (M) ($(C)!.5!(D)$) coordinate (N)
				;
				\draw
				(O') ellipse (\a cm and \b cm)        %elip trên
				(A) arc (180:360:\a cm and \b cm) %elip dưới liền
				(M) arc (180:360:\a cm and \b cm) %elip dưới liền
				(A') arc (180:360:2*\a cm and 2*\b cm) %elip dưới liền
				(A)--(B) (C)--(D) (A')--(M) (C')--(N)
				;
				\draw[dashed]
				(A) arc (180:0:\a cm and \b cm)    %elip dưới đứt
				(M) arc (180:0:\a cm and \b cm)    %elip dưới đứt
				(A') arc (180:0:2*\a cm and 2*\b cm)    %elip dưới đứt
				(M)--(O')--(N)
				;
				\draw[dashed] (O')--(O)node[midway,left]{$h$}--(C) node[midway,above]{$R$};
				\foreach \p/\r in {O/180,O'/0}		\fill (\p) circle (1pt) node[shift={(\r:3mm)}]{$\p$};
			\end{tikzpicture}
		}
		\begin{itemize}
			\item Thể tích phần khối nón nằm bên ngoài khối trụ là
			      $V_1=V_6-V_5=\dfrac{7}{6}V-\dfrac{1}{2}V=\dfrac{2}{3}V$.
			\item Thể tích khối trụ bên ngoài khối nón là
			      $V_2=V_5-V_4=\dfrac{1}{2}V-\dfrac{1}{6}V=\dfrac{1}{3}V$.
			\item Vậy $\dfrac{V_1}{V_2}=2$.
		\end{itemize}
	}
\end{ex}

\begin{ex}%[2D1V3-2]%[Dự án C đợt 3 - KSCL LeThanhTong-Võ Hoàng Nghĩa]
	Anh Nam có một cái ao với diện tích $50$ m$^2$ để nuôi cá diêu hồng. Vụ vừa qua anh nuôi với mật độ $40$ con/m$^2$ và thu được $3$ tấn cá thành phẩm. Theo kinh nghiệm nuôi cá của mình anh thấy cứ thả giảm đi $8$ con/m$^2$ thì mỗi con cá thành phẩm thu được tăng thêm $0{,}5$ kg. Để tổng năng suất cao nhất thì vụ tới anh Nam nên mua bao nhiêu cá giống để thả? (giả sử không có hao hụt trong quá trình nuôi)
	\shortans[]{$1600$}
	\loigiai{
		Ở vụ trước.
		\begin{itemize}
			\item Số con cá được nuôi là $50\cdot40=2\,000$ con.
			\item Cân nặng trung bình mỗi con cá là $\dfrac{3\,000}{2\,000} =1{,}5$ kg.
		\end{itemize}
		Ở vụ này, giả sử anh Nam giảm $8x$ con cá trên một mét vuông. Khi đó
		\begin{itemize}
			\item Mật độ cá là $40-8x$ (con/m$^2$).
			\item Số con cá được nuôi là $50(40-8x)$.
			\item Khối lượng trung bình mỗi con cá là $1{,}5+0{,}5x$ kg.
			\item Tổng khối lượng (kg) cá thành phẩm là
			      $\begin{aligned}[t]
					      f(x) & =50(40-8x)(1{,}5+0{,}5x)  \\
					           & =-200x^2 + 400x + 3\,000.
				      \end{aligned}$
			\item[] Ta có $f'(x)=-400x+400$ và $f'(x)=0\Leftrightarrow x=1$.\\
			      Bảng biến thiên của hàm số $f(x)$ như sau
			      \begin{center}
				      \begin{tikzpicture}
					      \tkzTabInit[nocadre=false,lgt=1.2,espcl=2.5,deltacl=0.6]
					      {$x$ /0.6,$f'(x)$ /0.6,$f(x)$ /2}
					      {,$1$,}
					      \tkzTabLine{,+,0,-,}
					      \tkzTabVar{-/,+/$3\,200$,-/}
				      \end{tikzpicture}
			      \end{center}
			      Dựa vào bảng biến thiên, ta thấy giá trị lớn nhất của $f(x)$ là $3\,200$ (kg) khi $x=1$.
		\end{itemize}
		Vậy để tổng năng suất của anh Nam ở vụ sau cao nhất thì vụ tới anh Nam số cá giống anh Nam nên mua để thả là $50(40-8\cdot1)=1600$ con cá.
	}
\end{ex}

\begin{ex}%[2H5C1-7]%[Dự án C đợt 3 - KSCL LeThanhTong-Võ Hoàng Nghĩa]
	\immini{
		Một cơ sở sản xuất Kem làm một mô hình Kem ốc quế lớn gồm 2 phần: Phần Kem có dạng hình cầu, phần ốc quế có dạng hình nón (như hình vẽ bên). Chủ cơ sở sản xuất muốn gắn một chiếc đèn Led lớn chiếu thẳng cây kem vào buổi tối, biết rằng chiếc đèn nằm trên mặt phẳng chứa đường tròn $(C)$ là phần tiếp xúc giữa phần Kem và phần ốc quế. Chọn hệ trục tọa độ $Oxy$ trong không gian thỏa mãn phần Kem hình cầu có tâm $I(1;2;3)$, bán kính $R_C=3$ và phần đỉnh của hình nón là điểm $H(0;1;-2)$ đáy là đường tròn có bán kính $R_N = \sqrt{6}$.
	}{
		\begin{tikzpicture}[line join = round, line cap = round, >=stealth, font=\footnotesize, scale=1]
			\tikzset{label style/.style={font=\footnotesize}}
			\def\r{1.5}

			\coordinate (I) at (0,0);
			\coordinate (a) at ($(I)+(-25:\r cm)$);
			\coordinate (b) at ($(I)+(-155:\r cm)$);
			\coordinate (a') at ($(a)!1!-90:(I)$);
			\coordinate (b') at ($(b)!1!-90:(I)$);
			\draw (intersection of  a--a' and b--b') coordinate (H);
			\draw (intersection of  I--H and a--b) coordinate (J);
			\draw[dashed] (a) arc (0:180:1.35cm and 0.35cm);
			\draw (a) arc (0:-180:1.35cm and 0.35cm);
			\draw[fill=black!20!white,opacity=0.5,draw=black] (I) circle (\r cm);
			\draw (a)--(H)--(b);
			\draw[fill=black!50!white, opacity=0.5] (a) arc (0:-180:1.35cm and 0.35cm)--(b)--(H)--(a)--cycle;
		\end{tikzpicture}
		\hspace{0.3cm}
		\begin{tikzpicture}[line join = round, line cap = round, >=stealth, font=\footnotesize, scale=1]
			\tikzset{label style/.style={font=\footnotesize}}
			\def\r{1.5}

			\coordinate[label={right}:{$I$}] (I) at (0,0);
			\coordinate (a) at ($(I)+(-25:\r cm)$);
			\coordinate (b) at ($(I)+(-155:\r cm)$);
			\coordinate (a') at ($(a)!1!-90:(I)$);
			\coordinate (b') at ($(b)!1!-90:(I)$);
			\draw (intersection of  a--a' and b--b') coordinate[label={right}:{$H$}] (H);
			\draw (intersection of  I--H and a--b) coordinate (J);
			\draw[dashed] (a) arc (0:180:1.35cm and 0.35cm);
			\draw (a) arc (0:-180:1.35cm and 0.35cm);
			\draw (I) circle (\r cm);
			\draw (a)--(H)--(b);
			\foreach \x in {I,H,J}
			\draw[fill=black] (\x) circle (1pt);
		\end{tikzpicture}
	}
	\noindent Để tối ưu hóa lượng ánh sáng chiếc đèn có thể chiếu vào cây kem người ta tính toán rằng chiếc đèn Led sẽ phải ở vị trí $M(a;b;2)$, $a\in \mathbb{Z}$ và từ điểm M kẻ được 2 tiếp tuyến với đường tròn $(C)$ sao cho góc giữa 2 tiếp tuyến đó không bé hơn $60^\circ$. Có bao nhiêu vị trí đặt chiếc đèn Led thỏa mãn yêu cầu của chủ cơ sở.
	\shortans{6}
	\loigiai{
	\immini{
		Ta có
		\allowdisplaybreaks
		\begin{eqnarray*}
			&& KI = \sqrt{R_C^2 - R_N^2} = \sqrt{3^2 - (\sqrt{6})^2} = \sqrt{3}.\\
			&& IH = \sqrt{(-1)^2 + (-1)^2 + (-5)^2} = 3\sqrt{3}.\\
			&\Rightarrow& \dfrac{IK}{IH} = \dfrac{1}{3} \\
			&\Rightarrow& \overrightarrow{IK} = \dfrac{1}{3}\overrightarrow{IH}
		\end{eqnarray*}
		Suy ra
		\[
			\heva{
				& x_K - 1 = -\dfrac{1}{3} \\
				& y_K - 2 = -\dfrac{1}{3} \\
				& z_K - 3 = -\dfrac{5}{3}
			}
			\Rightarrow
			\heva{
				&x_K = \dfrac{2}{3} \\
				&y_K = \dfrac{5}{3} \\
				&z_K = \dfrac{4}{3}
			}
			\Rightarrow K\left(\dfrac{2}{3};\dfrac{5}{3};\dfrac{4}{3}\right).
		\]
	}{
		\begin{tikzpicture}[line join = round, line cap = round, >=stealth, font=\footnotesize, scale=1]
			\tikzset{label style/.style={font=\footnotesize}}
			\def\r{1.5}

			\coordinate[label={right}:{$I$}] (I) at (0,0);
			\coordinate (a) at ($(I)+(-25:\r cm)$);
			\coordinate (b) at ($(I)+(-155:\r cm)$);
			\coordinate (a') at ($(a)!1!-90:(I)$);
			\coordinate (b') at ($(b)!1!-90:(I)$);
			\draw (intersection of  a--a' and b--b') coordinate[label={right}:{$H$}] (H);
			\draw (intersection of  I--H and a--b) coordinate[label={right}:{$K$}] (K);

			\coordinate[label={below}:{$B$}] (B) at ($(K)+(-110:1.35cm and 0.35cm)$);
			\coordinate[label={above}:{$A$}] (A) at ($(K)+(120:1.35cm and 0.35cm)$);
			\coordinate[label={left}:{$M$}] (M) at ($(K)!2.5!(b)$);

			\draw[dashed] (a) arc (0:180:1.35cm and 0.35cm);
			\draw (a) arc (0:-180:1.35cm and 0.35cm);
			\draw (I) circle (\r cm);
			\draw (a)--(H)--(b) (B)--(M)--(A) (M)--(b);
			\draw[dashed] (H)--(I) (b)--(K) (A)--(K)--(B);
			\foreach \x in {I,H,K,B,A,M}
			\draw[fill=black] (\x) circle (1pt);
		\end{tikzpicture}
	}
	Mặt phẳng $(MAB)$ đi qua $K$ và có VTPT là $\overrightarrow{IH} = (1; 1; 5)$ nên $(MAB)\colon x + y + 5z - 9 = 0$.\\
	Vì $M \in (MAB) \Rightarrow a + b + 5 \cdot 2 - 9 = 0 \Leftrightarrow b = -a - 1.$\\
	Khi đó
	$$KM = \sqrt{\left(a - \dfrac{2}{3}\right)^2 + \left(b - \dfrac{5}{3}\right)^2 + \left(2 - \dfrac{4}{3}\right)^2} = \sqrt{\left(a - \dfrac{2}{3}\right)^2 + \left(a + \dfrac{8}{3}\right)^2 + \left(2 - \dfrac{4}{3}\right)^2} = \sqrt{2a^2 + 4a + 8}.$$
	Và
	$$\widehat{AKM} = \dfrac{1}{2}\widehat{AKB} = \dfrac{1}{2} \left( 180^{\circ} - \widehat{AMB} \right) = 90^{\circ} - \dfrac{1}{2}\widehat{AMB} \le 90^{\circ} - \dfrac{1}{2} \cdot 60^{\circ} = 60^{\circ}.$$
			Vì $0^{\circ} < \widehat{AKM} \le 60^{\circ}$ nên $\dfrac{1}{2} \leq \cos \widehat{AKM} <1$. \hfill (1)\\
			Mặt khác $\cos \widehat{AKM} = \dfrac{KA}{KM} = \dfrac{\sqrt{6}}{\sqrt{2a^2 + 4a + 8}} = \sqrt{\dfrac{3}{a^2 + 2a + 4}}$. \hfill (2)\\
			Từ (1) và (2) suy ra $\dfrac{1}{2} \le \sqrt{\dfrac{3}{a^2 + 2a + 4}} < 1 \Leftrightarrow \dfrac{1}{4} \le \dfrac{3}{a^2 + 2a + 4} < 1 \Leftrightarrow 3 < a^2 + 2a + 4 \le 12$.\\
			Như vậy
			$$
				\heva{
					&a^2 + 2a - 8 \leq 0 \\
					&a^2 + 2a + 1 > 0
				} \Leftrightarrow
				\heva{
					&-4 \leq a \leq 2 \\
					&a \neq -1.
				}$$
			Vì $a \in \mathbb{Z}$ nên $a \in \{-4; -3; -2; 0; 1; 2\}$.\\
			Vậy có $6$ vị trí đặt đèn thỏa mãn.
		}
\end{ex}

\begin{ex}%[2D5C2-4]%[Dự án C đợt 3 - KSCL LeThanhTong-Võ Hoàng Nghĩa]
	Có hai hộp: hộp $I$ có $5$ quả bóng trắng và $7$ quả bóng đỏ, hộp $II$ có $10$ quả bóng trắng và $15$ quả bóng đỏ, các quả bóng có cùng kích thước và khối lượng. Lấy ngẫu nhiên hai quả bóng từ hộp $I$ bỏ vào hộp $II$. Sau đó, lấy ra ngẫu nhiên một quả bóng từ hộp $II$. Xác suất để quả bóng được lấy ra từ hộp $II$ là quả bóng được chuyển từ hộp $I$ sang, biết rằng quả bóng đó có màu trắng là $\dfrac{a}{b}$ (là phân số tối giản). Tính $a+b$.
	\shortans{14}
	\loigiai{
		Gọi $X$ là biến cố \lq\lq lấy được bóng trắng từ hộp $II$ sau khi đã chuyển $2$ bóng từ hộp $I$ sang\rq\rq.\\
		Gọi $Y$ là biến cố \lq\lq$2$ bóng được chuyển từ hộp $I$ sang hộp $II$\rq\rq.\\
		Ta có công thức Bayes: $\mathrm{P}(Y | X) = \dfrac{\mathrm{P}(X | Y) \cdot \mathrm{P}(Y)}{\mathrm{P}(X)}$. Trong đó:
		\begin{itemize}
			\item $\mathrm{P}(X | Y)$ là xác suất lấy được bóng trắng từ hộp $II$, biết rằng bóng đó được chuyển từ hộp $I$ sang.
			\item $\mathrm{P}(Y)$ là xác suất 2 bóng được chuyển từ hộp $I$ sang hộp $II$.
			\item $\mathrm{P}(X)$ là xác suất lấy được bóng trắng từ hộp $II$ sau khi chuyển $2$ bóng từ hộp $I$ sang.
		\end{itemize}
		Tính $\mathrm{P}(Y)$: Do ta không cần tính xác suất để $2$ bóng được chuyển từ hộp $I$ sang hộp $II$, biến cố này luôn xảy ra. Nên ta có thể bỏ qua yếu tố $\mathrm{P}(Y)$ trong công thức Bayes.\\
		Tính $\mathrm{P}(X)$: Có $3$ trường hợp xảy ra khi chuyển $2$ bóng từ hộp $I$ sang hộp $II$:
		\begin{itemize}
			\item TH1. Chuyển $2$ bóng trắng: Xác suất là $\dfrac{\mathrm{C}_{5}^{2}}{\mathrm{C}_{12}^{2}} = \dfrac{10}{66}$.
			      \begin{itemize}
				      \item Hộp $II$ có $12$ bóng trắng và $15$ bóng đỏ.
				      \item Xác suất lấy được bóng trắng từ hộp $II$ là $\dfrac{12}{27}$.
			      \end{itemize}
			\item TH2. Chuyển $1$ bóng trắng và $1$ bóng đỏ: Xác suất là $\dfrac{\mathrm{C}_{5}^{1} \cdot \mathrm{C}_{7}^{1}}{\mathrm{C}_{12}^{2}} = \dfrac{35}{66}$.
			      \begin{itemize}
				      \item Hộp $II$ có $11$ bóng trắng và $16$ bóng đỏ.
				      \item Xác suất lấy được bóng trắng từ hộp $II$ là $\dfrac{11}{27}$
			      \end{itemize}
			\item TH3. Chuyển $2$ bóng đỏ: Xác suất là $\dfrac{\mathrm{C}_{7}^{2}}{\mathrm{C}_{12}^{2}} = \dfrac{21}{66}$.
			      \begin{itemize}
				      \item Hộp $II$ có $10$ bóng trắng và $17$ bóng đỏ.
				      \item Xác suất lấy được bóng trắng từ hộp $II$ là $\dfrac{10}{27}$
			      \end{itemize}
		\end{itemize}
		Vậy, $\mathrm{P}(X) = \dfrac{10}{66} \cdot \dfrac{12}{27} + \dfrac{35}{66} \cdot \dfrac{11}{27} + \dfrac{21}{66} \cdot \dfrac{10}{27} = \dfrac{120 + 385 + 210}{66 \cdot 27} = \dfrac{715}{1782} = \dfrac{65}{162}$.\\
		Tính $\mathrm{P}(X|Y)$: Ta cần tính xác suất để bóng lấy ra từ hộp $II$ là bóng trắng được chuyển từ hộp $I$ sang.
		\begin{itemize}
			\item TH1. Trường hợp chuyển $2$ bóng trắng: Xác suất lấy được bóng trắng chuyển từ hộp $I$ là $\dfrac{2}{27}$.
			\item TH2. Trường hợp chuyển $1$ bóng trắng và $1$ bóng đỏ: Xác suất lấy được bóng trắng chuyển từ hộp $I$ là $\dfrac{1}{27}$.
			\item TH3. Trường hợp chuyển $2$ bóng đỏ: Xác suất lấy được bóng trắng chuyển từ hộp $I$ là $0$.
		\end{itemize}
		Vậy, $\mathrm{P}(X|Y) = \dfrac{10}{66} \cdot \dfrac{2}{27} + \dfrac{35}{66} \cdot \dfrac{1}{27} + \dfrac{21}{66} \cdot 0 = \dfrac{20 + 35}{66 \cdot 27} = \dfrac{55}{1782} = \dfrac{5}{162}$.\\
		Áp dụng công thức Bayes: $\mathrm{P}(Y|X) = \dfrac{\mathrm{P}(X|Y)}{\mathrm{P}(X)} = \dfrac{\dfrac{5}{162}}{\dfrac{65}{162}} = \dfrac{5}{65} = \dfrac{1}{13}$.\\
		Vậy $\dfrac{a}{b} = \dfrac{1}{13}$, suy ra $a = 1$ và $b = 13$. Suy ra $a + b = 1 + 13 = 14$.
	}
\end{ex}

\Closesolutionfile{ans}
\inputansbox{6,4,3}{ans/LeThanhTong}
% \begin{name}
	{\tenchude}
	{\tendethi}
	{SỞ GDĐT HÀ TĨNH}
	{\thoigian}
\end{name}

\caulc
\Opensolutionfile{ans}[Ans/TT-THPT-SGD-HaTinh-NH24-25]
%   \Opensolutionfile{ansbook}[Ansbook/TT-THPT-SGD-HaTinh-NH24-25-TN]%---Nên đặt tên theo bài
  \setcounter{ex}{0}
 %%%==============Cau_EX1==============%%%
  \begin{ex}%[Dự án C đề thi thử THPT QG 2025]%[Đỗ Chí Tâm]%[2D1B1-1]
 	Hàm số nào dưới đây đồng biến trên khoảng $(-\infty ;+\infty)$?
 	\choice
 	{$y=-x^3-2x+1$}
 	{$y=\dfrac{x-2}{x+1}$}
 	{\True $y=3x^3+3x-2$}
 	{$y=2x^3-5x+1$}
 	\loigiai{
 		Xét hàm số $y=3x^3+3x-2$ có 
 		$y'=9x^2+3>0$, $\forall x \in \mathbb{R}$.\\
 		Vậy hàm số $y=3 x^3+3 x-2$ đồng biến trên $\mathbb{R}$.
 	}
 \end{ex}
 %%%==============End-Cau_EX1==============%%%
 %%%==============Cau_EX2==============%%%
 \begin{ex}%[Dự án C đề thi thử THPT QG 2025]%[Đỗ Chí Tâm]%[2D1B2-1]
 	Cho hàm số $y=f(x)$ có đạo hàm $f'(x)=\left(x^2-4\right)(x+2)(x-3)$ và liên tục trên $\mathbb{R}$. Số điểm cực trị của hàm số đã cho là
 	\choice
 	{$5$}
 	{\True $2$}
 	{$3$}
 	{$1$}
 	\loigiai{
 		Ta có: $f'(x)=0 \Leftrightarrow\left(x^2-4\right)(x+2)(x-3)=0\Leftrightarrow(x+2)^2(x-2)(x-3)=0 \Leftrightarrow\hoac{&x=-2 \\& x=2 \\& x=3.}$\\Với $x=-2$ là nghiệm kép.\\
 		Vậy hàm số đã cho có $2$ cực trị.}
 \end{ex}
  %%%==============End-Cau_EX2==============%%%
 %%%==============Cau_EX3==============%%%
 \begin{ex}%[Dự án C đề thi thử THPT QG 2025]%[Đỗ Chí Tâm]%[2D1Y3-1]
 	Cho hàm số $y=f(x)$ có bảng biến thiên như hình bên. Giá trị lớn nhất của hàm số đã cho trên đoạn $[-2;4]$ bằng
 	\begin{center}
 		\begin{tikzpicture}[font=\normalsize,t style/.style={style=solid}]
 			\tkzTabInit[nocadre=true,lgt=1.2,espcl=2.5,deltacl=0.5]
 			{$x$ /0.75, $y'$/0.75, $y$/2}
 			{$ -\infty $,$ -1 $,$ 1 $,$ 3 $,$ +\infty $}
 			\tkzTabLine{,+,0,-,0,+,0,-, }  % z, t, d;
 			\tkzTabVar{-/$-\infty$,+/$10$,-/$-4$,+/$8$,-/$-\infty$} %+ hoac-
 		\end{tikzpicture}
 	\end{center}
 	\choice
 	{$-1$}
 	{\True $10$}
 	{$1$}
 	{$8$}
 	
 	\loigiai{
 		Từ bảng biến thiên, ta thấy giá trị lớn nhất của hàm số trên đoạn $[-2;4]$ là $10$.}
 \end{ex}
  %%%==============End-Cau_EX3==============%%%
 %%%==============Cau_EX4==============%%%
 \begin{ex}%[Dự án C đề thi thử THPT QG 2025]%[Đỗ Chí Tâm]%[2D1B5-3]
 	\immini{Cho hàm số đa thức bậc bốn $y=f(x)$ có đồ thị như hình vẽ bên. Phương trình $f(x)-1=0$ có bao nhiêu nghiệm thực phân biệt?
 	\choice
 	{\True $3$}
 	{$1$}
 	{$2$}
 	{$4$}}
 	{\begin{tikzpicture}[scale=0.6,>=stealth, font=\footnotesize, line join=round, line cap=round] 
 			\draw[->] (-4,0)--(2,0)node[below]{$x$};
 			\draw[->] (0,-3)--(0,3)node[left]{$y$};
 			\draw(-2.5,-2.5)..controls++(90:1) and++(180:0.3)..(-1.5,2.5)..controls++(0:0.3)and++(180:0.3)..(-0.3,-0.3)..controls++(0:0.5)and++(180:0.4)..(1,1)..controls++(0:0.3)and++(90:1)..(1.5,-2);
 			\draw(0,0)node[below right]{$O$};
 			\draw[dashed]
 			(-1,0)node[below]{$-1$}--(-1,1)--(0,1)node[above right]{$1$}--(1,1)--(1,0)node[below]{$1$};
 			\fill (0,0) circle (1pt);
 			\fill (0,1) circle (1pt);
 			\fill (-1,0) circle (1pt);
 			\fill (1,0) circle (1pt);
 		
 	\end{tikzpicture}}
 	\loigiai{
 		Ta thấy đường thẳng $y=1$ cắt đồ thị hàm số tại $3$ điểm phân biệt.
 		Do đó phương trình $f(x)-1=0$ có $3$ nghiệm phân biệt.}
 \end{ex}
  %%%==============End-Cau_EX4==============%%%
 %%%==============Cau_EX5==============%%%
 \begin{ex}%[Dự án C đề thi thử THPT QG 2025]%[Đỗ Chí Tâm]%[2D1B5-1]
 \immini{Đồ thị hàm số nào sau đây có hình dạng như hình vẽ?
 	\choice
 	{$y=x^3+3 x$}
 	{$y=x^3-3 x$}
 	{\True $y=x^3-3 x^2$}
 	{$y=x^3+3 x^2$}}
 	{\begin{tikzpicture}[scale=0.5,>=stealth, font=\footnotesize, line join=round, line cap=round]
 			\draw[->] (-2,0)--(4,0)node[below]{$x$};
 			\draw[->] (0,-4.7)--(0,3.2)node[left]{$y$};
 			\draw[samples=100,domain=-1.1:3.2] plot (\x,{(\x)^3-3*(\x)^2});
 			\draw(0,0)node[above left]{$O$};
 			\draw[dashed]
 			(1,0)node[above]{$1$}--(1,-2)--(0,-2)node[below right]{$-2$}
 			(2,0)node[above]{$2$}--(2,-4)--(0,-4)node[below right]{$-4$};
 		\end{tikzpicture}}
 	\loigiai{
 		Ta thấy đồ thị hàm số đi qua điểm $(2;-4)$ nên đồ thị cho là đồ thị của hàm số $y=x^3-3x^2$.}
 \end{ex}
  %%%==============End-Cau_EX5==============%%%
 %%%==============Cau_EX6==============%%%
 \begin{ex}%[Dự án C đề thi thử THPT QG 2025]%[Đỗ Chí Tâm]%[1D6H4-3]
 	Tập nghiệm của bất phương trình $\left(\dfrac{1}{2}\right)^x<\dfrac{1}{8}$ là
 	\choice
 	{\True $(3;+\infty)$}
 	{$(-\infty; 3)$}
 	{$[3;+\infty)$}
 	{$(-\infty; 3]$}
 	\loigiai{
 		Ta có: $\left(\dfrac{1}{2}\right)^x<\dfrac{1}{8} \Leftrightarrow\left(\dfrac{1}{2}\right)^x<\left(\dfrac{1}{2}\right)^3 \Leftrightarrow x>3$.}
 \end{ex}
  %%%==============End-Cau_EX6==============%%%
 %%%==============Cau_EX7==============%%%
 \begin{ex}%[Dự án C đề thi thử THPT QG 2025]%[Đỗ Chí Tâm]%[2H3Y1-1]
 	Trong không gian $Oxyz$, cho $\overrightarrow{a}=2 \overrightarrow{i}-3
 	\overrightarrow{j}+\overrightarrow{k}$. Tọa độ của $\overrightarrow{a}$ là
 	\choice
 	{$(-2;1;3)$}
 	{\True $(2;-3;1)$}
 	{$(2;1;3)$}
 	{$(2;1;-3)$}
 	
 	\loigiai{
 		Ta có $\overrightarrow{a}=2 \overrightarrow{i}-3 \overrightarrow{j}+\overrightarrow{k}$. Suy ra 
 		tọa độ của vectơ $\overrightarrow{a}$ là $(2;-3;1)$.
 	}
 \end{ex}
  %%%==============End-Cau_EX7==============%%%
 %%%==============Cau_EX8==============%%%
 \begin{ex}%[Dự án C đề thi thử THPT QG 2025]%[Đỗ Chí Tâm]%[2H3Y1-1]
 	Trong không gian $Oxyz$, cho tam giác $ABC$ với $A(1;3;4)$, $B(2;-1;0)$, $C(3;1;2)$. Tọa độ trọng tâm $G$ của tam giác $ABC$ là
 	\choice
 	{$G\left(3;\dfrac{2}{3};3\right)$}
 	{$G(2;-1;2)$}
 	{\True $G(2;1;2)$}
 	{$G(6;3;6)$}
 	\loigiai{
 		Tọa độ trọng tâm $G$ của tam giác $ABC$ là $\heva{&x_G=\dfrac{x_A+x_B+x_C}{3}=2\\& y_G=\dfrac{y_A+y_B+y_C}{3}=1 \\& z_G=\dfrac{z_A+z_B+z_C}{3}=2.}$\\
 		Tọa độ trọng tâm $G$ của tam giác $ABC$ là $G(2;1;2)$.
 	}
 \end{ex}
  %%%==============End-Cau_EX8==============%%%
 %%%==============Cau_EX9==============%%%
 \begin{ex}%[Dự án C đề thi thử THPT QG 2025]%[Đỗ Chí Tâm]%[2H3B1-2]
 	Trong không gian $Oxyz$, cho $\overrightarrow{a}=(1;-2;2)$, $\overrightarrow{b}=(-1;2;1)$. Giá trị của tích vô hướng $\overrightarrow{a} \cdot \overrightarrow{b}$ bằng
 	\choice
 	{$3$}
 	{\True $-3$}
 	{$2$}
 	{$-2$}
 	\loigiai{
 		Ta có $\overrightarrow{a} \cdot \overrightarrow{b}=1 \cdot(-1)+(-2) \cdot 2+2 \cdot 1=-3$.
 	}
 \end{ex}
  %%%==============End-Cau_EX9==============%%%
 %%%==============Cau_EX10==============%%%
 \begin{ex}%[Dự án C đề thi thử THPT QG 2025]%[Đỗ Chí Tâm]%[1H8H6-1]
 	Cho hình chóp $S \cdot ABCD$ có $ABCD$ là hình vuông cạnh $a$, tam giác $SAD$ đều. Góc giữa hai đường thẳng $BC$ và $SA$ bằng
 	\choice
 	{\True $60^{\circ}$}
 	{$30^{\circ}$}
 	{$90^{\circ}$}
 	{$45^{\circ}$}
 	
 	\loigiai{
 		Vì  $AB\parallel BC \Rightarrow(SA,BC)=(SA,AD)=\widehat{SAD}=60^{\circ}$.
 	}
 \end{ex}
  %%%==============End-Cau_EX10==============%%%
 %%%==============Cau_EX11==============%%%
 \begin{ex}%[Dự án C đề thi thử THPT QG 2025]%[Đỗ Chí Tâm]%[1D5H2-2]
 	Trong tuần lễ bảo vệ môi trường, các học sinh khối $12$ tiến hành thu nhặt vỏ chai nhựa để tái chế. Nhà trường thống kê kết quả thu nhặt vỏ chai của học sinh khối $11$ ở bảng sau:
 \begin{center}
 		\begin{tabular}{|c|c|c|c|c|c|}
 		\hline Số vỏ chai nhựa & $[10{,}5;15{,}5]$ & $[15{,}5;20{,}5]$ & $[20{,}5;25{,}5]$ & $[25{,}5;30{,}5]$ & $[30{,}5;35{,}5]$ \\
 		\hline Số học sinh & $53$ & $82$ & $48$ & $39$ & $18$ \\
 		\hline
 	\end{tabular}
 \end{center}
 	Hãy tìm trung vị của mẫu số liệu ghép nhóm trên.
 	\choice
 	{$19{,}51$}
 	{\True $19{,}59$}
 	{$20{,}1$}
 	{$18{,}3$}
 	 	\loigiai{
 		Ta có $53+82+48+39+18=240$.\\
 		Như vậy nhóm $[15{,}5;20{,}5]$ chứa trung vị.\\
 		Khi đó $C=n_1=53$.\\
 		Trung vị của mẫu số liệu là $M_e=15{,}5+\dfrac{\dfrac{240}{2}-53}{82} \cdot(20{,}5-15{,}5) \approx 19{,}59$.
 	}
 \end{ex}
  %%%==============End-Cau_EX11==============%%%
 %%%==============Cau_EX12==============%%%
 \begin{ex}%[Dự án C đề thi thử THPT QG 2025]%[Đỗ Chí Tâm]%[2D1K4-1]
 \immini{Cho hàm số $y=\dfrac{ax^2+bx+c}{x}$ ($ac \neq 0$) có đồ thị hàm số như hình vẽ. Đường tiệm cận xiên của đồ thị hàm số đã cho là đường thẳng
 	\choice
 	{Đường thẳng $y=x$}
 	{\True Đường thẳng $y=-x$}
 	{Đường thẳng $x=0$}
 	{Đường thẳng $y=2x$}}
 	{\begin{tikzpicture}[scale=0.5,>=stealth, font=\footnotesize, line join=round, line cap=round]
 			\draw[->] (-6,0)--(6,0)node[below]{$x$};
 			\draw[->] (0,-6.5)--(0,6.5)node[right]{$y$};
 			\draw[samples=100,domain=-5:-0.7] plot (\x,{(-(\x)^2-4)/(\x)});
 			\draw[samples=100,domain=0.7:5] plot (\x,{(-(\x)^2-4)/(\x)});
 			\draw[samples=100,domain=-5:5] plot (\x,{-1*(\x)});
 			\draw(0,0)node[below right]{$O$};
 			\draw[dashed]
 			(-2,0)node[below]{$-2$}--(-2,4)--(0,4)node[right]{$4$}
 			(2,0)node[above]{$2$}--(2,-4)--(0,-4)node[left]{$-4$};
 			\end{tikzpicture}}
 	\loigiai{
 		Đồ thị hàm số đi qua các điểm $(2;-4)$, $(-2;4)$ nên $\heva{&\dfrac{4a+2b+c}{2}=-4 \\& \dfrac{4a- b+c}{-2}=4} \Leftrightarrow\hoac{&4a+2b+c=-8 \\& 4a-2b+c=-8.}$\\
 		Ta có $y=\dfrac{ax^2+bx+c}{x}=ax+b+\dfrac{c}{x} \Rightarrow y'=a-\dfrac{c}{x^2}$.\\
 		Mà $x=2$ là cực trị của hàm số nên $y'(2)=0 \Leftrightarrow a-\dfrac{c}{4}=0 \Leftrightarrow 4 a-c=0$.\\
 		Từ $(1)$ và $(2)$ suy ra $a=-1$, $b=0$, $c=-4$.\\
 		Vậy hàm số đã cho là $y=\dfrac{-x^2-4}{x}$\\
 		Vì tiệm cận xiên của đồ thị hàm số đi qua $O(0;0)$ nên nó có dạng $y=mx$ ($m \neq 0$).\\
 		Ta có $m=\lim\limits_{x \rightarrow+\infty} \dfrac{y}{x}=\lim \limits_{x \rightarrow+\infty} \dfrac{-x^2-4}{x^2}=-1$.\\
 		Vậy tiệm cận xiên của đồ thị hàm số là $y=-x$.
 	}
 \end{ex}
 %%%==============HetCau_EX12==============%%%
%  \Closesolutionfile{ans}
%  \Closesolutionfile{ansbook}
 
\cauds
%   \Opensolutionfile{ansbook}[Ansbook/TT-THPT-SGD-HaTinh-NH24-25-TF]%---Nên đặt tên theo bài
%   \setcounter{ex}{0}
 %%%==============Cau_EX1==============%%%
 \begin{ex}%[Dự án C đề thi thử THPT QG 2025]%[Đỗ Chí Tâm]%[2D1K3-6]
 	Một loại thuốc được dùng cho một bệnh nhân và nồng độ thuốc trong máu của bệnh nhân được giám sát bởi bác sĩ. Biết rằng nồng độ thuốc trong máu của bệnh nhân sau khi tiêm vào cơ thể trong $t$ giờ được cho bởi công thức $c(t)=\dfrac{t}{t^2+1}$ (mg/l).
 	\choiceTF
 	{\True  Sau khi tiêm thuốc $2$ giờ thì nồng độ thuốc trong máu của bệnh nhân bằng $0{,}4$ (mg/l)}
 	{Sau khi tiêm thuốc thì nồng độ thuốc trong máu của bệnh nhân có thể vượt quá $0{,}5$ (mg/l)} 
 	{\True Sau khi tiêm thuốc $1$ giờ thì nồng độ thuốc trong máu của bệnh nhân cao nhất}
 	{\True Sau khi tiêm thuốc thì nồng độ thuốc trong máu của bệnh nhân cao nhất bằng $0{,}5$ (mg/l)}
 	\loigiai{
 		\begin{itemchoice}
 			\itemch Sau khi tiêm thuốc $2$ giờ, nồng độ thuốc trong máu bệnh nhân là $c(2)=\dfrac{2}{2^2+1}=0{,}4$ (mg/l).\\
 			Ta có $c'(t)=\dfrac{t^2+1-2 t^2}{\left(t^2+1\right)^2}=\dfrac{1-t^2}{\left(t^2+1\right)^2}$
 			$$c'(t)=0 \Leftrightarrow 1-t^2=0 \Leftrightarrow\hoac{&t=1\, (\text{nhận})\\&t=-1\, (\text{loại}).}$$
 			Bảng biến thiên
 				\begin{center}
 				\begin{tikzpicture}[font=\normalsize,t style/.style={style=solid}]
 					\tkzTabInit[nocadre=true,lgt=1.2,espcl=2.5,deltacl=0.5]
 					{$t$ /0.75, $c'(t)$/0.75, $c(t)$/2}
 					{$ 0$,$ 1 $,$ +\infty $}
 					\tkzTabLine{,+,0,-}  % z, t, d;
 					\tkzTabVar{-/$0$,+/$\tfrac{1}{2}$,-/$0$} %+ hoac-
 				\end{tikzpicture}
 			\end{center}
 			Từ bảng biến thiên ta thấy
 			\itemch Nồng độ thuốc trong máu không thể vượt quá $0{,}5$ (mg/l).
 			\itemch Sau khi tiêm thuốc $1$ giờ thì nồng độ thuốc trong máu bệnh nhân cao nhất.
 			\itemch Sau khi tiêm thuốc thì nồng độ thuốc trong máu của bệnh nhân cao nhất bằng $0{,}5$ (mg/l).
 		\end{itemchoice}	
 	}
 \end{ex}
 %%%==============HetCau_EX1==============%%%
  %%%==============Cau_EX2==============%%%
 \begin{ex}%[Dự án C đề thi thử THPT QG 2025]%[Đỗ Chí Tâm]%[2D1K3-6]
 	\immini{Một hồ nước nhân tạo được xây dựng trong một công viên giải trí. Trong mô hình minh họa, nó được giới hạn bởi các trục tọa độ và đồ thị hàm số $y=f(x)=-0{,}1 x^3+0{,}9 x^2-1{,}5 x+5{,}6$. Đơn vị đo độ dài trên mỗi trục tọa độ là $100$ m.}
 		{\begin{tikzpicture}[scale=0.5,>=stealth, font=\footnotesize, line join=round, line cap=round]
 			\draw[->] (-0.5,0)--(13,0)node[below]{$x$};
 			\draw[->] (0,-0.5)--(0,8)node[right]{$y$};
 			\draw[samples=100,domain=0:8.1] plot (\x,{-0.1*(\x)^3+0.9*(\x)^2-1.5*(\x)+5.6});
 			\fill[pattern = north east lines] (0,0)--plot[domain=0:8](\x,{-0.1*(\x)^3+0.9*(\x)^2-1.5*(\x)+5.6})--(8,0)--cycle; 
 			\draw[samples=100,domain=6:13] plot (\x,{-1.5*(\x)+18});
 			\path (10,3.7)node[rotate=-57]{$y=-1{,}5x+18$};
 			\draw(0,0)node[below left]{$O$};
 		 	\end{tikzpicture}}
 	\choiceTF
 	{Đường dạo ven hồ chạy dọc theo trục $Ox$ dài $600$ m}
 	{\True  Trên đường đi dạo ven hồ chạy dọc theo trục $Ox$, điểm cách gốc $O$ một đoạn $500$ m có khoảng cách theo phương thẳng đứng đến bờ hồ đối diện là lớn nhất}
 	{\True Khoảng cách nhỏ nhất theo phương thẳng đứng từ một điểm trên đường đi dạo ven hồ đến bờ hồ đối diện là $490$ m}
 	{\True  Trong công viên có một con đường chạy dọc theo đồ thị hàm số $y=-1{,}5 x+18$. Người ta dự định xây dựng bên bờ hồ một bến thuyền đạp nước sao cho khoảng cách từ bến thuyền đến con đường này là ngắn nhất. Biết tọa độ của điểm để xây bến thuyền này là $M(a; b)$. Giá trị của $a+5b$ bằng 43}
 	\loigiai{
 		\begin{itemchoice}
 			\itemch Xét phương trình hoành độ giao điểm của đồ thị hàm số $y=f(x)$ và trục $Ox$
 			$$-0{,}1x^3+0{,}9x^2-1{,}5x+5{,}6=0 \Leftrightarrow x=8.$$
 			Như vậy giao điểm của đồ thị hàm số $y=f(x)$ và trục $Ox$ là $A(8;0)$.\\
 			Vậy đường dạo ven hồ chạy dọc theo trục $Ox$ dài $8\cdot 100=800$ m.
 			\itemch Ta có $y=-0{,}1x^3+0{,}9x^2-1{,}5x+5{,}6\Rightarrow y'=-0{,}3x^2+1{,}8x-1{,}5$. \\
 			$y'=0 \Leftrightarrow\hoac{&x=1 \\&x=5.}$\\
 			Bảng biến thiên
 			\begin{center}
 				\begin{tikzpicture}[font=\normalsize,t style/.style={style=solid}]
 					\tkzTabInit[nocadre=true,lgt=1.2,espcl=2.5,deltacl=0.5]
 					{$x$ /0.75, $f'(x)$/0.75, $f(x)$/2}
 					{$ 0 $,$ 1 $,$ 5 $,$ 8 $}
 					\tkzTabLine{,-,0,+,0,-,}  % z, t, d;
 					\tkzTabVar{+/$5{,}6$,-/$4{,}9$,+/$8{,}1$,-/$0$} %+ hoac-
 				\end{tikzpicture}
 			\end{center}
 			Điểm cách $O$ một đoạn $500$ m có khoảng cách theo phương thẳng đứng đến bờ hồ đối diện là lớn nhất và bằng $810$ m.
 			\itemch Khoảng cách nhỏ nhất theo phương thẳng đứng từ một điểm trên đường đi dạo ven hồ đến bờ đối diện là $490$ m.
 			\itemch Gọi $d\colon y=mx+n$ ($a \neq 0$) là tiếp tuyến tại điểm $x=x_0$ của đồ thị hàm số $y=f(x)$ và song song với $y=-1{,}5x+18$.\\
 			Ta có $f'\left(x_0\right)=-0{,}3 x_0^2+1{,}8 x_0-1{,}5$.\\
 			Vì tiếp tuyến của đồ thị hàm số tại $x=x_0$ song song với $y=-1{,}5 x+18$ nên phương trình tiếp tuyến có dạng $y=-1{,}5 x+18$.\\		
 			Hay $-0{,}3 x_0^2+1{,}8 x_0-1{,}5=-1{,}5 \Leftrightarrow\hoac{&x_0=6 \, (\text{thỏa mãn}) \\& x_0=0\, (\text{loại}).}$\\
 			Với $x_0=6$ thì $f\left(x_0\right)=7{,}4$.\\
 			Tọa độ giao điểm của $d$ và đồ thị hàm số $y=f(x)$ là $M(6;7{,}4)$.\\
 			Ta có $\mathrm{d}(M,y=-1{,}5x+18)=\dfrac{|-1{,}5\cdot 6-7{,}4+18|}{\sqrt{(-1{,}5)^2+1^2}} \approx 0{,}89$ (m).\\
 			Khoảng cách ngắn nhất từ bến thuyền đến con đường là $0{,}89$ (m) ứng với điểm $M(6; 7{,}4)$ trên bờ hồ.\\
 			Vậy $a+5b=6+5\cdot 7{,}4=43$.
 		\end{itemchoice}	
 	}
 \end{ex}
 %%%==============HetCau_EX2==============%%%
 %%%==============Cau_EX3==============%%%
 \begin{ex}%[Dự án C đề thi thử THPT QG 2025]%[Đỗ Chí Tâm]%[2H3B1-1]
 	Trong không gian với hệ tọa độ $Oxyz$, cho tam giác $ABC$ với $A(1;0;-2)$, $B(-2;3;4)$, $C(4;-6;1)$
 	\choiceTF
 	{$\overrightarrow{AB}=(3;-3;6)$} 
 	{Hình chiếu vuông góc của $B$ lên trục $Ox$ là $B'(-2;3;0)$}
 	{Tồn tại 1 điểm $M$ thuộc trục hoành sao cho tam giác $MBC$ vuông tại $M$}
 	{\True Nếu $ABDC$ là hình bình hành thì tọa độ điềm $D$ là $(1;-3;7)$}
 	
 	\loigiai{
 		\begin{itemchoice}
 			\itemch $\overrightarrow{AB}=(-3;3;6)$.
 			\itemch Hình chiếu vuông góc của $B$ lên trục $Ox$ là $B'(-2;0;0)$.
 			\itemch Vì $M$ thuộc trục hoành nên $M(m;0;0)$.\\
 			Khi đó $\overrightarrow{MB}=(-2-m;3;4)$, $\overrightarrow{MC}=(4-m;-6;1)$.\\
 			Vì tam giác $MBC$ vuông tại $M$ nên $\overrightarrow{MB} \cdot \overrightarrow{MC}=0$
 			$\Rightarrow(m+2)(m-4)-18+4=0 \Leftrightarrow m=1 \pm \sqrt{23}$.\\
 			Như vậy tồn tại $2$ điểm $M$ thỏa mãn yêu cầu.
 			\itemch Gọi $D(a;b;c)$.\\
 			Vì $ABDC$ là hình bình hành nên $\overrightarrow{AB}=\overrightarrow{CD} \Rightarrow\heva{&a-4=-3\\& b+6=3\\& c-1=6} \Rightarrow\heva{&a=1 \\& b=-3 \\&c=7.}$\\
 			Vậy $D(1;-3;-5)$.
 		\end{itemchoice}	
 	}
 \end{ex}
 %%%==============HetCau_EX3==============%%%
 %%%==============Cau_EX4==============%%%
 \begin{ex}%[Dự án C đề thi thử THPT QG 2025]%[Đỗ Chí Tâm]%[1H8H5-3]
 	Cho lăng trụ đứng $ABC.A'B'C'$ có $AC=a$, $BC=2a$, $\widehat{ACB}=120^{\circ}$ có thể tích $V$. Gọi $M$ là trung điểm của $BB'$. Khi đó
 	\choiceTF
 	{Góc phẳng nhị diện $\left[A,CC',B\right]=60^{\circ}$} 
 	{\True Biết khoảng cách giữa hai mặt đáy lăng trụ bằng $2a$. Khi đó $V=a^3\sqrt{3}$}
 	{\True  $V_{M.ABC}=\dfrac{1}{6} V$}
 	{\True $d\left(C',\left(ABB'A'\right)\right)=\dfrac{a\sqrt{21}}{7}$}
 	\loigiai{
 		\immini{
 			\begin{itemchoice}
 				\itemch  Ta có
 				$\heva{&AC \perp CC'\\&BC \perp CC'} \Rightarrow\left[A,BB',C\right]=\widehat{ACB}=120^{\circ}$.
 				\itemch Ta có
 				$S_{ABC}=\dfrac{1}{2} AC \cdot BC \cdot \sin \widehat{ACB}=\dfrac{1}{2} \cdot a \cdot a \cdot \sin 120^{\circ}=\dfrac{a^2\sqrt{3}}{2}$.\\
 				Thể tích của khối lăng trụ là
 				$$V_{ABC.A'B'C'}=S_{ABC} \cdot AA'=\dfrac{a^2 \sqrt{3}}{2} \cdot 2a=a^3 \sqrt{3}.$$
 				\itemch  Ta có $V_{M.ABC}=\dfrac{1}{3} MB \cdot S_{ABC}=\dfrac{1}{3} \cdot \dfrac{1}{2} BB' \cdot S_{ABC}=\dfrac{1}{6}V$
 				\itemch Kẻ $CH\perp AB$ tại $H$.
 				Mà $AA' \perp CH \Rightarrow  CH \perp\left(ABB'A'\right) \Rightarrow \mathrm{d}\left(C',\left(ABB'A'\right)\right)=CH$.\\
 				Ta có: $AB=\sqrt{AC^2+BC^2-2AC \cdot BC \cos \widehat{ACB}}=\sqrt{a^2+4a^2-2 \cdot a \cdot 2a \cos 120^{\circ}}=a\sqrt{7}$.\\
 				$S_{ABC}=\dfrac{1}{2} CH \cdot AB \Rightarrow CH=\dfrac{2 S_{ABC}}{AB}=\dfrac{2 \cdot \dfrac{a^2 \sqrt{3}}{2}}{a \sqrt{7}}=\dfrac{a \sqrt{3}}{\sqrt{7}}=\dfrac{a \sqrt{21}}{7}$.
 			\end{itemchoice}
 		 }{
    \begin{tikzpicture}[scale=0.8]
  %%\draw[gray!20] (-3,-2) grid (6,5);
  \path (0,0) coordinate (A)  
  (2,-2) coordinate (B)
  (5,0) coordinate (C)
  (0,5) coordinate (A')
  (2,3) coordinate (B')
  ($(B)!0.5!(B')$) coordinate (M)
  (5,5) coordinate (C');
  \draw[dashed] (A)--(C);
  \draw (A)--(B)--(C)--(C')--(B')--(A')--(A) (B)--(B') (A')--(C');
  \foreach \p/\q in {A/180, B/-90, C/-90, A'/90, B'/70,C'/90, M/180}
  \fill[blue] (\p) circle(2pt) node[shift={(\q:3mm)}]{\bfseries $\p$};
   \end{tikzpicture}
 	      }
 		}
 	\end{ex}
 %%%==============HetCau_EX4==============%%%
%  \Closesolutionfile{ansbook}
 

\caukq
% \Opensolutionfile{ansbt}[Ansbook/TT-THPT-SGD-HaTinh-NH24-25-TLN]%---Nên đặt tên theo bài
% \setcounter{ex}{0}
 %%%==============Cau_EX1==============%%%
\begin{ex}%[Dự án C đề thi thử THPT QG 2025]%[Đỗ Chí Tâm]%[1D1V4-8]
	Cho đồ thị hàm số $f(x)=2 \sin x$ như hình vẽ bên. Tính diện tích tam giác $ABC$.
	\begin{center}
		\begin{tikzpicture}[thick,>=stealth,x=1cm,y=1cm,scale=0.9] 
		\draw[->] (0,0) -- (10,0) node[below] {\small $x$};
		\draw[->] (0,-2.3) -- (0,2.3) node[right] {\small $y$};
		\draw[thick,samples=100,domain=0:10] plot(\x,{2*sin((\x)*180/pi)});
		\draw[dashed] (0,2)--(10,2);
		\draw[dashed] (0,-2)--(10,-2);
		\path (1.57,2)node[above]{$B$} (7.85,2)node[above]{$C$} (0,-2)node[left]{$A$} ;
		\draw[red] (1.57,2)--(7.85,2)--(0,-2)--cycle;
		
	\end{tikzpicture}
\end{center}
	\shortans[]{12{,}6}
	\loigiai{
		Ta có $-1\leq \sin x \leq 1 \Rightarrow-2 \leq 2 \sin x \leq 2 \Rightarrow-2 \leq f(x) \leq 2$.\\
		Xét $f(x)=2 \Leftrightarrow 2 \sin x=2 \Leftrightarrow \sin x=1 \Leftrightarrow x=\dfrac{\pi}{2}+k 2 \pi$ ($k \in \mathbb{Z}$).\\
		Với $x>0$ thì $\dfrac{\pi}{2}+k 2 \pi>0 \Leftrightarrow k>-\dfrac{1}{4}$.\\
		Mà $k \in \mathbb{Z} \Rightarrow k \in\{0;1;2;\ldots\}$.\\
		Với $k=0$ ta có $x_B=\dfrac{\pi}{2} \Rightarrow B\left(\dfrac{\pi}{2};2\right)$.\\
		Với $k=1$ ta có $x_C=\dfrac{5 \pi}{2} \Rightarrow C\left(\dfrac{5 \pi}{2};2\right)$.\\
		Ta có $A(0;-2)$.\\
		Suy ra $\mathrm{d}(A,BC)=2OA=2\cdot 2=4$.\\
		$BC=\sqrt{(2 \pi)^2+0^2}=2 \pi$.\\
		Vậy $S_{ABC}=\dfrac{1}{2} \mathrm{d}(A,BC) \cdot BC=\dfrac{1}{2} \cdot 4 \cdot 2 \pi=4 \pi \approx 12{,}6$.
	}
\end{ex}
%%%==============HetCau_EX1==============%%%
%%%==============Cau_EX2==============%%%
\begin{ex}%[Dự án C đề thi thử THPT QG 2025]%[Đỗ Chí Tâm]%[0D0V2-9]
	Trong đề kiểm tra $15$ phút môn Toán có $20$ câu trắc nghiệm. Mỗi câu trắc nghiệm có $4$ phương án trả lời, trong đó chỉ có một phương án trả lời đúng. An giải chắc chắn đúng $10$ câu, $10$ câu còn lại lựa chọn ngẫu nhiên đáp án. Biết rằng mỗi câu trả lời đúng được $0{,}5$ điểm, trả lời sai không bị trừ điểm. Xác suất để An đạt được đúng $8$ điểm là $p$. Khi đó, $100p$ bằng
	\shortans[]{1{,}6}
	\loigiai{
		Vì An chắc chắn giải đúng $10$ câu nên chắc chắn An đã được $5$ điểm.\\
		Để An được $8$ điểm thì An cần làm đúng thêm $6$ câu nữa.\\
		Chọn $6$ câu trong số $10$ câu còn lại có $\mathrm{C}_{10}^6$ cách.\\
		Mỗi câu có $4$ phương án trả lời nên xác suất đúng $1$ câu là $0{,}25$, xác suất sai $1$ câu là $0{,}75$.\\
		Vậy xác suất để An được $8$ điểm là $\mathrm{C}_{10}^6 \cdot 0{,}25^6 \cdot 0{,}75^4 \approx 0{,}016$.\\
		Vậy $100p=1{,}6$.	
		
	}
\end{ex}
%%%==============HetCau_EX3==============%%%
%%%==============Cau_EX3==============%%%
\begin{ex}%[Dự án C đề thi thử THPT QG 2025]%[Đỗ Chí Tâm]%[2D4V3-2]
	\immini[thm]{Một công ty có ý định thiết kế một logo hình vuông có độ dài nửa đường chéo bằng $4$. Biểu tượng $4$ chiếc lá (được tô màu) được tạo thành bởi các đường cong đối xứng với nhau qua tâm của hình vuông và qua các đường chéo. Một trong số các đường cong ở nửa bên phải của logo là một phần của đồ thị hàm số bậc ba dạng $y = ax^3 + bx^2 - x$ với hệ số $a<0$. Để kỷ niệm ngày thành lập $2/3$, công ty thiết kế để tỉ số diện tích được tô màu so với phần không được tô màu bằng $\dfrac{2}{3}$. Tính $2a + 2b$.
	}
	{\begin{tikzpicture}[scale=0.7, font=\footnotesize, line join=round, line cap=round,>=stealth]
			\foreach \i in {0,1,2,3}{
				\fill[pattern = north east lines,draw,smooth,rotate=90*\i] plot [domain=0:3](\x,{-(\x)*(\x-3)^2/8})--plot [domain=3:0](\x,{(\x)*(\x-3)^2/8})--cycle;
				\draw (0,0)--(90*\i:3)--(90*\i+90:3);
			}
		\end{tikzpicture}
	}
	\shortans[]{$0{,}8$}
	\loigiai
	{
		Ta có nửa đường chéo hình vuông có độ dài là $4$, cạnh hình vuông sẽ là $4\sqrt{2}$ và diện tích hình vuông là $32$, khi đó ta có được diện tích phần tô màu là $\dfrac{64}{5}$.\\
		Gọi $f(x) = ax^3 + bx^2 - x$ là hàm số bậc ba biểu diễn đường cong trên logo.\\
		Ta có $x = 4$ là nghiệm của phương trình nên $64a + 16b - 4 = 0 \Leftrightarrow 4a+b=1$.\quad (1)\\
		Do đó phương trình $f(x) = 0$ sẽ có các nghiệm là $0$, $4$.\\
		Khi đó diện tích hình phẳng $S$ giới hạn bởi các đường $y = f(x)$, trục $Ox$, đường thẳng $x = 0$, $x = 4$ là
		\begin{eqnarray*}
			S&= & \displaystyle \int \limits_{0}^{4} |ax^3+bx^2-x|\mathrm{\,d}x = -\displaystyle \int \limits_{0}^{4} (ax^3+bx^2-x)\mathrm{\,d}x\\
			&= & -\left(\dfrac{ax^4}{4}+\dfrac{bx^3}{3}-\dfrac{x^2}{2}\right)\Bigg|_0^4 = -64a-\dfrac{64}{3}b+8.
		\end{eqnarray*}
		Mà $S = \dfrac{1}{8}\cdot \dfrac{64}{5} = \dfrac{8}{5}$ nên $-64a-\dfrac{64}{3}b+8 = \dfrac{8}{5} \Leftrightarrow 64a+\dfrac{64}{3}b=\dfrac{32}{5}$.\quad (2)\\
		Từ $(1)$ và $(2)$, ta có $\heva{&4a+b=1\\&64a+\dfrac{64}{3}b=\dfrac{32}{5}} \Leftrightarrow \heva{&a = -\dfrac{1}{20}\\&b=\dfrac{9}{20}} \Rightarrow 2a+2b = \dfrac{4}{5}$.
	}
\end{ex}
%%%==============HetCau_EX4==============%%%
%%%==============Cau_EX4==============%%%
\begin{ex}%[Dự án C đề thi thử THPT QG 2025]%[Đỗ Chí Tâm]%[2D1G3-6]
	Giả sử tỉ lệ sinh của tỉnh $A$ tuân theo quy luật logistic được mô hình hóa bằng hàm số $f(t)=\dfrac{200}{1+4 \mathrm{e}^{-t}}$, $t \geq 0$, $t \in \mathbb{N}$, trong đó thời gian $t$ được tính bằng tháng. Khi đó đạo hàm $f'(t)$ sẽ biểu thị tốc độ tăng dân số của tỉnh $A$. Hỏi sau bao nhiêu tháng tốc độ tăng trưởng của dân số tỉnh $A$ là lớn nhất?
	\shortans[]{2}
	\loigiai{
		Ta có: $f'(t)=200 \cdot \dfrac{4 \mathrm{e}^{-t}}{\left(1+4 \mathrm{e}^{-t}\right)^2}=\dfrac{800 \mathrm{e}^{-t}}{\left(1+4 \mathrm{e}^{-t}\right)^2}$
		\begin{eqnarray*}
			f^{\prime \prime}(t)&=&800 \cdot \dfrac{-\mathrm{e}^{-t}\left(1+4 \mathrm{e}^{-t}\right)^2-\mathrm{e}^{-t} \cdot 2 \cdot\left(-4 \mathrm{e}^{-t}\right)\left(1+4 \mathrm{e}^{-t}\right)}{\left(1+4 \mathrm{e}^{-t}\right)^4}\\
			&=&800 \cdot \dfrac{4\left(\mathrm{e}^{-t}\right)^2-\mathrm{e}^{-t}}{\left(1+4 \mathrm{e}^{-t}\right)^2}\\
			&=&\dfrac{800 \mathrm{e}^{-t}}{\left(1+4 \mathrm{e}^{-t}\right)^3}\left(4 \mathrm{e}^{-t}-1\right).
		\end{eqnarray*}
		$$f^{\prime \prime}(t)=0 \Leftrightarrow 4 \mathrm{e}^{-t}-1=0 \Leftrightarrow \mathrm{e}^{-t}=\dfrac{1}{4} \Leftrightarrow t=\ln 4.$$
		Bảng biến thiên
		\begin{center}
			\begin{tikzpicture}[font=\normalsize,t style/.style={style=solid}]
				\tkzTabInit[nocadre=true,lgt=1.2,espcl=2.5,deltacl=0.5]
				{$t$ /0.75, $f^{\prime\prime}(t)$/0.75, $f'(t)$/2}
				{$ 0 $,$\ln4 $,$+\infty $}
				\tkzTabLine{,+,0,-,}  % z, t, d;
				\tkzTabVar{-/$32$,+/$50$,-/$0$} %+ hoac-
			\end{tikzpicture}
		\end{center}
		Từ bảng biến thiên ta thấy giá trị lớn nhất của $f'(t)$ là $50$ xảy ra tại $t=\ln 4$.\\
		Vậy sau khoảng $2$ tháng thì tốc độ tăng trưởng dân số của tỉnh $A$ là lớn nhất.
		
	}
\end{ex}
%%%==============HetCau_EX4==============%%%
%%%==============Cau_EX5==============%%%
\begin{ex}%[Dự án C đề thi thử THPT QG 2025]%[Đỗ Chí Tâm]%[2D1G3-6]
	\immini{Một máy bay trình diễn có đường bay gắn với hệ trục $Oxy$ được mô phỏng như hình vẽ, trục $Ox$ gắn với mặt đất. Đường bay có dạng là một phần của đồ thị của hàm phân thức bậc hai trên bậc nhất $y=f(x)$ có đường tiệm cận đứng $x=2$. Điểm $G$ là giao điểm của đường tiệm cận xiên của đồ thị hàm số $y=f(x)$ và trục $Ox$ được gọi là điểm giới hạn. Biết rằng máy bay xuất phát tại vị trí $A$ cách gốc tọa độ $O$ một khoảng $2{,}5$ đơn vị và máy bay khi ở vị trí cao nhất cách điểm xuất phát $1{,}5$ đơn vị theo phương song song với trục $Ox$ và cách mặt đất $4{,}5$ đơn vị. Vị trí máy bay tiếp đất cách điểm giới hạn một khoảng bằng bao nhiêu?}
	{\begin{tikzpicture}[scale=0.5,>=stealth, font=\footnotesize, line join=round, line cap=round]
			\draw[->] (-0.5,0)--(11,0)node[below]{$x$};
			\draw[->] (0,-1.8)--(0,9)node[right]{$y$};
			\draw[samples=100,domain=2.4:10.3] plot (\x,{(-(\x)^2+12.5*(\x)-25)/((\x)-2)});
			\draw[samples=100,domain=1.6:10.6] plot (\x,{(-(\x)+10.5)});
			\draw(2,-1.8)--(2,9);
			\draw(0,0)node[below right]{$O$};
			\draw (4,4.5)circle (1.3pt)node[above]{$B$}(2.5,0)circle (1.3pt)node[below right]{$A$}(10,0)circle (1.3pt)node[below]{$C$}(10.5,0)circle (1.3pt)node[above]{$G$};%--(-2,4)--(0,4)node[right]{$4$}
		%	(2,0)node[above]{$2$}--(2,-4)--(0,-4)node[left]{$-4$};
	\end{tikzpicture}}
	\shortans[]{0{,}5}
	\loigiai{
		Hàm số bậc hai trên bậc nhất có dạng $y=f(x)=\dfrac{ax^2+bx+c}{x-2}$.\\
		Theo giả thiết ta có $A(2{,}5;0)$, $B(4; 4{,}5)$.\\
		Vì $A$, $B$ thuộc đồ thị hàm số nên $\heva{&6{,}25a+2{,}5b+c=0 \\& 16a+4b+c=9.}$\\
		Ta có
		$ f'(x)=\dfrac{ax^2-4ax-2b-c}{(x-2)^2}$.\\
		$f'\left(x_B\right)=0 \Rightarrow \dfrac{a \cdot 4^2-4a \cdot 4-2b-c}{(4-2)^2}=0 \Rightarrow 2b+c=0$.\\
		Từ $(1)$ và $(2)$ suy ra $a=-1$, $b=12{,}5$, $c=-25$.\\
		Khi đó $f(x)=\dfrac{-x^2+12{,}5x-25}{x-2}=-x+10{,}5-\dfrac{4}{x-2}$.\\
		\[\lim\limits_{x \rightarrow+\infty}[f(x)-(-x+10{,}5)]=\lim \limits_{x \rightarrow+\infty}\left(-x+10{,}5-\dfrac{4}{x-2}+x-10,5\right)=\lim \limits_{x \rightarrow+\infty} \dfrac{-4}{x-2}=0.\]		
		Vậy $y=-x+10{,}5$ là tiệm cận xiên của đồ thị hàm số.\\
		Tọa độ giao điểm $G$ là nghiệm của hệ phương trình $\heva{&y=-x+10{,}5 \\& y=0} \Leftrightarrow\heva{&x=10{,}5 \\& y=0} \Rightarrow G(10{,}5 0)$.\\
		Tương tự ta tìm được $A(2{,}5;0)$, $C(10;0)$.\\
		$CG=x_G-x_C=10{,}5-10=0{,}5$.\\
		Vậy vị trí máy bay tiếp đất cách điểm giới hạn một khoảng $0{,}5$ đơn vị.	
	}
\end{ex}
%%%==============HetCau_EX5==============%%%
%%%==============Cau_EX6==============%%%
\begin{ex}%[Dự án C đề thi thử THPT QG 2025]%[Đỗ Chí Tâm]%[0H5C2-6]
	\immini{Có ba lực cùng tác động vào một cái bàn như hình vẽ. Trong đó hai lực $\overrightarrow{F}_1, \overrightarrow{F}_2$ có giá nằm trên mặt phẳng chứa mặt bàn, tạo với nhau một góc $110^{\circ}$ và có độ lớn lần lượt là $9$ N, $4$ N, lực $\overrightarrow{F}_3$ vuông góc với mặt bàn và có độ lớn $7$ N. Độ lớn hợp lực của ba lực trên là $a$ (N), tìm giá trị của $a$ (\textit{kết quả làm tròn đến hàng đơn vị})}{\includegraphics[scale=0.4]{images/TT-THPT-SGD-HaTinh-NH24-25}}
	\shortans{11}
	\loigiai{
		Ta có \begin{eqnarray*}
			\left|\overrightarrow{F}_1+\overrightarrow{F}_2+\overrightarrow{F}_3\right|&=&\sqrt{\left(\overrightarrow{F}_1+\overrightarrow{F}_2+\overrightarrow{F}_3\right)} \\
			& =&\sqrt{F_1^2+F_2^2+F_3^2+2 \overrightarrow{F}_1 \overrightarrow{F}_2+2 \overrightarrow{F}_2 \overrightarrow{F}_3+2 \overrightarrow{F}_3 \overrightarrow{F}_1} \\
			& =&\sqrt{F_1^2+F_2^2+F_3^2+2 F_1 F_2 \cos 110^{\circ}+2 F_2 F_3 \cos 90^{\circ}+2 F_1 F_3 \cos 90^{\circ}} \\
			& =&\sqrt{9^2+4^2+7^2+2 \cdot 9 \cdot 4 \cdot \cos 10^{\circ}+2 \cdot 4 \cdot 7 \cdot 0+2 \cdot 9 \cdot 7 \cdot 0} \approx 11\,(\text{N}).
		\end{eqnarray*} 
}
\end{ex}
 \Closesolutionfile{ans}
\inputansbox{6,4,3}{ans/TT-THPT-SGD-HaTinh-NH24-25}
% \begin{name}
	{\tenchude}
	{\tendethi}
	{SỞ GDĐT TUYÊN QUANG}
	{\thoigian}
\end{name}

\caulc
\Opensolutionfile{ans}[Ans/TT-THPT-SGD-TuyenQuang-NH24-25]
%   \Opensolutionfile{ansbook}[Ansbook/TT-THPT-SGD-TuyenQuang-NH24-25-TN]%---Nên đặt tên theo bài
  \setcounter{ex}{0}
%%%==============Cau_EX1==============%%%
\begin{ex}%[Dự án C THPTQG 2025]%[ĐỖ CHÍ TÂM]%[2H1B1-1]
	Cho hình lập phương $ABCD.EFGH$ cạnh bằng $a$. Giá trị của $\vec{AC}\cdot\vec{EG}$ bằng
	\choice
	{$-a^2$}
	{$a^2$}
	{$-2a^2$}
	{\True$2a^2$}
	\loigiai{
		Ta có hình lập phương $ABCD.EFGH\Rightarrow \vec{AC}=\vec{EG}\Rightarrow \vec{AC}\cdot\vec{EG}=\vec{AC}\cdot\vec{AC}=AC^2$.\\
		Vì tam giác $ABC$ vuông tại $B$ nên $AC^2=BA^2+BC^2=a^2+a^2=2a^2$.
	}
\end{ex}
%%%==============HetCau_EX1==============%%%
%%%==============Cau_EX2==============%%%
\begin{ex}%[Dự án C THPTQG 2025]%[ĐỖ CHÍ TÂM]%[1D6N4-2]
	Tập nghiệm $S$ của phương trình $2^{x^2+7x+10}=1$ là 
	\choice
	{$S=\{2; 5\}$}
	{\True $\{-5; -2\}$}
	{$S=\{-5; 2\}$}
	{S=$\left\{\dfrac{-7-\sqrt{13}}{2}; \dfrac{-7+\sqrt{13}}{2}\right\}$}
	\loigiai{
		Ta có $2^{x^2+7x+10}=1\Rightarrow 2^{x^2+7x+10}=2^0\Rightarrow x^2+7x+10=0\Rightarrow x=-5\text{ hoặc }x=-2$
	}
\end{ex}
%%%==============HetCau_EX2==============%%%
%%%==============Cau_EX3==============%%%
\begin{ex} %[Dự án C THPTQG 2025]%[ĐỖ CHÍ TÂM]%[2D1Y2-2]
	\immini[thm]{
	Cho hàm số $y=f(x)$ có bảng biến thiên như hình vẽ. Tọa độ điểm cực đại của đồ thị hàm số là
	\choice
	{$(0; 2)$}
	{\True $(-2; 0)$}
	{$(0; -2)$}
	{$(2; 0)$}
	}{
	 \begin{tikzpicture}
	 	\tkzTabInit[nocadre=true,lgt=1.2,espcl=2.5,deltacl=0.6]
	 	{$x$ /0.6,$f'(x)$ /0.6,$f(x)$ /2}
	 	{$-\infty$,$0$,$3$,$+\infty$}
	 	\tkzTabLine{,+,$0$,-,$0$,+,}
	 	\tkzTabVar{-/$-\infty$, +/$2$,-/$-4$,+/$+\infty$}
	 \end{tikzpicture}}
	\loigiai{	
		Ta thấy tại điểm $x = 0$ thì $f'(x)$ đổi dấu từ dương sang âm. Do đó điểm cực đại của đồ thị hàm số là $(0;2)$.
		
	}
\end{ex}
%%%==============HetCau_EX3==============%%%
%%%==============Cau_EX4==============%%%
\begin{ex}%[Dự án C THPTQG 2025]%[ĐỖ CHÍ TÂM]%[2H3Y1-1]
	Trong không gian với hệ trục tọa độ Oxyz , cho $2$ điểm $A (1; -2; -3)$ và $B ( 7 ; -14 ; 11 )$. Tọa độ trung điểm của đoạn thẳng $AB$ là
	\choice
	{$(-4; 8; -4)$}
	{\True $(4; -8; 4)$}
	{$(3; -6; 7)$}
	{$(-3; 6; -7)$}
	\loigiai{
		Gọi $M(x_M; y_M; z_M)$ là trung điểm của đoạn thẳng $AB$, ta có $$\heva{&x_M=\dfrac{x_A+x_B}{2}=\dfrac{1+7}{2}=4\\&y_M=\dfrac{y_A+y_B}{2}=\dfrac{-2-14}{2}=-8\\&z_M=\dfrac{z_A+z_B}{2}=\dfrac{-3+11}{2}=4.}$$
		Vậy $M(4; -8; 4)$.
	}
\end{ex}
%%%==============HetCau_EX4==============%%%
%%%==============Cau_EX5==============%%%
\begin{ex}%[Dự án C THPTQG 2025]%[ĐỖ CHÍ TÂM]%[1D2N2-4]
	Cho cấp số cộng $(u_n)$ có $u_1=1$ và công sai $d=3$. Số hạng $u_3$ của cấp số cộng là
	\choice
	{$10$}{\True $7$}{$9$}{$4$}
	\loigiai{
		Ta có $u_3=u_1+2d=1+2\cdot 3=7$.
	}
\end{ex}
%%%==============HetCau_EX5==============%%%
%%%==============Cau_EX6==============%%%
\begin{ex}%[Dự án C THPTQG 2025]%[ĐỖ CHÍ TÂM]%[2D1Y5-3]
	\immini{Cho hàm số $y=\dfrac{ax+b}{cx+d}$ có đồ thị như hình vẽ bên. Tọa độ giao điểm của đồ thị hàm số đã cho với trục hoành là
	\choice
	{$(0; 2)$}{$(-2; 0)$}{$(0; -2)$}{\True $(2; 0)$}}{
		\begin{tikzpicture}[scale=0.5, line join=round, line cap=round, >=stealth,font=\scriptsize]
			\tikzset{every node/.style={scale=1}}
			\def\xmin{-5}\def\xmax{4}\def\ymin{-3}\def\ymax{5}
			\draw[->] (\xmin-0.2,0)--(\xmax+0.2,0) node[below]{$x$};
			\draw[->] (0,\ymin-0.2)--(0,\ymax+0.2) node[right]{$y$};
			\draw (0,0) node[below left]{$O$};
			\foreach \x in {-1}\draw (\x,0.1)--(\x,-0.1) node[below left]{$\x$};
			\foreach \x in {2}\draw (\x,0.1)--(\x,-0.1) node[below]{$\x$};
			\foreach \y in {1}\draw (0.1,\y)--(-0.1,\y) node[above left]{$\y$};
			\foreach \y in {-2}\draw (0.1,\y)--(-0.1,\y) node[right]{$\y$};
			\clip (\xmin,\ymin) rectangle (\xmax,\ymax);
			\draw[dashed] (\xmin,1.0)--(\xmax,1.0);
			\draw[dashed] (-1.0,\ymin)--(-1.0,\ymax);
			\draw[thick,smooth,samples=200,domain=\xmin:-1.1] plot (\x,{(1*(\x)+-2)/(1*(\x)+1)});
			\draw[thick,smooth,samples=200,domain=-0.9:\xmax] plot (\x,{(1*(\x)+-2)/(1*(\x)+1)});
		\end{tikzpicture}
	}
	\loigiai{
		Từ hình vẽ, ta suy ra đồ thị hàm số đã cho cắt trục hoành tại điểm $(2; 0)$.
	}
\end{ex}
%%%==============HetCau_EX6==============%%%
%%%==============Cau_EX7==============%%%
\begin{ex}%[Dự án C THPTQG 2025]%[ĐỖ CHÍ TÂM]%[2H3Y2-5]
	Trong không gian với hệ trục tọa độ $Oxyz$, cho hai véc-tơ $\vec u=(-1; 1; 0)$, $\vec v=(0; -1; 0)$. Góc giữa hai véc-tơ đã cho bằng
	\choice
	{$120^\circ$}{$60^\circ$}{\True $135^\circ$}{$45^\circ$}
	\loigiai{
		Ta có $$\cos(\vec u, \vec v)=\dfrac{\vec u\cdot\vec v}{|\vec u|\cdot|\vec v|}=\dfrac{-1\cdot 0+1\cdot(-1)+0\cdot 0}{\sqrt{(-1)^2+1^2+0^2}\sqrt{0^2+(-1)^2+0^2}}=-\dfrac{\sqrt2}{2}\Rightarrow(\vec u, \vec v)=135^\circ.
		$$
	}
\end{ex}
%%%==============HetCau_EX7==============%%%\\
%%%==============Cau_EX8==============%%%
\begin{ex}%[Dự án C THPTQG 2025]%[ĐỖ CHÍ TÂM]%[1D5H2-3]
	Kết quả thống kê chiều cao (đơn vị cm) của các bạn học sinh nữ lớp 12A ở bảng sau	
	\begin{center}
		\begin{tabular}{|c|c|c|c|c|c|}
			\hline
			Chiều cao (cm) & $[155; 160)$ & $[160; 165)$ & $[165; 170)$ & $[170; 175)$ & $[175; 180)$ \\
			\hline
			Số học sinh & $5$ & $9$ & $8$ & $2$ & $1$ \\
			\hline
		\end{tabular}
	\end{center}	
	Tứ phân vị thứ ba của mẫu số liệu ghép nhóm về chiều cao của học sinh nữ lớp 12A (làm tròn đến chữ số thập phân thứ hai) bằng
	\choice
	{$160{,}69$}{$168{,}59$}{$166{,}24$}{\True $167{,}97$}
	\loigiai{
		Ta có
		\begin{itemize}
			\item $n=5+9+8+2+1=25$.
			\item $\dfrac{3n}{4}=\dfrac{3\cdot 25}{4}=18{,}75$.
		\end{itemize}
		Vì $5+9<18{,}75<5+9+8$ nên $Q_3\in[165; 170)$.\\
		Suy ra $Q_3=165+\dfrac{18{,}75-(5+9)}{8}\cdot(170-165)\approx 167{,}97$.
		
	}
\end{ex}
%%%==============HetCau_EX8==============%%%
%%%==============Cau_EX9==============%%%
\begin{ex}%[Dự án C THPTQG 2025]%[ĐỖ CHÍ TÂM]%[1D6H4-3]
	Tập nghiệm $S$ của bất phương trình $\log_2(2x-1)\ge3$ là
	\choice
	{$\left[\dfrac52; +\infty\right)$}{\True $\left[\dfrac92; +\infty\right)$}{$\left(\dfrac72; +\infty\right)$}{$\left[\dfrac72; +\infty\right)$}
	\loigiai{
		Điều kiện xác định: $2x-1>0\Leftrightarrow x>\dfrac12$.\\
		Ta có $\log_2(2x-1)\ge 3\Leftrightarrow 2x-1\ge2^3\Leftrightarrow x\ge\dfrac92$.\\
		Kết hợp với điều kiện xác định, tập nghiệm bất phương trình là $S=\left[\dfrac92; +\infty\right)$.
	}
\end{ex}
%%%==============HetCau_EX9==============%%%
%%%==============Cau_EX10==============%%%
\begin{ex}%[Dự án C THPTQG 2025]%[ĐỖ CHÍ TÂM]%[2D1B1-2]
	\immini{Cho hàm số $y=ax^3+bx^2+cx+d$ ($a\ne0$) có đồ thị như hình vẽ bên. Hàm số đã cho nghịch biến trong khoảng nào sau đây?
	\choice
	{$(-\infty; -1)$ và $(1; +\infty)$}{$(0; +\infty)$}{$(-\infty; 0)$}{\True $(-1; 1)$}}{
		\begin{tikzpicture}[scale=0.55, line join=round, line cap=round, >=stealth,font=\scriptsize]
			\tikzset{every node/.style={scale=1}}
			\def\xmin{-2.1}\def\xmax{3}\def\ymin{-1.5}\def\ymax{4}
			\draw[->] (\xmin-0.2,0)--(\xmax+0.2,0) node[below]{$x$};
			\draw[->] (0,\ymin-0.2)--(0,\ymax+0.2) node[right]{$y$};
			\draw (0.2,0) node[below left]{$O$};
			\foreach \x in {-1}\draw (\x,0.1)--(\x,-0.1) node[below]{$\x$};
			\foreach \x in {1}\draw (\x,0.1)--(\x,-0.1) node[above]{$\x$};
			\foreach \y in {-1}\draw (0.1,\y)--(-0.1,\y) node[left]{$\y$};
			\foreach \y in {1,3}\draw (0.1,\y)--(-0.1,\y) node[right]{$\y$};
			\clip (\xmin,\ymin) rectangle (\xmax,\ymax);
			\draw[thick,smooth,samples=200,domain=\xmin:\xmax] plot (\x,{1*((\x)^3)+0*((\x)^2)+-3*(\x)+1});
			\draw[dashed] (-1,0)--(-1,3)--(0,3);\fill (-1,3)circle(1.5pt);
			\draw[dashed] (1,0)--(1,-1)--(0,-1);\fill (1,-1)circle(1.5pt);
		\end{tikzpicture}
	}
	\loigiai{
	Dựa vào đồ thị hàm số, ta suy ra hàm số đã cho nghịch biến trong khoảng $(-1; 1)$.}
\end{ex}
%%%==============HetCau_EX10==============%%%
%%%==============Cau_EX11==============%%%
\begin{ex}%[Dự án C THPTQG 2025]%[ĐỖ CHÍ TÂM]%[2D1Y4-1]
	\immini{Cho hàm số $y=f(x)$ có bảng biến thiên như hình vẽ bên. Đường tiệm cận ngang của đồ thị hàm số đã cho là
	\choice
	{$y=1$}{\True $y=2$}{$x=1$}{$x=2$}}{
		\begin{tikzpicture}[font=\scriptsize,scale=0.65]
			\tkzTabInit[nocadre=false,lgt=1.2,espcl=2.5,deltacl=0.6]
			{$x$/1,$f'(x)$/1,$f(x)$/2}{$-\infty$,$1$,$+\infty$}
			\tkzTabLine{,-,d,-,}
			\tkzTabVar{+/$2$,-D+/$-\infty$/$+\infty$,-/$2$}
		\end{tikzpicture}
		
	}
	\loigiai{
		Từ bảng biến thiên, ta có $\lim\limits_{x\to-\infty}y=2$ và $\lim\limits_{x\to+\infty}y=2$.\\
		Suy ra đồ thị hàm số có đường tiệm cận ngang là $y=2$.
	}
\end{ex}
%%%==============HetCau_EX11==============%%%
%%%==============Cau_EX12==============%%%
\begin{ex}%[Dự án C THPTQG 2025]%[ĐỖ CHÍ TÂM]%[1H8H6-1]
	\immini{Cho hình chóp $S.ABCD$ có đáy là hình chữ nhật, $AB=a\sqrt3$, $SA\perp(ABCD)$ và $SB=2a$ (minh họa như hình bên). Góc giữa $SB$ và mặt phẳng $(ABCD)$ bằng
	\choice
	{$90^\circ$}{$45^\circ$}{\True $30^\circ$}{$60^\circ$}}{
		\begin{tikzpicture}[line join = round, line cap = round,>=stealth,font=\scriptsize,scale=0.5]
			\coordinate (A) at (0,0);
			\coordinate (D) at (4,0);
			\coordinate (B) at (-2,-1);
			\coordinate (C) at ($(B)+(D)-(A)$);
			\coordinate (S) at ($(A)+(0,4)$);
			\draw [dashed] (S)--(A)--(B) (A)--(D);
			\draw (C)--(S)--(B)--(C)--(D)--(S);
			\draw [fill=black] (A)node[above left]{$A$}circle(1pt) (B)node[below left]{$B$}circle(1pt) (C)node[below right]{$C$}circle(1pt) (D)node[right]{$D$}circle(1pt) (S)node[above left]{$S$}circle(1pt);
		\end{tikzpicture}
	}
	\loigiai{
		Ta có $\heva{&SA\perp(ABCD)\text{ tại }A\\&SB\cap(ABCD)=B}\Rightarrow AB$ là hình chiếu của $SB$ lên $(ABCD)$.\\$\Rightarrow (SB,(ABCD))=(SB,AB)=\widehat{SBA}$.\\
		Xét tam giác $SBA$ vuông tại $A$, ta có $$\cos\widehat{SBA}=\dfrac{AB}{SB}=\dfrac{a\sqrt3}{2a}=\dfrac{\sqrt3}{2}\Rightarrow \widehat{SBA}=30^\circ.$$
		Vậy $(SB,(ABCD))=30^\circ$.
	}
\end{ex}
%%%==============HetCau_EX12==============%%%
%  \Closesolutionfile{ans}
%  \Closesolutionfile{ansbook}
 
\cauds
%   \Opensolutionfile{ansbook}[Ansbook/TT-THPT-SGD-TuyenQuang-NH24-25-TF]%---Nên đặt tên theo bài
%   \setcounter{ex}{0}
%%%==============Cau_EX1==============%%%
\begin{ex}%[Dự án C THPTQG 2025]%[ĐỖ CHÍ TÂM]%[2D1K3-6]
	Một trang sách có dạng hình chữ nhật có diện tích $486$ cm$^2$. Giả sử trang sách được đặt dọc trên bàn và lề trên, lề dưới đều để $3$ cm; lề trái, lề phải đều để $2$ cm; phần còn lại của trang sách được in chữ. Gọi $x$ là chiều rộng của trang sách.
	\choiceTF
	{Chiều dài của trang sách khi đó là $486-x$ (cm)}
	{\True Phần in chữ của trang sách có diện tích lớn nhất khi $x=18$ (cm)}
	{Phần in chữ của trang sách có diện tích lớn nhất là $276$ cm$^2$}
	{Khi diện tích phần in chữ lớn nhất thì phần diện tích lề để trống là $210$ cm$^2$}
	\loigiai{
		\begin{itemchoice}
			\itemch Sai. Trang sách có diện tích là $486$ cm$^2$.\\
			Chiều rộng là $x$ (cm), suy ra chiều dài của trang sách là $\dfrac{486}{x}$ (cm).
			\itemch Đúng. Chiều rộng của phần in chữ là $x-2\cdot 2=x-4$ (cm).\\
			Chiều dài của phần in chữ là $\dfrac{486}{x}-3\cdot 2=\dfrac{486}{x}-6$ (cm).\\
			Diện tích phần in chữ là $S=(x-4)\left(\dfrac{486}{x}-6\right)=510-6\left(x+\dfrac{324}{x}\right)$.\\
			Vì $x>0$ nên $\dfrac{324}{x}>0$.\\
			Áp dụng bất đẳng thức Cauchy cho hai số không âm $x$ và $\dfrac{324}{x}$, ta có $$x+\dfrac{324}{x}\ge2\sqrt{x\cdot\dfrac{324}{x}}=2\cdot 18=36\Rightarrow 510-6\left(x+\dfrac{324}{x}\right)\ge510-6\cdot 36=294.$$
			$S$ đạt giá trị lớn nhất bằng $294$ khi dấu ``$=$'' xảy ra.\\
			Dấu ``$=$'' xảy ra khi $x=\dfrac{324}{x}\Leftrightarrow x^2=324\Leftrightarrow x=18$ (vì $x>0$).\\
			Do đó, phần in chữ đạt giá trị lớn nhất bằng $294$ cm$^2$ khi $x=18$ (cm).
			\itemch Sai. Phần in chữ đạt giá trị lớn nhất bằng $294$ cm$^2$.
			\itemch Sai. Diện tích trang giấy là $486$ cm$^2$.\\
			Diện tích phần in chữ lớn nhất bằng $294$ cm$^2$.\\
			Diện tích phần để trống là $486-294=192$ cm$^2$.
		\end{itemchoice}
	}
\end{ex}
%%%==============HetCau_EX1==============%%%
%%%==============Cau_EX2==============%%%
\begin{ex}%[Dự án C THPTQG 2025]%[ĐỖ CHÍ TÂM]%[2H3B1-1]
	Trong không gian với hệ trục tọa độ $Oxyz$, cho tam giác $ABC$ với $A(4; 0; 2)$, $B(1; -4; -2)$ và $C(2; 1; 1)$.
	\choiceTF
	{Tọa độ trọng tâm tam giác $ABC$ là $G\left(\dfrac73; 1; \dfrac13\right)$}
	{\True Diện tích của tam giác $ABC$ bằng $\dfrac{\sqrt{210}}{2}$}
	{\True Tọa độ điểm $D$ thỏa mãn $ABCD$ là hình bình hành là $D(5; 5; 5)$}
	{Gọi điểm $E(a; b; c)$ là giao điểm của đường thẳng $BC$ với mặt phẳng tọa độ $(Oxz)$. Khi đó, $\dfrac{2a}{c}+b=\dfrac92$}
	\loigiai{
		
		\begin{itemchoice}
			\itemch Sai.
			$G(x_G; y_G; z_G)$ là trọng tâm tam giác $ABC$ thì $$\heva{&x_G=\dfrac{4+1+2}{3}=\dfrac73\\&y_G=\dfrac{0-4+1}{3}=-1\\&z_G=\dfrac{2-2+1}{3}=\dfrac13.}$$
			\itemch Đúng.
			Ta có $\heva{&\vec{AB}=(-3; -4; -4)\\&\vec{AC}=(-2;1;-1).}\Rightarrow \left[\vec{AB},\vec{AC}\right]=(8; 5;-11)$\\
			Suy ra $S_{\triangle ABC}=\dfrac12\left|\left[\vec{AB},\vec{AC}\right]\right|=\dfrac12\cdot\sqrt{210}=\dfrac{\sqrt{210}}{2}.$
			\itemch Đúng.
			Gọi $D(m; n; p)$, ta có $\heva{&\vec{DC}=(2-m; 1-n; 1-p)\\&\vec{AB}=(-3; -4; -4).}$\\
			$ABCD$ là hình bình hành khi chỉ khi $\vec{AB}=\vec{DC}\Leftrightarrow\heva{&2-m=-3\\&1-n=-4\\&1-p=-4}\Leftrightarrow\heva{&m=5\\&n=5\\&p=5.}$\\
			Vậy $D(5; 5; 5)$.
			\itemch Sai.
			Ta có $E(a; b; c)$ là giao điểm của đường thẳng $BC$ với mặt phẳng tọa độ $(Oxz)$.\\
			Khi đó $E\in(Oxz)$ nên $b=0$.\\
			Ba điểm $E$, $B$, $C$ thẳng hàng nên $\vec{BE}=(a-1; 4; c+2)$ và $\vec{BC}=(1; 5; 3)$ cùng phương.\\
			Suy ra $\dfrac{a-1}{1}=\dfrac{4}{5}=\dfrac{c+2}{3}\Leftrightarrow\heva{&a=\dfrac95\\&c=\dfrac25.}$\\
			Vậy $\dfrac{2a}{c}+b=9$.
		\end{itemchoice}
	}
\end{ex}
%%%==============HetCau_EX2==============%%%
%%%==============Cau_EX3==============%%%
\begin{ex}%[Dự án C THPTQG 2025]%[ĐỖ CHÍ TÂM] %[2D1B3-1]
	Cho hàm số $f(x)=\sqrt2x+2\cos x$.
	\choiceTF
	{\True $f(0)=2$; $f\left(\dfrac{\pi}{2}\right)=\dfrac{\pi\sqrt2}{2}$}
	{\True Đạo hàm của hàm số đã cho là $f'(x)=-2\sin x+\sqrt2$}
	{Trên đoạn $\left[0; \dfrac{\pi}{2}\right]$, phương trình $f'(x)=0$ có hai nghiệm}
	{\True Giá trị lớn nhất của $f(x)$ trên đoạn $\left[0; \dfrac{\pi}{2}\right]$ là $\dfrac{\sqrt2\pi}{4}+\sqrt2$}
	\loigiai{
		\begin{itemchoice}
			\itemch Đúng.
			Ta có $f(0)=\sqrt2\cdot 0+2\cos 0=2$; $f\left(\dfrac{\pi}{2}\right)=\sqrt 2\cdot\dfrac{\pi}{2}+2\cos\dfrac{\pi}{2}=\dfrac{\pi\sqrt2}{2}$.
			\itemch Đúng.
			Đạo hàm của hàm số là $f'(x)=\sqrt2-2\sin x$.
			\itemch Sai.
			Ta có $\begin{aligned}[t]
				f'(x)=0&\Leftrightarrow\sqrt2-2\sin x=0\\&\Leftrightarrow\sin x=\dfrac{\sqrt2}{2}\\&\Leftrightarrow\hoac{&x=\dfrac{\pi}{4}+k2\pi\\&x=\dfrac{3\pi}{4}+k2\pi}(k\in\mathbb{Z}).
			\end{aligned}$\\
			Vì $x\in\left[0; \dfrac{\pi}{2}\right]$ nên $x=\dfrac{\pi}{4}$.
			\itemch Đúng.
			Ta có $\heva{&f(0)=2\\&f\left(\dfrac{\pi}{2}\right)=\dfrac{\pi\sqrt2}{2}\\&f\left(\dfrac{\pi}{4}\right)=\dfrac{\pi\sqrt2}{2}+\sqrt2}$.\\
			Vậy giá trị lớn nhất của $f(x)$ trên đoạn $\left[0; \dfrac{\pi}{2}\right]$ là $\dfrac{\pi\sqrt2}{2}+\sqrt2$.
		\end{itemchoice}
	}
\end{ex}
%%%==============HetCau_EX3==============%%%
%%%==============Cau_EX4==============%%%
\begin{ex}%[Dự án C THPTQG 2025]%[ĐỖ CHÍ TÂM]%[1D9V1-3]
	Trên một bảng quảng cáo, người ta mắc hai hệ thống bóng đèn. Hệ thống I gồm hai bóng đèn mắc nối tiếp, hệ thống II gồm hai bóng mắc song song. Khả năng bị hỏng của mỗi bóng đèn sau $6$ giờ thắp sáng liên tục là $0{,}15$. Biết tình trạng mỗi bóng đèn là độc lập.
	\choiceTF
	{\True Xác suất hoạt động bình thường của mỗi bóng đèn sau $6$ giờ thắp sáng là $0{,}85$}
	{Xác suất để hệ thống I bị hỏng sau $6$ giờ thắp sáng là $0{,}7225$}
	{Xác suất để hệ thống II vẫn còn chiếu sáng sau $6$ giờ thắp sáng là $0{,}0225$}
	{\True Xác suất để cả hai hệ thống I, II đều bị hỏng (không còn chiếu sáng) sau $6$ giờ thắp sáng là $0,00624375$}
	\loigiai{
		\begin{itemchoice}
			\itemch Đúng.
			Vì khả năng bị hỏng của mỗi bóng đèn sau $6$ giờ thắp sáng liên tục là $0{,}15$ nên xác suất đề mỗi bóng đèn hoạt động bình thường sau $6$ giờ chiếu sáng liên tục là $1-0{,}15=0{,}85$.
			\itemch Sai.
			Gọi biến cố $A:$ ``Hệ thống I bị hỏng sau $6$ giờ thắp sáng''.\\
			Khi đó, $P(A)=0{,}15\cdot0{,}85+0{,}85\cdot0{,}15+0{,}15\cdot0{,}15=0{,}2775$.
			\itemch Sai.
			Gọi biến cố $B:$ ``Hệ thống II vẫn còn chiếu sáng sau $6$ giờ thắp sáng''.\\
			Khi đó, $P(B)=0{,}15\cdot0{,}85+0{,}85\cdot0{,}15+0{,}85\cdot0{,}85=0{,}9775$.
			\itemch Đúng.
			Xác suất để hai hệ thống I và II không còn chiếu sáng sau $6$ giờ thắp sáng là $$P(A\overline{B})=P(A)\cdot P(\overline B)=0{,}2775\cdot(1-0{,}9775)=0,00624375.$$
		\end{itemchoice}
	}
\end{ex}
%%%==============HetCau_EX4==============%%%
%  \Closesolutionfile{ansbook}
 

\caukq
% \Opensolutionfile{ansbt}[Ansbook/TT-THPT-SGD-TuyenQuang-NH24-25-TLN]%---Nên đặt tên theo bài
% \setcounter{ex}{0}
%%%==============Cau_EX1==============%%%
\begin{ex}%[Dự án C THPTQG 2025]%[ĐỖ CHÍ TÂM]%[0H9C4-7]	
	\immini{
		Một cái ao có hình $ABCDE$ (tham khảo hình vẽ bên), ở giữa ao có một mảnh vườn trồng hoa hình tròn bán kính $9$ mét, người ta muốn bắc một cây cầu từ bờ $AB$ của ao đến vườn. Hai bờ $AE$ và $BC$ và $BC$ nằm trên hai đường thẳng vuông góc nhau, hai đường thẳng này cắt nhau tại điểm $O$. Bờ $AB$ là một parabol có đỉnh là điểm $A$ và có trục đối xứng là đường thằng $OA$. Độ dài đoạn thẳng $OA$ và $OB$ lần lượt là $48$ mét và $20$ mét, tâm $I$ của mảnh vườn cách đường thẳng $AE$ và $BC$ lần lượt $48$ mét, $30$ mét. Độ dài ngắn nhất có thể của cây cầu là bao nhiêu mét (kết quả làm tròn đến hàng phần chục)?
	}{
		\begin{tikzpicture}[line join = round, line cap = round,>=stealth,font=\scriptsize,scale=0.65]
			%Bờ AB
			\draw[thick,smooth,samples=200,domain=0:2] plot (\x,{-6/5*((\x)^2)+0*\x+24/5});
			%Bờ AE ED DC CB
			\draw [dashed] (2,0)--(0,0)--(0,4.8);
			\draw (0,4.8)--(0,7)--(7,7)--(7,0)--(2,0);
			%Mảnh đất trồng rau
			\coordinate (I) at (4.8,3);
			\draw [thick] (I) circle(0.9);
			%Vẽ các điểm
			\draw [fill=black] (0,0)node[below left]{$O$}circle(1pt) (0,4.8)node[left]{$A$}circle(1pt) (0,7)node[above left]{$E$}circle(1pt) (7,7)node[above right]{$D$}circle(1pt) (7,0)node[below right]{$C$}circle(1pt) (2,0)node[below]{$B$}circle(1pt) (4.8,3)node[above]{$I$}circle(1pt);
			\draw [dashed] (0,3)--(I)--(4.8,0);
			%Vẽ cây cầu
			\draw [thick] (1.8,0.912)--(4.1,2.4343);
		\end{tikzpicture}
	}
	\shortans[oly]{$25,2$}
	\loigiai{
		Chọn hệ trục tọa độ $Oxy$ với gốc $O$, chiều dương trục hoành là tia $OC$, chiều dương trục tung là tia $OE$, đơn vị hai trục là {ex}vị độ dài (1 mét).\\
		Khi đó, ta có phương trình đường parabol là $y=-\dfrac3{25}x^2+48$ và p{ex}g trình đường tròn tâm $I(48; 30)$, bán kính $R=9$ là $(x-48)^2+(y-30)^2=81$.\\
		Xét điểm $M\left(a; -\dfrac{3}{25}a^2+48\right)$ với $0\le a\le 20$ nằm trên parabol thì khoảng cách từ đường tròn đến parabol là $d=MI-R=\sqrt{(48-a)^2+\left(-\dfrac3{25}a^2+48-30\right)^2}-9$.\\
		Khảo sát hàm số ta tìm được khoảng cách ngắn nhất xấp xỉ $25,2$ (mét).
	}
\end{ex}
%%%==============HetCau_EX1==============%%%
%%%==============Cau_EX2==============%%%
\begin{ex}%[Dự án C THPTQG 2025]%[ĐỖ CHÍ TÂM]	%[2H3K1-4]
	Hai chiếc máy bay không người lái cùng bay lên từ một địa điểm. Sau một giờ bay, chiếc thứ nhất cách điểm xuất phát về phía bắc $23$ km và về phía tây $18$ km, đồng thời cách mặt đất $2$ km. Chiếc thứ hai cách điểm xuất phát về phía đông $22$ km và về phía nam $27$ km, đồng thời cách mặt đất $3$ km. Chọn hệ trục tọa độ $Oxyz$ với gốc tọa độ $O$ đặt tại vị trí xuất của hai chiếc máy bay, mặt phẳng $Oxy$ trùng với mặt đất sao cho trục $Ox$ hướng về phía bắc, trục $Oy$ hướng về phía tây và trục $Oz$ hướng thẳng đứng lên trời, đơn vị đo lấy theo ki-lô-mét. Sau đúng một giờ bay, hai máy bay đó cùng bắn một mục tiêu di động trên mặt đất. Biết tổng khoảng cách từ mỗi máy bay đến mục tiêu là nhỏ nhất, lúc đó mục tiêu cách điểm xuất phát của hai máy bay bao nhiêu ki-lô-mét (kết quả làm tròn đến hàng phần trăm)?
	\par
	\shortans[oly]{$5$}
	\loigiai{
		Với hệ trục tọa độ được chọn. Sau một giờ bay, máy bay thứ nhất có vị trí là điểm $A(23; 18; 2)$, máy bay thứ hai có vị trí là điểm $B(-22; -27; 3)$.\\
		Gọi $M$ là vị trí của mục tiêu. Vì mục tiêu di chuyển trên mặt đất nên $M\in(Oxy)\Rightarrow M(a; b; 0)$.\\
		Ta cần tìm $M$ để $MA+MB$ nhỏ nhất.\\
		Ta có $A$, $B$ nằm cùng phía đối với mặt phẳng $(Oxy)$. Gọi $B'$ là điểm đối xứng với $B$ qua mặt phẳng $(Oxy)\Rightarrow B'(-22; -27; -3) $ và $MB=MB'$.\\
		Ta có $MA+MB=MA+MB'\ge AB'$.\\
		Suy ra $MA+MB$ nhỏ nhất bằng $AB'$ khi $M$ là giao điểm của $AB'$ với mặt phẳng $(Oxy)$ hay ba điểm $A$, $M$, $B'$ thẳng hàng.\\
		Ta có $\heva{&\vec{AM}=(a-23; b-18; -2)\\&\vec{AB'}=(-45; -45; -5)}$.\\
		Ba điểm $A$, $M$, $B'$ thẳng hàng khi chỉ khi $\dfrac{a-23}{-45}=\dfrac{b-18}{-45}=\dfrac{-2}{-5}\Leftrightarrow\heva{&a=5\\&b=0}\Rightarrow M(5; 0; 0)$.\\
		Vậy khoảng cách từ mục tiêu đến vị trí xuất phát ban đầu của máy bay là đoạn $$OA=\sqrt{(5-0)^2+(0-0)^2+(0-0)^2}=5\text{ km}.$$
	}
\end{ex}
%%%==============HetCau_EX2==============%%%
%%%==============Cau_EX3==============%%%
\begin{ex}%[Dự án C THPTQG 2025]%[ĐỖ CHÍ TÂM]	%[1H8V5-3]
	Cho hình chóp $S.ABCD$ có đáy $ABCD$ là hình thoi tâm $O$, cạnh $AB=7$ và $\widehat{BAD}=120^\circ$, $SO\perp (ABCD)$ và $SO=7$. Tính khoảng cách từ điểm $O$ đến mặt phẳng $(SBC)$. (Kết quả làm tròn đến hàng phần mười).
	\par
	\shortans[oly]{$2,8$}
	\loigiai{
		\immini{
			Kẻ $OH\perp BC$ tại $H$. Kẻ $OK\perp HS$ tại $K$. Khi đó $\mathrm{d}(O,(SBC))=OK$.\\
			Vì $ABCD$ là hình thoi tâm $O$ cạnh bằng $7$ và có $\widehat{BAD}=120^\circ$, do đó $BO=\dfrac{7\sqrt2}{2}$, $OC=\dfrac72$; $BC=7$.\\
			Xét tam giác $OBC$ vuông tại $O$ có $OH$ là đường cao nên $OH=\dfrac{BO\cdot OC}{BC}=\dfrac{7\sqrt3}{4}$.\\
			Xét tam giác $SOH$ vuông tại $O$ có đường cao $OK$ nên $OK=\dfrac{OS\cdot OH}{OS^2+OH^2}=\dfrac{7\sqrt{57}}{19}\approx 2{,}8$.\\
			Vậy khoảng cách từ điểm $O$ đến mặt phẳng $(SBC)$ xấp xỉ bằng $2{,}8$.
		}{
 \begin{tikzpicture}[thick, scale=0.7]
  %%\draw[gray!20] (-3,-2) grid (6,5);
  \path (0,0) coordinate (A)  
  (-2,-2) coordinate (D)
  (5,0) coordinate (B)
  (3,-2) coordinate (C)
  (3/2,4) coordinate (S)
  (3/2, -1) coordinate (O);
  \path ($(B)!0.6!(C)$) coordinate (H)
        ($(S)!0.7!(H)$) coordinate (K);
  \draw[dashed] (S)-- (A)--(D) (A)--(B) (O)--(S) (A)--(C) (D)--(B) (O)--(K) (O)--(H);
  \draw (D)--(C)--(B)--(S)--(D) (S)--(C) (S)--(H);
  \foreach \p/\q in {A/180, D/-90, C/-90, B/0, S/90, O/-90, H/-90, K/45}
  \fill[blue] (\p) circle(2pt) node[shift={(\q:3mm)}]{\bfseries $\p$};
  	\path pic[draw,angle radius=5pt]{right angle= O--H--C};
  	\path pic[draw,angle radius=5pt]{right angle= O--K--H};
 \end{tikzpicture}			
		}
	}
\end{ex}
%%%==============HetCau_EX3==============%%%
%%%==============Cau_EX4==============%%%
\begin{ex}%[Dự án C THPTQG 2025]%[ĐỖ CHÍ TÂM]%[2D1K3-6]	
	Công ty $A$ dự định tổ chức cho nhân viên đi tham quan Huế trong hai ngày. Công ty $A$ dự định nếu đặt giá tua của công ty du lịch $B$ là $2{,}1$ triệu đồng một người thì sẽ có khoảng $142$ người tham gia. Để kích thích mọi người tham gia, công ty du lịch $B$ quyết định giảm giá và cứ mỗi lần giảm giá tua $100$ nghìn đồng thì sẽ có thêm $20$ người tham gia. Hỏi công ty du lịch $B$ phải bán giá tua là bao nhiêu triệu đồng một người để doanh thu từ tua là lớn nhất (kết quả làm tròn đến hàng phần trăm)?
	\par
	\shortans[oly]{$1,41$}
	\loigiai{
		Gọi số lần giảm $100$ nghìn đồng là $x$ ($x>0$).\\
		Giá tham gia tua của một người là $2{,}1-0{,}1x$ (triệu đồng/người).\\
		Số người tham gia tua là $142+20x$ (người).\\
		Doanh thu $f(x)=(2{,}1-0{,}1x)(142+20x)=-2x^2+27{,}8x+298{,}2$.\\
		Do $f(x)$ là đa thức bậc hai có hệ số $a<0$ nên $f(x)$ đạt giá trị lớn nhất tại $x=\dfrac{-27{,}8}{2\cdot(-2)}=6{,}95$.\\
		Vậy giá vé tham gia tua của một người để doanh thu lớn nhất là $$2{,}1-0{,}1\cdot6{,}95=1{,}41\text{ (triệu đồng)}$$
	}
\end{ex}
%%%==============HetCau_EX4==============%%%
%%%==============Cau_EX5==============%%%
\begin{ex}%[Dự án C THPTQG 2025]%[ĐỖ CHÍ TÂM]	%[1C2V3-1]
	\immini{
		Một người khách nước ngoài sang Việt Nam dự định thuê ô-tô đi du lịch bằng cách lựa chọn xuất phát từ một tỉnh bất kỳ trong các tỉnh $A$, $B$, $C$, $D$, $E$ và lần lượt đi qua các tỉnh còn lại (mỗi tỉnh đi qua một lần duy nhất) rồi quay trở về tỉnh ban đầu với thời gian (đơn vị là giờ) đi giữa các tỉnh được cho như hình vẽ. Biết giá thuê xe ở thời điểm hiện tại là $50\,000$ đồng/giờ và không thay đổi trong suốt hành trình. Hỏi chi phí tiền thuê xe ít nhất bằng bao nhiêu triệu đồng để người đó có thể thể hoàn thành chuyến đi của mình?
	}{
		\begin{tikzpicture}[line join = round, line cap = round,>=stealth,font=\scriptsize,scale=0.65]
			\coordinate (A) at (-2,4.5);
			\coordinate (B) at (-2,1);
			\coordinate (C) at (-4,-2);
			\coordinate (D) at (4,-2);
			\coordinate (E) at (0,0);
			
			\coordinate (m) at ($(A)!0.5!(B)$);
			\coordinate (n) at ($(B)!0.5!(C)$);
			\coordinate (p) at ($(C)!0.5!(D)$);
			\coordinate (q) at ($(A)!0.5!(D)$);
			\coordinate (r) at ($(A)!0.5!(E)$);
			\coordinate (s) at ($(B)!0.5!(E)$);
			\coordinate (t) at ($(C)!0.5!(E)$);
			\coordinate (u) at ($(E)!0.5!(D)$);
			
			\draw (A)--(B)--(C)--(D)--(A)--(E)--(D) (B)--(E)--(C);
			\draw [fill=white] (A)circle(0.35)node{$A$} (B)circle(0.35)node{$B$} (C)circle(0.35)node{$C$} (D)circle(0.35)node{$D$} (E)circle(0.35)node{$E$};
			
			\draw (m)node[left]{$17$} (n)node[left]{$12$} (p)node[below]{$10$} (q)node[above]{$20$} (r)node[right]{$8$} (s)node[above]{$29$} (t)node[above]{$19$} (u)node[above]{$9$};
		\end{tikzpicture}
	}
	\par
	\shortans[oly]{$2,8$}
	\loigiai{
		Giả sử người đi du lịch xuất phát từ tỉnh $A$.\\
		Hành trình ngắn nhất người đó có thể đi là $A\to B\to C\to D\to E\to A$.\\
		Thời gian để xe di chuyển là $17+12+10+9+8=56$.\\
		Chi phí cần chi trả là $56\cdot 50\,000=2\,800\,000$ đồng.\\
		Vậy chi phí thấp nhất để người đó hoàn thành chuyến du lịch là $2{,}8$ triệu đồng.
	}
\end{ex}
%%%==============HetCau_EX5==============%%%
%%%==============Cau_EX6==============%%%
\begin{ex}%[Dự án C THPTQG 2025]%[ĐỖ CHÍ TÂM]%[0D0C2-4]
	Nhân dịp Tết Trung thu cô giáo tặng quà cho ba bạn Vũ, Hồng, Ngọc. Trong hộp quà có $9$ cây bút và $8$ quyển vở được để một cách lộn xộn. Cô giá gọi ba bạn xếp hàng theo thứ tự: Vũ đứng trước nhận quà đầu tiên, Hồng đứng sau Vũ nên được nhận quà thứ hai, Ngọc đứng sau cùng nên nhận quà sau cùng. Xác suất để Ngọc nhận quà là cây bút bằng bao nhiêu, biết rằng cô giáo tặng quà bằng cách rút ngẫu nhiên và mỗi bạn chỉ nhận một phần quà trong hộp (kết quả làm tròn đến hàng phần trăm)?
	\par
	\shortans[oly]{$0,53$}
	\loigiai{
		Ta xét các trường hợp sau
		\begin{itemize}
			\item \textbf{Trường hợp 1:} Vũ nhận bút, Hồng nhận bút, Ngọc nhận bút.\\
			Xác suất Vũ nhận bút là $\dfrac9{17}$.\\
			Xác suất Hồng nhận bút là $\dfrac{8}{16}=\dfrac12$.\\
			Xác suất Ngọc nhận bút là $\dfrac{7}{15}$.\\
			Xác suất của trường hợp này là $\dfrac{9}{17}\cdot\dfrac{1}{2}\dfrac{7}{15}=\dfrac{63}{510}$.
			\item \textbf{Trường hợp 2:} Vũ nhận bút, Hồng nhận vở, Ngọc nhận bút.\\
			Xác suất Vũ nhận bút là $\dfrac9{17}$.\\
			Xác suất Hồng nhận vở là $\dfrac{8}{16}=\dfrac12$.\\
			Xác suất Ngọc nhận bút là $\dfrac{8}{15}$.\\
			Xác suất của trường hợp này là $\dfrac{9}{17}\cdot\dfrac{1}{2}\dfrac{8}{15}=\dfrac{72}{510}$.
			\item \textbf{Trường hợp 3:} Vũ nhận vở, Hồng nhận bút, Ngọc nhận bút.\\
			Xác suất Vũ nhận bút là $\dfrac8{17}$.\\
			Xác suất Hồng nhận vở là $\dfrac{9}{16}$.\\
			Xác suất Ngọc nhận bút là $\dfrac{8}{15}$.\\
			Xác suất của trường hợp này là $\dfrac{8}{17}\cdot\dfrac{9}{16}\dfrac{8}{15}=\dfrac{72}{510}$.
			\item \textbf{Trường hợp 4:} Vũ nhận vở, Hồng nhận vở, Ngọc nhận bút.\\
			Xác suất Vũ nhận bút là $\dfrac8{17}$.\\
			Xác suất Hồng nhận vở là $\dfrac{7}{16}$.\\
			Xác suất Ngọc nhận bút là $\dfrac{9}{15}$.\\
			Xác suất của trường hợp này là $\dfrac{8}{17}\cdot\dfrac{7}{16}\dfrac{9}{15}=\dfrac{63}{510}$.
		\end{itemize}
		Vậy xác suất để Ngọc nhận được bút là $$\dfrac{63}{510}+\dfrac{72}{510}+\dfrac{72}{510}+\dfrac{63}{510}=\dfrac{9}{17}\approx 0{,}53.$$
	}
\end{ex}
%%%==============HetCau_EX6==============%%%
 \Closesolutionfile{ans}
 \inputansbox{6,4,3}{ans/TT-THPT-SGD-TuyenQuang-NH24-25}
% \begin{name}
	{\tenchude}
	{\tendethi}
	{\tentruong}
	{\thoigian}
\end{name}
%\part{ĐỀ ÔN TẬP}
% \subsection{Đề 7}
\Opensolutionfile{ans}[ans/ans-OTTNTHPT-DE7-LC]
\caulc
%Câu 1
\begin{ex}%[2D1V1-3]
	Có bao nhiêu số nguyên $m$ để hàm số $y=-x^3-3(m+1)x^2+3(m+1)x-1$ nghịch biến trên $\mathbb{R}$?
	\choice
	{$3$}
	{\True $2$}
	{$1$}
	{$0$}
	\loigiai{
		Ta có $y'=-3x^2-6(m+1)x+3(m+1)$.\\ 
		Hàm số nghịch biến trên $\mathbb{R}\Leftrightarrow y'\leq 0$ với mọi $x\in \mathbb{R}$.\\ 
		$\Leftrightarrow \Delta '=3\left[3(m+1)\right]^2-(-3)\cdot(3m+1)=9(m+1)^2+9(m+1)=9(m+1)(m+2)\leq 0$.\\ 
		$\Leftrightarrow-2\leq m\leq -1$.\\ 
		Vậy có hai giá trị $m$ nguyên thoả mãn yêu cầu là $m=-2$, $m=-1$.
	}
\end{ex}
%Câu 2
\begin{ex}%[2D1C2-1]
	Cho hàm số $y=\dfrac{x^2+2x+8}{x-2}$. Khẳng định nào sau đây \textbf{sai}?
	\choice
	{Hàm số đạt cực đại tại $x=-2$}
	{Hàm số đạt cực tiểu tại $x=6$}
	{\True Giá trị cực tiểu của hàm số là $y=6$}
	{Giá trị cực đại của hàm số là $y=-2$}
	\loigiai{
		Ta có $y'=\dfrac{x^2-4x-12}{(x-2)^2}$; $y'=0\Leftrightarrow x=-2$ hoặc $x=6$.\\ 
		Bảng biến thiên:
		\begin{center}
			\begin{tikzpicture}[scale=1, font=\footnotesize, line join=round, line cap=round, >=stealth]
				\tkzTabInit[nocadre=false,lgt=1.2,espcl=2.5,deltacl=0.6] {$x$ /0.6,$y'$ /0.6,$y$ /2} {$-\infty$,$-2$,$2$,$6$,$+\infty$}
				\tkzTabLine{,+,0,-,d,-,0,+,}
				\tkzTabVar{-/$-\infty$,+/$-2$,-D+/$-\infty$/$+\infty$,-/$14$,+/$+\infty$}
			\end{tikzpicture}
		\end{center}
		Từ đó ta thấy giá trị cực tiểu của hàm số là $y=14$.
	}
\end{ex}
%Câu 3
\begin{ex}%[2D3H2-2]
	Cho hai mẫu số liệu ghép nhóm $A$ và $B$ có bảng tần số ghép nhóm như sau:
	\begin{center}
		$A$:
		\begin{tabular}{{|>{\centering\arraybackslash}m{3cm}|>{\centering\arraybackslash}m{2cm}|>{\centering\arraybackslash}m{2cm}|>{\centering\arraybackslash}m{2cm}|>{\centering\arraybackslash}m{2cm}|>{\centering\arraybackslash}m{2cm}|}}
			\hline
			Nhóm&$[1{,}6;1{,}8)$&$[1{,}8;2{,}0)$&$[2{,}0;2{,}2)$&$[2{,}2;2{,}4)$&$[2{,}4;2{,}6)$\\ 
			\hline
			Tần số&$12$&$25$&$18$&$10$&$2$\\ 
			\hline
		\end{tabular}
	\end{center}
	\begin{center}
		$B$:
		\begin{tabular}{{|>{\centering\arraybackslash}m{3cm}|>{\centering\arraybackslash}m{2cm}|>{\centering\arraybackslash}m{2cm}|>{\centering\arraybackslash}m{2cm}|>{\centering\arraybackslash}m{2cm}|>{\centering\arraybackslash}m{2cm}|}}
			\hline
			Nhóm&$[5{,}0;5{,}2)$&$[5{,}2;5{,}4)$&$[5{,}4;5{,}6)$&$[5{,}6;5{,}8)$&$[5{,}8;6{,}0)$\\ 
			\hline
			Tần số&$2$&$10$&$18$&$25$&$12$\\ 
			\hline
		\end{tabular}
	\end{center}
	Gọi $S_A$ và $S_B$ lần lượt là độ lệch chuẩn của mẫu số liệu ghép nhóm $A$ và $B$. Khẳng định nào sau đây đúng?
	\choice
	{\True $S_A=S_B$}
	{$3S_A=S_B$}
	{$S_B=S_A+3{,}4$}
	{$\left|S_A=S_B\right|>3{,}4$}
	\loigiai{
		Áp dụng công thức tính độ lệch chuẩn của mẫu số liệu ghép nhóm, ta có $S_A=S_B$.
	}
\end{ex}
%Câu 4
\begin{ex}%[1D6V4-2]
	Nghiệm của phương trình $3^{3x-2}=9^x$ là
	\choice
	{$x=4$}
	{$x=1$}
	{\True $x=2$}
	{$x=3$}
	\loigiai{
		$3^{3x-2}=9^x\Leftrightarrow 3^{3x-2}=3^{2x}\Leftrightarrow 3x-2=2x\Leftrightarrow x=2$.
	}
\end{ex}
%Câu 5
\begin{ex}%[0H5V3-6]
	Cho hai véc-tơ $\overrightarrow{a}$, $\overrightarrow{b}$ thoả mãn điều kiện là $|\overrightarrow{a}|=1$, $\left|\overrightarrow{b}\right|=2$ và $\overrightarrow{a}\cdot \overrightarrow{b}=-1$. Khi đó $\left|\overrightarrow{a}-\overrightarrow{b}\right|$ bằng
	\choice
	{\True $\sqrt{7}$}
	{$\sqrt{5}$}
	{$3$}
	{$1$}
	\loigiai{
		$\left(\overrightarrow{a}-\overrightarrow{b}\right)^2=\overrightarrow{a}^2-2\cdot \overrightarrow{a}\cdot \overrightarrow{b}+\overrightarrow{b}^2=\left|\overrightarrow{a}\right|^2-2\overrightarrow{a}\cdot \overrightarrow{b}+\left|\overrightarrow{b}\right|^2=7$. Suy ra $\left|\overrightarrow{a}-\overrightarrow{b}\right|=\sqrt{\left(\overrightarrow{a}-\overrightarrow{b}\right)^2}=\sqrt{7}$.
	}
\end{ex}
%Câu 6
\begin{ex}%[1D2V3-4]
	Cho cấp số nhân $\left(u_n\right)$, biết $u_2\cdot u_6=64$. Giá trị của $u_3\cdot u_5$ là
	\choice
	{$-8$}
	{$-64$}
	{\True $64$}
	{$8$}
	\loigiai{
		Ta có $u_3\cdot u_5=u_2\cdot q\cdot u_5=u_2\cdot u_6=64$.
	}
\end{ex}
%Câu 7
\begin{ex}%[2D1V3-1]
	\immini[thm]{Giá trị lớn nhất của hàm số có đồ thị ở Hình 1 trên đoạn $[-3;3]$ là
		\choice
		{$2$}
		{$-2$}
		{$3$}
		{\True $4$}
		\loigiai{
			Quan sát đồ thị, ta thấy giá trị lớn nhất của hàm số là $y=4$ khi $x=3$ hoặc $x=-3$.
	}}{\begin{tikzpicture}[scale=0.5, font=\footnotesize, line join=round, line cap=round, >=stealth]
			\draw[->] (-4,0)--(4,0) node[above] {$x$};
			\draw[->] (0,-3)--(0,5) node[left] {$y$};
			\foreach \x/\y in {-3/-2,-2/-1,-1/1,1/2,2/3,3/4} {
				\draw (\x,-0.1)--(\x,0.1) (-0.1,\y)--(0.1,\y);
			}
			\draw[dashed] (-3,0)--(-3,4)--(3,4)--(3,0) (-2,0)--(-2,-2)--(2,-2)--(2,0);
			\foreach \x/\g in {-3/-90,-2/135,-1/135,1/-135,2/90,3/-90}{
				\node at ($(\x,0)+(\g:6mm)$) {$\x$};
			}
			\foreach \x/\g in {-2/-135,-1/180,1/180,2/135,3/180,4/135}{
				\node at ($(0,\x)+(\g:6mm)$) {$\x$};
			}
			\node at ($(0,0)+(-135:6mm)$) {$O$};
			\draw[samples=100] plot[domain=-3:3](\x,{(\x)^4/4-2*(\x)^2+2});
		\end{tikzpicture}\\ 
		\centering{\textit{Hình 1}}}
\end{ex}
%Câu 8
\begin{ex}%[1D6V4-3]
	Tập nghiệm của bất phương trình $\log _2(x+2)\geq \log _2(6-x)$ là
	\choice
	{$(-2;3]$}
	{\True $[2;6)$}
	{$[2;+\infty )$}
	{$(-\infty; 2]$}
	\loigiai{
		$\log _2(x+2)\geq \log _2(6-x)\Leftrightarrow \heva{&x+2\geq 6-x\\&6-x>0}
		\Leftrightarrow \heva{&x\geq 2\\&x<6}\Leftrightarrow x\in [2;6)$.
	}
\end{ex}
%Câu 9:
\begin{ex}%[2H2N2-2]
	Trong không gian $Oxyz$, cho điểm $A(-3;2;-1)$. Toạ độ điểm $A'$ là hình chiếu vuông góc của điểm $A$ trên trục $Oz$ là
	\choice
	{\True $(0;0;-1)$}
	{$(-3;2;0)$}
	{$(-3;0;0)$}
	{$(0;2;0)$}
	\loigiai{
		$A'$ là hình chiếu vuông góc của $A$ trên $Oz$ suy ra $x_{A'}=0$, $y_{A'}=0$, $z_{A'}=z_A=-1$.\\ 
		Vậy $A'(0;0;-1)$.
	}
\end{ex}
%Câu 10
\begin{ex}%[2D4V2-4]
	Giá trị của $\displaystyle \int \limits _2^2\left( 2x-e^x\right)\mathrm{\,d}x$ bằng
	\choice
	{$3-e^2$}
	{$3-e$}
	{\True $5-e^2$}
	{$5-e$}
	\loigiai{
		$\displaystyle \int \limits _2^2\left( 2x-e^x\right)\mathrm{\,d}x=\left.\left(x^2-e^x\right)\right|_0^2=\left(4-e^2\right)-\left(0-e^0\right)=5-e^2$.
	}
\end{ex}
%Câu 11
\begin{ex}%[1H8C6-1]
	\immini[thm]{Cho hình chóp $S.ABC$ có $SA\perp (ABC)$, $SA=4a\sqrt{3}$, tam giác $ABC$ vuông tại $B$, $AB=2a$ và $BC=2\sqrt{3}a$ (Hình 2).
		Góc giữa đường thẳng $SC$ và mặt phẳng $(ABC)$ bằng
		\choice
		{$30^{\circ}$}
		{$45^{\circ}$}
		{$90^{\circ}$}
		{\True $60^{\circ}$}}{\begin{tikzpicture}[scale=0.5, font=\footnotesize, line join=round, line cap=round, >=stealth]
			\coordinate (A) at (0,0);
			\coordinate (B) at (2,-2);
			\coordinate (C) at (3,0);
			\coordinate (S) at (0,4);
			\draw (S)--(A)--(B)--(C)--(S)--(B);
			\draw[dashed] (A)--(C);
			\foreach \x/\y in {A/180,B/-90,C/0,S/90}{\draw[fill=black] (\x) circle (1pt) ($(\x)+(\y:6mm)$) node {$\x$};}
			\foreach \x/\y/\z in {S/A/C,S/A/B,A/B/C}{\tkzMarkRightAngles(\x,\y,\z);}
		\end{tikzpicture}\\ 
		\centering{\textit{Hình 2}}}
	\loigiai{
		Ta có $\heva{&SA\perp (ABC)\\&SC\cap (ABC)=C}$ nên $AC$ là hình chiếu vuông góc của $SC$ lên $ABC$.\\ 
		Suy ra $\left(SC,(ABC)\right)=\left(SC,AC\right)=\widehat{SCA}$ (do tam giác $SAC$ vuông tại $A$).\\ 
		Xét $\triangle ABC$ có $AC=\sqrt{AB^2+BC^2}=\sqrt{4a^2+12a^2}=4a$.\\ 
		Xét $\triangle SAC$ có $\tan \widehat{SCA}=\dfrac{SA}{AC}=\dfrac{4a\sqrt{3}}{4a}=\sqrt{3}$.\\ 
		Vậy góc giữa $SC$ và mặt phẳng $ABC$ bằng $60^{\circ}$
	}
\end{ex}
%Câu 12
\begin{ex}%[2H5V3-3]
	Trong không gian $Oxyz$, cho hai điểm $I(1;-2;1)$ và $A(1;2;3)$. Phương trình mặt cầu có tâm $I$ và đi qua $A$ là
	\choice
	{\True $(x-1)^2+(y+2)^2+(z-1)^2=20$}
	{$(x+1)^2+(y-2)^2+(z+1)^2=5$}
	{$(x+1)^2+(y-2)^2+(z+1)^2=20$}
	{$(x-1)^2+(y+2)^2+(z-1)^2=5$}
	\loigiai{
		Ta có $R=IA=\sqrt{(1-1)^2+\left[2-(2)\right]^2+(3-1)^2}=2\sqrt{5}$.\\ 
		Phương trình mặt cầu có tâm $I$ và đi qua $A$ là $(x-1)^2+(y+2)^2+(z-1)^2=20$. 
	}
\end{ex}
\Closesolutionfile{ans}
% \indapan{6}{ans/ans-OTTNTHPT-DE7-LC}
\cauds
\Opensolutionfile{ans}[ans/ans-OTTNTHPT-DE7-DS]
\begin{ex}%[2D1H5-3]
	Cho hàm số $y=f(x)$ liên tục trên $\mathbb{R}$ và có bảng biến thiên như sau
	\begin{center}
		\begin{tikzpicture}
			\tkzTabInit[nocadre=false,lgt=1.2,espcl=2.5,deltacl=0.6]
			{$x$ /0.6, $y'$ /0.6, $y$ /2.5}
			{$-\infty$,$-5$,$-2$,$+\infty$}
			\tkzTabLine{,-,$0$,+,$0$,-,}
			\tkzTabVar{+/$6$,-/$-2$,+/$3$,-/$1$}
		\end{tikzpicture}
	\end{center}
	\choiceTF
	{\True Hàm số có hai cực trị}
	{Hàm số có hai tiệm cận ngang là $y=-2$ và $y=3$}
	{Hàm số có giá trị lớn nhất là $6$ và giá trị nhỏ nhất là $-2$}
	{Phương trình $f(x)=m-1$ có nghiệm khi $-1 \leq m \leq 7$}	
	\loigiai{
		Quan sát bảng biến thiên ta thấy
		\begin{itemchoice}
			\itemch Hàm số có hai cực trị là $x=-2$, $x=3$.
			\itemch Hàm số có hai tiệm cận ngang là $y=6$ và $y=1$.	
			\itemch Hàm số không có giá trị lớn nhất.
			\itemch Phương trình $f(x)=m-1$ có nghiệm khi $-2\leq m-1<6$ hay $-1 \leq m < 7$.
		\end{itemchoice}
	}
\end{ex}
\begin{ex}%[2D4V3-1]
	\immini{
		Cho hàm số $y=x^3-2 x^2-3 x+4$ có đồ thị $(C)$ và đường thẳng $d\colon y=2 x-2$.
		\choiceTF
		{\True Đồ thị $(C)$ và đường thẳng $d$ cùng đi qua các điểm $M(-2;-6)$, $N(1; 0)$, $P(3;4)$}
		{Diện tích hình phẳng giới hạn bởi đồ thị $(C)$, đường thẳng $d$ và hai đường thẳng $x=-2$, $x=1$ là $S_1=16$}
		{\True Diện tích hình phẳng giới hạn bởi đồ thị $(C)$ và đường thẳng $d$ là $S=\frac{253}{12}$}
		{Nếu diện tích hình phẳng giới hạn bởi đồ thị $(C)$, đường thẳng $d$ và hai đường thẳng $x=1$, $x=3$ là $S_2$ thì $S_1=3 S_2$}
	}
	{\begin{tikzpicture}[scale=.6,font=\footnotesize,samples=200,>=stealth]
			\tikzset{declare function={
					a=1;b=-2;c=-3;d=4; 
					xmin=-3;xmax=6;ymin=-7;ymax=6;}}
			%					\draw[color=gray!50,dashed] (xmin,ymin) grid (xmax,ymax);
			\draw[->] (xmin,0)--(xmax,0) node[below]{$x$};
			\draw[->] (0,ymin)--(0,ymax) node[left]{$y$};			
			\clip (xmin+0.1,ymin+0.1) rectangle (xmax-0.1,ymax-0.1);			
			\draw[dashed] (-2,0)|-(0,-6) (3,0)|-(0,4);
			\draw[teal,very thick] plot[domain=xmin+0.1:xmax-0.75](\x,{a*(\x)^3+b*(\x)^2+c*(\x)+d});
			\draw[violet,very thick] plot[domain=-2.2:4](\x,{2*(\x)-2});
			\foreach \x in {1,2,...,4}{
				\draw[thin] (\x,1.5pt)--(\x,-1.5pt);
				\draw[thin] (-\x,1.5pt)--(-\x,-1.5pt);}
			\foreach \y in {1,2,...,6}{
				\draw[thin] (1.5pt,\y)--(-1.5pt,\y);
				\draw[thin] (1.5pt,-\y)--(-1.5pt,-\y);}
			\draw[fill=white] 
			(-2,0) circle (0.05) node[above]{$-2$} 
			(1,0) circle (0.05) node[below]{$1$}
			(3,0) circle (0.05) node[below]{$3$}
			(0,4) circle (0.05) node[left]{$4$}			
			(0,-2) circle (0.05) node[left]{$-2$}
			(0,-6) circle (0.05) node[right]{$-6$}
			(0,5) circle (0.05) node[left]{$5$}
			; 		
			\draw[fill=red] (0,0) circle (0.07)node [below left]{$O$};
			\fill[pattern=vertical lines,opacity=0.3] plot[domain=1:3](\x,{2*(\x)-2})--plot[domain=3:1](\x,{a*(\x)^3+b*(\x)^2+c*(\x)+d});
			\draw[fill=red] (0,0) circle (0.07)node [below left]{$O$};
			\fill[pattern=grid,opacity=0.3] plot[domain=-2:1](\x,{2*(\x)-2})--plot[domain=1:-2](\x,{a*(\x)^3+b*(\x)^2+c*(\x)+d});			
			\draw 
			(4.5,4.0) node {$y=2x-2$}
			(0,-2.5) node[right] {$y=x^3-2x^2-3x+4$}
			(-0.5,1) node {$S_1$}
			(2,0.5) node {$S_2$}
			;
		\end{tikzpicture}
	}
	\loigiai{
		\begin{itemchoice}
			\itemch Đồ thị $(C)$ và đường thẳng $d$ cùng đi qua các điểm $M(-2;-6)$, $N(1;0)$, $P(3;4)$.
			\itemch $S_1=\displaystyle\int\limits_{-2}^1 (x^3-2x^2-3x+4-2x+2) \mathrm{\,d}x=\displaystyle\int\limits_{-2}^1 (x^3-2x^2-5x+6) \mathrm{\,d}x=\dfrac{63}{4}$.		
			\itemch 
			$S_1=\displaystyle\int\limits_{-2}^3 \left| x^3-2x^2-3x+4-2x+2\right|  \mathrm{\,d}x=\displaystyle\int\limits_{-2}^3 \left| x^3-2x^2-5x+6\right|  \mathrm{\,d}x=\dfrac{253}{12}$.
			\itemch 
			$S_2=S-S_1=\dfrac{253}{12}-\dfrac{63}{4}=\dfrac{16}{3}\Rightarrow\dfrac{S_1}{S_2}=\dfrac{189}{64}$.
		\end{itemchoice}
	}
\end{ex}
\begin{ex}%[2D6V1-4]
	Bạn An chơi tung đồng xu đổi bóng bay. Mỗi lượt chơi, An sẽ tung một đồng xu cân đối và đồng chất. Nếu đồng xu xuất hiện mặt ngửa, An được thưởng thêm $1$ quả bóng bay, ngược lại, An sẽ mất $1$ quả bóng bay bạn đang có. An đang có $10$ quả bóng bay.
	\choiceTF
	{\True Xác suất để An có $11$ quả bóng bay sau một lượt chơi là $\dfrac{1}{2}$}
	{\True Xác suất để An có $10$ quả bóng bay sau hai lượt chơi biết rằng An thắng ở lượt chơi thứ nhất là $\dfrac{1}{2}$}
	{Xác suất để An có $12$ quả bóng bay sau $3$ lượt chơi là $\dfrac{3}{8}$}
	{\True Sau $4$ lượt chơi, xác suất để An có $8$ quả bóng bay bằng xác suất để An có $12$ quả bóng bay}
	\loigiai{
		\begin{itemchoice}
			\itemch Vì An đang có $10$ quả bóng bay nên xác suất để An có $11$ quả bóng bay sau một lượt chơi bằng xác suất của biến cố \lq\lq An tung được mặt ngửa ở lượt chơi đầu tiên \rq\rq. Xác suất của biến cố này bằng $\dfrac{1}{2}$.
			\itemch Nếu An thắng ở lượt đầu tiên thì bạn sẽ có $11$ quả bóng bay.\\ Xác suất để An có $10$ quả bóng bay sau hai lượt chơi biết rằng bạn thắng ở lượt chơi thứ nhất bằng xác suất của biến cố \lq\lq An tung được mặt sấp ở lượt chơi thứ hai\rq\rq.\\ Xác suất của biến cố này bằng $\dfrac{1}{2}$.
			\itemch Sau mỗi lượt chơi An được hoặc mất $1$ quả bóng bay.\\ Do đó, sau một số lẻ lần chơi, số bóng của An sẽ là số lẻ.\\ Vậy xác suất để An có $12$ quả bóng bay sau $3$ lượt chơi là $0$.
			\itemch Xác suất để An có $8$ quả bóng bay sau $4$ lượt chơi bằng xác suất An tung được $3$ lần sấp và $1$ lần ngửa.\\
			Xác suất để An có $12$ quả bóng bay sau $4$ lượt chơi bằng xác suất An tung được $3$ lần ngửa và $1$ lần sấp.\\
			Do xác suất tung được mặt sấp bằng xác suất tung được mặt ngửa ở mỗi lần tung nên sau $4$ lượt chơi, xác suất để An có $8$ quả bóng bay bằng xác suất để An có $12$ quả bóng bay.
				\end{itemchoice}
		}
\end{ex}
\begin{ex}%[2H5V2-8]
	\immini[thm]{Các thiên thạch có đường kính lớn hơn $140$ m và có thể lại gần Trái Đất ở khoảng cách nhỏ hơn $7\,500\,000$ km được coi là những vật thể có khả năng va chạm gây nguy hiểm cho Trái Đất. Để theo dõi những thiên thạch này, người ta đã thiết lập các trạm quan sát các vật thể bay gần Trái Đất. Giả sử có một hệ thống quan sát có khả năng theo dõi các vật thể ở độ cao không vượt quá $6\,630$ km so với mực nước biển. Coi Trái Đất là khối cầu có bán kính $6\,370$ km.
	}
	{\begin{tikzpicture}[scale=.8,declare function={r=4;}]
			\path (0,0) coordinate (O)
			(0:r) coordinate (A)
			(0:r-1.5) coordinate (B)
			;
			
			\draw (O)--(B) node[above,pos=0.5,sloped]{\small$6370$ km};
			\draw (A)--(B) node[below,pos=0.5,sloped]{\tiny$6630$ km};
			
			\draw let \p1=($(O) - (A)$) in (O) circle ({veclen (\x1,\y1)});
			\draw let \p1=($(O) - (B)$) in (O) circle ({veclen (\x1,\y1)});
			\draw[<->] (O)--(B);
			\draw[<->] (A)--(B);
			\path let \p1=($ (O) - (A) $) in ($(O)+(35:{veclen(\x1,\y1)})$) coordinate (B);
			\path let \p1=($ (O) - (A) $) in ($(O)+(105:{veclen(\x1,\y1)})$) coordinate (A);
			\path ($(A)!-1/3!(B)$) coordinate (M);
			\path ($(A)!4/3!(B)$) coordinate (N);
			\draw (M)--(N);
			\foreach \t/\g in {M/90,A/90,O/-90,B/90}{
				\draw[fill=white] (\t) circle (1pt) node[shift={(\g:7pt)},font=\scriptsize]{$ \t $};
			}
	\end{tikzpicture}}
	\noindent
	Chọn hệ trục toạ độ $Oxyz$ trong không gian có gốc $O$ tại tâm Trái Đất và đơn vị độ dài trên mỗi trục toạ độ là $1\,000$ km. Một thiên thạch (coi như một hạt) chuyển động với tốc độ không đổi theo một đường thẳng từ điểm $M(6;15;-2)$, sau một thời gian vị trí đầu tiên thiên thạch di chuyển vào phạm vi theo dõi của hệ thống quan sát là điểm $A(5;12;0)$.
	\choiceTF
	{\True Đường thẳng $AM$ có phương trình chính tắc là $\dfrac{x-5}{1}=\dfrac{y-12}{3}=\dfrac{z}{-2}$}
	{Trên hệ trục toạ độ đã cho, thiên thạch di chuyển qua điểm $N(7; 18;-5)$}
	{\True Vị trí cuối cưng mà thiên thạch di chuyên trong phạm vi theo dõi cua hệ thống quan sát là $B\left(-\dfrac{6}{7};-\dfrac{39}{7};\dfrac{82}{7}\right)$}
	{\True Khoảng cách giữa vị trí đầu tiên và vị trí cuối cùng mà thiên thạch di chuyển trong phạm vi theo dơi của hệ thóng quan sát (làm tròn đên hàng đơn vị cua kilômét) là $21\,915$ km}
			\loigiai{
			\begin{itemchoice}
				\itemch Dường thẳng $AM$ đi qua $A(5;12;0)$ và có vectơ chỉ phương là $\overrightarrow{AM}=(1;3;-2)$ nên có phương trình chính tắc là $\dfrac{x-5}{1}=\dfrac{y-12}{3}=\dfrac{z}{-2}$.
				\itemch Thay toạ độ điểm $N(7;18;-5)$ vào phương trình $AM$ ta được
				\[\dfrac{7-5}{1}=\dfrac{18-12}{3}\neq \dfrac{-5}{-2}.\]
				 Suy ra thiên thạch không di chuyển qua điểm $N(7;18;-5)$.
				\itemch Vị trí cuối cùng mà thiên thạch di chuyển trong phạm vi theo dõi của hệ thống quan sát
				là $B\in AM\colon \dfrac{x-5}{1}=\dfrac{y-12}{3}=\dfrac{z}{-2} \Rightarrow B(5+t; 12+3t;-2t)$.\\
				Ngoài thực tế, khoảng cách từ tâm Trái Đất đến vị trí cuối cùng mà thiên thạch di chuyển trong phạm vi theo dõi của hệ thống quan sát là $6370+6630=13\,000$ (km) ứng với $13$ đơn vị trên hệ trục toạ độ, hay
				\allowdisplaybreaks
				\begin{eqnarray*}
					OB=13&\Leftrightarrow&OB^2=169\\
					&\Leftrightarrow &(5+t)^2+(12+3t)^2+(-2t)^2=169\\
					&\Leftrightarrow & 14 t^2+82 t=0 \Leftrightarrow \hoac{&t=0\\
						&t=-\dfrac{41}{7}.}
				\end{eqnarray*}
				Với $t=0\Rightarrow B(5;12;0)\equiv A$ (loại).\\
				Với $t=-\dfrac{41}{7}\Rightarrow B\left(-\dfrac{6}{7};-\dfrac{39}{7}; \dfrac{82}{7}\right)$.
				\itemch Khoảng cách giữa vị trí đầu tiên và vị trí cuối cùng mà thiên thạch di chuyển trong phạm vi theo dõi của hệ thống quan sát là khoảng cách giữa $A$ và $B$.\\
				Ta có $AB=\sqrt{\left(-\dfrac{6}{7}-5\right)^2+\left(-\dfrac{39}{7}-12\right)^2+\left(\dfrac{82}{7}\right)^2}=\dfrac{41\sqrt{14}}{7}$.\\
				Khoảng cách thực tế là $1\,000\cdot AB=1\,000\cdot \dfrac{41\sqrt{14}}{7} \approx 21\,915$ (km).
				\end{itemchoice}
			}
\end{ex}
\Closesolutionfile{ans}
% \indapan{2}{ans/ans-OTTNTHPT-DE7-DS}
\Opensolutionfile{ans}[ans/ans-OTTNTHPT-DE7-KQ]
\caukq
%Câu 1
\begin{ex}%[2D1C3-6]
	\immini[thm]{Bạn Linh có một tấm bìa hình vuông cạnh dài $40$ cm. Bạn dự định cắt bỏ phần tô màu như Hình $5$a rồi gấp vào dán lại để làm một hộp quà dạng hình chóp tứ giác đều như Hình $5$b (các mép dán không đáng kể). Để hộp quà có thể tích lớn nhất thì diện tích của phần bìa bị cắt bỏ là bao nhiêu cen-ti-mét vuông?}{
		\begin{tikzpicture}[scale=1, font=\footnotesize, line join=round, line cap=round, >=stealth]
			\coordinate (A) at (0,0);
			\coordinate (B) at (2,0);
			\coordinate (C) at (2,2);
			\coordinate (D) at (0,2);
			\coordinate (E) at (0.5,0.5);
			\coordinate (F) at (1.5,0.5);
			\coordinate (G) at (1.5,1.5);
			\coordinate (H) at (0.5,1.5);
			\coordinate (I) at ($(A)!0.5!(B)$);
			\coordinate (J) at ($(B)!0.5!(C)$);
			\coordinate (K) at ($(C)!0.5!(D)$);
			\coordinate (L) at ($(D)!0.5!(A)$);
			\fill[cyan] (A)--(B)--(C)--(D)--cycle;
			\fill[white] (L)--(E)--(I)--(F)--(J)--(G)--(K)--(H)--cycle;
			\draw (A)--(B)--(C)--(D)--cycle (L)--(E)--(I)--(F)--(J)--(G)--(K)--(H)--cycle (I)--(J)--(K)--(L);
			\path (A)--(B) node[midway, below]{$a)$};
		\end{tikzpicture}
		\vspace{1cm}
		\begin{tikzpicture}[scale=1, font=\footnotesize, line join=round, line cap=round, >=stealth]
			\coordinate (A) at (0,0);
			\coordinate (B) at (-0.5,-0.7);
			\coordinate (C) at (1,-0.7);
			\coordinate (D) at (1.5,0);
			\coordinate (S) at (0.5,1.3);
			\draw[dashed] (S)--(A)--(B) (A)--(D);
			\draw (S)--(B)--(C)--(D)--(S)--(C);
			\path (B)--(C) node[midway, below] {$b)$};
		\end{tikzpicture}\\
		\centering{\textit{Hình 5}}
	}
	\par
	\shortans[]{960}
	\loigiai{
		Gọi $x$ (cm) là độ dài cạnh đáy của hộp quà dạng hình chóp tứ giác đều $(0<x<40)$.\\
		Chiều cao mặt bên của hình chóp là $\dfrac{40-x}{2}$ (cm).\\ 
		Chiều cao của hình chóp là $\sqrt{\left(\dfrac{40-x}{2}\right)^2-\left(\dfrac{x}{2}\right)^2}=\dfrac{1}{2}\sqrt{1\,600-80x}$ (cm).\\ 
		Thể tích của hộp quà là $V=\dfrac{1}{3}\cdot x^2\cdot \dfrac{1}{2}\sqrt{1\,600-80x}=\dfrac{1}{6}x^2\sqrt{1\,600-80x}$ (cm$^3$).\\ 
		Ta có $V'=\dfrac{5\sqrt{5}x(16-x)}{3\sqrt{20-x}}$, $V'=0\Leftrightarrow x=16$.\\ 
		Lập bảng biến thiên, ta thấy $V$ lớn nhất khi $x=16$.\\ 
		Khi đó diện tích của phần bìa cắt bỏ là $40\cdot 40-16\cdot 16-4\cdot \dfrac{1}{2}\cdot 16\cdot \dfrac{40-16}{2}=960$ (cm$^2$).
	}
\end{ex}
%Câu 2
\begin{ex}%[2D4V3-5]
	\immini[thm]{Một khối bê tông cao $2$ m được đặt trên mặt đất phẳng. Nếu cắt khối bê tông này bằng mặt phẳng nằm ngang, cách mặt đất $x$ (m) thì được mặt cắt là hình chữ nhật có chiều dài $5$ m, chiều rộng $(0{,}5)^x$ (m), trong đó $0\leq x\leq 2$ (Hình 6). Tính thể tích của khối bê tông (làm tròn kết quả đến hàng phần trăm của mét khối).}{
		\begin{tikzpicture}[scale=0.7, font=\footnotesize, line join=round, line cap=round, >=stealth]
			\begin{scope}[rotate around={-45:(2,4)}]
				\draw plot[domain=0.5:2,samples=100] (\x,{-(\x)^2});
				\draw plot[domain=3.5:5,samples=100] (\x,{-(\x-3)^2+2});
				\coordinate (B) at (0.5,{-(0.5)^2});
				\coordinate (C) at (2,{-(2)^2});
				\coordinate (D) at (3.5,{-(3.5-3)^2+2});
				\coordinate (E) at (5,{-(5-3)^2+2});
				\coordinate (K) at (1.5,{-(1.5)^2});
				\coordinate (L) at (4.5,{-(4.5-3)^2+2});
			\end{scope}
			\coordinate (F) at ($(B)+(0.5,0.3)$);
			\coordinate (G) at ($(D)+(0.5,0.3)$);
			\coordinate (H) at ($(F)-(0,3)$);
			\coordinate (I) at ($(G)-(0,3)$);
			\coordinate (J) at ($(F)!0.6!(H)$);
			\coordinate (M) at ($(G)!0.6!(I)$);
			\fill[pattern=north east lines] (J)--(K)--(L)--(M)--cycle;
			\draw (B)--(D) (C)--(E) (B)--(F) (G)--(D) (F)--(G)--(I)--(E) (K)--(L)--(M);
			\draw[dashed] (F)--(H)--(I) (H)--(C) (K)--(J)--(M) (G)--($(G)+(0.5,0.3)$) (M)--($(M)+(1,0.6)$) (I)--($(I)+(1,0.6)$);
			\path (J)--(M) node[midway, above, scale=1] {$5$ m}; 
			\draw[<->] ($(I)+(0.5,0.3)$)--($(G)+(0.5,0.3)$) node[midway, above right] {$2$ m};
			\draw[<->] ($(I)+(1,0.6)$)--($(M)+(1,0.6)$) node[midway, right] {$x$ (m)};
		\end{tikzpicture}\\ 
		\centering{\textit{Hình 6}}
	}
	\par
	\shortans[]{$5{,}41$}
	\loigiai{
		Chọn trục $Ox$ thẳng đứng, gốc $O$ nằm trên mặt đáy của khối bê tông, chiều dương hướng lên trên. Khi đó, khối bê tông nằm trong khoảng không gian giữa hai mặt phẳng vuông góc với $Ox$ lần lượt tại các điểm $x=0$ và $x=2$. Mặt phẳng vuông góc với $Ox$ tại điểm có hoành độ $x(0\leq x\leq 2)$ cắt khối bê tông theo mặt cắt có diện tích là $S(x)=5\cdot (0{,}5)^x$ (m$^2$).\\ 
		Do đó, thể tích của khối bê tông là
		\[V=\displaystyle \int \limits _0^2S(x)\mathrm{\,d}x=\displaystyle \int \limits _0^2 5\cdot (0{,}5)^x\mathrm{\,d}x=\left.\dfrac{5}{\ln 0{,}5}\cdot (0{,}5)^x\right|_0^2=-\dfrac{5}{\ln 0{,}5}\left(\dfrac{1}{4}-1\right)=\dfrac{15}{4\ln 2}\approx 5{,}41 \text{ m}^3.\]
	}
\end{ex}
%Câu 3
\begin{ex}%[2D6V1-4]
	Bạn Minh có 9 viên bi có cùng kích thước và khối lượng, $3$ cái hộp được sơn màu khác nhau. Mỗi cái hộp có thể chứa tối đa $9$ viên bi. Minh bỏ ngẫu nhiên $9$ viên bi vào $3$ cái hộp. Tính xác suất để mỗi hộp đều có $3$ viên bi, biết rằng hộp nào cũng có ít nhất $2$ viên bi.
	\par
	\shortans[]{$\dfrac{20}{137}$}
	\loigiai{
		Số cách bỏ bi vào hộp sao cho hộp nào cũng có $3$ viên bi là $C_9^3C_6^3=1\,680$.\\ 
		Số cách bỏ bi vào hộp sao cho có $1$ hộp có $2$ viên bi, $1$ hộp có $3$ viên bi và $1$ hộp có $4$ viên bi là $3\!C_9^2C_7^3=7\,560$.\\ 
		Số cách bỏ bi vào hộp sao cho có $2$ hộp có $2$ viên bi và $1$ hộp có $5$ viên bi là $3C_9^2C_7^3=2\,268$.\\ 
		Số cách bỏ bi vào hộp sao cho hộp nào cũng có ít nhất $2$ viên bi là
		\[1\,680+7\,560+2\,268=11\,508.\]
		Vậy xác suất để mỗi hộp đều có $3$ viên bi biết rằng hộp nào cũng có ít nhất $2$ viên bi là $\dfrac{1\,680}{11\,508}=\dfrac{20}{137}$.
	}
\end{ex}
%Câu 4
\begin{ex}%[1D2C3-7]
	Xét các số thực dương $x$, $y$, $z$ theo thứ tự đó lập thành một cấp số cộng, đồng thời, các số $x$, $y-3$, $z+10$ theo thứ tự đó lập thành một cấp số nhân. Biết rằng $x+y+z=24$. Giá trị của tích $xyz$ bằng bao nhiêu?
	\par
	\shortans[]{120}
	\loigiai{
		Từ giả thiết bài toán, ta có $\heva{&x+y+z=24\\&2y=x+z}\Rightarrow x=8$.\\ 
		Hơn nữa $\heva{&x+z=2y\\&(y-3)^2=x(z+10)}\Rightarrow \heva{&x+z=16\\ &25=x(z+10)}\Rightarrow \hoac{&x=1,\, z=15\\ &x=25,\, z=-9.}$\\ 
		Vậy ba số $x$, $y$, $z$ cần tìm theo yêu cầu của bài toán là $x=1$, $y=8$, $z=15$. Suy ra $xyz=120$.
	}
\end{ex}
%Câu 5
\begin{ex}%[2H2C2-6]
	\immini[thm]{Một căn phòng có dạng hình hộp chữ nhật $ABCD.EFGH$ với $AB=6$ m, $AD=8$ m và chiều cao $10$ m. Cần giăng một dây trang trí trong phòng từ điểm $G$ đến điểm $I$ thuộc mặt sàn của phòng, rồi từ điểm đó giăng tiếp đến vị trí điểm $M$ là trung điểm của $AF$ (Hình $7$). Biết rằng $I$ là điểm sao cho dây trang trí được dùng ít nhất, khi đó $I$ cahs góc phòng $B$ bao nhiêu mét (kết quả làm tròn đến hàng phần trăm)?}{
		\begin{tikzpicture}[scale=1, font=\footnotesize, line join=round, line cap=round, >=stealth]
			\coordinate (A) at (0,0);
			\coordinate (B) at (2,0);
			\coordinate (E) at (0,2);
			\coordinate (F) at (2,2);
			\coordinate (D) at (2.5,-1.5);
			\coordinate (C) at (4.5,-1.5);
			\coordinate (H) at (2.5,0.5);
			\coordinate (G) at (4.5,0.5);
			\coordinate (M) at ($(A)!0.5!(F)$);
			\fill[orange!40!white] (A)--(B)--(C)--(D)--cycle;
			\draw (C)--(D)--(A)--(E)--(F)--(G)--(H)--(E);
			\draw (G)--(C);
			\draw (H)--(D);
			\draw[dashed] (A)--(F);
			\foreach \x in {A,F,C} {\draw[dashed] (B)--(\x);};
			\foreach \x in {A,B,C,D,E,F,G,H,M} {\draw[fill=black] (\x) circle (1pt);};
			\node[left] at (A) {$A$};
			\node[left] at (M) {$M$};
			\node[below] at (D) {$D$};
			\node[below] at (C) {$C$};
			\node[right] at (B) {$B$};
			\foreach \x in {E,F,G,H} {\node[above] at (\x) {$\x$};};
		\end{tikzpicture}\\ 
		\centering{\textit{Hình 7}}
	}
	\par
	\shortans[]{$3{,}33$}
	\loigiai{
		\immini{Dựng hệ trục $Oxyz$ như hình vẽ.\\ 
			Khi đó toạ độ các điểm là $B(0;0;0)$, $C(8;0;0)$, $D(8;6;0)$, $A(0;6;0)$, $G(8;0;10)$, $F(0;0;10)$.\\ 
			Ta có $M$ là trung điểm của $AF$ nên $M(0;3;5)$\\ 
			Dây giăng từ điểm $G$ đến chạm mặt sàn tại điểm $I(x;y;0)\in (Oxy)$ với $0\leq x\leq 8$, $0\leq y\leq 6$.\\ 
			Gọi $N$ là điểm đối xứng của điểm $M$ qua $(Oxy)$ thì $N(0;3;-5).$}{
			\begin{tikzpicture}[scale=1, font=\footnotesize, line join=round, line cap=round, >=stealth]
				\coordinate (A) at (0,0);
				\coordinate (B) at (2,0);
				\coordinate (E) at (0,2);
				\coordinate (F) at (2,2);
				\coordinate (D) at (2.5,-1.5);
				\coordinate (C) at (4.5,-1.5);
				\coordinate (H) at (2.5,0.5);
				\coordinate (G) at (4.5,0.5);
				\coordinate (M) at ($(A)!0.5!(F)$);
				\coordinate (P) at ($(A)!0.5!(B)$);
				\coordinate (N) at ($(M)!2!(P)$);
				%\fill[orange!40!white] (A)--(B)--(C)--(D)--cycle;
				\draw (C)--(D)--(A)--(E)--(F)--(G)--(H)--(E);
				\draw (G)--(C);
				\draw (H)--(D);
				\draw[dashed] (A)--(F) (M)--(P)--(N)--(G);
				\foreach \x in {A,F,C} {\draw[dashed] (B)--(\x);};
				\foreach \x in {A,B,C,D,E,F,G,H,M,N} {\draw[fill=black] (\x) circle (1pt);};
				\node[left] at (A) {$A$};
				\node[left] at (M) {$M$};
				\node[below] at (D) {$D$};
				\node[below] at (C) {$C$};
				\node[below] at (N) {$N$};
				\node[right] at (B) {$B$};
				\foreach \x in {E,F,G,H} {\node[above] at (\x) {$\x$};};
				\draw[->] (F)--($(F)+(0,1)$) node[left] {$z$};
				\draw[->] (A)--($(A)+(-1,0)$) node[below] {$y$};
				\draw[->] (C)--($(C)+(0.5,-0.3)$) node[below] {$x$};
				\tkzMarkRightAngles(M,P,B);
				\coordinate (I) at ($(N)!0.3!(G)$);
				\draw[fill=black] (I) circle (1pt) node[below right] {$I$};
				\draw (N)--($(N)!0.2!(M)$) (N)--($(N)!0.1!(G)$);
			\end{tikzpicture}
		}
		Ta có $IM+IG=IN+IG\geq GN$.\\ 
		Để $IM+IG$ nhỏ nhất thì ba điểm $I$, $G$, $N$ thẳng hàng.\\ 
		Suy ra $\overrightarrow{IG}$, $\overrightarrow{NG}$ cùng phương.\\ 
		Ta có $\overrightarrow{IG}=(8-x;-y;10)$, $\overrightarrow{NG}=(8;-3;15)$.\\ 
		Do đó $\dfrac{8-x}{8}=\dfrac{-y}{-3}=\dfrac{10}{15}$. Suy ra $x\dfrac{8}{3}$, $y=2\Rightarrow I\left(\dfrac{8}{3};2;0\right)$.\\ 
		Vậy $IB=\sqrt{\left(\dfrac{8}{3}\right)^2+2^2}=\dfrac{10}{3}\approx 3{,}33$.
	}
\end{ex}
%Câu 6
\begin{ex}%[2CD2V1-1]
	Bác An muốn gửi tiết kiệm $200$ triệu đồng vào một ngân hàng trong một năm theo hình thức lãi kép (tức là hết mỗi kì hạn thì tiền lãi nhập vào gốc để tính lãi cho kì hạn tiếp theo). Bác An phân vân giữa hai lựa chọn
	\begin{itemize}
		\item Phương án 1: Gửi tiền với lãi suất $4{,}2\%$/năm, kì hạn $3$ tháng;
		\item Phương án 2: Gửi tiền với lãi suất $5{,}1\%$/năm, kì hạn $6$ tháng.
	\end{itemize}
	Biết rằng lãi suất ngân hàng không thay đổi trong năm đó, sau khi tính toán bác An lựa chọn phương án $2$. Tính số tiền có lợi hơn nếu bác An chọn phương án $2$ so với phương án $1$ (kết quả làm tròn đến hàng phần mười của triệu đồng).
	\par
	\shortans[]{$1{,}8$}
	\loigiai{
		Nếu gửi số tiền $A$, lãi suất $r\%$ cho mỗi kì hạn và sau $n$ kì hạn thì nhận được tổng số tiền là $T=A(1+r\%)^n$.\\ 
		Tổng số tiền bác An nhận được nếu gửi theo phương án $1$ là\\ 
		$T_1=200\left(1+\dfrac{3}{12}\cdot 4{,}2\%\right)^4$.\\ 
		Tổng số tiền bác An nhận được nếu gửi theo phương án $2$ là\\ 
		$T_2=200\left(1+\dfrac{6}{12}\cdot 5{,}1\%\right)^2$.\\ 
		Số tiền chênh lệch nếu bác An chọn phương án $2$ so với phương án $1$ là\\ 
		$T_2-T_1\approx 1{,}8$ (triệu đồng).
	}
\end{ex}
\Closesolutionfile{ans}
% \indapan{6}{ans/ans-OTTNTHPT-DE7-KQ}


% --------Lời giải chi tiết
\FULLWIDTH \hienLG \hienDA % ẩn bảng đáp án
% % --
% \setcounter{deso}{10}
\chap{LỜI GIẢI CHI TIẾT} \setcounter{dang}{0} \setcounter{section}{0}
% Chương I. Hàm số
%%Bài 1. Đơn điệu, Cực trị
% \section{TÍNH ĐƠN ĐIỆU VÀ CỰC TRỊ CỦA HÀM SỐ}
\subsection{LÝ THUYẾT CẦN NHỚ}
\subsubsection{Tính đơn điệu của hàm số}
\begin{enumerate}[\iconMT]
	\item \indam{Định nghĩa:} Cho hàm số $y=f(x)$ xác định trên $K$ ($K$ là khoảng, đoạn hoặc nửa khoảng). \\
\begin{minipage}[b]{6cm}
\begin{khung4}{Ghi nhớ 1}
	Hàm số đồng biến trên $K$ nếu
	$\forall x_1,\,x_2 \in K$, $$ x_1<x_2 \Rightarrow f(x_1)<f(x_2)$$
	\centerline{\begin{tikzpicture}[>=stealth,scale=0.6]
		\draw[->] (-1,0)--(0,0)%
		node[below left]{$O$}--(5,0) node[below]{$x$};
		\draw[->] (0,-1) --(0,3) node[right]{$y$};
		\draw [black,thick, domain=0.2:4, samples=100] %
		plot (\x, {0.1*(\x)^2+1});
		\draw [dashed] (1,0)node[below]{\footnotesize$x_1$} --(1,1.1)--(0,1.1)node[left]{\footnotesize$f(x_1)$};
		\draw [dashed] (3,0)node[below]{\footnotesize$x_2$} --(3,1.9)--(0,1.9)node[left]{\footnotesize$f(x_2)$};
		\draw[fill=blue] (1,1.1) circle(2pt);
		\draw[fill=blue] (3,1.9) circle(2pt);
	\end{tikzpicture}}\\
Trên $K$, đồ thị là một "\textbf{đường đi lên}" khi xét từ trái sang phải.
\end{khung4}
\end{minipage}\hspace{0.5cm}
\begin{minipage}[b]{6cm}
\begin{khung4}{Ghi nhớ 2}
		Hàm số nghịch biến trên $K$ nếu
		$\forall x_1,\,x_2 \in K$, $$ x_1<x_2 \Rightarrow f(x_1)>f(x_2)$$
		\centerline{\begin{tikzpicture}[>=stealth,scale=0.6]
			\draw[->] (-1,0)--(0,0)%
			node[below left]{$O$}--(5,0) node[below]{$x$};
			\draw[->] (0,-1) --(0,3) node[right]{$y$};
			\draw [thick, domain=0.2:4, samples=100] %
			plot (\x, {-0.1*(\x)^2+2.5});
			\draw [dashed] (1,0)node[below]{\footnotesize$x_1$} --(1,2.4)--(0,2.4)node[left]{\footnotesize$f(x_1)$};
			\draw [dashed] (3,0)node[below]{\footnotesize$x_2$} --(3,1.6)--(0,1.6)node[left]{\footnotesize$f(x_2)$};
			\draw[fill=blue] (1,2.4) circle(2pt);
			\draw[fill=blue] (3,1.6) circle(2pt);
		\end{tikzpicture}}\\
	Trên $K$, đồ thị là một "\textbf{đường đi xuống}" khi xét từ trái sang phải.
\end{khung4}
\end{minipage}
	\item \indam{Liên hệ giữa đạo hàm và tính đơn điệu:}
	Cho hàm số $y=f(x)$ có đạo hàm trên khoảng $(a;b)$.
	\begin{boxdn}
	\begin{listEX}[1]
		\item [$\bullet$] Nếu $y'\ge 0$, $\forall x \in (a;b)$ và dấu bằng chỉ xảy ra tại hữu hạn điểm thì hàm số $y=f(x)$ đồng biến trên $(a;b)$.
		\item [$\bullet$] Nếu $y'\le 0$, $\forall x \in (a;b)$ và dấu bằng chỉ xảy ra tại hữu hạn điểm thì hàm số  $y=f(x)$ nghịch biến trên $(a;b)$.
	\end{listEX}
	\end{boxdn}
\end{enumerate}
\subsubsection{Cực trị của hàm số}
\begin{enumerate}[\iconMT]
	\item \indam{Định nghĩa:} Cho hàm số $y=f(x)$ xác định và liên tục trên khoảng $(a ; b)$ ( $a$ có thể là $-\infty, b$ có thể là $+\infty)$ và điểm $x_0 \in(a ; b)$.
	\begin{boxdn}
	\begin{itemize}
		\item [$\bullet$] Nếu tồn tại số $h>0$ sao cho $f(x)<f\left(x_0\right)$ với mọi $x \in\left(x_0-h ; x_0+h\right) \subset(a ; b)$ và $x \neq x_0$ thì ta nói hàm số $f(x)$ đạt cực đại tại $x_0$.
		\item [$\bullet$] Nếu tồn tại số $h>0$ sao cho $f(x)>f\left(x_0\right)$ với mọi $x \in\left(x_0-h ; x_0+h\right) \subset(a ; b)$ và $x \neq x_0$ thì ta nói hàm số $f(x)$ đạt cực tiểu tại $x_0$.
	\end{itemize}
	\end{boxdn}
	\item \indam{Định lý:} Giả sử hàm số $y=f(x)$ liên tục trên khoảng $(a ; b)$ chứa điểm $x_0$ và có đạo hàm trên các khoảng $\left(a ; x_0\right)$ và $\left(x_0 ; b\right)$. Khi đó:
	\begin{boxdn}
	\begin{itemize}
		\item [$\bullet$] Nếu $f^{\prime}(x)<0$ với mọi $x \in\left(a ; x_0\right)$ và $f^{\prime}(x)>0$ với mọi $x \in\left(x_0 ; b\right)$ thì $x_0$ là một điểm cực tiểu của hàm số $f(x)$.
		\item [$\bullet$] Nếu $f^{\prime}(x)>0$ với mọi $x \in\left(a ; x_0\right)$ và $f^{\prime}(x)<0$ với mọi $x \in\left(x_0 ; b\right)$ thì $x_0$ là một điểm cực đại của hàm số $f(x)$.
	\end{itemize}
	\end{boxdn}
	\item \indam{Các tên gọi:}\\
		\begin{tikzpicture}[smooth,samples=300,scale=1.15,>=stealth]
			\draw[->,>=stealth] (-2.5,0)--(2.7,0) node[below]{$x$};
			\draw[->,>=stealth] (0,-1.5)--(0,4) node[right]{$y$};
			\draw (0,0) node[above left]{$O$};
			\draw[blue,domain=-2:2,line width = 1.2pt] plot(\x,{(\x)^3-3*(\x)+1})node[right]{$y=f(x)$};
			\draw[fill=black] (1,0) circle(1pt) (1,-1) circle(2pt) (0,-1) circle(1pt) (-1,0) circle(1pt) (-1,3) circle(2pt) (0,3) circle(1pt);
			\draw[dashed] (1,0)node[above]{\small$x_2$}--(1,-1)--(0,-1)node[left]{\small$y_2$} (-1,0)node[below]{\small$x_1$}--(-1,3)--(0,3)node[right]{\small$y_1$};
			
			\draw[-,dotted] (-0.5,3.7)--(4,3.7)node[right]{$(x_1;y_1)$ là điểm cực đại của đồ thị hàm số;}; 
			\draw[->,dotted] (-0.5,3.7)--(-1,3.15);
			\node[right] at (4.5,3.1) {$\bullet$ $x_1$ là điểm cực đại của hàm số;};
			\node[right] at (4.5,2.5) {$\bullet$ $y_1$ là giá trị cực đại của hàm số.};
			
			\draw[-,dotted] (2,-1)--(2,1)--(4,1)node[right]{$(x_2;y_2)$ là điểm cực tiểu của đồ thị hàm số;}; \draw[->,dotted] (2,-1)--(1.15,-1);
			\node[right] at (4.5,0.4) {$\bullet$ $x_2$ là điểm cực tiểu của hàm số;};
			\node[right] at (4.5,-0.2) {$\bullet$ $y_2$ là giá trị cực tiểu của hàm số.};
		\end{tikzpicture}
\end{enumerate}
\subsection{PHÂN LOẠI VÀ PHƯƠNG PHÁP GIẢI TOÁN}
\begin{dang}{Bài toán tìm khoảng đơn điệu và cực trị của hàm số cho trước}
	\begin{listEX}[1]
		\item [\ding{172}] Tìm tập xác định $\mathscr{D}$ của hàm số $y=f(x)$ .
		\item [\ding{173}] Tính đạo hàm $f'(x)$. Tìm các điểm $x_i \,(i = 1, 2, ..., n)$ thuộc $\mathscr{D}$ mà tại đó đạo hàm bằng $0$ hoặc không xác định.
		\item [\ding{174}] Sắp xếp các điểm $x_i$ theo thứ tự tăng dần, xét dấu $y'$ và lập bảng biến thiên. Từ đây, nêu các khoảng đồng biến, nghịch biến và các điểm cực trị.
	\end{listEX}
\end{dang}
\indamm{Ghi nhớ cách xét dấu:}
\begin{note}
\begin{enumerate}[\iconCH]
		% \item Nếu $$f'(x)=(x-a)(x-b)^2(x-c)^{2n}(x-d)^{2n+1},\,\forall n \in \mathbb{N}*$$
		% thì phương trình $f'(x)=0$ có
		% \begin{itemize}
		% 	\item 	$x=a$ là nghiệm đơn;
		% 	\item  $x=b$ là nghiệm kép;
		% 	\item  $x=c$ là nghiệm bội chẵn;
		% 	\item  $x=d$ là nghiệm bội lẻ.
		% \end{itemize}
		\item Khi xét dấu $f'(x)$ thì $f'(x)$ sẽ không đổi dấu khi qua nghiệm kép (nghiệm bội chẵn) và đổi dấu khi qua nghiệm đơn (nghiệm bội lẻ).
	\end{enumerate}
	% \begin{tikzpicture}[smooth,samples=300,scale=0.8,>=stealth,font=\footnotesize]
	% 	\draw[->] (-3.5,0)--(6,0) node[below]{$x$};
	% 	\draw[->] (0,-1.5)--(0,4) node[left]{$y$};
	% 	\draw (0,0) node[above left]{$O$};
	% 	\draw[blue,line width=0.7pt,domain=-2.15:1.5] plot(\x,{(\x+2)*(\x-1)^2});
	% 	\draw[blue,line width=0.7pt,domain=1.5:4.7] plot(\x,{-1*(\x-1.64)*(\x-4)^2})node[below]{$y=f'(x)$};
	% 	\draw[fill=red] (-2,0)node[above left]{$x_1$} circle(1.5pt) (1,0)node[below]{$x_2$} circle(1.5pt) (4,0)node[above right]{$x_4$} circle(1.5pt) (1.64,0)node[above right]{$x_3$} circle(1.5pt);
	% 	\draw[dashed,<-] (-1.8,-0.2)--(0.5,-2.3)node[below]{\fbox{\scriptsize\text{Nghiệm bội lẻ}}};
	% 	\draw[dashed,->](0.5,-2.3)--(1.58,-0.2);
	% 	\draw[dashed,<-] (1,0.2)--(2,3)node[above]{\fbox{\scriptsize\text{Nghiệm bội chẵn}}};
	% 	\draw[dashed,->](2,3)--(3.9,0.1);
	% 	\end{tikzpicture}
\end{note}
\boxmini{BÀI TẬP TỰ LUẬN}

\begin{vd}
	Tìm các khoảng đơn điệu và các điểm cực trị của hàm số sau
	\begin{tasks}(3)
		\task $ y=-x^3+3x^2-4$;
		\task $ y=x^3-3x^2+1$;
		\task $y=x^3+3x^2+3x+2$;
		\task $y=-2x^4+4x^2$;
		\task $y=x^4+4x^3-1$;
		\task $y=-16x^4+x-1$.
	\end{tasks}
	\loigiai{
	\begin{enumEX}[a)]{1}
		\item Tập xác định: $\mathscr{D}=\mathbb{R}$. \\
		Đạo hàm: $y'=-3x^2+6x$.\\
		Xét $y'=0 \Leftrightarrow -3x^2+6x=0 \Leftrightarrow
		\hoac{
			& x=0 \\
			& x=2 }$
		Bảng biến thiên:\begin{center}
			\begin{tikzpicture}
				\tkzTabInit[nocadre=false,lgt=0.7,espcl=2.1,deltacl=0.6]
				{$x$ /0.6,$y'$ /0.6,$y$ /2}
				{$-\infty$,$0$,$2$,$+\infty$}
				\tkzTabLine{,-,$0$,+,$0$,-,}
				\tkzTabVar{+/$+\infty$, -/$-4$,+/$0$,-/$-\infty$}
			\end{tikzpicture}
		\end{center}
		\item Ta có: $ y'=3x^2-6x\Rightarrow y'=0\Leftrightarrow \hoac{&x=0\\&x=2.} $\\
		Từ bảng biến thiên suy ra hàm số đồng biến trên khoảng $ (-\infty;0) $ và $ (2;+\infty). $
		\begin{center}
			\begin{tikzpicture}
				\tkzTabInit[nocadre=false,lgt=1,espcl=3]
				{$x$ /1,$y'$ /1,$y$ /2}
				{$-\infty$,$0$, $2$,$+\infty$}
				\tkzTabLine{,+,$0$,-,$0$,+, }
				\tkzTabVar{-/ $-\infty$,+/$1 $ ,-/$-3$,+/$+\infty$}
			\end{tikzpicture}
		\end{center}
		\item Hàm số đã cho xác định trên $\mathscr{D}=\mathbb{R}$.\\
		Ta có $y'=3x^2+6x+3$. Cho $y'=0 \Leftrightarrow 3x^2+6x+3=0 \Leftrightarrow x=-1$.\\
		Bảng biến thiên
		\begin{center}
			\begin{tikzpicture}
				\tkzTabInit[lgt=1,espcl=3]
				{$x$/0.7,$y'$/0.7,$y$/2}
				{$-\infty$,$-1$,$+\infty$}
				\tkzTabLine{,+,0,+,}
				\tkzTabVar{-/$-\infty$,R/,+/$+\infty$}
				
			\end{tikzpicture}
		\end{center}
		Vậy hàm số đồng biến trên $\mathbb{R}$.
		\item Tập xác định của hàm số là $ \mathscr{D}=\mathbb{R}$.\\
		Ta có $y'=-8x^3+8x$.
		Cho $y'=0 \Leftrightarrow -8x^3+8x=0 \Leftrightarrow 8x(-x^2+1)=0$\\
		\centerline{$ \Leftrightarrow \left[\begin{aligned}
				&8x=0 \\
				&-x^2+1=0
			\end{aligned}\right. \Leftrightarrow \left[\begin{aligned}
				&x=0 \\
				&x^2=1
			\end{aligned}\right. \Leftrightarrow \left[\begin{aligned}
				&x=0 \\
				&x=\pm 1.
			\end{aligned}\right. $}
		Bảng biến thiên
		\begin{center}
			\begin{tikzpicture}
				\tkzTabInit[lgt=1,espcl=3]
				{$x$/0.7,$y'$/0.7,$y$/2}
				{$-\infty$,$-1$,$0$,$1$,$+\infty$}
				\tkzTabLine{,+,0,-,0,+,0,-,}
				\tkzTabVar{-/$-\infty$,+/ $2$/,-/$0$,+/$2$,-/$-\infty$}
			\end{tikzpicture}
		\end{center}
		Vậy hàm số đồng biến trên mỗi khoảng $(-\infty;-1)$ và $(0;1)$,\\
		\indent{ } hàm số nghịch biến trên mỗi khoảng $(-1;0)$ và $(1;+\infty)$.
		\item Hàm số đã cho xác định trên $\mathscr{D}=\mathbb{R}$.\\
		Ta có $y'=4x^3+12x^2=0=4x^2(x+3)$.\\
		Cho $y'=0 \Leftrightarrow 4x^2(x+3)=0 \Leftrightarrow \left[\begin{aligned}
			&x=0 \\
			&x=-3.
		\end{aligned}\right.$\\
		Bảng biến thiên
		\begin{center}
			\begin{tikzpicture}
				\tkzTabInit[lgt=1,espcl=3]
				{$x$/0.7,$y'$/0.7,$y$/2}
				{$-\infty$,$-3$,$0$,$+\infty$}
				\tkzTabLine{,-,0,+,0,+,}
				\tkzTabVar{+/$+\infty$,-/$-28$ /,R,+/$+\infty$}
			\end{tikzpicture}
		\end{center}
		Vậy hàm số nghịch biến trên khoảng $(-\infty;-3)$ và đồng biến trên khoảng $(-3;+\infty)$.
		\item Ta có $y'=-64x^3+1<0\Leftrightarrow x>\dfrac{1}{4}$ nên hàm số nghịch biến trên khoảng $\left(\dfrac{1}{4};+\infty\right)$.
\end{enumEX}}
\end{vd}

\begin{vd}
	Tìm các khoảng đơn điệu và cực trị của các hàm số sau:
	\begin{tasks}(3)
		\task $y=\dfrac{2x+1}{x+1}$;
		\task $y=\dfrac{3x+1}{x-1}$;
		\task $y=\dfrac{x^2+2x+2}{x+1}$;
		\task $y=x+\dfrac{4}{x}$;
		\task $y=\sqrt{x^2-2x}$;
		\task $y=x-3\sqrt[3]{x^2}$ .
	\end{tasks}
	\loigiai{
		\begin{enumEX}[a)]{1}
			\item Ta có $y'=\dfrac{1}{(x+1)^2} > 0, \forall x \in \mathbb{R} \backslash \{-1\}$.\\
			Vậy hàm số đồng biến trên $(-\infty ;-1)$ và $(-1 ;+\infty)$.\\
			Hàm số không có cực trị.
			\item Ta có $y'=\dfrac{-4}{(x-1)^2}>0,\,\forall x\in\mathbb{R}\setminus\{1\}$.\\
			Do vậy hàm số nghịch biến trên các khoảng  $(-\infty;1)$; $(1;+\infty)$.\\
			Hàm số không có cực trị.
			\item \begin{itemize}
				\item TXĐ: $\mathscr{D}=\mathbb{R}\setminus \left\{-1\right\}$.
				\item $y'=\dfrac{x^2+2x}{(x+1)^2}$, $y'=0\Leftrightarrow \hoac{& x=-2 \\ & x=0.}$\\
				Ta có bảng biến thiên
				\begin{center}
					\begin{center}
						\begin{tikzpicture}
							\tkzTabInit[nocadre=True,lgt=1,espcl=2]
							{$x$ /0.7,$y'$ /0.7,$y$ /2}
							{$-\infty$,$-2$,$-1$,$0$,$+\infty$}
							\tkzTabLine{,+,$0$,-,d,-,$0$,+,}
							\tkzTabVar{-/$-\infty$,+/$-2$,-D+/$-\infty$/$+\infty$,-/$2$,+/$+\infty$}
						\end{tikzpicture}
					\end{center}
				\end{center}
				Hàm số đồng biến trên khoảng $\left( -\infty;-2\right)$ và $\left( 0;+\infty\right)$;  nghịch biến trên $(-2;-1)$ và $(-1;0)$.\\
				Hàm số đạt cực đại tại $x=-2$, giá trị cực đại $y=-2$\\
				Hàm số đạt cực tiểu tại $x=0$, giá trị cực tiểu $y=2$.\\
			\end{itemize}
			\item Tập xác định $\mathscr{D}=\mathbb{R}\setminus\{0\}$.\\
			Ta có $y'=1-\dfrac{4}{x^2}=\dfrac{x^2-4}{x^2}$, $y'=0\Leftrightarrow x=\pm 2$.\\
			Bảng biến thiên
			\begin{center}
				\begin{tikzpicture}[yscale=.8,xscale=1.5,]
					\begin{scope}[shift={(-.5,.5)}]
						\draw
						(0,0) rectangle +(8,-5)
						(0,-1)--+(0:8) (0,-2)--+(0:8) (1,0)--+(-90:5);
					\end{scope}
					\path
					(0,0) node{$x$}          % <<< dòng 1
					++(0:1) node{$-\infty$}
					++(0:2) node{$-2$}
					++(0:1) node{$0$}
					++(0:1) node{$2$}
					++(0:2) node{$+\infty$}
					(0,-1)   node{$y'$}         % <<< dòng 2
					++(0:2) node{$+$}
					++(0:1) node{$0$}
					++(0:.5) node{$-$}
					++(0:1) node{$-$}
					++(0:.5) node{$0$}
					++(0:1) node{$+$}
					(0,-3)   node{$y$}       % <<< dòng 3
					++(0:1) ++(-90:1)  node (A) {$-\infty$}
					++(0:2) ++(90:2) node (B) {$-4$}
					++(0:1) ++(-90:2) node (C)[left]
					{$-\infty$}
					++(90:2) node (D)[right]{$+\infty$}
					++(0:1) ++(-90:2) node (E) {$4$}
					++(0:2) ++(90:2) node (F) {$+\infty$};
					\draw[-stealth] (A)--(B);
					\draw[-stealth] (B)--(C);
					\draw[-stealth] (D)--(E);
					\draw[-stealth] (E)--(F);
					\draw[double] (4,-.5)--(4,-4.5);
				\end{tikzpicture}
			\end{center}
			Hàm số đồng biến trên khoảng $\left( -\infty;-4\right)$ và $\left( 2;+\infty\right)$; nghịch biến trên các khoảng $(-2;0)$ và $(0;2)$.\\
			Hàm số đạt cực đại tại $x=-2$, giá trị cực đại $y=-4$\\
			Hàm số đạt cực tiểu tại $x=2$, giá trị cực tiểu $y=4$\\
			\item Tập xác định: $\mathscr{D}=(-\infty;0]\cup [2;+\infty)$.\\
			Ta có $y'=\dfrac{x-1}{\sqrt{x^2-2x}},\forall x\in (-\infty;0)\cup (2;+\infty)$.\\
			$y'=0 \Leftrightarrow \dfrac{x-1}{\sqrt{x^2-2x}}=0 \Rightarrow x-1=0 \Leftrightarrow x=1 \notin \mathscr{D}$.\\
			Bảng biến thiên:
			\begin{center}
				\begin{tikzpicture}
					\tkzTabInit[lgt=1,espcl=3]
					{$x$/0.7,$y'$/0.7,$y$/2}
					{$-\infty$,$0$,$2$,$+\infty$}
					\tkzTabLine{,-,d,h,d,+,}
					\tkzTabVar{+/$+\infty$,-H/$0$/,-/$0$,+/$+\infty$}
				\end{tikzpicture}
			\end{center}
			Vậy hàm số nghịch biến trên khoảng $(-\infty;0)$ và đồng biến trên khoảng $(2;+\infty)$.\\
			Hàm số không có cực trị.
			\item Tập xác định: $\mathscr{D}=\mathbb{R}$.\\
			Đạo hàm $y'=1-\dfrac{2}{\sqrt[3]{x}}$, xác định với mọi $x\neq 0$.\\
			$y'=0\Leftrightarrow \sqrt[3]{x}=2\Leftrightarrow x=8$.\\
			Đạo hàm không xác định tại $x=0$.\\
			Bảng biến thiên
			\begin{center}
				\begin{tikzpicture}
					\tkzTabInit[nocadre,lgt=1,espcl=2]{$x$/0.7,$y'$/0.7,$y$/2}{$-\infty$,$0$,$8$,$+\infty$}%
					\tkzTabLine{,+,d,-,z,+,}
					\tkzTabVar{-/$-\infty$ , +/$0$,-/$-4$, +/$+\infty$}%
				\end{tikzpicture}
			\end{center}
	\end{enumEX}}
\end{vd}

\begin{vd}
	Thể tích $V$ (đơn vị: centimét khối) của $1 \mathrm{~kg}$ nước tại nhiệt độ $T\,\left(0^{\circ} \mathrm{C} \leq T \leq 30^{\circ} \mathrm{C}\right)$ được tính bởi công thức	$$	V(T)=999,87-0,06426 T+0,0085043 T^2-0,0000679 T^3$$
	 Hỏi thể tích $V(T), \,0^{\circ} \mathrm{C} \leq T \leq 30^{\circ} \mathrm{C}$, giảm trong khoảng nhiệt độ nào?
	\loigiai{
		Xét hàm số  $V(T)=999{,}87-0{,}06426T+0{,}0085043T^2-0{,}0000679T^3$, với $T\in [0;30]$.\\
	Ta có $V'(T)=-0{,}0002037T^2+0{,}0170086T-0{,}06426$.\\
	$V'(T)=0\Leftrightarrow T=3{,}966514624=T_1$ hoặc $T=79{,}53176716\not\in [0;30]$.\\
	Bảng biến thiên của hàm số $V(T)$ như sau
	\begin{center}
		\begin{tikzpicture}[font=\footnotesize,thick,>=stealth]
			\tikzset{double style/.append style={double distance=1.5pt}}
			\tkzTabInit[nocadre=false,lgt=1.2,espcl=2.5,deltacl=0.6,lw=.75pt,color,colorL=green!50,colorV=green!50]
			{$T$ /0.7, $V'(T)$ /0.8, $V(T)$ /2}
			{$0$,$T_1$,$30$}
			\tkzTabLine{ ,-,$0$,+, }
			\tkzTabVar{+/$V(0)$,-/$V(T_1)$,+/$V(30)$}
		\end{tikzpicture}
	\end{center}
	Từ bảng biến thiên suy ra, thể tích $V(T), 0^{\circ}\mathrm{C}\leq T \leq 30^{\circ}\mathrm{C}$, giảm trong khoảng nhiệt độ từ $0^\circ$C đến $3{,}966514624^\circ$C.}
\end{vd}

\boxmini{BÀI TẬP TRẮC NGHIỆM}
\ind{PHẦN I.} \inden{Câu trắc nghiệm nhiều phương án lựa chọn. Mỗi câu hỏi học sinh chỉ chọn một phương án.}\\
\setcounter{ex}{0}
\Opensolutionfile{ans}[ans/2D1-B1-d1-1]

\begin{ex}%[KSCL, Sở GD \& ĐT Hà Nam, 2018]%[Lê Quốc Hiệp, dự án 12EX10-18]%[2D1B1-2]%
	\immini
	{Cho hàm số $y=f(x)$ có đồ thị như hình vẽ bên. Hàm số $y=f(x)$ nghịch biến trên khoảng nào dưới đây?
		\haicot
		{$(\sqrt{2};+\infty)$}
		{$(-2;2)$}
		{$(-\infty;0)$}
		{\True $(0;\sqrt{2})$}
	}
	{\begin{tikzpicture}[line cap=round,line join=round,x=1.0cm,y=1.0cm,>=stealth,scale=0.7]
			\draw[->,color=black,smooth,samples=100] (-2.5,0.) -- (2.5,0.) node[below] {\footnotesize $x$};
			\draw[->,color=black,smooth,samples=100] (0.,-2.5) -- (0.,3) node[left] {\footnotesize $y$};
			\draw plot[smooth,tension=.7] coordinates {(-2,3) (-1.41,-2)  (0,2) (1.41,-2) (2,3)};
			\draw[fill=black] (0,0) circle [radius=1pt] node[above left] {\footnotesize $O$};
			\fill (-1.41,0) node[shift={(90:2ex)}]{\footnotesize $-\sqrt{2}$} circle(1pt);
			\fill (1.41,0) node[shift={(90:2ex)}]{\footnotesize $\sqrt{2}$} circle(1pt);
			\fill (0,-2) node[shift={(-45:1.5ex)}]{\footnotesize $-2$} circle(1pt);
			\fill (0,2) node[shift={(45:1.5ex)}]{\footnotesize $2$} circle(1pt);
			\draw[dashed] (-1.41,0)|-(0,-2)-|(1.41,0);
	\end{tikzpicture}}
	\loigiai
	{
		Dựa vào đồ thị, ta thấy trên khoảng $(0;\sqrt{2})$ đồ thị đi xuống nên hàm số $y=f(x)$ nghịch biến trên khoảng đó.
	}
\end{ex}

\begin{ex}
	\immini{Cho hàm số $y=f(x)$ có đồ thị như hình vẽ bên. Mệnh đề nào sau đây là mệnh đề \textbf{sai}?
		\choice
		{Hàm số đạt cực đại tại $x=0$}
		{Hàm số có giá trị cực tiểu bằng $-2$}
		{\True Hàm số đồng biến trên $(-\infty; 2)$}
		{Hàm số nghịch biến trên $(0; 2)$}
	}
	{
		
		\begin{tikzpicture}[smooth,samples=300,scale=0.7,>=stealth]
			\draw[->] (-2,0)--(3.7,0) node[below]{$x$};
			\draw[->] (0,-2.5)--(0,2.5) node[right]{$y$};
			\draw (0,0) node[below right]{$O$};
			\draw[smooth,samples=100,domain=-1:3]
			plot(\x,{(\x)^3-3*(\x)^2+2});
			\draw[fill=black] (2,0) circle(1.5pt) (0,2) circle(1.5pt) (0,-2) circle(1.5pt);
			\draw[dashed] (2,0)node[above]{\small$2$}--(2,-2)--(0,-2)node[left]{\small$-2$} (0,2.1)node[left]{\small$2$};
		\end{tikzpicture}
	}
	
	\loigiai{
	}
	
\end{ex}

\begin{ex}
	\immini{
		Hàm số $y=f(x)$ có đồ thị là đường cong trong hình vẽ bên. Hàm số $y=f(x)$ đạt cực tiểu tại điểm nào dưới đây?
		\haicot
		{$x=2$}
		{\True $x=0$}
		{$x=-2$}
		{$x=4$}
	}{
		\begin{tikzpicture}[xscale=.7,yscale=.6, font=\footnotesize, line join=round, line cap=round, >=stealth]
			\draw[->] (-2.7,0)--(3,0) node[below]{$x$};
			\draw[->] (0,-1.25)--(0,5) node[left]{$y$};
			\draw[dashed] (-2^.5,0)--(-2^.5,4)--(2^.5,4)--(2^.5,0);
			\draw[domain=-2.05:2.05] plot(\x,{-(\x)^2*((\x)^2-4)});
			\path
			(0,0) node[below right]{$O$}
			(2,0) node[above right]{$2$}
			(-2,0) node[above left]{$-2$}
			(0,4) node[below right]{$4$}
			(-2^.5,0) node[below]{$-\sqrt{2}$}
			(2^.5,0) node[below]{$\sqrt{2}$};
		\end{tikzpicture}
	}
	\loigiai{
		Dựa vào đồ thị hàm số ta thấy hàm số đạt cực tiểu tại $x=0$.}
\end{ex}

\begin{ex}
	\immini{Cho hàm số $y=f(x)$ có bảng biến thiên như hình bên. Mệnh đề nào sau đây là mệnh đề đúng?
	\choice
	{Hàm số đồng biến trên khoảng $(-\infty;3)$}
	{Hàm số nghịch biến trên khoảng $(-2;+\infty)$}
	{Hàm số đạt cực đại tại $x=3$}
	{\True Hàm số đạt cực tiểu tại $x=2$}}{
	\begin{tikzpicture}
	\tkzTabInit[lgt=1.2,espcl=1.8,nocadre=True]
	{$x$/0.6,$f'(x)$/0.6,$f(x)$/2}{$-\infty$,$-2$,$2$,$+\infty$}
	\tkzTabLine{,+,0,-,0,+,}
	\tkzTabVar{-/$-\infty$,+/$3$,-/$0$,+/$+\infty$}
\end{tikzpicture}}
	\loigiai{
	}
	
\end{ex}

\begin{ex}
	Cho hàm số $y=f(x)$ có bảng biến thiên bên dưới
	\begin{center}
		\begin{tikzpicture}
			\tikzset{double style/.append style = {draw=\tkzTabDefaultWritingColor,double=\tkzTabDefaultBackgroundColor,double distance=2pt}}
			\tkzTabInit[lgt=1.2,espcl=2,nocadre=True]
			{$x$ /.7, $f’(x)$ /.7,$f(x)$ /2}
			{$-\infty$ , $-2$, $0$ ,$2$ ,$+\infty$}
			\tkzTabLine{ ,+,$0$,-,d,-,$0$,+, }
			\tkzTabVar{ -/ $-\infty$,+/ $-4$/,-D+/$- \infty$ /$+\infty$,-/ $4$,+ /$+\infty$}
		\end{tikzpicture}
	\end{center}
Khẳng định nào sau đây là khẳng định \textbf{sai}?
	\choice
	{Hàm số có hai điểm cực trị}
	{Tọa độ điểm cực đại của đồ thị hàm số là $(-2;-4)$}
	{\True Hàm số nghịch biến trên khoảng $(-2;2)$}
	{Hàm số đồng biến trên khoảng $(3;+\infty)$}
	\loigiai{
	}
\end{ex}


\begin{ex}
	Cho hàm số $y= - \dfrac{1}{3} x^3 - x -3 $. Mệnh đề nào dưới đây đúng?
	\choice
	{Hàm số đồng biến trên $(-\infty; 1)$ và trên $(1; +\infty)$}
	{\True Hàm số nghịch biến trên $\mathbb{R}$}
	{Hàm số đồng biến trên $(-1;1)$}
	{Hàm số đồng biến trên $\mathbb{R}$}
	\loigiai{
		Tập xác định $\mathscr D = \mathbb{R}$.\\
		$y'=-x^2 -1<0 $ với mọi $x$.\\
		Suy ra  hàm số đã cho nghịch biến trên $\mathbb{R}$.}
\end{ex} 


\begin{ex}
	Gọi $x_1$ là điểm cực đại $x_2$ là điểm cực tiểu của hàm số $y=-x^3+3x+2$. Tính $x_1+2x_2$.
	\choice{$2$}
	{$1$}
	{\True $-1$}
	{$0$}
	\loigiai{
		Ta có $y'=-3x^2+3$, $y'=0\Leftrightarrow x=\pm 1$.\\
		Vì $y'$ đổi dấu từ âm sang dương khi qua $x=-1$ và đổi dấu từ dương sang âm khi qua $x=1$ nên $x_2=-1$ là điểm cực tiểu và $x_1=1$ là điểm cực đại của hàm số. Do đó $x_1+2x_2=1-2=-1$.
	}
\end{ex} 


\begin{ex}
	Khoảng cách giữa hai điểm cực trị của đồ thị hàm số $y=x^3-3x^2+4$ bằng
	\choice
	{\True $2\sqrt{5}$}
	{$2\sqrt{2}$}
	{$2$}
	{$ 4 $}
	\loigiai{
		Ta có $y'=3x^2-6x$, $ y'=0\Rightarrow \hoac{&x=0\Rightarrow y=4\\&x=2\Rightarrow y=0.} $\\
		Suy ra hai điểm cực trị của đồ thị hàm số là $A(0;4),B(2;0)$.\\
		Do đó $AB=\sqrt{2^2+(-4)^2}=2\sqrt{5}$.
	}
\end{ex} 

\begin{ex}%[2D1B1]
	Hàm số $y=x^4-2x^2+1$ đồng biến trên khoảng nào dưới đây?
	\choice
	{\True $(-1;0)$}
	{$(-1;+ \infty)$}
	{$(-3;8)$}
	{$(- \infty ; -1)$}
	\loigiai
	{
		$y'= 4x^3-4x$ $\Rightarrow y'=0 \Leftrightarrow 4x^3-4x=0$ $\Leftrightarrow \hoac{x&= -1 \\ x&=0 \\ x &= 1}$\\
		Bảng xét dấu
		\begin{center}
			\begin{tikzpicture}
				\tkzTabInit[nocadre=false, lgt=1, espcl=2.5]{$x$ /1,$y$ /1}{$-\infty$,$-1$,$0$,$1$,$+\infty$}
				\tkzTabLine{,-,$0$,+,$0$,-,$0$,+}
			\end{tikzpicture}
		\end{center}
	}
	
\end{ex} 

\begin{ex}%[2HK1-13-ChuyenLeQuyDon-QuangTri]%[2D1B2-1]%
	Cho hàm số $ y = - \dfrac{1}{4}x^4 + \dfrac{1}{2}x^2 - 3 $. Khẳng định nào sau đây là khẳng định đúng?
	\choice
	{Hàm số đạt cực tiểu tại $ x = -3 $}
	{ \True Hàm số đạt cực tiểu tại $ x = 0 $}
	{Hàm số đạt cực đại tại $ x = 0 $}
	{Hàm số đạt cực tiểu tại $ x = -1 $}
	\loigiai{
		Ta có $ y' = - x^3 + x = - x (x^2 - 1) $.
		Ta có bảng biến thiên như hình bên
		\begin{center}
			\begin{tikzpicture}[scale=1]
				\tkzTabInit[lgt=1.5,espcl=2.5]{$x$  /1,$y'$  /1,$y$ /2}
				{$-\infty$,$ -1 $,$ 0 $,$ 1 $,$+\infty$}%
				\tkzTabLine{,+,z,-,z,+,z,-,}
				\tkzTabVar{-/$ -\infty $,+/   $\dfrac{-11}{4}$ /,-/ $-3$,+/$ \dfrac{-11}{4} $,-/$ -\infty $}
				%\tkzTabIma{1}{3}{2}{$ 0 $}
			\end{tikzpicture}
		\end{center}
	}
\end{ex} 

\begin{ex}
	Cho hàm số $y=\dfrac{3x-1}{x-2}$. Mệnh đề nào dưới đây là đúng?
	\choice
	{Hàm số nghịch biến trên $\mathbb{R}$}
	{Hàm số đồng biến trên các khoảng $(-\infty;2)$ và $(2;+\infty)$}
	{\True Hàm số nghịch biến trên các khoảng $(-\infty;2)$ và $(2;+\infty)$}
	{Hàm số đồng biến trên $\mathbb{R}\setminus\{2\}$}
	\loigiai{Tập xác định là $\mathscr{D}=\mathbb{R}\setminus\{2\}$.\\
		Có $y'=\dfrac{-5}{(x-2)^2}<0$, $\forall x\in\mathscr{D}$ nên hàm số nghịch biến trên các khoảng $(-\infty;2)$ và $(2;+\infty)$.}
	
\end{ex} 

\begin{ex}
	Cho hàm số $y=\dfrac{x-2}{x+3}$. Mệnh đề nào dưới đây đúng?
	\choice
	{Hàm số nghịch biến trên khoảng $(-\infty;-3)\cup (-3;+\infty) $}
	{\True Hàm số đồng biến trên khoảng $(-\infty;-3) $ và $(-3;+\infty)$}
	{Hàm số nghịch biến trên khoảng $(-\infty;-3)$ và $(-3;+\infty)$}
	{Hàm số đồng biến trên khoảng $(-\infty;-3)\cup (-3;+\infty) $}
	\loigiai{
		Tập xác định $\mathscr{D}=\mathbb{R}\setminus \{-3\}$. Ta có $y'=\dfrac{5}{(x+3)^2}>0$, $\forall x\in\mathscr{D}$.\\ Suy ra hàm số đồng biến trên khoảng $(-\infty;-3)$ và $(-3;+\infty)$.
	}
\end{ex} 

\begin{ex}
	Gọi $y_{\text{CĐ}},\,y_{\text{CT}}$ lần lượt là giá trị cực đại và giá trị cực tiểu của hàm số $y=\dfrac{x^2+3x+3}{x+2}$. Giá trị của biểu thức $y_{\text{CĐ}}^2-2y_{\text{CT}}^2$ bằng
	\choice
	{$8$}
	{\True $7$}
	{$9$}
	{$6$}
	\loigiai{
		Ta có $y'=\dfrac{x^2+4x+3}{(x+2)^2}$; $y'=0 \Leftrightarrow \left[\begin{aligned}
			&x=-1 \\
			&x=-3
		\end{aligned}\right. $. \\
		Bảng biến thiên
		\begin{center}
			\begin{tikzpicture}
				\tkzTab
				[lgt=1,espcl=2] % tùy chọn
				{$x$/0.7, $y'$/0.7, $y$/2} % cột đầu tiên
				{$-\infty$, $-3$, $-2$, $-1$, $+\infty$} % hàng 1 cột 2
				{,+,0,-,d,-,0,+,} % hàng 2 cột 2
				{-/ $-\infty$, +/ $-3$, -D+/ $-\infty$ / $+\infty$, -/ $1$, +/ $+\infty$} % hàng 3 cột 2
			\end{tikzpicture}
		\end{center}
		Từ bảng biến thiên ta tìm được $y_{\text{CĐ}}=-3;\,y_{\text{CT}}=1$ $ \Rightarrow $ $y_{\text{CĐ}}^2-2y_{\text{CT}}^2$ $=9-2=7$.}
\end{ex} 

\begin{ex}
	Tìm điểm cực tiểu của hàm số $f(x)=(x-3)\mathrm{e}^x$.
	\choice
	{$x=3$}
	{$x=0$}
	{\True $x=2$}
	{$x=1$}
	\loigiai{
		\begin{itemize}
			\item Ta có $f'(x)=\mathrm{e}^x(x-2)$, $f''(x)=\mathrm{e}^x(x-1)$.
			\item $f'(x)=0\Rightarrow x=2$ và $f''(2)=\mathrm{e}^2>0$.
		\end{itemize}
		Vậy hàm số đã cho đạt cực tiểu tại $x=2$.}
\end{ex} 

\begin{ex}
	Cho hàm số $y=x^2+4\ln(3-x)$. Tìm giá trị cực đai $y_\text{CĐ}$ của hàm số đã cho.
	\choice
	{$y_\text{CĐ}=2$}
	{\True $y_\text{CĐ}=4$}
	{$y_\text{CĐ}=1+4\ln2$}
	{$y_\text{CĐ}=1$}
	\loigiai{
		Tập xác định $\mathscr{D}=(-\infty;3)$.\\
		Đạo hàm $y'=2x-\dfrac{4}{3-x}=\dfrac{-2x^2+6x-4}{3-x}$.\\
		$y'=0\Leftrightarrow -2x^2+6x-4=0\Leftrightarrow \hoac{&x=1\\&x=2}$.\\
		Bảng biến thiên
		\begin{center}
			\begin{tikzpicture}[>=stealth]
				\tkzTabInit[nocadre=false,lgt=1,espcl=2,deltacl=0.5]{$x$/.7,$y'$/.7,$y$/2}
				{$-\infty$,$1$,$2$,$3$}
				\tkzTabLine{,-,0,+,0,-,d}
				\tkzTabVar{+/$+\infty$,-/$1+4\ln 2$,+/$4$,-D/$-\infty$}
			\end{tikzpicture}
		\end{center}
		Hàm số đạt cực đại tại $x=2$, $y_\text{CĐ}=4$.
	}
\end{ex} 


\begin{ex}%[2D1K2]
	Cho hàm số $y = f(x)$ xác định trên $\mathbb{R}$ và có đạo hàm $y' = f'(x) = 3x^3 - 3x^2$. Mệnh đề nào sau đây \textbf{sai}?
	\choice
	{Trên khoảng $(1;+\infty)$ hàm số đồng biến}
	{Trên khoảng $(-1;1)$ hàm số nghịch biến}
	{\True Đồ thị hàm số có hai điểm cực trị}
	{Đồ thị hàm số có một điểm cực tiểu}
	\loigiai
	{
		Ta có: $y' = 0 \Leftrightarrow 3x^3 - 3x^2 = 0 \Leftrightarrow \hoac{& x = 0 \\& x = 1.}$\\
		Bảng biến thiên:
		\begin{center}
			\begin{tikzpicture}[>=stealth]
				\tkzTabInit[nocadre, lgt=1, espcl=2.5]
				{$x$ /0.7,$y'$ /0.7,$y$ /1.7}
				{$-\infty$,$0$,$1$,$+\infty$}
				\tkzTabLine{,-,$0$,-,$0$,+,}
				\tkzTabVar{+/ $+\infty$, R, -/{\text{CT}}, +/ $+\infty$}
			\end{tikzpicture}
		\end{center}
		Hàm số đồng biến trên khoảng $(1;+\infty)$.\\
		Hàm số nghịch biến trên khoảng $(-\infty;1)$.\\
		Hàm số đạt cực tiểu tại $x = 1$.
	}
\end{ex} 

\begin{ex}%[2D1B2]
	Cho hàm số $ y=f(x) $ liên tục trên $ \mathbb{R} $ và có đạo hàm $ f'(x)=x(x-1)^2(x-2)^3 $. Số điểm cực trị của hàm số $ y=f(x) $ là
	\choice{1}{\True 2}{0}{3}
	\loigiai{	Ta có bảng xét dấu của $ f'(x) $:
		\begin{center}
			\begin{tikzpicture}
				\tkzTabInit[lgt=2,espcl=1.5]%
				{$x$ /1,$f'(x)$ /1}
				{$-\infty$ , $0$ , $1$ , $2$ ,$+\infty$}
				\tkzTabLine{ ,+,0,-,0,-,0,+,}
			\end{tikzpicture}
		\end{center}
		Dựa vào bảng xét dấu ta thấy $ f(x) $ có 2 điểm cực trị.
}\end{ex} 



\begin{ex}%[2D1K2-2]%
	\immini{Cho hàm số bậc bốn $ y=f(x) $. Biết $f'(x) $ có đồ thị như hình bên. Khẳng định nào sau đây là khẳng định đúng?
		\choice
		{Hàm số $f(x)$ đồng biến trên khoảng $(-\infty;0)$}
		{Hàm số $f(x)$ nghịch biến trên khoảng $(-1;1)$}
		{Hàm số $f(x)$ có đúng một điểm cực tiểu}
		{\True Hàm số $f(x)$ có đúng một điểm cực đại}
	}{
		\begin{tikzpicture}[>=stealth,line join=round,line cap=round,font=\footnotesize,scale=0.7,smooth]
			\draw[->] (-3,0)--(7,0)node[below]{$x$};
			\foreach \x in {-2,-1,1,2,3,4}\draw[shift={(\x,0)}] (0,2pt)--(0,-2pt) node[below]{\scriptsize $\x$};
			\draw[->] (0,-2)--(0,3)node[right]{$y$};
			\draw[] plot[smooth,tension=.65] coordinates{(-1.7,-2) (-1,0) (0,.7) (1,0)(2.7,-1.2)(4,0) (5,2.5)}node[right]{$y=f'(x)$};
		\end{tikzpicture}
	}
	\loigiai{
		\immini{Dựa vào đồ thị, ta có bảng biến thiên như hình vẽ. \\
		}{% Cần khai báo \usepackage{tkz-tab}
			\begin{tikzpicture}[scale=.8, font=\footnotesize, line join=round, line cap=round, >=stealth]
				\tkzTabInit[nocadre=false,lgt=1,espcl=2,deltacl=0.5]{$x$/.7 ,$y'$/.7,$y$/2}
				{$-\infty$ , $-1$ , $1$, $ 4 $, $+\infty$}
				\tkzTabLine{ , - , $0$ ,+, $ 0 $, -, $0$ , + , }
				\tkzTabVar{+/$+\infty$ , -/$f(-1)$ ,+/$f(-1)$ ,-/$ f(4) $, +/$+\infty$}
		\end{tikzpicture}}
	}
\end{ex} 

\begin{ex}
	\immini{
		Cho hàm số $y=f(x)$ xác định và liên tục trên $\mathbb{R}$. Biết rằng hàm số $f(x)$ có đạo hàm $f'(x)$ và hàm số $y=f'(x)$ có đồ thị như hình vẽ. Khi đó nhận xét nào sau đây đúng?
		\choice
		{\True Hàm số $f(x)$ không có cực trị}
		{Đồ thị hàm số $f(x)$ có đúng $2$ điểm cực tiểu}
		{Đồ thị hàm số $f(x)$ có đúng một cực đại}
		{Hàm số $f(x)$ có $3$ cực trị}
	}{
		\begin{tikzpicture}[scale=.8,font=\footnotesize, line join=round,line cap=round,>=stealth]
			\draw[->] (-2.5,0)--(2.5,0)node[below]{$x$};
			\draw[->] (0,-1)--(0,3.5)node[left]{$y$};
			\draw[samples=100,domain=-1.7:1.7] plot(\x,{(\x)^4-2*(\x)^2+1});
			\draw[dashed] (-1,0)node[below]{$-1$}circle(1pt) (1,0)node[below]{$1$}circle(1pt) (0,1)node[above right]{$1$}circle(1pt);
		\end{tikzpicture}
	}
	\loigiai{
		Dựa vào đồ thị ta thấy $f'(x)\geq 0$, với mọi $x\in\mathbb{R}$.\\
		Suy ra, hàm số $f(x)$ không có cực trị.
	}
\end{ex} 


\Closesolutionfile{ans}

\ind{PHẦN II.} \inden{Câu trắc nghiệm đúng sai. Trong mỗi ý a), b), c), d) ở mỗi câu, học sinh chọn đúng hoặc sai.}\\
\Opensolutionfile{ans}[ans/2D1-B1-d1-2]

\begin{ex}
	Cho hàm số $y=f(x)$ liên tục trên $\mathbb{R}$ và có bảng xét dấu đạo hàm như hình bên.
	\begin{center}
		\begin{tikzpicture}
			\tikzset{double style/.append style = {draw=\tkzTabDefaultWritingColor,double=\tkzTabDefaultBackgroundColor,double distance=2pt}}
			\tkzTabInit[nocadre=false, lgt=1, espcl=1.2]{$x$ /0.7,$y'$ /1}{$-\infty$,$0$,$1$,$2$,$+\infty$}
			\tkzTabLine{,+,$0$,-,d,+,$0$,+,}
		\end{tikzpicture}
	\end{center}
	% \immini{
		\choiceTF
		{Hàm số đồng biến trên khoảng $(-\infty;1)$}
		{\True Hàm số đồng biến trên khoảng $(1;+\infty)$}
		{Hàm số đạt cực đại tại $x=2$}
		{Hàm số có một điểm cực đại và hai điểm cực tiểu}
	% }{\vspace{0.1cm}
		%}
	\loigiai{
		Ta có bảng biến thiên như sau:
		\begin{center}
			\begin{tikzpicture}
				\tikzset{double style/.append style = {draw=\tkzTabDefaultWritingColor,double=\tkzTabDefaultBackgroundColor,double distance=2pt}}
				\tkzTabInit[lgt=1.1,espcl=2,nocadre=True]
				{$x$ /.7, $y'$ /.7,$y$ /2}
				{$-\infty$ , $0$, $1$ ,$2$ ,$+\infty$}
				\tkzTabLine{ ,+,$0$,-,d,+,$0$,+, }
				\tkzTabVar{ -/,+/ /,-/,R,+/$+\infty$}
			\end{tikzpicture}
		\end{center}
	Từ đây, suy ra:
		\begin{enumerate}[a)]
			\item Hàm số đồng biến trên khoảng $(-\infty;1)$ là khẳng định sai.
			\item Hàm số đồng biến trên khoảng $(1;+\infty)$ là khẳng định đúng.
			\item Hàm số đạt cực đại tại $x=2$ là khẳng định sai.
			\item Hàm số có một điểm cực đại và hai điểm cực tiểu là khẳng định sai.
		\end{enumerate}
	}
	
\end{ex} 

\begin{ex}
	Cho hàm số $y=x^3-3x^2+4$ có đồ thị $(C)$. Gọi $A$, $B$ là hai điểm cực trị của $(C)$.
	\choiceTF
	{\True Tập xác định của hàm số là $\mathbb{R}$}
	{Hàm số đồng biến trên khoảng $(0;2)$}
	{\True PTĐT qua hai điểm cực trị của đồ thị hàm số là $2x+y-4=0$}
	{\True Diện tích của tam giác $OAB$ bằng $4$, với $O$ là gốc tọa độ}
	\loigiai{
		\begin{enumerate}[a)]
			\item Hàm số đa thức nên có tập xác định là $D=\mathbb{R}$.
			\item Ta có 
			\begin{itemize}
				\item [$\bullet$] $y'=3x^2-6x$ và $y'=0 \Leftrightarrow x=0$ hoặc $x=2$.
			\end{itemize}
			Bảng biến thiên:
			\begin{center}
				\begin{tikzpicture}
					\tkzTabInit[lgt=1,espcl=3]
					{$x$ /0.7, $y'$ /0.7, $y$ /2.5}
					{$-\infty$,$0$,$2$,$+\infty$}
					\tkzTabLine{,+,$0$,-,$0$,+,}
					\tkzTabVar{-/$-\infty$,+/$4$,-/$0$,+/$+\infty$}
				\end{tikzpicture}
			\end{center}
		Suy ra hàm nghịch biến trên $(0;2)$.
			\item Tọa độ $A(0;4)$, $B(2;0)$. PTĐT $AB$ là
			$$\dfrac{x-0}{2-0}=\dfrac{y-4}{0-4} \Leftrightarrow 2x+y-4=0$$
			\item Diện tích tam giác vuông $OAB$ là $S_{OAB}=\dfrac{1}{2}OA \cdot OB=4$.
		\end{enumerate}

	}
\end{ex} 

\begin{ex}
	Cho hàm số $y=\dfrac{x^2+2x+2}{x+1}$ có đồ thị $(C)$. Gọi $A$, $B$ lần lượt là điểm cực tiểu và điểm cực đại của $(C)$.
	\choiceTF
	{Tập xác định của hàm số là $\mathbb{R}$}
	{Hàm số nghịch biến trên khoảng $(-2;0)$}
	{Tọa độ điểm $A(-2;-2)$, $B(0;2)$}
	{Khoảng cách giữa hai điểm cực trị là $AB=2\sqrt{5}$}
	\loigiai{
		\begin{enumerate}[a)]
			\item Đặt điều kiện mẫu số khác 0, ta được $x+1 \ne 0 \Leftrightarrow x \ne -1$. Suy ra $\mathscr{D}=\mathbb{R}\setminus \left\{-1\right\}$.
			\item $y'=\dfrac{x^2+2x}{(x+1)^2}\Rightarrow y'=0\Leftrightarrow \hoac{& x=-2 \\ & x=0.}$\\
			Ta có bảng xét dấu của hàm $f'(x)$ như sau
			\begin{center}
					\begin{tikzpicture}
					\tkzTabInit[nocadre=false,lgt=1,espcl=3]
					{$x$ /0.7,$y'$ /0.7,$y$ /2}
					{$-\infty$,$-2$,$-1$,$0$,$+\infty$}
					\tkzTabLine{,+,$0$,-,d,-,$0$,+,}
					\tkzTabVar{-/$-\infty$,+/$-2$,-D+/$-\infty$/$+\infty$,-/$2$,+/$+\infty$}
				\end{tikzpicture}
			\end{center}
			Dựa vào bảng xét dấu ta thấy rằng hàm số $y=f'(x)$ nghịch biến trên $(-2;-1)$ và $(-1;0)$.
			\item Tọa độ điểm $A(0;2)$, $B(-2;-2)$
			\item Độ dài $AB=\sqrt{(-2-0)^2+(-2-2)^2}=2\sqrt{5}$.
		\end{enumerate}

	}
\end{ex} 


\begin{ex}
	Xét một chất điểm chuyển động dọc theo trục $Ox$. Toạ độ của chất điểm tại thời điểm $t$ được xác định bởi hàm số $x(t)=t^3-6t^2+9t$ với $t\geq 0$. Khi đó $x'(t)$ là vận tốc của chất điểm tại thời điểm $t$, kí hiệu $v(t)$; $v'(t)$ là gia tốc chuyển động của chất điểm tại thời điểm $t$, kí hiệu $a(t)$.
	\choiceTF
	{Phương trình hàm vận tốc là $v(t)=3t^2-6t+9$}
	{\True Phương trình hàm gia tốc là $a(t)=6t-12$}
	{Vận tốc của chất điểm tăng khi $t\in (0;1)$ hoặc  $t \in (3;+\infty)$}
	{Vận tốc của chất điểm giảm khi $t\in (1;3)$}
	\loigiai{
		\begin{enumerate}
			\item $v(t)=x'(t)=3t^2-12t+9$
			\item $a(t)=v'(t)=6t-12$.
			\item Xét $v'(t)=6t-12$, $v'(t)=0\Leftrightarrow t=2$\\
			Bảng xét dấu
			\begin{center}
				\begin{tikzpicture}
					\tkzTabInit[nocadre=false,lgt=2,espcl=2.1]
					{$t$ /0.6,$v'(t)$ /0.6}
					{$0$,$2$,$+\infty$}
					\tkzTabLine{,-,$0$,+,}
				\end{tikzpicture}
			\end{center}
			Suy ra vận tốc của chất điểm tăng khi $t\in (2;+\infty) $, giảm khi $t\in (0;2)$.
		\end{enumerate}
	}
\end{ex} 

\Closesolutionfile{ans}
% \begin{dang}{Bài toán tìm m để hàm số đồng biến (nghịch biến) trên khoảng cho trước}
\begin{enumerate}[\iconCV]
\item Xét hàm số bậc ba $y=ax^3+bx^2+cx+d$ có $y'=3ax^2+2bx+c$.
	\begin{listEX}[1]
		\item [\ding{172}] Hàm số đồng biến trên  $\mathbb{R}$ khi và chỉ khi $$y' \ge 0,\,\forall x \in \mathbb{R} \Leftrightarrow \heva{&a>0\\&\Delta_{y'}\le 0}.$$
		\item [\ding{173}] Hàm số nghịch biến trên  $\mathbb{R}$ khi và chỉ khi $$y' \le 0, \,\forall x \in \mathbb{R} \Leftrightarrow \heva{&a<0\\&\Delta_{y'}\le 0}.$$
	\end{listEX}
\textit{Trường hợp hệ số $a$ có chứa tham số, ta kiểm tra thêm trường hợp $a=0$.}
\item Xét hàm phân thức $y=\displaystyle\frac{ax+b}{cx+d}$ có $y'=\dfrac{ad-cb}{(cx+d)^2}$, với $ad-cb \ne 0$ và $c \ne 0$.
\begin{itemize}
	\item [\ding{172}] Hàm số đồng biến trên từng khoảng xác định của nó khi và chỉ khi
	$$y'>0,\, \forall x \ne -\dfrac{d}{c}\Leftrightarrow ad-cb>0.$$
	\item [\ding{173}]  Hàm số nghịch biến trên từng khoảng xác định của nó khi và chỉ khi
	$$y'<0,\, \forall x \ne -\dfrac{d}{c}\Leftrightarrow ad-cb<0.$$
\end{itemize}
\item Xét hàm phân thức $y=\displaystyle\frac{ax^2+bx+c}{dx+e}$ có $y'=\dfrac{adx^2+2aex+be-dc}{(dx+e)^2}$, với $ad \ne 0$.
\begin{itemize}
	\item [\ding{172}] Hàm số đồng biến trên từng khoảng xác định của nó khi và chỉ khi
	$$y'\ge 0,\, \forall x \ne -\dfrac{e}{d}\Leftrightarrow adx^2+2aex+be-dc\ge 0,\, \forall x \ne -\dfrac{e}{d}.$$
	\item [\ding{173}]  Hàm số nghịch biến trên từng khoảng xác định của nó khi và chỉ khi
	$$y'\le 0,\, \forall x \ne -\dfrac{e}{d}\Leftrightarrow adx^2+2aex+be-dc\le 0,\, \forall x \ne -\dfrac{e}{d}.$$
\end{itemize}
\end{enumerate}
\end{dang}
\boxmini{BÀI TẬP TỰ LUẬN}
\setcounter{vd}{0}

\begin{vd}
	Tìm tất cả giá trị của tham số $m$ để hàm số
	\begin{tasks}
		\task $y=x^3+mx^2+2mx+2$ đồng biến trên $(-\infty;+\infty)$.
		\task $y=-\dfrac{1}{3}x^3-mx^2+\left(2m-3\right)x-m+2$ nghịch biến trên $\mathbb{R}$.
		\task $ y=\dfrac{1}{3}x^3-mx^2-(2m+1)x+1$ nghịch biến trên khoảng $(0;5)$.
		\task $y=x^3-3x^2+(5-m)x$ đồng biến trên khoảng $(2;+\infty)$.
	\end{tasks}
\loigiai{
\begin{enumerate}[a)]
	\item Hàm số đã cho có tập xác định $\mathscr{D}=\mathbb{R}$ và $y'=3x^2+2mx+2m$.\\
	Hàm số đã cho đồng biến trên $\mathbb{R}$ khi và chỉ khi
	\[y'\ge0,~\forall x\in\mathbb{R}\Leftrightarrow m^2-6m\le0\Leftrightarrow 0\le m\le6.\]
	\item Tập xác định: $D=\mathbb{R}$. Ta có $y'=-x^2-2mx+2m-3$.\\
	Để hàm số nghịch biến trên $\mathbb{R}$ thì:\\
	$y'\le 0,\forall x\in\mathbb{R} \Leftrightarrow\left\{
	\begin{aligned}
		&a_{y'}<0\\
		&\Delta'\le 0
	\end{aligned}
	\right.
	\Leftrightarrow \left\{
	\begin{aligned}
		&-1<0\\
		&m^2+2m-3\le0
	\end{aligned}
	\right.
	\Leftrightarrow -3\le m\le 1$.
	\item Tập xác định $\mathscr{D}=\mathbb{R}$.\\
	Ta có $y'=x^2-2mx-(2m+1)$, $ y'=0\Leftrightarrow\hoac{&x=-1\\&x=2m+1.}$\\
	Nếu $2m+1\leq-1\Leftrightarrow m\leq-1$ thì $y'\leq 0\Leftrightarrow x\in\left[2m+1;-1\right]$.\\
	Suy ra hàm số không nghịch biến trên khoảng $(0;5)$. \\
	$\Rightarrow m\leq-1$ không thỏa mãn.\\
	Nếu $2m+1>-1\Leftrightarrow m>-1$ thì $y'\leq 0\Leftrightarrow x\in\left[-1;2m+1\right]$.\\
	Để hàm số nghịch biến trên khoảng $(0;5)$ thì ta có $2m+1\geq 5\Leftrightarrow m\geq 2$.
	\item \textbf{\underline{Cách 1:}} Tập xác định $\mathscr{D}=\mathbb{R}$.\\
	Ta có $y'=3x^2-6x+5-m$.\\
	Hàm số $y=x^3-3x^2+(5-m)x$ đồng biến trên khoảng $(2;+\infty)$ khi và chỉ khi
	\allowdisplaybreaks
	\begin{eqnarray*}
		&&y'\ge 0,\,\forall x\in (2;+\infty)\\
		&\Leftrightarrow& 3x^2-6x+5-m\ge 0,\,\forall x\in (2;+\infty)\\
		&\Leftrightarrow& m\le 3x^2-6x+5, \,\forall x\in (2;+\infty)
	\end{eqnarray*}
	Xét hàm $g(x)=3x^2-6x+5$ trên $(2;+\infty)$ có $g'(x)=6x-6$ và $g'(x)=0\Leftrightarrow x=1$.\\
	Bảng biến thiên của $g(x)$
	\begin{center}
		\begin{tikzpicture}
			\tkzTabInit[nocadre=false,lgt=1.5,espcl=2,deltacl=0.5]
			{$x$/0.6,$g'(x)$/0.6,$g(x)$/1.5}
			{$2$,$+\infty$}
			\tkzTabLine{,+,}
			\tkzTabVar{-/$5$,+/$+\infty$}
		\end{tikzpicture}
	\end{center}
	Dựa vào bảng biến thiên của $g(x)$, ta được
	$$m\le 3x^2-6x+5, \,\forall x\in (2;+\infty) \Leftrightarrow m\le 5.$$
	\textbf{\underline{Cách 2:}} Tập xác định $\mathscr{D}=\mathbb{R}$.\\
	Ta có $y'=3x^2-6x+5-m$.\\
	Hàm số $y=x^3-3x^2+(5-m)x$ đồng biến trên khoảng $(2;+\infty)$ khi và chỉ khi
	$$y'\ge 0,\,\forall x\in (2;+\infty) 
	\Leftrightarrow \heva{& y'(2)\ge 0 \\ & -\dfrac{b}{2a} \le 2} 
	\Leftrightarrow \heva{& 5-m\ge 0 \\ & 1 \le 2}
	\Leftrightarrow m \le 5. $$
\end{enumerate}}
\end{vd}

\begin{vd}
	Tìm tất cả giá trị của tham số $m$ để hàm số
	\begin{tasks}
		\task $y=\dfrac{mx+2}{x+1}$ đồng biến trên từng khoảng xác định.
		\task $y=\dfrac{mx-2}{x+m-3}$ nghịch biến trên các khoảng xác định
		\task $y = \dfrac{mx-8}{x-2m}$ đồng biến trên $(3;+\infty )$.
		\task $y=\dfrac{mx+9}{4x+m}$ nghịch biến trên khoảng $(0;4)$.
	\end{tasks}
\loigiai{
\begin{enumerate}[a)]
	\item Từ yêu cầu bài toán, $\forall x \neq -1$ ta xét $y'>0$ $\Leftrightarrow m-2>0 \Leftrightarrow m>2$.
	\item Tập xác định $\mathbb{R}\setminus\{3-m\}$.\\
	$y' = \dfrac{m(m - 3) + 2}{\left( x + m - 3\right)^2} = \dfrac{m^2 - 3m + 2}{\left(x + m - 3\right)^2}$. \\
	Điều kiện để hàm số nghịch biến trên các khoảng xác định của nó là $y' < 0,\,\forall x \ne 3 - m$ hay $m^2 - 3m + 2 < 0 \Leftrightarrow m \in (1;2)$.
	\item Tập xác định: $\mathscr{D} = \mathbb{R} \setminus \{2m\}$.\\
	$y' = \dfrac{-2m^2+8}{(x-2m)^2}$.\\
	Hàm số luôn đơn điệu trên từng khoảng xác định $(-\infty; 2m)$ và $(2m; +\infty)$ khi $-2m^2 + 8 \ne 0$.\\
	Vậy hàm số đồng biến trên $(3;+\infty)$ khi và chỉ khi $-2m^2+8 > 0$ và $(3;+\infty) \subset (2m ;+\infty)$. \\
	Điều này tương đương $\heva{&-2<m<2\\&2m \le 3}$, hay $-2 < m \le \dfrac{3}{2}$.
	\item Tập xác định $\mathscr{D}=\mathbb{R}\setminus\left\{-\dfrac{m}{4}\right\}$.\\
	Ta có $y=\dfrac{mx+9}{4x+m}\Rightarrow y'=\dfrac{m^2-36}{(4x+m)^2}$.\\
	Để hàm số nghịch biến trên khoảng $(0;4)$ thì
	$$\heva{& y'<0 ,\forall x\in(0;4)\\ & -\dfrac{m}{4}\notin (0;4)}\Leftrightarrow\heva{& m^2-36<0 \\ &\hoac{&-\dfrac{m}{4}\geq4\\&-\dfrac{m}{4}\leq 0}}\Leftrightarrow\heva{& -6<m<6 \\ &\hoac{&m\leq-16\\&m\geq 0}}\Leftrightarrow 0\leq m<6.$$
\end{enumerate}}
\end{vd}

\begin{vd}
	Tìm tất cả giá trị của tham số $m$ để hàm số
	\begin{tasks}
		\task $ y = \dfrac{2x^2+3x+m+1}{x+1} $ đồng biến trên các khoảng xác định.
		\task $y=\dfrac{x^2+(m+1)x-1}{2-x}$ ($m$ là tham số) nghịch biến trên mỗi khoảng xác định.
	\end{tasks}
	\loigiai{
		\begin{enumerate}[a)]
			\item Tập xác định: $\mathbb{R}\setminus\{-1\}$.\\
			Ta có $y'=\dfrac{2x^2+4x+2-m}{(x+1)^2}$. Hàm số đồng biến trên các khoảng xác định khi 
			$$2x^2+4x+2-m\ge 0, \forall x\in \mathbb{R} \Leftrightarrow m\le \min\limits{\mathbb{R}\setminus \{-1\} } (2x^2+4x+2) = 0.$$
			\item Tập xác định $\mathscr{D}=\mathbb{R}\backslash\{2\}$.\\
			Đạo hàm: $y'=\dfrac{-x^2+4x+2m+1}{(2-x)^2}=\dfrac{g(x)}{(2-x)^2}$.\\
			Hàm số nghịch biến trên mỗi khoảng xác định của nó khi và chỉ khi $y'\le 0,\forall x\in \mathscr{D}$ (Dấu \lq\lq $=$\rq\rq~ chỉ xảy ra tại hữu hạn điểm thuộc $\mathscr{D}$).\\
			$\Leftrightarrow g(x)=-x^2+4x+2m+1\le 0,$  $\forall x\in \mathbb{R}$\\
			Điều kiện: ${\Delta}'\le 0$ (vì $a=-1<0$) $\Leftrightarrow 4-(-1)\cdot(2m+1)\le 0\Leftrightarrow 2m+5\le 0\Leftrightarrow m\le -\dfrac{5}{2}$.
	\end{enumerate}}
\end{vd}

\boxmini{BÀI TẬP TRẮC NGHIỆM}
\ind{PHẦN I.} \inden{Câu trắc nghiệm nhiều phương án lựa chọn. Học sinh trả lời từ câu 1 đến câu 17. Mỗi câu hỏi học sinh chỉ chọn một phương án.}\\
\setcounter{ex}{0}
\Opensolutionfile{ans}[ans/2D1-B1-d2-1]

\begin{ex}%[Nguyễn Trung Kiên, dự án 12-EX-7-2020]%[2D1B1-3]%
	Tất cả giá trị của $m$ để hàm số $y=\dfrac{x+m}{x-2}$ nghịch biến trên từng khoảng xác định là
	\choice
	{\True $m>-2$}
	{$m<-2$}
	{$m\leq -2$}
	{$m\geq -2$}
	\loigiai
	{Tập xác định $\mathscr{D}=\mathbb{R}\setminus \{2\}$ và $y'=\dfrac{-2-m}{(x-2)^2}$.\\
		Hàm số nghịch biến trên các khoảng $(-\infty;2)$ và $(2;+\infty)$ khi và chỉ khi
		\[y'<0,\, \forall x\neq 2\Leftrightarrow -2-m<0 \Leftrightarrow m>-2.\]}
\end{ex} 

\begin{ex}
	Cho hàm số $y=\dfrac{mx-2}{x+1-m}$. Tìm tất cả giá trị của tham số $m$ để hàm số đồng biến trên từng khoảng xác định.
	\choice
	{$\hoac{& m> 2\\& m< -1}$}
	{\True $-1<m<2$}
	{$-1\le m\le 2$}
	{$\hoac{& m\ge 2\\ &m\le -1}$}
	\loigiai{
		Yêu cầu bài toán $\Leftrightarrow ad-bc>0 \Leftrightarrow m(1-m)+2>0 \Leftrightarrow -1<m<2$.
	}
\end{ex} 

\begin{ex}
	Cho hàm số $ y=\dfrac{x+m}{x+2} $. Tập hợp tất cả các giá trị của $ m $ để hàm số đồng biến trên khoảng $ \left(0;+\infty\right)  $ là
	\choice
	{$ \left[2;+\infty\right) $}
	{$ \left(2;+\infty\right)  $}
	{$ \left(-\infty;2\right ]  $}
	{\True $\left(-\infty;2\right)   $}
	\loigiai{
		Hàm số xác định khi $ x\ne -2. $\\
		Có $ y'=\dfrac{2-m}{\left(x+2\right)^2 }, x\ne -2 $.\\
		Hàm số đồng biến trên $ (0;+\infty) $ khi và chỉ khi $ 2-m>0\Leftrightarrow m<2. $
	}
\end{ex} 

\begin{ex}
	Cho hàm số $f(x)=\dfrac{mx-4}{x-m}$ ( $m$ là tham số thực). Có bao nhiêu giá trị nguyên của $m$ để hàm số đồng biến trên khoảng $\left( 0;+\infty  \right)$?  
	\choice
	{$5$}
	{$4$}
	{$3$}
	{\True  $2$}
	\loigiai{
		Ta có $f'(x)=\dfrac{-m^2+4}{{{\left( x-m \right)}^{2}}}$\\
		Hàm số đồng biến trên khoảng $\left( 0;+\infty  \right)$ $\Leftrightarrow$ $\dfrac{-m^2+4}{\left( x-m \right)^2}>0,\,\, \forall x\in \left( 0;+\infty  \right)$\\
		$\Rightarrow \heva{
			& -m^2+4>0 \\ 
			& x\ne m\ \ \forall x\in \left( 0;+\infty  \right) \\ 
		}\Leftrightarrow \heva{
			& m\in \left( -2;2 \right) \\ 
			& m\in \left( -\infty ;0 \right] \\ 
		}\Leftrightarrow m\in \left( -2;0 \right]$\\
		Vậy có hai giá trị nguyên của $m$ là $-1$ và $0$.      
	}
\end{ex} 

\begin{ex}
	Tìm tất cả các giá trị của $m$ để hàm số $y=\dfrac{mx+4}{x+m}$ nghịch biến trên $(-\infty;1)$.
	\choice
	{$-2<m<2$}
	{$-2<m <-1$}
	{$-2\leq m <-1$}
	{\True $-2<m\leq-1$}
	\loigiai{
		ĐKXĐ: $x\neq-m$.\\
		Hàm số $y=\dfrac{mx+4}{x+m}$ nghịch biến trên $(-\infty;1)$\\$\Leftrightarrow y'=\dfrac{m^2-4}{(x+m)^2}<0$, $\forall x\in(-\infty;1)$
		$ \Leftrightarrow\heva{&m^2-4<0\\&-m\geq 1}\Leftrightarrow\heva{&-2<m<2\\&m\leq-1}\Leftrightarrow-2<m\leq-1 $.}
\end{ex} 

\begin{ex}%[THPT Tĩnh Gia - Thanh Hóa, 2020]%[Bùi Mạnh Tiến, 12EX7]%[2D1B1-3]%
	Số giá trị nguyên của tham số $m$ để hàm số $y=\dfrac{mx+10}{2x+m}$ nghịch biến trên khoảng $(0;2)$ là
	\choice
	{\True $6$}
	{$5$}
	{$4$}
	{$9$}
	\loigiai
	{
		Ta có $y'=\dfrac{m^2-20}{(2x+m)^2}$.\\
		Do đó hàm số $y=\dfrac{mx+10}{2x+m}$ nghịch biến trên $(0;2)$ khi và chỉ khi
		\begin{align*}
			\heva{& m^2-20<0 \\ & -\dfrac{m}{2}\notin (0;2)}\Leftrightarrow \heva{& -2\sqrt{5}<m<2\sqrt{5} \\ & \hoac{& -\dfrac{m}{2}\le 0 \\ & -\dfrac{m}{2}\ge 2}}\Leftrightarrow \hoac{& 0\le m<2\sqrt{5} \\ & -2\sqrt{5}<m\le -4.}
		\end{align*}
		Vì $m\in \mathbb{Z}$ nên $m\in \left\{-4;0;1;2;3;4\right\}$.\\
		Vậy có tất cả $6$ giá trị nguyên của $m$ thỏa mãn yêu cầu bài toán.
	}
\end{ex} 

\begin{ex}
	Có bao nhiêu giá trị nguyên của tham số $m$ để hàm số $y=x^3-2mx^2+\left(m^2+3\right)x$ đồng biến trên $\mathbb{R}$?
	\choice
	{$8$}
	{$6$}
	{\True $7$}
	{$0$}
	\loigiai{
		Hàm số $y=x^3-2mx^2+\left(m^2+3\right)x$ đồng biến trên $\mathbb{R}$
		\begin{eqnarray*}
			&\Leftrightarrow &y'=3x^2-4mx+m^2+3\ge 0, \, \forall x\in \mathbb{R}\\
			&\Leftrightarrow & \Delta'=4m^2-3\left(m^2+3\right)\le 0\\
			&\Leftrightarrow & m^2-9\le 0\Leftrightarrow-3\le m\le 3.
		\end{eqnarray*}
		Do $m$ là số nguyên nên $m\in \left\lbrace -3;-2;-1;0;1;2;3\right\rbrace $.\\
		Vậy có $7$ giá trị nguyên của tham số $m$.
	}
\end{ex} 

\begin{ex}
	Cho hàm số $y=-x^3-mx^2+(4m+9)x+5$. Có bao nhiêu giá trị nguyên của $m$ để hàm số nghịch biến trên $\mathbb{R}$?
	\choice
	{\True $7$}
	{$4$}
	{$5$}
	{$6$}
	\loigiai{
		Ta có $y'=-3x^2-2mx+(4m+9)$. Hàm số đã cho nghịch biến trên $\mathbb{R}$ khi và chỉ khi
		\[ \Delta'\le 0 \Leftrightarrow m^2+12m+27\le 0 \Leftrightarrow -9\le m\le -3. \]
		Vậy có tất cả $7$ giá trị nguyên của $m$ thỏa mãn bài toán.
	}
\end{ex} 

\begin{ex}
	Cho hàm số $y=(m-1)x^3 + (m-1)x^2 -2x+5$ với $m$ là tham số. Có bao nhiêu giá trị nguyên của $m$ để hàm số nghịch biến trên khoảng $(-\infty;+\infty)$?
	\choice
	{$5$}
	{\True $7$}
	{$8$}
	{$6$}
	\loigiai{
		\textbf{Trường hợp 1:} $m-1=0 \Leftrightarrow m=1$ khi đó $y=-2x+5$ nghịch biến trên $\mathbb{R}$. Do đó nhận $m=1$.\\
		\textbf{Trường hợp 2:} $m-1\ne 0 \Leftrightarrow m\ne 1$.\\
		Ta có $y'=3(m-1)x^2+2(m-1)x-2$. \\
		Hàm số nghịch biến trên $(-\infty;+\infty) $ $\Leftrightarrow y' \le 0 $, $\forall x\in (-\infty;+\infty)$
		$$\Leftrightarrow \heva{& 3(m-1)<0 \\ & (m-1)^2-3(m-1)\cdot (-2) \le 0} \Leftrightarrow \heva{& m<1 \\ & -5 \le m \le 1} \Leftrightarrow -5 \le m <1.$$.\\
		Do $m \in \mathbb{Z} \Rightarrow m\in \{-5;-4;-3;-2;-1;0\}$.\\
		Vậy cả $2$ trường hợp thì ta có tất cả $7$ giá trị $m$ thỏa yêu cầu bài toán là $\{-5;-4;-3;-2;-1;0;1\}$.
	}
\end{ex} 

\begin{ex}
	Tìm tất cả các giá trị thực của tham số $m$ để hàm số $y=x^3-3mx^2-9m^2x$ nghịch biến trên khoảng $(0;1)$.
	\choice
	{$-1<m<\dfrac{1}{3}$}
	{$m<-1$}
	{$m>\dfrac{1}{3}$}
	{\True $m\ge \dfrac{1}{3}$ hoặc $m\le -1$}
	\loigiai{
		Đặt $f(x)=y'=3x^2-6mx -9m^2$.\\
		Vì $y'$ là hàm số bậc hai với hệ số $a=3>0$ nên để hàm số nghịch biến trên $(0;1)$ thì phương trình $y'=0$ có hai nghiệm phân biệt $x_1, x_2$ thỏa mãn $x_1\le 0<1 \le x_2$ $$\Leftrightarrow \heva{&af(0)\le 0\\&af(1) \le0} \Leftrightarrow \heva{&-9m^2\le 0\\&3-6m-9m^2 \le 0} \Leftrightarrow \hoac{&x\le -1\\&x\ge \dfrac{1}{3}.}$$
	}
\end{ex} 

\begin{ex}
	Có bao nhiêu giá trị nguyên của tham số $ m$ thuộc khoảng $( -2019;2020 )$ để hàm số $ y=2x^3-3( 2m+1 )x^2+6m(m+1)x+2019$ đồng biến trên khoảng $(2;+\infty )$?
	\choice
	{\True $2020$}
	{$2018$}
	{$2021$}
	{$2019$}
	\loigiai{
		Ta có $y'=6x^2-6(2m+1)x+6m^2+6m$.\\
		Xét $y'=0$ $\Leftrightarrow x^2-(2m+1)x+m^2+m=0$, có $\Delta =(2m+1)^2-4\left( m^2+m \right)$ $=1>0$, $\forall m\in \mathbb{R}$. Suy ra phương trình $y'=0$ luôn có hai nghiệm phân biệt: $x_1=m$; $x_2=m+1$. Dễ thấy $x_1<x_2$.\\
		Bảng biến thiên
		\begin{center}
			\begin{tikzpicture}
				\tkzTabInit[nocadre=true,lgt=0.7,espcl=2.1]
				{$x$ /0.6,$y'$ /0.6,$y$ /2}
				{$-\infty$,$m$,$m+1$,$+\infty$}
				\tkzTabLine{,+,$0$,-,$0$,+,}
				\tkzTabVar{-/$-\infty$, +/$y(m)$,-/$y(m+1)$,+/$+\infty$}
			\end{tikzpicture}
		\end{center}
		Dựa vào bảng biến thiên ta thấy hàm số đồng biến trên mỗi khoảng $( -\infty ;m )$; $( m+1;+\infty )$. Vì thế, hàm số đồng biến trên $( 2:+\infty )$ khi $ m+1\le 2\Leftrightarrow m\le 1$.\\
		Suy ra có $2020$ giá trị nguyên của $ m$ thỏa mãn yêu cầu đề bài. }
\end{ex} 

\begin{ex}
	Tập hợp các giá trị thực của tham số $m$ để hàm số $y = - x^3 - 6x^2 + \left(4m - 9\right)x + 4$ nghịch biến trên khoảng $\left(- \infty; - 1\right)$ là
	\choice
	{$\left(- \infty; 0\right]$}
	{$\left[-\dfrac{3}{4}; +\infty\right)$}
	{\True $\left(- \infty; -\dfrac{3}{4}\right]$}
	{$\left[0; +\infty \right)$}
	\loigiai{ 
		Ta có $y'=-3x^2-12x+4m-9$. \\
		Hàm số đã cho nghịch biến trên khoảng $(-\infty;-1)$ khi và chỉ khi $y'\le 0$, $\forall x\in (-\infty;-1)$
		\begin{center}
			$\Leftrightarrow -3x^2-12x+4m-9\le 0\Leftrightarrow 4m\le 3x^2+12x+9$, $\forall x\in (-\infty;-1)$.
		\end{center}
		Đặt $g(x)=3x^2+12x+9\Rightarrow g'(x)=6x+12$. Giải $g'(x)=0\Leftrightarrow x=-2$.\\
		Bảng biến thiên của hàm số $g(x)$ trên $(-\infty;-1)$.
		\begin{center}
			\begin{tikzpicture}
				\tkzTabInit[nocadre=false,lgt=2,espcl=3.5,deltacl=0.6] %phần bắt buộc
				{$x$ /0.6,$g'(x)$ /0.6,$g(x)$ /2}%phần bắt buộc
				{$-\infty$,$-2$,$-1$}
				\tkzTabLine{,-,$0$,+,}
				\tkzTabVar{+/$+\infty$, -/$-3$,+/$0$}
			\end{tikzpicture}
		\end{center}
		Dựa vào bảng biến thiên suy  ra $4m\le g(x)$, $\forall x\in (-\infty;-1)\Leftrightarrow 4m\le -3\Leftrightarrow m\le -\dfrac{3}{4}$.
	}
\end{ex} 

\begin{ex}
	Tìm tất cả các giá trị thực của tham số $m$ sao cho hàm số $y=x^3-6x^2+mx+1$ đồng biến trên khoảng $\left(0;+\infty\right)$.
	\choice
	{$m\leq 12$}
	{\True $m\geq 12$}
	{$m\leq 0$}
	{$m\geq 0$}
	\loigiai{
		Tập xác định $\mathscr{D} =\mathbb{R}$.\\
		$y'=3x^2-12x+m$.\\
		Hàm số đồng biến trên khoảng $\left(0;+\infty\right)$ khi và chỉ khi
		{\allowdisplaybreaks
			\begin{eqnarray*}
				& & f'(x)\geq 0 , \forall x\in \left(0;+\infty\right) \\
				& \Leftrightarrow & 3x^2-12x+m \geq 0 , \forall x\in \left(0;+\infty\right) \\
				& \Leftrightarrow & m \geq -3x^2+12x , \forall x\in \left(0;+\infty\right).
		\end{eqnarray*}}
		Xét hàm số $g(x)= -3x^2+12x$ trên $\left(0;+\infty\right)$.
		Ta có $g'(x)=-6x+12 \Leftrightarrow x=2$.\\
		Bảng biến thiên của hàm số $g(x)$
		\begin{center}
			\begin{tikzpicture}
				\tkzTabInit[lgt=1.2,espcl=3]{$x$ /1, $y'$ /1,$y$ /2}{
					$0$,$2$,$+\infty$}
				\tkzTabLine{,+,0 ,-, }
				\tkzTabVar{-/$0$, +/$12$ ,-/$-\infty$ }
			\end{tikzpicture}
		\end{center}
		Suy ra hàm số đồng biến trên khoảng $\left(0;+\infty\right)$ khi $m \geq 12$.
	}
\end{ex} 

\begin{ex}
	Tìm tất cả các giá trị $m$ để hàm số $y=\dfrac{x^2-8x}{x+m}$ đồng biến trên mỗi khoảng xác định.
	\choice
	{$(-8;0)$}
	{$(0;8)$}
	{$[0;8]$}
	{\True $[-8;0]$}
	\loigiai{
		Ta có $y'=\dfrac{x^2+2mx-8m}{(x+m)^2}$. Khi đó
		\allowdisplaybreaks
		\begin{eqnarray*}
			\text{YCBT} &\Leftrightarrow & x^2+2mx-8m\ge 0, \forall x \Leftrightarrow \Delta' \le 0\\
			&\Leftrightarrow & m^2+8m\le 0\Leftrightarrow -8\le m\le 0.
		\end{eqnarray*}
	}
\end{ex} 

\begin{ex}
	Tập hợp các giá trị thực của tham số $m$ để hàm số $y=x+1+\dfrac{m}{x-2}$ đồng biến trên mỗi khoảng xác định của nó là
	\choice
	{$\left(-\infty;0\right)$}
	{$\left[0;1\right)$}
	{$\left[0;+\infty \right)\backslash \left\{1\right\}$}
	{\True $\left(-\infty;0\right]$}
	\loigiai{
		Tập xác định $\mathscr{D}=\mathbb{R}\backslash \left\{2\right\}$.
		Ta có $y'=1-\dfrac{m}{\left(x-2\right)^2}$.\\
		Hàm số đồng biến trên mỗi khoảng các định của nó khi và chỉ khi
		\begin{eqnarray*}
			&&y'\geq 0,\;\forall x\in \mathbb{R}\backslash \left\{2\right\}\Leftrightarrow 1-\dfrac{m}{\left(x-2\right)^2}\geq 0,\;\forall x\in \mathbb{R}\backslash \left\{2\right\}\\
			&\Leftrightarrow &m\le {\left(x-2\right)}^2,\;\forall x\in \mathbb{R}\backslash \left\{2\right\}\Leftrightarrow m\leq 0.
		\end{eqnarray*}
	}
\end{ex} 

\begin{ex}%[2D1K1-3]%
	Tìm tất cả các giá trị thực của tham số $ m $ để hàm số $ f(x)=2^{x^3-x^2+mx+1}$ đồng biến trên khoảng $(1; 2)$.
	\choice
	{$m\leq-8$}
	{$m>-8$}
	{\True $m\geq-1$}
	{$m<-1$}
	\loigiai{
		Ta có $ f'(x)=(3x^2-2x+m)\cdot 2^{x^3-x^2+mx+1}\cdot\ln 2 $.\\
		Ta thấy\allowdisplaybreaks{
			\begin{eqnarray*}
				&& f(x) \textrm{ đồng biến trên } (1; 2)\\
				\Leftrightarrow && (3x^2-2x+m)\cdot 2^{x^3-x^2+mx+1}\cdot\ln 2\geq 0,\forall x\in (1; 2)\\
				\Leftrightarrow && (3x^2-2x+m)\geq 0,\forall x\in (1; 2)\\
				\Leftrightarrow && m\geq (-3x^2+2x),\forall x\in (1; 2)\\
				\Leftrightarrow && m\geq\max\limits_{[1; 2]} (-3x^2+2x)\\
				\Leftrightarrow && m\geq-1.
			\end{eqnarray*}
		}
	}
\end{ex} 

\begin{ex}
	Có bao nhiêu giá trị nguyên dương của tham số $m$ để hàm số $f(x)=(x+1)\ln x+(2-m)x$ đồng biến trên khoảng $(0;\mathrm{e}^2)$?
	\choice
	{0}
	{3}
	{5}
	{\True 4}
	\loigiai
	{Hàm số đã cho xác định khi $x>0$ hay $D=\big(0;+\infty\big)$\\
			Với $x>0$, ta có $f'(x)=\ln x+\dfrac{x+1}{x}+2-m$.\\
			Hàm số đã cho đồng biến trên khoảng $(0;\mathrm{e}^2)$ khi
			\allowdisplaybreaks
			\begin{align*}
				f'(x) \geq 0, \forall x \in (0;\mathrm{e}^2) &\Leftrightarrow \ln x+\dfrac{x+1}{x}+2-m \geq 0, \forall x \in (0;\mathrm{e}^2)\\
				&\Leftrightarrow m \leq \ln x+\dfrac{x+1}{x}+2, \forall x \in (0;\mathrm{e}^2). \tag{$*$}
			\end{align*}
			Xét hàm số $g(x)=\ln x+\dfrac{x+1}{x}+2, \forall x \in (0;\mathrm{e}^2)$.\\
			Ta có $g'(x)=\dfrac{1}{x}-\dfrac{1}{x^2}=\dfrac{x-1}{x^2}$. Khi đó $g'(x)=0$ có nghiệm $x=1 \in (0;\mathrm{e}^2)$.\\
			Bảng biến thiên của hàm số $g$
			\begin{center}
				\begin{tikzpicture}
					\tkzTabInit[nocadre=false,lgt=1.5,espcl=3.5,deltacl=0.6] %phần bắt buộc
					{$x$/0.6, $g'(x)$/0.6, $g(x)$/2} %phần bắt buộc
					{$0$, $1$, $\mathrm{e}^2$} % hàng 1 cột 2
					\tkzTabLine{,-,z,+,}
					\tkzTabVar{+/$+\infty$,-/$4$,+/$g(\mathrm{e}^2)$}
				\end{tikzpicture}
			\end{center}
			Từ bảng biến thiên trên, bất phương trình $(*)$ thỏa mãn khi $m \leq 4$.
	}
\end{ex} 


\Closesolutionfile{ans}

\ind{PHẦN II.} \inden{Câu trắc nghiệm đúng sai. Học sinh trả lời từ câu 18 đến câu 20. Trong mỗi ý a), b), c), d) ở mỗi câu, học sinh chọn đúng hoặc sai.}\\
	
\Opensolutionfile{ans}[ans/2D1-B1-d2-2]

\begin{ex}
	Cho hàm số $ y=mx^3+mx^2-(m+1)x+1 $, với $m$ là tham số.
	\choiceTF
	{\True Hàm số là hàm số bậc ba khi $m \ne 0$}
	{\True Tập xác định của hàm số là $\mathbb{R}$}
	{Hàm số đồng biến trên $\mathbb{R}$ khi và chỉ khi $m<-\dfrac{3}{4}$ hoặc $m \ge 0$}
	{Hàm số nghịch biến trên $\mathbb{R}$ khi và chỉ khi $-\dfrac{3}{4}\leq m<0$}
	\loigiai{
		\begin{enumerate}[a)]
			\item Với $m \ne 0$ thì hàm số đã cho là một hàm số bậc ba.
			\item Hàm số là hàm đa thức nên có tập xác định là $\mathbb{R}$.
			\item Ta có $ y'=3mx^2+2mx-(m+1)$.
			\begin{itemize}
				\item [$\bullet$] Với $m=0$ thì $y'=-1<0$ (không thỏa)
				\item [$\bullet$] Với $m \ne 0$, yêu cầu bài toán tương đương với
				$\heva{&m>0\\&\Delta \le 0} \Leftrightarrow \heva{&m>0\\&4m^2+3m \le 0}$ (không tồn tại $m$)
			\end{itemize}
			\item 
			\begin{itemize}
				\item [$\bullet$] Với $m=0$ thì $y'=-1<0$ (thỏa)
				\item [$\bullet$] Với $m \ne 0$, yêu cầu bài toán tương đương với
				$$\heva{&m<0\\&\Delta \le 0} \Leftrightarrow \heva{&m<0\\&4m^2+3m \le 0} \Leftrightarrow -\dfrac{3}{4}\leq m<0$$
			\end{itemize}
		Suy ra $-\dfrac{3}{4}\leq m \leq 0$.
		\end{enumerate}
	}
\end{ex} 

\begin{ex}
	Cho hàm số $y=\dfrac{1}{3}x^3 + (m + 1)x^2 + \left(m^2 + 2m\right)x - 3$, với $m$ là tham số.
	\choiceTF
	{Tập xác định của hàm số là $\mathbb{R}$}
	{\True Phương trình $y'=0$ có hai nghiệm phân biệt $x_1=-m$ và $x_2=-m-2$}
	{\True Không tồn tại giá trị của tham số $m$ để hàm số đồng biến trên $\mathbb{R}$}
	{Hàm số nghịch biến trên khoảng $(- 1; 1)$ khi và chỉ khi $m \ge -1$}
	\loigiai{
		\begin{enumerate}[a)]
			\item Hàm số là hàm đa thức nên có tập xác định là $\mathbb{R}$
			\item Ta có $y'=x^2+2(m+1)x+m^2+2m$. Do $\Delta'=b'^2-ac=(m+1)^2-(m^2+2m)=1>0$ nên phương trình có hai nghiệm phân biệt
			$x_1=\dfrac{-b'+\sqrt{\Delta'}}{a}=-m$ và $x_2=\dfrac{-b'-\sqrt{\Delta'}}{a}=-m-2$.
			\item Bảng biến thiên
				\begin{center}
					\begin{tikzpicture}
						\tkzTabInit[lgt=1,espcl=3,nocadre=True]
						{$x$ /0.7, $y'$ /0.7, $y$ /2.5}
						{$-\infty$,$-m-2$,$-m$,$+\infty$}
						\tkzTabLine{,+,$0$,-,$0$,+,}
						\tkzTabVar{-/$-\infty$,+/$y(-m-2)$,-/$y(-m)$,+/$+\infty$}
					\end{tikzpicture}
				\end{center}
			Từ bảng biến thiên, suy ra không tồn tại giá trị của tham số $m$ để hàm số đồng biến trên $\mathbb{R}$
			\item Bảng biến thiên
			\begin{center}
				\begin{tikzpicture}
					\tkzTabInit[lgt=1,espcl=3,nocadre=True]
					{$x$ /0.7, $y'$ /0.7, $y$ /2.5}
					{$-\infty$,$-m-2$,$-m$,$+\infty$}
					\tkzTabLine{,+,$0$,-,$0$,+,}
					\tkzTabVar{-/$-\infty$,+/$y(-m-2)$,-/$y(-m)$,+/$+\infty$}
				\end{tikzpicture}
			\end{center}
			Từ bảng biến thiên, suy ra hàm số nghịch biến trên khoảng $(- 1; 1)$ khi và chỉ khi 
			$$\heva{&-m-2 \le -1\\& -m \ge 1} \Leftrightarrow m = -1.$$
		\end{enumerate}
		
	}
\end{ex} 

\begin{ex}
	Cho hàm số $ y=\dfrac{x+5}{x+m}$, với $m$ là tham số.
	\choiceTF
	{Tập xác định của hàm số là $\mathbb{R}$}
	{Hàm số đồng biến trên từng khoảng xác định khi và chỉ khi $m \ge 5$}
	{\True Hàm số nghịch biến trên từng khoảng xác định khi và chỉ khi $m < 5$}
	{Hàm số đồng biến trên khoảng $\left(-\infty ;\, -8\right)$ khi và chỉ khi $\left(5;\, 8\right)$}
	\loigiai{
		\begin{enumerate}[a)]
			\item Điều kiện $x+m \ne 0 \Leftrightarrow x \ne -m$. Tập xác định là $D=\mathbb{R} \backslash\{-m\}$.
			\item Ta có $y'=\dfrac{m-5}{\left( x+m \right)^2},\forall x\in \mathbb{R}\backslash \left\{ -m \right\}.$\\
			Hàm số đồng biến trên từng khoảng xác định $\Leftrightarrow m-5>0 \Leftrightarrow m>5$.
			\item Ta có $y'=\dfrac{m-5}{\left( x+m \right)^2},\forall x\in \mathbb{R}\backslash \left\{ -m \right\}.$\\
			Hàm số nghịch biến trên từng khoảng xác định $\Leftrightarrow m-5<0 \Leftrightarrow m<5$.
			\item 	Hàm số $ y=\dfrac{x+5}{x+m}$ đồng biến trên khoảng $\left(-\infty ;\, -8\right)$ khi và chỉ khi 
			$$\heva{
				&\dfrac{m-5}{\left(x+m\right)^2}> 0\\
				&-m\notin\left(-\infty ;\, -8\right)
			}\Leftrightarrow \heva{
				&m > 5\\
				&-m\ge-8
			} \Leftrightarrow 5 < m\le 8.$$
		\end{enumerate}
	}
\end{ex} 

\Closesolutionfile{ans}


% \begin{dang}{Bài toán tìm m để hàm số có cực trị hoặc đạt cực trị tại điểm cho trước}
	\begin{enumerate}[\iconCV]
		\item Tìm $m$ để hàm số $y=f(x)$ đạt cực trị tại điểm $x_0$ cho trước ($f(x)$ có đạo hàm tại $x_0$):
		\begin{listEX}[1]
			\item [\ding{172}] Giải điều kiện $y'(x_0)=0$, tìm $m$.
			\item [\ding{173}] Lập bảng biến thiên với $m $ vừa tìm được và chọn giá trị $m$ nào thỏa yêu cầu.
				\end{listEX}
	\item Biện luận cực trị hàm số $y=ax^3+bx^2+cx+d$.\\
	Tính $y'=3ax^2+2bx+c$ với $\Delta_{y'}=b^2-3ac$
	\begin{itemize}
		\item[\ding{172}] $\heva{&\Delta_{y'} >0\\&a \ne 0}$: Hàm số có hai điểm cực trị
		\item[\ding{173}]  $\Delta_{y'} \le 0$ hoặc suy biến $\heva{&a=0\\&b=0}$: Hàm số không có cực trị.
	\end{itemize}
	% \begin{note}
		\begin{enumerate}[\iconMT]
				\item Gọi $x_1$, $x_2$ là hai nghiệm phân biệt của $y'=0$ thì $x_1+x_2=-\dfrac{2b}{3a}$ và $x_1\cdot x_2 =\dfrac{c}{3a}$.
			\begin{itemize}
				\item [$\bullet$] $x_1^2+x_2^2=(x_1+x_2)^2-2x_1 x_2$
				\item [$\bullet$] $(x_1-x_2)^2=(x_1+x_2)^2-4x_1 x_2$
				\item [$\bullet$] $x_1^3+x_2^3=(x_1+x_2)^3-3x_1x_2(x_1+x_2)$.
			\end{itemize}
			\item Các công thức tính toán thường gặp:
			\begin{itemize}
				\item [$\bullet$] Độ dài $MN=\sqrt{(x_N-x_M)^2+(y_N-y_M)^2}$
				\item [$\bullet$]  Khoảng cách từ $M$ đến $\Delta$: $d(M,\Delta)=\dfrac{|Ax_M+By_M+C|}{\sqrt{A^2+B^2}}$, với $\Delta \colon Ax+By+C=0$.
				\item [$\bullet$] Tam giác $ABC$ vuông tại $A \Leftrightarrow \overrightarrow{AB} \cdot \overrightarrow{AC}=0 \Leftrightarrow \text{hoành}\cdot\text{hoành}+\text{tung}\cdot\text{tung}=0$.
				\item [$\bullet$] Diện tích tam giác $ABC$ là  $S=\dfrac{1}{2}|a_1b_2-a_2b_1|$, với $\overrightarrow{AB}=(a_1;b_1)$, $\overrightarrow{AC}=(a_2;b_2)$.
			\end{itemize}
			\item PTĐT qua hai điểm cực trị là $y=-\dfrac{2}{9a}(b^2-3ac)x+d-\dfrac{bc}{9a}$.
		\end{enumerate}
	% \end{note}
	\end{enumerate}
\end{dang}
\boxmini{BÀI TẬP TỰ LUẬN}
\setcounter{vd}{0}
\begin{vd}
	Tìm $m$ để hàm số
	\begin{tasks}
		\task  $y=\dfrac{x^3}{3}-mx^2+(m^2-m+1)x+1$ đạt cực tiểu tại $x=3$.
		\task  $y=x^3-3mx^2+3(m^2-1)x$ đạt cực đại tại $x_0=1$.
	\end{tasks}
\loigiai{
\begin{enumerate}[a)]
	\item Ta có $y'=x^2-2mx+m^2-m+1$. Hàm số đạt cực tiểu tại $x=3$ thì
	$$y'(3)=0 \Leftrightarrow 9-6m+m^2-m+1=0 \Leftrightarrow \hoac{&m=2\\&m=5}.$$
	Lập bảng biến thiên của hàm số với lần lượt hai giá trị $m$ vừa tìm được, ta thấy $m=2$ thỏa.\\
	Vậy $m=2$.
	\item Ta có $y'=3x^2-6mx+3(m^2-1)$\\
	Điều kiện cần và đủ để thỏa điều kiện bài toán là
	\begin{eqnarray*}
		\heva{&y'(1)=0 \\&y''(1)<0}
		\Leftrightarrow \heva{&3m^2-6m=0 \\&6-6m<0}
		\Leftrightarrow \heva{&m=0 \vee m=2 \\&m>1}
		\Leftrightarrow m=2.
	\end{eqnarray*}
	Vậy $m=2$ thì thỏa bài toán.
\end{enumerate}}
\end{vd}

\begin{vd}
	Tìm tất cả giá trị của tham số $m$ để hàm số (đồ thị hàm số)
	\begin{tasks}
		\task $ y=x^3-3x^2+2mx+m+2024$ có hai điểm cực trị.
		\task $ y=\dfrac{1}{3}x^3-mx^2+\left(m+2\right)x+2019$ không có cực trị.
		\task $y=x^3-3(m+1)x^2+12mx+2019$ có hai điểm cực trị $x_1,\ x_2$ thỏa mãn $x_1+x_2+2x_1x_2=-8$.
		\task $y=-x^3-3mx^2+m-2$ với $m$ là tham số có hai điểm cực trị $A,B$ sao cho $AB=2$.
	\end{tasks}
\loigiai{
\begin{enumerate}[a)]
	\item Ta có $y’=3x^2-6x+2m$.\\
	Hàm số có cực đại, cực tiểu khi và chỉ khi phương trình $y’=0$ có hai nghiệm phân biệt $\Leftrightarrow {\Delta }’_{y’}>0$ $\Leftrightarrow 9-6m>0$ $\Leftrightarrow m<\dfrac{3}{2}$.
	\item Ta có $y’=x^2-2mx+m+2$\\
	Hàm số đã cho không có cực trị $\Leftrightarrow$ phương trình $y’=0$ vô nghiệm hoặc có nghiệm kép hay ${\Delta }’_{y’} \le 0$ $\Leftrightarrow m^2-\left( m+2 \right)\le 0$ $\Leftrightarrow -1\le m\le 2$.
	\item Ta có $y'=3x^2-6(m+1)x+12m,\ y'=0\Leftrightarrow 3x^2-6(m+1)x+12m=0$. \\
	Hàm số có hai điểm cực trị $\Leftrightarrow \Delta '=9m^2-18m+9>0\Leftrightarrow m\ne 1$.\tagEX{1}
	Giả sử $x_1,\ x_2$ là hai nghiệm của phương trình $y'=0$, theo định lí Vi-ét ta có
	$$\heva{&x_1+x_2=-\dfrac{b}{a}=2(m+1)\\&x_1\cdot x_2=\dfrac{c}{a}=4m.}$$
	Do đó $x_1+x_2+2x_1\cdot x_2=-8\Leftrightarrow 2(m+1)+8m=-8\Leftrightarrow 10m=-10\Leftrightarrow m=-1$ thỏa mãn $(1)$.\\
	Vậy $m=-1$ là giá trị cần tìm của $m$.
	\item Ta có $y'=-3x^2-6mx$; $y'=0\Leftrightarrow
	\hoac{&x=0\\&x=-2m.\\}$\\
	Hàm số có hai điểm cực trị khi và chỉ khi $m\ne 0$.\\
	Gọi hai điểm cực trị của đồ thị hàm số là $A$, $B$.\\
	Ta có $A\left(0;m-2\right)$, $ B\left(-2m;-4{m}^{3}+m-2\right)$.\\
	Do đó
	{\allowdisplaybreaks
		\begin{align*}
			AB^2=4m^2+16m^6=4&\Leftrightarrow 4m^6+m^2-1=0\\
			&\Leftrightarrow m^2=\dfrac{1}{2}\Leftrightarrow m=\pm \dfrac{1}{\sqrt{2}}.
	\end{align*}}
\end{enumerate}}
\end{vd}

\boxmini{BÀI TẬP TRẮC NGHIỆM}
\ind{PHẦN I.} \inden{Câu trắc nghiệm nhiều phương án lựa chọn. Mỗi câu hỏi học sinh chỉ chọn một phương án.}\\
\setcounter{ex}{0}
\Opensolutionfile{ans}[ans/2D1-B1-d3-1]

\begin{ex}
	Tìm tất cả giá trị của tham số $m$ để hàm số $y=\dfrac{1}{3}x^3+(m+1)x^2+(1-3m)x+2$ có cực đại và cực tiểu.
	\choice
	{$m\leq-5;m\geq 0$}
	{\True $m <-5$; $m>0$}
	{$-5<m<0$}
	{$-5\leq m\leq 0$}
	\loigiai{
		Tập xác định $\mathscr{D}=\mathbb{R}$.\\
		Ta có $y’=x^2+2(m+1)x+1-3m$.\\
		Hàm số có cực đại và cực tiểu khi phương trình $y’=0$ có hai nghiệm phân biệt và đổi dấu qua các nghiệm đó.\\
		Khi đó $\Delta’_{y’}=(m+1)^2-(1-3m)>0\Leftrightarrow m^2+5m>0\Leftrightarrow \hoac{&m<-5\\&m>0.}$}
\end{ex} 

\begin{ex}
	Tìm tất cả các giá trị của tham số $ m $ để hàm số $ y=-x^3-3x^2+mx+2 $ có cực đại và cực tiểu.
	\choice
	{\True $m>-3$}
	{$m\geq 3$}
	{$m\geq-3$}
	{$m>3$}
	\loigiai{
		Ta có $ y'=-3x^2-6x+m $. Hàm số đã cho có cực đại và cực tiểu khi và chỉ khi phương trình $ y'=0 $ có $ 2 $ nghiệm phân biệt $\Leftrightarrow\Delta'>0\Leftrightarrow 9+3m>0\Leftrightarrow m>-3 $.
	}
\end{ex} 

\begin{ex}
	Cho hàm số $y=x^3-3(m+1)x^2+3(7m-3)x$. Số giá trị nguyên của tham số $m$ để hàm số không có cực trị là
	\choice
	{$2$}
	{$1$}
	{\True $4$}
	{$3$}
	\loigiai{
		Hàm số bậc $3$ không có cực trị khi và chỉ khi phương trình $y'=0 \Leftrightarrow 3x^2-6(m+1)x+3(7m-3)=0$ có nghiệm kép hoặc vô nghiệm hay
		$$\Delta' \le 0 \Leftrightarrow 9(m+1)^2-9(7m-3)\le 0 \Leftrightarrow m^2-5m+4 \le 0 \Leftrightarrow 1 \le m \le 4.$$
		Mà $m \in \mathbb{Z}$ nên $ m \in \{1;2;3;4\}$.\\
		Vậy có $4$ giá trị nguyên của $m$ thỏa mãn yêu cầu bài toán.
	}
\end{ex} 

\begin{ex}
	Cho hàm số $y=x^3-3(m+1)x^2+3(7m-3)x$. Gọi $S$ là tập hợp tất cả các giá trị nguyên của tham số $m$ để hàm số không có cực trị. Số phần tử của $S$ là
	\choice
	{$2$}
	{\True $4$}
	{$0$}
	{Vô số}
	\loigiai{
		Tập xác định là $\mathscr{D}=\mathbb{R}$.\\
		$y'=3x^2-6(m+1)x+3(7m-3)$.\\
		Hàm số không có cực trị khi và chỉ khi $\Delta'=9(m+1)^2-9(7m-3)\le 0\Leftrightarrow m^2-5m+4\le 0\Leftrightarrow 1\le m \le 4.$\\
		Vậy có $m\in \{1;2;3;4\}$.
	}
\end{ex} 

\begin{ex}
	Giả sử hàm số $ y=\dfrac{1}{3}x^3-x^2-\dfrac{1}{3}mx$ có hai điểm cực trị $x_1, x_2$ thỏa mãn $x_1+ x_2+2x_1x_2=0$. Giá trị của $ m$ là
	\choice
	{$ m=\dfrac{4}{3}$}
	{$ m=-3$}
	{\True $ m=3$}
	{$ m=2$}
	\loigiai{
		Ta có $y’=x^2-2x-\dfrac{1}{3}m$.\\
		$y’=0\Leftrightarrow  3x^2-6x-m=0$.\\
		Hàm số có hai cực trị $\Leftrightarrow y'=0$ có hai nghiệm phân biệt $\Leftrightarrow 9+3m>0 \Leftrightarrow m>-3$.\\
		Khi đó $x_1+ x_2 + 2x_1x_2=0 \Leftrightarrow 2-\dfrac{2m}{3}=0 \Leftrightarrow m=3$ (TM).}
\end{ex} 

\begin{ex}
	Cho hàm số $ f\left( x \right)=x^3-3x^2+mx-1$. Tìm giá trị của tham số $m$ để hàm số có hai cực trị $x_1, x_2$ thỏa $x_1^2+x_2^2=3$.
	\choice
	{$ m=\dfrac{1}{2}$}
	{$ m=-2$}
	{$ m=1$}
	{\True $ m=\dfrac{3}{2}$}
	\loigiai{
		TXĐ $D=\mathbb{R}$.\\
		${f}’\left( x \right)=3x^2-6x+m$.\\
		Hàm số có hai điểm cực trị $x_1, x_2 \Leftrightarrow {f}’\left( x \right)=0$ có hai nghiệm phân biệt $\Leftrightarrow 9-3m>0  \Leftrightarrow m<3$.\\
		Theo hệ thức Vi-et: $x_1+ x_2=2$; $x_1.x_2=\dfrac{m}{3}$.\\
		Khi đó: $x_1^2+x_2^2=3  \Leftrightarrow  \left( {x_1+ x_2} \right)^2 - 2x_1x_2=3 \Leftrightarrow 2^2-\dfrac{2m}{3}=3 \Leftrightarrow m=\dfrac{3}{2}$.}
\end{ex} 

\begin{ex}
	Tìm tất cả các giá trị của tham số $m$ để đồ thị hàm số $y=x^3-12x+m+2$ có hai cực trị và hai điểm cực
	trị này nằm về hai phía trục hoành?
	\choice
	{$m=-2$}
	{\True $-18<m<14$}
	{$\forall m\in \mathbb{R}$}
	{$m\neq 1$}
	\loigiai{
		Ta có $y'=3x^2-12$. Suy ra $y'=0\Leftrightarrow \hoac{& x=2\Rightarrow y=m-14 \\ & x=-2\Rightarrow y=m+18.}$\\
		Đồ thị hàm số có hai điểm cực trị nằm về hai phía trục hoành khi và chỉ khi
		$$(m-14)(m+18)<0\Leftrightarrow -18<m<14.$$
	}
\end{ex} 

\begin{ex}
	Tập hợp các giá trị của $m$ để đồ thị hàm số $y=x^3+mx^2-\left(m^2-4\right)x+1$ có hai điểm cực trị nằm ở hai phía của trục $Oy$ là
	\choice
	{$(-\infty;2)$}
	{\True $\mathbb{R}\setminus[-2;2]$}
	{$(-2;2)$}
	{$(2;+\infty)$}
	\loigiai{
		Ta có $y'=3x^2+2mx+4-m^2$.\\
		Đồ thị hàm số có hai cực trị nằm hai phía đối với trục $Oy$ khi và chỉ khi $y'=0$ có hai nghiệm trái dấu $\Leftrightarrow P=\dfrac{4-m^2}{3}<0\Leftrightarrow\hoac{&m>2\\&m<-2.}$}
\end{ex} 

\begin{ex}
	Cho hàm số $y=x^3+3mx^2+3(m^2-1)x+m^3.$ Tìm $m$ để hàm số đạt cực tiểu tại điểm $x=0.$
	\choice
	{$m=-1$}
	{\True $m=1$}
	{$m=0$}
	{$m=2$}
	\loigiai{
		Ta có $y'=3x^2+6mx+3(m^2-1)$ và $y''=6x+6m\Rightarrow y''(0)=6m.$\\
		Hàm số đạt cực tiểu tại $x=0\Rightarrow y'(0)=0\Leftrightarrow 3(m^2-1)=0\Leftrightarrow m=\pm 1.$\\
		Với $m=1\Rightarrow y''(0)=6>0\Rightarrow$  hàm số đạt cực tiểu tại $x=0.$\\
		Với $m=-1\Rightarrow y''(0)=-6<0\Rightarrow$  hàm số đạt cực đại tại $x=0.$\\
		Vậy $m=1$ thỏa mãn bài.
	}
\end{ex} 

\begin{ex}
	Hàm số $ y=x^3-2mx^2+m^2x-2 $ đạt cực tiểu tại $ x=1 $ khi
	\choice
	{$ m=3 $}
	{$ m=-3 $}
	{\True $ m=1 $}
	{$ m=-1 $}
	\loigiai{
		Ta có: $ y'=3x^2-4mx+m^2 ,
		y''=6x-4m. $\\
		Hàm số đạt cực tiểu tại $ x=1 $, suy ra $y'(1)=0\Leftrightarrow m^2-4m+3=0 \Leftrightarrow \hoac{&m=1\\&m=3.}$
		\begin{itemize}
			\item Với $m=1$ ta có $y'(1)=0, y''(1)=2>0$ nên hàm số đạt cực tiểu tại $x=1$.
			\item Với $m=3$ ta có $y'(1)=0, y''(1)=-6<0$ nên hàm số đạt cực đại tại $x=1$.
	\end{itemize}}
\end{ex} 

\begin{ex}
	Tìm giá trị thực của tham số $m$ để hàm số $y=\dfrac{1}{3}x^3-mx^2+(m^2-4)x+3$ đạt cực tiểu tại $x=3$.
	\choice
	{$m=-1$}
	{\True $m=1$}
	{$m=-7$}
	{$m=5$}
	\loigiai{
		Ta có $y'=x^2-2mx+m^2-4$ và $y''=2x-2m$.\\
		Hàm số đạt cực tiểu tại $x=3$ nên $y'(3)=0 \Leftrightarrow 9-6m+m^2-4=0 \Leftrightarrow \hoac{&m=5 \\ &m=1.}$\\
		Với $m=5$ thì $y''(3)=-4<0$, loại.\\
		Với $m=1$ thì $y''(3)=4>0$, thỏa mãn.
	}
\end{ex} 

\begin{ex}
	Đồ thị hàm số $y=x^3-3x^2+2ax+b$ (với $a, b \in \mathbb{R}$) có điểm cực tiểu $A(2;-2)$. Khi đó $a+b$ bằng
	\choice
	{$-4$}
	{$4$}
	{\True $2$}
	{$-2$}
	\loigiai{
		Ta có: $y’=3x^2-6x+2a; y''=6x-6$.\\
		Đồ thị hàm số có điểm cực tiểu $A(2;-2)$ nên ta có:
		$\heva{&y’(2)=0\\&y(2)=-2} \Leftrightarrow \heva{&2a=0\\&4a+b=2} \Leftrightarrow \heva{&a=0\\&b=2.}$\\
		Với $a=2,b=0$ ta thấy $y''(2)=6.2-6=6>0$ nên hàm số đạt cực tiểu tại $x=2$, thỏa yêu cầu bài toán.\\
		Suy ra $a+b=2$.
	}
\end{ex} 

\begin{ex}
	Gọi $m_1, m_2$ là các giá trị của tham số $m$ để đồ thị hàm số $y=2x^{3}-3x^{2}+m-1$ có hai điểm cực trị $B, C$ sao cho tam giác $OBC$ có diện tích bằng 2, với $O$ là gốc tọa độ. Tích $m_{1} \cdot m_{2}$ bằng
	\choice
	{$12$}
	{$6$}
	{\True $-15$}
	{$-20$}
	\loigiai{
		Tập xác định: $\mathscr{D}=\mathbb{R}$.\\
		Ta có \begin{eqnarray*}
			y'=6 x^{2}-6 x=0 &\Leftrightarrow&
			\hoac{x=0 \Rightarrow y=m-1 \Rightarrow B(0 ; m-1) \\ x=1 \Rightarrow y=m-2 \Rightarrow C(1 ; m-2)}\\
			&\Rightarrow& S_{\triangle OBC}=\dfrac{1}{2} d(C ; O B) \cdot O B=\dfrac{1}{2} \cdot 1 \cdot |m-1|=2\\
			&\Leftrightarrow& |m-1|=4
			\Leftrightarrow \hoac{&m_1=5 \\ &m_2=-3.}
		\end{eqnarray*}
		Vậy $m_1 \cdot m_2 = -15$.
	}
\end{ex} 

\begin{ex}
	Cho hàm số $y=x^3-3mx^2+3m^3$. Biết rằng có hai giá trị của tham số $m$ để đồ thị hàm số có hai điểm cực trị $A,B$ và tam giác $OAB$ có diện tích bằng $48$. Khi đó tổng các giá trị của $m$ là
	\choice
	{\True $0$}
	{$2$}
	{$\sqrt{2}$}
	{$-2$}
	\loigiai{
		Tập xác định $\mathscr{D}=\mathbb{R}$.\\
		Đạo hàm $y'=3x^2-6mx$, xác định với mọi $x\in\mathbb{R}$.\\
		$y'=0\Leftrightarrow\hoac{&x=0\\&x=2m.}$ \\
		Do đó hàm số có hai cực trị khi và chỉ khi $m\neq 0$.\\
		Khi đó $A\left(0;3m^3\right)$, $B\left(2m;-m^3\right)$.\\
		Suy ra $\overrightarrow{OA}=\left(0;3m^3\right)$, $\overrightarrow{OB}=\left(2m;-m^3\right)$.\\
		$S_{\triangle OAB}=48\Leftrightarrow \dfrac{1}{2}\left|\left[\overrightarrow{OA},\overrightarrow{OB}\right]\right|=48\Leftrightarrow \left|-6m^4\right|=96\Leftrightarrow m=\pm 2$.\\
		Vậy tổng các giá trị của $m$ là $0$.
	}
\end{ex} 
\Closesolutionfile{ans}

\ind{PHẦN II.} \inden{Câu trắc nghiệm đúng sai. Trong mỗi ý a), b), c), d) ở mỗi câu, học sinh chọn đúng hoặc sai.}\\
\Opensolutionfile{ans}[ans/2D1-B1-d3-2]

\begin{ex}
	Cho hàm số $ y=\dfrac{m}{3}x^3+2x^2+mx+1$, với $m$ là tham số.
	\choiceTF
	{Hàm số có hai điểm cực trị khi $-2<m<2$}
	{Hàm số có đúng một điểm cực trị khi $m=0$ hoặc $m=2$}
	{\True Hàm số không có cực trị khi $m \le -2$ hoặc $m \ge 2$}
	{\True Hàm số có $2$ điểm cực trị thỏa mãn $x_\text{CĐ}<x_{CT}$ khi $0<m<2$}
	\loigiai{
		\begin{enumerate}[a)]
			\item Ta có $y’=mx^2+4x+m$.\\
			Hàm số có $2$ điểm cực trị $\Leftrightarrow y’=0$ có $2$ nghiệm phân biệt $\Leftrightarrow \left\{ \begin{aligned}
				& m\ne 0 \\
				& 4-m^2>0 \\
			\end{aligned} \right.\Leftrightarrow \left\{ \begin{aligned}
				& m\ne 0 \\
				& -2<m<2 \\
			\end{aligned} \right.\quad(1)$.
			\item Hàm số có đúng 1 cực trị khi hàm số này bị suy biến về hàm bậc hai, nghĩa là $\dfrac{m}{3}=0 \Leftrightarrow m=0$.
			\item Với $m=0$ thì hàm số trở thành $y=2x^2+1$. Hàm số này có 1 điểm cực tiểu. Điều này không thỏa yêu cầu bài toán\\
			Với $m \ne 0$: Hàm số không có cực trị $\Leftrightarrow y’=0$ có vô nghiệm hoặc nghiệm kép. $\Leftrightarrow \left\{ \begin{aligned}
				& m\ne 0 \\
				& 4-m^2 \le 0 \\
			\end{aligned} \right.\Leftrightarrow \left\{ \begin{aligned}
				& m\ne 0 \\
				&  m \le -2,\,m \ge 2\\
			\end{aligned} \right.$.
			\item Dựa vào dạng đồ thị hàm số bậc $3$, hàm số có $2$ điểm cực trị thỏa mãn $x_\text{CĐ}<x_{CT}$ khi $ m>0$ $(2)$\\
			Từ $\left(1\right)$ và $\left(2\right)$ suy ra giá trị $ m$ cần tìm là $0<m<2$.
		\end{enumerate}
}
\end{ex} 

\begin{ex}
	Cho hàm số $y=x^3-3mx^2+3\left(m^2-1\right)x-m^3$ với $m$ là tham số.
	\choiceTF
	{\True Hàm số luôn có hai điểm cực trị với mọi $m$}
	{\True Hàm số đạt cực tiểu tại $x=3$ khi $m=2$}
	{\True Khi đồ thị hàm số có hai điểm cực trị thì khoảng cách giữa hai điểm cực trị bằng $2\sqrt{5}$}
	{\True Điểm cực tiểu của đồ thị hàm số luôn thuộc đường thẳng cố định với hệ số góc $k=-3$}
	\loigiai
	{
		\begin{enumerate}[a)]
			\item Ta có $y'=3x^2-6mx+3\left(m^2-1\right). y'=0\Leftrightarrow \hoac{&x_1=m-1\\&x_2=m+1}$.\\
			Do $x_1 \ne x_2, \,\forall m$ nên hàm số luôn có hai điểm cực trị.
			\item Dễ thấy $x=m+1$ là điểm cực tiểu. Suy ra, hàm số đạt cực tiểu tại $x=3$ khi $m+1=3 \Leftrightarrow m=2$.
			\item Với mọi $m$, tọa độ hai điểm cực trị là $A(m+1;-3m-2)$ và $B(m-1;-3m+2)$.\\
			Khoảng cách giữa hai điểm cực trị là $AB=\sqrt{(x_B-x_A)^2+(y_B-y_A)^2}=2\sqrt{5}$.
			\item Ta có $y'=3x^2-6mx+3\left(m^2-1\right). y'=0\Leftrightarrow \hoac{&x=m-1\\&x=m+1}$\\
			Vì là hàm số bậc ba với hệ số $a=1>0$ nên điểm cực tiểu của hàm số là $A\left(m+1;-3m-2\right)$. \\
			Lại có $-3m-2=-3\left(m+1\right)+1$ nên điểm cực tiểu của hàm số luôn thuộc đường thẳng $d:y=-3x+1$, hệ số góc $k=-3$.
		\end{enumerate}
	}
\end{ex} 

\begin{ex}
	Cho hàm số $y=\dfrac{x^2-2mx +m +2}{x-m}$, với $m$ là tham số.
	\choiceTF
	{\True Tập xác định của hàm số là $\mathbb{R}\backslash\{m\}$}
	{\True Có hai giá trị nguyên của tham số $m$ để hàm số có hai điểm cực trị}
	{\True Hàm số đạt cực đại tại $x=-1$ khi $m=\dfrac{1}{2}$}
	{Khi đồ thị hàm số có hai điểm cực trị thì đường thẳng qua hai điểm cực trị của đồ thị có phương trình là $y=2x-2m$}
	\loigiai{
		\begin{enumerate}[a)]
			\item Hàm số xác định khi $x-m \ne 0 \Leftrightarrow x \ne m$. Suy ra $\mathscr{D}=\mathbb{R}\backslash\{m\}$.
			\item $y'=\dfrac{x^2-2mx+2m^2-m-2}{(x-m)^2}$.\\
			Để hàm số có hai điểm cực trị thì $y'=0$ có hai nghiệm phân biệt khác $m$ hay $g(x)=x^2-2mx+2m^2-m-2$ có hai nghiệm phân biệt khác $m$.
			$$\Leftrightarrow \heva{&\Delta'>0\\&g(m) \ne 0} \Leftrightarrow \heva{&-m^2+m+2>0\\&m^2-m-2 \ne 0}  \Leftrightarrow m \in (-1;2).$$
			Vì $m$ nguyên nên $m \in \{0;1\}$.
			\item Hàm số đạt cực trị tại $x=-1$ thì $y'(-1)=0 \Leftrightarrow 2m^2+m-1 =0 \Leftrightarrow m=-1$ hoặc $m=\dfrac{1}{2}$.\\
			Thử lại với $m=\dfrac{1}{2}$, ta có $y'=\dfrac{x^2-x-2}{x-\dfrac{1}{2}}$.\\
				Bảng biến thiên
				\begin{center}
					\begin{tikzpicture}
						\tkzTabInit[nocadre=false,lgt=1,espcl=3]
						{$x$ /0.7,$y'$ /0.7,$y$ /2}
						{$-\infty$,$-1$,$0.5$,$2$,$+\infty$}
						\tkzTabLine{,+,$0$,-,d,-,$0$,+,}
						\tkzTabVar{-/$-\infty$,+/$y_1$,-D+/$-\infty$/$+\infty$,-/$y_2$,+/$+\infty$}
					\end{tikzpicture}
				\end{center}
			Suy ra $m=\dfrac{1}{2}$ thỏa yêu cầu bài toán.
			\item Cho hàm số $y=\dfrac{u(x)}{v(x)}$. Nếu đồ thị hàm số có hai điểm cực trị thì đường thẳng qua hai điểm cực trị có dạng $y=\dfrac{u'(x)}{v'(x)}$.\\
			Áp dụng, ta được $y=\dfrac{(x^2-2mx+m+2)'}{(x-m)'}=2x-2m$
		\end{enumerate}
	}
\end{ex} 

\Closesolutionfile{ans}

% \input{data/12/2D1-B1-4.tex}
% \begin{dang}{Cực trị hàm hợp, hàm chứa trị tuyệt đối}
    \begin{itemize}
        \item Các phép biến đổi đồ thị
        \begin{itemize}
            \item Đồ thị hàm $y=f(x+a)$ vẽ bằng cách dời đồ thị $y=f(x)$ sang trái $a$ đơn vị.
            \item Đồ thị hàm $y=f(x)+b$ vẽ bằng cách dời đồ thị $y=f(x)$ lên trên $b$ đơn vị.
            \item Đồ thị hàm $y=f(|x|)$ vẽ bằng cách "lật qua trái".
            \item Đồ thị hàm $y=|f(x)|$ vẽ bằng cách "lật lên".
            \item Đồ thị hàm $y=|f(|x|)|$ vẽ bằng cách "lật lên rồi lật qua trái".
        \end{itemize}
        \begin{note} Hàm $y=f(x)$ có $m$ điểm cực trị, $n$ nghiệm bội lẻ, $p$ điểm cực trị dương. Khi đó
            \begin{itemize}
                \item[-]Hàm $y=f(ax+b)+c$ cũng có $m$ điểm cực trị.
                \item[-]	Hàm $y=|f(x)|$ có $m+n$  điểm cực trị.
                \item[-] 	Hàm $y=f(|x|)$ có $2p+1$  điểm cực trị.
            \end{itemize}
        \end{note}
        \item Hàm $y=f(u)$.
        \begin{itemize}
            \item \textbf{Bước 1: } Tính đạo hàm $y'=u'f'(u)$.
            \item \textbf{Bước 2: } Lập bảng xét dấu của $y'$ hoặc đếm số nghiệm bội lẻ của $y'=0$.
            \item \textbf{Bước 3: } Kết luận.
        \end{itemize}
        \item Hàm $y=f(u)+g(x)$.
        \begin{itemize}
            \item \textbf{Bước 1: } Tính đạo hàm $y'=u'f'(u)+g'$.
            \item \textbf{Bước 2: } Lập bảng xét dấu của $y'$ hoặc đếm số nghiệm bội lẻ của $y'=0$ (dựa vào tương giao giữa hai đồ thị).
            \item \textbf{Bước 3: } Kết luận.
        \end{itemize}
    \end{itemize}
\end{dang}
\begin{vd}
    \immini{Cho hàm số $y=f(x)$ có bảng biến thiên như hình vẽ. Tìm các điểm cực trị, các cực trị của hàm số sau
        \begin{listEX}[1]
            \item $y=f(x+2)$
            \item $y=f(x)-3$
            \item $y=f(2x-3)+1$
            \item $y=f(1-2x)+2025$
    \end{listEX}}{\begin{tikzpicture}[>=stealth]
            \tkzTabInit[nocadre=false,lgt=1,espcl=2,deltacl=0.5]{$x$/.6 ,$y'$/.6,$y$/1.8}
            {$-\infty$ , $0$ , $2$ , $+\infty$}
            \tkzTabLine{ , - , $0$ , + , $0$ , - , }
            \tkzTabVar{+/$+\infty$ , -/$1$ , +/$5$ , -/$-\infty$}
    \end{tikzpicture}}
    \loigiai{}
\end{vd}
\begin{vd}
    \immini{Cho hàm số $y=f(x)$ có bảng biến thiên như hình vẽ.Tìm các điểm cực trị của hàm số sau
        \begin{listEX}[1]
            \item $y=f(x^2)$
            \item $y=f(3x^2-2x)$
            \item $y=f(\sqrt{x^2+2x+2})$
    \end{listEX}}{
        \begin{tikzpicture}[>=stealth]
            \tkzTabInit[nocadre=false,lgt=1,espcl=2,deltacl=0.5]{$x$/.6,$y'$/.6,$y$/1.8}
            {$-\infty$ , $0$ , $2$ , $+\infty$}
            \tkzTabLine{ , - , $0$ , + , $0$ , - , }
            \tkzTabVar{+/$+\infty$ , -/$1$ , +/$5$ , -/$-\infty$}
        \end{tikzpicture}
    }
    \loigiai{}
\end{vd}
\begin{vd}%[2D1G5-5]
    \immini{Cho hàm số $y=f(x)$ có đồ thị $y=f'(x)$ như hình vẽ. Tìm số điểm cực trị của các hàm số sau
        \begin{listEX}[2]
            \item $y=f(x)$
            \item $y=2f(x)-x$
            \item $y=f(3x)+2x$
            \item $y=f(x)+\dfrac{x^2}{2}-x$
            \item $y=3f(x)-2x^3$
            \item $y=f(2x+1)-4x$
    \end{listEX}}{\begin{tikzpicture}[smooth, >=latex, line cap =round, line join =round,font=\scriptsize,x=1.4cm]
            \begin{scope}[scale=.5]
                \draw[->] (-3,0)--(3,0) node[below]{$x$};
                \draw[->] (0,-2.5) -- (0,3) node[left] {$y$};
                \draw[ name path=dcong] (-2,-2)..controls +(90:0.3) and +(180:0.3)..(-0.7,2.7)..controls +(0:0.2) and +(180:0.3)..(0,0.5)..controls +(30:0.2) and +(180:0.3)..(1,2)..controls +(0:0.3) and +(90:0.1).. (2,-2);
                \draw[thick,dashed] (-1,0) node[below] {$-1$} --(-1,2) --(1,2) -- (1,0) node[below] {$1$} (0,2) node[above right] {$2$};
            \end{scope}
    \end{tikzpicture}}
    \loigiai{}
\end{vd}
\begin{vd}
    \immini{Cho hàm số $y=f(x)$ có đồ thị như hình vẽ. Tìm số điểm cực trị của hàm số
        \begin{listEX}[2]
            \item $y=f(|x|)$
            \item $y=|f(x)|$
            \item $y=|f(|x|)|$
            \item $y=f(|x|-a)$
            \item $y=f(|x+b|)$
            \item $y=|f(x+2025)|$
    \end{listEX}}{
        \begin{tikzpicture}[>=stealth,line join=round, line cap=round, font=\scriptsize]
            \begin{scope}[scale=.8]
                \draw[-stealth](-4,0)--(0,0)node[below right]{$O$}--(4,0)node[below left]{$x$};
                \draw[-stealth](0,-2)--(0,3)node[below left]{$y$};
                \draw[dashed]
                (-3,0)node[above]{$a$}--(-3,-2)
                (3,0)node[below]{$b$}--(3,3)
                ;
                \draw[smooth]
                (-3,-2)..controls+(85:3) and+(180:.5)..(-2,2)
                ..controls+(0:.5) and+(180:.5)..(-1,1)
                ..controls+(0:.5)and+(180:.5)..(0.5,2)
                ..controls+(0:.5)and+(180:.75)..(1.5,-1.5)
                ..controls+(0:.75)and+(-95:.3)..(3,3)
                ;
            \end{scope}
    \end{tikzpicture}}
    \loigiai{}
\end{vd}
\begin{vd}
    Tìm $m$ để
    \begin{listEX}
        \item  Hàm số $y=|f(x)|$ có $5$ điểm cực trị, với  $f(x)= 3x^3+3x^2+mx+m$
        \item Hàm số $y=f\left(\vert x\vert\right)$ có $5$ điểm cực trị, với $f(x)=x^3-(2m-1)x^2+(2-m)x+2$.
    \end{listEX}
    \loigiai{
        \begin{listEX}
            \item Đặt $f(x)=3x^3+3x^2+mx+m=3x^2(x+1)+m(x+1)=(x+1)(3x^2+m)$.\\
            Suy ra $f'(x)=9x^2+6x+m$.\\
            Phương trình $f'(x)=0$ có $2$ nghiệm phân biệt $x_1$, $x_2$ khi và chỉ khi $\Delta'=9-9m>0\Leftrightarrow m<1$. Khi đó ta có $x_1+x_2=-\dfrac{2}{3}$, $x_1x_2=\dfrac{m}{9}$.\\
            Hàm số $y=|f(x)|$ có $5$ điểm cực trị khi và chỉ khi $\heva{&\Delta'>0\\&y(x_1)\cdot y(x_2)<0.}$\\
            Thực hiện biến đổi
            \allowdisplaybreaks
            \begin{eqnarray*}
                y(x_1)\cdot y(x_2) &=&\ (x_1+1)(3x_1^2+m)\cdot(x_2+1)(3x_2^2+m)\\
                &=&\ \left[9(x_1x_2)^2+3m(x_1^2+x_2^2)+m^2\right]\left(x_1x_2+x_1+x_2+1\right)\\
                &=&\ \left[\dfrac{m^2}{9}+3m\left[\left(-\dfrac{2}{3}\right)^2-\dfrac{2m}{9}\right]+m^2\right]\left(\dfrac{m}{9}-\dfrac{2}{3}+1\right)\\
                &=&\ \dfrac{1}{9}(4m^2+12m)(m+3).
            \end{eqnarray*}
            Suy ra $y(x_1)\cdot y(x_2)<0\Leftrightarrow (4m^2+12m)(m+3)<0\Leftrightarrow -3\neq m<0$.\\
            Kết hợp với điều kiện $m$ là số nguyên thỏa $|m|<10$ ta được $m\in\{-1;-2;-4;-5;-6;-7;-8;-9\}$.\\
            Vậy có $8$ giá trị nguyên của tham số $m$.
            \item Tập xác định $\mathscr{D}=\mathbb{R}$.\\
            Ta có $f\left(|-x|\right)=f\left(|x|\right)$, $\forall x\in\mathbb{R}$ nên $y=f\left(|x|\right)$ là hàm số chẵn. \\
            Do đó, đồ thị hàm số $y=f\left(|x|\right)$ đối xứng qua trục tung.\\
            Suy ra hàm số $y=f\left(|x|\right)$ luôn có một điểm cực trị là $x=0$.\\
            Do đó, $y=f\left(|x|\right)$ có $5$ điểm cực trị $\Leftrightarrow$ hàm số $y=f(x)$ có $2$ điểm cực trị dương.\\
            \phantom{Do đó, số $y=f\left(|x|\right)$ có $5$ điểm cực trị} $\Leftrightarrow$ $f'(x)=0$ có hai nghiệm dương phân biệt.\\
            Ta có $f'(x)=3x^2-2(m-1)x+2-m$.\\
            Yêu cầu bài toán $\Leftrightarrow\heva{&\Delta'>0 \\ &S>0 \\ &P>0}\Leftrightarrow\heva{&4m^2-m-5>0 \\ &2m-1>0 \\ &2-m>0}\Leftrightarrow\heva{&m<-1\;\text{hoặc}\;m>\dfrac{5}{4} \\ &m>\dfrac{1}{2} \\ &m<3}\Leftrightarrow \dfrac{5}{4}<m<2$.
        \end{listEX}
    }
\end{vd}
\boxmini{BÀI TẬP TRẮC NGHIỆM}
\Opensolutionfile{ans}[ans/2D1-2-DANG-3]
\begin{ex}%[2D1K2-6]
    \immini
    {Cho hàm số $f(x)$ có đồ thị $f'(x)$ có đồ thị như hình vẽ bên dưới.\\ Hàm số $y=f(1-2x)$ có bao nhiêu cực trị ?
        \choice[2]
        {$4$}
        {$7$}
        {\True $3$}
        {$9$}
    }
    {
        \begin{tikzpicture}[>=stealth,font=\scriptsize]
            \begin{scope}[scale=0.55]
                \draw[->] (0,-1)--(0,3.5)node[right]{\scriptsize $y$};
                \draw[->] (-2,0)--(5,0)node[below]{\scriptsize $x$};
                \fill (0,0) node[below left]{\scriptsize  $O$} circle(1.5pt);
                \draw (-0.8,0) node[below left]{ $-1$} (0.9,0) node[below left]{ $1$} (2,0) node[below]{ $2$} (4,0) node[below]{ $4$};
                \clip (-2,-1) rectangle (5,3.5);
                \draw[] plot[smooth,tension=.65] coordinates{(-1.05,-0.9) (-0.3,2.5) (1.2,-0.5) (2.7,0.7) (4.2,0.2) (4.8,3.5)};
            \end{scope}
        \end{tikzpicture}
    }
    \loigiai{
        Đặt $g(x)=f(1-2x)$\\
        Dựa vào đồ thị, ta thấy $f'(x)=0$ có nghiệm $x_1=-1,x_2=1,x_3=2$ và $x_4=4$ nên $f'(x)$ có dạng $$f'(x)=k(x+1)(x-1)(x-2)(x-4)$$
        Khi đó $g'(x)=-2f'(1-2x)=-2k(2-2x)(-2x)(-1-2x)(-3-2x)^2$
        $$g'(x)=0 \Leftrightarrow \hoac{&x=1\\&x=0\\&x=-\dfrac{1}{2}\\&x=-\dfrac{3}{2} \text{ (kép)}}$$
        Bảng xét dấu $g'(x)$
        \begin{center}
            \begin{tikzpicture}[every node/.style={circle,fill=white,inner sep=0pt},arrow/.style={>=stealth,->,shorten <= 0.3cm,shorten >= 0.3cm},font=\footnotesize,xscale=1,yscale=1]
                \def\mnumline{1} %Số dòng
                \def\mnumcol{11} %Số cột
                \foreach \j in {0,...,\mnumline}
                \foreach \i in {0,...,\mnumcol}{
                    \coordinate (\j\i) at (\i,-\j);
                }
                \pgfmathsetmacro\yline{\mnumline/2-1}
                \path node at (00){$x$} node at (10){$g'(x)$};
                \foreach \x/\mnamex in {01/$-\infty$,03/$-\dfrac{3}{2}$,05/$-\dfrac{1}{2}$,07/$0$,09/$1$,0\mnumcol/$+\infty$} \path node at (\x) {\mnamex};
                \foreach \dy/\mnamedy in {12/$-$,13/$0$,14/$-$,15/$0$,16/$+$,17/$0$,18/$-$,19/$0$,110/$+$} \path node at (\dy) {\mnamedy};
                \draw[thick] (-.5,.5)rectangle([xshift=0.5cm,yshift=-0.5cm]\mnumline\mnumcol) ([xshift=-0.5cm,yshift=-0.5cm]00)--([xshift=0.5cm,yshift=-0.5cm]0\mnumcol)  ([xshift=0.5cm,yshift=0.5cm]00)--([xshift=0.5cm,yshift=-0.5cm]\mnumline0);
            \end{tikzpicture}
        \end{center}
        Dựa vào bảng xét dấu, ta thấy $g'(x)$ đổi dấu 3 lần nên $y=f(1-2x)$ có 3 cực trị.
    }
\end{ex}
\begin{ex}%[2D1K2-2]
    \immini{Cho hàm số $ f(x) $ có đạo hàm là $ f'(x) $. Đồ thị của hàm số $ y=f'(x) $ như hình vẽ bên. Khi đó hàm số $ y=f(x^2) $ có bao nhiêu điểm cực trị?
        \choice[2]
        {$2$}
        {$4$}
        {\True $3$}
        {$5$}}
    {
        \begin{tikzpicture}[line join=round, line cap=round,>=stealth,font=\scriptsize]
            \begin{scope}[scale=0.35]
                \tikzset{label style/.style={font=\footnotesize}}
                \def \xmin{-1.5}
                \def \xmax{6.5}
                \def \ymin{-2}
                \def \ymax{5.5}
                \def \hamso{-0.11*(\x)^3+1.09*(\x)^2-1.73*(\x)}
                \draw[->] (\xmin,0)--(\xmax,0) node[below left] {$x$};
                \draw[->] (0,\ymin)--(0,\ymax) node[below left] {$y$};
                \draw (0,0) node [below left] {$O$};
                \begin{scope}
                    \clip (\xmin+0.01,\ymin+0.01) rectangle (\xmax-0.01,\ymax-0.01);
                    \draw[samples=350,domain=-1.2:5.5,smooth,variable=\x] plot (\x,{\hamso});
                \end{scope}
                \draw [dashed] (5,4.6)--(5,0) node[below]{$5$} (2,0) node[below]{$2$};
            \end{scope}
        \end{tikzpicture}
    }
    \loigiai{$ y'=2xf'(x^2) $. Cho $ y'=0 \Leftrightarrow \hoac{&x=0\\&f'(x^2)=0} \Leftrightarrow \hoac{&x=0\\&x^2=0\\&x^2=2} \Leftrightarrow \hoac{&x=0\\&x=0 \text{ (nghiệm kép)}\\&x=\pm \sqrt{2}} $.\\
        $ y'=0 $ có 3 nghiệm bội bậc lẻ nên hàm số có 3 điểm cực trị.
    }
\end{ex}
\begin{ex}%[2D1K2-6]
    \immini{	Cho hàm số $y=f(x)$ xác định trên $\mathbb{R}$ và hàm số $y=f'(x)$ có đồ thị như hình vẽ. Hàm số $y=f(1-x^2)$ đạt cực đại tại điểm nào sau đây?
        \choice[2]
        {$x=-1$}
        {\True $x=\pm \sqrt{2}$}
        {$x=3$}
        {$x=0$}}{
        \begin{tikzpicture}[>=stealth, font=\scriptsize, line join=round, line cap=round,y=0.7cm]
            \begin{scope}[scale=.5]
                \def\a{1} \def\b{-2} \def\c{-2.5} % Hệ số
                \def\xmin{-2} \def\xmax{4}
                \def\ymin{-4} \def\ymax{1.5}
                %\draw[color=gray!50,dashed] (\xmin,\ymin) grid (\xmax,\ymax);
                \draw[->] (\xmin,0)--(\xmax,0);
                \draw[->] (0,\ymin)--(0,\ymax);
                \node at (0,0) [below right]{$O$};
                \node at (-1,0) [below left]{$-1$};
                \node at (3,0) [below right]{$3$};
                \clip (\xmin+0.1,\ymin+0.1) rectangle (\xmax-0.5,\ymax-0.1);
                \draw[smooth,samples=300,domain=-1.3:3.3] plot(\x,{\a*(\x)^2+\b*(\x)+\c});
            \end{scope}
    \end{tikzpicture}}
    \loigiai{Đặt $g(x)=f(1-x^2)$\\
        Khi đó $g'(x)=-2x\cdot f'(1-x^2)$\\
        Cho $g'(x)=0 \Leftrightarrow -2x \cdot f'(1-x^2) =0$
        $$ \Leftrightarrow \hoac{&x=0\\&f'(1-x^2)=0 \Leftrightarrow \hoac{&1-x^2=-1\Leftrightarrow x^2=2 \Leftrightarrow x=\pm \sqrt{2}\\&1-x^2=3}}$$
        Bảng xét dấu
        \begin{center}
            \begin{tikzpicture}[every node/.style={circle,fill=white,inner sep=0pt},arrow/.style={>=stealth,->,shorten <= 0.3cm,shorten >= 0.3cm},font=\footnotesize,xscale=1.4,yscale=.8]
                \def\mnumline{3} %Số dòng
                \def\mnumcol{9} %Số cột
                \foreach \j in {0,...,\mnumline}
                \foreach \i in {0,...,\mnumcol}{
                    \coordinate (\j\i) at (\i,-\j);
                    %	\draw[gray!30] ([xshift=-0.5cm,yshift=0.5cm]\j\i)--([xshift=0.5cm,yshift=0.5cm]\j\i)--([xshift=0.5cm,yshift=-0.5cm]\j\i)--([xshift=-0.5cm,yshift=-0.5cm]\j\i)--cycle (\j\i)node[]{\j\i}; %Ẩn lệnh này sau khi hoàn thành BBT
                }
                \pgfmathsetmacro\yline{\mnumline/2-1}
                \path node at (00){$x$} node at (10){$-x$} node at (20){\scriptsize $f'(1-x^2)$} node at (30){$g'(x)$};
                \foreach \x/\mnamex in {01/$-\infty$,03/$-\sqrt{2}$,05/$0$,07/$\sqrt{2}$,0\mnumcol/$+\infty$} \path node at (\x) {\mnamex};
                \foreach \dy/\mnamedy in {12/$-$,13/$0$,14/$+$,16/$+$} \path node at (\dy) {\mnamedy};
                \path node at ($(12)$){$+$} node at ($(13)$){$|$} node at ($(14)$){$+$} node at ($(15)$){$0$} node at ($(16)$){$-$} node at ($(17)$){$|$} node at ($(18)$){$-$} node at ($(22)$){$+$} node at ($(23)$){$0$} node at ($(24)$){$-$} node at ($(25)$){$|$} node at ($(26)$){$-$} node at ($(27)$){$0$} node at ($(28)$){$+$} node at ($(32)$){$+$} node at ($(33)$){$0$} node at ($(34)$){$-$} node at ($(35)$){$0$} node at ($(36)$){$+$} node at ($(37)$){$0$} node at ($(38)$){$-$};
                \draw[thick] (-.5,.5)rectangle([xshift=0.5cm,yshift=-0.5cm]\mnumline\mnumcol) ([xshift=-0.5cm,yshift=-0.5cm]00)--([xshift=0.5cm,yshift=-0.5cm]0\mnumcol) ([xshift=-0.5cm,yshift=-0.5cm]10)--([xshift=0.5cm,yshift=-0.5cm]1\mnumcol)

                ([xshift=-0.5cm,yshift=-0.5cm]20)--([xshift=0.5cm,yshift=-0.5cm]2\mnumcol)

                ([xshift=0.5cm,yshift=0.5cm]00)--([xshift=0.5cm,yshift=-0.5cm]\mnumline0); %Lệnh tự động kẻ bảng
            \end{tikzpicture}
        \end{center}
        Dựa vào bảng xét dấu ta xác định được hàm số đạt cực đại tại $x=\pm \sqrt{2}$.}
\end{ex}
\begin{ex}%[2D1K2-6]
    \immini{Cho hàm số $y=f(x)$ có đồ thị hàm $f'(x)=ax^2+bx+c$ như hình bên dưới. Hỏi hàm số $y=f(x-x^2)$ có bao nhiêu cực trị?
        \choice[2]
        {$0$}
        {\True $1$}
        {$2$}
        {$3$}}{
        \begin{tikzpicture}[>=stealth,x=1.2cm,y=0.7cm,font=\scriptsize]
            \begin{scope}[scale=0.35]
                \clip (-2,-2) rectangle (5,5.5);
                \def\a{1}
                \def\b{-3}
                \def\c{2}
                \draw[->] (-2,0) -- (4,0) node[below] { $x$};
                \draw[->] (0,-1) -- (0,5) node[left] {$y$};
                \draw (0,0)node[below left]{ $O$} circle(1.5pt);
                \draw (1,0) node[below]{$1$} (2,0) node[below]{  $2$} (0,2) node[left]{$2$};
                \pgfmathsetmacro\xdinh{-(\b)/2*(\a)}
                \pgfmathsetmacro\ydinh{(4*(\a)*(\c)-(\b)^2)/(4*(\a))}
                \draw[samples=150,smooth,domain=-5:5] plot(\x,{\a*(\x)^2+(\b)*\x+(\c)});
            \end{scope}
        \end{tikzpicture}
    }
    \loigiai{
        Đặt $g(x)=f\left(x-x^2\right)$\\
        Dựa vào đồ thị ta thấy $f'(x)=0$ có hai nghiệm $x_1=1,x_2=2$ nên $f'(x)$ có dạng $$f'(x)=k(x-1)(x-2)$$
        Khi đó $g'(x)=(1-2x)f'\left(x-x^2\right)=0$
        $$ \Leftrightarrow \hoac{&1-2x=0\\&f'\left(x-x^2\right)=0} \Leftrightarrow \hoac{&x=\dfrac{1}{2}\\&x-x^2=1\\&x-x^2=2} \Leftrightarrow \hoac{&x=\dfrac{1}{2}\\& \text{ vô nghiệm}\\&\text{ vô nghiệm.}}$$
        Bảng xét dấu
        \begin{center}
            \begin{tikzpicture}[every node/.style={circle,fill=white,inner sep=0pt},arrow/.style={>=stealth,->,shorten <= 0.3cm,shorten >= 0.3cm},font=\footnotesize,xscale=1,yscale=1]
                \def\mnumline{1} %Số dòng
                \def\mnumcol{5} %Số cột
                \foreach \j in {0,...,\mnumline}
                \foreach \i in {0,...,\mnumcol}{
                    \coordinate (\j\i) at (\i,-\j);
                }
                \pgfmathsetmacro\yline{\mnumline/2-1}
                \path node at (00){$x$} node at (10){$g'(x)$};
                \foreach \x/\mnamex in {01/$-\infty$,03/$\dfrac{1}{2}$,0\mnumcol/$+\infty$} \path node at (\x) {\mnamex};
                \foreach \dy/\mnamedy in {12/$+$,13/$0$,14/$-$} \path node at (\dy) {\mnamedy};
                \draw[thick] (-.5,.5)rectangle([xshift=0.5cm,yshift=-0.5cm]\mnumline\mnumcol) ([xshift=-0.5cm,yshift=-0.5cm]00)--([xshift=0.5cm,yshift=-0.5cm]0\mnumcol)  ([xshift=0.5cm,yshift=0.5cm]00)--([xshift=0.5cm,yshift=-0.5cm]\mnumline0);
            \end{tikzpicture}
        \end{center}
        Dựa vào bảng xét dấu, ta thấy $g(x)$ có 1 cực đại.
    }
\end{ex}
\begin{ex}%[2D1K2-2]
    \immini{Cho hàm số bậc bốn $y=f(x)$. Hàm số $y=f'(x)$
        có đồ thị như hình bên. Số điểm cực trị của hàm số $y=f\left(\sqrt{x^{2}+2 x+2}\right)$ là
        \choice[2]
        {$1$}
        {$2$}
        {$4$}
        {\True $3$}}
    {
        \begin{tikzpicture}[line join=round, line cap=round,>=stealth,font=\scriptsize]
            \begin{scope}[scale=0.5]
                \tikzset{label style/.style={font=\footnotesize}}
                \def \xmin{-2}
                \def \xmax{4.5}
                \def \ymin{-2}
                \def \ymax{3.5}
                \def \hamso{0.55*(\x)^3-1.76*(\x)^2-0.31*(\x)+2}
                \draw[->] (\xmin,0)--(\xmax,0) node[below left] {$x$};
                \draw[->] (0,\ymin)--(0,\ymax) node[below left] {$y$};
                \draw (0,0) node [below left] {$O$};
                \begin{scope}
                    \clip (\xmin+0.01,\ymin+0.01) rectangle (\xmax-0.01,\ymax-0.01);
                    \draw[samples=350,domain=-1.3:3.3,smooth,variable=\x] plot (\x,{\hamso});
                \end{scope}
                \draw (-1,0) node[below left]{$-1$} (1,0) node[below]{$1$} (3,0) node[below right]{$3$} (0,2) node[above left]{$2$};
            \end{scope}
        \end{tikzpicture}
    }
    \loigiai{
        $ y'=\dfrac{x+1}{\sqrt{x^2+2x+2}}f'(\sqrt{x^2+2x+2}) $.\\$ y'=0 \Leftrightarrow \hoac{&x=-1\\&f'(\sqrt{x^2+2x+2})=0} \Leftrightarrow \hoac{&x=-1\\&\sqrt{x^2+2x+2}=-1\\&\sqrt{x^2+2x+2}=1\\&\sqrt{x^2+2x+2}=3} \Leftrightarrow \hoac{&x=-1\\&x^2+2x+1=0\\&x^2+2x-7=0}\Leftrightarrow \hoac{&x=-1\\&x=-1 \text{ (nghiệm kép)}\\&x=-1\pm 2\sqrt{2}} $\\
        $ y'=0 $ có 3 nghiệm bội bậc lẻ nên hàm số có 3 điểm cực trị.
    }
\end{ex}
\begin{ex}%[2D1K2-2]
    \immini{Cho hàm số $ y=f(x) $ liên tục trên $ (a,b) $ và có đồ thị như hình bên. Số điểm cực trị của hàm số $ y=\left[f(x)\right]^2 $ trên $ (a;b) $ là
        \choice[2]
        {$4$}
        {$6$}
        {$2$}
        {\True $5$}}
    {
        \begin{tikzpicture}[line join=round, line cap=round,>=stealth,font=\scriptsize]
            \begin{scope}[scale=.35]
                \def \xmin{-3.5}
                \def \xmax{4.5}
                \def \ymin{-4}
                \def \ymax{3.5}
                \def \hamso{-0.37*(\x)^3+0.15*(\x)^2+2.41*(\x)-1}
                \draw[->] (\xmin,0)--(\xmax,0) node[below] {$x$};
                \draw[->] (0,\ymin)--(0,\ymax) node[left] {$y$};
                \draw (0,0) node [below left] {$O$};
                \clip (\xmin+0.01,\ymin+0.01) rectangle (\xmax-0.01,\ymax-0.01);
                \draw[samples=350,domain=-3:4,smooth,variable=\x] plot (\x,{\hamso});
                \draw[dashed] (-3,3.11)--(-3,0) node[below]{$a$} (3,-2.41)--(3,0) node[above]{$b$};
            \end{scope}
        \end{tikzpicture}
    }
    \loigiai{\immini{$ y=\left(f(x)\right)^2 $ nên $ y'=2f(x)f'(x) $.\\$ y'=0 \Leftrightarrow \hoac{&f(x)=0\\&f'(x)=0} \Leftrightarrow \hoac{&x=x_1,\ x=x_2,\ x=x_3\\&x=c,\ x=d}$.\\
            $ y'=0 $ có 5 nghiệm bội bậc lẻ thuộc $ (a,b) $ nên Số điểm cực trị của hàm số $ y=\left(f(x)\right)^2 $ trên $ (a;b) $ là 5.}
        {
            \begin{tikzpicture}[line join=round, line cap=round,>=stealth,thick,scale=0.8]
                \tikzset{label style/.style={font=\footnotesize}}
                \def \xmin{-3.5}
                \def \xmax{4.5}
                \def \ymin{-4}
                \def \ymax{3.5}
                \def \hamso{-0.37*(\x)^3+0.15*(\x)^2+2.41*(\x)-1}
                \draw[->] (\xmin,0)--(\xmax,0) node[below left] {$x$};
                \draw[->] (0,\ymin)--(0,\ymax) node[below left] {$y$};
                \draw (0,0) node [below left] {$O$};
                \begin{scope}
                    \clip (\xmin+0.01,\ymin+0.01) rectangle (\xmax-0.01,\ymax-0.01);
                    \draw[samples=350,domain=-3:4,smooth,variable=\x] plot (\x,{\hamso});
                \end{scope}
                \draw[dashed] (-3,3.11)--(-3,0) node[below]{\footnotesize $a$} (3,-2.41)--(3,0) node[above]{\footnotesize $b$} (2.54,0) node[below left]{\footnotesize $x_3$} (0.41,0) node[below right]{\footnotesize $x_2$} (-2.55,0) node[above right]{\footnotesize $x_1$} (-1.34,-3.07) -- (-1.34,0) node[above]{\footnotesize $c$} (1.61,1.72)--(1.61,0) node[below]{\footnotesize $d$};
            \end{tikzpicture}
        }
    }
\end{ex}
\begin{ex}%[2D1G2-1]
    \immini{Cho hàm số $y=f(x)$ có đạo hàm trên $\mathbb{R}$ và có bảng xét dấu $f'(x)$ như hình bên. Hàm số $y=f\left(x^{2}-2 x\right)$ có bao nhiêu điểm cực tiểu?
        \choice
        {\True $1$}
        {$2$}
        {$3$}
        {$4$}}{\begin{tikzpicture}
            \tkzTabInit[lgt=1,espcl=1.2]
            {$x$ /.7, $y'$ /.7}
            {$-\infty$,$-2$,$1$,$3$,$+\infty$}
            \tkzTabLine{ ,-,0,+,0,+,0,-, }
    \end{tikzpicture}}
    \loigiai{$ y'=(2x-2)f'(x^2-2x) $.
        \begin{eqnarray*}
            y'=0 	&\Leftrightarrow& \hoac{&x=1\\&f'(x^2-2x)=0}\\
            &\Leftrightarrow& \hoac{&x=1\\&x^2-2x=-2 \text{ (vô nghiệm)}\\&x^2-2x=1 \text{ (nghiệm bội bậc chẵn)}\\&x^2-2x=3} \\
            &\Leftrightarrow& \hoac{&x=1\\&x=1-\sqrt{2} \text{ (nghiệm bội bậc chẵn)}\\&x=1+\sqrt{2} \text{ (nghiệm bội bậc chẵn)}\\&x=3, \ x=-1.}
        \end{eqnarray*}
        $ y'=0 $ có 3 nghiệm bội bậc lẻ, khi đó $ y' $ đổi dấu qua các nghiệm này.\\
        $ y'=0 $ có 2 nghiệm bội bậc chẵn và $ y' $ sẽ không đổi dấu qua các nghiệm này.\\
        Tại $ x=4 $ thì $ y'(4)=(2\cdot 4 -2)f'(4^2-2\cdot 4)=6f'(8)<0 $.\\
        Bảng xét dấu
        \begin{center}
            \begin{tikzpicture}
                \tkzTabInit[lgt=1,espcl=1.2]
                {$x$ /1, $y'$ /1}
                {$-\infty$,$-1$,$1-\sqrt{2}$,$1+\sqrt{2}$,$3$,$+\infty$}
                \tkzTabLine{ ,-,0,+,0,+,0,+,0,-, }
            \end{tikzpicture}
        \end{center}
        Vậy hàm số có 1 điểm cực tiểu.
    }
\end{ex}
\begin{ex}%[2D1K2-6]
    \immini{Cho hàm số $f(x)$ có bảng biến thiên bên dưới. Trên khoảng $(-\sqrt{5};\sqrt{5})$ thì hàm số $y=f(x^2)$ đạt cực đại tại điểm nào sau đây?\choice
        {$x=\sqrt{2}$}
        {$x=-\sqrt{2}$}
        {\True $x=0$}
        {$x=2$}}{\begin{tikzpicture}
            \tkzTabInit[nocadre=false,lgt=1,espcl=1.6,deltacl=0.5]{$x$/.7 ,$f$/.7}
            {$-\infty$ , $0$ , $2$ , $+\infty$}
            \tkzTabLine{  , + , 0, - , 0 , +  }
    \end{tikzpicture}}
    \loigiai{Đặt $g(x)=f(x^2)$.\\
        Khi đó $g'(x)=2x \cdot f'(x^2)$.\\
        Cho $g'(x)=0 \Leftrightarrow 2x \cdot f'(x^2) =0 \Leftrightarrow
        \hoac{&x=0\\&f'(x^2)=0 \Leftrightarrow \hoac{x^2=0\\x^2=2} \Leftrightarrow \hoac{x=0\\x=\pm \sqrt{2}}}$\\
        Bảng xét dấu
        \begin{center}
            \begin{tikzpicture}[every node/.style={circle,fill=white,inner sep=0pt},arrow/.style={>=stealth,->,shorten <= 0.3cm,shorten >= 0.3cm},font=\footnotesize,xscale=1,yscale=.7]
                \def\mnumline{3} %Số dòng
                \def\mnumcol{9} %Số cột
                \foreach \j in {0,...,\mnumline}
                \foreach \i in {0,...,\mnumcol}{
                    \coordinate (\j\i) at (\i,-\j);
                }
                \pgfmathsetmacro\yline{\mnumline/2-1}
                \path node at (00){$x$} node at (10){$x$} node at (20){$f'(x^2)$} node at (30){$g'(x)$};
                \foreach \x/\mnamex in {01/$-\sqrt{5}$,03/$-\sqrt{2}$,05/$0$,07/$\sqrt{2}$,0\mnumcol/$\sqrt{5}$} \path node at (\x) {\mnamex};
                \foreach \dy/\mnamedy in {12/$-$,13/$0$,14/$+$,16/$+$} \path node at (\dy) {\mnamedy};
                \path node at ($(12)$){$-$} node at ($(13)$){$|$} node at ($(14)$){$-$} node at ($(15)$){$0$} node at ($(16)$){$+$} node at ($(17)$){$|$} node at ($(18)$){$+$} node at ($(22)$){$+$} node at ($(23)$){$0$} node at ($(24)$){$-$} node at ($(25)$){$0$} node at ($(26)$){$-$} node at ($(27)$){$0$} node at ($(28)$){$+$} node at ($(32)$){$-$} node at ($(33)$){$0$} node at ($(34)$){$+$} node at ($(35)$){$0$} node at ($(36)$){$-$} node at ($(37)$){$0$} node at ($(38)$){$+$};
                \draw[thick] (-.5,.5)rectangle([xshift=0.5cm,yshift=-0.5cm]\mnumline\mnumcol) ([xshift=-0.5cm,yshift=-0.5cm]00)--([xshift=0.5cm,yshift=-0.5cm]0\mnumcol) ([xshift=-0.5cm,yshift=-0.5cm]10)--([xshift=0.5cm,yshift=-0.5cm]1\mnumcol)

                ([xshift=-0.5cm,yshift=-0.5cm]20)--([xshift=0.5cm,yshift=-0.5cm]2\mnumcol)

                ([xshift=0.5cm,yshift=0.5cm]00)--([xshift=0.5cm,yshift=-0.5cm]\mnumline0); %Lệnh tự động kẻ bảng
            \end{tikzpicture}
        \end{center}
        Dựa vào bảng xét dấu ta xác định được hàm số đạt cực đại tại $x=0$.
    }
\end{ex}
\begin{ex}%[2D1K2-6]
    \immini{Cho hàm số $f(x)$ có bảng biến thiên bên dưới. Hàm số $y=f(x^2-2)$ đạt cực đại tại điểm nào sau đây?
        \choice
        {$x=-2$}
        {$x=-1$}
        {\True $x=0$}
        {$x=2$}}{\begin{tikzpicture}
            \tkzTabInit[nocadre=false,lgt=1,espcl=1.6,deltacl=0.5]{$x$/.7 ,$f$/.7}
            {$-\infty$ , $-1$ , $2$ , $+\infty$}
            \tkzTabLine{  , - , 0, - , 0 , +  }
    \end{tikzpicture}}
    \loigiai{Đặt $g(x)=f(x^2-2)$\\
        Khi đó $g'(x)=2x \cdot f'(x^2-2)$\\
        Cho $g'(x)=0 \Leftrightarrow 2x \cdot f'(x^2-2) =0$
        $$ \Leftrightarrow \hoac{&x=0\\&f'(x^2-2)=0 \Leftrightarrow \hoac{&x^2-2=-1\\&x^2-2=2} \Leftrightarrow \hoac{x^2=1\\x^2=4} \Leftrightarrow \hoac{x=\pm 1\\x=\pm 2}}$$
        Bảng xét dấu
        \begin{center}
            \begin{tikzpicture}[every node/.style={circle,fill=white,inner sep=0pt},arrow/.style={>=stealth,->,shorten <= 0.3cm,shorten >= 0.3cm},font=\footnotesize,xscale=1,yscale=1]
                \def\mnumline{3} %Số dòng
                \def\mnumcol{14} %Số cột
                \foreach \j in {0,...,\mnumline}
                \foreach \i in {0,...,\mnumcol}{
                    \coordinate (\j\i) at (\i,-\j);
                }
                \pgfmathsetmacro\yline{\mnumline/2-1}
                \path node at ([xshift=0.5cm]00){$x$} node at ([xshift=0.5cm]10){$x$}  node at ([xshift=0.5cm]20){$f'\left(x^2-2\right)$} node at ([xshift=0.5cm]\mnumline0){$g'(x)$};
                \foreach \x/\mnamex in {02/$-\infty$,04/$-2$,06/$-1$,08/$0$,010/$1$,012/$2$,0\mnumcol/$+\infty$} \path node at (\x) {\mnamex};
                \foreach \dy/\mnamedy in {13/$-$,14/$0$,15/$+$,16/$+$} \path node at (\dy) {\mnamedy};
                \path node at ($(13)$){$-$} node at ($(14)$){$|$} node at ($(15)$){$-$} node at ($(16)$){$|$} node at ($(17)$){$-$} node at ($(18)$){$0$} node at ($(19)$){$+$} node at ($(110)$){$|$} node at ($(111)$){$+$} node at ($(112)$){$|$} node at ($(113)$){$+$}
                node at ($(23)$){$+$} node at ($(24)$){$0$} node at ($(25)$){$-$} node at ($(26)$){$0$} node at ($(27)$){$-$} node at ($(28)$){$|$} node at ($(29)$){$-$} node at ($(210)$){$0$} node at ($(211)$){$-$} node at ($(212)$){$0$} node at ($(213)$){$+$}
                node at ($(33)$){$-$} node at ($(34)$){$0$} node at ($(35)$){$+$} node at ($(36)$){$0$} node at ($(37)$){$+$} node at ($(38)$){$0$} node at ($(39)$){$-$} node at ($(310)$){$0$} node at ($(311)$){$-$} node at ($(312)$){$0$} node at ($(313)$){$+$};
                \draw[thick] (-.5,.5)rectangle([xshift=0.5cm,yshift=-0.5cm]\mnumline\mnumcol) ([xshift=-0.5cm,yshift=-0.5cm]00)--([xshift=0.5cm,yshift=-0.5cm]0\mnumcol)
                ([xshift=-0.5cm,yshift=-0.5cm]20)--([xshift=0.5cm,yshift=-0.5cm]2\mnumcol)
                ([xshift=-0.5cm,yshift=-0.5cm]10)--([xshift=0.5cm,yshift=-0.5cm]1\mnumcol) ([xshift=0.5cm,yshift=0.5cm]01)--([xshift=0.5cm,yshift=-0.5cm]\mnumline1); %Lệnh tự động kẻ bảng
            \end{tikzpicture}
        \end{center}
        Dựa vào bảng xét dấu ta xác định được hàm số đạt cực đại tại $x=0$.
    }
\end{ex}
\begin{ex}%[2D1K2-1]
    Cho hàm số $ y=f(x) $ có đạo hàm $ f'(x)=x^2(x-1)(x-4)^2 $. Khi đó hàm số $ y=f(x^2) $ có bao nhiêu điểm cực trị?
    \choice
    {$4$}
    {\True $3$}
    {$5$}
    {$2$}
    \loigiai{$ f'(x)=0 \Leftrightarrow x=1 $ (nghiệm đơn), $ x=0 $ (nghiệm kép), $ x=4 $ (nghiệm kép).\\
        $ y=f(x^2) $ thì $ y'=2xf'(x^2) $.\\$y'=0 \Leftrightarrow \hoac{&x=0\\&x^2=1\\&x^2=0 \text{ (nghiệm kép)}\\&x^2=4 \text{ (nghiệm kép)}} \Leftrightarrow \hoac{&x=0\\&x=\pm 1\\&x=0 \text{ (nghiệm bội chẵn)}\\&x=\pm 2 \text{ (nghiệm bội chẵn).}} $\\
        Vậy hàm số có 3 điểm cực trị.
    }
\end{ex}
\begin{ex}%[2D1K2-6]
    Cho hàm $f(x)$ có đạo hàm $f'(x)=x^2-2x,\forall x\in \mathbb{R}$. Hàm số $y=f\left(1-\dfrac{1}{2}x\right)+4x$ có bao nhiêu điểm cực trị?
    \choice
    {0}
    {1}
    {\True 2}
    {3}
    \loigiai{Ta có $y'=-\dfrac{1}{2}f'\left(1-\dfrac{1}{2}x\right)+4$\\
        $y'=0 \Leftrightarrow
        f'\left(1-\dfrac{1}{2}x\right)=8\Leftrightarrow \left(1-\dfrac{1}{2}x\right)^2-2\left(1-\dfrac{1}{2}x\right)=8 \Leftrightarrow \dfrac{1}{4}x^2-9=0
        \Leftrightarrow \hoac{&x=-6\\&x=6}$\\
        Bảng xét dấu
        \begin{center}
            \begin{tikzpicture}[every node/.style={circle,fill=white,inner sep=0pt},arrow/.style={>=stealth,->,shorten <= 0.3cm,shorten >= 0.3cm},font=\footnotesize,xscale=1,yscale=1]
                \def\mnumline{1} %Số dòng
                \def\mnumcol{7} %Số cột
                \foreach \j in {0,...,\mnumline}
                \foreach \i in {0,...,\mnumcol}{
                }
                \pgfmathsetmacro\yline{\mnumline/2-1}
                \path node at (00){$x$} node at (10){$y'$};
                \foreach \x/\mnamex in {01/$-\infty$,03/$-6$,05/$6$,0\mnumcol/$+\infty$} \path node at (\x) {\mnamex};
                \foreach \dy/\mnamedy in {12/$+$,13/$0$,14/$-$,15/$0$,16/$+$} \path node at (\dy) {\mnamedy};
                \draw[thick] (-.5,.5)rectangle([xshift=0.5cm,yshift=-0.5cm]\mnumline\mnumcol) ([xshift=-0.5cm,yshift=-0.5cm]00)--([xshift=0.5cm,yshift=-0.5cm]0\mnumcol)  ([xshift=0.5cm,yshift=0.5cm]00)--([xshift=0.5cm,yshift=-0.5cm]\mnumline0); %Lệnh tự động kẻ bảng
            \end{tikzpicture}
        \end{center}
        Vậy hàm số $y=f\left(1-\dfrac{1}{2}x\right)+4x$ có 2 điểm cực trị.}
\end{ex}
\begin{ex}%[2D1G2-1]
    Cho hàm số $ y=f(x) $ có đạo hàm $ f'(x)=(x-1)^2(x^2-2x) $, với mọi $ x \in \mathbb{R} $. Có bao nhiêu giá trị nguyên dương của tham số $m$ để hàm số $ y=f(x^2-8x+m) $ có 5 điểm cực trị?
    \choice
    {\True $15$}
    {$16$}
    {$17$}
    {$18$}
    \loigiai{$ f'(x)=0 \Leftrightarrow x=1 $ (nghiệm kép), $ x=0 $ (nghiệm đơn), $ x=2 $ (nghiệm đơn).\\
        $ y=f(x^2-8x+m) $ thì $ y'=(2x-8)f'(x^2-8x+m) $.\\$y'=0 \Leftrightarrow \hoac{&x=4\\&x^2-8x+m=1 \text{ (nghiệm kép)}\\&x^2-8x+m=0 \quad (1)\\&x^2-8x+m=2 \quad (2)} $.\\
        Hàm số có 5 điểm cực trị $ \Leftrightarrow (1) $ có 2 nghiệm phân biệt khác 4 và $ (2) $ có 2 nghiệm phân biệt khác 4.\\$(1) $ có 2 nghiệm phân biệt khác 4 $ \Leftrightarrow \heva{&16-32+m \ne 0\\&\Delta'=16-m>0} \Leftrightarrow \heva{&m \ne 16\\&m<16}\Leftrightarrow m<16$.\\
        $(2) $ có 2 nghiệm phân biệt khác 4 $ \Leftrightarrow \heva{&16-32+m \ne 2\\&\Delta'=16-m+2>0} \Leftrightarrow \heva{&m \ne 18\\&m<18} \Leftrightarrow m<18$.\\
        Vậy ta có $ m<16 $ mà $ m $ nguyên dương nên $ m \in \{1,2,\cdots,15\} $ (15 số $ m $ thỏa mãn).
    }
\end{ex}
\begin{ex}%[2D1Y2-2]
    \immini
    {
        Cho hàm số $y=f(x)$ có đạo hàm liên tục trên $\mathbb{R}$. Đồ thị hàm số $y=f'(x)$ như hình vẽ bên. Số điểm cực trị của hàm số $y=f(x)-5x$ là
        \choice
        {$2$}
        {$3$}
        {$4$}
        {\True $1$}
    }
    {
        \begin{tikzpicture}[font=\scriptsize, line join=round, line cap=round, >=stealth,y=.8cm]
            \begin{scope}[scale=.6]
                \draw[->,>=latex](-3,0)--(3,0)node[above]{$x$};
                \draw[->,>=latex](0,-1)--(0,5)node[right]{$y$};
                \node[above left] at (0,0){$O$};
                \draw plot [samples=100,domain=-2.1:2.1] (\x,{(\x)^3-3*(\x)+2});
                \foreach\i in{-1,1}{\node[below] at (\i,0){$\i$};}
                \foreach\i in{4,2}{\node[right] at (0,\i){$\i$};}
                \draw[dashed](-1,0)--(-1,4)--(0,4);
            \end{scope}
        \end{tikzpicture}
    }
    \loigiai
    {
        Gọi $g(x)=f(x)-5x$. Ta có đạo hàm $g'(x)=f'(x)-5$. Bảng biến thiên của $g'(x)$ như hình dưới.
        \begin{center}
            \begin{tikzpicture}
                \tkzTabInit[nocadre=false,lgt=1.2,espcl=2.5,deltacl=0.6]
                {$x$/1, $f'(x)$/2, $g'(x)$/2}
                {$-\infty$, $-1$, $1$, $+\infty$}
                \tkzTabVar{-/ $-\infty$, +/$4$, -/$0$, +/$+\infty$}
                \tkzTabVar{-/ $-\infty$, +/$-1$, -/$-5$, +/$+\infty$}
            \end{tikzpicture}
        \end{center}
        Ta thấy $g'(x)$ chỉ đổi dấu một lần từ âm sang dương.\\
        Vì vậy hàm số $y=f(x)-5x$ có một điểm cực trị.
    }
\end{ex}
\begin{ex}%[2D1Y2-2]
    \immini
    {
        Cho hàm số $y=f(x)$ có đạo hàm trên $\mathbb{R}$. Biết hàm số $y=f'(x)$ có đồ thị như hình vẽ. Khẳng định nào sau đây đúng về cực trị của hàm số $g(x)=f(x)+x$?
        \choice
        {Hàm số có một điểm cực đại và một điểm cực tiểu}
        {Hàm số không có điểm cực đại và một điểm cực tiểu}
        {Hàm số có một điểm cực đại và hai điểm cực tiểu}
        {\True Hàm số có hai điểm cực đại và một điểm cực tiểu}
    }
    {
        \begin{tikzpicture}[ font=\scriptsize, line join=round, line cap=round, >=stealth]
            \begin{scope}[scale=.5]
                \foreach\x in{-1,0,...,3}{\draw[color=gray!30](\x,-3)--(\x,3.3);}
                \foreach\y in{-2,-1,...,3}{\draw[color=gray!30](-2,\y)--(4,\y);}
                \draw[->,>=latex](-2,0)--(4,0)node[above]{$x$};
                \draw[->,>=latex](0,-3)--(0,3.3)node[right]{$y$};
                \node[above left] at (0,0){$O$};
                \draw plot [samples=100,domain=-1.12:3.1] (\x,{-(\x)^3+3*(\x)^2-2});
            \end{scope}
        \end{tikzpicture}
    }
    \loigiai
    {
        Ta có $g'(x)=f'(x)+1$. Bảng biến thiên của $g'(x)$ như hình dưới.
        \begin{center}
            \begin{tikzpicture}
                \tkzTabInit[nocadre=false,lgt=1.2,espcl=2.5,deltacl=0.6]
                {$x$/1, $f'(x)$/2, $g'(x)$/2}
                {$-\infty$, $0$, $2$, $+\infty$}
                \tkzTabVar{+/ $+\infty$, -/$-2$, +/$2$, -/$-\infty$}
                \tkzTabVar{+/ $+\infty$, -/$-1$, +/$3$, -/$-\infty$}
            \end{tikzpicture}
        \end{center}
        Dựa vào bảng biến thiên của $g'(x)$, ta thấy đạo hàm đổi dấu từ dương sang âm hai lần, từ âm sang dương một lần.\\
        Do đó hàm số $g(x)$ có hai điểm cực đại và một điểm cực tiểu.
    }
\end{ex}
\begin{ex}%[2D1K2-6]
    \immini{	Cho hàm số $y=f(x)$ có đạo hàm trên $\mathbb{R}$ và có đồ thị hàm số $f'(x)$ như hình vẽ. Hàm số $y=2f(x)+x^2$ đạt cực đại tại điểm nào sau đây ?
        \choice[2]
        {\True $x=-1$}
        {$x=0$}
        {$x=1$}
        {$x=2$}}{\begin{tikzpicture}[>=stealth,font=\scriptsize,x=1.3cm]
            \begin{scope}[scale=.7]
                \draw[->] (-2,0) -- (3,0) node[below] {\scriptsize $x$};
                \draw[->] (0,-3) -- (0,2.5) node[left] { $y$};
                \draw (0,0)node[below left]{$O$} (-1.2,0) node[below]{ $-1$} (0,-2) node[below right]{ $-2$} (0,1) node[above left]{$1$} (0,-1) node[below right]{  $-1$} (1,0) node[above]{$1$} (2,0) node[above]{ $2$};
                \draw plot[smooth,tension=.65] coordinates{(-1.05,1.7) (-0.5,-2.6) (0.17,0.5) (0.9,-0.9) (1.5,-1.1) (2.1,-1.9) (2.3,2)};
                \draw[dashed] (-1,0) -- (-1,1) -- (0,1) (1,0)--(1,-1)--(0,-1) (2,0)--(2,-2)--(0,-2);
            \end{scope}
    \end{tikzpicture}}
    \loigiai{
        Đặt $g(x)=2f(x)+x^2$\\
        Khi đó $g'(x)=2f'(x)+2x=0 \Leftrightarrow 2\left(f'(x)+x\right)=0 \Leftrightarrow f'(x)=-x \quad (*)$\\
        Số nghiệm của phương trình $(*)$ là số giao điểm của đồ thị hàm số $y=f'(x)$ và $y=-x$\\
        Dựa vào hình bên ta thấy có $4$ giao điểm lần lượt có tọa độ là $(-1;1),(0;0),(1;-1)$ và $(2;-2)$ \\ $ \Rightarrow (*)  \Leftrightarrow \hoac{&x=-1 \quad \text{(đơn)}\\&x=0 \quad \text{(đơn)}\\&x=1 \quad \text{(kép)}\\&x=2 \quad \text{(kép)}.}$\\
        Bảng xét dấu
        \begin{center}
            \begin{tikzpicture}[every node/.style={circle,fill=white,inner sep=0pt},arrow/.style={>=stealth,->,shorten <= 0.3cm,shorten >= 0.3cm},font=\footnotesize,xscale=1,yscale=1]
                \def\mnumline{1} %Số dòng
                \def\mnumcol{11} %Số cột
                \foreach \j in {0,...,\mnumline}
                \foreach \i in {0,...,\mnumcol}{
                    \coordinate (\j\i) at (\i,-\j);
                }
                \pgfmathsetmacro\yline{\mnumline/2-1}
                \path node at (00){$x$} node at (10){$g'(x)$};
                \foreach \x/\mnamex in {01/$-\infty$,03/$-1$,05/$0$,07/$1$,09/$2$,0\mnumcol/$+\infty$} \path node at (\x) {\mnamex};
                \foreach \dy/\mnamedy in {12/$+$,13/$0$,14/$-$,15/$0$,16/$+$,17/$0$,18/$+$,19/$0$,110/$+$} \path node at (\dy) {\mnamedy};
                \draw[thick] (-.5,.5)rectangle([xshift=0.5cm,yshift=-0.5cm]\mnumline\mnumcol) ([xshift=-0.5cm,yshift=-0.5cm]00)--([xshift=0.5cm,yshift=-0.5cm]0\mnumcol)  ([xshift=0.5cm,yshift=0.5cm]00)--([xshift=0.5cm,yshift=-0.5cm]\mnumline0);
            \end{tikzpicture}
        \end{center}
        Dựa vào bảng xét dấu, ta thấy $g(x)$ đạt cực đại tại $x=-1$.
    }
\end{ex}
\begin{ex}%[2D1K2-6]
    \immini{Hàm số $y=f(x)$ liên tục trên $\mathbb{R}$ và có đồ thị hàm số $f'(x)$ như hình vẽ bên dưới. Hàm số $y=f(x)-\dfrac{1}{3}x^3+x^2-x+2$ đạt cực đại tại điểm nào sau đây ?
        \choice[2]
        {\True $x=1$}
        {$x=-1$}
        {$x=0$}
        {$x=2$}}{\begin{tikzpicture}[>=stealth,font=\scriptsize,y=.7cm]
            \begin{scope}[scale=.8]
                \draw[->] (-2,0) -- (3,0) node[below] {\scriptsize $x$};
                \draw[->] (0,-3) -- (0,2.5) node[left] {\scriptsize $y$};
                \draw (0,0)node[below left]{ $O$}  (-1.2,0) node[below left]{ $-1$} (0,-2) node[right]{ $-2$} (0,1) node[above left]{  $1$} (1,0) node[below]{ $1$} (2,0) node[below]{ $2$};
                \draw plot [samples=100,domain=-1.1:2.2] (\x,{(\x)^3-2*(\x)^2+1});
                \draw[dashed] (-1,0) -- (-1,-2) -- (0,-2) (0,1)--(2,1)--(2,0);
            \end{scope}
    \end{tikzpicture}}
    \loigiai{
        Đặt $g(x)=f(x)-\dfrac{1}{3}x^3+x^2-x+2$	\\
        Khi đó $g'(x)=f'(x)-x^2+2x-1$.\\
        $g'(x)=0 \Leftrightarrow f'(x)=x^2-2x+1 \quad (*)$
        \immini{Số nghiệm của $(*)$ cũng chính là số giao điểm của đồ thị hàm số $y=f'(x)$ với $y=x^2-2x+1$\\
            Dựa vào hình vẽ bên, ta thấy có $3$ giao điểm lần lượt có tọa độ là $(1;0),(2;1),(0;1)$. Khi đó,
            $(*) \Leftrightarrow \hoac{&x=1\\&x=0\\&x=2.}$
        }
        {\begin{tikzpicture}[>=stealth,x=1.0cm,y=1.0cm,scale=0.6]
                \draw[->] (-2,0) -- (3,0) node[below] {\scriptsize $x$};
                \draw[->] (0,-3) -- (0,2.5) node[left] {\scriptsize $y$};
                \draw (0,0)node[below right]{\scriptsize $O$} circle(1.5pt) (-1.2,0) node[below]{\scriptsize $-1$} (0,-2) node[right]{\scriptsize  $-2$} (0,1) node[left]{\scriptsize  $1$} (1,0) node[below]{\scriptsize  $1$} (2,0) node[below]{\scriptsize  $2$};
                \def\a{1}
                \def\b{-2}
                \def\c{1}
                \pgfmathsetmacro\xdinh{-(\b)/2*(\a)}
                \pgfmathsetmacro\ydinh{(4*(\a)*(\c)-(\b)^2)/(4*(\a))}
                \fill[dashed] (\xdinh,\ydinh)circle(2pt) edge (\xdinh,0) edge (0,\ydinh);
                \clip (-2,-3)rectangle(3,3);
                \draw[thick,samples=150,smooth,domain=-5:5] plot(\x,{\a*(\x)^2+(\b)*\x+(\c)});
                \draw[thick] plot[smooth,tension=.65] coordinates{(-1.1,-2.2) (-0.1,1) (1.4,-0.2) (2.5,2.8)};
                \draw[dashed] (-1,0) -- (-1,-2) -- (0,-2) (0,1)--(2,1)--(2,0);
        \end{tikzpicture}}
        \noindent
        Bảng xét dấu
        \begin{center}
            \begin{tikzpicture}[every node/.style={circle,fill=white,inner sep=0pt},arrow/.style={>=stealth,->,shorten <= 0.3cm,shorten >= 0.3cm},font=\footnotesize,xscale=1,yscale=1]
                \def\mnumline{1} %Số dòng
                \def\mnumcol{9} %Số cột
                \foreach \j in {0,...,\mnumline}
                \foreach \i in {0,...,\mnumcol}{
                    \coordinate (\j\i) at (\i,-\j);
                }
                \pgfmathsetmacro\yline{\mnumline/2-1}
                \path node at (00){$x$} node at (10){$g'(x)$};
                \foreach \x/\mnamex in {01/$-\infty$,03/$0$,05/$1$,07/$2$,0\mnumcol/$+\infty$} \path node at (\x) {\mnamex};
                \foreach \dy/\mnamedy in {12/$-$,13/$0$,14/$+$,15/$0$,16/$-$,17/$0$,18/$+$} \path node at (\dy) {\mnamedy};
                \draw[thick] (-.5,.5)rectangle([xshift=0.5cm,yshift=-0.5cm]\mnumline\mnumcol) ([xshift=-0.5cm,yshift=-0.5cm]00)--([xshift=0.5cm,yshift=-0.5cm]0\mnumcol)  ([xshift=0.5cm,yshift=0.5cm]00)--([xshift=0.5cm,yshift=-0.5cm]\mnumline0);
            \end{tikzpicture}
        \end{center}
        Hàm số đạt cực đại tại $x=1$.
    }
\end{ex}
\begin{ex}%[2D1G2-6]
    \immini{	Cho hàm số $f(x)$ có đạo hàm liên tục trên $\mathbb{R}$ và đồ thị $y=f'(x)$ như hình vẽ dưới đây. Xét trên khoảng $(-\pi;2\pi)$, số điểm cực trị của hàm số $g(x)=f(2\cos x)+2\cos2x$ là
        \choice[2]
        {$13$}
        {$10$}
        {\True $11$}
        {$9$}}{\begin{tikzpicture}[>=stealth,font=\scriptsize,x=1.3cm]
            \begin{scope}[scale=.5]
                \draw[->] (-2.5,0) -- (2.5,0) node[below] { $x$};
                \draw[->] (0,-2.5) -- (0,2.5) node[left] {$y$};
                \draw (0,0)node[below left]{ $O$};
                \draw (0,-2)node[below right]{$-2$} (0,2)node[above right]{\scriptsize $2$} (1,0) node[above]{ $1$} (-2,0) node[above]{ $-2$} (-1,0) node[below]{ $-1$} (2,0) node[below]{$2$};
                \draw[dashed] (-2,0)--(-2,-2)--(1,-2)--(1,0) (-1,0)--(-1,2)--(2,2)--(2,0);
                \clip (-2.5,-2.5)rectangle(2.5,3);
                \draw[samples=150,smooth,domain=-2.1:2.1] plot(\x,{(\x)^3-3*\x});

                \fill[black] (-2,0) circle(1.5pt) (-1,0) circle(1.5pt) (1,0) circle(1.5pt) (2,0) circle(1.5pt)(-2,-2) circle(1.5pt)(0,-2) circle(1.5pt)(1,-2) circle(1.5pt)(-1,2) circle(1.5pt)(0,2) circle(1.5pt)(2,2) circle(1.5pt);
            \end{scope}
    \end{tikzpicture}}
    \loigiai{
        Ta có $g'(x)=f'(2\cos x)\cdot(-2\sin x)-2\sin{2x}\cdot2=-2\sin{x}\left[f'(2\cos x)+4\cos x\right]$.\\
        Suy ra $g'(x)=0 \Leftrightarrow \hoac{&\sin x=0\\&f'(2\cos x)=-4\cos x.}$\\
        \begin{itemize}
            \item $\sin x=0 \Leftrightarrow x\in\{0;\pi\}$ vì $x\in(-\pi;2\pi)$.
            \item $f'(2\cos x)=-4\cos x$.\\
            Đặt $t=2\cos x$, vì $x\in(-\pi;2\pi)$ nên $t\in(-1;1)$.\\
            Phương trình trở thành $f'(t)=-2t$. Nghiệm của phương trình này là hoành độ giao điểm của đồ thị hàm số $y=f'(t)$ và đường thẳng $y=-2t$.\\
            \begin{center}
                \begin{tikzpicture}[>=stealth,x=1cm,y=1cm,scale=1]
                    \draw[->] (-2.5,0) -- (2.5,0) node[below] {\scriptsize $t$};
                    \draw[->] (0,-2.5) -- (0,2.5) node[left] {\scriptsize $y$};
                    \draw (0,0)node[below left]{\scriptsize $O$};
                    \draw (0,-2)node[below right]{\scriptsize $-2$} (0,2)node[above right]{\scriptsize $2$} (1,0) node[above]{\scriptsize $1$} (-2,0) node[above]{\scriptsize $-2$} (-1,0) node[below]{\scriptsize $-1$} (2,0) node[below]{\scriptsize $2$};
                    \draw[dashed] (-2,0)--(-2,-2)--(1,-2)--(1,0) (-1,0)--(-1,2)--(2,2)--(2,0);
                    \clip (-2.5,-2.5)rectangle(2.5,2.5);
                    \draw[thick,samples=150,smooth,domain=-2.1:2.1] plot(\x,{(\x)^3-3*\x}) node[right]{$(l)$};
                    \node[above left] at (2,2){\scriptsize $y=f'(t)$};
                    \fill[black] (-2,0) circle(1.5pt) (-1,0) circle(1.5pt) (1,0) circle(1.5pt) (2,0) circle(1.5pt)(-2,-2) circle(1.5pt)(0,-2) circle(1.5pt)(1,-2) circle(1.5pt)(-1,2) circle(1.5pt)(0,2) circle(1.5pt)(2,2) circle(1.5pt);
                    \draw[thick,samples=150,smooth,domain=-2.1:2.1] plot(\x,{-2*(\x)});
                \end{tikzpicture}
            \end{center}
            Suy ra $f'(t)=-2t \Leftrightarrow \hoac{&t=-1\\&t=0\\&t=1.}$

            \begin{itemize}
                \item Với $t=-1 \Rightarrow 2\cos x=-1 \Leftrightarrow \cos x=-\dfrac{1}{2} \Leftrightarrow x\in\left\{-\dfrac{2\pi}{3};\dfrac{2\pi}{3};\dfrac{4\pi}{3}
                \right\}$ vì $x\in(-\pi;2\pi)$.
                \item Với $t=0 \Rightarrow \cos x=0 \Leftrightarrow x\in\left\{-\dfrac{\pi}{2};\dfrac{\pi}{2};\dfrac{3\pi}{2}
                \right\}$ vì $x\in(-\pi;2\pi)$.
                \item Với $t=1 \Rightarrow 2\cos x=1 \Leftrightarrow \cos x=\dfrac{1}{2} \Leftrightarrow x\in\left\{-\dfrac{\pi}{3};\dfrac{\pi}{3};\dfrac{5\pi}{3}
                \right\}$ vì $x\in(-\pi;2\pi)$.
            \end{itemize}
        \end{itemize}
        Và
        \begin{itemize}
            \item $f'(t)+2t>0\Leftrightarrow f'(t)>-2t\Leftrightarrow \hoac{&-1<t<0\\&t>1}\\
            \Rightarrow \hoac{&-\dfrac{1}{2}<\cos x<0\\&\cos x>\dfrac{1}{2}}\Leftrightarrow \hoac{&-\dfrac{2\pi}{3}<x<-\dfrac{\pi}{3}\\&\dfrac{4\pi}{3}<x<\dfrac{5\pi}{3}\\&\dfrac{\pi}{3}<x<\dfrac{2\pi}{3}}$ (vì $x\in(-\pi;2\pi)$).
            \item $f'(t)+2t<0\Leftrightarrow f'(t)<-2t\Leftrightarrow \hoac{&t<-1\\&0<t<1}\\
            \Rightarrow \hoac{&\cos x<-\dfrac{1}{2}\\&0<\cos x<\dfrac{1}{2}} \Leftrightarrow \hoac{&-\pi<x<-\dfrac{2\pi}{3}\\&-\dfrac{\pi}{3}<x<\dfrac{\pi}{3}\\&\dfrac{2\pi}{3}<x<\dfrac{4\pi}{3}}$ (vì $x\in(-\pi;2\pi)$).
        \end{itemize}
        Bảng biến thiên hàm số $y=g(x)$
        \begin{center}
            \begin{tikzpicture}
                \tkzTabInit[nocadre=false,lgt=4,espcl=1]
                {$x$ /1.1,$-2\sin x$ /0.7,$f'(2\cos x)+4\cos x$ /0.7,$g'(x)$ /0.7,$g(x)$ /2}
                {$-\pi$,$-\dfrac{2\pi}{3}$,$-\dfrac{\pi}{2}$,$-\dfrac{\pi}{3}$,$0$,$\dfrac{\pi}{3}$,$\dfrac{\pi}{2}$,$\dfrac{2\pi}{3}$,$\pi$,$\dfrac{4\pi}{3}$,$\dfrac{3\pi}{2}$,$\dfrac{5\pi}{3}$,$2\pi$}
                \tkzTabLine{,+,|,+,|,+,|,+,$0$,-,|,-,|,-,|,-,$0$,+,|,+,|,+,|,+,}
                \tkzTabLine{,-,$0$,+,$0$,-,$0$,+,|,+,$0$,-,$0$,+,$0$,-,|,-,$0$,+,$0$,-,$0$,+,}
                \tkzTabLine{,-,$0$,+,$0$,-,$0$,+,|,-,$0$,+,$0$,-,$0$,+,|,-,$0$,+,$0$,-,$0$,+,}
                \tkzTabVar{+/,-/,+/,-/,+/,-/,+/,-/,+/,-/,+/,-/,+/,}
            \end{tikzpicture}
        \end{center}
        Từ bảng biến thiên ta suy ra hàm số $y=g(x)$ có $11$ điểm cực trị trên khoảng $(-\pi;2\pi)$.
    }
\end{ex}
\begin{ex}%[2D1G2-6]
    \immini{	Cho hàm số $y=f(x)$ có đồ thị của $y=f'(x)$ có đồ thị như hình vẽ bên dưới. Hàm số $g(x)=f(x^3-3x)-x^3+3x$ có bao nhiêu điểm cực tiểu? \choice[2]
        {$2$}
        {$4$}
        {$3$}
        {\True $5$}}{\begin{tikzpicture}[>=stealth,font=\scriptsize]
            \begin{scope}[scale=.5]
                \draw[->,line width = 1pt] (-2,0)--(0,0) node[below left]{$O$}--(5,0) node[below]{$x$};
                \draw[->,line width = 1pt] (0,-2) --(0,3) node[right]{$y$};
                \draw (-1,0) node[below left]{$-1$} circle (1pt);
                \draw (0,2) node[above right]{$2$} circle (1pt);
                \draw (2,0) node[below left]{$2$} circle (1pt);
                \draw (4,0) node[below right]{$4$} circle (1pt);
                \draw [ domain=-1.3:4.6, samples=100] %
                plot (\x, {0.25*(\x)^3-1.25*(\x)^2+0.5*(\x)+2});
            \end{scope}
    \end{tikzpicture}}
    \loigiai{
        $g'(x)=f'(x^3-3x)\cdot (3x^2-3)-3x^2+3=3(x^2-1)\left[f'(x^3-3x)-1\right].\\
        \Rightarrow g'(x)=0\Leftrightarrow \hoac{&x^2=1\\&f'(x^2-3x)=1} \Leftrightarrow \hoac{&x=\pm1\\&x^3-3x=a\quad (-1<a<0)\\&x^3-3x=b\quad (0<b<2)\\&x^3-3x=c\quad (c>4).}$\\
        \begin{center}
            \begin{tikzpicture}[>=stealth]
                \draw[->,line width = 1pt] (-2,0)--(0,0) node[below left]{$O$}--(5,0) node[below]{$x$};
                \draw[->,line width = 1pt] (0,-2) --(0,3) node[right]{$y$};
                \draw (-1,0) node[below left]{$-1$} circle (1pt);
                \draw (0,2) node[above right]{$2$} circle (1pt);
                \draw (2,0) node[below left]{$2$} circle (1pt);
                \draw (4,0) node[below right]{$4$} circle (1pt);
                \draw [thick, domain=-1.3:4.6, samples=100] %
                plot (\x, {0.25*(\x)^3-1.25*(\x)^2+0.5*(\x)+2});
                \draw [thick, domain=-2:5, samples=100] %
                plot (\x, {0*(\x)+1});
                \draw (-1,1) node[above left]{$y=1$};
                \draw (1,1.8) node[right]{$y=f'(x)$};
                \draw (4,0) node[below right]{$4$} circle(1pt);
                \draw (4.323404276086477,1) node[below right] {$c$} circle(1pt);
                \draw (1.3579263675184994,1) node[below left] {$b$} circle(1pt);
                \draw (-0.6813306436049771,1) node[below right] {$a$} circle(1pt);
            \end{tikzpicture}
        \end{center}
        \begin{itemize}
            \item Phương trình $x^3-3x=a$ có $3$ nghiệm $x_1$, $x_2$, $x_3$ với $x_1<x_2<x_3$.
            \item Phương trình $x^3-3x=b$ có $3$ nghiệm $x_4$, $x_5$, $x_6$ với $x_4<x_5<x_6$.
            \item Phương trình $x^3-3x=c$ có $1$ nghiệm $x_7\quad(x_7>x_6)$.
        \end{itemize}
        \begin{center}
            \begin{tikzpicture}[>=stealth]
                \draw[->,line width = 1pt] (-3,0)--(0,0) node[below left]{$O$}--(3,0) node[below]{$x$};
                \draw[->,line width = 1pt] (0,-3) --(0,6) node[right]{$y$};
                \draw (-1,0) node[below left]{$-1$} circle (1pt);
                \draw (0,2) node[above right]{$2$} circle (1pt);
                \draw (1,0) node[above left]{$1$} circle (1pt);
                \draw (2,0) node[below right]{$2$} circle (1pt);
                \draw (0,-2) node[below left]{$-2$} circle (1pt);
                \draw [thick,color=red, domain=-2.1:2.3, samples=100] %
                plot (\x, {(\x)^3-3*(\x)});
                \draw [thick, domain=-3:3, samples=100] plot (\x, {0*(\x)+4.32});
                \draw [thick, domain=-3:3, samples=100] plot (\x, {0*(\x)+1.36});
                \draw [thick, domain=-3:3, samples=100] plot (\x, {0*(\x)-0.68});
                \draw (2.3,5) node[above right]{$y=x^3-3x$};
                \draw (-2,-0.7) node[below left]{$y=a$};
                \draw (-2,1.4) node[below left]{$y=b$};
                \draw (-2,4) node[left]{$y=c$};
                \draw[dashed](-1,0)--(-1,2)--(0,2);
                \draw[dashed](0,-2)--(1,-2)--(1,0);
                \draw (-1.84,-0.68) node[below right] {$x_1$};
                \draw (0.23,-0.68) node[below right] {$x_2$};
                \draw (1.61,-0.68) node[below right] {$x_3$};
                \draw (-1.43,1.36) node[above left] {$x_4$};
                \draw (-0.56,1.36) node[above right] {$x_5$};
                \draw (1.93,1.36) node[below right] {$x_6$};
                \draw (2.22,4.32) node[below right] {$x_7$};
            \end{tikzpicture}
        \end{center}
        Bảng xét dấu $g'(x)$
        \begin{center}
            \begin{tikzpicture}
                \tkzTabInit[nocadre=false,lgt=2.3,espcl=1.3]
                {$x$ /1.1,$x^2-1$ /0.7,$f'(x^3-3x)$ /0.7,$g'(x)$ /0.7,$g(x)$ /2}
                {$-\infty$,$x_1$,$x_4$,$-1$,$x_5$,$x_2$,$1$,$x_3$,$x_6$,$x_7$,$+\infty$}
                \tkzTabLine{,+,|,+,|,+,$0$,-,|,-,|,-,$0$,+,|,+,|,+,|,+,}
                \tkzTabLine{,-,$0$,+,$0$,-,|,-,$0$,+,$0$,-,|,-,$0$,+,$0$,-,$0$,+,}
                \tkzTabLine{,-,$0$,+,$0$,-,$0$,+,$0$,-,$0$,+,$0$,-,$0$,+,$0$,-,$0$,+,}
                \tkzTabVar{+/,-/,+/,-/,+/,-/,+/,-/,+/,-/,+/}
            \end{tikzpicture}
        \end{center}
        Dựa vào bảng xét dấu ta kết luận hàm số $y=g(x)$ có $5$ điểm cực tiểu.
    }
\end{ex}
\begin{ex}%[2D1G2-2]
    \immini{Cho hàm số $y=f(x)$ có đạo hàm và liên tục trên $\mathbb{R}$ và có đồ thị $y=f'(x)$ như hình vẽ. Hàm $y=f(x^2-2)-\dfrac{1}{2}x^4+\dfrac{3}{2}x^2$ có bao nhiêu điểm cực tiểu?
        \choice[2]
        {$4$}
        {$1$}
        {$2$}
        {\True$3$}}
    {\begin{tikzpicture}[>=stealth,font=\scriptsize,x=1.2cm]
            \begin{scope}[scale=.7]
                \def\mx{-1} \def\max{3}
                \def\my{-1} \def\may{3.5}
                \def\hamso(#1,#2){plot [samples=200,smooth,domain=#1:#2](\x,{
                        2*(\x)^4-5*(\x)^3+1.5*(\x)^2+2.25*(\x)
                    })}
                \draw[fill=black]
                (-0.5,0)circle (.7pt)node[shift={(-110:.5)}]{$-\dfrac{1}{2}$}
                (0.5,0)circle (.7pt)node[shift={(-90:.5)}]{$\dfrac{1}{2}$}
                (2,0)circle (.7pt)node[shift={(-90:.5)}]{$2$}
                (1.5,0)circle (.7pt)node[shift={(-90:.5)}]{$\dfrac{3}{2}$}
                (0,1)circle (.7pt)node[shift={(180:.3)}]{$1$}
                (0,2.5)circle (.7pt)node[shift={(180:.3)}]{$\dfrac{5}{2}$}
                (0.5,1)circle (.7pt)
                (2,2.5)circle (.7pt)
                ;
                \draw[dashed,thin] (0.5,0)|-(0,1)(2,0)|-(0,2.5);
                %===========================================
                \draw[->] (\mx,0)--(0,0) node [below right] {$O$}--(\max,0) node[below] {$x$};
                \draw[->] (0,\my)--(0,\may) node[left] {$y$};
                \clip (\mx,\my) rectangle (\max,\may);
                \draw \hamso(\mx,\max);
            \end{scope}
    \end{tikzpicture}}
    \loigiai{
        \immini{
            Ta có $y'=2xf'(x^2-2)-2x^3+3x=2x\left(f'(x^2-2)-x^2+\dfrac{3}{2}\right)$.\\
            $y'=0\Leftrightarrow \hoac{&x=0\\&f'(x^2-2)-x^2+\dfrac{3}{2}=0.\quad(*)}$\\
            Đặt $t=x^2-2$ ta có $(*)\Leftrightarrow f'(t)-t-\dfrac{1}{2}=0\Leftrightarrow f'(t)=t+\dfrac{1}{2}$.\\
            Dựa vào đồ thị hàm số bên ta có
            $$f'(t)=t+\dfrac{1}{2}\Leftrightarrow \hoac{&t=-\dfrac{1}{2}\\&t=\dfrac{1}{2}\\&t=2.}$$
        }{
            \begin{tikzpicture}[scale=1.2,>=stealth,font=\footnotesize,y=.8cm]
                \def\mx{-1} \def\max{3}
                \def\my{-1} \def\may{3.5}
                \def\hamso(#1,#2){plot [samples=200,smooth,domain=#1:#2](\x,{
                        (\x)+0.5
                    })}
                \def\ham(#1,#2){plot [samples=200,smooth,domain=#1:#2](\x,{
                        2*(\x)^4-5*(\x)^3+1.5*(\x)^2+2.25*(\x)
                    })}
                \draw[fill=black]
                (-0.5,0)circle (.7pt)node[shift={(-110:.5)}]{$-\dfrac{1}{2}$}
                (0.5,0)circle (.7pt)node[shift={(-90:.5)}]{$\dfrac{1}{2}$}
                (2,0)circle (.7pt)node[shift={(-90:.5)}]{$2$}
                (1.5,0)circle (.7pt)node[shift={(-90:.5)}]{$\dfrac{3}{2}$}
                (0,1)circle (.7pt)node[shift={(180:.3)}]{$1$}
                (0,2.5)circle (.7pt)node[shift={(180:.3)}]{$\dfrac{5}{2}$}
                (0.5,1)circle (.7pt)
                (2,2.5)circle (.7pt)
                ;
                \draw[dashed,thin] (0.5,0)|-(0,1)(2,0)|-(0,2.5);
                %===========================================
                \draw[->] (\mx,0)--(0,0) node [below right] {$O$}--(\max,0) node[below] {$t$};
                \draw[->] (0,\my)--(0,\may) node[left] {$y$};
                \clip (\mx,\my) rectangle (\max,\may);
                \draw \hamso(\mx,\max)\ham(\mx,\max);
        \end{tikzpicture}}
        \noindent
        Suy ra $\hoac{&x^2-2=-\dfrac{1}{2}\\&x^2-2=\dfrac{1}{2}\\&x^2-2=2}\Leftrightarrow\hoac{&x=\pm\dfrac{\sqrt{6}}{2}\\&x=\pm\dfrac{\sqrt{10}}{2} &\text{ (nghiệm kép)}\\&x=\pm 2.}$\\
        Bảng xét dấu
        \begin{center}
            \begin{tikzpicture}
                \tkzTabInit[nocadre=false, lgt=0.7,espcl=1.6,deltacl=0.5]
                {$x$/1.2, $y'$ /0.6}
                {$-\infty$ , $-2$ , $-\dfrac{\sqrt{10}}{2}$ ,$-\dfrac{\sqrt{6}}{2}$,$0$,$\dfrac{\sqrt{6}}{2}$,$\dfrac{\sqrt{10}}{2}$ ,$2$, $+\infty$}
                \tkzTabLine{ ,-,0,+, 0 ,+, 0 ,-,0,+,0,-,0,-,0,+ }
            \end{tikzpicture}
        \end{center}
        Suy ra hàm số có $3$ điểm cực tiểu.
    }
\end{ex}
\begin{ex}%[2D1G2-6]
    \immini{Cho hàm số $y=f(x)$ có bảng biến thiên bên dưới. Số điểm cực đại và số điểm cực tiểu của hàm số $y=f^2(2x)-2f(2x)+1$ lần lượt là
        \choice
        {\True $2$ và $3$}
        {$3$ và $2$}
        {$1$ và $1$}
        {$2$ và $2$}}{\begin{tikzpicture}
            \tkzTabInit[nocadre=false,lgt=1.2,espcl=2,deltacl=0.6]
            {$x$ /0.6,$f'(x)$ /0.6,$f(x)$ /2}
            {$-\infty$,$-1$,$2$,$+\infty$}
            \tkzTabLine{,-,$0$,+,$0$,-,}
            \tkzTabVar{+/$+\infty$,-/$0$,+/$3$,-/$-\infty$}
    \end{tikzpicture}}
    \loigiai{
        Đặt $g(x)=	f^2(2x)-2f(2x)+1=\left[f(2x)-1\right]^2$.\\
        $\Rightarrow g'(x)=2\cdot\left[f(2x)-1\right]\cdot f'(2x)$.\\
        $\Rightarrow g'(x)=0\Leftrightarrow \hoac{&f(2x)=1\\&f'(2x)=0.}$\\
        \begin{itemize}
            \item $f(2x)=1\Leftrightarrow \hoac{&2x=a\quad(a<-1)\\&2x=b\quad(-1<b<2)\\&2x=c\quad(2<c)}\Leftrightarrow \hoac{&x=\dfrac{a}{2}\quad(\dfrac{a}{2}<-\dfrac{1}{2})\\&x=\dfrac{b}{2}\quad(-\dfrac{1}{2}<\dfrac{b}{2}<1)\\&x=\dfrac{c}{2}\quad(1<\dfrac{c}{2}).}$
            \item $f'(2x)=0\Leftrightarrow\hoac{&2x=-1\\&2x=2}\Leftrightarrow\hoac{&x=-\dfrac{1}{2}\\&x=1.}$
        \end{itemize}
        Bảng biến thiên hàm số $y=g(x)$
        \begin{center}
            \begin{tikzpicture}
                \tkzTabInit[nocadre=false,lgt=2.1,espcl=2,deltacl=0.6]
                {$x$ /1.1,$f'(2x)$ /0.7,$f(2x)-1$ /0.7,$g'(x)$ /0.7,$g(x)$ /2}
                {$-\infty$,$\dfrac{a}{2}$,$-\dfrac{1}{2}$,$\dfrac{b}{2}$,$1$,$\dfrac{c}{2}$,$+\infty$}
                \tkzTabLine{,-,|,-,$0$,+,|,+,$0$,-,|,-,}
                \tkzTabLine{,+,$0$,-,|,-,$0$,+,|,+,$0$,-,}
                \tkzTabLine{,-,$0$,+,$0$,-,$0$,+,$0$,-,$0$,+,}
                \tkzTabVar{+/,-/,+/,-/,+/,-/,+/}
            \end{tikzpicture}
        \end{center}
        Dựa vào bảng biến thiên ta thấy hàm số $y=g(x)$ có $2$ điểm cực đại và $3$ điểm cực tiểu.
    }
\end{ex}

\begin{ex}%[2D1G2-6]
    Cho hàm số bậc ba $y=f(x)$ có đồ thị như hình bên. Có bao nhiêu giá trị nguyên của tham số $m$ để hàm số $y=\vert f^2(x)+2f(x)+m\vert$ có $9$ điểm cực trị?
    \choice
    {\True $24$}
    {Vô số}
    {$25$}
    {$23$}
    \loigiai{
        Đặt $y=g(x)=f^2(x)+2f(x)+m=\left[f(x)+1\right]^2+m-1.\\
        \Rightarrow g'(x)=2\left[f(x)+1\right]\cdot f'(x).\\
        \Rightarrow g'(x)=0\Leftrightarrow \hoac{&f'(x)=0\\&f(x)=-1}\Leftrightarrow \hoac{&x=1\\&x=3\\&x=a\quad(0<a<1)\\&x=b\quad(1<b<3)\\&x=c\quad(3<c).}$\\
        Từ đồ thị ta suy ra
        \begin{itemize}
            \item $f'(x)+1>0\Leftrightarrow f'(x)>-1\Leftrightarrow a<x<b \text{ hoặc } x>c$.
            \item $f'(x)+1<0\Leftrightarrow f'(x)<-1\Leftrightarrow x<a \text{ hoặc } b<x<c$.
        \end{itemize}
        Bảng biến thiên hàm số $y=g(x)$
        \begin{center}
            \begin{tikzpicture}
                \tkzTabInit[nocadre=false,lgt=2.3,espcl=1.8]
                {$x$ /1.1,$f'(x)$ /0.7,$f(x)+1$ /0.7,$g'(x)$ /0.7,$g(x)$ /2}
                {$-\infty$,$0$,$a$,$1$,$b$,$3$,$c$,$+\infty$}
                \tkzTabLine{,+,t,+,|,+,$0$,-,|,-,$0$,+,|,+,}
                \tkzTabLine{,-,t,-,$0$,+,|,+,$0$,-,|,-,$0$,+,}
                \tkzTabLine{,-,t,-,$0$,+,$0$,-,$0$,+,$0$,-,$0$,+,}
                \tkzTabVar{+/,R,-/$m-1$,+/$m+24$,-/$m-1$,+/$m$,-/$m-1$,+/}
            \end{tikzpicture}
        \end{center}
        Đồ thị hàm số $y=|g(x)|$ gồm có $2$ phần như sau:
        \begin{itemize}
            \item Phần 1: Trùng với đồ thị hàm số $y=g(x)$ với $g(x)\ge0$.
            \item Phần 2: Là phần đối xứng với phần đồ thị của hàm số $y=g(x)$ với $g(x)<0$ qua trục $\text{Ox}$.
        \end{itemize}
        Kết hợp với bảng biến thiên hàm số $y=g(x)$ ta suy ra hàm số $y=|g(x)|$ có $9$ điểm cực trị khi và chỉ khi $m\le 0<m+24 \Leftrightarrow -24<m\le0$. Mà $m$ là số nguyên nên ta được $24$ giá trị của $m$.
    }
\end{ex}
\begin{ex}%[2D1K2-6]
    Có bao giá trị nguyên của tham số $m$ thoả mãn $\vert m\vert<10$ sao cho hàm số $y=\vert x^3-(m-2)x^2-mx-m^2\vert$ có $3$ điểm cực tiểu?
    \choice
    {$9$}
    {$10$}
    {\True $8$}
    {$16$}
    \loigiai{
        Đặt hàm số $y=f(x)=x^3-(m-2)x^2-mx-m^2=(x-m)(x^2+2x+m)=(x-m)\left[x(x+2)+m\right]$.
        Suy ra $f'(x)=3x^2-2(m-2)x-m=0$ có $\Delta'=(m-2)^2+3m=m^2-m+4>0$ với mọi $m$.\\
        Theo định lí Vi-ét ta có $\heva{&x_1+x_2=\dfrac{2(m-2)}{3}\\&x_1x_2=-\dfrac{m}{3}}$.\\
        Hàm số $y=|f(x)|$ có $3$ điểm cực tiểu khi và chỉ khi $y(x_1)\cdot y(x_2)<0$.\\
        Thực hiện biến đổi\\
        $y(x_1)\cdot y(x_2)=  (x_1-m)(x_2-m)\left[x_1(x_1+2)+m\right]\left[x_2(x_2+2)+m\right]\\
        = (x_1-m)(x_2-m)\left[x_1x_2(x_1+2)(x_2+2)+m(x_1^2+x_2^2)+2m(x_1+x_2)+m^2\right]\\
        = \left[x_1x_2-m(x_1+x_2)+m^2\right]\left[x_1x_2\left(x_1x_2+2(x_1+x_2)+4\right)+m(x_1^2+x_2^2)+2m(x_1+x_2)+m^2\right]\\
        = \left(\dfrac{m^2}{3}+m\right)\left[-\dfrac{m}{3}\left(m+\dfrac{4}{3}\right)+m\left(\dfrac{4m^2}{9}-\dfrac{10m}{9}+\dfrac{16}{9}\right)+\dfrac{4m^2}{3}-\dfrac{8m}{3}+m^2\right]\\
        = \dfrac{2}{27}m^2(m+3)(2m^2+4m-5)$.\\
        Suy ra $y(x_1)\cdot y(x_2)<0\Leftrightarrow m^2(m+3)(2m^2+4m-5)<0\Leftrightarrow \hoac{&m<-3\\&\dfrac{-2-\sqrt{14}}{2}<m<0\\&0<m<\dfrac{-2+\sqrt{14}}{2}}$.\\
        Kết hợp với điều kiện $m$ là số nguyên thỏa $|m|<10$ ta được $m\in\{-9;-8;-7;-6;-5;-4;-2;-1\}$.\\
        Vậy có $8$ giá trị nguyên của tham số $m$.
    }
\end{ex}

\begin{ex}%[2D1G2-6]
    \immini{	Cho hàm số $f(x)=ax^4+bx^3+cx^2+dx+e, (ae<0)$. Đồ thì hàm số $y=f'(x)$ như hình bên dưới. Hàm số $y=\left|4f(x)-x^2\right|$ có bao nhiêu điểm cực tiểu?
        \choice[2]
        {$4$}
        {$5$}
        {\True $3$}
        {$2$}
    }{\begin{tikzpicture}[>=stealth,font=\scriptsize,x=1.3cm,y=1.5cm]
            \begin{scope}[scale=0.5]
                \draw[->] (-2,0) -- (3,0) node[below] { $x$};
                \draw[->] (0,-1) -- (0,3) node[left] {\ $y$};
                \draw (0,0) node[above left] { $O$} circle (1pt);
                \draw[smooth](0,0) parabola bend (-0.6,-1)(-1.7,1.7);
                \draw(0,0) parabola bend (1.3,2.5)(2,1);
                \draw [dashed] (0,1)--(2,1)--(2,0);
                \draw [dashed] (-1.05,0)--(-1.05,-0.5)--(0,-0.5);
                \draw (-1,0) node[above] { $-1$};
                \draw (0,-0.5) node[right] { $-\dfrac{1}{2}$};
                \draw (0,1) node[left] { $1$};
                \draw (2,0) node[below] { $2$};
            \end{scope}
    \end{tikzpicture}}
    \loigiai{
        Ta có $f'(x)=4ax^3+3bx^2+2cx+d$. Từ đồ thị hàm số $f'(x)$ suy ra $a<0$, do đó $e>0$.\\
        Đặt $y=g(x)=4f(x)-x^2\Rightarrow g'(x)	=4f'(x)-2x=4\left[f'(x)-\dfrac{x}{2}\right]$.\\
        \begin{center}
            \begin{tikzpicture}[>=stealth,x=1.0cm,y=1.0cm,thick, scale=1.0]
                \draw[->] (-2,0) -- (3,0) node[below] {\footnotesize $x$};
                \draw[->] (0,-1) -- (0,3) node[left] {\footnotesize $y$};
                \draw (0,0) node[below left] {\footnotesize $O$} circle (1pt);
                \draw[smooth](0,0) parabola bend (-0.6,-1)(-1.7,1.7);
                \draw(0,0) parabola bend (1.3,2.5)(2,1);
                \draw [dashed] (0,1)--(2,1)--(2,0);
                \draw [dashed] (-1.05,0)--(-1.05,-0.5)--(0,-0.5);
                \draw (-1,0) node[above] {\footnotesize $-1$};
                \draw (0,-0.5) node[right] {\footnotesize $-\dfrac{1}{2}$};
                \draw (0,1) node[left] {\footnotesize $1$};
                \draw (2,0) node[below] {\footnotesize $2$};
                \draw [thick, domain=-2.1:2.5, samples=100] %
                plot (\x, {0.5*(\x)});
            \end{tikzpicture}
        \end{center}
        Suy ra $g'(x)=0\Leftrightarrow f'(x)-\dfrac{x}{2}=0\Leftrightarrow f'(x)=\dfrac{x}{2}\Leftrightarrow \hoac{&x=-1\\&x=0\\&x=2}$.\\
        Bảng biến thiên
        \begin{center}
            \begin{tikzpicture}
                \tkzTabInit[nocadre=false,lgt=1.2,espcl=2.3]
                {$x$ /0.6,$g'(x)$ /0.6,$g(x)$ /2}
                {$-\infty$,$-1$,$0$,$2$,$+\infty$}
                \tkzTabLine{,+,$0$,-,$0$,+,$0$,-,}
                \tkzTabVar{-/,+/,-/$4e$,+/,-/}
            \end{tikzpicture}
        \end{center}
        Vì $4e>0$ nên từ bảng biến thiên hàm số $g(x)$ ta suy ra hàm số $y=\left|g(x)\right|$ có $3$ điểm cực tiểu.
    }
\end{ex}

\begin{ex}%[2D1G2-6]
    \immini{	Cho hàm số bậc bốn $f(x)$ có $f(0)=-1$. Hàm số $y=f'(x)$ có đồ thị là hình bên. Số điểm cực trị của hàm số $y=\vert 4f(x+1)+x^2+2x\vert$ là
        \choice[2]
        {$3$}
        {\True $5$}
        {$4$}
        {$6$}}{\begin{tikzpicture}[>=stealth,font=\scriptsize,y=.8cm]
            \begin{scope}[scale=.5]
                \draw[->] (-3.5,0) -- (5,0) node[below] {$x$};
                \draw[->] (0,-4) -- (0,2) node[left] { $y$};
                \draw (0,0) node[below left] {$O$} circle (1pt);
                \draw (-3.6,-3.6) ..controls +(60:0.2) and +(-180:1.3).. (-1.5,1.2) ..controls +(0:1.6) and +(-180:1.6) .. (2.8,-3.4)..controls +(0:0.5) and +(-100:4.5) .. (4.95,1.8);
                \draw [dashed] (-2,0)--(-2,1)--(0,1);
                \draw [dashed] (0,-2)--(4,-2)--(4,0);
                \draw (-2,0) node[below] { $-2$};
                \draw (0,-2) node[left] { $-2$};
                \draw (0,1) node[right] { $1$};
                \draw (4,0) node[above] {$4$};
            \end{scope}
    \end{tikzpicture}}
    \loigiai{
        Đặt $y=g(x)=4f(x+1)+x^2+2x\Rightarrow g'(x)=4f'(x+1)+2x+2=4\left[f'(x+1)+\dfrac{x+1}{2}\right]$.\\
        Suy ra $g'(x)=0\Leftrightarrow f'(x+1)=-\dfrac{x+1}{2}$.\\
        Đặt $t=x+1$ thì phương trình trở thành $f'(t)=-\dfrac{t}{2}$. Nghiệm của phương trình này là hoành độ giao điểm của đồ thị hàm số $y=f'(t)$ và $y=-\dfrac{t}{2}$.
        \begin{center}
            \begin{tikzpicture}[>=stealth,x=1.0cm,y=1.0cm,thick, scale=0.8]
                \draw[->] (-3.5,0) -- (5,0) node[below] {\footnotesize $t$};
                \draw[->] (0,-4) -- (0,2) node[left] {\footnotesize $y$};
                \draw (0,0) node[below left] {\footnotesize $O$} circle (1pt);
                \draw[very thick] (-3.6,-3.6) ..controls +(60:0.2) and +(-180:1.3).. (-1.5,1.2) ..controls +(0:1.6) and +(-180:1.6) .. (2.8,-3.4)..controls +(0:0.5) and +(-100:4.5) .. (4.95,1.8);
                \draw [thick, domain=-3.5:5, samples=100] %
                plot (\x, {-0.5*(\x)});
                \draw [dashed] (-2,0)--(-2,1)--(0,1);
                \draw [dashed] (0,-2)--(4,-2)--(4,0);
                \draw (-2,0) node[below] {\footnotesize $-2$};
                \draw (0,-2) node[left] {\footnotesize $-2$};
                \draw (0,1) node[right] {\footnotesize $1$};
                \draw (4,0) node[above] {\footnotesize $4$};
            \end{tikzpicture}
        \end{center}
        Do đó\\
        $$f'(t)=-\dfrac{t}{2}\Leftrightarrow\hoac{&t=-2\\&t=0\\&t=4}\Rightarrow \hoac{&x+1=-2\\&x+1=0\\&x+1=4}\Leftrightarrow \hoac{&x=-3\\&x=-1\\&x=3.}$$
        Bảng biến thiên
        \begin{center}
            \begin{tikzpicture}
                \tkzTabInit[nocadre=false,lgt=1.2,espcl=2.3]
                {$x$ /0.6,$g'(x)$ /0.6,$g(x)$ /2}
                {$-\infty$,$-3$,$-1$,$3$,$+\infty$}
                \tkzTabLine{,-,$0$,+,$0$,-,$0$,+,}
                \tkzTabVar{+/,-/,+/$-5$,-/,+/}
            \end{tikzpicture}
        \end{center}
        Từ bảng biến thiên suy ra hàm số $y=g(x)$ có $3$ cực trị âm, do đó hàm số $y=\left|g(x)\right|$ có $5$ điểm cực trị.
    }
\end{ex}
\begin{ex}%[2D1Y2-6]
    \immini{	Cho hàm số $y=f(x)$ có bảng biến thiên như hình vẽ. Hàm số $y=f\left(|x|\right)$ đạt cực đại tại.
        \choice[2]
        {$x=-1$}
        {\True $x=0$}
        {$x=2$}
        {$x=-2$}}{\begin{tikzpicture}[>=stealth,scale=0.9]
            \tkzTabInit[nocadre=false,lgt=1,espcl=2.6,deltacl=0.5]{$x$/.7 ,$y'$/.7,$y$/2}
            {$-\infty$ , $-1$ , $2$ , $+\infty$}
            \tkzTabLine{ , + , $0$ , - , $0$ , + , }
            \tkzTabVar{-/$-\infty$ , +/$3$ , -/$1$ , +/$+\infty$}
    \end{tikzpicture}}
    \loigiai{Từ bảng biến thiên của hàm số $y=f(x)$ ta có bảng biến thiên của hàm số $y=f\left(|x|\right)$ như sau
        \begin{center}
            \begin{tikzpicture}[scale=0.9]
                \foreach \x/\texn in {0/x,2/-\infty,4/-2,6/0,8/2,10/+\infty} \path (\x,3.5)node{$\texn$};
                \foreach \x/\texn in
                {0/y',3/-,4/0,5/+,6/0,7/-,8/0,9/+} \path (\x,2.5)node{$\texn$};
                \foreach \x/\y/\texn in {0/1/y,
                    2/1.5/+\infty,4/0/1,6/1.0/f(0),8/0/1,10/1.5/+\infty}
                \path (\x,\y) node(\x){$\texn$};
                \foreach \x/\y/\texn in {2/4,4/6,6/8,8/10}
                \draw[-stealth] (\x)--(\y);
                \draw
                (-.5,3)--(10.5,3) (1,4)--(1,0)
                (-.5,2)--(10.5,2);
            \end{tikzpicture}
        \end{center}
        Từ bảng biến thiên ta thấy hàm số $y=f\left(|x|\right)$ đạt cực đại tại $x=0$.
    }
\end{ex}
%%=====Câu 70
\begin{ex}%[2D1Y2-6]
    \immini{	Cho hàm số $y=f(x)$ có bảng biến thiên như hình vẽ. Tổng các giá trị cực đại của hàm số $y=\left|f(x)\right|$ là
        \choice[2]
        {\True $9$}
        {$-3$}
        {$3$}
        {$7$}}{\begin{tikzpicture}[>=stealth]
            \tkzTabInit[nocadre=false,lgt=1,espcl=1.7,deltacl=0.5]{$x$/.7 ,$y'$/.7,$y$/2}
            {$-\infty$ , $-1$ , $0$ , $1$ , $+\infty$}
            \tkzTabLine{ , - , $0$ , + , $0$ , - , $0$ , + , }
            \tkzTabVar{+/$+\infty$ , -/$-2$, +/$3$ , -/$-4$ , +/$+\infty$}
    \end{tikzpicture}}
    \loigiai{Từ bảng biến thiên của hàm số $y=f(x)$ ta có bảng biến thiên của hàm số $y=\left|f(x)\right|$ như sau
        \begin{center}
            \begin{tikzpicture}[scale=0.8]
                \foreach \x/\texn in {0/x,2/-\infty,4/x_1,6/-1,8/x_2,10/0,12/x_3,14/1,16/x_4,18/+\infty} \path (\x,3.5)node{$\texn$};
                \foreach \x/\texn in
                {0/y',3/-,4/||,5/+,6/0,7/-,8/||,9/+,10/0,11/-,12/||,13/+,14/0,15/-,16/||,17/+} \path (\x,2.5)node{$\texn$};
                \foreach \x/\y/\texn in {0/1/y,
                    2/1.5/+\infty,4/0/0,6/0.5/2,8/0/0,10/0.7/3,12/0/0,14/1.0/4,16/0/0,18/1.5/+\infty}
                \path (\x,\y) node(\x){$\texn$};
                \foreach \x/\y/\texn in {2/4,4/6,6/8,8/10,10/12,12/14,14/16,16/18}
                \draw[-stealth] (\x)--(\y);
                \foreach \x in {4,8,12}\draw[dashed,red] (\x)--+(3.7,0);
                \draw[dashed,red] (1.2,0)--+(2.7,0) (16)--+(2.0,0) node[below=-.1]{\small $y=0$};
                \draw
                (-.5,3)--(18.5,3) (1,4)--(1,0)
                (-.5,2)--(18.5,2);
            \end{tikzpicture}
        \end{center}
        Từ bảng biến thiên ta thấy hàm số $y=\left|f(x)\right|$ có $3$ giá trị cực đại lần lượt là $2$, $3$, $4$.\\
        Tổng các giá trị cực đại là $9$.
    }
\end{ex}
\begin{ex}%[2D1Y2-6]
    Cho hàm số $y=f(x)$ có đạo hàm $y=f'(x)=(x-1)(x-2)^4(x^2-4)$. Số điểm cực trị của hàm số $y=f(|x|)$ là
    \choice
    {$3$}
    {$2$}
    {$4$}
    {\True $5$}
\end{ex}
\begin{ex}%[2D1Y2-6]
    Cho hàm số $y=f(x)$ có đạo hàm $y=f'(x)=(x^3-2x^2)(x^3-2x)$ trên $\mathbb{R}$. Hàm số $y=|f(4-2021x)|$ có nhiều nhất bao nhiêu điểm cực trị?
    \choice
    {\True$9$}
    {$11$}
    {$2021$}
    { $5$}
\end{ex}
\begin{ex}%[2D1B2-6]
    Có bao nhiêu giá trị nguyên của tham số $m$ để hàm số $y=|3x^4-4x^3-12x^2+m|$ có $7$ điểm cực trị?
    \choice
    {$3$}
    {$5$}
    {$6$}
    {\True $4$}
    \loigiai{
        Đặt $f(x)=3x^4-4x^3-12x^2+m$ $\Rightarrow f'(x)=12x^3-12x^2-24x=0 \Rightarrow x=0; x=-1; x=2$.\\
        Qua BBT của $y=f(x)$ ta suy ra $y=|f(x)|$ có $7$ điểm cực trị $\Rightarrow \heva{&m>0\\&m-5<0} \Rightarrow 0<m<5$. Vậy có $4$ giá trị nguyên $m$ thỏa yêu cầu bài toán.
    }
\end{ex}
\begin{ex}%[2D1B2-6]
    Tìm các giá trị của $m$ để hàm số $f(x)=|x^3+3x^2+m-3|$ có ba điểm cực trị.
    \choice
    {$m=3; m=-1$}
    {$m\ge 1; m \le-3$}
    {$1\le m \le 3$}
    {\True $m\ge 3; m \le -1$}
\end{ex}
\begin{ex}%[2D1B2-6]
    Cho hàm số $y=f(x)=x^3-3mx^2+3(m^2-4)x+1$, có bao nhiêu số nguyên $m \in (-10;10)$  để hàm số $y=f(|x|)$ có đúng $5$ điểm cực trị.
    \choice
    {$3$}
    {$6$}
    {$8$}
    {\True $7$}
    \loigiai{
        $y=f(|x|)$ có đúng $5$ điểm cực trị $\Rightarrow y=f(x)$ có hai điểm cực trị dương.\\
        $f'(x)=3x^2-6mx+3(m^2-4)=0 \Rightarrow x=m-2; x=m+2$ có hai nghiệm dương $\Leftrightarrow m-2>0 \Leftrightarrow m >2$.\\ Vậy có $7$ giá trị $m$ thỏa yêu cầu bài toán.
    }
\end{ex}
\begin{ex}%[2D1G2-6]
    Cho hàm số $f(x)=\dfrac{1}{3}x^3-(2m-1)x^2+(8-m)x+2020$ với $m$ là tham số. Tập hợp tất cả các giá trị của tham số $m$ để hàm số $y=f\left(\vert x\vert\right)$ có điểm $5$ cực trị là khoảng $(a;b)$. Tích $a\cdot b$ bằng
    \choice
    {$12$}
    {$16$}
    {$10$}
    {\True $14$}
    \loigiai{
        Tập xác định $\mathscr{D}=\mathbb{R}$.\\
        Ta có $f\left(|-x|\right)=f\left(|x|\right)$, $\forall x\in\mathbb{R}$ nên $y=f\left(|x|\right)$ là hàm số chẵn. \\
        Do đó, đồ thị hàm số $y=f\left(|x|\right)$ đối xứng qua trục tung.\\
        Suy ra hàm số $y=f\left(|x|\right)$ luôn có một điểm cực trị là $x=0$.\\
        Do đó, $y=f\left(|x|\right)$ có $5$ điểm cực trị $\Leftrightarrow$ hàm số $y=f(x)$ có $2$ điểm cực trị dương.\\
        \phantom{Do đó, $y=f\left(|x|\right)$ có $5$ điểm cực trị} $\Leftrightarrow$  $f'(x)=0$ có hai nghiệm dương phân biệt.\\
        Ta có $f'(x)=x^2-2(m-1)x+8-m$.\\
        Yêu cầu bài toán $\Leftrightarrow\heva{&\Delta'>0 \\ &S>0 \\ &P>0}\Leftrightarrow\heva{&4m^2-3m-7>0 \\ &2m-1>0 \\ &8-m>0}\Leftrightarrow\heva{&m<-1\;\text{hoặc}\;m>\dfrac{7}{4} \\ &m>\dfrac{1}{2} \\ &m<8}\Leftrightarrow \dfrac{7}{4}<m<8$.
        Suy ra $a\cdot b=14$.
    }
\end{ex}
\begin{ex}%[2D1G2-6]
    \immini{	Cho hàm số $f(x)$ có đạo hàm liên tục trên $\mathbb{R}$ và đồ thị hàm số $f'(x)$ như hình vẽ. Hàm số $y=f\left(x^2-2\vert x\vert\right)$ có bao nhiêu điểm cực tiểu?
        \choice[2]
        {$1$}
        {\True $2$}
        {$5$}
        {$3$}}{\begin{tikzpicture}[>=stealth,font=\scriptsize]
            \draw[->] (-2,0) -- (2,0) node[below] {$x$};
            \draw[->] (0,-1) -- (0,2) node[left] {$y$};
            \draw (0,0) node[below right] {$O$} circle(1pt) (-1,0) node[above left]{ $-1$} (1,0) node[below]{ $1$} (0,1) node[right]{ $1$} (-1/3,0) node[below]{\ $-\dfrac{1}{3}$};
            \draw [dashed] (-1/3,0)--(-1/3,1.2);
            \draw plot[smooth,tension=.65] coordinates{(-1.2,-0.5) (-1/3,1.2) (1,0) (1.8,1.5)};
    \end{tikzpicture}}
    \loigiai{
        Đặt $g(x)=f(x^2-2x)\Rightarrow g'(x)=2(x-1)f'(x^2-2x).\\
        g'(x)=0\Leftrightarrow \hoac{&x=1\\&f'(x^2-2x)=0}\Leftrightarrow \hoac{&x=1\\&x^2-2x=-1\\&x^2-2x=1}\Leftrightarrow \hoac{&x=1\text{ (bội 3)}\\&x=1-\sqrt{2}\\&x=1+\sqrt{2}.}$\\
        Ta có
        \begin{itemize}
            \item $f'(x)>0\Leftrightarrow \heva{&x>-1\\&x\neq1}$ nên $f'(x^2-2x)>0\Leftrightarrow \heva{&x^2-2x>-1\\&x^2-2x\neq -1}\Leftrightarrow \heva{&x\neq 1\\&x=1\pm\sqrt{2}.}$
            \item $f'(x)<0\Leftrightarrow x<-1$ nên $f'(x^2-2x)<0\Leftrightarrow x^2-2x<-1 \text{ (Vô nghiệm)}.$
        \end{itemize}
        Bảng biến thiên hàm số $y=g(x)$
        \begin{center}
            \begin{tikzpicture}
                \tkzTabInit[nocadre=false,lgt=2.5,espcl=2.1,deltacl=0.6]
                {$x$ /0.7,$x-1$ /0.7,$f'(x^2-2x)$ /0.7,$g'(x)$ /0.7,$g(x)$ /2}
                {$-\infty$,$1-\sqrt{2}$,$0$,$1$,$1+\sqrt{2}$,$+\infty$}
                \tkzTabLine{,-,|,-,t,-,$0$,+,|,+,}
                \tkzTabLine{,+,$0$,+,t,+,$0$,+,$0$,+,}
                \tkzTabLine{,-,$0$,-,t,-,$0$,+,$0$,+,}
                \tkzTabVar{+/,R,R,-/,R,+/}
            \end{tikzpicture}
        \end{center}
        Do hàm số $y=f(x^2-2\vert x\vert)$ là hàm số chẵn nên từ bảng biến thiên trên ta suy ra đồ thị hàm số $y=f(x^2-2\vert x\vert)$ gồm hai nhánh như sau
        \begin{itemize}
            \item Nhánh thứ nhất là phần đồ thị hàm số $y=g(x)$ với $x\ge 0$.
            \item Nhánh thứ hai là phần đối xứng với nhánh thức nhất qua trục $Oy$
        \end{itemize}
        Do đó hàm số $y=f(x^2-2\vert x\vert)$ có $2$ điểm cực tiểu.

    }
\end{ex}
\begin{ex}%[2D1G2-6]
    \immini{	Cho hàm bậc bốn $y=f(x)$ có đồ thị như hình vẽ dưới đây. Số điểm cực trị của hàm số $g(x)=f\left(\vert x\vert^3-3\vert x\vert\right)$ là
        \choice[2]
        {$5$}
        {$3$}
        {\True $7$}
        {$11$}}{\begin{tikzpicture}[>=stealth,font=\scriptsize,y=.6cm,x=.7cm]
            \begin{scope}[scale=.5]
                \draw[->] (-5,0) -- (5,0) node[below] {\footnotesize $x$};
                \draw[->] (0,-4.5) -- (0,4) node[left] {\footnotesize $y$};
                \draw (0,0) node[below left] {\footnotesize $O$} circle (1pt);
                \draw[smooth](-1.1,0) parabola bend (-2,-2)(-4,4);
                \draw(-1.1,0) parabola bend (0,2)(1.5,-2.1);
                \draw[smooth](1.5,-2.1) parabola bend (2.6,-4)(4,4);
                \draw (1.5,0) node[above] {\footnotesize $2$};
                \draw (-1.1,0) node[above left] {\footnotesize $-2$};
            \end{scope}
    \end{tikzpicture}}
    \loigiai{
        Đặt $g(x)=f(x^3-3x)\Rightarrow g'(x)=3(x^2-1)f'(x^3-3x)$.\\
        Suy ra $g'(x)=0\Leftrightarrow \hoac{&x^2=1\\&f'(x^3-3x)=0}\Leftrightarrow \hoac{&x=\pm1\\&x^3-3x=2\\&x^3-3x=-2\\&x^3-3x=a\quad(a<-2)\\&x^3-3x=b\quad(b>2).}$\\
        Ta có
        \begin{itemize}
            \item $x^3-3x=2\Leftrightarrow \hoac{&x=2\\&x=-1.}$
            \item $x^3-3x=-2\Leftrightarrow \hoac{&x=-2\\&x=1.}$
            \item $x^3-3x=a\Leftrightarrow x=m \text{ (với $m<-2$)}$.
            \item $x^3-3x=b\Leftrightarrow x=n \text{ (với $n>2$)}$.
        \end{itemize}
        Từ đồ thị hàm số $f'(x)$ ta có $f'(x)>0\Leftrightarrow \hoac{&x<a\\&-2<x<2\\&x>b.}$\\
        Suy ra $h'(x)>0\Leftrightarrow \hoac{&x^3-3x<a\\&-2<x^3-3x<2\\&x^3-3x>b}\Leftrightarrow \hoac{&x<m\\&-2<x<-1\\&-1<x<2\\&x>n.}$\\
        Bảng biến thiên hàm số $y=g(x)$
        \begin{center}
            \begin{tikzpicture}
                \tkzTabInit[nocadre=false,lgt=2.5,espcl=1.7,deltacl=0.5]
                {$x$ /0.7,$x^2-1$ /0.7,$f'(x^3-3x)$ /0.7,$g'(x)$ /0.7,$g(x)$ /2}
                {$-\infty$,$m$,$-2$,$-1$,$0$,$1$,$2$,$n$,$+\infty$}
                \tkzTabLine{,+,|,+,|,+,$0$,-,t,-,$0$,+,|,+,|,+,}
                \tkzTabLine{,+,|,-,|,+,$0$,+,t,+,$0$,+,|,-,|,+,}
                \tkzTabLine{,+,|,-,|,+,$0$,-,t,-,$0$,+,|,-,|,+,}
                \tkzTabVar{-/,+/,-/,+/,R,-/,+/,-/,+/}
            \end{tikzpicture}
        \end{center}
        Từ bảng biến thiên suy ra hàm số $y=g(x)$ có $3$ điểm cực trị ứng với $x>0$ nên hàm số $y=f(|x|^3-3|x|)$ có $7$ điểm cực trị.

    }
\end{ex}
\begin{ex}%[2D1K2-2]
    \immini{
        Hình vẽ dưới đây là đồ thị của hàm số $y=f(x)$.
        Có bao nhiêu giá trị nguyên dương của tham số $m$ để hàm số $y=\left|f(x+1)+m\right|$ có $5$ cực trị?
        \choice
        {$0$}
        {\True $3$}
        {$2$}
        {$1$}
    }{
        \begin{tikzpicture}[>=stealth, font=\footnotesize, line join=round, line cap=round,y=.7cm]
            \begin{scope}[scale=.5]
                \def\xmin{-4} \def\xmax{3}
                \def\ymin{-5.5} \def\ymax{4}
                %\draw[color=gray!50,dashed] (\xmin,\ymin) grid (\xmax,\ymax);
                \draw[->] (\xmin,0)--(\xmax,0) node [below]{$x$};
                \draw[->] (0,\ymin)--(0,\ymax) node [left]{$y$};
                \node at (0,0) [below right]{$O$};
                \clip (\xmin+0.1,\ymin+0.1) rectangle (\xmax-0.5,\ymax-0.1);
                \draw[smooth,samples=300,domain=-3.5:0] plot(\x,{-1.24*(\x)^3-5.74*(\x)^2-5.78*(\x)});
                \draw[smooth,samples=300,domain=0:2.3] plot(\x,{1.78*(\x)^3-1.61*(\x)^2-4.57*(\x)-0.04});
                \draw[dashed](-2.5,0)|-(0,-2.07)  (-0.7,0)|-(0,1.7) (1.3,0)|-(0,-4.8);
                \draw[fill=black](0,-2.07)node[below left]{$-3$}circle(1pt)
                (0,-4.8)node[left]{$-6$}circle(1pt)
                (0,1.7)node[right]{$2$}circle(1pt);
            \end{scope}
        \end{tikzpicture}
    }
    \loigiai{
        Nhận xét
        \begin{itemize}
            \item  Hàm số $y=\left|f(x)-\alpha\right|$ có số điểm cực trị bằng số cực trị của hàm $y=f(x)$ và số giao điểm của đồ thị hàm $y=f(x)$ với đường thẳng $y=\alpha$ (không tính giao điểm là các điểm cực trị).
            \item  Số điểm cực trị của hàm $y=f(x)$ bằng số điểm cực trị của hàm $y=f(x+a)$.
        \end{itemize}
        Từ nhận xét trên ta có: Hàm số $y=f(x+1)$ có $3$ cực trị.\\
        Vậy ta cần đường thẳng $y=-m$ cắt đồ thị hàm số $y=f(x+1)$ tại 2 điểm khác cực trị.\\
        Từ đồ thị ta suy ra: $\hoac{&-6 <-m\leq-3\\&-m\geq 2}\Leftrightarrow\hoac{&3\leq m<6\\&m\leq-2.}$ \\
        Do $m\in\mathbb{N}^*$ nên $m\in\{3,4,5\}$.
    }
\end{ex}
\Closesolutionfile{ans}
%% \indapan{10}{ans/2D1-2-DANG-3}
%%Bài 2. Max-min
% \setcounter{section}{1}
\section{GIÁ TRỊ LỚN NHẤT - NHỎ NHẤT CỦA HÀM SỐ}

\subsection{LÝ THUYẾT CẦN NHỚ}
Cho hàm số $y=f(x)$ xác định trên tập $\mathscr{D}$. Ta có
\immini{\begin{itemize}
		\item[\ding{172}] $M$ là giá trị lớn nhất của hàm số nếu $\heva{&f(x) \le M,\forall x \in \mathscr{D}\\& \exists x_0 \in \mathscr{D}: f(x_0)=M.}$\\
		Kí hiệu \fbox{$\displaystyle\max_{x \in \mathscr{D}}f(x)=M$}
		\vskip 0.5cm
		\item[\ding{173}] $n$ là giá trị nhỏ nhất của hàm số nếu $\heva{&f(x) \ge n,\forall x \in \mathscr{D}\\& \exists x_0 \in \mathscr{D}: f(x_0)=n.}$\\
		Kí hiệu \fbox{$\displaystyle\min_{x \in \mathscr{D}}f(x)=n$}
\end{itemize}
}{
\begin{tikzpicture}[smooth,samples=300,scale=0.7,>=stealth]
	\draw[->] (-1.5,0)--(4.8,0) node[below]{$x$};
	\draw[->] (0,-2)--(0,4) node[right]{$y$};
	\draw (0,0) node[above left]{$O$};
	\draw[line width = 1.2pt,domain=-1:4,blue] plot(\x,{0.5*((\x)^2-4*(\x)+1)});
	\draw[fill=black] (-1,0) circle(1.5pt) (-1,3) circle(2pt) (0,3) circle(1.5pt) (0,-1.5) circle(1.5pt) (2,0) circle(1.5pt) (2,-1.5) circle(2pt) (4,0) circle(1.5pt) (4,0.5) circle(1.5pt);
	\draw[dashed] (-1,0)node[below]{\small$a$}--(-1,3)--(0,3)node[right]{\small$f(a)$} (2,0)node[above]{\small$x_0$}--(2,-1.5)--(0,-1.5)node[left]{\small$f(x_0)$}
	(4,0)node[below]{\small$b$}--(4,0.5);
	\node[above] at (-1,3) {\small $y_{\max}$};
	\node[below] at (2,-1.5) {\small $y_{\min}$};
\end{tikzpicture}}
\begin{note}
	\begin{listEX}[1]
		\item [\ding{172}] Khi yêu cầu tìm max min của hàm số mà không nói rõ xét trên tập nào, thì ta hiểu là tìm max min trên miền xác định của hàm số đó.
		\item [\ding{173}] Để tìm max min của hàm số $y=f(x)$ trên miền $\mathscr{D}$, ta thường lập bảng biến thiên của hàm số $y=f(x)$ trên $\mathscr{D}$. Từ bảng biến thiên, ta kết luận:
		\begin{itemize}
			\item [$\bullet$] Điểm ở vị trí cao nhất $\longrightarrow$ Kết luận max;
			\item [$\bullet$] Điểm ở vị trí thấp nhất $\longrightarrow$ Kết luận min.
		\end{itemize}
		\item [\ding{174}] Để tìm max min của hàm số $y=f(x)$ trên đoạn $[a;b]$ (\textit{$f(x)$ liên tục trên đoạn $[a ; b]$ và có đạo hàm trên $(a ; b)$ (có thể trừ một số hữu hạn các điểm) và $f^{\prime}(x)=0$ chỉ tại một số hữu hạn các điểm trong $(a ; b)$}), thì ta có thể giải như sau:
		\begin{itemize}
			\item [$\bullet$] Giải $f'(x)=0$ tìm các nghiệm $x_0 \in (a;b)$; 
			\item [$\bullet$] Tìm các điểm $x_i\in (a;b)$ mà tại đó đạo hàm không xác định (nếu có).
			\item [$\bullet$] Tính toán $f(a)$, $f(x_0)$, $f(x_i)$, $f(b)$ \quad ($\star$)
			\item [$\bullet$]  Gọi $M$, $n$ lần lượt là số lớn nhất và số nhỏ nhất của các kết quả tính toán ở bước ($\star$) thì
			$$M=\displaystyle\max_{[a;b]}f(x); \quad n=\displaystyle\min_{[a;b]}f(x)$$
		\end{itemize}
	\item [\ding{175}] Ta có thể dùng các bất đẳng thức có sẵn để đánh giá biểu thức cần tìm max, min. 
	% \begin{itemize}
	% 	\item [$\bullet$] Bất đẳng thức Cauchy cho hai số không âm $a$, $b$:
	% 	$$a+b \ge 2\sqrt{ab}$$
	% 	Dấu "=" xảy ra khi $a=b$.
	% 	\item [$\bullet$]  Bất đẳng thức Cauchy cho ba số không âm $a$, $b$, $c$:
	% 	$$a+b +c\ge 3\sqrt[3]{abc}$$
	% 	Dấu "=" xảy ra khi $a=b=c$.
	% 	\item [$\bullet$]  Bất đẳng thức Cauchy cho $n$ số không âm $a_1$, $a_2$,..., $a_n$:
	% 	$$a_1+a_2 +...+a_n \ge n\sqrt[n]{a_1a_2...a_n}$$
	% 	Dấu "=" xảy ra khi $a_1=a_2=...=a_n$.
	% \end{itemize}
	\end{listEX}
\end{note}
% \newpage
\subsection{PHÂN LOẠI VÀ PHƯƠNG PHÁP GIẢI TOÁN}
\begin{dang} {Bài toán tìm max, min của hàm số $y=f(x)$ trên miền $\mathscr{D}$}
	\begin{enumerate}[\iconMT]
		\item \indam{Phương pháp giải:} 
		\begin{listEX}[1]
			\item [\ding{172}] Tính $y'$. Giải phương trình $y'=0$ tìm các nghiệm $x_i \in \mathscr{D}$ và tìm các điểm $x_j \in \mathscr{D}$ mà tại đó $y'$ không xác định.
			\item [\ding{173}] Lập bảng biến thiên của hàm số trên $\mathscr{D}$.
			\item [\ding{174}] Từ bảng biến thiên, kết luận:
			\begin{itemize}
				\item [$\bullet$] Điểm ở vị trí cao nhất $\longrightarrow$ Kết luận max;
				\item [$\bullet$] Điểm ở vị trí thấp nhất $\longrightarrow$ Kết luận min.
			\end{itemize}
		\end{listEX}
		\item \indam{Lưu ý:} Nếu $\mathscr{D}$ là đoạn $\left[a;b\right]$ và hàm số $y=f(x)$ liên tục trên đoạn $\left[a;b\right]$ thì ta có thể làm như sau:
		\begin{listEX}[1]
			\item [\ding{172}] Giải $f'(x)=0$ tìm các nghiệm $x_0 \in (a;b)$;
			\item [\ding{173}] Tìm các điểm $x_i\in (a;b)$ mà tại đó đạo hàm không xác định (nếu có).
			\item [\ding{174}] Tính toán $f(a)$, $f(x_0)$, $f(x_i)$, $f(b)$ \quad ($\star$)
			\item [\ding{175}] Gọi $M$, $n$ lần lượt là số lớn nhất và số nhỏ nhất của các kết quả tính toán ở bước ($\star$) thì
			$$M=\displaystyle\max_{[a;b]}f(x); \quad n=\displaystyle\min_{[a;b]}f(x)$$
		\end{listEX}
		\begin{note}
			\begin{itemize}
				\item[\iconCH] Nếu hàm số $y=f(x)$ đồng biến trên đoạn $\left[a;b\right]$ thì $\min\limits_{[a;b]} f(x)=f(a)$ và $\max\limits_{[a;b]} f(x)=f(b)$.
				\item[\iconCH]  Nếu hàm số $y=f(x)$ nghịch biến trên đoạn $\left[a;b\right]$ thì $\min\limits_{[a;b]} f(x)=f(b)$ và $\max\limits_{[a;b]} f(x)=f(a)$.
			\end{itemize}
		\end{note}
	\end{enumerate}
\end{dang}

\boxmini{BÀI TẬP TỰ LUẬN}
\begin{vd}
	Tìm giá trị lớn nhất và nhỏ nhất (nếu có) của hàm số sau trên đoạn đã chỉ ra.
	\begin{tasks}(2)
		\task $f(x)=-x^3+3x^2+10$ trên đoạn $[-3;1]$.
		\task $f(x)=\dfrac{x^3}{3}-2x^2+3x+1$ trên đoạn $[-3;2]$.
		\task $f(x) = - 2x^4 + 4x^2 + 3$ trên đoạn $\left[0;2\right]$
		\task $f(x)=\dfrac{2x+3}{x+1}$ trên đoạn $[0;4]$.
		\task $f(x)=x+\dfrac{4}{x}$ trên khoảng $(0;+\infty)$;
		\task $f(x)=3x+\dfrac{4}{x^2}$ trên $(0;+\infty)$.
		\task $f(x)=\dfrac{2x^2+4x+5}{x^2+1}$ trên $\mathbb{R}$.
		\task $f(x)=\sqrt{-x^2+2x}$ trên miền xác định.
	\end{tasks}
\loigiai{
\begin{enumerate}[a)]
	\item Hàm số liên tục trên $[-3;1]$. Ta có $f'(x)=-3x^2+6x$; $f'(x)=0 \,\Leftrightarrow \hoac{&x=0 \in [-3;1]\\&x=2 \notin [-3;1]}$.\\
	Ta có $f(-3)=64$, $f(0)=10$, $f(1)=12$. Suy ra, $\max\limits_{[-3;1]} f(x)=f(-3)=64$; $\min\limits_{[-3;1]} f(x)=f(0)=10$.
	
	\item Hàm số liên tục trên $[-3;2]$
	Ta có $f^{\prime}(x)=x^2-4x+3$; $f^{\prime}(x)=0\Leftrightarrow \heva{&x=1\\x=3\notin[-3;2]}$.\\
	$f(1)=\dfrac{7}{3}$, $f(3)=-35$, $f(2)=\dfrac{5}{3}$.\\
	Vậy 
	$\max\limits_{[-3;2]}f(x)=\dfrac{7}{3}$ và
	$\min\limits_{[-3;2]}f(x)=-35$.	
	\item Ta có $f'(x)=- 8x^3 + 8x =- 8x(x^2 - 1) =- 8x(x - 1)(x + 1)$.\\
	Xét $f(0) = 3$, $f(1) = 5$ và $f(2) =- 13$.\\
	Vậy 
	$\max\limits_{[0;2]}f(x)=5$ và
	$\min\limits_{[0;2]}f(x)=-13$.	
	\item Hàm số đã cho liên tục trên đoạn $[0;4]$.\\
	Ta có $y'=-\dfrac{1}{(x+1)^2} < 0$, $\forall x \in [0;4]$. Suy ra hàm số đã cho nghịch biến trên đoạn $[0;4]$.\\
	Vậy $\max\limits_{[0;4]} y = y(0) = 3$ và $\min\limits_{[0;4]} y = y(4) = \dfrac{11}{5}$.
	
	\item Xét hàm số $f(x)=x+\dfrac{4}{x}$ trên khoảng $(0;+\infty)$.\\
	Đạo hàm $f'(x)=1-\dfrac{4}{x^2}=\dfrac{x^2-4}{x^2}$.\\
	Cho $f'(x)=0 \Leftrightarrow x^2-4=0 \Leftrightarrow \hoac{&x=2\in(0;+\infty)\\&x=-2\notin(0;+\infty).}$\\
	Bảng biến thiên
	\begin{center}
		\begin{tikzpicture}[font=\footnotesize,thick,>=stealth]
			\tikzset{double style/.append style={double distance=1.75pt}}
			\tkzTabInit[nocadre=false,lgt=1.2,espcl=2.5,deltacl=0.6,lw=.5pt,color,colorL=green!50,colorV=green!50]
			{$x$ /0.6,$f'(x)$ /0.6,$f(x)$ /2}
			{$-\infty$,$-2$,$0$,$2$,$+\infty$}
			\tkzTabLine{,+,$0$,-,d,-,$0$,+,}
			\tkzTabVar{-/$-\infty$,+/$-4$,-D+/$-\infty$/$+\infty$,-/$4$,+/$+\infty$}
			%\draw[pattern={Lines[angle=60,distance=1.25mm]},pattern color=blue,thin] (N11)--(N31)--(N33)--(N13);
		\end{tikzpicture}
	\end{center}
	Căn cứ vào bảng biến thiên, ta có $\min\limits_{(0;+\infty)}f(x)=4$.
	
	\item Áp dụng bất đẳng thức Cauchy cho $3$ số dương, ta có
	$$y=\dfrac{3x}{2}+\dfrac{3x}{2}+\dfrac{4}{x^2} \geq 3\sqrt[3]{\dfrac{3x}{2}\cdot \dfrac{3x}{2}\cdot \dfrac{4}{x^2}}=3\sqrt[3]{9}.$$
	Đẳng thức xảy ra khi $\dfrac{3x}{2}=\dfrac{4}{x^2} \Leftrightarrow x=\dfrac{2}{\sqrt[3]{2}}=2\sqrt[3]{2}$.
	
	\item Tập xác định $\mathscr{D}= \mathbb{R}$.\\
	Ta có $y'= \dfrac{-4x^2-6x+4}{(x^2+1)^2}$, \; $y'=0 \Leftrightarrow -4x^2-6x+4=0\Leftrightarrow \hoac{&x=-2\\&x=\dfrac{1}{2}.}$
	\begin{center}
		\begin{tikzpicture}\tkzTabInit[nocadre=false,lgt=1.2,espcl=2.5,deltacl=0.6]
			{$x$ /1, $y'$ /0.6, $y$ /2.5}
			{$-\infty$,$-2$,$\dfrac{1}{2}$,$+\infty$}
			\tkzTabLine{,-,$0$,+,$0$,-,}
			\tkzTabVar{+/$2$,-/$1$,+/$6$,-/$2$}
		\end{tikzpicture}
	\end{center}
	Suy ra $M=6$ và $m=1$.
	
	\item Hàm số $f(x)=\sqrt{-x^2+2x}$ liên tục trên $[0;2]$.\\
	$f'(x)=\dfrac{1-x}{\sqrt{-x^2+2x}}$, $f'(x)=0 \Leftrightarrow x=1$.\\
	Ta có $f(0)=0$, $f(2)=0$, $f(1)=1$.\\
	Vậy $\displaystyle\max_{x\in [0;2]}f(x)=1$ và $\displaystyle\min_{x\in [0;2]}f(x)=0$.
\end{enumerate}}
\end{vd}
\dongcham{54}
\begin{vd}
	Tìm giá trị lớn nhất và nhỏ nhất của hàm số sau trên miền đã chỉ ra.
	\begin{tasks}(2)
		\task $y=x-\sin 2x$ trên đoạn $\left[-\dfrac{\pi}{2};\pi\right]$
		\task $y = \mathrm{e}^{x^3 - 3x + 3}$ trên đoạn $[0; 2]$
		\task $y=\mathrm{e}^{x}(x^{2}-3)$ trên đoạn $[-2;2]$
		\task $y=\dfrac{\ln^2x}{x}$ trên đoạn $\left[1;\mathrm{e}^5\right]$
	\end{tasks}
	\loigiai{
		\begin{enumerate}[a)]
			\item Ta có
			\begin{itemize}
				\item $y'=1-2\cos 2x$.
				\item $\heva{&x\in \left(-\dfrac{\pi}{2};\pi\right)\\& y'=0}\Leftrightarrow \heva{&x\in \left(-\dfrac{\pi}{2};\pi\right)\\&\cos 2x=\dfrac{1}{2}}\Leftrightarrow \heva{&x\in \left(-\dfrac{\pi}{2};\pi\right)\\&x=\pm \dfrac{\pi}{6}+k\pi} \Leftrightarrow \hoac {&x=\pm\dfrac{\pi}{6}\\& x=\dfrac{5\pi}{6}.}$
				\item $f\left(-\dfrac{\pi}{2}\right)=-\dfrac{\pi}{2}$,  $f(\pi)=\pi$,
				$f\left(-\dfrac{\pi}{6}\right)=-\dfrac{\pi}{6}+\dfrac{\sqrt{3}}{2}$,  $f\left(\dfrac{\pi}{6}\right)=\dfrac{\pi}{6}-\dfrac{\sqrt{3}}{2}$,  $f\left(\dfrac{5\pi}{2}\right)=\dfrac{5\pi}{6}+\dfrac{\sqrt{3}}{2}$.
			\end{itemize}
			Vậy giá trị lớn nhất và giá trị nhỏ nhất của hàm số $y=x-\sin 2x$ trên đoạn $\left[-\dfrac{\pi}{2};\pi\right]$ lần lượt là $\dfrac{5\pi+3\sqrt{3}}{6}$ và $-\dfrac{\pi}{2}$.
			\item Ta có $y' = (3x^2 - 3)\cdot \mathrm{e}^{x^3 - 3x + 3}$.\\
			$y' = 0 \Leftrightarrow 3x^2 - 3 = 0 \Leftrightarrow x = 1$ do $x \in [0;2]$.\\
			Khi đó $y(0) = \mathrm{e}^3$; $y(2) = \mathrm{e}^5$; $y(1) = \mathrm{e}$.
			Vậy $ \max \limits_{[0; 2]} y = \mathrm{e}^5 $ khi $x = 2$.
			\item Ta có $y'=\mathrm{e}^x(x^2+2x-3)=0\Leftrightarrow \hoac{&x=-3\\ &x=1}$.
			Xét các giá trị: $f(-2)=\mathrm{e}^{-2}$; $f(1)=-2\mathrm{e}$; $f(2)=\mathrm{e}^2$, từ đó suy ra $y_{\min}=-2\mathrm{e}$.
			\item $y'=\dfrac{2\ln x-\ln^2x}{x^2}$, $y'=0\Leftrightarrow \hoac{
				& \ln x=0 \\
				& \ln x=2 \\} \Leftrightarrow \hoac{
				& x=1 \\
				& x=\mathrm{e}^2. \\}$\\
			Tính $y(1)=0$, $y(\mathrm{e}^2)=\dfrac{4}{\mathrm{e}^2}\approx 0{,}54$, $y(\mathrm{e}^5)=\dfrac{9}{\mathrm{e}^5}\approx 0{,}16$.\\
			Vậy $\max\limits_{x \in \left[1;\mathrm{e}^5\right]}y=\dfrac{4}{\mathrm{e}^2}$
	\end{enumerate}}
\end{vd}
\dongcham{40}
\begin{vd}
	Tìm giá trị lớn nhất và nhỏ nhất (nếu có) của hàm số sau trên miền đã chỉ ra.
	\begin{tasks}(2)
		\task $f(x)=\dfrac{5\sin x+1}{\sin x+2}$ trên đoạn $\left[0;\dfrac{\pi}{6}\right]$.
		\task $ y=\cos^3x +2\sin^2x+\cos x$ trên miền xác định.
	\end{tasks}
	\loigiai{
		\begin{enumerate}[a)]
			\item Đặt $t=\sin x,\; x\in \left[0;\dfrac{\pi}{6}\right]\Rightarrow t \in \left[0;\dfrac{1}{2}\right]$.\\
			Ta được hàm số $y=g(t)=\dfrac{5t+1}{t+2}$.\\
			$g'(t)=\dfrac{9}{(t+2)^2}>0,\forall t \in \left[0;\dfrac{1}{2}\right]$.\\
			Vì $g(0)=\dfrac{1}{2}$, $g\left(\dfrac{1}{2}\right)=\dfrac{7}{5}$ nên $\min\limits_{\left[0;\tfrac{1}{2}\right]}g(t)=g(0)=\dfrac{1}{2}.$\\
			Vậy $\min\limits_{\left[0;\tfrac{\pi}{6}\right]}f(x)=\min\limits_{\left[0;\tfrac{1}{2}\right]}g(t)=\dfrac{1}{2}$
			\item Ta có $ y=\cos^3x +2\sin^2x+\cos x =\cos^3x +2(1-\cos^2x)+\cos x =\cos^3x-2\cos^2x+\cos x+2$.\\
			Đặt $ t=\cos x,\, t\in [-1;1] $. Ta được $ f(t)=t^3-2t^2 +t+2$.\\
			Ta có $ f'(t)=3t^2-4t+1;\,y'=0\Leftrightarrow \hoac{&t=1\in[-1;1]\\&t=\dfrac{1}{3}\in[-1;1].} $\\
			Mà $f\left(-1\right)=-2$, $ f\left(\dfrac{1}{3}\right)=\dfrac{58}{27} $, $f(1)=2$ nên $\max\limits_{x\in \mathbb{R}}y=\max\limits_{\left[-1;1\right]}f(t)=\dfrac{58}{27}$
			\item
			\item
	\end{enumerate}}
\end{vd}
\dongcham{43}
\boxmini{BÀI TẬP TRẮC NGHIỆM}
\ind{PHẦN I.} \inden{Câu trắc nghiệm nhiều phương án lựa chọn. Mỗi câu hỏi học sinh chỉ chọn một phương án.}\\
\setcounter{ex}{0}
\Opensolutionfile{ans}[ans/2D1-B2-d1-1]

\begin{ex}
	\immini
	{Hàm số $y=f(x)$ liên tục trên đoạn $[-1;3]$ và có bảng biến thiên như sau.\\
		Gọi $M$ là giá trị lớn nhất của hàm số $y=f(x)$ trên đoạn $[-1;3]$. Khẳng định nào sau đây là khẳng định đúng?
		\choice
		{\True $M=f(0)$}
		{$M=f(-1)$}
		{$M=f(3)$}
		{$M=f(2)$}
	}
	{\begin{tikzpicture}
			\tkzTabInit[nocadre=True,lgt=1.2,espcl=2]
			{$x$ /0.7,$y'$ /0.7,$y$ /2.1}
			{$-1$,$0$,$2$,$3$}
			\tkzTabLine{,+,$0$,-,$0$,+,}
			\tkzTabVar{-/$0$, +/$5$,-/$1$,+/$4$}
	\end{tikzpicture}}
	\loigiai
	{Dựa vào bảng biến thiên ta có $M=f(0)=5$.}
\end{ex} \dongcham{1}

\begin{ex}
	\immini{Cho hàm số $f(x)$ liên tục trên đoạn $[-1;5]$ và có đồ thị như hình vẽ bên. Gọi $M$ và $m$ lần lượt là giá trị lớn nhất và nhỏ nhất của hàm số đã cho trên $[-1;5]$. Giá trị của $M+m$ bằng
		\choice
		{$5$}
		{$6$}
		{$3$}
		{\True $1$}
	}{
		\begin{tikzpicture}[scale=0.65, font=\footnotesize, line join=round, line cap=round, >=stealth]
			%%Nhập giới hạn đồ thị và hàm số cần vẽ
			\def \xmin{-1.5}
			\def \xmax{6.3}
			\def \ymin{-2.8}
			\def \ymax{4}
			%%Tự động
			\draw[->] (\xmin,0)--(\xmax,0) node[below left] {$x$};
			\draw[->] (0,\ymin)--(0,\ymax) node[below left] {$y$};
			\draw[fill=black] (0,0) circle(1pt) node [below right] {$O$};
			%%Vẽ các điểm trên 2 hệ trục
			\foreach \x in {3,4,5}
			\draw[fill=black] (\x,0) circle(1pt) node [below] {$\x$};
			\foreach \x in {-1,2}
			\draw[fill=black] (\x,0) circle(1pt) node [above] {$\x$};
			\foreach \y in {-2,1,3}
			\draw[fill=black] (0,\y) circle(1pt) node [above right] {$\y$};
			\draw[dashed](-1,0)--(-1,-2)--(0,-2)--(2,-2)--(2,0) (5,0)--(5,1)--(0,1) (3,0)--(3,1) (4,0)--(4,3)--(0,3);
			%%Tự động
			\draw
			(-1.1,-2.7) to[out=80, in=-100] (-1,-2)
			..controls +(80:1.2) and +(180:.5)..(0,1)
			..controls +(0:.6) and +(180:0.7)..(2,-2)
			..controls +(0:0.4) and +(-100:1.2)..(2.8,0)
			to[in=80, out=-100] (3,1)
			..controls +(75:1.5) and +(180:0.3)..(4,3)
			..controls +(0:0.5) and +(-75:1)..(5,1)
			to[in=105, out=-75] (6,-2.7);
			\fill[black]
			(-1,-2) circle(1pt)
			(2,-2) circle(1pt)
			(3,1) circle(1pt)
			(4,3) circle(1pt)
			(5,1) circle(1pt)
			;
		\end{tikzpicture}
	}
	\loigiai{
		Dựa vào đồ thị, suy ra $m=\min\limits_{[-1;5]} f(x)=f(-1)=-2$, $M=\max\limits_{[-1;5]} f(x)=f(4)=3$.\\
		Do đó $M+m=3-2=1$.
	}
\end{ex} \dongcham{1}

\begin{ex}
	\immini{Cho hàm số $y=f(x)$ có đồ thị là đường cong ở hình bên. Tìm giá trị nhỏ nhất $m$ của hàm số $y=f(x)$ trên đoạn $[-1;1] $.
		\haicot
		{$m=2 $}
		{\True $m=-2 $}
		{$m=1 $}
		{$m=-1 $}}{\vspace{-0.5cm}
		\begin{tikzpicture}[smooth,samples=300,scale=0.68,>=stealth]
			\draw[->] (-2.3,0)--(2.3,0) node[below]{$x$};
			\draw[->] (0,-2.5)--(0,2.5) node[right]{$y$};
			\draw (0,0) node[above right]{$O$};
			\draw[line width = 1pt,domain=-2:2] plot(\x,{(\x)^(3)-3*(\x)});
			\draw[fill=black] (-1,2) circle(1.5pt) (1,-2) circle(1.5pt);
			\draw[dashed] (-1,0)node[below]{\small$-1$}--(-1,2)--(0,2)node[right]{\small$2$} (1,0)node[above]{\small$1$}--(1,-2)--(0,-2)node[left]{\small$-2$};
	\end{tikzpicture}}
	\loigiai{
		Dựa vào đồ thị ta có giá trị nhỏ nhất của hàm số trên đoạn $[-1;1] $ bằng $-2$.	
		
	}
	
\end{ex} \dongcham{1}

\begin{ex}
	Cho hàm số $y=f(x)$ có bảng biến thiên trên đoạn $[-2;3]$ như hình bên dưới.
	\begin{center}
		\begin{tikzpicture}[>=stealth,scale=1]
			\tkzTabInit[nocadre=false,lgt=1.2,espcl=2,deltacl=0.6]
			{$x$/0.6,$f’(x)$/0.6,$f(x)$/2}
			{$-\infty$,$-2$,$-1$,$1$,$3$,$+\infty$}
			\tkzTabLine{,h,,+,z,-,d,+,,h}
			\tkzTabVar{+H/,-/$0$,+/$1$,-/$-2$,+H/$5$}
		\end{tikzpicture}
	\end{center}
	Gọi $M$ và $m$ lần lượt là giá trị lớn nhất và giá trị nhỏ nhất của hàm số đã cho trên đoạn $[-1;3]$. Giá trị của biểu thức $M-m$ là
	\choice
	{$5$}
	{\True $7$}
	{$-1$}
	{$3$}
	\loigiai{
		Dựa vào bảng biến thiên ta thấy giá trị lớn nhất của hàm số là $M=5$ và giá trị nhỏ nhất của hàm số là $m=-2$ nên $M-m=7$.}
\end{ex}
% \newpage
\begin{ex}
	Giá trị lớn nhất và nhỏ nhất của hàm số $y=x^3-12x+1$ trên đoạn $[-2;3]$ lần lượt là
	\choice
	{\True $17$, $-15$}
	{$10$, $-26$}
	{$-15$, $17$}
	{$6$, $-26$}
	\loigiai{
		Ta có $y'=3x^2-12$, do đó $y'=0\Leftrightarrow 3x^2-12=0\Leftrightarrow x=\pm 2\in [-2;3]$.\\
		Mặt khác $f(-2)=17$, $f(2)=-15$, $f(3)=-8$.\\
		Vậy giá trị lớn nhất và nhỏ nhất cần tìm lần lượt là $17$, $-15$.
	}
\end{ex} \dongcham{12}

\begin{ex}
	Gọi $ M, m $ lần lượt là giá trị lớn nhất và giá trị nhỏ nhất của hàm số $ y = x^3 + 3x^2 - 9x + 1 $ trên $ [-4;4] $. Tính tổng $ M + m. $
	\choice 
	{$ 12 $}
	{$ 98 $}
	{$ 17 $}
	{\True $ 73 $}
	\loigiai
	{
		Ta có $ y' = 3x^2 + 6x - 9 = 0 \Leftrightarrow \hoac{&x = 1\\ &x = -3.} $\\
		Khi đó: $ y(-4) = 21 $,\, $ y(-3) = 28, $
		\, $ y(1) = -4, $
		\, $ y(4) = 77. $\\
		Do đó $ M + m = 77 + (-4) = 73. $
	}
\end{ex} \dongcham{12}

\begin{ex}
	Giá trị lớn nhất của hàm số $f(x)=-x^4+12x^2+1$ trên đoạn $\left[ -1;2\right] $ bằng
	\choice
	{\True $33$}
	{$37$}
	{$12$}
	{$1$}
	\loigiai{
		Hàm số $f(x)=-x^4+12x^2+1$ liên tục trên đoạn $\left[ -1;2\right] $.\\
		Ta có $f'(x)=-4x^3+24x=-4x(x^2-6)$.\\
		$f'(x)=0 \Leftrightarrow \hoac{& x=-\sqrt{6} \not \in \left[ -1;2\right] \\ &x=0 \in \left[ -1;2\right] \\&x=\sqrt{6} \not \in \left[ -1;2\right]. }$\\
		Ta có $f(-1)=12; f(0)=1; f(2)=33$.\\
		Vậy $\max\limits_{\left[ -1;2\right] } f(x)=33$.
	}
\end{ex} \dongcham{12}
\begin{ex}
	Giá trị lớn nhất của hàm số $y=x^4-3x^2+2$ trên đoạn $\left[ 0;3\right] $ bằng
	\choice
	{ $ 57 $}
	{\True $ 56 $}
	{$ 54$}
	{$ 55 $}
	\loigiai{
		Hàm số $y$ liên tục trên đoạn $\left[ 0;3\right] $ và có đạo hàm $y'=4x^3-6x$.\\
		Ta có $y'=0 \Leftrightarrow 4x^3-6x=0 \Leftrightarrow \hoac{& x=0 \in \left[ 0;3\right]  \\&x=\sqrt{\dfrac{3}{2}} \in \left[ 0;3\right]\\ & x=- \sqrt{\dfrac{3}{2}}\notin \left[ 0;3\right].}$\\
		Ta có $y(0)=2$, $y(3)=56$, $y\left(\sqrt{\dfrac{3}{2}}\right) =-\dfrac{1}{4} $.\\
		Do đó giá trị lớn nhất của hàm số $y=x^4-3x^2+2$ trên đoạn $\left[ 0;3\right] $ bằng $56$.
	}
\end{ex} \dongcham{7}

\begin{ex}%[2D1Y3-1]
	Giá trị nhỏ nhất của hàm số $y=\dfrac{x-1}{x+1}$ trên đoạn $[0;3]$ là
	\choice
	{$\min\limits_{[0;3]}y=\dfrac{1}{2}$}
	{$\min\limits_{[0;3]}y=-3$}
	{$\min\limits_{[0;3]}y=1$}
	{\True $\min\limits_{[0;3]}y=-1$}	
	\loigiai{
		Trên đoạn $[0;3]$ hàm số luôn xác định.\\
		Ta có $y'=\dfrac{2}{(x+1)^2}>0$, $\forall x \in [0;3]$ nên hàm số đã cho đồng biến trên đoạn $[0;3]$.\\
		Do đó $\min\limits_{[0;3]}y=y(0)=-1$.
	}	
\end{ex} \dongcham{7}

\begin{ex}%
	Giá trị nhỏ nhất của hàm số $y=\dfrac{2x+3}{x+1}$ trên đoạn $[0;4]$ là
	\choice
	{$2$}
	{$\dfrac{7}{5}$}
	{$3$}
	{\True $\dfrac{11}{5}$}
	\loigiai
	{
		Ta có $y'=\dfrac{-1}{(x+1)^2}<0$ nên $\min\limits_{[0;4]} y=y(4)=\dfrac{11}{5}$.
	}
\end{ex} \dongcham{7}

\begin{ex}%[2D1B3]
	Giá trị lớn nhất của hàm số $y=\dfrac{x^2-3x+3}{x-1}$ trên đoạn $\left[-2;\dfrac{1}{2}\right]$ bằng
	\choice
	{$4$}
	{\True $-3$}
	{$-\dfrac{7}{2}$}
	{$-\dfrac{13}{3}$}
	\loigiai{
		Ta có $y'=\dfrac{x^2-2x}{(x-1)^2}$. Xét $y'=0\Leftrightarrow  x^2-2x=0\Leftrightarrow \hoac{&x=0\in \left[-2;\dfrac{1}{2}\right]\\&x=2\notin\left[-2;\dfrac{1}{2}\right]}$.\\
		Ta có $y(0)=-3$, $y(-2)=\dfrac{-13}{3}$, $y\left(\dfrac{1}{2}\right)=\dfrac{-7}{2}$.\\
		Suy ra $\underset{x\in \left[-2;\dfrac{1}{2}\right]}{\max y}=-3$
	}
\end{ex} \dongcham{10}

\begin{ex}
	Giá trị lớn nhất của hàm số $y=\sqrt{4-x^2}$ là
	\choice
	{$M=-2$}
	{\True $M=2$ }
	{$M=4$}
	{$M=0$}
	\loigiai
	{
		Tập xác định: $\mathscr{D}=\left[-2;2\right]$.\\
		Đạo hàm $y'=\dfrac{-x}{\sqrt{4-x^{2}}}$; $y'=0 \Leftrightarrow x=0 \in \left[-2;2\right]$.\\
		Ta có $y(2)=y(-2)=0$; $y(0)=2$.\\
		Vậy giá trị lớn nhất của hàm số đã cho bằng $2$.
	}
\end{ex} \dongcham{8}

\begin{ex}%[2D1B3-1]
	Tìm giá trị lớn nhất $M$ của hàm số $y=\sqrt{7+6x-x^2}$.
	\choice
	{\True $M=4$}
	{$M=\sqrt{7}$}
	{$M=7$}
	{$M=3$}
	\loigiai{
		Tập xác định $\mathscr{D}=[-1;7]$.\\
		$y'=\dfrac{-x+3}{\sqrt{7+6x-x^2}}$.\\
		Cho $y'=0\Leftrightarrow x= 3\in \mathscr{D}$.\\
		Có $y(3)=4, y(-1)=0, y(7)=0$. Vậy $M=4$.	
	}
\end{ex} \dongcham{8}

\begin{ex}%[Nguyễn Quang Hiệp - Phát triển đề minh họa 2021]%[2D2B4-4]%
	Tính giá trị lớn nhất của hàm số $y=x-\ln x$ trên $\left[\dfrac{1}{2};\mathrm{e}\right]$.\\
	\choice
	{$\max\limits_{x \in \left[\frac{1}{2};\mathrm{e}\right]}y=1$}
	{\True $\max\limits_{x \in \left[\frac{1}{2};\mathrm{e}\right]}y=\mathrm{e}-1$}
	{$\max\limits_{x \in \left[\frac{1}{2};\mathrm{e}\right]}y=\mathrm{e}$}
	{$\max\limits_{x \in \left[\frac{1}{2};\mathrm{e}\right]}y=\dfrac{1}{2}+\ln 2$}
	\loigiai{
		Hàm số $y=x-\ln x$liên tục trên đoạn $\left[\dfrac{1}{2};\mathrm{e}\right]$.\\
		Ta có $y'=1-\dfrac{1}{x}\Rightarrow y'=0\Leftrightarrow x=1\in \left[\dfrac{1}{2};\mathrm{e}\right]$.\\
		Do $y\left(\dfrac{1}{2}\right)=\dfrac{1}{2}+\ln 2$; $y(\mathrm{e})=\mathrm{e}-1$; $y(1)=1$ nên $\max\limits_{x \in \left[\frac{1}{2};\mathrm{e}\right]}y=\mathrm{e}-1$.}
\end{ex} \dongcham{8}

\begin{ex}
	Gọi $M, N$ lần lượt là giá trị lớn nhất và nhỏ nhất của hàm số $y = x^2 - 4\ln (1 - x)$ trên đoạn $[-2;0]$. Tính $M - N$.
	\choice
	{$M - N = 4\ln 2$}
	{$M - N = -1$}
	{\True $M - N = 4\ln 2 -1$}
	{$M - N = 4\ln 3 -4$}
	\loigiai{
		Tập xác định: $\mathscr{D} = (-\infty;1)$.\\
		Ta có $y' = 2x + \dfrac{4}{1 - x} = \dfrac{-2x^2 + 2x + 4}{1 - x}$.\\
		Khi đó $y' = 0 \Leftrightarrow -2x^2 + 2x + 4 = 0 \Leftrightarrow \hoac{&x = -1 \quad \mbox{(nhận)} \\&x = 2 \quad \mbox{(loại)}. }$\\
		Khi đó $\heva{& y(-2) = 4 - 4\ln 3 \approx -0{,}4 \\& y(-1) = 1 - 4\ln 2  \approx -1{,}7\\& y(0) = 0.} \Rightarrow M = 0, N = 1 -4\ln 2$\\
		Vậy $M - N = 4\ln 2 -1$.
	}
\end{ex} \dongcham{8}

\begin{ex}
	Cho hàm số $f(x)$ nghịch biến trên $\mathbb{R}$. Giá trị nhỏ nhất của hàm số $g(x)=\mathrm{e}^{3x^2-2x^3}-f(x)$ trên đoạn $[0;1]$ bằng
	\choice
	{$\mathrm{e}-f(1)$}
	{$f(1)$}
	{$f(0)$}
	{\True $1-f(0)$}
	\loigiai{
		Ta có $g'(x)=(6x-6x^2)\mathrm{e}^{3x^2-2x^3}-f'(x)$.\\
		Trên đoạn $[0;1]$ thì $6x-6x^2\ge 0$, $f'(x)\le 0$ nên $g'(x)\ge 0$, suy ra hàm số $g(x)$ đồng biến, suy ra giá trị nhỏ nhất là $g(0)=1-f(0)$.
	}
\end{ex} \dongcham{8}

\begin{ex}
	\immini{Cho hàm số $y=f(x)$ xác định và liên tục trên đoạn $\left[0;\dfrac{7}{2}\right]$, có
		đồ thị của hàm số $y=f'(x)$ như hình vẽ. Hỏi hàm số $y=f(x)$ đạt giá trị nhỏ nhất trên đoạn $\left[0;\dfrac{7}{2}\right]$ tại điểm $x_0$ nào dưới đây?
		\choice
		{\True $x_0=3$}
		{$x_0=2$}
		{$x_0=1$}
		{$x_0=0$}
	}
	{\begin{tikzpicture}[scale=1,font=\footnotesize, line join=round,line cap=round,>=stealth]
			\draw [->] (-1,0)--(0,0)
			node[below left]{$O$}--(4.5,0)node[below]{$x$}; % Hệ trục tọa độ
			\draw[->] (0,-1.5) --(0,2) node[left]{$y$};
			\draw[dashed](3.5,0)node[below]{$\tfrac{7}{2}$}--(3.5,25/16);
			\draw(1,0)node[above]{$1$}(3,0)node[above left]{$3$};
			\draw [domain=0:3.5,samples=100] plot (\x, {(\x-1)^2*(\x-3)/2});
	\end{tikzpicture}}
	\loigiai{
		Từ đồ thị hàm số ta có $f'(x)=0 \Leftrightarrow \hoac{&x=1\\&x=3.}$\\
		Bảng biến thiên của hàm số $y=f(x)$ trên đoạn $\left[0;\dfrac{7}{2}\right]$
		\begin{center}
			\begin{tikzpicture}
				\tkzTabInit[nocadre=false,lgt=1.2,espcl=2.5,deltacl=0.7]{$x$ / 1.1 , $f’(x)$ /0.7, $f(x)$ / 2}
				{$0$,$1$, $3$ , $\dfrac{7}{2}$}%
				\tkzTabLine{,-,0,-,0,+,}%
				\tkzTabVar{+ /$f(0)$,R/,-/$f(3)$,+ / $f\left(\dfrac{7}{2}\right)$}%
				\tkzTabIma{1}{3}{2}{$f(0)$}
			\end{tikzpicture}
		\end{center}
		Từ bảng biến thiên ta có hàm số $y=f(x)$ đạt giá trị nhỏ nhất trên đoạn $\left[0;\dfrac{7}{2}\right]$ tại điểm $x_0=3.$
	}
\end{ex} \dongcham{10}

\begin{ex}%[2D1K3-1]
	\immini{Cho hàm số $y=f(x)$, biết hàm số $y=f'(x)$ có đồ thị như hình vẽ dưới đây. Hàm số $y=f(x)$ đạt giá trị nhỏ nhất trên đoạn $\left[\dfrac{1}{2};\dfrac{3}{2} \right]$ tại điểm nào sau đây?
		\choicew{0,25 \textwidth}
		\choice
		{$x=\dfrac{3}{2}$}
		{$x=\dfrac{1}{2}$}
		{\True $x=1$}
		{$x=0$}}{\vspace{-0.5cm}\begin{tikzpicture}[>=stealth,scale=1.5]
			\draw[->] (-0.5,0)--(2.5,0) node[below]{\footnotesize $x$};
			\draw[->] (0,-0.5)--(0,1.5) node[right]{\footnotesize $y$};
			\draw (0,0) node[below left]{\footnotesize $O$};
			\draw[line width = 1pt,smooth,domain=-0.4:1.7] plot({\x},{(\x)^2-\x});
			\draw[fill=black] (1.5,0.75) circle(1pt);
			\draw [dashed] (1.5,0)
			node[below]{\footnotesize$\dfrac{3}{2}$}--(1.5,0.75)--(0,0.75)(1,0)node[below]{$1$};
	\end{tikzpicture}}
	\loigiai{
		Dựa vào đồ thị hàm số $y=f'(x)$. Ta có bảng biến thiên
		\begin{center}\begin{tikzpicture}
				\tkzTabInit[nocadre=false,lgt=1,espcl=2.1]
				{$x$ /1,$y'$ /0.6,$y$ /2}
				{$\dfrac{1}{2}$,$1$,$\dfrac{3}{2}$}
				\tkzTabLine{,-,$0$,+,$0$,}
				\tkzTabVar{+/, -/,+/}
			\end{tikzpicture}
		\end{center}
		Suy ra hàm số đạt giá trị nhỏ nhất trên $\left[\dfrac{1}{2};\dfrac{3}{2} \right]$ tại $x=1$.}
\end{ex} \dongcham{10}

\begin{ex}
	\immini{ Cho hàm số $f(x)$ có đồ thị của hàm số $y=f'(x)$ như hình vẽ. Biết $f(0)+f(1)-2f(2)=f(4)-f(3)$. Giá trị nhỏ nhất $m$, giá trị lớn nhất $M$ của hàm số $f(x)$ trên đoạn $[0;4]$ là
		\choice
		{$m=f(4)$, $M=f(1)$}
		{\True $m=f(4)$, $M=f(2)$}
		{$m=f(1)$, $M=f(2)$}
		{$m=f(0)$, $M=f(2)$}
	}{
		\begin{tikzpicture}[scale=0.79, >=stealth]
			\draw[->] (-0.6,0.) -- (5.3,0.);
			\draw[->] (0.,-1.7) -- (0.,1.6);
			\draw[dashed] (4,0) -- (4,-1.2);
			\clip(-0.6,-1.7) rectangle (5.3,1.7);
			\draw[smooth,samples=100,domain=0:2] plot(\x,{-0.8*((\x)^2-2*(\x))});
			\draw[smooth,samples=100,domain=2:4.5] plot(\x,{0.2*(((\x)-2)*((\x)-7)});
			\draw (-0.3,-0.3) node {$O$} (5.2,0.3) node {$x$} (0.35,1.5) node {$y$} (1.9,-0.3) node {$2$} (4.0,0.3) node {$4$} (2.0,1.15) node {$y=f'(x)$};
			\fill (0,0) circle(1pt) (2,0) circle(1pt) (4,0) circle(1pt); 
		\end{tikzpicture}
	}
	\loigiai{
		Từ đồ thị hàm số $y=f'(x)$ ta suy ra $f'(x)=0 \Leftrightarrow \hoac{&x=0\\&x=2.}$\\
		Ta có bảng biến thiên: 
		\begin{center}\begin{tikzpicture}[>=stealth,scale=1]
				\tkzTabInit[lgt=1.2,espcl=2.5]
				{$x$/1,$f'(x)$/1,$f(x)$/2.5}
				{$0$,$2$,$4$}
				\tkzTabLine{$0$,+,$0$,- }
				\tkzTabVar{-/$f(0)$,+/$f(2)$,-/$f(4)$}
		\end{tikzpicture}\end{center}
		Từ bảng biến thiên ta thấy $M=f(2)$.\\
		Mặt khác, từ bảng biến thiên ta có $\heva{&f(1)<f(2)\\&f(3)<f(2)}\Rightarrow f(1)+f(3)<2f(2)$.\\
		Do đó $f(4)=f(0)+f(1)+f(3)-2f(2)<f(0)+f(2)+f(2)-2f(2)=f(0) \Rightarrow m=f(4)$.	
	}
\end{ex} \dongcham{10}


\begin{ex}
	Giá trị lớn nhất, giá trị nhỏ nhất của hàm số $y={\sin}^3x-3{\sin}^2x+2$ lần lượt là $M$, $m$. Tổng $M+m$ bằng
	\choice
	{\True $0$}
	{$4$}
	{$1$}
	{$3$}
	\loigiai{
		Đặt $t=\sin x \, (-1\le t\le 1)$. Ta có $y=f(t)=t^3-3t^2+2 \, (-1\le t\le 1)$.\\$y'=3t^2-6t=0\Leftrightarrow\hoac{&t=0\in \left[-1;1\right]\\&t=2\notin \left[-1;1\right].}$\\
		Ta có $f(-1)=-2,\,f(1)=0, \,f(0)=2$. Vậy $M=2$ và $m=-2\Rightarrow M+m=0$.}
\end{ex} \dongcham{11}

\begin{ex}
	Giá trị nhỏ nhất của hàm số $f(x)=(x+1)(x+2)(x+3)(x+4)+2019$ là
	\choice
	{$2017$}
	{$2020$}
	{\True $2018$}
	{$2019$}
	\loigiai{
		Tập xác định $\mathscr{D}=\mathbb{R}$.\\
		Biến đổi $f(x)=(x+1)(x+2)(x+3)(x+4)+2019=(x^2+5x+4)(x^2+5x+6)+2019$.\\
		Đặt $t=x^2+5x+4\Rightarrow t=\left( x+\dfrac{5}{2} \right)^2-\dfrac{9}{4}\Rightarrow t\ge-\dfrac{9}{4}\,\forall x\,\in \,\mathbb{R}$.\\
		Hàm số đã cho trở thành $f(x)=t^2+2t+2019=(t+1)^2+2018\ge 2018 \,\forall t\ge -\dfrac{9}{4}$.\\
		Vậy giá trị nhỏ nhất của hàm số đã cho bằng $2018$ tại $t=-1\in \left[-\dfrac{9}{4};+\infty \right)$.
	}
\end{ex} \dongcham{11}

\Closesolutionfile{ans}

\ind{PHẦN II.} \inden{Câu trắc nghiệm đúng sai. Trong mỗi ý a), b), c), d) ở mỗi câu, học sinh chọn đúng hoặc sai.}\\
\Opensolutionfile{ans}[ans/2D1-B2-d1-2]

\begin{ex}%[2D1Y3]
		Cho hàm số $y=f(x)$ là hàm số liên tục trên $\mathbb{R}$ và có bảng biến thiên như hình vẽ dưới đây. 
		\begin{center}
			\begin{tikzpicture}
				\tkzTabInit[nocadre=false, lgt=1.2, espcl=1.5]{$x$ /0.6,$f'(x)$ /0.6,$f(x)$ /1.7}{$-\infty$,$-1$,$0$,$1$,$+\infty$}
				\tkzTabLine{,+,$0$,-,$0$,+,$0$,-,}
				\tkzTabVar{-/ $-\infty$ ,+/$4$,-/$3$,+/$4$,-/$-\infty$}
			\end{tikzpicture}
		\end{center}
	Xét tính đúng, sai của các khẳng định sau:
		\choiceTF
		{\True Cực đại của hàm số là $4$}
		{\True Cực tiểu của hàm số là $3$}
		{\True $\max\limits_{\mathbb{R}}{y}=4$}
		{$\min\limits_{\mathbb{R}}{y}=3$}
	\loigiai{
		Tử bảng biến thiên ta thấy $\lim\limits_{x\to+\infty}f(x)=-\infty$ nên hàm số không có giá trị nhỏ nhất trên $\mathbb{R}.$}
\end{ex} \dongcham{8} 

\begin{ex}
	Hình bên cho biết sự thay đổi của nhiệt độ ở một thành phố trong một ngày. Xét tính đúng, sai của các khẳng định sau:
	\begin{center}
		\begin{tikzpicture}[>=stealth,x=0.25cm,y=0.15cm]
			\draw[->] (-2,0)--(0,0) node[below left]{$O$}--(28,0) node[below right]{$x$ (giờ)};
			\draw[->] (0,-4)--(0,40) node[left]{$t$ ($^\circ C$)};
			\foreach \x/\g in {4/-90,8/-90,12/-90,16/-90,20/-90,24/-90}
			\draw[thin] (\x,2pt)--(\x,-2pt) + (\g:3mm) node {$\x$};
			%Vẽ các điểm trên trục Oy
			\foreach \y/\g in {25/180}
			\draw[thin] (2pt,\y)--(-2pt,\y) + (\g:3mm) node {$\y$};
			\path
			(0,25) coordinate (25)
			(4,20) coordinate (20)
			(8,31) coordinate (31)
			(12,28) coordinate (28)
			(16,34) coordinate (34)
			(20,27) coordinate (27)
			(24,24) coordinate (24);
			\draw [dashed] (4,0)--(4,20) (8,0)--(8,31) (12,0)--(12,28) (16,0)--(16,34) (20,0)--(20,27) (24,0)--(24,24); 
			\draw[smooth, thick, red]
			(25) .. controls +(-10:1) and +(-180:1) .. (20)
			(20) .. controls +(0:1) and +(-180:1) .. (31)
			(31) .. controls +(0:1) and +(160:1) .. (28)
			(28) .. controls +(0:1) and +(-180:2) .. (34)
			(34) .. controls +(0:1.5) and +(130:1.5) .. (27)
			(27) .. controls +(-60:1.5) and +(-180:2) .. (24);
			\foreach \x in {20,31,28,34,27,24}
			\fill (\x) +(90:3mm) node {$\x$};
		\end{tikzpicture}
	\end{center}
		\choiceTF
		{Nhiệt độ cao nhất trong ngày là $28^{\circ} \mathrm{C}$}
		{\True Nhiệt độ thấp nhất trong ngày là $20^{\circ} \mathrm{C}$}
		{\True Thời điểm có nhiệt độ cao nhất trong ngày là lúc $16$ giờ}
		{\True Thời điểm có nhiệt độ thấp nhất trong ngày là lúc $4$ giờ}
	\loigiai{}
\end{ex} \dongcham{8}

\begin{ex}
	Cho hàm số $y=f\left(x\right)$ có đạo hàm $y=f'\left(x\right)$ liên tục trên $\mathbb{R}$ và đồ thị của hàm số $f'\left(x\right)$ trên đoạn $\left[-2;6\right]$ như hình vẽ bên. 	Xét tính đúng, sai của các khẳng định sau:
	\begin{center}
		\begin{tikzpicture}[line join=round, line cap=round,>=stealth,scale=.7]
			\def\xmin{-3}\def\xmax{6.5}\def\ymin{-1}\def\ymax{3.5}
			\draw[->] (\xmin-0.2,0)--(\xmax+0.2,0) node[below] {\small $x$};
			\draw[->] (0,\ymin-0.2)--(0,\ymax+0.2) node[right] {\small $y$};
			\draw (0,0) node [below left] {\footnotesize $O$};
			\foreach \x in {-2,-1,2,6}\draw (\x,0.05)--(\x,-0.05) node [below] {\scriptsize $\x$};
			\foreach \y in {-1,1,2,3}\draw (0.05,\y)--(-0.05,\y) node [left] {\scriptsize $\y$};
			\clip (\xmin,\ymin) rectangle (\xmax,\ymax);
			\draw[thick,smooth,samples=200,domain=-2:6] plot (\x,{13/3360*(\x)^4-61/672*(\x)^3+173/336*(\x)^2-11/42*(\x)-61/70});
			\draw[dashed](-2,0)|-(0,2.5)(6,0)|-(0,1.5);
		\end{tikzpicture}
	\end{center}
		\choiceTF
		{$\max\limits_{\left[-2;6\right]}f\left(x\right)=f\left(-1\right)$}
		{$\max\limits_{\left[-2;6\right]}f\left(x\right)=f\left(6\right)$}
		{$\max\limits_{\left[-2;6\right]}f\left(x\right)=f\left(-2\right)$}
		{\True $\max\limits_{\left[-2;6\right]}f\left(x\right)=\max\left\{ f\left(-1\right),f\left(6\right)\right\}$}
	
	\loigiai{
		\begin{center}
			\begin{tikzpicture}
				\tkzTabInit[nocadre,,lgt=1.2,espcl=2.5,deltacl=0.6]
				{$x$/0.6,$y'$/0.6,$y$/2}
				{$-2$,$-1$,$2$,$6$}
				\tkzTabLine{,+,0,-,0,+,}
				\tkzTabVar{-/$f(-2)$,+/$f(-1)$,-/$f(2)$,+/$f(6)$}
			\end{tikzpicture}
		\end{center}
		Dựa vào bảng biến thiên, ta thấy
		\begin{itemize}
			\item Hàm số đồng biến trên $\left( { - 2; - 1} \right)$ và $\left( {2;6} \right)$ do $f'\left( x \right) > 0$, suy ra
			\begin{center}
				$f\left( { - 1} \right) > f\left( { - 2} \right)$ và $f\left( 6 \right) > f\left( 2 \right)$ (1).
			\end{center}
			\item Hàm số nghịch biến trên $\left( { - 1;2} \right)$ do $f'\left( x \right) < 0$, suy ra
			\begin{center}
				$f\left( { - 1} \right) > f\left( 2 \right)$  $ (2) $.
			\end{center}
		\end{itemize}
		Từ $ (1) $, $ (2) $ suy ra $\mathop {\max }\limits_{\left[ { - 2;6} \right]} f\left( x \right) = \max \left\{ {f\left( { - 2} \right),f\left( { - 1} \right),f\left( 2 \right),f\left( 6 \right)} \right\} = \max \left\{ {f\left( { - 1} \right),f\left( 6 \right)} \right\}$.
	}
\end{ex} \dongcham{13}

\begin{ex}
	Cho hàm số $f(x)$ có đạo hàm là $f'(x)$. Đồ thị $y=f'(x)$ được cho như hình vẽ. Biết rằng $f(0)+f(3)=f(2)+f(5)$. Xét tính đúng, sai của các khẳng định sau:
	\begin{center}
		\begin{tikzpicture}[scale=1, font=\footnotesize, line join=round, line cap=round, >=stealth]
			\draw[->](-1,0)--(5.5,0) node[right] {$x$};
			\draw[->](0,-1)--(0,2.5) node[right] {$y$};
			\node (0,0) [below left]{$O$};
			\foreach \x in {1,...,5}
			\draw[shift={(\x,0)},color=black] (0pt,2pt) -- (0pt,-2pt);
			\foreach \y in {1,...,2}
			\draw[shift={(0,\y)},color=black] (2pt,0pt) -- (-2pt,0pt);
			\draw (-0.3,1.2) .. controls (0.1,-1.8) and (1.5,-0.5) .. (2,0) .. controls (3,1.) and (4,1.2) .. (5,1.3) .. controls (5.1,1.3) and (5.3,1.3) .. (5.5,1.3);
			\clip (-1,-1) rectangle (5.5,2.5);
			\draw[dashed](5,0)--(5,1.3);
			\fill (0,0) circle(1pt) (2,0) circle(1pt) node[below right]{$2$} (5,0) circle(1pt) node[below]{$5$};
		\end{tikzpicture}
	\end{center}
	\choiceTF
	{Hàm số nghịch biến trên khoảng $(-\infty;0)$}
	{\True Hàm số nghịch biến trên khoảng $(0;2)$}
	{$\min\limits_{[0;5]}f(x)=f(0)$ và $\max\limits_{[0;5]}f(x)=f(5)$}
	{\True $\min\limits_{[0;5]}f(x)=f(2)$ và $\max\limits_{[0;5]}f(x)=f(5)$}
	
	\loigiai
	{
		Bảng biến thiên của hàm số trên đoạn $[0;5]$
		\begin{center}
			\begin{tikzpicture}
				\tkzTabInit[lgt=1.5,espcl=3,deltacl=0.6]
				{$x$/0.6, $f'(x)$/0.6, $f(x)$/2}
				{$0$, $2$, $3$, $5$}
				\tkzTabLine{,-,z,+, ,+,}
				\tkzTabVar{+/$f(0)$, -/$f(2)$, R, +/$f(5)$}
				\tkzTabVal[draw]{2}{4}{0.5}{}{$f(3)$}
			\end{tikzpicture}
		\end{center}
		Từ bảng biến thiên suy ra $\min\limits_{[0;5]}f(x)=f(2)$ và $\max\limits_{[0;5]}f(x)=\max\{f(0);f(5)\}$.\\
		Theo bảng biến thiên thì $f(3)>f(2)$ nên $f(3)-f(2)>0$.\\
		Theo giả thiết, ta có
		\[f(0)+f(3)=f(2)+f(5) \Leftrightarrow f(5)=f(0)+\left[f(3)-f(2)\right]>f(0).\]
		Suy ra $\max\limits_{[0;5]}f(x)=f(5)$.\\
		Vậy $\min\limits_{[0;5]}f(x)=f(2)$ và $\max\limits_{[0;5]}f(x)=f(5)$.
	}
\end{ex} \dongcham{13}

\Closesolutionfile{ans}
\begin{dang}{Bài toán max, min có chứa tham số $m$}
\end{dang}
\boxmini{BÀI TẬP TỰ LUẬN}
\begin{vd}
	Tìm tất cả giá trị của tham số $m$ để 
	\begin{tasks}
		\task giá trị lớn nhất của hàm số $f(x)= - x^3 -3x^2 +m$ trên $[-1;1]$ bằng $0$.
		\task giá trị nhỏ nhất của hàm số $ f(x)=\dfrac{x+5m}{x-3} $ trên $[1;2]$ bằng $4$.
	\end{tasks}
	\loigiai{
		\begin{enumerate}[a)]
			\item Hàm số liên tục và xác định trên đoạn $[-1;1]$.\\
			Ta có $f'(x)= -3x^2 -6x$.\\
			Cho $f'(x)=0 \Leftrightarrow \hoac{& x=0 \in [-1;1]\\& x= -2 \notin [-1;1].}$\\
			Xét $f(-1)= -2 + m $; $f(1)= -4 + m$.\\
			Suy ra $\displaystyle \max_{[-1;1]} f(x) = -2 + m$.\\
			Theo đề bài, $-2+ m=0 \Leftrightarrow m=2.$
			\item Ta có $ y'=\dfrac{-3-5m}{(x-3)^2} $.
			\begin{itemize}
				\item Trường hợp $ -3-5m>0\Leftrightarrow m<-\dfrac{3}{5}$\\
				$\Rightarrow y'>0 $ thì $ \displaystyle\min_{[1;2]}y=y(1)\Leftrightarrow -\dfrac{1}{2}(1+5m)=4\Leftrightarrow m=-\dfrac{9}{5}$ (nhận vì $ -\dfrac{9}{5}<-\dfrac{3}{5} $).
				\item Trường hợp $ -3-5m<0\Leftrightarrow m>-\dfrac{3}{5}$\\
				$ \Rightarrow y'<0 $ thì $ \displaystyle\min_{[1;4]}y=y(2)\Leftrightarrow -(2+5m)=4\Leftrightarrow m=-\dfrac{6}{5} $ (loại vì $ -\dfrac{6}{5}<-\dfrac{3}{5} $).
			\end{itemize}
			Vậy  $m=-\dfrac{9}{5}$.
		\end{enumerate}
		}
\end{vd}
\dongcham{20}
\boxmini{BÀI TẬP TRẮC NGHIỆM}

\setcounter{ex}{0}
\Opensolutionfile{ans}[ans/2D1-B2-d2-1]

\begin{ex}
	Cho hàm số $f(x) = 2x^3 -3x^2 + m$ thoả mãn $\displaystyle \min_{[0;5]} f(x) = 5$. Khi đó giá trị của $m$ bằng
	\choice
	{$10 $}
	{$ 5$}
	{\True $ 6$}
	{$ 7$}
	\loigiai{
		Ta có $f'(x)= 6x^2 -6x$.\\
		Cho $f'(x)=0 \Leftrightarrow \hoac{&x=0 \in [0;5] \\& x=1 \in [0;5].}$\\
		Xét $f(0)= m$; $f(1)= -1+ m$; $f(5)= 175 +m$.\\
		Suy ra $\displaystyle \min_{[0;5]} f(x)= -1+m$.\\
		Theo giả thiết $-1+ m= 5 \Leftrightarrow m=6$.}
\end{ex} \dongcham{10}

\begin{ex}
	Tìm $m$ để giá trị nhỏ nhất của hàm số $f(x) = 3x^3 - 4x^2 + 2(m - 10)$ trên đoạn $[1; 3]$ bằng $-5$.
	\choice
	{$m = \dfrac{15}{2}$}
	{$m = - 15$}
	{\True $m = 8$}
	{$m = -8$}
	\loigiai{
		$\bullet$ $f'(x) = 9x^2-8x$. Ta có $f'(x) = 0 \Leftrightarrow \hoac{&x = 0\\&x = \dfrac{8}{9}.}$\\
		$\bullet$ Ta có bảng biến thiên
		\begin{center}
			\begin{tikzpicture}
				\tkzTabInit[espcl=4,lgt=2,deltacl=1]{$x$/1,$f'(x)$/1,$f(x)$/2}
				{$1$,$3$}
				\tkzTabLine{,+,}
				\tkzTabVar{-/$2m-21$,+/$2m+25$}
			\end{tikzpicture}
		\end{center}
		$\bullet$ Giá trị nhỏ nhất của $f(x)$ trên đoạn $[1;3]$ bằng $-5 \Leftrightarrow 2m - 21 = -5 \Leftrightarrow m= 8$.
	}
\end{ex} \dongcham{12}

\begin{ex}
	Tìm $m$ để giá trị nhỏ nhất của hàm số $f(x)=\dfrac{x-m^2+m}{x+1}$ trên đoạn $[0;1]$ bằng $-2$.
	\choice
	{$\hoac{&m=1\\&m=-2}$}
	{$\hoac{&m=1\\&m=2}$}
	{$m=\dfrac{1\pm\sqrt{21}}{2}$}
	{\True $\hoac{&m=-1\\&m=2}$}
	\loigiai{
		$\mathscr{D}=\mathbb{R}\setminus\{-1\}$.\\
		Ta có $f'(x)=\dfrac{m^2-m+1}{(x+1)^2}>0$, $\forall x\in\mathscr{D}$.\\
		Khi đó $\min\limits_{x\in[0;1]}f(x)=f(0)\Leftrightarrow -2=-m^2+m\Leftrightarrow \hoac{&m=-1\\&m=2}$.
	}
\end{ex} \dongcham{12}

\begin{ex}
	Hàm số $y=\dfrac{x-m}{x+2}$ thỏa mãn $\min \limits_{x\in[0;3]}y+\max \limits_{x\in[0;3]}y=\dfrac{7}{6}$. Hỏi giá trị $m$ thuộc khoảng nào trong các khoảng dưới đây?
	\choice
	{$(2;+\infty)$}
	{$(0;2)$}
	{$(-\infty;-1)$}
	{\True $(-1;0)$}
	\loigiai{
		Do hàm số $y=\dfrac{x-m}{x+2}$ luôn đơn điệu trên đoạn $[0;3]$.\\
		Do đó $\min \limits_{x\in[0;3]}y+\max \limits_{x\in[0;3]}y=y(0)+y(3)=\dfrac{-m}{2}+\dfrac{3-m}{5}=\dfrac{7}{6}\Leftrightarrow\dfrac{-7m}{10}=\dfrac{17}{30}\Leftrightarrow m=\dfrac{-17}{21}$.}
\end{ex} \dongcham{11}

\begin{ex}
	Cho hàm số $y=\dfrac{x+m}{x+1}$ ($m$ là tham số thực) thỏa mãn $\min\limits_{[1;2]} y+\max\limits_{[1;2]} y=\dfrac{16}{3}$. Mệnh đề nào dưới đây đúng?
	\choice
	{\True $m>4$}
	{$m\le 0$}
	{$0<m\le 2$}
	{$2<m\le 4$}
	\loigiai{
		Tập xác định $\mathscr{D}=\mathbb{R}$.\\
		Ta có $y'=\dfrac{1-m}{(x+1)^2}$.
		\begin{itemize}
			\item Với $m=1$ thì $y=1$ nên $\min\limits_{[1;2]} y+\max\limits_{[1;2]} y=2$ (không thỏa mãn).
			\item Với $m\neq 1$ thì hàm số đơn điệu trên $[1;2]$ nên
			\begin{eqnarray*}
				&& \min\limits_{[1;2]} y+\max\limits_{[1;2]} y=\dfrac{16}{3}\\
				& \Leftrightarrow & y(1)+y(2)=\dfrac{16}{3}\\
				& \Leftrightarrow & \dfrac{m+1}{2}+\dfrac{m+2}{3}=\dfrac{16}{3}\\
				& \Leftrightarrow & m=5>4.
			\end{eqnarray*}
		\end{itemize}
	}
\end{ex} \dongcham{11}

\begin{ex}
	Cho hàm số $ f(x)=\dfrac{x+m}{x-1} $ ($ m $ là tham số thực) thỏa mãn $ \min\limits_{[2 ; 4]} f(x)=3 $. Mệnh đề nào dưới đây đúng ?
	\choice
	{$1\leq m<3$}
	{$m < -1$}
	{$3<m\leq 4$}
	{\True$m>4$}
	\loigiai{
		Tập xác định $ \mathscr{D} = \mathbb{R} \setminus\{1\}$.\\
		Ta có $ f'(x)=\dfrac{-1-m}{(x-1)^{2}} $.\\
		\underline{\textbf{TH1}}: $ -1-m<0 \Leftrightarrow m >-1 $.\\
		Ta có $ \min\limits_{[2 ; 4]}y=y(4)=\dfrac{4+m}{4-1}=3\Leftrightarrow m=5$ (thỏa mãn).\\
		\underline{\textbf{TH2}}: $  -1-m>0 \Leftrightarrow m <-1 $.\\
		Ta có $ \min\limits_{[2 ; 4]}y=y(2)=\dfrac{2+m}{2-1}=3\Leftrightarrow m=1$ (loại).\\
		Vậy $ m=5>4 $.
	}
\end{ex} \dongcham{11}

\begin{ex}
	Gọi $S$ là tổng giá trị của $m$ để hàm số $f(x) = \dfrac{x - m^2 - m}{x+1}$ có giá trị nhỏ nhất trên $[0;1]$ bằng $-2$. Mệnh đề nào sau đây đúng?
	\choice
	{\True $S=-1 $}
	{$S=1 $}
	{$S=-2 $}
	{$ S=-3$}
	\loigiai{
		Ta có $f'(x)= \dfrac{m^2 + m -1 }{(x+1)^2}$.
		\begin{itemize}
			\item Trường hợp $1$: $y'<0 \Leftrightarrow m^2 + m -1 <0$.\\
			Khi đó hàm số nghịch biến trên $[0;1]$.\\
			Suy ra $\displaystyle \min_{[0;1]} f(x) = f(1)= \dfrac{-m^2 -m +1}{2}$.\\
			Theo giả thiết $\dfrac{-m^2 -m +1}{2} = -2 \Leftrightarrow m^2 + m =5$ (không thoả điều kiện $m^2 +m <1$).
			\item Trường hợp $2$: $y'>0 \Leftrightarrow m^2 + m -1>0$.\\
			Khi đó $\displaystyle \min_{[0;1]} f(x) = f(0)=-m^2 -m$.\\
			Theo giả thiết $-m^2 -m =-2  \Leftrightarrow \hoac{&m= 1 \text{ (nhận) }\\& m=-2 \text{ (nhận).}}$
		\end{itemize}
		Vậy tổng các giá trị của $m$ là $-2+1 =-1.$
	}
\end{ex} \dongcham{11}

\begin{ex}
	Cho hàm số $f(x)=x^3+m x^2-m^2x+2$ với tham số $m>0$. Biết $\min\limits_{[-m ; m]}f(x)=\dfrac{14}{ 27}$. Mệnh đề nào dưới đây đúng
	\choice
	{$m\in (-\infty;-3)$}
	{$m\in (3;+\infty)$}
	{\True $m\in (1;3)$}
	{$m\in (-3;-1)$}
	\loigiai{
		Ta có $f'(x)=3x^2+2mx-m^2=(x+m)(3x-m)$.\\
		$f'(x)=0\Leftrightarrow \hoac{& x=-m \\ & x=\dfrac{m}{3}}$. Suy ra $\heva{& f(-m)=m^3 +2\\ & f(m)=m^3+2\\ &f\left(\dfrac{m}{3}\right)=-\dfrac{5m^3}{27}+2.}$\\
		Vì $m>0$ nên $f(m)=f(-m)>f\left(\dfrac{m}{3}\right)$, suy ra $\min\limits_{  [-m;m]} f(x)=f\left(\dfrac{m}{3}\right)=\dfrac{14}{27}$.\\
		Do đó $m=2$, vậy $m\in(1;3)$.
	}
\end{ex} \dongcham{11}

\begin{ex}%[2D1K3-1]
	Có tất cả bao nhiêu giá trị nguyên của tham số $m$ để giá trị nhỏ nhất của hàm số $y=x^3+\left(m^2-m+1\right)x+m^3-4m^2+m+2025$ trên đoạn $[0;2]$ bằng $2019$?
	\choice
	{$0$}
	{$1$}
	{$2$}
	{\True $3$}
	\loigiai{
		Ta có $y'=f'(x)=3x^2+\left(m^2-m+1\right)$ trên đoạn $[0;2]$.\\
		Ta có $y'=3x^2+\left(m-\dfrac{1}{2}\right)^2+\dfrac{3}{4}>0,\forall x\in\mathbb{R}$.\\
		Do đó hàm số đồng biến trên $\mathbb{R}\Rightarrow$ ta có $\min\limits_{[0;2]}y=f(0)=m^3-4m^2+m+2025$.\\
		Ta có $f(0)=2019\Leftrightarrow m^3-4m^2+m+2025=2019\Leftrightarrow m^3-4m^2+m+6=0\Leftrightarrow\hoac{&m=-1\\&m=2\\&m=3.}$\\
		Vậy tập các giá trị $m$ thỏa mãn là $\{-1;2;3\}$. Hay có tất cả $3$ giá trị $m$ thỏa mãn.}
\end{ex} \dongcham{11}

\begin{ex}
	Gọi $S$ là tập tất cả các giá trị của $m$ sao cho giá trị nhỏ nhất của hàm số $y=\left(x^3-3x+m \right)^2 $ trên
	đoạn $[-1;1]$ bằng $4$. Tính tổng các phần tử của $S$.
	\choice
	{\True  $ 0 $}
	{$ 6 $}
	{$ -5 $}
	{$ 3 $}
	\loigiai{
		\immini{Ta có  $\displaystyle\min\limits_{[-1;1]}\left(x^3-3x+m \right)^2=4  \Leftrightarrow \displaystyle\min\limits_{[-1;1]}\left|x^3-3x+m \right|=2$.\\Xét hàm số $y=f(x)=x^3-3x+m$ trên $[-1;1]$.\\
			Ta có $y'=3x^2-3=3(x^2-1)$, $y'=0\Leftrightarrow x=\pm1$.\\
			Bảng biến thiên hàm số như hình bên.
		}{\begin{tikzpicture}[scale=.8,line join=round, line cap=round,font=\footnotesize,>=stealth]
				\def\a{6}
				\def\b{3.7}
				\draw[shift={(-.5,.5)},blue!50!black]
				(0,0) rectangle +(\a,-\b)
				(0,-1)--+(0:\a)
				(0,-2)--+(0:\a)
				(1,0)--+(-90:\b)
				;
				\path
				(0,0) node{$x$}
				(0,-1) node{$y'$}
				(0,-2.3) node{$y$}
				(1,0) node{$-1$}
				(5,0) node{$1$}
				(1,-1) node{$0$}
				(3,-1) node{$-$}
				(5,-1) node{$0$}
				(1.2,-1.8) node (A) {$m+2$}
				(4.8,-3) node (C){$m-2$}
				;
				\draw[->] (A)--(C);
		\end{tikzpicture}}
		\noindent Từ bảng biến thiên của hàm số $y=f(x)$, ta có $\displaystyle\min\limits_{[-1;1]}\left|x^3-3x+m \right|=2$ khi và chỉ khi
		\begin{enumerate}[TH1.]
			\item $\heva{&m+2<0\\&m+2=-2}\Leftrightarrow m=-4$.
			\item $\heva{&m-2>0\\&m-2=2}\Leftrightarrow m=4$.
		\end{enumerate}
		Vậy $S=\{-4,4\}\Rightarrow $ Tổng các phần tử của $S$ bằng $0$.
	}
\end{ex} \dongcham{12}

\Closesolutionfile{ans}



% 
\begin{dang}{Bài toán vận dụng, thực tiễn có liên quan đến max min}
	\begin{enumerate}[\iconMT]
		\item \indam{Bài toán chuyển động:}
		\begin{itemize}
			\item [$\bullet$] Gọi $s(t)$ là hàm quãng đường; $v(t)$ là hàm vận tốc; $a(t)$ là hàm giá tốc;
			\item [$\bullet$] Khi đó $s'(t)=v(t)$; $v'(t)=a(t)$.
		\end{itemize}
		\item \indam{Bài toán thực tế -- tối ưu:}
		\begin{itemize}
			\item[$\bullet$] Biểu diễn dữ kiện cần đạt max -- min qua một hàm $f(t)$. 
			\item[$\bullet$] Khảo sát hàm $f(t)$ trên miền điều kiện của hàm và suy ra kết quả.
		\end{itemize}
	\end{enumerate}
\end{dang}
\boxmini{BÀI TẬP TỰ LUẬN}
\begin{vd}%[2D1B3-6]
	Một chất điểm chuyển động có vận tốc tức thời $v(t)$ phụ thuộc vào thời gian $t$ theo hàm số $v(t)=-t^4+24t^2+500$ (m/s). Trong khoảng thời gian từ $t=0$ (s) đến $t=5$ (s) chất điểm đạt vận tốc lớn nhất tại thời điểm nào?
	\loigiai{Ta có $v'(t)=-4t^3+48t=-4t(t^2-12)$\\
		$v'(t)=0\Leftrightarrow \hoac{&t=0\\&t=\pm 2\sqrt{3}}$.\\
		Bài toán trở thành tìm giá trị lớn nhất của hàm số $v(t)$ trên đoạn $[0;10]$, ta có:\\
		$v(0)=500$, $v(2\sqrt{3})=664$, $v(5)=475$.\\
		Vậy vận tốc lớn nhất khi $t=2\sqrt{3}\approx 4$ (s).
	}
\end{vd}
\dongcham{8}
\begin{vd}
	\immini{
		Sự phân huỷ của rác thải hữu cơ có trong nước sẽ làm tiêu hao oxygen hoà tan trong nước. Nồng độ oxygen (mg/l) trong một hồ nước sau $t$ giờ $(t \geq 0)$ khi một lượng rác thải hữu cơ bị xả vào hồ được xấp xỉ bởi hàm số (có đồ thị như đường màu đỏ ở hình bên)
		$$
		y(t)=5-\frac{15 t}{9 t^2+1}.
		$$
	}{
		\begin{tikzpicture}[>=stealth,x=1cm,y=0.3cm,scale=1.5,font=\footnotesize]
			\draw[->] (-0.5,0) -- (4,0) node[below] {$t$};
			\draw[->] (0,-1) -- (0,6) node[left] {$y$};
			\filldraw (0,0) circle (1pt)node[below left]{$O$};
			\draw[domain=0:4,samples=200,red] plot (\x,{5-(15*(\x))/(9*(\x)^2+1)});
			\draw[dashed] (0,5) node [left] {$5$}--(4,5);
			\foreach \x/\g in {1/-90,2/-90,3/-90}
			\draw[thin] (\x,2pt)--(\x,-2pt) + (\g:3mm) node {$\x$};
		\end{tikzpicture}
	}
	\noindent
	Vào các thời điểm nào nồng độ oxygen trong nước cao nhất và thấp nhất?\
	\loigiai{
		Xét hàm số $y(t)=5-\dfrac{15t}{9t^2+1}$ xác định và liên tục trên khoảng $[0;+\infty)$ .\\
		Ta có $y'(t)=\dfrac{135t^2-15}{(9t^2+1)^2}=0\Leftrightarrow t=\dfrac{1}{3}$ (giờ).\\
		Mặt khác $\lim\limits_{t\to+\infty}y(t)=\lim\limits_{t\to+\infty}\left[5-\dfrac{15t}{9t^2+1}\right]=5$ và $\lim\limits_{t\to 0}y(t)=\lim\limits_{t\to 0}\left[5-\dfrac{15t}{9t^2+1}\right]=5$.\\
		Bảng biến thiên
		\begin{center}
			\begin{tikzpicture}
				\tkzTabInit[espcl=3,lgt=1.5]
				{$t$/0.6,$y'(t)$/0.6,$y(t)$/1.5}
				{$0$,$\frac{1}{3}$,$+\infty$}
				\tkzTabLine{,-,0,+,}
				\tkzTabVar{+/$5$,-/$0$,+/$5$}
			\end{tikzpicture}
		\end{center}
		Từ bảng biến thiên, ta thấy $\min\limits_{[0;+\infty)}y(x)=0$ và $\mathop{\rm{max}}\limits_{[0;+\infty)}y(x)=5$.
	}
\end{vd}
\dongcham{10}
\begin{vd}%[2D1T3-6]
	\immini[0.02]{
		Tính diện tích lớn nhất $S_{\max}$ của một hình chữ nhật nội tiếp trong nửa đường tròn bán kính $R=6$ cm nếu một cạnh của hình chữ nhật nằm dọc theo đường kính của hình tròn mà hình chữ nhật đó nội tiếp.
	}{
		\begin{tikzpicture}[line join = round, line cap = round,>=stealth,font=\footnotesize,scale=1] 
			\def\R{2}
			\coordinate[label = below:$O$] (O) at (0,0); 
			\coordinate (A) at (-\R,0); 
			\coordinate (B) at ($(A)!2!(O)$);
			\coordinate[label = above right:$C$] (C) at (50:\R); 
			\coordinate[label = above left:$D$] (D) at (130:\R);
			\coordinate[label = below:$A$] (AA) at ($(A)!(D)!(B)$); 
			\coordinate[label = below:$B$] (BB) at ($(A)!(C)!(B)$); 
			\draw (A) arc(180:0:\R)--cycle;
			\draw[fill=cyan!20] (BB)--(C)--(D)--(AA)--cycle;
			\foreach \x in {AA,O,BB} \fill[black] (\x) circle (1.5pt); 
		\end{tikzpicture}
	}
	\loigiai{
		\immini{
			{\bf Cách 1.}\\
			Gọi chiều dài $AD=2x$ ($0<x<6$)\\
			$\Rightarrow AB=\sqrt{36-x^{2}}$.\\
			Diện tích hình chữ nhật là $S=2x\sqrt{36-x^{2}}$.\\
			Xét $f(x)=x\sqrt{36-x^{2}}$ trên $(0;6)$, ta có $$f'(x)=\sqrt{36-x^{2}}-\dfrac{x^{2}}{\sqrt{36-x^{2}}}=0\Leftrightarrow x=\pm 3\sqrt{2}.$$
		}{
			\begin{tikzpicture}
				\tikzset{on double/.style = {fill = \tkzTabDefaultBackgroundColor}} 
				\tikzset{h style/.style = {pattern=north west lines}} 
				\tkzTabInit[lgt=1.2,espcl=2]
				{$x$ /.6,$f'(x)$ /.6, $f(x)$ /1.5}
				{$0$,$3\sqrt{2}$,$6$}
				\tkzTabLine{d,+,0,-,d}
				\tkzTabVar{-/$0$,+/$36$,-/$0$}
			\end{tikzpicture}
		}
		Bảng biến thiên hàm số $f(x)$ trên $(0,6)$ ở hình bên\\
		Vậy giá trị lớn nhất của diện tích hình chữ nhật $ABCD$ là $36$ cm$^2$.\\
		{\bf Cách 2.}\\
		Đặt $AB=CD=2x$ ($0<x<6$). Khi đó $AD=\sqrt{DO^2-AO^2}=\sqrt{36-x^2}$. Suy ra
		\begin{align*}
			S_{ABCD}=2x\sqrt{36-x^2}\le 2\cdot \dfrac{x^2+36-x^2}{2}=36.
		\end{align*}
		Dấu bằng xảy ra khi $x=\sqrt{36-x^2}$ hay $x=3\sqrt{2}$.\\
		Vậy giá trị lớn nhất của diện tích hình chữ nhật $ABCD$ là $36$ cm$^2$.
	}
\end{vd}
\dongcham{14}
\begin{vd}%[2D1K3-6]
	\immini{Một người muốn xây một cái bể chứa nước, dạng một khối hộp chữ nhật không nắp có thể tích
	bằng $288$ dm$^3$. Đáy bể là hình chữ nhật có chiều dài gấp đôi chiều rộng, giá thuê nhân công để xây bể là
	$500000$ đồng/ m$^2$. Nếu người đó biết xác định các kích thước của bể hợp lí thì chi phí thuê nhân công sẽ
	thấp nhất. Hỏi người đó trả chi phí thấp nhất để thuê nhân công xây dựng bể đó là bao nhiêu?}{\hspace{1cm}
	\begin{tikzpicture}[scale=0.8, line join=round, line cap=round]
		\tkzDefPoints{0/0/A,-1.3/-1.1/B,2/-1.1/C}
		\coordinate (D) at ($(A)+(C)-(B)$);
		\coordinate (A') at ($(A)+(0,2.5)$);
		\tkzDefPointsBy[translation=from A to A'](B,C,D){B'}{C'}{D'}
		\tkzDrawPolygon(A',B',B,C,D,D')
		\tkzDrawSegments(B',C' C',D' C,C')
		\tkzDrawSegments[dashed](A,B A,D A,A')
\end{tikzpicture}}
	\loigiai{
		Gọi $x(x>0)$ là chiều rộng của đáy bể. Khi đó, chiều dài của bể là $2x$ và chiều cao của bể là $\dfrac{0,144}{x^2}$.\\
		Diện tích cần xây $2x^2+\dfrac{0,864}{x}$\\
		Xét $f(x) = 2x^2 + \dfrac{0,864}{x}$, có
		$f'(x) = 4x - \dfrac{0,864}{x^2}$\\
		$f'(x) = 0 \Leftrightarrow 4x - \dfrac{0,864}{x^2} \Leftrightarrow x=0,6.$\\
		Bảng biến thiên
		\begin{center}
			\begin{tikzpicture}
				\tkzTabInit[nocadre=false, lgt=1.2, espcl=3]
				{$x$ /0.6,$f'(x)$ /0.6,$f(x)$ /1.5} 	
				{$0$, $0{,}6$, $+\infty$}
				\tkzTabLine{,-,$0$,+}
				\tkzTabVar{+/ $+\infty$ ,-/$2{,}16$,+/$+\infty$}
			\end{tikzpicture}
		\end{center}
		Từ bảng biến thiên ta có $\min f(x)= 2,16.$\\
		Vậy chi phí thấp nhất để thuê nhân công xây bể là $2,16 \times 500000 = 1080000$ đồng.
	}
\end{vd}
\dongcham{18}
\begin{vd}%[2D1T3-2]
	\immini{Một nhà sản xuất cần làm ra những chiếc bình có dạng hình trụ với dung tích $1000\mathrm{~cm}^3$. Mặt trên và mặt dưới của bình được làm bằng vật liệu có giá 1,2 nghìn đồng$/\mathrm{cm}^2$, trong khi mặt bên của bình được làm bằng vật liệu có giá $0{,}75$ nghìn đồng$/\mathrm{cm}^2$. Tìm các kích thước của bình để chi phí vật liệu sản xuất mỗi chiếc bình là nhỏ nhất.}{\hspace{1cm}
	\begin{tikzpicture}[line join=round,line cap=round,line width=.6pt,font=\footnotesize,scale=0.46,>=stealth]
		\coordinate[label=right:$A$] (A) at (3,0);
		\coordinate[label=left:$O$] (O) at (0,0);
		\coordinate[label=right:$A'$] (A1) at ($(A)+(90:6)$);
		\coordinate[label=left:$O'$] (O1) at ($(O)+(90:6)$);
		\draw (A) arc (0:-180:3 and 3/4)--($(A1)!2!(O1)$) arc (180:0:3 and 3/4) arc (0:-180:3 and 3/4) (A)--(A1)--(O1);
		\draw[dashed] (O1)--(O)--(A) arc (0:180:3 and 3/4);
		\fill (O)circle(1.5pt) (O1)circle(1.5pt) (A)circle(1.5pt) (A1)circle(1.5pt);
\end{tikzpicture}}
	\loigiai{
			Gọi bán kính đáy của bình là $x$ (cm), ($x > 0$).\\
			Chiều cao của bình là $\dfrac{1000}{\pi \cdot x^2}$ (cm).\\
			Chi phí để sản xuất một chiếc bình là 
			\[
			T(x)=2\cdot1{,}2\cdot\pi \cdot x^2+0{,}75\cdot \dfrac{2000}{x}=2{,}4\pi \cdot x^2+\dfrac{1500}{x}~\text{(nghìn đồng)}.
			\]
			Để chi phí sản xuất mỗi chiếc bình là thấp nhất thì $T(x)$ là nhỏ nhất.\\
			$T^{\prime}(x)=4,8\pi x-\dfrac{1500}{x^2}, T^{\prime}(x)=0\Leftrightarrow x=\sqrt[3]{\dfrac{625}{2\pi}}$ (thỏa mãn).\\
			Bảng biến thiên:
			\begin{center}
				\begin{tikzpicture}[scale=1, font=\footnotesize]
					\tkzTabInit[nocadre=false, lgt=1.2, espcl=2, deltacl=0.6]
					{$x$/0.8,$T'(x)$/0.6,$T(x)$/2}
					{$0$,$\sqrt[3]{\frac{625}{2\pi}}$,$12$};
					\tkzTabLine{,-,$0$,+,};
					\tkzTabVar{+/$+\infty$,-/$T\left(\sqrt[3]{\frac{625}{2\pi}}\right)$,+/$T(12)$};
				\end{tikzpicture}
			\end{center}
			Để chi phí sản xuất mỗi chiếc bình là nhỏ nhất thì bán kính đáy của bình là $\sqrt[3]{\dfrac{625}{2\pi}}$ cm và chiều cao của bình là $\dfrac{1000}{\pi \cdot\left(\sqrt[3]{\dfrac{625}{2\pi}}\right)^2}$ cm.
	}
\end{vd}
\dongcham{20}
\boxmini{BÀI TẬP TRẮC NGHIỆM}
\ind{PHẦN I.} \inden{Câu trắc nghiệm nhiều phương án lựa chọn. Mỗi câu hỏi học sinh chỉ chọn một phương án.}\\
\setcounter{ex}{0}
\Opensolutionfile{ans}[ans/2D1-B2-d3-1]
\begin{ex}%[2D1K3]
	Một chất điểm chuyển động với quãng đường $s(t)$ cho bởi công thức $s(t)=6t^2-t^3$, $t$ (giây) là thời gian. Hỏi trong khoảng thời gian từ $0$ đến $4$ giây, vận tốc tức thời của chất điểm đạt giá trị lớn nhất tại thời điểm  $t$ (giây) bằng bao nhiêu?
	\choice
	{$t=3$ s}
	{$t=4$ s}
	{\True $t=2$ s}
	{$t=6$ s}
	\loigiai{Ta có $v(t)=s'(t)=12t-3t^2$.\\
		$v'(t)=12-6t$, $v'(t)=0\Leftrightarrow t=2$. \\
		Lập bảng biến thiên ta thấy $v(t)$ đạt giá trị lớn nhất tại $t=2$.
	}
\end{ex} \dongcham{7}

\begin{ex}
	Trong $3$ giây đầu tiên, một chất điểm chuyển động theo phương trình $s(t)=-t^3+6t^2+t+5,$ trong đó $t$ tính bằng giây và $s$ tính bằng mét. Chất điểm có vận tốc tức thời lớn nhất bằng bao nhiêu trong $3$ giây đầu tiên đó?
	\choice
	{\True 13 m/s}
	{10 m/s}
	{9 m/s}
	{12 m/s}
	\loigiai{
		Ta có $v(t)=s'(t)=-3t^2+12t+1.$ Xét hàm số $v(t)=-3t^2+12t+1$ trên đoạn $[0;5]$.\\
		$v'(t)=-6t+12$; $v'(t)=0 \Leftrightarrow t=2$.\\
		Tính các giá trị $v(0)=1$, $v\left(2\right)=13$, $v(3)=10$.\\
		So sánh các giá trị, ta có $\max\limits_{[0;3]}v(t)=13$.
	}
\end{ex}
\dongcham{7}
\begin{ex}
	Độ giảm huyết áp của một bệnh nhân được cho bởi công thức $G(x)=0{,}025x^2(30-x)$, trong đó $x$ là liều lượng thuốc được tiêm cho bệnh nhân ($x$ được tính bằng miligam). Liều lượng thuốc cần tiêm cho bệnh nhân là bao nhiêu để huyết áp được giảm nhanh nhất?
	\choice
	{$24$ mg}
	{\True $20$ mg}
	{$15$ mg}
	{$10$ mg}
	\loigiai
	{ 
		Bài toán trở thành: Tìm $x\in[0;30]$ để hàm số $G(x)=0{,}025x^2(30-x)$ đạt giá trị lớn nhất. \\
		Ta có $G(x)=0{,}025\left(30x^2-x^3\right) \Rightarrow G'(x)=0{,}025\left(60x-3x^2\right)$. \\
		Xét $G'(x)=0 \Leftrightarrow \hoac{ & x=0 \\ & x=20.}$ \\
		Bảng biến thiên hàm số $G(x)$
		\begin{center}
			\begin{tikzpicture}[scale=1]
				\tkzTabInit[nocadre=false, lgt=1.2, espcl=3.5, deltacl=0.6]{$x$/0.6, $G'(x)$/0.6, $G(x)$/2}{$0$, $20$, $30$}
				\tkzTabLine{0,+,0,-,}
				\tkzTabVar{-/$0$, +/$100$, -/$0$}
			\end{tikzpicture}
		\end{center}
		Từ bảng biến thiên ta có $\max\limits_{[0;30]} G(x)=G(20)=100$. \\
		Vậy liều lượng thuốc cần tiêm cho bệnh nhân để huyết áp giảm nhanh nhất là $20$ mg.
	}
\end{ex}
\dongcham{7}
\begin{ex}
	Trong thí nghiệm y học, người ta cấy $1\,000$ vi khuẩn vào môi trường dinh dưỡng. Bằng thực nghiệm, người ta xác định số lượng vi khuẩn thay đổi theo thời gian bởi công thức \[N(t)=1\,000+\dfrac{100t}{100+t^2}\,\text(con).\]
	trong đó $t$ là thời gian tính bằng giây. Tính số lượng vi khuẩn lớn nhất kể từ khi thực hiện cấy vi khuẩn vào môi trường dinh dưỡng.
	\choice
	{$1\,008$ con}
	{$1\,012$ con}
	{\True $1\,005$ con}
	{$1\,020$ con}
	\loigiai{
		Xét hàm số $N(t)=1\,000+\dfrac{100t}{100+t^2}$ ($t>0$).\\
		Ta có $N'(t)=\dfrac{100\cdot (100+t^2)-100t\cdot 2t}{\left(100+t^2\right)^2}=\dfrac{100\cdot (100-t^2)}{\left(100+t^2\right)^2}$.\\
		Khi đó, với $t>0$, $N'(t)=0\Leftrightarrow 100-t^2=0\Leftrightarrow t^2=100\Leftrightarrow t=10$.\\
		Bảng biến thiên của hàm số $N(t)$ như sau
		\begin{center}
			\begin{tikzpicture}[>=stealth]
				\tkzTabInit[nocadre=false,lgt=1.5,espcl=3,deltacl=0.6]{$t$/.6 ,$N'(t)$/.6,$N(t)$/1.5}
				{$0$ , $10$ , $+\infty$}
				\tkzTabLine{ ,+ , $0$ , - , }
				\tkzTabVar{-/$1\,000$ , +/$1\,005$ , -/$1\,000$}
			\end{tikzpicture}
		\end{center}
		Căn cứ vào bảng biến thiên, ta thấy trên khoảng $(0;+\infty)$, hàm số $N(t)$ đạt giá trị lớn nhất bằng $1\,005$ tại $t=10$.\\
		Vậy số lượng vi khuẩn lớn nhất kể từ khi thực hiện nuôi cấy vi khuẩn vào môi trường dinh dưỡng là $1\,005$ con.
	}
\end{ex}
\dongcham{14}
\begin{ex}
	Tam giác vuông có cạnh huyền bằng $5 \mathrm{~cm}$ có thể có diện tích lớn nhất bằng bao nhiêu?
	\choice
	{25 $\text{cm}^2$}
	{$\dfrac{125}{4}\,\text{cm}^2$}
	{$\dfrac{625}{4}\,\text{cm}^2$}
	{$125 \text{cm}^2$}
	\loigiai{Gọi một cạnh góc vuông là $x$ ($0<x<5$) thì cạnh góc vuông còn lại là $\sqrt{25-x^2}$.\\ Như vậy, diện tích tam giác là $S=\dfrac{x\cdot\sqrt{25-x^2}}{2}$.
		Đặt $f(x)=25x^2-x^4$. 
		\\Ta có $f'(x)=50x-4x^3$. Khi đó
		$f'(x)=0 \Leftrightarrow x=\dfrac{5\sqrt{2}}{2}$.\\
		Vì vậy $\displaystyle\max _{(0;5)} f(x)=f\left( \dfrac{5\sqrt{2}}{2}\right) =\dfrac{625}{4}$.\\
		Vậy tam giác vuông có cạnh huyền bằng $5 \mathrm{~cm}$ có thể có diện tích lớn nhất bằng $\dfrac{625}{4}$.}
\end{ex}
\dongcham{18}
\begin{ex}
	\immini{
		Từ một tấm tôn có hình dạng là nửa hình tròn bán kính $R=3$, người ta muốn cắt ra một hình chữ nhật (hình vẽ bên). Diện tích lớn nhất có thể của tấm tôn hình chữ nhật là
		\choice
		{$\dfrac{9}{2}$}
		{$6\sqrt2$}
		{\True $9$}
		{$9\sqrt2$}
	}
	{
		\begin{tikzpicture}[thick,scale=0.57]
			\draw [-] (-4,0)--(4,0);
			\draw [-] (-3,0)--(-2.99,2.65)--(2.99,2.65)--(3,0);
			\draw[smooth,samples=200,variable=\t,domain=0:180] plot({(4)*cos (\t)},{(4)*sin(\t)});
			\draw (0,0) [fill=black] circle (1pt) node[below]{$O$};
			\draw (-3,0) [fill=black] circle (1pt) node[below]{$Q$};
			\draw (3,0) [fill=black] circle (1pt) node[below]{$P$};
			\draw (-2.99,2.65) [fill=black] circle (1pt) node[left]{$M$};
			\draw (2.99,2.65) [fill=black] circle (1pt) node[right]{$N$};
			\draw[pattern=north east lines,pattern color=black!50!] (-3,0)--(-2.99,2.65)--(2.99,2.65)--(3,0);
		\end{tikzpicture}
	}
	\loigiai{
		Đặt $OQ=x,\ (0<x<3) \Rightarrow MQ=\sqrt{MO^2-OQ^2}=\sqrt{9-x^2}$.\\
		Ta có  $S_{MNPQ}=PQ\cdot MQ=2x\cdot\sqrt{9-x^2}\le 2\cdot\dfrac{x^2+9-x^2}{2}=9.$\\
		Dấu $=$ xảy ra khi $x=\dfrac{3\sqrt2}{2}.$
	}
\end{ex} \dongcham{18}

\begin{ex}%[2D1K3-6]
	Cho một tấm tôn hình chữ nhật có kích thước $10$ cm $\times$ $16$ cm. Người ta cắt bỏ $4$ góc của tấm tôn $4$ miếng hình vuông bằng nhau rồi gò lại thành một hình hộp chữ nhật không có nắp. Để thể tích của hình hộp đó lớn nhất thì độ dài cạnh hình vuông của các miếng tôn bị cắt bỏ bằng
	\choice
	{$3$ m}
	{$4$ m}
	{$5$ m}
	{\True $2$ m}
	\loigiai{
		\immini
		{Giả sử độ dài cạnh hình vuông của các miếng tôn bị cắt bỏ bằng $x$ $(0<2x<10\Leftrightarrow 0<x<5)$. Khi đó hình hộp chữ nhật có chiều cao bằng $x$, chiều rộng bằng $10-2x$ và chiều dài bằng $16-2x$. Suy ra hình hộp chữ nhật có thể tích $V=x(10-2x)(16-2x)=4x^3-52x^2+160x$.}
		{
			\begin{tikzpicture}[scale=0.7]
				\tkzInit[xmin=-5,xmax=6,ymin=-3,ymax=6]
				\tkzDefPoints{0/0/A, 8/0/D, 8/6/C, 0/6/B, 0/1/E, 0/5/F, 1/6/G, 7/6/H, 8/5/I, 8/1/J, 1/0/M, 7/0/N}
				\tkzDrawPoints(A,B,C,D,M,N,E,F,G,H,I,J)
				\tkzLabelSegments[above](B,G H,C){$x$}
				\tkzLabelSegments[right](C,I J,D){$x$}
				\tkzLabelSegment[left](A,B){$10$}
				\tkzLabelSegment[below](A,D){$16$}
				\tkzDrawSegments[thin](A,B A,D B,C C,D F,I E,J G,M H,N)
			\end{tikzpicture}
		}
		Xét hàm $f(x)=4x^3-52x^2+160x$ trên $(0; 5)$. Tập xác định $\mathscr{D}=\mathbb{R}$,\\ $f'(x)=12x^2-104x+160=0\Leftrightarrow\hoac{&x=2\\&x=\dfrac{20}{3}.}$
		Bảng biến thiên hàm số $f(x)$ trên $(0; 5)$:
		\begin{center}
			\begin{tikzpicture}
				\tkzTabInit%
				{$x$/1,%
					$f’(x)$ /1,%
					$f(x)$ /2}%
				{$0$ ,$2$ , $5$}%
				\tkzTabLine{ ,+, 0 ,-,}
				\tkzTabVar %
				{
					-/,+/ ,-/
				}
			\end{tikzpicture}
		\end{center}
		Dựa vào bảng biến thiên ta có hàm số đạt giá trị lớn nhất trên $(0; 5)$ tại $x=2$ hay hình hộp chữ nhật có thể tích lớn nhất khi độ dài cạnh hình vuông của miếng tôn bị cắt bỏ bằng $2$ m.
	}
\end{ex}
\dongcham{18}

\begin{ex}%[2H1K3-6]
	Ông Bình dự định sử dụng hết $5,5\,\mathrm{m^2}$ kính để làm một bể cá bằng kính có dạng hình hộp chữ nhật không nắp, chiều dài gấp đôi chiều rộng (các mối ghép có kích thước không đáng kể). Bể cá có dung tích lớn nhất bằng bao nhiêu (làm tròn đến hàng phần trăm)?
	\choice
	{ $1{,}01\,\mathrm{m^3}$}
	{\True $1{,}17\,\mathrm{m^3}$}
	{ $1{,}51\,\mathrm{m^3}$}
	{ $1{,}40\,\mathrm{m^3}$}
	\loigiai{
		\immini{
			Gọi $x,2x,y$ với $x,y>0$  lần lượt là chiều rộng, chiều dài, chiều cao của bể cá.
			Theo giả thiết ta có: $$2\cdot 2xy+2\cdot xy+2x^2=5{,}5\Leftrightarrow 6xy+2x^2=5{,}5\Rightarrow y=\dfrac{5{,}5-2x^2}{6x}.$$
			Do $y>0$ nên $5,5 - 2x^2 >0 \Rightarrow 0<x<\dfrac{\sqrt{11}}{2}$.\\
			Thể tích bể cá là $$V(x)=2x^2y=2x^2\cdot \dfrac{5{,}5-2x^2}{6x}=-\dfrac{2}{3}{x^3}+\dfrac{11}{6}x.$$
			Khảo sát hàm số $V(x)=-\dfrac{2}{3}{x^3}+\dfrac{11}{6}x$ trên khoảng $\left( 0;\dfrac{\sqrt{11}}{2} \right) $
			\begin{itemize}
				\item [$\bullet$] $V'(x)=-2x^2+\dfrac{11}{6}$; $V'(x)=0\Leftrightarrow x=\sqrt{\dfrac{11}{12}}$.
				\item [$\bullet$] Bảng biến thiên:
				\begin{center}
					\begin{tikzpicture}
						\tkzTabInit[nocadre=True,lgt=1,espcl=3]
						{$x$ /1,$V'$ /0.6,$V$ /2}
						{$0$,$\sqrt{\frac{11}{12}}$,$\frac{\sqrt{11}}{2}$}
						\tkzTabLine{,+,$0$,-,}
						\tkzTabVar{-/, +/$y_0$,-/}
					\end{tikzpicture}
				\end{center}
			\end{itemize}
			Thể tích lớn nhất của bể cá là $V\left( \sqrt{\dfrac{11}{12}} \right)=1{,}17\,\mathrm{m^3}$.}{
			\begin{tikzpicture}
				\def\tls{.4}
				\path
				(0,0) coordinate (A)
				++ (0:4)coordinate (B)
				++ (30:2.3)coordinate (C)
				($(A)+(C)-(B)$)coordinate (D)
				\foreach \x in {A,B,C,D}{(\x)++(90:2.5) coordinate (\x_1)}
				;
				\draw[dashed]
				(A)--(D) node[pos=.5,sloped,above]{$x$}
				(D)--(C) node[pos=.4,sloped,above]{$2x$}
				(D)--(D_1) node[pos=.4, right]{$y$}
				;
				\draw
				(A)--(B)--(C)
				(A_1)--(B_1)--(C_1)--(D_1)--cycle
				(A)--(A_1) (B)--(B_1) (C)--(C_1)
				;
		\end{tikzpicture}}
	}
\end{ex} \dongcham{20}

\begin{ex}%[2D1T3-6]
	Người ta muốn xây một chiếc bể nước có hình dạng là	một khối hộp chữ nhật không nắp có thể tích bằng $\dfrac {500}{3}$ m$^3$. Biết đáy bể là một hình chữ nhật có chiều dài gấp đôi chiều rộng và giá thuê thợ xây là $700.000$ đồng/m$^2$. Để chi phí thuê nhân công ít nhất thì chi phí thuê nhân công là
	\choice
	{$120$ triệu đồng}	
	{\True $105$ triệu đồng}
	{$115$ triệu đồng}	
	{$110$ triệu đồng}
	\loigiai{
		Gọi $x,y$ lần lượt là chiều rộng và chiều cao của bể cá (điều kiện $x,y>0$ ).
		\immini{Với giả thiết của bài toán, thể tích bể cá là $$V=2x^2y=\dfrac {500}{3}\Rightarrow y=\dfrac {250}{3x^2}.$$
			Để chi phí thuê nhân công ít nhất thì tổng diện tích các mặt của bể cá phải nhỏ nhất. Tổng diện tích các mặt của bể cá} 
		{\begin{tikzpicture}[scale=0.8, font=\footnotesize, line join=round, line cap=round, >=stealth]
				\tkzDefPoints{0/0/A,-1.3/-1.1/B,2/-1.1/C}
				\coordinate (D) at ($(A)+(C)-(B)$);
				\coordinate (A') at ($(A)+(0,2.5)$);
				\tkzDefPointsBy[translation=from A to A'](B,C,D){B'}{C'}{D'}
				\tkzDrawPolygon(A',B',B,C,D,D')
				\tkzDrawSegments(B',C' C',D' C,C')
				\tkzDrawSegments[dashed](A,B A,D A,A')
				\tkzDrawPoints[fill=black](A,B,D,C,A',B',C',D')
				\tkzLabelSegment[left](B',B){$y$}
				\tkzLabelSegment[below](B,C){$2x$}
				\tkzLabelSegment[right](A,B){$x$}
		\end{tikzpicture}}
		$S=2xy+2\cdot 2xy+2x^2=6xy+2x^2=\dfrac {500}{x}+2x^2$.\\
		Xét hàm số $S(x)=\dfrac {500}{x}+2x^2$ trên khoảng $(0;+\infty)$.\\
		$\Rightarrow S'(x)=-\dfrac {500}{x^2}+4x$.\\
		$S'(x)=0\Leftrightarrow -500+4x^3=0\Leftrightarrow x=5$.\\
		Bảng biến thiên
		\begin{center}
			\begin{tikzpicture}[scale=1, font=\footnotesize, line join=round, line cap=round, >=stealth]
				\tkzTabInit[nocadre=false,lgt=1.2,espcl=2, deltacl=0.5]
				{$x$/0.6,$S’(x)$/0.6,$S(x)$/1.5}
				{$0$,$5$,$+\infty$}
				\tkzTabLine{,-,z,+,}
				\tkzTabVar{+/$+\infty$,-/$150$,+/$+\infty$}
			\end{tikzpicture}
		\end{center}
		Do đó $\min S=150$ tại $x=5$. \\
		Khi đó, chi phí thuê nhân công là $150\cdot 700000=105$ triệu đồng.\\Vậy chi phí thuê nhân công ít nhất là $105$ triệu đồng.}
\end{ex}
\dongcham{13}
\begin{ex}%[2D1V3-6]
	Từ một tấm bìa hình chữ nhật có chiều rộng $30 \mathrm{~cm}$ và chiều dài $80 \mathrm{~cm}$ (Hình a), người ta cắt ở bốn góc bốn hình vuông có cạnh $x(\mathrm{~cm})$ với $5 \leq x \leq 10$ và gấp lại để tạo thành chiếc hộp có dạng hình hộp chữ nhật không nắp như Hình b. Tìm $x$ để thể tích chiếc hộp là lớn nhất (kết quả làm tròn đến hàng phần trăm).
	\begin{center}
		\begin{tikzpicture}[line join=round, line cap=round,scale=0.9]
			\coordinate (A) at (0,3);
			\coordinate (B) at (5,3);
			\coordinate (D) at (0,0);
			\coordinate (C) at ($(B)+(D)-(A)$);
			\draw(A)--(B)--(C)--(D)--cycle;
			\draw (0,0) rectangle (1,1) (A) rectangle (1,2) (B) rectangle (4,2) (4,1) rectangle (C);
			\draw[dashed] (1,1) rectangle (4,2);
			%	\foreach \i/\g in {A/90,B/90,C/-90,D/-90}{\draw[fill=black](\i) circle (1pt) ($(\i)+(\g:3mm)$) node[scale=1]{$\i$};}
			\draw (0,.5) node [left] {$x$};
			\draw (.5,0) node [below] {$x$};
			\draw (0,2.5) node [left] {$x$};
			\draw (0.5,3) node [above] {$x$};
			%%%%%%%%%
			\draw (4.5,0) node [below] {$x$};
			\draw (5,0.5) node [right] {$x$};
			\draw (5,2.5) node [right] {$x$};
			\draw (4.5,3) node [above] {$x$};
			%%%%%%%%
			\draw[<->] (-1,0)--(-1,3) node[above,midway,sloped] {$30$cm};
			\draw[<->] (0,-1)--(5,-1) node[above,midway] {$80$cm};
			\path (current bounding box.south) node[below, black]{a)}; %dưới
		\end{tikzpicture}
		\hspace*{1cm}
		\begin{tikzpicture}[scale=0.9, font=\footnotesize, line join=round, line cap=round, >=stealth]
			\def\bc{3} % cạnh BC
			\def\ba{1} % cạnh BA
			\def\h{1.5} % đường cao
			\def\gocnghieng{90} % góc nghiêng
			\def\gocB{35} % góc B của đáy
			\coordinate (B) at (0,0);
			\coordinate (A) at (\gocB:\ba);
			\coordinate (C) at (\bc,0);
			\coordinate (D) at ($(C)-(B)+(A)$);
			\coordinate (A') at ($(A)+(\gocnghieng:\h)$);
			\coordinate (B') at ($(B)-(A)+(A')$);
			\coordinate (C') at ($(C)-(A)+(A')$);
			\coordinate (D') at ($(D)-(A)+(A')$);
			\draw (B')--(B)--(C)--(D)--(D')--(A')--(B')--(C')--(D') (C)--(C');
			\draw[dashed] (A')--(A)--(D) (A)--(B);
			\path (current bounding box.south) node[below, black]{b)}; %dưới
		\end{tikzpicture}
	\end{center}
	\choice
	{\True $x=\dfrac{20}{3} \mathrm{~cm}$}
	{$x=\dfrac{20}{7} \mathrm{~cm}$}
	{$x=\dfrac{25}{3} \mathrm{~cm}$}
	{$x=\dfrac{25}{7} \mathrm{~cm}$}
	\loigiai{
		Thể tích chiếc hộp là $V(x)=x(30-2 x)(80-2 x)=2400 x-220 x^2+4 x^3$ với $5 \leq x \leq 10$.\\
		Ta có: $V'(x)=12 x^2-440 x+2400$;\\
		$V'(x)=0 \Leftrightarrow x=\dfrac{20}{3}$ hoặc $x=30$ (loại vì không thuộc $[5 ; 10]$);
		\begin{center}
			$V(5)=7000 ; V\left(\dfrac{20}{3}\right)=\dfrac{200000}{27} ; V(10)=6000$.
		\end{center}
		Do đó $\max \limits_{[5 ; 10]} V(x)=\dfrac{200000}{27}$ khi $x=\dfrac{20}{3}$.
		Vậy để thể tích chiếc hộp là lớn nhất thì $x=\dfrac{20}{3} \mathrm{~cm}$.}
\end{ex}
\dongcham{13}
\begin{ex}%[2D1K3]
	Một sợi dây có chiều dài là $6$ m, được chia thành $2$ phần. Phần thứ nhất được uốn thành hình tam giác đều, phần thứ hai uốn thành hình vuông. Hỏi độ dài của cạnh hình tam giác đều bằng bao nhiêu để tổng diện tích $2$ hình thu được là nhỏ nhất?
	\begin{center}
		\begin{tikzpicture}[scale=0.8,>=stealth]
			\draw(0,0)--(7,0);
			\draw (0,0)circle (1pt)(7,0) circle (1pt)(3,0) circle (1pt);
			\draw[->](1.5,-0.3)--(1.5,-0.7);
			\draw[->](5,-0.3)--(5,-0.7);
			\draw(4.5,-0.9)--(5.5,-0.9)--(5.5,-1.9)--(4.5,-1.9)--(4.5,-0.9);
			\draw(1.5,-0.9)--(2,-1.9)--(1,-1.9)--(1.5,-0.9);
		\end{tikzpicture}
	\end{center}
	\choice
	{$\dfrac{12}{4+\sqrt{3}}$ m}
	{$\dfrac{18\sqrt{3}}{4+\sqrt{3}}$ m}
	{$\dfrac{36\sqrt{3}}{4+\sqrt{3}}$ m}
	{\True $\dfrac{18}{9+4\sqrt{3}}$ m}
	\loigiai{
		Gọi độ dài cạnh hình tam giác đều là $x$ (m). Khi đó độ dài cạnh hình vuông là $\dfrac{6-3x}{4}$.\\
		Tổng diện tích khi đó là $S =\dfrac{\sqrt{3}}{4}x^2 + \left(\dfrac{{6 - 3x}}{4}\right)^2 =\dfrac{1}{16}\left[\left(9+4\sqrt{3}\right)x^2 - 36x + 36 \right)]$.\\
		Xét hàm số $f(x)=\left(9+4\sqrt{3}\right)x^2-36x+36, x\in(0;6)$.\\
		Ta có $f(x)$ là tam thức bậc $2$ có $-\dfrac{b}{2a}=\dfrac{18}{9+4\sqrt{3}} \in (0;6)$ và $a>0$.\\
		Suy ra $f(x)$ đạt giá trị nhỏ nhất tại
		$x=-\dfrac{b}{2a}\dfrac{18}{9+4\sqrt{3}}$.\\
		Vậy diện tích nhỏ nhất khi $x=\dfrac{18}{9+4\sqrt{3}}$ m.
	}
\end{ex}
\dongcham{14}
\begin{ex}
	Một doanh nghiệp tư nhân $A$ chuyên kinh doanh xe gắn máy các loại. Hiện nay doanh nghiệp đang tập trung vào chiến lược kinh doanh xe $X$ với chi phí mua vào một chiếc là 27 triệu đồng và bán ra với giá 31 triệu đồng. Với giá bán này, số lượng xe mà khách hàng đã mua trong một năm là 600 chiếc. Nhằm mục tiêu đẩy mạnh hơn nữa lượng tiêu thụ dòng xe đang bán chạy này, doanh nghiệp dự định giảm giá bán. Bộ phận nghiên cứu thị trường ước tính rằng nếu giảm 1 triệu đồng mỗi chiếc xe thì số lượng xe bán ra trong một năm sẽ tăng thêm 200 chiếc. Hỏi theo đó, giá bán mới là bao nhiêu thì lợi nhuận thu được cao nhất?
	\choice
	{$30$ triệu đồng}
	{\True $30,5$ triệu đồng}
	{$29,5$ triệu đồng}
	{$32$ triệu đồng}
	\loigiai{
		Gọi giá bán mới là $x$ (triệu đồng) với $x \in [27;31]$.\\
		Khi đó số xe bán ra là $600+(31-x) \cdot 200$.\\
		Lợi nhuận thu được là 
		\begin{eqnarray*}
			f(x) &=& [600+(31-x) \cdot 200](x-27)\\
			&=& (-200x+6800)(x-27)\\
			&=& -200x^2+12200x-183600\\
			&=& -200\left(x-\dfrac{61}{2}\right)^2+2450\\
			&\leq&2450.
		\end{eqnarray*}
		Vậy giá bán mới là $30,5$ triệu đồng thì lợi nhuận thu được là lớn nhất là $2\,450$ (triệu đông).
	}
\end{ex}
\dongcham{14}
\Closesolutionfile{ans}
\ind{PHẦN II.} \inden{Câu trắc nghiệm đúng sai. Trong mỗi ý a), b), c), d) ở mỗi câu, học sinh chọn đúng hoặc sai.}\\
\Opensolutionfile{ans}[ans/2D1-B2-d3-2]

\begin{ex}
	Người ta bơm xăng vào bình xăng của một xe ô tô. Biết rằng thể tích $V$ (lít) của lượng xăng trong bình xăng tính theo thời gian bơm xăng $t$ (phút) được cho bởi công thức $$V(t)=300(t^2-t^3)+4 \text{ với } 0\le t\le 0{,}5.$$
Gọi $V'(t)$ là tốc độ tăng thể tích tại thời điểm $t$ với $0\le t\le 0{,}5$.
\choiceTF
{Lượng xăng trong bình ban đầu là $1$ lít}
{\True Lượng xăng lớn nhất bơm vào bình xăng là $41{,}5$ lít}
{$V'(t)=300(2t-3t^2)+4$, với $0\le t\le 0{,}5$}
{\True Xăng chảy vào bình xăng vào thời điểm ở giây thứ $30$ có tốc độ tăng thể tích là lớn nhất}
	\loigiai{
		\begin{enumerate}[a)]
			\item Số xăng trong bình ban đầu là $V(0)=4$ lít.
			\item Lượng xăng lớn nhất bơm vào bình xăng là $V=V\left(\dfrac{1}{2}\right)=41{,}5$ lít.
			\item Xét hàm số $V(t)=300(t^2-t^3)+4 \text{ với } 0\le t\le 0{,}5.$\\
			Đạo hàm $V'(t)=300t(2-3t)$.\\
			\item Cho $V'(t)=0 \Leftrightarrow 300t(t-3t)=0 \Leftrightarrow \hoac{&t=0\in[0;0{,}5]\\&t=\dfrac{2}{3}\notin[0;0{,5}].}$\\
			Các giá trị $V(0)=4$, $V\left(\dfrac{1}{2}\right)=41{,}5$.\\
			Xăng chảy vào bình xăng vào thời điểm ở giây thứ $30$ có tốc độ tăng thể tích là lớn nhất.
		\end{enumerate}
	}
\end{ex}
\dongcham{20}
\begin{ex}
	Tại một xí nghiệp chuyên sản xuất vật liệu xây dựng, nếu trong một ngày xí nghiệp sản xuất $x$ (m$^3$) sản phẩm thì phải bỏ ra các khoản chi phí bao gồm: $4$ triệu đồng chi phí cố định; $0{,}2$ triệu đồng chi phí cho mỗi mét khối sản phẩm và $0{,}001 x^2$ triệu đồng chi phí bảo dưỡng máy móc. Biết rằng, mỗi ngày xí nghiệp sản xuất được tối đa $100$ m$^3$ sản phẩm. Goi $C(x)$ là tổng chi phí để xí nghiệp sản xuất $x$ (m$^3$) sản phẩm trong một ngày và $\overline{C}$ là chi phí trung bình  trên mỗi mét khối sản phẩm.
	\choiceTF
	{$C=0{,}2 x+0{,}001 x^2 \quad \text { với } 0 \leq x \leq 100$}
	{\True Tổng chi phí khi sản xuất 100 m$^3$ sản phẩm là 34 triệu đồng}
	{\True $\overline{C}=0{,}001 x+\dfrac{4}{x}+0{,}2 \quad\text { với } 0<x \leq 100$}
	{\True $\overline{C}$ có giá trị thấp nhất bằng 0,326 triệu đồng (\textit{kết quả làm tròn 3 chữ số thập phân})}
	\loigiai{
		\begin{enumerate}
			\item Tổng chi phí (triệu đồng) để xí nghiệp sản xuất $x$ (m$^3$) sản phẩm trong một ngày là
			$$
			C=C(x)=4+0{,}2 x+0{,}001 x^2 \text { với } 0 \leq x \leq 100.
			$$
			\item Thay $x=100$ vào hàm $C(x)$, ta được kết quả 34 (triệu đồng).
			\item Chi phí trung bình (triệu đồng) trên mỗi mét khối sản phẩm là
			$$
			\overline{C}=\overline{C}(x)=\dfrac{C(x)}{x}=\dfrac{4+0{,}2 x+0{,}001 x^2}{x}=0{,}001 x+\dfrac{4}{x}+0{,}2 \text { với } 0<x \leq 100.
			$$
			\item Ta có $\bar{C}'(x)=0{,}001-\dfrac{4}{x^2}$;
			$$
			\overline{C}'(x)=0 \Leftrightarrow 0{,}001-\dfrac{4}{x^2}=0 \Leftrightarrow x^2=4\,000 \Leftrightarrow x=20 \sqrt{10} \in(0 ; 100].
			$$
			
			Ta có $\overline{C}(20 \sqrt{10})=\dfrac{\sqrt{10}}{25}+\dfrac{1}{5} \approx 0,326$.\\
			Bảng biến thiên
			\begin{center}
				\begin{tikzpicture}
					\tikzset{double style/.append style={double distance=2pt}}
					\tkzTabInit[lgt=1.2, espcl=2]
					{$x$/0.6,$\overline{C'}(x)$/0.6,$\overline{C}(x)$/2.5}{$0$,$20\sqrt{10}$,$100$}
					\tkzTabLine{,-,0,+,}
					\tkzTabVar{+/$+\infty$,-/$\dfrac{\sqrt{10}}{25}+\dfrac{1}{5}$,+/$0{,}34$}
				\end{tikzpicture}
			\end{center}
			Từ bảng biến thiên, ta thấy chi phí trung bình thấp nhất là $\bar{C}(20 \sqrt{10}) \approx 0{,}326$ (triệu đồng/m$^3$ sản phẩm), đạt được khi $x=20 \sqrt{10} \approx 63$ (m$^3$).
		\end{enumerate}
	}
\end{ex}
\dongcham{20}
\begin{ex}
	Nhà máy $A$ chuyên sản xuất một loại sản phẩm cung cấp cho nhà máy $B$. Hai nhà máy thoả thuận rằng, hằng tháng $A$ cung cấp cho $B$ số lượng sản phẩm theo đơn đặt hàng của $B$ (tối đa $100$ tấn sản phẩm). Nếu số lượng đặt hàng là $x$ tấn sản phẩm thì giá bán cho mỗi tấn sản phẩm là $P(x)=45-0{,}001 x^2$ (triệu đồng). Chi phí để $A$ sản xuất $x$ tấn sản phẩm trong một tháng là $C(x)=100+30 x$ (triệu đồng) (gồm $100$ triệu đồng chi phí cố định và $30$ triệu đồng cho mỗi tấn sản phẩm).
	\choiceTF
	{\True Chi phí để  A sản xuất 10 tấn sảm phẩm trong một tháng là 400 triệu đồng}
	{Số tiền  A thu được khi bán 10 tấn sản phẩm cho B là 600 triệu đồng}
	{\True Lợi nhuận mà A thu được khi bán $x$ tấn sản phẩm ($0\le x \le 100)$ cho  B là $-0{,}001 x^3+15 x-100$}
	{\True A bán cho $B$ khoảng 70,7 tấn sản phẩm mỗi tháng thì thu được lợi nhuận lớn nhất}
	\loigiai{
		\begin{enumerate}[a)]
			\item Chi phí để  A sản xuất 10 tấn sảm phẩm trong một tháng là $C(10)=100+30\cdot 10=400$ (triệu)
			\item Số tiền mà $A$ thu được (gọi là doanh thu) từ việc bán $x$ tấn sản phẩm $(0 \leq x \leq 100)$ cho $B$ là
			$$
			R(x)=x \cdot P(x)=x\left(45-0{,}001 x^2\right)=45 x-0{,}001 x^3 \text { (triệu đồng). }
			$$
			Thay $x=10$, ta được $R(10)=449$ (triệu đồng).
			\item Lợi nhuận (triệu đồng) mà $A$ thu được là
			$$
			P(x)=R(x)-C(x)=x\left(45-0{,}001 x^2\right)-(100+30 x)=-0{,}001 x^3+15 x-100.
			$$
			\item Xét hàm số $P(x)=-0{,}001 x^3+15 x-100$ với $0 \leq x \leq 100$, ta có
			$$
			\begin{aligned}
				& P'(x)=-0{,}003 x^2+15; \\
				& P'(x)=0 \Leftrightarrow-0{,}003 x^2+15=0 \Leftrightarrow x^2=5\,000 \Leftrightarrow x=50 \sqrt{2} \in[0 ; 100].
			\end{aligned}
			$$
			
			Ta có $P(0)=-100$; $P(50 \sqrt{2})=500 \sqrt{2}-100 \approx 607$; $P(100)=400$.\\
			Bảng biến thiên
			\begin{center}
				\begin{tikzpicture}
					\tkzTabInit[lgt=1, espcl=4]
					{$x$/1,$y'$/0.6,$y$/3}{$0$,$50\sqrt{2}$,$100$}
					\tkzTabLine{,+,0,-,}
					\tkzTabVar{-/$100$,+/$500\sqrt{2}-100$,-/$400$}
				\end{tikzpicture}
			\end{center}
			
			Từ bảng biến thiên, ta có $\max \limits_{[0 ; 100]} P=P(50 \sqrt{2})=500 \sqrt{2}-100 \approx 607$.\\
			Vậy $A$ thu được lợi nhuận lớn nhất khi bán $50 \sqrt{2} \approx 70{,}7$ tấn sản phẩm cho $B$ mỗi tháng và lợi nhuận lớn nhất thu được khoảng $607$ triệu đồng.
		\end{enumerate}
	}
\end{ex}
\dongcham{20}
\Closesolutionfile{ans}
%%Bài 3. Tiệm cận
% \setcounter{section}{2}
\section{ĐƯỜNG TIỆM CẬN CỦA ĐỒ THỊ HÀM SỐ}
\subsection{LÝ THUYẾT CẦN NHỚ}
\subsubsection{Đường tiệm cận ngang (TCN):}
\begin{enumerate}[\iconMT]
	\item \indam{Định nghĩa:} Đường thẳng $y=m$ được gọi là một \inden{đường tiệm cận ngang} (hay \inden{tiệm cận ngang}) của đồ thị hàm số $y=f(x)$ nếu 
	$$\lim\limits_{x \rightarrow-\infty} f(x)=m \text{ hoặc }\lim\limits_{x \rightarrow+\infty} f(x)=m.$$
Đường thẳng $y=m$ là tiệm cận ngang của đồ thị hàm số $y=f(x)$ được minh hoạ như hình bên dưới\\
	\begin{tikzpicture}[scale=1,>=stealth, font=\footnotesize, line join=round, line cap=round]
		\def\xmin{-4} \def\xmax{2}
		\def\ymin{-0.5} \def\ymax{3}
		%\draw[color=gray!50,dashed] (\xmin,\ymin) grid (\xmax,\ymax);
		\draw[->] (\xmin,0)--(\xmax,0) node [below]{$x$};
		\draw[->] (0,\ymin)--(0,\ymax) node [left]{$y$};
		\fill (0,0) circle (1pt) node[shift={(-135:2.5mm)}]{$O$};
		\node at (current bounding box.south) [below=-2pt] {a) $\lim\limits_{x \rightarrow-\infty} f(x)=m$};
		\clip (\xmin+0.1,\ymin+0.1) rectangle (\xmax-0.1,\ymax-0.1);
		\draw[red,thick,smooth,samples=300,domain=\xmin:\xmax]
		(-4,0.9)..controls +(0:2) and +(180:0.5)
		..(-1.5,0.5)..controls +(0:0.5) and +(180:0.5)
		..(-0.3,1.4)..controls +(0:0.5) and +(135:1)
		..(1.8,0.3);
		\draw [blue](\xmin,1)--(\xmax,1);
		\path[blue] (-3,1)node[above]{$y=m$};
		\path[red] (0,1.3)node[above left]{$y=f(x)$};
		\fill (0,1) circle (1pt) node[shift={(-135:3mm)}]{$m$};
	\end{tikzpicture}\hspace*{.5cm}
	\begin{tikzpicture}[scale=1,>=stealth, font=\footnotesize, line join=round, line cap=round]
		\def\xmin{-1.5} \def\xmax{4}
		\def\ymin{-0.5} \def\ymax{3}
		%\draw[color=gray!50,dashed] (\xmin,\ymin) grid (\xmax,\ymax);
		\draw[->] (\xmin,0)--(\xmax,0) node [below]{$x$};
		\draw[->] (0,\ymin)--(0,\ymax) node [left]{$y$};
		\fill (0,0) circle (1pt) node[shift={(-135:2.5mm)}]{$O$};
		\node at (current bounding box.south) [below=-2pt] {b) $\lim\limits_{x \rightarrow+\infty} f(x)=m$};
		\clip (\xmin+0.1,\ymin+0.1) rectangle (\xmax-0.1,\ymax-0.1);
		\draw[red,thick,smooth,samples=300,domain=\xmin:\xmax]
		(-1,3)..controls +(-80:1) and +(170:1)
		..(0.5,1.1)..controls +(170:-1) and +(180:-0.5)
		..(3.9,0.8);
		\draw [blue](\xmin,0.7)--(\xmax,0.7);
		\path[blue] (4,0.7)node[below left]{$y=m$};
		\path[red] (0.5,1)node[above right]{$y=f(x)$};
		\fill (0,0.7) circle (1pt) node[shift={(-135:3mm)}]{$m$};
	\end{tikzpicture}
	\item \indam{Các bước tìm TCN:}
	\begin{boxdn}
		\begin{listEX}[1]
			\item [\ding{172}] Tính $\lim \limits_{x \to +\infty} f(x)$ và $\lim \limits_{x \to -\infty} f(x)$.
			\item [\ding{173}] Xem ở "vị trí" nào ra kết quả hữu hạn thì ta kết luận có tiệm cận ngang ở "vị trí" đó.
		\end{listEX}
	\end{boxdn}
\end{enumerate}
\subsubsection{Đường tiệm cận đứng (TCĐ)}
\begin{enumerate}[\iconMT]
	\item \indam{Định nghĩa:}	Đường thẳng $x=a$ được gọi là một \inden{đường tiệm cận đứng} (hay \inden{tiệm cận đứng}) của đồ thị hàm số $y=f(x)$ nếu ít nhất một trong các điều kiện sau thoả mãn:		
	$$
	\lim\limits_{x \rightarrow a^{-}} f(x)=+\infty,\,\, \lim\limits_{x \rightarrow a^{+}} f(x)=+\infty,\,\, \lim\limits_{x \rightarrow a^{-}} f(x)=-\infty,\,\, \lim\limits_{x \rightarrow a^{+}} f(x)=-\infty \text {. }
	$$
		Đường thẳng $x=a$ là tiệm cận đứng của đồ thị hàm số $y=f(x)$ được minh hoạ như hình bên dưới.\\
		\begin{center}
		\begin{tikzpicture}[scale=.7,>=stealth, font=\footnotesize, line join=round, line cap=round]
			%Hình a
			\def\xmin{-2.2} \def\xmax{3.5}
			\def\ymin{-2} \def\ymax{2} 
			%\draw[color=gray!50,dashed] (\xmin,\ymin) grid (\xmax,\ymax); 
			\draw[->] (\xmin,0)--(\xmax,0) node [below]{$x$};
			\draw[->] (0,\ymin)--(0,\ymax) node [left]{$y$};
			\fill (0,0) circle (1pt) node[shift={(-45:2.5mm)}]{$O$};
			\draw (2.1,\ymin)--(2.1,\ymax)node[below right]{$x=a$};
			\fill (2.1,0) circle (1pt) node[shift={(-45:3mm)}]{$a$};
			%\clip (\xmin+0.1,\ymin+0.1) rectangle (\xmax-0.1,\ymax-0.1);
			\draw[red] (-2,-1)..controls +(80:0.5) and +(0:-.5)..(-1,0.5)node[above]{$y=f(x)$}
			..controls +(0:0.5) and +(180:0.5)..(0.5,-1.5)
			..controls +(0:0.5) and +(87:-0.2)..(1.6,0)
			..controls +(87:-.2) and +(90:-0.2)
			..(2,1.85);
			\node at (current bounding box.south) [below=-2pt] {a) $\lim\limits_{x \rightarrow a^{-}} f(x)=+\infty$};
		\end{tikzpicture}
		\begin{tikzpicture}[scale=.7,>=stealth, font=\footnotesize, line join=round, line cap=round]
			%Hình b
			\def\xmin{-1.2} \def\xmax{4}
			\def\ymin{-2} \def\ymax{2} 
			%\draw[color=gray!50,dashed] (\xmin,\ymin) grid (\xmax,\ymax); 
			\draw[->] (\xmin,0)--(\xmax,0) node [below]{$x$};
			\draw[->] (0,\ymin)--(0,\ymax) node [left]{$y$};
			\fill (0,0) circle (1pt) node[shift={(-45:2.5mm)}]{$O$};
			\draw (1,\ymin)node[above right]{$x=a$}--(1,\ymax);
			\fill (1,0) circle (1pt) node[shift={(-135:3mm)}]{$a$};
			\path[red] (1.25,1)node[above right]{$y=f(x)$};
			%\clip (\xmin+0.1,\ymin+0.1) rectangle (\xmax-0.1,\ymax-0.1);
			\draw[red] (1.2,2)..controls +(80:0) and +(0:-1.4)..(2.5,-0.8)
			..controls +(0:0.1) and +(-80:-0.6)
			..(3.5,-1.5);
			\node at (current bounding box.south) [below=-2pt] {b) $\lim\limits_{x \rightarrow a^{+}} f(x)=+\infty$};		
		\end{tikzpicture}\\
		\begin{tikzpicture}[scale=.7,>=stealth, font=\footnotesize, line join=round, line cap=round]
			%Hình c
			\def\xmin{-2.2} \def\xmax{3.5}
			\def\ymin{-2} \def\ymax{2} 
			%\draw[color=gray!50,dashed] (\xmin,\ymin) grid (\xmax,\ymax); 
			\draw[->] (\xmin,0)--(\xmax,0) node [below]{$x$};
			\draw[->] (0,\ymin)--(0,\ymax) node [left]{$y$};
			\fill (0,0) circle (1pt) node[shift={(-45:2.5mm)}]{$O$};
			\draw (2,\ymin)--(2,\ymax)node[below right]{$x=a$};
			\fill (2,0) circle (1pt) node[shift={(-45:3mm)}]{$a$};
			\path[red] (-2.25,1.2)node[below right]{$y=f(x)$};
			%\clip (\xmin+0.1,\ymin+0.1) rectangle (\xmax-0.1,\ymax-0.1);
			\draw[red] (-2,1.4)..controls +(-10:-0.2) and +(-55:-.7)
			..(1.3,0.65)..controls +(-50:0.4) and +(-90:0)
			..(1.8,-2)
			;
			\node at (current bounding box.south) [below=-2pt] {c) $\lim\limits_{x \rightarrow a^{-}} f(x)=-\infty$};		
		\end{tikzpicture}
		\begin{tikzpicture}[scale=.7,>=stealth, font=\footnotesize, line join=round, line cap=round]
			%Hình d
			\def\xmin{-2.2} \def\xmax{3.5}
			\def\ymin{-2} \def\ymax{2} 
			%\draw[color=gray!50,dashed] (\xmin,\ymin) grid (\xmax,\ymax); 
			\draw[->] (\xmin,0)--(\xmax,0) node [below]{$x$};
			\draw[->] (0,\ymin)--(0,\ymax) node [left]{$y$};
			\fill (0,0) circle (1pt) node[shift={(-135:2.5mm)}]{$O$};
			\draw (.6,\ymin)--(.6,\ymax)node[below right]{$x=a$};
			\fill (.6,0) circle (1pt) node[shift={(-135:3mm)}]{$a$};
			%\clip (\xmin+0.1,\ymin+0.1) rectangle (\xmax-0.1,\ymax-0.1);
			\draw[red] (0.7,-2)..controls +(85:0.2) and +(180:0.2)
			..(1.2,-0.3)..controls +(0:0.2) and +(180:0.2)
			..(1.7,-0.6)..controls +(0:0.4) and +(90:0)
			..(2.5,2)
			;
			\node at (current bounding box.south) [below=-2pt] {d) $\lim\limits_{x \rightarrow a^{+}} f(x)=-\infty$};		
		\end{tikzpicture}
	\end{center}
	\item \indam{Các bước tìm TCĐ:}
	\begin{boxdn}
		\begin{listEX}[1]
			\item [\ding{172}] Tìm nghiệm của mẫu, giả sử nghiệm đó là $x=x_0$.
			\item [\ding{173}] Tính giới hạn một bên tại $x_0$. Nếu xảy ra $\lim \limits_{x \to x_0^{-}} f(x) =\infty \text{ hoặc} \lim \limits_{x \to x_0^{+}} f(x) =\infty$
			thì ta kết luận $x=x_0$ là đường tiệm cận đứng.
		\end{listEX}
	\end{boxdn}
\end{enumerate}
\subsubsection{Đường tiệm cận xiên}
\begin{enumerate}[\iconMT]
	\item \indam{Định nghĩa:} Đường thẳng $y=ax+b$, $a \neq 0$, được gọi là \inden{đường tiệm cận xiên} (hay \inden{tiệm cận xiên}) của đồ thị hàm số $y=f(x)$ nếu 
	$$\lim\limits_{x \rightarrow-\infty}[f(x)-(ax+b)]=0 \text{ hoặc }\lim\limits_{x \rightarrow+\infty}[f(x)-(ax+b)]=0.$$
	Đường thẳng $y=ax+b$ là tiệm cận xiên của đồ thị hàm số $y=f(x)$ được minh hoạ như hình bên dưới:\\	
		\begin{tikzpicture}[scale=1,>=stealth, font=\footnotesize, line join=round, line cap=round]
			\def\xmin{-4} \def\xmax{2.5}
			\def\ymin{-0.5} \def\ymax{3}
			%\draw[color=gray!50,dashed] (\xmin,\ymin) grid (\xmax,\ymax);
			\draw[->] (\xmin,0)--(\xmax,0) node [below]{$x$};
			\draw[->] (0,\ymin)--(0,\ymax) node [left]{$y$};
			\fill (0,0) circle (1pt) node[shift={(-135:2.5mm)}]{$O$};
			\node at (current bounding box.south) [below=-2pt] {a) $\lim\limits_{x \rightarrow-\infty}\left[f(x)-(ax+b)\right]=0$};
			\clip (\xmin+0.1,\ymin+0.1) rectangle (\xmax-0.1,\ymax-0.1);
			\draw[red,thick,smooth,samples=300,domain=\xmin:\xmax]
			(-3.8,-0.6)..controls +(34:0.5) and +(180:.75)
			..(-0.2,1.2)..controls +(0:0.75) and +(180:.75)
			..(1,0.3)..controls +(0:0.5) and +(80:0)
			..(2.2,1);
			\draw[blue,smooth,samples=300,domain=\xmin:\xmax] plot(\x,{2/3*(\x)+2});
			\path[blue] (-3,0)--(0,2)node[below,sloped,pos=1.3]{$y=ax+b$};
			\path[red] (0.5,1)node[above right]{$y=f(x)$};
		\end{tikzpicture}\hspace{.5cm}
		\begin{tikzpicture}[scale=1,>=stealth, font=\footnotesize, line join=round, line cap=round]
			\def\xmin{-3.5} \def\xmax{3}
			\def\ymin{-0.5} \def\ymax{3}
			%\draw[color=gray!50,dashed] (\xmin,\ymin) grid (\xmax,\ymax);
			\draw[->] (\xmin,0)--(\xmax,0) node [below]{$x$};
			\draw[->] (0,\ymin)--(0,\ymax) node [left]{$y$};
			\fill (0,0) circle (1pt) node[shift={(-135:2.5mm)}]{$O$};
			\node at (current bounding box.south) [below=-2pt] {a) $\lim\limits_{x \rightarrow+\infty}\left[f(x)-(ax+b)\right]=0$};
			\clip (\xmin+0.1,\ymin+0.1) rectangle (\xmax-0.1,\ymax-0.1);
			\draw[red,thick,smooth,samples=300,domain=\xmin:\xmax]
			(-3,0.8)..controls +(60:0.5) and +(180:.75)
			..(-1.5,2)..controls +(0:.5) and +(180:.75)
			..(0.5,1.3)..controls +(0:.75) and +(-160:.5)
			..(2.8,1.8);
			\draw[blue,smooth,samples=300,domain=\xmin:\xmax] plot(\x,{1/3*(\x)+0.75});
			\path[blue] (-3,-0.25)--(0,0.75)node[below,sloped,pos=1.6]{$y=ax+b$};
			\path[red] (-2.5,2)node[above right]{$y=f(x)$};
		\end{tikzpicture}
	\item \indam{Các bước tìm TCX y = ax + b:}
	Ta xác định hệ số của $a$ và $b$ trong 2 trường hợp sau:
	\begin{boxdn}
		\begin{listEX}[1]
			\item [\ding{172}] Tính $a=\lim\limits_{x \rightarrow+\infty} \dfrac{f(x)}{x}$, $b=\lim\limits_{x \rightarrow+\infty}[f(x)-ax]$.
			\item [\ding{173}] Tính $a=\lim\limits_{x \rightarrow-\infty} \dfrac{f(x)}{x}$, $b=\lim\limits_{x \rightarrow-\infty}[f(x)-ax]$.
		\end{listEX}
	\end{boxdn}
\end{enumerate}
\subsection{PHÂN LOẠI VÀ PHƯƠNG PHÁP GIẢI TOÁN}
\begin{dang}{Bài toán tìm tiệm cận đứng và tiệm cận ngang của đồ thị hàm số}
	Cho hàm số $y=f(x)$. Để tìm tiệm cận đứng và tiệm cận ngang, ta làm như sau:
	\begin{enumerate}[\iconCH]
		\item \indamm{Các bước tìm tiệm cận đứng:}
		\begin{listEX}[1]
			\item [\ding{172}] Tìm nghiệm của mẫu, giả sử nghiệm đó là $x=x_0$.
			\item [\ding{173}] Tính giới hạn một bên tại $x_0$. Nếu xảy ra $\lim \limits_{x \to x_0^{-}} f(x) =\infty \text{ hoặc} \lim \limits_{x \to x_0^{+}} f(x) =\infty$
			thì ta kết luận $x=x_0$ là đường tiệm cận đứng.
		\end{listEX}
		\item \indamm{Các bước tìm tiệm cận ngang:}
		\begin{listEX}[1]
			\item [\ding{172}] Tính $\lim \limits_{x \to +\infty} f(x)$ và $\lim \limits_{x \to -\infty} f(x)$.
			\item [\ding{173}] Xem ở "vị trí" nào ra kết quả hữu hạn thì ta kết luận có tiệm cận ngang ở "vị trí" đó.
		\end{listEX}
		\item \indamm{Lưu ý:} Đồ thị hàm số $y=\dfrac{ax+b}{cx+d}$ luôn có TCĐ $x=-\dfrac{d}{c}$ và TCN: $y=\dfrac{a}{c}$.
	\end{enumerate}
\end{dang}
\boxmini{BÀI TẬP TỰ LUẬN}
\begin{vd}
	Xác định tiệm cận đứng và tiệm cận ngang của đồ thị hàm số cho bởi công thức sau:
	\begin{enumEX}[a)]{4}
		\item $y=\dfrac{2x-1}{x+1}$;
		\item $y=\dfrac{2 x-3}{1-2 x}$;
		\item $y=\dfrac{x^2-5x+4}{x^2-1}$;
		\item $y=\dfrac{2x-1}{x^2-3x+2}$.
	\end{enumEX}
\loigiai{
\begin{enumerate}[a)]
	\item Xét $\lim\limits_{x \to -1^+} \dfrac{2x-1}{x+1}=-\infty$ (hoặc $\lim\limits_{x \to -1^-} \dfrac{2x-1}{x+1}=+\infty$) nên đường thẳng $x=-1$ là tiệm cận đứng.\\
	Xét $\lim\limits_{x \to \pm \infty } \dfrac{2x-1}{x+1}=2$ nên đường thẳng $y=2$ là tiệm cận ngang.
	\item Ta có
	\begin{itemize}
		\item $\lim\limits_{x \to \pm\infty} y=\lim\limits_{x \to \pm\infty} \dfrac{2x-3}{1-2x}=-1$ suy ra $y=-1$ là tiệm cận ngang.
		\item $\heva{& \lim\limits_{x \to \tfrac{1}{2}^+} \dfrac{2x-3}{1-2x}=+\infty \\ & \lim\limits_{x \to \tfrac{1}{2}^-} \dfrac{2x-3}{1-2x}=-\infty}$ suy ra $x=\dfrac{1}{2}$ là tiệm cận đứng.
	\end{itemize}
	\item Điều kiện xác định: $\heva{&x\neq-1\\ &x\neq1.}$
	\begin{itemize}
		\item $\lim\limits_{x\to\pm\infty}\dfrac{x^2-5x+4}{x^2-1}=1$
		\item $\lim\limits_{x\to(-1)^-}\dfrac{x^2-5x+4}{x^2-1}=+\infty$
		\item $\lim\limits_{x\to1}\dfrac{x^2-5x+4}{x^2-1}=-\dfrac{3}{2}$
	\end{itemize}
	Vậy đồ thị hàm số có một tiệm cận ngang $y=1$ và một tiệm cận đứng $x=-1$.
	\item Tập xác định $\mathscr{D}=\mathbb{R}\setminus\{1; 2\}$.\\
	Ta có\begin{itemize}
		\item $\lim\limits_{x\to\pm\infty}y=\lim\limits_{x\to\pm\infty}\dfrac{2x-1}{x^2-3x+2}=0$ nên $y=0$ là đường tiệm cận ngang.
		\item $\lim\limits_{x\to 1^-}y=\lim\limits_{x\to 1^-}\dfrac{2x-1}{x^2-3x+2}=\lim\limits_{x\to 1^-}\dfrac{2x-1}{(x-1)(x-2)}=+\infty$ và $\lim\limits_{x\to 1^+}y=-\infty$ nên $x=1$ là đường tiệm cận đứng.
		\item $\lim\limits_{x\to 2^-}y=\lim\limits_{x\to 2^-}\dfrac{2x-1}{x^2-3x+2}=\lim\limits_{x\to 2^-}\dfrac{2x-1}{(x-1)(x-2)}=-\infty$ và $\lim\limits_{x\to 2^+}y=2=+\infty$ nên $x=2$ là đường tiệm cận đứng.
	\end{itemize}
\end{enumerate}}
\end{vd}

\boxmini{BÀI TẬP TRẮC NGHIỆM}
\ind{PHẦN I.} \inden{Câu trắc nghiệm nhiều phương án lựa chọn. Mỗi câu hỏi học sinh chỉ chọn một phương án.}\\
\setcounter{ex}{0}
\Opensolutionfile{ans}[ans/2D1-B3-d1-1]
\begin{ex}
	Đường tiệm cận ngang của đồ thị hàm số $y=\dfrac{2x-4}{x+2}$ là
	\choice
	{\True $y=2$}
	{  $x=2$}
	{ $x=-2$}
	{$y=-2$}
	
	\loigiai{$\underset{x\to -\infty }{\mathop{\lim \limits_{n \to +\infty}}}\,\dfrac{2x-4}{x+2}=2$ và $\underset{x\to +\infty }{\mathop{\lim \limits_{n \to +\infty}}}\,\dfrac{2x-4}{x+2}=2$ nên hàm số có tiệm cận ngang là $y=2$.
	}
\end{ex}

\begin{ex}
	Tìm tiệm cận ngang của đồ thị hàm số $ y = \dfrac{2x + 1}{ x +1} $.
	\choice
	{ \True $  y = -2 $}
	{$  x = -2 $}
	{ $  y = 2 $}
	{$ x = 1 $}
	\loigiai{
		Ta có $ \displaystyle \lim_{ x \rightarrow \pm \infty } \dfrac{2x + 1}{-x + 1} = -2  $.	
	}		
\end{ex}

\begin{ex}
	Đường thẳng $y=3$ là tiệm cận ngang của đồ thị hàm số nào sau đây?
	\choice
	{$y=\dfrac{1-3x}{2+x}$}
	{$y=\dfrac{x^2+3x+2}{x-2}$}
	{\True $y=\dfrac{1+3x}{1+x}$}
	{$y=\dfrac{3x^2+2}{2-x}$}
	\loigiai{
		Ta có $\lim\limits_{x\to \pm \infty}\dfrac{1+3x}{1+x}=3$ nên $y=3$ là tiệm cận ngang của đồ thị hàm số $y=\dfrac{1+3x}{1+x}$.}
\end{ex}

\begin{ex}
	Hàm số nào có đồ thị nhận đường thẳng $x = 2$ làm đường tiệm cận đứng?
	\choice
	{$y=x-2+\dfrac{1}{x+1}$}
	{$y=\dfrac{1}{x+1}$}
	{$y=\dfrac{2}{x+2}$}
	{\True $y=\dfrac{5x}{2-x}$}
	\loigiai{ Xét hàm số $y=\dfrac{5x}{2-x}$\\
		Ta có $\lim\limits_{x\to 2^+}5x=10>0$; $\lim\limits_{x\to 2^+}(2-x)$ và $x-2<0$ khi $x>2$ suy ra $\lim\limits_{x\to 2^+}\dfrac{5x}{2-x}=-\infty$.\\
		Vậy đồ thị hàm số $y=\dfrac{5x}{2-x}$ nhận đường thẳng $x=2$ làm tiệm cận đứng.
	}
\end{ex}

\begin{ex}
	Đường tiệm cận đứng của đồ thị hàm số $y=\dfrac{3x+1}{x-2}$ là đường thẳng
	\choice
	{$x=-2$}
	{\True $x=2$}
	{$y=3$}
	{$y=-\dfrac{1}{2}$}
	\loigiai{Ta có: $\lim \limits_{x\to 2^+}{\dfrac{3x+1}{x-2}}=+\infty$.
	}
\end{ex}

\begin{ex}
	Đường tiệm cận đứng của đồ thị hàm số $y=\dfrac{x+1}{x^2+4x-5}$ có phương trình là
	\choice
	{$x=-1$}
	{$y=1;y=-5$}
	{\True $x=1;x=-5$}
	{$x=\pm 5$}
	\loigiai{
		Ta có $\mathop{\lim}\limits_{x\rightarrow 1^+}y=+\infty$, $\mathop{\lim}\limits_{x\rightarrow 1^-}y=-\infty$, $\mathop{\lim}\limits_{x\rightarrow 5^+}y=+\infty$, $\mathop{\lim}\limits_{x\rightarrow 5^-}y=-\infty$.\\
		Vậy đồ thị hàm số có hai đường tiệm cận đứng là $x=1$ và $x=-5$.}
\end{ex}

\begin{ex}
	Tìm số đường tiệm cận của đồ thị hàm số $ y = \dfrac{x^2 - 3x + 2}{x^2 - 4}. $
	\choice
	{$1$}
	{$ 0$}
	{\True $2$}
	{$3$}
	\loigiai
	{
		Tập xác định: $ \mathscr D = \mathbb{R} \backslash \{\pm2 \} $.\\
		Ta có $ \lim \limits_{x \to \pm  \infty} y = 1 \Rightarrow  $ đồ thị hàm số có 1 tiệm cận ngang là $ y = 1. $\\
		Ta lại có $\lim \limits_{x \to 2} y =  \lim \limits_{x \to 2} \dfrac{x-1}{x+2} = \dfrac{1}{4} $ và $\lim \limits_{x \to -2^+} y =  \lim \limits_{x \to -2^+} \dfrac{x-1}{x+2} = -\infty$ nên đồ thị hàm số có 1 tiệm cận đứng là $ x = -2. $\\
		Vậy đồ thị hàm số đã cho có 2 đường tiệm cận.
	}
\end{ex}

\begin{ex}
	Số đường tiệm cận của đồ thị hàm số $y=\dfrac{3}{x-2}$ là
	\choice
	{$1$}
	{\True $2$}
	{$0$}
	{$3$}
	\loigiai{
		Tiệm cận đứng $x=2$.\\
		Tiệm cận ngang $y=0$.
	}
\end{ex}

\begin{ex}
	Cho hàm số $y=f(x)$ có đồ thị là đường cong $(C)$ và các giới hạn $\lim\limits_{x\to 2^{+}}f(x)=1$, $\lim\limits_{x\to 2^{-}}f(x)=1$, $\lim\limits_{x\to +\infty}f(x)=2$, $\lim\limits_{x\to -\infty}f(x)=2$. Hỏi mệnh đề nào sau đây đúng?
	\choice
	{\True Đường thẳng $y=2$ là tiệm cận ngang của $(C)$}
	{Đường thẳng $y=1$ là tiệm cận ngang của $(C)$}
	{Đường thẳng $x=2$ là tiệm cận ngang của $(C)$}
	{Đường thẳng $x=2$ là tiệm cận đứng của $(C)$}
	\loigiai{
		Ta có $\lim\limits_{x\to +\infty}f(x)=2$, $\lim\limits_{x\to -\infty}f(x)=2\Rightarrow y=2$ là tiệm cận ngang của $(C)$.
	}
\end{ex}

\begin{ex}
	Số tiệm cận đứng của đồ thị hàm số $y=\dfrac{\sqrt{x+9}-3}{x^2+x}$ là
	\choice
	{$3$}
	{$2$}
	{$0$}
	{\True $1$}
	\loigiai{
		Tập xác định $\mathscr{D}=[-9;+\infty)\setminus \{-1;0\}$. \\
		Ta có $\left\{\begin{aligned}
			&\lim\limits_{x\to -1^+} \dfrac{\sqrt{x+9}-3}{x^2+x}=+\infty \\
			&\lim\limits_{x\to -1^-} \dfrac{\sqrt{x+9}-3}{x^2+x}=-\infty
		\end{aligned}\right. \Rightarrow x=-1$ là tiệm cận đứng. \\
		Ngoài ra $\lim\limits_{x\to 0} \dfrac{\sqrt{x+9}-3}{x^2+x}=\dfrac{1}{6}$ nên $x=0$ không thể là một tiệm cận được.}
\end{ex} 

\begin{ex}%[2D1B4]
	\immini{Cho hàm số $y=f(x)$ xác định trên $\mathbb{R}\setminus\left\{\pm1\right\}$ liên tục trên mỗi khoảng xác định và có bảng biến thiên như hình vẽ. Số đường tiệm cận của đồ thị hàm số là
	\choice
	{$1$}
	{$2$}
	{\True $3$}
	{$4$}}{\hspace{0.5cm}
\begin{tikzpicture}
	\tikzset{double style/.append style = {draw=\tkzTabDefaultWritingColor,double=\tkzTabDefaultBackgroundColor,double distance=2pt}}
	\tikzset{double style/.append style = {double distance=0.5pt}} 
	\tkzTabInit[nocadre=false,lgt=1,espcl=1.7]
	{$x$/.7,$y'$ /.7, $y$ /2.3}
	{$-\infty$ ,$-1$,$0$,$1$,$+\infty$}
	\tkzTabLine{,-,d,-,0,+,d,+,}
	\tkzTabVar {+/$-2$,-D+/$-\infty$/$+\infty$,-/$1$,+D-/$+\infty$/$-\infty$,+/$-2$}
\end{tikzpicture}}
	\loigiai{
		Dựa vào bảng biến thiên ta có:\\
		$\lim\limits_{x\to -1^\pm}f(x)=\pm\infty$. 
		$\lim\limits_{x\to 1^\pm}f(x)=\mp\infty$.\\
		Do đó $x=1$ và $x=-1$ là các đường tiệm cận đứng của đồ thị hàm số.\\
		Lại có $\lim\limits_{x\to \pm\infty}f(x)=-2$. Do đó $y=-2$ là tiệm cận ngang của đồ thị hàm số.\\
		Vậy đồ thị hàm số có $3$ đường tiệm cận.
	}
\end{ex}

\begin{ex}
	\immini{Cho hàm số $y=f(x)$ xác định trên $\mathbb{R}\backslash \left\{0\right\},$ liên tục trên mỗi khoảng xác định và có bảng biến thiên như hình bên. Chọn khẳng định đúng.
	\choice
	{Đồ thị hàm số có đúng một tiệm cận ngang}
	{Đồ thị hàm số có hai tiệm cận ngang}
	{\True Đồ thị hàm số có đúng một tiệm cận đứng}
	{Đồ thị hàm số không có tiệm đứng và tiệm cận ngang}}{
	\begin{tikzpicture}[>=stealth]
		\tikzset{double style/.append style = {draw=\tkzTabDefaultWritingColor,double=\tkzTabDefaultBackgroundColor,double distance=2pt}}
		\tkzTabInit[nocadre=false,lgt=1,espcl=2]{$x$/.6,$y'$/.7,$y$/2}{$-\infty$,$0$,$1$,$+\infty$}
		\tkzTabLine{,-, d ,+,z,-,} 
		\tkzTabVar{+/$+\infty$ / , -D- / $-1$ /$-\infty$,+/$2$,-/$-\infty$}
\end{tikzpicture}}
	\loigiai{
		Do $\lim\limits_{x \to +\infty} y=-\infty$ và $\lim\limits_{x \to -\infty} y=+\infty$  nên đồ thị hàm số không có tiệm cận ngang.\\
		Do $\lim\limits_{x \to 0^+} y=+\infty$ suy ra $x=0$ là tiệm cận đứng của đồ thị hàm số.
	} 
\end{ex}

\begin{ex}
	\immini{Cho hàm số $ y=f(x) $ có bảng biến thiên như hình bên. Hỏi đồ thị hàm số đã cho có bao nhiêu đường tiệm cận?
	\choice
	{\True $ 2 $}
	{$ 3 $}
	{$ 4 $}
	{$ 1 $}}{
\begin{tikzpicture}[yscale=.8,xscale=1.15,
	kxd/.pic={\draw[double] (90:.4)--(-90:.4);}]
	\begin{scope}[shift={(-.5,.5)}]
		\fill[pattern=north east lines,pattern color=black]
		(1,-1) rectangle +(1.45,-4);
		\draw
		(0,0) rectangle +(7,-5)
		(0,-1)--+(0:7) (0,-2)--+(0:7) (1,0)--+(-90:5);
	\end{scope}
	\path
	(0,0) node{$ x $}
	++ (0:1) node{$ -\infty $}
	++(0:1)node{$ -2 $}
	++(0:2)node{$ 0 $}
	++(0:2)node{$ +\infty $}
	(0,-1)node{$ y' $}
	++(0:2)pic{kxd}
	++(0:1)node{$ + $}
	++(0:1)pic{kxd}
	++(0:1)node{$ - $}
	(0,-3)node{$ y $}
	++(0:2)pic[yscale=3]{kxd}
	+(-90:1)node[below right](A){$ -\infty $}
	++(0:2) pic[yscale=3]{kxd}
	node[above right](C){$ 1 $}
	+(90:1)node[left](B){$ 2 $}
	++(0:2)node[below](D){$ 0 $};
	\draw[-stealth,black](A)--(B)
	;
	\draw[-stealth,black] (C)--(D);
\end{tikzpicture}}
	\loigiai{
		Dựa vào bảng biến thiên của hàm số, suy ra
		\begin{itemize}
			\item  $ \lim\limits_{x \to +\infty} f(x)=0 $, đồ thị hàm số có tiệm cận ngang là $ y=0 $.
			\item $ \lim\limits_{x \to (-2)^+} f(x)=-\infty $, đồ thị hàm số có tiệm cận đứng là $ x=-2 $.
			Vậy đồ thị hàm số đã cho có $ 2 $ đường tiệm cận.
		\end{itemize}
	}
\end{ex}

\Closesolutionfile{ans}

\ind{PHẦN II.} \inden{Câu trắc nghiệm đúng sai. Trong mỗi ý a), b), c), d) ở mỗi câu, học sinh chọn đúng hoặc sai.}\\
\Opensolutionfile{ans}[ans/2D1-B3-d1-2]
\begin{ex}
	Cho hàm số $y=f(x)$ có bảng biến thiên như hình bên. Xét tính đúng, sai của các khẳng định sau:
	\begin{center}
		\begin{tikzpicture}
			\tikzset{double style/.append style = {draw=\tkzTabDefaultWritingColor,double=\tkzTabDefaultBackgroundColor,double distance=2pt}}
			\tkzTab[nocadre=false,lgt=1.2,espcl=1.7,deltacl=0.6]
			{$x$/0.6, $y'$/0.6, $y$/2}
			{$-\infty$, $0$, $2$, $+\infty$}
			{,-,d,-,$0$,+,}
			{+/ $2$, -D+/ $-\infty$ / $+\infty$, -/ $2$,+/$+\infty$}
		\end{tikzpicture}
	\end{center}
	\choiceTF
	{\True $f(-5)<f(4)$}
	{Hàm số có giá trị nhỏ nhất bằng $2$}
	{\True Đồ thị hàm số có tiệm cận đứng $x=0$}
	{Đồ thị hàm số không có tiệm cận ngang}
	\loigiai{
		\begin{enumerate}[a)]
			\item Từ bảng biến thiên ta thấy $f(-5)<2$ và $f(4)>2$ nên $f(-5)<f(4)$.
			\item Do $\lim \limits_{x\to 0^-}y=-\infty$ nên hàm số không có giá trị nhỏ nhất.
			\item Do $\lim \limits_{x\to 0^-}y=-\infty$ nên đồ thị hàm số có tiệm cận đứng $x=0$.
			\item Do $\lim \limits_{x\to -\infty}y=2y$ nên đồ thị hàm số có tiệm cận ngang $y=2$.
		\end{enumerate}
}
\end{ex}


\begin{ex}
	Cho hàm số hàm số $y=\dfrac{-4x+5}{2x+3}$ có đồ thị $(C)$.
	Xét tính đúng sai của các khẳng định sau:
	\choiceTF
	{\True Hàm số không có cực trị}
	{Đồ thị hàm số có tiệm cận đứng $x=-3$}
	{Đồ thị hàm số có tiệm cận ngang $y=-2$}
	{\True Các đường tiệm cận của đồ thị tạo với hai trục toạ độ một hình chữ nhật có diện tích bằng $3$}
	\loigiai{
		Tập xác định $\mathscr D=\mathbb{R}\setminus \left \{-\dfrac{3}{2}\right \}$\\
		$\lim \limits_{x\to \left (-\frac{3}{2}\right )^+}y=+\infty; \ \lim \limits_{x\to \left (-\frac{3}{2}\right )^-}y=-\infty$ nên đồ thị hàm số có tiệm cận đứng $x=-\dfrac{3}{2}$\\
		$\lim \limits_{x\to -\infty}y=-2, \ \lim \limits_{x\to +\infty}y=-2$ nên đồ thị hàm số có một tiệm cận ngang là $y=-2$\\
		Diện tích hình chữ nhật cần tìm là $S=\left |-\dfrac{3}{2}\right |\cdot \left |-2\right |=3$
	}
\end{ex}

\Closesolutionfile{ans}

\begin{dang}{Bài toán tìm tiệm cận đứng và tiệm cận xiên của đồ thị hàm số}
	\begin{enumerate}[\iconCH]
		\item \indamm{Các bước tìm TCX y = ax + b:}
		Ta xác định hệ số của $a$ và $b$ trong 2 trường hợp sau:
			\begin{listEX}[1]
				\item [\ding{172}] Tính $a=\lim\limits_{x \rightarrow+\infty} \dfrac{f(x)}{x}$, $b=\lim\limits_{x \rightarrow+\infty}[f(x)-ax]$.
				\item [\ding{173}] Tính $a=\lim\limits_{x \rightarrow-\infty} \dfrac{f(x)}{x}$, $b=\lim\limits_{x \rightarrow-\infty}[f(x)-ax]$.
			\end{listEX}
		\item \indamm{Lưu ý:} 
		\begin{listEX}[1]
			\item [\ding{172}] Nếu $a=0$ thì tiệm cận xiên chính là tiệm cận ngang.
			\item [\ding{173}] Đối với hàm số phân thức $f(x)=\dfrac{ax^2+bx+c}{mx+n}$, ta có thể chia đa thức, biến đổi về dạng
			$$f(x)=a'x+b'+\dfrac{e}{mx+n}, \, \text{ với } e \ne0$$
			Suy ra $y=a'x+b'$ là đường tiệm cận xiên của đồ thị hàm số.
		\end{listEX}
	\end{enumerate}
	
\end{dang}
\boxmini{BÀI TẬP TỰ LUẬN}

\begin{vd}
	Tìm các tiệm cận đứng và tiệm cận xiên của đồ thị hàm số sau:
	\begin{listEX}[3]
		\item $y=\dfrac{x^{2}+2}{2x-4}$;
		\item $y=\dfrac{2x^{2}-3x-6}{x+2}$;
		\item $y=\dfrac{2x^{2}+9x+11}{2x+5}$.
	\end{listEX}
	\loigiai{
		\begin{listEX}
			\item Hàm số $y=f(x)=\dfrac{x^{2}+2}{2x-4}$ có tập xác định $\mathscr{D}=\mathbb{R} \setminus \left\lbrace 2\right\rbrace$.
			\begin{itemize}
				\item Ta có $\lim\limits_{x \rightarrow 2^{-}} \dfrac{x^{2}+2}{2x-4}=-\infty$; $\lim\limits_{x \rightarrow 2^{+}} \dfrac{x^{2}+2}{2x-4}=+\infty$.\\
				Suy ra đường thẳng $x=2$ là một tiệm cận đứng của đồ thị hàm số.
				\item Ta có $\begin{aligned}[t]
					a&=\lim\limits_{x \rightarrow+\infty} \dfrac{f(x)}{x}=\lim\limits_{x \rightarrow+\infty} \dfrac{x+\dfrac{2}{x}}{2x-4}=\dfrac{1}{2};\\
					b&=\lim\limits_{x \rightarrow+\infty}[f(x)-ax]=\lim\limits_{x \rightarrow+\infty}\left(\dfrac{x^{2}+2}{2x-4}-\dfrac{1}{2}x\right)=\lim\limits_{x \rightarrow+\infty} \dfrac{2x+2}{2x-4}=1.
				\end{aligned}$\\
				Ta cũng có $\lim\limits_{x \rightarrow-\infty} \dfrac{f(x)}{x}=\dfrac{1}{2}$; $\lim\limits_{x \rightarrow-\infty}[f(x)-\dfrac{1}{2}x]=1$.
				\\
				Do đó, đồ thị hàm số có tiệm cận xiên là đường thẳng $y=\dfrac{1}{2}x+1$.
			\end{itemize}	
			\item Hàm số $y=f(x)=\dfrac{2x^{2}-3x-6}{x+2}$ có tập xác định $\mathscr{D}=\mathbb{R} \setminus \left\lbrace -2\right\rbrace$.
			\begin{itemize}
				\item Ta có $\lim\limits_{x \rightarrow \left(-2\right)^{-}} \dfrac{2x^{2}-3x-6}{x+2}=-\infty$; $\lim\limits_{x \rightarrow \left(-2\right)^{+}} \dfrac{2x^{2}-3x-6}{x+2}=+\infty$.\\
				Suy ra đường thẳng $x=-2$ là một tiệm cận đứng của đồ thị hàm số.
				\item Ta có $\begin{aligned}[t]
					a&=\lim\limits_{x \rightarrow+\infty} \dfrac{f(x)}{x}=\lim\limits_{x \rightarrow+\infty} \dfrac{2x-3-\dfrac{6}{x}}{x+2}=2;\\
					b&=\lim\limits_{x \rightarrow+\infty}[f(x)-ax]=\lim\limits_{x \rightarrow+\infty}\left(\dfrac{2x^{2}-3x-6}{x+2}-2x\right)=\lim\limits_{x \rightarrow+\infty} \dfrac{-7x-6}{x+2}=-7.
				\end{aligned}$\\
				Ta cũng có $\lim\limits_{x \rightarrow-\infty} \dfrac{f(x)}{x}=2$; $\lim\limits_{x \rightarrow-\infty}[f(x)-2x]=-7$.\\
				Do đó, đồ thị hàm số có tiệm cận xiên là đường thẳng $y=2x-7$.
			\end{itemize}
			\item Hàm số $y=f(x)=\dfrac{2x^{2}+9x+11}{2x+5}$ có tập xác định $\mathscr{D}=\mathbb{R} \setminus \left\lbrace -\dfrac{5}{2}\right\rbrace$.
			\begin{itemize}
				\item 
				Ta có $\lim\limits_{x \rightarrow \left(-\tfrac{5}{2}\right)^{-}} \dfrac{2x^{2}+9x+11}{2x+5}=-\infty$; $\lim\limits_{x \rightarrow \left(-\tfrac{5}{2}\right)^{+}} \dfrac{2x^{2}+9x+11}{2x+5}=+\infty$.\\
				Suy ra đường thẳng $x=-\dfrac{5}{2}$ là một tiệm cận đứng của đồ thị hàm số.
				\item Ta có $\begin{aligned}[t]
					a&=\lim\limits_{x \rightarrow+\infty} \dfrac{f(x)}{x}=\lim\limits_{x \rightarrow+\infty} \dfrac{2x+9+\dfrac{11}{x}}{2x+5}=1;\\
					b&=\lim\limits_{x \rightarrow+\infty}[f(x)-ax]=\lim\limits_{x \rightarrow+\infty}\left(\dfrac{2x^{2}+9x+11}{2x+5}-x\right)=\lim\limits_{x \rightarrow+\infty} \dfrac{4x+11}{2x+5}=2.
				\end{aligned}$\\
				Ta cũng có $\lim\limits_{x \rightarrow-\infty} \dfrac{f(x)}{x}=1$; $\lim\limits_{x \rightarrow-\infty}[f(x)-x]=2$.\\
				Do đó, đồ thị hàm số có tiệm cận xiên là đường thẳng $y=x+2$.
			\end{itemize}
		\end{listEX}	
	}
\end{vd}

\boxmini{BÀI TẬP TRẮC NGHIỆM}
\ind{PHẦN I.} \inden{Câu trắc nghiệm nhiều phương án lựa chọn. Mỗi câu hỏi học sinh chỉ chọn một phương án.}\\
\setcounter{ex}{0}
\Opensolutionfile{ans}[ans/2D1-B3-d2-1]
\begin{ex}
	Đường tiệm cận xiên của đồ thị hàm số $y=f(x)=2x-1-\dfrac{1}{x+1}$ có phương trình là
	\choice
	{$y=x+1$}
	{\True $y=2x-1$}
	{$y=x-1$}
	{$y=2x+1$}
	\loigiai{
		Do $\lim\limits_{x\to +\infty}[f(x)-(2x-1)]=\lim\limits_{x\to +\infty}\dfrac{-1}{x+1}=0$ nên đường thẳng $y=2x-1$
		là tiệm cận xiên của đồ thị hàm số đã cho.}
\end{ex}

\begin{ex}
	Đường tiệm cận xiên của đồ thị hàm số $y=f(x)=x+3+\dfrac{1}{2x+1}$ có phương trình là
	\choice
	{$y=2x+1$}
	{$y=x-3$}
	{\True $y=x+3$}
	{$y=2x-1$}
	\loigiai{
		Do $\lim\limits_{x\to \pm\infty}[f(x)-(x+3)]=\lim\limits_{x\to \pm\infty}\dfrac{1}{2x+1}=0$ nên đường thẳng $y=x+3$
		là tiệm cận xiên của đồ thị hàm số đã cho.}
\end{ex}

\begin{ex}
	Tìm tiệm cận xiên của đồ thị hàm số $y=f(x)=\dfrac{x^2+3x}{x-2}$.
	\choice
	{$y=2x-5$}
	{$y=x-2$}
	{\True $y=x+5$}
	{$y=x-5$}
	\loigiai{
	Ta có
	\begin{itemize}
		\item $a=\lim\limits_{x\to +\infty}\dfrac{f(x)}{x}=\lim\limits_{x\to +\infty}\dfrac{x^2+3x}{x(x-2)}=1$
		\item và $b=\lim\limits_{x\to +\infty}[f(x)-x]=\lim\limits_{x\to +\infty}\dfrac{5x}{x-2}=5$.
	\end{itemize}
	Vậy đường thẳng $y=x+5$ là tiệm cận xiên của đồ thị hàm số đã cho (khi $x \to +\infty$).\\
	Tương tự, do $\lim\limits_{x\to -\infty}\dfrac{f(x)}{x}=1$ và $\lim\limits_{x\to -\infty}[f(x)-x]=5$ nên đường thẳng $y=x+5$ cũng là tiệm cận xiên của đồ thị hàm số đã cho (khi $x \to -\infty$).}
\end{ex}

\begin{ex}%[2D1H4-1]
	Tiệm cận xiên của đồ thị hàm số $y=\dfrac{x^2+2x-2}{x+2}$ là
	\choice
	{$y=-2$}
	{$y=1$}
	{$y=x+2$}
	{\True $y=x$}
	\loigiai{
		Ta có $y=\dfrac{x^2+2x-2}{x+2}=\dfrac{x(x+2)-2}{x+2}=x-\dfrac{2}{x+2}$.\\
		$\underset{x\to +\infty}{\mathop{\lim}} [ y-x ] =\underset{x\to +\infty}{\mathop{\lim}}\dfrac{-2}{x+2}=0$ và $\underset{x\to -\infty}{\mathop{\lim}} [ y-x ] =\underset{x\to -\infty}{\mathop{\lim}}\dfrac{-2}{x+2}=0$. \\ 
		Vậy đồ thị hàm số có tiệm cận xiên là đường thẳng $y=x$. 
	}
\end{ex}

\begin{ex}
	Tìm tiệm cận xiên của đồ thị hàm số $f(x)=\dfrac{x^{2}-3 x+1}{x-2}$.
	\choice
	{$y=x+1$}
	{$y=-3x+1$}
	{$y=x-2$}
	{\True $y=x-1$}
	\loigiai{
		Tập xác định: $\mathscr{D}=\mathbb{R} \setminus\{2\}$.
		\\
		Ta có $\begin{aligned}[t]
			a&=\lim\limits_{x \rightarrow+\infty} \dfrac{f(x)}{x}=\lim\limits_{x \rightarrow+\infty} \dfrac{x^{2}-3 x+1}{x^{2}-2 x}=1;\\
			b&=\lim\limits_{x \rightarrow+\infty}[f(x)-a x]=\lim\limits_{x \rightarrow+\infty}\left(\dfrac{x^{2}-3 x+1}{x-2}-x\right)=\lim\limits_{x \rightarrow+\infty} \dfrac{-x+1}{x-2}=-1.
		\end{aligned}$\\
		Ta cũng có $\lim\limits_{x \rightarrow-\infty} \dfrac{f(x)}{x}=1$; $\lim\limits_{x \rightarrow-\infty}[f(x)-x]=-1$.
		\\
		Do đó, đồ thị hàm số có tiệm cận xiên là đường thẳng $y=x-1$.
	}
\end{ex}

\begin{ex}
	Đường tiệm cận xiên của đồ thị hàm số $y=\dfrac{2x^{2}-3x}{x+5}$ đi qua điểm nào sau đây?
	\choice
	{$(5;3)$}
	{$(-4;-5)$}
	{\True $(6;-1)$}
	{$(2;-10)$}
	\loigiai{
		Tập xác định: $\mathscr{D}=\mathbb{R} \setminus\{-5\}$.
		\\
		Ta có $\begin{aligned}[t]
			a&=\lim\limits_{x \rightarrow+\infty} \dfrac{f(x)}{x}=\lim\limits_{x \rightarrow+\infty} \dfrac{2x^{2}-3x}{x^2+5x}=2;\\
			b&=\lim\limits_{x \rightarrow+\infty}[f(x)-ax]=\lim\limits_{x \rightarrow+\infty}\left(\dfrac{2x^{2}-3x}{x+5}-2x\right)=\lim\limits_{x \rightarrow+\infty} \dfrac{-13x}{x+5}=-13.
		\end{aligned}$\\
		Ta cũng có $\lim\limits_{x \rightarrow-\infty} \dfrac{f(x)}{x}=2$; $\lim\limits_{x \rightarrow-\infty}[f(x)-x]=-13$.
		\\
		Do đó, đồ thị hàm số có tiệm cận xiên là đường thẳng $y=2x-13$.	Đường thẳng này qua $(6;-1)$.
	}
\end{ex}

\begin{ex}
	Giao điểm của đường tiệm cận đứng và đường tiệm cận xiên của đồ thị hàm số $y=\dfrac{2x^2-3x+2}{x-1}$ là
	\choice
	{$(1;2)$}
	{\True $(1;1)$}
	{$(1;-1)$}
	{$(1;0)$}
	\loigiai{
	Ta viết lại $y=\dfrac{2x^2-3x+2}{x-1}=2x-1+\dfrac{1}{x-1}$. Suy ra
	\begin{itemize}
		\item [$\bullet$] Tiệm cận đứng $x=1$;
		\item [$\bullet$] Tiệm cận ngang $y=2x-1$.
	\end{itemize}
Xét hệ $\heva{&x=1\\&y=2x-1} \Leftrightarrow \heva{&x=1\\&y=1}$}
\end{ex}

\Closesolutionfile{ans}

\ind{PHẦN II.} \inden{Câu trắc nghiệm đúng sai. Trong mỗi ý a), b), c), d) ở mỗi câu, học sinh chọn đúng hoặc sai.}\\
\Opensolutionfile{ans}[ans/2D1-B3-d2-2]

\begin{ex}
	\immini{Cho hàm số $y=f(x)=\dfrac{ax^2+bx+c}{dx+e}$ có đồ thị như hình bên. 
		\choiceTF
		{Tập xác định của hàm số là $\mathbb{R}$}
		{\True Hàm số có hai điểm cực trị}
		{Đồ thị hàm số có đường tiệm cận đứng là $x=0$}
		{Đồ thị hàm số có đường tiệm cận xiên là $y=x+1$}
	}{
		\begin{tikzpicture}[scale=.4, font=\footnotesize, line join=round, line cap=round, >=stealth]
			\draw[->] (-6,0)--(0,0) node[below left]{$O$}--(6,0) node[below]{$x$};
			\draw[->] (0,-8) --(0,6) node[right]{$y$};
			\clip (-6,-8) rectangle (6,6);
			\draw[violet] [domain=-0.8:6, samples=100,thick] %
			plot (\x, {\x-1+ (2)/((\x)+1)});
			\draw[violet] [domain=-6:-1.3, samples=100,thick] %
			plot (\x, {\x-1+ (2)/((\x)+1)});
			\draw[fill] (0,0) circle (1pt) (-1,0) circle (1pt) (-1,-2) circle (1pt) (1,0) circle (1pt)node[above] {$1$} (0,-1) circle (1pt)node[right] {$-1$};
			\draw[domain=-8:7, samples=100] %
			plot (\x, {\x-1});
			\draw (-1,-8)--(-1,0)node[above left] {$-1$}--(-1,6);
	\end{tikzpicture}}
\end{ex}

\begin{ex}
	\immini{Cho đồ thị của hàm số $y=f(x)=\dfrac{2 x^2}{x^2-1}$. Xét tính đúng sai của các khẳng định sau:
	\choiceTF
	{Đồ thị hàm số có 3 điểm cực trị}
	{$\lim \limits_{x \rightarrow-\infty} f(x)=2$ ; $\lim \limits_{x \rightarrow 1^{-}} f(x)=-\infty$}
	{Đồ thị hàm số có 3 đường tiệm cận đứng $x=-1$, $x=0$, $x=1$} 
	{Đồ thị hàm số có hai đường tiệm cận ngang $y=2$ và $y=0$} 
	}{
\begin{tikzpicture}[scale=.5,>=stealth, font=\footnotesize, line join=round, line cap=round]
	\def\xmin{-6} \def\xmax{6}
	\def\ymin{-5} \def\ymax{7}
	%\draw[color=gray!50,dashed] (\xmin,\ymin) grid (\xmax,\ymax);
	\draw[->] (\xmin,0)--(\xmax,0) node [below]{$x$};
	\draw[->] (0,\ymin)--(0,\ymax) node [left]{$y$};
	\fill (0,0) circle (1pt) node[shift={(135:2.5mm)}]{$O$};
	\clip (\xmin+0.1,\ymin+0.1) rectangle (\xmax-0.1,\ymax-0.1);
	\draw[thick,smooth,violet,samples=300,domain=(\xmin:-1.01)] plot(\x,{(2*(\x)^2)/((\x)^2-1)});		
	\draw[thick,smooth,violet,samples=300,domain=(-0.9:0.9)] plot(\x,{(2*(\x)^2)/((\x)^2-1)});
	\draw[thick,smooth,violet,samples=300,domain=(1.1:\xmax)] plot(\x,{(2*(\x)^2)/((\x)^2-1)});
	\draw[blue] (\xmin,2)--(\xmax,2);	
	\draw[blue] (-1,\ymin)--(-1,\ymax);	
	\draw[blue] (1,\ymin)--(1,\ymax);		
	\foreach \x in {\xmin,...,\xmax}
	\draw (\x,-0.1)--(\x,0.1);
	\foreach \y in {\ymin,...,\ymax}
	\draw (-0.1,\y)--(0.1,\y);
	\node at (-5,2)[below]{$y=2$};
	\node at (-1.2,-4)[left]{$x=-1$};
	\node at (1.2,-4)[right]{$x=1$};
	%\node at (-1,0)[shift={(-135:2.5mm)}]{$-1$};
	%\node at (.5,0)[shift={(-75:2.5mm)}]{$\dfrac{1}{2}$};
	%\node at (0,-1)[left]{$-1$};
	%\node at (0,2)[shift={(135:2.5mm)}]{$2$};		
\end{tikzpicture}}
	\loigiai{
	\begin{enumerate}[a)]
		\item Đồ thị hàm số có một điểm cực trị $(0;0)$.
		\item Theo hình vẽ thì $\lim \limits_{x \rightarrow-\infty} f(x)=2$; $\lim \limits_{x \rightarrow 1^{-}} f(x)=-\infty$.
		\item Đồ thị hàm số có 2 đường tiệm cận đứng $x= \pm 1$.
		\item Đồ thị hàm số có 1 đường tiệm cận ngang $y= 2$.
\end{enumerate}}
\end{ex}

\Closesolutionfile{ans}
\begin{dang}{Bài toán về đường tiệm cận có chứa tham số}
\end{dang}
\boxmini{BÀI TẬP TỰ LUẬN}
\begin{vd}%[2D1Y4-2]
	Tìm tham số $m$ để đồ thị hàm số 
	\begin{tasks}
		\task $y=\dfrac{3x-1}{x-m}$ có đường tiệm cận đứng là $x=5$.
		\task $y=\dfrac{(m+1)x-5m}{2x-m}$ có tiệm cận ngang là đường thẳng $y=1$.
	\end{tasks}
	\loigiai{
		\begin{enumerate}[a)]
			\item Điều kiện để đồ thị hàm số có tiệm cận đứng là $-3m+1\neq 0\Leftrightarrow m\neq \dfrac{1}{3}$.\\
			Đồ thị hàm số có tiệm cận đứng $x=m$.\\
			Theo đề bài ta có $m=5$ (thoả mãn).
			\item Điều kiện để đồ thị hàm số có tiệm cận ngang là $-m(m+1)+10m\neq 0$.\\
			Tiệm cận ngang là $y=\dfrac{a}{c}=\dfrac{m+1}{2}.$\\
			Theo đề bài ta có $\dfrac{m+1}{2}=1\Leftrightarrow m+1=2\Leftrightarrow m=1$ (thoả mãn).
		\end{enumerate}
	}
\end{vd}

\begin{vd}%[2D1K4-2]
	Tìm $m$ để đồ thị hàm số 
	\begin{tasks}
		\task $y=\dfrac{x-2}{x^2-mx+1}$ có hai đường tiệm cận đứng.
		\task $y=\dfrac{2x^2-3x+m}{x-m}$ có đường tiệm cận xiên.
	\end{tasks}
	\loigiai{
		\begin{enumerate}[a)]
			\item Đồ thị hàm số có hai tiệm cận đứng $\Leftrightarrow$ phương trình $g(x)=x^2-mx+1=0$ có hai nghiệm phân biệt khác $2$.
			$$\Leftrightarrow\heva{&a=1\neq 0 \, (\textrm{LĐ})\\ & \Delta =m^2-4>0\\&g(2)=2^2-2m+1\neq 0} \Leftrightarrow \heva{&\hoac{&m<-2\\&m>2}\\& m\neq \dfrac{5}{2}}.$$
			Vậy $m\in\left(-\infty; -2\right) \cup \left(2; +\infty\right) \setminus \left\{\dfrac{5}{2}\right\}$.
			\item 	Đồ thị hàm số có đường tiệm cận xiên khi và chỉ khi phương trình $g(x)=2x^2-3x+m=0$ không có nghiệm $x=m$. Tức là:
			$$g(m)\neq 0 \Leftrightarrow 2m^2-2m\neq 0 \Leftrightarrow \heva{&m\neq 0\\ &n\neq 1}.$$
			Vậy $m\in\mathbb{R}\setminus\left\{0; 1\right\}$ là các giá trị cần tìm.
		\end{enumerate}
	}	
\end{vd}


\boxmini{BÀI TẬP TRẮC NGHIỆM}
\ind{PHẦN I.} \inden{Câu trắc nghiệm nhiều phương án lựa chọn. Mỗi câu hỏi học sinh chỉ chọn một phương án.}\\
\setcounter{ex}{0}
\Opensolutionfile{ans}[ans/2D1-B3-d3-1]

\begin{ex}
	Tìm tất cả các giá trị của $m$ để đồ thị hàm số $y=\dfrac{mx+2}{x-5}$ có đường tiệm cận ngang đi qua điểm $A(1; 3)$.
	\choice
	{$m=-3$}
	{$m=1$}
	{$m=-1$}
	{\True $m=3$}
	\loigiai{
		Tiệm cận ngang $y=m$ đi qua điểm $A(1; 3)$ nên $m=3$.
	}
\end{ex} 

\begin{ex}
	Tìm tham số thực $m$ để đồ thị hàm số $y=\dfrac{mx+3}{x-m}$ có tiệm cận đứng là đường $x=1$, tiệm cận ngang là đường $y=1$.
	\choice
	{\True $m=1$}
	{$m=2$}
	{$m=-1$}
	{$m=3$}
	\loigiai{
		\begin{itemize}
			\item Điều kiện để đồ thị hàm số có tiệm cận là $-m^2-3\ne 0 \ \forall m$
			\item Phương trình đường tiệm cận đứng là $x=m$ nên có $m=1$
			\item Phương trình đường tiệm cận ngang là $y=m$ nên có $m=1$\\
			Vậy $m=1$.
		\end{itemize}
	}
\end{ex}

\begin{ex}
	Biết rằng hai đường tiệm cận của đồ thị hàm số $y=\dfrac{2x+1}{x-m}$ (với $m$ là tham số) tạo với hai trục tọa độ một hình chữ nhật có diện tích bằng $2$. Giá trị của $m$ là
	\choice
	{$m=\pm 2$}
	{$m=-1$}
	{$m=2$}
	{\True $m=\pm 1$}
	\loigiai{
		Điều kiện $ m\neq -\dfrac{1}{2} $.\\
		Ta có $\lim\limits_{x\to+\infty}\dfrac{2x+1}{x-m}=2$ và $\lim\limits_{x\to-\infty}\dfrac{2x+1}{x-m}=2\Rightarrow y=2$ là tiệm cận ngang của đồ thị hàm số.\\
		\begin{itemize}
			\item Xét $ m>-\dfrac{1}{2} $, ta có $\lim\limits_{x\to m^{+}}\dfrac{2x+1}{x-m}=+\infty$, $\lim\limits_{x\to m^{-}}\dfrac{2x+1}{x-m}=-\infty\Rightarrow x=m$ là tiệm cận đứng của đồ thị hàm số.
			\item Xét $ m<-\dfrac{1}{2} $, ta có $\lim\limits_{x\to m^{+}}\dfrac{2x+1}{x-m}=-\infty$, $\lim\limits_{x\to m^{-}}\dfrac{2x+1}{x-m}=+\infty\Rightarrow x=m$ là tiệm cận đứng của đồ thị hàm số.
		\end{itemize}
		Diện tích hình chữ nhật là $|2m|=2\Rightarrow m=\pm 1$ (thỏa mãn).
	}
\end{ex} 


\begin{ex}
	Tìm giá trị của $m$ để đồ thị hàm số $y=\dfrac{2x^2-5x+m}{x-m}$ có tiệm cận đứng.
	\choice
	{$\hoac{&m=0\\&m=2}$}
	{$m\ne 0$}
	{$m\ne 2$}
	{\True $\heva{&m\ne 0\\&m\ne 2}$}
	\loigiai{
		Ta có $x-m=0\Leftrightarrow x=m$. \\
		Để đồ thị hàm số có tiệm cận đứng thì $2(m)^2-5(m)+m\ne 0\Leftrightarrow 2m^2-4m\ne 0\Leftrightarrow \heva{&m\ne 0\\&m\ne 2}$.
	}
\end{ex} 

\begin{ex}%[2D1Y4-1]
	Tìm tất cả các giá trị thực của tham số $m$ để đồ thị hàm số $y=\dfrac{x-4}{x^2-mx+4}$ có hai đường tiệm cận đứng?
	\choice
	{$m \in \left (-\infty;-4\right] \cup \left [4;+\infty \right )$}
	{$m \ne 5$}
	{\True $m \in \left (-\infty;-4\right) \cup \left (4;+\infty \right ) \setminus \left \{5\right \}$}
	{$m \in \left (-\infty;-4\right) \cup \left (4;+\infty \right )$}
	\loigiai{
		Đồ thị hàm số có hai tiệm cận đứng khi phương trình $x^2-mx+4=0$ có hai nghiệm phân biệt khác $4\Leftrightarrow \heva{&m^2-16>0\\&16-4m+4\ne 0}\Leftrightarrow m \in \left (-\infty;-4\right) \cup \left (4;+\infty \right ) \setminus \left \{5\right \}$ 
	}
\end{ex}

\begin{ex}%[2D1B4-2]
	Cho hàm số $ y = \dfrac{2x^2-3x+m}{x-m} $ có đồ thị $ (C) $. Tìm tất cả các giá trị của tham số $ m $ để $ (C) $ không có tiệm cận đứng.
	\choice
	{\True $ m = 0 $ hoặc $ m = 1 $}
	{$ m = 2 $}
	{$ m = 1 $}
	{$ m = 0 $}
	\loigiai{
		Đồ thị $ (C) $ không có tiệm cận đứng khi $ m $ là nghiệm của $ 2x^2-3x+m $
		\begin{align*}
			\Leftrightarrow 2m^2 - 3m + m = 0 \Leftrightarrow \hoac{& m = 0 \\& m = 1.}
		\end{align*}
	}
\end{ex}

\begin{ex}
	Tìm tất cả các giá trị của tham số thực $m$ để đồ thị hàm số $y=\dfrac{x-2}{x^2-mx+1}$ có đúng $3$ đường tiệm cận.
	\choice
	{\True $\left[\begin{aligned}
			&\left\{\begin{aligned}
				&m>2 \\
				&m\ne \dfrac{5}{2}
			\end{aligned}\right. \\
			&m<-2
		\end{aligned}\right. $}
	{$\left[\begin{aligned}
			&m>2 \\
			&\left\{\begin{aligned}
				&m<-2 \\
				&m\ne -\dfrac{5}{2}
			\end{aligned}\right.
		\end{aligned}\right. $}
	{$\left[\begin{aligned}
			&m>2 \\
			&m<-2
		\end{aligned}\right. $}
	{$-2<m<2$}
	\loigiai{
		ĐKXĐ : $x^2-mx+1\ne 0$ \\
		Ta có $\displaystyle\lim \limits_{x\to \pm \infty}y=\displaystyle\lim \limits_{x\to \pm \infty}\dfrac{x-2}{x^2-mx+1}=0$ $ \Rightarrow y=0$ là tiệm cận ngang. \\
		Do đó đồ thị hàm số $y=\dfrac{x-2}{x^2-mx+1}$ có đúng $3$ đường tiệm cận khi và chỉ khi phương trình $x^2-mx+1=0$ có hai nghiệm phân biệt khác $2$. \\
		$ \Leftrightarrow \left\{\begin{aligned}
			& \Delta =m^2-4>0 \\
			&2^2-2m+1\ne 0
		\end{aligned}\right. \Leftrightarrow \left\{\begin{aligned}
			&\left[\begin{aligned}
				&m>2 \\
				&m<-2
			\end{aligned}\right. \\
			&m\ne \dfrac{5}{2}
		\end{aligned}\right. $. }
\end{ex} 

\begin{ex}
	Cho hàm số $y=\dfrac{ax+1}{bx-2}$, xác định $a$ và $b$ để đồ thị của hàm số trên nhận đường thẳng $x=1$ làm tiệm cận đứng và đường thẳng $y=\dfrac{1}{2}$ làm tiệm cận ngang.
	\choice
	{$ \heva{&a=-1\\&b=-2} $}
	{\True $ \heva{&a=1\\&b=2} $}
	{$ \heva{&a=2\\&b=2} $}
	{$ \heva{&a=2\\&b=-2} $}
	\loigiai{Yêu cầu bài toán $\Leftrightarrow\heva{&\dfrac{a}{b}=\dfrac{1}{2}\\&\dfrac{2}{b}=1}\Leftrightarrow\heva{&b=2\\&a=1}$.}
\end{ex} 


\begin{ex}%[2D1Y4-1]
	Cho hàm số $y=\dfrac{mx+1}{x+3n+1}$. Đồ thị hàm số nhận trục hoành và trục tung làm tiệm cận ngang và tiệm cận đứng. Tính $m+n$.
	\choice
	{\True $m+n=-\dfrac{1}{3}$}
	{$m+n=\dfrac{1}{3}$}
	{$m+n=\dfrac{2}{3}$}
	{$m+n=0$}
	\loigiai{
		\begin{itemize}
			\item Điều kiện để đồ thị hàm số có tiệm cận là $m\left (3n+1\right )\ne 0$
			\item Phương trình đường tiệm cận đứng là $x=-3n-1$ nên có $n=-\dfrac{1}{3}$
			\item Phương trình đường tiệm cận ngang là $y=m$ nên có $m=0$\\
			Vậy $m+n=-\dfrac{1}{3}$.
		\end{itemize}
	}
\end{ex}

\begin{ex}%[2D1K4-2]
	Đồ thị hàm số $y=\dfrac{(4a-b)x^2+ax+1}{x^2+ax+b-12}$ nhận trục hoành và trục tung làm hai tiệm cận. Tính giá trị của $a+b$.
	\choice
	{$a+b=10$}
	{$a+b=12$}
	{$a+b=18$}
	{\True $a+b=15$}
	\loigiai{
		Tiệm cận đứng $x=0 \Rightarrow 0^2+a.0+b-12=0\Leftrightarrow b=12.$\\
		Tiệm cận ngang $y=0 \Rightarrow 4a-b=0\Leftrightarrow 4a-12=0 \Leftrightarrow a=3.$\\
		\textbf{Kết luận:} $a+b=15.$
	}
\end{ex}

\Closesolutionfile{ans}

\ind{PHẦN II.} \inden{Câu trắc nghiệm đúng sai. Trong mỗi ý a), b), c), d) ở mỗi câu, học sinh chọn đúng hoặc sai.}\\
\Opensolutionfile{ans}[ans/2D1-B3-d3-2]
\begin{ex}%[2D1B4-2]
	Cho hàm số $y=\dfrac{mx^2+6x-2}{x+2}$, với $m$ là tham số.
	\choiceTF
	{\True Tập xác định của hàm số là $\mathbb{R}\backslash\{-2\}$}
	{Đồ thị hàm số có tiệm cận ngang khi $m>0$}
	{Đồ thị hàm số có tiệm cận đứng khi $m\ne 0$}
	{\True Tập hợp tất cả giá trị của $m$ đề đồ thị có hai đường tiệm cận là $\mathbb{R}\setminus\left\{\dfrac{7}{2}\right\}$}
	\loigiai
	{
		\begin{enumerate}[a)]
			\item Điều kiện $x+2 \ne 0 \Leftrightarrow x \ne -2$. Vậy Tập xác định là $\mathbb{R}\backslash\{-2\}$
			\item Đồ thị hàm số có tiệm cận ngang khi hệ số của $x^2$ trên tử số phải bằng 0. Suy ra $m=0$.
			\item Đồ thị hàm số có tiệm cận đứng khi $x=-2$ không là nghiệm của tam thức $g(x)=mx^2+6x-2$. Suy ra
			$$g(-2)\ne 0 \Leftrightarrow m \ne \dfrac{7}{2}$$
			\item Đồ thị hàm số chắc chắn có 1 tiệm cận xiên (hoặc ngang). Suy ra, để đồ thị có hai đường tiệm cận thì nó phải có 1 tiệm cận đứng. Điều này tương đương với $m \ne \dfrac{7}{2}$.
		\end{enumerate}
	}
\end{ex}

\Closesolutionfile{ans}

% \section[TIỆM CẬN]{ĐƯỜNG TIỆM CẬN CỦA ĐỒ THỊ HÀM SỐ}
\subsection{TÓM TẮT LÝ THUYẾT}
\subsubsection{Đường tiệm cận ngang}%[Lý Văn Hoàng, Dự án TeX hóa Lý Thuyết]
\begin{dn}
    Đường thẳng $y=m$ là đường tiệm cận ngang (hay tiệm cận ngang)
    của đồ thị hàm số $y=f(x)$ nếu ít nhất một trong các điều kiện sau được thỏa mãn:\\
    \centerline{$\lim\limits_{x\to+\infty}f(x)=m, \quad\lim\limits_{x\to-\infty}f(x)=m $.}
\end{dn}
\begin{center}
    \begin{tikzpicture}[scale=1,>=stealth, font=\footnotesize, line join=round, line cap=round]
        \def\xmin{-4} \def\xmax{2}
        \def\ymin{-0.5} \def\ymax{3}
        %\draw[color=gray!50,dashed] (\xmin,\ymin) grid (\xmax,\ymax);
        \draw[->] (\xmin,0)--(\xmax,0) node [below]{$x$};
        \draw[->] (0,\ymin)--(0,\ymax) node [left]{$y$};
        \fill (0,0) circle (1pt) node[shift={(-135:2.5mm)}]{$O$};
        \node at (current bounding box.south) [below=-2pt] {a) $\lim\limits_{x \rightarrow-\infty} f(x)=m$};
        \clip (\xmin+0.1,\ymin+0.1) rectangle (\xmax-0.1,\ymax-0.1);
        \draw[red,thick,smooth,samples=300,domain=\xmin:\xmax]
        (-4,0.9)..controls +(0:2) and +(180:0.5)
        ..(-1.5,0.5)..controls +(0:0.5) and +(180:0.5)
        ..(-0.3,1.4)..controls +(0:0.5) and +(135:1)
        ..(1.8,0.3);
        \draw [blue](\xmin,1)--(\xmax,1);
        \path[blue] (-3,1)node[above]{$y=m$};
        \path[red] (0,1.3)node[above left]{$y=f(x)$};
        \fill (0,1) circle (1pt) node[shift={(-135:3mm)}]{$m$};
    \end{tikzpicture}\hspace*{1cm}
    \begin{tikzpicture}[scale=1,>=stealth, font=\footnotesize, line join=round, line cap=round]
        \def\xmin{-1.5} \def\xmax{4}
        \def\ymin{-0.5} \def\ymax{3}
        %\draw[color=gray!50,dashed] (\xmin,\ymin) grid (\xmax,\ymax);
        \draw[->] (\xmin,0)--(\xmax,0) node [below]{$x$};
        \draw[->] (0,\ymin)--(0,\ymax) node [left]{$y$};
        \fill (0,0) circle (1pt) node[shift={(-135:2.5mm)}]{$O$};
        \node at (current bounding box.south) [below=-2pt] {b) $\lim\limits_{x \rightarrow+\infty} f(x)=m$};
        \clip (\xmin+0.1,\ymin+0.1) rectangle (\xmax-0.1,\ymax-0.1);
        \draw[red,thick,smooth,samples=300,domain=\xmin:\xmax]
        (-1,3)..controls +(-80:1) and +(170:1)
        ..(0.5,1.1)..controls +(170:-1) and +(180:-0.5)
        ..(3.9,0.8);
        \draw [blue](\xmin,0.7)--(\xmax,0.7);
        \path[blue] (4,0.7)node[below left]{$y=m$};
        \path[red] (0.5,1)node[above right]{$y=f(x)$};
        \fill (0,0.7) circle (1pt) node[shift={(-135:3mm)}]{$m$};
    \end{tikzpicture}
\end{center}
\begin{nx} \quad
    \begin{itemize}
        \item Để tìm tiệm cận ngang của đồ thị hàm số ta cần tính giới hạn của hàm số tại vô cực $(\pm \infty)$.
        \item Tìm giới hạn ở vô cực của hàm $y=\dfrac{P(x)}{Q(x)}$ với $P(x)$, $Q(x)$ là các đa thức không căn.
        \begin{enumerate}[i)]
            \item Bậc của $P(x)$ nhỏ hơn bậc của $Q(x) \Rightarrow \lim\limits_{x\to \pm\infty} y =0 \Rightarrow$ Tiệm cận ngang $Ox \colon y=0$.
            \item Bậc của $P(x)$ bằng bậc của $Q(x) \Rightarrow \lim\limits_{x\to \pm\infty} y = \dfrac{\text{Hệ số x bậc cao của P(x) }}{\text{Hệ số x bậc cao của Q(x)}} = \alpha$ (một số cụ thể) $\Rightarrow y= \alpha$ là tiệm cận ngang.
            \item Bậc của $P(x)$ lớn hơn bậc của $Q(x) \Rightarrow \lim\limits_{x\to \pm\infty} y = \pm \infty \Rightarrow$ Không có tiệm cận ngang.
        \end{enumerate}
    \end{itemize}
\end{nx}
\subsubsection{Đường tiệm cận đứng}%[Lý Văn Hoàng, Dự án TeX hóa Lý Thuyết]
\begin{dn}
    Đường thẳng $x=a$ được gọi là đường tiệm cận đứng (hay tiệm cận đứng) của đồ thị hàm số $y=f(x)$ nếu ít nhất một trong các điều kiện sau được thỏa mãn:
    $$ \lim\limits_{x \to a^{+} } f(x)= + \infty; \lim\limits_{x \to a^{+} } f(x)= - \infty ;$$ $$ \lim\limits_{x \to a^{-} } f(x)= + \infty; \lim\limits_{x \to a^{-}} f(x)= - \infty.$$
\end{dn}
\begin{center}
    \begin{tikzpicture}[scale=.7,>=stealth, font=\footnotesize, line join=round, line cap=round]
        %Hình a
        \def\xmin{-2.2} \def\xmax{3.5}
        \def\ymin{-2} \def\ymax{2}
        %\draw[color=gray!50,dashed] (\xmin,\ymin) grid (\xmax,\ymax);
        \draw[->] (\xmin,0)--(\xmax,0) node [below]{$x$};
        \draw[->] (0,\ymin)--(0,\ymax) node [left]{$y$};
        \fill (0,0) circle (1pt) node[shift={(-45:2.5mm)}]{$O$};
        \draw (2.1,\ymin)--(2.1,\ymax)node[below right]{$x=a$};
        \fill (2.1,0) circle (1pt) node[shift={(-45:3mm)}]{$a$};
        %\clip (\xmin+0.1,\ymin+0.1) rectangle (\xmax-0.1,\ymax-0.1);
        \draw[red] (-2,-1)..controls +(80:0.5) and +(0:-.5)..(-1,0.5)node[above]{$y=f(x)$}
        ..controls +(0:0.5) and +(180:0.5)..(0.5,-1.5)
        ..controls +(0:0.5) and +(87:-0.2)..(1.6,0)
        ..controls +(87:-.2) and +(90:-0.2)
        ..(2,1.85);
        \node at (current bounding box.south) [below=-2pt] {a) $\lim\limits_{x \rightarrow a^{-}} f(x)=+\infty$};
    \end{tikzpicture}
    \begin{tikzpicture}[scale=.7,>=stealth, font=\footnotesize, line join=round, line cap=round]
        %Hình b
        \def\xmin{-1.2} \def\xmax{4}
        \def\ymin{-2} \def\ymax{2}
        %\draw[color=gray!50,dashed] (\xmin,\ymin) grid (\xmax,\ymax);
        \draw[->] (\xmin,0)--(\xmax,0) node [below]{$x$};
        \draw[->] (0,\ymin)--(0,\ymax) node [left]{$y$};
        \fill (0,0) circle (1pt) node[shift={(-45:2.5mm)}]{$O$};
        \draw (1,\ymin)node[above right]{$x=a$}--(1,\ymax);
        \fill (1,0) circle (1pt) node[shift={(-135:3mm)}]{$a$};
        \path[red] (1.25,1)node[above right]{$y=f(x)$};
        %\clip (\xmin+0.1,\ymin+0.1) rectangle (\xmax-0.1,\ymax-0.1);
        \draw[red] (1.2,2)..controls +(80:0) and +(0:-1.4)..(2.5,-0.8)
        ..controls +(0:0.1) and +(-80:-0.6)
        ..(3.5,-1.5);
        \node at (current bounding box.south) [below=-2pt] {b) $\lim\limits_{x \rightarrow a^{+}} f(x)=+\infty$};
    \end{tikzpicture}
    \begin{tikzpicture}[scale=.7,>=stealth, font=\footnotesize, line join=round, line cap=round]
        %Hình c
        \def\xmin{-2.2} \def\xmax{3.5}
        \def\ymin{-2} \def\ymax{2}
        %\draw[color=gray!50,dashed] (\xmin,\ymin) grid (\xmax,\ymax);
        \draw[->] (\xmin,0)--(\xmax,0) node [below]{$x$};
        \draw[->] (0,\ymin)--(0,\ymax) node [left]{$y$};
        \fill (0,0) circle (1pt) node[shift={(-45:2.5mm)}]{$O$};
        \draw (2,\ymin)--(2,\ymax)node[below right]{$x=a$};
        \fill (2,0) circle (1pt) node[shift={(-45:3mm)}]{$a$};
        \path[red] (-2.25,1.2)node[below right]{$y=f(x)$};
        %\clip (\xmin+0.1,\ymin+0.1) rectangle (\xmax-0.1,\ymax-0.1);
        \draw[red] (-2,1.4)..controls +(-10:-0.2) and +(-55:-.7)
        ..(1.3,0.65)..controls +(-50:0.4) and +(-90:0)
        ..(1.8,-2)
        ;
        \node at (current bounding box.south) [below=-2pt] {c) $\lim\limits_{x \rightarrow a^{-}} f(x)=-\infty$};
    \end{tikzpicture}
    \begin{tikzpicture}[scale=.7,>=stealth, font=\footnotesize, line join=round, line cap=round]
        %Hình d
        \def\xmin{-2.2} \def\xmax{3.5}
        \def\ymin{-2} \def\ymax{2}
        %\draw[color=gray!50,dashed] (\xmin,\ymin) grid (\xmax,\ymax);
        \draw[->] (\xmin,0)--(\xmax,0) node [below]{$x$};
        \draw[->] (0,\ymin)--(0,\ymax) node [left]{$y$};
        \fill (0,0) circle (1pt) node[shift={(-135:2.5mm)}]{$O$};
        \draw (.6,\ymin)--(.6,\ymax)node[below right]{$x=a$};
        \fill (.6,0) circle (1pt) node[shift={(-135:3mm)}]{$a$};
        %\clip (\xmin+0.1,\ymin+0.1) rectangle (\xmax-0.1,\ymax-0.1);
        \draw[red] (0.7,-2)..controls +(85:0.2) and +(180:0.2)
        ..(1.2,-0.3)..controls +(0:0.2) and +(180:0.2)
        ..(1.7,-0.6)..controls +(0:0.4) and +(90:0)
        ..(2.5,2)
        ;
        \node at (current bounding box.south) [below=-2pt] {d) $\lim\limits_{x \rightarrow a^{+}} f(x)=-\infty$};
    \end{tikzpicture}
\end{center}
\immini{\textbf{Đặc biệt} Đối với hàm số $y= \dfrac{ax+b}{cx+d}$ có tiệm cận ngang $y=\dfrac{a}{c}$ và tiệm cận đứng $x= -\dfrac{d}{c}$. Tâm đối xứng là giao điểm của hai đường tiệm cận.
}{
    \begin{tikzpicture}[>=stealth, line join=round, line cap=round, font=\scriptsize,x=.8cm,y=.7cm]
        \begin{scope}[scale=.7]
            \def\a{1}
            \def\b{1}
            \def\c{1}
            \def\d{-2}
            \def\mau{red}
            \draw[->] (-5,0) -- (8,0) node[below] {$x$};
            \draw[->] (0,-5) -- (0,5) node[left] {$y$};
            \draw (0,0)node[below left]{$O$};
            \draw[dashed,blue] ({-\d/\c},-5)--({-\d/\c},5) (-5,{\a/\c})--(8,{\a/\c}); % Vẽ TCĐ và TCN
            \clip (-5,-5)rectangle(8,5);
            \draw ({-\d/\c},0)node[below right]{$2$};
            \draw (0,{\a/\c})node[above left]{$1$};
            \draw (7,-4)node[above left]{{\normalsize }$y=\dfrac{x+1}{x-2}$};
            \pgfmathsetmacro{\can}{-(\d)/(\c)}
            \draw[\mau,samples=150,smooth,domain=-5:{\can-.1}] plot(\x,{(\a*\x+(\b))/(\c*\x+(\d))}); % Vẽ nhánh bên trái TCĐ
            \draw[\mau,samples=150,smooth,domain={\can+.1}:8] plot(\x,{(\a*\x+(\b))/(\c*\x+(\d))}); % Vẽ nhánh bên phải TCĐ
        \end{scope}
\end{tikzpicture}}
\begin{nx} \quad
    \begin{itemize}
        \item Để tìm tiệm cận đứng của đồ thị hàm số, ta cần tính giới hạn một bên của $x_0$, với $x_0$ thường là điều kiện của hàm số (hay tại $x_0$ thì hàm số không xác định).
        \item Kỹ năng sử dụng máy tính (tham khảo):
        \begin{enumerate}[i)]
            \item Tính $\lim\limits_{x \to x_0^+} f(x)$ thì nhập $f(x)$ và CALC $x= x_0 + 10^{-9}$.
            \item Tính $\lim\limits_{x \to x_0^-} f(x)$ thì nhập $f(x)$ và CALC $x= x_0 - 10^{-9}$.
        \end{enumerate}
    \end{itemize}
\end{nx}
\subsubsection{Đường tiệm cận xiên}
\begin{dn}
    Đường thẳng $y=ax+b$ được gọi là đường tiệm cận xiên của đồ thị $(C):y=f(x)$ nếu \[\lim \limits_{x \to -\infty} \left[f(x)-(ax+b)\right]=0 \text{ hoặc }\lim \limits_{x \to +\infty} \left[f(x)-(ax+b)\right]=0\]
\end{dn}
\begin{center}
    \begin{tikzpicture}[scale=0.8,>=stealth, font=\footnotesize, line join=round, line cap=round]
        \def\xmin{-4} \def\xmax{2.5}
        \def\ymin{-0.5} \def\ymax{3}
        %\draw[color=gray!50,dashed] (\xmin,\ymin) grid (\xmax,\ymax);
        \draw[->] (\xmin,0)--(\xmax,0) node [below]{$x$};
        \draw[->] (0,\ymin)--(0,\ymax) node [left]{$y$};
        \fill (0,0) circle (1pt) node[shift={(-135:2.5mm)}]{$O$};
        \node at (current bounding box.south) [below=-2pt] {a) $\lim\limits_{x \rightarrow-\infty}\left[f(x)-(ax+b)\right]=0$};
        \clip (\xmin+0.1,\ymin+0.1) rectangle (\xmax-0.1,\ymax-0.1);
        \draw[red,thick,smooth,samples=300,domain=\xmin:\xmax]
        (-3.8,-0.6)..controls +(34:0.5) and +(180:.75)
        ..(-0.2,1.2)..controls +(0:0.75) and +(180:.75)
        ..(1,0.3)..controls +(0:0.5) and +(80:0)
        ..(2.2,1);
        \draw[blue,smooth,samples=300,domain=\xmin:\xmax] plot(\x,{2/3*(\x)+2});
        \path[blue] (-3,0)--(0,2)node[below,sloped,pos=1.3]{$y=ax+b$};
        \path[red] (0.5,1)node[above right]{$y=f(x)$};
    \end{tikzpicture}\hspace{1cm}
    \begin{tikzpicture}[scale=0.8,>=stealth, font=\footnotesize, line join=round, line cap=round]
        \def\xmin{-3.5} \def\xmax{3}
        \def\ymin{-0.5} \def\ymax{3}
        %\draw[color=gray!50,dashed] (\xmin,\ymin) grid (\xmax,\ymax);
        \draw[->] (\xmin,0)--(\xmax,0) node [below]{$x$};
        \draw[->] (0,\ymin)--(0,\ymax) node [left]{$y$};
        \fill (0,0) circle (1pt) node[shift={(-135:2.5mm)}]{$O$};
        \node at (current bounding box.south) [below=-2pt] {a) $\lim\limits_{x \rightarrow+\infty}\left[f(x)-(ax+b)\right]=0$};
        \clip (\xmin+0.1,\ymin+0.1) rectangle (\xmax-0.1,\ymax-0.1);
        \draw[red,thick,smooth,samples=300,domain=\xmin:\xmax]
        (-3,0.8)..controls +(60:0.5) and +(180:.75)
        ..(-1.5,2)..controls +(0:.5) and +(180:.75)
        ..(0.5,1.3)..controls +(0:.75) and +(-160:.5)
        ..(2.8,1.8);
        \draw[blue,smooth,samples=300,domain=\xmin:\xmax] plot(\x,{1/3*(\x)+0.75});
        \path[blue] (-3,-0.25)--(0,0.75)node[below,sloped,pos=1.6]{$y=ax+b$};
        \path[red] (-2.5,2)node[above right]{$y=f(x)$};
    \end{tikzpicture}
\end{center}
\begin{nx}\quad
    \begin{itemize}
        \item Để tìm TCX của đồ thị hàm số $y=f(x)$ ta giải hệ phương trình: $\heva{& \lim \limits_{x \to +\infty} \dfrac{f(x)}{x}=a \ne 0 \\ & \lim \limits_{x \to +\infty} \left[f(x)-ax\right]=b}$ hoặc $\heva{& \lim \limits_{x \to -\infty} \dfrac{f(x)}{x}=a \ne 0 \\ & \lim \limits_{x \to -\infty} \left[f(x)-ax\right]=b}$, khi đó tiệm cận xiên của đồ thị hàm số $y=f(x)$ là đường thẳng $y=ax+b$.
        \item Đồ thị hàm số $y=\dfrac{mx^2+nx+p}{cx+d}=ax+b+\dfrac{r}{cx+d}$ có đường tiệm cận xiên là đường thẳng $y=ax+b$.
        \item Hàm phân thức có bậc tử bé hơn hoặc bằng bậc mẫu, bậc tử lớn hơn bậc mẫu 2 bậc thì không có tiệm cận xiên.
    \end{itemize}
\end{nx}
%\subsection{CÁC DẠNG TOÁN}
\begin{dang}{Tìm các đường tiệm cận qua biểu thức hàm số, bảng biến thiên}
\end{dang}
\begin{vd} Tìm các đường tiệm cận đứng, ngang, xiên (nếu có) của đồ thị hàm số sau
    \begin{listEX}[3]
        \item $y=\dfrac{2x+1}{x+1}$.
        \item $y=\dfrac{x}{2x-1}$.
        \item $y=\dfrac{3-x}{x+1}$.
        \item $y=2x+1+\dfrac{1}{x-3}$
        \item $y=\dfrac{4x^2-3x+10}{x-1}$.
        \item $y=\dfrac{x^2-4x+3}{x^2-1}$.
        \item $y=\dfrac{2x+4}{x^2+x-2}$.
        \item $y=\dfrac{\sqrt{9-x^2}}{x-1}$.
        \item $y=x+\sqrt{x^2-1}$
        %	\item $y=\dfrac{x}{\sqrt{x^2+1}}$.
        %	\item $y=\dfrac{\sqrt{x+25}-5}{x^2+x}$.
    \end{listEX}
    \loigiai{}
\end{vd}
\begin{vd}
    Tìm các đường tiệm cận của đồ thị hàm số $y=f(x)$, biết
    \begin{listEX}[2]
        \item \begin{tikzpicture}[scale=.7,>=stealth, font=\footnotesize, line join=round, line cap=round]
            \def\xmin{-2} \def\xmax{4}
            \def\ymin{-3} \def\ymax{3}
            %\draw[color=gray!50,dashed] (\xmin,\ymin) grid (\xmax,\ymax);
            \draw[->] (\xmin,0)--(\xmax,0) node [below]{$x$};
            \draw[->] (0,\ymin)--(0,\ymax) node [left]{$y$};
            \fill (0,0) circle (1pt) node[shift={(135:2.5mm)}]{$O$};
            %\node at (current bounding box.south) [below=-2pt] {a) $y=\dfrac{2x-3}{5x^{2}-15x+10}$};
            \clip (\xmin+0.1,\ymin+0.1) rectangle (\xmax-0.1,\ymax-0.1);
            \draw[thick,smooth,samples=300,domain=\xmin:0.99] plot(\x,{(2*(\x)-3)/(5*(\x)^2-15*(\x)+10)});
            \draw[thick,smooth,samples=300,domain=1.01:1.99] plot(\x,{(2*(\x)-3)/(5*(\x)^2-15*(\x)+10)});
            \draw[thick,smooth,samples=300,domain=2.01:\xmax] plot(\x,{(2*(\x)-3)/(5*(\x)^2-15*(\x)+10)});
            \draw[dashed] (1,\ymin)--(1,\ymax);
            \draw[dashed] (2,\ymin)--(2,\ymax);
            \foreach \s/\t in {2/-45,1/-45}
            \fill (\s,0) circle (1pt) node[shift={(\t:3mm)}]{$\s$};
        \end{tikzpicture}
        \item \begin{tikzpicture}[scale=.5,>=stealth, font=\footnotesize, line join=round, line cap=round]
            \def\xmin{-4} \def\xmax{4}
            \def\ymin{-3} \def\ymax{5}
            %\draw[color=gray!50,dashed] (\xmin,\ymin) grid (\xmax,\ymax);
            \draw[->] (\xmin,0)--(\xmax,0) node [below]{$x$};
            \draw[->] (0,\ymin)--(0,\ymax) node [right]{$y$};
            \fill (0,0) circle (1pt) node[shift={(-135:2.5mm)}]{$O$};
            %\node at (current bounding box.south) [below=-2pt] {c) $y=\dfrac{16x^{2}-8x}{16x^{2}+1}$};
            \clip (\xmin+0.1,\ymin+0.1) rectangle (\xmax-0.1,\ymax-0.1);
            \draw[thick,smooth,samples=300,domain=\xmin:\xmax] plot(\x,{(16*(\x)^2-8*(\x))/(16*(\x)^2+1)});
            \draw[dashed](\xmin,1)--(\xmax,1);
            \foreach \p/\r in {1/45}
            \fill (0,\p) circle (1pt) node[shift={(\r:3mm)}]{$\p$};
        \end{tikzpicture}
        \item 	\begin{tikzpicture}[scale=.7,>=stealth, font=\footnotesize, y=.7cm]
            \def\xmin{-.5} \def\xmax{6}
            \def\ymin{-.5} \def\ymax{5}
            \draw[->] (\xmin,0)--(\xmax,0) node [below]{$x$};
            \draw[->] (0,\ymin)--(0,\ymax) node [left]{$y$};
            \fill (0,0) circle (1pt) node[shift={(-135:2.5mm)}]{$O$};
            \node at (1,-.5)[right]{$x=1$};
            \clip (\xmin+0.1,\ymin+0.1) rectangle (\xmax-0.1,\ymax-0.1);
            \draw[smooth,thick,samples=300,domain=(1.01:\xmax)] plot(\x,{2/sqrt(\x-1)});
            \draw[blue,dashed] (1,\ymin)--(1,\ymax);
            \draw[blue,dashed] (-1,.8)--(6,.8)node[below left]{$y=0.5$};
            \foreach \x in {\xmin,...,\xmax}
            \draw (\x,-0.1)--(\x,0.1);
            \foreach \y in {\ymin,...,\ymax}
            \draw (-0.1,\y)--(0.1,\y);
        \end{tikzpicture}
        \item \begin{tikzpicture}[scale=0.5, font=\footnotesize, line join=round, line cap=round, >=stealth]
            \clip(-3,-2) rectangle (5.1,4.1);
            \draw[->] (-3,0) -- (5,0);\draw (4.9,0) node[below] { $x$};
            \draw[->] (0,-2) -- (0,4);\draw (0,3.9) node[right] { $y$};
            \draw[fill=black] (0,0) node[below right]{$O$} circle (1pt);
            \draw (1,0) node[below right]{$2$};
            \draw (0,1) node[above left]{$1$};
            \draw[thick] plot[domain=-3:0.5,samples=100] (\x, {(1 + \x)/(\x - 1)});
            \draw[thick] plot[domain= 1.5:5,samples=100] (\x, {(1 + \x)/(\x - 1)});
            \draw [-,dashed] (-3,1)--(5,1); %TCN
            \draw [-,dashed] (1,-2)--(1,4); %TCĐ
            \draw[fill=black] (0,0) circle(1pt);
        \end{tikzpicture}
        \item \begin{tikzpicture}[scale=.7,>=stealth, font=\footnotesize, line join=round, line cap=round]
            \def\xmin{-4} \def\xmax{4}
            \def\ymin{-3} \def\ymax{5}
            %\draw[color=gray!50,dashed] (\xmin,\ymin) grid (\xmax,\ymax);
            \draw[->] (\xmin,0)--(\xmax,0) node [below]{$x$};
            \draw[->] (0,\ymin)--(0,\ymax) node [right]{$y$};
            \fill (0,0) circle (1pt) node[shift={(-135:2.5mm)}]{$O$};
            %\node at (current bounding box.south) [below=-2pt] {b) $y=\dfrac{x^{2}+x-1}{x}$};
            \clip (\xmin+0.1,\ymin+0.1) rectangle (\xmax-0.1,\ymax-0.1);
            \draw[thick,smooth,samples=300,domain=\xmin:-0.01] plot(\x,{((\x)^2+(\x)-1)/(\x)});
            \draw[thick,smooth,samples=300,domain=0.01:\xmax] plot(\x,{((\x)^2+(\x)-1)/(\x)});
            \draw[dashed,smooth,samples=300,domain=\xmin:\xmax] plot(\x,{(\x)+1});
            \foreach \s/\t in {-1/-90}
            \fill (\s,0) circle (1pt) node[shift={(\t:3mm)}]{$\s$};
            \foreach \p/\r in {1/-20}
            \fill (0,\p) circle (1pt) node[shift={(\r:3mm)}]{$\p$};
        \end{tikzpicture}
        \item \begin{tikzpicture}[scale=.7,>=stealth, font=\footnotesize,x=.7cm,y=.7cm]
            \def\xmin{-6} \def\xmax{6}
            \def\ymin{-5} \def\ymax{7}
            %\draw[color=gray!50,dashed] (\xmin,\ymin) grid (\xmax,\ymax);
            \draw[->] (\xmin,0)--(\xmax,0) node [below]{$x$};
            \draw[->] (0,\ymin)--(0,\ymax) node [left]{$y$};
            \fill (0,0) circle (1pt) node[shift={(135:2.5mm)}]{$O$};
            \clip (\xmin+0.1,\ymin+0.1) rectangle (\xmax-0.1,\ymax-0.1);
            \draw[smooth,thick,samples=300,domain=(\xmin:-1.01)] plot(\x,{(2*(\x)^2)/((\x)^2-1)});
            \draw[smooth,thick,samples=300,domain=(-0.9:0.9)] plot(\x,{(2*(\x)^2)/((\x)^2-1)});
            \draw[smooth,thick,samples=300,domain=(1.1:\xmax)] plot(\x,{(2*(\x)^2)/((\x)^2-1)});
            \draw[dashed] (\xmin,2)--(\xmax,2);
            \draw[dashed] (-1,\ymin)--(-1,\ymax);
            \draw[dashed] (1,\ymin)--(1,\ymax);
            \foreach \x in {\xmin,...,\xmax}
            \draw (\x,-0.1)--(\x,0.1);
            \foreach \y in {\ymin,...,\ymax}
            \draw (-0.1,\y)--(0.1,\y);
            \node at (-5,2)[below]{$y=2$};
            \node at (-1.2,-4)[left]{$x=-1$};
            \node at (1.2,-4)[right]{$x=1$};
        \end{tikzpicture}
        \item
        \begin{tikzpicture}[>=stealth]
            \tkzTabInit[nocadre=false,lgt=1,espcl=2,deltacl=0.5]{$x$/.7 ,$y'$/.7,$y$/2}
            {$-\infty$ , $1$ , $+\infty$}
            \tkzTabLine{ , - , d , - , }
            \tkzTabVar{+/$2$ ,-D+/$-\infty$/$+\infty$ , -/$2$}
        \end{tikzpicture}
        \item
        \begin{tikzpicture}[>=stealth]
            \tkzTabInit[nocadre=false,lgt=1,espcl=1.5,deltacl=0.5]{$x$/.7 ,$y'$/.7,$y$/2}
            {$-\infty$ , $0$,$1$ , $+\infty$}
            \tkzTabLine{ , + , 0,-, d , + , }
            \tkzTabVar{-/$0$, +/$2$ ,-D-/$-\infty$/$3$ , +/$5$}
        \end{tikzpicture}
        % \item
        % \begin{tikzpicture}[>=stealth]
        %     \tkzTabInit[nocadre=false,lgt=1,espcl=1.8,deltacl=0.5]{$x$/.7 ,$y'$/.7,$y$/2}
        %     {$-\infty$ , $-1$,$1$ , $+\infty$}
        %     \tkzTabLine{ , - , d,-, 0 , + , }
        %     \tkzTabVar{+/$2$ ,-D+/$-5$/$3$, -/$-1$ , +/$+\infty$}
        % \end{tikzpicture}
        % \item
        % \begin{tikzpicture}[>=stealth]
        %     \tkzTabInit[nocadre=false,lgt=1,espcl=1.4,deltacl=0.5]{$x$/.7 ,$y'$/.7,$y$/2}
        %     {$-\infty$ , $-2$, $0$,$1$ , $+\infty$}
        %     \tkzTabLine{ , - , d,-, 0 , + ,d,-, }
        %     \tkzTabVar{+/$-1$ ,-D+/$-\infty$/$2$, -/$-4$, +/$3$ , -/$0$}
        % \end{tikzpicture}
    \end{listEX}
    \loigiai{}
\end{vd}
\begin{vd}
    Một bể bơi chứa $5\,000$ lít nước tinh khiết. Người ta bơm vào bể đó nước muối có nồng đồ $30$ gam muối cho mỗi lít nước với tốc độ $25$ lít/phút.
    \begin{listEX}
        \item Lập hàm số biểu diễn nồng độ muối trong bể sau $t$ phút.
        \item Tìm tiệm cận ngang của hàm số vừa tìm được.
        \item Nêu nhận xét về nồng độ muối trong bể khi thời gian $t$ ngày càng lớn.
    \end{listEX}
    \loigiai{
        \begin{enumerate}[a)]
            \immini{\item Sau $t$ phút, ta có: khối lượng muối trong bể là $25\cdot 30\cdot t=750t$ (gam); thể tích của lượng nước trong bể là $5\,000+25t$ (lít). Vậy nồng độ muối sau $t$ phút là
                $$f(t)=\dfrac{750t}{5\,000+25t}=\dfrac{30t}{200+t}\,\text{(gam/lít)}.$$
                \item Ta có\\
                $\lim\limits_{t\to +\infty}f(t)=\lim\limits_{t \to +\infty}\dfrac{30t}{200+t}=\lim\limits_{t\to +\infty}\left(30-\dfrac{6\,000}{200+t}\right)=30$.\\
                Vậy đường thẳng $y=30$ là tiệm cận ngang của đồ thị hàm số $f(t)$ \texttt{(Hình 17).}}{\begin{tikzpicture}[scale=.1,xscale=0.1, font=\footnotesize, line join=round, line cap=round, >=stealth]
                    \draw[->] (-.5,0)--(0,0) node[below right]{$O$}--(500,0) node[below]{$x$};
                    \draw[->] (0,-1) --(0,34) node[right]{$y$};
                    \draw[blue] [domain=0:500, samples=100] %
                    plot (\x, {(30*(\x))/((\x)+200)});
                    \draw[fill] (0,0) circle (1pt);
                    \foreach \y/\g in {30/180}
                    \draw[fill] (0,\y) circle(1pt)node [shift={(\g:.3)}] {$\y$};
                    \draw[thick] (-.1,30)--(500,30);
                    \draw (250,-5) node{Hình 17};
            \end{tikzpicture}}
            \item Ta có đồ thị hàm số $y=f(t)$ nhận đường thẳng $y=30$ làm đường tiệm cận ngang, tức là khi $t$ càng lớn thì nồng độ muối trong bể sẽ tiến gần đến mức $30$ (gam/lít). Lúc đó, nồng độ muối trong bể sẽ gần như bằng nồng độ nước muối bơm vào bể.
        \end{enumerate}
    }
\end{vd}
\begin{vd}
    Một mô hình kinh tế mô tả lượng cung cầu theo giá cả được cho bởi hàm:
    \[
    Q(p) = \frac{k}{p - p_0}
    \]
    trong đó \( Q(p) \) là lượng cung cầu, \( p \) là giá cả, \( p_0 \) là mức giá tối thiểu, và \( k \) là hằng số tỷ lệ. Xác định tiệm cận đứng của hàm số này và nêu ý nghĩa của nó.
    % \shortans{$Q=p$, khi giá giảm về mức tối thiểu thì nhu cầu tăng lên vô hạn}
    \loigiai{
        Để tìm tiệm cận đứng, ta xem xét các giá trị của \( p \) làm cho mẫu số của phương trình bằng 0:
        \[
        p - p_0 = 0 \Rightarrow p = p_0
        \]
        Vậy đường thẳng \( p = p_0 \) là tiệm cận đứng của đồ thị hàm số.
        \textbf{Ý nghĩa:} Từ đó ta suy ra khi giá cả \( p \) càng sát với \( p_0 \), lượng cung cầu \( Q(p) \) sẽ tăng lên vô hạn. Điều này có nghĩa là nếu giá cả của sản phẩm giảm gần bằng mức giá tối thiểu \( p_0 \), thì nhu cầu đối với sản phẩm đó sẽ tăng lên vô hạn.}
\end{vd}
\BTTN
\Opensolutionfile{ans}[ans/2D1-4-DANG-1]
\begin{ex}%[Nguyễn Văn Sang, dự án Tex hoá đề cương trường Marie Curie - Lần 6]%[2D1Y4-1]
    Đường thẳng nào dưới đây là tiệm cận ngang của đồ thị hàm số $y=\dfrac{x-1}{x+1}$?
    \choice
    {$y=-1$}
    {$x=-1$}
    {\True $y=1$}
    {$x=1$}
    \loigiai{
        Tập xác định $\mathscr{D}=\mathbb{R}\setminus\left\lbrace -1\right\rbrace$.
        \begin{itemize}
            \item $\lim\limits_{x \to \pm\infty} y=\lim\limits_{x \to \pm\infty} \dfrac{x-1}{x+1}=1$ suy ra $y=1$ là tiệm cận ngang.
            \item $\heva{& \lim\limits_{x \to -1^+} \dfrac{x-1}{x+1}=-\infty \\ & \lim\limits_{x \to -1^-} \dfrac{x-1}{x+1}=+\infty}$ suy ra $x=-1$ là tiệm cận đứng.
        \end{itemize}
    }
\end{ex}
%%=====Câu 13
\begin{ex}%[Nguyễn Văn Sang, dự án Tex hoá đề cương trường Marie Curie - Lần 6]%[2D1Y4-1]
    Đồ thị hàm số $y=\dfrac{2 x-3}{1-2 x}$ có tiệm cận đứng là đường thẳng
    \choice
    {$x=3$}
    {$x=2$}
    {\True $x=\dfrac{1}{2}$}
    {$x=\dfrac{3}{2}$}
    \loigiai{
        Tập xác định $\mathscr{D}=\mathbb{R}\setminus\left\lbrace \dfrac{1}{2}\right\rbrace$.
        \begin{itemize}
            \item $\lim\limits_{x \to \pm\infty} y=\lim\limits_{x \to \pm\infty} \dfrac{2x-3}{1-2x}=-1$ suy ra $y=-1$ là tiệm cận ngang.
            \item $\heva{& \lim\limits_{x \to \tfrac{1}{2}^+} \dfrac{2x-3}{1-2x}=+\infty \\ & \lim\limits_{x \to \tfrac{1}{2}^-} \dfrac{2x-3}{1-2x}=-\infty}$ suy ra $x=\dfrac{1}{2}$ là tiệm cận đứng.
        \end{itemize}
    }
\end{ex}
\begin{ex}%[Nguyễn Văn Sang, dự án Tex hoá đề cương trường Marie Curie - Lần 6]%[2D1Y4-1]
    Đồ thị hàm số $y=\dfrac{2-3 x}{2 x-3}$ có tiệm cận đứng và ngang lần lượt là
    \choice
    {\True $x=\dfrac{3}{2}$ và $y=-\dfrac{3}{2}$}
    {$x=\dfrac{3}{2}$ và $y=1$}
    {$x=\dfrac{2}{3}$ và $y=-\dfrac{3}{2}$}
    {$x=\dfrac{2}{3}$ và $y=1$}
    \loigiai{
        Tập xác định $\mathscr{D}=\mathbb{R}\setminus\left\lbrace \dfrac{3}{2}\right\rbrace$.
        \begin{itemize}
            \item $\lim\limits_{x \to \pm\infty} y=\lim\limits_{x \to \pm\infty} \dfrac{2-3 x}{2 x-3}=\dfrac{-3}{2}$ suy ra $y=-\dfrac{3}{2}$ là tiệm cận ngang.
            \item $\heva{& \lim\limits_{x \to \tfrac{3}{2}^+} \dfrac{2-3 x}{2 x-3}=-\infty \\ & \lim\limits_{x \to \tfrac{3}{2}^-} \dfrac{2-3 x}{2 x-3}=+\infty}$ suy ra $x=\dfrac{3}{2}$ là tiệm cận đứng.
        \end{itemize}
    }
\end{ex}
\begin{ex}%[BGD-THPT-2020-104-L2]%[2D1Y4-1]
    Tiệm cận đứng của đồ thị hàm số $y=\dfrac{x+1}{x+3}$ có phương trình là
    \choice
    {$x=-1$}
    {$x=1$}
    {\True $x=-3$}
    {$x=3$}
    \loigiai{
        Tập xác định của hàm số đã cho $\mathscr{D}=\mathbb{R}\setminus\{-3\}$.\\
        Ta có $\lim\limits_{x\rightarrow-3^-}y=\lim\limits_{x\rightarrow-3^-}\dfrac{x+1}{x+3}=+\infty$ và $\lim\limits_{x\rightarrow-3^+}y=\lim\limits_{x\rightarrow-3^+}\dfrac{x+1}{x+3}=-\infty$.\\
        Khi đó đường tiệm cận đứng của đồ thị hàm số đã cho là $x=-3$.
    }
\end{ex}
%%==========Câu 11
\begin{ex}%[BGD-Minh Họa-2020-L2]%[2D1Y4-1]
    Tiệm cận ngang của đồ thị hàm số $y=\dfrac{x-2}{x+1}$ có phương trình là
    \choice
    {$y=-2$}
    {\True $y=1$}
    {$x=-1$}
    {$x=2$}
    \loigiai
    {
        Tập xác định: $\mathscr{D}=\mathbb{R}\setminus \{-1\}$.\\
        Ta có $\lim \limits_{x \to +\infty} y=\lim \limits_{x \to +\infty} \dfrac{x-2}{x+1}=\lim \limits_{x \to +\infty} \dfrac{1-\dfrac{2}{x}}{1+\dfrac{1}{x}}=1$ và $\lim \limits_{x \to -\infty} y=\lim \limits_{x \to -\infty} \dfrac{x-2}{x+1}=\lim \limits_{x \to -\infty} \dfrac{1-\dfrac{2}{x}}{1+\dfrac{1}{x}}=1$ nên đường thẳng $y=1$ là đường tiệm cận ngang của đồ thị.
    }
\end{ex}
%%==========Câu 12
\begin{ex}%[BGD-THPT-2021-101-L1]%[2D1Y4-1]
    Tiệm cận đứng của đồ thị hàm số $ y=\dfrac{2x-1}{x-1}$ là đường thẳng có phương trình là
    \choice
    {\True $x=1$}
    {$x=-1$}
    {$x=2$}
    {$x=\dfrac{1}{2}$}
    \loigiai{
        Vì $\lim\limits_{x\to 1^+}\dfrac{2x-1}{x-1}=+\infty $ và $\lim\limits_{x\to 1^-}\dfrac{2x-1}{x-1}=-\infty $ nên đồ thị hàm số $ y=\dfrac{2x-1}{x-1}$ có một tiệm cận đứng là đường thẳng $ x=1 $.
    }
\end{ex}
%%==========Câu 13
\begin{ex}%[BGD-THPT-2021-102-L1]%[2D1Y4-1]
    Tiệm cận đứng của đồ thị hàm số $ y=\dfrac{x+1}{x-2}$ là đường thẳng có phương trình là
    \choice
    {$x=-1$}
    {$x=-2$}
    {\True $x=2$}
    {$x=1$}
    \loigiai{
        Ta có $\displaystyle\lim\limits_{x\to 2^+}\dfrac{x+1}{x-2}=+\infty $; $\displaystyle\lim\limits_{x\to 2^-}\dfrac{x+1}{x-2}=-\infty $.\\
        Vậy đồ thị hàm số $ y=\dfrac{x+1}{x-2}$ có tiệm cận đứng là đường thẳng $ x=2 $.
    }
\end{ex}
%%==========Câu 14
\begin{ex}%[BGD-THPT-2021-103-L1]%[2D1B4-1]
    Tiệm cận đứng của đồ thị hàm số $ y=\dfrac{2x+1}{x-1}$ là đường thẳng có phương trình là
    \choice
    {$x=2$}
    {\True $x=1$}
    {$x=-\dfrac{1}{2}$}
    {$x=-1$}
    \loigiai{
        Ta có $\lim\limits_{x\to 1^+}y=\lim\limits_{x\to 1^+}\dfrac{2x+1}{x-1}=+\infty $ nên tiệm cận đứng của đồ thị hàm số là đường thẳng $ x=1 $.
    }
\end{ex}
%%==========Câu 15
\begin{ex}%[BGD-THPT-2021-104-L1]%[2D1Y4-1]
    Tiệm cận đứng của đồ thị hàm số $ y=\dfrac{x-1}{x+2}$ là đường thẳng có phương trình là
    \choice
    {$x=2$}
    {$x=-1$}
    {\True $x=-2$}
    {$x=1$}
    \loigiai{
        Ta có $\lim\limits_{x\to (-2)^{+}}\dfrac{x-1}{x+2}=-\infty $, $\lim\limits_{x\to (-2)^{-}}\dfrac{x-1}{x+2}=+\infty $.\\
        Đồ thị hàm số có tiệm cận đứng là đường thẳng có phương trình $ x=-2 $.
    }
\end{ex}
\begin{ex}%[2D1Y4-1]
    Giao điểm của tiệm cận đứng và tiệm cận ngang của đồ thị hàm số $y=\dfrac{-2}{3x-1}$ là điểm
    \choice
    {$Q\left(\dfrac{1}{3};-2\right)$}
    {$M\left(\dfrac{1}{3};-\dfrac{2}{3}\right)$}
    {$N\left(\dfrac{1}{3};2\right)$}
    {\True $P\left(\dfrac{1}{3};0\right)$}
    \loigiai{
        Tiệm cận đứng, tiệm cận ngang của đồ thị hàm số lần lượt là $x=\dfrac{1}{3}$ và $y=0$. Giao điểm của $2$ tiệm cận là $P\left(\dfrac{1}{3};0\right)$.
    }
\end{ex}
\begin{ex}%[2D1Y4-1]
    Đồ thị hàm số $y=\dfrac{3-4x}{x-5}$ có tâm đối xứng là điểm
    \choice
    {$M\left(5;-\dfrac{3}{5}\right)$}
    {$P\left(5;\dfrac{4}{5}\right)$}
    {$Q(5;3)$}
    {\True $N(5;-4)$}
    \loigiai{
        Tiệm cận đứng, tiệm cận ngang của đồ thị hàm số lần lượt là $x=5$ và $y=-4$. Tâm đối xứng là điểm $N(5;-4)$.
    }
\end{ex}
\begin{ex}%[2D1B4-1]
    Đồ thị hàm số nào dưới đây có tiệm cận đứng?
    \choice
    {$y=\dfrac{x^2-3x+2}{x-1}$}
    {$y=\dfrac{x^2}{x^2+1}$}
    {$y=\sqrt{x^2-1}$}
    {\True $y=\dfrac{x}{x+1}$}
    \loigiai{
    }
\end{ex}
\begin{ex}%[2D1B4-1]
    Cho hàm số $y=f(x)$ có bảng biến thiên như hình bên. Tổng số tiệm cận đứng và tiệm cận ngang của đồ thị hàm số đã cho là
    \begin{center}
        \begin{tikzpicture}
            \tkzTabInit[nocadre=false,lgt=1.5,espcl=3,deltacl=0.6]
            {$x$ /0.6,$y’$ /0.6,$y$ /2}
            {$-\infty$ ,$0$, $1$, $+\infty$}
            \tkzTabLine{,-,d,+,0,-,}
            \tkzTabVar{+/$+\infty$,-D-/$-\infty$/$-1$,+/$2$,-/$-3$}
        \end{tikzpicture}
    \end{center}
    \choice
    {$1$}
    {$3$}
    {\True $2$}
    {$4$}
    \loigiai{Dựa vào bảng biến thiên ta thấy đồ thị hàm số có tiệm cận đứng $x=0$ và tiệm cận ngang $y=-3$.}
\end{ex}
\begin{ex}%[2D1B4-1]
    Cho hàm số $y=f(x)$ có bảng biến thiên như hình bên. Tổng số tiệm cận đứng và tiệm cận ngang của đồ thị hàm số đã cho là
    \begin{center}
        \begin{tikzpicture}[scale=0.8]
            \tkzTabInit[nocadre=false,lgt=1.5,espcl=3,deltacl=0.6]
            {$x$ /0.6,$y’$ /0.6,$y$ /2}
            {$-\infty$ , $0$,$2$, $+\infty$}
            \tkzTabLine{,-,0,+,d,-,}
            \tkzTabVar{+/$8$,-/$1$,+/$4$,-/$2$}
        \end{tikzpicture}
    \end{center}
    \choice
    {$1$}
    {$3$}
    {\True $2$}
    {$4$}
    \loigiai{
        Dựa vào bảng biến thiên ta thấy đồ thị hàm số có tiệm cận ngang $y=8$ và $y=2$.
    }
\end{ex}
\begin{ex}%[2D1B4-1]
    Cho hàm số $y=f(x)$ có bảng biến thiên như hình bên. Tổng số tiệm cận đứng và tiệm cận ngang của đồ thị hàm số đã cho là
    \begin{center}
        \begin{tikzpicture}[scale=0.8]
            \tkzTabInit[nocadre=false,lgt=1.5,espcl=3,deltacl=0.6]
            {$x$ /0.6,$y’$ /0.6,$y$ /2}
            {$-\infty$ ,$1$, $2$, $+\infty$}
            \tkzTabLine{,+,d,-,d,+,}
            \tkzTabVar{-/$-4$,+/$3$,-/$-5$,+/$+\infty$}
        \end{tikzpicture}
    \end{center}
    \choice
    {\True $1$}
    {$3$}
    {$2$}
    {$0$}
    \loigiai{
        Dựa vào bảng biến thiên ta thấy đồ thị hàm số có một tiệm cận ngang $y=-4$.
    }
\end{ex}
\begin{ex}%[2D1B4-1]
    Cho hàm số $y=f(x)$ có bảng biến thiên như hình bên. Đồ thị hàm số đã cho có tiệm cận đứng là đường thẳng
    \begin{center}
        \begin{tikzpicture}[scale=0.8, font=\footnotesize, line join=round, line
            cap=round, >=stealth]
            \tkzTabInit[espcl=2.5,lgt=1,nocadre=false]
            {$x$/0.7,$f(x)$/2.1}
            {$-\infty$,$0$,$1$,$2$,$+\infty$}
            \tkzTabVar{-/$-\infty$,+/$2$,-D+/$-\infty$/$+\infty$,-/$4$,+/$+\infty$}
        \end{tikzpicture}
    \end{center}
    \choice
    {$x=0$}
    {\True $x=1$}
    {$x=2$}
    {$x=4$}
    \loigiai{Dựa vào bảng biến thiên ta thấy đồ thị hàm có tiệm cận đứng $x=1$.}
\end{ex}
%%==========Câu 16
\begin{ex}%[BGD-THPT-2019-103]%[2D1B4-1]
    Cho hàm số $y=f(x)$ có bảng biến thiên như sau
    \begin{center}
        \begin{tikzpicture}[scale=1, font=\footnotesize,line join=round, >=stealth]
            \tkzTabInit[nocadre=false,lgt=1.5,espcl=3]{$x$/.7,$y'$/.7,$y$/2.5}{$-\infty$,$0$,$3$,$+\infty$}%
            \tkzTabLine{,-,d,+,0,-,}%
            \tkzTabVar{+/$1$ , -D+/$-\infty$/$2$,-/$-3$, +/$3$}%
        \end{tikzpicture}
    \end{center}
    Tổng số tiệm cận đứng và tiệm cận ngang của đồ thị hàm số đã cho là
    \choice
    {1}
    {2}
    {\True 3}
    {4}
    \loigiai{
        Nhìn bảng biến thiên ta thấy\\
        $\lim\limits_{x \to 0^-} f(x)=-\infty \Rightarrow x=0$ là TCĐ của đồ thị hàm số.\\
        $\lim\limits_{x \to +\infty} f(x)=3 \Rightarrow y=3$ là TCN của đồ thị hàm số.\\
        $\lim\limits_{x \to -\infty} f(x)=1 \Rightarrow y=1$ là TCN của đồ thị hàm số.\\
        Vậy hàm số có 3 tiệm cận.}
\end{ex}
%%==========Câu 17
\begin{ex}%[BGD-THPT-2019-102]%[2D1B4-1]
    Cho hàm số $f(x)$ có bảng biến thiên như sau
    \begin{center}
        \begin{tikzpicture}[scale=1, font=\footnotesize,line join=round, >=stealth]
            \tkzTabInit[lgt=1.2,espcl=3]
            {$x$/0.8,$f’(x)$/0.8,$f(x)$/2}
            {$-\infty$,$0$,$1$,$+\infty$}
            \tkzTabLine{ ,-,d,-,0,+,}
            \tkzTabVar{+/$0$,-D+/$-\infty$/$2$,-/$-2$,+/$+\infty$}
        \end{tikzpicture}
    \end{center}
    Tổng số tiệm cận đứng và tiệm cận ngang của đồ thị hàm số đã cho là
    \choice
    {$3$}
    {$1$}
    {\True $2$}
    {$4$}
    \loigiai{
        Từ bảng biến thiên đã cho ta có\\
        $\lim\limits_{x \to -\infty} f(x)=0$ nên đường thẳng $y=0$ là một tiệm cận ngang của đồ thị hàm số.\\
        $\lim\limits_{x \to 0^-} f(x)=-\infty$ nên đường thẳng $x=0$ là một tiệm cận đứng của đồ thị hàm số.\\
        Vậy đồ thị hàm số đã cho có hai đường tiệm cận.}
\end{ex}
\begin{ex}%[2D1B4-1]
    Cho hàm số $y=f(x)$ có bảng biến thiên như hình bên. Tổng số tiệm cận đứng và tiệm cận ngang của đồ thị hàm số đã cho là
    \begin{center}
        \begin{tikzpicture}[scale=0.8]
            \tkzTabInit[nocadre=false,lgt=1.5,espcl=3,deltacl=0.6]
            {$x$ /0.6,$y’$ /0.6,$y$ /2}
            {$-\infty$ ,$0$, $1$, $+\infty$}
            \tkzTabLine{,+,0,-,d,-,}
            \tkzTabVar{-/$4$,+/$2$,-D+/$-\infty$/$5$,-/$-3$}
        \end{tikzpicture}
    \end{center}
    \choice
    {$1$}
    {\True $3$}
    {$2$}
    {$4$}
    \loigiai{
        Dựa vào bảng biến thiên ta thấy đồ thị hàm số có tiệm cận đứng $x=1$, tiệm cận ngang $y=4$ và $y=-3$.
    }
\end{ex}
\begin{ex}%[2D1B4-1]
    Cho hàm số $y=f\left(x\right)$ có bảng biến thiên như sau
    \begin{center}
        \begin{tikzpicture}[scale=1,line join=round,>=stealth]\tikzset{double style/.append style={double distance=2pt}}
            \tkzTabInit[nocadre=false,lgt=1.2,espcl=2.2,deltacl=0.6]
            {$x$ /.6,$y'$ /.6,$y$ /2.2}
            {$ -\infty $,$-2$,$0$,$+\infty$}
            \tkzTabLine{,-,d,+,d,-}
            \tkzTabVar{+/$+\infty$,-D-/$1$/$-\infty$,+D+/$+\infty$/$1$,-/$0$,}
        \end{tikzpicture}
    \end{center}
    Tổng số đường tiệm cận đứng và tiệm cận ngang của đồ thị hàm số đã cho bằng
    \choice
    {$2$}
    {$1$}
    {$0$}
    {\True $3$}
    \loigiai{
        Ta có
        \begin{itemize}
            \item $\lim\limits_{x \to -2^{+}} y=-\infty \Rightarrow x=-2$ là tiệm cận đứng.
            \item $\lim\limits_{x \to 0^{-}} y=+\infty \Rightarrow x=0$ là tiệm cận đứng.
            \item $\lim\limits_{x \to +\infty} y=0 \Rightarrow y=0$ là tiệm cận ngang.
        \end{itemize}
        Vậy đồ thị hàm số đã cho có tổng đường tiệm cận đứng và tiệm cận ngang là $3$.}
\end{ex}
\begin{ex}%[2D1B4-1]
    Cho hàm số $y=f\left(x\right)$ liên tục trên $\mathbb{R} \backslash\{1\}$ có bảng biến thiên như bảng sau:
    \begin{center}
        \begin{tikzpicture}[scale=1,line join=round,>=stealth]
            \tikzset{double style/.append style={double distance=2pt}}
            \tkzTabInit[nocadre=false,lgt=1.2,espcl=2.8,deltacl=0.6]
            {$x$ /0.6,$y'$ /0.6,$y$ /2.2}
            {$ -\infty $,$-1$,$1$,$+\infty$}
            \tkzTabLine{,-,0,+,d,+}
            \tkzTabVar{+/$1$,-/$-\sqrt 2$,+D-/$+\infty$/$-\infty$,+/$-1$,}
        \end{tikzpicture}
    \end{center}
    Tổng số đường tiệm cận đứng và đường tiệm cận ngang của đồ thị hàm số $y=f\left(x\right)$ là
    \choice
    {$1$}
    {$4$}
    {$2$}
    {\True $3$}
    \loigiai{
        Do $\lim\limits_{x \to 1^{+}} y=-\infty \Rightarrow$ Tiệm cận đứng $x=1$.\\
        Lại có $\lim\limits_{x \to +\infty} y=-1 ; \lim\limits_{x \to -\infty} y=1 \Rightarrow$ Đồ thị có $2$ tiệm cận ngang là $y=\pm 1$.\\
        Vậy, đồ thị hàm số đã cho có tổng số tiệm cận là $3$.}
\end{ex}
\begin{ex}%[2D1B4-1]
    Cho hàm số $y=f\left(x\right)$ có bảng biến như sau:
    \begin{center}
        \begin{tikzpicture}[scale=1,line join=round,>=stealth]
            \tikzset{double style/.append style={double distance=2pt}}
            \tkzTabInit[nocadre=false,lgt=1.2,espcl=2.5,deltacl=0.6]{$x$ /.6,$y'$ /.6,$y$ /2}
            {$ -\infty $,$-3$,$3$,$+\infty$}
            \tkzTabLine{,+,d,+,d,+}
            \tkzTabVar{-/$0$,+D-/$+\infty$/$-\infty$,+D-/$+\infty$/$-\infty$,+/$0$,}
        \end{tikzpicture}
    \end{center}
    Số đường tiệm cận của đồ thị hàm số là
    \choice
    {\True $3$}
    {$1$}
    {$4$}
    {$2$}
    \loigiai{
        Từ bảng biến thiên của hàm số ta có
        \begin{itemize}
            \item $\lim\limits_{x \to -\infty} y=0 ; \lim\limits_{x \to +\infty} y=0 \Rightarrow$ Đường thẳng $y=0$ là tiệm cận ngang.
            \item $\lim\limits_{x \to (-3)^{-}} y=+\infty \Rightarrow$ Đường thẳng $x=-3$ là tiệm cận đứng.
            \item $+\lim\limits_{x \to 3^{-}} y=+\infty \Rightarrow$ Đường thẳng $x=3$ là tiệm cận đứng.
        \end{itemize}
        Vậy số đường tiệm cận của đồ thị hàm số là $3$.}
\end{ex}
\begin{ex}%[2D1B4-1]
    Cho hàm số $y=f\left(x\right)$ có bảng biến thiên như sau
    \begin{center}
        \begin{tikzpicture}[scale=1,line join=round,>=stealth]
            \tikzset{double style/.append style={double distance=2pt}}
            \tkzTabInit[nocadre=false,lgt=1.2,espcl=2.5,deltacl=0.6]
            {$x$ /.6,$y'$ /.6,$y$ /2.2}
            {$ -\infty $,$-2$,$2$,$+\infty$}
            \tkzTabLine{,-,d,-,d,-}
            \tkzTabVar{+/$0$,-D+/$-\infty$/$+\infty$,-D+/$-\infty$/$+\infty$,-/$-\infty$,}
        \end{tikzpicture}
    \end{center}
    Tổng số tiệm cận đứng và tiệm cận ngang của đồ thị hàm số đã cho là
    \choice
    {$4$}
    {$2$}
    {\True $3$}
    {$1$}
    \loigiai{
        Dựa vào bảng biến thiên, ta có:
        \begin{itemize}
            \item $\lim\limits_{x \to -\infty} f(x)=0$ nên đường thẳng $y=0$ là đường tiệm cận ngang.
            \item $\lim\limits_{x \to -2^{+}} f(x)=+\infty $ nên đường thẳng $x=-2$ là đường tiệm cận đứng.
            \item $\lim\limits_{x \to 2^{+}} f(x)=+\infty$ nên đường thẳng $x=2$ là đường tiệm cận đứng.
        \end{itemize}
        Vậy, tổng số tiệm cận đứng và tiệm cận ngang của đồ thị hàm số đã cho là $3$.}
\end{ex}
%%==========Câu 20
\begin{ex}%[THPT Yên Định - Thanh Hóa 2019]%[2D1B4-1]
    Cho hàm số $ y=f(x) $ xác định và có đạo hàm trên $ \mathbb{R}\setminus\{\pm 1\} $. Hàm số có bảng biến thiên như hình vẽ dưới đây.
    \begin{center}
        \begin{tikzpicture}[scale=1, font=\footnotesize,line join=round, >=stealth]
            \tkzTabInit[nocadre=false,lgt=1.2,espcl=2.5,deltacl=0.6]{$x$/.6 ,$y'$/.6,$y$/2.5} {$-\infty$ , $-1$ , $0$ , $1$ , $+\infty$}
            \tkzTabLine{ , + , d , - , d , + , d , + , }
            \tkzTabVar{-/$-4$ , +D-/$+\infty$/$-\infty$ , +/$2$,-D-/$-\infty$/$-\infty$,+/$-1$}
        \end{tikzpicture}
    \end{center}
    Tổng số đường tiệm cận đứng và tiệm cận ngang của đồ thị hàm số đã cho là
    \choice
    { $ 1 $ }
    { $ 2 $ }
    { $ 3 $ }
    {\True $ 4 $ }
    \loigiai{
        Dựa vào bảng biến thiên, suy ra:\\
        $ \lim \limits_{x \to - \infty} y=-4 $, $ \lim \limits_{x \to + \infty} y=-1$. Đồ thị có hai tiệm cận ngang là $ y=-4 $ và $ y=-1 $.\\
        Lại có $ \lim \limits_{x \to (-1)^+} y=+\infty $ và $ \lim \limits_{x \to 1^-} y=+\infty $, $ \lim \limits_{x \to 1^-} y=-\infty $. Đồ thị hàm số có hai đường tiệm cận đứng là $ x=1 $ và $ x=-1 $.
    }
\end{ex}
\begin{ex}%[2D1B4-1]
    Cho hàm số $y=f\left(x\right)$ có bảng biến thiên như hình vẽ dưới đây.
    \begin{center}
        \begin{tikzpicture}[line cap=round,line join=round,>=triangle 45,x=1.0cm,y=1.0cm]
            \clip(-1.58,-2.4) rectangle (12.58,2.);
            \fill[line width=1.2pt,dash pattern=on 15 pt off 5pt,color=white,fill=black,pattern=north east lines,pattern color=black] (0.,1.) -- (2.84,1.) -- (2.84,-1.96) -- (0.,-1.96) -- cycle;
            \draw (-1.,1.)-- (12.,1.);
            \draw (-1.,0.)-- (12.,0.);
            \draw (0.,1.62)-- (0.,-1.96);
            \draw (0.08,1.5) node[anchor=north west] {$-\infty$};
            \draw (2.46,1.5) node[anchor=north west] {$-2$};
            \draw (6.85,1.5) node[anchor=north west] {$0$};
            \draw (11.14,1.5) node[anchor=north west] {$+\infty$};
            \draw (2.84,1.)-- (2.84,-1.96);
            \draw (3.,1.)-- (3.,-1.96);
            \draw (7.,1.)-- (7.,-1.96);
            \draw (7.14,1.)-- (7.14,-1.96);
            \draw (-0.7,1.5) node[anchor=north west] {$x$};
            \draw (-0.72,0.78) node[anchor=north west] {$y'$};
            \draw (-0.64,-0.8) node[anchor=north west] {$y$};
            \draw [->] (3.96,-1.54) -- (6.24,-0.64);
            \draw [->] (7.64,-0.58) -- (11.12,-1.7);
            \draw (11.34,-1.5) node[anchor=north west] {$0$};
            \draw (8.98,0.7) node[anchor=north west] {$-$};
            \draw (4.68,0.7) node[anchor=north west] {$+$};
            \draw (6.0,-0.2) node[anchor=north west] {$+\infty$};
            \draw (7.22,-0.2) node[anchor=north west] {$1$};
            \draw (3.08,-1.5) node[anchor=north west] {$-\infty$};
        \end{tikzpicture}
    \end{center}
    Hỏi đồ thị của hàm số đã cho có bao nhiêu đường tiệm cận?
    \choice
    {\True $3$}
    {$2$}
    {$4$}
    {$1$}
    \loigiai{
        Dựa vào bảng biến thiên ta có:\\
        $\lim\limits_{x \to -2^{+}} f(x)=-\infty$, suy ra đường thẳng $x=-2$ là tiệm cận đứng của đồ thị hàm số.\\
        $\lim\limits_{x \to 0^{-}} f(x)=+\infty$, suy ra đường thẳng $x=0$ là tiệm cận đứng của đồ thị hàm số.\\
        $\lim\limits_{x \to +\infty} f(x)=0$, suy ra đường thẳng $y=0$ là tiệm cận ngang của đồ thị hàm số.\\
        Vậy đồ thị hàm số có $3$ đường tiệm cận.}
\end{ex}
%%=====Câu 5
\begin{ex}%[Nguyễn Văn Sang, dự án Tex hoá đề cương trường Marie Curie - Lần 6]%[2D1Y4-1]
    Cho hàm số $y=f(x)$ có $\lim\limits_{x \rightarrow 3} f(x)=+\infty$, $\lim\limits_{x \rightarrow+\infty} f(x)=-\infty$, $\lim\limits_{x \rightarrow-\infty} f(x)=8$ và $\lim\limits_{x \rightarrow 7} f(x)=5 $. Tổng số tiệm cận ngang và tiệm cận đứng của đồ thị hàm số đã cho là
    \choice
    {$4$}
    {\True $2$}
    {$1$}
    {$3$}
    \loigiai{
        Ta có
        \begin{itemize}
            \item $\lim\limits_{x \rightarrow-\infty} f(x)=8$, suy ra $y=8$ là tiệm cận ngang.
            \item $\lim\limits_{x \rightarrow 3} f(x)=+\infty$, suy ra $x=3$ là tiệm cận đứng.
            \item $\lim\limits_{x \rightarrow 7} f(x)=5 $, suy ra $x=7$ không là tiệm cận đứng.
        \end{itemize}
        Vậy đồ thị hàm số có $1$ tiệm cận đứng và $1$ tiệm cận ngang.
    }
\end{ex}
%%=====Câu 7
\begin{ex}%[Nguyễn Văn Sang, dự án Tex hoá đề cương trường Marie Curie - Lần 6]%[2D1Y4-1]
    Cho hàm số $y=f(x)$ có $\lim\limits_{x \rightarrow 1^{+}} f(x)=+\infty$ và $\lim\limits_{x \rightarrow 1^{-}} f(x)=2$. Mệnh đề nào sau đây đúng?
    \choice
    {Đồ thị hàm số không có tiệm cận}
    {\True Đồ thị hàm số có tiệm cận đứng $x=1$}
    {Đồ thị hàm số có hai tiệm cận}
    {Đồ thị hàm số tiệm cận ngang $y=2$}
    \loigiai{
        Ta có $\lim\limits_{x \rightarrow 1^{-}} f(x)=2$, suy ra $x=1$ là tiệm cận đứng.
    }
\end{ex}
\begin{ex}%[2D1B4-1]
    Cho hàm số $y=\dfrac{\sqrt{x+1}}{\sqrt{x^2-4}}$ mệnh đề nào sau đây đúng?
    \choice
    {\True Đồ thị hàm số có một tiệm cận đứng và một tiệm cận ngang}
    {Đồ thị hàm số có một tiệm cận đứng và hai tiệm cận ngang}
    {Đồ thị hàm số có hai tiệm cận đứng và hai tiệm cận ngang}
    {Đồ thị hàm số có hai tiệm cận đứng và một tiệm cận ngang}
    \loigiai{
        Tập xác định $\mathscr{D}=[-1;+\infty) \setminus \{2\}$. \\
        Đồ thị hàm số có một tiệm cận đứng $x=2$, tiệm cận ngang là $y=0$.
    }
\end{ex}
\begin{ex}%[2-HK1-49-THPT-NKKN-TPHCM, 12EX5]%[Nhật Thiện, ID6]%[2D1K4-2]%
    Với giá trị nào của $m$ thì đồ thị hàm số $y=\dfrac{mx-1}{2x+m}$ có tiệm cận đứng là đường thẳng $x=-1$?
    \choice
    {$m=2$}
    {\True $m=-2$}
    {$m=\dfrac{1}{2}$}
    {$m=0$}
    \loigiai{
        Đồ thị hàm số $y=\dfrac{mx-1}{2x+m}$ có tiệm cận đứng là đường thẳng $x=-1$ khi và chỉ khi $$\heva{&m(-1)-1\ne 0\\&2(-1)+m=0}\Leftrightarrow \heva{&m\ne -1\\&m=-2(n).}$$
    }
\end{ex}
\begin{ex}%[2D1K4-1]
    Đồ thị hàm số $y=\dfrac{2x-1-\sqrt{x^2+x+3}}{x^2-5x+6}$ có tất cả đường tiệm cận đứng là đường thẳng
    \choice
    {$x=-3$ và $x=-2$}
    {$x=-3$}
    {$x=3$ và $x=-2$}
    {\True $x=3$}
    \loigiai{
        Điều kiện xác định $x \ne 3$, $x \ne 2$.\\
        Với điều kiện xác định trên, ta có
        {\allowdisplaybreaks
            \begin{eqnarray*}
                y&=&\dfrac{2x-1-\sqrt{x^2+x+3}}{x^2-5x+6}=\dfrac{(3x+1)(x-2)}{(x-2)(x-3)\left(2x-1+\sqrt{x^2+x+3}\right)}\\
                &=&\dfrac{3x+1}{(x-3)\left(2x-1+\sqrt{x^2+x+3}\right)}.
        \end{eqnarray*} }
        Tiệm cận đứng của đồ thị hàm số là $x=3$.
    }
\end{ex}
\begin{ex}%[2D1K4-1]
    Số tiệm cận đứng của đồ thị hàm số $y=\dfrac{\sqrt{x+9}-3}{x^2+x}$ là
    \choice
    {$3$}
    {$2$}
    {$0$}
    {\True $1$}
    \loigiai{
        Tập xác định $\mathscr{D}=[-9;+\infty)\setminus \{-1;0\}$. \\
        Ta có $\left\{\begin{aligned}
            &\lim\limits_{x\to -1^+} y=\lim\limits_{x\to -1^+} \dfrac{\sqrt{x+9}-3}{x^2+x}=+\infty \\
            &\lim\limits_{x\to -1^-} y =\lim\limits_{x\to -1^-} \dfrac{\sqrt{x+9}-3}{x^2+x}=-\infty
        \end{aligned}\right. \Rightarrow x=-1$ là tiệm cận đứng. \\
        Ngoài ra $\lim\limits_{x\to 0} y =\lim\limits_{x\to 0} \dfrac{\sqrt{x+9}-3}{x^2+x}=\dfrac{1}{6}$ nên $x=0$ không phải là một tiệm cận đứng.}
\end{ex}
\BTTF
\begin{ex}%[EX-TF-2024, Lê Đạt]%[2D1N4-1]
    Cho hàm số $y=\dfrac{2x-3}{x-1}$. Xét tính đúng sai các khẳng định dưới đây
    \choiceTF
    {\True Đường tiệm cận đứng của đồ thị hàm số là $ x=1 $}
    {Đường tiệm cận đứng của đồ thị hàm số là $ y=2 $}
    {Đường tiệm cận ngang của đồ thị hàm số là $ x=1 $}
    {\True Đường tiệm cận ngang của đồ thj hàm số là $ y=2 $}
    \loigiai{
        Ta có $\lim\limits_{x\to -\infty}y=\lim\limits_{x\to +\infty}y=2$ nên đồ thị hàm số đã cho có tiệm cận ngang là $y=2$.\\
        Ta có $\lim\limits_{x\to 1^+}y=-\infty$ nên đồ thị hàm số đã cho có tiệm cận ngang là $ x=1 $.
        \begin{itemchoice}
            \itemch Đường tiệm cận đứng của đồ thị hàm số là $ x=1 $.
            \itemch Đường tiệm cận đứng của đồ thị hàm số là $ x=1 $.
            \itemch Đường tiệm cận ngang của đồ thj hàm số là $ y=2 $.
            \itemch Đường tiệm cận ngang của đồ thj hàm số là $ y=2 $.
        \end{itemchoice}
    }
\end{ex}
\begin{ex}%[EX-TF-2024, Lê Đạt]%[2D1N4-1]
    Cho hàm số $y=f(x)$ có bảng biến thiên như sau
    \begin{center}
        \begin{tikzpicture}[>=stealth]
            \tkzTabInit[nocadre=false,lgt=1,espcl=3,deltacl=0.6]
            {$x$/.7 ,$y'$/.7,$y$/2}
            {$-\infty$ , $-2$ , $0$, $+\infty$}
            \tkzTabLine{ , - , d , + , d , -, }
            \tkzTabVar{+/$+\infty$ , -D-/$1$/$-\infty$ , +D+/$+\infty$ /$1$, -/$0$}
        \end{tikzpicture}
    \end{center}
    Xét tính đúng sai của các khẳng định sau
    \choiceTF
    {\True $ x=0 $ là tiệm cận đứng của đồ thị hàm số $ y=f(x) $}
    {\True $ x=-2 $ là tiệm cận đứng của đồ thị hàm số $ y=f(x) $}
    {$ x=1 $ là tiệm cận đứng của đồ thị hàm số $ y=f(x) $}
    {\True $ y=0 $ là tiệm cận ngang của đồ thị hàm số $ y=f(x) $}
    \loigiai{
        \begin{itemchoice}
            \itemch $\lim \limits_{x \to 0^-} f(x)=+\infty\Rightarrow x=0$ là đường tiệm cận đứng của đồ thị hàm số $f(x)$.
            \itemch $\lim \limits_{x \to (-2)^+} f(x)=-\infty\Rightarrow x=-2$ là đường tiệm cận đứng của đồ thị hàm số $f(x)$.
            \itemch Đồ thị hàm số chỉ có hai tiệm cận đứng là $ x=0 $ và $ x=-2 $.
            \itemch $\lim \limits_{x \to +\infty} f(x)=0\Rightarrow y=0$ là đường tiệm cận ngang của đồ thị hàm số $f(x)$.
        \end{itemchoice}
    }
\end{ex}
%===== DẠNG 2
\begin{ex}%[EX-TF-2024, Lê Đạt]%[2D1H4-2]
    Cho hàm số $ y=\dfrac{m^2x+1}{x-1} $. Xét tính đúng sai của các khẳng định sau
    \choiceTF
    {\True Đồ thị hàm số luôn có tiệm cận ngang}
    {\True Đồ thị hàm số luôn có tiệm cận đứng}
    {\True Khi $ m=1$ đồ thị hàm số có $ 2 $ đường tiệm cận}
    {Khi $ m=0 $ đồ thị hàm số có $ 1 $ đường tiệm cận}
    \loigiai{
        \begin{itemchoice}
            \itemch $\lim\limits_{x\to -\infty}y=\lim\limits_{x\to +\infty}y=m^2$ suy ra hàm số luôn có tiệm cận ngang.
            \itemch $\lim\limits_{x\to 1^+}y=+\infty$ nên đồ thị hàm số đã cho có tiệm cận ngang là $ x=1 $.
            \itemch Khi $ m=1 $ ta được hàm số $ y=\dfrac{x+1}{x-1} $ suy ra đồ thì hàm số có $ x=1 $ là tiệm cận đứng và $ y=1 $ là tiệm cận ngang nên đồ thị hàm số có $ 2 $ tiệm cận.
            \itemch Khi $ m=0 $ ta được hàm số $ y=\dfrac{1}{x-1} $ suy ra đồ thì hàm số có $ x=1 $ là tiệm cận đứng và $ y=0 $ là tiệm cận ngang nên đồ thị hàm số có $ 2 $ tiệm cận.
        \end{itemchoice}
    }
\end{ex}
\begin{ex}%[EX-TF-2024, Lê Đạt]%[2D1H4-2]
    Cho hàm số $y=\dfrac{m x^{2}+6 x-2}{x+2}$. Xét tính đúng sai của các khẳng định sau
    \choiceTF
    {Đồ thị hàm số luôn có tiệm cận đứng với mọi $ m $}
    {Đồ thị hàm số không có tiệm cận ngang với mọi $ m $}
    {\True Khi $ m=1 $ đồ thị hàm số có một tiệm cận xiên là $ y=x+4 $ }
    {Đồ thị hàm số luôn có tiệm cận xiên}
    \loigiai{
        \begin{itemchoice}
            \itemch Khi $ m=\dfrac{7}{2} $ hàm số trở thành $y=\dfrac{\dfrac{7}{2} x^{2}+6 x-2}{x+2}=\dfrac{7}{2}\left(x-\dfrac{2}{7} \right) $ suy ra đồ thị hàm số không có tiệm cận đứng.
            \itemch Khi $ m=0 $ hàm số trở thành $ y=\dfrac{6x-2}{x+2} $ từ đó suy ra đồ thị hàm số có $ y=6 $ là tiệm cận ngang.
            \itemch Khi $ m=1 $ hàm số trở thành $ y=\dfrac{x^2+6x-2}{x+2}=x+4-\dfrac{10}{x+2} $ từ đó suy ra $ y=x+4 $ là một tiệm cận ngang.
            \itemch Khi $ m=0 $ hàm số trở thành $ y=\dfrac{6x-2}{x+2} $ từ đó suy ra đồ thị hàm số có $ y=6 $ là tiệm cận ngang, $ x=-2 $ là tiệm cận đứng và không có tiệm cận xiên.
        \end{itemchoice}
    }
\end{ex}
\begin{ex}
    Cho hàm số $y=\dfrac{x-1}{x^2-8 x+m}$, $m$ là tham số. Các mệnh đề sau đúng hay sai?
    \choiceTF
    {\True Đồ thị hàm số có 1 đường tiệm cận ngang}
    {Khi $m<16$ thì đồ thị hàm số có 3 đường tiệm cận}
    {Khi $m=16$ thì đồ thị hàm số có 2 đường tiệm cận đứng}
    {\True Có 14 giá trị nguyên dương của $m$ để đồ thị hàm số có 3 đường tiệm cận}
    \loigiai{
        Ta có $\lim \limits{n \to +\infty}_{x \rightarrow-\infty} \frac{x-1}{x^2-8 x+m}=\lim \limits{n \to +\infty}_{x \rightarrow+\infty} \frac{x-1}{x^2-8 x+m}=0$ nên hàm số có một tiện cận ngang $y=0$.
        Hàm số có 3 đường tiệm cận khi và chỉ khi hàm số có hai đường tiệm cận đứng $\Leftrightarrow$ phương trình $x^2-8 x+m=0$ có hai nghiệm phân biệt khác $1 \Leftrightarrow\left\{\begin{array}{l}\Delta^{\prime}=16-m>0 \\ m-7 \neq 0\end{array} \Leftrightarrow\left\{\begin{array}{l}m<16 \\ m \neq 7\end{array}\right.\right.$.
        Kết hợp với điều kiện $m$ nguyên dương ta có $\quad m \in\{1 ; 2 ; 3 ; \ldots ; 6 ; 8 ; \ldots ; 15\}$. Vậy có 14 giá trị của $m$ thỏa mãn đề bài.}
\end{ex}
\begin{ex}
    Cho hàm số $y=\dfrac{x^2+m x-1}{x-1}\left(C_m\right)$ ( $m$ là tham số). Các mệnh đề sau đúng hay sai?
    \choiceTF
    {\True Để đồ thị $\left(C_m\right)$ của hàm số có tiệm cận xiên thì $m \neq 0$.}
    {\True Để tiệm cận xiên của $\left(C_m\right)$ đi qua $M(2,-5)$ thì $m=-8$}
    { Để tiệm cận xiên của $\left(C_m\right)$ tạo với hai trục toạ độ một tam giác có diện tích bằng 8 thì tổng tất cả các giá trị $m$ tìm được bằng 2}
    { Với $m=3$ thì giao điểm của hai đường tiệm cận của $\left(C_m\right)$ nằm trên Parapol $y=x^2+3$}
    \loigiai{
        Hàm số xác định trên $\mathbb{R} \backslash\{1\}$.
        \begin{listEX}
            \item Ta có $y=x+m+1+\frac{m}{x-1}$
            Để đồ thị $\left(C_m\right)$ của hàm số có tiệm cận xiên thì $m \neq 0$.
            - Với $m \neq 0,\left(C_m\right)$ có tiệm cận xiên
            $y=x+m+1\left(\Delta_m\right)$ vì $\lim \limits{n \to +\infty}_{x \rightarrow \infty}[y-(x+m+1)]=\lim \limits{n \to +\infty}_{x \rightarrow \infty} \frac{m}{x-1}=0$.
            \item Để $\left(\Delta_m\right)$ qua $M(2,-5)$ thì $-5=2+m+1 \Leftrightarrow m=-8$. (thỏa mãn $m \neq 0$ ).
            \item Gọi $A$ là giao điểm của $\Delta_m$ với $O x$. Khi đó $A(-m-1 ; 0)$
            Gọi $B$ là giao điểm của $\Delta_m$ với $O y$. Khi đó $B(0 ; m+1)$.
            Suy ra $S_{\triangle O A B}=\frac{1}{2} O A \cdot O B=\frac{1}{2}|-m-1||m+1|=\frac{1}{2}(m+1)^2$
            Để $S_{\triangle O A B}=8 \Leftrightarrow \frac{1}{2}(m+1)^2=8 \Leftrightarrow\left[\begin{array}{l}m=-5 \\ m=3\end{array}\right.$ (thỏa mãn $m \neq 0$ ).
            \item Ta có với $m \neq 0, x=1$ là tiệm cận đứng vì $\lim \limits{n \to +\infty}_{x \rightarrow 1} y=\infty$ nên $y=x+m+1$ là tiệm cận xiên.
            Khi đó giao điểm của 2 tiệm cận là $I(1, m+2)$.
            Để $I$ nằm trên Parabol $y=x^2+3$ thì $m+2=1+3 \Leftrightarrow m=2(\mathrm{t} / \mathrm{m} m \neq 0)$.
        \end{listEX}
    }
\end{ex}
%===== DẠNG 3
\begin{ex}%[EX-TF-2024, Lê Đạt]%[2D1N4-3]
    \immini{Cho hàm số $y=f(x)$ có đồ thị như hình bên. Xét tính đúng sai của các khẳng định sau
        \choiceTF
        {$ x=2 $ là đường tiệm cận ngang của đồ thị hàm số}
        {\True $ x=-1 $ là đường tiệm cận đứng của đồ thị hàm số}
        {\True Đồ thị hàm số có hai đường tiệm cận}
        {\True Đồ thị hàm số không có tiệm cận xiên}
    }{
        \begin{tikzpicture}[scale=0.5, font=\footnotesize, line join=round, line cap=round, >=stealth]
            \draw[->](-5,0)--(5,0)node[below]{ $x$};
            \draw[->](0,-4)--(0,5)node[right]{ $y$};
            \draw [fill=black,draw=black] (0,0) circle (1pt)node[above left] { $O$};
            \foreach \x in {-1}\draw[shift={(\x,0)}](0pt,-2pt)--(0pt,2pt) node[below left]{ $\x$};
            \foreach \y in {2}\draw[shift={(0,\y)}](-2pt,0pt)--(2pt,0pt)node[above right]{ $\y$};
            \clip(-5,-4) rectangle (5,5);
            \draw[smooth,samples=100,domain=-5:-1.1] plot(\x,{(2*(\x)-1)/((\x)+1)});
            \draw[smooth,samples=100,domain=-0.9:5] plot(\x,{(2*(\x)-1)/((\x)+1)});
            \draw[dashed](-5,2)--(5,2) (-1,-4)--(-1,5);
        \end{tikzpicture}
    }
    \loigiai{
        \begin{itemchoice}
            \itemch $ y=2 $ là đường tiệm cận ngang của đồ thị hàm số.
            \itemch $ x=-1 $ là đường tiệm cận đứng của đồ thị hàm số.
            \itemch $ x=-1 $ là đường tiệm cận đứng và $ y=2 $ là đường tiệm cận ngang của đồ thị hàm số suy ra đồ thị hàm số có hai đường tiệm cận.
            \itemch Đồ thị hàm số không có tiệm cận xiên.
        \end{itemchoice}
    }
\end{ex}
\begin{ex}%[EX-TF-2024, Lê Đạt]%[2D1H4-3]
    \immini{Cho hàm số $y=f(x)$ có đồ thị như hình bên. Xét tính đúng sai của các khẳng định sau
        \choiceTF
        {\True $ x=0 $ là một đường tiệm cận đứng của đồ thị hàm số}
        {$ y=-x $ là một đường tiệm cận xiên của đồ thị hàm số}
        {\True $ y=x $ là một đường tiệm cận xiên của đồ thị hàm số}
        {Đồ thị hàm số có ba đường tiệm cận}
    }{
        \begin{tikzpicture}[scale=.9, font=\footnotesize, line join=round, line cap=round,>=stealth]
            \def\a{0} \def\b{1} \def\c{1} \def\d{-1} % Hệ số
            \def\xmin{-3} \def\xmax{3.5}
            \def\ymin{-2.8} \def\ymax{3.3}
            \draw[color=gray!50,dashed] (\xmin,\ymin) grid (\xmax,\ymax);
            \draw[->] (\xmin,0)--(\xmax,0) node [below]{$x$};
            \draw[->] (0,\ymin)--(0,\ymax) node [left]{$y$};
            \fill (0,0) circle(1pt) node[shift=(-45:0.25)]{$O$};
            \clip (\xmin+0.1,\ymin+0.1) rectangle (\xmax-0.1,\ymax-0.1);
            \draw[smooth,samples=300,domain=-3:3] plot(\x,{\x+1/(7*\x)});
            \draw[dashed,smooth,samples=300,domain=-3:3] plot(\x,{\x});
            %	\fill (-1,0) circle (1.0pt) node[below]{$-1$} (1,0) circle (1.0pt) node[below right]{$1$};
    \end{tikzpicture}}
    \loigiai{
        \begin{itemchoice}
            \itemch $ x=0 $ là một đường tiệm cận đứng của đồ thị hàm số.
            \itemch	$ y=x $ là một đường tiệm cận xiên của đồ thị hàm số.
            \itemch $ y=x $ là một đường tiệm cận xiên của đồ thị hàm số.
            \itemch Đồ thị hàm số có $ x=0 $ là tiệm cận đứng và $ y=x $ là tiệm cận xiên nên có hai tiệm cận.
        \end{itemchoice}
    }
\end{ex}
\BTTL
\begin{ex}%[2D1K4-2]
    Nếu đồ thị hàm số $y=\dfrac{(m+1)x+2}{x-n+1}$ lần lượt nhận trục hoành và trục tung làm đường đường tiệm cận ngang và tiệm cận đứng thì $m+n$ bằng bao nhiêu?
    \shortans{$0$}
    \loigiai{
        Theo đề bài, ta có $\heva{&m+1=0\\&n-1=0} \Leftrightarrow \heva{&m=-1\\&n=1.}$\\
        Suy ra $m+n=0$.
    }
\end{ex}
\begin{ex}%[2D1K4-2]
    Tìm giá trị của $m$ để đồ thị hàm số $y=\dfrac{(2m+1)x+3}{x+1}$ có đường đường tiệm cận đi qua điểm $A(-2;7)$.
    \shortans{m=3}
    \loigiai{
        Từ đề bài, suy ra $2m+1=7 \Leftrightarrow m=3$.\\
        Suy ra $m+n=0$.
    }
\end{ex}
\begin{ex}%[2D1K4-2]
    Cho hàm số $y=\dfrac{-3+mx}{x+n}$. Tìm giá trị của $m$ và $n$ để đồ thị hàm số đã cho có tiệm cận đứng $x=2$ và tiệm cận ngang $y=2$.
    \shortans{$m=2, n=-2$}
    \loigiai{
        Từ yêu cầu đề bài, suy ra $\heva{&m=2\\&-n=2} \Leftrightarrow \heva{&m=2\\&n=-2.}$
    }
\end{ex}
\begin{ex}%[2D1K4-2]
    Để đường tiệm cận đứng và tiệm cận ngang của đồ thị hàm số $y=\dfrac{mx+1}{2m+1-x}$ cùng với hai trục tọa độ tạo thành một hình chữ nhật có diện tích bằng $3$ thì giá trị của $m$ bằng bao nhiêu?
    \shortans{$1$ hay $-\dfrac{3}{2}$}
    \loigiai{
        Từ yêu cầu đề bài, suy ra $|-m| \cdot |2m+1|=3 \Leftrightarrow \hoac{&m=1\\&m=-\dfrac{3}{2}.}$
    }
\end{ex}
\begin{ex}%[2D1K4-2]%[Thầy Hải Toán]%Câu 2.
    Đường tiệm cận đứng và đường tiệm cận ngang của đồ thị hàm số $y=\dfrac{mx+1}{2m+1-x}$ cùng với hai trục tọa độ tạo thành một hình chữ nhật có diện tích bằng $3$. Tính giá trị của $m$.
    \shortans{$m=1$; $m=-\dfrac{3}{2}$}
    \loigiai{
        Ta có $\lim\limits_{x\to+\infty}\dfrac{mx+1}{2m+1-x}=-m$; $\lim\limits_{x\to(2m+1)^+}\dfrac{mx+1}{2m+1-x} =\lim\limits_{x\to(2m+1)^+}\dfrac{m(2m+1)+1}{2m+1-x} =\lim\limits_{x\to(2m+1)^+}\dfrac{2m^2+m+1}{2m+1-x}$
        $\lim\limits_{x\to(2m+1)^+}\left(2m^2+m+1\right)=2m^2+m+1>0$; $\lim\limits_{x\to(2m+1)^+}(2m+1-x)=0$ và $2m+1-x<0\forall x>2m+1$ \\
        $ \Rightarrow\lim\limits_{x\to(2m+1)^+}\dfrac{mx+1}{2m+1-x}=-\infty $.\\
        Vậy đồ thị hàm số có hai đường tiệm cận $x=2m+1$ và $y=-m$.\\
        Hai đường tiệm cận tạo với hai trục tọa độ một hình chữ nhật có diện tích bằng $3$ suy ra $|2m+1|\cdot|m|=3\Leftrightarrow\hoac{&2m^2+m=3\\&2m^2+m=-3(PTVN)}\Leftrightarrow 2m^2+m-3=0\Leftrightarrow\hoac{&m=1\\&m=\dfrac{-3}{2}}$.}
\end{ex}
\begin{ex}%[KSCL L1, THPT Nhã Nam - Bắc Giang, 2019]%[Phạm An Bình, 12EX3]%[2D1K4-2]%
    Biết rằng đồ thị của hàm số $y=\dfrac{(n-3)x+n-2017}{x+m+3}$ ($m$, $n$ là tham số) nhận trục hoành làm tiệm cận ngang và trục tung làm tiệm cận đứng. Tính tổng $m-2n$.
    \shortans{$-9$}
    \loigiai{
        $\bullet$ $\lim\limits_{x\to +\infty}y=\lim\limits_{x\to +\infty}\dfrac{n-3+\dfrac{n-2017}{x}}{1+\dfrac{m+3}{x}} =n-3$.\\
        Vì đồ thị nhận trục hoành làm tiệm cận ngang nên $n-3=0\Leftrightarrow n=3$.\\
        $\bullet$ Vì đồ thị hàm số nhận trục tung làm tiệm cận đứng nên $\heva{&n-2017\ne 0\\&m+3=0}\Leftrightarrow \heva{&n\ne 2017\\&m= -3.}$\\
        Vậy $m-2n=-9$.
    }
\end{ex}
\begin{ex}%[TT Nguyễn Đăng Đạo, Bắc Ninh, lần 3, đề 152, 2018]%[2D1K4-2]%[Nguyễn Vân Trường, 12EX-8]%
    Tìm $m$ để tiệm cận đứng của đồ thị hàm số $y = \dfrac{m^2x-4m}{2x-m^2}$ đi qua điểm $A(2;1)$.
    \shortans{$m=-2$}
    \loigiai{
        Để hàm số có tiệm cận đứng thì \\
        $\hoac{& m \ne 0 \\ & m^2\cdot \dfrac{m^2}{2} - 4m \ne 0} \Leftrightarrow \hoac{& m \ne 0 \\ & m(m^3-8) \ne 0} \Leftrightarrow \hoac{& m \ne 0 \\ & m \ne 2}.$\\
        Khi đó tiệm cận đứng của hàm số là $x = \dfrac{m^2}{2}.$ Theo giả thiết ta có $ \dfrac{m^2}{2} = 2 \Leftrightarrow \hoac{& m =2 \text{ (loại)} \\ & m=-2 \text{ (thỏa mãn).}}$ Vậy $m=-2$.
    }
\end{ex}
\begin{ex}%[TT, Chuyên Lê Quý Đôn, Lai Châu, 2018]%[2D1K4-2]%[Nguyễn Tiến Thùy, 12EX-8]%
    Tìm $m$ để đồ thị hàm số $y=\dfrac{(m+1)x-5m}{2x-m}$ có tiệm cận ngang là đường thẳng $y=1$.
    \shortans{$m=1$}
    \loigiai{
        Ta có $\lim\limits_{x\rightarrow \pm\infty}f(x)=\lim\limits_{x\rightarrow \pm\infty}\dfrac{(m+1)x-5m}{2x-m}=\dfrac{m+1}{2}$, suy ra $y=\dfrac{m+1}{2}$ là tiệm cận ngang.\\
        Theo bài ra ta có $y=\dfrac{m+1}{2}=1\Leftrightarrow m=1$.
    }
\end{ex}
\begin{ex}%[2D1K4-1]
    Tìm tất cả các đường tiệm cận ngang của đồ thị hàm số $y=\dfrac{\sqrt{4x^2-x+1}}{2x+1}$.
    \shortans{$y=1$ và $y=-1$}
    \loigiai{
        Điều kiện xác định $x \ne \dfrac{-1}{2}$.\\
        Ta có $\lim\limits_{x \to +\infty} \dfrac{\sqrt{4x^2-x+1}}{2x+1}=\lim\limits_{x \to +\infty} \dfrac{|2x|\sqrt{1-\dfrac{1}{4x}+\dfrac{1}{4x^2}}}{2x\left(1+\dfrac{1}{2x}\right)}=1$.\\
        $\lim\limits_{x \to -\infty} \dfrac{\sqrt{4x^2-x+1}}{2x+1}=\lim\limits_{x \to -\infty} \dfrac{|2x|\sqrt{1-\dfrac{1}{4x}+\dfrac{1}{4x^2}}}{2x\left(1+\dfrac{1}{2x}\right)}=-1$.\\
        Tiệm cận ngang của đồ thị hàm số là $y= \pm 1$.
    }
\end{ex}
\begin{ex}%[2D1K4-1]
    Đồ thị hàm số $y=\dfrac{1-\sqrt{x^2+x+1}}{x^3+1}$ có tất cả bao nhiêu tiệm cận đứng và ngang?
    \shortans{$1$}
    \loigiai{
        Tập xác định $\mathscr{D}=\mathbb{R} \setminus \{-1\}$.
        \begin{itemize}
            \item
            {\allowdisplaybreaks
                \begin{eqnarray*}
                    \lim\limits_{x\to -1} \dfrac{1-\sqrt{x^2+x+1}}{x^3+1}&=&\lim\limits_{x\to -1} \dfrac{-x(x+1)}{(x+1)\left(x^2-x+1\right)\left(1+\sqrt{x^2+x+1}\right)}\\
                    &=&\lim\limits_{x\to -1} \dfrac{-x}{\left(x^2-x+1\right)\left(1+\sqrt{x^2+x+1}\right)}\\
                    &=&\dfrac{1}{6}.
            \end{eqnarray*} }
            \item $\lim\limits_{x\to +\infty}\dfrac{1-\sqrt{x^2+x+1}}{x^3+1}=0$.
        \end{itemize}
        Đồ thị hàm số không có tiệm cận đứng, tiệm cận ngang là $y=0$.
    }
\end{ex}
\begin{ex}%[2D1K4-1]
    Đồ thị hàm số $y=\dfrac{|x|}{\sqrt{x^2-1}}$ có tất cả bao nhiêu tiệm cận đứng và ngang?
    \shortans{$3$}
    \loigiai{
        Tập xác định $\mathscr{D}=(-\infty;-1) \cup (1;+\infty)$.
        \begin{itemize}
            \item $\lim\limits_{x\to-1^-}\dfrac{|x|}{\sqrt{x^2-1}}=+\infty$.
            \item $\lim\limits_{x\to 1^+}\dfrac{|x|}{\sqrt{x^2-1}}=+\infty$.
            \item $\lim\limits_{x\to +\infty}\dfrac{|x|}{\sqrt{x^2-1}}=1$.
            \item $\lim\limits_{x\to -\infty}\dfrac{|x|}{\sqrt{x^2-1}}=1$.
        \end{itemize}
        Đồ thị hàm số có $2$ tiệm cận đứng là $x=\pm 1$, tiệm cận ngang là $y=1$.
    }
\end{ex}
\begin{ex}%[2D1K4-1]
    Đồ thị hàm số $y=\dfrac{x}{\sqrt{x^2+1}}$ có tất cả bao nhiêu tiệm cận đứng và ngang?
    \shortans{$2$}
    \loigiai{
        Tập xác định $\mathscr{D}=\mathbb{R}$.
        \begin{itemize}
            \item $\lim\limits_{x\to +\infty}\dfrac{x}{\sqrt{x^2+1}}=1$.
            \item $\lim\limits_{x\to -\infty}\dfrac{x}{\sqrt{x^2+1}}=-1$.
        \end{itemize}
        Đồ thị hàm số không có tiệm cận đứng, tiệm cận ngang là $y=\pm 1$.
    }
\end{ex}
\begin{ex}%[2D1K4-1]
    Đồ thị hàm số $y=\dfrac{\sqrt{x^2-4}}{x^2-5x+6}$ có tất cả bao nhiêu tiệm cận đứng và ngang?
    \shortans{$3$}
    \loigiai{
        Tập xác định $\mathscr{D}=(-\infty;-2] \cup (2;+\infty) \setminus \{3\}$.
        \begin{itemize}
            \item $\lim\limits_{x\to 2^+}\dfrac{\sqrt{x^2-4}}{x^2-5x+6}=-\infty$.
            \item $\lim\limits_{x\to -2^-}\dfrac{\sqrt{x^2-4}}{x^2-5x+6}=-\infty$.
            \item $\lim\limits_{x\to +\infty}\dfrac{\sqrt{x^2-4}}{x^2-5x+6}=0$.
            \item $\lim\limits_{x\to -\infty}\dfrac{\sqrt{x^2-4}}{x^2-5x+6}=0$.
        \end{itemize}
        Đồ thị hàm số có $2$ tiệm cận đứng là $x=\pm 2$, tiệm cận ngang là $y=0$.
    }
\end{ex}
\begin{ex}
    Nồng độ thuốc trong máu $C(t)$ sau $t$ giờ khi uống một liều thuốc có thể được mô tả bởi hàm $C(t) = \dfrac{3}{1 + 2t}$. Tìm đường tiệm cận của nồng độ thuốc khi thời gian tăng lên rất lớn.
    \shortans{$0$}
\end{ex}
\begin{ex}
    Tốc độ (km/h) của một chiếc xe hơi tăng theo thời gian được mô tả bởi hàm $ v(t) = \dfrac{120t}{3+ t}$. Tìm đường tiệm cận của tốc độ khi thời gian tăng lên rất lớn.
    \shortans{$120$}
\end{ex}
\begin{ex}%[TeX hóa SGK CTST 12]%[Phạm Phương]%[2D1V4-4]
    Nồng độ oxygen trong hồ theo thời gian $t$ cho bởi công thức $y(t)=5-\dfrac{15t}{9t^{2}+1}$, với $y$ được tính theo mg/l và $t$ được tính theo giờ, $t \geq 0$. Tìm các đường tiệm cận của đồ thị hàm số $y(t)$. Từ đó, có nhận xét gì về nồng độ oxygen trong hồ khi thời gian $t$ trở nên rất lớn?
    \shortans{$y=5$, nồng độ tiến về $5$ mg/l}
    \loigiai{
        Hàm số $y(t)=5-\dfrac{15t}{9t^{2}+1}$ có tập xác định $\mathscr{D}=\mathbb{R}$.\\
        Ta có $\lim\limits_{x \rightarrow+\infty} \left(5-\dfrac{15t}{9t^{2}+1}\right)=5$.\\
        Vậy đồ thị hàm số có tiệm cận ngang là đường thẳng $y=5$.\\
        Khi thời gian $t$ trở nên rất lớn thì nồng độ oxygen trong hồ sẽ tiến dần về $5$ mg/l.
    }
\end{ex}
\begin{ex}
    Mô hình phát triển số lượng lợi khuẩn $P(t)$ theo thời gian có thể được mô tả bởi hàm $P(t) = \dfrac{100}{1 + 5e^{-2t}}$. Tính số lượng lợi khuẩn khi thời gian tăng lên rất lớn.\\
    \shortans{$100$}
\end{ex}
\begin{ex}
    Đáp ứng xung của một hệ thống điện tử the thời gian $t$ được mô tả bởi hàm \( h(t) = 120 e^{-\sqrt{3}t} \sin(2 t + \pi) \). Tìm và nêu ý nghĩa của đường tiệm cận của đáp ứng xung khi thời gian tăng.
    \shortans{$0$}
\end{ex}
\begin{ex}
    Điện áp của pin sạc theo thời gian được mô tả bởi hàm \( V(t) = 220 \left(1 - e^{-\dfrac{t}{\tau}}\right) \), trong đó \( \tau \) là hằng số thời gian. Tìm và nêu ý nghĩa của đường tiệm cận của điện áp khi thời gian tăng.
    \shortans{$220$}
\end{ex}
\begin{ex}%[0D1K1-4]
    Số lượng sản phẩm bán được của một công ty trong $x$ (tháng) được tính theo công thức $S(x)=200\left(5-\dfrac{9}{2+x}\right)$, trong đó $x\ge 1$ \emph{(Nguồn: R.Larson and B.Edwards, Calculus 10e, Cengage 2014).}
    \begin{enumerate}[a)]
        \item Xem $y=S(x)$ là một hàm số xác định trên nửa khoảng $[1;+\infty)$, hãy tìm tiệm cận ngang của đồ thị hàm số đó.
        \item Nêu nhận xét về số lượng sản phẩm bán được của công ty trong $x$ (tháng) khi $x$ đủ lớn.
    \end{enumerate}
    \shortans{$y=1$, sản phẩm gần $1\,000$}
    \loigiai{
        \begin{enumerate}[a)]
            \item Ta có $\lim\limits_{x\to +\infty}\left[200\left(5+\dfrac{9}{2-x}\right)\right]=200\cdot 5=1000$.\\
            Vậy $y=1\,000$ là tiệm cận ngang của đồ thị hàm số $y=S(x)$.
            \item Từ phần trên ta có thể rút ra nhận xét: khi số tháng đủ lớn thì công ty có thể bán được số sản phẩm gần bằng $1\,000$.
        \end{enumerate}
    }
\end{ex}
\begin{ex}
    Công ty cung cấp dịch vụ internet tính $75\$$ phí lắp đặt thiết bị ban đầu và phí sử dụng internet $40\$$ mỗi tháng
    \begin{listEX}
        \item Lập hàm số thể hiện chi phí sử dụng trung bình mỗi tháng sau $x$ tháng sử dụng
        \item Chi phí sử dụng trung bình thay đổi thế nào khi số tháng sử dụng tăng lên rất nhiều.
    \end{listEX}
    \shortans{$y=\dfrac{40x+75}{40}$, chi phí tiến về $40\$$}
\end{ex}
\begin{ex}
    Nhà trường dự định tổ chức tiệc liên hoan chào mừng lớp 12, tiền thuê hội trường là $1$ tỷ. Cứ mỗi người tham gia sẽ tính thêm phí phục vụ là $2$ triệu mỗi người. Gọi $x$ là số người tham gia bữa tiệc
    \begin{listEX}
        \item Lập hàm số thể hiện tổng chi phí của bữa tiệc
        \item Lập hàm số thể hiện chi phí trung bình của mỗi người bỏ ra cho bữa tiệc
        \item Chi phí trung bình của mỗi người thay đổi thế nào khi số người tham gia tăng lên rất lớn.
    \end{listEX}
    \shortans{$y=\dfrac{0,02x+1}{x}$, tiến về $2$ triệu}
\end{ex}
\begin{ex}
    Số lượng vi khuẩn trong một môi trường dinh dưỡng có thể được mô tả bởi hàm:
    \[
    N(t) = \dfrac{N_0}{1 - \dfrac{t}{T}}
    \]
    trong đó \( N(t) \) là số lượng vi khuẩn tại thời gian \( t \), \( N_0 \) là số lượng vi khuẩn ban đầu, và \( T \) là thời gian mà môi trường dinh dưỡng không còn đủ để hỗ trợ sự tăng trưởng của vi khuẩn. Xác định tiệm cận đứng của hàm số này và nêu ý nghĩa của nó.
    \shortans{$t=T$, khi $t$ tiến về $T$ thì số lượng vi khuẩn tăng lên vô hạn}
    \loigiai{
        Để tìm tiệm cận đứng, ta xem xét các giá trị của \( t \) làm cho mẫu số của phương trình bằng 0:
        \[
        1 - \frac{t}{T} = 0 \Rightarrow t = T
        \]
        Vậy đường thẳng \( t = T \) là tiệm cận đứng của đồ thị hàm số.
        \textbf{Ý nghĩa:} Từ đó ta suy ra khi thời gian \( t \) càng sát với \( T \), số lượng vi khuẩn \( N(t) \) sẽ tăng lên vô hạn. Điều này có nghĩa là khi thời gian tiếp cận \( T \) thì số lượng vi khuẩn sẽ tăng lên nhanh chóng đến mức vô hạn.}
\end{ex}
\begin{ex}
    Trong vật lý, vận tốc tối đa \(V\) của một vật rơi qua một chất lỏng được mô tả bằng phương trình:
    \[
    V(t) = \frac{mg}{b} \left(1 - e^{-\dfrac{bt}{m}}\right)
    \]
    trong đó \(m\) là khối lượng của vật, \(g\) là gia tốc trọng trường, \(b\) là hệ số ma sát, và \(t\) là thời gian. Xác định tiệm cận đứng của hàm số này và nêu ý nghĩa của nó.
    \shortans{Không có TCĐ}
    \loigiai{Để tìm tiệm cận đứng, ta xem xét các giá trị của \(t\) làm cho mẫu số của phương trình bằng 0. Tuy nhiên, trong trường hợp này, hàm số không có tiệm cận đứng vì biểu thức mũ đảm bảo hàm số được xác định cho tất cả các số thực.
        \textbf{Ý nghĩa:} Điều này ngụ ý rằng không có giới hạn về thời gian để vật đạt đến vận tốc tối đa. Khi thời gian tăng lên, vận tốc của vật sẽ tiệm cận đến vận tốc tối đa, nhưng vật không bao giờ thực sự đạt được nó.}
\end{ex}
\begin{ex}
    Trong sinh học, sự tăng trưởng của dân số \(P\) theo thời gian \(t\) có thể được mô hình bằng hàm số:
    \[
    P(t) = \frac{P_0}{1 - kP_0t}
    \]
    trong đó \(P_0\) là kích thước dân số ban đầu và \(k\) là hằng số tốc độ tăng trưởng. Xác định tiệm cận đứng của hàm số này và nêu ý nghĩa của nó.
    \shortans{$t = \frac{1}{kP_0}$}
    \loigiai{Để tìm tiệm cận đứng, ta xem xét các giá trị của \(t\) làm cho mẫu số của phương trình bằng 0:
        \[
        1 - kP_0t = 0 \Rightarrow t = \frac{1}{kP_0}
        \]
        Vậy tiệm cận đứng là \(t = \frac{1}{kP_0}\).
        \textbf{Ý nghĩa:} Điều này ngụ ý rằng hàm số tăng trưởng dân số có một tiệm cận đứng tại thời điểm \(t\) bằng nghịch đảo của tích của hằng số tốc độ tăng trưởng \(k\) và kích thước dân số ban đầu \(P_0\). Điều này chỉ ra một giới hạn cho tốc độ tăng trưởng dân số theo thời gian.}
\end{ex}
\begin{ex}
    Trong khoa học máy tính, độ phức tạp thời gian \(T(n)\) của một thuật toán với kích thước đầu vào \(n\) có thể được biểu diễn bằng hàm số:
    \[
    T(n) = \frac{an^2 + bn + c}{n}
    \]
    trong đó \(a\), \(b\), và \(c\) là các hằng số. Xác định tiệm cận đứng của hàm số này và nêu ý nghĩa của nó.
    \shortans{$T=0$}
    \loigiai{Để tìm tiệm cận đứng, ta xem xét các giá trị của \(n\) làm cho mẫu số của phương trình bằng 0:
        \[
        n = 0
        \]
        Vậy tiệm cận đứng là \(n = 0\).
        \textbf{Ý nghĩa:} Điều này ngụ ý rằng hàm số độ phức tạp thời gian không có tiệm cận đứng. Trong phân tích tính toán, một tiệm cận đứng tại \(n = 0\) sẽ ngụ ý rằng thuật toán có độ phức tạp thời gian vô hạn cho các đầu vào có kích thước bằng 0, điều này không có ý nghĩa trong hầu hết các trường hợp.}
\end{ex}
\Closesolutionfile{ans}
\begin{dang}{Đường tiệm cận liên quan tham số $m$}
\end{dang}
\begin{vd}
    Tìm $m$ để đồ thị hàm số
    \begin{listEX}[2]
        \item $y=\dfrac{x-2}{x^2-mx+1}$ có hai đường tiệm cận đứng.
        \item $y=\dfrac{x-1}{x^2-mx+1}$ có đúng ba đường tiệm cận.
        \item $y=\dfrac{\sqrt{x-3}}{x^2+x-m}$ có đúng hai đường tiệm cận.
        \item $y=\dfrac{\sqrt{1-x}}{x^2+4x+m}$ có ba đường tiệm cận.
        %	\item* $y=\dfrac{x}{x^2-2(m+1)x+m^2}$ có đúng hai đường tiệm cận.
    \end{listEX}
    \loigiai{}
\end{vd}
\BTTN
\Opensolutionfile{ans}[ans/2D1-4-DANG-2]
\begin{ex}%[Phạm Văn Long]%[Latex-HK2-TT-2020-2021]%[2D1K4-2]%
    Tìm $m$ để đồ thị hàm số $y=\dfrac{2x^2-3x+4}{x^2+mx+1}$ có duy nhất một đường tiệm cận?
    \choice
    {\True $m\in (-2;2)$}
    {$m\in [-2;2]$}
    {$m\in \{-2;2\}$}
    {$m\in (2;+\infty)$}
    \loigiai{
        Ta thấy đồ thị hàm số đã cho luôn có một tiệm cận ngang là đường $y=2$.\\
        Do đó, để đồ thị hàm số đã cho có duy nhất một đường tiệm cận thì đồ thị hàm số đã cho không có tiệm cận đứng.\\
        $\Rightarrow$ Phương trình $x^2+mx+1=0$ vô nghiệm $\Leftrightarrow \Delta <0 \Leftrightarrow m^2-4<0\Leftrightarrow m\in (-2;2)$.
    }
\end{ex}
\begin{ex}%[2D1K4-2]%[Đề GHK1, THPT Trần Nhân Tông, Hà Nội 2018]%[WTT2D1-128]%
    Tìm giá trị thực của tham số $m$ để đồ thị hàm số $y=\dfrac{x-4}{m-x^2}$ có đường tiệm cận đứng.
    \choice
    {$m\ge0;\,m\ne16$}
    {\True $m\ge0$}
    {$m>0$}
    {$m>0;\,m\ne16$}
    \loigiai{
        Điều kiện xác định: $m-x^2\neq0$.\\
        Để đồ thị hàm số có đường tiệm cận đứng thì phương trình $m-x^2=0$ có nghiệm, tức là $m\ge0$.\\
        Với $m=16$ thì $y=\dfrac{-1}{4+x}$ có một tiệm cận đứng là $x=-4$. Vậy giá trị $m$ cần tìm là $m\ge0$.
    }
\end{ex}
\begin{ex}%[2D1K4-2]%
    Có bao nhiêu giá trị của tham số $m$ thoả mãn đồ thị hàm số $y=\dfrac{x+3}{x^2-x-m}$ có đúng hai đường tiệm cận?
    \choice
    {$1$}
    {$4$}
    {\True $2$}
    {$3$}
    \loigiai{
        Đồ thị hàm số có đúng hai đường tiệm cận khi phương trình $x^2-x-m=0$ có nghiệm kép hoặc có hai nghiệm phân biệt với một nghiệm bằng $-3$. Khi đó
        \[\hoac{&\Delta=0\\&\heva{&\Delta>0\\&g(-3)=0}}\Leftrightarrow\hoac{&4m+1=0\\&\heva{&4m+1>0\\&m=12}}\hoac{&m=-\dfrac{1}{4}\\&m=12.}\]
        Vậy có hai giá trị của m.}
\end{ex}
\begin{ex}%[2D1K4-2]%[Đề kiểm tra giữa học kì I, 2017 - 2018 trường THPT Chu Văn An, Hà Nội]%[WTT2D1-156]%
    Tìm tất cả các giá trị thưc của tham số $m$ để đồ thị hàm số $y=\dfrac{x^2+m}{x^2-3x+2}$ có đúng hai tiệm cận.
    \choice
    {$m=-1$}
    {$m\in\left\{1;4\right\}$}
    {\True $m\in\left\{-1;-4\right\}$}
    {$m=4$}
    \loigiai{
        Vì $\lim\limits_{x\to\pm\infty}\dfrac{x^2+m}{x^2-3x+2}=1,\,\forall m$ nên đồ thị hàm số luôn có một tiêm cận ngang là $y=1$.\\
        Để đồ thị hàm số có đúng hai tiệm cận thì đồ thị hàm số có thêm một tiệm cận đứng là $x=1$ hoặc là $x=2$.
        \begin{itemize}
            \item Đồ thị hàm số có một tiệm cận đứng $x=1$, suy ra pt $x^2+m=0$ và phương trình $x^2-3x+2=0$ có nghiệm chung là $x=1\Rightarrow m=-1$.
            \item Đồ thị hàm số có một tiệm cận đứng $x=2$, suy ra pt $x^2+m=0$ và phương trình $x^2-3x+2=0$ có nghiệm chung là $x=2\Rightarrow m=-4$.
        \end{itemize}
        Vậy $m\in\left\{-1;4\right\}$ thỏa yêu cầu bài toán.
    }
\end{ex}
\begin{ex}%[Thi thử, THPT Lục Ngạn - Bắc Giang, 2019]%[Trần Như Ngọc, 12EX3-2019]%[2D1K4-2]%
    Có bao nhiêu giá trị nguyên dương của tham số $m$ để đồ thị hàm số
    $y=\dfrac{\sqrt{9-x}}{x^2-2(m+1)x+m^2+2m}$
    có đúng hai đường tiệm cận.
    \choice
    {\True $2$}
    {$1$}
    {$4$}
    {$3$}
    \loigiai{
        Ta có $ x^2-2(m+1)x+m^2+2m = 0 \Leftrightarrow \hoac{& x=m \\ & x=m+2}$
        $( \Delta ' = 1 )$. \\
        Hàm số xác định khi $ \heva{& x \le 9 \\ & x \ne m \\ & x \ne m+2.} $\\
        Ta có $\lim \limits_{x\to -\infty}y = 0$ nên đồ thị hàm số có một tiệm cận ngang là $ y = 0 $.\\
        Đồ thị hàm số có đúng hai tiệm cận khi và chỉ khi nó có đúng một tiệm cận đứng \\
        $\Leftrightarrow$ phương trình trên có một nghiệm nhỏ hơn hoặc bằng $ 9 $.\\
        $\Leftrightarrow m \le 9 < m+2 \Leftrightarrow 7 < m \le 9 $.\\
        Vậy có $ 2 $ giá trị $ m $ nguyên dương thỏa mãn điều kiện bài toán.
    }
\end{ex}
\begin{ex}%[Thi học kỳ I, Trường THPT Chuyên Lê Quý Đôn - Khánh Hòa, 2021]%[Lê Hồng Phi, 12EX5]%[2D1K4-2]%
    Cho hàm số $y=\dfrac{2x-3}{\sqrt{x^2+2(m-2)x+m^2}}$ với $m$ là tham số thực và $m>1$. Hỏi đồ thị hàm số có bao nhiêu đường tiệm cận (tiệm cận ngang và tiệm cận đứng)?
    \choice
    {$1$}
    {\True $2$}
    {$3$}
    {$4$}
    \loigiai
    {Phương trình $x^2+2(m-2)x+m^2=0$ có $\Delta'=(m-2)^2-m^2=-2(2m-2)=-4(m-1)<0,\ \forall m>1$ nên vô nghiệm.\\
        Do đó tập xác định của hàm số là $\mathscr{D}=\mathbb{R}$.\\
        Như thế đồ thị hàm số không có đường tiệm cận đứng.\\
        Ta tính được
        \begin{itemize}
            \item $\lim\limits_{x\to +\infty}y=\lim\limits_{x\to +\infty}\dfrac{2-\dfrac{3}{x}}{\sqrt{1+\dfrac{2(m-2)}{x}+\dfrac{m^2}{x}}}=2$ nên $y=2$ là đường tiệm cận ngang.
            \item $\lim\limits_{x\to -\infty}y=\lim\limits_{x\to -\infty}\dfrac{2-\dfrac{3}{x}}{-\sqrt{1+\dfrac{2(m-2)}{x}+\dfrac{m^2}{x}}}=-2$ nên $y=-2$ là đường tiệm cận ngang.
        \end{itemize}
        Vậy đồ thị hàm số đã cho có $2$ đường tiệm cận.
    }
\end{ex}
\begin{ex}%[Đề Khảo sát lần 1 THPT Quang Hà - Vĩnh Phúc, 2021]%[Trần Nhân Kiệt, 12EX4-2021]%[2D1K4-2]%
    Có bao nhiêu giá trị nguyên của tham số $m$ để đồ thị tham số $y=\dfrac{1+\sqrt{x+1}}{\sqrt{x^2-(1-m)x+2m}}$ có hai tiệm cận đứng?
    \choice
    {$2$}
    {\True $3$}
    {$1$}
    {$0$}
    \loigiai{
        Điều kiện $\heva{& x\ge -1 \\ & x^2-(1-m)x+2m>0.}$\\
        Đồ thị hàm số có hai tiệm cận đứng khi và chỉ khi phương trình $x^2-(1-m)x+2m=0$ có hai nghiệm phân biệt lớn hơn hoặc bằng $-1$.\\
        Ta có $x^2-(1-m)x+2m=0\Leftrightarrow x^2-x+m(x+2)=0\Leftrightarrow m=\dfrac{-x^2+x}{x+2}$.\\
        Đặt $f(x)=\dfrac{-x^2+x}{x+2}$, $x\ge -1$.\\
        Ta có $f'(x)=\dfrac{-x^2-4x+2}{(x+2)^2}$, suy ra $f'(x)=0\Leftrightarrow -x^2-4x+2=0\Leftrightarrow x=-2\pm \sqrt{6}$.
        \begin{center}
            \begin{tikzpicture}[>=stealth]
                \tkzTabInit[nocadre=false,lgt=1.2,espcl=3,deltacl=0.5]
                {$x$/.7 ,$f'(x)$/.7,$f(x)$/2}
                {$-1$ , $-2+\sqrt{6}$ , $+\infty$}
                \tkzTabLine{ , - , $0$ , + , }
                \tkzTabVar{-/$-2$ , +/$5-2\sqrt{6}$ , -/$-\infty$}
            \end{tikzpicture}
        \end{center}
        Từ bảng biến thiên suy ra $m\in [-2;5-2\sqrt{6})$.\\
        Vì $m$ nguyên nên $m\in \{-2;-1;0\}$.\\
        Vậy có $3$ giá trị nguyên của $m$ thỏa mãn bài.
    }
\end{ex}
\BTTL
\begin{ex}%[2D1K4-2]%
    Cho hàm số $y=\dfrac{2x^2-3x+m}{x-m}$ có đồ thị $(C)$. Với tất cả các giá trị thực nào của tham số $m$ thì đồ thị $(C)$ không có tiệm cận đứng?
    \shortans{$m=0$ hoặc $m=1$}
    %	\choice
    %	{$m=2$}
    %	{$m=0$}
    %	{$m=1$}
    %	{\True $m=0$ hoặc $m=1$}
    \loigiai{
        Đồ thị không có tiệm cận đứng khi $x=m$ là nghiệm của phương trình $2x^2-3x+m=0$, suy ra $2m^2-3m+m=0 \Leftrightarrow \hoac{&m=0\\&m=1}$.
    }
\end{ex}
\begin{ex}%[2D1K4-2]%
    Với tất cả các giá trị thực nào của tham số $m$ thì đồ thị hàm số $y=\dfrac{x^2+x-2}{x^2+x+m}$ có ba đường tiệm cận?
    \shortans{$m<\dfrac{1}{4}$ và $m\ne -2$}
    %	\choice
    %	{$m>\dfrac{1}{4}$ và $m\ne 2$}
    %	{$m>\dfrac{1}{4}$}
    %	{$m<\dfrac{1}{4}$}
    %	{\True $m<\dfrac{1}{4}$ và $m\ne -2$}
    \loigiai{
        Đồ thị hàm số chỉ có $1$ tiệm cận ngang là $y=1$.\\
        Ta có $x^2+x-2 \Leftrightarrow \hoac{&x=1\\&x=-2.}$\\
        Đồ thị hàm số có ba đường tiệm cận khi và chỉ khi có $2$ tiệm cận đứng. Điều này tương đương với phương trình $x^2+x+m=0$ có $2$ nghiệm phân biệt khác $1$ và $-2$, nghĩa là\\
        $\heva{&1-4m>0\\&1^2+1+m \ne 0\\& (-2)^2-2+m\ne 0} \Leftrightarrow \heva{&m<\dfrac{1}{4}\\&m\ne -2.}$
    }
\end{ex}
\begin{ex}%[đề thi thử THPT Quốc gia, đề số 3, nguyễn hoàng thanh]%[2D1K4-2]%
    Tìm số giá trị nguyên thuộc đoạn $ [-2025;2025] $ của tham số $ m $ để đồ thị hàm số $ y=\dfrac{\sqrt{x-3}}{x^2+x-m} $ có đúng hai đường tiệm cận.
    \shortans{$2014$}
    %	\choice
    %	{$ 2007 $}
    %	{$ 2010 $}
    %	{$ 2009 $}
    %	{\True $ 2008 $}
    \loigiai{
        Điều kiện xác định của hàm số $ \heva{& x\ge 3\\& x^2+x-m\ne 0.} $\\
        Vì $ \lim\limits_{x\to +\infty}\dfrac{\sqrt{x-3}}{x^2+x-m}=\lim\limits_{x\to +\infty}\dfrac{\sqrt{\frac{1}{x}-\frac{3}{x^2 }}}{1+\frac{1}{x}-\frac{m}{x^2}}=0 $, suy ra $ y=0 $ là tiệm cận ngang.\\
        Để đồ thị hàm số có đúng hai tiệm cận thì đồ thị hàm số chỉ có thêm một tiệm cận đứng, tương đương $ f(x)=x^2+x-m $ có đúng một nghiệm lớn hơn $ 3 $. Xét các trường hợp xảy ra như sau
        \begin{enumerate}
            \item $ f(x)=0 $ có nghiệm kép $ x_{1}=x_2=-\dfrac{1}{2}<3 $ (không thỏa mãn).
            \item $ f(x)=0 $ có hai nghiệm thỏa $ x_1<3\le x_2\Leftrightarrow a\cdot f(3)\le 0\Leftrightarrow 12-m\le 0\Leftrightarrow m\ge 12 $.
        \end{enumerate}
        Kết hợp với yêu cầu bài toán ta suy ra $ \heva{&m\in \mathbb{Z}\\ &m\in[12;2025]} $, suy ra có $ 2025-12+1=2014 $ giá trị nguyên của $ m $ thỏa mãn bài toán.
    }
\end{ex}
\begin{ex}%[Đề tham khảo THPT Quốc gia 2021 - Đề 5]%[Đoàn Minh Tân]%[2D1K4-2]%
    Tìm tất cả giá trị thực của tham số $m$ để đồ thị hàm số $y=\dfrac{3x+2018}{\sqrt{mx^2+5x+6}}$ có hai đường tiệm cận ngang.
    \shortans{$m>0$}
    %	\choice
    %	{$m\in \varnothing$}
    %	{$m<0$}
    %	{$m=0$}
    %	{\True $m>0$}
    \loigiai{
        Ta có $\displaystyle \lim \limits_{x\to +\infty} y=\displaystyle\lim\limits_{x\to +\infty}\dfrac{3x+2018}{\sqrt{mx^2+5x+6}}=\displaystyle\lim\limits_{x\to +\infty}\dfrac{3+\dfrac{2018}{x}}{\sqrt{m+\dfrac{5}{x}+\dfrac{6}{x^2}}}=\dfrac{3}{\sqrt{m}}$ tồn tại khi $m>0$.\\
        $\displaystyle\lim\limits_{x\to -\infty}=\displaystyle\lim\limits_{x\to -\infty}\dfrac{3x+2018}{\sqrt{mx^2+5x+6}}=\displaystyle\lim\limits_{x\to -\infty}\dfrac{3+\dfrac{2018}{x}}{-\sqrt{m+\dfrac{5}{x}+\dfrac{6}{x^2}}}=-\dfrac{3}{\sqrt{m}}$ tồn tại khi $m>0$.\\
        Hiên nhiên $\displaystyle\lim\limits_{x\to +\infty}y\ne \displaystyle \lim \limits_{x\to -\infty}y$.\\
        Vậy đồ thị hàm số đã cho có hai tiệm cận ngang khi và chỉ khi $m>0$.
    }
\end{ex}
\begin{ex}%[2D1K4-2]%
    Có bao nhiêu giá trị nguyên của tham số thực $m$ thuộc đoạn $[-20; 10]$ để đồ thị hàm số $y=\dfrac{x+2}{\sqrt{x^2-4x+m}}$ có hai đường tiệm cận đứng?
    \shortans{$23$}
    %	\choice
    %	{$20$}
    %	{$21$}
    %	{$22$}
    %	{\True $23$}
    \loigiai{
        Đồ thị hàm số có hai đường tiệm cận đứng $\Leftrightarrow$ phương trình $x^2-4x+m=0$ có hai nghiệm phân biệt khác $-2$ \\
        $ \Leftrightarrow\heva{&2^2-m>0\\&(-2)^2-4\cdot (-2)+m\neq 0}\Leftrightarrow\heva{&m<4\\&m\neq-12.} $ \\
        Do $m$ nguyên và $m\in[-20; 10]$ nên $m\in\left\{-20;-19;\ldots;-13;-11;\ldots; 2; 3\right\}$, gồm $23$ giá trị thỏa mãn.}
\end{ex}
\Closesolutionfile{ans}
\begin{dang}{Tìm các đường tiệm cận đồ thị hàm ẩn}
\end{dang}
\begin{vd}
    Cho hàm số $y=f(x)$ có bảng biến thiên như hình vẽ sau
    \begin{center}
        \begin{tikzpicture}[>=stealth]
            \tkzTabInit[nocadre=false,lgt=1,espcl=1.5,deltacl=0.5]{$x$/.7 ,$y'$/.7,$y$/2}
            {$-\infty$ , $-1$ , $2$ , $+\infty$}
            \tkzTabLine{ , + , $0$ , - , d , + , }
            \tkzTabVar{-/$1$ , +/$4$ , -/$-5$ , +/$+\infty$}
        \end{tikzpicture}
    \end{center}
    Tìm TCĐ, TCN của đồ thị hàm số
    \begin{listEX}[3]
        \item $y=\dfrac{2}{f(x)-3}$
        \item $y=\dfrac{-3}{f(x)+2}$
        \item $y=\dfrac{x-2}{f(x)+5}$
        \item $y=\dfrac{x+1}{f(x)-4}$
        \item $y=\dfrac{2}{f(x^2)+3}$
        \item $y=\dfrac{4f(x)-5}{3f(x)+1}$
    \end{listEX}
    \loigiai{}
\end{vd}
\begin{vd}\immini{Cho hàm bậc ba $y=f(x)$ có đồ thị như hình vẽ. Tìm số tiệm cận đứng của đồ thị hàm số
        \begin{listEX}[2]
            \item $y=\dfrac{\sqrt{x+3}}{(x-1)f(x)}$
            \item $g(x)=\dfrac{(x^2+4x+3)\sqrt{x^2+x}}{x\left[f^2(x)-2f(x)\right]}$ .
    \end{listEX}}{\begin{tikzpicture}[line cap=round,line join=round, >=stealth,font=\footnotesize]
            \begin{scope}[scale=.5]
                \def\a{-1} % Hệ số a phải khác 0
                \def\b{-13/2}
                \def\c{-12}
                \def\d{-9/2}
                \draw[->] (-5,0) -- (2,0)node[below]{$x$};
                \draw[->] (0,-3) -- (0,4) node[left] {$y$};
                \draw (0,0)node[below right]{$O$} (-3,0)node[below]{$-3$};
                \draw[dashed] (-1,0)node[below]{$-1$}|-(0,2)node[right]{$2$};
                \draw[samples=150,smooth,domain=-4:.-.2] plot(\x,{\a*(\x)^3+(\b)*(\x)^2+(\c)*\x+(\d)});
            \end{scope}
    \end{tikzpicture}}
    \loigiai{
        \begin{center}
            \begin{tikzpicture}[line cap=round,line join=round, >=stealth,font=\footnotesize,scale=1]
                \def\a{-1} % Hệ số a phải khác 0
                \def\b{-13/2}
                \def\c{-12}
                \def\d{-9/2}
                \draw[->] (-5,0) -- (2,0)node[below]{$x$};
                \draw[->] (0,-3) -- (0,4) node[left] {$y$};
                \draw (0,0)node[below right]{$O$} (-3,0)node[below]{$-3$} (-.3,0)node[above]{$a$};
                \draw[dashed] (-3.78,0)node[below]{$c$}|-(0,2)|-(-1.71,0)node[below]{$b$}|-(0,2) (-1,0)node[below]{$-1$}|-(0,2)node[right]{$2$};
                \draw[samples=150,smooth,domain=-4:.-.2] plot(\x,{\a*(\x)^3+(\b)*(\x)^2+(\c)*\x+(\d)});
            \end{tikzpicture}
        \end{center}
        $g(x)=\dfrac{(x^2+4x+3)\sqrt{x^2+x}}{x\left[f^2(x)-2f(x)\right]}=\dfrac{(x+1)(x+3)\sqrt{x(x+1)}}{x\left[f^2(x)-2f(x)\right]}$.\\
        Điều kiện của căn là $x\le -1; x\ge 0$.\\
        Dựa vào đồ thị ta có \[x\left[f^2(x)-2f(x)\right]=0 \Leftrightarrow \hoac{&x=0\\&f(x)=0\\& f(x)=2} \Leftrightarrow \hoac{&x=0\text{ (nhận)}\\&x=-3\text{ (nhận)};\ x=a \text{ (loại)} \\&x=-1\text{ (nhận)};\ x=b\text{ (nhận)};\ x=c\text{ (nhận)}}\]\\
        Số TCĐ lúc này chính là số nghiệm không bị rút gọn của mẫu, vậy có bốn TCĐ là $x=0; x=-3; x=b; x=c$.
    }
\end{vd}
\BTTN
\Opensolutionfile{ans}[ans/2D1-4-DANG-3]
\begin{ex}%[2D1K4-1]
    Cho hàm số $y=f(x)$ có bảng biến thiên như hình bên. Đồ thị hàm số $y=\dfrac{-5}{f(x)+4}$ có bao nhiêu tiệm cận đứng?
    \begin{center}
        \begin{tikzpicture}[scale=0.8]
            \tkzTabInit[nocadre=false,lgt=1.5,espcl=3,deltacl=0.6]
            {$x$ /0.6,$y’$ /0.6,$y$ /2}
            {$-\infty$ ,$1$, $2$, $+\infty$}
            \tkzTabLine{,+,d,-,d,+,}
            \tkzTabVar{-/$-4$,+/$3$,-/$-5$,+/$+\infty$}
        \end{tikzpicture}
    \end{center}
    \choice
    {$1$}
    {$3$}
    {\True $2$}
    {$4$}
    \loigiai{
        Dựa vào bảng biến thiên suy ra
        $f(x)+4=0 \Leftrightarrow f(x) =-4$, phương trình này có $2$ nghiệm phân biệt nên đồ thị hàm số $y=\dfrac{-5}{f(x)+4}$ có $2$ tiệm cận đứng.
    }
\end{ex}
\begin{ex}%[2D1K4-1]
    Cho hàm số $y=f(x)$ có bảng biến thiên như hình bên. Đồ thị hàm số $y=\dfrac{x+2}{2f(x)-1}$ có bao nhiêu tiệm cận đứng?
    \begin{center}
        \begin{tikzpicture}[scale=0.8]
            \tkzTabInit[nocadre=false,lgt=1.5,espcl=3,deltacl=0.6]
            {$x$ /0.6,$y’$ /0.6,$y$ /2}
            {$-\infty$ ,$-1$, $0$, $1$, $+\infty$}
            \tkzTabLine{,+,0,-,0,+,0,-,}
            \tkzTabVar{-/$-\infty$,+/$0$,-/$-\dfrac{5}{3}$,+/$0$,-/$-\infty$}
        \end{tikzpicture}
    \end{center}
    \choice
    {$1$}
    {$3$}
    {$2$}
    {\True $0$}
    \loigiai{
        Dựa vào bảng biến thiên suy ra
        $2f(x)-1=0 \Leftrightarrow f(x) =\dfrac{1}{2}$, phương trình này có $0$ nghiệm nên đồ thị hàm số $y=\dfrac{x+2}{2f(x)-1}$ không có tiệm cận đứng.
    }
\end{ex}
%69
\begin{ex}%[2D1K4-1]
    Cho hàm số $y=f(x)$ có bảng biến thiên như hình bên. Đồ thị hàm số $y=\dfrac{1}{2f(x)-3}$ có bao nhiêu tiệm cận đứng?
    \begin{center}
        \begin{tikzpicture}[scale=0.8]
            \tkzTabInit[nocadre=false,lgt=1.5,espcl=3,deltacl=0.6]
            {$x$ /0.6,$y’$ /0.6,$y$ /2}
            {$-\infty$ ,$0$, $1$, $+\infty$}
            \tkzTabLine{,+,0,-,0,+,}
            \tkzTabVar{-/$-\infty$,+/$5$,-/$-1$,+/$+\infty$}
        \end{tikzpicture}
    \end{center}
    \choice
    {$1$}
    {\True $3$}
    {$2$}
    {$0$}
    \loigiai{
        Dựa vào bảng biến thiên suy ra
        $2f(x)-3=0 \Leftrightarrow f(x) =-\dfrac{3}{2}$, phương trình này có $3$ nghiệm phân biệt nên đồ thị hàm số $y=\dfrac{1}{2f(x)-3}$ có ba tiệm cận đứng.
    }
\end{ex}
%70
%71
%72
\begin{ex}%[2D1K4-1]
    Cho hàm số $y=f(x)$ có bảng biến thiên như hình bên. Đồ thị hàm số $y=\dfrac{x}{f(x)-3}$ có bao nhiêu tiệm cận đứng?
    \begin{center}
        \begin{tikzpicture}[scale=0.8]
            \tkzTabInit[nocadre=false,lgt=1.5,espcl=3,deltacl=0.6]
            {$x$ /0.6,$y’$ /0.6,$y$ /2}
            {$-\infty$ ,$-1$, $0$, $1$, $+\infty$}
            \tkzTabLine{,-,0,+,0,-,0,+,}
            \tkzTabVar{+/$+\infty$,-/$0$,+/$3$,-/$0$,+/$+\infty$}
        \end{tikzpicture}
    \end{center}
    \choice
    {$1$}
    {\True $3$}
    {$2$}
    {$4$}
    \loigiai{
        Dựa vào bảng biến thiên suy ra
        $f(x)-3=0 \Leftrightarrow f(x) =3$, phương trình này có $2$ nghiệm phân biệt khác $0$ và một nghiệm bội chẵn $x=0$ nên đồ thị hàm số $y=\dfrac{x}{f(x)-3}$ có ba tiệm cận đứng.
    }
\end{ex}
\begin{ex}%[2D1K4-1]
    Cho hàm số $y=f(x)$ có bảng biến thiên như hình bên. Đồ thị hàm số $y=\dfrac{4}{f(x)+1}$ có tiệm cận ngang là đường thẳng
    \begin{center}
        \begin{tikzpicture}[scale=0.8]
            \tkzTabInit[nocadre=false,lgt=1.5,espcl=3,deltacl=0.6]
            {$x$ /0.6,$y’$ /0.6,$y$ /2}
            {$-\infty$ ,$-1$, $2$, $+\infty$}
            \tkzTabLine{,+,0,-,0,+,}
            \tkzTabVar{-/$1$,+/$4$,-/$-5$,+/$1$}
        \end{tikzpicture}
    \end{center}
    \choice
    {$y=1$}
    {$y=-5$}
    {\True $y=2$}
    {$y=4$}
    \loigiai{
        Dựa vào bảng biến thiên suy ra
        $\lim \limits_{x \to \pm \infty} f(x)=1 \Leftrightarrow \lim \limits_{x \to \pm \infty} \dfrac{4}{f(x)+1} =2$ nên đồ thị hàm số đã cho có tiệm cận ngang là $y=2$.
    }
\end{ex}
\begin{ex}%[2D1K4-1]
    Cho hàm số $y=f(x)$ có bảng biến thiên như hình bên. Đồ thị hàm số $y=\dfrac{2-f(x)}{f(x)+3}$ có tiệm cận ngang là đường thẳng
    \begin{center}
        \begin{tikzpicture}[scale=0.8]
            \tkzTabInit[nocadre=false,lgt=1.5,espcl=3,deltacl=0.6]
            {$x$ /0.6,$y’$ /0.6,$y$ /2}
            {$-\infty$ ,$0$, $2$, $+\infty$}
            \tkzTabLine{,-,0,+,0,-,}
            \tkzTabVar{+/$+\infty$,-/$1$,+/$5$,-/$-\infty$}
        \end{tikzpicture}
    \end{center}
    \choice
    {$y=1$}
    {$y=-3$}
    {$y=2$}
    {\True $y=-1$}
    \loigiai{
        Dựa vào bảng biến thiên suy ra
        $\lim \limits_{x \to \pm \infty} f(x)=\pm \infty \Leftrightarrow \lim \limits_{x \to \pm \infty} \dfrac{2-f(x)}{f(x)+3} =-1$ nên đồ thị hàm số $y=\dfrac{2-f(x)}{f(x)+3}$ có tiệm cận ngang là $y=-1$.
    }
\end{ex}
\begin{ex}%[2D1K4-1]
    Cho hàm số $y=f(x)$ có bảng biến thiên như hình bên. Đồ thị hàm số $y=\dfrac{1}{f^2(x)-4f(x)+4}$ có bao nhiêu tiệm cận đứng?
    \begin{center}
        \begin{tikzpicture}[scale=0.8]
            \tkzTabInit[nocadre=false,lgt=1.5,espcl=3,deltacl=0.6]
            {$x$ /0.6,$y’$ /0.6,$y$ /2}
            {$-\infty$, $2$, $+\infty$}
            \tkzTabLine{,-,0,+,}
            \tkzTabVar{+/$1$,-/$-3$,+/$1$}
        \end{tikzpicture}
    \end{center}
    \choice
    {$1$}
    {$3$}
    {$2$}
    {$0$}
    \loigiai{
        Dựa vào bảng biến thiên suy ra $f^2(x)-4f(x)+4=0 \Leftrightarrow f(x)=2$, phương trình $f(x)=2$ vô nghiệm nên đồ thị hàm số đã cho không có tiệm cận đứng.
    }
\end{ex}
%83
\begin{ex}%[2D1K4-1]
    Cho hàm số $y=f(x)$ có bảng biến thiên như hình bên. Đồ thị hàm số $y=\dfrac{1}{f(3-x)-2}$ có bao nhiêu tiệm cận đứng?
    \begin{center}
        \begin{tikzpicture}[scale=0.8]
            \tkzTabInit[nocadre=false,lgt=1.5,espcl=3,deltacl=0.6]
            {$x$ /0.6,$y’$ /0.6,$y$ /2}
            {$-\infty$ ,$-2$, $2$, $+\infty$}
            \tkzTabLine{,+,0,-,0,+,}
            \tkzTabVar{-/$-\infty$,+/$3$,-/$0$,+/$+\infty$}
        \end{tikzpicture}
    \end{center}
    \choice
    {$1$}
    {\True $3$}
    {$2$}
    {$0$}
    \loigiai{
        Dựa vào bảng biến thiên suy ra $f(3-x)-2=0 \Leftrightarrow f(3-x)=2$, phương trình này có $3$ nghiệm phân biệt nên đồ thị hàm số đã cho có $3$ tiệm cận đứng.
    }
\end{ex}
\begin{ex}%[2D1G4-1]
    Cho hàm số $y=f(x)$ có bảng biến thiên như hình bên. Đồ thị hàm số $y=\dfrac{4}{f(x^2)-2}$ có bao nhiêu tiệm cận đứng?
    \begin{center}
        \begin{tikzpicture}[scale=0.8]
            \tkzTabInit[nocadre=false,lgt=1.5,espcl=3,deltacl=0.6]
            {$x$ /0.6,$y’$ /0.6,$y$ /2}
            {$-\infty$ ,$0$, $3$, $+\infty$}
            \tkzTabLine{,-,0,+,d,-,}
            \tkzTabVar{+/$8$,-/$1$,+/$4$,-/$2$}
        \end{tikzpicture}
    \end{center}
    \choice
    {$5$}
    {$3$}
    {\True $2$}
    {$4$}
    \loigiai{
        Dựa vào bảng biến thiên suy ra
        $f(x^2)-2=0 \Leftrightarrow f(x^2) =2$. Kẻ đường thẳng $y=2$ ta thấy đường thẳng cắt đồ thị hàm số tại hai điểm phân biệt. Suy ra
        $$\hoac{&x^2=a \; (a<0)\\&x^2=b \; (b >0)} \Rightarrow x=\pm \sqrt{b}.$$
        Do đó đồ thị hàm số đã cho có $2$ tiệm cận đứng.
    }
\end{ex}%89
\begin{ex}%[2D1G4-1]
    Cho hàm số $y=f(x)$ có bảng biến thiên như hình bên. Đồ thị hàm số $y=\dfrac{2}{f(|x|)-3}$ có bao nhiêu tiệm cận ngang?
    \begin{center}
        \begin{tikzpicture}[scale=0.8]
            \tkzTabInit[nocadre=false,lgt=1.5,espcl=3,deltacl=0.6]
            {$x$ /0.6,$y’$ /0.6,$y$ /2}
            {$-\infty$ ,$0$, $2$, $+\infty$}
            \tkzTabLine{,+,0,-,0,+,}
            \tkzTabVar{-/$-\infty$,+/$3$,-/$-1$,+/$+\infty$}
        \end{tikzpicture}
    \end{center}
    \choice
    {$4$}
    {\True $3$}
    {$5$}
    {$6$}
    \loigiai{
        Dựa vào bảng biến thiên suy ra
        $f(|x|)-3=0 \Leftrightarrow f(|x|) =3$.\\
        Bảng biến thiên hàm số $y=f(|x|)$ như sau
        \begin{center}
            \begin{tikzpicture}[scale=0.8]
                \tkzTabInit[nocadre=false,lgt=1.5,espcl=3,deltacl=0.6]
                {$x$ /0.6,$y’$ /0.6,$y$ /2}
                {$-\infty$ ,$-2$, $0$, $2$, $+\infty$}
                \tkzTabLine{,-,0,+,0,-,0,+,}
                \tkzTabVar{+/$+\infty$,-/$-1$,+/$3$,-/$-1$,+/$+\infty$}
            \end{tikzpicture}
        \end{center}
        Dựa vào bảng biến thiên hàm số $y=f(|x|)$, phương trình $f(|x|) =3$ có ba nghiệm phân biệt, do đó đồ thị hàm số $y=\dfrac{2}{f(|x|)-3}$ có $3$ tiệm cận đứng.
    }
\end{ex}
\begin{ex}
    \immini{ %Câu 90
        Cho hàm số bậc ba $f(x)= ax^3 +bx^2 +cx +d$ có đồ thị như hình vẽ bên. Đồ thị hàm số $g(x) = \dfrac{\sqrt{x+1}}{(x-3)\cdot f(x)}$ có bao nhiêu đường tiệm cận đứng?
        \choice
        {5}
        {2}
        {4}
        {\True 3}}{\begin{tikzpicture}[scale=.5, font=\footnotesize, line join=round, line cap=round, >=stealth]
            \def\xmin{-3}\def\xmax{3}\def\ymin{-5}\def\ymax{1}
            \draw[->] (\xmin-0.2,0)--(\xmax+0.2,0) node[below] {\footnotesize $x$};
            \draw[->] (0,\ymin-0.2)--(0,\ymax+0.2) node[right] {\footnotesize $y$};
            \draw (0,0) node [below left] {\footnotesize $O$};
            \foreach \x in {-1}\draw (\x,-0.1)--(\x,0.1) node [above] {\footnotesize $\x$};
            \foreach \x in {2}\draw (\x,-0.1)--(\x,0.1) node [above right] {\footnotesize $\x$};
            \foreach \y in {}\draw (-0.1,\y)--(0.1,\y) node [right] {\footnotesize $\y$};
            \clip (\xmin,\ymin) rectangle (\xmax,\ymax);
            \draw[smooth,samples=200,domain=\xmin:\xmax] plot (\x,{1*((\x)^3)+0*((\x)^2)+-3*(\x)+-2});
        \end{tikzpicture}
    }
    \loigiai{
        * Điều kiện: $\heva{&x \ne 3\\&f(x) \ne 0\\&x \ge -1.}$\\
        Nhìn hình vẽ ta thấy
        $f(x)=0\Leftrightarrow \hoac{&x=-1&(\text{nghiệm kép}) \\&x=2&(\text{nghiệm đơn}).}$\\
        Vậy $g(x) = \dfrac{\sqrt{x+1}}{(x-3)\cdot a(x+1)^2 (x-2)}.$ \\
        Đồ thị hàm số $g(x)$ có 3 đường tiệm cận đứng.}
\end{ex}
\begin{ex}
    \immini{ %Câu 92.
        Đường cong ở hình bên là đồ thị của hàm số $y = ax^3 +bx^2 +cx+d$. Đồ thị hàm số $y =\dfrac{(2x+1)\sqrt{x-1}}{x\cdot f(x-2)}$ có tất cả bao nhiêu tiệm cận đứng?
        \choice
        {1}
        {3}
        {4}
        {\True 2}}{\begin{tikzpicture}[scale=.6, font=\footnotesize, line join=round, line cap=round, >=stealth]
            \def\xmin{-3}\def\xmax{3}\def\ymin{-3}\def\ymax{3}
            \draw[->] (\xmin-0.2,0)--(\xmax+0.2,0) node[below] {\footnotesize $x$};
            \draw[->] (0,\ymin-0.2)--(0,\ymax+0.2) node[right] {\footnotesize $y$};
            \draw (0,0) node [below left] {\footnotesize $O$};
            \foreach \x in {-2}\draw (\x,-0.1)--(\x,0.1) node [above left] {\footnotesize $\x$};
            \foreach \x in {2}\draw (\x,-0.1)--(\x,0.1) node [above right] {\footnotesize $\x$};
            \foreach \y in {}\draw (-0.1,\y)--(0.1,\y) node [right] {\footnotesize $\y$};
            \clip (\xmin,\ymin) rectangle (\xmax,\ymax);
            \draw[smooth,samples=200,domain=\xmin:\xmax] plot (\x,{(2/3)*((\x)^3)+0*((\x)^2)+-(8/3)*(\x)});
    \end{tikzpicture}}
    \loigiai{
        * Điều kiện: $\heva{&x \ne 0\\&f(x-2) \ne 0\\&x \ge 1.}$\\
        Nhìn hình vẽ ta thấy
        $f(x-2)=0\Leftrightarrow \hoac{&x-2=-2\\&x-2=0\\&x-2=2}\Leftrightarrow \hoac{&x=0&(\text{không thỏa mãn})\\&x=2&(\text{nghiệm đơn})\\&x=4&(\text{nghiệm đơn}).}$\\
        Vậy $g(x) =\dfrac{(2x+1)\sqrt{x-1}}{x\cdot f(x-2)}=\dfrac{(x-1)\sqrt{x+2}}{x\cdot ax(x-2)(x-4)}.$ \\
        Đồ thị hàm số $g(x)$ có 2 đường tiệm cận đứng.}
\end{ex}
\begin{ex}
    \immini{ %Câu 93.
        Cho hàm số $y= f(x)$ có đồ thị cắt trục hoành tại đúng 3 điểm như hình bên. Đồ thị hàm số $y =\dfrac{(x+2)\sqrt{3-x}}{f(|x|)}$
        có tất cả bao nhiêu tiệm cận đứng?
        \choice
        {1}
        {3}
        {4}
        {\True 2}}{\begin{tikzpicture}[scale=.5, font=\footnotesize, line join=round, line cap=round, >=stealth]
            \def\xmin{-2}\def\xmax{5}\def\ymin{-3}\def\ymax{5}
            \draw[->] (\xmin-0.2,0)--(\xmax+0.2,0) node[below] {\footnotesize $x$};
            \draw[->] (0,\ymin-0.2)--(0,\ymax+0.2) node[right] {\footnotesize $y$};
            \draw (0,0) node [below left] {\footnotesize $O$};
            \foreach \x in {-1,2,4}\draw (\x,-0.1)--(\x,0.1) node [above left] {\footnotesize $\x$};
            \foreach \y in {}\draw (-0.1,\y)--(0.1,\y) node [right] {\footnotesize $\y$};
            \clip (\xmin,\ymin) rectangle (\xmax,\ymax);
            \draw[smooth,samples=200,domain=-1.2:0] plot(\x,{0-8.48*(\x)^(2.0)-5.48*(\x)+3.0});
            \draw[smooth,samples=200,domain=0:2]
            plot(\x,{0-2.7989489689153735*(\x)^(3.0)+8.326740175055514*(\x)^(2.0)-6.957684474449535*(\x)+3.0});
            \draw[smooth,samples=200,domain=2:5]
            plot(\x,{2.395330112721417*(\x)^(2.0)-14.371980676328501*(\x)+19.162640901771336});
    \end{tikzpicture}}
    \loigiai{
        * Điều kiện: $\heva{&f(|x|) \ne 0\\&x \le 3.}$\\
        Nhìn hình vẽ ta thấy
        $f(|x|)=0\Leftrightarrow \hoac{&|x|=-1\\&|x|=2\\&|x|=4}\Leftrightarrow \hoac{&x=\pm 2&(\text{nghiệm đơn})\\&x=- 4&(\text{nghiệm đơn})\\&x=4&(\text{không thỏa mãn}).}$\\
        Vậy $y =\dfrac{(x+2)\sqrt{3-x}}{a(x-2)(x+2)(x+4)(x-4)}$ \\
        Đồ thị hàm số có 2 đường tiệm cận đứng.}
\end{ex}
\begin{ex}
    \immini{ %Câu 94.
        Đường cong ở hình bên là đồ thị của hàm số $y = ax^3 +bx^2 +cx+d$. Đồ thị hàm số $y =\dfrac{(2x+1)\sqrt{1-x}}{f(|x|)}$ có tất cả bao nhiều tiệm cận đứng?
        \choice
        { 1}
        {3}
        {4}
        {\True 2}}{\begin{tikzpicture}[scale=.8, font=\footnotesize, line join=round, line cap=round, >=stealth]
            \def\xmin{-1}\def\xmax{2}\def\ymin{-1.5}\def\ymax{1.5}
            \draw[->] (\xmin-0.2,0)--(\xmax+0.2,0) node[below] {\footnotesize $x$};
            \draw[->] (0,\ymin-0.2)--(0,\ymax+0.2) node[right] {\footnotesize $y$};
            \draw (0.15,0) node [below left] {\footnotesize $O$};
            \foreach \x in {}\draw (\x,0.1)--(\x,-0.1) node [below] {\footnotesize $\x$};
            \foreach \y in {-1,1}\draw (0.1,\y)--(-0.1,\y) node [left] {\footnotesize $\y$};
            \clip (\xmin,\ymin) rectangle (\xmax,\ymax);
            \draw[smooth,samples=200,domain=\xmin:\xmax] plot (\x,{4*((\x)^3)+-6*((\x)^2)+0*(\x)+1});
            \draw[dashed] (0.5,0)--(0.5,0.0)--(0,0.0);
            \draw (0.5,-1pt)--(0.5,1pt) node [above] {\footnotesize $\frac{1}{2}$};
            \draw (-0.7,-1pt)--(-0.7,1pt) node [above] {\footnotesize $-\frac{1}{2}$};
            \draw (1,-1pt)--(1,1pt) node [above] {\footnotesize $1$};
            \draw[dashed] (0.0,0)--(0.0,1.0)--(0,1.0);
            \draw[dashed] (1.0,0)--(1.0,-1.0)--(0,-1.0);
    \end{tikzpicture}}
    \loigiai{
        * Điều kiện: $\heva{&f(|x|) \ne 0\\&x \le 1.}$\\
        Nhìn hình vẽ ta thấy
        $f(|x|)=0\Leftrightarrow \hoac{&|x|=-\dfrac{1}{2}\\&|x|=\dfrac{1}{2}\\&|x|=x_1>1}\Leftrightarrow \hoac{&x=\pm \dfrac{1}{2}&(\text{hai nghiệm đơn})\\&x=- x_1&(\text{nghiệm đơn})\\&x=x_1&(\text{không thỏa mãn}).}$\\
        Vậy $y =\dfrac{(2x+1)\sqrt{1-x}}{f(|x|)}=\dfrac{(2x+1)\sqrt{1-x}}{a\left(x-\dfrac{1}{2}\right)\left(x+\dfrac{1}{2}\right)(x+x_1)(x-x_1)}$ \\
        Đồ thị hàm số có 2 đường tiệm cận đứng.}
\end{ex}
\begin{ex}
    \immini{ %Câu 96.
        Cho đồ thị hàm số $y =f(x)$ và trục hoành có đúng 2 điểm chung như hình bên. Đồ thị hàm số $y =\dfrac{(x-1)\sqrt{3-x}}{f(x^2)}$ có tất cả bao nhiêu tiệm cận đứng?
        \choice
        {1}
        {3}
        {4}
        {\True 2}}{\begin{tikzpicture}[scale=.8, font=\footnotesize, line join=round, line cap=round, >=stealth]
            \def\xmin{-1.5}\def\xmax{2}\def\ymin{-1}\def\ymax{4.5}
            \draw[->] (\xmin-0.2,0)--(\xmax+0.2,0) node[below] {\footnotesize $x$};
            \draw[->] (0,\ymin-0.2)--(0,\ymax+0.2) node[right] {\footnotesize $y$};
            \draw (0,0) node [below left] {\footnotesize $O$};
            \foreach \x in {1}\draw (\x,0.1)--(\x,-0.1) node [below] {\footnotesize $\x$};
            \foreach \x in {-1}\draw (\x,0.1)--(\x,-0.1) node [below left] {\footnotesize $\x$};
            \clip (\xmin,\ymin) rectangle (\xmax,\ymax);
            \draw[smooth,samples=200,domain=-1.1:0] plot(\x,{21.044670464836045*(\x)^(3.0)+24.701786337609526*(\x)^(2.0)+5.65711587277348*(\x)+2.0});
            \draw[smooth,samples=200,domain=0:\xmax] plot(\x,{10.591704641658401*(\x)^(3.0)-19.26315454354621*(\x)^(2.0)+6.6714499018878115*(\x)+2.0});
    \end{tikzpicture}}
    \loigiai{
        * Điều kiện: $\heva{&f(x^2) \ne 0\\&x \le 3.}$\\
        Nhìn hình vẽ ta thấy
        $f(x^2)=0\Leftrightarrow \hoac{&x^2=-1\\&x^2=1}\Leftrightarrow x=\pm 1\,(\text{nghiệm kép}).$\\
        Vậy $y=\dfrac{(x-1)\sqrt{3-x}}{f(x^2)}=\dfrac{(x-1)\sqrt{3-x}}{(x-1)^2(x+1)^2}$ \\
        Đồ thị hàm số có 2 đường tiệm cận đứng.}
\end{ex}
\begin{ex}%[2D1G4-3]%Câu 52
    Cho hàm số $y=ax^3+bx^2+cx+d$ có đồ thị như hình vẽ. Đồ thị của hàm số $g(x)=\dfrac{x^2-x}{f^2(x)-2f(x)}$ có bao nhiêu đường tiệm cận đứng?
    \choice
    {$2$}
    {$3$}
    {\True $4$}
    {$5$}
    \begin{center}
        \begin{tikzpicture}[thick,>=stealth,x=1cm,y=1cm,scale=.7]
            \draw[thin,color=gray!50] (-3.3,-1.3) grid (3.9,5.9);
            \draw[->] (-3.2,0) -- (4.2,0) node[right] {$x$};
            \draw[->] (0,-1.2) -- (0,5.2) node[above] {$y$};
            \draw[color=blue, domain=-2.15:2.15,samples=300] plot (\x,{(\x)^3-3*(\x)+2}) node[right] {$y=f(x)$};
            \draw (-2,0) circle (1.5pt) node[below left]{$-2$};
            \draw (-1,0) circle (1.5pt) node[below]{$-1$};
            \draw (0,0) circle (1.5pt) node[above left]{$O$};
            \draw (1,0) circle (1.5pt) node[below]{$1$};
            \draw (0,4) circle (1.5pt) node[right]{$4$};
            \draw (-1,4) circle (1.5pt);
            \draw[dashed] (-1,0)--(-1,4)--(0,4);
            \draw[red] (-3,2)--(3.2,2);
            \draw[red] (3.5,2) node[right]{$f(x)=2$};
        \end{tikzpicture}
    \end{center}
    \loigiai{
        Xét phương trình $f^2(x)-2f(x)=0 \Leftrightarrow \hoac{&f(x)=0\\&f(x)=2}\Leftrightarrow \hoac{&x=1 \, (\textrm{nghiệm kép trùng nghiệm đơn ở tử số})\\&x=-2\, (\textrm{nghiệm đơn khác nghiệm của tử})\\&x=a\in(-2; -1)\\&x=0\, (\textrm{nghiệm đơn trùng nghiệm ở tử})\\&x=b\in(1; 2)}$\\
        \textbf{Kết luận:} Đồ thị hàm số có $4$ đường tiệm cận đứng.
    }
\end{ex}
\begin{ex}%[Thi thử L3, Lương Thế Vinh, Hà Nội, 2018]%[Phạm Toàn, Dự án (12EX-10)]%[2D1G4-3]%
    \immini{Cho hàm số $y=f(x)$ có đạo hàm liên tục trên $\mathbb{R}$. Đồ thị hàm $f(x)$ như hình vẽ. Số đường tiệm cận đứng của đồ thị hàm số $y=\dfrac{x^2-1}{f^2(x)-4f(x)}$ bằng
        \choice
        {$3$}
        {$1$}
        {$2$}
        {\True $4$}
    }{\begin{tikzpicture}[>=stealth,x=1cm,y=0.75cm,scale=0.7]
            \draw[->] (-2.5,0)--(0,0)%
            node[below right]{$O$}--(2.5,0) node[below]{$x$};
            \draw[->] (0,-2) --(0,5) node[right]{$y$};
            \foreach \x in {-1,1}{
                \draw (\x,0) node[below]{\footnotesize $\x$} circle (1pt);%Ox
            }
            \foreach \y in {2,4}{
                \draw (0,\y) node[right]{\footnotesize $\y$} circle (1pt);%Oy
            }
            \draw[samples=100,domain=-2.05:2] plot (\x,{(\x -1)^2*(\x+2)});
            \draw [dashed] (-1,0)--(-1,4)--(0,4);
            \draw(-1,4) circle (1pt);
    \end{tikzpicture}}
    \loigiai{Xét $f^2(x)-4f(x)=0\Leftrightarrow \hoac{& f(x)=0\\ &f(x)=4.}$\\
        Xét $f(x)=0$ có hai nghiệm, nghiệm $x_1\ne \pm 1$ và nghiệm $x_2=1$ là nghiệm bội (do đồ thị tiếp xúc với trục hoành tại $x=1$. Trường hợp này có $2$ tiệm cận đứng.\\
        Xét $f(x)=4$ có hai nghiệm, nghiệm $x_3\ne \pm 1$ và nghiệm $x_4=-1$ là nghiệm bội (do đồ thị tiếp xúc với đường thẳng $y=4$ tại $x=-1$. Trường hợp này có $2$ tiệm cận đứng.\\
        Vậy đồ thị có $4$ tiệm cận đứng.}
\end{ex}
\begin{ex}%[Thi thử, Trường THPT Lý Thái Tổ - Bắc Ninh, 2019]%[Duong Xuan Loi, 12EX3]%[2D1G4-3]%
    \immini{
        Cho hàm số $f(x)$ có đồ thị như hình bên. Số đường tiệm cận đứng của đồ thị hàm số
        $y=\dfrac{(x^2-4)(x^2+2x)}{[f(x)]^2+2f(x)-3}$ là
        \choice
        {\True $4$}
        {$5$}
        {$3$}
        {$2$}
    }{
        \begin{tikzpicture}[scale=0.5, font=\footnotesize, line join=round, line cap=round, >=stealth]
            \def\a{1} \def\b{-8} \def\c{1} % Hệ số
            \def\xt{-3.7} \def\xp{4} \def\yt{2} \def\yd{-3.7} % x_trái, x_phải, y_trên, y_dưới (giới hạn)
            \draw[->] (\xt,0)--(\xp,0) node [below]{$x$};
            \draw[->] (0,\yd)--(0,\yt) node [left]{$y$};
            \node at (0,0) [below left]{$O$};
            \clip (\xt-0.1,\yd+0.1) rectangle (\xp-0.1,\yt-0.1);
            \draw[smooth,samples=300] plot(\x,{1/4*(\a*(\x)^4+\b*(\x)^2)+\c});
            \draw[dashed] (-2,0)node[above]{$-2$}--(-2,-3)--(2,-3)--(2,0)node[above]{$2$};
            \node at (0,-3)[above left]{$-3$};
            \node at (-3,0)[above left]{$-3$};
            \node at (0,1)[above right]{$1$};
            \node at (3,0)[above right]{$3$};
            \fill (0,0) circle (1pt) (0,-3) circle (1pt) (2,0) circle (1pt) (-2,0) circle (1pt) (-3,0) circle (1pt) (0,1) circle (1pt) (3,0) circle (1pt);
        \end{tikzpicture}
    }
    \loigiai{
        Ta có $y=\dfrac{(x^2-4)(x^2+2x)}{[f(x)]^2+2f(x)-3}$ có các nghiệm ở tử là $x=0$ (bội $1$), $x=2$ (bội $1$), $x=-2$ (bội $2$).\\
        Mặt khác, từ đồ thị $f(x)$ ta thấy hàm số $y=\dfrac{(x^2-4)(x^2+2x)}{[f(x)]^2+2f(x)-3}$ có các nghiệm ở mẫu là
        $f^2(x)+2f(x)-3=0\Leftrightarrow \hoac{& f(x)=1 \\ & f(x)=-3}
        \Leftrightarrow \hoac{& x=0,x=x_1,x=x_2 \\ & x=-2,x=2.}$\\
        Trong đó nghiệm $x=0$, $x=-2$, $x=2$ đều có bội $2$ và $x_1$, $x_2$ khác các nghiệm của tử.\\
        So sánh bội nghiệm ở mẫu và bội nghiệm ở tử thì thấy đồ thị có các tiệm cận đứng là $x=0$, $x=2$; $x=x_1$; $x=x_2$.
    }
\end{ex}
\begin{ex}%[Thi thử, THPT Sơn Tây, Hà Nội, 2019]%[Huỳnh Xuân Tín, 12EX3]%[2D1G4-3]%
    \immini{Cho hàm số $ f(x)=(x+3)(x+1)^2(x-1)(x-3)$ có đồ thị như hình vẽ. Đồ thị hàm số $ g(x)=\dfrac{\sqrt{x-1}}{f^2(x)-9f(x)}$ có bao nhiêu tiệm cận đứng và tiệm cận ngang?
        \choice
        {$3$}
        {\True$ 4$}
        {$ 9$}
        { $8$}
    }{\begin{tikzpicture}[scale=0.3, font=\footnotesize, line join=round, line cap=round, >=stealth]
            %\draw[dashed, line width=0.1pt, gray] (-3.2,-5.5) grid (5.2,4.5);
            \draw[->] (-3.5,0)--(0,0) node[below right]{$O$}--(3.6,0) node[below]{$x$};
            \draw[fill=black] (0,0) circle (1pt);
            \draw[->] (0,-7.7) --(0,6.5) node[right]{$y$};
            \foreach \x in {-3,-1,3}{
                \draw[fill=black] (\x,0) node[below left]{$\x$} circle (1pt);}
            \draw[fill=black] (1,0) node[below right]{$1$} circle (1pt);
            \draw[fill=black] (0,1.35) node[above left]{$9$} circle (1pt);
            \draw [black, domain=-3.2:3.18, samples=100] %
            plot(\x,{0.15*(\x+3)*(\x+1)^2*(\x-1)*(\x-3)});
    \end{tikzpicture}}
    \loigiai{Điều kiện xác định của hàm số $g(x)$ là $\heva{&x\ge1\\ &f^2(x)-9f(x)\not=0.}$\\
        Từ $f^2(x)-9f(x)=0\Leftrightarrow \hoac{&f(x)=0\\&f(x)=9.}$\\
        Với $f(x)=0$ có nghiệm là $x=\pm 1, x=\pm 3$.\\
        Dựa vào đồ thị ta thấy nghiệm của phương trình $f(x)=9$ là hoành độ giao điểm của đường thẳng $y=9$ với đồ thị hàm số $y=f(x)$ nên có nghiệm là $-3<x_3<x_2<-1<0<x_1<1<3<x_0$.\\
        Do đó tập xác định của hàm số $y=g(x)$ là $\mathscr{D}=\left[1;+\infty \right)\setminus\left\lbrace1;3;x_0 \right\rbrace $.\\
        Khi đó ta có \begin{itemize}
            \item $\lim\limits_{x\rightarrow1^+ } g(x)=\lim\limits_{x\rightarrow1^+ }\dfrac{\sqrt{x-1}}{f(x)\left(f(x)-9 \right)}=+\infty$ (vì $x$ tiến gần bên phải $1$ thì $f(x)<0, f(x)-9<0$), suy ra đường thẳng $x=1$ là tiệm cận đứng.
            \item $\lim\limits_{x\rightarrow3^+ } g(x)=\lim\limits_{x\rightarrow3^+ }\dfrac{\sqrt{x-1}}{f(x)\left(f(x)-9 \right)}=-\infty$ (vì $x$ tiến gần bên phải $3$ thì $f(x)>0, f(x)-9<0$), suy ra đường thẳng $x=3$ là tiệm cận đứng.
            \item $\lim\limits_{x\rightarrow x_0^+} g(x)=\lim\limits_{x\rightarrow x_0^+ }\dfrac{\sqrt{x-1}}{f(x)\left(f(x)-9 \right)}=+\infty$ (vì $x$ tiến gần bên phải $x_0$ thì $f(x)>0, f(x)-9>0$), suy ra đường thẳng $x=x_0$ là tiệm cận đứng.
        \end{itemize}
        Và $\lim\limits_{x\rightarrow +\infty} g(x)=\lim\limits_{x\rightarrow +\infty }\dfrac{\sqrt{x-1}}{f(x)\left(f(x)-9 \right)}=0$ (vì bậc ở mẫu của $y=g(x)$ là $10$ và bậc tử của nó là $\dfrac{1}{2}$). Do vậy đồ thị hàm số $y=g(x)$ có một tiệm cận ngang là đường thẳng $y=0$.\\
        Vậy đồ thị hàm số $y=g(x)$ có bốn tiệm cận ngang và đứng. }
\end{ex}
\begin{ex}%[Thi thử, Chuyên Quang Trung-Bình Phước, 2021,lần 1]%[Trần Hòa, 12EX6]%[2D1G4-3]%
    \immini{Cho hàm số $y=f(x)=ax^3+bx^2+cx+d$, có đồ thị như hình vẽ. Số đường tiệm cận đứng của đồ thị hàm số $y=\dfrac{x^2+x-2}{f^2(x)-f(x)}$ là
        \choice
        {$3$}
        {$2$}
        {\True $4$}
        {$5$}}
    {\begin{tikzpicture}[scale=.5, font=\footnotesize, line join=round, line cap=round, >=stealth]
            \draw[->] (-2.5,0)--(0,0) node[below right]{$O$}--(2,0) node[below]{$x$};
            \draw[->] (0,-.5) --(0,4.5) node[right]{$y$};
            \draw [domain=-2.05:2.05, samples=100] %
            plot (\x, {(\x+2)*(\x-1)^2});
            \draw[fill] (0,0) circle (1pt);
            \foreach \x/\g in {-2/140,-1/-90,1/-90}
            \draw[fill] (\x,0) circle(.5pt)node [shift={(\g:.3)}] {$\x$};
            \foreach \y/\g in {2/0,4/0}
            \draw[fill] (0,\y) circle(.5pt)node [shift={(\g:.3)}] {$\y$};
            \draw[dashed] (-1,0)--(-1,4)--(0,4);
    \end{tikzpicture}}
    \loigiai{
        \begin{itemize}
            \item $x^2+x-2=(x-1)(x+2)$.\\
            \item Dựa vào đồ thị hàm số $y=f(x)$ ta có $f^2(x)-f(x)=0\Leftrightarrow\hoac{&f(x)=0\\&f(x)=1.}$\\
            $f(x)=0\Leftrightarrow x=-2$, $x=1$ (nghiệm kép).\\
            $f(x)=1\Leftrightarrow\hoac{&x=x_1,(x_1\in (-2;-1))\\&x=x_2,(x_2\in (0;1))\\&x=x_3,(x_3>1). }$
            \item Do đó $y=\dfrac{(x-1)(x+2)}{a^2(x+2)(x-1)^2(x-x_1)(x-x_2)(x-x_3)}$.
        \end{itemize}
        Suy ra đồ thị có các đườn tiệm cận đứng $x=1$, $x=x_1$, $x=x_2$, $x=x_3$.
    }
\end{ex}
\begin{ex}%[Đề thi hết học kì 2, Bình Minh, Ninh Bình 2018]%[Nguyễn Tuấn Anh, dự án EX9]%[2D1G4-3]%
    \immini{Cho hàm số bậc ba $f(x)=ax^3+bx^2+cx+d$ có đồ thị như hình vẽ bên dưới. Hỏi đồ thị hàm số $g(x)=\dfrac{(x^2-3x+2)\sqrt{x-1}}{x[f^2(x)-f(x)]}$ có bao nhiêu tiệm cận đứng?
        \choice
        {$5$}
        {$6$}
        {\True $3$}
        {$4$}
    }{
        \begin{tikzpicture}[line width=1.0pt,line join=round,>=stealth,x=1cm,y=1cm,scale=1.0]
            \draw[->,line width = 1pt] (-1,0)--(0,0) node[below right]{$O$}--(4,0) node[below]{$x$};
            \draw[->,line width = 1pt] (0,-1.5) --(0,2.5) node[right]{$y$};
            \foreach \x in {1,2}{
                \draw (\x,0) node[below]{$\x$} circle (1pt);
            }
            \foreach \y in {1}{
                \draw (0,\y) node[left]{$\y$} circle (1pt);
            }
            \clip(-0.8,-1) rectangle (3.8,2.3);
            \draw [line width=1.0pt, thick, domain=-0.5:3.5, samples=100]%,domain=-1.5:3] %
            plot (\x, {(5*(\x)-4)*((\x)-2)^2});
            \draw [dash pattern=on 4pt off 4pt] (1.,0.)-- (1.,1.)-- (0.,1.);
            \draw (1,1) circle (1pt);
        \end{tikzpicture}
    }
    \loigiai{
        Điều kiện $\heva{&x\geq 1\\ &x\ne 0\\ &f^2(x)-f(x)\ne 0}\Leftrightarrow \heva{&x\geq 1\\ &f(x)\ne 0\\ & f(x)\ne 1.}$\\
        Dựa vào đồ thị hàm số $y=f(x)$, ta thấy $f(x)=0$ có hai nghiệm, một nghiệm $x_1<1$ và một nghiệm kép bằng $2$. Do đó ta biểu diễn được $f(x)$ dưới dạng
        $$ f(x)=a(x-x_1)(x-2)^2. $$
        Dựa vào đồ thị hàm số $y=f(x)$, ta thấy phương trình $f(x)=1$ có ba nghiệm $1,x_2, x_3$, với $1<x_2<2<x_3$. Do đó ta biểu diễn được $f(x)-1$ dưới dạng
        $$ f(x)-1=a(x-1)(x-x_2)(x-x_3). $$
        Lúc này điều kiện được viết lại như sau $\heva{&x>1\\ &x\ne x_2, x\ne 2, x\ne x_3.}$\\
        Với điều kiện đó thì $g(x)$ được viết lại là
        $$ g(x)=\dfrac{\sqrt{x-1}}{a^2x(x-x_1)(x-x_2)(x-2)(x-x_3)}. $$
        Ta có
        \begin{align*}
            &\lim\limits_{x\to 1^+}g(x)=\lim\limits_{x\to 1^+}\dfrac{\sqrt{x-1}}{a^2x(x-x_1)(x-x_2)(x-2)(x-x_3)}=0,\\
            & (x=1\mbox{ \textbf{không} là tiệm cận đứng}) \\
            &\lim\limits_{x\to x_2^+}g(x)=\lim\limits_{x\to x_2^+}\dfrac{\sqrt{x-1}}{a^2x(x-x_1)(x-x_2)(x-2)(x-x_3)}=+\infty,\\
            & (x=x_2\mbox{ là tiệm cận đứng}) \\
            &\lim\limits_{x\to 2^+}g(x)=\lim\limits_{x\to 2^+}\dfrac{\sqrt{x-1}}{a^2x(x-x_1)(x-x_2)(x-2)(x-x_3)}=-\infty,\\
            & (x=2\mbox{ là tiệm cận đứng}) \\
            &\lim\limits_{x\to x_3^+}g(x)=\lim\limits_{x\to x_3^+}\dfrac{\sqrt{x-1}}{a^2x(x-x_1)(x-x_2)(x-2)(x-x_3)}=+\infty,\\
            & (x=x_3\mbox{ là tiệm cận đứng}) \\
        \end{align*}
        Vậy đồ thị hàm số $g(x)$ có tất cả $3$ tiệm cận đứng.
    }
\end{ex}
\begin{ex}%[VDC5-Đỗ Đường Hiếu]%[2D1G4-3]%
    \immini{Cho hàm số $f(x)=(x+3)(x+1)^2(x-1)(x-3)$ có đồ thị như hình vẽ. Đồ thị hàm số $g(x)=\dfrac{\sqrt{x-1}}{f^2(x)-9f(x)}$ có bao nhiêu tiệm cận đứng và tiệm cận ngang?
        \choice
        {$3$}
        {\True $4$}
        {$9$}
        {$8$}}
    {\begin{tikzpicture}[xscale=0.8,yscale=0.05, line join=round, line cap=round,font=\footnotesize,>=stealth]
            \draw[->] (-4,0)--(4,0) node[below]{$x$};
            \draw[->] (0,-56)--(0,30) node[left]{$y$};
            \coordinate[label=below left:$O$] (O) at (0,0);
            \draw (-1,0) node[below] { $-1$}(1,0) node[below] { $1$};
            \draw (-3,0) node[below left] { $-3$};
            \draw (3,0) node[below right] { $3$};
            \clip (-3.3,-60) rectangle (3.5,26);
            \draw[smooth,samples=300,domain=-3.5:3.5] plot(\x,{(\x+3)*(\x+1)^2*(\x-1)*(\x-3)});
            \foreach \x in {-3,-1,1,3}
            \draw[shift={(\x,0)},color=black] (0pt,20pt) -- (0pt,-20pt);
            \draw[shift={(0,9)},color=black] (2pt,0pt) -- (-2pt,0pt) node[left] {$9$};
        \end{tikzpicture}
    }
    \loigiai{%GV tổng quát hóa bài toán:
        Cho hàm số đa thức $y=f(x)$ có đồ thị $(C)$. Tìm số đường tiệm cận của đồ thị hàm số $g(x)=\dfrac{\sqrt{ax+b}}{P\left(f(x) \right) }$, trong đó $P\left(f(x) \right)$ là một đa thức của $f(x)$.
        Nếu $a>0$ thì $\lim\limits_{x\to +\infty}g(x)=0$.\\
        Nếu $a<0$ thì $\lim\limits_{x\to -\infty}g(x)=0$.\\
        Do đó đồ thị hàm số $y=g(x)$ luôn có duy nhất một đường tiệm cận ngang là $y=0$.\\
        Gọi $x=x_0$ là một nghiệm của phương trình $P\left(f(x) \right) =0$ thỏa mãn điều kiện $ax+b\ge 0$. Rõ ràng khi đó $\lim\limits_{x\to x_0^+}g(x)=+\infty$ hoặc $\lim\limits_{x\to x_0^+}g(x)=-\infty$.\\
        Bởi vậy, số đường tiệm cận đứng của đồ thị hàm số $y=g(x)$ chính là số nghiệm của phương trình $P\left(f(x) \right) =0$ thỏa mãn điều kiện $ax+b\ge 0$.
        \immini{Ta có $f^2(x)-9f(x)=0\Leftrightarrow \hoac{&f(x)=0\\&f(x)=9.}$\\
            \begin{itemize}
                \item $f(x)=0$ có các nghiệm thuộc $\left[1;+\infty\right)$ là $x=1$ và $x=3$.
                \item Đường thẳng $y=9$ cắt đồ thị hàm số $y=f(x)$ tại duy nhất một điểm có hoành độ thuộc $\left[1;+\infty\right)$ là $x=a>3$.
            \end{itemize}
        }
        {\begin{tikzpicture}[xscale=0.8,yscale=0.05, line join=round, line cap=round,font=\footnotesize,>=stealth]
                \draw[->] (-4,0)--(4,0) node[below]{$x$};
                \draw[->] (0,-56)--(0,30) node[left]{$y$};
                \coordinate[label=below left:$O$] (O) at (0,0);
                \draw (-4,9)--(4,9);
                \draw (-1,0) node[below] { $-1$}(1,0) node[below] { $1$};
                \draw (-3,0) node[below left] { $-3$};
                \draw (3,0) node[below right] { $3$};
                \clip (-3.3,-60) rectangle (3.5,26);
                \draw[smooth,samples=300,domain=-3.5:3.5] plot(\x,{(\x+3)*(\x+1)^2*(\x-1)*(\x-3)});
                \foreach \x in {-3,-1,1,3}
                \draw[shift={(\x,0)},color=black] (0pt,20pt) -- (0pt,-20pt);
                \draw[shift={(0,9)},color=black] (2pt,0pt) -- (-2pt,0pt) node[above left] {$9$};
        \end{tikzpicture}}
        \noindent
        Bởi vậy, hàm số $g(x)=\dfrac{\sqrt{x-1}}{f^2(x)-9f(x)}$ có tập xác định là $\mathscr D=\left[1;3\right) \cup \left(3;a\right) \cup\left( a;+\infty\right)$.\\
        Khi đó ta có
        \begin{itemize}
            \item $\lim\limits_{x\to+\infty}g(x)=0$ nên đồ thị hàm số $y=g(x)$ có một đường tiệm cận ngang là đường thẳng $y=0$.
            \item $\lim\limits_{x\to 1^+}g(x)=\lim\limits_{x\to 1^+}\dfrac{\sqrt{x-1}}{f(x)\left[f(x)-9\right] }=+\infty$;\\
            $\lim\limits_{x\to 3^+}g(x)=\lim\limits_{x\to 3^+}\dfrac{\sqrt{x-1}}{f(x)\left[f(x)-9\right] }=-\infty$;\\
            $\lim\limits_{x\to a^+}g(x)=\lim\limits_{x\to a^+}\dfrac{\sqrt{x-1}}{f(x)\left[f(x)-9\right] }=+\infty$.\\
            Do đó nên đồ thị hàm số $y=g(x)$ có $3$ đường tiệm cận đứng là các đường thẳng $x=1$, $x=3$ và $x=a$.
        \end{itemize}
        Như vậy, đồ thị hàm số $y=g(x)$ có $4$ đường tiệm cận, trong đó có $1$ đường tiệm cận ngang và $3$ đường tiệm cận đứng.
    }
\end{ex}
\begin{ex}%[VDC5-Đỗ Đường Hiếu]%[2D1G4-3]%
    \immini{Cho hàm số bậc ba $y=f(x)$ có đồ thị như hình vẽ bên. Đồ thị hàm số $g(x)=\dfrac{x\sqrt{x+1}}{f(x)\left[f^2(x)-16 \right] }$ có bao nhiêu tiệm cận đứng?
        \choice
        {\True $4$}
        {$5$}
        {$6$}
        {$7$}}
    {\begin{tikzpicture}[scale=0.6,line join=round, line cap=round,font=\footnotesize,>=stealth]
            \draw[->] (-2.5,0)--(4,0) node[below]{$x$};
            \draw[->] (0,-5)--(0,2.5) node[left]{$y$};
            \coordinate[label=below left:$O$] (O) at (0,0);
            \draw[dashed] (-1,0)--(-1,-4)--(0,-4);
            \clip (-2.3,-5) rectangle (3.5,2.5);
            \draw[smooth,samples=300,domain=-3.5:3.5] plot(\x,{-0.5*(\x+2)*(\x-1)*(\x-3)});
            \foreach \x in {-2,-1,1,3}
            \draw[shift={(\x,0)},color=black] (0pt,2pt) -- (0pt,-2pt) node[above] { $\x$};
            \foreach \y in {-4,-3,1}
            \draw[shift={(0,\y)},color=black] (2pt,0pt) -- (-2pt,0pt) node[right] {$\y$};
        \end{tikzpicture}
    }
    \loigiai{
        Xét phương trình $f(x)\left[f^2(x)-16 \right]=0$ \, $(*)$, với điều kiện $x\in\left[-1;+\infty \right) $.\\
        Ta có $f(x)\left[f^2(x)-16 \right]=0\Leftrightarrow \hoac{&f(x)=0\\&f(x)=4\\&f(x)=-4.}$\\
        \begin{itemize}
            \item Phương trình $f(x)=0$ có hai nghiệm $x\in\left[-1;+\infty \right) $ là $x=1$ và $x=3$.
            \item Phương trình $f(x)=4$ có không có nghiệm $x\in\left[-1;+\infty \right) $.
            \item Phương trình $f(x)=-4$ có hai nghiệm $x\in\left[-1;+\infty \right) $ là $-1<x_1<0$ và $x_2>3$.
        \end{itemize}
        Rõ ràng $\lim\limits_{x\to x_0^+}g(x)=+\infty$ hoặc $\lim\limits_{x\to x_0^+}g(x)=-\infty$, trong đó $x=x_0$ là nghiệm thuộc $\left[-1;+\infty \right) $ của phương trình $(*)$. Do đó đường thẳng $x=x_0$ là tiệm cận đứng của đồ thị hàm số $y=g(x)$.\\
        Từ đó suy ra đồ thị hàm số $g(x)=\dfrac{x\sqrt{x+1}}{f(x)\left[f^2(x)-16 \right] }$ có $4$ tiệm cận đứng.
    }
\end{ex}
\begin{ex}%[VDC5-Đỗ Đường Hiếu]%[2D1G4-3]%
    \immini{Cho $y=f(x)$ là hàm số đa thức có đồ thị như hình vẽ bên. Đặt $g(x)=\dfrac{\sqrt{x-1}}{\left[f(x)\right]^2-2f(x)}$ có bao nhiêu đường tiệm cận đứng?
        \choice
        {$5$}
        {$3$}
        {$4$}
        {\True $2$}}
    {\begin{tikzpicture}[scale=0.6,line join=round, line cap=round,font=\footnotesize,>=stealth]
            \draw[->] (-3,0)--(2.5,0) node[below]{$x$};
            \draw[->] (0,-1)--(0,5) node[left]{$y$};
            \coordinate[label=above left:$O$] (O) at (0,0);
            \draw[dashed] (-1,0)--(-1,4)--(0,4);
            \clip (-2.3,-1) rectangle (2.5,4.5);
            \draw[smooth,samples=300,domain=-3.5:3.5] plot(\x,{(\x)^3-3*(\x)+2});
            \foreach \x in {-2,-1,1}
            \draw[shift={(\x,0)},color=black] (0pt,2pt) -- (0pt,-2pt) node[below] { $\x$};
            \foreach \y in {2,4}
            \draw[shift={(0,\y)},color=black] (2pt,0pt) -- (-2pt,0pt) node[right] {$\y$};
        \end{tikzpicture}
    }
    \loigiai{
        Xét phương trình $\left[f(x)\right]^2-2f(x)=0$ \, $(*)$, với điều kiện $x\in\left[1;+\infty \right) $.\\
        Ta có $\left[f(x)\right]^2-2f(x)=0\Leftrightarrow \hoac{&f(x)=0\\&f(x)=2.}$\\
        \begin{itemize}
            \item Phương trình $f(x)=0$ có một nghiệm $x\in\left[1;+\infty \right) $ là $x=1$.
            \item Phương trình $f(x)=2$ có một nghiệm $x\in\left[1;+\infty \right) $ là $x=x_1>1$.
        \end{itemize}
        Rõ ràng $\lim\limits_{x\to x_0^+}g(x)=+\infty$ hoặc $\lim\limits_{x\to x_0^+}g(x)=-\infty$, trong đó $x=x_0$ là nghiệm thuộc $\left[1;+\infty \right) $ của phương trình $(*)$. Do đó đường thẳng $x=x_0$ là tiệm cận đứng của đồ thị hàm số $y=g(x)$.\\
        Từ đó suy ra đồ thị hàm số $g(x)=\dfrac{\sqrt{x-1}}{\left[f(x)\right]^2-2f(x)}$ có $2$ tiệm cận đứng.
    }
\end{ex}
\begin{ex}%[VDC5-NgocDungHo]%[2D1G4-3]%
    \immini
    {
        Cho hàm số $f(x)$ có đồ thị như hình bên. Số đường tiệm cận đứng của đồ thị hàm số $y=\dfrac{(x^2-4)(x^2+2x)}{[f(x)]^2-4f(x)+3}$ là
        \choice
        {$4$}
        {\True $5$}
        {$3$}
        {$2$}
    }
    {\begin{tikzpicture}[>=stealth,scale=0.5, line join=round, line cap=round]
            \def\f[#1]{-0.25*((#1)^4-8*(#1)^2+4)}
            \draw[->] (-4.1,0)--(4,0) node [below]{$x$};
            \draw[->] (0,-2)--(0,4) node [left]{$y$};
            \node at (0,0) [above left]{$O$};
            % \clip;
            \draw[domain=-2.9:2.9,samples=300,thick] plot (\x,{\f[\x]});
            \foreach \x in {-2,2} \filldraw (\x,0) node[below]{\x} circle (2pt);
            %\foreach \x in {-3,3} \filldraw (\x,0) node[below left]{\x} circle (2pt);
            \filldraw (-3,0) node[below left]{$-3$} circle (2pt);
            \filldraw (3,0) node[below right]{$3$} circle (2pt);
            \filldraw (0,1) node[left]{$1$} circle (2pt);
            \filldraw (0,3) node[above left]{$3$} circle (2pt);
            \draw[dashed](-2,0)--(-2,3)--(2,3)--(2,0);
            \draw (3,-1.75) node[right]{$y=f(x)$};
        \end{tikzpicture}
    }
    \loigiai{
        Xét hàm số $y=g(x)=\dfrac{(x^2-4 )(x^2+2x)}{[f(x)]^2-4f(x)+3}$.
        \immini
        {
            Giải phương trình $(x^2-4)(x^2+2x)=0 $\\
            $\Leftrightarrow \hoac{& x^2-4=0 \\ & x^2+2x=0}\Leftrightarrow \hoac{& x=\pm 2 \\ & x=0.}$\\
            Giải phương trình $[f(x)]^2-4f(x)+3=0$\\
            $ \Leftrightarrow \hoac{& f(x)=1 \\ & f(x)=3} \Leftrightarrow \hoac{& x = \pm 2 \\ & x=a\\&x=b\\&x=c\\&x=d.}$\\ với $-3<a<-2<b<c<2<d<3$.\\
        }
        {\begin{tikzpicture}[>=stealth,scale=0.8, line join=round, line cap=round]
                \def\f[#1]{-0.25*((#1)^4-8*(#1)^2+4)}
                \def\g[#1]{1}
                \def\h[#1]{3}
                \draw[->] (-4.1,0)--(4,0) node [below]{$x$};
                \draw[->] (0,-2)--(0,4) node [left]{$y$};
                \node at (0,0) [above left]{$O$};
                % \clip;
                \draw[domain=-2.9:2.9,samples=300,thick] plot (\x,{\f[\x]});
                \draw[domain=-4:4,samples=300,thick] plot (\x,{\g[\x]});
                \draw[domain=-4:4,samples=300,thick] plot (\x,{\h[\x]});
                \foreach \x in {-3,-2,2,3} \filldraw (\x,0) node[below]{\x} circle (2pt);
                % \filldraw (-3,0) node[above left]{$-3$} circle (2pt);
                % \filldraw (3,0) node[above ]{$3$} circle (2pt);
                \filldraw (0,1) node[below left]{$1$} circle (2pt);
                \filldraw (0,-1) node[below left]{$-1$} circle (2pt);
                \filldraw (0,3) node[above left]{$3$} circle (2pt);
                \draw[dashed](-2,0)--(-2,3) (2,3)--(2,0) (2.61,0)node[below]{$d$}--(2.61,1) (-2.61,0)node[below]{$a$}--(-2.61,1) (1.08,0)node[below]{$c$}--(1.08,1)(-1.08,0)node[below]{$b$}--(-1.08,1);
                \draw (3,2.75) node[right]{$y=f(x)$};
            \end{tikzpicture}
        }
        Trong điều kiện xác định của hàm số $y=g(x)$ ta có thể viết $$y=g(x)=\dfrac{x(x-2)(x+2)^2}{(x-a)(x-b)(x-c)(x-d) (x-2)^2(x+2)^2}=\dfrac{x}{(x-a)(x-b)(x-c)(x-d)(x-2)}$$
        Vậy số tiệm cận đứng của đồ thị hàm số $y=g(x)$ bằng $5$.
    }
\end{ex}
\Closesolutionfile{ans}
%\subsection{ĐỀ ÔN LUYỆN}
%\boxde
\BTTN
\begin{ex}%[2D1N3-1]Câu 2
 Đường thẳng $x = a$ là một đường tiệm cận đứng của
 đồ thị hàm số $ y = f (x)$ nếu điều kiện sau thoả mãn
 \choice
 {$\displaystyle\lim_{x\to +\infty }f(x)=a$}
 {\True $\displaystyle\lim_{x\to a^-}f(x)=+\infty $}
 {$\displaystyle\lim_{x\to -\infty }f(x)=a$}
 {$\displaystyle\lim_{x\to a^-}f(x)=a $}
 \loigiai{ Đường thẳng $x = a$ được gọi là một đường tiệm cận đứng (hay tiệm cận đứng) của đồ thị hàm số $ y = f (x)$ nếu ít nhất một trong các điều kiện sau thoả mãn: \\$\displaystyle\lim_{x\to a^+}f(x)=+\infty $, $\displaystyle\lim_{x\to a^+}f(x)=-\infty $, $\displaystyle\lim_{x\to a^-}f(x)=-\infty $, $\displaystyle\lim_{x\to a^-}f(x)=+\infty $.}
\end{ex}
\begin{ex}%[2D1N3-1]Câu 4
 Đường thẳng $y = ax + b$ ($a \neq 0$) được gọi là đường tiệm cận xiên của đồ thị hàm số $y = f(x)$ nếu
 \choice
 {\True $\displaystyle\lim_{x\to -\infty }\big(f(x)-ax-b\big)=0$ hoặc $\displaystyle\lim_{x\to +\infty }\big(f(x)-ax-b\big)=0$}
 {$\displaystyle\lim_{x\to -\infty }\big(f(x)-ax+b\big)=0$ hoặc $\displaystyle\lim_{x\to +\infty }\big(f(x)-ax+b\big)=0$}
 {$\displaystyle\lim_{x\to 0 }\big(f(x)-ax+b\big)=+\infty$ hoặc $\displaystyle\lim_{x\to 0 }\big(f(x)-ax+b\big)=+\infty$}
 {$\displaystyle\lim_{x\to 0 }\big(f(x)-ax-b\big)=-\infty$ hoặc $\displaystyle\lim_{x\to 0 }\big(f(x)-ax-b\big)=-\infty$}
 \loigiai{Đường thẳng $y = ax + b, a \neq 0$, được gọi là đường tiệm cận xiên (hay tiệm cận xiên) của đồ thị hàm số $y = f(x)$ nếu\\ $\displaystyle\lim_{x\to -\infty }[f(x)-(ax+b)]=\displaystyle\lim_{x\to -\infty }(f(x)-ax-b)=0$ hoặc\\ $\displaystyle\lim_{x\to +\infty }[f(x)-(ax+b)]=\displaystyle\lim_{x\to +\infty }(f(x)-ax-b)=0$.
 }
\end{ex}
\begin{ex}
 Tiệm cận ngang của đồ thị hàm số $ y=\dfrac{2x-1}{x+1} $ là đường thẳng
 \choice
 {$y=-1$}
 {$ x=-1 $}
 {\True $ y=2 $}
 {$ x=2 $}
 \loigiai
 {
 Ta có $ \lim\limits_{x\to \pm\infty}y=2$ suy ra đường thẳng $ y=2 $ là tiệm cận ngang của đồ thị hàm số $ y=\dfrac{2x-1}{x+1} $.
 }
\end{ex}
\begin{ex}
 Tiệm cận ngang của đồ thị hàm số $y=\dfrac{1}{2x-3}$ là đường thẳng
 \choice
 {$y=\dfrac{3}{2}$}
 {$x=\dfrac{3}{2}$}
 {\True $y=0$}
 {$y=\dfrac{1}{2}$}
 \loigiai{
 Vì $\lim\limits_{x\to -\infty} \dfrac{1}{2x-3}=\lim\limits_{x\to +\infty} \dfrac{1}{2x-3}=0$ nên đồ thị hàm số có tiệm cận ngang $y=0$.
 }
\end{ex}
\begin{ex}
 Đồ thị hàm số $f(x)=\dfrac{2x-3}{x+1}$ có đường tiệm cận đứng là
 \choice
 {$y=2$}
 {\True $x=-1$}
 {$y=-1$}
 {$x=2$}
 \loigiai{
 Ta có $\displaystyle \lim_{x \to (-1)^-}f(x)=\displaystyle \lim_{x \to (-1)^-}\dfrac{2x-3}{x+1}=+\infty $; $\displaystyle \lim_{x \to (-1)^+}f(x)=\displaystyle \lim_{x \to (-1)^+}\dfrac{2x-3}{x+1}=-\infty$ nên đường thẳng $x=-1$ là đường tiệm cận đứng của đồ thị hàm số.}
\end{ex}
\begin{ex}
 Hàm số nào sau đây có đồ thị nhận đường thẳng $x=2$ là đường tiệm cận đứng?
 \choice
 {$y=\dfrac{2}{x+2}$}
 {\True $y=\dfrac{5x}{2-x}$}
 {$y=\dfrac{1}{x+1}$}
 {$y=x-2+\dfrac{1}{x+1}$}
 \loigiai{
 Ta có $\lim\limits_{x\to 2^+} \dfrac{5x}{2-x}=-\infty $ và $\lim\limits_{x\to 2^-} \dfrac{5x}{2-x}=+\infty $ nên đồ thị hàm số $y=\dfrac{5x}{2-x}$ nhận $x=2$ làm tiệm cận đứng.}
\end{ex}
\begin{ex}%[2D1V3-1]Câu 12
 Đồ thị của hàm số nào sau đây có giao điểm của hai đường tiệm cận thuộc đường thẳng $y=x$?
 \choice
 {$y=\dfrac{2x-1}{x+3}$}
 {\True$y=\dfrac{x+4}{x-1}$}
 {$y=\dfrac{2x+1}{x+2}$}
 {$\dfrac{1}{x+3}$}
 \loigiai{
 Đáp án $y=\dfrac{2x-1}{x+3}$ có giao hai đường tiệm tiệm cận là $(-3;2)\notin d$\\
 Đáp án $y=\dfrac{x+4}{x-1}$ có giao hai đường tiệm cận là $(1;1)\in d$\\
 Đáp án $y=\dfrac{2x+1}{x+2}$ có giao hai đường tiệm cận là $(-2;2)\notin d$\\
 Đáp án $\dfrac{1}{x+3}$ có giao hai đường tiệm cận là $(-3;0)\notin d$\\
 }
\end{ex}
\begin{ex}%[2D1N3-1]Câu 6
 Đồ thị hàm số $y=\dfrac{x-2}{x^{2}-4}$ có mấy đường tiệm cận?
 \choice
 {$3$}
 {$1$}
 {\True$2$}
 {$0$}
 \loigiai{ Hàm số $y=\dfrac{x-2}{x^{2}-4}=\dfrac{x-2}{(x-2)(x+2)}=\dfrac{1}{x+2}$.\\
 $\heva{&\displaystyle\lim_{x\to +\infty }\dfrac{1}{x+2}=0\\&
 \displaystyle\lim_{x\to -\infty }\dfrac{1}{x+2}=0.}$\\
 Nên $y=0$ là đường tiệm cận ngang của hàm số, hàm số có tiệm cận ngang thì không có tiệm cận xiên.\\
 $\heva{&\displaystyle\lim_{x\to -2^- }\dfrac{1}{x+2}= - \infty \\&
 \displaystyle\lim_{x\to -2^+ }\dfrac{1}{x+2}= + \infty.}$\\
 Nên $x=-2$ là đường tiệm cận đứng của hàm số.\\
 Vậy hàm số có hai đường tiệm cận.
 }
\end{ex}
\begin{ex}
 Tiệm cận xiên của đồ thị hàm số $y=\dfrac{x^2+x-1}{x}$ có phương trình là
 \choice
 {$y=x-1$}
 {$y=x-2$}
 {$y=x-3$}
 {\True$y=x+1$}
 \loigiai{
 Ta có $y=\dfrac{x^2+x-1}{x}=x+1-\dfrac{1}{x}$.\\
 Xét $$\displaystyle\lim_{x\to \pm \infty }\big(y-(x+1)\big)=\displaystyle\lim_{x\to \pm \infty }\dfrac{-1}{x}=0$$
 Vậy đường tiệm cận xiên cần tìm của hàm số $f(x)$ có phương trình $y=x+1$.}
\end{ex}
\begin{ex}
 Tiệm cận xiên của đồ thị hàm số $y=\dfrac{2x^2-3x+4}{x-1}$ có phương trình là
 \choice
 {$y=x-1$}
 {\True $y=2x-1$}
 {$y=2x+1$}
 {$y=x+1$}
 \loigiai{
 Ta có $y=\dfrac{2x^2-3x+4}{x-1}=2x-1+\dfrac{3}{x-1}$. Suy ra $y=2x-1$ là đường tiệm cận xiên của đồ thị hàm số.
 }
\end{ex}
\begin{ex}
 Cho hàm số $y=f(x)$ xác định $ \mathbb{R} \setminus \left\lbrace 0\right\rbrace $, liên tục trên mỗi khoảng xác định và có bảng biến thiên như sau.\\
 \begin{center}
 \begin{tikzpicture}[>=stealth,font=\footnotesize,scale=1]
 \tikzset{double style/.append style = {draw=\tkzTabDefaultWritingColor,double=\tkzTabDefaultBackgroundColor,double distance=2pt}}
 \tkzTabInit[nocadre=false,lgt=1.2,espcl=2.5,deltacl=0.6]
 {$x$ /0.6,$y'$ /0.6,$y$ /2}
 {$-\infty$,$0$,$ 1 $,$+\infty$}
 \tkzTabLine{,-,d,+,$ 0 $,- }
 \tkzTabVar{+/ $+\infty$,-D- /$-1$/$-\infty$,+/$2$,-/$ -\infty $}
 \end{tikzpicture}
 \end{center}
 Chọn khẳng định đúng
 \choice
 {Đồ thị hàm số có hai tiệm cận ngang}
 {\True Đồ thị hàm số có đúng một tiệm cận đứng}
 {Đồ thị hàm số không có tiệm cận đứng và tiệm cận ngang}
 {Đồ thị hàm số có đúng một tiệm cận ngang}
 \loigiai
 {
 Dựa vào bảng biến thiên ta thấy
 $ \lim\limits_{x \to + \infty} f(x)=+\infty$; $ \lim\limits_{x \to -\infty}f(x)=-\infty$; $ \lim\limits_{x \to 0^+}f(x)=-\infty$.\\
 Suy ra đồ thị hàm số có đúng một tiệm cận đứng.
 }
\end{ex}
\begin{ex}%[2D1N3-1]Câu 5
 \immini{Cho hàm số $y=f(x)$ có đồ thị như hình bên dưới. Khẳng định nào sau đây là khẳng định đúng?
 \choice
 {Đồ thị hàm số chỉ có 2 đường tiệm cận đứng $x=-1$ và $x=1$}
 {\True Đồ thị hàm số có 3 đường tiệm cận}
 {Đồ thị hàm số có 4 đường tiệm cận}
 {Đồ thị hàm số có 2 đường tiệm cận đứng và 1 đường tiệm cận xiên}}{
 \begin{tikzpicture}[>=stealth]
 \draw[->] (-4,0) --(4,0);
 \draw[->](0,-4)--(0,4);
 \draw (0,0) node[below left]{$O$};
 \draw (4,0) node[below]{$x$};
 \draw (0,4) node[left]{$y$};
 \draw (1,0) node[above left]{$1$};
 \draw (-1,0) node[above left]{$-1$};
 \clip (-4,-4) rectangle(4,4);
 \draw[thick,samples=100] plot[domain=-4:4]
 (\x,{(\x)/((\x)^(2)-1)});
 \draw (-1.7,1.5) node
 {$x=-1$};
 \draw (1.5,-1.5) node
 {$x=1$};
 \end{tikzpicture}}
 \loigiai{ Đồ thị hàm số có 2 đường tiệm cận đứng $x=-1$ và $x=1$ và một đường tiệm cận ngang $y=0$, hàm số không có đường tiệm cận xiên.}
\end{ex}
\begin{ex}
 Biết rằng đồ thị hàm số $ y=\dfrac{ax+1}{bx-2}$ có tiệm cận đứng là $x=2$ và tiệm cận ngang là $y=3$. Giá trị của $a+b$ bằng
 \choice
 {$0$}
 {\True $4$}
 {$5$}
 {$1$}
 \loigiai{
 Điều kiện để đồ thị hàm số $ y=\dfrac{ax+1}{bx-2}$ có tiệm cận đứng và tiệm cận ngang là $-2a-b\ne 0$. \quad$(*)$\\
 $b\ne 0$ vì nếu $ b=0$, đồ thị hàm số $ y=\dfrac{ax+1}{-2}$ không có tiệm cận.\\
 Tập xác định của hàm số $y=\dfrac{ax+1}{bx-2}$ là $\mathscr{D}=\left(-\infty;\dfrac{2}{b}\right)\cup\left(\dfrac{2}{b};+\infty\right)$.\\
 $\lim\limits_{x\to\pm\infty}\dfrac{ax+1}{bx-2}=\dfrac{a}{b}\Rightarrow y=\dfrac{a}{b}$ là đường tiệm cận ngang của đồ thị hàm số.\\
 Theo giả thiết ta có $\dfrac{a}{b}=3\Leftrightarrow a=3b$.\\
 Đồ thị hàm số $y=\dfrac{ax+1}{bx-2}$ có $ x=\dfrac{2}{b}$ là đường tiệm cận đứng.\\
 Theo giả thiết ta có $\dfrac{2}{b}=2\Leftrightarrow b=1\Rightarrow a=3$ (thỏa mãn điều kiện $(*)$).\\
 Vậy $a+b=4$.
 }
\end{ex}
\begin{ex}
 Tìm tất cả giá trị của tham số $m$ để đường tiệm cận xiên của đồ thị hàm số $y=2mx+3-\dfrac{4}{x+1}$ đi qua điểm $M(1;7)$.
 \choice
 {$m=1$}
 {$m=3$}
 {\True $m=2$}
 {$m=-2$}
 \loigiai{
 Xét $\displaystyle\lim_{x\to \pm \infty }\left( y-\left( 2mx+3\right) \right) =\displaystyle\lim_{x\to \pm \infty }\dfrac{-4}{x+1}=0$.\\
 Vậy đường tiệm cận xiên có phương trình $y=2mx+3$.\\
 Đường thẳng này qua điểm $M(1;7)$, suy ra $2m \cdot 1 ++3=7 \Leftrightarrow m=2$.
 }
\end{ex}
\begin{ex}
 Tại một công ty sản xuất đồ chơi A, công ty phải chi 50000 USD để thiết lập dây chuyền sản xuất ban đầu. Sau đó, cứ sản xuất được một sản phẩm đồ chơi A, công ty phải trả 5 USD cho nguyên liệu thô và nhân công. Gọi $x\,(x \geq 1)$ là số đồ chơi A mà công ty đã sản xuất và $T(x)$ (đơn vị USD) là tổng số tiền bao gồm cả chi phí ban đầu mà công ty phải chi trả khi sản xuất $x$ đồ chơi A. Người ta xác định chi phí trung bình cho mỗi sản phẩm đồ chơi A là $M(x)=\dfrac{T(x)}{x}$. Khi $x$ đủ lớn ($x\to +\infty$) thì chi phí trung bình (USD) cho mỗi sản phẩm đồ chơi $A$ gần nhất với kết quả nào sau đây?
 \choice
 {$50\,000$}
 {$50\,005$}
 {10}
 {\True $5$}
 \loigiai{
 Gọi $T(x)$ (đơn vị USD) là tổng số tiền bao gồm cả chi phí ban đầu mà công ty phải chi trả khi sản xuất $x$ đồ chơi A thì $T(x)=50\,000 + 5x$.\\
 Ta có $$\displaystyle\lim_{x\to + \infty }\dfrac{T(x)}{x} =\displaystyle\lim_{x\to + \infty }\left(\dfrac{50\,000}{x}+5\right) =5.$$
 }
\end{ex}
\BTTF
\begin{ex}
 Cho hàm số $y=f(x)$ có $\displaystyle\lim_{x\rightarrow 3^{-}}f(x)=1$, $\displaystyle\lim\limits_{x\rightarrow 3^{+}}f(x)=+\infty$ và $\displaystyle\lim_{x\rightarrow -\infty}f(x)=1$, $\displaystyle\lim\limits_{x\rightarrow +\infty}f(x)=+\infty$. Xét tính đúng sai của các khẳng định sau:
 \choiceTF
 {\True Đồ thị của hàm số $y=f(x)$ có tiệm cận ngang là đường thẳng $y=1$}
 {\True Đồ thị của hàm số $y=f(x)$ có tiệm cận đứng là đường thẳng $x=3$}
 {Đồ thị của hàm số $y=f(x)$ không có tiệm cận ngang}
 {Đồ thị của hàm số $y=f(x)$ không có tiệm cận đứng}
 \loigiai{
 \begin{itemchoice}
 \itemch Do $\displaystyle\lim_{x\rightarrow -\infty}f(x)=1$ nên $y=1$ là đường tiệm cận ngang của đồ thị hàm số. (1)
 \itemch Do $\displaystyle\lim\limits_{x\rightarrow 3^{+}}f(x)=+\infty$ nên $x=3$ là đường tiệm cận đứng của đồ thị hàm số. (2)
 \itemch Từ (1) suy ra khẳng định này sai.
 \itemch Từ (2) suy ra khẳng định này sai.
 \end{itemchoice}
 }
\end{ex}
\begin{ex}
 Cho hàm số $y=f(x)$ xác định trên $\mathbb{R}\backslash\{\pm 2\}$ và có bảng biến thiên như hình vẽ bên dưới.
 \begin{center}
 \begin{tikzpicture}[scale=0.8,>=stealth]
 \tikzset{double style/.append style = {draw=\tkzTabDefaultWritingColor,double=\tkzTabDefaultBackgroundColor,double distance=2pt}}
 \tkzTabInit[nocadre=false, lgt=1, espcl=4,deltacl=1pt]{$x$ /1,$y'$ /1,$y$ /2.2}{$-\infty$,$-2$,$2$,$+\infty$}
 \tkzTabLine{,-,d,-,d,-,}
 \tkzTabVar{+/ $0$ ,-D+/ $-10$/$+\infty$ , -D+/ $-\infty$/$+\infty$,-/$0$}
 \end{tikzpicture}
 \end{center}
 Xét tính đúng sai của các khẳng định sau:
 \choiceTF
 {\True Hàm số không có điểm cực trị}
 {$\lim\limits_{x\to -2^{-}}f(x)=+\infty$}
 {\True Đồ thị hàm số có đúng 1 tiệm cận ngang}
 {Đồ thị hàm số có đúng $1$ tiệm cận đứng}
 \loigiai{
 Dựa vào bảng biến thiên ta thấy
 \begin{itemchoice}
 \itemch Hàm số không có điểm cực trị;
 \itemch $\lim\limits_{x\to -2^{-}}f(x)=-10$;
 \itemch $\lim\limits_{x\to \pm \infty}f(x)=0$. Suy ra đồ thị có đúng 1 đường tiệm cận ngang là $y=0$.
 \itemch $\lim\limits_{x\to -2^{+}}f(x)=+\infty$ và $\lim\limits_{x\to 2^{+}}f(x)=+\infty$ nên đồ thị hàm số có đúng 2 đường tiệm cận đứng $x = \pm 2$.
 \end{itemchoice}
 }
\end{ex}
\begin{ex}
 Cho hàm số $y=\dfrac{\sqrt{x^2-x+2}}{x-1}$. Xét tính đúng sai của các khẳng định sau:
 \choiceTF
 {\True Tập xác định của hàm số là $\mathbb{R} \backslash\{1\}$}
 {\True Đồ thị hàm số có các đường tiệm cận ngang là $y=1,\,y=-1$}
 {Đồ thị hàm số đã cho có tất cả 2 đường tiệm cận}
 {Các đường tiệm cận của đồ thị cùng với trục $O y$ tạo thành 1 đa giác có diện tích bằng 1}
 \loigiai{
 \begin{itemchoice}
 \itemch Điều kiện xác định $\heva{&x^2-x+2>0\text{ luôn đúng}\\& x-1 \ne 0} \Leftrightarrow x \ne 1$. Vậy tập xác định của hàm số là $\mathbb{R} \backslash\{1\}$
 \itemch Ta có
 \begin{itemize}
 \item [$\bullet$] $\displaystyle\lim_{x\rightarrow -\infty}f(x)=-1$ nên $y=-1$ là đường tiệm cận ngang;
 \item [$\bullet$] $\displaystyle\lim_{x\rightarrow +\infty}f(x)=1$ nên $y=1$ là đường tiệm cận ngang;
 \end{itemize}
 \itemch Do $\displaystyle\lim_{x\rightarrow 1^+}f(x)=+\infty$ nên $x=1$ là đường tiệm cận đứng. Vậy đồ thị hàm số có tất cả 3 đường tiệm cận (2 TCN và 1 TCĐ).
 \itemch Minh họa miền giới hạn của các đường tiệm cận và trục $Oy$ như sau:
 \begin{center}
 \begin{tikzpicture}[smooth,samples=300,scale=0.8,>=stealth]
 \draw[->] (-3,0)--(6,0) node[below]{$x$};
 \draw[->] (0,-3)--(0,3) node[right]{$y$};
 \draw (0,0) node[below left]{$O$};
 \draw[pattern = north west lines] (0,-1)--(1,-1)--(1,1)--(0,1);
 \draw
 (-3,-1)--(4,-1)node[below]{\scriptsize TCN $y=-1$}
 (-3,1)--(4,1)node[above]{\scriptsize TCN $y=1$}
 (1,-3)--(1,3)node[above right]{\scriptsize TCĐ $x=1$};
 \end{tikzpicture}
 \end{center}
 Miền giới hạn là hình chữ nhật có diện tích là $S=2 \cdot 1 =2$.
 \end{itemchoice}
 }
\end{ex}
\begin{ex}
 Cho hàm số $y=f(x)=\dfrac{2 x^2+2 x+5}{2 x+1}$. Xét tính đúng sai của các khẳng định sau:
 \choiceTF
 {\True Đạo hàm của hàm số đã cho là $y'=\dfrac{4\left(x^2+x-2\right)}{(2 x+1)^2}$}
 {\True Các điểm cực trị của đồ thị hàm số có toạ độ là $(-2 ;-3)$ và $(1 ; 3)$}
 {\True Đường tiệm cận đứng của đồ thị hàm số có phương trình là $x=-\dfrac{1}{2}$}
 {\True Đường tiệm cận xiên của đồ thị hàm số có phương trình là $y=x+\dfrac{1}{2}$}
 \loigiai{
 \begin{itemchoice}
 \itemch Ta có $y'=\dfrac{(2 x^2+2 x+5)'(2 x+1)-(2 x+1)'(2 x^2+2 x+5)}{(2x+1)^2}=\dfrac{4\left(x^2+x-2\right)}{(2 x+1)^2}$.
 \itemch $y'=0 \Leftrightarrow x^2+x-2 =0 \Leftrightarrow \hoac{&x=1\\&x=-2}$.\\
 Thay vào hàm số, ta tính được toạ độ các điểm cực trị là $(-2 ;-3)$ và $(1 ; 3)$.
 \itemch Điều kiện xác định $x \ne -\dfrac{1}{2}$.\\
 $\displaystyle\lim_{x\rightarrow -\frac{1}{2}^+}f(x)=+\infty$ nên $x=-\dfrac{1}{2}$ là đường tiệm cận đứng;
 \itemch $y=\dfrac{2 x^2+2 x+5}{2 x+1}=x+\dfrac{1}{2}+\dfrac{9}{2(2x+1)}$. \\
 Suy ra đồ thị có đường tiệm cận xiên là $y=x+\dfrac{1}{2}$.
 \end{itemchoice}
 }
\end{ex}
\BTTL
\begin{ex}%[2D1B4-1]%
 Các đường tiệm cận của đồ thị hàm số $ y=\dfrac{2x+3}{x-1}$ tạo với hai trục tọa độ một hình chữ nhật có
 diện tích bằng bao nhiêu?\\
 \shortans[3]{$2$}
 \loigiai{
 Tập xác định $\mathscr{D}=\mathbb{R}\setminus\{1\}$.
 \begin{itemize}
 \item $\lim\limits_{x\to 1} y=\lim\limits_{x\to 1^+}\dfrac{2x+3}{x-1}=+\infty\Rightarrow x=1 $ là tiệm cận đứng của đồ thị hàm số.
 \item $\lim\limits_{x\to+\infty} y=\lim\limits_{x\to+\infty}\dfrac{2x+1}{x-1}=2\Rightarrow y=2 $ là tiệm cận ngang của đồ thị hàm số.
 \end{itemize}
 Hai đường tiệm cận của đồ thị hàm số tạo với hai trục tọa độ một hình chữ nhật có diện tích $ S=1\cdot 2=2 $.
 }
\end{ex}
\begin{ex}%[2D1B4-1]%
 Cho hàm số $y=\dfrac{x+1}{x-3}$ có đồ thị $(C)$ và đường thẳng $\Delta: y=mx+m-3.$ Biết đường thẳng $\Delta$ đi qua giao điểm hai đường tiệm cận của
 $(C).$ Khi đó giá trị của $m$ bằng bao nhiêu?\\
 \shortans[3]{$1$}
 \loigiai{
 Đồ thị (C) có TCĐ là $x=3$ và TCN là $y=1$, suy ra $I(3 ; 1)$ là giao điểm hai tiệm cận của $(C)$.\\
 Do $I \in \Delta \Rightarrow 1=3m+m-3 \Leftrightarrow 4m-4=0 \Leftrightarrow m=1$.
 }
\end{ex}
\begin{ex}
 Cho hàm số $y=\dfrac{3x^2+2x}{4x+4}$. Khoảng cách từ điểm $M(3;-2)$ đến đường tiệm cận xiên của đồ thị hàm số này bằng bao nhiêu?\\
 \shortans[3]{$3{,}2$}
 \loigiai{
 $y=\dfrac{3x^2+2x}{4x+4}=\dfrac{3}{4}x-\dfrac{1}{4}+\dfrac{1}{4x+4}$.\\
 Xét $\displaystyle\lim_{x\to \pm \infty }\left( y-\left( \dfrac{3}{4}x-\dfrac{1}{4}\right) \right) =\displaystyle\lim_{x\to \pm \infty }\dfrac{1}{4x+4}=0$.\\
 Vậy đường tiệm cận xiên có phương trình $y=\dfrac{3}{4}x-\dfrac{1}{4} \Leftrightarrow 3x-4y-1=0$.\\
 Khoảng cách từ điểm $M$ đến đường tiệm cận xiên là
 $$d=\dfrac{\big|3 \cdot 3 -4 \cdot (-2)-1\big|}{\sqrt{3^2+(-4)^2}}=\dfrac{16}{5}=3,2$$
 }
\end{ex}
\begin{ex}
 Nồng độ oxygen trong hồ theo thời gian $t$ cho bởi công thức $y(t)=5-\dfrac{15 t}{9 t^2+1}$, với $y$ được tính theo $\mathrm{mg} / l$ và $t$ được tính theo giờ, $t \geq 0$. Đường tiệm cận ngang của đồ thị hàm số $y=y(t)$ khi $t \to +\infty$ có dạng $y=a$. Giá trị của $a$ bằng bao nhiêu?\\
 \shortans[3]{$5$}
 \loigiai{
 $\displaystyle\lim_{t\rightarrow +\infty}y(t)=\lim_{t\rightarrow +\infty}\left( 5-\dfrac{15 t}{9 t^2+1}\right) =5$ nên $y=5$ là đường tiệm cận ngang.
 }
\end{ex}
\begin{ex}
 Số lượng sản phẩm bán được của một công ty trong $x$ (tháng) được tính theo công thức $S(x)=200\left(5-\dfrac{9}{2+x}\right)$, trong đó $x \geq 1$. Xem $y=S(x)$ là một hàm số xác định trên nửa khoảng $[1 ;+\infty)$. Biết $y=a$ là tiệm cận ngang của đồ thị hàm số đó. Giá trị của $a$ bằng bao nhiêu?\\
 \shortans[3]{$1000$}
 \loigiai{
 Ta có
 $S(x)=200\left(5-\dfrac{9}{2+x}\right)=1000-\dfrac{1800}{2+x}.$\\
 Vì $\displaystyle\lim \limits_{n \to +\infty}_{x \rightarrow\pm\infty} S(x)=\lim \limits_{n \to +\infty}_{x \rightarrow\pm\infty} \left(1000-\dfrac{1800}{2+x}\right)=1000$
 nên đường thẳng $y=1000$ là tiệm cận ngang của đồ thị hàm số đã cho.
 }
\end{ex}
\begin{ex}%
 \immini{Cho hàm đa thức bậc ba $y=f(x)$ có đồ thị như hình vẽ.	Đồ thị hàm số $y=\dfrac{(x+1)(x^2-1)}{f(x)}$ có bao nhiêu đường tiệm cận (đứng và ngang)?\\
 \shortans[3]{$3$}}{
 \begin{tikzpicture}[>=stealth]
 \draw[->] (-3,0) --(3,0)node[below]{$x$};
 \draw[->](0,-3)--(0,2.5)node[left]{$y$};
 \draw (0,0.5) node[below left]{$O$};
 \draw (2,0) node[above left]{$2$};
 \draw (-1,0) node[above left]{$-1$};
 \draw[dashed] (0,-2)
 node[left]{$-2$} -- (1,-2) --
 (1,0) node[above]{$1$};
 \draw[thick,samples=100] plot[domain=-2.2:2.3]
 (\x,{(1/2)*(\x)^3-(3/2)*\x-1})node[above]{$y=f(x)$};
 \end{tikzpicture}}
 \loigiai{ Hàm số có dạng $f(x)=ax^3+bx^2+cx-1$ (vì là hàm bậc ba cắt trục tung tại điểm có tung độ $-1$)\\
 Đồ thị hàm số đã cho đi qua các điểm có tọa độ là $(-1;0)$, $(1;-2)$, $(2;0)$ \\
 $\to \heva{&8a+4b+2c=1\\& -a+b-c=1 \\&a+b+c=-1}\Leftrightarrow \heva{&a=\dfrac{1}{2}\\&b=0 \\&c=\dfrac{-3}{2}.}$\\
 $\to f(x)=\dfrac{1}{2}x^3-\dfrac{3}{2}x-1=\dfrac{1}{2}(x-1)^2(x-2)$.\\
 Khi đó $y=\dfrac{(x+1)(x^2-1)}{f(x)}=\dfrac{(x+1)(x^2-1)}{\dfrac{1}{2}(x-1)^2(x-2)}=\dfrac{2(x+1)^2}{(x-1)(x-2)}$.\\
 Đồ thị hàm số trên có tiệm cận ngang $y=2$ và tiệm cận đứng là $x=1,x=2$.\\
 Vậy đồ thị hàm số $y=\dfrac{(x+1)(x^2-1)}{f(x)}$ có 3 đường tiệm cận.
 }
\end{ex}
%\boxde
\BTTN
\begin{ex}%[2D1B4-1]
    Phương trình đường tiệm cận ngang của đồ thị hàm số $y=\dfrac{x-3}{x-1}$ là
    \choice
    {$y=5$}
    {$y=0$}
    {$x=1$}
    {\True $y=1$}
    \loigiai
    {
        Ta có $\lim\limits_{x \to \pm \infty}\dfrac{x-3}{x-1} = \lim\limits_{x \to \pm \infty}\dfrac{1-\dfrac{3}{x}}{1-\dfrac{1}{x}}=1$, nên đường thẳng $y=1$ là tiệm cận ngang của đồ thị hàm số đã cho.
    }
\end{ex}


\begin{ex}
    Đường thẳng nào dưới đây là tiệm cận đứng của đồ thị hàm số $y=\dfrac{2x}{x+2}$?
    \choice
    {$x=2$}
    {$x=0$}
    {\True $x=-2$}
    {$x=1$}
    \loigiai{
        Tập xác định $\mathscr{D}=\mathbb{R}\setminus\{-2\}$.
        \begin{itemize}
            \item $\lim\limits_{x\to -2^+}\dfrac{2x}{x+2}=-\infty$.
            \item $\lim\limits_{x\to -2^-}\dfrac{2x}{x+2}=+\infty$.
        \end{itemize}
        Vậy $x=2$ là đường tiệm cận đứng của đồ thị hàm số.
    }
\end{ex}

\begin{ex}%[2D1Y4-1]
    Cho hàm số $y=\dfrac{x+1}{2x-2}$. Khẳng định nào sau đây đúng?
    \choice
    {Đồ thị hàm số có tiệm cận đứng là $x=\dfrac{1}{2}$}
    {Đồ thị hàm số có tiệm cận ngang là $y=-\dfrac{1}{2}$}
    {\True Đồ thị hàm số có tiệm cận ngang là $y=\dfrac{1}{2}$}
    {Đồ thị hàm số có tiệm cận đứng là $x=2$}
    \loigiai{
        Đồ thị hàm số $y=\dfrac{x+1}{2x-2}$ có tiệm cận đứng $x=1$ và tiệm cận ngang $y=\dfrac{1}{2}$.
    }
\end{ex}


\begin{ex}%[2D1Y4-1]
    Cho hàm số $y=f(x)$ có $\lim\limits_{x\to -\infty} f(x)= -2$ và $\lim\limits_{x\to +\infty} f(x)= 2$. Khẳng định nào sau đây đúng?
    \choice
    {Đồ thị hàm số đã cho có đúng một tiệm cận ngang}
    {Đồ thị hàm số đã cho không có tiệm cận ngang}
    {Đồ thị hàm số đã cho có hai tiệm cận ngang là hai đường thẳng $x=-2$ và $x=2$}
    {\True Đồ thị hàm số đã cho có hai tiệm cận ngang là hai đường thẳng $y=-2$ và $y=2$}
    \loigiai{
        $\lim\limits_{x\to -\infty} f(x)= -2$ nên $y=-2$ là tiệm cận ngang.\\
        $\lim\limits_{x\to +\infty} f(x)= 2$ nên $y=2$ là tiệm cận ngang.
    }
\end{ex}

\begin{ex}%[2D1Y4-1]
    Cho hàm số $y= \dfrac{2017}{x-2}$ có đồ thị $(H)$. Số đường tiệm cận của $(H)$ là
    \choice
    {$0$}
    {\True $2$}
    {$3$}
    {$4$}
    \loigiai{
        Đồ thị $(H)$ có tiệm cận đứng là $x=2$, tiệm cận ngang là $y=0$.\\
        Vậy số đường tiệm cận của $(H)$ là $2$.
    }
\end{ex}

\begin{ex}
    Tìm số đường tiệm cận của đồ thị hàm số $ y = \dfrac{x^2 - 3x + 2}{x^2 - 4}. $
    \choice
    {$1$}
    {$ 0$}
    {\True $2$}
    {$3$}
    \loigiai
    {
        Tập xác định: $ \mathscr D = \mathbb{R} \backslash \{\pm2 \} $.\\
        Ta có $ \lim \limits_{x \to \pm  \infty} y = 1 \Rightarrow  $ đồ thị hàm số có 1 tiệm cận ngang là $ y = 1. $\\
        Ta lại có $\lim \limits_{x \to 2} y =  \lim \limits_{x \to 2} \dfrac{x-1}{x+2} = \dfrac{1}{4} $ và $\lim \limits_{x \to -2^+} y =  \lim \limits_{x \to -2^+} \dfrac{x-1}{x+2} = -\infty$ nên đồ thị hàm số có 1 tiệm cận đứng là $ x = -2. $\\
        Vậy đồ thị hàm số đã cho có 2 đường tiệm cận.
    }
\end{ex}


\begin{ex}%[Đề minh họa BGD 2018-2019]%[2D1B4-1]
    \immini[thm]{Cho hàm số $y=f(x)$ có bảng biến thiên như sau. Tổng số tiệm cận ngang và tiệm cận đứng của đồ thị hàm số đã cho là
        \haicot
        {$4$}
        {$1$}
        {\True $3$}
        {$2$}
    }{
        \begin{tikzpicture}
            \tikzset{double style/.append style = {draw=\tkzTabDefaultWritingColor,double=\tkzTabDefaultBackgroundColor,double distance=2pt}}
            \tkzTabInit[nocadre=false, lgt=1.2, espcl=2.5,deltacl=0.6
            ]{$x$ /0.6,$y'$ /0.6,$y$ /1.6}{$-\infty$,$1$,$+\infty$}
            \tkzTabLine{,+, d ,+,}
            \tkzTabVar{-/ $2$ / , +D-/ $+\infty$ / $3$ , +/ $5$ /}
    \end{tikzpicture}}
    \loigiai{
        Từ bảng biến thiên ta có
        \begin{itemize}
            \item $\lim\limits_{x \to -\infty} y =2$ suy ra $y=2$ là tiệm cận ngang.
            \item $\lim\limits_{x \to +\infty} y =5$ suy ra $y=5$ là tiệm cận ngang.
            \item $\lim\limits_{x \to 1^-} y = +\infty$ suy ra $x=1$ là tiệm cận đứng.
        \end{itemize}
        Vậy đồ thị hàm số tổng cộng có $3$ đường tiệm cận ngang và tiệm cận đứng.

    }
\end{ex}

\begin{ex}%[Đề tập huấn Sở Ninh Bình, 2019]%[Nguyễn Văn Hải, dự án(12EX-5-2019)]%[2D1B4-1]
    \immini[thm]{Cho hàm số $y=f(x)$ có bảng biến thiên như hình bên. Hỏi đồ thị hàm số $y=f(x)$ có tổng số bao nhiêu tiệm cận (tiệm cận đứng và tiệm cận ngang)?
        \haicot
        {$0$}
        {\True $2$}
        {$3$}
        {$1$}
    }{
        \begin{tikzpicture}
            \tikzset{double style/.append style = {draw=\tkzTabDefaultWritingColor,double=\tkzTabDefaultBackgroundColor,double distance=2pt}}
            \tkzTabInit[nocadre=false,lgt=1,espcl=2,deltacl=0.6]
            {$x$ /0.6,$y’$ /0.6,$y$ /2.2}
            {$-\infty$ , $1$ , $3$ , $+\infty$}
            \tkzTabLine{,+,d,+,0,-,}
            \tkzTabVar{-/$-1$ ,+D- / $+\infty$ /  $-\infty$,+/ $2$, -/$-\infty$}
    \end{tikzpicture}}
    \loigiai{
        Ta có $\lim\limits_{x\to -\infty}f(x)=-1$, $\lim\limits_{x\to +\infty}f(x)=-\infty$ nên $y=-1$ là tiệm cận ngang.\\
        Ta có $\lim\limits_{x\to 1^+}f(x)=-\infty$ nên $x=1$ là tiệm cận đứng.\\
        Vậy đồ thị hàm số có $2$ đường tiệm cận.
    }
\end{ex}


\begin{ex}%[GHK1, THCS - THPT Nguyễn Khuyến, HCM, 2019]%[Vinh Vo, 12Ex3-2019]%[2D1B4-1]
    \immini[thm]{Cho hàm số $ y = f(x) $ xác định trên $ (-2;0) \cup (0;+\infty) $ và có bảng biến thiên như hình vẽ. Số đường tiệm cận của đồ thị hàm số $ f(x) $ là
        \haicot
        {$ 4 $}
        {$ 2 $}
        {$ 1 $}
        {\True $ 3 $}
    }{
        \begin{tikzpicture}
            \tikzset{double style/.append style = {double distance=2pt}}
            \tkzTabInit[lgt=1.2,espcl=3.5,nocadre=false]
            {$x$ /0.6, $f’(x)$ /0.7,$f(x)$ /1.7}
            { $-2$ , $0$ , $+\infty$}
            \tkzTabLine{d,+,d,-, }
            \tkzTabVar{D-/ /$ -\infty $, +D+/$ +\infty $/ $ 1 $, -/ $ 0 $}
            \draw[pattern = north west lines] ($(N13)-(0.2ex,0)$) rectangle (T11);
    \end{tikzpicture}}
    \loigiai{
        Từ bảng biến thiên, ta thấy $ \heva{& \lim \limits_{ x \to -2^{+} } f(x) = - \infty \\ & \lim \limits_{x \to 0^{-} } f(x) = + \infty \\ & \lim \limits_{x \to + \infty} f(x) = 0  } $, suy ra đồ thị hàm số $ f(x) $ có $ 3 $ tiệm cận trong đó có $ 2 $ tiệm cận đứng và $ 1 $ tiệm cận ngang.
    }
\end{ex}

\begin{ex}%[Đề thi khảo sát chất lượng trường THCS-THPT Lômônôxốp, Hà Nội 2018 ,Nhật Thiện 12EX1-2019]%[2D1Y4-1]
    \immini[thm]{Cho hàm số $y=f(x)$ xác định trên $\mathbb{R}\backslash \left\{{0}\right\}$, liên tục trên mỗi khoảng xác định và có bảng biến thiên như hình bên. Hỏi đồ thị hàm số có bao nhiêu đường tiệm cận?
        \haicot
        {$1$}
        {\True $2$}
        {$3$}
        {$4$}}{\begin{tikzpicture}
            \tikzset{double style/.append style = {draw=\tkzTabDefaultWritingColor,double=\tkzTabDefaultBackgroundColor,double distance=2pt}}
            \tkzTabInit[nocadre=false,lgt=1,espcl=2.3,deltacl=0.6]{$x$ /0.6,$y'$ /0.6,$y$ /1.8}{$-\infty$,$0$,$1$,$+\infty$}
            \tkzTabLine{,+,d,-,0,+}
            \tkzTabVar{+/ $2$ ,-D-/ $-\infty$/$-\infty$, +/ $1$ ,-/ $-\infty$ /}
    \end{tikzpicture}}
    \loigiai{
        Dựa vào bảng biến thiên, ta có $$\lim\limits_{x\to -\infty}y=2;\qquad \lim\limits_{x\to 0^{\pm}}y=-\infty$$
        Vậy hàm số có một tiệm cận ngang $y=2$, một tiệm cận đứng $x=0$.
    }
\end{ex}

\begin{ex}
    Số tiệm cận đứng của đồ thị hàm số $y=\dfrac{\sqrt{x+9}-3}{x^2+x}$ là
    \choice
    {$3$}
    {$2$}
    {$0$}
    {\True $1$}
    \loigiai{
        Tập xác định $\mathscr{D}=[-9;+\infty)\setminus \{-1;0\}$. \\
        Ta có $\left\{\begin{aligned}
            &\lim\limits_{x\to -1^+} \dfrac{\sqrt{x+9}-3}{x^2+x}=+\infty \\
            &\lim\limits_{x\to -1^-} \dfrac{\sqrt{x+9}-3}{x^2+x}=-\infty
        \end{aligned}\right. \Rightarrow x=-1$ là tiệm cận đứng. \\
        Ngoài ra $\lim\limits_{x\to 0} \dfrac{\sqrt{x+9}-3}{x^2+x}=\dfrac{1}{6}$ nên $x=0$ không là tiệm cận.}
\end{ex}

\begin{ex}
    Phương trình đường tiệm cận xiên của đồ thị hàm số $y=\dfrac{2x^2-x+1}{x-1}$ là
    \choice
    {$y=x-1$}
    {\True $y=2x+1$}
    {$y=2x+3$}
    {$y=x+1$}
    \loigiai{
        Sau khi chia đa thức, ta viết lại hàm số $y=2x+1+\dfrac{2}{x-1}$.\\
        Do $\lim\limits_{x\to \pm \infty}\left[y-(2x+1)\right]=\lim\limits_{x\to \pm \infty}\dfrac{2}{x-1}=0$ nên $y=2x+1$ là đường tiệm cận xiên.}
\end{ex}

\begin{ex}
    Giao điểm của đường tiệm cận đứng và đường tiệm cận xiên của đồ thị hàm số $y=\dfrac{x^2-3x+5}{x-2}$ có tọa độ là
    \choice
    {$(2;3)$}
    {$(-2;1)$}
    {\True $(2;1)$}
    {$(-2;3)$}
    \loigiai{
        Sau khi chia đa thức, ta viết lại hàm số $y=x-1+\dfrac{3}{x-2}$.
        \begin{itemize}
            \item [$\bullet$] Đồ thị hàm số có tiệm cận đứng là $x=2$
            \item [$\bullet$] Do $\lim\limits_{x\to \pm \infty}\left[y-(x-1)\right]=\lim\limits_{x\to \pm \infty}\dfrac{3}{x-2}=0$ nên $y=x-1$ là đường tiệm cận xiên.
        \end{itemize}
        Giải hệ $\heva{&x=2\\&y=x-1} \Leftrightarrow \heva{&x=2\\&y=1}$. Suy ra, giao hai đường tiệm cận có tọa độ $(2;1)$.
    }
\end{ex}

\begin{ex}%[Đề thi giữa HK1, THPT Bình Sơn Đồng Nai, 2019]%[Phan Minh Tâm, dự án EX3]%[2D1B4-2]
    Tiệm cận đứng của đồ thị hàm số $ y=\dfrac{2x+1}{x-m} $ đi qua điểm $ M(2;5) $ khi $ m $ bằng bao nhiêu?
    \choice
    {$ m=-2 $}
    {$ m=-5 $}
    {$ m=5 $}
    {\True $ m=2 $}
    \loigiai{
        Với $m \ne -\dfrac{1}{2}$ đồ thị có tiệm cận đứng là đường thẳng $ x=m $. Tiệm cận đứng $ x=m $ đi qua $ M(2;5) $ khi chỉ khi $ m=2 $.
    }
\end{ex}

\begin{ex}%[GHK1, THPT Quế Võ 2-Bắc Ninh, 2019]%[TranTony,12EX2]%[2D1B4-2]
    Cho hàm số $ y = \dfrac{2x^2-3x+m}{x-m} $ có đồ thị $ (C) $. Tìm tất cả các giá trị của tham số $ m $ để $ (C) $ không có tiệm cận đứng.
    \choice
    {\True $ m = 0 $ hoặc $ m = 1 $}
    {$ m = 2 $}
    {$ m = 1 $}
    {$ m = 0 $}
    \loigiai{
        Đồ thị $ (C) $ không có tiệm cận đứng khi $ m $ là nghiệm của $ 2x^2-3x+m $
        \begin{align*}
            \Leftrightarrow 2m^2 - 3m + m = 0 \Leftrightarrow \hoac{& m = 0 \\& m = 1.}
        \end{align*}
    }
\end{ex}

\BTTF

\begin{ex}
    Cho hàm số $y=f(x)=\dfrac{3-2x}{x+1}$. Xét tính đúng sai của các khẳng định sau:
    \choiceTF
    {\True Tập xác định của hàm số là $\mathbb{R}\backslash\{-1\}$}
    {\True  Đồ thị hàm số có đường tiệm cận đứng là $x=-1$}
    {Đồ thị hàm số có đường tiệm cận ngang là $y=3$}
    {Hai đường tiệm cận (đứng và ngang) của đồ thị tạo với hai trục tọa độ một hình phẳng có diện tích bằng $3$}
    \loigiai{
        \begin{itemchoice}
            \itemch Điều kiện xác định $x+1 \ne 0 \Leftrightarrow x \ne -1$. Suy ra $D=\mathbb{R}\backslash\{-1\}$.
            \itemch Đồ thị hàm số có tiệm cận đứng là $x=-1$.
            \itemch Đồ thị hàm số có tiệm cận ngang là $y=\dfrac{-2}{1}=-2$.
            \itemch Hai đường tiệm cận (đứng và ngang) của đồ thị tạo với hai trục tọa độ một hình chữ nhật như hình vẽ
            \begin{center}
                \begin{tikzpicture}[smooth,samples=300,scale=0.8,>=stealth]
                    \draw[->] (-3,0)--(2,0) node[below]{$x$};
                    \draw[->] (0,-3)--(0,1) node[right]{$y$};
                    \draw (0,0) node[below right]{$O$};
                    \draw[pattern = north west lines] (0,0)--(0,-2)--(-1,-2)--(-1,0);
                    \draw (-3,-2)--(2,-2)node[above]{\scriptsize TCN $y=-2$} (-1,-3)--(-1,1)node[above]{\scriptsize TCĐ $x=-1$};
                    \draw[fill=black] (-1,0) circle(1.5pt) (-1,-2) circle(1pt) (0,-2) circle(1.5pt);
                    \node[right] at (0,-2.3) {$A$};
                    \node[left] at (-1,0.3) {$B$};
                \end{tikzpicture}
            \end{center}
            Diện tích hình chữ nhật này là
            $$S=OA \cdot OB=2 \cdot 1=2.$$
    \end{itemchoice}}
\end{ex}

\begin{ex}%[2D1K4]
    Cho hàm số $y=f(x)$ xác định trên $(-\infty;2) \backslash\{-2\}$ và có bảng biến thiên như hình vẽ dưới đây.
    \begin{center}
        \begin{tikzpicture}
            \tikzset{double style/.append style = {draw=\tkzTabDefaultWritingColor,double=\tkzTabDefaultBackgroundColor,double distance=2pt}}
            \tkzTabInit[nocadre=false,lgt=1,espcl=3]
            {$x$ /0.7,$y'$ /0.7,$y$ /2.1}
            {$-\infty$,$-2$,$0$,$2$,$+\infty$}
            \tkzTabLine{,+,d,-,0,+,d,}
            \tkzTabVar{-/$-2$/,+D+/ $3$ / $+\infty$,-/$-2$/,+D/ $+\infty$ / }
            \draw[pattern = north west lines] ($(N43)+(0.1ex,0)$) rectangle (T21);
        \end{tikzpicture}
    \end{center}
    Xét tính đúng sai của các khẳng định sau:
    \choiceTF
    {\True Hàm số có giá trị nhỏ nhất bằng $-2$}
    {Hàm số có giá trị lớn nhất bằng $3$}
    {\True Đồ thị hàm số có hai đường tiệm cận đứng là $x=-2$ và $x=2$}
    {Đồ thị hàm số có hai đường tiệm cận ngang là $y=-2$ và $y=3$}
    \loigiai
    {
        Căn cứ vào bảng biến thiên của hàm số, ta có
        \begin{itemchoice}
            \itemch Hàm số đạt giá trị nhỏ nhất bằng $-2$ khi $x=0$.
            \itemch Do $\lim\limits_{x\to 2^{-}}f(x)=+\infty$ nên hàm số không có giá trị lớn nhất.
            \itemch Do $\lim\limits_{x\to -2^{+}}f(x)=+\infty$ và $\lim\limits_{x\to 2^{-}}f(x)=+\infty$ nên đồ thị hàm số có hai đường tiệm cận đứng là $x=-2$ và $x=2$.
            \itemch Do $\lim\limits_{x\to -\infty}f(x)=-2$ nên đồ thị hàm số có hai đường tiệm cận ngang là $y=-2$.
    \end{itemchoice}}
\end{ex}

\begin{ex}
    Cho hàm số $y=f(x)=\dfrac{5 x^2+9 x+9}{x-4}$. Xét tính đúng sai của các khẳng định sau:
    \choiceTF
    {Tập xác định của hàm số là $\mathbb{R}\backslash\{-4\}$}
    {\True Đường tiệm cận đứng của đồ thị hàm số có phương trình là $x=4$}
    {\True Đường tiệm cận xiên của đồ thị hàm số có phương trình là $y=5x+29$}
    {Giao điểm hai đường tiệm cận của đồ thị hàm số có toạ độ là $(4;29)$}
    \loigiai{
        Hàm số được viết thành $y=5x+29+\dfrac{125}{x-4}$.
        \begin{itemchoice}
            \itemch Điều kiện $x-4 \ne 0 \Leftrightarrow x \ne 4$. Suy ra $D=\mathbb{R}\backslash\{4\}$.
            \itemch Đường tiệm cận đứng của đồ thị hàm số có phương trình là $x=4$
            \itemch Đường tiệm cận xiên của đồ thị hàm số có phương trình là $y=5x+29$
            \itemch Giải hệ $\heva{&x=4\\&y=5x+29}\Leftrightarrow \heva{&x=4\\&y=49}$. Giao điểm hai đường tiệm cận của đồ thị hàm số có toạ độ là $(4;49)$.
    \end{itemchoice}}
\end{ex}

\begin{ex}
    Cho hàm số $y=f(x)=\dfrac{x^2-4 x+7}{x-1}$. Xét tính đúng sai của các khẳng định sau:
    \choiceTF
    {\True Đường tiệm cận đứng của đồ thị hàm số có phương trình là $x=1$}
    {\True Đường tiệm cận xiên của đồ thị hàm số có phương trình là $y=x-3$}
    {\True Giao điểm hai đường tiệm cận của đồ thị hàm số có toạ độ là $(1 ;-2)$}
    {Diện tích tam giác tạo bởi đường tiệm cận xiên của đồ thị hàm số và hai trục toạ độ là $\dfrac{9}{4}$}
    \loigiai{
        Hàm số được viết thành $y=x-3+\dfrac{4}{x-1}$.
        \begin{itemchoice}
            \itemch Đường tiệm cận đứng của đồ thị hàm số có phương trình là $x=1$
            \itemch Đường tiệm cận xiên của đồ thị hàm số có phương trình là $y=x-3$
            \itemch Giải hệ $\heva{&x=1\\&y=x-3}\Leftrightarrow \heva{&x=1\\&y=-2}$. Giao điểm hai đường tiệm cận của đồ thị hàm số có toạ độ là $(1;-2)$.
            \itemch Giao của đường thẳng $d \colon y=x-3$ với các trục tọa độ lần lượt tại $A(0;-3)$ và $B(3;0)$.\\
            Diện tích tam giác $OAB$ là $S=\dfrac{1}{2} OA \cdot OB=\dfrac{9}{2}$.
    \end{itemchoice}}
\end{ex}

\BTTL

\begin{ex}%[2D1N4-1]Câu 1
    Cho hàm số $y=\dfrac{2x-1}{x+3}$. Gọi $x=m$ và $y=n$ lần lượt là đường tiệm cận đứng và tiệm cận ngang của đồ thị hàm số. Tính giá trị của biểu thức $P=\dfrac{2m-1}{n+3}$.\\
    \shortans[3]{$-1{,}4$}
    \loigiai{Ta có:\\
        $\bullet \underset{x \to -3}{\lim}\,y=\infty \Rightarrow x=-3$ là đường tiệm cận đứng.\\
        $\bullet \underset{x \to +\infty}{\lim}\,y=2$ và $\underset{x \to -\infty}{\lim}\,y=2 \Rightarrow y=2$ là đường tiệm cận ngang.\\
        Vậy $m=-3; n=2 \Rightarrow P=\dfrac{2\cdot(-3)-1}{2+3}=\dfrac{-7}{5}=-1{,}4$.}
\end{ex}

\begin{ex}%[2D1B4-3]
    Cho đồ thị $(C)\colon y=\dfrac{x-3}{x+2}$ có hai đường tiệm cận cắt nhau tại $I$. Với $O$ là gốc tọa độ, hãy tính độ dài đoạn thẳng $OI$ (làm tròn đến hàng phần trăm).\\
    \shortans[3]{$2{,}24$}
    \loigiai{
        Ta có tiệm cận đứng của đồ thị $(C)$ là $x=-2$ và tiệm cận ngang là $y=1$. Do đó $I(-2;1)$ là giao điểm của hai đường tiệm cận của đồ thị $(C)$.\\
        Ta có $OI=\sqrt{ (-2-0)^2+(1-0)^2}=\sqrt{5} \approx 2{,}24$.
    }
\end{ex}

\begin{ex}
    Nếu trong một ngày, một xưởng sản xuất được $x$ kilôgam sản phẩm thì chi phí trung bình (tính bằng nghìn đồng) cho một sản phẩm được cho bởi công thức:
    $$
    y=C(x)=\dfrac{50x+2000}{x}
    $$
    Đồ thị hàm số $C(x)$ có một đường tiệm cận ngang (khi $x \to +\infty$) là $y=y_0$. Giá trị $y_0$ bằng bao nhiêu?\\
    \shortans[3]{$50$}
    \loigiai{
        Ta có $\lim\limits_{x \rightarrow+\infty} \dfrac{50x+2000}{x}=\lim\limits_{x \rightarrow+\infty} \left(50+\dfrac{2000}{x}\right)=50$.\\
        Vậy đường thẳng $y=50$ là tiệm cận ngang của đồ thị hàm số.
    }
\end{ex}

\begin{ex}%[2D1C4-3]Câu 6
    Cho hàm số $y=\dfrac{x-2}{x^2-3mx+m}$ tìm $m$ để đồ thị hàm số có đúng một tiệm cận đứng. Biết tổng các giá trị của tham số $m$ có dạng phân số $\dfrac{a}{b}$, tính tổng $S=a+b$.\\
    \shortans[3]{$101$}
    \loigiai{Dễ thấy tử thức có một nghiệm là $x=2$ do đó để đồ thị hàm số có đúng một tiệm cận đứng thì phương trình $x^2-3mx+m=0$ có nghiệm kép hoặc có hai nghiệm phân biệt trong đó có một nghiệm bằng $2$.\\
        $\Rightarrow \hoac{&\Delta=0\\&\heva{&\Delta>0\\&4-3\cdot2m+m=0}} \Leftrightarrow \hoac{&9m^2-4m=0\\&\heva{&9m^2-4m>0\\&4-3\cdot2m+m=0}} \Leftrightarrow \hoac{&m=0\\&m=\dfrac{4}{9}\\&\heva{&\hoac{&m<0\\&m>\dfrac{4}{9}}\\&m=\dfrac{4}{5}}} \Leftrightarrow \hoac{&m=0\\&m=\dfrac{4}{9}\\&m=\dfrac{4}{5}}$\\
        Vậy tổng các giá trị của tham số $m$ bằng $\dfrac{56}{45} \Rightarrow S=101$.
    }
\end{ex}

\begin{ex}%[2D1H4-1]Câu 4
    Cho hàm số $y=\dfrac{x^2+2x-3}{x-2}$, đồ thị hàm số có đường tiệm cận xiên có dạng $(C) \colon y=ax+b$. Tính giá trị của biểu thức $P=\dfrac{a}{b}$.\\
    \shortans[3]{$0{,}25$}
    \loigiai{Ta xét $y=\dfrac{x^2+2x-3}{x-2}=x+4+\dfrac{5}{x-2} \Rightarrow (C) \colon y=x+4$ là đường tiệm cận xiên của đồ thị hàm số. Vậy $P=\dfrac{1}{4}$.}
\end{ex}

\begin{ex}
    Gọi $d$ là đường tiệm cận xiên của đồ thị hàm số $y=mx+4-3m+\dfrac{3}{x+2}$, $m$ là tham số. Đường thẳng $d$ luôn qua điểm cố định $M$. Tính độ dài đoạn $OM$, với $O$ là gốc tọa độ.\\
    \shortans[3]{$5$}
    \loigiai{
        Đường tiệm cần xiên của đồ thị là $y=mx+4-3m \Leftrightarrow (x-3)m+4-y=0$.\\
        Ta có $(x-3)m+4-y=0,\,\forall m \Leftrightarrow \heva{&x-3=0\\&4-y=0} \Leftrightarrow \heva{&x=3\\&y=4}$.\\
        Đường thẳng này luôn qua điểm cố định $M(3;4)$. Khi đó $OM=\sqrt{3^2+4^2}=5$.
    }
\end{ex}
%\boxde
\BTTN
\Opensolutionfile{ans}[ans/2D1-4-DEON-1]
\begin{ex}%[2D1B4-1]
    Cho hàm số $y=f(x)$ có bảng biến thiên như hình bên. Đồ thị hàm số đã cho có tiệm cận ngang là đường thẳng
    \begin{center}
        \begin{tikzpicture}[scale=0.8, font=\footnotesize, line join=round, line
            cap=round, >=stealth]
            \tkzTabInit[espcl=2.5,lgt=1,nocadre=false]
            {$x$/0.7,$f(x)$/2.1}
            {$-\infty$,$0$,$1$,$2$,$+\infty$}
            \tkzTabVar{-/$-\infty$,+/$2$,-D+/$-\infty$/$+\infty$,-/$4$,+/$6$}
        \end{tikzpicture}
    \end{center}
    \choice
    {$y=2$}
    {$y=1$}
    {\True $y=6$}
    {$y=4$}
    \loigiai{Dựa vào bảng biến thiên ta thấy đồ thị hàm có tiệm cận ngang $y=6$.
    }
\end{ex}
%56
\begin{ex}%[2D1B4-1]
    Cho hàm số $y=f(x)$ có bảng biến thiên như hình bên. Tổng số tiệm cận đứng và tiệm cận ngang của đồ thị hàm số đã cho là
    \begin{center}
        \begin{tikzpicture}[scale=0.8]
            \tkzTabInit[nocadre=false,lgt=1.5,espcl=3,deltacl=0.6]
            {$x$ /0.6,$y’$ /0.6,$y$ /2}
            {$-\infty$ , $1$, $+\infty$}
            \tkzTabLine{,+,d,+,}
            \tkzTabVar{-/$2$,+D-/$+\infty$/$3$,+/$5$}
        \end{tikzpicture}
    \end{center}
    \choice
    {$1$}
    {\True $3$}
    {$2$}
    {$4$}
    \loigiai{Dựa vào bảng biến thiên ta thấy đồ thị hàm số có tiệm cận đứng $x=1$ và tiệm cận ngang $y=2$ và $y=5$.}
\end{ex}
\begin{ex}%[2D1B4-1]
    Cho hàm số $y=f(x)$ có bảng biến thiên như hình bên. Tổng số tiệm cận đứng và tiệm cận ngang của đồ thị hàm số đã cho là
    \begin{center}
        \begin{tikzpicture}[scale=0.8]
            \tkzTabInit[nocadre=false,lgt=1.5,espcl=3,deltacl=0.6]
            {$x$ /0.6,$y’$ /0.6,$y$ /2}
            {$-\infty$ ,$0$, $1$, $+\infty$}
            \tkzTabLine{,+,0,-,d,-,}
            \tkzTabVar{-/$4$,+/$2$,-D+/$-1$/$+\infty$,-/$-3$}
        \end{tikzpicture}
    \end{center}
    \choice
    {$1$}
    {\True $3$}
    {$2$}
    {$4$}
    \loigiai{
        Dựa vào bảng biến thiên ta thấy đồ thị hàm số có tiệm cận đứng $x=1$, tiệm cận ngang $y=4$ và $y=-3$.
    }
\end{ex}
%61
\begin{ex}%[2D1B4-1]
    Cho hàm số $y=f(x)$ có bảng biến thiên như hình bên. Tổng số tiệm cận đứng và tiệm cận ngang của đồ thị hàm số đã cho là
    \begin{center}
        \begin{tikzpicture}[scale=0.8]
            \tkzTabInit[nocadre=false,lgt=1.5,espcl=3,deltacl=0.6]
            {$x$ /0.6,$y’$ /0.6,$y$ /2}
            {$-\infty$ ,$0$, $1$, $+\infty$}
            \tkzTabLine{,-,0,+,d,+,}
            \tkzTabVar{+/$5$,-/$-4$,+D-/$+\infty$/$-\infty$,+/$2$}
        \end{tikzpicture}
    \end{center}
    \choice
    {$1$}
    {\True $3$}
    {$2$}
    {$4$}
    \loigiai{Dựa vào bảng biến thiên ta thấy đồ thị hàm số có một tiệm cận đứng $x=1$, hai tiệm cận ngang $y=5$ và $y=2$.}
\end{ex}
\begin{ex}%[2D1B4-1]
    Đồ thị hàm số nào trong các hàm số dưới đây có tiệm cận đứng?
    \choice
    {\True $y=\dfrac{1}{\sqrt{x}}$}
    {$y=\dfrac{1}{x^2+x+1}$}
    {$y=\dfrac{1}{x^4+1}$}
    {$y=\dfrac{1}{x^2+1}$}
    \loigiai{
    }
\end{ex}
\begin{ex}%[2D1K4-1]
    Số tiệm cận đứng của đồ thị hàm số $y=\dfrac{\sqrt{x+4}-2}{x^2+x}$ là
    \choice
    {$3$}
    {$0$}
    {$2$}
    {\True $1$}
    \loigiai{
        Tập xác định hàm số $ \mathscr{D}=[-4;+\infty)\setminus\lbrace -1;0\rbrace
        $.\\
        Ta có $ \lim\limits_{x\to -1^{+}}y=+\infty $, $ \lim\limits_{x\to 0^{+}}y=1
        $ và $ \lim\limits_{x\to 0^{-}}y=1 $.\\
        Suy ra đồ thị hàm số chỉ có $ 1 $ tiệm cận đứng là $ x=-1 $.
    }
\end{ex}
\begin{ex}%[Nguyễn Văn Sang, dự án Tex hoá đề cương trường Marie Curie - Lần 6]%[2D1Y4-1]
    Đường thẳng nào dưới đây là tiệm cận ngang của đồ thị hàm số $y=\dfrac{3+2 x}{x+1}$?
    \choice
    {$y=3$}
    {$x=-1$}
    {\True $y=2$}
    {$x=2$}
    \loigiai{
        Tập xác định $\mathscr{D}=\mathbb{R}\setminus\left\lbrace -1\right\rbrace$.
        \begin{itemize}
            \item $\lim\limits_{x \to \pm\infty} y=\lim\limits_{x \to \pm\infty} \dfrac{3+2 x}{x+1}=2$ suy ra $y=2$ là tiệm cận ngang.
            \item $\heva{& \lim\limits_{x \to -1^+} \dfrac{3+2 x}{x+1}=+\infty \\ & \lim\limits_{x \to -1^-} \dfrac{3+2 x}{x+1}=-\infty}$ suy ra $x=-1$ là tiệm cận đứng.
        \end{itemize}
    }
\end{ex}
%%=====Câu 15
\begin{ex}%[2D1Y4-1]
    Giao điểm của tiệm cận đứng và tiệm cận ngang của đồ thị hàm số $y=\dfrac{3x-2}{1-x}$ là điểm
    \choice
    {$M(1;3)$}
    {$P(-3;1)$}
    {\True $Q(1;-3)$}
    {$N\left(\dfrac{2}{3};3\right)$}
    \loigiai{
        Tiệm cận đứng, tiệm cận ngang của đồ thị hàm số lần lượt là $x=1$ và $y=-3$. Giao điểm của $2$ tiệm cận là $Q(1;-3)$.
    }
\end{ex}
\begin{ex}%[2D1K4-2]%
    Nếu đồ thị hàm số $y=\dfrac{(m+1)x+2}{x-n+1}$ lần lượt nhận trục hoành và trục tung làm đường đường tiệm cận ngang và tiệm cận đứng thì $m+n$ bằng bao nhiêu?
    \choice
    {\True $m+n=0$}
    {$m+n=2$}
    {$m+n=-1$}
    {$m+n=1$}
    \loigiai{
        Theo đề bài, ta có $\heva{&m+1=0\\&n-1=0} \Leftrightarrow \heva{&m=-1\\&n=1.}$\\
        Suy ra $m+n=0$.
    }
\end{ex}
\begin{ex}%[2D1K4-1]
    Cho hàm số $y=f(x)$ có bảng biến thiên như hình bên. Đồ thị hàm số $y=\dfrac{x-2}{f(x)-1}$ có bao nhiêu tiệm cận đứng?
    \begin{center}
        \begin{tikzpicture}
            \tkzTabInit[espcl=3]{$x$ / 1 , $f’(x)$ / 1, $f(x)$ / 2}
            {$-\infty$, $-1$ , $5$, $+\infty$}%
            \tkzTabLine{,-,0,+,0,-,}%
            \tkzTabVar{+/ $+\infty$, - / $-1$, + / $3$,-/$-2$}%
            \tkzTabVal[draw]{2}{3}{0.4}{$2$}{$1$}
        \end{tikzpicture}
    \end{center}
    \choice
    {$1$}
    {$3$}
    {\True $2$}
    {$4$}
    \loigiai{
        Dựa vào bảng biến thiên suy ra
        $f(x)-1=0 \Leftrightarrow f(x) =1$, phương trình này có $2$ nghiệm phân biệt khác $2$ và một nghiệm $x=2$ nên đồ thị hàm số $y=\dfrac{x-2}{f(x)-1}$ có hai tiệm cận đứng.
    }
\end{ex}
%68
\begin{ex}%[2D1K4-1]
    Cho hàm số $y=f(x)$ có bảng biến thiên như hình bên. Đồ thị hàm số $y=\dfrac{1}{2f(x)+1}$ có bao nhiêu tiệm cận đứng?
    \begin{center}
        \begin{tikzpicture}[scale=0.8]
            \tkzTabInit[nocadre=false,lgt=1.5,espcl=3,deltacl=0.6]
            {$x$ /0.6,$y’$ /0.6,$y$ /2}
            {$-\infty$ ,$-2$, $2$, $+\infty$}
            \tkzTabLine{,+,0,-,0,+,}
            \tkzTabVar{-/$-\infty$,+/$3$,-/$0$,+/$+\infty$}
        \end{tikzpicture}
    \end{center}
    \choice
    {\True $1$}
    {$3$}
    {$2$}
    {$0$}
    \loigiai{
        Dựa vào bảng biến thiên suy ra
        $2f(x)+1=0 \Leftrightarrow f(x) =-\dfrac{1}{2}$, phương trình này có $1$ nghiệm nên đồ thị hàm số $y=\dfrac{1}{2f(x)+1}$ có một tiệm cận đứng.
    }
\end{ex}
\begin{ex}%[2D1K4-1]
    Cho hàm số $y=f(x)$ có bảng biến thiên như hình bên. Đồ thị hàm số $y=\dfrac{1}{2f(x)-1}$ có bao nhiêu tiệm cận ngang?
    \begin{center}
        \begin{tikzpicture}[scale=0.8]
            \tkzTabInit[nocadre=false,lgt=1.5,espcl=3,deltacl=0.6]
            {$x$ /0.6,$y’$ /0.6,$y$ /2}
            {$-\infty$, $2$, $+\infty$}
            \tkzTabLine{,-,0,+,}
            \tkzTabVar{+/$1$,-/$-3$,+/$1$}
        \end{tikzpicture}
    \end{center}
    \choice
    {$1$}
    {\True $3$}
    {$2$}
    {$0$}
    \loigiai{
        Dựa vào bảng biến thiên suy ra
        \begin{itemize}
            \item 	$\lim \limits_{x \to \pm \infty} f(x)=1 \Leftrightarrow \lim \limits_{x \to \pm \infty}\dfrac{1}{2f(x)-1} =1$ nên đồ thị hàm số đã cho có tiệm cận ngang là $y=1$.
            \item $2f(x)-1=0 \Leftrightarrow f(x)=\dfrac{1}{2}$, phương trình này có $2$ nghiệm phân biệt nên đồ thị hàm số đã cho có hai tiệm cận đứng.
        \end{itemize}
    }
\end{ex}
%80
\begin{ex}%[2D1K4-1]
    Cho hàm số $y=f(x)$ có bảng biến thiên như hình bên. Đồ thị hàm số $y=\dfrac{1}{f^2(x)+f(x)}$ có bao nhiêu tiệm cận đứng?
    \begin{center}
        \begin{tikzpicture}[scale=0.8]
            \tkzTabInit[nocadre=false,lgt=1.5,espcl=3,deltacl=0.6]
            {$x$ /0.6,$y’$ /0.6,$y$ /2}
            {$-\infty$ ,$-4$, $6$, $+\infty$}
            \tkzTabLine{,-,0,+,0,-,}
            \tkzTabVar{+/$+\infty$,-/$-2$,+/$5$,-/$-\infty$}
        \end{tikzpicture}
    \end{center}
    \choice
    {$4$}
    {$3$}
    {$2$}
    {\True $6$}
    \loigiai{
        Dựa vào bảng biến thiên suy ra $f^2(x)+f(x)=0 \Leftrightarrow \hoac{&f(x)=0\\&f(x)=-1}$, mỗi phương trình này có $3$ nghiệm phân biệt nên đồ thị hàm số đã cho có $6$ tiệm cận đứng.
    }
\end{ex}
\begin{ex}%[2D1K4-1]
    Cho hàm số $y=f(x)$ có bảng biến thiên như hình bên. Đồ thị hàm số $y=\dfrac{3}{f(x^2)+1}$ có bao nhiêu tiệm cận đứng?
    \begin{center}
        \begin{tikzpicture}[scale=0.8]
            \tkzTabInit[nocadre=false,lgt=1.5,espcl=3,deltacl=0.6]
            {$x$ /0.6,$y’$ /0.6,$y$ /2}
            {$-\infty$ ,$0$, $2$, $+\infty$}
            \tkzTabLine{,+,d,-,0,+,}
            \tkzTabVar{-/$-\infty$,+/$1$,-/$-2$,+/$+\infty$}
        \end{tikzpicture}
    \end{center}
    \choice
    {\True $4$}
    {$3$}
    {$6$}
    {$0$}
    \loigiai{
        Dựa vào bảng biến thiên suy ra
        $f(x^2)+1=0 \Leftrightarrow f(x^2) =-1$. Kẻ đường thẳng $y=-1$ ta thấy đường thẳng cắt đồ thị hàm số tại 3 điểm phân biệt. Suy ra
        $$\hoac{&x^2=a \; (a<0)\\&x^2=b \; (b \in (0;2)\\&x^2=c \; (c>2)} \Rightarrow \hoac{&x=\pm \sqrt{b}\\&x=\pm \sqrt{c}.}$$
        Do đó đồ thị hàm số $y=\dfrac{2}{f(x^2)+1}$ có $4$ tiệm cận đứng.
    }
\end{ex}
\BTTF
\begin{ex}%[EX-TF-2024, Lê Đạt]%[2D1N4-1]
    Cho hàm số $y=\dfrac{2x-3}{x-1}$. Xét tính đúng sai các khẳng định dưới đây
    \choiceTF
    {\True Đường tiệm cận đứng của đồ thị hàm số là $ x=1 $}
    {Đường tiệm cận đứng của đồ thị hàm số là $ y=2 $}
    {Đường tiệm cận ngang của đồ thị hàm số là $ x=1 $}
    {\True Đường tiệm cận ngang của đồ thj hàm số là $ y=2 $}
    \loigiai{
        Ta có $\lim\limits_{x\to -\infty}y=\lim\limits_{x\to +\infty}y=2$ nên đồ thị hàm số đã cho có tiệm cận ngang là $y=2$.\\
        Ta có $\lim\limits_{x\to 1^+}y=-\infty$ nên đồ thị hàm số đã cho có tiệm cận ngang là $ x=1 $.
        \begin{itemchoice}
            \itemch Đường tiệm cận đứng của đồ thị hàm số là $ x=1 $.
            \itemch Đường tiệm cận đứng của đồ thị hàm số là $ x=1 $.
            \itemch Đường tiệm cận ngang của đồ thj hàm số là $ y=2 $.
            \itemch Đường tiệm cận ngang của đồ thj hàm số là $ y=2 $.
        \end{itemchoice}
    }
\end{ex}
%===== DẠNG 2
\begin{ex}%[EX-TF-2024, Lê Đạt]%[2D1H4-2]
    Cho hàm số $ y=\dfrac{m^2x+1}{x-1} $. Xét tính đúng sai của các khẳng định sau
    \choiceTF
    {\True Đồ thị hàm số luôn có tiệm cận ngang}
    {\True Đồ thị hàm số luôn có tiệm cận đứng}
    {\True Khi $ m=1$ đồ thị hàm số có $ 2 $ đường tiệm cận}
    {Khi $ m=0 $ đồ thị hàm số có $ 1 $ đường tiệm cận}
    \loigiai{
        \begin{itemchoice}
            \itemch $\lim\limits_{x\to -\infty}y=\lim\limits_{x\to +\infty}y=m^2$ suy ra hàm số luôn có tiệm cận ngang.
            \itemch $\lim\limits_{x\to 1^+}y=+\infty$ nên đồ thị hàm số đã cho có tiệm cận ngang là $ x=1 $.
            \itemch Khi $ m=1 $ ta được hàm số $ y=\dfrac{x+1}{x-1} $ suy ra đồ thì hàm số có $ x=1 $ là tiệm cận đứng và $ y=1 $ là tiệm cận ngang nên đồ thị hàm số có $ 2 $ tiệm cận.
            \itemch Khi $ m=0 $ ta được hàm số $ y=\dfrac{1}{x-1} $ suy ra đồ thì hàm số có $ x=1 $ là tiệm cận đứng và $ y=0 $ là tiệm cận ngang nên đồ thị hàm số có $ 2 $ tiệm cận.
        \end{itemchoice}
    }
\end{ex}
%===== DẠNG 3
\begin{ex}%[EX-TF-2024, Lê Đạt]%[2D1N4-3]
    \immini{Cho hàm số $y=f(x)$ có đồ thị như hình bên. Xét tính đúng sai của các khẳng định sau
        \choiceTF
        {$ x=2 $ là đường tiệm cận ngang của đồ thị hàm số}
        {\True $ x=-1 $ là đường tiệm cận đứng của đồ thị hàm số}
        {\True Đồ thị hàm số có hai đường tiệm cận}
        {\True Đồ thị hàm số không có tiệm cận xiên}
    }{
        \begin{tikzpicture}[scale=0.5, font=\footnotesize, line join=round, line cap=round, >=stealth]
            \draw[->](-5,0)--(5,0)node[below]{ $x$};
            \draw[->](0,-4)--(0,5)node[right]{ $y$};
            \draw [fill=black,draw=black] (0,0) circle (1pt)node[above left] { $O$};
            \foreach \x in {-1}\draw[shift={(\x,0)}](0pt,-2pt)--(0pt,2pt) node[below left]{ $\x$};
            \foreach \y in {2}\draw[shift={(0,\y)}](-2pt,0pt)--(2pt,0pt)node[above right]{ $\y$};
            \clip(-5,-4) rectangle (5,5);
            \draw[smooth,samples=100,domain=-5:-1.1] plot(\x,{(2*(\x)-1)/((\x)+1)});
            \draw[smooth,samples=100,domain=-0.9:5] plot(\x,{(2*(\x)-1)/((\x)+1)});
            \draw[dashed](-5,2)--(5,2) (-1,-4)--(-1,5);
        \end{tikzpicture}
    }
    \loigiai{
        \begin{itemchoice}
            \itemch $ y=2 $ là đường tiệm cận ngang của đồ thị hàm số.
            \itemch $ x=-1 $ là đường tiệm cận đứng của đồ thị hàm số.
            \itemch $ x=-1 $ là đường tiệm cận đứng và $ y=2 $ là đường tiệm cận ngang của đồ thị hàm số suy ra đồ thị hàm số có hai đường tiệm cận.
            \itemch Đồ thị hàm số không có tiệm cận xiên.
        \end{itemchoice}
    }
\end{ex}
\BTTL
\begin{ex}%[2D1K4-2]%
    Đường tiệm cận đứng và tiệm cận ngang của đồ thị hàm số $y=\dfrac{mx+1}{2m+1-x}$ cùng với hai trục tọa độ tạo thành một hình chữ nhật có diện tích bằng $3$. Khi đó $m$ bằng
    \shortans{$1$ hay $-\dfrac{3}{2}$}
    % \choice
    % {$1$ hay $\dfrac{3}{2}$}
    % {$-1$ hay $-\dfrac{3}{2}$}
    % {\True $1$ hay $-\dfrac{3}{2}$}
    % {$-1$ hay $3$}
    \loigiai{
        Từ yêu cầu đề bài, suy ra $|-m| \cdot |2m+1|=3 \Leftrightarrow \hoac{&m=1\\&m=-\dfrac{3}{2}.}$
    }
\end{ex}
\begin{ex}%[2D1K4-2]%
    Tìm tất cả các giá trị thực $m$ sao cho đồ thị hàm số $y=\dfrac{5x-3}{x^2-2mx+1}$ không có tiệm cận đứng.
    \shortans{$-1<m<1$}
    % \choice
    % {\True $-1<m<1$}
    % {$m=1$}
    % {$m=-1$}
    % {$m <-1$ hoặc $m>1$}
    \loigiai{
        Xét $f(x)=5x-3$, có $f(x)=0\Leftrightarrow x=\dfrac{3}{5}$; $g(x)=x^2-2mx+1$ có $\Delta’=m^2-1$.\\
        Đồ thị hàm số không có tiệm cận đứng khi phương trình $g(x)=0$ vô nghiệm $\Leftrightarrow m^2-1<0\Leftrightarrow-1<m<1$.\\
        Vậy với $-1<m<1$ thì đồ thị hàm số đã cho không có tiệm cận đứng.}
\end{ex}
\begin{ex}%[Nguyễn Văn Sang, dự án VDC-Hàm số 2020 - Lần 2]%[2D1K4-2]%
    Cho hàm số $y=\dfrac{1+\sqrt{x+1}}{\sqrt{x^2-mx-3m}}$ với $m$ là tham số. Tìm tập hợp các giá trị của tham số $m$ để đồ thị hàm số có hai tiệm cận đứng.
    \shortans{$\left(0;\dfrac{1}{2}\right)$}
    % \choice
    % {\True $\left(0;\dfrac{1}{2}\right)$}
    % {$\left(\left. 0;\dfrac{1}{2}\right]\right.$}
    % {$\left(0;+\infty \right)$}
    % {$\left(-\infty;-12\right)\cup \left(0;+\infty \right)$}
    \loigiai{
        Ta có $\sqrt{x+1}$ xác định khi $x\ge-1.$\\
        Yêu cầu bài toán $\Leftrightarrow $ phương trình $x^2-mx-3m=0$ có hai nghiệm phân biệt $x_1$, $x_2$ thỏa mãn $$-1<x_1<x_2\Leftrightarrow \heva{
            & \Delta >0 \\
            & x_1+x_2>-2 \\
            & a\cdot f\left(-1\right)>0 \\}\Leftrightarrow \heva{
            & m^2+12m>0 \\
            & m>-2 \\
            & 1\cdot \left(1-2m\right)>0 \\}\Leftrightarrow 0<m<\dfrac{1}{2}.$$
    }
\end{ex}
\begin{ex}%[VDC5-NgocDungHo]%[2D1G4-3]%
    \immini{Cho hàm số $f(x)$ có đồ thị như hình bên. Số đường tiệm cận đứng của đồ thị hàm số $y=\dfrac{(x^2-4)(x^2+2x)}{[f(x)]^2+2f(x)-3}$ là bao nhiêu?
        \shortans{$4$}
        % \choice
        % {\True $4$}
        % {$5$}
        % {$3$}
        % {$2$}
    }{\begin{tikzpicture}[>=stealth,scale=0.5, line join=round, line cap=round]
            \def\f[#1]{0.25*((#1)^4-8*(#1)^2+4)}
            \draw[->] (-4.1,0)--(4,0) node [below]{$x$};
            \draw[->] (0,-3.5)--(0,4) node [left]{$y$};
            \node at (0,0) [below left]{$O$};
            % \clip;
            \draw[domain=-3:3,samples=300,thick] plot (\x,{\f[\x]});
            \foreach \x in {-2,2} \filldraw (\x,0) node[above]{\x} circle (2pt);
            \foreach \x in {-3,3} \filldraw (\x,0) node[below]{\x} circle (2pt);
            \filldraw (0,1) node[above left]{$1$} circle (2pt);
            \filldraw (0,-3) node[below left]{$-3$} circle (2pt);
            \draw[dashed](-2,0)--(-2,-3)--(2,-3)--(2,0);
            \draw (3,2.75) node[right]{$y=f(x)$};
    \end{tikzpicture}}
    \loigiai{%GV tổng quát hóa bài toán:
        Cho hàm số $f(x)$ có đồ thị $(C)$ cho trước. Xác định số đường tiệm cận đứng của đồ thị hàm số $y=\dfrac{u(x)}{v[f(x)]}$.
        \begin{enumerate}
            \item Tìm tập xác định của hàm số $y=\dfrac{u(x)}{v[f(x)]}$.\\
            \item Tìm nghiệm của phương trình $u(x)=0\quad (1)$.\\
            \item Tìm nghiệm của phương trình $v[f(x)]=0\quad (2)$. Giả sử $f(x)=m_1$, $f(x)=m_2,\ldots$.
        \end{enumerate}
        Dựa vào đồ thị $(C)$, xác định hoành độ giao điểm của $(C)$ với các đường thẳng $d_1\colon f(x)=m_1$, $d_2\colon f(x)=m_2,\ldots$.\\
        Số đường tiệm cận đứng của đồ thị hàm số $y=\dfrac{u(x)}{v[f(x)]}$ chính là tổng của:
        \begin{itemize}
            \item Số nghiệm riêng của phương trình $(2)$.
            \item Số nghiệm chung $x=x_0$ của $(1) $ và $(2)$ mà bậc của $(x-x_0)$ ở mẫu lớn hơn bậc của $(x-x_0)$ ở tử.
        \end{itemize}
        \noindent
        Xét hàm số $y=g(x)=\dfrac{(x^2-4)(x^2+2x)}{[f(x)]^2+2f(x)-3}$.
        \immini
        {
            Giải phương trình $(x^2-4)(x^2+2x)=0\,(1)$\\$ \Leftrightarrow \hoac{& x^2-4=0 \\ & x^2+2x=0}\Leftrightarrow \hoac{& x=\pm 2 \\ & x=0.}$\\
            Giải phương trình $[f(x)]^2+2f(x)-3=0\,(2)$\\
            $ \Leftrightarrow \hoac{& f(x)=1 \\ & f(x)=-3.}$\\
        }
        {\begin{tikzpicture}[>=stealth,scale=0.7, line join=round, line cap=round]
                \def\f[#1]{0.25*((#1)^4-8*(#1)^2+4)}
                \def\g[#1]{1}
                \def\h[#1]{-3}
                \draw[->] (-4.1,0)--(4,0) node [below]{$x$};
                \draw[->] (0,-3.5)--(0,4) node [left]{$y$};
                \node at (0,0) [below left]{$O$};
                % \clip;
                \draw[domain=-3:3,samples=300,thick] plot (\x,{\f[\x]});
                \draw[domain=-4:4,samples=300,thick] plot (\x,{\g[\x]});
                \draw[domain=-4:4,samples=300,thick] plot (\x,{\h[\x]});
                \foreach \x in {-2,2} \filldraw (\x,0) node[above]{\x} circle (2pt);
                \filldraw (-3,0) node[above left]{$-3$} circle (2pt);
                \filldraw (3,0) node[above right]{$3$} circle (2pt);
                \filldraw (-2.85,0) node[below]{$a$} circle (2pt);
                \filldraw (2.85,0) node[below]{$b$} circle (2pt);
                \filldraw (0,1) node[above left]{$1$} circle (2pt);
                \filldraw (0,-3) node[below left]{$-3$} circle (2pt);
                \draw[dashed](-2,0)--(-2,-3)--(2,-3)--(2,0) (-2.85,0)--(-2.85,1) (2.85,0)--(2.85,1);
                \draw (3,2.75) node[right]{$(C):y=f(x)$};
                \draw (4.2,1) node[above]{$d_1:y=1$};
                \draw (4,-3) node[below]{$d_2:y=-3$};
            \end{tikzpicture}
        }
        Dựa vào đồ thị đã cho $(2)\Leftrightarrow \hoac{& x = \pm 2 \\ & x=0\\&x=a\\&x=b.}$
        với $-3<a<-2<2<b<3$.\\
        Trong điều kiện xác định của hàm số $y=g(x)$ ta có thể viết $$y=g(x)=\dfrac{x(x-2)(x+2)^2}{x^2(x-a)(x-b)(x-2)^2(x+2)^2}=\dfrac{1}{x(x-a)(x-b)(x-2)}$$
        Vậy số tiệm cận đứng của đồ thị hàm số $y=g(x)$ bằng $4$.
    }
\end{ex}
\begin{ex}
    \immini{%Câu 97.
        Đường cong ở hình bên là đồ thị của hàm số $y = ax^3 +bx^2 +cx+d$. Đồ thị hàm số $y =\dfrac{(x+1)\sqrt{1-x}}{f(x^2)}$ có tất cả bao nhiêu tiệm cận đứng?
        \shortans{$2$}
        % \choice
        % {1}
        % {6}
        % {4}
        % {\True 2}
    }{\begin{tikzpicture}[scale=.6, font=\footnotesize, line join=round, line cap=round, >=stealth]
            \def\xmin{-2}\def\xmax{4}\def\ymin{-3}\def\ymax{3}
            \draw[->] (\xmin-0.2,0)--(\xmax+0.2,0) node[below] {\footnotesize $x$};
            \draw[->] (0,\ymin-0.2)--(0,\ymax+0.2) node[right] {\footnotesize $y$};
            \draw (0,0) node [below left] {\footnotesize $O$};
            \foreach \x in {1,2}\draw (\x,-0.1)--(\x,0.1) node [above ] {\footnotesize $\x$};
            \foreach \x in {-1,3}\draw (\x,-0.1)--(\x,0.1) node [above left] {\footnotesize $\x$};
            \foreach \y in {-2}\draw (0.1,\y)--(-0.1,\y) node [left] {\footnotesize $\y$};
            \foreach \y in {2}\draw (-0.1,\y)--(0.1,\y) node [right] {\footnotesize $\y$};
            \clip (\xmin,\ymin) rectangle (\xmax,\ymax);
            \draw[smooth,samples=200,domain=\xmin:\xmax] plot (\x,{0.6666666666666666*((\x)^3)+-2*((\x)^2)+-0.6666666666666666*(\x)+2});
            \draw[dashed] (1.0,0)--(1.0,0.0)--(0,0.0);\fill (1.0,0.0) circle (1pt);
            \draw[dashed] (2,0)--(2,-2)--(0,-2);
    \end{tikzpicture}}
    \loigiai{
        * Điều kiện: $\heva{&f(x^2) \ne 0\\&x \le 1.}$\\
        Nhìn hình vẽ ta thấy
        $f(x^2)=0\Leftrightarrow \hoac{&x^2=-1\\&x^2=1\\&x^2=3}\Leftrightarrow \hoac{&x=\pm 1\,(\text{nghiệm đơn})\\&x=- \sqrt{3}\,(\text{nghiệm đơn})\\&x= \sqrt{3}\,(\text{không thỏa mãn})}.$\\
        Vậy $y=\dfrac{(x+1)\sqrt{1-x}}{f(x^2)}=\dfrac{(x+1)\sqrt{1-x}}{(x - 1)(x + 1)(x + \sqrt{3})}$ \\
        Đồ thị hàm số có 2 đường tiệm cận đứng.}
\end{ex}
\begin{ex}
    \immini{ %Câu 95.
        Đường cong ở hình bên là đồ thị của hàm số $y = ax^3 +bx^2 +cx+d$. Đồ thị hàm số $y =\dfrac{(2x+1)\sqrt{x-1}}{f(|x|)}$ có tất cả bao nhiêu tiệm cận đứng?
        \shortans{$1$}
        % \choice
        % {\True 1}
        % {3}
        % {4}
        % {2}
    }{\begin{tikzpicture}[scale=.5, font=\footnotesize, line join=round, line cap=round, >=stealth]
            \def\xmin{-3}\def\xmax{3}\def\ymin{-5}\def\ymax{5}
            \draw[->] (\xmin-0.2,0)--(\xmax+0.2,0) node[below] {\footnotesize $x$};
            \draw[->] (0,\ymin-0.2)--(0,\ymax+0.2) node[right] {\footnotesize $y$};
            \draw (0,0) node [below left] {\footnotesize $O$};
            \foreach \x in {-1,2}\draw (\x,0.1)--(\x,-0.1) node [below] {\footnotesize $\x$};
            \foreach \x in {-2,1}\draw (\x,-0.1)--(\x,0.1) node [above] {\footnotesize $\x$};
            \foreach \y in {-4,2}\draw (-0.1,\y)--(0.1,\y) node [right] {\footnotesize $\y$};
            \foreach \y in {-2,4}\draw (0.1,\y)--(-0.1,\y) node [left] {\footnotesize $\y$};
            \clip (\xmin,\ymin) rectangle (\xmax,\ymax);
            \draw[smooth,samples=200,domain=\xmin:\xmax] plot (\x,{1.3333333333333333*((\x)^3)+0*((\x)^2)+-3.3333333333333335*(\x)+0});
            \draw[dashed] (-2,0)--(-2,-4)--(0,-4);
            \draw[dashed] (2,0)--(2,4)--(0,4);
            \draw[dashed] (1,0)--(1,-2)--(0,-2);
            \draw[dashed] (-1,0)--(-1,2)--(0,2);
    \end{tikzpicture}}
    \loigiai{
        * Điều kiện: $\heva{&f(|x|) \ne 0\\&x \ge 1.}$\\
        Nhìn hình vẽ ta thấy
        $f(|x|)=0\Leftrightarrow \hoac{&|x|=x_1\,(-2<x_1<-1)\\&|x|=0\\&|x|=x_2\,(1<x_2<2)}\Leftrightarrow \hoac{&x=0&(\text{không thỏa mãn})\\&x=- x_2&(\text{không thỏa mãn})\\&x=x_2&(\text{nghiệm đơn}).}$\\
        Vậy $y =\dfrac{(2x+1)\sqrt{x-1}}{f(|x|)}=\dfrac{(2x+1)\sqrt{x-1}}{ax(x+x_2)(x-x_2)}.$ \\
        Đồ thị hàm số có 1 đường tiệm cận đứng.}
\end{ex}
\begin{ex}
    Đáp ứng tần số của một hệ thống điều khiển có thể được mô tả bởi hàm truyền \( H(s) = \dfrac{\omega_n^2}{s^2 + 2\zeta\omega_ns + \omega_n^2} \), trong đó \( \omega_n \) là tần số tự nhiên và \( \zeta \) là hệ số tắt dần. Tìm đường tiệm cận ngang của đáp ứng tần số khi tần số góc \( s \) tăng và nêu ý nghĩa của nó.
    \shortans{$y=0$}
    \loigiai{
        Khi \( s \) tăng vô hạn, các thành phần bậc cao trong mẫu số chiếm ưu thế:
        \[
        H(s) \approx \frac{\omega_n^2}{s^2}
        \]
        Đường tiệm cận ngang của \( H(s) \) khi \( s \to \infty \) là:
        \[
        |H(s)| \approx \frac{\omega_n^2}{s^2} \to 0
        \]}
\end{ex}
\begin{ex}
    Trong thuyết tương đối của Einstein, khối lượng của vật chuyển động với vận tốc $v$ được cho bởi công thức:
    $$m(v)=\dfrac{m_0}{\sqrt{1-\dfrac{v^2}{c^2}}},$$
    trong đó $m_0$ là khối lượng của vật khi nó đứng yên, $c$ là vận tốc ánh sáng.\\
    (nguồn: https://www.britannica.com/science/relativity/Relativistic-mass)\\
    Xem $m$ là hàm số theo vận tốc $v$, tìm đường tiệm cận đứng của đồ thị hàm số. Từ đó nhận xét khối lượng của vật khi vận tốc của nó càng gần với vận tốc ánh sáng.
    \shortans{$v=c$, khối lượng tăng lên vô hạn}
    \loigiai{
        Điều kiện xác định: $\heva{&1-\dfrac{v^2}{c^2}>0\\
            &v>0}\Leftrightarrow\heva{& -c<v<c\\
            &v>0}\Leftrightarrow 0<v<c$.\\
        Ta có $\lim\limits_{v\to c^{-}} m(v)=\lim\limits_{v\to c^{-}}\dfrac{m_0}{\sqrt{1-\dfrac{v^2}{c^2}}}=+\infty$ nên đường thẳng $v=c$ là tiệm cận đứng của đồ thị hàm số.\\
        Từ đó ta suy ra khi vận tốc của vật càng sát với vận tốc ánh sáng thì khối lượng của vật tăng lên vô hạn.
    }
\end{ex}
\Closesolutionfile{ans}
%\boxde
\BTTN
\Opensolutionfile{ans}[ans/2D1-4-DEON-2]
\begin{ex}%[2D1B4-1]
    Cho hàm số $y=f(x)$ có bảng biến thiên như hình bên. Tổng số tiệm cận đứng và tiệm cận ngang của đồ thị hàm số đã cho là
    \begin{center}
        \begin{tikzpicture}
            \tkzTabInit[nocadre=false,lgt=1.5,espcl=3,deltacl=0.6]
            {$x$ /0.6,$y’$ /0.6,$y$ /2}
            {$-\infty$ ,$0$, $1$, $+\infty$}
            \tkzTabLine{,-,d,+,0,-,}
            \tkzTabVar{+/$+\infty$,-D-/$-1$/$-\infty$,+/$2$,-/$-\infty$}
        \end{tikzpicture}
    \end{center}
    \choice
    {\True $1$}
    {$3$}
    {$2$}
    {$4$}
    \loigiai{
        Dựa vào bảng biến thiên ta thấy đồ thị hàm số có một tiệm cận đứng $x=0$.
    }
\end{ex}
%62
%63
\begin{ex}%[2D1B4-1]
    Cho hàm số $y=f(x)$ xác định, liên tục trên $\mathbb{R} \backslash \{0;1\}$ và có bảng biến thiên như hình bên. Đồ thị hàm số $y=f(x)$ có
    \begin{center}
        \begin{tikzpicture}
            \tkzTabInit[nocadre=false,lgt=1.5,espcl=3,deltacl=0.6]
            {$x$ /0.6,$y’$ /0.6,$y$ /2}
            {$-\infty$ ,$0$, $1$, $+\infty$}
            \tkzTabLine{,+,d,+,d,+,}
            \tkzTabVar{-/$-5$,+D-/$+\infty$/$-\infty$,+D-/$3$/$-\infty$,+/$+\infty$}
        \end{tikzpicture}
    \end{center}
    \choice
    {\True $2$ tiệm cận đứng và $1$ tiệm cận ngang}
    {$2$ tiệm cận đứng và $2$ tiệm cận ngang}
    {$1$ tiệm cận đứng và $1$ tiệm cận ngang}
    {$1$ tiệm cận đứng và $2$ tiệm cận ngang}
    \loigiai{
        Dựa vào bảng biến thiên ta thấy đồ thị hàm số có hai tiệm cận đứng $x=0$ và $x=1$; một tiệm cận ngang $y=-5$.
    }
\end{ex}
%64
\begin{ex}%[2D1B4-1]
    Cho hàm số $y=f(x)$ có bảng biến thiên như hình bên. Tổng số tiệm cận đứng và tiệm cận ngang của đồ thị hàm số đã cho là
    \begin{center}
        \begin{tikzpicture}[scale=0.8]
            \tkzTabInit[nocadre=false,lgt=1.5,espcl=3,deltacl=0.6]
            {$x$ /0.6,$y’$ /0.6,$y$ /2}
            {$-\infty$ ,$-1$, $1$, $+\infty$}
            \tkzTabLine{,+,d,+,0,-,}
            \tkzTabVar{-/$2$,+D-/$4$/$-\infty$,+/$3$,-/$-1$}
        \end{tikzpicture}
    \end{center}
    \choice
    {$1$}
    {\True $3$}
    {$2$}
    {$4$}
    \loigiai{Dựa vào bảng biến thiên ta thấy đồ thị hàm số có một tiệm cận đứng $x=-1$; hai tiệm cận ngang $y=-1$ và $y=2$.
    }
\end{ex}
%65
\begin{ex}%[2D1B4-1]
    Cho hàm số $y=f(x)$ có bảng biến thiên như hình bên. Tổng số tiệm cận đứng và tiệm cận ngang của đồ thị hàm số đã cho là
    \begin{center}
        \begin{tikzpicture}[scale=0.8]
            \tkzTabInit[nocadre=false,lgt=1.5,espcl=3,deltacl=0.6]
            {$x$ /0.6,$y’$ /0.6,$y$ /2}
            {$-\infty$ ,$0$, $3$, $+\infty$}
            \tkzTabLine{,-,0,+,d,-,}
            \tkzTabVar{+/$8$,-/$1$,+/$4$,-/$2$}
        \end{tikzpicture}
    \end{center}
    \choice
    {$1$}
    {$3$}
    {\True $2$}
    {$4$}
    \loigiai{
        Dựa vào bảng biến thiên ta thấy đồ thị hàm số có hai tiệm cận ngang $y=2$ và $y=8$.
    }
\end{ex}
\begin{ex}%[Nguyễn Văn Sang, dự án Tex hoá đề cương trường Marie Curie - Lần 6]%[2D1Y4-1]
    Đường thẳng nào dưới đây là tiệm cận đứng của đồ thị hàm số $y=\dfrac{2 x+1}{x+1}$?
    \choice
    {$x=1$}
    {$y=-1$}
    {$y=2$}
    {\True $x=-1$}
    \loigiai{
        Tập xác định $\mathscr{D}=\mathbb{R}\setminus\left\lbrace -1\right\rbrace$.
        \begin{itemize}
            \item $\lim\limits_{x \to \pm\infty} y=\lim\limits_{x \to \pm\infty} \dfrac{2x+1}{x+1}=2$ suy ra $y=2$ là tiệm cận ngang.
            \item $\heva{& \lim\limits_{x \to -1^+} \dfrac{2x+1}{x+1}=-\infty \\ & \lim\limits_{x \to -1^-} \dfrac{2x+1}{x+1}=+\infty}$ suy ra $x=-1$ là tiệm cận đứng.
        \end{itemize}
    }
\end{ex}
\begin{ex}%[2D1Y4-1]
    Đồ thị hàm số $y=\dfrac{2x-3}{2x+1}$ có tâm đối xứng là điểm
    \choice
    {\True $M\left(-\dfrac{1}{2};1\right)$}
    {$P\left(-\dfrac{1}{2};2\right)$}
    {$Q\left(-\dfrac{1}{2};-3\right)$}
    {$N\left(1;-\dfrac{1}{2}\right)$}
    \loigiai{
        Tiệm cận đứng, tiệm cận ngang của đồ thị hàm số lần lượt là $x=-\dfrac{1}{2}$ và $y=3$. Tâm đối xứng là điểm $M\left(-\dfrac{1}{2};1\right)$.
    }
\end{ex}
\begin{ex}%[2D1K4-1]
    Đồ thị hàm số $y=\dfrac{\sqrt{x}}{x+1}-\dfrac{1}{x}$ có tất cả bao nhiêu tiệm cận đứng và ngang?
    \choice
    {$0$}
    {$3$}
    {\True $2$}
    {$1$}
    \loigiai{
        Tập xác định $\mathscr{D}=(0;+\infty)$.
        \begin{itemize}
            \item $\lim\limits_{x\to 0^+} \left(\dfrac{\sqrt{x}}{x+1}-\dfrac{1}{x}\right)=-\infty$.
            \item $\lim\limits_{x\to +\infty}\left(\dfrac{\sqrt{x}}{x+1}-\dfrac{1}{x}\right)=0$.
        \end{itemize}
        Suy ra đồ thị hàm số có tiệm cận đứng $x=0$, tiệm cận ngang $y=0$.
    }
\end{ex}
\begin{ex}%[2D1K4-1]
    Số tiệm cận đứng của đồ thị hàm số $y=\dfrac{x^2-3x-4}{x^2-16}$ là
    \choice
    {$2$}
    {$3$}
    {\True $1$}
    {$0$}
    \loigiai{
        Điều kiện xác định $x \ne \pm 4$.\\
        Với điều kiện xác định trên, ta có $y=\dfrac{x^2-3x-4}{x^2-16}=\dfrac{(x+1)(x-4)}{(x-4)(x+4)}=\dfrac{x+1}{x+4}$.\\
        Tiệm cận đứng của đồ thị hàm số là $x=-4$.
    }
\end{ex}
\begin{ex}%[2D1K4-1]
    Số đường tiệm cận đứng và ngang của đồ thị hàm số $y=\dfrac{x-1}{x^2-x-2}$ là
    \choice
    {\True $3$}
    {$1$}
    {$0$}
    {$2$}
    \loigiai{
        Điều kiện xác định $x \ne -1$, $x \ne 2$.\\
        Với điều kiện xác định trên, ta có $y=\dfrac{x-1}{x^2-x-2}=\dfrac{x-1}{(x+1)(x-2)}$.\\
        Tiệm cận đứng của đồ thị hàm số là $x=-1$, $x=2$, tiệm cận ngang của đồ thị hàm số là $y=0$.
    }
\end{ex}
%81
\begin{ex}%[2D1K4-1]
    Cho hàm số $y=f(x)$ có bảng biến thiên như hình bên. Đồ thị hàm số $y=\dfrac{1}{f^2(x)-2f(x)}$ có bao nhiêu tiệm cận đứng?
    \begin{center}
        \begin{tikzpicture}[scale=0.8]
            \tkzTabInit[nocadre=false,lgt=1.5,espcl=3,deltacl=0.6]
            {$x$ /0.6,$y’$ /0.6,$y$ /2}
            {$-\infty$ ,$-1$, $2$, $+\infty$}
            \tkzTabLine{,+,d,-,0,+,}
            \tkzTabVar{-/$-\infty$,+/$1$,-/$-2$,+/$+\infty$}
        \end{tikzpicture}
    \end{center}
    \choice
    {\True $4$}
    {$3$}
    {$2$}
    {$6$}
    \loigiai{
        Dựa vào bảng biến thiên suy ra $f^2(x)-2f(x)=0 \Leftrightarrow \heva{&f(x)=0\\&f(x)=2}$, phương trình $f(x)=0$ có $3$ nghiệm phân biệt và phương trình $f(x)=2$ có $1$ nghiệm nên đồ thị hàm số đã cho có $4$ tiệm cận đứng.
    }
\end{ex}
%79
\begin{ex}%[2D1K4-1]
    Cho hàm số $y=f(x)$ có bảng biến thiên như hình bên. Tổng số tiệm cận ngang và tiệm cận đứng của đồ thị hàm số $y=\dfrac{2}{f(x)+3}$ là
    \begin{center}
        \begin{tikzpicture}[scale=0.8]
            \tkzTabInit[nocadre=false,lgt=1.5,espcl=3,deltacl=0.6]
            {$x$ /0.6,$y’$ /0.6,$y$ /2}
            {$-\infty$ ,$-4$, $6$, $+\infty$}
            \tkzTabLine{,-,0,+,0,-,}
            \tkzTabVar{+/$+\infty$,-/$-2$,+/$5$,-/$-\infty$}
        \end{tikzpicture}
    \end{center}
    \choice
    {$4$}
    {$3$}
    {\True $2$}
    {$1$}
    \loigiai{
        Dựa vào bảng biến thiên suy ra
        \begin{itemize}
            \item 	$\lim \limits_{x \to \pm \infty} f(x)=\pm \infty \Leftrightarrow \lim \limits_{x \to \pm \infty}\dfrac{2}{f(x)+3} =0$ nên đồ thị hàm số đã cho có tiệm cận ngang là $y=0$.
            \item $f(x)+3=0 \Leftrightarrow f(x) =-3$, phương trình này có $1$ nghiệm $x=a>6$ nên đồ thị hàm số đã cho có một tiệm cận đứng.
        \end{itemize}
    }
\end{ex}
\begin{ex}%[2D1K4-1]
    Cho hàm số $y=f(x)$ có bảng biến thiên như hình bên. Đồ thị hàm số $y=\dfrac{x+1}{f(x)-4}$ có bao nhiêu tiệm cận đứng?
    \begin{center}
        \begin{tikzpicture}[scale=0.8]
            \tkzTabInit[nocadre=false,lgt=1.5,espcl=3,deltacl=0.6]
            {$x$ /0.6,$y’$ /0.6,$y$ /2}
            {$-\infty$ ,$-1$, $2$, $+\infty$}
            \tkzTabLine{,+,0,-,0,+,}
            \tkzTabVar{-/$1$,+/$4$,-/$-5$,+/$+\infty$}
        \end{tikzpicture}
    \end{center}
    \choice
    {$1$}
    {$3$}
    {\True $2$}
    {$4$}
    \loigiai{
        Dựa vào bảng biến thiên suy ra
        $f(x)-4=0 \Leftrightarrow f(x) =4$, phương trình này có $1$ nghiệm khác $-1$ và một nghiệm bội chẵn $x=-1$ nên đồ thị hàm số $y=\dfrac{x+1}{f(x)-4}$ có hai tiệm cận đứng.
    }
\end{ex}
\begin{ex}%[2D1K4-1]
    Cho hàm số $y=f(x)$ có bảng biến thiên như hình bên. Đồ thị hàm số $y=\dfrac{x-5}{f(x)-1}$ có bao nhiêu tiệm cận đứng?
    \begin{center}
        \begin{tikzpicture}
            \tkzTabInit[espcl=3]{$x$ / 1 , $f’(x)$ / 1, $f(x)$ / 2}
            {$-\infty$, $-1$ , $2$, $+\infty$}%
            \tkzTabLine{,-,0,+,d,-,}%
            \tkzTabVar{+/ $+\infty$, - / $-1$, + / $3$,-/$-\infty$}%
            \tkzTabVal[draw]{3}{4}{0.4}{$5$}{$1$}%
            %\tkzTabVal[draw]{2}{3}{0.4}{$e^2$}{$1$}%
        \end{tikzpicture}
    \end{center}
    \choice
    {$1$}
    {$3$}
    {\True $2$}
    {$4$}
    \loigiai{
        Dựa vào bảng biến thiên suy ra
        $f(x)-1=0 \Leftrightarrow f(x) =1$, phương trình này có $2$ nghiệm phân biệt khác $5$ và một nghiệm $x=5$ nên đồ thị hàm số $y=\dfrac{x-5}{f(x)-1}$ có hai tiệm cận đứng.
    }
\end{ex}
\begin{ex}%[VDC5-NgocDungHo]%[2D1G4-3]%
    \immini
    {
        Cho hàm số $f(x)$ có đồ thị như hình bên. Số đường tiệm cận đứng của đồ thị hàm số $y=\dfrac{(x^2-1)(x^2+x)}{[f(x)]^2-2f(x)-3}$ là
        \choice
        {$4$}
        {$5$}
        {\True $3$}
        {$2$}
    }
    {
        \begin{tikzpicture}[>=stealth,scale=0.7, line join=round, line cap=round]
            \def\f[#1]{(#1)^3-3*(#1)+1)}
            \draw[->] (-2.2,0)--(2.4,0) node [below]{$x$};
            \draw[->] (0,-1.5)--(0,3.5) node [left]{$y$};
            \node at (0,0) [below left]{$O$};
            % \clip;
            \draw[domain=-2.1:2.1,samples=300,thick] plot (\x,{\f[\x]});
            \filldraw (-1,0) node[below]{$-1$} circle (2pt);
            \filldraw (1,0) node[above]{$1$} circle (2pt);
            \filldraw (0,-1) node[ left]{$1$} circle (2pt);
            \filldraw (0,3) node[ right]{$3$} circle (2pt);
            \draw[dashed](-1,0)--(-1,3)--(0,3) (1,0)--(1,-1)--(0,-1);
            \draw (2,3) node[right]{$y=f(x)$};
        \end{tikzpicture}
    }
    \loigiai{
        Xét hàm số $y=g(x)=\dfrac{(x^2-1)(x^2+x)}{[f(x)]^2-2f(x)-3}$.
        \immini
        {
            Giải phương trình $(x^2-1)(x^2+x)=0 \Leftrightarrow \hoac{& x^2-1=0 \\ & x^2+x=0}\Leftrightarrow \hoac{& x=\pm 1 \\ & x=0.}$\\
            Giải phương trình $[f(x)]^2-2f(x)-3=0$\\$ \Leftrightarrow \hoac{& f(x)=-1 \\ & f(x)=3} \Leftrightarrow \hoac{& x = \pm 1 \\ & x=a\\&x=b\;(a<-1<1<b).}$
        }
        {
            \begin{tikzpicture}[>=stealth,scale=0.7, line join=round, line cap=round]
                \def\f[#1]{(#1)^3-3*(#1)+1)}
                \def\g[#1]{3}
                \def\h[#1]{-1}
                \draw[->] (-2.5,0)--(4,0) node [below]{$x$};
                \draw[->] (0,-1.5)--(0,3.5) node [left]{$y$};
                \node at (0,0) [below left]{$O$};
                % \clip;
                \draw[domain=-2.5:4,samples=300,thick] plot (\x,{\g[\x]});
                \draw[domain=-2.5:4,samples=300,thick] plot (\x,{\h[\x]});
                \draw[domain=-2.1:2.1,samples=300,thick] plot (\x,{\f[\x]});
                \filldraw (-1,0) node[below]{$-1$} circle (2pt);
                \filldraw (1,0) node[above]{$1$} circle (2pt);
                \filldraw (0,-1) node[ left]{$1$} circle (2pt);
                \filldraw (0,3) node[ right]{$3$} circle (2pt);
                \draw[dashed](-1,0)--(-1,3)--(0,3) (1,0)--(1,-1) (2,0)node[below]{$b$}--(2,3) (-2,0)node[above]{$a$}--(-2,-1);
                \draw (2,2) node[right]{$y=f(x)$};
                \draw (3.3,3) node[above]{$d_1:y=3$};
                \draw (3,-1) node[below]{$d_2:y=-1$};
            \end{tikzpicture}
        }
        Trong điều kiện xác định của hàm số $y=g(x)$ ta có thể viết
        $$y=g(x)=\dfrac{x(x-1)(x+1)^2}{(x-a)(x-b)(x-1)^2(x+1)^2}=\dfrac{x}{(x-a)(x-b)(x-1)}$$
        Vậy số tiệm cận đứng của đồ thị hàm số $y=g(x)$ bằng $3$.
    }
\end{ex}
\begin{ex}
    \immini{ %Câu 91.
        Đường cong ở hình bên là đồ thị của hàm số $y = ax^4 + bx^2 +c$. Đồ thị hàm số $g(x) =\dfrac{(x^2-x)\sqrt{x+2}}{(x-2)\cdot f(x+1)}$
        có bao nhiêu đường tiệm cận đứng?
        \choice
        {1}
        {3}
        {4}
        {2}}{
        \begin{tikzpicture}[scale=.8, font=\footnotesize, line join=round, line cap=round, >=stealth]
            \def\xmin{-2}\def\xmax{2}\def\ymin{-3}\def\ymax{1}
            \draw[->] (\xmin-0.2,0)--(\xmax+0.2,0) node[below] {\footnotesize $x$};
            \draw[->] (0,\ymin-0.2)--(0,\ymax+0.2) node[right] {\footnotesize $y$};
            \draw (0,0) node [below left] {\footnotesize $O$};
            \foreach \x in {-1,1}\draw (\x,-0.1)--(\x,0.1) node [above left] {\footnotesize $\x$};
            \foreach \y in {-2}\draw (0.1,\y)--(-0.1,\y) node [ below left] {\footnotesize $\y$};
            \clip (\xmin,\ymin) rectangle (\xmax,\ymax);
            \draw[smooth,samples=200,domain=\xmin:\xmax] plot (\x,{((\x)^4)+((\x)^2)+-2});
        \end{tikzpicture}
    }
    \loigiai{
        * Điều kiện: $\heva{&x \ne 2\\&f(x+1) \ne 0\\&x \ge -2.}$\\
        Nhìn hình vẽ ta thấy
        $f(x+1)=0\Leftrightarrow \hoac{&x+1=-1\\&x+1=1}\Leftrightarrow \hoac{&x=-2&(\text{nghiệm đơn})\\&x=0&(\text{nghiệm đơn}).}$\\
        Vậy $g(x) = \dfrac{(x^2-x)\sqrt{x+2}}{(x-2)\cdot ax^2(x^2+2) }=\dfrac{(x-1)\sqrt{x+2}}{(x-2)\cdot ax(x^2+2)}.$ \\
        Đồ thị hàm số $g(x)$ có 2 đường tiệm cận đứng.}
\end{ex}
\begin{ex}%[Thi thử THPT Yên Phong 1 - Bắc Ninh, 2021]%[Duong Xuan Loi,12EX 6- 2021]%[2D1G4-3]%
    \immini{
        Cho hàm số $y=f(x)$ có đồ thị như hình vẽ. Biết $f'(x)<0$, $\forall x <-1$ và $f'(x)>0$, $\forall x>1$. Khi đó, tổng số tiệm cận của đồ thị hàm số $y=\dfrac{2024}{\sqrt{xf(x+1)}[xf(x+1)+1]-2}$ là
        \choice
        {$1$}
        {$3$}
        {$4$}
        {\True $2$}
    }{
        \begin{tikzpicture}[scale=0.7, font=\footnotesize, line join=round, line cap=round,>=stealth]
            \def\xmin{-2} \def\xmax{2}
            \def\ymin{-2} \def\ymax{3.3}
            \draw[color=gray!50,dashed] (\xmin,\ymin) grid (\xmax,\ymax);
            \draw[->] (\xmin,0)--(\xmax,0) node [below]{$x$};
            \draw[->] (0,\ymin)--(0,\ymax) node [left]{$y$};
            \node at (0,0) [above right]{$O$};
            \clip (\xmin+0.1,\ymin+0.1) rectangle (\xmax-0.1,\ymax-0.1);
            \draw[smooth,samples=300,domain=-1.8:1.4] plot(\x,{(\x+1)*(\x+1)*(\x)*(\x-1)});
            \fill (-1,0) circle (1.0pt) node[below]{$-1$} (1,0) circle (1.0pt) node[below right]{$1$};
        \end{tikzpicture}
    }
    \loigiai{
        Xét phương trình $\sqrt{xf(x+1)}[xf(x+1)+1]-2=0.\quad(1)$\\
        Đặt $t=\sqrt{xf(x+1)}(t\geq 0)$, ta được phương trình $t\left(t^2+1\right)=2\Leftrightarrow t^3+t-2=0\Leftrightarrow t=1$.\\
        Với $t=1\Rightarrow\sqrt{xf(x+1)}=1\Leftrightarrow xf(x+1)-1=0$.\\
        Đặt $u=x+1\Rightarrow x=u-1$, ta được phương trình $(u-1)f(u)-1=0\Leftrightarrow f(u)=\dfrac{1}{u-1}.\quad(2)$
        \begin{center}
            \begin{tikzpicture}[scale=1, font=\footnotesize, line join=round, line cap=round,>=stealth]
                \def\a{0} \def\b{1} \def\c{1} \def\d{-1} % Hệ số
                \def\xmin{-3} \def\xmax{3.5}
                \def\ymin{-2.8} \def\ymax{3.3}
                \draw[color=gray!50,dashed] (\xmin,\ymin) grid (\xmax,\ymax);
                \draw[->] (\xmin,0)--(\xmax,0) node [below]{$u$};
                \draw[->] (0,\ymin)--(0,\ymax) node [left]{$y$};
                \node at (0,0) [above right]{$O$};
                \clip (\xmin+0.1,\ymin+0.1) rectangle (\xmax-0.1,\ymax-0.1);
                \draw[smooth,samples=300,domain=-1.8:1.4] plot(\x,{(\x+1)*(\x+1)*(\x)*(\x-1)});
                \draw[smooth,samples=300,domain=\xmin:(-\d/\c-0.1)] plot(\x,{(\a*(\x)+\b)/(\c*(\x)+\d)});
                \draw[smooth,samples=300,domain=(-\d/\c+0.1:\xmax)] plot(\x,{(\a*(\x)+\b)/(\c*(\x)+\d)});
                \fill (-1,0) circle (1.0pt) node[below]{$-1$} (1,0) circle (1.0pt) node[below right]{$1$};
            \end{tikzpicture}
        \end{center}
        Nhận thấy đồ thị của các hàm số $y=f(u)$, $y=\dfrac{1}{u-1}$ chỉ cắt nhau tại $1$ điểm do đó phương trình $(2)$ có nghiệm duy nhất $\Rightarrow(1)$ có nghiệm duy nhất, suy ra đồ thị có $1$ tiệm cận đứng.\\
        Mặt khác: $\lim\limits_{x\to+\infty} f(x+1)=+\infty\Rightarrow\lim\limits_{x\to+\infty}\dfrac{2021}{\sqrt{xf(x+1)}[xf(x+1)+1]-2}=a>0$.\\
        $\lim\limits_{x\to-\infty} xf(x+1)=-\infty\Rightarrow\lim\limits_{x\to+\infty}\dfrac{2021}{\sqrt{xf(x+1)}[xf(x+1)+1]-2}$ không tồn tại.\\
        Do đó đường thẳng $y=a$ là tiệm cận ngang.}
\end{ex}
\BTTF
\begin{ex}%[EX-TF-2024, Lê Đạt]%[2D1N4-1]
    Cho hàm số $y=f(x)$ có bảng biến thiên như sau
    \begin{center}
        \begin{tikzpicture}[>=stealth]
            \tkzTabInit[nocadre=false,lgt=1,espcl=3,deltacl=0.6]
            {$x$/.7 ,$y'$/.7,$y$/2}
            {$-\infty$ , $-2$ , $0$, $+\infty$}
            \tkzTabLine{ , - , d , + , d , -, }
            \tkzTabVar{+/$+\infty$ , -D-/$1$/$-\infty$ , +D+/$+\infty$ /$1$, -/$0$}
        \end{tikzpicture}
    \end{center}
    Xét tính đúng sai của các khẳng định sau
    \choiceTF
    {\True $ x=0 $ là tiệm cận đứng của đồ thị hàm số $ y=f(x) $}
    {\True $ x=-2 $ là tiệm cận đứng của đồ thị hàm số $ y=f(x) $}
    {$ x=1 $ là tiệm cận đứng của đồ thị hàm số $ y=f(x) $}
    {\True $ y=0 $ là tiệm cận ngang của đồ thị hàm số $ y=f(x) $}
    \loigiai{
        \begin{itemchoice}
            \itemch $\lim \limits_{x \to 0^-} f(x)=+\infty\Rightarrow x=0$ là đường tiệm cận đứng của đồ thị hàm số $f(x)$.
            \itemch $\lim \limits_{x \to (-2)^+} f(x)=-\infty\Rightarrow x=-2$ là đường tiệm cận đứng của đồ thị hàm số $f(x)$.
            \itemch Đồ thị hàm số chỉ có hai tiệm cận đứng là $ x=0 $ và $ x=-2 $.
            \itemch $\lim \limits_{x \to +\infty} f(x)=0\Rightarrow y=0$ là đường tiệm cận ngang của đồ thị hàm số $f(x)$.
        \end{itemchoice}
    }
\end{ex}
\begin{ex}%[EX-TF-2024, Lê Đạt]%[2D1H4-2]
    Cho hàm số $y=\dfrac{m x^{2}+6 x-2}{x+2}$. Xét tính đúng sai của các khẳng định sau
    \choiceTF
    {Đồ thị hàm số luôn có tiệm cận đứng với mọi $ m $}
    {Đồ thị hàm số không có tiệm cận ngang với mọi $ m $}
    {\True Khi $ m=1 $ đồ thị hàm số có một tiệm cận xiên là $ y=x+4 $ }
    {Đồ thị hàm số luôn có tiệm cận xiên}
    \loigiai{
        \begin{itemchoice}
            \itemch Khi $ m=\dfrac{7}{2} $ hàm số trở thành $y=\dfrac{\dfrac{7}{2} x^{2}+6 x-2}{x+2}=\dfrac{7}{2}\left(x-\dfrac{2}{7} \right) $ suy ra đồ thị hàm số không có tiệm cận đứng.
            \itemch Khi $ m=0 $ hàm số trở thành $ y=\dfrac{6x-2}{x+2} $ từ đó suy ra đồ thị hàm số có $ y=6 $ là tiệm cận ngang.
            \itemch Khi $ m=1 $ hàm số trở thành $ y=\dfrac{x^2+6x-2}{x+2}=x+4-\dfrac{10}{x+2} $ từ đó suy ra $ y=x+4 $ là một tiệm cận ngang.
            \itemch Khi $ m=0 $ hàm số trở thành $ y=\dfrac{6x-2}{x+2} $ từ đó suy ra đồ thị hàm số có $ y=6 $ là tiệm cận ngang, $ x=-2 $ là tiệm cận đứng và không có tiệm cận xiên.
        \end{itemchoice}
    }
\end{ex}
\begin{ex}%[EX-TF-2024, Lê Đạt]%[2D1H4-3]
    \immini{Cho hàm số $y=f(x)$ có đồ thị như hình bên. Xét tính đúng sai của các khẳng định sau
        \choiceTF
        {\True $ x=0 $ là một đường tiệm cận đứng của đồ thị hàm số}
        {$ y=-x $ là một đường tiệm cận xiên của đồ thị hàm số}
        {\True $ y=x $ là một đường tiệm cận xiên của đồ thị hàm số}
        {Đồ thị hàm số có ba đường tiệm cận}
    }{
        \begin{tikzpicture}[scale=.9, font=\footnotesize, line join=round, line cap=round,>=stealth]
            \def\a{0} \def\b{1} \def\c{1} \def\d{-1} % Hệ số
            \def\xmin{-3} \def\xmax{3.5}
            \def\ymin{-2.8} \def\ymax{3.3}
            \draw[color=gray!50,dashed] (\xmin,\ymin) grid (\xmax,\ymax);
            \draw[->] (\xmin,0)--(\xmax,0) node [below]{$x$};
            \draw[->] (0,\ymin)--(0,\ymax) node [left]{$y$};
            \fill (0,0) circle(1pt) node[shift=(-45:0.25)]{$O$};
            \clip (\xmin+0.1,\ymin+0.1) rectangle (\xmax-0.1,\ymax-0.1);
            \draw[smooth,samples=300,domain=-3:3] plot(\x,{\x+1/(7*\x)});
            \draw[dashed,smooth,samples=300,domain=-3:3] plot(\x,{\x});
            %	\fill (-1,0) circle (1.0pt) node[below]{$-1$} (1,0) circle (1.0pt) node[below right]{$1$};
    \end{tikzpicture}}
    \loigiai{
        \begin{itemchoice}
            \itemch $ x=0 $ là một đường tiệm cận đứng của đồ thị hàm số.
            \itemch	$ y=x $ là một đường tiệm cận xiên của đồ thị hàm số.
            \itemch $ y=x $ là một đường tiệm cận xiên của đồ thị hàm số.
            \itemch Đồ thị hàm số có $ x=0 $ là tiệm cận đứng và $ y=x $ là tiệm cận xiên nên có hai tiệm cận.
        \end{itemchoice}
    }
\end{ex}
\begin{ex}
    \immini
    {
        Cho hàm số $y=f(x)$ có đạo hàm liên tục trên $R$. Hàm số $y=f^{\prime}(x)$ có đồ thị như hình bên. Xác định tính đúng, sai của các mệnh đề sau
        \choiceTF
        {Hàm số $y=f(x)$ có hai cực trị}
        {Hàm số $y=f(x)$ đồng biến trên khoảng $(1 ;+\infty)$}
        {\True $f(1)>f(2)>f(4)$.}
        {\True Trên đoạn $[-1 ; 4]$, giá trị lớn nhất của hàm số $y=f(x)$ là $f(1)$.}
    }
    {
        \begin{tikzpicture}[line join=round, line cap=round,>=stealth,font=\scriptsize]
            \begin{scope}[scale=0.5]
                \tikzset{label style/.style={font=\footnotesize}}
                \def \xmin{-2}
                \def \xmax{4.5}
                \def \ymin{-2}
                \def \ymax{3.5}
                \def \hamso{0.55*(\x)^3-1.76*(\x)^2-0.31*(\x)+2}
                \draw[->] (\xmin,0)--(\xmax,0) node[below left] {$x$};
                \draw[->] (0,\ymin)--(0,\ymax) node[below left] {$y$};
                \draw (0,0) node [below left] {$O$};
                \begin{scope}
                    \clip (\xmin+0.01,\ymin+0.01) rectangle (\xmax-0.01,\ymax-0.01);
                    \draw[samples=350,domain=-1.3:3.3,smooth,variable=\x] plot (\x,{\hamso});
                \end{scope}
                \draw (-1,0) node[below left]{$-1$} (1,0) node[below]{$1$} (3,0) node[below right]{$4$} (0,2) node[above left]{$2$};
            \end{scope}
        \end{tikzpicture}
    }
    \loigiai{}
\end{ex}
\begin{ex}
    \immini{Cho hàm số $y=f(x)$ liên tục trên đoạn $\left[0 ; \frac{7}{2}\right]$ có đồ thị hàm số $y=f^{\prime}(x)$ như hình vẽ.
        \choiceTF
        {\True Hàm số $y=f(x)$ đồng biến trên khoảng $\left(3 ; \frac{7}{2}\right)$}
        {\True $f(0)>f(3)$}
        {$f(3)>f\left(\frac{7}{2}\right)$}
        {Hàm số $y=f(x)$ đạt giá trị nhỏ nhất trên đoạn $\left[0 ; \frac{7}{2}\right]$ tại điểm $x_0=\frac{7}{2}$}
    }{\begin{tikzpicture}[>=stealth, samples=100,smooth,y=.7cm,font=\scriptsize]
            \begin{scope}[scale=.7]
                \draw[->] (-1,0)--(4.5,0) node[below] {$x$};
                \draw[->] (0,-2)--(0,4) node[right] {$y$};
                \draw (0,0) node [below left] {$O$};
                \draw[dashed] (3.6,0)--(3.6,4);
                \draw[samples=200,domain=0.2:3.6,smooth,variable=\x]
                plot (\x,{1.06*(\x)^3-5.3*(\x)^2+7.23*(\x)-3});
                \path
                (3.6,0)node[below]{$\dfrac{7}{2}$}
                (3,0)node[above left]{$3$}
                (1,0)node[above]{$1$}
                ;
            \end{scope}
    \end{tikzpicture}}
    \loigiai{}
\end{ex}
\BTTL
\begin{ex}%[2D1K4-2]%
    Đồ thị hàm số $y=\dfrac{(2m+1)x+3}{x+1}$ có đường đường tiệm cận đi qua điểm $A(-2;7)$ khi và chỉ khi
    \shortans{$m=3$}
    %	\choice
    %	{\True $m=3$}
    %	{$m=1$}
    %	{$m=-1$}
    %	{$m=-3$}
    \loigiai{
        Từ đề bài, suy ra $2m+1=7 \Leftrightarrow m=3$.\\
        Suy ra $m+n=0$.
    }
\end{ex}
\begin{ex}%[Học kì 1, THPT Nguyễn Thi Minh Khai - Hà Nội, 2020-2021]%[Bùi Mạnh Tiến, 12EX5]%[2D1K4-2]%
    Cho hàm số $ y=\dfrac{2mx+m}{x-1}$. Với giá trị nào của tham số $m$ thì đường tiệm cận đứng, tiệm cận ngang của đồ thị hàm số cùng hai trục tọa độ tạo thành một hình chữ nhật có diện tích bằng $8$?
    \shortans{$m=\pm 4$}
    %	\choice
    %	{$ m=2$}
    %	{ $m=\pm 2$}
    %	{\True $m=\pm 4$}
    %	{$ m=\pm\dfrac{1}{2}$}
    \loigiai{
        Hàm số $y=\dfrac{2mx+m}{x-1}$ có $a=2m$, $b=m$, $c=1$, $d=-1$.\\
        Tiệm cận ngang $y=\dfrac{a}{c}=2m$.\\
        Tiệm cận đứng $x=-\dfrac{d}{c}=1$.\\
        Diện tích hình chữ nhật tạo thành bởi hai đường tiệm cận và hai trục tọa độ có diện tích
        \begin{align*}
            |2m|\cdot 1=8\Leftrightarrow m=\pm 4.
        \end{align*}
    }
\end{ex}
\begin{ex}%[KSCL lần 1, Liễn Sơn - Vĩnh Phúc, 2021]%[Phạm Doãn Lê Bình, 12EX4-2021]%[2D1K4-2]%
    Cho hàm số $y=\dfrac{x-\sqrt{x^2+2x}}{x^2+mx-m-3}$ có đồ thị $(C)$. Giá trị của $m$ để $(C)$ có đúng hai tiệm cận thuộc tập nào sau đây?
    \shortans{$(-5;2)$}
    %	\choice
    %	{$(-2;1)$}
    %	{$(1;5)$}
    %	{$(5;8)$}
    %	{\True $(-5;2)$}
    \loigiai{
        Điều kiện xác định của hàm số đã cho $\heva{& \hoac{ & x\ge 0 \\ & x\le -2}\\ & x^2+mx-m-3\ne 0.}$\\
        Ta có $\lim \limits_{x\to +\infty} y = \lim \limits_{x\to -\infty} y = 0$ nên $(C)$ có một tiệm cận ngang $y=0$.\\
        Xét phương trình $x^2+mx-m-3=0$.\hfill $(1)$\\
        Ta có
        \begin{itemize}
            \item $\Delta = m^2+4m+12>0$, $\forall m \in \mathbb{R}$.\\ Vậy phương trình $(1)$ luôn có hai nghiệm phân biệt $x_1,x_2$ ($x_1<x_2$).
            \item $x-\sqrt{x^2+2x}=0 \Leftrightarrow \heva{& x\ge 0 \\ & x^2=x^2+2x} \Leftrightarrow x=0$.
            \item Phương trình $(1)$ có nghiệm $x=0 \Leftrightarrow m=-3$. Với $m=-3$ ta có
            $ y =\dfrac{x-\sqrt{x^2+2x}}{x^2-3x}.$
            Khi đó
            \begin{eqnarray*}
                & \lim \limits_{x\to 0^+} y & =\lim \limits_{x\to 0^+} \dfrac{x-\sqrt{x^2+2x}}{x^2-3x}\\
                & & =\lim \limits_{x\to 0^+} \dfrac{-2x}{(x^2-3x)\left( x+\sqrt{x^2+2x}\right)}\\
                & & = \lim \limits_{x\to 0^+} \dfrac{-2}{(x-3)\left( x+\sqrt{x^2+2x}\right)}=+\infty
            \end{eqnarray*}
            và $\lim \limits_{x\to 3^+} y =-\infty$
            nên $(C)$ có thêm hai tiệm cận đứng $x=0$ và $x=3$ (không thỏa yêu cầu bài toán).
            \item Với $m\ne -3$ thì $(C)$ có đúng hai tiệm cận khi và chỉ khi $\hoac{& x_1<-2<x_2<0 &(2)\\ & -2<x_1<0<x_2. & (3)}$
            \item Đặt $f(x)=x^2+mx-m-3$. Ta có
            $(2)\Leftrightarrow \heva{& f(-2)< 0 \\ & f(0) >0 \\ & 0>-m} \Leftrightarrow \heva{& m>\dfrac{1}{3}\\ & m<-3 \\ & m>0}\Leftrightarrow m \in \varnothing.$
            \item $(3)\Leftrightarrow \heva{& f(-2)> 0 \\ & f(0) <0 \\ & -2<-m} \Leftrightarrow \heva{& m<\dfrac{1}{3}\\ & m>-3 \\ & m<2}\Leftrightarrow -3<m<\dfrac{1}{3}.$
        \end{itemize}
        Vậy $m\in \left(-3;\dfrac{1}{3}\right)$.
    }
\end{ex}
\begin{ex}%[kiểm tra GHK1, Sở GD và ĐT - Vĩnh Phúc, 2021]%[Huỳnh Xuân Tín, 12EX4]%[2D1K4-2]%
    Gọi $S$ là tập tất cả các giá trị của tham số $m$ để đồ thị hàm số	$y=\dfrac{x-3}{x^2-2x-m}$ có đúng một đường
    tiệm cận đứng. Tính tổng các phần tử của tập $S$.
    \shortans{$2$}
    %	\choice
    %	{$-1$}
    %	{\True $2$}
    %	{$-6$}
    %	{$1$}
    \loigiai{
        Để đồ thị hàm số	$y=\dfrac{x-3}{x^2-2x-m}$ có đúng một đường
        tiệm cận đứng, ta có hai trường hợp sau
        \begin{enumerate}[TH 1.]
            \item $x^2-2x-m=0$ có nghiệm kép $\Leftrightarrow \Delta'=1+m=0\Leftrightarrow m=-1$.
            \item $x^2-2x-m=0$ có hai nghiệm phân biệt trong đó có một nghiệm bằng $3$
            \[\Leftrightarrow \heva{&\Delta'=1+m>0\\& 3^2-6-m=0}\Leftrightarrow\heva{&m>-1\\&m=3}\Leftrightarrow m=3.\]
        \end{enumerate}
        Khi đó $S=\{-1;3\}$ và có tổng là $2$.
    }
\end{ex}
\begin{ex}
    Tốc độ phản ứng của enzyme theo nồng độ cơ chất \( S \) được mô tả bởi phương trình Michaelis-Menten: $v(S) = \dfrac{V_{\text{max}} S}{K_m + S}$,
    trong đó \( v(S) \) là tốc độ phản ứng, \( S \) là nồng độ cơ chất, \( V_{\text{max}} \) là tốc độ tối đa, và \( K_m \) là hằng số Michaelis. Xác định và nêu ý nghĩa của đường tiệm cận đứng của hàm số này.
    \shortans{không có TCĐ, tốc độ phản ứng không thể tới vô hạn}
    \loigiai{
        Để tìm tiệm cận đứng, ta xét các giá trị của \( S \) làm cho mẫu số của phương trình bằng 0:
        \[
        K_m + S = 0 \Rightarrow S = -K_m
        \]
        Vì nồng độ cơ chất \( S \) không thể âm, không có tiệm cận đứng trong trường hợp này.
        \textbf{Ý nghĩa:} Điều này có nghĩa là tốc độ phản ứng enzyme không có giá trị nào dẫn đến tốc độ phản ứng tiến đến vô hạn trong phạm vi các giá trị hợp lý của \( S \).}
\end{ex}
\begin{ex}
    \immini{Một ống khói của nhà máy điện hạt nhân có mặt cắt là một hypebol $(H)$ có phương trình chính tắc là $\dfrac{x^2}{27^2}-\dfrac{y^2}{40^2}=1$ (Hình $1.25$). Xét hai nhánh bên trên $Ox$ của $(H)$ là đồ thị $(C)$ của hàm số $y=\dfrac{40}{27}\sqrt{x^2-27^2}$ (phần nét liền đậm). Tìm tất cả các đường tiệm cận xiên của $(C)$.}{\begin{tikzpicture}[>=latex,line join=round, line cap=round, scale=.04, font=\footnotesize]
            \draw[->] (-90,0)--(90,0) node[above]{$x$};
            \draw[->] (0,-130)--(0,80) node[left]{$y$};
            \foreach \x in {-80,-60,-40,-20,20,40,60,80}
            \draw[fill=black] (\x,0) circle (15pt) node[below, fill=white]{$\x$};
            \foreach \y in {-120,-100,-80,-60,-40,-20,20,40,60}
            \draw[fill=black] (0,\y) circle (15pt) node[left]{$\y$};
            \clip (-90,-130) rectangle (90,80);
            \draw[samples=200,smooth,blue,line width=1] plot[domain=-90:-27] (\x,{40*sqrt((\x)^2-27^2)/27});
            \draw[samples=200,smooth,blue,line width=1] plot[domain=27:90] (\x,{40*sqrt((\x)^2-27^2)/27});
            \draw[samples=200,smooth,blue,line width=1, dashed] plot[domain=-90:-27] (\x,{-40*sqrt((\x)^2-27^2)/27});
            \draw[samples=200,smooth,blue,line width=1, dashed] plot[domain=27:90] (\x,{-40*sqrt((\x)^2-27^2)/27});
            \draw (0,0) node[above right]{$O$};
    \end{tikzpicture}}
    \shortans{$y=\pm \dfrac{40}{27}$}
    \loigiai{
        Ta có
        \allowdisplaybreaks
        \begin{eqnarray*}
            a&=&\lim\limits_{x\to+\infty}\dfrac{f(x)}{x}	 =\lim\limits_{x\to+\infty}\dfrac{\dfrac{40}{27}\sqrt{x^2-27^2}}{x}\\
            &=&\lim\limits_{x\to+\infty}\dfrac{\dfrac{40}{27}x\sqrt{1-\dfrac{27^2}{x^2}}}{x}
            =\lim\limits_{x\to+\infty}\dfrac{40}{27}\sqrt{1-\dfrac{27^2}{x^2}}
            =\dfrac{40}{27}.\\
            b&=&\lim\limits_{x\to+\infty}\left[f(x)-ax\right]
            =\lim\limits_{x\to+\infty}\left[\dfrac{40}{27}\sqrt{x^2-27^2}-\dfrac{40}{27}x\right]\\
            &=&\lim\limits_{x\to+\infty}\dfrac{40}{27}\left(\sqrt{x^2-27^2}-x\right)
            =\lim\limits_{x\to+\infty}\dfrac{40}{27}\cdot\dfrac{x^2-27^2-x^2}{\sqrt{x^2-27^2}+x}\\
            &=&\lim\limits_{x\to+\infty}\dfrac{40}{27}\cdot\dfrac{-27^2}{x\left(\sqrt{1-\dfrac{27}{x^2}}+1\right)}
            =0.
        \end{eqnarray*}
        Vậy đường thẳng $y=\dfrac{40}{27}x$ là một tiệm cận xiên của đồ thị.\\
        Tương tự, $\lim\limits_{x\to-\infty}\dfrac{f(x)}{x}=-\dfrac{40}{27}\Rightarrow a=-\dfrac{40}{27}$; $\lim\limits_{x\to-\infty} \left[f(x)-ax\right]=0\Rightarrow b=0$.\\
        Vậy đường thẳng $y=-\dfrac{40}{27}x$ là tiệm cận xiên của đồ thị.
    }
\end{ex}

\Closesolutionfile{ans}
% \begin{dang}{Tìm các đường tiệm cận đồ thị hàm ẩn}
\end{dang}
\begin{vd}
    Cho hàm số $y=f(x)$ có bảng biến thiên như hình vẽ sau
    \begin{center}
        \begin{tikzpicture}[>=stealth]
            \tkzTabInit[nocadre=false,lgt=1,espcl=1.5,deltacl=0.5]{$x$/.7 ,$y'$/.7,$y$/2}
            {$-\infty$ , $-1$ , $2$ , $+\infty$}
            \tkzTabLine{ , + , $0$ , - , d , + , }
            \tkzTabVar{-/$1$ , +/$4$ , -/$-5$ , +/$+\infty$}
        \end{tikzpicture}
    \end{center}
    Tìm TCĐ, TCN của đồ thị hàm số
    \begin{listEX}[3]
        \item $y=\dfrac{2}{f(x)-3}$
        \item $y=\dfrac{-3}{f(x)+2}$
        \item $y=\dfrac{x-2}{f(x)+5}$
        \item $y=\dfrac{x+1}{f(x)-4}$
        \item $y=\dfrac{2}{f(x^2)+3}$
        \item $y=\dfrac{4f(x)-5}{3f(x)+1}$
    \end{listEX}
    \loigiai{}
\end{vd}
\begin{vd}\immini{Cho hàm bậc ba $y=f(x)$ có đồ thị như hình vẽ. Tìm số tiệm cận đứng của đồ thị hàm số
        \begin{listEX}[2]
            \item $y=\dfrac{\sqrt{x+3}}{(x-1)f(x)}$
            \item $g(x)=\dfrac{(x^2+4x+3)\sqrt{x^2+x}}{x\left[f^2(x)-2f(x)\right]}$ .
    \end{listEX}}{\begin{tikzpicture}[line cap=round,line join=round, >=stealth,font=\footnotesize]
            \begin{scope}[scale=.5]
                \def\a{-1} % Hệ số a phải khác 0
                \def\b{-13/2}
                \def\c{-12}
                \def\d{-9/2}
                \draw[->] (-5,0) -- (2,0)node[below]{$x$};
                \draw[->] (0,-3) -- (0,4) node[left] {$y$};
                \draw (0,0)node[below right]{$O$} (-3,0)node[below]{$-3$};
                \draw[dashed] (-1,0)node[below]{$-1$}|-(0,2)node[right]{$2$};
                \draw[samples=150,smooth,domain=-4:.-.2] plot(\x,{\a*(\x)^3+(\b)*(\x)^2+(\c)*\x+(\d)});
            \end{scope}
    \end{tikzpicture}}
    \loigiai{
        \begin{center}
            \begin{tikzpicture}[line cap=round,line join=round, >=stealth,font=\footnotesize,scale=1]
                \def\a{-1} % Hệ số a phải khác 0
                \def\b{-13/2}
                \def\c{-12}
                \def\d{-9/2}
                \draw[->] (-5,0) -- (2,0)node[below]{$x$};
                \draw[->] (0,-3) -- (0,4) node[left] {$y$};
                \draw (0,0)node[below right]{$O$} (-3,0)node[below]{$-3$} (-.3,0)node[above]{$a$};
                \draw[dashed] (-3.78,0)node[below]{$c$}|-(0,2)|-(-1.71,0)node[below]{$b$}|-(0,2) (-1,0)node[below]{$-1$}|-(0,2)node[right]{$2$};
                \draw[samples=150,smooth,domain=-4:.-.2] plot(\x,{\a*(\x)^3+(\b)*(\x)^2+(\c)*\x+(\d)});
            \end{tikzpicture}
        \end{center}
        $g(x)=\dfrac{(x^2+4x+3)\sqrt{x^2+x}}{x\left[f^2(x)-2f(x)\right]}=\dfrac{(x+1)(x+3)\sqrt{x(x+1)}}{x\left[f^2(x)-2f(x)\right]}$.\\
        Điều kiện của căn là $x\le -1; x\ge 0$.\\
        Dựa vào đồ thị ta có \[x\left[f^2(x)-2f(x)\right]=0 \Leftrightarrow \hoac{&x=0\\&f(x)=0\\& f(x)=2} \Leftrightarrow \hoac{&x=0\text{ (nhận)}\\&x=-3\text{ (nhận)};\ x=a \text{ (loại)} \\&x=-1\text{ (nhận)};\ x=b\text{ (nhận)};\ x=c\text{ (nhận)}}\]\\
        Số TCĐ lúc này chính là số nghiệm không bị rút gọn của mẫu, vậy có bốn TCĐ là $x=0; x=-3; x=b; x=c$.
    }
\end{vd}
\BTTN
\Opensolutionfile{ans}[ans/2D1-4-DANG-3]
\begin{ex}%[2D1K4-1]
    Cho hàm số $y=f(x)$ có bảng biến thiên như hình bên. Đồ thị hàm số $y=\dfrac{-5}{f(x)+4}$ có bao nhiêu tiệm cận đứng?
    \begin{center}
        \begin{tikzpicture}[scale=0.8]
            \tkzTabInit[nocadre=false,lgt=1.5,espcl=3,deltacl=0.6]
            {$x$ /0.6,$y’$ /0.6,$y$ /2}
            {$-\infty$ ,$1$, $2$, $+\infty$}
            \tkzTabLine{,+,d,-,d,+,}
            \tkzTabVar{-/$-4$,+/$3$,-/$-5$,+/$+\infty$}
        \end{tikzpicture}
    \end{center}
    \choice
    {$1$}
    {$3$}
    {\True $2$}
    {$4$}
    \loigiai{
        Dựa vào bảng biến thiên suy ra
        $f(x)+4=0 \Leftrightarrow f(x) =-4$, phương trình này có $2$ nghiệm phân biệt nên đồ thị hàm số $y=\dfrac{-5}{f(x)+4}$ có $2$ tiệm cận đứng.
    }
\end{ex}
\begin{ex}%[2D1K4-1]
    Cho hàm số $y=f(x)$ có bảng biến thiên như hình bên. Đồ thị hàm số $y=\dfrac{x+2}{2f(x)-1}$ có bao nhiêu tiệm cận đứng?
    \begin{center}
        \begin{tikzpicture}[scale=0.8]
            \tkzTabInit[nocadre=false,lgt=1.5,espcl=3,deltacl=0.6]
            {$x$ /0.6,$y’$ /0.6,$y$ /2}
            {$-\infty$ ,$-1$, $0$, $1$, $+\infty$}
            \tkzTabLine{,+,0,-,0,+,0,-,}
            \tkzTabVar{-/$-\infty$,+/$0$,-/$-\dfrac{5}{3}$,+/$0$,-/$-\infty$}
        \end{tikzpicture}
    \end{center}
    \choice
    {$1$}
    {$3$}
    {$2$}
    {\True $0$}
    \loigiai{
        Dựa vào bảng biến thiên suy ra
        $2f(x)-1=0 \Leftrightarrow f(x) =\dfrac{1}{2}$, phương trình này có $0$ nghiệm nên đồ thị hàm số $y=\dfrac{x+2}{2f(x)-1}$ không có tiệm cận đứng.
    }
\end{ex}
%69
\begin{ex}%[2D1K4-1]
    Cho hàm số $y=f(x)$ có bảng biến thiên như hình bên. Đồ thị hàm số $y=\dfrac{1}{2f(x)-3}$ có bao nhiêu tiệm cận đứng?
    \begin{center}
        \begin{tikzpicture}[scale=0.8]
            \tkzTabInit[nocadre=false,lgt=1.5,espcl=3,deltacl=0.6]
            {$x$ /0.6,$y’$ /0.6,$y$ /2}
            {$-\infty$ ,$0$, $1$, $+\infty$}
            \tkzTabLine{,+,0,-,0,+,}
            \tkzTabVar{-/$-\infty$,+/$5$,-/$-1$,+/$+\infty$}
        \end{tikzpicture}
    \end{center}
    \choice
    {$1$}
    {\True $3$}
    {$2$}
    {$0$}
    \loigiai{
        Dựa vào bảng biến thiên suy ra
        $2f(x)-3=0 \Leftrightarrow f(x) =-\dfrac{3}{2}$, phương trình này có $3$ nghiệm phân biệt nên đồ thị hàm số $y=\dfrac{1}{2f(x)-3}$ có ba tiệm cận đứng.
    }
\end{ex}
%70
%71
%72
\begin{ex}%[2D1K4-1]
    Cho hàm số $y=f(x)$ có bảng biến thiên như hình bên. Đồ thị hàm số $y=\dfrac{x}{f(x)-3}$ có bao nhiêu tiệm cận đứng?
    \begin{center}
        \begin{tikzpicture}[scale=0.8]
            \tkzTabInit[nocadre=false,lgt=1.5,espcl=3,deltacl=0.6]
            {$x$ /0.6,$y’$ /0.6,$y$ /2}
            {$-\infty$ ,$-1$, $0$, $1$, $+\infty$}
            \tkzTabLine{,-,0,+,0,-,0,+,}
            \tkzTabVar{+/$+\infty$,-/$0$,+/$3$,-/$0$,+/$+\infty$}
        \end{tikzpicture}
    \end{center}
    \choice
    {$1$}
    {\True $3$}
    {$2$}
    {$4$}
    \loigiai{
        Dựa vào bảng biến thiên suy ra
        $f(x)-3=0 \Leftrightarrow f(x) =3$, phương trình này có $2$ nghiệm phân biệt khác $0$ và một nghiệm bội chẵn $x=0$ nên đồ thị hàm số $y=\dfrac{x}{f(x)-3}$ có ba tiệm cận đứng.
    }
\end{ex}
\begin{ex}%[2D1K4-1]
    Cho hàm số $y=f(x)$ có bảng biến thiên như hình bên. Đồ thị hàm số $y=\dfrac{4}{f(x)+1}$ có tiệm cận ngang là đường thẳng
    \begin{center}
        \begin{tikzpicture}[scale=0.8]
            \tkzTabInit[nocadre=false,lgt=1.5,espcl=3,deltacl=0.6]
            {$x$ /0.6,$y’$ /0.6,$y$ /2}
            {$-\infty$ ,$-1$, $2$, $+\infty$}
            \tkzTabLine{,+,0,-,0,+,}
            \tkzTabVar{-/$1$,+/$4$,-/$-5$,+/$1$}
        \end{tikzpicture}
    \end{center}
    \choice
    {$y=1$}
    {$y=-5$}
    {\True $y=2$}
    {$y=4$}
    \loigiai{
        Dựa vào bảng biến thiên suy ra
        $\lim \limits_{x \to \pm \infty} f(x)=1 \Leftrightarrow \lim \limits_{x \to \pm \infty} \dfrac{4}{f(x)+1} =2$ nên đồ thị hàm số đã cho có tiệm cận ngang là $y=2$.
    }
\end{ex}
\begin{ex}%[2D1K4-1]
    Cho hàm số $y=f(x)$ có bảng biến thiên như hình bên. Đồ thị hàm số $y=\dfrac{2-f(x)}{f(x)+3}$ có tiệm cận ngang là đường thẳng
    \begin{center}
        \begin{tikzpicture}[scale=0.8]
            \tkzTabInit[nocadre=false,lgt=1.5,espcl=3,deltacl=0.6]
            {$x$ /0.6,$y’$ /0.6,$y$ /2}
            {$-\infty$ ,$0$, $2$, $+\infty$}
            \tkzTabLine{,-,0,+,0,-,}
            \tkzTabVar{+/$+\infty$,-/$1$,+/$5$,-/$-\infty$}
        \end{tikzpicture}
    \end{center}
    \choice
    {$y=1$}
    {$y=-3$}
    {$y=2$}
    {\True $y=-1$}
    \loigiai{
        Dựa vào bảng biến thiên suy ra
        $\lim \limits_{x \to \pm \infty} f(x)=\pm \infty \Leftrightarrow \lim \limits_{x \to \pm \infty} \dfrac{2-f(x)}{f(x)+3} =-1$ nên đồ thị hàm số $y=\dfrac{2-f(x)}{f(x)+3}$ có tiệm cận ngang là $y=-1$.
    }
\end{ex}
\begin{ex}%[2D1K4-1]
    Cho hàm số $y=f(x)$ có bảng biến thiên như hình bên. Đồ thị hàm số $y=\dfrac{1}{f^2(x)-4f(x)+4}$ có bao nhiêu tiệm cận đứng?
    \begin{center}
        \begin{tikzpicture}[scale=0.8]
            \tkzTabInit[nocadre=false,lgt=1.5,espcl=3,deltacl=0.6]
            {$x$ /0.6,$y’$ /0.6,$y$ /2}
            {$-\infty$, $2$, $+\infty$}
            \tkzTabLine{,-,0,+,}
            \tkzTabVar{+/$1$,-/$-3$,+/$1$}
        \end{tikzpicture}
    \end{center}
    \choice
    {$1$}
    {$3$}
    {$2$}
    {$0$}
    \loigiai{
        Dựa vào bảng biến thiên suy ra $f^2(x)-4f(x)+4=0 \Leftrightarrow f(x)=2$, phương trình $f(x)=2$ vô nghiệm nên đồ thị hàm số đã cho không có tiệm cận đứng.
    }
\end{ex}
%83
\begin{ex}%[2D1K4-1]
    Cho hàm số $y=f(x)$ có bảng biến thiên như hình bên. Đồ thị hàm số $y=\dfrac{1}{f(3-x)-2}$ có bao nhiêu tiệm cận đứng?
    \begin{center}
        \begin{tikzpicture}[scale=0.8]
            \tkzTabInit[nocadre=false,lgt=1.5,espcl=3,deltacl=0.6]
            {$x$ /0.6,$y’$ /0.6,$y$ /2}
            {$-\infty$ ,$-2$, $2$, $+\infty$}
            \tkzTabLine{,+,0,-,0,+,}
            \tkzTabVar{-/$-\infty$,+/$3$,-/$0$,+/$+\infty$}
        \end{tikzpicture}
    \end{center}
    \choice
    {$1$}
    {\True $3$}
    {$2$}
    {$0$}
    \loigiai{
        Dựa vào bảng biến thiên suy ra $f(3-x)-2=0 \Leftrightarrow f(3-x)=2$, phương trình này có $3$ nghiệm phân biệt nên đồ thị hàm số đã cho có $3$ tiệm cận đứng.
    }
\end{ex}
\begin{ex}%[2D1G4-1]
    Cho hàm số $y=f(x)$ có bảng biến thiên như hình bên. Đồ thị hàm số $y=\dfrac{4}{f(x^2)-2}$ có bao nhiêu tiệm cận đứng?
    \begin{center}
        \begin{tikzpicture}[scale=0.8]
            \tkzTabInit[nocadre=false,lgt=1.5,espcl=3,deltacl=0.6]
            {$x$ /0.6,$y’$ /0.6,$y$ /2}
            {$-\infty$ ,$0$, $3$, $+\infty$}
            \tkzTabLine{,-,0,+,d,-,}
            \tkzTabVar{+/$8$,-/$1$,+/$4$,-/$2$}
        \end{tikzpicture}
    \end{center}
    \choice
    {$5$}
    {$3$}
    {\True $2$}
    {$4$}
    \loigiai{
        Dựa vào bảng biến thiên suy ra
        $f(x^2)-2=0 \Leftrightarrow f(x^2) =2$. Kẻ đường thẳng $y=2$ ta thấy đường thẳng cắt đồ thị hàm số tại hai điểm phân biệt. Suy ra
        $$\hoac{&x^2=a \; (a<0)\\&x^2=b \; (b >0)} \Rightarrow x=\pm \sqrt{b}.$$
        Do đó đồ thị hàm số đã cho có $2$ tiệm cận đứng.
    }
\end{ex}%89
\begin{ex}%[2D1G4-1]
    Cho hàm số $y=f(x)$ có bảng biến thiên như hình bên. Đồ thị hàm số $y=\dfrac{2}{f(|x|)-3}$ có bao nhiêu tiệm cận ngang?
    \begin{center}
        \begin{tikzpicture}[scale=0.8]
            \tkzTabInit[nocadre=false,lgt=1.5,espcl=3,deltacl=0.6]
            {$x$ /0.6,$y’$ /0.6,$y$ /2}
            {$-\infty$ ,$0$, $2$, $+\infty$}
            \tkzTabLine{,+,0,-,0,+,}
            \tkzTabVar{-/$-\infty$,+/$3$,-/$-1$,+/$+\infty$}
        \end{tikzpicture}
    \end{center}
    \choice
    {$4$}
    {\True $3$}
    {$5$}
    {$6$}
    \loigiai{
        Dựa vào bảng biến thiên suy ra
        $f(|x|)-3=0 \Leftrightarrow f(|x|) =3$.\\
        Bảng biến thiên hàm số $y=f(|x|)$ như sau
        \begin{center}
            \begin{tikzpicture}[scale=0.8]
                \tkzTabInit[nocadre=false,lgt=1.5,espcl=3,deltacl=0.6]
                {$x$ /0.6,$y’$ /0.6,$y$ /2}
                {$-\infty$ ,$-2$, $0$, $2$, $+\infty$}
                \tkzTabLine{,-,0,+,0,-,0,+,}
                \tkzTabVar{+/$+\infty$,-/$-1$,+/$3$,-/$-1$,+/$+\infty$}
            \end{tikzpicture}
        \end{center}
        Dựa vào bảng biến thiên hàm số $y=f(|x|)$, phương trình $f(|x|) =3$ có ba nghiệm phân biệt, do đó đồ thị hàm số $y=\dfrac{2}{f(|x|)-3}$ có $3$ tiệm cận đứng.
    }
\end{ex}
\begin{ex}
    \immini{ %Câu 90
        Cho hàm số bậc ba $f(x)= ax^3 +bx^2 +cx +d$ có đồ thị như hình vẽ bên. Đồ thị hàm số $g(x) = \dfrac{\sqrt{x+1}}{(x-3)\cdot f(x)}$ có bao nhiêu đường tiệm cận đứng?
        \choice
        {5}
        {2}
        {4}
        {\True 3}}{\begin{tikzpicture}[scale=.5, font=\footnotesize, line join=round, line cap=round, >=stealth]
            \def\xmin{-3}\def\xmax{3}\def\ymin{-5}\def\ymax{1}
            \draw[->] (\xmin-0.2,0)--(\xmax+0.2,0) node[below] {\footnotesize $x$};
            \draw[->] (0,\ymin-0.2)--(0,\ymax+0.2) node[right] {\footnotesize $y$};
            \draw (0,0) node [below left] {\footnotesize $O$};
            \foreach \x in {-1}\draw (\x,-0.1)--(\x,0.1) node [above] {\footnotesize $\x$};
            \foreach \x in {2}\draw (\x,-0.1)--(\x,0.1) node [above right] {\footnotesize $\x$};
            \foreach \y in {}\draw (-0.1,\y)--(0.1,\y) node [right] {\footnotesize $\y$};
            \clip (\xmin,\ymin) rectangle (\xmax,\ymax);
            \draw[smooth,samples=200,domain=\xmin:\xmax] plot (\x,{1*((\x)^3)+0*((\x)^2)+-3*(\x)+-2});
        \end{tikzpicture}
    }
    \loigiai{
        * Điều kiện: $\heva{&x \ne 3\\&f(x) \ne 0\\&x \ge -1.}$\\
        Nhìn hình vẽ ta thấy
        $f(x)=0\Leftrightarrow \hoac{&x=-1&(\text{nghiệm kép}) \\&x=2&(\text{nghiệm đơn}).}$\\
        Vậy $g(x) = \dfrac{\sqrt{x+1}}{(x-3)\cdot a(x+1)^2 (x-2)}.$ \\
        Đồ thị hàm số $g(x)$ có 3 đường tiệm cận đứng.}
\end{ex}
\begin{ex}
    \immini{ %Câu 92.
        Đường cong ở hình bên là đồ thị của hàm số $y = ax^3 +bx^2 +cx+d$. Đồ thị hàm số $y =\dfrac{(2x+1)\sqrt{x-1}}{x\cdot f(x-2)}$ có tất cả bao nhiêu tiệm cận đứng?
        \choice
        {1}
        {3}
        {4}
        {\True 2}}{\begin{tikzpicture}[scale=.6, font=\footnotesize, line join=round, line cap=round, >=stealth]
            \def\xmin{-3}\def\xmax{3}\def\ymin{-3}\def\ymax{3}
            \draw[->] (\xmin-0.2,0)--(\xmax+0.2,0) node[below] {\footnotesize $x$};
            \draw[->] (0,\ymin-0.2)--(0,\ymax+0.2) node[right] {\footnotesize $y$};
            \draw (0,0) node [below left] {\footnotesize $O$};
            \foreach \x in {-2}\draw (\x,-0.1)--(\x,0.1) node [above left] {\footnotesize $\x$};
            \foreach \x in {2}\draw (\x,-0.1)--(\x,0.1) node [above right] {\footnotesize $\x$};
            \foreach \y in {}\draw (-0.1,\y)--(0.1,\y) node [right] {\footnotesize $\y$};
            \clip (\xmin,\ymin) rectangle (\xmax,\ymax);
            \draw[smooth,samples=200,domain=\xmin:\xmax] plot (\x,{(2/3)*((\x)^3)+0*((\x)^2)+-(8/3)*(\x)});
    \end{tikzpicture}}
    \loigiai{
        * Điều kiện: $\heva{&x \ne 0\\&f(x-2) \ne 0\\&x \ge 1.}$\\
        Nhìn hình vẽ ta thấy
        $f(x-2)=0\Leftrightarrow \hoac{&x-2=-2\\&x-2=0\\&x-2=2}\Leftrightarrow \hoac{&x=0&(\text{không thỏa mãn})\\&x=2&(\text{nghiệm đơn})\\&x=4&(\text{nghiệm đơn}).}$\\
        Vậy $g(x) =\dfrac{(2x+1)\sqrt{x-1}}{x\cdot f(x-2)}=\dfrac{(x-1)\sqrt{x+2}}{x\cdot ax(x-2)(x-4)}.$ \\
        Đồ thị hàm số $g(x)$ có 2 đường tiệm cận đứng.}
\end{ex}
\begin{ex}
    \immini{ %Câu 93.
        Cho hàm số $y= f(x)$ có đồ thị cắt trục hoành tại đúng 3 điểm như hình bên. Đồ thị hàm số $y =\dfrac{(x+2)\sqrt{3-x}}{f(|x|)}$
        có tất cả bao nhiêu tiệm cận đứng?
        \choice
        {1}
        {3}
        {4}
        {\True 2}}{\begin{tikzpicture}[scale=.5, font=\footnotesize, line join=round, line cap=round, >=stealth]
            \def\xmin{-2}\def\xmax{5}\def\ymin{-3}\def\ymax{5}
            \draw[->] (\xmin-0.2,0)--(\xmax+0.2,0) node[below] {\footnotesize $x$};
            \draw[->] (0,\ymin-0.2)--(0,\ymax+0.2) node[right] {\footnotesize $y$};
            \draw (0,0) node [below left] {\footnotesize $O$};
            \foreach \x in {-1,2,4}\draw (\x,-0.1)--(\x,0.1) node [above left] {\footnotesize $\x$};
            \foreach \y in {}\draw (-0.1,\y)--(0.1,\y) node [right] {\footnotesize $\y$};
            \clip (\xmin,\ymin) rectangle (\xmax,\ymax);
            \draw[smooth,samples=200,domain=-1.2:0] plot(\x,{0-8.48*(\x)^(2.0)-5.48*(\x)+3.0});
            \draw[smooth,samples=200,domain=0:2]
            plot(\x,{0-2.7989489689153735*(\x)^(3.0)+8.326740175055514*(\x)^(2.0)-6.957684474449535*(\x)+3.0});
            \draw[smooth,samples=200,domain=2:5]
            plot(\x,{2.395330112721417*(\x)^(2.0)-14.371980676328501*(\x)+19.162640901771336});
    \end{tikzpicture}}
    \loigiai{
        * Điều kiện: $\heva{&f(|x|) \ne 0\\&x \le 3.}$\\
        Nhìn hình vẽ ta thấy
        $f(|x|)=0\Leftrightarrow \hoac{&|x|=-1\\&|x|=2\\&|x|=4}\Leftrightarrow \hoac{&x=\pm 2&(\text{nghiệm đơn})\\&x=- 4&(\text{nghiệm đơn})\\&x=4&(\text{không thỏa mãn}).}$\\
        Vậy $y =\dfrac{(x+2)\sqrt{3-x}}{a(x-2)(x+2)(x+4)(x-4)}$ \\
        Đồ thị hàm số có 2 đường tiệm cận đứng.}
\end{ex}
\begin{ex}
    \immini{ %Câu 94.
        Đường cong ở hình bên là đồ thị của hàm số $y = ax^3 +bx^2 +cx+d$. Đồ thị hàm số $y =\dfrac{(2x+1)\sqrt{1-x}}{f(|x|)}$ có tất cả bao nhiều tiệm cận đứng?
        \choice
        { 1}
        {3}
        {4}
        {\True 2}}{\begin{tikzpicture}[scale=.8, font=\footnotesize, line join=round, line cap=round, >=stealth]
            \def\xmin{-1}\def\xmax{2}\def\ymin{-1.5}\def\ymax{1.5}
            \draw[->] (\xmin-0.2,0)--(\xmax+0.2,0) node[below] {\footnotesize $x$};
            \draw[->] (0,\ymin-0.2)--(0,\ymax+0.2) node[right] {\footnotesize $y$};
            \draw (0.15,0) node [below left] {\footnotesize $O$};
            \foreach \x in {}\draw (\x,0.1)--(\x,-0.1) node [below] {\footnotesize $\x$};
            \foreach \y in {-1,1}\draw (0.1,\y)--(-0.1,\y) node [left] {\footnotesize $\y$};
            \clip (\xmin,\ymin) rectangle (\xmax,\ymax);
            \draw[smooth,samples=200,domain=\xmin:\xmax] plot (\x,{4*((\x)^3)+-6*((\x)^2)+0*(\x)+1});
            \draw[dashed] (0.5,0)--(0.5,0.0)--(0,0.0);
            \draw (0.5,-1pt)--(0.5,1pt) node [above] {\footnotesize $\frac{1}{2}$};
            \draw (-0.7,-1pt)--(-0.7,1pt) node [above] {\footnotesize $-\frac{1}{2}$};
            \draw (1,-1pt)--(1,1pt) node [above] {\footnotesize $1$};
            \draw[dashed] (0.0,0)--(0.0,1.0)--(0,1.0);
            \draw[dashed] (1.0,0)--(1.0,-1.0)--(0,-1.0);
    \end{tikzpicture}}
    \loigiai{
        * Điều kiện: $\heva{&f(|x|) \ne 0\\&x \le 1.}$\\
        Nhìn hình vẽ ta thấy
        $f(|x|)=0\Leftrightarrow \hoac{&|x|=-\dfrac{1}{2}\\&|x|=\dfrac{1}{2}\\&|x|=x_1>1}\Leftrightarrow \hoac{&x=\pm \dfrac{1}{2}&(\text{hai nghiệm đơn})\\&x=- x_1&(\text{nghiệm đơn})\\&x=x_1&(\text{không thỏa mãn}).}$\\
        Vậy $y =\dfrac{(2x+1)\sqrt{1-x}}{f(|x|)}=\dfrac{(2x+1)\sqrt{1-x}}{a\left(x-\dfrac{1}{2}\right)\left(x+\dfrac{1}{2}\right)(x+x_1)(x-x_1)}$ \\
        Đồ thị hàm số có 2 đường tiệm cận đứng.}
\end{ex}
\begin{ex}
    \immini{ %Câu 96.
        Cho đồ thị hàm số $y =f(x)$ và trục hoành có đúng 2 điểm chung như hình bên. Đồ thị hàm số $y =\dfrac{(x-1)\sqrt{3-x}}{f(x^2)}$ có tất cả bao nhiêu tiệm cận đứng?
        \choice
        {1}
        {3}
        {4}
        {\True 2}}{\begin{tikzpicture}[scale=.8, font=\footnotesize, line join=round, line cap=round, >=stealth]
            \def\xmin{-1.5}\def\xmax{2}\def\ymin{-1}\def\ymax{4.5}
            \draw[->] (\xmin-0.2,0)--(\xmax+0.2,0) node[below] {\footnotesize $x$};
            \draw[->] (0,\ymin-0.2)--(0,\ymax+0.2) node[right] {\footnotesize $y$};
            \draw (0,0) node [below left] {\footnotesize $O$};
            \foreach \x in {1}\draw (\x,0.1)--(\x,-0.1) node [below] {\footnotesize $\x$};
            \foreach \x in {-1}\draw (\x,0.1)--(\x,-0.1) node [below left] {\footnotesize $\x$};
            \clip (\xmin,\ymin) rectangle (\xmax,\ymax);
            \draw[smooth,samples=200,domain=-1.1:0] plot(\x,{21.044670464836045*(\x)^(3.0)+24.701786337609526*(\x)^(2.0)+5.65711587277348*(\x)+2.0});
            \draw[smooth,samples=200,domain=0:\xmax] plot(\x,{10.591704641658401*(\x)^(3.0)-19.26315454354621*(\x)^(2.0)+6.6714499018878115*(\x)+2.0});
    \end{tikzpicture}}
    \loigiai{
        * Điều kiện: $\heva{&f(x^2) \ne 0\\&x \le 3.}$\\
        Nhìn hình vẽ ta thấy
        $f(x^2)=0\Leftrightarrow \hoac{&x^2=-1\\&x^2=1}\Leftrightarrow x=\pm 1\,(\text{nghiệm kép}).$\\
        Vậy $y=\dfrac{(x-1)\sqrt{3-x}}{f(x^2)}=\dfrac{(x-1)\sqrt{3-x}}{(x-1)^2(x+1)^2}$ \\
        Đồ thị hàm số có 2 đường tiệm cận đứng.}
\end{ex}
\begin{ex}%[2D1G4-3]%Câu 52
    Cho hàm số $y=ax^3+bx^2+cx+d$ có đồ thị như hình vẽ. Đồ thị của hàm số $g(x)=\dfrac{x^2-x}{f^2(x)-2f(x)}$ có bao nhiêu đường tiệm cận đứng?
    \choice
    {$2$}
    {$3$}
    {\True $4$}
    {$5$}
    \begin{center}
        \begin{tikzpicture}[thick,>=stealth,x=1cm,y=1cm,scale=.7]
            \draw[thin,color=gray!50] (-3.3,-1.3) grid (3.9,5.9);
            \draw[->] (-3.2,0) -- (4.2,0) node[right] {$x$};
            \draw[->] (0,-1.2) -- (0,5.2) node[above] {$y$};
            \draw[color=blue, domain=-2.15:2.15,samples=300] plot (\x,{(\x)^3-3*(\x)+2}) node[right] {$y=f(x)$};
            \draw (-2,0) circle (1.5pt) node[below left]{$-2$};
            \draw (-1,0) circle (1.5pt) node[below]{$-1$};
            \draw (0,0) circle (1.5pt) node[above left]{$O$};
            \draw (1,0) circle (1.5pt) node[below]{$1$};
            \draw (0,4) circle (1.5pt) node[right]{$4$};
            \draw (-1,4) circle (1.5pt);
            \draw[dashed] (-1,0)--(-1,4)--(0,4);
            \draw[red] (-3,2)--(3.2,2);
            \draw[red] (3.5,2) node[right]{$f(x)=2$};
        \end{tikzpicture}
    \end{center}
    \loigiai{
        Xét phương trình $f^2(x)-2f(x)=0 \Leftrightarrow \hoac{&f(x)=0\\&f(x)=2}\Leftrightarrow \hoac{&x=1 \, (\textrm{nghiệm kép trùng nghiệm đơn ở tử số})\\&x=-2\, (\textrm{nghiệm đơn khác nghiệm của tử})\\&x=a\in(-2; -1)\\&x=0\, (\textrm{nghiệm đơn trùng nghiệm ở tử})\\&x=b\in(1; 2)}$\\
        \textbf{Kết luận:} Đồ thị hàm số có $4$ đường tiệm cận đứng.
    }
\end{ex}
\begin{ex}%[Thi thử L3, Lương Thế Vinh, Hà Nội, 2018]%[Phạm Toàn, Dự án (12EX-10)]%[2D1G4-3]%
    \immini{Cho hàm số $y=f(x)$ có đạo hàm liên tục trên $\mathbb{R}$. Đồ thị hàm $f(x)$ như hình vẽ. Số đường tiệm cận đứng của đồ thị hàm số $y=\dfrac{x^2-1}{f^2(x)-4f(x)}$ bằng
        \choice
        {$3$}
        {$1$}
        {$2$}
        {\True $4$}
    }{\begin{tikzpicture}[>=stealth,x=1cm,y=0.75cm,scale=0.7]
            \draw[->] (-2.5,0)--(0,0)%
            node[below right]{$O$}--(2.5,0) node[below]{$x$};
            \draw[->] (0,-2) --(0,5) node[right]{$y$};
            \foreach \x in {-1,1}{
                \draw (\x,0) node[below]{\footnotesize $\x$} circle (1pt);%Ox
            }
            \foreach \y in {2,4}{
                \draw (0,\y) node[right]{\footnotesize $\y$} circle (1pt);%Oy
            }
            \draw[samples=100,domain=-2.05:2] plot (\x,{(\x -1)^2*(\x+2)});
            \draw [dashed] (-1,0)--(-1,4)--(0,4);
            \draw(-1,4) circle (1pt);
    \end{tikzpicture}}
    \loigiai{Xét $f^2(x)-4f(x)=0\Leftrightarrow \hoac{& f(x)=0\\ &f(x)=4.}$\\
        Xét $f(x)=0$ có hai nghiệm, nghiệm $x_1\ne \pm 1$ và nghiệm $x_2=1$ là nghiệm bội (do đồ thị tiếp xúc với trục hoành tại $x=1$. Trường hợp này có $2$ tiệm cận đứng.\\
        Xét $f(x)=4$ có hai nghiệm, nghiệm $x_3\ne \pm 1$ và nghiệm $x_4=-1$ là nghiệm bội (do đồ thị tiếp xúc với đường thẳng $y=4$ tại $x=-1$. Trường hợp này có $2$ tiệm cận đứng.\\
        Vậy đồ thị có $4$ tiệm cận đứng.}
\end{ex}
\begin{ex}%[Thi thử, Trường THPT Lý Thái Tổ - Bắc Ninh, 2019]%[Duong Xuan Loi, 12EX3]%[2D1G4-3]%
    \immini{
        Cho hàm số $f(x)$ có đồ thị như hình bên. Số đường tiệm cận đứng của đồ thị hàm số
        $y=\dfrac{(x^2-4)(x^2+2x)}{[f(x)]^2+2f(x)-3}$ là
        \choice
        {\True $4$}
        {$5$}
        {$3$}
        {$2$}
    }{
        \begin{tikzpicture}[scale=0.5, font=\footnotesize, line join=round, line cap=round, >=stealth]
            \def\a{1} \def\b{-8} \def\c{1} % Hệ số
            \def\xt{-3.7} \def\xp{4} \def\yt{2} \def\yd{-3.7} % x_trái, x_phải, y_trên, y_dưới (giới hạn)
            \draw[->] (\xt,0)--(\xp,0) node [below]{$x$};
            \draw[->] (0,\yd)--(0,\yt) node [left]{$y$};
            \node at (0,0) [below left]{$O$};
            \clip (\xt-0.1,\yd+0.1) rectangle (\xp-0.1,\yt-0.1);
            \draw[smooth,samples=300] plot(\x,{1/4*(\a*(\x)^4+\b*(\x)^2)+\c});
            \draw[dashed] (-2,0)node[above]{$-2$}--(-2,-3)--(2,-3)--(2,0)node[above]{$2$};
            \node at (0,-3)[above left]{$-3$};
            \node at (-3,0)[above left]{$-3$};
            \node at (0,1)[above right]{$1$};
            \node at (3,0)[above right]{$3$};
            \fill (0,0) circle (1pt) (0,-3) circle (1pt) (2,0) circle (1pt) (-2,0) circle (1pt) (-3,0) circle (1pt) (0,1) circle (1pt) (3,0) circle (1pt);
        \end{tikzpicture}
    }
    \loigiai{
        Ta có $y=\dfrac{(x^2-4)(x^2+2x)}{[f(x)]^2+2f(x)-3}$ có các nghiệm ở tử là $x=0$ (bội $1$), $x=2$ (bội $1$), $x=-2$ (bội $2$).\\
        Mặt khác, từ đồ thị $f(x)$ ta thấy hàm số $y=\dfrac{(x^2-4)(x^2+2x)}{[f(x)]^2+2f(x)-3}$ có các nghiệm ở mẫu là
        $f^2(x)+2f(x)-3=0\Leftrightarrow \hoac{& f(x)=1 \\ & f(x)=-3}
        \Leftrightarrow \hoac{& x=0,x=x_1,x=x_2 \\ & x=-2,x=2.}$\\
        Trong đó nghiệm $x=0$, $x=-2$, $x=2$ đều có bội $2$ và $x_1$, $x_2$ khác các nghiệm của tử.\\
        So sánh bội nghiệm ở mẫu và bội nghiệm ở tử thì thấy đồ thị có các tiệm cận đứng là $x=0$, $x=2$; $x=x_1$; $x=x_2$.
    }
\end{ex}
\begin{ex}%[Thi thử, THPT Sơn Tây, Hà Nội, 2019]%[Huỳnh Xuân Tín, 12EX3]%[2D1G4-3]%
    \immini{Cho hàm số $ f(x)=(x+3)(x+1)^2(x-1)(x-3)$ có đồ thị như hình vẽ. Đồ thị hàm số $ g(x)=\dfrac{\sqrt{x-1}}{f^2(x)-9f(x)}$ có bao nhiêu tiệm cận đứng và tiệm cận ngang?
        \choice
        {$3$}
        {\True$ 4$}
        {$ 9$}
        { $8$}
    }{\begin{tikzpicture}[scale=0.3, font=\footnotesize, line join=round, line cap=round, >=stealth]
            %\draw[dashed, line width=0.1pt, gray] (-3.2,-5.5) grid (5.2,4.5);
            \draw[->] (-3.5,0)--(0,0) node[below right]{$O$}--(3.6,0) node[below]{$x$};
            \draw[fill=black] (0,0) circle (1pt);
            \draw[->] (0,-7.7) --(0,6.5) node[right]{$y$};
            \foreach \x in {-3,-1,3}{
                \draw[fill=black] (\x,0) node[below left]{$\x$} circle (1pt);}
            \draw[fill=black] (1,0) node[below right]{$1$} circle (1pt);
            \draw[fill=black] (0,1.35) node[above left]{$9$} circle (1pt);
            \draw [black, domain=-3.2:3.18, samples=100] %
            plot(\x,{0.15*(\x+3)*(\x+1)^2*(\x-1)*(\x-3)});
    \end{tikzpicture}}
    \loigiai{Điều kiện xác định của hàm số $g(x)$ là $\heva{&x\ge1\\ &f^2(x)-9f(x)\not=0.}$\\
        Từ $f^2(x)-9f(x)=0\Leftrightarrow \hoac{&f(x)=0\\&f(x)=9.}$\\
        Với $f(x)=0$ có nghiệm là $x=\pm 1, x=\pm 3$.\\
        Dựa vào đồ thị ta thấy nghiệm của phương trình $f(x)=9$ là hoành độ giao điểm của đường thẳng $y=9$ với đồ thị hàm số $y=f(x)$ nên có nghiệm là $-3<x_3<x_2<-1<0<x_1<1<3<x_0$.\\
        Do đó tập xác định của hàm số $y=g(x)$ là $\mathscr{D}=\left[1;+\infty \right)\setminus\left\lbrace1;3;x_0 \right\rbrace $.\\
        Khi đó ta có \begin{itemize}
            \item $\lim\limits_{x\rightarrow1^+ } g(x)=\lim\limits_{x\rightarrow1^+ }\dfrac{\sqrt{x-1}}{f(x)\left(f(x)-9 \right)}=+\infty$ (vì $x$ tiến gần bên phải $1$ thì $f(x)<0, f(x)-9<0$), suy ra đường thẳng $x=1$ là tiệm cận đứng.
            \item $\lim\limits_{x\rightarrow3^+ } g(x)=\lim\limits_{x\rightarrow3^+ }\dfrac{\sqrt{x-1}}{f(x)\left(f(x)-9 \right)}=-\infty$ (vì $x$ tiến gần bên phải $3$ thì $f(x)>0, f(x)-9<0$), suy ra đường thẳng $x=3$ là tiệm cận đứng.
            \item $\lim\limits_{x\rightarrow x_0^+} g(x)=\lim\limits_{x\rightarrow x_0^+ }\dfrac{\sqrt{x-1}}{f(x)\left(f(x)-9 \right)}=+\infty$ (vì $x$ tiến gần bên phải $x_0$ thì $f(x)>0, f(x)-9>0$), suy ra đường thẳng $x=x_0$ là tiệm cận đứng.
        \end{itemize}
        Và $\lim\limits_{x\rightarrow +\infty} g(x)=\lim\limits_{x\rightarrow +\infty }\dfrac{\sqrt{x-1}}{f(x)\left(f(x)-9 \right)}=0$ (vì bậc ở mẫu của $y=g(x)$ là $10$ và bậc tử của nó là $\dfrac{1}{2}$). Do vậy đồ thị hàm số $y=g(x)$ có một tiệm cận ngang là đường thẳng $y=0$.\\
        Vậy đồ thị hàm số $y=g(x)$ có bốn tiệm cận ngang và đứng. }
\end{ex}
\begin{ex}%[Thi thử, Chuyên Quang Trung-Bình Phước, 2021,lần 1]%[Trần Hòa, 12EX6]%[2D1G4-3]%
    \immini{Cho hàm số $y=f(x)=ax^3+bx^2+cx+d$, có đồ thị như hình vẽ. Số đường tiệm cận đứng của đồ thị hàm số $y=\dfrac{x^2+x-2}{f^2(x)-f(x)}$ là
        \choice
        {$3$}
        {$2$}
        {\True $4$}
        {$5$}}
    {\begin{tikzpicture}[scale=.5, font=\footnotesize, line join=round, line cap=round, >=stealth]
            \draw[->] (-2.5,0)--(0,0) node[below right]{$O$}--(2,0) node[below]{$x$};
            \draw[->] (0,-.5) --(0,4.5) node[right]{$y$};
            \draw [domain=-2.05:2.05, samples=100] %
            plot (\x, {(\x+2)*(\x-1)^2});
            \draw[fill] (0,0) circle (1pt);
            \foreach \x/\g in {-2/140,-1/-90,1/-90}
            \draw[fill] (\x,0) circle(.5pt)node [shift={(\g:.3)}] {$\x$};
            \foreach \y/\g in {2/0,4/0}
            \draw[fill] (0,\y) circle(.5pt)node [shift={(\g:.3)}] {$\y$};
            \draw[dashed] (-1,0)--(-1,4)--(0,4);
    \end{tikzpicture}}
    \loigiai{
        \begin{itemize}
            \item $x^2+x-2=(x-1)(x+2)$.\\
            \item Dựa vào đồ thị hàm số $y=f(x)$ ta có $f^2(x)-f(x)=0\Leftrightarrow\hoac{&f(x)=0\\&f(x)=1.}$\\
            $f(x)=0\Leftrightarrow x=-2$, $x=1$ (nghiệm kép).\\
            $f(x)=1\Leftrightarrow\hoac{&x=x_1,(x_1\in (-2;-1))\\&x=x_2,(x_2\in (0;1))\\&x=x_3,(x_3>1). }$
            \item Do đó $y=\dfrac{(x-1)(x+2)}{a^2(x+2)(x-1)^2(x-x_1)(x-x_2)(x-x_3)}$.
        \end{itemize}
        Suy ra đồ thị có các đườn tiệm cận đứng $x=1$, $x=x_1$, $x=x_2$, $x=x_3$.
    }
\end{ex}
\begin{ex}%[Đề thi hết học kì 2, Bình Minh, Ninh Bình 2018]%[Nguyễn Tuấn Anh, dự án EX9]%[2D1G4-3]%
    \immini{Cho hàm số bậc ba $f(x)=ax^3+bx^2+cx+d$ có đồ thị như hình vẽ bên dưới. Hỏi đồ thị hàm số $g(x)=\dfrac{(x^2-3x+2)\sqrt{x-1}}{x[f^2(x)-f(x)]}$ có bao nhiêu tiệm cận đứng?
        \choice
        {$5$}
        {$6$}
        {\True $3$}
        {$4$}
    }{
        \begin{tikzpicture}[line width=1.0pt,line join=round,>=stealth,x=1cm,y=1cm,scale=1.0]
            \draw[->,line width = 1pt] (-1,0)--(0,0) node[below right]{$O$}--(4,0) node[below]{$x$};
            \draw[->,line width = 1pt] (0,-1.5) --(0,2.5) node[right]{$y$};
            \foreach \x in {1,2}{
                \draw (\x,0) node[below]{$\x$} circle (1pt);
            }
            \foreach \y in {1}{
                \draw (0,\y) node[left]{$\y$} circle (1pt);
            }
            \clip(-0.8,-1) rectangle (3.8,2.3);
            \draw [line width=1.0pt, thick, domain=-0.5:3.5, samples=100]%,domain=-1.5:3] %
            plot (\x, {(5*(\x)-4)*((\x)-2)^2});
            \draw [dash pattern=on 4pt off 4pt] (1.,0.)-- (1.,1.)-- (0.,1.);
            \draw (1,1) circle (1pt);
        \end{tikzpicture}
    }
    \loigiai{
        Điều kiện $\heva{&x\geq 1\\ &x\ne 0\\ &f^2(x)-f(x)\ne 0}\Leftrightarrow \heva{&x\geq 1\\ &f(x)\ne 0\\ & f(x)\ne 1.}$\\
        Dựa vào đồ thị hàm số $y=f(x)$, ta thấy $f(x)=0$ có hai nghiệm, một nghiệm $x_1<1$ và một nghiệm kép bằng $2$. Do đó ta biểu diễn được $f(x)$ dưới dạng
        $$ f(x)=a(x-x_1)(x-2)^2. $$
        Dựa vào đồ thị hàm số $y=f(x)$, ta thấy phương trình $f(x)=1$ có ba nghiệm $1,x_2, x_3$, với $1<x_2<2<x_3$. Do đó ta biểu diễn được $f(x)-1$ dưới dạng
        $$ f(x)-1=a(x-1)(x-x_2)(x-x_3). $$
        Lúc này điều kiện được viết lại như sau $\heva{&x>1\\ &x\ne x_2, x\ne 2, x\ne x_3.}$\\
        Với điều kiện đó thì $g(x)$ được viết lại là
        $$ g(x)=\dfrac{\sqrt{x-1}}{a^2x(x-x_1)(x-x_2)(x-2)(x-x_3)}. $$
        Ta có
        \begin{align*}
            &\lim\limits_{x\to 1^+}g(x)=\lim\limits_{x\to 1^+}\dfrac{\sqrt{x-1}}{a^2x(x-x_1)(x-x_2)(x-2)(x-x_3)}=0,\\
            & (x=1\mbox{ \textbf{không} là tiệm cận đứng}) \\
            &\lim\limits_{x\to x_2^+}g(x)=\lim\limits_{x\to x_2^+}\dfrac{\sqrt{x-1}}{a^2x(x-x_1)(x-x_2)(x-2)(x-x_3)}=+\infty,\\
            & (x=x_2\mbox{ là tiệm cận đứng}) \\
            &\lim\limits_{x\to 2^+}g(x)=\lim\limits_{x\to 2^+}\dfrac{\sqrt{x-1}}{a^2x(x-x_1)(x-x_2)(x-2)(x-x_3)}=-\infty,\\
            & (x=2\mbox{ là tiệm cận đứng}) \\
            &\lim\limits_{x\to x_3^+}g(x)=\lim\limits_{x\to x_3^+}\dfrac{\sqrt{x-1}}{a^2x(x-x_1)(x-x_2)(x-2)(x-x_3)}=+\infty,\\
            & (x=x_3\mbox{ là tiệm cận đứng}) \\
        \end{align*}
        Vậy đồ thị hàm số $g(x)$ có tất cả $3$ tiệm cận đứng.
    }
\end{ex}
\begin{ex}%[VDC5-Đỗ Đường Hiếu]%[2D1G4-3]%
    \immini{Cho hàm số $f(x)=(x+3)(x+1)^2(x-1)(x-3)$ có đồ thị như hình vẽ. Đồ thị hàm số $g(x)=\dfrac{\sqrt{x-1}}{f^2(x)-9f(x)}$ có bao nhiêu tiệm cận đứng và tiệm cận ngang?
        \choice
        {$3$}
        {\True $4$}
        {$9$}
        {$8$}}
    {\begin{tikzpicture}[xscale=0.8,yscale=0.05, line join=round, line cap=round,font=\footnotesize,>=stealth]
            \draw[->] (-4,0)--(4,0) node[below]{$x$};
            \draw[->] (0,-56)--(0,30) node[left]{$y$};
            \coordinate[label=below left:$O$] (O) at (0,0);
            \draw (-1,0) node[below] { $-1$}(1,0) node[below] { $1$};
            \draw (-3,0) node[below left] { $-3$};
            \draw (3,0) node[below right] { $3$};
            \clip (-3.3,-60) rectangle (3.5,26);
            \draw[smooth,samples=300,domain=-3.5:3.5] plot(\x,{(\x+3)*(\x+1)^2*(\x-1)*(\x-3)});
            \foreach \x in {-3,-1,1,3}
            \draw[shift={(\x,0)},color=black] (0pt,20pt) -- (0pt,-20pt);
            \draw[shift={(0,9)},color=black] (2pt,0pt) -- (-2pt,0pt) node[left] {$9$};
        \end{tikzpicture}
    }
    \loigiai{%GV tổng quát hóa bài toán:
        Cho hàm số đa thức $y=f(x)$ có đồ thị $(C)$. Tìm số đường tiệm cận của đồ thị hàm số $g(x)=\dfrac{\sqrt{ax+b}}{P\left(f(x) \right) }$, trong đó $P\left(f(x) \right)$ là một đa thức của $f(x)$.
        Nếu $a>0$ thì $\lim\limits_{x\to +\infty}g(x)=0$.\\
        Nếu $a<0$ thì $\lim\limits_{x\to -\infty}g(x)=0$.\\
        Do đó đồ thị hàm số $y=g(x)$ luôn có duy nhất một đường tiệm cận ngang là $y=0$.\\
        Gọi $x=x_0$ là một nghiệm của phương trình $P\left(f(x) \right) =0$ thỏa mãn điều kiện $ax+b\ge 0$. Rõ ràng khi đó $\lim\limits_{x\to x_0^+}g(x)=+\infty$ hoặc $\lim\limits_{x\to x_0^+}g(x)=-\infty$.\\
        Bởi vậy, số đường tiệm cận đứng của đồ thị hàm số $y=g(x)$ chính là số nghiệm của phương trình $P\left(f(x) \right) =0$ thỏa mãn điều kiện $ax+b\ge 0$.
        \immini{Ta có $f^2(x)-9f(x)=0\Leftrightarrow \hoac{&f(x)=0\\&f(x)=9.}$\\
            \begin{itemize}
                \item $f(x)=0$ có các nghiệm thuộc $\left[1;+\infty\right)$ là $x=1$ và $x=3$.
                \item Đường thẳng $y=9$ cắt đồ thị hàm số $y=f(x)$ tại duy nhất một điểm có hoành độ thuộc $\left[1;+\infty\right)$ là $x=a>3$.
            \end{itemize}
        }
        {\begin{tikzpicture}[xscale=0.8,yscale=0.05, line join=round, line cap=round,font=\footnotesize,>=stealth]
                \draw[->] (-4,0)--(4,0) node[below]{$x$};
                \draw[->] (0,-56)--(0,30) node[left]{$y$};
                \coordinate[label=below left:$O$] (O) at (0,0);
                \draw (-4,9)--(4,9);
                \draw (-1,0) node[below] { $-1$}(1,0) node[below] { $1$};
                \draw (-3,0) node[below left] { $-3$};
                \draw (3,0) node[below right] { $3$};
                \clip (-3.3,-60) rectangle (3.5,26);
                \draw[smooth,samples=300,domain=-3.5:3.5] plot(\x,{(\x+3)*(\x+1)^2*(\x-1)*(\x-3)});
                \foreach \x in {-3,-1,1,3}
                \draw[shift={(\x,0)},color=black] (0pt,20pt) -- (0pt,-20pt);
                \draw[shift={(0,9)},color=black] (2pt,0pt) -- (-2pt,0pt) node[above left] {$9$};
        \end{tikzpicture}}
        \noindent
        Bởi vậy, hàm số $g(x)=\dfrac{\sqrt{x-1}}{f^2(x)-9f(x)}$ có tập xác định là $\mathscr D=\left[1;3\right) \cup \left(3;a\right) \cup\left( a;+\infty\right)$.\\
        Khi đó ta có
        \begin{itemize}
            \item $\lim\limits_{x\to+\infty}g(x)=0$ nên đồ thị hàm số $y=g(x)$ có một đường tiệm cận ngang là đường thẳng $y=0$.
            \item $\lim\limits_{x\to 1^+}g(x)=\lim\limits_{x\to 1^+}\dfrac{\sqrt{x-1}}{f(x)\left[f(x)-9\right] }=+\infty$;\\
            $\lim\limits_{x\to 3^+}g(x)=\lim\limits_{x\to 3^+}\dfrac{\sqrt{x-1}}{f(x)\left[f(x)-9\right] }=-\infty$;\\
            $\lim\limits_{x\to a^+}g(x)=\lim\limits_{x\to a^+}\dfrac{\sqrt{x-1}}{f(x)\left[f(x)-9\right] }=+\infty$.\\
            Do đó nên đồ thị hàm số $y=g(x)$ có $3$ đường tiệm cận đứng là các đường thẳng $x=1$, $x=3$ và $x=a$.
        \end{itemize}
        Như vậy, đồ thị hàm số $y=g(x)$ có $4$ đường tiệm cận, trong đó có $1$ đường tiệm cận ngang và $3$ đường tiệm cận đứng.
    }
\end{ex}
\begin{ex}%[VDC5-Đỗ Đường Hiếu]%[2D1G4-3]%
    \immini{Cho hàm số bậc ba $y=f(x)$ có đồ thị như hình vẽ bên. Đồ thị hàm số $g(x)=\dfrac{x\sqrt{x+1}}{f(x)\left[f^2(x)-16 \right] }$ có bao nhiêu tiệm cận đứng?
        \choice
        {\True $4$}
        {$5$}
        {$6$}
        {$7$}}
    {\begin{tikzpicture}[scale=0.6,line join=round, line cap=round,font=\footnotesize,>=stealth]
            \draw[->] (-2.5,0)--(4,0) node[below]{$x$};
            \draw[->] (0,-5)--(0,2.5) node[left]{$y$};
            \coordinate[label=below left:$O$] (O) at (0,0);
            \draw[dashed] (-1,0)--(-1,-4)--(0,-4);
            \clip (-2.3,-5) rectangle (3.5,2.5);
            \draw[smooth,samples=300,domain=-3.5:3.5] plot(\x,{-0.5*(\x+2)*(\x-1)*(\x-3)});
            \foreach \x in {-2,-1,1,3}
            \draw[shift={(\x,0)},color=black] (0pt,2pt) -- (0pt,-2pt) node[above] { $\x$};
            \foreach \y in {-4,-3,1}
            \draw[shift={(0,\y)},color=black] (2pt,0pt) -- (-2pt,0pt) node[right] {$\y$};
        \end{tikzpicture}
    }
    \loigiai{
        Xét phương trình $f(x)\left[f^2(x)-16 \right]=0$ \, $(*)$, với điều kiện $x\in\left[-1;+\infty \right) $.\\
        Ta có $f(x)\left[f^2(x)-16 \right]=0\Leftrightarrow \hoac{&f(x)=0\\&f(x)=4\\&f(x)=-4.}$\\
        \begin{itemize}
            \item Phương trình $f(x)=0$ có hai nghiệm $x\in\left[-1;+\infty \right) $ là $x=1$ và $x=3$.
            \item Phương trình $f(x)=4$ có không có nghiệm $x\in\left[-1;+\infty \right) $.
            \item Phương trình $f(x)=-4$ có hai nghiệm $x\in\left[-1;+\infty \right) $ là $-1<x_1<0$ và $x_2>3$.
        \end{itemize}
        Rõ ràng $\lim\limits_{x\to x_0^+}g(x)=+\infty$ hoặc $\lim\limits_{x\to x_0^+}g(x)=-\infty$, trong đó $x=x_0$ là nghiệm thuộc $\left[-1;+\infty \right) $ của phương trình $(*)$. Do đó đường thẳng $x=x_0$ là tiệm cận đứng của đồ thị hàm số $y=g(x)$.\\
        Từ đó suy ra đồ thị hàm số $g(x)=\dfrac{x\sqrt{x+1}}{f(x)\left[f^2(x)-16 \right] }$ có $4$ tiệm cận đứng.
    }
\end{ex}
\begin{ex}%[VDC5-Đỗ Đường Hiếu]%[2D1G4-3]%
    \immini{Cho $y=f(x)$ là hàm số đa thức có đồ thị như hình vẽ bên. Đặt $g(x)=\dfrac{\sqrt{x-1}}{\left[f(x)\right]^2-2f(x)}$ có bao nhiêu đường tiệm cận đứng?
        \choice
        {$5$}
        {$3$}
        {$4$}
        {\True $2$}}
    {\begin{tikzpicture}[scale=0.6,line join=round, line cap=round,font=\footnotesize,>=stealth]
            \draw[->] (-3,0)--(2.5,0) node[below]{$x$};
            \draw[->] (0,-1)--(0,5) node[left]{$y$};
            \coordinate[label=above left:$O$] (O) at (0,0);
            \draw[dashed] (-1,0)--(-1,4)--(0,4);
            \clip (-2.3,-1) rectangle (2.5,4.5);
            \draw[smooth,samples=300,domain=-3.5:3.5] plot(\x,{(\x)^3-3*(\x)+2});
            \foreach \x in {-2,-1,1}
            \draw[shift={(\x,0)},color=black] (0pt,2pt) -- (0pt,-2pt) node[below] { $\x$};
            \foreach \y in {2,4}
            \draw[shift={(0,\y)},color=black] (2pt,0pt) -- (-2pt,0pt) node[right] {$\y$};
        \end{tikzpicture}
    }
    \loigiai{
        Xét phương trình $\left[f(x)\right]^2-2f(x)=0$ \, $(*)$, với điều kiện $x\in\left[1;+\infty \right) $.\\
        Ta có $\left[f(x)\right]^2-2f(x)=0\Leftrightarrow \hoac{&f(x)=0\\&f(x)=2.}$\\
        \begin{itemize}
            \item Phương trình $f(x)=0$ có một nghiệm $x\in\left[1;+\infty \right) $ là $x=1$.
            \item Phương trình $f(x)=2$ có một nghiệm $x\in\left[1;+\infty \right) $ là $x=x_1>1$.
        \end{itemize}
        Rõ ràng $\lim\limits_{x\to x_0^+}g(x)=+\infty$ hoặc $\lim\limits_{x\to x_0^+}g(x)=-\infty$, trong đó $x=x_0$ là nghiệm thuộc $\left[1;+\infty \right) $ của phương trình $(*)$. Do đó đường thẳng $x=x_0$ là tiệm cận đứng của đồ thị hàm số $y=g(x)$.\\
        Từ đó suy ra đồ thị hàm số $g(x)=\dfrac{\sqrt{x-1}}{\left[f(x)\right]^2-2f(x)}$ có $2$ tiệm cận đứng.
    }
\end{ex}
\begin{ex}%[VDC5-NgocDungHo]%[2D1G4-3]%
    \immini
    {
        Cho hàm số $f(x)$ có đồ thị như hình bên. Số đường tiệm cận đứng của đồ thị hàm số $y=\dfrac{(x^2-4)(x^2+2x)}{[f(x)]^2-4f(x)+3}$ là
        \choice
        {$4$}
        {\True $5$}
        {$3$}
        {$2$}
    }
    {\begin{tikzpicture}[>=stealth,scale=0.5, line join=round, line cap=round]
            \def\f[#1]{-0.25*((#1)^4-8*(#1)^2+4)}
            \draw[->] (-4.1,0)--(4,0) node [below]{$x$};
            \draw[->] (0,-2)--(0,4) node [left]{$y$};
            \node at (0,0) [above left]{$O$};
            % \clip;
            \draw[domain=-2.9:2.9,samples=300,thick] plot (\x,{\f[\x]});
            \foreach \x in {-2,2} \filldraw (\x,0) node[below]{\x} circle (2pt);
            %\foreach \x in {-3,3} \filldraw (\x,0) node[below left]{\x} circle (2pt);
            \filldraw (-3,0) node[below left]{$-3$} circle (2pt);
            \filldraw (3,0) node[below right]{$3$} circle (2pt);
            \filldraw (0,1) node[left]{$1$} circle (2pt);
            \filldraw (0,3) node[above left]{$3$} circle (2pt);
            \draw[dashed](-2,0)--(-2,3)--(2,3)--(2,0);
            \draw (3,-1.75) node[right]{$y=f(x)$};
        \end{tikzpicture}
    }
    \loigiai{
        Xét hàm số $y=g(x)=\dfrac{(x^2-4 )(x^2+2x)}{[f(x)]^2-4f(x)+3}$.
        \immini
        {
            Giải phương trình $(x^2-4)(x^2+2x)=0 $\\
            $\Leftrightarrow \hoac{& x^2-4=0 \\ & x^2+2x=0}\Leftrightarrow \hoac{& x=\pm 2 \\ & x=0.}$\\
            Giải phương trình $[f(x)]^2-4f(x)+3=0$\\
            $ \Leftrightarrow \hoac{& f(x)=1 \\ & f(x)=3} \Leftrightarrow \hoac{& x = \pm 2 \\ & x=a\\&x=b\\&x=c\\&x=d.}$\\ với $-3<a<-2<b<c<2<d<3$.\\
        }
        {\begin{tikzpicture}[>=stealth,scale=0.8, line join=round, line cap=round]
                \def\f[#1]{-0.25*((#1)^4-8*(#1)^2+4)}
                \def\g[#1]{1}
                \def\h[#1]{3}
                \draw[->] (-4.1,0)--(4,0) node [below]{$x$};
                \draw[->] (0,-2)--(0,4) node [left]{$y$};
                \node at (0,0) [above left]{$O$};
                % \clip;
                \draw[domain=-2.9:2.9,samples=300,thick] plot (\x,{\f[\x]});
                \draw[domain=-4:4,samples=300,thick] plot (\x,{\g[\x]});
                \draw[domain=-4:4,samples=300,thick] plot (\x,{\h[\x]});
                \foreach \x in {-3,-2,2,3} \filldraw (\x,0) node[below]{\x} circle (2pt);
                % \filldraw (-3,0) node[above left]{$-3$} circle (2pt);
                % \filldraw (3,0) node[above ]{$3$} circle (2pt);
                \filldraw (0,1) node[below left]{$1$} circle (2pt);
                \filldraw (0,-1) node[below left]{$-1$} circle (2pt);
                \filldraw (0,3) node[above left]{$3$} circle (2pt);
                \draw[dashed](-2,0)--(-2,3) (2,3)--(2,0) (2.61,0)node[below]{$d$}--(2.61,1) (-2.61,0)node[below]{$a$}--(-2.61,1) (1.08,0)node[below]{$c$}--(1.08,1)(-1.08,0)node[below]{$b$}--(-1.08,1);
                \draw (3,2.75) node[right]{$y=f(x)$};
            \end{tikzpicture}
        }
        Trong điều kiện xác định của hàm số $y=g(x)$ ta có thể viết $$y=g(x)=\dfrac{x(x-2)(x+2)^2}{(x-a)(x-b)(x-c)(x-d) (x-2)^2(x+2)^2}=\dfrac{x}{(x-a)(x-b)(x-c)(x-d)(x-2)}$$
        Vậy số tiệm cận đứng của đồ thị hàm số $y=g(x)$ bằng $5$.
    }
\end{ex}
\Closesolutionfile{ans}
%%Bài 4. Đồ thị
% \section{KHẢO SÁT SỰ BIẾN THIÊN VÀ VẼ ĐỒ THỊ HÀM SỐ}
\subsection{LÝ THUYẾT CẦN NHỚ}
\subsubsection{Sơ đồ khảo sát hàm số y= f(x)}
\begin{tcolorbox}[colframe=cyan,colback=red!3!white,boxrule=0.5mm]
		\begin{itemize}
		\item[\iconCH] \indamm{Bước 1.} Tìm tập xác định của hàm số.
		\item [\iconCH] \indamm{Bước 2.} Khảo sát sự biến thiên của hàm số
		\begin{itemize}
			\item Tính đạo hàm $y'$. Tìm các điểm mà tại đó $y'$ bằng $0$ hoặc đạo hàm không tồn tại.
			\item Tìm các giới hạn tại vô cực, giới hạn vô cực và tìm tiệm cận của đồ thị hàm số.
			\item Lập bảng biến thiên; xác định chiều biến thiên và cực trị của hàm số.
		\end{itemize}
		\item [\iconCH] \indamm{Bước 3.} Cho thêm điểm và vẽ đồ thị của hàm số dựa vào bảng biến thiên.
	\end{itemize}
\end{tcolorbox}
\subsubsection{Hàm số bậc ba $\mathbf{y=ax^3+bx^2+cx+d}$}
\	\begin{minipage}[b]{10cm}
		\begin{enumerate}[\iconCH]
			\item \indamm{TH1.} $y'=0$ có hai nghiệm phân biệt $x_1$ và $x_2$. Khi đó, hàm số có hai điểm cực trị $x=x_1$ và $x=x_2$.\\
			\begin{tikzpicture}[smooth,samples=300,line width=0.6pt,scale=0.8,>=stealth,font=\footnotesize]
				\draw[->] (-2.5,0)--(2.5,0) node[below]{$x$};
				\draw[->] (0,-1)--(0,2) node[right]{$y$};
				\draw (0,0) node[below left]{$O$};
				\draw[blue,line width=1pt,domain=-2.1:2.1] plot(\x,{0.4*((\x)^3-3*(\x)+1)});
				\draw[fill=black] (0,0.4) circle(2pt) (-1,1.2) circle(2pt) (1,-0.4) circle(2pt);
				\draw[dashed] (1,0)node[above]{\footnotesize$x_2$}--(1,-0.4)--(0,-0.4) (-1,0)node[below]{\footnotesize$x_1$}--(-1,1.2)--(0,1.2);
				\node[right] at (0,0.6) {\footnotesize $I$};
				\node[right] at (-2,2) {\tiny\fbox{$a>0$}};
			\end{tikzpicture}
			\hspace{0.3cm}
			\begin{tikzpicture}[smooth,samples=300,line width=0.6pt,scale=0.8,>=stealth,font=\footnotesize]
				\draw[->] (-2.5,0)--(2.5,0) node[below]{$x$};
				\draw[->] (0,-1)--(0,2) node[right]{$y$};
				\draw (0,0) node[below right]{$O$};
				\draw[blue,line width=1pt,domain=-2.1:2.1] plot(\x,{0.4*(-(\x)^3+3*(\x)+1)});
				\draw[fill=black] (0,0.4) circle(2pt) (1,1.2) circle(2pt) (-1,-0.4) circle(2pt);
				\draw[dashed] (-1,0)node[above]{\footnotesize$x_1$}--(-1,-0.4)--(0,-0.4) (1,0)node[below]{\footnotesize$x_2$}--(1,1.2)--(0,1.2);
				\node[left] at (0,0.6) {\footnotesize$I$};
				\node[right] at (-2,2) {\tiny\fbox{$a<0$}};
			\end{tikzpicture}
			\item \indamm{TH2.} $y'=0$ có nghiệm kép $x_0$. Khi đó, hàm số không có cực trị.\\
			\begin{tikzpicture}[smooth,samples=300,line width=0.6pt,scale=0.8,>=stealth,font=\footnotesize]
				\draw[->] (-2,0)--(2.5,0) node[below]{$x$};
				\draw[->] (0,-1)--(0,2) node[right]{$y$};
				\draw (0,0) node[below right]{$O$};
				\draw[blue,line width=1pt,domain=-0.7:1.6] plot(\x,{(\x-0.5)^3+0.7});
				\draw[fill=black] (0.5,0.7) circle(2pt);
				\node[above] at (0.5,0.7) {\footnotesize$I$};
				\node[right] at (-2,2) {\tiny\fbox{$a>0$}};
			\end{tikzpicture}
			\hspace{0.5cm}
			\begin{tikzpicture}[smooth,samples=300,line width=0.6pt,scale=0.8,>=stealth,font=\footnotesize]
				\draw[->] (-2,0)--(2.5,0) node[below]{$x$};
				\draw[->] (0,-1)--(0,2) node[right]{$y$};
				\draw (0,0) node[below right]{$O$};
				\draw[blue,line width=1pt,domain=-0.6:1.6] plot(\x,{-((\x-0.5)^3-0.5)});
				\draw[fill=black] (0.5,0.5) circle(2pt);
				\node[above] at (0.5,0.5) {\footnotesize$I$};
				\node[right] at (-2,2) {\tiny\fbox{$a<0$}};
			\end{tikzpicture}
			\item \indamm{TH3.} $y'=0$ vô nghiệm. Khi đó, hàm số không có cực trị.\\
			\begin{tikzpicture}[smooth,samples=300,line width=0.6pt,scale=0.8,>=stealth,font=\footnotesize]
				\draw[->] (-2,0)--(2.5,0) node[below]{$x$};
				\draw[->] (0,-1)--(0,2.5) node[right]{$y$};
				\draw (0,0) node[below right]{$O$};
				\draw[blue,line width=1pt,domain=-0.5:1.5] plot(\x,{((\x-0.6)^3+0.7*(\x)+0.7)});
				\draw[fill=black] (0.6,1.12) circle(2pt);
				\node[below right] at (0.5,1.12) {\footnotesize$I$};
				\node[right] at (-2,2) {\tiny\fbox{$a>0$}};
			\end{tikzpicture}
			\hspace{0.5cm}
			\begin{tikzpicture}[smooth,samples=300,line width=0.6pt,scale=0.8,>=stealth,font=\footnotesize]
				\draw[->] (-2,0)--(2.5,0) node[below]{$x$};
				\draw[->] (0,-1)--(0,2.5) node[right]{$y$};
				\draw (0,0) node[below left]{$O$};
				\draw[blue,line width=1pt,domain=-0.5:1.5] plot(\x,{-(\x-0.6)^3-0.7*(\x)+0.7)});
				\draw[fill=black] (0.6,0.28) circle(1.5pt);
				\node[right] at (-2,2) {\tiny\fbox{$a<0$}};
				\node[above] at (0.6,0.28) {\footnotesize$I$};
			\end{tikzpicture}
		\end{enumerate}
		\vspace{0.4cm}
	\end{minipage}\hspace{0.5cm}
	\begin{minipage}[b]{6.5cm}
		\begin{khung4}{GHI NHỚ}
			\ding{172} Hàm số không có điểm cực trị
			$$b^2-3ac\le 0 \text{ hoặc } \heva{&a=0 \\&b=0.}$$
			\ding{173} Hàm số có hai điểm cực trị
			$$\heva{&a \ne 0\\&b^2-3ac >0.}$$
			\ding{174} Liên hệ tổng tích hai nghiệm
			$$\heva{&x_1+x_2=-\dfrac{2b}{3a}\\&x_1x_2=\dfrac{c}{3a}}$$
			\ding{175} Tọa độ tâm đối xứng của đồ thị, nó chính là trung điểm của đoạn nối 2 điểm cực trị. Hoành độ tâm đối xứng là nghiệm phương trình $y''=0 \Leftrightarrow x=-\dfrac{b}{3a}$.
			\ding{176} Tiếp tuyến tại tâm đối xứng sẽ có hệ số góc nhỏ nhất nếu $a>0$ và lớn nhất nếu $a<0$.
		\end{khung4}
		\vspace{0.1cm}
	\end{minipage}
\newpage
\subsubsection{Hàm số $\mathbf{y = \dfrac{{ax + b}}{{cx + d}}\left( {c \ne 0,ad - bc \ne 0} \right)}$}
\begin{minipage}[b]{10cm}
	\begin{enumerate}[\iconCH]
		\item Tập xác định $D=\mathbb{R}\backslash \left\{-\dfrac{d}{c}\right\}$; Đạo hàm $y'=\dfrac{ad-cb}{(cx+d)^2}$.
		\item Đồ thị nhận giao điểm của hai đường tiệm cận làm tâm đối xứng.
		\item Hình dạng đồ thị:\\
		\begin{tikzpicture}[smooth,samples=300,line width=0.6pt,>=stealth, scale=0.45]
			\draw[->] (-5,0)--(3.5,0) node[below]{$x$};
			\draw[->] (0,-1.6)--(0,5.7) node[right]{$y$};
			\draw (0,0) node[below right]{$O$};
			\node at (-3,5.3) {\tiny\fbox{$y'>0$}};
			\clip (-5,-1.5) rectangle (3,5.5);
			\draw[dashed] (-1,-2)--(-1,5.5) (-5,2)--(3,2);
			\draw[blue,line width=1pt,domain=-5:-1.1] plot(\x,{(2*(\x)+1)/((\x)+1)});
			\draw[blue,line width=1pt,domain=-0.9:3] plot(\x,{(2*(\x)+1)/((\x)+1)});
			\draw[fill=black] (-1,2) circle(1.5pt) circle(1.5pt) (-1,0) circle(1pt) (0,2) circle(1pt);
			\node[left] at (-1,1.5) {\footnotesize $I$};
			\node[below left] at (-1,0) {\tiny $-\dfrac{d}{c}$};
			\node[above right] at (0,2) {\tiny $\dfrac{a}{c}$};
		\end{tikzpicture}
		\hspace{0.5cm}
		\begin{tikzpicture}[smooth,samples=300,line width=0.6pt,>=stealth, scale=0.45]
			\draw[->] (-3,0)--(5.5,0) node[below]{$x$};
			\draw[->] (0,-1.6)--(0,5.7) node[left]{$y$};
			\draw (0,0) node[below left]{$O$};
			\node at (3,5.3) {\tiny \fbox{$y'<0$}};
			\clip (-3,-1.5) rectangle (5,5.5);
			\draw[dashed] (1,-2)--(1,5.5) (-3,2)--(5,2);
			\draw[blue,line width=1pt,domain=-3:0.9] plot(\x,{(2*(\x)-1)/((\x)-1)});
			\draw[blue,line width=1pt,domain=1.1:5] plot(\x,{(2*(\x)-1)/((\x)-1)});
			\draw[fill=black] (1,2) circle(1.5pt) (1,0) circle(1pt) (0,2) circle(1pt);
			\node[right] at (1,1.5) {\footnotesize $I$};
			\node[below right] at (1,0) {\tiny $-\dfrac{d}{c}$};
			\node[above left] at (0,2) {\tiny $\dfrac{a}{c}$};
		\end{tikzpicture}
	\end{enumerate}
	%	\vspace{1.5cm}
\end{minipage}\hspace{0.5cm}
\begin{minipage}[b]{6.5cm}
	\begin{khung4}{GHI NHỚ}
		\ding{172} Tiệm cận đứng $x=-\dfrac{d}{c}$.\\
		\ding{173} Tiệm cận ngang $y=\dfrac{a}{c}$.\\
		\ding{174} Giao với $Ox$: $y=0 \Rightarrow x=-\dfrac{b}{a}$.\\
		\ding{175} Giao với $Oy$: $x=0 \Rightarrow y=\dfrac{b}{d}$.\\
	\end{khung4}
\end{minipage}
\subsubsection{Hàm số $\mathbf{y = \dfrac{{a{x^2} + bx + c}}{{mx + n}}\left( {a \ne 0,m \ne 0} \right)}$ (đa thức tử không chia hết cho đa thức mẫu)}
\begin{enumerate}[\iconCH]
	\item Tập xác định $D=\mathbb{R}\backslash \left\{-\dfrac{n}{m}\right\}$; Đạo hàm $y'=\dfrac{am\cdot x^2+2an\cdot x + b.n - m.c}{(mx+n)^2}$.
	\item Hàm số $2$ điểm cực trị khi $y'=0$ có $2$ nghiệm phân biệt; Hàm số không có cực trị khi $y'=0$ vô nghiệm.
	\item Đồ thị nhận giao điểm của tiệm cận đứng và tiệm cận xiên làm tâm đối xứng.
	\item Hình dạng đồ thị:\\
	\begin{tikzpicture}[line cap=butt,line join=miter,>=stealth,scale=0.4,font=\footnotesize]
		\tikzset{declare function={xmin=-5.5;xmax=3.5;ymin=-4.6;ymax=4.6;},
			smooth,samples=450}
		\draw[->] (xmin,-0.5)--(xmax,-0.5) node[above]{$ x $};
		\draw[->] (0,ymin)--(0,ymax) node[right]{$ y $};
		\fill (0,-0.5) node[above right]{$ O $};
		\path (current bounding box.south) node[below, black]{\tiny\fbox{$a>0$, $y'=0$ có $2$ nghiệm phân biệt}};
		\clip (xmin,ymin) rectangle (xmax,ymax);
		\def\f(#1){((#1)^2+2*(#1)+2)/((#1)+1)} % Hàm số
		\def\q(#1){((#1)+1)} % Tiệm cận xiên	
		\draw[blue,thick,samples=250] plot[domain=xmin:-1.1] (\x,{\f(\x)});	
		\draw[blue,thick,samples=250] plot[domain=-0.9:xmax] (\x,{\f(\x)});
		%--------- Tiệm cận
		\draw[dashed] plot [domain=xmin:xmax] (\x,{\q(\x)}) ;
		\draw[dashed] (-1,ymin)--(-1,ymax);
	\end{tikzpicture}	
	\hspace{.25cm}
	\begin{tikzpicture}[line cap=butt,line join=miter,>=stealth,scale=0.4,font=\footnotesize]
		\tikzset{declare function={xmin=-5.5;xmax=3.5;ymin=-4.6;ymax=4.6;},
			smooth,samples=450}
		\draw[->] (xmin,-0.5)--(xmax,-0.5) node[above]{$ x $};
		\draw[->] (0,ymin)--(0,ymax) node[right]{$ y $};
		\fill (0,-0.5) node[above right]{$ O $};
		\path (current bounding box.south) node[below, black]{\tiny\fbox{$a<0$, $y'=0$ có $2$ nghiệm phân biệt}};
		\clip (xmin,ymin) rectangle (xmax,ymax);
		\def\f(#1){(-(#1)^2-2*(#1)-2)/((#1)+1)} % Hàm số
		\def\q(#1){(-(#1)-1)} % Tiệm cận xiên	
		\draw[blue,thick,samples=250] plot[domain=xmin:-1.1] (\x,{\f(\x)});	
		\draw[blue,thick,samples=250] plot[domain=-0.9:xmax] (\x,{\f(\x)});
		%--------- Tiệm cận
		\draw[dashed] plot [domain=xmin:xmax] (\x,{\q(\x)}) ;
		\draw[dashed] (-1,ymin)--(-1,ymax);
	\end{tikzpicture}	
	\hspace{.25cm}
	\begin{tikzpicture}[line cap=butt,line join=miter,>=stealth,scale=0.4,font=\footnotesize]
		\tikzset{declare function={xmin=-5.5;xmax=3.5;ymin=-4.6;ymax=4.6;},
			smooth,samples=450}
		\draw[->] (xmin,-0.5)--(xmax,-0.5) node[above]{$ x $};
		\draw[->] (0,ymin)--(0,ymax) node[right]{$ y $};
		\fill (0,-0.5) node[above right]{$ O $};
		\path (current bounding box.south) node[below, black]{\tiny\fbox{$a>0$, $y'=0$ vô nghiệm}};
		\clip (xmin,ymin) rectangle (xmax,ymax);
		\def\f(#1){((#1)^2+2*(#1))/((#1)+1)} % Hàm số
		\def\q(#1){((#1)+1)} % Tiệm cận xiên	
		\draw[blue,thick,samples=250] plot[domain=xmin:-1.1] (\x,{\f(\x)});	
		\draw[blue,thick,samples=250] plot[domain=-0.9:xmax] (\x,{\f(\x)});
		%--------- Tiệm cận
		\draw[dashed] plot [domain=xmin:xmax] (\x,{\q(\x)}) ;
		\draw[dashed] (-1,ymin)--(-1,ymax);
	\end{tikzpicture}	
	\hspace{.25cm}
	\begin{tikzpicture}[line cap=butt,line join=miter,>=stealth,scale=0.4,font=\footnotesize]
		\tikzset{declare function={xmin=-5.5;xmax=3.5;ymin=-4.6;ymax=4.6;},
			smooth,samples=450}
		\draw[->] (xmin,-0.5)--(xmax,-0.5) node[above]{$ x $};
		\draw[->] (0,ymin)--(0,ymax) node[right]{$ y $};
		\fill (0,-0.5) node[above right]{$ O $};
		\path (current bounding box.south) node[below, black]{\tiny\fbox{$a<0$, $y'=0$ vô nghiệm}};
		\clip (xmin,ymin) rectangle (xmax,ymax);
		\def\f(#1){(-(#1)^2-2*(#1))/((#1)+1)} % Hàm số
		\def\q(#1){(-(#1)-1)} % Tiệm cận xiên	
		\draw[blue,thick,samples=250] plot[domain=xmin:-1.1] (\x,{\f(\x)});	
		\draw[blue,thick,samples=250] plot[domain=-0.9:xmax] (\x,{\f(\x)});
		%--------- Tiệm cận
		\draw[dashed] plot [domain=xmin:xmax] (\x,{\q(\x)}) ;
		\draw[dashed] (-1,ymin)--(-1,ymax);
	\end{tikzpicture}
\end{enumerate}
\subsection{PHÂN LOẠI VÀ PHƯƠNG PHÁP GIẢI TOÁN}
\begin{dang}{Khảo sát và vẽ đồ thị hàm số bậc ba}
	Ta khảo sát theo sơ đồ đã nhắc đến ở phần lý thuyết.
\end{dang}
\boxmini{BÀI TẬP TỰ LUẬN}
\begin{vd}
	Khảo sát sự biến thiên và vẽ đồ thị các hàm số sau:
	\begin{tasks}(2)
		\task $y=x^3-3x^2+1$;
		\task $y =-2{x^3}-3{x^2}+1$;
		\task $y = {x^3}+3{x^2}+3x+2$;
		\task $y=x^3-3x^2+4x-2$.
	\end{tasks}
\loigiai{
\begin{enumerate}[a)]
	\item Tập xác định $\mathbb{R}$.\\
	Sự biến thiên:
	\begin{itemize}
		\item [$\bullet$] $y'=3x^2-6x$; $y'=0\Leftrightarrow \hoac{&x=0\\&x=2.}$.
		\item [$\bullet$]  Giới hạn: $\lim\limits_{x\to -\infty}y=-\infty$; $\lim\limits_{x\to +\infty}y=+\infty$.
		\item [$\bullet$] \immini{Bảng biến thiên như hình bên:\\
		Suy ra hàm số đồng biến trên các khoảng $(-\infty;0)$ và $(2;+\infty)$; nghịch biến trên $(0;2)$.\\
	Hàm số đạt cực đại tại $x=0; y_{\text{CĐ}}=1$; hàm số đạt cực tiểu tại $x=2; y_{\text{CT}}=-3$.}
	{\hspace{1cm}
		\begin{tikzpicture}
			\tkzTabInit[lgt=1.1,espcl=2,nocadre=True]{$x$/0.6,$y'$/0.6,$y$/2}{$-\infty$,$0$,$2$,$+\infty$}
			\tkzTabLine{,+,z,-,z,+,}
			\tkzTabVar{-/$-\infty$ , +/$1$,-/$-3$, +/$+\infty$}%
		\end{tikzpicture}}
			\end{itemize}
		Đồ thị:
			\immini{
				\begin{itemize}
					\item [$\bullet$] Đồ thị đi qua các điểm $(2;-3)$, $(-1;-3)$, $(3;1)$
					\item [$\bullet$] Đồ thị nhận điểm $I(1;-1)$ làm tâm đối xứng.
				\end{itemize}
				}
			{
				\begin{tikzpicture}[smooth,samples=300,scale=0.8,>=stealth]
					\draw[->] (-2,0)--(4,0) node[below]{$x$};
					\draw[->] (0,-3.9)--(0,2) node[right]{$y$};
					\draw (0,0) node[below left]{$O$};
					\draw[domain=-1.1:3.1] plot(\x,{(\x)^3-3*(\x)^2+1});
					\draw[fill=black] (-1,-3) circle(1.5pt) (0,1) circle(1pt) (2,-3) circle(1pt) (3,1) circle(1pt);
					\draw[dashed] (-1,0)--(-1,-3)--(0,-3)node[below left]{$-3$}--(2,-3)--(2,0)node[above]{$2$}
					(1,0)node[above]{$1$}--(1,-1)--(0,-1)node[left]{$-1$}
					(3,0)node[below]{$3$}--(3,1)--(0,1)node[left]{$1$}
					;
				\end{tikzpicture}
			}
	\item Tập xác định: $\mathbb{R}$.\\
	Sự biến thiên:
	\begin{itemize}
		\item [$\bullet$] $y' = - 6{x^2} - 6x;\,\,y' = 0 \Leftrightarrow x = 0$ hoặc $x = - 1$.
		\item [$\bullet$]  Giới hạn: $\lim\limits_{x\to -\infty}y=+\infty$; $\lim\limits_{x\to +\infty}y=-\infty$.
		\item [$\bullet$] \immini{Bảng biến thiên như hình bên:\\
			Suy ra hàm số nghịch biến trên các khoảng $(-\infty;-1)$ và $(0;+\infty)$; đồng biến trên $(-1;0)$.\\
			Hàm số đạt cực đại tại $x=0; y_{\text{CĐ}}=1$; hàm số đạt cực tiểu tại $x=-1; y_{\text{CT}}=0$.}
		{\hspace{1cm}
			\begin{tikzpicture}[scale=1, font=\footnotesize, line join=round, line cap=round, >=stealth]
				\tkzTabInit[nocadre=false,lgt=1.2,espcl=1.6,deltacl=0.6]
				{$x$ /0.6,$f'(x)$ /0.6,$f(x)$ /1.5}
				{$-\infty$,$-1$,$0$,$+\infty$}
				\tkzTabLine{,-,0,+,0,-,}
				\tkzTabVar{+/$+\infty$,-/$0$,+/$1$,-/$-\infty$}
		\end{tikzpicture}}
	\end{itemize}
Đồ thị:
	\immini{
		\begin{itemize}
			\item [$\bullet$] Đồ thị qua các điểm $(1;-4)$, $(-2;5)$.
			\item [$\bullet$] Đồ thị của hàm số có tâm đối xứng là điểm $I\left({- \dfrac{1}{2};\dfrac{1}{2}} \right)$
		\end{itemize}
		}{	
		\begin{tikzpicture}[scale=1.0,>=stealth, font=\footnotesize, line join=round, line cap=round]
			\def\xmin{-2} \def\xmax{2}
			\def\ymin{-2} \def\ymax{2}
			\draw[->] (\xmin,0)--(\xmax,0) node [below]{$x$};
			\draw[->] (0,\ymin)--(0,\ymax) node [left]{$y$};
			\fill (0,0) circle (1pt) node[shift={(-135:2.5mm)}]{$O$};
			\clip (\xmin+0.1,\ymin+0.1) rectangle (\xmax-0.1,\ymax-0.1);
			\draw[smooth,red,samples=300,domain=(\xmin:3.01)] plot(\x,{-2*(\x)^3-3*(\x)^2+1});	
			\foreach \x in {\xmin,...,\xmax}
			\draw (\x,-0.05)--(\x,0.05);
			\foreach \y in {\ymin,...,\ymax}
			\draw (-0.05,\y)--(0.05,\y);
			\node at (-1,0)[below]{$ -1 $};
			\node at (1,0)[below]{$ 1 $};	
			\node at (0,1)[shift={(135:1.5mm)}]{$ 1 $};	
	\end{tikzpicture}}
	\item Tập xác định $\mathbb{R}$.\\
	Sự biến thiên:
	\begin{itemize}
		\item [$\bullet$] $y' = 3{x^2} + 6x + 3;\,\,y' = 0 \Leftrightarrow x = - 1$.
		\item [$\bullet$] Giới hạn: $\lim\limits_{x\to -\infty}y=-\infty$; $\lim\limits_{x\to +\infty}y=+\infty$.
		\item [$\bullet$] \immini{Bảng biến thiên như hình bên:\\
			Suy ra hàm số đồng biến trên $\mathbb{R}$.\\
			Hàm số không có cực trị.}
		{\hspace{1cm}
			\begin{tikzpicture}[>=stealth]
				\tkzTabInit[nocadre=false,lgt=1,espcl=2.5,deltacl=0.5]{$x$/.6 ,$y'$/.6,$y$/1.5}
				{$-\infty$ , $-1$ , $+\infty$}
				\tkzTabLine{ , + , $0$ , + , }
				\tkzTabVar{-/$-\infty$ , R , +/$+\infty$}
				\tkzTabIma{1}{3}{2}{$1$}
		\end{tikzpicture}}
	\end{itemize}
	Đồ thị:
		\immini{	
	Đồ thị của hàm số có tâm đối xứng là điểm $I\left( {- 1;1} \right)$
	}{	
		\begin{tikzpicture}[scale=0.7,>=stealth, font=\footnotesize, line join=round, line cap=round]
			\def\xmin{-3} \def\xmax{2}
			\def\ymin{-2} \def\ymax{4}
			\draw[->] (\xmin,0)--(\xmax,0) node [below]{$x$};
			\draw[->] (0,\ymin)--(0,\ymax) node [left]{$y$};
			\fill (0,0) circle (1pt) node[shift={(-135:2.5mm)}]{$O$};
			\clip (\xmin+0.1,\ymin+0.1) rectangle (\xmax-0.1,\ymax-0.1);
			\draw[smooth,red,samples=300,domain=(\xmin:3.01)] plot(\x,{(\x)^3+3*(\x)^2+3*(\x)+2});	
			\foreach \x in {\xmin,...,\xmax}
			\draw (\x,-0.05)--(\x,0.05);
			\foreach \y in {\ymin,...,\ymax}
			\draw (-0.05,\y)--(0.05,\y);	
			\node at (1,0)[below]{$ 1 $};
			\node at (-1,1)[shift={(70:2mm)}]{$ I $};
			\draw[dashed]
			(-1,0)node[below]{$ -1 $}|-(0,1)node[right]{$1$}	
			;	
		\end{tikzpicture}}				
	\item Tập xác định: $\mathbb{R}$.\\
	Sự biến thiên
	\begin{itemize}
		\item [$\bullet$] $y'=3x^2-6x+4>0$ với $\forall x\in\mathbb{R}$.
		\item [$\bullet$] Giới hạn: $\lim\limits_{x\to -\infty}y=-\infty$; $\lim\limits_{x\to +\infty}y=+\infty$
		\item [$\bullet$] \immini{Bảng biến thiên như hình bên:\\
			Suy ra hàm số đồng biến trên $\mathbb{R}$.\\
			Hàm số không có cực trị.}
		{\hspace{1cm}
			\begin{tikzpicture}
				\tkzTabInit[lgt=1.1,espcl=4]{$x$/0.6,$y'$/0.6,$y$/2}{$-\infty$,$+\infty$}
				\tkzTabLine{,+,}
				\tkzTabVar{-/$-\infty$ ,+/$+\infty$}
		\end{tikzpicture}}
	\end{itemize}
Đồ thị\\
\immini{
	\begin{itemize}
		\item [$\bullet$] Đồ thị đi qua $(2;2)$, $(0;-2)$, $(1;0)$.
		\item [$\bullet$] Đồ thị nhận $I(1;0)$ làm tâm đối xứng.
	\end{itemize}
}
{
	\begin{tikzpicture}[line cap=round,line join=round,x=1cm,y=1cm]
		\draw[->](-3.08,0)--(4.06,0);
		\foreach \x in {-1,2,3}
		\draw[shift={(\x,0)},color=black] (0pt,2pt)--(0pt,-2pt) node[below]{$\x$};
		\draw[->,color=black](0,-4.06)--(0,2.98);
		\foreach \y in {-2,-1,1,2}
		\draw[shift={(0,\y)},color=black](2pt,0pt)--(-2pt,0pt) node[left]{\normalsize $\y$};
		\draw[color=black](3.8,.2)node[right]{$x$};
		\draw[color=black](.2,3)node[right]{$y$};
		\draw[color=black](0pt,-8pt)node[right]{\normalsize $O$};
		\clip(-3.08,-4.06) rectangle (4.06,2.98);
		%Vẽ đồ thị
		\draw[smooth,samples=100,domain=-4:4]plot(\x,{(\x)^3-3*(\x)^2+4*(\x)-2});
		%Vẽ râu ria
		\draw[dashed](2,0)--(2,2)--(0,2);
		\node[below right] at (1,0){$I$};
		\node[above] at (1,0) {$1$};
	\end{tikzpicture}
}
\end{enumerate}
}
\end{vd}
\dongcham{45}
\boxmini{BÀI TẬP TRẮC NGHIỆM}
\ind{PHẦN I.} \inden{Câu trắc nghiệm nhiều phương án lựa chọn. Mỗi câu hỏi học sinh chỉ chọn một phương án.}\\
\setcounter{ex}{0}
\Opensolutionfile{ans}[ans/2D1-B4-d1-1]
\begin{ex}
	\immini[thm]{Bảng biến thiên ở hình bên là của một trong bốn hàm số sau đây. Hỏi đó là hàm số nào?
		\choice
		{$y=-x^3-2x^2+5$}
		{\True $y=x^3-3x^2+5$}
		{$y=-x^3-3x+5$}
		{$y=x^3+3x^2+5$}}{
		\begin{tikzpicture}
			\tkzTabInit[nocadre=false, lgt=1.2, espcl=1.6]{$x$ /0.6,$f'(x)$ /0.6,$f(x)$ /1.5}{$-\infty$,$0$,$2$,$+\infty$}
			\tkzTabLine{,+,$0$,-,$0$,+,}
			\tkzTabVar{-/ $-\infty$/, +/$5$ , -/$1$  , +/$+\infty$/}
	\end{tikzpicture}}
\end{ex} \dongcham{1}

\begin{ex}
	\immini[thm]{Bảng biến thiên ở hình bên là của một trong bốn hàm số sau đây. Hỏi đó là hàm số nào?
	\choice
	{$ y=-x^3+3x^2 $}
	{$ y=x^3-3x^2-1$}
	{$ y=x^4+2x^2+1 $}
	{\True$ y=-x^3+3x^2+1 $}}{
\begin{tikzpicture}
	\tkzTabInit[nocadre=false,lgt=1.2,espcl=1.6,deltacl=0.6]
	{$x$/0.6, $y'$/0.6, $y$/1.5}
	{$-\infty$,$0$,$2$,$+\infty$}
	\tkzTabLine{,-,z,+,z,-,}
	\tkzTabVar{+/$+\infty$ ,-/ $1$ ,+/$5$, -/$-\infty$}
\end{tikzpicture}}
	\loigiai{
		Ta thấy đây là hàm số bậc ba và $\displaystyle\lim\limits_{x\rightarrow-\infty}=-\infty$ nên $a<0$.\\
		Ta có $f(0)=1$ nên hàm số cần tìm là $y=-x^3+3x^2+1$.
	}
\end{ex} \dongcham{1}

\begin{ex}
	\immini[thm]{Bảng biến thiên ở hình bên là của một trong bốn hàm số sau đây. Hỏi đó là hàm số nào?
		\haicot
		{$y=x^3-3x^2+x+3$}
		{$y=x^3-3x+4$}
		{\True $y=x^3-3x^2+3x+1$}
		{$y=x^3+3x^2+5$}}{
		\begin{tikzpicture}
			\tkzTabInit[lgt=1,espcl=2.5]
			{$x$/0.6,$y'$/0.6,$y$/1.5}
			{$-\infty$,$1$,$+\infty$}
			\tkzTabLine{,+,$0$,+,}
			\tkzTabVar{-/$-\infty$,R,+/$+\infty$}
			\tkzTabIma[draw]{1}{3}{2}{$2$}
	\end{tikzpicture}}
\end{ex} \dongcham{1}

\begin{ex}%[2D1B5-1]
	\immini[thm]{Đường cong bên là đồ thị của một trong bốn hàm số đã cho sau đây. Hỏi đó là hàm số nào?
		\choice
		{$y=-x^3+x^2-2$}
		{\True $y=x^3+3x^2-2$}
		{$y=x^3-3x+2$}
		{$y=x^2-3x-2$}
	}{
		\begin{tikzpicture}[scale=0.55, font=\footnotesize,line join=round, line cap=round,>=stealth]
			\draw[->] (-3.7,0.) -- (2.5,0.) node[below]{$x$};
			\draw[->] (0,-2.5) -- (0,2.5) node[right]{$y$};
			\fill (0,0) node[above left]{$O$};
			\fill (0,-2) circle(2pt) node[below left]{$-2$};
			\draw[line width=1pt,smooth,samples=300,domain=-2.9:1] plot(\x,{(\x+2)^3-3*(\x+2)^2+2});
		\end{tikzpicture}
	}
	\loigiai{
		Dựa vào hình dáng đồ thị, ta thấy đây là đồ thị của hàm số bậc ba $y=ax^3+bx^2+cx+d$ với $a>0$ nên loại các hàm $y=x^4+x^2-2$, $y=-x^2-3x-2$. Mặt khác, đồ thị đi qua điểm $(0;-2)$ nên loại hàm $y=x^3-3x+2$.\\
		(Ngoài ra, ta có thể đánh giá dấu của các hệ số $a,~b,~c$ thông qua hoành độ $2$ điểm cực trị và hoành độ trung điểm của hai điểm cực trị. Trong đồ thị này ta còn thấy hàm số có điểm cực tiểu $x=0$ nên $c=0$)
	}
\end{ex} \dongcham{1}

\begin{ex}%
	\immini[thm]{Đường cong bên là đồ thị của một trong bốn hàm số đã cho sau đây. Hỏi đó là hàm số nào?
		\choice
		{$y=x^3+3x-2  $}
		{$ y=x^3-3x+2$}
		{\True $y=-x^3+3x+2$}
		{$y=-x^3-3x-2$}
	}{
		\begin{tikzpicture}[scale=0.6, font=\footnotesize, line join=round, line cap=round, >=stealth]
			\clip(-2.5,-1.2) rectangle (5,5);
			\draw[->] (-2.5,0) -- (3,0) node[below]{ $x$};
			\draw[->] (0,-1.5) -- (0,4.7) node[left]{ $y$};
			\draw[line width=1pt,smooth,samples=100,domain=-2.05:2.05] plot(\x,{-(\x)^3+3*(\x)+2});
			\draw [fill=black] (0,0) circle (1pt)node[below left]{\footnotesize $O$}(-1,1);
			\draw[dashed](-2,0)node[below]{\scriptsize $-2$}--(-2,4)--(0,4)node[below left]{\scriptsize $4$}--(1,4)--(1,0)node[below]{\scriptsize $1$};
			\draw(2,0)node[below right]{\scriptsize $2$};
	\end{tikzpicture}}
	
	\loigiai{
		Quan sát đồ thị, ta thấy nhánh cuối của đồ thị hướng xuống dưới nên $\lim\limits_{x\rightarrow +\infty}y=-\infty$, suy ra hệ số $a<0$. Như vậy hai hàm số 	$y=x^3+3x-2; y=x^3-3x+2$ không thỏa mãn.
		\\Mặt khác hàm số có hai điểm cực trị nên hàm số $y=-x^3-3x-2$ có $y'=-3x^2-3<0$ $\forall x\in \mathbb{R}$ không thỏa mãn.
	}
\end{ex} \dongcham{1}


\begin{ex}
	\immini[thm]{
		Đường cong bên là đồ thị của một trong bốn hàm số đã cho sau đây. Hỏi đó là hàm số nào?
		\choice
		{$y= - x^3 + 3x^2 + 1$}
		{$y= - x^2 - 3x - 1$}
		{$y=x^4 + 2x^2 - 1$}
		{\True $y=x^3 - 3x + 1$}
	}{
		\begin{tikzpicture}[scale=0.6, font=\footnotesize, line join=round, line cap=round, >=stealth]
			\draw[->] (-2.7,0)--(0,0) node[below left]{$O$}--(2.5,0) node[below]{$x$};
			\draw[->] (0,-1.5) --(0,3.8) node[right]{$y$};
			\tkzDefPoints{0/0/O}
			\draw(-1.2,0) node[below]{$-1$};
			\draw(1,0) node[above]{$1$};
			\draw(0,-1) node[left]{$-1$};
			\draw(0,3) node[right]{$3$};
			\draw [domain=-2.02:2.02, samples=100] %
			plot (\x, {(\x)^3-3*(\x)+1}) ;
			\draw [dashed] (0,3)--(-1,3)--(-1,0);
			\draw [dashed] (1,0)--(1,-1)--(0,-1);
			\tkzDrawPoints[fill=black](O)
		\end{tikzpicture}
	}
	\loigiai{
		Đường cong trong hình là đồ thị của hàm số bậc ba có hệ số $a<0$. Trong các hàm số đã cho, chỉ có duy nhất hàm số $y=x^3 - 3x + 1$ thỏa mãn.
	}
\end{ex} \dongcham{1}

\begin{ex}
	\immini[thm]{Đường cong bên là đồ thị của một trong bốn hàm số đã cho sau đây. Hỏi đó là hàm số nào?
		\choice
		{$y=x^3-3x^2-4$}
		{$y=-x^3-4$}
		{$y=-x^3+3x^2-2$}
		{\True $y=-x^3+3x^2-4$}
	}{\begin{tikzpicture}[scale=0.6,>=stealth, font=\footnotesize, line join=round, line cap=round]
			\def\a{-1} \def\b{3} \def\c{0} \def\d{-4} % Hệ số
			\def\xmin{-1.5} \def\xmax{3.8}
			\def\ymin{-4.5} \def\ymax{1.5}
			%\draw[color=gray!50,dashed] (\xmin,\ymin) grid (\xmax,\ymax);
			\foreach \x in {-1,2}
			\draw[thin] (\x,1pt)--(\x,-1pt) node [above] {$\x$};
			\foreach \y in {-4}
			\draw[thin] (1pt,\y)--(-1pt,\y) node [left] {$\y$};
			\draw[->] (\xmin,0)--(\xmax,0) node [below]{$x$};
			\draw[->] (0,\ymin)--(0,\ymax) node [left]{$y$};
			\node at (0,0) [below left]{$O$};
			\clip (\xmin+0.1,\ymin+0.1) rectangle (\xmax-0.5,\ymax-0.1);
			\draw[smooth,samples=300] plot(\x,{\a*(\x)^3+\b*(\x)^2+\c*(\x)+\d});
	\end{tikzpicture}}
	\loigiai{
		\begin{itemize}
			\item Đồ thị hàm số có dạng chữ N ngược nên đây là đồ thị hàm số $y=ax^3+bx^2+cx+d$ với $a<0$. Loại phương án $y=x^3-3x^2-4$.
			\item Đồ thị hàm số giao $Oy$ tại điểm có tung độ bằng $-4$ nên $d=-4$, loại phương án $y=-x^3+3x^2-2$.
			\item Hàm số có hai điểm cực trị $x=0, x=2$ nên loại phương án $y=-x^3-4$ (vì phương án này có $y'=-3x^2$, hàm số không có điểm cực trị).
	\end{itemize}}
\end{ex} \dongcham{1}

\begin{ex}
	\immini[thm]{Đường cong bên là đồ thị của một trong bốn hàm số đã cho sau đây. Hỏi đó là hàm số nào?
		\haicot
		{$y=x^3-1$}
		{$y=(x+1)^3$}
		{\True $y=(x-1)^3$}
		{$y=x^3+1$}}
	{
		\begin{tikzpicture}[scale=0.8,>=stealth]
			\draw[->] (-1,0)--(0,0)node[above left]{$O$}--(2.2,0)node[below]{$x$};
			\draw[->] (0,-2)--(0,1.7)node[left]{$y$};
			\draw[line width=1pt,smooth,samples=100,domain=-0.25:2.2] plot(\x,{(\x-1)^3});
			\draw [fill=black] (1,0) circle (1.5pt);
			\draw (1,0)node[above]{$1$} (0,-1)node[left]{$-1$};
		\end{tikzpicture}
	}
	\loigiai{
		$(C)$ tiếp xúc với $Ox$ tại điểm uốn, suy ra $f(x)$ có nghiệm bội ba $x=1$ nên hàm số có dạng $y=a(x-1)^3$. Mà $(0;-1)\in (C)$ nên $a=1$.
	}
\end{ex} \dongcham{1}

\begin{ex}
	\immini[thm]{Cho hàm số $y = ax^3 + bx^2 + cx + d$ có đồ thị như hình vẽ bên. Khẳng định nào sau đây là đúng?
		
		\choice
		{$a > 0$, $b > 0$, $c > 0$, $d > 0$}
		{$a < 0$, $b < 0$, $c > 0$, $d > 0$}
		{$a > 0$, $b < 0$, $c < 0$, $d > 0$}
		{\True $a > 0$, $b < 0$, $c > 0$, $d > 0$}}
	{\begin{tikzpicture}[scale=0.9, font= \footnotesize, line join=round, line cap=round, >=stealth]
			\draw[->] (-2,0) -- (4,0) node[above] {$x$};
			\draw[->] (0,-1.3) -- (0,2) node[right] {$y$};
			\draw[fill=black] (1,0) circle (1.5pt);
			\draw[fill=black] (0,0) circle (1.5pt);
			\draw[line width=1pt,smooth,samples=100,domain=-0.4:3.1] plot(\x,{(\x)^3-4*(\x)^2 + 3*(\x) + 1});
			\node[below right] at (0,0) {$O$};
			\node[above] at (1,0) {$1$};
	\end{tikzpicture}}
	
	\loigiai{
		Nhìn vào đồ thị, ta thấy đồ thị hàm số đi từ $-\infty$ lên $+\infty$ nên $a > 0$. \\
		Giao điểm với trục tung nằm trên trục hoành, do đó $d > 0$.\\
		Hàm số có hai điểm cực trị, và hai điểm cực trị đều dương. Suy ra tổng hai điểm cực trị và tích hai điểm cực trị đều dương.\\ 	Ta có $f'(x) = 3ax^2 + 2bx + c$ nên tổng hai điểm cực trị là $\dfrac{-2b}{3a}$. Suy ra $\dfrac{-2b}{3a} > 0$, hay $b < 0$.\\ Còn tích hai điểm cực trị là $\dfrac{c}{3a}$. Suy ra $\dfrac{c}{3a} > 0$ hay $c > 0$.}
\end{ex} \dongcham{1}

\begin{ex}%[2D1B5-1]
	\immini[thm]{Cho hàm số $ y=ax^3+bx^2+cx+d $ có đồ thị như hình vẽ bên. Mệnh đề nào sau đây đúng?
		\choice
		{$ a<0 $, $ b<0 $, $ c<0 $, $ d>0 $}
		{$ a<0 $, $ b>0 $, $ c<0 $, $ d>0 $}
		{\True $ a<0 $, $ b>0 $, $ c>0 $, $ d<0 $}
		{$ a<0 $, $ b<0 $, $ c>0 $, $ d<0 $}}{
		\begin{tikzpicture}[smooth,samples=300,scale=0.7,>=stealth]
			\draw[->] (-2,0)--(3.7,0) node[below]{$x$};
			\draw[->] (0,-1.7)--(0,2.8) node[right]{$y$};
			\draw (0,0) node[above left]{$O$};
			\draw[line width=1pt,domain=-1.5:2.5] plot(\x,{-(\x-0.5)^3+3*(\x-0.5)+0.5});
		\end{tikzpicture}
	}
	\loigiai{
		Dựa vào hình dáng đồ thị suy ra $ a<0 $.\\
		Dựa vào vị trí điểm cực đại và điểm cực tiểu, suy ra $ x_{\text{CT}}+x_{\text{CĐ}}>0 \Rightarrow -\dfrac{b}{a}>0\Rightarrow b>0$.\\
		Hai điểm cực trị có hoành độ trái dấu nên $ x_{\text{CT}}\cdot x_{\text{CĐ}}<0\Rightarrow \dfrac{c}{a}<0\Rightarrow c>0 $.\\
		Đồ thị hàm số cắt trục tung tại điểm có tung độ dương nên $ d>0 $.\\
		Vậy $ a<0 $, $ b>0 $, $ c>0 $ và $ d>0 $.
	}
\end{ex} \dongcham{1}

\begin{ex}%[2D1K5-1]
	\immini[thm]{Cho hàm số $y=ax^3+bx^2+cx+d$ có đồ thị như hình vẽ bên. Mệnh đề nào dưới đây đúng?
		\choice
		{$a<0$, $b>0$, $c>0$, $d>0$}
		{$a<0$, $b<0$, $c=0$, $d>0$}
		{\True $a<0$, $b>0$, $c=0$, $d>0$}
		{$a>0$, $b<0$, $c>0$, $d>0$}}{
		\begin{tikzpicture}[smooth,samples=300,scale=0.7,>=stealth]
			\draw[->] (-2,0)--(3.7,0) node[below]{$x$};
			\draw[->] (0,-1)--(0,4.5) node[right]{$y$};
			\draw (0,0) node[below left]{$O$};
			\draw[line width=1pt,domain=-1:3.1] plot(\x,{-(\x)^3+3*(\x)^2+0.3});
			%\draw[fill=black] (2,-1) circle(1.5pt) (2,0) circle(1pt) (0,-1) circle(1pt);
			%\draw[dashed] (2,-1.5)--(2,2.5) (2,-1)--(0,-1)node[left]{\small$-\dfrac{\Delta}{4a}$};
			%\node[right] at (2,2.4) {\small $x=-\tfrac{b}{2a}$};
			%\node[right] at (0.5,-2) {\fbox{$a>0$}};
		\end{tikzpicture}
	}
	\loigiai{
		Dựa vào đồ thị ta có thể thấy $a<0$, đồ thị cắt trục tung tại điểm có tung độ dương nên $d>0$.\\
		Hàm số có hai cực trị thỏa $\heva{&S>0\\&P=0}\Leftrightarrow\heva{&-\dfrac{b}{a}>0\\&\dfrac{c}{a}=0}\Leftrightarrow\heva{&b>0\\&c=0.}$
	}
\end{ex} \dongcham{4}

\begin{ex}
	\immini[thm]{Cho hàm số $y=ax^3+bx^2+cx+d$ có bảng biến
	thiên như hình bên. Trong các hệ số $a$, $b$, $c$ và $d$ có bao nhiêu số âm?
	\choice
	{$2$}
	{\True $1$}
	{$4$}
	{$3$}}{
\begin{tikzpicture}[>=stealth,scale=1]
	\tkzTabInit[lgt=1.2,espcl=2]
	{$x$ /0.6, $f’(x)$ /0.6, $f(x)$ /2}
	{$-\infty$,$-1$,$2$,$+\infty$}
	\tkzTabLine{ ,-,z,+,z,-, }
	\tkzTabVar{+/,-/$0$,+/,-/}
\end{tikzpicture}}
	\loigiai
	{
		Từ bảng biến thiên ta thấy hàm số có $2$ điểm cực trị nên bậc của đa thức phải lớn hơn $2\Rightarrow a\ne 0$. Mà $\lim \limits_{x \to +\infty} y=-\infty\Rightarrow a<0$.\\
		Từ bảng biến thiên ta có $d=y(0)>y(-1)=0$.\\
		Ta có $y'=3ax^2+2bx+c$ có hai nghiệm là $-1$ và $2$ nên $\heva{& -\dfrac{2b}{3a}=-1+2=1>0 \\ & \dfrac{c}{3a}=(-1)\cdot 2=-2<0}\Rightarrow \heva{& b>0 \\ & c>0.}$
	}
\end{ex} \dongcham{4}

\Closesolutionfile{ans}

\ind{PHẦN II.} \inden{Câu trắc nghiệm đúng sai. Trong mỗi ý a), b), c), d) ở mỗi câu, học sinh chọn đúng hoặc sai.}\\
\Opensolutionfile{ans}[ans/2D1-B4-d1-2]

\begin{ex}
	\immini[thm]{Cho hàm số $y=f(x)=ax^3+bx^2+cx+d$ có đồ thị như hình vẽ.
		\choiceTF
		{Hàm số đạt cực tiểu tại $x=1$}
		{\True Đồ thị hàm số cắt trục $Oy$ tại điểm $(0;1)$}
		{Hàm số đồng biến trên khoảng $(-\infty;-1)$}
		{$2a+3b+c=9$}
	}{
		\begin{tikzpicture}
			[scale=1,line join=round, line cap=round, >=stealth]
			\draw[->] (-3,0)--(0,0) node[below left]{$O$}--(2,0) node[below]{$x$};
			\draw[->] (0,-1) --(0,3) node[right]{$y$};
			\draw [domain=-2.3:.7, samples=100] %
			plot (\x, {(\x)^3+2*(\x)^2+1});
			\draw [dashed] (-2,0)node[below]{$-2$}--(-2,1) --(0,1)node[below right]{$1$}
			(-1,0)node[below]{$-1$}--(-1,2)--(0,2)node[right]{$2$};
			\draw[fill] (0,1) circle (1pt) (-2,1) circle (1pt) (-1,2) circle (1pt);
		\end{tikzpicture}}
	\loigiai{
		Theo hình vẽ thì:
		\begin{enumerate}[a)]
			\item Hàm số đạt cực tiểu tại $x=0$, giá trị cực tiểu $y=1$;
			\item Đồ thị hàm số cắt trục $Oy$ tại điểm $(0;1)$;
			\item Hàm số đồng biến trên khoảng $(-\infty;x_0)$, với $-2<x_0<-1$;
			\item Đồ thị qua 3 điểm $(-2;1)$, $(-1;2)$, $(0;1)$ và đạt cực trị tại $x=1$ nên ta được hệ
			$$\heva{&-8a+4b-2c+d=1\\&-a+b-c+d=2 \\& d=1\\&c=0} \Leftrightarrow a=1;\,b=2,\,c=0,\,d=1$$
			nên $2a+3b+c=8$.
		\end{enumerate}
	}
\end{ex} \dongcham{10}

\begin{ex}
	\immini[thm]{Cho hàm số bậc ba $ f(x)=ax^3+bx^2+cx+d $ có đồ thị như hình vẽ.\\
		Tính tổng $ T=$.
		\choiceTF
		{\True Đồ thị hàm số cắt trục tung tại điểm $(0;1)$}
		{\True Đường thẳng đi qua điểm $(0;1)$ luôn cắt đồ thị tại ba điểm phân biệt có hoành độ lập thành 1 cấp số cộng}
		{\True $a-b+c+d =-1$}
		{Đồ thị hàm số đi qua điểm $(3;18)$}
	}{\begin{tikzpicture}[>=stealth,line join=round,line cap=round,scale=.8]
			\draw[->] (-2.3,0)--(2.5,0)node[below]{$x$};
			\draw[->] (0,-1.5)--(0,3.5)node[right]{$y$};
			\draw[domain=-2:2, samples=100] plot (\x,{(\x)^3-3*(\x)+1});
			\draw[fill] (-1,3) circle (1pt) (0,1) circle (1pt) (1,-1) circle (1pt);
			\draw[dashed] (-1,0)node[below]{$-1$}|-(0,3)node[right]{$3$} (0,-1)node[left]{$-1$}-|(1,0)node[above]{$1$}
			;
	\end{tikzpicture}}
	\loigiai{
		\begin{enumerate}[a)]
			\item Đồ thị hàm số có hai điểm cực trị $(-1;3)$ và $(1;-1)$. Suy ra tọa độ tâm đối xứng là $(0;1)$. Suy ra đồ thị hàm số cắt trục tung tại điểm $(0;1)$
			\item Do $I(0;1)$ là tâm đối xứng của đồ thị, nên đường thẳng qua nó sẽ cắt đồ thị tại ba điểm phân biệt $I$, $A$, $B$ với $I$ là trung điểm của $AB$. Suy ra $x_A+x_B=2x_I$. Vậy ba điểm này có hoành độ lập thành 1 cấp số cộng.
			\item Ta có $ f'(x)=3ax^2+2bx+c $. Từ hình vẽ, ta có
			$$\heva{&f(-1)=3\\&f(1)=-1\\&f'(-1)=0\\&f'(1)=0} \Leftrightarrow \heva{&-a+b-c+d=3\\&a+b+c+d=-1\\&3a-2b+c=0\\&3a+2b+c=0}$$
			Giải hệ, ta được $a=1$, $b=0$, $c=-3$,$d=1$.
			Vậy $ T=a-b+c+d=-1 $.
			\item Ta có $ f'(x)=3ax^2+2bx+c $. Từ hình vẽ, ta có
			$$\heva{&f(-1)=3\\&f(1)=-1\\&f'(-1)=0\\&f'(1)=0} \Leftrightarrow \heva{&-a+b-c+d=3\\&a+b+c+d=-1\\&3a-2b+c=0\\&3a+2b+c=0}$$
			Giải hệ, ta được $a=1$, $b=0$, $c=-3$,$d=1$. Suy ra $y=x^2-3x+1$.\\
			Thay tọa độ $(3;18)$ vào phương trình, không thỏa mãn. Vậy đồ thị hàm số không đi qua điểm $(3;18)$.
		\end{enumerate}
		}
\end{ex} \dongcham{14}

\begin{ex}
	\immini[thm]{Cho hàm số $ y=f(x)=ax^3+bx^2+cx+d$ có bảng biến thiên như hình bên.
		\choiceTF
		{Hàm số đạt giá trị lớn nhất là $ 4 $}
		{\True Đường thẳng $ y=2$ cắt đồ thị hàm số $ y=f(x)$ tại $ 3 $ điểm phân biệt}
		{\True Trong bốn hệ số $a$, $b$, $c$, $d$ có đúng hai số âm}
		{\True Đồ thị hàm số đi qua điểm $(-4;20)$}
	}{
		\begin{tikzpicture}
			\tkzTabInit[nocadre=false,lgt=1.2,espcl=1.6,deltacl=0.6]
			{$x$ /0.6, $y'$ /0.6, $y$ /2.3}
			{$-\infty$,$-2$,$0$,$+\infty$}
			\tkzTabLine{,-,0,+,0,-,}
			\tkzTabVar{+/$+\infty$,-/$0$,+/$4$,-/$-\infty$}
	\end{tikzpicture}}
	\loigiai{
		Dựa vào bảng biến thiên ta thấy:
		\begin{enumerate}[a)]
			\item Hàm số $ y=f(x)$ không có giá trị lớn nhất trên $\mathbb{R}$.
			\item Vẽ đường thẳng $y=2$ qua điểm $(0;2)$ và song song với $Ox$, rõ ràng đường thẳng này cắt đồ thị tại ba điểm phân biệt.
			\item Từ các thông số trên hình, ta có thể giải ra chính xác giá trị $a$, $b$, $c$, $d$ bởi hệ
			$$\heva{&f(-2)=0\\&f(0)=4\\&f'(-2)=0\\&f'(0)=0} \Leftrightarrow a=-1,\,b=-3,\,c=0,\,d=4.$$
			Vậy trong 4 hệ số, có đúng 2 số âm.
			\item Từ các thông số trên hình, ta có thể giải ra chính xác giá trị $a$, $b$, $c$, $d$ bởi hệ
			$$\heva{&f(-2)=0\\&f(0)=4\\&f'(-2)=0\\&f'(0)=0} \Leftrightarrow a=-1,\,b=-3,\,c=0,\,d=4.$$
			Suy ra $y=-x^3-3x^2+4$. Thay tọa độ $(-4;20)$ vào phương trình, thỏa mãn. Suy ra Đồ thị hàm số đi qua điểm $(-4;20)$.
		\end{enumerate}
		
	}
\end{ex} \dongcham{14}

\Closesolutionfile{ans}





% \begin{dang}{Khảo sát và vẽ đồ thị hàm số phân thức hữu tỉ bậc I/I}
	Ta khảo sát theo sơ đồ
	\begin{itemize}
		\item[\iconCH] \indamm{Bước 1.} Tìm tập xác định $D=\mathbb{R}\backslash \left\{-\dfrac{d}{c}\right\}$.
		\item [\iconCH] \indamm{Bước 2.} Khảo sát sự biến thiên của hàm số
		      \begin{itemize}
			      \item Tính đạo hàm $y'=\dfrac{ad-cb}{(cx+d)^2}$.
			      \item Tìm các giới hạn tại vô cực, giới hạn vô cực và tìm tiệm cận của đồ thị hàm số.
			      \item Lập bảng biến thiên; xác định chiều biến thiên và cực trị của hàm số.
		      \end{itemize}
		\item [\iconCH] \indamm{Bước 3.} Cho thêm điểm và vẽ đồ thị của hàm số dựa vào bảng biến thiên.
	\end{itemize}
\end{dang}
\boxmini{BÀI TẬP TỰ LUẬN}
\begin{vd}
	Khảo sát sự biến thiên và vẽ đồ thị các hàm số sau:
	\begin{tasks}(3)
		\task $y=\dfrac{x-1}{x+1}$;
		\task $y=\dfrac{2 x+1}{x-1}$;
		\task $y = \dfrac{5 + x}{2 - x}$.
	\end{tasks}
	\loigiai{
		\begin{enumerate}[a)]
			\item Tập xác định: $\mathbb{R} \backslash\{-1\}$.\\
			      Sự biến thiên:
			      \begin{itemize}
				      \item [$\bullet$] Đạo hàm $y^{\prime}=\dfrac{2}{(x+1)^{2}}>0$ với mọi $x \neq -1$.
				      \item [$\bullet$] Giới hạn và tiệm cận:\\
				            $\displaystyle\lim _{x \rightarrow -1^{-}} y= +\infty, \displaystyle\lim _{x \rightarrow -1^{+}} y= -\infty$. Do đó, đường thẳng $x=-1$ là tiệm cận đứng của đồ thị hàm số.\\
				            $\displaystyle\lim _{x \rightarrow-\infty} y=1, \displaystyle\lim _{x \rightarrow +\infty} y=1$. Do đó, đường thẳng $y=1$ là tiệm cận ngang của đồ thị hàm số.
				      \item Bảng biến thiên:
				            \begin{center}
					            \begin{tikzpicture}[font=\normalsize,t style/.style={style=solid},scale=.8]
						            %dòng khai báo
						            \tkzTabInit[lgt=1.2,espcl=4,deltacl=0.9]
						            {$x$ /0.75, $y^{\prime}$/0.75, $y$/2.5}
						            {$ -\infty $,$ -1 $,$ +\infty $}
						            %dòng xét dấu
						            \tkzTabLine{ , + ,d , - , } % z, t, d;
						            %dòng biến thiên
						            \path ($(N12)!0.5!(N13)$) node (A1){$ 1 $}
						            ($(N22)!0.1!(N23)+(-17pt,-0)$) node (A2){$ +\infty $}
						            ($(N22)!0.9!(N23)+(12pt,0)$) node (A3){$ -\infty $}
						            ($(N32)!0.5!(N33)$) node (A4){$ 1 $};
						            \draw[double] (N22)--(N23);
						            \foreach \x/\y in {A1/A2,A3/A4}{
								            \draw[-stealth] (\x)--(\y);
							            }
					            \end{tikzpicture}
				            \end{center}
				            Hàm số đồng biến trên mỗi khoảng $(-\infty ; -1)$ và $(-1 ;+\infty)$.\\
				            Hàm số không có cực trị.
			      \end{itemize}
			      Đồ thị:\\
			      \immini{	\begin{itemize}
					      \item Giao điểm của đồ thị với trục tung: $(0 ;-1)$.
					      \item Giao điểm của đồ thị với trục hoành: $\left(1 ; 0\right)$.
					      \item Đồ thị hàm số đi qua các điểm $(0 ;-1)$, $\left(1 ; 0\right)$,  $(-3 ;2)$, $(-2 ;3)$.
				      \end{itemize}
			      }
			      {		\begin{tikzpicture}[line cap=butt,line join=miter,>=stealth,scale=.7,font=\footnotesize]
					      \tikzset{declare function={xmin=-6.1;xmax=4.1;ymin=-4.1;ymax=6.1;},
						      smooth,samples=450}
					      \draw[->] (xmin,0)--(xmax,0) node[shift={(0:7pt)}]{$ x $};
					      \draw[->] (0,ymin)--(0,ymax) node[shift={(90:7pt)}]{$ y $};
					      \fill (0,0) node[shift={(130:8pt)}]{$ O $};
					      \clip (-6,-4.6) rectangle (4,6);
					      \foreach \i in {-3,-2,-1,1}{
							      \draw(\i,1.5pt)--(\i,-1.5pt)node[below]{$\i$};}
					      \foreach \j in {-1,2,3}{
							      \draw(-1.5pt,\j)--(1.5pt,\j) node[right]{$\j$};}
					      \draw(-1.5pt,1)--(1.5pt,1)node[shift={(6pt,3pt)}]{$1$};
					      \def\f(#1){((#1)-1)/((#1)+1)}
					      \def\a{-2}
					      \def\b{-3}
					      \def\c{1}
					      \def\d{0}
					      \pgfmathsetmacro\fa{\f(\a)}
					      \pgfmathsetmacro\fb{\f(\b)}
					      \pgfmathsetmacro\fc{\f(\c)}
					      \pgfmathsetmacro\fd{\f(\d)}
					      \draw[samples=100] plot[domain=-6:-1.1] (\x,{\f(\x)});
					      \draw[samples=100] plot[domain=-0.9:4] (\x,{\f(\x)});
					      \draw[] (-1,-4)--(-1,6);
					      \draw[] (-6,1)--(4,1);
					      \foreach \x/\y in {\a/\fa,\b/\fb,\c/\fc,\d/\fd}{
							      \draw[dashed] (\x,0)|-(0,\y);
							      %\draw[dashed] (-2,3)--(0,-1) (-3,2)--(1,0);
							      \fill[white,draw=black] (\x,\y) circle (1pt);}
					      \node at (-1,1) [ shift = (45:7pt)] {I};
				      \end{tikzpicture}	}
			\item Tập xác định $\mathbb{R} \backslash\{1\}$.\\
			      Sự biến thiên:
			      \begin{itemize}
				      \item [$\bullet$] Đạo hàm: $y^{\prime}=\dfrac{-3}{(x-1)^{2}}<0$ với mọi $x \neq 1$.
				      \item [$\bullet$] Giới hạn và các đường tiệm cận:\\
				            $\displaystyle\lim _{x \rightarrow 1^{-}} y=-\infty, \displaystyle\lim _{x \rightarrow 1^{+}} y=+\infty$. Do đó, đường thẳng $x=1$ là tiệm cận đứng của đồ thị hàm số.\\
				            $\displaystyle\lim _{x \rightarrow+\infty} y=2, \displaystyle\lim _{x \rightarrow-\infty} y=2$. Do đó, đường thẳng $y=2$ là tiệm cận ngang của đồ thị hàm số.
				      \item [$\bullet$] Bảng biến thiên:
				            \begin{center}
					            \begin{tikzpicture}[font=\normalsize,t style/.style={style=solid},scale=.8]
						            %dòng khai báo
						            \tkzTabInit[lgt=1.2,espcl=4,deltacl=0.75]
						            {$x$ /0.75, $y^{\prime}$/0.75, $y$/2.5}
						            {$ -\infty $,$ 1 $,$ +\infty $}
						            %dòng xét dấu
						            \tkzTabLine{ , -,d , -, } % z, t, d;
						            %dòng biến thiên
						            \path ($(N12)!0.5!(N13)$) node (A1){$ 2 $}
						            ($(N22)!0.9!(N23)+(-17pt,0)$) node (A2){$ -\infty $}
						            ($(N22)!0.1!(N23)+(12pt,0)$) node (A3){$ +\infty $}
						            ($(N32)!0.5!(N33)$) node (A4){$ 2 $};
						            \draw[double] (N22)--(N23);
						            \foreach \x/\y in {A1/A2,A3/A4}{
								            \draw[-stealth] (\x)--(\y);
							            }
					            \end{tikzpicture}
				            \end{center}
				            Hàm số nghịch biến trên mỗi khoảng $(-\infty ; 1)$ và $(1 ;+\infty)$.\\
				            Hàm số không có cực trị.
			      \end{itemize}
			      Đồ thị:\\
			      \immini{
				      \begin{itemize}
					      \item Giao điểm của đồ thị với trục tung: $(0 ;-1)$.
					      \item Giao điểm của đồ thị với trục hoành: $\left(-\dfrac{1}{2} ; 0\right)$.
					      \item Đồ thị hàm số đi qua các điểm $(0 ;-1),\left(-\dfrac{1}{2} ; 0\right)$, $(-2 ; 1),(2 ; 5),\left(\dfrac{5}{2} ; 4\right)$ và $(4 ; 3)$.
				      \end{itemize}
			      }
			      {		\begin{tikzpicture}[line cap=butt,line join=miter,>=stealth,scale=0.7,font=\tiny]
					      \tikzset{declare function={xmin=-3.1;xmax=5.1;ymin=-2.1;ymax=6.1;},
						      smooth,samples=450}
					      \draw[->] (xmin-.1,0)--(xmax+.1,0) node[shift={(0:7pt)}]{$ x $};
					      \draw[->] (0,ymin-.1)--(0,ymax+.1) node[shift={(90:7pt)}]{$ y $};
					      \fill (0,0) node[shift={(55:6pt)}]{$ O $};
					      \clip (xmin,ymin-.5) rectangle (xmax,ymax);
					      \foreach \i in {-2,-1,2,3,4}{
							      \draw(\i,1.5pt)--(\i,-1.5pt)node[below]{$\i$};}
					      \foreach \j in {-1,3,4,5}{
							      \draw(-1.5pt,\j)--(1.5pt,\j) node[left]{$\j$};}
					      \draw(-1.5pt,1)--(1.5pt,1)node[shift={(0:3pt)}]{};
					      \draw(-1.5pt,2)--(1.5pt,2)node[shift={(-135:7.5pt)}]{$2$};
					      \draw(1,-1.5pt)--(1,1.5pt)node[shift={(3pt,-7.2pt)}]{$1$};
					      \def\f(#1){(2*(#1)+1)/((#1)-1)} % Hàm số: ( 2x+1 )/( x-1 )
					      \def\a{-2}
					      \def\b{-1}
					      \def\c{-0.5}
					      \def\d{0}
					      \def\e{2}
					      \def\g{2.5}
					      \def\h{4}
					      \pgfmathsetmacro\fa{\f(\a)}
					      \pgfmathsetmacro\fb{\f(\b)}
					      \pgfmathsetmacro\fc{\f(\c)}
					      \pgfmathsetmacro\fd{\f(\d)}
					      \pgfmathsetmacro\fe{\f(\e)}
					      \pgfmathsetmacro\fg{\f(\g)}
					      \pgfmathsetmacro\fh{\f(\h)}
					      \draw[samples=100] plot[domain=-4.8:0.7] (\x,{\f(\x)});
					      \draw[samples=100] plot[domain=1.05:5] (\x,{\f(\x)});
					      \draw[] (1,ymin)--(1,ymax);
					      \draw[] (xmin,2)--(xmax,2);
					      \foreach \x/\y in {\a/\fa,\b/\fb,\c/\fc,\d/\fd,\e/\fe,\g/\fg,\h/\fh}{
							      %\draw[dashed] (0,-1)--(2,5)  (-.5,0)--(2.5,4) ;
							      \draw[dashed] (\x,0)|-(0,\y);
							      \fill[black] (\x,\y) circle (1pt);}
					      \node at (1,2) [shift = (135:5pt)] {I};
				      \end{tikzpicture}	}
			\item Tập xác định: $D = \mathbb{R} \setminus \left\{ 2\right\}$.\\
			      Sự biến thiên:
			      \begin{itemize}
				      \item [$\bullet$] Đạo hàm $y' = \dfrac{ 7}{(-x + 2)^2}>0$, với mọi $x \neq 2$
				      \item [$\bullet$] Giới hạn và tiệm cận:\\
				            $\displaystyle\lim _{x \rightarrow 2^{-}} y=+\infty, \displaystyle\lim _{x \rightarrow 2^{+}} y=-\infty$. Do đó, đường thẳng $x=2$ là tiệm cận đứng của đồ thị hàm số.\\
				            $\displaystyle\lim _{x \rightarrow+\infty} y=-1, \displaystyle\lim _{x \rightarrow-\infty} y=-1$. Do đó, đường thẳng $y=-1$ là tiệm cận ngang của đồ thị hàm số.
				      \item [$\bullet$] Bảng biến thiên:
				            \begin{center}
					            \begin{tikzpicture}[scale=1, font=\footnotesize, line join=round, line cap=round, >=stealth]
						            \tkzTabInit[nocadre=false,lgt=1.2,espcl=2.6,deltacl=0.6]
						            {$x$ /0.6,$y'$ /0.6,$y$ /1.6}
						            {$-\infty$,$2$,$+\infty$}
						            \tkzTabLine{,+,d,+,}
						            \tkzTabVar{-/$-1$,+D-/$+\infty$/$-\infty$,+/$-1$}
					            \end{tikzpicture}
				            \end{center}
				            Hàm số đồng biến trên khoảng $(-\infty;2)$ và $(2;+\infty)$.\\
				            Hàm số không có cực trị.
			      \end{itemize}
			      Đồ thị:
			      \begin{center}
				      \begin{tikzpicture}[scale=0.7,>=stealth, font=\footnotesize, line join=round, line cap=round]
					      \def\xmin{-5} \def\xmax{8}
					      \def\ymin{-8} \def\ymax{8}
					      %\draw[color=gray!50,dashed] (\xmin,\ymin) grid (\xmax,\ymax);
					      \draw[->] (\xmin,0)--(\xmax,0) node [below]{$x$};
					      \draw[->] (0,\ymin)--(0,\ymax) node [left]{$y$};
					      \fill (0,0) circle (1pt) node[shift={(-135:2.5mm)}]{$O$};
					      \node at (current bounding box.south) [below=-2pt] {c)};
					      \clip (\xmin+0.1,\ymin+0.1) rectangle (\xmax-0.1,\ymax-0.1);
					      \draw[smooth,red,samples=300,domain=(\xmin:1.8)] plot(\x,{((\x)+5)/(-(\x)+2)});
					      \draw[smooth,red,samples=300,domain=(2.2:\xmax)] plot(\x,{((\x)+5)/(-(\x)+2)});
					      \draw[blue] (\xmin,-1)--(\xmax,-1);
					      \draw[blue] (2,\ymin)--(2,\ymax);
					      \foreach \x in {\xmin,...,\xmax}
					      \draw (\x,-0.1)--(\x,0.1);
					      \foreach \y in {\ymin,...,\ymax}
					      \draw (-0.1,\y)--(0.1,\y);
					      \node at (0,2.5)[right]{$\frac{5}{2}$};
					      \node at (7,-1)[below]{$y=-1$};
					      \node at (2,7)[right]{$x=2$};
					      \node at (2,-1)[shift={(45:2.5mm)}]{$I$};
				      \end{tikzpicture}\hspace*{2cm}
			      \end{center}
		\end{enumerate}}
\end{vd}
\dongcham{48}
\boxmini{BÀI TẬP TRẮC NGHIỆM}
\ind{PHẦN I.} \inden{Câu trắc nghiệm nhiều phương án lựa chọn. Mỗi câu hỏi học sinh chỉ chọn một phương án.}\\
\setcounter{ex}{0}
\Opensolutionfile{ans}[ans/2D1-B4-d2-1]
\begin{ex}%[2D1B5-1]
	\immini{Hàm số nào trong bốn hàm số dưới đây có bảng biến thiên như hình bên?
		\choice
		{$ y=\dfrac{2x-1}{x+3} $}
		{$ y=\dfrac{4x-6}{x-2} $}
		{$ y=\dfrac{3-x}{2-x}$}
		{\True $ y=\dfrac{x+5}{x-2} $}}
	{\begin{tikzpicture}
			% \tikzset{double style/.append style = {draw=\tkzTabDefaultWritingColor,double=\tkzTabDefaultBackgroundColor,double distance=2pt}}
		\tkzTabInit[nocadre=false,lgt=1,espcl=2.5,deltacl=0.6]
		{$x$/0.6,$y'$/0.6,$y$/1.5}
		{$-\infty$,$2$,$+\infty$}
		\tkzTabLine{,-,d,-,}
		\tkzTabVar{+/$1$,-D+/$-\infty$/$+\infty$,-/$1$}
	\end{tikzpicture}
	}
	\loigiai{
		Xét hàm số $ y=\dfrac{x+5}{x-2} $ có
		$$\heva{&y'=\dfrac{-7}{(x-2)^2}<0, \forall x \in \mathbb{R} \setminus \{2\} \\ & \lim\limits_{x \to \pm \infty} y=1.}$$
	}
\end{ex} \dongcham{1}

\begin{ex}%[2D1B5-1]
	\immini{Hàm số nào trong bốn hàm số dưới đây có bảng biến thiên như hình bên?
		\choice
		{$y=\dfrac{x-1}{x-3}$}
		{$y=\dfrac{x-1}{-x-3}$}
		{\True $y=\dfrac{x+5}{-x+3}$}
		{$y=\dfrac{1}{x-3}$}
	}{
		\begin{tikzpicture}
			% \tikzset{double style/.append style = {draw=\tkzTabDefaultWritingColor,double=\tkzTabDefaultBackgroundColor,double distance=2pt}}
			\tkzTabInit[lgt=1,espcl=2.6]
			{$x$/0.6,$y'$/0.6,$y$/1.5}{$-\infty$,$3$,$+\infty$}
			\tkzTabLine{,+,d,+,}
			\tkzTabVar{-/$-1$,+D-/$+\infty$/$-\infty$,+/$-1$}
		\end{tikzpicture}
	}
	\loigiai{Dựa vào bảng biến thiên, ta suy ra
		\begin{itemize}
			\item Hàm số nghịch biến trên từng khoảng xác định.
			\item Đồ thị hàm số nhận đường thẳng $x=2$ và đường thẳng $y=1$ làm tiệm cận đứng và tiệm cận ngang.
		\end{itemize}
		Vậy ta nhận hàm số $y=\dfrac{x+5}{x-2}$.}
\end{ex} \dongcham{1}

\begin{ex}
	\immini
	{Đường cong trong hình vẽ bên là đồ thị của một trong bốn hàm số sau. Hỏi đó là hàm số nào?
		\haicot
		{\True $y=\dfrac{2x-1}{x+1}$}
		{$y=\dfrac{1-2x}{x+1}$}
		{$y=\dfrac{2x+1}{x-1}$}
		{$y=\dfrac{2x+1}{x+1}$}
	}
	{
		\begin{tikzpicture}[smooth,samples=300,scale=0.45,>=stealth]
			\draw[->] (-5,0)--(3,0) node[below]{$x$};
			\draw[->] (0,-2.5)--(0,4.5) node[right]{$y$};
			\draw (0,0) node[above left]{$O$};
			\draw[line width=1pt,domain=-0.3:3] plot(\x,{(2*\x-1)/(\x+1)});
			\draw[line width=1pt,domain=-5:-2.2] plot(\x,{(2*\x-1)/(\x+1)});
			\draw[fill=black] (0,2) circle(1.5pt) (0,-1) circle(1.5pt) (-1,0) circle(1.5pt);
			\draw (-5,2)--(3,2) (-1,-2.5)--(-1,4.5);
			\draw (0,-1) node[right]{$-1$};
			\draw (-1,0) node[below left]{$-1$};
			\draw (0,2) node[above right]{$2$};
		\end{tikzpicture}
	}
	\loigiai
	{
		Đồ thị hàm số có tiệm cận đứng là $x=-1$ nên loại đáp án $ y=\dfrac{2x+1}{x-1}$.\\
		Đồ thị hàm số đi qua điểm $A(0;-1)$ nên loại đáp án $y=\dfrac{1-2x}{x+1}$ và $ y=\dfrac{2x+1}{x+1}$.
	}
\end{ex} \dongcham{1}

\begin{ex}
	\immini{Đường cong trong hình vẽ bên là đồ thị của một trong bốn hàm số sau. Hỏi đó là hàm số nào?
		\choice
		{$y=\dfrac{x-1}{x-2}$}
		{$y=x+2$}
		{$y=x^4-3x^2+1$}
		{\True $y=\dfrac{2x+1}{x-1}$}
	}{\begin{tikzpicture}[scale=0.5, line join=round, line cap=round,font=\footnotesize,>=stealth,x=0.7cm,y=0.7cm]
			\draw[fill,->] (-5,0)--(0,0) node[below left]{$O$}circle(0.05)--(6,0) node [below] {$x$};
			\draw[->] (0,-4)--(0,6) node [left] {$y$};
			\draw[black,domain=1.75:6, samples=100]plot(\x,{(2*(\x)+1)/((\x)-1)});
			\draw[black,domain=-4.9:0.5, samples=100]plot(\x,{(2*(\x)+1)/((\x)-1)});
			\draw[black,domain=-5:6, samples=100]plot(\x,{2});
			\draw[black,domain=-4:6, samples=100, variable=\t]plot(1,\t);
			\foreach \x in {1}
			\draw (\x,0.05)--(\x,-0.05) node [below right] {\x};
			\foreach \y in {2}
			\draw (0.05,\y)--(-0.05,\y) node [below left] {\y};
		\end{tikzpicture}}
	\loigiai{
		Đồ thị hàm số như hình vẽ nhận đường thẳng $x=1$ là tiệm cận đứng.\\
		Do đó, hàm số cần tìm là $y=\dfrac{2x+1}{x-1}$.
	}
\end{ex} \dongcham{1}

\begin{ex}
	\immini{Đường cong trong hình vẽ bên là đồ thị của một trong bốn hàm số sau. Hỏi đó là đồ thị của hàm số nào?
		\haicot
		{$y=\dfrac{x-2}{x+1}$}
		{$y=\dfrac{x+2}{x-2}$}
		{\True $y=\dfrac{x-2}{x-1}$}
		{$y=\dfrac{x+2}{x-1}$}
	}
	{
		\begin{tikzpicture}[>=stealth,x=1cm,y=1cm,scale=0.5]
			\draw[->] (-3,0)--(0,0) node[below left]{$O$}--(5,0) node[above]{$x$};
			\draw[->] (0,-3) --(0,5) node[left]{$y$};
			\foreach \x in {1,2}{\draw[-] (\x,-0.1)--(\x,0.1);}
			\foreach \y in {1,2}{\draw[-] (-0.1,\y)--(0.1,\y);}
			\draw [domain=-3:0.75, samples=100] plot (\x, {(\x-2)/(\x-1)});
			\draw [domain=1.25:5, samples=100] plot (\x, {(\x-2)/(\x-1)});
			\draw [dashed](-3,1)--(5,1) (1,-3)--(1,5);
			\draw (0,1) node[below left]{$1$};
			\draw (0,2) node[above left]{$2$};
			\draw (1,0) node[below left]{$1$};
			\draw (2,0) node[below right]{$2$};
		\end{tikzpicture}
	}
	\loigiai{
		Từ đồ thị ta thấy
		\begin{itemize}
			\item Tiệm cận ngang là $y=1$, tiệm cận đứng là $x=1$ nên các hàm số $y=\dfrac{x+2}{x-2}$, $y=\dfrac{x-2}{x+1}$ không thỏa mãn.
			\item Giao điểm của đồ thị với trục tung là $(0;2)$ nên hàm số $y=\dfrac{x+2}{x-1}$ không thỏa mãn, hàm số $y=\dfrac{x-2}{x-1}$ thỏa mãn.
		\end{itemize}
	}
\end{ex} \dongcham{1}


\begin{ex}
	\immini
	{Cho hàm số $y=\dfrac{ax-b}{x+c}$ ($a,b,c\in \mathbb{R}$) có đồ thị như hình vẽ bên. Giá trị của biểu thức $2a+b-3c$ bằng
		\haicot
		{$-3$}
		{$4$}
		{\True $7$}
		{$-5$}
	}
	{\begin{tikzpicture}[scale=0.7, font=\footnotesize, line join=round, line cap=round, >=stealth]
			\def\xt{-2.5} \def\xp{4.5} \def\yt{4.5} \def\yd{-2.5}
			\draw[->] (\xt,0)--(\xp,0) node [below]{$x$};
			\draw[->] (0,\yd)--(0,\yt) node [left]{$y$};
			\node at (0,0) [below left]{$O$};
			\clip (\xt,\yd) rectangle (\xp,\yt);
			\draw[smooth,samples=200,domain=\xt:0.99] plot(\x,{(\x-2)/(\x-1)});
			\draw[smooth,samples=300,domain=1.01:\xp] plot(\x,{(\x-2)/(\x-1)});
			\draw[dashed] (1,\yd)--(1,\yt);
			\draw[dashed] (\xt,1)--(\xp,1);
			\fill (1,0)node[shift={(-120:0.3)}]{$1$} circle(1pt);
			\fill (2,0)node[shift={(-60:0.3)}]{$2$} circle(1pt);
			\fill (0,1)node[shift={(230:0.3)}]{$1$} circle(1pt);
			\fill (0,2)node[shift={(150:0.3)}]{$2$} circle(1pt);
		\end{tikzpicture}}
	\loigiai
	{Từ đồ thị hàm số ta có:\\
		Đường tiệm cận đứng là $x=1$ nên $-c=1 \Leftrightarrow c=-1$.\\
		Đường tiệm cận ngang là $y=1$ nên $a=1$.\\
		Đồ thị hàm số đi qua điểm $(0;2)$ nên $\dfrac{-b}{c}=2 \Leftrightarrow b=2$.\\
		Vậy $2a+b-3c = 2+2+3=7$.}
\end{ex} \dongcham{1}

\begin{ex}
	\immini{Cho hàm số $ y=\dfrac{ax+1}{bx-2} $ có đồ thị như hình vẽ. Tính $T=a+b$
		\haicot
		{\True $ T=2 $}
		{$ T=0 $}
		{$ T=-1 $}
		{$ T=3 $}}{
		\begin{tikzpicture}[scale=0.8, font=\footnotesize, line join=round, line cap=round, >=stealth,x=0.7cm,y=0.7cm]
			\def\xmin{-1.3}\def\xmax{6}\def\ymin{-2}\def\ymax{4}
			\draw[->] (\xmin-0.2,0)--(\xmax+0.4,0) node[below] {\footnotesize $x$};
			\draw[->] (0,\ymin-0.2)--(0,\ymax+0.4) node[right] {\footnotesize $y$};
			\draw (0,1) node [above left] {\footnotesize $1$};
			\draw (2,0) node [below right] {\footnotesize $2$};
			\draw (0,0) node [above left] {\footnotesize $O$};
			\foreach \x in {-1,1,3,4,5,6}\draw (\x,0.1)--(\x,-0.1) node [below] {\footnotesize $\x$};
			\foreach \y in {-2,-1,2,3,4}\draw (0.1,\y)--(-0.1,\y) node [left] {\footnotesize $\y$};
			\clip (\xmin,\ymin) rectangle (\xmax,\ymax);
			\draw[dashed] (\xmin,1.0)--(\xmax,1.0);
			\draw[dashed] (2.0,\ymin)--(2.0,\ymax);
			\draw[line width=1pt,smooth,samples=200,domain=\xmin:1.5] plot (\x,{(1*(\x)+1)/(1*(\x)+-2)});
			\draw[line width=1pt,smooth,samples=200,domain=2.3:\xmax] plot (\x,{(1*(\x)+1)/(1*(\x)+-2)});
		\end{tikzpicture}
	}
	\loigiai{
		Từ biểu thức của hàm số, suy ra tiệm cận đứng là $ x=\dfrac{2}{b} $, tiệm cận ngang là $ y=\dfrac{a}{b} $.\\
		Dựa vào hình vẽ, suy ra tiệm cận đứng $ x=2 $, tiệm cận ngang $ y=1 $.\\
		Từ hai điều trên suy ra $ a=1 $, $ b=1 $. Vậy $ T=1+1=2 $.
	}
\end{ex} \dongcham{1}

\begin{ex}
	\immini{
		Cho hàm số $y=\dfrac{ax-b}{cx+2}$ ($a$, $b$, $c\in\mathbb{R}$; $c\neq 0$) có đồ thị như hình vẽ bên. Giá trị của biểu thức $a+b+c$ bằng
		\choice
		{$-3$}
		{$5$}
		{$-4$}
		{\True $3$}
	}{
		\begin{tikzpicture}[scale=0.7, font=\footnotesize, line join=round, line cap=round, >=stealth]
			\def\a{1} \def\b{-3} \def\c{-1} \def\d{2} % Hệ số
			\def\xt{-2} \def\xp{6} \def\yt{2} \def\yd{-4} % x_trái, x_phải, y_trên, y_dưới (giới hạn)
			\draw[->] (\xt,0)--(\xp,0) node [below]{$x$};
			\draw[->] (0,\yd)--(0,\yt) node [left]{$y$};
			\fill (0,0) circle (1.5pt) node[above left]{$O$} (1,0) circle (1.5pt) node[below]{$1$} (2,0) circle (1.5pt) node[below left]{$2$} (3,0) circle (1.5pt) node[below]{$3$} (0,-1) circle (1.5pt) node[above left]{$-1$} (0,-1.5) circle (1.5pt) node[below left]{$-\dfrac{3}{2}$};
			\clip (\xt+0.1,\yd+0.1) rectangle (\xp-0.1,\yt-0.1);
			\draw[smooth,samples=300,domain=\xt:(-\d/\c-0.1)] plot(\x,{(\a*(\x)+\b)/(\c*(\x)+\d)});
			\draw[smooth,samples=300,domain=(-\d/\c+0.1:\xp)] plot(\x,{(\a*(\x)+\b)/(\c*(\x)+\d)});
			\draw[dashed] (-\d/\c,\yd)--(-\d/\c,\yt);
			\draw[dashed] (\xt,\a/\c)--(\xp,\a/\c);
		\end{tikzpicture}
	}
	\loigiai{
		Từ hình vẽ, ta thấy đồ thị hàm số có
		\begin{itemize}
			\item Đường tiệm cận đứng $x=2$, suy ra $-\dfrac{2}{c}=2 \Leftrightarrow c=-1$.
			\item Đường tiệm cận ngang $y=-1$, suy ra $\dfrac{a}{c}=-1 \Leftrightarrow a=-c=1$.
			\item Giao điểm với trục $Oy$ tại điểm $\left(0;-\dfrac{3}{2}\right)$, suy ra $-\dfrac{b}{2}=-\dfrac{3}{2} \Leftrightarrow b=3$.
		\end{itemize}
		Vậy $a+b+c=1+3-1=3$.
	}
\end{ex} \dongcham{1}

\begin{ex}
	\immini{Hãy xác định $a$, $b$ để hàm số $y = \dfrac{2 - ax}{x + b}$ có đồ thị như hình vẽ?
		\choice
		{$a = 1$; $b = - 2$}
		{$a = b = 2$}
		{\True $a = - 1$; $b = -2$}
		{$a = b = -2$}}{

		\begin{tikzpicture}[smooth,samples=300,scale=0.5,>=stealth]
			\draw[->] (-3.5,0)--(6.5,0) node[below]{$x$};
			\draw[->] (0,-2.5)--(0,5) node[right]{$y$};
			\draw (0,0) node[above left]{$O$};
			\draw[line width=1pt,domain=-3.5:0.8] plot(\x,{(\x+2)/(\x-2)});
			\draw[line width=1pt,domain=3:6.5] plot(\x,{(\x+2)/(\x-2)});
			\draw[fill=black] (0,1) circle(1.5pt) (-2,0) circle(1.5pt) (2,0) circle(1.5pt) (0,-1) circle(1.5pt);
			\draw [dashed](-3.5,1)--(6.5,1) (2,-2.5)--(2,5);
			\draw (0,-1) node[right]{$-1$};
			\draw (2,0) node[below right]{$2$};
			\draw (-2,0) node[below left]{$-2$};
			\draw (0,1) node[above right]{$1$};
		\end{tikzpicture}
	}
	\loigiai{
		Đồ thị hàm số có đường tiệm cận đứng là $x = 2$ nên $b + 2 = 0 \Leftrightarrow b = -2$.\\
		Đồ thị hàm số cắt trục hoành tại điểm $\left(-2; 0\right)$ nên $2 + 2a = 0 \Rightarrow a = -1$.
	}
\end{ex} \dongcham{1}


\begin{ex}
	\immini{Cho đồ thị hàm số $y=\dfrac{ax-b}{x-1}$ như hình vẽ. Tìm khẳng định đúng?
		\haicot
		{$a<0$, $b<0$}
		{$0<b<a$}
		{\True $b<0<a$}
		{$a<b<0$}}{
		\begin{tikzpicture}[>=stealth,scale=0.4, line join=round, line
				cap=round,font=\footnotesize]
			\draw[->] (-4,0)--(6,0) node [below]{$x$};
			\draw[->] (0,-4)--(0,6) node [right]{$y$};
			\draw[fill=black] (0,0) circle (2pt) node[below left]{$O$};
			\draw[smooth,samples=300,domain=-4:0.4] plot(\x,{(\x+2)/(\x-1)});
			\draw[smooth,samples=300,domain=1.6:6] plot(\x,{(\x+2)/(\x-1)});
			\draw[dashed] (-4,1)--(6,1) (1,-4)--(1,6);
			\draw[fill=black] (1,0) circle (2pt) node[below left]{$1$};
			\draw[fill=black] (0,1) circle (2pt) node[below left]{$1$};
			\draw[fill=black] (-2,0) circle (2pt) node[below left]{$-2$};
			\draw[fill=black] (0,-2) circle (2pt) node[below left]{$-2$};
		\end{tikzpicture}}
	\loigiai{
		Hàm số có dạng $y=\dfrac{ax-b}{x-1}$.
		\begin{itemize}
			\item Tiệm cận ngang $y=1 \Rightarrow a=1$.
			\item Đồ thị đi qua $(-2;0) \Rightarrow -2a-b=0$
		\end{itemize}
		Suy ra $b<0<a$.
	}
\end{ex} \dongcham{1}

\begin{ex}
	\immini{Cho hàm số $y=\dfrac{ax+4}{bx+c}\ (a,\ b,\ c\in \mathbb{R})$ có bảng biến thiên như sau. Trong các số $a,\ b,\ c$ có bao nhiêu số dương?
		\choice
		{$0$}
		{\True $1$}
		{$2$}
		{$3$}}{
		\begin{tikzpicture}
			\tikzset{double style/.append style = {draw=\tkzTabDefaultWritingColor,double=\tkzTabDefaultBackgroundColor,double distance=2pt}}
			\tkzTabInit[nocadre=false,lgt=1.2,espcl=2.5,deltacl=0.6]
			{$x$ /0.6, $f'(x)$ /0.6, $f(x)$ /1.5}
			{$-\infty$,$1$,$+\infty$}
			\tkzTabLine{ ,+,d,+, }
			\tkzTabVar{-/$3$,+D-/$+\infty$/$-\infty$,+/$3$}
		\end{tikzpicture}}
	\loigiai
	{
		Dựa vào bảng biến thiên, ta có $y(0)>3\Rightarrow\dfrac{4}{c}>0\Rightarrow c>0$.\\
		Đồ thị có tiệm cận đứng $x=1$ và tiệm cận ngang $y=3$ nên $\heva{& -\dfrac{c}{b}>0\\& \dfrac{a}{b}>0}\Rightarrow\heva{&b<0\\&a<0.}$\\
		Vậy $c>0$, $a<0$, $b<0$.
	}
\end{ex} \dongcham{4}

\begin{ex}%[2D1K5-1]%
	\immini
	{
		Cho hàm số $y=\dfrac{ax+b}{cx+d}$ với $a>0$ có đồ thị như hình vẽ bên. Mệnh đề nào sau đây đúng?
		\choice
		{$b<0$, $c<0$, $d<0$}
		{$b>0$, $c<0$, $d<0$}
		{$b<0$, $c>0$, $d<0$}
		{\True $b>0$, $c>0$, $d<0$}
	}
	{\begin{tikzpicture}[scale=0.7, font=\footnotesize, line join=round, line cap=round,>=stealth,x=0.4cm,y=0.4cm]
			\def \xmin{-5.0};
			\def \xmax{6.3};
			\def \ymin{-4.0};
			\def \ymax{5.5};
			\draw[->] (\xmin, 0.) -- (\xmax,0.) node[anchor=north] {$x$};
			\draw[->] (0.,\ymin) -- (0.,\ymax) node[anchor=west] {$y$};
			\clip(\xmin,\ymin) rectangle (\xmax,\ymax);
			\draw[smooth,samples=100,domain=\xmin-0.1:1-0.1] plot(\x,{((\x)+2)/((\x)-1)});
			\draw[smooth,samples=100,domain=1+0.1:\xmax-0.1] plot(\x,{((\x)+2)/((\x)-1)});
			\draw[dashed] (\xmin,1)--(\xmax,1) (1,\ymin)--(1,\ymax);
			\draw[fill=black] (0,0) circle (1pt) node[above left] {$O$};
		\end{tikzpicture}
	}
	\loigiai{
		Đồ thị hàm số có đường tiệm cận ngang $y=\dfrac{a}{c}$ nằm trên trục $Ox$ nên $\dfrac{a}{c}>0\overset{a>0}{\Rightarrow} c>0$.\\
		Đồ thị hàm số có đường tiệm cận đứng $x=-\dfrac{d}{c}$ nằm bên phải trục $Oy$ nên $-\dfrac{d}{c}>0\overset{c>0}{\Rightarrow}d<0$.\\
		Vậy mệnh đề đúng là \lq\lq $b>0$, $c>0$, $d<0$\rq\rq.
	}
\end{ex} \dongcham{4}

\begin{ex}
	\immini{Hình vẽ bên là đồ thị của hàm số $y=\dfrac{ax+b}{cx+d}$. Mệnh đề nào sau đây là đúng?
		\choice
		{$ab>0,bd<0$}
		{$ab<0,ad>0$}
		{\True $ab<0,ad<0$}
		{$bd>0,ad>0$}
	}{\begin{tikzpicture}[smooth,samples=300,line width=0.6pt,>=stealth, scale=0.5]
			\draw[->] (-4,0)--(4.5,0) node[below]{$x$};
			\draw[->] (0,-2)--(0,4.5) node[right]{$y$};
			\draw (0,0) node[below right]{$O$};
			\draw[dashed] (-0.5,-2)--(-0.5,4.5) (-4,1)--(4.5,1);
			\draw[line width=1pt,domain=-4:-0.8] plot(\x,{(2*(\x)-1)/(2*(\x)+1)});
			\draw[line width=1pt,domain=-0.15:4.5] plot(\x,{(2*(\x)-1)/(2*(\x)+1)});
		\end{tikzpicture}
	}
	\loigiai{
		Ta có
		\begin{itemize}
			\item [$\bullet$] Đường tiệm cận đứng $x=-\dfrac{d}{c}$. Theo hình vẽ thì $-\dfrac{d}{c}<0 \Rightarrow cd >0$ \quad (1).
			\item [$\bullet$] Đường tiệm cận ngang $y=\dfrac{a}{c}$. Theo hình vẽ thì $\dfrac{a}{c}<0 \Rightarrow ac <0$ \quad (2).
			\item [$\bullet$] Giao điểm với trục tung tại điểm có tung độ $y=\dfrac{b}{d}$. Theo hình vẽ thì $\dfrac{b}{d}>0 \Rightarrow bd >0$ \quad (3).
			\item [$\bullet$] Giao điểm với trục hoành tại điểm có hoành độ $x=-\dfrac{b}{a}$. Theo hình vẽ thì $-\dfrac{b}{a}>0 \Rightarrow ab <0$ \quad (4).
		\end{itemize}
		Lấy (3) nhân với (4), ta được $ad \cdot b^2 <0$. Suy ra $ad<0$.\\
		Mặt khác theo (4) thì $ab<0$.
	}
\end{ex} \dongcham{4}


\begin{ex}
	\immini{Hình vẽ dưới đây là đồ thị hàm số $y=\dfrac{ax+b}{cx+d}$ $ac\ne0$, $ad-cb\ne0$. Mệnh đề nào sau đây đúng?
		\choice
		{\True $ad>0$ và $ab<0$}
		{$bd<0$ và $ab>0$}
		{$ad<0$ và $ab<0$}
		{$ad>0$ và $bd>0$}
	}
	{\begin{tikzpicture}[>=stealth,font=\footnotesize,scale=0.6]
			\draw[->](-4,0)--(3,0)node[below]{$x$};
			\draw[->](0,-3.5)--(0,3)node[right]{$y$};
			\draw[smooth,samples=100,domain=-4:-1.4]plot(\x,{(\x-1)/(2*\x+2)});
			\draw[smooth,samples=100,domain=-0.75:3]plot(\x,{(\x-1)/(2*\x+2)});
			\draw(-4,0.5)--(3,0.5) (-1,-3.5)--(-1,3);
			\fill (0,0)node[below left]{$O$}circle (1.2pt);
		\end{tikzpicture}}
	\loigiai{
		\begin{itemize}
			\item Đồ thị hàm số cắt trục $Oy$ tại điểm có tung độ âm $\Rightarrow\dfrac{b}{d}< 0\Rightarrow bd<0$.
			\item Đồ thị hàm số cắt trục $Ox$ tại điểm có hoành độ dương $\Rightarrow-\dfrac{b}{a}> 0\Rightarrow ab<0$.
			\item Đồ thị hàm số có tiệm cận ngang $y=\dfrac{a}{c}>0\Rightarrow ac>0.\quad(1)$
			\item Đồ thị hàm số có tiệm cận đứng $x=-\dfrac{d}{c}<0\Rightarrow cd>0.\quad(2)$
		\end{itemize}
		Từ $(1)$ và $(2)\Rightarrow ad>0$.}
\end{ex} \dongcham{4}
\Closesolutionfile{ans}

\ind{PHẦN II.} \inden{Câu trắc nghiệm đúng sai. Trong mỗi ý a), b), c), d) ở mỗi câu, học sinh chọn đúng hoặc sai.}\\
\Opensolutionfile{ans}[ans/2D1-B4-d2-2]

\begin{ex}
	\immini{Cho hàm số $y = \dfrac{x + a}{b x +c}$, $\left( a, b, c \in \mathbb{Z}\right) $.
		\choiceTF
		{\True Đồ thị hàm số có tiệm cận đứng $x=1$}
		{Đồ thị hàm số có tiệm cận ngang $y=0$}
		{Hàm số đồng biến trên $\mathbb{R}$}
		{\True $a - 3b - 2c=-3$}
	}{
		\begin{tikzpicture}[font=\footnotesize,line join=round, line cap=round,>=stealth,scale=0.7]
			\tikzset{label style/.style={font=\footnotesize}}
			\def \xmin{-2.7}
			\def \xmax{4}
			\def \ymin{-2.2}
			\def \ymax{4}
			\draw[->] (\xmin,0)--(\xmax,0) node[below left] {$x$};
			\draw[->] (0,\ymin)--(0,\ymax) node[below left] {$y$};
			\draw (0,0) node [below left] {$O$};
			\draw (1,0) node [below left] {$1$} circle (1.2pt);
			\draw (2,0) node [below right] {$2$} circle (1.2pt);
			\draw (0,1) node [above left] {$1$} circle (1.2pt);
			\draw (0,2) node [above left] {$2$} circle (1.2pt);
			\begin{scope}
				\clip (\xmin+0.01,\ymin+0.01) rectangle (\xmax-0.01,\ymax-0.01);
				\draw[samples=350,domain=\xmin+0.01:\xmax-0.01,smooth,variable=\x] plot (\x,{(\x-2)/(\x-1)});
				\draw[samples=200,domain=\xmin+0.01:\xmax-0.01,smooth,variable=\x] plot (\x,{1});
			\end{scope}
		\end{tikzpicture}
	}
	\loigiai{
		Căn cứ vào đồ thị, ta có
		\begin{enumerate}[a)]
			\item Đồ thị hàm số có tiệm cận đứng $x=1$.
			\item Đồ thị hàm số có tiệm cận ngang $y=1$
			\item Hàm số đồng biến trên các khoảng $(-\infty,1)$ và $(1;+\infty)$
			\item Đồ thị hàm số có tiệm cận ngang $y = 1$ nên $\dfrac{1}{b} = 1 \Rightarrow b = 1$.\\
			      Đồ thị hàm số có tiệm cận đứng $x = 1$ nên $-\dfrac{c}{b} = 1$ mà $b = 1$ $\Rightarrow c = -1$.\\
			      Đồ thị hàm số cắt trục tung tại điểm $(0; 2)$ nên $\dfrac{a}{c} = 2$ mà $c = -1$ nên $a = -2$.\\
			      Vậy $T = a - 3b - 2c = -2 - 3 \cdot 1 -2 \cdot (-1) =-3 $.
		\end{enumerate}

	}
\end{ex} \dongcham{4}

\begin{ex}%[2D1K5-1]%
	Cho hàm số $ f(x)=\dfrac{a x-1}{b x+c}\ (a, b, c\in\mathbb{R})$ có bảng biến thiên như sau.
	\begin{center}
		\begin{tikzpicture}
			\tikzset{double style/.append style = {draw=\tkzTabDefaultWritingColor,double=\tkzTabDefaultBackgroundColor,double distance=2pt}}
			\tkzTabInit[espcl=2.5,lgt=1.2,nocadre=false]
			{$x $ /0.7, $ f'(x)$ /0.7, $ f(x)$ /2.1}
			{$-\infty $, $ 3 $, $+\infty$}
			\tkzTabLine{,-,d,-,}
			\tkzTabVar{+/ $\dfrac{1}{2}$,-D+/ $-\infty $ / $+\infty $,-/ $\dfrac{1}{2}$}
		\end{tikzpicture}
	\end{center}
		\choiceTF
		{\True Hàm số nghịch biến trên khoảng $\left( -\infty,\dfrac{1}{2}\right)$}
		{Đồ thị hàm số có tiệm cận đứng $x=\dfrac{1}{2}$}
		{\True Đồ thị giao với trục hoành tại điểm có hoành độ nhỏ hơn $3$}
		{\True $\hoac{&b>\dfrac{2}{3}\\ &b<0}$}
	\loigiai{
		\begin{enumerate}[a)]
			\item Hàm số đồng biến trên các khoảng $(-\infty,3)$ nên nghịch biến trên khoảng $\left( -\infty,\dfrac{1}{2}\right)$.
			\item Đồ thị hàm số có tiệm cận đứng $x=3$.
			\item Đồ thị giao với trục hoành tại điểm thuộc nhánh trái của đồ thị, suy ra hoành độ giao điểm này nhỏ hơn $3$.
			\item Từ bảng biến thiên suy ra
			      \[
				      \heva{&\dfrac{a}{b}=\dfrac{1}{2}\\&-\dfrac{c}{b}=3.}\quad\quad (1)
			      \]
			      Ta có $ y'=\dfrac{ac+b}{(bx+c)^2}<0 $, $\forall x\ne-\dfrac{c}{b}\Leftrightarrow ac+b<0 $.\quad\quad (2)\\
			      Từ (1) và (2) suy ra $\dfrac{b}{2}\cdot (-3b)+b<0\Leftrightarrow\hoac{&b>\dfrac{2}{3}\\ &b<0.}$
		\end{enumerate}
	}
\end{ex} \dongcham{4}

\begin{ex}
	\immini{Cho hàm số $f(x)=\dfrac{ax+b}{cx+d}$ với $a$, $b$, $c$, $d \in \mathbb{R}$ có đồ thị hàm số $y=f'(x)$ nhận $x=-1$ làm tiệm cận đứng như hình vẽ bên. Biết rằng giá trị lớn nhất của hàm số $y=f(x)$ trên đoạn $[-3;-2]$ bằng $8$.
		\choiceTF
		{\True $f'(0)=3$}
		{Hàm số $f(x)$ nghịch biến trên khoảng $(-1;+\infty)$}
		{Giá trị của $f(-3)$ bằng $8$}
		{\True  Giá trị của $f(2)$ bằng $4$}
	}
	{\begin{tikzpicture}[>=stealth,scale=0.6, line join=round, line cap=round]
			\def\a{3} \def\b{0} \def\c{1} \def\d{1} % Hệ số
			\def\xt{-4.5} \def\xp{4.5} \def\yt{5.5} \def\yd{-1}
			\draw[->] (\xt,0)--(\xp,0) node [below]{$x$};
			\draw[->] (0,\yd)--(0,\yt) node [left]{$y$};
			\node at (0,0) [below left]{$O$};
			\clip (\xt-0.1,\yd+0.1) rectangle (\xp-0.1,\yt-0.1);
			\draw[smooth,samples=300,domain=\xt:(-\d/\c-0.1)] plot(\x,{(\a)/(\c*(\x)+\d)^2});
			\draw[smooth,samples=300,domain=(-\d/\c+0.1:\xp)] plot(\x,{(\a)/(\c*(\x)+\d)^2});
			\draw (-\d/\c,\yd)--(-\d/\c,\yt);
			\draw (-1,0) node [below left] {$-1$} circle (1.2pt)
			(0,3) node [right] {$3$} circle (1.2pt);
		\end{tikzpicture}}
	\loigiai{
		\begin{enumerate}[a)]
			\item Theo hình vẽ, đồ thị $f'(x)$ qua điểm $(0;3)$ nên $f'(0)=3$.
			\item Do $f'(x)>0$, $\forall x \ne -1$ nên hàm số $f(x)$ đồng biến trên các khoảng $(-\infty;-1)$ và $(-1;+\infty)$.
			\item Vì $f'(x)>0$, $\forall x \ne -1 \Rightarrow \max \limits_{[-3;-2]} f(x)=f(-2)=8$. Suy ra $f(-3) \ne 8$.
			\item Ta có $f'(x)=\dfrac{ad-bc}{(cx+d)^2}$.\\
			      Đồ thị hàm số đi qua điểm $(0;3)$ nên $f'(0)=3 \Leftrightarrow \dfrac{ad-bc}{d^2}=3$.\\
			      Mặt khác, đồ thị hàm số $y=f'(x)$ có tiệm cận đứng $x=-1$ nên $-c+d=0$.\\
			      Vì $f'(x)>0$, $\forall x \ne -1 \Rightarrow \max \limits_{[-3;-2]} f(x)=f(-2)=8 \Leftrightarrow \dfrac{-2a+b}{-2c+d}=8$.\\
			      Vậy ta có hệ phương trình $\heva{&ad-bc=3d^2\\&-c+d=0\\&b-2a=8(d-2c)}\Leftrightarrow\heva{&c=d\\&a-b=3d\\&b-2a=-8d}\Leftrightarrow\heva{&a=5d\\&b=2d\\&c=d.}$\\
			      Từ đó suy ra $f(x)=\dfrac{5\mathrm{\,d}x+2d}{\mathrm{\,d}x+d}=\dfrac{5x+2}{x+1} \Rightarrow f(2)=4$.
		\end{enumerate}
	}
\end{ex} \dongcham{5}
\Closesolutionfile{ans}
% \begin{dang}{Khảo sát và vẽ đồ thị hàm số phân thức hữu tỉ bậc II/I}
	\begin{itemize}
		\item[\iconCH] \indamm{Bước 1.} Tập xác định $D=\mathbb{R}\backslash \left\{-\dfrac{n}{m}\right\}$.
		\item [\iconCH] \indamm{Bước 2.} Khảo sát sự biến thiên của hàm số
		\begin{itemize}
			\item Tính đạo hàm $y'=\dfrac{am\cdot x^2+2an\cdot x + b.n - m.c}{(mx+n)^2}$. Giải $y'=0 \Leftrightarrow am\cdot x^2+2an\cdot x + b.n - m.c=0$, tìm nghiệm.
			\item Tìm các giới hạn tại vô cực, giới hạn vô cực và tìm tiệm cận của đồ thị hàm số.
			\item Lập bảng biến thiên; xác định chiều biến thiên và cực trị của hàm số.
		\end{itemize}
		\item [\iconCH] \indamm{Bước 3.} Cho thêm điểm và vẽ đồ thị của hàm số dựa vào bảng biến thiên.
	\end{itemize}
\end{dang}
\boxmini{BÀI TẬP TỰ LUẬN}
\begin{vd}
	Khảo sát sự biến thiên và vẽ đồ thị các hàm số sau:
	\begin{tasks}(3)
		\task $ y = \dfrac{x^2+ 2x - 2}{x - 1}$;
		\task $y=-x+2-\dfrac{1}{x+1}$;
		\task $y=\dfrac{-x^2-3x+4}{x+2}$.
	\end{tasks}
\loigiai{
\begin{enumerate}[a)]
	\item Ta viết lại hàm số $ y = \dfrac{x^2+ 2x - 2}{x - 1}=x+3+\dfrac{1}{x-1}$.\\
	Tập xác định: $D = \mathbb{R} \setminus \left\{ 1 \right\}$.\\
	Sự biến thiên:
	\begin{itemize}
		\item [$\bullet$] Đạo hàm $y'= \dfrac{x^2 - 2x}{(x - 1)^2}$; $y' = 0 \Leftrightarrow x = 0$  hoặc $x = 2$.
		\item [$\bullet$] Giới hạn và tiệm cận:\\
		$\displaystyle\lim \limits{n \to +\infty}_{x \rightarrow-\infty} y=-\infty,\, \displaystyle\lim \limits{n \to +\infty}_{x \rightarrow +\infty} y=+\infty$.\\
		$\displaystyle\lim \limits{n \to +\infty}_{x \rightarrow 1^{-}} y= -\infty,\, \displaystyle\lim \limits{n \to +\infty}_{x \rightarrow 1^{+}} y=+\infty$. Suy ra $x=1$ là tiệm cận đứng.\\
		$\displaystyle\lim \limits{n \to +\infty}_{x \rightarrow -\infty} \left(y-(x+3) \right) = 0,\, \displaystyle\lim \limits{n \to +\infty}_{x \rightarrow +\infty} \left(y-(x+3) \right) = 0$. Suy ra $y=x+3$ là tiệm cận xiên.
		\item [$\bullet$] Bảng biến thiên:
		\begin{center}
			\begin{tikzpicture}[scale=1, font=\footnotesize, line join=round, line cap=round, >=stealth]
				\tikzset{double style/.append style = {draw=\tkzTabDefaultWritingColor,double=\tkzTabDefaultBackgroundColor,double distance=2pt}}
				\tkzTabInit[nocadre=false,lgt=1.2,espcl=2.2,deltacl=0.6]
				{$x$ /0.6,$y'$ /0.6,$y$ /1.6}
				{$-\infty$,$0$,$1$,$2$,$+\infty$}
				\tkzTabLine{,+,0,-,d,-,0,+,}
				\tkzTabVar{-/$-\infty$,+/$2$,-D+/$-\infty$/$+\infty$,-/$6$,+/$+\infty$}
			\end{tikzpicture}
		\end{center}
	Hàm số đồng biến trên khoảng $(-\infty;0)$ và $(2;+\infty)$; nghịch biến trên khoảng $(0;1)$ và $(1;2)$.\\
	Hàm số đạt cực tiểu tại $x = 2$  và ${y_{CT}} = 6$ .\\
	Hàm số đạt cực đại tại $x = 0$ và ${y_{CĐ}} = 2$ .\\
	\end{itemize}
	Đồ thị:\\
	\immini{
		\begin{itemize}
			\item [$\bullet$] Đồ thị hàm số giao với trục $Ox$ tại điểm $(-1+\sqrt{3}; 0)$ và điểm $(-1-\sqrt{3}; 0)$.
			\item [$\bullet$] Đồ thị nhận $I(1;4)$ làm tâm đối xứng.
		\end{itemize}
	}{
	\begin{tikzpicture}[line join=round, line cap=round,>=stealth,thick,x=0.8cm,y=0.8cm]
		\tikzset{every node/.style={scale=0.9}}
		\draw[->] (-3.8,0)--(5.6,0) node[below] {$x$};
		\draw[->] (0,-1.1)--(0,7.6) node[below left] {$y$};
		\draw (0,0) node [below left] {$O$};
		\draw (1,4) circle (1pt) node [below right] {$I$};
		\draw (1,-1.5) node [right] {Hình 5};
		\draw[dashed,thin] (1.01,-1)--(1.01,7.5) node [pos=0.4,sloped,black,below] {$x=1$} ;
		\begin{scope}
			\clip (-4,-1) rectangle (6.5,7.5);
			\draw[samples=200,domain=-3.5:0.99,smooth,variable=\x] plot (\x,{(1*((\x)^2)+2*(\x)+-2)/(1*(\x)+-1)});
			\draw[samples=200,domain=1.01:6,smooth,variable=\x] plot (\x,{(1*((\x)^2)+2*(\x)+-2)/(1*(\x)+-1)});
			\draw[dashed,thin] (-3.6,-0.6)--(4.1,7.1) node [pos=0.8,sloped,black,below] {$y=x+3$};
		\end{scope}
		\foreach \x/\g in {-3/-90,-2/-90,-1/-90,1/-60,2/-90,3/-90,4/-90,5/-90}
		\draw[thin] (\x,2pt)--(\x,-2pt) + (\g:3mm) node [scale=0.8] {$\x$};
		%Vẽ các điểm trên trục Oy
		\foreach \y/\g in {1/180,2/140,3/180,6/180,4/180,5/180}
		\draw[thin] (2pt,\y)--(-2pt,\y) + (\g:3mm) node [scale=0.8] {$\y$};
\end{tikzpicture}}

	\item Tập xác định: $\mathscr{D}=\mathbb{R} \backslash\{-1\}$.\\
	Sự biến thiên:
	\begin{itemize}
		\item [$\bullet$] Đạo hàm $y'=-1+\dfrac{1}{(x+1)^2}$, $y'=0\Leftrightarrow x=-2$ hoặc $x=0$.
		\item [$\bullet$] Giới hạn và tiệm cận:\\
		\begin{itemize}
			\item $\lim\limits_{x \rightarrow +\infty} y= -\infty, \lim\limits_{x \rightarrow -\infty} y= +\infty$.
			\item $\lim\limits_{x \rightarrow (-1)^{-}} y= +\infty, \lim\limits_{x \rightarrow (-1)^{+}} y= -\infty$.
		\end{itemize}
		Do đó, đường thẳng $x=-1$ là tiệm cận đứng của đồ thị hàm số.
		\begin{itemize}
			\item $\lim\limits_{x \rightarrow+\infty}[y - (-x+2)]=\lim\limits_{x \rightarrow +\infty} \dfrac{-1}{x+1}=0$,
			\item $ \lim\limits_{x \rightarrow-\infty}[y - (-x+2)]=\lim\limits_{x \rightarrow +\infty} \dfrac{-1}{x+1}=0$.
		\end{itemize}
		Do đó, đường thẳng $y= -x+2 $ là tiệm cận xiên của đồ thị hàm số.
		\item [$\bullet$] Bảng biến thiên:
		\begin{center}
			\begin{tikzpicture}
				\tikzset{double style/.append style = {draw=\tkzTabDefaultWritingColor,double=\tkzTabDefaultBackgroundColor,double distance=2pt}}
				\tkzTabInit[lgt=1.2, espcl=2.5, deltacl=0.6]
				{$x$/0.6, $y'$/0.6, $y$/2}
				{$-\infty$, $-2$, $-1$, $0$, $+\infty$}
				\tkzTabLine{, -, 0, +, d, +, 0, -, }
				\tkzTabVar{+/$+\infty$, -/$5$, +D-/$+\infty$/$-\infty$, +/$1$, -/$-\infty$}
			\end{tikzpicture}
		\end{center}
		Hàm số đồng biến trên các khoảng $(-2;-1)$, $(-1;0)$ và nghịch biến trên các khoảng $(-\infty;-2)$, $(0;+\infty)$.\\
		Hàm số đạt cực tiểu tại $x=-2$, $y_{_\text{CT}}=5$; đạt cực đại tại $x=0$, $y_{_\text{CĐ}}=1$.\\
	\end{itemize}
	Đồ thị:\\
	\immini{
		\begin{itemize}
			\item [$\bullet$] Đồ thị hàm số qua các điểm $\left(-3;-\dfrac{11}{2} \right)$, $\left(3;-\dfrac{5}{4} \right)$.
			\item [$\bullet$] Đồ thị nhận $I(-1;3)$ làm tâm đối xứng.
		\end{itemize}
	}{
		\begin{tikzpicture}[>=stealth, scale=0.6, font=\footnotesize,x=1cm,y=1cm]
			\draw[->] (-5,0)--(5,0) node[below] {$x$};
			\draw[->] (0,-4)--(0,8.5) node[left] {$y$};
			\draw[domain=-0.85:3.8, smooth] plot (\x, {-(\x)^2+(\x)+1)/(\x+1)});
			\draw[domain=-4:-1.2, smooth] plot (\x, {-(\x)^2+(\x)+1)/(\x+1)});
			\draw[domain=-4.5:4, smooth] plot (\x, {-\x+2});
			\draw (-1,-4)--(-1,8.2);
			\draw[fill=black] (0,0) node[below left=-0.1] {$O$} circle (1.2pt);
			\draw[fill=black] (1,0) node[below] {$1$} circle (1.2pt);
			\draw[fill=black] (2,0) node[above] {$2$} circle (1.2pt);
			\draw[fill=black] (-1,0) node[above] {$-1$} circle (1.2pt);
			\draw[fill=black] (-2,0) node[below ] {$-2$} circle (1.2pt);
			\draw[fill=black] (3,0) node[above ] {$3$} circle (1.2pt);
			%	\draw[fill=black] (-4,0) node[below] {$-4$} circle (1.2pt);
			\draw[fill=black] (0,5) node[right] {$5$} circle (1.2pt);
			%		\draw[fill=black] (0,17) node[right] {$17$} circle (1.2pt);
			\draw[fill=black] (0,-1.25) node[ left] {$-\dfrac{5}{4}$} circle (1.2pt);
			\draw[fill=black] (0,1) node[above right] {$1$} circle (1.2pt);
			\draw[dashed] (-2,0)--(-2,5)--(0,5) (3,0)--(3,-1.25)--(0,-1.25) (1,0)--(1,0.5)--(0,0.5) ;
	\end{tikzpicture}}
	
	\item Ta viết lại hàm số $ y = \dfrac{x^2+ 2x - 2}{x - 1}=x+3+\dfrac{1}{x-1}$.\\
	Tập xác định: $D=\mathbb{R} \setminus\{-2\}$.\\
	Sự biến thiên:
	\begin{itemize}
		\item [$\bullet$] Đạo hàm Đạo hàm $y'=\dfrac{-x^2-4x-10}{(x+2)^2}<0$, với mọi $x \ne -2$.
		\item [$\bullet$] Giới hạn và tiệm cận:\\
		$$
		\lim\limits_{x \to-\infty} y=\lim\limits_{x \to-\infty} \dfrac{-x^2-3 x+4}{x+2}=+\infty; \lim\limits_{x \to+\infty} y=\lim\limits_{x \to+\infty} \dfrac{-x^2-3 x+4}{x+2}=-\infty.
		$$		
		Ta có 
		\begin{itemize}
			\item $a=\lim\limits_{x \to+\infty} \dfrac{-x^2-3x+4}{x^2+2x}=-1$.
			\item $b=\lim\limits_{x \to+\infty}\left[\dfrac{-x^2-3x+4}{x+2}-(-1) x\right]=\lim\limits_{x \to+\infty}\left(\dfrac{-x+4}{x+2}\right)=-1$.
		\end{itemize}	
		Suy ra đường thẳng $y=-x-1$ là tiệm cận xiên của đồ thị hàm số.\\		
		Ta có $\lim\limits_{x \to-2^{-}} y=\lim\limits_{x \to-2^{-}} \dfrac{-x^2-3x+4}{x+2}=-\infty; \lim\limits_{x \to-2^{+}} y=\lim\limits_{x \to-2^{+}} \dfrac{-x^2-3x+4}{x+2}=+\infty$. Suy ra đường thẳng $x=-2$ là tiệm cận đứng của đồ thị hàm số.
		\item [$\bullet$] Bảng biến thiên:
		\begin{center}
			\begin{tikzpicture}
				\tikzset{double style/.append style = {draw=\tkzTabDefaultWritingColor,double=\tkzTabDefaultBackgroundColor,double distance=2pt}}
				\tkzTabInit[nocadre=false,espcl=3,lgt=1.5]
				{$x$/0.7,$y'$/0.7,$y$/2.1}
				{$-\infty$,$-2$,$+\infty$}
				\tkzTabLine{,-,d,-,}
				\tkzTabVar{+/$+\infty$,-D+/$-\infty$/$+\infty$,-/$-\infty$}
			\end{tikzpicture}
		\end{center}
		Hàm số nghịch biến trên khoảng $(-\infty;-2)$ và $(-2;+\infty)$.\\
		Hàm số không có cực trị.
	\end{itemize}
	Đồ thị:\\
	\immini{
		\begin{itemize}
			\item [$\bullet$] Đồ thị hàm số giao với trục $Ox$ tại điểm $(-4; 0)$ và điểm $(1; 0)$.
			\item [$\bullet$] Đồ thị nhận $I(-2;1)$ làm tâm đối xứng.
		\end{itemize}
	}{
		\begin{tikzpicture}[>=stealth, scale=0.6, font=\footnotesize]
			\draw[->] (-7,0)--(5.5,0) node[below] {$x$};
			\draw[->] (0,-7)--(0,7) node[left] {$y$};
			\draw[domain=-1.1:5, smooth] plot (\x, {-(\x)^2-3*(\x)+4)/(\x+2)});
			\draw[domain=-6.4:-2.7, smooth] plot (\x, {-(\x)^2-3*(\x)+4)/(\x+2)});
			\draw[domain=-6:5, smooth] plot (\x, {-\x-1});
			\draw (-2,-7)--(-2,7);
			\draw[fill=black] (0,0) node[below left=-0.1] {$O$} circle (1.2pt);
			\draw[fill=black] (1,0) node[below] {$1$} circle (1.2pt);
			\draw[fill=black] (-1,0) node[below left] {$-1$} circle (1.2pt);
			\draw[fill=black] (2,0) node[below right=0 and -0.1] {$2$} circle (1.2pt);
			\draw[fill=black] (4,0) node[above] {$4$} circle (1.2pt);
			\draw[fill=black] (0,6) node[right] {$6$} circle (1.2pt);
			\draw[fill=black] (0,2) node[below left] {$2$} circle (1.2pt);
			\draw[fill=black] (0,-2.8) node[left] {$-\dfrac{14}{5}$} circle (1.2pt);
			\draw[fill=black] (0,-4) node[left] {$-4$} circle (1.2pt);
			\draw[dashed] (3,0)--(3,-2.8)--(0,-2.8) (4,0)--(4,-4)--(0,-4) (-1,0)--(-1,6)--(0,6);
			\node [above=-1mm, fill=white,font=\footnotesize] at (1.5,-7) {\it Hình $6$};
	\end{tikzpicture}}

\end{enumerate}}
\end{vd}

\boxmini{BÀI TẬP TRẮC NGHIỆM}
\ind{PHẦN I.} \inden{Câu trắc nghiệm nhiều phương án lựa chọn. Mỗi câu hỏi học sinh chỉ chọn một phương án.}\\
\setcounter{ex}{0}
\Opensolutionfile{ans}[ans/2D1-B4-d3-1]
\begin{ex}
	\immini{Bảng biến thiên sau là của một trong bốn hàm số sau. Hỏi đó là hàm số nào?
	\choice
	{$y=\dfrac{x^2-3x+4}{-x-4}$}
	{\True $y=\dfrac{x^2-4x+4}{-x-4}$}
	{$y=\dfrac{x^2-5x+4}{x+4}$}
	{$y=\dfrac{x^2-4x+4}{x+4}$}}{
	\begin{tikzpicture}
		\tikzset{double style/.append style = {draw=\tkzTabDefaultWritingColor,double=\tkzTabDefaultBackgroundColor,double distance=2pt}}
		\tkzTabInit[nocadre=false,lgt=1,espcl=1.6]
		{$x$ /0.6,$y'$ /0.6,$y$ /1.5}
		{$-\infty$,$-10$,$-4$,$2$,$+\infty$}
		\tkzTabLine{,-,$0$,+,d,+,$0$,-,}
		\tkzTabVar{+/$+\infty$,-/$24$,+D-/$+\infty$/$-\infty$,+/$0$,-/$-\infty$}
	\end{tikzpicture}
}
\loigiai{
}
\end{ex}

\begin{ex}
	\immini{Bảng biến thiên sau là của một trong bốn hàm số sau. Hỏi đó là hàm số nào?
		\choice
		{$y=\dfrac{x^2-4x+3}{x-3}$}
		{$y=\dfrac{-x^2-x+2}{x-3}$}
		{\True $y=\dfrac{-x^2+x+2}{x-3}$}
		{$y=\dfrac{x^2-4x+4}{-x+3}$}}{
		\begin{tikzpicture}
			\tikzset{double style/.append style = {draw=\tkzTabDefaultWritingColor,double=\tkzTabDefaultBackgroundColor,double distance=2pt}}
			\tkzTabInit[nocadre=false,lgt=1,espcl=1.6]
			{$x$ /0.6,$y'$ /0.6,$y$ /1.5}
			{$-\infty$,$1$,$3$,$5$,$+\infty$}
			\tkzTabLine{,-,$0$,+,d,+,$0$,-,}
			\tkzTabVar{+/$+\infty$,-/$-1$,+D-/$+\infty$/$-\infty$,+/$-9$,-/$-\infty$}
		\end{tikzpicture}
	}
\loigiai{
}
\end{ex}

\begin{ex}
	\immini{Bảng biến thiên sau là của một trong bốn hàm số sau. Hỏi đó là hàm số nào?
		\choice
		{\True $y=\dfrac{x^2-2x+1}{x+4}$}
		{$y=\dfrac{x^2-4x+2}{x+4}$}
		{$y=\dfrac{x^2-x+2}{-x-4}$}
		{$y=\dfrac{x^2-3x+4}{-x-4}$}}{
		\begin{tikzpicture}
			\tikzset{double style/.append style = {draw=\tkzTabDefaultWritingColor,double=\tkzTabDefaultBackgroundColor,double distance=2pt}}
			\tkzTabInit[nocadre=false,lgt=1,espcl=1.6]
			{$x$ /0.7,$y'$ /0.7,$y$ /2}
			{$-\infty$,$-9$,$-4$,$1$,$+\infty$}
			\tkzTabLine{,+,$0$,-,d,-,$0$,+,}
			\tkzTabVar{-/$-\infty$,+/$-20$,-D+/$-\infty$/$+\infty$,-/$0$,+/$+\infty$}
		\end{tikzpicture}
	}
\loigiai{
}
\end{ex}

\begin{ex}
	\immini{Bảng biến thiên sau là của một trong bốn hàm số sau. Hỏi đó là hàm số nào?
		\choice
		{$y=\dfrac{x^2-3}{x-2}$}
		{\True $y=\dfrac{x^2-4x+2}{x-2}$}
		{$y=\dfrac{x^2-x}{x-2}$}
		{$y=\dfrac{x^2-4x+5}{x-2}$}}{
			\begin{tikzpicture}
				\tkzTabInit[nocadre=false,lgt=1,espcl=3]
				{$x$ /0.6,$y'$ /0.6,$y$ /2}
				{$-\infty$,$2$,$+\infty$}
				\tkzTabLine{,+,d,+,}
				\tkzTabVar{-/$-\infty$,+D-/$+\infty$/$-\infty$,+/$+\infty$}
			\end{tikzpicture}
	}
\loigiai{
}
\end{ex}

\begin{ex}
	\immini{Đồ thị hình bên là của một trong bốn hàm số sau. Hỏi đó là hàm số nào?
		\choice
		{$y=\dfrac{x^2+x-1}{x-1}$}
		{\True $y=\dfrac{x^{2}-x+1}{x-1}$}
		{$y=\dfrac{x^2-4x-1}{-x+1}$}
		{$y=\dfrac{x^2-3x-1}{-x+1}$}}{
		\begin{tikzpicture}[line cap=butt,line join=miter,>=stealth,scale=0.4,font=\footnotesize]
			\tikzset{declare function={xmin=-3.5;xmax=4.7;ymin=-3.5;ymax=6;},
				smooth,samples=450}
			\draw[->] (xmin,0)--(xmax,0) node[shift={(0:7pt)}]{$ x $};
			\draw[->] (0,ymin-.2)--(0,ymax) node[shift={(90:7pt)}]{$ y $};
			\fill (0,0) node[shift={(140:6pt)}]{$ O $};
			\clip (xmin,ymin) rectangle (xmax,ymax);
			\foreach \i in {-3,-2,2,3,4}{
				\draw(\i,1.5pt)--(\i,-1.5pt)node[below]{$\i$};}	
			\foreach \j in {-2,1,2,3,4,5}{
				\draw(-1.5pt,\j)--(1.5pt,\j) node[left]{$\j$};}
			\draw(-1.5pt,-1)--(1.5pt,-1)node[shift={(160:6.5pt)}]{$-1$};
			\draw(1,-1.5pt)--(1,1.5pt)node[shift={(-75:7pt)}]{$1$};
			\draw(-1,-1.5pt)--(-1,1.5pt)node[shift={(100:5pt)}]{$-1$};
			\def\f(#1){((#1)^2-(#1)+1)/((#1)-1)}
			\def\a{-1}
			\def\b{0}
			\def\c{0.5}
			\def\d{1.5}	
			\def\e{2}
			\def\g{3}	
			\pgfmathsetmacro\fa{\f(\a)}
			\pgfmathsetmacro\fb{\f(\b)}
			\pgfmathsetmacro\fc{\f(\c)}
			\pgfmathsetmacro\fd{\f(\d)}	
			\pgfmathsetmacro\fe{\f(\e)}
			\pgfmathsetmacro\fg{\f(\g)}	
			\draw[samples=100] plot[domain=-5.3:0.9] (\x,{\f(\x)});	
			\draw[samples=100] plot[domain=1.05:5.2] (\x,{\f(\x)});
			\draw[] (1,ymin)--(1,ymax) node [pos=0.95,sloped, above]{$x=1$};
			\draw[] (xmin,ymin)--(6,ymax) node [pos=0.08,sloped, above]{$y=x$};
	\end{tikzpicture}
	}
\loigiai{
}
\end{ex}

\begin{ex}
	\immini{Đồ thị hình bên là của một trong bốn hàm số sau. Hỏi đó là hàm số nào?
		\choice
		{$y=\dfrac{x^2-x}{x+1}$}
		{$y=\dfrac{x^2-3x}{x+1}$}
		{$y=\dfrac{x^2+1x+2}{x+1}$}
		{\True $y=\dfrac{-x^{2}}{x+1}$}}{
		\begin{tikzpicture}[line cap=butt,line join=miter,>=stealth,scale=0.35,font=\footnotesize]
			\tikzset{declare function={xmin=-6.2;xmax=4.8;ymin=-4.6;ymax=7.8;},
				smooth,samples=450}
			\draw[->] (xmin,0)--(xmax,0) node[shift={(0:7pt)}]{$ x $};
			\draw[->] (0,ymin)--(0,ymax) node[shift={(90:7pt)}]{$ y $};
			\fill (0,0) node[shift={(140:5pt)}]{$ O $};
			\clip (xmin,ymin-.7) rectangle (xmax,ymax);
			\foreach \i in {-2,2}{
				\draw(\i,1.5pt)--(\i,-1.5pt)node[below]{$\i$};}
			\foreach \j in {-2,2,4}{
				\draw(-1.5pt,\j)--(1.5pt,\j) node[right]{$\j$};}	
			\def\f(#1){(-(#1)^2)/((#1)+1)} % Hàm số
			\def\q(#1){(-(#1)+1)} % Tiệm cận xiên
			\def\a{0}
			\def\b{-2}	
			\pgfmathsetmacro\fa{\f(\a)}
			\pgfmathsetmacro\fb{\f(\b)}	
			\draw[samples=250] plot[domain=-7.4:-1.1] (\x,{\f(\x)});	
			\draw[samples=250] plot[domain=-0.9:15] (\x,{\f(\x)});
			\draw[] plot [domain=-7.4:7] (\x,{\q(\x)}) ;
			\draw[] (-1,ymin)--(-1,ymax) node[sloped,pos=0.9,below] {$x=-1$};
			\foreach \x/\y in {\a/\fa,\b/\fb}{	
				\draw[dashed] (\x,0)|-(0,\y);}
			\foreach \x/\y in {\a/\fa,\b/\fb}{
				\fill[white,draw=black] (\x,\y) circle (1pt);}
			\fill[white,draw=black] (-1,2) circle (1pt) node[text=black,shift = {(14pt,5pt)}] {$I $};
		\end{tikzpicture}
	}
\loigiai{
}
\end{ex}

\begin{ex}
	\immini{Đồ thị hình bên là của một trong bốn hàm số sau. Hỏi đó là hàm số nào?
		\choice
		{$y=\dfrac{x^2-x+4}{x+1}$}
		{$y=\dfrac{x^2-2x+3}{x+1}$}
		{\True $y=\dfrac{-x^2-x+2}{x+1}$}
		{$y=\dfrac{x^2+x-1}{x+1}$}}{
		\begin{tikzpicture}[>=stealth, scale=0.35, font=\footnotesize]
			\draw[->] (-5,0)--(4.4,0) node[below] {$x$};
			\draw[->] (0,-5)--(0,6) node[left] {$y$};
			\draw[domain=-0.6:4, smooth] plot (\x, {-(\x)^2-(\x)+2)/(\x+1)});
			\draw[domain=-5:-1.35, smooth] plot (\x, {-(\x)^2-(\x)+2)/(\x+1)});
			\draw[domain=-5:4, smooth] plot (\x, {-\x});
			\draw (-1,-5)--(-1,6);
			\draw[fill=black] (0,0) node[below left=-0.1] {$O$} circle (1.2pt);
			\draw[fill=black] (1,0) node[below] {$1$} circle (1.2pt);
			\draw[fill=black] (-1,0) node[below left] {$-1$} circle (1.2pt);
			\draw[fill=black] (3,0) node[above] {$3$} circle (1.2pt);
			\draw[fill=black] (0,2) node[below left] {$2$} circle (1.2pt);
			\draw[fill=black] (0,-2.5) node[left] {$-\dfrac{5}{2}$} circle (1.2pt);
			\draw[dashed] (3,0)--(3,-2.5)--(0,-2.5) ;
			\end{tikzpicture}
	}
\loigiai{
}
\end{ex}

\begin{ex}
	\immini{Đồ thị hình bên là của một trong bốn hàm số sau. Hỏi đó là hàm số nào?
		\choice
		{$y=\dfrac{x^2+3}{x-1}$}
		{\True 	$y=\dfrac{x^{2}+x-3}{x-1}$}
		{$y=\dfrac{x^2-2x+3}{-x+1}$}
		{$y=\dfrac{x^2+3}{-x+1}$}}{
		\begin{tikzpicture}[line cap=butt,line join=miter,>=stealth,scale=0.35,font=\footnotesize]
			\tikzset{declare function={xmin=-3.8;xmax=4.8;ymin=-3.6;ymax=7.8;},
				smooth,samples=450}
			\draw[->] (xmin,0)--(xmax,0) node[shift={(0:7pt)}]{$ x $};
			\draw[->] (0,ymin)--(0,ymax) node[shift={(90:7pt)}]{$ y $};
			\fill (0,0) node[shift={(140:5pt)}]{$ O $};
			\clip (xmin,ymin-.7) rectangle (xmax,ymax);
			\foreach \i in {-2,2}{
				\draw(\i,1.5pt)--(\i,-1.5pt)node[below]{$\i$};}
			\foreach \j in {-2,4}{
				\draw(-1.5pt,\j)--(1.5pt,\j) node[left]{$\j$};}	
			\draw(-1.5pt,2)--(1.5pt,2) node[right]{$2$};
			\def\f(#1){((#1)^2+(#1)-3)/((#1)-1)} % Hàm số
			\def\q(#1){((#1)+2)} % Tiệm cận xiên
			\def\a{0}	
			\pgfmathsetmacro\fa{\f(\a)}	
			\draw[samples=250] plot[domain=-7.4:0.9] (\x,{\f(\x)});	
			\draw[samples=250] plot[domain=1.1:15] (\x,{\f(\x)});
			\draw[] plot [domain=-7.4:7] (\x,{\q(\x)});
			\draw[] (1,ymin)--(1,ymax) node[rotate=180 ,pos=0.9,sloped,above] {$x=1$};
			\foreach \x/\y in {\a/\fa}{	
				\draw[dashed] (\x,0)|-(0,\y);}
			\foreach \x/\y in {\a/\fa}{
				\fill[white,draw=black] (\x,\y) circle (1pt);}
			;
			\node at (2.6,5.4) [rotate=45,right,fill=white]{$y=x+2$};
		\end{tikzpicture}
	}
\loigiai{
}
\end{ex}

\Closesolutionfile{ans}

\ind{PHẦN II.} \inden{Câu trắc nghiệm đúng sai. Trong mỗi ý a), b), c), d) ở mỗi câu, học sinh chọn đúng hoặc sai.}\\
\Opensolutionfile{ans}[ans/2D1-B4-d3-2]
\begin{ex}
	\immini{Cho hàm số $y=\dfrac{ax^2+bx+c}{mx+n}$ có đồ thị như hình bên.
	\choiceTF
	{Tập xác định của hàm số là $\mathbb{R}\backslash\{1\}$}
	{\True Hàm số nghịch biến trên khoảng $(-\infty;2)$ và $(2;+\infty)$}
	{\True Điểm $I(2;1)$ là tâm đối xứng của đồ thị}
	{\True Hệ số $a$ và $m$ trái dấu}}{
\begin{tikzpicture}[line join=round, line cap=round,>=stealth,x=0.5cm, y=0.5cm]
	\tikzset{every node/.style={scale=0.9}}
	\draw[->] (-4.1,0)--(6.1,0) node[below left] {$x$};
	\draw[->] (0,-6.1)--(0,6.1) node[below left] {$y$};
	\draw (0,0) node [below left] {\scriptsize$O$};
	\foreach \x/\nx in {-4/-4,-2/-2,1/ ,2/ ,4/4}
	\draw[thin] (\x,1pt)--(\x,-1pt) node [below] {$\nx$};
	\draw (2,0) node[above left]{\scriptsize$2$};
	\foreach \y/\ny in {-4/-4,-2/-2,-1/-1,2/2,4/4}
	\draw[thin] (1pt,\y)--(-1pt,\y) node [left] {$\ny$};
	\draw[dashed,thin](2,0)--(2,-1)--(0,-1);
	\draw[dashed,thin] (2,-6)--(2,6);
	\begin{scope}
		\clip (-4,-6) rectangle (6,6);
		\draw[samples=200,domain=-4:1.99,smooth,variable=\x] plot (\x,{(-1*((\x)^2)+3*(\x)+-1)/(1*(\x)+-2)});
		\draw[samples=200,domain=2.01:6,smooth,variable=\x] plot (\x,{(-1*((\x)^2)+3*(\x)+-1)/(1*(\x)+-2)});
		\draw[dashed,thin] (-6.1,7.1)--(6.1,-5.1);
	\end{scope}
\end{tikzpicture}}
\loigiai{
\begin{enumerate}[a)]
	\item 
	\item
	\item
	\item
\end{enumerate}}
\end{ex}

\begin{ex}
	\immini{Cho hàm số $y=\dfrac{ax^2+bx+c}{x+n}$ có đồ thị như hình bên.
		\choiceTF
		{\True Tập xác định của hàm số là $\mathbb{R}\backslash\{1\}$}
		{Điểm $I(1;2)$ là tâm đối xứng của đồ thị}
		{$a+2b=4$}
		{\True Đồ thị qua điểm $(2;10)$ khi $c=4$}}{
		\begin{tikzpicture}[line cap=round, line join=round,font=\footnotesize,>=stealth, scale=1,x=0.5cm, y=0.25cm]
			\tikzset{label style/.style={font=\footnotesize}}
			\draw[->] (-4,0)--(6,0) node[below] {$x$};
			\draw[->] (0,-8)--(0,15) node[left] {$y$};
			\draw[smooth, samples=100] plot[domain=-4:0.5] (\x, {  (2*(\x)^2-(\x)+4)/(\x-1) });
			\draw[smooth, samples=100] plot[domain=1.5:6] (\x, {  (2*(\x)^2-(\x)+4)/(\x-1) });
			\draw[dashed] (1,-8)node [right]{$x=1$}--(1,15) 
			plot[domain=-4:6](\x, {2*(\x)+1}) node[rotate=45,below]{$y=2x+1$};
	\end{tikzpicture}}
\loigiai{
	\begin{enumerate}[a)]
		\item
		\item
		\item
		\item
\end{enumerate}}
\end{ex}

\Closesolutionfile{ans}
% \begin{dang}{Sự tương giao của hai đồ thị}
	\begin{enumerate}[\iconCH]
		\item \indamm{Xác định tọa độ giao điểm của hai đồ thị $y=f(x)$ và $y=g(x)$:}
		\begin{listEX}[1]
			\item [\ding{172}] Giải phương trình hoành độ giao điểm $f(x)=g(x)$, tìm các nghiệm $x_0 \in \mathscr{D}_f \cap \mathscr{D}_g$.
			\item [\ding{173}] Với $x_0$ vừa tìm, thay vào một trong hai hàm số ban đầu để tìm $y_0$.
			\item [\ding{174}] Kết luận giao điểm $(x_0;y_0)$.
		\end{listEX}
		\item \indamm{Ứng dụng đồ thị để biện luận nghiệm phương trình:}
		\immini{
		\begin{enumerate}[]
			\item Xét phương trình $f(x)=m$, với $m$ là tham số. Nghiệm của phương trình này có thể coi là hoành độ giao điểm của đồ thị $y=f(x)$ (cố định) với đường thẳng $y=m$ (nằm ngang).
			\item Từ đó, để biện luận nghiệm phương trình $f(x)=m$, ta có thể thực hiện các bước như sau:
				\begin{itemize}
					\item [$\bullet$] Lập bảng biến thiên của hàm số $y=f(x)$ trên miền xác định mà đề bài yêu cầu.
					\item [$\bullet$] Tịnh tiến đường thẳng $y=m$ theo hướng "\textit{lên, xuống}". Quan sát số giao điểm để quy ra số nghiệm tương ứng.
				\end{itemize}
		\end{enumerate}}{
	\begin{tikzpicture}[scale=0.7, font=\footnotesize, line join=round, line cap=round, >=stealth]
		\draw[->] (-2.5,0) -- (3,0) node[below]{ $x$};
		\draw[->] (0,-1.5) -- (0,4) node[right]{ $y$};
		\draw[blue,line width=1pt,smooth,samples=100,domain=-2.09:2.1] plot(\x,{-(\x)^3+3*(\x)+1})node[right]{\footnotesize $y=f(x)$};
		\draw[dashed](-2,3)--(0,3)node[above left]{\footnotesize $3$}--(1,3);
		\draw[dashed](-1,-1)--(0,-1)node[right]{\footnotesize $-1$};
		\draw[fill=black] (-2,3) circle(1.5pt) (-1,-1) circle(1.5pt) (1,3) circle(1.5pt);
		\draw[fill=red] (0.3473,2) circle(2.5pt) (1.5321,2) circle(2.5pt) (-1.8794,2) circle(2.5pt);
		\draw[line width=1pt,red](-2.5,2)--(3,2)node[above]{\footnotesize $y=m$};
\end{tikzpicture}}
	\end{enumerate}
\end{dang}
\boxmini{BÀI TẬP TỰ LUẬN}
\begin{vd}
	Xác định tọa độ giao điểm của hai đồ thị hàm số sau:
	\begin{tasks}(2)
		\task $y=x^3-2x^2+x-1$ và $y=1-2x$;
		\task $y=\dfrac{x+8}{x-2}$ và $y=x+2$.
	\end{tasks}
\loigiai{
\begin{enumerate}[a)]
	\item Xét phương trình hoành độ giao điểm\\
	\centerline{$x^3-2x^2+x-1=1-2x\Leftrightarrow x^3-2x^2+3x-2=0\Leftrightarrow(x-1)\left(x^2-x+2\right)=0\Leftrightarrow x=1$.}\\
	Do đó $2$ đồ thị  hàm số có giao điểm là $(1;-1)$.
	\item Với điều kiện $x\ne 2$ ta có\\
	Phương trình hoành độ giao điểm $x+2=\dfrac{x+8}{x-2}\Leftrightarrow x^2-4=x+8 \Leftrightarrow x^2 -x -12 =0 \Leftrightarrow \hoac{&x=3\\&x=-4.}$\\
	Từ đó được $A(3;5)$ và $B(-4;-2)$.
\end{enumerate}}
\end{vd}

\begin{vd}
	Tìm tập hợp các giá trị thực của tham số $m$  để đồ thị hàm số $y=(x-2)(x^2+mx+m^2-3)$ cắt trục hoành tại ba điểm phân biệt.
	\loigiai{
		Đồ thị hàm số đã cho cắt trục hoành tại ba điểm phân biệt khi phương trình $$(x-2)(x^2+mx+m^2-3)=0$$ có $3$ nghiệm phân biệt hay phương trình $x^2+mx+m^2-3=0$ có $2$ nghiệm phân biệt khác $2$ 
		$$\Leftrightarrow \heva{\Delta& =-3m^2+12>0\\m&^2+2m+1\ne 0}\Leftrightarrow \heva{-&2<m<2\\m&\ne -1}.$$
	}
\end{vd}

\begin{vd}
	Tìm tham số m để phương trình $x^3 - 3x + 2-m=0$ có ba nghiệm phân biệt.
	\loigiai{
		Phương trình tương đương với  $x^3 - 3x + 2=m$.
		\immini {
			\begin{itemize}
				\item [$\bullet$] Số nghiệm của phương trình bằng số giao điểm của đồ thị $y=x^3 - 3x + 2$ với đường thẳng $y=m$ (nằm ngang).
				\item [$\bullet$] Đồ thị hàm số $ y = x^3 - 3x +2 $ như hình bên. Để đường thẳng $ y = m $ cắt đồ thị tại 3 điểm phân biệt khi và chỉ khi $ 0 < m < 4. $
			\end{itemize}
		Vậy $ 0 < m < 4. $
			}
		{\begin{tikzpicture}[>=stealth,scale=0.6,every node/.style={scale=0.8}]
				\draw[->,black] (-2.5,0) -- (3,0)node[above left] {$x$};
				từ tọa độ (-1.5,0) đến tạo độ (3.5,0) ghi tên x ở trên bên trái
				\draw[->,black] (0,-1) -- (0,4.5)node[right] {$y$};
				\foreach \x in {-1,1,2}
				\draw[shift={(\x,0)}] (0pt,1pt) -- (0pt,-1pt) node[below] {\footnotesize $\x$};
				\foreach \y in {2,3,4}
				\draw[shift={(0,\y)}] (1pt,0pt) -- (-1pt,0pt) node[right] {\footnotesize $\y$};
				\node at (0,0) [below right] {\footnotesize $O$};
				\draw[smooth,samples=100,domain=-2.1:2.02] plot(\x,{(\x)^3-3*(\x)+2});
				\draw [dashed] (-1,0)--(-1,4)--(0,4);
				\draw [blue] (-2.4,2.5)--(2.9,2.5);
				\node at (2,2.5) [below right] {\footnotesize $y=m$};
			\end{tikzpicture}
		}
	}
\end{vd}

\boxmini{BÀI TẬP TRẮC NGHIỆM}
\setcounter{ex}{0}
\Opensolutionfile{ans}[ans/2D1-B4-d4-1]
\begin{ex}%[2D1B5-4]
	Đường thẳng $y=x-1$ cắt đồ thị hàm số $y=x^3-x^2+x-1$ tại hai điểm. Tìm tổng tung độ các giao điểm đó.
	\choice
	{$-3 $}
	{$2 $}
	{$0 $}
	{\True $-1 $}
	\loigiai{Phương trình hoành độ giao điểm
		$$x^3-x^2+x-1=x-1\Leftrightarrow \left[ \begin{aligned} &x=1 \Rightarrow y=0\\ &x=0 \Rightarrow y=-1.\end{aligned} \right.$$
		Tổng tung độ các giao điểm là $0+(-1)=-1$.
	}
\end{ex}

\begin{ex}%[2D1Y5-4]
	Số giao điểm của đồ thị hàm số $y=(x-1)(x^2-3x+2)$ và trục hoành là
	\choice
	{$0$}
	{$1$}
	{\True $2$}
	{$3$}
	\loigiai{
		Phương trình $y=0$ có hai nghiệm là $x=1$ và $x=2$.
	}
\end{ex}

\begin{ex}
	Đồ thị hàm số $y=x^3-3x^2+2x-1$ cắt đồ thị hàm số $y=x^2-3x+1$ tại hai điểm phân biệt $A,B$. Tính độ dài $AB$.
	\choice
	{$AB=3$}
	{$AB=2\sqrt2$}
	{$AB=2$}
	{\True $AB=1$}
	\loigiai{
		Phương trình hoành độ giao điểm $$x^3-3x^2+2x-1=x^2-3x+1\Leftrightarrow x^3-4x^2+5x-2=0\Leftrightarrow \hoac{& x=1\\& x=2}\Rightarrow \hoac{& y=-1\\& y=-1}.$$
		Không mất tính tổng quát, ta giả sử $A(1;-1),B(2;-1)$. Suy ra $\vec{AB}=(1;0)\Rightarrow AB=1$.
	}
\end{ex}

\begin{ex}
	Đồ thị của hàm số $ y = \dfrac{x - 1}{x+1} $ cắt hai trục $ Ox $ và $ Oy $ tại $ A $ và $ B $. Khi đó diện tích của tam giác $ OAB $ (với $ O $ là gốc tọa độ) bằng
	\choice
	{$ 1 $}
	{$ \dfrac{1}{4} $}
	{$ 2 $}
	{\True $ \dfrac{1}{2} $}
	\loigiai{
		Ta có $ A(1;0), B(0;-1) $. Diện tích $ S_{\triangle OAB} = \dfrac{OA\cdot OB}{2} = \dfrac{1}{2} $.
	}	
\end{ex}

\begin{ex}
	Biết đường thẳng $y=x-2$ cắt đồ thị hàm số $ y=\dfrac{x}{x-1} $ tại $ 2 $ điểm phân biệt $ A, $ $ B. $ Tìm hoành độ trọng tâm tam giác $OAB$ với $O$ là gốc tọa độ.
	\choice
	{$ \dfrac{2}{3} $}
	{$ 2 $}
	{\True $ \dfrac{4}{3} $}
	{$ 4 $}
	\loigiai{
		
		Xét phương trình hoành độ giao điểm $ x-2=\dfrac{x}{x-1} $ (Điều kiện $ x\neq 1 $).
		
		$ \Rightarrow (x-2)(x-1)=x\Leftrightarrow x^2-4x+2=0 \,  (1).$
		
		Khi đó $ A(x_1;x_1-2), $ $ B(x_2;x_2-2) $  với $ x_1, x_2 $ là $ 2 $ nghiệm của phương trình $ (1) $ thỏa mãn 
		
		$ \heva{&x_1+x_2=4\\&x_1.x_2=2}. $ Gọi $ G\left(x_G;y_G\right) $ là trọng tâm tam giác $ OAB. $
		
		$ \Rightarrow x_G=\dfrac{0+x_1+x_2}{3}=\dfrac{4}{3}. $
		
	}
\end{ex}

\begin{ex}%[2D1B5]
	Gọi $ M, N $ là giao điểm của đường thẳng $ y = x + 1 $ và đường cong $ y = \dfrac{2x+4}{x-1} $. Tìm hoành độ trung điểm của đoạn thẳng $ MN. $
	\choice
	{$ x = -1 $}
	{\True $ x = 1 $}
	{$ x = -2 $}
	{ $ x = 2$}
	\loigiai
	{Xét phương trình hoành độ giao điểm $ x+ 1 = \dfrac{2x +4}{x-1} \Leftrightarrow \heva{&x \ne 1\\ &x^2 - 2x- 5 = 0}$\\
		$ \Rightarrow x_M + x_N = 2 \Rightarrow x_I = \dfrac{x_M+x_N}{2} = 1.$}
\end{ex}

\begin{ex}
	Cho hàm số $y=\dfrac{2x}{x+1}$ có đồ thị $(C)$. Gọi $A,B$ là giao điểm của đường thẳng $d:y=x$ với đồ thị $(C)$. Tính độ dài đoạn $AB$.
	\choice
	{\True $AB=\sqrt{2}$}
	{ $AB=\dfrac{\sqrt{2}}{2}$}
	{$AB=1$}
	{ $AB=2$}
	\loigiai{
		Phương trình hoành độ giao điểm\\
		$\dfrac{2x}{x+1}=x,\left({x\ne -1}\right)\Rightarrow x^2-x=0\Rightarrow \left[{\begin{aligned}&{x=0\Rightarrow y=0\Rightarrow A\left({0;0}\right)} \\ &{x=1\Rightarrow y=1\Rightarrow B\left({1;1}\right)} \\ \end{aligned}}\right.$\\
		Vậy $AB=\sqrt{2}$.
	}
\end{ex}

\begin{ex}%[2D1B5-3]
	\immini
	{
		Cho hàm số $y=f(x)$ có đồ thị như hình vẽ. Số nghiệm của phương trình $2f(x)-3=0$ là
		\haicot
		{$2$}
		{$1$}
		{$0$}
		{\True $3$}
	}
	{\begin{tikzpicture}[smooth,samples=300,scale=0.5,>=stealth]
			\draw[->] (-2.3,0)--(3,0) node[below]{$x$};
			\draw[->] (0,-1.5)--(0,4) node[right]{$y$};
			\draw (0,0) node[above left]{$O$};
			\draw[thick,domain=-2.05:2.05] plot(\x,{1*((\x)^3)-3*(\x)+1});
			\draw[fill=black] (0,3) circle(1pt) (-1,3) circle(1.5pt) (0,-1) circle(1pt) (1,-1) circle(1.5pt);
			\draw[dashed] (1,-1)--(0,-1)node[left]{$-1$} (-1,3)--(0,3)node[right]{$3$};
		\end{tikzpicture}
	}
	
	\loigiai{
		Ta có $2f(x)-3=0\Leftrightarrow f(x)=\dfrac{3}{2}$.\\
		Từ đồ thị suy ra phương trình có $3$ nghiệm phân biệt.
	}
\end{ex}


\begin{ex}%[2D1B5-3]
	\immini
	{Cho hàm số $f(x)=ax^3 +bx^2 +cx +d$ $(d\ne 0)$ có đồ thị như hình vẽ bên. Số nghiệm của phương trình $3f(x) -1 =0$ bằng
		\haicot
		{$0$}
		{\True $1$}
		{$2$}
		{$3$}
	}
	{\hspace{1cm}\begin{tikzpicture}[line join=round, line cap=round,>=stealth,scale=0.5]
			\tikzset{label style/.style={font=\footnotesize}}
			\draw[->] (-1.1,0)--(3.1,0) node[above right] {$x$};
			\draw[->] (0,-2.1)--(0,5.1) node[right] {$y$};
			\draw (0,0) node [above left] {$O$};
			\foreach \x in {1,2}
			\draw[thin] (\x,1pt)--(\x,-1pt) node [above] {$\x$};
			\foreach \y in {-1,4}
			\draw[thin] (1pt,\y)--(-1pt,\y) node [left] {$\y$};
			%\draw[dashed,thin](-1,0)--(-1,3)--(0,3);
			\draw[dashed,thin](1,0)--(1,-1)--(0,-1);
			\begin{scope}
				\clip (-1,-2) rectangle (3,5);
				\draw[samples=200,domain=-1:3,smooth,variable=\x] plot (\x,{-2*((\x)^3)+9*((\x)^2)+-12*(\x)+4});
			\end{scope}
	\end{tikzpicture}}
	\loigiai{
		Ta có $3f(x)-1=0 \Leftrightarrow f(x) = \dfrac{1}{3}$.\\
		Khi đó số giao điểm của đồ thị $y=f(x)$ và đường thẳng $y=\dfrac{1}{3}$ chính là số nghiệm của phương trình $3f(x) -1=0$. Dựa vào đồ thị ta có số nghiệm của phương trình là 1.}
\end{ex}

\begin{ex}%[2D1B5-3]
	\immini{Cho hàm số $y = f(x)$ có bảng biến thiên như sau. Số giao điểm của đồ thị hàm số $y = f(x)$ với trục hoành là
		\haicot
		{$ 1$}
		{$ 0$}
		{$ 2  $}
		{\True $ 3 $}}{
		\begin{tikzpicture}
			\tkzTabInit[lgt=1,espcl=1.8]
			{$x$/0.6, $y’$/0.6, $y$/1.6}
			{$-\infty$,$0$,$1$,$+\infty$}
			\tkzTabLine{ ,-,$0$,+,$0$,-, }
			\tkzTabVar{+/$+\infty$,-/$-1$,+/$3$,-/$-\infty$}
	\end{tikzpicture}}
	\loigiai{
		Dựa vào bảng biến thiên thì đồ thị hàm số $y = f(x)$ và trục hoành có $3$ điểm chung.	
	}
\end{ex}

\begin{ex}%[2D1B5-3]
	\immini{Cho hàm số $y=f(x)$ liên tục trên $(-\infty;+\infty)$ và có bảng biến thiên như hình bên. Số nghiệm thực của phương trình $2\big|f(x)\big|=7$ bằng
		\choice
		{$3$}
		{\True $2$}
		{$4$}
		{$2$}	
		
		
	}{\begin{tikzpicture}[>=stealth,line join=round,line cap=round,font=\footnotesize,scale=.8]
			\begin{scope}[xscale=1.15,yscale=0.8]
				\begin{scope}[shift={(-0.5,0.5)}]
					\def\a{8}
					\def\b{4}
					\draw (0,0)rectangle +(\a,-\b)
					(1,0)--+(-90:\b)
					(0,-1)--+(0:\a)
					(0,-2)--+(0:\a)
					;
				\end{scope}
				\draw
				(0,0)node{$x$}++(0:1)node{$-\infty$}++(0:2)node{$1$}++(0:2)node{$2$}
				++(0:2)node{$+\infty$}
				(0,-1)node{$y'$}	++(0:2)node{$+$}++(0:1)node{$0$}++(0:1)node{$-$}++(0:1)node{$0$}++(0:1)node{$+$}
				(0,-2.5)node{$y$}
				(1,-3.2) node (A)  {$-\infty$} 
				(3,-2)node (B) {$5$} 
				(5,-2.8) node (C){$4$} 
				(7,-1.9)node (D){$+\infty$} 
				;
				\draw[->] (A)--(B);
				\draw[->] (B)--(C);
				\draw[->] (C)--(D);
			\end{scope}	
	\end{tikzpicture}}
	
	\loigiai{
		
	}
\end{ex}

\begin{ex}%[2D1K5-3]
	\immini{Cho hàm số $y=f(x)$ liên tục trên $\mathbb{R}\setminus\{0\}$ và có bảng biến thiên như hình bên. Hỏi phương trình $3|f(x)|-10=0$ có bao nhiêu nghiệm?
		\choice
		{$2$ nghiệm}
		{$4$ nghiệm}
		{\True $3$ nghiệm}
		{$1$ nghiệm}
	}{
		\begin{tikzpicture}
			\tikzset{double style/.append style = {draw=\tkzTabDefaultWritingColor,double=\tkzTabDefaultBackgroundColor,double distance=2pt}}
			\tkzTabInit[nocadre=false,lgt=1.2,espcl=1.7,deltacl=0.6]
			{$x$ /.6,$f'(x)$ /.6,$f(x)$ /1.7}{$-\infty$,$0$,$1$,$+\infty$}
			\tkzTabLine{,-,d,-,0,+,}
			\tkzTabVar{+/$2$,-D+/$-\infty$/+$\infty$,-/$3$,+/$+\infty$}
	\end{tikzpicture}}
	\loigiai
	{
		Từ bảng biến thiên đề bài, ta có bảng biến thiên của hàm số $y=|f(x)|$ như sau
		\begin{center}
			\begin{tikzpicture}
				\tkzTabInit[nocadre=false,lgt=1.3,espcl=2.5,deltacl=0.6]
				{$x$ /.6,$f'(x)$ /.6,$|f(x)|$ /2}{$-\infty$,,$0$,$1$,$+\infty$}
				\tkzTabLine{,,-,,d,-,0,+,}
				\tkzTabVar{+/$2$,-/$0$,+D+/$+\infty$/+$\infty$,-/$3$,+/$+\infty$}
			\end{tikzpicture}
		\end{center}
		Ta có $3|f(x)|-10=0\Leftrightarrow |f(x)|=\dfrac{10}{3}.\qquad(1)$\\
		Số nghiệm của phương trình (1) bằng số giao điểm của đồ thị $y=|f(x)|$ và đường thẳng $y=3$.\\
		Dựa vào bảng biến thiên trên, suy ra phương trình (1) có $3$ nghiệm. 
	}
\end{ex}

\begin{ex}%[2D1K5-3]
	\immini{Cho hàm số $y = f(x)$ xác định và liên tục trên $\mathbb{R}$, có bảng biến thiên như sau. Số nghiệm của phương trình $2[f(x)]^2- 3 f(x)+ 1 = 0$ là
		\haicot
		{$2$}
		{\True $3$}
		{$6$}
		{$0$}}
	{\begin{tikzpicture}[scale=0.8]
			\tkzTabInit[espcl=2.3,lgt=1.2,deltacl=0.6]
			{$x$/0.6,$y'$/0.6,$y$/2}
			{$-\infty$,$-1$,$1$,$+\infty$}
			\tkzTabLine{,+,0,-,0,+,}
			\tkzTabVar{-/$1$,+/$3$,-/$\dfrac{1}{3}$,+/$1$}
	\end{tikzpicture}}
	\loigiai{
		Ta có $ 2[f(x)]^2- 3 f(x)+ 1 = 0\Leftrightarrow \left[\begin{array}{l}{f(x)= 1}\\{f(x)= \dfrac{1}{2}.}\end{array}\right.$\\
		Phương trình $f(x)= 1$ có duy nhất nghiệm $ x_0 $.\\
		Phương trình $f(x)= \dfrac{1}{2}$ có $2$ nghiệm phân biệt khác $x_{0}$.  Vậy phương trình có ba nghiệm.
	}
\end{ex}

\begin{ex}
	\immini{Cho hàm số $f(x)$ có bảng biến thiên như hình bên. Tìm tất cả các giá trị thực của tham số $m$ để phương trình $f(x)=m+1$ có ba nghiệm thực phân biệt.
		\choice
		{$-3\le m \le 3$}
		{$-2\le m \le 4$}
		{$-2<m<4$}
		{\True $-3<m<3$}
	}{
		\begin{tikzpicture}
			\tkzTabInit[nocadre=false,lgt=1,espcl=1.9,deltacl=0.6]
			{$x$ /0.6, $y'$ /0.6, $y$ /1.6}
			{$-\infty$,$-1$,$3$,$+\infty$}
			\tkzTabLine{,+,$0$,-,$0$,+,}
			\tkzTabVar{-/$-\infty$,+/$4$,-/$-2$,+/$+\infty$}
	\end{tikzpicture}}
	\loigiai{
		Dựa vào bảng biến thiên phương trình $f(x)=m+1$ có ba nghiệm thực phân biệt khi
		\begin{center}
			$-2<m+1<4 \Leftrightarrow -3<m<3$.
		\end{center}
	}
\end{ex}

\begin{ex}
	\immini{Cho hàm số $y=f(x)$ có bảng biến thiên như hình bên. Phương trình $f(4x-x^2)-2=0$ có bao nhiêu nghiệm thực?
		\choice 
		{$2$}
		{$6$}
		{$0$}
		{\True $4$}
	}{
		\begin{tikzpicture}
			\tkzTabInit[nocadre=false,lgt=1.2,espcl=2.2,deltacl=0.6]
			{$x$ /0.6,$y’$ /0.6,$y$ /1.6}
			{$-\infty$ ,$0$ , $4$, $+\infty$}
			\tkzTabLine{,-,0,+,0,-}
			\tkzTabVar{+/ $+\infty $ / , -/ $-1$ /,+/ $3$/ , -/ $-\infty$ /}  
	\end{tikzpicture} }
	\loigiai{ 
		Đặt $t=4x-x^2$. Khi đó $t=-(x-2)^2+4 \leq 4$.\\
		Từ mỗi giá trị $t<4$ ta tìm được hai giá trị $x$. Với $t=4$ ta tìm được $x=2$.\\
		Từ bảng biến thiên, ta thấy phương trình $f(t)=2 \Leftrightarrow \left [ \begin{aligned} &t=\alpha \in (-\infty;0)\\ 
			&t=\beta \in (0;4)  \\
			&t=\gamma \in  (4;+\infty)  \end{aligned} \right.$\\
		Vậy phương trình $f(4x-x^2)-2=0$ có $4$ nghiệm. 
	}   
\end{ex}

\Closesolutionfile{ans}


%%Bài 5. Ứng dụng TT
% \setcounter{section}{4}
\section{ỨNG DỤNG ĐẠO HÀM VÀ KHẢO SÁT HÀM SỐ ĐỂ GIẢI QUYẾT MỘT SỐ BÀI TOÁN THỰC TIỄN}
\subsection{LÝ THUYẾT CẦN NHỚ}
\subsubsection{Tốc độ thay đổi của một đại lượng}
Ta có đạo hàm $f'(a)$ là tốc độ thay đổi tức thời của đại lượng $y=f(x)$ đối với $x$ tại điểm $x=a$. Dưới đây, chúng ta xem xét một số ứng dụng của ý tưởng này đối với vật lí, hoá học, sinh học và kinh tế: 
\begin{itemize}
	\item Nếu $s=s(t)$ là hàm vị trí của một vật chuyển động trên một đường thẳng thì $v=s'(t)$ biểu thị vận tốc tức thời của vật (tốc độ thay đổi của độ dịch chuyển theo thời gian). Tốc độ thay đổi tức thời của vận tốc theo thời gian là gia tốc tức thời của vật:
	$$
	a(t)=v'(t)=s''(t).
	$$
	\item Nếu $C=C(t)$ là nồng độ của một chất tham gia phản ứng hoá học tại thời điểm $t$, thì $C'(t)$ là tốc độ phản ứng tức thời (tức là độ thay đổi nồng độ) của chất đó tại thời điểm $t$.
	\item Nếu $P=P(t)$ là số lượng cá thể trong một quần thể động vật hoặc thực vật tại thời điểm $t$, thì $P'(t)$ biểu thị tốc độ tăng trưởng tức thời của quần thể tại thời điểm $t$.
	\item  Nếu $C=C(x)$ là hàm chi phí, tức là tổng chi phí khi sản xuất $x$ đơn vị hàng hoá, thì tốc độ thay đổi tức thời $C'(x)$ của chi phí đối với số lượng đơn vị hàng được sản xuất được gọi là chi phí biên.
	\item Về ý nghĩa kinh tế, chi phí biên $C'(x)$ xấp xỉ với chi phí để sản xuất thêm một đơn vị hàng hoá tiếp theo, tức là đơn vị hàng hoá thứ $x+1$ (xem SGK Toán 11 tập hai, trang 87, bộ sách Kết nối tri thức với cuộc sống). 
\end{itemize}
\subsubsection{Bài toán tối ưu hóa}
Một trong những ứng dụng phổ biến nhất của đạo hàm là cung cấp một phương pháp tổng quát, hiệu quả để giải những bài toán tối ưu hoá. Trong mục này, chúng ta sẽ giải quyết những vấn đề thường gặp như tối đa hoá diện tích, khối lượng, lợi nhuận, cũng như tối thiểu hoá khoảng cách, thời gian, chi phí.\\
Khi giải những bài toán như vậy, khó khăn lớn nhất thường là việc chuyển đổi bài toán thực tế cho bằng lời thành bài toán tối ưu hoá toán học bằng cách thiết lập một hàm số phù hợp mà ta cần tìm giá trị lớn nhất hoặc giá trị nhỏ nhất của nó, trên miền biến thiên phù hợp của biến số.\\
Quy trình giải một số bài toán tối ưu hoá  đơn giản:
\begin{itemize}
	\item[\iconCH]\indamm{Bước 1.} Xác định đại lượng Q mà ta cần làm cho giá trị của đại lượng ấy lớn nhất hoặc nhỏ nhất và biểu diễn nó qua các đại lượng khác trong bài toán.
	
	\item[\iconCH]\indamm{Bước 2.}  Chọn một đại lượng thích hợp nào đó, kí hiệu là $x$, và biểu diễn các đại lượng khác ở \indamm{Bước 1} theo $x$. Khi đó, đại lượng $Q$ sẽ là hàm số của một biến $x$. Tìm tập xác định của hàm số $Q=Q(x)$.
	
	\item[\iconCH]\indamm{Bước 3.}  Tìm giá trị lớn nhât hoặc giá trị nhỏ nhất của hàm số $Q=Q(x)$ bằng các phương pháp đã biết và kết luận.
\end{itemize}

\subsection{PHÂN LOẠI VÀ PHƯƠNG PHÁP GIẢI TOÁN}
\begin{dang}{Bài toán về tốc độ thay đổi của một đại lượng}
\end{dang}
\begin{vd}
	Khi bỏ qua sức cản của không khí, độ cao (mét) của một vật được phóng thẳng đứng lên trên từ điểm cách mặt đất $2$ m với vận tốc ban đầu $24{,}5$ m/s là $h(t)=2+24{,}5t-4{,}9t^2$ (theo Vật lí đại cương, NXB Giáo dục Việt Nam, $2016$).
	\begin{enumerate}
		\item Tìm vận tốc của vật sau $2$ giây.
		\item Khi nào vật đạt độ cao lớn nhất và độ cao lớn nhất đó là bao nhiêu?
		\item Khi nào thì vật chạm đất và vận tốc của vật lúc chạm đất là bao nhiêu?
	\end{enumerate}
	\loigiai{
		\begin{enumerate}
			\item Theo ý nghĩa cơ học của đạo hàm, vận tốc của vật là $v=h'(t)=24{,}5-9{,}8t$ m/s.\\				
			Do đó, vận tốc của vật sau $2$ giây là $v(2)=24{,}5-9{,}8\cdot 2=4{,}9$ m/s.
			\item Vì $h(t)$ là hàm số bậc hai có hệ số $a=-4{,}9< 0$ nên $h(t)$ đạt giá trị lớn nhất tại $t=-\dfrac{b}{2a}=\dfrac{24{,}5}{2\cdot 4{,}9}=2{,}5$ (giây). Khi đó, độ cao lớn nhất của vật là $h(2{,}5)=32{,}625$ m.
			\item Vật chạm đất khi độ cao bằng 0, tức là $h=2+24{,}5t-4{,}9t^2=0$, hay $t \approx 5{,}08$ (giây).\\
			Vận tốc của vật lúc chạm đất là $v(5{,}08)=24{,}5-9{,}8\cdot 5{,}08=-25{,}284$ m/s.\\
			Vận tốc âm chứng tỏ chiều chuyển động của vật là ngược chiều dương (hướng lên trên) của trục đã chọn (khi lập phương trình chuyển động của vật).
		\end{enumerate}
	}
\end{vd}

\begin{vd}
	Xét phản ứng hóa học tạo ra chất $C$ từ hai chất $A$ và $B$: $A+B\longrightarrow C$. Giả sử nồng độ của hai chất $A$ và $B$ bằng nhau $[A]=[B]=a$ (mol/l). Khi đó, nồng độ của chất $C$ theo thời gian $t$ ($t>0$) được cho bởi công thức: $[C]=\dfrac{a^2Kt}{aKt+1}$ (mol/l), trong đó $K$ là hằng số dương.
	\begin{enumerate}
		\item Tìm tốc độ phản ứng ở thời điểm $t>0$.
		\item Chứng minh nếu $x=[C]$ thì $x'(t)=K(a-x)^2$.
		\item Nêu hiện tượng xảy ra với nồng độ các chất khi $t\longrightarrow +\infty$.
		\item Nêu hiện tượng xảy ra với tốc độ phản ứng khi $t\longrightarrow +\infty$.
	\end{enumerate}
	\loigiai{
		\begin{enumerate}
			\item Tìm tốc độ phản ứng ở thời điểm $t>0$.\\
			Tốc độ của phản ứng là đạo hàm của $[C]=\dfrac{a^2Kt}{aKt+1}$ theo biến $t$. Do đó 
			\[[C]^\prime =\left(\dfrac{a^2Kt}{aKt+1}\right)^\prime=\dfrac{a^2K\left(aKt+1\right)-a^2Kt\cdot aK}{\left(aKt+1\right)^2}=\dfrac{a^2K}{\left(aKt+1\right)^2}.\]
			\item Chứng minh nếu $x=[C]$ thì $x'(t)=K(a-x)^2$.\\
			Theo câu trên, nếu nếu $x=[C]$ thì $x^\prime(t)=\dfrac{a^2K}{\left(aKt+1\right)^2}$.\\
			Ta lại có 
			\[K(a-x)^2=K \left(a-\dfrac{a^2Kt}{aKt+1}\right)^2=\dfrac{a^2K}{\left(aKt+1\right)^2}.\]
			Vậy $x'(t)=K(a-x)^2$.
			\item Nêu hiện tượng xảy ra với nồng độ các chất khi $t\longrightarrow +\infty$.\\
			Ta có $\lim\limits_{t\to +\infty}[C]=\lim\limits_{t\to +\infty}\dfrac{a^2Kt}{aKt+1}=a\  (mol/l)$.\\
			Vậy nồng độ của chất $C$ dần đến $a\  (mol/l)$.
			\item Nêu hiện tượng xảy ra với tốc độ phản ứng khi $t\longrightarrow +\infty$.\\
			Ta có $\lim\limits_{t\to +\infty}x^\prime(t)=\lim\limits_{t\to +\infty}\dfrac{a^2K}{\left(aKt+1\right)^2}=0$.\\
			Vậy tốc độ  của phản ứng  dần đến $0$.\\
	\end{enumerate}}
\end{vd}

\begin{vd}
	Giả sử số lượng của một quần thể nấm men tại môi trường nuôi cấy trong phòng thí nghiệm được mô hình hoá bằng hàm số $P(t)=\dfrac{a}{b+\mathrm{e}^{-0{,}75t}}$, trong đó thời gian $t$ được tính bằng giờ. Tại thời điểm ban đầu $t=0$, quần thể có 20 tế bào và tăng với tốc độ $12$ tế bào/giờ. Tìm các giá trị của $a$ và $b$. Theo mô hình này, điều gì xảy ra với quần thể nấm men về lâu dài?
	\loigiai{
		Ta có $P'(t)=\dfrac{0,75a \mathrm{e}^{-0,75t}}{\left(b+\mathrm{e}^{-0{,}75t}\right)^2}, t \geq 0$.\\
		Theo đề bài, ta có $P(0)=20$ và $P'(0)=12$. Do đó, ta có hệ phương trình:
		$$
		\heva{&\dfrac{a}{b+1}=20 \\& \dfrac{0,75a}{(b+1)^2=12}} \Leftrightarrow \heva{&a=20(b+1)\\&\dfrac{15}{b+1}=12}
		$$
		Giải hệ phương trình này, ta được $a=25$ và $b=\dfrac{1}{4}$.\\
		Khi đó, $P'(t)=\dfrac{18{,}75\mathrm{e}^{-0{,}75t}}{\left(\dfrac{1}{4}+\mathrm{e}^{-0{,}75 t}\right)^2} > 0, \forall t \geq 0$, tức là số lượng quần thể nấm men luôn tăng.\\
		Tuy nhiên, do $\lim\limits_{t \rightarrow+\infty} P(t)=\lim\limits_{t \rightarrow+\infty} \dfrac{25}{\dfrac{1}{4}+\mathrm{e}^{-0{,}75t}}=100$ nên số lượng quần thể nấm men tăng nhưng không vượt quá $100$ tế bào. 
	}
\end{vd}

\begin{vd}
	Giả sử chi phí $C(x)$ (nghìn đồng) để sản xuất $x$ đơn vị của một loại hàng hoá nào đó được cho bởi hàm số $C(x)=30\,000+300x-2{,}5x^2+0{,}125x^3$.
	\begin{enumerate}
		\item Tìm hàm chi phí biên.
		\item Tìm $C'(200)$ và giải thích ý nghĩa.
		\item So sánh $C'(200)$ với chi phí sản xuất đơn vị hàng hoá thứ 201.
	\end{enumerate}
	\loigiai{
		\begin{enumerate}
			\item Hàm chi phí biên là $C'(x)=300-5x+0{,}375x^2$.
			\item Ta có $C'(200)=300-5\cdot 200+0,375\cdot 200^2=14300$.\\				
			Chi phí biên tại $x=200$ là $14\,300$ nghìn đồng, nghĩa là chi phí để sản xuất thêm một đơn vị hàng hoá tiếp theo (đơn vị hàng hoá thứ 201) là khoảng $14\,300$ nghìn đồng.
			\item Chi phí sản xuất đơn vị hàng hoá thứ $201$ là
			$$
			C(201)-C(200)=1\,004\,372{,}625- 990\,000=14\,372{,}625 \text { (nghìn đồng).}
			$$				
			Giá trị này xấp xỉ với chi phí biên $C'(200)$ đã tính ở câu b.
		\end{enumerate}
		
	}
\end{vd}

\begin{dang}{Bài toán tối ưu hoá đơn giản}
\end{dang}

\begin{vd}
	\immini{Một nhà sản xuất cần làm những hộp đựng hình trụ có thể tích $ 1 $ lít. Tìm các kích thước của hộp đựng để chi phi vật liệu dùng để sản xuất là nhỏ nhất (kết quả được tính theo centimét và làm tròn đến chứ số thập phân thứ hai).}{
	\begin{tikzpicture}[line join=round,line cap=round,line width=.6pt,font=\footnotesize,scale=0.45,>=stealth]
		\coordinate[label=right:$A$] (A) at (3,0);
		\coordinate[label=left:$O$] (O) at (0,0);
		\coordinate[label=right:$A'$] (A1) at ($(A)+(90:6)$);
		\coordinate[label=left:$O'$] (O1) at ($(O)+(90:6)$);
		\draw (A) arc (0:-180:3 and 3/4)--($(A1)!2!(O1)$) arc (180:0:3 and 3/4) arc (0:-180:3 and 3/4) (A)--(A1)--(O1);
		\draw[dashed] (O1)--(O)--(A) arc (0:180:3 and 3/4);
		\fill (O)circle(1.5pt) (O1)circle(1.5pt) (A)circle(1.5pt) (A1)circle(1.5pt);
\end{tikzpicture}}
	\loigiai{
		Đổi $1 \text{ lít} =1000 \text{ cm}^3$.
		\\
		Gọi $r( cm )$ là bán kính đáy của hình trụ, $h( cm )$ là chiều cao của hình trụ.
		\\
		Diện tích toàn phần của hinh trụ là $S=2 \pi r^2+2 \pi r h$.
		\\
		Do thể tích của hình trụ là $1000 \text{ cm}^3$ nên ta có: $1000=V=\pi r^2 h$, hay $h=\dfrac{1000}{\pi r^2}$.
		\\
		Do đó, diện tích toàn phần của hình trụ là $S=2 \pi r^2+\dfrac{2000}{r},\, r>0$.
		\\
		Ta cần tìm $r$ sao cho $S$ đạt giá trị nhỏ nhất. Ta có
		\begin{align*}
			&S'=4 \pi r-\dfrac{2000}{r^2}=\dfrac{4 \pi r^3-2000}{r^2};
			\\
			&S'=0 \Leftrightarrow \pi r^3=500 \Leftrightarrow r=\sqrt[3]{\dfrac{500}{\pi}}
		\end{align*}
		Bảng biến thiên
		\begin{center}
			\begin{tikzpicture}[font=\footnotesize,thick,>=stealth]
				\tikzset{double style/.append style={double distance=1.5pt}}\tkzTabInit[nocadre=false,lgt=1.2,espcl=3.5,deltacl=0.6,lw=.75pt,color,colorL=green!50,colorV=green!50]
				{$r$ /1.2, $S'(r)$ /1, $S(r)$ /2.5}
				{$0$,$\sqrt[3]{\dfrac{500}{\pi}}$,$+\infty$}
				\tkzTabLine{ ,-,$0$,+, }
				\tkzTabVar{+/$+\infty$,-/$S\left( \sqrt[3]{\dfrac{500}{\pi}} \right)$,+/$+\infty$}
			\end{tikzpicture}
		\end{center}
		Khi đó
		$$
		h=\dfrac{1000}{\pi r^2}=\dfrac{1000}{\pi \sqrt[3]{\frac{250000}{\pi^2}}}=\dfrac{100}{\sqrt[3]{250 \pi}}.
		$$
		Vậy cần sản xuất các hộp đựng hình trụ có bán kinh đáy $r=\sqrt[3]{\dfrac{500}{\pi}} \approx 5,42 \text{ (cm)}$ và chiều cao $h=\dfrac{100}{\sqrt[3]{250 \pi}} \approx 10,84\text{ (cm)}$.
	}
\end{vd}

\begin{vd}
	Một bác nông dân có ba tấm lưới B40, mỗi tấm dài $a \text{ (m)}$ và muốn rào một mảnh vườn dọc bờ sông có dạng hình thang cân $ABCD$ như {\it Hình 36} (bờ sông là đường thẳng $CD$ không phải rào). Hỏi bác đó có thể rào được mảnh vườn có diện tích lớn nhất là bao nhiêu mét vuông?
	\begin{center}
		\begin{tikzpicture}[scale=.6]
			\path 
			(-3,0) coordinate (D)
			(3,0) coordinate (C)
			($(D)+(65:3)$) coordinate (A)
			($(C)+(115:3)$) coordinate (B)
			;
			\fill[cyan!50] (-4.5,-1) rectangle (4,0);
			\draw[thick] (A)--node[above]{$a \text{(m)}$}(B)--node[right]{$a \text{(m)}$}(C)--(D)--node[left]{$a \text{(m)}$} cycle;
			\node at (0,-1.5) {\it Hình 36};
			\foreach \x/\g in {A/120,B/60,C/-60,D/-120}		\fill[black] 	(\x) circle (1pt)
			($(\g:3mm)+(\x)$) node {$\x$};
		\end{tikzpicture}
		
	\end{center}
	
	\loigiai{
		\begin{center}
			\begin{tikzpicture}
				\path 
				(-3,0) coordinate (D)
				(3,0) coordinate (C)
				($(D)+(65:3)$) coordinate (A)
				($(C)+(115:3)$) coordinate (B)
				($(C)!(A)!(D)$) coordinate (M)
				($(C)!(B)!(D)$) coordinate (N)
				;
				\draw[thick] (A)--(B)--(C)--(D)-- cycle;
				\draw[dashed] (A)--(M) (B)--(N);
				\foreach \x/\g in {A/120,B/60,C/-60,D/-120,M/-90,N/-90}		\fill[black] 	(\x) circle (1pt)
				($(\g:3mm)+(\x)$) node {$\x$};
			\end{tikzpicture}
		\end{center}
		Gọi $M$, $N$ lần lượt là hình chiếu vuông góc của $A$, $B$ trên $CD$.\\
		Đặt $x=MD$, $\left( 0<x<a\right)$. Suy ra $AM=\sqrt{AD^2-MD^2}=\sqrt{a^2-x^2}$.\\
		Diện tích của mảnh vườn hình thang cân là $S(x)=\dfrac{(AB+CD)AM}{2}=(a+x)\sqrt{a^2-x^2}$.\\
		Xét hàm số $f(x)= (a+x)\sqrt{a^2-x^2}$ trên khoảng $\left( 0<x<a\right)$.\\
		$f^\prime (x)=\dfrac{-2x^2-ax+a^2}{\sqrt{a^2-x^2}}$, $f^\prime (x)=0\Leftrightarrow \dfrac{-2x^2-ax+a^2}{\sqrt{a^2-x^2}}=0\Leftrightarrow \hoac{&x=-a \notin \left( 0<x<a\right) \\&x=\dfrac{a}{2}\in \left( 0<x<a\right) }$.\\
		Bảng biến thiên hàm số $f(x)$ trên khoảng $\left( 0;a\right)$.
		\begin{center}
			\begin{tikzpicture}
				\tkzTabInit[lgt=1.2,espcl=4.5,deltacl=0.6]
				{$x$/1,$f'(x)$/1,$f(x)$/3} {$0$,$\dfrac{a}{2}$,$a$}
				\tkzTabLine{,+,0,-,}
				\tkzTabVar{-/$a^2$,+/$\dfrac{3\sqrt{3}a^2}{4}$,-/$0$}
			\end{tikzpicture}
		\end{center}
		Từ bảng biến thiên suy ra $\max\limits_{(0;a)} f(x)=f\left(\dfrac{a}{2}\right)=\dfrac{3\sqrt{3}a^2}{4}$.\\
		Vậy bác nông dân có thể rào được mảnh vườn có diện tích lớn nhất $\dfrac{3\sqrt{3}a^2}{4} \text{ m}^2$.
	}
\end{vd}

\begin{vd}
	Có hai xã $A$, $B$ cùng ở một bên bờ sông Lam, khoảng cách từ hai xã đó đến bờ sông lần lượt là $AA'=500 \text{ m}$, $BB'=600 \text{ m}$ và người ta đo được $A'B'=2\,200 \text{ m}$ {\it Hình 37}. Các kĩ sư muốn xây một trạm cung cấp nước sạch nằm bên bờ sông Lam cho dân hai xã. Để tiết kiệm chi phí, các kĩ sư cần phải chọn vị trí $M$ của trạm cung cấp nước sạch đó trên đoạn $A'B'$ sao cho tổng khoảng cách từ hai xã đến vị trí $M$ là nhỏ nhất. Hãy tìm giá trị nhỏ nhất của tổng khoảng cách đó.
	\begin{center}
		\begin{tikzpicture}[scale=.6]
			\path 
			(0:0) coordinate (A')
			(0:6) coordinate (B')
			(0:2) coordinate (M)
			($(A')+(90:2.5)$) coordinate (A)
			($(B')+(90:3)$) coordinate (B)
			;
			\fill[cyan!50] (-1.5,-1) rectangle (7.5,0);
			\draw[thick] (A')--node[left]{$500 \text{(m)}$}(A)--(M)--(B)--node[right]{$600 \text{(m)}$}(B');
			
			\foreach \i/\j in{A'/-100,B'/-80,A/100,B/80,M/-90}{\fill [black](\i) circle (1pt) ($(\i)+(\j:3mm)$) node {$\i$};}
			
			\draw [dashed,<->]	(0,.6)--(6,.6) node[pos=0.75,sloped,above]{$2\,200\text{(m)}$}; %Tùy chọn sloped,above,below
			\node at (3,-1.5){\it Hình 37};
		\end{tikzpicture}
	\end{center}
	\loigiai{
		\begin{center}
			\begin{tikzpicture}
				\path 
				(0:0) coordinate (A')
				(0:6) coordinate (B')
				(0:2) coordinate (M)
				($(A')+(90:2.5)$) coordinate (A)
				($(B')+(90:3)$) coordinate (B)
				;
				\draw[thick] (A')--node[left]{$500 \text{(m)}$}(A)--(M)--(B)--node[right]{$600 \text{(m)}$}(B') (A')--(B');
				
				\foreach \i/\j in{A'/-100,B'/-80,A/100,B/80,M/-90}{\fill [black](\i) circle (1pt) ($(\i)+(\j:3mm)$) node {$\i$};}
				
				\draw [dashed,<->]	(0,.6)--(6,.6) node[pos=0.75,sloped,above]{$2\,200\text{(m)}$}; %Tùy chọn sloped,above,below
				\node at (3,-1.5){\it Hình 37};
			\end{tikzpicture}
		\end{center}
		Đặt $A'M=x$, $(0<x<2200)$,  $B'M=2200-x$.\\
		Ta có: $AM=\sqrt{x^2+500^2}$, $BM=\sqrt{(2200-x)^2+600^2}$.\\
		Khi đó tổng khoảng cách từ hai xã đến vị trí $M$ là $AM+BM= \sqrt{x^2+500^2}+\sqrt{(2200-x)^2+600^2} $.\\
		Xét hàm số $f(x)= \sqrt{x^2+500^2}+\sqrt{(2200-x)^2+600^2}$ trên khoảng $(0<x<2200)$.\\
		%$f(x)=\sqrt{x^2+500}+\sqrt{x^2-4400x+4840600}$ .\\
		$f^\prime (x)=\dfrac{x}{\sqrt{x^2+500^2}}-\dfrac{2200-x}{\sqrt{(2200-x)^2+600^2}}$, $f^\prime (x)=0\Leftrightarrow \dfrac{x}{\sqrt{x^2+500^2}}=\dfrac{2200-x}{\sqrt{(2200-x)^2+600^2}}$\\
		$\Leftrightarrow \dfrac{x^2}{x^2+500^2}=\dfrac{(2200-x)^2}{(2200-x)^2+600^2}$\\
		$\Leftrightarrow \dfrac{x^2+500^2}{x^2}=\dfrac{(2200-x)^2+600^2}{(2200-x)^2}$\\
		$\Leftrightarrow 1+\dfrac{500^2}{x^2}=1+\dfrac{600^2}{(2200-x)^2}$\\
		$\Leftrightarrow \dfrac{25}{x^2}=\dfrac{36}{(2200-x)^2}$\\
		$\Leftrightarrow \dfrac{5}{x}=\dfrac{6}{2200-x}\Leftrightarrow x=1000$, vì $ x>0$.\\ 
		Bảng biến thiên hàm số $f(x)$ trên khoảng $\left( 0;2200\right)$.
		\begin{center}
			\begin{tikzpicture}
				\tkzTabInit[lgt=1.2,espcl=4.5,deltacl=0.6]
				{$x$/1,$f'(x)$/1,$f(x)$/3} {$0$,$1000$,$2200$}
				\tkzTabLine{,-,0,+,}
				\tkzTabVar{+/$2780$,-/$2460$,+/$2856$}
			\end{tikzpicture}
		\end{center}
		Vậy giá trị nhỏ nhất của tổng khoảng cách từ hai xã đó đến bờ sông  là khoảng $2460 \text{ m}$, tại vị trí $M$ cách điểm $A'$  là $1000 \text{ m}$.
	}
\end{vd}

\subsection{BÀI TẬP TỰ LUYỆN}
\begin{bt}
	Một tàu đổ bộ tiếp cận Mặt Trăng theo cách tiếp cận thẳng đứng và đốt cháy các tên lửa hãm ở độ cao $250$ km so với bề mặt của Mặt Trăng.\\
	Trong khoảng $50$ giây đầu tiên kể từ khi đốt cháy các tên lửa hãm, độ cao $h$ của con tàu so với bề mặt của Mặt Trăng được tính (gần đúng) bởi hàm $h(t)=-0{,}01t^3+1{,}1t^2-30t+250$, trong đó $t$ là thời gian tính bằng giây và $h$ là độ cao tính bằng kilômét.\\
	\textit{(Nguồn: A. Bigalke et al., Mathematik, Grundkurs ma-1, Cornelsen 2016).}
	\begin{enumerate}
		\item Vẽ đồ thị của hàm số $y=h(t)$ với $0\leq t\leq 50$ (đơn vị trên trục hoành là $10$ giây, đơn vị trên trục tung là $10$ km).
		\item Gọi $v(t)$ là vận tốc tức thời của con tàu ở thời điểm $t$ (giây) kể từ khi đốt cháy các tên lửa hãm với $(0\leq t\leq 50$). Xác định hàm số $v(t)$.
		\item Vận tốc tức thời của con tàu lúc bắt đầu hãm phanh là bao nhiêu? Tại thời điểm $t=25$ (giây) là bao nhiêu?
		\item Tại thời điểm $t=25$ (giây), vận tốc tức thời của con tàu vẫn giảm hay đang tăng trở lại?
		\item Tìm thời điểm $t$ ($0\leq t\leq 50$) sao cho con tàu đạt khoảng cách nhỏ nhất so với bề mặt của Mặt Trăng. Khoảng cách nhỏ nhất này là bao nhiêu?
	\end{enumerate}
	\loigiai{
		\begin{enumerate}
			\item Vẽ đồ thị của hàm số $h(t)=-0{,}01t^3+1{,}1t^2-30t+250$.
			\begin{itemize}
				\item Miền khảo sát: $[0;50]$.
				\item Đạo hàm: $h'(t)=-0{,}03t^2+2{,}2t-30$.
				\[h'(t)=0\Leftrightarrow -0{,}03t^2+2{,}2t-30=0\Leftrightarrow \hoac{&t\approx 18\\ &t\approx 55.}\]
				\item Bảng biến thiên:
				\begin{center}
					\begin{tikzpicture}
						\tkzTabInit[lgt=1.2, espcl=3, deltacl=0.6]
						{$t$/0.6, $h'(t)$/0.6, $h(t)$/2}
						{$0$, $18$, $50$}
						\tkzTabLine{, -, 0, +, }
						\tkzTabVar{+/ $250$, -/$8{,}08$, +/$250$}
					\end{tikzpicture}
				\end{center}
				\begin{itemize}
					\item Hàm số nghịch biến trên các khoảng $(0;18)$ và đồng biến trên khoảng $(18;50)$.
					\item Hàm số đạt cực tiểu tại $t=18$, $y_{_\text{CT}}=h(18)=8{,}08$.
				\end{itemize}
				\item Bảng giá trị:
				\begin{center}
					\begin{tikzpicture}
						\tkzTabInit[lgt=1.2, espcl=2.5, deltacl=1]
						{$x$/0.7, $y$/0.7}
						{$0$, $18$, $50$}
						\tkzTabLine{250, , 8.08, , 250}
					\end{tikzpicture}
				\end{center}
				\item Đồ thị:
				\begin{center}
					\begin{tikzpicture}[>=stealth, scale=1, font=\footnotesize]
						\draw[->] (-1,0)--(4.5,0) node[below] {$t$};
						\draw[->] (0,-1)--(0,8) node[left] {$h(t)$};
						\draw[fill=black] (0,0) node[below left=-0.1] {$O$} circle (1.2pt);
						\draw[fill=black] (0.8,0) node[below] {$18$} circle (1.2pt);
						\draw[fill=black] (2.34,0) node[below] {$50$} circle (1.2pt);
						\draw[fill=black] (0,0.3) node[above left = -0.1 and 0] {$8{,}08$} circle (1.2pt);
						\draw[fill=black] (0,7) node[left] {$250$} circle (1.2pt);
						\draw[dashed] (0.8,0)--(0.8,0.3)--(0,0.3) (2.34,0)--(2.34,7)--(0,7);
						\clip (0,0) rectangle (3,7);
						\draw (0,7) parabola bend (0.8,0.3) (1.5,3) parabola bend (3,8) (3,8);
					\end{tikzpicture}
				\end{center}
			\end{itemize}
			\item Xác định $v(t)$.\\
			Ta có $v(t)=h'(t)=-0{,}03t^2+2{,}2t-30$.
			\item Tính vận tốc tức thời lúc bắt đầu hãm phanh và lúc $t=25$ (giây).
			\begin{itemize}
				\item Vận tốc tức thời lúc bắt đầu hãm phanh là: $v(0)=-30$ (km/s).
				\item Vận tốc tức thời lúc $t=25$ (giây) là: $v(25)=6{,}25$ (km/s).
			\end{itemize}
			\item Tại thời điểm $t=25$ (giây), vận tốc tức thời của con tàu vẫn giảm hay tăng trở lại?
			\begin{itemize}
				\item Ta có phương trình gia tốc: $a(t)=v'(t)=-0{,}06t+2{,}2t$.
				\item Vì $a(25)=53{,}5>0$ nên tại thời điểm $t=25$ (giây), vận tốc tức thời của con tàu đang tăng trở lại.
			\end{itemize}
			\item Tìm thời điểm mà khoảng cách giữa con tàu và Mặt Trăng nhỏ nhất.\\
			Dựa vào đồ thị ta thấy tại thời điểm $t=18$ (giây) thì khoảng cách giữa con tàu và Mặt Trăng nhỏ nhất, khoảng cách này bằng $8{,}08$ km.
		\end{enumerate}
	}
\end{bt}

\begin{bt}
	Để loại bỏ $x\%$ chất gây ô nhiễm không khí từ khí thải của một nhà máy, người ta ước tính chi phí cần bỏ ra là
	$$
	C(x)=\dfrac{300 x}{100-x} \text { (triệu đồng), } 0 \leq x < 100.
	$$		
	Khảo sát sự biến thiên và vẽ đồ thị của hàm số $y=C(x)$. Từ đó, hãy cho biết:
	\begin{enumerate}
		\item Chi phí cần bỏ ra sẽ thay đổi như thế nào khi $x$ tăng?
		\item Có thể loại bỏ được $100 \%$ chất gây ô nhiễm không khí không? Vì sao?
	\end{enumerate}
	\loigiai{
		Xét hàm số $y=C(x)=\dfrac{300x}{100-x}, 0\leq x < 100$.\\
		Ta có 
		\begin{itemize}
			\item  $y'=\dfrac{30\,000}{(100-x)^2} > 0$, với mọi $x \in[0; 100)$.\\
			Do đó hàm số luôn đồng biến trên nửa khoảng $[0; 100)$.
			\item  $\lim\limits_{x \to 100^{-}} C(x)=\lim\limits_{x \to 100^{-}} \dfrac{300x}{100-x}=+\infty$, nên đồ thị hàm số có tiệm cận đứng là $x=100$.
		\end{itemize}
		Bảng biến thiên:
		\begin{center}
			\begin{tikzpicture}
				\tkzTabInit%[nocadre,lgt=1.2,espcl=2]
				{$x$/0.7,$C'(x)$/0.7,$C(x)$/2.5}{$0$,$+\infty$}  
				\tkzTabLine{ ,$+$, }
				\tkzTabVar{-/$0$,+/$\infty$}
			\end{tikzpicture}
		\end{center}
		Đồ thị hàm số như Hình $1.34$.
		\begin{enumerate}
			\item  Chi phí cần bỏ ra $C(x)$ sẽ luôn tăng khi $x$ tăng.
			\item  Vì $\lim\limits_{x \rightarrow 100^{-}} C(x)=+\infty$ (hàm số $C(x)$ không xác định khi $x=100$) nên nhà máy không thể loại bỏ $100\%$ chất gây ô nhiễm không khí (dù bỏ ra chi phí là bao nhiêu đi chăng nữa).
		\end{enumerate}
		\begin{tikzpicture}[xscale=1/50,yscale=1/50,>=stealth, font=\footnotesize, line join=round, line cap=round]
			\def\xmin{-100} \def\xmax{200}
			\def\ymin{-100} \def\ymax{450}
			%\draw[color=gray!50,dashed] (\xmin,\ymin) grid (\xmax,\ymax);
			\draw[->] (\xmin,0)--(\xmax,0) node [below]{$x$};
			\draw[->] (0,\ymin)--(0,\ymax) node [left]{$y$};
			\fill (0,0) circle (1pt) node[shift={(-45:2.5mm)}]{$O$};	
			\clip (\xmin+0.1,\ymin+0.1) rectangle (\xmax-0.1,\ymax-0.1);
			\draw[red,smooth,samples=50,domain=0:48] plot(\x,{(300*\x)/(100-1.5*\x)});
			\foreach \x in {-50,0,50,100,150}
			\draw (\x,-0.1)--(\x,0.1);	
			\foreach \x/\r in {-50/-100, 50/100, 100/200, 150/300}
			\node at (\x,0) [below,scale=0.8] {\r};
			\foreach \y in {\ymin,...,\ymax}
			\draw (-0.1,\y)--(0.1,\y);
			\foreach \y/\r in {-50/-100, 50/100, 100/200, 150/300, 200/400,250/500,300/600,350/700,400/800}
			\node at (0,\y) [left,scale=0.8] {\r};
			\draw (50,-50)--(50,450)	;
		\end{tikzpicture}
	}
\end{bt}

\begin{bt}
	Khi máu di chuyển từ tim qua các động mạch chính rồi đến các mao mạch và quay trở lại qua các tĩnh mạch, huyết áp tâm thu (tức là áp lực của máu lên động mạch khi tim co bóp) liên tục giảm xuống. Giả sử một người có huyết áp tâm thu $P$ (tính bằng mmHg) được cho bởi hàm số
	$$
	P(t)=\dfrac{25 t^2+125}{t^2+1}, 0 \leq t \leq 10,
	$$
	trong đó thời gian $t$ được tính bằng giây. Tính tốc độ thay đổi của huyết áp sau $5$ giây kể từ khi máu rời tim.
	\loigiai{
		Ta có tốc độ thay đổi của huyết áp là $P'(t)=\dfrac{-100t}{(t^2+1)^2}$.\\
		Do đó tốc độ thay đổi huyết áp sau $5$ s là $P'(5)=-\dfrac{125}{169}$.
	}
\end{bt}

\begin{bt}
	Bạn Việt muốn dùng tấm bìa hình vuông cạnh $6$ dm làm một chiếc hộp không nắp, có đáy là hình vuông bằng cách cắt bỏ đi $4$ hình vuông nhỏ ở bốn góc của tấm bìa (Hình bên dưới).
	\begin{center}
		\begin{tikzpicture}[>=stealth,line join=round,line cap=round,font=\footnotesize,scale=1]
			\fill[blue!20] (0,0)--(3,0)--(3,3)--(0,3);
			\fill[white] (0,0)--(0,0.4)--(0.4,0.4)--(0.4,0) (2.6,0)--(3,0)--(3,0.4)--(2.6,0.4)
			(2.6,2.6)--(2.6,3)--(3,3)--(3,2.6) (0,2.6)--(0.4,2.6)--(0.4,3)--(0,3);
			\draw (0,0)--(3,0)--(3,3)--(0,3)--cycle;
			\draw[line width=0.2pt] (0.4,0.4)--(2.6,0.4)--(2.6,2.6)--(0.4,2.6)--cycle;
			\draw [line width=0.05pt,<->] (3.05,2.6)--(3.05,3);
			\path (3,2.6)--(3,3)node[pos=0.5,sloped,black,below]{$x$};					
		\end{tikzpicture}~\begin{tikzpicture}[>=stealth,line join=round,line cap=round,font=\footnotesize,scale=1]
			\fill[blue!20] (0,0)--(1.5,-1)--(3,0)--(1.5,1);		
			\fill[black!20] (0,0)--(-0.1,0.4)--(1.4,1.4)--(1.5,1);
			\draw (0,0)--(-0.1,0.4)--(1.4,1.4)--(1.5,1)--cycle;
			\fill[black!20] (1.5,1)--(1.6,1.4)--(3.1,0.4)--(3,0);
			\draw (1.5,1)--(1.6,1.4)--(3.1,0.4)--(3,0)--cycle;
			\fill[black!20] (1.5,-1)--(1.5,-0.6)--(3.1,0.4)--(3,0);
			\draw (1.5,-1)--(1.5,-0.6)--(3.1,0.4)--(3,0)--cycle;
			\fill[black!20] (1.5,-1)--(1.3,-0.7)--(-0.3,0.3)--(0,0);
			\draw (1.5,-1)--(1.3,-0.7)--(-0.3,0.3)--(0,0)--cycle;
		\end{tikzpicture}
	\end{center}
	
	Bạn Việt muốn tìm độ dài cạnh hình vuông cần cắt bỏ để chiếc hộp đạt thể tích lớn nhất.
	\begin{enumerate}
		\item Hãy thiết lập hàm số biểu thị thể tích hộp theo $x$ với $x$ là độ dài cạnh hình vuông cần cắt đi.
		\item Khảo sát và vẽ đồ thị hàm số tìm được.\\		
		Từ đó, hãy tư vấn cho bạn Việt cách giải quyết vấn đề và giải thích vì sao cần chọn giá trị này. (Làm tròn kết quả đến hàng phần mười.)
	\end{enumerate}
	\loigiai{
		\begin{enumerate}
			\item Hãy thiết lập hàm số biểu thị thể tích hộp theo $x$ với $x$ là độ dài cạnh hình vuông cần cắt đi.
			Mặt đáy của hộp là hình vuông có cạnh bằng $6-2x$ (cm), với $0<x<3$. Vậy diện tích của đáy hộp là $S=(6-2x)^2$.\\
			Khối hộp có chiều cao $h=x$ (cm).\\
			Vậy thể tích hộp là $V=S\cdot h=(6-2x)^2 \cdot x=4x^3-24x^2+36x$ (cm$^3$).
			\item Khảo sát và vẽ đồ thị hàm số tìm được.\\
			Xét hàm $f(x)=4x^3-24x^2+36x,\,\,0<x<3$.
			\begin{enumerate}
				\item Tập xác định: $\mathscr{D}=(0;3)$.
				\item Sự biến thiên.
				\begin{itemize}
					\item Giới hạn tại vô cực: $\displaystyle\displaystyle\lim \limits{n \to +\infty}_{x \rightarrow+\infty} f(x)=+\infty, \displaystyle\displaystyle\lim \limits{n \to +\infty}_{x \rightarrow-\infty} f(x)=-\infty$.
					\item Ta có $f'(x)=12x^2-48x+36\Rightarrow f'(x)=0\Leftrightarrow x^2-4x+3=0\Leftrightarrow \hoac{& x=1\\ & x=3.}$\\
					Ta có bảng biến thiên:
					\begin{center}
						\begin{tikzpicture}[>=stealth]
							\tkzTabInit[nocadre=false,lgt=1,espcl=2,deltacl=0.5]{$x$/.7 ,$y'$/.7,$y$/2}
							{$0$, $1$ , $3$ }
							\tkzTabLine{ , + , $0$ , - , $0$ }
							\tkzTabVar{-/$0$ , +/$16$ , -/$0$ }
						\end{tikzpicture}
					\end{center}
					Hàm số đồng biến trên $(0;1)$ và nghịch biến trên khoảng $(1;3)$.\\
					Hàm số không có cực trị.
					\item Đồ thị hàm số đi qua các điểm $(0 ; 0),(1 ; 16),(3 ; 0)$.
					\begin{center}
						\begin{tikzpicture}[line cap=butt,line join=miter,>=stealth,scale=0.8,font=\footnotesize,y=0.5cm]
							\tikzset{declare function={xmin=-1;xmax=4;ymin=-1;ymax=17;},
								smooth,samples=450}
							\draw[->] (xmin,0)--(xmax,0) node[shift={(0:7pt)},]{$ x $};
							\draw[->] (0,ymin)--(0,ymax) node[shift={(90:7pt)}]{$ y $};
							\fill (0,0) node[shift={(-150:7pt)}]{$ O $};
							\clip (xmin,ymin) rectangle (xmax,ymax);
							\foreach \i in {1}{
								\draw(\i,1.5pt)--(\i,-1.5pt)node[below]{$\i$};}
							\foreach \j in {16}{
								\draw(-1.5pt,\j)--(1.5pt,\j) node[left]{$\j$};}
							%	\draw(-1.5pt,-1)--(1.5pt,-1)node[shift={(7pt,0pt)}]{$-1$};	
							\def\f(#1){4*(#1)^3-24*(#1)^2+36*(#1)} % Đồ thị hàm số y=x^3+3x^2+3x+1
							\def\c{-1}
							\def\d{0}
							\def\e{-2}	
							\pgfmathsetmacro\fc{\f(\c)}
							\pgfmathsetmacro\fd{\f(\d)}
							\pgfmathsetmacro\fe{\f(\e)}
							\draw[samples=100] plot[domain=0:3] (\x,{\f(\x)});
							\foreach \x/\y in {\c/\fc,\d/\fd,\e/\fe}{
								\draw[dashed] (1,0)--(1,16)--(0,16);
								\fill[white,draw=black] (\x,\y) circle (1pt);}	
							%\node at (-2,-3.2) [right,fill=white,font=\footnotesize]{\it Hình LT $2b$};
						\end{tikzpicture}
					\end{center}
				\end{itemize}
			\end{enumerate}
			Vậy hình vuông mà bạn Việt cần cắt bỏ pải có độ dài cạnh $x=1$ dm thì chiếc hộp đạt thể tích lớn nhất.
		\end{enumerate}
	}
\end{bt}

%%Tổng hợp vdc
% \setcounter{ex}{0}
\section*{TỔNG HỢP VDC - CHƯƠNG I}
\Opensolutionfile{ans}[ans/ansBTchoice]
\TN
\begin{ex}%[2D1C5-7]
Tìm được trên đồ thị $(C)$ của hàm số $y=\dfrac{x^2+4x+5}{x+2}$ hai điểm $M(a;b)$ và $N(c;d)$ có khoảng cách đến đường thẳng $\Delta\colon 3x+y+6=0$ nhỏ nhất. Khi đó $a+b+c+d$ bằng
\choice
{$4$}
{$9$}
{$-9$}
{\True $-4$}
\loigiai
{Tập xác định $\mathscr{D}=\mathbb{R}\setminus\{-2\}$.
Gọi $M\left(x_0;y_0\right) \in (C)$. Ta có $x_0\neq -2$ và $y_0=\dfrac{x_0^2+4x_0+5}{x_0+2}$.\\
Khoảng cách từ $M$ đến đường thẳng $\Delta\colon 3x+y+6=0$ là
$$\mathrm{d}\left(M,\Delta\right)=\dfrac{1}{\sqrt{10}}\left|\dfrac{4x_0^2+16x_0+17}{x_0+2}\right|=\dfrac{1}{\sqrt{10}}\left|4\left(x_0+2\right)+\dfrac{1}{x_0+2}\right|.$$
Hàm số $f(t)=4t+\dfrac{1}{t}$ với $t\neq 0$ có $f'(t)=\dfrac{4t^2-1}{t^2}$ và $f'(t)=0\Leftrightarrow 4t^2-1=0\Leftrightarrow t=\pm \dfrac{1}{2}$.\\
Bảng biến thiên \begin{center}
\begin{tikzpicture}
\tkzTabInit[nocadre=false, lgt=1.2, espcl=2.5, deltacl=0.6]{$t$/1.2,$f'(t)$/0.6,$f(t)$/2}
{$-\infty$, $-\dfrac{1}{2}$, $0$, $\dfrac{1}{2}$, $+\infty$}
\tkzTabLine {,+,0,-,d,-,0,+,}
\tkzTabVar{-/$-\infty$, +/$-4$, -D+/$-\infty$/$+\infty$,-/$4$, +/$+\infty$}
\end{tikzpicture}
\end{center}
Suy ra $\min\limits_{t\neq 0} \left|f(t)\right|=4$ khi $t=\pm\dfrac{1}{2}$.\\
Do đó $\min\mathrm{d}(M,\Delta)=\dfrac{4}{\sqrt{10}}$ khi
$x_0+2=\pm\dfrac{1}{2} \Leftrightarrow \hoac{& x_0=-\dfrac{3}{2} \Rightarrow y_0=\dfrac{5}{2}\\ &x_0=-\dfrac{5}{2} \Rightarrow y_0=-\dfrac{5}{2}.}$\\
Như thế có hai điểm thoả yêu cầu bài toán là $M\left(-\dfrac{3}{2};\dfrac{5}{2}\right)$ và $N\left(-\dfrac{5}{2};-\dfrac{5}{2}\right)$.\\
Vậy $a+b+c+d=-\dfrac{3}{2}+\dfrac{5}{2}-\dfrac{5}{2}-\dfrac{5}{2}=-4$.
}
\end{ex}

\begin{ex}%[Dự Án Giảng 12 4 in 1, Lê Văn Toàn]%[2D1C5-6]
Trên đồ thị của hàm số $y=\dfrac{3x}{x-2}$ có điểm $M\left(x_0;y_0\right)$ $\left( \text{ với }x_0<0\right)$ sao cho tiếp tuyến tại điểm đó cùng với các trục tọa độ tạo thành một tam giác có diện tích bằng $\dfrac{3}{4}$. Khi đó $x_0+2y_0$ bằng
\choice
{$\dfrac{1}{2}$}
{$-1$}
{$-\dfrac{1}{2}$}
{\True $1$}
\loigiai{
Gọi $(C)$ là đồ thị của hàm số $y=\dfrac{3x}{x-2}$, $M\left(x_0;y_0\right)\in (C)$, suy ra $y_0=\dfrac{3x_0}{x_0-2}$ và $y'\left(x_0\right)=\dfrac{-6}{\left(x_0-2\right)^2}$.\\
Phương trình tiếp tuyến của $(C)$ tại $M\left(x_0;y_0\right)$ là $\Delta\colon y=\dfrac{-6}{\left(x_0-2\right)^2}\left(x-x_0\right)+\dfrac{3x_0}{x_0-2}$.\\
Gọi $A=\Delta \cap Ox\Rightarrow -6x+3x^2_0=0\Rightarrow x=\dfrac{x^2_0}{2}\Rightarrow A\left(\dfrac{x^2_0}{2};0\right)$.\\
Gọi $B=\Delta\cap Oy\Rightarrow y=\dfrac{6x_0}{\left(x_0-2\right)^2}+\dfrac{3x_0}{x_0-2}=\dfrac{3x_0^2}{\left(x_0-2\right)^2}\Rightarrow B\left(0;\dfrac{3x_0^2}{\left(x_0-2\right)^2}\right).$\\
Ta có $$S_{OAB}=\dfrac{1}{2}OA\cdot OB=\dfrac{1}{2}\cdot \dfrac{x^2_0}{2}\cdot \dfrac{3x^2_0}{\left(x_0-2\right)^2}=\dfrac{3}{4}\Leftrightarrow x^4_0=\left(x_0-2\right)^2\Leftrightarrow \hoac{&x^2_0=x_0-2\\&x^2_0=-x_0+2}\Leftrightarrow \hoac{&x_0=1\\&x_0=2.}$$
Do $x_0<0$ nên  nhận $x_0=-2\Rightarrow y_0=\dfrac{3}{2}$.\\
Vậy $x_0+2y_0=1$.
}
\end{ex}

\begin{ex}%[Mức độ C]%[2D1C5-3]
Tìm tất cả các giá trị thực của $m$ để phương trình $|x^4-2x^2-3|=2m-1$ có đúng $6$ nghiệm thực phân biệt.
\choice{$1<m<\dfrac{1}{3}$}{$4<m<5$}{$3<m<4$}{\True $2<m<\dfrac{5}{2}$}
\loigiai{Xét $g(x)=x^4-2x^2-3$; $g'(x)=4x^3-4x$.\\
$g'(x)=0\Leftrightarrow 4x^3-4x=0 \Leftrightarrow \left[\begin{array}{l}
x=0\\x=1\\x=-1.
\end{array}\right.$ \\
Bảng biến thiên của hàm số $g(x)$
\begin{center}
\begin{tikzpicture}
\tkzTabInit[nocadre=false, lgt=1.5,espcl=3.5]
{$x$/1,$g'(x)$/1,$g(x)$/2}
{$-\infty$,$-1$,$0$,$1$,$+\infty$}
\tkzTabLine{,-,0,+,0,-,0,+, }
\tkzTabVar{+/$+\infty$,-/$-4$,+/$-3$,-/$-4$,+/$+\infty$/}
\end{tikzpicture}
\end{center}
Bảng biến thiên của hàm số $f(x)=|x^4-2x^2-3|$ là
\begin{center}
\begin{tikzpicture}[scale=0.75]
\tkzTabInit[nocadre=false, lgt=1.5,espcl=3.5]
{$x$/1,$f(x)$/2}
{$-\infty$,$-\sqrt{3}$,$-1$,$0$,$1$,$\sqrt{3}$,$+\infty$}
\tkzTabVar{+/$+\infty$,-/$0$,+/$4$,-/$3$,+/$4$,-/$0$,+/$+\infty$/}
\end{tikzpicture}
\end{center}
Để phương trình $|x^4-2x^2-3|=2m-1$ có đúng $6$ nghiệm thực phân biệt khi và chỉ khi $$ 3<2m-1<4 \Leftrightarrow 2<m<\dfrac{5}{2}.$$.
}
\end{ex}

\begin{ex}%[Mức độ C]%[2D1C5-3]
Cho hàm số $y=f(x)$ liên tục trên $\mathbb{R}$ và có đồ thị như hình vẽ dưới đây. Tập hợp tất cả các giá trị thực của tham số $m$ để phương trình $f(x^2+2x-2)=3m+1$ có nghiệm thuộc khoảng $[0;1]$.
\begin{center}
\begin{tikzpicture}[>=stealth]
\draw [->] (-3.5,0)--(1.5,0);
\draw [->] (0,-1)--(0,5);
\draw (0,0) node[below right]{$O$};
\draw (1.5,0) node[below]{$x$};
\draw (0,5) node[below left]{$y$};
\draw (1,0) node[below]{$1$};
\draw (0,4) node[below left]{$4$};
\draw (-2,0) node[below]{$-2$};
\clip (-4,-1) rectangle (5,5);
\draw [thick,samples=100] plot[domain=-5:5](\x,{(\x)^3+3*(\x)^2});
\draw[dashed] (0,4) -- (1,4) --(1,0);
\fill[black] (1,4) circle(2pt);
\fill[black] (-2,4) circle(2pt);
\fill[black] (0,0) circle(2pt);
\draw[dashed] (0,4) -- (-2,4) --(-2,0);
\draw (1.1,5) node[below right]{$f(x)$};
\end{tikzpicture}
\end{center}
\choice
{$[0;4]$}
{$[-1;0]$}
{$[0;1]$}
{\True $\bigg[-\dfrac{1}{3};1\bigg]$}
\loigiai{Đặt $t=x^2+2x-2$, Với $x \in [0;1] \Rightarrow t \in [-2;1].$\\
Để phương trình $f(x^2+2x-2)=3m+1$ có nghiệm thuộc đoạn $[0;1]$ khi và chỉ khi phương trình $f(t)=3m+1$ có nghiệm thuộc $[-2;1].$\\ Do đó $ 0\leq m \leq 4 \Leftrightarrow -\dfrac{1}{3}\leq m \leq 1$.\\
Vậy $m \in \bigg[-\dfrac{1}{3};1\bigg]$}
\end{ex}

\begin{ex}%[Dự án TL12New-4in1-NCT]%[2D1C4-2]
Tìm tất cả các giá trị thực của tham số $m$ sao cho đồ thị hàm số $y=\dfrac{x+2}{\sqrt{mx^2+1}+\sqrt{(1-m)x^2+1}}$ có hai tiệm cận ngang.
\choice
{$m>0$}
{$m<1$}
{\True $0\leq m\leq 1$}
{$0<m<1$}
\loigiai{
Xét các trường hợp
\begin{itemize}
\item $m<0$ hoặc $m>1$, khi đó $\lim\limits_{x\to \infty}y$ không tồn tại nên đồ thị hàm số không thể có hai tiệm cận ngang.
\item Với $m=0$ hoặc $m=1$ thì hàm số trở thành $y=\dfrac{x+2}{1+\sqrt{x^2+1}}$. Đồ thị hàm số có đúng hai đường tiệm cận ngang $y=1$ và $y=-1$. Do đó $m=0$ và $m=1$ thỏa mãn yêu cầu bài toán.
\item Với $0<m<1$ ta có:
\begin{enumerate}[*]
\item $\lim\limits_{x\to +\infty}y=\lim\limits_{x\to +\infty}\dfrac{1+\dfrac{2}{x}}{\sqrt{m+\dfrac{1}{x^2}}+\sqrt{1-m+\dfrac{1}{x^2}}}=\dfrac{1}{\sqrt{m}+\sqrt{1-m}}$
\item $\lim\limits_{x\to -\infty}y=\lim\limits_{x\to -\infty}\dfrac{1+\dfrac{2}{x}}{-\sqrt{m+\dfrac{1}{x^2}}-\sqrt{1-m+\dfrac{1}{x^2}}}=\dfrac{-1}{\sqrt{m}+\sqrt{1-m}}$.
\end{enumerate}
Do đó đồ thị có hai tiệm cận ngang khi và chỉ khi\\ $\dfrac{1}{\sqrt{m}+\sqrt{1-m}}\neq \dfrac{-1}{\sqrt{m}+\sqrt{1-m}} \Leftrightarrow \dfrac{1}{\sqrt{m}+\sqrt{1-m}}\neq 0$ (hiển nhiên).
\end{itemize}
Tóm lại, $0\leq m\leq 1$ là các giá trị của $m$ thỏa mãn yêu cầu bài toán.
}
\end{ex}

\begin{ex}%[BG-12NEW-4in1, Nguyen Huynh]%[2D1C4-1]
Đồ thị hàm số $y=\log\dfrac{x^2-4x-5}{x^2-4}$ có tất cả bao nhiêu đường tiệm cận?
\choice
{$1$}
{$3$}
{\True $5$}
{$2$}
\loigiai{
Tập xác định của hàm số $\mathscr{D}=(-\infty;-2)\cup(-1;2)\cup(5;+\infty)$.\\
Mà $\lim\limits_{x\to \pm\infty}\log\dfrac{x^2-4x-5}{x^2-4}=\log 1=0$, suy ra $y=0$ là tiệm cận ngang của đồ thị hàm số.\\
Mà $\lim\limits_{x\to -2^-}\log\dfrac{x^2-4x-5}{x^2-4}=\lim\limits_{x\to +\infty}\log (x)=+\infty$, suy ra $x=-2$ là tiệm cận đứng của đồ thị hàm số.\\
Mà $\lim\limits_{x\to 2^-}\log\dfrac{x^2-4x-5}{x^2-4}=\lim\limits_{x\to +\infty}\log (x)=+\infty$, suy ra $x=2$ là tiệm cận đứng của đồ thị hàm số.\\
Mà $\lim\limits_{x\to -1^+}\log\dfrac{x^2-4x-5}{x^2-4}=\lim\limits_{x\to 0^+}\log (x)=-\infty$, suy ra $x=-1$ là tiệm cận đứng của đồ thị hàm số.\\
Mà $\lim\limits_{x\to 5^+}\log\dfrac{x^2-4x-5}{x^2-4}=\lim\limits_{x\to 0^+}\log (x)=-\infty$, suy ra $x=5$ là tiệm cận đứng của đồ thị hàm số.\\
Vậy đồ thị hàm số có $5$ đường tiệm cận.
}
\end{ex}

\begin{ex}%[THPTGQ 2018, mã 103, MĐ4]%[2D1C2-6]
Có bao nhiêu giá trị nguyên của tham số $m$ để hàm số $y=x^8+(m-4)x^5-(m^2-16)x^4+1$ đạt cực tiểu tại $x=0$.
\choice
{\True $8$}
{Vô số}
{$7$}
{$9$}
\loigiai{
Ta có $y'=8x^7+5(m-4)x^4-4(m^2-16)x^3$. \\
Đặt $g(x)=8x^4+5(m-4)x-4(m^2-16)$. Có $2$ trường hợp cần xét liên quan $(m^2-16)$:
\begin{itemize}
\item Trường hợp 1: $m^2-16=0 \Leftrightarrow m=\pm 4$.
\begin{itemize}
\item[+] Khi $m=4$ ta có $y'=8x^7 \Rightarrow x=0$ là điểm cực tiểu.
\item[+] Khi $m=-4$ ta có $y'=x^4(8x^4-40) \Rightarrow x=0$ không là điểm cực tiểu.
\end{itemize}
\item Trường hợp 2: $m^2-16\ne 0 \Leftrightarrow m\ne \pm 4$. Khi đó $x=0$ không là nghiệm của $g(x)$.\\
Ta có $x^3$ đổi dấu từ $-$ sang $+$ khi qua $x_0=0$, do đó\\
$y'=x^3\cdot g(x)$ đổi dấu từ $-$ sang $+$ khi qua $x_0=0 \Leftrightarrow \lim\limits_{x \to 0} g(x)>0 \Leftrightarrow m^2-16<0$.
\end{itemize}
Kết hợp các trường hợp giải được ta nhận $m \in \{-3;-2;-1;0;1;2;3;4\}$.}
\end{ex}

\begin{ex}%[THPTGQ 2018, mã 102, MĐ4]%[2D1C2-6]
Có bao nhiêu giá trị nguyên của tham số $m$ để hàm số
\begin{eqnarray*}
y =x^8+ (m - 1)x^5- (m^2- 1)x^4+ 1
\end{eqnarray*}
đạt cực tiểu tại $x = 0$?
\choice
{$3$}
{\True $2$}
{Vô số}
{$1$}
\loigiai{
Ta có $ y'=8x^7+5(m-1)x^4-4(m^2-1)x^3+1=x^3\left[ 8x^4+5(m-1)x-4(m^2-1) \right]  $,
$$ y'=0\Leftrightarrow \hoac{&x=0\\&8x^4+5(m-1)x-4(m^2-1)=0.\ \ (*)} $$
\begin{itemize}
\item Nếu $ m=1 $ thì $ y'=8x^7 $, suy ra hàm số đạt cực tiểu tại $ x=0 $.
\item Nếu $ m=-1 $ thì $ $$y'=0\Leftrightarrow\hoac{&x=0\\&8x^4-10x=0}\Leftrightarrow\hoac{&x=0\mbox{ (nghiệm kép)}\\&x=\sqrt[3]{\dfrac{5}{4}.}}$
Do đó $x=0$ không phải là điểm cực trị.
\item Nếu $ m\ne\pm1 $ thì $ x=0 $ là nghiệm đơn.\\
Đặt $ g(x)=8x^4+5(m-1)x-4(m^2-1) $. Hàm số đã cho đạt cực tiểu tại $ x=0 $ khi chỉ khi $$ \lim_{x\to 0^{-}}g(x)>0\Leftrightarrow -4(m^2-1)>0\Leftrightarrow m^2-1<0\Leftrightarrow -1<m<1. $$
Vì $ m\in\mathbb{Z} $ nên $ m=0 $.
\end{itemize}
Vậy giá trị $ m $ thỏa mãn yêu cầu bài toán là $ m=0 $, $ m=1 $.
}
\end{ex}

\begin{ex}%[MĐ4]%[2D1C2-6]
Có bao nhiêu giá trị nguyên dương của tham số $ m $ không vượt quá $ 2019 $ để hàm số $ f(x) = \dfrac{x^2}{8} + \sqrt{x + m + 2} $ không có điểm cực trị?
\choice
{ $ 0 $}
{\True$ 1 $}
{$ 2018 $}
{$ 2019 $}
\loigiai{
Tập xác định  $ \mathscr{D} = [ - m - 2;+ \infty)$.\\
Ta thấy \allowdisplaybreaks{
\begin{eqnarray*}
&& f'(x) =  \dfrac{x}{4}  + \dfrac{1}{ 2\sqrt{x + m + 2} }, x \neq - m - 2 \\
& \Leftrightarrow & 4 f'(x) = x + \dfrac{2}{ \sqrt{x + m  + 2} } \\
& \Leftrightarrow & 4 f'(x) =  (x + m + 2) + \dfrac{1}{ \sqrt{x + m  + 2}} + \dfrac{1}{ \sqrt{x + m  + 2}} - (m + 2) \\
& \Leftrightarrow & 4f'(x) \geq 3 - (m + 2)\\
& \Leftrightarrow & f'(x) \geq \dfrac{1 - m}{4}. \quad \quad (1)
\end{eqnarray*}
}%
Đẳng thức $ (1) $ xảy ra $ \Leftrightarrow x + m + 2 = \dfrac{1}{ \sqrt{x + m + 2} } \Leftrightarrow x = - m - 1 \in \mathscr{D} \setminus \{ - m - 2 \} $.\\
Vì $ \lim \limits_{x \to + \infty} f'(x) = + \infty $ nên hàm số $ f(x) $ không có cực trị khi và chỉ khi $ 1 - m \geq 0 \Leftrightarrow m \leq 1 $.\\
Vì $ m $ nguyên dương và không vượt quá $ 2019 $ nên $ m = 1 $.\\
Vậy có đúng $ 1 $ giá trị $ m $ thỏa mãn đề bài.
}
\end{ex}

\begin{ex}%[Mức độ 4]%[Dự án giảng new 4in1, Trần Quang Thạnh]%[2D1C1-4]
Có bao nhiêu cặp số nguyên dương $(x;y)$ thoả mãn $y\leq 1000$ và $$\log\dfrac{x+1}{3y+1}\leq 9y^2-x^2+6y-2x?$$
\choice
{$1501100$}
{$1501300$}
{$1501400$}
{\True $1501500$}
\loigiai
{Ta có $$\log\dfrac{x+1}{3y+1}\leq 9y^2-x^2+6y-2x \Leftrightarrow \log(x+1)+(x+1)^2\leq \log(3y+1)+ (3y+1)^2.$$
Xét hàm $f(t)=\log t+t^2$ trên $(0;+\infty)$.\\
Ta có $f'(t)=\dfrac{1}{t \ln 10}+2 t>0$, $\forall t \in(0;+\infty)$.\\
Suy ra $f(t)$ là hàm đồng biến trên $t\in(0;+\infty)$.\\
Khi đó $(*)\Leftrightarrow f(x+1) \leq f\left(3y+1\right) \Leftrightarrow x+1 \leq 3y+1 \Leftrightarrow x \leq 3y$.\\
Vì $y \leq 1000$ nên ta có các trường hợp sau
\begin{itemize}
\item $y=1 \Rightarrow x \in\{1;2;3\}$.
\item $y=2 \Rightarrow x \in\{1;2;3;4;5;6\}$.\\
$\ldots$
\item $y=1000 \Rightarrow x \in\{1;2;\cdots;3000\}$.
\end{itemize}
Vậy số cặp nghiệm thoả mãn điều kiện đề bài là $3+6+9+\ldots+3000=1501500$.
}
\end{ex}

\begin{ex}%[Mức độ 3]%[Dự án bài giảng new 4in1, Trần Quang Thạnh]%[2D1C1-3]
Có bao nhiêu giá trị nguyên của tham số $m$ để hàm số $y=\dfrac{16-m^2}{(x+1)^2}$ đồng biến trên $(0;+\infty)$?
\choice
{\True $7$}
{$9$}
{Vô số}
{$11$}
\loigiai{
Ta có $y'=-\dfrac{2(16-m^2)}{(x+1)^3}$.\\
Nhận thấy $y'=0 \Leftrightarrow m=\pm 4$ và khi đó hàm số đã cho là hàm hằng.\\
Do đó, hàm số đã cho đồng biến trên đồng biến trên $(0;+\infty)$ khi và chỉ khi $y'>0$, với mọi $x>0$, tức là $16-m^<0$ hay $-4<m<4$.\\
Vậy có $7$ giá trị nguyên của tham số $m$ để hàm số đã cho đồng biến trên $(0;+\infty)$ là $-3; -2; -1; 0; 1; 2; 3$.
}
\end{ex}


\Closesolutionfile{ans}

\Opensolutionfile{ans}[ans/ansBTchoiceTF]

\TNTF
\begin{ex}%[2D1C5-7]
Cho hàm số $y=\dfrac{-x^2+2(m+1)x-5}{x-1}$. Xét tính đúng sai của các mệnh đề sau.
\choiceTF[t]
{\True Khi $m=0$ thì đồ thị hàm số có tiệm cận xiên là $y=-x+1$}
{\True Khi $m=0$ thì đồ thị hàm số không cắt $Ox$}
{Để hàm số có cực đại cực tiểu thì $m>2$}
{\True Khi $m=0$ thì hàm số có đồ thị là $(C)$. Biết rằng tồn tại điểm $M$ thuộc đồ thị $(C)$ sao cho $x_M>1$  và $IM$ ngắn nhất ($I$ là tâm đối xứng của $(C)$), khi đó $y_M<-4$}
\loigiai
{\begin{itemchoice}
\itemch Đúng. Khi $m=0$ thì $y=\dfrac{-x^2+2x-5}{x-1}=-x+1-\dfrac{4}{x-1}$ nên đồ thị có tiệm cận xiên $y=-x+1$.
\itemch Đúng. Khi $m=0$ thì $y=\dfrac{-x^2+2x-5}{x-1}$ và $y=0\Leftrightarrow -x^2+2x-5=0$ vô nghiệm nên đồ thị hàm số không cắt $Ox$.
\itemch Sai. Ta có $y'=\dfrac{-x^2+2x-2m+3}{(x-1)^2}$.\\
Hàm số có cực đại, cực tiểu khi phương trình $-x^2+2x-2m+3=0$ có $2$ nghiệm phân biệt khác $1$. Điều kiện tương đương là
$$\heva{& \Delta'=(-1)^2-2m+3>0 \\ & -1^2+2\cdot 1-2m+3\neq 0}\Leftrightarrow\heva{& 2m<4 \\ & 2m\neq 4}\Leftrightarrow m<2.$$
\itemch Đúng. Khi $m=0$ thì đồ thị $(C)$ của hàm số $y=\dfrac{-x^2+2x-5}{x-1}=-x+1-\dfrac{4}{x-1}$ có tiệm cận đứng là $x=1$ và tiệm cận xiên là $y=-x+1$. Suy ra giao điểm của hai tiệm cận là $I(1;0)$.\\
Gọi $M\left(x_M;y_M\right)$ là điểm thuộc $(C)$ có $x_M>1$.\\
Ta có $y_M=-x_M+1-\dfrac{4}{x_M-1}$ và \begin{eqnarray*}
IM^2&=&\left(x_M-1\right)^2+\left(-x_M+1\right)^2+\dfrac{16}{\left(x_M-1\right)^2}+8\\
&=&2\left(x_M-1\right)^2+\dfrac{16}{\left(x_M-1\right)^2}+8\\
&\geq & 8\sqrt{2}+8.
\end{eqnarray*}
Dấu ``$=$'' xảy ra khi $2\left(x_M-1\right)^2=\dfrac{16}{\left(x_M-1\right)^2}\Leftrightarrow\left(x_M-1\right)^2=8\Leftrightarrow x_M=1+\sqrt[4]{8}$ (do $x_M>1$).\\
Suy ra $IM$ ngắn nhất bằng $\sqrt{8\sqrt{2}+8}$ khi $x_M=1+\sqrt[4]{8}$.\\
Khi đó $y_M=-\sqrt[4]{8}-\dfrac{4}{\sqrt[4]{8}}<-4$.
\end{itemchoice}
}
\end{ex}

\begin{ex}%[2D1C5-6]
Cho hàm số $y=\dfrac{x^2+3 x+3}{x+2}$ có đồ thị là $(C)$. Xét tính đúng sai của các mệnh đề sau.
\choiceTF[t]
{Biết hàm số có $2$ điểm cực trị khi đó tổng của giá trị cực đại và giá trị cực tiểu bằng $-4$}
{\True Đường tiệm cận xiên của đồ thị hàm số đi qua điểm $A(0;1)$}
{\True Gọi $\Delta$ là tiếp tuyến của $(C)$ và vuông góc với đường thẳng $x-3 y-6=0$. Khi đó $\Delta$ đi qua điểm $B\left(-\dfrac{3}{2};\dfrac{3}{2}\right)$}
{Để phương trình $x^2+3x+3=m|x+2|$ có $4$ nghiệm phân biệt thì $m>2$}
\loigiai
{\begin{itemchoice}
\itemch Sai. Tập xác định $\mathscr{D}=\mathbb{R}\setminus\{-2\}$.\\
Ta có $y'=\dfrac{x^2+4x+3}{(x+2)^2}$ và $y'=0\Leftrightarrow x^2+4x+3=0\Leftrightarrow\hoac{& x=-1 \\ & x=-3}$.\\
Bảng biến thiên
\begin{center}
\begin{tikzpicture}
\tkzTabInit[nocadre=false, lgt=1.2, espcl=2.5, deltacl=0.6]{$x$/0.6,$y'$/0.6,$y$/2}
{$-\infty$, $-3$, $-2$, $-1$, $+\infty$}
\tkzTabLine {,+,0,-,d,-,0,+,}
\tkzTabVar{-/$-\infty$, +/3, -D+/$-\infty$/$+\infty$, -/$-1$, +/$+\infty$}
\end{tikzpicture}
\end{center}
Vậy tổng của giá trị cực đại và giá trị cực tiểu là $3+(-1)=2$.
\itemch Đúng. Ta có $y=x+1+\dfrac{1}{x+2}$ nên đồ thị $(C)$ có tiệm cận xiên là $y=x+1$. Tiệm cận xiên này đi qua $A(0;1)$.
\itemch Đúng. Đường thẳng $x-3y-6=0$ có hệ số góc bằng $\dfrac{1}{3}$ nên tiếp tuyến $\Delta$ có hệ số góc bằng $-3$.\\
Gọi $M\left(x_0;y_0\right)$ là tiếp điểm. Ta có hệ số góc của $\Delta$ là $y'(x_0)=\dfrac{x_0^2+4x_0+3}{\left(x_0+2\right)^2}$. Khi đó
\allowdisplaybreaks\begin{eqnarray*}
y'(x_0)=-3&\Leftrightarrow & \dfrac{x_0^2+4x_0+3}{\left(x_0+2\right)^2}=-3\\
&\Leftrightarrow &\heva{& x_0\neq -2 \\ & x_0^2+4x_0+3=-3\left(x_0^2+4x_0+4\right)}\\
&\Leftrightarrow &\heva{& x_0\neq -2 \\ & 4x_0^2+16x_0+15=0}\\
&\Leftrightarrow &\heva{& x_0\neq -2 \\ & \hoac{& x_0=-\dfrac{3}{2} \\ & x_0=-\dfrac{5}{2}}}\\
&\Leftrightarrow &\hoac{& x_0=-\dfrac{3}{2}\Rightarrow y_0=\dfrac{3}{2} \\ & x_0=-\dfrac{5}{2}\Rightarrow y_0=-\dfrac{7}{2}.}
\end{eqnarray*}
Suy ra có tiếp tuyến $\Delta$ đi qua điểm $B\left(-\dfrac{3}{2};\dfrac{3}{2}\right)$.
\itemch Sai. Nhận thấy $x=-2$ không là nghiệm của phương trình $x^2+3x+3=m|x+2|$ nên ta viết lại $\dfrac{x^2+3x+3}{|x+2|}=m$. Đây là phương trình hoành độ giao điểm giữa đồ thị hàm số $y=\dfrac{x^2+3x+3}{|x+2|}$ và đường thẳng $y=m$.\\
Gọi $(C)$ là đồ thị của hàm số $y=\dfrac{x^2+3x+3}{x+2}$.\\
Ta có $y=\dfrac{x^2+3x+3}{|x+2|}=\heva{& \dfrac{x^2+3x+3}{x+2} &&\text{nếu } x\geq -2 \\ & -\dfrac{x^2+3x+3}{x+2}&& \text{nếu } x<-2.}$\\
Do đó, đồ thị $(C')$ của hàm số $y=\dfrac{x^2+3x+3}{|x+2|}$ gồm phần đồ thị $(C_1)$ trùng với $(C)$ khi $x\geq -2$ và $(C_2)$ đối xứng với $(C)$ qua trục $Ox$ khi $x<-2$.
\begin{center}
\begin{tikzpicture}[scale=1, font=\footnotesize, line join=round, line cap=round,x=0.5cm,y=0.5cm,>=stealth]
\def \xmin{-8.0};
\def \xmax{6.1};
\def \ymin{-7.0};
\def \ymax{6.5};
\def\f(#1){0.03*(#1)^4-0.07*(#1)^3-0.44*(#1)^2+0.9*(#1)+1};
\def\g(#1){0.03*(#1-2)^4-0.07*(#1-2)^3-0.44*(#1-2)^2+0.9*(#1-2)+1};
\draw[->] (\xmin, 0.) -- (\xmax,0.) node[anchor=north] {$x$};
\draw[->] (0.,\ymin) -- (0.,\ymax) node[anchor=west] {$y$};
\clip(\xmin,\ymin) rectangle (\xmax,\ymax);
\begin{scope}
\clip (\xmin,\ymin) rectangle (6,0);
\draw[smooth,dashed,samples=100] plot[domain=\xmin-0.1:-2.01] (\x,{((\x)^2+3*(\x)+3)/((\x)+2)});
\end{scope}
\begin{scope}[yscale=-1]
\clip (\xmin,\ymin) rectangle (6,0);
\draw[smooth,samples=100] plot[domain=\xmin-0.1:-2.01] (\x,{((\x)^2+3*(\x)+3)/((\x)+2)});
\path[font=\tiny,postaction={decorate,decoration={text along path,text align=right, raise=1mm,text={|\tiny|{{$(C_2)$}}}}}] plot[domain=-6.5:-6.0] (\x,{((\x)^2+3*(\x)+3)/((\x)+2)});
\end{scope}
\begin{scope}
\clip (\xmin,0) rectangle (\xmax,\ymax);
\draw[smooth,samples=100] plot[domain=-1.9:\xmax] (\x,{((\x)^2+3*(\x)+3)/((\x)+2)});
\path[font=\tiny,postaction={decorate,decoration={text along path,text align=right, raise=1mm,text={|\tiny|{{$(C_1)$}}}}}] plot[domain=4.5:5] (\x,{((\x)^2+3*(\x)+3)/((\x)+2)});
\end{scope}
\draw (-2,\ymin)--(-2,\ymax);
\draw[dashed] (-3,0)node[below left]{$-3$}|-(0,3)node[right]{$3$} (-3,0)|-(0,-3)node[right]{$-3$} (-1,0)node[below]{$-1$}|-(0,1)node[right]{$1$};
\draw[smooth,samples=100] plot[domain=\xmin:\xmax] (\x,{(\x)+1});
\draw[fill=black] (0,0) circle (1pt) node[below right] {$O$} (-3,3) circle (1pt) (-3,-3) circle (1pt) (0,-3) circle (1pt) (0,3) circle (1pt) (0,1) circle (1pt) (-1,0) circle (1pt) (-1,1) circle (1pt) (-3,0) circle (1pt) (-2,0) circle (1pt);
\end{tikzpicture}
\end{center}
Dựa vào đồ thị, phương trình đã cho có $4$ nghiệm phân biệt khi và chỉ khi $m>3$.
\end{itemchoice}
}
\end{ex}

\begin{ex}%[2D1C5-6]
Cho hàm số $y=x-\dfrac{1}{x+1}$ có đồ thị là $(C)$. Xét tính đúng sai của các mệnh đề sau.
\choiceTF[t]
{Đồ thị của hàm số có tiệm cận đứng là $x=1$}
{\True Đồ thị hàm số cắt trục $Oy$ tại $M$. Phương trình tiếp tuyến của $(C)$ tại $M$ là $y=2x-1$}
{Tồn tại hai tiếp tuyến của đồ thị vuông góc với nhau}
{\True Để đường thẳng $y=k$ cắt $(C)$ tại hai điểm phân biệt $A$ và $B$ sao cho $OA\perp OB$ thì $k$ là nghiệm của phương trình $k^2-k-1=0$}
\loigiai
{\begin{itemchoice}
\itemch Sai. Đồ thị  $(C)$ có tiệm cận đứng là $x=-1$.
\itemch Đúng. Đồ thị $(C)$ cắt trục $Oy$ tại $M(0;-1)$.\\
Ta có $y'=1+\dfrac{1}{(x+1)^2}\Rightarrow y'(0)=2$.\\
Phương trình tiếp tuyến của $(C)$ tại $M$ là $y=2x-1$.
\itemch Sai. Tiếp tuyến của đồ thị $(C)$ tại tiếp điểm $M_1(x_1;y_1)$ có hệ số góc $k_1=y'\left(x_1\right)=1+\dfrac{1}{(x_1+1)^2}>0$.\\
Tiếp tuyến của đồ thị $(C)$ tại tiếp điểm $M_2(x_2;y_2)$ có hệ số góc $k_2=y'\left(x_2\right)=1+\dfrac{1}{(x_2+1)^2}>0$.\\
Khi đó $k_1k_2>0$ nên không tồn tại hai tiếp tuyến của đồ thị vuông góc với nhau.
\itemch Đúng. Phương trình hoành độ giao điểm giữa đồ thị $(C)$ và đường thẳng $y=k$ là $$x-\dfrac{1}{x+1}=k\Leftrightarrow\heva{& x\neq -1 \\ & x^2+x-1=k(x+1).\quad (1)}\quad (I)$$
Nhận thấy $x=-1$ không thỏa mãn $(1)$ nên $$(I)\Leftrightarrow x^2+(1-k)x-1-k=0.\quad (2)$$
Phương trình $(2)$ có $\Delta=(1-k)^2+4(1+k)=k^2+2k+5=(k+1)^2+4>0,\ \forall k$.\\
Do đó, đường thẳng $y=k$ luôn cắt đồ thị $(C)$ tại hai điểm phân biệt $A(x_A;k)$, $B(x_B;k)$ với $x_A$, $x_B$ là nghiệm của phương trình $(2)$.\\
Theo Vi-et thì $x_A x_B=-1-k$.\\
Ta có $OA\perp OB\Leftrightarrow\overrightarrow{OA}\cdot\overrightarrow{OB}=0\Leftrightarrow x_Ax_B+k^2=0\Leftrightarrow -1-k+k^2=0$.\\
Vậy $OA\perp OB$ thì $k$ là nghiệm của phương trình $k^2-k-1=0$.
\end{itemchoice}
}
\end{ex}

\begin{ex}%[2D1C5-2]
Cho hàm số $y= \log_3 \left(\dfrac{1}{x} \right)$ có đồ thị $(C_1)$ và hàm số $y=f(x)$ có đồ thị $(C_2)$ đối xứng với $(C_1)$ qua gốc tọa độ. Xét tính đúng sai của các mệnh đề sau.
\choiceTF[t]
{Hàm số $y=f(x)$ có tập xác định $\mathscr{D}=(0;+\infty)$}
{\True Đồ thị hàm số $y=f(x)$ đi qua điểm $M(-3;1)$}
{Đồ thị hàm số $y=f(x)$ có tiệm cận ngang là trục hoành}
{\True Hàm số $y=\left|f(x)\right|$ nghịch biến trên $(-\infty;-1)$}
\loigiai{
\begin{itemchoice}
\itemch Sai. Hàm số $y=\log_3\left(\dfrac{1}{x}\right)=-\log_3 x$ có tập xác định là $(0;+\infty)$ nên hàm số $y=f(x)$ có tập xác định là $(-\infty;0)$.
\itemch Đúng. Hàm số $y=\log_3\left(\dfrac{1}{x}\right)$ đi qua điểm $N(3;-1)$.\\
Ta có $M(-3;1)$ đối xứng với $N(3;-1)$ qua gốc tọa độ $O$ nên $M$ thuộc đồ thị hàm số $y=f(x)$.
\itemch Sai. Đồ thị hàm số $y=\log_3\left(\dfrac{1}{x}\right)=-\log_3 x$ chỉ có tiệm cận đứng là trục $Oy$ nên đồ thị $(C_2)$ cũng chỉ có tiệm cận đứng là $Oy$.
\itemch Gọi $M(x_0;y_0)$ là điểm thuộc $(C_2)$, $x_0<0$. Khi đó $y_0=f(x_0)$\\
Điểm $N$ đối xứng với $M$ qua gốc tọa độ $O$ có tọa độ là $N(-x_0;-y_0)$.\\
Ta có $N$ thuộc $(C_1)$ nên $-y_0=\log_3\left(\dfrac{1}{-x_0}\right)$ hay $y_0=\log_3(-x_0)$.\\
Do đó $f(x_0)=\log_3(-x_0)$ với $x_0<0$.\\
Suy ra $f(x)=\log_3(-x)$ với $x<0$.\\
Khi đó $y=|f(x)|=|\log_3(-x)|=\heva{&\log_3(-x),&&x \leq -1\\& -\log_3(-x),&&-1<x<0.}$\\
Suy ra $y'=\heva{&\dfrac{1}{x\ln 3}, &&x<-1\\ &-\dfrac{1}{x\ln 3},&&-1<x<0.}$\\
Như thế $y'<0$ khi $x<-1$. Vậy hàm số $y=\left|f(x)\right|$ nghịch biến trên $(-\infty;-1)$.
\end{itemchoice}
}
\end{ex}

\begin{ex}%[2D1C5-2]
Cho hàm số $y=f(x)$ có đồ thị đối xứng với đồ thị hàm số $y=2^x+x$ qua đường thẳng $y=x$. Xét tính đúng sai của các mệnh đề sau.
\choiceTF[t]
{\True Hàm số $y=f(x)$ có tập xác định $\mathscr{D}=\mathbb{R}$}
{Đồ thị hàm số $y=f(x)$ không có đường tiệm cận xiên}
{\True Đồ thị hàm số $y=f(x)$ nên bên dưới đường thẳng $y=x$}
{\True Đồ thị hàm số $y=f(x)$ là một đường đi lên từ trái sang phải}

\loigiai
{\begin{itemchoice}
\itemch Đúng. Hàm số $y=2^x+x$  xác định và liên tục với mọi $x$.\\
Ta có $\lim\limits_{x\to -\infty}y=-\infty$ và $\lim\limits_{x\to +\infty}y=+\infty$ nên nó có tập giá trị là $(-\infty;+\infty)$.\\
Vì đồ thị hàm số $y=f(x)$ đối xứng với đồ thị hàm số $y=2^x+x$ qua đường thẳng $y=x$ nên tập giá trị của hàm số $y=2^x+x$ là tập xác định của hàm số $y=f(x)$.\\
Vậy hàm số $y=f(x)$ có tập xác định $\mathscr{D}=\mathbb{R}$.
\itemch Sai. Hàm số $y=2^x+x$ có tập xác định $\mathscr{D}=\mathbb{R}$ và $\lim\limits_{x\to -\infty}(y-x)=\lim\limits_{x\to -\infty} 2^x=0$ nên đồ thị có tiệm cận xiên $y=x$.\\
Do  đồ thị hàm số $y=f(x)$ đối xứng với đồ thị hàm số $y=2^x+x$ qua đường thẳng $y=x$ nên đồ thị hàm số $y=f(x)$ cũng có tiệm cận xiên.
\itemch Đúng. Ta có $2^x+x>x$, $\forall x\in\mathbb{R}$ nên đồ thị hàm số $y=2^x+x$ nằm phía trên đường thẳng $y=x$.\\
Vì đồ thị hàm số $y=f(x)$ đối xứng với đồ thị hàm số $y=2^x+x$ qua đường thẳng $y=x$ nên đồ thị của $y=f(x)$ nằm bên dưới đường thẳng $y=x$.
\itemch Gọi $M(x_0;y_0)$ là điểm tùy ý thuộc đồ thị hàm số $y=f(x)$. Khi đó, $y_0=f(x_0)$.\\
Ta có điểm đối xứng với $M$ qua đường thẳng $y=x$ là $N(y_0;x_0)$.\\
Hàm số $y=2^x+x$ có $y'=2^x\ln x+1>0$, $\forall x\in\mathbb{R}$ nên đồng biến trên $\mathbb{R}$.\\
Lấy hai điểm $M_1(x_1,y_1)$ và $M_2(x_2;y_2)$ thuộc đồ thị hàm số $y=f(x)$ sao cho $x_1<x_2$.\hfill $(1)$\\
Gọi $N_1$, $N_2$ lần lượt là điểm đối xứng của $M_1$, $M_2$ qua đường thẳng $y=x$.\\
Khi đó $N_1(y_1;x_1)$, $N_2(y_2;x_2)$ thuộc đồ thị hàm số $y=2^x+x$ và do hàm số này đồng biến nên từ $x_1<x_2$ suy ra $y_1<y_2$ hay $f(x_1)<f(x_2)$.\hfill $(2)$\\
Từ $(1)$ và $(2)$ suy ra hàm số $y=f(x)$ là hàm số đồng biến.\\
Vậy đồ thị hàm số $y=f(x)$ là đường đi lên từ trái sang phải.
\end{itemchoice}
}
\end{ex}

\begin{ex}%[Dự án TL12New-4in1-NCT]%[2D1C4-1]
\immini
{
Cho hàm số bậc ba $f(x)$ có đồ thị như hình vẽ.
Xét tính đúng sai của các khẳng định sau.
\choiceTF
{\True Đồ thị hàm số $ g_1(x)=\dfrac{1}{f(x)} $ có $3$ tiệm cận đứng}
{Đồ thị hàm số $ g_2(x)=\dfrac{1}{f(x)-2} $ có $3$ tiệm cận đứng}
{\True Đồ thị hàm số $ g_3(x)=\dfrac{x^2-x}{f(x)} $ có $2$ tiệm cận đứng}
{\True Đồ thị hàm số $ g_4(x)=\dfrac{x^2-x}{\left[f(x)\right]^2-2f(x)} $ có $4$ tiệm cận đứng và $1$ tiệm cận ngang}
}{
\begin{tikzpicture}[scale=.6,>=stealth]
\draw[->](-1.9,0)--(3.5,0)node[below]{$x$};
\draw[->](0,-2.9)--(0,2.9)node[left]{$y$};
\draw[dashed](2,0)--(2,-2)circle(1.5 pt)--(0,-2);
\node at (1,0) [above ] {\footnotesize $1$};
\node at (2,0) [below left] {\footnotesize $2$};
\node at (0,2) [right] {\footnotesize $2$};
\node at (0,-2) [left] {\footnotesize $-2$};
\draw [fill] (0,0) circle (1.5 pt)node[below right] {\footnotesize $O$};
\draw[smooth,samples=100,domain=-1.1:3.1] plot(\x,{(\x)^3-3*(\x)^2+2});
\end{tikzpicture}
}
\loigiai{
\begin{center}
\begin{tikzpicture}[scale=.6,>=stealth]
\draw[->](-1.9,0)--(3.5,0)node[below]{$x$};
\draw[->](0,-2.9)--(0,2.9)node[left]{$y$};
\draw[dashed](2,0)--(2,-2)circle(1.5 pt)--(0,-2) (-1.9,2)--(3.5,2) (3,2)--(3,0);
\node at (1,0) [above ] {\footnotesize $1$};
\node at (2,0) [below left] {\footnotesize $2$};
\node at (0,2) [right] {\footnotesize $2$};
\node at (0,-2) [left] {\footnotesize $-2$};
\node at (3,0) [below] {\footnotesize $3$};
\draw [fill] (0,0) circle (1.5 pt)node[above right] {\footnotesize $O$};
\draw [fill] (-0.75,0) circle (1.5 pt)node[below right] {\footnotesize $a$};
\draw [fill] (2.7,0) circle (1.5 pt)node[above left] {\footnotesize $b$};
\draw[smooth,samples=100,domain=-1.1:3.1] plot(\x,{(\x)^3-3*(\x)^2+2});
\end{tikzpicture}
\end{center}
\begin{itemchoice}
\itemch Dựa vào đồ thị ta thấy $f(x)=0\Leftrightarrow\hoac{&x=a<0\\&x=1\\&x=b>2}$.\\
Từ đó suy ra đồ thị hàm số $g_1(x)$ có $3$  tiệm cận đứng là $ x=a,  x=1, x=b$.
\itemch Dựa vào đồ thị ta thấy $f(x)=2\Leftrightarrow\hoac{&x=0 \text{ nghiệm bội 2}\\&x=b>2}$.\\
Từ đó suy ra đồ thị hàm số $g_2(x)$ có $2$  tiệm cận đứng là $ x=0, x=b$.
\itemch Dựa vào đồ thị ta thấy $f(x)=0\Leftrightarrow\hoac{&x=a<0\\&x=1\\&x=b>2}$.\\
Từ đó suy ra đồ thị hàm số $g_3(x)$ có $2$  tiệm cận đứng là $ x=a, x=b$.
\itemch Dựa vào đồ thị ta thấy $f^2(x)-2f(x)=0\Leftrightarrow\hoac{&f(x)=0\\&f(x)=2}\Leftrightarrow \hoac{&x=a \quad (a<0)\\&x=1\\&x=b\quad  (b>2)\\&x=0 \\&x=3.}$\\
Do đó ta viết $f^2(x)-2f(x)=k (x-a)(x-1)(x-b)x^2(x-3)$.\\
Xét hàm số $  g(x)\dfrac{x^2-x}{\left[f(x)\right]^2-2f(x)} =\dfrac{x(x-1)}{k (x-a)(x-1)(x-b)x^2(x-3)} $.\\
Tập xác định $ \mathscr{D}=\mathbb{R}\backslash\{a;1;b;0;3\} $.\\
Từ đó suy ra đồ thị hàm số $g(x)$ có $4$  tiệm cận đứng là $ x=a,  x=0, x=b, x=3 $ và $1$ tiệm cận ngang là $y=0$.
\end{itemchoice}
}
\end{ex}

\begin{ex}%[Dự án TL12New-4in1-NCT]%[2D1C4-1]
\immini{Cho hàm số $y=f(x)=ax^4+bx^2+c (a\ne 0)$ có đồ thị như hình vẽ.
Xét tính đúng sai của các khẳng định sau.
\choiceTF
{\True Đồ thị hàm số $g(x)=\dfrac{2025(x-2)^3\sqrt{x^2+2026}}{f(x)}$ có $1$ tiệm cận đứng}
{Đồ thị hàm số $g(x)=\dfrac{2025(x+2)^3\sqrt{x^2+2026}}{f(x)}$ có $2$ tiệm cận đứng}
{Đồ thị hàm số $g(x)=\dfrac{2025(x+2)^3\sqrt{x^2+2026}}{f(x)}$ có $1$ tiệm cận ngang}
{\True Đồ thị hàm số $g(x)=\dfrac{2025(x-2)^3\sqrt{x^2+2026}}{f(x)}$ có $2$ tiệm cận ngang}
}{
\begin{tikzpicture}[scale=0.5,>=stealth]
\path
(1,1) coordinate (A);
\draw[->](-3.5,0)--(3.5,0)node[below]{$x$};
\draw[->](0,-1)--(0,5)node[left]{$y$};
\draw[smooth,samples=200,domain=-3:3]plot(\x,{3/16*(\x)^4-3/2*(\x)^2+3});
\draw(0,0)node[below left]{$O$} (-2,0)node[below]{$-2$} (2,0)node[below]{$2$} (0,3)node[ left]{$3$};
\end{tikzpicture}}
\loigiai{
Ta có $f(x)=0\Leftrightarrow\hoac{&x=-2\\&x=2.}$\\
Do đó ta viết $f(x)=a(x+2)^2(x-2)^2$.
\begin{itemchoice}
\itemch Xét hàm số  $g(x)=\dfrac{2025(x-2)^3\sqrt{x^2+2026}}{f(x)}=\dfrac{2025(x-2)^3\sqrt{x^2+2026}}{a(x+2)^2(x-2)^2}$.\\
Hàm số $g(x)$ có tập xác định $\mathscr{D}=\mathbb{R}\backslash\{-2;2\}$.\\
Từ đó suy ra đồ thị hàm số $g(x)$ có một tiệm cận đứng là $x=-2$.
\itemch Xét hàm số  $g(x)=\dfrac{2025(x+2)^3\sqrt{x^2+2026}}{f(x)}=\dfrac{2025(x+2)^3\sqrt{x^2+2026}}{a(x+2)^2(x-2)^2}$.\\
Hàm số $g(x)$ có tập xác định $\mathscr{D}=\mathbb{R}\backslash\{-2;2\}$.\\
Từ đó suy ra đồ thị hàm số $g(x)$ có một tiệm cận đứng là $x=2$.
\itemch Xét hàm số  $g(x)=\dfrac{2025(x+2)^3\sqrt{x^2+2026}}{f(x)}=\dfrac{2025(x+2)^3\sqrt{x^2+2026}}{a(x+2)^2(x-2)^2}$.\\
Hàm số $g(x)$ có tập xác định $\mathscr{D}=\mathbb{R}\backslash\{-2;2\}$.\\
Từ đó suy ra đồ thị hàm số $g(x)$ có hai tiệm cận ngang là $y=-\dfrac{2025}{a}$, $y=\dfrac{2025}{a}$.
\itemch Xét hàm số  $g(x)=\dfrac{2025(x-2)^3\sqrt{x^2+2026}}{f(x)}=\dfrac{2025(x-2)^3\sqrt{x^2+2026}}{a(x+2)^2(x-2)^2}$.\\
Hàm số $g(x)$ có tập xác định $\mathscr{D}=\mathbb{R}\backslash\{-2;2\}$.\\
Từ đó suy ra đồ thị hàm số $g(x)$ có hai tiệm cận ngang là $y=-\dfrac{2025}{a}$, $y=\dfrac{2025}{a}$.
\end{itemchoice}
}
\end{ex}

\begin{ex}%[BG-12NEW-4in1, Nguyen Huynh]%[2D1C4-1]
Cho hàm số $y=\dfrac{x-3}{x+1}$ có đồ thị $(C)$. Xét tính đúng sai của các mệnh đề sau
\choiceTF[t]
{ Hàm số đã cho đồng biến trên $\mathbb R\setminus\{1\}$}
{Đồ thị của hàm số chỉ có tiệm cận ngang là $y=3$}
{\True Hai đường tiệm cận của đồ thị hàm số giao nhau tại điểm $I\left(-1;1 \right)$}
{\True Có hai điểm $M$ trên $(C)$ sao cho tiếp tuyến tại $M$ của $(C)$ tạo với hai đường tiệm cận của $(C)$ một tam giác có bán kính đường tròn nội tiếp lớn nhất}
\loigiai{
\begin{itemchoice}
\itemch Ta có $y=1-\dfrac{4}{x+1} \Rightarrow y'=\dfrac{4}{(x+1)^2}>0$ với $x \neq-1$.
\\Do đó hàm số đã cho đồng biến trên từng khoảng xác định.
\itemch Ta có $\lim\limits_{x\to +\infty}f(x)=\lim\limits_{x\to+\infty}\dfrac{x-3}{x+1}=1$ và $\lim\limits_{x\to -\infty}f(x)=\lim\limits_{x\to-\infty}\dfrac{x-3}{x+1}=1$.
Suy ra đường thẳng $y=1$ là tiệm cận ngang của đồ thị $(C)$.
\itemch	Do $\lim\limits_{x\to (-1)^+}f(x)=\lim\limits_{x\to(-1)^+}\dfrac{x-3}{x+1}=-\infty$ nên đường thẳng $x=-1$ là tiệm cận đứng của đồ thị hàm số $y=f(x)$.
\\Vậy $I(-1;1)$ là giao điểm của hai tiệm cận của đồ thị $(C)$.
\itemch  Gọi $M\left(a ; 1-\dfrac{4}{a+1}\right), a \neq-1$.\\
Phương trình tiếp tuyến của $(C)$ tại $M$ là $$d\colon y=\dfrac{4}{(a+1)^2}(x-a)+1-\dfrac{4}{a+1}.$$
Gọi $A$ và $B$ lần lượt là giao điểm của tiếp tuyến $d$ với đường tiệm cận đứng và tiệm cận ngang.\\
Giao điểm của $d$ và tiệm cận đứng là $A\left(-1 ; 1-\dfrac{8}{a+1}\right)$.\\
Giao điểm của $d$ và tiệm cận ngang là $B(2 a+1 ; 1)$.\\
Suy ra $I A=\dfrac{8}{|a+1|}, I B=2|a+1|, A B=\sqrt{4(a+1)^2+\dfrac{64}{(a+1)^2}}$.\\
Vì $\triangle I A B$ vuông tại $I$ nên $S_{\triangle I A B}=\dfrac{1}{2} I A \cdot I B=8$.\\
Nửa chu vi của $\triangle I A B$ là $p=\dfrac{I A+I B+A B}{2}=\dfrac{4}{|a+1|}+|a+1|+\sqrt{(a+1)^2+\dfrac{16}{(a+1)^2}}$.\\
Bán kính đường tròn nội tiếp $\Delta I A B$ là $r=\dfrac{S_{\Delta I A B}}{p}$ nên $r$ lớn nhất khi $p$ nhỏ nhất.\\
Áp dụng bất đắng AM-GM ta có
$$
\dfrac{4}{|a+1|}+|a+1|+\sqrt{(a+1)^2+\dfrac{16}{(a+1)^2}} \geq 2 \sqrt{\dfrac{4}{|a+1|} \cdot|a+1|}+\sqrt{2 \cdot \sqrt{(a+1)^2 \cdot \dfrac{16}{(a+1)^2}}}=4+2 \sqrt{2}.
$$
Suy ra $p \geq 4+2 \sqrt{2}$.\\
$p$ đạt giá trị nhỏ nhất bằng $4+2 \sqrt{2}$ khi $\dfrac{4}{|a+1|}=|a+1| \Leftrightarrow(a+1)^2=4 \Leftrightarrow\hoac{&a=1 \\& a=-3.}$\\
Vậy có hai điểm thỏa mãn yêu cầu bài toán là $M(1 ;-1), M(-3 ; 3)$.
\end{itemchoice}
}
\end{ex}


\Closesolutionfile{ans}

\Opensolutionfile{ans}[ans/ansBTshortans]

\TNSA
\begin{ex}%[2D1C5-7]
Trong mặt phẳng $Oxy$, xét tứ giác tứ giác $ABCD$ có các đỉnh có hoành độ là các số nguyên liên tiếp và nằm trên đồ thị của hàm số $y=\ln x$. Biết diện tích tứ giác $ABCD$ bằng $\ln \dfrac{91}{90}$, tính tổng các chữ số của hoành độ đỉnh xa gốc tọa độ nhất.
\shortans{$6$}
\loigiai{
\immini
{
Giả sử hoành độ của các đỉnh của tứ giác lần lượt là $a$, $a+1$, $a+2$, $a+3$ ($a\in\mathbb{N}^*$) tương ứng với các đỉnh $A$, $B$, $C$, $D$.\\
Khi đó $A(a;\ln a)$, $B(a+1;\ln (a+1))$, $C(a+2;\ln (a+2))$, $D(a+3;\ln (a+3))$.\\
Xét các điểm $M(a;0)$, $N(a+1;0)$, $P(a+2;0)$, $Q(a+3;0)$ thì các tứ giác $ABNM$, $BCPN$, $CDQP$ và $ADQM$ là các hình thang vuông. Khi đó
}
{
\begin{tikzpicture}[line join=round, line cap = round, >=stealth, scale=.8,font=\footnotesize,transform shape]
\pgfmathsetmacro{\a}{ln(2)/ln(1.5)};
\pgfmathsetmacro{\b}{ln(3)/ln(1.5)};
\pgfmathsetmacro{\c}{ln(4)/ln(1.5)};
\pgfmathsetmacro{\d}{ln(5)/ln(1.5)}
\foreach \x/\y/\z/\g in
{
2/\a/A/135,3/\b/B/90,4/\c/C/90,5/\d/D/90,
2/0/M/-90,3/0/N/-90,4/0/P/-90,5/0/Q/-90,0/0/O/-135
}
\draw[fill=black] (\x,\y) circle(1pt) coordinate (\z) ($(\z)+(\g:3mm)$) node{$\z$};
\draw[->] (-.5,0)--(6,0) node[anchor=north]{$x$};
\draw[->] (0,-.5)--(0,4.3) node[anchor=east]{$y$};
\draw[samples=100,domain=.9:5.5] plot(\x,{ln(\x)/ln(1.5)});
\draw (M)--(A) (N)--(B) (P)--(C) (Q)--(D) (A)--(B)--(C)--(D)--(A);
\end{tikzpicture}
}
\begin{eqnarray*}
2S_{ABCD} &=& 2S_{ABNM} + 2S_{BCPN} + 2S_{CDQP} - 2S_{ADQM}\\
&= & \left[ \ln a + \ln (a+1) \right] + \left[ \ln (a+1) + \ln (a+2) \right] + \left[ \ln (a+2) + \ln (a+3) \right] - 3\left[\ln a + \ln (a+3)\right]\\
&= &2\ln \dfrac{(a+1)(a+2)}{a(a+3)}.
\end{eqnarray*}
Kết hợp với $S_{ABCD}=\ln \dfrac{91}{90}$ ta có
$$
\dfrac{(a+1)(a+2)}{a(a+3)} = \dfrac{91}{90} \Leftrightarrow 91a(a+3) = 90(a+1)(a+2)
\Leftrightarrow a^2 + 3a - 180 = 0 \Leftrightarrow \hoac{&a=12\text{ (thỏa mãn)}\\ &a=-15\text{ (loại)}.}
$$
Suy ra đỉnh xa gốc tọa độ nhất là $D(15;\ln 15)$.\\
Vậy $1+5=6$.
}
\end{ex}

\begin{ex}%[2D1C5-7]
Cho hàm số $y=\dfrac{x^2+mx+m^2-2m-4}{x-2}$ có đồ thị $(C)$.
Tìm $m$ để đồ thị $(C)$ có hai điểm cực trị và hai điểm cực trị cách đều đường thẳng $\Delta\colon 2 x+y+1=0$.
\shortans{$-9$}
\loigiai{
Tập xác định $\mathscr{D}=\mathbb{R}\setminus\{2\}$.\\
Ta có  $y'=\dfrac{x^2-4x+4-m^2}{(x-2)^2}$.\\
Dấu của $y'$ là dấu của $g(x)=x^2-4x+4-m^2$.\\
Hàm số có hai điểm cực trị khi và chỉ khi phương trình $g(x)=0$ có hai nghiệm phân biệt khác $2$. Điều kiện tương đương là
$$\heva{& \Delta'=4-4+m^2=m^2>0\\ &
4-8+4-m^2 \neq 0} \Leftrightarrow m \neq 0.$$
Nghiệm của $g(x)=0$ là $x_1=2-m$, $x_2=2+m$, suy ra hai điểm cực trị của đồ thị hàm số là $A(2-m;4-m)$, $B(2+m;4+3m)$.\\
Khoảng cách từ $A$, $B$ đến đường thẳng $\Delta$ lần lượt là $\mathrm{d}(A,\Delta)=\dfrac{|9-3m|}{\sqrt{5}}$ và $\mathrm{d}(B,\Delta)=\dfrac{|9+5m|}{\sqrt{5}}$.\\
Khi đó $$\mathrm{d}(A,\Delta)=\mathrm{d}(B,\Delta)\Leftrightarrow |9-3m|=|9+5m|\Leftrightarrow\hoac{& 9-3m=9+5m\\ &9-3m=-9-5m}\Leftrightarrow\hoac{& m=0 \\ & m=-9.}$$
So với điều kiện $m \neq 0$ ta nhận $m=-9$.\\
Vậy giá trị $m$ cần tìm là $m=-9$.
}
\end{ex}

\begin{ex}%[Dự Án Giảng 12 4 in 1, Lê Văn Toàn]%[2D1C5-7]
Cho hàm số $y=\dfrac{x+2}{x-1}$ có đồ thị $(C)$. Gọi $I$ là giao điểm hai đường tiệm cận của $(C)$. Biết tọa độ điểm $M(a; b)$ có hoành độ dương thuộc đồ thị $(C)$ sao cho $M I$ ngắn nhất. Tính giá trị của $ab-2\sqrt{3}$.
\shortans{$4$}
\loigiai{
Giả sử $M\left(x_0; \dfrac{x_0+2}{x_0-1}\right) \in(C)\left(x_0>0; x_0 \neq 1\right)$.\\
Giao điểm của hai đường tiệm cận của $(C)$ là $I(1; 1)$.\\
Khi đó $M I=\sqrt{\left(1-x_0\right)^2+\left(1-\dfrac{x_0+2}{x_0-1}\right)^2}=\sqrt{\left(1-x_0\right)^2+\dfrac{9}{\left(x_0-1\right)^2}} \geq \sqrt{6}$.\\
Dấu bằng xảy ra khi $$\left(1-x_0\right)^2=\dfrac{9}{\left(x_0-1\right)^2} \Leftrightarrow\left[\begin{array}{l}x_0=1+\sqrt{3} \Rightarrow y_0=1+\sqrt{3} \\ x_0=1-\sqrt{3} \text { (loại).}\end{array}\right.$$
Suy ra $M(1+\sqrt{3}; 1+\sqrt{3}) \Rightarrow(1+\sqrt{3})^2=4+2 \sqrt{3}$.
}
\end{ex}

\begin{ex}%[Dự Án Giảng 12 4 in 1, Lê Văn Toàn]%[2D1C5-6]
Cho hàm số $y=\dfrac{1}{4}x^4-\dfrac{7}{2}x^2$ có đồ thị $(C)$. Tiếp tuyến tại điểm $A$ thuộc $(C)$ cắt $(C)$ tại hai điểm phân biệt $M\left(x_1;y_1\right)$, $N\left(x_2;y_2\right)$ ($M$, $N$ khác $A)$ thỏa mãn $y_1-y_2=6\left(x_1-x_2\right)$. Các điểm $A$ thỏa mãn có tổng các hoành độ là
\shortans{$-3$}
\loigiai{
Gọi $A\left(x_0;y_0\right)\in\,(C)$ là tọa độ tiếp điểm của phương trình tiếp tuyến.\\
Ta có hệ số góc $k=y'\left(x_0\right)=x_0^3-7x_0$.\\
Phương trình tiếp tuyến $y=k\left(x-x_0\right)+y_0=\left(x_0^3-7x_0\right)\left(x-x_0\right)+y_0$.\\
Ta có
\begin{eqnarray*}
&&y_1-y_2=6\left(x_1-x_2\right)\\
&\Leftrightarrow& k\left(x_1-x_0\right)+y_0-\left[k\left(x_2-x_0\right)+y_0\right]=6\left(x_1-x_2\right)\\
&\Leftrightarrow& k\left(x_1-x_2\right)=6\left(x_1-x_2\right)\\
&\Leftrightarrow& k=6\\
&\Leftrightarrow& x_0^3-7x_0=6\\
&\Leftrightarrow& x_0^3-7x_0-6=0\\
&\Leftrightarrow& \hoac{&x_0=3\Rightarrow y_0=-\dfrac{45}{4}\\&x_0=-1\Rightarrow y_0=-\dfrac{13}{4}\\&x_0=-2\Rightarrow y_0=-10.}
\end{eqnarray*}
Khi đó các phương trình tiếp tuyến tương ứng là
$$\hoac{&d_1\colon y=6(x-3)-\dfrac{45}{4}=6x-\dfrac{117}{4}\\&d_2\colon y=6(x+1)-\dfrac{13}{4}=6x+\dfrac{11}{4}\\&d_3\colon y=6(x+2)-10=6x+2.}$$
Phương trình hoành độ giao điểm của $(C)$ với các tiếp tuyến là
$$\hoac{&\dfrac{1}{4}x^4-\dfrac{7}{2}x^2-6x+\dfrac{117}{4}=0\text{ (có 1 nghiệm nên không thỏa)}\\&\dfrac{1}{4}x^4-\dfrac{7}{2}x^2-6x-\dfrac{11}{4}=0\text{ (có 3 nghiệm nên thỏa mãn)}\\&\dfrac{1}{4}x^4-\dfrac{7}{2}x^2-6x-2=0\text{ (có 3 nghiệm nên thỏa mãn).}}$$
Do đó tổng các hoành độ điểm các tiếp điểm là $-1-2=-3$.
}
\end{ex}

\begin{ex}%[2D1C5-5]
\immini
{
Cho hàm số $ f(x)=\dfrac{ax+b}{cx+d} $ (với $ a,\, b,\, c,\, d $ là các số thực) có đồ thị hàm số $ f'(x) $ như hình vẽ. Biết rằng giá trị lớn nhất của hàm số $ y=f(x) $ trên đoạn $ [-3;-2] $ bằng $ 7 $. Giá trị $ f(2) $ bằng
}
{
\begin{tikzpicture}[>=stealth,line join=round,line cap=round,scale=.7]
\def\f(#1){-3/((#1)+1)}
\draw[->] (0,-1)--(0,8)node[right]{$y$};
\draw[->] (-5,0)--(5,0)node[below]{$x$};
\clip (-5,-1) rectangle (5-0.1,8-0.1);
\draw[thick,samples=150,smooth,domain=5:-5] plot(\x,{abs(\f(\x))});
\draw (-1,-1)--(-1,8);
\fill (-1,0)node[below left]{$ -1 $};
\fill (0,0)node[below left]{$O$}circle(1pt);
\end{tikzpicture}
}
\shortans{$3$}
\loigiai{
Ta có $ f'(x)=\dfrac{ad-bc}{(cx+d)^2} $.\\
Từ đồ thị ta có $\heva{&-c+d=0\\ &a d-bc=3d^2}\Leftrightarrow\heva{&c=d\\ &a d-b d=3d^2}\Leftrightarrow\heva{&c=d\\ &a-b=3d.}$ \\
Từ đồ thị $ f'(x)>0 $ nên hàm số $ f(x)=\dfrac{ax+b}{cx+d} $ đồng biến trên $ (-\infty;-1)  $ và $ (-1;+\infty) $.\\
$\Rightarrow\max\limits_{[-3 ; -2]} f(x)=f(-2)=7 \Rightarrow \dfrac{-2a+b}{2c+d} = 7 \Leftrightarrow \dfrac{-2(3d+b)+b}{-2d+d}=7 \Leftrightarrow -6d-b=-7d \Leftrightarrow b=d$.\\
Vậy $f(2)=\dfrac{2a+b}{2c+d}=\dfrac{9d}{3d}=3$.
}
\end{ex}

\begin{ex}%[2D1C5-4]
Cho đường thẳng $d: y=mx+m+2$ ($m$ là tham số) và đường cong $(C): y=\dfrac{2x-1}{x+1}$. Biết rằng khi $m=m_0$ thì $(C)$ cắt $d$ tại hai điểm $A, B$ thỏa mãn độ dài $AB$ ngắn nhất. Tìm $m_0$.
\shortans{$-1$}
\loigiai{Tập xác đinh $\mathscr{D}=\mathbb{R}\setminus \{-1\}$.\\
Phương trình hoành độ giao điểm $mx+m+2=\dfrac{2x-1}{x+1}\Leftrightarrow \dfrac{mx^2+2mx+m+3}{x+1}=0$.\\
Điều kiện cần và đủ để $d$ cắt $C$ tại hai điểm phân biệt là phương trình $mx^2+2mx+m+3=0$ có hai nghiệm phân biệt khác $-1.$ \\
Điều này tương đương
$\heva{&m\ne 0\\&\Delta'=-3m>0\\&3\ne 0}\Leftrightarrow m<0.$\\
Với $m<0$ thì $d$ cắt $C$ tại hai điểm $A(x_1;mx_1+m+2)$ và $B(x_2;mx_2+m+2)$.\\
Theo Vi-et $x_1+x_2=-2$, $x_1x_2=1+\dfrac{3}{m}$. Ta có
\begin{eqnarray*}
AB^2&=& (x_1-x_2)^2+(mx_1-mx_2)^2\\
&=& (m^2+1)((x_1+x_2)^2-4x_1x_2)\\
&=& -\dfrac{12}{m}(m^2+1)\\
&=& -12m-\dfrac{12}{m}\ge 2\sqrt{\dfrac{-12}{m}(-12m)}=24.
\end{eqnarray*}
Dấu bằng xảy ra khi và chỉ khi $m^2=1\Leftrightarrow \heva{&m=1\\&m=-1.}$\\
Kết hợp với $m<0$ ta có $m=-1$ thỏa yêu cầu bài toán.
}
\end{ex}

\begin{ex}%[2D1C5-4]
%[Thi thử L1, chuyên Hùng Vương, Gia Lai 2018]%[2D1G5-4]%[Nguyễn Tài Chung, 12EX-7]
Cho hàm số đa thức bậc ba $y=f(x)$ có đồ thị đi qua các điểm $A(2;4)$, $B(3;9)$, $C(4;16)$. Các đường thẳng $AB, AC, BC$ lại cắt đồ thị tại lần lượt tại các điểm $D$, $E$, $F$ ($D$ khác $A$ và $B$; $E$ khác $A$ và $C$; $F$ khác $B$ và $C$). Biết rằng tổng các hoành độ của $D$, $E$, $F$ bằng $24$. Tính $f(0)$.
\shortans{$6{,}25$}
\loigiai{
Giải sử $f(x)=a(x-2)(x-3)(x-4)+x^2$ ($a\ne 0$). Ta có
$$AB\colon y=5x-6;AC\colon y=6x-8;BC\colon y=7x-12.$$
Hoành độ điểm $D$ là nghiệm của phương trình
\begin{eqnarray*}
& & a\left({x-2}\right)\left({x-3}\right)\left({x-4}\right)=-x^2+5x-6\\
&\Leftrightarrow & a\left({x-2}\right)\left({x-3}\right)\left({x-4}\right)=-\left({x-2}\right)\left({x-3}\right)\\
&\Leftrightarrow & a(x-4)=-1\Rightarrow x=-\dfrac{1}{a}+4.
\end{eqnarray*}
Hoành độ điểm $E$ là nghiệm của phương trình
\begin{eqnarray*}
& & a\left({x-2}\right)\left({x-3}\right)\left({x-4}\right)=-x^2+6x-8\\
&\Leftrightarrow & a\left({x-2}\right)\left({x-3}\right)\left({x-4}\right)=-\left({x-2}\right)\left({x-4}\right)\\
&\Leftrightarrow & a(x-3)=-1\Rightarrow x=-\dfrac{1}{a}+3.
\end{eqnarray*}
Hoành độ điểm $F$ là nghiệm của phương trình
\begin{eqnarray*}
& & a\left({x-2}\right)\left({x-3}\right)\left({x-4}\right)=-x^2+7x-12\\
&\Leftrightarrow & a\left({x-2}\right)\left({x-3}\right)\left({x-4}\right)=-\left({x-3}\right)\left({x-4}\right)\\
&\Leftrightarrow & a(x-2)=-1\Rightarrow x=-\dfrac{1}{a}+2.
\end{eqnarray*}
Theo giả thiết ta có $$-\dfrac{1}{a}+2-\dfrac{1}{a}+3+-\dfrac{1}{a}+4=24\Leftrightarrow -\dfrac{3}{a}=15\Leftrightarrow a=-\dfrac{1}{5}.$$
Do đó $f(0)=a\left({-2}\right)\left({-3}\right)\left({-4}\right)=\dfrac{24}{5}=6{,}25$.}
\end{ex}

\begin{ex}%[2D1C5-4]
Tập hợp tất cả các giá trị thực của tham số $m$ để đồ thị của hàm số $y=x^3-3x^2+2m+1$ cắt trục hoành tại ba điểm phân biệt cách đều nhau là
\shortans{$0{,}5$}
\loigiai{
Phương trình hoành độ giao điểm $x^3-3x^2+2m+1=0\quad (*)$.\\
Giả sử $x_1$; $x_2$; $x_3$ là ba nghiệm của $(*)$.\\
Để $x_1$; $x_2$; $x_3$ cách đều nhau $\Leftrightarrow$ $2x_2=x_1+x_3$ \quad $(1)$\\
Mặt khác\\ $x^3-3x^2+2m+1=(x-x_1)(x-x_2)(x-x_3)$
$=x^3-(x_1+x_2+x_3)x^2+(x_1x_2+x_2x_3+x_3x_1)x-x_1x_2x_3$\\
Nên $x_1+x_2+x_3=3$ $(2)$\\
Từ $(1)$ và $(2)$ suy ra $x_2=1$ $(3)$\\
Thế $(3)$ vào $(*)$ ta được $1-3+2m+1=0 \Leftrightarrow m=\dfrac{1}{2}$.\\
Với $m=\dfrac{1}{2}$ thế vào $(*)$, ta được $x^3-3x^2+2=0 \Leftrightarrow \hoac{&x=1-\sqrt{3}\\&x=1\\&x=1+\sqrt{3}.}$\\
Rõ ràng $3$ nghiệm này cách đều nhau.\\ Vậy $m=\dfrac{1}{2}=0{,}5$ là giá trị cần tìm.}
\end{ex}

\begin{ex}%[2D1C5-4]
Số giao điểm của hai đồ thị hàm số $f(x)=2(m+1)x^3+2mx^2-2(m+1)x-2m$, $\left( m \text{ là tham số khác} -\dfrac{3}{4}\right)$  và $g(x)=-x^4+x^2$ là
\shortans{$4$}
\loigiai{
Phương trình hoành độ giao điểm của hai đồ thị hàm số là
\begin{eqnarray*}
&&-x^4+x^2=2(m+1)x^3+2mx^2-2(m+1)x-2m\\
&\Leftrightarrow & -x^2(x^2-1)=2m(x^3+x^2-x-1)+2x^3-2x\\
&\Leftrightarrow & -x^2(x^2-1)=2m(x^2-1)(x+1)+2x(x^2-1)\\
&\Leftrightarrow &  (x^2-1)\left[x^2+2(m+1)x+2m\right]=0\\
&\Leftrightarrow & \hoac{&x^2-1=0\\&h(x)=x^2+2(m+1)x+2m=0}\\
&\Leftrightarrow & \hoac{&x=\pm 1\\&h(x)=x^2+2(m+1)x+2m=0 \quad(1)}
\end{eqnarray*}
Xét $(1)$ có $\heva{&\Delta =m^2+1>0,\forall m\\&h(-1)=-1\ne 0, \forall m\\&h(1)=4m+3 \ne 0, \forall m \ne -\dfrac{3}{4}.}$\\
$\Rightarrow$ Phương trình $(1)$ luôn có $2$ nghiệm phân biệt khác $\pm 1$.\\
Vậy phương trình đã cho có $4$ nghiệm phân biệt.
}
\end{ex}

\begin{ex}%[2D1C5-4]
\immini{
Cho hàm số bậc ba $y=f(x)$ có đồ thị như hình vẽ bên. Tìm số điểm cực trị của hàm số $g(x)=f\left(\mathrm{e}^x-x\right)$.

}{
\begin{tikzpicture}[scale=0.6, font=\footnotesize, line join=round, line cap=round,>=stealth]
\def\xmin{-1} \def\xmax{5}
\def\ymin{-2} \def\ymax{3}
\draw[->] (\xmin,0)--(\xmax,0) node [below]{$x$};
\draw[->] (0,\ymin)--(0,\ymax) node [right]{$y$};
\node at (0,0) [above left]{$O$};
\draw[color=black] (-0.7,-2) parabola bend (1,1.7) (2,0) parabola bend (3,-1.6) (4.6,3);
\draw[dashed] (1,0) -- (1,1.7);
\foreach\i/\j/\goc/\diem in{1/0/-90/1,2/0/-100/2}
\fill(\i,\j) circle(1.0pt) node[shift={(\goc:10pt)}]{$\diem$};
\end{tikzpicture}
}
\shortans{$3$}
\loigiai{
Từ đồ thị của hàm số $y=f(x)$ ta có $f'(x)=0\Leftrightarrow\hoac{&x=1\\&x=a>2.}$ \\
Ta có $g'(x)=f'\left(\mathrm{e}^x-x\right)\cdot \left(\mathrm{e}^x-1\right)$ và $g'(x)=0\Leftrightarrow\hoac{&\mathrm{e}^x-1=0\\&\mathrm{e}^x-x=1\\&\mathrm{e}^x-x=a>2}\Leftrightarrow\hoac{&x=0 &&\\&\mathrm{e}^x-x=1&&(1)\\&\mathrm{e}^x-x=a>2.&&(2)}$ \\
Xét hàm số $h(x)=\mathrm{e}^x-x$ trên $\mathbb{R}$, ta có $h'(x)=\mathrm{e}^x-1=0\Leftrightarrow x=0$.\\
Bảng biến thiên của hàm số $y=h(x)$
\begin{center}
\begin{tikzpicture}
\tkzTabInit[nocadre=false,lgt=1.2,espcl=2.5,deltacl=0.6]
{$x$ /0.8, $h'(x)$ /0.8, $h(x)$ /2.5}
{$-\infty$,$0$,$+\infty$}
\tkzTabLine{,-,$0$,+,}
\tkzTabVar{+/$+\infty$, -/$1$,+/$+\infty$}
\end{tikzpicture}
\end{center}
Phương trình $(2)$ có $2$ nghiệm phân biệt khác $0$, phương trình $(1)$ có nghiệm kép $x=0$, do đó phương trình $g'(x)=0$ có $3$ nghiệm trong đó $x=0$ là nghiệm bội $3$.\\
Vậy hàm số $g(x)=f\left(\mathrm{e}^x-x\right)$ có $3$ điểm cực trị.
}
\end{ex}

\begin{ex}%[Mức độ C]%[2D1C5-3]
Cho hàm số $y=f(x)$ có đạo hàm liên tục trên $\mathbb{R}$ và có đồ thị $y=f'(x)$ như hình vẽ. Đặt $g(x)=f(x-m)-\dfrac{1}{2}(x-m-1)^2+2024$, với $m$ là tham số thực. Gọi $S$ là tập hợp các giá trị nguyên dương của $m$ để hàm số $y=g(x)$ đồng biến trên khoảng $(5;6)$. Tổng tất cả các phần tử trong $S$ bằng bao nhiêu?
\begin{center}
\begin{tikzpicture}[>=stealth]
\draw [->] (-2,0)--(4,0);
\draw [->] (0,-3)--(0,3);
\draw (0,0) node[below left]{$O$};
\draw (4,0) node[below]{$x$};
\draw (0,3) node[below left]{$y$};
\draw (3,0) node[below]{$3$};
\draw (0,2) node[above left]{$2$};
\foreach \x in {-1,1,2,}{\draw (\x,-.1)--(\x,.1) node[below left,black]{$\x$};}
\foreach \y in {-2}{\draw [-] (-.1,\y)--(.1,\y) node[below left,black]{$\y$};}
\clip (-2,-3) rectangle (4,3);
\draw [thick,samples=100] plot[domain=-4:4](\x,{(\x)^3-3*(\x)^2+2});
\draw (3.1,2.9) node[below left]{$(C)$};
\draw[dashed] (2,0)--(2,-2)--(0,-2);
\draw[dashed] (-1,0)--(-1,-2)--(0,-2);
\draw[dashed] (3,0)--(3,2)--(0,2);
\end{tikzpicture}
\end{center}
\shortans{$4$}
\loigiai{Xét hàm số $g(x)=f(x-m)-\dfrac{1}{2}(x-m-1)^2+2019\text{; }g'(x)=f'(x-m)-(x-m-1)$.\\
Cho $g'(x)=0 (1)$.\\
Đặt $x-m=t$, phương trình $(1)$ trở thành $f'(t)-(t-1)=0 \Leftrightarrow f'(t)=t-1 \text{  (2)}$.\\
Nghiệm của phương trình  $(2)$ là hoành độ giao điểm của hai đồ thị hàm số $y=f'(t)$ và $y=t-1$.\\
Đồ thị hai hàm số $y=f'(t)$ và $y=t-1$.
\begin{center}
\begin{tikzpicture}[>=stealth]
\draw [->] (-2,0)--(4,0);
\draw [->] (0,-3)--(0,3);
\draw (0,0) node[below left]{$O$};
\draw (4,0) node[below]{$x$};
\draw (0,3) node[below left]{$y$};
\draw (3,0) node[below]{$3$};
\draw (0,2) node[above right]{$2$};
\foreach \x in {-1,1,2,}{\draw (\x,-.1)--(\x,.1) node[above,black]{$\x$};}
\foreach \y in {-2}{\draw [-] (-.1,\y)--(.1,\y) node[below left,black]{$\y$};}
\clip (-2,-3) rectangle (4,3);
\draw [thick,samples=100] plot[domain=-4:4](\x,{(\x)^3-3*(\x)^2+2});
\draw (3.1,2.9) node[below left]{$(C)$};
\draw[dashed] (2,0)--(2,-2)--(0,-2);
\draw[dashed] (-1,0)--(-1,-2)--(0,-2);
\draw[dashed] (3,0)--(3,2)--(0,2);
\draw [thick,samples=100] plot[domain=-4:4](\x,{(\x)-1});
\fill[black] (-1,-2) circle(2pt);
\fill[black] (1,0) circle(2pt);
\fill[black] (3,2) circle(2pt);
\end{tikzpicture}
\end{center}
Từ đồ thị ta có phương trình $(2)$ có nghiệm là $\left[\begin{array}{l}
t=-1\\t=1\\t=3
\end{array}\right. \Rightarrow \left[\begin{array}{l}
x=m-1\\x=m+1\\x=m+3.
\end{array}\right.$\\
Bảng biến thiên
\begin{center}
\begin{tikzpicture}
\tkzTabInit[nocadre=false, lgt=1.5,espcl=3.5]
{$x$/1,$y'$/1,$y$/2}
{$-\infty$,$m-1$,$m+1$,$m+3$,$+\infty$}
\tkzTabLine{,-,0,+,0,-,0,+, }
\tkzTabVar{+/$+\infty$,-/$ $,+/$ $,-/$ $,+/$+\infty$/}
\end{tikzpicture}
\end{center}
Để hàm số $y=g(x)$ đồng biến trên khoảng $(5;6)$ cần $\left[\begin{array}{l}
\begin{cases}
m-1 \leq 5\\m+1 \geq 6
\end{cases}\\
m+3 \leq 5
\end{array}\right. \Leftrightarrow \left[\begin{array}{l}
5 \leq m \leq 6\\
m \leq 2.
\end{array}\right.$\\
Vì $m \in \mathbb{N^{*}}$ mên $m = \{1;2;5;6\}$.\\
Vậy $S=1+2+5+6=14$.
}


\end{ex}

\begin{ex}%[2D1C5-2]
Cho hàm số $f(x)=\dfrac{x^2+5x+2}{2x+1}$. Có tất cả bao nhiêu giá trị nguyên dương của tham số $m$ để bất phương trình $2021f\left(\sqrt{3x^2-18x+28}\right)-m\sqrt{3x^2-18x+28} \geq m+4042$ nghiệm đúng với mọi $x$ thuộc đoạn $[2;4]$?
\shortans{$673$}
\loigiai{
Đặt $u=\sqrt{3 x^2-18 x+28}=\sqrt{3(x-3)^2+1}=\sqrt{3(x-2)(x-4)+4}$.\\
Hàm số $t=3x^2-18x+28$ có $t'=6x-18$ và $t'=0\Leftrightarrow t=3$.\\
Bảng biến thiên của $t$ trên đoạn $[2;4]$ như sau \begin{center}
\begin{tikzpicture}
\tkzTabInit[nocadre=false, lgt=1.2, espcl=2.5, deltacl=0.6]{$x$/0.6,$t'$/0.6,$t$/2}
{$2$, $3$, $4$}
\tkzTabLine {,-,0,+,}
\tkzTabVar{+/$4$, -/$1$, +/$4$}
\end{tikzpicture}
\end{center}
Suy ra $u\in [1;2]$ khi $x\in [2;4]$.\\
Bất phương trình đã cho được viết lại $2021f(u)-mu \geq m+4042 \Leftrightarrow 2021[f(u)-2] \geq m(u+1)$.\\
Ta có $f(x)=\dfrac{x^2+5x+2}{2x+1}$ nên $f(u)-2=\dfrac{u^2+5u+2}{2u+1}-2=\dfrac{u^2+u}{2u+1}$.\\ Do vậy bất phương trình được viết lại thành $\dfrac{2021\left(u^2+u\right)}{2u+1} \geq m(u+1) \Leftrightarrow m \leq \dfrac{2021u}{2u+1}$.\\
Lúc này yêu cầu bài toán tương đương $m \leq \dfrac{2021u}{2u+1},\forall u \in [1;2] \Leftrightarrow m \leq \min\limits_{u \in 1;2]}g(u)$.\\
Xét hàm số $g(u)=\dfrac{2021u}{2u+1}$, $u \in [1;2]$ ta có $g'(u)=\dfrac{2021}{(2u+1)^2}>0$, $\forall u \in [1;2]$.\\
Do vậy hàm số $g(u)$ tăng trên đoạn $[1;2]$.\\
Vì vậy $\min\limits_{u \in [1;2]}g(u)=\dfrac{2021u}{2u+1}=g(1)=\dfrac{2021}{3}$.\\
Kết hợp với $m$ là các số nguyên dương ta được $m \in\{1 ; 2 ; 3 ; \ldots ; 673\}$.\\
Vậy tìm được $673$ số nguyên dương thỏa mãn yêu cầu bài toán.
}
\end{ex}

\begin{ex}%[Dự án Giảng 12 Nhóm Toán & LaTex, Lê Minh Thiện Anh]%[2D1C5-1]
Cho hàm số $f(x)=\dfrac{2-ax}{bx-c}\,(a, b, c \in \mathbb{R}, b \neq 0)$ có bảng biến thiên như sau
\begin{center}
\begin{tikzpicture}
\tkzTabInit[nocadre=false,lgt=1.2,espcl=3]
{$x$/.6,$f'(x)$/.6,$f(x)$/2}
{$-\infty$,$1$,$+\infty$}
\tkzTabLine{ ,+,d,+,}
\tkzTabVar{-/$3$,+D-/$+\infty$/$-\infty$,+/$3$}
\end{tikzpicture}
\end{center}
Tổng $(a+b+c)^2$ thuộc khoảng $\left(0;\dfrac{4}{n}\right)$. Tìm $n$.
\shortans{$9$}
\loigiai{
Ta có $\lim\limits_{x \rightarrow \infty}\dfrac{2-a x}{b x-c}=\dfrac{-a}{b}$, theo giả thiết suy ra $\dfrac{-a}{b}=3 \Leftrightarrow a=-3b$.\\
Hàm số không xác định tại $x=1 \Rightarrow b-c=0 \Leftrightarrow b=c$.\\
Hàm số đồng biến trên từng khoảng xác định nên $f'(x)=\dfrac{ac-2b}{(bx-c)^2}>0$, $\forall x\neq 1$.\\
Suy ra $ac-2b>0 \Leftrightarrow-3b^2-2b>0 \Leftrightarrow-\dfrac{2}{3}<b<0 \Leftrightarrow 0<-b<\dfrac{2}{3}$.\\
Lại có $a+b+c=-3b+b+b=-b$. Suy ra $(a+b+c)^2=b^2 \in\left(0 ; \dfrac{4}{9}\right)$.\\
Vậy tổng $a+b+c$ thuộc khoảng $\left(0;\dfrac{4}{9}\right)$. Vậy $n=9$.
}
\end{ex}

\begin{ex}%[Dự án Giảng 12 Nhóm Toán & LaTex, Lê Minh Thiện Anh]%[2D1C5-1]
Biết hàm số $f(x)=x^3+ax^2+bx+c$ đạt cực đại tại điểm $x=-3$, $f(-3)=28$ và đồ thị của hàm số cắt trục tung tai điểm có tung độ bằng $1$. Tính $S=a^2+b^2-c^2$.
\shortans{$89$}
\loigiai{
Ta có $f'(x)=3 x^2+2 a x+b$; $f''(x)=6x+2a$.\\
Hàm số $f(x)$ đạt cực đại tại điểm $x=-3$ khi và chi khi $\heva{& f'(-3)=0 \\& f''(-3)<0} \Leftrightarrow\heva{& -6a+b=-27 \\ & a<9}$ (1).\\
Mà $f(-3)=28 \Rightarrow 9 a-3 b+c=55(2)$.\\
Ngoài ra, đồ thị của hàm số $f(x)$ cắt trục tung tại điểm có tung độ bằng 1 nên $c=1$ (3).\\
Tù (1), (2), (3) suy ra $\heva{& -6 a+b=-27 \\& 9a-3b+c=55 \\& c=1 \\& a<9} \Leftrightarrow\heva{& a=3 \\& b=-9 \\& c=1 \\ & a<9.}$\\
Do đó $S=3^2+(-9)^2-1^2=89$.
}
\end{ex}

\begin{ex}%[Mức độ 4]giảng 12, Phạm Tiến Long]%[2D1C4-3]
Cho hàm số $f(x)=\dfrac{x^2-2}{x-4}$ có đồ thị $(C)$. Biết đường thẳng $\Delta\colon y=-x+m$ cắt tiệm cận đứng và tiệm cận xiên của $(C)$ lần lượt tại hai điểm $B$, $C$ sao cho tam giác $OBC$ có diện tích bằng $\dfrac{11}{4}$ (với $O$ là gốc tọa độ). Biết $m$ là số nguyên và lớn hơn $1$. Tính giá trị $m^2-1$.
\shortans{$120$}
\loigiai{
Hàm số đã cho có tập xác định là $\mathbb{R}\backslash\{4\}$.\\
Ta có $\lim\limits_{x\to 4^+}f(x)=+\infty$  và 	$\lim\limits_{x\to 4^-}f(x)=-\infty$.\\
Suy ra tiệm cận đứng của $(C)$ là đường thẳng $d\colon x=4$.\\
Mặt khác,	ta có $\begin{aligned}[t]
a&=\lim\limits_{x \rightarrow+\infty} \dfrac{f(x)}{x}=\lim\limits_{x \rightarrow+\infty} \dfrac{x^2-2}{x^2-4x}=1;\\
b&=\lim\limits_{x \rightarrow+\infty}[f(x)-x]=\lim\limits_{x \rightarrow+\infty}\left(\dfrac{x^2-2}{x-4}-x\right)=\lim\limits_{x \rightarrow+\infty} \dfrac{4x-2}{x-4}=4.
\end{aligned}$\\
Ta cũng có $\lim\limits_{x \rightarrow-\infty} \dfrac{f(x)}{x}=1$; $\lim\limits_{x \rightarrow-\infty}[f(x)-x]=4$.
\\
Do đó, đồ thị hàm số có tiệm cận xiên là đường thẳng $d'\colon y=x+4\Leftrightarrow x-y+4=0$.\\
Đường thẳng $\Delta\colon y=-x+m$ cắt hai đường thẳng $d$ và $d'$ lần lượt tại hai điểm $B(4;m-4)$ và $C\left(\dfrac{m-4}{2};\dfrac{m+4}{2}\right)$.\\
Ta có
\begin{itemize}
\item $BC=\sqrt{\left(\dfrac{m-4}{2}-4\right)^2+\left(\dfrac{m+4}{2}-m+4\right)^2}=\sqrt{\dfrac{1}{2}m^2-12m+72}$.
\item $\mathrm{d}(O,\Delta)=\dfrac{|m|\sqrt{2}}{2}$.
\end{itemize} .\\
Theo giả thiết ta có
\begin{eqnarray*}
& & S_{\triangle OBC}=\dfrac{11}{4}\\
&\Leftrightarrow & \dfrac{1}{2}\cdot BC \cdot \mathrm{d}(O,\Delta)=\dfrac{11}{4}\\
&\Leftrightarrow & \dfrac{1}{2}\sqrt{\dfrac{1}{2}m^2-12m+72} \cdot \dfrac{|m|\sqrt{2}}{2}=\dfrac{11}{4}\\
&\Leftrightarrow & \dfrac{1}{4}\cdot \left(\dfrac{1}{2}m^2-12m+72\right)\cdot \dfrac{m^2}{2} =\dfrac{121}{16} \\
&\Leftrightarrow & \dfrac{1}{16}m^4-\dfrac{3}{2}m^3+9m^2-\dfrac{121}{16}=0\\
&\Leftrightarrow & \hoac{&m=1\\&m=11\\&m=6-\sqrt{47}\\&m=6+\sqrt{47}.}
\end{eqnarray*}
Vì $m$ là số nguyên và $m>1$ nên $m=11\Rightarrow m^2-1=120$.
}
\end{ex}

\begin{ex}%[Mức độ 4]giảng 12, Phạm Tiến Long]%[2D1C4-3]
Cho hàm số $f(x)=\dfrac{x^2-4x+5}{x+2}$ có đồ thị $(C)$. Gọi $I$ là giao điểm của tiệm cận đứng và tiệm cận xiên của $(C)$. Đường thẳng $y=m$ (với $m\ne 0$) cắt tiệm cận đứng và tiệm cận xiên của $(C)$ tại hai điểm $A$, $B$ sao cho tam giác $IAB$ có diện tích bằng $32$. Tìm $m$.
\shortans{$-16$}
\loigiai{
Hàm số đã cho có tập xác định là $\mathbb{R}\backslash\{-2\}$.\\
Ta có $\lim\limits_{x\to -2^+}f(x)=+\infty$  và 	$\lim\limits_{x\to -2^-}f(x)=-\infty$.\\
Suy ra tiệm cận đứng của $(C)$ là đường thẳng $d\colon x=-2$.\\
Mặt khác,	ta có $\begin{aligned}[t]
a&=\lim\limits_{x \rightarrow+\infty} \dfrac{f(x)}{x}=\lim\limits_{x \rightarrow+\infty} \dfrac{x^2-4x+5}{x^2+2x}=1;\\
b&=\lim\limits_{x \rightarrow+\infty}[f(x)-x]=\lim\limits_{x \rightarrow+\infty}\left(\dfrac{x^2-4x+5}{x+2}-x\right)=\lim\limits_{x \rightarrow+\infty} \dfrac{-6x+5}{x+2}=-6.
\end{aligned}$\\
Ta cũng có $\lim\limits_{x \rightarrow-\infty} \dfrac{f(x)}{x}=1$; $\lim\limits_{x \rightarrow-\infty}[f(x)-x]=-6$.
\\
Do đó, đồ thị hàm số có tiệm cận xiên là đường thẳng $d'\colon y=x-6$.\\
$I$ là giao điểm của $d$ và $d' \Rightarrow I(-2;-8)$.\\
Đường thẳng $y=m$ cắt hai đường thẳng $d$ và $d'$ lần lượt tại hai điểm $A(-2;m)$ và $B(m+6;m)$.\\
Ta có $\heva{&IA=\sqrt{(m+8)^2}=|m+8|\\&AB=\sqrt{(m+8)^2}=|m+8|}\Rightarrow IA=AB$\quad(1)\\
Dễ thấy đường thẳng $y=m$ vuông góc với $d$ tại $A$.\quad(2)\\
Từ (1) và (2) suy ra tam giác $IAB$ vuông cân tại $A$.\\
Theo giả thiết ta có
\begin{eqnarray*}
& & S_{\triangle IAB}=32\\
&\Leftrightarrow & \dfrac{1}{2}\cdot AB^2=32\\
&\Leftrightarrow & (m+8)^2=64\\
&\Leftrightarrow & m^2+16m=0\\
&\Leftrightarrow & \hoac{&m=0\\&m=-16.}
\end{eqnarray*}
Vì $m\ne 0$  nên suy ra $m=-16$.
}
\end{ex}

\begin{ex}%[Dự án TL12New-4in1-NCT]%[2D1C4-2]
Cho hàm số $y=\dfrac{2x+1}{x-3}$ có đồ thị là $(C)$. Gọi $M$ là điểm bất kì trên đồ thị $(C)$, tìm giá trị nhỏ nhất của tổng khoảng cách từ $M$ đến hai tiệm cận của đồ thị (làm tròn đến $1$ chữ số thập phân).
\shortans{$5{,}3$}
\loigiai{Gọi $M\left(x_M;\dfrac{2x_M+1}{x_M-3}\right),x_M\ne 3$. Các đường tiệm cận ngang, tiệm cận đứng của đồ thị có phương trình lần lượt là $y=2,x=3$.\\
Tổng khoảng cách từ điểm $M$ đến hai tiệm cận là\newline $d=|x_M-3|+\left|\dfrac{2x_M+1}{x_M-3}-2\right|=|x_M-3|+\dfrac{7}{\left|x_M-3\right|}\ge 2\sqrt{7}$.\\
Đẳng thức xảy ra khi và chỉ khi $|x_M-3|=\dfrac{7}{\left|x_M-3\right|}\Leftrightarrow \left[\begin{aligned}&x_M=3+\sqrt{7}\\&x_M=3-\sqrt{7}\end{aligned}\right.$\\
Vậy giá trị giá trị nhỏ nhất của tổng khoảng cách từ $M$ đến hai tiệm cận của đồ thị là $2\sqrt 7\approx5{,}3$.}
\end{ex}

\begin{ex}%[Dự án TL12New-4in1-NCT]%[2D1C4-1]
\immini{
Cho hàm số bậc ba $f(x)$ có đồ thị như hình vẽ. Xác định tổng số các đường tiệm cận đứng và tiệm cận ngang của đồ thị hàm số $ g(x)=\dfrac{\left(x^2-2x-3\right)\sqrt{x+2}}{(x^2-x)\left[f^2(x)+f(x)\right]} $.
\shortans{$7$}
}{
\begin{tikzpicture}[scale=.7,>=stealth]
\draw[->](-2.5,0)--(4,0)node[below]{$x$};
\draw[->](0,-2.5)--(0,3)node[left]{$y$};
\draw[dashed](-1,0)--(-1,-1)circle(1.5 pt)--(0,-1);
\node at (-1,0) [above] {\footnotesize $-1$};
\node at (2,0) [above] {\footnotesize $2$};
\node at (0,-1) [ right] {\footnotesize $-1$};
\draw [fill] (0,0) circle (1.5 pt)node[above right] {\footnotesize $O$};
\draw[smooth,samples=100,domain=-2.1:3.5] plot(\x,{-20/81*((\x)+1.45)*((\x)-2)^2});
\end{tikzpicture}
}
\loigiai{
\begin{center}
\begin{tikzpicture}[scale=.7,>=stealth]
\draw[->](-2.5,0)--(4,0)node[below]{$x$};
\draw[->](0,-2.5)--(0,3)node[left]{$y$};
\draw[dashed](-1,0)--(-1,-1)circle(1.5 pt)--(0,-1);
\node at (-1,0) [above] {\footnotesize $-1$};
\node at (2,0) [above] {\footnotesize $2$};
\node at (0,-1) [above left] {\footnotesize \footnotesize $-1$};
\draw [fill] (0,0) circle (1.5 pt)node[above left] {\footnotesize $O$};
\draw [fill] (-1.45,0) circle (1.5 pt)node[above left] {\footnotesize $a$};
\draw [fill] (2.9,0) circle (1.5 pt)node[above] {\footnotesize $c$};
\draw [dashed] (-2,-1)--(3.5,-1) (0.55,0)--(0.55,-1)(2.9,0)--(2.9,-1);
\draw [fill] (0.55,0) circle (1.5 pt)node[above ] {\footnotesize $b$};
\draw[smooth,samples=100,domain=-2.1:3.5] plot(\x,{-20/81*((\x)+1.45)*((\x)-2)^2});
\end{tikzpicture}
\end{center}
Dựa vào đồ thị ta thấy $f^2(x)+f(x)=0\Leftrightarrow\hoac{&f(x)=0\\&f(x)=-1}\Leftrightarrow \hoac{&x=a \quad (-2<a<-1)\\&x=2\\&x=-1\\&x=b\quad  (0<b<1)\\&x=c \quad (c>2).}$\\
Do đó ta viết $f^2(x)+f(x)=k x(x-1)(x-a)(x-2)^2(x+1)(x-b)(x-c)$.\\
Xét hàm số $  g(x)=\dfrac{\left(x^2-2x-3\right)\sqrt{x+2}}{(x^2-x)\left[f^2(x)+f(x)\right]}=\dfrac{(x+1)(x-3)\sqrt{x+2}}{k x(x-1)(x-a)(x-2)^2(x+1)(x-b)(x-c)} $.\\
Tập xác định $ \mathscr{D}=[-2;+\infty)\backslash\{0;1;a;2;-1;b;c\} $.\\
Từ đó suy ra đồ thị hàm số $g(x)$ có $6$  tiệm cận đứng là $ x=0, x=1, x=a, x=2, x=b, x=c $ và $1$ tiệm cận ngang là $y=0$.
}
\end{ex}

\begin{ex}%[Dự án TL12New-4in1-NCT]%[2D1C4-1]
\immini{
Cho hàm số $y=ax^4+bx^2+c$ có đồ thị như hình vẽ. Đồ thị hàm số $y=\dfrac{(x^2-4)(x^2+2x)}{\left[f(x)\right]^2+2f(x)-3}$ có bao nhiêu đường tiệm cận đứng?\shortans{$4$}}	{\hspace{0.5cm}
\begin{tikzpicture}[scale=0.5,>=stealth]
\path
(1,1) coordinate (A);
\draw[->](-3.5,0)--(3.5,0)node[below]{$x$};
\draw[->](0,-4)--(0,2.5)node[left]{$y$};
\draw[smooth,samples=200,domain=-2.9:2.9]plot(\x,{1/4*(\x)^4-2*(\x)^2+1});
\draw[dashed](-2,0)--(-2,-3)-- (2,-3)--(2,0);
\draw(0,0)node[below left]{$O$} (-2,0)node[above]{$-2$} (2,0)node[above]{$2$} (0,1)node[ left]{$1$} (0,-3)node[below left]{$-3$};
\end{tikzpicture}}
\loigiai{
\begin{center}
\begin{tikzpicture}[scale=0.7,>=stealth]
\path
(1,1) coordinate (A);
\draw[->](-3.5,0)--(4,0)node[below]{$x$};
\draw[->](0,-4)--(0,2.5)node[left]{$y$};
\draw[smooth,samples=200,domain=-2.9:2.9]plot(\x,{1/4*(\x)^4-2*(\x)^2+1});
\draw[dashed] (-3.3,1)--(3.3,1);
\draw[dashed] (-2.85,1)--(-2.85,0)node[below ]{$m$};
\draw[dashed] (2.85,1)--(2.85,0)node[below ]{$n$};
\draw[dashed] (-3.3,-3)--(3.3,-3);
\draw[dashed](-2,0)--(-2,-3)-- (2,-3)--(2,0);
\draw(0,0)node[below left]{$O$} (-2,0)node[above]{$-2$} (2,0)node[above]{$2$} (0,1)node[above left]{$1$} (0,-3)node[below left]{$-3$};
\end{tikzpicture}
\end{center}
Ta có $\left[f(x)\right]^2+2f(x)-3=0\Leftrightarrow\hoac{&f(x)=1\\&f(x)=-3}\Leftrightarrow\hoac{&x=m \quad (m<-2)\\&x=0\\&x=n\quad (n>2)\\&x=2\\&x=-2.}$\\
Do đó ta viết $\left[f(x)\right]^2+2f(x)-3=kx^2(x+2)^2(x-2)^2(x-m)(x-n)$.\\
Xét hàm số $y=\dfrac{(x^2-4)(x^2+2x)}{\left[f(x)\right]^2+2f(x)-3}=\dfrac{x(x+2)^2(x-2)}{kx^2(x+2)^2(x-2)^2(x-m)(x-n)}$.\\
Hàm số có tập xác định $\mathscr{D}=\mathbb{R}\backslash\{m;-2;0;2;n\}$.\\
Từ đó suy ra đồ thị hàm số đã cho có bốn tiệm cận đứng là $x=0$, $x=2$, $x=m, x=n$.
}

\end{ex}

\begin{ex}%[Dự án TL12New-4in1-NCT]%[2D1C4-1]
\immini
{
Cho hàm số bậc ba $f(x)$  có đồ thị như hình vẽ. Hỏi đồ thị hàm số \break $ g(x)=\dfrac{\left(x^2+4x+3\right)\sqrt{x^2+x}}{x\left[f^2(x)-2f(x)\right]} $ có bao nhiêu đường tiệm cận đứng? 	\shortans{$4$}
}{
\begin{tikzpicture}[scale=.6,>=stealth]
\draw[->](-4.5,0)--(2,0)node[below]{$x$};
\draw[->](0,-2.5)--(0,4)node[left]{$y$};
\draw[dashed](-1,0)--(-1,2)circle(1.5 pt)--(0,2);
\node at (-3,0) [below] {\footnotesize $-3$};
\node at (-1,0) [below] {\footnotesize $-1$};
\node at (0,2) [right] {\footnotesize $2$};
\draw [fill] (0,0) circle (1.5 pt)node[below right] {\footnotesize $O$};
\draw[smooth,samples=100,domain=-4:-0.3] plot(\x,{-(\x)^3-6.5*(\x)^2-12*(\x)-4.5});
\end{tikzpicture}
}
\loigiai{
\begin{center}
\begin{tikzpicture}[scale=.8,>=stealth]
\draw[->](-4.5,0)--(2,0)node[below]{$x$};
\draw[->](0,-2.5)--(0,4)node[left]{$y$};
\draw[dashed](-1,0)--(-1,2)circle(1.5 pt)--(0,2);
\node at (-3,0) [below] {\footnotesize $-3$};
\node at (-1,0) [below] {\footnotesize $-1$};
\node at (0,2) [right] {\footnotesize $2$};
\draw [fill] (0,0) circle (1.5 pt)node[below right] {\footnotesize $O$};
\draw [dashed] (-4.5,2)--(1,2);
\draw [fill](-0.5,0)circle (1.5 pt)node[below right]{$x_3$};
\draw [dashed][fill](-3.8,2)--(-3.8,0)circle (1.5 pt)node[below ]{$x_1$};
\draw [dashed][fill](-1.7,2)--(-1.7,0)circle (1.5 pt)node[below ]{$x_2$};
\draw[smooth,samples=100,domain=-4:-0.3] plot(\x,{-(\x)^3-6.5*(\x)^2-12*(\x)-4.5});
\end{tikzpicture}
\end{center}
Dựa vào đồ thị ta thấy $ f(x)=0\Leftrightarrow \hoac{&x=-3\\&x=x_3\in (-1;0).} $\\
Do đó, ta viết $ f(x)=a(x+3)^2(x-x_3) $.\\
Đồng thời, $ f(x)=2\Leftrightarrow \hoac{&x=x_1\in (-\infty;-3)\\&x=x_2\in (-3;-1)\\&x=-1} $. Do đó, ta viết $ f(x)-2=a(x-x_1)(x-x_2)(x+1) $.\\
Xét hàm số $ g(x)=\dfrac{\left(x^2+4x+3\right)\sqrt{x^2+x}}{x\left[f^2(x)-2f(x)\right]}=\dfrac{(x+1)(x+3)\sqrt{x^2+x}}{a^2x(x+3)^2(x+1)(x-x_1)(x-x_2)(x-x_3)} $.\\
Tập xác định $ \mathscr{D}=(-\infty;x_1)\cup(x_1;-3)\cup(-3;x_2)\cup(x_2;-1)\cup(0;+\infty) $.\\
Từ đó suy ra đồ thị hàm số đã cho có bốn tiệm cận đứng là $ x=0,x=3,x=x_1,x=x_2$.
}
\end{ex}

\begin{ex}%[BG-12NEW-4in1, Nguyen Huynh]%[2D1C4-1]
Gọi $S$ là tập các giá trị nguyên của tham số $m$ sao cho đồ thị hàm số $y = \log (mx^{2} - 2(m+1)x + m+1)$ có hai tiệm cận đứng mà khoảng cách giữa chúng lớn hơn 1. Tích của các phần tử của $S$ bằng bao nhiêu?
\shortans{$-24$}
\loigiai{
Yêu cầu bài toán tương đương với\\ Phương trình $mx^{2} - 2(m+1)x + m+1=0$ có 2 nghiệm phân biệt $x_{1}, x_{2}$ sao cho $|x_{1}-x_{2}|>1$.\\
$\Leftrightarrow \heva{& m \neq 0 \\ & \Delta' > 0 \\ & (x_{1}+x_{2})^{2}-4x_{1}x_{2}>1} \heva{& m\neq 0 \\ & m > -1 \\ & -m^{2}+4m+4 > 0.}$\\
Do $m \in \mathbb{Z}$ nên $S= \{ -1; 1; 2; 3; 4 \}$. Suy ra tích cần tìm bằng $-24$.
}
\end{ex}

\begin{ex}%[BG-12NEW-4in1, Nguyen Huynh]%[2D1C4-1]
Đồ thị hàm số $y=\log\dfrac{x^2-4x+3}{x(x-2)}$ có tất cả bao nhiêu đường tiệm cận?
\shortans{$5$}
\loigiai{
Tập xác định của hàm số là $\mathscr{D}=(-\infty;0)\cup(1;2)\cup(3;+\infty)$.\\
Ta có $\lim\limits_{x\to \pm\infty}\log\dfrac{x^2-4x+3}{x(x-2)}=\log(1)=0$, suy ra đồ thị có tiệm cận ngang là $y=0$.\\
Ta có $\lim\limits_{x\to 0^-}\log f(x)=\lim\limits_{x\to +\infty}\log x=+\infty$, suy ra đồ thị có tiệm cận đứng là $x=0$.\\
Ta có $\lim\limits_{x\to 1^+}\log f(x)=\lim\limits_{x\to 0^+}\log x=-\infty$, suy ra đồ thị có tiệm cận đứng là $x=1$.\\
Ta có $\lim\limits_{x\to 2^-}\log f(x)=\lim\limits_{x\to +\infty}\log x=+\infty$, suy ra đồ thị có tiệm cận đứng là $x=2$.\\
Ta có $\lim\limits_{x\to 3^+}\log f(x)=\lim\limits_{x\to 0^+}\log x=-\infty$, suy ra đồ thị có tiệm cận đứng là $x=3$.
}
\end{ex}

\begin{ex}%[BG-12NEW-4in1, Nguyen Huynh]%[2D1C4-1]
Đồ thị hàm số $y=\log\dfrac{x-2}{x+1}$ có tất cả bao nhiêu đường tiệm cận?
\shortans{$3$}
\loigiai{
Tập xác định của hàm số $\mathscr{D}=(-\infty;-1)\cup(2;+\infty)$.\\
Mà $\lim\limits_{x\to \pm\infty}\log(\dfrac{x-2}{x+1})=\log 1=0$, suy ra $y=0$ là tiệm cận ngang.\\
Mà $\lim\limits_{x\to -1^-}\log\left( \dfrac{x-2}{x+1}\right) =\lim\limits_{x\to -1^-}\log(+\infty)=+\infty$, suy ra $x=-1$ là tiệm cận đứng.\\
Mà $\lim\limits_{x\to 2^+}\log\left( \dfrac{x-2}{x+1}\right) =\lim\limits_{x\to 0^+}\log x=-\infty$, suy ra $x=2$ là tiệm cận đứng.\\
Vậy đồ thị hàm số đã cho có $3$ tiệm cận.
}
\end{ex}

\begin{ex}%[BG-12NEW-4in1, Nguyen Huynh]%[2D1C4-1]
Có tất cả bao nhiêu điểm trên đồ thị hàm số $y=\dfrac{x+1}{x-2}$ sao cho tổng khoảng cách từ điểm đó đến hai đường tiệm cận là nhỏ nhất?
\shortans{$2$}
\loigiai{
Xét $M_0\left(x_0;\dfrac{x_0+1}{x_0-2}\right)$ thuộc đồ thị hàm số.\\
Hai đường tiệm cận của đồ thị hàm số là $x=2$ (tiệm cận đứng) và $y=1$ (tiệm cận ngang).\\
Tổng khoảng cách từ $M_0$ đến hai đường tiệm cận là $$\left|x_0-2\right|+\left|\dfrac{x_0+1}{x_0-2}-1\right|=\left|x_0-2\right|+\dfrac{3}{\left|x_0-2\right|}\geq 2\sqrt{3}.$$
Đẳng thức xảy ra khi và chỉ khi $$\left|x_0-2\right|=\dfrac{3}{\left|x_0-2\right|} \Leftrightarrow |x_0-2|=\sqrt{3} \Leftrightarrow \hoac{x_0=2+\sqrt{3}\\ x_0=2-\sqrt{3}} \Rightarrow \hoac{y_0=1+\sqrt{3}\\ y_0=1-\sqrt{3}.}$$
}
\end{ex}

\begin{ex}%[Mức độ C]%[2D1C3-6]
Ông $A$ muốn xây một cái bể chứa nước lớn dạng một khối hộp chữ nhật không nắp có thể tích bằng $288$cm$^2$. Đáy bể là hình chữ nhật có chiều dài gấp đôi chiều rộng. Hỏi tổng diện tích bể bằng bao nhiêu để chi phí thuê nhân công xây dựng là thấp nhất?

\shortans{$216$}
\loigiai{Theo bài toán, để chi phí thuê nhân công thấp nhất thì ta phải xây dựng bể sao cho tổng diện tích xung quanh và diện tích đáy là nhỏ nhất.\\
Gọi các kích thước của bể lần lượt là $a$(m), $2a$(m), $c$(m).\\
%	\begin{center}
%		\begin{tikzpicture}\def\a{3}\def\b{1}\def\g{30}\def\h{2}
%			\path
%			(0:0) coordinate (A)--++(\g:\b) coordinate (B)--++(0:\a) coordinate (C)--++(\g-180:\b) coordinate (D)
%			\foreach \x in {A,B,C,D}{
%				($(\x)+(90:\h)$) coordinate (\x
%				’)};
%			\draw[dashed] (B’)--(B)--(A)
%			(B)--(C);
%			\draw
%			(A)--(D)--(D’)--(A’)--cycle(A’)--(B’)--(C’)--(D’)(D)--(C)--(C’)			; \end{tikzpicture}
%	\end{center}
Ta có tổng diện tích các mặt cần xây là $S=2a^2+4ac+2ac=2a^2+6ac$.\\
Thể tích bể $V=a \cdot 2a\cdot c=2a^2\cdot c=288 \Rightarrow c=\dfrac{144}{a^2}$.\\
Suy ra $S=2a^2+6a\dfrac{144}{a^2}=2a^2+\dfrac{864}{a}=2a^2+\dfrac{432}{a}+\dfrac{432}{a} \geq 3.\sqrt{2a^2\cdot \dfrac{432}{a}\cdot \dfrac{432}{a}}=216.$\\
Do đó diện tích bể nhỏ nhất là $S=216$.\\
Vậy diện tích bể $S=216$m$^2$ thì chi phí thuê nhân công xây dựng là thấp nhất.}
\end{ex}

\begin{ex}%[Mức độ 4]%[BG12-4IN1, Nguyễn Khánh Trọng]%[2D1C3-6]
\immini[thm]{
Cho một tấm gỗ hình vuông cạnh $200$ cm. Người ta cắt một tấm gỗ có hình một tam giác vuông $ABC$ từ tấm gỗ hình vuông đã cho như hình vẽ bên. Biết $AB=x$ ($0<x<60$ cm) là một cạnh góc vuông của tam giác $ABC$ và tổng độ dài cạnh góc vuông $AB$ với cạnh huyền $BC$ bằng $120$ cm. Tìm $x$ để tam giác $ABC$ có diện tích lớn nhất.
\shortans{$40$}
}{
\begin{tikzpicture}[scale=0.72, font=\footnotesize, line join=round, line cap=round, >=stealth]
\draw[dashed] (0,0)--(4,0)--(0,1)--(0,0);
\draw (4,0)--(5,0)--(5,5)--(0,5)--(0,1);
\node at (0,0.5)[below left] {$x$}; \node at (2,0.5)[above,rotate=-13] {$120-x$}; \node at (2.5,5)[above] {$200$};
\fill (0,0) circle (1.5pt) node[below left]{$A$} (4,0) circle (1.5pt) node[below]{$C$} (0,1) circle (1.5pt) node[left]{$B$};
\end{tikzpicture}
}

\loigiai{
Độ dài cạnh huyền $BC$ là $120-x$.\\
Khi đó độ dài cạnh $AC=\sqrt{BC^2-AB^2}=\sqrt{(120-x)^2-x^2}=\sqrt{14400-240x}$.\\
Diện tích tam giác $ABC$ là $S=\dfrac{1}{2}AB\cdot AC=\dfrac{1}{2}x\sqrt{14400-240x}$.\\
Xét hàm số $f(x)=x\sqrt{14400-240x}$ với $0<x<60$.\\
Ta có $f'(x)=\sqrt{14400-240x}-\dfrac{120x}{\sqrt{14400-240x}}=\dfrac{14400-360x}{\sqrt{14400-240x}}$;\\
$f'(x)=0\Leftrightarrow x=40\in(0;60)$.\\
Bảng biến thiên
\begin{center}
\begin{tikzpicture}
\tkzTabInit[nocadre=false,lgt=1.2,espcl=2.5,deltacl=0.6]
{$x$ /0.6,$f'(x)$ /0.6,$f(x)$ /2}
{$0$,$40$,$60$}
\tkzTabLine{,+,$0$,-,}
\tkzTabVar{-/, +/,-/}
\end{tikzpicture}
\end{center}
Vậy tam giác $ABC$ có diện tích lớn nhất khi $AB=40$ cm.
}
\end{ex}

\begin{ex}%[SGK 12 - Cùng Khám Phá, Mức độ 4]%[BG12-4IN1, Nguyễn Khánh Trọng]%[2D1C3-6]
\immini{Một thùng chứa nhiên liệu gồm phần ở giữa là một hình trụ có chiều dài $h$ mét $(h>0)$ và hai đầu là các nửa hình cầu bán kính $r$ $(r>0)$ (\textit{Hình 1.11}). Biết rằng thể tích của thùng chứa là $144\,000 \pi$ m$^3$. Để sơn mặt ngoài của phần hình cầu cần $20\,000$ đồng cho $1$ m$^2$, còn sơn mặt ngoài cho phần hình trụ cần $10\,000$ đồng cho $1$ m$^2$. Xác định $r$ để chi phí cho việc sơn diện tích mặt ngoài thùng chứa (bao gồm diện tích xung quanh hình trụ và diện tích hai nửa hình cầu) là nhỏ nhất, biết rằng bán kính $r$ không được vượt quá $50$ m.
\shortans{$30$}
}{

\begin{tikzpicture}[scale=.7]
\draw[white,fill=blue!10] (0,0) rectangle (5,3);
\draw(0,0)--(5,0);
\draw(0,3)--(5,3);
\draw[red](0,3)--(0,4);
\draw[red](5,3)--(5,4);
\draw[fill=blue!10] (5,3) arc(90:-90:1.5);
\draw[fill=blue!10] (0,0) arc(-90:-270:1.5);
\draw[white] (0,3) arc (90:270:0.75 and 1.5);
\draw (0,0) arc (-90:90:0.75 and 1.5);
\draw[white] (5,3) arc (90:270:0.75 and 1.5);
\draw (5,0) arc (-90:90:0.75 and 1.5);
\draw[red,<->] (0,3.5)--(5,3.5) node[midway,above]{$h$};
\draw[->] (0,1.5)--(0,3) node[midway,left]{$r$};
\draw (2,0) node[below right]{\textit{Hình 1.11}};
\end{tikzpicture}
}
\loigiai{
Ta có thể tích của thùng chứa nhiên liệu là $V=\pi \cdot r^2 \cdot h + \dfrac{4}{3} \pi \cdot r^3 = 144\, 000 \pi$ \\
Suy ra $h=\dfrac{(144\,000-\dfrac{4}{3} \cdot r^3)}{r^2}$ \\
Khi đó chi phí sơn diện tích mặt ngoài thùng chứa là
$$2 \pi \cdot r \cdot \dfrac{(144\,000-\dfrac{4}{3} \cdot r^3)}{r^2} \cdot 10^4 + 4 \pi \cdot r^2 \cdot 2 \cdot 10^4 =2 \pi \cdot 10^4 \left( \dfrac{144\,000}{r} + \dfrac{8}{3} r^2 \right). $$
Xét hàm số $f(r)=\dfrac{144\,000}{r} + \dfrac{8}{3} r^2$ với $r \in (0;50]$.\\
Ta có $f'(r)=-\dfrac{144\,000}{r^2} + \dfrac{16}{3} r=\dfrac{16r^3-432\,000}{3r^2}$ và $f'(r)=0 \Leftrightarrow r=30$ m.\\
Bảng biến thiên\\
\begin{center}
\begin{tikzpicture}
\tkzTabInit[nocadre=false,lgt=1.2,espcl=3.5,deltacl=0.6]
{$r$/1,$f'(r)$/1,$f(r)$/3}
{$0$,	$30$,	$50$}
\tkzTabLine{,	-,	$0$,	+}
\tkzTabVar{+/$+\infty$,	-/$7200$,	+/$9546{,}7$}
\end{tikzpicture}
\end{center}
Vậy với $r=30$ m thì chi phí cho việc sơn diện tích mặt ngoài của thùng chứa là nhỏ nhất.
}
\end{ex}

\begin{ex}%[Mức độ 4]%[Dự án giảng 12 - Nguyễn Sĩ Đạt]%[2D1C3-4]
Gọi $S$ là tập hợp các giá trị nguyên của tham số $m\in \left[ 0;2024 \right]$ để bất phương trình ${{x}^{2}}-m+\sqrt{{{\left( 1-{{x}^{2}} \right)}^{3}}}\le 0$ nghiệm đúng với mọi $x\in \left[ -1;1 \right]$. Tập $S$ có bao nhiêu phần tử?
\shortans{$2024$}
\loigiai{
Đặt $t=\sqrt{1-{{x}^{2}}}$, với $x\in \left[ -1;1 \right]\Rightarrow t\in \left[ 0;1 \right]$.\\
Bất phương trình đã cho trở thành ${{t}^{3}}-{{t}^{2}}+1-m\le 0\Leftrightarrow m\ge {{t}^{3}}-{{t}^{2}}+1$.(1)\\
Yêu cầu của bài toán tương đương với bất phương trình (1) nghiệm đúng với mọi $t\in \left[ 0;1 \right]$.\\
Xét hàm số $f\left( t \right)={{t}^{3}}-{{t}^{2}}+1\Rightarrow {f}'\left( t \right)=3{{t}^{2}}-2t$.\\
${f}'\left( t \right)=0\Leftrightarrow \hoac{&t=0\notin \left( 0;1 \right)  \\&t=\frac{2}{3}\in \left( 0;1 \right) .}$\\
Vì $f\left( 0 \right)=f\left( 1 \right)=1$, $f\left( \dfrac{2}{3} \right)=\dfrac{23}{27}$ nên $\underset{\left[ 0;1 \right]}{\mathop{\max }}\,f\left( t \right)=1$.\\
Do đó bất phương trình (1) nghiệm đúng với mọi $t\in \left[ 0;1 \right]$ khi và chỉ khi $m\ge 1$.\\
Mặt khác $m$ là số nguyên thuộc $\left[ 0;2024 \right]$ nên $m\in \left\{ 1;2;3;\ldots;2024 \right\}$.\\
Vậy có $2024$ giá trị của $m$ thỏa mãn bài toán.
}
\end{ex}

\begin{ex}%[Mức độ 4]%[BG12-4IN1, Nguyễn Khánh Trọng]%[2D1C3-4]
Cho hàm số $y=f(x)$ có bảng biến thiên như sau
\begin{center}
\begin{tikzpicture}
\tkzTabInit[nocadre=false,lgt=1.2,espcl=2.5,deltacl=0.6]
{$x$ /0.6,$f'(x)$ /0.6,$f(x)$ /2}
{$0$,  $1$, $3$}
\tkzTabLine{,+,$0$,-}
\tkzTabVar{-/ $8$ ,+/$9$,-/$5$}
\end{tikzpicture}
\end{center}
Gọi $S$ là tập hợp các số nguyên dương $m$ để bất phương trình $f(x) \ge mx^2\left(x^2-2\right)+2m$ có nghiệm thuộc đoạn $[0;3]$. Tìm số phần tử của tập $S$.
\shortans{$9$}
\loigiai{
Bất phương trình đã cho tương đương với $\dfrac{f(x)}{x^4-2x^2+2} \ge m$.\\
Từ bảng biến thiên ta thấy $5 \le f(x) \le 9$ với mọi $x\in [0;3]$.\\
Xét hàm số $g(x)=x^4-2x^2+2$ với $x\in [0;3]$ ta có $g'(x)=4x^3-4x$, $g'(x)=0 \Leftrightarrow \hoac{&x=0\\&x=1.}$\\
Ta lại có $g(0)=2$, $g(1)=1$, $g(3)=65$. Từ đó suy ra $1 \le g(x) \le 65$ với mọi $x\in [0;3]$.\\
Xét hàm số $h(x)=\dfrac{f(x)}{g(x)}$, $x\in [0;3]$. Từ đó ta có đánh giá $\dfrac{5}{65} \le h(x) \le 9$ với mọi $x\in [0;3]$.\\
Từ đó suy ra $\min\limits_{x\in [0;3]} h(x)=\dfrac{5}{65}$ khi $x=3$; $\max\limits_{x\in [0;3]} h(x)=9$ khi $x=1$.\\
Vậy bất phương trình đã cho có nghiệm thuộc đoạn $[0;3]$ khi và chỉ khi $ m \le 9$.\\
Vì $m$ nguyên dương nên có tất cả $9$ giá trị thỏa đề bài.
}
\end{ex}

\begin{ex}%[Mức độ 4]%[BG12-4IN1, Nguyễn Khánh Trọng]%[2D1C3-4]
Cho hàm số $ y=f(x) $ liên tục trên $ \mathbb{R} $ và có bảng biến thiên sau
\begin{center}
\begin{tikzpicture}
\tkzTabInit[lgt=1.5,espcl=2]
{$x$/1,$f’(x)$/.7,$f(x)$/2}
{$-\infty$,$-1$,$1$,$\dfrac{21}{4}$,$7$,$10$,$+\infty$}
\tkzTabLine{ ,+,z,-,z,+,z,-,d,+,z,- }
\tkzTabVar{-/$-\infty$,+/$4$,-/$2$,+/$5$,-/$0$,+/$8$,-/$-\infty$}
\end{tikzpicture}
\end{center}
Gọi $S$ là tập hợp các số nguyên của tham số $m\in[-5;15]$ để bất phương trình $f(x^2-2x)-m\ge 0$ có nghiệm
trên khoảng $\left(-\dfrac{3}{2};\dfrac{7}{2}\right)$. Tìm số phần tử của tập $S$.
\shortans{$10$}
\loigiai{
Đặt $t=x^2-2x$. Với $x\in \left[-\dfrac{3}{2};\dfrac{7}{2}\right] \Leftrightarrow -1 \le (x-1)^2-1 \le \dfrac{21}{4}$ nên $t\in \left[-1;\dfrac{21}{4}\right]$.\\
Xét hàm số $ y=f(t)$, với $t \in \left[-1;\dfrac{21}{4}\right] $.\\
Từ BBT, ta có $\max \limits_{t \in \left[-1;\tfrac{21}{4}\right]}f(t)= f\left(\dfrac{21}{4}\right)=5$.\\
Bất phương trình $f(x^2-2x)-m\ge 0$ có nghiệm
trên khoảng $\left(-\dfrac{3}{2};\dfrac{7}{2}\right)$ khi và chỉ khi
$$\max \limits_{t \in \left[-1;\tfrac{21}{4}\right]}f(t)>m\Leftrightarrow m<5.$$
Vì $m$ nguyên thuộc đoạn $[-5;15]$ nên $m\in\left\{-5;-4;-3;\ldots;3;4\right\}$, suy ra ta có $10$ giá trị thỏa đề bài.
}
\end{ex}

\begin{ex}giảng 12-4in1, Nhật Thiện]%[2D1C3-1]
Giá trị lớn nhất của hàm số $y=\dfrac{x^3+x^2-m}{x+1}$ trên $[0;2]$ bằng $5$. Tham số $m$ nhận giá trị là
\shortans{$-3$}
\loigiai{
Đặt $f(x)=\dfrac{x^3+x^2-m}{x+1}$.\\
Giá trị lớn nhất của $y=f(x)$ trên $[0; 2]$ bằng $5\Leftrightarrow \heva{& f(x)\leq 5,  \forall x\in [0;  2] \\ & \exists x_0\in [0;2]\colon f(x_0)=5.}$\\
\begin{itemize}
\item $f(x)\leq 5$, $\forall x\in [0;2]\Leftrightarrow \dfrac{x^3+x^2-m}{x+1}\leq 5$, $\forall x\in [0;2]$\\
\phantom{$f(x)\leq 5$, $\forall x\in [0;2]$} $\Leftrightarrow m\geq x^3+x^2-5x-5$, $\forall x\in [0;2]$\\
\phantom{$f(x)\leq 5$, $\forall x\in [0;2]$} $\Leftrightarrow m\geq \max\limits_{[0;2]} h(x)$, với $h(x)=x^3+x^2-5x-5$.\\
Ta có $h'(x)=3x^2+2x-5$, $h'(x)=0\Leftrightarrow 3x^2+2x-5=0\Leftrightarrow \hoac{& x=1 \\ & x=-\dfrac{5}{3}\;\text{(loại)}.}$\\
Ta có $h(0)=-5$, $h(2)=-3$, $h(1)=-8$.\\
Suy ra $\max\limits_{[0;2]} h(x)=-3$, $\min\limits_{[0;2]} h(x)=-8$.\\
Vậy $m\geq-3$. \hfill $(1)$
\item $\exists x_0\in [0;2]\colon f(x_0)=5\Leftrightarrow \dfrac{x^3+x^2-m}{x+1}=5$ có nghiệm trên $[0;2]$.\\
\phantom{$\exists x_0\in [0;2]\colon f(x_0)=5$} $\Leftrightarrow m=x^3+x^2-5x-5$ có nghiệm trên $[0;2]$.\\
Theo phần trên, ta suy ra $-8\leq m\leq-3$. \hfill $(2)$
\end{itemize}
Từ $(1)$ và $(2)$ suy ra $m=-3$.
}
\end{ex}

\begin{ex}giảng 12-4in1, Nhật Thiện]%[2D1C2-7]
Gia đình An xây bể hình trụ có thể tích $150$\text{m}$^3$. Đáy bể làm bằng bê tông giá $100000$\text{đ/m}$^2$. Phần thân làm bằng vật liệu chống thấm giá $90000$\text{đ/m}$^2$, nắp bằng nhôm giá $120000$\text{đ/m}$^2$. Hỏi tỷ số giữa chiều cao bể và bán kính đáy là bao nhiêu để chi phí sản xuất bể đạt cực đại? (làm tròn đến hai chữ số thập phân)
\shortans{$2{,}44$}
\loigiai{
Ta có $\pi r^2\cdot h=150 \Rightarrow h=\dfrac{150}{\pi r^2}$.\\
$S_{xq}+S_{đáy}+S_{nắp}=2\pi r\cdot h+\pi r^2+\pi r^2=\dfrac{300}{r}+2\pi r^2$.\\
Chi phí sản xuất bể là $S=\dfrac{300}{r}\cdot 90000+\pi r^2\cdot 220000$.\\
Ta có $S'=-\dfrac{27000000}{r^2}+440000\pi\cdot r$; $S'=0 \Leftrightarrow r=\sqrt[3]{\dfrac{675}{11\pi}}$.\\
Bảng biến thiên
\begin{center}
\begin{tikzpicture}
\tkzTabInit[nocadre=false,lgt=1.2,espcl=2.5,deltacl=0.6]
{$r$ /0.96,$S'$ /0.6,$S$ /3}
{$0$,$\sqrt[3]{\dfrac{675}{11\pi}}$,$+\infty$}
\tkzTabLine{,-,$0$,+,}
\tkzTabVar{+/$+\infty$, -/$S\left(\sqrt[3]{\dfrac{675}{11\pi}}\right)$,+/$+\infty$}
\end{tikzpicture}
\end{center}
Suy ra chi phí sản xuất bể đạt cực trị khi $r=\sqrt[3]{\dfrac{675}{11\pi}}\approx 2{,}44$.
}
\end{ex}

\begin{ex}%[MĐ4]%[2D1C2-6]
Cho hàm số $f(x)=\dfrac{x^2-m(m+1)x+m^3+1}{x-m}$ có đồ thị là $(C_m)$. Điểm $A(a;b)$ vừa là điểm cực đại của $(C_{m_1})$ vừa là điểm cực tiểu của $(C_{m_2})$. Tính $a-b$. \shortans{$1{,}25$}
\loigiai{
Ta có $f(x)=\dfrac{x^2-2mx+m^2-1}{(x-m)^2}$. Suy ra $f'(x)= 0 \Leftrightarrow x=m\pm 1$. Do đó, ta có bảng biến thiên
\begin{center}
\begin{tikzpicture}[font=\footnotesize,>=stealth, scale=1]
\tkzTabInit[nocadre=false,lgt=1.2,espcl=3.5,deltacl=0.6]
{$x$ /0.6,$f'(x)$ /0.6,$f$ /2}
{$-\infty$, $m-1$, $m$, $m+1$, $+\infty$}
\tkzTabLine{,+,0,-,d,-,0,+,}
\tkzTabVar{-/$-\infty$, +/$-m^2+m-2$, -D+/$-\infty$/$+\infty$, -/$-m^2+m+2$,+/$+\infty$}
\end{tikzpicture}
\end{center}
Từ giả thiết, ta có
$$ \heva{&m_1-1=m_2+1\\ &-m_1^2+m_1-2=-m_2^2+m_2+2} \Leftrightarrow \heva{&m_1=m_2+2\\ &4m_2=-6} \Leftrightarrow \heva{&m_1=\frac{1}{2}\\ &m_2=-\frac{3}{2}.} $$
Khi đó $A\left(-\dfrac{1}{2};-\dfrac{7}{4}\right)$, vậy $a-b=-\dfrac{1}{2}+\dfrac{7}{4}=1{,}25$.
}
\end{ex}

\begin{ex}%[MĐ4]%[2D1C2-6]
Cho hàm số $f(x)=\dfrac{x^2+m\left(m^2-1\right)x-m^4+1}{x-m}$, với $m$ là tham số, có đồ thị $(C_m)$. Biết rằng tồn tại duy nhất một điểm vừa điểm cực đại của $(C_{m_1})$ và là cực tiểu của $(C_{m_2})$, tính giá trị của $m_1m_2$.
\shortans{$-1$}
\loigiai{
Ta có $f(x)=x+m^3+\dfrac{1}{x-m}$, $f'(x)=1-\dfrac{1}{(x-m)^2}$. Suy ra $f'(x)=0 \Leftrightarrow x=m\pm 1$. Từ đó, ta có bảng biến thiên
\begin{center}
\begin{tikzpicture}[font=\footnotesize,>=stealth, scale=1]
\tkzTabInit[nocadre=false,lgt=1.2,espcl=2.5,deltacl=0.6]
{$x$ /0.6,$f'(x)$ /0.6,$f(x)$ /2}
{$-\infty$, $m-1$, $m$, $m+1$, $+\infty$}
\tkzTabLine{,+,0,-,d,-,0,+,}
\tkzTabVar{-/$-\infty$, +/$y_1$, -D+/$-\infty$/$+\infty$, -/$y_2$, +/$+\infty$}
\end{tikzpicture}
\end{center}
Ta có đường thẳng đi qua hai điểm cực trị có phương trình
$$ y=\dfrac{\left(x^2+m\left(m^2-1\right)x-m^4+1\right)'}{(x-m)'} \text{ hay } y=2x+m\left(m^2-1\right). $$
Suy ra $y_1=m^3+m-2$ và $y_2=m^3+m+2$. Theo giả thiết, tồn tại một điểm vừa là điểm cực đại của $(C_{m_1})$ và vừa là điểm cực tiểu của $(C_{m_2})$ nên
$$ \heva{&m_1-1=m_2+1\\ &m_1^3+m_1-2=m_2^3+m_2+2} \Leftrightarrow \heva{&m_1=m_2+2\\ &m_2^2+2m_2+1=0} \Leftrightarrow \heva{&m_1=1\\ &m_2=-1.} $$
Vậy $m_1m_2=-1$.
}
\end{ex}

\begin{ex}%[Mức độ C]%[Dự án giảng 12 - Trung Anh]%[2D1C2-5]
Với giá trị nào của tham số $m$ thì hàm số $y=x^4+2mx^2+m^2+m$ có ba điểm cực trị lập thành một tam giác có một góc bằng $120^\circ$? (lấy giá trị xấp xỉ đến hàng phần trăm)
\shortans{$-0{,}69$}
\loigiai
{
Tập xác định của hàm số là $\mathscr{D}=\mathbb{R}$.\\
Ta có $y'=4x^3+4mx = 4x(x^2+m)$.
$$y'=0 \Leftrightarrow 4x(x^2+m)=0 \Leftrightarrow \left[\begin{aligned}&x=0 \\&x^2=-m.\end{aligned}\right.$$
Đồ thị hàm số đã cho có ba điểm cực trị khi phương trình $x^2=-m$ có hai nghiệm phân biệt $x\neq 0$, suy ra $-m >0$ hay $m < 0$.\\
Như vậy, với $m<0$ đồ thị hàm số đã cho có điểm cực trị là $A(0;m^2+m)$, $B\left(\sqrt{-m};m\right)$, $C\left(-\sqrt{-m};m\right)$.\\
Dễ thấy tam giác $ABC$ cân tại $A$. Khi đó $\overrightarrow{AB}=\left(\sqrt{-m};-m^2\right)$, $\overrightarrow{AC} = \left(-\sqrt{-m}; -m^2\right)$.\\
Tam giác $ABC$ có một góc bằng $120^\circ$ nên $\widehat{A} = \left(\overrightarrow{AB},\overrightarrow{AC}\right) = 120^\circ$.\\
Suy ra
\begin{eqnarray*}
\cos\left(\overrightarrow{AB},\overrightarrow{AC}\right) = -\dfrac{1}{2} \Leftrightarrow \dfrac{m^4+m}{m^4-m} = -\dfrac{1}{2} \Leftrightarrow 3m^4+m=0 \Leftrightarrow m(3m^3+1)=0 \Leftrightarrow \left[\begin{aligned}&m=0 \\&m=-\dfrac{1}{\sqrt[3]{3}}.\end{aligned}\right.
\end{eqnarray*}
Kết hợp điều kiện $m<0$ ta được $m=-\dfrac{1}{\sqrt[3]{3}}\approx -0{,}69$ là giá trị thỏa mãn yêu cầu bài toán.
}
\end{ex}

\begin{ex}%[Sách tham khảo, Mức độ C]%[Dự án giảng 12 - Trung Anh]%[2D1C2-5]
Cho hàm số $y=x^4-2m^2x^2+m^2$ có đồ thị $(C)$. Tích các giá trị của $m$ để đồ thị $(C)$ có ba điểm cực trị $A$, $B$, $C$ sao cho bốn điểm $A$, $B$, $C$, $O$ là bốn đỉnh của hình thoi ($O$ là gốc tọa độ).
\shortans{$-0{,}5$}
\loigiai{
Ta có $y'=4x^3-4m^2x$; $y'=0\Leftrightarrow \left[ \begin{aligned}
x=0 \\
x=m^2 \\
\end{aligned} \right. $.\\
Điều kiện để hàm số có ba cực trị là $y'=0$ có ba nghiệm phân biệt $\Leftrightarrow m\ne 0$.\\
Khi đó: $y'=0\Leftrightarrow \left[ \begin{aligned}
x=0 \\
x=\pm m \\
\end{aligned} \right. $.\\
Tọa độ các điểm cực trị là $A(0;m^2)$, $B(m;-m^4+m^2)$, $C(m;-m^4+m^2)$.\\
Ta có $OA\bot BC$, nên bốn điểm $A$, $B$, $C$, $O$ là bốn đỉnh của hình thoi điều kiện cần và đủ là $OA$ và $BC$ cắt nhau tại trung điểm mỗi đoạn\\
$\Leftrightarrow \left\{ \begin{aligned}
{{x}_A}+{{x}_O}={{x}_B}+{{x}_C} \\
{{y}_A}+{{y}_O}={{y}_B}+{{y}_C} \\
\end{aligned} \right. \Leftrightarrow \left\{ \begin{aligned}
0=0 \\
m^2+0=(-m^4+m^2)+(-m^4+m^2) \\
\end{aligned} \right. $\\
$\Leftrightarrow 2m^4-m^2=0 \Leftrightarrow m^2=\dfrac{1}{2} \Leftrightarrow m=\pm \dfrac{\sqrt{2}}{2}$.\\
Vậy $m=\pm \dfrac{\sqrt{2}}{2}$.
}
\end{ex}

\begin{ex}%[Mức độ C]%[Dự án giảng 12 - Trung Anh]%[2D1C2-4]
Hàm số $f(x)=\dfrac{1}{3}x^3-x^2+(m^2-3)x+2018$ có hai điểm cực trị $x_1, x_2$. Tìm giá trị lớn nhất của biểu thức $P=|x_1(x_2-2)-2(x_2+1)|$.
\shortans{$9$}
\loigiai{
Tập xác định $\mathbb{R}$. Đạo hàm $y'=x^2-2x+m^2-3$.\\
Hàm số có hai điểm cực trị $\Leftrightarrow $ phương trình $y'=0$ có hai nghiệm phân biệt $\Leftrightarrow \Delta '>0 \Leftrightarrow 4-m^2>0\Leftrightarrow m \in (-2;2)$.\\
Áp dụng định lí Vi-et ta có $x_1+x_2=2;x_1x_2=m^2-3$.\\
Ta có $P=|x_1x_2-2(x_1+x_2)-2|=|m^2-9|$.\\
Xét hàm số $f(m)=m^2-9, m \in (-2;2)$. Ta có $f'(m)=2m=0\Leftrightarrow m=0$.\\
Bảng biến thiên:
\begin{center}
\begin{tikzpicture}
\tkzTabInit[nocadre=false,lgt=1.5,espcl=2.5,deltacl=0.6]
{$x$ /0.6,$f'(m)$ /0.6,$f(m)$ /2}
{$-2$,$0$,$2$}
\tkzTabLine{,-,0,+,}
\tkzTabVar{+/ $-5$ / , -/ $-9$ /, +/ $-5$ /}
\end{tikzpicture}
\end{center}
Vậy $P_{\max}=9$ đạt tại $m=0$.
}
\end{ex}

\begin{ex}%[Sách tham khảo, Mức độ C]%[Dự án giảng 12 - Trung Anh]%[2D1C2-4]
Gọi $S$ là tập hợp giá trị $m$ là số nguyên để hàm số $y = \dfrac{1}{3}x^3 - \left(m + 1\right)x^2 + \left(m - 2\right)x + 2m - 3$ đạt cực trị tại hai điểm $x_{1}$, $x_{2}$ thỏa mãn $x^2_{1} + x^2_{2} = 18$. Tính tổng các phần tử nguyên thuộc tập $S$.
\shortans{$1$}
\loigiai{Tập xác định $\mathscr{D} = \mathbb{R}$.\\
Ta có $y' = x^2 - 2\left(m + 1\right)x + m - 2$.\\
Xét $y' = 0$ suy ra $x^2 - 2\left(m + 1\right)x + m - 2 = 0\quad (*)$.\\
Để hàm số có hai điểm cực trị khi và chỉ khi phương trình $(*)$ có hai nghiệm phân biệt.
$$\Leftrightarrow \Delta' > 0\Leftrightarrow \left(m + 1\right)^2 - \left(m - 2\right) > 0\Leftrightarrow m^2 + m + 3 > 0\Leftrightarrow \left(m + \dfrac{1}{2}\right)^2 + \dfrac{11}{4} > 0.$$
Dễ thấy $(*)$ luôn có $2$ nghiệm phân biệt với mọi $m$.\\
Khi đó $x_{1}$, $x_{2}$ là nghiệm của phương trình $(*)$.\\
Theo định lý Vi-ét ta có $\heva{&x_{1} + x_{2}  = 2m  +2\\ &x_{1}\cdot x_{2} = m - 2}$\\
Để thỏa mãn bài toán $x_{1}^2 + x_{2}^2  = 18\Leftrightarrow \left(x_{1} + x_{2}\right)^2 - 2x_{1}x_{2} - 18 = 0\quad (**)$.\\
Áp dụng định lý Vi-ét  $(**)$ trở thành
\begin{eqnarray*}
&{ }&\left(2m + 2\right)^2- 2\left(m - 2\right) - 18 = 0\\
&\Leftrightarrow& 4m^2 + 6m - 10 = 0\Leftrightarrow\hoac{& m = 1 \\ &m = - \dfrac{5}{2}.}
\end{eqnarray*}
Do giả thiết suy ra $m = 1$ nên $P = 1$.
}
\end{ex}

\begin{ex}%[BG12, Tran Tony]%[2D1C2-3]
Có bao nhiêu giá trị nguyên của tham số $m$ để hàm số $y=x^6+(m+4)x^5+(16-m^2)x^4+2$ đạt cực tiểu tại $x=0$?
\shortans{$8$}
\loigiai{
Tập xác định $\mathscr D=\mathbb{R}$.\\
Ta có $y'=x^3\cdot\left[6x^2+5(m+4)x+64-4m^2\right]$; $y'=0 \Leftrightarrow\hoac{& x=0\,(\text{bội } 3)\\& 6x^2+5(m+4)x+64-4m^2=0. \quad (*)}$\\
Để hàm số đã cho đạt cực tiểu tại $x=0$ thì $\hoac{& (*) \text{ vô nghiệm}\\& (*) \text{ có nghiệm kép } x=0\\& (*) \text{ có hai nghiệm phân biệt cùng dấu (do } a=6>0).}$
\begin{itemize}
\item \textbf{Trường hợp 1.} $(*)$ vô nghiệm $\Leftrightarrow \Delta<0 \Leftrightarrow 25\cdot (m+4)^2-24\cdot (64-4m^2)<0$.\\
$$\Leftrightarrow 121m^2+200m-1136<0 \Leftrightarrow -4<m<\dfrac{284}{121}.$$
\item \textbf{Trường hợp 2.} $(*)$ có nghiệm kép $x=0 \Leftrightarrow \heva{& \Delta=0\\& x=0.}$\\
$$\Leftrightarrow \heva{& 121m^2+200m-1136=0\\& 64-4m^2=0} \Leftrightarrow \heva{& m=-4 \vee m=\dfrac{284}{121}\\& m=\pm 4} \Leftrightarrow m=-4.$$
\item \textbf{Trường hợp 3.} $(*)$ có hai nghiệm phân biệt khác cùng dấu $\Leftrightarrow \heva{& \Delta>0\\& 6\cdot (64-4m^2)>0.}$\\
$$\Leftrightarrow \heva{& 121m^2+200m-1136>0\\& 64-4m^2>0} \Leftrightarrow \heva{& m\in (-\infty; -4) \cup \left(\dfrac{284}{121}; +\infty\right)\\& m\in (-4;4)} \Leftrightarrow m\in \left(\dfrac{284}{121}; 4\right).$$
\end{itemize}
Do đó $m\in \left[-4; \dfrac{284}{121}\right)\cup \left(\dfrac{284}{121};4\right)$.\\
Lại có $m\in\mathbb{Z}$ nên $m\in \{-4;-3;-2;-1;0;1;2;3\}$.\\
Vậy có $8$ giá trị nguyên của $m$ thỏa mãn yêu cầu bài toán.
}
\end{ex}

\begin{ex}%[BG12, Tran Tony]%[2D1C2-3]
Cho hàm số $y=x^2-2mx-2\ln \left(x^2-2mx+m^2+1\right)$, với $m$ là tham số. Gọi $S$ là tập hợp các giá trị của $m$ để hàm số đã cho đạt cực tiểu tại điểm $x=2$. Tính tổng các phần tử của $S$.
\shortans{$4$}
\loigiai{
Hàm số xác định khi $x^2-2mx+m^2+1>0\Leftrightarrow (x-m)^2+1>0,\,\forall x\in\mathbb{R}$.\\
Do đó hàm số có tập xác định $\mathscr D=\mathbb{R}$.\\
Ta có
\allowdisplaybreaks
\begin{eqnarray*}
y'&=&2x-2m-\dfrac{2(2x-2m)}{x^2-2mx+m^2+1}\\
&=&2(x-m)\left[1-\dfrac{2}{x^2-2mx+m^2+1}\right]\\
&=&\dfrac{2(x-m)(x^2-2mx+m^2-1)}{x^2-2mx+m^2+1}\\
&=&\dfrac{2(x-m)(x-m-1)(x-m+1)}{x^2-2mx+m^2+1}.
\end{eqnarray*}
Bảng xét dấu $y'$
\begin{center}
\begin{tikzpicture}
\tkzTabInit[lgt=1.2,espcl=3]
{$x$ /0.6, $y'$ /0.6}
{$-\infty$,$m-1$,$m$,$m+1$,$+\infty$}
\tkzTabLine{ ,-,$0$,+,$0$,-,$0$, +, }
\end{tikzpicture}
\end{center}
Từ bảng xét dấu $y'$, suy ra hàm số đạt cực tiểu tại các điểm $x=m-1$ và $x=m+1$.\\
Do đó, để hàm số đạt cực tiểu tại điểm $x=2$ thì $\hoac{&m-1=2\\&m+1=2}\Leftrightarrow\hoac{&m=3\\&m=1.}$\\
Suy ra $S=\{1;3\}$.\\
Vậy tổng các phần tử của $S$ là $4$.
}
\end{ex}

\begin{ex}%[BG12, Tran Tony]%[2D1C2-2]
\immini{Cho hàm số $f(x)$ có đạo hàm liên tục trên $\mathbb{R}$. Đồ thị của hàm số $y=f(5-2x)$ như hình vẽ bên. Có bao nhiêu giá trị thực của tham số $m$ thuộc khoảng $(-9;9)$ thoả mãn $2m\in \mathbb{Z}$ và hàm số $y=\left|2f\left(4x^3+1\right)+m-\dfrac{1}{2}\right|$ có $5$ điểm cực trị?
\shortans{$26$}
}
{
\begin{tikzpicture}[>=stealth,line cap=round,line join=round,scale=0.5,font=\footnotesize]
\draw[->] (-1.5,0) -- (5,0) node[below] {\scriptsize $x$};
\draw[->] (0,-4.5) -- (0,4) node[left] {\scriptsize $y$};
\draw (0,0)node[below left]{\scriptsize $O$};
\clip (-1.5,-4.5) rectangle(5,4);
\draw[samples=150,smooth,domain=-1.5:3] plot(\x,{-9/16*(\x)^2*(\x-3)});
\draw[samples=150,smooth,domain=3:5] plot(\x,{2.37*(\x)^3-22.21*(\x)^2+63.88*(\x)-55.67});
\draw[dashed] (2,0)|-(0,2.25) (4,0)|-(0,-3.8);
\draw[fill] (0,-3.8) circle(1pt) node[left]{$-4$} (2,0) circle(1pt) node[below]{$2$} (4,0) circle(1pt) node[above]{$4$} (0,2.25) circle(1pt) node[left]{$\dfrac{9}{4}$} (2,2.25) circle(1pt) (4,-3.8) circle(1pt) (0,0) circle(1pt);
\end{tikzpicture}
}
\loigiai{
Dựa vào đồ thị, ta thấy hàm số 	$y=f(5-2x)$ có ba điểm cực trị là $0$, $2$, $4$.\\
Suy ra phương trình $y'=-2f'(5-2x)=0$ có ba nghiệm phân biệt $0$, $2$, $4$ $\Rightarrow \hoac{& 5-2x=0\\& 5-2x=2\\ & 5-2x=4}\Leftrightarrow \hoac{& x=\dfrac{5}{2}\\ & x=\dfrac{3}{2}\\ & x=\dfrac{1}{2}.}$\\
Suy ra hàm số $f(x)$ có ba điểm cực trị là $\dfrac{1}{2}$, $\dfrac{3}{2}$, $\dfrac{5}{2}$.\\
Ta có bảng biến thiên của hàm số $y=f(x)$ như sau
\begin{center}
\begin{tikzpicture}
\tkzTabInit[espcl=2.5,lgt=1.5]
{$x$/1,$f'(x)$/0.7,$f(x)$/1.8}
{$-\infty$,$\dfrac{1}{2}$, $\dfrac{3}{2}$,$\dfrac{5}{2}$,$+\infty$}
\tkzTabLine{,-,0,,+,0,-,0,+}
\tkzTabVar{+/$+\infty$,-/$-4$,+/$\dfrac{9}{4}$,-/$0$,+/$+\infty$}
\end{tikzpicture}
\end{center}
Đặt $t=4x^3+1$, dễ thấy hàm số $u$ đồng biến trên $\mathbb{R}$ và ứng với mỗi giá trị $t$, ta tìm được duy nhất giá trị $x$.\\
Ta cũng có bảng biến thiên của hàm số $y=f\left(4x^3+1\right)$ như sau
\begin{center}
\begin{tikzpicture}
\tkzTabInit[espcl=2.5,lgt=2.5]
{$x$/0.7,$f\left(4x^3+1\right)$/2}
{$-\infty$,$a$, $b$, $c$,$+\infty$}
\tkzTabVar{+/$+\infty$,-/$-4$,+/$\dfrac{9}{4}$,-/$0$,+/$+\infty$}
\end{tikzpicture}
\end{center}
Từ đó suy ra hàm số $f\left(4x^3+1\right)$ có ba điểm cực trị.\\
Suy ra hàm số $y=2f\left(4x^3+1\right)+m-\dfrac{1}{2}$ có ba điểm cực trị.\\
Do đó, hàm số $y=\left|2f\left(4x^3+1\right)+m-\dfrac{1}{2}\right|$ có $5$ điểm cực trị khi và chỉ khi phương trình $$2f\left(4x^3+1\right)+m-\dfrac{1}{2}=0$$ có hai nghiệm bội lẻ phân biệt.\\
Xét phương trình $2f\left(4x^3+1\right)+m-\dfrac{1}{2}=0\Leftrightarrow -\dfrac{m}{2}+\dfrac{1}{4}=f\left(4x^3+1\right)$.\quad $(1)$\\
Dựa vào bảng biến thiên của $f\left(4x^3+1\right)$, phương trình $(1)$ có hai nghiệm bội lẻ khi và chỉ khi
$$\hoac{&-\dfrac{m}{2}+\dfrac{1}{4}\ge \dfrac{9}{4}\\ & -4<-\dfrac{m}{2}+\dfrac{1}{4}\le 0}\Leftrightarrow \hoac{& m\le -4\\ & \dfrac{1}{2}\le m<\dfrac{17}{2}}\Leftrightarrow \hoac{& 2m\le -8\\ & 1\le 2m<17.}$$
Do $m\in (-9;9)$ nên $2m\in (-18;18)$ và $2m\in \mathbb{Z}$ nên $2m\in \{-17;-16;\ldots;-8;1;2;\ldots;16\}$.\\
Vậy có $26$ giá trị $m$ cần tìm.
}
\end{ex}

\begin{ex}%[BG12, Tran Tony]%[2D1C2-2]
\immini{
Cho hàm số bậc bốn $y=f(x)$ có đồ thị hàm số $y=f'(x)$ như hình vẽ. Gọi $m$, $n$ lần lượt là số điểm cực đại và số điểm cực tiểu của hàm số $$h(x)=2f\left(\left|3-x\right|\right)+1.$$ Tính $T=2m+3n$
\shortans{$13$}
}
{
\begin{tikzpicture}[line join = round, line cap = round,>=stealth,font=\footnotesize,scale=0.8]
\def \xmin{-1.5};
\def \xmax{3};
\def \ymin{-1.3};
\def \ymax{2.3};
\draw[->] (\xmin,0) -- (\xmax,0) node[below] {$x$};
\draw[->] (0,\ymin) -- (0,0) node[below left] {$O$} -- (0,\ymax) node[left] {$y$};
\clip (\xmin+0.1,\ymin+0.1) rectangle (\xmax+0.1,\ymax-0.1);
\draw[smooth, samples=100] plot[domain=-1.5:1.45] (\x,{2*(\x+1)*(\x-0.5)*(\x-1)}) node[below right] {$y=f'(x)$};
\end{tikzpicture}
}
\loigiai{
Dựa vào đồ thị ta thấy $f'(x)=0\Leftrightarrow \hoac{& x=a<0 \\ & x=b>0\\& x=c>b.}$\\
Ta có $h'(x)=2\dfrac{(x-3)}{|x-3|}f'\left(|3-x|\right)$, $h'(x)=0\Leftrightarrow \hoac{& |3-x|=b \\ & |3-x|=c}\Leftrightarrow \hoac{& x=3+b \\ & x=3-b\\& x=3+c\\& x=3-c.}$\\
Bảng xét dấu đạo hàm
\begin{center}
\begin{tikzpicture}[line join = round, line cap = round,>=stealth,font=\footnotesize,scale=1]
\tkzTabInit[nocadre=false,lgt=1.2,espcl=2.5,deltacl=0.6]
{$x$ /0.6, $h'(x)$ /0.6}
{$-\infty$,$3-c$,$3-b$,$3$,$3+b$,$3+c$,$+\infty$}
\tkzTabLine{ ,-,z,+,z,-,d,+,z,-,z,+, }
\end{tikzpicture}
\end{center}
Dựa vào bảng xét dấu của $h'(x)$ ta thấy hàm số $y=h(x)$ có $2$ điểm cực đại và $3$ điểm cực tiểu.\\
Vậy $T=2m+3n=13$.
}
\end{ex}

\begin{ex}%[Mức độ 4]%[2D1C2-1]
Giả sử $A$, $B$ là hai điểm cực trị của đồ thị hàm số $y=x^3+a x^2+b x+c$ và đường thẳng $(AB)$ đi qua gốc tọa độ. Giá trị nhỏ nhất $\mathrm{P}_{\min}$ của $P=a b c+a b+c$ bằng $-\dfrac{m}{n}$ (với $\dfrac{m}{n}$ là phân số tối giản, $m;n$ nguyên dương). Tính $m+n$.
\shortans{$34$}
\loigiai
{
Tập xác định $\mathscr{D}=\mathbb{R}$.\\
$f'(x)=3x^2+2ax+b$.\\
Điều kiện để hàm số có hai điểm cực trị là $f'(x)=0$ có hai nghiệm phân biệt $\Rightarrow a^2-3b>0$.\\
Lấy $f(x)$ chia cho $f'(x)$, ta có $f(x)=f'(x)\left(\dfrac{1}{3}x+\dfrac{1}{9}a\right)+\left(\dfrac{2}{3}b-\dfrac{2}{9}\right)x+c-\dfrac{1}{9}ab$.\\
Suy ra, đường thẳng qua hai cực trị là $(AB): y=\left(\dfrac{2}{3} b-\dfrac{2a^2}{9}\right) x+c-\dfrac{a b}{9}$.\\
Do $(AB)$ qua gốc $O$ nên $c-\dfrac{a b}{9}=0\Leftrightarrow a b=9c$.\\
Khi đó $P=a b c+a b+c=9c^2+10c=\left(3c+\dfrac{5}{3}\right)^2-\dfrac{25}{9}\ge-\dfrac{25}{9},\forall c\in\mathbb{R}$.\\
Vậy $\mathrm{P}_{\text{min}}=-\dfrac{25}{9}$ khi $\heva{&c=-\dfrac{5}{9}\\&a b=-5.}$\\
Suy ra $m+n=34$.
}
\end{ex}

\begin{ex}%[Mức độ 4]%[2D1C2-1]
Cho hàm số $y=f(x)$ có đúng ba điểm cực trị là $-2;-1;0$ và có đạo hàm liên tục trên $\mathbb{R}$. Khi đó hàm số $y=f\left(x^2-2x\right)$ có bao nhiêu điểm cực trị?
\shortans{$7$}
\loigiai
{
Do hàm số $y=f(x)$ có đúng ba điểm cực trị là $-2;-1;0$ và có đạo hàm liên tục trên $\mathbb{R}$ nên $f'(x)=0$ có ba nghiệm là $x=-2;x=-1;x=0$.\\
Đặt $g(x)=f\left(x^2-2x\right)\Rightarrow g'(x)=(2x-2)\cdot f'\left(x^2-2x\right)$. \\
Vì $f'(x)$ liên tục trên $\mathbb{R}$ nên $g'(x)$ cũng liên tục trên $\mathbb{R}$.\\
Do đó những điểm $g'(x)$ có thể đổi dấu khi đi qua các điểm thỏa mãn
$$\hoac{&{2 x-2=0}\\&
{x^2-2 x=-2}\\&
{x^2-2 x=-1}\\&
{x^2-2 x=0}}
\Leftrightarrow\hoac{&x=1\\&x=0\\&x=2.}$$
Vậy hàm số $g(x)$ có ba điểm cực trị.
}
\end{ex}

\begin{ex}%[CD12-CTST, Mức độ 4]%[2D1C1-5]
\immini{Mặt cắt ngang của một máng dẫn nước là một hình thang cân có độ dài đáy bé bằng độ dài cạnh bên và bằng $a$ (cm) không đổi (Hình vẽ). Gọi $\alpha$ là một góc của hình thang cân tạo bởi đáy bé và cạnh bên $\left(\dfrac{\pi}{2} \leq \alpha<\pi\right)$. Tìm $\alpha$ để diện tích mặt cắt ngang của máng lớn nhất.}
{\begin{tikzpicture}[>=stealth,line join=round,line cap=round,font=\footnotesize,scale=0.6]
\path
(0,0) coordinate (D)
(4,0) coordinate (C)
($(C)!1!-120:(D)$) coordinate (B)
($(D)!1!120:(C)$) coordinate (A)
;
\draw(A)--(D)node[pos=0.5,left]{$a$};
\draw(D)--(C)node[pos=0.5,below]{$a$};
\draw(C)--(B)node[pos=0.5,right]{$a$};
\draw[fill=cyan!80!blue] (A)--(B)--(C)--(D)--(A);
\draw (A)--(D)--(C)--(B);
\draw pic[draw, angle radius=3mm, angle eccentricity=1.5]{ angle = B--C--D};
\node at ($(C)+(100:0.75)$){\small $\alpha$};
%			\draw (1.8,-0.7)node[below]{$\text{Hình 5}$ };
\end{tikzpicture}}
Hàm số $S(\alpha)$ mô tả diện tích mặt cắt ngang theo góc $\alpha$ có bảng biến thiên như sau
\begin{center}
\begin{tikzpicture}
\tkzTabInit[lgt=1.5, espcl=4]
{$x$/.6,$S'(\alpha)$/0.6,$S(\alpha)$/3}{$0$,$a$,$\pi$}
\tkzTabLine{,+,0,-,}
\tkzTabVar{-/$0$,+/$b$,-/$1$}
\end{tikzpicture}
\end{center}
Tính $a\cdot b$. (làm tròn đến hàng phần trăm)
\shortans{$2{,}72$}
\loigiai{
\immini{Vì $ABCD$ là hình thang cân nên $\heva{&\widehat{A}=\widehat{B}\\&\widehat{C}=\widehat{D}=\alpha.}$\\
$\widehat{A}+\widehat{B}+\widehat{C}+\widehat{D}=2\pi$.\\
$2\widehat{A}+2\alpha=2\pi$ hay $\widehat{A}=\pi-\alpha$.\\
$DH=AD\sin A=a\cdot \sin \left(\pi-a\right)=a\sin \alpha$.\\
$\begin{aligned}[t]AB&=DC+2AH\\
&=a+2a\cos \left(\pi-a\right)\\
&=a-2a\cos \alpha\\
&=a\left(1-2\cos \alpha\right).
\end{aligned}$}
{\begin{tikzpicture}[>=stealth,line join=round,line cap=round,font=\footnotesize,scale=0.6]
\path
(0,0) coordinate (D)
(4,0) coordinate (C)
($(C)!1!-120:(D)$) coordinate (B)
($(D)!1!120:(C)$) coordinate (A)
($(A)!(D)!(B)$) coordinate (H)
;
\draw(A)--(D)node[pos=0.5,left]{$a$};
\draw(D)--(C)node[pos=0.5,below]{$a$};
\draw(C)--(B)node[pos=0.5,right]{$a$};
\draw[fill=cyan!80!blue] (A)--(B)--(C)--(D)--(A);
\draw(A)--(D)--(C)--(B) (D)--(H);
\foreach \x/\y in {A/90,B/90,C/-90,D/-90,H/90}
\draw[fill=black] (\x) circle (1.1pt) + (\y:0.5cm) node{$\x$};
\end{tikzpicture}}
\noindent Diện tích mặt cắt ngang
\begin{eqnarray*}
S&=&\dfrac{1}{2}\cdot \left(AB+CD\right)\cdot DH\\
&=&\dfrac{1}{2}\left[a\left(1-2\cos \alpha\right)+a\right]a\sin \alpha\\
&=&a^2\left(1-\cos \alpha\right)\sin \alpha.
\end{eqnarray*}
$S'(\alpha)=2a^2\sin \dfrac{3\alpha}{2}\sin \dfrac{\alpha}{2}$.\\
$S'(\alpha)=0\Leftrightarrow \hoac{&\alpha=\dfrac{2k\pi}{3}\\&\alpha=2k\pi.}\, (k\in \mathbb{Z})$.\\
Vì $\dfrac{\pi}{2}\leq \alpha<\pi$ nên  $x=\dfrac{2\pi}{3}$.\\
$S\left(\dfrac{2\pi}{3}\right)=\dfrac{3\sqrt{3}}{4}$.\\
$S\left(\dfrac{\pi}{2}\right)=1$.\\
Bảng biến thiên
\begin{center}
\begin{tikzpicture}
\tkzTabInit[lgt=1.5, espcl=4]
{$x$/1,$S'(\alpha)$/0.6,$S(\alpha)$/3}{$0$,$\dfrac{2\pi}{3}$,$\pi$}
\tkzTabLine{,+,0,-,}
\tkzTabVar{-/$0$,+/$\dfrac{3\sqrt{3}}{4}$,-/$1$}
\end{tikzpicture}
\end{center}
Vậy $a\cdot b =\dfrac{2\pi}{3}\cdot \dfrac{3\sqrt{3}}{4}\approx2{,}72$.
}
\end{ex}

\begin{ex}%[CD12-CTST, Mức độ 4]%[2D1C1-5]
\immini{

Người ta muốn thiết kế một lồng nuôi cá có bề mặt hình chữ nhật bao gồm phần mặt nước có diện tích bằng $54$ m$^2$ và phần đường đi xung quanh với kích thước (đơn vị: m) như Hình vẽ.
Khi kích thước $a$ thay đổi trong khoảng $(3;+\infty)$ thì giá trị hàm số mô tả diện tích lối đi theo kích thước $a$ sẽ giảm đến giá trị $S_0$ rồi tăng lên. Xác định giá trị $S_0$.

}
{\begin{tikzpicture}[>=stealth,line join=round,line cap=round,font=\footnotesize,scale=0.75]
\path
(0,0) coordinate (A)
(5,0) coordinate (B)
(0,4) coordinate (D)
($(D)+(B)-(A)$) coordinate (C)
($(B)+(-0.5,0)$) coordinate (M)
($(B)+(0,0.5)$) coordinate (N)
($(A)+(1,0)$) coordinate (H)
($(A)+(0,0.5)$) coordinate (T)
($(C)+(-0.5,0)$) coordinate (Q)
($(C)+(0,-0.5)$) coordinate (P)
($(D)+(1,0)$) coordinate (R)
($(D)+(0,-0.5)$) coordinate (S)
($(T)+(H)-(A)$) coordinate (A')
($(M)+(N)-(B)$) coordinate (B')
($(Q)+(P)-(C)$) coordinate (C')
($(R)+(S)-(D)$) coordinate (D')
($(D)+(0,0.5)$) coordinate (x)
($(C)+(0,0.5)$) coordinate (y)
($(D)+(-0.5,0)$) coordinate (u)
($(A)+(-0.5,0)$) coordinate (v)
($(A)+(0,-0.5)$) coordinate (x')
($(H)+(0,-0.5)$) coordinate (y')
($(M)+(0,-0.5)$) coordinate (u')
($(B)+(0,-0.5)$) coordinate (v')
($(C)+(0.5,0)$) coordinate (x'')
($(P)+(0.5,0)$) coordinate (y'')
($(N)+(0.5,0)$) coordinate (u'')
($(B)+(0.5,0)$) coordinate (v'')
;
\draw[fill=cyan!20!brown](A)--(B)--(C)--(D)--(A);
\draw[fill=cyan!90!blue](A')--(B')--(C')--(D')--(A');
\draw(D)--(x) (C)--(y)(D)--(u)(A)--(v)(A)--(x')(H)--(y')(M)--(u')(B)--(v')(C)--(x'') (P)--(y'') (N)--(u'') (B)--(v'');
\draw[<->] (x)--(y)node[pos=0.5,above]{$a$};
\draw[<->] (u)--(v)node[pos=0.5,left]{$b$};
\draw[<->] (x')--(y')node[pos=0.5,below]{$2$};
\draw[<->] (u')--(v')node[pos=0.5,below]{$1$};
\draw[<->] (u'')--(v'')node[pos=0.5,right]{$1$};
\draw[<->] (x'')--(y'')node[pos=0.5,right]{$1$};
%			\draw (2.2,-0.5)node[below]{$\text{Hình 8}$ };
\end{tikzpicture}}
\shortans{$42$}
\loigiai{
Gọi $x$, $y$ lần lượt là độ dài, rộng của mặt nước. Điều kiện $x$, $y>0$.\\
Phần mặt nước có diện tích bằng $54$ m$^2$ nên ta có $$x\cdot y=54.\, \hfill(*)$$
Theo đề bài ta có $x=a-3$, $y=b-2$.\\
Từ $(*)$ suy ra $$(a-3)(b-2)=54\Rightarrow b=\dfrac{54}{a-3}+2=\dfrac{2a+48}{a-3}.$$
Diện tích lối đi là \begin{eqnarray*}
S(a)&=&a\cdot b-x\cdot y\\
&=&ab-54\\
&=&a\cdot \dfrac{2a+48}{a-3}-54\\
&=&\dfrac{2a^2+48a}{a-3}-54.
\end{eqnarray*}
$S'(a)=\dfrac{2a^2-12a-144}{\left(a-3\right)^2}$;\\
$S'(a)=0\Leftrightarrow \hoac{&a=-6\\&a=12.}$\\
Bảng biến thiên
\begin{center}
\begin{tikzpicture}
\tkzTabInit[nocadre=false, lgt=1.2, espcl=2.4]{$a$ /0.7,$S'(a)$ /0.7,$S(a)$ /2.5}{$0$,$3$,$12$,$+\infty$}
\tkzTabLine{,-,d,-,$0$,+,}
\tkzTabVar{+/$-54$  ,-D+/$-\infty$/$+\infty$,-/$42$,+/$+\infty$}
\end{tikzpicture}
\end{center}
Vậy $S_0=42$.
}
\end{ex}

\begin{ex}%[Mức độ 3]%[Dự án giảng new 4in1, Trần Quang Thạnh]%[2D1C1-4]
Tính tổng tất cả các nghiệm của phương trình $\log_{3}\dfrac{x^{2}+x+3}{2x^{2}+4x+5}=x^{2}+3x+2 $.
\shortans{$1$}
\loigiai{Xét phương trình $\log_{3}\dfrac{x^{2}+x+3}{2x^{2}+4x+5}=x^{2}+3x+2$. $\qquad(*)$\\
Điều kiện xác định $ \heva{&\dfrac{x^{2}+x+3}{2x^{2}+4x+5}>0\\ &2x^{2}+4x+5\neq 0}\Leftrightarrow \forall x\in\mathbb{R} $.\\
Ta có
\begin{eqnarray*}
(*)&\Leftrightarrow& \log_{3} (x^{2}+x+3)-\log_{3}(2x^{2}+4x+5)=(2x^{2}+4x+5)-(x^{2}+x+3)\\
&\Leftrightarrow& (x^{2}+x+3)+\log_{3} (x^{2}+x+3)=\log_{3}(2x^{2}+4x+5)+(2x^{2}+4x+5).
\end{eqnarray*}
Xét hàm số $ f(t)=t+\log_{3}t $ trên khoảng $ (0;+\infty ) $.\\
Ta có cơ số $ 3>1 $ và hàm số $ y=t $ đồng biến nên $ f(t) $ đồng biến trên khoảng $ (0;+\infty) $.\\
Do đó $ f(x^{2}+x+3)=f(2x^{2}+4x+5)\Leftrightarrow x^{2}+x+3=2x^{2}+4x+5\Leftrightarrow x^{2}+3x+2=0\Leftrightarrow\hoac{&x=-1\\ &x=2.} $\\
Do đó $ x=-1 $, $ x=2 $ là các nghiệm của phương trình.\\
Vậy tổng các nghiệm của phương trình là $1$.
}
\end{ex}

\begin{ex}%[Mức độ 4]%[Dự án giảng new 4in1, Trần Quang Thạnh]%[2D1C1-4]
Có bao nhiêu số nguyên dương $y>4$ sao cho tồn tại số thực $x\in(1;6)$ thỏa mãn $4(x-1)\mathrm{e}^x=y(\mathrm{e}^x+xy-2x^2-3)$?
\shortans{$14$}
\loigiai{
\allowdisplaybreaks
\begin{eqnarray*}
&&	4(x-1)\mathrm{e}^x=y(\mathrm{e}^x+xy-2x^2-3)\\
&\Leftrightarrow& 4(x-1)\mathrm{e}^x-y(\mathrm{e}^x+xy-2x^2-3)=0.
\end{eqnarray*}
Xét hàm số $y=f(x)=4(x-1)\mathrm{e}^x-y(\mathrm{e}^x+xy-2x^2-3)$ liên tục trên $[1;6]$ có
\allowdisplaybreaks
\begin{eqnarray*}
f'(x)&=&4\mathrm{e}^x+4(x-1)\mathrm{e}^x-y(\mathrm{e}^x+y-4x)\\
&=&(\mathrm{e}^x+y)(4x-y).
\end{eqnarray*}
\noindent
Cho $f'(x)=0\Leftrightarrow x=\dfrac{y}{4}$.\\
Do $x\in(1;6)$ nên hàm số $y=f(x)$ sẽ tồn tại điểm cực trị $x=\dfrac{y}{4}$ khi $ y\in (4;24)$.\\
Từ đó ta có cơ sở chia các trường hợp như sau
\begin{itemize}
\item Trường hợp 1: $y\ge 24$.
\begin{center}
\begin{tikzpicture}
\tkzTabInit[nocadre=false,lgt=1.2,espcl=2.5,deltacl=0.6]
{$x$/0.7,$f'(x)$/0.7,$f(x)$/2}
{$1$,$6$}
\tkzTabLine{,-,}
\tkzTabVar{+/$f(1)$,-/$f(6)$}
\end{tikzpicture}
\end{center}
Ta có $\heva{&f(1)=-y(\mathrm{e}+y-5)\\&f(6)=20\mathrm{e}^6-y(\mathrm{e}^6+6y-75).}$\\
Điều kiện cần và đủ để tồn tại $x$ là
$$\heva{&f(6)<0\\&f(1)\cdot f(6)<0}\Rightarrow f(1)>0.$$
Mặt khác ta thấy $-y(\mathrm{e}+y-5)<0,\;\forall y\ge 24$ (vô lí) nên loại.
\item Trường hợp 2: $4<y<24$.
\begin{center}
\begin{tikzpicture}
\tkzTabInit[nocadre=false,lgt=1.2,espcl=2.5,deltacl=0.6]
{$x$/0.7,$f'(x)$/0.7,$f(x)$/2}
{$1$,$\tfrac{y}{4}$,$6$}
\tkzTabLine{,-,z,+,}
\tkzTabVar{+/$f(1)$,-/$f\left(\tfrac{y}{4}\right)$,+/$f(6)$}
\end{tikzpicture}
\end{center}
Do $f(1)<0$ nên để tồn tại nghiệm $x\in(1;6)$ thì $f(6)>0$
\allowdisplaybreaks
\begin{eqnarray*}
&\Leftrightarrow&20\mathrm{e}^6-y(\mathrm{e}^6+6y-75>0\\
&\Leftrightarrow&\heva{&-6y^2-(\mathrm{e}^6-75)y+20\mathrm{e}^6>0\\&y\in\mathbb{N}^*;\;y\in(4;24)}\\
&\Leftrightarrow&y\in\{5;6;\ldots;18\}.
\end{eqnarray*}
\end{itemize}
Vậy có tất cả $14$ giá trị nguyên dương $y$ thỏa đề bài.
}
\end{ex}

\begin{ex}%[Đề Tham khảo 2021, Mức độ 4]%[Dự án giảng new 4in1, Trần Quang Thạnh]%[2D1C1-4]
Có bao nhiêu số nguyên $a ~(a\ge2)$ sao cho tồn tại số thực $x$ thỏa mãn $\left(a^{\log x}+2\right)^{\log a}=x-2$?
\shortans{$8$}
\loigiai{
Với $x>0$ đặt $y=a^{\log x}+2>0$ ta được $y^{\log a}=x-2\Leftrightarrow x=a^{\log y}+2$. \\
Từ đó ta có $y=a^{\log x}+2$ và $x=a^{\log y}+2$.\\
Do $a\ge2$ nên $f(t)=a^t+2$ đồng biến trên $\mathbb R$.\\ Giả sử $x\ge y$ thì $f(y)\ge f(x)$, suy ra $y\ge x$ tức là $x=y$. Tương tự $x\le y$ cũng có $x=y$.\\
Vì thế chỉ xét phương trình $x=a^{\log x}+2$ với $x>0$ hay $x-x^{\log a}=2$.\\
Ta phải có $x>2$ và $x>x^{\log a}\Leftrightarrow 1>\log a\Leftrightarrow a<10$.\\
Ngược lại $a<10$ thì xét hàm số liên tục $g(x)=x-x^{\log a}-2=x^{\log a}\left(x^{1-\log a}-1\right)-2$ có $\lim\limits_{x\to+\infty}g(x)=+\infty$ và $g(2)<0$ nên $g(x)$ sẽ có nghiệm trên $(2;+\infty)$. \\
Do đó các số $a\in \{2,3,\ldots,9\}$ đều thỏa mãn.
}
\end{ex}

\begin{ex}%[Mức độ 4]%[Dự án giảng new 4in1, Trần Quang Thạnh]%[2D1C1-3]
Có bao nhiêu giá trị nguyên của tham số $m$ trên khoảng $(-100;100)$ sao cho hàm số $y=\dfrac{-\mathrm{e}^x+3}{\mathrm{e}^x+m}$ nghịch biến trên khoảng $(0;+\infty)$?
\shortans{$101$}
\loigiai{
Điều kiện: $\mathrm{e}^x\neq-m$.\\
Ta có $y'=\mathrm{e}^x\cdot\dfrac{-m-3}{\left(\mathrm{e}^x+m\right)^2}$.\\
Ta có $\heva{&\mathrm{e}^x>0\\&\mathrm{e}^x\in(1;+\infty)}$, $\forall x\in(0;+\infty)$ và khi $x\in(0;+\infty)$ thì $\mathrm{e}^x \in (1;+\infty)$.\\
Suy ra hàm số $y$ nghịch biến trên khoảng $(0;+\infty)$ khi và chỉ khi
$$\heva{&-m-3<0\\&-m\notin(1;+\infty)}\Leftrightarrow\heva{&m >-3\\&m\geq-1}\Leftrightarrow m\geq-1.$$
Vì $\heva{&m\in\mathbb{Z}\\&m\in(-100;100)}$ nên $m\in\left\{-1;0;1;\ldots;99\right\}$.\\
Suy ra có $101$ giá trị của $m$ thỏa mãn.
}
\end{ex}

\begin{ex}%[Mức độ 4]%[Dự án giảng new 4in1, Trần Quang Thạnh]%[2D1C1-3]
Có bao nhiêu giá trị nguyên của tham số $a$ trên đoạn $[-100; 100]$ để hàm số $f(x)=\dfrac{(a+1)\ln x-6}{\ln x-3a}$ nghịch biến trên khoảng $(1; \mathrm{e})$?
\shortans{$198$}
\loigiai{
Ta có $f'(x)=(\ln x)' \cdot \dfrac{-3a^2-3a+6}{(\ln x-3a)^2}=\dfrac{1}{x}\cdot \dfrac{-3a^2-3a+6}{(\ln x-3a)^2}.$\\
Với
$1<x<\mathrm{e}$ thì  $0<\ln x<1$.\\
Do đó hàm số $f(x)$ nghịch biến trên khoảng $(1; \mathrm{e})$ khi và chỉ khi
$$\heva{&-3a^2-3a+6<0\\&3a\notin(0; 1)}\Leftrightarrow\heva{&\hoac{&a <-2\\&a>1}\\&\hoac{&a\leq 0\\&a\geq\dfrac{1}{3}}}\Leftrightarrow\hoac{&a <-2\\&a>1.}$$
Vì $m\in [-100;100]$ nên $m\in \{-100;-99;\ldots;-3;1;\ldots;100\}$.\\
Vậy có $198$ giá trị nguyên của tham số $a$ để hàm số đã cho nghịch biến trên khoảng $(1; \mathrm{e})$.
}
\end{ex}

\begin{ex}%[BG12new-4in1, Trần Hoà]%[2D1C1-2]
\immini{
Cho hàm số $f(x)$ có đồ thị như hình vẽ bên. Xét $x\in \left(0;\dfrac{\pi}{2}\right)$, biết hàm số $f(\sin x)$ nghịch biến trên khoảng $(a;b)$. Khi đó giá trị lớn nhất của $|a-b|$ bằng bao  nhiêu? (kết quả làm tròn đến hàng phần phần trăm).
\shortans[]{$0{,}52$}
}{
\begin{tikzpicture}[>=stealth,font=\footnotesize,yscale=1.2, xscale=1.5]
\draw[->] (-0.5,0) -- (1.3,0) node[below] {$x$};
\draw[->] (0,-1.2) -- (0,1) node[left] {$y$};
\filldraw (0,0) node[above left=-0.1] {$O$} circle (1pt);
\draw[smooth,samples=100,domain=-0.4:0.9] plot(\x,{(16/3)*(\x)^3-4*(\x)^2}) node[right]{$f(x)$};
\filldraw (0.5,0) node[above] {$\frac{1}{2}$} circle (1pt);
\draw[dashed] (0.5,0) -- (0.5,-0.33);
\end{tikzpicture}
}
\loigiai{
Từ đồ thị của hàm số $f(x)$, suy ra
$$f'(x)<0\Leftrightarrow 0<x<\dfrac{1}{2};\qquad f'(x)>0\Leftrightarrow\hoac{&x<0\\ &x>\dfrac{1}{2}.}$$
Đặt $g(x)=f(\sin x)$, ta có $g'(x)=\cos x\cdot f'(\sin x)$. Xét trên khoảng $(0;\pi)$:
$$g'(x)<0\Leftrightarrow\hoac{&\cos x>0\ \text{và}\ f'(\sin x)<0\\ &\cos x<0\ \text{và}\ f'(\sin x)>0}\Leftrightarrow \hoac{&0<x< \dfrac{\pi}{2}\ \text{và}\ f'(\sin x)<0\qquad \qquad\ (1)\\ &\dfrac{\pi}{2}<x<\pi\ \text{và}\ f'(\sin x)>0.\qquad \qquad (2)}$$
Ta có
{\allowdisplaybreaks
\begin{align*}
&(1)\Leftrightarrow \heva{&0<x<\dfrac{\pi}{2}\\ &0<\sin x<\dfrac{1}{2}}\Leftrightarrow 0<x<\dfrac{\pi}{6}.\\
&(2)\Leftrightarrow \heva{&\dfrac{\pi}{2}<x<\pi\\ &\hoac{&\sin x<0\\ &\sin x>\dfrac{1}{2}}}\Leftrightarrow \dfrac{\pi}{2}<x<\dfrac{5\pi}{6}.
\end{align*}}
Vậy hàm số nghịch biến trên khoảng $\left(0;\dfrac{\pi}{6}\right)$. Suy ra $a=0$, $b=\dfrac{\pi}{6}$. Vậy $a+b\approx 0{,}52$.
}
\end{ex}

\begin{ex}%[2D1C1-1]
Cho hàm số $y= f(x)$ có đạo hàm $f'(x)=(x-1)(x-2)$.
Biết hàm số $y = f(x-x^2)$ nghịch biến trên khoảng có dạng $\left(\dfrac{a}{b};+\infty\right)$ với $\dfrac{a}{b}$ là tối giản và $b>0$. Giá trị của biểu thức $a^2+b^2$ bằng bao nhiêu?
\shortans[]{$5$}
\loigiai{
Đặt $y = g(x) = f(x-x^2) \Rightarrow g'(x) = f'(x-x^2) \cdot (x-x^2)' = (1-2x)f'(x-x^2)$.\\
Khi đó
\allowdisplaybreaks
\begin{eqnarray*}
g'(x) = 0
&\Leftrightarrow& \hoac{&1-2x=0\\&f'(x-x^2) =0 }\\
&\Leftrightarrow&  \hoac{& 1-2x=0\\ & x-x^2 =1\\& x-x^2 =2}\\
&\Leftrightarrow&  x = \dfrac{1}{2}.
\end{eqnarray*}
Với $x< \dfrac{1}{2}$ thì $\heva{&1-2x>0\\& f'\left[ - \left( x- \dfrac{1}{2}\right)^2 + \dfrac{1}{4}  \right] >0  } $ nên $g'(x)>0$.\\
Với $x> \dfrac{1}{2}$ thì $\heva{&1-2x<0\\& f'\left[ - \left( x- \dfrac{1}{2}\right)^2 + \dfrac{1}{4}  \right] >0  } $ nên $g'(x)<0$.\\
Hay hàm số $g(x) = f(x-x^2)$ nghịch biến trên khoảng $\left( \dfrac{1}{2}; + \infty \right) $.\\
Suy ra $a=1$, $b=2$ nên $a^2+b^2=5$.
}
\end{ex}

\begin{ex}%[2D1C5-5]
\immini{Cho hàm số $y=f(x)$ có đạo hàm trên $\mathbb{R}$. Biết rằng hàm số $y=f'(x)$ có đồ thị như hình vẽ bên. Hỏi đồ thị hàm số $y=f(2x-3)$ cắt đường thẳng $y=-3x+2$ tại nhiều nhất bao nhiêu điểm?
}
{
\begin{tikzpicture}[>=stealth,line join=round,line cap=round,font=\scriptsize,scale=0.7]
\draw [->] (-2.5,0)--(2.5,0)node[below]{$ x $};
\draw [->] (0,-3.5)--(0,2)node[left]{$ y $};
\draw [fill=black] (0,0)node[below left]{$ O $}circle(1pt) (-1,0)node[below left]{$ -1 $}circle(1pt) (1,0)node[below]{$ 1 $}circle(1pt) (0,1)node[left]{$ 1 $}circle(1pt) (0,-1)node[right]{$ -1 $}circle(1pt) (0,-3)node[right]{$-3$} circle (1pt) ;
\draw [smooth,domain=-2.1:2.1] plot(\x,{-(\x)^3+3*(\x)-1});
\draw[dashed] (-1,0)|-(0,-3) (1,0)|-(0,1);
\clip (-2.5,-3.5) rectangle (2.5,2.2);
\end{tikzpicture}
}
\shortans{$4$}
\loigiai{
Phương trình hoành độ giao điểm của hai đồ thị hàm số $y=f(2x-3)$ và $y=-3x+2$
$$f(2x-3)=-3x+2\Leftrightarrow f(2x-3)+3x-2=0.$$
Đặt $g(x)=f(2x-3)+3x-2$, ta có $g'(x)=2f'(2x-3)+3=0\Leftrightarrow f'(2x-3)=-\dfrac{3}{2}$.\\
Dựa vào đồ thị, đường thẳng $y=-\dfrac{3}{2}$ cắt đồ thị $f'(x)$ tại ba điểm phân biệt nên phương trình $f'(2x-3)=-\dfrac{3}{2}$ cũng có ba nghiệm phân biệt, giả sử ba nghiệm đó lần lượt là $a$, $b$, $c$ với $a<b<c$.\\
Ta có bảng biến thiên
\begin{center}
\begin{tikzpicture}
\tkzTabInit[nocadre=false,lgt=1.2,espcl=2.5,deltacl=.6]
{$x$/1,$g'(x)$/0.6,$g(x)$/2}
{$-\infty$, $a$, $b$, $c$, $+\infty$}
\tkzTabLine{,+,0,-,0,+,0,-,}
\tkzTabVar{-/,+/,-/,+/,-/}
\end{tikzpicture}
\end{center}
Dựa vào bảng biến thiên, suy ra phương trình $g(x)=0$ có tối đa $4$ nghiệm hay đồ thị hàm số $y=f(2x-3)$ cắt đường thẳng $y=-3x+2$ tại nhiều nhất $4$ điểm.
}
\end{ex}


\Closesolutionfile{ans}

\newpage
\begin{center}
    \bfseries\faGg~\faGg~\faGg~BẢNG ĐÁP ÁN TRẮC NGHIỆM~\faGg~\faGg~\faGg
\end{center}
\inputansbox{8}{ans/ansBTchoice}
\inputansbox[3]{2}{ans/ansBTchoiceTF}
\inputansbox{6}{ans/ansBTshortans}
% \newpage

%Chương II. Vector trong KG
%%Bài 1. Vector trong khong gian
% \section{VECTƠ TRONG KHÔNG GIAN}
\subsection{LÝ THUYẾT CẦN NHỚ}
\subsubsection{Tổng của hai véc tơ}
\begin{enumerate}[\iconMT]
	\item \textbf{Định nghĩa:}
	 \immini{Trong không gian, cho hai véctơ $\vec{a}$ và $\vec{b}$. Lấy ba điểm $O$, $A$, $B$ sao cho $\vec{OA}=\vec{a}$, $\vec{AB}=\vec{b}$. Ta gọi $\vec{OB}$ là \textbf{tổng của hai véctơ} $\vec{a}$ và $\vec{b}$, ký hiệu $\vec{a}+\vec{b}$.\\
	 Phép lấy tổng của hai véctơ $\vec{a}$ và $\vec{b}$ được gọi là \textbf{phép cộng véctơ}.}
	 {
	 \begin{tikzpicture}[>=stealth,scale=.5,font=\footnotesize]
	 \foreach \x\y\t in {3/0.4/A1,4./4.4/A2,9.5/4.3/B1,14.5/1.3/B2,5/0/O}
	 \coordinate (\t) at (\x,\y);
	 \coordinate (A) at ($(A2)-(A1)+(O)$);
	 \coordinate (B) at ($(B2)-(B1)+(A)$);
	 \foreach \a/\b in {A1/A2,B1/B2,O/A,A/B,O/B}
	 {\draw[-{Stealth[length=2.5mm]}](\a)--(\b);}
	 \node at ($(A1)!1/2!(A2)$)[left=-2pt]{$\vec{a}$};
	 \node at ($(B1)!1/2!(B2)$)[above right=-2pt]{$\vec{b}$};
	 \node at ($(O)!1/2!(A)$)[left=-2pt]{$\vec{a}$};
	 \node at ($(A)!1/2!(B)$)[above right=-2pt]{$\vec{b}$};
	 \node at ($(O)!1/2!(B)$)[rotate=7,above=-2pt]{$\vec{a}+\vec{b}$};
	 \foreach \t/\g in {O/-140,A/90,B/0}
	 \draw[fill=black] (\t)circle(1.2pt) +(\g:12pt)node{$\t$};
	 \end{tikzpicture}}
	\item \textbf{Các quy tắc cần nhớ:}
	 \begin{listEX}[1]
	 \immini{\item [\ding{172}] Quy tắc ba điểm: Với ba điểm $A$, $B$, $C$, ta có
	 \fbox{$\vec{AB} + \vec{BC} = \vec{AC}$}
	 \item [\ding{173}] Quy tắc hình bình hành: Cho $ABCD$ là hình bình hành, ta có
	 \fbox{$\vec{AB} + \vec{AD} = \vec{AC}$}}{
	 \begin{tikzpicture}[scale=0.6, font=\footnotesize, line join=round, line cap=round]
	 \begin{scope}
	 \foreach \x\y\t in {0/0/A, -2/-2/B, 2.5/-1.5/C}
	 \coordinate (\t) at (\x,\y);
	 \foreach \a\b in {A/B, B/C,A/C}
	 \draw[-{Stealth[length=2mm]}] (\a)--(\b);
	 \foreach \t\g in {A/90, B/-100, C/-80}
	 \draw[fill=black] (\t)circle(0.6pt) +(\g:8pt)node{$\t$};
	 \end{scope}
	 \begin{scope}[xshift=5cm]
	 \foreach \x\y\t in {0/0/A, -1.5/-2/B, 2.5/-2/C,4/0/D}
	 \coordinate (\t) at (\x,\y);
	 \foreach \a\b in {A/B, A/D,A/C}
	 \draw[-{Stealth[length=2mm]}] (\a)--(\b);
	 \draw[dashed] (B)--(C)--(D);
	 \foreach \t\g in {A/90, B/-90, C/-45,D/50}
	 \draw[fill=black] (\t)circle(0.6pt) +(\g:8pt)node{$\t$};
	 \end{scope}
	 \end{tikzpicture}}
	 \immini{\item [\ding{174}] Quy tắc hình hộp:
	 Cho hình hộp $ABCD.A'B'C'D'$. Ta có
	 \fbox{$\vec{AB} + \vec{AD} + \vec{AA'} = \vec{AC'}$}
	 \begin{note}
	 Hệ thức tương tự: \quad $\vec{BA} + \vec{BC} + \vec{BB'} = \vec{BD'}$.
	 \end{note}
	 }{
	 \begin{tikzpicture}[scale=0.6, font=\footnotesize, line join=round, line cap=round]
	 \def\h{4}
	 \foreach \x\y\t in {0/0/A',-1/-1.1/B',2.6/-1.1/C'}
	 \coordinate (\t) at (\x,\y);
	 \coordinate (D') at ($(A')+(C')-(B')$);
	 \coordinate (A) at ($(A')+(0,2.5)$);
	 \coordinate (B) at ($(B')+(0,2.5)$);
	 \coordinate (C) at ($(C')+(0,2.5)$);
	 \coordinate (D) at ($(D')+(0,2.5)$);
	 \foreach \a\b in {A/B, A/D,A/C}
	 \draw[-{Stealth[length=2mm]}] (\a)--(\b);
	 \foreach \a\b in {A/A', A/C'}
	 \draw[-{Stealth[length=2mm]},dashed] (\a)--(\b);
	 \draw (D)--(C)--(B)--(B')--(C')--(D')--(D) (C')--(C);
	 \draw[dashed](B')--(A')--(D');
	 \foreach \t/\g in {A'/170,B'/-150,C'/-70,D'/0,A/100,B/170,C/-20,D/50}
	 \draw[fill=black] (\t) circle(1pt)
	 node[shift={(\g:7pt)}]{$\t$};
	 \end{tikzpicture}
	 }
	 \end{listEX}
	\item \textbf{Tính chất:}
	 \begin{itemize}
	 \item[\ding{172}] Tính chất giao hoán: $\vec{a}+\vec{b}=\vec{b}+\vec{a}$;
	 \item[\ding{173}] Tính chất kết hợp: $\left(\vec{a}+\vec{b}\right)+\vec{c}=\vec{a}+\left(\vec{b}+\vec{c}\right)$;
	 \item[\ding{174}] Với mọi véctơ $\vec{a}$, ta luôn có: $\vec{a}+\vec{0}=\vec{0}+\vec{a}=\vec{a}$.
	 \item[\ding{175}] Tổng của ba véctơ $\vec{a}$, $\vec{b}$, $\vec{c}$:\quad $\vec{a}+\vec{b}+\vec{c}=\left(\vec{a}+\vec{b}\right)+\vec{c}.$
	 \end{itemize}
\end{enumerate}
\subsubsection{Hiệu của hai véc tơ}
\begin{enumerate}[\iconMT]
	\item \textbf{Véctơ đối:}
	 \begin{listEX}[1]
	 \item [\ding{172}] Vectơ đối của $\vec{a}$ kí hiệu là $-\vec{a}$.
	 \item [\ding{173}] Vectơ đối của $\vec{AB}$ là $\vec{BA}$: $-\vec{AB}=\vec{BA}$.
	 \item [\ding{174}] Vectơ $\vec{0}$ được coi là vectơ đối của chính nó.
	 \end{listEX}
	 \immini{
	\item \textbf{Định nghĩa hiệu của hai véctơ:} Trong không gian, cho hai véctơ $\vec{a}$, $\vec{b}$. Ta gọi $\vec{a}+\left(-\vec{b}\right)$ là \textbf{hiệu của hai véctơ} $\vec{a}$ và $\vec{b}$, ký hiệu $\vec{a}-\vec{b}$.\\
	 Phép lấy hiệu của hai véctơ được gọi là \textbf{phép trừ véctơ}.
	\item \textbf{Các quy tắc cần nhớ:}
	 \begin{listEX}[1]
	 \item [\ding{172}] Với ba điểm $A$, $B$, $C$ ta có $\vec{AB}-\vec{AC}=\vec{CB}$.
	 \item [\ding{173}] Hai véc tơ $\vec{a}$ và $\vec{b}$ đối nhau thì $\vec{a}+\vec{b}=\vec{0}$.
	 \end{listEX}
	 }{
	 \begin{tikzpicture}[scale=1, font=\footnotesize, line join=round, line cap=round]
	 \foreach \x\y\t in {0/0/O,1.5/1/A,3.1/-0.4/B,-0.3/1/a1,1.1/1.6/b1}
	 \coordinate (\t) at (\x,\y);
	 \coordinate (a2) at ($(a1)+(A)$);
	 \coordinate (b2) at ($(b1)+(B)$);
	 \foreach \a\b in {a1/a2, b1/b2,O/A,O/B,B/A}
	 \draw[-{Stealth[length=2mm]}] (\a)--(\b);
	 \path (a1)--(a2)node[pos=0.5,above left]{$\vec{a}$}
	 (b1)--(b2)node[pos=0.5,above]{$\vec{b}$}
	 (O)--(A)node[pos=0.5,above left]{$\vec{a}$}
	 (O)--(B)node[pos=0.5,below]{$\vec{b}$}
	 (B)--(A)node[pos=0.7,right=3pt]{$\vec{a}-\vec{b}$};
	 \foreach \t\g in {A/90, O/180, B/0}
	 \draw[fill=black] (\t)circle(0.2pt) +(\g:5pt)node{$\t$};
	 \end{tikzpicture}}
\end{enumerate}
\subsubsection{Tích của một số với một véc-tơ}
\begin{enumerate}[\iconMT]
	\item \textbf{Định nghĩa:} Cho số thực $k\ne 0$ và vectơ $\vec{a} \ne \vec{0}$. Tích của một số $k$ với vectơ $\vec{a}$ là một vectơ, kí hiệu là $k\vec{a}$, được xác định như sau:
	 \begin{itemize}
	 \item Cùng hướng với vectơ $\vec{a}$ nếu $k>0$, ngược hướng với vectơ $\vec{a}$ nếu $k<0$.
	 \item Có độ dài bằng $|k| \cdot |\vec{a}|$.
	 \end{itemize}
	 \begin{note}
	 $0\cdot \vec{a}=\vec{0}$ và $k\cdot \vec{0}=\vec{0}$.
	 \end{note}
	 % \immini{\textbf{Ví dụ:} Theo hình vẽ bên, thì $\vec{b}=3\vec{a}$; $\vec{c}=-2\vec{a}$; $\vec{c}=-\dfrac{2}{3}\vec{b}$.
	 % }{
	 % \begin{tikzpicture}[>=stealth,scale=0.5, line join=round, line cap=round]
	 % 	 \draw[line width=0.05pt,gray,dashed] (-0.7,-0.7) grid (8.7,3.7);
	 % 	 \draw[->,thick](1,2)--(2,3)node[above left]{$\vec{a}$};
	 % 	 \draw[->,thick](1,0)--(4,3)node[above right]{$\vec{b}$};
	 % 	 \draw[->,thick](7,3)--(5,1)node[below right]{$\vec{c}$};
	 % \end{tikzpicture}}
	\item \textbf{Hệ thức trung điểm, trọng tâm:}
	 \immini{
	 \begin{itemize}
	 \item [\ding{172}] $I$ là trung điểm của đoạn thẳng $AB$ thì
	 \begin{itemize}
	 \item [$\bullet$] $\vec{IA} + \vec{IB} = \vec 0$;
	 \item [$\bullet$] $\vec{IA}=-\vec{IB}$; $\vec{AI}=\dfrac{1}{2}\vec{AB}$;...
	 \end{itemize}
	 \item [\ding{173}] $G$ là trọng tâm của tam giác $ABC$ thì
	 \begin{listEX}[1]
	 \item [$\bullet$] $\vec{GA}+\vec{GB}+\vec{GC}=\vec{0}$;
	 \item [$\bullet$] $\vec{GA}=-\dfrac{2}{3}\vec{AK}$; $\vec{GA}=-2\vec{GK}$;...
	 \end{listEX}
	 \end{itemize}}{
	 \begin{tikzpicture}[scale=0.8, font=\footnotesize, line join=round, line cap=round]
	 \begin{scope}
	 \foreach \x\y\t in {-2/-2/A, 0/0/B}
	 \coordinate (\t) at (\x,\y);
	 \coordinate (I) at ($(A)!0.5!(B)$);
	 \foreach \a\b in {A/B}
	 \draw[] (\a)--(\b);
	 \foreach \t\g in {A/-90, B/40,I/1200}
	 \draw[fill=black] (\t)circle(0.6pt) +(\g:8pt)node{$\t$};
	 \end{scope}
	 \begin{scope}[xshift=4cm]
	 \foreach \x\y\t in {0/0/A, -2/-2/B, 2.5/-2/C}
	 \coordinate (\t) at (\x,\y);
	 \coordinate (M) at ($(A)!0.5!(B)$);
	 \coordinate (N) at ($(A)!0.5!(C)$);
	 \coordinate (K) at ($(C)!0.5!(B)$);
	 \coordinate (G) at ($(A)!2/3!(K)$);
	 \foreach \a\b in {A/B, B/C, A/C, A/K, M/C, B/N}
	 \draw[] (\a)--(\b);
	 \foreach \t\g in {A/90, B/-100, C/-80, M/120, N/40, K/-90,G/60}
	 \draw[fill=black] (\t)circle(0.8pt) +(\g:10pt)node{$\t$};
	 \end{scope}
	 \end{tikzpicture}}
	\item \textbf{Nhận xét:}
	 \begin{itemize}
	 \item[\ding{172}] Với hai véctơ $\vec{a}$ và $\vec{b}$ bất kỳ, với mọi số $h$ và $k$, ta luôn có
	 \begin{enumEX}[$\bullet$]{3}
	 \item $k\left(\vec{a}+\vec{b}\right)=k\vec{a}+k\vec{b}$;
	 \item $\left(h+k\right)\vec{a}=h\vec{a}+k\vec{a}$;
	 \item $h\left(k\vec{a}\right)=\left(hk\right)\vec{a}$;
	 \item $1\cdot \vec{a}=\vec{a}$;
	 \item $\left(-1\right)\cdot\vec{a}=-\vec{a}$;
	 \item $k\vec{a}=\vec{0} \Leftrightarrow \hoac{&\vec{a}=\vec{0}\\& k=0}$.
	 \end{enumEX}
	 \item[\ding{173}] Hai véctơ $\vec{a}$ và $\vec{b}$ ($\vec{b}$ khác $\vec{0}$) cùng phương khi và chỉ khi có số $k$ sao cho $\vec{a}=k\vec{b}$.
	 \item[\ding{174}] Ba điểm phân biệt $A$, $B$, $C$ thẳng hàng khi và chỉ khi có số $k \neq 0$ để $\vec{AB}=k\vec{AC}$.
	 \end{itemize}
\end{enumerate}
\subsubsection{Tích vô hướng của hai véc-tơ}
\begin{enumerate}[\iconMT]
	\item \textbf{Góc giữa hai véctơ:}
	 \immini{
	 Trong không gian, cho $\vec{u}$ và $\vec{v}$ là hai véctơ khác $\vec{0}$. Lấy một điểm $A$ bất kỳ, gọi $B$ và $C$ là hai điểm sao cho $\vec{AB}=\vec{u}$, $\vec{AC}=\vec{v}$. Khi đó, ta gọi $\widehat{BAC}$ là góc giữa hai véctơ $\vec{u}$ và $\vec{v}$, ký hiệu $\left(\vec{u}, \vec{v}\right)$.
	 \begin{note}
	 $0^{\circ} \leq \left(\vec{u},\vec{v}\right) \leq 180^{\circ}$.
	 \end{note}
	 \begin{note}
	 \begin{itemize}
	 \item [$\bullet$] Nếu $\vec{u}$ cùng hướng với $\vec{v}$ thì $\left(\vec{u}, \vec{v}\right)=0^\circ$;
	 \item [$\bullet$] Nếu $\vec{u}$ ngược hướng với $\vec{v}$ thì $\left(\vec{u}, \vec{v}\right)=180^\circ$;
	 \item [$\bullet$] Nếu $\vec{u}$ vuông góc với $\vec{v}$ thì $\left(\vec{u}, \vec{v}\right)=90^\circ$.
	 \end{itemize}
	 \end{note}
	 }{\vspace{-0.5cm}
	 \begin{tikzpicture}[scale=0.8, font=\footnotesize, line join=round, line cap=round]
	 \foreach \x\y\t in {0/0/A,2/0.8/B,3.2/-1./C,-1/1/u1,-0.5/-1.5/v1}
	 \coordinate (\t) at (\x,\y);
	 \coordinate (u2) at ($(u1)+(B)$);
	 \coordinate (v2) at ($(v1)+(C)$);
	 \draw (-1.5,-1.2)--(3.5,-1.2)--(4.5,1)--(-0.5,1)--cycle;
	 \draw[dashed] (A)--(u1) (B)--(u2) (A)--(v1) (C)--(v2);
	 \foreach \a\b in {A/B, A/C,u1/u2,v1/v2}
	 \draw[-{Stealth[length=2mm]}] (\a)--(\b);
	 \foreach \t\g in {A/-170, B/0, C/50}
	 \draw[fill=black] (\t)circle(0.6pt) +(\g:8pt)node{$\t$};
	 \path (A) pic[draw,angle radius=9]{angle=C--A--B};
	 \path
	 (u1)--(u2)node[pos=0.5,above]{$\vec{u}$}
	 (v1)--(v2)node[pos=0.5,above]{$\vec{v}$};
	 \end{tikzpicture}}
	\item \textbf{Định nghĩa tích vô hướng của hai véc tơ:}
	 Trong không gian, cho hai véctơ $\vec{u}$ và $\vec{v}$ khác $\vec{0}$.\\
	 Tích vô hướng của hai véctơ $\vec{u}$ và $\vec{v}$ là một số, kí hiệu $\vec{u} \cdot \vec{v}$, được xác định bởi công thức
	 \fbox{$\vec{u} \cdot \vec{v}=|\vec{u}| \cdot |\vec{v}| \cdot \cos (\vec{u}, \vec{v})$}
	 \vspace{-0.6cm}
	 \begin{note}
	 \begin{itemize}
	 \item[\ding{172}] Trong trường hợp $\vec{u}=0$ hoặc $\vec{v}=0$, ta quy ước $\vec{u} \cdot \vec{v}=0$.
	 \item[\ding{173}] $\vec{u} \cdot \vec{u}=\vec{u}^2=|\vec{u}|^2$; \quad $\vec{u}^2 \geqslant 0$. $ \vec{u}^2 = 0 \Leftrightarrow \vec{u}=\vec{0}$.
	 \item[\ding{174}] Với hai véctơ $\vec{u}$, $\vec{v}$ khác $\vec{0}$, ta có $\cos (\vec{u},\vec{v}) = \dfrac{\vec{u} \cdot \vec{v}}{|\vec{u}| \cdot |\vec{v}|}$
	 \item[\ding{175}] Với hai véctơ $\vec{u}$, $\vec{v}$ khác $\vec{0}$, ta có $\vec{u} \perp \vec{v} \Leftrightarrow \vec{u} \cdot \vec{v}= \vec{0}$.
	 \end{itemize}
	 \end{note}
	\item \textbf{TÍnh chất:} Với ba véctơ $\vec{a}$, $\vec{b}$, $\vec{c}$ và số thực $k$, ta có:
	 \begin{enumEX}[$\bullet$]{3}
	 \item $\vec{a} \cdot \vec{b}= \vec{b} \cdot \vec{a}$;
	 \item $\vec{a} \cdot \left( {\vec{b} + \vec{c}} \right) = \vec{a} \cdot \vec{b} + \vec{a} \cdot \vec{c}$;
	 \item $(k\vec{a}) \cdot \vec{b}= k(\vec{a} \cdot \vec{b}) = \vec{a} \cdot (k\vec{b})$.
	 \end{enumEX}
\end{enumerate}
\subsection{PHÂN LOẠI VÀ PHƯƠNG PHÁP GIẢI TOÁN}
\begin{dang}{Xác định véc-tơ, chứng minh đẳng thức véc tơ,độ dài véc tơ}
\end{dang}
\boxmini{BÀI TẬP TỰ LUẬN}
\setcounter{vd}{0}
\begin{vd}
	\immini{Cho hình hộp $ABCD.A'B'C'D'$. Hãy xác định các véc-tơ (khác $\vec{0}$) có điểm đầu, điểm cuối là các đỉnh của hình hộp $ABCD.A'B'C'D'$ thỏa
	\begin{tasks}(2)
	\task cùng phương với $\vec{AB}$;
	\task cùng phương $\vec{AA'}$;
	\task bằng với $\vec{AD}$;
	\task bằng với $\vec{A'B}$;
	\task đối với $\vec{CD'}$;
	\task đối với $\vec{B'C}$.
	\end{tasks}}{
	\begin{tikzpicture}[scale=0.7, font=\footnotesize,>=stealth]
	%Gán số liệu.
	\def\canhAD{3};\def\canhBA{2};\def\gocBAD{-130};\def\h{3};\def\xdinhA'{-0.5};
	%Gán tọa độ.
	\coordinate (A) at (0,0);
	\coordinate (B) at ($(A)+(\gocBAD:\canhBA)$);
	\coordinate (C) at ($(B)+(0:\canhAD)$);
	\coordinate (D) at ($(A)+(0:\canhAD)$);
	\coordinate (A') at ($(A)+(\xdinhA',\h)$);
	\coordinate (B') at ($(B)+(\xdinhA',\h)$);
	\coordinate (C') at ($(C)+(\xdinhA',\h)$);
	\coordinate (D') at ($(D)+(\xdinhA',\h)$);
	%Vẽ khối lẳng trụ ABCD.A'B'C'D'.
	\draw (A')--(B')--(B)--(C)--(C')--(D')--cycle (B')--(C') (D')--(D)--(C);
	\draw[dashed] (A)--(D) ;
	\draw[->,dashed] (A)--(C');
	\draw[->,dashed] (A)--(A');
	\draw[->, dashed] (A)--(D);
	\draw[->, dashed] (A)--(B);
	\draw[->, dashed] (A)--(C);
	%Gán nhãn.
	\foreach \x/\y in {A/180, B/180, C/0, D/0, A'/180, B'/180, C'/0, D'/0}{\fill (\x) circle(1pt) ($(\x)+(\y:0.3cm)$) node{$\x$};}
	\end{tikzpicture}}
\end{vd}
\dongcham{3}
\begin{vd}
	Cho hình chóp $S . A B C D$ có đáy $A B C D$ là hình bình hành. Gọi $M$, $N$, $O$ lần lượt là trung điểm của $A B, C D$ và $AC$. Chứng minh rằng
	\begin{listEX}[3]
	\item $\vec{B N}$ và $\vec{D M}$ đối nhau;
	\item $\vec{SA}+\vec{SB}+\vec{SC}+\vec{SD}=4\vec{SO}$;
	\item $\vec{S D}-\vec{B N}-\vec{C M}=\vec{S C}$.
	\end{listEX}
	\loigiai{
	\immini{\vspace*{-3mm}
	\begin{enumerate}
	\item Tứ giác $A B C D$ là hình bình hành nên $A B=C D$ và $A B\parallel C D$, suy ra $B M=DN$ và $B M \parallel D N$.\\
	 Do đó $BMDN$ là hình bình hành.\\
	 Hai véc-tơ $\vec{BN}$ và $\vec{DN}$ có cùng độ dài và ngược hướng nên chúng là hai véc-tơ đối nhau.
	\item Ta có $\vec{SA}+\vec{SC}=2\vec{SO}$; $\vec{SB}+\vec{SD}=2\vec{SO}$. Suy ra
	 $$\vec{SA}+\vec{SB}+\vec{SC}+\vec{SD}=4\vec{SO}.$$
	\item Từ câu a, ta có $\vec{BN}=-\vec{DM}$.\\
	 Suy ra $\vec{S D}-\vec{B N}-\vec{C M}=\vec{S D}+\vec{DM}-\vec{CM}=\vec{SM}+\vec{MC}=\vec{S C}$.
	\end{enumerate}
	}{
	\begin{tikzpicture}[line join=round, line cap = round, >=stealth, scale=.9,font=\footnotesize]
	\def\a{4}
	\path 	(0:0) coordinate (A)
	++(0:\a) coordinate (D)
	++(-130:\a/2) coordinate (C)
	($(A)+(C)-(D)$) coordinate (B)
	($(A)+(80:0.7*\a)$) coordinate (S)
	(intersection of A--C and B--D) coordinate (O)
	($(A)!0.5!(B)$) coordinate (M)
	($(C)!0.5!(D)$) coordinate (N)
	($(A)!0.5!(C)$) coordinate (O)
	;%giao điểm O
	\draw[dashed] 	(B)--(A)--(D)	(A)--(S) (A)--(C) (B)--(D);
	\draw 	(B)-- (C)--(D)
	(B)--(S)	(C)--(S)	(D)--(S);
	\foreach \x/\g in {A/135,B/-135,C/-45,D/45,S/90,M/-50,N/-30,O/-90}
	\fill[black] 	(\x) circle (1pt)
	($(\g:3mm)+(\x)$) node {$\x$};
	\draw [dashed] (D)--(M) (B)--(N);
	\end{tikzpicture}}
	}
\end{vd}
\dongcham{16}
\begin{vd}
	Cho hình lập phương $A B C D . A' B' C' D'$ cạnh bằng $a$. Gọi $G$ là trọng tâm tam giác $AB'D'$.
	\begin{tasks}(2)
	\task Tìm vectơ: $\vec{C C'}+\vec{B A}$; \quad $\vec{C C'}+\vec{B A}+\vec{D' A'}$.
	\task Chứng minh: $\vec{B C}+\vec{D C}+\vec{A A'}=\vec{A C'}$.
	\task Chứng minh: $\vec{B'B} + \vec{AD} + \vec{CD} = \vec{B'D}$.
	\task Chứng minh: $\vec{BB'} - \vec{C'B'} - \vec{D'C'} = \vec{BD'}$.
	\task Chứng minh: $\vec{A'C} = 3\vec{A'G}$.
	\task Tính độ dài véc tơ $\vec{u}= \vec{AB}+\vec{A'D'}+\vec{AA'}$.
	\end{tasks}
	\loigiai{
	\begin{enumerate}[a)]
	\immini{
	\item Vì $A B C D . A' B' C' D'$ là hình hộp nên $\vec{B A}=\vec{C D}$ và $\vec{D' A'}=\vec{C B}$.\\
	 Suy ra $\vec{C C'}+\vec{B A}+\vec{D' A'}=\vec{C C'}+\vec{C D}+\vec{C B}=\vec{C A'}$.
	\item Vì tứ giác $A B C D$ là hình bình hành nên $\vec{B C}=\vec{A D}$ và $\vec{D C}=\vec{A B}$. Áp dụng quy tắc hình hộp suy ra $$\vec{B C}+\vec{D C}+\vec{A A'}=\vec{A D}+\vec{A B}+\vec{A A'}=\vec{A C'}$$
	\item Ta có $\vec{AD} = \vec{B'C'}$, $\vec{CD} = \vec{B'A'}$. Do đó
	 $$\vec{B'B} + \vec{AD} + \vec{CD} = \vec{B'B} + \vec{B'C'} + \vec{B'A'} = \vec{B'D}.$$
	\item Ta có \begin{eqnarray*}
	 \vec{BB'} - \vec{C'B'} - \vec{D'C'} &=& \vec{BB'} - \left( \vec{D'C'} + \vec{C'B'} \right)
	 = \vec{BB'} - \vec{D'B'} \\
	 &=& \vec{BB'} + \left( - \vec{D'B'}\right)
	 = \vec{BB'} + \vec{B'D'}= \vec{BD'}.
	 \end{eqnarray*}
	 }{
	 \begin{tikzpicture}[scale=0.8, font=\footnotesize, line join=round, line cap=round]
	 \def\h{4}
	 \foreach \x\y\t in {0/0/A,-1/-1.1/B,2.6/-1.1/C}
	 \coordinate (\t) at (\x,\y);
	 \coordinate (D) at ($(A)+(C)-(B)$);
	 \coordinate (A') at ($(A)+(0,3.2)$);
	 \coordinate (B') at ($(B)+(0,3.2)$);
	 \coordinate (C') at ($(C)+(0,3.2)$);
	 \coordinate (D') at ($(D)+(0,3.2)$);
	 \draw (B')--(A')--(D')--(C')--(B')--(B)--(C)--(D)--(D') (C')--(C);
	 \draw[dashed](B)--(A)--(D) (A)--(A');
	 \foreach \t/\g in {A/170,B/-150,C/-70,D/0,A'/100,B'/170,C'/-20,D'/50}
	 \draw[fill=black] (\t) circle(1pt)
	 node[shift={(\g:7pt)}]{$\t$};
	 \end{tikzpicture}
	 }
	\item Do $G$ là trọng tâm tam giác $AB'D'$ nên $\vec{GA} + \vec{GB'} + \vec{GD'} = \vec{0}$. Khi đó, theo quy tắc hình hộp ta có
	 \begin{eqnarray*}
	 & \vec{A'C} & = \vec{A'A} + \vec{A'B'} + \vec{A'D'}\\
	 & & = \vec{A'G} + \vec{GA} + \vec{A'G} + \vec{GB'} + \vec{A'G} + \vec{GD'}\\
	 & & = 3\vec{A'G}.
	 \end{eqnarray*}
	\item Ta có $\vec{u}= \vec{AB}+\vec{A'D'}+\vec{AA'}=\vec{AB}+\vec{AD}+\vec{AA'}=\vec{AC'}$. Suy ra
	 $\big|\vec{u}\big|=AC'=a\sqrt{3}.$
	\end{enumerate}
	}
\end{vd}
\dongcham{34}
\begin{vd}%[2H2H1-4]
	\immini{
	Ba lực $\vec{F_1}$, $\vec{F_2}$, $\vec{F_3}$ cùng tác động vào một vật có phương đôi một vuông góc nhau và có độ lớn lần lượt là $2 \mathrm{\,N}$, $3 \mathrm{\,N}$, $4 \mathrm{\,N}$.
	\begin{tasks}
	\task Tính độ lớn hợp lực của $\vec{F_2}$, $\vec{F_3}$.
	\task Tính độ lớn hợp lực của ba lực đã cho.
	\end{tasks}}
	{
	\begin{tikzpicture}[scale=1.3, font=\footnotesize, line join=round, line cap=round]
	\foreach \x\y\t in {0/0/O,0/1/a,1.3/0/b,-1.2/-1/c}
	\coordinate (\t) at (\x,\y);
	\foreach \a\b in {O/a,O/b,O/c}
	\draw[-{Stealth[length=2mm]}] (\a)--(\b);
	\path (O)--(a) node[pos=0.5,left]{$\vec{F_1}$}
	(O)--(b) node[pos=0.5,above]{$\vec{F_2}$}
	(O)--(c) node[pos=0.5,above left]{$\vec{F_3}$};
	\path
	pic[draw,angle radius=4]{right angle=a--O--b}
	pic[draw,angle radius=4]{right angle=c--O--b}
	pic[draw,angle radius=4]{right angle=c--O--a};
	\end{tikzpicture}}
	\loigiai{
	\immini{
	\begin{enumerate}[a)]
	\item Gọi $O$ là vị trí trên vật mà ba lực cùng tác động vào. Gọi$A$, $B$, $C$ là các điểm sao cho $\vec{F_1}=\vec{OA}$, $\vec{F_2}=\vec{OB}$, $\vec{F_3}=\vec{OC}$. Khi đó
	 $$\left|\vec{F_2}+\vec{F_3}\right|=OE=\sqrt{3^2+4^2}=5 \text{N}.$$
	\item Dựng các hình chữ nhật $OBEC$ và $OEFA$ thì ta có
	 $$\heva{&\vec{OB}+\vec{OC}=\vec{OE}\\&\vec{OA}+\vec{OE}=\vec{OF}.}$$
	 Do đó $\vec{F_1}+\vec{F_2}+\vec{F_3}=\vec{OA}+\vec{OB}+\vec{OC}=\vec{OA}+\vec{OE}=\vec{OF}.$\\
	 Vậy độ lớn hợp lực của $F_1$, $\vec{F_2}$ và $\vec{F_3}$ là
	 $$\begin{aligned}
	 \left|\vec{F_1}+\vec{F_2}+\vec{F_3}\right|=OF
	 & =\sqrt{OA^2+OE^2} \\
	 & =\sqrt{OA^2+OB^2+OC^2} \\
	 & =\sqrt{2^2+3^2+4^2}=\sqrt{29} \mathrm{\,N}.
	 \end{aligned}$$
	\end{enumerate}
	}
	{
	\begin{tikzpicture}[scale=1.8, font=\footnotesize, line join=round, line cap=round]
	\foreach \x\y\t in {0/0/O,0/1/A,1.3/0/B,-0.9/-1.2/C}
	\coordinate (\t) at (\x,\y);
	\coordinate (E) at ($(B)+(C)$);
	\coordinate (F) at ($(A)+(E)$);
	\foreach \a\b in {O/A,O/B,O/C,O/F,O/E}
	\draw[-{Stealth[length=2mm]}] (\a)--(\b);
	\path
	(O)--(A) node[pos=0.5,left]{$\vec{F_1}$}
	(O)--(B) node[pos=0.5,above]{$\vec{F_2}$}
	(O)--(C) node[pos=0.5,above left]{$\vec{F_3}$};
	\path
	pic[draw,angle radius=4]{right angle=A--O--B}
	pic[draw,angle radius=4]{right angle=C--O--B}
	pic[draw,angle radius=4]{right angle=C--O--A};
	\foreach \t\g in {A/90, B/0, C/180,E/-80,F/0,O/180}
	\draw[fill=black] (\t)circle(0.1pt) +(\g:4pt)node{$\t$};
	\draw[dashed] (C)--(E)--(B) (A)--(F)--(E);
	\end{tikzpicture}}
	}
\end{vd}
\dongcham{23}
\boxmini{BÀI TẬP TRẮC NGHIỆM}
\textbf{PHẦN I.} \textit{Câu trắc nghiệm nhiều phương án lựa chọn. Mỗi câu hỏi học sinh chỉ chọn một phương án.}\\
\setcounter{ex}{0}
\Opensolutionfile{ans}[ans/2H2-B1-d1-1]
%%==========Câu 1
\begin{ex}%[1H3B1-1]
	\immini{Cho hình hộp $ABCD.EFGH$. Các véc-tơ có điểm đầu và điểm cuối là các đỉnh của hình hộp và bằng véc-tơ $\vec{AB}$ là các véc-tơ nào sau đây?
	\choice
	{$\vec{CD}$, $\vec{HG}$, $\vec{EF}$}
	{\True $\vec{DC}$, $\vec{HG}$, $\vec{EF}$}
	{$\vec{DC}$, $\vec{HG}$, $\vec{FE}$}
	{$\vec{DC}$, $\vec{GH}$, $\vec{EF}$}}{
	\begin{tikzpicture}[scale=0.75, font=\footnotesize, line join=round, line cap=round]
	\foreach \x\y\t in {0/0/A,-0.8/-1.1/B,2.8/-1.1/C}
	\coordinate (\t) at (\x,\y);
	\coordinate (D) at ($(A)+(C)-(B)$);
	\coordinate (E) at ($(A)+(-0.5,2.5)$);
	\coordinate (F) at ($(B)+(E)-(A)$);
	\coordinate (G) at ($(C)+(E)-(A)$);
	\coordinate (H) at ($(D)+(E)-(A)$);
	\foreach \a\b in {A/B, A/D,A/E}
	\draw[dashed] (\a)--(\b);
	\foreach \a\b in {C/D,C/B,C/G}
	\draw[] (\a)--(\b);
	\draw (F)--(E)--(H)--(G)--(F)--(B) (D)--(H);
	\foreach \t/\g in {A/170,B/-130,C/-60,D/0,E/90,F/180,G/-20,H/70}
	\draw[fill=black] (\t) circle(1pt)
	node[shift={(\g:7pt)}]{$\t$};
	\end{tikzpicture}}
	\loigiai{
	Các véc-tơ bằng với véc-tơ $\vec{AB}$ là $\vec{DC}$, $\vec{HG}$, $\vec{EF}$
	}
\end{ex} \dongcham{2}
%%==========Câu 2
\begin{ex}%[1H3B1-2]
	\immini{Cho hình hộp $ABCD.A'B'C'D'.$ Trong các khẳng định sau, khẳng định nào \textbf{sai}?
	\choice
	{$\vec{AB}+\vec{B'D'}=\vec{AD}$}
	{$\vec{AB}+\vec{CD}=\vec{0}$}
	{$\vec{AC'}+\vec{A'C}=2\vec{AC}$}
	{\True $\vec{AC}-\vec{D'D}=\vec{0}$}}{
	\begin{tikzpicture}[scale=0.55, font=\footnotesize, line join=round, line cap=round]
	\foreach \x\y\t in {0/0/A,-2/-1.5/B,3.9/0/D,-0.5/3.5/A'}
	\coordinate (\t) at (\x,\y);
	\coordinate (C) at ($(B)+(D)-(A)$);
	\coordinate (B') at ($(A')+(B)-(A)$);
	\coordinate (C') at ($(B')+(C)-(B)$);
	\coordinate (D') at ($(C')+(D)-(C)$);
	\draw (A')--(B')--(B)--(C)--(C');
	\draw (A')--(D')--(D);
	\draw (D')--(C') (C)--(D);
	\draw (B')--(C') (D)--(D');
	\draw[dashed] (A)--(B) (A')--(A)--(D);
	\foreach \t/\g in {A/180,B/180,C/0,D/0,A'/180,B'/180,C'/0,D'/0} \draw (\t) node[shift={(\g:10pt)}]{$\t$};
	\end{tikzpicture}}
	\loigiai{
	\immini{
	\begin{itemize}
	\item [$\bullet$] $\vec{AB}+\vec{B'D'}=\vec{AB}+\vec{BD}=\vec{AD}$'
	\item [$\bullet$] $\vec{AB}$ và $\vec{CD}$ đối nhau nên $\vec{AB}+\vec{CD}=\vec{0}$.
	\item [$\bullet$] Theo quy tắc hình bình hành ta có\\ $\vec{AC'}+\vec{A'C}=\vec{AC}+\vec{AA'}+\vec{A'A}+\vec{A'C'}=2\cdot\vec{AC}.$
	\item [$\bullet$] $\vec{AC}-\vec{D'D}=\vec{AC}+\vec{CC'}=\vec{AC'}$
	\end{itemize}
	}
	{\begin{tikzpicture}[scale=0.55, font=\footnotesize, line join=round, line cap=round]
	\foreach \x\y\t in {0/0/A,-2/-1.5/B,3.9/0/D,-0.5/3.5/A'}
	\coordinate (\t) at (\x,\y);
	\coordinate (C) at ($(B)+(D)-(A)$);
	\coordinate (B') at ($(A')+(B)-(A)$);
	\coordinate (C') at ($(B')+(C)-(B)$);
	\coordinate (D') at ($(C')+(D)-(C)$);
	\draw (A')--(B')--(B)--(C)--(C');
	\draw (A')--(D')--(D);
	\draw (D')--(C') (C)--(D);
	\draw (B')--(C') (D)--(D');
	\draw[dashed] (A)--(B) (A')--(A)--(D) (A)--(C') (A')--(C);
	\foreach \t/\g in {A/180,B/180,C/0,D/0,A'/180,B'/180,C'/0,D'/0} \draw (\t) node[shift={(\g:10pt)}]{$\t$};
	\end{tikzpicture}}
	}
\end{ex} \dongcham{8}
%%==========Câu 3
\begin{ex}
	\immini{Cho hình lập phương $ ABCD.A'B'C'D'$ cạnh $ a$. Khẳng định nào sau đây là khẳng định \textbf{sai}?
	\choice
	{$\big|\vec{AC}\big|=a\sqrt{2}$}
	{$\big|\vec{AC'}\big|=a\sqrt{3}$}
	{$\vec{BD}+\vec{D'B'}=\vec{0}$}
	{\True $\vec{BA}+\vec{BC}+\vec{BB'}=\vec{BC'}$}
	}{
	\begin{tikzpicture}[scale=0.55, font=\footnotesize, line join=round, line cap=round]
	\foreach \x\y\t in {0/0/A,-2/-1.5/B,3.9/0/D,0/3.5/A'}
	\coordinate (\t) at (\x,\y);
	\coordinate (C) at ($(B)+(D)-(A)$);
	\coordinate (B') at ($(A')+(B)-(A)$);
	\coordinate (C') at ($(B')+(C)-(B)$);
	\coordinate (D') at ($(C')+(D)-(C)$);
	\draw (A')--(B')--(B)--(C)--(C');
	\draw (A')--(D')--(D);
	\draw (D')--(C') (C)--(D);
	\draw (B')--(C') (D)--(D');
	\draw[dashed] (A)--(B) (A')--(A)--(D) (C)--(A)--(C');
	\foreach \t/\g in {A/180,B/180,C/0,D/0,A'/180,B'/180,C'/0,D'/0} \draw (\t) node[shift={(\g:10pt)}]{$\t$};
	\end{tikzpicture}
	}
	\loigiai{
	}
\end{ex} \dongcham{8}
%%==========Câu 4
\begin{ex}%[1H3B1-3]
	\immini{Cho hình lập phương $ABCD.A'B'C'D'$. Gọi $O$ là tâm của hình lập phương. Khẳng định nào dưới đây là đúng?
	\choice
	{$\vec{AO}=\dfrac{1}{3}\left(\vec{AB}+\vec{AD}+\vec{AA'}\right)$}
	{\True $\vec{AO}=\dfrac{1}{2}\left(\vec{AB}+\vec{AD}+\vec{AA'}\right)$}
	{$\vec{AO}=\dfrac{1}{4}\left(\vec{AB}+\vec{AD}+\vec{AA'}\right)$}
	{$\vec{AO}=\dfrac{2}{3}\left(\vec{AB}+\vec{AD}+\vec{AA'}\right)$}}{
	\begin{tikzpicture}[scale=0.55, font=\footnotesize, line join=round, line cap=round]
	\foreach \x\y\t in {0/0/A,-2/-1.5/B,3.9/0/D,0/3.5/A'}
	\coordinate (\t) at (\x,\y);
	\coordinate (C) at ($(B)+(D)-(A)$);
	\coordinate (B') at ($(A')+(B)-(A)$);
	\coordinate (C') at ($(B')+(C)-(B)$);
	\coordinate (D') at ($(C')+(D)-(C)$);
	\coordinate (O) at ($(A)!0.5!(C')$);
	\draw (A')--(B')--(B)--(C)--(C');
	\draw (A')--(D')--(D);
	\draw (D')--(C') (C)--(D);
	\draw (B')--(C') (D)--(D');
	\draw[dashed] (A)--(B) (A')--(A)--(D) (A)--(C') (A')--(C);
	\foreach \t/\g in {A/180,B/180,C/0,D/0,A'/180,B'/180,C'/0,D'/0,O/-100} \draw (\t) node[shift={(\g:10pt)}]{$\t$};
	\end{tikzpicture}}
	\loigiai{\vspace{-0.5cm}
	\immini{
	Theo quy tắc hình hộp, ta có $\vec{AC'}=\vec{AB}+\vec{AD}+\vec{AA'}$. \\
	Mà $O$ là trung điểm của $AC'$\\
	nên $\vec{AO}=\dfrac{1}{2}\vec{AC'}=\dfrac{1}{2}\left(\vec{AB}+\vec{AD}+\vec{AA'}\right)$.}
	{\vspace{-0.5cm}
	\begin{tikzpicture}[scale=0.55, font=\footnotesize, line join=round, line cap=round]
	\foreach \x\y\t in {0/0/A,-2/-1.5/B,3.9/0/D,0/3.5/A'}
	\coordinate (\t) at (\x,\y);
	\coordinate (C) at ($(B)+(D)-(A)$);
	\coordinate (B') at ($(A')+(B)-(A)$);
	\coordinate (C') at ($(B')+(C)-(B)$);
	\coordinate (D') at ($(C')+(D)-(C)$);
	\coordinate (O) at ($(A)!0.5!(C')$);
	\draw (A')--(B')--(B)--(C)--(C');
	\draw (A')--(D')--(D);
	\draw (D')--(C') (C)--(D);
	\draw (B')--(C') (D)--(D');
	\draw[dashed] (A)--(B) (A')--(A)--(D) (A)--(C') (A)--(C);
	\foreach \t/\g in {A/180,B/180,C/0,D/0,A'/180,B'/180,C'/0,D'/0,O/-90} \draw (\t) node[shift={(\g:10pt)}]{$\t$};
	\end{tikzpicture}}}
\end{ex} \dongcham{8}
%%==========Câu 5
\begin{ex}%[1H3B1-2]%
	Cho hình lập phương $ ABCD.A'B'C'D'$ cạnh $ a$. Tính độ dài vectơ $\vec x=\vec{AB'}+\vec{AD'}$ theo $ a$.
	\choice
	{$\left|\vec x\right|=a\sqrt 2 $}
	{$\left|\vec x\right|=2a\sqrt 2 $}
	{$\left|\vec x\right|=2a\sqrt 6 $}
	{\True $\left|\vec x\right|=a\sqrt 6 $}
	\loigiai{
	\immini{Ta có $\vec x=\vec{AB'}+\vec{AD'}=2\vec{AI}$, với $ I$ là trung điểm của $ B'D'$. Khi đó $\left|\vec x\right|=2AI$.\\
	Do tam giác $ AB'D'$ đều cạnh $ a\sqrt 2 $ nên $ AI=\dfrac{a\sqrt 6}{2}$. \\
	Vậy $\left|\vec x\right|=a\sqrt 6 $.}
	{\begin{tikzpicture}[scale=1, font=\footnotesize, line join=round, line cap=round, >=stealth]
	\def\bc{4} % cạnh BC
	\def\ba{2} % cạnh BA
	\def\gocB{35} % góc B của đáy
	\coordinate[label=below left:$B$] (B) at (0,0);
	\coordinate[label=above left:$A$] (A) at (\gocB:\ba);
	\coordinate[label=below:$C$] (C) at (\bc,0);
	\coordinate[label=right:$D$] (D) at ($(C)-(B)+(A)$);
	\coordinate[label=above left:$A'$] (A') at ($(A)+(90:\bc)$);
	\coordinate[label=left:$B'$] (B') at ($(B)-(A)+(A')$);
	\coordinate[label=below right:$C'$] (C') at ($(C)-(A)+(A')$);
	\coordinate[label=right:$D'$] (D') at ($(D)-(A)+(A')$);
	\tkzDefMidPoint(B',D') \tkzGetPoint{I}
	\tkzLabelPoints[above](I);
	\draw (B')--(B)--(C)--(D)--(D')--(A')--(B')--(C')--(D') (C)--(C') (B')--(D');
	\draw[dashed] (A')--(A)--(D) (A)--(B) (A)--(B') (A)--(D') (A)--(I);
	\foreach \diem in {A,B,C,D,A',B',C',D',I}\fill (\diem)circle(1.5pt);
	\end{tikzpicture}}
	}
\end{ex} \dongcham{8}
%%==========Câu 6
\begin{ex}%[1H3B1-3]
	\immini{Hình lập phương $ABCD.A'B'C'D'$ cạnh $a$. Tính độ dài véctơ $\vec{x}=\vec{AA'}+\vec{AC'}$ theo~$a$.
	\haicot
	{$a\sqrt{2}$}
	{$\left(1+\sqrt{3}\right)a$}
	{\True $a\sqrt{6}$}
	{$\dfrac{a\sqrt{6}}{2}$}}{\hspace{1cm}
	\begin{tikzpicture}[scale=0.7, line join=round, line cap=round]
	\tikzset{label style/.style={font=\footnotesize}}
	\tkzDefPoints{0/0/A,-1.3/-1.1/B,2/-1.1/C}
	\coordinate (D) at ($(A)+(C)-(B)$);
	\coordinate (A') at ($(A)+(0,2.5)$);
	\tkzDefPointsBy[translation=from A to A'](B,C,D){B'}{C'}{D'}
	\tkzDrawPolygon(A',B',B,C,D,D')
	\tkzDrawSegments(B',C' C',D' C,C')
	\tkzDrawSegments[dashed](A,B A,D A,A')
	\tkzDrawPoints[fill=black,size=4](A,B,D,C,A',B',C',D')
	\tkzLabelPoints[above](A',D')
	\tkzLabelPoints[below](A,B,C)
	\tkzLabelPoints[left](B')
	\tkzLabelPoints[right](C',D)
	\end{tikzpicture}}
	\loigiai{
	\immini{
	Gọi $O'$ là tâm $A'B'C'D'\Rightarrow A'O'=\dfrac{a\sqrt{2}}{2}$.\\
	Ta có $\vec{AA'}+\vec{AC'}=2\vec{AO'}\Rightarrow \vert \vec{x} \vert =2\left| \vec{AO'} \right| =2AO'$.\\
	$\triangle AA'O'$ vuông tại $A'\Rightarrow AO'=\sqrt{AA'^2+A'O'^2}=\dfrac{a\sqrt{6}}{2}$.\\
	Vậy $\vert \vec{x} \vert =2AO'=a\sqrt{6}$.
	}{
	\begin{tikzpicture}[scale=.5, line join=round, line cap=round,>=stealth]
	\tkzDefPoints{0/0/B,3/2/A,10/2/D,7/0/C,0/7/B',3/9/A',7/7/C',10/9/D'}
	\tkzInterLL(A',C')(B',D')\tkzGetPoint{O'}
	\tkzDrawSegments[](A',B' B',C' C',D' D',A' B,C C,D B,B' C,C' D,D' A',C' B',D')
	\tkzDrawSegments[dashed](A,A' A,B A,D A,C' A,O')
	\tkzLabelPoints[above](A',D',O')
	\tkzLabelPoints[above left](A,B')
	\tkzLabelPoints[below left](B)
	\tkzLabelPoints[below right](C,C')
	\tkzLabelPoints[right](D)
	\tkzDrawPoints[fill=black](A,B,C,D,A',B',C',D',O')
	\end{tikzpicture}
	}
	}
\end{ex} \dongcham{8}
%%==========Câu 7
\begin{ex}%[Trần Toàn]%[1H3Y1-2]%
	\immini{Cho tứ diện $ABCD$. Mệnh đề nào dưới đây là mệnh đề đúng?
	\choice
	{$\vec {AB}-\vec {AD}=\vec {CD}+\vec {BC}$}
	{$\vec {AC}-\vec {AD}=\vec {BD}-\vec {BC}$}
	{$\vec {BC}+\vec {AB}=\vec {DA}-\vec {DC}$}
	{\True $\vec {AB}-\vec {AC}=\vec {DB}-\vec {DC}$}}{
	\begin{tikzpicture}[scale=0.8, font=\footnotesize,>=stealth]
	\path
	(0,0) coordinate (A)
	(5,0) coordinate (C)
	(1.2,-1.5) coordinate (B)
	($(B)!0.5!(C)$)coordinate (M)
	($(A)!2/3!(M)$)coordinate (G)
	($(G)+(0,3)$)coordinate (D)
	;
	\draw (D)--(A)--(B)--(D)--(C)--(B);
	\draw[dashed] (A)--(C);
	\foreach \x/\g in {A/180,B/-90,C/0,D/90}\draw[fill=black] (\x) circle (.04) +(\g:.4)node{\footnotesize$\x$};
	\end{tikzpicture}}
	\loigiai{
	Ta có $\vec{AB}-\vec{AC}=\vec{CB}=\vec{DB}-\vec{DC}$.
	}
\end{ex} \dongcham{6}
%%==========Câu 8
\begin{ex}%[1H3B1-2]%
	\immini{Cho tứ diện $ABCD$. Gọi $ G$ là trọng tâm tam giác $ABC$. Tìm $ k$ thỏa đẳng thức vectơ $\vec{DA}+\vec{DB}+\vec{DC}=k\cdot\vec{DG}$.
	\haicot
	{$ k=1$}
	{$ k=3$}
	{$ k=2$}
	{\True $ k=3$}}{
	\begin{tikzpicture}[scale=0.8, font=\footnotesize,>=stealth]
	\path
	(0,0) coordinate (A)
	(5,0) coordinate (C)
	(1.2,-1.5) coordinate (B)
	($(B)!0.5!(C)$)coordinate (M)
	($(A)!2/3!(M)$)coordinate (G)
	($(G)+(0,3)$)coordinate (D)
	;
	\draw (D)--(A)--(B)--(D)--(C)--(B);
	\draw[dashed] (A)--(C) (D)--(G);
	\foreach \x/\g in {A/180,B/-90,C/0,D/90,G/0}\draw[fill=black] (\x) circle (.04) +(\g:.4)node{\footnotesize$\x$};
	\end{tikzpicture}}
	\loigiai{
	\immini{
	$\vec{DA}+\vec{DB}+\vec{DC}=\vec{DG}+\vec{GA}+\vec{DG}+\vec{GB}+\vec{DG}+\vec{GC}=3\vec{DG}$.}
	{\begin{tikzpicture}[scale=1, font=\footnotesize,>=stealth]
	\path
	(0,0) coordinate (A)
	(4,0) coordinate (C)
	(1.5,-1.5) coordinate (B)
	($(B)!0.5!(C)$)coordinate (M)
	($(A)!2/3!(M)$)coordinate (G)
	($(G)+(0,3)$)coordinate (D)
	;
	\draw (D)--(A)--(B)--(D)--(C)--(B);
	\draw[dashed] (A)--(C) (D)--(G) (B)--(G)--(A) (G)--(C);
	\foreach \x/\g in {A/180,B/-90,C/0,D/90,G/0}\draw[fill=black] (\x) circle (.04) +(\g:.4)node{\footnotesize$\x$};
	\end{tikzpicture}
	}
	}
\end{ex} \dongcham{8}
%%==========Câu 9
\begin{ex}%[1H3B1-3]
	\immini{Cho hình lăng trụ $ABC.A'B'C'$. Gọi $G'$ là trọng tâm của tam giác $A'B'C'$. Đặt $\vec{a}=\vec{AA'}, \vec{b}=\vec{AB}, \vec{c}=\vec{AC}$. Véc-tơ $\vec{AG'}$ bằng
	\choice
	{$\dfrac{1}{3}\left(\vec{a}+3\vec{b}+\vec{c}\right)$}
	{\True $\dfrac{1}{3}\left(3\vec{a}+\vec{b}+\vec{c}\right)$}
	{$\dfrac{1}{3}\left(\vec{a}+\vec{b}+3\vec{c}\right)$}
	{$\dfrac{1}{3}\left(\vec{a}+\vec{b}+\vec{c}\right)$}}{
	\begin{tikzpicture}[scale=0.8, font=\footnotesize,>=stealth]
	\path
	(0,0) coordinate (A)
	(4,0) coordinate (C)
	(1.5,-1.5) coordinate (B)
	($(A)+(0.4,3)$)coordinate (A')
	($(B)+(0.4,3)$)coordinate (B')
	($(C)+(0.4,3)$)coordinate (C')
	($(B')!0.5!(C')$)coordinate (I)
	($(A')!2/3!(I)$)coordinate (G')
	;
	\draw (B)--(C)--(C')--(B')--(B)--(A)--(A')--(B') (I)--(A')--(C');
	\draw[dashed] (A)--(C);
	\foreach \x/\g in {A/180,B/-45,C/0,A'/180,B'/190,C'/0,G'/-90}\draw[fill=black] (\x) circle (.04) +(\g:.4)node{\footnotesize$\x$};
	\end{tikzpicture}}
	\loigiai{\vspace{-0.5cm}
	\immini{
	Gọi $I$ là trung điểm của $B'C'$. \\
	Vì $G'$ là trọng tâm của tam giác $A'B'C' \Rightarrow \vec{A'G'}=\dfrac{2}{3}\vec{A'I}$. \\
	$\begin{aligned}
	\text{Ta có} \vec{AG'} & =\vec{AA'}+\vec{A'G'}=\vec{AA'}+\dfrac{2}{3}\vec{A'I} \\
	 & =\vec{AA'}+\dfrac{1}{3}\left(\vec{A'B'}+\vec{A'C'}\right) \\
	 & =\vec{AA'}+\dfrac{1}{3}\left(\vec{AB}+\vec{AC}\right) \\
	 & =\dfrac{1}{3}\left(3\vec{AA'}+\vec{AB}+\vec{AC}\right)=\dfrac{1}{3}\left(3\vec{a}+\vec{b}+\vec{c}\right).
	\end{aligned}$}
	{\begin{tikzpicture}[scale=0.7, font=\footnotesize,>=stealth]
	\path
	(0,0) coordinate (A)
	(4,0) coordinate (C)
	(1.5,-1.5) coordinate (B)
	($(A)+(0.4,3)$)coordinate (A')
	($(B)+(0.4,3)$)coordinate (B')
	($(C)+(0.4,3)$)coordinate (C')
	($(B')!0.5!(C')$)coordinate (I)
	($(A')!2/3!(I)$)coordinate (G')
	;
	\draw (B)--(C)--(C')--(B')--(B)--(A)--(A')--(B') (I)--(A')--(C');
	\draw[dashed] (A)--(C);
	\foreach \x/\g in {A/180,B/-45,C/0,A'/180,B'/190,C'/0,G'/-90,I/-90}\draw[fill=black] (\x) circle (.04) +(\g:.4)node{\footnotesize$\x$};
	\end{tikzpicture}}}
\end{ex} \dongcham{12}
%%==========Câu 10
\begin{ex}%[1H3B1-2]
	\immini{Cho hình chóp $S.ABCD$ có đáy $ABCD$ là hình bình hành. Đặt $\vec{SA}=\vec{a}$, $\vec{SB}=\vec{b}$, $\vec{SC}=\vec{c}$, $\vec{SD}=\vec{d}$. Khẳng định nào dưới đây là đúng?
	\choice
	{\True $\vec{a}+\vec{c}=\vec{b}+\vec{d}$}
	{$\vec{a}+\vec{b}+\vec{c}+\vec{d}=\vec{0}$}
	{$\vec{a}+\vec{d}=\vec{b}+\vec{c}$}
	{$\vec{a}+\vec{b}=\vec{c}+\vec{d}$}}{
	\begin{tikzpicture}[scale=0.8, font=\footnotesize,>=stealth]
	\path
	(0,0) coordinate (A)
	(-1.6,-1.5) coordinate (B)
	(5,0) coordinate (D)
	($(B)+(D)-(A)$)coordinate (C)
	($(A)!1/2!(C)$)coordinate (O)
	($(A)+(0.4,3)$)coordinate (S)
	;
	\draw (C)--(D)--(S)--(C)--(B)--(S);
	\draw[dashed] (D)--(A)--(B) (S)--(A);
	\foreach \x/\g in {A/160,B/-90,C/-90,D/0,S/90}\draw[fill=black] (\x) circle (.04) +(\g:.4)node{\footnotesize$\x$};
	\end{tikzpicture}}
	\loigiai{
	\immini{
	Gọi $O$ là tâm hình bình hành $ABCD$. \\
	Vì $O$ là trung điểm của $AC$\\ \indent\qquad
	nên $\vec{SA}+\vec{SC}=2\vec{SO} \Leftrightarrow 2\vec{SO}=\vec{a}+\vec{c}$. \hfill (1) \\
	Và $O$ là trung điểm của $BD$\\\indent\qquad
	nên $\vec{SB}+\vec{SD}=2\vec{SO} \Leftrightarrow 2\vec{SO}=\vec{b}+\vec{d}$.\hfill (2)\\
	Từ $(1)$ và $(2)$, suy ra $\vec{a}+\vec{c}=\vec{b}+\vec{d}$.}
	{\begin{tikzpicture}[scale=0.5, font=\footnotesize,>=stealth]
	\path
	(0,0) coordinate (A)
	(-1.6,-1.5) coordinate (B)
	(5,0) coordinate (D)
	($(B)+(D)-(A)$)coordinate (C)
	($(A)!1/2!(C)$)coordinate (O)
	($(A)+(0.4,3)$)coordinate (S)
	;
	\draw (C)--(D)--(S)--(C)--(B)--(S);
	\draw[dashed] (D)--(A)--(B)--(D) (S)--(A)--(C) (O)--(S);
	\foreach \x/\g in {A/160,B/-90,C/-90,D/0,S/90}\draw[fill=black] (\x) circle (.04) +(\g:.4)node{\footnotesize$\x$};
	\end{tikzpicture}}}
\end{ex} \dongcham{15}
%%==========Câu 11
\begin{ex}
	\immini{
	Cho tứ diện $ABCD$. Các vectơ có điểm đầu là $A$ và điểm cuối là các đỉnh còn lại của hình tứ diện là
	\choice
	{$\vec{AB},\vec{CA},\vec{AD}$}
	{$\vec{BA},\vec{AC},\vec{AD}$}
	{$\vec{AB},\vec{AC},\vec{DA}$}
	{\True $\vec{AB},\vec{AC},\vec{AD}$}
	}{\begin{tikzpicture}[line join = round, line cap = round, thick, font = \small, scale = .6]
	\path
	(0:0) coordinate (B)
	+(0:5) coordinate (C)
	+(-70:3) coordinate (D)
	++(90:4) coordinate (A)
	;
	\draw[dashed]
	(B)--(C)
	;
	\draw
	(A)--(B)--(D)--(C)--cycle
	(A)--(D)
	;
	\foreach \x/\g in {B/180,C/0,D/-90,A/90}
	\fill (\x) circle (1.5pt)
	+(\g:3mm) node {$\x$};
	\end{tikzpicture}
	}
	\loigiai{
	}
\end{ex}
%%==========Câu 12
\begin{ex}
	\immini{
	Cho hình lăng trụ tam giác $ABC.A'B'C'$.Gọi $M$, $N$ lần lượt là trung điểm của $AB$, $AC$. Trong 4 vectơ $\vec{AB}$, $\vec{CB}$, $\vec{B'C'}$, $\vec{A'C'}$ vectơ nào cùng hướng với vectơ $\vec{MN}$
	\choice
	{$\vec{AB}$}
	{$\vec{CB}$}
	{\True $\vec{B'C'}$}
	{$\vec{A'C'}$}
	}{\begin{tikzpicture}[line join = round, line cap = round, thick, font = \small, scale = .7]
	\path
	(0:0) coordinate (A)
	+(0:4) coordinate (C)
	+(-50:2) coordinate (B)
	+(75:3.5) coordinate (A')
	($(A')+(B)-(A)$) coordinate (B')
	($(A')+(C)-(A)$) coordinate (C')
	($(A)!.5!(B)$) coordinate (M)
	($(A)!.5!(C)$) coordinate (N)
	;
	\draw[dashed]
	(A)--(C) (M)--(N)
	;
	\draw
	(A)--(B)--(C)--(C')--(A')--cycle
	(B')--(A') (B')--(B) (B')--(C')
	;
	\foreach \x/\g in {A/180,B/-90,C/0,A'/-180,B'/70,C'/0,M/-135,N/-45}
	\fill (\x) circle (1.5pt)
	+(\g:3mm) node {$\x$};
	\end{tikzpicture}
	}
	\loigiai{
	Vì $MN$ là đường trung bình của tam giác $ABC$ nên $MN$ song song với $BC$. Mà tứ giác $BCC'B'$ là hình bình hành. Do đó $MN$ song song với $B'C'$. Vậy hai vectơ $\vec{MN}$ và $\vec{B'C'}$ cùng hướng.
	}
\end{ex}
%%==========Câu 13
\begin{ex}
	\immini{
	Cho hình hộp $ABCD.A'B'C'D'$.Số các vectơ có điểm đầu, điểm cuối là các đỉnh của hình hộp và bằng vectơ $\vec{AB}$ là
	\choice
	{$1$}
	{$2$}
	{\True $3$}
	{$4$}
	}{\begin{tikzpicture}[line join = round, line cap = round, thick, font = \small, scale = .7]
	\path
	(0:0) coordinate (D')
	+(75:3.5) coordinate (D)
	+(0:3) coordinate (C')
	+(40:2) coordinate (A')
	($(C')+(D)-(D')$) coordinate (C)
	($(D)+(A')-(D')$) coordinate (A)
	($(C')+(A')-(D')$) coordinate (B')
	($(C)+(A)-(D)$) coordinate (B)
	;
	\draw[dashed]
	(A')--(A) (A')--(B') (A')--(D')
	;
	\draw
	(A)--(B)--(B')--(C')--(D')--(D)--cycle
	(C)--(B) (C)--(D) (C)--(C')
	;
	\foreach \x/\g in {D'/-90,C'/-90,D/180,A'/135,C/-45,A/90,B'/0,B/90}
	\fill (\x) circle (1.5pt)
	+(\g:3mm) node {$\x$};
	\end{tikzpicture}
	}
	\loigiai{
	$\vec{AB}=\vec{DC}=\vec{D'C'}=\vec{A'B'}$
	}
\end{ex}
%%==========Câu 14
\begin{ex}
	Cho hình hộp $ABCD.A'B'C'D'$. Trong các khẳng định dưới đây, đâu là khẳng định đúng?
	\choice
	{$\vec{AB}+\vec{AC}+\vec{AD}=\vec{AC'}$}
	{\True $\vec{AB}+\vec{AA'}+\vec{AD}=\vec{AC'}$}
	{$\vec{AB}+\vec{AA'}+\vec{AD}=\vec{AC}$}
	{$\vec{AB}+\vec{AA'}+\vec{AD}=\vec{0}$}
	\loigiai{
	Xét hình hộp $ABCD.A'B'C'D'$ ta có $\vec{AB}+\vec{AA'}+\vec{AD}=\vec{AC'}$
	}
\end{ex}
%%==========Câu 15
\begin{ex}
	Trong không gian cho tam giác $ABC$ có $G$ là trọng tâm và điểm $M$ nằm ngoài mặt phẳng $(ABC)$. Khẳng định nào sau đây là đúng?
	\choice
	{$\vec{MA}+\vec{MB}+\vec{MC}=\vec{0}$}
	{$\vec{GA}+\vec{GB}+\vec{GC}=0$}
	{$\vec{MA}+\vec{MB}+\vec{MC}=\vec{MG}$}
	{\True $\vec{MA}+\vec{MB}+\vec{MC}=3\vec{MG}$}
	\loigiai{
	Vì $G$ là trọng tâm tam giác $ABC$ nên $\vec{MA}+\vec{MB}+\vec{MC}=3\vec{MG}$
	}
\end{ex}
%%==========Câu 16
\begin{ex}
	Cho hình chóp đều $S.ABCD$ tất cả các cạnh bằng $2\sqrt{3}$. Tính độ dài vectơ $\vec{u}=\vec{SA}-\vec{SC}$.
	\choice
	{$\sqrt{3}$}
	{$\sqrt{2}$}
	{\True $2\sqrt{6}$}
	{$2\sqrt{2}$}
	\loigiai{
	Ta có: $|\vec{u}| = |\vec{SA}-\vec{SC}| = |\vec{CA}| = AB\sqrt{2} =2\sqrt{6}$.
	}
\end{ex}
%%==========Câu 17
\begin{ex}
	Cho tứ diện $ABCD$. Mệnh đề nào dưới đây là mệnh đề đúng?
	\choice
	{$\vec{BC}-\vec{BA}=\vec{DA}-\vec{DC}$}
	{$\vec{AC}-\vec{AD}=\vec{BD}-\vec{BC}$}
	{\True $\vec{AB}-\vec{AC}=\vec{DB}-\vec{DC}$}
	{$\vec{AB}-\vec{AD}=\vec{CD}-\vec{CB}$}
	\loigiai{
	Ta có: $\heva{& \vec{AB}-\vec{AC}=\vec{CB} \\& \vec{DB}-\vec{DC}=\vec{CB}}\Rightarrow \vec{AB}-\vec{AC}=\vec{DB}-\vec{DC}$.
	}
\end{ex}
%%==========Câu 18
\begin{ex}
	Cho hình lăng trụ $ABC.A'B'C'$, $M$ là trung điểm của $BB'$. Đặt $\vec{CA}=\vec{a}$, $\vec{CB}=\vec{b}$, $\vec{AA'}=\vec{c}$. Khẳng định nào sau đây đúng?
	\choice
	{$\vec{AM}=\vec{b}+\vec{c}-\dfrac{1}{2}\vec{a}$}
	{$\vec{AM}=\vec{a}-\vec{c}+\dfrac{1}{2}\vec{b}$}
	{$\vec{AM}=\vec{a}+\vec{c}-\dfrac{1}{2}\vec{b}$}
	{\True $\vec{AM}=\vec{b}-\vec{a}+\dfrac{1}{2}\vec{c}$}
	\loigiai{
	\immini{
	Ta có: $\vec{AM}=\vec{AB}+\vec{BM}=\vec{CB}-\vec{CA}+\dfrac{1}{2}\vec{BB'}=\vec{CB}-\vec{CA}+\dfrac{1}{2}\vec{AA'}=\vec{b}-\vec{a}+\dfrac{1}{2}\vec{c}$
	}{\begin{tikzpicture}[line join = round, line cap = round, thick, font = \small, scale = .6]
	\path
	(0:0) coordinate (A)
	+(0:4) coordinate (C)
	+(-50:2) coordinate (B)
	+(75:4) coordinate (A')
	($(A')+(B)-(A)$) coordinate (B')
	($(A')+(C)-(A)$) coordinate (C')
	($(B)!.5!(B')$) coordinate (M)
	;
	\draw[dashed]
	(A)--(C)
	;
	\draw
	(A)--(B)--(C)--(C')--(A')--cycle
	(B')--(A') (B')--(B) (B')--(C') (A)--(M)
	;
	\foreach \x/\g in {A/180,B/-90,C/0,A'/-180,B'/70,C'/0,M/135}
	\fill (\x) circle (1.5pt)
	+(\g:3mm) node {$\x$};
	\end{tikzpicture}
	}
	}
\end{ex}
%%==========Câu 19
\begin{ex}
	Cho hình lập phương $ABCD.A'B'C'D'$ cạnh $a$. Tính độ dài véctơ $\vec{x}=\vec{A'C'}-\vec{A'A}$ theo $a$?
	\choice
	{$a\sqrt{2}$}
	{$\dfrac{a\sqrt{3}}{2}$}
	{$a\sqrt{6}$}
	{\True $a\sqrt{3}$}
	\loigiai{
	Ta có $\vec{x}=\vec{A'C'}-\vec{A'A}=\vec{AC'}=a\sqrt{3}$.
	}
\end{ex}
%%==========Câu 20
\begin{ex}
	\immini{Cho tứ diện $S.ABC$ có $M$, $N$, $P$ là trung điểm của $SA$, $SB$, $SC$. Tìm khẳng định đúng?}
	{\begin{tikzpicture}[line join = round, line cap = round, thick, font = \small, scale = .7]
	\path
	(0:0) coordinate (A)
	+(0:5) coordinate (C)
	+(-70:2) coordinate (B)
	+(65:4) coordinate (S)
	($(S)!.5!(A)$) coordinate (M)
	($(S)!.5!(B)$) coordinate (N)
	($(S)!.5!(C)$) coordinate (P)
	;
	\draw[dashed]
	(A)--(C) (M)--(P)
	;
	\draw
	(A)--(B)--(C)--(S)--cycle
	(S)--(B) (M)--(N)--(P)
	;
	\foreach \x/\g in {A/180,C/0,B/-90,S/45,M/135,N/-45,P/45}
	\fill (\x) circle (1.5pt)
	+(\g:3mm) node {$\x$};
	\end{tikzpicture}}
	\choice
	{$\vec{AB}=\dfrac{1}{2}\left(\vec{PN}-\vec{PM}\right)$}
	{$\vec{AB}=\vec{PN}-\vec{PM}$}
	{$\vec{AB}=2\left(\vec{PM}-\vec{PN}\right)$}
	{\True $\vec{AB}=2\left(\vec{PN}-\vec{PM}\right)$}
	\loigiai{
	Ta có: $\vec{AB}=2\vec{MN}=2\left(\vec{PN}-\vec{PM}\right)$.
	}
\end{ex}
%%==========Câu 21
\begin{ex}
	\immini{Cho tứ diện $S.ABC$ có đáy là tam giác đều cạnh $a$, $SB$ vuông góc với đáy và $SB=\sqrt{3}a$. Góc giữa hai vectơ $(\vec{AB},\vec{AS})$ là}
	{\begin{tikzpicture}[line join = round, line cap = round, thick, font = \small, scale = .7]
	\path
	(0:0) coordinate (B)
	+(0:5) coordinate (C)
	+(-50:3) coordinate (A)
	+(90:4) coordinate (S)
	;
	\draw[dashed]
	(B)--(C)
	;
	\draw
	(S)--(B)--(A)--(C)--cycle
	(S)--(A)
	\foreach \x/\y/\z in {S/B/C,S/B/A}{
	pic[draw, angle radius = 8pt]{right angle = \x--\y--\z}
	}
	;
	\foreach \x/\g in {B/180,C/0,A/-90,S/90}
	\fill (\x) circle (1.5pt)
	+(\g:3mm) node {$\x$};
	\end{tikzpicture}}
	\choice
	{\True $60^\circ$}
	{$30^\circ$}
	{$45^\circ$}
	{$90^\circ$}
	\loigiai{
	Ta có: $\left(\vec{AB},\vec{AS}\right)=\widehat{SAB}$.\\
	Xét $\triangle SBA$ vuông tại $B$ ta có: $\tan \left(\widehat{SAB}\right)=\dfrac{SB}{AB}=\sqrt{3}$. Suy ra: $\left(\vec{AB},\vec{AS}\right)=60^\circ$
	}
\end{ex}
%%==========Câu 22
\begin{ex}
	Cho hình chóp $S.ABC$ có $AB=4$, $\widehat{BAC}=60^\circ$, $\vec{AB} \cdot \vec{AC}=6$. Khi đó độ dài $\vec{AC}$ là
	\choice
	{\True $3$}
	{$6$}
	{$4$}
	{$12$}
	\loigiai{
	Ta có: $\vec{AB} \cdot \vec{AC}=AB \cdot AC \cdot \cos \widehat{BAC}\Leftrightarrow 6=4 \cdot AC \cdot \cos 60^\circ \Leftrightarrow AC=3$.
	}
\end{ex}
%%==========Câu 23
\begin{ex}
	Trong không gian cho vectơ $\vec{AB}$. Khi đó:
	\choice
	{Giá của vectơ $\vec{AB}$ là $\vec{AB}$}
	{Giá của vectơ $\vec{AB}$ là $\left| \vec{AB} \right|$}
	{\True Giá của vectơ $\vec{AB}$ là đường thẳng $AB$}
	{Giá của vectơ $\vec{AB}$ là đoạn thẳng $AB$}
	\loigiai{
	Giá của vectơ $\vec{AB}$ là đường thẳng $AB$.
	}
\end{ex}
%%==========Câu 24
\begin{ex}
	Cho hình hộp chữ nhật $ABCD.A'B'C'D'$. Trong các vectơ dưới đây, vectơ nào cùng phương với vectơ $\vec{AB}$?
	\choice
	{Vectơ$\vec{AD}$}
	{Vectơ$\vec{CC'}$}
	{Vectơ$\vec{BD}$}
	{\True Vectơ$\vec{CD}$}
	\loigiai{
	$AB \parallel CD$ nên $\vec{AB}$ và $\vec{CD}$ cùng phương.
	}
\end{ex}
%%==========Câu 25
\begin{ex}
	Cho hình hộp $ABCD.A'B'C'D'$. Vectơ $\vec{u}=\vec{A'A}+\vec{A'B'}+\vec{A'D'}$ bằng vectơ nào dưới đây?
	\choice
	{\True $\vec{A'C}$}
	{$\vec{CA'}$}
	{$\vec{AC'}$}
	{$\vec{C'A}$}
	\loigiai{
	Do $A'B'BA$ là hình bình hành nên $\vec{A'A}+\vec{A'B'}=\vec{A'B}$. Lại có, $A'BCD'$ cũng là hình bình hành nên $\vec{A'B}+\vec{A'D'}=\vec{A'C}$. Vậy $\vec{A'A}+\vec{A'B'}+\vec{A'D'}=\vec{A'C}$
	}
\end{ex}
%%==========Câu 26
\begin{ex}
	Cho hình lăng trụ tam giác $ABC.A'B'C'$. Đặt $\vec{AA'}=\vec{a}$, $\vec{AB}=\vec{b}$, $\vec{AC}=\vec{c}$, $\vec{BC}=\vec{d}$. Trong các biểu thức vec tơ sau đây, biểu thức nào là đúng?
	\choice
	{$\vec{a}=\vec{b}+\vec{c}$}
	{$\vec{a}+\vec{b}+\vec{c}+\vec{d}=\vec{0}$}
	{\True $\vec{b}-\vec{c}+\vec{d}=\vec{0}$}
	{$\vec{a}+\vec{b}+\vec{c}=\vec{d}$}
	\loigiai{
	Ta có: $\vec{b}-\vec{c}+\vec{d}=\vec{AB}-\vec{AC}+\vec{BC}=\vec{CB}+\vec{BC}=\vec{0}$.
	}
\end{ex}
%%==========Câu 27
\begin{ex}
	Cho lập phương $ABCD.A'B'C'D'$ có độ dài mỗi cạnh bằng $1$. Tính độ dài của vectơ $\vec{AC}+\vec{C'D'}$.
	\choice
	{$\sqrt{3}$}
	{$\sqrt{2}$}
	{\True $1$}
	{$2\sqrt{2}$}
	\loigiai{
	Ta có: $A'C'CA$ là hình chữ nhật nên $\vec{A'C'}=\vec{AC}$.\\
	Khi đó, $\vec{AC}+\vec{C'D'}=\vec{A'C'}+\vec{C'D'}=\vec{A'D'}$. Vậy $\left| \vec{AC}+\vec{C'D'} \right| =\left| \vec{A'D'} \right| =A'D'=1$
	}
\end{ex}
%%==========Câu 28
\begin{ex}
	Cho $O$ là tâm hình bình hành $ABCD$. Hỏi vectơ $\left(\vec{AO}-\vec{DO}\right)$ bằng vectơ nào?
	\choice
	{$\vec{BA}$}
	{\True $\vec{AD}$}
	{$\vec{DC}$}
	{$\vec{AC}$}
	\loigiai{
	Ta có: $\vec{AO}-\vec{DO}=\vec{AO}+\vec{OD}=\vec{AD}$.
	}
\end{ex}
%%==========Câu 29
\begin{ex}
	Cho ba điểm phân biệt $A$, $B$, $C$. Nếu $\vec{AB}=-3\vec{AC}$ thì đẳng thức nào dưới đây đúng?
	\choice
	{$\vec{BC}=-4\vec{AC}$}
	{$\vec{BC}=-2\vec{AC}$}
	{$\vec{BC}=2\vec{AC}$}
	{\True $\vec{BC}=4\vec{AC}$}
	\loigiai{
	Ta có: $\vec{AB}=-3\vec{AC}\Leftrightarrow \vec{CB}-\vec{CA}=-3\vec{AC}\Leftrightarrow \vec{AC}+3\vec{AC}=-\vec{CB}\Leftrightarrow \vec{BC}=4\vec{AC}$.
	}
\end{ex}
%%==========Câu 30
\begin{ex}
	Cho tam giác $ABC$ có điểm $O$ thỏa mãn: $\left| \vec{OA}+\vec{OB}-2\vec{OC} \right| = \left| \vec{OA}-\vec{OB} \right|$. Khẳng định nào sau đây là đúng?
	\choice
	{Tam giác $ABC$ đều}
	{Tam giác $ABC$ cân tại $C$}
	{\True Tam giác $ABC$ vuông tại $C$}
	{Tam giác $ABC$ cân tại $B$}
	\loigiai{
	Gọi $M$ là trung điểm $AB$, ta có $\vec{OA}+\vec{OB}=2\vec{OM}$.\\
	Do đó, $\left| \vec{OA}+\vec{OB}-2\vec{OC} \right| =\left| \vec{OA}-\vec{OB} \right|\Leftrightarrow \left| 2\vec{OM}-2\vec{OC} \right| =\left| \vec{BA} \right|\Leftrightarrow 2\left| \vec{CM} \right| =BA\Leftrightarrow CM=\dfrac{1}{2}BA$ \hfill $(1)$\\
	Vì $M$ là trung điểm $AB$ nên $CM$ là đường trung tuyến của $\triangle ABC$, Từ $(1)$ suy ra, tam giác $\triangle ABC$ vuông tại $C$.
	}
\end{ex}
%%==========Câu 31
\begin{ex}
	Cho hình hộp $ABCD.A'B'C'D'$. Đẳng thức nào dưới đây là đúng?
	\choice
	{$\vec{AC'}=\vec{AB}+\vec{AD}+\vec{AC}$}
	{$\vec{AC'}=\vec{AA'}+\vec{AD}+\vec{AC}$}
	{\True $\vec{AC'}=\vec{AB'}+\vec{AD}$}
	{$\vec{AC'}=\vec{AC}+\vec{AB}+\vec{AA'}$}
	\loigiai{
	Do $AB'C'D$ là hình bình hành nên $\vec{AC'}=\vec{A'B'}+\vec{AD}$.
	}
\end{ex}
%%==========Câu 32
\begin{ex}
	Cho hình lập phương $ABCD.A'B'C'D'$ có độ dài cạnh bằng $a$. Tính độ dài của vectơ $\vec{AD'}+\vec{BA'}$.
	\choice
	{$\sqrt{3}a$}
	{$\sqrt{2}a$}
	{\True $\sqrt{6}a$}
	{$2\sqrt{3}a$}
	\loigiai{
	Gọi $O'$ là tâm của hình vuông $A'B'C'D'$.\\
	Ta có $ABC'D'$ là hình bình hành nên $\vec{AD'}=\vec{BC'}$, do đó $\vec{BA'}+\vec{AD'}=\vec{BA'}+\vec{BC'}=2\vec{BO'}$.\\
	Tam giác $BA'C'$ là tam giác đều cạnh $a\sqrt{2}$ nên $BO'=\dfrac{\sqrt{3}}{2}a\sqrt{2}=\dfrac{\sqrt{6}}{2}a$.\\
	Từ đó độ dài của vectơ $\vec{AD'}+\vec{BA'}$ bằng $\sqrt{6}a$.
	}
\end{ex}
%%==========Câu 33
\begin{ex}%[2H2H1-4]
	\immini{
	Trong điện trường đều, lực tĩnh điện $\vec{F}$ (đơn vị: N) tác dụng lên điện tích điểm có điện tích $q$ (đơn vị: C) được tính theo công thức $\vec{F}=q \cdot \vec{E}$, trong đó $\vec{E}$ là cường độ điện trường (đơn vị: N/C). Tính độ lớn của lực tĩnh điện tác dụng lên điện tích điểm khi $q=10^{-9}$ C và độ lớn điện trường $E=10^5$ N/C.
	\choice
	{$10^{-3}$ N}
	{$10^{4}$ N}
	{$10^{-14}$ N}
	{\True $10^{-4}$ N}
	}{\hspace{1cm}
	\begin{tikzpicture}[scale=.7]
	\def\drong{3} % khoảng cách giữa 2 thanh
	\def\drog{0.3} % một nữa độ rộng của thanh
	\def\num{8}
	\filldraw [green!5, draw=green!80!black] ($(0,0)+(-\drog,\drog)$) rectangle ($(0,0)+(\num,0)+(\drog,-\drog)$)
	($(0,-\drong)+(-\drog,\drog)$) rectangle ($(0,-\drong)+(\num,0)+(\drog,-\drog)$);
	\foreach \a in {0,1,...,\num}{
	\draw[->,>=Latex] ($(0,0)+(\a,0)$)node[red]{$+$} ($(0,0)+(\a,0)+(0,-\drog)$)-- ($(0,-0.5*\drong)+(\a,0)$);
	\draw ($(0,-0.5*\drong)+(\a,0)$) -- ($(0,-\drong)+(\a,0)+(0,\drog)$) ($(0,-\drong)+(\a,0)$)node[red]{$-$} ;}
	\draw[->,>=Latex] (3.5,-0.4*\drong)--++(-90:1)node[below]{$\vec{F}$};
	\filldraw [green!5, draw=green!80!black] (3.5,-0.4*\drong)node[green!90!black]{$+$} circle (0.2cm) ;
	\draw (4.5,-0.5*\drong)node{$M$} (8.5,-0.5*\drong)node[red]{$\vec{E}$};
	\end{tikzpicture}
	}
	\loigiai{
	Từ công thức $\vec{F}=q \cdot \vec{E}$ suy ra $\begin{aligned}[t]
	|\vec{F}| & =q|\vec{E}| \\
	 & =10^{-9} \cdot 10^5 \\
	 & =10^{-4} \text{N}.
	\end{aligned}$\\
	Vậy độ lớn của lực tĩnh điện tác dụng lên điện tích điểm là $10^{-4}$ N.
	}
\end{ex} \dongcham{4}
\Closesolutionfile{ans}
\textbf{PHẦN II.} \textit{Câu trắc nghiệm đúng sai. Trong mỗi ý a), b), c), d) ở mỗi câu, học sinh chọn đúng hoặc sai.}\\
\Opensolutionfile{ans}[ans/2H2-B1-d1-2]
%%==========Câu 34
\begin{ex}
	\immini{Cho hình hộp chữ nhật $ABCD.A'B'C'D'$ có cạnh $AB=a$; $AD=a\sqrt{3}$; $AA'=2a$. Xét tính đúng, sai của các khẳng định sau:
	\choiceTF
	{$\vec{AB'}+\vec{CD'}=\vec{0}$}
	{\True $\vec{A'D}+\vec{CB'}=\vec{0}$}
	{$\big|\vec{AB}+\vec{AD}\big|=a\sqrt{5}$}
	{\True $\big|\vec{AB}+\vec{A'D'}+\vec{CC'}\big|=2\sqrt{2}a$}}{
	\begin{tikzpicture}[scale=0.6, font=\footnotesize, line join=round, line cap=round]
	\def\h{4}
	\foreach \x\y\t in {0/0/A,-1/-1.1/B,4.6/-1.1/C}
	\coordinate (\t) at (\x,\y);
	\coordinate (D) at ($(A)+(C)-(B)$);
	\coordinate (A') at ($(A)+(0,3.2)$);
	\coordinate (B') at ($(B)+(0,3.2)$);
	\coordinate (C') at ($(C)+(0,3.2)$);
	\coordinate (D') at ($(D)+(0,3.2)$);
	\draw (B')--(A')--(D')--(C')--(B')--(B)--(C)--(D)--(D') (C')--(C);
	\draw[dashed](B)--(A)--(D) (A)--(A');
	\foreach \t/\g in {A/170,B/-150,C/-70,D/0,A'/100,B'/170,C'/-20,D'/50}
	\draw[fill=black] (\t) circle(1pt)
	node[shift={(\g:7pt)}]{$\t$};
	\end{tikzpicture}}
	\loigiai{\begin{enumerate}[a)]
	\item $\vec{AB'}$ và $\vec{CD'}$ không đối nhau nên $\vec{AB'}+\vec{CD'} \ne \vec{0}$
	\item $\vec{A'D}$ và $\vec{CB'}$ đối nhau nên $\vec{AB'}+\vec{CD'} = \vec{0}$
	\item $\big|\vec{AB}+\vec{AD}\big|=\big|\vec{AC}\big|=AC=\sqrt{AB^2+AD^2}=2a$
	\item $\big|\vec{AB}+\vec{A'D'}+\vec{CC'}\big|=\big|\vec{AB}+\vec{AD}+\vec{AA'}\big|=AC'=\sqrt{AB^2+AD^2+AA^2}=2\sqrt{2}a$
	\end{enumerate}}
\end{ex} \dongcham{8}
%%==========Câu 35
\begin{ex}%[2H2H1-2]
	\immini{Cho hình lập phương $ABCD.A'B'C'D'$ có cạnh bằng $a$. Xét tính đúng, sai của các khẳng định sau:
	\choiceTF
	{\True $\vec{B'B} - \vec{DB} = \vec{B'D}$}
	{$\vec{BA}+\vec{BC}+\vec{BB'}=\vec{BD}$}
	{$\big|\vec{BA}+\vec{BC}+\vec{BB'}\big|=a\sqrt{2}$}
	{\True $\big|\vec{BC}-\vec{BA}+\vec{C'A}\big|=a$}
	}{
	\begin{tikzpicture}[scale=0.6, font=\footnotesize, line join=round, line cap=round]
	\def\h{4}
	\foreach \x\y\t in {0/0/A,-1/-1.1/B,2.6/-1.1/C}
	\coordinate (\t) at (\x,\y);
	\coordinate (D) at ($(A)+(C)-(B)$);
	\coordinate (A') at ($(A)+(0,3.2)$);
	\coordinate (B') at ($(B)+(0,3.2)$);
	\coordinate (C') at ($(C)+(0,3.2)$);
	\coordinate (D') at ($(D)+(0,3.2)$);
	\draw (B')--(A')--(D')--(C')--(B')--(B)--(C)--(D)--(D') (C')--(C);
	\draw[dashed](B)--(A)--(D) (A)--(A');
	\foreach \t/\g in {A/170,B/-150,C/-70,D/0,A'/100,B'/170,C'/-20,D'/50}
	\draw[fill=black] (\t) circle(1pt)
	node[shift={(\g:7pt)}]{$\t$};
	\end{tikzpicture}}
	\loigiai{
	\immini{\vspace*{-3mm}
	\begin{listEX}
	\item Ta có \begin{eqnarray*}
	\vec{B'B} - \vec{DB} &=& \vec{B'B} + \left( - \vec{DB} \right) \\
	&=& \vec{B'B} + \vec{BD} \\
	&=& \vec{B'D}.
	\end{eqnarray*}
	\item Áp dụng quy tắc hình hộp ta có $\vec{BA}+\vec{BC}+\vec{BB'}=\vec{BD'}$.\\
	\item $\big|\vec{BA}+\vec{BC}+\vec{BB'}\big|=\big|\vec{BD'}\big|=BD'=a\sqrt{3}$
	\item Ta có $\vec{BC}-\vec{BA}+\vec{C'A}=\vec{AC}+\vec{C'A}=\vec{C'C}$.\\
	Do đó $\big|\vec{BC}-\vec{BA}+\vec{C'A}\big|=C'C=a$
	\end{listEX}}
	{
	\begin{tikzpicture}[scale=0.6, font=\footnotesize, line join=round, line cap=round]
	\def\h{4}
	\foreach \x\y\t in {0/0/A,-1/-1.1/B,2.6/-1.1/C}
	\coordinate (\t) at (\x,\y);
	\coordinate (D) at ($(A)+(C)-(B)$);
	\coordinate (A') at ($(A)+(0,3.2)$);
	\coordinate (B') at ($(B)+(0,3.2)$);
	\coordinate (C') at ($(C)+(0,3.2)$);
	\coordinate (D') at ($(D)+(0,3.2)$);
	\draw (B')--(A')--(D')--(C')--(B')--(B)--(C)--(D)--(D') (C')--(C);
	\draw[dashed](B)--(A)--(D) (A)--(A');
	\foreach \t/\g in {A/170,B/-150,C/-70,D/0,A'/100,B'/170,C'/-20,D'/50}
	\draw[fill=black] (\t) circle(1pt)
	node[shift={(\g:7pt)}]{$\t$};
	\end{tikzpicture}}
	}
\end{ex} \dongcham{14}
%%==========Câu 36
\begin{ex}%[2H2N1-2]
	\immini{Cho hình lăng trụ tam giác $A B C.A' B' C'$ có $\vec{A A'}=\vec{a}$, $\vec{A B}=\vec{b}$ và $\vec{A C}=\vec{c}$. Gọi $M$ là trung điểm của $BC$. Xét tính đúng, sai của các khẳng định sau:
	\choiceTF
	{\True $\vec{B'C}=-\vec{a}-\vec{b}+\vec{c}$}
	{\True $\vec{BC'}=\vec{a}-\vec{b}+\vec{c}$}
	{$\vec{AM}=\vec{b}+\vec{c}$}
	{\True $\vec{A'M}=-\vec{a}+\dfrac{1}{2}\vec{b}+\dfrac{1}{2}\vec{c}$}
	}{
	\begin{tikzpicture}[scale=0.8, font=\footnotesize,>=stealth]
	\path
	(0,0) coordinate (A)
	(4,0) coordinate (C)
	(1.5,-1.5) coordinate (B)
	($(A)+(0.4,3)$)coordinate (A')
	($(B)+(0.4,3)$)coordinate (B')
	($(C)+(0.4,3)$)coordinate (C')
	($(B)!1/2!(C)$)coordinate (M)
	;
	\draw (B)--(C)--(C')--(B')--(B)--(A)--(A')--(B') (A')--(C');
	\draw[dashed] (C)--(A)--(M)--(A');
	\foreach \x/\g in {A/180,B/-45,C/0,A'/180,B'/-30,C'/0,M/-90}\draw[fill=black] (\x) circle (.04) +(\g:.4)node{\footnotesize$\x$};
	\end{tikzpicture}}
	\loigiai{
	\immini{\begin{enumerate}[a)]
	\item $\vec{B'C}=\vec{B'A'}+\vec{A'C'}+\vec{C'C}=-\vec{AB}+\vec{AC}-\vec{AA'}$ hay $\vec{B'C}=-\vec{a}-\vec{b}+\vec{c}$;
	\item $\vec{BC'}=\vec{BB'}+\vec{B'A'}+\vec{A'C'}=\vec{AA'}-\vec{AB}+\vec{AC}$ hay $\vec{BC'}=\vec{a}-\vec{b}+\vec{c}$;
	\item Ta có $\vec{AB}+\vec{AC}=2\vec{AM}$, suy ra $\vec{AM}=\dfrac{1}{2}\vec{AB}+\dfrac{1}{2}\vec{AC}=\dfrac{1}{2}\vec{b}+\dfrac{1}{2}\vec{c}$
	\item $\vec{A'M}=\vec{A'A}+\vec{AM}=\vec{A'A}+\dfrac{1}{2}\vec{AB}+\dfrac{1}{2}\vec{AC}=-\vec{a}+\dfrac{1}{2}\vec{b}+\dfrac{1}{2}\vec{c}$
	\end{enumerate}}{\begin{tikzpicture}[scale=0.6, font=\footnotesize,>=stealth]
	\path
	(0,0) coordinate (A)
	(4,0) coordinate (C)
	(1.5,-1.5) coordinate (B)
	($(A)+(0.4,3)$)coordinate (A')
	($(B)+(0.4,3)$)coordinate (B')
	($(C)+(0.4,3)$)coordinate (C')
	($(B)!1/2!(C)$)coordinate (M)
	;
	\draw (B)--(C)--(C')--(B')--(B)--(A)--(A')--(B') (A')--(C');
	\draw[dashed] (C)--(A)--(M)--(A');
	\foreach \x/\g in {A/180,B/-45,C/0,A'/180,B'/-30,C'/0,M/-90}\draw[fill=black] (\x) circle (.04) +(\g:.4)node{\footnotesize$\x$};
	\end{tikzpicture}}
	}
\end{ex} \dongcham{14}
%%==========Câu 37
\begin{ex}
	\immini{
	Cho tứ diện $ABCD$. Gọi $M$, $N$ lần lượt là trung điểm của các cạnh $AD$ và $BC$, $I$ là trung điểm $MN$. Xét tính đúng, sai của các khẳng định sau:
	\choiceTF
	{$\vv{A B}-\vv{C D}=\vv{A C}-\vv{B D}$}
	{\True $\vec{AB} + \vec{CD} = \vec{AD} + \vec{CB}$}
	{\True $\vec{AB} + \vec{DC}=2\vec{MN}$}
	{\True $\vec{IA} + \vec{IB} + \vec{IC} + \vec{ID} = \vec{0}$}
	}{
	\vspace*{-3mm}
	\begin{tikzpicture}[scale=0.5, font=\footnotesize, line join=round, line cap=round]
	\foreach \x\y\t in {0/0/B,6/0/D,1.5/-2/C,1.5/5/A}
	\coordinate (\t) at (\x,\y);
	\coordinate (M) at ($(A)!1/2!(D)$);
	\coordinate (N) at ($(B)!1/2!(C)$);
	\coordinate (I) at ($(M)!1/2!(N)$);
	\draw (A)--(B)--(C)--(D)--(A)--(C);
	\draw[dashed] (D)--(I)--(A) (B)--(I)--(C) (M)--(N) (B)--(D);
	\foreach \t/\g in {A/90,B/180,C/-90,D/0,M/0,N/180,I/20} \draw (\t) node[shift={(\g:10pt)}]{$\t$};
	\end{tikzpicture}}
	\loigiai{
	\begin{enumerate}
	\item Sử dụng quy tắc ba điểm và quy tắc hiệu, ta có
	 \begin{align*}
	 \vv{A B}-\vv{C D} & \ =\left(\vv{A C}+\vv{C B}\right)-\vv{C D} \\
	 & \ =\vv{A C}+\left(\vv{C B}-\vv{C D}\right) \\
	 & \ =\vv{A C}+\vv{D B} \\
	 & \ =\vv{A C}-\vv{B D}.
	 \end{align*}
	\item Theo quy tắc ba điểm, ta có $\vec{AB} = \vec{AD} + \vec{DB}$. Do đó
	 \begin{eqnarray*}
	 \vec{AB} + \vec{CD} &=& \vec{AD} + \vec{DB} + \vec{CD} \\
	 &=&\vec{AD}+ \left( \vec{CD} + \vec{DB} \right) \\
	 &=& \vec{AD} + \vec{CB}.
	 \end{eqnarray*}
	\item Ta có
	\item
	\end{enumerate}
	}
\end{ex} \dongcham{8}
%%==========Câu 38
\begin{ex}
	\immini
	{
	Một chiếc ô tô được đặt trên mặt đáy dưới của một khung sắt có dạng hình hộp chữ nhật với đáy trên là hình chữ nhật $ABCD$, mặt phẳng $(ABCD)$ song song với mặt phẳng nằm ngang. Khung sắt đó được buộc vào móc $E$ của chiếc cần cẩu sao cho các đoạn dây cáp $EA$, $EB$, $EC$, $ED$ có độ dài bằng nhau và cùng tạo với mặt phẳng $(ABCD)$ một góc bằng $60^\circ$. Chiếc cần cẩu kéo khung sắt lên theo phương thẳng đứng. Biết rằng các lực căng $\vec{F_1}$, $\vec{F_2}$, $\vec{F_3}$, $\vec{F_4}$ đều có cường độ là $4700$ N và trọng lượng của khung sắt là $3000$ N.
	\choiceTF
	{$\vec{F_1}+\vec{F_2}=\vec{F_3}+\vec{F_4}$}
	{\True $\vec{F_1}+\vec{F_3}=\vec{F_2}+\vec{F_4}$}
	{\True $\big|\vec{F_1}+\vec{F_3}\big|=8141$ N (\textit{làm tròn đến hàng đơn vị})}
	{Trọng lượng của chiếc xe ô tô là $16282$ N (\textit{làm tròn đến hàng đơn vị})}
	}
	{\hspace{1cm}
	\includegraphics[scale=.09]{images/xe-1.jpg}
	}
	\loigiai{
	Lấy các điểm $M$, $N$, $P$, $Q$ lần lượt trên các tia $EA$, $EB$, $EC$, $ED$ sao cho
	\[
	\vec{EM} = \vec{F_1},\ \vec{EN} = \vec{F_2},\ \vec{EP} = \vec{F_3},\ \vec{EQ} = \vec{F_4}.
	\]
	Do các lực căng $\vec{F_1}$, $\vec{F_2}$, $\vec{F_3}$, $\vec{F_4}$ đều có cường độ là $4700$ N nên $EM = EN = EP = EQ = 4700$.
	\begin{center}
	\begin{tikzpicture}[line join=round, line cap = round, >=stealth, scale=.8,font=\footnotesize,transform shape]
	\foreach \x/\y/\z/\g in
	{
	-3/0/A/180, -1/-1/B/-90, 3/0/C/0, 1/1/D/45, 0/4/E/90
	}
	\draw[fill=black] (\x,\y) circle(1pt) coordinate (\z) ($(\z)+(\g:3.5mm)$) node{$\z$};
	\path
	($(E)!.75!(A)$) coordinate (M)
	($(E)!.75!(B)$) coordinate (N)
	($(E)!.75!(C)$) coordinate (P)
	($(E)!.75!(D)$) coordinate (Q)
	($(M)!.5!(P)$) coordinate (O)
	;
	\draw (E)--(A)--(B)--(E)--(C)--(B) (M)--(N)--(P);
	\draw[dashed] (M)--(P)--(Q)--(M) (N)--(Q) (O)--(E)--(D)--(C) (A)--(D);
	\foreach \x/\g in {M/135, N/-45,P/45,Q/45,O/135}
	\draw[fill = white] (\x) circle(1pt) ($(\x)+(\g:3mm)$) node{$\x$};
	\end{tikzpicture}
	\end{center}
	\begin{enumerate}[a)]
	\item Ta có
	 \begin{itemize}
	 \item [$\bullet$] $\vec{F_1}+\vec{F_2}=\vec{EM}+\vec{EN}=2\vec{EH}$, với $H$ là trung điểm của $MN$.
	 \item [$\bullet$] $\vec{F_3}+\vec{F_4}=\vec{EP}+\vec{EQ}=2\vec{EK}$, với $K$ là trung điểm của $PQ$.
	 \end{itemize}
	 Suy ra $\vec{F_1}+\vec{F_2}\ne \vec{F_3}+\vec{F_4}$
	\item Ta có
	 \begin{itemize}
	 \item [$\bullet$] $\vec{F_1}+\vec{F_3}=\vec{EM}+\vec{EP}=2\vec{EO}$, với $O$ là trung điểm của $MP$.
	 \item [$\bullet$] $\vec{F_2}+\vec{F_4}=\vec{EN}+\vec{EQ}=2\vec{EO}$, với $O$ là trung điểm của $MP$.
	 \end{itemize}
	 Suy ra $\vec{F_1}+\vec{F_3}=\vec{F_2}+\vec{F_4}$.
	\item $\big|\vec{F_1}+\vec{F_3}\big|=\big|2\vec{EO}\big|=2EO$.\\
	 Theo giả thiết, góc giữa $EA$ với $(ABCD)$ bằng $60^\circ$, suy ra góc giữa $EM$ với $(MNPQ)$ cũng bằng $60^\circ$ hay $\widehat{SMO}=60^\circ$.\\
	 Xét $\triangle EMO$ có $EM=4700$, $\widehat{SMO}=60^\circ$. Suy ra $EO = EM \sin 60^\circ = 2350\sqrt{3}$.\\
	 Từ đây, ta tính được $\big|\vec{F_1}+\vec{F_3}\big|=2EO=8141$ N.
	\item Gọi $\vec{P}$ là trọng lực tác dụng lên cả hệ, do $O$ là trung điểm $MP$, $NQ$ nên ta có:
	 \begin{eqnarray*}
	 & \vec{P} & = \vec{F_1}+\vec{F_2}+\vec{F_3}+\vec{F_4}\\
	 & & = \vec{EM} + \vec{EN} + \vec{EP} + \vec{EQ}\\
	 & & = \vec{EO} + \vec{OM} + \vec{EO} + \vec{ON} + \vec{EO} + \vec{OP} + \vec{EO} + \vec{OQ}\\
	 & & = 4\vec{EO} + \left(\vec{OM} + \vec{OP}\right) + \left(\vec{ON} + \vec{OQ}\right)\\
	 & & = 4\vec{EO}.
	 \end{eqnarray*}
	 Suy ra trọng lượng của toàn bộ hệ là $\left| \vec{P} \right| = 4\left| \vec{EO}\right| = 4EO = 9400\sqrt{3}$ N.\\
	 Do trọng trượng khung sắt là $3000$ N nên trọng lượng của xe ô tô là $9400\sqrt{3} - 3000 \approx 13281$ N.
	\end{enumerate}
	}
\end{ex} \dongcham{14}
%%==========Câu 39
\begin{ex}
	\immini{Cho tứ diện $ABCD$ có $AB=AC=AD=a$ và $\widehat{BAC}=\widehat{BAD}=60^\circ ,\widehat{CAD}=90^\circ $. Gọi $I$ là điểm trên cạnh $AB$ sao cho $AI=3IB$ và $J$ là trung điểm của $CD$. Gọi $\alpha $ là góc giữa hai vectơ $\vec{AB}$ và $\vec{IJ}$.
	\choiceTF
	{\True Tam giác $BCD$ vuông cân}
	{$\vec{IJ}=\dfrac{1}{2}\vec{AC}+\dfrac{1}{2}\vec{AD}+\dfrac{3}{2}\vec{AB}$}
	{$\vec{AB} \cdot \vec{AC}+\vec{AC} \cdot \vec{AD}+\vec{AD} \cdot \vec{AB}=\dfrac{a^2}{2}$}
	{\True $\cos \alpha =-\dfrac{\sqrt{5}}{5}$}
	}{\begin{tikzpicture}[line join = round, line cap = round, thick, font = \small, scale = .7]
	\path
	(0:0) coordinate (B)
	+(0:5) coordinate (C)
	+(-70:2) coordinate (D)
	+(75:4) coordinate (A)
	($(B)!1/4!(A)$) coordinate (I)
	($(C)!.5!(D)$) coordinate (J)
	;
	\draw[dashed]
	(B)--(C) (I)--(J)
	;
	\draw
	(A)--(B)--(D)--(C)--cycle
	(A)--(D)
	;
	\foreach \x/\g in {B/180,C/0,D/-90,A/90,I/135,J/-45}
	\fill (\x) circle (1.5pt)
	+(\g:3mm) node {$\x$};
	\end{tikzpicture}}
	\loigiai{
	\begin{enumerate}[a)]
	\item Tam giác $ABC$, $ABD$ đều cạnh bằng $a$, tam giác $ACD$ vuông cân đỉnh $A\Rightarrow CD=a\sqrt{2}$. Vậy tam giác $BCD$ có $BC=BD=a,CD=a\sqrt{2}$ nên tam giác $BCD$ vuông cân.
	\item $\vec{IJ}=\vec{IA}+\vec{AJ}=-\dfrac{3}{4}\vec{AB}+\dfrac{1}{2}\left(\vec{AC}+\vec{AD}\right)=\dfrac{1}{2}\vec{AC}+\dfrac{1}{2}\vec{AD}-\dfrac{3}{4>}\vec{AB}$.
	\item Ta có: $\vec{AC} \cdot \vec{AD}=0$, $\vec{AB} \cdot \vec{AD}=AB \cdot AD \cdot \cos 60^\circ =\dfrac{a^2}{2}$, $\vec{AC} \cdot \vec{AB}=\dfrac{a^2}{2}$. Suy ra $\vec{AB} \cdot \vec{AC}+\vec{AC} \cdot \vec{AD}+\vec{AD} \cdot \vec{AB}=a^2$.\\
	\item $IJ^2=\vec{IJ}^2=\dfrac{1}{4}{{\left(\vec{AC}+\vec{AD}-\dfrac{3}{2}\vec{AB}\right)}^2}
	 =\dfrac{1}{4}\left(\dfrac{17}{4}a^2+2\vec{AC} \cdot \vec{AD}-3\vec{AC} \cdot \vec{AB}-3\vec{AB} \cdot \vec{AD}\right)
	 =\dfrac{5a^2}{16}\Rightarrow IJ=\dfrac{a\sqrt{5}}{4}$.\\
	 $\vec{IJ} \cdot \vec{AB}=\dfrac{1}{2}\left(\vec{AC}+\vec{AD}-\dfrac{3}{2}\vec{AB}\right) \cdot \vec{AB}= \dfrac{1}{2}\left(\vec{AC} \cdot \vec{AB}+\vec{AD} \cdot \vec{AB}-\dfrac{3}{2}{{\vec{AB}}^2}\right)=-\dfrac{a^2}{4}$.\\
	 $\cos \left(\vec{IJ},\vec{AB}\right)=\dfrac{\vec{IJ} \cdot \vec{AB}}{IJ \cdot AB}=\dfrac{-\dfrac{a^2}{4}}{\dfrac{a\sqrt{5}}{4} \cdot a}=-\dfrac{\sqrt{5}}{5}$.
	\end{enumerate}
	}
\end{ex}
%%==========Câu 40
\begin{ex}
	\immini{Cho tứ diện $ABCD$. Gọi $M$, $N$, $P$, $Q$, $R$, $S$, $G$ lần lượt là trung điểm các đoạn thẳng $AB$, $CD$, $AC$, $BD$, $AD$, $BC$, $MN$.
	\choiceTF
	{\True $\vec{MR}=\vec{SN}$}
	{\True $\vec{GA}+\vec{GB}+\vec{GC}+\vec{GD}=\vec{0}$}
	{$2\vec{PQ}=\vec{AB}+\vec{AC}+\vec{AD}$}
	{\True $|\vec{IA}+\vec{IB}+\vec{IC}+\vec{ID}|$ nhỏ nhất khi và chỉ khi điểm $I$ trùng với điểm $G$}
	}{\begin{tikzpicture}[line join = round, line cap = round, thick, font = \small, scale = .7]
	\path
	(0:0) coordinate (B)
	+(0:5) coordinate (C)
	+(-70:2) coordinate (D)
	+(75:4) coordinate (A)
	($(A)!.5!(B)$) coordinate (M)
	($(C)!.5!(D)$) coordinate (N)
	($(A)!.5!(C)$) coordinate (P)
	($(B)!.5!(D)$) coordinate (Q)
	($(A)!.5!(D)$) coordinate (R)
	($(B)!.5!(C)$) coordinate (S)
	($(M)!.5!(N)$) coordinate (G)
	;
	\draw[dashed]
	(B)--(C) (M)--(N)
	;
	\draw
	(A)--(B)--(D)--(C)--cycle
	(A)--(D)
	;
	\foreach \x/\g in {B/180,C/0,D/-90,A/90,M/135,N/-45,P/45,Q/-135,R/180,S/45,G/45}
	\fill (\x) circle (1.5pt)
	+(\g:3mm) node {$\x$};
	\end{tikzpicture}}
	\loigiai{
	\begin{center}
	\begin{tikzpicture}[line join = round, line cap = round, thick, font = \small, scale = .7]
	\path
	(0:0) coordinate (B)
	+(0:5) coordinate (C)
	+(-70:2) coordinate (D)
	+(75:4) coordinate (A)
	($(A)!.5!(B)$) coordinate (M)
	($(C)!.5!(D)$) coordinate (N)
	($(A)!.5!(C)$) coordinate (P)
	($(B)!.5!(D)$) coordinate (Q)
	($(A)!.5!(D)$) coordinate (R)
	($(B)!.5!(C)$) coordinate (S)
	($(M)!.5!(N)$) coordinate (G)
	;
	\draw[dashed]
	(B)--(C) (M)--(N) (P)--(Q) (R)--(S)
	;
	\draw
	(A)--(B)--(D)--(C)--cycle
	(A)--(D)
	;
	\foreach \x/\g in {B/180,C/0,D/-90,A/90,M/135,N/-45,P/45,Q/-135,R/180,S/45,G/45}
	\fill (\x) circle (1.5pt)
	+(\g:3mm) node {$\x$};
	\end{tikzpicture}
	\end{center}
	\begin{enumerate}[a)]
	\item $\vec{MR}=\vec{SN}=\dfrac12 \vec{BD}$.
	\item Vì $M$ là trung điểm của $AB$ nên $\vec{GA}+\vec{GB}=2\vec{GM}$\\
	 Vì $N$ là trung điểm của $CD$ nên $\vec{GC}+\vec{GD}=2\vec{GN}$\\
	 Vì $G$ là trung điểm của $MN$ nên $\vec{GM}+\vec{GN}=\vec{0}$\\
	 Do đó: $\vec{GA}+\vec{GB}+\vec{GC}+\vec{GD}=2\left(\vec{GM}+\vec{GN}\right)=2 \cdot \vec{0}=\vec{0}$.
	\item $\vec{PQ}=\vec{AQ}-\vec{AP}=\dfrac{1}{2}\left(\vec{AB}+\vec{AD}\right)-\dfrac{1}{2}\vec{AC}\Leftrightarrow 2\vec{PQ}=\vec{AB}-\vec{AC}+\vec{AD}$
	\item $\vec{IA}+\vec{IB}+\vec{IC}+\vec{ID}=4\vec{IG}+\left(\vec{GA}+\vec{GB}+\vec{GC}+\vec{GD}\right)=4\vec{IG}$.\\
	 $\Rightarrow | \vec{IA}+\vec{IB}+\vec{IC}+\vec{ID}|=| 4\vec{IG}|=4IG$\\
	 Do đó: $| \vec{IA}+\vec{IB}+\vec{IC}+\vec{ID}|$ nhỏ nhất khi $IG=0\Leftrightarrow I\equiv G$
	\end{enumerate}
	}
\end{ex}
%%==========Câu 41
\begin{ex}
	Cho tứ diện đều $SABC$ có cạnh $a$. Gọi $M$, $N$ lần lượt là trung điểm $SA$, $BC$. Các mệnh đề sau đúng hay sai?
	\begin{center}
	\begin{tikzpicture}[scale=1, font=\footnotesize, line join=round, line cap=round, >=stealth]
	\def\ac{4} % cạnh AC
	\def\ab{2} % cạnh AB
	\def\as{4} % cạnh AS
	\def\gocA{50} % góc A của đáy
	\path
	(0,0) coordinate (A)
	(\ac,0) coordinate (C)
	(-\gocA:\ab) coordinate (B)
	(70:\as) coordinate (S)
	($(S)!.5!(A)$) coordinate (M)
	($(B)!.5!(C)$) coordinate (N)
	;
	\draw (A)--(B)--(C)--(S)--cycle (S)--(B);
	\draw (S)--(A)node[midway,above left]{$a$};
	\draw[dashed] (A)--(C) (M)--(N);
	\foreach \x/\g in {A/180,B/-90,C/0,S/90}\fill (\x) circle (1pt)+(\g:3mm) node[black]{$\x$};
	\end{tikzpicture}
	\end{center}
	\choiceTF
	{\True Độ dài của vectơ $\vec{SA}$ bằng $a$.}
	{\True $\vec{SA} \cdot \vec{SB}=\dfrac{a^2\sqrt{3}}{2}$}
	{$\vec{SB}+\vec{AB}+\vec{SC}+\vec{AC}=4\vec{MN}$}
	{Gọi $I$ là trọng tâm của tứ diện. Khoảng cách từ $I$ đến $(ABC)$ bằng $\dfrac{3a\sqrt{6}}{4}$}
	\loigiai{
	\begin{center}
	\begin{tikzpicture}[scale=1, font=\footnotesize, line join=round, line cap=round, >=stealth]
	\def\ac{4} % cạnh AC
	\def\ab{2} % cạnh AB
	\def\as{4} % cạnh AS
	\def\gocA{50} % góc A của đáy
	\path
	(0,0) coordinate (A)
	(\ac,0) coordinate (C)
	(-\gocA:\ab) coordinate (B)
	(70:\as) coordinate (S)
	($(S)!.5!(A)$) coordinate (M)
	($(B)!.5!(C)$) coordinate (N)
	($(M)!.5!(N)$) coordinate (I)
	($(A)!2/3!(N)$) coordinate (G)
	;
	\draw (A)--(B)--(C)--(S)--cycle (S)--(B) (S)--(N) (M)--(B);
	\draw (S)--(A)node[midway,above left]{$a$};
	\draw[dashed] (A)--(C)--(M)--(N)--(A) (S)--(G) ;
	\foreach \x/\g in {A/180,B/-90,C/0,S/90}\fill (\x) circle (1pt)+(\g:3mm) node[black]{$\x$};
	\end{tikzpicture}
	\end{center}
	\begin{enumerate}[a)]
	\item $|\vec{SA}|=SA=a$.
	\item $\vec{SA} \cdot \vec{SB}=\left| \vec{SA} \right| \cdot \left| \vec{SB} \right| \cdot \sin \widehat{ASB}=a \cdot a \cdot \sin 60^\circ=\dfrac{a^2\sqrt{3}}{2}$.
	\item Do $N$ là trung điểm của $BC$ nên $\vec{SB}+\vec{SC}=2\vec{SN}$ và $\vec{AB}+\vec{AC}=2\vec{MB}$.\\
	Suy ra $\vec{SB}+\vec{SC}+\vec{AB}+\vec{AC}=2\left(\vec{SN}+\vec{AN}\right)$\\
	Do $M$ là trung điểm của $SA$ nên $\vec{NA}+\vec{NS}=2\vec{NM}\Leftrightarrow \vec{AN}+\vec{SN}=2\vec{MN}$.\\
	Do đó $\vec{SB}+\vec{SC}+\vec{AB}+\vec{AC}=2 \cdot 2 \cdot \vec{MN}=4\vec{MN}$.
	\item Gọi $G$ là trọng tâm tam giác $ABC$.\\
	Do tứ diện $SABC$ là tứ diện đều và $I$ là trọng tâm tứ diện nên $d\left(I,(ABC)\right)=IG$\\
	Tam giác $ABC$ đều cạnh $a$, $N$ là trung điểm của $BC$, suy ra $AN=\dfrac{a\sqrt{3}}{2}$.\\
	Do $G$ là trọng tâm tam giác$ABC$ nên $AG=\dfrac{2}{3}AN=\dfrac{a\sqrt{3}}{3}$.\\
	Do tứ diện $SABC$ là tứ diện đều nên $SG\bot (ABC)\Rightarrow SG\bot AG$.\\
	Tam giác $SAG$ vuông tại $G$ nên $SG=\sqrt{SA^2-AG^2}=\sqrt{a^2-\dfrac{a^2}{3}}=\dfrac{a\sqrt{6}}{3}$.\\
	Do $I$ là trọng tâm tứ diện$SABC$ nên $IG=\dfrac{1}{4}SG=\dfrac{1}{4} \cdot \dfrac{a\sqrt{6}}{3}=\dfrac{a\sqrt{6}}{12}$.\\
	Vậy $d\left(I,(ABC)\right)=\dfrac{a\sqrt{6}}{12}$.
	\end{enumerate}
	}
\end{ex}
%%==========Câu 42
\begin{ex}
	\immini{Cho hình hộp chữ nhật $ABCD \cdot EFGH$ có $AB=AE=2$, $AD=3$ và đặt $\vec{a}=\vec{AB},\vec{b}=\vec{AD},\vec{c}=\vec{AE}$. Lấy điểm $M$ thỏa $\vec{AM}=\dfrac{1}{5}\vec{AD}$ và điểm $N$ thỏa $\vec{EN}=\dfrac{2}{5}\vec{EC}$. (tham khảo hình vẽ).
	\choiceTF
	{\True $\vec{MA}=-\dfrac{1}{5}\vec{b}$}
	{\True $\vec{EN}=\dfrac{2}{5}\left(\vec{a}-\vec{b}+\vec{c}\right)$}
	{${{\left(m \cdot \vec{a}+n \cdot \vec{b}+n \cdot \vec{c}\right)}^2}=m^2 \cdot {{\vec{a}}^2}+n^2 \cdot {{\vec{b}}^2}+p^2 \cdot {{\vec{c}}^2}$ với $m,n,p$ là các số thực}
	{\True $MN=\dfrac{\sqrt{61}}{5}$}
	}{\begin{tikzpicture}[line join = round, line cap = round, thick, font = \small, scale = .7]
	\path
	(0:0) coordinate (H)
	+(75:3.5) coordinate (D)
	+(0:3) coordinate (G)
	+(40:2) coordinate (E)
	($(G)+(D)-(H)$) coordinate (C)
	($(D)+(E)-(H)$) coordinate (A)
	($(G)+(E)-(H)$) coordinate (F)
	($(C)+(A)-(D)$) coordinate (B)
	;
	\draw[dashed]
	(E)--(A) (E)--(F) (E)--(H)
	;
	\draw
	(A)--(B)--(F)--(G)--(H)--(D)--cycle
	(C)--(B) (C)--(D) (C)--(G)
	;
	\foreach \x/\g in {H/-90,G/-90,D/180,E/135,C/-45,A/90,F/0,B/90}
	\fill (\x) circle (1.5pt)
	+(\g:3mm) node {$\x$};
	\end{tikzpicture}}
	\loigiai{
	\begin{enumerate}[a)]
	\item $\vec{MA}=-\vec{AM}=-\dfrac{1}{5}\vec{AD}=-\dfrac{1}{5}\vec{b}$.
	\item $\vec{EN}=\dfrac{2}{5}\vec{EC}=\dfrac{2}{5}\left(\vec{EF}+\vec{EH}+\vec{EA}\right)=\dfrac{2}{5}\left(\vec{a}+\vec{b}-\vec{c}\right)$.
	\item ${{\left(m \cdot \vec{a}+n \cdot \vec{b}+p \cdot \vec{c}\right)}^2}=m^2 \cdot {{\vec{a}}^2}+n^2 \cdot {{\vec{b}}^2}+p^2 \cdot {{\vec{c}}^2}+2mn \cdot \vec{a} \cdot \vec{b}+2np \cdot \vec{b} \cdot \vec{c}+2mp \cdot \vec{a} \cdot \vec{c}$\\
	 $=m^2 \cdot {{\vec{a}}^2}+n^2 \cdot {{\vec{b}}^2}+p^2 \cdot {{\vec{c}}^2}$. (vì $\vec{a},\vec{b},\vec{c}$ đôi một vuông góc nên $\vec{a} \cdot \vec{b}=\vec{b} \cdot \vec{c}=\vec{a} \cdot \vec{c}=0$).
	\item $\vec{MN}=\vec{MA}+\vec{AE}+\vec{EN}=-\dfrac{1}{5}\vec{b}+\vec{c}+\dfrac{2}{5}\left(\vec{a}+\vec{b}-\vec{c}\right)=\dfrac{2}{5}\vec{a}+\dfrac{1}{5}\vec{b}+\dfrac{3}{5}\vec{c}$.\\
	 $MN^2={{\vec{MN}}^2}={{\left(\dfrac{2}{5}\vec{a}+\dfrac{1}{5}\vec{b}+\dfrac{3}{5}\vec{c}\right)}^2}=\dfrac{4}{25}{{\vec{a}}^2}+\dfrac{1}{25}{{\vec{b}}^2}+\dfrac{9}{25}{{\vec{c}}^2}=\dfrac{4}{25} \cdot 4+\dfrac{1}{25} \cdot 9+\dfrac{9}{25} \cdot 4=\dfrac{61}{25}$.\\
	 Suy ra $MN=\dfrac{\sqrt{61}}{5}$.
	\end{enumerate}
	}
\end{ex}
%%==========Câu 43
\begin{ex}
	\immini{Cho hình lăng trụ tam giác đều $ABC.A'B'C'$ có cạnh đáy bằng $x$ và chiều cao bằng $y$. (tham khảo hình vẽ)
	\choiceTF
	{\True $\vec{AB} \cdot \vec{AC}=\dfrac{1}{2}x^2$}
	{\True $\vec{AC'}=\vec{AC}+\vec{AA'}$}
	{$\vec{CB'}=\vec{AB}-\vec{CA}+\vec{AA'}$}
	{Góc $\left(AC',CB'\right)>60^\circ $ khi $\dfrac{y}{x}<\sqrt{2}$}
	}{\begin{tikzpicture}[line join = round, line cap = round, thick, font = \small, scale = .6]
	\path
	(0:0) coordinate (A)
	+(0:4) coordinate (C)
	+(-50:2) coordinate (B)
	+(90:4) coordinate (A')
	($(A')+(B)-(A)$) coordinate (B')
	($(A')+(C)-(A)$) coordinate (C')
	;
	\draw[dashed]
	(A)--(C)
	;
	\draw
	(A)--(B)--(C)--(C')--(A')--cycle
	(B')--(A') (B')--(B) (B')--(C')
	;
	\foreach \x/\g in {A/180,B/-90,C/0,A'/-180,B'/70,C'/0}
	\fill (\x) circle (1.5pt)
	+(\g:3mm) node {$\x$};
	\end{tikzpicture}}
	\loigiai{
	\begin{enumerate}[a)]
	\item $\vec{AB} \cdot \vec{AC}=AB \cdot AC \cdot \cos 60^\circ =\dfrac{1}{2}x^2$.
	\item $\vec{AC'}=\vec{AC}+\vec{AA'}$ (vì $ACC'A'$ là hình chữ nhật).
	\item $\vec{CB'}=\vec{CB}+\vec{CC'}=\vec{AB}-\vec{AC}+\vec{AA'}$.
	\item Ta có $\vec{AC'} \cdot \vec{CB'}=\left(\vec{AC}+\vec{AA'}\right) \cdot \left(\vec{AB}-\vec{AC}+\vec{AA'}\right)=y^2-\dfrac{1}{2}x^2$ và $AC'=CB'=\sqrt{x^2+y^2}$.\\
	 Khi đó $\cos \left(AC',CB'\right)=\left| \cos \left(\vec{AC'},\vec{CB'}\right) \right|=\dfrac{\left| \vec{AC'} \cdot \vec{CB'} \right|}{AC' \cdot CB'}=\dfrac{\left| y^2-\dfrac{1}{2}x^2\right|}{x^2+y^2}$.\\
	 Theo đề $\left(AC',CB'\right)>60^\circ $, suy ra $\dfrac{\left| y^2-\dfrac{1}{2}x^2\right|}{x^2+y^2}<\dfrac{1}{2}\Leftrightarrow 3y^4-6x^2y^2<0\Leftrightarrow \dfrac{y}{x}<\sqrt{2}$.
	\end{enumerate}
	}
\end{ex}
\textbf{PHẦN III.} \textit{Câu trắc nghiệm trả lời ngắn.}\\
%%==========Câu 44
\begin{ex}
	\immini{Cho hình lăng trụ $ABC.A'B'C'$. Đặt $\vec{AB}=\vec{a},\vec{AA'}=\vec{b},\vec{AC}=\vec{c}$. Ta biểu diễn $\vec{B'C}=m\vec{a}+n\vec{b}+p\vec{c}$, khi đó $m+n+p$ bằng bao nhiêu?}
	{\begin{tikzpicture}[line join = round, line cap = round, thick, font = \small, scale = .6]
	\path
	(0:0) coordinate (A)
	+(0:4) coordinate (C)
	+(-50:2) coordinate (B)
	+(75:4) coordinate (A')
	($(A')+(B)-(A)$) coordinate (B')
	($(A')+(C)-(A)$) coordinate (C')
	;
	\draw[dashed]
	(A)--(C)
	;
	\draw
	(A)--(B)--(C)--(C')--(A')--cycle
	(B')--(A') (B')--(B) (B')--(C')
	;
	\foreach \x/\g in {A/180,B/-90,C/0,A'/-180,B'/70,C'/0}
	\fill (\x) circle (1.5pt)
	+(\g:3mm) node {$\x$};
	\end{tikzpicture}}
	\loigiai{
	\SA{-1}
	$\vec{B'C}=\vec{B'B}+\vec{BC}=-\vec{BB'}+\vec{BA}+\vec{AC}=-\vec{BB'}-\vec{AB}+\vec{AC}=-\vec{b}-\vec{a}+\vec{c}$\\
	$\Rightarrow \vec{B'C}=-\vec{a}-\vec{b}+\vec{c}$.\\
	Suy ra $m=-1$, $n=-1$, $p=1$. Do đó $m+n+p=-1$.
	}
\end{ex}
%%==========Câu 45
\begin{ex}
	Cho tứ diện $ABCD$, gọi $I$, $J$ lần lượt là trung điểm của $AB$ và $CD$. Biết $\vec{IJ}=\dfrac{a}{b}\vec{AC}+\dfrac{c}{d}\vec{BD}$. Giá trị biểu thức $P=ab+cd$ bằng
	\loigiai{
	\SA{4}
	$\vec{AC}+\vec{BD}=\vec{AI}+\vec{IJ}+\vec{JC}+\vec{BI}+\vec{IJ}+\vec{JD}=2\vec{IJ}\Rightarrow \vec{IJ}=\dfrac{1}{2}\left(\vec{AC}+\vec{BD}\right)$.
	}
\end{ex}
%%==========Câu 46
\begin{ex}
	Cho tứ diện đều $ABCD$ có cạnh bằng $15$. Biết độ dài của $\vec{AB}+\vec{AC}+\vec{AD}$ bằng $a\sqrt{6}$, khi đó giá trị của $a$ là?
	\loigiai{
	\SA{15}
	\immini{
	Gọi $G$ là trọng tâm tâm giác $BCD$, $M$ là trung điểm $CD$.\\
	Ta có $\vec{GB}+\vec{GC}+\vec{GD}=\vec{0}\Leftrightarrow \left(\vec{GA}+\vec{AB}\right)+\left(\vec{GA}+\vec{AC}\right)+\left(\vec{GA}+\vec{AD}\right)=\vec{0}\Leftrightarrow 3\vec{GA}+\left(\vec{AB}+\vec{AC}+\vec{AD}\right)=\vec{0}$\\
	$\Leftrightarrow \vec{AB}+\vec{AC}+\vec{AD}=-3\vec{GA}=3\vec{AG}\Rightarrow | \vec{AB}+\vec{AC}+\vec{AD}|=| 3\vec{AG}|=3AG$.\\
	Xét tam giác đều $BCD$ có $BM=BC \cdot \dfrac{\sqrt{3}}{2}=\dfrac{15\sqrt{3}}{2}\Rightarrow BG=\dfrac{2}{3}BM=5\sqrt{3}$.\\
	Vì tứ diện $ABCD$ đều nên $AG\bot (BCD)\Rightarrow \widehat{AGB}=90^\circ $.\\
	Xét tam giác $ABG$ có $AG=\sqrt{AB^2-BG^2}=\sqrt{{{15}^2}-{{\left(5\sqrt{3}\right)}^2}}=5\sqrt{6}$.\\
	Do đó $| \vec{AB}+\vec{AC}+\vec{AD}|=3AG=15\sqrt{6}\Rightarrow a=15$.\\
	Vậy giá trị của $a=15$.
	}{\begin{tikzpicture}[line join = round, line cap = round, thick, font = \small, scale = .7]
	\path
	(0:0) coordinate (B)
	+(0:5) coordinate (C)
	+(-70:3) coordinate (D)
	(barycentric cs:B=1,C=1,D=1) coordinate (G)
	++(90:4) coordinate (A)
	($(C)!.5!(D)$) coordinate (M)
	;
	\draw[dashed]
	(M)--(B)--(C) (A)--(G)
	;
	\draw
	(A)--(B)--(D)--(C)--cycle
	(A)--(D)
	;
	\foreach \x/\g in {B/180,C/0,D/-90,G/45,A/90,M/-45}
	\fill (\x) circle (1.5pt)
	+(\g:3mm) node {$\x$};
	\end{tikzpicture}}
	}
\end{ex}
%%==========Câu 47
\begin{ex}
	Một chiếc cân đòn tay đang cân một vật có khối lượng $m=3\,\text{kg}$ được thiết kế với đĩa cân được giữ bởi bốn đoạn xích $SA$, $SB$, $SC$, $SD$ sao cho $S.ABCD$ là hình chóp tứ giác đều có $\widehat{ASC}=90^\circ $. Biết độ lớn của lực căng cho mỗi sợi xích có dạng $\dfrac{a\sqrt{2}}{4}$. Lấy $g=10 \mathrm{m}/\mathrm{s}^2$, khi đó giá trị của $a$ bằng bao nhiêu?
	\begin{center}
	\includegraphics{images/candon.png}
	\end{center}
	\loigiai{
	\SA{30}
	\immini{
	Gọi $O$ là tâm của hình vuông $ABCD$.\\
	Ta có $\vec{OA}+\vec{OB}+\vec{OC}+\vec{OD}=\vec{0}\Leftrightarrow \vec{OS}+\vec{SA}+\vec{OS}+\vec{SB}+\vec{OS}+\vec{SC}+\vec{OS}+\vec{SD}=\vec{0}$\\
	$\Leftrightarrow \vec{SA}+\vec{SB}+\vec{SC}+\vec{SD}=-4\vec{OS}=4\vec{SO}\Rightarrow | \vec{SA}+\vec{SB}+\vec{SC}+\vec{SD}| =| 4\vec{SO}|=4SO$.\\
	Trọng lượng của vật nặng là $P=mg=3 \cdot 10=30$ (N). Suy ra $4| \vec{SO}|=P=30$ (N) $\Rightarrow SO=\dfrac{15}{2}$.\\
	Lại có tam giác $ASC$ vuông cân tại $S$ nên\\
	$SO=SA \cdot \sin \widehat{SAC}\Rightarrow SA=\dfrac{SO}{\sin \widehat{SAC}}=\dfrac{\dfrac{15}{2}}{\sin 45^\circ}=\dfrac{15\sqrt{2}}{2}=\dfrac{30\sqrt{2}}{4}\Rightarrow a=30$.\\
	Vậy $a=30$.
	}{\begin{tikzpicture}[line join = round, line cap = round, thick, font = \small, scale = .7]
	\path
	(0:0) coordinate (A)
	+(0:5) coordinate (B)
	+(-140:2.5) coordinate (D)
	($(B)+(D)-(A)$) coordinate (C)
	(intersection of A--C and B--D) coordinate (O)
	++(90:5) coordinate (S)
	;
	\draw[dashed]
	(A)--(B) (A)--(D) (A)--(S) (A)--(C) (B)--(D) (S)--(O)
	;
	\draw
	(D)--(C)--(B)
	(S)--(B) (S)--(C) (S)--(D)
	;
	\foreach \x/\g in {A/135,B/0,C/-45,D/-135,O/-90,S/90}
	\fill (\x) circle (1.5pt)
	+(\g:3mm) node {$\x$};
	\end{tikzpicture}}
	}
\end{ex}
%%==========Câu 48
\begin{ex}
	Cho tứ diện $ABCD$. Trên các cạnh $AD$ và $BC$ lần lượt lấy $M$, $N$ sao cho $AM=3MD$, $BN=3NC$. Gọi $P$, $Q$ lần lượt là trung điểm của $AD$ và $BC$. Phân tích vectơ $\vec{MN}$ theo hai vectơ $\vec{PQ}$ và $\vec{DC}$ ta được $\vec{MN}=a\vec{PQ}+b\vec{DC}$. Tính $a+2b$.
	\loigiai{
	\SA{1,5}
	\begin{center}
	\begin{tikzpicture}[line join = round, line cap = round, thick, font = \small, scale = .7]
	\path
	(0:0) coordinate (B)
	+(0:5) coordinate (D)
	+(-30:4) coordinate (C)
	+(70:4) coordinate (A)
	($(A)!1/2!(D)$) coordinate (P)
	($(P)!1/2!(D)$) coordinate (M)
	($(B)!1/2!(C)$) coordinate (Q)
	($(Q)!1/2!(C)$) coordinate (N)
	;
	\draw[dashed]
	(B)--(D) (P)--(Q) (M)--(N)
	;
	\draw
	(A)--(B)--(C)--(D)--cycle
	(A)--(C)
	;
	\foreach \x/\g in {B/180,D/0,C/-90,A/90,M/45,P/45,N/-135,Q/-135}
	\fill (\x) circle (1.5pt)
	+(\g:3mm) node {$\x$};
	\end{tikzpicture}
	\end{center}
	Do $AM=3MD$, $BN=3NC$ và $P$, $Q$ lần lượt là trung điểm của $AD$ và $BC$ nên $M$, $N$ lần lượt là trung điểm của $PD$ và $QC$. \\
	Ta có $\heva{& \vec{MN}=\vec{MP}+\vec{PQ}+\vec{QN} \\& \vec{MN}=\vec{MD}+\vec{DC}+\vec{CN}}\Rightarrow 2\vec{MN}=\vec{PQ}+\vec{DC}\Rightarrow \vec{MN}=\dfrac{1}{2}\left(\vec{PQ}+\vec{DC}\right)$\\
	$\Rightarrow a=\dfrac{1}{2};\ b=\dfrac{1}{2}\Rightarrow a+2b=\dfrac{3}{2}=1,5$.
	}
\end{ex}
%%==========Câu 49
\begin{ex}
	Cho hình chóp $S.ABCD$ có đáy $ABCD$ là hình bình hành. Một mặt phẳng $(\alpha)$ cắt các cạnh $SA$, $SB$, $SC$, $SD$ lần lượt tại $A',B',C',D'$. Giá trị của biểu thức $P=\dfrac{SA}{SA'}+\dfrac{SC}{SC'}-\dfrac{SB}{SB'}-\dfrac{SD}{SD'}$.
	\loigiai{
	\SA{0}
	\begin{center}
	\begin{tikzpicture}[line join = round, line cap = round, thick, font = \small, scale = 1]
	\path
	(0:0) coordinate (A)
	+(0:5) coordinate (B)
	+(-140:2.5) coordinate (D)
	($(B)+(D)-(A)$) coordinate (C)
	($(A)!.5!(C)$) coordinate (O)
	++(100:5) coordinate (S)
	($(S)!6/13!(A)$) coordinate (A')
	($(S)!.6!(B)$) coordinate (B')
	($(S)!.5!(C)$) coordinate (C')
	($(S)!.4!(D)$) coordinate (D')
	;
	\draw[dashed]
	(A)--(B) (A)--(D) (A)--(S) (A)--(C) (B)--(D) (S)--(O) (B')--(A')--(D')--cycle (A')--(C')
	;
	\draw
	(D)--(C)--(B)
	(S)--(B) (S)--(C) (S)--(D) (B')--(C')--(D')
	;
	\foreach \x/\g in {A/135,B/0,C/-45,D/-135,O/-90,S/90,A'/135,B'/45,C'/-30,D'/135}
	\fill (\x) circle (1.5pt)
	+(\g:3mm) node {$\x$};
	\end{tikzpicture}
	\end{center}
	Gọi $O$ là tâm của hình bình hành $ABCD$ thì $\vec{SA}+\vec{SC}=\vec{SB}+\vec{SD}=2\vec{SO}$\\
	$\Leftrightarrow \dfrac{SA}{SA'}\vec{SA'}+\dfrac{SC}{SC'}\vec{SC'}=\dfrac{SB}{SB'}\vec{SB'}+\dfrac{SD}{SD'}\vec{SD'}$\\
	Do $A',B',C',D'$ đồng phẳng nên $\Rightarrow \dfrac{SA}{SA'}+\dfrac{SC}{SC'}=\dfrac{SB}{SB'}+\dfrac{SD}{SD'}\Rightarrow P=\dfrac{SA}{SA'}+\dfrac{SC}{SC'}-\dfrac{SB}{SB'}-\dfrac{SD}{SD'}=0$.
	}
\end{ex}
%%==========Câu 50
\begin{ex}
	Cho hình lập phương $B'C$ có đường chéo $A'C=\dfrac{3}{16}$. Gọi $O$ là tâm hình vuông $ABCD$ và điểm $20$ thỏa mãn: $\vec{OS}=\vec{OA}+\vec{OB}+\vec{OC}+\vec{OD}+\vec{OA'}+\vec{OB'}+\vec{OC'}+\vec{OD'}$. Khi đó độ dài của đoạn $OS$ bằng $\dfrac{a\sqrt{3}}{b}$ với $a,b\in \mathbb{N}$ và $\dfrac{a}{b}$ là phân số tối giản. Tính giá trị của biểu thức $P=a^2+b^2$.
	\loigiai{
	\SA{17}
	\begin{center}
	\begin{tikzpicture}[line join = round, line cap = round, thick, font = \small, scale = .7]
	\def \canh{4}
	\path
	(0:0) coordinate (D')
	+(90:\canh) coordinate (D)
	+(0:\canh) coordinate (C')
	+(40:.6*\canh) coordinate (A')
	($(C')+(D)-(D')$) coordinate (C)
	($(D)+(A')-(D')$) coordinate (A)
	($(C')+(A')-(D')$) coordinate (B')
	($(C)+(A)-(D)$) coordinate (B)
	($(A)!.5!(C)$) coordinate (O)
	($(A')!.5!(C')$) coordinate (O')
	;
	\draw[dashed]
	(A')--(A) (A')--(B') (C')--(A')--(D')--(B') (O)--(O')
	;
	\draw
	(A)--(B)--(B')--(C')--(D')--(D)--cycle
	(A)--(C)--(B)--(D)--(C) (C)--(C')
	;
	\foreach \x/\g in {D'/-90,C'/-90,D/180,A'/135,C/-45,A/90,B'/0,B/90,O/90,O'/-90}
	\fill (\x) circle (1.5pt)
	+(\g:3mm) node {$\x$};
	\end{tikzpicture}
	\end{center}
	Ta có: $A'C^2=A'A^2+AC^2=3A'A^2\Rightarrow A'A=\dfrac{A'C}{\sqrt{3}}=\dfrac{\sqrt{3}}{16}$.\\
	Gọi $O'$ là tâm của hình vuông $A'B'C'D'$.\\
	Lại có :
	$\begin{aligned}[t]
	\vec{OS}
	 & =\vec{OA}+\vec{OB}+\vec{OC}+\vec{OD}+\vec{OA'}+\vec{OB'}+\vec{OC'}+\vec{OD'} \\
	 & =\left(\vec{OA}+\vec{OC}\right)+\left(\vec{OB}+\vec{OD}\right)+\left(\vec{OA'}+\vec{OC'}\right)+\left(\vec{OB'}+\vec{OD'}\right) \\
	 & =2\vec{OO'}+2\vec{OO'}=4\vec{OO'}
	\end{aligned}$\\
	Suy ra $OS=\left| \vec{OS} \right| =\left| 4\vec{OO'} \right| =4OO'=4 \cdot \dfrac{\sqrt{3}}{16}=\dfrac{\sqrt{3}}{4}$.\\
	Khi đó $a=1,b=4\Rightarrow P=a^2+b^2=17$.
	}
\end{ex}
%%==========Câu 51
\begin{ex}
	Khi chuyển động trong không gian, máy bay luôn chịu tác động của 4 lực chính: lực đẩy của động cơ, lực cản của không khí, trọng lực và lực nâng khí động học (hình ảnh 2.20).
	\begin{center}
	\includegraphics*{images/h2.20.png}
	\end{center}
	Lực cản của không khí ngược hướng với lực đẩy của động cơ và có độ lớn tỉ lệ thuận với bình phương vận tốc máy bay. Một chiếc máy bay tăng vận tốc từ 900(km/h) lên 920(km/h), trong quá trình tăng tốc máy bay giữ nguyên hướng bay. Lực cản của không khí khi máy bay đạt vận tốc 900(km/h) và 920(km/h) lần lượt biểu diễn bởi hai véc tơ $\vec{F_1}$ và $\vec{F_2}$ với $\vec{F_1}=k\vec{F_2}(k\in \mathbb{R};k>0)$. Tính giá trị của $k$ (làm tròn kết quả đến chữ số thập phân thứ hai).
	\loigiai{
	\SA{0,96}
	Vì trong quá trình máy bay tăng vận tốc từ 900(km/h) lên 900(km/h), máy bay giữ nguyên hướng bay nên hai véc tơ $\vec{F_1}$ và $\vec{F_2}$ có cùng hướng và $\vec{F_1}=k\vec{F_2}(k>0)$.\\
	Gọi $v_1,v_2$ lần lượt là vận tốc của chiếc máy bay khi đạt 900(km/h) và 920(km/h).\\
	Suy ra $v_1=900$(km/h), $v_2=920$(km/h).\\
	Vì lực cản của không khí ngược hướng với lực đẩy của động cơ và có độ lớn tỉ lệ thuận với bình phương vận tốc máy bay nên $\left| \dfrac{\vec{F_1}}{\vec{F_2}} \right| =\dfrac{v_1^2}{v_2^2}=\dfrac{900^2}{920^2}=\dfrac{2025}{2116}\Rightarrow \left| \vec{F_1} \right| =\dfrac{2025}{2116}\left| \vec{F_2} \right|\Rightarrow \vec{F_1}=\dfrac{2025}{2116}\vec{F_2}$.\\
	Từ đó suy ra: $k=\dfrac{2025}{2116}\approx 0{,}96$.
	}
\end{ex}
%%==========Câu 52
\begin{ex}
	\immini{
	Một chiếc đèn tròn được treo song song với mặt phẳng nằm ngang bởi ba sợi dây không dãn xuất phát từ điểm $O$ trên trần nhà và lần lượt buộc vào ba điểm $A$, $B$, $C$ trên đèn tròn sao cho các lực căng $\vec{F_1}$, $\vec{F_2}$, $\vec{F_3}$ lần lượt trên mỗi dây $OA$, $OB$, $OC$ đôi một vuông góc với nhau và $\left| \vec{F_1} \right| = \left| \vec{F_2} \right| = \left| \vec{F_3} \right|$ = $15$ (N). Tính trọng lượng của chiếc đèn tròn đó (làm tròn đến hàng phần chục).
	}{\hspace{1cm}
	\begin{tikzpicture}[scale=0.55, font=\footnotesize, line
	join=round, line cap=round]
	%Toa do cac diem
	\coordinate (O) at (0,5);
	\coordinate (A) at (-2.2163, -0.4626);
	\coordinate (C) at (2.5, 0.1);
	\coordinate (B) at (-1.245, 0.867);
	\coordinate (A_1) at ($(O)!1/2!(A)$);
	\coordinate (B_1) at ($(O)!3/4!(B)$);
	\coordinate (C_1) at ($(O)!1/2!(C)$);
	%Ve hai day
	\draw[fill=blue!30] (0,-0.44) ellipse ({2.5} and {1});
	\draw[fill=blue!30] (0,0) ellipse ({2.5} and {1});
	%Ve cac doan thang
	\draw[line width=3] (-3,5)--(3,5);
	\draw(O)--(A);
	\draw(O)--(B);
	\draw(O)--(C);
	\draw[dashed] (0,0)--(C);
	\draw[dashed] (0,0)--(0,-2.5/2);
	%Ve cac diem
	\draw(A) node[above right]{$A$};
	\draw(B) node[above right]{$B$};
	\draw(C) node[above right]{$C$};
	\draw(C) node[above right]{$C$};
	\draw (0.2,5) node [below right]{$O$};
	\draw [fill=red] (0,0) circle (3.2pt);
	%Ve cac vecto
	\draw[-stealth, very thick, blue] (O)--(A_1) node [pos=0.5, left]{$\vec{F_1}$};
	\draw[-stealth, very thick, blue] (O)--(B_1) node [pos=0.5, right]{$\vec{F_2}$};
	\draw[-stealth, very thick, blue] (O)--(C_1) node [pos=0.5, right]{$\vec{F_3}$};
	\draw[-stealth,very thick](0,-2.5/2)--(0,-3.5) node [pos=0.5, right]{$\vec{P}$};
	\end{tikzpicture}}
	\loigiai{
	\SA{26,0}
	\immini{
	Gọi $A_1$, $B_1$, $C_1$ lần lượt là các điểm sao cho $\vec{OA_1} = \vec{F_1}$, $\vec{OB_1} = \vec{F_2}$, $\vec{OC_1} = \vec{F_3}$. Lấy các điểm $D_1$, $A_1'$, $B_1'$, $D_1'$ sao cho $OA_1D_1B_1.C_1A_1'D_1'B_1'$ là hình hộp (như hình bên). Khi đó, áp dụng quy tắc hình hộp ta có
	$$\vec{OA_1}+\vec{OB_1}+\vec{OC_1}=\vec{OD_1'}.$$
	Mặt khác, do các lực căng $\vec{F_1}$, $\vec{F_2}$, $\vec{F_3}$ đôi một vuông góc và $\left| \vec{F_1} \right| = \left| \vec{F_2} \right| = \left| \vec{F_3} \right|$ = $15$ (N) nên hình hộp $OA_1D_1B_1.C_1A_1'D_1'B_1'$ có ba cạnh $OA_1$, $OB_1$, $OC_1$ đôi một vuông góc và bằng nhau. Vì thế hình hộp đó là hình lập phương có độ dài cạnh bằng $15$. Suy ra độ dài đường chéo $OD_1'$ của hình lập phương đó bằng $15 \sqrt{3}$.\\
	Do chiếc đèn ở vị trí cân bằng nên $\vec{F_1}+\vec{F_2}+\vec{F_3}=\vec{P}$, ở đó $\vec{P}$ là trọng lực tác dụng lên chiếc đèn. Suy ra trọng lượng của chiếc đèn là $\left| \vec{P} \right| = \left| \vec{OD_1'} \right| =15\sqrt{3}$ (N).
	}{\begin{tikzpicture}[scale=0.5, font=\footnotesize, line join=round, line cap=round]
	\foreach \x\y\t in {0/0/B_1,-3.5/-3.5/D_1,2.5/-2.5/B_1',-0.6/3/O}
	\coordinate (\t) at (\x,\y);
	\coordinate (D_1') at ($(B_1')+(D_1)-(B_1)$);
	\coordinate (A_1) at ($(O)+(D_1)-(B_1)$);
	\coordinate (A_1') at ($(A_1)+(D_1')-(D_1)$);
	\coordinate (C_1) at ($(O)+(B_1')-(B_1)$);
	\draw (D_1)--(D_1')--(B_1');
	\draw (D_1)--(A_1)--(A_1')--(D_1');
	\draw (A_1')--(C_1)--(B_1');
	\draw[dashed] (D_1)--(B_1)--(B_1');
	\draw[-stealth] (O)--(A_1)node [pos=0.5, above left]{$\vec{F_1}$};
	\draw[-stealth] (O)--(C_1) node [pos=0.5, above right]{$\vec{F_3}$};
	\draw[dashed, -stealth] (O)--(B_1) node [pos=0.5, right]{$\vec{F_2}$};
	\draw[dashed, -stealth] (O)--(D_1');
	\foreach \t/\g in {B_1/-80,D_1/180,D_1'/-30,B_1'/0,O/100,A_1/180,A_1'/-160,C_1/0} \draw (\t) node[shift={(\g:10pt)}]{$\t$};
	\end{tikzpicture}}
	}
\end{ex}
%%==========Câu 53
\begin{ex}
	Cho hình hộp $ABCD.A'B'C'D'$. Xét các điểm $M$, $N$ lần lượt thuộc các đường thẳng $A'C$, $C'D$ sao cho đường thẳng $MN$ song song với đường thẳng $BD'$. Khi đó tỉ số $\dfrac{MN}{BD'}$ bằng
	\loigiai{
	\shortans{0,25}
	\begin{center}
	\begin{tikzpicture}[line join = round, line cap = round, thick, font = \small, scale = 1]
	\path 
	(0:0) coordinate (D)
	+(100:3) coordinate (D')
	+(0:3) coordinate (C)
	+(35:2) coordinate (A)
	($(C)+(D')-(D)$) coordinate (C')
	($(D')+(A)-(D)$) coordinate (A')
	($(C)+(A)-(D)$) coordinate (B)
	($(C')+(A')-(D')$) coordinate (B')
	($(C)!1/4!(A')$) coordinate (M)
	($(C')!.5!(D)$) coordinate (N)
	;
	\draw[dashed] 
	(A')--(A) (A)--(B) (A)--(D) (B)--(D') (A')--(C) (M)--(N)
	;
	\draw 
	(A')--(B')--(B)--(C)--(D)--(D')--cycle
	(C')--(B') (C')--(D') (D)--(C')--(C)
	;
	\foreach \x/\g in {D/-90,C/-90,D'/180,A/-60,C'/-45,A'/90,B/0,B'/90,M/-135,N/135}
	\fill (\x) circle (1.5pt)
	+(\g:3mm) node {$\x$};
	\end{tikzpicture}
	\end{center}
	Đặt $\vec{BA}=\vec{x}$, $\vec{BB'}=\vec{y}$, $\vec{BC}=\vec{z}$.\\
	Do $\vec{CM}$, $\vec{CA'}$ là hai vectơ cùng phương $\Rightarrow \exists \,k\in \mathbb{R}\colon \,\vec{CM}=k \cdot \vec{CA'}$.\\
	Và $\vec{C'N}$, $\vec{C'D}$ là hai vectơ cùng phương $\Rightarrow \exists \,h\in \mathbb{R}\colon \,\vec{C'N}=h \cdot \vec{C'D}$.\\
	Ta có: $\vec{BD'}=\vec{BA}+\vec{BC}+\vec{BB'}=\vec{x}+\vec{y}+\vec{z}$, \hfill (1)\\
	$\begin{aligned}[t]
	\vec{MN} & =\vec{CN}-\vec{CM}=\vec{CC'}+\vec{C'N}-\vec{CM}=\vec{CC'}+h \cdot \vec{C'D}-k \cdot \vec{CA'} \\
	& =\vec{y}+h \cdot (-\vec{y}+\vec{x})-k \cdot \left(\vec{y}-\vec{z}+\vec{x}\right)=(h-k) \cdot \vec{x}+(1-h-k) \cdot \vec{y}+k \cdot \vec{z}
	\end{aligned}$ \hfill (2)\\
	Do $MN\parallel B'D$ nên tồn tại $t\in \mathbb{R} \colon \vec{MN}=t \cdot \vec{BD'}$.\\
	Từ (1) và (2) ta có$\heva{& h-k=t \\& 1-h-k=t \\& k=t}\Leftrightarrow \heva{& k=t \\& h=2t \\& 1-3t=t}\Rightarrow t=\dfrac{1}{4}\Rightarrow \vec{MN}=\dfrac{1}{4}\vec{BD'}$.\\
	Vậy $\dfrac{MN}{BD'}=\dfrac{1}{4}=0,25$.
	}
\end{ex}
\Closesolutionfile{ans}
% \begin{dang}{Xác định góc và tính tích vô hướng của hai véctơ}
\end{dang}
\boxmini{BÀI TẬP TỰ LUẬN}
\setcounter{vd}{0}
\begin{vd}
	Cho hình lập phương $ ABCD.A'B'C'D' $ có cạnh bằng 5.
	\begin{tasks}
	\task Tìm góc giữa các cặp véc-tơ sau: $\vec{AC}$ và $\vec{AB}$; $\vec{AC}$ và $\vec{B'D'}$; $\vec{AC}$ và $\vec{CD}$; $\vec{AD'}$ và $\vec{BD}$.
	\task Tính các tích vô hướng $\vec{AC}\cdot \vec{AB}$; $\vec{AC}\cdot \vec{B'D'}$; $\vec{AD'}\cdot\vec{BD}$;
	\task Chứng minh $\vec{AC'}$ vuông góc với $\vec{BD}$.
	\end{tasks}
	\loigiai{
	\immini{\begin{enumerate}[a)]
	\item Ta có :
	 \begin{itemize}
	 \item [$\bullet$] $\left( \vec{AC},\vec{AB}\right)=\widehat{CAB}=45^\circ$
	 \item [$\bullet$] $\left(\vec{AC},\vec{B'D'}\right)=\left(\vec{AC},\vec{BD}\right)=90^\circ$.
	 \item [$\bullet$] $\left(\vec{AC},\vec{CD}\right)=\left(\vec{CE},\vec{CD}\right)=180^\circ-45^\circ=135^\circ$ (E là điểm đối xứng của $A$ qua $C$).
	 \item [$\bullet$] $ \vec{AD'}=\vec{BC'} \Rightarrow \left(\vec{AD'},\vec{BD}\right)=\left(\vec{BC'},\vec{BD}\right)=\widehat{C'BD}$.
	 Lại có, tam giác $ C'BD $ là tam giác đều nên $ \widehat{C'BD}=60^\circ\Rightarrow \left(\vec{AD'},\vec{BD}\right)=60^\circ $.
	 \end{itemize}
	\item Ta có $AC=BD=B'D'=5\sqrt{2}$. Suy ra
	 \begin{itemize}
	 \item [$\bullet$] $\vec{AC}\cdot \vec{AB}=AC.AB.\cos45^\circ =25$.
	 \item [$\bullet$] Do $AC$ vuông góc $B'D'$ nên $\vec{AC}\cdot \vec{B'D'}=0$.
	 \item [$\bullet$] $\vec{AD'}\cdot \vec{BD}=AD'.BD.\cos60^\circ =5\sqrt{2}.5\sqrt{2}.\dfrac{1}{2}=25$.
	 \end{itemize}
	\item Ta cần chứng minh $\vec{AC'}\cdot \vec{BD}=0$.\\
	 Ta có: $\vec{AC'}=\vec{AB}+\vec{AD}+\vec{AA'}$ và $\vec{BD}=\vec{AD}-\vec{AB}$ nên
	 \begin{eqnarray*}
	 \vec{AC'}\cdot \vec{BD}
	 &=&\left( \vec{AB}+\vec{AD}+\vec{AA'}\right)\cdot \left(\vec{AD}-\vec{AB} \right) \\
	 &=&\vec{AB}.\vec{AD}-\vec{AB}^2+\vec{AD}^2-\vec{AD}.\vec{AB}+\vec{AA'}.\vec{AD}-\vec{AA'}.\vec{AB}=5^2-5^2=0
	 \end{eqnarray*}
	 Suy ra $\vec{AC'}$ vuông góc với $\vec{BD}$.
	\end{enumerate}}{
	\begin{tikzpicture}[scale=0.6, font=\footnotesize, line join=round, line cap=round]
	\def\h{4}
	\foreach \x\y\t in {0/0/A,-1/-1.1/B,2.6/-1.1/C}
	\coordinate (\t) at (\x,\y);
	\coordinate (D) at ($(A)+(C)-(B)$);
	\coordinate (A') at ($(A)+(0,3.2)$);
	\coordinate (B') at ($(B)+(0,3.2)$);
	\coordinate (C') at ($(C)+(0,3.2)$);
	\coordinate (D') at ($(D)+(0,3.2)$);
	\coordinate (E) at ($2*(C)-(A)$);
	\draw (B')--(A')--(D')--(C')--(B')--(B)--(C)--(D)--(D') (C')--(C)--(E);
	\draw[dashed](B)--(A)--(D) (A)--(A') (A)--(C);
	\foreach \t/\g in {A/170,B/-150,C/-100,D/0,A'/100,B'/170,C'/-20,D'/50,E/0}
	\draw[fill=black] (\t) circle(1pt)
	node[shift={(\g:7pt)}]{$\t$};
	\end{tikzpicture}
	}
	}
\end{vd}

\begin{vd}%[2H2H1-3]
	Cho tứ diện đều $ABCD$ có cạnh bằng $a$ và $M$ là trung điểm của $CD$.
	\begin{listEX}[2]
	\item Tính các tích vô hướng $\vec{AB} \cdot \vec{AC}$, $\vec{AB} \cdot \vec{AM}$.
	\item Tính góc $(\vec{AB}, \vec{CD})$.
	\end{listEX}
	\loigiai{
	\begin{enumerate}
	\item Ta có $\begin{aligned}[t]
	 \vec{AC} \cdot \vec{AC} & = |\vec{AB}| \cdot |\vec{AC}| \cdot \cos (\vec{AB},\vec{AC}) \\
	 & = AB \cdot AC \cdot \cos \widehat{BAC} \\
	 & = a \cdot a \cdot \cos {60^0} \\
	 & = \frac{a^2}{2}.
	 \end{aligned}$\\
	 Tương tự ta cũng có $\vec{AB} \cdot \vec{AD} = \dfrac{a^2}{2}$.
	 \immini{
	 Ta lại có $\vec{AM} = \dfrac{1}{2}(\vec{AC} + \vec{AD})$, suy ra
	 \[ \vec{AB} \cdot \vec{AM} = \vec{AB} \cdot \frac{1}{2}(\vec{AC} +\vec{AD}) = \frac{1}{2}(\vec{AB} \cdot \vec{AC} + \vec{AB} \cdot \vec{AD}) = \frac{1}{2}\left(\frac{a^2}{2} + \frac{a^2}{2}\right) = \frac{a^2}{2}.\]
	\item Ta có $\vec{AB} \cdot \vec{CD}=(\vec{AM}+\vec{MB}) \cdot \vec{CD}=\vec{AM} \cdot \vec{CD} +\vec{MB} \cdot \vec{CD}$.\\
	 Mà $AM$, $BM$ là trung tuyến của các tam giác đều $ACD$, $BCD$ nên $\vec{AM} \perp \vec{CD}, \vec{MB} \perp \vec{CD}$.\\
	 Suy ra $\vec{AM} \cdot \vec{CD} = \vec{MB} \cdot \vec{CD} = 0$.\\
	 Từ các kết quả trên ta có $\vec{AM} \cdot \vec{CD}=0$.
	 Suy ra $(\vec{AB}, \vec{CD})=90^\circ$.
	 }{
	 \begin{tikzpicture}[scale=1, font=\footnotesize, line join=round, line cap=round, >=Stealth]
	 \def\a{3}
	 \path
	 (0:0) coordinate (B)
	 (5:\a) coordinate (D)
	 ($(D)+(-135:\a/2)$) coordinate (C)
	 ($(B)+(65:\a)$) coordinate (A)
	 ($(C)!.5!(D)$) coordinate (M)
	 ;
	 \draw[->] (A)--(D);
	 \draw[->] (A)--(C);
	 \draw[->] (A)--(B);
	 \draw[->] (A)--(M);
	 \draw[dashed] (B)--(D);
	 \draw (B)--(C)--(D);
	 \foreach \x/\g in {A/90,B/180,C/-90,D/0,M/0}
	 \draw[fill=black] 	(\x) circle (.5pt)
	 ($(\g:.3)+(\x)$) node {$\x$};
	 \end{tikzpicture}
	 }
	\end{enumerate}
	}
\end{vd}

\begin{vd}
	\immini{Cho biết công $A$ (đơn vị: $J$) sinh bởi lực $\vec{F}$ tác dụng lên một vật được tính bằng công thức $A = \vec{F}\cdot\vec{d}$, trong đó $\vec{d}$ là vectơ biểu thị độ dịch chuyển của vật (đơn vị của $\left|\vec{d}\right|$ là m) khi chịu tác dụng của lực $\vec{F}$.
	}{
	\begin{tikzpicture}[scale=0.68,font=\footnotesize, line join=round, line cap=round, >=stealth]
	\def\xe{orange!60}
	\def\den{red}
	\draw (0,3)--(10,0) (0,0)--(10,0);
	\fill[black!30] (2.48,2.5)--(2.57,2.75)--(5.8,1.7)--(5.73,1.47)--cycle ;
	\fill[black!70] (2.48,2.5)--(2.57,2.75)--(5.2,2)--(5.18,1.6)--cycle ;
	\fill[black!20] (2.57,2.75)--(3,3.9)--(5.2,3.2)--(4.78,2)--cycle ;
	\fill[\xe] (4.78,2)--(5.13,3)--(5.7,2.7)--(5.9,2.2)--(5.75,1.72)--cycle ;
	\fill[green] (5.2,2.8)--(5.65,2.56)--(5.8,2.2)--(5.72,1.94)--(5,2.2)--cycle ;
	\fill[black!70] (3.2,2.36) circle (0.3) (5,1.81) circle (0.3) ;
	\fill[black!20] (3.2,2.36) circle (0.2) (5,1.81) circle (0.2) ;
	\fill[blue](3.2,2.36) circle (3pt) (5,1.81) circle (3pt) (4,2.7) circle (2pt) ;
	\draw[dashed] (4,2.7)--(3.3,0.3) ;
	\draw[->] (4,2.7)--(4,0.3) node[right] {$\vec{P}$};
	\draw[->] (4,2.7)--(10,0.8) node[above] {$\vec{d}$};
	\end{tikzpicture} }
	Một chiếc xe có khối lượng $1{,}5$ tấn đang đi xuống trên một đoạn đường dốc có góc nghiêng $5^\circ$ so với phương ngang. Tính công sinh bởi trọng lực $\vec{P}$ khi xe đi hết đoạn đường dốc dài $30$ m (làm tròn kết quả đến hàng đơn vị), biết rằng trọng lực $\vec{P}$ được xác định bởi công thức $\vec{P} = m\vec{g}$, với $m$ (đơn vị: kg) là khối lượng của vật và $\vec{g}$ là gia tốc rơi tự do có độ lớn $g = 9{,}8$ m/s$^2$.
	\loigiai{
	\begin{center}
	\begin{tikzpicture}[scale=0.8,font=\footnotesize, line join=round, line cap=round, >=stealth]
	\def\xe{orange!60}
	\def\den{red}
	\draw (0,3)--(10,0) (0,0)--(10,0);
	\fill[black!30] (2.48,2.5)--(2.57,2.75)--(5.8,1.7)--(5.73,1.47)--cycle ;
	\fill[black!70] (2.48,2.5)--(2.57,2.75)--(5.2,2)--(5.18,1.6)--cycle ;
	\fill[black!20] (2.57,2.75)--(3,3.9)--(5.2,3.2)--(4.78,2)--cycle ;
	\fill[\xe] (4.78,2)--(5.13,3)--(5.7,2.7)--(5.9,2.2)--(5.75,1.72)--cycle ;
	\fill[green] (5.2,2.8)--(5.65,2.56)--(5.8,2.2)--(5.72,1.94)--(5,2.2)--cycle ;
	\fill[black!70] (3.2,2.36) circle (0.3) (5,1.81) circle (0.3) ;
	\fill[black!20] (3.2,2.36) circle (0.2) (5,1.81) circle (0.2) ;
	\fill[blue](3.2,2.36) circle (3pt) (5,1.81) circle (3pt) (4,2.7) circle (2pt) ;
	\draw[dashed] (4,2.7)--(3.3,0.3) ;
	\draw[->] (4,2.7)--(4,0.3) node[right] {$\vec{P}$};
	\draw[->] (4,2.7)--(10,0.8) node[above] {$\vec{d}$};
	\end{tikzpicture}
	%\includegraphics{hinhanh/H1.png} 
	\end{center}
	\noindent
	Ta có $1{,}5$ tấn = $1~500$ kg.\\
	Độ lớn của trọng lực tác dụng lên chiếc xe là $\left|\vec{P}\right| = m \left|\vec{g}\right| = 1~500\cdot 9{,}8 = 14~700$ (N).\\
	Vectơ d biểu thị độ dịch chuyển của xe có độ dài là $\left|\vec{d}\right| = 30$ (m) và $ \left(\vec{P},\vec{d}\right)= 90^\circ - 5^\circ = 85^\circ$.\\
	Công sinh ra bởi trọng lực $\vec{P}$ khi xe đi hết đoạn đường dốc dài $30$ m là
	$$A=\vec{P}\cdot\vec{d}=\left|\vec{P}\right|\cdot\left|\vec{d}\right|\cdot\cos\left(\vec{P},\vec{d}\right) = 14~700\cdot 30\cdot \cos 85^\circ \approx 38~436~(J).$$
	}
\end{vd}

\begin{vd}
	\immini{
	Một chất điểm $A$ nằm trên mặt phẳng nằm ngang $\left(\alpha\right)$, chịu tác động bởi ba lực $\vec{F}_1$, $\vec{F}_2$, $\vec{F}_3$. Các lực $\vec{F}_1$, $\vec{F}_2$ có giá nằm trong $\left(\alpha\right)$ và $\left(\vec{F}_1, \vec{F}_2\right)=135^\circ$, còn lực $\vec{F}_3$ có giá vuông góc với $\left(\alpha\right)$ và hướng lên trên. Xác định cường độ hợp lực của các lực $\vec{F}_1$, $\vec{F}_2$, $\vec{F}_3$ biết rằng độ lớn của ba lực đó lần lượt là $20$ N, $15$ N và $10$ N.
	}{
	\begin{tikzpicture}[line join=round, line cap = round, >=stealth, scale=0.75,font=\footnotesize]
	\path
	(0,0) coordinate (A)
	(-2,-1.5) coordinate (B)
	(3,0) coordinate (C)
	(0,3) coordinate (D)
	;
	\draw[->] (A)--(B) node[below] {$\vec{F}_1$};
	\draw[->] (A)--(C) node[below] {$\vec{F}_2$};
	\draw[->] (A)--(D) node[right] {$\vec{F}_3$};
	\draw pic[draw,angle eccentricity=1.2,angle radius=0.2cm]{right angle= D--A--C};
	\draw pic[draw,angle eccentricity=1.2,angle radius=0.2cm]{right angle= D--A--B};
	\draw pic[draw,angle eccentricity=1.2,angle radius=0.3cm]{angle= B--A--C};
	\draw ($(A)-(-0.3,0.6)$) node {$135^\circ$};
	\foreach \x/\g in {A/140} \fill[black](\x) circle (1pt) ($(\x)+(\g:4mm)$)node{$\x$};
	\end{tikzpicture}
	}
	\loigiai{
	Gọi $\vec{F}$ là hợp lực của các lực $\vec{F}_1$, $\vec{F}_2$, $\vec{F}_3$, tức là $\vec{F}= \vec{F}_1+ \vec{F}_2+ \vec{F}_3$, ta có
	\allowdisplaybreaks
	\begin{eqnarray*}
	\left|\vec{F}\right|^2 &=& \left(\vec{F}_1+ \vec{F}_2+ \vec{F}_3\right)^2\\
	&=& \vec{F}_1^2+ \vec{F}_2^2+ \vec{F}_3^2+ 2\vec{F}_1\cdot\vec{F}_2+ 2\vec{F}_2\cdot\vec{F}_3+ 2\vec{F}_3\cdot\vec{F}_1\\
	&=& 20^2+ 15^2+ 10^2+ 2\cdot 20\cdot 15\cdot\cos 135^\circ\\
	&=& 725-300\sqrt2.
	\end{eqnarray*}
	Vậy $\left|\vec{F}\right|= \sqrt{725-300\sqrt2} \approx 17{,}34$ (N).
	}
\end{vd}

\boxmini{BÀI TẬP TRẮC NGHIỆM}
\textbf{PHẦN I.} \textit{Câu trắc nghiệm nhiều phương án lựa chọn. Mỗi câu hỏi học sinh chỉ chọn một phương án.}\\
\setcounter{ex}{0}
\Opensolutionfile{ans}[ans/2H2-B1-d2-1]
%%==========Câu 1
\begin{ex}
	\immini{Cho hình lập phương $ABCD.A'B'C'D'$. Khẳng định nào sau đây là khẳng định \textbf{sai}?
	\choice
	{ $\left(\vec{A'C'},\vec{AD}\right)=45^\circ$}
	{$\left(\vec{A'C'},\vec{B'B}\right)=90^\circ$}
	{\True $\left(\vec{A'A}, \vec{CB'}\right)=45^\circ$}
	{$\left(\vec{AB},\vec{CD}\right)=180^\circ$}
	}{
	\begin{tikzpicture}[scale=0.65, font=\footnotesize,>=stealth]
	%Gán số liệu.
	\def\canhAD{3};\def\canhBA{2};\def\gocBAD{-130};\def\h{3};\def\xdinhA'{0};
	%Gán tọa độ.
	\coordinate (A) at (0,0);
	\coordinate (B) at ($(A)+(\gocBAD:\canhBA)$);
	\coordinate (C) at ($(B)+(0:\canhAD)$);
	\coordinate (D) at ($(A)+(0:\canhAD)$);
	\coordinate (A') at ($(A)+(\xdinhA',\h)$);
	\coordinate (B') at ($(B)+(\xdinhA',\h)$);
	\coordinate (C') at ($(C)+(\xdinhA',\h)$);
	\coordinate (D') at ($(D)+(\xdinhA',\h)$);
	%Vẽ khối lẳng trụ ABCD.A'B'C'D'.
	\draw (A')--(B')--(B)--(C)--(C')--(D')--cycle (B')--(C') (D')--(D)--(C) (A')--(C');
	\draw[dashed] (A)--(D) (A')--(A)--(B) (A)--(C);
	%Gán nhãn.
	\foreach \x/\y in {A/180, B/180, C/0, D/0, A'/180, B'/180, C'/0, D'/0}{\fill (\x) circle(1pt) ($(\x)+(\y:0.3cm)$) node{$\x$};}
	\end{tikzpicture}}
	\loigiai{
	\begin{itemize}
	\item [$\bullet$] Ta có $\left(\vec{A'C'},\vec{AD}\right)=\left(\vec{A'C'},\vec{A'D'}\right)=\widehat{C'A'D'}=45^\circ$.
	\item [$\bullet$] $\left(\vec{A'C'},\vec{B'B}\right)=\left(\vec{A'C'},\vec{A'A}\right)=\widehat{AA'C'}=90^\circ$.
	\item [$\bullet$] Ta có $\vec{B'B}=\vec{A'A}$, suy ra\\
	 $\left(\vec{A'A},\vec{CB'}\right)=\left(\vec{B'B},\vec{CB'}\right)=180^{\circ}-\widehat{BB'C}=180^{\circ}-45^{\circ}=135^{\circ}$
	\item [$\bullet$] $\vec{AB}$ ngược hướng với $\vec{CD}$ nên $\left(\vec{AB},\vec{CD}\right)=180^\circ$.
	\end{itemize}
	}
\end{ex} 
%%==========Câu 2
\begin{ex}
	\immini{Cho tứ diện đều $ABCD$, Gọi $M$, $N$ lần lượt là trung điểm các cạnh $AB$, $AC$. Hãy tính góc giữa hai vectơ $\vec{MN}$ và $\vec{BD}$.
	\choice
	{$ \left(\vec{MN}, \vec{BD} \right) = 150^\circ$}
	{$ \left(\vec{MN}, \vec{BD} \right) = 120^\circ$}
	{$ \left(\vec{MN}, \vec{BD} \right) = 30^\circ$}
	{\True $ \left(\vec{MN}, \vec{BD} \right) = 60^\circ$}}{
	\begin{tikzpicture}[scale=0.55, font=\footnotesize, line join=round, line cap=round]
	\foreach \x\y\t in {0/0/B,6/0/D,1.5/-2/C,1.5/5/A}
	\coordinate (\t) at (\x,\y);
	\coordinate (M) at ($(A)!1/2!(B)$);
	\coordinate (N) at ($(A)!1/2!(C)$);
	\draw (A)--(B)--(C)--(D)--(A)--(C) (M)--(N);
	\draw[dashed] (B)--(D);
	\foreach \t/\g in {A/90,B/180,C/-90,D/0,M/180,N/0} \draw (\t) node[shift={(\g:10pt)}]{$\t$};
	\end{tikzpicture}}
	\loigiai{
	\immini{Xét tam giác $ABC$ có $M$, $N$ là trung điểm của $AB$, $AC$ nên $MN$ là đường trung bình của tam giác $ABC$. Do đó $MN \parallel BC$.\\
	Ta có $ \left(\vec{MN}, \vec{BD} \right)= \left(\vec{BC}, \vec{BD} \right) = \widehat{CBD}$. \\
	Vì $ABCD$ là tứ diện đều nên $BC=CD=DB$. Do đó tam giác $BCD$ đều suy ra $\widehat{CBD} = 60^\circ$.\\
	Vậy $ \left(\vec{MN}, \vec{BD} \right) = 60^\circ$.}
	{\begin{tikzpicture}[scale=0.55, font=\footnotesize, line join=round, line cap=round]
	\foreach \x\y\t in {0/0/B,6/0/D,1.5/-2/C,1.5/5/A}
	\coordinate (\t) at (\x,\y);
	\coordinate (M) at ($(A)!1/2!(B)$);
	\coordinate (N) at ($(A)!1/2!(C)$);
	\draw (A)--(B)--(C)--(D)--(A)--(C);
	\draw[dashed,-stealth,blue,very thick] (B)--(D);
	\draw[-stealth,blue,very thick](M)--(N);
	\foreach \t/\g in {A/90,B/180,C/-90,D/0,M/180,N/0} \draw (\t) node[shift={(\g:10pt)}]{$\t$};
	\end{tikzpicture}}
	}
\end{ex} 
%%==========Câu 3
\begin{ex}
	\immini{Cho hình chóp $S. A B C D$ có đáy $A B C D$ là hình bình hành và mặt bên $S A B$ là tam giác đều. Tính góc giữa hai vectơ $\vec{D C}$ và $\vec{B S}$.
	\haicot
	{\True $\left(\vec{D C}, \vec{B S}\right)=120^{\circ}$}
	{$\left(\vec{D C}, \vec{B S}\right)=60^{\circ}$}
	{$\left(\vec{D C}, \vec{B S}\right)=90^{\circ}$}
	{$\left(\vec{D C}, \vec{B S}\right)=150^{\circ}$}
	}{
	\begin{tikzpicture}[scale=0.8, font=\footnotesize,>=stealth]
	%Gán số liệu.
	\def\canhAD{4};\def\canhBA{2};\def\gocBAD{-130};\def\h{2};\def\xdinhS{-1};
	%Gán tọa độ.
	\coordinate (A) at (0,0);
	\coordinate (B) at ($(A)+(\gocBAD:\canhBA)$);
	\coordinate (C) at ($(B)+(0:\canhAD)$);
	\coordinate (D) at ($(A)+(0:\canhAD)$);
	\coordinate (S) at ($(A)+(\xdinhS,\h)$);
	\draw (B)--(S)--(C)--cycle (S)--(D)--(C);
	\draw[dashed] (A)--(D) (S)--(A)--(B);
	\foreach \x/\y in {A/180,B/-90,C/-90,D/0,S/90}{\fill (\x) circle(1pt) ($(\x)+(\y:0.3cm)$) node{$\x$};}
	\end{tikzpicture}
	}
	\loigiai{
	\immini{Vì $A B C D$ là hình bình hành nên $A B \parallel D C$.\\
	Trên tia $A B$ lấy điểm $E$ sao cho $\vec{B E}=\vec{D C}$ (Hình $2.20$). Ta có
	$$
	\left(\vec{D C}, \vec{B S}\right)=\left(\vec{B E}, \vec{B S}\right)=\widehat{E B S}=180^{\circ}-60^{\circ}=120^{\circ}.
	$$
	Vậy $\left(\vec{D C}, \vec{B S}\right)=120^{\circ}$.
	}{
	\begin{tikzpicture}[scale=0.8, font=\footnotesize,>=stealth]
	%Gán số liệu.
	\def\canhAD{4};\def\canhBA{2};\def\gocBAD{-130};\def\h{2};\def\xdinhS{-1};
	%Gán tọa độ.
	\coordinate (A) at (0,0);
	\coordinate (B) at ($(A)+(\gocBAD:\canhBA)$);
	\coordinate (C) at ($(B)+(0:\canhAD)$);
	\coordinate (D) at ($(A)+(0:\canhAD)$);
	\coordinate (S) at ($(A)+(\xdinhS,\h)$);
	\coordinate (E) at ($(B)!-1!(A)$);
	%Vẽ khối chóp S.ABCD.
	\draw (B)--(S)--(C)--cycle (S)--(D)--(C);
	\draw[dashed] (A)--(D) (S)--(A)--(B);
	\draw[->] (B)--(E);
	\draw[->] (D)--(C);
	%	%Gán nhãn.
	\draw pic["$60^\circ$",draw,angle eccentricity=1.6,angle radius=0.5cm]{angle=A--B--S};
	\draw pic["$120^\circ$",draw,double,angle eccentricity=1.7,angle radius=0.4cm]{angle=S--B--E};
	\foreach \x/\y in {A/180,B/-90,C/-90,D/0,S/90, E/90}{\fill (\x) circle(1pt) ($(\x)+(\y:0.3cm)$) node{$\x$};}
	\end{tikzpicture}
	}
	}
\end{ex} 
%%==========Câu 4
\begin{ex}
	\immini{Cho hình chóp $S.A B C D$ có đáy $A B C D$ là hình bình hành. Mặt bên $A S B$ là tam giác vuông cân tại $S$ và có cạnh $A B=a$. Tính $\vec{D C} \cdot \vec{A S}$.
	\haicot
	{$\dfrac{a^2}{4}$}
	{$-\dfrac{a^2}{4}$}
	{$-\dfrac{a^2}{2}$}
	{\True $\dfrac{a^2}{2}$}}{
	\begin{tikzpicture}[scale=0.79, font=\footnotesize,>=stealth]
	\def\canhAD{4};\def\canhBA{2};\def\gocBAD{-130};\def\h{2};\def\xdinhS{-1};
	%Gán tọa độ.
	\coordinate (A) at (0,0);
	\coordinate (B) at ($(A)+(\gocBAD:\canhBA)$);
	\coordinate (C) at ($(B)+(0:\canhAD)$);
	\coordinate (D) at ($(A)+(0:\canhAD)$);
	\coordinate (S) at ($(A)+(\xdinhS,\h)$);
	\draw (B)--(S)--(C)--cycle (S)--(D)--(C);
	\draw[dashed] (A)--(D) (S)--(A)--(B);
	\draw[->] (D)--(C);
	%\draw pic["$45^\circ$",draw,angle eccentricity=1.6,angle radius=0.3cm]{angle=S--A--B};
	\foreach \x/\y in {A/60,B/-90,C/-90,D/0,S/90}{\fill (\x) circle(1pt) ($(\x)+(\y:0.3cm)$) node{$\x$};}
	\end{tikzpicture}}
	\loigiai{
	\immini{$\vec{D C} \cdot \vec{A S}=\vec{A B} \cdot \vec{A S}=\left|\vec{A B}\right| \cdot\left|\vec{A S}\right| \cdot \cos \left(\vec{A B}, \vec{A S}\right)=a \cdot \dfrac{a \sqrt{2}}{2} \cdot \cos 45^{\circ}=\dfrac{a^2}{2}$.
	}{
	\begin{tikzpicture}[scale=0.9, font=\footnotesize,>=stealth]
	\def\canhAD{4};\def\canhBA{2};\def\gocBAD{-130};\def\h{2};\def\xdinhS{-1};
	%Gán tọa độ.
	\coordinate (A) at (0,0);
	\coordinate (B) at ($(A)+(\gocBAD:\canhBA)$);
	\coordinate (C) at ($(B)+(0:\canhAD)$);
	\coordinate (D) at ($(A)+(0:\canhAD)$);
	\coordinate (S) at ($(A)+(\xdinhS,\h)$);
	\draw (B)--(S)--(C)--cycle (S)--(D)--(C);
	\draw[dashed] (A)--(D) (S)--(A)--(B);
	\draw[->] (D)--(C);
	\draw pic["$45^\circ$",draw,angle eccentricity=1.6,angle radius=0.3cm]{angle=S--A--B};
	\foreach \x/\y in {A/60,B/-90,C/-90,D/0,S/90}{\fill (\x) circle(1pt) ($(\x)+(\y:0.3cm)$) node{$\x$};}
	\end{tikzpicture}}
	}
\end{ex} 
%%==========Câu 5
\begin{ex}
	\immini{Cho hình lập phương $ABCD.EFGH$ có các cạnh bằng $ a $. Tính $\vec{AB}\cdot\vec{EG}$.
	\haicot
	{$a^2\sqrt{2}$}
	{\True $a^2$}
	{$\dfrac{a^2\sqrt{2}}{2}$}
	{$a^2\sqrt{3}$}}{
	\begin{tikzpicture}[scale=0.5, font=\footnotesize, line join=round, line cap=round, >=stealth]
	\coordinate (A) at (0,0);
	\coordinate (B) at (-1.5,-1.5);
	\coordinate (D) at (3,0);
	\coordinate (A') at (0,3);
	\coordinate (C) at ($(D)+(B)-(A)$);
	\coordinate (B') at ($ (A')-(A)+(B) $);
	\coordinate (C') at ($ (A')-(A)+(C) $);
	\coordinate (D') at ($ (A')-(A)+(D) $);
	\draw (B)--(C)--(C')--(B')--cycle (A')--(B')--(C')--(D')--cycle (C)--(D)--(D') (C)--(D);
	\draw[dashed] (B)--(A)--(D) (A)--(A');
	\draw[dashed] (A)--(C);
	\draw[](A')--(C');
	\foreach \p/\t/\q in {A/A/150,B/B/-90,C/C/-90,D/D/30, A'/E/90, B'/F/180, C'/G/-30, D'/H/90} \draw[black,fill=white] (\p) circle(0.8pt)node[shift={(\q:6pt)}]{\color{black}$\t$};
	\end{tikzpicture}}
	\loigiai{
	\immini
	{
	Ta có $\vec{AB}\cdot\vec{EG}=\vec{AB}\cdot\vec{AC}=AB\cdot AC\cdot\cos45^\circ=a\cdot a\sqrt{2}\cdot\dfrac{\sqrt2}{2}=a^2$.
	}
	{
	\begin{tikzpicture}[scale=0.5, font=\footnotesize, line join=round, line cap=round, >=stealth]
	\coordinate (A) at (0,0);
	\coordinate (B) at (-1.5,-1.5);
	\coordinate (D) at (3,0);
	\coordinate (A') at (0,3);
	\coordinate (C) at ($(D)+(B)-(A)$);
	\coordinate (B') at ($ (A')-(A)+(B) $);
	\coordinate (C') at ($ (A')-(A)+(C) $);
	\coordinate (D') at ($ (A')-(A)+(D) $);
	\draw (B)--(C)--(C')--(B')--cycle (A')--(B')--(C')--(D')--cycle (C)--(D)--(D') (C)--(D);
	\draw[dashed] (B)--(A)--(D) (A)--(A');
	\draw[->,dashed] (A)--(C);
	\draw[->](A')--(C');
	\foreach \p/\t/\q in {A/A/150,B/B/-90,C/C/-90,D/D/30, A'/E/90, B'/F/180, C'/G/-30, D'/H/90} \draw[black,fill=white] (\p) circle(0.8pt)node[shift={(\q:6pt)}]{\color{black}$\t$};
	\end{tikzpicture}
	}
	}
\end{ex} 
%%==========Câu 6
\begin{ex}
	\immini{Cho hình lập phương $ABCD.A'B'C'D'$ có cạnh bằng $a$. Tính $ \vec{A B'} \cdot \vec{A' C'}$.
	\haicot
	{$\dfrac{a^2}{2}$}
	{$-a^2$}
	{\True $a^2$}
	{$-\dfrac{a^2}{2}$}
	}{\hspace{1.5cm}
	\begin{tikzpicture}[scale=0.7, font=\footnotesize,>=stealth]
	%Gán số liệu.
	\def\canhAD{3};\def\canhBA{2};\def\gocBAD{-130};\def\h{3};\def\xdinhA'{0};
	%Gán tọa độ.
	\coordinate (A) at (0,0);
	\coordinate (B) at ($(A)+(\gocBAD:\canhBA)$);
	\coordinate (C) at ($(B)+(0:\canhAD)$);
	\coordinate (D) at ($(A)+(0:\canhAD)$);
	\coordinate (A') at ($(A)+(\xdinhA',\h)$);
	\coordinate (B') at ($(B)+(\xdinhA',\h)$);
	\coordinate (C') at ($(C)+(\xdinhA',\h)$);
	\coordinate (D') at ($(D)+(\xdinhA',\h)$);
	%\coordinate (F) at ($(A)!-1!(E)$);
	%\draw pic[draw,angle radius=0.3cm]{right angle=B'--H--E};
	%Vẽ khối lẳng trụ ABCD.A'B'C'D'.
	\draw (A')--(B')--(B)--(C)--(C')--(D')--cycle (B')--(C') (D')--(D)--(C) (A')--(C') ;
	\draw[dashed] (A)--(D) (A')--(A)--(B) (A)--(C) (A)--(B') (B)--(D) ;
	%Gán nhãn.
	\foreach \x/\y in {A/180, B/-90, C/0, D/0, A'/180, B'/180, C'/0, D'/0}{\fill (\x) circle(1pt) ($(\x)+(\y:0.3cm)$) node{$\x$};}
	\end{tikzpicture}}
	\loigiai{
	Ta có $A'C'=AC$.\\
	Vì $AB'=AC=B'C=a\sqrt{2}$ nên tam giác $AB'C$ đều. Suy ra $\widehat{B'AC}=60^\circ$.\\
	Ta có $\begin{aligned}[t]
	\vec{A B'} \cdot \vec{A' C'} & \ =\left|\vec{AB'}\right|\cdot\left|\vec{A'C'}\right|\cdot\cos \left(\vec{AB'},\vec{A'C'}\right) \\
	 & \ = AB'\cdot A'C' \cdot \cos \left(\vec{AB'}, \vec{AC}\right) \\
	 & \ = AB'\cdot A'C' \cdot \cos \widehat{B'AC} \\
	 & \ = a\sqrt{2}\cdot a\sqrt{2}\cdot \cos 60^\circ= a^2.
	\end{aligned}$
	}
\end{ex} 
%%==========Câu 7
\begin{ex}
	\immini{Cho hình lập phương $ABCD.A'B'C'D'$ có cạnh bằng $a$. Tính $\vec{A B'} \cdot \vec{B D} $.
	\haicot
	{$\dfrac{a^2}{2}$}
	{$-a^2$}
	{\True $a^2$}
	{$-\dfrac{a^2}{2}$}
	}{\hspace{1.5cm}
	\begin{tikzpicture}[scale=0.7, font=\footnotesize,>=stealth]
	%Gán số liệu.
	\def\canhAD{3};\def\canhBA{2};\def\gocBAD{-130};\def\h{3};\def\xdinhA'{0};
	%Gán tọa độ.
	\coordinate (A) at (0,0);
	\coordinate (B) at ($(A)+(\gocBAD:\canhBA)$);
	\coordinate (C) at ($(B)+(0:\canhAD)$);
	\coordinate (D) at ($(A)+(0:\canhAD)$);
	\coordinate (A') at ($(A)+(\xdinhA',\h)$);
	\coordinate (B') at ($(B)+(\xdinhA',\h)$);
	\coordinate (C') at ($(C)+(\xdinhA',\h)$);
	\coordinate (D') at ($(D)+(\xdinhA',\h)$);
	%\coordinate (F) at ($(A)!-1!(E)$);
	%\draw pic[draw,angle radius=0.3cm]{right angle=B'--H--E};
	%Vẽ khối lẳng trụ ABCD.A'B'C'D'.
	\draw (A')--(B')--(B)--(C)--(C')--(D')--cycle (B')--(C') (D')--(D)--(C) (A')--(C') ;
	\draw[dashed] (A)--(D) (A')--(A)--(B) (A)--(C) (A)--(B') (B)--(D) ;
	%Gán nhãn.
	\foreach \x/\y in {A/180, B/-90, C/0, D/0, A'/180, B'/180, C'/0, D'/0}{\fill (\x) circle(1pt) ($(\x)+(\y:0.3cm)$) node{$\x$};}
	\end{tikzpicture}}
	\loigiai{
	Ta có $ABCD.A'B'C'D$ là hình lập phương nên $\heva{&AA'\perp AB\\ &AB\perp BC\\ &\vec{AD}=\vec{BC}\\
	&\vec{AB'}=\vec{AA'}+\vec{AB}\\ &\vec{BD}=\vec{BA}+\vec{BC}.}$\\
	Khi đó $\begin{aligned}[t]
	\vec{A B'} \cdot \vec{B D} & \ =\left(\vec{AA'}+\vec{AB}\right)\cdot \left(\vec{BA}+\vec{BC}\right) \\
	 & \ =\vec{AA'}\cdot \vec{BA} +\vec{AA'}\cdot\vec{BC}+\vec{AB}\cdot\vec{BA}+\vec{AB}\cdot \vec{BC}. \\
	 & \ = 0 + 0 - AB^2 + 0 =-a^2.
	\end{aligned}$
	}
\end{ex} 
%%==========Câu 8
\begin{ex}
	\immini{Cho hình chóp tứ giác đều $S . A B C D$ có độ dài tất cả các cạnh bằng $a$. Tính $\vec{A S} \cdot \vec{B C}$.
	\haicot
	{$-\dfrac{a^2}{4}$}
	{\True $\dfrac{a^2}{2}$}
	{$-\dfrac{a^2}{2}$}
	{$\dfrac{a^2}{4}$}
	}{
	\begin{tikzpicture}[line join=round, line cap = round, >=stealth, scale=.6,font=\footnotesize]
	\def\a{4}
	\def\h{3}
	\path 	(0:0) coordinate (A)
	++(0:\a) coordinate (D)
	++(-150:\a/2) coordinate (C)
	($(A)+(C)-(D)$) coordinate (B)
	(intersection of A--C and B--D) coordinate (O)
	($(O)+(90:\h)$) coordinate (S);
	\draw[dashed] 	(A)--(D)
	(B)--(D)	(S)--(O)	;
	\draw	(C)--(D)
	(B)--(S)	(C)--(S)	(D)--(S);
	\foreach \x/\g in {A/135,B/-135,C/-45,D/45,S/90,O/-90}
	\fill[black] 	(\x) circle (1.5pt)
	($(\g:3mm)+(\x)$) node {$\x$};
	\draw[] (B)--(C);
	\draw[dashed] (A)--(S) (A)--(C) (A)--(B);
	\end{tikzpicture}	}
	\loigiai{
	Tam giác $S A D$ có ba cạnh bằng nhau nên là tam giác đều, suy ra $\widehat{S A D}=60^{\circ}$.\\
	Tứ giác $A B C D$ là hình vuông nên $\vec{A D}=\vec{B C}$, suy ra $(\vec{A S}, \vec{B C})=(\vec{A S}, \vec{A D})=\widehat{S A D}=60^{\circ}$.\\
	Do đó $\vec{A S} \cdot \vec{B C}=|\vec{A S}| \cdot|\vec{B C}| \cdot \cos 60^{\circ}=a \cdot a \cdot \dfrac{1}{2}=\dfrac{a^2}{2}$.
	}
\end{ex} 
%%==========Câu 9
\begin{ex}
	\immini{Cho hình chóp tứ giác đều $S . A B C D$ có độ dài tất cả các cạnh bằng $a$. Tính $\vec{A S} \cdot \vec{A C}$.
	\haicot
	{$-a^2$}
	{$\dfrac{a^2}{2}$}
	{$-\dfrac{a^2}{2}$}
	{\True $a^2$}
	}{
	\begin{tikzpicture}[line join=round, line cap = round, >=stealth, scale=.6,font=\footnotesize]
	\def\a{4}
	\def\h{3}
	\path 	(0:0) coordinate (A)
	++(0:\a) coordinate (D)
	++(-150:\a/2) coordinate (C)
	($(A)+(C)-(D)$) coordinate (B)
	(intersection of A--C and B--D) coordinate (O)
	($(O)+(90:\h)$) coordinate (S);
	\draw[dashed] 	(A)--(D)
	(B)--(D)	(S)--(O)	;
	\draw	(C)--(D)
	(B)--(S)	(C)--(S)	(D)--(S);
	\foreach \x/\g in {A/135,B/-135,C/-45,D/45,S/90,O/-90}
	\fill[black] 	(\x) circle (1.5pt)
	($(\g:3mm)+(\x)$) node {$\x$};
	\draw[] (B)--(C);
	\draw[dashed] (A)--(S) (A)--(C) (A)--(B);
	\end{tikzpicture}	}
	\loigiai{
	Tứ giác $A B C D$ là hình vuông có độ dài mỗi cạnh là a nên độ dài đường chéo $A C$ là $\sqrt{2} a$.\\
	Tam giác $S A C$ có $S A=S C=a$ và $A C=\sqrt{2} a$ nên tam giác $S A C$ vuông cân tại $S$, suy ra $\widehat{S A C}=45^{\circ}$.\\
	Do đó $\vec{A S} \cdot \vec{A C}=|\vec{A S}| \cdot|\vec{A C}| \cdot \cos \widehat{S A C}=a \cdot \sqrt{2} a \cdot \dfrac{\sqrt{2}}{2}=a^2$.
	}
\end{ex} 
%%==========Câu 10
\begin{ex}
	\immini{Cho tứ diện $ABCD$ biết $AB=AD=BD=a$, $AC=2a$ và $\widehat{CAD}=120^{\circ}$. Tính $\vec{BC}\cdot \vec{AD}$.
	\haicot
	{\True $-\dfrac{3}{2}a^2$}
	{$\dfrac{3}{2} a^2$}
	{$\dfrac{1}{2} a^2$}
	{$-\dfrac{1}{2} a^2$}}{
	\begin{tikzpicture}[scale=0.65, font=\footnotesize,>=stealth]
	\path
	(0,0) coordinate (B)
	(5,0) coordinate (C)
	(1.5,-1.5) coordinate (D)
	(1,3) coordinate (A)
	;
	\draw (B)--(A)node[midway,sloped,scale=0.7]{$||$}--(D)node[midway,sloped,scale=0.7]{$||$}--(C)--(A) (B)--(D)node[midway,sloped,scale=0.7]{$||$};
	\draw[dashed](B)--(C);
	\foreach \x/\g in {B/180,A/90,C/0,D/-90}\draw[fill=black] (\x) circle (.05) +(\g:.5)node{\footnotesize$\x$};
	\draw pic["$120^\circ$",draw,angle eccentricity=1.6,angle radius=0.5cm]{angle=D--A--C};
	\end{tikzpicture}}
	\loigiai{
	Theo giả thiết tam giác $ABD$ là tam giác đều. Ta có
	\begin{eqnarray*}
	\vec{BC}\cdot\vec{AD}&=&\left(\vec{AC}-\vec{AB}\right)\cdot \vec{AD}\\&=&\vec{AC}\cdot\vec{AD}-\vec{AB}\cdot\vec{AD}\\
	&=&AC \cdot AD \cdot \cos 120^{\circ}-AB\cdot AD\cdot \cos 60^{\circ}\\
	&=&\dfrac{-3}{2}a^2.
	\end{eqnarray*}
	}
\end{ex} 
%%==========Câu 11
\begin{ex}
	\immini{ Cho hình chóp $S.A B C$ có $S A=S B=S C=A B=A C=a$ và $B C=a \sqrt{2}$. Tính góc giữa các vectơ $\vec{S C}$ và $\vec{A B}$.
	\haicot
	{$60^{\circ}$}
	{$90^{\circ}$}
	{\True $120^{\circ}$}
	{$150^{\circ}$}}{
	\begin{tikzpicture}[scale=0.6, font=\footnotesize,>=stealth]
	\path
	(0,0) coordinate (A)
	(5,0) coordinate (C)
	(1.5,-1.5) coordinate (B)
	(1,3) coordinate (S)
	;
	\draw (S)--(A)node[midway,sloped,scale=0.7]{$||$}--(B)node[midway,sloped,scale=0.7]{$||$}--(C)--(S)node[midway,sloped,scale=0.7]{$||$} (S)--(B)node[midway,sloped,scale=0.7]{$||$};
	\draw[dashed](A)--(C)node[midway,sloped,scale=0.7]{$||$};
	\foreach \x/\g in {A/180,S/90,C/0,B/-90}\draw[fill=black] (\x) circle (.05) +(\g:.5)node{\footnotesize$\x$};
	\end{tikzpicture}}
	\loigiai{
	\immini{Ta có
	\begin{align*}
	\cos \left(\vec{S C}, \vec{A B}\right) & \ =\dfrac{\vec{S C} \cdot \vec{A B}}{\left|\vec{S C}\right| \cdot\left|\vec{A B}\right|}=\dfrac{\left(\vec{S A}+\vec{A C}\right) \cdot \vec{A B}}{a^2} \\
	 & \ =\dfrac{\vec{S A} \cdot \vec{A B}+\vec{A C} \cdot \vec{A B}}{a^2}.
	\end{align*}
	Từ giả thiết suy ra $S A B$ là tam giác đều và $A B C$ là tam giác vuông cân tại $A$. Từ đó ta tính được
	$\vec{S A} \cdot \vec{A B}=a\cdot a \cdot \cos 120^{\circ}=-\dfrac{a^2}{2}$ và $\vec{A C} \cdot \vec{A B}=0$.\\
	Suy ra $\cos \left(\vec{S C}, \vec{A B}\right)=-\dfrac{1}{2}$.\\
	Vậy $\cos \left(\vec{S C}, \vec{A B}\right)=120^{\circ}$.
	}{
	\begin{tikzpicture}[scale=0.7, font=\footnotesize,>=stealth]
	\path
	(0,0) coordinate (A)
	(5,0) coordinate (C)
	(1.5,-1.5) coordinate (B)
	(1,3) coordinate (S)
	;
	\draw (S)--(A)node[midway,sloped,scale=0.7]{$||$}--(B)node[midway,sloped,scale=0.7]{$||$}--(C)--(S)node[midway,sloped,scale=0.7]{$||$} (S)--(B)node[midway,sloped,scale=0.7]{$||$};
	\draw[dashed](A)--(C)node[midway,sloped,scale=0.7]{$||$};
	\foreach \x/\g in {A/180,S/90,C/0,B/-90}\draw[fill=black] (\x) circle (.05) +(\g:.5)node{\footnotesize$\x$};
	\draw pic[draw,angle radius=0.3cm]{right angle=B--S--C};
	\end{tikzpicture}}
	}
\end{ex} 
%%==========Câu 12
\begin{ex}
	\immini{Cho tứ diện $OABC$ có các cạnh $OA$, $OB$, $OC$ đôi một vuông góc và $OA=OB=OC=1$. Gọi $M$ là trung điểm của cạnh $AB$. Tính góc giữa hai vectơ $\vec{OM}$ và $\vec{AC}$.
	\haicot
	{$90^\circ$}
	{\True $120^\circ$}
	{$60^\circ$}
	{$30^\circ$}}{
	\begin{tikzpicture}[scale=0.4, font=\footnotesize, line join=round, line cap=round]
	\foreach \x\y\t in {0/0/O,3/-3.6/A,7/2/B,0.3/5/C} \coordinate (\t) at (\x,\y);
	\draw (O)--(C)--(B)--(A)--(O);
	\coordinate (M) at ($(A)!1/2!(B)$);
	\draw[-stealth,thick] (A)--(C);
	\draw[-stealth,dashed,thick] (O)--(M);
	\draw[dashed] (O)--(B);
	\foreach \t/\g in {A/-90,B/0,C/90,O/180,M/0} \draw (\t) node[shift={(\g:10pt)}]{$\t$};
	\end{tikzpicture}}
	\loigiai{
	\immini{Đặt $\vec{OA}=\vec{a}$, $\vec{OB}=\vec{b}$, $\vec{OC}=\vec{c}$. \\
	Khi đó, $\left| \vec{a} \right| = \left| \vec{b} \right| = \left| \vec{c} \right| = 1$ và $\vec{a} \cdot \vec{b} = \vec{a} \cdot \vec{c} = \vec{b} \cdot \vec{c} = 0$.\\
	Ta có $\cos \left(\vec{OM},\vec{AC} \right) = \dfrac{\vec{OM} \cdot \vec{AC}}{\left| \vec{OM} \right| \cdot \left| \vec{AC} \right|}$.\\
	Mặt khác do $\vec{OM} = \dfrac{1}{2} \left(\vec{OA}+\vec{OB} \right) = \dfrac{1}{2} \left(\vec{a}+\vec{b} \right)$\\
	và $\vec{AC} = \vec{OC} - \vec{OA} = \vec{c} - \vec{a}$\\
	nên $\begin{aligned}[t]
	\vec{OM} \cdot \vec{AC} & =\dfrac{1}{2} \left(\vec{a}+\vec{b} \right) \cdot \left(\vec{c}-\vec{a} \right) \\
	 & =\dfrac{1}{2} \left(\vec{a} \cdot \vec{c}-\vec{a}^2+ \vec{b} \cdot \vec{c} - \vec{b} \cdot \vec{a} \right) = -\dfrac{1}{2}. \\
	\end{aligned}$
	}
	{\begin{tikzpicture}[scale=0.45, font=\footnotesize, line join=round, line cap=round]
	\foreach \x\y\t in {0/0/O,3/-3.6/A,7/2/B,0.3/5/C} \coordinate (\t) at (\x,\y);
	\draw (O)--(C)--(B)--(A)--(O);
	\coordinate (M) at ($(A)!1/2!(B)$);
	\draw[-stealth,red,very thick] (A)--(C);
	\draw[-stealth,dashed,red,very thick] (O)--(M);
	\draw[dashed] (O)--(B);
	\foreach \t/\g in {A/-90,B/0,C/90,O/180,M/0} \draw (\t) node[shift={(\g:10pt)}]{$\t$};
	\end{tikzpicture}}
	Ta lại có $\left| \vec{OM} \right| = OM =\dfrac{\sqrt{2}}{2}$, $\left| \vec{AC} \right| = AC = \sqrt{2}$. \\
	Do đó $\cos \left(\vec{OM},\vec{AC} \right) = \dfrac{\vec{OM} \cdot \vec{AC}}{\left| \vec{OM} \right| \cdot \left| \vec{AC} \right|} = \dfrac{\dfrac{-1}{2}}{\dfrac{\sqrt{2}}{2} \cdot \sqrt{2}} = \dfrac{-1}{2}$. \\
	Vậy $\left( \vec{OM}, \vec{AC}\right) = 120^\circ$.
	}
\end{ex} 
%%==========Câu 13
\begin{ex}
	Cho hình lập phương $ABCD.A'B'C'D'$ cạnh bằng $a$. Tích vô hướng của hai vectơ $\vec{DD'}$ và $\vec{A'C'}$ bằng
	\choice
	{$\sqrt{2}a^2$}
	{$a^2$}
	{$-\sqrt{2}a^2$}
	{\True $0$}
	\loigiai{
	Ta có: $\vec{A'C'}=\vec{A'D'}+\vec{D'C'}$, mà tứ giác $ADD'A'$ và $DCC'D'$ là hình vuông nên $\vec{DD'} \cdot \vec{A'D'}=\vec{DD'} \cdot \vec{D'C'}=0$. Do đó $\vec{DD'} \cdot \left(\vec{A'D'}+\vec{D'C'}\right)=0$.
	}
\end{ex}
\Closesolutionfile{ans}
\textbf{PHẦN II.} \textit{Câu trắc nghiệm đúng sai. Trong mỗi ý a), b), c), d) ở mỗi câu, học sinh chọn đúng hoặc sai.}\\
\Opensolutionfile{ans}[ans/2H2-B1-d2-2]
%%==========Câu 14
\begin{ex}%[2H2H1-3]
	Trong không gian, cho hai véc-tơ $\vec{a}$ và $\vec{b}$ cùng có độ dài bằng $1$. Biết rằng góc giữa hai véc-tơ đó là $45^{\circ}$.
	\choiceTF
	{\True $\vec{a}\cdot \vec{b}=\dfrac{\sqrt{2}}{2}$}
	{\True $\left( \vec{a}+3 \vec{b}\right) \cdot\left( \vec{a}-2 \vec{b}\right)=-5+\dfrac{\sqrt{2}}{2}$}
	{$\left| \vec{a}+ \vec{b}\right|=2+\sqrt{2} $}
	{$\left| \vec{a}-\sqrt{2}\vec{b}\right|=0$}
	\loigiai{
	\begin{enumerate}[a)]
	\item $\vec{a}\cdot \vec{b}=\left| \vec{a}\right|\cdot \left| \vec{b}\right|\cos \left(\vec{a},\vec{b} \right)=\dfrac{\sqrt{2}}{2}$.
	\item $\left( \vec{a}+3 \vec{b}\right) \cdot\left( \vec{a}-2 \vec{b}\right)=\left| \vec{a}\right|^2+\cdot\vec{a}\cdot \vec{b}-6\left| \vec{b}\right|^2 =1+\cdot\dfrac{\sqrt{2}}{2}-6=-5+\dfrac{\sqrt{2}}{2} $.
	\item $\left( \vec{a}+ \vec{b}\right)^2= \vec{a}^2+2\vec{a}\cdot \vec{b}+\vec{b}^2=1+2\cdot\dfrac{\sqrt{2}}{2}+1=2+\sqrt{2}$. Suy ra $\left| \vec{a}+ \vec{b}\right|=\sqrt{2+\sqrt{2}}$.
	\item $\left( \vec{a}-\sqrt{2} \vec{b}\right)^2= \vec{a}^2+2\sqrt{2}\vec{a}\cdot \vec{b}+2\vec{b}^2=1+2\sqrt{2}\cdot\dfrac{\sqrt{2}}{2}+2=2$. Suy ra $\left| \vec{a}- \sqrt{2}\vec{b}\right|=\sqrt{2}$.
	\end{enumerate}
	}
\end{ex} 
%%==========Câu 15
\begin{ex}%[2H2H1-3]
	\immini{Cho tứ diện đều $ABCD$ có cạnh bằng $a$ và $M$ là trung điểm của $CD$.
	\choiceTF
	{\True $\vec{AM} \cdot \vec{CD}=0$}
	{\True $\vec{AB} \cdot \vec{AC}=\dfrac{a^2}{2}$}
	{\True $\vec{AB}\cdot\vec{CD}=0$}
	{$\vec{AM}\cdot\vec{AB} =-\dfrac{a^2}{2}$}
	}{
	\begin{tikzpicture}[scale=1, font=\footnotesize, line join=round, line cap=round, >=Stealth]
	\def\a{3}
	\path
	(0:0) coordinate (B)
	(5:\a) coordinate (D)
	($(D)+(-135:\a/2)$) coordinate (C)
	($(B)+(65:\a)$) coordinate (A)
	($(C)!.5!(D)$) coordinate (M)
	;
	\draw[->] (A)--(D);
	\draw[->] (A)--(C);
	\draw[->] (A)--(B);
	\draw[->] (A)--(M);
	\draw[dashed] (B)--(D);
	\draw (B)--(C)--(D);
	\foreach \x/\g in {A/90,B/180,C/-90,D/0,M/0}
	\draw[fill=black] 	(\x) circle (.5pt)
	($(\g:.3)+(\x)$) node {$\x$};
	\end{tikzpicture}}
	\loigiai{
	\begin{enumerate}[a)]
	\item Tam giác $ACD$ đều, suy ra $AM$ vuông góc với $CD$ nên $\vec{AM}\cdot \vec{CD}=0$.
	\item Ta có $\begin{aligned}[t]
	 \vec{AB} \cdot \vec{AC} & = |\vec{AB}| \cdot |\vec{AC}| \cdot \cos (\vec{AB},\vec{AC}) \\
	 & = AB \cdot AC \cdot \cos \widehat{BAC} \\
	 & = a \cdot a \cdot \cos {60^0} \\
	 & = \dfrac{a^2}{2}.
	 \end{aligned}$\\
	\item Ta có $\vec{AB} \cdot \vec{CD}=(\vec{AM}+\vec{MB}) \cdot \vec{CD}=\vec{AM} \cdot \vec{CD} +\vec{MB} \cdot \vec{CD}$.\\
	 Mà $AM$, $BM$ là trung tuyến của các tam giác đều $ACD$, $BCD$ nên $\vec{AM} \perp \vec{CD}, \vec{MB} \perp \vec{CD}$.\\
	 Suy ra $\vec{AM} \cdot \vec{CD} = \vec{MB} \cdot \vec{CD} = 0$.\\
	 Từ các kết quả trên ta có $\vec{AM} \cdot \vec{CD}=0$.
	 Suy ra $(\vec{AB}, \vec{CD})=90^\circ$.
	\item Ta có $\vec{AM} = \dfrac{1}{2}(\vec{AC} + \vec{AD})$, suy ra
	 \[ \vec{AB} \cdot \vec{AM} = \vec{AB} \cdot \frac{1}{2}(\vec{AC} +\vec{AD}) = \frac{1}{2}(\vec{AB} \cdot \vec{AC} + \vec{AB} \cdot \vec{AD}) = \frac{1}{2}\left(\frac{a^2}{2} + \frac{a^2}{2}\right) = \dfrac{a^2}{2}.\]
	\end{enumerate}
	}
\end{ex} 
%%==========Câu 16
\begin{ex}
	\immini{
	Một chất điểm ở vị trí đỉnh $A$ của hình lập phương $ABCD.A'B'C'D'$. Chất điểm chịu tác động bởi ba lực $\vec{a}$, $\vec{b}$, $\vec{c}$ lần lượt cùng hướng với $\vec{AD}$, $\vec{AB}$ và $\vec{AC'}$ như hình vẽ. Độ lớn của các lực $\vec{a}$, $\vec{b}$ và $\vec{c}$ tương ứng là $10$ N, $10$ N và $20$ N.
	\choiceTF
	{$\vec{a}+\vec{b}=\vec{c}$}
	{$\big|\vec{a}+\vec{b}\big|=20$ (N)}
	{\True $\big|\vec{a}+\vec{c}\big|=\big|\vec{b}+\vec{c}\big|$}
	{\True $\big|\vec{a}+\vec{b}+\vec{c}\big|=32{,}59$ (N) (\textit{làm tròn kết quả đến hàng phần mười})}
	}{\hspace{0.5cm}
	\begin{tikzpicture}[line join=round, line cap = round, >=stealth, scale=0.65,font=\footnotesize]
	\path
	(0,0) coordinate (A')
	(-1.5,-1.5) coordinate (D')
	(2,-1.5) coordinate (C')
	(3.5,0) coordinate (B')
	(0,3.5) coordinate (A)
	($(A)+(B')-(A')$) coordinate (B)
	($(A)+(C')-(A')$) coordinate (C)
	($(A)+(D')-(A')$) coordinate (D)
	($(A)!1/2!(D)$) coordinate (M)
	($(A)!1/2!(B)$) coordinate (N)
	($(A)!1/2!(C')$) coordinate (P)
	;
	\draw[->,thick](A)--node [left]{$\vec{a}$}(M);
	\draw[->,thick](A)--node [above]{$\vec{b}$}(N);
	\draw[->,thick](A)--node[right]{$\vec{c}$}(P);
	\draw[dashed] (D')--(A')--(B') (A')--(A)--(C');
	\draw (D)--(C)--(B)--(A)--(D)--(D')--(C')--(C) (C')--(B')--(B);
	\draw pic[draw,angle eccentricity=1.2,angle radius=0.25cm]{right angle= D--A--B};
	\foreach \x/\g in {A/90,B/80,C/-40,D/110,A'/160,B'/-65,C'/-90,D'/-100} \fill[black](\x) circle (1pt) ($(\x)+(\g:3mm)$)node{$\x$};
	\end{tikzpicture}
	}
	\loigiai{
	Từ giả thiết, ta có $\vec{a} \perp \vec{b};\cos\left(\vec{a},\vec{c}\right)= \cos \widehat{DAC'}= \dfrac{1}{\sqrt3}; \cos \left(\vec{b},\vec{c}\right)= \cos \widehat{BAC'}= \dfrac{1}{\sqrt3}$.\\
	\begin{enumerate}[a)]
	\item Giả sử $\vec{a}+\vec{b}=\vec{d}$. Theo quy tắc hình bình hành thì $\vec{d}$ cùng hướng với $\vec{AC}$. Suy ra $\vec{a}+\vec{b}\ne \vec{c}$
	\item $\big|\vec{a}+\vec{b}\big|=10\sqrt{2}$ (đường chéo hình vuông cạnh bằng 10).
	\item Ta có
	 \begin{itemize}
	 \item [$\bullet$] $\big(\vec{a}+\vec{c}\big)^2=|\vec{a}|^2+2\vec{a} \cdot \vec{c}+|\vec{c}|^2=10^2+2.10.20.\dfrac{1}{\sqrt{3}}+20^2=500+\dfrac{400\sqrt{3}}{3}$.\\
	 Suy ra $\big|\vec{a}+\vec{c}\big|=\sqrt{500+\dfrac{400\sqrt{3}}{3}}$.
	 \item [$\bullet$] $\big(\vec{b}+\vec{c}\big)^2=|\vec{b}|^2+2\vec{b} \cdot \vec{c}+|\vec{c}|^2=10^2+2.10.20.\dfrac{1}{\sqrt{3}}+20^2=500+\dfrac{400\sqrt{3}}{3}$.\\
	 Suy ra $\big|\vec{b}+\vec{c}\big|=\sqrt{500+\dfrac{400\sqrt{3}}{3}}$.
	 \end{itemize}
	 Vậy $\big|\vec{a}+\vec{c}\big|=\big|\vec{b}+\vec{c}\big|$.
	\item Giả sử lực tổng hợp là $\vec{m}$, tức là $\vec{m}=\vec{a}+\vec{b}+\vec{c}$.\\
	 Do đó
	 \allowdisplaybreaks
	 \begin{eqnarray*}
	 \vec{m}=\vec{a}+\vec{b}+\vec{c} &\Leftrightarrow& \left|\vec{m}\right|^2= \left(\vec{a}+\vec{b}+\vec{c}\right)^2\\
	 &\Leftrightarrow& \left|\vec{m}\right|^2= \vec{a}^2+\vec{b}^2+\vec{c}^2+2\vec{a}\cdot\vec{b}+2\vec{b}\cdot\vec{c}+2\vec{c}\cdot\vec{a}\\
	 &\Leftrightarrow& \left|\vec{m}\right|^2= 10^2+10^2+20^2+0+2\cdot 10\cdot 20 \cdot \dfrac{1}{\sqrt3} +2\cdot 10\cdot 20 \cdot \dfrac{1}{\sqrt3}\\
	 &\Leftrightarrow& \left|\vec{m}\right|^2= 10^2+10^2+20^2+0+2\cdot 10\cdot 20 \cdot \dfrac{1}{\sqrt3} +2\cdot 10\cdot 20 \cdot \dfrac{1}{\sqrt3}\\
	 &\Leftrightarrow& \left|\vec{m}\right| \approx 32{,}59.
	 \end{eqnarray*}
	 Vậy cường độ hợp lực của $\vec{a}$, $\vec{b}$ và $\vec{c}$ là $\approx 32{,}59$ (N).
	\end{enumerate}
	}
\end{ex} 
%%==========Câu 17
\begin{ex}
	Cho hình chóp $S.ABCD$ có đáy $ABCD$ là hình chữ nhật. Biết rằng cạnh $AB=a$, $AD=2a$, cạnh bên $SA=2a$ và vuông góc với mặt đáy. Gọi $M$, $N$ lần lượt là trung điểm của các cạnh $SB$, $SD$. Các mệnh đề sau đúng hay sai ?
	\choiceTF
	{Hai vectơ $\vec{AB}$, $\vec{CD}$ là hai vectơ cùng phương, cùng hướng}
	{Góc giữa hai vectơ $\vec{SC}$ và $\vec{AC}$ bằng $60^\circ $}
	{\True Tích vô hướng $\vec{AM} \cdot \vec{AB}=\dfrac{a^2}{2}$}
	{Độ dài của vectơ $\vec{AM}-\vec{AN}$ là $\dfrac{a\sqrt{3}}{2}$}
	\loigiai{
	\begin{center}
	\begin{tikzpicture}[line join = round, line cap = round, thick, font = \small, scale = .7]
	\path
	(0:0) coordinate (A)
	+(0:5) coordinate (B)
	+(-150:2.5) coordinate (D)
	+(90:5) coordinate (S)
	($(B)+(D)-(A)$) coordinate (C)
	($(S)!.5!(B)$) coordinate (M)
	($(S)!.5!(D)$) coordinate (N)
	;
	\draw[dashed]
	(D)--(A)--(B) (C)--(A)--(S) (M)--(A)--(N)
	;
	\draw
	(D)--(C)--(B)
	(S)--(B) (S)--(C) (S)--(D)
	\foreach \x/\y/\z in {S/A/B,S/A/D}{
	pic[draw, angle radius = 6pt]{right angle = \x--\y--\z}
	}
	;
	\foreach \x/\g in {A/135,B/0,C/-45,D/-135,S/90,M/45,N/135}
	\fill (\x) circle (1.5pt)
	+(\g:3.5mm) node {$\x$};
	\end{tikzpicture}
	\end{center}
	\begin{enumerate}[a)]
	\item $\vec{AB}=-\vec{CD}$. Suy ra hai vectơ $\vec{AB}$, $\vec{CD}$ là hai vectơ ngược hướng.
	\item Ta có: $ABCD$ là hình chữ nhật nên: $AC=\sqrt{AB^2+AD^2}=a\sqrt{5}$.\\
	 Hình chóp $S.ABCD$ có $SA$ vuông góc với mặt đáy nên tam giác $SAC$ là tam giác vuông tại $A$. Suy ra: $\tan \widehat{SCA}=\dfrac{SA}{AC}=\dfrac{2a}{a\sqrt{5}}\Rightarrow \widehat{SCA}\approx 41^\circ 48'$.\\
	 Ta có: $\left(\vec{SC},\vec{AC}\right)=\left(\vec{CS},\vec{CA}\right)=\widehat{SCA}\approx 41^\circ 48'$.
	\item Hình chóp $S.ABCD$ có $SA$ vuông góc với mặt đáy nên tam giác $SAB$ là tam giác vuông tại $A$.\\
	 Suy ra: $SB=\sqrt{SA^2+AB^2}=a\sqrt{5}$.\\
	 Trong tam giác $SAB$ vuông tại $A$ có $AM$ là đường trung tuyến nên: \\
	 $AM=\dfrac{1}{2}SB=\dfrac{a\sqrt{5}}{2}$.\\
	 Lại có: $M$ là trung điểm của $SB$ nên $MB=\dfrac{1}{2}SB=\dfrac{a\sqrt{5}}{2}$. \\
	 Ta tính được: $\cos \widehat{MAB}=\dfrac{MA^2+AB^2-MB^2}{2MA \cdot AB}=\dfrac{\sqrt{5}}{5}$.\\
	 Mà: $\left(\vec{AM},\vec{AB}\right)=\widehat{MAB}$, suy ra: \\
	 $\vec{AM}\cdot \vec{AB}=\left| \vec{AM} \right| \cdot \left| \vec{AB} \right| \cdot \cos \left(\vec{AM},\vec{AB}\right)=\dfrac{a\sqrt{5}}{2} \cdot a \cdot \dfrac{\sqrt{5}}{5}=\dfrac{a^2}{2}$.
	\item Ta có: $M$, $N$ lần lượt là trung điểm của các cạnh $SB$, $SD$ nên $MN$ là đường trung bình của tam giác $SBD$.
	 Do đó: $MN=\dfrac{1}{2}BD=\sqrt{AB^2+AD^2}=\dfrac{a\sqrt{5}}{2}$.\\
	 Suy ra: $\left| \vec{AM}-\vec{AN} \right| =\left| \vec{MN} \right| =\dfrac{a\sqrt{5}}{2}$.
	\end{enumerate}
	}
\end{ex}
%%==========Câu 18
\begin{ex}
	Cho hình lập phương $ABCD.A'B'C'D'$ có cạnh bằng $a$. Trên các cạnh $AA'$, $CC'$ lần lượt lấy các điểm $M$, $N$ sao cho $AM=\dfrac{2}{3}AA'$, $CN=NC'$. Các mệnh đề sau đúng hay sai?
	\choiceTF
	{Góc giữa hai vectơ $\vec{AN}$ và $\vec{AC}$ bằng $60^\circ $}
	{\True Độ dài của vectơ $\vec{MN}+\vec{AM}$ là $\dfrac{3a}{2}$}
	{Tích vô hướng $\vec{AN}\cdot \vec{AC}=a^2$}
	{\True Tích vô hướng $\vec{MN}\cdot \vec{A'C'}=2a^2$}
	\loigiai{
	\begin{center}
	\begin{tikzpicture}[line join = round, line cap = round, thick, font = \small, scale = .7]
	\def \canh{4}
	\path
	(0:0) coordinate (D')
	+(90:\canh) coordinate (D)
	+(0:\canh) coordinate (C')
	+(40:.4*\canh) coordinate (A')
	($(C')+(D)-(D')$) coordinate (C)
	($(D)+(A')-(D')$) coordinate (A)
	($(C')+(A')-(D')$) coordinate (B')
	($(C)+(A)-(D)$) coordinate (B)
	($(A)!2/3!(A')$) coordinate (M)
	($(C)!.5!(C')$) coordinate (N)
	($(C)!2/3!(C')$) coordinate (M')
	;
	\draw[dashed]
	(A')--(A) (A')--(B') (A')--(D') (M')--(M)--(N)
	;
	\draw
	(A)--(B)--(B')--(C')--(D')--(D)--cycle
	(C)--(B) (C)--(D) (C)--(C')
	;
	\foreach \x/\g in {D'/-90,C'/-90,D/180,A'/135,C/-45,A/90,B'/0,B/90,M/180,N/0,M'/0}
	\fill (\x) circle (1.5pt)
	+(\g:3mm) node {$\x$};
	\end{tikzpicture}
	\end{center}
	\begin{enumerate}[a)]
	\item Ta có: $AC=\sqrt{AB^2+AC^2}=a\sqrt{2}$.\\
	 Lại có: $CN=NC'$ nên $CN=NC'=\dfrac{a}{2}$.\\
	 $ABCD.A'B'C'D'$ là hình lập phương nên tam giác $NAC$ là tam giác vuông tại $C$.\\
	 Suy ra: $\tan NAC=\dfrac{CN}{AC}=\dfrac{\sqrt{2}}{4}\Rightarrow \widehat{NAC}\approx 19^\circ 28'$\\
	 Ta có: $\left(\vec{AN},\vec{AC}\right)=\widehat{NAC}\approx 19^\circ 28'$.
	\item Trong tam giác $NAC$ vuông tại $C$ có: $AN=\sqrt{AC^2+CN^2}=\dfrac{3a}{2}$.\\
	 Ta có: $\left| \vec{MN}+\vec{AM} \right| =\left| \vec{AN} \right| =\dfrac{3a}{2}$.
	\item Ta có: $\tan \widehat{NAC}=\dfrac{\sqrt{2}}{4}\Rightarrow \cos \widehat{NAC}=\dfrac{2\sqrt{2}}{3}$ (Do $\widehat{NAC}<90^\circ $).\\
	 Do đó: $\vec{AN}\cdot \vec{AC}=\left| \vec{AN} \right| \cdot \left| \vec{AC} \right| \cdot \cos \left(\vec{AN},\vec{AC}\right)=\dfrac{3a}{2} \cdot a\sqrt{2} \cdot \dfrac{2\sqrt{2}}{3}=2a^2$.
	\item Trên cạnh $CC'$ lấy điểm $M'$ sao cho: $\dfrac{CM'}{CC'}=\dfrac{2}{3}$.\\
	 Suy ra: $\heva{& NM'=NC'-M'C'=\dfrac{a}{6} \\ & MM'\parallel AC \\ & MM'=AC=a\sqrt{2}} $.\\
	 Ta có: $\cos \widehat{NMM'}=\dfrac{NM^2+M'M^2-M'N^2}{2 \cdot NM \cdot M'M}=\dfrac{6\sqrt{146}}{73}$.\\
	 Mặt khác: $\left(\vec{MN},\vec{A'C'}\right)=\left(\vec{MN},\vec{MM'}\right)=\widehat{NMM'}$.\\
	 Tam giác $MNM'$ vuông tại $M'$ có: $MN=\sqrt{M'N^2+M'M^2}=\dfrac{a\sqrt{73}}{6}$.\\
	 Do đó: $\vec{MN}\cdot \vec{A'C'}=\left| \vec{MN} \right| \cdot \left| \vec{A'C'} \right| \cdot \cos \left(\vec{MN},\vec{A'C'}\right)=2a^2$.
	\end{enumerate}
	}
\end{ex}
%%==========Câu 19
\begin{ex}
	Cho hình lăng trụ đứng $ABC.A'B'C'$ đáy là tam giác đều cạnh $2a,AA'=a\sqrt{3}$. $H$, $K$ lần lượt là trung điểm $BC$, $B'C'$. Các mệnh đề sau đúng hay sai?
	\choiceTF
	{Hai vectơ $\vec{AH}$, $\vec{KA'}$ là hai vectơ cùng phương, cùng hướng}
	{Góc giữa hai vectơ $\vec{A'H}$ và $\vec{AH}$ bằng $60^\circ $}
	{Tích vô hướng $\vec{AK}\cdot \vec{AB'}=\dfrac{5a^2}{2}$}
	{\True Độ dài của vectơ $\vec{AK}+\vec{AH}$ là $\dfrac{a\sqrt{3}}{2}$}
	\loigiai{
	\begin{center}
	\begin{tikzpicture}[line join = round, line cap = round, thick, font = \small, scale = .7]
	\path
	(0:0) coordinate (A)
	+(0:4) coordinate (C)
	+(-50:2) coordinate (B)
	+(90:4) coordinate (A')
	($(A')+(B)-(A)$) coordinate (B')
	($(A')+(C)-(A)$) coordinate (C')
	($(B)!.5!(C)$) coordinate (H)
	($(B')!.5!(C')$) coordinate (K)
	($(H)!.5!(K)$) coordinate (I)
	;
	\draw[dashed]
	(A)--(C) (H)--(A)--(K) (A)--(I)
	;
	\draw
	(A)--(B)--(C)--(C')--(A')--cycle
	(B')--(A') (B')--(B) (B')--(C') (A')--(K)--(H)
	;
	\foreach \x/\g in {A/180,B/-90,C/0,A'/-180,B'/-150,C'/0,H/-45,K/-30,I/0}
	\fill (\x) circle (1.5pt)
	+(\g:3mm) node {$\x$};
	\end{tikzpicture}
	\end{center}
	\begin{enumerate}[a)]
	\item Ta có tam giác $\triangle ABC,\triangle A'B'C'$ đều cạnh $2a$ suy ra $A'K=AH=a\sqrt{3}$\\
	 Xét tứ giác $AA'KH$ có $AA'=KH=AH=A'K=a\sqrt{3}$, $AA'\perp AH$ suy ra tứ giác $AA'KH$ là hình vuông , từ đó dễ thấy hai vectơ $\vec{AH}$, $\vec{KA'}$ là hai vecto cùng phương ngược hướng.
	\item Ta có: $AA'KH$ là hình vuông suy ra $\widehat{A'HA}=45^\circ $\\
	 Có $A'A\perp AH\Rightarrow \triangle A'AH$ vuông tại $A\Rightarrow \left(\vec{A'H},\vec{AH}\right)=\widehat{A'HA}=45^\circ $.
	\item Ta có $\triangle AB'C'$ cân tại $A$, suy ra $AK\perp B'C'$, $AK=a\sqrt{6},B'K=a$\\
	 $AB'=\sqrt{AB^2+BB'^2}=\sqrt{4a^2+3a^2}=a\sqrt{7}$\\
	 Xét $\triangle AKB'$ có $\cos \widehat{KAB'}=\dfrac{AK}{AB'}=\dfrac{a\sqrt{6}}{a\sqrt{7}}=\sqrt{\dfrac{6}{7}}$.\\
	 $\vec{AK} \cdot \vec{AB'}=AK \cdot AB'\cdot \cos \widehat{KAB'}=a\sqrt{6} \cdot a\sqrt{7} \cdot \sqrt{\dfrac{6}{7}}=6a^2$.
	\item Gọi $I$ là trung điểm $HK\Rightarrow IH=\dfrac{a\sqrt{3}}{2}$, $AI=\sqrt{IH^2+AH^2}=\sqrt{\dfrac{3a^2}{4}+3a^2}=\dfrac{a\sqrt{15}}{2}$.\\
	 Ta có $\left| \vec{AK}+\vec{AH} \right| =\left| 2 \cdot \vec{AI} \right| =2AI=a\sqrt{15}$.
	\end{enumerate}
	}
\end{ex}
%%==========Câu 20
\begin{ex}
	Cho tứ diện đều $ABCD$ cạnh $a$. $E$ là điểm trên đoạn $CD$ sao cho $ED=2CE$. Các mệnh đề sau đúng hay sai?
	\choiceTF
	{Có $6$ vectơ (khác vectơ $\vec{0}$) có điểm đầu và điểm cuối được tạo thành từ các đỉnh của tứ diện}
	{Góc giữa hai vectơ $\vec{AB}$ và $\vec{BC}$ bằng $60^\circ $}
	{Nếu $\vec{BE}=m\vec{BA}+n\vec{BC}+p\vec{BD}$ thì $m+n+p=\dfrac{2}{3}$}
	{\True Tích vô hướng $\vec{AD} \cdot \vec{BE}=\dfrac{a^2}{6}$}
	\loigiai{
	\begin{center}
	\begin{tikzpicture}[line join = round, line cap = round, thick, font = \small, scale = .7]
	\path
	(0:0) coordinate (B)
	+(0:5) coordinate (C)
	+(-70:2) coordinate (D)
	+(70:4) coordinate (A)
	($(C)!1/3!(D)$) coordinate (E)
	;
	\draw[dashed]
	(B)--(C) (B)--(E)
	;
	\draw
	(A)--(B)--(D)--(C)--cycle
	(E)--(A)--(D)
	;
	\foreach \x/\g in {B/180,C/0,D/-90,A/90,E/-45}
	\fill (\x) circle (1.5pt)
	+(\g:3mm) node {$\x$};
	\end{tikzpicture}
	\end{center}
	\begin{enumerate}[a)]
	\item Số vectơ (khác $\vec{0}$) có điểm đầu và điểm cuối được tạo thành từ các đỉnh của tứ diện là $A_4^2=12$.
	\item $(\vec{AB},\vec{BC})=180^\circ-(\vec{BA},\vec{BC})={{180}^\circ}-\widehat{ABC}=120^\circ$.
	\item $\vec{BE}=\vec{BC}+\vec{CE}=\vec{BC}+\dfrac{1}{3}\vec{CD}=\vec{BC}+\dfrac{1}{3}\left(\vec{BD}-\vec{BC}\right)=\dfrac{2}{3}\vec{BC}+\dfrac{1}{3}\vec{BD}$.\\
	 Do đó $m=0$,$n=\dfrac{2}{3}$,$p=\dfrac{1}{3}$. Suy ra $m+n+p=1$.
	\item Ta có: $\vec{BE}=\vec{AE}-\vec{AB}=\left(\vec{AC}+\vec{CE}\right)-\vec{AB}=\vec{AC}+\dfrac{1}{3}\vec{CD}-\vec{AB}$\\
	 $=\vec{AC}+\dfrac{1}{3}\left(\vec{AD}-\vec{AC}\right)-\vec{AB}=\dfrac{2}{3}\vec{AC}+\dfrac{1}{3}\vec{AD}-\vec{AB}$\\
	 Suy ra: $\vec{AD} \cdot \vec{BE}=\vec{AD} \cdot \left(\dfrac{2}{3}\vec{AC}+\dfrac{1}{3}\vec{AD}-\vec{AB}\right)=\dfrac{2}{3} \cdot \vec{AD} \cdot \vec{AC}+\dfrac{1}{3} \cdot {{\vec{AD}}^2}-\vec{AD} \cdot \vec{AB}$\\
	 $=\dfrac{2}{3} \cdot a \cdot a \cdot \cos 60^\circ +\dfrac{1}{3}a^2-a \cdot a \cdot \cos 60^\circ =\dfrac{a^2}{6}$.
	\end{enumerate}
	}
\end{ex}
%%==========Câu 21
\begin{ex}
	Cho tứ diện $ABCD$ có cạnh $a$. Gọi $M$, $N$ lần lượt là trung điểm của $AB$, $CD$. Các mệnh đề sau đúng hay sai?
	\choiceTF
	{$\vec{AB}$ và $\vec{CD}$ cùng hướng}
	{\True $\vec{EA}+\vec{EB}+\vec{EC}+\vec{ED}=\vec{0}$ với $E$ là trung điểm $MN$}
	{\True $\vec{AB} \cdot \vec{CD}+\vec{AC} \cdot \vec{DB}+\vec{AD} \cdot \vec{BC}=\vec{0}$}
	{\True Điểm $I$ xác định bởi $P=3\vec{IA}^2+\vec{IB}^2+\vec{IC}^2+\vec{ID}^2$ có giá trị nhỏ nhất. Khi đó giá trị nhỏ nhất của $P$ là $2a^2$}
	\loigiai{
	\begin{center}
	\begin{tikzpicture}[line join = round, line cap = round, thick, font = \small, scale = 1]
	\path 
	(0:0) coordinate (B)
	+(0:5) coordinate (C)
	+(-45:2) coordinate (D)
	+(70:4) coordinate (A)
	($(A)!.5!(B)$) coordinate (M)
	($(C)!.5!(D)$) coordinate (N)
	($(M)!.5!(N)$) coordinate (E)
	($(B)!2/3!(N)$) coordinate (G)
	($(A)!.5!(G)$) coordinate (O)
	;
	\draw[dashed] 
	(B)--(C) (A)--(G) (M)--(N)
	;
	\draw 
	(A)--(B)--(D)--(C)--cycle
	(A)--(D)
	;
	\foreach \x/\g in {B/180,C/0,D/-90,G/0,A/90,M/135,N/-45,E/0,O/0}
	\fill (\x) circle (1.5pt)
	+(\g:3mm) node {$\x$};
	\end{tikzpicture}
	\end{center}
	\begin{enumerate}[a)]
	\item $\vec{AB}$ và $\vec{CD}$ ngược hướng.
	\item Vì $M$ là trung điểm $AB$ nên $\vec{EA}+\vec{EB}=2\vec{EM}$, $N$ là trung điểm $CD$ nên $\vec{EC}+\vec{ED}=2\vec{EN}$.\\
	Ta có $\vec{EA}+\vec{EB}+\vec{EC}+\vec{ED}=2\left(\vec{EM}+\vec{EN}\right)=\vec{0}$.
	\item $\begin{aligned}[t]
	&\vec{AB} \cdot \vec{CD}+\vec{AC} \cdot \vec{DB}+\vec{AD} \cdot \vec{BC}=\left(\vec{AC}+\vec{CB}\right) \cdot \vec{CD}+\vec{AC} \cdot \vec{DB}+\vec{AD} \cdot \vec{BC}\\
	 = & \vec{AC} \cdot \left(\vec{CD}+\vec{DB}\right)+\vec{AD} \cdot \vec{BC}+\vec{CB \cdot }\vec{CD}=\vec{AC} \cdot \vec{CB}+\vec{AD} \cdot \vec{BC}+\vec{CB \cdot }\vec{CD} \\
	 = &\vec{CB}\left(\vec{AC}-\vec{AD}\right)+\vec{CB \cdot }\vec{CD}=\vec{0} 
	\end{aligned}$
	\item Gọi $O$ là điểm thoả mãn hệ thức $3\vec{OA}+\vec{OB}+\vec{OC}+\vec{OD}=\vec{0}$ suy ra $O$ cố định vì $A$, $B,C$, $D$ cố định. Ta có
	\begin{align*}
	P& =3\vec{IA}^2+\vec{IB}^2+\vec{IC}^2+\vec{ID}^2 \\
	& =3\left(\vec{IO}+\vec{OA}\right)^2+\left(\vec{IO}+\vec{OB}\right)^2+\left(\vec{IO}+\vec{OC}\right)^2+\left(\vec{IO}+\vec{OD}\right)^2 \\
	& =6IO^2+3OA^2+OB^2+OC^2+OD^2+2\vec{IO}\left(3\vec{OA}+\vec{OB}+\vec{OC}+\vec{OD}\right) \\
	& =6IO^2+3OA^2+OB^2+OC^2+OD^2.
	\end{align*}
	Do đó để $P$ nhỏ nhất thì $I$ trùng với $O$. Gọi $G$ là trọng tâm tam giác $BCD$.\\
	Vì $3\vec{OA}+\vec{OB}+\vec{OC}+\vec{OD}=3\vec{OA}+\left(\vec{OB}+\vec{OC}+\vec{OD}\right) =3\vec{OA}+3\vec{OG}$ nên $\vec{OA}+\vec{OG}=\vec{0}$.\\	
	Suy ra $O$ là trung điểm của $AG$.\\
	Ta có $BG=\dfrac{2}{3} \cdot \dfrac{a\sqrt{3}}{2}=\dfrac{a}{\sqrt{3}}\Rightarrow AG=\sqrt{AB^2-BG^2}=\sqrt{a^2-{{\left(\dfrac{a}{\sqrt{3}}\right)}^2}}=\dfrac{a\sqrt{2}}{\sqrt{3}}$\\
	$\Rightarrow OA=\dfrac{1}{2}AG=\dfrac{a}{\sqrt{6}}\Rightarrow OA^2=\dfrac{a^2}{6}$.\\
	Lại có $OD^2=OC^2=OB^2=OG^2+BG^2=\dfrac{a^2}{6}+\dfrac{a^2}{3}=\dfrac{a^2}{2}$.\\
	Vậy giá trị nhỏ nhất là $P=3 \cdot \dfrac{a^2}{6}+3 \cdot \dfrac{a^2}{2}=2a^2$ khi $I$ trùng với $O$.
	\end{enumerate}
	}
\end{ex}
\textbf{PHẦN III.} \textit{Câu trắc nghiệm trả lời ngắn.}\\
%%==========Câu 22
\begin{ex}
	Cho tứ diện đều $ABCD$ có cạnh bằng $4$. Giá trị tích vô hướng $\vec{AB}\left(\vec{AB}-\vec{CA}\right)$ bằng
	\loigiai{
	\SA{24}
	$\begin{aligned}[t]
	\vec{AB}\left(\vec{AB}-\vec{CA}\right)
	 & =\vec{AB} \cdot \vec{AB}+\vec{AB} \cdot \vec{AC}={{\vec{AB}}^2}+| \vec{AB}| \cdot | \vec{AC}| \cdot \cos \left(\vec{AB},\vec{AC}\right) \\
	 & =AB^2+AB \cdot AC \cdot \cos \left(\widehat{BAC}\right)=4^2+4 \cdot 4 \cdot \cos 60^\circ=4^2+\dfrac{4^2}{2}=\dfrac{{{3 \cdot 4}^2}}{2}=24.
	\end{aligned}$
	}
\end{ex}
%%==========Câu 23
\begin{ex}
	Trong không gian, cho hai vectơ $\vec{a}$ và $\vec{b}$ có cùng độ dài bằng $6$. Biết độ dài của vectơ $\vec{a}+2\vec{b}$ bằng $6\sqrt{3}$. Biết số đo góc giữa hai vectơ $\vec{a}$ và $\vec{b}$ là $x$ độ. Giá trị của $x$ là bao nhiêu?
	\loigiai{
	\SA{120}
	$\vec{a} \cdot \vec{b} = \dfrac{1}{4} \left[\left(\vec{a}+2 \vec{b}\right)^2 - \vec{a}^2 - 4\vec{b}^2\right]
	= \dfrac{1}{4} \left[\left|\vec{a}+2 \vec{b}\right|^2 - |\vec{a}|^2 - 4|\vec{b}|^2\right]
	= \dfrac{1}{4} \left[\left(6\sqrt{3}\right)^2 - 6^2 - 4\cdot 6^2\right] = -18$.\\
	Lại có $\vec{a} \cdot \vec{b}=| \vec{a}| \cdot | \vec{b}| \cdot \cos \left(\vec{a}\,,\vec{b}\right)\Leftrightarrow \cos \left(\vec{a}\,,\vec{b}\right)=\dfrac{\vec{a} \cdot \vec{b}}{| \vec{a}| \cdot | \vec{b}|}=\dfrac{-18}{6 \cdot 6}=\dfrac{-1}{2}\Leftrightarrow \left(\vec{a}\,,\vec{b}\right)=120^\circ $.\\
	Khi đó góc giữa hai vectơ $\vec{a}$ và $\vec{b}$ là $120^\circ $.
	}
\end{ex}
%%==========Câu 24
\begin{ex}
	Cho hình lập phương $ABCD.A'B'C'D'$ có cạnh bằng $2$. Tính $\vec{AB} \cdot \vec{A'C'}$.
	\loigiai{
	\SA{4}
	Ta có: $\left(\vec{AB},\vec{A'C'}\right)=\left(\vec{AB},\vec{AC}\right)=45^\circ $.\\
	Khi đó: $\vec{AB} \cdot \vec{A'C'}=AB \cdot A'C'\cdot \cos \left(\vec{AB},\vec{A'C'}\right)=2 \cdot 2\sqrt{2} \cdot \cos 45^\circ =4$.
	}
\end{ex}
%%==========Câu 25
\begin{ex}
	Cho tứ diện $ABCD$, gọi $M$, $N$ lần lượt là trung điểm của $BC$ và $AD$, biết $AB=a$, $CD=a$, $MN=\dfrac{a\sqrt{3}}{2}$. Tìm số đo (đơn vị độ) góc giữa hai đường thẳng $AB$ và $CD$.
	\loigiai{
	\SA{60}
	\begin{center}
	\begin{tikzpicture}[line join = round, line cap = round, thick, font = \small, scale = .7]
	\path
	(0:0) coordinate (B)
	+(0:5) coordinate (D)
	+(-30:4) coordinate (C)
	+(70:4) coordinate (A)
	($(B)!1/2!(C)$) coordinate (M)
	($(A)!1/2!(D)$) coordinate (N)
	($(A)!1/2!(C)$) coordinate (I)
	;
	\draw[dashed]
	(B)--(D) (M)--(N)
	;
	\draw
	(A)--(B)--(C)--(D)--cycle
	(A)--(C) (M)--(I)--(N)
	;
	\foreach \x/\g in {B/180,D/0,C/-90,A/90,M/-135,I/180,N/45}
	\fill (\x) circle (1.5pt)
	+(\g:3mm) node {$\x$};
	\end{tikzpicture}
	\end{center}
	Gọi $I$ là trung điểm của $AC$.\\
	Ta có $\heva{& IM \parallel AB \\& IN \parallel CD}\Rightarrow \widehat{\left(AB,CD\right)}=\widehat{\left(IM,IN\right)}$.\\
	Đặt $\widehat{MIN}=\alpha $. Xét tam giác $IMN$, có: $IM=\dfrac{AB}{2}=\dfrac{a}{2}$, $IN=\dfrac{CD}{2}=\dfrac{a}{2}$, $MN=\dfrac{a\sqrt{3}}{2}$.\\
	Theo định lý cosin, có $\cos \alpha =\dfrac{IM^2+IN^2-MN^2}{2 \cdot IM \cdot IN}=-\dfrac{1}{2}<0$.\\
	$\Rightarrow \widehat{MIN}=120^\circ \Rightarrow \widehat{\left(AB,CD\right)}=60^\circ $.
	}
\end{ex}
%%==========Câu 26
\begin{ex}
	Cho hình lập phương $ABCD.A'B'C'D'$. Góc giữa hai vectơ $\vec{A'B}$ và $\vec{AC'}$ bằng
	\loigiai{
	\SA{90}
	$\vec{A'B}=\vec{A'A}+\vec{AB}=\vec{AB}-\vec{AA'}$.\\
	$\vec{AC'}=\vec{AB}+\vec{AD}+\vec{AA'}$.\\
	$\Rightarrow \vec{A'B} \cdot \vec{AC'} = \left(\vec{AB}-\vec{AA'}\right) \cdot \left(\vec{AB}+\vec{AD}+\vec{AA'}\right)={{\vec{AB}}^2}-{{\vec{AA'}}^2}=0$.\\
	$\Rightarrow$ Góc giữa hai vectơ $\vec{A'B}$ và $\vec{AC'}$ bằng $90^\circ$.
	}
\end{ex}
%%==========Câu 27
\begin{ex}
	Cho hình chóp $S.ABC$ có $SA$, $SB$, $SC$ đôi một vuông góc nhau và $SA=SB=SC=a$. Gọi $M$ là trung điểm của $AB$. Góc giữa hai vectơ $\vec{SM}$ và $\vec{BC}$ bằng
	\loigiai{
	\shortans{120}	
	Ta có $\cos \left(\vec{SM},\vec{BC}\right)=\dfrac{\vec{SM} \cdot \vec{BC}}{|\vec{SM}| \cdot |\vec{BC}|}=\dfrac{\vec{SM} \cdot \vec{BC}}{SM \cdot BC}$.\\
	\begin{align*}
	\vec{SM} \cdot \vec{BC} & =\dfrac{1}{2}\left(\vec{SA}+\vec{SB}\right) \cdot \left(\vec{SC}-\vec{SB}\right)\\
	& =\dfrac{1}{2}\left(\vec{SA} \cdot \vec{SC}-\vec{SA} \cdot \vec{SB}+\vec{SB} \cdot \vec{SC}-\vec{SB} \cdot \vec{SB}\right) \\
	& =-\dfrac{1}{2}\vec{SB} \cdot \vec{SB}=-\dfrac{1}{2}SB^2=-\dfrac{a^2}{2}.
	\end{align*}
	Tam giác $SAB$ và $SBC$ vuông cân tại $S$ nên $AB=BC=a\sqrt{2}$.\\
	$\Rightarrow SM=\dfrac{AB}{2}=\dfrac{a\sqrt{2}}{2}$.\\
	Do đó $\cos \left(\vec{SM},\vec{BC}\right)=\dfrac{-\dfrac{a^2}{2}}{\dfrac{a\sqrt{2}}{2} \cdot a\sqrt{2}}=-\dfrac{1}{2}$. Suy ra $\left(\vec{SM},\vec{BC}\right)={120}^\circ$.
	}
\end{ex}
\Closesolutionfile{ans}
%%Bài 2. Tọa độ vector trong không gian
% \setcounter{section}{1}
\section{TỌA ĐỘ CỦA VÉC TƠ TRONG KHÔNG GIAN}
\subsection{LÝ THUYẾT CẦN NHỚ}
\subsubsection{Hệ tọa độ trong không gian}
Trong không gian, ba trục $O x$, $O y$, $O z$ đôi một vuông góc với nhau tại gốc $O$ của mỗi trục. Gọi $\vec{i}$, $\vec{j}$, $\vec{k}$ lần lượt là các véc-tơ đơn vị trên các trục $O x$, $O y$, $O z$.
\immini{
	\begin{itemize}
		\item  Hệ ba trục như vậy được gọi là hệ trục toạ độ Descartes vuông góc $Oxyz$, hay đơn giản là hệ toạ độ $Oxyz$. Điểm $O$ được gọi là gốc toạ độ.
		\item  Các mặt phẳng $(O x y)$, $(O y z)$, $(O z x)$ đôi một vuông góc với nhau được gọi là các mặt phẳng toạ độ.
		\item  ${\vec{i}^2} = {\vec{j}^2} = {\vec{k}^2} = 1$ \\
		và $\vec{i} \cdot \vec{j} = \vec{j} \cdot \vec{k} = \vec{k} \cdot \vec{i}  = 0$
	\end{itemize}
}{\hspace{1cm}
	\begin{tikzpicture}[>=stealth,line join=round,line cap=round,scale=1]
		\def\a{3.0}
		\path
		(0,0) coordinate (A1)
		(\a,0) coordinate (A2)
		(\a,\a) coordinate (A3)
		(0,\a) coordinate (A4);
		\foreach \i in {1,...,4}
		\path (A\i)+(45:.75) coordinate (B\i);
		\draw (B1)--(A1) (B1)--(B2) (B1)--(B4);
		%	\draw(A4)--(B4)--(B3)--(B2)--(A2) (A3)--(B3)
		%	(A1)--(A2)--(A3)--(A4)--cycle;
		\draw[-stealth] (B1)--(B2)node[right]{$y$};
		\draw[-stealth] (B1)--(B4)node[above]{$z$};
		\draw[dashed](B1)--+(45:0.85)[dashed](B1)--+(180:0.85)(B1)--+(270:0.85);
		\draw[-stealth] (A1)--+(-135:.95)node[below]{$x$};
		\draw[-stealth,blue] (B1)--+(0:.95)node[above]{$\vec{j}$};
		\draw[-stealth,blue] (B1)--+(90:.85)node[above left]{$\vec{k}$};
		\draw[-stealth,blue] (B1)--+(-135:0.95)node[right]{$\vec{i}$};
		\fill(B1)circle(1pt) node[below right]{$O$};
	\end{tikzpicture}}
Không gian với hệ toạ độ $Oxyz$ còn được gọi là không gian $Oxyz$.
\subsubsection{Tọa độ của điểm}
Trong KG $Oxyz$, cho điểm $M$. Tọa độ điểm $M$ được xác định như sau:
\immini{
	\begin{itemize}
		\item Xác định hình chiếu $M_1$ của điểm $M$ trên mặt phẳng $Oxy$. Trong mặt phẳng tọa độ $Oxy$, tìm hoành độ $a$, tung độ $b$ của điểm $M_1$.
		\item Xác định hình chiếu $P$ của điểm $M$ trên trục cao $Oz$, điểm $P$ ứng với số $c$ trên trục $Oz$. Số $c$ là cao độ của điểm $M$.
	\end{itemize}
	Bộ số $(a;b;c)$ là toạ độ của điểm $M$ trong không gian với hệ toạ độ $Oxyz$, kí hiệu là $M(a;b;c)$.
}{
	\begin{tikzpicture}[scale=0.6, font=\small,>=stealth]
		\path
		(0,0) coordinate (O)
		(-2,-2) coordinate (H)
		(3,-2) coordinate (M_1)
		(5,0) coordinate (K)
		(3,1) coordinate (M)
		(0,3) coordinate (P)
		;
		\draw[->] (0,0)--(6.7,0) node[below]{$y$};
		\draw[->] (0,0)--(-3,-3) node[below]{$x$};
		\draw[->] (0,0)--(0,4.3) node[left]{$z$};
		\draw[dashed] (P)node[left]{$c$}--(M)--(M_1)--(H)node[left]{$a$} (O)--(M_1)--(K)node[above]{$b$} (O)--(M);
		\foreach \x/\g in {O/160,M_1/-90,M/30,H/-80,K/-70,P/30}\draw[fill=black] (\x) circle (.05) +(\g:.5)node{\small$\x$};
		\foreach \x/\y/\z in {M_1/H/O,M_1/K/O,M/P/O}{\path pic[draw,angle radius=5pt]{right angle= \x--\y--\z};}
	\end{tikzpicture}
}
\subsubsection{Tọa độ của vectơ}
Trong KG $Oxyz$:
\immini{
	\begin{itemize}
		\item Toạ độ của điểm $M$ cũng là toạ độ của vectơ $\overrightarrow{OM}$.
		\item Cho $\vec{u}$. Dựng điểm $M(a;b;c)$ thỏa $\vec{OM}=\vec{u}$ thì tọa độ của điểm $M$ là tọa độ của $\vec{u}$. Theo hình vẽ thì
		      $$\vec{u}=\vec{OM}=\vec{OH}+\vec{OK}+\vec{OP}=a\vec{i}+b\vec{j}+c\vec{k}.$$
		      Suy ra
		      $$\vec{u}=\left(a;b;c \right)\Leftrightarrow \vec{u}=a\vec{i}+b\vec{j}+c\vec{k}. $$
	\end{itemize}
}{
	\begin{tikzpicture}[scale=0.6, font=\small,>=stealth]
		\path
		(0,0) coordinate (O)
		(-2,-2) coordinate (H)
		(3,-2) coordinate (M_1)
		(5,0) coordinate (K)
		(3,1) coordinate (M)
		(0,3) coordinate (P)
		;
		\draw[->] (0,0)--(6.7,0) node[below]{$y$};
		\draw[->] (0,0)--(-3,-3) node[below]{$x$};
		\draw[->] (0,0)--(0,4.3) node[left]{$z$};
		\draw[-stealth,blue,thick] (O)--(-1,-1)node[above]{$\vec{i}$};
		\draw[-stealth,blue,thick](O)--(1,0)node[below right]{$\vec{j}$};
		\draw[-stealth,blue,thick] (O)--(0,1)node[above left]{$\vec{k}$};
		\draw[dashed] (P)node[left]{$c$}--(M)--(M_1)--(H)node[left]{$a$} (O)--(M_1)--(K)node[above]{$b$};
		\draw[thick,->](O)--(M)node[midway,sloped,above,scale=1]{$\vec{u}$};
		\foreach \x/\g in {O/160,M_1/-90,M/30,H/-80,K/-70,P/30}\draw[fill=black] (\x) circle (.05) +(\g:.5)node{\small$\x$};
		\foreach \x/\y/\z in {M_1/H/O,M_1/K/O,M/P/O}{\path pic[draw,angle radius=5pt]{right angle= \x--\y--\z};}
	\end{tikzpicture}}
\begin{note}
	Tọa độ các véc tơ đơn vị lần lượt là: $\vec{i}=(1;0;0)$,\quad $\vec{j}=(0;1;0)$,\quad $\vec{k}=(0;0;1)$.
\end{note}
\subsection{PHÂN LOẠI VÀ PHƯƠNG PHÁP GIẢI TOÁN}

\begin{dang}{Tọa độ điểm, tọa độ vec tơ}
	\indamm{Khi xác định tọa độ điểm, tọa độ véc tơ ta chú ý các kết quả sau:}
	\begin{enumerate}
		\item $\vec{u}=a\vec{i}+b\vec{j}+c\vec{k} \Leftrightarrow \vec{u}=\big(a;b;c\big)$.
		\item $\vec{u}\big(u_1;u_2;u_3\big)=\vec{v}\big(v_1;v_2;v_3\big) \Leftrightarrow \heva{&u_1=v_1\\&u_2=v_2\\&u_3=v_3}$
		\item $\vec{OM}=(a;b;c)$ thì $M\big(a;b;c\big)$.
		\item $\vec{AB}=\big(x_B-x_A;y_B-y_A;z_B-z_A \big).$
		\item Chiếu điểm $M(a;b;c)$ lên mặt phẳng tọa độ (hoặc trục tọa độ) thì "thành phần bị khuyết" bằng $0$. Chẳng hạn: $M(1;2;3)$ chiếu lên $(Oxy)$ thì $z=0$. Suy ra hình chiếu là $M_1(1;2;0)$.
		\item Tứ giác $ABCD$ là hình bình hành khi và chỉ khi $$\vec{AD}=\vec{BC}$$
	\end{enumerate}
\end{dang}
\BTTL
\begin{vd}
	Trong KG $Oxyz$, cho $A(3 ;-2 ;-1)$. Gọi $ A_1, A_2, A_3$ lần lượt là hình chiếu của điểm $A$ trên các mặt phẳng toạ độ $(Oxy),(Oyz),(Oxz)$. Tìm toạ độ của các điểm $ A_1, A_2, A_3$.
	\loigiai{
		Toạ độ của các điểm $ A_1=(3 ;-2 ;0)$.\\
		Toạ độ của các điểm $ A_2=(3 ;0 ;-1)$.\\
		Toạ độ của các điểm $ A_3=(0 ;-2 ;-1)$
	}
\end{vd}

\begin{vd}
	Trong KG $Oxyz$, cho $A(-2;3;4)$. Gọi $H, K, P$ lần lượt là hình chiếu của điểm $A$ trên các trục $Ox, Oy, Oz$. Tìm tọa độ của các điểm $H,K,P$.
	\loigiai{
		Tìm tọa độ của các điểm $H=(-2;0;0)$.\\
		Tìm tọa độ của các điểm $K=(0;3;0)$.\\
		Tìm tọa độ của các điểm $P=(0;0;4)$.
	}
\end{vd}

\begin{vd} Trong KG $Oxyz$, cho $A(1; 1;-2)$, $B(4; 3; 1)$ và $C(-1;-2; 2)$.
	\begin{tasks}
		\task Tìm tọa độ của véctơ $\overrightarrow{A B}$.
		\task Tìm toạ độ của điểm $D$ sao cho $ABCD$ là hình bình hành.
	\end{tasks}
	\loigiai{
		\begin{enumerate}
			\item Ta có $
				      \overrightarrow{AB}=(4-1; 3-1; 1-(-2))=(3; 2; 3) .
			      $
			\item Gọi tọa độ của điểm $D$ là $\left(x_D; y_D; z_D\right)$, ta có
			      $
				      \overrightarrow{DC}=\left(-1-x_D;-2-y_D; 2-z_D\right) .
			      $\\
			      Tứ giác $A B C D$ là hình bình hành khi và chỉ khi
			      $$
				      \overrightarrow{DC}=\overrightarrow{A B} \Leftrightarrow\heva{&
					      - 1 - x _ { D } = 3 \\&
					      - 2 - y _ { D } = 2 \\&
					      2 - z _ { D } = 3.}
				      \Leftrightarrow \heva{&
					      x_D=-4 \\&
					      y_D=-4 \\&
					      z_D=-1.}$$
			      Vậy $D(-4;-4;-1)$.
		\end{enumerate}}
\end{vd}

\begin{vd}
	Trong KG $Oxyz$, cho hình hộp $ABCD \cdot A'B'C'D'$ có $A(4;6;-5)$, $B(5;7;-4)$, $C(5;6;-4)$, $D'(2;0;2)$. Tìm tọa độ các đỉnh còn lại của hình hộp $ABCD\cdot A'B'C'D'$.
	\loigiai{
		\begin{center}
			\begin{tikzpicture}[scale=0.7, font=\small, line join=round, line cap=round, >=stealth]
				\def\bc{4} % cạnh BC
				\def\ba{3} % cạnh BA
				\def\gocB{35} % góc B của đáy
				\coordinate[label=below left:$B(5;7;-4)$] (B) at (0,0);
				\coordinate[label=above left:$A(4;6;-5)$] (A) at (\gocB:\ba);
				\coordinate[label=below:$C(5;6;-4)$] (C) at (\bc,0);
				\coordinate[label=right:$D$] (D) at ($(C)-(B)+(A)$);
				\coordinate[label=above left:$A'$] (A') at ($(A)+(90:\bc)$);
				\coordinate[label=left:$B'$] (B') at ($(B)-(A)+(A')$);
				\coordinate[label=below right:$C'$] (C') at ($(C)-(A)+(A')$);
				\coordinate[label=right:$D'(2;0;2)$] (D') at ($(D)-(A)+(A')$);
				\draw (B')--(B)--(C)--(D)--(D')--(A')--(B')--(C')--(D') (C)--(C');
				\draw[dashed] (A')--(A)--(D) (A)--(B);
				\foreach \diem in {A,B,C,D,A',B',C',D'}	\fill (\diem)circle(1.5pt);
			\end{tikzpicture}
		\end{center}
		Ta có  $\overrightarrow{AD}=\overrightarrow{BC}\Leftrightarrow \heva{x_D&=x_A-x_B+x_C\\y_D&=y_A-y_B+y_C\\z_D&=z_A-z_B+z_C}\Leftrightarrow \heva{x_D&=4\\y_D&=5\\z_D&=-5}$. Suy ra $D(4;5;-5)$.\\
		Do đó $\overrightarrow{DD'}=(2-4;0-5;2-(-5)) =(-2;-5;7)$.\\
		Theo tính chất của hình hộp ta có $\overrightarrow{AA'}=\overrightarrow{BB'}=\overrightarrow{CC'}=\overrightarrow{DD'}=(-2;-5;7)$. Suy ra tọa độ đỉnh còn lại của hình hộp là $A'=(2;1;2)$, $B'(3;2;3)$, $C'(3;1;3)$.
	}
\end{vd}

\BTTN
\setcounter{ex}{0}
\Opensolutionfile{ans}[ans/2H2-B2-d1-1]

\begin{ex}
	Trong KG $Oxyz$, cho $\overrightarrow{a}=-2\overrightarrow{i}+3\overrightarrow{j}+5\overrightarrow{k}$. Toạ độ của véc-tơ $\overrightarrow{a}$ là
	\choice
	{$(2;-3;-5)$}
	{$(2;3;-5)$}
	{\True $(-2;3;5)$}
	{$(2;3;5)$}
	\loigiai{
		Toạ độ của véc-tơ $\overrightarrow{a}$ là $(-2;3;5)$.}
\end{ex} 

\begin{ex}
	Trong KG $Oxyz$, cho véc-tơ $\overrightarrow{u}=3\overrightarrow{i}+4\overrightarrow{k}-\overrightarrow{j}$. Tọa độ của véc-tơ $\overrightarrow{u}$ là
	\choice
	{\True $(3;-1;4)$}
	{$(3;4;-1)$}
	{$(4;-1;3)$}
	{$(4;3;-1)$}
	\loigiai
	{
		Tọa độ của véc-tơ $\overrightarrow{u}$ là $(3;-1;4)$.
	}
\end{ex} 

\begin{ex}
	Trong KG $Oxyz$, điểm nào sau đây thuộc trục $Oz$?
	\choice
	{$M(1;0;0)$}
	{$M(1;0;2)$}
	{$M(1;2;0)$}
	{\True $M(0;0;-2)$}
	\loigiai{
		Ta có $M(0;0;-2) \in Oz$.
	}
\end{ex} 

\begin{ex}%[An Do - Dự án 2H3-LVD]%[2H3Y1-1]%
	Trong KG $Oxyz$, cho điểm $M$ thỏa $\vec{OM} = 2\vec{i} + \vec{j}$. Tọa độ điểm $M$ là
	\choice
	{$M(0;2;1)$}
	{$M(1;2;0)$}
	{$M(2;0;1)$}
	{\True$M(2;1;0)$}
	\loigiai{
		Tọa độ $\vec{OM} = 2\vec{i} + \vec{j} = (2;0;0) + (0;1;0) = (2;1;0)$.\\
		Vậy $M (2;1;0)$.
	}
\end{ex} 

\begin{ex}%[2H3Y1-1]%[Đoàn Mạnh Hùng]%
	Trong KG $Oxyz$, cho vectơ $\overrightarrow{OA}=\overrightarrow{j}-2\overrightarrow{k}$. Tọa độ điểm $A$ là
	\choice
	{$(1;0;-2)$}
	{\True $(0;1;-2)$}
	{$(0;-1;2)$}
	{$(1;-2;0)$}
	\loigiai{
		Ta có $\overrightarrow{OA}=\vec{j}-2\vec{k}\Leftrightarrow A(0;1;-2)$.
	}
\end{ex} 

\begin{ex}
	Trong không gian $O x y z$, xác định toạ độ của điểm $A$ biết $A$ nằm trên tia $O x$ và $O A=2$.
	\choice
	{$A(0;0;2)$}
	{$A(2;2;0)$}
	{$A(0;2;0)$}
	{\True $A(2;0;0)$}
	\loigiai{$A$ nằm trên tia $O x$ và $O A=2$ nên $A(2;0;0)$.
	}
\end{ex} 

\begin{ex}
	Trong không gian $O x y z$, xác định toạ độ của điểm $A$ biết $A$ nằm trên tia đối của tia $O y$ và $O A=3$.
	\choice
	{$A(0;3;0)$}
	{\True $A(0;-3;0)$}
	{$A(0;-9;0)$}
	{$A(3;-3;0)$}
	\loigiai{
		$A$ nằm trên tia đối của tia $O y$ và $O A=3$ nên $A(0;-3;0)$.}
\end{ex} 

\begin{ex}
	Trong KG $Oxyz$, cho hai điểm $A(1;-1;2)$ và $B(2;1;-4)$. Véc-tơ $\vec{AB}$ có tọa độ là
	\choice
	{$(-1;-2;6)$}
	{$(3;0;-2)$}
	{$(1;0;-6)$}
	{\True $(1;2;-6)$}
	\loigiai{
		Ta có $\vec{AB} = (1;2;-6)$.
	}
\end{ex} 

\begin{ex}
	Trong không gian $ Oxyz $, cho hai điểm $ A(1;3;-2) $, $ B(3;-2;4) $. Véc-tơ $ \overrightarrow{AB} $ có tọa độ là
	\choice
	{\True $ (2;5;6) $}
	{$ (4;1;2) $}
	{\True $ (2;-5;6) $}
	{$ (-2;5;6) $}
	\loigiai{
		Véc-tơ $ \overrightarrow{AB} $ có tọa độ là $ (2;-5;6) $.}
\end{ex} 

\begin{ex}
	Cho hai điểm $A$, $B$ thỏa mãn $\vec{OA} = (2;-1; 3)$ và  $\vec{OB}= (5;2;-1)$. Tìm tọa độ véc-tơ $\vec{AB}$.
	\choice
	{$\vec{AB} =(2;-1;3)$}
	{\True  $\vec{AB} =(3;3;-4)$}
	{$\vec{AB} = (7;1;2)$}
	{$\vec{AB} =(3;-3;4)$}
	\loigiai{
		$\vec{AB} = \vec{OB} - \vec{OA} = (5-2;2+1;-1-3)=(3;3;-4)$.}
\end{ex} 

\begin{ex}
	Trong KG $Oxyz$, cho hai điểm $M$ và $N$ biết $M(2;1;-1)$ và $\vv{MN}=(-1;2-3)$. Tọa độ $N$ là
	\choice
	{$N(1;-3;-4)$}
	{\True $N(1;3;-4)$}
	{$N(-1;3;-4)$}
	{$N(1;3;4)$}
	\loigiai
	{
		Gọi $N(x,y,z)$, khi đó ta có $\heva{&x-2=-1\\&y-1=2\\&z+1=-3}\Leftrightarrow \heva{&x=1\\&y=3\\&z=-4}\Rightarrow N(1;3;-4)$.\\
	}
\end{ex} 

\begin{ex}
	Hình chiếu vuông góc của điểm $A(3;-4;5)$ trên mặt phẳng $(Oxz)$ là điểm
	\choice
	{$M(3;0;0)$}
	{$M(0;-4;5)$}
	{$M(0;0;5)$}
	{\True $M(3;0;5)$}
	\loigiai{
		Hình chiếu vuông góc của điểm $A(3;-4;5)$ trên mặt phẳng $(Oxz)$ là điểm $M(3;0;5)$.}
\end{ex} 

\begin{ex}
	Hình chiếu vuông góc của điểm $A(1;2;3)$ trên mặt phẳng $(Oxy)$ là điểm
	\choice
	{$M(0;0;3)$}
	{\True $N(1;2;0)$}
	{$Q(0;2;0)$}
	{$P(1;0;0)$}
	\loigiai
	{
		Hình chiếu vuông góc của điểm $A(1;2;3)$ trên mặt phẳng $(Oxy)$ là điểm $N(1;2;0)$.
	}
\end{ex} 

\begin{ex}
	Hình chiếu vuông góc của điểm $M(2;1;-3)$ lên mặt phẳng $(Oyz)$ có tọa độ là
	\choice
	{$(2;0;0)$}
	{$(2;1;0)$}
	{\True $(0;1;-3)$}
	{$(2;0;-3)$}
	\loigiai{
		Điểm thuộc $(Oyz)$ có tọa độ $(0;y;z)$ nên hình chiếu của $M$ lên $(Oyz)$ có tọa độ là $(0;-1;3)$.
	}
\end{ex} 

\begin{ex}
	Hình chiếu vuông góc của điểm $A(3;2;1)$ trên trục $Ox$ có tọa độ là
	\choice
	{$(0;2;1)$}
	{$(0;2;0)$}
	{\True $(3;0;0)$}
	{$(0;0;1)$}
	\loigiai{
		Hình chiếu vuông góc của điểm $A(3;2;1)$ lên trục $Ox$ là $A'(3;0;0)$.
	}
\end{ex} 

\begin{ex}
	Hình chiếu của điểm $M(2;3;-2)$ trên trục $Oy$ có tọa độ là
	\choice
	{$ (2;0;0) $}
	{\True $ (0;3;0) $}
	{$ (0;0;-2) $}
	{$ (2;0;-2) $}
	\loigiai{
		Hình chiếu của điểm $M(2;3;-2)$ trên trục $Oy$ có tọa độ là $(0;3;0)$.
	}
\end{ex} 

\begin{ex}
	\immini{Trong KG $Oxyz$, cho hình bình hành $ABCD$ với $A(-2;3;1)$, $B(3;0;-1)$, $C(6;5;0)$. Tọa độ đỉnh $D$ là
		\choice
		{$D(11;2;2)$}
		{\True $D(1;8;2)$}
		{$D(11;2;-2)$}
		{$D(1;8;-2)$}}{
		\begin{tikzpicture}[scale=0.7, font=\small,>=stealth]
			\path
			%	Vẽ mp
			(0,0) coordinate (A)
			(1,1.5) coordinate (B)
			(4,0) coordinate (D)
			($(D)+(B)-(A)$)coordinate (C)
			;
			\draw (A)--(B)--(C)--(D)--(A);
			\foreach \x/\g in {A/-90,B/90,C/0,D/-90}\draw[fill=black] (\x) circle (.05) +(\g:.5)node{\small$\x$};
		\end{tikzpicture}
	}
	\loigiai{
		Ta có $\heva{& x_D = x_A+x_C-x_B = 1 \\ & y_D = y_A +y_C -y_B = 8 \\ & z_D = z_A + z_C - z_B = 2}\Rightarrow D(1;8;2)$.
	}
\end{ex} 

\begin{ex}
	\immini{Trong KG $Oxyz$, cho các điểm $A(1;0;3)$, $B(2;3;-4)$,$C(-3;1;2)$. Tìm tọa độ điểm $D$ sao cho tứ giác $ABCD$ là hình bình hành.
		\choice
		{$D(4;2;9) $}
		{$D(-2;4;-5) $}
		{$D(6;2;-3) $}
		{\True $(-4;-2;9) $}}{
		\begin{tikzpicture}[scale=0.7, font=\small,>=stealth]
			\path
			%	Vẽ mp
			(0,0) coordinate (A)
			(1,1.5) coordinate (B)
			(4,0) coordinate (D)
			($(D)+(B)-(A)$)coordinate (C)
			;
			\draw (A)--(B)--(C)--(D)--(A);
			\foreach \x/\g in {A/-90,B/90,C/0,D/-90}\draw[fill=black] (\x) circle (.05) +(\g:.5)node{\small$\x$};
		\end{tikzpicture}}
	\loigiai{
		Gọi $D(x;y;z) \Rightarrow \vec{CD}=(x+3;y-1;z-2)$ và $\vec{BA}=(-1;-3;7)$.\\
		Để tứ giác $ABCD$ là hình bình hành ta có $\vec{BA}=\vec{CD}$ $\Rightarrow \heva{&x+3=-1\\&y-1=-3\\&z-2=7} \Rightarrow D(-4;-2;9)$.
	}
\end{ex} 

\begin{ex}
	\immini{Cho hình hộp $A B C D . A' B' C' D'$ có $A(1 ; 0 ; 1)$, $B(2 ; 1 ; 2)$, $D(1 ;-1 ; 1), C'(4 ; 5 ;-5)$. Tìm tọa độ đỉnh $C$ của hình hộp.
		\haicot
		{$C(2;0;2)$}
		{$C(2;0;2)$}
		{$C(2;0;2)$}
		{$C(2;0;2)$}}{
		\begin{tikzpicture}[scale=0.65, font=\small, line join=round, line cap=round, >=stealth]
			\def\bc{4} % cạnh BC
			\def\ba{2} % cạnh BA
			\def\gocB{35} % góc B của đáy
			\coordinate[label=below left:$B$] (B) at (0,0);
			\coordinate[label=above left:$A$] (A) at (\gocB:\ba);
			\coordinate[label=below:$C$] (C) at (\bc,0);
			\coordinate[label=right:$D$] (D) at ($(C)-(B)+(A)$);
			\coordinate[label=above left:$A'$] (A') at ($(A)+(100:\bc)$);
			\coordinate[label=left:$B'$] (B') at ($(B)-(A)+(A')$);
			\coordinate[label=below right:$C'$] (C') at ($(C)-(A)+(A')$);
			\coordinate[label=right:$D'$] (D') at ($(D)-(A)+(A')$);
			\draw (B')--(B)--(C)--(D)--(D')--(A')--(B')--(C')--(D') (C)--(C');
			\draw[dashed] (A')--(A)--(D) (A)--(B);
			\foreach \diem in {A,B,C,D,A',B',C',D'}	\fill (\diem)circle(1.5pt);
		\end{tikzpicture}}
	\loigiai{
		Ta có $\overrightarrow{AB}=\overrightarrow{DC}\Leftrightarrow \heva{& 2-1=x_C-1\\ & 1-0=y_C-(-1) \\ & 2-1=z_C-1} \Leftrightarrow \heva{&x_C = 2 \\ & y_C = 0 \\ & z_C=2}\Rightarrow C(2;0;2)$.
	}
\end{ex} 


\begin{ex} %[2H2H2-2]
	Cho hình hộp $A B C D . A' B' C' D'$ có $A(1 ; 0 ; 1)$, $B(2 ; 1 ; 2)$, $D(1 ;-1 ; 1), C'(4 ; 5 ;-5)$. Tìm tọa độ đỉnh $A'$ của hình hộp.
	\choice
	{$A'(-1;-5;8)$}
	{$A'(-1;-5;8)$}
	{$A'(-1;-5;8)$}
	{$A'(-1;-5;8)$}
	\loigiai{
		Ta có
		\begin{itemize}
			\item $\overrightarrow{AB}=\overrightarrow{DC}\Leftrightarrow \heva{& 2-1=x_C-1\\ & 1-0=y_C-(-1) \\ & 2-1=z_C-1} \Leftrightarrow \heva{&x_C = 2 \\ & y_C = 0 \\ & z_C=2}\Rightarrow C(2;0;2)$;
			\item $\overrightarrow{AA'}=\overrightarrow{CC'}\Leftrightarrow \heva{& x_{A'}-1=2-4\\ & y_{A'}-0=0-5 \\ & z_{A'}-1=2-(-5)} \Leftrightarrow \heva{& x_{A'} = -1 \\ & y_{A'} = -5 \\ & z_{A'}=8}\Rightarrow A'(-1;-5;8)$;
		\end{itemize}
	}
\end{ex} 


\begin{ex}%[2H2H2-2]
	\immini{Cho hình hộp $A B C D . A' B' C' D'$ có $A(1 ; 0 ; 1)$, $B(2 ; 1 ; 2)$, $D(1 ;-1 ; 1), C'(4 ; 5 ;-5)$. Tìm tọa độ đỉnh $D'$ của hình hộp.
		\haicot
		{$D'(-1;-6;8)$}
		{$D'(-1;-6;8)$}
		{$D'(-1;-6;8)$}
		{$D'(-1;-6;8)$}}{
		\begin{tikzpicture}[scale=0.65, font=\small, line join=round, line cap=round, >=stealth]
			\def\bc{4} % cạnh BC
			\def\ba{2} % cạnh BA
			\def\gocB{35} % góc B của đáy
			\coordinate[label=below left:$B$] (B) at (0,0);
			\coordinate[label=above left:$A$] (A) at (\gocB:\ba);
			\coordinate[label=below:$C$] (C) at (\bc,0);
			\coordinate[label=right:$D$] (D) at ($(C)-(B)+(A)$);
			\coordinate[label=above left:$A'$] (A') at ($(A)+(100:\bc)$);
			\coordinate[label=left:$B'$] (B') at ($(B)-(A)+(A')$);
			\coordinate[label=below right:$C'$] (C') at ($(C)-(A)+(A')$);
			\coordinate[label=right:$D'$] (D') at ($(D)-(A)+(A')$);
			\draw (B')--(B)--(C)--(D)--(D')--(A')--(B')--(C')--(D') (C)--(C');
			\draw[dashed] (A')--(A)--(D) (A)--(B);
			\foreach \diem in {A,B,C,D,A',B',C',D'}	\fill (\diem)circle(1.5pt);
		\end{tikzpicture}}
	\loigiai{
		Ta có
		\begin{itemize}
			\item $\overrightarrow{AB}=\overrightarrow{DC}\Leftrightarrow \heva{& 2-1=x_C-1\\ & 1-0=y_C-(-1) \\ & 2-1=z_C-1} \Leftrightarrow \heva{&x_C = 2 \\ & y_C = 0 \\ & z_C=2}\Rightarrow C(2;0;2)$;
			\item $\overrightarrow{DD'}=\overrightarrow{CC'}\Leftrightarrow \heva{& x_{D'}-1=2-4\\ & y_{D'}-(-1)=0-5 \\ & z_{D'}-1=2-(-5)} \Leftrightarrow \heva{& x_{D'} = -1 \\ & y_{D'} = -6 \\ & z_{D'}=8}\Rightarrow D'(-1;-6;8)$.
		\end{itemize}
	}
\end{ex} 

\Closesolutionfile{ans}
\BTTF
\Opensolutionfile{ans}[ans/2H2-B2-d1-2]
\begin{ex}
	Trong KG $Oxyz$, cho $\vec{a}=\vec{i}+3\vec{k}-4\vec{j}$ và $\vec{b}=\big(m-n;4m-6n;n^2-3m+2\big)$, với $m$, $n$ là tham số.
	\choiceTF
	{Tọa độ $\vec{a}=\big(1;3;-4\big)$}
	{\True Dựng điểm $A$ thỏa $\vec{OA}=\vec{a}$ thì $A(1;-4;3)$}
	{Tồn tại giá trị của $m$ và $n$ để $\vec{b}=\vec{0}$}
	{\True Nếu $\vec{a}=\vec{b}$ thì $m+n=9$}
	\loigiai{
		\begin{enumerate}[a)]
			\item Tọa độ $\vec{a}=\big(1;-4;3\big)$.
			\item Khi $\vec{OA}=\vec{a}$ thì tọa độ $\vec{a}$ cũng là tọa độ điểm $A$. Suy ra $A(1;-4;3)$.
			\item $\vec{b}=\vec{0} \Leftrightarrow \heva{&m-n=0\\&4m-6n=0\\&n^2-3m+2=0} \Leftrightarrow \heva{&m=0\\&n=0\\&n^2-3m+2=0}$ (vô nghiệm).\\
			      Vậy, không tồn tại $m$, $n$ để $\vec{b}=\vec{0}$.
			\item $\vec{a}=\vec{b} \Leftrightarrow \heva{&m-n=1\\&4m-6n=-4\\&n^2-3m+2=3} \Leftrightarrow \heva{&m=5\\&n=4}$.\\
			      Suy ra $m+n=9$.
		\end{enumerate}}
\end{ex} 
\begin{ex}
	\immini{Trong KG $Oxyz$, cho $\vec{a}=(2;2;0)$, $\vec{b}=2\vec{j}+2\vec{k}$. Dựng $\vec{OA}=\vec{a}$ và $\vec{OB}=\vec{b}$.
		\choiceTF
		{$\vec{a}=2\vec{i}+2\vec{k}$}
		{\True Toạ độ $\vec{b}=(0;2;2)$}
		{\True Toạ độ $\vec{AB}=(-2;2;0)$}
		{Góc $\widehat{AOB}=45^\circ$}}{
		\begin{tikzpicture}[scale=0.5, font=\small,>=stealth]
			\path
			(0,0) coordinate (O)
			(-2,-2) coordinate (H)
			(4,0) coordinate (K)
			(0,3.5) coordinate (P)
			($(P)+(H)-(O)$)coordinate (A)
			($(P)+(K)-(O)$)coordinate (B)
			;
			\draw[->] (0,0)--(6.7,0) node[below]{$y$};
			\draw[->] (0,0)--(-3,-3) node[below]{$x$};
			\draw[->] (0,0)--(0,5) node[left]{$z$};
			\draw[-stealth,blue,thick] (O)--(-1,-1)node[above]{$\vec{i}$};
			\draw[-stealth,blue,thick](O)--(1,0)node[below right]{$\vec{j}$};
			\draw[-stealth,blue,thick] (O)--(0,1)node[above right]{$\vec{k}$};
			\draw[dashed] (H)--(A)--(P)--(B)--(K);
			\draw[thick,->](O)--(A)node[midway,sloped,above,scale=1]{$\vec{a}$};
			\draw[thick,->](O)--(B)node[midway,sloped,below,scale=1]{$\vec{b}$};
			\foreach \x/\g in {O/-90,A/180,B/10}\draw[fill=black] (\x) circle (.05) +(\g:.5)node{\small$\x$};
		\end{tikzpicture}}
	\loigiai{
		\immini{
			\begin{enumerate}[a)]
				\item Ta có $\vec{a}=(2;0;2)\Rightarrow \vec{a}=2\vec{i}+2\vec{k} $.
				\item Ta có $\vec{b}=2\vec{j}+2\vec{k} \Rightarrow \vec{b}=(0;2;2)$.
				\item Ta có $\vec{OA}=\vec{a}$ thì toạ độ véc tơ $\vec{a}$ cũng chính là toạ độ $A$. Suy ra $A(2;0;2)$. Tương tự $B(0;2;2)$. Từ đây, ta tính được
				      $$\vec{AB}=(-2;2;0).$$
				\item Nhận xét $OHMK.PANB$ là hình lập phương. Suy ra $\triangle OAB$ đều. Vậy $\widehat{AOB}=60^\circ$.
			\end{enumerate}}{
			\begin{tikzpicture}[scale=0.8, font=\small,>=stealth]
				\path
				(0,0) coordinate (O)
				(-2,-2) coordinate (H)
				(4,0) coordinate (K)
				(0,3.5) coordinate (P)
				($(P)+(H)-(O)$)coordinate (A)
				($(P)+(K)-(O)$)coordinate (B)
				($(H)+(K)-(O)$)coordinate (M)
				($(A)+(B)-(P)$)coordinate (N)
				;
				\draw[->] (0,0)--(6.7,0) node[below]{$y$};
				\draw[->] (0,0)--(-3,-3) node[below]{$x$};
				\draw[->] (0,0)--(0,4.3) node[left]{$z$};
				\draw[-stealth,blue,thick] (O)--(-1,-1)node[above]{$\vec{i}$};
				\draw[-stealth,blue,thick](O)--(1,0)node[below right]{$\vec{j}$};
				\draw[-stealth,blue,thick] (O)--(0,1)node[above right]{$\vec{k}$};
				\draw[dashed] (H)--(A)--(P)--(B)--(K)--(M)--(H) (A)--(N)--(B) (M)--(N);
				\draw[thick,->](O)--(A)node[midway,sloped,above,scale=1]{$\vec{a}$};
				\draw[thick,->](O)--(B)node[midway,sloped,above,scale=1]{$\vec{b}$};
				\foreach \x/\g in {O/-90,A/180,B/10,H/-90,M/-90,K/20,P/30,N/0}\draw[fill=black] (\x) circle (.05) +(\g:.5)node{\small$\x$};
			\end{tikzpicture}
		}
	}
\end{ex} 

\begin{ex}
	\immini{Trong không gian $O x y z$, cho hình hộp $O A B C . O' A' B' C'$ có $A(1 ; 1 ;-1)$, $B(0 ; 3 ; 0)$, $\vec{BC'}=(2 ;-6 ; 6)$. Gọi $H$, $K$ lần lượt là trọng tâm của tam giác $OA'O'$ và $CB'C'$.
	\choiceTF
		{\True Tọa độ điểm $C'$ là $(2;-3;6)$}
		{\True Tọa độ điểm $O'$ là $(3;-5;5)$}
		{Tọa độ véc tơ $\vec{AB'}=(-2;3;-6)$}
		{Tọa độ véc tơ $\vec{HK}=(-1;2;-1)$}}
		{\begin{tikzpicture}[scale=0.5, font=\small,>=stealth]
			\path
			%	Vẽ mp
			(0,0) coordinate (O)
			(-1.5,-1) coordinate (A)
			(5,0) coordinate (C)
			($(A)+(C)-(O)$)coordinate (B)
			($(O)+(-.5,4)$)coordinate (O')
			($(A)+(C)-(O)$)coordinate (B)
			($(A)+(O')-(O)$)coordinate (A')
			($(A')+(B)-(A)$)coordinate (B')
			($(B')+(C)-(B)$)coordinate (C')
			;
			\draw (B)--(A)--(A')--(B')--(B)--(C)--(C')--(O')--(A') (B')--(C');
			\draw[thick,->] (B)--(C');
			\draw[dashed] (C)--(O)--(O') (O)--(A);
			\foreach \x/\g in {O/170,A/-90,B/-90,C/0,O'/90,A'/180,B'/-5,C'/10}\draw[fill=black] (\x) circle (.05) +(\g:.4)node{\small$\x$};
		\end{tikzpicture}}
	\loigiai{
		\begin{enumerate}[a)]
			\item Gọi $C'(x;y;z)$. Ta có $$\vec{BC'}=(2 ;-6 ; 6) \Rightarrow \heva{&x-0=2\\&y-3=-6\\&z-0=6} \Leftrightarrow \heva{&x=2\\&y=-3\\&z=6}$$
			      Vậy $C(2;-3;6)$.
			\item Gọi $O'(x;y;z)$. Theo hình vẽ thì
			      $$\vec{AO'}=\vec{BC'} \Leftrightarrow \heva{&x-1=2\\&y-1=-6\\&z+1=6} \Leftrightarrow \heva{&x=3\\&y=-5\\&z=5}$$
			      Vậy $O'(3;-5;5)$.
			\item Theo hình vẽ thì $\vec{AB'}=\vec{OC'}=(2;-3;6)$.
			\item Ta có $\vec{HK}=\vec{AB}=(-1;2;1)$.
		\end{enumerate}
	}
\end{ex} 
\Closesolutionfile{ans}

\begin{dang}{Tọa độ hóa một số hình không gian}
	\begin{listEX}[1]
		\item [\ding{172}] Chọn một điểm mà từ đó có ba đường đôi một vuông góc nhau làm gốc tọa độ.
		\item [\ding{173}] Xây dựng tọa độ các điểm trên hình đã cho tương ứng với hệ trục vừa chọn.
		\item [\ding{173}] Tọa độ các điểm đặc biệt:
		\begin{listEX}[3]
			\item [$\bullet$] $M \in Ox \Rightarrow M(x;0;0)$.
			\item [$\bullet$] $M \in Oy \Rightarrow M(0;y;0)$.
			\item [$\bullet$] $M \in Oz \Rightarrow M(0;0;z)$.
			\item [$\bullet$] $M \in (Oxy) \Rightarrow M(x;y;0)$.
			\item [$\bullet$] $M \in (Oxz) \Rightarrow M(x;0;z)$.
			\item [$\bullet$] $M \in (Oyz) \Rightarrow M(0;y;z)$.
		\end{listEX}
	\end{listEX}
\end{dang}
\BTTL
\begin{vd}
	\immini{Cho hình hộp chữ nhật $ABCD.A'B'C'D'$ có cạnh $AB=AA'=2$, $AD=4$. Gọi $E$ là tâm của hình chữ nhật $ABCD$, $F$ là trung điểm $AC'$. Với hệ toạ độ $Oxyz$ được thiết lập như hình bên (gốc tọa độ $O$ trùng với $A$), hãy xác định tọa độ các đỉnh của hình hộp chữ nhật và tọa độ hai điểm $E$, $F$.
	}{
		\begin{tikzpicture}[scale=0.7, font=\small,>=stealth]
			\path
			(0,0) coordinate (A)
			(-2,-2) coordinate (B)
			(5,0) coordinate (D)
			(0,3) coordinate (A')
			($(B)+(D)-(A)$)coordinate (C)
			($(A')+(B)-(A)$)coordinate (B')
			($(B')+(C)-(B)$)coordinate (C')
			($(A')+(D)-(A)$)coordinate (D')
			($(A)!0.5!(C)$)coordinate (E)
			($(A)!0.5!(C')$)coordinate (F)
			;
			\draw[->] (D)--(6.7,0) node[below]{$y$};
			\draw[->] (B)--(-3,-3) node[below]{$x$};
			\draw[->] (A')--(0,4.5) node[left]{$z$};
			\draw (B')--(B)--(C)--(D)--(D')--(A')--(B')--(C')--(D') (C)--(C');
			\draw[dashed] (A')--(A)--(B)--(D)--(A)--(C)--(A') (A)--(C');
			\draw[-stealth,blue,thick] (O)--(-0.6,-0.6)node[above]{$\vec{i}$};
			\draw[-stealth,blue,thick](O)--(1,0)node[below right]{$\vec{j}$};
			\draw[-stealth,blue,thick] (O)--(0,0.7)node[left]{$\vec{k}$};
			\foreach \x/\g in {A/-90,B/180,C/-70,D/10,A'/40,B'/180,C'/10,D'/0,E/-90,F/-10}\draw[fill=black] (\x) circle (.04) +(\g:.5)node{\small$\x$};
		\end{tikzpicture}}
	\loigiai{}
\end{vd}

\begin{vd}%[2H2V2-2]
	\immini{
		Một máy bay $M$ đang cất cánh từ phi trường. Với hệ toạ độ $Oxyz$ được thiết lập như Hình bên, cho biết $M$ là vị trí của máy bay với $OM=14$, $\widehat{NOB}=32^\circ$, $\widehat{MOC}=65^\circ$. Tính toạ độ điểm $M$.
	}{
		\begin{tikzpicture}[scale=0.6, font=\small,>=stealth]
			\path
			(0,0) coordinate (O)
			(-2,-2) coordinate (A)
			(3,-2) coordinate (N)
			(5,0) coordinate (B)
			(3,1) coordinate (M)
			(0,3) coordinate (C)
			;
			\draw[->] (0,0)--(6,0) node[below]{$y$};
			\draw[->] (0,0)--(-3,-3) node[below]{$x$};
			\draw[->] (0,0)--(0,4) node[left]{$z$};
			\draw[dashed] (C)--(M)--(N)--(A) (O)--(N)--(B);
			\draw[fill=blue] (M)circle (0.15)--(O)node[midway,sloped,scale=1,above]{$14$};
			\foreach \x/\g in {O/160,N/-90,M/30,A/-80,B/-70,C/30}\draw[fill=black] (\x) circle (.05) +(\g:.5)node{\small$\x$};
			\foreach \x/\y/\z in {N/A/O,N/B/O,M/C/O}{\path pic[draw,angle radius=5pt]{right angle= \x--\y--\z};}
			\draw pic["$65^\circ$",draw,angle eccentricity=1.9,angle radius=0.3cm]{angle=M--O--C};
			\draw pic["$32^\circ$",draw,angle eccentricity=1.9,angle radius=0.4cm]{angle=N--O--B};
		\end{tikzpicture}
	}
	\loigiai{
	\immini{
	Ta có:\\
	$OC=OM\cos 65^\circ\approx 5{,}9$.\\
	$ON=CN=OM\sin 65^\circ\approx 12{,}7$.\\
	$OB=ON\cos 32^\circ\approx 10{,}8$.\\
	$OA=BN=ON\sin 32^\circ\approx 6{,}7$.\\
	Vì $OANB$ là hình chữ nhật nên $\vec{ON}=\vec{OA}+\vec{OB}$.\\
	Vì $OCMN$ là hình chữ nhật nên $$\vec{OM}=\vec{OC}+\vec{ON}=\vec{OA}+\vec{OB}+\vec{OC}=6{,}7\vec{i}+10{,}8\vec{j}+5{,}9\vec{k}.$$
	Do đó $M(6{,}7; 10{,}8; 5{,}9)$.
	}{
	\begin{tikzpicture}[scale=1, font=\small, line join=round, line cap=round, >=stealth]
		\def\x{2.5}
		\def\y{4}
		\def\z{3}
		\def\gocXY{-150} % góc B của đáy
		\coordinate[label=above left:$O$] (O) at (0,0);
		\coordinate[label=below:$x$] (x) at (\gocXY:\x);
		\coordinate[label=below:$y$] (y) at (\y,0);
		\coordinate[label=right:$z$] (z) at (0,\z);
		\def\vtdv{1}
		\coordinate (i) at (\gocXY:\vtdv);
		\coordinate (j) at (\vtdv,0);
		\coordinate (k) at (0,\vtdv);
		\coordinate[label=above left:$A$] (A) at (\gocXY:0.7*\x);
		\coordinate[label=above:$B$] (B) at (0.8*\y,0);
		\coordinate[label=left:$C$] (C) at (0,0.8*\z);
		\coordinate[label=below:$N$] (N) at ($(A)+(B)$);
		\coordinate[label=right:$M$] (M) at ($(N)+(C)$);
		\draw[->] (O)--(x);
		\draw[->] (O)--(y);
		\draw[->] (O)--(z);
		\draw[->] (O)--(M);
		\draw[->, red] (O)--(i) node[left]{$\vec{i}$};
		\draw[->, red] (O)--(j) node[above]{$\vec{j}$};
		\draw[->, red] (O)--(k) node[left]{$\vec{k}$};
		\draw[dashed] (A)--(N)--(B) (O)--(N) (C)--(M)--(N);
		\foreach \diem in {A,B,C,O,N,M}	\fill (\diem)circle(1.5pt);
		\foreach \A/\B/\C in {O/C/M,N/B/y,N/A/O}
		\draw pic[draw=black,angle radius=6pt] {right angle = \A--\B--\C};
		\draw pic[draw,% double,% nét đôi
				blue,angle radius=5mm,angle eccentricity=2.5,"$32^\circ$"] {angle = N--O--y};
		\draw pic[draw,% double,% nét đôi
				blue,angle radius=3mm,angle eccentricity=2.5,"$65^\circ$"] {angle = M--O--C};
	\end{tikzpicture}
	}
	}
\end{vd}

\BTTN
\Opensolutionfile{ans}[ans/2H2-B2-d2-1]

\begin{ex}
	\immini{Hình bên mô tả một sân cầu lông với kích thước theo tiêu chuẩn quốc tế. Với hệ toạ độ $Oxyz$ được thiết lập như hình bên (đơn vị trên mỗi trục là mét), giả sử $AB$ là một trụ cầu lông để căng lưới, hãy xác định tọa độ của $B$.
		\choice
		{$\big(6,1;6,7;1,55\big)$}
		{\True $\big(6,7;6,1;1,55\big)$}
		{$\big(6,1;0;1,55\big)$}
		{$\big(0;6,7;1,55\big)$}
	}{
		\begin{tikzpicture}[scale=0.55, font=\small,>=stealth]
			\path
			%	Vẽ mp
			(0,0) coordinate (O)
			(8,0) coordinate (M)
			(10,2) coordinate (N)
			(2,2) coordinate (K)
			(4,1) coordinate (F)
			(4,0) coordinate (E)
			(6,3) coordinate (B)
			(6,2) coordinate (A)
			(2.3,-0.7) coordinate (I)
			($(A)!0.5!(B)$)coordinate (C)
			($(E)!0.5!(F)$)coordinate (D)
			;
			\draw[->] (M)--(10.5,0) node[below]{$x$};
			\draw[->] (K)--(3,3) node[above]{$y$};
			\draw[->] (O)--(0,4.5) node[left]{$z$};
			\draw[fill=green!20] (O)--(M)--(N)--(K)--cycle;
			\draw[pattern=north west lines] (C)--(B)--(F)--(D)--cycle;
			\draw (O)--(M)--(N)--(K)--(O) (E)--(F) (A)--(B);
			\draw[<->,dashed] (0,-0.3)--(8,-0.3)node[midway,sloped,below]{\scriptsize$13,40$ m};
			\draw[<->,dashed] (8.3,0)--(10.3,2)node[midway,right]{\scriptsize$6,10$ m};
			\draw[<->,dashed] (6.3,2)--(6.3,3)node[midway,right]{\scriptsize$1,55$ m};
			\foreach \x/\g in {O/180,A/-80,B/90}\draw[fill=black] (\x) circle (.05) +(\g:.5)node{\small$\x$};
		\end{tikzpicture}}
	\loigiai{
		\begin{itemize}
			\item Gọi toạ độ điểm $A$ là $\left(x_A;y_A;z_A\right)$. Vì chiều rộng của sân là $6,1 \mathrm{~m}$ nên $x_A=6,1$. Do một nửa chiều dài của sân là $6,7 \mathrm{~m}$ nên $y_A=6,7$. Điểm $A$ thuộc mặt phẳng $(Oxy)$ nên $z_A=0$. Vì vậy, điểm $A$ có tọa độ là $(6,1;6,7;0)$.
			\item Độ dài đoạn thẳng $AB$ là $1,55 \mathrm{~m}$ nên điểm $B$ có toạ độ là $(6,1;6,7;1,55)$.
		\end{itemize}
		Vậy ta có: $\overrightarrow{AB}=(6,1-6,1;6,7-6,7;1,55-0)$, tức là $\overrightarrow{AB}=(0;0;1,55)$.
	}
	\loigiai{
	}
\end{ex} 

\begin{ex}
	\immini{Cho hình lập phương $ABCD.A'B'C'D'$ có cạnh bằng 2. Với hệ toạ độ $Oxyz$ được thiết lập như hình bên (gốc tọa độ $O$ trùng với điểm $A$), tọa độ điểm $B'$ là
		\haicot
		{$B(0;2;0)$}
		{$B(2;2;2)$}
		{$B(2;2;0)$}
		{\True $B(2;0;2)$}
	}{
		\begin{tikzpicture}[scale=0.5, font=\small,>=stealth]
			\path
			(0,0) coordinate (A)
			(-2,-2) coordinate (B)
			(5,0) coordinate (D)
			(0,3) coordinate (A')
			($(B)+(D)-(A)$)coordinate (C)
			($(A')+(B)-(A)$)coordinate (B')
			($(B')+(C)-(B)$)coordinate (C')
			($(A')+(D)-(A)$)coordinate (D')
			;
			\draw[->] (D)--(6.7,0) node[below]{$y$};
			\draw[->] (B)--(-3,-3) node[below]{$x$};
			\draw[->] (A')--(0,4.5) node[left]{$z$};
			\draw (B')--(B)--(C)--(D)--(D')--(A')--(B')--(C')--(D') (C)--(C');
			\draw[dashed] (A')--(A)--(B) (A)--(D);
			\foreach \x/\g in {A/-90,B/180,C/-70,D/40,A'/40,B'/180,C'/10,D'/20}\draw[fill=black] (\x) circle (.04) +(\g:.6)node{\small$\x$};
		\end{tikzpicture}
	}
	\loigiai{
	}
\end{ex} 
\begin{ex}
	\immini{Cho hình lập phương $ABCD.A'B'C'D'$ có cạnh bằng 2. Với hệ toạ độ $Oxyz$ được thiết lập như hình bên (gốc tọa độ $O$ trùng với điểm $A$), tọa độ điểm $C'$ là
		\haicot
		{$C'(2;2;0)$}
		{\True $C'(2;2;2)$}
		{$C'(2;2;0)$}
		{$C'(2;0;2)$}
	}{
		\begin{tikzpicture}[scale=0.65, font=\small,>=stealth]
			\path
			(0,0) coordinate (A)
			(-2,-2) coordinate (B)
			(5,0) coordinate (D)
			(0,3) coordinate (A')
			($(B)+(D)-(A)$)coordinate (C)
			($(A')+(B)-(A)$)coordinate (B')
			($(B')+(C)-(B)$)coordinate (C')
			($(A')+(D)-(A)$)coordinate (D')
			;
			\draw[->] (D)--(6.7,0) node[below]{$y$};
			\draw[->] (B)--(-3,-3) node[below]{$x$};
			\draw[->] (A')--(0,4.5) node[left]{$z$};
			\draw (B')--(B)--(C)--(D)--(D')--(A')--(B')--(C')--(D') (C)--(C');
			\draw[dashed] (A')--(A)--(B) (A)--(D);
			\foreach \x/\g in {A/-90,B/180,C/-70,D/40,A'/40,B'/180,C'/10,D'/20}\draw[fill=black] (\x) circle (.04) +(\g:.6)node{\small$\x$};
		\end{tikzpicture}
	}
	\loigiai{
	}
\end{ex} 


\begin{ex}
	\immini{Cho hình chóp tứ giác đều $S.ABCD$ có cạnh đáy bằng $a\sqrt{2}$, cạnh bên bằng $a\sqrt{5}$. Gọi $O$ là tâm của hình vuông $ABCD$. Với hệ toạ độ $Oxyz$ được thiết lập như hình bên (gốc tọa độ $O$ trùng với tâm hình vuông $ABCD$), tọa độ $\vec{SC}$ là
		\choice
		{$\vec{SC}=(2a;0;-2a)$}
		{$\vec{SC}=(2a;-a;-2a)$}
		{\True $\vec{SC}=(a;0;-2a)$}
		{$\vec{SC}=(a;0;2a)$}}{
		\begin{tikzpicture}[scale=0.65, font=\small,>=stealth]
			\path
			(0,0) coordinate (A)
			(-3,-2) coordinate (B)
			(5,0) coordinate (D)
			($(B)+(D)-(A)$)coordinate (C)
			($(A)!0.5!(C)$)coordinate (O)
			($(O)+(0,4)$)coordinate (S)
			;
			\draw[->] (D)--(7,0.5) node[below]{$y$};
			\draw[->] (C)--(3,-3) node[below]{$x$};
			\draw[->] (S)--(1,4.5) node[left]{$z$};
			\draw (C)--(D)--(S)--(C)--(B)--(S);
			\draw[dashed] (S)--(A)--(D)--(B)--(A)--(C) (S)--(O);
			\pic[draw,thin,angle radius=2mm] {right angle = C--O--D};
			\foreach \x/\g in {A/180,B/-90,C/-100,D/-30,S/10,O/-90}\draw[fill=black] (\x) circle (.04) +(\g:.6)node{\small$\x$};
		\end{tikzpicture}}
	\loigiai{
	}
\end{ex} 

\begin{ex}%[2H2H2-2]
	\immini{
		Cho tứ diện $SABC$ có $ABC$ là tam giác vuông tại $B$, $BC=3$, $BA=2$, $SA$ vuông góc với mặt phẳng $(ABC)$ và có độ dài bằng $2$. Với hệ toạ độ $Oxyz$ được thiết lập như hình bên (gốc tọa độ $O$ trùng với điểm $B$), tìm khẳng định \textbf{sai}.
		\haicot
		{$A(0; 2; 0)$}
		{$B(0; 0; 0)$}
		{$C(0; 0; 3)$}
		{\True $S(-2; 2; 2)$}
	}{
		\begin{tikzpicture}[scale=1, font=\small, line join=round, line cap=round, >=Stealth]
			\path
			(0:0) coordinate (B)
			(20:4) coordinate (x)
			(90:3) coordinate (z)
			(130:2) coordinate (y)
			($(B)!.7!(x)$) coordinate (C)
			($(B)!4/6!(y)$) coordinate (A)
			($(B)!3/5!(z)$) coordinate (H)
			($(A)+(H)-(B)$) coordinate (S)
			;
			\draw[->] (B)--(x);
			\draw[->] (B)--(y);
			\draw[->] (B)--(z);
			\draw[dashed] 	(A)--(C)
			;
			\draw (A)--(S) (B)--(S)--(C);
			\pic[draw,angle radius=2mm]{right angle=C--B--A}
			pic[draw,angle radius=2mm]{right angle=C--B--H}
			pic[draw,angle radius=2mm]{right angle=H--B--A}
			;
			\foreach \x/\g in {B/-90,x/90,y/180,z/0,C/-90,A/210,H/0,S/90}
			\draw[fill=black] 	(\x)
			($(\g:.2)+(\x)$) node {$\x$};
		\end{tikzpicture}
	}
	\loigiai{
	}
\end{ex} 

\begin{ex}%[2H2H2-2]
	\immini{Cho hình chóp $S.ABC$ có đáy $ABC$ là tam giác đều cạnh bằng $2$, $SA$ vuông góc với đáy và $SA =1$. Với hệ toạ độ $Oxyz$ được thiết lập như hình bên (gốc tọa độ $O$ trùng với trung điểm của đoạn $BC$), hãy tìm toạ độ điểm $S$.
		\haicot
		{$S(0;\sqrt{3};1)$}
		{$S(0;\sqrt{3};1)$}
		{$S(0;\sqrt{3};1)$}
		{$S(0;\sqrt{3};1)$}
	}{
		\begin{tikzpicture}[ font = \small, scale =1,>=stealth]
			\path
			(0:0) coordinate (A)
			++(0:4) coordinate (C)
			++(-160:3)coordinate(B)
			(A)++(90:2) coordinate (S)
			($(B)!1/2!(C)$) coordinate (O)
			($(S)+(O)-(A)$) coordinate (H)
			($(O)!1.4!(A)$) coordinate (y) node[above]{$y$}
			($(O)!1.7!(C)$) coordinate (x) node[above]{$x$}
			($(O)!1.4!(H)$) coordinate (z) node[right]{$z$}
			(intersection of S--C and O--H) coordinate (t)
			;
			\draw (S)--(A)--(B)--(C) (S)--(B) (O)--(z) (S)--(H) (C)--(t)
			;
			\draw[dashed] (C)--(A)--(O) (S)--(t)
			;
			\draw[->] (C)--(x);
			\draw[->] (A)--(y);
			\draw[->] (H)--(z);
			\pic[draw,thin,angle radius=2mm] {right angle = A--O--B}
			pic[draw,thin,angle radius=2mm] {right angle = O--H--S}
			pic[draw,thin,angle radius=2mm] {right angle = H--O--C}
			;
			\foreach \x/\g in {A/-100,C/-50,B/-90,O/-70,H/0,S/90}
			\fill (\x) circle (1pt)
			+(\g:3mm) node{$\x$};
		\end{tikzpicture}
	}
	\loigiai{
	}
\end{ex} 
\begin{ex}%[2H2V2-6]
	\immini{Ở một sân bay, vị trí của máy bay được xác định bởi điểm $M$ Trong KG $Oxyz$ như hình bên. Gọi $H$ là hình chiếu vuông góc của $M$ xuống mặt phẳng $(Oxy)$. Cho biết $OM = 50$, $\left(\overrightarrow{i},\overrightarrow{OH}\right) = 64^\circ$, $\left(\overrightarrow{OH},\overrightarrow{OM}\right) = 48^\circ$. Tìm toạ độ của điểm $M$.
		\choice
		{$M(14{,}7; 30{,}1; 37{,}2)$}
		{$M(14{,}7; 30{,}1; 37{,}2)$}
		{$M(14{,}7; 30{,}1; 37{,}2)$}
		{$M(14{,}7; 30{,}1; 37{,}2)$}
	}{
		\begin{tikzpicture}[scale=0.85, font=\small,>=stealth]
			\path
			(0,0) coordinate (O)
			(-2,-2) coordinate (A)
			(3,-2) coordinate (H)
			(5,0) coordinate (B)
			(3,1) coordinate (M)
			(0,3) coordinate (C)
			;
			\draw[->] (0,0)--(6,0) node[below]{$y$};
			\draw[->] (0,0)--(-3,-3) node[below]{$x$};
			\draw[->] (0,0)--(0,4) node[left]{$z$};
			\draw[dashed] (C)--(M)--(H)--(A) (O)--(H)--(B);
			\draw[fill=blue] (M)circle (0.15)--(O)node[midway,sloped,scale=1,above]{$50$};
			\foreach \x/\g in {O/160,H/-90,M/30,A/-80,B/-70,C/30}\draw[fill=black] (\x) circle (.05) +(\g:.5)node{\small$\x$};
			\foreach \x/\y/\z in {H/A/O,H/B/O,M/C/O}{\path pic[draw,angle radius=5pt]{right angle= \x--\y--\z};}
			\draw pic["\scriptsize$48^\circ$",draw,angle eccentricity=1.9,angle radius=0.45cm]{angle=H--O--M};
			\draw pic["\scriptsize$64^\circ$",draw,angle eccentricity=1.9,angle radius=0.3cm]{angle=A--O--H};
		\end{tikzpicture}
	}
	\loigiai{
		\immini{
			Tam giác $OMH$ vuông tại $H$, $OM = 50$; $\widehat{MOH} = 48^\circ$ nên ta có
			\begin{itemize}
				\item [$\bullet$] $OH = OM\cdot \cos 48 \approx 33{,}5$
				\item [$\bullet$] $OC = MH = OM \cdot \sin 48 \approx 37{,}2$.
			\end{itemize}
			Tam giác $OAH$ vuông tại $A$, $OH = 33{,}5$; $\widehat{AOH} = 64^\circ$ nên ta có
			\begin{itemize}
				\item [$\bullet$] $OA = OH\cdot \cos 64 \approx 14{,}7$,
				\item [$\bullet$] $OB = AH = OH\cdot \sin 64 \approx 30{,}1$.
			\end{itemize}
			Suy ra
			\begin{eqnarray*}
				\overrightarrow{OM} & = & \overrightarrow{OC} + \overrightarrow{OH} = \overrightarrow{OC} + \overrightarrow{OA}+\overrightarrow{OB} \\
				& = & 14{,}7\overrightarrow{i}+30{,}1\overrightarrow{j}+37{,}2\overrightarrow{k}.
			\end{eqnarray*}
			Vậy $M(14{,}7; 30{,}1; 37{,}2)$.
		}{
			\begin{tikzpicture}[scale=0.85, font=\small,>=stealth]
				\path
				(0,0) coordinate (O)
				(-2,-2) coordinate (A)
				(3,-2) coordinate (H)
				(5,0) coordinate (B)
				(3,1) coordinate (M)
				(0,3) coordinate (C)
				;
				\draw[->] (0,0)--(6,0) node[below]{$y$};
				\draw[->] (0,0)--(-3,-3) node[below]{$x$};
				\draw[->] (0,0)--(0,4) node[left]{$z$};
				\draw[dashed] (C)--(M)--(H)--(A) (O)--(H)--(B);
				\draw[fill=blue] (M)circle (0.15)--(O)node[midway,sloped,scale=1,above]{$50$};
				\foreach \x/\g in {O/160,H/-90,M/30,A/-80,B/-70,C/30}\draw[fill=black] (\x) circle (.05) +(\g:.5)node{\small$\x$};
				\foreach \x/\y/\z in {H/A/O,H/B/O,M/C/O}{\path pic[draw,angle radius=5pt]{right angle= \x--\y--\z};}
				\draw pic["\scriptsize$48^\circ$",draw,angle eccentricity=1.9,angle radius=0.45cm]{angle=H--O--M};
				\draw pic["\scriptsize$64^\circ$",draw,angle eccentricity=1.9,angle radius=0.3cm]{angle=A--O--H};
			\end{tikzpicture}
		}
	}
\end{ex} 

\Closesolutionfile{ans}
\BTTF
\Opensolutionfile{ans}[ans/2H2-B2-d2-2]
\begin{ex}
	\immini{Cho hình chóp $S.ABCD$ có đáy $ABCD$ là hình chữ nhật, $AB=1$, $AD=2$, $SA$ vuông góc với mặt đáy và $SA=3$. Với hệ toạ độ $Oxyz$ được thiết lập như sau: Gốc tọa độ $O$ trùng với điểm $A$, các véc tơ $\vec{AB}$, $\vec{AD}$, $\vec{AS}$ lần lượt cùng hướng với $\vec{i}$, $\vec{j}$ và $\vec{k}$. Xét tính đúng sai của các khẳng định sau
		\choiceTF
		{\True Tọa độ $D(0;2;0)$}
		{Tọa độ $C(1;2;3)$}
		{\True Tọa độ $S(2;0;0)$}
		{Tọa độ $I(1;1;0)$}
	}{
		\begin{tikzpicture}[scale=0.6, font=\small,>=stealth]
			\path
			(0,0) coordinate (A)
			(-2,-2) coordinate (B)
			(5,0) coordinate (D)
			($(B)+(D)-(A)$)coordinate (C)
			($(A)!0.5!(C)$)coordinate (I)
			($(A)+(0,3)$)coordinate (S)
			;
			\draw (C)--(D)--(S)--(C)--(B)--(S);
			\draw[dashed] (S)--(A)--(D)--(B)--(A)--(C);
			\pic[draw,thin,angle radius=2mm] {right angle = B--A--D};
			\foreach \x/\g in {A/180,B/-90,C/-100,D/-80,S/90,I/-90}\draw[fill=black] (\x) circle (.04) +(\g:.5)node{\small$\x$};
		\end{tikzpicture}	}
	\loigiai{
		\immini{
			Với hệ trục đã chọn như hình vẽ thì
			\begin{enumerate}[a)]
				\item Điểm $D \in Oy$ và $AD=2$ nên $D(0;2;0)$.
				\item Điểm $C \in (Oxy)$ và có hình chiếu lên $Ox$, $Oy$ lần lượt là điểm $B$ và $D$.\\
				      Do $AB=1$ và $AD=2$ nên $C(2;2;0)$.
				\item Điểm $S \in Oz$ và $AS=3$ nên $S(0;0;3)$.
				\item Điểm $I \in (Oxy)$ và và có hình chiếu lên $Ox$, $Oy$ lần lượt là trung điểm của $AB$ và $AD$ nên $I(0,5;1;0)$.
			\end{enumerate}}{
			\begin{tikzpicture}[scale=0.6, font=\small,>=stealth]
				\path
				(0,0) coordinate (A)
				(-2,-2) coordinate (B)
				(5,0) coordinate (D)
				($(B)+(D)-(A)$)coordinate (C)
				($(A)!0.5!(C)$)coordinate (I)
				($(A)+(0,3)$)coordinate (S)
				;
				\draw[->] (D)--(7,0) node[below]{$y$};
				\draw[->] (B)--(-3,-3) node[below]{$x$};
				\draw[->] (S)--(0,4) node[left]{$z$};
				\draw (C)--(D)--(S)--(C)--(B)--(S);
				\draw[dashed] (S)--(A)--(D)--(B)--(A)--(C);
				\pic[draw,thin,angle radius=2mm] {right angle = B--A--D};
				\foreach \x/\g in {A/180,B/-90,C/-100,D/-80,S/170,I/-90}\draw[fill=black] (\x) circle (.04) +(\g:.5)node{\small$\x$};
			\end{tikzpicture}}
	}
\end{ex} 


\begin{ex}
	\immini{Cho hình lập phương $ABCD.A'B'C'D'$ có cạnh bằng $2$. Với hệ toạ độ $Oxyz$ được thiết lập như hình bên (gốc tọa độ $O$ trùng với tâm hình vuông $ABCD$), hãy xét tính đúng sai của các khẳng định sau:
		\choiceTF
		{Tọa độ $A(-1;0;0)$}
		{\True $\vec{AC'}=(2\sqrt{2};0;2)$}
		{\True Tọa độ $D'(0;\sqrt{2};2)$}
		{$\vec{BD'}=(0;0;2)$}
	}{
		\begin{tikzpicture}[scale=0.45, font=\small,>=stealth]
			\path
			(0,0) coordinate (A)
			(-2,-2) coordinate (B)
			(6,0) coordinate (D)
			(0,4) coordinate (A')
			($(B)+(D)-(A)$)coordinate (C)
			($(A')+(B)-(A)$)coordinate (B')
			($(B')+(C)-(B)$)coordinate (C')
			($(A')+(D)-(A)$)coordinate (D')
			($(A)!0.5!(C)$)coordinate (O)
			($(A')!0.5!(C')$)coordinate (O')
			;
			\draw[->] (D)--(8,0.5) node[below]{$y$};
			\draw[->] (C)--(6,-3) node[below]{$x$};
			\draw[->] (O')--(2,5.5) node[left]{$z$};
			\draw (B')--(B)--(C)--(D)--(D')--(A')--(B')--(C')--(D') (C)--(C')--(A') (B')--(D');
			\draw[dashed] (A')--(A)--(B)--(D)--(A)--(C) (O)--(O');
			\pic[draw,thin,angle radius=2mm] {right angle = C--O--D};
			\foreach \x/\g in {A/-90,B/180,C/-70,D/-40,A'/40,B'/180,C'/10,D'/20,O/-90}\draw[fill=black] (\x) circle (.04) +(\g:.65)node{\small$\x$};
		\end{tikzpicture}
	}
	\loigiai{
		Độ dài $AC=2\sqrt{2}$. Với hệ trục $Oxyz$ đã chọn như hình vẽ thì
		\begin{enumerate}[a)]
			\item Điểm $A \in Ox$, nằm ngược chiều dương và $OA=\sqrt{2}$ nên $A(-\sqrt{2};0;0)$.
			\item Tọa độ $C'(\sqrt{2};0;2)$. Suy ra $\vec{AC'}=(2\sqrt{2};0;2)$.
			\item Điểm $D'$ có hình chiếu vuông góc xuống $(Oxy)$ là điểm $D(0;\sqrt{2};0)$ và $DD'=2$ nên $D'(0;\sqrt{2};2)$.
			\item Tọa độ $B(0;-\sqrt{2};0)$, $D'(0;\sqrt{2};2)$. Suy ra $\vec{BD'}=(0;2\sqrt{2};2)$.
		\end{enumerate}
	}
\end{ex} 

\begin{ex}
	\immini{Cho hình lăng trụ $ABC.A'B'C'$ có đáy $ABC$ là tam giác đều cạnh bằng $2$ như hình vẽ. Hình chiếu vuông góc của $A'$ lên $(ABC)$ trùng với trung điểm cạnh $AB$, góc $\widehat{A'AO}=60^\circ$. Với hệ toạ độ $Oxyz$ được thiết lập như hình bên (gốc tọa độ $O$ trùng với trung điểm của đoạn $BC$), hãy xét tính đúng sai của các khẳng định sau:
		\choiceTF
		{\True Tọa độ điểm $A(-1;0;0)$}
		{\True Tọa độ điểm $C(0;\sqrt{3};0)$}
		{Tọa độ điểm $A'(0;-1;\sqrt{3})$}
		{\True Tọa độ điểm $C'\big(1;\sqrt{3};\sqrt{3}\big)$}
	}{
		\begin{tikzpicture}[scale=0.7, font=\small,>=stealth]
			\path
			(0,0) coordinate (A)
			(2,-2) coordinate (B)
			(5,0) coordinate (C)
			(1,4) coordinate (A')
			($(A)!0.5!(B)$)coordinate (O)
			($(A')+(B)-(A)$)coordinate (B')
			($(A')+(C)-(A)$)coordinate (C')
			;
			\draw[->] (B)--(3,-3) node[below]{$x$};
			\draw[->] (C)--(7,0.5) node[below]{$y$};
			\draw[->] (A')--(1,5) node[left]{$z$};
			\draw (B)--(A)--(A')--(B')--(B)--(C)--(C')--(B') (A')--(C') (O)--(A');
			\draw[dashed] (A)--(C)--(O);
			\pic[draw,thin,angle radius=2mm] {right angle = C--O--B};
			\pic[draw,thin,angle radius=2mm] {right angle = A'--O--B};
			\pic[draw,thin,angle radius=2mm] {right angle = A'--O--C};
			\foreach \x/\g in {A/180,B/-90,C/-90,A'/180,B'/-20,C'/0,O/-100}\draw[fill=black] (\x) circle (.05) +(\g:.5)node{\small$\x$};
		\end{tikzpicture}}
	\loigiai{
		Độ dài $OC=2.\dfrac{\sqrt{3}}{2}=\sqrt{3}$. $OA'=OA.\tan60^\circ=\sqrt{3}$. Với hệ trục $Oxyz$ đã chọn như hình vẽ trên thì
		\begin{enumerate}[a)]
			\item Điểm $A \in Ox$, nằm ngược chiều dương và $OA=1$ nên $A(-1;0;0)$.
			\item Điểm $A' \in Oy$, nằm cùng chiều dương và $OC=\sqrt{3}$ nên $C(0;\sqrt{3};0)$.
			\item $A' \in Oz$, nằm cùng chiều dương và $OA'=\sqrt{3}$ nên $A'(0;0;\sqrt{3})$.
			\item Gọi $C'(x;y;z)$. Ta có
			      $$\vec{A'C'}=\vec{AC} \Leftrightarrow\heva{&x-0=1\\&y-0=\sqrt{3}\\&z-\sqrt{3}=0}\Leftrightarrow\heva{&x=1\\&y=\sqrt{3}\\&z=\sqrt{3}}.$$
		\end{enumerate}
	}
\end{ex} 


\Closesolutionfile{ans}

%%Bài 3. Biểu thức tọa độ
% \setcounter{section}{2}
\setcounter{dang}{0}
\section{BIỂU THỨC TỌA ĐỘ CỦA CÁC PHÉP TOÁN VECTƠ}
\subsection{LÝ THUYẾT CẦN NHỚ}
\subsubsection{Biểu thức tọa độ của phép toán cộng, trừ, nhân một số thực với một vectơ}
Trong không gian $Oxyz$, cho hai véc-tơ $\vec{a} = (a_1;a_2;a_3)$, $\vec{b} = (b_1; b_2; b_3)$ và số $k$. Khi đó
\begin{listEX}[1]
	\item [\ding{172}] $\vec{a}+\vec{b}=(a_1+b_1;a_2+b_2;a_3+b_3)$;
	\item [\ding{173}] $\vec{a}-\vec{b}=(a_1-b_1;a_2-b_2;a_3-b_3)$;
	\item [\ding{174}] $k\vec{a} = (ka_1; ka_2; ka_3)$.
\end{listEX}
\begin{note}
	Cho hai véc-tơ $\vec{a}=(a_1;a_2;a_3)$, $\vec{b}=(b_1;b_2;b_3)$, $\vec{b}\ne \vec{0}$. Hai véc-tơ $\vec{a}$, $\vec{b}$ cùng phương khi và chỉ khi tồn tại một số thực $k$ sao cho $\heva{&a_1=k b_1\\& a_2= k b_2\\& a_3= k b_3.}$
\end{note}
\subsubsection{Biểu thức tọa độ của tích vô hướng hai vectơ}
Trong không gian $Oxyz$, tích vô hướng của hai véc-tơ $\vec{a} = (a_1;a_2;a_3)$ và $\vec{b} = (b_1; b_2; b_3)$ được xác định bởi công thức
\[\vec{a} \cdot \vec{b} = a_1b_1 + a_2b_2 + a_3b_3. \]
\begin{note}
	\begin{itemize}
		\item[\ding{172}] $\vec{a} \perp \vec{b} \Leftrightarrow a_1b_1 + a_2b_2 + a_3b_3 = 0$;
		\item[\ding{173}] $\left| \vec{a} \right| = \sqrt{a_1^2 + a_2^2 +a_3^2}$; \quad $AB=\sqrt{(x_B-x_A)^2+(y_B-y_A)^2+(z_B-z_A)^2}$.
		\item[\ding{174}] $\cos \left(\vec{a}; \vec{b}\right) = \dfrac{\vec{a}\cdot \vec{b}}{\left|\vec{a}\right| \cdot \left|\vec{b}\right|} = \dfrac{a_1b_1 + a_2b_2 + a_3b_3}{\sqrt{a_1^2 + a_2^2 +a_3^2} \cdot \sqrt{b_1^2 + b_2^2 +b_3^2}}$ (với $\vec{a},\vec{b} \ne \vec{0}$).
	\end{itemize}
\end{note}

\subsubsection{Biểu thức tọa độ của tích có hướng hai vectơ}
Cho hai véc-tơ $\vec{a}=(a_1;a_2;a_3)$ và $\vec{b}=(b_1;b_2;b_3)$ không cùng phương. Khi đó vec tơ $$\vec{w}=\bigg(a_2b_3-b_2a_3\,;\,a_3b_1-b_3a_1\,;\,a_1b_2-b_1a_2 \bigg)$$ vuông góc với cả hai véc tơ $\vec{a}$ và $\vec{b}$.
\begin{note}
	\begin{itemize}
		\item [\ding{172}] Véc tơ $\vec{w}$ xác định như trên còn gọi là \textbf{tích có hướng} của hai vectơ $\vec{a}$, $\vec{b}$, kí hiệu  $\vec{w}=\left[\vec{a},\vec{a}\right]$.
		\item [\ding{173}] Quy ước $\left|\begin{array}{l}
				      {a_1}\quad{a_2} \\
				      {b_1}\quad{b_2}
			      \end{array}\right|=a_1b_2-a_2b_1$ thì
		      $$\left[\vec a ,\vec b\right]=\left(\left|\begin{array}{l}
					      {a_2}\quad{a_3} \\
					      {b_2}\quad{b_3}
				      \end{array}\right|;\left|\begin{array}{l}
					      {a_3}\quad {a_1} \\
					      {b_3}\quad{b_1}
				      \end{array}\right|;\left|\begin{array}{l}
					      {a_1}\quad{a_2} \\
					      {b_1}\quad{b_2}
				      \end{array}\right|\right)$$
		\item [\ding{174}] $\vec{a}$ không cùng phương với $\vec{b}$ $\Leftrightarrow \left[\vec a ,\vec b\right] \ne \vec{0}$.
	\end{itemize}
\end{note}
\subsubsection{Ứng dụng của tích có hướng của hai véc-tơ}
	\begin{enumerate}
		\item Xét sự đồng phẳng của ba véc-tơ:
		\begin{itemize}
			\item Ba vectơ $\vec{a}$; $\vec{b}$; $\vec{c}$ đồng phẳng $\Leftrightarrow \left[ \vec{a},\vec{b} \right]\cdot \vec{c}=0$.
			\item Bốn điểm $A$, $B$, $C$, $D$ tạo thành tứ diện $\Leftrightarrow \left[ \vec{AB},\vec{AC} \right]\cdot \vec{AD}\ne 0$.
		\end{itemize}
		\item Diện tích hình bình hành: $S_{ABCD}=\left| \left[ \vec{AB},\vec{AD} \right] \right|$.
		\item Tính diện tích tam giác: $S_{\triangle ABC}=\dfrac{1}{2}\left| \left[ \vec{AB},\vec{AC} \right] \right|$.
		\item Tính thể tích hình hộp: $V_{ABCD.A'B'C'D'}=\left| \left[ \vec{AB},\vec{AC} \right]\cdot\vec{AA'} \right|$.
		\item Tính thể tích tứ diện: $V_{ABCD}=\dfrac{1}{6}\left| \left[ \vec{AB},\vec{AC} \right]\cdot \vec{AD} \right|$.
	\end{enumerate} 
\subsubsection{Biểu thức tọa độ trung điểm đoạn thẳng, trọng tâm tam giác}
\immini{Trong không gian $Oxyz$, tọa độ trung điểm và trong tâm được xác định như sau:
	\begin{itemize}
		\item [\ding{172}] Tọa độ trung điểm $M$ của đoạn thẳng $AB$ là
		      \[ M\left(\dfrac{x_A + x_B}{2}; \dfrac{y_A + y_B}{2}; \dfrac{z_A + z_B}{2} \right).\]
		\item [\ding{173}] Tọa độ trọng tâm $G$ của tam giác $ABC$ là
		      \[ G\left(\dfrac{x_A + x_B +x_C}{3}; \dfrac{y_A + y_B +y_C}{3}; \dfrac{z_A + z_B + z_C}{3} \right).\]
	\end{itemize}}
	{\begin{tikzpicture}[scale=0.8, font=\footnotesize, line join=round, line cap=round]
		\begin{scope}
			\foreach \x\y\t in {-2/-2/A, 0/0/B}
			\coordinate (\t) at (\x,\y);
			\coordinate (M) at ($(A)!0.5!(B)$);
			\foreach \a\b in {A/B}
			\draw[] (\a)--(\b);
			\foreach \t\g in {A/-90, B/40,M/1200}
			\draw[fill=black] (\t)circle(0.6pt) +(\g:8pt)node{$\t$};
		\end{scope}
		\begin{scope}[xshift=3cm]
			\foreach \x\y\t in {0/0/A, -2/-2/B, 2.5/-2/C}
			\coordinate (\t) at (\x,\y);
			\coordinate (M) at ($(A)!0.5!(B)$);
			\coordinate (N) at ($(A)!0.5!(C)$);
			\coordinate (K) at ($(C)!0.5!(B)$);
			\coordinate (G) at ($(A)!2/3!(K)$);
			\foreach \a\b in {A/B, B/C, A/C, A/K, M/C, B/N}
			\draw[] (\a)--(\b);
			\foreach \t\g in {A/90, B/-100, C/-80, M/120, N/40, K/-90,G/60}
			\draw[fill=black] (\t)circle(0.8pt) +(\g:10pt)node{$\t$};
		\end{scope}
	\end{tikzpicture}}
\subsection{PHÂN LOẠI VÀ PHƯƠNG PHÁP GIẢI TOÁN}
\begin{dang}{Tọa độ của các phép toán vec tơ, tọa độ điểm, độ dài đoạn thẳng}
\end{dang}
\BTTL
\begin{vd}
	Cho $\vec{a}=(-2 ; 3 ; 2), \vec{b}=(2 ; 1 ;-1), \vec{c}=(1 ; 2 ; 3)$. Tính tọa độ của mỗi vectơ sau:
	\begin{listEX}[3]
		\item $3 \vec{a}$;
		\item $2 \vec{a}-\vec{b}$;
		\item $\vec{a}+2 \vec{b}-\dfrac{3}{2} \vec{c}$.
	\end{listEX}
	\loigiai{
		Ta có
		\begin{listEX}
			\item $3 \vec{a}=(3 \cdot(-2) ; 3 \cdot 3 ; 3 \cdot 2)$. Vậy $3 \vec{a}=(-6 ; 9 ; 6)$.
			\item Ta có $2 \vec{a}=(-4 ; 6 ; 4)$ và $\vec{b}=(2 ; 1 ;-1)$.\\ Do đó, $2 \vec{a}-\vec{b}=(-4-2 ; 6-1 ; 4-(-1))$.\\
			Vậy $2 \vec{a}-\vec{b}=(-6 ; 5 ; 5)$.
			\item Do $\vec{a}=(-2 ; 3 ; 2)$ và $2 \vec{b}=(4 ; 2 ;-2)$ nên
			\[\vec{a}+2 \vec{b}=(2 ; 5 ; 0).\]
			Ngoài ra, vì $-\dfrac{3}{2} \vec{c}=\left(-\dfrac{3}{2} ;-3 ;-\dfrac{9}{2}\right)$ nên $\vec{a}+2 \vec{b}-\dfrac{3}{2} \vec{c}=\left(\dfrac{1}{2} ; 2 ;-\dfrac{9}{2}\right)$.
		\end{listEX}}
\end{vd}
\dongcham{8}
\begin{vd}
	Trong không gian $Oxyz$, cho các véc-tơ $\vec{u}=3\vec{i}-2\vec{j}+\vec{k}$, $\vec{v}=-\dfrac{3}{2}\vec{i}+\vec{j}-\dfrac{1}{2}\vec{k}$, $\vec{w}=6\vec{i}+m\vec{j}-n\vec{k}$.
	\begin{enumerate}
		\item Chứng minh $\vec{u}$ và $\vec{v}$ cùng phương.
		\item Tìm giá trị của $m$ và $n$ để véc-tơ $\vec{u}$ và $\vec{w}$ cùng phương.
	\end{enumerate}
	\loigiai{
		Ta có $\vec{u}=(3;-2;1)$, $\vec{v}=\left(-\dfrac{3}{2}; 1; -\dfrac{1}{2}\right)$, $\vec{w}=\left(6; m; -n\right)$.
		\begin{enumerate}
			\item Hai véc-tơ $\vec{u}$ và $\vec{v}$ cùng phương khi và chỉ khi
			      $$\vec{v}=k\vec{u}\Leftrightarrow{ \left\{\begin{aligned}& -\dfrac{3}{2}=3k\\&1=-2k\\&-\dfrac{1}{2}=k\end{aligned}\right.}\Leftrightarrow k=-\dfrac{1}{2}$$
			      Như vậy $ \vec{v}=-\dfrac{1}{2}\vec{u} $ nên hai véc-tơ $\vec{u}$ và $\vec{v}$ cùng phương.
			\item Hai véc-tơ $\vec{u}$ và $\vec{w}$ cùng phương khi và chỉ khi
			      $$\vec{w}=k\vec{u}\Leftrightarrow{ \left\{\begin{aligned}&6=3k\\&m=-2k\\&-n=k\end{aligned}\right.}\Leftrightarrow { \left\{\begin{aligned}&k=2\\&m=-4\\&n=-2\end{aligned}\right.}$$
			      Như vậy $ m=-4$ và $ n=-2 $ thì hai véc-tơ $\vec{u}$ và $\vec{w}$ cùng phương. Khi đó $\vec{w}=\left(6; -4; 2\right)$.
		\end{enumerate}
	}
\end{vd}
\dongcham{8}
\begin{vd}
	Trong không gian với hệ tọa độ $Oxyz$, cho ba điểm $A(3;-1;2)$, $B(1;2;3)$, $C(4;-2;1)$.
	\begin{tasks}
		\task Chứng minh ba điểm $A, B, C$ không thẳng hàng. Xác định tọa độ trọng tâm tam giác $ABC$.
		\task Tìm tọa độ điểm $D$ biết $ABCD$ là hình bình hành.
		\task Tìm tọa độ giao điểm $E$ của đường thẳng $BC$ với mặt phẳng tọa độ $\left(Oxz\right)$.
	\end{tasks}
	\loigiai{
		\begin{enumerate}
			\item Ta có $\vec{AB}=(-2;3;1)$, $\vec{AC}=(1;-1;-1)$.
			      Vì $\dfrac{-2}{1}\neq \dfrac{-3}{-1}$ nên hai véc-tơ $\vec{AB}$, $\vec{AC}$ không cùng phương.\\
			      Hay ba điểm $A$, $B$, $C$ không thẳng hàng. Suy ra, tọa độ trọng tâm là $G\left(\dfrac{8}{3};-\dfrac{1}{3};2 \right)$.
			\item
			      \immini{Tứ giác $ABCD$ là hình bình hành khi và chỉ khi
				      $$ \vec{DC}=\vec{AB}\Leftrightarrow{ \left\{\begin{aligned}&4-x_D=-2\\&-2-y_D=3\\&1-z_D=1\end{aligned}\right.}\Leftrightarrow {\left\{\begin{aligned}&x_D=6\\&y_D=-5\\&z_D=0\end{aligned}\right.}$$
				      Vậy $D(6;-5;0)$.
			      }{
				      \begin{tikzpicture}[scale=.8]
					      \tkzDefPoints{0/4/A,-2/0/B,3/0/C}
					      \coordinate (D) at ($(A)+(C)-(B)$);
					      \tkzDrawSegments(A,B B,C C,D D,A)
					      \tkzLabelPoints[left](A,B)
					      \tkzLabelPoints[right](C,D)
					      \tkzDrawPoints(A,B,C,D)
				      \end{tikzpicture}}
			\item Vì $E$ thuộc mặt phẳng $Oxz$ nên $E=(x;0;z)$.\\
			      Ta có $\vec{AE}=(x-3;1;z-2)$.\\
			      Mặt khác $A, B, E$ thẳng hàng nên hai véc-tơ $\vec{AB}$, $\vec{AE}$ cùng phương, do đó:
			      $$ \vec{AE}=k\vec{AB}\Leftrightarrow{ \left\{\begin{aligned}&x-3=-2k\\&1=3k\\&z-2=k\end{aligned}\right.}\Leftrightarrow {\left\{\begin{aligned}&x=\dfrac{7}{3}\\&k=\dfrac{1}{3}\\&z=\dfrac{7}{3}\end{aligned}\right.}$$
			      Vậy $E=\left(\dfrac{7}{3}; 0; \dfrac{7}{3}\right)$.
		\end{enumerate}
	}
\end{vd}
\dongcham{18}
\begin{vd}
	Trong không gian $Oxyz$, cho ba điểm $A(5;-3;0)$, $B(2;1;-1)$, $C(4;1;2)$.
	\begin{enumerate}
		\item Tìm tọa độ của vectơ $\vec{u}=2\vec{AB}+\vec{AC}-5\vec{BC}$.
		\item Tìm tọa độ điểm $N$ sao cho $2\vec{NA}=-\vec{NB}$.
	\end{enumerate}
	\loigiai{
		\begin{enumerate}
			\item Ta có $\heva{&A(5;-3;0)\\ &B(2;1;-1)\\&C(4;1;2)}\Rightarrow\heva{&\vec{AB}=(-3;4;-1)\\&\vec{AC}=(-1;4;2)\\&\vec{BC}=(2;0;3)}\Rightarrow\heva{&2\vec{AB}=(-6;8;-2)\\&\vec{AC}=(-1;4;2)\\&-5\vec{BC}=(-10;0;-15)}\Rightarrow \vec{u}=(-17;12;-15)$.
			\item Gọi $N(x;y;z)$, khi đó $\heva{&\vec{NA}=(5-x;-3-y;-z)\\&\vec{NB}=(2-x;1-y;-1-z)}$\\
			      $$2\vec{NA}=-\vec{NB}\Leftrightarrow \heva{&2(5-x)=-2+x\\&2(-3-y)=-1+y\\&-2z=1+z}\Leftrightarrow \heva{&x=4\\&y=-\dfrac{5}{3}\\&z=-\dfrac{1}{3}}\Rightarrow N\left(4;-\dfrac{5}{3};-\dfrac{1}{3}\right).$$
		\end{enumerate}
	}
\end{vd}
\dongcham{14}
\begin{vd}%[2H2H2-6]
	\immini{Một phòng học có thiết kế dạng hình hộp chữ nhật với chiều dài là $8$ m, chiều rộng là $6$ m và chiều cao là $3$ m. Một chiếc đèn được treo tại chính giữa trần nhà của phòng học. Xét hệ trục toạ độ $Oxyz$ có gốc $O$ trùng với một góc phòng và mặt phẳng $(Oxy)$ trùng với mặt sàn, đơn vị đo được lấy theo mét (\textit{Hình minh họa bên}). Hãy tìm toạ độ của điểm treo đèn.}{
		\begin{tikzpicture}[line cap=round,line join=round, >=stealth,scale=0.6]
			\path (0,0)coordinate[label=left:$O$](O) (-1,-1)coordinate(B) (4,0)coordinate(D) (3,-1)coordinate(C) (0,2)coordinate(O') (-1,1)coordinate(B') (3,1)coordinate(C') (4,2)coordinate(D') (0,3)coordinate[label=right:$z$](E) (5,0)coordinate[label=above:$y$](F) (-2,-2)coordinate[label=left:$x$](G) (-1,0)coordinate[label=left:$3$ m](H) (1,-1)coordinate[label=below:$8$ m](I)
			(3.5,-0.5)coordinate[label=right:$6$ m](K);
			\draw (B)--(B')--(C')--(C)--cycle (O')--(B') (O')--(D') (C')--(D') (D')--(D) (D)--(C);
			\draw[dashed] (O)--(B) (O)--(D) (O)--(O');
			\draw[->] (B)--(G);
			\draw[->] (O')--(E);
			\draw[->] (D)--(F);
		\end{tikzpicture}
	}
	\loigiai{
		\immini{Gọi các điểm $B(3;0;0)$, $C(3;6;0)$, $D(0;6;0)$ như hình vẽ.\\
			$N$ là trung điểm $OC$, $N'$ là hình chiếu của $N$ lên mặt phẳng trần nhà.\\
			Suy ra $N'$ là điểm treo đèn.\\
			Ta có $N$ có tọa độ là $\left(\dfrac{0+3}{2};\dfrac{0+6}{2};\dfrac{0+0}{2}\right)$, suy ra $N\left(\dfrac{3}{2};3;0\right)$.\\
			Suy ra $N'\left(\dfrac{3}{2};3;3\right)$.\\
			Vậy tọa độ của điểm treo đèn là $\left(\dfrac{3}{2};3;3\right)$.
		}
		{\begin{tikzpicture}[line cap=round,line join=round, >=stealth,scale=0.6]
				\path (0,0)coordinate[label=left:$O$](O) (-1,-1)coordinate[label=left:$B$](B) (4,0)coordinate[label=above right:$D$](D) (3,-1)coordinate[label=below:$C$](C) (0,2)coordinate(O') (-1,1)coordinate(B') (3,1)coordinate(C') (4,2)coordinate(D') (0,3)coordinate[label=right:$z$](E) (6,0)coordinate[label=above:$y$](F) (-2,-2)coordinate[label=left:$x$](G) (-1,0)coordinate[label=left:$3$ m](H) (1,-1)coordinate[label=below:$8$ m](I)
				(3.5,-0.5)coordinate[label=right:$6$ m](K);
				\coordinate[label=left:$N$] (N) at (intersection cs:first line={(O)--(C)}, second line={(B)--(D)});
				\coordinate[label=left:$N'$] (N') at (intersection cs:first line={(O')--(C')}, second line={(B')--(D')});
				\draw (B)--(B')--(C')--(C)--cycle (O')--(B') (O')--(D') (C')--(D') (D')--(D) (D)--(C);
				\draw[dashed] (O)--(B) (O)--(D) (O)--(O') (O)--(C) (B)--(D) (N)--(N');
				\draw[->] (B)--(G);
				\draw[->] (O')--(E);
				\draw[->] (D)--(F);
			\end{tikzpicture}}
	}
\end{vd}
\dongcham{12}
\BTTN
\Opensolutionfile{ans}[ans/2H2-B3-d1-1]
Các câu hỏi sau đều xét trong không gian $Oxyz$.
\begin{ex}
	Cho $\vec{a}=(1;2;-3),\vec{b}=(-2;-4;6)$. Khẳng định nào sau đây đúng?
	\choice
	{$\vec{a}=2\vec{b}$}
	{$\vec{b}=2\vec{a}$}
	{\True $\vec{b}=-2\vec{a}$}
	{$\vec{a}=-2\vec{b}$}
	\loigiai{
		Ta có: $-2\vec{a}=\left(-2;-4;6\right)=\vec{b}$.
	}
\end{ex} \dongcham{3}

\begin{ex}
	Cho hai véc-tơ $\vec{x}=(2;1;-3),\vec{y}=(1;0;-1)$. Tìm tọa độ của véc-tơ $\vec{a}=\vec{x}+2\vec{y}$.
	\choice
	{\True $\vec{a}(4;1;-5)$}
	{$\vec{a}(4;1;-1)$}
	{$\vec{a}(3;1;-4)$}
	{$\vec{a}(0;1;-1)$}
	\loigiai{
		Ta có $\vec{a}=(2;1;-3)+2\cdot (1;0;-1)=(4;1;-5)$.}
\end{ex} \dongcham{5}

\begin{ex}
	Cho $\vec{a}=(1;-1;3)$, $\vec{b}=(2;0;-1)$. Tìm tọa độ véc-tơ $\vec{u}=2\vec{a}-3\vec{b}$.
	\choice
	{\True $\vec{u}=(-4;-2;9)$}
	{$\vec{u}=(4;2;-9)$}
	{$\vec{u}=(-4;-5;9)$}
	{$\vec{u}=(1;3;-11)$}
	\loigiai{
		$\vec{u}=2\vec{a}-3\vec{b}=(-4;-2;9)$.
	}
\end{ex} \dongcham{5}

\begin{ex}
	Cho hai véc-tơ  $\vec{a}=(3;0;1)$,  $\vec{c}=(1;1;0)$. Tìm tọa độ của véc-tơ $\vec{b}$ thỏa mãn biểu thức  $\vec{b}-\vec{a}+2\vec{c}=\vec{0}$.
	\choice
	{$\vec{b}=(-2;1;-1)$}
	{$\vec{b}=(-1;2;-1)$}
	{$\vec{b}=(5;2;1)$}
	{\True $\vec{b}=(1;-2;1)$}
	\loigiai
	{Gọi $\vec{b}=\left(x; y; z\right)$. Ta có
		$$\vec{b}-\vec{a}+2\vec{c}=\vec{0}\Leftrightarrow\heva{& x-3+2\cdot 1=0 \\ & y-0+2\cdot 1=0\\ & z-1+2\cdot 0=0}\Leftrightarrow\heva{& x=1 \\ & y=-2\\ & z=1.}$$
		Vậy $\vec{b}=(1;-2;1)$.
	}
\end{ex} \dongcham{5}

\begin{ex}
	Cho vectơ $\vec{a}=(1;-3;4)$. Vectơ  nào sau đây cùng phương với $\vec{a}$?
	\choice
	{$\vec{b}=(-2;-6;8)$}
	{\True $\vec{c}=(-2;6;-8)$}
	{$\vec{d}=(-2;6;8)$}
	{$\vec{m}=(2;-6;-8)$}
	\loigiai
	{
		$$\vec{b}=(-2;6;-8)=-2\vec{a}.$$
	}
\end{ex} \dongcham{4}

\begin{ex}
	Hai véc-tơ $\vec{a}= (m; 2; 3)$ và $\vec{b}= (1; n; 2)$ cùng phương khi
	\choice
	{$\heva{&m=\dfrac{1}{2}\\ &n= \dfrac{4}{3}.}$}
	{\True $\heva{&m=\dfrac{3}{2}\\ &n= \dfrac{4}{3}.}$}
	{$\heva{&m=\dfrac{3}{2}\\ &n= \dfrac{2}{3}.}$}
	{$\heva{&m=\dfrac{2}{3}\\ &n= \dfrac{4}{3}.}$}
	\loigiai
	{
		YCBT $\Leftrightarrow \exists k\in {{\mathbb{R}}^{*}}:\vec{a}=k.\vec{b}\Leftrightarrow \heva{
				& m=k.1 \\
				& 2=k.n \\
				& 3=k.2 \\
			}\Rightarrow \heva{
				& m=\dfrac{3}{2} \\
				& n=\dfrac{4}{3}. \\
			}$
	}
\end{ex} \dongcham{4}

\begin{ex}
	Cho hai điểm $A(2;3;1)$ và  $B(3;1;5)$. Tính độ dài đoạn thẳng $AB$.
	\choice
	{\True  $AB= \sqrt{21}$}
	{$AB= 2\sqrt{3}$}
	{$AB= 2\sqrt{5}$}
	{$AB= \sqrt{13}$}
	\loigiai{
		$AB = \sqrt{(3-2)^2 + (1-3)^2 +(5-1)^2} = \sqrt{21}$.}
\end{ex} \dongcham{4}

\begin{ex}
	Cho hai điểm $M(3;-2;1)$ và $N(0;1;-1)$. Tính độ dài đoạn thẳng $MN$.
	\choice
	{$MN=\sqrt{17}$}
	{$MN=22$}
	{\True $MN=\sqrt{22}$}
	{$MN=\sqrt{19}$}
	\loigiai
	{
		Ta có $\vec{MN}=(-3;3;-2)\Rightarrow MN=\sqrt{9+9+4}=\sqrt{22}$.
	}
\end{ex} \dongcham{4}

\begin{ex}
	Cho hai điểm $A(-1;1;2)$ và $B(3;-5;0)$. Tọa độ trung diểm của đoạn thẳng $AB$ là
	\choice
	{\True $(1;-2;1)$}
	{$(4;-6;2)$}
	{$(2;-3;-1)$}
	{$(2;-4;2)$}
	\loigiai{
		Gọi $M$ là trung điểm $AB$, khi đó tọa độ của $M$ được tính bởi
		\[ \heva{&x_M=\dfrac{x_A+y_A}{2}=1\\ &y_M=\dfrac{y_A+y_B}{2}=-2\\ &z_M=\dfrac{z_A+z_B}{2}=1.} \]
	}
\end{ex} \dongcham{4}

\begin{ex}
	Cho hai điểm $A(1;1;0)$, $B(3;-1;2)$. Tọa độ điểm $C$ sao cho $B$ là trung điểm của đoạn $AC$ là
	\choice
	{\True $C(5;-3;4)$}
	{$C(4;-3;5)$}
	{$C(-1;3;-2)$}
	{$C(2;0;1)$}
	\loigiai{
		Ta có $\heva{&x_B=\dfrac{x_A+x_C}{2}\\&y_B=\dfrac{y_A+y_C}{2}\\&z_B=\dfrac{z_A+z_C}{2}}\Rightarrow \heva{&x_C=2x_B-x_A=5\\&y_C=2y_B-y_A=-3\\&z_C=2z_B-z_A=4.}$
	}
\end{ex} \dongcham{5}

\begin{ex}
	Cho tam giác $ABC$ với $A(0;-1;3)$, $B(2;1;1)$, $C(1;0;-1)$. Tọa độ trọng tâm của tam giác $ABC$ là
	\choice
	{\True$(1;0;1)$}
	{$(-1;0;1)$}
	{$(0;1;1)$}
	{$(1;1;0)$}
	\loigiai{
		Gọi $G$ là trọng tâm của tam giác $ABC$. Khi đó $\heva{&x_G=\dfrac{x_A+x_B+x_C}{3}\\&y_G=\dfrac{y_A+y_B+y_C}{3}\\&z_G=\dfrac{z_A+z_B+z_C}{3}}$ $\Leftrightarrow \heva{&x_G=1\\&y_G=0\\&z_G=1.}$\\
		Vậy tọa độ trọng tâm tam giác $ABC$ là $(1;0;1)$.
	}
\end{ex} \dongcham{5}

\begin{ex}
	Cho $\vec{OA}=\vec{i}-2\vec{j}+3\vec{k}$, điểm $B(3;-4;1)$ và $C(2;0;-1)$. Tọa độ trọng tâm của tam giác $ABC$ là
	\choice
	{$(1;-2;3)$}
	{$(-1;2;-3)$}
	{\True $(2;-2;1)$}
	{$(-2;2;-1)$}
	\loigiai{
		Từ giả thiết: $\vec{OA}=\vec{i}-2\vec{j}+3\vec{k} \Rightarrow A(1;-2;3)$. \\
		Gọi $G$ là trọng tâm tam giác $ABC$, ta có: $\left\{\begin{aligned}
				 & x_G=\dfrac{x_A+x_B+x_C}{3}=2  \\
				 & y_G=\dfrac{y_A+y_B+y_C}{3}=-2 \\
				 & z_G=\dfrac{z_A+z_B+z_C}{3}=1  \\
			\end{aligned}\right. \Rightarrow G(2;-2;1)$. \\
		Vậy trọng tâm của tam giác $ABC$ là điểm $G(2;-2;1)$.}
\end{ex} \dongcham{5}

\begin{ex}
	Cho tam giác $ABC$ trọng tâm $G$. Biết $A(0;2;1)$, $B(1;-1;2)$, $G(1;1;1)$. Khi đó điểm $C$ có tọa độ là
	\choice
	{$(2;2;4)$}
	{$(-2;0;2)$}
	{$(-2;-3;-2)$}
	{\True $(2;2;0)$}
	\loigiai{
		\begin{itemize}
			\item [$\bullet$] Giả sử tọa độ $C$ là $C(a;b;c)$ khi đó $\heva{&\dfrac{0+1+a}{3}=1 \\ &\dfrac{2-1+b}{3}=1 \\ &\dfrac{1+2+c}{3}=1} \Leftrightarrow \heva{&a=2 \\ &b=2 \\ &c=0.}$
			\item [$\bullet$] Vậy điểm $C$ có tọa độ là $(2;2;0)$.
		\end{itemize}
	}
\end{ex} \dongcham{5}

\begin{ex}
	Cho bốn điểm $A(1;0;3)$, $B(2;-1;1)$, $C(-1;3;-4)$, $D(2;6;0)$ tạo thành một hình tứ diện. Gọi $M$, $N$ lần lượt là trung điểm các đoạn thẳng $AB$, $CD$. Tìm tọa độ trung điểm $G$ của đoạn $MN$.
	\choice
	{$G\left(\dfrac{4}{3};\dfrac{8}{3};0\right)$}
	{$G(2;4;0)$}
	{\True $G(1;2;0)$}
	{$G(4;8;0)$}
	\loigiai{
		Gọi $M$ là trung điểm đoạn thẳng $AB\Rightarrow M\left(\dfrac{3}{2};-\dfrac{1}{2};2\right)$.\\
		Gọi $N$ là trung điểm đoạn thẳng $CD\Rightarrow N\left(\dfrac{1}{2};\dfrac{9}{2};-2\right))$.\\
		Gọi $G$ là trung điểm đoạn thẳng $MN\Rightarrow G(1;2;0)$.
	}
\end{ex} \dongcham{5}


\begin{ex}
	Cho hai điểm $B(1;2;-3)$, $C(7;4;-2)$. Nếu $E$ là điểm thỏa mãn đẳng thức $\vec{CE}=2\vec{EB}$ thì tọa độ điểm $E$ là
	\choice
	{$\left(3;\dfrac{8}{3};\dfrac{8}{3}\right)$}
	{$\left(1;2;\dfrac{1}{3}\right)$}
	{$\left(3;3;-\dfrac{8}{3}\right)$}
	{\True $\left(\dfrac{8}{3};3;-\dfrac{8}{3}\right)$}
	\loigiai{
		$E(x;y;z)$, từ $\vec{CE}=2\vec{EB}\Rightarrow\heva{&x=\dfrac{8}{3}\\&y=3\\&z=-\dfrac{8}{3}.}$}
\end{ex} \dongcham{5}

\begin{ex}
	Cho các điểm $A(1;-1;0)$, $B(0;2;0)$, $C(2;1;3)$ và $M$ là điểm thỏa mãn hệ thức $\vec{MA}-\vec{MB}+\vec{MC}=\vec{0}$. Khi đó điểm $M$ có tọa độ là
	\choice
	{$(3;2;3)$}
	{$(3;-2;-3)$}
	{\True$(3;-2;3)$}
	{$(3;2;-3)$}
	\loigiai
	{
		Gọi $M(x;y;z)$, ta có $\heva{&1-x-(0-x)+(2-x)&=0\\&-1-y-(2-y)+1-y&=0\\&0-z-(0-z)+3-z&=0}\Leftrightarrow \heva{&x=3\\&y=-2\\&z=3}\Rightarrow M(3;-2;3).$
	}
\end{ex} \dongcham{5}

\begin{ex}
	Cho tọa độ các điểm $A(-1;3);B(2;-2)$ và $C(m;1)$. Tìm $m$ để $3$ điểm $A,B,C$ thẳng hàng.
	\choice
	{$m=\dfrac{2}{5}$}
	{$m=\dfrac{1}{5}$}
	{$m=-\dfrac{1}{3}$}
	{\True $m=-\dfrac{1}{5}$}
	\loigiai{
		Ta có $\vec{AB}=(3;-5);\vec{AC}=(m+1;-2)$.\\
		$A,B,C$ thẳng hàng $\Leftrightarrow$ $\vec{AB}$ cùng phương với $\vec{AC} \Leftrightarrow \dfrac{3}{m+1}=\dfrac{-5}{-2} \Leftrightarrow m=-\dfrac{1}{5}$.
	}
\end{ex} \dongcham{5}

\begin{ex}
	Cho ba điểm $A\left(-1;1;2\right),\ B(0;1;-1),\ C(x+2;y;-2)$ thẳng hàng. Tổng $x+y$ bằng
	\choice
	{$\dfrac{7}{3}$}
	{$-\dfrac{8}{3}$}
	{\True $-\dfrac{2}{3}$}
	{$-\dfrac{1}{3}$}
	\loigiai{
		\begin{itemize}
			\item [$\bullet$] Ta có $\vec{AB}=(1;0-3),\ \vec{AC}=(x+3;y-1;-4)$.
			\item [$\bullet$] Các điểm $A,\ B,\ C$ thẳng hàng $\Leftrightarrow$ có số thực $t$ thỏa mãn $\vec{AC}=t\vec{AB}$.\tagEX{1}
			      Ta có $(1)\Leftrightarrow\heva{&x+3=t\\&y-1=0\\&-4=-3t}\Leftrightarrow\heva{&x=-\dfrac{5}{3}\\&y=1\\&t=\dfrac{4}{3}}\Rightarrow x+y=-\dfrac{2}{3}$.
			\item [$\bullet$] Vậy tổng $x+y=-\dfrac{2}{3}$.
		\end{itemize}
	}
\end{ex} \dongcham{5}

\begin{ex}
	Tứ giác $ABCD$ là hình bình hành, biết $A(1; 0; 1)$, $B(2; 1; 2)$, $D(1; -1; 1)$. Tìm tọa độ điểm $C$.
	\choice
	{$(0; -2; 0)$}
	{$(2; 2; 2)$}
	{\True $(2; 0; 2)$}
	{$(2; -2; 2)$}
	\loigiai{
		\begin{itemize}
			\item [$\bullet$] Tứ giác $ABCD$ là hình bình hành khi
			      $$\vec{AB}=\vec{DC} \Leftrightarrow \heva{&x_C -1=2-1\\&y_C+1=1-0\\&z_C-1=2-1}\Leftrightarrow \heva{&x_C =2\\&y_C=0\\&z_C=2}.$$
			\item [$\bullet$] Tọa độ điểm $C(2; 0; 2)$.
		\end{itemize}
	}
\end{ex} \dongcham{5}

\begin{ex}
	\immini[thm]{Cho hình hộp $ABCD.A'B'C'D'$ có $A(0;0;0)$, $B(a;0;0)$, $D(0;2a;0)$, $A'(0;0;2a),a\ne 0$. Tính độ dài đoạn thẳng $AC'$.
		\haicot
		{$|a|$}
		{$2|a|$}
		{\True $3|a|$}
		{$\dfrac{3|a|}{2}$}}{
		\begin{tikzpicture}[scale=0.65, font=\footnotesize, line join=round, line cap=round, >=stealth]
			\def\bc{4} % cạnh BC
			\def\ba{2} % cạnh BA
			\def\gocB{35} % góc B của đáy
			\coordinate[label=below left:$B$] (B) at (0,0);
			\coordinate[label=above left:$A$] (A) at (\gocB:\ba);
			\coordinate[label=below:$C$] (C) at (\bc,0);
			\coordinate[label=right:$D$] (D) at ($(C)-(B)+(A)$);
			\coordinate[label=above left:$A'$] (A') at ($(A)+(99:\bc)$);
			\coordinate[label=left:$B'$] (B') at ($(B)-(A)+(A')$);
			\coordinate[label=below right:$C'$] (C') at ($(C)-(A)+(A')$);
			\coordinate[label=right:$D'$] (D') at ($(D)-(A)+(A')$);
			\draw (B')--(B)--(C)--(D)--(D')--(A')--(B')--(C')--(D') (C)--(C');
			\draw[dashed] (A')--(A)--(D) (A)--(B);
			\foreach \diem in {A,B,C,D,A',B',C',D'}	\fill (\diem)circle(1.5pt);
		\end{tikzpicture}}
	\loigiai{
		\immini{
			Ta có: $\vec{AB}=(a;0;0); \vec{AD}=(0;2a;0); \vec{AA'}=(0;0;2a)$.
			$$\vec{AC'}=\vec{AB}+\vec{AD}+\vec{AA'}\Rightarrow \vec{AC'}=(a;2a;2a).$$
			Suy ra $AC'=\sqrt{a^2+4a^2+4a^2}=3|a|$.
		}
		{
			\begin{tikzpicture}[scale=0.65, font=\footnotesize, line join=round, line cap=round, >=stealth]
				\def\bc{4} % cạnh BC
				\def\ba{2} % cạnh BA
				\def\gocB{35} % góc B của đáy
				\coordinate[label=below left:$B$] (B) at (0,0);
				\coordinate[label=above left:$A$] (A) at (\gocB:\ba);
				\coordinate[label=below:$C$] (C) at (\bc,0);
				\coordinate[label=right:$D$] (D) at ($(C)-(B)+(A)$);
				\coordinate[label=above left:$A'$] (A') at ($(A)+(99:\bc)$);
				\coordinate[label=left:$B'$] (B') at ($(B)-(A)+(A')$);
				\coordinate[label=below right:$C'$] (C') at ($(C)-(A)+(A')$);
				\coordinate[label=right:$D'$] (D') at ($(D)-(A)+(A')$);
				\draw (B')--(B)--(C)--(D)--(D')--(A')--(B')--(C')--(D') (C)--(C');
				\draw[dashed] (A')--(A)--(D) (A)--(B);
				\foreach \diem in {A,B,C,D,A',B',C',D'}	\fill (\diem)circle(1.5pt);
			\end{tikzpicture}

		}
	}
\end{ex} \dongcham{5}

\begin{ex}
	\immini[thm]{Cho hình hộp $ABCD.A'B'C'D'$ có $A(0;0;1)$, $B'(1;0;0)$, $C'(1;1;0)$. Tìm tọa độ của điểm $D$.
		\haicot
		{$D(0;-1;1)$}
		{\True $D(0;1;1)$}
		{$D(0;1;0)$}
		{$D(1;1;1)$}}{
		\begin{tikzpicture}[scale=0.65, font=\footnotesize, line join=round, line cap=round, >=stealth]
			\def\bc{4} % cạnh BC
			\def\ba{2} % cạnh BA
			\def\gocB{35} % góc B của đáy
			\coordinate[label=below left:$B$] (B) at (0,0);
			\coordinate[label=above left:$A$] (A) at (\gocB:\ba);
			\coordinate[label=below:$C$] (C) at (\bc,0);
			\coordinate[label=right:$D$] (D) at ($(C)-(B)+(A)$);
			\coordinate[label=above left:$A'$] (A') at ($(A)+(99:\bc)$);
			\coordinate[label=left:$B'$] (B') at ($(B)-(A)+(A')$);
			\coordinate[label=below right:$C'$] (C') at ($(C)-(A)+(A')$);
			\coordinate[label=right:$D'$] (D') at ($(D)-(A)+(A')$);
			\draw (B')--(B)--(C)--(D)--(D')--(A')--(B')--(C')--(D') (C)--(C');
			\draw[dashed] (A')--(A)--(D) (A)--(B);
			\foreach \diem in {A,B,C,D,A',B',C',D'}	\fill (\diem)circle(1.5pt);
		\end{tikzpicture}}
	\loigiai
	{
		\immini
		{
			Gọi $D(x_D;y_D;z_D)$.\\
			Ta có $\vec{B'C'}=(0;1;0)$, $\vec{AD}=\left(x_D;y_D;z_D-1\right)$. Vì $B'C'DA$ là hình bình hành nên
			\begin{align*}
				\vec{B'C'}=\vec{AD}\Leftrightarrow \heva{ & x_D=0 \\ & y_D=1\\& z_D-1=0}\Leftrightarrow \heva{& x_D=0 \\ & y_D=1\\& z_D=1.}
			\end{align*}
			Vậy $D(0;1;1)$.
		}
		{
			\begin{tikzpicture}[scale=0.65, font=\footnotesize, line join=round, line cap=round, >=stealth]
				\def\bc{4} % cạnh BC
				\def\ba{2} % cạnh BA
				\def\gocB{35} % góc B của đáy
				\coordinate[label=below left:$B$] (B) at (0,0);
				\coordinate[label=above left:$A$] (A) at (\gocB:\ba);
				\coordinate[label=below:$C$] (C) at (\bc,0);
				\coordinate[label=right:$D$] (D) at ($(C)-(B)+(A)$);
				\coordinate[label=above left:$A'$] (A') at ($(A)+(99:\bc)$);
				\coordinate[label=left:$B'$] (B') at ($(B)-(A)+(A')$);
				\coordinate[label=below right:$C'$] (C') at ($(C)-(A)+(A')$);
				\coordinate[label=right:$D'$] (D') at ($(D)-(A)+(A')$);
				\draw (B')--(B)--(C)--(D)--(D')--(A')--(B')--(C')--(D') (C)--(C');
				\draw[dashed] (A')--(A)--(D) (A)--(B);
				\foreach \diem in {A,B,C,D,A',B',C',D'}	\fill (\diem)circle(1.5pt);
			\end{tikzpicture}
		}
	}
\end{ex} \dongcham{5}


\Closesolutionfile{ans}
\BTTF
\Opensolutionfile{ans}[ans/2H2-B3-d1-2]
\begin{ex}
	Cho các điểm $A(1 ;-2 ; 3), B(-2 ; 1 ; 2), C(3 ;-1 ; 2)$.
	\choiceTF
	{\True $\vec{A B}=(-3 ; 3 ;-1)$}
	{$\vec{A C}=(-2 ;-1 ; 1)$}
	{$\vec{A B}=3 \vec{A C}$}
	{\True  Ba điểm $A, B, C$ không thẳng hàng}
	\loigiai{
		\begin{enumerate}[a)]
			\item $\vec{A B}=\big(x_B-x_A;y_B-y_A;z_B-z_A\big)=(-3 ; 3 ;-1)$.
			\item $\vec{A C}=\big(x_C-x_A;y_C-y_A;z_C-z_A\big)=(2 ; 1 ;-1)$
			\item $\vec{A B}=(-3 ; 3 ;-1)$, $\vec{A C}=(2 ; 1 ;-1)$. Hai vec tơ này không cùng phương nên không tồn tại số thực $k$ để $\vec{A B}=k \vec{A C}$.
			\item Hai vec tơ $\vec{A B}$ và $\vec{A C}$ không cùng phương nên ba điểm $A, B, C$ không thẳng hàng.
		\end{enumerate}
	}
\end{ex} \dongcham{5}

\begin{ex}
	\immini[thm]{Cho ba điểm $ A(3;3;-6) $, $ B(1;3;2) $ và $ C(-1;-3;1)$. Gọi $M$, $N$, $K$ lần lượt là trung điểm của $AB$, $BC$ và $CA$.
		\choiceTF
		{Tọa độ $M\left(2;3;2 \right)$}
		{Với $G$ là trọng tâm tam giác $ABC$ thì $GC=2\sqrt{5}$}
		{\True Trọng tâm tam giác $MNK$ là $E(1;1;-1)$}
		{\True Với $D(-3;-3;9)$ thì tứ giác $ABDC$ là hình bình hành}}{
		\begin{tikzpicture}[scale=1, font=\footnotesize,>=stealth]
			\path
			%	Vẽ mp
			(0,0) coordinate (B)
			(5,0) coordinate (C)
			(2,3) coordinate (A)
			($(A)!0.5!(B)$)coordinate (M)
			($(B)!0.5!(C)$)coordinate (N)
			($(A)!0.5!(C)$)coordinate (K)
			($(A)!2/3!(N)$)coordinate (G)
			;
			\draw (B)--(A)--(C)--(B) (M)--(N)--(K)--(M);
			\foreach \x/\g in {A/90,B/180,C/0,M/160,N/-90,K/10}\draw[fill=black] (\x) circle (.05) +(\g:.5)node{\footnotesize$\x$};
		\end{tikzpicture}}
	\loigiai{
		\begin{enumerate}[a)]
			\item $M$ là trung điểm của $AB$, suy ra $M\left(\dfrac{x_A+x_B}{2}; \dfrac{y_A+y_B}{2};\dfrac{z_A+z_B}{2}\right)$ hay $M(2;3;-2)$.
			\item Ta có $G(1;1;-1)$. Suy ra $GC=\sqrt{(-1-1)^2+(-3-1)^2+(1+1)^2}=2\sqrt{6}$.
			\item Hai tam giác $ABC$ và $MNK$ có cùng trọng tâm. Suy ra $E$ trùng với $G(1;1;-1)$.
			\item Ta có $\vec{AC}=(-4;-6;7)$, $\vec{BD}=(-4;-6;7)$, suy ra $\vec{AC}=\vec{BD}$. Vậy $ABDC$ là hình bình hành.
		\end{enumerate}
	}
\end{ex} \dongcham{5}

\begin{ex}
	\immini[thm]{Cho hình hộp $ ABCD.A'B'C'D' $, biết điểm $ A(0; 0; 0)$, $ B(1; 0; 0)$, $ C(1; 2; 0)$, $ D'(-1; 3; 5)$. Gọi $M$, $N$ là tâm của các hình bình hành $ABB'A'$, $ADD'A'$.
		\choiceTF
		{\True Tọa độ $D(0; 2; 0)$}
		{\True Tọa độ $A'(-1; 1; 5)$}
		{Tọa độ $\vec{MN}=(-1;1;0)$}
		{$\big|\vec{AB}+\vec{AD}+\vec{CC'}\big|=\sqrt{29}$}}{
		\begin{tikzpicture}[scale=0.7, font=\footnotesize, line join=round, line cap=round,>=stealth]
			\tkzDefPoints{0/0/A,-2/-2/B, 4/0/D,-0.5/3.5/A'}
			\coordinate (C) at ($(B)+(D)$);
			\coordinate (B') at ($(B)+(A')$);
			\coordinate (C') at ($(C)+(A')$);
			\coordinate (D') at ($(D)+(A')$);
			\coordinate (x) at ($(A)!1.5!(B)$);
			\coordinate (y) at ($(A)!1.5!(D)$);
			\coordinate (z) at ($(A)!1.5!(A')$);
			\tkzDrawPoints[fill=black](A,B,C,D,A',B',C',D')
			\draw[dashed] (A)--(B) (A)--(D) (A)--(A');
			\draw (A')--(B')--(C')--(D')--(A') (B)--(B') (C)--(C') (D)--(D') (B)--(C)--(D);
			\tkzLabelPoints[left](A,B,B',A')
			\tkzLabelPoints[below right=-0.1](C,D)
			\tkzLabelPoints[right](C',D')
		\end{tikzpicture}}
	\loigiai{
		\immini{
			\begin{enumerate}[a)]
				\item Theo qui tắc hình bình hành, ta có
				      \[\vec{AD}=\vec{AC}-\vec{AB}=(0; 2; 0)\Rightarrow D(0; 2; 0).\]
				\item Ta có
				      \[\vec{AA'}=\vec{DD'}=(-1; 1; 5)\Rightarrow A'(-1; 1; 5).\]
				\item Theo hình vẽ $\vec{MN}=\vec{BC}=(0;2;0)$.
				\item Ta có $\vec{AC'}=\vec{AB}+\vec{AD}+\vec{AA'}=(0;3;5)$.\\ Xét
				      \begin{eqnarray*}
					      \big|\vec{AB}+\vec{AD}+\vec{CC'}\big|
					      &=&\big|\vec{AB}+\vec{AD}+\vec{AA'}\big|=\big|\vec{AC'}\big|\\
					      &=& \sqrt{0^2+3^2+5^2}=\sqrt{34}.
				      \end{eqnarray*}
			\end{enumerate}
		}{
			\begin{tikzpicture}[scale=0.7, font=\footnotesize, line join=round, line cap=round,>=stealth]
				\tkzDefPoints{0/0/A,-2/-2/B, 4/0/D,-0.5/3.5/A'}
				\coordinate (C) at ($(B)+(D)$);
				\coordinate (B') at ($(B)+(A')$);
				\coordinate (C') at ($(C)+(A')$);
				\coordinate (D') at ($(D)+(A')$);
				\coordinate (x) at ($(A)!1.5!(B)$);
				\coordinate (y) at ($(A)!1.5!(D)$);
				\coordinate (z) at ($(A)!1.5!(A')$);
				\coordinate (M) at ($(A')!0.5!(B)$);
				\coordinate (N) at ($(C)!0.5!(D')$);
				\tkzDrawPoints[fill=black](A,B,C,D,A',B',C',D')
				\draw[dashed] (A)--(B) (A)--(D) (A)--(A') (M)--(N);
				\draw (A')--(B')--(C')--(D')--(A') (B)--(B') (C)--(C') (D)--(D') (B)--(C)--(D) (A')--(B) (C)--(D');
				\draw[->] (B)--(x) node[above] {$x$};
				\draw[->] (D)--(y) node[above] {$y$};
				%\draw[->] (A')--(z) node[left] {$z$};
				\tkzLabelPoints[left](A,B,B',A',M)
				\tkzLabelPoints[below right=-0.1](C,D)
				\tkzLabelPoints[right](C',D',N)
			\end{tikzpicture}
		}
	}
\end{ex} \dongcham{5}

\begin{ex}
	\immini[thm]{Hai chiếc khinh khí cầu bay lên từ cùng một địa điểm. Chiếc thứ nhất cách điểm xuất phát $2$ km về phía nam và $1$ km về phía đông, đồng thời cách mặt đất $0{,}5$ km. Chiếc thứ hai nằm cách điểm xuất phát $1$ km về phía bắc và $1{,}5$ km về phía tây, đồng thời cách mặt đất $0{,}8$ km.\\
		Chọn hệ trục $Oxyz$ với gốc $O$ đặt tại điểm xuất phát của hai khinh khí cầu, mặt phẳng $(Oxy)$ trùng với mặt đất với trục $Ox$ hướng về phía nam, trục $Oy$ hướng về phía đông và trục $Oz$ hướng thẳng đứng lên trời (Hình bên dưới), đơn vị đo lấy theo kilomet.
	}{
		\begin{tikzpicture}[smooth,samples=300,scale=0.8,>=stealth,font=\footnotesize]
			\draw[->] (-4,0)node[above right]{Bắc}--(6,0) node[below]{$x$} node[above left]{Nam};
			\draw[->] (3,3)node[above right]{Tây}--(-2.3,-2.3) node[left]{$y$}node[below right]{Đông};
			\draw[->] (0,0)--(0,4.5) node[right]{$z$};
			\draw (0,0) node[below]{$O$};
			\fill [scale=.1,black,yshift=12 cm,xshift=25 cm]
			(-1,0) rectangle (1,1)
			(-1,2).. controls +(135:1) and +(180:4) .. (0,7)
			.. controls +(180:2) and +(135:2).. (-.5,2)
			.. controls +(120:2) and +(180:1) .. (0,6.99)
			.. controls +(0:1) and +(60:2) .. (.5,2)
			.. controls +(45:2) and +(0:2) .. (0,7)
			.. controls +(0:4) and +(45:1).. (1,2)
			;
			\draw[scale=.1,yshift=12 cm,xshift=25 cm] (-1,0) rectangle (1,2) (0,1)--(0,2);
			;
			\fill [scale=.1,blue,yshift=37 cm,xshift=-20 cm]
			(-1,0) rectangle (1,1)
			(-1,2).. controls +(135:1) and +(180:4) .. (0,7)
			.. controls +(180:2) and +(135:2).. (-.5,2)
			.. controls +(120:2) and +(180:1) .. (0,6.99)
			.. controls +(0:1) and +(60:2) .. (.5,2)
			.. controls +(45:2) and +(0:2) .. (0,7)
			.. controls +(0:4) and +(45:1).. (1,2)
			;
			\draw[blue,scale=.1,yshift=37 cm,xshift=-20 cm] (-1,0) rectangle (1,2) (0,1)--(0,2);
			;
			\draw[dashed] (4,0)--(2.5,-1.5)--(-1.5,-1.5)
			(2.5,1)--(0,0)--(2.5,-1.5)--(2.5,1);
			\draw[dashed] (-3,0)--(-2,1)--(1,1) (-2,1)--(0,0)--(-2,3.5)--(-2,1)
			;
			\draw[fill=black] (2.5,1) circle(2pt) (-2,3.5) circle(2pt);
		\end{tikzpicture}}
	\choiceTF
	{\True Với hệ tọa độ đã chọn, toạ độ khinh khí cầu thứ nhất là $(2;1;0{,}5)$}
	{Với hệ tọa độ đã chọn, toạ độ khinh khí cầu thứ hai  là $(-1{,}5;-1;0{,}8)$}
	{Khoảng cách từ điểm xuất phát đến khinh khí cầu thứ nhất bằng $\sqrt{21}$ km}
	{\True Khoảng cách hai chiếc khinh khí cầu là $3{,}92\text{ km}$ (\textit{Kết quả làm tròn đến hàng phần trăm})}
	\loigiai{
		\begin{enumerate}
			\item Chiếc khinh khí cầu thứ nhất có tọa độ là $(2;1;0{,}5)$.
			\item Chiếc khinh khí cầu thứ hai có tọa độ là $(-1;-1{,}5;0{,}8)$.
			\item Khoảng cách từ điểm xuất phát đến khinh khí cầu thứ nhất bằng $\sqrt{2^2+1^2+0,5^2}=\dfrac{\sqrt{21}}{2}$ (km)
			\item Khoảng cách hai chiếc khinh khí cầu là
			      $\sqrt{(-1-2)^2+(1{,}5-1)^2+(0{,}8-0{,}5)^2}=\sqrt{15{,}34}\approx3{,}92\text{ (km)}.$
		\end{enumerate}
	}
\end{ex}

\Closesolutionfile{ans}

\begin{dang}{Tích vô hướng, tích có hướng hai vec tơ và ứng dụng}
\end{dang}
\BTTL
\begin{vd}
	Cho ba véc-tơ $\vec{a} = (3; 0; 1)$, $\vec{b} = (1; -1; -2)$, $\vec{c} = (2; 1; -1)$, $\vec{d} = (1; 7; -3)$.
	\begin{tasks}(3)
		\task Tính $\vec{a}\cdot \vec{b}$, $\vec{b}\cdot \vec{c}$.
		\task Tính $\left| \vec{a} \right|$, $\left| \vec{b} \right|$, $\cos \left(\vec{a}, \vec{b}\right)$.
		\task Chứng minh $\vec{d} \perp \vec{a}$.
	\end{tasks}
	\loigiai{
		\begin{enumerate}
			\item Ta có $\vec{a} \cdot \vec{b} = 3\cdot 1 + 0\cdot (-1) + 1\cdot (-2) = 1$ và $\vec{b} \cdot \vec{c} = 1\cdot 2 + (-1)\cdot 1 + (-2)\cdot (-1) = 3$.
			\item Ta có $\left| \vec{a} \right| = \sqrt{3^2 + 0^2 + 1^2} = \sqrt{10}$, $\left| \vec{b} \right| \sqrt{1^2 + (-1)^2 + (-2)^2} = \sqrt{6}$.\\
			      $\cos \left(\vec{a}, \vec{b}\right) = \dfrac{\vec{a}\cdot \vec{b}}{\left|\vec{a}\right| \cdot \left|\vec{b}\right|} = \dfrac{1}{\sqrt{10}\cdot \sqrt{6}} = \dfrac{\sqrt{15}}{60}$.
			\item Ta có $\vec{d}\cdot \vec{a} = 1\cdot 3 + 7 \cdot 0 + (-3)\cdot 1 = 0 \Rightarrow \vec{d} \perp \vec{a}$.
		\end{enumerate}
	}
\end{vd}
\dongcham{16}
\begin{vd}
	Trong không gian $Oxyz$, cho $\vec{a}=(1;0;1)$, $\vec{b}=(1;1;0)$ và $\vec{c}=(-4;3;m)$.
	\begin{listEX}
		\item Tính góc giữa hai vectơ $\vec{a}$ và $\vec{b}$.
		\item Tìm $m$ để vectơ $\vec{d}=2\vec{a}+3\vec{b}$ vuông góc với $\vec{c}$.
	\end{listEX}
	\loigiai{
		\begin{listEX}
			\item Ta có $\heva{&\vec{a}=(1;0;1)\\ &\vec{b}=(1;1;0)}\Rightarrow \cos(\vec{a};\vec{b})=\dfrac{\vec{a}\cdot\vec{b}}{|\vec{a}|\cdot |\vec{b}|}=\dfrac{1}{2}$.
			\item Ta có $\vec{d}=2\vec{a}+3\vec{b}=(5;3;2)$.\\
			Ta có $\vec{d}\perp \vec{c}\Leftrightarrow \vec{d}\cdot\vec{c}=0\Leftrightarrow -20+9+2m=0\Leftrightarrow m=\dfrac{11}{2}$.
		\end{listEX}
	}
\end{vd}
\dongcham{16}
\begin{vd}
	Trong không gian $Oxyz$, cho tam giác $ABC$ có $A(-1; 0; 2)$, $B(0; 4; 3)$ và $C(-2; 1; 2)$.
	\begin{tasks}
		\task Chỉ ra tọa độ một véc tơ (khác $\vec{0}$) vuông góc với hai véc tơ $\vec{AB}$, $\vec{AC}$.
		\task Tính chu vi tam giác $ABC$.
		\task Tính $\cos \widehat{BAC}$.
		\task Tìm độ dài đường phân giác trong $AD$ của tam giác $ABC$.
	\end{tasks}
	\loigiai{
		\begin{enumerate}[a)]
			\item
			\item Ta có $AB = \sqrt{1 + 16 + 1} = 3\sqrt{2}$ và $AC = \sqrt{1 + 1 + 0}$.
			\item
			\item Theo tính chất đường phân giác trong của tam giác, ta có $\dfrac{DB}{DC} = \dfrac{AB}{AC} = 3$.\\
			      Suy ra $\overrightarrow{DB} = -3\overrightarrow{DC} \Leftrightarrow \heva{&x_D = \dfrac{x_B + 3x_C}{4} = -\dfrac{3}{2} \\&y_D = \dfrac{y_B + 3y_C}{4} = \dfrac{7}{4} \\&z_D = \dfrac{z_B + 3z_C}{4} = \dfrac{9}{4}\cdot}$\\
			      $\Rightarrow D\left(-\dfrac{3}{2}; \dfrac{7}{4}; \dfrac{9}{4}\right)$.\\
			      Vậy $AD = \sqrt{\dfrac{1}{4} + \dfrac{49}{16} + \dfrac{1}{16}} = \dfrac{3\sqrt{6}}{4}\cdot$
		\end{enumerate}
	}
\end{vd}
\dongcham{20}
\begin{vd}
	Trong không gian $Oxyz$, cho 3 điểm $A\left(0;1;-2\right);B\left(3;0;0\right)$ và điểm $C$ thuộc trục $Oz$. Biết $ABC$ là tam giác cân tại $C$. Tìm toạ độ điểm $C$.
	\loigiai{
		Gọi $C\left(0;0;z\right)$ là điểm thuộc trục $Oz$.\\
		Tam giác $ABC$ cân tại $C$ nên $CA=CB$. \\
		Suy ra $CA^2=CB^2 \Rightarrow 1+(z+2)^2=9+z^2\Rightarrow z=1\Rightarrow C(0;0;1)$.}

\end{vd}
\dongcham{14}
\begin{vd}
	Trong không gian $Oxyz$, cho ba điểm $M\left(2;3;-1\right)$, $N\left(-1;1;1\right)$, $P\left(1;m-1;2\right)$. Với những giá trị nào của $m$ thì tam giác $MNP$ vuông tại $N$?
	\loigiai{Ta có $\vec{NM}=\left(3;2;-2\right)$ và $\vec{NP}=\left(2;m-2;1\right)$.\\
		Vì tam giác $MNP$ vuông tại $N$ nên ta có $\vec{NM} \perp \vec{NP} \Leftrightarrow \vec{NM}.\vec{NP} =0 \Leftrightarrow 2m=0 \Leftrightarrow m=0$.\\
		Vậy $m=0$ thỏa yêu cầu bài toán.}
\end{vd}
\dongcham{10}
\begin{vd}
	Cho hai điểm $A\left({2,-1,1}\right);B\left({3,-2,-1}\right)$. Tìm điểm $N$ trên trục  $Ox$ cách đều $A$ và $B$.
	\loigiai{$N$ nằm trên trục $Ox$ nên $N\left(x;0;0\right).$\\
		Khi đó, ta có $\overrightarrow {AN}  = \left({x-2;1;-1}\right)$;\quad $\overrightarrow {BN}  = \left({x-3;2;1}\right)$.\\
		Vì $N$ cách đều $A$ và $B$ nên $AN=BN\Leftrightarrow \sqrt {{(x-2)^2}+1+1}  = \sqrt {{(x-3)^2}+4+1} \Leftrightarrow x = 4.$\\
		Suy ra $N(4;0;0)$.}

\end{vd}
\dongcham{10}
\begin{vd}
	\immini{Trong Hóa học, cấu tạo của phân tử ammoniac ($\mathrm{NH}_3$) có dạng hình chóp tam giác đều mà đỉnh là nguyên tử nitrogen ($\mathrm{N}$) và đáy là tam giác $H_1H_2H_3$ với $H_1$, $H_2$, $H_3$ là vị trí của ba nguyên tử hydrogen ($\mathrm{H}$). Góc tạo bởi liên kết $\mathrm{H}-\mathrm{N}-\mathrm{H}$, có hai cạnh là hai đoạn thẳng nối $N$ với hai trong ba điểm $H_1$, $H_2$, $H_3$ (chẳng hạn $\widehat{H_1NH_2}$), gọi là góc liên kết của phân tử $\mathrm{NH}_3$. Góc này xấp xỉ $107^{\circ}$.\\
		Trong không gian $Oxyz$, cho một phân tử $\mathrm{NH}_3$ được biểu diễn bởi hình chóp tam giác đều $N.H_1H_2H_3$ với $O$ là tâm của đáy. Nguyên tử nitrogen được biểu diễn bởi điểm $N$ thuộc trục $Oz$, ba nguyên tử hydrogen ở các vị trí $H_1$, $H_2$, $H_3$ trong đó $H_1(0;-2;0)$ và $H_2H_3$ song song với trục $Ox$ (Hình bên).}{
		\begin{tikzpicture}[>=stealth,line join=round,line cap=round,scale=3]
			\draw (0,0)coordinate(H1)--(-35:0.8)coordinate(H2)--(1,0)coordinate(H3);
			\draw[dashed,black] ($(H2)!.5!(H3)$)coordinate(M1)--($(H1)!2/3!(M1)$)coordinate(H)--(H1)--(H3) (H)--($(H)+(90:1)$)coordinate(N);
			\draw[black,scale=2.5] (H1)node[left]{$H_1$}--(N)node[right]{$N$}--(H3)node[right]{$H_3$} (N)--(H2)node[below]{$H_2$};
			\draw[->] (H)node[below]{$O$}--($(H)+(90:1.2)$)node[right]{$z$};
			\draw[->] (H)--($4*(M1)-4*(H)$)node[above]{$y$};
			\draw[->] (H)--($(H2)-(H3)+(H)$)node[below]{$x$};
			\foreach \diem in {N,H1,H2,H3}\fill[blue] (\diem)circle(0.7pt);
		\end{tikzpicture}}
	\begin{listEX}
		\item Tính khoảng cách giữa hai nguyên tử hydrogen.
		\item Tính khoảng cách giữa hai nguyên tử nitrogen với mỗi nguyên tử hydrogen.
	\end{listEX}
	\loigiai{
		\begin{listEX}
			\item Gọi $x=H_1H_2$, khi đó độ dài $OH_1=x\dfrac{\sqrt{3}}{3}\Leftrightarrow 2=x\dfrac{\sqrt{3}}{3}\Leftrightarrow x=2\sqrt{3}$.
			\item Gọi $y$ là khoảng cách giữa hai nguyên tử nitrogen với mỗi nguyên tử hydrogen; khi đó $NH_2=y$.\\
			Áp dụng định lí cosin ta có $$ H_1H_2^2=NH_1^2+NH_2^2-2\cdot NH_1\cdot NH_2\cos \widehat{H_1NH_2}\Leftrightarrow 2y^2-2y^2\cos 107^\circ=12$$ $$\Leftrightarrow y^2=\dfrac{12}{2-2\cos 107}\Leftrightarrow y=2{,}155 $$
		\end{listEX}
	}
\end{vd}
\dongcham{18}
\begin{vd}%[2H2V2-6]
	\immini{
		Một chậu cây được đặt trên một giá đỡ có bốn chân với điểm đặt $S(0; 0; 20)$ và các điểm chạm mặt đất của bốn chân lần lượt là $A(20; 0; 0)$, $B(0; 20; 0)$, $C(-20; 0; 0)$, $D(0; -20; 0)$ (đơn vị cm). Cho biết trọng lực tác dụng lên chậu cây có độ lớn $40$(N) và được phân bố thành bốn lực $\overrightarrow{F_1}$, $\overrightarrow{F_2}$, $\overrightarrow{F_3}$, $\overrightarrow{F_4}$ có độ lớn bằng nhau như Hình 4.  Tìm toạ độ của các lực
		nói trên (mỗi centimét biểu diễn 1 N).
	}{
		\includegraphics[scale=0.7]{images/luc-1.png}
	}
	\loigiai{
		Tứ giác $ABCD$ có hai đường chéo bằng nhau và vuông góc với nhau tại trung điểm của mỗi đường nên là hình vuông.
		\begin{center}
			\begin{tikzpicture}[scale=0.7, font=\footnotesize,>=stealth]
				\path
				%	Vẽ mp
				(0,0) coordinate (O)
				(-2,-1) coordinate (A)
				(5,-1) coordinate (B)
				(2,1) coordinate (C)
				(-5,1) coordinate (D)
				($(O)+(0,6)$)coordinate (S)
				($(A)!0.5!(S)$)coordinate (A')
				($(B)!0.5!(S)$)coordinate (B')
				($(C)!0.5!(S)$)coordinate (C')
				($(D)!0.5!(S)$)coordinate (D')
				($(B')!0.5!(D')$)coordinate (O')
				;
				\draw (A)--(D)--(S)--(A)--(B)--(S) (D')--(A')--(B');
				\draw[dashed] (A)--(C)--(B)--(D)--(C)--(S)--(O) (C')--(B')--(D')--(C')--(A');
				\foreach \x/\g in {A/-90,B/0,C/0,D/180,S/90,O/30,O'/-30,A'/200,B'/0,C'/20,D'/180}\draw[fill=black] (\x) circle (.05) +(\g:.5)node{\footnotesize$\x$};
			\end{tikzpicture}
		\end{center}
		Ta có $\overrightarrow{SA} = (20; 0; -20)$, $\overrightarrow{SB} = (0; 20; -20)$, $\overrightarrow{SC} = (-20; 0; -20)$ , $\overrightarrow{SD} = (0; -20; -20)$.\\
		Suy ra $SA = SB = SC = SD = 20\sqrt{2}$. Do đó $S.ABCD$ là hình chóp tứ giác đều.
		Các vectơ $\overrightarrow{F_1}$, $\overrightarrow{F_2}$, $\overrightarrow{F_3}$, $\overrightarrow{F_4}$ có điểm đầu tại $S$ và điểm cuối lần lượt là $A'$, $B'$, $C'$, $D'$.\\
		Ta có $SA' = SB' = SC' = SD'$ nên $S.A'B'C'D'$ cũng là hình chóp tứ giác đều.\\
		Gọi $\overrightarrow{F} $ là trọng lực tác dụng lên chậu cây và $O'$ là tâm của hình vuông $A'B'C'D'$. Ta có
		\[ \overrightarrow{F} =\overrightarrow{F_1} + \overrightarrow{F_2}+ \overrightarrow{F_3}+ \overrightarrow{F_4} = \overrightarrow{SA'} + \overrightarrow{SB'}+ \overrightarrow{SC'}+ \overrightarrow{SD'} = 4\overrightarrow{SO'}.\]
		Ta có $\left| \overrightarrow{F} \right| =40$, suy ra $\left| \overrightarrow{SO'} \right| = SO' = 10$.
		Do tam giác $SO'A'$ vuông cân nên $SA' = \sqrt{2}SO' = 10\sqrt{2}$.\\
		Suy ra $\overrightarrow{F_1} = \overrightarrow{SA'} = \dfrac{1}{2}\overrightarrow{SA}  = (10; 0; -10)$.
		Chứng minh tương tự ta cũng có\\
		\[ \overrightarrow{F_2} = \dfrac{1}{2}\overrightarrow{SB}  = (0; 10; -10), \overrightarrow{F_3} = \dfrac{1}{2}\overrightarrow{SC}  = (-10; 0; -10), \overrightarrow{F_4} = \dfrac{1}{2}\overrightarrow{SD}  = (0; -10; -10). \]
	}
\end{vd}
\dongcham{24}
\BTTN
\Opensolutionfile{ans}[ans/2H2-B3-d2-1]

\begin{ex}
	Tích vô hướng của hai vectơ $\vec{u}=(3;0;1)$ và $\vec{v}=(2;1;0)$ là
	\choice
	{$0$}
	{\True$6$}
	{$8$}
	{$-6$}
	\loigiai
	{
		$$\vec{u}\cdot\vec{v}=6+0+0=6.$$
	}
\end{ex} \dongcham{4}

\begin{ex}
	Tích vô hướng của hai vectơ $\vec{u} = \vec{i} + 2 \vec{j} - \vec{k}$ và $ \vec{v} = (0;1; -2)$ bằng
	\choice
	{$ -4 $}
	{$ 0 $}
	{\True $ 4 $}
	{$ -2 $}
	\loigiai{
		Ta có $ \vec{u}=(1;2;-1) $.\\
		Suy ra $ \vec{u}\cdot \vec{v}=1\cdot0+2\cdot1+(-1)\cdot(-2)=4 $.
	}
\end{ex} \dongcham{4}


\begin{ex}
	Cho các véc-tơ $\vec{a}=(1;2;1)$ và $\vec{b}=(2;2;1)$. Tính tích vô hướng $\vec{a} \cdot \left(\vec{a}-\vec{b}\right)$.
	\choice
	{\True $-1$}
	{$-2$}
	{$2$}
	{$1$}
	\loigiai{
		Ta có: $\left(\vec{a}-\vec{b}\right)=(-1;0;0) \Rightarrow \vec{a} \cdot \left(\vec{a}-\vec{b}\right)=1 \cdot (-1)+2 \cdot 0+1 \cdot 0=-1$.}
\end{ex} \dongcham{5}

\begin{ex}
	Một thiết bị thăm dò đáy biển được đẩy bởi một lực $\overrightarrow{f} = (5; 4; -2)$ (đơn vị: N) giúp thiết bị thực hiện độ dời $\overrightarrow{a} = (70; 20; -40)$ (đơn vị: m). Tính công sinh bởi lực $\overrightarrow{f}$.
	\choice
	{$480\,(\text{J})$}
	{$530\,(\text{J})$}
	{\True $510\,(\text{J})$}
	{$500\,(\text{J})$}
	\loigiai{
		Công sinh bởi lực $\overrightarrow{f}$ là
		\[ A = \left| \overrightarrow{f} \right| \cdot \left| \overrightarrow{a} \right|\cdot \cos \left(\overrightarrow{f}, \overrightarrow{a}\right) = \overrightarrow{f} \cdot \overrightarrow{a} = 5\cdot 70 + 4\cdot 20 + (-2)\cdot (-40) = 510(\text{J}).\]
	}
\end{ex} \dongcham{5}

\begin{ex}
	Góc giữa hai véc-tơ $ \vec{i} $ và $ \vec{u}=(-\sqrt{3};0,;1) $ bằng
	\choice
	{$ 60^\circ $}
	{$ 120^\circ $}
	{\True $ 150^\circ $}
	{$ 30^\circ $}
	\loigiai{
		$\cos \left(\vec{i},\vec{u}\right)=\dfrac{\vec{i}\cdot \vec{u}}{\vert \vec{i} \vert \cdot \vert \vec{u} \vert}=\dfrac{1\cdot (-\sqrt3)}{1\cdot\sqrt{3+1}}=-\dfrac{\sqrt{3}}{2}$.\\
		Vậy góc của hai véc-tơ đã cho bằng $ 150^\circ $.
	}
\end{ex} \dongcham{5}

\begin{ex}
	Cho hai véc-tơ $ \vec{u}=(-1;1;0) $ và $ \vec{v}=(0;-1;0) $. Góc hợp bởi hai véc-tơ $ \vec{u} $ và $ \vec{v} $ bằng
	\choice
	{$ 60^\circ $}
	{$ 45^\circ $}
	{\True $ 135^\circ $}
	{$ 120^\circ $}
	\loigiai{
		$\cos \left(\vec{u},\vec{v}\right)=\dfrac{\vec{u}\cdot \vec{v}}{\vert \vec{u} \vert \cdot \vert \vec{v} \vert}=\dfrac{(-1)\cdot 0+1\cdot (-1)+0 \cdot 0}{\sqrt{(-1)^2+1^2+0^2}\sqrt{0^2+(-1)^2+0^2}}=-\dfrac{1}{\sqrt{2}}$.\\
		Vậy góc của hai véc-tơ đã cho bằng $ 135^\circ $.}
\end{ex} \dongcham{6}

\begin{ex}
	Cho hai véc-tơ $ \vec{a}(-2;-3;1) $ và $ \vec{b}(1;0;1) $. Tính $ \cos(\vec{a},\vec{b}) $.
	\choice
	{\True $ \cos(\vec{a},\vec{b})=-\dfrac{1}{2\sqrt{7}} $}
	{$ \cos(\vec{a},\vec{b})=-\dfrac{3}{2\sqrt{7}} $}
	{$ \cos(\vec{a},\vec{b})=\dfrac{1}{2\sqrt{7}} $}
	{$ \cos(\vec{a},\vec{b})=\dfrac{3}{2\sqrt{7}} $}
	\loigiai{
		Ta có $ \cos(\vec{a},\vec{b})=\dfrac{(-2)\cdot 1+(-3)\cdot 0+1\cdot 1}{\sqrt{14}\cdot \sqrt{2}}=-\dfrac{1}{2\sqrt{7}} $.
	}
\end{ex} \dongcham{6}

\begin{ex}
	Cho $\vec{a}=(3;2;1)$, $\vec{b}=(-2;2;-4)$. Giá trị của $\left| \vec{a}-\vec{b} \right|$ bằng
	\choice
	{\True$5\sqrt{2}$}
	{$50$}
	{$2\sqrt{5}$}
	{$3$}
	\loigiai
	{
		Gọi $\vec{c}=\vec{a}-\vec{b}=(5;0-5)\Rightarrow \left| \vec{c} \right|=\sqrt{5^2+(-5)^2}=5\sqrt{2}$.}
\end{ex} \dongcham{6}

\begin{ex}
	Cho hai véc-tơ $\vec{u}=(-1;0;2)$ và $\vec{v}=(x;-2;1)$. Biết rằng $\vec{u}\cdot \vec{v}=4$. Khi đó $|\vec{v}|$ bằng
	\choice
	{$\sqrt{21}$}
	{$2$}
	{\True $3$}
	{$5$}
	\loigiai{
		Ta có $\vec{u}\cdot \vec{v}=-x+2=4\Leftrightarrow x=-2$.\\
		Vậy $|\vec{v}|=3$.
	}
\end{ex} \dongcham{6}

\begin{ex}%[2H3Y1-2]%
	Tìm số thực $a$ để vec-tơ $\vec{u}=(a;0;1)$ vuông góc với vec-tơ $\vec{v}=(2;-1;4)$.
	\choice
	{\True $a=-2$}
	{$a=-4$}
	{$a=4$}
	{$a=2$}
	\loigiai{
		Ta có $\vec{u}\perp\vec{v}\Leftrightarrow \vec{u}\cdot\vec{v}=0\Leftrightarrow 2a+0(-1)+4=0\Leftrightarrow a=-2.$
	}
\end{ex} \dongcham{6}

\begin{ex}
	Tìm $x$ để hai véc-tơ $\vec{a}=(x;x-2;2)$ và $\vec{b}=(x;1;-2)$ vuông góc với nhau.
	\choice
	{$x=3$}
	{$x=1$}
	{\True $\hoac{&x=2\\ &x=-3}$}
	{$\hoac{&x=-2\\&x=3}$}
	\loigiai{
		Hai véc-tơ đã cho vuông góc khi $0=\vec{a}\cdot \vec{b}=x^2+x-2-4$ hay $x=2\ \text{hoặc}\ x=-3$.
	}
\end{ex} \dongcham{6}

\begin{ex}
	Cho hai véc-tơ $\vec{u}=(1;-2;1)$ và $\vec{v}=(2;1;-1)$. Véc-tơ nào dưới đây vuông góc với cả hai véc-tơ $\vec{u}$ và $\vec{v}$?
	\choice
	{\True $\vec{w_2}=(1;3;5)$}
	{$\vec{w_3}=(1;-4;7)$}
	{$\vec{w_4}=(1;4;7)$}
	{$\vec{w_1}=(1;-3;5)$}
	\loigiai{
		Ta có $\vec{u}\cdot\vec{w_2}=0$, $\vec{v}\cdot\vec{w_2}=0$. Do đó $\vec{w_2}$ thỏa mãn đề bài.
	}
\end{ex} \dongcham{6}

\begin{ex}
	Tích có hướng của hai véc-tơ $\vec{a}=(-1;2;0)$ và $\vec{b}=(0;4;-3)$ có tọa độ là
	\choice
	{$(-6;3;-4)$}
	{$(6;-3;4)$}
	{$(6;3;4)$}
	{\True $(-6;-3;-4)$}
	\loigiai{
		Ta có
		$\left[\vec{a},\vec{b}\right]=\left(\left|\begin{array}{lr}
					2 & 0 \\ 4 & -3
				\end{array}\right|;\left|\begin{array}{lr}
					0 & -1 \\ -3 & 0
				\end{array}\right|; \left|\begin{array}{lr}
					-1 & 2 \\ 0 & 4
				\end{array}\right|\right)=(-6;-3;-4)$.
	}
\end{ex} \dongcham{6}

\begin{ex}
	Cho $ A(2;1;4), B(-2;2;-6), C(6;0;-1) $. Tính tích vô hướng $ \vec{AB}\cdot \vec{AC} $.
	\choice
	{$\vec{AB}\cdot \vec{AC}=67$}
	{$\vec{AB}\cdot \vec{AC}=-67$}
	{\True$\vec{AB}\cdot \vec{AC}=33$}
	{$\vec{AB}\cdot \vec{AC}=65$}
	\loigiai{
		Ta có: $ \heva{& \vec{AB}=(-4;1;-10) \\ & \vec{AC}=(4;-1;-5).} $\\
		$ \vec{AB}\cdot \vec{AC}=(-4)\cdot6+1\cdot (-1)+(-10)\cdot (-5)=33 $.
	}
\end{ex} \dongcham{6}

\begin{ex}
	Cho $ A(1;-2; 3)$, $ B(2;-4; 1)$, $ C(2; 0; 2)$, khi đó tích vô hướng $\vec{AB}\cdot\vec{AC}$ bằng
	\choice
	{$4$}
	{\True $-1$}
	{$7$}
	{$-5$}
	\loigiai
	{
		Ta có $\vec{AB}=(1;-2;-2)$ và $\vec{AC}=(1; 2;-1)$.\\
		Vì vậy $\vec{AB}\cdot\vec{AC}=1\cdot 1+(-2)\cdot 2+(-2)\cdot (-1)=-1 $.
	}
\end{ex} \dongcham{5}

\begin{ex}%
	Cho tam giác $ABC$ với $A(8; 9; 2)$, $B(3; 5; 1)$, $C(11; 10; 4)$. Số đo góc $A$ của tam giác $ABC$ là
	\choice
	{$60^\circ$}
	{\True $150^\circ$}
	{$30^\circ$}
	{$120^\circ$}
	\loigiai{
		Ta có $\widehat{BAC} = \left( \vec{AB}; \vec{AC} \right)$, $\vec{AB} = (-5; -4; -1)$, $\vec{AC} = (3;1;2)$. Ta có
		$$\cos \left( \vec{AB}; \vec{AC} \right) = \dfrac{\vec{AB} \cdot \vec{AC}}{\left| \vec{AB} \right| \cdot \left| \vec{AC} \right|} = \dfrac{-21}{\sqrt{42} \cdot \sqrt{14}} = - \dfrac{\sqrt{3}}{2} \Rightarrow \widehat{BAC} = \left( \vec{AB}; \vec{AC} \right) = 150^\circ.$$
	}
\end{ex} \dongcham{6}

\begin{ex}
	Cho điểm $A(3;-1;5)$, $B(m;2;7)$. Tìm tất cả các giá trị của $m$ để độ dài đoạn $AB=7$.
	\choice
	{$m=3$ hoặc $m=-3$}
	{\True $m=9$ hoặc $m=-3$}
	{$m=-3$ hoặc $m=-9$}
	{$m=9$ hoặc $m=3$}
	\loigiai{
		$$AB=7\Leftrightarrow \sqrt{\left(m-3\right)^2+3^2+2^2}=7\Leftrightarrow (m-3)^2=36\Leftrightarrow \hoac{&m-3=6\\&m-3=-6}\Leftrightarrow \hoac{&m=9\\&m=-3.}$$
	}
\end{ex} \dongcham{5}



\begin{ex}
	Cho ba điểm $A(3;2;8)$, $B(0;1;3)$ và $C(2;m;4)$. Tìm $m$ để tam giác $ABC$ vuông tại $B$.
	\choice
	{$m=4$}
	{\True $m=-10$}
	{$m=25$}
	{$m=-1$}
	\loigiai{
		Tam giác $ABC$ vuông tại $B$ tương đương với $\vec{BA}\cdot\vec{BC}=\vec{0}$.\\
		Ta có $\vec{BA}=(3;1;5)$, $\vec{BC}=(2;m-1;1)$.\\
		Nên $\vec{BA}\cdot\vec{BC}=0\Leftrightarrow 3\cdot 2+(m-1)+5\cdot 1=0\Leftrightarrow m=-10$.}
\end{ex} \dongcham{5}

\begin{ex}
	Cho ba điểm $M(2;3;-1)$, $N(-1;1;1)$ và $P(1;m-1;2)$. Tìm $m$ để tam giác $MNP$ vuông tại $N$.
	\choice
	{\True $ m=0 $  }
	{ $ m=-4 $}
	{$ m=2 $ }
	{ $ m=-6 $}
	\loigiai{
		$\vec{MN}(-3;-2;2)$; $\vec{NP}(2;m-2;1)$.\\
		Tam giác $MNP$ vuông tại $N \Leftrightarrow \vec{MN} \cdot \vec{NP}=0 \Leftrightarrow -6-2(m-2)+2=0 \Leftrightarrow m-2=-2 \Leftrightarrow m=0$.}
\end{ex} \dongcham{5}

\begin{ex}%[2H2V2-4]%[2H2H2-4]
	Cho tam giác $ABC$ có $A(7; 3; 3)$, $B(1; 2; 4)$, $C(2; 3; 5)$. Tìm toạ độ điểm $H$ là chân đường cao kẻ từ $A$ của tam giác $ABC$.
	\choice
	{\True $H(3; 4; 6)$}
	{$H(-3; 4; 7)$}
	{$H(2; 4; 1)$}
	{$H(2; -4; 3)$}
	\loigiai{
		Ta có $\overrightarrow{BC} = (1; 1; 1)$.\\
		Gọi $H(x; y; z)$ là chân đường cao của tam giác $ABC$ kẻ từ $A$.\\
		Suy ra $\overrightarrow{BH} = (x-1; y-2; z-4)$.\\
		$\overrightarrow{BH}$ cùng phương với $\overrightarrow{BC}$, do đó $x-1 = t$; $y-2 = t$; $z-4=t$. Suy ra $H(1+t; 2+t; 4+t)$.\\
		Ta có $\overrightarrow{AH} = (x_H-x_A; y_H-y_A; z_H-z_A) = (t-6; t-1; t+1)$.\\
		$\overrightarrow{AH} \perp \overrightarrow{BC} \Leftrightarrow \overrightarrow{AH}\cdot \overrightarrow{BC} = 0 \Leftrightarrow t-6 + t-1 + t+ 1 =0\Leftrightarrow 3t =6 \Leftrightarrow t =2$.\\
		Suy ra $H(3; 4; 6)$.
	}
\end{ex} \dongcham{7}

\begin{ex}
	Cho hai điểm $A(1;1;0)$, $B(2;-1;2)$. Gọi $M(0;0;z)$ là điểm thuộc trục $Oz$ sao cho $MA^2+MB^2$ nhỏ nhất. Khẳng định nào sau đây là đúng?
	\choice
	{\True $z \in (0;1]$}
	{$z \in (1;2]$}
	{$z \in (-1;0]$}
	{$z \in (-2;-1]$}
	\loigiai{Gọi $M(0;0; z)$.Khi đó $MA^2+MB^2=2z^2-4z+11=2(z-1)^2+9 \geq 9$.\\
		Dấu $"="$ xảy ra khi và chỉ khi $z=1$. Do đó, $ M(0;0;1)$.}

\end{ex} \dongcham{11}

\Closesolutionfile{ans}

\BTTF
\Opensolutionfile{ans}[ans/2H2-B3-d2-2]

\begin{ex}
	Cho ba vec-tơ $ \overrightarrow a=(-1;1;0)$, $ \overrightarrow b=(1;1;0)$ và $ \overrightarrow c=(1;1;1)$.
	\choiceTF
	{$\left| {\overrightarrow a } \right| = 2$}
	{\True $\left| {\overrightarrow c } \right| = \sqrt 3 $}
	{$\cos\left(\vec{a},\vec{c} \right)=\dfrac{2}{\sqrt{5}}$}
	{ $\overrightarrow b \perp \overrightarrow c$}
	\loigiai{
		\begin{enumerate}[a)]
			\item $\left| \overrightarrow {a} \right| = \sqrt{(-1)^2+1^2}=\sqrt{2}$.
			\item $\left|\overrightarrow {c} \right| = \sqrt{1^2+1^2+1^2}=\sqrt{3}$
			\item $\cos\left(\vec{a},\vec{c} \right)=\dfrac{\vec{a}\cdot \vec{c}}{|\vec{a}|.|\vec{c}|}=0$
			\item $\vec{b} \cdot \vec{c}=2$, suy ra $\vec{b}$ không vuông $\vec{c}$.
		\end{enumerate}
	}
\end{ex} \dongcham{10}

\begin{ex}
	Cho hai vectơ $\vec{u}=(0;2;3)$ và $\vec{v}=(m-1;2m;3)$.
	\choiceTF
	{\True $\big|\vec{u}\big|=\sqrt{13}$}
	{$\big|\vec{u}\big|=\big|\vec{v}\big| \Leftrightarrow m=-\dfrac{3}{5}$}
	{\True $\vec{u}=\vec{v} \Leftrightarrow m=1$}
	{$\vec{u}\perp\vec{v} \Leftrightarrow m=\dfrac{9}{4}$}
	\loigiai{
		\begin{enumerate}[a)]
			\item $\big|\vec{u}\big|=\sqrt{0^2+2^2+3^2}=\sqrt{13}$
			\item $\big|\vec{u}\big|=\big|\vec{v}\big|\Leftrightarrow \sqrt{13}=\sqrt{(m-1)^2+4m^2+9} \Leftrightarrow 5m^2-2m-3=0 \Leftrightarrow m=1$ hoặc $m=-\dfrac{3}{5}$.
			\item khi $m=1$ thì $\vec{v}=(0;2;3)$. Suy ra $\vec{u}=\vec{v}$.
			\item $\vec{u} \perp \vec{u} \Leftrightarrow 4m+9=0 \Leftrightarrow m=-\dfrac{9}{4}$.
		\end{enumerate}
	}
\end{ex} \dongcham{10}
%Câu 1
\begin{ex}
	Trong không gian với hệ trục tọa độ $Oxyz$, cho ba vectơ $\vec{a}(1;2;3)$, $\vec{b}(2;2;-1)$, $\vec{c}(4;0;-4)$.
	\choiceTF
	{\True Tọa độ của vectơ $\vec{x}=\vec{a}+\vec{b}$ là $\vec{x}=(3;4;2)$}
	{Tọa độ của vectơ $\vec{y}=\vec{a}+\vec{c}$ là $\vec{y}=(5;2;1)$}
	{Tọa độ của vectơ $\vec{z}=\vec{b}+\vec{c}$ là $\vec{z}=(6;-2;-5)$}
	{\True Vectơ $\vec{k}=(7;4;-2)$ thỏa mãn đẳng thức $\vec{k}=\vec{a}+\vec{b}+\vec{c}$}
	\loigiai{
		\begin{enumerate}[a)]
			\item $\vec{x}=\vec{a}+\vec{b}=(3;4;2)$.
			\item $\vec{y}=\vec{a}+\vec{c}=(5;2;-1)$.
			\item $\vec{z}=\vec{b}+\vec{c}=(6;2;-5)$.
			\item $\vec{k}=\vec{a}+\vec{b}+\vec{c}=(7;4;-2)$.
		\end{enumerate}
	}
\end{ex}
%Câu 2
\begin{ex}
	Trong không gian $Oxyz$, cho hai vectơ $\vec{a}(1;-1;5)$, $\vec{b}(3;2;-1)$.
	\choiceTF
	{\True $\vec{a}+\vec{b}\ne \vec{0}$}
	{$\vec{a}-\vec{b}=(-2;-3;4)$}
	{$\vec{v}=\vec{b}-\vec{a}$ có tung độ âm}
	{\True Xét $\vec{x}$ thỏa $\vec{a}-\vec{x}=\vec{b}$. Hoành độ của vectơ $\vec{x}$ thuộc khoảng $(-3;1)$}
	\loigiai{
		\begin{enumerate}
			\item $\vec{a}+\vec{b}=(4;1;4)$.
			\item $\vec{a}-\vec{b}=(-2;-3;6)$.
			\item $\vec{b}-\vec{a}=\left(2;3;-4\right)$.
			\item $\vec{a}-\vec{x}=\vec{b}\Leftrightarrow \vec{x}=\vec{a}-\vec{b}=(-2;-3;6)$. Suy ra hoành độ của vectơ $\vec{x}$ là $-2\in (-3;1)$.
		\end{enumerate}
	}
\end{ex}
%Câu 4
\begin{ex}
	Trong không gian $Oxyz$, cho điểm $D(4;-1;3)$ và các điểm $M$, $N$, $P$ lần lượt thuộc các trục
	$Ox$, $Oy$, $Oz$ sao cho $DM$, $DN$, $DP$ đôi một vuông góc với nhau
	\choiceTF
	{Tung độ của điểm $N$ bằng $13$}
	{Cao độ của điểm $P$ bằng $\dfrac{13}{4}$}
	{\True $V_{DMNP}>29$}
	{Gọi $\vec{x}$ là vectơ thỏa $\vec{x} \cdot \vec{DM}=1$; $\vec{x} \cdot \vec{DN}=2$; $\vec{x} \cdot \vec{DP}=-3$ thì tổng hoành độ, tung độ và cao độ của vectơ $\vec{x}$ thuộc khoảng $(3;7)$}
	\loigiai{
		\begin{itemize}
			\item Gọi $M(a;0;0)$, $N(0;b;0)$, $P(0;0;c)$.\\
			      $\vec{DM}=(a-4;1;-3)$, $\vec{DN}=(-4;b+1;-3)$, $\vec{DP}=(-4;1;c-3)$\\
			      Ta có $DM$, $DN$, $DP$ đôi một vuông góc với nhau nên \\
			      $\heva{& \vec{DM} \cdot \vec{DN}=0 \\& \vec{DM} \cdot \vec{DP}=0 \\& \vec{DN} \cdot \vec{DP}=0}\Leftrightarrow \heva{& -4(a-4)+b+1+9=0 \\& -4(a-4)+1-3(c-3)=0 \\& 16+b+1-3(c-3)=0}\Leftrightarrow \heva{& -4a+b=-26 \\& -4a-3c=-26 \\& b-3c=-26}\Leftrightarrow \heva{& a=\dfrac{13}{4} \\& b=-13 \\& c=\dfrac{13}{3}}$.
			\item ${{V}_{DMNP}}=\dfrac{1}{6}DM \cdot DN \cdot DP=\dfrac{1}{6} \cdot \dfrac{13}{4} \cdot 13 \cdot \dfrac{13}{3}=\dfrac{2197}{72}>29$.
			\item Gọi $\vec{x}=\left(m;n;p\right)$\\
			      $\vec{DM}=\left(-\dfrac{3}{4};1;-3\right);\vec{DN}=(-4;-12;-3);\vec{DP}=\left(-4;1;\dfrac{4}{3}\right)$\\
			      $\heva{& \vec{x} \cdot \vec{DM}=1 \\& \vec{x} \cdot \vec{DN}=2 \\& \vec{x} \cdot \vec{DP}=-3}\Leftrightarrow \heva{& -\dfrac{3}{4}m+n-3p=1 \\& -4m-12n-3p=2 \\& -4m+n+\dfrac{4}{3}p=-3}\Leftrightarrow \heva{& m=\dfrac{88}{169} \\& n=-\dfrac{35}{169} \\& p=-\dfrac{90}{169}}$\\
			      $m+n+p=\dfrac{-37}{169}$.
		\end{itemize}
	}
\end{ex}

\begin{ex}
	Cho tam giác $ABC$ có $ A(1;2;0) $, $ B(0;1;1) $, $ C(2;1;0) $.
	\choiceTF
	{\True  Tam giác $ABC$ vuông tại $A$}
	{Chu vi tam giác là $ \sqrt{7}+\sqrt{3}+\sqrt{2} $}
	{Diện tích tam giác $ ABC $ là $ \sqrt{6} $}
	{\True Tâm đường tròn ngoại tiếp tam giác $ ABC $ là $ I\left(1;1;\dfrac{1}{2}\right) $}
	\loigiai{
		Ta có $ \overrightarrow{AB}=(-1;-1;1) \Rightarrow AB=\sqrt{3}$, $ \overrightarrow{AC}=(1;-1;0) \Rightarrow AC=\sqrt{2}$,\\ $\overrightarrow{BC}=(2;0;-1) \Rightarrow BC=\sqrt{5}$.
		\begin{enumerate}[a)]
			\item $ \overrightarrow{AB}\cdot \overrightarrow{AC}=0 $ do đó $ AB\perp AC $, tam giác $ ABC $ vuông tại $ A $.
			\item Chu vi của tam giác là $ AB+AC+BC=\sqrt{3}+\sqrt{2}+\sqrt{5} $.
			\item Diện tích là\\ $ S=\dfrac{1}{2}\cdot AB\cdot AC=\dfrac{\sqrt{6}}{2} $
			\item Tâm đường tròn ngoại tiếp là trung điểm của $ BC $ có tọa độ $ I\left(1;1;\dfrac{1}{2}\right)$.
		\end{enumerate}
	}
\end{ex} \dongcham{25}

\begin{ex}
	Hình minh họa sơ đồ một ngôi nhà trong hệ trục tọa độ $Oxyz$, trong đó nền nhà, bốn bức tường và hai mái nhà đều là hình chữ nhật.
		\choiceTF
		{Tọa độ của các điểm $A(5;0;0)$}
		{Tọa độ của các điểm $H(0;5;3)$}
		{Góc nhị diện có cạnh là đường thẳng $FG$, hai mặt lần lượt là $(FGQP)$ và $(FGHE)$ gọi là góc dốc của mái nhà. Số đo của góc dốc của mái nhà bằng $26{,}6^\circ$ (làm tròn kết quả đến hàng phần mười của độ)}
		{Chiều cao của ngôi nhà là 4}
	\begin{center}
		\begin{tikzpicture}[scale=1, font=\footnotesize, line join=round, line cap=round, >=stealth]
			\path
			(0,0) coordinate (O) node[below]{$O(0;0;0)$}
			(4,0) coordinate (A) node[below]{$A$}
			(0,3) coordinate (E) node[left]{$E(0;0;3)$}
			($(A)+(E)-(O)$) coordinate (F)node[right]{$F$}
			(A)+(1,2) coordinate (B) node[right]{$B(4;5;0)$}
			($(O)+(B)-(A)$) coordinate (C) node[left]{$C$}
			($(C)+(E)-(O)$) coordinate (H) node[above left]{$H$}
			($(H)+(B)-(C)$) coordinate (G) node [right]{$G(4;5;3)$}
			($(O)!.6!(A)$) coordinate (x)
			(x)+(0,4.5) coordinate (P) node[right]{$P(2;0;4)$}
			($(P)+(H)-(E)$) coordinate (Q) node[above]{$Q(2;5;4)$}
			;
			\draw[->] (O)--(E)--($(E)+(90:1)$)node[above]{$z$};
			\draw[->] (O)--(A)--($(A)+(0:1)$) node[below]{$x$};
			\draw[dashed,->] (O)--(C)--($(O)!1.3!(C)$) node[above]{$y$};
			\draw (A)--(B)--(G)--(Q)--(H)--(E)--(F)--cycle (E)--(P)--(F) (P)--(Q) (F)--(G);
			\draw[dashed] (C)--(B) (C)--(H)--(G);
		\end{tikzpicture}
	\end{center}
	\loigiai{
		\begin{enumerate}
			\item Vì nền nhà là hình chữ nhật nên tứ giác $OABC$ là hình chữ nhật, suy ra $x_A=x_B=4$, $y_C=y_B=5$. Do $A$ nằm trên trục $Ox$ nên tọa độ điểm $A$ là $(4;0;0)$.
			\item Tường nhà là hình chữ nhật, suy ra $y_H=y_C=5$, $z_H=z_E=3$. Do $H$ nằm trên mặt phẳng $(Oyz)$ nên tọa độ điểm $H$ là $(0;5;3)$.\\
			\item Để tính góc dốc của mái nhà, ta đi tính số đo góc nhị diện có cạnh là đường thẳng $FG$, hai mặt phẳng lần lượt là $(FGQP)$ và $(FGHE)$. Do mặt phẳng $(Ozx)$ vuông góc với hai mặt phẳng $(FGQP)$ và $(FGHE)$ nên góc $PFE$ là góc phẳng nhị diện ứng với góc nhị diện đó. \\
			      Ta có $\vec{FP}=(-2;0;1)$, $\vec{FE}=(-4;0;0)$.\\
			      Suy ra
			      \begin{eqnarray*}
				      \cos \widehat{PFE}&=&\cos \left(\vec{FP},\vec{FE}\right)=\dfrac{\vec{FP}\cdot \vec{FE}}{\left|\vec{FP}\right|\cdot \left|\vec{FE}\right|}\\
				      &=&\dfrac{(-2)\cdot (-4)+0\cdot 0+1\cdot 0}{\sqrt{(-2)^2+0^2+1^2}\cdot \sqrt{(-4)^2+0^2+0^2}}=\dfrac{2\sqrt{5}}{5}.
			      \end{eqnarray*}
			      Do đó, $\widehat{PFE}\approx 26{,}^\circ$. Vậy góc dốc của mái nhà khoảng $26{,}6^\circ$.
			\item Chiều cao bằng cao độ của điểm $P$. Suy ra $h=4$.
		\end{enumerate}
	}
\end{ex} \dongcham{45}
\BTTL
\begin{ex}
	Trong không gian $Oxyz$, cho hai vectơ $\vec{a}=(1;2;-3);\vec{b}=(-1;-2;z)$. Tìm giá trị $z$ sao cho	$\vec{a}+\vec{b}=\vec{0}$
	\loigiai{
		\SA{3}
		Ta có: $\vec{a}+\vec{b}=\left(0;0;z-3\right)$.\\
		$\vec{a}+\vec{b}=\vec{0}\Leftrightarrow z-3=0\Leftrightarrow z=3$.\\
		Vậy $z=3$.
	}
\end{ex}
%Câu 2
\begin{ex}
	Trong không gian $Oxyz$, cho hai vectơ $\vec{a}=2\vec{i}-3\vec{j}+6\vec{k}$ và $\vec{b}=6\vec{j}+\vec{k}$. Khi đó độ dài của
	$\vec{a}-2\vec{b}$ (làm tròn đến hàng phần mười)
	\loigiai{
		\SA{15,7}
		Ta có: $\vec{a}=2\vec{i}-3\vec{j}+6\vec{k}\Rightarrow \vec{a}=(2;-3;6)$\\
		$\vec{b}=6\vec{j}+\vec{k}\Rightarrow \vec{b}=(0;6;1)$\\
		Khi đó: $\vec{a}-2\vec{b}=(2;-15;4)\Rightarrow \left| \vec{a}-2\vec{b} \right|=7\sqrt{5}\approx 15{,}7$
	}
\end{ex}
%Câu 4
\begin{ex}
	Trong không gian $Oxyz$, cho các vectơ $\vec{a}=(1;0;-2),\text{ }\vec{b}=(-2;1;3)$,$\vec{c}=(3;2;-1)$, $\vec{d}=(9;0;-11)$ và $3$ số thực $m,n,p$ thỏa $m \cdot \vec{a}+n \cdot \vec{b}+p\vec{c}=\vec{d}$. Tính giá trị biểu thức $T=m+n+p$.
	\loigiai{
		\SA{1}
		Ta có: $m \cdot \vec{a}+n \cdot \vec{b}+p\vec{c}=\left(m-2n+3p;n+2p;-2m+3n-p\right)$, $\vec{d}=\left(9;0;-11\right)$.\\
		$m \cdot \vec{a}+n \cdot \vec{b}+p\vec{c}=\vec{d}\Leftrightarrow \heva{& m-2n+3p=9 \\& n+2p=0 \\& -2m+3n-p=-11} \Leftrightarrow \heva{& m=2 \\& n=-2 \\& p=1.} $\\
		Vậy $T=m+n+p=1$.
	}
\end{ex}
\Closesolutionfile{ans}

%Chương III. Mẫu số liệu ghép nhóm
%%Bài 1
% \setcounter{section}{0}
\section{KHOẢNG BIẾN THIÊN, KHOẢNG TỨ PHÂN VỊ CỦA MSL GHÉP NHÓM}
\subsection{LÝ THUYẾT CẦN NHỚ}
\subsubsection{Khoảng biến thiên}
\begin{enumerate}[\iconMT] 
	\item \indam{Định nghĩa:} Xét mẫu số liệu ghép nhóm được cho ở bảng sau:
	\begin{center}
		\begin{tikzpicture}
			\matrix[matrix of nodes,nodes in empty cells,
			row sep=-\pgflinewidth,column sep=-\pgflinewidth,
			nodes={minimum height=7mm,minimum width=20mm,draw=black,anchor=center},
			column 1/.style={nodes={minimum width=24mm,color=black}},
			row 1/.style={nodes={fill=cyan!10}},
			row 2/.style={nodes={minimum height=7mm}},
			]{
				Nhóm &$[u_1;u_2)$&$[u_1;u_2)$&\dots&$[u_k;u_{k+1})$\\ 
				\node[align=center]{Tần số}; &$n_1$&$n_2$&\dots&$n_k$\\
			};
		\end{tikzpicture}
	\end{center}
	Nếu $n_1$ và $n_k$ cùng khác $0$ thì khoảng biến thiên của mẫu số liệu ghép nhóm được tính theo công thức
		\boxmini{$R=u_{k+1}-u_1$}
	% \item \indam{Ý nghĩa:}
	% \begin{listEX}[1]
	% 	\item [\iconCH] Khoảng biến thiên của mẫu số liệu ghép nhóm là giá trị xấp xỉ khoảng biến thiên của mẫu số liệu gốc và có thể dùng để đo mức độ phân tán của mẫu số liệu. Khoảng biến thiên càng lớn thì mẫu số liệu càng phân tán.
	% 	\item [\iconCH] Trong các đại lượng đo mức độ phân tán của mẫu số liệu ghép nhóm, khoảng biến thiên là đại lượng dễ hiểu, dễ tính toán. Tuy nhiên, do khoảng biến thiên chỉ sử dụng hai giá trị $u_1$ và $u_{m+1}$ của mẫu số liệu nên đại lượng đó dễ bị ảnh hưởng bởi các giá trị bất thuờng.
	% \end{listEX}
\end{enumerate}

\subsubsection{Khoảng tứ phân vị}
\begin{enumerate}[\iconMT] 
	\item \indam{Định nghĩa:}
	Khoảng tứ phân vị của mẫu số liệu ghép nhóm, kí hiệu $\Delta_Q$, là hiệu giữa tứ phân vị thứ ba $Q_3$ và tứ phân vị thứ nhất $Q_1$ của mẫu số liệu ghép nhóm đó, tức là \boxmini{$\Delta_Q=Q_3-Q_1$}
	\item \indam{Ý nghĩa:}
	\begin{listEX}[1]
		\item [\iconCH] Khoảng tứ phân vị của mẫu số liệu ghép nhóm là giá trị xấp xỉ cho khoảng tứ phân vị của mẫu số liệu gốc và có thể dùng để đo mức độ phân tán của nửa giữa của mẫu số liệu (tập hợp gồm $50 \%$ số liệu nằm chính giữa mẫu số liệu).
		% \item [\iconCH] Khoảng tứ phân vị của mẫu số liệu ghép nhóm càng nhỏ thì dữ liệu càng tập trung xung quanh trung vị.
		\item [\iconCH] Khoảng tứ phân vị được dùng để xác định giá trị bất thường trong mẫu số liệu. Giá trị $x$ trong mẫu số liệu là giá trị bất thường nếu $x>Q_3+1,5 \Delta_Q$ hoặc $x<Q_1-1,5 \Delta_Q$.
		% \item [\iconCH] Khoảng tứ phân vị của mẫu số liệu ghép nhóm không bị ảnh hưởng nhiều bởi các giá trị bất thường trong mẫu số liệu.
	\end{listEX}
	% \begin{note}
	% 	$
	% 	Q_1=a_p+\dfrac{\frac{n}{4}-\left(m_1+\ldots+m_{p-1}\right)}{m_p}\cdot\left(a_{p+1}-a_p\right),
	% 	$\\
	% 	$
	% 	Q_3=a_p+\dfrac{\frac{3 n}{4}-\left(m_1+\ldots+m_{p-1}\right)}{m_p}\cdot\left(a_{p+1}-a_p\right) .
	% 	$
	% \end{note}
\end{enumerate}

\subsection{PHÂN LOẠI VÀ PHƯƠNG PHÁP GIẢI TOÁN}
\begin{dang}{Tìm khoảng biến thiên của mẫu số liệu ghép nhóm}
% \begin{listEX}[1]
	% \item [\ding{172}] Xác định $ u_1 $ là giá trị đầu mút trái của nhóm đầu tiên và $ u_{k+1} $ là giá trị đầu mút phải của nhóm cuối cùng có chứa dữ liệu (tần số khác $0$).
	% \item [\ding{173}] Khoảng biến thiên $ R=u_{k+1}-u_1 $.
% \end{listEX}
\end{dang}
% \boxmini{BÀI TẬP TỰ LUẬN}
\viduminhhoa
\begin{vd}%[1T5B1-1]
	Cân nặng của $28$ học sinh nam lớp $11$ được cho như sau:
	\begin{center}
		\begin{tabular}{lllllll}
			$55{,}4$ & $62{,}6$ & $54{,}2$ & $56{,}8$ & $58{,}8$ & $59{,}4$ & $60{,}7$ \\
			$58$ & $59{,}5$ & $63{,}6$ & $61{,}8$ & $52{,}3$ & $63{,}4$ & $57{,}9$\\
			$49{,}7$ & $45{,}1$ & $56{,}2$ & $63{,}2$ & $46{,}1$ & $49{,}6$ & $59{,}1$\\
			$55{,}3$ & $55{,}8$ & $45{,}5$ & $46{,}8$ & $54$ & $49{,}2$ & $52{,}6$
		\end{tabular}
	\end{center}
\begin{tasks}
	\task Hãy chuyển mẫu số liệu trên sang mẫu số liệu ghép nhóm gồm $5$ nhóm có độ dài bằng nhau với nhóm đầu tiên là $[45; 49)$.
	\task Tìm khoảng biến thiên của mẫu số liệu gốc và bảng biến thiên của mẫu số liệu ghép nhóm tương ứng.
\end{tasks}
\loigiai{
	\begin{enumEX}[a)]{1}
		\item Các nhóm $[45; 49)$, $[49; 53)$, $[53; 57)$, $[57; 61)$, $[61; 65)$. Khi đó ta có bảng tần số ghép nhóm sau:
		\begin{center}
			\begin{tabular}{|c|c|c|c|c|c|}
				\hline Cân nặng &{$[45; 49)$} &{$[49; 53)$} &{$[53; 57)$} &{$[57; 61)$} &{$[61; 65)$} \\
				\hline Số học sinh & 4 & 5 & 7 & 7 & 5 \\
				\hline
			\end{tabular}
		\end{center}
		\item Khoảng biến thiên của mẫu số liệu gốc là $63{,}6-45{,}1=18{,}5$.\\
		Khoảng biến thiên của mẫu số liệu ghép nhóm $65-45=20$.
	\end{enumEX}=
	}
\end{vd}

\begin{vd}
	Bảng sau thống kê thời gian tập thể dục buổi sáng mỗi ngày trong tháng 9/2022 của bác Bình và bác An.
	\begin{center}
		\begin{tabular}{|c|c|c|c|c|c|}
			\hline \begin{tabular}{c} 
				Thời gian \\
				(phút)
			\end{tabular} &{$[15; 20)$} &{$[20; 25)$} &{$[25; 30)$} &{$[30; 35)$} &{$[35; 40)$} \\
			\hline \begin{tabular}{c} 
				Số ngày tập
				của bác Bình
			\end{tabular} & $ 5 $ & $ 12 $ & $ 8 $ & $ 3 $ & $ 2 $ \\
			\hline \begin{tabular}{c} 
				Số ngày tập
				của bác An
			\end{tabular} & $ 0 $ & $ 25 $ & $ 5 $ & $ 0 $ & $ 0 $ \\
			\hline
		\end{tabular}
	\end{center}
	\begin{enumEX}{1}
		\item Hãy tìm khoảng biến thiên của mẫu số liệu ghép nhóm về thời gian tập thể dục buổi sáng mỗi ngày của bác Bình và bác An.
		\item Sử dụng khoảng biến thiên, hãy cho biết bác nào có thời gian tập phân tán hơn.
	\end{enumEX}
	\loigiai{
		\begin{enumEX}{1}
			\item Khoảng biến thiên của mẫu số liệu ghép nhóm về thời gian tập thể dục buổi sáng của bác Bình là $40-15=25$ (phút).\\
			Trong mẫu số liệu ghép nhóm về thời gian tập thể dục buổi sáng của bác An, khoảng đầu tiên chứa dữ liệu là $[20; 25)$ và khoảng cuối cùng chứa dữ liệu là $[25; 30)$.\\
			Do đó khoảng biến thiên của mẫu số liệu ghép nhóm về thời gian tập thể dục buổi sáng của bác An là $30-20=10$ (phút).
			\item Nếu căn cứ theo khoảng biến thiên thì bác Bình có thời gian tập phân tán hơn bác An.
		\end{enumEX}	
	}
\end{vd}

\begin{vd}
	Thống kê thời gian sử dụng mạng xã hội trong ngày của các bạn Tổ 1, Tổ 2 lớp 12A, được kết quả như bảng sau:
	\begin{center}
		\begin{tabular}{|l|c|c|c|c|}
			\hline Thời gian sử dụng (phút) &{$[0; 10)$} &{$[10; 30)$} &{$[30; 60)$} &{$[60; 90)$} \\
			\hline Số học sinh Tổ 1 & $ 2 $ & $ 4 $ & $ 3 $ & $ 1 $ \\
			\hline Số học sinh Tổ 2 & $ 5 $ & $ 1 $ & $ 3 $ & $ 0 $ \\
			\hline
		\end{tabular}
	\end{center}
	Tìm khoảng biến thiên cho thời gian sử dụng mạng xã hội của học sinh mỗi tổ và giải thích ý nghĩa.
	\loigiai{
		Gọi $R_1, R_2$ tương ứng là khoảng biến thiên của mẫu số liệu ghép nhóm về thời gian sử dụng mạng xã hội trong ngày của các bạn Tổ 1 và Tổ 2.\\
		Ta có: $R_1=90-0=90$ và $R_2=60-0=60$.\\
		Do $R_1>R_2$ nên nếu dựa vào khoảng biến thiên, ta kết luận rằng thời gian sử dụng mạng xã hội trong ngày của các bạn Tổ 1 phân tán hơn thời gian sử dụng mạng xã hội của các bạn Tổ 2.	
	}
\end{vd}
\baitaptn
% \boxmini{BÀI TẬP TRẮC NGHIỆM}
\Opensolutionfile{ans}[ans/2D3-B1-d1]
\begin{ex}
	Khảo sát thời gian tập thể dục của một số học sinh khối $11$ thu được mẫu số liệu ghép nhóm sau:
	\vspace*{-10pt}
	\begin{center}
		\begin{tabular}{|c|c|c|c|c|c|}
			\hline Thời gian & {$[0 ; 20)$} & {$[20 ; 40)$} & {$[40 ; 60)$} & {$[60 ; 80)$} &  {$[80 ;100)$} \\
			\hline Số học sinh & $5$ & $9$ & $12$ & $10$ & $6$  \\
			\hline
		\end{tabular}
	\end{center}
	Tìm khoảng biến thiên của mẫu số liệu ghép nhóm trên.
	\choice
	{$80$}
	{$60$}
	{\True $100$}
	{$12$}
	\loigiai{
		Xác định $ u_1=0 $ là giá trị đầu mút trái của nhóm đầu tiên và $ u_{k+1}=100 $ là giá trị đầu mút phải của nhóm cuối cùng có chứa dữ liệu. Suy ra $R=u_{k+1}-u_{1}=100-0=100$.
}
\end{ex}

\begin{ex}
	Mức thưởng tết (triệu đồng) cho các nhân viên của một công ty được thống kê trong bảng sau:
	\vspace*{-10pt}
	\begin{center}
		\begin{tabular}{|c|c|c|c|c|c|}
			\hline Mức thưởng tết & {$[5 ; 10)$} & {$[10 ; 15)$} & {$[15 ; 20)$} & {$[20 ; 25)$} &  {$[25 ;30)$} \\
			\hline Số nhân viên & $13$ & $35$ & $47$ & $25$ & $10$  \\
			\hline
		\end{tabular}
	\end{center}
	Tìm khoảng biến thiên của mẫu số liệu ghép nhóm trên.
	\choice
	{$20$}
	{\True $25$}
	{$47$}
	{$23$}
	\loigiai{
			Xác định $ u_1=5 $ là giá trị đầu mút trái của nhóm đầu tiên và $ u_{k+1}=30 $ là giá trị đầu mút phải của nhóm cuối cùng có chứa dữ liệu. Suy ra $R=u_{k+1}-u_{1}=30-5=25$.
}
\end{ex}

\begin{ex}
	Cho bảng phân bố tần số ghép lớp sau
	\vspace*{-10pt}
	\begin{center}
		Chiều cao của $40$ học sinh nam ở một trường THPT\\
		\begin{tabular}{|c|c|c|c|c|c|}
			\hline
			Lớp chiều cao (cm) & [160; 163,5) & [164; 167,5) & [168; 171,5) & [172; 175,5) & Cộng\\
			\hline
			Tần số & 9 & 20 & 7 & 4 & 40\\
			\hline
		\end{tabular}
	\end{center}
	Tìm khoảng biến thiên của mẫu số liệu ghép nhóm trên.
	\choice
	{$31$}
	{\True $15,5$}
	{$175,5$}
	{$12$}
	\loigiai{
			Xác định $ u_1=160 $ là giá trị đầu mút trái của nhóm đầu tiên và $ u_{k+1}=175,5 $ là giá trị đầu mút phải của nhóm cuối cùng có chứa dữ liệu. Suy ra $R=u_{k+1}-u_{1}=175,5-160=15,5$.
	}
\end{ex}


\begin{ex}
	Thời gian truy cập Internet mỗi buổi tối của một số học sinh được cho trong bảng sau:
%	\vspace*{-10pt}
	\begin{center}
		\begin{tabular}{|c|c|c|c|c|c|}
			\hline Thời gian (phút) & {$[9{,}5 ; 12{,}5)$} & {$[12{,}5 ; 15{,}5)$} & {$[15{,}5 ; 18{,}5)$} & {$[18{,}5 ; 21{,}5)$} & {$[21{,}5 ; 24{,}5)$} \\
			\hline Số học sinh & $0$ & $12$ & $15$ & $24$ & $26$ \\
			\hline
		\end{tabular}
	\end{center}
	Tìm khoảng biến thiên của mẫu số liệu ghép nhóm trên.
	\choice
	{$26$}
	{$14$}
	{$20$}
	{\True $12$}
	\loigiai{
		Xác định $ u_1=12,5 $ là giá trị đầu mút trái của nhóm đầu tiên và $ u_{k+1}=24,5 $ là giá trị đầu mút phải của nhóm cuối cùng có chứa dữ liệu. Suy ra $R=u_{k+1}-u_{1}=24,5-12,5=12$.
	}
\end{ex}

\begin{ex}%[2D3H1-2]
	Thời gian hoàn thành bài kiểm tra môn Toán của các bạn trong lớp $12$C được cho trong bảng sau:
	%\vspace*{-10pt}
	\begin{center}
		\begin{tabular}{|l|c|c|c|c|}
			\hline
			Thời gian (phút) & $[25;30 )$ & $[30;35)$ &$[35;40 )$ & $[40;45)$ \\
			\hline
			Số học sinh & $8$ & $16$ & $12$ & $2$ \\
			\hline
		\end{tabular}
	\end{center}
	Tìm khoảng biến thiên của mẫu số liệu ghép nhóm trên.
	\choice
	{$24$}
	{$15$}
	{$2$}
	{\True $20$}
	\loigiai{
		Xác định $ u_1=25 $ là giá trị đầu mút trái của nhóm đầu tiên và $ u_{k+1}=45 $ là giá trị đầu mút phải của nhóm cuối cùng có chứa dữ liệu. Suy ra $R=u_{k+1}-u_{1}=45-25=20$.
	}
\end{ex}

\begin{dang}{Tìm tứ phân vị của mẫu số liệu ghép nhóm}
	\indamm{Với mẫu số liệu ghép nhóm}
	\begin{center}
		\begin{tabular}{|l|c|c|c|c|c|}
			\hline Nhóm &{$\left[a_1; a_2\right)$} & $\ldots$ &{$\left[a_i; a_{i+1}\right)$} & $\ldots$ &{$\left[a_k; a_{k+1}\right)$} \\
			\hline Tần số & $m_1$ & $\ldots$ & $m_i$ & $\ldots$ & $m_k$ \\
			\hline
		\end{tabular}
	\end{center}
	\indamm{Các bước thực hiện:}
	\begin{listEX}[1]
		\item [\ding{172}] Tìm tứ phân vị $ Q_1$ và $Q_3 $ theo công thức:
		$$Q_r=a_p+\dfrac{\dfrac{r \cdot n}{4}-\left(m_1+\cdots+m_{p-1}\right)}{m_p} \cdot\left(a_{p+1}-a_p\right), $$
		trong đó $\left[a_p; a_{p+1}\right)$ là nhóm chứa tứ phân vị thứ $r$ với $r=1$, $3$; \quad $n$ là cỡ mẫu.
		\item [\ding{173}] Khoảng tứ phân vị của mẫu số liệu ghép nhóm là $\Delta_Q=Q_3-Q_1$.
	\end{listEX}
\end{dang}
\viduminhhoa
% \boxmini{BÀI TẬP TỰ LUẬN}
\setcounter{vd}{0}
\begin{vd}
	Bảng sau thống kê cân nặng của $ 50 $ quả xoài được lựa chọn ngẫu nhiên sau khi thu hoạch ở một nông trường.
	\begin{center}
		\begin{tabular}{|c|c|c|c|c|c|}
			\hline Cân nặng $(\mathrm{g})$ &{$[250; 290)$} &{$[290; 330)$} &{$[330; 370)$} &{$[370; 410)$} &{$[410; 450)$} \\
			\hline Số quả xoài & $ 3 $ & $ 13 $ & $ 18 $ & $ 11 $ & $ 5 $ \\
			\hline
		\end{tabular}
	\end{center}
	Hãy tìm khoảng tứ phân vị của mẫu số liệu ghép nhóm đã cho.
	\loigiai{
		Cỡ mẫu $n=50$.\\
		Gọi $x_1; x_2; \ldots; x_{50}$ là mẫu số liệu gốc gồm cân nặng của $ 50 $ quả xoài được xếp theo thứ tự không giảm.\\
		Ta có 
		\begin{listEX}[3]
			\item [] $x_1, x_2, x_3 \in[250; 290)$
			\item [] $x_4, \ldots, x_{16} \in[290; 330)$
			\item [] $x_{17}, \ldots, x_{34} \in[330; 370)$
			\item [] $x_{35}, \ldots, x_{45} \in[370; 410)$
			\item [] $x_{46}, \ldots, x_{50} \in[410; 450).$
		\end{listEX}
		Tứ phân vị thứ nhất của mẫu số liệu gốc là $x_{13} \in[290; 330)$.\\
		Do đó, tứ phân vị thứ nhất của mẫu số liệu ghép nhóm là
		$$Q_1=290+\dfrac{\dfrac{50}{4}-3}{13} \cdot(330-290)=\dfrac{4150}{13} .$$	
		Tứ phân vị thứ ba của mẫu số liệu gốc là $x_{38} \in[370; 410)$. Do đó, tứ phân vị thứ ba của mẫu số liệu ghép nhóm là 
		$$Q_3=370+\dfrac{\dfrac{3 \cdot 50}{4}-(3+13+18)}{11} \cdot(410-370)=\dfrac{4210}{11}.$$
		Vậy khoảng tứ phân vị của mẫu số liệu ghép nhóm là
		$$\Delta_Q=\dfrac{4210}{11}-\dfrac{4150}{13}=\dfrac{9080}{143} \approx 63,5.$$
	}
\end{vd}

\begin{vd}%[2D3H1-4]
	Bảng sau đây cho biết chiều cao của các học sinh lớp 12A và 12B.
	\begin{center}
		\begin{tabular}{|c|c|c|c|c|c|c|}
			\hline
			Chiều cao (cm) & $[145;150 )$ & $[150;155)$ &$[155;160 )$ & $[160;165)$ & $[165;170)$ & $[170;175)$ \\
			\hline
			Số học sinh của lớp 12A & $1$ & $0$ & $15$ & $12$ & $10$ & $5$\\
			\hline
			Số học sinh của lớp 12B & $0$ & $0$ & $17$ & $10$ & $9$ & $6$\\
			\hline
		\end{tabular}
	\end{center}
	\begin{enumerate}
		\item Tính khoảng biến thiên, khoảng tứ phần vị cho các mẫu số liệu ghép nhóm của học sinh lớp 12A, 12B.
		\item Để so sánh độ phân tán về chiều cao của học sinh hai lớp này ta nên dùng khoảng biến thiên hay khoảng tứ phân vị? Vì sao?	
	\end{enumerate}
	\loigiai{
		\begin{enumerate}
			\item Ta có
			\begin{center}
				\begin{tabular}{|c|c|c|c|c|c|c|}
					\hline
					Chiều cao (cm) & $[145;150 )$ & $[150;155)$ &$[155;160 )$ & $[160;165)$ & $[165;170)$ & $[170;175)$ \\
					\hline
					Số học sinh \\
					của lớp 12A & $1$ & $0$ & $15$ & $12$ & $10$ & $5$\\
					\hline
					Số học sinh \\
					của lớp 12B & $0$ & $0$ & $17$ & $10$ & $9$ & $6$\\
					\hline
				\end{tabular}
			\end{center}
			Khoảng biến thiên là $175-145 = 30$ (cm).\\
			Xét lớp 12A,\\
			\[Q_1 = 155 + \dfrac{\dfrac{43}{4}-1}{15}\cdot 5 = 158{,}25.\]
			\[Q_3 = 165 + \dfrac{\dfrac{43\cdot 3}{4}-28}{10}\cdot 5 = 167{,}125.\]
			\[\triangle Q = Q_3 -Q_1 = 8{,}875.\]
			Xét lớp 12B,\\
			\[Q_1 = 155 + \dfrac{\dfrac{42}{4}-0}{17}\cdot 5 = 158{,}5\]
			\[Q_3 = 165 + \dfrac{\dfrac{42\cdot 3}{4}-27}{9}\cdot 5 = 167{,}5\]
			\[\triangle Q = Q_3 -Q_1 = 9{,}4.\]
			\item Để so sánh độ phân tán về chiều cao của học sinh hai lớp này ta nên dùng khoảng tứ phân vị, vì khoảng biến thiên của $2$ lớp này là bằng nhau.
		\end{enumerate}
	}
\end{vd}

\begin{vd}
	Hằng ngày ông Thắng đều đi xe buýt từ nhà đến cơ quan. Dưới đây là bảng thống kê thời gian của $ 100 $ lần ông Thắng đi xe buýt từ nhà đến cơ quan.
	\begin{center}
		\begin{tabular}{|c|c|c|c|c|c|c|}
			\hline Thời gian(phút) &{$[15; 18)$} &{$[18; 21)$} &{$[21; 24)$} &{$[24; 27)$} &{$[27; 30)$} &{$[30; 33)$} \\
			\hline Số lần & $ 22 $ & $ 38 $ & $ 27 $ & $ 8 $ & $ 4 $ & $ 1 $ \\
			\hline
		\end{tabular}
	\end{center}
	\begin{enumEX}{1}
		\item Hãy tìm khoảng tứ phân vị của mẫu số liệu ghép nhóm trên. (Làm tròn kết quả đến hàng phần trăm.)
		\item Biết rằng trong $ 100 $ lần đi trên, chỉ có đúng một lần ông Thắng đi hết $ 32 $ phút. Thời gian của lần đi đó có phải là giá trị ngoại lệ không?
	\end{enumEX}
	\loigiai{
		\begin{enumEX}{1}
			\item Cỡ mẫu $n=100$.\\
			Gọi $x_1; x_2; \ldots; x_{100}$ là mẫu số liệu gốc gồm thời gian 100 lần đi xe buýt của ông Thắng.\\
			Ta có: 
			\begin{listEX}[3]
				\item [] $x_1, \ldots, x_{22} \in[15; 18)$
				\item [] $x_{23}, \ldots, x_{60} \in[18; 21)$
				\item [] $x_{61}, \ldots, x_{87} \in[21; 24)$
				\item [] $x_{88}, \ldots, x_{95} \in[24; 27)$
				\item [] $x_{96}, \ldots, x_{99} \in[27; 30)$
				\item [] $x_{100} \in[30; 33)$.
			\end{listEX}
			Tứ phân vị thứ nhất của mẫu số liệu gốc là $\dfrac{1}{2}\left(x_{25}+x_{26}\right) \in[18; 21)$.\\
			Do đó, tứ phân vị thứ nhất của mẫu số liệu ghép nhóm là
			$$Q_1=18+\dfrac{\dfrac{100}{4}-22}{38} \cdot(21-18)=\dfrac{693}{38}.$$
			Tứ phân vị thứ ba của mẫu số liệu gốc là $\dfrac{1}{2}\left(x_{75}+x_{76}\right) \in[21; 24)$.\\
			Do đó, tứ phân vị thứ ba của mẫu số liệu ghép nhóm là
			$$Q_3=21+\dfrac{\dfrac{3 \cdot 100}{4}-(22+38)}{27} \cdot(24-21)=\dfrac{68}{3}.$$
			Vậy khoảng tứ phân vị của mẫu số liệu ghép nhóm là
			$$\Delta_Q=\dfrac{68}{3}-\dfrac{693}{38}=\dfrac{505}{114} \approx 4,43.$$
			\item Trong lần duy nhất ông Thắng đi hết $ 32 $ phút, thời gian đi của ông thuộc nhóm $[30; 33)$.\\
			Vì $Q_3+1,5 \Delta_Q=\dfrac{6683}{228} \approx 29,31<30$ nên thời gian của lần ông Thắng đi hết $ 32 $ phút là giá trị ngoại lệ của mẫu số liệu ghép nhóm.
		\end{enumEX}	
	}
\end{vd}

\begin{vd}
	\immini
	{
		Bảng bên biểu diễn mẫu số liệu ghép nhóm về chiều cao của $ 42 $ mẫu cây ở một vườn thực vật (đơn vị: centimét). Tính khoảng tứ phân vị của mẫu số liệu ghép nhóm đó (làm tròn kết quả đến hàng phần mười nếu cần).
	}
	{
		\begin{tabular}{|c|c|c|}
			\hline Nhóm & Tần số & Tần số tích luỹ\\
			\hline$[40; 45)$ & $ 5 $ & $ 5 $ \\
			{$[45; 50)$} & $ 10 $ & $ 15 $ \\
			{$[50; 55)$} & $ 7 $ & $ 22 $ \\
			{$[55; 60)$} & $ 9 $ & $ 31 $ \\
			{$[60; 65)$} & $ 7 $ & $ 38 $ \\
			{$[65; 70)$} & $ 4 $ & $ 42 $ \\
			\hline & $n=42$ & \\
			\hline
		\end{tabular}
	}
	\loigiai{
		Cỡ mẫu là $n=42$.
		\begin{itemize}
			\item Ta có: $\dfrac{n}{4}=\dfrac{42}{4}=10,5$ mà $5<10,5<15$.\\
			Suy ra nhóm $ 2 $ là nhóm đầu tiên có tần số tích luỹ lớn hơn hoặc bằng $ 10,5 $ nên nhóm $ 2 $ ( nhóm $[45; 50$) ) là chứa tứ phân vị thứ nhất. Áp dụng công thức, ta có tứ phân vị thứ nhất là
			$$Q_1=45+\left(\dfrac{10,5-5}{10}\right) \cdot 5=47,75.$$	
			\item Ta có: $\dfrac{3 n}{4}=\dfrac{3 \cdot 42}{4}=31,5$ mà $31<31,5<38$.\\
			Suy ra nhóm $ 5 $ là nhóm đầu tiên có tần số tích luỹ lớn hơn hoặc bằng $ 31,5 $ nên nhóm $ 5 $ ( nhóm $[60; 65)$)  là nhóm chứa tứ phân vị thứ ba. Áp dụng công thức, ta có tứ phân vị thứ ba là
			$$Q_3=60+\left(\dfrac{31,5-31}{7}\right) \cdot 5 \approx 60,4.$$
		\end{itemize}
		Vậy khoảng tứ phân vị của mẫu số liệu ghép nhóm đã cho là
		$$\Delta_Q=Q_3-Q_1 \approx 60,4-47,75=12,65.$$
	}
\end{vd}
\baitaptn
% \boxmini{BÀI TẬP TRẮC NGHIỆM}
% \ind{PHẦN I.} \inden{Câu trắc nghiệm nhiều phương án lựa chọn. Mỗi câu hỏi học sinh chỉ chọn một phương án.}\\
\setcounter{ex}{0}
\Opensolutionfile{ans}[ans/2D3-B1-d2-1]



\begin{ex}%[1D1B2-2]
	Khảo sát về cân nặng của các học sinh lớp 11D3 người ta được một mẫu dữ liệu ghép nhóm như sau
	\begin{center}
		\begin{tabular}{|c|c|c|c|c|c|c|}
			\hline Cân nặng & {$[30 ; 40)$} & {$[40 ; 50)$} & {$[50 ; 60)$} & {$[60 ; 70)$} & {$[70 ; 80)$} & {$[80 ; 90)$} \\
			\hline Số học sinh & $2$ & $10$ & $16$ & $8$ & $2$ & $2$ \\
			\hline
		\end{tabular}
	\end{center}
	Khoảng tứ phân vị của bảng số liệu ghép nhóm trên là
	\choice
	{$17$}
	{\True $14.5$}
	{$14$}
	{$17.5$}
	\loigiai{
		Ta có $n=40\Rightarrow\dfrac{n}{4}=10$. \\
		Gọi $x_1, \ldots, x_{40}$ là mẫu số liệu gốc về cân nặng của 40 học sinh lớp 11D3 và giả sử rằng dãy số liệu gốc này đã được sắp xếp theo thứ tự tăng dần.\\
		Tứ phân vị thứ nhất của mẫu số liệu gốc là $\dfrac{1}{2}\left( x_{10}+x_{11}\right) $ nên nhóm chứa tứ phân vị thứ nhất là nhóm $\left[40\,;\,50\right)$. Do đó tứ phân vị thứ nhất của mẫu số liệu trên là
		$$Q_1=40+\dfrac{10-2}{10}\cdot10=48.$$
		Ta có $\dfrac{3 n}{4}=30$.\\
		Tứ phân vị thứ ba của mẫu số liệu gốc là $\dfrac{1}{2}\left( x_{30}+x_{31}\right) $ nên nhóm chứa tứ phân vị thứ ba là nhóm $[60 ; 70)$. Do đó tứ phân vị thứ ba của mẫu số liệu trên là
		$$Q_3=60+\dfrac{30-28}{8} \cdot 10=62{,}5.$$
		Khoảng tứ phân vị $\Delta _Q=Q_3-Q_1=62{,}5-48=14,5$.
	}
\end{ex}

\begin{ex}%[1D1B2-2]
	Doanh thu bán hàng trong $20$ ngày được lựa chọn ngẫu nhiên của một của hàng được ghi lại ở bảng sau (đơn vị: triệu đồng)
	\begin{center}
		\begin{tabular}{|c|c|c|c|c|c|}
			\hline Doanh thu & {$[5 ; 7)$} & {$[7 ; 9)$} & {$[9 ; 11)$} & {$[11 ; 13)$} & {$[13 ; 15)$} \\
			\hline Số ngày & $2$ & $7$ & $7$ & $3$ & $1$ \\
			\hline
		\end{tabular}
	\end{center}
	Khoảng tứ phân vị của mẫu số liệu ghép nhóm này là
	\choice
	{$\dfrac{25}{7}$}
	{$\dfrac{13}{7}$}
	{ \True $\dfrac{20}{7}$}
	{$\dfrac{55}{7}$}
	\loigiai{
		Ta có $n=20$. Gọi $x_1$, $x_2$, $\ldots$, $x_{20}$ là doanh thu bán hàng trong 20 ngày xếp theo thứ tự không giảm.\\
		Khi đó 
		\begin{listEX}[3]
			\item [] $x_1$, $x_2 \in[5 ; 7)$
			\item [] $x_3, \ldots, x_9 \in[7 ; 9)$
			\item [] $x_9$, $\ldots$, $x_{16} \in[9 ; 11)$
			\item [] $x_{17}$, $\ldots$, $x_{19} \in[11 ; 13)$
			\item [] $x_{20} \in[13 ; 15)$.
		\end{listEX}
		Tứ phân vị thứ nhất của mẫu số liệu gốc là $\dfrac{1}{2}\left( x_{5}+x_{6}\right)$ nên tứ phân vị thứ nhất của mẫu số liệu thuộc nhóm $[7 ; 9)$.\\
		Tứ phân vị thứ nhất của mẫu số liệu là
				$$Q_1=7+\dfrac{\dfrac{1.20}{4}-2}{7}(9-7) =\dfrac{55}{7}.$$
		Tứ phân vị thứ ba của mẫu số liệu gốc là $\dfrac{1}{2}\left( x_{15}+x_{16}\right)$ nên tứ phân vị thứ ba của mẫu số liệu thuộc nhóm $[9 ; 11)$.\\
		Tứ phân vị thứ ba của mẫu số liệu là
		$$
		Q_3=9+\dfrac{\dfrac{3\cdot20}{4}-9}{7}(11-9) =\dfrac{75}{7}.
		$$
		Khoảng tứ phân vị $\Delta _Q=Q_3-Q_1=\dfrac{20}{7}$.
	}
\end{ex}

\begin{ex}%[1D1B2-2]
	Trung tâm ngoại ngữ thống kê bảng điểm môn Tiếng Anh của một khóa học trong bảng bên dưới
	\begin{center}
		\begin{tabular}{|l|c|c|c|c|c|}
			\hline
			Điểm     & [0;2) & [2;4) & [4;6) & [6;8) & [8;10) \\ \hline
			Học viên & 10   & 30   & 55   & 42   & 9     \\ \hline
		\end{tabular}
	\end{center}
	Khoảng tứ phân vị của mẫu số liệu ghép nhóm này là (làm tròn đến hàng phần trăm)
	\choice
	{\True $2{,}92$}
	{$2{,}93$}
	{$3{,}93$}
	{$3,92$}
	\loigiai{
		Ta có $n=146$. Gọi $x_{1}, x_{2}, ..., x_{146}$ là số liệu được sắp xếp theo thứ tự không giảm. \\
		Tứ phân vị thứ nhất của của dãy số liệu gốc là $x_{37}\in [2;4)$. Do đó, tứ phân vị thứ nhất của mẫu số liệu ghép nhóm trên là 
		$$Q_{1}=2+\dfrac{\dfrac{1.146}{4}-10}{30}.(4-2)=\dfrac{113}{30}.$$
		Tứ phân vị thứ ba của của dãy số liệu gốc là $x_{110}\in [6;8)$. Do đó, tứ phân vị thứ ba của mẫu số liệu ghép nhóm trên là \\
		$$Q_{3}=6+\dfrac{\dfrac{3.146}{4}-(10+30+55)}{42}.(8-6)=\dfrac{281}{42}$$
		Khoảng tứ phân vị $Q_3-Q_1=\dfrac{307}{105}\approx 2{,}92$.
	}
\end{ex}

\begin{ex}%[1T5K2-2]
	Thời gian luyện tập trong một ngày (tính theo giờ) của một số vận động viên được ghi lại ở bảng sau:
	\begin{center}
		\begin{tabular}{|c|c|c|c|c|c|}
			\hline 
			Thời gian luyện tập (giờ)	& $ \left[ 0 ; 2\right) $ & $ \left[ 2 ; 4\right) $ & $ \left[ 4 ; 6\right) $ & $ \left[ 6 ; 8\right) $ & $ \left[8 ; 10 \right) $ \\ 
			\hline 
			Số vận động viên	& $ 3 $ & $ 8 $ & $ 12 $ & $ 12 $ & $ 4 $ \\ 
			\hline 
		\end{tabular} 
	\end{center}
	Hãy xác định khoảng tứ phân vị của mẫu số liệu đã cho (làm tròn đến hàng phần trăm).
	\choice
	{$4{,}52$}
	{\True $3{,}35$}
	{$2{,}85$}
	{$3{,}36$}
	\loigiai{
	Số vận động viên được khảo sát là $ n=3+8+12+12+4=39$.\\
	Gọi $ x_1 $; $ x_2 $; \ldots ;$ x_{39} $ là thời gian luyện tập của $ 39 $ vận động viên được xếp theo thứ tự không giảm.	Ta có 
	\begin{enumEX}[]{3}
		\item $ x_1, x_2, x_3 \in \left[ 0 ; 2\right) $;
		\item $ x_4, \ldots, x_{11}\in \left[2 ; 4 \right) $;
		\item $ x_{12}, \ldots, x_{23}\in \left[ 4 ; 6\right) $;
		\item $ x_{24}, \ldots, x_{35}\in \left[6;8\right) $;
		\item $ x_{36},\ldots , x_{39}\in \left[ 8 ; 10\right) $.
	\end{enumEX}
	\begin{itemize}
		\item Tứ phân vị thứ nhất là $ x_{10} $ thuộc nhóm $\left[ 2 ; 4\right)$;
		\item Tứ phân vị thứ ba là $ x_{30} $ thuộc nhóm $ \left[ 6 ; 8\right) $.
	\end{itemize}
	Tứ phân vị thứ nhất của mẫu số liệu ghép nhóm là $$Q_1=2+\dfrac{\dfrac{1\cdot 39}{4}-3}{8}\cdot(4-2)=\dfrac{59}{16}.$$\\
	Tứ phân vị thứ ba của mẫu số liệu ghép nhóm là $$Q_3=6+\dfrac{\dfrac{3\cdot 39}{4}-(3+8+12)}{12}\cdot(8-6)=\dfrac{169}{24}.$$ 
	Khoảng tứ phân vị $Q_3-Q_1=\dfrac{161}{48}\approx 3{,}35$.
	}
\end{ex}

\begin{ex}
	Ở một phòng điều trị nội trú của bệnh viện, dữ liệu thống kê thời gian ngủ hằng đêm của một bệnh nhân trong suốt một tháng được tổng hợp bởi bảng dưới đây
	\begin{center}
		\begin{tabular}{|c|c|c|}
			\hline Thời gian (phút) & Tần số & \begin{tabular}{c} 
				Tần số \\
				tích luỹ
			\end{tabular} \\
			\hline$[180 ; 240)$ & $2$ & $2$ \\
			\hline$[240 ; 300)$ & $9$ & $11$ \\
			\hline$[300 ; 360)$ & $12$ & $23$\\
			\hline$[360 ; 420)$ & $5$ & $28$ \\
			\hline$[420 ; 480)$ & $2$ & $30$ \\
			\hline
		\end{tabular}
	\end{center}
\choice
{$75{,}53$}
{$84{,}83$}
{\True $80{,}83$}
{$72{,}53$}
\loigiai{
	Kích thước mẫu $n=30$. Ta có $\dfrac{n}{4}=\dfrac{15}{2}=7{,}5 ;\, \dfrac{3 n}{4}=\dfrac{45}{2}=22{,}5$.	\\
\begin{itemize}
	\item [$\bullet$] Nhóm chứa $Q_1$ là $[240 ; 300)$. Suy ra
	$$Q_1=240+\dfrac{7{,}5 -2}{9} \cdot 60 =\dfrac{830}{3}.$$
	\item [$\bullet$] Nhóm chứa $Q_3$ là $[300 ; 360)$.Suy ra
	$$Q_3=300+\dfrac{22{,}5 -11}{12} \cdot 60=357{,}5$$
\end{itemize}
	Vậy $\Delta_Q=357{,}5-\dfrac{830}{3}\approx 80{,}83$.
	}
\end{ex}

\begin{ex}%[2D3H1-3]
	Biểu đồ dưới đây biểu diễn số lượt khách hàng đặt bàn qua hình thức trực tuyến mỗi ngày trong quý III năm 2022 của một nhà hàng. Cột thứ nhất biểu diễn số ngày có từ $1$ đến dưới $6$ lượt đặt bàn; cột thứ hai biểu diễn số ngày có từ $6$ đến dưới $11$ lượt đặt bàn;\ldots.
	\begin{center}
		\begin{tikzpicture}[font=\small, line join=round, line cap=round, >=stealth,x=0.25cm,y=0.6cm]
			\draw[->](0,0)--(0,8)node[left]{\textbf{Số ngày}};
			\draw[->](0,0)--(35,0)node[below]{\textbf{Số lượt đặt bàn}};
			\foreach \i in{1,...,7} \pgfmathsetmacro{\gti}{int(5*(\i))}
			\draw [dotted](0,\i) circle(1pt)node[left]{$\gti$} -- (30,\i);
			\foreach \i/\a/\b in {1/15/20,2/20/25,3/25/30,4/30/35,5/35/40}
			\foreach \i/\j in {1/14,6/30,11/25,16/18,21/5}
			{
				\draw[fill=cyan!50](\i,0)rectangle(\i+5,\j/5);
				\draw (\i,0) node [below] {$\i$};
				\draw (\i+2.5,\j/5) node [above] {$\j$};
			}
			\draw (26,0) node [below] {$26$};
		\end{tikzpicture}
	\end{center}
	Hãy tìm khoảng tứ phân vị của mẫu số liệu ghép nhóm cho bởi biểu đồ trên.
	\choice
	{$9{,}5$}
	{\True $8{,}5$}
	{$10{,}5$}
	{$7{,}5$}
	\loigiai{Dựa vào biểu đồ, ta lập được bảng ghép nhóm như bên dưới.
		\begin{center}
			\begin{tabular}{|c|c|c|c|c|c|}
				\hline
				Lượt đặt bàn & $[1;6)$ & $[6;11)$ & $[11;16)$ & $[16;21)$ & $[21;26)$ \\
				\hline
				Số ngày & $14$ & $30$ & $25$ & $18$ & $5$ \\
				\hline
			\end{tabular}
		\end{center}
		Ta có cỡ mẫu $n=92$.\\
		Gọi $x_1$; $x_2$; \ldots; $x_{92}$ là mẫu số liệu đã cho.\\
		Ta có: 
		\begin{enumEX}[]{3}
			\item $x_1$, \ldots, $x_{14}\in[1;6)$; 
			\item $x_{15}$, \ldots, $x_{44}\in[6;11)$; 
			\item $x_{45}$, \ldots, $x_{69}\in[11;16)$;
			\item $x_{70}$, \ldots, $x_{87}\in[16;21)$;
			\item $x_{88}$, \ldots, $x_{92}\in[21;26)$.
		\end{enumEX} 
		Tứ phân vị thứ nhất của mẫu số liệu là $\dfrac{x_{23}+x_{24}}{2}\in[6;11)$. Do đó, tứ phân vị thứ nhất của mẫu số liệu là
		$$Q_1=6+\dfrac{\dfrac{92}{4}-14}{30}\cdot(11-6)=7{,}5.$$
		Tứ phân vị thứ ba của mẫu số liệu là $\dfrac{x_{69}+x_{70}}{2}$ với $x_{69}\in[11;16)$ và $x_{70}\in[16;21)$. Do đó, tứ phân vị thứ ba của mẫu số liệu là $Q_3=16$.\\
		%Đoạn này sách giáo khoa 12 không đề cập, tôi lấy từ kiến thức của sách CTST lớp 11.
		Vậy khoảng tứ phân vị của mẫu số liệu là $\Delta_Q=Q_3-Q_1=8{,}5$.}
\end{ex}

\Closesolutionfile{ans}

% \ind{PHẦN II.} \inden{Câu trắc nghiệm đúng sai. Trong mỗi ý a), b), c), d) ở mỗi câu, học sinh chọn đúng hoặc sai.}\\
\Opensolutionfile{ans}[ans/2D3-B1-d2-2]

\begin{ex}%[2D3H1-3]
	Kết quả đo chiều cao của $100$ cây keo 3 năm tuổi tại một nông trường được cho ở bảng sau
	\begin{center}
		\begin{tabular}{|c|c|c|c|c|c|}
			\hline
			Chiều cao (m) & $[8{,}4;8{,}6)$ & $[8{,}6;8{,}8)$ & $[8{,}8;9{,}0)$ & $[9{,}0;9{,}2)$ & $[9{,}2;9{,}4)$ \\
			\hline
			Số cây & $5$ & $12$ & $25$ & $44$ & $14$ \\
			\hline
		\end{tabular}
	\end{center}
	\choiceTF
	{\True Khoảng biến thiên của mẫu số liệu này là $R=1$}
	{Tứ phân vị thứ nhất của mẫu số liệu là $Q_1=8$}
	{\True Khoảng tứ phân vị của mẫu số liệu là $\Delta Q=0{,}286$}
	{\True Biết rằng trong $100$ cây keo trên có $1$ cây cao $8{,}4$ m. Chiều cao của cây keo này là giá trị ngoại lệ}
	\loigiai{\begin{enumerate}
			\item Khoảng biến thiên của mẫu số liệu là $R=9{,}4-8{,}4=1$.
			\item
			Ta có cỡ mẫu $n=100$.\\
			Gọi $x_1$; $x_2$; \ldots; $x_{100}$ là mẫu số liệu gồm chiều cao của $100$ cây keo.\\
			Ta có: 
			\begin{enumEX}[]{3}
				\item $x_1$, \ldots, $x_5\in[8{,}4;8{,}6)$; 
				\item $x_6$, \ldots, $x_{17}\in[8{,}6;8{,}8)$;
				\item $x_{18}$, \ldots, $x_{42}\in[8{,}8;9{,}0)$; 
				\item $x_{43}$, \ldots, $x_{86}\in[9{,}0;9{,}2)$; 
				\item $x_{87}$, \ldots, $x_{100}\in[9{,}2;9{,}4)$.
			\end{enumEX}
			Tứ phân vị thứ nhất của mẫu số liệu là $\dfrac{x_{25}+x_{26}}{2}\in[8{,}8;9{,}0)$. Do đó, tứ phân vị thứ nhất của mẫu số liệu ghép nhóm là
			$$Q_1=8{,}8+\dfrac{\dfrac{100}{4}-(5+12)}{25}\cdot(9{,}0-8{,}8)=8{,}864.$$
			\item	Tứ phân vị thứ ba của mẫu số liệu là $\dfrac{x_{75}+x_{76}}{2}\in[9{,}0;9{,}2)$. Do đó, tứ phân vị thứ ba của mẫu số liệu ghép nhóm là
			$$Q_3=9{,}0+\dfrac{\dfrac{3\cdot100}{4}-(5+12+25)}{44}\cdot(9{,}2-9{,}0)=9{,}15.$$
			Vậy khoảng tứ phân vị của mẫu số liệu ghép nhóm là $\Delta_Q=Q_3-Q_1=0{,}286$.
			\item Vì $Q_1-1{,}5\Delta_Q=8{,}435$ và $Q_3+1{,}5\Delta_Q=9{,}579$ nên cây keo có chiều cao $8{,}4$ m là giá trị ngoại lệ của mẫu số liệu ghép nhóm.
	\end{enumerate}}
\end{ex}

\begin{ex}
	\immini{Bảng bên biểu diễn mẫu số liệu ghép nhóm thống kê mức lương của một công ty (đơn vị: triệu đồng).
		\choiceTF
		{Khoảng biến thiên của mẫu số liệu này là $R=25$}
		{\True Tứ phân vị thứ nhất của mẫu số liệu là $Q_1=15$}
		{Tứ phân vị thứ ba của mẫu số liệu là $Q_3=27$}
		{Khoảng tứ phân vị của mẫu số liệu là $\Delta Q=12$}
	}{\begin{tabular}{|c|c|}
			\hline Nhóm & Tần số \\
			\hline$[10 ; 15)$ & $15$ \\
			{$[15 ; 20)$} & $18$ \\
			{$[20 ; 25)$} & $10$ \\
			{$[25 ; 30)$} & $10$ \\
			{$[30 ; 35)$} & $5$ \\
			{$[35 ; 40)$} & $2$ \\
			\hline & $n=60$ \\
			\hline
	\end{tabular}}
	\loigiai{
		\begin{enumerate}
			\item Trong mẫu số liệu ghép nhóm ở bảng, ta có đầu mút trái của nhóm $1$ là $a_1=10$, đầu mút phải của nhóm $6$ là $a_7=40$.\\Vậy khoảng biến thiên của mẫu số liệu ghép nhóm đó là $R=a_7-a_1=40-10=30.$
			\item Ta có bảng sau
			\begin{center}
				\begin{tabular}{|c|c|c|}
					\hline Nhóm & Tần số & Tần số tích luỹ\\
					\hline$[10 ; 15)$ & $15$ & $15$\\
					{$[15 ; 20)$} & $18$ & $33$\\
					{$[20 ; 25)$} & $10$ & $43$\\
					{$[25 ; 30)$} & $10$ & $53$\\
					{$[30 ; 35)$} & $5$ & $58$\\
					{$[35 ; 40)$} & $2$ & $60$\\
					\hline & $n=60$ & \\
					\hline
				\end{tabular}
			\end{center}
			Số phần tử của mẫu là $n=60$. \\
			Nhóm $[15;20)$ là nhóm chứa tứ phân vị thứ nhất. 
			Áp dụng công thức, ta có tứ phân vị thứ nhất là $$Q_1=15+\left(\dfrac{15-15}{18}\right)\cdot 5=15 ~\text{(triệu đồng)}.$$
			\item Nhóm $[25;30)$ là nhóm chứa tứ phân vị thứ 3. Áp dụng công thức, ta có tứ phân vị thứ ba là
			$$Q_3=25+\left(\dfrac{45-43}{10}\right)\cdot5=26 ~\text{(triệu đồng)}.$$
			\item  Khoảng tứ phân vị của mẫu số liệu ghép nhóm đã cho là 
			$$\Delta _Q=Q_3-Q_1=26-15=11 ~\text{(triệu đồng)}.$$
		\end{enumerate}
	}
\end{ex}

\begin{ex}
	Điều tra một số hộ gia đình thu nhập ở mức trung bình sinh sống trên hai địa bàn $A$, $B$, người ta thấy diện tích nhà ở của họ đều nhỏ hơn $100$ m$^2$. Hai biểu đồ dưới biểu diễn kết quả thống kê. 
	\begin{center}
		\begin{tikzpicture}[>=stealth,scale=1]
			%========================
			\draw[opacity=.25,thin,step=.2,cyan](0,0) grid(7,3);
			\draw[opacity=.5,cyan](0,0) grid (7,3);
			\draw[stealth-stealth](0,3) node[left]{Tần số}|-(7,0)node[below]{m$^2$};
			\foreach\x/\dientich[count=\i from 1] in {0/50,.4/60,1/70,2.5/80,.9/90,.2/100}{
				\draw[fill=gray](\i-1,0) rectangle +(1,\x);
				\draw (\i,0) node[below]{\dientich};
			}
			\foreach \y [count=\i from 1] in {10,20,30,40,50}{
				\draw (-.1,\i/2)--(.1,\i/2)(0,\i/2) node[left]{$\y$};}
			\foreach \z [count=\i from 1] in {8,20,50,18,4}{
				\draw (\i+0.5,\z/20) node[above] {$\z$};}
		\end{tikzpicture}\\	
		\textit{Hình a. Diện tích nhà ở của cư dân địa bàn $A$}
		%========================
	\end{center}
	\begin{center}
		\begin{tikzpicture}[>=stealth,scale=1]
			%========================
			\draw[opacity=.25,thin,step=.2,cyan](0,0) grid(7,3);
			\draw[opacity=.5,cyan](0,0) grid (7,3);
			\draw[stealth-stealth](0,3) node[left]{Tần số}|-(7,0)node[below]{m$^2$};
			\foreach\x/\dientich[count=\i from 1] in {0/50,.75/60,1/70,1.5/80,1/90,.75/100}{
				\draw[fill=gray](\i-1,0) rectangle +(1,\x);
				\draw (\i,0) node[below]{\dientich};
			}
			\foreach \y [count=\i from 1] in {10,20,30,40,50}{
				\draw (-.1,\i/2)--(.1,\i/2)(0,\i/2) node[left]{$\y$};}
			\foreach \z [count=\i from 1] in {15,20,30,20,15}{
				\draw (\i+0.5,\z/20-0.05) node[above] {$\z$};}
			%========================
		\end{tikzpicture}\\
		\textit{Hình b. Diện tích nhà ở của cư dân địa bàn $B$}
	\end{center}
	\choiceTF
	{\True Khoảng biến thiên của hai mẫu số liệu này bằng nhau}
	{\True Khoảng tứ phân vị ghép nhóm diện tích căn hộ của địa phương A là $10{,}9$}
	{Khoảng tứ phân vị ghép nhóm diện tích căn hộ của địa phương B là $8{,}5$.}
	{Số liệu về diện tích nhà ở của cư dân thuộc địa bàn A phân tán hơn địa bàn B}
	\loigiai{
		Ta có bảng tần số tích luỹ như sau:
		\begin{center}
			\begin{tabular}{|c|c|c|c|c|c|}
				\hline \begin{tabular}{c}
					Diện tích nhà ở \\
					Địa bàn $A$ (m$^2$) 
				\end{tabular} & Tần số  & \begin{tabular}{c}
					Tần số \\
					tích luỹ 
				\end{tabular}  & \begin{tabular}{c}
					Diện tích nhà ở \\
					Địa bàn $B$ (m$^2$) 
				\end{tabular} & Tần số & \begin{tabular}{c}
					Tần số  \\
					tích luỹ 
				\end{tabular}   \\
				\hline$[50 ; 60)$ & $8$ & $8$& $[50 ; 60)$ & $15$& $15$ \\
				\hline$[60 ; 70)$ & $20$ &$28$&  $[60 ; 70)$ & $20$& $35$ \\
				\hline$[70 ; 80)$ & $50$ &$78$&  $[70 ; 80)$ & $30$& $65$ \\
				\hline$[80 ; 90)$ & $18$ &$96$&  $[80 ; 90)$ & $20$& $85$ \\
				\hline$[90 ; 100)$ & $4$ &$100$&  $[90 ; 100)$ & $15$& $100$ \\
				\hline
			\end{tabular}
		\end{center}
		\begin{enumerate}[a)]
			\item Khoảng biến thiên của hai mẫu số liệu này bằng nhau và bằng $100=50=50$.
			\item Xét bảng số liệu $A$, ta có $N=100; \dfrac{N}{4}=25; \dfrac{N}{2}=50; \dfrac{3N}{4}=75$.
			\begin{itemize}
				\item [$\bullet$] Nhóm chứa $Q_1^A$ là $[60 ; 70)$. Suy ra
						$$Q_1^A=60+\dfrac{25-8}{20} \cdot 10 = 68,5 $$
				\item [$\bullet$] Nhóm chứa $Q_3^A$ là $[70;80)$. Suy ra
						$$Q_3^A=70+\dfrac{75 -28}{50} \cdot 10=79{,}4$$
			\end{itemize}
		Vậy khoảng tứ phân vị ghép nhóm diện tích căn hộ của địa phương A là\\ $\Delta_{Q_A} =79{,}4-68{,}5=10{,}9$. 
			\item  Xét bảng số liệu $B$, ta có $N=100; \dfrac{N}{4}=25; \dfrac{N}{2}=50; \dfrac{3N}{4}=75$.
			\begin{itemize}
				\item [$\bullet$] Nhóm chứa $Q_1^B$ là $[60 ; 70)$. Suy ra
						$$Q_1^B=60+\dfrac{25 -15}{20} \cdot 10=65.$$
				\item [$\bullet$] Nhóm chứa $Q_3^B$ là $[80;90)$.Suy ra
					$$Q_3^B=80+\dfrac{75 -65}{20} \cdot 10= 85.$$
			\end{itemize}
			Vậy khoảng tứ phân vị  ghép nhóm diện tích căn hộ của địa phương B là là $\Delta_{Q_B} =85-65=20$. 
			\item $\Delta_{Q_B}>\Delta_{Q_A}$ nên dựa vào khoảng tứ phân vị về diện tích căn hộ người dân hai địa phương, ta thấy địa phương B phân tán hơn.
		\end{enumerate}
	}
\end{ex}

\begin{ex}%[2D3H1-3]
	Bảng tần số ghép nhóm dưới đây thể hiện kết quả điều tra về tuổi thọ trung bình của nam giới và nữ giới ở $50$ quốc gia.
	\begin{center}
		\begin{tabular}{|c|c|c|}
			\hline
			\diagbox{Nhóm (Tuổi thọ)}{Giới tính} & Nam & Nữ \\
			\hline
			$[50;55)$ & $4$ & $3$ \\
			\hline
			$[55;60)$ & $7$ & $4$ \\
			\hline
			$[60;65)$ & $4$ & $5$ \\
			\hline
			$[65;70)$ & $6$ & $3$ \\
			\hline
			$[70;75)$ & $15$ & $7$ \\
			\hline
			$[75;80)$ & $12$ & $14$ \\
			\hline
			$[80;85)$ & $2$ & $13$ \\
			\hline
			$[85;90)$ & $0$ & $1$ \\
			\hline	
		\end{tabular}
	\end{center}
	\choiceTF
	{Khoảng biến thiên của mẫu số liệu về độ tuổi trung bình của nam giới là $50$}
	{Khoảng tứ phân vị của mẫu số liệu về độ tuổi trung bình của nam giới là $14{,}75$}
	{Khoảng tứ phân vị của mẫu số liệu về độ tuổi trung bình của nữ giới là $15$}
	{\True Dựa vào khoảng tứ phân vị thì tuổi thọ trung bình của nam giới đều hơn tuổi thọ trung bình của nữ giới}
	\loigiai{
		\begin{enumerate}
			\item Khoảng biến thiên của mẫu số liệu về độ tuổi trung bình của nam giới là $90-50=40$.
			\item Xét ở nam giới, ta có cỡ mẫu $n=50$.\\
			Gọi $x_1$; $x_2$; \ldots; $x_{50}$ là mẫu số liệu gồm tuổi thọ của $50$ nam giới.\\
			Ta có: $x_1$, \ldots, $x_4\in[50;55)$; $x_5$, \ldots, $x_{11}\in[55;60)$; $x_{12}$, \ldots, $x_{15}\in[60;65)$; $x_{16}$, \ldots, $x_{21}\in[65;70)$; $x_{22}$, \ldots, $x_{36}\in[70;75)$; $x_{37}$, \ldots, $x_{48}\in[75;80)$; $x_{49}$, $x_{50}\in[80;85)$.\\
			Tứ phân vị thứ nhất của mẫu số liệu là $x_{13}\in[60;65)$. Do đó, tứ phân vị thứ nhất của mẫu số liệu nam giới là
			$$Q_1=60+\dfrac{\dfrac{50}{4}-(4+7)}{4}\cdot(65-60)=\dfrac{495}{8}.$$
			Tứ phân vị thứ ba của mẫu số liệu là $x_{38}\in[75;80)$. Do đó, tứ phân vị thứ ba của mẫu số liệu nam giới là
			$$Q_3=75+\dfrac{\dfrac{3\cdot50}{4}-(4+7+4+6+15)}{12}\cdot(80-75)=\dfrac{605}{8}.$$
			Vậy khoảng tứ phân vị của mẫu số liệu nam giới là $\Delta_Q=Q_3-Q_1=\dfrac{55}{4}=13{,}75$.
			\item Xét ở nữ giới, ta có cỡ mẫu $n=50$.\\
			Gọi $x_1$; $x_2$; \ldots; $x_{50}$ là mẫu số liệu gồm tuổi thọ của $50$ nữ giới.\\
			Ta có: $x_1$, $x_2$, $x_3\in[50;55)$; $x_4$, \ldots, $x_7\in[55;60)$; $x_8$, \ldots, $x_{12}\in[60;65)$; $x_{13}$, $x_{14}$, $x_{15}\in[65;70)$; $x_{16}$, \ldots, $x_{22}\in[70;75)$; $x_{23}$, \ldots, $x_{36}\in[75;80)$; $x_{37}$, \ldots, $x_{49}\in[80;85)$; $x_{50}\in[85;90)$.\\
			Tứ phân vị thứ nhất của mẫu số liệu là $x_{13}\in[65;70)$. Do đó, tứ phân vị thứ nhất của mẫu số liệu nữ giới là
			$$Q_1=65+\dfrac{\dfrac{50}{4}-(3+4+5)}{3}\cdot(70-65)=\dfrac{395}{6}.$$
			Tứ phân vị thứ ba của mẫu số liệu là $x_{38}\in[80;85)$. Do đó, tứ phân vị thứ ba của mẫu số liệu nữ giới là
			$$Q_3=80+\dfrac{\dfrac{3\cdot50}{4}-(3+4+5+3+7+14)}{13}\cdot(85-80)=\dfrac{2095}{26}.$$
			Vậy khoảng tứ phân vị của mẫu số liệu nữ giới là $\Delta_Q=Q_3-Q_1=\dfrac{575}{39}\approx14{,}74$.
			\item Do khoảng tứ phân vị của mẫu số liệu của nam giới nhỏ hơn mẫu số liệu của nữ giới nên tuổi thọ của nam giới đều hơn tuổi thọ của nữ giới.
	\end{enumerate}}
\end{ex}


\Closesolutionfile{ans}

%%Bài 2
% \setcounter{section}{1}
\setcounter{dang}{0}
\section{PHƯƠNG SAI VÀ ĐỘ LỆCH CHUẨN CỦA MSL GHÉP NHÓM}
\subsection{LÝ THUYẾT CẦN NHỚ}
Xét mẫu số liệu ghép nhóm cho bởi bảng sau:
\begin{center}
	\begin{tabular}{|c|c|c|c|c|}
		\hline Nhóm             & {$\left[u_1; u_2\right)$} & {$\left[u_2; u_3\right)$} & $\ldots$ & {$\left[u_k; u_{k+1}\right)$} \\
		\hline Giá trị đại diện & $c_1$                     & $c_2$                     & $\ldots$ & $c_k$                         \\
		\hline Tần số           & $n_1$                     & $n_2$                     & $\ldots$ & $n_k$                         \\
		\hline
	\end{tabular}
\end{center}
\begin{enumerate}[\iconMT]
	\item \indam{Phương sai:} Phuơng sai của mẫu số liệu ghép nhóm, kí hiệu $S^2$, được tính bởi công thức
	      \begin{align*}
			S^2&=\dfrac{1}{n}\left[n_1\left(c_1-\bar{x}\right)^2+n_2\left(c_2-\bar{x}\right)^2+\cdots+n_k\left(c_k-\bar{x}\right)^2\right]\\
		  &=\dfrac{1}{n}\left(n_1 c_1^2+n_2 c_2^2+\cdots+n_k c_k^2\right)-\overline{x}^2
		  \end{align*}
		  
	      trong đó: $n=n_1+n_2+\cdots+n_k$ là cỡ mẫu; $\bar{x}=\dfrac{1}{n}\left(n_1 c_1+n_2 c_2+\cdots+n_k c_k\right)$ là số trung bình.
	\item \indam{Độ lệch chuẩn:} Độ lệch chuẩn của mẫu số liệu ghép nhóm, kí hiệu $S$, là căn bậc hai số học của phương sai, nghĩa là $S=\sqrt{S^2}$.
	% \item \indam{Ý nghĩa:}
	%       \begin{listEX}[1]
	% 	    %   \item [\iconCH] Phương sai (độ lệch chuẩn) của mẫu số liệu ghép nhóm là giá trị xấp xỉ cho phương sai (độ lệch chuẩn) của mẫu số liệu gốc. Chúng được dùng để đo mức độ phân tán của mẫu số liệu ghép nhóm xung quanh số trung bình của mẫu số liệu. Phương sai và độ lệch chuẩn càng lớn thì dữ liệu càng phân tán.
	% 	      \item [\iconCH] Độ lệch chuẩn có cùng đơn vị với đơn vị của mẫu số liệu.
	%       \end{listEX}
\end{enumerate}

\subsection{PHÂN LOẠI VÀ PHƯƠNG PHÁP GIẢI TOÁN}
% \begin{dang}{Tính trung bình cộng của mẫu số liệu ghép nhóm}
% 	Xét mẫu số liệu ghép nhóm cho bởi bảng sau:
% 	\begin{center}
% 		\begin{tabular}{|c|c|c|c|c|}
% 			\hline Nhóm             & {$\left[u_1; u_2\right)$} & {$\left[u_2; u_3\right)$} & $\ldots$ & {$\left[u_k; u_{k+1}\right)$} \\
% 			\hline Giá trị đại diện & $c_1$                     & $c_2$                     & $\ldots$ & $c_k$                         \\
% 			\hline Tần số           & $n_1$                     & $n_2$                     & $\ldots$ & $n_k$                         \\
% 			\hline
% 		\end{tabular}
% 	\end{center}
% 	Số trung bình cộng của mẫu số liệu ghép nhóm trên được tính bằng công thức
% 	\boxmini{$\bar{x}=\dfrac{1}{n}\left(n_1 c_1+n_2 c_2+\cdots+n_k c_k\right)$}
% \end{dang}
% \boxmini{BÀI TẬP TỰ LUẬN}
% \begin{vd}%[1K3B9-1] 
% 	Tìm cân nặng trung bình của học sinh lớp $11D$ cho trong bảng sau:
% 	\begin{center}
% 		\begin{tabular}{|c|c|c|c|c|c|c|}
% 			\hline
% 			Cân nặng    & $\left[40{,}5;45{,}5 \right)$ & $\left[45{,}5;50{,}5 \right)$ & $\left[50{,}5;55{,}5 \right)$ & $\left[55{,}5;60{,}5 \right)$ & $\left[60{,}5;65{,}5 \right)$ & $\left[65{,}5;70{,}5 \right)$ \\
% 			\hline
% 			Số học sinh & $10$                          & $7$                           & $16$                          & $4$                           & $2$                           & $3$                           \\
% 			\hline
% 		\end{tabular}
% 	\end{center}
% 	\loigiai{
% 		Trong mỗi khoảng cân nặng, giá trị đại diện là trung bình cộng của hai giá trị đầu mút nên ta có bảng sau:
% 		\begin{center}
% 			\begin{tabular}{|c|c|c|c|c|c|c|}
% 				\hline
% 				Cân nặng (kg) & $43$ & $48$ & $53$ & $58$ & $63$ & $68$ \\
% 				\hline
% 				Số học sinh   & $10$ & $7$  & $16$ & $4$  & $2$  & $3$  \\
% 				\hline
% 			\end{tabular}
% 		\end{center}
% 		Tổng số học sinh là $n=42$. Cân nặng trung bình của học sinh lớp $11D$ là $$\overline{x}=\dfrac{10\cdot 43+7\cdot 48+16\cdot 53+4\cdot 58+2\cdot 63+3\cdot 68}{42}\approx51{,}81\,\mathrm{(kg)}.$$
% 	}
% \end{vd}

% \begin{vd}%[1T5B1-2]
% 	Kết quả khảo sát cân nặng của $25$ quả cam ở mỗi lô hàng $A$ và $B$ được cho ở bảng sau:
% 	\begin{center}
% 		\begin{tabular}{|c|c|c|c|c|c|}
% 			\hline \multicolumn{1}{|c|}{Cân nặng $(\mathrm{g})$} & {$[150; 155)$} & {$[155; 160)$} & {$[160; 165)$} & {$[165; 170)$} & {$[170; 175)$} \\
% 			\hline Số quả cam ở lô hàng $A$                      & 2              & 6              & 12             & 4              & 1              \\
% 			\hline Số quả cam ở lô hàng $B$                      & 1              & 3              & 7              & 10             & 4              \\
% 			\hline
% 		\end{tabular}
% 	\end{center}
% 	\begin{enumerate}
% 		\item Hãy ước lượng cân nặng trung bình của mỗi quả cam ở lô hàng $A$ và lô hàng $B$.
% 		\item Nếu so sánh theo số trung bình thì cam ở lô hàng nào nặng hơn?
% 	\end{enumerate}
% 	\loigiai{
% 		Ta có bảng thống kê số lượng cam theo giá trị đại diện:
% 		\begin{center}
% 			\begin{tabular}{|c|c|c|c|c|c|}
% 				\hline \multicolumn{1}{|c|}{Cân nặng $(\mathrm{g})$} & {$152{,}5$} & {$157{,}5$} & {$162{,}5$} & {$167{,}5$} & $172{,}5$ \\
% 				\hline Số quả cam ở lô hàng $A$                      & 2           & 6           & 12          & 4           & 1         \\
% 				\hline Số quả cam ở lô hàng $B$                      & 1           & 3           & 7           & 10          & 4         \\
% 				\hline
% 			\end{tabular}
% 		\end{center}
% 		\begin{enumerate}
% 			\item Cân nặng trung bình của mỗi quả cam ở lô hàng $A$ xấp xỉ bằng
% 			      \[(2\cdot 152{,}5+6\cdot 157{,}5+12\cdot 162{,}5+4\cdot 167{,}5+1\cdot 172{,}5): 25=161{,}7\ (\mathrm{g}). \]
% 			      Cân nặng trung bình của mỗi quả cam ở lô hàng $B$ xấp xỉ bằng
% 			      \[(1\cdot 152{,}5+3\cdot 157{,}5+7\cdot 162{,}5+10\cdot 167{,}5+4\cdot 172{,}5): 25=165{,}1\ (\mathrm{g}). \]
% 			\item Nếu so sánh theo số trung bình thì cam ở lô hàng $B$ nặng hơn cam ở lô hàng $A$.
% 		\end{enumerate}
% 	}
% \end{vd}

% \boxmini{BÀI TẬP TRẮC NGHIỆM}
% \Opensolutionfile{ans}[ans/2D3-B2-d1]
% \begin{ex}%%[1D1Y1-2]
% 	Cho mẫu số liệu với cỡ mẫu $n$ được cho dưới bảng tần số ghép nhóm
% 	\begin{center}
% 		\begin{tabular}{|c|c|c|c|c|}
% 			\hline Nhóm             & {$\left[u_1 ; u_2\right)$} & {$\left[u_2 ; u_3\right)$} & $\ldots$ & {$\left[u_k ; u_{k+1}\right)$} \\
% 			\hline Giá trị đại diện & $c_1$                      & $c_2$                      & $\ldots$ & $c_k$                          \\
% 			\hline Tần số           & $n_1$                      & $n_2$                      & $\ldots$ & $n_k$                          \\
% 			\hline
% 		\end{tabular}
% 	\end{center}
% 	Số trung bình $\overline x $ của mẫu số liệu trên được tính bằng công thức nào sau đây
% 	\choice
% 	{$\overline x=\dfrac{u_1+u_2+\ldots+u_k}{n}$}
% 	{$\overline x=\dfrac{c_1+c_2+\ldots+c_k}{n}$}
% 	{$\overline x=\dfrac{n_1u_1+n_2u_2+\ldots+n_k{u_k}}{n}$}
% 	{\True $\overline x=\dfrac{n_1c_1+n_2c_2+\ldots+n_k{c_k}}{n}$}
% 	\loigiai{}
% \end{ex}

% \begin{ex}%[1D1B1-2]
% 	Khảo sát về cân nặng của các học sinh lớp $11D3$ người ta được một mẫu dữ liệu ghép nhóm như sau:
% 	\begin{center}
% 		\begin{tabular}{|c|c|c|c|c|c|c|}
% 			\hline Cân nặng    & {$[30 ; 40)$} & {$[40 ; 50)$} & {$[50 ; 60)$} & {$[60 ; 70)$} & {$[70 ; 80)$} & {$[80 ; 90)$} \\
% 			\hline Số học sinh & $2$           & $10$          & $16$          & $8$           & $2$           & $2$           \\
% 			\hline
% 		\end{tabular}
% 	\end{center}
% 	Số trung bình của mẫu số liệu trên là
% 	\choice
% 	{\True $56$}
% 	{$50$}
% 	{$60$}
% 	{$55$}
% 	\loigiai{
% 		Ta có: Số phần tử của mẫu là $n=40$ và
% 		\begin{center}
% 			\begin{tabular}{|c|c|c|c|c|c|c|}
% 				\hline Cân nặng         & {$[30 ; 40)$} & {$[40 ; 50)$} & {$[50 ; 60)$} & {$[60 ; 70)$} & {$[70 ; 80)$} & {$[80 ; 90)$} \\
% 				\hline Giá trị đại diện & $35$          & $45$          & $55$          & $65$          & $75$          & $85$          \\
% 				\hline Số học sinh      & $2$           & $10$          & $16$          & $8$           & $2$           & $2$           \\
% 				\hline
% 			\end{tabular}
% 		\end{center}
% 		Do đó giá trị trung bình của mẫu số liệu trên là\\
% 		$\overline x=\dfrac{35\cdot2+45\cdot10+55\cdot16+65\cdot8+75\cdot2+85\cdot2}{40}=56$.}
% \end{ex}

% \begin{ex}%[1D1B1-2]
% 	Thống kê về thời lượng mỗi trận đấu bi-a trong vòng tứ kết giải đấu European Open người ta được mẫu số liệu ghép nhóm như sau
% 	\begin{center}
% 		\begin{tabular}{|c|c|c|c|c|c|}
% 			\hline Thời gian & {$[9{,}5 ; 12{,}5)$} & {$[12{,}5 ; 15{,}5)$} & {$[15{,}5 ; 18{,}5)$} & {$[18{,}5 ; 21{,}5)$} & {$[21{,}5 ; 24{,}5)$} \\
% 			\hline Số trận   & $3$                  & $12$                  & $15$                  & $24$                  & $2$                   \\
% 			\hline
% 		\end{tabular}
% 	\end{center}
% 	Số trung bình của mẫu số liệu trên gần nhất với giá trị nào sau đây
% 	\choice{$17$}
% 	{\True $17{,}5$}
% 	{$18$}
% 	{$18{,}5$}
% 	\loigiai{
% 		Ta có số phần tử của mẫu là $n=56$ và
% 		\begin{center}
% 			\begin{tabular}{|c|c|c|c|c|c|}
% 				\hline Thời gian        & {$[9{,}5 ; 12{,}5)$} & {$[12{,}5 ; 15{,}5)$} & {$[15{,}5 ; 18{,}5)$} & {$[18{,}5 ; 21{,}5)$} & {$[21{,}5 ; 24{,}5)$} \\
% 				\hline Giá trị đại diện & $11$                 & $14$                  & $17$                  & $20$                  & $2$3                  \\
% 				\hline Số trận          & $3$                  & $12$                  & $15$                  & $24$                  & $2$                   \\
% 				\hline
% 			\end{tabular}
% 		\end{center}
% 		Do đó giá trị trung bình của mẫu số liệu trên là
% 		$$
% 			\overline{x}=\dfrac{11\cdot3+14\cdot12+17\cdot15+20\cdot24+23\cdot2}{56}=\dfrac{491}{28} \approx 17{,}54.$$
% 	}
% \end{ex}

% \begin{ex}%[1D1B1-2]
% 	Doanh thu bán hàng trong $20$ ngày được lựa chọn ngẫu nhiên của một của hàng được ghi lại ở bảng sau (đơn vị: triệu đồng)
% 	\begin{center}
% 		\begin{tabular}{|c|c|c|c|c|c|}
% 			\hline Doanh thu & {$[5 ; 7)$} & {$[7 ; 9)$} & {$[9 ; 11)$} & {$[11 ; 13)$} & {$[13 ; 15)$} \\
% 			\hline Số ngày   & $2$         & $7$         & $7$          & $3$           & $1$           \\
% 			\hline
% 		\end{tabular}
% 	\end{center}
% 	Số trung bình của mẫu số liệu trên thuộc khoảng nào trong các khoảng dưới đây?
% 	\choice{ $[7 ; 9)$}
% 	{ \True $[9 ; 11)$}
% 	{ $[11 ; 13)$}
% 	{ $[13 ; 15)$}
% 	\loigiai{
% 	Bảng tần số ghép nhóm theo giá trị đại diện là
% 	\begin{center}
% 		\begin{tabular}{|c|c|c|c|c|c|}
% 			\hline Doanh thu        & {$[5 ; 7)$} & {$[7 ; 9)$} & {$[9 ; 11)$} & {$[11 ; 13)$} & {$[13 ; 15)$} \\
% 			\hline Giá trị đại diện & $6$         & $8$         & $10$         & $12$          & $14$          \\
% 			\hline Số ngày          & $2$         & $7$         & $7$          & $3$           & $1$           \\
% 			\hline
% 		\end{tabular}
% 	\end{center}
% 	Số trung bình $\overline{x}=\dfrac{2\cdot 6+7\cdot8+7\cdot10+3\cdot12+1\cdot 14}{20}=9{,}4$.
% 	}
% \end{ex}

% \begin{ex}%[1D1B1-2]
% 	Trung tâm ngoại ngữ thống kê bảng điểm môn Tiếng Anh của một khóa học trong bảng bên dưới
% 	\begin{center}
% 		\begin{tabular}{|l|c|c|c|c|c|}
% 			\hline
% 			Điểm     & [0;2) & [2;4) & [4;6) & [6;8) & [8;10) \\ \hline
% 			Học viên & 10    & 30    & 55    & 42    & 9      \\ \hline
% 		\end{tabular}
% 	\end{center}
% 	Số trung bình của mẫu số liệu thuộc khoảng nào trong các khoảng dưới đây?
% 	\choice
% 	{$[8;10)$}
% 	{\True $[4;6)$}
% 	{$[2;4)$}
% 	{$[6;8)$}
% 	\loigiai{
% 		Ta có bảng thống kê theo giá trị đại diện như sau:
% 		\begin{center}
% 			\begin{tabular}{|l|c|c|c|c|c|}
% 				\hline
% 				Giá trị đại diện & 1  & 3  & 5  & 7  & 9 \\ \hline
% 				Tần số           & 10 & 30 & 55 & 42 & 9 \\ \hline
% 			\end{tabular}
% 		\end{center}
% 		Khi đó ta có số trung bình của mẫu số liệu trên được tính như sau: \\
% 		$$\bar{x}=\dfrac{1.10+3.30+5.55+7.42+9.9}{10+30+55+42+9}\approx 5,14.$$
% 	}
% \end{ex}

% \Closesolutionfile{ans}
\begin{dang}{Tính phương sai và độ lệch chuẩn của mẫu số liệu ghép nhóm}
	\begin{listEX}[1]
		\item [\ding{172}] Xác định cỡ của mẫu số liệu;
		\item [\ding{173}] Tính số trung bình của mẫu số liệu;
		\item [\ding{174}] Áp dụng công thức tính phương sai và độ lệch chuẩn.
	\end{listEX}
\end{dang}
\setcounter{vd}{0}
\setcounter{ex}{0}
% \boxmini{BÀI TẬP TỰ LUẬN}
\viduminhhoa
\begin{vd}
	Cân nặng của một số quả mít trong một khu vườn được thống kê ở bảng sau:
	\begin{center}
		\begin{tabular}{|c|c|c|c|c|c|}
			\hline Cân nặng (kg) & {$[4; 6)$} & {$[6; 8)$} & {$[8; 10)$} & {$[10; 12)$} & {$[12; 14)$} \\
			\hline Số quả mít    & $ 6 $      & $ 12 $     & $ 19 $      & $ 9 $        & $ 4 $        \\
			\hline
		\end{tabular}
	\end{center}
	Hãy tính phương sai và độ lệch chuẩn của mẫu số liệu ghép nhóm trên. (Kết quả các phép tính làm tròn đến hàng phần trăm.)
	\loigiai{
		Ta có bảng thống kê cân nặng của các quả mít theo giá trị đại diện:
		\begin{center}
			\begin{tabular}{|c|c|c|c|c|c|}
				\hline Cân nặng đại diện $(\mathrm{kg})$ & $ 5 $ & $ 7 $  & $ 9 $  & $ 11 $ & $ 13 $ \\
				\hline Tần số                            & $ 6 $ & $ 12 $ & $ 19 $ & $ 9 $  & $ 4 $  \\
				\hline
			\end{tabular}
		\end{center}
		Cỡ mẫu $n=6+12+19+9+4=50$.\\
		Số trung bình của mẫu số liệu ghép nhóm là
		$$\bar{x}=\dfrac{6\cdot 5+12\cdot 7+19\cdot 9+9\cdot 11+4\cdot 13}{50}=8,72.$$
		Phương sai của mẫu số liệu ghép nhóm là
		$$S^2=\dfrac{1}{50}\left(6 \cdot 5^2+12 \cdot 7^2+19 \cdot 9^2+9 \cdot 11^2+4 \cdot 13^2\right)-8,72^2 \approx 4,80.$$
		Độ lệch chuẩn của mẫu số liệu ghép nhóm là
		$$S \approx \sqrt{4,80} \approx 2,19.$$
	}
\end{vd}

\begin{vd}
	Thống kê tổng số giờ nắng trong tháng 9 tại một trạm quan trắc đặt ở Cà Mau trong các năm từ 2002 đến 2021 được thống kê như sau:
	\begin{center}
		\begin{tabular}{cccccccccc}
			$ 111,6 $ & $ 134,9 $ & $ 130,3 $ & $ 134,2 $ & $ 140,9 $ & $ 109,3 $ & $ 154,4 $ & $ 156,3 $ & $ 116,1 $ & $ 96,7 $ \\
			$ 105,2 $ & $ 80,8 $  & $ 80,8 $  & $ 110 $   & $ 109 $   & $ 139 $   & $ 145 $   & $ 161 $   & $ 126 $   & $ 114 $
		\end{tabular}
	\end{center}
	\begin{flushright}
		(Nguồn: Tổng cục Thống kê)
	\end{flushright}
	\begin{enumEX}{1}
		\item Hãy tính phương sai và độ lệch chuẩn của mẫu số liệu trên.
		\item Hãy lập bảng tần số ghép nhóm với nhóm đầu tiên là $[80; 98)$ và độ dài mỗi nhóm bằng $ 18 $. Tính phương sai, độ lệch chuẩn của mẫu số liệu ghép nhóm.
		\item Hãy tính sai số tương đối của độ lệch chuẩn của mẫu số liệu ghép nhóm so với độ lệch chuẩn của mẫu số liệu gốc.
	\end{enumEX}
	\loigiai{
		\begin{enumEX}{1}
			\item Cỡ mẫu là $n=20$.\\
			Số trung bình của mẫu số liệu trên là
			$$\bar{x}_1=\dfrac{111,6+134,9+\cdots+114}{20}=122,755.$$
			Phương sai của mẫu số liệu trên là
			$$S_1^2=\dfrac{1}{20}\left(111,6^2+134,9^2+\cdots+114^2\right)-122,755^2 \approx 515,453.$$
			Độ lệch chuẩn của mẫu số liệu trên là
			$$S_1 \approx \sqrt{515,453} \approx 22,704.$$
			\item Ta có bảng sau:
			\begin{center}
				\begin{tabular}{|c|c|c|c|c|c|}
					\hline Số giờ nắng      & {$[80; 98)$} & {$[98; 116)$} & {$[116; 134)$} & {$[134; 152)$} & {$[152; 170)$} \\
					\hline Giá trị đại diện & 89           & 107           & 125            & 143            & 161            \\
					\hline Số năm           & 3            & 6             & 3              & 5              & 3              \\
					\hline
				\end{tabular}
			\end{center}
			Số trung bình của mẫu số liệu ghép nhóm là
			$$\bar{x}_2=\dfrac{3\cdot 89+6\cdot 107+3\cdot 125+5\cdot 143+3\cdot 161}{20}=124,1.$$
			Phương sai của mẫu số liệu ghép nhóm là
			$$S_2^2=\dfrac{1}{20}\left(3\cdot 89^2+6\cdot 107^2+3\cdot 125^2+5\cdot 143^2+3\cdot 161^2\right)-124,1^2=566,19.$$
			Độ lệch chuẩn của mẫu số liệu ghép nhóm là
			$$S_2=\sqrt{566,19} \approx 23,795.$$
			\item Sai số tương đối của độ lệch chuẩn của mẫu số liệu ghép nhóm so với độ lệch chuẩn của mẫu số liệu gốc là
			$$\dfrac{\left|S_2-S_1\right|}{S_1}=\dfrac{|23,795-22,704|}{22,704} \cdot 100 \% \approx 4,805 \%.$$
		\end{enumEX}
	}
\end{vd}

\begin{vd}
	Thầy Tuấn thống kê lại điểm trung bình cuối năm của các học sinh lớp $11 \mathrm{A}$ và $ 11\mathrm{B} $ ở bảng sau:
	\begin{center}
		\begin{tabular}{|c|c|c|c|c|c|}
			\hline Điểm trung bình                  & {$[5; 6)$} & {$[6; 7)$} & {$[7; 8)$} & {$[8; 9)$} & {$[9; 10)$} \\
			\hline Số học sinh lớp $ 11\mathrm{A} $ & $ 1 $      & $ 0 $      & $ 11 $     & $ 22 $     & $ 6 $       \\
			\hline Số học sinh lớp $ 11\mathrm{B} $ & $ 0 $      & $ 6 $      & $ 8 $      & $ 14 $     & $ 12 $      \\
			\hline
		\end{tabular}
	\end{center}
	\begin{enumEX}{1}
		\item Nếu so sánh theo khoảng biến thiên thì học sinh lớp nào có điểm trung bình ít phân tán hơn?
		\item Nếu so sánh theo độ lệch chuẩn thì học sinh lớp nào có điểm trung bình ít phân tán hơn?
	\end{enumEX}
	\loigiai{
		\begin{enumEX}{1}
			\item Khoảng biến thiên của điểm trung bình của học sinh lớp $11 \mathrm{A}$ là: $10-5=5$.\\
			Khoảng biến thiên của điểm trung bình của học sinh lớp $ 11\mathrm{B} $ là: $10-6=4$.\\
			Nếu so sánh theo khoảng biến thiên thì điểm trung bình của các học sinh lớp $ 11\mathrm{B} $ ít phân tán hơn điểm trung bình của các học sinh lớp $ 11\mathrm{A} $.
			\item Ta có bảng thống kê điểm trung bình theo giá trị đại diện:
			\begin{center}
				\begin{tabular}{|c|c|c|c|c|c|}
					\hline Giá trị đại diện                 & $ 5,5 $ & $ 6,5 $ & $ 7,5 $ & $ 8,5 $ & $ 9,5 $ \\
					\hline Số học sinh lớp $ 11\mathrm{A} $ & $ 1 $   & $ 0 $   & $ 11 $  & $ 22 $  & $ 6 $   \\
					\hline Số học sinh lớp $ 11\mathrm{B} $ & $ 0 $   & $ 6 $   & $ 8 $   & $ 14 $  & $ 12 $  \\
					\hline
				\end{tabular}
			\end{center}
			\begin{itemize}
				\item Xét mẫu số liệu của lớp $ 11\mathrm{A} $:
				      \begin{itemize}
					      \item Cỡ mẫu là $n_1=1+11+22+6=40$.
					      \item Số trung bình của mẫu số liệu ghép nhóm là
					            $$\bar{x}_1=\dfrac{1 \cdot 5,5+11 \cdot 7,5+22 \cdot 8,5+6 \cdot 9,5}{40}=8,3.$$
					      \item Phương sai của mẫu số liệu ghép nhóm là
					            $$S_1^2=\dfrac{1}{40}\left(1 \cdot 5,5^2+11 \cdot 7,5^2+22 \cdot 8,5^2+6 \cdot 9,5^2\right)-8,3^2=0,61.$$
					      \item Độ lệch chuẩn của mẫu số liệu ghép nhóm là $S_1=\sqrt{0,61}$.
				      \end{itemize}
				\item Xét mẫu số liệu của lớp $ 11\mathrm{B} $:
				      \begin{itemize}
					      \item Cỡ mẫu là $n_2=6+8+14+12=40$.
					      \item Số trung bình của mẫu số liệu ghép nhóm là
					            $$
						            \bar{x}_2=\dfrac{6 \cdot 6,5+8 \cdot 7,5+14 \cdot 8,5+12 \cdot 9,5}{40}=8,3.
					            $$
					      \item Phương sai của mẫu số liệu ghép nhóm là
					            $$
						            S_2^2=\dfrac{1}{40}\left(6 \cdot 6,5^2+8 \cdot 7,5^2+14 \cdot 8,5^2+12 \cdot 9,5^2\right)-8,3^2=1,06.
					            $$
					      \item Độ lệch chuẩn của mẫu số liệu ghép nhóm là $S_2=\sqrt{1,06}$.
				      \end{itemize}
			\end{itemize}
			Do $S_1<S_2$ nên nếu so sánh theo độ lệch chuẩn thì học sinh lớp $11 \mathrm{A}$ có điểm trung bình ít phân tán hơn học sinh lớp $ 11\mathrm{B} $.
		\end{enumEX}
	}
\end{vd}

\begin{vd}
	Giá đóng cửa của một cổ phiếu là giá của cổ phiếu đó cuối một phiên giao dịch. Bảng sau thống kê giá đóng cửa (đơn vị: nghìn đồng) của hai mã cổ phiếu $A$ và $B$ trong $ 50 $ ngày giao dịch liên tiếp.
	\begin{center}
		\begin{tabular}{|c|c|c|c|c|c|}
			\hline Giá đóng cửa       & {$[120; 122)$} & {$[122; 124)$} & {$[124; 126)$} & {$[126; 128)$} & {$[128; 130)$} \\
			\hline \begin{tabular}{c}
				       Số ngày giao dịch \\
				       của cổ phiếu $A$
			       \end{tabular} & $ 8 $          & $ 9 $          & $ 12 $         & $ 10 $         & $ 11 $              \\
			\hline \begin{tabular}{c}
				       Số ngày giao dịch \\
				       của cổ phiếu $B$
			       \end{tabular} & $ 16 $         & $ 4 $          & $ 3 $          & $ 6 $          & $ 21 $              \\
			\hline
		\end{tabular}
	\end{center}
	Người ta có thể dùng phương sai và độ lệch chuẩn để so sánh mức độ rủi ro của các loại cổ phiếu có giá trị trung bình gần bằng nhau. Cổ phiếu nào có phương sai, độ lệch chuẩn cao hơn thì được coi là có độ rủi ro lớn hơn.\\
	Theo quan điểm trên, hãy so sánh độ rủi ro của cổ phiếu $A$ và cổ phiếu $B$.
	\loigiai{
		Ta có bảng thống kê giá đóng cửa theo giá trị đại diện:
		\begin{center}
			\begin{tabular}{|c|c|c|c|c|c|}
				\hline Giá đóng cửa       & $ 121 $ & $ 123 $ & $ 125 $ & $ 127 $ & $ 129 $ \\
				\hline \begin{tabular}{c}
					       Số ngày giao dịch \\
					       của cổ phiếu $A$
				       \end{tabular} & $ 8 $   & $ 9 $   & $ 12 $  & $ 10 $  & $ 11 $       \\
				\hline \begin{tabular}{c}
					       Số ngày giao dịch \\
					       của cổ phiếu $B$
				       \end{tabular} & $ 16 $  & $ 4 $   & $ 3 $   & $ 6 $   & $ 21 $       \\
				\hline
			\end{tabular}
		\end{center}
		\begin{itemize}
			\item Xét mẫu số liệu của cổ phiếu $A$:
			      \begin{itemize}
				      \item Số trung bình của mẫu số liệu ghép nhóm là
				            $$
					            \bar{x}_1=\dfrac{8 \cdot 121+9\cdot 123+12 \cdot 125+10\cdot 127+11\cdot 129}{50}=125,28.
				            $$
				      \item Phương sai của mẫu số liệu ghép nhóm là
				            $$
					            S_1^2=\dfrac{1}{50}\left(8 \cdot 121^2+9 \cdot 123^2+12 \cdot 125^2+10 \cdot 127^2+11 \cdot 129^2\right)-(125,28)^2=7,5216.
				            $$
				      \item Độ lệch chuẩn của mẫu số liệu ghép nhóm là $S_1=\sqrt{S_1^2}=\sqrt{7,5216}$.
			      \end{itemize}
			\item Xét mẫu số liệu của cổ phiếu $B$:
			      \begin{itemize}
				      \item Số trung bình của mẫu số liệu ghép nhóm là
				            $$
					            \bar{x}_2=\dfrac{16\cdot 121+4\cdot 123+3\cdot 125+6\cdot 127+21\cdot 129}{50}=125,28.
				            $$
				      \item Phương sai của mẫu số liệu ghép nhóm là
				            $$
					            S_2^2=\dfrac{1}{50}\left(16\cdot 121^2+4\cdot 123^2+3 \cdot 125^2+6 \cdot 127^2+21\cdot 129^2\right)-(125,48)^2=12,4096.$$
				      \item Độ lệch chuẩn của mẫu số liệu ghép nhóm là $S_2=\sqrt{S_2^2}=\sqrt{12,4096}$.
			      \end{itemize}
		\end{itemize}
		Vậy nếu đánh giá độ rủi ro theo phương sai và độ lệch chuẩn thì cổ phiếu $A$ có độ rủi ro thấp hơn cổ phiếu $B$.
	}
\end{vd}
\baitaptn
% \boxmini{BÀI TẬP TRẮC NGHIỆM}
% \ind{PHẦN I.} \inden{Câu trắc nghiệm nhiều phương án lựa chọn. Mỗi câu hỏi học sinh chỉ chọn một phương án.}\\
\setcounter{ex}{0}
\Opensolutionfile{ans}[ans/2D3-B2-d2-1]
\begin{ex}
	Trong các khẳng định sau, khẳng định nào sai?
	\choice
	{Phương sai luôn luôn là số không âm}
	{Phương sai là bình phương của độ lệch chuẩn}
	{Phương sai càng lớn thì độ phân tán của các giá trị quanh số trung bình càng lớn}
	{\True Phương sai luôn luôn lớn hơn độ lệch chuẩn}
	\loigiai{
		Ta có khi $s \in (0;1)$ thì $s^2 < s$. Do đó khẳng định phương sai luôn lớn hơn độ lệch chuẩn là sai.}
\end{ex}

\begin{ex}
	Số đặc trưng nào không sử dụng thông tin của nhóm số liệu đầu tiên và nhóm số liệu cuối cùng?
	\choice
	{Khoảng biến thiên}
	{\True Khoảng tứ phân vị}
	{Phương sai}
	{Độ lệch chuẩn}
	\loigiai{
		Số đặc trưng tứ phân vị không sử dụng thông tin của nhóm số liệu đầu tiên và nhóm số liệu cuối cùng
	}
\end{ex}

\begin{ex}%[2D4H2-2]
	Mỗi ngày bác Hương đều đi bộ để rèn luyện sức khỏe. Quãng đường đi bộ mỗi ngày (đơn vị km) của bác Hương trong $20$ ngày được thống kê lại ở bảng sau
	\begin{center}
		\begin{tabular}{|c|c|c|c|c|c|}
			\hline
			Quãng đường (km) & $[2{,}7;3{,}0)$ & $[3{,}0;3{,}3)$ & $[3{,}3;3{,}6)$ & $[3{,}6;3{,}9)$ & $[3{,}9;4{,}2)$ \\
			\hline
			Số ngày          & $3$             & $6$             & $5$             & $4$             & $2$             \\
			\hline
		\end{tabular}
	\end{center}
	Phương sai của mẫu số liệu ghép nhóm là
	\choice
	{$3{,}39$}
	{$11{,}62$}
	{\True $0{,}1314$}
	{$0{,}36$}
	\loigiai
	{
	Xét mẫu số liệu ghép nhóm cho bởi bảng sau
	\begin{center}
		\begin{tabular}{|c|c|c|c|c|c|}
			\hline
			Nhóm             & $[2{,}7;3{,}0)$ & $[3{,}0;3{,}3)$ & $[3{,}3;3{,}6)$ & $[3{,}6;3{,}9)$ & $[3{,}9;4{,}2)$ \\
			\hline
			Giá trị đại diện & $2{,}85$        & $3{,}15$        & $3{,}45$        & $3{,}75$        & $4{,}05$        \\
			\hline
			Tần số           & $3$             & $6$             & $5$             & $4$             & $2$             \\
			\hline
		\end{tabular}
	\end{center}
	Số trung bình của mẫu số liệu là
	$$\overline{x}=\dfrac{1}{20}\cdot (2{,}85\cdot 3+3{,}15\cdot 6+3{,}45\cdot 5+3{,}75\cdot 4+4{,}05\cdot 2)=3{,}39.$$
	Phương sai của mẫu số liệu ghép nhóm là
	$$S^2=\dfrac{1}{20}\left(3\cdot 2{,}85^2+6\cdot 3{,}15^2+5\cdot 3{,}45^2+4\cdot 3{,}45^2+2\cdot 4{,}05^2\right)-3{,}39^2=0{,}1314.$$
	}
\end{ex}


\begin{ex}%[2D4H2-2]
	Bạn Chi rất thích nhảy hiện đại. Thời gian tập nhảy mỗi ngày trong thời gian gần đây của bạn Chi được thống kê lại ở bảng sau
	\begin{center}
		\begin{tabular}{|c|c|c|c|c|c|}
			\hline
			Thời gian (phút) & $[20;25)$ & $[25;30)$ & $[30;35)$ & $[35;40)$ & $[40;45)$ \\
			\hline
			Số ngày          & $6$       & $6$       & $4$       & $1$       & $1$       \\
			\hline
		\end{tabular}
	\end{center}
	Phương sai của mẫu số liệu ghép nhóm có giá trị gần nhất với giá trị nào dưới đây?
	\choice
	{$31{,}77$}
	{$32$}
	{$31$}
	{\True $31{,}44$}
	\loigiai
	{
	Xét mẫu số liệu ghép nhóm cho bởi bảng sau
	\begin{center}
		\begin{tabular}{|c|c|c|c|c|c|}
			\hline
			Nhóm             & $[20;25)$ & $[25;30)$ & $[30;35)$ & $[35;40)$ & $[40;45)$ \\
			\hline
			Giá trị đại diện & $22{,}5$  & $27{,}5$  & $32{,}5$  & $37{,}5$  & $42{,}5$  \\
			\hline
			Tần số           & $6$       & $6$       & $4$       & $1$       & $1$       \\
			\hline
		\end{tabular}
	\end{center}
	Số trung bình của mẫu số liệu là
	$$\overline{x}=\dfrac{1}{18}\cdot (22{,}5\cdot 6+27{,}5\cdot 6+32{,}5\cdot 4+37{,}5\cdot 1+42{,}5\cdot 1)=\dfrac{85}{3}.$$
	Phương sai của mẫu số liệu ghép nhóm là
	$$S^2=\dfrac{1}{18}\left(6\cdot 22{,}5^2+6\cdot 27{,}5^2+4\cdot 32{,}5^2+1\cdot 37{,}5^2+1\cdot 42{,}5^2\right)-\left(\dfrac{85}{3}\right)^2=31{,}25.$$
	Vậy phương sai của mẫu số liệu ghép nhóm gần nhất với $31{,}44$.
	}
\end{ex}


\begin{ex}%[2D4H2-2]
	Mỗi ngày bác Hương đều đi bộ để rèn luyện sức khỏe. Quãng đường đi bộ mỗi ngày (đơn vị km) của bác Hương trong $20$ ngày được thống kê lại ở bảng sau
	\begin{center}
		\begin{tabular}{|c|c|c|c|c|c|}
			\hline
			Quãng đường (km) & $[2{,}7;3{,}0)$ & $[3{,}0;3{,}3)$ & $[3{,}3;3{,}6)$ & $[3{,}6;3{,}9)$ & $[3{,}9;4{,}2)$ \\
			\hline
			Số ngày          & $3$             & $6$             & $5$             & $4$             & $2$             \\
			\hline
		\end{tabular}
	\end{center}
	Độ lệch chuẩn của mẫu số liệu ghép nhóm có giá trị gần nhất với giá trị nào dưới đây?
	\choice
	{$3{,}41$}
	{$11{,}62$}
	{$0{,}017$}
	{\True $0{,}36$}
	\loigiai
	{
	Xét mẫu số liệu ghép nhóm cho bởi bảng sau
	\begin{center}
		\begin{tabular}{|c|c|c|c|c|c|}
			\hline
			Nhóm             & $[2{,}7;3{,}0)$ & $[3{,}0;3{,}3)$ & $[3{,}3;3{,}6)$ & $[3{,}6;3{,}9)$ & $[3{,}9;4{,}2)$ \\
			\hline
			Giá trị đại diện & $2{,}85$        & $3{,}15$        & $3{,}45$        & $3{,}75$        & $4{,}05$        \\
			\hline
			Tần số           & $3$             & $6$             & $5$             & $4$             & $2$             \\
			\hline
		\end{tabular}
	\end{center}
	Số trung bình của mẫu số liệu là
	$$\overline{x}=\dfrac{1}{20}\cdot (2{,}85\cdot 3+3{,}15\cdot 6+3{,}45\cdot 5+3{,}75\cdot 4+4{,}05\cdot 2)=3{,}39.$$
	Phương sai của mẫu số liệu ghép nhóm là
	$$S^2=\dfrac{1}{20}\left(3\cdot 2{,}85^2+6\cdot 3{,}15^2+5\cdot 3{,}45^2+4\cdot 3{,}45^2+2\cdot 4{,}05^2\right)-3{,}39^2=0{,}1314.$$
	Độ lệch chuẩn của mẫu số liệu ghép nhóm là $S=\sqrt{0{,}1314}\approx 0{,}36$.
	}
\end{ex}


\begin{ex}
	Dũng là học sinh rất giỏi chơi rubik, bạn có thể giải nhiều loại khối rubik khác nhau. Trong một lần tập luyện giải khối rubik $3\times 3$, bạn Dũng đã tự thống kê lại thời gian giải	rubik trong $25$ lần giải liên tiếp ở bảng sau
	\begin{center}
		\begin{tabular}{|c|c|c|c|c|c|}
			\hline
			Thời gian giải rubik (giây) & $[8;10)$ & $[10;12)$ & $[12;14)$ & $[14;16)$ & $[16;18)$ \\
			\hline
			Số ngày                     & $4$      & $6$       & $8$       & $4$       & $3$       \\
			\hline
		\end{tabular}
	\end{center}
	Độ lệch chuẩn của mẫu số liệu ghép nhóm có giá trị gần nhất với giá trị nào dưới đây?
	\choice
	{$5{,}98$}
	{$6$}
	{\True $2{,}44$}
	{$2{,}5$}
	\loigiai
	{
	Xét mẫu số liệu ghép nhóm cho bởi bảng sau
	\begin{center}
		\begin{tabular}{|c|c|c|c|c|c|}
			\hline
			Nhóm             & $[8;10)$ & $[10;12)$ & $[12;14)$ & $[14;16)$ & $[16;18)$ \\
			\hline
			Giá trị đại diện & $9$      & $11$      & $13$      & $15$      & $17$      \\
			\hline
			Tần số           & $4$      & $6$       & $8$       & $4$       & $3$       \\
			\hline
		\end{tabular}
	\end{center}
	Số trung bình của mẫu số liệu là
	$$\overline{x}=\dfrac{1}{25}\cdot (9\cdot 4+11\cdot 6+13\cdot 8+15\cdot 4+17\cdot 3)=12{,}68.$$
	Phương sai của mẫu số liệu ghép nhóm là
	$$S^2=\dfrac{1}{25}\left(4\cdot 9^2+6\cdot 11^2+8\cdot 13^2+4\cdot 15^2+3\cdot 17^2\right)-12{,}68^2=5{,}9776.$$
	Độ lệch chuẩn của mẫu số liệu là
	$$S=\sqrt{5{,}9776}=\approx 2{,}445.$$
	Vậy độ lệch chuẩn của mẫu số liệu ghép nhóm gần nhất với $2{,}44$.
	}
\end{ex}

\begin{ex}
	Để đánh giá chất lượng một lọa pin điện thoại mới, người ta ghi lại thời gian nghe nhạc liên tục của điện thoại được sạc đầy pin cho đến khi hết pin cho kết quả sau
	\begin{center}
		\begin{tabular}{|p{5cm}|c|c|c|c|c|}
			\hline
			Thời gian (giờ)              & $ [5;5{,}5) $ & $ [5{,}5;6) $ & $ [6;6{,}5) $ & $ [6{,}5;7) $ & $ [7;7{,}5) $ \\
			\hline
			Số chiếc điện thoại (tần số) & $ 2 $         & $ 8 $         & $ 15 $        & $ 10 $        & $ 5 $         \\
			\hline
		\end{tabular}
	\end{center}
	Tính độ lệch chuẩn của mẫu số liệu ghép nhóm trên (làm tròn đến 4 chữ số thập phân).
	\choice
	{$0{,}4252$}
	{$0{,}5314$}
	{$0{,}6214$}
	{\True $0{,}5268$}
	\loigiai{\begin{center}
		\begin{tabular}{|p{5cm}|c|c|c|c|c|}
			\hline
			Thời gian (giờ)              & $ [5;5{,}5) $ & $ [5{,}5;6) $ & $ [6;6{,}5) $ & $ [6{,}5;7) $ & $ [7;7{,}5) $ \\
			\hline
			Giá trị đại diện             & $ 5{,}25 $    & $ 5{,}75 $    & $ 6{,}25$     & $ 6{,}75$     & $ 7{,}25 $    \\
			\hline
			Số chiếc điện thoại (tần số) & $ 2 $         & $ 8 $         & $ 15 $        & $ 10 $        & $ 5 $         \\
			\hline
		\end{tabular}
	\end{center}
	Số trung bình của mẫu số liệu\\
	$ \overline{x}=\dfrac{m_{1}\cdot x_{1}+\dots+m_{k}\cdot x_{k}}{n}=\dfrac{2\cdot5{,}25+8\cdot 5{,}75+15\cdot 6{,}25+10\cdot 6{,}75+5\cdot7{,}25}{40}=6{,}35 $.\\
		Phương sai của mẫu số liệu ghép nhóm
		\begin{center}
			$ s^{2}=\dfrac{1}{40}\cdot\left(2\cdot 5{,}25^{2}+8\cdot 5{,}75^{2}+15\cdot 6{,}25^{2}+10\cdot 6{,}75^{2}+5\cdot 7{,}25^{2}\right)-6{,}35^{2}=0{,}2775 $.
		\end{center}
		Độ lệch chuẩn của mẫu số liệu ghép nhóm
		\begin{center}
			$ s=\sqrt{s^{2}}=\sqrt{0{,}2775}\approx 0{,}5268 $.
		\end{center}
	}
\end{ex}

\Closesolutionfile{ans}

% \ind{PHẦN II.} \inden{Câu trắc nghiệm đúng sai. Trong mỗi ý a), b), c), d) ở mỗi câu, học sinh chọn đúng hoặc sai.}\\
\Opensolutionfile{ans}[ans/2D3-B2-d2-2]

\begin{ex}
	Một trang trại phân $1 \, 000$ quả trứng thành $5$ loại, tuỳ theo khối lượng (đã được làm tròn) của chúng	được thống kê bởi bảng dưới đây:
	\begin{center}
		\begin{tabular}{|l|c|c|c|c|c|}
			\hline
			Khối lượng (gam) & $[30; 36)$ & $ [36; 42)$ & $ [42; 48)$ & $ [48; 54)$ & $ [54; 60)$ \\
			\hline
			Số trứng         & $45$       & $190$       & $500$       & $250$       & $15$        \\
			\hline
		\end{tabular}
	\end{center}
	\choiceTF
	{\True Khoảng biến thiên của mẫu số liệu là $30$}
	{\True Khoảng tứ phân vị của mẫu số liệu là $6{,} 48$}
	{\True Khối lượng trung bình của 100 quả trứng là 45 gam}
	{\True Độ lệch chuẩn của mẫu số liệu là $\dfrac{6\sqrt{17}}{5}$}
	\loigiai{
		\begin{enumerate}[a)]
			\item Khoảng biến thiên là $60-30=30$.
			\item Nhóm chứa $Q_1$ là nhóm $[42; 48)$.\\
			      Suy ra $Q_1= 42 + \dfrac{250- 235}{500} \cdot 16=42{,} 48$.\\
			      $\dfrac{3N}{4}= 750$.\\
			      Nhóm chứa $Q_3$ là nhóm $[48; 54)$.\\
			      Khi đó $Q_3 =48 +\dfrac{750- 735 }{250} \cdot 16 = 48{,} 96$.\\
			      Suy ra khoảng tứ phân vị $\Delta_Q = Q_3 - Q_1= 6{,} 48$.
			\item Ta có bảng sau:
			      \begin{center}
				      \begin{tabular}{|l|c|c|c|c|c|}
					      \hline \hline
					      \textbf{Khối lượng (gam)} & $[30; 36)$ & $ [36; 42)$ & $ [42; 48)$ & $ [48; 54)$ & $ [54; 60)$ \\
					      \hline
					      \textbf{Giá trị đại diện} & $33$       & $39$        & $45$        & $51$        & $57$        \\ \hline
					      \textbf{Số trứng }        & $45$       & $190$       & $500$       & $250$       & $15$        \\
					      \hline \hline
				      \end{tabular}
			      \end{center}
			      Khối lượng trung bình $$\overline{x}= \dfrac{33 \cdot 45 + 39 \cdot 190 + 45 \cdot 500 + 51 \cdot 250 + 57 \cdot 15}{1\, 000}= 45\text{ gam}.$$
			\item Phương sai: $\dfrac{33^2 \cdot 45 + 39^2 \cdot 190 + 45^2 \cdot 500 + 51^2 \cdot 250 + 57^2 \cdot 15}{1\, 000} - 45^2=24{,}48$
			      Độ lệch chuẩn $$s= \sqrt{\dfrac{33^2 \cdot 45 + 39^2 \cdot 190 + 45^2 \cdot 500 + 51^2 \cdot 250 + 57^2 \cdot 15}{1\, 000} - 45^2} =\dfrac{6\sqrt{17}}{5} \text{ gam}.$$
		\end{enumerate}
	}
\end{ex}

\begin{ex}
	Kết quả $ 40 $ lần nhảy xa của hai vận động viên nam Dũng và Huy được lần lượt thống kê trong Bảng ở bên (đơn vị: mét).
	\begin{center}
		% \begin{tabular}{|c|c|c|}
		% 	\hline Nhóm          & Dũng   & Huy    \\
		% 	\hline$[6,22; 6,46)$ & $ 3 $  & $ 2 $  \\
		% 	{$[6,46; 6,70)$}     & $ 7 $  & $ 5 $  \\
		% 	{$[6,70; 6,94)$}     & $ 5 $  & $ 8 $  \\
		% 	{$[6,94; 7,18)$}     & $ 20 $ & $ 19 $ \\
		% 	{$[7,18; 7,42)$}     & $ 5 $  & $ 6 $  \\
		% 	\hline               & $n=40$ & $n=40$ \\
		% 	\hline
		% \end{tabular}
		\begin{tabular}{|c|c|c|c|c|c|c|}
			\hline
			Nhóm & $[6,22; 6,46)$ & $[6,46; 6,70)$ & $[6,70; 6,94)$ & $[6,94; 7,18)$ & $[7,18; 7,42)$ & $n$ \\
			\hline
			Dũng & 3              & 7              & 5              & 20             & 5              & 40  \\
			\hline
			Huy  & 2              & 5              & 8              & 19             & 6              & 40  \\
			\hline
		\end{tabular}
	\end{center}
	\choiceTF
	{\True Số trung bình cộng của mẫu số liệu ghép nhóm biểu diễn kết quả $ 40 $ lần nhảy xa của vận động viên Dũng (làm tròn kết quả đến hàng phần trăm) là $6,92\,(\mathrm{m})$}
	{Số trung bình cộng của mẫu số liệu ghép nhóm biểu diễn kết quả $ 40 $ lần nhảy xa của vận động viên Huy (làm tròn kết quả đến hàng phần trăm) là $6,85\,(\mathrm{m})$}
	{\True Độ lệch chuẩn của mẫu số liệu ghép nhóm biểu diễn kết quả $ 40 $ lần nhảy xa của vận động viên Huy (làm tròn kết quả đến hàng phần trăm) là $0,24\,(\mathrm{m})$}
	{\True Dựa vào độ lệch chuẩn thì kết quả nhảy xa của vận động viên Huy đồng đều hơn kết quả nhảy xa của vận động viên Dũng}

	\loigiai{
		Ta có bảng thống kê sau:
		\begin{center}
			\begin{tabular}{|c|c|c|c|}
				\hline Nhóm          & Giá trị đại diện & Dũng   & Huy    \\
				\hline$[6,22; 6,46)$ & $ 6,34 $         & $ 3 $  & $ 2 $  \\
				{$[6,46; 6,70)$}     & $ 6,58 $         & $ 7 $  & $ 5 $  \\
				{$[6,70; 6,94)$}     & $ 6,82 $         & $ 5 $  & $ 8 $  \\
				{$[6,94; 7,18)$}     & $ 7,06 $         & $ 20 $ & $ 19 $ \\
				{$[7,18; 7,42)$}     & $ 7,30 $         & $ 5 $  & $ 6 $  \\
				\hline               &                  & $n=40$ & $n=40$ \\
				\hline
			\end{tabular}
		\end{center}
		\begin{enumEX}{1}
			\item Số trung bình cộng của mẫu số liệu ghép nhóm biểu diễn kết quả $ 40 $ lần nhảy xa của vận động viên Dũng là:
			$$\bar{x}_D=\dfrac{3 \cdot 6,34+7 \cdot 6,58+5 \cdot 6,82+20 \cdot 7,06+5 \cdot 7,30}{40}=\dfrac{276,88}{40} \approx 6,92\,(\mathrm{m}).$$
			\item Số trung bình cộng của mẫu số liệu ghép nhóm biểu diễn kết quả $ 40 $ lần nhảy xa của vận động viên Huy là:
			$$\bar{x}_H=\dfrac{2 \cdot 6,34+5 \cdot 6,58+8 \cdot 6,82+19 \cdot 7,06+6 \cdot 7,30}{40}=\dfrac{278,08}{40} \approx 6,95\,(\mathrm{m}).$$
			\item Phương sai của mẫu số liệu ghép nhóm biểu diễn kết quả $ 40 $ lần nhảy xa của vận động viên Huy (làm tròn kết quả đến hàng phần trăm) là:
			$s_H^2 =\dfrac{1}{40}[2 \cdot(6,34-6,95)^2+5 \cdot(6,58-6,95)^2+8\cdot(6,82-6,95)^2+19 \cdot(7,06-6,95)^2+6 \cdot(7,30-6,95)^2]=\dfrac{2,5288}{40} \approx 0,06.$\\
			Độ lệch chuẩn của mẫu số liệu ghép nhóm trên là:
			$$s_H \approx \sqrt{0,06} \approx 0,24\,(\mathrm{m}).$$
			\item Phương sai của mẫu số liệu ghép nhóm biểu diễn kết quả $ 40 $ lần nhảy xa của vận động viên Dũng (làm tròn kết quả đến hàng phần trăm) là:
			$s_D^2=\dfrac{1}{40}[3 \cdot(6,34-6,92)^2+7 \cdot(6,58-6,92)^2+5 \cdot(6,82-6,92)^2+20 \cdot(7,06-6,92)^2+5 \cdot(7,30-6,92)^2]=\dfrac{2,9824}{40} \approx 0,07.$\\
			Độ lệch chuẩn của mẫu số liệu ghép nhóm trên là: $s_D \approx \sqrt{0,07} \approx 0,26\,(\mathrm{m})$.\\
			Do $s_H \approx 0,24<s_D \approx 0,26$ nên kết quả nhảy xa của vận động viên Huy đồng đều hơn kết quả nhảy xa của vận động viên Dũng.
		\end{enumEX}
	}
\end{ex}

\begin{ex}
	Một công ty giống cây trồng đã thử nghiệm hai phương pháp chăm sóc khác nhau cho cây hướng dương. Sau hai tuần, người ta thấy cây được chăm sóc theo cả hai phương pháp đều thấp hơn $50$ cm.\\
	\begin{tikzpicture}[scale=1]
		\def\y {{1.2, 1.6, 2.4,1.6, 1.2} }
		\foreach \i in {0,...,4} \draw[fill=blue!50] ( 1* \i,0) rectangle ++(1, \y[\i]);
		\foreach \i in {1,...,5}  \draw(\i,0) circle (1pt) node[ below]{$\i 0$};
		\draw(0,1) circle (1pt) node[left]{$5$};  \draw(0,2) circle (1pt) node[left]{$10$};

		\def\d{1.5}
		\def\l{1}
		\draw[gray!,step=0.2,line width=0.05pt](0,0)grid(6,3);
		\draw[red!50,thin,opacity=.5]
		(0,0) grid (6,3);
		\draw[->] (0,0)node[below right]{$O$}--(6,0)node[below]{cm};
		\draw[->] (0,0)--(0,3)node[above]{Tần số};
		\node[right] at (1,-1) {\text{Chiều cao của cây chăm sóc}};
		\node[right] at (1.5,-1.5) {\text{theo phương pháp A}};
	\end{tikzpicture}
	\begin{tikzpicture}[scale=1]
		\def\y {{2.6, 1.2, 0.4,1.2, 2.6} }
		\foreach \i in {0,...,4} \draw[fill=blue!50] ( 1* \i,0) rectangle ++(1, \y[\i]);
		\foreach \i in {1,...,5}  \draw(\i,0) circle (1pt) node[ below]{$\i 0$};
		\draw(0,1) circle (1pt) node[left]{$5$};  \draw(0,2) circle (1pt) node[left]{$10$};

		\def\d{1.5}
		\def\l{1}
		\draw[gray!,step=0.2,line width=0.05pt](0,0)grid(6,3);
		\draw[red!50,thin,opacity=.5]
		(0,0) grid (6,3);
		\draw[->] (0,0)node[below right]{$O$}--(6,0)node[below]{cm};
		\draw[->] (0,0)--(0,3)node[above]{Tần số};
		\node[right] at (1,-1) {\text{Chiều cao của cây chăm sóc}};
		\node[right] at (1.5,-1.5) {\text{theo phương pháp B}};
	\end{tikzpicture}
	\choiceTF
	{\True Khoảng biến thiên của chiều cao các cây được chăm sóc theo mỗi phương pháp A và B bằng nhau}
	{\True Trung bình của chiều cao các cây được chăm sóc theo mỗi phương pháp A và B bằng nhau}
	{\True Độ lệch chuẩn của chiều cao các cây được chăm sóc theo phương án $A$ là 12{,} 65 (cm)}
	{Dựa vào độ lệch chuẩn thì chiều cao của các loại cây được chăm sóc theo phương án $B$ ít bị chênh lệch hơn so với phương án $A$.}
	\loigiai{
		\begin{enumEX}{1}
			\item Khoảng biến thiên của chiều cao các cây được chăm sóc theo mỗi phương pháp A và B bằng nhau và cùng bằng 50.
			\item	Ước tính số trung bình và độ lệch chuẩn của chiều cao các cây được chăm sóc theo mỗi phương pháp.\\
			Cỡ mẫu của hai mẫu số liệu thống kê là $N= 40$.\\
			Ta có bảng tần số ghép nhóm về chiều cao của cây được chăm sóc theo phương pháp A như sau:
			\begin{center}
				\begin{tabular}{|l|c|c|c|c|c|}
					\hline \hline
					\textbf{Chiều cao (cm)}   & $[0; 10)$ & $ [10; 20)$ & $ [20; 30)$ & $ [30; 40)$ & $ [40; 50)$ \\
					\hline
					\textbf{Giá trị đại diện} & $5$       & $15$        & $25$        & $35$        & $45$        \\ \hline
					\textbf{Tần số }          & $ 6$      & $8$         & $12$        & $8$         & $6$         \\
					\hline \hline
				\end{tabular}
			\end{center}
			Chiều cao trung bình của các cây được chăm sóc theo phương án $A$ là $$\overline{x}_A= \dfrac{5 \cdot 6 + 15 \cdot 8 + 25 \cdot 12 + 35 \cdot 8 + 45 \cdot 6}{40}=25.$$
			Ta có bảng tần số ghép nhóm về chiều cao của cây được chăm sóc theo phương pháp B như sau:
			\begin{center}
				\begin{tabular}{|l|c|c|c|c|c|}
					\hline \hline
					\textbf{Chiều cao (cm)}   & $[0; 10)$ & $ [10; 20)$ & $ [20; 30)$ & $ [30; 40)$ & $ [40; 50)$ \\
					\hline
					\textbf{Giá trị đại diện} & $5$       & $15$        & $25$        & $35$        & $45$        \\ \hline
					\textbf{Tần số }          & $ 13 $    & $6$         & $2$         & $6$         & $13$        \\
					\hline \hline
				\end{tabular}
			\end{center}
			Chiều cao trung bình của các cây được chăm sóc theo phương án $B$ là $$\overline{x}_B= \dfrac{5 \cdot 13 + 15 \cdot 6 + 25 \cdot 2 + 35 \cdot 6 + 45 \cdot 13}{40}=25 \text{ cm}.$$
			\item 	Độ lệch chuẩn của chiều cao các cây được chăm sóc theo phương án $A$ là $$s_A =\sqrt{\dfrac{5^2 \cdot 6 + 15^2 \cdot 8 + 25^2 \cdot 12 + 35^2 \cdot 8 + 45 ^2 \cdot 6}{40} - 25^2}\approx 12{,} 65. $$
			\item Độ lệch chuẩn của chiều cao các cây được chăm sóc theo phương án $B$ là $$s_B =\sqrt{\dfrac{5^2 \cdot 13 + 15^2 \cdot 6 + 25^2 \cdot 2 + 35^2 \cdot 6+ 45 ^2 \cdot 13}{40} - 25^2}\approx 17{,} 03 \text{ cm}. $$
			Do $s_A< s_B$ nên chiều cao của các loại cây được chăm sóc theo phương án $A$ ít bị chênh lệch hơn so với phương án $B$.
		\end{enumEX}}
\end{ex}

\Closesolutionfile{ans}

%Chương IV. Nguyên hàm. Tích phân.
% %%Bài 1. Nguyên hàm
% \chap{NGUYÊN HÀM VÀ TÍCH PHÂN}
\section{NGUYÊN HÀM}
\subsection{Tóm tắt lý thuyết}
\subsection{Kiến thức cần nắm}
% \subsubsection{ĐỊNH NGHĨA VÀ TÍNH CHẤT}
\subsubsection{Định nghĩa nguyên hàm}
Cho hàm số $f(x)$ xác định trên khoảng $K$. Hàm số $F(x)$ được gọi là nguyên hàm của hàm số $f(x)$ nếu $F'(x)=f(x)$ với mọi $x\in K$.\\
\textbf{Nhận xét:} Nếu $F(x)$ là một nguyên hàm của $f(x)$ thì $F(x)+C$, $(C\in\mathbb{R})$ cũng là nguyên hàm của $f(x)$.\\
Ký hiệu $\displaystyle\int f(x)\mathrm{\,d}x=F(x)+C$.\\
\subsubsection{Một số tính chất của nguyên hàm}
\begin{itemize}
	\item $\left(\displaystyle\int f(x)\mathrm{\,d}x\right)'=f(x)$.
	\item $\displaystyle\int a\cdot f(x)\mathrm{\,d}x=a\cdot\displaystyle\int f(x)\mathrm{\,d}x\quad\left(a\in\mathbb{R}, a\neq 0\right)$.
	\item $\displaystyle\int\left[f(x)\pm g(x)\right]\mathrm{\,d}x=\displaystyle\int f(x)\mathrm{\,d}x\pm\displaystyle\int g(x)\mathrm{\,d}x$.
\end{itemize}
\subsubsection{Một số nguyên hàm cơ bản}
\begin{longtable}{|c|c|}
	\hline
	 Nguyên hàm của hàm số cơ bản & Nguyên hàm mở rộng \\
	\hline
	$\displaystyle\int a\cdot\mathrm{\,d}x=ax+C, a\in\mathbb{R}$ & \\
	\hline
	$\displaystyle\int x^{\alpha}\mathrm{\,d}x=\dfrac{x^{\alpha+1}}{\alpha+1}+C,\alpha\neq-1$ & $\displaystyle\int(ax+b)^{\alpha}\mathrm{\,d}x=\dfrac{1}{a}\cdot\dfrac{(ax+b)^{\alpha+1}}{\alpha+1}+C$ \\
	\hline
	$\displaystyle\int\dfrac{\mathrm{\,d}x}{x}=\ln |x|+C, x\neq 0$ & $\displaystyle\int\dfrac{\mathrm{\,d}x}{ax+b}=\dfrac{1}{a}\cdot\ln |ax+b|+C$ \\
	\hline
	$\displaystyle\int\dfrac{\mathrm{\,d}x}{\sqrt{x}}=2\sqrt{x}+C, x>0$ & $\displaystyle\int\dfrac{\mathrm{\,d}x}{\sqrt{ax+b}}=\dfrac2a\sqrt{ax
	+b}+C, x>0$ \\
	\hline
	$\displaystyle\int\dfrac{\mathrm{\,d}x}{x^2}=-\dfrac{1}{x}+C, x\neq 0$ & $\displaystyle\int\dfrac{\mathrm{\,d}x}{(ax+b)^2}=-\dfrac{1}{a}\cdot \dfrac{1}{ax+b}+C$ \\
	\hline
	$\displaystyle\int\dfrac{\mathrm{\,d}x}{x^{\alpha}}=-\dfrac{1}{(\alpha-1)x^{\alpha-1}}+C$ & $\displaystyle\int\dfrac{\mathrm{\,d}x}{(ax+b)^{\alpha}}=-\dfrac{1}{a}\cdot \dfrac{1}{(\alpha-1)}\cdot (ax+b)^{\alpha-1}+C$ \\
	\hline
	$\displaystyle\int\mathrm{e}^x\mathrm{\,d}x=\mathrm{e}^x+C$ & $\displaystyle\int\mathrm{e}^{ax+b}\mathrm{\,d}x=\dfrac{1}{a}\cdot\mathrm{e}^{ax+b}+C$ \\
	\hline
	$\displaystyle\int a^x\mathrm{\,d}x=\dfrac{a^x}{\ln a}+C$ & $\displaystyle\int a^{\alpha x+\beta}\mathrm{\,d}x=\dfrac{1}{\alpha}\cdot\dfrac{a^{\alpha x+\beta}}{\ln a}+C$ \\
	\hline
	$\displaystyle\int\cos x\mathrm{\,d}x=\sin x+C$ & $\displaystyle\int\cos (ax+b)\mathrm{\,d}x=\dfrac{1}{a}\cdot\sin (ax+b)+C$ \\
	\hline
	$\displaystyle\int\sin x\mathrm{\,d}x=-\cos x+C$ & $\displaystyle\int\sin (ax+b)\mathrm{\,d}x=-\dfrac{1}{a}\cdot\cos (ax+b)+C$ \\
	\hline
	$\displaystyle\int\dfrac{1}{\cos^2x}\mathrm{\,d}x=\tan x+C$ & $\displaystyle\int\dfrac{1}{\cos^2(ax+b)}\mathrm{\,d}x=\dfrac{1}{a}\cdot \tan (ax+b)+C$ \\
	\hline
	$\displaystyle\int\dfrac{1}{\sin^2x}\mathrm{\,d}x=-\cot x+C$ & $\displaystyle\int\dfrac{1}{\sin^2(ax+b)}\mathrm{\,d}x=-\dfrac{1}{a}\cdot \cot(ax+b)+C$ \\
	\hline
\end{longtable}
\textit{\textbf{Nhận xét:} $[F(ax+b)]'=af(ax+b) \Rightarrow \int f(ax+b) \mathrm{\,d}x = \dfrac{1}{a} F(ax+b)+C$}.
\subsection{Phân loại và phương pháp giải bài tập}
\begin{dang}{Sử dụng định nghĩa nguyên hàm và bảng nguyên hàm}
\end{dang}
\subsubsection{Các ví dụ}
\begin{vd}%Câu 1  %[2D3Y1-1]
	Tìm họ nguyên hàm của các hàm số sau
    \begin{listEX}[2]
        \item $f(x)=4x^3+x+5$.
        \item $f(x)=3x^2-2x$.
        \item $f(x)=\dfrac{1}{x^5}+x^2$.
        \item $f(x)=\dfrac{1}{x^3}+x^2-1$.
    \end{listEX}
	\loigiai{
        \begin{listEX}[1]
            \item Ta có $F(x)=\displaystyle\int f(x)\mathrm{\,d}x =\displaystyle\int{(4x^3+x+5)\textrm{ d}x=x^4+\dfrac{x^2}{2}+5x+C}$.
            \item Ta có $F(x)=\displaystyle\int f(x)\mathrm{\,d}x =\displaystyle\int{(3x^2-2x)\textrm{ d}x=x^3-x^2+C}$.
            \item Ta có $F(x)=\displaystyle\int f(x)\mathrm{\,d}x=\displaystyle\int ({x^{-5}}+x^2)\mathrm{\,d}x =-\dfrac{{x^{-4}}}{4}+\dfrac{x^3}{3}+C$.
            \item Ta có $F(x)=\displaystyle\int{f(x)\mathrm{\,d}x}=\displaystyle\int{\left( {x^{-3}}+x^2-1 \right)\mathrm{\,d}x}=-\dfrac{{x^{-2}}}{2}+\dfrac{x^3}{3}-x$.
        \end{listEX}
		}
    \end{vd}
\begin{vd}%Câu 5 %[2D3Y1-1]
	Tính
    \begin{listEX}[3]
        \item $I=\displaystyle\int{(x^2-3x)(x+1)\mathrm{\,d}x}$.
        \item $I=\displaystyle\int{(x-1)(x^2+2)\mathrm{\,d}x}$.
        \item $I=\displaystyle\int{{{(2x+1)}^5}\mathrm{\,d}x}$
        \item $I=\displaystyle\int{{{(2x-10)}^{2020}}\mathrm{\,d}x}$.
        \item $I=\displaystyle\int{\left( 3x^2+\dfrac{1}{x}-2 \right)\mathrm{\,d}x}$.
        \item $I=\displaystyle\int{\left( 3x^2-\dfrac{2}{x}-\dfrac{1}{x^2} \right)\mathrm{\,d}x}$.
        \item $I=\displaystyle\int{\dfrac{x^2-3x+1}{x}\mathrm{\,d}x}$.
        \item $I=\displaystyle\int{\dfrac{2x^2-6x+3}{x}\mathrm{\,d}x}$.
        \item $I=\displaystyle\int{\dfrac{1}{2x-1}\mathrm{\,d}x}$.
        \item $I=\displaystyle\int{\dfrac{2}{3-4x}\mathrm{\,d}x}$.
        \item $I=\displaystyle\int{\dfrac{1}{{{\left( 2x-1 \right)}^2}}\mathrm{\,d}x}$.
        \item $I=\displaystyle\int{\left[ \dfrac{12}{{{\left( x-1 \right)}^2}}+\dfrac{2}{2x-3} \right]\mathrm{\,d}x}$.
        \item $I=\displaystyle\int{\dfrac{3}{4x^2+4x+1}\textrm{ d}x}$.
        \item $I=\displaystyle\int{\dfrac{4}{x^2+6x+9}\textrm{ d}x}$.
            \item (*) $I=\displaystyle\int{\dfrac{2x-1}{{{\left( x+1 \right)}^2}}\textrm{ d}x}$.
    \end{listEX}
	\loigiai{
        \begin{listEX}[1]
            \item Phân phối được: $I=\displaystyle\int{(x^3-2x^2-3x)\mathrm{\,d}x} =\dfrac{x^4}{4}-\dfrac{2}{3}x^3-\dfrac{3}{2}x^2+C$.
            \item Phân phối được: $I=\displaystyle\int{(x^3-x^2+2x-2)\mathrm{\,d}x} =\dfrac{x^4}{4}-\dfrac{x^3}{3}+x^2-2x+C$.
            \item $I=\displaystyle\int{{{(2x+1)}^5}\mathrm{\,d}x}=\dfrac{1}{2}\dfrac{{{(2x+1)}^6}}{6}+C$.
            \item $I=\displaystyle\int{{{(2x-10)}^{2020}}\mathrm{\,d}x}=\dfrac{1}{2}\dfrac{{{(2x-10)}^{2021}}}{2021}+C$.
            \item Ta có $I=\displaystyle\int{\left( 3x^2+\dfrac{1}{x}-2 \right)\mathrm{\,d}x}=x^3+\ln \left| x \right|-2x+C$.
            \item Ta có $I=\displaystyle\int{\left( 3x^2-\dfrac{2}{x}-\dfrac{1}{x^2} \right)\mathrm{\,d}x}=x^3-2\ln \left| x \right|+\dfrac{1}{x}+C$.
            \item Ta có $I=\displaystyle\int{\dfrac{x^2-3x+1}{x}\mathrm{\,d}x}=\displaystyle\int{\left( x-3+\dfrac{1}{x} \right)\mathrm{\,d}x}=x^2-3x+\ln \left| x \right|+C$.
            \item Ta có $I=\displaystyle\int{\dfrac{2x^2-6x+3}{x}\mathrm{\,d}x}=\displaystyle\int{\left( 2x-6+\dfrac{3}{x} \right)\mathrm{\,d}x}=x^2-6x+3\ln \left| x \right|+C$.
            \item Ta có $I=\displaystyle\int{\dfrac{1}{2x-1}\mathrm{\,d}x}=\dfrac{1}{2}\ln \left| 2x-1 \right|+C$.
            \item Ta có $I=\displaystyle\int{\dfrac{2}{3-4x}\mathrm{\,d}x}=2.\dfrac{1}{-4}.\ln \left| 3-4x \right|+C=-\dfrac{1}{2}\ln \left| 3-4x \right|+C$.
            \item $I=\displaystyle\int{\dfrac{1}{{{\left( 2x-1 \right)}^2}}\mathrm{\,d}x=-\dfrac{1}{2}}.\dfrac{1}{2x-1}+C=\dfrac{-1}{4x-2}+C$.
            \item $I=\displaystyle\int{\left[ \dfrac{12}{{{\left( x-1 \right)}^2}}+\dfrac{2}{2x-3} \right]\mathrm{\,d}x=-\dfrac{12}{1}}.\dfrac{1}{x-1}+\dfrac{2}{2}\ln \left| 2x-3 \right|+C=\dfrac{-12}{x-1}+\ln \left| 2x-3 \right|+C$.
            \item $I=\displaystyle\int{\dfrac{1}{4x^2+4x+1}\textrm{ d}x=}\displaystyle\int{\dfrac{1}{{{\left( 2x+1 \right)}^2}}\textrm{ d}x=-\dfrac{1}{2}}.\dfrac{1}{2x+1}+C=\dfrac{-1}{4x+2}+C$.
            \item $I=\displaystyle\int{\dfrac{4}{x^2+6x+9}\textrm{ d}x=}\displaystyle\int{\dfrac{4}{{{\left( x+3 \right)}^2}}\textrm{ d}x=-\dfrac{4}{1}}.\dfrac{1}{x+3}+C=\dfrac{-4}{x+3}+C$.
            \item $I=\displaystyle\int{\dfrac{2x+2-3}{{{\left( x+1 \right)}^2}}\textrm{ d}x=\displaystyle\int{\left[ \dfrac{2(x+1)}{{{\left( x+1 \right)}^2}}-\dfrac{3}{{{\left( x+1 \right)}^2}} \right]}}\textrm{ d}x=\displaystyle\int{\dfrac{2}{x+1}\textrm{ d}x-\displaystyle\int{\dfrac{3}{{{\left( x+1 \right)}^2}}\textrm{ d}x}}$.\\
            $I=2\ln \left| x+1 \right|-\dfrac{-3}{x+1}+C=2\ln \left| x+1 \right|+\dfrac{3}{x+1}+C$.
            \item $I=\displaystyle\int{\dfrac{2x-2}{{{\left( 2x+1 \right)}^2}}\textrm{ d}x=\displaystyle\int{\left[ \dfrac{2x+1}{{{\left( 2x+1 \right)}^2}}-\dfrac{3}{{{\left( 2x+1 \right)}^2}} \right]}}\textrm{ d}x = \displaystyle\int{\dfrac{1}{2x+1}\textrm{ d}x-\displaystyle\int{\dfrac{3}{{{\left( 2x+1 \right)}^2}}\textrm{ d}x}}$.\\
            $I=\dfrac{1}{2}\ln \left| 2x+1 \right|-\dfrac{-3}{2\left( 2x+1 \right)}+C \Rightarrow I=\dfrac{1}{2}\ln \left| 2x+1 \right|+\dfrac{3}{2\left( 2x+1 \right)}+C$.
        \end{listEX}
		}
\end{vd}
\begin{vd}%[2D3B1-1]%BT3.
    Tìm họ nguyên hàm của các hàm số sau
    \begin{listEX}[3]
        \item $I=\displaystyle\int(\sin x-\cos x) \mathrm{\,d}x$.
        \item $I=\displaystyle\int (3 \cos x-2 \sin x) \mathrm{\,d}x$.
        \item $I=\displaystyle\int (2 \sin 2x-3 \cos 6x) \mathrm{\,d}x$.
        \item $I=\displaystyle\int \sin x \cos x \mathrm{\,d}x$.
        \item $I=\displaystyle\int \cos \left(\dfrac{x}{2}+\dfrac{\pi}{6}\right)\mathrm{\,d}x$.
        \item $I=\displaystyle\int \sin \left(\dfrac{\pi}{3}-\dfrac{x}{3}\right)\mathrm{\,d}x$.
        \item $I=\displaystyle\int (\sin x-\cos x)^2 \mathrm{\,d}x$.
        \item $I=\displaystyle\int (\cos x+\sin x)^2 \mathrm{\,d}x$.
        %    \item $I=\displaystyle\int \left(\cos ^2x-\sin ^2x\right) \mathrm{\,d}x$.
        %    \item $I=\displaystyle\int \left(\cos ^{4}x-\sin ^{4}x\right) \mathrm{\,d}x$.
    \end{listEX}
    \loigiai{
        \begin{listEX}[1]
            \item $I=\displaystyle\int(\sin x-\cos x) \mathrm{\,d}x=-\cos x-\sin x +C$.
            \item $I=\displaystyle\int (3 \cos x-2 \sin x) \mathrm{\,d}x=3\sin x + 2\cos x+C $.
            \item $I=\displaystyle\int (2 \sin 2x-3 \cos 6x) \mathrm{\,d}x=-\cos 2x -\dfrac{1}{2} \sin 6x+C$.
            \item $I=\dfrac{1}{2}\displaystyle\int \sin 2x \mathrm{\,d}x=-\dfrac{1}{4}\cos 2x+C$.
            \item $I=\displaystyle\int \cos \left(\dfrac{x}{2}+\dfrac{\pi}{6}\right)\mathrm{\,d}x=\displaystyle\int \left(\dfrac{\sqrt{3}}{2}\cos\dfrac{x}{2} -\dfrac{1}{2}\sin \dfrac{x}{2}\right) \mathrm{\,d}x = \sqrt{3}\sin \dfrac{x}{2}+\cos \dfrac{x}{2}+C$.
            \item $I=\displaystyle\int \sin \left(\dfrac{\pi}{3}-\dfrac{x}{3}\right)\mathrm{\,d}x= \displaystyle\int  \left(\dfrac{\sqrt{3}}{2}\cos \dfrac{x}{3}-\dfrac{1}{2}\sin \dfrac{x}{3} \right)\mathrm{\,d}x =\dfrac{3\sqrt{3}}{2}\sin \dfrac{x}{3}+\dfrac{3}{2}\cos\dfrac{x}{3}+C$.
            \item $I=\displaystyle\int (\sin x-\cos x)^2 \mathrm{\,d}x=\displaystyle\int (1-\sin 2x)\mathrm{\,d}x=x+\dfrac{1}{2}\cos 2x+C$.
            \item $I=\displaystyle\int (\cos x+\sin x)^2 \mathrm{\,d}x=\displaystyle\int(1+\sin 2x)\mathrm{\,d}x=x-\dfrac{1}{2}\cos 2x+C$.
            \item $I=\displaystyle\int \left(\cos ^2x-\sin ^2x\right) \mathrm{\,d}x= \displaystyle\int \cos 2x \mathrm{\,d}x=\dfrac{1}{2}\sin 2x+C$.
            \item $I=\displaystyle\int \left(\cos ^{4}x-\sin ^{4}x\right) \mathrm{\,d}x=\displaystyle\int \left(\cos ^2x-\sin ^2x\right) \mathrm{\,d}x= \displaystyle\int \cos 2x \mathrm{\,d}x=\dfrac{1}{2}\sin 2x+C$.
        \end{listEX}
    }
\end{vd}

\begin{vd} %[2D3B1-1]
    Tìm họ nguyên hàm của các hàm số sau
    \begin{listEX}[3]
        \item $I=\displaystyle\int \dfrac{1}{\sin ^2x} \mathrm{\,d}x$.
        \item $I=\displaystyle\int \dfrac{6}{\cos ^2 3x} \mathrm{\,d}x$.
        \item $I=\displaystyle\int (\tan x+\cot x)^2 \mathrm{\,d}x$.
        \item $I=\displaystyle\int \sin ^2x \mathrm{\,d}x$.
        \item $I=\displaystyle\int \cos ^2 2x \mathrm{\,d}x$.
        \item $I=\displaystyle\int \sin 4x \cos x \mathrm{\,d}x$.
        \item $I=\displaystyle\int \dfrac{1}{\sin x \cos x} \mathrm{\,d}x$.
    \end{listEX}

    \loigiai{
        \begin{listEX}[1]
            \item $I=\displaystyle\int\left( \dfrac{1}{\cos ^2x}-\dfrac{1}{\sin ^2x}\right) \mathrm{\,d}x=\tan x+\cot x +C$.
            \item $I=\displaystyle\int \dfrac{6}{\cos ^2 3x} \mathrm{\,d}x=2\tan 3x+C$.
            \item $I=\displaystyle\int (\tan x+\cot x)^2 \mathrm{\,d}x=\displaystyle\int (\tan^2 x+\cot^2x+2) \mathrm{\,d}x            =\displaystyle\int (\tan^2 x+1+\cot^2x+1) \mathrm{\,d}x=\tan x-\cot x+C$.
			\item $I=\displaystyle\int \sin ^2x \mathrm{\,d}x = \displaystyle\int \dfrac{1-\cos 2x}{2} \mathrm{\,d}x = \dfrac{1}{2}x-\dfrac{1}{4}\sin 2x+C$.
			\item $I=\displaystyle\int \cos ^2 2x \mathrm{\,d}x = \displaystyle\int \dfrac{1+\cos 4x}{2} \mathrm{\,d}x = \dfrac{1}{2}x+\dfrac{1}{8}\sin 4x+C$.
        \end{listEX}
    }
\end{vd}
\begin{vd} %[2D3B1-1]
    Tìm họ nguyên hàm của các hàm số sau
    \begin{listEX}[3]
        \item $I=\displaystyle\int \mathrm{e} ^{2x} \mathrm{\,d}x$.
        \item $I=\displaystyle\int \mathrm{e}^{1-2x} \mathrm{\,d}x$.
        \item $I=\displaystyle\int \left(2x-\mathrm{e}^{-x}\right) \mathrm{\,d}x$.
        \item $I=\displaystyle\int \mathrm{e}^x\left(1-3 \mathrm{e}^{-2x}\right) \mathrm{\,d}x$.
        \item $I=\displaystyle\int \left(3-\mathrm{e}^x\right)^2 \mathrm{\,d}x$.
        \item $I=\displaystyle\int \left(2+\mathrm{e}^{3x}\right)^2 \mathrm{\,d}x$.
        \item $I=\displaystyle\int 2^{2x+1} \mathrm{\,d}x$.
        \item $I=\displaystyle\int 4^{1-2x} \mathrm{\,d}x$.
        \item $I=\displaystyle\int 3^x \cdot 5^x \mathrm{\,d}x$.
        \item $I=\displaystyle\int 4^x \cdot 3^{x-1} \mathrm{\,d}x$.
        \item $I=\displaystyle\int \dfrac{\mathrm{\,d}x}{\mathrm{e}^{2-5x}}$.
        \item $I=\displaystyle\int \dfrac{\mathrm{\,d}x}{2^{3-2x}}$.
        \item $I=\displaystyle\int \dfrac{4^{x+1} \cdot 3^{x-1}}{2^x} \mathrm{\,d}x$.
        \item $I=\displaystyle\int \dfrac{4^{2x-1} \cdot 6^{x-1}}{3^x} \mathrm{\,d}x$.
    \end{listEX}
    \loigiai{
        \begin{listEX}[1]
            \item Ta có $I=\displaystyle\int \mathrm{e} ^{2x} \mathrm{\,d}x=\dfrac{1}{2} \mathrm{e}^{2x}+C$.
            \item Ta có $I=\displaystyle\int \mathrm{e}^{1-2x} \mathrm{\,d}x=-\dfrac{1}{2}\mathrm{e}^{1-2x}+C$.
            \item $I=\displaystyle\int \left(2x-\mathrm{e}^{-x}\right) \mathrm{\,d}x=x^2+\mathrm{e}^{-x}+C$.
            \item Ta có $I=\displaystyle\int \mathrm{e}^x\left(1-3 \mathrm{e}^{-2x}\right) \mathrm{\,d}x=\displaystyle\int \left(e^x-3e^{-x}\right) \mathrm{\,d}x=e^x+3e^{-x}+C$.
            \item $I=\displaystyle\int \left(3-\mathrm{e}^x\right)^2 \mathrm{\,d}x=\displaystyle\int\left( 9-6\mathrm{e}^x+\mathrm{e}^{2x}\right) \mathrm{\,d}x=9x-6\mathrm{e}^x+\dfrac{1}{2}\mathrm{e}^{2x}+C$.
            \item Ta có $I=\displaystyle\int \left(2+\mathrm{e}^{3x}\right)^2 \mathrm{\,d}x= \displaystyle\int \left(4+4\mathrm{e}^{3x}+ \mathrm{e}^{6x}\right) \mathrm{\,d}x=4x+\dfrac{4}{3}\mathrm{e}^{3x}+\dfrac{1}{6}\mathrm{e}^{6x}+C$.
            \item Ta có $I=\displaystyle\int 2^{2x+1} \mathrm{\,d}x=\dfrac{2^{2x+1}}{2\ln 2}+C$.
            \item Ta có $I=\displaystyle\int 4^{1-2x} \mathrm{\,d}x=-\dfrac{4^{1-2x}}{2\ln 4}+C$.
            \item Ta có $I=\displaystyle\int 15^x \mathrm{\,d}x = \dfrac{15^x}{\ln 15}+C$.
            \item Ta có $I=\dfrac{1}{3}\displaystyle\int 12^x \mathrm{\,d}x=\dfrac{12^x}{3\ln 12}+C$.
            \item $I=\displaystyle\int \mathrm{e}^{5x-2}\mathrm{\,d}x=\dfrac{\mathrm{e}^{5x-2}}{5}+C$.
            \item Ta có $I=\displaystyle\int 2^{2x-3} \mathrm{\,d}x =\dfrac{ 2^{2x-3}}{2\ln 2}+C$.
            \item Ta có $I=\displaystyle\int \dfrac{4^{x+1} \cdot 3^{x-1}}{2^x} \mathrm{\,d}x=\dfrac{4}{3}\displaystyle\int 6^x\mathrm{\,d}x= \dfrac{4\cdot 6^x}{3\cdot \ln 6}+C$.
            \item Ta có $I=\displaystyle\int \dfrac{4^{2x-1} \cdot 6^{x-1}}{3^x} \mathrm{\,d}x=\dfrac{1}{24}\displaystyle\int 32^x \mathrm{\,d}x=\dfrac{32^x}{24\ln 32}+C=\dfrac{2^{5x}}{120\ln 2}+C$.
        \end{listEX}
    }
\end{vd}
\subsubsection{Câu hỏi trắc nghiệm}
% \TN
\Opensolutionfile{ans}[ans/ans-2-B1-D2-LC]
\begin{ex}%[2D4N1-1]
	Cho hàm số $F(x)$ là một nguyên hàm của hàm số $f(x)$ trên $K$. Các mệnh đề sau, mệnh đề nào \textbf{sai}.
	\choice
	{$\displaystyle\int{f(x)\mathrm{\,d}x=}F(x)+C$}
	{$\displaystyle{\left(\displaystyle\int{f(x)\mathrm{\,d}x}\right)'}=f(x)$}
	{\True $\displaystyle{\left(\displaystyle\int{f(x)\mathrm{\,d}x}\right)'}=f'(x)$}
	{$\displaystyle{\left(\displaystyle\int{f(x)\mathrm{\,d}x}\right)'}=F'(x)$}
	\loigiai{
		Ta có $\displaystyle\int{f(x)\mathrm{\,d}x=}F(x)+C\Leftrightarrow F'(x)=f(x)$ nên phương án $\left(\displaystyle\int{f(x)\mathrm{\,d}x}\right)'=f'(x)$ sai.}
\end{ex}

\begin{ex}%[2D4N1-2]
	Họ tất cả các nguyên hàm của hàm số $f(x)=2x+6$ là
	\choice
	{$x^2+C$}
	{\True $x^2+6x+C$}
	{$2x^2+C$}
	{$2x^2+6x+C$}
	\loigiai{
		$\displaystyle\int{(2x+6)\mathrm{\,d}x=x^2+6x+C}$.}
\end{ex}

\begin{ex}%[2D4N1-2]
	$\displaystyle\int{x^2\mathrm{\,d}x}$ bằng
	\choice
	{$2x+C$}
	{\True $\dfrac{1}{3}x^3+C$}
	{$x^3+C$}
	{$3x^3+C$}
	\loigiai{
		Ta có $\displaystyle\int{x^2\mathrm{\,d}x}=\dfrac{1}{3}x^3+C$.}
\end{ex}

\begin{ex}%[2D4N1-2]
	Họ nguyên hàm của hàm số $f(x)=3x^2+1$ là
	\choice
	{$x^3+C$}
	{$\dfrac{x^3}{3}+x+C$}
	{$6x+C$}
	{\True $x^3+x+C$}
	\loigiai{
		$\displaystyle\int{(3x^2+1)\mathrm{\,d}x=x^3+x+C}$.}
\end{ex}

\begin{ex}%[2D4N1-2]
	Nguyên hàm của hàm số $f(x)=x^3+x$ là
	\choice
	{\True $\dfrac{1}{4}x^4+\dfrac{1}{2}x^2+C$}
	{$3x^2+1+C$}
	{$x^3+x+C$}
	{$x^4+x^2+C$}
	\loigiai{
		$\displaystyle\int{(x^3+x^2)\mathrm{\,d}x}=\dfrac{1}{4}x^4+\dfrac{1}{2}x^2+C$.}
\end{ex}

\begin{ex}%[2D4N1-2]
	Nguyên hàm của hàm số $f(x)=x^4+x^2$ là
	\choice
	{\True $\dfrac{1}{5}x^5+\dfrac{1}{3}x^3+C$}
	{$x^4+x^2+C$}
	{$x^5+x^3+C$}
	{$4x^3+2x+C$}
	\loigiai{
		$\displaystyle\int{f(x)\mathrm{\,d}x}=\displaystyle\int{(x^4+x^2)\mathrm{\,d}x}$ $=\dfrac{1}{5}x^5+\dfrac{1}{3}x^3+C$.}
\end{ex}

\begin{ex}%[2D4H1-2]
	Hàm số nào trong các hàm số sau đây không là nguyên hàm của hàm số $y=x^{2022}$?
	\choice
	{$\dfrac{x^{2023}}{2023}+1$}
	{$\dfrac{x^{2023}}{2023}$}
	{\True $y=2022x^{2021}$}
	{$\dfrac{x^{2023}}{2023}-1$}
	\loigiai{
		Ta có $\displaystyle\int{x^{2022}\mathrm{\,d}}x=\dfrac{x^{2023}}{2023}+C$, $C$ là hằng số nên $y=2022x^{2021}$ không là nguyên hàm của hàm số $y=x^{2022}$.}
\end{ex}

\begin{ex}%[2D4H1-2]
	Nguyên hàm của hàm số $f(x)=$ $\dfrac{1}{3}x^3-2x^2+x-2024$ là
	\choice
	{$\dfrac{1}{12}x^4-\dfrac{2}{3}x^3+\dfrac{x^2}{2}+C$}
	{$\dfrac{1}{9}x^4-\dfrac{2}{3}x^3+\dfrac{x^2}{2}-2024x+C$}
	{\True $\dfrac{1}{12}x^4-\dfrac{2}{3}x^3+\dfrac{x^2}{2}-2024x+C$}
	{$\dfrac{1}{9}x^4+\dfrac{2}{3}x^3-\dfrac{x^2}{2}-2024x+C$}
	\loigiai{
		Sử dụng công thức $\displaystyle\int{x^n\mathrm{\,d}x=\dfrac{x^{n+1}}{n+1}+C}$ ta được
		\begin{eqnarray*}
		\displaystyle\int\left(\dfrac{1}{3}x^3-2x^2+x-2024\right)\mathrm{\,d}x&=&\dfrac{1}{3}\cdot \dfrac{x^4}{4}-2\cdot \dfrac{x^3}{3}+\dfrac{x^2}{2}-2024x+C\\&=&\dfrac{1}{12}x^4-\dfrac{2}{3}x^3+\dfrac{1}{2}x^2-2024x+C.
		\end{eqnarray*}
		}
\end{ex}

\begin{ex}%[2D4H1-2]
	Tìm nguyên $F(x)$ của hàm số $f(x)=(x+1)(x+2)(x+3)?$
	\choice
	{$F(x)=\dfrac{x^4}{4}-6x^3+\dfrac{11}{2}x^2-6x+C$}
	{$F(x)=x^4+6x^3+11x^2+6x+C$}
	{\True $F(x)=\dfrac{x^4}{4}+2x^3+\dfrac{11}{2}x^2+6x+C$}
	{$F(x)=x^3+6x^2+11x^2+6x+C$}
	\loigiai{
		Ta có $f(x)=(x+1)(x+2)(x+3)=x^3+6x^2+11x+6$ nên\\
		$\displaystyle F(x)=\displaystyle\int{(x^3+6x^2+11x+6)}\mathrm{\,d}x=\dfrac{x^4}{4}+2x^3+\dfrac{11}{2}x^2+6x+C$.}
\end{ex}

\begin{ex}%[2D4H1-2]
	Tìm nguyên hàm của hàm số $f(x)=(5x+3)^5$.
	\choice
	{$(5x+3)^6+C$}
	{$(5x+3)^4+C$}
	{\True $\dfrac{(5x+3)^6}{30}+C$}
	{$\dfrac{(5x+3)^4}{30}+C$}
	\loigiai{
		$f(x)=(5x+3)^5$ $\displaystyle \Rightarrow \displaystyle\int{f(x)\mathrm{\,d}x=}\displaystyle\int{(5x+ 3)^5\mathrm{\,d}x=}\dfrac{1}{5}\cdot \dfrac{(5x+3)^6}{6}+C=\dfrac{(5x+3)^6}{30}+C$.}
\end{ex}

\begin{ex}%[2D4H1-2]
	Tìm nguyên hàm của hàm số $f(x)=x^2+\dfrac{2}{x^2}$.
	\choice
	{\True $\displaystyle\int{f(x)\mathrm{\,d}x}=\dfrac{x^3}{3}+\dfrac{1}{x}+C$}
	{$\displaystyle\int{f(x)\mathrm{\,d}x}=\dfrac{x^3}{3}-\dfrac{2}{x}+C$}
	{$\displaystyle\int{f(x)\mathrm{\,d}x}=\dfrac{x^3}{3}-\dfrac{1}{x}+C$}
	{$\displaystyle\int{f(x)\mathrm{\,d}x}=\dfrac{x^3}{3}+\dfrac{2}{x}+C$}
	\loigiai{
		Ta có $\displaystyle\int{\left(x^2+\dfrac{2}{x^2}\right)\mathrm{\,d}x}=\dfrac{x^3}{3}-\dfrac{2}{x}+C$.}
\end{ex}

\begin{ex}%[2D4H1-4]
	Tính $\displaystyle\int{\sqrt{x\sqrt{x\sqrt{x}}}\mathrm{\,d}x}$.
	\choice
	{$\dfrac{4}{15}x\sqrt[8]x^7+C$}
	{\True $\dfrac{8}{15}x\sqrt[8]x^7+C$}
	{$\dfrac{8}{15}x\sqrt[8]x+C$}
	{$\dfrac{4}{15}x\sqrt[8]x+C$}
	\loigiai{
		\begin{eqnarray*}
		\displaystyle\int{\sqrt{x\sqrt{x\sqrt{x}}}\mathrm{\,d}x}&=&\displaystyle\int{\sqrt{x\sqrt{x\cdot{x^{\frac{1}{2}}}}}\mathrm{\,d}x}=\displaystyle\int{\sqrt{x\cdot{x^{\frac{3}{4}}}}\mathrm{\,d}x}=\displaystyle\int{x^{\frac{7}{8}}\mathrm{\,d}x}\\&=&\dfrac{x^{\frac{7}{8}+1}}{\dfrac{7}{8}+1}+C=\dfrac{8}{15}x\sqrt[8]x^7+C.	
		\end{eqnarray*}
		}
\end{ex}

\begin{ex}%[2D4H1-4]
	Tính $\displaystyle\int{\dfrac{\sqrt{x}-2\sqrt[3]x^2+1}{\sqrt[4]x}\mathrm{\,d}x}$.
	\choice
	{$x\sqrt[5]x-2x\sqrt[12]x^5+\sqrt[4]x^3+C$}
	{\True $\dfrac{4}{5}x\sqrt[4]x-\dfrac{24}{17}x\sqrt[12]x^5+\dfrac{4}{3}\sqrt[4]x^3+C$}
	{$x\sqrt[5]x-\dfrac{24}{17}x\sqrt[12]x^5+\sqrt[4]x^3+C$}
	{$\dfrac{4}{5}x\sqrt[5]x-2x\sqrt[12]x^5+\dfrac{4}{3}\sqrt[4]x^3+C$}
	\loigiai{
		\begin{eqnarray*}
			\displaystyle\int{\dfrac{\sqrt{x}-2\sqrt[3]x^2+1}{\sqrt[4]x}\mathrm{\,d}x}&=&\displaystyle\int{\dfrac{x^{\frac{1}{2}}-2x^{\frac{2}{3}}+1}{x^{\frac{1}{4}}}\mathrm{\,d}x=}\displaystyle\int{\left(\dfrac{x^{\frac{1}{2}}}{x^{\frac{1}{4}}}-2\dfrac{x^{\frac{2}{3}}}{x^{\frac{1}{4}}}+\dfrac{1}{x^{\frac{1}{4}}}\right)\mathrm{\,d}x}\\
			&=&\displaystyle\int\left(x^{\frac{1}{4}}-2x^{\frac{5}{12}}+x^{-\frac{1}{4}}\right)\mathrm{\,d}x
			=\dfrac{4}{5}x\sqrt[4]x-\dfrac{24}{17}x\sqrt[12]x^5+\dfrac{4}{3}\sqrt[4]x^3+C.
		\end{eqnarray*}
	}
\end{ex}

\begin{ex}%[2D4N1-2]
	Cho hàm số $f(x)=x^2+4$. Mệnh đề nào sau đây đúng?
	
	\choice
	{$\displaystyle{\displaystyle\int f(x)\mathrm{\,d}x=2 x+C}$}
	{$\displaystyle{\displaystyle\int f(x)\mathrm{\,d}x=x^2+4 x+C}$}
	{\True $\displaystyle{\displaystyle\int f(x)\mathrm{\,d}x=\dfrac{x^3}{3}+4 x+C}$}
	{$\displaystyle{\displaystyle\int f(x)\mathrm{\,d}x=x^3+4 x+C}$}
	\loigiai{
		Ta có $f(x)=x^2+4 $ nên $ \displaystyle\int f(x)\mathrm{\,d}x=\dfrac{x^3}{3}+4 x+C$.}
\end{ex}
\begin{ex}%[2D4N1-4]
	Trên khoảng $(0;+\infty)$, cho hàm số $f(x)=x^{\frac{3}{2}}$. Mệnh đề nào sau đây đúng?
	\choice
	{$\displaystyle\int{f(x)}\mathrm{\,d}x=\dfrac{3}{2}x^{\frac{1}{2}}+C$}
	{$\displaystyle\int{f(x)}\mathrm{\,d}x=\displaystyle\int{\sqrt{x^3}}\mathrm{\,d}x$}
	{\True $\displaystyle\int{f(x)}\mathrm{\,d}x=\dfrac{2}{5}x^{\frac{5}{2}}+C$}
	{$\displaystyle\int{f(x)}\mathrm{\,d}x=\dfrac{2}{3}x^{\frac{1}{2}}+C$}
	\loigiai{
		Ta có $\displaystyle\int{f(x)}\mathrm{\,d}x=\displaystyle\int{x^{\frac{3}{2}}}\mathrm{\,d}x=\dfrac{2}{5}x^{\frac{5}{2}}+C$.}
\end{ex}

\begin{ex}%[2D4H1-2]
	Cho hàm số $f(x)=\dfrac{x^4+2}{x^2}$. Mệnh đề nào sau đây đúng?
	\choice
	{$\displaystyle\int{f(x)\mathrm{\,d}x=}\dfrac{x^3}{3}-\dfrac{1}{x}+C$}
	{$\displaystyle\int{f(x)\mathrm{\,d}x=}\dfrac{x^3}{3}+\dfrac{2}{x}+C$}
	{$\displaystyle\int{f(x)\mathrm{\,d}x=}\displaystyle\int{\left(x^2+\dfrac{2}{x^2}\right)}\mathrm{\,d}x$}
	{\True $\displaystyle\int{f(x)\mathrm{\,d}x=}\dfrac{x^3}{3}-\dfrac{2}{x}+C$}
	\loigiai{
		Ta có $\displaystyle\int{f(x)\mathrm{\,d}x=}\displaystyle\int{\dfrac{x^4+2}{x^2}}\mathrm{\,d}x=\displaystyle\int{\left(x^2+\dfrac{2}{x^2}\right)}\mathrm{\,d}x=\dfrac{x^3}{3}-\dfrac{2}{x}+C$.}
\end{ex}

\Closesolutionfile{ans}
\indapan{10}{ans/ans-2-B1-D2-LC}
% \TNTF
\Opensolutionfile{ans}[ans/ans-2-B1-D2-DS]
\begin{ex}%[2D4H1-4]
	Các mệnh đề sau đây đúng hay sai
	\choiceTF
	{\True $\displaystyle\int{(\sqrt[3]x^2+x-2)\mathrm{\,d}x}=\dfrac{3}{5}\sqrt[3]x^5+\dfrac{1}{2}x^2-2x+C$}
	{\True $\displaystyle\int{\dfrac{1}{2023x^{2024}}\mathrm{\,d}x}=\dfrac{1}{2023^2x^{2023}}+C$}
	{$\displaystyle\int{(2x-2024)^2\mathrm{\,d}x}=x-1012+C$}
	{\True $\displaystyle\int{\left(\dfrac{1}{4}x^4+4x^3\right)\mathrm{\,d}x}=\dfrac{1}{20}x^5+\dfrac{4}{3}x^4+C$}
	\loigiai{
	$\displaystyle\int{(\sqrt[3]x^2+x-2)\mathrm{\,d}x}=\dfrac{3}{5}\sqrt[3]x^5+\dfrac{1}{2}x^2-2x+C$.\\
	$\displaystyle\int{\dfrac{1}{2023x^{2024}}\mathrm{\,d}x}=\dfrac{1}{2023}\displaystyle\int{x^{-2024}\mathrm{\,d}x}=\dfrac{1}{2023^2x^{2023}}+C$.\\
	$\displaystyle\int{(2x-2024)^2\mathrm{\,d}x}=\dfrac{(2x-2024)^3}{3}+C$.\\
	$\displaystyle\int{\left(\dfrac{1}{4}x^4+4x^3\right)\mathrm{\,d}x}=\dfrac{1}{20}x^5+\dfrac{4}{3}x^4+C$.}
\end{ex}
\begin{ex}%[2D4H1-2][Lê Công Trường]
	Cho	các mệnh đề sau đây 
	\choiceTF
	{\True $F(x)=\dfrac{x^4}{4}-\dfrac{3}{2}{x^2}+\ln \left| x\right|+C$ là nguyên hàm của hàm số $f(x)=x^3-3x+\dfrac{1}{x}$}
	{$F(x)=\dfrac{(5x+3)^6}{6}+C$ là nguyên hàm của hàm số $f(x)=\left(5x+3\right)^5$}
	{$F(x)=\dfrac{3}{2}x\sqrt x+\dfrac{4}{3}x\sqrt[3]{x}+\dfrac{5}{4}x\sqrt[4]{x}+C$ là nguyên hàm của hàm số $f(x)=\sqrt x+\sqrt[3]{x}+\sqrt[4]{x}$}
	{\True $F(x)=\dfrac{1}{3}{x^3}-2024x+C$ là nguyên hàm của hàm số $f(x)=\dfrac{x^3-2024x}{x}$}
	\loigiai{
		\begin{itemchoice}
			\itemch {\bf Đúng}. Vì $f(x)=x^3-3x+\dfrac{1}{x}$\\
			$\Rightarrow F(x)=\displaystyle\int f(x)dx=\displaystyle\int{(x^3-3x{\rm}+\dfrac{1}{x})dx}$\\
			$=\displaystyle\int{x^3dx}-3\displaystyle\int{xdx}+\displaystyle\int{\dfrac{1}{x}dx}=\dfrac{x^4}{4}-\dfrac{3}{2}{x^2}+\ln \left| x\right|+C$.
			\itemch {\bf Sai.} Vì $f(x)=\left(5x+3\right)^5$ \\
			$\Rightarrow F(x)=\displaystyle\int{f(x)dx=}\displaystyle\int(5x+3)^{5}dx$\\
			$=\displaystyle\int{\rm{(5x+3)}^{\rm{5}}\dfrac{d(5x+3)}{5}=\dfrac{(5x+3)^6}{30}+C}$.
			\itemch {\bf Sai.} Vì $f(x)=\sqrt x+\sqrt[3]{x}+\sqrt[4]{x}$\\
			$\Rightarrow F(x)=\displaystyle\int{\left(\sqrt x+\sqrt[3]{x}+\sqrt[4]{x}\right)}dx=\displaystyle\int{\left(x^{\frac{1}{2}}+x^{\frac{1}{3}}+x^{\frac{1}{4}}\right)}dx$\\
			$=\dfrac{2}{3}{x^{\frac{3}{2}}}+\dfrac{3}{4}{x^{\frac{4}{3}}}+\dfrac{4}{5}{x^{\frac{5}{4}}}+C=\dfrac{2}{3}x\sqrt x+\dfrac{3}{4}x\sqrt[3]{x}+\dfrac{4}{5}x\sqrt[4]{x}+C$.
			\itemch {\bf Đúng.} $f(x)=\dfrac{x^3-2024x}{x}\Rightarrow F(x)=\displaystyle\int{\dfrac{x^3-2024x}{x}dx}=\displaystyle\int\left(x^2-2024\right)dx$\\
			$=\dfrac{1}{3}{x^3}-2024x+C$.
		\end{itemchoice}
	}
\end{ex}
\Closesolutionfile{ans}
\indapan{3}{ans/ans-2-B1-D2-DS}
\Opensolutionfile{ans}[ans/ans-2-B1-D1-KQ]
% \TN
\begin{ex}%[2D4H1-2][Lê Công Trường]
	Hệ số của $x^2$ trong nguyên hàm $F(x)$ của hàm số $f(x)=\dfrac{2}{\sqrt{x}}+3^x+3x-2$ là
	\shortans{$1{,}5$}
	\loigiai{
		$F(x)=\displaystyle\int{\left(\dfrac{2}{\sqrt{x}}+3^x+3x-2\right)\mathrm{\,d}x}=4\sqrt{x}+\dfrac{3^x}{\ln 3}+\dfrac{3}{2}{x^2}-2x+C$.
	}
\end{ex}

\begin{ex}%[2D4H1-2][Lê Công Trường]
	Hệ số của $x^3$ trong nguyên hàm $F(x)$ của hàm số $f(x)=m{x^3}-3x^2+\dfrac{4m}{x^3}+\dfrac{5}{2x}-7m$ ($m$ là tham số) là
	\shortans{$-1$}
	\loigiai{
		$F(x)=\displaystyle\int{\left(m{x^3}-3x^2+\dfrac{4m}{x^3}+\dfrac{5}{2x}-7m\right)\mathrm{\,d}x}=\dfrac{m}{4}{x^4}-x^3-\dfrac{2m}{x^2}-\dfrac{5}{2}\ln {|x|}-7mx+C$
	}
\end{ex}

\begin{ex}% [2D4H1-2][Lê Công Trường]
	Tìm nguyên hàm $F(x)$ của hàm số $f(x)=\dfrac{1}{\sqrt{x}}-\dfrac{2}{\sqrt[3]{x}}$. Tổng hệ số của biến $x$ là
	\shortans{$-1$}
	\loigiai{
		$F(x)=\displaystyle\int f(x)\mathrm{\,d}x=\displaystyle\int\left(\dfrac{1}{\sqrt{x}}-\dfrac{2}{\sqrt[3]{x}}\right)\mathrm{\,d}x=\displaystyle\int\dfrac{1}{\sqrt{x}}\mathrm{\,d}x-\displaystyle\int\dfrac{2}{\sqrt[3]{x}}=\displaystyle\int{x^{\frac{-1}{2}}}\mathrm{\,d}x-\displaystyle\int{2x^{\frac{-1}{3}}}\mathrm{\,d}x$\\
		$=\dfrac{x^{\frac{1}{2}}}{\dfrac{1}{2}}-2.\dfrac{x^{\frac{2}{3}}}{\dfrac{2}{3}}+C=2{x^{\frac{1}{2}}}-3x^{\frac{2}{3}}+C=2\sqrt{x}-3\sqrt[3]{x^2}+C$.
	}
\end{ex}

\begin{ex}%%[2D4H1-2][Lê Công Trường]
	Tìm nguyên hàm $F(x)$ của hàm số $f(x)=\dfrac{(x^2-1)^2}{x^2}$. Tổng hệ số của bậc $3$ và bậc $1$ là (làm tròn đến hàng phần chục).
	\shortans{$-1{,}6$}
	\loigiai{
		$\displaystyle\int  f(x)\mathrm{\,d}x=\displaystyle\int\dfrac{(x^2-1)^2}{x^2}\mathrm{\,d}x=\displaystyle\int\dfrac{x^4-2x^2+1}{x^2}\mathrm{\,d}x=\displaystyle\int\left(x^2-2+\dfrac{1}{x^2}\right)\mathrm{\,d}x$\\
		$=\dfrac{x^3}{3}-2x-\dfrac{1}{x}+C$.
	}
\end{ex}

\begin{ex}%%[2D4H1-2][Lê Công Trường]
	Tính $\displaystyle\int{\left(\dfrac{\left(1-x\right)^3}{\sqrt[3]{x}}\right)\mathrm{\,d}x}$. Giá trị tổng hệ số chứa biến là (làm tròn đến hàng phần trăm).
	\shortans{$0{,}55$}
	\loigiai{$\displaystyle\int\left(\dfrac{\left(1-x\right)^3}{\sqrt[3]{x}}\right)\mathrm{\,d}x=\displaystyle\int\dfrac{1-3x+3x^2-x^3}{x^{\frac{1}{3}}}\mathrm{\,d}x=\displaystyle\int\left(x^{\frac{-1}{3}}-3x^{\frac{2}{3}}+3x^{\frac{5}{3}}-x^{\frac{8}{3}}\right)\mathrm{\,d}x$\\
		$=\dfrac{x^{\frac{2}{3}}}{\dfrac{2}{3}}-3\dfrac{x^{\frac{5}{3}}}{\dfrac{5}{3}}+3\dfrac{x^{\frac{8}{3}}}{\dfrac{8}{3}}-\dfrac{x^{\frac{11}{3}}}{\dfrac{11}{3}}+C=\dfrac{3}{2}{x^{\frac{2}{3}}}-\dfrac{9}{5}{x^{\frac{5}{3}}}+\dfrac{9}{8}{x^{\frac{8}{3}}}-\dfrac{3}{11}{x^{\frac{11}{3}}}+C$.
		
	}
\end{ex}

\begin{ex}%[2D4H1-2][Lê Công Trường]
	Tính $\displaystyle\int{\left(\sqrt[3]{x^2}-\sqrt[4]{x^3}+\sqrt[5]{x^4}\right)\mathrm{\,d}x}$. Giá trị tổng hệ số chứa biến là (làm tròn đến hàng phần trăm).
	\shortans{$0{,}58$}
	\loigiai{
		$\displaystyle\int\left(\sqrt[3]{x^2}-\sqrt[4]{x^3}+\sqrt[5]{x^4}\right)\mathrm{\,d}x=\displaystyle\int\left(x^{\frac{2}{3}}-x^{\frac{3}{4}}+x^{\frac{4}{5}}\right)\mathrm{\,d}x=\dfrac{x^{\frac{5}{3}}}{\dfrac{5}{3}}-\dfrac{x^{\frac{7}{4}}}{\dfrac{7}{4}}+\dfrac{x^{\frac{9}{5}}}{\dfrac{9}{5}}+C$\\
		$=\dfrac{3}{5}{x^{\frac{5}{3}}}-\dfrac{4}{7}{x^{\frac{7}{4}}}+\dfrac{5}{9}{x^{\frac{9}{5}}}+C$.
	}
\end{ex}

\begin{ex}%%[2D4H1-2][Lê Công Trường]
	Tính $\displaystyle\int\left(\sqrt{x}+1\right)\left(x-\sqrt{x}+1\right)\mathrm{\,d}x$. Giá trị tổng hệ số chứa biến là (làm tròn đến hàng phần chục).
	\shortans{$1{,}4 $}
	\loigiai{
		$\left(\sqrt{x}+1\right)\left(x-\sqrt{x}+1\right)=\left(\sqrt{x}+1\right)\left[x-\left(\sqrt{x}-1\right)\right]=x\left(\sqrt{x}+1\right)-\left(\sqrt{x}+1\right)\left(\sqrt{x}-1\right)$\\
		$=x\sqrt{x}+x-\left(x-1\right)=x\sqrt{x}+1$.\\
		Do đó $\displaystyle\int\left(\sqrt{x}+1\right)\left(x-\sqrt{x}+1\right)\mathrm{\,d}x=\displaystyle\int\left(x\sqrt{x}+1\right)\mathrm{\,d}x=\displaystyle\int\left(x^{\frac{3}{2}}+1\right)\mathrm{\,d}x$\\
		$=\dfrac{2}{5}x^{\frac{5}{2}}+x+C$.
	}
\end{ex}

\begin{ex}%%[2D4H1-2][Lê Công Trường]
	Tính $\displaystyle\int{\left(2\sqrt{x}-\dfrac{3}{\sqrt[3]{x}}\right)\mathrm{\,d}x}$. Giá trị tổng hệ số chứa biến là (làm tròn đến hàng phần chục).
	\shortans{$-3{,}1$}
	\loigiai{$\displaystyle\int{\left(2\sqrt {x}-\dfrac{3}{\sqrt[3]{x}}\right)\mathrm{\,d}x=\displaystyle\int{\left(2x^{\frac{1}{2}}-3x^{\frac{-1}{3}}\right)}}\mathrm{\,d}x=\dfrac{4}{3}{x^{\frac{3}{2}}}-\dfrac{9}{2}{x^{\frac{2}{3}}}+C=\dfrac{4}{3}\sqrt[2]{x^3}-\dfrac{9}{2}\sqrt[3]{x^2}+C$.\\
	}
\end{ex}	

\begin{ex}%[2D4H1-2][Lê Công Trường]
	Tính $\displaystyle\int{\dfrac{1}{\sqrt{2x}+\sqrt{3x}}\mathrm{\,d}x}=a\left(\sqrt {b}-\sqrt {c}\right)\sqrt {x}$. Giá trị của tổng $a+b+c$ là
	\shortans{$7$}
	\loigiai{
		Ta có: $\dfrac{1}{\sqrt{2x}+\sqrt{3x}}=\dfrac{\sqrt{3x}-\sqrt{2x}}{\left(\sqrt{3x}-\sqrt{2x}\right)\left(\sqrt{3x}+\sqrt{2x}\right)}=\dfrac{\sqrt{3x}-\sqrt{2x}}{x}=\dfrac{\sqrt {x}}{x}\left(\sqrt 3-\sqrt 2\right)$\\
		$=\left(\sqrt 3-\sqrt 2\right){x^{\frac{-1}{2}}}.$\\
		$\displaystyle\int{\dfrac{1}{\sqrt{2x}+\sqrt{3x}}\mathrm{\,d}x}=\displaystyle\int{\left(\sqrt 3-\sqrt 2\right){x^{\frac{-1}{2}}}\mathrm{\,d}x=}\left(\sqrt 3-\sqrt 2\right)\dfrac{x^{\frac{1}{2}}}{\dfrac{1}{2}}=2\left(\sqrt 3-\sqrt 2\right)\sqrt {x}$.}
\end{ex}

\begin{ex}%%[2D4H1-2][Lê Công Trường]
	Tính $\displaystyle\int{\dfrac{1}{\sqrt{5x}-\sqrt{3x}}\mathrm{\,d}x=\left(\sqrt{a}+\sqrt{b}\right)\sqrt {x}+C}$. Giá trị $a+b$ bằng\\
	\shortans{$8$}
	\loigiai{
		$\dfrac{1}{\sqrt{5x}-\sqrt{3x}}=\dfrac{\sqrt{5x}+\sqrt{3x}}{\left(\sqrt{5x}-\sqrt{3x}\right)\left(\sqrt{5x}+\sqrt{3x}\right)}=\dfrac{\sqrt{5x}+\sqrt{3x}}{2x}=\dfrac{\sqrt {x}}{2x}\left(\sqrt 5+\sqrt 3\right).$\\
		$\displaystyle\int{\dfrac{1}{\sqrt{5x}-\sqrt{3x}}\mathrm{\,d}x}=\displaystyle\int{\dfrac{\sqrt {x}}{2x}\left(\sqrt 5+\sqrt 3\right)\mathrm{\,d}x}=\dfrac{\left(\sqrt 5+\sqrt 3\right)}{2}\displaystyle\int{x^{\frac{-1}{2}}}\mathrm{\,d}x=\dfrac{\left(\sqrt 5+\sqrt 3\right)}{2}\cdot\dfrac{x^{\frac{1}{2}}}{\dfrac{1}{2}}$\\
		$=\left(\sqrt 5+\sqrt 3\right)\sqrt {x}+C$.}
\end{ex}

\begin{ex}%%[2D4H1-2][Lê Công Trường]
	Tính $\displaystyle\int{\left(x^2-1\right)^3\mathrm{\,d}x}$. Giá trị tổng hệ số chứa biến là (làm tròn đến hàng phần chục).
	\shortans{$-0{,}5$}
	\loigiai{
		$\displaystyle\int\left(x^2-1\right)^3\mathrm{\,d}x=\displaystyle\int\left(x^6-3x^4+3x^2-1\right)\mathrm{\,d}x=\dfrac{x^7}{7}-3\dfrac{x^5}{5}+x^3-x+C$.}
\end{ex}

\begin{ex}%%[2D4H1-2][Lê Công Trường]
	Tính $\displaystyle\int{\left(2-x^2\right)^4\mathrm{\,d}x}$. Giá trị tổng hệ số chứa biến là (làm tròn đến hàng phần chục).
	\shortans{$9{,}1 $}
	\loigiai{
		Sử dụng khai triển theo nhị thức Newton, ta có:\\
		$\left(2-x^2\right)^4=x^8-8x^6+24x^4-32x^2+16$.\\
		Do đó\\
		$\displaystyle\int{\left(2-x^2\right)^4\mathrm{\,d}x}=\displaystyle\int{\left(x^8-8x^6+24x^4-32x^2+16\right)}\mathrm{\,d}x$\\
		$=\dfrac{x^8}{9}-\dfrac{8}{7}{x^7}+\dfrac{24}{5}{x^5}-\dfrac{32}{3}{x^3}+16x+C$.}
\end{ex}

\begin{ex}%%[2D4H1-2][Lê Công Trường]
	Tính $\displaystyle\int{\left(x-\sqrt[3]{x}\right)^2\mathrm{\,d}x}$. Giá trị tổng hệ số chứa biến là (làm tròn đến hàng phần chục).
	\shortans{$-1{,}1 $}
	\loigiai{
		$\left(x-\sqrt[3]{x}\right)^2=x^2-2x\sqrt[3]{x}-\sqrt[3]{x^2}=x^2-2x^{\frac{4}{3}}-x^{\frac{2}{3}}$.\\
		$\displaystyle\int{\left(x-\sqrt[3]{x}\right)^2\mathrm{\,d}x}=\displaystyle\int{\left(x^2-2x^{\frac{4}{3}}-x^{\frac{2}{3}}\right)\mathrm{\,d}x=}\dfrac{x^3}{3}-\dfrac{6}{7}{x^{\frac{7}{3}}}-\dfrac{3}{5}{x^{\frac{5}{3}}}+C.$}
\end{ex}

\begin{ex}%%[2D4H1-2][Lê Công Trường]
	Tính $\displaystyle\int{\left(\dfrac{x^2+2\sqrt[3]{x}}{x}\right)^2\mathrm{\,d}x}$.  Giá trị tổng hệ số chứa biến là (làm tròn đến hàng phần chục).
	\shortans{$-8{,}7$}
	\loigiai{
		Ta có: $\left(\dfrac{x^2+2\sqrt[3]{x}}{x}\right)^2=\dfrac{x^4+4x^2\sqrt[3]{x}+4\sqrt[3]{x^2}}{x^2}=x^2+4x^{\frac{1}{3}}+4x^{\frac{-4}{3}}$.\\
		$\displaystyle\int{\left(\dfrac{x^2+2\sqrt[3]{x}}{x}\right)^2\mathrm{\,d}x}=\displaystyle\int{\left(x^2+4x^{\frac{1}{3}}+4x^{\frac{-4}{3}}\right)\mathrm{\,d}x=\dfrac{x^3}{3}+4\dfrac{x^{\frac{4}{3}}}{\dfrac{4}{3}}}+4\dfrac{x^{\frac{-1}{3}}}{\dfrac{-1}{3}}+C$\\
		$=\dfrac{1}{3}{x^3}+3x^{\frac{4}{3}}-12x^{\frac{-1}{3}}+C$.}
\end{ex}

\begin{ex}%[2D4H1-2][Lê Công Trường]
	Tìm $m$ để $F(x)=m{x^3}+(3m+2){x^2}-4x+3$ là một nguyên hàm của hàm số $f(x)=3x^2+10x-4$.
	\shortans{$1$}
	\loigiai{
		$\displaystyle\int{f(x)\mathrm{\,d}x}=\displaystyle\int{\left(3x^2+10x-4\right)}\mathrm{\,d}x=x^3+5x^2-4x+C.$ Suy ra $ m=1$.
	}
\end{ex}

\begin{ex}%%[2D4V1-2][Lê Công Trường]
	Tìm $a,b,c$ để $F(x)=(a{x^2}+bx+c)\sqrt{x^2-4x}$ là một nguyên hàm của hàm số $f(x)=(x-2)\sqrt{x^2-4x}$. Giá trị biểu thức $a+b+c$ bằng.
	\shortans{$-1$}
	\loigiai{
		Đặt $ t=\sqrt{x^2-4x}\Rightarrow{t^2}=x^2-4x\Rightarrow 2t\mathrm{\,d}t=\left(2x-4\right)\mathrm{\,d}x=2\left(x-2\right)\mathrm{\,d}x$.\\
		$\Rightarrow \mathrm{\,d}x=\dfrac{2t\mathrm{\,d}t}{2\left(x-2\right)}=\dfrac{t\mathrm{\,d}t}{x-2}$.\\
		$\displaystyle\int{(x-2)\sqrt{x^2-4x}}\mathrm{\,d}x=\displaystyle\int{t.t.\mathrm{\,d}t=\displaystyle\int{t^2\mathrm{\,d}t}=\dfrac{1}{3}}{t^3}+C=\dfrac{1}{3}\sqrt{\left(x^2-4x\right)^3}+C$\\$=\dfrac{1}{3}\left(x^2-4x\right)\sqrt{x^2-4x}+C=\left(\dfrac{1}{3}{x^2}-\dfrac{4}{3}x\right)\sqrt{x^2-4x}+C$.\\
		Vậy $ a=\dfrac{1}{3};\,\,b=-\dfrac{4}{3};\,\,c=0$.}
\end{ex}

\begin{ex}%%[2D4V1-2][Lê Công Trường]
	Tìm $a,b,c$ để $F(x)=(a{x^2}+bx+c)\sqrt{2x-3}$ là một nguyên hàm của hàm số $f(x)=\dfrac{20x^2-30x+7}{\sqrt{2x-3}}$.  Giá trị biểu thức $a+b+c$ bằng\\
	\shortans{$3$}
	\loigiai{
		Theo định nghĩa nguyên hàm thì $ F'(x)=f(x)$.\\
		Ta có 
		\begin{eqnarray*}
			F'(x) & =& \left(2ax+b\right)\sqrt{2x-3}+(a{x^2}+bx+c)\dfrac{2}{2\sqrt{2x-3}}\\
			&= & \dfrac{\left(2ax+b\right)\left(2x-3\right)+a{x^2}+bx+c}{\sqrt{2x-3}}\\
			&= & \dfrac{5a{x^2}+\left(-6a+3b\right)x-3b+c}{\sqrt{2x-3}}.
		\end{eqnarray*}
		Từ đó ta có $\dfrac{5a{x^2}+\left(-6a+3b\right)x-3b+c}{\sqrt{2x-3}}=\dfrac{20x^2-30x+7}{\sqrt{2x-3}}$.\\
		Sử dụng phương pháp đồng nhất hệ số, ta được\\
		$\heva{
			&5a=20\\
			&-6a+3b=-30\\
			&-3b+c=7.
		}\Leftrightarrow
		\heva{
			&a=4\\
			&b=-2\\
			&c=1.
		}$
	}
\end{ex}
\Closesolutionfile{ans}
\indapan{6}{ans/ans-2-B1-D1-KQ}

\begin{ex}%[2D4H1-3][Lê Công Trường]
	Hàm số $F(x)=\cot x$ là một nguyên hàm của hàm số nào dưới đây trên khoảng $\left(0;\dfrac{\pi}{2}\right)$
	\choice
	{$f_2(x)=\dfrac{1}{\sin^2x}$}
	{$f_1(x)=-\dfrac{1}{\cos^2x}$}
	{$f_4(x)=\dfrac{1}{\cos^2x}$}
	{\True $f_3(x)=-\dfrac{1}{\sin^2x}$}
	\loigiai{
		Có $\displaystyle\int{\dfrac{1}{\sin^2x}\mathrm{\,d}x}=-\cot x+C$ suy ra $F(x)=\cot x$ trên khoảng $\left(0;\dfrac{\pi}{2}\right)$ là một nguyên hàm của hàm số $f_3(x)=-\dfrac{1}{\sin^2x}$.}
\end{ex}

\begin{ex}%[2D4H1-3][Lê Công Trường]
	Cho hàm số $f(x)=1+\sin x$. Khẳng định nào dưới đây đúng?
	\choice
	{\True $\displaystyle\int{f(x){\rm{d}}x}=x-\cos x+C$}
	{$\displaystyle\int{f(x){\rm{d}}x}=x+\sin x+C$}
	{$\displaystyle\int{f(x){\rm{d}}x}=x+\cos x+C$}
	{$\displaystyle\int{f(x){\rm{d}}x}=\cos x+C$}
	\loigiai
	{Ta có $\displaystyle\int{f(x){\rm{d}}x=\displaystyle\int{\left(1+\sin x\right){\rm{d}}x}=\displaystyle\int{1\rm{d}x}+\displaystyle\int{\sin x{\rm{d}}x}=x-\cos x+C}$.}
\end{ex}

\begin{ex}%[2D4H1-3][Lê Công Trường]
	Tìm nguyên hàm $F(x)$ của hàm số $f(x)=\cos ^2\dfrac{x}{2}$
	\choice
	{$F(x)=2\cos\dfrac{x}{2}+C$}
	{\True $F(x)=\dfrac{1}{2}\left(1+\sin x\right)+C$}
	{$F(x)=2\sin\dfrac{x}{2}+C$}
	{$F(x)=\dfrac{1}{2}\left(1-\sin x\right)+C$}
	\loigiai{
		Ta có:$f(x)=\cos ^2\dfrac{x}{2}\Rightarrow F(x)=\displaystyle\int{\cos^2\dfrac{x}{2}\mathrm{\,d}x}=\displaystyle\int{\dfrac{1+\cos x}{2}\mathrm{\,d}x}=\dfrac{1}{2}\displaystyle\int{\left(1+\cos x\right)\mathrm{\,d}x}$\\
		$=\dfrac{1}{2}\left(1+\sin x\right)+C$.}
\end{ex}

\begin{ex}%[2D4H1-3][Lê Công Trường]
	Cho hàm số $f(x)=1-\dfrac{1}{\cos^2x}$. Khẳng định nào dưới đây đúng?
	\choice
	{$\displaystyle\int{f(x){\rm{d}}x}=x+\tan x+C$}
	{$\displaystyle\int{f(x){\rm{d}}x}=x+\cot x+C$}
	{\True $\displaystyle\int{f(x){\rm{d}}x}=x-\tan x+C$}
	{$\displaystyle\int{f(x){\rm{d}}x}=x-\cot x+C$}
	\loigiai
	{
		$\displaystyle\int{f(x){\rm{d}}x}=\displaystyle\int{\left(1-\dfrac{1}{\cos^2x}\right){\rm{d}}x}=x-\tan x+C$.}
\end{ex}

\begin{ex}%%[2D4H1-3][Lê Công Trường]
	Họ nguyên hàm của hàm số $f(x)=\cos x+6x$ là
	\choice
	{\True $\sin x+3x^2+C$}
	{$-\sin x+3x^2+C$}
	{$\sin x+6x^2+C$}
	{$-\sin x+C$}
	\loigiai
	{
		Ta có $\displaystyle\int{f(x){\rm{d}}x=\displaystyle\int{\left(\cos x+6x\right){\rm{d}}x=\sin x+3x^2+C}}$.}
\end{ex}

\begin{ex}%%[2D4H1-3][Lê Công Trường]
	Tìm nguyên hàm của hàm số $f(x)=2\sin x+3x$.
	\choice
	{\True $\displaystyle\int{\left(2\sin x+3x\right)\mathrm{\,d}x=-2\cos x+\dfrac{3}{2}{x^2}+C}$}
	{$\displaystyle\int{\left(2\sin x+3x\right)\mathrm{\,d}x=2\cos x+3x^2+C}$}
	{$\displaystyle\int{\left(2\sin x+3x\right)\mathrm{\,d}x=\sin^2x+\dfrac{3}{2}x+C}$}
	{$\displaystyle\int{\left(2\sin x+3x\right)\mathrm{\,d}x=\sin 2x+\dfrac{3}{2}{x^2}+C}$}
	\loigiai
	{
		$\displaystyle\int{\left(2\sin x+3x\right)\mathrm{\,d}x}=-2\cos x+\dfrac{3}{2}{x^2}+C$}
\end{ex}

\begin{ex}%%[2D4H1-3][Lê Công Trường]
	Tính$\displaystyle\int{\left(x-\sin x\right)}{\rm{d}}x$.
	\choice
	{$\dfrac{x^2}{2}+\sin x+C$}
	{$\dfrac{x^2}{2}-\cos x+C$}
	{$\dfrac{x^2}{2}-\sin x+C$}
	{\True $\dfrac{x^2}{2}+\cos x+C$}
	\loigiai
	{
		Ta có $\displaystyle\int{\left(x-\sin x\right){\rm{d}}x\,\rm{=}\,}\dfrac{x^2}{2}+\cos x+C$.}
\end{ex}
\begin{ex}%[2D4H1-3][Lê Công Trường]
	Họ nguyên hàm của hàm số $f(x)=3x^2+\sin x$ là
	\choice
	{$x^3+\cos x+C$}
	{$6x+\cos x+C$}
	{\True $x^3-\cos x+C$}
	{$6x-\cos x+C$}
	\loigiai
	{
		Ta có $\displaystyle\int\left(3x^2+\sin x\right){\rm{d}}x=x^3-\cos x+C$.}
\end{ex}
\begin{ex}%[2D4H1-3][Lê Công Trường]
	Họ nguyên hàm của hàm số $ f(x)=\dfrac{1}{x}+\sin x$ là
	\choice
	{$\ln x-\cos x+C$}
	{$-\dfrac{1}{x^2}-\cos x+C$}
	{$\ln \left| x\right|+\cos x+C$}
	{\True $\ln \left| x\right|-\cos x+C$}
	\loigiai
	{
		Ta có $\displaystyle\int{f(x){\rm{d}}x}=\displaystyle\int{\left(\dfrac{1}{x}+\sin x\right){\rm{d}}x}=\displaystyle\int{\dfrac{1}{x}{\rm{d}}x}+\displaystyle\int{\sin x{\rm{d}}x}=\ln \left| x\right|-\cos x+C$.}
\end{ex}
\begin{ex}%%[2D4H1-3][Lê Công Trường]
	Cho $\displaystyle\int{f(x)}\,\rm{d}x=-\cos x+C$. Khẳng định nào dưới đây đúng?
	\choice
	{$ f(x)=-\sin x$}
	{$ f(x)=-\cos x$}
	{\True $ f(x)=\sin x$}
	{$ f(x)=\cos x$}
	\loigiai{
		Áp dụng công thức $\smallint{\rm{sin}}x{\rm{\;d}}x=-\rm{cos}x+C$. Suy ra $ f(x)=\rm{sin}x$.}
\end{ex}
\begin{ex}%[2D4H1-3][Lê Công Trường]
	Cho hàm số $ f(x)=\displaystyle\int{\cos\dfrac{x}{2}\sin\dfrac{x}{2}}$. Khẳng định nào dưới đây đúng?
	\choice
	{$\displaystyle\int{\cos\dfrac{x}{2}\sin\dfrac{x}{2}}=\dfrac{1}{2}\sin+C$}
	{$\displaystyle\int{\cos\dfrac{x}{2}\sin\dfrac{x}{2}}=\dfrac{1}{2}\cos x+C$}
	{$\displaystyle\int{\cos\dfrac{x}{2}\sin\dfrac{x}{2}}=-\dfrac{1}{2}\sin x+C$}
	{\True $\displaystyle\int{\cos\dfrac{x}{2}\sin\dfrac{x}{2}}=-\dfrac{1}{2}\cos x+C$}
	\loigiai{
		$\displaystyle\int{\cos\dfrac{x}{2}\sin\dfrac{x}{2}}=\dfrac{1}{2}\displaystyle\int{\sin x}\mathrm{\,d}x=-\dfrac{1}{2}\cos x+C$.}
\end{ex}
\Closesolutionfile{ans}
\indapan{10}{ans/ans-2-B1-D2-TN}
\Opensolutionfile{ans}[ans/ans-2-B1-D2-DS]
% \TNTF
\begin{ex}%%[2D4H1-3][Lê Công Trường]
	Các mệnh đề sau đây đúng hay sai?
	\choiceTF
	{\True $\displaystyle\int{\left(2+\cot^2x\right)\mathrm{\,d}x}=x-\cot x+C$}
	{ $\displaystyle\int{\left(1-\cos^2\dfrac{x}{2}\right)\mathrm{\,d}x}=\dfrac{1}{2}\left(x+\sin x\right)+C$}
	{$\displaystyle\int{\left(\sin\dfrac{x}{2}+\cos\dfrac{x}{2}\right)^2}\mathrm{\,d}x=x+\cos x+C$}
	{ $\displaystyle\int{\left(\sin\dfrac{x}{2}-\cos\dfrac{x}{2}\right)^2}\mathrm{\,d}x=x-\cos x+C$}
	\loigiai{
		\begin{itemchoice}
			\itemch {\bf Đúng}. Vì
			$\displaystyle\int{\left(2+\cot^2x\right)\mathrm{\,d}x}$\\
			$=\displaystyle\int{\left(1+1+\cot^2x\right)\mathrm{\,d}x}=\displaystyle\int{\left(1+\dfrac{1}{\sin^2x}\right)\mathrm{\,d}x}=x-\cot x+C$.
			\itemch {\bf Sai}. Vì $\displaystyle\int{\left(1-\cos^2\dfrac{x}{2}\right)\mathrm{\,d}x}=\displaystyle\int{\sin^2\dfrac{x}{2}\mathrm{\,d}x}=\displaystyle\int{\dfrac{1-\cos x}{2}\mathrm{\,d}x}=\dfrac{1}{2}\left(x-\sin x\right)+C$.
			\itemch {\bf Sai}. Vì $\displaystyle\int{\left(\sin\dfrac{x}{2}+\cos\dfrac{x}{2}\right)^2}\mathrm{\,d}x=\displaystyle\int{\left(1+\sin x\right)}\mathrm{\,d}x=x-\cos x+C$.
			\itemch {\bf Sai}. Vì $\displaystyle\int{\left(\sin\dfrac{x}{2}-\cos\dfrac{x}{2}\right)^2}\mathrm{\,d}x=\displaystyle\int{\left(1-\sin x\right)}\mathrm{\,d}x=x+\cos x+C$.
		\end{itemchoice}
	}
\end{ex}
\Closesolutionfile{ans}
\indapan{2}{ans/ans-2-B1-D2-DS}
\Opensolutionfile{ans}[ans/ans-2-B1-D2-KQ]
% \TN
\begin{ex}%%[2D4H1-3][Lê Công Trường]
	Tìm nguyên hàm $ F(x)$của hàm số $f(x)=2024-2\sin ^2\dfrac{x}{2}$. Hệ số của biến $x$ là
	\shortans{$2023$}
	\loigiai{
		$\Rightarrow F(x)=\displaystyle\int{\left(2024-2\sin^2\dfrac{x}{2}\right)}\mathrm{\,d}x=\displaystyle\int{\left(2023+\cos x\right)}\mathrm{\,d}x=2023x-\sin x+C$.}
\end{ex}
\begin{ex}%%[2D4H1-3][Lê Công Trường]
	Tìm nguyên hàm $ F(x)$của hàm số $f(x)=\dfrac{1}{\sin^2\dfrac{x}{2}\cdot\cos^2\dfrac{x}{2}}==a\cot x+C$. Giá trị $a$ là
	\shortans{$-4$}
	\loigiai{
		Ta có $\dfrac{1}{\sin^2\dfrac{x}{2}\cdot\cos^2\dfrac{x}{2}}=\dfrac{1}{\left(\sin\dfrac{x}{2}\cdot\cos\dfrac{x}{2}\right)^2}=\dfrac{1}{\left(\dfrac{\sin x}{2}\right)^2}=\dfrac{4}{\sin^2x}\cdot$\\
		$ F(x)=\displaystyle\int{f(x)\mathrm{\,d}x=\displaystyle\int{\dfrac{1}{\sin^2\dfrac{x}{2}\cdot\cos^2\dfrac{x}{2}}}}\mathrm{\,d}x=\displaystyle\int{\dfrac{4}{\sin^2x}=-4\cot x+C}$.}
\end{ex}
\begin{ex}%%[2D4H1-3][Lê Công Trường]
	Tìm nguyên hàm $ F(x)$ của hàm số $f(x)=\dfrac{1}{3}{x^2}-2x+\dfrac{1}{2}{\tan ^2}x=\dfrac{x^3}{a}+bx^2+\dfrac{1}{c}x+d\tan x+C$. Giá trị của $a+b+c+d$ là
	\shortans{$6{,}5$}
	\loigiai{$F(x)=\displaystyle\int{f(x)\mathrm{\,d}x}$\\
		$=\displaystyle\int{\left(\dfrac{1}{3}{x^2}-2x+\dfrac{1}{2}{\tan^2}x\right)}\mathrm{\,d}x=\displaystyle\int{\left(\dfrac{1}{3}{x^2}-2x+\dfrac{1}{2}\dfrac{\sin^2x}{\cos^2x}\right)}\mathrm{\,d}x\\
		=\displaystyle\int{\left[\dfrac{1}{3}{x^2}-2x+\dfrac{1}{2}\left(\dfrac{1-\cos^2x}{\cos^2x}\right)\right]}\mathrm{\,d}x=\displaystyle\int{\left[\dfrac{1}{3}{x^2}-2x+\dfrac{1}{2}\left(\dfrac{1}{\cos^2x}-1\right)\right]}\mathrm{\,d}x\\
		=\dfrac{x^3}{9}-x^2+\dfrac{1}{2}\left(\tan x-x\right)+C=\dfrac{x^3}{9}-x^2-\dfrac{1}{2}x+\dfrac{1}{2}\tan x+C$.
	}
\end{ex}
% \begin{ex}%%[2D4V1-3][Lê Công Trường]
% 	Tính $I=\displaystyle\int{x\left(1-\dfrac{\sin^2\dfrac{x}{2}}{2}\right)\mathrm{\,d}x}$. Hệ số của hạng tử $\cos {x}$ của $I$ là
% 	\shortans{$-1$} 
% 	\loigiai{
% 		Đáp án: Ta có $x\left(1-\dfrac{\sin^2\dfrac{x}{2}}{2}\right)=x\left(1-\dfrac{1-cox}{4}\right)=\dfrac{3}{4}x+\dfrac{1}{4}x\cos x$.\\
% 		$\displaystyle\int x\left(1-\dfrac{\sin^2\dfrac{x}{2}}{2}\right)\mathrm{\,d}x=\displaystyle\int\left(\dfrac{3}{4}x+\dfrac{1}{4}x\cos x\right)\mathrm{\,d}x=\displaystyle\int\dfrac{3}{4}x\mathrm{\,d}x+\displaystyle\int\dfrac{1}{4}x\cos x\mathrm{\,d}x$\\
% 		$=\dfrac{3}{8}{x^2}+C_1+\dfrac{1}{4}\displaystyle\int{x\cos x\mathrm{\,d}x.}$\\
% 		Đặt $\heva{
% 			&u=x\Rightarrow \mathrm{\,d}u=\mathrm{\,d}x\\
% 			&dv=\cos x\mathrm{\,d}x\Rightarrow v=\sin x.
% 		}$\\
% 		Sử dụng phương pháp tích phân từng phần, ta có\\
% 		$\displaystyle\int{x\cos x\mathrm{\,d}x}=x\sin x+\displaystyle\int{\sin x\mathrm{\,d}x=x\sin x-\cos x+C_2}$.\\
% 		Vậy $\displaystyle\int{x\left(1-\dfrac{\sin^2\dfrac{x}{2}}{2}\right)\mathrm{\,d}x}=\dfrac{3}{8}{x^2}+x\sin x-\cos x+C.$}
% \end{ex}	
\begin{ex}%%[2D4H1-3][Lê Công Trường]
	Tính $\displaystyle\int{x^2\left(1+\dfrac{1}{x}-\dfrac{\tan^2x}{x^2}\right)\mathrm{\,d}x}=\dfrac{x^m}{n}+\dfrac{x^p}{q}+x+r\tan x+C$. Giá trị biểu thức $P=\dfrac{m}{n}+\dfrac{p}{q}+2r$ là	
	\shortans{$0$} 
	\loigiai{
		$\displaystyle\int{x^2\left(1+\dfrac{1}{x}-\dfrac{\tan^2x}{x^2}\right)\mathrm{\,d}x}=\displaystyle\int{\left(x^2+x-\tan^2x\right)\mathrm{\,d}x}=\dfrac{x^3}{3}+\dfrac{x^2}{2}-(\tan x-x)+C$\\
		$=\dfrac{x^3}{3}+\dfrac{x^2}{2}+x-\tan x+C$.}
\end{ex}
\begin{ex}%%[2D4V1-3][Lê Công Trường]
	Tính $T=\displaystyle\int{x\left(2024-\dfrac{1}{x^3}+\dfrac{\sin x}{x}\right)\mathrm{\,d}x}$. Hệ số của hạng tử $\cos {x}$ của $T$ là
	\shortans{$-1$} 
	\loigiai{
		$\displaystyle\int{x\left(2024-\dfrac{1}{x^3}+\dfrac{\sin x}{x}\right)\mathrm{\,d}x}=\displaystyle\int{\left(2024x-\dfrac{1}{x^2}+\sin x\right)}\mathrm{\,d}x=1012x^2+\dfrac{1}{x}-\cos x+C.$}
\end{ex}
\begin{ex}%Câu 27%[2D4H1-5]
	Tính $R=\displaystyle\int{x^3\left[\dfrac{\left(\sin\dfrac{x}{2}+\cos\dfrac{x}{2}\right)^2}{x^3}-2x+\dfrac{1}{x^{2024}}\right]}\mathrm{\,d}x= ax+b\cos x+c{x^5}-\dfrac{1}{d\cdot x^{2020}}+C$. Giá trị $a+b+c+d+7$ là (làm tròn đến hàng đơn vị)
	\shortans{$2025$} 
	\loigiai{
		Ta có
		\begin{eqnarray*}
			{x^3}\left[\dfrac{\left(\sin\dfrac{x}{2}+\cos\dfrac{x}{2}\right)^2}{x^3}-2x+\dfrac{1}{x^{2024}}\right] &=& \left(\sin\dfrac{x}{2}+\cos\dfrac{x}{2}\right)^2-2x^4+x^{-2021}\\
			&=& \sin ^2\dfrac{x}{2}+\cos^2\dfrac{x}{2}+2\sin\dfrac{x}{2}\cos\dfrac{x}{2}-2x^4+x^{-2021}\\
			&=&1+2\sin x-2x^4+x^{-2021}.
		\end{eqnarray*}
		Khi đó\\
		\begin{eqnarray*}
			\displaystyle\int{x^3\left[\dfrac{\left(\sin\dfrac{x}{2}+\cos\dfrac{x}{2}\right)^2}{x^3}-2x+\dfrac{1}{x^{2024}}\right]}\mathrm{\,d}x&=& \displaystyle\int{\left(1+2\sin x-2x^4+x^{-2021}\right)\mathrm{\,d}x}\\
			&= & x-2\cos x-\dfrac{2}{5}{x^5}-\dfrac{1}{2020x^{2020}}+C.
		\end{eqnarray*}
	}
\end{ex}	
\begin{ex}%[2D4V1-3][Lê Công Trường]
	Tính $\displaystyle\int{x^2\left[\dfrac{1}{x^2\sin^2\dfrac{x}{2}\cdot\cos^2\dfrac{x}{2}}+\dfrac{3}{x^3}-\dfrac{4}{x^4}\right]}\mathrm{\,d}x=a\cot{x}+b\ln \left| x\right|+\dfrac{c}{x}+C$. Giá trị $a+b+c$ là
	\shortans{$3$} 
	\loigiai{
		Ta có\\
		$\dfrac{1}{\sin^2\dfrac{x}{2}\cdot\cos^2\dfrac{x}{2}}=\dfrac{1}{\left(\sin\dfrac{x}{2}\cdot\,\cos\dfrac{x}{2}\right)^2}=\dfrac{1}{\left(\dfrac{\sin x}{2}\right)^2}=\dfrac{4}{\sin^2x}.$\\
		$x^2\left[\dfrac{1}{x^2\sin^2\dfrac{x}{2}\cdot\cos^2\dfrac{x}{2}}+\dfrac{3}{x^3}-\dfrac{4}{x^4}\right]=\dfrac{1}{\sin^2\dfrac{x}{2}\cdot\cos^2\dfrac{x}{2}}+\dfrac{3}{x}-\dfrac{4}{x^2}=\dfrac{4}{\sin^2x}+\dfrac{3}{x}-\dfrac{4}{x^2}$.\\
		Khi đó
		\begin{eqnarray*}
			\displaystyle\int{x^2\left[\dfrac{1}{x^2\sin^2\dfrac{x}{2}\cdot\cos^2\dfrac{x}{2}}+\dfrac{3}{x^3}-\dfrac{4}{x^4}\right]}\mathrm{\,d}x	&= & \displaystyle\int{\left(\dfrac{4}{\sin^2x}+\dfrac{3}{x}-\dfrac{4}{x^2}\right)}\mathrm{\,d}x\\
			&= & -4\cot x+3\ln \left| x\right|+\dfrac{4}{x}+C.
		\end{eqnarray*}
	}
\end{ex}
\Closesolutionfile{ans}
\indapan{6}{ans/ans-2-B1-D2-KQ}

\Opensolutionfile{ans}[ans/ans-C4B1CD1-LC]
% \TN
\begin{ex}%[2D4N2-4]
	Họ nguyên hàm của hàm số $f(x)=e^{3x}$ là hàm số nào sau đây?
	\choice
	{$3e^x+C$}
	{\True $\dfrac{1}{3}e^{3x}+C$}
	{$\dfrac{1}{3}e^{x}+C$}
	{$3e^{3x}+C$}
	\loigiai{
		\textbf{Cách 1:} $\displaystyle\int e^{3x} \mathrm{\,d}x=\displaystyle\int (e^{3})^x \mathrm{\,d}x=\dfrac{(e^3)^x}{\ln e^3}+C=\dfrac{e^{3x}}{3}+C$.\\
		\textbf{Cách 2 (Trắc nghiệm): } $\displaystyle\int e^{3x} \mathrm{\,d}x=\dfrac{1}{3}e^{3x}+C$, với $C$ là hằng số bất kì.
	}
\end{ex}

\begin{ex}%[2D4N2-4]
	Nguyên hàm của hàm số $y=e^{2x-1}$  là
	\choice
	{$2e^{2x-1}+C$}
	{$e^{2x-1}+C$}
	{\True $\dfrac{1}{2}e^{2x-1}+C$}
	{$\dfrac{1}{2}e^{x}+C$}
	\loigiai{
		\textbf{Cách 1:} $\displaystyle\int e^{2x-1} \mathrm{\,d}x=\displaystyle\int e^{-1}(e^{2})^x \mathrm{\,d}x=e^{-1}\dfrac{(e^2)^x}{\ln e^2}+C=\dfrac{e^{2x-1}}{2}+C$.\\
		\textbf{Cách 2:} $\displaystyle\int e^{2x-1} \mathrm{\,d}x=\dfrac{1}{2}\displaystyle\int e^{2x-1} \mathrm{\,d}(2x-1)=\dfrac{1}{2}e^{2x-1}+C$.
	}
\end{ex}

\begin{ex}%[2D4N2-4]
	Cho hàm số $f(x)=e^x+2$. Khẳng định nào dưới đây là \textbf{đúng}?
	\choice
	{$\displaystyle\int f(x) \mathrm{\,d}x=e^{x-2}+C$}
	{\True $\displaystyle\int f(x) \mathrm{\,d}x=e^{x}+2x+C$}
	{$\displaystyle\int f(x) \mathrm{\,d}x=e^{x}+C$}
	{$\displaystyle\int f(x) \mathrm{\,d}x=e^{x}-2x+C$}
	\loigiai{
		Ta có $\displaystyle\int f(x) \mathrm{\,d}x=\displaystyle\int (e^x+2) \mathrm{\,d}x=e^x+2x+C$.
	}
\end{ex}

\begin{ex}%[2D4N2-4]
	Cho hàm số $f(x)=e^x+2x$. Khẳng định nào dưới đây \textbf{đúng}?
	\choice
	{\True $\displaystyle\int f(x) \mathrm{\,d}x=e^{x}+x^2+C$}
	{$\displaystyle\int f(x) \mathrm{\,d}x=e^{x}+C$}
	{$\displaystyle\int f(x) \mathrm{\,d}x=e^{x}-x^2+C$}
	{$\displaystyle\int f(x) \mathrm{\,d}x=e^{x}+2x^2+C$}
	\loigiai{
		Ta có $\displaystyle\int f(x) \mathrm{\,d}x=\displaystyle\int (e^x+2x) \mathrm{\,d}x=e^x+x^2+C$.
	}
\end{ex}

\begin{ex}%[2D4N2-4]
	Tìm nguyên hàm của hàm số  $f(x)=7^x$.
	\choice
	{\True $\displaystyle\int 7^x \mathrm{\,d}x=\dfrac{7^x}{\ln 7}+C$}
	{$\displaystyle\int 7^x \mathrm{\,d}x=7^{x+1}+C$}
	{$\displaystyle\int 7^x \mathrm{\,d}x=\dfrac{7^{x+1}}{x+1}+C$}
	{$\displaystyle\int 7^x \mathrm{\,d}x=7^x\ln 7+C$}
	\loigiai{
		Ta có $\displaystyle\int 7^x \mathrm{\,d}x=\dfrac{7^x}{\ln 7}+C$.
	}
\end{ex}

\begin{ex}%[2D4N2-4]
	Nguyên hàm của hàm số  $f(x)=2^x$ là
	\choice
	{$\displaystyle\int 2^x \mathrm{\,d}x=\ln 2\cdot 2^x+C$}
	{$\displaystyle\int 2^x \mathrm{\,d}x=2^x+C$}
	{\True $\displaystyle\int 2^x \mathrm{\,d}x=\dfrac{2^{x}}{\ln 2}+C$}
	{$\displaystyle\int 2^x \mathrm{\,d}x=\dfrac{2^x}{x+1}\ln 7+C$}
	\loigiai{
		Ta có $\displaystyle\int 2^x \mathrm{\,d}x=\dfrac{2^x}{\ln 2}+C$.
	}
\end{ex}

\begin{ex}%[2D4N2-4]
	Tất cả các nguyên hàm của hàm số  $f(x)=3^{-x}$ là
	\choice
	{\True $-\dfrac{3^{-x}}{\ln 3}+C$}
	{$-3^{-x}+C$}
	{$-3^{-x}\ln 3+C$}
	{$\dfrac{3^{-x}}{\ln 3}+C$}
	\loigiai{
		Ta có $\displaystyle\int 3^{-x} \mathrm{\,d}x=\displaystyle\int (3^{-1})^{x} \mathrm{\,d}x=-\dfrac{3^{-x}}{\ln 3}+C$.
	}
\end{ex}

\begin{ex}%[2D4N2-4]
	Tìm nguyên hàm của hàm số $f(x)=3^x+2x$.
	\choice
	{\True $\displaystyle\int (3^x+2x) \mathrm{\,d}x=\dfrac{3^x}{\ln 3}+x^2+C$}
	{$\displaystyle\int (3^x+2x) \mathrm{\,d}x=3^x\ln 3+x^2+C$}
	{$\displaystyle\int (3^x+2x) \mathrm{\,d}x=\dfrac{3^x}{\ln 3}+x+C$}
	{$\displaystyle\int (3^x+2x) \mathrm{\,d}x=3^x\ln 3+x+C$}
	\loigiai{
		Ta có $\displaystyle\int (3^x+2x) \mathrm{\,d}x=\dfrac{3^x}{\ln 3}+x^2+C$.
	}
\end{ex}

\begin{ex}%[2D4N2-4]
	Họ nguyên hàm của hàm số $f(x)=e^x-2x$ là
	\choice
	{$e^x+x^2+C$}
	{\True $e^x-x^2+C$}
	{$\dfrac{1}{x+1}e^x-x^2+C$}
	{$e^x-2+C$}
	\loigiai{
		Ta có $\displaystyle\int (e^x-2x) \mathrm{\,d}x=e^x-x^2+C$.
	}
\end{ex}

\begin{ex}%[2D4H2-4]
	Tìm nguyên hàm của hàm số $f(x)=e^x\left(2017-\dfrac{2018e^{-x}}{x^5}\right) $.
	\choice
	{$\displaystyle\int f(x) \mathrm{\,d}x=2017e^x-\dfrac{2018}{x^4}+C$}
	{$\displaystyle\int f(x) \mathrm{\,d}x=2017e^x+\dfrac{2018}{x^4}+C$}
	{\True $\displaystyle\int f(x) \mathrm{\,d}x=2017e^x+\dfrac{504{,}5}{x^4}+C$}
	{$\displaystyle\int f(x) \mathrm{\,d}x=2017e^x-\dfrac{504{,}5}{x^4}+C$}
	\loigiai{
		\begin{eqnarray*}
			\displaystyle\int f(x) \mathrm{\,d}x
			&=&\displaystyle\int e^x\left(2017-\dfrac{2018e^{-x}}{x^5}\right)\mathrm{\,d}x\\
			&=&\displaystyle\int \left(2017e^x-\dfrac{2018}{x^5}\right)\mathrm{\,d}x\\
			&=&2017e^x+\dfrac{504{,}5}{x^4}+C
		\end{eqnarray*}
	}
\end{ex}

\begin{ex}%[2D4H2-4]
	Họ nguyên hàm của hàm số $y=e^x\left(2+\dfrac{e^{-x}}{\cos^2x}\right) $ là
	\choice
	{\True $2e^x+\tan x+C$}
	{$2e^x-\tan x+C$}
	{$2e^x-\dfrac{1}{\cos x}+C$}
	{$2e^x+\dfrac{1}{\cos x}+C$}
	\loigiai{
		Ta có $\displaystyle\int y \mathrm{\,d}x=\displaystyle\int e^x\left(2+\dfrac{e^{-x}}{\cos^2x}\right)\mathrm{\,d}x=\displaystyle\int \left(2e^x+\dfrac{1}{\cos^2x}\right)\mathrm{\,d}x=2e^x+\tan x+C$.
	}
\end{ex}

\begin{ex}%[2D4N2-4]
	Tìm họ nguyên hàm của hàm số $y=x^2-3^x+\dfrac{1}{x}$.
	\choice
	{$\dfrac{x^3}{3}-\dfrac{3^x}{\ln 3}-\dfrac{1}{x^2}+C,\,C\in \mathbb{R}$}
	{$\dfrac{x^3}{3}-3^x+\dfrac{1}{x^2}+C,\,C\in \mathbb{R}$}
	{\True $\dfrac{x^3}{3}-\dfrac{3^x}{\ln 3}+\ln \left|x\right|+C,\,C\in \mathbb{R}$}
	{$\dfrac{x^3}{3}-\dfrac{3^x}{\ln 3}-\ln \left|x\right|+C,\,C\in \mathbb{R}$}
	\loigiai{
		Ta có $\displaystyle\int \left( x^2-3^x+\dfrac{1}{x}\right)  \mathrm{\,d}x=\dfrac{x^3}{3}-\dfrac{3^x}{\ln 3}+\ln \left|x\right|+C,\,C\in \mathbb{R}$.
	}
\end{ex}

\begin{ex}%[2D4N2-4]
	Khẳng định nào dưới đây \textbf{đúng}?
	\choice
	{$\displaystyle\int e^x \mathrm{\,d}x=xe^x+C$}
	{$\displaystyle\int e^x \mathrm{\,d}x=e^{x+1}+C$}
	{$\displaystyle\int e^x \mathrm{\,d}x=-e^{x+1}+C$}
	{\True $\displaystyle\int e^x \mathrm{\,d}x=e^x+C$}
	\loigiai{
		Ta có $\displaystyle\int e^x \mathrm{\,d}x=e^x+C$.
	}
\end{ex}

\begin{ex}%[2D4N2-4]
	Cho hàm số $f(x)=1+e^{2x}$. Khẳng định nào dưới đây \textbf{đúng}?
	\choice
	{$\displaystyle\int f(x) \mathrm{\,d}x=x+\dfrac{1}{2}e^x+C$}
	{$\displaystyle\int f(x) \mathrm{\,d}x=x+2e^{2x}+C$}
	{\True $\displaystyle\int f(x) \mathrm{\,d}x=x+\dfrac{1}{2}e^{2x}+C$}
	{$\displaystyle\int f(x) \mathrm{\,d}x=x+e^{2x}+C$}
	\loigiai{
		Ta có $\displaystyle\int (1+e^{2x}) \mathrm{\,d}x=x+\dfrac{1}{2}e^{2x}+C$.
	}
\end{ex}
\Closesolutionfile{ans}
\indapan{6}{ans/ans-C4B1CD1-LC}
% \TNTF
\Opensolutionfile{ans}[ans/ans-C4B1CD1-DS]
\begin{ex}%[2D4N2-4]
	Các mệnh đề sau đây \textbf{đúng} hay \textbf{sai}?
	\choiceTF
	{$\displaystyle\int \dfrac{1}{x} \mathrm{\,d}x=\ln x+C$}
	{\True $\displaystyle\int \dfrac{1}{\cos^2x} \mathrm{\,d}x=\tan x+C$}
	{\True $\displaystyle\int \sin x \mathrm{\,d}x=-\cos x+C$}
	{\True $\displaystyle\int e^x \mathrm{\,d}x=e^x+C$}
	\loigiai{
		\begin{itemchoice}
			\itemch Ta có $\displaystyle\int \dfrac{1}{x} \mathrm{\,d}x=\ln \left|x\right|+C$.
			\itemch Ta có $\displaystyle\int \dfrac{1}{\cos^2x} \mathrm{\,d}x=\tan x+C$
			\itemch Ta có $\displaystyle\int \sin x \mathrm{\,d}x=-\cos x+C$.
			\itemch Ta có $\displaystyle\int e^x \mathrm{\,d}x=e^x+C$.
		\end{itemchoice}
	}
\end{ex}

\begin{ex}%[2D4N2-4]
	Các mệnh đề sau đây \textbf{đúng} hay \textbf{sai}?
	\choiceTF
	{\True $\displaystyle\int \cos x \mathrm{\,d}x=\sin x+C$}
	{\True $\displaystyle\int x^e \mathrm{\,d}x=\dfrac{x^{e+1}}{e+1}+C$}
	{\True $\displaystyle\int \dfrac{1}{x} \mathrm{\,d}x=\ln \left|x\right|+C$}
	{$\displaystyle\int e^x \mathrm{\,d}x=\dfrac{e^{x+1}}{x+1}+C$}
	\loigiai{
		\begin{itemchoice}
			\itemch Ta có $\displaystyle\int \cos x \mathrm{\,d}x=\sin x+C$.
			\itemch Ta có $\displaystyle\int x^e \mathrm{\,d}x=\dfrac{x^{e+1}}{e+1}+C$
			\itemch Ta có $\displaystyle\int \dfrac{1}{x} \mathrm{\,d}x=\ln \left|x\right|+C$.
			\itemch Ta có $\displaystyle\int e^x \mathrm{\,d}x=e^x+C$.
		\end{itemchoice}
	}
\end{ex}

\begin{ex}%[2D4H2-4]
	Các mệnh đề sau đây \textbf{đúng} hay \textbf{sai}?
	\choiceTF
	{$\displaystyle\int 2^x \mathrm{\,d}x=2^x\ln 2+C$}
	{\True $\displaystyle\int e^{2x} \mathrm{\,d}x=\dfrac{e^{2x}}{2}+C$}
	{$\displaystyle\int e^x(e^x-1) \mathrm{\,d}x=\dfrac{1}{2}e^{2x}+e^x+C$}
	{\True $\displaystyle\int e^{3x}\cdot 3^x \mathrm{\,d}x=\dfrac{(3e^{3})^x}{3+\ln 3}+C$}
	\loigiai{
		\begin{itemchoice}
			\itemch Ta có $\displaystyle\int 2^x \mathrm{\,d}x=\dfrac{2^x}{\ln 2}+C$.
			\itemch Ta có $\displaystyle\int e^{2x} \mathrm{\,d}x=\dfrac{e^{2x}}{2}+C$
			\itemch Ta có $\displaystyle\int e^x(e^x-1) \mathrm{\,d}x=\displaystyle\int (e^{2x}-e^x) \mathrm{\,d}x=\dfrac{1}{2}e^{2x}-e^x+C$.
			\itemch Ta có $\displaystyle\int e^{3x}\cdot 3^x \mathrm{\,d}x=\displaystyle\int (3e^{3})^x \mathrm{\,d}x=\dfrac{(3e^{3})^x}{\ln (3e^3)}+C=\dfrac{(3e^{3})^x}{3+\ln (3)}+C$.
		\end{itemchoice}
	}
\end{ex}
\Closesolutionfile{ans}
\indapan{3}{ans/ans-C4B1CD1-DS}
% \TNSA
\Opensolutionfile{ans}[ans/ans-C4B1CD1-KQ]
\begin{ex}%[2D4H2-4]
	Biết rằng $\displaystyle\int (2^x+3^x) \mathrm{\,d}x=\dfrac{2^x}{\ln a}+\dfrac{3^x}{\ln b}+C,\,a,b\in \mathbb{Z}$. Tính $P=a+b$.
	\shortans[4]{$5$}
	\loigiai{
		Ta có $\displaystyle\int (2^x+3^x) \mathrm{\,d}x=\dfrac{2^x}{\ln 2}+\dfrac{3^x}{\ln 3}+C$.\\
		Do đó $a=2$, $b=3\Rightarrow P=a+b=2+3=5$.
	}
\end{ex}

\begin{ex}%[2D4H2-4]
	Cho $\displaystyle\int e^{3x+2024} \mathrm{\,d}x=\dfrac{a}{b}e^{cx+d}+C$ với $a,b,c,d\in \mathbb{Z}$ và $\dfrac{a}{b}$ là phân số tối giãn . Tính giá trị của biểu thức $P=a+b-c+d$.
	\shortans[4]{$2025$}
	\loigiai{
		Ta có $\displaystyle\int e^{3x+2024} \mathrm{\,d}x=\dfrac{1}{3}e^{3x+2024}+C$.\\
		Do đó $a=1$, $b=3$, $c=3$, $d=2024\Rightarrow P=a+b-c+d=1+3-3+2024=2025$.
	}
\end{ex}

\begin{ex}%[2D4H2-4]
	Biết rằng $\displaystyle\int 3^{x+2}\cdot 2^{2x+1} \mathrm{\,d}x=\dfrac{a\cdot 12^x}{b\ln 2+c\ln 3}+C$ với $a,b,c\in \mathbb{Z}$. Tính giá trị của biểu thức $P=\dfrac{a}{b+c}$.
	\shortans[4]{$6$}
	\loigiai{
		Ta có $\displaystyle\int 3^{x+2}\cdot 2^{2x+1} \mathrm{\,d}x=\displaystyle\int 3^2\cdot 3^x\cdot 2\cdot4^x \mathrm{\,d}x=\displaystyle\int 18\cdot 12^x \mathrm{\,d}x=18\cdot \dfrac{12^x}{\ln 12}+C=\dfrac{18\cdot 12^x}{2\ln 2+\ln 3}+C$.\\
		Do đó $a=18$, $b=2$, $c=1\Rightarrow P=\dfrac{a}{b+c}=\dfrac{18}{2+1}=6$.
	}
\end{ex}

\begin{ex}%[2D4H2-4]
	Biết rằng $\displaystyle\int (3^{x}+5^{x})^2\mathrm{\,d}x=\dfrac{9^x}{a\ln 3}+\dfrac{30^x}{b\ln 5+c\ln 2+d\ln 3}+\dfrac{25^x}{e\ln 5}+C$. Tính giá trị của biểu thức $P=a+b+c+d+e$.
	\shortans[4]{$7$}
	\loigiai{
		\begin{eqnarray*}
			\displaystyle\int (3^{x}+5^{x})\mathrm{\,d}x&=&\displaystyle\int (9^{x}+30^{x}+25^{x})\mathrm{\,d}x\\
			&=&
			\dfrac{9^x}{\ln 9}+\dfrac{30^x}{\ln 30+\ln 25}+C\\
			&=&\dfrac{9^x}{2\ln 3}+\dfrac{30^x}{\ln 5+\ln 2+\ln 3}+\dfrac{25^x}{2\ln 5}+C.
		\end{eqnarray*}
		Do đó $a=2$, $b=c=d=1$, $e=2\Rightarrow P=a+b+c+d+e=7$.
	}
\end{ex}

\begin{ex}%[2D4H2-4]
	Cho $\displaystyle\int \dfrac{e^{3x}+1}{e^x+1}\mathrm{\,d}x=\dfrac{a}{b}e^{2x}+ce^x+dx+C$ với $a,b,c,d\in \mathbb{Z}$ và $\dfrac{a}{b}$ là phân số tối giãn. Tính giá trị của biểu thức $P=a^2+b^2+c^2+d^2$.
	\shortans[4]{$7$}
	\loigiai{
		Ta có $\displaystyle\int \dfrac{e^{3x}+1}{e^x+1}\mathrm{\,d}x=\displaystyle\int \dfrac{(e^{x}+1)(e^{2x}-e^x+1)}{e^x+1}\mathrm{\,d}x=\displaystyle\int (e^{2x}-e^x+1)\mathrm{\,d}x=\dfrac{1}{2}e^{2x}-e^x+x+C$.\\
		Do đó $a=d=1$, $b=2$, $c=-1\Rightarrow P=a^2+b^2+c^2+d^2=7$.
	}
\end{ex}

\begin{ex}%[2D4H2-4]
	Biết rằng $\displaystyle\int (e^x+e^{-x})^2\mathrm{\,d}x=\dfrac{1}{m}e^{2x}+\dfrac{1}{n}e^{-2x}+px+C$ với $m,m,p\in \mathbb{Z}$. Tính giá trị của biểu thức $P=m+n+p$.
	\shortans[4]{$2$}
	\loigiai{
		Ta có $\displaystyle\int (e^x+e^{-x})^2\mathrm{\,d}x=\displaystyle\int (e^{2x}+e^{-2x}+2)\mathrm{\,d}x=\dfrac{1}{2}e^{2x}-\dfrac{1}{2}e^{2x}+2x+C$.\\
		Do đó $m=p=2$, $n=-2\Rightarrow P=m+n+p=2$.
	}
\end{ex}

\begin{ex}%[2D4H2-4]
	Biết rằng $\displaystyle\int \dfrac{e^{2x}-1}{1-e^{-x}}\mathrm{\,d}x=\dfrac{1}{m}e^{nx}+pe^x+C$ với $m,m,p\in \mathbb{Z}$. Tính giá trị của biểu thức $P=m+n-p$.
	\shortans[4]{$5$}
	\loigiai{
		Ta có $\displaystyle\int \dfrac{e^{2x}-1}{1-e^{-x}}\mathrm{\,d}x=\displaystyle\int \dfrac{e^x(e^x-1)(e^x+1)}{e^x-1}\mathrm{\,d}x=\displaystyle\int e^x(e^x-1) \mathrm{\,d}x$\\$=\displaystyle\int (e^{2x}-e^x) \mathrm{\,d}x=\dfrac{1}{2}e^{2x}-e^x+C$.\\
		Do đó $m=n=2$, $p=-1\Rightarrow P=m+n-p=5$.
	}
\end{ex}

\begin{ex}%[2D4H2-4]
	Biết rằng $F(x)=(ax+b)\cdot e^x$ là một nguyên hàm của hàm số $f(x)=(4x-1)\cdot e^x$. Tính giá trị biểu thức $P=a+b$.
	\shortans[4]{$-1$}
	\loigiai{
		Ta có $F'(x)=a\cdot e^x+(ax+b)\cdot e^x=e^x(ax+a+b)$.\\
		Mà $F'(x)=f(x)\Rightarrow \heva{&a=4\\&a+b=-1}\Rightarrow \heva{&a=4\\&b=-5.}$\\
		Vậy $P=a+b=-1$.
	}
\end{ex}

\begin{ex}%[2D4H2-4]
	Biết rằng $F(x)=8e^x+\dfrac{na^x}{\ln a}+p\cos x$ (với $m,n,p\in \mathbb{Z}$) là một nguyên hàm của hàm số $f(x)=me^x+2a^x-2\sin x$. Tính giá trị của biểu thức $P=m+n+p$.
	\shortans[4]{$12$}
	\loigiai{
		Ta có $F'(x)=8e^x+\dfrac{na^x}{\ln a}\cdot \ln a-p\sin x=8e^x+na^x-p\sin x$.\\
		Mà $F'(x)=f(x)\Rightarrow m=8, n=2, p=2$.\\
		Vậy $P=m+n+p=12$.
	}
\end{ex}

\begin{ex}%[2D4H2-4]
	Biết rằng  $F(x)=(ax^2+bx+c)e^{-2x}$ (với $a,b,c\in \mathbb{R}$) là một nguyên hàm của hàm số $f(x)=(-2x^2+8x-7)e^{-2x}$. Tính giá trị biểu thức $P=a+b+c$.
	\shortans[4]{$-7$}
	\loigiai{
		Ta có $F'(x)=(2ax+b)e^{-2x}-2(ax^2+bx+c)e^{-2x}=\left[-2ax^2+2(a-b)x+b-2c \right]e^{-2x}$.\\
		Mà $F'(x)=f(x)\Rightarrow \heva{&-2a=-2\\&2(a-b)=8\\&b-2c=7}\Rightarrow \heva{&a=1\\&b=-3\\&c=-5.}$\\
		Vậy $P=a+b+c=1-3-5=-7$.
	}
\end{ex}
\Closesolutionfile{ans}
\indapan{6}{ans/ans-C4B1CD1-KQ}

% % \subsection{NGUYÊN HÀM CÓ ĐIỀU KIỆN}
\begin{dang}{Tìm nguyên hàm khi biết giá trị nguyên hàm}
	Phương pháp: Tìm $F(x)=\int f(x)\mathrm{\,d}x$. Sau đó dựa vào $F(x_0)=a$ để suy ra $C$.
\end{dang}
\Opensolutionfile{ans}[ans/ans-C4B1CD2-LC]
% \TN
\begin{ex}%[2D4H2-2]
Hàm số $F(x)$ là một nguyên hàm của hàm số $f(x)=\dfrac{1}{x}$ trên $(-\infty;0)$ thỏa mãn $F(-2)=0$. Khẳng định nào sau đây \textbf{đúng}?
\choice
{\True $F(x)=\ln \left(-\dfrac{x}{2} \right),\,\forall x\in (-\infty;0)$}
{$F(x)=\ln \left|x\right|+C,\,\forall x\in (-\infty;0)$ với $C$ là một số thực bất kì}
{$F(x)=\ln \left|x\right|+\ln 2,\,\forall x\in (-\infty;0)$}
{$F(x)=\ln \left(-x\right)+C,\,\forall x\in (-\infty;0)$ với $C$ là một số thực bất kì}
\loigiai{
Ta có $F(x)=\displaystyle\int \dfrac{1}{x}\mathrm{\,d}x=\ln \left|x\right|+C=\ln (-x)+C,\,\forall x\in (-\infty;0)$.\\
Lại có $F(-2)=0\Rightarrow \ln 2+C=0\Rightarrow C=-\ln 2$.\\
Do đó $F(x)=\ln (-x)-\ln 2=\ln \left(-\dfrac{x}{2}\right)$.
}
\end{ex}

\begin{ex}%[2D4H2-4]
Biết $F(x)$ là một nguyên hàm của hàm số $f(x)=e^{2x}$ và $F(0)=0$. Giá trị của $F(\ln 3)$ bằng
\choice
{$2$}
{$6$}
{$8$}
{\True $4$}
\loigiai{
Ta có $F(x)=\displaystyle\int e^{2x}\mathrm{\,d}x=\dfrac{1}{2}e^{2x}+C$.\\
Lại có $F(0)=0\Rightarrow \dfrac{1}{2}+C=0\Rightarrow C=-\dfrac{1}{2}$.\\
Do đó $F(\ln 3)=\dfrac{1}{2}e^{2\ln 3}-\dfrac{1}{2}=4$.
}
\end{ex}

\begin{ex}%[2D4H2-4]
Cho $F(x)$ là một nguyên hàm của $f(x)=2^x+x+1$. Biết $F(0)=1$. Giá trị của $F(-1)$ bằng
\choice
{$F(-1)=\dfrac{1}{2\ln 2}$}
{\True $F(-1)=\dfrac{1}{2}-\dfrac{1}{2\ln 2}$}
{$F(-1)=1+\dfrac{1}{2\ln 2}$}
{$F(-1)=\dfrac{1}{2}-\dfrac{1}{\ln 2}$}
\loigiai{
Ta có $F(x)=\displaystyle\int (2^x+x+1)\mathrm{\,d}x=\dfrac{2^x}{\ln 2}+\dfrac{x^2}{2}+x+C$.\\
Lại có $F(0)=1\Rightarrow \dfrac{1}{\ln 2}+C=1\Rightarrow C=1-\dfrac{1}{\ln 2}$.\\
Do đó $F(-1)=\dfrac{1}{2\ln 2}+\dfrac{1}{2}-1+1-\dfrac{1}{\ln 2}=\dfrac{1}{2}-\dfrac{1}{2\ln 2}$.
}
\end{ex}

\begin{ex}%[2D4H2-3]
Tìm nguyên hàm $F(x)$ của hàm số $f(x)=\sin x+\cos x$ thoả mãn $F\left(\dfrac{\pi}{2}\right)=2$.
\choice
{$F(x)=-\cos x+\sin x+3$}
{$F(x)=-\cos x+\sin x-1$}
{\True $F(x)=-\cos x+\sin x+1$}
{$F(x)=\cos x-\sin x+3$}
\loigiai{
Ta có $F(x)=\displaystyle\int (\sin x+\cos x)\mathrm{\,d}x=-\cos x+\sin x+C$.\\
Lại có $F\left(\dfrac{\pi}{2}\right)=2\Rightarrow -\cos\dfrac{\pi}{2}+\sin\dfrac{\pi}{2}+C=2\Rightarrow C=1$.\\
Do đó $F(x)=-\cos x+\sin x+1$.
}
\end{ex}

\begin{ex}%[2D4H2-4]
Cho $F(x)$ là một nguyên hàm của hàm số $f(x)=e^x+2x$ thỏa mãn $F(0)=\dfrac{3}{2}$. Tìm $F(x)$.
\choice
{\True $F(x)=e^x+x^2+\dfrac{1}{2}$}
{$F(x)=e^x+x^2+\dfrac{5}{2}$}
{$F(x)=e^x+x^2+\dfrac{3}{2}$}
{$F(x)=e^x+x^2-\dfrac{1}{2}$}
\loigiai{
Ta có $F(x)=\displaystyle\int (e^x+2x)\mathrm{\,d}x=e^x+x^2+C$.\\
Lại có $F(0)=\dfrac{3}{2}\Rightarrow 1+C=\dfrac{3}{2}\Rightarrow C=\dfrac{1}{2}$.\\
Do đó $F(x)=e^x+x^2+\dfrac{1}{2}$.
}
\end{ex}

\begin{ex}%[2D4H2-2]
Cho hàm số $f(x)=\heva{&2x-1&\text{khi}\quad&x\ge 1\\&3x^2-2&\text{khi}\quad&x<1}$, giả sử $F$ là nguyên hàm của  $f$ trên $\mathbb{R}$ thỏa mãn $F(0)=2$. Giá trị của $F(-1)+2F(2)$ bằng
\choice
{\True $9$}
{$15$}
{$11$}
{$6$}
\loigiai{
Ta có $\displaystyle\int (2x-1)\mathrm{\,d}x=x^2-x+C_1$ và $\displaystyle\int (3x^2-2)\mathrm{\,d}x=x^3-2x+C_2$.\\
Suy ra $F(x)=\displaystyle\int f(x)\mathrm{\,d}x=\heva{&x^2-x+C_1&\text{khi}\quad&x\ge 1\\&x^3-2x+C_2&\text{khi}\quad&x<1.}$ 
Lại có $F(0)=2\Rightarrow C_2=2$.\\
Mặt khác hàm số $F$ là nguyên hàm của $f$ trên $\mathbb{R}$ nên $y=F(x)$ liên tục tại $x=1$.\\
Suy ra  $\lim\limits_{ x\to 1^{+}} F(x)=\lim\limits_{ x\to 1^{-}} F(x)\Rightarrow C_1=1$.\\
Khi đó ta có $F(x)=\heva{&x^2-x+1&\text{khi}\quad&x\ge 1\\&x^3-2x+2&\text{khi}\quad&x<1}\Rightarrow \heva{&F(-1)=3\\&F(2)=3.}$  \\
Vậy $F(-1)+2F(2)=9$.  
}
\end{ex}

\Opensolutionfile{ans}[ans/ans-C4B1CD2-CAU7_8-LC]
\setcounter{ex}{6}
\begin{ex}%[2D4H1-2]
	Cho hàm số $f(x)=\heva{&2x+3 &\text{khi } &x\ge 1\\ &3x^2+2 &\text{khi } &x<1.}$ Giả sử $F$ là nguyên hàm của hàm số $f$ trên $\mathbb{R}$ thỏa mãn $F(0)=2$. Giá trị của $F(-1)+2F(2)$ bằng
	\choice
	{$23$}
	{$11$}
	{$10$}	 
	{\True $21$}
	\loigiai{
		Khi $x\ge 1$ thì $F(x)=\displaystyle\int f(x)\mathrm{\,d}x=\displaystyle\int (2x+3)\mathrm{\,d}x=x^2+3x+\mathrm{C}_1$.\\
		Khi $x<1$ thì $F(x)=\displaystyle\int f(x)\mathrm{\,d}x=\displaystyle\int (3x^2+2)\mathrm{\,d}x=x^3+2x+C_2$.\\
		Theo giả thiết $F(0)=2 \Rightarrow C_2=2$.\\
		Ta có $\lim\limits_{x \to 1^+} f(x)=\lim\limits_{x \to 1^-} f(x)=f(1)=5$ nên hàm số $f(x)$ liên tục tại $x=1$.\\
		Suy ra hàm số $f(x)$ liên tục trên $\mathbb{R}$.\\
		Do đó hàm số $F(x)$ liên tục trên $\mathbb{R} \Rightarrow \lim\limits_{x \to 1^+} F(x)=\lim\limits_{x \to 1^-} F(x) \Rightarrow C_1+4=C_2+3 \Rightarrow C_1=1$.\\
		Vậy $F(-1)+2F(2)=-3+C_2+2(10+C_1)=21$.
	}
\end{ex}
\begin{ex}%[2D4H1-2]
	Cho hàm số $f(x)=\heva{&2x+2 &\text{khi } &x\ge 1\\ &3x^2+1 &\text{khi } &x<1.}$ Giả sử $F$ là nguyên hàm của hàm số $f$ trên $\mathbb{R}$ thỏa mãn $F(0)=2$. Giá trị của $F(-1)+2F(2)$ bằng
	\choice
	{\True $18$}
	{$20$}
	{$9$}	 
	{$24$}
	\loigiai{
		$F$ là nguyên hàm của $f$ trên $\mathbb{R}$ nên $F(x)=\heva{&x^2+2x+C_1 &\text{khi } &x\ge 1\\ &x^3+x+C_2 &\text{khi } &x<1.}$\\
		Ta có $F(0)=2 \Rightarrow C_2=2$. \quad $(1)$\\
		Do $F$ liên tục tại $x=1$ nên $\lim\limits_{x \to 1^+} F(x)=\lim\limits_{x \to 1^-} F(x)=F(1)$.\\
		$\Leftrightarrow C_1+3=C_2+2 \mathop  \Leftrightarrow \limits^{(1)} C_1+3=4 \Leftrightarrow C_1=1$.\\
		Do đó $F(x)=\heva{&x^2+2x+1 &\text{khi } &x\ge 1\\ &x^3+x+2 &\text{khi } &x<1.}$\\
		Suy ra $F(-1)+2F(2)=18$.
	}
\end{ex}

\begin{ex}%[2D4H1-2]
	Cho hàm số $y=f(x)$ có đạo hàm là $f'(x)=12x^2+2, \forall x\in \mathbb{R}$ và $f(1)=3$. Biết $F(x)$ là nguyên hàm của $f(x)$ thỏa mãn $F(0)=2$, khi đó $F(1)$ bằng
	\choice
	{$-3$}
	{\True $1$}
	{$2$}
	{$7$}
	\loigiai{
		Ta có $f'(x)=12x^2+2, \forall x\in \mathbb{R} \Rightarrow f(x)=4x^3+2x+C_1$.\\
		Mà $f(1)=3\Rightarrow 3=6+C_1\Rightarrow C_1=-3\Rightarrow f(x)=4x^3+2x-3\Rightarrow F(x)=x^4+x^2-3x+C_2$.\\
		Lại có $F(0)=2\Rightarrow C_2=2\Rightarrow F(x)=x^4+x^2-3x+2$.\\
		Do đó $F(1)=1$.\\
		\textbf{Cách khác:}\\
		Ta có $F(1)=\displaystyle \int\limits_0^1 {f(x)\mathrm{\,d}x}+F(0)=\displaystyle \int\limits_0^1{(4x^3+2x-3)\mathrm{\,d}x}+2=-1+2=1$.}
\end{ex}
\begin{ex}%[2D4H1-3]
	Cho hàm số $f(x)$ thỏa mãn $f'(x)=3-5\sin x$ và $f(0)=10$. Mệnh đề nào dưới đây \textbf{đúng}?
	\choice
	{$f(x)=3x-5\cos x+15$}
	{$f(x)=3x-5\cos x+2$}
	{\True $f(x)=3x+5\cos x+5$}
	{$f(x)=3x+5\cos x+2$}
	\loigiai{
		Ta có $f(x)=\displaystyle \int (3-5\sin x)\mathrm{\,d}x=3x+5\cos x+C$.\\
		Theo giả thiết $f(0)=10$ nên $5+C=10\Rightarrow C=5$.\\
		Vậy $f(x)=3x+5\cos x+5$.
	}
\end{ex}
\begin{ex}%[2D4H1-4]
	Hàm số $f(x)$ có đạo hàm liên tục trên $\mathbb{R}$ và $f'(x)=2\mathrm{e}^{2x}+1, \forall x; f(0)=2$. Hàm $f(x)$ là
	\choice
	{$y=2\mathrm{e}^x+2x$}
	{$y=2\mathrm{e}^x+2$}
	{$y=\mathrm{e}^{2x}+x+2$}
	{\True $y=\mathrm{e}^{2x}+x+1$}
	\loigiai{
		Ta có $\displaystyle \int f'(x)\mathrm{\,d}x=\displaystyle \int(2\mathrm{e}^{2x}+1)\mathrm{\,d}x=\mathrm{e}^{2x}+x+C$.\\
		Suy ra $f(x)=\mathrm{e}^{2x}+x+C$.\\
		Theo bài ra ta có $f(0)=2\Rightarrow 1+C=2\Leftrightarrow C=1$.\\
		Vậy $f(x)=\mathrm{e}^{2x}+x+1$.
	}
\end{ex}
\begin{ex}%[2D4H1-3]
	Cho hàm số $f(x)$ thỏa mãn $f'(x)=2-5\sin x$ và $ f(0)=10$. Mệnh đề nào dưới đây \textbf{đúng}?
	\choice
	{$f(x)=2x+5\cos x+3$}
	{$f(x)=2x-5\cos x+15$}
	{\True $f(x)=2x+5\cos x+5$}
	{$f(x)=2x-5\cos x+10$}
	\loigiai{
		Ta có $f(x)=\displaystyle \int f'(x)\mathrm{\,d}x=\displaystyle\int (2-5\sin x)\mathrm{\,d}x=2x+5\cos x+C$.\\
		Mà $f(0)=10$ nên $5+C=10\Rightarrow C=5$.\\
		Vậy $f(x)=2x+5\cos x+5$.}
\end{ex}
\begin{ex}%[2D4V1-2]
	Cho hàm số $f(x)$ thỏa mãn $f'(x)=ax^2+\dfrac{b}{x^3}$, $f'(1)=3$, $f(1)=2$, $f\left(\dfrac{1}{2}\right)=-\dfrac{1}{12}$. Khi đó $2a+b$ bằng
	\choice
	{$-\dfrac{3}{2}$}
	{$0$}
	{\True $5$}
	{$\dfrac{3}{2}$}
	\loigiai{
		Ta có $f'(1)=3\Rightarrow a+b=3. \quad (1)$\\
		Hàm số có đạo hàm liên tục trên khoảng $(0;+\infty)$, các điểm $x=1$, $x=\dfrac{1}{2}$ đều thuộc $(0;+\infty)$ nên\\
		$f(x)=\displaystyle \int f'(x)\mathrm{\,d}x=\displaystyle \int (ax^2+\dfrac{b}{x^3})\mathrm{\,d}x=\dfrac{ax^3}{3}-\dfrac{b}{2x^2}+C$.\\
		\begin{itemize}
			\item $f(1)=2\Rightarrow \dfrac{a}{3}-\dfrac{b}{2}+C=2. \quad (2)$
			\item $f\left(\dfrac{1}{2}\right)=-\dfrac{1}{12}\Rightarrow \dfrac{a}{24}-2b+C=-\dfrac{1}{12}. \quad (3)$
		\end{itemize}
		Từ $(1)$, $(2)$ và $(3)$ ta được hệ phương trình $\heva{ &a+b=3\\ &\dfrac{a}{3}-\dfrac{b}{2}+C=2\\ &\dfrac{a}{24}-2b+C=-\dfrac{1}{12}}\Leftrightarrow \heva{&a=2\\ &b=1\\ &C=\dfrac{11}{6}.}$\\
		Vậy $2a+b=2\cdot 2+1=5$.
	}
\end{ex}
\begin{ex}%[2D4V1-2]
	Tìm một nguyên hàm $F(x)$ của hàm số $f(x)=ax+\dfrac{b}{x^2} \quad (x\ne  0)$, biết rằng $F(-1)=1$, $F(1)=4$, $f(1)=0$.
	\choice
	{$F(x)=\dfrac{3}{2}x^2+\dfrac{3}{4x}-\dfrac{7}{4}$}
	{$F(x)=\dfrac{3}{4}x^2-\dfrac{3}{2x}-\dfrac{7}{4}$}
	{\True $F(x)=\dfrac{3}{4}x^2+\dfrac{3}{2x}+\dfrac{7}{4}$}
	{$F(x)=\dfrac{3}{2}x^2-\dfrac{3}{2x}-\dfrac{1}{2}$}
	\loigiai{
		Ta có $F(x)=\displaystyle \int f(x)\mathrm{\,d}x=\displaystyle \int \left(ax+\dfrac{b}{x^2}\right)\mathrm{\,d}x=\dfrac{1}{2}ax^2-\dfrac{b}{x}+C$.\\
		Theo bài ra $\heva{&F(-1)=1\\ &F(1)=4\\ &f(1)=0}\Leftrightarrow \heva{&\dfrac{1}{2}a+b+C=1\\ &\dfrac{1}{2}a-b+C=4\\ &a+b=0} \Leftrightarrow \heva{&b=-\dfrac{3}{2}\\ &a=\dfrac{3}{2}\\ &C=\dfrac{7}{4}.} $\\
		Vậy $F(x)=\dfrac{3}{4}{x^2}+\dfrac{3}{2x}+\dfrac{7}{4}$.
	}
\end{ex}
\begin{ex}%[2D4V1-2]
	Cho hàm số $f(x)$ xác định trên $\mathbb{R}\setminus \{0\}$ thỏa mãn $f'(x)=\dfrac{x+1}{x^2}$, $f(-2)=\dfrac{3}{2}$ và $f(2)=2\ln 2-\dfrac{3}{2}$. Giá trị của biểu thức $f(-1)+f(4)$ bằng
	\choice
	{$\dfrac{6\ln 2-3}{4}$}
	{$\dfrac{6\ln 2+3}{4}$}
	{\True $\dfrac{8\ln 2+3}{4}$}
	{$\dfrac{8\ln 2-3}{4}$}
	\loigiai{
		Có $f(x)=\displaystyle \int f'(x)\mathrm{\,d}x=\displaystyle \int \dfrac{x+1}{x^2}\mathrm{\,d}x=\ln x-\dfrac{1}{x}+C$.\\
		Tìm được $f(x)=\heva{&\ln |x|-\dfrac{1}{x}+C_1 &\text{khi } &x<0\\ &\ln |x|-\dfrac{1}{x}+C_2 &\text{khi } &x>0.}$\\
		Do $f(-2)=\dfrac{3}{2} \Rightarrow \ln 2+\dfrac{1}{2}+C_1=\dfrac{3}{2} \Rightarrow C_1=1-\ln 2$.\\
		Do $f(2)=2\ln 2-\dfrac{3}{2} \Rightarrow \ln 2-\dfrac{1}{2}+C_2=2\ln 2-\dfrac{3}{2} \Rightarrow C_2=\ln 2-1$.\\
		Suy ra $f(x)=\heva{&\ln |x|-\dfrac{1}{x}+1-\ln 2 &\text{khi } &x<0\\ &\ln |x|-\dfrac{1}{x}+\ln 2-1 &\text{khi } &x>0.}$\\
		Vậy $f(-1)+f(4)=(2-\ln 2)+\left(\ln 4-\dfrac{1}{4}+\ln 2-1\right)=\dfrac{8\ln 2+3}{4}$.
	}
\end{ex}

\begin{ex}%[2D4V1-2]
	\immini{Cho hàm số $y=f(x)$. Đồ thị của hàm số \break $y=f'(x)$ trên $[-5 ; 3]$ như hình vẽ (phần cong của đồ thị là một phần của parabol \break $y=a x^2+b x+c$). Biết $f(0)=0$, giá trị của $2 f(-5)+3 f(2)$ bằng
		\choice
		{$33$}
		{$\dfrac{109}{3}$}
		{\True $\dfrac{35}{3}$}
		{$11$}
	}{
		\begin{tikzpicture}[scale=0.7, font=\footnotesize, line join=round, line cap=round,>=stealth]
			%Gán số liệu.
			\def\xmin{-6};\def\ymin{-2};\def\xmax{4};\def\ymax{5};
			%Gán tọa độ.
			\coordinate (O) at (0,0);
			%Trục Oxy.
			\draw[->] (\xmin,0)--(\xmax,0) node[below]{$x$};
			\draw[->] (0,\ymin)--(0,\ymax) node[left]{$y$};
			\fill (O) node[below left]{$O$} circle(1pt);
			%Giới hạn đồ thị.
			\clip ({\xmin-0.1},{\ymin-0.1}) rectangle ({\xmax+0.1},{\ymax+0.1});
			\foreach \x in {-5,-4,-1,1,2,3}{
					\fill (\x,0) node[below]{$\x$} circle(1pt);
				}
			\foreach \y in {-1,2,3,4}{
					\fill (0,\y) node[left]{$\y$} circle(1pt);
				}
			\draw (-5,-1)--(-4,2)--(-1,0);
			\draw[thick,samples=100] plot[domain=-1:3.5](\x,{-(\x)^2+2*\x+3});
			\draw[dashed] (-5,0)|-(0,-1) (-4,0)|-(0,2) (1,0)|-(0,4) (2,0)|-(0,3);
		\end{tikzpicture}
	}
	\loigiai{
		Parabol $y=a x^2+b x+c$ qua các điểm $(2 ; 3)$, $(1 ; 4)$, $(0 ; 3)$, $(-1 ; 0)$, $(3 ; 0)$ nên xác định được $y=-x^2+2 x+3$, $\forall x \geq-1$ suy ra $f(x)=-\dfrac{x^3}{3}+x^2+3 x+C_1$.\\
		Mà $f(0)=0 \Rightarrow C_1=0$, $f(x)=-\dfrac{x^3}{3}+x^2+3 x$.\\
		Có $f(-1)=-\dfrac{5}{3}$, $ f(2)=\dfrac{22}{3}$.\quad $(1)$\\
		Đồ thị $f'(x)$ trên đoạn $[-4 ;-1]$ qua các điểm $(-4 ; 2)$, $(-1 ; 0)$.\\
		Nên $f'(x)=-\dfrac{2}{3}(x+1) \Rightarrow f(x)=-\dfrac{2}{3}\left(\dfrac{x^2}{2}+x\right)+C_2$.\\
		Mà $f(-1)=-\dfrac{5}{3} \Leftrightarrow C_2=-\dfrac{5}{3}+\dfrac{2}{3}\left(-\dfrac{1}{2}\right)=-2 \Rightarrow f(x)=-\dfrac{2}{3}\left(\dfrac{x^2}{2}+x\right)-2$, hay $f(-4)=-\dfrac{14}{3}$.\\
		Đồ thị $f'(x)$ trên đoạn $[-5 ;-4]$ qua các điểm $(-4 ; 2)$, $(-5 ;-1)$.\\
		Nên $f'(x)=3 x+14 \Rightarrow f(x)=\dfrac{3 x^2}{2}+14 x+C_3$.\\
		Mà $f(-4)=-\dfrac{14}{3} \Leftrightarrow \dfrac{3 \cdot(-4)^2}{2}+14 \cdot(-4)+C_3=-\dfrac{14}{3}$ suy ra $C_3=\dfrac{82}{3}$.\\
		Ta có $f(x)=\dfrac{3 x^2}{2}+14 x+\dfrac{82}{3} \Rightarrow f(-5)=-\dfrac{31}{6}$.\quad $(2)$\\
		Từ $(1)$ và $(2)$ ta được $2 f(-5)+3 f(2)=-\dfrac{31}{3}+22=\dfrac{35}{3}$.
	}
\end{ex}
\begin{ex}%[2D4H1-4]
	Cho hàm số $f(x)=2x+\mathrm{e}^x$. Một nguyên hàm $F(x)$ của hàm số $f(x)$ thỏa mãn $F(0)=2024$. Biết $F(x)=ax^2+b\mathrm{e}^x+c$, giá trị của $a+b+c$ là
	\shortans{$2025$}
	\loigiai{
		Ta có $\displaystyle\int f(x)\mathrm{\,d}x=\displaystyle\int (2x+\mathrm{e}^x)\mathrm{\,d}x=x^2+\mathrm{e}^x+C$.\\
		Có $F(x)$ là một nguyên hàm của $f(x)$ và $F(0)=2024$.\\
		Tìm được $\heva{&F(x)=x^2+\mathrm{e}^x+C\\ &F(0)=2024} \Rightarrow 1+C=2024 \Leftrightarrow C=2023$.\\
		Suy ra $F(x)=x^2+\mathrm{e}^x+2023$.\\
		Vậy $a+b+c=2025$.
	}
\end{ex}
% \begin{ex}%[2D4H1-3]
% 	Cho $F(x)$ là một nguyên hàm của hàm số $f(x)=\sin x+1$ biết $F\left( \dfrac{\pi}{6}\right) =0$. Tính giá trị của $F(\pi)$. (Làm tròn đến chữ số thập phân thứ hai)
% 	\shortans{$4{,}48$}
% 	\loigiai{
% 		$F(x)=\displaystyle \int(\sin x+1)\mathrm{\,d}x=x-\cos x+C$.\\
% 		Do $F\left( \dfrac{\pi}{6}\right) =0 \Rightarrow \dfrac{\pi}{6}-\cos \left( \dfrac{\pi}{6}\right) +C=0\Leftrightarrow C=\dfrac{\sqrt{3}}{2}-\dfrac{\pi}{6}$.\\
% 		Suy ra $F(x)=x-\cos x+\dfrac{\sqrt{3}}{2}-\dfrac{\pi}{6}$.\\
% 		Vậy $F(\pi)=4{,}48$. 
% 	}
% \end{ex}
% \begin{ex}%[2D4H1-2]
% 	Cho $F(x)$ là một nguyên hàm của $f(x)=(5x+3)^5$. Biết $F(1)=0$. Tính giá trị của $\sqrt{|F(0)|}$. (Làm tròn đến chữ số thập phân thứ nhất)
% 	\shortans{$93{,}3$}
% 	\loigiai{
% 		Ta có $F(x)=\displaystyle \int f(x)\mathrm{\,d}x=\displaystyle \int (5x+3)^5\mathrm{\,d}x=\dfrac{(5x+3)^6}{30}+C$.\\
% 		Do $F(1)=0\Rightarrow 0=\dfrac{(5\cdot 1+3)^6}{30}+C\Rightarrow C=-\dfrac{131072}{15}$.\\
% 		Suy ra $F(x)=\dfrac{(5x+3)^6}{30}-\dfrac{131072}{15}$.\\
% 		Do đó $F(0)=\dfrac{(5\cdot 0+3)^6}{30}-\dfrac{131072}{15}=-\dfrac{52283}{6}$.\\
% 		Vậy $\sqrt{|F(0)|}=93{,}3$.
% 	}
% \end{ex}
% \begin{ex}%[2D4H1-2]
% 	Cho $F(x)$ là một nguyên hàm của $f(x)=x^3-4x+5$. Biết $F(1)=3$. Tính $|F(0)|$.
% 	\shortans{$0{,}25$}
% 	\loigiai{
% 		Ta có $F(x)=\displaystyle \int f(x)\mathrm{\,d}x=\displaystyle \int(x^3-4x+5)\mathrm{\,d}x=\dfrac{x^4}{4}-2x^2+5x+C$.\\
% 		Do $F(1)=3\Rightarrow 3=\dfrac{1^4}{4}-2\cdot 1^2+5\cdot 1+C\Rightarrow C=-\dfrac{1}{4}$.\\
% 		Suy ra $F(x)=\dfrac{x^4}{4}-2x^2+5x-\dfrac{1}{4}$.\\
% 		Vậy $|F(0)|=0{,}25$.
% 	}
% \end{ex}
% \begin{ex}%[2D4H1-3]
% 	Cho $F(x)$ là một nguyên hàm của $f(x)=3-5\cos x$. Biết $F(\pi )=2$. Tính $F\left(\dfrac{\pi}{2}\right)$. (Làm tròn đến chữ số thập phân thứ nhất)
% 	\shortans{$-7{,}7$}
% 	\loigiai{
% 		Ta có $F(x)=\displaystyle \int f(x)\mathrm{\,d}x=\displaystyle \int(3-5\cos x)\mathrm{\,d}x=3x-5\sin x+C$.\\
% 		Do $F(\pi )=2\Rightarrow 2=3\pi -5\sin \pi+C\Rightarrow C=-3\pi +2$.\\
% 		Suy ra $F(x)=3x-5\sin x-3\pi +2$.\\
% 		Vậy $F\left(\dfrac{\pi}{2}\right)=-7{,}7$.
% 	}
% \end{ex}
% \begin{ex}%[2D4H1-2]
% 	Cho $F(x)$ là một nguyên hàm của $f(x)=\dfrac{3-5x^2}{x}$. Biết $F(\mathrm{e})=1$. Tính $F(2)$. (Làm tròn đến chữ số thập phân thứ hai)
% 	\shortans{$8{,}55$}
% 	\loigiai{
% 		Hàm số $f(x)=\dfrac{3-5x^2}{x}=\dfrac{3}{x}-5x$.\\
% 		Có $F(x)=\displaystyle \int f(x)\mathrm{\,d}x=\displaystyle \int\left(\dfrac{3}{x}-5x\right)\mathrm{\,d}x=3\ln |x|-\dfrac{5}{2} x^2 +C$.\\
% 		Do $F(\mathrm{e})=1\Rightarrow 1=3\ln |\mathrm{e}|-\dfrac{5}{2} \mathrm{e}^2 +C \Rightarrow C=\dfrac{5}{2} \mathrm{e}^2 -2$.\\
% 		Suy ra $F(x)=3\ln |x|-\dfrac{5}{2} x^2 +\dfrac{5}{2} \mathrm{e}^2 -2$.\\
% 		Vậy $F(2)=8{,}55$.
% 	}
% \end{ex}
% \begin{ex}%[2D4H1-2]
% 	Cho $F(x)$ là một nguyên hàm của $f(x)=\dfrac{x^2+1}{x}$. Biết $F(1)=\dfrac{3}{2}$. Tính $F(-1)$.
% 	\shortans{$1{,}5$}
% 	\loigiai{
% 		Hàm số $f(x)=\dfrac{x^2+1}{x}=x+\dfrac{1}{x}$.\\
% 		Có $F(x)=\displaystyle \int f(x)\mathrm{\,d}x=\displaystyle \int\left(x+\dfrac{1}{x}\right)\mathrm{\,d}x=\dfrac{x^2}{2}+\ln |x|+C$.\\
% 		Do $F(1)=\dfrac{3}{2}\Rightarrow \dfrac{3}{2}=\dfrac{1^2}{2}+\ln |1|+C \Rightarrow C=1$.\\
% 		Suy ra $F(x)=\dfrac{x^2}{2}+\ln |x|+1$.\\
% 		Vậy $F(-1)=\dfrac{(-1)^2}{2}+\ln |-1|+1=\dfrac{3}{2}=1{,}5$.
% 	}
% \end{ex}
% \begin{ex}%[2D4H1-2]
% 	Cho $F(x)$ là một nguyên hàm của hàm số $f(x)=\dfrac{x^3-1}{x^2}$. Biết $F(-2)=0$. Tính giá trị của $F(2)$.
% 	\shortans{$1$}
% 	\loigiai{
% 		Hàm số $f(x)=\dfrac{x^3-1}{x^2}=x-\dfrac{1}{x^2}$.\\
% 		Có $F(x)=\displaystyle \int f(x)\mathrm{\,d}x=\displaystyle \int\left(x-\dfrac{1}{x^2}\right)\mathrm{\,d}x=\dfrac{x^2}{2}+\dfrac{1}{x}+C$.\\
% 		Do $F(-2)=0\Rightarrow 0=\dfrac{(-2)^2}{2}+\dfrac{1}{(-2)}+C\Rightarrow C=-\dfrac{3}{2}$.\\
% 		Suy ra $F(x)=\dfrac{x^2}{2}+\dfrac{1}{x}-\dfrac{3}{2}$.\\
% 		Vậy $F(2)=1$.
% 	}
% \end{ex}
% \begin{ex}%[2D4H1-4]
% 	Cho $F(x)$ là một nguyên hàm của hàm số $f(x)=x\sqrt{x}+\dfrac{1}{\sqrt{x}}$. Biết $F(1)=-2$. Tính $F(0)$.
% 	\shortans{$-4{,}4$}
% 	\loigiai{
% 		Hàm số $f(x)=x\sqrt{x}+\dfrac{1}{\sqrt{x}}=x^{\tfrac{3}{2}}+x^{-\tfrac{1}{2}}$.\\
% 		Có $F(x)=\displaystyle \int f(x)\mathrm{\,d}x=\displaystyle \int\left(x^{\tfrac{3}{2}}+x^{-\tfrac{1}{2}}\right)\mathrm{\,d}x=\dfrac{2}{5}x^{\tfrac{5}{2}}+2\sqrt{x}+C$.\\
% 		Do $F(1)=-2\Rightarrow -2=\dfrac{2}{5}\cdot 1^{\tfrac{5}{2}}+2\sqrt{1} +C \Rightarrow C=-\dfrac{22}{5}$.\\
% 		Suy ra $F(x)=\dfrac{2}{5}x^{\tfrac{5}{2}}+2\sqrt{x} -\dfrac{22}{5}$.\\
% 		Vậy $F(0)=-4{,}4$.
% 	}
% \end{ex}
% \begin{ex}%[2D4H1-3]
% 	Cho $F(x)$ là một nguyên hàm của hàm số $f(x)=\sin x +1$. Biết $F\left(\dfrac{\pi}{6}\right)=0$. Tính $F(-1)$. (Làm tròn đến chữ số thập phân thứ nhất)
% 	\shortans{$-1{,}2$}
% 	\loigiai{
% 		Ta có $F(x)=\displaystyle \int f(x)\mathrm{\,d}x=\displaystyle \int(\sin x +1)\mathrm{\,d}x=-\cos x +x +C$.\\
% 		Do $F\left(\dfrac{\pi}{6}\right)=0\Rightarrow 0=-\cos \dfrac{\pi}{6} +\dfrac{\pi}{6} +C\Rightarrow C=-\dfrac{\pi}{6} +\dfrac{\sqrt{3}}{2}$.\\
% 		Suy ra $F(x)=-\cos x +x -\dfrac{\pi}{6} +\dfrac{\sqrt{3}}{2}$.\\
% 		Vậy $F(-1)=-1{,}2$.
% 	}
% \end{ex}
% \begin{ex}%[2D4V1-3]
% 	Cho $F(x)$ là một nguyên hàm của $f(x)=2024-\sin^2 \dfrac{x}{2}$. Biết $F\left(\dfrac{\pi}{2}\right)=2025$. Tính $\sqrt{|F(0)|}$. (Làm tròn đến chữ số thập phân thứ nhất)
% 	\shortans{$34$}
% 	\loigiai{
% 		Hàm số $f(x)=2024-\sin^2 \dfrac{x}{2}=2024-\dfrac{1-\cos x}{2}=\dfrac{4047+\cos x}{2}$.\\
% 		Có $F(x)=\displaystyle \int f(x)\mathrm{\,d}x=\displaystyle \int\left(\dfrac{4047+\cos x}{2}\right)\mathrm{\,d}x=\dfrac{1}{2}(4047x+\sin x)+C$.\\
% 		Do $F\left(\dfrac{\pi}{2}\right)=2025\Rightarrow 2025=\dfrac{1}{2}(4047\cdot \dfrac{\pi}{2} +\sin \dfrac{\pi}{2})+C\Rightarrow C=-\dfrac{4047}{4}\pi +\dfrac{4049}{2}$.\\
% 		Suy ra $F(x)=\dfrac{1}{2}(4047x+\sin x)-\dfrac{4047}{4}\pi +\dfrac{4049}{2}$.\\
% 		Vậy $\sqrt{|F(0)|}=34$.
% 	}
% \end{ex}
% \begin{ex}%[2D4V1-3]
% 	Cho $F(x)$ là một nguyên hàm của $f(x)=\sin^2 \dfrac{x}{4} \cdot \cos^2 \dfrac{x}{4}$. Biết $F\left(\dfrac{\pi}{3}\right)=0$. Tính giá trị của $F(\pi)$. (Làm tròn đến chữ số thập phân thứ hai)
% 	\shortans{$0{,}37$}
% 	\loigiai{
% 		Hàm số $f(x)=\sin^2 \dfrac{x}{4} \cdot \cos^2 \dfrac{x}{4}=\dfrac{1}{8}(1-\cos x)$.\\
% 		Có $F(x)=\displaystyle \int f(x)\mathrm{\,d}x=\displaystyle \int \dfrac{1}{8}(1-\cos x)\mathrm{\,d}x=\dfrac{1}{8}(x-\sin x)+C$.\\
% 		Do $F\left(\dfrac{\pi}{3}\right)=0\Rightarrow 0=\dfrac{1}{8}\left(\dfrac{\pi}{3}-\sin \dfrac{\pi}{3}\right)+C\Rightarrow C=-\dfrac{\pi}{24}+\dfrac{\sqrt{3}}{16}$.\\
% 		Suy ra $F(x)=\dfrac{1}{8}(x-\sin x)-\dfrac{\pi}{24}+\dfrac{\sqrt{3}}{16}$.\\
% 		Vậy $F(\pi)=0{,}37$.
% 	}
% \end{ex}
% \begin{ex}%[2D4H1-2]
% 	Cho hàm số $f(x)=\heva{&2x+5 &\text{khi } &x\ge 1\\ &3x^2+4 &\text{khi } &x<1.}$ Giả sử $F$ là nguyên hàm của $f$ trên $\mathbb{R}$ thỏa mãn $F(0)=2$. Giá trị của $F(-1)+2F(2)$.
% 	\shortans{$27$}
% 	\loigiai{
% 		Ta có $f(x)=\heva{&2x+5 &\text{khi } &x\ge 1\\ &3x^2+4 &\text{khi } &x<1}\Rightarrow \heva{&F(x)=x^2+5x+C_1 &\text{khi } &x\ge 1\\ &F(x)=x^3+4x+C_2 &\text{khi } &x<1.}$\\
% 		Vì $F$ là nguyên hàm của $f$ trên $\mathbb{R}$ thỏa mãn $F(0)=2$ nên $C_2=2\Rightarrow F(x)=x^3+4x+2$.\\
% 		Vì $F(x)$ liên tục trên $\mathbb{R}$ nên $F(x)$ liên tục tại $x=1$ nên:\\
% 		$\lim\limits_{x\to1^+} F(x)=\lim\limits_{x\to1^-} F(x)=F(1)\Rightarrow 6+C_1=7\Rightarrow C_1=1$.\\
% 		Vậy ta có $\heva{&F(x)=x^2+5x+1 &\text{khi } &x\ge 1\\ &F(x)=x^3+4x+2 &\text{khi } &x<1}\Rightarrow F(-1)+2F(2)=27$.
% 	}
% \end{ex}
\begin{ex}%[2D4V1-4]
	Gọi $F(x)$ là một nguyên hàm của hàm số $f(x)=2^x$, thỏa mãn $F(0)=\dfrac{1}{\ln 2}$. Giá trị biểu thức $T=F(0)+F(1)+\cdots +F(2018)+F(2019)$ có dạng $\dfrac{{2^{2020}}+a}{\ln b}$. Giá trị của $\dfrac{a}{b}$ là
	\shortans{$-0{,}5$}
	\loigiai{
		Ta có $\displaystyle \int f(x) \mathrm{\,d}x=\displaystyle \int 2^x \mathrm{\,d}x=\dfrac{2^x}{\ln 2}+C$.\\
		$F(x)$ là một nguyên hàm của hàm số $f(x)=2^x$, ta có $F(x)=\dfrac{2^x}{\ln 2}+C$ mà $F(0)=\dfrac{1}{\ln 2}$.\\
		$\Rightarrow C=0\Rightarrow F(x)=\dfrac{2^x}{\ln 2}$.\\
		\begin{eqnarray*}
			T&=&F(0)+F(1)+\cdots +F(2018)+F(2019)\\
			&=&\dfrac{1}{\ln 2}(1+2+2^2+\cdots +2^{2018}+2^{2019})\\
			&=&\dfrac{1}{\ln 2}\cdot \dfrac{{2^{2020}}-1}{2-1}\\
			&=&\dfrac{{2^{2020}}-1}{\ln 2}.\\
		\end{eqnarray*}
		Vậy $\dfrac{a}{b}=-\dfrac{1}{2}=-0{,}5$
	}
\end{ex}
\begin{ex}%[2D4V1-3]
	Cho $F(x)$ là một nguyên hàm của hàm số $f(x)=\dfrac{1}{\cos^2 x}$. Biết $F\left(\dfrac{\pi}{4}+k\pi \right)=k$ với mọi $k\in \mathbb{Z}$. Tính giá trị của biểu thức $T=F(0)+F(\pi )+F(2\pi )+\cdots +F(10\pi )$.
	\shortans{$44$}
	\loigiai{
		Ta có $\displaystyle \int f(x) \mathrm{\,d}x=\displaystyle \int \dfrac{\mathrm{\,d}x}{\cos^2 x}=\tan x+C$.\\
		Suy ra 
		$F(x)=\heva{&\tan x+C_0, \quad x\in \left(-\dfrac{\pi}{2};\dfrac{\pi}{2}\right)\\ &\tan x+C_1, \quad x\in \left(\dfrac{\pi}{2};\dfrac{3\pi}{2}\right)\\ &\tan x+C_2, \quad x\in \left(\dfrac{3\pi}{2};\dfrac{5\pi}{2}\right)\\ &\cdots\\ &\tan x+C_9, \quad x\in \left(\dfrac{17\pi}{2};\dfrac{19\pi}{2}\right)\\ &\tan x+C_{10}, \quad x\in \left(\dfrac{19\pi}{2};\dfrac{21\pi}{2}\right)}\Rightarrow \heva{&F\left(\dfrac{\pi}{4}+0\pi\right)=1+C_0=0\Rightarrow C_0=-1\\ &F\left(\dfrac{\pi}{4}+\pi\right)=1+C_1=1\Rightarrow C_1=0\\ &F\left(\dfrac{\pi}{4}+2\pi\right)=1+C_2=2\Rightarrow C_2=1\\ &\cdots \\ &F\left(\dfrac{\pi}{4}+9\pi\right)=1+C_9=9\Rightarrow C_9=8\\ &F\left(\dfrac{\pi}{4}+10\pi\right)=1+C_{10}=0\Rightarrow C_{10}=9.}$\\
		Vậy \begin{eqnarray*}
			T&=&F(0)+F(\pi )+F(2\pi )+\cdots +F(10\pi )\\ &=&\tan 0-1+\tan \pi +\tan 2\pi +1+\cdots +\tan 10\pi +9\\ &=&44.
		\end{eqnarray*}
	}
\end{ex}
% \begin{ex}%[2D4V1-3]
% 	Hàm số $f(x)$ có đạo hàm liên tục trên $\mathbb{R}$ và $f'(x)=2024-2\sin^2 \dfrac{x}{2}$, $\forall x$;\hfill \break $f\left(\dfrac{\pi}{2}\right)=\dfrac{2023\pi}{2}$. Tính giá trị của $f(0)$.
% 	\shortans{$-1$}
% 	\loigiai{
% 		$f(x)=\displaystyle \int \left(2024-2\sin^2 \dfrac{x}{2}\right)\mathrm{\,d}x=\displaystyle \int (2023+\cos x)\mathrm{\,d}x=2023x+\sin x+C$.\\
% 		Tìm được $f(x)=2023x+\sin x+C$.\\
% 		Do $f\left(\dfrac{\pi}{2}\right)=\dfrac{2023\pi}{2} \Leftrightarrow \dfrac{2023\pi}{2}=2023\cdot \dfrac{\pi}{2}+\sin \dfrac{\pi}{2}+C\Leftrightarrow C=-1$.\\
% 		Vậy $f(x)=2023x+\sin x-1$.\\
% 		Do đó $f(0)=-1$.
% 	}
% \end{ex}
% \begin{ex}%[2D4H1-4]
% 	Hàm số $f(x)$ có đạo hàm liên tục trên $\mathbb{R}$ và $f'(x)=1+\mathrm{e}^{2x}$, $\forall x$; $f(0)=2$. Tính giá trị của $f(2)$. (Làm tròn đến số thập phân thứ nhất)
% 	\shortans{$30{,}8$}
% 	\loigiai{
% 		Hàm số $f(x)=\displaystyle \int (1+\mathrm{e}^{2x})\mathrm{\,d}x=x+\dfrac{1}{2}\mathrm{e}^{2x}+C$.\\
% 		Do $f(0)=2\Leftrightarrow 2=\dfrac{1}{2}+C\Leftrightarrow C=\dfrac{3}{2}$.\\
% 		Suy ra $f(x)=x+\dfrac{1}{2}\mathrm{e}^{2x}+\dfrac{3}{2}$.\\
% 		Vậy $f(2)=30{,}8$.
% 	}
% \end{ex}
% \begin{ex}%[2D4H1-4]
% 	Hàm số $f(x)$ có đạo hàm liên tục trên $\mathbb{R}$ và $f'(x)=2^x+3^x$, $\forall x$; $f(0)=\dfrac{1}{\ln 3}$. Tính giá trị của $f(1)$. (Làm tròn đến số thập phân thứ hai)
% 	\shortans{$4{,}17$}
% 	\loigiai{
% 		Hàm số $f(x)=\displaystyle \int (2^x+3^x)\mathrm{\,d}x= \displaystyle \int 2^x \mathrm{\,d}x+\displaystyle \int 3^x \mathrm{\,d}x=\dfrac{2^x}{\ln 2}+\dfrac{3^x}{\ln 3}+C$.\\
% 		$f(x)=\dfrac{2^x}{\ln 2}+\dfrac{3^x}{\ln 3}+C$.\\
% 		Do $f(0)=\dfrac{1}{\ln 3} \Leftrightarrow \dfrac{1}{\ln 3}=\dfrac{1}{\ln 2}+\dfrac{1}{\ln 3}+C\Leftrightarrow C=-\dfrac{1}{\ln 2}$\\
% 		Suy ra $f(x)=\dfrac{2^x}{\ln 2}+\dfrac{3^x}{\ln 3}-\dfrac{1}{\ln 2}$.\\
% 		Vậy $f(1)=4{,}17$.
% 	}
% \end{ex}
%%%%%----------Câu 34
\begin{ex}%[2D4H1-4]
	Hàm số $f(x)$ có đạo hàm liên tục trên $\mathbb{R}$ và $f'(x)=\mathrm{e}^{3x+2024}$, $\forall x $ thoả mã $f(-675)=1$. Giá trị của $f(-674)$ bằng
	\shortans[3]{$3{,}34$}
	\loigiai{
		Hàm số $f(x)$ có đạo hàm $f'(x)=\mathrm{e}^{3x+2024}$.\\
		Ta có $f(x)=\displaystyle\int\!\!\mathrm{e}^{3x+2024}\mathrm{d}x =\dfrac{1}{3}\mathrm{e}^{3x+2024}+C$.\\ 
		Suy ra $f(x)=\dfrac{1}{3}\mathrm{e}^{3x+2024}+C$.\\
		Với $f(-675)=1 \Rightarrow 1 =\dfrac{1}{3}\mathrm{e}^{3\cdot(-675)+2024}+C \Rightarrow C=1-\dfrac{1}{3\mathrm{e}}$.\\
		Vậy $f(x)=\dfrac{1}{3}\mathrm{e}^{3x+2024}+1-\dfrac{1}{3\mathrm{e}}$.\\
		Giá trị $f(-274)=\dfrac{1}{3}\mathrm{e}^2+1-\dfrac{1}{3\mathrm{e}}=3{,}34$.
	}
\end{ex}
%%%%%----------Câu 35
\begin{ex}%[2D4H1-4]
	Hàm số $f(x)$ có đạo hàm liên tục trên $\mathbb{R}$ và $f'(x)=3^{x+2}\cdot2^{2x+1}$, $\forall x$ thoả mãn $f(0)~=~\dfrac{1}{2\ln 2}$. Giá trị của $f(1)$ bằng
	\shortans[3]{$80{,}4$}
	\loigiai{
		Hàm số $f(x)$ có đạo hàm $f'(x)=3^{x+2}\cdot2^{2x+1}$.\\
		Ta có $f(x)=\displaystyle\int3^{x+2}\cdot2^{2x+1}\mathrm{d}x =\int3^2\cdot3^x\cdot2\cdot4^x\mathrm{d}x=18\int12^x\mathrm{d}x =18\cdot\dfrac{12^x}{\ln 12}+C$.\\
		Suy ra $f(x)=18\cdot\dfrac{12^x}{\ln 12}+C$.\\
		Với $f(0)=\dfrac{1}{2\ln 2}$
		$\Rightarrow \dfrac{1}{2\ln 2}=18\dfrac{1}{\ln 12}+C \Rightarrow C=\dfrac{1}{2\ln 2}-\dfrac{18}{2\ln 2+\ln 3}$.\\
		Vậy $f(x)=18\cdot\dfrac{12^x}{\ln 12}+\dfrac{1}{2\ln 2}-\dfrac{18}{2\ln 2+\ln 3}$.\\
		Giá trị $f(1)=18\cdot\dfrac{12}{\ln 12}+\dfrac{1}{2\ln 2}-\dfrac{18}{2\ln 2+\ln 3}=\dfrac{216}{\ln 12}+\dfrac{1}{\ln 4}-\dfrac{18}{\ln 4+\ln 3}=80{,}4$.
	}
\end{ex}
%%%%%----------Câu 36
\begin{ex}%[2D4H1-4]
	Hàm số $f(x)$ có đạo hàm liên tục trên $\mathbb{R}$ và $f'(x)=\left( 3^x+5^x \right)^2$, $\forall x$ thoả mãn\break $f(0)=\dfrac{1}{\ln 5+\ln 3+\ln 2}$. Giá trị của $f(1)$ bằng
	\shortans[3]{$19{,}9$}
	\loigiai{
		Hàm số $f(x)$ có đạo hàm $f'(x)=\left( 3^x+5^x \right)^2$.\\
		Ta có 
		$\begin{aligned}[t]
			f(x)&=\displaystyle\int(3^x+5^x)^2\mathrm{d}x\\
			&=\int(9^x+30^x+25^x)\mathrm{d}x\\
			&=\dfrac{9^x}{\ln 9}+\dfrac{30^x}{\ln 30}+\frac{25^x}{\ln 25}+C\\
			&=\dfrac{9^x}{2\ln 3}+\dfrac{30^x}{\ln 5+\ln 3+\ln 2}+\dfrac{25^x}{2\ln 5}+C.
		\end{aligned}$\\
		Suy ra $f(x)=\dfrac{9^x}{2\ln 3}+\dfrac{30^x}{\ln 5+\ln 3+\ln 2}+\dfrac{25^x}{2\ln 5}+C$.\\
		Với 
		$\begin{aligned}[t]
			f(0)=&\ \dfrac{1}{\ln 5+\ln 3+\ln 2}\\
			\Rightarrow &\ \dfrac{1}{\ln 5+\ln 3+\ln 2} =\dfrac{1}{2\ln 3}+\dfrac{1}{\ln 5+\ln 3+\ln 2}+\dfrac{1}{2\ln 5}+C\\
			\Leftrightarrow &\ C=-\dfrac{1}{2\ln 3}-\dfrac{1}{2\ln 5}.
		\end{aligned}$\\
		Vậy
		$\begin{aligned}[t]
			f(x)&=\dfrac{9^x}{2\ln 3}+\dfrac{30^x}{\ln 5+\ln 3+\ln 2}+\dfrac{25^x}{2\ln 5}-\dfrac{1}{2\ln 3}-\dfrac{1}{2\ln 5}\\
			&=\dfrac{9^x}{\ln 9}+\dfrac{30^x}{\ln 30}+\dfrac{25^x}{\ln 25}-\dfrac{1}{\ln 9}-\dfrac{1}{\ln 25}.
		\end{aligned}$\\
		Giá trị của
		$\begin{aligned}[t]
			f(1)&=\dfrac{9}{\ln 9}+\dfrac{30}{\ln 30}+\dfrac{25}{\ln 25}-\dfrac{1}{\ln 9}-\dfrac{1}{\ln 25}\\
			&=\dfrac{8}{\ln 9}+\dfrac{30}{\ln 30}+\dfrac{24}{\ln 25}\\
			&=19{,}9.
		\end{aligned}$
	}
\end{ex}
\Closesolutionfile{ans}
% \indapan{6}{ans/ans-C4B1CD2-CAU31_33-KQ}
% % \subsection{ỨNG DỤNG NGUYÊN HÀM TRONG THỰC TIỄN}
\begin{dang}{Ứng dụng trong bài toán thực tiễn}
	Giả sử $v(t)$ là vận tốc của vật $ {M}$ tại thời điểm $t$ và $s(t)$ là quãng đường vật đi được sau khoảng thời gian $t$ tính từ lúc bắt đầu chuyển động. Ta có mối liên hệ giữa $s(t)$ và $v(t)$ như sau.
	\begin{itemize}
		\item  Đạo hàm của quãng đường là vận tốc $s'(t)=v(t)$.
		\item  Nguyên hàm của vận tốc là quãng đường $s(t)=\displaystyle\int v(t)  \mathrm{\,d} t$.
	\end{itemize}
	Nếu gọi $a(t)$ là gia tốc của vật M thì ta có mối liên hệ giữa $v(t)$ và $a(t)$ như sau.
	\begin{itemize}
		\item Đạo hàm của vận tốc là gia tốc $v'(t)=a(t)$.
		\item Nguyên hàm của gia tốc là vận tốc $v(t)=\displaystyle\int\limits a(t)  \mathrm{\,d} t$.
	\end{itemize}
\end{dang}
% \TN
\Opensolutionfile{ans}[ans/ans2-C4B1CD4-D1]
\begin{ex}%[2D4V1-6]
	Một ô tô đang chạy với vận tốc $20$ m/s thì người lái đạp phanh. Sau khi đạp phanh, ô tô chuyển động chậm dần đều với vận tốc $v(t)=-40t+20$ m/s, trong đó $t$ là khoảng thời gian tính bằng giây kể từ lúc bắt đầu đạp phanh. Gọi  $s(t)$ là quãng đường xe ô tô đi được trong thời gian $t$  (giây) kể từ lúc đạp phanh. Hỏi từ lúc đạp phanh đến khi dừng hẳn, ô tô còn di chuyển bao nhiêu mét?
	\choice{$5$ cm}{$7{,}5$ m}{$\dfrac{5}{2}$ m}{\True $5$ m}
	\loigiai{
		Ta có $v(t)=-40t+20$.\\
		Suy ra $s(t)=\displaystyle\int v(t)\mathrm{\,d}t=\displaystyle\int (-40t+20)\mathrm{\,d}t=-20t^2+20t+C$.\\
		Chọn $t=0$ suy ra $s(0)=0\Rightarrow C=0$.\\
		Khi đó $s(t)=-20t^2+20t$.\\
		Khi xe dừng hẳn $v(t)=0\Leftrightarrow -40t+20=0\Leftrightarrow t=0{,}5$.\\
		Từ lúc đạp phanh đến khi dừng hẳn, ô tô còn di chuyển được\\ $s(0{,}5)=-20\cdot (0{,}5)^2+20\cdot 0{,}5=5$ m.
	}
\end{ex}
\Opensolutionfile{ans}[ans/ansMyLT]
%\begin{ex}%[2D4H1-6]
%Một ô tô đang chạy với vận tốc $20$ m/s thì người người đạp phanh. Sau khi đạp phanh, ô tô chuyển động chậm dần đều với vận tốc $v(t)=-40t+20$ m/s, trong đó $t$ là khoảng thời gian tính bằng giây kể từ lúc bắt đầu đạp phanh. Gọi $s(t)$ là quãng đường xe ô tô đi được trong thời gian $t$ (giây) kể từ lúc đạp phanh. Hỏi từ lúc đạp phanh đến khi dừng hẳn, ô tô còn di chuyển bao nhiêu mét?
%\choice
%{$5$ cm}
%{$7,5$ m}
%{$\dfrac{5}{2}$ m}
%{\True $5$ m}
%\begin{center}
%	\color{red}HÌNH Ở ĐÂY
%\end{center}
%\loigiai{
	%Ta có $v(t)=-40t+20$\\
	%$\Rightarrow s(t)=\displaystyle\int{v(t)}dt=\displaystyle\int{\left(-40t+20\right)}dt=-20t^2+20t+C$\\
	%$\Rightarrow s(t)=-20t^2+20t+C$\\
	%Chọn $t=0\Rightarrow s(0)=0$ $\Rightarrow C=0$\\
	%$\Rightarrow s(t)=-20t^2+20t$\\
	%Khi xe dừng hẳn thì $v(t)=0\Leftrightarrow-40t+20=0\Rightarrow t=0{,}5$.\\
	%từ lúc đạp phanh đến khi dừng hẳn, ô tô còn di chuyển được:\\ $s\left(0{,}5\right)=-20\left(0{,}5\right)^2+20\left(0{,}5\right)=5$ m.}
%\end{ex}

\begin{ex}%[2D4H1-6]
	Bạn Minh Hiền ngồi trên máy bay đi du lịch thế giới với vận tốc chuyển động của máy báy là $v(t)=3t^2+5$ (m/s). Quãng đường máy bay bay từ giây thứ $4$ đến giây thứ $10$ là
	\begin{center}
		\begin{tikzpicture}
			\clip (-4.5,-1) rectangle (4.5,1);
			%\path (0,0) node[opacity=.5,scale=.3] {\includegraphics{6}};
			%\draw[gray!50] (-4,-2) grid (4,2);
			\begin{pgfinterruptboundingbox}
				%Đuôi máy bay
				\def\D{ (-2,-.45)--(-2.36,.24)--(-2.2,.3)--(-1.35,-.25)--cycle
					;}
				\draw \D;
				\fill[brown!40!] \D;
				%Quạt máy bay
				\def\Q{ (.15,-.1)
					..controls +(-140:0.1) and +(140:0.1) .. (.2,-.35)--(.8,-.17)--(.7,.07)--cycle;}
				\draw \Q;
				\draw[xshift=.5cm] \Q;
				\fill[brown!40!,xshift=.5cm] \Q;
				%---------elip2
				\draw[rotate=-70,xshift=-.23cm,yshift=1.27cm] (.7,-.1) ellipse (.12cm and .07cm);
				%Thân máy bay
				\def\T{ (-2.1,-.5)
					..controls +(40:0) and +(-170:0.7) .. (2,.74)
					..controls +(-40:0) and +(170:0.3) .. (2.55,.7)
					..controls +(-150:0) and +(30:.65) .. (2.1,.3)
					..controls +(-150:0) and +(-18:1.25) .. (-2.1,-.6)
					..controls +(90:0) and +(-90:0) .. (-2.1,-.51)
					-- (-2.5,-.52)--cycle;
				}
				\draw \T;
				\fill[brown!70!] \T;
				%Cánh máy bay
				\def\C{ (.5,.05)
					..controls +(170:0) and +(-10:0) .. (-1.48,.12)
					..controls +(-80:0) and +(170:0.02) .. (-1.65,.3)
					..controls +(180:0) and +(0:0) .. (-1.77,.3)
					..controls +(-80:0) and +(110:0) .. (-1.64,.12)
					..controls +(-35:0) and +(145:0) .. (-.2,-.17)
					--cycle;}
				\draw \C;
				\fill[brown!40!] \C;
				%Quạt sau
				\fill[brown!40!] \Q;
				
				%----elip 1
				\draw[rotate=-70,xshift=-.4cm,yshift=.8cm] (.7,-.1) ellipse (.12cm and .07cm);
				%Ô cửa
				\draw (2.18,0.73)--(2.5,0.7)--(2.14,0.58)--cycle;
				\fill[brown!40!](2.18,0.73)--(2.51,0.71)--(2.14,0.57)--cycle;
			\end{pgfinterruptboundingbox}
		\end{tikzpicture}
	\end{center}
	\choice
	{$36$ m}
	{$252$ m}
	{$1134$ m}
	{\True $966$ m}
	
	\loigiai{
		Ta có $v(t)=3t^2+5$.\\
		$\Rightarrow s(t)=\displaystyle\int{v(t)\mathrm{\,d}t}=\displaystyle\int{\left(3t^2+5\right)} \mathrm{\,d}t=t^3+5t+C$.\\
		$\Rightarrow s(t)=t^3+5t+C$.\\
		Chọn $t=0\Rightarrow s(0)=0 \Rightarrow C=0$.\\
		$\Rightarrow s(t)=t^3+5t$.\\
		Quãng đường máy bay bay từ giây thứ $4$ là $s(4)=4^3+5\cdot4=84$ (m).\\
		Quãng đường máy bay bay từ giây thứ $10$ là $s\left(10\right)=10^3+5\cdot 10=1050$ (m).\\
		Quãng đường máy bay bay từ giây thứ $4$ đến giây thứ $10$ là $s(10)-s(4)=966$ (m).}
\end{ex}

\begin{ex}%[2D4H1-6]
	Một ô tô đang chạy với vận tốc $12$ m/s thì người lái đạp phanh; từ thời điểm đó, ô tô chuyển động chậm dần đều với vận tốc $v(t)=-6t+12$ (m/s), trong đó $t$ là khoảng thời gian tính bằng giây kể từ lúc đạp phanh. Hỏi từ lúc đạp phanh đến khi ô tô dừng hẳn, ô tô còn di chuyển được bao nhiêu mét?
	\choice
	{$24$ m}
	{\True $12$ m}
	{$6$ m}
	{$0{,}4$ m}
	\loigiai{
		Ta có 
		\begin{eqnarray*}
			& & v(t)=-6t+12\\
			&\Rightarrow & s(t)=\displaystyle\int{v(t) \mathrm{\,d}t}\\
			&\Leftrightarrow &  s(t)=\displaystyle\int (-6t+12)\mathrm{\,d}t\\
			&\Leftrightarrow &  s(t)=-3t^2+12t+C.
		\end{eqnarray*}
		Chọn $t=0\Rightarrow s(0)=0 \Rightarrow C=0\Rightarrow s(t)=-3t^2+12t$.\\
		Khi xe dừng hẳn thì $v(t)=0\Leftrightarrow-6t+12=0\Rightarrow t=2$.\\
		Từ lúc đạp phanh đến khi ô tô dừng hẳn thì ô tô còn di chuyển được quãng đường là
		$$S=s(2)-s(0)=s(2)=-3\cdot 2^2 +12\cdot 2 =12\text{ (m).}$$
	}
\end{ex}

\begin{ex}%[2D4H1-6]
	Một ô tô đang chạy với vận tốc $36$ km/h thì tăng tốc chuyển động nhanh dần đều với gia tốc $a(t)=1+\dfrac{t}{3}$ (m/s$^2$) tính quãng đường ô tô đi được sau $6$ giây kể từ khi ô tô bắt đầu tăng tốc.
	\choice
	{\True $S=90$ m}
	{$S=246$ m}
	{$S=58$ m}
	{$S=100$ m}
	\loigiai{
		Đổi $36$ km/h $= 36\cdot \dfrac{1000}{3600}=10$ m/s.\\
		Ta có $a(t)=1+\dfrac{t}{3}$.\\
		$\Rightarrow v(t)=\displaystyle\int{a(t) \mathrm{\,d}t}=\displaystyle\int \left(1+\dfrac{t}{3}\right) \mathrm{\,d}t=t+\dfrac{1}{6}t^2+C$.\\
		Từ lúc bắt đầu tăng tốc thì vận tốc của xe là $10$ m/s nên ta có 
		\begin{eqnarray*}
			& & v(0)=10\\
			&\Rightarrow & C=10\\
			&\Rightarrow & v(t)=t+\dfrac{1}{6}t^2+10\\
			&\Rightarrow & s(t)=\displaystyle\int{v(t) \mathrm{\,d}t}\\
			&\Rightarrow & s(t)=\displaystyle\int (t+\dfrac{1}{6}t^2+10)\\
			&\Rightarrow & s(t)=\dfrac{t^2}{2}+\dfrac{t^3}{18}+10t + C_1.
		\end{eqnarray*}
		Quãng đường tính từ lúc xe bắt đầu tăng tốc nên $s(0)=0 \Rightarrow C_1=0$.\\
		Vậy $s(6)=\dfrac{6^2}{2}+\dfrac{6^3}{18}+10\cdot 6 =90$ (m).
	}
\end{ex}

\begin{ex}%[2D4H1-6]
	Một ca nô đang chạy trên hồ Tây với vận tốc $20$ m/s thì hết xăng; từ thời điểm đó, ca nô chuyển động chậm dần đều với vận tốc $v(t)=-5t+20$ (m/s), trong đó $t$ là khoảng thời gian tính bằng giây, kể từ lúc hết xăng. Hỏi từ lúc hết xăng đến lúc ca nô dừng hẳn thì ca nô đi được bao nhiêu mét?
	\choice
	{$10$ m}
	{$20$ m}
	{$30$ m}
	{\True $40$ m}
	\loigiai{
		Ta có $v(t)=-5t+20$.\\
		$\Rightarrow s(t)=\displaystyle\int{v(t)} \mathrm{\,d}t=\displaystyle\int{\left(-5t+20\right)} \mathrm{\,d}t=-\dfrac{5}{2}t^2+20t+C$.\\
		Chọn $t=0\Rightarrow s(0)=0 \Rightarrow C=0$. Suy ra $s(t)=-\dfrac{5}{2}t^2+20t$.\\
		Khi xe dừng hẳn thì $v(t)=0\Leftrightarrow-5t+20=0\Rightarrow t=4$ (s).\\
		Từ lúc đạp phanh đến khi dừng hẳn, ô tô còn di chuyển được $s = -\dfrac{5}{2}\cdot 4^2+20\cdot 4=40$ (m).
	}
\end{ex}

\begin{ex}%[2D4H1-6]
	Một vật chuyển động với vận tốc $10$ m/s thì tăng tốc với gia tốc được tính theo thời gian $t$ là $a(t)=3t+t^2$ (m$^2$/s). Tính quãng đường vật đi được trong $10$s kể từ khi bắt đầu tăng tốc.
	\choice
	{$\dfrac{130}{3}$ m}
	{$\dfrac{310}{3}$ m}
	{$\dfrac{3400}{3}$ m}
	{\True $\dfrac{4300}{3}$ m}
	\loigiai{
		Ta có $a(t)=3t+t^2$\\
		$\Rightarrow v(t)=\displaystyle\int{a(t) \mathrm{\,d}t}=\displaystyle\int (3t+t^2) \mathrm{\,d}t=\dfrac{3}{2}t^2+\dfrac{1}{3}t^3+C$.\\
		Từ lúc bắt đầu tăng tốc thì vận tốc của xe là $10$ m/s nên ta có \\ 
		$v(0)=10 \Rightarrow C=10$.\\
		$\Rightarrow v(t)=\dfrac{3}{2}t^2+\dfrac{1}{3}t^3+10$.\\
		$\Rightarrow s(t)=\displaystyle\int{v(t) \mathrm{\,d}t}=\displaystyle\int (\dfrac{3}{2}t^2+\dfrac{1}{3}t^3+10) \mathrm{\,d}t=\dfrac{1}{2}t^3+\dfrac{1}{12}t^4+10t+C_1$.\\
		Quãng đường tính từ lúc xe bắt đầu tăng tốc nên $s(0)=0 \Rightarrow C_1=0$.\\
		Vậy $s(10)=\dfrac{1}{2}\cdot 10^3+\dfrac{1}{12}\cdot 10^4+10\cdot 10=\dfrac{4300}{3}$ m.
	}
\end{ex}

\begin{ex}%[2D4H1-6]
	Tại một nơi không có gió, một chiếc khí cầu đang đứng yên ở độ cao $162$ m so với mặt đất đã được phi công cài đặt cho nó chế độ chuyển động đi xuống. Biết rằng, khí cầu đã chuyển động theo phương thẳng đứng với vận tốc tuân theo quy luật $v(t)=10t-t^2$, trong đó $t$ (phút) là thời gian tính từ lúc bắt đầu chuyển động, $v(t)$ được tính theo đơn vị mét/phút (m/p). Nếu như vậy thì khi bắt đầu tiếp đất vận tốc $v$ của khí cầu là
	\choice
	{$5$ m/p}
	{$7$ m/p}
	{\True $9$ m/p}
	{$3$ m/p}
	\loigiai{
		Ta có $v(t)=10t-t^2$.\\
		$\Rightarrow s(t)=\displaystyle\int{v(t) \mathrm{\,d}t}=\displaystyle\int (10t-t^2) \mathrm{\,d}t=5t^2-\dfrac{1}{3}t^3+C$.\\
		Từ lúc bắt đầu giảm độ cao thì khinh khí cầu ở độ cao $162$ m nên ta có \\ 
		$s(0)=0 \Rightarrow C=0\Rightarrow s(t)=5t^2-\dfrac{1}{3}t^3$.\\
		mà $s(t)= 162 \Rightarrow 5t^2-\dfrac{1}{3}t^3=162 \Leftrightarrow t=9$ (s).\\
		Suy ra vận tốc khi chạm đất của khinh khí cầu là
		$$v(9)= 10\cdot 9 -9^2=9 \text{ (m/p)}.$$
	}
\end{ex}

\begin{ex}%[2D4H1-6]
	Một viên đạn được bắn lên theo phương thẳng đứng với vận tốc ban đầu là $25$ m/s, gia tốc trọng trường là $9{,}8$ m/s$^2$. Quãng đường viên đạn đi được từ lúc bắn cho đến khi chạm đất gần bằng kết quả nào nhất trong các kết quả sau?
	\choice
	{$30{,}78$ m}
	{\True $31{,}89$ m}
	{$32{,}43$ m}
	{$33{,}88$ m}
	\loigiai{
		Ta có $a(t)=-9{,}8$ (m/s$^2$).\\
		$\Rightarrow v(t)=\displaystyle\int{a(t) \mathrm{\,d}t}=\displaystyle\int (-9{,}8) \mathrm{\,d}t=-9{,}8t+C$.\\
		Từ lúc bắt đầu bắn viên đạn thì viên đạn có vận tốc $25$ m/s nên ta có\\
		$v(0)=25 \Rightarrow C=25 \Rightarrow v(t)=-9{,}8t+25$.\\ 
		$\Rightarrow s(t)=\displaystyle\int{v(t) \mathrm{\,d}t}=\displaystyle\int (-9{,}8t+25) \mathrm{\,d}t=-4{,}9t^2+25t+C_1$.\\
		Ta có $s(0)=0 \Rightarrow C_1=0$.\\
		$s(t)=-4{,}9t^2+25t$.\\
		Tới khi vận tốc viên đạn bằng không thì ta có $v(t)=0 \Rightarrow -9{,}8t+25=0 \Leftrightarrow t \approx 2{,}55$ s.
		Suy ra quãng đường viên đạn đi được từ lúc bắn cho đến khi chạm đất là\\
		$S= 2\cdot (-4{,}9\cdot (2{,}55)^2+25\cdot 2{,}55) \approx 31{,}89$ m.
	}
\end{ex}

\begin{ex}%[2D4H1-6]
	Trong một đợt xả lũ, nhà máy thủy điện đã xả lũ trong $40$ phút với tốc độ lưu lượng nước tại thời điểm $t$ giây là $h'(t)=10t+500$ (m$^3$/s). Hỏi sau thời gian xả lũ trên thì hồ thoát nước của nhà máy đã thoát đi một lượng nước là bao nhiêu?
	\choice
	{$5\cdot 10^4$ m$^3$}
	{$4\cdot 10^6$ m$^3$}
	{\True $3\cdot 10^7$ m$^3$}
	{$6\cdot 10^6$ m$^3$}
	\loigiai{
		Ta có $h'(t)=10t+500$.\\
		$\Rightarrow h(t)=\displaystyle\int{\left(10t+500\right)}\mathrm{\,dx}=5t^2+500t+C$.\\
		$\Rightarrow h(t)=5t^2+500t+C$.\\
		Chọn $t=0\Rightarrow h(0)=0\Rightarrow C=0$.\\
		$\Rightarrow h(t)=5t^2+500t$.\\
		Thủy điện đã xả lũ trong $40$ phút=$2400$ giây thì thoát đi một lượng nước là
		$$h\left(2400\right)=5\cdot 2400^2+500\cdot 2400=3\cdot 10^7\text{ (m$^3$)}.$$
	}
\end{ex}

\begin{ex}%[2D4H1-6]
	Một bác thợ xây bơm nước vào bể chứa nước. Gọi $h(t)$ là thể tích nước bơm được sau $t$ giây. Cho $h'(t)=3a{t^2}+bt$ (m$^3$/s) và ban đầu bể không có nước. Sau $5$ giây thì thể tích nước trong bể là $150$ m$^3$. Sau $10$ giây thì thể tích nước trong bể là $1100$ m$^3$. Hỏi thể tích nước trong bể sau khi bơm được $20$ giây là bao nhiêu.
	\choice
	{\True $8400$ m$^3$}
	{$7400$ m$^3$}
	{$6000$ m$^3$}
	{$4200$ m$^3$}
	\loigiai{
		Ta có $h'(t)=3a{t^2}+bt$.\\
		$\Rightarrow h(t)=\displaystyle\int{\left(3a{t^2}+bt\right)} \mathrm{\,d}t=a{t^3}+\dfrac{1}{2}b{t^2}+C$.\\
		$\Rightarrow h(t)=a{t^3}+\dfrac{1}{2}b{t^2}+C$.\\
		Chọn $t=0\Rightarrow h(0)=0\Rightarrow C=0$.\\
		$\Rightarrow h(t)=a{t^3}+\dfrac{1}{2}b{t^2}$.\\
		Sau $5$ giây thì thể tích nước trong bể là $150$ m$^3$ là  $h(5)=150\Leftrightarrow 125a+\dfrac{25}{2}b=150$.\hfill(1)\\
		Sau $10$ giây thì thể tích nước trong bể là $1100$ m$^3$ là  $h(10)=1100\Leftrightarrow 1000a+50b=1100$.\hfill(2)\\
		Từ (1) và (2) ta có hệ  $\heva{
			&125a+\dfrac{25}{2}b=150\\
			&1000a+50b=1100
		}\Leftrightarrow \heva{&a=1\\&b=2.}$\\
		$\Rightarrow h(t)=t^3+t^2$.\\
		Thể tích nước trong bể sau khi bơm được $20$ giây là $h\left(20\right)=20^3+20^2=8400$ (m$^3$).}
\end{ex}

\begin{ex}%[2D4H1-6]
	Gọi $h(t)$ (m) là mực nước ở bồn chứa sau khi bơm nước được $t$ giây. Biết rằng $h'(t)=\dfrac{1}{5}\sqrt[3]{t}$ (m/s) và lúc đầu bồn không có nước. Tìm mức nước ở bồn sau khi bơm nước được $6$ giây (\textit{làm tròn kết quả đến hàng phần trăm}).
	\choice
	{$2{,}64$ m}
	{$1{,}22$ m}
	{$2{,}22$ m}
	{\True $1{,}64$ m}
	\loigiai{
		Ta có $h'(t)=\dfrac{1}{5}\sqrt[3]{t}$.\\
		$\Rightarrow h(t)=\displaystyle\int{\dfrac{1}{5}\sqrt[3]{t}}\mathrm{\,dx}=\dfrac{1}{5}\displaystyle\int{t^{\frac{1}{3}}}\mathrm{\,dx}=\dfrac{1}{5}\dfrac{t^{\frac{1}{3}+1}}{\frac{1}{3}+1}+C=\dfrac{3}{20}t\sqrt[3]{t}+C$.\\
		$\Rightarrow h(t)=\dfrac{3}{20}t\sqrt[3]{t}+C$\\
		Chọn $t=0\Rightarrow h(0)=0\Rightarrow C=0$.\\
		$\Rightarrow h(t)=\dfrac{3}{20}t\sqrt[3]{t}$.\\
		Mức nước ở bồn sau khi bơm nước được $6$ giây $h(6)=\dfrac{3}{20}\cdot 6\sqrt[3]{6}\approx 1{,}64$ (m).}
\end{ex}

\begin{ex}%[2D4H1-6]
	Sự sản sinh vi rút Zika ngày thứ $t$ có số lượng là $N(t)$ con, biết $N'(t)=\dfrac{1000}{t}$ và lúc đầu đám vi rút có số lượng $250{,}000$ con. Tính số lượng vi rút sau $10$ ngày.
	\choice
	{$272304$ con}
	{$212302$ con}
	{$242102$ con}
	{\True $252302$ con}
	\loigiai{
		Ta có $N'(t)=\dfrac{1000}{t}$.\\
		$\Rightarrow N(t)=\displaystyle\int{\dfrac{1000}{t} \mathrm{\,d}t=1000\ln \left| t\right|}+C$.\\
		$\Rightarrow N(t)=1000\ln \left| t\right|+C$.\\
		Chọn $t=1\Rightarrow N(1)=250000\Rightarrow C=250000$.\\
		$\Rightarrow N(t)=1000\ln \left| t\right|+250000$.\\
		Số lượng vi rút sau $10$ ngày là $N\left(10\right)=1000\ln 10+250000\approx 252302$ (con).
	}
\end{ex}
% \TNSA
\begin{ex}%[2D4H1-6]
	Một chiếc ô tô đang chạy với vận tốc $15$ m/s thì nhìn thấy chướng ngại vật trên đường cách đó $50$ m, người lái xe hãm phanh khẩn cấp. Sau khi hãm phanh, ô tô chuyển động chậm dần đều với vận tốc $v(t)=-3t+15$ (m/s), trong đó $t$ (giây). Gọi $s(t)$ là quãng đường xe ô tô đi được trong thời gian $t$ (giây) kể từ lúc đạp phanh. Hỏi từ lúc hãm phanh đến khi dừng hẳn, ô tô di chuyển được bao nhiêu mét?
	\shortans[]{$37{,}5$}
	\loigiai{
		Quãng đường xe ô tô đi được trong thời gian $t$ (giây) là một nguyên hàm của $v(t)$ nên\\
		$s(t)=\displaystyle\int{v(t)}\mathrm{\,dx}=\displaystyle\int{(-3t+15)}\mathrm{\,dx}=-\dfrac{3t^2}{2}+15t+C$.\\
		$\Rightarrow s(t)=-\dfrac{3t^2}{2}+15t+C$.\\
		Chọn $t=0\Rightarrow s(0)=0$\\
		$\Rightarrow C=0$\\
		$\Rightarrow s(t)=-\dfrac{3t^2}{2}+15t$\\
		Khi xe dừng hẳn thì $v(t)=0\Leftrightarrow-3t+15=0\Rightarrow t=5$ (s).\\
		Thời gian kể từ lúc đạp phanh đến khi dừng hẳn là $5$ giây\\
		Sau khi đạp phanh đến khi dừng hẳn, xe đi được quãng đường
		$$s(5)=-\dfrac{3.5^2}{2}+15\cdot 5=37{,}5 \text{ (m).}$$
	}
\end{ex}

\begin{ex}%[2D4H1-6]
	Một chiếc ô tô đang chạy với vận tốc $72$ km/h thì nhìn thấy chướng ngại vật trên đường cách đó $40$ m, người lái xe hãm phanh khẩn cấp. Sau khi hãm phanh, ô tô chuyển động chậm dần đều với vận tốc $v(t)=-10t+20$ (m/s), trong đó $t$ tính bằng giây. Gọi $s(t)$ là quãng đường xe ô tô đi được trong thời gian $t$ (giây) kể từ lúc đạp phanh.
	Hỏi từ lúc hãm phanh đến khi dừng hẳn, ô tô di chuyển được bao nhiêu mét?
	\shortans[]{$20$}
	\loigiai{
		Quãng đường xe ô tô đi được trong thời gian $t$ (giây) là một nguyên hàm của $v(t)$ nên\\
		$s(t)=\displaystyle\int{v(t)}\mathrm{\,dx}=\displaystyle\int{(-10t+20)}\mathrm{\,dx}=-5t^2+20t+C$.\\
		$\Rightarrow s(t)=-5t^2+20t+C$.\\
		Chọn $t=0\Rightarrow s(0)=0$.\\
		$\Rightarrow C=0$.\\
		$\Rightarrow s(t)=-5t^2+20t$.\\
		Khi xe dừng hẳn thì $v(t)=0\Leftrightarrow-10t+20=0\Rightarrow t=2$ (s).\\
		Thời gian kể từ lúc đạp phanh đến khi dừng hẳn là $2$ giây.\\
		Sau khi đạp phanh đến khi dừng hẳn, xe đi được quãng đường
		$$s(2)=-5\cdot 2^2+20\cdot 2=20 \text{ (m).}$$
	}
\end{ex}

\begin{ex}%[2D4H1-6]
	Một viên đạn được bắn lên theo phương thẳng đứng từ mặt đất. Tại thời điểm $t$ giây vận tốc của nó được cho bởi công thức $v(t)=24{,}5-9{,}8t$ (m/s). Tính quãng đường viên đạn đi từ lúc bắn lên cho tới khi rơi xuống đất (\textit{làm tròn tới hàng đơn vị}).
	\shortans[]{$61$}
	\loigiai{
		Quãng đường viên đạn đi được là\\ $s(t)=\displaystyle\int{\left(24{,}5-9{,}8t\right)}\mathrm{\,dx}=24{,}5t-4{,}9t^2+C$.\\
		$\Rightarrow s(t)=24{,}5t-4{,}9t^2+C$.\\
		Chọn $t=0\Rightarrow s(0)=0$.\\
		$\Rightarrow C=0$.\\
		$\Rightarrow s(t)=24{,}5t-4{,}9t^2$.\\
		Khi viên đạt đạt độ cao lớn nhất thì $v(t)=0\Leftrightarrow 24,5-9,8t=0\Leftrightarrow t=2{,}5$ (s).\\
		Quãng đường viên đạn đi từ lúc bắn lên cho tới khi rơi xuống đất là
		$$2\cdot s\left(2{,}5\right)=2\left(24{,}5\cdot2{,}5-4{,}9\cdot2{,}5^2\right)=61{,}25 \approx 61 \text{ (m).}$$
	}
\end{ex}
\begin{ex}%[2D4V1-6]
	Mực nước trong hồ chứa của nhà máy điện thủy triều thay đổi trong suốt một ngày do nước chảy ra khi thủy triều xuống và nước chảy vào khi thủy triều lên (như hình vẽ). Tốc độ thay đổi của mực nước được xác định bởi hàm số $h'(t)=\dfrac{1}{90}\left(t^2-17t+60\right)$, trong đó $t$ tính bằng giờ $\left(0\le t\le 24\right)$, $h'(t)$ tính bằng mét/giờ. Tại thời điểm $t=0$, mực nước trong hồ chứa cao $8$ m. Mực nước trong hồ cao nhất là bao nhiêu?
	\shortans[]{$20{,}8$}
	\loigiai{
		Ta có $h'(t)=\dfrac{1}{90}\left(t^2-17t+60\right)$.\\
		$\Rightarrow h(t)=\dfrac{1}{90}\displaystyle\int{\left(t^2-17t+60\right) \mathrm{\,d}t=}\dfrac{1}{90}\left(\dfrac{1}{3}{t^3}-\dfrac{17}{2}{t^2}+60t\right)+C$.\\
		$\Rightarrow h(t)=\dfrac{1}{90}\left(\dfrac{1}{3}{t^3}-\dfrac{17}{2}{t^2}+60t\right)+C$.\\
		Tại thời điểm $t=0$, mực nước trong hồ chứa cao $8$ nên $h(0)=8\Rightarrow C=8$.\\
		$\Rightarrow h(t)=\dfrac{1}{90}\left(\dfrac{1}{3}{t^3}-\dfrac{17}{2}{t^2}+60t\right)+8\rm\left(0\le t\le 24\right)$.\\
		Ta có $h'(t)=0\Leftrightarrow{t^2}-17t+60=0\Leftrightarrow \hoac{
			&t=5\\
			&t=12.
		}$\\
		Lập bảng biến thiên:
		\begin{center}
			\begin{tikzpicture}
				\tkzTabInit[nocadre=false,lgt=1.5,espcl=2.5,deltacl=0.6]
				{$x$/0.7,$h'(x)$/0.7,$h(x)$/2.5}{$0$,$5$,$12$,$24$}
				\tkzTabLine{,+,0,-,0,+,}  
				\tkzTabVar{-/$8$,+/$\dfrac{1019}{108}$,-/$\dfrac{44}{5}$,+/$\dfrac{104}{5}$}
			\end{tikzpicture}
		\end{center}
		Mực nước trong hồ cao nhất: $\dfrac{104}{5}=20{,}8$ m\\
	}
\end{ex}

\begin{ex}
	Gọi $h(t)$ là chiều cao của cây keo (tính theo mét) sau khi trồng $t$ năm. Biết rằng năm đầu tiên cây cao $1{,}5$ m, trong những năm tiếp theo, cây phát triển với tốc độ $h'(t)=\dfrac{1}{\sqrt[4]{t}}$ (mét/năm). Sau bao nhiêu năm cây cao được $3$ m (\textit{kết quả làm tròn tới hàng phần trăm}).
	\shortans[]{$2{,}73$}
	\loigiai{
		Ta có $h'(t)=\dfrac{1}{\sqrt[4]{t}}$.\\
		$\Rightarrow h(t)=\displaystyle\int{\dfrac{1}{\sqrt[4]{t}}} \mathrm{\,d}t=\displaystyle\int{t^{-\frac{1}{4}}} \mathrm{\,d}t=\dfrac{t^{-\frac{1}{4}+1}}{^{-\frac{1}{4}+1}}+C=\dfrac{4}{3}\sqrt[4]{t^3}+C$.\\
		$\Rightarrow h(t)=\dfrac{4}{3}\sqrt[4]{t^3}+C$.\\
		Năm đầu tiên cây cao $1{,}5$ m nên $h(1)=1{,}5\Leftrightarrow 1{,}5=\dfrac{4}{3}\sqrt[4]{1}+C\Rightarrow C=\dfrac{1}{6}$.\\
		$\Rightarrow h(t)=\dfrac{4}{3}\sqrt[4]{t^3}+\dfrac{1}{6}$.\\
		Cây cao được $3$ m nên $h(t)=3\Leftrightarrow\dfrac{4}{3}\sqrt[4]{t^3}+\dfrac{1}{6}=3\Leftrightarrow\sqrt[4]{t^3}=\dfrac{17}{8}\Rightarrow t\approx 2{,}73$ (m).
	}
\end{ex}

\begin{ex}%[2D4H1-6]
	Người ta bơm nước vào một bồn chứa, lúc đầu bồn không chứa nước, mức nước ở bồn chứa sau khi bơm phụ thuộc vào thời gian bơm nước theo một hàm số $h=h(t)$ trong đó $h$ tính bằng cm, $t$ tính bằng giây. Biết rằng $h'(t)=\sqrt[3]{2t}$ (cm/s). Mức nước ở bồn sau khi bơm được $13$ giây là bao nhiêu? (\textit{kết quả làm tròn tới hàng đơn vị}).
	\shortans[]{$23$}
	\loigiai{
		Ta có $h'(t)=\sqrt[3]{2t}$ (cm/s).\\
		$\Rightarrow h(t)=\displaystyle\int{\sqrt[3]{2t}} \mathrm{\,d}t=\sqrt[3]{2}\displaystyle\int{t^{\frac{1}{3}}} \mathrm{\,d}t=\sqrt[3]{2}\dfrac{t^{\frac{1}{3}+1}}{^{\frac{1}{3}+1}}+C=\dfrac{3\sqrt[3]{2}}{4}\sqrt[3]{t^4}+C$.\\
		$\Rightarrow h(t)=\dfrac{3}{4}\sqrt[3]{t^4}+C$.\\
		Lúc đầu bồn không chứa nước nên $h(0)=0\Rightarrow C=0$.\\
		$\Rightarrow h(t)=\dfrac{3}{4}\sqrt[3]{t^4}$ (cm).\\
		Mức nước ở bồn sau khi bơm được $13$ giây là $h(t)=\dfrac{3}{4}\sqrt[3]{13^4}\approx 23$ cm.
	}
\end{ex}

\begin{ex}%[2D4H1-6]
	Khi quan sát một đám vi khuẩn trong phòng thí nghiệm người ta thấy tại ngày thứ $t$ có số lượng là $N(t)$. Biết rằng $N'(t)=\dfrac{1500}{t}$ và tại ngày thứ nhất số lượng vi khuẩn là $5000$ con. Tính số lượng vi khuẩn tại ngày thứ $12$ (\textit{làm tròn đến hàng đơn vị}).
	\shortans[]{$8727$}
	\loigiai{
		Ta có $N'(t)=\dfrac{1500}{t}$.\\
		$\Rightarrow N(t)=\displaystyle\int{\dfrac{1500}{t}} \mathrm{\,d}t=1500 \cdot \ln t +C$.\\
		$\Rightarrow N(t)=1500\ln t +C$.\\
		Tại ngày thứ nhất số lượng vi khuẩn là $5000$ con nên\\
		$N(1)=5000 \Rightarrow C= 5000$
		$\Rightarrow N(t)=1500\ln t + 5000$ (con).\\
		Số lượng vi khuẩn tại ngày thứ $12$ là
		$N(12)=1500\ln 12 + 5000 \approx 8727$ con.
	}
\end{ex}

\begin{ex}%[2D4H1-6]
	Vi khuẩn HP (Helicobacter pylori) gây đau dạ dày, tại ngày thứ $t$ với số lượng là $F(t)$. Biết $F'(t)=\dfrac{600}{t}$ và ban đầu bệnh nhân có $2000$ con vi khuẩn. Sau $15$ ngày bệnh nhân phát hiện ra bị bệnh. Hỏi khi đó có bao nhiêu con vi khuẩn trong dạ dày (lấy xấp xỉ tới hàng đơn vị)? Biết rằng nếu phát hiện sớm khi số lượng không vượt quá $4000$ con thì bệnh nhân sẽ được cứu chữa.
	\shortans[]{$3625$}
	\loigiai{
		Ta có 
		\begin{eqnarray*}
			& & F'(t)=\dfrac{600}{t}\\
			&\Leftrightarrow & F(t)=\displaystyle\int{\dfrac{600}{t}} \mathrm{\,d}t\\
			&\Leftrightarrow & F(t)=600 \cdot \ln t +C.
		\end{eqnarray*}
		Tại ngày thứ nhất số lượng vi khuẩn là $2000$ con nên\\
		$ F(1)=2000 \Rightarrow C= 2000$.\\
		$\Rightarrow F(t)=600\ln t + 2000$\\
		Số lượng vi khuẩn sau $15$ ngày là	$F(15)=600\ln 15 + 2000 \approx 3625$ (con).
	}
\end{ex}
\Closesolutionfile{ans}
% \indapan{6}{ans/ans2-C4B1CD4-D2-KQ}
% \subsection{NGUYÊN HÀM HÀM ẨN}

\subsubsection*{Cần nhớ các công thức đạo hàm của hàm hợp}
\begin{itemize}[\color{blue}\faPencilSquare]
	\item $\int{f'(x)\mathrm{d}x}=f(x)+C$
	\item $f'(x)\cdot g(x)+f(x)\cdot g'(x)=\left[f(x)\cdot g(x)\right]'$
	\item $\dfrac{f'(x)\cdot g(x)-f(x)\cdot g'(x)}{g^2(x)} =\left[\dfrac{f(x)}{g(x)}\right]'$
	\item $\dfrac{f'(x)}{f(x)}=\left[\ln f(x) \right]'$
	\item $-\dfrac{f'(x)}{f^2(x)}=\left[ \dfrac{1}{f(x)} \right]'$
	\item $-\dfrac{f'(x)}{f^n(x)}=\left[ \dfrac{1}{(n-1)\left[ f(x) \right]^{n-1}} \right]'$
	\item $n\cdot f'(x)\cdot f^{n-1}(x)=\left[ f^n(x) \right]'$
	\item $\dfrac{f'(x)}{\sqrt{f(x)}}=\left[ 2\sqrt{f(x)} \right]'$
\end{itemize}

\begin{dang}{~}
	\subsubsection{Điều kiện hàm ẩn có dạng}
	$$\left[ \begin{aligned}
			 & f'(x)=g(x)\cdot h\left[ f(x) \right]  \\
			 & f'(x)\cdot h\left[ f(x) \right]=g(x).
		\end{aligned} \right.$$
	\textbf{Phương pháp giải}
	\begin{itemize}[\color{blue}\faPencilSquareO]
		\item $\dfrac{f'(x)}{h[f(x)]}=g(x) \Leftrightarrow \displaystyle\int \dfrac{f'(x)}{h[f(x)]}\mathrm{d}x =\int  {g(x)}\mathrm{d}x \Leftrightarrow \int\dfrac{\mathrm{d}\left[ f(x) \right]}{h\left[ f(x) \right]} =\int  {g(x)\mathrm{d}x}$.
		\item $f'(x)h[f(x)]=g(x)
			      \Leftrightarrow
			      \displaystyle\int f'(x)h[f(x)]\mathrm{d}x=\int g(x)\mathrm{d}x
			      \Leftrightarrow
			      \int h[f(x)]\mathrm{d}\left[ f'(x) \right]=\int g(x)\mathrm{d}x$.
	\end{itemize}
	Chú ý: Ngoài việc nguyên hàm hai vế, ta có thể lấy tích phân hai vế (tùy câu hỏi của bài toán)
	\subsubsection{Điều kiện hàm ẩn có dạng}
	$$u(x)f'(x)+u'(x)f(x)=h(x)$$
	\textbf{Phương pháp giải}
	Dễ dàng thấy rằng $u(x)f'(x)+u'(x)f(x)=[u(x)f(x)]'$.\\
	Do dó $u(x)f'(x)+u'(x)f(x)=h(x) \Leftrightarrow [u(x)f(x)]'=h(x)$.\\
	Suy ra $u(x)f(x)=\displaystyle\int   h(x)\mathrm{d}x$.\\
	Từ đây ta dễ dàng tính được $f(x)$.
\end{dang}

%PHẦN I. Câu trắc nghiệm nhiều phương án lựa chọn. Mỗi câu hỏi thí sinh chỉ chọn một phương án.
% \TN
\Opensolutionfile{ans}[ans/ans-LC-2-C4B1CD3_1-8]

%%%==============EX_1============%%%
\begin{ex}%[2D4V1-3]
	Cho hàm số $f(x)$ thỏa mãn $f\left(\dfrac{\pi}{4} \right)=0$ và $f'(x)\sin^2\dfrac{x}{2}\cos^2\dfrac{x}{2}=1$. Tính $f\left(\dfrac{\pi}{2} \right)$.
	\choice
	{$f\left(\dfrac{\pi}{2} \right)=1$}
	{$f\left(\dfrac{\pi}{2} \right)=-1$}
	{$f\left(\dfrac{\pi}{2} \right)=2$}
	{\True $f\left(\dfrac{\pi}{2} \right)=4$}
	\loigiai{
		Ta có
		$\begin{aligned}[t]
				            & \quad f'(x)\sin^2\dfrac{x}{2}\cos^2\dfrac{x}{2}=1             \\
				\Rightarrow & \quad f'(x)=\dfrac{1}{\sin^2\dfrac{x}{2}\cos^2\dfrac{x}{2}}   \\
				\Rightarrow & \quad f'(x)=\dfrac{1}{\tfrac{1}{4}\sin^2}x                    \\
				\Rightarrow & \quad f(x)=4\displaystyle\int  \dfrac{1}{\sin^2x}\mathrm{d}x.
			\end{aligned}$\\
		Tìm được $f(x)=-4\cot x+C$.\\
		Với $f\left(\dfrac{\pi}{4} \right)=0$ thì $C=4$.\\
		Suy ra $f(x)=-4\cot x+4$.\\
		Vậy $f\left(\dfrac{\pi}{2}\right) =-4\cot\dfrac{\pi}{2}+4=4$.
	}
\end{ex}
%%%==============EX_2============%%%
\begin{ex}%[2D4V1-2]
	Cho hàm số $y=f(x)$ thỏa mãn $f'(x)\cdot f(x)=x^4+x^2$. Biết $f(0)=2$. Tính $f^2(2)$.
	\choice
	{$f^2(2)=\dfrac{313}{15}$}
	{\True $f^2(2)=\dfrac{332}{15}$}
	{$f^2(2)=\dfrac{324}{15}$}
	{$f^2(2)=\dfrac{323}{15}$}
	\loigiai{
		Ta có
		$\begin{aligned}[t]
				f'(x)\cdot f(x)=x^4+x^2
				 & \Leftrightarrow f(x)\mathrm{d}f(x)=(x^4+x^3)\mathrm{d}x                                    \\
				 & \Leftrightarrow \displaystyle\int f'(x)\cdot f(x)\mathrm{d}x =\int{(x^4+x^2)\mathrm{d}x}+C \\
				 & \Rightarrow \dfrac{f^2(x)}{2} =\dfrac{x^5}{5}+\dfrac{x^3}{3}+C.
			\end{aligned}$\\
		Do
		$\begin{aligned}[t]
				f(0)=2 & \Rightarrow \dfrac{f^2(0)}{2}=\dfrac{0^5}{5}+\dfrac{0^3}{3}+C
				       & \Rightarrow C=2.
			\end{aligned}$\\
		Vậy $f^2(2)=2\left(\dfrac{32}{5}+\dfrac{8}{3}+2 \right) =\dfrac{332}{15}$.
	}
\end{ex}
%%%==============EX_3============%%%
\begin{ex}%[2D4V1-2]
	Cho hàm số $y=f(x)$ có đạo hàm liên tục trên đoạn $[-2;1]$ thỏa mãn $f(0)=3$ và $\left(f(x)\right)^2\cdot f'(x)=3x^2+4x+2$. Giá trị $f(1)$ là
	\choice
	{$2\sqrt[3]{42}$}
	{$2\sqrt[3]{15}$}
	{\True $\sqrt[3]{42}$}
	{$\sqrt[3]{15}$}
	\loigiai{
		Ta có $\left(f(x)\right)^2\cdot f'(x)=3x^2+4x+2$ (*).\\
		Lấy nguyên hàm 2 vế của phương trình trên ta được
		\begin{eqnarray*}
			\displaystyle\int  f(x)^2\cdot f'(x)\mathrm{d}x =\int\left(3x^2+4x+2\right)\mathrm{d}x
			& \Leftrightarrow & \int \left(f(x)\right)^2\mathrm{d}f(x) =x^3+2x^2+2x+C\\
			& \Leftrightarrow & \dfrac{\left(f(x)\right)^3}{3} =x^3+2x^2+2x+C\\
			& \Leftrightarrow & \left(f(x)\right)^3 =3(x^3+2x^2+2x+C) \quad(1).
		\end{eqnarray*}
		Theo đề bài
		$\begin{aligned}[t]
				f(0)=3 & \overset{(1)}{\Rightarrow} \left(f(0)\right)^3 =3(0^3+2.0^2+2\cdot0+C) \\
				       & \Leftrightarrow 27=3C                                                  \\
				       & \Leftrightarrow C=9.
			\end{aligned}$\\
		Suy ra
		$\begin{aligned}[t]
				            & \left(f(x)\right)^3=3(x^3+2x^2+2x+9)
				\Rightarrow & \ f(x)=\sqrt[3]{3(x^3+2x^2+2x+9)}.
			\end{aligned}$\\
		Vậy $f(1)=\sqrt[3]{42}$.
	}
\end{ex}
%%%==============EX_4============%%%
\begin{ex}%[2D4C1-2]
	Cho hàm số $f(x)$ thỏa mãn $f(2)=-\dfrac{1}{3}$ và $f'(x)=x\left[f(x)\right]^2$ với mọi $x\in \mathbb{R}$. Giá trị của $f(1)$ bằng
	\choice
	{\True $f(1)=-\dfrac{2}{3}$}
	{$f(1)=-\dfrac{2}{9}$}
	{$f(1)=-\dfrac{7}{6}$}
	{$f(1)=-\dfrac{11}{6}$}
	\loigiai{
		Từ hệ thức đề cho: $f'(x)=x\left[f(x)\right]^2$ (1), suy ra $f'(x)\ge 0$ với mọi $x\in [1;2]$ \\
		Do đó $f(x)$ là hàm không giảm trên đoạn $[1;2]$, ta có $f(x)\le f(2) < 0$ với mọi $x\in [1;2]$
		\begin{enumerate}[\color{blue}\bf Cách 1.]
			\item Lấy nguyên hàm\\
			      Ta có
			      $\begin{aligned}[t]
					      f'(x)=x\left[f(x)\right]^2
					       & \Rightarrow \dfrac{f'(x)}{\left[f(x)\right]^2}=x    \\
					       & \Rightarrow \left(-\dfrac{1}{f(x)} \right)'=x       \\
					       & \Rightarrow \left(\dfrac{1}{f(x)} \right)'=-x       \\
					       & \Rightarrow \dfrac{1}{f(x)}=\displaystyle\int(-x)dx \\
					       & \Rightarrow \dfrac{1}{f(x)}=-\dfrac{x^2}{2}+C.
				      \end{aligned}$\\
			      Mà
			      $\begin{aligned}[t]
					      f(2)=-\dfrac{1}{3}
					       & \Rightarrow \dfrac{1}{f(2)}=-2+C              \\
					       & \Leftrightarrow \dfrac{1}{-\tfrac{1}{3}}=-2+C \\
					       & \Rightarrow C=-1.
				      \end{aligned}$\\
			      Tìm được $\dfrac{1}{f(x)}=-\dfrac{x^2}{2}-1$.\\
			      Cho nên $\dfrac{1}{f(1)}=-\dfrac{1}{2}-1 \Leftrightarrow f(1)=-\dfrac{2}{3}$.
			\item Chia 2 vế hệ thức $(1)$ cho $\left[f(x)\right]^2$, ta được $ \dfrac{f'(x)}{\left[f(x)\right]^2}=x,\forall x\in[1;2]$. \\
			      Lấy tích phân 2 vế trên đoạn $[1;2]$ hệ thức vừa tìm được, ta được:
			      \begin{align*}
				      \displaystyle\int\limits_1^2\dfrac{f'(x)}{\left[f(x)\right]^2}\mathrm{d}x =\displaystyle\int\limits_1^2x\mathrm{d}x
				       & \Rightarrow \int\limits_1^2\dfrac{1}{\left[f(x)\right]^2}\mathrm{d}f(x)=\dfrac{3}{2} \\
				       & \Leftrightarrow \dfrac{-1}{f(x)} \bigg|_1^2=\dfrac{3}{2}                             \\
				       & \Leftrightarrow \dfrac{1}{f(1)}-\dfrac{1}{f(2)}=\dfrac{3}{2}                         \\
				       & \Leftrightarrow \dfrac{1}{f(1)}=\dfrac{1}{f(2)}+\dfrac{3}{2}                         \\
				       & \Leftrightarrow f(1)=\dfrac{2f(2)}{2+3f(2)}.
			      \end{align*}
			      Với $f(2)=-\dfrac{1}{3}$ thì $f(1)=\dfrac{2\cdot\left(-\frac{1}{3}\right)}{2+3\cdot\left(-\frac{1}{3}\right)}=-\dfrac{2}{3}$.
		\end{enumerate}
	}
\end{ex}
%%%==============EX_5============%%%
\begin{ex}%[2D4V1-2]
	Cho hàm số $f(x)$ thỏa mãn $f(2)=-\dfrac{1}{25}$ và $f'(x)=4x^3\left[f(x)\right]^2$ với mọi $x\in\mathbb{R}$. Giá trị của $f(1)$ bằng
	\choice
	{$-\dfrac{391}{400}$}
	{$-\dfrac{1}{40}$}
	{$-\dfrac{41}{400}$}
	{\True $-\dfrac{1}{10}$}
	\loigiai{
		Ta có
		$\begin{aligned}[t]
				f'(x)=4x^3\left[f(x)\right]^2
				 & \Rightarrow-\dfrac{f'(x)}{\left[f(x)\right]^2}=-4x^3 \\
				 & \Rightarrow \left[\dfrac{1}{f(x)}\right]'=-4x^3      \\
				 & \Rightarrow \dfrac{1}{f(x)}=-x^4+C.
			\end{aligned}$\\
		Với $f(2)=-\dfrac{1}{25}$ thì $\dfrac{1}{f(2)}=-2^4+C \Leftrightarrow -25=-16+C \Leftrightarrow C=-9$. \\
		Suy ra $f(x)=-\dfrac{1}{x^4+9}$.\\
		Vậy $f(1)=-\dfrac{1}{10}$.
	}
\end{ex}
%%%==============EX_6============%%%
\begin{ex}%[2D4V1-2]
	Cho hàm số $f(x)$ thỏa mãn $f(2)=-\dfrac{1}{5}$ và $f'(x)=x^3\left[f(x)\right]^2$ với mọi $x\in\mathbb{R}$. Giá trị của $f(1)$ bằng
	\choice
	{$-\dfrac{4}{35}$}
	{$-\dfrac{71}{20}$}
	{$-\dfrac{79}{20}$}
	{\True $-\dfrac{4}{5}$}
	\loigiai{
		Ta có
		$\begin{aligned}[t]
				f'(x)=x^3\left[f(x)\right]^2
				 & \Rightarrow \dfrac{f'(x)}{f^2(x)}=x^3                                                                                                   \\
				 & \Rightarrow \displaystyle\int\limits_1^2\dfrac{f'(x)}{f^2(x)}\mathrm{d}x =\int\limits_1^2x^3\mathrm{d}x                                 \\
				 & \Leftrightarrow -\dfrac{1}{f(x)} \bigg|_1^2 =\dfrac{1}{4}\cdot(2^4-1^4)                                                                 \\
				 & \Leftrightarrow -\dfrac{1}{f(2)}+\dfrac{1}{f(1)} =\dfrac{15}{4}                                                                         \\
				 & \Leftrightarrow \dfrac{1}{f(1)}=\dfrac{4+15f(2)}{4f(2)}                                                                                 \\
				 & \Leftrightarrow f(1)=\dfrac{4f(2)}{4+15f(2)}=\dfrac{4\cdot\left(-\frac{1}{5}\right)}{4+15\cdot\left(-\frac{1}{5}\right)}=-\dfrac{4}{5}.
			\end{aligned}$\\
		Vậy $f(1)=-\dfrac{4}{5}$.
	}
\end{ex}
%%%==============EX_7============%%%
\begin{ex}%[2D4V1-2]
	Cho hàm số $y=f(x)$ thỏa mãn $f(2)=-\dfrac{4}{19}$ và $f'(x)=x^3f^2(x)\, \forall x\in \mathbb{R}$. Giá trị của $f(1)$ bằng
	\choice
	{$-\dfrac{2}{3}$}
	{$-\dfrac{1}{2}$}
	{\True $-1$}
	{$-\dfrac{3}{4}$}
	\loigiai{
		Ta có
		$\begin{aligned}[t]
				f'(x)=x^3f^2(x) & \Leftrightarrow \dfrac{f'(x)}{f^2(x)}=x^3                                          \\
				                & \Rightarrow \displaystyle\int\dfrac{f'(x)}{f^2(x)}\mathrm{d}x =\int x^3\mathrm{d}x \\
				                & \Leftrightarrow -\dfrac{1}{f(x)} =\dfrac{x^4}{4}+C.
			\end{aligned}$\\
		Với $f(2)=-\dfrac{4}{9}$ thì $\dfrac{19}{4} =\dfrac{16}{4}+C \Rightarrow C=\dfrac{3}{4}$.\\
		Tìm được $f(x)=-\dfrac{4}{x^4+3}$.\\
		Vậy $f(1)=-1$.
	}
\end{ex}
%%%==============EX_8============%%%
\begin{ex}%[2D4V1-4]
	Cho hàm số $f(x)>0$ xác định và liên tục trên $\mathbb{R}$ đồng thời thỏa mãn $f(0)=\dfrac{1}{2}$, $f'(x)=-\mathrm{e}^xf^2(x),\, \forall x\in \mathbb{R}$. Tính giá trị của $f(\ln 2)$.
	\choice
	{$f(\ln 2)=\dfrac{1}{4}$}
	{\True $f(\ln 2)=\dfrac{1}{3}$}
	{$f(\ln 2)=\ln 2+\dfrac{1}{2}$}
	{$f(\ln 2)=\ln ^22+\dfrac{1}{2}$}
	\loigiai{
		Ta có
		$\begin{aligned}[t]
				f'(x)=-\mathrm{e}^xf^2(x)
				 & \Leftrightarrow \dfrac{f'(x)}{f^2(x)}=-\mathrm{e}^x\ (\text{do } f(x)>0)                          \\
				 & \Leftrightarrow \displaystyle\int\dfrac{f'(x)}{f^2(x)}\mathrm{d}x =\int(-\mathrm{e}^x)\mathrm{d}x \\
				 & \Rightarrow -\dfrac{1}{f(x)}=-\mathrm{e}^x+C                                                      \\
				 & \Rightarrow f(x)=\dfrac{1}{\mathrm{e}^x-C}.
			\end{aligned}$\\
		Với $f(0)=\dfrac{1}{2}$ thì $\dfrac{1}{\mathrm{e}^0-C} =\dfrac{1}{2} \Rightarrow C=-1$.\\
		Suy ra $f(x)=\dfrac{1}{\mathrm{e}^x+1}$.\\
		Vậy $f(\ln 2) =\dfrac{1}{\mathrm{e}^{\ln 2}+1}=\dfrac{1}{3}$.
	}
\end{ex}
%%%==============EX_9============%%%
\begin{ex}%[2D4V1-2]
	Cho hàm số $f(x)\ne0$ thỏa mãn điều kiện $f'(x)=(2x+3)f^2(x)$ và $f(0)=-\dfrac{1}{2}$. Biết rằng tổng $f(1)+f(2)+f(3)+\cdots+f(2024)+f(2025) =\dfrac{a}{b}$ với $\left(a\in \mathbb{Z}, b\in \mathbb{N}^{*} \right)$ và $\dfrac{a}{b}$ là phân số tối giản. Mệnh đề nào sau đây đúng?
	\choice
	{$\dfrac{a}{b} <-1$}
	{$\dfrac{a}{b} > 1$}
	{$a+b=1010$}
	{\True $b-a=1519$}
	\loigiai{
		Ta có
		$\begin{aligned}[t]
				f'(x)=(2x+3)f^2(x)
				 & \Leftrightarrow \dfrac{f'(x)}{f^2(x)}=2x+3                                             \\
				 & \Leftrightarrow \displaystyle\int\dfrac{f'(x)}{f(x)}\mathrm{d}x =\int(2x+3)\mathrm{d}x \\
				 & \Leftrightarrow -\dfrac{1}{f(x)}=x^2+3x+C.
			\end{aligned}$\\
		Với $f(0)=-\dfrac{1}{2}$, thì $-\dfrac{1}{-\frac{1}{2}}=0^2+3\cdot0+C \Rightarrow C=2$\\
		Suy ra $f(x)=-\dfrac{1}{(x+1)(x+2)}=\dfrac{1}{x+2}-\dfrac{1}{x+1}$\\
		Ta có: $\left\{\begin{aligned}
				 & f(1)=\dfrac{1}{3}-\dfrac{1}{2}          \\
				 & f(2)=\dfrac{1}{4}-\dfrac{1}{3}          \\
				 & f(3)=\dfrac{1}{5}-\dfrac{1}{4}          \\
				 & \vdots                                  \\
				 & f(2025)=\dfrac{1}{2026}-\dfrac{1}{2025} \\
			\end{aligned} \right.$ \\
		Tổng $f(1)+f(2)+f(3)+\cdots+f(2024)+f(2025) =-\dfrac{1}{2}+\dfrac{1}{2026}=-\dfrac{506}{1013}$\\
		Do $a\in \mathbb{Z}, b\in \mathbb{N}^{*}
			\Rightarrow a=-506, b=1013\Rightarrow b-a=1519$.
	}
\end{ex}
%%%==============EX_10============%%%
\begin{ex}%[2D4V1-2]
	Cho hàm số $y=f(x)$ đồng biến trên $(0;+\infty)$; $y=f(x)$ liên tục, nhận giá trị dương trên $(0;+\infty)$ và thỏa mãn $f(3)=\dfrac{4}{9}$ và $\left[f'(x)\right]^2=xf(x)$. Tính $f(8)$.
	\choice
	{\True $f(8)=\dfrac{43-24\sqrt{3}}{9}$}
	{$f(8)=\dfrac{43+24\sqrt{3}}{9}$}
	{$f(8)=\dfrac{43-\sqrt{3}}{3}$}
	{$f(8)=\dfrac{43+\sqrt{3}}{3}$}
	\loigiai{
		Với $\forall x\in (0;+\infty)$ thì $y=f(x) > 0$; $x+1 > 0$\\
		Hàm số $y=f(x)$ đồng biến trên $(0;+\infty)$ nên $f'(x)\ge0, \forall x\in (0;+\infty)$\\
		Ta có
		$\begin{aligned}[t]
				\left[f'(x)\right]^2=xf(x)
				 & \Rightarrow f'(x)=\sqrt{xf(x)}                                  \\
				 & \Rightarrow \dfrac{f'(x)}{\sqrt{f(x)}}=\sqrt{x}                 \\
				 & \Rightarrow 2\left(\sqrt{f(x)}\right)'=\sqrt{x}                 \\
				 & \Rightarrow \left(\sqrt{f(x)}\right)'=\dfrac{1}{2}\sqrt{x}      \\
				 & \Rightarrow \sqrt{f(x)}=\dfrac{1}{2}\displaystyle\int\sqrt{x}dx \\
				 & \Rightarrow \sqrt{f(x)}=\dfrac{1}{3}\sqrt{x^3}+C.
			\end{aligned}$\\
		Với $f(3)=\dfrac{4}{9}$, thì $\sqrt{f(3)}=\dfrac{1}{3}\cdot\sqrt{3^3}+C \Leftrightarrow \dfrac{2}{3}=\sqrt{3}+C \Leftrightarrow C=\dfrac{2-\sqrt{3}}{3}$.\\
		Tìm được $\sqrt{f(x)}=\dfrac{1}{3}\sqrt{x^3}+\dfrac{2-3\sqrt{3}}{3}$
		$\Rightarrow f(x)=\left(\dfrac{1}{3}\sqrt{x^3}+\dfrac{2-3\sqrt{3}}{3}\right)^2$.\\
		Vậy $f(8)=\dfrac{43-24\sqrt{3}}{9}$.
	}
\end{ex}
%%%==============EX_11============%%%
\begin{ex}%[2D4V1-4]%[2D4V1-4]%[2D4H1-4]%[2D4H1-4]
	Cho hàm số $f(x)>0$ với mọi $x\in\mathbb{R}$, $f(0)=1$ và $f(x)=\sqrt{x}\cdot f'(x)$ với mọi $x\in\mathbb{R}$. Mệnh đề nào dưới đây đúng?
	\choice
	{$f(3)<2$}
	{$2<f(3)<4$}
	{\True $f(3)>6$}
	{$4<f(3)<6$}
	\loigiai{
	Ta có
	$\begin{aligned}[t]
			f(x)=\sqrt{x}\cdot f'(x)
			 & \Rightarrow \dfrac{f'(x)}{f(x)}=\dfrac{1}{\sqrt{x}}                  \\
			 & \Rightarrow \ln f(x)=\displaystyle\int\dfrac{1}{\sqrt{x}}\mathrm{d}x \\
			 & \Leftrightarrow \ln f(x)=2\sqrt{x}+C                                 \\
			 & \Leftrightarrow f(x)=\mathrm{e}^{2\sqrt{x}+C}.
		\end{aligned}$\\
	Với $f(0)=1$ thì $\ln f(0)=2\sqrt{0}+C \Leftrightarrow C=0$.\\
	Suy ra $f(x)=\mathrm{e}^{2\sqrt{x}}$.\\
	Vậy $f(3)=\mathrm{e}^{2\sqrt{3}}>6$.
	}
\end{ex}
\begin{ex}%[2D4V1-2]
	Cho hàm số $ f(x)$ có đạo hàm cấp hai trên đoạn $\left[0;1\right]$ đồng thời thỏa mãn các điều kiện $f'(0)=-1$, $f'(x)<0$, $\left[f'(x)\right]^2=f''(x)$, $\forall x\in\left[0;1\right]$. Giá trị $ f'(2)$ thuộc khoảng
	\choice
	{$(2;3)$}
	{\True $(-2;0)$}
	{$(0;2)$}
	{$(-3;-2)$}
	\loigiai
	{
		Ta có
		\begin{align*}
			\left[f'(x)\right]^2=f''(x) \Leftrightarrow\dfrac{f''(x)}{\left[f'(x)\right]^2}=1\Leftrightarrow-\left(\dfrac{1}{f'(x)}\right)'=1\Rightarrow\dfrac{1}{f'(x)}=-\displaystyle\int{\mathrm{d} x} \Leftrightarrow\dfrac{1}{f'(x)}=-x+C.
		\end{align*}
		Mà $f'(0)=-1\Rightarrow C=-1$ suy ra
		$$\dfrac{1}{f'(x)}=-x-1\Rightarrow f'(x)=-\dfrac{1}{x+1} \Rightarrow{f}'(2)=-\dfrac{1}{3}.$$
	}
\end{ex}

\begin{ex}%[2D4V2-2]
	Cho hàm số $ f(x)$ đồng biến có đạo hàm đến cấp hai trên đoạn $\left[0;2\right]$ và thỏa mãn $\left[f(x)\right]^2-f(x)\cdot f''(x)+\left[f'(x)\right]^2=0$. Biết $ f(0)=1$, $ f(2)={\mathrm{e}}^6$. Khi đó $ f(1)$ bằng
	\choice
	{$\mathrm{e}^{\tfrac{3}{2}}$}
	{$\mathrm{e}^3$}
	{\True $\mathrm{e}^{\tfrac{5}{2}}$}
	{$\mathrm{e}^2$}
	\loigiai
	{
	Theo đề bài, ta có
	\begin{align*}
		\left[f(x)\right]^2-f(x)\cdot f''(x)+\left[f'(x)\right]^2=0
		 & \Rightarrow\dfrac{f(x)\cdot f''(x)-\left[f'(x)\right]^2}{\left[f(x)\right]^2}=1 \\
		 & \Rightarrow{\left[\dfrac{f'(x)}{f(x)}\right]'}=1                                \\& \Rightarrow\dfrac{f'(x)}{f(x)}=x+C\\
		 & \Rightarrow\ln f(x)=\dfrac{x^2}{2}+Cx+D.
	\end{align*}
	Mà $\heva{
			& f(0)=1\\
			& f(2)=\mathrm{e}^6} \Leftrightarrow\heva{
			& C=2\\
			& D=0.}$\\
	Suy ra $f(x)=\mathrm{e}^{\tfrac{x^2}{2}+2x}\Rightarrow f(1)=\mathrm{e}^{\tfrac{5}{2}}$.}
\end{ex}
\begin{ex}%[2D4V1-2]
	Cho hàm số $ f(x)$ thỏa mãn $(f'(x))^2+f(x)\cdot f''(x)=x^3-2x$, $\forall x\in\mathbb{R}$ và \break $ f(0)=f'(0)=1$. Giá trị của $ T=f^2(2)$ bằng
	\choice
	{$\dfrac{43}{30}$}
	{$\dfrac{16}{15}$}
	{\True $\dfrac{43}{15}$}
	{$\dfrac{26}{15}$}
	\loigiai
	{
		Ta có
		\begin{align*}
			\left( f'(x)\right)^2+f(x)\cdot f''(x)=x^3-2x & \Leftrightarrow \left( f(x)\cdot f'(x)\right)'=x^3-2x \\& \Rightarrow f(x)\cdot f'(x)=\displaystyle\int{(x^3-2x})\mathrm{\,d} x\\&\Rightarrow f(x)\cdot f'(x)=\dfrac{1}{4}{x^4}-x^2+C.
		\end{align*}
		Từ $ f(0)=f'(0)=1$ suy ra $C=1$. Do đó $ f(x)\cdot f'(x)=\dfrac{1}{4}{x^4}-x^2+1$.\\
		Lại có
		\begin{align*}
			2f(x)\cdot f'(x)=\dfrac{1}{2}{x^4}-2x^2+2 & \Leftrightarrow \left( f^2(x)\right) '=\dfrac{1}{2}{x^4}-2x^2+2 \\& \Rightarrow{f^2}(x)=\displaystyle\int{\left(\dfrac{1}{2}{x^4}-2x^2+2\right)}\mathrm{\,d} x\\& \Rightarrow{f^2}(x)=\dfrac{1}{10}{x^5}-\dfrac{2}{3}{x^3}+2x+C.
		\end{align*}
		Vì $ f(0)=1$ nên $C=1$. Do đó $f^2(x)=\dfrac{1}{10}{x^5}-\dfrac{2}{3}{x^3}+2x+1$.\\
		Vậy $T=\dfrac{43}{15}$.}
\end{ex}

\begin{ex}%[2D4V1-2]
	Cho hàm số $ f(x)$ thỏa mãn $\left[f'(x)\right]^2+f(x)\cdot f''(x)=2x^2-x+1$, $\forall x\in\mathbb{R}$ và $ f(0)=f'(0)=3$. Giá trị của $\left[f(1)\right]^2$ bằng
	\choice
	{\True $ 28$}
	{$ 22$}
	{$\dfrac{19}{2}$}
	{$ 10$}
	\loigiai
	{
		Ta có $\left[f(x){f}'(x)\right]'=\left[f'(x)\right]^2+f(x)\cdot f''(x)$.\\
		Do đó theo giả thiết ta được $\left[f(x){f}'(x)\right]'=2x^2-x+1$.\\
		Suy ra $f(x){f}'(x)=\dfrac{2}{3}{x^3}-\dfrac{x^2}{2}+x+C$.\\
		Hơn nữa $ f(0)=f'(0)=3$ suy ra $ C=9$.\\
		Tương tự vì $\left[f^2(x)\right]'=2f(x){f}'(x)$ nên $\left[f^2(x)\right]'=2\left(\dfrac{2}{3}{x^3}-\dfrac{x^2}{2}+x+9\right)$.\\
		Suy ra $f^2(x)=\displaystyle\int{2\left(\dfrac{2}{3}{x^3}-\dfrac{x^2}{2}+x+9\right)\mathrm{\,d}x}=\dfrac{1}{3}{x^4}-\dfrac{x^3}{3}+x^2+18x+C$.\\
		Vì $ f(0)=3$ nên $C=9$ suy ra $f^2(x)=\dfrac{1}{3}{x^4}-\dfrac{x^3}{3}+x^2+18x+9$.\\
		Do đó $\left[f(1)\right]^2=28$.
	}
\end{ex}

\Closesolutionfile{ans}
% \indapan{10}{ans/ans-LC-2-C4B1CD3_1-8}
% \TNSA
\Opensolutionfile{ans}[ans/ans-KQ-2-C4B1CD3]
\begin{ex}%[2D4H1-2]
	Cho hàm số $ y=f(x)$ thỏa mãn $y'=x{y^2}$ và $ f\left(-1\right)=1$. Tính giá trị $f(2)$. (\textit{Kết quả làm tròn đến hàng phần mười}).
	\shortans{$20{,}1$}
	\loigiai{
	Ta có $y'=x{y^2} \Rightarrow\dfrac{y'}{y}=x^2\Rightarrow\displaystyle\int{\dfrac{y'}{y}\mathrm{\,d}x}=\displaystyle\int{x^2\mathrm{\,d}x}\Leftrightarrow\ln y=\dfrac{x^3}{3}+C\Leftrightarrow y=\mathrm{e}^{\tfrac{x^3}{3}+C}$.\\
	Theo giả thiết $ f(-1)=1$ nên $\mathrm{e}^{-\tfrac{1}{3}+C}=1\Leftrightarrow C=\dfrac{1}{3}$.\\
	Do đó	 $ y=f(x)= \mathrm{e}^{\tfrac{x^3}{3}+\tfrac{1}{3}}$.\\
	Vậy $f(2)=\mathrm{e}^3\approx 20{,}1$.}
\end{ex}

\begin{ex}%[2D4V2-2]
	Cho hàm số $ f(x)\ne 0$, liên tục trên đoạn $\left[1;2\right]$ và thỏa mãn $ f(1)=3$, \break $x^2\cdot f'(x)=f^2(x)$ với $\forall x\in\left[1;2\right]$. Tính $f(2)$.
	\shortans{$-6$}
	\loigiai{
		Ta có
		\begin{align*}
			x^2\cdot f'(x)=f^2(x) & \Rightarrow\dfrac{f'(x)}{f^2(x)}=\dfrac{1}{x^2} \Rightarrow{\left(-\dfrac{1}{f(x)}\right)'}=\dfrac{1}{x^2} \\& \Rightarrow\displaystyle\int\limits_1^2\left(-\dfrac{1}{f(x)}\right)'\mathrm{\,d} x=\displaystyle\int\limits_1^2\dfrac{1}{x^2}\mathrm{\,d} x\\&\Rightarrow\left.\left(-\dfrac{1}{f(x)}\right)\right|_1^2=-\left.\dfrac{1}{x}\right|_1^2\\& \Rightarrow-\dfrac{1}{f(2)}+\dfrac{1}{f(1)}=-\dfrac{1}{2}+1\\& \Rightarrow-\dfrac{1}{f(2)}+\dfrac{1}{f(1)}=\dfrac{1}{2}.
		\end{align*}
		Vì $f(1)=3\Rightarrow-\dfrac{1}{f(2)}+\dfrac{1}{3}=\dfrac{1}{2}\Rightarrow f(2)=-6$.}
\end{ex}
\begin{ex}%Câu 7%[2D4C1-2]
	Cho hàm số $ y=f(x)$ thỏa mãn $ f(x)<0$, $\forall x>0$ và có đạo hàm $f'(x)$ liên tục trên khoảng $\left( 0;+\infty\right) $ thỏa mãn $f'(x)=(2x+1){f^2}(x)$, $\forall x>0$ và $ f(1)=-\dfrac{1}{2}$. Tính giá trị của biểu thức $ T=f(1)+f(2)+\ldots +f\left(2023\right)+f\left(2024\right)$. (\textit{Kết quả làm tròn đến hàng đơn vị}).
	\shortans{$-1$}
	\loigiai{
		Ta có
		\begin{align*}
			f'(x)=(2x+1){f^2}(x)
			 & \Rightarrow \dfrac{f'(x)}{f^2(x)}=2x+1                                                              \\
			 & \Rightarrow\displaystyle\int\dfrac{f'(x)}{f^2(x)}\mathrm{\,d}x=\displaystyle\int(2x+1)\mathrm{\,d}x \\&\Rightarrow-\dfrac{1}{f(x)}=x^2+x+C.
		\end{align*}
		Mà $ f(1)=-\dfrac{1}{2}$ $\Rightarrow C=0$ $\Rightarrow f(x)=\dfrac{-1}{x^2+x}$ $=\dfrac{1}{x+1}-\dfrac{1}{x}$.\\
		Ta có $\heva{
				& f(1)=\dfrac{1}{2}-1\\
				& f(2)=\dfrac{1}{3}-\dfrac{1}{2}\\
				& f(3)=\dfrac{1}{4}-\dfrac{1}{3}\\
				&\ldots\\
				& f\left(2024\right)=\dfrac{1}{2023}-\dfrac{1}{2024}.}$\\
		$
			\Rightarrow T=f(1)+f(2)+\ldots+f\left(2024\right)=-1+\dfrac{1}{2025}=-\dfrac{2024}{2025}\approx -1$.}
\end{ex}


\begin{ex}%[2D4V1-4]
	Cho hàm số $f(x)$ thỏa mãn $ f(0)=1-\ln 2$ và $\mathrm{e}^ xf'(x)=2^x\left[f(x)\right]^2$ với mọi $x\in\mathbb{R}$. Giá trị của $f(1)$ bằng bao nhiêu? (\textit{Kết quả làm tròn đến hàng phần trăm}).
	\shortans{$0{,}42$}
	\loigiai{
		Từ giả thiết ta có $f'(x)=\dfrac{2^x}{\mathrm{e}^ x}{\left[f(x)\right]^2}$ với mọi $ x\in\left(1;2\right]$.\\
		Do đó $ f(x)\ge f(1)=1>0$ với mọi $ x\in\left[1;2\right]$.\\
		Xét với mọi $ x\in [1 ; 2]$ ta có
		\begin{align*}
			\mathrm{e}^ x{f}'(x)=2^x{\left[f(x)\right]^2} & \Rightarrow{f}'(x)=\dfrac{2^x}{\mathrm{e}^ x}{\left[f(x)\right]^2} \\&\Rightarrow\dfrac{f'(x)}{\left[f(x)\right]^2}=\left(\dfrac{2}{\mathrm{e}}\right)^x\\&\Rightarrow-\left(\dfrac{1}{f(x)}\right)'=\left(\dfrac{2}{\mathrm{e}}\right)^x \\&\Rightarrow{\left(\dfrac{1}{f(x)}\right)'}=-\left(\dfrac{2}{\mathrm{e}}\right)^x\\&\Rightarrow\dfrac{1}{f(x)}=-\displaystyle\int\left(\dfrac{2}{\mathrm{e}}\right)^x\mathrm{\,d} x\\&\Rightarrow\dfrac{1}{f(x)}=-\dfrac{\left(\dfrac{2}{\mathrm{e}}\right)^x}{\ln \dfrac{2}{\mathrm{e}}}+C\\&\Rightarrow\dfrac{1}{f(x)}=\dfrac{\left(\dfrac{2}{\mathrm{e}}\right)^x}{1-\ln 2}+C.
		\end{align*}
		Mà $ f(0)=1-\ln 2\Rightarrow C=0$. \\Do đó
		$\dfrac{1}{f(x)}=\dfrac{\left(\dfrac{2}{\mathrm{e}}\right)^x}{1-\ln 2}$
		$\Rightarrow f(x)=\dfrac{1-\ln 2}{\left(\dfrac{2}{\mathrm{e}}\right)^x}=\dfrac{(1-\ln 2)\mathrm{e}^x}{2^x}$.\\
		Vậy $ f(1)=\dfrac{\mathrm{e}-\mathrm{e}\ln 2}{2}\approx 0{,}42$.}
\end{ex}
\begin{ex}%[2D4V1-4]
	Cho hàm số $ y=f(x)$ đồng biến và có đạo hàm liên tục trên $\mathbb{R}$ thỏa mãn $\left(f'(x)\right)^2=f(x)\cdot\mathrm{e}^x$, $\forall x\in\mathbb{R}$ và $f(0)=2$. Tính $ f(2)$. (Kết quả làm tròn đến hàng phần trăm).
	\shortans{$9{,}81$}
	\loigiai{
	Vì hàm số $ y=f(x)$ đồng biến và có đạo hàm liên tục trên $\mathbb{R}$ đồng thời $ f(0)=2$ nên $f'(x)\ge 0$ và $ f(x)>0$ với mọi $ x\in\left[0;+\infty\right)$.\\
	Từ giả thiết $\left(f'(x)\right)^2=f(x)\cdot \mathrm{e}^x$, $\forall x\in\mathbb{R}$ suy ra $f'(x)=\sqrt{f(x)}\cdot\mathrm{e}^{\tfrac{x}{2}}$, $\forall x\in\left[0;+\infty\right).$\\
	Do đó $\dfrac{f'(x)}{2\sqrt{f(x)}}=\dfrac{1}{2}{\mathrm{e}^{\tfrac{x}{2}}}$, $\forall x\in\left[0;+\infty\right).$\\
	Lấy nguyên hàm hai vế, ta được $\sqrt{f(x)}=e^{\tfrac{x}{2}}+C$, $\forall x\in\left[0;+\infty\right)$ với $C$ là hằng số.\\
	Kết hợp với $ f(0)=2$, ta được $C=\sqrt{2}-1$.\\
	Suy ra $ f(2)=\left(\mathrm{e}+\sqrt{2}-1\right)^2\approx 9{,}81$.}
\end{ex}
\begin{ex}%[2D4H1-2]
	Giả sử hàm số $ y=f(x)$ liên tục, nhận giá trị dương trên $\left(0;+\infty\right)$ và thỏa mãn $ f(1)=1$, $ f(x)=f'(x)\cdot \sqrt{3x}$, với mọi $x>0$. Tính $f(5)$ \textit{(kết quả làm tròn đến hàng phần trăm}).
	\shortans{$4{,}17$}
	\loigiai{
	Ta có
	\begin{align*}
		f(x)=f'(x)\cdot\sqrt{3x} & \Rightarrow\dfrac{f'(x)}{f(x)}=\dfrac{1}{\sqrt{3x}}             \\&
		\Rightarrow\ln f(x)=\dfrac{1}{\sqrt{3}}\displaystyle\int \dfrac{1}{\sqrt{x}}\mathrm{\,d} x \\ &\Rightarrow\ln f(x)=\dfrac{2}{\sqrt{3}}\sqrt{x}+C\\&\Rightarrow f(x)=e^{\tfrac{2}{\sqrt{3}}\sqrt{x}+C}.
	\end{align*}
	Mà $ f(1)=1$ nên $1=e^{\tfrac{2}{\sqrt{3}}+C}\Rightarrow C=-\dfrac{2}{\sqrt{3}}$
	$\Rightarrow f(x)=e^{\tfrac{2}{\sqrt{3}}\sqrt{x}-\tfrac{2}{\sqrt{3}}}$.\\
	Suy ra $ f(5)=e^{\tfrac{2}{\sqrt{3}}\sqrt{5}-\tfrac{2}{\sqrt{3}}}=e^{\tfrac{2\sqrt{5}-2}{\sqrt{3}}}\approx 4{,}17$.}
\end{ex}

\begin{ex}%[2D4V1-2]
	Cho hàm số $ f(x)$ có đạo hàm trên $\mathbb{R}$ thỏa mãn $\mathrm{e}^{f(x)}-\dfrac{x}{f'(x)}=0$, $\forall x\in\mathbb{R}$. Biết $f(1)=1$, tính $f\left(\mathrm{e}^2\right)$ (\textit{kết quả làm tròn đến hàng phần trăm}).
	\shortans{$ 3{,}38$}
	\loigiai{
	Ta có
	\begin{align*}
		\mathrm{e}^{f(x)}-\dfrac{x}{f'(x)}=0
		 & \Rightarrow f'(x)\mathrm{e}^{f(x)}=x                                \\
		 & \Leftrightarrow \left(\mathrm{e}^{f(x)} \right)'=x                  \\
		 & \Leftrightarrow \mathrm{e}^{f(x)}=\displaystyle\int x\mathrm{\,d} x \\
		 & \Leftrightarrow \mathrm{e}^{f(x)}=\dfrac{x^2}{2}+C.
	\end{align*}
	Mà 	$f(1)=1$ nên $\mathrm{e}=\dfrac{1}{2}+C\Rightarrow C=\mathrm{e}-\dfrac{1}{2}$.\\
	Do đó $\mathrm{e}^{f(x)}=\dfrac{x^2}{2}+ \mathrm{e}-\dfrac{1}{2} \Rightarrow \mathrm{e}^{f\left(\mathrm{e}^2\right)}= \dfrac{\mathrm{e}^4}{2}+ \mathrm{e}-\dfrac{1}{2}\Rightarrow f\left(\mathrm{e}^2\right)=\ln \left(\dfrac{\mathrm{e}^4}{2}+ \mathrm{e}-\dfrac{1}{2}\right) \approx 3{,}38$.}
\end{ex}

\begin{ex}%[2D4V1-4]
	Cho hàm số $ f(x)$ nhận giá trị dương và thỏa mãn $ f(0)=1$, $\left(f'(x)\right)^3=\mathrm{\mathrm{e}}^ x{\left(f(x)\right)^2}$, $\forall x\in\mathbb{R}$. Tính $ f(3)$ (\textit{kết quả làm tròn đến hàng phần mười}).
	\shortans{$20{,}1$}
	\loigiai{
	Ta có

	\begin{align*}
		\left(f'(x)\right)^3=\mathrm{e}^x{\left(f(x)\right)^2},\,\forall x\in\mathbb{R}
		 & \Leftrightarrow{f}'(x)=\sqrt[3]{\mathrm{e}^x}\cdot \sqrt[3]{\left(f(x)\right)^2}\Leftrightarrow\dfrac{f'(x)}{\sqrt[3]{\left(f(x)\right)^2}}=\sqrt[3]{\mathrm{e}^x}     \\
		 & \Leftrightarrow\dfrac{f'(x)}{\sqrt[3]{\left(f(x)\right)^2}}=\sqrt[3]{\mathrm{e}^x}\Leftrightarrow{f}'(x)\cdot \left(f(x)\right)^{-\tfrac{2}{3}}=\sqrt[3]{\mathrm{e}^x} \\&\Leftrightarrow 3\left[\left(f(x)\right)^{\tfrac{1}{3}}\right]'=\sqrt[3]{\mathrm{e}^x}\Leftrightarrow{\left[\left(f(x)\right)^{\tfrac{1}{3}}\right]'}=\dfrac{1}{3}\sqrt[3]{\mathrm{e}^x}\\&\Leftrightarrow{\left(f(x)\right)^{\tfrac{1}{3}}}=\dfrac{1}{3}\displaystyle\int{\sqrt[3]{\mathrm{e}^x}}\mathrm{\,d} x \Leftrightarrow{\left(f(x)\right)^{\tfrac{1}{3}}}=e^{\tfrac{x}{3}}+C.
	\end{align*}
	Vì	$f(0)=1$ nên $1=1+C\Rightarrow C=0\Rightarrow{\left(f(x)\right)^{\tfrac{1}{3}}}=e^{\tfrac{x}{3}}\Rightarrow f(x)=\mathrm{e}^x$.\\
	Vậy	$f(3)=e^3\approx 20{,}1$.
	}
\end{ex}

\begin{ex}%Câu 13%[2D4V1-2]
	Cho hàm số $ y=f(x)$ có đạo hàm liên tục trên $\mathbb{R}$ và thỏa mãn điều kiện $x^6\left( f'(x)\right) ^3+27\left[f(x)-1\right]^4=0$, $\forall x\in\mathbb{R}$ và $ f(1)=0$. Tính giá trị của $f(2)$.
	\shortans{$-7$}
	\loigiai{
		Ta có
		\begin{align*}
			x^6\left( f'(x)\right)^3+27\left[f(x)-1\right]^4=0
			 & \Leftrightarrow{x^6}{\left( f'(x)\right)^3}=-27\left( f(x)-1\right)^4                                       \\&\Leftrightarrow\dfrac{\left( f'(x)\right) ^3}{\left( f(x)-1\right)^4}=-\dfrac{27}{x^6}\\
			 & \Leftrightarrow\dfrac{\left( f'(x)\right) ^3}{\left( f(x)-1\right) ^3\left( f(x)-1\right)}=-\dfrac{27}{x^6} \\&\Leftrightarrow\dfrac{f'(x)}{\left(f(x)-1\right)\sqrt[3]{f(x)-1}}=-\dfrac{3}{x^2}\\&\Leftrightarrow\dfrac{f'(x)}{-3\left(f(x)-1\right)\sqrt[3]{f(x)-1}}=\dfrac{1}{x^2}\\&\Leftrightarrow{\left[\dfrac{1}{\sqrt[3]{f(x)-1}}\right]'}=\dfrac{1}{x^2}
		\end{align*}
		Do đó $\displaystyle\int{\left( \dfrac{1}{\sqrt[3]{f(x)-1}}\right)'}\mathrm{\,d}x=\displaystyle\int{\dfrac{1}{x^2}\mathrm{\,d}x}=-\dfrac{1}{x}+C.$
		\\
		Suy ra $\dfrac{1}{\sqrt[3]{f(x)-1}}=-\dfrac{1}{x}+C$.\\
		Ta có $ f(1)=0\Rightarrow C=0 \Rightarrow f(x)=1-x^3$.\\
		Khi đó $ f(2)=-7$.}
\end{ex}
\begin{ex}%[2D4V1-2]
	Cho hàm số $f(x)$ thỏa mãn $\left[x{f}'(x)\right]^2+1=x^2\left[1-f(x).f''(x)\right]$ với mọi $x$ dương. Biết $f(1)=f'(1)=1$. Tính giá trị $f^2(2)$ (\textit{kết quả làm tròn đến hàng phần trăm}).
	\shortans{$3{,}39$}
	\loigiai{
		Với mọi $x$ dương, ta có
		\begin{align*}
			\left[x{f}'(x)\right]^2+1=x^2\left[1-f(x)\cdot f''(x)\right]; x>0 & \Leftrightarrow{x^2}\cdot\left[f'(x)\right]^2+1=x^2\left[1-f(x)\cdot f''(x)\right] \\
			                                                                  & \Leftrightarrow{\left[f'(x)\right]^2}+\dfrac{1}{x^2}=1-f(x)\cdot f''(x)            \\
			                                                                  & \Leftrightarrow{\left[f'(x)\right]^2}+f(x)\cdot f''(x)=1-\dfrac{1}{x^2}            \\
			                                                                  & \Leftrightarrow\left[f(x)\cdot f'(x)\right]'=1-\dfrac{1}{x^2}.
		\end{align*}
		Do đó $\displaystyle\int\left[f(x)\cdot f'(x)\right]'\mathrm{\, d}x=\displaystyle\int\left(1-\dfrac{1}{x^2}\right)\mathrm{\, d}x\Rightarrow f(x)\cdot f'(x)=x+\dfrac{1}{x}+C.$\\
		Vì $ f(1)=f'(1)=1\Rightarrow 1=2+C\Leftrightarrow C=-1.$\\
		Nên $\displaystyle\int f(x)\cdot f'(x)\mathrm{\, d}x=\displaystyle\int\left(x+\dfrac{1}{x}-1\right) \mathrm{\, d}x$ $\Leftrightarrow\displaystyle\int f(x)\mathrm{\, d}\left(f(x)\right)=\displaystyle\int{\left(x+\dfrac{1}{x}-1\right)}\mathrm{\, d}x$.\\
		Suy ra				$\dfrac{f^2(x)}{2}=\dfrac{x^2}{2}+\ln x-x+C.$\\
		Vì $ f(1)=1\Rightarrow\dfrac{1}{2}=\dfrac{1}{2}-1+C\Leftrightarrow C=1.$\\
		Vậy $\dfrac{f^2(x)}{2}=\dfrac{x^2}{2}+\ln x-x+1\Rightarrow{f^2}(2)=2\ln 2+2\approx 3{,}39$.
	}
\end{ex}
\Closesolutionfile{ans}
% \indapan{6}{ans/ans-KQ-2-C4B1CD3}
\begin{dang}{~}
	\subsubsection{Điều kiện hàm ẩn có dạng} $$A(x)f(x)+B(x)f'(x)=h(x)\quad (1)$$
	\textbf{Phương pháp giải}
	\begin{itemize}
		\item Ta cần nhân thêm một lượng $u(x)$ vào  $(1)$ để tạo thành \break $u'(x) f(x)+u(x) f'(x)=u(x) \cdot h(x)$ và lúc này.
		      \begin{align*}
			                      & \, u'(x) f(x)+u(x) f'(x)=u(x) \cdot h(x)                                                        
			      \Leftrightarrow \,\left[u(x) f(x)\right]'=u(x) \cdot h(x)                                                       \\
			      \Rightarrow     & \, \int\left[u(x) f(x)\right] \mathrm{\,d} x=\int u(x) \cdot h(x) d x 
			      \Rightarrow   \, u(x) f(x)=\int u(x) \cdot h(x) \mathrm{\,d} x                                   \\
			      \Rightarrow     & \, f(x)=\dfrac{\int u(x) \cdot h(x) \mathrm{\,d} x}{u(x)}
		      \end{align*}
		\item Cách tìm $u(x)$.\\
		      $u(x)$ được chọn sao cho  $\heva{&u'(x)=A(x) \\ &u(x)=B(x).}$\\
		      Suy ra
		      \begin{align*}
			                  & \,\dfrac{u'(x)}{u(x)}=\dfrac{A(x)}{B(x)}                                                                    
			      \Rightarrow \, \int \dfrac{u'(x)}{u(x)} \mathrm{\,d} x=\int \dfrac{A(x)}{B(x)} \mathrm{\,d} x \\
			      \Rightarrow & \, \ln |u(x)|=\int \dfrac{A(x)}{B(x)} \mathrm{\,d} x                                           
			      \Rightarrow \, u(x)=\mathrm{e}^{\int\tfrac{A(x)}{B(x)} \mathrm{\,d} x}
		      \end{align*}
	\end{itemize}
	Tóm lại phương pháp giải $A(x)f(x)+B(x)f'(x)=h(x)$\quad $(1)$ như sau.
	\begin{itemize}
		\item \textbf{Bước 1.} Tìm $u(x)$. $u(x)=\mathrm{e}^{\int\tfrac{A(x)}{B(x)} \mathrm{\,d} x}$.
		\item \textbf{Bước 2.} Nhân $u(x)$ vào $(1)$ suy ra $f(x)=\dfrac{\int\limits u(x) \cdot h(x) \mathrm{\,d}x}{u(x)}$.
	\end{itemize}
	\subsubsection*{Một số dạng đặc biệt của $(1)$.}
	\begin{enumerate}
		\item Điều kiện hàm ẩn có dạng $\hoac{&f'(x)+f(x)=h(x)\\ &f'(x)-f(x)=h(x).}$\\
		      Phương pháp giải.
		      \begin{itemize}
			      \item $f'(x)+f(x)=h(x)$.\\
			            Nhân hai vế với $\mathrm{e}^x$ ta được $$\mathrm{e}^x \cdot f'(x)+\mathrm{e}^x \cdot f(x)=\mathrm{e}^x \cdot h(x) \Leftrightarrow\left[\mathrm{e}^x \cdot f(x)\right]'=\mathrm{e}^x \cdot h(x).$$
			            Suy ra $\mathrm{e}^x \cdot f(x)=\int \mathrm{e}^x \cdot h(x)  \mathrm{\,d} x$.\\
			            Từ đây ta dễ dàng tính được $f(x)$.
			      \item $f'(x)-f(x)=h(x)$.\\
			            Nhân hai vế với $\mathrm{e}^{-x}$ ta được $$\mathrm{e}^{-x} \cdot f'(x)-\mathrm{e}^{-x} \cdot f(x)=\mathrm{e}^{-x} \cdot h(x) \Leftrightarrow\left[\mathrm{e}^{-x} \cdot f(x)\right]'=\mathrm{e}^{-x} \cdot h(x).$$
			            Suy ra $\mathrm{e}^{-x} \cdot f(x)=\int \mathrm{e}^{-x} \cdot h(x)  \mathrm{\,d} x$.\\
			            Từ đây ta dễ dàng tính được $f(x)$.
		      \end{itemize}
		\item Điều kiện hàm ẩn có dạng $f'(x)+p(x)\cdot f(x)=h(x)$.\\
		      \textbf{Phương pháp giải.}\\
		      Nhân hai vế với $\mathrm{e}^{\int\limits p(x) \mathrm{\,d} x}$ ta được
		      \begin{align*}
			                      & \,f'(x) \cdot \mathrm{e}^{\int\limits p(x) \mathrm{\,d} x}+p(x) \cdot \mathrm{e}^{\int\limits p(x) \mathrm{\,d} x} \cdot f(x)=h(x) \cdot \mathrm{e}^{\int\limits p(x) \mathrm{\,d}x} \\
			      \Leftrightarrow & \, \left[f(x) \cdot \mathrm{e}^{\int\limits p(x) \mathrm{\,d} x}\right]'=h(x) \cdot \mathrm{e}^{\int\limits p(x) \mathrm{\,d} x}.
		      \end{align*}
		      Suy ra $f(x) \cdot e^{\int p(x)\mathrm{\,d} x}=\int \mathrm{e}^{\int\limits p(x) \mathrm{e} x} h(x)  \mathrm{\,d} x$.\\
		      Từ đây ta dễ dàng tính được $f(x)$.
	\end{enumerate}
\end{dang}
% \TN
\Opensolutionfile{ans}[ans/ans-LC-2-C4B1CD3.1]
\begin{ex}%[2D4V1-4]
	Cho hàm số $f(x)$ thỏa mãn $f(x)+f'(x)= \mathrm{e}^{-x}$, $\forall x \in \mathbb{R}$ và $f(0)=2$. Tất cả các nguyên hàm của $f(x)\mathrm{e}^x$ là
	\choice
	{$x^2+x+C$}
	{$2 x^2+2 x+C$}
	{$2 x^2+x+C$}
	{\True $\dfrac{1}{2} x^2+2 x+C$}
	\loigiai{
		Ta có \begin{align*}
			f(x)+f'(x)= \mathrm{e}^{-x}
			 & \, \Leftrightarrow f(x)  \mathrm{e}^x+f'(x)  \mathrm{e}^x=1          \\
			 & \, \Leftrightarrow\left(f(x)  \mathrm{e}^x\right)'=1                 \\
			 & \, \Rightarrow f(x)  \mathrm{e}^x=\displaystyle\int x \mathrm{\,d} x \\
			 & \, \Leftrightarrow f(x)  \mathrm{e}^x=x+C.
		\end{align*}
		Vì $f(0)=2$ nên $ C=2$.\\
		Suy ra $f(x)  \mathrm{e}^x=x+2
			\Rightarrow \displaystyle\int f(x)  \mathrm{e}^x d x=\displaystyle\int(x+2) \mathrm{\,d} x=\dfrac{1}{2} x^2+2 x+C$.
	}
\end{ex}

\begin{ex}%[2D4V1-4]
	Cho hàm số $y=f(x)$ liên tục trên $\mathbb{R}$ thỏa mãn $f'(x)+2 x \cdot f(x)= \mathrm{e}^{-x^2}$, $\forall x \in \mathbb{R}$ và $f(0)=0$. Tính $f(1)$.
	\choice
	{$f(1)= \mathrm{e}^2$}
	{$f(1)=-\dfrac{1}{ \mathrm{e}}$}
	{$f(1)=\dfrac{1}{ \mathrm{e}^2}$}
	{\True $f(1)=\dfrac{1}{ \mathrm{e}}$}
	\loigiai{
		Ta có
		\begin{align*}
			                & \, f'(x)+2 x \cdot f(x)= \mathrm{e}^{-x^2}                                                                   \\
			\Leftrightarrow & \,  \mathrm{e}^{x^2} f'(x)+2 x \cdot  \mathrm{e}^{x^2} \cdot f(x)=1                                          \\
			\Leftrightarrow & \,\left( \mathrm{e}^{x^2} \cdot f(x)\right)'=1                                                               \\
			\Rightarrow     & \,\displaystyle\int\left(\mathrm{e}^{x^2} \cdot f(x)\right)' \mathrm{\,d} x=\displaystyle\int \mathrm{\,d} x \\
			\Rightarrow     & \,  \mathrm{e}^{x^2} \cdot f(x)=x+C                                                                          \\
			\Rightarrow     & \, f(x)=\dfrac{x+C}{\mathrm{e}^{x^2}}.
		\end{align*}
		Vì $f(0)=0 \Rightarrow C=0$.\\
		Do đó $f(x)=\dfrac{x}{ \mathrm{e}^{x^2}}$.\\
		Vậy $f(1)=\dfrac{1}{ \mathrm{e}}$.
	}
\end{ex}

\begin{ex}%[2D4V1-2]
	Cho hàm số $y=f(x)$ liên tục trên $\mathbb{R} \setminus \{-1 ; 0\}$ thỏa mãn điều kiện $f(1)=-2 \ln 2$ và $x \cdot(x+1) \cdot f'(x)+f(x)=x^2+x$. Biết $f(2)=a+b \cdot \ln 3$  ($a$, $b \in \mathbb{Q}$). Giá trị $2\left(a^2+b^2\right)$ là
	\choice
	{$\dfrac{27}{4}$}
	{\True  $9$}
	{$\dfrac{3}{4}$}
	{$\dfrac{9}{2}$}
	\loigiai{
		Chia cả hai vế của biểu thức $x \cdot(x+1) \cdot f'(x)+f(x)=x^2+x$ cho $(x+1)^2$ ta có
		$$ \dfrac{x}{x+1} \cdot f'(x)+\dfrac{1}{(x+1)^2} f(x)=\dfrac{x}{x+1} \\
			\Leftrightarrow\left[\dfrac{x}{x+1} \cdot f(x)\right]'=\dfrac{x}{x+1}.$$
		Do đó $$\dfrac{x}{x+1} \cdot f(x)=\displaystyle\int\limits\left[\dfrac{x}{x+1} \cdot f(x)\right]'  \mathrm{\,d} x=\displaystyle\int\limits \dfrac{x}{x+1} \mathrm{\,d} x=\displaystyle\int\limits\left(1-\dfrac{1}{x+1}\right)  \mathrm{\,d} x=x-\ln |x+1|+C.$$
		Do $f(1)=-2 \ln 2$ nên ta có $\dfrac{1}{2} \cdot f(1)=1-\ln 2+C \Leftrightarrow-\ln 2=1-\ln 2+C \Leftrightarrow C=-1$.\\
		Khi đó $f(x)=\dfrac{x+1}{x}(x-\ln |x+1|-1)$.\\
		Vậy ta có $f(2)=\dfrac{3}{2}(2-\ln 3-1)=\dfrac{3}{2}(1-\ln 3)=\dfrac{3}{2}-\dfrac{3}{2} \ln 3 \Rightarrow a=\dfrac{3}{2}$, $b=-\dfrac{3}{2}$.\\
		Suy ra $2\left(a^2+b^2\right)=2\left[\left(\dfrac{3}{2}\right)^2+\left(-\dfrac{3}{2}\right)^2\right]=9$.
	}
\end{ex}

\begin{ex}%[2D4V1-2]
	Cho hàm số $y=f(x)$ liên tục trên $\mathbb{R} \setminus \{-1 ; 0\}$ thỏa mãn $f(1)=2 \ln 2+1$, $x(x+1) f'(x)+(x+2) f(x)=x(x+1)$, $\forall x \in \mathbb{R} \backslash\{-1 ; 0\}$. Biết $f(2)=a+b \ln 3$, với $a$, $b$ là hai số hữu tỉ. Tính $T=a^2-b$.
	\choice
	{\True $T=-\dfrac{3}{16}$}
	{$T=\dfrac{21}{16}$}
	{$T=\dfrac{3}{2}$}
	{$T=0$}
	\loigiai{
		Ta có \begin{align*}
			x(x+1) f'(x)+(x+2) f(x)=x(x+1) & \Leftrightarrow f'(x)+\dfrac{x+2}{x(x+1)} f(x)=1                                     \\
			                               & \Leftrightarrow \dfrac{x^2}{x+1} f'(x)+\dfrac{x(x+2)}{(x+1)^2} f(x)=\dfrac{x^2}{x+1} \\
			                               & \Leftrightarrow\left[\dfrac{x^2}{x+1} f(x)\right]'=\dfrac{x^2}{x+1}                  \\
			                               & \Leftrightarrow \dfrac{x^2}{x+1} f(x)=\displaystyle\int\limits \dfrac{x^2}{x+1} d x  \\
			                               & \Leftrightarrow \dfrac{x^2}{x+1} f(x)=\dfrac{x^2}{2}-x+\ln |x+1|+c                   \\
			                               & \Leftrightarrow f(x)=\dfrac{x+1}{x^2}\left(\dfrac{x^2}{2}-x+\ln |x+1|+c\right).
		\end{align*}
		Từ $f(1)=2 \ln 2+1 \Leftrightarrow c=1$.\\
		Từ đó $f(x)=\dfrac{x+1}{x^2}\left(\dfrac{x^2}{2}-x+\ln |x+1|+1\right)$.\\
		$\Rightarrow f(2)=\dfrac{3}{4}+\dfrac{3}{4} \ln 3$.\\
		Nên $\heva{&a=\dfrac{3}{4} \\ &b=\dfrac{3}{4}.}$\\
		Vậy $T=a^2-b=-\dfrac{3}{16}$.
	}
\end{ex}

\begin{ex}%[2D4V1-2]
	Cho hàm số $y=f(x)$ có đạo hàm liên tục trên $(0 ;+\infty)$ thỏa mãn \break $f'(x)+\dfrac{f(x)}{x}=4 x^2+3 x$ và $f(1)=2$. Phương trình tiếp tuyến của đồ thị hàm số $y=f(x)$ tại điểm có hoành độ $x=2$ là
	\choice
	{$y=-16 x-20$}
	{\True $y=16 x-20$}
	{$y=16 x+20$}
	{$y=-16 x+20$}
	\loigiai{
		$$
			f'(x)+\dfrac{f(x)}{x}=4 x^2+3 x \Leftrightarrow x f'(x)+f(x)=4 x^3+3 x^2 \Leftrightarrow \left(x.f(x)\right)'=4x^3+3x^2.
		$$
		Lấy nguyên hàm hai vế ta được $x f(x)=\displaystyle\int\limits\left(4 x^3+3 x^2\right)  \mathrm{\,d} x=x^4+x^3+C$.\\
		Với $x=1$ ta có $f(1)=2+C$.\\
		Theo đề bài ta có: $f(1)=2 \Leftrightarrow 2+C=2 \Leftrightarrow C=0$.\\
		Vậy $x f(x)=x^4+x^3 \Leftrightarrow f(x)=x^3+x^2$.\\
		Ta có $f'(x)=3 x^2+2 x$, $f'(2)=16$, $ f(2)=12$.\\
		Phương trình tiếp tuyến của đồ thị hàm số $y=f(x)$ tại điểm có hoành độ $x=2$ là
		$$
			y=16(x-2)+12 \Leftrightarrow y=16 x-20.
		$$
	}
\end{ex}

\begin{ex}%[2D4V1-2]
	Cho hàm số $y=f(x)$ liên tục trên $(0 ;+\infty)$ thỏa mãn $2 x f'(x)+f(x)=3 x^2 \sqrt{x}$. Biết $f(1)=\dfrac{1}{2}$. Tính $f(4)$.
	\choice
	{$24$}
	{$14$}
	{$4$}
	{\True  $16$}
	\loigiai{
		Trên khoảng $(0 ;+\infty)$ ta có
		\begin{align*}
			2 x f'(x)+f(x)=3 x^2 \sqrt{x}
			 & \Leftrightarrow \sqrt{x} f'(x)+\dfrac{1}{2 \sqrt{x}}.f(x)=\dfrac{3}{2} x^2                                   \\
			 & \Rightarrow(\sqrt{x} \cdot f(x))'=\dfrac{3}{2} x^2                                                           \\
			 & \Rightarrow \displaystyle\int\limits(\sqrt{x} \cdot f(x))' d x=\displaystyle\int\limits \dfrac{3}{2} x^2 d x \\
			 & \Rightarrow \sqrt{x} \cdot f(x)=\dfrac{1}{2} x^3+C.\quad(\ast)
		\end{align*}
		Mà $f(1)=\dfrac{1}{2}$ nên từ $(\ast)$ có $$\sqrt{1} \cdot f(1)=\dfrac{1}{2} \cdot 1^3+C \Leftrightarrow \dfrac{1}{2}=\dfrac{1}{2}+C \Leftrightarrow C=0 \Rightarrow f(x)=\dfrac{x^2 \sqrt{x}}{2}.$$
		Vậy $f(4)=\dfrac{4^2\cdot  \sqrt{4}}{2}=16$.
	}
\end{ex}

\begin{ex}%[2D4V1-2]
	Cho hàm số $f(x)$ thỏa mãn $f(1)=4$ và $f(x)=x f'(x)-2 x^3-3 x^2$ với mọi $x>0$. Giá trị của $f(2)$ bằng
	\choice
	{$5$}
	{$10$}
	{\True $20$}
	{$15$}
	\loigiai{
		Ta có
		\begin{align*}
			f(x)-x f'(x)=-2 x^3-3 x^2
			 & \Leftrightarrow \dfrac{1 \cdot f(x)-x \cdot f'(x)}{x^2}=\dfrac{-2 x^3-3 x^2}{x^2} \\
			 & \Leftrightarrow\left[\dfrac{f(x)}{x}\right]'=2 x+3.
		\end{align*}
		Suy ra $\dfrac{f(x)}{x}$ là một nguyên hàm của hàm số $ {g}(x)=2 x+3$.\\
		Ta có $\displaystyle\int(2 x+3) \mathrm{\,d} x=x^2+3 x+C$, $C \in \mathbb{R}$.\\
		Do đó $\dfrac{f(x)}{x}=x^2+3 x+\mathrm{C}_1$\quad $(1)$ với $\mathrm{C}_1 \in \mathbb{R}$.\\
		Vì $f(1)=4$ theo giả thiết, nên thay $x=1$ vào hai vế của $(1)$ ta thu được $\mathrm{C}_1=0$, từ đó $f(x)=x^3+3 x^2$.\\ Vậy $f(2)=20$.
	}
\end{ex}

\begin{ex}%[2D4V1-2]
	Cho hàm số $y=f(x)$ liên tục trên $(0 ;+\infty)$ thỏa mãn \break $3 x \cdot f(x)-x^2 \cdot f'(x)=2 f^2(x)$, với $f(x) \neq 0$,  $\forall x \in(0 ;+\infty)$ và $f(1)=\dfrac{1}{3}$. Gọi $M$,  $m$ lần lượt là giá trị lớn nhất, giá trị nhỏ nhất của hàm số $y=f(x)$ trên đoạn $[1 ; 2]$. Tính $M+m$.
	\choice
	{$\dfrac{9}{10}$}
	{$\dfrac{21}{10}$}
	{\True  $\dfrac{5}{3}$}
	{$\dfrac{7}{3}$}
	\loigiai{
		Ta có
		\begin{align*}
			3x\cdot f(x)-x^2 \cdot f'(x)=2 f^2(x)
			 & \Rightarrow 3 x^2 \cdot f(x)-x^3 \cdot f'(x)=2 x \cdot f^2(x)                                             \\
			 & \Rightarrow \dfrac{3 x^2 \cdot f(x)-x^3 \cdot f'(x)}{f^2(x)}=2 x,  f(x) \neq 0, \forall x \in(0 ;+\infty) \\
			 & \Rightarrow\left(\dfrac{x^3}{f(x)}\right)'=2 x                                                            \\ &\Rightarrow \dfrac{x^3}{f(x)}=\displaystyle\int 2 x  \mathrm{\,d} x=x^2+C . \\
			 &
		\end{align*}
		Mà $f(1)=\dfrac{1}{3} \Rightarrow C=2 \Rightarrow f(x)=\dfrac{x^3}{x^2+2}$.\\
		Ta có $f(x)=\dfrac{x^3}{x^2+2} \Rightarrow f'(x)=\dfrac{x^4+6 x^2}{\left(x^2+2\right)^2}>0$, $\forall x \in(0 ;+\infty)$.\\
		Vậy, hàm số $f(x)=\dfrac{x^3}{x^2+2}$ đồng biến trên khoảng $(0 ;+\infty)$.\\
		Mà $[1 ; 2] \subset(0 ;+\infty)$ nên hàm số $f(x)=\dfrac{x^3}{x^2+2}$ đồng biến trên đoạn $[1 ; 2]$.\\
		Suy ra $M=f(2)=\dfrac{4}{3}$, $ m=f(1)=\dfrac{1}{3}$.\\
		Vậy $ M+m=\dfrac{5}{3}$.
	}
\end{ex}

\begin{ex}%[2D4V1-4]
	Cho $F(x)$ là một nguyên hàm của hàm số $f(x)=e^{x^2}\left(x^3-4 x\right)$. Hàm số $F\left(x^2+x\right)$ có bao nhiêu điểm cực trị?
	\choice
	{$6$}
	{\True $5$}
	{$3$}
	{$4$}
	\loigiai{
		Ta có $F'(x)=f(x)$. Khi đó
		\begin{align*}
			F'\left(x^2+x\right) & \,=f\left(x^2+x\right) \cdot\left(x^2+x\right)'                                                   \\
			                     & \,=(2 x+1)\left(x^2+x\right) \mathrm{e}^{\left(x^2+x\right)^2}\left[\left(x^2+x\right)^2-4\right] \\
			                     & \, =(2 x+1) x(x+1) \mathrm{e}^{\left(x^2+x\right)^2}\left(x^2+x-2\right)\left(x^2+x+2\right)      \\
			                     & \, =(2 x+1) x(x+1)(x+2)(x-1)\left(x^2+x+2\right) \mathrm{e}^{\left(x^2+x\right)^2}.
		\end{align*}
		$F'(x)=0\Leftrightarrow \hoac{&x=-2\\ &x=\dfrac{-1}{2}\\ &x=1\\ &x=-1\\& x=0.}$\\
		$F'\left(x^2+x\right)=0$ có $5$ nghiệm đơn nên $F\left(x^2+x\right)$ có $5$ điểm cực trị.
	}
\end{ex}

\begin{ex}%[2D4V1-2]
	\immini{Cho hàm số $y=f(x)$. Đồ thị của hàm số \break $y=f'(x)$ trên $[-5 ; 3]$ như hình vẽ (phần cong của đồ thị là một phần của parabol \break $y=a x^2+b x+c$). Biết $f(0)=0$, giá trị của $2 f(-5)+3 f(2)$ bằng
		\choice
		{$33$}
		{$\dfrac{109}{3}$}
		{\True $\dfrac{35}{3}$}
		{$11$}
	}{
		\begin{tikzpicture}[scale=0.7, font=\footnotesize, line join=round, line cap=round,>=stealth]
			%Gán số liệu.
			\def\xmin{-6};\def\ymin{-2};\def\xmax{4};\def\ymax{5};
			%Gán tọa độ.
			\coordinate (O) at (0,0);
			%Trục Oxy.
			\draw[->] (\xmin,0)--(\xmax,0) node[below]{$x$};
			\draw[->] (0,\ymin)--(0,\ymax) node[left]{$y$};
			\fill (O) node[below left]{$O$} circle(1pt);
			%Giới hạn đồ thị.
			\clip ({\xmin-0.1},{\ymin-0.1}) rectangle ({\xmax+0.1},{\ymax+0.1});
			\foreach \x in {-5,-4,-1,1,2,3}{
					\fill (\x,0) node[below]{$\x$} circle(1pt);
				}
			\foreach \y in {-1,2,3,4}{
					\fill (0,\y) node[left]{$\y$} circle(1pt);
				}
			\draw (-5,-1)--(-4,2)--(-1,0);
			\draw[thick,samples=100] plot[domain=-1:3.5](\x,{-(\x)^2+2*\x+3});
			\draw[dashed] (-5,0)|-(0,-1) (-4,0)|-(0,2) (1,0)|-(0,4) (2,0)|-(0,3);
		\end{tikzpicture}
	}
	\loigiai{
		Parabol $y=a x^2+b x+c$ qua các điểm $(2 ; 3)$, $(1 ; 4)$, $(0 ; 3)$, $(-1 ; 0)$, $(3 ; 0)$ nên xác định được $y=-x^2+2 x+3$, $\forall x \geq-1$ suy ra $f(x)=-\dfrac{x^3}{3}+x^2+3 x+C_1$.\\
		Mà $f(0)=0 \Rightarrow C_1=0$, $f(x)=-\dfrac{x^3}{3}+x^2+3 x$.\\
		Có $f(-1)=-\dfrac{5}{3}$, $ f(2)=\dfrac{22}{3}$.\quad $(1)$\\
		Đồ thị $f'(x)$ trên đoạn $[-4 ;-1]$ qua các điểm $(-4 ; 2)$, $(-1 ; 0)$.\\
		Nên $f'(x)=-\dfrac{2}{3}(x+1) \Rightarrow f(x)=-\dfrac{2}{3}\left(\dfrac{x^2}{2}+x\right)+C_2$.\\
		Mà $f(-1)=-\dfrac{5}{3} \Leftrightarrow C_2=-\dfrac{5}{3}+\dfrac{2}{3}\left(-\dfrac{1}{2}\right)=-2 \Rightarrow f(x)=-\dfrac{2}{3}\left(\dfrac{x^2}{2}+x\right)-2$, hay $f(-4)=-\dfrac{14}{3}$.\\
		Đồ thị $f'(x)$ trên đoạn $[-5 ;-4]$ qua các điểm $(-4 ; 2)$, $(-5 ;-1)$.\\
		Nên $f'(x)=3 x+14 \Rightarrow f(x)=\dfrac{3 x^2}{2}+14 x+C_3$.\\
		Mà $f(-4)=-\dfrac{14}{3} \Leftrightarrow \dfrac{3 \cdot(-4)^2}{2}+14 \cdot(-4)+C_3=-\dfrac{14}{3}$ suy ra $C_3=\dfrac{82}{3}$.\\
		Ta có $f(x)=\dfrac{3 x^2}{2}+14 x+\dfrac{82}{3} \Rightarrow f(-5)=-\dfrac{31}{6}$.\quad $(2)$\\
		Từ $(1)$ và $(2)$ ta được $2 f(-5)+3 f(2)=-\dfrac{31}{3}+22=\dfrac{35}{3}$.
	}
\end{ex}
\Closesolutionfile{ans}
% \indapan{10}{ans/ans-LC-2-C4B1CD3.1}
% %%Bài 2. Tích phân
% \setcounter{section}{1}
\section{Tích Phân}
\subsection{Lý thuyết cần nhớ}
\subsubsection{Diện tích hình thang cong}
\begin{center}
	\begin{tikzpicture}[>=stealth]
		%		\tkzInit[xmin=-0.5,ymin=-2.5,xmax=6.5,ymax=2.5] \tkzClip
		\draw[->] (-0.5,0)--(5.3,0) node[below] {$x$} ;
		\draw[->] (0,-.5)--(0,2.3) node[right] {$y$} ;
		\draw (0,0) node[below left] {$O$};
		%		\draw (1,0) ellipse (0.16 and 1);
		%		\draw (4,0) ellipse (0.25 and 1.73);
		%Nhánh trên
		\draw[domain=1:4] 
		plot(\x,{0.31*(\x)^3-2.28*(\x)^2+5.14*(\x)-2.17}) ;
		%Nhánh dưới
		%	\draw[domain=1:4]
		%	plot(\x,{-0.31*(\x)^3+2.28*(\x)^2-5.14*(\x)+2.17}) ;
		%Tô màu
		\draw[pattern = north east lines,opacity=.3, line width = 1.2pt,draw=none] (1,1) plot[domain=1:4] (\x,{0.31*(\x)^3-2.28*(\x)^2+5.14*(\x)-2.17})--(4,0)--(1,0)--cycle;
		
		%Các yếu tố khác
		\draw (1,0) node[below] {$a$};
		\draw (4,0) node[below] {$b$};
		\draw[dashed] (1,0)--(1,1);
		\draw[dashed] (4,0)--(4,1.72);
		\draw (2.5,1.7) node {$y=f(x)$} ;
		\draw (2.5,0.7) node {$S$} ;
		%	\draw[->] (5,0.25) arc (90:270:0.3);
	\end{tikzpicture}
\end{center}
Nếu hàm số $f(x)$ liên tục và không âm trên đoạn $\left[a;b\right]$ thì diện tích $S$ của hình thang cong giới hạn bởi đồ thị $y=f(x)$, trục hoành và hai đường thẳng $x=a$, $x=b$ được tính bởi:
$S=F(b)-F(a)$
trong đó $F(x)$ là một nguyên hàm của $f(x)$ trên đoạn $\left[a;b\right]$.
\subsubsection{Khái niệm tích phân}
Cho hàm số $f(x)$ liên tục trên đoạn $\left[a;b\right]$. Nếu $F(x)$ là nguyên hàm của hàm số $f(x)$ trên đoạn $\left[a;b\right]$ thì hiệu số $F(b)-F(a)$ được gọi là tích phân từ $a$ đến $b$ của hàm số $f(x)$, kí hiệu $\displaystyle\int\limits_a^bf(x)\mathrm{d}x$.\\
\begin{note}Chú ý:
	\begin{itemize}
		\item Hiệu số $F(b)-F(a)$ còn được kí hiệu là $ F(x)\big|_a^b$.\\
		Vậy $\displaystyle\int\limits_a^bf(x)\mathrm{d}x= F(x)\big|_a^b=F(b)-F(a)$.
		\item Ta gọi $\displaystyle\int\limits_a^b{}$ là dấu tích phân, $a$ là cận dưới, $b$ là cận trên, $f(x)\mathrm{d}x$ là biểu thức dưới dấu tích phân và $f(x)$ là hàm số dưới dấu tích phân.
		\item Quy ước: $\displaystyle\int\limits_a^af(x)\mathrm{d}x=0$; $\displaystyle\int\limits_a^bf(x)\mathrm{d}x=-\displaystyle\int\limits_b^af(x)\mathrm{d}x$.
		\item Tích phân của hàm số $f$ từ $a$ đến $b$ chỉ phụ thuộc vào $f$ và các cận $a$, $b$ mà không phụ thuộc vào biến $x$ hay $t$, nghĩa là $\displaystyle\int\limits_a^bf(x)\mathrm{d}x=\displaystyle\int\limits_a^bf(t)\mathrm{d}t$.
		\item Ý nghĩa hình học của tích phân.\\
		\immini{
			Nếu hàm số $f(x)$ liên tục và không âm trên đoạn $\left[a;b\right]$ thì $\displaystyle\int\limits_a^bf(x)\mathrm{d}x$ là diện tích $S$ của hình thang cong giới hạn bởi đồ thị $y=f(x)$, trục hoành và hai đường thẳng $x=a$, $x=b$.
			$$S=\displaystyle\int\limits_a^bf(x)\mathrm{\,d}x.$$}{\begin{tikzpicture}[>=stealth,scale=0.8]
				%		\tkzInit[xmin=-0.5,ymin=-2.5,xmax=6.5,ymax=2.5] \tkzClip
				\draw[->] (-0.5,0)--(5.3,0) node[below] {$x$} ;
				\draw[->] (0,-.5)--(0,2.3) node[right] {$y$} ;
				\draw (0,0) node[below left] {$O$};
				%		\draw (1,0) ellipse (0.16 and 1);
				%		\draw (4,0) ellipse (0.25 and 1.73);
				%Nhánh trên
				\draw[domain=1:4] 
				plot(\x,{0.31*(\x)^3-2.28*(\x)^2+5.14*(\x)-2.17}) ;
				%Nhánh dưới
				%	\draw[domain=1:4]
				%	plot(\x,{-0.31*(\x)^3+2.28*(\x)^2-5.14*(\x)+2.17}) ;
				%Tô màu
				\draw[pattern = north east lines,opacity=.3, line width = 1.2pt,draw=none] (1,1) plot[domain=1:4] (\x,{0.31*(\x)^3-2.28*(\x)^2+5.14*(\x)-2.17})--(4,0)--(1,0)--cycle;
				
				%Các yếu tố khác
				\draw (1,0) node[below] {$a$};
				\draw (4,0) node[below] {$b$};
				\draw[dashed] (1,0)--(1,1);
				\draw[dashed] (4,0)--(4,1.72);
				\draw (2.5,1.7) node {$y=f(x)$} ;
				\draw (2.5,0.7) node {$S$} ;
				%	\draw[->] (5,0.25) arc (90:270:0.3);
		\end{tikzpicture}}
	\end{itemize}
\end{note}
\begin{nx}
	\begin{itemize}
		\item Nếu hàm số $f(x)$ có đạo hàm $f'(x)$ và $f'(x)$ liên tục trên đoạn $\left[a;b\right]$ thì\\
		$f(b)-f(a)=\displaystyle\int\limits_a^bf'(x)\mathrm{d}x$.
		\item Cho hàm số $f(x)$ liên tục trên đoạn $\left[a;b\right]$. Khi đó $\dfrac{1}{b-a}\displaystyle\int\limits_a^bf(x)\mathrm{d}x$ được gọi là giá trị trung bình của hàm số $f(x)$ trên đoạn $\left[a;b\right]$.
		\item Đạo hàm của quãng đường di chuyển của vật theo thời gian bằng tốc độ của chuyển động tại mọi thời điểm $v(t)=s'(t)$. Do đó, nếu biết tốc độ $v(t)$ tại mọi thời điểm $t\in\left[a;b\right]$ thì tính được quãng đường di chuyển trong khoảng thời gian từ $a$ đến $b$ theo công thức: $s=s(b)-s(a)=\displaystyle\int\limits_a^bv(t)\mathrm{d}t$.
	\end{itemize}
\end{nx}
\subsubsection{Tính chất của tích phân}
Cho hai hàm số $f(x)$, $g(x)$ liên tục trên đoạn $\left[a;b\right]$. Khi đó:
\begin{enumerate}
	\item $\displaystyle\int\limits_a^bkf(x)\mathrm{d}x=k\displaystyle\int\limits_a^bf(x)\mathrm{d}x$, với $k$ là hằng số.
	\item $\displaystyle\int\limits_a^b\left[f(x)\pm g(x)\right]\mathrm{\,d}x=\displaystyle\int\limits_a^b{f(x)\mathrm{\,d}x}\pm\displaystyle\int\limits_a^bg(x)\mathrm{\,d}x$.
	\item $\displaystyle\int\limits_a^bf(x)\mathrm{\,d}x=\displaystyle\int\limits_a^cf(x)\mathrm{\,d}x+\displaystyle\int\limits_c^bf(x)\mathrm{\,d}x$ với $c\in\left(a;b\right)$.
\end{enumerate}
\subsection{Phân loại và phương pháp giải bài tập}
\begin{dang}{Tính chất của tích phân}

\end{dang}
\setcounter{ex}{0}
\TN
\Opensolutionfile{ans}[ans/ans-2C4B2CD3-LC]
\begin{ex}%[Câu 1]%[2D4N2-1]
	Nếu $\displaystyle\int\limits_0^3f(x)\mathrm{\,d}x=6$ thì $\displaystyle\int\limits_0^3\left[\dfrac{1}{3}f(x)+2\right]\mathrm{\,d}x$ bằng
	\choice
	{\True $8$}
	{$5$}
	{$9$}
	{$6$}
	\loigiai{
		Ta có $\displaystyle\int\limits_0^3\left[\dfrac{1}{3}f(x)+2\right]\mathrm{\,d}x=\dfrac{1}{3}\displaystyle\int\limits_0^3f(x)\mathrm{\,d}x+\displaystyle\int\limits_0^32\mathrm{\,d}x=\dfrac{1}{3}\cdot 6+6=8$.}
\end{ex}
\begin{ex}%[Câu 2]%[2D4N2-1]
	Nếu $\displaystyle\int_1^4 f(x) \mathrm{\,d}x=3$ và $\displaystyle\int_1^4 g(x) \mathrm{\,d}x=-2$ thì $\displaystyle\int_1^4\left(f(x)-g(x)\right)\mathrm{\,d}x$ bằng
	\choice
	{$-1$}
	{$-5$}
	{\True $5$}
	{$1$}
	\loigiai{
		Ta có $\displaystyle\int _1^4\left[f(x)-g(x)\right]\mathrm{\,d}x=\displaystyle\int _1^4f(x)\mathrm{\,d}x-\displaystyle\int _1^4g(x)\mathrm{\,d}x=3-(-2)=5$.}
\end{ex}
\begin{ex}%[Câu 3]%[2D4N2-1]
	Nếu $\displaystyle\int\limits_1^4f(x)\mathrm{\,d}x=5$ và $\displaystyle\int\limits_1^4g(x)\mathrm{\,d}x=-4$ thì $\displaystyle\int\limits_1^4\left[f(x)-g(x)\right]\mathrm{\,d}x$ bằng
	\choice
	{$-1$}
	{$-9$}
	{$1$}
	{\True $9$}
	\loigiai{
		Ta có $\displaystyle\int\limits_1^4\left[f(x)-g(x)\right]\mathrm{\,d}x=\displaystyle\int\limits_1^4f(x)\mathrm{\,d}x-\displaystyle\int\limits_1^4g(x)\mathrm{\,d}x=5-(-4)=9$.}
\end{ex}
\begin{ex}%[Câu 4]%[2D4N2-1]
	Biết $\displaystyle\int\limits_1^{2024}f(x)\mathrm{\,d}x=-3$ và $\displaystyle\int\limits_{2024}^1g(x)\mathrm{\,d}x=2$. Khi đó $\displaystyle\int\limits_1^{2024}\left[f(x)-g(x)\right]\mathrm{\,d}x$ bằng
	\choice
	{$6$}
	{$-5$}
	{$5$}
	{\True $-1$}
	\loigiai{
		Ta có $\displaystyle\int\limits_{2024}^1g(x)\mathrm{\,d}x=2\Leftrightarrow \displaystyle\int\limits_1^{2024}g(x)\mathrm{\,d}x=-2$.\\
		Do đó $\displaystyle\int\limits_1^{2024}\left[f(x)-g(x)\right]\mathrm{\,d}x=\displaystyle\int\limits_1^{2024}f(x)\mathrm{\,d}x-\displaystyle\int\limits_1^{2024}g(x)\mathrm{\,d}x=-3-(-2)=-1$.}
\end{ex}
\begin{ex}%[Câu 5]%[2D4N2-1]
	Nếu $\displaystyle\int\limits_0^3f(x)\mathrm{\,d}x=3$ thì $\displaystyle\int\limits_0^34f(x)\mathrm{\,d}x$ bằng
	\choice
	{$3$}
	{\True $12$}
	{$36$}
	{$4$}
	\loigiai{
		Ta có $\displaystyle\int\limits_0^34f(x)\mathrm{\,d}x=4\displaystyle\int\limits_0^3f(x)\mathrm{\,d}x=4\cdot 3=12$.}
\end{ex}
\begin{ex}%[Câu 6]%[2D4N2-1]
	Cho $\displaystyle\int\limits_0^2f(x)\mathrm{\,d}x=\dfrac{1}{2024}$. Tính $I=\displaystyle\int\limits_0^2 2024f(x)\mathrm{\,d}x$.
	\choice
	{$I=5$}
	{$I=\dfrac{1}{2024}$}
	{\True $I=1$}
	{$I=2024$}
	\loigiai{
		Ta có $I=\displaystyle\int\limits_0^2 2024f(x)\mathrm{\,d}x=2024\displaystyle\int\limits_0^2f(x)\mathrm{\,d}x=2024\cdot \dfrac{1}{2024}=1$.}
\end{ex}
\begin{ex}%[Câu 7]%[2D4N2-1]
	Nếu $\displaystyle\int\limits_0^5f(x)\mathrm{\,d}x=5$ thì $\displaystyle\int\limits_5^05f(x)\mathrm{\,d}x$ bằng
	\choice
	{$1$}
	{$-1$}
	{$25$}
	{\True $-25$}
	\loigiai{
		Ta có $\displaystyle\int\limits_5^05f(x)\mathrm{\,d}x=5\displaystyle\int\limits_5^0f(x)\mathrm{\,d}x=-5\cdot\displaystyle\int\limits_0^5f(x)\mathrm{\,d}x=(-5)\cdot 5=-25$.
		
	}
\end{ex}
\begin{ex}%[Câu 8]%[2D4N2-1]
	Nếu $\displaystyle\int\limits_0^2f(x)\mathrm{\,d}x=5$ thì $\displaystyle\int\limits_0^2\left[2f(x)-1\right]\mathrm{\,d}x$ bằng
	\choice
	{\True $8$}
	{$9$}
	{$10$}
	{$12$}
	\loigiai{
		Ta có $\displaystyle\int _0^2\left[2f(x)-1\right]\mathrm{\,d}x=2\displaystyle\int _0^2f(x)\mathrm{\,d}x-\displaystyle\int _0^21\mathrm{\,d}x=2\cdot 5-2=8$.}
\end{ex}
\begin{ex}%[Câu 9]%[2D4N2-1]
	Nếu $\displaystyle\int_0^2 f(x) d x=3$ thì $\displaystyle\int_0^2\left[2f(x)-1\right]\mathrm{\,d}x$ bằng
	\choice
	{$6$}
	{\True $4$}
	{$8$}
	{$5$}
	\loigiai{
		Ta có $\displaystyle\int_0^2\left[2f(x)-1\right]\mathrm{\,d}x=2\displaystyle\int_0^2f(x)\mathrm{\,d}x-\displaystyle\int_0^2\mathrm{\,d}x=2\cdot 3-2=4$.}
\end{ex}
\begin{ex}%[Câu 10]%[2D4N2-1]
	Cho $\displaystyle\int\limits_0^1f(x)\mathrm{\,d}x=2$ và $\displaystyle\int\limits_0^1g(x)\mathrm{\,d}x=5$, khi $\displaystyle\int\limits_0^1\left[f(x)-2g(x)\right]\mathrm{\,d}x$ bằng
	\choice
	{\True $-8$}
	{$1$}
	{$-3$}
	{$12$}
	\loigiai{
		Ta có $\displaystyle\int\limits_0^1\left[f(x)-2g(x)\right]\mathrm{\,d}x=\displaystyle\int\limits_0^1f(x)\mathrm{\,d}x-2\displaystyle\int\limits_0^1g(x)\mathrm{\,d}x=2-2\cdot 5=-8$.}
\end{ex}
\begin{ex}%[Câu 11]%[2D4H2-1]
	Cho $\displaystyle\int\limits_0^{\frac{\pi}{2}}f(x)\mathrm{\,d}x=5$. Tính $I=\displaystyle\int\limits_0^{\frac{\pi}{2}}\left[f(x)+2\sin x\right]\mathrm{\,d}x$.
	\choice
	{\True $I=7$}
	{$I=5+\dfrac{\pi}{2}$}
	{$I=3$}
	{$I=5+\pi $}
	\loigiai{
		Ta có
		\begin{eqnarray*}
			&I&=\displaystyle\int\limits_0^{\frac{\pi}{2}}\left[f(x)+2\sin x\right]\mathrm{\,d}x\\
			&&=\displaystyle\int\limits_0^{\frac{\pi}{2}}f(x)\mathrm{\,d}x\text{+2}\displaystyle\int\limits_0^{\tfrac{\pi}{2}}\sin x\mathrm{\,d}x\\
			&&=\displaystyle\int\limits_0^{\frac{\pi}{2}}f(x)\mathrm{\,d}x-2\cos x\bigg|_0^{\frac{\pi}{2}}\\
			&&=5-2(0-1)=7.
		\end{eqnarray*}
	}
\end{ex}
\begin{ex}%[Câu 12]%[2D4H2-1]
	Cho $\displaystyle\int\limits_1^2\left[4f(x)-2x\right]\mathrm{\,d}x=1$. Khi đó $\displaystyle\int\limits_1^2f(x)\mathrm{\,d}x$ bằng
	\choice
	{\True $1$}
	{$-3$}
	{$3$}
	{$-1$}
	\loigiai{
		Ta có \begin{eqnarray*}
			&&\displaystyle\int\limits_1^2\left[4f(x)-2x\right]\mathrm{\,d}x=1\\
			&\Leftrightarrow&4\displaystyle\int\limits_1^2f(x)\mathrm{\,d}x-2\displaystyle\int\limits_1^2x\mathrm{\,d}x=1\\
			&\Leftrightarrow&4\displaystyle\int\limits_1^2f(x)\mathrm{\,d}x-2\cdot  \dfrac{x^2}{2}\bigg|_1^2=1\\
			&\Leftrightarrow&4\displaystyle\int\limits_1^2f(x)\mathrm{\,d}x=4\\
			&\Leftrightarrow&\displaystyle\int\limits_1^2f(x)\mathrm{\,d}x=1.
		\end{eqnarray*}
	}
\end{ex}
% \begin{ex}%[Câu 13]%[2D4H2-1]
% 	Cho $\displaystyle\int\limits_0^1f(x)\mathrm{\,d}x=1$, tích phân $\displaystyle\int\limits_0^1\left(2f(x)-3x^2\right)\mathrm{\,d}x$ bằng
% 	\choice
% 	{\True $1$}
% 	{$0$}
% 	{$3$}
% 	{$-1$}
% 	\loigiai{Ta có 
% 		$\displaystyle\int\limits_0^1(2f(x)-3x^2)\mathrm{\,d}x=2\displaystyle\int\limits_0^1f(x)\mathrm{\,d}x-3\displaystyle\int\limits_0^1x^2\mathrm{\,d}x=2-1=1$.}
% \end{ex}
% \begin{ex}%[Câu 14]%[2D4H2-1]
% 	Cho $\displaystyle\int\limits_{-1}^2f(x)\mathrm{\,d}x=2$ và $\displaystyle\int\limits_{-1}^2g(x)\mathrm{\,d}x=-1$. Tính $I=\displaystyle\int\limits_{-1}^2\left[x+2f(x)-3g(x)\right]\mathrm{\,d}x$.
% 	\choice
% 	{\True $I=\dfrac{17}{2}$}
% 	{$I=\dfrac{5}{2}$}
% 	{$I=\dfrac{7}{2}$}
% 	{$I=\dfrac{11}{2}$}
% 	\loigiai{
% 		Ta có 
% 		\begin{eqnarray*}
% 			&I&=\displaystyle\int\limits_{-1}^2\left[x+2f(x)-3g(x)\right]\mathrm{\,d}x\\
% 			&&= \dfrac{x^2}{2}\bigg|_{-1}^2+2\displaystyle\int\limits_{-1}^2f(x)\mathrm{\,d}x-3\displaystyle\int\limits_{-1}^2g(x)\mathrm{\,d}x\\
% 			&&=\dfrac{3}{2}+2\cdot 2-3(-1)=\dfrac{17}{2}.
% 		\end{eqnarray*}
% 	}
% \end{ex}
% \begin{ex}%[Câu 15]%[2D4H2-1]
% 	Cho $\displaystyle\int\limits_0^2f(x)\mathrm{\,d}x=3$,$\displaystyle\int\limits_0^2g(x)\mathrm{\,d}x=-1$ thì $\displaystyle\int\limits_0^2\left[f(x)-5g(x)+x\right]\mathrm{\,d}x$ bằng
% 	\choice
% 	{$12$}
% 	{$0$}
% 	{$8$}
% 	{\True $10$}
% 	\loigiai{Ta có 
% 		$\displaystyle\int\limits_0^2\left[f(x)-5g(x)+x\right]\mathrm{\,d}x=\displaystyle\int\limits_0^2f(x)\mathrm{\,d}x-5\displaystyle\int\limits_0^2\mathrm{g}(x)\mathrm{\,d}x+\displaystyle\int\limits_0^2x\mathrm{\,d}x=3+5+2=10$.}
% \end{ex}
% \begin{ex}%[Câu 16]%[2D4H2-1]
% 	Cho $\displaystyle\int\limits_0^5f(x)\mathrm{\,d}x=-2$. Tích phân $\displaystyle\int\limits_0^5\left[4f(x)-3x^2\right]\mathrm{\,d}x$ bằng
% 	\choice
% 	{$-140$}
% 	{$-130$}
% 	{$-120$}
% 	{\True $-133$}
% 	\loigiai{Ta có
% 		$\displaystyle\int\limits_0^5\left[4f(x)-3x^2\right]\mathrm{\,d}x=4\displaystyle\int\limits_0^5f(x)\mathrm{\,d}x-\displaystyle\int\limits_0^53x^2\mathrm{\,d}x=-8-x^3\bigg|_0^5=-8-125=-133$.}
% \end{ex}
% \begin{ex}%[Câu 17]%[2D4H2-1]
% 	Cho $\displaystyle\int\limits_1^2\left[4f(x)-2x\right]\mathrm{\,d}x=1$. Khi đó $\displaystyle\int\limits_1^2f(x)\mathrm{\,d}x$ bằng:
% 	\choice
% 	{\True $1$}
% 	{$-3$}
% 	{$3$}
% 	{$-1$}
% 	\loigiai{Ta có
% 		\begin{eqnarray*}
% 			&&\displaystyle\int\limits_1^2\left[4f(x)-2x\right]\mathrm{\,d}x=1\\
% 			&\Leftrightarrow&4\displaystyle\int\limits_1^2f(x)\mathrm{\,d}x-2\displaystyle\int\limits_1^2x\mathrm{\,d}x=1\\
% 			&\Leftrightarrow&4\displaystyle\int\limits_1^2f(x)\mathrm{\,d}x-2\cdot  \dfrac{x^2}{2}\bigg|_1^2=1\\
% 			&\Leftrightarrow&4\displaystyle\int\limits_1^2f(x)\mathrm{\,d}x=4\\
% 			&\Leftrightarrow&\displaystyle\int\limits_1^2f(x)\mathrm{\,d}x=1.
% 		\end{eqnarray*}
% 	}
% \end{ex}
% \begin{ex}%[Câu 18]%[2D4H2-1]
% 	Cho $\displaystyle\int\limits_{-2}^2f(x)\mathrm{\,d}x=1$, $\displaystyle\int\limits_{-2}^4f(t)\mathrm{\,d}t=-4$. Tính $\displaystyle\int\limits_2^4f(y)\mathrm{\,d}y$.
% 	\choice
% 	{$I=5$}
% 	{$I=-3$}
% 	{$I=3$}
% 	{\True $I=-5$}
% 	\loigiai{	
% 		Ta có $\displaystyle\int\limits_{-2}^4f(t)\mathrm{\,d}t=\displaystyle\int\limits_{-2}^4f(x)\mathrm{\,d}x$, $\displaystyle\int\limits_2^4f(y)\mathrm{\,d}y=\displaystyle\int\limits_2^4f(x)\mathrm{\,d}x$.\\
% 		Khi đó $\displaystyle\int\limits_{-2}^2f(x)\mathrm{\,d}x+\displaystyle\int\limits_2^4f(x)\mathrm{\,d}x=\displaystyle\int\limits_{-2}^4f(x)\mathrm{\,d}x$. Do đó
% 		$$ \displaystyle\int\limits_2^4f(x)\mathrm{\,d}x=\displaystyle\int\limits_{-2}^4f(x)\mathrm{\,d}x-\displaystyle\int\limits_{-2}^2f(x)\mathrm{\,d}x=-4-1=-5.$$
% 		Vậy $\displaystyle\int\limits_2^4f(y)\mathrm{\,d}y=-5$.}
% \end{ex}
% \begin{ex}%[Câu 19]%[2D4H2-1]
% 	Cho hàm số $f(x)$ liên tục trên $\mathbb{R}$ và có $\displaystyle\int\limits_0^2f(x)\mathrm{\,d}x=9;\displaystyle\int\limits_2^4f(x)\mathrm{\,d}x=4$. Tính $I=\displaystyle\int\limits_0^4f(x)\mathrm{\,d}x$.
% 	\choice
% 	{$I=5$}
% 	{$I=36$}
% 	{$I=\dfrac{9}{4}$}
% 	{\True $I=13$}
% 	\loigiai{
% 		Ta có $I=\displaystyle\int\limits_0^4f(x)\mathrm{\,d}x=\displaystyle\int\limits_0^2f(x)\mathrm{\,d}x+\displaystyle\int\limits_2^4f(x)\mathrm{\,d}x=9+4=13$.}
% \end{ex}
% \begin{ex}%[Câu 20]%[2D4H2-1]
% 	Cho hàm số $f(x)$ liên tục trên $\mathbb{R}$ và $\displaystyle\int\limits_0^4f(x)\mathrm{\,d}x=10$, $\displaystyle\int\limits_3^4f(x)\mathrm{\,d}x=4$. Tích phân $\displaystyle\int\limits_0^3f(x)\mathrm{\,d}x$ bằng
% 	\choice
% 	{$4$}
% 	{$7$}
% 	{$3$}
% 	{\True $6$}
% 	\loigiai{
% 		Theo tính chất của tích phân, ta có $\displaystyle\int\limits_0^3f(x)\mathrm{\,d}x+\displaystyle\int\limits_3^4f(x)\mathrm{\,d}x=\displaystyle\int\limits_0^4f(x)\mathrm{\,d}x$.\\
% 		Suy ra  $\displaystyle\int\limits_0^3f(x)\mathrm{\,d}x=\displaystyle\int\limits_0^4f(x)\mathrm{\,d}x-\displaystyle\int\limits_3^4f(x)\mathrm{\,d}x=10-4=6$.\\
% 		Vậy $\displaystyle\int\limits_0^3f(x)\mathrm{\,d}x=6$.}
% \end{ex}
% \begin{ex}%[Câu 21]%[2D4H2-1]
% 	Cho hàm số $f(x)$ liên tục trên đoạn $[0;10]$ và $\displaystyle\int\limits_0^{10}f(x)\mathrm{\,d}x=7$; $\displaystyle\int\limits_2^6f(x)\mathrm{\,d}x=3$.\\
% 	Tính $P=\displaystyle\int\limits_0^2f(x)\mathrm{\,d}x+\displaystyle\int\limits_6^{10}f(x)\mathrm{\,d}x$.
% 	\choice
% 	{\True $P=4$}
% 	{$P=10$}
% 	{$P=7$}
% 	{$P=-4$}
% 	\loigiai{
% 		Ta có $\displaystyle\int\limits_0^{10}f(x)\mathrm{\,d}x=\displaystyle\int\limits_0^2f(x)\mathrm{\,d}x+\displaystyle\int\limits_2^6f(x)\mathrm{\,d}x+\displaystyle\int\limits_6^{10}f(x)\mathrm{\,d}x$ hay $7=P+3\Leftrightarrow P=4$.}
% \end{ex}
% \begin{ex}%[Câu 22]%[2D4H2-1]
% 	Cho hàm số $f(x)$ liên tục trên đoạn $[0; 6]$ thỏa mãn $\displaystyle\int\limits_0^6f(x)\mathrm{\,d}x=10$ và $\displaystyle\int\limits_2^4f(x)\mathrm{\,d}x=6$. 	Tính giá trị của biểu thức $P=\displaystyle\int\limits_0^2f(x)\mathrm{\,d}x+\displaystyle\int\limits_4^6f(x)\mathrm{\,d}x$.
% 	\choice
% 	{\True$P=4$}
% 	{$P=16$}
% 	{$P=8$}
% 	{$P=10$}
% 	\loigiai{
% 		Ta có $\displaystyle\int\limits_0^{6}f(x)\mathrm{\,d}x=\displaystyle\int\limits_0^2f(x)\mathrm{\,d}x+\displaystyle\int\limits_2^4f(x)\mathrm{\,d}x+\displaystyle\int\limits_4^{6}f(x)\mathrm{\,d}x$ hay $7=P+3\Leftrightarrow P=4$.	
% 	}
% \end{ex}
\Closesolutionfile{ans}
% \indapan{10}{ans/ans-2C4B2CD3-LC}
\TNTF
\Opensolutionfile{ans}[ans/ans-2C4B2CD3-DS]
\begin{ex}%[Câu 23]%[2D4H2-1]
	Cho hai hàm $f$, $g$ liên tục trên $K$ và $a$, $b$ là các số bất kỳ thuộc $K$.
	\choiceTF
	{\True $\displaystyle\int\limits_a^b\left[f(x)+2g(x)\right]\mathrm{\,d}x=\displaystyle\int\limits_a^bf(x)\mathrm{\,d}x\text{+2}\displaystyle\int\limits_a^bg(x)\mathrm{\,d}x$}
	{$\displaystyle\int\limits_a^b\dfrac{f(x)}{g(x)}\mathrm{\,d}x=\dfrac{\displaystyle\int\limits_a^bf(x)\mathrm{\,d}x}{\displaystyle\int\limits_a^bg(x)\mathrm{\,d}x}$}
	{$\displaystyle\int\limits_a^b\left[f(x)\cdot g(x)\right]\mathrm{\,d}x=\displaystyle\int\limits_a^bf(x)\mathrm{\,d}x \displaystyle\int\limits_a^bg(x)\mathrm{\,d}x$}
	{$\displaystyle\int\limits_a^bf^2(x)\mathrm{\,d}x=\left[\displaystyle\int\limits_a^bf(x)\mathrm{\,d}x\right]^2$}
	\loigiai{
		\begin{itemchoice}
			\itemch Đúng. Theo tính chất tích phân ta có
			$\displaystyle\int\limits_a^b\left[f(x)+g(x)\right]\mathrm{\,d}x=\displaystyle\int\limits_a^bf(x)\mathrm{\,d}x+\displaystyle\int\limits_a^bg(x)\mathrm{\,d}x;\displaystyle\int\limits_a^bkf(x)\mathrm{\,d}x=k\displaystyle\int\limits_a^bf(x)\mathrm{\,d}x$, với $k\in \mathbb{R}$.
			\itemch Sai. Cho $a=1,b=2$ và $f(x)=x+1, g(x)=x$. Khi đó
			$$VT=\displaystyle\int\limits_{1}^2\dfrac{x+1}{x}\mathrm{\,d}x==\displaystyle\int\limits_{1}^2\left(1+\dfrac{1}{x}\right)\mathrm{\,d}x=\left(x+\ln x\right)\bigg|_1^2=1+\ln 2.$$
			và $$VP=\dfrac{\displaystyle\int\limits_1^2(x+1)\mathrm{\,d}x}{\displaystyle\int\limits_1^2x\mathrm{\,d}x}=\dfrac{\left(\dfrac{x^2}{2}+x\right)\bigg|_1^2}{\dfrac{x^2}{2}\bigg|_1^2}=\dfrac{1}{3}.$$
			Do đó $VT\neq VP$.
			\itemch Sai. Cho $a=1, b=2$ và $f(x)=x, g(x)=\dfrac{1}{x}$. Khi đó
			$$VT=\displaystyle\int\limits_1^2\left[x\cdot \dfrac{1}{x}\right]\mathrm{\,d}x=x\bigg|_1^2=1.$$
			và $$VP=\displaystyle\int\limits_1^2x\mathrm{\,d}x\cdot \displaystyle\int\limits_1^2\dfrac{1}{x}\mathrm{\,d}x=\left(\dfrac{x^2}{2}\right)\bigg|_1^2\cdot \ln x\bigg|_1^2=\dfrac{3}{2}\ln 2.$$
			Do đó $VT\neq VP$.
			\itemch Sai. Cho $a=1,b=2$ và $f(x)=x$. Khi đó
			$$VT=\displaystyle\int\limits_1^2x^2\mathrm{\,d}x=\left(\dfrac{x^3}{3}\right)\bigg|_1^2=\dfrac{7}{3}.$$
			và $$VP=\left(\displaystyle\int\limits_1^2x\mathrm{\,d}x\right)^2=\left(\dfrac{x^2}{2}\bigg|_1^2\right)^2=\dfrac{9}{4}.$$
			Do đó $VT\neq VP$.
		\end{itemchoice}
	}
\end{ex}
\begin{ex}%[Câu 24]%[2D4H2-1]
	Cho hàm số $f(x),g(x)$ liên tục trên $\mathbb{R}$.
	\choiceTF
	{\True Nếu $\displaystyle\int\limits_0^2f(x)\mathrm{\,d}x=4$ thì $\displaystyle\int\limits_0^2\left[\dfrac{1}{2}f(x)+2\right]\mathrm{\,d}x=6$}
	{\True Nếu $\displaystyle\int\limits_2^5f(x)\mathrm{\,d}x=3$ và $\displaystyle\int\limits_2^5g(x)\mathrm{\,d}x=-2$ thì $\displaystyle\int\limits_2^5\left[f(x)+g(x)\right]\mathrm{\,d}x=1$}
	{Nếu $\displaystyle\int\limits_1^4f(x)\mathrm{\,d}x=6$ và $\displaystyle\int\limits_1^4g(x)\mathrm{\,d}x=-5$ thì $\displaystyle\int\limits_1^4\left[f(x)-g(x)\right]\mathrm{\,d}x=1$}
	{\True Nếu $\displaystyle\int\limits_2^3f(x)\mathrm{\,d}x=4$ và$\displaystyle\int\limits_2^3g(x)\mathrm{\,d}x=1$ thì $\displaystyle\int\limits_2^3\left[f(x)-g(x)\right]\mathrm{\,d}x=3$}
	\loigiai{
		\begin{itemchoice}
			\itemch Đúng. Ta có $\displaystyle\int\limits_0^2\left[\dfrac{1}{2}f(x)+2\right]\mathrm{\,d}x=\dfrac{1}{2}\displaystyle\int\limits_0^2f(x)\mathrm{\,d}x+\displaystyle\int\limits_0^22\mathrm{\,d}x=\dfrac{1}{2}\cdot 4+4=6$.
			\itemch Đúng. Ta có $\displaystyle\int\limits_2^5\left[f(x)+g(x)\right]\mathrm{\,d}x=\displaystyle\int\limits_2^5f(x)\mathrm{\,d}x+\displaystyle\int\limits_2^5g(x)\mathrm{\,d}x=3+(-2)=1$.
			\itemch Sai. Ta có $\displaystyle\int\limits_1^4\left[f(x)-g(x)\right]\mathrm{\,d}x=\displaystyle\int\limits_1^4f(x)\mathrm{\,d}x-\displaystyle\int\limits_1^4g(x)\mathrm{\,d}x=6-(-5)=11$.
			\itemch Đúng. Ta có $\displaystyle\int\limits_2^3\left[f(x)-g(x)\right]\mathrm{\,d}x=\displaystyle\int\limits_2^3f(x)\mathrm{\,d}x-\displaystyle\int\limits_2^3g(x)\mathrm{\,d}x=4-1=3$.
		\end{itemchoice}
	}
\end{ex}
\begin{ex}%[Câu 25]%[2D4H2-1]
	Cho hàm số $f(x),g(x)$ liên tục trên $\mathbb{R}$.
	\choiceTF
	{Biết $\displaystyle\int\limits_2^3f(x)\mathrm{\,d}x=3$ và $\displaystyle\int\limits_3^2g(x)\mathrm{\,d}x=1$. Khi đó $\displaystyle\int\limits_2^3\left[f(x)+g(x)\right]\mathrm{\,d}x=4$}
	{\True Biết $\displaystyle\int\limits_1^3f(x)\mathrm{\,d}x=2022$ và $\displaystyle\int\limits_3^1g(x)\mathrm{\,d}x=1$. Khi đó $\displaystyle\int\limits_1^3\left[f(x)+g(x)\right]\mathrm{\,d}x=2021$}
	{\True Biết $\displaystyle\int\limits_1^2f(x)\mathrm{\,d}x=3$ và $\displaystyle\int\limits_1^2g(x)\mathrm{\,d}x=2$. Khi đó $\displaystyle\int\limits_1^2\left[f(x)-g(x)\right]\mathrm{\,d}x=1$}
	{Biết $\displaystyle\int\limits_2^5f(x)\mathrm{\,d}x=2$. Khi đó $\displaystyle\int\limits_2^53f(x)\mathrm{\,d}x=2$}
	\loigiai{
		\begin{itemchoice}
			\itemch Sai. Ta có
			$\displaystyle\int\limits_2^3\left[f(x)+g(x)\right]\mathrm{\,d}x=\displaystyle\int\limits_2^3f(x)\mathrm{\,d}x+\displaystyle\int\limits_2^3g(x)\mathrm{\,d}x=\displaystyle\int\limits_2^3f(x)\mathrm{\,d}x-\displaystyle\int\limits_3^2g(x)\mathrm{\,d}x=2$.
			\itemch Đúng. Ta có $\displaystyle\int\limits_3^1g(x)\mathrm{\,d}x=1\Leftrightarrow \displaystyle\int\limits_1^3g(x)\mathrm{\,d}x=-1$. Do đó 
			$$\displaystyle\int\limits_1^3\left[f(x)+g(x)\right]\mathrm{\,d}x=\displaystyle\int\limits_1^3f(x)\mathrm{\,d}x+\displaystyle\int\limits_1^3g(x)\mathrm{\,d}x=2022+(-1)=2021.$$
			\itemch Đúng. Ta có $\displaystyle\int\limits_1^2\left[f(x)-g(x)\right]\mathrm{\,d}x=\displaystyle\int\limits_1^2f(x)\mathrm{\,d}x-\displaystyle\int\limits_1^2g(x)\mathrm{\,d}x=3-2=1$.
			\itemch Sai. Ta có $\displaystyle\int\limits_2^53f(x)\mathrm{\,d}x=3\displaystyle\int\limits_2^5f(x)\mathrm{\,d}x=3\cdot 2=6$.
		\end{itemchoice}
	}
\end{ex}
\begin{ex}%[Câu 26]%[2D4H2-1]
	Cho hàm số $f(x)$ liên tục trên $\mathbb{R}$.
	\choiceTF
	{\True Nếu $\displaystyle\int\limits_0^3f(x)\mathrm{\,d}x=3$ thì $\displaystyle\int\limits_0^32f(x)\mathrm{\,d}x=6$}
	{\True Nếu $\displaystyle\int\limits_1^4f(x)\mathrm{\,d}x=2024$ thì $\displaystyle\int\limits_4^1f(x)\mathrm{\,d}x=-2024$}
	{Nếu $\displaystyle\int\limits_6^0f(x)\mathrm{\,d}x=12$ thì $\displaystyle\int\limits_0^62022f(x)\mathrm{\,d}x=24264$}
	{\True Nếu $\displaystyle\int\limits_0^1f(x)\mathrm{\,d}x=4$ thì $\displaystyle\int\limits_0^12f(x)\mathrm{\,d}x=8$}
	\loigiai{
		\begin{itemchoice}
			\itemch Đúng. Ta có $\displaystyle\int\limits_0^32f(x)\mathrm{\,d}x=2\displaystyle\int\limits_0^3f(x)\mathrm{\,d}x=2\cdot 3=6$.
			\itemch Đúng. Ta có $\displaystyle\int\limits_4^1f(x)\mathrm{\,d}x=-\displaystyle\int\limits_1^4f(x)\mathrm{\,d}x=-2024$.
			\itemch Sai. Ta có $\displaystyle\int\limits_0^62022f(x)\mathrm{\,d}x=2022\displaystyle\int\limits_0^6f(x)\mathrm{\,d}x=2022\cdot (-12)=-24264$.
			\itemch Đúng. Ta có $\displaystyle\int\limits_0^12f(x)\mathrm{\,d}x=2\displaystyle\int\limits_0^1f(x)\mathrm{\,d}x=2\cdot 4=8$.
		\end{itemchoice}
	}
\end{ex}
% \begin{ex}%[Câu 27]%[2D4H2-1]
% 	Cho hàm số $f(x),g(x)$ liên tục trên $\mathbb{R}$.
% 	\choiceTF
% 	{Nếu $\displaystyle\int_0^2 f(x)d x=6$ thì $\displaystyle\int_0^2\left[2f(x)-1\right]\mathrm{\,d}x=-10$}
% 	{\True Nếu $\displaystyle\int\limits_0^2f(x)\mathrm{\,d}x=4$ thì $\displaystyle\int\limits_0^2\left[2f(x)-1)\right]\mathrm{\,d}x=6$}
% 	{\True Nếu $\displaystyle\int_0^2f(x)\mathrm{\,d}x=3$ và $\displaystyle\int_0^2g(x)\mathrm{\,d}x=7$ thì $\displaystyle\int_0^2\left[f(x)+3g(x)\right]\mathrm{\,d}x=24$}
% 	{\True Nếu $\displaystyle\int\limits_0^1\left[f(x)+2x\right]\mathrm{\,d}x=3$ thì $\displaystyle\int\limits_0^1f(x)\mathrm{\,d}x=2$}
% 	\loigiai{
% 		\begin{itemchoice}
% 			\itemch Sai. Ta có $\displaystyle\int _0^2\left[2f(x)-1\right]\mathrm{\,d}x=2\displaystyle\int _0^2f(x)\mathrm{\,d}x-\displaystyle\int _0^2\mathrm{\,d}x=2\cdot 6-2=10$.
% 			\itemch Đúng. Ta có $\displaystyle\int\limits_0^2\left[2f(x)-1)\right]\mathrm{\,d}x=\displaystyle\int\limits_0^22f(x)\mathrm{\,d}x-\displaystyle\int\limits_0^2\mathrm{\,d}x=2\cdot 4-2=6$.
% 			\itemch Đúng. Ta có $\displaystyle\int_0^2\left[f(x)+3g(x)\right]\mathrm{\,d}x=\displaystyle\int_0^2f(x)\mathrm{\,d}x+3\displaystyle\int_0^2g(x)\mathrm{\,d}x=3+3\cdot 7=24$.
% 			\itemch Đúng. Ta có:\\ $\displaystyle\int\limits_0^1\left[f(x)+2x\right]\mathrm{\,d}x=3\Leftrightarrow \displaystyle\int\limits_0^1f(x)\mathrm{\,d}x+2\displaystyle\int\limits_0^1x\mathrm{\,d}x=3\Leftrightarrow \displaystyle\int\limits_0^1f(x)\mathrm{\,d}x+2\cdot \dfrac{x^2}{2}\bigg|_0^1=3$.\\
% 			Suy ra $\displaystyle\int\limits_0^1f(x)\mathrm{\,d}x=3-x^2\bigg|_0^1=3-(1-0)=2$.
% 		\end{itemchoice}
% 	}
% \end{ex}
% \begin{ex}%[Câu 28]%[2D4H2-1]
% 	Cho hàm số $f(x),g(x)$ liên tục trên $\mathbb{R}$.
% 	\choiceTF
% 	{\True Nếu $\displaystyle\int\limits_{-1}^5f(x)\mathrm{\,d}x=-3$ thì $\displaystyle\int\limits_5^{-1}f(x)\mathrm{\,d}x=3$}
% 	{\True Nếu $\displaystyle\int\limits_2^3f(x)\mathrm{\,d}x=-6$ thì $\displaystyle\int\limits_3^22f(x)\mathrm{\,d}x=12$}
% 	{Nếu $\displaystyle\int\limits_1^2f(x)\mathrm{\,d}x=2$ và $\displaystyle\int\limits_1^2g(x)\mathrm{\,d}x=6$ thì $\displaystyle\int\limits_2^1\left[f(x)-g(x)\right]\mathrm{\,d}x=-4$}
% 	{Nếu $\displaystyle\int\limits_0^1f(x)\mathrm{\,d}x=3$ và $\displaystyle\int\limits_0^1g(x)\mathrm{\,d}x=-4$ thì $\displaystyle\int\limits_1^0\left[f(x)+g(x)\right]\mathrm{\,d}x=-1$}
% 	\loigiai{
% 		\begin{itemchoice}
% 			\itemch Đúng. Ta có $\displaystyle\int _5^{-1}f(x)\mathrm{\,d}x=-\displaystyle\int _{-1}^5f(x)\mathrm{\,d}x=-(-3)=3$.
% 			\itemch Đúng. Ta có $\displaystyle\int\limits_3^22f(x)\mathrm{\,d}x=-\displaystyle\int\limits_2^32f(x)\mathrm{\,d}x=-2\displaystyle\int\limits_2^3f(x)\mathrm{\,d}x=-2\cdot (-6)=12$.
% 			\itemch Sai. Ta có \\
% 			$\displaystyle\int\limits_2^1\left[f(x)-g(x)\right]\mathrm{\,d}x=-\displaystyle\int\limits_1^2\left[f(x)-g(x)\right]\mathrm{\,d}x=-\displaystyle\int\limits_1^2f(x)\mathrm{\,d}x+\displaystyle\int\limits_1^2g(x)\mathrm{\,d}x=-2+6=4$.
% 			\itemch Sai. Ta có\\
% 			$\displaystyle\int\limits_1^0\left[f(x)+g(x)\right]\mathrm{\,d}x=-\displaystyle\int\limits_0^1\left[f(x)+g(x)\right]\mathrm{\,d}x=-\displaystyle\int\limits_0^1f(x)\mathrm{\,d}x-\displaystyle\int\limits_0^1g(x)\mathrm{\,d}x=-3+4=1$.
% 		\end{itemchoice}
% 	}
% \end{ex}
% \begin{ex}%[Câu 29]%[2D4V2-1]
% 	Cho hàm số $f(x),g(x)$ liên tục trên $\mathbb{R}$.
% 	\choiceTF
% 	{\True Nếu $\displaystyle\int\limits_0^1f(x)\mathrm{\,d}x=-1$ và $\displaystyle\int\limits_0^3f(x)\mathrm{\,d}x=5$ thì $\displaystyle\int\limits_1^3f(x)=6$}
% 	{Nếu $\displaystyle\int\limits_1^2f(x)\mathrm{\,d}x=-3$ và $\displaystyle\int\limits_2^3f(x)\mathrm{\,d}x=4$ thì $\displaystyle\int\limits_1^3f(x)\mathrm{\,d}x=-1$}
% 	{Nếu $\displaystyle\int\limits_{-1}^0f(x)\mathrm{\,d}x=3, \displaystyle\int\limits_{0}^3f(x)\mathrm{\,d}x=1$ thì $\displaystyle\int\limits_{-1}^3f(x)\mathrm{\,d}x=-4$}
% 	{Nếu $\displaystyle\int\limits_{-2}^{5}f(x)\mathrm{\,d}x=8$ và $\displaystyle\int\limits_5^{-2}g(x)\mathrm{\,d}x=3$ thì $\displaystyle\int\limits_{-2}^5\left(f(x)-4g(x)-1\right)\mathrm{\,d}x=-13$}
% 	\loigiai{
% 		\begin{itemchoice}
% 			\itemch Đúng. Ta có 
% 			$\displaystyle\int\limits_0^3f(x)\mathrm{\,d}x =\displaystyle\int\limits_0^1f(x)\mathrm{\,d}x +\displaystyle\int\limits_1^3f(x)\mathrm{\,d}x$.\\
% 			Do đó $\displaystyle\int\limits_1^3f(x)\mathrm{\,d}x =\displaystyle\int\limits_0^3f(x)\mathrm{\,d}x-\displaystyle\int\limits_0^1f(x)\mathrm{\,d}x = 5+ 1= 6$.
% 			\itemch Sai. Ta có $\displaystyle\int\limits_1^3f(x)\mathrm{\,d}x=\displaystyle\int\limits_1^2f(x)\mathrm{\,d}x+\displaystyle\int\limits_2^3f(x)\mathrm{\,d}x=-3+4=1$.
% 			\itemch Sai. Ta có $\displaystyle\int\limits_{-1}^0f(x)\mathrm{\,d}x=3;\displaystyle\int\limits_{0}^3f(x)\mathrm{\,d}x=1;\displaystyle\int\limits_{-1}^3f(x)\mathrm{\,d}x=\displaystyle\int\limits_{-1}^0f(x)\mathrm{\,d}x+\displaystyle\int\limits_{0}^3f(x)\mathrm{\,d}x=3+1=4$.
% 			\itemch Sai. Ta có 
% 			\begin{eqnarray*}
% 				&&\displaystyle\int\limits_{-2}^5\left[f(x)-4g(x)-1\right]\mathrm{\,d}x\\
% 				&&=\displaystyle\int\limits_{-2}^5f(x)\mathrm{\,d}x-\displaystyle\int\limits_{-2}^54g(x)\mathrm{\,d}x-\displaystyle\int\limits_{-2}^5\mathrm{\,d}x\\
% 				&&=\displaystyle\int\limits_{-2}^5f(x)\mathrm{\,d}x-4\displaystyle\int\limits_{-2}^5g(x)\mathrm{\,d}x-\displaystyle\int\limits_{-2}^5\mathrm{\,d}x\\
% 				&&=\displaystyle\int\limits_{-2}^5f(x)\mathrm{\,d}x+4\displaystyle\int\limits_5^{-2}g(x)\mathrm{\,d}x-\displaystyle\int\limits_{-2}^5\mathrm{\,d}x\\
% 				&&=8+4\cdot 3-x\bigg|_{-2}^5=8+4\cdot 3-7=13.
% 			\end{eqnarray*}
% 		\end{itemchoice}
% 	}
% \end{ex}
% \begin{ex}%[Câu 30]%[2D4V2-1]
% 	Cho hàm số $f(x),g(x)$ liên tục trên $\mathbb{R}$.
% 	\choiceTF
% 	{\True Biết $\displaystyle\int\limits_1^2f(x)\mathrm{\,d}x=2$. Giá trị của  $\displaystyle\int\limits_2^13f(x)\mathrm{\,d}x=-6$}
% 	{Biết $\displaystyle\int\limits_1^2f(x)\mathrm{\,d}x=-1$ và $\displaystyle\int\limits_1^2g(x)\mathrm{\,d}x=3$, khi đó $\displaystyle\int\limits_2^1\left[f(x)-g(x)\right]\mathrm{\,d}x=5$}
% 	{\True Nếu $\displaystyle\int\limits_1^2f(x)\mathrm{\,d}x=-2$ và $\displaystyle\int\limits_2^3f(x)\mathrm{\,d}x=1$ thì $\displaystyle\int\limits_1^3f(x)\mathrm{\,d}x=-1$}
% 	{\True Nếu $\displaystyle\int\limits_0^2(f(x)+3x^2)\mathrm{\,d}x=10$ thì $\displaystyle\int\limits_0^2f(x)\mathrm{\,d}x=2$}
% 	\loigiai{
% 		\begin{itemchoice}
% 			\itemch Đúng. Biết $\displaystyle\int\limits_1^2f(x)\mathrm{\,d}x=2$. Giá trị của $\displaystyle\int\limits_2^13f(x)\mathrm{\,d}x=-6$.\\
% 			Ta có $\displaystyle\int\limits_2^13f(x)\mathrm{\,d}x=-\displaystyle\int\limits_1^23f(x)\mathrm{\,d}x=-3\displaystyle\int\limits_1^2f(x)\mathrm{\,d}x=-3\cdot 2=-6$.
% 			\itemch Sai. Biết $\displaystyle\int\limits_1^2f(x)\mathrm{\,d}x=-1$ và $\displaystyle\int\limits_1^2g(x)\mathrm{\,d}x=3$.\\
% 			Ta có $\displaystyle\int\limits_1^2f(x)\mathrm{\,d}x=-1\Leftrightarrow \displaystyle\int\limits_2^1f(x)\mathrm{\,d}x=1$ và $\displaystyle\int\limits_1^2g(x)\mathrm{\,d}x=3\Leftrightarrow \displaystyle\int\limits_2^1g(x)\mathrm{\,d}x=-3$.\\
% 			Do vậy,  $\displaystyle\int\limits_2^1\left[f(x)-g(x)\right]\mathrm{\,d}x=\displaystyle\int\limits_2^1f(x)\mathrm{\,d}x-\displaystyle\int\limits_2^1g(x)\mathrm{\,d}x=1-(-3)=4$.
% 			\itemch Đúng. Nếu $\displaystyle\int\limits_1^2f(x)\mathrm{\,d}x=-2$ và $\displaystyle\int\limits_2^3f(x)\mathrm{\,d}x=1$ thì $\displaystyle\int\limits_1^3f(x)\mathrm{\,d}x=-1$.\\
% 			Ta có $\displaystyle\int\limits_1^3f(x)\mathrm{\,d}x=\displaystyle\int\limits_1^2f(x)\mathrm{\,d}x+\displaystyle\int\limits_2^3f(x)\mathrm{\,d}x=-2+1=-1$.
% 			\itemch Đúng. Ta có
% 			\begin{eqnarray*}
% 				&&\displaystyle\int\limits_0^2(f(x)+3x^2)\mathrm{\,d}x=10\\
% 				&\Leftrightarrow&\displaystyle\int\limits_0^2f(x)\mathrm{\,d}x+\displaystyle\int\limits_0^23x^2\mathrm{\,d}x=10\\
% 				&\Leftrightarrow&\displaystyle\int\limits_0^2f(x)\mathrm{\,d}x=10-\displaystyle\int\limits_0^23x^2\mathrm{\,d}x\\
% 				&\Leftrightarrow&\displaystyle\int\limits_0^2f(x)\mathrm{\,d}x=10-x^3\bigg|_0^2\\
% 				&\Leftrightarrow&\displaystyle\int\limits_0^2f(x)\mathrm{\,d}x=10-8=2.	\end{eqnarray*}
% 		\end{itemchoice}
% 	}
% \end{ex}
\Closesolutionfile{ans}
% \indapan{3}{ans/ans-2C4B2CD3-DS}
\Opensolutionfile{ans}[ans/ans-2-C4B2CD3-KQ]
\TNSA

\begin{ex}%[2D4H2-1]
	Cho $\displaystyle\int\limits_0^3f(x)\mathrm{\,d}x=4$. Tính $I=\displaystyle\int\limits_0^33f(x)\mathrm{\,d}x$.\\
	\shortans{$12$}
	\loigiai{
		Ta có $\displaystyle\displaystyle\int\limits_0^3 3 f(x){d}x=3\displaystyle\displaystyle\int\limits_0^3 f(x){d}x=12$.}
\end{ex}
\begin{ex}%[2D4H2-1]
	Cho $\displaystyle\int\limits_1^3f(x)\mathrm{\,d}x=2$. Tính $I=\displaystyle\int\limits_1^3\left[f(x)+2x\right]\mathrm{\,d}x$.\\
	\shortans{$10$}
	\loigiai{
		Ta có$\colon $ $\displaystyle\int\limits_1^3\left[f(x)+2x\right]\mathrm{\,d}x=\displaystyle\int\limits_1^3f(x)\mathrm{\,d}x+\displaystyle\int\limits_1^32x\mathrm{\,d}x=2+\left.x^2\right|_1^3=2+3^2-1^2=10$.}
\end{ex}

\begin{ex}%[2D4H2-1]
	Cho $\displaystyle\int\limits_{-1}^2f(x)\mathrm{\,d}x=2$ và $\displaystyle\int\limits_{-1}^2g(x)\mathrm{\,d}x=-1$. Tính $ I=\displaystyle\int\limits_{-1}^2\left[x+2f(x)+3g(x)\right]\mathrm{\,d}x$.\\
	\shortans{$2{,}5$}
	\loigiai{
		Ta có $\displaystyle\int\limits_{-1}^2\left[x+2f(x)+3g(x)\right]\mathrm{\,d}x=\displaystyle\int\limits_{-1}^2x\mathrm{\,d}x+2\displaystyle\int\limits_{-1}^2f(x)\mathrm{\,d}x+3\displaystyle\int\limits_{-1}^2g(x)\mathrm{\,d}x=\dfrac{3}{2}+4-3=\dfrac{5}{2}=2{.}5$.}
\end{ex}

\begin{ex}%[2D4H2-1]
	Cho $\displaystyle\int\limits_0^1f(x)\mathrm{\,d}x=1$. Tính tích phân $ I=\displaystyle\int\limits_0^1\left[2f(x)-3x^2\right]\mathrm{\,d}x.$\\
	\shortans{$1$}
	\loigiai{
		$\displaystyle\int\limits_0^1\left[2f(x)-3x^2\right]\mathrm{\,d}x=2\displaystyle\int\limits_0^1f(x)\mathrm{\,d}x-3\displaystyle\int\limits_0^1x^2\mathrm{\,d}x=2-1=1$.}
\end{ex}

\begin{ex}%[2D4H2-1]
	Biết $\displaystyle\int\limits_1^3f(x)\mathrm{\,d}x=3$. Tính giá trị của $ I=\displaystyle\int\limits_3^12f(x)\mathrm{\,d}x$.\\
	\shortans{$-6$}
	\loigiai{
		Ta có $\displaystyle\int\limits_3^12f(x)\mathrm{\,d}x=-\displaystyle\int\limits_1^32f(x)\mathrm{\,d}x=-2\displaystyle\int\limits_1^3f(x)\mathrm{\,d}x=-2\cdot3=-6$.}
\end{ex}

% \begin{ex}%[2D4H2-1]
% 	Biết $\displaystyle\int\limits_0^1f(x)\mathrm{\,d}x=-2$ và $\displaystyle\int\limits_1^0g(x)\mathrm{\,d}x=-3$.. Tính $ I=\displaystyle\int\limits_0^1\left[f(x)-g(x)\right]\mathrm{\,d}x$.\\
% 	\shortans{$-5$}
% 	\loigiai{
% 		$\displaystyle\int\limits_0^1\left[f(x)-g(x)\right]\mathrm{\,d}x=\displaystyle\int\limits_0^1f(x)\mathrm{\,d}x-\displaystyle\int\limits_0^1g(x)\mathrm{\,d}x=-2-3=-5$.}
% \end{ex}

% \begin{ex}%[2D4H2-1]
% 	Biết $\displaystyle\int\limits_1^2f(x)\,\mathrm{\,d}x=3$ và $\displaystyle\int\limits_1^2g(x)\mathrm{\,d}x=2$ và $\displaystyle\int\limits_1^2h(x)\mathrm{\,d}x=2022$. Tính $\linebreak I=\displaystyle\int\limits_1^2\left[f(x)-g(x)+h(x)\right]\mathrm{\,d}x$.\\
% 	\shortans{$2023$}
% 	\loigiai{
% 		Ta có $\displaystyle\int\limits_1^2\left[f(x)-g(x)+h(x)\right]\,\mathrm{\,d}x=\displaystyle\int\limits_1^2f(x)\,\mathrm{\,d}x-\displaystyle\int\limits_1^2g(x)\mathrm{\,d}x+\displaystyle\int\limits_1^2h(x)\mathrm{\,d}x$\\
% 		$=3-2+2022=2023$.}
% \end{ex}

% \begin{ex}%[2D4H2-1]
% 	Cho $\displaystyle\int\limits_{-1}^2f(x)\mathrm{\,d}x=2$ và $\displaystyle\int\limits_2^5f(x)\mathrm{\,d}x=-5$. Tính $ I=\displaystyle\int\limits_{-1}^5f(x)\mathrm{\,d}x$.\\
% 	\shortans{$-3$}
% 	\loigiai{
% 		Ta có $\displaystyle\int\limits_{-1}^5f(x)\mathrm{\,d}x=\displaystyle\int\limits_{-1}^2f(x)\mathrm{\,d}x+\displaystyle\int\limits_2^5f(x)\mathrm{\,d}x=2-5=-3$.}
% \end{ex}

% \begin{ex}%[2D4H2-1]
% 	Cho $ f$, $ g$ là hai hàm liên tục trên đoạn $\left[1;\,3\right]$ thoả$\colon $ $\displaystyle\int\limits_1^3\left[f(x)+3g(x)\right]\mathrm{\,d}x=10$, $\displaystyle\int\limits_1^3\left[2f(x)-g(x)\right]\mathrm{\,d}x=6$. Tính $I=\displaystyle\int\limits_1^3\left[f(x)+g(x)\right]\mathrm{\,d}x$.\\
% 	\shortans{$6$}
% 	\loigiai{
% 		Đặt $ a=\displaystyle\int\limits_1^3f(x)\mathrm{\,d}x$ và $ b=\displaystyle\int\limits_1^3g(x)\mathrm{\,d}x$.\\
% 		Khi đó, $\displaystyle\int\limits_1^3\left[f(x)+3g(x)\right]\mathrm{\,d}x=a+3b$, $\displaystyle\int\limits_1^3\left[2f(x)-g(x)\right]\mathrm{\,d}x=2a-b$.\\
% 		Theo giả thiết, ta có $\heva{
% 			& a+3b=10\\ 
% 			& 2a-b=6\\ 
% 		}\Leftrightarrow\heva{
% 			& a=4\\ 
% 			& b=2.\\ 
% 		}$\\
% 		Vậy $ I=a+b=6$.}
% \end{ex}

% \begin{ex}%[2D4H2-1]
% 	Cho hàm số $ f(x)$ liên tục trên $\mathbb{R}$ thoả mãn $\displaystyle\int\limits_1^8f(x)\,\mathrm{\,d}x=9$, $\displaystyle\int\limits_4^{12}{f(x)}\,\mathrm{\,d}x=3$, $\displaystyle\int\limits_4^8f(x)\,\mathrm{\,d}x=5$. Tính $ I=\displaystyle\int\limits_1^{12}{f(x)}\,\mathrm{\,d}x$.\\
% 	\shortans{$7$}
% 	\loigiai{
% 		Ta có $ I=\displaystyle\int\limits_1^{12}{f(x)}\,\mathrm{\,d}x=\displaystyle\int\limits_1^8f(x)\,\mathrm{\,d}x+\displaystyle\int\limits_8^{12}{f(x)}\,\mathrm{\,d}x$ $=\displaystyle\int\limits_1^8f(x)\,\mathrm{\,d}x+\displaystyle\int\limits_4^{12}{f(x)}\,\mathrm{\,d}x-\displaystyle\int\limits_4^8f(x)\,\mathrm{\,d}x$\\$=9+3-5=7$.}
% \end{ex}

% \begin{ex}%[2D4H2-1]
% 	Cho hàm số $ f(x)$ liên tục trên $\left[0;10\right]$ thỏa mãn $\displaystyle\int\limits_0^{10}{f(x)\mathrm{\,d}x}=7$, $\displaystyle\int\limits_2^6f(x)\mathrm{\,d}x=3$. Tính $ P=\displaystyle\int\limits_0^2f(x)\mathrm{\,d}x+\displaystyle\int\limits_6^{10}{f(x)\mathrm{\,d}x}$.\\
% 	\shortans{$4$}
% 	\loigiai{
% 		Ta có $\displaystyle\int\limits_0^{10}{f(x)\mathrm{\,d}x}=\displaystyle\int\limits_0^2f(x)\mathrm{\,d}x+\displaystyle\int\limits_2^6f(x)\mathrm{\,d}x+\displaystyle\int\limits_6^{10}{f(x)\mathrm{\,d}x}$\\
% 		Suy ra $\displaystyle\int\limits_0^2f(x)\mathrm{\,d}x+\displaystyle\int\limits_6^{10}{f(x)\mathrm{\,d}x}=\displaystyle\int\limits_0^{10}{f(x)\mathrm{\,d}x}-\displaystyle\int\limits_2^6f(x)\mathrm{\,d}x=7-3=4$.}
% \end{ex}

% \begin{ex}%[2D4H2-1]
% 	Giả sử $\displaystyle\int\limits_0^1f(x)\mathrm{\,d}x=3$ và $\displaystyle\int\limits_0^5f(z)\mathrm{\,d}z=9$. Tổng $I=\displaystyle\int\limits_1^3f(t)\mathrm{\,d}t+\displaystyle\int\limits_3^5f(t)\mathrm{\,d}t$ bằng\\
% 	\shortans{$6$}
% 	\loigiai{
% 		$\displaystyle\int\limits_0^1f(x)\mathrm{\,d}x=3\Leftrightarrow\displaystyle\int\limits_0^1f(t)\mathrm{\,d}t=3\Leftrightarrow\displaystyle\int\limits_1^0f(t)\mathrm{\,d}t=-3.$\\
% 		$\displaystyle\int\limits_0^5f(z)\mathrm{\,d}z=9\Leftrightarrow\displaystyle\int\limits_0^5f(t)\mathrm{\,d}t=9.$\\
% 		$\Rightarrow\displaystyle\int\limits_1^0f(t)\mathrm{\,d}t+\displaystyle\int\limits_0^5f(t)\mathrm{\,d}t=6\Leftrightarrow\displaystyle\int\limits_1^5f(t)\mathrm{\,d}t=6.$\\
% 		$I=\displaystyle\int\limits_1^3f(t)\mathrm{\,d}t+\displaystyle\int\limits_3^5f(t)\mathrm{\,d}t=\displaystyle\int\limits_1^5f(t)\mathrm{\,d}t=6.$}
% \end{ex}
\Closesolutionfile{ans}
% \indapan{6}{ans/ans-2-C4B2CD3-KQ}
\begin{dang}{Tích phân hàm số sơ cấp}	
\end{dang}
\TN
\Opensolutionfile{ans}[ans/ans-C4B2CD1]
\begin{ex}%[2D4N2-2]%Câu 1
	Tích phân $ I=\displaystyle\int\limits_0^2(2x+1)\mathrm{\,d}x$ bằng
	\choice
	{$ I=5$}
	{\True $ I=6$}
	{$ I=2$}
	{$ I=4$}
	\loigiai{
		Ta có $ I=\displaystyle\int\limits_0^2(2x+1)\mathrm{\,d}x=\left(x^2+x\right)\big|_0^2=4+2=6$.}
\end{ex}
%
\begin{ex}%[2D4H2-2]%Câu 2
	Tích phân $\displaystyle\int\limits_0^1\left(3x+1\right)\left(x+3\right)\mathrm{\,d}x$ bằng
	\choice
	{$ 12$}
	{\True $ 9$}
	{$ 5$}
	{$ 6$}
	\loigiai{
		Ta có $\displaystyle\int\limits_0^1\left(3x+1\right)\left(x+3\right)\mathrm{\,d}x=\displaystyle\int\limits_0^1\left(3x^2+10x+3\right)\mathrm{\,d}x=\left(x^3+5x^2+3x\right)\big|_0^1=9$.\\
		Vậy $\displaystyle\int\limits_0^1\left(3x+1\right)\left(x+3\right)\mathrm{\,d}x=9$.}
\end{ex}
%
\begin{ex}%[2D4N2-2]%Câu 3
	Tính tích phân $ I=\displaystyle\int\limits_1^\mathrm{e}{\left(\dfrac{1}{x}-\dfrac{1}{x^2}\right)}\mathrm{\,d}x$
	\choice
	{\True $I=\dfrac{1}{\mathrm{e}}$}
	{$I=\dfrac{1}{\mathrm{e}}+1$}
	{$I=1$}
	{$I=\mathrm{e}$}
	\loigiai{
		$ I=\displaystyle\int\limits_1^\mathrm{e}{\left(\dfrac{1}{x}-\dfrac{1}{x^2}\right)}\mathrm{\,d}x=\left(\ln \left| x\right|+\dfrac{1}{x}\right)\Big|_1^\mathrm{e}=\dfrac{1}{\mathrm{e}}$.}
\end{ex}

\begin{ex}%[2D4N2-2]%Câu 4
	Biết $\displaystyle\int\limits_1^3\dfrac{x+2}{x}\mathrm{\,d}x=a+b\ln c,$ với $a$, $b$, $c\in\mathbb{Z}$, $c<9.$ Tính tổng $S=a+b+c.$
	\choice
	{\True $ S=7$}
	{$ S=5$}
	{$ S=8$}
	{$ S=6$}
	\loigiai{
		Ta có $\displaystyle\int\limits_1^3\dfrac{x+2}{x}\mathrm{\,d}x=\displaystyle\int\limits_1^3\left(1+\dfrac{2}{x}\right)\mathrm{\,d}x=\displaystyle\int\limits_1^3\mathrm{d}x+\displaystyle\int\limits_1^3\dfrac{2}{x}\mathrm{d}x=2+2\ln \left| x\right|\big|_1^3=2+2\ln 3.$\\
		Do đó $ a=2$, $b=2$, $c=3\Rightarrow S=7.$}
\end{ex}
%
\begin{ex}%[2D4H2-4]%Câu 5
	Tích phân $\displaystyle\int\limits_0^1\mathrm{e}^{3x+1}\mathrm{\,d}x$ bằng
	\choice
	{$\dfrac{1}{3}\left(\mathrm{e}^4+\mathrm{e}\right)$}
	{$\mathrm{e}^3-\mathrm{e}$}
	{\True $\dfrac{1}{3}\left(\mathrm{e}^4-\mathrm{e}\right)$}
	{$\mathrm{e}^4-\mathrm{e}$}
	\loigiai{
		$\displaystyle\int\limits_0^1\mathrm{e}^{3x+1}\mathrm{\,d}x=\dfrac{1}{3}\displaystyle\int\limits_0^1\mathrm{e}^{3x+1}\mathrm{\,d}\left(3x+1\right)=\dfrac{1}{3}{\mathrm{e}^{3x+1}}\big|_0^1=\dfrac{1}{3}\left(\mathrm{e}^4-\mathrm{e}\right)$.}
\end{ex}

\begin{ex}%[2D4H2-4]%Câu 6
	Biết $\displaystyle\int\limits_0^1\dfrac{\mathrm{e}^x}{2^x}\mathrm{\,d}x=\dfrac{\mathrm{e-1}}{a-\ln b }$, $\left(a,b\in\mathbb{Z}\right)$. Khi đó giá trị của $ P=a+b$ là
	\choice
	{$ P=-3$}
	{\True $ P=6$}
	{$ P=-1$}
	{$ P=3$}
	\loigiai{
		$ I=\displaystyle\int\limits_0^1\dfrac{\mathrm{e}^x}{2^x}\mathrm{\,d}x=\displaystyle\int\limits_0^1\left(\dfrac{\mathrm{e}}{2}\right)^x\mathrm{\,d}x=\left[\left(\dfrac{\mathrm{e}}{2}\right)^x\cdot\dfrac{1}{1-\ln 2}\right]\Big|_0^1=\dfrac{\mathrm{e}-1}{2-\ln 4}$.}
\end{ex}

\begin{ex}%[2D4H2-4]%Câu 7
	Giá trị của $ I=\displaystyle\int\limits_0^1\dfrac{\mathrm{e}^{2x}-4}{\mathrm{e}^x+2}\mathrm{\,d}x$ bằng
	\choice
	{$ I=2\left(\mathrm{e}+3\right)$}
	{$ I=\dfrac{1}{2}\left(\mathrm{e}+3\right)$}
	{\True $ I=\mathrm{e}-3$}
	{$ I=2\left(\mathrm{e}-3\right)$}
	\loigiai{
		$ I=\displaystyle\int\limits_0^1\dfrac{\mathrm{e}^{2x}-4}{\mathrm{e}^x+2}\mathrm{\,d}x=\displaystyle\int\limits_0^1\dfrac{\left(\mathrm{e}^x-2\right)\left(\mathrm{e}^x+2\right)}{\mathrm{e}^x+2}\mathrm{\,d}x=\displaystyle\int\limits_0^1\left(\mathrm{e}^x-2\right)\mathrm{\,d}x=\left(\mathrm{e}^x-2x\right)\big|_0^1=e-3$.}
\end{ex}
%
\begin{ex}%[2D4H2-4]%Câu 8
	Biết $\displaystyle\int\limits_1^2\mathrm{e}^x\left(1-\dfrac{\mathrm{e}^{-x}}{x}\right)\mathrm{d}x=\mathrm{e}^2+a\cdot \mathrm{e}+b\ln 2$, $\left(a,b\in\mathbb{Z}\right)$. Khi đó giá trị của $ P=\dfrac{a+b}{a\cdot b}$ là
	\choice
	{$ P=-3$}
	{$ P=1$}
	{$ P=-1$}
	{\True $ P=-2$}
	\loigiai{
		$ I=\displaystyle\int\limits_1^2\mathrm{e}^x\left(1-\dfrac{\mathrm{e}^{-x}}{x}\right)\mathrm{\,d}x=\displaystyle\int\limits_1^2\left(\mathrm{e}^x-\dfrac{1}{x}\right)\mathrm{\,d}x=\left(\mathrm{e}^x-\ln \left| x\right|\right)\big|_1^2=\mathrm{e}^2-\mathrm{e}-\ln 2$.}
\end{ex}

\begin{ex}%[2D4H2-4]%Câu 9
	Biết $ I=\displaystyle\int\limits_0^1\dfrac{\mathrm{e}^{2x-1}-\mathrm{e}^{-3x}+1}{\mathrm{e}^x}\mathrm{\,d}x=\dfrac{1}{a}+b$, $\left(a,b\in\mathbb{R}\right)$. Khi đó giá trị của $ P=\dfrac{a+b}{a\cdot b}$ là
	\choice
	{$ P=\mathrm{e}^4-1$}
	{$ P=\dfrac{\mathrm{e}^4-1}{\mathrm{e}^2}$}
	{$ P=\dfrac{\mathrm{e}^4-1}{\mathrm{e}^4}$}
	{\True $ P=\dfrac{1-\mathrm{e}^4}{\mathrm{e}^4}$}
	\loigiai{
		\allowdisplaybreaks
		\begin{eqnarray*} I&=&\displaystyle\int\limits_0^1\dfrac{\mathrm{e}^{2x-1}-\mathrm{e}^{-3x}+1}{\mathrm{e}^x}\mathrm{\,d}x=\displaystyle\int\limits_0^1\left(\mathrm{e}^{x-1}-\mathrm{e}^{-4x}+\mathrm{e}^{-x}\right)\mathrm{\,d}x\\
			&=&\left(\mathrm{e}^{x-1}-\dfrac{\mathrm{e}^{-4x}}{-4}+\dfrac{\mathrm{e}^{-x}}{-1}\right)\Big|_0^1=\dfrac{1-\mathrm{e}^4}{\mathrm{e}^4}=\dfrac{1}{\mathrm{e}^4}-1
		\end{eqnarray*}
		$\Rightarrow P=\dfrac{a+b}{a\cdot b}=\dfrac{1-\mathrm{e}^4}{\mathrm{e}^4}$.}
\end{ex}
%
\begin{ex}%[2D4N2-3]%Câu 10
	Giá trị của $\displaystyle\int\limits_0^{\frac{\pi}{2}}{\sin x\mathrm{\,d}x}$ bằng
	\choice
	{0}
	{\True 1}
	{$-1$}
	{$\dfrac{\pi}{2}$}
	\loigiai{
		Tính được $\displaystyle\int\limits_0^{\frac{\pi}{2}}{\sin x\mathrm{\,d}x}=-\cos x\Big|_0^{\frac{\pi}{2}}=1$.}
\end{ex}

\begin{ex}%[2D4H2-3]%Câu 11
	Biết $\displaystyle\int\limits_{\tfrac{\pi}{3}}^{\tfrac{\pi}{2}}{\left(2\sin x+3\cos x+x\right)\mathrm{\,d}x}=\dfrac{a+b\sqrt{3}}{2}+\dfrac{\pi^2}{c}$, $\left(a,b,c\in\mathbb{Z}\right)$. Khi đó giá trị của $ P=a+2b+3c$ là
	\choice
	{$ P=45$}
	{\True $ P=60$}
	{$ P=65$}
	{$ P=70$}
	\loigiai{
		$\displaystyle\int\limits_{\tfrac{\pi}{3}}^{\tfrac{\pi}{2}}\left(2\sin x+3\cos x+x\right)\mathrm{\,d}x=\left(-2\cos x+3\sin x+\dfrac{1}{2}{x^2}\right)\Big|_{\tfrac{\pi}{3}}^{\tfrac{\pi}{2}}=\dfrac{12-3\sqrt{3}}{2}+\dfrac{\pi^2}{18}$\\
		$\Rightarrow P=a+2b+3c=60$.
	}
\end{ex}

\begin{ex}%[2D4H2-3]%Câu 12
	Biết $\displaystyle\int\limits_{\tfrac{\pi}{4}}^{\tfrac{\pi}{3}}{3\tan^2x\mathrm{\,d}x}=a\sqrt{3}+b+\dfrac{\pi}{c}$, $\left(a,b,c\in\mathbb{Z}\right)$. Khi đó giá trị của $ P=a+b+c$ là
	\choice
	{$ P=6$}
	{\True $ P=-4$}
	{$ P=4$}
	{$ P=-6$}
	\loigiai{
		$\displaystyle\int\limits_{\tfrac{\pi}{4}}^{\tfrac{\pi}{3}}{3\tan^2x\mathrm{\,d}x}=3\displaystyle\int\limits_{\tfrac{\pi}{4}}^{\tfrac{\pi}{3}}{\left(\dfrac{1}{\cos^2x}-1\right)\mathrm{\,d}x= 3\left(\tan x-x\right)\big|_{\tfrac{\pi}{4}}^{\tfrac{\pi}{3}}=3\sqrt{3}-3-\dfrac{\pi}{4}}$\\
		$\Rightarrow P=a+b+c=3-3-4=-4$.}
\end{ex}
%
\begin{ex}%[2D4H2-3]%Câu 13
	Biết $\displaystyle\int\limits_{\tfrac{\pi}{6}}^{\tfrac{\pi}{4}}{\left(2\cot^2x+5\right)\mathrm{\,d}x}=\dfrac{\pi}{a}+b\sqrt{3}+c$, $\left(a,b,c\in\mathbb{Z}\right)$. Khi đó giá trị của \break $ P=a+b+c$ là
	\choice
	{$ P=6$}
	{$ P=-4$}
	{\True $ P=4$}
	{$ P=-6$}
	\loigiai{\allowdisplaybreaks
		\begin{eqnarray*}
			\displaystyle\int\limits_{\tfrac{\pi}{6}}^{\tfrac{\pi}{4}}{\left(2\cot^2x+5\right)\mathrm{\,d}x}&=&\displaystyle\int\limits_{\tfrac{\pi}{6}}^{\tfrac{\pi}{4}}{\left(2\left(\dfrac{1}{\sin^2x}-1\right)+5\right)\mathrm{\,d}x}\\
			&=&\displaystyle\int\limits_{\dfrac{\pi}{6}}^{\dfrac{\pi}{4}}{\left(3-\dfrac{-2}{\sin^2x}\right)\mathrm{\,d}x=\left(3x-\cot x\right)\Big|_{\tfrac{\pi}{6}}^{\tfrac{\pi}{4}}=\dfrac{\pi}{4}+\sqrt{3}-1}.
	\end{eqnarray*}}
\end{ex}

\begin{ex}%[2D4H2-3]%Câu 14
	Biết $\displaystyle\int\limits_0^{\tfrac{\pi}{2}}\sin^2\dfrac{x}{4}{\cos^2}\dfrac{x}{4}\mathrm{\,d}x=\dfrac{\pi}{c}+\dfrac{a}{b}$ với $a$, $b\in\mathbb{Z}$ và $\dfrac{a}{b}$ là phân số tối giản. Khi đó giá trị của $ P=a+b+c$ là
	\choice
	{$ P=17$}
	{$ P=16$}
	{$ P=32$}
	{\True $ P=49$}
	\loigiai{\allowdisplaybreaks
		\begin{eqnarray*}
			\displaystyle\int\limits_0^{\tfrac{\pi}{2}}{\sin^2\dfrac{x}{4}{\cos^2}\dfrac{x}{4}\mathrm{\,d}x}&=&\dfrac{1}{4}\displaystyle\int\limits_0^{\tfrac{\pi}{2}}\sin^2\dfrac{x}{2}\mathrm{\,d}x\\
			&=&\dfrac{1}{4}\displaystyle\int\limits_0^{\tfrac{\pi}{2}}\left(\dfrac{1-\cos x}{2}\right)\mathrm{\,d}x\\
			&=&\dfrac{1}{8}\left(x-\dfrac{1}{4}\sin x\right)\Big|_0^{\tfrac{\pi}{2}}=\dfrac{\pi}{16}+\dfrac{1}{32}.
		\end{eqnarray*}
		$\Rightarrow P=a+b+c=1+32+16=49$.}
\end{ex}
\Closesolutionfile{ans}
% \indapan{6}{ans/ans-C4B2CD1}
\TNTF
\Opensolutionfile{ans}[ans/ans-C4B2CD1-DS]
\begin{ex}%[2D4H2-1]%Câu 15
	Cho hàm số $y=f(x)$ liên tục trên $\left[a;b\right]$. Các mệnh đề sau đây đúng hay sai?
	\choiceTF
	{$\displaystyle\int\limits_a^b{f(x)\mathrm{\,d}x}=\displaystyle\int\limits_b^a{f(x)\mathrm{\,d}x}$}
	{\True $\displaystyle\int\limits_a^b{f(x)\mathrm{\,d}x}=-\displaystyle\int\limits_b^a{f(x)\mathrm{\,d}x}$}
	{$\displaystyle\int\limits_a^bf(x)\mathrm{\,d}x=2\displaystyle\int\limits_a^bf(x)\mathrm{\,d}\left(2x\right)$}
	{\True $\displaystyle\int\limits_a^a{2024f(x)\mathrm{\,d}x=0}$}
	\loigiai{
		\begin{itemchoice}
			\itemch Sai. Vì
			$\displaystyle\int\limits_a^b{f(x)\mathrm{\,d}x}=-\displaystyle\int\limits_b^a{f(x)\mathrm{\,d}x}$.
			\itemch Đúng. Vì $\displaystyle\int\limits_a^b{f(x)\mathrm{\,d}x}=-\displaystyle\int\limits_b^a{f(x)\mathrm{\,d}x}$.
			\itemch Sai. Vì $2\displaystyle\int\limits_a^bf(x)\mathrm{\,d}\left(2x\right)=4\displaystyle\int\limits_a^bf(x)\mathrm{\,d}\left(x\right)$.
			\itemch Đúng. 
			$\displaystyle\int\limits_a^a2024f(x)\mathrm{\,d}x=0.$
		\end{itemchoice}
	}
\end{ex}
%
\begin{ex}%[2D4H2-1]%Câu 16
	Cho hàm số $y=f(x)$, $y=g(x)$ liên tục trên $\left[a;b\right]$. Các mệnh đề sau đây đúng hay sai?
	\choiceTF
	{\True $\displaystyle\int\limits_a^b{\left[f(x)+g(x)\right]\mathrm{\,d}x}=\displaystyle\int\limits_a^b{f(x)}\mathrm{\,d}x+\displaystyle\int\limits_a^b{g(x)\mathrm{\,d}x}$}
	{$\displaystyle\int\limits_a^b{f(x)\cdot g(x)\mathrm{\,d}x}=\displaystyle\int\limits_a^b{f(x)\mathrm{\,d}x}\cdot\displaystyle\int\limits_a^b{g(x)\mathrm{\,d}x}$}
	{\True $\displaystyle\int\limits_a^b{kf(x)\mathrm{\,d}x=k\displaystyle\int\limits_a^b{f(x)\mathrm{\,d}x}}$}
	{$\displaystyle\int\limits_a^b{\dfrac{f(x)}{g(x)}\mathrm{\,d}x}=\dfrac{\displaystyle\int\limits_a^bf(x)\mathrm{\,d}x}{\displaystyle\int\limits_a^bg(x)\mathrm{\,d}x}$}
	\loigiai{
		\begin{itemchoice}
			\itemch Đúng.
			$\displaystyle\int\limits_a^b{\left[f(x)+g(x)\right]\mathrm{\,d}x}=\displaystyle\int\limits_a^b{f(x)}\mathrm{\,d}x+\displaystyle\int\limits_a^b{g(x)\mathrm{\,d}x}$.
			\itemch Sai. Vì không có tính chất.
			\itemch Đúng.
			$\displaystyle\int\limits_a^b{kf(x)\mathrm{\,d}x=k\displaystyle\int\limits_a^b{f(x)\mathrm{\,d}x}}$.
			\itemch Sai.
	\end{itemchoice}}
\end{ex}
%
% \begin{ex}%[2D4H2-1]%Câu 17
% 	Cho hàm số $y=f(x)$ liên tục trên $\mathbb{R}$ và $a$, $b$, $c\in\mathbb{R}$ thỏa mãn $a<b<c$. Các mệnh đề sau đây đúng hay sai?
% 	\choiceTF
% 	{$\displaystyle\int\limits_a^c{f(x)\mathrm{\,d}x=\displaystyle\int\limits_a^b{f(x)\mathrm{\,d}x}}\cdot \displaystyle\int\limits_b^c{f(x)\mathrm{\,d}x}$}
% 	{\True $\displaystyle\int\limits_a^c{f(x)\mathrm{\,d}x=\displaystyle\int\limits_a^b{f(x)\mathrm{\,d}x}}+\displaystyle\int\limits_b^c{f(x)\mathrm{\,d}x}$}
% 	{$\displaystyle\int\limits_a^c{f(x)\mathrm{\,d}x=\displaystyle\int\limits_a^b{f(x)\mathrm{\,d}x}}-\displaystyle\int\limits_b^c{f(x)\mathrm{\,d}x}$}
% 	{$\displaystyle\int\limits_a^c{f(x)\mathrm{\,d}x=\displaystyle\int\limits_a^b{f(x)\mathrm{\,d}x}}+\displaystyle\int\limits_c^b{f(x)\mathrm{\,d}x}$}
% 	\loigiai{\begin{itemchoice}
% 			\itemch Sai. Không đúng với lý thuyết.
% 			\itemch Đúng. $\displaystyle\int\limits_a^c{f(x)\mathrm{\,d}x=\displaystyle\int\limits_a^b{f(x)\mathrm{\,d}x}}+\displaystyle\int\limits_b^c{f(x)\mathrm{\,d}x}$.
% 			\itemch Sai.
% 			\itemch Sai.
% 	\end{itemchoice}}
% \end{ex}
% %
% \begin{ex}%[2D4H2-1]%Câu 18
% 	Cho $f(x)$, $g(x)$ là hai hàm số liên tục trên $\mathbb{R}$. Các mệnh đề sau đây đúng hay sai?
% 	\choiceTF
% 	{\True $\displaystyle\int\limits_a^bf(x)\mathrm{\,d}x=\displaystyle\int\limits_a^bf(y)\mathrm{\,d}y$}
% 	{\True $\displaystyle\int\limits_a^b{\left(f(x)+g(x)\right)\mathrm{\,d}x}=\displaystyle\int\limits_a^b{f(x)\mathrm{\,d}x+\displaystyle\int\limits_a^b{g(x)\mathrm{\,d}x}}$}
% 	{$\displaystyle\int\limits_a^b{f(x)\mathrm{\,d}x=\displaystyle\int\limits_a^b{f(t)\mathrm{\,d}x}}$}
% 	{$\displaystyle\int\limits_a^b{\left(f(x)g(x)\right)\mathrm{\,d}x}=\displaystyle\int\limits_a^b{f(x)\mathrm{\,d}x\displaystyle\int\limits_a^b{g(x)\mathrm{\,d}x}}$}
% 	\loigiai{
% 		\begin{itemchoice}
% 			\itemch Đúng. $\displaystyle\int\limits_a^b{f(x)\mathrm{\,d}x=\displaystyle\int\limits_a^b{f(y)\mathrm{\,d}}y}$
% 			\itemch Đúng. $\displaystyle\int\limits_a^b{\left(f(x)+g(x)\right)\mathrm{\,d}x}=\displaystyle\int\limits_a^bf(x)\mathrm{\,d}x+\displaystyle\int\limits_a^b g(x)\mathrm{\,d}x$.
% 			\itemch Sai. Không đúng với lý thuyết.
% 			\itemch Sai. Không đúng với lý thuyết.
% 		\end{itemchoice}
% 	}
% \end{ex}

% \begin{ex}%[2D4H2-1]%Câu 19
% 	Các mệnh đề sau đây đúng hay sai?
% 	\choiceTF
% 	{\True $\displaystyle\int\limits_{-2024}^{2024}\mathrm{\,d}x=4048$}
% 	{$\displaystyle\int\limits_a^bf_1(x)\cdot f_2(x)\mathrm{\,d}x=\displaystyle\int\limits_a^bf_1(x)\mathrm{\,d}x\cdot\displaystyle\int\limits_a^bf_2(x)\mathrm{\,d}x$}
% 	{\True Cho hàm số $f(x)$ liên tục trên đoạn $\left[a;b\right]$. Khi đó $\dfrac{1}{b-a}\displaystyle\int\limits_a^bf(x)\mathrm{\,d}x$ được gọi là giá trị trung bình của hàm số $f(x)$ trên đoạn $\left[a;b\right]$}
% 	{\True Nếu hàm số $f(x)$ có đạo hàm $f'(x)$ và $f'(x)$ liên tục trên đoạn $\left[a;b\right]$ thì $f(b)-f(a)=\displaystyle\int\limits_a^bf'(x)\mathrm{\,d}x$}
% 	\loigiai{\begin{itemchoice}
% 			\itemch Đúng.
% 			\itemch Sai. 
% 			\itemch Đúng.
% 			\itemch Đúng.
% 		\end{itemchoice}
		
% 	}
% \end{ex}
%
\begin{ex}%[2D4H2-1]%Câu 20
	Cho hàm $ f(x)$ là hàm liên tục trên đoạn $\left[a;b\right]$ với $ a<b$ và $F(x)$ là một nguyên hàm của hàm $ f(x)$ trên $\left[a;b\right]$. Các mệnh đề sau đây đúng hay sai?
	\choiceTF
	{\True $\displaystyle\int\limits_a^b{kf(x)\mathrm{\,d}x}=k\left[F(b)-F(a)\right]$}
	{$\displaystyle\int\limits_b^af(x)\mathrm{\,d}x=F(b)-F(a)$}
	{Diện tích $S$ của hình phẳng giới hạn bởi đường thẳng $x=a$; $x=b$; đồ thị của hàm số $ y=f(x)$ và trục hoành được tính theo công thức $ S=F(b)-F(a)$}
	{$\displaystyle\int\limits_a^b{f\left(2x+3\right)\mathrm{\,d}x}=F\left(2x+3\right)\big|_a^b$}
	\loigiai{
		\begin{itemchoice}
			\itemch Đúng.
			\itemch Sai. $\displaystyle\int\limits_b^a{f(x)\mathrm{\,d}x}=F(a)-F(b)$. 
			\itemch Sai. Diện tích $S$ của hình phẳng giới hạn bởi đường thẳng $x=a$; $x=b$; đồ thị của hàm số $ y=f(x)$ và trục hoành được tính theo công thức $ S=|F(b)-F(a)|$.
			\itemch Sai. $\displaystyle\int\limits_a^bf\left(2x+3\right)\mathrm{\,d}x=\dfrac12 F\left(2x+3\right)\big|_a^b$
		\end{itemchoice}
	}
\end{ex}
%
\begin{ex}%[2D4H2-4]%Câu 21
	Các mệnh đề sau đây đúng hay sai.
	\choiceTF
	{\True $\displaystyle\int\limits_0^1\dfrac{\mathrm{e}^{2x}-4}{\mathrm{e}^x+2}\mathrm{\,d}x=\mathrm{e}-3$}
	{$\displaystyle\int\limits_0^1\dfrac{\mathrm{e}^x}{2^x}\mathrm{\,d}x=\dfrac{\mathrm{e}}{2}+1$}
	{\True $\displaystyle\int\limits_1^2\mathrm{e}^x\left(1-\dfrac{\mathrm{e}^{-x}}{x}\right)\mathrm{\,d}x=\mathrm{e}^2-\mathrm{e}-\ln 2$}
	{$\displaystyle\int\limits_0^1\dfrac{\mathrm{e}^{2x-1}-\mathrm{e}^{-3x}+1}{\mathrm{e}^x}\mathrm{\,d}x=\mathrm{e}^4-1$}
	\loigiai{\begin{itemchoice}
			\itemch Đúng. \allowdisplaybreaks
			\begin{eqnarray*} \displaystyle\int\limits_0^1\dfrac{\mathrm{e}^{2x}-4}{\mathrm{e}^x+2}\mathrm{\,d}x&=&\displaystyle\int\limits_0^1\dfrac{\left(\mathrm{e}^x-2\right)\left(\mathrm{e}^x+2\right)}{\mathrm{e}^x+2}\mathrm{\,d}x\\
				&=&\displaystyle\int\limits_0^1\left(\mathrm{e}^x-2\right)\mathrm{\,d}x=\left(\mathrm{e}^x-2x\right)\big|_0^1=\mathrm{e}-3.
			\end{eqnarray*}
			\itemch Sai.  $\displaystyle\int\limits_0^1\dfrac{\mathrm{e}^x}{2^x}\mathrm{\,d}x=\displaystyle\int\limits_0^1\left(\dfrac{\mathrm{e}}{2}\right)^x\mathrm{\,d}x=\left[\left(\dfrac{\mathrm{e}}{2}\right)^x\right]\Big|_0^1=\dfrac{\mathrm{e}}{2}-1$.
			\itemch Đúng. $\displaystyle\int\limits_1^2\mathrm{e}^x\left(1-\dfrac{\mathrm{e}^{-x}}{x}\right)\mathrm{\,d}x=\displaystyle\int\limits_1^2\left(\mathrm{e}^x-\dfrac{1}{x}\right)\mathrm{\,d}x=\left(\mathrm{e}^x-\ln \left| x\right|\right)\big|_1^2=\mathrm{e}^2-\mathrm{e}-\ln 2$.
			\itemch Sai.\allowdisplaybreaks
			\begin{eqnarray*} \displaystyle\int\limits_0^1\dfrac{\mathrm{e}^{2x-1}-\mathrm{e}^{-3x}+1}{\mathrm{e}^x}\mathrm{\,d}x&=&\displaystyle\int\limits_0^1\left(\mathrm{e}^{x-1}-\mathrm{e}^{-4x}+\mathrm{e}^{-x}\right)\mathrm{\,d}x\\
				&=&\left(\mathrm{e}^{x-1}-\mathrm{e}^{-4x}+\mathrm{e}^{-x}\right)\big|_0^1=\dfrac{1-\mathrm{e}^4}{\mathrm{e}^4}=\mathrm{e}^{-4}-1.
			\end{eqnarray*}
		\end{itemchoice}
	}
\end{ex}
\Closesolutionfile{ans}
% \indapan{3}{ans/ans-C4B2CD1-DS}
\TNSA
\Opensolutionfile{ans}[ans/ans-C4B2CD1-KQ]
\begin{ex}%[2D4H2-2]%Câu 22
	Với $a$, $b$ là các tham số thực. Tích phân $$I=\displaystyle\int\limits_0^b\left(3x^2-2ax-1\right)\mathrm{\,d}x=b^t-b^ya+zb.$$ Tính $t+y+z$.
	\shortans{$4$}
	\loigiai{
		Ta có $\displaystyle\int\limits_0^b{\left(3x^2-2ax-1\right)\mathrm{\,d}x}=\left(x^3-a{x^2}-x\right)\big|_0^b=b^3-a{b^2}-b$. \\
		Suy ra $t=3$, $y=2$, $z=-1$ nên $t+y+z=4$.}
\end{ex}

\begin{ex}%[2D4H2-2]%Câu 23
	Cho $\displaystyle\int\limits_0^m{\left(3x^2-2x+1\right)}\mathrm{\,d}x=6$. Tính giá trị của tham số $m$.
	\shortans{$2$}
	\loigiai{
		Ta có $\displaystyle\int\limits_0^m{\left(3x^2-2x+1\right)}\mathrm{\,d}x=6\Leftrightarrow\left.\left(x^3-x^2+x\right)\right|_0^m=6\Leftrightarrow{m^3}-m^2+m-6=0\Leftrightarrow m=2$.}
\end{ex}
%%%==============EX_1============%%%
\begin{ex}%[2D4H2-2]
	Tính tích phân $I=\displaystyle\int\limits\limits_1^2\dfrac{x-1}{x} \mathrm{d}x$ (\textit{\textit{làm tròn đến hàng phần trăm}}).
	\shortans{$0{,}31$}	
	\loigiai{
		\begin{eqnarray*}
			I	&= &\displaystyle\int\limits_1^2\dfrac{x-1}{x} \mathrm{d}x\\
			&=& \displaystyle\int\limits_1^2\left(1-\dfrac{1}{x} \right) \mathrm{d}x\\
			&= & \left(x-\ln |x|\right)\Bigg|_1^2\\
			&=&	\left(2-\ln 2\right)-\left(1-\ln 1\right)=1-\ln 2.
	\end{eqnarray*}}
\end{ex}
%%%==============EX_2============%%%
\begin{ex}%[2D4H2-2]
	Tính $I=\displaystyle\int\limits_1^2\left(\dfrac{x-\sqrt[{4}]{x^3}}{x} \right)^2 \mathrm{\,d}x$ (\textit{\textit{làm tròn đến hàng phần trăm}}).
	\shortans{$0{,}01$}	
	\loigiai{
		\begin{eqnarray*}
			I	&= &\displaystyle\int\limits_1^2\left(\dfrac{x-\sqrt[{4}]{x^3}}{x} \right)^2 \mathrm{\,d}x\\
			&=& \displaystyle\int\limits_1^2\left(1-x^{-\tfrac{1}{4}}\right)^2 \mathrm{\,d}x\\
			&= &\displaystyle\int\limits_1^2\left(1-2x^{-\tfrac{1}{4}}+x^{-\tfrac{1}{8}}\right) \mathrm{\,d}x\\
			&=&	\left(x-\dfrac{8}{3}x^{\tfrac{3}{4}}+\dfrac{8}{7}x^{\tfrac{7}{8}} \right)\Bigg|_1^2\\
			&\approx& 0{,}01. 
		\end{eqnarray*}
	}
\end{ex}
%%%==============EX_3============%%%
\begin{ex}%[2D4H2-2]
	Tính $I=\displaystyle\int\limits_1^2\left(\sqrt{x}+1\right)\left(\sqrt[{3}]{x}-1\right)\mathrm{\,d}x$ (\textit{\textit{làm tròn đến hàng phần trăm}}).
	\shortans{$0{,}32$}	
	\loigiai{
		\begin{eqnarray*}
			I&= &\displaystyle\int\limits_1^2\left(\sqrt{x}+1\right)\left(\sqrt[{3}]{x}-1\right)\mathrm{\,d}x\\
			&=& \displaystyle\int\limits_1^2\left(x^{\tfrac{5}{6}}-x^{\tfrac{1}{2}}+x^{\tfrac{1}{3}}-1\right) \mathrm{\,d}x\\
			&=&	\left(\dfrac{6}{11}x^{\tfrac{11}{6}}-\dfrac{2}{3}x^{\tfrac{3}{2}}+\dfrac{3}{4}x^{\tfrac{4}{3}}-x \right)\Bigg|_1^2\\
			&\approx& 0{,}32. 
		\end{eqnarray*}
	}
\end{ex}
%%%==============EX_4============%%%
\begin{ex}%[2D4H2-2]
	Tính $I=\displaystyle\int\limits_1^2\dfrac{(x^2+1)^3}{x^2} \mathrm{\,d}x$ (\textit{làm tròn đến hàng phần chục}).
	\shortans{$16{,}7$}	
	\loigiai{
		\begin{eqnarray*}
			I&= &\displaystyle\int\limits_1^2\dfrac{(x^2+1)^3}{x^2} \mathrm{\,d}x\\
			&=& \displaystyle\int\limits_1^2\left(x^4+3x^2+3+\dfrac{1}{x^2}\right) \mathrm{\,d}x\\
			&=&	\left(\dfrac{x^5}{5}+x^3+3x-\dfrac{1}{x}\right)\Bigg|_1^2\\
			&=& 16{,}7. 
		\end{eqnarray*}
	}
\end{ex}
%%%==============EX_5============%%%
\begin{ex}%[2D4H2-4]
	Tính $I=\displaystyle\int\limits _0^15^{x+1}\cdot7^{2x-1} \mathrm{\,d}x$ (\textit{làm tròn đến hàng đơn vị}).
	\shortans{$959$}	
	\loigiai{
		\begin{eqnarray*}
			I&= &\displaystyle\int\limits _0^15^{x+1}\cdot7^{2x-1} \mathrm{\,d}x\\
			&=&\dfrac{5}{7} \displaystyle\int\limits_0^15^x\cdot49^x \mathrm{\,d}x\\
			&=&	\dfrac{5}{7} \displaystyle\int\limits_0^1245^x \mathrm{\,d}x\\
			&=&	\dfrac{5}{7}\left(245^x\ln 245\right)\Bigg|_0^1\\
			&=&\dfrac{5}{7}\left(245\ln 245-\ln 245\right)\approx 959. 
		\end{eqnarray*}
	}
\end{ex}
%%%==============EX_6============%%%
\begin{ex}%[2D4H2-4]
	Tính $I=\displaystyle\int\limits _0^1\left(x+\mathrm{e}^{-x-2} \right)\mathrm{\,d}x$ (\textit{\textit{làm tròn đến hàng phần trăm}}).
	\shortans{$0{,}59$}	
	\loigiai{
		\begin{eqnarray*}
			I&= &\displaystyle\int\limits _0^1\left(x+\mathrm{e}^{-x-2} \right)\mathrm{\,d}x\\
			&=&	\left(\dfrac{x^2}{2}-\mathrm{e}^{-x-2}\right)\Bigg|_0^1\\
			&=&\left(\dfrac{1}{2}+\mathrm{e}^{-2}-\mathrm{e}^{-3}\right)\approx 0{,}59. 
		\end{eqnarray*}
	}
\end{ex}
%%%==============EX_7============%%%
\begin{ex}%[2D4H2-3]
	Tính $I=\displaystyle\int\limits _{\tfrac{\pi}{6}}^{\tfrac{\pi}{3}}x^2 \left(1-\dfrac{\sin x}{x^2} \right)\mathrm{\,d}x$ (\textit{\textit{làm tròn đến hàng phần trăm}}).
	\shortans{$-0{,}03$}	
	\loigiai{
		\begin{eqnarray*}
			I&= &\displaystyle\int\limits _{\tfrac{\pi}{6}}^{\tfrac{\pi}{3}}x^2 \left(1-\dfrac{\sin x}{x^2} \right)\mathrm{\,d}x\\
			&=&\displaystyle\int\limits _{\tfrac{\pi}{6}}^{\tfrac{\pi}{3}}\left(x^2-\sin x \right)\mathrm{\,d}x\\
			&=&\left(\dfrac{x^3}{3}+\cos x\right)\Bigg| _{\tfrac{\pi}{6}}^{\tfrac{\pi}{3}}\approx -0{,}03.  
		\end{eqnarray*}
	}
\end{ex}
%%%==============EX_8============%%%
\begin{ex}%[2D4H2-3]
	Tính $I=\displaystyle\int\limits _{\tfrac{\pi}{6}}^{\tfrac{\pi}{2}}\left(\sin x-\dfrac{1}{\sqrt[{3}]{x^2}} \right) \mathrm{\,d}x$ \textit{(\textit{làm tròn đến hàng phần trăm})}.
	\shortans{$0{,}38$}	
	\loigiai{
		\begin{eqnarray*}
			I&= &\displaystyle\int\limits _{\tfrac{\pi}{6}}^{\tfrac{\pi}{2}}\left(\sin x-\dfrac{1}{\sqrt[{3}]{x^2}} \right) \mathrm{\,d}x\\
			&=&\left(-\cos x-3\sqrt[{3}]{x}\right)\Bigg| _{\tfrac{\pi}{6}}^{\tfrac{\pi}{2}}\approx 0{,}38.  
		\end{eqnarray*}
	}
\end{ex}
%%%==============EX_9============%%%
\begin{ex}%[2D4H2-4]
	Biết $\displaystyle\int\limits _0^1\dfrac{\left(e^{-x}+2\right)^2}{e^{x-1}} \mathrm{\,d}x=ae+b+\dfrac{c}{e}+\dfrac{1}{e^2}$ $\left(a,b,c\in \mathbb{Z}\right)$. Tính giá trị của $P=a+b+c$.
	\shortans{$-1$}	
	\loigiai{
		\begin{eqnarray*}
			I	&=& \displaystyle\int\limits _0^1\dfrac{\left(e^{-x}+2\right)^2}{e^{x-1}} \mathrm{\,d}x\\
			&= & \displaystyle\int\limits _0^1\dfrac{e^{-2x}+4e^{-x}+4}{e^{x-1}} \mathrm{\,d}x\\
			&=&	\displaystyle\int\limits _0^1\left(e^{-3x+1}+4e^{-2x+1}+4e^{-x+1} \right)\mathrm{\,d}x\\
			&=&\left. \left(\dfrac{e^{-3x+1}}{-3}+\dfrac{4e^{-2x+1}}{-2}+\dfrac{4e^{-x+1}}{-1} \right)\right|_0^1\\
			&=&\dfrac{-9e^3+4e^2+4e+1}{e^2}=-9e+4+\dfrac{4}{e}+\dfrac{1}{e^2}.
		\end{eqnarray*}
		Vậy $ P=a+b+c=-1$.
	}
\end{ex}
%%%==============EX_10============%%%
\begin{ex}%[2D4H2-3]
	Biết $\displaystyle\int\limits _0^{\tfrac{\pi}{3}}\dfrac{1-\cos 2x}{1+\cos 2x} \mathrm{\,d}x=a\sqrt{3}+\dfrac{\pi}{b}$ $\left(a,b\in \mathbb{Z}\right)$. Tính $a+b$.
	\shortans{$0$}	
	\loigiai{
		\begin{eqnarray*}
			I&= & \displaystyle\int\limits _0^{\tfrac{\pi}{3}}\dfrac{1-\cos 2x}{1+\cos 2x} \mathrm{\,d}x\\
			&= & \displaystyle\int\limits _0^{\tfrac{\pi}{3}}\dfrac{2\sin^2 x}{2\cos^2 x} \mathrm{\,d}x\\
			&=& \displaystyle\int\limits _0^{\tfrac{\pi}{3}}\left(\dfrac{1}{\cos^2 x}-1\right)\mathrm{\,d}x\\
			&=&  \left(\tan x-x\right)\Bigg|_0^{\tfrac{\pi}{3}}=\sqrt{3}-\dfrac{\pi}{3}.
		\end{eqnarray*}
		Vậy  $\heva{&a=1\\
			&b=-1}\Rightarrow a+b=0.$
	}
\end{ex}
%%%==============EX_11============%%%
\begin{ex}%[2D4H2-4]
	Tính $I=\displaystyle\int\limits _0^1\dfrac{\left(2024^x+1\right)^2}{e^{-3x}} \mathrm{\,d}x$ (\textit{làm tròn đến hàng phần trăm}).
	\shortans{$0$}	
	\loigiai{
		\begin{eqnarray*} 
			I&=&\int_0^1 \frac{\left(2024^x+1\right)^2}{e^{-3 x}} d x\\
			&=&\int_0^1 \frac{2024^{2 x}+2 \cdot 2024^x+1}{e^{-3 x}} d x\\
			&=&\left[\left(\frac{2024^2}{e^{-3}}\right)^x+2 \cdot\left(\frac{2024}{e^{-3}}\right)^x+e^{3 x}\right]\Bigg|_0 ^1 \\ 
			& =&\dfrac{\left(\dfrac{2024^2}{e^{-3}}\right)^x}{\ln \dfrac{2024^2}{e^{-3}}}+\dfrac{2 \cdot\left(\dfrac{2024}{e^{-3}}\right)^x}{\ln \dfrac{2024}{e^{-3}}}+\dfrac{1}{3} e^{3 x}\\
			&=&\dfrac{2024^{2 x} e^{3 x}}{2 \ln 2024-3}+\dfrac{2.2024^{2 x} e^{3 x}}{\ln 2024-3}+\dfrac{1}{3} e^{3 x} \\ 
			& =&\left(\dfrac{2024^{2 x}}{2 \ln 2024-3}+\dfrac{2\cdot2024^{2 x}}{\ln 2024-3}+\dfrac{1}{3}\right) e^{3 x}. 
		\end{eqnarray*}
	}
\end{ex}
%%%==============EX_12============%%%
\begin{ex}%[2D4H2-4]
	Tính $I=\dfrac{1}{1000}\displaystyle\int\limits _0^1\dfrac{\left(e^{-x}+2\right)^2}{e^{x-1}} \mathrm{\,d}x$ (\textit{làm tròn đến hàng đơn vị}).
	\shortans{$4522$}	
	\loigiai{
		\begin{eqnarray*}
			I&= &\dfrac{1}{1000}\displaystyle\int\limits _0^1\dfrac{\left(e^{-x}+2\right)^2}{e^{x-1}} \mathrm{\,d}x\\
			&=& \dfrac{1}{1000}\displaystyle\int\limits _0^1\dfrac{e^{-2x}+4e^{-x}+4}{e^{x-1}} \mathrm{\,d}x\\
			&= &\dfrac{1}{1000} \displaystyle\int\limits _0^1\left(e^{-3x+1}+4e^{-2x+1}+4e^{-x+1} \right)\mathrm{\,d}x\\
			&=& \dfrac{1}{1000}\left(e^{-3x+1}+4e^{-2x+1}+4e^{-x+1} \right)\Bigg|_0^1\\
			&=&\dfrac{1}{1000} \dfrac{-9e^3+4e^2+4e+1}{e^2}\approx 4522.
		\end{eqnarray*}
	}
\end{ex}
%%%==============EX_13============%%%
\begin{ex}%[2D4H2-4]
	Tính $I=\dfrac{1}{100}\displaystyle\int\limits_1^2e^{2x} \left(2023+\dfrac{2024e^{-2x}}{x^3} \right) \mathrm{\,d}x$ (\textit{làm tròn đến hàng phần chục}).
	\shortans{$48{,}5$}	
	\loigiai{
		\begin{eqnarray*}
			I&= &\dfrac{1}{100}\displaystyle\int\limits_1^2e^{2x} \left(2023+\dfrac{2024e^{-2x}}{x^3} \right) \mathrm{\,d}x\\
			&=&\dfrac{1}{100}\displaystyle\int\limits_1^2\left(2023e^{2x} +\dfrac{2024}{x^3} \right) \mathrm{\,d}x\\
			&=&\dfrac{1}{100}\left(2023\dfrac{e^{2x}}{2}-\dfrac{1012}{x}\right)\Bigg|_1^2\\
			&\approx& 48{,}5.
		\end{eqnarray*}
	}
\end{ex}
%%%==============EX_14============%%%
\begin{ex}%[2D4H2-4]
	Tính $I=\displaystyle\int\limits_1^2\left(4x^3-2\cdot3^{x+1}+\dfrac{1}{x^2} \right) \mathrm{\,d}x$ (\textit{làm tròn đến hàng phần chục}).
	\shortans{$-17{,}3$}	
	\loigiai{
		\begin{eqnarray*}
			I&= &\displaystyle\int\limits_1^2\left(4x^3-2\cdot3^{x+1}+\dfrac{1}{x^2} \right) \mathrm{\,d}x\\
			&=&\left(x^4-\dfrac{2\cdot3^{x+1}}{\ln 3}-\dfrac{1}{x}\right)\Bigg|_1^2\\
			&\approx&-17{,}3.
		\end{eqnarray*}
	}
\end{ex}
\Closesolutionfile{ans}
% \indapan{3}{ans/ans-C4B2CD1-KQ}

\begin{dang}{Tích phân hàm chứa trị tuyệt đối}
	Tính tích phân $I=\displaystyle\int\limits_a^b|f(x)| \mathrm{\,d}x$?\\
	\textbf{Phương pháp}
	\begin{itemize}
		\item \textbf{Bước 1.} Xét dấu $f(x)$ trên đoạn $[a ; b]$.
		\item \textbf{Bước 2.} Dựa vào bảng xét dấu trên đoạn $[a ; b]$ để khử $|f(x)|$. Sau đó sử dụng các phương pháp tính tích phân đã học để tính $I=\displaystyle\int\limits_a^b|f(x)| \cdot \mathrm{\,d}x$.
	\end{itemize}
\end{dang}

\Opensolutionfile{ans}[ans/ans-C4B2CD1-Dang2]
\TN
%%%==============EX_1============%%%
\begin{ex}%[2D4V2-3]
	Giá trị của $I=\displaystyle\int\limits _0^{2\pi}\sqrt{1-\cos 2x} \mathrm{\,d}x$ bằng
	\choice
	{$\sqrt{3}$}
	{\True $4\sqrt{2}$}
	{$2\sqrt{3}$}
	{$\dfrac{\pi}{2}$}
	\loigiai{
		Ta có	$I=\displaystyle\int\limits_0^{2\pi}\sqrt{1-\cos 2x} \mathrm{\,d}x=\displaystyle\int\limits _0^{2\pi}\sqrt{2\sin^2 x} \mathrm{\,d}x=\sqrt{2} \displaystyle\int\limits_0^{2\pi}\left|\sin x\right|\mathrm{\,d}x.$
		\\
		Vì $x\in \left[0;\pi \right]\to \sin x > 0\Rightarrow \left|\sin x\right|=\sin x$;\\
		$x\in \left[\pi;2\pi \right]\to \sin x < 0\Rightarrow \left|\sin x\right|=-\sin x$.
		\\		
		Vậy $I=\sqrt{2} \left(\displaystyle\int\limits_0^{\pi}\sin x \mathrm{\,d}x+\displaystyle\int\limits_{\pi}^{2\pi}-\sin x\mathrm{\,d}x \right)=\sqrt{2} \left(-\cos x\Bigg|_0^\pi+\cos x\Bigg|_\pi^{2\pi}\right) =4\sqrt{2}$.
	}
\end{ex}
%%%==============EX_2============%%%
\begin{ex}%[2D4H2-2]
	Tính tích phân $I=\displaystyle\int\limits _0^2\left|x-2\right|\mathrm{\,d}x$.
	\choice
	{$I=-2$}
	{$I=4$}
	{\True $I=2$}
	{$I=0$}
	\loigiai{
		Ta có $I=\displaystyle\int\limits _0^2\left|x-2\right|\mathrm{\,d}x.$\\
		Do $x\in \left[0;2\right]\Rightarrow x-2< 0\Leftrightarrow \left|x-2\right|=2-x$.\\
		Vậy $I=\displaystyle\int\limits _0^2\left(2-x\right)\mathrm{\,d}x=\left(2x-\dfrac{1}{2} x^2 \right)\Bigg|_0^2=4-2=2$.
	}
\end{ex}


%%%==============EX_3============%%%
\begin{ex}%[2D4H2-2]
	Tính tích phân $I=\displaystyle\int\limits _0^2\left|x^3-x\right|\mathrm{\,d}x$.
	\choice
	{$I=-\dfrac{1}{2}$}
	{$I=5$}
	{$I=\dfrac{1}{2}$}
	{\True $I=\dfrac{5}{2}$}
	\loigiai{
		Ta có $I=\displaystyle\int\limits _0^2\left|x^3-x\right|\mathrm{\,d}x.$\\
		Ta có $f(x)=x^3-x=x\left(x^2-1\right)=0\leftrightarrow \hoac{&x=0\\&x=-1\\&x=1.}$\\
		\[\Rightarrow f(x) > 0\forall x\in \left[1;2\right];\quad f(x) < 0\forall x\in \left[0;1\right].
		\]
		Vậy $I=\displaystyle\int\limits _0^1\left(x-x^3 \right)\mathrm{\,d}x+\displaystyle\int\limits_1^2\left(x^3-x\right)\mathrm{\,d}x=\left(\dfrac{1}{2} x^2-\dfrac{1}{4} x^4 \right)\Bigg|_0^1+\left(\dfrac{1}{4} x^4-\dfrac{1}{2}^2 \right)\Bigg|_1^2=\dfrac{5}{2}$.
	}
\end{ex}

%%%==============EX_4============%%%
\begin{ex}%[2D4H2-2]
	Tính tích phân $I=\displaystyle\int\limits _0^2\left|x^2+2x-3\right|\mathrm{\,d}x$.
	\choice
	{$I=-2$}
	{$I=4$}
	{$I=5$}
	{\True $I=-4$}
	\loigiai{
		Ta có		$I=\displaystyle\int\limits _0^2\left|x^2+2x-3\right|\mathrm{\,d}x.$\\
		Ta có $f(x)=x^2+2x-3=0\Rightarrow\hoac{&x=1\\&x=-3}\Rightarrow f(x) > 0$, $\forall x\in \left[1;2\right]$; $f(x) < 0$, $\forall x\in \left[0;1\right]$.
		\begin{eqnarray*} 
			I&=&\displaystyle\int\limits _0^1-f(x)\mathrm{\,d}x+\displaystyle\int\limits_1^2f(x)\mathrm{\,d}x\\
			&=&\displaystyle\int\limits _0^1\left(3-2x-x^2 \right)\mathrm{\,d}x+\displaystyle\int\limits_1^2\left(x^2+2x-3\right)\mathrm{\,d}x\\
			&=&	\left(3x-x^2-\dfrac{1}{3} x^3 \right)\Bigg|_0^1+\left(\dfrac{1}{3} x^3+x^2-3x\right)\Bigg|_1^2\\
			&=&\left(3-1-\dfrac{1}{3} \right)+\left[\left(\dfrac{8}{3}+4-6\right)-\left(\dfrac{1}{3}+1-3\right)\right]=4.
		\end{eqnarray*} 	
	}
\end{ex}
%%%==============EX_5============%%%

\begin{ex}%[2D4V2-2]
	Cho tích phân $I=\left(\sqrt{3}+\sqrt{2} \right)\displaystyle\int\limits _{-3}^3\left|x^2-1\right|\mathrm{\,d}x=a\sqrt{3}+b\sqrt{2}$ với $a,b\in \mathbb{Q}$. Tính $P=a+b$.
	\choice
	{$P=\dfrac{44}{3}$}
	{\True $P=\dfrac{88}{3}$}
	{$P=\dfrac{17}{3}$}
	{$P=\dfrac{98}{3}$}
	\loigiai{
		Ta có		$I=\left(\sqrt{3}+\sqrt{2} \right)\displaystyle\int\limits _{-3}^3\left|x^2-1\right|\mathrm{\,d}x$.\\
		Tính $J=\displaystyle\int\limits _{-3}^3\left|x^2-1\right|\mathrm{\,d}x$.\\
		Ta có $f(x)=x^2-1=0\Rightarrow \hoac{&x=1\\&x=-1.}$\\
		$\Rightarrow f(x) > 0$, $\forall x\in \left[-3;-1\right]\cup \left[1;3\right]$; và $f(x) < 0$, $\forall x\in \left[-1;1\right]$.\\
		Vậy
		\begin{eqnarray*} 
			I&=&\displaystyle\int\limits _{-3}^{-1}\left(x^2-1\right)\mathrm{\,d}x+\displaystyle\int\limits _{-1}^1\left(1-x^2 \right)\mathrm{\,d}x+\displaystyle\int\limits_1^3\left(x^2-1\right)\mathrm{\,d}x\\
			&=&\left(\dfrac{1}{3} x^3-x\right)\Bigg|_{-3}^{-1}+\left(x-\dfrac{1}{3} x^3 \right)\Bigg|_{-1}^{1}+\left(\dfrac{1}{3} x^3-x\right)\Bigg|_{1}^3\\
			&=&\dfrac{20}{3}+\dfrac{4}{3}+\dfrac{20}{3}=\dfrac{44}{3}.
		\end{eqnarray*}
		\[\Rightarrow I=\left(\sqrt{3}+\sqrt{2} \right)\displaystyle\int\limits _{-3}^3\left|x^2-1\right|\mathrm{\,d}x=\dfrac{44}{3} \sqrt{3}+\dfrac{44}{3} \sqrt{2}.
		\]
		Khi đó $a=\dfrac{44}{3}$, $b=\dfrac{44}{3}$. Suy ra $P=a+b=\dfrac{88}{3}$.
	}
\end{ex}
%%%==============EX_6============%%%
\begin{ex}%[2D4V2-2]
	Tính tích phân $I=\displaystyle\int\limits _{-2}^5\left(\left|x+2\right|-\left|x-2\right|\right)\mathrm{\,d}x$.
	\choice
	{$I=18$}
	{\True $I=12$}
	{$I=28$}
	{$I=30$}
	\loigiai{
		Ta có $I=\displaystyle\int\limits _{-2}^5\left(\left|x+2\right|-\left|x-2\right|\right)\mathrm{\,d}x.$\\
		Gọi $f(x)=\left|x+2\right|-\left|x-2\right|$ trên $x\in [-2;5]$. Khi đó
		\begin{itemize}
			\item Với $ x\in \left[-2;2\right]$ thì $f(x)=2x$.
			\item Với $ x\in \left[2;5\right]$ thì $f(x)=4$.
		\end{itemize} 
		Vậy $\displaystyle\int\limits _{-2}^5f(x)\mathrm{\,d}x=\displaystyle\int\limits _{-2}^2 2x\mathrm{\,d}x+\displaystyle\int\limits_2^5 4\mathrm{\,d}x=x^2\Bigg|_{-2}^2+4x\Bigg|_2^5=0+12=12$.
	}
\end{ex}
%%%==============EX_7============%%%
\begin{ex}%[2D4V2-4]
	Cho tích phân $I=\displaystyle\int\limits _0^3\left|2^x-4\right|\mathrm{\,d}x=a+\dfrac{b}{c\ln 2}$ với $a,b,c\in \mathbb{Z}$ và $\dfrac{b}{c}$ là phân số tối giản. Tính $P=a^2+b^2+c^2$.
	\choice
	{$P=15$}
	{$P=10$}
	{$P=5$}
	{\True $P=18$}
	\loigiai{
		Ta có $I=\displaystyle\int\limits _0^3\left|2^x-4\right|\mathrm{\,d}x$.
		Ta có $2^x-4> 0\Leftrightarrow x > 2\Rightarrow f(x) > 0,~\forall x\in \left[2;3\right]$; và $f(x) < 0,~\forall x\in \left[0;2\right]$.\\
		Vậy
		\begin{eqnarray*} 
			I&=&\displaystyle\int\limits _0^2\left(4-2^x \right)\mathrm{\,d}x+\displaystyle\int\limits_2^3\left(2^x-4\right)\mathrm{\,d}x\\
			&=&\left(4x-\dfrac{1}{\ln 2} 2^x \right)\Bigg|_0^2+\left(\dfrac{1}{\ln 2} 2^x-4x\right)\Bigg|_2^3\\
			&=&\left(8-\dfrac{3}{\ln 2} \right)+\left(\dfrac{4}{\ln 2}-4\right)=4+\dfrac{1}{\ln 2}.
		\end{eqnarray*} 
		\[\Rightarrow P=a^2+b^2+c^2=4^2+1^2+1^2=18.
		\]
	}
\end{ex}
%%%==============EX_8============%%%
\begin{ex}%[2D4V2-4]
	Tính tích phân $I=\displaystyle\int\limits _{-1}^1\left|2^x-2^{-x} \right|\mathrm{\,d}x$.
	\choice
	{\True $\dfrac{1}{\ln 2}$}
	{$\ln 2$}
	{$2\ln 2$}
	{$\dfrac{2}{\ln 2}$}
	\loigiai{
		$I=\displaystyle\int\limits _{-1}^1\left|2^x-2^{-x} \right|\mathrm{\,d}x$.\\
		Ta có $2^x-2^{-x}=0$ $\Rightarrow x=0$.
		\begin{eqnarray*} 
			I&=&\displaystyle\int\limits _{-1}^1\left|2^x-2^{-x} \right|\mathrm{\,d}x\\
			&=&\displaystyle\int\limits _{-1}^0\left|2^x-2^{-x} \right|\mathrm{\,d}x+\displaystyle\int\limits _0^1\left|2^x-2^{-x} \right|\mathrm{\,d}x\\
			&=&\left|\displaystyle\int\limits _{-1}^0\left(2^x-2^{-x} \right) \mathrm{\,d}x\right|+\left|\displaystyle\int\limits _0^1\left(2^x-2^{-x} \right) \mathrm{\,d}x\right|\\
			&=&\left|\left(\dfrac{2^x+2^{-x}}{\ln 2} \right)\Bigg|_{-1}^0 \right|+\left| \left(\dfrac{2^x+2^{-x}}{\ln 2} \right)\Bigg|_0^1 \right|=\dfrac{1}{\ln 2}.
		\end{eqnarray*} 		
	}
\end{ex}
%%%==============EX_9============%%%

\begin{ex}%[2D4V2-2]
	Tính tích phân $I=\displaystyle\int\limits _{-1}^2\left(\left|x\right|-\left|x-1\right|\right)\mathrm{\,d}x$.
	\choice
	{\True $I=0$}
	{$I=2$}
	{$I=-2$}
	{$I=-3$}
	\loigiai{
		Ta có $I=\displaystyle\int\limits _{-1}^2\left(\left|x\right|-\left|x-1\right|\right)\mathrm{\,d}x$.
		\begin{eqnarray*} 
			I&=&\displaystyle\int\limits _{-1}^2\left(\left|x\right|-\left|x-1\right|\right)\mathrm{\,d}x\\
			&=&\displaystyle\int\limits _{-1}^2\left|x\right|\mathrm{\,d}x-\displaystyle\int\limits _{-1}^2\left|x-1\right|\mathrm{\,d}x\\
			&=&-\displaystyle\int\limits _{-1}^0x\mathrm{\,d}x+\displaystyle\int\limits _0^2x\mathrm{\,d}x+\displaystyle\int\limits _{-1}^1(x-1)\mathrm{\,d}x-\displaystyle\int\limits_1^2(x-1)\mathrm{\,d}x\\
			&=&-\dfrac{x^2}{2}\Bigg|_{-1}^0+ \dfrac{x^2}{2}\Bigg|_0^2+ \left(\dfrac{x^2}{2}-x\right)\Bigg|_{-1}^1- \left(\dfrac{x^2}{2}-x\right)\Bigg|_1^2=0.
		\end{eqnarray*} 		
	}
\end{ex}


%%%==============EX_10============%%%
\begin{ex}%[2D4V2-2]
	Cho $a$ là số thực dương, tính tích phân $I=\displaystyle\int\limits _{-1}^a\left|x\right|\mathrm{d}x$ theo $a$.
	\choice
	{\True $I=\dfrac{a^2+1}{2}$}
	{$I=\dfrac{a^2+2}{2}$}
	{$I=\dfrac{-2a^2+1}{2}$}
	{$I=\dfrac{\left|3a^2-1\right|}{2}$}
	\loigiai{
		Vì $a > 0$ nên $I=-\displaystyle\int\limits_{-1}^0x \mathrm{\,d}x+\displaystyle\int\limits_0^ax \mathrm{\,d}x=\dfrac{1}{2}+\dfrac{a^2}{2}=\dfrac{1+a^2}{2}$.
	}
\end{ex}
%%%==============EX_11============%%%
\begin{ex}%[2D4V2-2]
	Cho số thực $m > 1$ thỏa mãn $\displaystyle\int\limits_1^m\left|2mx-1\right|\mathrm{\,d}x=1$. Khẳng định nào sau đây đúng?
	\choice
	{$m\in \left(4;6\right)$}
	{$m\in \left(2;4\right)$}
	{$m\in \left(3;5\right)$}
	{\True $m\in \left(1;3\right)$}
	\loigiai{
		Do $m > 1\Rightarrow 2m > 2\Rightarrow \dfrac{1}{2m} < 1$. Do đó với $m > 1, x\in \left[1;m\right]\Rightarrow 2mx-1> 0$.\\		
		Vậy
		\begin{eqnarray*} 
			\displaystyle\int\limits_1^m\left|2mx-1\right|\mathrm{\,d}x&=&\displaystyle\int\limits_1^m\left(2mx-1\right)\mathrm{\,d}x\\
			&=&\left(mx^2-x\right)\Bigg|_1^m\\
			&=&m^3-m-m+1=m^3-2m+1.
		\end{eqnarray*}
		Từ đó theo bài ra ta có $m^3-2m+1=1\Leftrightarrow \hoac{&m=0 \\&m=\pm \sqrt{2}.} $\\ Do $m > 1$ vậy $m=\sqrt{2}$.
	}
\end{ex}
%%%==============EX_12============%%%
\begin{ex}%[2D4V2-2]
	Khẳng định nào sau đây là đúng?
	\choice
	{$\displaystyle\int\limits _{-1}^1\left|x\right|^3 \mathrm{d}x=\left|\displaystyle\int\limits _{-1}^1x^3 \mathrm{d}x \right|$}
	{\True $\displaystyle\int\limits _{-1}^{2024}\left|x^4-x^2+1\right|\mathrm{d}x=\displaystyle\int\limits _{-1}^{2024}\left(x^4-x^2+1\right)\mathrm{d}x$}
	{$\displaystyle\int\limits _{-2}^3\left|e^x \left(x+1\right)\mathrm{d}x\right|=\displaystyle\int\limits _{-2}^3e^x \left(x+1\right)\mathrm{d}x$}
	{$\displaystyle\int\limits _{-\tfrac{\pi}{2}}^{\tfrac{\pi}{2}}\sqrt{1-\cos^2 x} \mathrm{d}x=\displaystyle\int\limits _{-\tfrac{\pi}{2}}^{\tfrac{\pi}{2}}\sin x\mathrm{d}x$}
	\loigiai{
		Ta có: $x^4-x^2+1=x^4-2\cdot x^2\cdot\dfrac{1}{2}+\dfrac{1}{4}+\dfrac{3}{4}$ $=\left(x^2-\dfrac{1}{2} \right)^2+\dfrac{3}{4} > 0,\forall x\in {\bf \mathbb{R}}$.\\
		Do đó $\displaystyle\int\limits _{-1}^{2024}\left|x^4-x^2+1\right|\mathrm{d}x=\displaystyle\int\limits _{-1}^{2024}\left(x^4-x^2+1\right)\mathrm{d}x$.
	}
\end{ex}
%%%==============EX_13============%%%
\begin{ex}%[2D4V2-2]
	Tính tích phân $I=\displaystyle\int\limits_1^4\sqrt{x^2-6x+9} \mathrm{\,d}x$.
	\choice
	{\True $I=\dfrac{5}{2}$}
	{$I=-\dfrac{1}{2}$}
	{$I=-2$}
	{$I=\dfrac{1}{2}$}
	\loigiai{
		Ta có $I=\displaystyle\int\limits_1^4\sqrt{x^2-6x+9} \mathrm{\,d}x=\displaystyle\int\limits_1^4\left|x-3\right|\mathrm{\,d}x$.\\
		Ta có $x-3> 0,~\forall x\in \left[3;4\right];~x-3< 0,~\forall x\in \left[1;3\right]$.\\
		Vậy
		\begin{eqnarray*} 
			I&=&\displaystyle\int\limits_1^3\left(3-x\right)\mathrm{\,d}x+\displaystyle\int\limits_3^4\left(x-3\right)\mathrm{\,d}x\\
			&=&\left(3x-\dfrac{1}{2} x^2 \right)\Bigg|_1^3+\left(\dfrac{1}{2} x^2-3x\right)\Bigg|_3^4\\
			&=&2+\dfrac{1}{2}=\dfrac{5}{2}.
		\end{eqnarray*}
	}
\end{ex}
\Closesolutionfile{ans}
% \indapan{6}{ans/ans-C4B2CD1-Dang2}



\Opensolutionfile{ans}[ans/ans-C4B2CD1-Dang2-KQ]
\TNSA

\begin{ex}%[2D4V2-2]
	Tính tích phân $I=\displaystyle\int\limits _{-3}^{3}\left|x^{2} -1\right|\mathrm{\,d}x $ (tính gần đúng đến hàng phần chục).
	\shortans{$13{,}3$}	
	\loigiai{
		\[I=\displaystyle\int\limits _{-3}^{3}\left|x^{2} -1\right|\mathrm{\,d}x.\] 
		Vì  $f(x)=x^{2} -1=0\to\hoac{&x=-1\\&x=1} \Rightarrow f(x)>0,~\forall x\in \left[-3;-1\right]\cup \left[1;3\right]$; $f(x)<0,~\forall x\in \left[-1;1\right]$.\\
		Vậy  
		\begin{eqnarray*} 
			I&=&\displaystyle\int\limits _{-3}^{-1}\left(x^{2} -1\right)\mathrm{\,d}x+\displaystyle\int\limits _{-1}^{1}\left(1-x^{2} \right)\mathrm{\,d}x+\displaystyle\int\limits _{1}^{3}\left(x^{2} -1\right)\mathrm{\,d}x\\
			&=&\left(\frac{1}{3} x^{3} -x\right)\Bigg|_{-3}^{-1}+\left(x-\frac{1}{3} x^{3} \right)\Bigg|_{-1}^1+\left(\frac{1}{3} x^{3} -x\right)\Bigg|_1^3\\
			&=&\frac{20}{3} +\frac{4}{3} +\frac{16}{3} =\frac{40}{3}\approx 13{,}3.
		\end{eqnarray*}
	}
\end{ex}

\begin{ex}%[2D4V2-2]
	Tính tích phân $I=\displaystyle\int\limits _{-1}^{2}\left|-x^{2} -2x+3\right|\mathrm{\,d}x $ (tính gần đúng đến hàng phần trăm).
	\shortans{$7{,}67$}	
	\loigiai{
		Vì  $f(x)=-x^{2} -2x+3=0\Rightarrow\hoac{&x=1\\&x=-3} \Rightarrow f(x)>0,~\forall x\in \left[-1;-1\right]$; $f(x)<0,~\forall x\in \left[1;2\right]$\\
		Vậy  
		\begin{eqnarray*} 
			I&=&\displaystyle\int\limits _{-1}^{1}\left(-x^{2} -2x+3\right)\mathrm{\,d}x+\displaystyle\int\limits _{1}^{2}\left(x^{2}+2x-3 \right)\mathrm{\,d}x\\
			&=&\left(-\dfrac{1}{3} x^{3}-x^2 +3x\right)\Bigg|_{-1}^{1}+\left(\dfrac{1}{3} x^{3}+x^2 -3x \right)\Bigg|_{1}^2\\
			&=&-\dfrac{1}{3}-1+3-\dfrac{1}{3}+1+3 +\dfrac{8}{3}+4-6-\dfrac{1}{3}-1+3 \approx7{,}67.
		\end{eqnarray*}		
	}
\end{ex}

\begin{ex}%[2D4V2-2]
	Tính tích phân $I=\displaystyle\int\limits _{1}^{2}\left|\frac{x+1}{x} \right|\mathrm{\,d}x $ (tính gần đúng đến hàng phần trăm).
	\shortans{$1{,}69$}	
	\loigiai{
		Vì $\frac{x+1}{x}>0$, $\forall x\in [1;2]$ nên
		\[I=\displaystyle\int\limits _{1}^{2}\left(\dfrac{x+1}{x}\right)\mathrm{\,d}x=\displaystyle\int\limits _{1}^{2}\left(1+\dfrac{1}{x}\right)\mathrm{\,d}x=\left(x+\ln x \right)\Bigg|1^2=2+\ln 2-1=1+\ln 2\approx1{,}69.  \]		
	}
\end{ex}

\begin{ex}%[2D4V2-2]
	Tính tích phân $I=\displaystyle\int\limits _{2}^{6}\sqrt{x^{2} -8x+16} \mathrm{\,d}x $.
	\shortans{$4$}	
	\loigiai{
		Ta có $I=\displaystyle\int\limits _{2}^{6}\left| x-4\right|  \mathrm{\,d}x $.\\
		Ta có $x-4\le 0$, $\forall x\in [2;4]$	và 	 $x-4\ge 0$, $\forall x\in [4;6]$. Khi đó
		\[I=\displaystyle\int\limits _{2}^{4}\left( 4- x\right)   \mathrm{\,d}x+\displaystyle\int\limits _{4}^{6}\left( x- 4\right)   \mathrm{\,d}x=\left( 4 x-\dfrac{x^2}{2}\right)\Bigg|_{2}^{4}+\left( -4 x+\dfrac{x^2}{2}\right)\Bigg|_{4}^{6}=4.\]
	}
\end{ex}

\begin{ex}%[2D4V2-2]
	Tính tích phân $I=\displaystyle\int\limits _{-2}^{1}\sqrt{4x^{2} +6x+9} \mathrm{\,d}x $ (\textit{làm tròn đến hàng phần trăm}).
	\shortans{$9{,}38$}	
	\loigiai{
		Ta có $I=\displaystyle\int\limits _{-2}^{1}\sqrt{4x^{2} +6x+9} \mathrm{\,d}x=\displaystyle\int\limits _{-2}^{1}\left|2x+3 \right|  \mathrm{\,d}x$.\\
		Ta có $2x+3\le 0$, $\forall x\in \left[-2;-\dfrac{3}{2} \right] $	và 	 $2x+3\ge 0$, $\forall x\in \left[-\dfrac{3}{2};1 \right]$. Khi đó
		\begin{eqnarray*} 
			I&=&\displaystyle\int\limits _{-2}^{-\tfrac{3}{2}}\left( -2x-3 \right)   \mathrm{\,d}x+\displaystyle\int\limits _{-\tfrac{3}{2}}^{1}\left( 2x+3 \right)  \mathrm{\,d}x\\
			&=&\left(-x^2-3x\right)\Bigg|_{-2}^{-\tfrac{3}{2}}+\left(x^2+3x \right)\Bigg|_{-\tfrac{3}{2}}^{1}\\
			&\approx&9{,}38.
		\end{eqnarray*}				
	}
\end{ex}

\begin{ex}%[2D4V2-2]
	Tính tích phân $I=\displaystyle\int\limits _{0}^{1}\sqrt{9x^{2} -6x+1} \mathrm{\,d}x $ (\textit{làm tròn đến hàng phần trăm}).
	\shortans{$0{,}83$}	
	\loigiai{
		Ta có $I=\displaystyle\int\limits _{0}^{1}\sqrt{9x^{2} -6x+1} \mathrm{\,d}x =\displaystyle\int\limits _{0}^{1}\left| 3x-1\right| \mathrm{\,d}x $.\\
		Ta có $3x+1\le 0$, $\forall x\in \left[1;\dfrac{1}{3} \right] $	và 	 $3x+1\ge 0$, $\forall x\in \left[\dfrac{1}{3};1 \right]$. Khi đó
		\begin{eqnarray*} 
			I&=&\displaystyle\int\limits _{0}^{\tfrac{1}{3}}\left(-3x-1 \right)   \mathrm{\,d}x+\displaystyle\int\limits _{\tfrac{1}{3}}^{1}\left( 3x+1 \right)  \mathrm{\,d}x\\
			&=&\left(-\dfrac{3x^2}{2}-x\right)\Bigg| _{0}^{\tfrac{1}{3}}+\left(\dfrac{3x^2}{2}+x\right)\Bigg|_{\tfrac{1}{3}}^{1}\\
			&\approx&0{,}83.
		\end{eqnarray*}		
	}
\end{ex}

\begin{ex}%[2D4V2-3]
	Tính tích phân $I=\displaystyle\int\limits _{0}^{2\pi }\sqrt{1+\cos 2x} \mathrm{\,d}x $ (\textit{làm tròn đến hàng phần trăm}).
	\shortans{$5{,}66$}	
	\loigiai{
		Ta có  $I=\displaystyle\int\limits _{0}^{2\pi }\sqrt{1+\cos 2x} \mathrm{\,d}x =\sqrt{2}\displaystyle\int\limits _{0}^{2\pi }|\cos x|\mathrm{\,d}x $.\\
		Ta có $\cos x\ge 0, \forall x\in \left[0;\dfrac{\pi}{2} \right]\cup\left[\dfrac{3\pi}{2};2\pi \right] $ và $\cos x\le 0, \forall x\in \left[\dfrac{\pi}{2};\dfrac{3\pi}{2} \right] $. Khi đó
		\begin{eqnarray*} 
			I&=&\sqrt{2}\displaystyle\int\limits _{0}^{\tfrac{\pi}{2} }\cos x\mathrm{\,d}x-\sqrt{2}\displaystyle\int\limits _{\tfrac{\pi}{2} }^{\tfrac{3\pi}{2} }\cos x\mathrm{\,d}x+\sqrt{2}\displaystyle\int\limits _{\tfrac{3\pi}{2} }^{2\pi}\cos x\mathrm{\,d}x\\
			&=&\sqrt{2}\sin x\Bigg|_{0}^{\tfrac{\pi}{2} } -\sqrt{2}\sin x\Bigg|_{\tfrac{\pi}{2} }^{\tfrac{3\pi}{2} }+\sqrt{2}\sin x\Bigg|_{\tfrac{3\pi}{2} }^{2\pi}\\
			&=&4\sqrt{2}\approx5{,}66.
	\end{eqnarray*}		}
\end{ex}

\begin{ex}%[2D4V2-3]
	Tính tích phân $I=\displaystyle\int\limits_{0}^{2\pi }\sqrt{1-\cos 2x} \mathrm{\,d}x $ (\textit{làm tròn đến hàng phần trăm}).
	\shortans{$5{,}66$}	
	\loigiai{
		Ta có  $I=\displaystyle\int\limits _{0}^{2\pi }\sqrt{1-\cos 2x} \mathrm{\,d}x =2\displaystyle\int\limits _{0}^{2\pi }|\sin  x|\mathrm{\,d}x $.\\
		Ta có $\sin x\ge 0, \forall x\in \left[0;\pi \right]$ và $\sin x\le 0, \forall x\in \left[\pi;2\pi \right]$. Khi đó
		\begin{eqnarray*} 
			I&=&\sqrt{2}\displaystyle\int\limits _{0}^{\pi}\sin x\mathrm{\,d}x-\sqrt{2}\displaystyle\int\limits _{\pi}^{2\pi}\sin x\mathrm{\,d}x\\
			&=&-\sqrt{2}\cos x\Bigg|_{0}^{\pi } +\sqrt{2}\cos x\Bigg|_{\pi }^{2\pi}\\
			&=&4\sqrt{2}\approx5{,}66.
		\end{eqnarray*}			
	}
\end{ex}


\begin{ex}%[2D4V2-3]
	Tính tích phân $I=\displaystyle\int\limits _{0}^{2\pi }\sqrt{1-\sin 2x} \mathrm{\,d}x $, (\textit{làm tròn đến hàng phần trăm}).
	\shortans{$0{,}31$}	
	\loigiai{
		Ta có  $I=\displaystyle\int\limits _{0}^{2\pi }\sqrt{1-\sin 2x} \mathrm{\,d}x =\displaystyle\int\limits _{0}^{2\pi }|\sin  x-\cos x|\mathrm{\,d}x $.\\
		Ta có $\sin x-\cos x\le 0, \forall x\in \left[0;\dfrac{\pi}{4} \right]\cup\left[\dfrac{5\pi}{4};2\pi \right] $ và $\sin x-\cos x\ge 0, \forall x\in \left[\dfrac{\pi}{4};\dfrac{5\pi}{4} \right]$. Khi đó
		\begin{eqnarray*} 
			I&=&\displaystyle\int\limits _{0}^{\tfrac{\pi}{4} }\left( \cos x-\sin x\right) \mathrm{\,d}x+\displaystyle\int\limits _{\tfrac{\pi}{4} }^{\tfrac{5\pi}{4} }\left( \sin x-\cos x\right) \mathrm{\,d}x+\displaystyle\int\limits _{\tfrac{5\pi}{4} }^{2\pi}\left( \cos x-\sin x\right) \mathrm{\,d}x\\
			&=&\left( \sin x+\cos x\right) \Bigg|_{0}^{\tfrac{\pi}{4} }+\left( -\cos x-\sin x\right) \Bigg|_{\tfrac{\pi}{4} }^{\tfrac{5\pi}{4} }+\left( \sin x+\cos x\right) \Bigg|_{\tfrac{5\pi}{4} }^{2\pi}\\
			&=&4\sqrt{2}\approx5{,}66.
		\end{eqnarray*}				
	}
\end{ex}

\begin{ex}%[2D4V2-3]
	Tính tích phân $I=\displaystyle\int\limits _{0}^{2\pi }\sqrt{1+\sin 2x} \mathrm{\,d}x $ (\textit{làm tròn đến hàng phần trăm}).
	\shortans{$5{,}66$}	
	\loigiai{
		Ta có  $I=\displaystyle\int\limits _{0}^{2\pi }\sqrt{1+\sin 2x} \mathrm{\,d}x =\displaystyle\int\limits _{0}^{2\pi }|\sin  x+\cos x|\mathrm{\,d}x $.\\
		Ta có $\sin x+\cos x\ge 0, \forall x\in \left[0;\dfrac{3\pi}{4} \right]\cup\left[\dfrac{7\pi}{4};2\pi \right] $ và $\sin x+\cos x\le 0, \forall x\in \left[\dfrac{3\pi}{4};\dfrac{7\pi}{4} \right]$. \\
		Khi đó:
		\begin{eqnarray*} 
			I&=&\displaystyle\int\limits _{0}^{\tfrac{3\pi}{4} }\left( \cos x+\sin x\right) \mathrm{\,d}x-\displaystyle\int\limits _{\tfrac{3\pi}{4} }^{\tfrac{7\pi}{4} }\left( \sin x+\cos x\right) \mathrm{\,d}x+\displaystyle\int\limits _{\tfrac{7\pi}{4} }^{2\pi}\left( \cos x+\sin x\right) \mathrm{\,d}x\\
			&=&\left( \sin x-\cos x\right) \Bigg|_{0}^{\tfrac{3\pi}{4} }-\left( \sin x-\cos x\right) \Bigg|_{\tfrac{3\pi}{4} }^{\tfrac{7\pi}{4} }+\left( \sin x-\cos x\right) \Bigg|_{\tfrac{7\pi}{4} }^{2\pi}\\
			&=&4\sqrt{2}\approx5{,}66.
		\end{eqnarray*}				
	}
\end{ex}
\Closesolutionfile{ans}
% \indapan{6}{ans/ans-C4B2CD1-Dang2-KQ}

% \begin{dang}{Tích phân có điều kiện}
\end{dang}
\TN
\Opensolutionfile{ans}[ans/ans-2C4B2CD2-LC]
\begin{ex}%[2D4H1-1]
	Nếu $F'(x) = \dfrac{1}{2x}$ và $F(1) = 1$ thì giá trị của $F(4)$ bằng
	\choice
	{$\ln 2$}
	{\True $1 + \ln 2$}
	{$1 + \dfrac{1}{2} \ln 2 $}
	{$ \dfrac{1}{2} \ln 2 $}
\loigiai{	
	Ta có
	$$
	\displaystyle\int\limits_{1}^{4} F'(x) \, \mathrm{d}x = 	\displaystyle\int\limits_{1}^{4} \dfrac{1}{2x} \, \mathrm{d}x = \dfrac{1}{2} \ln \left| x \right| \bigg|_{1}^{4} = \ln 2 .
	$$	
	Lại có 
	$$
	\displaystyle\int\limits_{1}^{4} F'(x) \, dx = F(x) \bigg|_{1}^{4} = F(4) - F(1) .
	$$	
	Suy ra 
	$	F(4) - F(1) = \ln 2	$	.
	Do đó 
	$	F(4) = F(1) + \ln 2 = 1 + \ln 2 $.
}
\end{ex}
\begin{ex}%[2D4H1-1]
	Cho $F(x)$ là một nguyên hàm của $f(x) = \dfrac{2}{x}$. Biết $F(-1) = 0$. Tính $F(2)$ kết quả là
	\choice
	{$2 \ln 2 + 1$}
	{$\ln 2$}
	{$ 2 \ln 3 + 2$}
	{\True $2 \ln 2 $}
\loigiai{
	Ta có 
\allowdisplaybreaks
\begin{eqnarray*}
&&	\displaystyle\int\limits_{-1}^{2} f(x) \, \mathrm{d}x = F(x) \bigg|_{-1}^{2} = F(2) - F(-1)\\
&&	\displaystyle\int\limits_{-1}^{2} \dfrac{2}{x} \, dx = 2 \ln \left| x \right| \bigg|_{-1}^{2} = 2 \ln 2 - 2 \ln 1 = 2 \ln 2\\
&	\Rightarrow& F(2) - F(-1) = 2 \ln 2\\
&\Leftrightarrow& F(2) = 2 \ln 2 \, \text{(do } F(-1) = 0).	
\end{eqnarray*}
}
\end{ex}
\begin{ex}%[2D4H1-1]
	Cho hàm số $f(x)$ liên tục, có đạo hàm trên $[-1;2]$, $f(-1) = 8$, $f(2) = -1$. Tích phân $\displaystyle\int\limits_{-1}^{2} f'(x) \, \mathrm{d}x$ bằng
	\choice
	{$1$}
	{$7$}
	{\True $-9$}
	{$9$}
\loigiai{
	Ta có 
	$$
	\displaystyle\int\limits_{-1}^{2} f'(x) \, \mathrm{d}x = f(x) \bigg|_{-1}^{2} = f(2) - f(-1) = -1 - 8 = -9
	.$$
}
\end{ex}
\begin{ex}%[2D4H1-1]
	Biết $F(x) = x^2$ là một nguyên hàm của hàm số $f(x)$ trên $\mathbb{R}$. Giá trị của $\displaystyle\int\limits_{1}^{3} \left[ 1 + f(x) \right] \mathrm{d}x$ bằng
	\choice
	{\True $10$}
	{$8$}
	{$ \dfrac{26}{3}$}
	{$ \dfrac{32}{3}$}


\loigiai{
	Ta có 
	$$\displaystyle\int\limits_{1}^{3} \left[ 1 + f(x) \right] \mathrm{d}x = (x + F(x)) \bigg|_{1}^{3} = (x + x^2) \bigg|_{1}^{3} = 12 - 2 = 10.$$
}
\end{ex}
\begin{ex}%[2D4H2-2]
	Biết $F(x) = x^3$ là một nguyên hàm của hàm số $f(x)$ trên $\mathbb{R}$. Giá trị của $\displaystyle\int\limits_{1}^{3} \left[ 1 + f(x) \right] \mathrm{d}x$ bằng
	\choice
	{$20$}
	{$22$}
	{$26$}
	{\True $28$}
\loigiai{	
	Ta có 
	$$
	\displaystyle\int\limits_{1}^{3} \left[ 1 + f(x) \right] \mathrm{d}x = \left[ x + F(x) \right] \bigg|_{1}^{3} = \left[ x + x^3 \right] \bigg|_{1}^{3} = 30 - 2 = 28
	.$$
}

\end{ex}
\begin{ex}%[2D4H2-2]
	Biết $F(x) = x^2$ là một nguyên hàm của hàm số $f(x)$ trên $\mathbb{R}$. Giá trị của $\displaystyle\int\limits_{1}^{2} \left[ 2 + f(x) \right] \mathrm{d}x$ bằng
	\choice
	{\True$5$}
	{$3$}
	{$\dfrac{13}{3}$}
	{$\dfrac{7}{3}$}
\loigiai{
	Ta có 
	$$	\displaystyle\int\limits_{1}^{2} \left[ 2 + f(x) \right] \mathrm{d}x = (2x + x^2) \bigg|_{1}^{2} = 8 - 3 = 5	.$$
}
\end{ex}
\begin{ex}%[2D4H2-2]
	Biết $F(x) = x^3$ là một nguyên hàm của hàm số $f(x)$ trên $\mathbb{R}$. Giá trị của $\displaystyle\int\limits_{1}^{2} \left[ 2 + f(x) \right] \mathrm{d}x$ bằng
	\choice
	{$\dfrac{23}{4}$}
	{$7$}
	{\True$9$}
	{$\dfrac{15}{4}$}
\loigiai{
	Ta có 
	$$	\displaystyle\int\limits_{1}^{2} \left[ 2 + f(x) \right] \mathrm{d}x = \displaystyle\int\limits_{1}^{2} 2 \, \mathrm{d}x + \displaystyle\int\limits_{1}^{2} f(x) \, \mathrm{d}x = 2x \bigg|_{1}^{2} + F(x) \bigg|_{1}^{2} = 2x \bigg|_{1}^{2} + x^3 \bigg|_{1}^{2} = 9	.$$
}
\end{ex}
\begin{ex}%[2D4H2-3]
	Cho hàm số $f(x)$. Biết $f(0) = 4$ và $f'(x) = 2 \sin^2 \dfrac{x}{2} + 1$, $\forall x \in \mathbb{R}$, khi đó $\displaystyle\int_{0}^{\frac{\pi}{4}} f(x) \mathrm{d}x$ bằng
	\choice
	{\True $\dfrac{\pi^2 + 16\pi + 8\sqrt{2} - 16}{16}$}
	{$\dfrac{\pi^2 + 16\pi + 2\sqrt{2} - 4}{16}$}
	{$\dfrac{\pi^2 + 16\pi + 8\sqrt{2}}{16}$}
	{$\dfrac{\pi^2 + 16\pi - 16}{16}$}
\loigiai{	
	Ta có 	$$
	f(x) = \displaystyle\int \left( 2 \sin^2 \dfrac{x}{2} + 1 \right) \mathrm{d}x = \displaystyle\int (2 - \cos x) \mathrm{d}x = 2x - \sin x + C	.$$	
	Vì $f(0) = 4 \Rightarrow C = 4 \Rightarrow f(x) = 2x - \sin x + 4$.\\	
	Suy ra 
	\allowdisplaybreaks
	\begin{eqnarray*}
&&		\displaystyle\int\limits_{0}^{\frac{\pi}{4}} f(x) \mathrm{d}x = \displaystyle\int\limits_{0}^{\frac{\pi}{4}} (2x - \sin x + 4) \mathrm{d}x\\
&	= &\left( x^2 + \cos x + 4x \right) \bigg|_{0}^{\frac{\pi}{4}} = \dfrac{\pi^2}{16} + \dfrac{\sqrt{2}}{2} + \pi - 1 = \dfrac{\pi^2 + 16\pi + 8\sqrt{2} - 16}{16}.		
	\end{eqnarray*}
}
\end{ex}
\begin{ex}%[2D4H2-3]
	Cho hàm số $f(x)$. Biết $f(0) = 4$ và $f'(x) = 2 \cos^2 \dfrac{x}{2} + 3$, $\forall x \in \mathbb{R}$, khi đó $\displaystyle\int\limits_{0}^{\frac{\pi}{4}} f(x) \mathrm{d}x$ bằng?
	\choice
	{$\dfrac{\pi^2 + 8\pi - 8 - \sqrt{2}}{8}$}
	{\True$\dfrac{\pi^2 + 8\pi - 8 - 4\sqrt{2}}{8}$}
	{$\dfrac{\pi^2 + 6\pi + 8}{8}$}
	{$\dfrac{\pi^2 + 8\pi - 4\sqrt{2}}{8}$}

\loigiai{
	Ta có 
\allowdisplaybreaks
\begin{eqnarray*}
		f(x) &= &\displaystyle\int f'(x) \mathrm{d}x = \displaystyle\int (2 \cos^2 \dfrac{x}{2} + 3) \mathrm{d}x\\
	&=&\displaystyle\int \left( 2 \cdot \dfrac{1 + \cos x}{2} + 3 \right) \mathrm{d}x = \displaystyle\int (\cos x + 4) \mathrm{d}x\\
	&	\Rightarrow& f(x) = \sin x + 4x + C	.
\end{eqnarray*}	
Do $f(0) = 4 \Rightarrow C = 4\Rightarrow  	f(x) = \sin x + 4x + 4$. Vậy\\
$$ \displaystyle\int\limits_{0}^{\frac{\pi}{4}} f(x) \mathrm{d}x = \displaystyle\int\limits_{0}^{\frac{\pi}{4}} (\sin x + 4x + 4) \mathrm{d}x	= \left( -\cos x + 2x^2 + 4x \right) \bigg|_{0}^{\frac{\pi}{4}} = \dfrac{\pi^2 + 8\pi - 8 - 4\sqrt{2}}{8}.$$
}
\end{ex}
\begin{ex}%[2D4H2-4]
	Cho hàm số $f(x) = \heva{
		&e^{2x} \text{ khi } x \geq 0 \\
		&x^2 + x + 2 \text{ khi } x < 0 
	}$. Biết tích phân $\displaystyle\int\limits_{-1}^{1} f(x) \mathrm{d}x = \dfrac{a}{b} + \dfrac{e^2}{c}$ ($\dfrac{a}{b}$ là phân số tối giản). Giá trị $a + b + c$ bằng
	\choice
	{$7$}
	{$8$}
	{\True$9$}
	{$10$}
\loigiai{
	Ta có
	$$
	I = \displaystyle\int\limits_{-1}^{1} f(x) \mathrm{d}x = \displaystyle\int\limits_{-1}^{0} (x^2 + x + 2) \mathrm{d}x + \displaystyle\int\limits_{0}^{1} e^{2x} \mathrm{d}x = \dfrac{4}{3} + \dfrac{e^2}{2}
	.$$	
	Vậy $a + b + c = 9$.
}
\end{ex}
\begin{ex}%[2D4H2-2]
	Cho hàm số $f(x) = \heva{
		&x^2 - 1 \text{ khi } x \geq 2 \\
		&x^2 - 2x + 3 \text{ khi } x < 2 
	}$. Tích phân $I = \dfrac{1}{2} \displaystyle\int\limits_{1}^{3} f(x) \mathrm{d}x$ bằng:
	\choice
	{$\dfrac{23}{3}$}
	{\True $\dfrac{23}{6}$}
	{$\dfrac{17}{6}$}
	{$\dfrac{17}{3}$}
\loigiai{
Ta có
	$$I = \dfrac{1}{2} \displaystyle\int\limits_{1}^{3} f(x) \mathrm{d}x = \dfrac{1}{2} \left[ \displaystyle\int\limits_{1}^{2} (x^2 - 2x + 3) \mathrm{d}x + \displaystyle\int\limits_{2}^{3} (x^2 - 1) \mathrm{d}x \right] = \dfrac{23}{6}.$$
}
\end{ex}
\begin{ex}%[2D4H2-2]
	Cho hàm số $f(x) = \heva{
		&\dfrac{x(1 + x^2)}{x - 4} \text{ khi } x \geq 3 \\
		&\dfrac{1}{x - 4} \text{ khi } x < 3 
	}$. Tích phân $I = \displaystyle\int\limits_{2}^{4} f(t) \mathrm{d}t$ bằng:
	\choice
	{$\dfrac{40}{3} - \ln 2$}
	{$\dfrac{95}{6} + \ln 2$}
	{$\dfrac{189}{4} + \ln 2$}
	{\True $\dfrac{189}{4} - \ln 2$}
\loigiai{
Ta có
$$	I = \displaystyle\int\limits_{2}^{4} f(t) \mathrm{d}t = \displaystyle\int\limits_{2}^{3} \dfrac{1}{x - 4} \mathrm{d}x + \displaystyle\int\limits_{3}^{4} \dfrac{x(1 + x^2)}{x - 4} \mathrm{d}x = \dfrac{189}{4} - \ln 2
	.$$
}
\end{ex}
\begin{ex}%[2D4H2-2]
	Cho số thực $a$ và hàm số $f(x) = \heva{
		&2x \text{ khi } x \leq 0 \\
		&a(x - x^2) \text{ khi } x > 0 
	}$. Tính tích phân $\displaystyle\int\limits_{-1}^{1} f(x) \mathrm{d}x$ bằng:
	\choice
	{\True $\dfrac{a}{6} - 1$}
	{$\dfrac{2a}{3} + 1$}
	{$\dfrac{a}{6} + 1$}
	{$\dfrac{2a}{3} - 1$}
\loigiai{
	Ta có
\allowdisplaybreaks
\begin{eqnarray*}
&&	\displaystyle\int\limits_{-1}^{1} f(x) \mathrm{d}x = \displaystyle\int\limits_{-1}^{0} f(x) \mathrm{d}x + \displaystyle\int\limits_{0}^{1} f(x) \mathrm{d}x = \displaystyle\int\limits_{-1}^{0} 2x \mathrm{d}x + \displaystyle\int\limits_{0}^{1} a(x - x^2) \mathrm{d}x\\
&	=& (x^2) \bigg|_{-1}^{0} + a \left( \dfrac{x^2}{2} - \dfrac{x^3}{3} \right) \bigg|_{0}^{1} = -1 + a \left( \dfrac{1}{6} \right) = \dfrac{a}{6} - 1.	
\end{eqnarray*}
}
\end{ex}
\Closesolutionfile{ans}
\indapan{6}{ans/ans-2C4B2CD2-LC}
\TNTF
\Opensolutionfile{ans}[ans/ans-2C4B2CD2-DS]
\begin{ex}%[2D4H2-2]
	Cho hàm số $f(x) = \heva{
		&2x^2 + 3 \text{ khi } x \geq 1 \\
		&2 - x^3 \text{ khi } x < 1 
	}$.
	\choiceTF
	{\True $\displaystyle\int\limits_{1}^{2024} f(x) \mathrm{d}x = \displaystyle\int\limits_{1}^{2024} (2x^2 + 3) \mathrm{d}x$}
{\True $\displaystyle\int\limits_{-2024}^{1} f(x) \mathrm{d}x = \displaystyle\int\limits_{-2024}^{1} (2 - x^3) \mathrm{d}x$}
	{$\displaystyle\int\limits_{-2024}^{2024} f(x) \mathrm{d}x = \displaystyle\int\limits_{1}^{2024} (2x^2 + 3) \mathrm{d}x + \displaystyle\int\limits_{-2024}^{1} (2 - x^3) \mathrm{d}x$}
	{\True $\displaystyle\int_{-2024}^{2024} f(x) \mathrm{d}x = \displaystyle\int\limits_{1}^{2024} (2x^2 + 3) \mathrm{d}x + \displaystyle\int\limits_{-2024}^{1} (2 - x^3) \mathrm{d}x$}
\loigiai{
Do $f(x) = \heva{
	&2x^2 + 3 \text{ khi } x \geq 1 \\
	&2 - x^3 \text{ khi } x < 1 
}$ nên\\
\begin{itemize}
	\item $\displaystyle\int\limits_{1}^{2024} f(x) \mathrm{d}x = \displaystyle\int\limits_{1}^{2024} (2x^2 + 3) \mathrm{d}x.
	$
	\item $\displaystyle\int\limits_{-2024}^{1} f(x) \mathrm{d}x = \displaystyle\int\limits_{-2024}^{1} (2 - x^3) \mathrm{d}x.
	$
\item $
\displaystyle\int\limits_{-2024}^{2024} f(x) \mathrm{d}x = \displaystyle\int\limits_{1}^{2024} (2x^2 + 3) \mathrm{d}x + \displaystyle\int\limits_{-2024}^{1} (2 - x^3) \mathrm{d}x.
$
\item $
\displaystyle\int\limits_{-2024}^{2024} f(x) \mathrm{d}x = \displaystyle\int\limits_{1}^{2024} (2x^2 + 3) \mathrm{d}x + \displaystyle\int\limits_{-2024}^{1} (2 - x^3) \mathrm{d}x.
$
\end{itemize}
}

\end{ex}

\begin{ex}%[2D4H2-2]
	Cho hàm số $f(x) = \heva{
		&x^2 - 2x + 3 \text{ khi } x \geq 2 \\
		&x + 1 \text{ khi } x < 2 
	}$.
	\choiceTF
	{\True $\displaystyle\int\limits_{1}^{2} f(x) \mathrm{d}x = \displaystyle\int\limits_{1}^{2} (x + 1) \mathrm{d}x$}
	{\True $\displaystyle\int\limits_{2}^{3} f(x) \mathrm{d}x = \displaystyle\int\limits_{2}^{3} (x^2 - 2x + 3) \mathrm{d}x$}
	{\True $\displaystyle\int\limits_{1}^{3} \dfrac{1}{2} f(x) \mathrm{d}x = \dfrac{41}{12}$}
	{$\displaystyle\int\limits_{1}^{2} f(x) \mathrm{d}x = \displaystyle\int\limits_{1}^{2} (x^2 - 2x + 3) \mathrm{d}x$}
\loigiai{
	Do $f(x) = \heva{
		&x^2 - 2x + 3 \text{ khi } x \geq 2 \\
		&x + 1 \text{ khi } x < 2 
	}$ nên\\
\begin{itemize}
	\item $
	\displaystyle\int\limits_{1}^{2} f(x) \mathrm{d}x = \displaystyle\int\limits_{1}^{2} (x + 1) \mathrm{d}x	$.
\item 	$
\displaystyle\int\limits_{2}^{3} f(x) \mathrm{d}x = \displaystyle\int\limits_{2}^{3} (x^2 - 2x + 3) \mathrm{d}x$.
\item 	$
 \displaystyle\int\limits_{1}^{3} \dfrac{1}{2} f(x) \mathrm{d}x = \dfrac{1}{2} \left( \displaystyle\int\limits_{1}^{2} (x + 1) \mathrm{d}x + \displaystyle\int\limits_{2}^{3} (x^2 - 2x + 3) \mathrm{d}x \right) = \dfrac{41}{12}
$.
\end{itemize}
}
\end{ex}
\Closesolutionfile{ans}
\indapan{2}{ans/ans-2C4B2CD2-DS}
\TNSA
\Opensolutionfile{ans}[ans/ans-2C4B2CD2-KQ]
\begin{ex}%[2D4H1-2]
	Cho hàm số $f(x) = \heva{
		&\dfrac{1}{x} \text{ khi } x \geq 1 \\
		&x + 1 \text{ khi } x < 1 
	}$. Tích phân $I = \displaystyle\int\limits_{2}^{0} -3t^2 f(t) \mathrm{d}t$. (\textit{\textit{làm tròn đến hàng phần trăm}})
\shortans{$2{,}08$}
\loigiai{
	Ta có\\
	$
	I = -3 \displaystyle\int\limits_{2}^{0} t^2 f(t) \mathrm{d}t = 3 \displaystyle\int\limits_{0}^{2} t^2 f(t) \mathrm{d}t = 3 \left[ \displaystyle\int\limits_{0}^{1} x^2 (x + 1) \mathrm{d}x + \displaystyle\int\limits_{1}^{2} x^2 \cdot \dfrac{1}{x} \mathrm{d}x \right] = \dfrac{25}{12}\approx 2{,}08
	$.
}
\end{ex}
\begin{ex}%[2D4H1-2]
	Cho hàm số $f(x) = \heva{
		&2x^2 - 1 \text{ khi } x < 0 \\
		&x - 1 \text{ khi } 0 \leq x \leq 2 \\
		&5 - 2x \text{ khi } x > 2 
	}$. Tính tích phân $I = \displaystyle\int\limits_{-5}^{9} \dfrac{1}{7} f(t) \mathrm{d}t$. (\textit{làm tròn đến hàng phần trăm})
\shortans{$5{,}19$}
\loigiai{
	Ta có
\allowdisplaybreaks
\begin{eqnarray*}
		I &=& \dfrac{1}{7} \displaystyle\int\limits_{-5}^{9} f(t) \mathrm{d}t = \dfrac{1}{7} \displaystyle\int\limits_{-5}^{9} f(x) \mathrm{d}x = \dfrac{1}{7} \left( \displaystyle\int\limits_{-5}^{0} f(x) \mathrm{d}x + \displaystyle\int\limits_{0}^{2} f(x) \mathrm{d}x + \displaystyle\int\limits_{2}^{9} f(x) \mathrm{d}x \right)\\
&	=& \dfrac{1}{7} \displaystyle\int\limits_{-5}^{0} (2x^2 - 1) \mathrm{d}x + \dfrac{1}{7} \displaystyle\int\limits_{0}^{2} (x - 1) \mathrm{d}x + \dfrac{1}{7} \displaystyle\int\limits_{2}^{9} (5 - 2x) \mathrm{d}x = \dfrac{109}{21}	\approx 5{,}19.
\end{eqnarray*}
}
\end{ex}
\begin{ex}%[2D4H1-2]
	Cho hàm số $f(x) = \heva{
		&x^2 - x \text{ khi } x \geq 0 \\
		&x \text{ khi } x < 0 
	}$. Khi đó $I = \displaystyle\int\limits_{-1}^{1} f(x) \mathrm{d}x + \displaystyle\int\limits_{-1}^{3} f(x) \mathrm{d}x$ bằng bao nhiêu? (\textit{làm tròn đến hàng phần trăm})
\shortans{$3{,}33$}
\loigiai{
	Đặt 	$	I_1 = \displaystyle\int\limits_{-1}^{1} f(x) \mathrm{d}x$ và 	$	I_2 = \displaystyle\int\limits_{-1}^{3} f(x) \mathrm{d}x
	$. \\	
	Vì  $f(x) = \heva{
		&x^2 - x \text{ khi } x \geq 0 \\
		&x \text{ khi } x < 0 
	}$ nên \\
	$$ I_1 = \displaystyle\int\limits_{-1}^{0} x \mathrm{d}x + \displaystyle\int\limits_{0}^{1} (x^2 - x) \mathrm{d}x = -\dfrac{2}{3}.$$
Và
	$$
 I_2 = \displaystyle\int\limits_{-1}^{0} x \mathrm{d}x + \displaystyle\int_{0}^{3} (x^2 - x) \mathrm{d}x = 4.$$	
	Vậy $I = I_1 + I_2 = \dfrac{10}{3}\approx 3{,}33$.
}
\end{ex}

\begin{ex}%[2D4H1-2]
	Cho hàm số $f(x) = \heva{
		&4x \text{ khi } x > 2 \\
		&-2x + 12 \text{ khi } x \leq 2 
	}$. Tính tích phân $I = \displaystyle\int\limits_{1}^{2} f(t) \mathrm{d}t + \dfrac{1}{2} \displaystyle\int\limits_{5}^{10} f(t) \mathrm{d}t$.
\shortans{$84$}
\loigiai{
Đặt 	$I_1 = \displaystyle\int\limits_{1}^{2} f(t) \mathrm{d}t = \displaystyle\int\limits_{1}^{2} f(x) \mathrm{d}x$ và 	$
I_2 = \dfrac{1}{2} \displaystyle\int\limits_{5}^{10} f(t) \mathrm{d}t = \dfrac{1}{2} \displaystyle\int\limits_{5}^{10} f(x) \mathrm{d}x$.\\
	Vì  $f(x) = \heva{
		&4x \text{ khi } x > 2 \\
		&-2x + 12 \text{ khi } x \leq 2 
	}$ nên\\	
$$I_1 = \displaystyle\int\limits_{1}^{2} (-2x + 12) \mathrm{d}x = 9.$$
Và 
	$$
I_2 = \dfrac{1}{2} \displaystyle\int\limits_{5}^{10} 4x \mathrm{d}x = 75.$$	
	Vậy $I = I_1 + I_2 = 84$.
}
\end{ex}

\begin{ex}%[2D4H1-2]
	Biết rằng hàm số $f(x) = mx + n$ thỏa mãn $\displaystyle\int\limits_{0}^{1} f(x) \mathrm{d}x = 3$, $\displaystyle\int\limits_{0}^{2} f(x) \mathrm{d}x = 8$. Tính $m + n$.
\shortans{$4$}
\loigiai{
	Ta có 
	$
	\displaystyle\int f(x) \mathrm{d}x = \displaystyle\int (mx + n) \mathrm{d}x = \dfrac{m}{2} x^2 + nx + C
	$.\\	
	Lại có
	$
	\displaystyle\int\limits_{0}^{1} f(x) \mathrm{d}x = 3 \Rightarrow \left( \dfrac{m}{2} x^2 + nx \right) \bigg|_{0}^{1} = 3 \Rightarrow \dfrac{1}{2} m + n = 3 \quad (1)
	$.\\
	$	\displaystyle\int\limits_{0}^{2} f(x) \mathrm{d}x = 8 \Rightarrow \left( \dfrac{m}{2} x^2 + nx \right) \bigg|_{0}^{2} = 8 \Rightarrow 2m + 2n = 8 \quad (2)
	$.\\	
	Từ (1) và (2) ta có hệ phương trình
$$
\heva{&\dfrac{1}{2} m + n = 3 \\
		&2m + 2n = 8 }
	\Rightarrow \heva{&	m = 2 \\&	n = 2.}
	$$	
Vậy $ m + n = 4$.
}
\end{ex}
\begin{ex}%[2D4V1-2]
	Biết rằng hàm số $f(x) = ax^2 + bx + c$ thỏa mãn $\displaystyle\int\limits_{0}^{1} f(x) \mathrm{d}x = -\dfrac{7}{2}$, $\displaystyle\int\limits_{0}^{2} f(x) \mathrm{d}x = -2$ và $\displaystyle\int\limits_{0}^{3} f(x) \mathrm{d}x = \dfrac{13}{2}$. Tính $P = a + b + c$. (\textit{làm tròn đến hàng phần trăm}).
	\shortans{$-1{,}33$}
\loigiai{
	Ta có
	$	\displaystyle\int f(x) \mathrm{d}x = \displaystyle\int (ax^2 + bx + c) \mathrm{d}x = \dfrac{a}{3} x^3 + \dfrac{b}{2} x^2 + cx + C
	$.\\	
	Lại có
	$	\displaystyle\int\limits_{0}^{1} f(x) \mathrm{d}x = -\dfrac{7}{2} \Rightarrow \left( \dfrac{a}{3} x^3 + \dfrac{b}{2} x^2 + cx \right) \bigg|_{0}^{1} = -\dfrac{7}{2} \Rightarrow \dfrac{1}{3} a + \dfrac{1}{2} b + c = -\dfrac{7}{2} \quad (1)
	$.	
	$	\displaystyle\int\limits_{0}^{2} f(x) \mathrm{d}x = -2 \Rightarrow \left( \dfrac{a}{3} x^3 + \dfrac{b}{2} x^2 + cx \right) \bigg|_{0}^{2} = -2 \Rightarrow \dfrac{8}{3} a + 2b + 2c = -2 \quad (2)
	$.\\	
	$	\displaystyle\int\limits_{0}^{3} f(x) \mathrm{d}x = \dfrac{13}{2} \Rightarrow \left( \dfrac{a}{3} x^3 + \dfrac{b}{2} x^2 + cx \right) \bigg|_{0}^{3} = \dfrac{13}{2} \Rightarrow 9a + \dfrac{9}{2} b + 3c = \dfrac{13}{2} \quad (3)
	$.\\	
	Từ (1), (2) và (3) ta có hệ phương trình:
	$$\heva{&\dfrac{1}{3} a + \dfrac{1}{2} b + c = -\dfrac{7}{2} \\&
		\dfrac{8}{3} a + 2b + 2c = -2 \\&
		9a + \dfrac{9}{2} b + 3c = \dfrac{13}{2}}
	\Rightarrow \heva{&	a = 1 \\&
		b = 3 \\&
		c = -\dfrac{16}{3}.}	$$	
Vậy	$P = a + b + c = 1 + 3 + \left( -\dfrac{16}{3} \right) = -\dfrac{4}{3}\approx -1{,}33$.
}
\end{ex}

% \begin{ex}%[2D4H1-2]
% 	Có hai giá trị của số thực $a$ là $a_1$, $a_2$ ($0 < a_1 < a_2$) thỏa mãn $\displaystyle\int\limits_{1}^{a} (2x - 3) \mathrm{d}x = 0$. Hãy tính $T = 3^{a_1} + 3^{a_2} + \log_2 \left( \dfrac{a_2}{a_1} \right)$.
% 	\shortans{$13$}
% \loigiai{
% 	Ta có
% 	$\displaystyle\int\limits_{1}^{a} (2x - 3) \mathrm{d}x = \left( x^2 - 3x \right) \bigg|_{1}^{a} = a^2 - 3a + 2	$.\\	
% 	Vì $\displaystyle\int\limits_{1}^{a} (2x - 3) \mathrm{d}x = 0$ nên $a^2 - 3a + 2 = 0$, suy ra $\hoac{&a = 1 \\ &a = 2.}$\\	
% 	Lại có $0 < a_1 < a_2$ nên $a_1 = 1$, $a_2 = 2$.\\	
% 	Như vậy $T = 3^{a_1} + 3^{a_2} + \log_2 \left( \dfrac{a_2}{a_1} \right) = 3^1 + 3^2 + \log_2 \left( \dfrac{2}{1} \right) = 13$.
% }
% \end{ex}
\begin{ex}%[2D4H1-2]
	Cho $\displaystyle\int\limits_{0}^{m} (3x^2 - 2x + 1) \mathrm{d}x = 6$. Tính giá trị của tham số $m$.
	\shortans{$2$}
\loigiai{
	Ta có\\
	$	\displaystyle\int\limits_{0}^{m} (3x^2 - 2x + 1) \mathrm{d}x = \left( x^3 - x^2 + x \right) \bigg|_{0}^{m} = m^3 - m^2 + m	$.\\	
	$
	\displaystyle\int\limits_{0}^{m} (3x^2 - 2x + 1) \mathrm{d}x = 6 \Leftrightarrow m^3 - m^2 + m - 6 = 0 \Leftrightarrow m = 2
	$.
}
\end{ex}
\begin{ex}%[2D4V1-2]
	Cho $I = \displaystyle\int\limits_{0}^{1} (4x - 2m^2) \mathrm{d}x$. Có bao nhiêu giá trị nguyên của $m$ để $I + 6 > 0$?
\shortans{$3$}
\loigiai{
	Theo định nghĩa tích phân ta có:
	$	I = \displaystyle\int\limits_{0}^{1} (4x - 2m^2) \mathrm{d}x = \left( 2x^2 - 2m^2 x \right) \bigg|_{0}^{1} = -2m^2 + 2
	$.\\
		Khi đó $I + 6 > 0 \Leftrightarrow -2m^2 + 2 + 6 > 0 \Leftrightarrow -2m^2 + 8 > 0  \Leftrightarrow -2 < m < 2$.\\	
	Mà $m$ là số nguyên nên $m \in \{-1; 0; 1\}$.\\	
	Vậy có $3$ giá trị nguyên của $m$ thỏa mãn yêu cầu.
}
\end{ex}
\begin{ex}%[2D4V1-2]
	Có bao nhiêu giá trị nguyên dương của $a$ để $\displaystyle\int\limits_{0}^{a} (2x - 3) \mathrm{d}x \leq 4$?
\shortans{$4$}
\loigiai{
	Ta có
	$
	\displaystyle\int\limits_{0}^{a} (2x - 3) \mathrm{d}x = \left( x^2 - 3x \right) \bigg|_{0}^{a} = a^2 - 3a
	$.\\	
	Khi đó
	$
	\displaystyle\int\limits_{0}^{a} (2x - 3) \mathrm{d}x \leq 4 \Leftrightarrow a^2 - 3a \leq 4 \Leftrightarrow -1 \leq a \leq 4
	$.\\	
	Mà $a \in \mathbb{N}^*$ nên $a \in \{1; 2; 3; 4\}$.\\	
	Vậy có $4$ giá trị của $a$ thỏa đề bài.
}
\end{ex}
\begin{ex}%[2D4V1-2]
	Có bao nhiêu số thực $b$ thuộc khoảng $(\pi; 3\pi)$ sao cho $\displaystyle\int\limits_{\pi}^{b} 4 \cos 2x \mathrm{d}x = 1$?
\shortans{$4$}
\loigiai{
	Ta có
	$
	\displaystyle\int\limits_{\pi}^{b} 4 \cos 2x \mathrm{d}x = 1 \Leftrightarrow 2 \sin 2x \bigg|_{\pi}^{b} = 1 \Leftrightarrow \sin 2b - \sin 2\pi = \dfrac{1}{2} \Leftrightarrow \sin 2b = \dfrac{1}{2}
	$.\\	
	$
	\Rightarrow 2b = \dfrac{\pi}{6} + k2\pi \quad \text{ hoặc } \quad 2b = \dfrac{5\pi}{6} + k2\pi
	$\\	
	$
	\Rightarrow b = \dfrac{\pi}{12} + k\pi \quad \text{ hoặc } \quad b = \dfrac{5\pi}{12} + k\pi\qquad (k\in \mathbb{Z})
	$.\\
Khi $	b = \dfrac{\pi}{12} + k\pi$, ta xét\\
\allowdisplaybreaks
\begin{eqnarray*}
&& \pi< \dfrac{\pi}{12} + k\pi<3\pi\\
&\Leftrightarrow& \dfrac{11}{12}<k<\dfrac{35}{12}\\
&\Leftrightarrow& k \in \{1;2\}.
\end{eqnarray*}
Khi $	b = \dfrac{5\pi}{12} + k\pi$, ta xét\\
\allowdisplaybreaks
\begin{eqnarray*}
&& \pi< \dfrac{5\pi}{12} + k\pi<3\pi\\
	&\Leftrightarrow& \dfrac{7}{12}<k<\dfrac{31}{12}\\
	&\Leftrightarrow& k \in \{1;2\}.
\end{eqnarray*}
	Vậy có $4$ số thực $b$ thỏa mãn yêu cầu bài toán.
}
\end{ex}
\Closesolutionfile{ans}
\indapan{2}{ans/ans-2C4B2CD2-KQ}
% \Opensolutionfile{ans}[ans/ans-2C4B2CD3-LC]
\begin{dang}{Ứng dụng tích phân trong thực tiễn}
    \begin{itemize}
        \item Cho hàm số$f\left(x \right)$ liên tục trên đoạn $\left[a;b \right]$. Khi đó $\dfrac{1}{b-a}\displaystyle\int\limits_a^b{f\left(x \right)dx}$ được gọi là giá trị trung bình của hàm số $f\left(x \right)$ trên đoạn $\left[a;b \right]$.
        \item Đạo hàm của quãng đường di chuyển của vật theo thời gian bằng tốc độ của chuyển động tại mọi thời điểm $v(t)=s'(t)$. Do đó, nếu biết tốc độ $v(t)$ tại mọi thời điểm $t\in \left[a;b \right]$ thì tính được quãng đường di chuyển trong khoảng thời gian từ $a$ đến $b$ theo công thức
        $$s=s\left(b \right)-s\left(a \right)=\displaystyle\int\limits_a^b v(t)\mathrm{\,d}t.$$
        \item Giả sử là vận tốc của vật tại thời điểm và là quãng đường vật đi được sau khoảng thời gian tính từ lúc bắt đầu chuyển động. Ta có mối liên hệ giữa vận tốc và quãng đường như sau
        \begin{itemize}
            \item Đạo hàm của quãng đường là vận tốc $s'(t)=v(t)$.
            \item Nguyên hàm của vận tốc là quãng đường $s(t)= \displaystyle\int v(t)\mathrm{\,d}t$.
        \end{itemize}
        $\Rightarrow$ Từ đây ta cũng có quãng đường vật đi được trong khoảng thời gian từ $a$ đến $b$ là 
        $$\displaystyle\int\limits_a^b v(t)\mathrm{\,d}t=s(b)-s(a).$$ 
        Nếu gọi $a(t)$ là gia tốc của vật thì ta có mối liên hệ giữa gia tốc và vận tốc như sau
        \begin{itemize}
            \item Đạo hàm của vận tốc là gia tốc $v'(t)=a(t)$.
            \item Nguyên hàm của gia tốc là vận tốc $v(t)= \displaystyle\int a(t)\mathrm{\,d}t$.
        \end{itemize}
    \end{itemize}
\end{dang}
\TN
\begin{ex}%[2D4H2-6] 
    Một ô tô đang chạy với vận tốc $10\,m/s$ thì gặp chướng ngại vật, người lái xe đạp phanh. Từ thời điểm đó, ô tô chuyển động chậm dần đều với vận tốc $v\,\left(t \right)=-2t+10\,\left(m/s \right)$, trong đó $t$ là khoảng thời gian tính bằng giây, kể từ lúc bắt đầu đạp phanh. Tính quãng đường ô tô di chuyển được trong $8$ giây cuối cùng.
    \choice
    {\True $55\,m$}
    {$25\,m$}
    {$50\,m$}
    {$16\,m$}
    \loigiai{
    Ta có $-2t+10=0\Leftrightarrow t=5\Rightarrow$ thời gian tính từ lúc bắt đầu đạp phanh đến khi dừng hẳn là $5$ giây.\\ 
    Vậy trong $8$ giây cuối cùng thì có $3$ giây ô tô chuyển động với vận tốc $10\,m/s$ và $5$ giây chuyển động chậm dần đều với vận tốc $v\left(t \right)=-2t+10\,\left(m/s \right)$.\\
    Khi đó quãng đường ô tô di chuyển là $$S=3\cdot 10+\displaystyle\int\limits_0^5 \left(-2t+10\right)\mathrm{\,d}t=30+25=55\,m.$$
    }
\end{ex}

\begin{ex}%[2D4H2-6]
    Một ô tô đang chạy với tốc độ $20\,\left(m/s \right)$ thì gặp chướng ngại vật, người lái đạp phanh, từ thời điểm đó ô tô chuyển động chậm dần đều với vận tốc $v\left(t \right)=-5t+20\,\left(m/s \right)$, trong đó $t$ là khoảng thời gian tính bằng giây, kể từ lúc bắt đầu đạp phanh. Hỏi từ lúc đạp phanh đến khi dừng hẳn, ô tô còn di chuyển bao nhiêu mét ($m$)?
    \choice
    {$20\,m$}
    {$30\,m$}
    {$10\,m$}
    {\True $40\,m$}
    \loigiai{
    Khi ô tô dừng hẳn thì $v\left(t \right)=0\Leftrightarrow-5t+20=0\Leftrightarrow t=4\,\left(s \right)$.\\
    Vậy từ lúc đạp phanh đến khi dừng hẳn, ô tô di chuyển được 
    $$s=\displaystyle\int\limits_0^4 \left(-5t+20\right)\mathrm{\,d}t=40\,\left(m \right).$$
}
\end{ex}

% \begin{ex}%[2D4V2-6]
%     Một chất điểm $A$ xuất phát từ $O$, chuyển động thẳng với vận tốc biến thiên theo thời gian bởi quy luật $v\left(t \right)=\dfrac{1}{120}t^2+\dfrac{58}{45}t\,\left(m/s \right)$, trong đó $t$ (giây) là khoảng thời gian tính từ lúc $A$ bắt đầu chuyển động. Từ trạng thái nghỉ, một chất điểm $B$ cũng xuất phát từ $O$, chuyển động thẳng cùng hướng với $A$ nhưng chậm hơn $3$ giây so với $A$ và có gia tốc bằng $a\,\left(m/s^2 \right)$ ($a$ là hằng số). Sau khi $B$ xuất phát được $15$ giây thì đuổi kịp $A$. Vận tốc của $B$ tại thời điểm đuổi kịp $A$ bằng
%     \choice
%     {$21\,\left(m/s \right)$}
%     {$25\,\left(m/s \right)$}
%     {$36\,\left(m/s \right)$}
%     {\True $30\,\left(m/s \right)$}
%     \loigiai{
%     Thời điểm chất điểm $B$ đuổi kịp chất điểm $A$ thì chất điểm $B$ đi được $15$ giây, chất điểm $A$ đi được $18$ giây.\\
%     Biểu thức vận tốc của chất điểm $B$ có dạng $v_B\left(t \right)=\displaystyle\int a\mathrm{\,d}t =at+C$ mà $v_B\left(0\right)=0$ nên $v_B\left(t \right)=at$.\\
%     Do từ lúc chất điểm $A$ bắt đầu chuyển động cho đến khi chất điểm $B$ đuổi kịp thì quãng đường hai chất điểm đi được bằng nhau.\\
%     Do đó $\displaystyle\int\limits_0^{18} \left(\dfrac{1}{120}t^2+\dfrac{58}{45} \right)\mathrm{\,d}t=\displaystyle\int\limits_0^{15} at\mathrm{\,d}t \Leftrightarrow 225=a\cdot\dfrac{225}{2}\Leftrightarrow a=2$.\\
%     Vậy vận tốc của chất điểm $B$ tại thời điểm đuổi kịp $A$ bằng 
%     $$v_B\left(t \right)=2\cdot 15=30\,\left(m/s \right).$$
%     }
% \end{ex}

\begin{ex}%[2D4V2-6]
    Một chất điểm $A$ xuất phát từ $O$, chuyển động thẳng với vận tốc biến thiên theo thời gian bởi quy luật $v\left(t \right)=\dfrac{1}{150}t^2+\dfrac{59}{75}t\,\left(m/s \right)$, trong đó $t$ (giây) là khoảng thời gian tính từ lúc $a$ bắt đầu chuyển động. Từ trạng thái nghỉ, một chất điểm $B$ cũng xuất phát từ $O$, chuyển động thẳng cùng hướng với $A$ nhưng chậm hơn $3$ giây so với $A$ và có gia tốc bằng $a\,\left(m/s^2 \right)$ ($a$ là hằng số). Sau khi $B$ xuất phát được $12$ giây thì đuổi kịp $A$. Vận tốc của $B$ tại thời điểm đuổi kịp $A$ bằng
    \choice
    {$15\,\left(m/s \right)$}
    {$20\,\left(m/s \right)$}
    {\True $16\,\left(m/s \right)$}
    {$13\,\left(m/s \right)$}
    \loigiai{
    Quãng đường chất điểm $A$ đi từ đầu đến khi $B$ đuổi kịp là 
    $$S=\displaystyle\int\limits_0^{15} \left(\dfrac{1}{150}t^2+\dfrac{59}{75}t \right)\mathrm{\,d}t=96\,\left(m \right).$$
    Vận tốc của chất điểm $B$ là 
    $$v_B\left(t \right)=\displaystyle\int a\mathrm{\,d}t=at+C.$$
    Tại thời điểm $t=3$ vật $B$ bắt đầu từ trạng thái nghỉ nên $v_B\left(3\right)=0\Leftrightarrow C=-3a$.\\
    Lại có quãng đường chất điểm $B$ đi được đến khi gặp $A$ là 
    $$S_2=\displaystyle\int\limits_3^{15} \left(at-3a \right)\mathrm{\,d}t=\left. \left(\dfrac{at^2}{2}-3at \right) \right|_3^{15}=72a\,\left(m \right).$$
    Vậy $72a=96\Leftrightarrow a=\dfrac{4}{3}\,\left(m/s^2 \right)$.\\
    Tại thời điểm đuổi kịp $A$ thì vận tốc của $B$ là $v_B\left(15\right)=16\,\left(m/s \right)$.
    }
\end{ex}

\begin{ex}%[2D4V2-6]
    Một ô tô bắt đầu chuyển động thẳng đều với vận tốc $v_0$, sau $6$ giây chuyển động thì gặp chướng ngại vật nên bắt đầu giảm tốc độ với vận tốc chuyển động $v(t)=-\dfrac{5}{2}t+a\,(m/s)$ với $t\ge 6$ cho đến khi dừng hẳn. Biết rằng kể từ lúc chuyển động đến lúc dừng hẳn thì ô tô đi được quãng đường là $80\,m$. Tìm $v_0$.
    \choice
    {$v_0=35\,m/s$}
    {$v_0=25\,m/s$}
    {\True $v_0=10\,m/s$}
    {$v_0=20\,m/s$}
    \loigiai{
    Tại thời điểm $t=6$ vật đang chuyển động với vận tốc $v_0$ nên có 
    $$v(6)=v_0 \Leftrightarrow -\dfrac{5}{2}\cdot 6+a=v_0 \Leftrightarrow a=v_0+15 \Rightarrow v(t)=-\dfrac{5}{2}t+v_0+15.$$
    Gọi $k$ là thời điểm vật dừng hẳn, ta có 
    $$v(k)=0 \Leftrightarrow k=\dfrac{2}{5}\cdot\left(v_0+15\right)\Leftrightarrow k=\dfrac{2v_0}{5}+6.$$
    Tổng quãng đường vật đi được là 
    \allowdisplaybreaks 
    \begin{eqnarray*}
        && 80=6\cdot v_0+\displaystyle\int\limits_6^k \left(-\dfrac{5}{2}t+v_0+15\right)\mathrm{\,d}t\\
        &\Leftrightarrow& 80=6\cdot v_0+\left. \left(-\dfrac{5}{4}t^2+v_0\cdot t+15t \right) \right|_6^k \\ 
        &\Leftrightarrow& 80=6\cdot v_0-\dfrac{5}{4}\left(k^2-6^2\right)+v_0\cdot (k-6)+15(k-6) \\ 
        &\Leftrightarrow& 80=6\cdot v_0-\dfrac{5}{4}\left(\dfrac{4\left(v_0 \right)^2}{25}+\dfrac{24v_0}{5} \right)+v_0\cdot\dfrac{2v_0}{5}+15\cdot\dfrac{2v_0}{5} \\
        &\Leftrightarrow& \left(v_0 \right)^2+36\cdot v_0-400=0\\ 
        &\Leftrightarrow& v_0=10. 
    \end{eqnarray*}
    }
\end{ex}

\begin{ex}%[2D4H2-6]
    Để đảm bảo an toàn khi lưu thông trên đường, các xe ô tô khi dừng đèn đỏ phải cách nhau tối thiểu $1\,m$. Một ô tô $A$ đang chạy với vận tốc $16\,m/s$ bỗng gặp ô tô $B$ đang dừng đèn đỏ nên ô tô $A$ hãm phanh và chuyển động chậm dần đều với vận tốc được biểu thị bởi công thức $v_A\left(t \right)=16-4t$ (đơn vị tính bằng $m/s$), thời gian tính bằng giây. Hỏi rằng để hai ô tô $A$ và $B$ đạt khoảng cách an toàn khi dừng lại thì ô tô $A$ phải hãm phanh khi cách ô tô $B$ một khoảng ít nhất là bao nhiêu?
    \choice
    {$33$}
    {$12$}
    {$31$}
    {\True $32$}
    \loigiai{
    Ta có $v_A\left(0\right)=16\,m/s$.\\
    Khi xe $A$ dừng hẳn $v_A\left(t \right)=0 \Leftrightarrow t=4\,s$.\\
    Quãng đường từ lúc xe $A$ hãm phanh đến lúc dừng hẳn là 
    $$s=\displaystyle\int\limits_0^4 \left(16-4t \right)\mathrm{\,d}t=32\,m.$$
    }
\end{ex}

\begin{ex}%[2D4H2-6]
    Do các xe phải cách nhau tối thiểu $1\,m$ để đảm bảo an toàn nên khi dừng lại ô tô $A$ phải hãm phanh khi cách ô tô $B$ một khoảng ít nhất là $33\,m$. Một chất điểm đang chuyển động với vận tốc $v_0=15\,m/s$ thì tăng tốc với gia tốc $a\left(t \right)=t^2+4t\, \left(m/s^2\right)$. Tính quãng đường chất điểm đó đi được trong khoảng thời gian $3$ giây kể từ lúc bắt đầu tăng vận tốc.
    \choice
    {$70{,}25\, {m}$}
    {$68{,}25\, {m}$}
    {$67{,}25\, {m}$}
    {\True $69{,}75\, {m}$}
    \loigiai{
    Ta có 
    $$a\left(t \right)=t^2+4t \Rightarrow v\left(t \right)=\displaystyle\int a\left(t \right)\mathrm{\,d}t=\dfrac{t^3}{3}+2t^2+C,\, \left(C\in \mathbb{R} \right).$$
    Mà $v\left(0\right)=C=15 \Rightarrow v\left(t \right)=\dfrac{t^3}{3}+2t^2+15$.\\
    Vậy $S=\displaystyle\int\limits_0^3 \left(\dfrac{t^3}{3}+2t^2+15\right)\mathrm{\,d}t=69{,}75\, {m}$.
    }
\end{ex}

\begin{ex}%[2D4V2-6]
    Một vật chuyển động với vận tốc $10\,m/s$ thì tăng tốc với gia tốc được tính theo thời gian là $a\left(t \right)=t^2+3t$. Tính quãng đường vật đi được trong khoảng thời gian $6$ giây kể từ khi vật bắt đầu tăng tốc.
    \choice
    {$136\,{m}$}
    {$126\,{m}$}
    {$276\,{m}$}
    {\True $216\,{m}$}
    \loigiai{
    Ta có $v\left(0\right)=10\,m/s$ và 
    $$v\left(t \right)=\displaystyle\int\limits_0^t a\left(t \right)\mathrm{\,d}t=\displaystyle\int\limits_0^t \left(t^2+3t \right)\mathrm{\,d}t=\left. \left(\dfrac{t^3}{3}+\dfrac{3t^2}{2} \right) \right|_0^t=\dfrac{1}{3}t^3+\dfrac{3}{2}t^2.$$
    Quãng đường vật đi được là 
    $$S=\displaystyle\int\limits_0^6 v\left(t \right)\mathrm{\,d}t=\displaystyle\int\limits_0^6 \left(\dfrac{1}{3}t^3+\dfrac{3}{2}t^2 \right)\mathrm{\,d}t=\left. \left(\dfrac{1}{12}t^4+\dfrac{1}{2}t^3 \right) \right|_0^6=216\,{m}.$$
    }
\end{ex}

% \begin{ex}%[2D4V2-6]
%     Một chiếc máy bay chuyển động trên đường băng với vận tốc $v\left(t \right)=t^2+10t$ $\left(m/s \right)$ với $t$ là thời gian được tính theo đơn vị giây kể từ khi máy bay bắt đầu chuyển động. Biết khi máy bay đạt vận tốc $200\,\left(m/s \right)$ thì rời đường băng. Quãng đường máy bay đã di chuyển trên đường băng là
%     \choice
%     {\True $\dfrac{2500}{3}\,\left(m \right)$}
%     {$2000\,\left(m \right)$}
%     {$500\,\left(m \right)$}
%     {$\dfrac{4000}{3}\,\left(m \right)$}
%     \loigiai{
%     Thời điểm máy bay đạt vận tốc $200\,\left(m/s \right)$ là 
%     $$v\left(t \right)=200 \Leftrightarrow t^2+10t=200 \Leftrightarrow \hoac{& t=10\\ & t=-20}\Leftrightarrow t=10.$$
%     Quãng đường máy bay đã di chuyển trên đường băng là
%     $$s=\displaystyle\int\limits_0^{10} \left(t^2+10t \right)\mathrm{\,d}t=\left.\left(\dfrac{t^3}{3}+5t \right)\right|_0^{10}=\dfrac{2500}{3}\,\left(m \right).$$
%     }
% \end{ex}

\begin{ex}%[2D4V2-6]
    Một ô tô bắt đầu chuyển động nhậnh dần đều với vận tốc $v_1\left(t \right)=7t\,\left(m/s \right)$. Đi được $5\,s$, người lái xe phát hiện chướng ngại vật và phanh gấp, ô tô tiếp tục chuyển động chậm dần đều với gia tốc $a=-70\,\left(m/s^2 \right)$. Tính quãng đường $S$ đi được của ô tô từ lúc bắt đầu chuyển bánh cho đến khi dừng hẳn.
    \choice
    {\True $S=96{,}25\,\left(m\right)$}
    {$S=87{,}5\,\left(m\right)$}
    {$S=94\,\left(m\right)$}
    {$S=95{,}7\,\left(m\right)$}
    \loigiai{
    Chọn gốc thời gian là lúc ô tô bắt đầu đi.\\ 
    Sau $5\,s$ ô tô đạt vận tốc là $v\left(5\right)=35\,\left(m/s\right)$.\\
    Sau khi phanh vận tốc ô tô là $v\left(t\right)=35-70\left(t-5\right)$.\\
    Ô tô dừng tại thời điểm $t=5{,}5\,s$.\\
    Quãng đường ô tô đi được là 
    $$S=\displaystyle\int\limits_0^5 7t\mathrm{\,d}t+\displaystyle\int\limits_5^{5{,}5} \left[35-70\left(t-5\right) \right]\mathrm{\,d}t=96{,}25\,\left(m\right).$$
    }
\end{ex}

\begin{ex}%[2D4V2-6]
    Một ô tô bắt đầu chuyển động nhanh dần đều với vận tốc $v_1\left(t \right)=2t\,\left(m/s\right)$. Đi được $12$ giây, người lái xe gặp chướng ngại vật và phanh gấp, ô tô tiếp tục chuyển động chậm dần đều với gia tốc $a=-12\,\left(m/s^2\right)$. Tính quãng đường $s\left(m\right)$ đi được của ôtô từ lúc bắt đầu chuyển động đến khi dừng hẳn.
    \choice
    {\True $s=168\,\left(m\right)$}
    {$s=166\,\left(m\right)$}
    {$s=144\,\left(m\right)$}
    {$s=152\,\left(m\right)$}
    \loigiai{
    \textbf{Giải đoạn 1:} Xe bắt đầu chuyển động đến khi gặp chướng ngại vật.\\
    Quãng đường xe đi được là
    $$S_1=\displaystyle\int\limits_0^{12} v_1\left(t \right)\mathrm{\,d}t=\displaystyle\int\limits_0^{12} 2t\mathrm{\,d}t =\left. t^2 \right|_0^{12}=144\,\left(m\right).$$
    \textbf{Giải đoạn 2:} Xe gặp chướng ngại vật đến khi dừng hẳn.\\
    Ôtô chuyển động chậm dần đều với vận tốc 
    $$v_2\left(t \right)=\displaystyle\int a\mathrm{\,d}t=-12t+c.$$
    Vận tốc của xe khi gặp chướng ngại vật là $$v_2\left(0\right)=v_1\left(12\right)=2\cdot 12=24\,\left(m/s\right).$$
    Suy ra $-12\cdot 0+c=24 \Rightarrow c=24\Rightarrow v_2\left(t \right)=-12t+24$.\\
    Thời gian khi xe gặp chướng ngại vật đến khi xe dừng hẳn là nghiệm phương trình
    $$-12t+24=0\Leftrightarrow t=2.$$
    Khi đó, quãng đường xe đi được là
    $$S_2=\displaystyle\int\limits_0^2 v_2\left(t \right)\mathrm{\,d}t=\displaystyle\int\limits_0^2 \left(-12t+24\right)\mathrm{\,d}t=\left. \left(-6t^2+24t \right) \right|_0^2=24\,\left(m\right).$$
    Vậy tổng quãng đường xe đi được là $S=S_1+S_2=168\,\left(m\right)$.
    }
\end{ex}
\begin{ex}%Cau12D%[2D4H2-6]
	Một ô tô đang dừng và bắt đầu chuyển động theo một đường thẳng với gia tốc $a\left(t\right)= 6-2t$ (m/s$^2$), trong đó $t$ là khoảng thời gian tính bằng giây kể từ lúc ô tô bắt đầu chuyển động. Hỏi quảng đường ô tô đi được từ lúc bắt đầu chuyển động đến khi vận tốc của ô tô đạt giá trị lớn nhất là bao nhiêu mét?
	\choice
	{\True $18$ m}
	{$36$ m}
	{$22{,}5$ m}
	{$6{,}75$ m}
	\loigiai{
		$a\left(t\right) = 6-2t$ (m/s$^2$) $\Rightarrow v\left(t\right) = \displaystyle\int \left(6-2t\right) \mathrm{\,d}t = 6t - t^2+C$.\\
		Xe dừng và bắt đầu chuyển động nên khi $t=0$ thì $v=0 \Rightarrow C=0 \Rightarrow v\left(t\right) = 6t-t^2$.\\
		$v\left(t\right) = 6t-t^2$ là hàm số bậc $2$ nên đạt giá trị lớn nhất khi $t=-\dfrac{b}{2a}=3$ (s).\\
		Quãng đường xe đi trong $3$ giây đầu là: $S= \displaystyle\int\limits_0^3 \left(6t-t^2\right) \mathrm{\,d}t = 18$ (m).
	}
\end{ex}

% \begin{ex}%Cau13D%[2D4V2-6]
% 	Một chất điểm $A$ xuất phát từ $O$, chuyển động thẳng với vận tốc biến thiên theo thời gian bởi quy luật $v \left(t\right) = \dfrac{1}{180}t^2 + \dfrac{11}{18}t$ (m/s), trong đó $t$ (giây) là khoảng thời gian tính từ lúc $A$ bắt đầu chuyển động. Từ trạng thái nghỉ, một chất điểm $B$ cũng xuất phát từ $O$, chuyển động thẳng cùng hướng với $A$ nhưng chậm hơn $5$ giây so với $A$ và có gia tốc bằng $a$ (m/s$^2$) ($a$ là hằng số). Sau khi $B$ xuất phát được $10$ giây thì đuổi kịp $A$. Vận tốc của $B$ tại thời điểm đuổi kịp $A$ bằng
% 	\choice
% 	{\True $15$ (m/s)}
% 	{$10$ (m/s)}
% 	{$7$ (m/s)}
% 	{$22$(m/s)}
% 	\loigiai{
% 		Thời gian tính từ khi $A$ xuất phát đến khi bị $B$ đuổi kịp là $15$ giây, suy ra quãng đường đi được tới lúc đó là:
% 		$$\displaystyle\int\limits_0^{15} v\left(t\right) \mathrm{\,d}t = \displaystyle\int\limits_0^{15} \left(\dfrac{1}{180}t^2 + \dfrac{11}{18}t \right) \mathrm{\,d}t= \left(\dfrac{1}{540}t^3 + \dfrac{11}{36}t^2 \right)\Big|_0^{15} = 75 \left(\text{m}\right).$$
% 		Vận tốc của chất điểm $B$ là $y\left(t\right) = \displaystyle\int a \mathrm{\,d}t = a \cdot t+C$ ( $C$ là hằng số); do $B$ xuất phát từ trạng thái nghỉ nên có $y\left(0\right)=0 \Leftrightarrow C=0$.\\
% 		Quãng đường của $B$ từ khi xuất phát đến khi đuổi kịp $A$ là
% 		$$\displaystyle\int\limits_0^{10} y\left(t\right) \mathrm{\,d}t = 75 \Leftrightarrow \displaystyle\int\limits_0^{10} a \cdot t \mathrm{\,d}t =75 \Leftrightarrow \dfrac{a \cdot t^2}{2}\Big|_0^{10} = 75 \Leftrightarrow 50a=75 \Leftrightarrow a = \dfrac{3}{2}.$$
% 		Vậy có $y\left(t\right) = \dfrac{3t}{2}$; suy ra vận tốc của $B$ tại thời điểm đuổi kịp $A$ bằng $y\left(10\right) = 15$ (m/s).
% 	}
% \end{ex}

% \begin{ex}%Cau14D%[2D4V2-6]
% 	Một vật chuyển động trong $3$ giờ với vận tốc $v$ (km/h) phụ thuộc thời gian $t$ (h) có đồ thị là một phần của đường parabol có đỉnh $I\left(2;9\right)$ và trục đối xứng song song với trục tung như hình bên. Tính quãng đường $s$ mà vật di chuyển được trong $3$ giờ đó.
% 	\begin{center}
% 		\begin{tikzpicture}[>=stealth, font=\footnotesize, line join=round, line cap=round, thick, smooth, samples=250, scale=0.6, yscale=.7]
% 			% Vẽ 2 trục, điền các số lên trục
% 			\draw[->] (-0.5,0)--(0,0) node[below left]{$O$}--(4,0) node[above]{$t$};
% 			\foreach \x in {2,3}
% 			\draw[shift={(\x,0)},color=black] (0pt,2pt)--(0pt,-2pt) 
% 			node[below] { $\x$};
% 			\draw[->,color=black] (0,-0.5)--(0,10) node[right]{$v$};
% 			\foreach \y in {6,9}
% 			\draw[shift={(0,\y)},color=black] (2pt,0pt) -- (-2pt,0pt) 
% 			node[left] {$\y$};
% 			\clip(-1,-1) rectangle (3,10); %vùng đồ thị
% 			%\draw[gray!50,thin,opacity=.5] (-1,-1) grid (4,10); %ô vuông
% 			%Vẽ đồ thị
% 			\draw[smooth,samples=100,domain=0:10] 
% 			plot(\x,{(-0.75)*(\x)^2+3*(\x)+6});
% 			\draw[dashed] (3,0)--(3,8.25) circle(1.5pt);  \draw[dashed] (2,0)--(2,9) circle(1.5pt) node[above]{$I$}--(0,9) circle(1.5pt);
% 			% Vẽ thêm mấy cái râu ria
			
% 			%Vẽ dấu chấm tròn 
% 			\fill (0cm,0cm) circle (1.5pt); 
% 		\end{tikzpicture}
% 	\end{center}
% 	\choice
% 	{$s = 25{,}25$ (km)}
% 	{$s = 24{,}25$ (km)}
% 	{\True $s = 24{,}75$ (km)}
% 	{$s = 26{,}75$ (km)}
% 	\loigiai{Gọi $v \left(t\right) = at^2 +bt +c$.\\
% 		Đồ thị $v\left(t\right)$ là một phần parabol có đỉnh $I\left(2;9\right)$ và đi qua điểm $A\left(0;6\right)$ nên\\
% 		$\heva{&\dfrac{-b}{2a} = 2\\&a \cdot 2^2 +b\cdot 2 +c=9\\&a \cdot 0^2 + b \cdot 0+c =6} \Rightarrow \heva{&a = -\dfrac{3}{4}\\&b=3\\&c=6}$. Tìm được $v\left(t\right) = -\dfrac{3}{4}t^2 + 3t +6$.\\
% 		Vậy $S = \displaystyle\int\limits_0^{3} \left(-\dfrac{3}{4} t^2 + 3t +6 \right) \mathrm{\,d}t = 24{,}75$ (km).
% 	}
% \end{ex}

\begin{ex}%Cau15D%[2D4H2-6]
	Một vật chuyển động trong $3$ giờ với vận tốc $v$ (km/h) phụ thuộc vào thời gian $t$ (h) có đồ thị vận tốc như hình bên. Trong thời gian $1$ giờ kể từ khi bắt đầu chuyển động, đồ thị đó là một phần của đường parabol có đỉnh $I\left(2;9\right)$ và trục đối xứng song song với trục tung, khoảng thời gian còn lại đồ thị là một đoạn thẳng song song với trục hoành. Tính quãng đường $s$ mà vật chuyển động được trong $3$ giờ đó (kết quả làm tròn đến hàng phần trăm).
	\begin{center}
		\begin{tikzpicture}[>=stealth,scale=0.6, yscale=.7]
			% Vẽ 2 trục, điền các số lên trục
			\draw[->] (-0.5,0)--(0,0) node[below left]{$O$}--(4,0) node[above]{$t$}; %định dạng trục Ox
			\foreach \x in {1,2,3}
			\draw[shift={(\x,0)},color=black] (0pt,2pt)--(0pt,-2pt) 
			node[below] { $\x$};
			\draw[->,color=black] (0,-0.5)--(0,10) node[right]{$v$};  %định dạng trục Oy
			\foreach \y in {4,9}
			\draw[shift={(0,\y)},color=black] (2pt,0pt) -- (-2pt,0pt) 
			node[left] {$\y$};
			\clip(-1,-1) rectangle (3,10); %vùng đồ thị
			%\draw[gray!50,thin,opacity=.5] (-1,-1) grid (4,10); %ô vuông
			%Vẽ đồ thị
			\draw[smooth,samples=100,domain=0:1,font=\footnotesize, line join=round, line cap=round, thick] 
			plot(\x,{(-5/4)*(\x)^2+5*(\x)+4});
			\draw[smooth,samples=100,domain=1:3,dashed] 
			plot(\x,{(-5/4)*(\x)^2+5*(\x)+4});
			\draw[smooth,samples=100,domain=1:3,font=\footnotesize, line join=round, line cap=round, thick] 
			plot(\x,{31/4});
			% Vẽ thêm mấy cái râu ria
			\draw[dashed] (3,0)--(3,31/4) circle(1.5pt);  \draw[dashed] (2,0)--(2,9) circle(1.5pt) node[above]{$I$}--(0,9) circle(1.5pt); \draw[dashed] (1,0)--(1,31/4) circle(1.5pt) --(0,31/4);
			%Vẽ dấu chấm tròn 
			\fill (0cm,0cm) circle (1.5pt); 
		\end{tikzpicture}
	\end{center}
	\choice
	{\True $s = 21{,}58$ (km)}
	{$s = 23{,}25$ (km)}
	{$s = 13{,}83$ (km)}
	{$s = 15{,}50$ (km)}
	\loigiai{Gọi phương trình parabol $v = at^2+bt+c$ ta có hệ như sau
		$$\heva{&c=4\\&4a+2b+c=9\\&-\dfrac{b}{2a}=2} \Leftrightarrow \heva{&b=5\\&c=4\\&a=-\dfrac{5}{4}.}$$
		Với $t=1$ ta có $v = \dfrac{31}{4}$.
		Vậy quãng đường vật chuyển động được là
		$$s = \displaystyle\int\limits_0^1 \left(-\dfrac{5}{4}t^2 + 5t +4 \right) \mathrm{\,d}t + \displaystyle\int\limits_1^3 \dfrac{31}{4} \mathrm{\,d}t = \dfrac{259}{12} \approx 21{,}58.$$
	}
\end{ex}

% \begin{ex} %Cau16D %[2D4H2-6]
% 	Một người chạy trong $2$ giờ, vận tốc $v$ (km/h) phụ thuộc vào thời gian $t$ (h) có đồ thị là $1$ phần của đường Parabol với đỉnh $I\left(1;5\right)$ và trục đối xứng song song với trục tung $Ov$ như hình vẽ. Tính quảng đường $S$ người đó chạy được trong $1$ giờ $30$ phút kể từ lúc bắt đầu chạy (kết quả làm tròn đến $2$ chữ số thập phân).
% 	\begin{center}
% 		\begin{tikzpicture}[>=stealth, font=\footnotesize, line join=round, line cap=round, thick, smooth, samples=250, scale=0.7]
% 			% Vẽ 2 trục, điền các số lên trục
% 			\draw[->] (-0.5,0)--(0,0) node[below left]{$O$}--(3,0) node[above]{$t$}; %định dạng trục Ox
% 			\foreach \x in {2,1}
% 			\draw[shift={(\x,0)},color=black] (0pt,2pt)--(0pt,-2pt) 
% 			node[below] { $\x$};
% 			\draw[->,color=black] (0,-0.5)--(0,6) node[right]{$v$};  %định dạng trục Oy
% 			\foreach \y in {5}
% 			\draw[shift={(0,\y)},color=black] (2pt,0pt) -- (-2pt,0pt) 
% 			node[left] {$\y$};
% 			\clip(-0.5,-0.5) rectangle (3,6); %vùng đồ thị
% 			%\draw[gray!50,thin,opacity=.5] (-1,-1) grid (4,10); %ô vuông
% 			%Vẽ đồ thị
% 			\draw[smooth,samples=100,domain=0:2] 
% 			plot(\x,{-5*(\x)^2+10*(\x)});
% 			% Vẽ thêm mấy cái râu ria
% 			\draw[dashed] (1,0)--(1,5) circle(1.5pt)--(0,5);
% 			%Vẽ dấu chấm tròn 
% 			\fill (0cm,0cm) circle (1.5pt); 
% 		\end{tikzpicture} 
% 	\end{center}
% 	\choice
% 	{$2{,}11$ km}
% 	{$6{,}67$ km}
% 	{\True $5{,}63$ km}
% 	{$6{,}63$ km}
% 	\loigiai{
% 		Ta có $1$ giờ $30$ phút = $1,5$ giờ $\Rightarrow S = \displaystyle\int\limits_0^{1,5} v\left(t\right) \mathrm{\,d}t$.\\
% 		Đồ thị $v = v\left(t\right)$ đi qua gốc tọa độ nên $v\left(t\right)$ có dạng $v\left(t\right) = at^2+bt$.\\
% 		Đồ thị $v\left(t\right)$ có đỉnh là $I\left(1;5\right)$ nên $\heva{&-\dfrac{b}{2a}=1\\&a+b=5} \Leftrightarrow \heva{&b=-2a\\&a+b=5} \Leftrightarrow \heva{&a=-5\\&b=10.}$\\
% 		Suy ra $v\left(t\right) = -5t^2+10$. Do đó
% 		$$S = \displaystyle\int\limits_0^{1,5} \left(-5t^2+10\right) \mathrm{\,d}t = \dfrac{45}{8} \approx 5{,}63.$$
% 	}
% \end{ex}
%--------------------------------------------------------------------
\Closesolutionfile{ans}
\indapan{6}{ans/ans-2C4B2CD3-LC}
\TNSA
\Opensolutionfile{ans}[ans/ans-2C4B2CD3-KQ]
\begin{ex}%Cau17D%[2D4H2-6] 
	Một ô tô đang chạy với vận tốc là $12$ (m/s) thì người lái đạp phanh; từ thời điểm đó ô tô chuyển động chậm dần đều với vận tốc $v \left(t\right) = -6t+12$ (m/s), trong đó $t$ là khoảng thời gian tính bằng giây kể từ lúc đạp phanh. Hỏi từ lúc đạp phanh đến lúc ô tô dừng hẳn, ô tô còn di chuyển được bao nhiêu mét?
	\shortans{$12$}
	\loigiai{
		Lấy mốc thời gian $\left(t=0\right)$ là lúc đạp phanh.\\
		Khi ô tô dừng hẳn thì vận tốc $v\left(t\right)=0$, tức là $v\left(t\right) = -6t+12 = 0 \Leftrightarrow t =2$.\\
		Vậy từ lúc đạp phanh đến lúc ô tô dừng hẳn, ô tô còn di chuyển được quãng đường là:
		$$\displaystyle\int\limits_0^2 \left(-6t+12\right) \mathrm{\,d}t = \left(-3t^2 +12t \right) \Big|_0^2 = 12 \left(\text{m}\right).$$
	}
\end{ex}

\begin{ex}%Cau18D%[2D4H2-6]
	Một ô tô đang chạy với vận tốc $10$ m/s thì người lái đạp phanh; từ thời điểm đó, ô tô chuyển động chậm dần đều với vận tốc $v \left(t\right) = -5t+10$ (m/s), trong đó $t$ là khoảng thời gian tính bằng giây, kể từ lúc bắt đầu đạp phanh. Hỏi từ lúc đạp phanh đến khi dừng hẳn, ô tô còn di chuyển bao nhiêu mét?
	\shortans{$10$}
	\loigiai{
		Xét phương trình $-5t+10=0 \Leftrightarrow t=2$. Do vậy, kể từ lúc người lái đạp phanh thì sau $2s$ ô tô dừng hẳn.\\
		Quãng đường ô tô đi được kể từ lúc người lái đạp phanh đến khi ô tô dừng hẳn là:
		$$s = \displaystyle\int\limits_0^2 \left(-5t+10\right) \mathrm{\,d}t = \left(-\dfrac{5}{2}t^2 + 10t \right) \Big|_0^2 = 10 (\text{m}).$$
	}
\end{ex}

% \begin{ex}%Cau19D%[2D4V2-6]
% 	Một chất điểm $A$ xuất phát từ $O$, chuyển động thẳng với vận tốc biến thiên theo thời gian bởi quy luật $v \left(t\right) = \dfrac{1}{100}t^2 + \dfrac{13}{30}t$ (m/s), trong đó $t$ (giây) là khoảng thời gian tính từ lúc $A$ bắt đầu chuyển động. Từ trạng thái nghỉ, một chất điểm $B$ cũng xuất phát từ $O$, chuyển động thẳng cùng hướng với $A$ nhưng chậm hơn $10$ giây so với $A$ và có gia tốc bằng $a$ (m/s$^2$ ) ( $a$ là hằng số). Sau khi $B$ xuất phát được $15$ giây thì đuổi kịp $A$. Vận tốc của $B$ tại thời điểm đuổi kịp $A$ bằng bao nhiêu m/s?
% 	\shortans{$25$}
% 	\loigiai{
% 		Ta có $v_{B}(t) = \displaystyle\int a \cdot \mathrm{\,d}t = at + C$, $v_{B} (0) = 0 \Rightarrow C = 0 \Rightarrow v_{B} \left(t\right) = at$.\\
% 		Quãng đường chất điểm $A$ đi được trong $25$ giây là
% 		$$S_{A} = \displaystyle\int\limits_0^{25} \left(\dfrac{1}{100}t^2 + \dfrac{13}{30}t \right) \mathrm{\,d}t = \left(\dfrac{1}{300}t^3 + \dfrac{13}{60}t^2 \right) \Big|_0^{25} = \dfrac{375}{2}.$$
% 		Quãng đường chất điểm $B$ đi được trong $15$ giây là
% 		$$S_{B} = \displaystyle\int\limits_0^{15} at \cdot \mathrm{\,d}t = \dfrac{at^2}{2} \Big|_0^{15} = \dfrac{225a}{2}.$$
% 		Ta có $\dfrac{375}{2} = \dfrac{225a}{2} \Leftrightarrow a = \dfrac{5}{3}$.\\
% 		Vận tốc của $B$ tại thời điểm đuổi kịp $A$ là $v_{B} \left(15\right) = \dfrac{5}{3} \cdot 15 = 25$ (m/s).
% 	}
% \end{ex}

\begin{ex}%Cau20D%[2D4H2-6] 
	Một ô tô chuyển động nhanh dần đều với vận tốc $v \left(t\right) = 7t$ (m/s). Đi được $5$ (s) người lái xe phát hiện chướng ngại vật và phanh gấp, ô tô tiếp tục chuyển động chậm dần đều với gia tốc $a = -35$ (m/s$^2$). Tính quãng đường của ô tô đi được từ lúc bắt đầu chuyển bánh cho đến khi dừng hẳn (đơn vị tính bằng mét)?
	\shortans{$105$}
	\loigiai{
		Quãng đường ô tô đi được trong $5 \left(s\right)$ đầu là $s_1 = \displaystyle\int\limits_0^5 7t \mathrm{\,d}t = 7 \dfrac{t^2}{2} \Big|_0^5 = 87{,}5$ (mét).\\
		Phương trình vận tốc của ô tô khi người lái xe phát hiện chướng ngại vật là $v_2 \left(t\right) = 35-35t$ (m/s). Khi xe dừng lại hẳn thì $v_2 \left(t\right) = 0 \Leftrightarrow 35-35t = 0 \Leftrightarrow t =1$.\\
		Quãng đường ô tô đi được từ khi phanh gấp đến khi dừng lại hẳn là
		$$s_2 = \displaystyle\int\limits_0^1 \left(35-35t \right) \mathrm{\,d}t = \left(35-35t\right) \Big|_0^1 = 17,5 \left(\text{mét}\right).$$
		Vậy quãng đường của ô tô đi được từ lúc bắt đầu chuyển bánh cho đến khi dừng hẳn là
		$$s = s_1 + s_2 = 87,5 + 17,5 = 105 \left(\text{mét}\right).$$ 
	}
\end{ex}

% \begin{ex}%Cau21D%[2D4H2-6] 
% 	\immini{Một người chạy trong thời gian $1$ giờ, vận tốc $v$ (km/h) phụ thuộc vào thời gian $t \left(h\right)$ có đồ thị là một phần parabol với đỉnh $I \left(\dfrac{1}{2};8\right)$ và trục đối xứng song song với trục tung như hình bên. Tính quảng đường $s$ người đó chạy được trong khoảng thời gian $45$ phút, kể từ khi chạy (đơn vị tính bằng km)?
% 	}{
% 		\begin{tikzpicture}[>=stealth, font=\footnotesize, line join=round, line cap=round, thick, smooth, samples=250, scale=0.6,yscale=.5]
% 			% Vẽ 2 trục, điền các số lên trục
% 			\draw[->] (-0.5,0)--(0,0) node[below left]{$O$}--(2,0) node[above]{$t$}; %định dạng trục Ox
% 			\foreach \x in {1}
% 			\draw[shift={(\x,0)},color=black] (0pt,2pt)--(0pt,-2pt) 
% 			node[below] { $\x$};
% 			\draw[->,color=black] (0,-0.5)--(0,9) node[right]{$v$};  %định dạng trục Oy
% 			\foreach \y in {8}
% 			\draw[shift={(0,\y)},color=black] (2pt,0pt) -- (-2pt,0pt) 
% 			node[left] {$\y$};
% 			\clip(-1,-1) rectangle (2,9); %vùng đồ thị
% 			%\draw[gray!50,thin,opacity=.5] (-1,-1) grid (4,10); %ô vuông
% 			%Vẽ đồ thị
% 			\draw[smooth,samples=50,domain=0:1] 
% 			plot(\x,{-32*(\x)^2+32*(\x)});
% 			% Vẽ thêm mấy cái râu ria
% 			\draw[dashed] (1/2,0)--(1/2,8) circle(1.5pt)--(0,8);
% 			%Vẽ dấu chấm tròn 
% 			\fill (0cm,0cm) circle (1.5pt); 
% 		\end{tikzpicture} 
% 	}
% 	\shortans{$4,5$}
% 	\loigiai{
% 		\immini{
% 			Gọi parabol là $\left(P\right) \colon y = ax^2 + bx + c$. Từ hình vẽ ta có $\left(P\right)$ đi qua $O\left(0;0\right)$, $A \left(1;0\right)$ và điểm $I\left(\dfrac{1}{2};8\right)$.\\
% 			Ta có hệ: $\heva{&c=0\\&a+b+c=0\\&\dfrac{a}{4}+\dfrac{b}{2}+c = 8} \Leftrightarrow \heva{&a=-32\\&b=32\\&c=0.}$\\
% 			Suy ra $\left(P\right) \colon y = -32x^2 + 32x$.\\
% 			Vậy quãng đường người đó đi được là $s = \displaystyle\int\limits_0^{\tfrac{3}{4}} \left(-32x^2 + 32x \right) \mathrm{\,d}x = 4{,}5$ (km).
% 		}
% 		{\begin{tikzpicture}[>=stealth, font=\footnotesize, line join=round, line cap=round, thick, smooth, samples=250, scale=0.6]
% 				% Vẽ 2 trục, điền các số lên trục
% 				\draw[->] (-0.5,0)--(0,0) node[below left]{$O$}--(2,0) node[above]{$t$}; %định dạng trục Ox
% 				\foreach \x in {1}
% 				\draw[shift={(\x,0)},color=black] (0pt,2pt)--(0pt,-2pt) 
% 				node[below] { $\x$};
% 				\draw[->,color=black] (0,-0.5)--(0,9) node[right]{$v$};  %định dạng trục Oy
% 				\foreach \y in {8}
% 				\draw[shift={(0,\y)},color=black] (2pt,0pt) -- (-2pt,0pt) 
% 				node[left] {$\y$};
% 				\clip(-1,-1) rectangle (2,9); %vùng đồ thị
% 				%\draw[gray!50,thin,opacity=.5] (-1,-1) grid (4,10); %ô vuông
% 				%Vẽ đồ thị
% 				\draw[smooth,samples=50,domain=0:1] 
% 				plot(\x,{-32*(\x)^2+32*(\x)});
% 				% Vẽ thêm mấy cái râu ria
% 				\draw[dashed] (1/2,0)--(1/2,8) circle(1.5pt)--(0,8);
% 				%Vẽ dấu chấm tròn 
% 				\fill (0cm,0cm) circle (1.5pt); 
% 		\end{tikzpicture} }
% 	}
% \end{ex}

% \begin{ex}%Cau22D%[2D4V2-6]
% 	Một vật chuyển động trong $4$ giờ với vận tốc $v$ (km/h) phụ thuộc thời gian $t$ (h) có đồ thị của vận tốc như hình bên. Trong khoảng thời gian $3$ giờ kể từ khi bắt đầu chuyển động, đồ thị đó là một phần của đường parabol có đỉnh $I \left(2;9\right)$ với trục đối xứng song song với trục tung, khoảng thời gian còn lại đồ thị là một đoạn thẳng song song với trục hoành. Tính quãng đường $s$ mà vật di chuyển được trong $4$ giờ đó (đơn vị tính bằng km).
% 	\begin{center}
% 		\begin{tikzpicture}[>=stealth,scale=0.45]
% 			% Vẽ 2 trục, điền các số lên trục
% 			\draw[->] (-0.5,0)--(0,0) node[below left]{$O$}--(5,0) node[above]{$t$};
% 			\foreach \x in {2,3,4}
% 			\draw[shift={(\x,0)},color=black] (0pt,2pt)--(0pt,-2pt) 
% 			node[below] { $\x$};
% 			\draw[->,color=black] (0,-0.5)--(0,10) node[right]{$v$};
% 			\foreach \y in {9}
% 			\draw[shift={(0,\y)},color=black] (2pt,0pt) -- (-2pt,0pt) 
% 			node[left] {$\y$};
% 			\clip(-1,-1) rectangle (5,10); %vùng đồ thị
% 			%\draw[gray!50,thin,opacity=.5] (-1,-1) grid (4,10); %ô vuông
% 			%Vẽ đồ thị
% 			\draw[smooth,samples=100,domain=0:3, font=\footnotesize, line join=round, line cap=round, thick, smooth] 
% 			plot(\x,{(-9/4)*(\x)^2+9*(\x)});
% 			\draw[smooth,samples=100, font=\footnotesize, line join=round, line cap=round, thick, smooth,domain=3:4] 
% 			plot(\x,{27/4});
% 			% Vẽ thêm mấy cái râu ria
% 			\draw[dashed] (3,0)--(3,27/4) circle(1.5pt);  \draw[dashed] (2,0)--(2,9) circle(1.5pt) node[above]{$I$}--(0,9) circle(1.5pt); \draw[dashed] (4,0)--(4,27/4) circle(1.5pt);
% 			%Vẽ dấu chấm tròn 
% 			\fill (0cm,0cm) circle (1.5pt); 
% 		\end{tikzpicture} 
% 	\end{center}
% 	\shortans{$27$}
% 	\loigiai{
% 		Gọi $\left(P\right) \colon y = ax^2+bx+c$.\\
% 		Vì $\left(P\right)$ qua $O\left(0;0\right)$ và có đỉnh $I\left(2;9\right)$ nên dễ tìm được phương trình là $y = \dfrac{-9}{4}x^2 + 9x$.\\
% 		Ngoài ra tại $x=3$ ta có $y = \dfrac{27}{4}$.\\
% 		Vậy quãng đường cần tìm là: $S = \displaystyle\int\limits_0^3 \left(\dfrac{-9}{4}x^2 +9x \right) \mathrm{\,d}x + \displaystyle\int\limits_3^4 \dfrac{27}{4} \mathrm{\,d}x = 27$ (km).
% 	}
% \end{ex}

% \begin{ex}%Cau23D%[2D4V2-6]
% 	Một vật chuyển động trong $6$ giờ với vận tốc $v$ (km/h) phụ thuộc vào thời gian $t$ (h) có đồ thị như hình bên dưới. Trong khoảng thời gian $2$ giờ từ khi bắt đầu chuyển động, đồ thị là một phần đường Parabol có đỉnh $I\left(3;9\right)$ và có trục đối xứng song song với trục tung. Khoảng thời gian còn lại, đồ thị vận tốc là một đường thẳng có hệ số góc bằng $\dfrac{1}{4}$. Tính quảng đường $s$ mà vật di chuyển được trong $6$ giờ? (đơn vị tính bằng km, làm tròn đến chữ số thập phân thứ nhất).
% 	\begin{center}
% 		\begin{tikzpicture}[>=stealth,scale=0.5]
% 			% Vẽ 2 trục, điền các số lên trục
% 			\draw[->] (-0.5,0)--(0,0) node[below left]{$O$}--(7,0) node[above]{$t$}; %định dạng trục Ox
% 			\foreach \x in {2,3,6}
% 			\draw[shift={(\x,0)},color=black] (0pt,2pt)--(0pt,-2pt) 
% 			node[below] { $\x$};
% 			\draw[->,color=black] (0,-0.5)--(0,10) node[right]{$v$};  %định dạng trục Oy
% 			\foreach \y in {8,9}
% 			\draw[shift={(0,\y)},color=black] (2pt,0pt) -- (-2pt,0pt) 
% 			node[left] {$\y$};
% 			\clip(-1,-1) rectangle (7,10); %vùng đồ thị
% 			%\draw[gray!50,thin,opacity=.5] (-1,-1) grid (4,10); %ô vuông
% 			%Vẽ đồ thị
% 			\draw[smooth,samples=100,domain=0:2,font=\footnotesize, line join=round, line cap=round, thick] 
% 			plot(\x,{(-1)*(\x)^2+6*(\x)});
% 			\draw[smooth,domain=2:6, line join=round, line cap=round,dashed] 
% 			plot(\x,{(-1)*(\x)^2+6*(\x)});
% 			\draw[smooth,samples=100,domain=2:6,font=\footnotesize, line join=round, line cap=round, thick] 
% 			plot(\x,{(1/4)*(\x)+15/2});
% 			% Vẽ thêm mấy cái râu ria
% 			\draw[dashed] (3,0)--(3,9) circle(1.5pt) node[above]{$I$}--(0,9) circle(1.5pt); 
% 			\draw[dashed] (2,0)--(2,8) circle(1.5pt) --(0,8);
% 			\draw[dashed] (6,0)--(6,9) circle(1.5pt) --(0,9);
% 			%Vẽ dấu chấm tròn 
% 			\fill (0cm,0cm) circle (1.5pt); 
% 		\end{tikzpicture}
% 	\end{center}
% 	\shortans{$43{,}3$}
% 	\loigiai{
% 		Vì Parabol đi qua $O\left(0;0\right)$ và có tọa độ đỉnh $I\left(3;9\right)$ nên thiết lập được phương trình Parabol là $\left(P\right) \colon y = v\left(t\right) = -t^2+6t$; $\forall t \in \left[0;2\right]$.\\
% 		Sau $2$ giờ đầu thì hàm vận tốc có dạng là hàm bậc nhất $y = \dfrac{1}{4}t + m$, dựa trên đồ thị ta thấy đi qua điểm có tọa độ $\left(6;9\right)$ nên thế vào hàm số và tìm được $m = \dfrac{15}{2}$.\\
% 		Nên hàm vận tốc từ giờ thứ $2$ đến giờ thứ $6$ là: $y = \dfrac{1}{4}t + \dfrac{15}{2},\forall t \in \left[2;6\right]$.\\
% 		Quảng đường vật đi được bằng tổng đoạn đường $2$ giờ đầu và đoạn đường $4$ giờ sau.
% 		$$S = S_1 +S_2 = \displaystyle\int\limits_0^2 \left(-t^2+6t\right) \mathrm{\,d}t + \displaystyle\int\limits_2^6 \left(\dfrac{1}{4}t+\dfrac{15}{2} \right) \mathrm{\,d}t = \dfrac{130}{3} \approx 43{,}3 \left(\text{km}\right).$$
% 	}
% \end{ex}

% \begin{ex}%Cau24D%[2D4H2-6]
% 	\immini{Một người chạy trong thời gian $1$ giờ, với vận tốc $v$ (km/h) phụ thuộc vào thời gian $t$ (h) có đồ thị là một phần của parabol có đỉnh $I \left(\dfrac{1}{2};8\right)$ và trục đối xứng song song với trục tung như hình vẽ. Tính quãng đường $S$ người đó chạy được trong thời gian $45$ phút, kể từ khi bắt đầu chạy (đơn vị tính bằng km).
% 	}{
% 		\begin{tikzpicture}[>=stealth, scale=0.6, yscale=.5]
% 			% Vẽ 2 trục, điền các số lên trục
% 			\draw[->] (-0.5,0)--(0,0) node[below left]{$O$}--(2,0) node[above]{$t$}; %định dạng trục Ox
% 			\foreach \x in {1}
% 			\draw[shift={(\x,0)},color=black] (0pt,2pt)--(0pt,-2pt) 
% 			node[below] { $\x$};
% 			\draw[->,color=black] (0,-0.5)--(0,9) node[right]{$v$};  %định dạng trục Oy
% 			\foreach \y in {8}
% 			\draw[shift={(0,\y)},color=black] (2pt,0pt) -- (-2pt,0pt) 
% 			node[left] {$\y$};
% 			\clip(-1,-1) rectangle (2,9); %vùng đồ thị
% 			\draw[smooth,samples=50,domain=0:1, font=\footnotesize, line join=round, line cap=round, thick] 
% 			plot(\x,{-32*(\x)^2+32*(\x)});
% 			\draw[dashed] (1/2,0)--(1/2,8) circle(1.5pt)--(0,8);
% 			\fill (0cm,0cm) circle (1.5pt); 
% 		\end{tikzpicture}
% 	}
% 	\shortans{$4{,}5$}
% 	\loigiai{
% 		Trước hết ta tìm công thức biểu thị vận tốc theo thời gian, giả sử $v\left(t\right) = at^2+bt+c$.\\  .
% 		Khi đó dựa vào hình vẽ ta có hệ phương trình\\
% 		$$\heva{&c=0\\&a\left(\dfrac{1}{2}\right)^2+b\left(\dfrac{1}{2}\right)+c =8\\&a+b+c=0} \Leftrightarrow \heva{&a=-32\\&b=32\\&c=0.}$$
% 		Do đó quãng đường người đó đi được sau $45$ phút là $S = \displaystyle\int\limits_0^{\tfrac{45}{60}} \left(32t-32t^2\right) \mathrm{\,d}t = 4{,}5$ (km).
% 	}
% \end{ex}

\begin{ex}%Cau25D%[2D4H2-6]
	Một vật chuyển động trong $4$ giờ với vận tốc $v$ (km/h) phụ thuộc thời gian $t$ (h) có đồ thị là một phần của đường parabol có đỉnh $I\left(1;1\right)$ và trục đối xứng song song với trục tung như hình bên. Tính quãng đường $s$ mà vật di chuyển được trong $4$ giờ kể từ lúc xuất phát (làm tròn đến chữ số thập phân thứ nhất).
	\begin{center}
		\begin{tikzpicture}[>=stealth,scale=0.5,yscale=.7]
			% Vẽ 2 trục, điền các số lên trục
			\draw[->] (-0.5,0)--(0,0) node[below left]{$O$}--(5,0) node[above]{$t$};
			\foreach \x in {1,4}
			\draw[shift={(\x,0)},color=black] (0pt,2pt)--(0pt,-2pt) 
			node[below] { $\x$};
			\draw[->,color=black] (0,-0.5)--(0,11) node[right]{$v$};
			\foreach \y in {1,2,10}
			\draw[shift={(0,\y)},color=black] (2pt,0pt) -- (-2pt,0pt) 
			node[left] {$\y$};
			\clip(-1,-1) rectangle (5,11); %vùng đồ thị
			\draw[smooth,samples=100,domain=0:10, font=\footnotesize, line join=round, line cap=round, thick] 
			plot(\x,{(\x)^2-2*(\x)+2});
			% Vẽ thêm mấy cái râu ria
			\draw[dashed] (4,0)--(4,10) circle(1.5pt)--(0,10);  \draw[dashed] (1,0)--(1,1) circle(1.5pt) node[above]{$I$}--(0,1);
			%Vẽ dấu chấm tròn 
			\fill (0cm,0cm) circle (1.5pt); 
		\end{tikzpicture} 
	\end{center}
	\shortans{$13{,}3$}
	\loigiai{Hàm biểu diễn vận tốc có dạng $v\left(t\right) = at^2+bt+c$. Dựa vào đồ thị ta có
		$$\heva{&c=2\\&-\dfrac{b}{2a}=1\\&a+b+c=1} \Leftrightarrow \heva{&a=1\\&b=-2\\&c=2} \Leftrightarrow v\left(t\right) = t^2 -2t+2.$$
		Với $t=4 \Rightarrow v\left(4\right) = 10$ (thõa mãn).\\
		Từ đó $s = \displaystyle\int\limits_0^4 \left(t^2-2t+2\right) \mathrm{\,d}t = \dfrac{40}{3} \approx 13{,}3$ (km).
	}
\end{ex}

% \begin{ex}%Cau26D%[2D4V2-6]
% 	Chất điểm chuyển động theo quy luật vận tốc $v\left(t\right)$ (m/s) có dạng đường Parapol khi $0 \leq t\leq 5$ (s) và $v\left(t\right)$ có dạng đường thẳng khi $5 \leq t \leq 10$ (s). Cho đỉnh Parapol là $I\left(2;3\right)$. Hỏi quãng đường đi được chất điểm trong thời gian $0 \leq t \leq 10$ (s) là bao nhiêu mét? (làm tròn đến hàng đơn vị)
% 	\begin{center}
% 		\begin{tikzpicture}[>=stealth,scale=0.3,yscale=.7]
% 			% Vẽ 2 trục, điền các số lên trục
% 			\draw[->] (-0.5,0)--(0,0) node[below left]{$O$}--(13,0) node[above]{$t$};
% 			\foreach \x in {2,5,10}
% 			\draw[shift={(\x,0)},color=black] (0pt,2pt)--(0pt,-2pt) 
% 			node[below] { $\x$};
% 			\draw[->,color=black] (0,-0.5)--(0,22) node[right]{$v$};
% 			\foreach \y in {3,11}
% 			\draw[shift={(0,\y)},color=black] (2pt,0pt) -- (-2pt,0pt) 
% 			node[left] {$\y$};
% 			\clip(-1,-1) rectangle (11,23); %vùng đồ thị
% 			\draw[smooth,samples=100,domain=0:5, font=\footnotesize, line join=round, line cap=round, thick] 
% 			plot(\x,{2*(\x)^2-8*(\x)+11});
% 			\draw[smooth,samples=100,domain=5:10, font=\footnotesize, line join=round, line cap=round, thick] 
% 			plot(\x,{(-21/5)*(\x)+42});
% 			% Vẽ thêm mấy cái râu ria
% 			\draw[dashed] (5,0)--(5,21) circle(1.5pt);  \draw[dashed] (2,0)--(2,3) circle(1.5pt)--(0,3);
% 			%Vẽ dấu chấm tròn 
% 			\fill (0cm,0cm) circle (1.5pt); 
% 		\end{tikzpicture}
% 	\end{center}
% 	\shortans{$91$}
% 	\loigiai{
% 		Gọi Parapol $\left(P\right) \colon y = ax^2+bx+c$ khi $0 \leq t \leq 5 \left(s\right)$.\\  
% 		Do $\left(P\right) \colon y = ax^2+bx+c$ đi qua $I\left(3;2\right)$; $A \left(0;11\right)$ nên
% 		$$\heva{&4a+2b+c=3\\&c=11\\&4a+b=0} \Rightarrow \heva{&a=2\\&b=-8\\&c=11.}$$
% 		Khi đó quãng đường vật di chuyển trong khoảng thời gian từ $0 \leq t \leq 5$ (s) là
% 		$$S_1 = \displaystyle\int\limits_0^5 \left(2x^2 - 8x +11\right) \mathrm{\,d}x = \dfrac{115}{3} \left(\text{m}\right).$$
% 		Ta có $f\left(5\right) = 21$.\\
% 		Gọi $d \colon y = ax+b$ khi $5 \leq t \leq 10$ (s), do $d$ đi qua điểm $B\left(5;21\right)$ và $C\left(10;0\right)$ nên
% 		$$\heva{&5a+b=11\\&10a+b=0} \Rightarrow \heva{&a=-\dfrac{21}{5}\\&b=42.}$$
% 		Khi đó quãng đường vật di chuyển trong khoảng thời gian từ $5 \leq t \leq 10$ (s) là:
% 		$$S_2 = \displaystyle\int\limits_5^{10} \left(-\dfrac{21}{5}x+42\right) \mathrm{\,d}x = \dfrac{105}{2} \left(\text{m}\right).$$
% 		Quãng đường đi được chất điểm trong thời gian  $0 \leq t \leq 10$ (s) là:
% 		$$S = \dfrac{115}{3}+\dfrac{105}{2}= \dfrac{545}{6} \approx 91 \left(\text{m}\right).$$
% 	}
% \end{ex}
\Closesolutionfile{ans}
\indapan{6}{ans/ans-2C4B2CD3-KQ}
% \subsection{Tích phân hàm ẩn biến đổi phức tạp}
% \begin{tomtat}
% 	Cần nhớ các công thức đạo hàm của hàm hợp
% 	\begin{itemize}
% 		\item $\displaystyle\int f'(x)\mathrm{\,d}x=f(x)+C$
% 		\item $f'(x)\cdot g(x)+f(x)\cdot g'(x)=\left[f(x)\cdot g(x)\right]'$
% 		\item $\dfrac{f'(x)\cdot g(x)-f(x)\cdot g'(x)}{g^2(x)}=\left[\dfrac{f(x)}{g(x)}\right]'$
% 		\item $\dfrac{f'(x)}{f(x)}=\left[\ln \left(f(x)\right)\right]'$
% 		\item $ -\dfrac{f'(x)}{f^2(x)}=\left[\dfrac 1{f(x)}\right]'$
% 		\item $-\dfrac{f'(x)}{f^n(x)}=\left[\dfrac 1{(n-1)[f(x)]^{n-1}}\right]'$
% 		\item $n\cdot f'(x)\cdot \left(f(x)\right)^{n-1}=\left[f(x)^n\right]'$
% 		\item $\dfrac{f'(x)}{\sqrt{f(x)}}=\left[2\sqrt{f(x)}\right]'$
% 	\end{itemize}	 
% \end{tomtat}
% \begin{dang}{.}
% \begin{enumerate}
		
% \item[1.]  Điều kiện hàm ẩn có dạng$\colon $ $\hoac{&f'(x)=g(x) \cdot h\left[f(x)\right] \\ & f'(x) \cdot h[f(x)]=g(x).}$

% Phương pháp giải$\colon $
% 	\begin{itemize}
% 	\item $\dfrac{f'(x)}{h[f(x)]}=g(x) \Leftrightarrow \displaystyle\int \dfrac{f'(x)}{h[f(x)]} \mathrm{\,d}x=\displaystyle\int g(x) \mathrm{\,d}x \Leftrightarrow \displaystyle\int \dfrac{d[f(x)]}{h[f(x)]}=\displaystyle\int g(x)\mathrm{\,d}x.$
% 	\item $f'(x) h[f(x)]=g(x) \Leftrightarrow \displaystyle\int f'(x) h[f(x)] \mathrm{\,d}x=\displaystyle\int g(x)\mathrm{\,d}x$\\$ \Leftrightarrow \displaystyle\int h[f(x)]  d\left[f'(x)\right]=\displaystyle\int g(x).$
% 	\end{itemize}
% Chú ý$\colon$ Ngoài việc nguyên hàm hai vế, ta có thể lấy tích phân hai vế (tùy câu hỏi của bài toán).\\
% \item[2.] Điều kiện hàm ẩn có dạng$\colon $ $\hoac{&f'(x)+p(x) \cdot f(x)=0 \\ & f'(x)+p(x) \cdot[f(x)]^n=0.}$

% Phương pháp giải$\colon $
% \begin{itemize}
% 	\item $f'(x)+p(x) \cdot f(x)=0.$\\
% Chia hai vế với $f(x)$ ta đựơc $\dfrac{f'(x)}{f(x)}+p(x)=0 \Leftrightarrow \dfrac{f'(x)}{f(x)}=-p(x).$\\
% Suy ra $\displaystyle\int \dfrac{f'(x)}{f(x)} \mathrm{d} x=-\displaystyle\int p(x) \mathrm{d} x \Leftrightarrow \ln |f(x)|=-\displaystyle\int p(x) \mathrm{d} x$.\\
% Từ đây ta dễ dàng tính được $f(x).$
% 	\item $f'(x)+p(x) \cdot[f(x)]^n=0$\\
% Chia hai vế với $[f(x)]^n$ ta được $\dfrac{f'(x)}{[f(x)]^n}+p(x)=0 \Leftrightarrow \dfrac{f'(x)}{[f(x)]^n}=-p(x).$
% \end{itemize}
% \end{enumerate}
% \end{dang}
\setcounter{ex}{0}
\Opensolutionfile{ans}[ans/ans-2-B1]
\TN  
\begin{ex}%[2D4C2-2]
	Cho hàm số $f(x)$ nhận giá trị không âm và có đạo hàm liên tục trên $\mathbb{R}$ thỏa mãn $f'(x)=(2x+1){{\left[f(x) \right]}^2},\forall x\in \mathbb{R}$ và $f(0)=-1$. Tính tích phân $\displaystyle\int\limits_0^1\left(x^3-1\right)f(x)\mathrm{\,d}x$.
	\choice
	{$1$}
	{$\dfrac{2}{3}$}
	{\True $\dfrac{1}{2}$}
	{$\dfrac{3}{2}$}
	\loigiai{
		Ta có
		$$
		\begin{aligned}
			 &&f'(x)=(2x+1)[f(x)]^2,\forall x\in\mathbb{R}\\
			&\Rightarrow&\dfrac{-f'(x)}{[f(x)]^2}=-(2x+1),\forall x\in\mathbb{R}\\ 			
			&\Rightarrow&\left[\dfrac 1{f(x)}\right]'=-(2x+1),\forall x\in\mathbb{R}.
		\end{aligned}
		$$
		Suy ra $\dfrac{1}{f(x)}=-\displaystyle\int{\left(2x+1\right)}\mathrm{\,d}x=-x^2-x+C\Rightarrow f(x)=\dfrac{1}{-x^2-x+C}$.\\
		Vì  $f(0)=-1\Rightarrow C=-1$.\\
		Suy ra $f(x)=-\dfrac{1}{x^2+x+1}$.\\
		$\displaystyle\int\limits_0^1\left(x^3-1\right)f(x)\mathrm{\,d}x=-\displaystyle\int\limits_0^1\left(x^3-1\right)\left(\dfrac{1}{x^2+x+1}\right)\mathrm{\,d}x=\displaystyle\int\limits_0^1\left(1-x\right)\mathrm{\,d}x$\\
		$=\left.\left(x-\dfrac{x^2}{2}\right)\right|_0^1=\dfrac{1}{2}$.}
\end{ex}

\begin{ex}%[2D4C2-2]
	Cho hàm số $f(x)\ne 0$, liên tục trên đoạn $\left[1;2\right]$ và thỏa mãn $f(1)=\dfrac{1}{3}$; $\linebreak x^2\cdot f'(x)=f^2(x)$ với $\forall x\in\left[1;2\right]$. Tính tích phân $I=\displaystyle\int\limits_1^2\left(2x+1\right)^2f(x)\mathrm{\,d}x$.
	\choice
	{$I=\dfrac{7}{6}$}
	{$I=\dfrac{5}{6}$}
	{\True $I=\dfrac{37}{6}$}
	{$I=\dfrac{1}{6}$}
	\loigiai{
		Ta có
		$$
		\begin{aligned}
		&x^2\cdot f'(x)=f^2(x)\\ 
		\Rightarrow&\dfrac{f'(x)}{f^2(x)}=\dfrac 1{x^2}\\ 
		\Rightarrow&{\left[-\dfrac 1{f(x)}\right]'}=\dfrac 1{x^2}\\ 
		\Rightarrow&-\dfrac 1{f(x)}=\displaystyle\int{\dfrac 1{x^2}}\mathrm{\,d}x\\
		 \Rightarrow&\dfrac 1{f(x)}=-\displaystyle\int{\dfrac 1{x^2}}\mathrm{\,d}x\\
		  \Rightarrow&\dfrac 1{f(x)}=\dfrac 1 x+C.\\ 
		\end{aligned}
		$$
		Mà $f(1)=\dfrac{1}{3}$ $\Rightarrow 3=1+C\Rightarrow C=2.$\\
		Do đó $\dfrac{1}{f(x)}=\dfrac{1}{x}+2 \Rightarrow f(x)=\dfrac{x}{2x+1}.$\\
		Vậy $I=\displaystyle\int\limits_1^2\left(2x+1\right)^2f(x)\mathrm{\,d}x=\displaystyle\int\limits_1^2\left(2x+1\right)^2\dfrac{x}{2x+1}\mathrm{\,d}x=\displaystyle\int\limits_1^2\left(2x^2+x\right)\mathrm{\,d}x=\dfrac{37}{6}$.}
\end{ex}

\begin{ex}%[2D4C2-2]
	Cho hàm số $f(x)$ có đạo hàm trên $\mathbb{R}$ thỏa mãn $3f'(x)\cdot \mathrm{e}^{f^3(x)}-\dfrac{2x}{f^2(x)}=0$ với $\forall x\in\mathbb{R}$. Biết $f(1)=0$, tính tích phân $I=\displaystyle\int\limits_0^{2024}{\dfrac{1}{\sqrt[3]{2\ln x}}\cdot f(x){\mathrm{\,d}}x}$.
	\choice
	{$1$}
	{$\dfrac{1}{2024}$}
	{\True $2024$}
	{$0$}
	\loigiai{
		Ta có
		$$
		\begin{aligned}
		&3f'(x)\cdot\mathrm{e}^{f^3(x)}-\dfrac{2x}{f^2(x)}=0\\ 
		\Rightarrow& 3f^2(x)\cdot f'(x)\cdot\mathrm{e}^{f^3(x)}=2x \\
		\Rightarrow&\left[\mathrm{e}^{f^3(x)}\right]'=2x \\
		\Rightarrow&\mathrm{e}^{f^3(x)}=\displaystyle\int{2x}\mathrm{\,d}x \\
		\Rightarrow&\mathrm{e}^{f^3(x)}=x^2+C.\\ 
		\end{aligned}
		$$
		Mặt khác $f(1)=0\Rightarrow\mathrm{e}^{f^3(1)}=1+C\Rightarrow C=0.$\\
		Suy ra $\mathrm{e}^{f^3(x)}=x^2\Rightarrow{f^3}(x)=\ln {x^2}\Rightarrow f(x)=\sqrt[3]{2\ln x}$.\\
		Vậy $I=\displaystyle\int\limits_0^{2024}\dfrac 1{\sqrt[3]{2\ln x}}\cdot f(x)\mathrm{\,d}x=\displaystyle\int\limits_0^{2024}\dfrac 1{\sqrt[3]{2\ln x}}\cdot \sqrt[3]{2\ln x}\mathrm{\,d}x=\displaystyle\int\limits_0^{2024}\mathrm{\,d}x=2024$}
\end{ex}

\begin{ex}%[2D4C2-2]
	Cho hàm số $f(x)$ đồng biến, có đạo hàm trên đoạn $\left[1;4\right]$ và thoả mãn $x+2x\cdot f(x)=\left[f'(x)\right]^2$ với $\forall x\in\left[1;4\right]$. Biết $f(1)=\dfrac{3}{2}$, tính $I=\displaystyle\int\limits_1^4f(x)\mathrm{\,d}x$.
	\choice
	{\True $I=\dfrac{1186}{45}$}
	{$I=\dfrac{1186}{9}$}
	{$I=\dfrac{1186}{5}$}
	{$I=\dfrac{1186}{41}$}
	\loigiai{
		Do $f(x)$ đồng biến trên đoạn $\left[1;4\right]$ $\Rightarrow f'(x)\ge 0,\forall x\in\left[1;4\right].$\\
		Ta có  $x+2x \cdot f(x)=\left[f'(x)\right]^2
		\Leftrightarrow x\left(1+2\cdot f(x)\right)=\left[f'(x)\right]^2$, \\Do $x\in\left[1;4\right]$ và $f'(x)\ge 0,\forall x\in\left[1;4\right]$
		$\Rightarrow f(x) >\dfrac{-1}{2}$ và
		$$
		\begin{aligned}
			&f'(x)=\sqrt x \cdot \sqrt{1+2f(x)}\\
			\Leftrightarrow&\dfrac{f'(x)}{\sqrt{1+2f(x)}}=\sqrt x\\
			\Leftrightarrow&\left(\sqrt{1+2f(x)}\right)'=\sqrt x \\
			\Leftrightarrow&\sqrt{1+2f(x)}=\displaystyle\int{\sqrt x}\mathrm{\,d}x\\
			\Leftrightarrow&\sqrt{1+2f(x)}=\dfrac{2}{3}x\sqrt x+C.
		\end{aligned}
		$$
		Vì $f(1)=\dfrac{3}{2}\Rightarrow\sqrt{1+2\cdot\dfrac{3}{2}}=\dfrac{2}{3}+C\Leftrightarrow C=\dfrac{4}{3}$.\\
		Suy ra
		$$
		\begin{aligned}
			&\sqrt{1+2f(x)}=\dfrac{2}{3}x\sqrt x+\dfrac{4}{3}\\
			\Leftrightarrow & 1+2f(x)=\left(\dfrac{2}{3}x\sqrt x+\dfrac{4}{3}\right)^2\\
			\Leftrightarrow & f(x)=\dfrac{2}{9}{x^3}+\dfrac{8}{9}{x^{\dfrac{3}{2}}}+\dfrac{7}{18}.
		\end{aligned}
		$$
		Khi đó\\ $I=\displaystyle\int\limits_1^4f(x)\mathrm{\,d}x=\displaystyle\int\limits_1^4\left(\dfrac{2}{9}{x^3}+\dfrac{8}{9}{x^{\tfrac{3}{2}}}+\dfrac{7}{18}\right)\mathrm{\,d}x=\left.\left(\dfrac{1}{18}{x^4}+\dfrac{16}{45}{x^{\tfrac{5}{2}}}+\dfrac{7}{18}x\right)\right|_1^4=\dfrac{1186}{45}$.}
\end{ex}

\begin{ex}%[2D4C2-2]
	Cho hàm số $f(x)$ nhận giá trị dương và thỏa mãn $f(0)=1$, $\left[f'(x)\right]^3=\mathrm{e}^x\left[f(x)\right]^2,\forall x\in\mathbb{R}$.
	Tính $I=\displaystyle\int\limits_1^2f(x)\mathrm{\,d}x$.
	\choice
	{$I=\mathrm{e}^2+1$}
	{$I=\mathrm{e}-1$}
	{\True $I=\mathrm{e}^2-e$}
	{$I=\mathrm{e}$}
	\loigiai{
		Ta có
		$$
		\begin{aligned}
			&\left[f'(x)\right]^3=\mathrm{e}^x\left[f(x)\right]^2\\
		\Leftrightarrow& f'(x)=\sqrt[3]{\mathrm{e}^x}\cdot\sqrt[3]{\left[f(x)\right]^2}\\ 
		\Leftrightarrow&\dfrac{f'(x)}{\sqrt[3]{\left[f(x)\right]^2}}=\sqrt[3]{\mathrm{e}^x}\\
		 \Leftrightarrow&\dfrac{f'(x)}{\sqrt[3]{\left[f(x)\right]^2}}=\sqrt[3]{\mathrm{e}^x}\\
		  \Leftrightarrow &f'(x)\cdot \left[f(x)\right]^{-\tfrac 23}=\sqrt[3]{\mathrm{e}^x}\\ 
		  \Leftrightarrow& 3\left[\left(f(x)\right)^{\tfrac 13}\right]'=\sqrt[3]{\mathrm{e}^x}\\ 
		  \Leftrightarrow&\left[\left(f(x)\right)^{\tfrac 13}\right]'=\dfrac 13\sqrt[3]{\mathrm{e}^x}\\ 
		  \Leftrightarrow&\left[f(x)\right]^{\tfrac 13}=\dfrac 13\displaystyle\int{\sqrt[3]{\mathrm{e}^x}}\mathrm{\,d}x \\
		  \Leftrightarrow&\left[f(x)\right]^{\tfrac 13}=\mathrm{e}^{\tfrac x3}+C.
		\end{aligned}
		$$	
		Mà $f(0)=1\Rightarrow 1=1+C\Rightarrow C=0$.\\
		Do đó $\left[f(x)\right]^{\tfrac{1}{3}}=\mathrm{e}^{\tfrac{x}{3}}\Rightarrow f(x)=\mathrm{e}^x$.\\
		Vậy $I=\displaystyle\int\limits_1^2\mathrm{e}^x\mathrm{\,d}x=\mathrm{e}^2-\mathrm{e}$.}
\end{ex}

\begin{ex}%[2D4C2-2]
	Cho hàm số $y=f(x)$ có đạo hàm liên tục trên $\mathbb{R}$ và thỏa mãn điều kiện ${{x}^6}{{\left[f'(x) \right]}^3}+27{{\left[f(x)-1 \right]}^4}=0\,,\,\forall x\in \mathbb{R}$ và $f(1)=0$. Tính $I=\displaystyle\int\limits_2^3f(x)\mathrm{\,d}x$.
	\choice
	{$I=\dfrac{31}{2}$}
	{$I=-\dfrac{31}{2}$}
	{$I=\dfrac{61}{4}$}
	{\True $I=-\dfrac{61}{4}$}
	\loigiai{
		Ta có
		$$
		\begin{aligned}
		&x^6\left[f'(x)\right]^3+27\left[f(x)-1\right]^4=0\\ 
		\Leftrightarrow&{x^6}{\left[f'(x)\right]^3}=-27\left[f(x)-1\right]^4\\ 
		\Leftrightarrow&\dfrac{\left[f'(x)\right]^3}{\left[f(x)-1\right]^4}=-\dfrac{27}{x^6}\\ 
		\Leftrightarrow&\dfrac{\left[f'(x)\right]^3}{\left[f(x)-1\right]^3\left[f(x)-1\right]}=-\dfrac{27}{x^6}\\ 
		\Leftrightarrow&\dfrac{f'(x)}{\left[f(x)-1\right]\sqrt[3]{f(x)-1}}=-\dfrac 3{x^2}\\
		 \Leftrightarrow&\dfrac{f'(x)}{-3\left[f(x)-1\right]\sqrt[3]{f(x)-1}}=\dfrac 1{x^2}\\ 
		 \Leftrightarrow&{\left[\dfrac 1{\sqrt[3]{f(x)-1}}\right]'}=\dfrac 1{x^2}.\\ 
		\end{aligned}
		$$
		Do đó $\displaystyle\int{\left[\dfrac{1}{\sqrt[3]{f(x)-1}}\right]'}\mathrm{\,d}x=\displaystyle\int{\dfrac{1}{x^2}\mathrm{\,d}x}=-\dfrac{1}{x}+C.$\\
		Suy ra $\dfrac{1}{\sqrt[3]{f(x)-1}}=-\dfrac{1}{x}+C$.\\
		Mà  $f(1)=0\Rightarrow C=0$.\\
		Nên  $f(x)=1-x^3$.\\
		Khi đó $I=\displaystyle\int\limits_2^3f(x)\mathrm{\,d}x=\displaystyle\int\limits_2^3(1-x^3)\mathrm{\,d}x=-\dfrac{61}{4}$.}
\end{ex}
\begin{ex}%[2D4C2-4]
	Cho hàm số $f(x) > 0$ và thỏa mãn $\left[f'(x)\right]^2+f(x)\cdot f''(x)=\mathrm{e}^x$, $\forall x\in \mathbb{R}$ và $f(0)=f'(0)=1$. Tính $I=\displaystyle\int\limits_1^2 f(x) \mathrm{\,d}x$.
	\choice
	{$I=2\sqrt{\mathrm{e}}$}
	{$I=\mathrm{e}-\sqrt{\mathrm{e}}$}
	{\True $I=2\mathrm{e}-2\sqrt{\mathrm{e}}$}
	{$I=2\mathrm{e}+2\sqrt{\mathrm{e}}$}
	\loigiai{
		Ta có
		\allowdisplaybreaks
		\begin{eqnarray*}
			&&\left[f'(x)\right]^2+f(x)\cdot f''(x)=\mathrm{e}^x\\
			&\Leftrightarrow& \left[f(x)\cdot f'(x)\right]'=\mathrm{e}^x\\
			&\Rightarrow& f(x)\cdot f'(x)=\displaystyle\int\limits_{\mathrm{e}}^x \mathrm{e}^x \mathrm{\,d}x\\
			&\Rightarrow& f(x)\cdot f'(x)=\mathrm{e}^x+C.
		\end{eqnarray*}
		Từ $f(0)=f'(0)=1$ ta suy ra $C=0$.\\
		Vậy $f(x)\cdot f'(x)=\mathrm{e}^x$\\
		Tiếp đến có
		\allowdisplaybreaks
		\begin{eqnarray*}
			&&2f(x)\cdot f'(x)=\mathrm{e}^x\\
			&\Leftrightarrow& \left[f^2(x)\right]'=\mathrm{e}^x\\
			&\Rightarrow& f^2(x)=\displaystyle\int\limits_{\mathrm{e}}^x \mathrm{e}^x \mathrm{\,d}x\\
			&\Rightarrow& f^2(x)=\mathrm{e}^x+C
		\end{eqnarray*}
		Từ $f(0)=1$ ta suy ra $C=0$.\\
		Vậy $f^2(x)=\mathrm{e}^x\Rightarrow f(x)=\sqrt{\mathrm{e}^x}$ (do $f(x) > 0$).\\
		Khi đó $I=\displaystyle\int\limits_1^2 f(x) \mathrm{\,d}x = \displaystyle\int\limits_1^2 \sqrt{\mathrm{e}^x}\mathrm{\,d}x = \displaystyle\int\limits_1^2 \mathrm{e}^{\tfrac{x}{2}} \mathrm{\,d}x = \left.2\mathrm{e}^{\tfrac{x}{2}}\right|_1^2 = 2\mathrm{e}-2\sqrt{\mathrm{e}}$.
	}
\end{ex}

\begin{ex}%[2D4C2-2]
	Cho hàm số $f(x)$ thỏa mãn $\left[f'(x)\right]^2+f(x)\cdot f''(x)=2x$, và $f(0)=f'(0)=2$. Tính $I=\displaystyle\int\limits_1^2f^2(x)\mathrm{\,d}x$.
	\choice
	{\True $I=\dfrac{15}{2}$}
	{$I=\dfrac{1}{2}$}
	{$I=\dfrac{19}{2}$}
	{$I=15$}
	\loigiai{
		Ta có $\left[f(x)f'(x)\right]'=\left[f'(x)\right]^2+f(x)f''(x)$.\\
		Do đó theo giả thiết ta được $\left[f(x)f'(x)\right]'=2x$.\\
		Suy ra $f(x)f'(x)=x^2+C$.\\
		Hơn nữa $f(0)=f'(0)=2$ suy ra $C=1$.\\
		$\Rightarrow f(x)f'(x)=x^2+1$.\\
		Tương tự vì $\left[f^2(x)\right]'=2f(x)f'(x)$ nên $\left[f^2(x)\right]'=2\left(x^2+1\right)$.\\
		Suy ra $f^2(x)=\displaystyle\int 2\left(x^2+1\right) \mathrm{\,d}x \Rightarrow f^2(x)=\dfrac{2}{3}{x^3}+2x+C$.\\
		Mặt khác $f(0)=2$ nên  suy ra $C=2$.\\
		$\Rightarrow f^2(x)=\dfrac{2}{3}{x^3}+2x+2$.\\
		Vậy $I=\displaystyle\int\limits_1^2 f^2(x)\mathrm{\,d}x=\displaystyle\int\limits_1^2 \left(\dfrac{2}{3}{x^3}+2x+2\right)\mathrm{\,d}x=\dfrac{15}{2}$.
	}
\end{ex}

\begin{ex}%[2D4C2-2]
	Cho hàm số $f(x)$ thỏa mãn: $\left[f'(x)\right]^2+f(x)\cdot f''(x)=15x^4+12x$, $\forall x\in\mathbb{R}$ và $f(0)=f'(0)=1$. Giá trị của $f^2(1)$ bằng
	\choice
	{$\dfrac{5}{2}$}
	{\True $8$}
	{$10$}
	{$4$}
	\loigiai{
		Theo giả thiết
		\allowdisplaybreaks
		\begin{eqnarray*}
			& & \forall x\in\mathbb{R}\colon \left[f'(x)\right]^2+f(x)\cdot f''(x)=15x^4+12x\\
			&\Leftrightarrow& f'(x)\cdot f'(x)+f(x)\cdot f''(x)=15x^4+12x\\
			&\Leftrightarrow& \left[f(x)\cdot f'(x)\right]'=15x^4+12x\\
			&\Leftrightarrow& f(x)\cdot f'(x)=\displaystyle\int \left(15x^4+12x\right)\mathrm{\,d}x=3x^5+6x^2+C.\quad (1)
		\end{eqnarray*}
		Thay $x=0$ vào $(1)$, ta được $f(0)\cdot f'(0)=C \Leftrightarrow C=1$.\\
		Khi đó $(1)$ trở thành $\begin{aligned}[t]& f(x)\cdot f'(x)=3x^5+6x^2+1\\
			&\Rightarrow \displaystyle\int\limits_0^1 f(x)\cdot f'(x) \mathrm{\,d}x = \displaystyle\int\limits_0^1 \left(3x^5+6x^2+1\right) \mathrm{\,d}x\\
			&\Leftrightarrow \left.\left[\dfrac{1}{2} f^2(x)\right] \right|_0^1 = \left.\left(\dfrac{1}{2}{x^6}+2x^3+x\right) \right|_0^1
			\Leftrightarrow \dfrac{1}{2}\left[f^2(1)-f^2(0)\right]=\dfrac{7}{2} \\
			&\Leftrightarrow f^2(1)-1=7\Leftrightarrow f^2(1)=8.\end{aligned}$\\
		Vậy $f^2(1)=8$.
	}
\end{ex}

\begin{ex}%[2D4C2-2]
	Cho hàm số $y=f(x)$ thỏa mãn $\left[f'(x)\right]^2+f(x)\cdot f''(x)=x^3-2x,\,\forall x\in \mathbb{R}$ và $f(0)=f'(0)=2$. Tính giá trị của $T=f^2(2)$.
	\choice
	{$\dfrac{160}{15}$}
	{\True $\dfrac{268}{15}$}
	{$\dfrac{4}{15}$}
	{$\dfrac{268}{30}$}
	\loigiai{
		Ta có $\left[f'(x)\right]^2+f(x)\cdot f''(x)=x^3-2x,\,\forall x\in \mathbb{R} \Leftrightarrow \left[f'(x)\cdot f(x)\right]'=x^3-2x,\,\forall x\in \mathbb{R}$.\\
		Lấy nguyên hàm hai vế ta có 
		\allowdisplaybreaks
		\begin{eqnarray*}
			& & \displaystyle\int \left[f'(x)\cdot f(x)\right]' \mathrm{\,d}x= \displaystyle\int \left(x^3-2x\right)\mathrm{\,d}x\\ &\Leftrightarrow& f'(x)\cdot f(x)=\dfrac{x^4}{4}-x^2+C.
		\end{eqnarray*}
		Theo đề ra ta có $f(0)\cdot f(0)=C=4$.\\
		Suy ra $\displaystyle\int\limits_0^2 f'(x)\cdot f(x)\mathrm{\,d}x = \displaystyle\int\limits_0^2 \left(\dfrac{x^4}{4}-x^2+4\right)\mathrm{\,d}x \Leftrightarrow \left.\dfrac{f^2(x)}{2}\right|_0^2=\dfrac{104}{15}$ $\Leftrightarrow f^2(2)=\dfrac{268}{15}$.
	}
\end{ex}
\Closesolutionfile{ans}
% \indapan{10}{ans/ans-2-B1}
% \begin{dang}{}
% 	\begin{enumerate}
% 		\item Điều kiện hàm ẩn có dạng: $A(x)f(x)+B(x)f'(x)=h(x)$.\quad$(1)$\\
% 		\textit{Ý tưởng giải:}
% 		\begin{itemize}
% 			\item Ta cần nhân thêm một lượng $u(x)$ vào $(1)$ để tạo thành \break  $u'(x)f(x)+u(x)f'(x)=u(x).h(x)$ và lúc này:
% 			\allowdisplaybreaks
% 			\begin{eqnarray*}
% 				& & u'(x)f(x)+u(x)f'(x)=u(x)\cdot h(x)\\
% 				&\Leftrightarrow & \left[u(x)f(x)\right]'=u(x)\cdot .h(x)\\
% 				&\Rightarrow& \displaystyle\int \left[u(x)f(x)\right]'\mathrm{\,d}x= \displaystyle\int u(x)\cdot h(x)\mathrm{\,d}x\\
% 				&\Rightarrow& u(x)f(x)=\displaystyle\int u(x)\cdot h(x)\mathrm{\,d}x\\
% 				&\Rightarrow& f(x)=\dfrac{\displaystyle\int u(x)\cdot h(x)\mathrm{\,d}x}{u(x)}
% 			\end{eqnarray*}
% 			\item Cách tìm $u(x)$\\
% 			$u(x)$ được chọn sao cho: $\heva{&u'(x)=A(x)\\&u(x)=B(x)}$\\
% 			$\Rightarrow \dfrac{u'(x)}{u(x)}=\dfrac{A(x)}{B(x)} \Rightarrow \displaystyle\int \dfrac{u'(x)}{u(x)}\mathrm{\,d}x =\displaystyle\int\dfrac{A(x)}{B(x)}\mathrm{\,d}x$\\ $\Rightarrow \ln \left|u(x)\right|=\displaystyle\int \dfrac{A(x)}{B(x)}\mathrm{\,d}x \Rightarrow u(x)=\mathrm{e}^{\displaystyle\int \dfrac{A(x)}{B(x)}\mathrm{\,d}x}$.\\
% 		\end{itemize}
% 		\textbf{Tóm lại phương pháp giải:} $A(x)f(x)+B(x)f'(x)=h(x)$ $(1)$ như sau:
% 		\begin{itemize}
% 			\item Tìm $u(x)$: $u(x)=\mathrm{e}^{\displaystyle\int \dfrac{A(x)}{B(x)} \mathrm{\,d}x}$.
% 			\item Nhân $u(x)$ vào $(1)$ $\Rightarrow f(x)=\dfrac{\displaystyle\int{u(x)\cdot h(x)} \mathrm{\,d}x}{u(x)}$. 
% 		\end{itemize}
% 		\item Một số dạng đặc biệt của $(1)$
% 		\begin{enumerate}
% 			\item Điều kiện hàm ẩn có dạng: $\heva{&f'(x)+f(x)=h(x)\\&f'(x)-f(x)=h(x).}$\\
% 			\textbf{Phương pháp giải}
% 			\begin{itemize}
% 				\item $f'(x)+f(x)=h(x)$.\\
% 				Nhân hai vế với $\mathrm e^x$ ta được $$\mathrm e^x\cdot f'(x)+\mathrm e^x\cdot f(x)=\mathrm e^x\cdot h(x)\Leftrightarrow \left[\mathrm e^x\cdot f(x)\right]'=\mathrm e^x\cdot h(x).$$
% 				Suy ra $\mathrm e^x\cdot f(x)=\displaystyle\int \mathrm e^x\cdot h(x) \mathrm{\,d}x$.\\
% 				Từ đây ta dễ dàng tính được $f(x)$.
% 				\item $f'(x)-f(x)=h(x)$.\\
% 				Nhân hai vế với $\mathrm e^{-x}$ ta được $$\mathrm e^{-x}\cdot f'(x)-\mathrm e^{-x}\cdot f(x)=\mathrm e^{-x}\cdot h(x)\Leftrightarrow \left[e^{-x}\cdot f(x)\right]'=\mathrm e^{-x}\cdot h(x).$$
% 				Suy ra $\mathrm e^{-x}\cdot f(x)=\displaystyle\int \mathrm e^{-x}\cdot h(x) \mathrm{\,d}x$.\\
% 				Từ đây ta dễ dàng tính được $f(x)$.
% 			\end{itemize}
% 			\item Điều kiện hàm ẩn có dạng: $f'(x)+p(x)\cdot f(x)=h(x)$.\\
% 			\textbf{Phương pháp giải}\\
% 			Nhân hai vế với $\mathrm e^{\displaystyle\int p (x)\mathrm{\,d}x}$ ta được
% 			\allowdisplaybreaks
% 			\begin{eqnarray*}
% 				& & f'(x)\cdot \mathrm e^{\displaystyle\int p(x)\mathrm{\,d}x}+p(x)\cdot \mathrm e^{\displaystyle\int p (x)dx}\cdot f(x)=h(x)\cdot{\mathrm e^{\displaystyle\int p (x)dx}}\\
% 				&\Leftrightarrow& \left[f(x)\cdot{e^{\displaystyle\int p (x)dx}}\right]'=h(x)\cdot \mathrm e^{\displaystyle\int p (x)\mathrm{\,d} x}.
% 			\end{eqnarray*}		
% 			Suy ra $f(x)\cdot \mathrm e^{\displaystyle\int p(x)\mathrm{\,d}x}=\displaystyle\int \mathrm e^{\displaystyle\int p (x)\mathrm{\,d}x}h(x) \mathrm{\,d}x$.\\
% 			Từ đây ta dễ dàng tính được $f(x)$.
% 		\end{enumerate}
% 	\end{enumerate}
% \end{dang}
\Opensolutionfile{ans}[ans/ans-2-B1-D2]
% \TN
\begin{ex}%[2D4C2-4]
	Cho hàm số $f(x)$ thỏa mãn $f(x)+f'(x)=\mathrm{e}^{-x}$, $\forall x\in\mathbb{R}$ và $f(0)=2$. Tính $I=\displaystyle\int\limits_1^2 \dfrac{f(x) \mathrm{e}^x}{x}\mathrm{\,d}x$.
	\choice
	{$I=2\ln 2$}
	{$I=\ln 2$}
	{$I=1+\ln 2$}
	{\True $I=1+2\ln 2$}
	\loigiai{
		Ta có
		\allowdisplaybreaks
		\begin{eqnarray*}
			& & f(x)+f'(x)=\mathrm{e}^{-x}\\
			&\Leftrightarrow& f(x) \mathrm{e}^x+f'(x)\mathrm{e}^x=1\\
			&\Leftrightarrow& \left[f(x) \mathrm{e}^x\right]'=1\\
			&\Rightarrow& f(x)\mathrm{e}^x=\displaystyle\int x \mathrm{\,d}x\\
			&\Leftrightarrow& f(x) \mathrm{e}^x=x+C.
		\end{eqnarray*}
		Vì $f(0)=2$ nên $C=2$.\\
		$\Rightarrow f(x)\mathrm{e}^x=x+2$.\\
		Vậy 
		$I=\displaystyle\int\limits_1^2 \dfrac{f(x) \mathrm{e}^x}{x} \mathrm{\,d}x = \displaystyle\int\limits_1^2 \dfrac{x+2}{x}\mathrm{\,d}x=\displaystyle\int\limits_1^2 \left(1+\dfrac{2}{x}\right)\mathrm{\,d}x= \left(x+2\ln | x|\right)\bigg|_1^2=1+2\ln 2$.
	}
\end{ex}

\begin{ex}%[2D4C2-4]
	Cho hàm số $f(x)$ có đạo hàm trên $\mathbb{R}$ thỏa mãn $\left(x+2\right)f(x)+\left(x+1\right)f'(x)=\mathrm{e}^x$ và $f(0)=\dfrac{1}{2}$. Tính $I=\displaystyle\int\limits_1^2 \left(2x+2\right)f(x)\mathrm{\,d}x$.
	\choice
	{$I=\mathrm{e}^2$}
	{$I=1+\mathrm{e}$}
	{$I=1+\mathrm{e}^2$}
	{\True $I=\mathrm{e}^2-\mathrm{e}$}
	\loigiai{
		Ta có
		\allowdisplaybreaks
		\begin{eqnarray*}
			& & \left(x+2\right)f(x)+\left(x+1\right)f'(x)=\mathrm{e}^x\\
			&\Leftrightarrow& \left(x+1\right)f(x)+f(x)+\left(x+1\right)f'(x)=\mathrm{e}^x\\
			&\Leftrightarrow& \left[\left(x+1\right)f(x)\right]+\left[\left(x+1\right)f(x)\right]'=\mathrm{e}^x\\
			&\Leftrightarrow& \mathrm{e}^x\left[\left(x+1\right)f(x)\right]+\mathrm{e}^x\left[\left(x+1\right)f(x)\right]'=\mathrm{e}^{2x}\\
			&\Leftrightarrow& \left[\mathrm{e}^x\left(x+1\right)f(x)\right]'=\mathrm{e}^{2x}\\
			&\Rightarrow& \displaystyle\int \left[\mathrm{e}^x\left(x+1\right)f(x)\right]'\mathrm{\,d}x=\displaystyle\int \mathrm{e}^{2x}\mathrm{\,d}x\\
			&\Leftrightarrow& \mathrm{e}^x\left(x+1\right)f(x)=\dfrac{1}{2}{\mathrm{e}^{2x}}+C.
		\end{eqnarray*}
		Mà $f(0)=\dfrac{1}{2}$ $\Rightarrow C=0$.\\
		Vậy $f(x)=\dfrac{1}{2}\cdot \dfrac{\mathrm{e}^x}{x+1}$.\\
		Do đó 
		$I=\displaystyle\int\limits_1^2 \left(2x+2\right)\dfrac{1}{2}\cdot \dfrac{\mathrm{e}^x}{x+1}\mathrm{\,d}x=\displaystyle\int\limits_1^2 \mathrm{e}^x\mathrm{\,d}x=\mathrm e^2-\mathrm e$.
	}
\end{ex}

\begin{ex}%[2D4C2-3]
	Cho hàm số $y=f(x)$ liên tục, có đạo hàm trên $\mathbb{R}$ thỏa mãn điều kiện \break $f(x)+x\left[f'(x)-2\sin x\right]=x^2\cos x$, $x\in \mathbb{R}$ và $f\left(\dfrac{\pi}{2}\right)=\dfrac{\pi}{2}$. Tính $I=\displaystyle\int\limits_0^{\tfrac{\pi}{2}} \dfrac{f(x)}{x}\mathrm{\,d}x$.
	\choice
	{\True $I=1$}
	{$I=\dfrac{\pi}{2}$}
	{$I=-1$}
	{$I=-\pi$}
	\loigiai{
		Từ giả thiết $\begin{aligned}[t] &f(x)+x\left(f'(x)-2\sin x\right)=x^2\cos x\\
			&\Leftrightarrow f(x)+xf'(x)=x^2\cos x+2x\sin x\\
			&\Leftrightarrow \left(xf(x)\right)'=\left(x^2\sin x\right)'\\
			&\Leftrightarrow xf(x)=x^2\sin x+C.\end{aligned}$\\
		Mặt khác $f\left(\dfrac{\pi}{2}\right)=\dfrac{\pi}{2}\Rightarrow C=0\Rightarrow f(x)=x\sin x$.\\
		Vậy
		$I=\displaystyle\int\limits_0^{\tfrac{\pi}{2}}{\dfrac{f(x)}{x}\mathrm{\,d}x}=\displaystyle\int\limits_0^{\tfrac{\pi}{2}}{\dfrac{x\sin x}{x}\mathrm{\,d}x}=\displaystyle\int\limits_0^{\tfrac{\pi}{2}}{\sin x\mathrm{\,d}x}=1$.
	}
\end{ex}

\begin{ex}%[2D4C2-4]
	Cho hàm số $y=f(x)$ có đạo hàm trên $(0;+\infty)$ thỏa mãn $2xf'(x)+f(x)=2x$, $\forall x\in(0;+\infty)$, $f(1)=1$. Giá trị của biểu thức $f(4)$ là
	\choice
	{$\dfrac{25}{6}$}
	{$\dfrac{25}{3}$}
	{\True $\dfrac{17}{6}$}
	{$\dfrac{17}{3}$}
	\loigiai{
		Xét phương trình $2xf'(x)+f(x)=2x$ $(1)$ trên $(0;+\infty)$ ta có $$(1)\Leftrightarrow f'(x)+\dfrac{1}{2x}\cdot f(x)=1.\quad(2)$$
		Đặt $g(x)=\dfrac{1}{2x}$, ta tìm một nguyên hàm $G(x)$ của $g(x)$.\\
		Ta có $\displaystyle\int g(x)\mathrm{\,d}x=\displaystyle\int \dfrac{1}{2x}\mathrm{\,d}x=\dfrac{1}{2}\ln x+C=\ln \sqrt x+C$. Ta chọn $G(x)=\ln \sqrt x $.\\
		Nhân cả 2 vế của $(2)$ cho $\mathrm{e}^{G(x)}=\sqrt x$, ta được 
		$$\sqrt x\cdot f'(x)+\dfrac{1}{2\sqrt x}\cdot f(x)=\sqrt x \Leftrightarrow \left[\sqrt x \cdot f(x)\right]'=\sqrt x. \quad(3)$$
		Lấy tích phân 2 vế của $(3)$ từ $1$ đến $4$, ta được \\
		$\displaystyle\int\limits_1^4 \left[\sqrt x \cdot f(x)\right]'\mathrm{\,d}x= \displaystyle\int\limits_1^4 \sqrt x\mathrm{\,d}x$ $\Rightarrow \left[\sqrt x \cdot f(x)\right]\bigg|_1^4=\left.\left(\dfrac{2}{3}\sqrt{x^3}\right)\right|_1^4\Rightarrow 2f(4)-f(1)=\dfrac{14}{3}$\\
		$\Rightarrow f(4)=\dfrac{1}{2}\left(\dfrac{14}{3}+1\right)=\dfrac{17}{6}$ (vì $f(1)=1$).\\
		Vậy $f(4)=\dfrac{17}{6}$.
	}
\end{ex}

\begin{ex}%[2D4C2-2]
	Cho hàm số $f(x)$ không âm, có đạo hàm trên đoạn $[0;1]$ và thỏa mãn $f(1)=1$, $\left[2f(x)+1-x^2\right]f'(x)=2x\left[1+f(x)\right]$, $\forall x\in[0;1]$. Tích phân $\displaystyle\int\limits_0^1 f(x)\mathrm{\,d}x$ bằng
	\choice
	{$1$}
	{$2$}
	{\True $\dfrac{1}{3}$}
	{$\dfrac{3}{2}$}
	\loigiai{
		Xét trên đoạn $[0;1]$, theo đề bài ta có
		\allowdisplaybreaks
		\begin{eqnarray*}
			&&\left[2f(x)+1-x^2\right]f'(x)=2x\left[1+f(x)\right]\\
			&\Leftrightarrow& 2f(x)\cdot f'(x)=2x+\left(x^2-1\right)\cdot f'(x)+2x\cdot f(x)\\
			&\Leftrightarrow& \left[f^2(x)\right]'=\left[x^2+\left(x^2-1\right)\cdot f(x)\right]'\\
			&\Leftrightarrow& f^2(x)=x^2+\left(x^2-1\right)\cdot f(x)+C.\quad (1)
		\end{eqnarray*}
		Thay $x=1$ vào $(1)$ ta được $f^2(1)=1+C\Leftrightarrow C=0$ (vì $f(1)=1$).\\
		Do đó, $(1)$ trở thành
		\allowdisplaybreaks
		\begin{eqnarray*} &&f^2(x)=x^2+\left(x^2-1\right)\cdot f(x)\\
			&\Leftrightarrow& f^2(x)-1=x^2-1+\left(x^2-1\right)\cdot f(x)\\
			&\Leftrightarrow& \left[f(x)-1\right]\cdot \left[f(x)+1\right]=\left(x^2-1\right)\cdot \left[f(x)+1\right]\\
			&\Leftrightarrow& f(x)-1=x^2-1 ~(\text{vì } f(x)\ge 0\Rightarrow f(x)+1 > 0,\,\forall x\in[0;1])\\
			&\Leftrightarrow& f(x)=x^2.
		\end{eqnarray*}
		Vậy $\displaystyle\int\limits_0^1 f(x)\mathrm{\,d}x=\displaystyle\int\limits_0^1 x^2\mathrm{\,d}x=\left.\dfrac{x^3}{3}\right|_0^1=\dfrac{1}{3}$.
	}
\end{ex}

\begin{ex}%[2D4C2-2]
	Cho hàm số $y=f(x)$ có đạo hàm liên tục trên $[0;1]$, thỏa mãn \break  $\left[f'(x)\right]^2+4f(x)=8x^2+4,\,\forall x\in[0;1]$ và $f(1)=2$. Tính $\displaystyle\int\limits_0^1 f(x) \mathrm{\,d}x$.
	\choice
	{$\dfrac{1}{3}$}
	{$2$}
	{\True $\dfrac{4}{3}$}
	{$\dfrac{21}{4}$}
	\loigiai{
		Ta có
		\allowdisplaybreaks
		\begin{eqnarray*}
			& & \left[f'(x)\right]^2+4f(x)=8x^2+4\\
			&\Rightarrow& \displaystyle\int\limits_0^1\left[f'(x)\right]^2\mathrm{\,d}x+4\displaystyle\int\limits_0^1 f(x)\mathrm{\,d}x =\displaystyle\int\limits_0^1\left(8x^2+4\right)\mathrm{\,d}x=\dfrac{20}{3}.\quad (1)	
		\end{eqnarray*}
		Và 
		\allowdisplaybreaks
		\begin{eqnarray*}
			&&\displaystyle\int\limits_0^1 xf'(x)\mathrm{\,d}x=xf(x)\big|_0^1-\displaystyle\int\limits_0^1 f(x)\mathrm{\,d}x=2-\displaystyle\int\limits_0^1 f(x)\mathrm{\,d}x\\
			&\Rightarrow&-4\displaystyle\int\limits_0^1 xf'(x)\mathrm{\,d}x=-8+4\displaystyle\int\limits_0^1 f(x)\mathrm{\,d}x.\quad (2)
		\end{eqnarray*}
		Lại có $$\displaystyle\int\limits_0^1\left(2x\right)^2\rm{d}x=\dfrac{4}{3}.\quad (3)$$
		Cộng vế với vế của (1), (2), (3) ta được $$\displaystyle\int\limits_0^1\left(f'(x)-2x\right)^2\mathrm{\,d}x=0\Rightarrow f'(x)=2x\Rightarrow f(x)=x^2+C.$$
		Mặt khác $f(1)=C+1=2\Rightarrow C=1\Rightarrow f(x)=x^2+1$.\\
		Do đó $\displaystyle\int\limits_0^1 f(x)\mathrm{\,d}x=\displaystyle\int\limits_0^1 \left(x^2+1\right)\mathrm{\,d}x=\dfrac{4}{3}$.
	}
\end{ex}

\begin{ex}%[2D4C2-2]
	Cho hàm số $y=f(x)$ có đạo hàm liên tục trên $[0;1]$ thỏa mãn \break $3f(x)+xf'(x)\ge x^{2018}$, $\forall x\in[0;1]$. Tìm giá trị nhỏ nhất của $\displaystyle\int_0^1 f(x)\mathrm{\,d}x$.
	\choice
	{$\dfrac{1}{2018\cdot2020}$}
	{$\dfrac{1}{2019\cdot2020}$}
	{$\dfrac{1}{2020\cdot2021}$}
	{\True $\dfrac{1}{2019\cdot2021}$}
	\loigiai{
		Ta có
		\allowdisplaybreaks
		\begin{eqnarray*}
			& & 3f(x)+xf'(x)\ge{x^{2018}},\, \forall x\in[0;1]\\
			&\Leftrightarrow& 3x^2f(x)+x^3\cdot f'(x)\ge x^{2020},\, \forall x\in[0;1]\\
			&\Leftrightarrow& \left[x^3f(x)\right]'\ge x^{2020}, \, \forall x\in[0;1]\\
			&\Rightarrow& x^3f(x)\ge\displaystyle\int x^{2020}\mathrm{\,d}x,\, \forall x\in[0;1]\\
			&\Rightarrow& x^3f(x)\ge\dfrac{x^{2021}}{2021}+C,\, \forall x\in[0;1].
		\end{eqnarray*}
		Cho $x=0\Rightarrow C=0\Rightarrow x^3f(x)\ge\dfrac{x^{2021}}{2021}$, $\forall x\in[0;1]$ $\Rightarrow f(x)\ge\dfrac{x^{2018}}{2021},\,\forall x\in[0;1]$.\\
		$\Rightarrow\displaystyle\int_0^1 f(x) \mathrm{\,d}x\ge\displaystyle\int_0^1 \dfrac{x^{2018}}{2021}\mathrm{\,d}x=\left.\left(\dfrac{x^{2019}}{2019\cdot2021}\right)\right|_0^1=\dfrac{1}{2019\cdot2021}$.
	}
\end{ex}
\Closesolutionfile{ans}
% \indapan{10}{ans/ans-2-B1-D2}
% \begin{dang}{MỘT SỐ DẠNG KHÁC}
% \end{dang}
\Opensolutionfile{ans}[ans/ans-2-B1-D3]
% \TNSA
\begin{ex}%[2D4C2-2]
	Cho hàm số $y=f(x)$ có đạo hàm trên $\mathbb{R}$ thỏa mãn $$\heva{&f(0)=f'(0)=1\\&		f(x+y)=f(x)+f(y)+3xy(x+y)-1} \text{ với }x,y\in\mathbb{R}$$
	Tính $\displaystyle\int\limits_0^1 f(x-1)\mathrm{\,d}x$.
	\choice
	{$\dfrac{1}{2}$}
	{$-\dfrac{1}{4}$}
	{\True $\dfrac{1}{4}$}
	{$\dfrac{7}{4}$}
	\loigiai{
		Lấy đạo hàm theo hàm số $y$ ta được $f'(x+y)=f'(y)+3x^2+6xy$, $\forall x\in\mathbb{R}$.\\
		Cho $y=0\Rightarrow f'(x)=f'(0)+3x^2\Rightarrow f'(x)=1+3x^2$\\
		$\Rightarrow f(x)=\displaystyle\int f'(x)\mathrm{\,d}x=x^3+x+C$ mà $f(0)=1 \Rightarrow C=1$.\\
		Do đó $f(x)=x^3+x+1\Rightarrow f(x-1)=(x-1)^3+x-1+1=x^3-3x^2+4x-1$.\\
		Vậy $\displaystyle\int\limits_0^1 f(x-1)\mathrm{\,d}x=\displaystyle\int\limits_0^1 \left(x^3-3x^2+4x-1\right)\mathrm{\,d}x=\dfrac{1}{4} \displaystyle\int\limits_{-1}^0 f(x)\mathrm{\,d}x= \displaystyle\int\limits_{-1}^0 \left(x^3+x+1\right)\mathrm{\,d}x=\dfrac{1}{4}$.
	}
\end{ex}

\begin{ex}%[2D4C2-2]
	Cho hai hàm $f(x)$ và $g(x)$ có đạo hàm trên $[1;4]$, thỏa mãn $\heva{&f(1)+g(1)=4\\&g(x)=-xf'(x)\\&f(x)=-xg'(x)}$, 
	với mọi $x\in[1;4]$. Tính tích phân $I=\displaystyle\int\limits_1^4\left[f(x)+g(x)\right]\mathrm{\,d}x$.
	\choice
	{$3\ln 2$}
	{$4\ln 2$}
	{$6\ln 2$}
	{\True $8\ln 2$}
	\loigiai{
		Từ giả thiết ta có 
		\allowdisplaybreaks
		\begin{eqnarray*}
			&& f(x)+g(x)=-x\cdot f'(x)-x\cdot g'(x)\\
			&\Leftrightarrow& \left[f(x)+x\cdot f'(x)\right]+\left[g(x)+x\cdot g'(x)\right]=0\\ &\Leftrightarrow& \left[x\cdot f(x)\right]'+\left[x\cdot g(x)\right]'=0\\
			&\Rightarrow& x\cdot f(x)+x\cdot g(x)=C\\
			&\Rightarrow& f(x)+g(x)=\dfrac{C}{x}
		\end{eqnarray*}
		Mà $f(1)+g(1)=4\Rightarrow C=4\Rightarrow f(x)+g(x)=\dfrac{4}{x}$.\\
		Vậy $I=\displaystyle\int\limits_1^4 \left[f(x)+g(x)\right]\mathrm{\,d}x=\displaystyle\int\limits_1^4\dfrac{4}{x}\mathrm{\,d}x=8\ln 2$.
	}
\end{ex}
\begin{ex}%[2D4C2-5]
	Cho hai hàm $f(x)$ và $g(x)$ có đạo hàm trên $\left[1;2\right]$ thỏa mãn $f(1)=g(1)=0$ và $\heva{& \dfrac{x}{(x+1)^2}g(x)+2023x=(x+1)f'(x) \\ & \dfrac{x^3}{x+1}g'(x)+f(x)=2024x^2}\,,\forall x\in \left[1;2\right]$. \\
	Tính tích phân $I=\displaystyle\int\limits_1^2 \left[\dfrac{x}{x+1}g(x)-\dfrac{x+1}{x}f(x) \right]\mathrm{\,d}x$.
	\choice
	{\True $I=\dfrac{1}{2}$}
	{$I=1$} 
	{$I=\dfrac{3}{2}$}
	{$I=2$}
	\loigiai{
		Từ giả thiết ta có $\heva{& \dfrac{1}{(x+1)^2}g(x)-\dfrac{x+1}{x}f'(x)=-2023\\ & \dfrac{x}{x+1}g'(x)+\dfrac{1}{x^2}f(x)=2024}\,,\forall x\in \left[1;2\right]$.\\
		Suy ra
		\allowdisplaybreaks 
		\begin{eqnarray*}
			&& \left[\dfrac{1}{(x+1)^2}g(x)+\dfrac{x}{x+1}g'(x) \right]-\left[\dfrac{x+1}{x}f'(x)-\dfrac{1}{x^2}f(x) \right]=1\\  
			&\Leftrightarrow& \left[\dfrac{x}{x+1}g(x) \right]'-\left[\dfrac{x+1}{x}f(x) \right]'=1\\ 
			&\Rightarrow& \dfrac{x}{x+1}g(x)-\dfrac{x+1}{x}f(x)=x+C.
		\end{eqnarray*}
		Mà $f(1)=g(1)=0\Rightarrow C=-1 \Rightarrow \dfrac{x}{x+1}g(x)-\dfrac{x+1}{x}f(x)=x-1$.\\
		Vậy $I=\displaystyle\int\limits_1^2 \left[\dfrac{x}{x+1}g(x)-\dfrac{x+1}{x}f(x) \right]\mathrm{\,d}x=\displaystyle\int\limits_1^2 (x-1)\mathrm{\,d}x=\dfrac{1}{2}$.
	}
\end{ex}

\begin{ex}%[2D4C2-5]
	Cho hàm số $f\left(x \right)$ xác định và liên tục trên $\mathbb{R}\setminus \left\{0\right\}$ thỏa mãn $x^2f^2\left(x \right)+\left(2x-1\right)f\left(x \right)=xf'\left(x \right)-1$, với mọi $x\in \mathbb{R}\setminus \left\{0\right\}$ đồng thời thỏa mãn $f\left(1\right)=-2$. Tính $\displaystyle\int\limits_1^2 f\left(x \right)\mathrm{\,d}x$.
	\choice
	{$-\dfrac{\ln 2}{2}-1$}
	{\True $-\ln 2-\dfrac{1}{2}$}
	{$-\ln 2-\dfrac{3}{2}$}
	{$-\dfrac{\ln 2}{2}-\dfrac{3}{2}$}
	\loigiai{
		Ta có 
		\allowdisplaybreaks 
		\begin{eqnarray*}
			&& x^2f^2\left(x \right)+2xf\left(x \right)+1=xf'\left(x \right)+f\left(x \right) \\ 
			&\Leftrightarrow& \left(xf\left(x \right)+1\right)^2=\left(xf\left(x \right)+1\right)'.
		\end{eqnarray*}
		Do đó
		\allowdisplaybreaks 
		\begin{eqnarray*}
			&& \dfrac{\left(xf\left(x \right)+1\right)'}{\left(xf\left(x \right)+1\right)^2}=1\\ 
			&\Rightarrow& \displaystyle\int \dfrac{\left(xf\left(x \right)+1\right)'}{\left(xf\left(x \right)+1\right)^2}\mathrm{\,d}x=\displaystyle\int 1\mathrm{\,d}x\\
			&\Rightarrow& -\dfrac{1}{xf\left(x \right)+1}=x+C\\
			&\Rightarrow& xf\left(x \right)+1=-\dfrac{1}{x+C}.
		\end{eqnarray*}
		Mặt khác $f\left(1\right)=-2$ nên $-2+1=-\dfrac{1}{1+C}\Rightarrow C=0$.\\
		Nên suy ra $xf\left(x \right)+1=-\dfrac{1}{x}\Rightarrow f\left(x \right)=-\dfrac{1}{x^2}-\dfrac{1}{x}$.\\
		Vậy $\displaystyle\int\limits_1^2 f\left(x \right)\mathrm{\,d}x=\displaystyle\int\limits_1^2 \left(-\dfrac{1}{x^2}-\dfrac{1}{x} \right)\mathrm{\,d}x=\left.\left(-\ln x+\dfrac{1}{x} \right)\right|_1^2=-\ln 2-\dfrac{1}{2}$.
	}
\end{ex}

\begin{ex}%[2D4C2-5]
	Cho hàm số $y=f(x)$ có đạo hàm liên tục trên $\mathbb{R}$ thỏa mãn $x\cdot f(x)\cdot f'(x)=f^2(x)-x,\,\forall x\in \mathbb{R}$ và có $f(2)=1$. Tích phân $\displaystyle\int\limits_0^2 f^2(x)\mathrm{\,d}x$ bằng
	\choice
	{$\dfrac{3}{2}$}
	{$\dfrac{4}{3}$}
	{\True $2$}
	{$4$}
	\loigiai{
		Ta có
		\allowdisplaybreaks 
		\begin{eqnarray*}
			x\cdot f(x)\cdot f'(x)=f^2(x)-x &\Leftrightarrow& 2x\cdot f(x)\cdot f'(x)=2f^2(x)-2x \\
			&\Leftrightarrow& 2x\cdot f(x)\cdot f'(x)+f^2(x)=3f^2(x)-2x \\ 
			&\Leftrightarrow& \displaystyle\int\limits_0^2 \left(x\cdot f^2(x) \right)'\mathrm{\,d}x=3\displaystyle\int\limits_0^2 f^2(x)\mathrm{\,d}x-\displaystyle\int\limits_0^2 2x\mathrm{\,d}x \\ 
			&\Leftrightarrow& \left.\left(x\cdot f^2(x) \right)\right|_0^2 =3I-4\\ 
			&\Leftrightarrow& 2=3I-4\\ 
			&\Leftrightarrow& I=2.
		\end{eqnarray*}
	}
\end{ex}

\begin{ex}%[2D4C2-5]
	Cho hàm số $f\left(x \right)$ có đạo hàm liên tục trên $\mathbb{R}$, $f\left(0\right)=0$, $f'\left(0\right)\ne 0$ và thỏa mãn hệ thức $f\left(x \right)\cdot f'\left(x \right)+18x^2=\left(3x^2+x \right)f'\left(x \right)+\left(6x+1\right)f\left(x \right),\,\forall x \in \mathbb{R}$. Biết $\displaystyle\int\limits_0^1 \left(x+1\right)\mathrm{e}^{f\left(x \right)}\mathrm{\,d}x=a\mathrm{e}^2+b,\,\left(a,b\in \mathbb{Q} \right)$. Giá trị của $a-b$ bằng
	\choice
	{\True $1$}
	{$2$}
	{$0$}
	{$\dfrac{2}{3}$}
	\loigiai{
		Ta có $f\left(x \right)\cdot f'\left(x \right)+18x^2=\left(3x^2+x \right)f'\left(x \right)+\left(6x+1\right)f\left(x \right)$.\\
		Lấy nguyên hàm hai vế ta được 
		\allowdisplaybreaks 
		\begin{eqnarray*}
			\dfrac{f^2\left(x \right)}{2}+6x^3=\left(3x^2+x \right)f\left(x \right)
			&\Rightarrow& f^2\left(x \right)-2\left(3x^2+x \right)f\left(x \right)+12x^3=0\\
			&\Rightarrow& \hoac{& f\left(x \right)=6x^2 \\ & f\left(x \right)=2x.}
		\end{eqnarray*}
		\begin{enumerate}[\bf TH1:]
			\item $f\left(x \right)=6x^2$ không thoả mãn kết quả $\displaystyle\int\limits_0^1 \left(x+1\right)\mathrm{e}^{f\left(x \right)}\mathrm{\,d}x=a\mathrm{e}^2+b,\,\left(a,b\in \mathbb{Q} \right)$.
			\item $f\left(x \right)=2x \Rightarrow \displaystyle\int\limits_0^1 \left(x+1\right)\mathrm{e}^{f\left(x \right)}\mathrm{\,d}x= \displaystyle\int\limits_0^1 \left(x+1\right)\mathrm{e}^{2x}\mathrm{\,d}x=\dfrac{3}{4}\mathrm{e}^2-\dfrac{1}{4}$.\\ 
			Suy ra $a=\dfrac{3}{4};b=-\dfrac{1}{4}$.
		\end{enumerate}
		Vậy $a-b=1$.
	}
\end{ex}

\begin{ex}%[2D4C2-5]
	Cho hàm số $y=f(x)$ xác định và có đạo hàm $f'\left(x \right)$ liên tục trên $[1;3]$; $f\left(x \right)\ne 0,\,\forall x\in \left[1;3\right]$; $f'\left(x \right)\left[1+f\left(x \right) \right]^2=\left(x-1\right)^2\left[f\left(x \right) \right]^4$ và $f\left(1\right)=-1$. Biết rằng $\displaystyle\int\limits_{\mathrm{e}}^3 f\left(x \right)\mathrm{\,d}x=a\ln 3+b\,\left(a,b\in \mathbb{Z} \right)$. Giá trị của $a+b^2$ bằng
	\choice
	{$4$}
	{\True $0$}
	{$2$}
	{$-1$}
	\loigiai{
		Ta có 
		\allowdisplaybreaks 
		\begin{eqnarray*}
			f'(x)\left[1+f(x)\right]^2=(x-1)^2\left[f(x)\right]^4
			&\Rightarrow& \dfrac{f'(x)}{f^4(x)}+\dfrac{2f'(x)}{f^3(x)}+\dfrac{f'(x)}{f^2(x)}=(x-1)^2\\
			&\Rightarrow& \displaystyle\int \left(\dfrac{f'(x)}{f^4(x)}+\dfrac{2f'(x)}{f^3(x)}+\dfrac{f'(x)}{f^2(x)} \right) \mathrm{\,d}x=\displaystyle\int (x-1)^2\mathrm{\,d}x\\
			&\Rightarrow& -\left(\dfrac{1}{3f^3(x)}+\dfrac{1}{f^2(x)}+\dfrac{1}{f(x)} \right)=\dfrac{1}{3}(x-1)^3+C. \quad (*)
		\end{eqnarray*}
		Do $f(1)=-1$ nên $C=\dfrac{1}{3}$.\\ 
		Thay vào $(*)$ ta được $\left(\dfrac{1}{f(x)}+1\right)^3=-(x-1)^3 \Rightarrow f(x)=\dfrac{-1}{x}$.\\
		Khi đó $\displaystyle\int\limits_{\mathrm{e}}^3 \dfrac{-1}{x}\mathrm{\,d}x=\left.-\ln \left|x \right|\right|_{\mathrm{e}}^3=-\ln 3+1\Rightarrow a=-1,b=1$.\\ 
		Vậy $a+b^2=0$.\\
	}
\end{ex}
\Closesolutionfile{ans}
% \indapan{10}{ans/ans-2-B1-D3}

% %%Bài 3. Ứng dụng
% \section{ỨNG DỤNG HÌNH HỌC CỦA TÍCH PHÂN}
\subsection{Diện tích hình thang cong}
\subsubsection{Hình phẳng giới hạn bởi đồ thị hàm số, trục hoành và hai đường thẳng $x=a$ và $x=b$}
\begin{center}
	\begin{tikzpicture}[scale=1,font=\footnotesize,line join=round,line cap=round,>=stealth]
	% Draw axes
	\draw[->] (-0.5,0) -- (6,0) node[right] {$x$};
	\draw[->] (0,-0.5) -- (0,4) node[above] {$y$};
	
	% Labels
	\node at (0,0) [below left] {$O$};
	\node at (0.7,0) [below] {$a$};
	\node at (4.3,0) [below] {$b$};
	\node at (5.3,2) {$y = f(x)$};
	
	% Draw function curve
	\draw[thick,domain=0.5:4.7,samples=100] plot (\x,{-0.3*(\x-0.8)*(\x-2.5)*(\x-5)+1.5});
	
	% Draw vertical lines
	\draw[dashed] (0.7,0) -- (0.7,{-0.3*(0.7-0.8)*(0.7-2.5)*(0.7-5)+1.5});
	\draw[dashed] (4.3,0) -- (4.3,{-0.3*(4.3-0.8)*(4.3-2.5)*(4.3-5)+1.5});
	
	% Draw shaded area
	\fill[pattern=north east lines, pattern color=black!50] 
	(0.7,0) -- plot[domain=0.7:4.3,samples=100] (\x,{-0.3*(\x-0.8)*(\x-2.5)*(\x-5)+1.5}) -- (4.3,0) -- cycle;
	
	% Additional labels
	\node at (0.5,2.7) [below] {$x = a$};
	\node at (4.5,3.5) [below] {$x = b$};
	\node at (3.5,-0.1) [below left] {$y = 0$};
\end{tikzpicture}
\end{center}
Cho hàm số $y=f(x)$ liên tục trên $[a;b]$. Khi đó, diện tích hình phẳng giới hạn bởi đồ thị hàm số $y=f(x)$, trục hoành $Ox$ $(y=0)$ và hai đường thẳng $x=a$ và $x=b$ được tính bởi công thức

	$$S=\displaystyle\int\limits_a^b \left|f(x)\right|\mathrm{\,d}x$$

\textbf{Chú ý:} Giả sử hàm số $y=f(x)$ liên tục trên $[a;b]$. Nếu $f(x)$ không đổi dấu trên $[a;b]$ thì

	$$\displaystyle\int\limits_a^b \left|f(x)\right|\mathrm{\,d}x=\displaystyle\left|\int\limits_a^b f(x)\mathrm{\,d}x\right|.$$

\subsubsection{Hình phẳng giới hạn bởi hai đồ thị hàm số và hai đường thẳng $x=a$ và $x=b$}

\begin{center}
	\begin{tikzpicture}[scale=1,font=\footnotesize,line join=round,line cap=round,>=stealth]
	% Draw axes
\draw[->] (-0.5,0) -- (6,0) node[right] {$x$};
\draw[->] (0,-0.5) -- (0,4) node[above] {$y$};

% Labels
\node at (0,0) [below left] {$O$};
\node at (0.7,0) [below] {$a$};
\node at (4.3,0) [below] {$b$};
\node at (5,3.1) {$y = f(x)$};
\node at (5,2.1) {$y = g(x)$};
% Draw function curve
\draw[thick,domain=1:3.7,samples=100] plot (\x,{sqrt(4-(\x-2.5)^2)+1.5});
\draw[thick,domain=0.9:3.9,samples=100] plot (\x,{-sqrt(4-(\x-2.5)^2)+3.3});

% Draw vertical lines
\draw[dashed] (1.2,0) -- (1.2,{sqrt(4-(1.2-2.5)^2)+1.5});
\draw[dashed] (3.4,0) -- (3.4,{sqrt(4-(3.4-2.5)^2)+1.5});

% Draw shaded area
\fill[pattern=north east lines, pattern color=black!50] 
(1.2,1.78) -- plot[domain=1.2:3.4,samples=100] (\x,{sqrt(4-(\x -2.5)^2)+1.5}) -- (3.4,1.514) -- plot[domain=1.2:3.4,samples=100] (\x,{-sqrt(4-(\x-2.5)^2)+3.3}) -- cycle;

% Additional labels
\node at (1.2,4) [below] {$x = a$};
\node at (3.4,4) [below] {$x = b$};

\end{tikzpicture}
\end{center}
Cho 2 hàm số $y=f(x)$ và $y=g(x)$ liên tục trên $[a;b]$. Khi đó diện tích của hình phẳng giới hạn bởi đồ thị hai hàm số $y=f(x)$ và $y=g(x)$ và hai đường thẳng $x=a$ và $x=b$ được tính bởi công thức
$$
S=\displaystyle\int_a^b|f(x)-g(x)|\mathrm{\,d}x
$$
\subsection{Thể tích hình khối}
\subsubsection{Thể tích của vật thể}
\begin{center}
\begin{tikzpicture}[scale=1,font=\footnotesize,line join=round,line cap=round,>=stealth]
	\draw plot[smooth,tension=.65] coordinates{(1,2) (2.5,2.3) (3.5,2.2)};
	\draw[dashed] plot[smooth,tension=.65] coordinates{(3.5,2.2) (4,2)};
	\draw plot[smooth,tension=.65] coordinates{(4,2) (5,2.2) (5.5,2.1)};
	\draw[dashed] plot[smooth,tension=.65] coordinates{(5.5,2.1) (6,2)};
	\draw plot[smooth,tension=.65] coordinates{(1,1) (2.3,0.5) (3.5,0.8)};
	\draw[dashed] plot[smooth,tension=.65] coordinates{(3.5,0.8) (4,1)};
	\draw plot[smooth,tension=.65] coordinates{(4,1) (5,0.7) (5.5,0.8)};
	\draw[dashed] plot[smooth,tension=.65] coordinates{(5.5,0.8) (6,1)};
	\draw[dashed] (1,1) arc (-90:90:.2 and 0.5);
	\draw (1,2) arc (90:270:.2 and 0.5);
	\draw[dashed] (4,1) arc (-90:90:.2 and 0.5);
	\draw (4,2) arc (90:270:.2 and 0.5);
	\draw (6,1) arc (-90:270:.2 and 0.5);
	\fill[pattern=north east lines] (4,1) arc (-90:90:.2 and 0.5)--(4,2) arc (90:270:.2 and 0.5)--cycle;
	\draw (-.5,0)--(0.5,0) (1,0)--(3.5,0) (4,0)--(5.5,0);
	\draw[dashed] (0.5,0)--(1,0) (3.5,0)--(4,0) (5.5,0)--(6,0);
	\draw[->] (6,0)--(7,0)node[below]{$x$};
	\draw (0.5,-1)--(0.5,3)--(1.5,3.5)--(1.5,2.2) (1.5,.8)--(1.5,-0.5)--(0.5,-1);
	\draw[dashed](1.5,2.2)--(1.5,.8);
	\draw[dashed] (1,1)--(1,0)node[below]{$a$};
	\coordinate (A) at (0.5,3);
	\coordinate (B) at (1.5,3.5);
	\coordinate (C) at (1.5,2.2);
	%\tkzMarkAngle[size=.6](A,B,C);
	\draw pic[draw=black, angle eccentricity=1.6, angle radius=0.5cm]{angle=A--B--C};
	\draw (1.3,3.2) node {\footnotesize $P$};
	\draw (3.5,-1)--(3.5,3)--(4.5,3.5)--(4.5,2) (4.5,1)--(4.5,-0.5)--(3.5,-1);
	\draw[dashed](4.5,2)--(4.5,1);
	\draw[dashed] (4,1)--(4,0)node[below]{$x$};
	\coordinate (D) at (3.5,3);
	\coordinate (E) at (4.5,3.5);
	\coordinate (F) at (4.5,2);
	%	\tkzMarkAngle[size=.6](D,E,F);
	\draw pic[draw=black, angle eccentricity=1.6, angle radius=0.5cm]{angle=D--E--F};
	\draw (4.3,3.2) node {\footnotesize $R$};
	\draw (5.5,-1)--(5.5,3)--(6.5,3.5)--(6.5,-0.5)--(5.5,-1);
	\draw[dashed] (6,1)--(6,0)node[below]{$b$};
	\coordinate (G) at (5.5,3);
	\coordinate (H) at (6.5,3.5);
	\coordinate (K) at (6.5,-0.5);
	%	\tkzMarkAngle[size=.6](G,H,K);
	\draw pic[draw=black, angle eccentricity=1.6, angle radius=0.5cm]{angle=G--H--K};	\draw (6.3,3.2) node {\footnotesize $Q$};
	\draw (0,.3) node {$O$};
	\fill (0,0) circle(1pt);
	\draw[->] (4,1.5)--(4.7,1.7) node[right] {\scriptsize $S(x)$};
\end{tikzpicture}
\end{center}
Trong không gian, cho một vật thể nằm trong khoảng không gian giữa hai mặt phẳng $(P)$ và $(Q)$ cùng vuông góc với trục $Ox$ tại các điểm $a$ và $b$. Mặt phẳng vuông góc với trục $Ox$ tại điểm $x(a\leq x \leq b)$ cắt vật thể theo mặt cắt có diện tích $S(x)$. Khi đó, nếu $S(x)$ là hàm số liên tục trên $[a;b]$ thì thể tích của vật thể được tính bởi công thức
	$$V=\displaystyle\int\limits_a^b S(x)\mathrm{\,d}x$$
\subsubsection{Thể tích khối tròn xoay}
\begin{center}
	\begin{tikzpicture}[scale=1,font=\footnotesize,line join=round,line cap=round,>=stealth]
	 \draw[->] (-1,0) -- (6,0) node[right] {$x$};
	\draw[->] (0,-3) -- (0,3) node[above] {$y$};
	
	\draw[black, thick] plot[domain=0.5:5] (\x, {0.8*(0.4*(\x-1)-0.4)^2+1});
	\draw[black, thick, dashed] plot[domain=0.5:5] (\x, {-0.8*(0.4*(\x-1)-0.4)^2-1});
	
	\draw[black, thick,dashed, domain=4.54:5.54, samples=100] plot (\x, {sqrt(4.63 * (1 - 4 * (\x - 5.04)^2))});
	\draw[black, thick,dashed, domain=4.54:5.54, samples=100] plot (\x, {-sqrt(4.63 * (1 - 4 * (\x - 5.04)^2))});
	
	
	\draw[thick,dashed,domain=0.5:1.50 ,samples=100] plot (\x,{sqrt(0.5^2-(\x -1)^2)*1.5*1.5});
	\draw[thick,dashed,domain=0.5:1.50,samples=100] plot (\x,-{sqrt(0.5^2-(\x-1)^2)*1.5*1.5});
	
	\draw[thick,dashed,domain=2.55:3.551,samples=100] plot (\x,-{sqrt(1.274-5.09*(\x-3.05)^2)});
	\draw[thick,dashed,domain=2.55:3.551,samples=100] plot (\x,{sqrt(1.274-5.09*(\x-3.05)^2)});
	
	% Additional labels
	\node at (1,0) [below] {$a$};
	\node at (3,0) [below] {$x$};
	\node at (5,0) [below] {$b$};
	\node at (0,0) [below left] {$O$};
	\node at (3,2) [below] {$y=f(x)$};
	% Draw vertical lines
	\draw[thick] (1,0) -- (1,{0.8*(0.4*(1-1)-0.4)^2+1});
	\draw[thick] (3,0) -- (3,{0.8*(0.4*(3-1)-0.4)^2+1});
	\draw[thick] (5,0) -- (5,{0.8*(0.4*(5-1)-0.4)^2+1});
	
	%fill
	% Draw shaded area
	\fill[pattern=north east lines, pattern color=black!50] 
	(1,0) -- plot[domain=1:5,samples=100] (\x,{0.8*(0.4*(\x-1)-0.4)^2+1}) -- (5,0) -- cycle;
	
\end{tikzpicture}
\end{center}

Cho hàm số $y=f(x)$ liên tục, không âm trên $[a;b]$. Hình phẳng $(H)$ giới hạn bởi đồ thị hàm số $y=f(x)$, trục hoành $O x$ và hai đường thẳng $x=a$ và $x=b$ quay quanh trục $O x$ tạo thành một khối tròn xoay có thể tích bằng

	$$V=\displaystyle\pi\int\limits_a^b \left[f(x)\right]^2\mathrm{\,d}x$$

\begin{dang}{TÍNH DIỆN TÍCH HÌNH GIỚI HẠN BỞI CÁC ĐƯỜNG CONG}
\end{dang}

%\TN
\Opensolutionfile{ans}[ans/ans2-C4B3CD1-D1]
\begin{ex}%[2D4N3-1]
	Cho hai hàm số $f(x)$ và $g(x)$ liên tục trên $[a;b]$. Diện tích hình phẳng giới hạn bởi đồ thị của các hàm số $y=f(x), y=g(x)$ và các đường thẳng $x=a, x=b$ bằng
\choice
{$\left|\displaystyle\int\limits_a^b \left[f(x)-g(x)\right]\mathrm{\,d}x\right|$}
{$\displaystyle\int\limits_a^b \left|f(x)+g(x)\right|\mathrm{\,d}x$}
{\True $\displaystyle\int\limits_a^b \left|f(x)-g(x)\right|\mathrm{\,d}x$}
{$\displaystyle\int\limits_a^b \left[f(x)-g(x)\right]\mathrm{\,d}x$}
\loigiai{
Theo lý thuyết thì diện tích hình phẳng được giới hạn bởi đồ thị của các đường\\ $y=f(x), y=g(x)$, $x=a, x=b$ 
\\Được tính theo công thức $S=\displaystyle\int\limits_a^b \left|f(x)-g(x)\right|\mathrm{\,d}x$.
}
\end{ex}
\begin{ex}%[2D4N3-1]
	Gọi $S$ là diện tích của hình phẳng giới hạn bởi các đường $y=3^x$, $y=0$, $x=0$, $x=2$. Mệnh đề nào dưới đây đúng?
\choice
{\True $\displaystyle\int\limits_0^2 3^x\mathrm{\,d}x$}
{$S=\pi\displaystyle\int\limits_0^2 3^{2x}\mathrm{\,d}x$}
{$S=\pi\displaystyle\int\limits_0^2 3^x\mathrm{\,d}x$}
{$S=\displaystyle\int\limits_0^2 3^{2x}\mathrm{\,d}x$}

\loigiai{
Diện tích hình phẳng đã cho được tính bởi công thức $S=\displaystyle\int\limits_0^2 3^x\mathrm{\,d}x$.
}
\end{ex}%[2D4N3-1]
\begin{ex}%[2D4N3-1]
	Diện tích hình phẳng giới hạn bởi đồ thị hàm số $y=(x-2)^2-1$, trục hoành và hai đường thẳng $x=1, x=2$ bằng
\choice
{\True $\dfrac{2}{3}$}
{$\dfrac{3}{2}$}
{$\dfrac{1}{3}$}
{$\dfrac{7}{3}$}
\loigiai{
Ta có $S=\displaystyle\int\limits_1^2 \left|(x-2)^2-1\right|\mathrm{\,d}x=\displaystyle\int\limits_1^2\left|x^2-4 x+3\right| \mathrm{d}x=\left|\displaystyle\int\limits_1^2\left(x^2-4 x+3\right) \mathrm{d}x\right|=\dfrac{2}{3}$.
}

\end{ex}
\begin{ex}%[2D4H3-1]
	Tính diện tích $S$ hình phẳng giới hạn bởi các đường $y=x^2+1$, $x=-1$, $x=2$ và trục hoành.
\choice
{\True $S=6$}
{$S=16$}
{$S=\dfrac{13}{6}$}
{$S=13$}
\loigiai{
Ta có $S=\displaystyle\int\limits_{-1}^2\left|x^2+1\right| \mathrm{d}x=\displaystyle\int\limits_{-1}^2\left(x^2+1\right) \mathrm{d}x=6$.
}
\end{ex}
\begin{ex}%[2D4H3-1]
	Gọi $S$ là diện tích hình phẳng giới hạn bởi các đường $y=x^2+5, y=6 x, x=0, x=1$. Tính $S$.
\choice
{$\dfrac{4}{3}$}
{\True $\dfrac{7}{3}$}
{$\dfrac{8}{3}$}
{$\dfrac{5}{3}$}
\loigiai{
Diện tích hình phẳng cần tìm $S=\displaystyle\int\limits_0^1\left|x^2-6 x+5\right| \mathrm{d}x=\dfrac{7}{3}$.
}
\end{ex}
\begin{ex}%[2D4V3-1]
	Diện tích hình phẳng giới hạn bởi đồ thị các hàm số $y=\ln x, y=1$ và hai đường thẳng $x=1, x=e$ bằng
\choice
{$e^2$}
{$e+2$}
{$2 e$}
{\True $e-2$}
\loigiai{
\begin{eqnarray*}
S&=&\displaystyle\int\limits_1^e|\ln x-1|\mathrm{d}x\\&=&\left|\displaystyle\int\limits_1^e(\ln x-1) \mathrm{d}x\right|\\&=&|x(\ln x-1)|_1^e-\displaystyle\int\limits_1^e \mathrm{d}x|\\&=&| 1-\left.x\right|_1 ^e|\\&=&| 1-(e-1)|=| 2-e \mid\\&=&e-2.
\end{eqnarray*}
}
\end{ex}
\begin{ex}%[2D4H3-1]
	Diện tích hình phẳng giới hạn bởi đồ thị của hàm số $y=4 x-x^2, y=2 x$ và hai đường thẳng $x=1, x=e$ bằng
\choice
{$4$}
{$\dfrac{20}{3}$}
{\True $\dfrac{4}{3}$}
{$\dfrac{16}{3}$}
\loigiai{
	Diện tích hình phẳng cần tìm là\[S=\displaystyle\int\limits_0^2\left|x^2-2 x\right| \mathrm{d}x=\displaystyle\int\limits_0^2\left(2 x-x^2\right) \mathrm{d}x=\left.\left(x^2-\dfrac{x^3}{3}\right)\right|_0 ^2=\dfrac{4}{3}.\]
}
\end{ex}
\begin{ex}%[2D4V3-1]
	Tính diện tích $S$ của hình phẳng giới hạn bởi các đường $y=x^2-2 x$, $y=0$, $x=-10$, $x=10$.
\choice
{$S=\dfrac{2000}{3}$}
{$S=2008$}
{$S=2000$}
{\True $S=\dfrac{2008}{3}$}
\loigiai{
Phương trình hoành độ giao điểm của hai đường $(C)\colon y=x^2-2 x$ và $(d)\colon y=0$ là
$$
x^2-2 x=0 \Leftrightarrow\hoac{
	x=0 \\
	x=2
}.
$$
Bảng xét dấu\\
\begin{center}
	\begin{tikzpicture}
	\tkzTabInit[nocadre=false,lgt=2.5,espcl=2.5,deltacl=0.6]
	{$x$ /0.6, VT/0.6}
	{$-\infty$, $0$, $2$, $+\infty$}
	\tkzTabLine{,+,,-,,+,}
\end{tikzpicture}
\end{center}
Diện tích cần tìm

\begin{eqnarray*}
	S&=&\displaystyle\int\limits_{-10}^{10}\left|x^2-2 x\right| \mathrm{d}x\\&=&\displaystyle\int\limits_{-10}^0\left(x^2-2 x\right) \mathrm{d}x-\displaystyle\int\limits_0^2\left(x^2-2 x\right) \mathrm{d}x+\displaystyle\int\limits_2^{10}\left(x^2-2 x\right) \mathrm{d}x 
	\\&=&\left.\left(\dfrac{x^3}{3}-x^2\right)\right|_{-10} ^0-\left.\left(\dfrac{x^3}{3}-x^2\right)\right|_0 ^2+\left.\left(\dfrac{x^3}{3}-x^2\right)\right|_2 ^{10}\\&=&\dfrac{1300}{3}+\dfrac{4}{3}+\dfrac{704}{3}\\&=&\dfrac{2008}{3} .
\end{eqnarray*}
}
\end{ex}
\Closesolutionfile{ans}
% \indapan{10}{ans/ans2-C4B3CD1-D1}
%\TNTF
\Opensolutionfile{ans}[ans/ans2-C4B3CD1-D1-DS]
\begin{ex}%[2D4N3-1]
	Gọi $S$ là diện tích của hình phẳng giới hạn bời các đường $y=2^x$, $y=0$, $x=0$, $x=2$. Các mệnh đề sau đây đúng hay sai?
\choiceTF
{\True $S=\displaystyle\int\limits_0^2 2^x \mathrm{d}x$}
{\True $S=\dfrac{3}{\ln 2}$}
{$S=\pi \displaystyle\int\limits_0^2 2^x \mathrm{d}x$}
{$S=\dfrac{3 \pi}{\ln 2}$}
\loigiai{
\[
S=\displaystyle\int\limits_0^2\left|2^x\right| \mathrm{d}x=\displaystyle\int\limits_0^2 2^x \mathrm{d}x=\dfrac{2^2}{\ln 2}-\dfrac{2^0}{\ln 2}=\dfrac{3}{\ln 2}\left( \text {do } 2^x>0, \forall x \in[0;2]\right) .
\]
}
\end{ex}
\begin{ex}%[2D4N3-1]
	Gọi $S$ là diện tích hình phẳng giới hạn bởi các đường $y=\mathrm{e}^x, y=0, x=0, x=2$. Các mệnh đề sau đây đúng hay sai?
\choiceTF
{\True $S=\displaystyle\int\limits_0^2 \mathrm{e}^x \mathrm{d}x$}
{$S=e^2$}
{$S=\pi \displaystyle\int\limits_0^2 \mathrm{e}^x \mathrm{d}x$}
{$S=\left(e^2-1\right) \pi$}
\loigiai{
Diện tích hình phẳng giới hạn bời các đường $y=\mathrm{e}^x, y=0, x=0, x=2$ là
\[
S=\displaystyle\int\limits_0^2 e^x \mathrm{d}x=e^2-1.
\]
}
\end{ex}
\begin{ex}%[2D4V3-1]
	Các mệnh đề sau đây đúng hay sai
\choiceTF
{\True Diện tích hình phẳng giới hạn bởi đồ thị hàm số $y=x^2$, $y=2 x$, $x=0$, $x=1$ là $\dfrac{4}{3}$}
{\True Diện tích hình phẳng giới hạn bởi đồ thị hàm số $y=-x^2+2 x+1$, $y=2 x^2-4 x+1$, $x=0$, $x=2$ là $4$}
{\True Diện tích hình phẳng giới hạn bởi đồ thị hàm số $y=\dfrac{x-1}{x+1}$, trục hoành, $x=0$, $x=1$ là $2 \ln 2-1$}
{\True Diện tích hình phẳng giới hạn bởi đồ thị hàm số $y=-x^3+12 x$, $y=-x^2$, $x=-3$, $x=4$ là $\dfrac{937}{12}$}
\loigiai{
\begin{itemchoice}
\itemch Đúng.\\Diện tích hình phẳng giới hạn bời đồ thị hàm số $y=x^2$, $y=2 x$, $x=0$, $x=1$ là
\begin{eqnarray*}
S&=&\displaystyle\int\limits_0^1\left|x^2-x\right| \mathrm{d}x\\&=&\left|\displaystyle\int\limits_0^1\left(x^2-x\right) \mathrm{d}x\right|\\&=&\dfrac{4}{3}.
\end{eqnarray*}

\itemch Đúng.\\Diện tích hình phẳng giới hạn bởi đồ thị hàm số $y=-x^2+2x+1$, $y=2 x^2-4 x+1$, $x=0$, $x=2$ là
\begin{eqnarray*}
&&\displaystyle\int\limits_0^2\left|2x^2-4x+1-\left(-x^2+2 x+1\right)\right| \mathrm{d}x
\\&=&\displaystyle\int\limits_0^2\left|3 x^2-6 x\right| \mathrm{d}x
\\&=&\displaystyle\int\limits_0^2\left(6 x-3 x^2\right) \mathrm{d}x
\\&=&\left(3 x^2-x^3\right)|_0^2=4.
\end{eqnarray*}

\itemch Đúng.\\Diện tích hình phẳng giới hạn bởi đồ thị hàm số $y=\dfrac{x-1}{x+1}$, trục hoành, $x=0$, $x=1$ là

\begin{eqnarray*}
S&=&\displaystyle\int\limits_0^1\left|\dfrac{x-1}{x+1}\right| \mathrm{d}x
\\&=&\left|\displaystyle\int\limits_0^1\left(\dfrac{x-1}{x+1}\right)\mathrm{d}x\right|
\\&=&\left|\displaystyle\int\limits_0^1\left(1-\dfrac{2}{x+1}\right)\mathrm{d}x\right|
\\&=&|(x-2 \ln |x+1|)|_0^1|
\\&=&2 \ln {2}-1.
\end{eqnarray*}
\itemch Đúng.\\Diện tích hình phẳng giới hạn bởi đồ thị hàm số $y=-x^3+12$, $y=-x^2$, $x=-3$ là

\begin{eqnarray*}
	S&=&\displaystyle\int\limits_{-3}^4\left|x^3-x^2-12x\right| \mathrm{d}x
	\\&=&\displaystyle\int\limits_{-3}^0\left|x^3-x^2-12x\right| \mathrm{d}x+\displaystyle\int\limits_0^4\left|x^3-x^2-12 x\right| \mathrm{d}x 
	\\&=&\left|\displaystyle\int\limits_{-3}^0\left(x^3-x^2-12 x\right) \mathrm{d}x\right|+\left|\displaystyle\int\limits_0^4\left(x^3-x^2-12 x\right) \mathrm{d}x\right|
	\\&=&\left| \left|\left(\dfrac{x^4}{4}-\dfrac{x^3}{3}-6 x^2\right)\right|_{-3}^0\right| +\left| \left(\dfrac{x^4}{4}-\dfrac{x^3}{3}-6 x^2\right)\right|_0^4
	\\&=&\left|\dfrac{-99}{4}\right|+\left|\dfrac{-160}{3}\right|=\dfrac{937}{12}.
\end{eqnarray*}
\end{itemchoice}
}
\end{ex}
\Closesolutionfile{ans}
% \indapan{3}{ans/ans2-C4B3CD1-D1-DS}
%\TN
\Opensolutionfile{ans}[ans/ans2-C4B3CD1-D1-KQ]
\begin{ex}%[2D4V3-1]
	Tính diện tích hình phẳng giới hạn bời đồ thị hàm số $y=x^2+x-1$, $y=x^4+x-1$, $x=-1$, $x=1$.
\shortans{$0,27$}
\loigiai{
Diện tích hình phẳng giới hạn bởi đồ thị hàm số $y=x^2+x-1, y=x^4+x-1, x=-1, x=1$ là
\begin{eqnarray*}
	S&=&\displaystyle\int\limits_{-1}^1\left|x^2-x^4\right| \mathrm{d}x\\
	&=&\displaystyle\int\limits_{-1}^0\left|x^2-x^4\right| \mathrm{d}x+\displaystyle\int\limits_0^1\left|x^2-x^4\right| \mathrm{d}x\\
	&=&\left|\displaystyle\int\limits_{-1}^0\left(x^2-x^4\right) \mathrm{d}x\right|+\left|\displaystyle\int\limits_0^1\left(x^2-x^4\right) \mathrm{d}x\right|\\
	&=&\left|\left(\dfrac{x^3}{3}-\dfrac{x^5}{5}\right)\right| 0|+|\left(\dfrac{x^3}{3}-\dfrac{x^5}{5}\right)|0|\\
	&=&\dfrac{2}{15}+\dfrac{2}{15}=\dfrac{4}{15}\approx0,27.
\end{eqnarray*}
}
\end{ex}

\begin{ex}%[2D4V3-1]
	Kí hiệu $S(t)$ là diện tích của hình phẳng giới hạn bởi các đường $y=2 x+1$, $y=0$, $x=1$, $x=t\left(t>1\right)$. Tìm $t$ để $S(t)=10$.
\shortans{$3$}
\loigiai{
\textbf{Cách 1.} Ta có $S(t)=\displaystyle\int\limits_1^t|2 x+1| \mathrm{d}x=\displaystyle\int\limits_1^t(2 x+1) \mathrm{d}x$.\\
Suy ra $S(t)=\left.\left(x^2+x\right)\right|_1^t=t^2+t-2$.\\
Do đó $S(t)=10 \Leftrightarrow t^2+t-2=10 \Leftrightarrow t^2+t-12=0 \Leftrightarrow\hoac{&t=3 \\ &t=-4\text{ (L)}}.$\\
Vậy $t=3$.\\
\textbf{Cách 2}. Hình phẳng đã cho là hình thang có đáy nhỏ bằng $y(1)=3$, đáy lớn bằng $y(t)=2 t+1$ và chiều cao bằng $t-1$.
Ta có \[\dfrac{(3+2t+1)(t-1)}{2}=10 \Leftrightarrow 2 t^2+2 t-24=0 \Leftrightarrow\hoac{t=3 \\ t=-4}.\]\\ 
Vì $t>1$ nên $t=3$.
}
\end{ex}
\begin{ex}%[2D4V3-1]
	 Gọi $S$ là diện tích hình phẳng giới hạn bởi các đường $m y=x^2$, $m x=y^2(m>0)$. Tìm giá trị của $m$ để $S=3$.

\shortans{$3$}
\loigiai{
Tọa độ giao điểm của hai đồ thị hàm số là nghiệm của hệ phương trình $\heva{my=x^2 &\quad(1)\\mx=y^2 &\quad(2)}$
Thế $(1)$ vào $(2)$ ta được $m x=\left(\dfrac{x^2}{m}\right)^2 \Leftrightarrow m^3 x-x^4=0 \Leftrightarrow\hoac{&x=0\\&x=m>0.}$\\
Vì $y=\dfrac{x^2}{m}>0$ nên $m x=y^2 \text{ (với $y>0$) }  \Leftrightarrow y=\sqrt{m x}$\\
Khi đó diện tích hình phẳng cần tìm là 
\begin{eqnarray*}
	S&=&\displaystyle\int\limits_0^m\left|\sqrt{m x}-\dfrac{x^2}{m}\right| \mathrm{d}x=\left|\displaystyle\int\limits_0^m\left(\sqrt{m x}-\dfrac{x^2}{m}\right) \mathrm{d}x\right|\\
	&=&\left|\left(\dfrac{2 \sqrt{m}}{3} \cdot x^{\dfrac{3}{2}}-\dfrac{x^3}{3 m}\right)\right|_0^m\\
	&=& \left|\dfrac{1}{3} m^2 \right|=\dfrac{1}{3} m^2
\end{eqnarray*}
	Yêu cầu bài toán $S=3 \Leftrightarrow \dfrac{1}{3} m^2=3 \Leftrightarrow m^2=9 \Leftrightarrow m=3$.

}
\end{ex}
\begin{ex}%[2D4V3-1]
	Giá trị dương của tham số $m$ sao cho diện tích hình phẳng giới hạn bởi đồ thị của hàm số $y=2 x+3$ và các đường thẳng $y=0$, $x=0$, $x=m$ bằng 10 là?

\shortans{$2$}
\loigiai{
Vì $m>0$ nên $2 x+3>0, \forall x \in[0;m]$.
Diện tích hình phẳng giới hạn bởi đồ thị hàm số $y=2 x+3$ và các đường thẳng $y=0$, $x=0$, $x=m$ là
\[S=\displaystyle\int\limits_0^m(2 x+3) \cdot \mathrm{d}x=\left.\left(x^2+3 x\right)\right|_0^m=m^2+3m.\]
Theo giả thiết ta có
\[S=10 \Leftrightarrow m^2+3 m=10 \Leftrightarrow m^2+3 m-10=0 \Leftrightarrow \hoac{&m=2\\&m=-5}
 \Leftrightarrow m=2 \text { do } m>0.\]
}
\end{ex}
\begin{ex}%[2D4V3-1]
	Cho hàm số  $f(x)=\heva{&7-4x^3\text{ khi }  0 \leq x \leq 1\\&4-x^2 \text{ khi } x>1}$. Tính diện tích hình phẳng giới hạn bởi đồ thị hàm số $f(x)$ và các đường thẳng $x=0$, $x=3$, $y=0$.
\shortans{10}
\loigiai{

\begin{center}
	\begin{tikzpicture}[scale=0.6, font=\footnotesize, line join=round, line cap=round, >=stealth]
		% Vẽ trục
		\draw[->] (-0.5,0) -- (4,0) node[right] {$x$};
		\draw[->] (0,-5) -- (0,7) node[above] {$y$};
		
		% Đánh dấu các điểm trên trục x
		\foreach \x in {1,2,3}{
			\draw[fill=black] (\x,0) circle(0.03) node[below]{$\x$};
		}
		
		% Đánh dấu các điểm trên trục y
		\foreach \y in {-5,-4,-3,-2,-1,1,2,3,4,5,6,7}{
			\draw[fill=black] (0,\y) circle(0.03) node[left]{$\y$};
		}
		% Đánh dấu gốc tọa độ
		\draw[fill=black] (0,0) circle(0.03) node[below left] {$0$};
		
		
			
		% Draw function curve
		\draw[thick,domain=0:1,samples=100] plot (\x,{-4*(\x)*(\x)+7});
		\draw[thick,domain=1:3,samples=100] plot (\x,{-(\x)*(\x)+4});
		% Draw lines
		\draw (1,0) -- (1,3);
		\draw (3,0) -- (3,-5);
		\draw[dashed] (0,3) -- (1,3);
		\draw[dashed] (0,-5) -- (3,-5);
		% Draw shaded area
		\fill[pattern=north east lines, pattern color=black!50] 
		(0,0) -- plot[domain=0:1,samples=100] (\x,{-4*(\x)*(\x)+7}) -- (1,0) -- cycle;
		\fill[pattern=north east lines, pattern color=black!50] 
		(1,0) -- plot[domain=1:2,samples=100] (\x,{-(\x)*(\x)+4}) -- (2,0) -- cycle;
		\fill[pattern=north east lines, pattern color=black!50] 
		(2,0) -- plot[domain=2:3,samples=100] (\x,{-(\x)*(\x)+4}) -- (3,0) -- cycle;
	\end{tikzpicture}
\end{center}

\begin{eqnarray*}
	 S&=&\displaystyle\int\limits_0^1\left(7-4 x^3\right) \mathrm{d}x+\displaystyle\int\limits_1^2\left(4-x^2\right) \mathrm{d}x+\displaystyle\int\limits_2^3\left(x^2-4\right) \mathrm{d}x \\ & =&\left.\left(7 x-x^4\right)\right|_0 ^1+\left.\left(4 x-\dfrac{x^3}{3}\right)\right|_1 ^2+\left.\left(\dfrac{x^3}{3}-4 x\right)\right|_2 ^3
	\\&=&6+4-\dfrac{7}{3}-3-\dfrac{8}{3}+8=10 .
\end{eqnarray*}
}
\end{ex}
\Closesolutionfile{ans}
% \indapan{5}{ans/ans2-C4B3CD1-D1-KQ}
% \begin{dang}
% 	{TÍNH DIỆN TÍCH GIỚI HẠN BỞI CÁC ĐƯỜNG CONG KHI BIẾT ĐỒ THỊ HÀM SỐ CỦA CÁC ĐƯỜNG CONG}
% \end{dang}
\Opensolutionfile{ans}[ans/ans-2-C4B3CD1_10-19]

%\TN
%%%%-------------Câu 17
\begin{ex}%[2D4N3-1]
	\immini{
		Gọi $S$  là diện tích hình phẳng giới hạn bởi đồ thị hàm số  $y=f(x)$, trục hoành, đường thẳng $x=a$, $x=b$  (như hình vẽ bên). Hỏi cách tính $S$  nào dưới đây đúng?
	}{
		\begin{tikzpicture}[scale=.7,>=stealth, font=\footnotesize, line join=round, line cap=round]
			\def\a{-0.25} \def\b{2.5} \def\c{-6.75} \def\d{4.5} % Hệ số
			\def\xmin{-1} \def\xmax{7}
			\def\ymin{-2} \def\ymax{3} 
			\draw[->] (\xmin,0)--(\xmax,0) node [below]{$x$};
			\draw[->] (0,\ymin)--(0,\ymax) node [left]{$y$};
			\node at (0,0) [below left]{$O$};
			\draw[smooth,samples=300] plot[domain=1:6](\x,{\a*(\x)^3+\b*(\x)^2+\c*(\x)+\d});
			\draw[pattern=north east lines] plot[domain=1:6](\x,{\a*(\x)^3+\b*(\x)^2+\c*(\x)+\d});
			\node[below left] at (1,0) {$a$};
			\node[below right] at (3,0) {$c$};
			\node[below] at (6,0) {$b$};
			\node[] at (4,2.5) {$y=f(x)$};
		\end{tikzpicture}
	}
	\choice
	{$S=\displaystyle\int\limits_a^b f(x) \mathrm{\,d}x$}
	{$ S= \left|\displaystyle\int\limits_a^c f(x) \mathrm{\,d}x + \displaystyle\int\limits_c^b f(x) \mathrm{\,d}x \right|$}
	{\True  $S=-\displaystyle\int\limits_a^c f(x) \mathrm{\,d}x + \displaystyle\int\limits_c^b f(x) \mathrm{\,d}x$}
	{$S=\displaystyle\int\limits_a^c f(x) \mathrm{\,d}x + \displaystyle\int\limits_c^b f(x) \mathrm{\,d}x$}
	\loigiai{
		Ta có $y=f(x)$ liên tục trên đoạn $\left[a; b\right]$.\\
		Dựa vào đồ thị ta có $\left|f(x)\right|=\heva{& -f(x), & a\le x \le c\\& f(x), & c< x \le b.}$\\
		Suy ra 
		$S= \displaystyle\int\limits_a^b \left|f(x)\right| \mathrm{\,d}x = 
		\displaystyle\int\limits_a^c \left|f(x)\right| \mathrm{\,d}x +\displaystyle\int\limits_c^b \left|f(x)\right| \mathrm{\,d}x = -\displaystyle\int\limits_a^c f(x) \mathrm{\,d}x +
		\displaystyle\int\limits_c^b f(x) \mathrm{\,d}x$.
		
	}
\end{ex}	

%%%%%-------------Câu 18
\begin{ex}%[2D4N3-1]
	\immini{
		Cho hàm số $y=f(x)$  liên tục trên đoạn  $\left[a; b\right]$. Gọi $D$  là diện tích hình phẳng giới hạn bởi đồ thị  $\left(C\right)\colon y=f(x)$, trục hoành, hai đường thẳng  $x=a$, $x=b$ (như hình vẽ). Giả sử  $S_D$ là diện tích hình phẳng  $D$. Chọn phương án đúng trong các phương án {\bf A}, {\bf B}, {\bf C}, {\bf D} cho dưới đây?
	}{
		\begin{tikzpicture}[yscale=.7,xscale=1,>=stealth, font=\footnotesize, line join=round, line cap=round]
			\def\a{1/3} \def\b{0} \def\c{0} \def\d{0} % Hệ số
			\def\xmin{-3} \def\xmax{3}
			\def\ymin{-3} \def\ymax{3} 
			\draw[->] (\xmin,0)--(\xmax,0) node [below]{$x$};
			\draw[->] (0,\ymin)--(0,\ymax) node [left]{$y$};
			\node at (0,0) [below left]{$O$};
			\draw[smooth,samples=300] plot[domain=-2.1:2.1](\x,{\a*(\x)^3+\b*(\x)^2+\c*(\x)+\d});
			\draw[pattern=north east lines] plot[domain=0:-2](\x,{\a*(\x)^3+\b*(\x)^2+\c*(\x)+\d})--(-2,0)--cycle
			plot[domain=0:2](\x,{\a*(\x)^3+\b*(\x)^2+\c*(\x)+\d})--(2,0)--cycle;
			\node[above] at (-2,0) {$a$};
			\node[below] at (2,0) {$b$};
		\end{tikzpicture}
	}
	\choice
	{$S_D=\displaystyle\int\limits_a^0 f(x) \mathrm{\,d}x +\displaystyle\int\limits_0^b f(x) \mathrm{\,d}x$}
	{\True  $S_D=-\displaystyle\int\limits_a^0 f(x) \mathrm{\,d}x + \displaystyle\int\limits_0^b f(x) \mathrm{\,d}x$}
	{$S_D=\displaystyle\int\limits_a^0 f(x) \mathrm{\,d}x -\displaystyle\int\limits_0^b f(x) \mathrm{\,d}x$}
	{$S_D=-\displaystyle\int\limits_a^0 f(x) \mathrm{\,d}x -\displaystyle\int\limits_0^b f(x) \mathrm{\,d}x$}
	\loigiai{
		Ta có $y=f(x)$ liên tục trên đoạn $\left[a; b\right]$.\\
		Dựa vào đồ thị ta có $\left|f(x)\right|=\heva{& -f(x), & a\le x \le 0\\& f(x), & 0< x \le b.}$\\
		Suy ra $S_D= \displaystyle\int\limits_a^b \left|f(x)\right| \mathrm{\,d}x = 
		\displaystyle\int\limits_a^0 \left|f(x)\right| \mathrm{\,d}x +\displaystyle\int\limits_0^b \left|f(x)\right| \mathrm{\,d}x = -\displaystyle\int\limits_a^0 f(x) \mathrm{\,d}x + \displaystyle\int\limits_0^b f(x) \mathrm{\,d}x$.
	}
\end{ex}	

%%%%%-------------Câu 19
\begin{ex}%[2D4N3-1]
	\immini{
		Diện tích của hình phẳng được giới hạn bởi đồ thị hàm số $y=f(x)$, trục hoành và hai đường thẳng  $x=a$,  $x=b$  $(a<b)$ (phần tô đậm trong hình vẽ) tính theo công thức nào dưới đây?
	}{
		\begin{tikzpicture}[yscale=1,xscale=.8,>=stealth, font=\footnotesize, line join=round, line cap=round]
			\def\xmin{-3.5} \def\xmax{3}
			\def\ymin{-1.5} \def\ymax{2} 
			\draw[->] (\xmin,0)--(\xmax,0) node [below]{$x$};
			\draw[->] (0,\ymin)--(0,\ymax) node [left]{$y$};
			\node at (0,0) [below left]{$O$};
			\draw[smooth,samples=300] plot[domain=-3:2](\x,{((\x)+3)^.5-1});
			\draw[pattern=north west lines] plot[domain=-2:-3](\x,{((\x)+3)^.5-1})--(-3,0)--cycle
			plot[domain=-2:2](\x,{((\x)+3)^.5-1})--(2,0)--cycle;
			\node[above] at (-2,0) {$c$};
			\node[above] at (-3,0) {$a$};
			\node[below] at (2,0) {$b$};
			\node[left] at (0,1) {$(C)\colon y = f(x)$};
		\end{tikzpicture}
	}
	\choice
	{$S=\displaystyle\int\limits_a^c f(x) \mathrm{\,d}x + 
		\displaystyle\int\limits_c^b f(x) \mathrm{\,d}x$}
	{$S=\displaystyle\int\limits_a^b f(x) \mathrm{\,d}x$}
	{\True  $S=-\displaystyle\int\limits_a^c f(x) \mathrm{\,d}x + \displaystyle\int\limits_c^b f(x) \mathrm{\,d}x$}
	{$ S= \left|\displaystyle\int\limits_a^b f(x) \mathrm{\,d}x\right|$}
	
	\loigiai{
		Ta có $y=f(x)$ liên tục trên đoạn $\left[a; b\right]$.\\
		Dựa vào đồ thị ta có $\left|f(x)\right|=\heva{& -f(x), & a\le x \le c\\& f(x), & c< x \le b.}$\\
		Suy ra $S= \displaystyle\int\limits_a^b \left|f(x)\right| \mathrm{\,d}x = 
		\displaystyle\int\limits_a^c \left|f(x)\right| \mathrm{\,d}x +\displaystyle\int\limits_c^b \left|f(x)\right| \mathrm{\,d}x = -\displaystyle\int\limits_a^c f(x) \mathrm{\,d}x + \displaystyle\int\limits_c^b f(x) \mathrm{\,d}x$.
		
	}
\end{ex}	

%%%%%-------------Câu 20
\begin{ex}%[2D4H3-1]
	Diện tích phần hình phẳng gạch chéo trong hình vẽ bên dưới được tính theo công thức nào dưới đây?
	\begin{center}
		\begin{tikzpicture}[yscale=.8,xscale=.8,>=stealth, font=\footnotesize, line join=round, line cap=round]
			\def\xmin{-2} \def\xmax{3.5}
			\def\ymin{-2} \def\ymax{4} 
			\draw[->] (\xmin,0)--(\xmax,0) node [below]{$x$};
			\draw[->] (0,\ymin)--(0,\ymax) node [left]{$y$};
			\node [right] at (3,2){$y=x^2-2x-1$};
			\node [right] at (2.2,-2){$y=-x^2+3$};
			\clip (-2,-2) rectangle (3,3);
			\draw[smooth,samples=300] plot[domain=-1.4:3](\x,{(\x)^2-2*(\x)-1}) ;
			\draw[smooth,samples=300] plot[domain=-1.4:3](\x,{-(\x)^2+3});
			\draw[pattern=north west lines]plot[domain=-1:2](\x,{-(\x)^2+3})-- plot[domain=-1:2](\x,{(\x)^2-2*(\x)-1});
			\node at (0,0) [below right]{$O$};
			\draw[dashed] (-1,0) node [below]{$-1$}--(-1,2);
			\draw[dashed] (2,0) node [above]{$2$}--(2,-1);
		\end{tikzpicture}
	\end{center}
	
	\choice
	{$\displaystyle\int\limits_{-1}^{2} (-2x +2) \mathrm{\,d}x$}
	{$\displaystyle\int\limits_{-1}^{2} (2x -2) \mathrm{\,d}x$}
	{\True$\displaystyle\int\limits_{-1}^{2} (-2x^2 + 2x + 4) \mathrm{\,d}x$}
	{$\displaystyle\int\limits_{-1}^{2} (2x^2 -2x - 4) \mathrm{\,d}x$}
	
	\loigiai{
		Ta có 
		$S=\displaystyle\int\limits_{-1}^{2} \left|(-x^2 + 3) - (x^2 - 2x -1) \right| \mathrm{\,d}x = 
		\displaystyle\int\limits_{-1}^{2} \left|-2x^2 + 2x +4\right| \mathrm{\,d}x$.\\
		Vì $-2x^2 + 2x +4 > 0 , \forall x \in (-1; 2) $ nên ta có \\
		$S = \displaystyle\int\limits_{-1}^{2} \left|-2x^2 + 2x +4\right| \mathrm{\,d}x = \displaystyle\int\limits_{-1}^{2} (-2x^2 + 2x + 4) \mathrm{\,d}x$.
		
	}
\end{ex}	

%%%%%-------------Câu 21
\begin{ex}%[2D4N3-1]
	Cho hàm số  $y=f(x)$ liên tục trên $\mathbb{R}$. Gọi $S$  là diện tích hình phẳng giới hạn bởi các đường $y=f(x)$, $y=0$, $x= -1$, $x = 5$ (như hình vẽ bên dưới).
	\begin{center}
		\begin{tikzpicture}[scale=.7,>=stealth, font=\footnotesize, line join=round, line cap=round]
			\def\a{1/5} \def\b{-1} \def\c{-1/5} \def\d{1} % Hệ số
			\def\xmin{-2} \def\xmax{7}
			\def\ymin{-4} \def\ymax{2} 
			\draw[->] (\xmin,0)--(\xmax,0) node [below]{$x$};
			\draw[->] (0,\ymin)--(0,\ymax) node [left]{$y$};
			\node at (0,0) [below left]{$O$};
			\draw[smooth,samples=300] plot[domain=-1.7:5.4](\x,{\a*(\x)^3+\b*(\x)^2+\c*(\x)+\d}) node [left]{$y=f(x)$};
			\draw[pattern=north east lines] plot[domain=-1:5](\x,{\a*(\x)^3+\b*(\x)^2+\c*(\x)+\d});
			\node[above left] at (-1,0) {$-1$};
			\node[above right] at (1,0) {$1$};
			\node[below right] at (5,0) {$5$};
		\end{tikzpicture}
	\end{center}
	Mệnh đề nào sau đây đúng?
	\choice
	{$S=-\displaystyle\int\limits_{-1}^{1} f(x) \mathrm{\,d}x - \displaystyle\int\limits_{1}^{5} f(x) \mathrm{\,d}x$}
	{$S=\displaystyle\int\limits_{-1}^{1} f(x) \mathrm{\,d}x + \displaystyle\int\limits_{1}^{5} f(x) \mathrm{\,d}x$}
	{\True  $S=\displaystyle\int\limits_{-1}^{1} f(x) \mathrm{\,d}x - \displaystyle\int\limits_{1}^{5} f(x) \mathrm{\,d}x$}
	{$S=-\displaystyle\int\limits_{-1}^{1} f(x) \mathrm{\,d}x + \displaystyle\int\limits_{1}^{5} f(x) \mathrm{\,d}x$}
	\loigiai{
		Ta có $y=f(x)$ liên tục trên đoạn $\left[-1; 5\right]$.\\
		Dựa vào đồ thị ta có $\left|f(x)\right|=\heva{& f(x), & -1\le x \le 1\\& -f(x), & 1< x \le 5.}$\\
		Suy ra $S= \displaystyle\int\limits_{1}^{5} \left|f(x)\right| \mathrm{\,d}x = 
		\displaystyle\int\limits_{-1}^{1} \left|f(x)\right| \mathrm{\,d}x +\displaystyle\int\limits_{1}^{5} \left|f(x)\right| \mathrm{\,d}x = \displaystyle\int\limits_{-1}^{1} f(x) \mathrm{\,d}x - \displaystyle\int\limits_{1}^{5} f(x) \mathrm{\,d}x$.
		
	}
\end{ex}	
%%%%%-------------Câu 22
\begin{ex}%[2D4N3-1]
	Cho hàm số  $y=f(x)$ liên tục trên $\mathbb{R}$. Gọi $S$  là diện tích hình phẳng giới hạn bởi các đường $y=f(x)$, $y=0$, $x= -1$, $x = 2$ (như hình vẽ bên dưới).
	\begin{center}
		\begin{tikzpicture}[scale=1,>=stealth, font=\footnotesize, line join=round, line cap=round]
			\def\a{1} \def\b{-2} \def\c{-1} \def\d{2} % Hệ số
			\def\xmin{-2} \def\xmax{3}
			\def\ymin{-2} \def\ymax{3} 
			\draw[->] (\xmin,0)--(\xmax,0) node [below]{$x$};
			\draw[->] (0,\ymin)--(0,\ymax) node [left]{$y$};
			\node at (0,0) [below left]{$O$};
			\draw[smooth,samples=300] plot[domain=-1.2:2.5](\x,{\a*(\x)^3+\b*(\x)^2+\c*(\x)+\d}) node [left]{$y=f(x)$};
			\draw[pattern=north east lines] plot[domain=-1:2](\x,{\a*(\x)^3+\b*(\x)^2+\c*(\x)+\d});
			\node[above left] at (-1,0) {$-1$};
			\node[above right] at (1,0) {$1$};
			\node[below right] at (2,0) {$2$};
		\end{tikzpicture}
	\end{center}
	Mệnh đề nào sau đây đúng?
	\choice
	{$S=\displaystyle\int\limits_{-1}^{1} f(x) \mathrm{\,d}x + \displaystyle\int\limits_{1}^{2} f(x) \mathrm{\,d}x$}
	{$S=-\displaystyle\int\limits_{-1}^{1} f(x) \mathrm{\,d}x - \displaystyle\int\limits_{1}^{2} f(x) \mathrm{\,d}x$}
	{$S=-\displaystyle\int\limits_{-1}^{1} f(x) \mathrm{\,d}x + \displaystyle\int\limits_{1}^{2} f(x) \mathrm{\,d}x$}
	{\True  $S=\displaystyle\int\limits_{-1}^{1} f(x) \mathrm{\,d}x - \displaystyle\int\limits_{1}^{2} f(x) \mathrm{\,d}x$}
	\loigiai{
		Ta có $y=f(x)$ liên tục trên đoạn $\left[-1; 2\right]$.\\
		Dựa vào đồ thị ta có $\left|f(x)\right|=\heva{& f(x), & -1\le x \le 1\\& -f(x), & 1< x \le 2.}$\\
		Suy ra $S= \displaystyle\int\limits_{1}^{2} \left|f(x)\right| \mathrm{\,d}x = 
		\displaystyle\int\limits_{-1}^{1} \left|f(x)\right| \mathrm{\,d}x +\displaystyle\int\limits_{1}^{2} \left|f(x)\right| \mathrm{\,d}x = \displaystyle\int\limits_{-1}^{1} f(x) \mathrm{\,d}x - \displaystyle\int\limits_{1}^{2} f(x) \mathrm{\,d}x$.
		
	}
\end{ex}	
%%%%%-------------Câu 23
\begin{ex}%[2D4N3-1]
	\immini{
		Gọi $S$ là diện tích hình phẳng $(H)$ giới hạn bởi các đường $y=f(x)$, trục hoành và hai đường thẳng  $x=-1$, $x=2$. Đặt $a=\displaystyle\int\limits_{-1}^{0} f(x) \mathrm{\,d}x$, $b=\displaystyle\int\limits_{0}^{2} f(x) \mathrm{\,d}x$ (như hình vẽ bên). Mệnh đề nào sau đây đúng?  
		\choice
		{\True $S=b-a$}
		{$S=b+a$}
		{$S=-b+a$}
		{$S=-b-a$}
	}{
		\begin{tikzpicture}[yscale=.7,xscale=.8,>=stealth, font=\footnotesize, line join=round, line cap=round]
			\def\a{0.37} \def\b{0} \def\c{0.33} \def\d{0} % Hệ số
			\def\xmin{-2} \def\xmax{3}
			\def\ymin{-2.5} \def\ymax{4} 
			\draw[->] (\xmin,0)--(\xmax,0) node [below]{$x$};
			\draw[->] (0,\ymin)--(0,\ymax) node [left]{$y$};
			\node at (0,0) [below right]{$O$};
			\draw[smooth,samples=300] plot[domain=-1.5:2.1](\x,{\a*(\x)^3+\b*(\x)^2+\c*(\x)+\d});
			\draw[pattern=north east lines] plot[domain=0:-1](\x,{\a*(\x)^3+\b*(\x)^2+\c*(\x)+\d})--(-1,0)--cycle
			plot[domain=0:2](\x,{\a*(\x)^3+\b*(\x)^2+\c*(\x)+\d})--(2,0)--cycle;
			\node[above] at (-1,0) {$-1$};
			\node[below] at (2,0) {$2$};
		\end{tikzpicture}
	}
	\loigiai{
		Ta có $y=f(x)$ liên tục trên đoạn $\left[-1; 2\right]$.\\
		Dựa vào đồ thị ta có $\left|f(x)\right|=\heva{& -f(x), & -1\le x \le 0\\& f(x), & 0< x \le 2.}$\\
		Suy ra $S= \displaystyle\int\limits_{-1}^{2} \left|f(x)\right| \mathrm{\,d}x = 
		\displaystyle\int\limits_{-1}^{0} \left|f(x)\right| \mathrm{\,d}x +\displaystyle\int\limits_{0}^{2} \left|f(x)\right| \mathrm{\,d}x = -\displaystyle\int\limits_{-1}^{0} f(x) \mathrm{\,d}x + \displaystyle\int\limits_{0}^{2} f(x) \mathrm{\,d}x$.\\
		Hay $S=-a + b = b - a$.
		
	}
\end{ex}	
%%%%%-------------Câu 24
\begin{ex}%[2D4N3-1]
	\immini{
		Gọi $S$ là diện tích hình phẳng $(H)$ giới hạn bởi các đường $y=f(x)$, trục hoành và hai đường thẳng  $x=-3$, $x=2$. Đặt $a=\displaystyle\int\limits_{-3}^{1} f(x) \mathrm{\,d}x$, $b=\displaystyle\int\limits_{1}^{2} f(x) \mathrm{\,d}x$ (như hình vẽ bên). Mệnh đề nào sau đây đúng?  
	}{
		\begin{tikzpicture}[yscale=.7,xscale=.7,>=stealth, font=\footnotesize, line join=round, line cap=round]
			\def\a{-0.05} \def\b{-0.08} \def\c{1.07} \def\d{-.94} % Hệ số
			\def\xmin{-4} \def\xmax{3}
			\def\ymin{-4} \def\ymax{1} 
			\draw[->] (\xmin,0)--(\xmax,0) node [below]{$x$};
			\draw[->] (0,\ymin)--(0,\ymax) node [left]{$y$};
			\node at (0,0) [above left]{$O$};
			\draw[smooth,samples=300] plot[domain=-3:2](\x,{\a*(\x)^3+\b*(\x)^2+\c*(\x)+\d});
			\draw[pattern=north west lines] (-3,0)-- (-3,-3.5)-- plot[domain=-3:2](\x,{\a*(\x)^3+\b*(\x)^2+\c*(\x)+\d})-- (2,.5)--(2,0);
			\node[above] at (-3,0) {$-3$};
			\node[below] at (2,0) {$2$};
			\node[below] at (1,0) {$1$};
		\end{tikzpicture}
	}
	\choice
	{$S=a+b$}
	{$S=a-b$}
	{$S=-a-b$}
	{\True $S=b-a$}
	\loigiai{
		Ta có $y=f(x)$ liên tục trên đoạn $\left[-3; 2\right]$.\\
		Dựa vào đồ thị ta có $\left|f(x)\right|=\heva{& -f(x), & -3\le x \le 1\\& f(x), & 1< x \le 2.}$\\
		Suy ra $S= \displaystyle\int\limits_{-3}^{2} \left|f(x)\right| \mathrm{\,d}x = 
		\displaystyle\int\limits_{-3}^{1} \left|f(x)\right| \mathrm{\,d}x +\displaystyle\int\limits_{1}^{2} \left|f(x)\right| \mathrm{\,d}x = -\displaystyle\int\limits_{-3}^{1} f(x) \mathrm{\,d}x + \displaystyle\int\limits_{1}^{2} f(x) \mathrm{\,d}x$.\\
		Hay $S=-a + b = b - a$.
		
	}
\end{ex}	
%%%%%-------------Câu 25
% \begin{ex}%[2D4V3-1]
% 	\immini{
% 		Cho các số $p$, $q$  thỏa mãn các điều kiện $p>0$, $q>1$, $\dfrac{1}{p}+\dfrac{1}{q} = 1$ và các số dương $a, b$. Xét hàm số $y=x^{p-1}$ $(x>0)$ có đồ thị $(C)$. Gọi $S_1$  là diện tích hình phẳng giới hạn bởi  $(C)$, trục hoành, đường thẳng  $x=a$. Gọi $S_2$  là diện tích hình phẳng giới hạn bởi  $(C)$, trục tung, đường thẳng  $y=b$. Gọi $S$ là diện tích hình phẳng giới hạn bởi trục hoành, trục tung và hai đường thẳng  $x=a$,  $y=b$ (như hình vẽ bên). Khi so sánh  $S_1 + S_2$ và  $S$ ta nhận được bất đẳng thức nào trong các bất đẳng thức dưới đây?
% 	}{
% 		\begin{tikzpicture}[yscale=.7,xscale=.7,>=stealth, font=\footnotesize, line join=round, line cap=round]
% 			\def\a{1/8} \def\b{1} \def\c{0} \def\d{0} % Hệ số
% 			\def\xmin{-1} \def\xmax{4}
% 			\def\ymin{-1} \def\ymax{6} 
% 			\draw[->] (\xmin,0)--(\xmax,0) node [below]{$x$};
% 			\draw[->] (0,\ymin)--(0,\ymax) node [left]{$y$};
% 			\node at (0,0) [below left]{$O$};
% 			\draw[smooth,samples=300] plot[domain=0:2.2](\x,{\a*(\x)^3+\b*(\x)^2+\c*(\x)+\d}) node[above right]{$y=x^{p-1}$};
% 			\fill[pattern=north west lines] plot[domain=0:2](\x,{\a*(\x)^3+\b*(\x)^2+\c*(\x)+\d})--(2,5)--(2,0)--cycle;
% 			\fill[pattern=north east lines](0,4)-- plot[domain=0:1.802](\x,{\a*(\x)^3+\b*(\x)^2+\c*(\x)+\d})--cycle;
% 			\draw (-1,4)--(4,4) node[pos=.8,sloped,above]{$y=b$};
% 			\draw (2,-1)--(2,6)node[pos=.5,sloped,below]{$x=a$};
% 			\node[circle] at (.8,3){$S_2$};
% 			\node[circle] at (1.5,.5){$S_1$};
% 			\node[above left] at (0,4) {$b$};
% 			\node[below right] at (2,0) {$a$};
% 		\end{tikzpicture}
% 	}
% 	\choice
% 	{$\dfrac{a^p}{p}+\dfrac{b^q}{q}\le ab$}
% 	{$\dfrac{a^{p-1}}{p-1}+\dfrac{b^{q-1}}{q-1}\le ab$}
% 	{$\dfrac{a^{p+1}}{p+1}+\dfrac{b^{q+1}}{q+1}\le ab$}
% 	{\True $\dfrac{a^p}{p}+\dfrac{b^q}{q}\ge ab$}
% 	\loigiai{
% 		\begin{itemize}
% 			\item Diện tích hình phẳng giới hạn bởi trục hoành, trục tung và hai đường thẳng  $x=a$,  $y=b$ là $S = ab$.
% 			\item $S_1 = \displaystyle\int\limits_0^a x^{p-1} \mathrm{\,d}x=
% 			\left.\dfrac{x^p}{p}\right|_0^a = \dfrac{a^p}{p}$.
% 			\item Ta có $\dfrac{1}{p}+\dfrac{1}{q} = 1 \Leftrightarrow \dfrac{1}{q} = 1 - \dfrac{1}{p} = \dfrac{p - 1}{p} \Leftrightarrow q= \dfrac{p}{p-1}$. Tương tự $p=\dfrac{q}{q-1}$.\\
% 			Phương trình hoành độ giao điểm $ x^{p-1}=b\Leftrightarrow x= b^{\tfrac{1}{p-1}} \in (0;2)$. Suy ra\\
% 			$S_2 = \displaystyle\int\limits_0^{b^{\frac{1}{p-1}}} \left(b-x^{p-1}\right)\mathrm{\,d}x=
% 			\left.\left(bx -\dfrac{x^p}{p}\right)\right|_0^{b^{\frac{1}{p-1}}} $\\
% 			$= b\cdot b^{\frac{1}{p-1}}-
% 			\dfrac{\left( b^{\frac{1}{p-1}}\right)^p}{p}= b^{\frac{p}{p-1}}-
% 			\dfrac{b^{\frac{p}{p-1}}}{\dfrac{q}{q-1}} = b^q - \dfrac{ b^q(q-1)}{q} = \dfrac{b^q}{q}$.
% 			\item Dựa và hình vẽ đồ thị  ta có $S_1 + S_2 \ge S $.
% 			Vậy $\dfrac{a^p}{p}+\dfrac{b^q}{q}\ge ab $.
% 		\end{itemize}
% 	}
% \end{ex}	
%%%%%-------------Câu 26
\begin{ex}%[2D4N3-1]
	Diện tích phần hình phẳng được gạch sọc trong hình vẽ sau được tính theo công thức nào dưới đây?
	
	\begin{center}
		\begin{tikzpicture}[scale=1,>=stealth, font=\footnotesize, line join=round, line cap=round]
			\def\a{0} \def\b{1} \def\c{0} \def\d{-2} % Hệ số
			\def\xmin{-4} \def\xmax{4}
			\def\ymin{-3} \def\ymax{3} 
			\draw[->] (\xmin,0)--(\xmax,0) node [below]{$x$};
			\draw[->] (0,\ymin)--(0,\ymax) node [left]{$y$};
			\node at (0,0) [below left]{$O$};
			\draw[smooth,samples=300] plot[domain=-2:2](\x,{\a*(\x)^3+\b*(\x)^2+\c*(\x)+\d}) node [right]{$y=x^2 -2$};
			\draw[smooth,samples=300] plot[domain=0:3](\x,{-(\x)^.5})node [below]{$y=-\sqrt{|x|}$};
			\draw[smooth,samples=300] plot[domain=-3:0](\x,{-(-\x)^.5});
			\draw[pattern=north east lines] (-1,-1)--  plot[domain=-1:0](\x,{-(-\x)^.5})--(0,0)--plot[domain=0:-1](\x,{\a*(\x)^3+\b*(\x)^2+\c*(\x)+\d}) --cycle;
			\draw[pattern=north east lines] (0,0)--  plot[domain=0:1](\x,{-(\x)^.5})--(1,-1)--plot[domain=1:0](\x,{\a*(\x)^3+\b*(\x)^2+\c*(\x)+\d}) --cycle;
			\foreach \x in {-3,-2,-1,1,2,3} \draw[fill] (\x,0) circle (1pt) node [above] { $\x$};
			\foreach \y in {-2,1,2} \draw[fill] (0,\y) circle (1pt) node [ below left] { $\y$};
			\draw[dashed] (-1,0)--(-1,-1) (1,0)--(1,-1);
		\end{tikzpicture}
	\end{center}
	\choice
	{$\displaystyle\int\limits_{-1}^{1} \left( x^2 -2 + \sqrt{|x|}\right)\mathrm{\,d}x$}
	{$\displaystyle\int\limits_{-1}^{1} \left( x^2 -2 - \sqrt{|x|}\right)\mathrm{\,d}x$}
	{$\displaystyle\int\limits_{-1}^{1} \left( -x^2 + 2 + \sqrt{|x|}\right)\mathrm{\,d}x$}
	{\True $\displaystyle\int\limits_{-1}^{1} \left( -x^2 + 2 - \sqrt{|x|}\right)\mathrm{\,d}x$}
	\loigiai{
		Ta có $ -\sqrt{|x|}\ge x^2 -2$, $\forall x\in [-1; 1]$.\\
		Do đó $-\sqrt{|x|}- (x^2 -2) = -x^2  +2 -\sqrt{|x|} \ge 0, \forall x\in [-1; 1] $.\\
		Diện tích phần hình phẳng được gạch sọc trong hình vẽ là\\
		$\displaystyle\int\limits_{-1}^{1} \left|-\sqrt{|x|}- (x^2 -2)\right|\mathrm{\,d}x = \displaystyle\int\limits_{-1}^{1} \left( -x^2 + 2 - \sqrt{|x|}\right)\mathrm{\,d}x$
		
		
	}
\end{ex}

\Closesolutionfile{ans}
% \indapan{6}{ans/ans-2-C4B3CD1_10-19}


%\TNTF
\Opensolutionfile{ans}[ans/ans-2-C4B3CD1_10-19-DS]
\begin{ex}%[2D4H3-1]
	Cho hàm số  $y=f(x)$ liên tục trên  $\mathbb{R}$. Gọi $S$  là diện tích hình phẳng giới hạn bởi các đường  $y=f(x)$, $y=0$, $x=-1$, $x=4$ (như hình vẽ). Các mệnh đề sau đây đúng hay sai?
	\begin{center}
		\begin{tikzpicture}[scale=1,>=stealth, font=\footnotesize, line join=round, line cap=round]
			\def\a{1/3} \def\b{-4/3} \def\c{-1/3} \def\d{4/3} % Hệ số
			\def\xmin{-2} \def\xmax{5}
			\def\ymin{-3} \def\ymax{2.5} 
			\draw[->] (\xmin,0)--(\xmax,0) node [below]{$x$};
			\draw[->] (0,\ymin)--(0,\ymax) node [left]{$y$};
			\node at (0,0) [below left]{$O$};
			\draw[smooth,samples=300] plot[domain=-1.5:4.2](\x,{\a*(\x)^3+\b*(\x)^2+\c*(\x)+\d}) node [right]{$y=f(x)$};
			\draw[pattern=north east lines] plot[domain=-1:4](\x,{\a*(\x)^3+\b*(\x)^2+\c*(\x)+\d});
			\node[above left] at (-1,0) {$-1$};
			\node[above right] at (1,0) {$1$};
			\node[below right] at (4,0) {$4$};
		\end{tikzpicture}
	\end{center}
	\choiceTF
	{\True $S= \displaystyle\int\limits_{-1}^{1} f(x) \mathrm{\,d}x - \displaystyle\int\limits_{1}^{4} f(x) \mathrm{\,d}x$}
	{\True $S= \displaystyle\int\limits_{-1}^{1} \left|f(x)\right| \mathrm{\,d}x +\displaystyle\int\limits_{1}^{4} \left|f(x)\right| \mathrm{\,d}x$}
	{$S= \left|\displaystyle\int\limits_{-1}^{4} f(x)\mathrm{\,d}x\right|$}
	{$S= \displaystyle\int\limits_{-1}^{1} f(x) \mathrm{\,d}x + \displaystyle\int\limits_{1}^{4} f(x) \mathrm{\,d}x$}
	\loigiai{
		Ta có $y=f(x)$ liên tục trên đoạn $\left[-1; 4\right]$.\\
		Dựa vào đồ thị ta có $\left|f(x)\right|=\heva{& f(x), & -1\le x \le 1\\& -f(x), & 1< x \le 4.}$\\
		Suy ra
		$S= \displaystyle\int\limits_{-1}^{4} \left|f(x)\right| \mathrm{\,d}x = 
		\displaystyle\int\limits_{-1}^{1} \left|f(x)\right| \mathrm{\,d}x + \displaystyle\int\limits_{1}^{4} \left|f(x)\right| \mathrm{\,d}x = \displaystyle\int\limits_{-1}^{1} f(x) \mathrm{\,d}x - \displaystyle\int\limits_{1}^{4} f(x) \mathrm{\,d}x$.\\
		Do đó suy ra
		\begin{itemchoice}
			\itemch {\bf Đúng.}
			Vì $S= \displaystyle\int\limits_{-1}^{1} f(x) \mathrm{\,d}x - \displaystyle\int\limits_{1}^{4} f(x) \mathrm{\,d}x$ đúng.
			\itemch {\bf Đúng.}
			Vì $S= \displaystyle\int\limits_{-1}^{1} \left|f(x)\right| \mathrm{\,d}x +\displaystyle\int\limits_{1}^{4} \left|f(x)\right| \mathrm{\,d}x$ đúng.
			\itemch {\bf Sai.} 
			Vì $\left|f(x)\right|=\heva{& f(x), & -1\le x \le 1\\& -f(x), & 1< x \le 4.}$ nên $\left|\displaystyle\int\limits_{-1}^{4} f(x)\mathrm{\,d}x\right| \ne \displaystyle\int\limits_{-1}^{4} \left|f(x)\right| \mathrm{\,d}x$.
			\itemch {\bf Sai.} 
			Vì $S= \displaystyle\int\limits_{-1}^{1} f(x) \mathrm{\,d}x -\displaystyle\int\limits_{1}^{4} f(x) \mathrm{\,d}x$ sai.
		\end{itemchoice}
	}
\end{ex}

\begin{ex}%[2D4H3-1]
	Cho hình phẳng được gạch chéo trong hình bên dưới.
	\begin{center}
		\begin{tikzpicture}[yscale=.8,xscale=.8,>=stealth, font=\footnotesize, line join=round, line cap=round]
			\def\xmin{-2} \def\xmax{3.5}
			\def\ymin{-3} \def\ymax{3} 
			\draw[->] (\xmin,0)--(\xmax,0) node [below]{$x$};
			\draw[->] (0,\ymin)--(0,\ymax) node [left]{$y$};
			\node [right] at (3,1){$y=x^2-2x-2$};
			\node [right] at (2.2,-3){$y=-x^2+2$};
			\clip (-2,-3) rectangle (3,3);
			\draw[smooth,samples=300] plot[domain=-1.4:3](\x,{(\x)^2-2*(\x)-2}) ;
			\draw[smooth,samples=300] plot[domain=-1.4:3](\x,{-(\x)^2+2});
			\fill[pattern=north west lines]plot[domain=-1:2](\x,{-(\x)^2+2})-- plot[domain=-1:2](\x,{(\x)^2-2*(\x)-2});
			\node at (0,0) [below right]{$O$};
			\draw[dashed] (-1,0) node [below]{$-1$}--(-1,1);
			\draw[dashed] (2,0) node [above]{$2$}--(2,-2);
		\end{tikzpicture}
	\end{center}
	Các mệnh đề sau đây đúng hay sai?
	\choiceTF
	{\True Hình phẳng được gạch chéo trong hình trên được giới hạn các đồ thị $y=x^2-2x-2$, $y=-x^2+2$ và hai đường thẳng $x=-1$, $x=2$}
	{Diện tích hình phẳng gạch chéo trong hình vẽ là\\
		$S= \displaystyle\int\limits_{-1}^{2} \left|x^2 -2x -2\right|\mathrm{\,d}x+\displaystyle\int\limits_{-1}^{2} \left|-x^2 + 2\right|\mathrm{\,d}x$}
	{\True Hình phẳng được gạch chéo trong hình trên được giới hạn các đồ thị $y=x^2-2x-2$ và  $y=-x^2+2$}
	{\True Diện tích hình phẳng gạch chéo trong hình vẽ là $S=9$}
	\loigiai{
		\begin{itemchoice}
			\itemch {\bf Đúng.}
			Hình phẳng được gạch chéo trong hình trên được giới hạn các đồ thị $y=x^2-2x-2$, $y=-x^2+2$ và hai đường thẳng $x=-1$, $x=2$.
			\itemch {\bf Sai.}\\
			Vì $\displaystyle\int\limits_{-1}^{2} \left|x^2 -2x -2\right|\mathrm{\,d}x+
			\displaystyle\int\limits_{-1}^{2} \left|-x^2 + 2\right|\mathrm{\,d}x\ge
			\displaystyle\int\limits_{-1}^{2} \left|(x^2 -2x -2)- (-x^2 + 2) \right|\mathrm{\,d}x=S$
			\itemch {\bf Đúng.} 
			Phương trình hoành độ giao điểm\\
			$ x^2 -2x -2 = -x^2 +2 \Leftrightarrow
			2x^2 -2x - 4 = 0 \Leftrightarrow $ $x= -1$ hoặc $x=2$.\\
			Suy ra $S= \displaystyle\int\limits_{-1}^{2} \left|2x^2 -2x - 4\right|\mathrm{\,d}x$.
			\itemch {\bf Đúng}. 
			Vì $2x^2 -2x - 4<0, \forall x\in (-1;2)$.\\
			$S= \displaystyle\int\limits_{-1}^{2} \left|2x^2 -2x - 4\right|\mathrm{\,d}x=\displaystyle\int\limits_{-1}^{2} (-2x^2 + 2x + 4) \mathrm{\,d}x=\left.\left(\dfrac{2x^3}{3}+x^2+4x\right)\right|_{-1}^{2}= 9$.
		\end{itemchoice}
	}
\end{ex}


\begin{ex}%[2D4H3-1]
	Cho hình phẳng được gạch chéo trong hình bên dưới.
	\begin{center}
		\begin{tikzpicture}[yscale=.8,xscale=.8,>=stealth, font=\footnotesize, line join=round, line cap=round]
			\def\xmin{-3} \def\xmax{3}
			\def\ymin{-1} \def\ymax{5} 
			\node at (0,0) [below left]{$O$};
			\draw[->] (\xmin,0)--(\xmax,0) node [below]{$x$};
			\draw[->] (0,\ymin)--(0,\ymax) node [left]{$y$};
			\node [left] at (-2,4){$y=x^2$};
			\draw[smooth,samples=300] plot[domain=-2.2:2.2](\x,{(\x)^2}) ;
			\draw (1,-1)--(1,5) node[sloped,pos=.6,above]{$x=1$} 
			(2,-1)--(2,5) node[sloped,pos=.6,below]{$x=2$};
			\fill[pattern=north west lines](1,0)--(1,1)-- plot[domain=1:2](\x,{(\x)^2})--(2,4)--(2,0)--cycle;
			\foreach \x in {-1,-2,1,2} \draw[fill] (\x,0) circle (1pt) node [below left] { $\x$};
			\foreach \y in {1,2,3,4} \draw[fill] (0,\y) circle (1pt) node [left] { $\y$};
		\end{tikzpicture}
	\end{center}
	Các mệnh đề sau đây đúng hay sai?
	\choiceTF
	{\True Hình phẳng được gạch chéo trong hình trên được giới hạn các đồ thị $y=x^2$, $y=0$ và hai đường thẳng $x=1$, $x=2$}
	{\True  Diện tích hình phẳng gạch chéo trong hình vẽ là $S= \displaystyle\int\limits_{1}^{2} x^2 \mathrm{\,d}x$}
	{Diện tích hình phẳng gạch chéo trong hình vẽ là $S=\dfrac{4}{3}$}
	{Hình phẳng được gạch chéo trong hình trên được giới hạn đồ thị $y=x^2$ và hai đường thẳng $x=1$, $x=2$}
	\loigiai{
		\begin{itemchoice}
			\itemch {\bf Đúng}.
			Hình phẳng được gạch chéo trong hình trên được giới hạn các đồ thị $y=x^2$, $y=0$ và hai đường thẳng $x=1$, $x=2$.
			\itemch {\bf Đúng}.
			Vì $S= \displaystyle\int\limits_{1}^{2} \left|x^2\right| \mathrm{\,d}x = \displaystyle\int\limits_{1}^{2} x^2 \mathrm{\,d}x$.
			\itemch {\bf Sai}. 
			Vì $S= \displaystyle\int\limits_{1}^{2} x^2 \mathrm{\,d}x = \left.\dfrac{x^3}{3}\right|_1^2 = \dfrac{8}{3}-\dfrac{1}{3}=\dfrac{7}{3}$.
			\itemch {\bf Sai}. 
			Vì hình phẳng được giới hạn đồ thị $y=x^2$ và hai đường thẳng $x=1$, $x=2$ không xác định được diện tích.
		\end{itemchoice}
	}
\end{ex}



\begin{ex}%[2D4H3-1]
	Cho hình phẳng được gạch chéo trong hình bên dưới.
	\begin{center}
		\begin{tikzpicture}[yscale=.8,xscale=.8,>=stealth, font=\footnotesize, line join=round, line cap=round]
			\def\xmin{-1} \def\xmax{6}
			\def\ymin{-1} \def\ymax{6.5} 
			\draw[->] (\xmin,0)--(\xmax,0) node [below]{$x$};
			\draw[->] (0,\ymin)--(0,\ymax) node [left]{$y$};
			\node at (0,0) [below right]{$O$};
			\draw[smooth,samples=300] plot[domain=-.2:5.2](\x,{-(\x)^2+5*(\x)}) node[left]{$y=5x-x^2$};
			\draw[smooth,samples=300] plot[domain=-1:5.2](\x,{(\x)})node[below right]{$y=x$} ;
			\fill[pattern=north west lines] plot[domain=0:4](\x,{-(\x)^2+5*(\x)});
			\foreach \x/\y in {4/4} \draw[fill] (\x,\y) circle (1pt) node [right] { $(\x,\y)$};
		\end{tikzpicture}
	\end{center}
	Các mệnh đề sau đây đúng hay sai?
	\choiceTF
	{\True Hình phẳng được gạch chéo trong hình trên được giới hạn các đồ thị $y=5x-x^2$, $y=x$ và các đường thẳng $x=0$, $x=4$}
	{Diện tích hình phẳng gạch chéo trong hình vẽ là $S= \displaystyle\int\limits_{0}^{4} \left(x^2 - 4x\right) \mathrm{\,d}x$}
	{\True Diện tích hình phẳng gạch chéo trong hình vẽ là $S= \displaystyle\int\limits_{0}^{4} \left|x^2 - 4x \right| \mathrm{\,d}x $}
	{Diện tích hình phẳng gạch chéo trong hình vẽ $S= \dfrac{56}{3}$}
	\loigiai{
		\begin{itemchoice}
			\itemch {\bf Đúng.}
			Hình phẳng được gạch chéo trong hình trên được giới hạn các đồ thị $y=5x-x^2$, $y=x$ và các đường thẳng $x=0$, $x=4$.
			\itemch {\bf Sai}
			Phương trình hoành độ giao điểm\\
			$ x  = 5x -x^2 \Leftrightarrow
			x^2 -4x = 0 \Leftrightarrow $ $x= 0$ hoặc $x=4$.\\
			Vì $x^2-4x<0, \forall x\in (0;4)$.
			Do đó $S= \displaystyle\int\limits_{0}^{4} \left|x^2 - 4x \right| \mathrm{\,d}x = \displaystyle\int\limits_{0}^{4} \left(- x^2 + 4x \right) \mathrm{\,d}x$.
			\itemch {\bf Đúng}. 
			Vì $S= \displaystyle\int\limits_{0}^{4} \left|x^2 - 4x \right| \mathrm{\,d}x$.
			\itemch {\bf Sai}. 
			Vì $S= \displaystyle\int\limits_{0}^{4} \left|x^2 - 4x \right| \mathrm{\,d}x= \displaystyle\int\limits_{0}^{4} \left(- x^2 + 4x \right) \mathrm{\,d}x= \left.\left(-\dfrac{x^3}{3}+ 2x^2\right)\right|_{0}^{4}=\dfrac{32}{3}$.
		\end{itemchoice}
	}
\end{ex}


\begin{ex}%[2D4H3-1]
	Cho hình phẳng được gạch chéo trong hình bên dưới.
	\begin{center}
		\begin{tikzpicture}[yscale=1,xscale=1,>=stealth, font=\footnotesize, line join=round, line cap=round]
			\def\xmin{-1} \def\xmax{4}
			\def\ymin{-1} \def\ymax{3.5} 
			\draw[->] (\xmin,0)--(\xmax,0) node [below]{$x$};
			\draw[->] (0,\ymin)--(0,\ymax) node [left]{$y$};
			\node at (0,0) [below left]{$O$};
			\draw[smooth,samples=300] plot[domain=0.5:3](\x,{1+1/(\x)});
			\draw[pattern=north east lines](1,0)--(1,2)-- plot[domain=1:2](\x,{1+1/(\x)})--(2,1.5)--(2,0)--cycle;
			\foreach \x in {1,2} \draw[fill] (\x,0) circle (1pt) node [below] { $\x$};
			\foreach \y in {1,2} \draw[fill] (0,\y) circle (1pt) node [left] { $\y$};
			\node[above right] at (1,2) {$y=1+\dfrac{1}{x}$};
		\end{tikzpicture}
	\end{center}
	Các mệnh đề sau đây đúng hay sai?
	\choiceTF
	{Hình phẳng được gạch chéo trong hình trên được giới hạn đồ thị $y=1 + \dfrac{1}{x}$ và các đường thẳng $x=1$, $x=2$}
	{\True  Diện tích hình phẳng gạch chéo trong hình vẽ là $S= \displaystyle\int\limits_{1}^{2} \left(1 + \dfrac{1}{x}\right) \mathrm{\,d}x$}
	{Diện tích hình phẳng gạch chéo trong hình vẽ là $S=2$}
	{\True Diện tích hình phẳng gạch chéo trong hình vẽ là $S= 1+ \displaystyle\int\limits_{1}^{2} \dfrac{1}{x} \mathrm{\,d}x$}
	\loigiai{
		\begin{itemchoice}
			\itemch {\bf Sai.}
			Hình phẳng  giới hạn đồ thị
			$y=1 + \dfrac{1}{x}$ và các đường thẳng $x=1$, $x=2$ không xác định được diện tích.
			\itemch {\bf Đúng.}
			Vì $1+\dfrac{1}{x}>0, \forall x\in (1;2)$ nên $\left|1+\dfrac{1}{x}\right| = 1+\dfrac{1}{x}, \forall x\in (1;2)$.\\
			Do đó $S= \displaystyle\int\limits_{1}^{2} \left|1+\dfrac{1}{x}\right| \mathrm{\,d}x= \displaystyle\int\limits_{1}^{2} \left(1 + \dfrac{1}{x}\right) \mathrm{\,d}x$.
			\itemch {\bf Sai.}
			Vì $S= \displaystyle\int\limits_{1}^{2} \left(1 + \dfrac{1}{x}\right) \mathrm{\,d}x =\left(x+ \ln |x|\right)\Big|_{1}^{2} = 1+ \ln 2$.
			\itemch {\bf Đúng.} 
			Vì $ S=  \displaystyle\int\limits_{1}^{2} \left(1 + \dfrac{1}{x}\right) \mathrm{\,d}x = 1+ \displaystyle\int\limits_{1}^{2} \dfrac{1}{x} \mathrm{\,d}x = 1+ \ln 2$.
		\end{itemchoice}
	}
\end{ex}
%câu 33
\begin{ex}%[2D4N3-1]
	Cho hình phẳng được tô màu trong hình bên dưới
	\begin{center}
		\begin{tikzpicture}[font=\footnotesize, line join=round, line cap=round, >=stealth, scale = 0.8]
			\draw[->] (-1.5,0) --(0,0) node[below right]{$O$}--(1.5,0) node[below]{$x$};
			\draw[->] (0,-.7) --(0,3.2) node[right]{$y$};
			\draw[fill = black] (1,0) node[below]{$1$} circle (1pt);
			\draw[fill = black] (-1,0) node[below left]{$-1$} circle (1pt);
			\draw[fill = black] (0,1) node[left]{$1$} circle (1pt);
			\draw[fill = black] (0,0) circle (1pt);
			\draw[dashed](1,0)--(1,2.71) (-1,0)--(-1,0.37);
			%	\clip (-1.3,-.4) rectangle (1.5,4);
			\draw [samples=100, domain=-1.2:1.1] plot (\x, {e^(\x)});
			\fill[pattern = north west lines] (-1,0) -- plot[smooth,samples=100,domain=-1:1] (\x, {e^(\x)}) -- (1,0) -- cycle;
			\draw (1.3,3)node[above]{\scriptsize $ y=e^x $};
		\end{tikzpicture}
	\end{center}
	Các mệnh đề sau đây đúng hay sai?
	\choiceTF
	{Hình phẳng được tô màu trong hình vẽ trên được giới hạn bởi các đồ thị $ y=\mathrm{e}^x $; $ y=0 $; $ x=0 $; $ x=1 $}
	{\True Diện tích hình phẳng tô màu trong hình vẽ là $ \displaystyle\int\limits_{-1}^1 \mathrm{e}^x \mathrm{\,d}x$}
	{Diện tích hình phẳng tô màu trong hình vẽ là $ \displaystyle\int\limits_0^1 \mathrm{e}^x \mathrm{\,d}x$}
	{\True Hình phẳng được tô màu trong hình vẽ trên được giới hạn bởi các đồ thị $ y=\mathrm{e}^x $; $ y=0 $; $ x=-1 $; $ x=1 $}
	\loigiai{
		\begin{itemchoice}
			\itemch Sai. Vì hình phẳng được tô màu trong hình vẽ trên được giới hạn bởi các đồ thị $ y=\mathrm{e}^x $; $ y=0 $; $ x=-1 $; $ x=1 $.
			\itemch Đúng. Ta có $ S=\displaystyle\int\limits_{-1}^1 \mathrm{e}^x \mathrm{d}x $.
			\itemch Sai.
			\itemch Đúng.
	\end{itemchoice}}
\end{ex}

%CÂU 34
\begin{ex}%[2D4N3-1]
	Cho hình phẳng được tô màu trong hình bên dưới.
	\begin{center}
		\begin{tikzpicture}[font=\footnotesize, line join=round, line cap=round, >=stealth, scale = 0.8]
			\draw[->] (-.5,0) --(0,0) node[below right]{$O$}--(3.3,0) node[below]{$x$};
			\draw[->] (0,-.7) --(0,4) node[right]{$y$};
			\draw[fill = black] (2,0) node[below left]{$2$} circle (1pt);
			\draw[fill = black] (0,1) node[left]{$1$} circle (1pt);
			\draw[fill = black] (0,0) circle (1pt);
			\draw[dashed](2,0)--(2,3);
			\fill[pattern=north west lines] plot[domain=2:0](\x,{0.5^(\x)})-- plot[domain=0:2](\x,{(\x)+1}) --cycle;
			\draw [samples=100, domain=-.2:2.6] plot (\x, {(\x)+1});
			\draw [samples=100, domain=-.2:2.8] plot (\x, {0.5^(\x)});
			\draw (1.5,2.6)node[above,rotate=45]{\scriptsize $ y=x+1 $};
			\draw (3.3,.02)node[above]{\scriptsize $ y=\left(\dfrac{1}{2}\right)^x $};
		\end{tikzpicture}
	\end{center}
	Các mệnh đề sau đúng hay sai?
	\choiceTF
	{\True Hình phẳng được tô màu trong hình vẽ trên được giới hạn bởi các đồ thị $ y=x+1 $; $ y=\left(\dfrac{1}{2}\right)^x $; $ x=0 $; $ x=2 $}
	{Diện tích hình phẳng tô màu trong hình vẽ là $ \displaystyle\int\limits_0^2\left[\left(\dfrac{1}{2}\right)^x-x-1\right]\mathrm{\,d}x$}
	{\True Diện tích hình phẳng tô màu trong hình vẽ bằng $ S=4-\dfrac{3}{4\ln 2}$}
	{Hình phẳng được tô màu trong hình vẽ trên được giới hạn bởi các đồ thị $ y=x+1 $; $ y=\left(\dfrac{1}{2}\right)^x $; $ x=1 $; $ x=2 $}
	\loigiai{
		\begin{itemchoice}
			\itemch Đúng. Vì hình phẳng được tô màu trong hình vẽ trên được giới hạn bởi các đồ thị $ y=x+1 $; $ y=\left(\dfrac{1}{2}\right)^x $; $ x=0 $; $ x=2 $.
			\itemch Sai. Trên đoạn $ \left[0;2\right] $, đồ thị hàm số $ y=x+1 $ nằm trên đồ thị hàm số $ y=\left(\dfrac{1}{2}\right)^x $ nên với mọi $ x \in \left[0;2\right]$ ta có $ x+1 \ge \left(\dfrac{1}{2}\right)^x  \Rightarrow \left|\left(\dfrac{1}{2}\right)^x-(x+1)\right|=x+1-\left(\dfrac{1}{2}\right)^x$.\\
			Vậy diện tích hình phẳng tô màu là $ \displaystyle\int\limits_0^2\left[ x+1-\left(\dfrac{1}{2}\right)^x\right]\mathrm{\,d}x$.
			\itemch Đúng. Ta có $ S=\displaystyle\int\limits_0^2\left[ x+1-\left(\dfrac{1}{2}\right)^x\right]\mathrm{\,d}x=\left( \dfrac{x^2}{2}+x+\dfrac{2^{-x}}{\ln 2}\right) \Bigg|_0^2 =4+\dfrac{1}{4\ln 2}-\dfrac{1}{\ln 2}=4-\dfrac{3}{4\ln 2}$.
			\itemch Sai.
		\end{itemchoice}
	}
\end{ex}

%Câu 35
\begin{ex}%[2D4H3-1]
	Cho đồ thị hàm số $y=f(t)$ như hình vẽ.
	\begin{center}
		\begin{tikzpicture}[font=\footnotesize, line join=round, line cap=round, >=stealth, scale = 0.8]
			\draw[->] (-.5,0) --(0,0) node[below left]{$O$}--(5.3,0) node[below]{$t$};
			\draw[->] (0,-2.5) --(0,3) node[right]{$y$};
			\draw[fill = black] (1,0) node[below left]{$1$} circle (1pt) (2,0) node[below left]{$2$} circle (1pt) (3,0) node[below left]{$3$} circle (1pt) (4,0) node[below left]{$4$} circle (1pt) (5,0) node[below left]{$5$} circle (1pt);
			\draw[fill = black] (0,2) node[left]{$2$} circle (1pt) (0,-2) node[left] {$-2$} circle (1pt);
			\draw[line width = 1pt, red] (0,0)--(1,2)--(2,2);
			\draw[fill = black] (0,0) circle (1pt);
			\draw[line width = 1pt, red] 
			plot[domain=2:5, samples=100] (\x, {2/3*((\x)^3-9*(\x)^2+23*(\x)-15)});
			\draw[dashed] (1,0)--(1,2)--(0,2) (2,0)--(2,2);
		\end{tikzpicture}
	\end{center}
	Các mệnh đề sau đây đúng hay sai?
	\choiceTF
	{\True Diện tích hình phẳng được giới hạn các đồ thị hàm số $y=f(t)$, trục $O t$ và hai đường thẳng $t=0$; $t=1$ là $S=\dfrac{1}{2} \displaystyle\int\limits_{0}^{1} t \mathrm{\,d} t=\dfrac{1}{4}$}
	{\True Diện tích hình phẳng được giới hạn các đồ thị hàm số $y=f(t)$, trục $Ot$ và hai đường thẳng $t=1$; $t=2$ là $S=\displaystyle\int\limits_{1}^{2} 2 \mathrm{\,d}t=2$}
	{\True Tích phân $\displaystyle\int\limits_{2}^{3} f(x) \mathrm{\,d} x$ biểu thị cho phần diện tích của hình phẳng giới hạn các đồ thị hàm số $y=f(t)$, trục $O t$ và hai đường thẳng $t=2$; $ t=3$}
	{Tích phân $\displaystyle\int\limits_{3}^{5} f(x) \mathrm{\,d} x$ biểu thị cho phần diện tích của hình phẳng giới hạn các đồ thị hàm số $y=f(t)$, trục $O t$ và hai đường thẳng $t=3$; $ t=5$}
	\loigiai{
		\begin{itemchoice}
			\itemch Đúng. Vì đồ thị hàm số $y=f(t)$ trên đoạn $\left[0 ; 1\right]$ là $y=\dfrac{1}{2} t$. Do đó diện tích hình phẳng được giới hạn các đồ thị hàm số $y=f(t)$, trục $Ot$ và hai đường thẳng $t=0$; $t=1$ là $S=\dfrac{1}{2}\displaystyle\int\limits_{0}^{1} t \mathrm{\,d} t=\dfrac{1}{4}$.
			\itemch Đúng. Vì trên đoạn $ \left[1;2\right] $ đồ thị hàm số $y=f(t)=2$ nên hình phẳng được giới hạn bởi các đồ thị hàm số $ y=f(t) $, trục $O t$ và hai đường thẳng $t=1$; $t=2$ có diện tích là $S=\displaystyle\int\limits_{1}^{2} 2 \mathrm{\,d}t=2$.
			\itemch Đúng. Tích phân $\displaystyle\int\limits_{2}^{3} f(x) \mathrm{\,d} x=\displaystyle\int\limits_{2}^{3} f(t) \mathrm{\,d} t$ nên giá trị của tích phân $\displaystyle\int\limits_{2}^{3} f(t) \mathrm{\,d} t$ là diện tích của hình phẳng giới hạn các đồ thị hàm số $y=f(t)$, trục $Ot$ và hai đường thẳng $t=2$; $t=3$.
			\itemch Sai. Tích phân $\displaystyle\int\limits_{3}^{5} f(x) \mathrm{\,d} x=\displaystyle\int\limits_{3}^{5} f(t) \mathrm{\,d} t$.\\
			Diện tích hình phẳng được giới hạn các đồ thị hàm số $y=f(t)$, trục $O t$ và hai đường thẳng $t=3$; $t=5$ là $S=\displaystyle\int\limits_{3}^{5} \left|f(t)\right| \mathrm{\,d} t$.
		\end{itemchoice}
	}
\end{ex}

\Closesolutionfile{ans}
% \indapan{3}{ans/ans-2-C4B3CD1_10-19-DS}

\Opensolutionfile{ans}[ans/ans-C4B3CD1_20-26-KQ]
%\TNSA
%Câu 36

\begin{ex}%[2D4N3-1]
	Tính diện tích hình phẳng được tô màu trong hình bên dưới.
	\begin{center}
		\begin{tikzpicture}[font=\footnotesize, line join=round, line cap=round, >=stealth, scale = 1]
			\draw[->] (-.5,0) --(0,0) node[below left]{$O$}--(2.7,0) node[below]{$x$};
			\draw[->] (0,-.5) --(0,2.5) node[right]{$y$};
			\draw[fill = black] (1,0) node[below]{$1$} circle (1pt) (2,0) node[below]{$2$} circle (1pt);
			\draw[fill = black] (0,1) node[left]{$1$} circle (1pt)  (0,2) node[left]{$2$} circle (1pt);
			\draw[fill = black] (0,0) circle (1pt);
			\draw[dashed](0,2)--(2,2);
			\draw[line width =0.5 pt] (0,1) node[above right]{$ A $}--(2,2) node[above ]{$ B $}--(2,0) node[above right]{$ C $};
			\fill[pattern=north west lines] (0,1)--(2,2)--(2,0)--(0,0)--cycle;
			
		\end{tikzpicture}
	\end{center}
	\shortans{$ 3 $}
	\loigiai{
		\textbf{Cách 1:} Hình phẳng đã cho là hình thang vuông $ AOCB $, vuông tại $ A $, $ O $. Ta có
		$$ S=\dfrac{\left(AO+BC\right)\cdot OC}{2}=3.$$
		\textbf{Cách 2:} Đường thẳng $ AB $ đi qua hai điểm $ A\left(0;1\right) $ và $ B\left(2;2\right) $ nên đường thẳng $ AB $ có phương trình là $ y=\dfrac{1}{2}x+1 $.\\
		Hình phẳng đã cho giới hạn bởi đường thẳng $ y=\dfrac{1}{2}x+1 $, $ y=0 $, $ x=0 $, $ x=2 $ nên diện tích của hình phẳng là $ S=\displaystyle\int\limits_0^2 \left|\dfrac{1}{2}x+1 \right| \mathrm{\,d} x=3$.
	}
\end{ex}

%Câu 37
\begin{ex}%[2D4H3-1]
	Biết diện tích phần hình phẳng gạch chéo trong hình vẽ bên có diện tích là $ \dfrac{a}{b} $ với $ a$, $b \in \mathbb{Z} $ và phân số $ \dfrac{a}{b} $ tối giản. Tính tổng $ a+b $.
	\begin{center}
		\begin{tikzpicture}[font=\footnotesize, line join=round, line cap=round, >=stealth, scale = 1]
			\draw[->] (-.5,0) --(0,0) node[above left]{$O$}--(4.5,0) node[below]{$x$};
			\draw[->] (0,-.5) --(0,3.8) node[right]{$y$};
			\draw[fill = black] (1,0) node[below]{$1$} circle (1pt) (2,0) node[below]{$2$} circle (1pt);
			\draw[fill = black] (0,1) node[left]{$1$} circle (1pt)  (0,2) node[left]{$2$} circle (1pt);
			\draw[fill = black] (0,0) circle (1pt);
			\draw[dashed](1,0)--(1,1)--(0,1);
			\draw[line width = 0.5pt] plot[domain=0.2:3.8, samples=100] (\x, {((\x)-2)^2});
			\draw[line width = 0.5pt] plot[domain=-.5:3.6, samples=100] (\x, {(\x)});
			\fill[pattern=north west lines] (0,0)-- plot[domain=0:1](\x,{(\x)}) --(1,1)-- plot[domain=1:2](\x, {((\x)-2)^2}) --(2,0) --cycle;
			\draw (1.5,1.5)node[above,rotate=45]{\scriptsize $ y=x $};
			\draw (3.8,1)node[below]{\scriptsize $ y=\left(x-2\right)^2 $};
		\end{tikzpicture}
	\end{center}
	\shortans{$ 11 $}
	\loigiai{
		Dựa vào đồ thị, diện tích hình phẳng cần tìm là
		
		$S = \displaystyle\int\limits_{0}^{1} x \mathrm{\,d} x + \displaystyle\int\limits_{1}^{2}(x-2)^{2} \mathrm{\,d} x = \dfrac{1}{2} + \dfrac{1}{3} = \dfrac{5}{6}$.\\
		Vậy $ a=5 $; $ b=6 $ và $ a+b=11 $.
		
	}
\end{ex}

%Câu 38
\begin{ex}%[2D4H3-1]
	Biết diện tích phần tam giác cong $ OAB $ trong hình vẽ bên có diện tích là $ \dfrac{a}{b} $ với $ a$, $b \in \mathbb{Z} $ và phân số $ \dfrac{a}{b} $ tối giản. Tính hiệu $ b-a $.
	\begin{center}
		\begin{tikzpicture}[font=\footnotesize, line join=round, line cap=round, >=stealth, scale = 1]
			\draw[->] (-1.5,0) --(0,0) node[above left]{$O$}--(5.3,0) node[below]{$x$};
			\draw[->] (0,-1.5) --(0,4.5) node[right]{$y$};
			\foreach \x in {-1,1,2,3,4,5} \draw[fill] (\x,0) circle (1pt) node [below] { $\x$};
			\foreach \y in {-1,1,2,3,4} \draw[fill] (0,\y) circle (1pt) node [left] { $\y$};
			\draw[fill = red] (0,0) circle (1.2pt);
			\draw[line width = 0.5pt] plot[domain=-1.1:1.6, samples=100] (\x, {(\x)^3});
			\draw[line width = 0.5pt] plot[domain=-.1:4.1, samples=100] (\x, {((\x)-2)^2});
			\draw (1.5,3.5)node[left]{\scriptsize $ y=x^3 $};
			\draw (4,4.2)node[right]{\scriptsize $ y=x^2-4x+4 $};
			\draw[fill=red] (1,1) node[right]{$ A $} circle (1pt) (2,0) node[above right]{$ B $} circle (1pt);
		\end{tikzpicture}
	\end{center}
	\shortans{$ 5 $}
	\loigiai{
		Dựa vào hình vẽ ta thấy hình phẳng cần tính diện tích gồm 2 phần.\\
		Phần 1: Hình phẳng giới hạn bởi đồ thị hàm số $y=x^3$, trục $Ox$, $x=0$, $x=1$.\\
		Phần 2: Hình phẳng giới hạn bởi đồ thị hàm số $y=x^2-4 x+4$, trục $O x$, $x=1$, $x=2$.\\
		Do đó diện tích cần tính là 
		
		$S=\displaystyle\int\limits_{0}^{1}\left|x^3\right| \mathrm{\,d} x+\displaystyle\int\limits_{1}^{2}\left|x^2-4 x+4\right| \mathrm{\,d} x = \displaystyle\int\limits_{0}^{1} x^3 \mathrm{\,d} x + \displaystyle\int\limits_{1}^{2}\left(x^2-4 x+4\right) \mathrm{\,d} x = \dfrac{7}{12}$.\\
		Vậy $ a=7 $, $ b=12 $ và $ b-a=5 $.
	}
\end{ex}

%Câu 39
\begin{ex}%[2D4H3-1]
	Hình vuông $OABC$ có cạnh bằng $4$ được chia thành hai phần bởi đường cong $(C)$ có phương trình $y=\dfrac{1}{4} x^{2}$. Gọi $S_2$, $S_2$ lần lượt là diện tích của phần không tô màu và phần tô màu như hình vẽ bên dưới. Tỉ số $\dfrac{S_1}{S_2}$ bằng bao nhiêu?
	\begin{center}
		\begin{tikzpicture}[font=\footnotesize, line join=round, line cap=round, >=stealth, scale = 0.8]
			\draw[->] (-.5,0) --(0,0) node[above left]{$O$}--(5.3,0) node[below]{$x$};
			\draw[->] (0,-1) --(0,5) node[right]{$y$};
			\foreach \x in {2,4} \draw[fill] (\x,0) circle (1pt) node [below] { $\x$};
			\foreach \y in {2,4} \draw[fill] (0,\y) circle (1pt) node [left] { $\y$};
			\draw (0,0) circle (1.2pt);
			\draw[line width = 0.5pt] plot[domain=4:-.2, samples=100] (\x, {0.25*(\x)^2});
			\draw (4,0)--(4,4)--(0,4);
			\fill[gray] (0,0)-- plot[domain=0:4](\x,{0.25*(\x)^2})--(4,0)--cycle;
			\draw[fill=black] (0,4) node[above right]{$ A $} circle (1pt) (4,4) node[above right]{$ B $} circle (1pt) (4,0)node[above right]{$ C $} circle (1pt);
			\draw (1,2.5) node[right]{$ S_1 $} (3,1) node[above]{$ S_2 $};
		\end{tikzpicture}
	\end{center}
	\shortans{$ 2 $}
	\loigiai{Ta có diện tích hình vuông $O A B C$ là $ 16 $ và bằng $S_1+S_2$.\\
		Ta có $S_2=\displaystyle\int\limits_{0}^{4} \dfrac{1}{4} x^2 \mathrm{\,d} x =\left.\dfrac{x^3}{12}\right|_{0} ^{4}=\dfrac{16}{3} \Rightarrow \dfrac{S_1}{S_2}=\dfrac{16-S_2}{S_2}=\dfrac{16-\dfrac{16}{3}}{\dfrac{16}{3}}=2$.}
\end{ex}


%câu 40
\begin{ex}%[2D4V3-1]
	Cho hình thang cong $(H)$ giới hạn bởi các đường $y=\mathrm{e}^{x}$, $y=0$, $x=0$, $x=\ln 4$. Đường thẳng $x=k$, $(0<k<\ln 4)$ chia $(H)$ thành hai phần có diện tích là $S_1$ và $S_2$ như hình vẽ bên. Tìm $k$ để $S_1=2 S_2$ (làm tròn kết quả đến hàng phần chục).
	\begin{center}
		\begin{tikzpicture}[font=\footnotesize, line join=round, line cap=round, >=stealth, scale =1]
			\draw[->] (-1,0) --(0,0) node[below left]{$O$}--(2.5,0) node[below]{$x$};
			\draw[->] (0,-.7) --(0,4.6) node[right]{$y$};
			\draw[fill = black] (.8,0) node[below left]{$k$} circle (1pt);
			\draw[fill = black] (1.4,0) node[below right]{$\ln 4$} circle (1pt);
			\draw[fill = black] (0,1) node[above left]{$1$} circle (1pt);
			\draw[fill = black] (0,0) circle (1pt);
			\draw (.8,-.7)--(.8,4.4) (1.4,-.7)--(1.4,4.4);
			\draw [samples=100, domain=-.9:1.5] plot (\x, {e^(\x)});
			\fill[color=gray!10!black,shading=axis,opacity=0.2] (0,0) -- plot[smooth,samples=100,domain=0:1.4] (\x, {e^(\x)}) -- (1.4,0) -- cycle;
			\draw (0.2,.3) node[right]{$ S_1 $} (1,1) node[above]{$ S_2 $};
		\end{tikzpicture}
	\end{center}
	\shortans{$1,1$}
	\loigiai{
		Diện tích hình thang cong $(H)$ giới hạn bởi các đường $y=\mathrm{e}^{x}$, $y=0$, $x=0$, $x=\ln 4$ là
		$$S=\displaystyle\int\limits_{0}^{\ln 4} \mathrm{e}^{x} \mathrm{\,d} x = \mathrm{e}^{x}\Bigg|_{0}^{\ln 4}=\mathrm{e}^{\ln 4}-\mathrm{e}^{0}=4-1=3.$$\\
		Ta có $S=S_1+S_2=S_1+\dfrac{1}{2} S_1=\dfrac{3}{2} S_1$. Suy ra $S_1=\dfrac{2 S}{3}=\dfrac{2 \cdot 3}{3}=2$.\\
		Vì $S_1$ là phần diện tích được giới hạn bởi các đường $y=\mathrm{e}^{x}$, $y=0$, $x=0$, $x=k$ nên\\
		$
		2=S_1=\displaystyle\int\limits_{0}^{k} \mathrm{e}^{x} \mathrm{\,d} x=\mathrm{e}^{x}\Bigg|_{0}^{k}=\mathrm{e}^{k}-\mathrm{e}^{0}=\mathrm{e}^{k}-1$.\\
		Do đó $\mathrm{e}^{k}=3 \Leftrightarrow k=\ln 3 \approx 1{,}1$.}
\end{ex}

%câu 41
% \begin{ex}%[2D4V3-1]
% 	Cho hình phẳng $(H)$ giới hạn bởi các đường $y=\left|x^{2}-1\right|$ và $y=k$, với $0<k<1$. Tìm $k$ để diện tích hình phẳng $(H)$ gấp hai lần diện tích hình phẳng được kẻ sọc ở hình vẽ bên (làm tròn kết quả đến hàng phần trăm).
% 	\begin{center}
% 		\begin{tikzpicture}[font=\footnotesize, line join=round, line cap=round, >=stealth, scale =1]
% 			\draw[->] (-2,0) --(0,0) node[below left]{$O$}--(2,0) node[below]{$x$};
% 			\draw[->] (0,-3) --(0,3.3) node[right]{$y$};
% 			\draw[fill = black] (1,0) node[below left]{$1$} circle (1pt);
% 			\draw[fill = black] (0,1) node[above left]{$1$} circle (1pt);
% 			\draw[fill = black] (0,0) circle (1pt);
% 			\draw (-2,.4)--(2,.4) node[above right]{$ y=k $};
% 			\draw [samples=100, domain=-1:1] plot (\x, {-(\x)^2+1});
% 			\draw [samples=100, domain=-1:-1.8] plot (\x, {(\x)^2-1});
% 			\draw [samples=100, domain=1:1.8] plot (\x, {(\x)^2-1});
% 			\draw[dashed,samples=100,domain=-1:-1.8] plot (\x, {-(\x)^2+1});
% 			\draw[dashed,samples=100,domain=1:1.8] plot (\x, {-(\x)^2+1});
% 			\fill[pattern=north west lines] (-.77,.4) -- plot[smooth,samples=100,domain=-.77:.77] (\x, {-(\x)^2+1}) -- (.77,.4) -- cycle;
			
% 		\end{tikzpicture}
% 	\end{center}
% 	\shortans{$ 0{,}59 $}
% 	\loigiai{\begin{center}
% 			\begin{tikzpicture}[font=\footnotesize, line join=round, line cap=round, >=stealth, scale =1]
% 				\draw[->] (-2,0) --(0,0) node[below left]{$O$}--(2,0) node[below]{$x$};
% 				\draw[->] (0,-3) --(0,3.3) node[right]{$y$};
% 				\draw[fill = black] (1,0) node[below left]{$1$} circle (1pt);
% 				\draw[fill = black] (0,1) node[above left]{$1$} circle (1pt);
% 				\draw[fill = black] (0,0) circle (1pt);
% 				\draw (-2,.4)--(2,.4) node[above right]{$ y=k $};
% 				\draw [samples=100, domain=-1:1] plot (\x, {-(\x)^2+1});
% 				\draw [samples=100, domain=-1:-1.8] plot (\x, {(\x)^2-1});
% 				\draw [samples=100, domain=1:1.8] plot (\x, {(\x)^2-1});
% 				\draw[dashed,samples=100,domain=-1:-1.8] plot (\x, {-(\x)^2+1});
% 				\draw[dashed,samples=100,domain=1:1.8] plot (\x, {-(\x)^2+1});
% 				\fill[pattern=dots] (0,.4) -- plot[smooth,samples=100,domain=0:.77] (\x, {-(\x)^2+1}) -- (.77,.4) -- cycle;
% 				\fill[pattern=north west lines] (.77,.4)--plot[smooth,samples=100,domain=.77:1] (\x, {-(\x)^2+1}) -- plot[smooth,samples=100,domain=1:1.18] (\x, {(\x)^2-1}) -- (1.18,.4) -- cycle;
% 				\draw (0.77,.4) node[above]{$ A $} circle (1pt) (1.18,.4) node[above right]{$ B $} circle (1pt);
% 			\end{tikzpicture}
% 		\end{center}
% 		Gọi $S$ là diện tích hình phẳng $(H)$. Lúc dó $S=2 S_1+2 S_2$, trong đó $S_1$ là diện tích phần chấm bi và $S_2$ là diện tích phần gạch sọc trong hình vẽ bên.\\
% 		Gọi $A$, $ B$ là các giao điếm có hoành độ dương của đường thẳng $y=k$ và đồ thị hàm số $y=\left|x^{2}-1\right|$, trong đó $A(\sqrt{1-k}; k)$ và $B(\sqrt{1+k}; k)$.\\
% 		Theo yêu cầu bài toán
% 		\begin{eqnarray*}
% 			& & S=2 \cdot 2 S_1 \\
% 			& \Leftrightarrow & S_1=S_2 \\
% 			& \Leftrightarrow &\displaystyle\int\limits_{0}^{\sqrt{1-k}} \left(1-x^{2}-k\right) \mathrm{\,d} x =\displaystyle\int\limits_{\sqrt{1-k}}^{1}\left(k-1+x^{2}\right) \mathrm{\,d} x+\displaystyle\int\limits_{1}^{\sqrt{1+k}}\left(k-x^{2}+1\right) \mathrm{\,d} x \\
% 			& \Leftrightarrow & (1-k) \sqrt{1-k}-\dfrac{1}{3}(1-k) \sqrt{1-k}=\dfrac{1}{3}-(1-k)-\dfrac{1}{3}(1-k) \sqrt{1-k}+(1-k) \sqrt{1-k}+(1+k) \sqrt{1+k}-\dfrac{1}{3}(1+k) \sqrt{1+k}-(1+k)+\dfrac{1}{3} \\
% 			& \Leftrightarrow &\dfrac{2}{3}(1+k) \sqrt{1+k}=\dfrac{4}{3} \\
% 			& \Leftrightarrow & (\sqrt{1+k})^{3}=2\\
% 			& \Leftrightarrow & k=\sqrt[3]{4}-1\approx 0{,}59.
% 	\end{eqnarray*}}
	
% \end{ex}
\Closesolutionfile{ans}
% \indapan{6}{ans/ans-C4B3CD1_20-26-KQ}
% %\setcounter{chude}{1}
\begin{dang}{	THỂ TÍCH KHỐI TRÒN XOAY}
\end{dang}
% \begin{tomtat}
% 	\subsection{Thế tích của vật thế}
% 	\begin{center}
% 	\begin{tikzpicture}[>=stealth, scale=0.8]
% 		\draw plot[smooth,tension=.65] coordinates{(1,2) (2.5,2.3) (3.5,2.2)};
% 		\draw[dashed] plot[smooth,tension=.65] coordinates{(3.5,2.2) (4,2)};
% 		\draw plot[smooth,tension=.65] coordinates{(4,2) (5,2.2) (5.5,2.1)};
% 		\draw[dashed] plot[smooth,tension=.65] coordinates{(5.5,2.1) (6,2)};
% 		\draw plot[smooth,tension=.65] coordinates{(1,1) (2.3,0.5) (3.5,0.8)};
% 		\draw[dashed] plot[smooth,tension=.65] coordinates{(3.5,0.8) (4,1)};
% 		\draw plot[smooth,tension=.65] coordinates{(4,1) (5,0.7) (5.5,0.8)};
% 		\draw[dashed] plot[smooth,tension=.65] coordinates{(5.5,0.8) (6,1)};
% 		\draw[dashed] (1,1) arc (-90:90:.2 and 0.5);
% 		\draw (1,2) arc (90:270:.2 and 0.5);
% 		\draw[dashed] (4,1) arc (-90:90:.2 and 0.5);
% 		\draw (4,2) arc (90:270:.2 and 0.5);
% 		\draw (6,1) arc (-90:270:.2 and 0.5);
% 		\fill[pattern=north east lines] (4,1) arc (-90:90:.2 and 0.5)--(4,2) arc (90:270:.2 and 0.5)--cycle;
% 		\draw (-.5,0)--(0.5,0) (1,0)--(3.5,0) (4,0)--(5.5,0);
% 		\draw[dashed] (0.5,0)--(1,0) (3.5,0)--(4,0) (5.5,0)--(6,0);
% 		\draw[->] (6,0)--(7,0)node[below]{$x$};
% 		\draw (0.5,-1)--(0.5,3)--(1.5,3.5)--(1.5,2.2) (1.5,.8)--(1.5,-0.5)--(0.5,-1);
% 		\draw[dashed](1.5,2.2)--(1.5,.8);
% 		\draw[dashed] (1,1)--(1,0)node[below]{$a$};
% 		\coordinate (A) at (0.5,3);
% 		\coordinate (B) at (1.5,3.5);
% 		\coordinate (C) at (1.5,2.2);
% 		%\tkzMarkAngle[size=.6](A,B,C);
% 		\draw pic[draw=black, angle eccentricity=1.6, angle radius=0.5cm]{angle=A--B--C};
% 		\draw (1.3,3.2) node {\footnotesize $P$};
% 		\draw (3.5,-1)--(3.5,3)--(4.5,3.5)--(4.5,2) (4.5,1)--(4.5,-0.5)--(3.5,-1);
% 		\draw[dashed](4.5,2)--(4.5,1);
% 		\draw[dashed] (4,1)--(4,0)node[below]{$x$};
% 		\coordinate (D) at (3.5,3);
% 		\coordinate (E) at (4.5,3.5);
% 		\coordinate (F) at (4.5,2);
% 	%	\tkzMarkAngle[size=.6](D,E,F);
% 		\draw pic[draw=black, angle eccentricity=1.6, angle radius=0.5cm]{angle=D--E--F};
% 		\draw (4.3,3.2) node {\footnotesize $R$};
% 		\draw (5.5,-1)--(5.5,3)--(6.5,3.5)--(6.5,-0.5)--(5.5,-1);
% 		\draw[dashed] (6,1)--(6,0)node[below]{$b$};
% 		\coordinate (G) at (5.5,3);
% 		\coordinate (H) at (6.5,3.5);
% 		\coordinate (K) at (6.5,-0.5);
% 	%	\tkzMarkAngle[size=.6](G,H,K);
% 		\draw pic[draw=black, angle eccentricity=1.6, angle radius=0.5cm]{angle=G--H--K};	\draw (6.3,3.2) node {\footnotesize $Q$};
% 		\draw (0,.3) node {$O$};
% 		\fill (0,0) circle(1pt);
% 		\draw[->] (4,1.5)--(4.7,1.7) node[right] {\scriptsize $S(x)$};
% 	\end{tikzpicture}
% 	\end{center}
% 	Trong không gian, cho một vật thể nằm trong khoảng không gian giữa hai mặt phẳng $(P)$ và $(Q)$ cùng vuông góc với trục $O x$ tại các điểm $a$ và $b$. Mặt phẳng vuông góc với trục $O x$ tại điểm $x(a \leq x \leq b)$ cắt vật thể theo mặt cắt có diện tích $S(x)$. Khi đó, nếu $S(x)$ là hàm số liên tục trên $\left[a ; b\right]$ thì thể tích của vật thể được tính bởi công thức
% 	$$
% 	V=\displaystyle\int\limits_a^b S(x) \mathrm{\,d} x.
% 	$$
% \subsection{Thế tích khối tròn xoay}
% 	\begin{center}
% 		\begin{tikzpicture}[line join=round, line cap=round,>=stealth,thick,scale=.4]
% 		\tikzset{label style/.style={font=\normalsize}}
% 		%%Nhập giới hạn đồ thị và hàm số cần vẽ
% 		\def \xmin{-.5}
% 		\def \xmax{11}
% 		\def \ymin{-4}
% 		\def \ymax{4.5}
% 		%\draw[xstep=1 cm, ystep=1 cm,gray,thin] (\xmin,\ymin) grid (\xmax,\ymax);
% 		\def \hamso{0-0.01864083398323228*((\x)-1.0)^(3.0)+0.35126127715584465*((\x)-1.0)^(2.0)-1.5117402856674746*((\x)-1.0)+3.106314636847822}
% 		%%Tự động
% 		\draw[->] (\xmin,0)--(11,0) node[below left] {$x$};
% 		\draw[->] (0,\ymin)--(0,\ymax) node[below left] {$y$};
% 		\draw (0,0) node [below left] {$O$};
% 		\draw[dashed] (2.04,-1.89) arc(-90:90:.5 cm and 1.89 cm);
% 		\draw (2.04,1.89) arc(90:270:.5 cm and 1.89 cm);
		
% 		%\draw[dashed] (5.59,-1.77) arc(-90:90:.5 cm and 1.77 cm);
% 		%	\draw (5.59,1.77) arc(90:270:.5 cm and 1.77 cm);
		
% 		\draw (9,0) ellipse (1 cm and 3.96 cm);
		
% 		\draw[fill=black] (2.04,0) node [below] {$a$} circle (1.2pt);
% 		\draw[fill=black] (9,0) node [below] {$b$} circle (1.2pt);
% 		\draw[fill=black] (5.5,1.8) node [above,rotate=30] {\scriptsize $y=f(x)$};
% 		%%Tự động
% 		\begin{scope}
% 			\clip (\xmin+0.01,\ymin+0.01) rectangle (\xmax-0.01,\ymax-0.01);
% 			\draw[samples=350,domain=1:9.,smooth,variable=\x] plot (\x,{\hamso});
% 			\draw[samples=350,domain=1:9,smooth,variable=\x] plot (\x,{-0+0.01864083398323228*((\x)-1.0)^(3.0)-0.35126127715584465*((\x)-1.0)^(2.0)+1.5117402856674746*((\x)-1.0)-3.106314636847822});		
% 			\draw[pattern=north west lines,opacity=0.5] (2.04,0)--(2.04,1.89)plot[domain=2.04:9] (\x,{0-0.01864083398323228*((\x)-1.0)^(3.0)+0.35126127715584465*((\x)-1.0)^(2.0)-1.5117402856674746*((\x)-1.0)+3.106314636847822})--(9.0,0)--(2.04,0);
% 		\end{scope}
% 	\end{tikzpicture}
% 	\end{center}
% 	Cho hàm số $y=f(x)$ liên tục, không âm trên $\left[a ; b\right]$. Hình phẳng $(H)$ giới hạn bởi đồ thị hàm số $y=f(x)$, trục hoành $O x$ và hai đường thẳng $x=a$ và $x=b$ quay quanh trục $O x$ tạo thành một khối tròn xoay có thể tích bằng
% 	$$
% 	V=\pi \displaystyle\int\limits_a^b\left[f(x)\right]^2 \mathrm{\,d} x
% 	$$
% \end{tomtat}
%\TN
\Opensolutionfile{ans}[ans/ans-2-C4B3CD2-lc]
\begin{ex}%[2D4N3-3]
Viết công thức tính thể tích $V$ của khối tròn xoay được tạo ra khi quay hình thang cong, giới hạn bới đồ thị hàm số $y=f(x)$, trục $O x$ và hai đường thẳng $x=a$, $x=b$, $(a<b)$ xung quanh trục $O x$.
\choice
{$V=\displaystyle\int\limits_a^b \left|f(x)\right| \mathrm{\,d} x$}
{\True $V=\pi \displaystyle\int\limits_a^b f^2(x) \mathrm{\,d} x $}
{$V=\displaystyle\int\limits_a^b f^2(x) \mathrm{\,d} x$}
{$V=\pi \displaystyle\int\limits_a^b f(x) \mathrm{\,d} x$}
\loigiai{
Theo lí thuyết.
}
\end{ex}
\begin{ex}%[2D4N3-4]
	Cắt một vật thể bởi hai mặt phẳng vuông góc với trục $O x$ tại $x=1$ và $x=2$. Một mặt phẳng tùy ý vuông góc với trục $O x$ tại điểm có hoành độ $x$, $(1 \leq x \leq 2)$ cắt vật thể đó có diện tích $S(x)=2024 x$. Tính thể tích của phần vật thể giới hạn bởi hai mặt phẳng trên.
	\choice
	{\True $V=3036$}
	{$V=3036 \pi$}
	{$V=1518$}
	{$V=1518 \pi$}
\loigiai{Thể tích vật thể là $ V=\displaystyle\int\limits_1^2 2024x \mathrm{\,d}x=3036 $.}
\end{ex}

\begin{ex}%[2D4H3-4]
Cắt một vật thể bởi hai mặt phẳng vuông góc với trục $O x$ tại $x=1$ và $x=3$. Một mặt phẳng tùy ý vuông góc với trục $O x$ tại điểm có hoành độ $x$, $(1 \leq x \leq 3)$ cắt vật thể đó theo thiết diện là một hình chữ nhật có độ dài hai cạnh là $3 x$ và $3 x^2-2$. Tính thể tích của phần vật thể giới hạn bởi hai mặt phẳng trên.
\choice
{\True $V=156$}
{$V=156 \pi$}
{$ V=312 $}
{$V=312 \pi$}
\loigiai{Diện tích thiết diện là $S(x)=3 x \cdot \left(3 x^2-2\right)=9 x^3-6 x$.\\
Thể tích vật thể là $V=\displaystyle\int\limits_1^3\left(9 x^3-6 x\right) \mathrm{\,d} x=156$.
}
\end{ex}

\begin{ex}%[2D4N3-3]
Gọi $D$ là hình phẳng giới hạn bởi các đường $y=\mathrm{e}^{3 x}$, $y=0$, $x=0$ và $x=1$. Thể tích của khối tròn xoay tạo thành khi quay $D$ quanh trục $O x$ bằng
\choice
{$\pi \displaystyle\int\limits_0^1 \mathrm{e}^{3 x} \mathrm{\,d} x$}
{$\displaystyle\int\limits_0^1 \mathrm{e}^{6 x} \mathrm{\,d} x$}
{\True $\pi \displaystyle\int\limits_0^1 \mathrm{e}^{6 x} \mathrm{\,d} x$}
{$\displaystyle\int\limits_0^1 \mathrm{e}^{3 x} \mathrm{\,d} x$}
\loigiai{
Thể tích của khối tròn xoay tạo thành khi quay $D$ quanh trục $O x$ là\\
$$\pi \displaystyle\int\limits_0^1\left(\mathrm{e}^{3 x}\right)^2 \mathrm{\,d} x=\pi \displaystyle\int\limits_0^1 \mathrm{e}^{6 x} \mathrm{\,d} x.$$}
\end{ex}

\begin{ex}%[2D4N3-3]
Gọi $D$ là hình phẳng giới hạn bởi các đường $y=\mathrm{e}^{4 x}$, $y=0$, $x=0$ và $x=1$. Thể tích của khối tròn xoay tạo thành khi quay $D$ quanh trục $O x$ bằng
\choice
{$\displaystyle\int\limits_0^1 \mathrm{e}^{4 x} \mathrm{\,d} x$}
{\True $\pi \displaystyle\int\limits_0^1 \mathrm{e}^{8 x} \mathrm{\,d} x$}
{$\pi \displaystyle\int\limits_0^1 \mathrm{e}^{4 x} \mathrm{\,d} x$}
{$\displaystyle\int\limits_0^1 \mathrm{e}^{8 x} \mathrm{\,d} x$}
\loigiai{
Thể tích của khối tròn xoay tạo thành khi quay $D$ quanh trục $O x$ là 
$$V=\pi \displaystyle\int\limits_0^1\left(\mathrm{e}^{4 x}\right)^2 \mathrm{\,d} x=\pi \displaystyle\int\limits_0^1 \mathrm{e}^{8 x} \mathrm{\,d} x.$$}
\end{ex}

\begin{ex}%[2D4N3-3]
Cho hình phẳng $(H)$ giới hạn bởi các đường $y=x^2+3$, $y=0$, $x=0$, $x=2$. Gọi $V$ là thể tích của khối tròn xoay được tạo thành khi quay $(H)$ xung quanh trục $O x$. Mệnh đề nào dưới đây đúng?
\choice
{$V=\displaystyle\int\limits_0^2\left(x^2+3\right) \mathrm{\,d}x$}
{$V=\pi \displaystyle\int\limits_0^2\left(x^2+3\right) \mathrm{\,d}x$}
{$V=\displaystyle\int\limits_0^2\left(x^2+3\right)^2 \mathrm{\,d}x$}
{\True $V=\pi \displaystyle\int\limits_0^2\left(x^2+3\right)^2 \mathrm{\,d}x$}
\loigiai{Thể tích của khối tròn xoay được tạo thành khi quay $(H)$ xung quanh trục $O x$ là\\
$V=\pi \displaystyle\int\limits_0^2\left(x^2+3\right)^2 \mathrm{\,d} x$.
}
\end{ex}

\begin{ex}%[2D4H3-3]
Cho hình phẳng $D$ giới hạn bởi đường cong $y=\mathrm{e}^x$, trục hoành và các đường thẳng $x=0$, $x=1$. Khối tròn xoay tạo thành khi quay $D$ quanh trục hoành có thể tích $V$ bằng bao nhiêu?
\choice
{$V=\dfrac{\pi\left(\mathrm{e}^2+1\right)}{2}$}
{$V=\dfrac{\mathrm{e}^2-1}{2}$}
{$V=\dfrac{\pi \mathrm{e}^2}{3}$}
{\True $V=\dfrac{\pi\left(\mathrm{e}^2-1\right)}{2}$}
\loigiai{
$V=\pi \displaystyle\int\limits_0^1 \mathrm{e}^{2 x} \mathrm{\,d} x=\pi \dfrac{\mathrm{e}^{2x}}{2} \Bigg|_0 ^1=\dfrac{\pi\left(e^2-1\right)}{2}$.}
\end{ex}

\begin{ex}%[2D4H3-3]
Cho hình phẳng $D$ giới hạn bởi đường cong $y=\sqrt{x^2+1}$, trục hoành và các đường thẳng $x=0$, $x=1$. Khối tròn xoay tạo thành khi quay $D$ quanh trục hoành có thể tích $V$ bằng bao nhiêu?
\choice
{$ V=2 $}
{\True $V=\dfrac{4 \pi}{3} $}
{$V=2 \pi$}
{$V=\dfrac{4}{3}$}
\loigiai{Thể tích khối tròn xoay được tính theo công thức
	$$
	V=\pi \displaystyle\int\limits_0^1\left(\sqrt{x^2+1}\right)^2 \mathrm{\,d} x=\pi \displaystyle\int\limits_0^1\left(x^2+1\right) \mathrm{\,d} x=\pi\left(\dfrac{x^3}{3}+x\right)\Bigg|_0 ^1=\dfrac{4 \pi}{3} .
	$$}
\end{ex}

\begin{ex}%[2D4H3-3]
Cho hình phẳng $D$ giới hạn bởi đường cong $y=\sqrt{2+\cos x}$, trục hoành và các đường thẳng $x=0, x=\dfrac{\pi}{2}$. Khối tròn xoay tạo thành khi $D$ quay quanh trục hoành có thể tích $V$ bằng bao nhiêu?
\choice
{\True $V=(\pi+1) \pi$}
{$V=\pi-1$}
{$V=\pi+1$}
{$V=(\pi-1) \pi$}
\loigiai{Ta có 
	$$
	V=\pi \displaystyle\int\limits_0^{\tfrac{\pi}{2}}(\sqrt{2+\cos x})^2 \mathrm{\,d} x=\pi(2 x+\sin x)\Bigg|_0 ^{\tfrac{\pi}{2}}=\pi(\pi+1).
	$$}
\end{ex}

\begin{ex}%[2D4H3-3]
Cho hình phẳng $D$ giới hạn bởi đường cong $y=\sqrt{2+\sin x}$, trục hoành và các đường thẳng $x=0$, $ x=\pi$. Khối tròn xoay tạo thành khi quay $D$ quay quanh trục hoành có thể tích $V$ bằng bao nhiêu?
\choice
{\True $V=2 \pi(\pi+1)$}
{$V=2 \pi$}
{$V=2(\pi+1)$}
{$V=2 \pi^2$}
\loigiai{Ta có $V=\pi \displaystyle\int\limits_0^\pi\left(\sqrt{2+\sin x}\right)^2 \mathrm{\,d} x=\pi \displaystyle\int\limits_0^\pi\left(2+\sin x\right) \mathrm{\,d} x=\pi(2 x-\cos x)\Bigg|_0 ^\pi=2 \pi\left(\pi+1\right)$.}
\end{ex}

\begin{ex}%[2D4H3-3]
Tìm công thức tính thể tích của khối tròn xoay khi cho hình phẳng giới hạn bởi parabol $(P)\colon y=x^2$, đường thẳng $d\colon y=2 x$ và đường thẳng $x=0$, $x=2$ quay xung quanh trục $O x$.
\choice
{$\pi \displaystyle\int\limits_0^2\left(x^2-2 x\right)^2 \mathrm{\,d} x$}
{\True $\pi \displaystyle\int\limits_0^2 4 x^2 \mathrm{\,d} x-\pi \int_0^2 x^4 \mathrm{\,d} x$}
{ $\pi \displaystyle\int\limits_0^2 4 x^2 \mathrm{\,d} x+\pi \int_0^2 x^4 \mathrm{\,d} x$}
{$\pi \displaystyle\int\limits_0^2\left(2 x-x^2\right) \mathrm{\,d} x$}
\loigiai{Với mọi $ x \in \left[0;2\right] $ ta có $ 2x\ge 0 $, $ x^2\ge 0 $ và $ 2x\ge x^2 $ nên $V=\pi \displaystyle\int\limits_0^2 4 x^2 \mathrm{\,d} x-\pi \displaystyle\int\limits_0^2 x^4 \mathrm{\,d} x$.
}
\end{ex}

\begin{ex}%[2D4N3-3]
 Cho hình phẳng $(H)$ giới hạn bởi các đường $y=x^2+3$, $y=0$, $x=0$, $x=2$. Gọi $V$ là thể tích khối tròn xoay được tạo hành khi quay $ (H) $ xung quanh trục $ Ox $. Mệnh đề nào sau đây đúng?
 \choice
 {\True $ V=\pi \displaystyle\int\limits_0^2 \left(x^2+3\right)^2 \mathrm{\,d}x $}
 {$ V=\displaystyle\int\limits_0^2 \left(x^2+3\right)\mathrm{\,d}x $}
 {$ V=\displaystyle\int\limits_0^2 \left(x^2+3\right)^2 \mathrm{\,d}x $}
 {$ V=\pi \displaystyle\int\limits_0^2 \left(x^2+3\right) \mathrm{\,d}x $}
 \loigiai{Thể tích của vật tròn xoay là $ V=\pi \displaystyle\int\limits_0^2 \left(x^2+3\right)^2 \mathrm{\,d}x $.}
\end{ex}
	%Câu 13
\begin{ex}%[2D4N3-3]
	Gọi $V$ là thể tích của khối tròn xoay thu được khi quay hình thang cong, giới hạn bởi đồ thị hàm số $y=\sin x$, trục $Ox$, trục $Oy$ và đường thẳng $x=\dfrac{\pi}{2}$, xung quanh trục $Ox$. Mệnh đề nào dưới đây đúng?
	\choice
	{$V=\displaystyle\int\limits_0^{\tfrac{\pi}{2}}{\sin^2x\mathrm{\,d}x}$}
	{$V=\displaystyle\int\limits_0^{\tfrac{\pi}{2}}{\sin x\mathrm{\,d}x}$}
	{\True $V=\pi\displaystyle\int\limits_0^{\tfrac{\pi}{2}}{\sin^2x\mathrm{\,d}x}$}
	{$V=\pi\displaystyle\int\limits_0^{\tfrac{\pi}{2}}{\sin x\mathrm{\,d}x}$}
	\loigiai{
		Công thức tính $V=\pi\displaystyle\int\limits_a^b{f^2(x)\mathrm{\,d}x}$.
	}
\end{ex}

%Câu 14
\begin{ex}%[2D4H3-3]
	Thể tích khối tròn xoay được sinh ra khi quay hình phẳng giới hạn bởi đồ thị của hàm số $y=x^2-2x$, trục hoành, đường thẳng $x=0$ và $x=1$ quanh trục hoành bằng
	\choice
	{$\dfrac{16\pi}{15}$}
	{$\dfrac{2\pi}{3}$}
	{$\dfrac{4\pi}{3}$}
	{\True $\dfrac{8\pi}{15}$}
	\loigiai{
		Ta có
		\allowdisplaybreaks
		\begin{eqnarray*}
			V&=&\pi\displaystyle\int\limits_0^1\left(x^2-2x\right)^2\mathrm{\,d}x\\
			&=&\pi\displaystyle\int\limits_0^1\left(x^4-4x^3+4x^2\right)\mathrm{\,d}x\\
			&=&\pi \cdot \left(\dfrac{x^5}{5}-x^4+\dfrac{4x^3}{3}\right)\Bigg|_0^1\\
			&=&\pi \cdot \left(\dfrac{1}{5}-1+\dfrac{4}{3}\right)=\dfrac{8\pi}{15}.
		\end{eqnarray*}
	}
\end{ex}

%Câu 15
\begin{ex}%[2D4N3-3]
	Cho miền phẳng $(D)$ giới hạn bởi $y=\sqrt x$, hai đường thẳng $x=1$, $x=2$ và trục hoành. Tính thể tích khối tròn xoay tạo thành khi quay $(D)$ quanh trục hoành.
	\choice
	{$3\pi $}
	{\True $\dfrac{3\pi}{2}$}
	{$\dfrac{2\pi}{3}$}
	{$\dfrac{3}{2}$}
	\loigiai{
		$V=\pi\displaystyle\int\limits_1^2x\mathrm{\,d}x=\dfrac{\pi x^2}{2}\Bigg|_1^2=\dfrac{3\pi}{2}$.
	}
\end{ex}

%Câu 16
\begin{ex}%[2D4N3-3]
	Cho hình phẳng $(H)$ giới hạn bởi các đường $y=2x-x^2$, $y=0$. Quay $(H)$ quanh trục hoành tạo thành khối tròn xoay có thể tích là
	\choice
	{$\displaystyle\int\limits_0^2\left(2x-x^2\right)\mathrm{\,d}x$}
	{\True $\pi\displaystyle\int\limits_0^2\left(2x-x^2\right)^2\mathrm{\,d}x$}
	{$\displaystyle\int\limits_0^2\left(2x-x^2\right)^2\mathrm{\,d}x$}
	{$\pi\displaystyle\int\limits_0^2\left(2x-x^2\right)\mathrm{\,d}x$}
	\loigiai{
		Theo công thức ta chọn $V=\pi\displaystyle\int\limits_0^2\left(2x-x^2\right)^2\mathrm{\,d}x$.
	}
\end{ex}

%Câu 17
\begin{ex}%[2D4H3-3]
	Cho hình phẳng giới hạn bởi các đường $y=\sqrt{x}-2$, $y=0$ và $x=4$, $x=9$ quay xung quanh trục $Ox$. Tính thể tích khối tròn xoay tạo thành.
	\choice
	{$V=\dfrac{7}{6}$}
	{$V=\dfrac{5\pi}{6}$}
	{$V=\dfrac{7\pi}{11}$}
	{\True $V=\dfrac{11\pi}{6}$}
	\loigiai{
		Thể tích của khối tròn xoay tạo thành là
		\allowdisplaybreaks
		\begin{eqnarray*}
			V&=&\pi\displaystyle\int\limits_4^9\left(\sqrt{x}-2\right)^2\mathrm{\,d}x\\ &=&\pi\displaystyle\int\limits_4^9\left(x-4\sqrt{x}+4\right)\mathrm{\,d}x\\
			&=&\pi\cdot \left(\dfrac{x^2}{2}-\dfrac{8x\sqrt{x}}{3}+4x\right)\Bigg|_4^9\\
			&=&\pi\left(\dfrac{81}{2}-72+36\right)-\pi\left(\dfrac{16}{2}-\dfrac{64}{3}+16\right)=\dfrac{11\pi}{6}.
		\end{eqnarray*}
	}
\end{ex}

%Câu 18
\begin{ex}%[2D4H3-3]
	Cho hình phẳng $(H)$ giới hạn bởi các đường thẳng $y=x^2+2$, $y=0$, $x=1$, $x=2$. Gọi $V$ là thể tích của khối tròn xoay được tạo thành khi quay $(H)$ xung quanh trục $Ox$. Mệnh đề nào dưới đây đúng?
	\choice
	{$V=\displaystyle\int\limits_1^2\left(x^2+2\right)\mathrm{\,d}x$}
	{\True $V=\pi\displaystyle\int\limits_1^2\left(x^2+2\right)^2\mathrm{\,d}x$}
	{$V=\displaystyle\int\limits_1^2\left(x^2+2\right)^2\mathrm{\,d}x$}
	{$V=\pi\displaystyle\int\limits_1^2\left(x^2+2\right)\mathrm{\,d}x$}
	\loigiai{
		Ta có $V=\pi\displaystyle\int\limits_1^2\left(x^2+2\right)^2\mathrm{\,d}x$.
	}
\end{ex}
\Closesolutionfile{ans}
% \indapan{6}{ans/ans-2-C4B3CD2-lc}

\Opensolutionfile{ans}[ans/ans-2-C4B3CD2_5-10-KQ]
%\TNSA

%Câu 19
\begin{ex}%[2D4H3-4]
	Cắt một vật thể $(T)$ bởi hai mặt phẳng vuông góc với trục $Ox$ tại $x=0$ và $x=2$. Một mặt phẳng tùy ý vuông góc với trục $Ox$ tại điểm có hoành độ $x$ ($0\le x\le 2$) cắt vật thể đó có theo một thiết diện là một hình vuông có cạnh bằng $\sqrt{x^3}$. Thể tích vật thể $(T)$ là số hữu tỉ có dạng phân số tối giản $\dfrac{a}{b}$. Tính $a+b$.
	\shortans{$135$}
	\loigiai{
		Diện tích thiết diện là $S(x)=\sqrt{x^3}\cdot \sqrt{x^3}=x^6$.\\
		Thể tích của vật thể $(T)$ là $V=\displaystyle\int\limits_0^2S(x)\mathrm{\,d}x=\displaystyle\int\limits_0^2x^6\mathrm{\,d}x=\dfrac{128}{7}$.\\
		Suy ra $a=128$ và $b=7$. Khi đó, $a+b=135$.
	}
\end{ex}

%Câu 20
\begin{ex}%[2D4H3-4]
	Cắt một vật thể bởi hai mặt phẳng vuông góc với trục $Ox$ tại $x=1$; $x=3$. Khi cắt một vật thể bởi mặt phẳng vuông góc với trục $Ox$ tại điểm có hoành độ $x$ ($1\le x\le 3$), mặt cắt là tam giác vuông có một góc $45^\circ$ và độ dài một cạnh góc vuông là $\sqrt{4-\dfrac{1}{2} x^2}$. Thể tích vật thể trên là một số hữu tỉ có dạng phân số tối giản $\dfrac{a}{b}$. Tính $a\cdot b$.
	\shortans{$66$}
	\loigiai{
		Diện tích tam giác vuông cân là $S(x)=\dfrac{1}{2}\sqrt{4-\dfrac{1}{2} x^2}\cdot \sqrt{4-\dfrac{1}{2}x^2}=\dfrac{1}{2}\left(4-\dfrac{1}{2}x^2\right)$.\\
		Vậy thể tích vật thể là \[V=\displaystyle\int\limits_1^3\dfrac{1}{2}\left(4-\dfrac{1}{2}{x^2}\right)\mathrm{\,d}x=\dfrac{11}{6}.\]
		Suy ra $a=11$; $b=6$. Khi đó $a\cdot b=66$.
	}
\end{ex}

%Câu 21
\begin{ex}%[2D4H3-3]
	Tính thể tích khối tròn xoay khi quay hình phẳng $(H)$ xác định bởi các đường $y=\dfrac{1}{3}x^3-x^2$, $y=0$, $x=0$ và $x=3$ quanh trục $Ox$ (kết quả viết dưới dạng số thập phân và làm tròn đến hàng phần trăm).
	\shortans{$7{,}27$}
	\loigiai{
		Thể tích khối tròn xoay sinh ra khi quay hình phẳng $(H)$ quanh trục $Ox$ là
		$$V=\pi\displaystyle\int\limits_0^3\left(\dfrac{1}{3}x^3-x^2\right)^2\mathrm{\,d}x=\pi\displaystyle\int\limits_0^3\left(\dfrac{1}{9}x^6-\dfrac{2}{3}x^5+x^4\right)\mathrm{\,d}x=\dfrac{81\pi}{35} \approx 7{,}27.$$
	}
\end{ex}

%Câu 22
\begin{ex}%[2D4H3-3]
	Tính thể tích của vật thể tạo nên khi quay quanh trục $Ox$ hình phẳng $D$ giới hạn bởi đồ thị $(P)\colon y=2x-x^2$, trục $Ox$ và hai đường thẳng $x=0$, $x=2$ (Kết quả viết dưới dạng số thập phân và làm tròn đến hàng phần trăm).
	\shortans{$3{,}35$}
	\loigiai{
		Ta có
		\allowdisplaybreaks
		\begin{eqnarray*}
			V&=&\pi\displaystyle\int\limits_0^2\left(2x-x^2\right)^2\mathrm{\,d}x\\
			&=&\pi\displaystyle\int\limits_0^2\left(4x^2-4x^3+x^4\right)\mathrm{\,d}x\\
			&=&\pi\left(\dfrac{4}{3}{x^3}-x^4+\dfrac{1}{5}{x^5}\right)\Bigg|_0^2\\
			&=&\dfrac{16}{15}\pi\approx 3{,}35.
		\end{eqnarray*}
	}
\end{ex}

%Câu 23
\begin{ex}%[2D4H3-3]
	Cho hình phẳng giới hạn bởi các đường $y=\tan x$, $y=0$, $x=0$, $x=\dfrac{\pi}{4}$ quay xung quanh trục $Ox$. Tính thể tích vật thể tròn xoay được sinh ra (kết quả viết dưới dạng số thập phân và làm tròn một chữ số thập phân sau dấu phẩy).
	\shortans{$0{,}8$}
	\loigiai{
		Thể tích vật thể tròn xoay được sinh ra là
		\[V=\pi\displaystyle\int\limits_0^{\tfrac{\pi}{4}}{\tan^2x\mathrm{\,d}x}=\pi\displaystyle\int\limits_0^{\tfrac{\pi}{4}}{\left(\dfrac{1}{\cos^2x-1}\right)}\mathrm{\,d}x=\pi\left(\tan x-x\right)\Bigg|_0^{\tfrac{\pi}{4}}=\dfrac{4\pi-\pi^2}{4} \approx 0{,}8.\]
	}
\end{ex}

%Câu 24
\begin{ex}%[2D4H3-3]
	Gọi $V$ là thể tích khối tròn xoay tạo thành do quay xung quanh trục hoành một elip có phương trình $\dfrac{x^2}{25}+\dfrac{y^2}{16}=1$. Tính $V$ (Kết quả làm tròn đến hàng đơn vị).
	\shortans{$335$}
	\loigiai{
		Quay elip đã cho xung quanh trục hoành chính là quay hình phẳng $H$ giới hạn bởi $y=4\sqrt{1-\dfrac{x^2}{25}}$, $y=0$, $x=-5$, $x=5$.\\
		Vậy thể tích khối tròn xoay sinh ra bởi $H$ khi quay xung quanh trục hoành là
		\[V=\pi\displaystyle\int_{-5}^5\left(16-\dfrac{16x^2}{25}\right)\mathrm{\,d}x=\pi\left(16x-\dfrac{16x^3}{75}\right)\Bigg|^5_{-5}=\dfrac{320\pi}{3}\approx 335.\]
	}
\end{ex}

%Câu 25
\begin{ex}%[2D4H3-3]%Câu 13
	\immini{Cho hình phẳng $(H)$ được gạch chéo trong hình bên. Tính thể hình tròn xoay sinh ra bởi $(H)$ khi quay $(H)$ quanh trục $Ox$ (Kết quả viết dưới dạng số thập phân và làm tròn đến hàng phần chục).
	}{
		\begin{tikzpicture}[line join=round, line cap=round,>=stealth,thick,scale=0.7]
			\tikzset{every node/.style={scale=0.8}}
			\draw[->] (-3.1,0)--(3.1,0) node[below left] {$x$};
			\draw[->] (0,-1.1)--(0,5.1) node[below left] {$y$};
			\draw (0,0) node [below left] {$O$};
			\foreach \x/\nx in {1/1,2/2}
			\draw (\x,1pt)--(\x,-1pt) node [below left] {$\nx$};
			\foreach \y/\ny in {1/1,2/2,3/3,4/4}
			\draw (1pt,\y)--(-1pt,\y) node [left] {$\ny$};
			\begin{scope}
				\clip (-3,-1) rectangle (3,5);
				\draw[samples=200,domain=-2:2,smooth,variable=\x] plot (\x,{1*(\x)^2+0*(\x)+0});
				\fill[pattern=north east lines](1,0)--plot[samples=200,domain=1:2,smooth,variable=\x] (\x,{(\x)^2})--(2,0);
				\draw plot[samples=200,domain=-2:2.15,smooth,variable=\x] (\x,{(\x)^2}) node[left=2cm]{$y=x^2$};
				\draw (1,-0.8)--(1,4.3) (2,-0.8)--(2,4.3);
				\fill[black](1,1) circle (2pt);
				\fill[black](2,4) circle (2pt);
			\end{scope}
		\end{tikzpicture}
	}
	\shortans{$19{,}5$}
	\loigiai{
		Ta có $V=\pi\displaystyle\int_1^2{\left(x^2\right)^2\mathrm{\,d}x}=\pi\dfrac{x^5}{5}\Bigg|^2_1=\dfrac{31\pi}{5}\approx 19{,}5$.
	}
\end{ex}

%Câu 26
\begin{ex}%[2D4H3-3]
	\immini{Cho hình phẳng $(D)$ được tô màu trong hình bên. Tính thể hình tròn xoay sinh ra bởi $(D)$ khi quay $(D)$ quanh trục $Ox$ (Kết quả viết dưới dạng số thập phần và làm tròn đến hàng phần trăm).
	}{
		\begin{tikzpicture}[line join=round, line cap=round,>=stealth,thick]
			\tikzset{every node/.style={scale=0.9}}
			\draw[->] (-1.1,0)--(3.1,0) node[below left] {$x$};
			\draw[->] (0,-1.1)--(0,3.1) node[below left] {$y$};
			\draw (0,0) node [below left] {$O$};
			\foreach \x/\nx in {1/1,2/2}
			\draw[thin] (\x,1pt)--(\x,-1pt) node [below] {$\nx$};
			\foreach \y/\ny in {1/1,2/2}
			\draw[thin] (1pt,\y)--(-1pt,\y) node [left] {$\ny$};
			\begin{scope}
				\clip (-1,-1) rectangle (3,3);
				\draw[pattern=north east lines](1,0)--plot[samples=200,domain=1:2,smooth,variable=\x] (\x,{1+1/(\x)})--(2,0);
				\draw plot[samples=200,domain=0.1:2.7,smooth,variable=\x] (\x,{1+1/(\x)});
				\draw (1.7,2.5) node{$y=1+\dfrac{1}{x}$};
				\draw[dashed](1,2)--(0,2);			
			\end{scope}
			\draw (1.5,1) node[circle, fill=white] {$\mathrm{D}$};
		\end{tikzpicture}
	}
	\shortans{$9{,}08$}
	\loigiai{
		Ta có $V=\pi\displaystyle\int_1^2{\left(1+\dfrac{1}{x}\right)^2\mathrm{\,d}x}=\pi\displaystyle\int_1^2{\left(1+\dfrac{2}{x}+\dfrac{1}{x^2}\right)\mathrm{\,d}x}=\pi\left(x+\ln x-\dfrac{1}{x}\right)\Bigg|^2_1 \approx 9{,}08$.
	}
\end{ex}

%Câu 27
\begin{ex}%[2D4H3-3]
	\immini{Cho hình phẳng $(H)$ được tô màu trong hình bên. Tính thể hình tròn xoay sinh ra bởi $(H)$ khi quay $(H)$ quanh trục $Ox$ (Kết quả viết dưới dạng số thập phân và làm tròn đến hàng phần chục)
	}{
		\begin{tikzpicture}[line join=round, line cap=round,>=stealth,thick]
			\tikzset{every node/.style={scale=0.9}}
			\draw[->] (-1.6,0)--(2.1,0) node[below left] {$x$};
			\draw[->] (0,-1.1)--(0,3.1) node[below left] {$y$};
			\draw (0,0) node [below left] {$O$};
			\foreach \x/\nx in {-1/-1,1/1}
			\draw[thin] (\x,1pt)--(\x,-1pt) node [below] {$\nx$};
			\foreach \y/\ny in {1/1}
			\draw[thin] (1pt,\y)--(-1pt,\y) node [left] {$\ny$};
			\begin{scope}
				\clip (-1.5,-1) rectangle (2,3);
				\draw[pattern=north east lines](-1,0)--plot[samples=200,domain=-1:1,smooth,variable=\x] (\x,{e^(\x)})--(1,0);
				\draw[samples=200,domain=-1.5:2,smooth,variable=\x] plot (\x,{e^(\x)});
				\path (0,1)--(1,e) node[pos=0.7, above, sloped]{$y=\mathrm{e}^x$};
			\end{scope}
		\end{tikzpicture}
	}
	\shortans{$11{,}4$}
	\loigiai{
		Ta có $V=\pi\displaystyle\int_{-1}^1{\left(\mathrm{e}^x\right)^2\mathrm{\,d}x}=\pi\displaystyle\int_{-1}^1{\left(\mathrm{e}^{2x}\right)\mathrm{\,d}x}=\dfrac{\pi}{2}\mathrm{e}^{2x}\Bigg|^1_{-1} \approx 11{,}4$.
	}
\end{ex}

%Câu 28
\begin{ex}%[2D4H3-3]
	\immini{Cho hình phẳng $(H)$ được tô màu trong hình bên. Tính thể hình tròn xoay sinh ra bởi $(H)$ khi quay $(H)$ quanh trục $Ox$ (Kết quả viết dưới dạng số thập phân và làm tròn đến hàng phần chục).
	}{
		\begin{tikzpicture}[line join=round, line cap=round,>=stealth,thick]
			\tikzset{every node/.style={scale=0.9}}
			\draw[->] (-1.1,0)--(3.1,0) node[below left] {$x$};
			\draw[->] (0,-1.1)--(0,3.1) node[below left] {$y$};
			\draw (0,0) node [below left] {$O$};
			\foreach \x/\nx in {1/1,2/2}
			\draw[thin] (\x,1pt)--(\x,-1pt) node [below] {$\nx$};
			\draw[thin] (1pt,1)--(-1pt,1) node [below left] {$1$};
			\draw[thin] (1pt,2)--(-1pt,2) node [left] {$2$};
			\begin{scope}
				\clip (-1,-1) rectangle (3,3);
				\draw[pattern=north east lines](0,0)--(0,1)--(2,2)--(2,0);
				\draw[dashed](0,2)--(2,2);
				\fill[black](0,1) circle (1.5pt) node[above left]{$A$};
				\fill[black](2,2) circle (1.5pt) node[right]{$B$};
				\fill[black](2,0) circle (1.5pt) node[above right]{$C$};
			\end{scope}
		\end{tikzpicture}
	}
	\shortans{$14{,}7$}
	\loigiai{
		Gọi đường thẳng $d$ đi qua $A$ và $B$ có phương trình dạng $y=ax+b$.\\
		Ta có hệ phương trình $\heva{&b=1\\&2a+b=2} \Rightarrow \heva{&a=\dfrac{1}{2}\\&b=1.}$\\
		Suy ra $d \colon y=\dfrac{1}{2}x+1$.\\
		Khi đó
		$V=\pi\displaystyle\int_0^1{\left(\dfrac{1}{2}x+1\right)^2\mathrm{\,d}x} \approx 14{,}7$.
	}
\end{ex}

%Câu 29
\begin{ex}%[2D4V3-3]
	\immini{Cho hình phẳng $(H)$ là tam giác cong $OAB$ trong hình vẽ bên. Tính thể hình tròn xoay sinh ra bởi $(H)$ khi quay $(H)$ quanh trục $Ox$ (Kết quả viết dưới dạng số thập phân và làm tròn đến hàng phần trăm).
	}{
		\begin{tikzpicture}[line join=round, line cap=round,>=stealth,thick]
			\tikzset{every node/.style={scale=0.9}}
			\draw[dashed, step=1, gray!50,very thin] (-.5,-0.9) grid (4.5,4.5);
			\draw[->] (-1.6,0)--(5.1,0) node[below] {$x$};
			\draw[->] (0,-1.8)--(0,4.5) node[right] {$y$};
			\draw (0,0) node [below left] {$O$};
			\foreach \x/\nx in {-1/-1,1/1,2/2,3/3,4/4}
			\draw[thin] (\x,1pt)--(\x,-1pt) node [below] {$\nx$};
			\foreach \y/\ny in {-1/-1,1/1,2/2,3/3,4/4}
			\draw[thin] (1pt,\y)--(-1pt,\y) node [left] {$\ny$};
			\begin{scope}
				\clip (-1.5,-1.5) rectangle (4.5,4.5);
				\draw plot[samples=200,domain=-1.2:1.7,smooth,variable=\x] (\x,{(1*(\x)^3});
				\path (1,1)--(2,8) node[pos=0.4,above, sloped]{$y=x^3$};
				\draw plot[samples=200,domain=-0.3:4.3,smooth,variable=\x] (\x,{1*(\x)^2+-4*(\x)+4});
				\fill [pattern=north east lines](0,0)--plot[samples=200,domain=0:1,smooth,variable=\x] (\x,{(\x)^3})--plot[samples=200,domain=1:2,smooth,variable=\x] (\x,{(\x)^2-4*(\x)+4})--(2,0)--cycle;
				\path (3,1)--(4,4) node[pos=0.55,above, sloped]{$y=x^2-4x+4$};
				\fill[black](1,1) circle (1.5pt) node[right]{$A$};
				\fill[black](2,0) circle (1.5pt) node[above]{$B$};
			\end{scope}
		\end{tikzpicture}
	}
	\shortans{$1{,}08$}
	\loigiai{
		Ta có $V=\pi\displaystyle\int_0^1{\left(x^3\right)^2\mathrm{\,d}x}+\pi\displaystyle\int_1^2{\left(x^2-4x+4\right)^2\mathrm{\,d}x} \approx 1{,}08$.
	}
\end{ex}

%Câu 30
\begin{ex}%[2D4V3-3]
	\immini{Gọi $V$ là thể tích khối tròn xoay tạo thành khi quay hình phẳng giới hạn bởi các đường $y=\sqrt{x}$, $y=0$ và $x=4$ quanh trục $Ox$. Đường thẳng $x=a$, $\left(0<a<4\right)$ cắt đồ thị hàm số $y=\sqrt{x}$ tại $M$ (hình vẽ). Gọi $V_1$ là thể tích khối tròn xoay tạo thành khi quay tam giác $OMH$ quanh trục $Ox$. Biết rằng $V=2V_1$. Tìm $a$.
	}{
		\begin{tikzpicture}[line join=round, line cap=round,>=stealth,thick]
			\tikzset{every node/.style={scale=0.9}}
			\draw[->] (-0.6,0)--(5.1,0) node[below left] {$x$};
			\draw[->] (0,-0.4)--(0,2.5) node[below left] {$y$};
			\draw (0,0) node [below left] {$O$};
			\foreach \x/\nx in {3/a,4/4}
			\draw[thin] (\x,1pt)--(\x,-1pt) node [below] {$\nx$};
			\begin{scope}
				\clip (-1.0,-1.0) rectangle (4.5,2.5);
				\draw plot[samples=200,domain=0:4.5,smooth,variable=\x] (\x,{sqrt((\x))});
				\path (0,0)--(4,1) node[pos=0.45,above=0.8cm, sloped]{$y=\sqrt x$};
				\fill[black](3,1.732) circle (1.5pt) node[above right]{$M$} (4,0) node[above right]{$H$};
				\draw (3,2)--(3,-0.1);
				\draw[pattern=north east lines](0,0)--(3,1.732)--(4,0);
			\end{scope}
		\end{tikzpicture}
	}
	\shortans{$3$}
	\loigiai{
		\immini{
			Ta có $V=\pi\displaystyle\int\limits_0^4x\mathrm{\,d}x=\pi\dfrac{x^2}{2}\Bigg|_0^4=8\pi$.\\
			Mà $V=2V_1\Rightarrow{V_1}=4\pi$.\\
			Gọi $K$ là hình chiếu của $M$ trên $Ox$.\\
			Suy ra $OK=a$, $KH=4-a$, $MK=\sqrt a$.\\
			Khi xoay tam giác $OMH$ quanh $Ox$ ta được khối
		}{
			\begin{tikzpicture}[line join=round, line cap=round,>=stealth,thick]
				\tikzset{every node/.style={scale=0.9}}
				\draw[->] (-0.6,0)--(5.1,0) node[below left] {$x$};
				\draw[->] (0,-0.4)--(0,2.5) node[below left] {$y$};
				\draw (0,0) node [below left] {$O$};
				\foreach \x/\nx in {3/a,4/4}
				\draw[thin] (\x,1pt)--(\x,-1pt) node [below] {$\nx$};
				\begin{scope}
					\clip (-1.0,-1.0) rectangle (4.5,2.5);
					\draw plot[samples=200,domain=0:4.5,smooth,variable=\x] (\x,{sqrt((\x))});
					\path (0,0)--(4,1) node[pos=0.45,above=0.8cm, sloped]{$y=\sqrt x$};
					\fill[black](3,1.732) circle (1.5pt) node[above right]{$M$} (4,0) node[above right]{$H$} (3,0)node[below right]{$K$};
					\draw (3,2)--(3,-0.1);
					\draw[pattern=north east lines](0,0)--(3,1.732)--(4,0);
				\end{scope}
			\end{tikzpicture}
		}\hspace{-0.77cm}
		tròn xoay là sự lắp ghép của hai khối nón sinh bởi các tam giác $OMK$, $MHK$, hai khối nón đó có cùng mặt đáy và có tổng chiều cao là $OH=4$ nên thể tích của khối tròn xoay đó là $V_1=\dfrac{1}{3} \cdot \pi \cdot 4 \cdot \left(\sqrt a\right)^2=\dfrac{4\pi a}{3}$, từ đó suy ra $a=3$.
	}
\end{ex}
\Closesolutionfile{ans}
% \indapan{6}{ans/ans-2-C4B3CD2_5-10-KQ}
% \begin{dang}{Ứng dụng diện tích hình phẳng và thể tích khối tròn xoay trong bt thực tiễn}
\end{dang}

% \begin{dang}{Ứng dụng diện tích hình phẳng trong bài toán thực tiễn}
% \end{dang}

\Opensolutionfile{ans}[ans/ans-2-C4B3CD3_1-4-lc]
%\TN

%Câu 1
\begin{ex}%[2D4V3-2]
	Trường Nguyễn Văn Trỗi muốn làm một cái cửa nhà hình parabol có chiều cao từ mặt đất đến đỉnh là $2{,}25$\,mét, chiều rộng tiếp giáp với mặt đất là $3$\,mét. Giá thuê mỗi mét vuông là $1\,500\,000$\,đồng. Vậy số tiền nhà trường phải trả là
	\choice
	{$33\,750\,000$\,đồng}
	{$3\,750\,000$\,đồng}
	{$12\,750\,000$\,đồng}
	{\True $6\,750\,000$\,đồng}
	\loigiai{
		\immini{Gọi phương trình parabol
			\[(P)\colon y=ax^2+bx+c.\]
			Do tính đối xứng của parabol nên ta có thể chọn hệ trục tọa độ $Oxy$ sao cho $(P)$ có đỉnh $I\in Oy$ (như hình vẽ).\\
			Ta có hệ phương trình\\ $\heva{&\dfrac{9}{4}=c,\Big(I\in(P)\Big)\\&\dfrac{9}{4}a-\dfrac{3}{2}b+c=0\Big(A\in(P)\Big)\\&\dfrac{9}{4}a+\dfrac{3}{2}b+c=0\Big(B\in(P)\Big)} \Leftrightarrow \heva{&c=\dfrac{9}{4}\\&a=-1\\& b=0.}$\\
			Vậy $(P)\colon y=-x^2+\dfrac{9}{4}$.
		}{
			\begin{tikzpicture}[line join=round, line cap=round,>=stealth,thick]
				\tikzset{every node/.style={scale=0.9}}
				\begin{scope}
					\clip (-3,-1) rectangle (3,3.5);
					\draw[fill=green!20](-1.5,0)--plot[samples=200,domain=-1.5:1.5,smooth,variable=\x] (\x,{-1*(\x)^2+9/4})--(1.5,0);
					\draw (-1.5,0) circle (1.5pt) node[below]{$A\left(-\frac{3}{2},0\right)$};
					\draw (1.5,0) circle (1.5pt) node[below right]{$B\left(\frac{3}{2},0\right)$};
					\draw (0,2.25) circle (1.5pt) node[above right]{$I\left(\frac{3}{2},0\right)$};
				\end{scope}
				\draw[->] (-3.1,0)--(4.1,0) node[below left] {$x$};
				\draw[->] (0,-1.1)--(0,3.6) node[below left] {$y$};
				\draw (0,0) node [below left] {$O$};
				\foreach \x/\nx in {-1/-1,1/1}
				\draw[thin] (\x,1pt)--(\x,-1pt) node [above] {$\nx$};
				\foreach \y/\ny in {1/1,2/2}
				\draw[thin] (1pt,\y)--(-1pt,\y) node [left] {$\ny$};
			\end{tikzpicture}
		}
		\noindent
		Dựa vào đồ thị, diện tích của parabol là
		\[S=\displaystyle\int\limits_{-\tfrac{3}{2}}^{\tfrac{3}{2}}{\left(-x^2+\dfrac{9}{4}\right)\mathrm{\,d}x}=2\displaystyle\int\limits_0^{\tfrac{3}{2}}{\left(-x^2+\dfrac{9}{4}\right)\mathrm{\,d}x}=2\left(\dfrac{-x^3}{3}+\dfrac{9}{4}x\right)\Bigg|_0^{\tfrac{9}{4}}=\dfrac{9}{2}\,\mathrm{(m^2)}.\]
		Số tiền phải trả là $\dfrac{9}{2}\cdot 1\,500\,000=6\,750\,000$\,(đồng).
	}
\end{ex}

%Câu 2
\begin{ex}%[2D4V3-2]
	\immini{Chị Minh Hiền muốn làm một cái cổng hình Parabol như hình vẽ bên. Chiều cao $GH=4$\,m, chiều rộng $AB=4$\,m, $AC=BD=0{,}9$\,m. Chị Minh Hiền làm hai cánh cổng khi đóng lại là hình chữ nhật $CDEF$ tô đậm có giá là $1\,200\,000$\,đồng/$\mathrm{m^2}$, còn các phần để trắng làm xiên hoa có giá là $900\,000$\,đồng/$\mathrm{m^2}$. Hỏi tổng số tiền để làm hai phần nói trên gần nhất với số tiền nào dưới đây?
		\choice
		{\True $11\,445\,000$\,đồng}
		{$4\,077\,000$\,đồng}
		{$7\,368\,000$\,đồng}
		{$11\,370\,000$\,đồng}
	}{
		\begin{tikzpicture}[line join=round, line cap=round,>=stealth,thick]
			\tikzset{every node/.style={scale=0.9}}
			\begin{scope}
				\clip (-0.1,-0.5) rectangle (4.1,4.5);
				\draw(0,0)--plot[samples=200,domain=0:4,smooth,variable=\x] (\x,{-1*(\x)^2+4*(\x)})--(4,0);
				\draw[fill=black](0,0) circle (1.5pt) node[below]{$A$} (4,0) circle (1.5pt) node[below]{$B$} (2,4) circle (1.5pt) node[above]{$G$} (0.9,2.79) circle (1.5pt) node[left]{$F$} (3.1,2.79) circle (1.5pt) node[right]{$E$} (3.1,0) circle (1.5pt) node[below]{$D$} (0.9,0) circle (1.5pt) node[below]{$C$} (2,0) circle (1.5pt) node[below]{$H$};
				\draw[fill=gray!20](0.9,0)--(0.9,2.79)--(3.1,2.79)--(3.1,0) (0,0)--(4,0);
				\draw[dashed](2,0)--(2,4);
			\end{scope}
		\end{tikzpicture}
	}
	\loigiai{
		\immini{Gắn hệ trục tọa độ Oxy sao cho $AB$ trùng $Ox$, $A$ trùng $O$ khi đó parabol có đỉnh $G(2;4)$ và đi qua gốc tọa độ.\\
			Giả sử phương trình của parabol có dạng $y=ax^2+bx+c$, $(a\ne 0)$.\\
			Vì parabol có đỉnh là $G(2;4)$ và đi qua điểm $O(0;0)$ nên ta có \[\heva{&c=0\\&-\dfrac{b}{2a}=2\\&a\cdot 2^2+b\cdot 2+c=4}\Leftrightarrow\heva{&a=-1\\&b=4\\&c=0.}\]
		}{
			\begin{tikzpicture}[line join=round, line cap=round,>=stealth,thick]
				\tikzset{every node/.style={scale=0.9}}
				\begin{scope}
					\clip (-0.5,-0.5) rectangle (4.5,4.5);
					\draw(0,0)--plot[samples=200,domain=0:4,smooth,variable=\x] (\x,{-1*(\x)^2+4*(\x)})--(4,0);
					\draw plot[samples=200,domain=-0.5:4.5,smooth,variable=\x] (\x,{-1*(\x)^2+4*(\x)});
					\draw[fill=black](0,0) circle (1.5pt) node[below right]{$A$} (4,0) circle (1.5pt) node[below left]{$B$} (2,4) circle (1.5pt) node[above]{$G$} (0.9,2.79) circle (1.5pt) node[left]{$F$} (3.1,2.79) circle (1.5pt) node[right]{$E$} (3.1,0) circle (1.5pt) node[above left]{$D$} (0.9,0) circle (1.5pt) node[above left]{$C$} (2,0) circle (1.5pt) node[above left]{$H$};
					\draw(0.9,0)--(0.9,2.79)--(3.1,2.79)--(3.1,0) (0,0)--(4,0);
					\draw[dashed](2,0)--(2,4)--(0,4);
				\end{scope}
				\draw[->] (-1,0)--(5.1,0) node[below left] {$x$};
				\draw[->] (0,-1)--(0,5.1) node[below left] {$y$};
				\draw (0,0) node [above left] {$O$};
				\foreach \x/\nx in {0.9/0{,}9,2/2,3.1/3{,}1}
				\draw[thin] (\x,1pt)--(\x,-1pt) node [below] {$\nx$};
				\draw[thin] (4,1pt)--(4,-1pt) node [above right] {$4$};
				\foreach \y/\ny in {4/4}
				\draw[thin] (1pt,\y)--(-1pt,\y) node [left] {$\ny$};
			\end{tikzpicture}				
		}
		\noindent
		Suy ra phương trình parabol là $y=f(x)=-x^2+4x$.\\
		Diện tích của cả cổng là $S=\displaystyle\int\limits_0^4\left(-x^2+4x\right)\mathrm{\,d}x=\left(-\dfrac{x^3}{3}+2x^2\right)\Bigg|_0^4=\dfrac{32}{3}\,\mathrm{\left(m^2\right)}$.\\
		Mặt khác chiều cao $CF=DE=f(0{,}9)=2{,}79$\,(m); $CD=4-2\cdot 0{,}9=2{,}2$\,(m).\\
		Diện tích hai cánh cổng là $S_{CDEF}=CD\cdot CF=6{,}138\,\mathrm{(m^2)}$.\\
		Diện tích phần xiên hoa là $S_{xh}=S-S_{CDEF}=\dfrac{32}{3}-6{,}14=\dfrac{6793}{1500}\,\mathrm{(m^2)}$.\\
		Vậy tổng số tiền để làm cổng là $6{,}138\cdot 1\,200\,000+\dfrac{6793}{1500}\cdot 900\,000=11\,441\,400$\,(đồng).
	}
\end{ex}

%Câu 3
\begin{ex}%[2D4V3-2]
	\immini{Một cổng chào có dạng hình Parabol chiều cao $18$\,m, chiều rộng chân đế $12$\,m. Người ta căng hai sợi dây trang trí $AB$, $CD$ nằm ngang đồng thời chia hình giới hạn bởi Parabol và mặt đất thành ba phần có diện tích bằng nhau (xem hình vẽ bên). Tỉ số $\dfrac{AB}{CD}$ bằng
		\choice
		{$\dfrac{1}{\sqrt{2}}$}
		{$\dfrac{4}{5}$}
		{\True $\dfrac{1}{\sqrt[3]{2}}$}
		{$\dfrac{3}{1+2\sqrt{2}}$}
	}{
		\begin{tikzpicture}[line join=round, line cap=round,>=stealth,thick,scale=0.6]
			\tikzset{every node/.style={scale=0.9}}
			\begin{scope}
				\draw(-3,-9)--plot[samples=200,domain=-3:3,smooth,variable=\x] (\x,{-1*(\x)^2})--(3,-9)--cycle;
				\draw[dashed](0,0)--(3.5,0) (3,-9)--(3.5,-9) (-3,-9)--(-3,-9.5) (3,-9)--(3,-9.5);
				\draw[<->](3.5,0)--(3.5,-9) node[pos=0.5, right]{$18$\,m}; \draw[<->](-3,-9.5)--(3,-9.5) node[pos=0.5, below]{$12$\,m};
				\draw[fill=black](-2,-4) circle (1.5pt) node[left]{$A$} (2,-4) circle (1.5pt) node[right]{$B$} (-2.5,-6.25) circle (1.5pt) node[left]{$C$} (2.5,-6.25) circle (1.5pt) node[right]{$D$} (-3,-9) circle (1.5pt) (3,-9) circle (1.5pt);
				\draw (-2,-4)--(2,-4) (-2.5,-6.25)--(2.5,-6.25);
			\end{scope}
		\end{tikzpicture}
	}
	\loigiai{
		\immini{Chọn hệ trục tọa độ $Oxy$ như hình vẽ.
			Phương trình Parabol $(P)$ có dạng $y=ax^2$.\\
			$(P)$ đi qua điểm có tọa độ $(-6;-18)$.\\
			Suy ra $-18=a\cdot (-6)^2\Leftrightarrow a=-\dfrac{1}{2}.\\
			\Rightarrow(P)\colon y=-\dfrac{1}{2}x^2$.\\
			Từ hình vẽ ta có $\dfrac{AB}{CD}=\dfrac{x_1}{x_2}$.\\
			Diện tích hình phẳng giới bạn bởi Parabol và đường thẳng $AB\colon y=-\dfrac{1}{2}x_1^2$ là
			\allowdisplaybreaks
			\begin{eqnarray*}
				S_1&=&2\displaystyle\int\limits_0^{x_1}{\left[-\dfrac{1}{2}{x^2}-\left(-\dfrac{1}{2}x_1^2\right)\right]\mathrm{\,d}x}\\
				&=&2\left(-\dfrac{1}{2}\cdot \dfrac{x^3}{3}+\dfrac{1}{2}x_1^2x\right)\Bigg|_0^{x_1}=\dfrac{2}{3}x_1^3.
			\end{eqnarray*}
		}{
			\begin{tikzpicture}[line join=round, line cap=round,>=stealth,thick,scale=0.7]
				\tikzset{every node/.style={scale=0.8}}
				\begin{scope}
					\draw(-3,-9)--plot[samples=200,domain=-3:3,smooth,variable=\x] (\x,{-1*(\x)^2})--(3,-9)--cycle;
					\draw[dashed] (3,-9)--(3.5,-9) (-3,-9)--(-3,-9.5) (3,-9)--(3,-9.5);
					\draw[<->](3.5,0)--(3.5,-9) node[pos=0.5, right]{$18$\,m}; \draw[<->](-3,-9.5)--(3,-9.5) node[pos=0.7, below]{$12$\,m};
					\draw[fill=black](-2,-4) circle (1.5pt) node[left]{$A$} (2,-4) circle (1.5pt) node[right]{$B$} (-2.5,-6.25) circle (1.5pt) node[left]{$C$} (2.5,-6.25) circle (1.5pt) node[right]{$D$} (-3,-9) circle (1.5pt) (3,-9) circle (1.5pt);
					\draw (-2,-4)--(2,-4) (-2.5,-6.25)--(2.5,-6.25);
				\end{scope}
				\draw[->] (-4,0)--(4,0) node[below left] {$x$};
				\draw[->] (0,-10)--(0,2.0) node[below left] {$y$};
				\draw (0,0) node [above left] {$O$};
				\foreach \x/\nx in {-3/-6,2/x_1,2.5/x_2}
				\draw[thin] (\x,1pt)--(\x,-1pt) node [above] {$\nx$};
				\draw[thin] (1pt,-9)--(-1pt,-9) node [above left] {$-18$};
				\draw[dashed](-3,0)--(-3,-9) (2,0)--(2,-4) (2.5,0)--(2.5,-6.25);
			\end{tikzpicture}
		}
		\noindent
		Diện tích hình phẳng giới hạn bởi Parabol và đường thẳng $CD\colon y=-\dfrac{1}{2}x_2^2$ là
		\[S_2=2\displaystyle\int\limits_0^{x_2}{\left[-\dfrac{1}{2}{x^2}-\left(-\dfrac{1}{2}x_2^2\right)\right]\mathrm{\,d}x}=2\left(-\dfrac{1}{2}\cdot \dfrac{x^3}{3}+\dfrac{1}{2}x_2^2x\right)\Bigg|_0^{x_2}=\dfrac{2}{3}x_2^3.\]
		Từ giả thiết suy ra $S_2=2S_1\Leftrightarrow x_2^3=2x_1^3\Leftrightarrow\dfrac{x_1}{x_2}=\dfrac{1}{\sqrt[3]{2}}$.\\
		Vậy $\dfrac{AB}{CD}=\dfrac{x_1}{x_2}=\dfrac{1}{\sqrt[3]{2}}$.
	}
\end{ex}

%Câu 4
\begin{ex}%[2D4C3-2]
	\immini{Một họa tiết hình cánh bướm như hình vẽ bên. Phần tô đậm được đính đá với giá thành $500\,000$/$\,\mathrm{m^2}$. Phần còn lại được tô màu với giá thành $250\,000$/$\,\mathrm{m^2}$. Cho $AB=4$\,dm; $BC=8$\,dm. Hỏi để trang trí $1\,000$ họa tiết như vậy cần số tiền gần nhất với số nào sau đây.
		\choice
		{$105\,660\,667$}
		{\True $106\,666\,667$}
		{$ 107\,665\,667$}
		{$ 108\,665\,667$}
	}{
		\begin{tikzpicture}[line join=round, line cap=round,>=stealth,thick,scale=0.5]
			\tikzset{every node/.style={scale=0.9}}
			\begin{scope}
				\draw[fill=gray!35](-2,0)--plot[samples=200,domain=-2:2,smooth,variable=\x] (\x,{(\x)^2})--(2,0);
				\draw[fill=gray!35](-2,0)--plot[samples=200,domain=-2:2,smooth,variable=\x] (\x,{-1*(\x)^2})--(2,0);
				\draw[fill=black](-2,4) circle (1.5pt) node[left]{$A$} (2,4) circle (1.5pt) node[right]{$B$} (2,-4) circle (1.5pt) node[right]{$C$} (-2,-4) circle (1.5pt) node[left]{$D$};
				\draw (-2,4)--(2,4) (2,-4)--(-2,-4);
			\end{scope}
			\draw[->] (-3,0)--(3,0) node[below left] {$x$};
			\draw[->] (0,-5)--(0,5) node[below left] {$y$};
		\end{tikzpicture}
	}
	\loigiai{
		Vì $AB=4$\,dm; $BC=8$\,dm $\Rightarrow A(-2;4)$, $B(2;4)$, $C(2;-4)$, $D(-2;-4)$.\\
		parabol là $y=x^2$ hoặc $y=-x^2$.\\
		Diện tích phần tô đậm là $S_1=4\displaystyle\int\limits_0^2{x^2}\mathrm{\,d}x=\dfrac{32}{3}\mathrm{\,(dm^2)}$.\\
		Diện tích hình chữ nhật là $S=4\cdot 8=32\mathrm{\,(dm^2)}$.\\
		Diện tích phần trắng là $S_2=S-S_1=32-\dfrac{32}{3}=\dfrac{64}{3}\mathrm{\,(dm^2)}$.\\
		Tổng chi phí trang chí là $T=\left(\dfrac{32}{3}\cdot 5\,000+\dfrac{64}{3}\cdot 2\,500\right)\cdot 1\,000\approx 106\,666\,667$.
	}
\end{ex}
%%==========Câu 5
% \begin{ex}%[2D4C3-2]
% 	\immini{Một hoa văn trang trí được tạo ra từ một miếng bìa mỏng hình vuông cạnh bằng $10$ cm bằng cách khoét đi bốn phần bằng nhau có hình dạng parabol như hình bên. Biết $AB=5$ cm, $OH=4$ cm. Biết giá trang trí hoa văn $1$ cm$^2$ là $50000$ đồng, tính số tiền cần bỏ ra để trang trí hoa văn đó.
% 		\choice
% 		{$2\,553\,333$ đồng}
% 		{\True $2\,333\,333$ đồng}
% 		{$2\,780\,333$ đồng}
% 		{$2\,123\,333$ đồng}
% 	}{
% 		\begin{tikzpicture}[line join=round, line cap=round,>=stealth,thick,scale=0.5]
% 			\tikzset{every node/.style={scale=0.9}}
% 			\begin{scope}
% 				\fill[black] (0,0)--(10,0)--(10,10)--(0,10)--cycle;
% 				\fill[white](2.5,0)--plot[samples=200,domain=2.5:7.5,smooth,variable=\x] (\x,{-16/25*(\x-2.5)^2+16/5*(\x-2.5)})--(7.5,0);
% 				\fill[white](2.5,10)--plot[samples=200,domain=2.5:7.5,smooth,variable=\x] (\x,{16/25*(\x-2.5)^2-16/5*(\x-2.5)+10})--(7.5,10);
% 				\fill[white] plot[samples=200,domain=6:10,smooth,variable=\x] (\x,{sqrt(1.5625*((\x)-6))+5})--(10,5)--plot[samples=200,domain=6:10,smooth,variable=\x] (\x,{-sqrt(1.5625*((\x)-6))+5})--(10,5);
% 				\fill[white] plot[samples=200,domain=4:0,smooth,variable=\x] (\x,{sqrt(1.5625*(4-(\x)))+5})--(0,5)--plot[samples=200,domain=4:0,smooth,variable=\x] (\x,{-sqrt(1.5625*(4-(\x)))+5})--(0,5);				
% 			\end{scope}
% 			\draw[fill=white](6,5) circle (1.5pt) node[above right]{$O$} (10,2.5) circle (1.5pt) node[right]{$B$} (10,7.5) circle (1.5pt) node[right]{$A$} (10,5) circle (1.5pt) node[right]{$H$};
% 			\draw[dashed] (6,5)--(10,5) (10,2.5)--(10,7.5);
% 		\end{tikzpicture}
% 	}
% 	\loigiai{
% 		\immini{Đưa parabol vào hệ trục $Oxy$ ta tìm được phương trình là $(P)\colon y=-\dfrac{16}{25}x^2+\dfrac{16}{5}x$.\\
% 			Diện tích hình phẳng giới hạn bởi $(P)\colon y=-\dfrac{16}{25}x^2+\dfrac{16}{5}x$, trục hoành và các đường thẳng $x=0$, $x=5$ là
% 			\[S=\displaystyle\int\limits_0^5 \left(-\dfrac{16}{25}x^2+\dfrac{16}{5}x\right)\mathrm{\,d}x=\dfrac{40}{3}.\]
			
% 		}
% 		{\begin{tikzpicture}[line join=round, line cap=round,>=stealth,thick,scale=0.7]
% 				\tikzset{every node/.style={scale=0.8}}
% 				\draw[step=0.5, gray!50,very thin] (0,0) grid (5.5,4.5);
% 				\draw[->] (-0.5,0)--(6.1,0) node[below left] {$x$};
% 				\draw[->] (0,-0.5)--(0,5.1) node[below left] {$y$};
% 				\foreach \x/\nx in {1/1,2/2,3/3,4/4,5/5}
% 				\draw[thin] (\x,1pt)--(\x,-1pt) node [below] {$\nx$};
% 				\foreach \y/\ny in {1/1,2/2,3/3,4/4}
% 				\draw[thin] (1pt,\y)--(-1pt,\y) node [left] {$\ny$};
% 				\draw (0,0) node [below left] {$O$};
% 				\begin{scope}
% 					\clip (-0.5,-0.5) rectangle (5.5,5);
% 					\draw[samples=200,domain=-0.5:5.5,smooth,variable=\x] plot (\x,{-0.64*(\x)^2+3.2*(\x)+0});
% 				\end{scope}
% 				\draw[fill=black](0,0) circle (1.5pt) (2.5,4) circle (1.5pt) (5,0) circle (1.5pt);
% 			\end{tikzpicture}
% 		}\noindent
% 		Tổng diện tích phần bị khoét đi $S_1=4S=\dfrac{160}{3}$ cm$^2$.\\
% 		Diện tích của hình vuông là $S_{hv}=100$ cm$^2$.\\
% 		diện tích bề mặt hoa văn là $S_2=S_{hv}-S_1=100-\dfrac{160}{3}=\dfrac{140}{3}\mathrm{\,(cm^2)}$.\\
% 		Vậy số tiền cần bỏ ra để trang trí hoa văn đó là $\dfrac{140}{3}\cdot 50\,000\approx 2\,333\,333$ (đồng).
% 	}
% \end{ex}
%%==========Câu 6
\begin{ex}%[2D4V3-2]
	\immini{Một viên gạch hoa hình vuông cạnh $40$ cm. Người thiết kế đã sử dụng bốn đường parabol có chung đỉnh tại tâm viên gạch để tạo ra bốn cánh hoa (được tô đen như hình vẽ dưới). Diện tích mỗi cánh hoa của viên gạch bằng
		\choice
		{$800$ cm$^2$} 
		{$\dfrac{800}{3}$ cm$^2$} 
		{\True $\dfrac{400}{3}$ cm$^2$} 
		{$250$ cm$^2$}
	}
	{\begin{tikzpicture}[scale=1.2,>=stealth, line join = round, line cap = round,font=\footnotesize]
			\draw (-1,1) rectangle (1,-1);
			\foreach \i in {0,90,180,270}
			\draw [fill=gray, smooth, samples=100,rotate=\i] plot [domain=0:1] (\x, {(\x)^2})-- plot [domain=1:0] (\x, {sqrt(\x)});
			\draw [<-] (-1,-1.5)--(0,-1.5) node[below] {$40$ cm};
			\draw [->] (0,-1.5)--(1,-1.5);
		\end{tikzpicture}
	} 
	\loigiai{
		\immini{Chọn hệ tọa độ như hình vẽ (1 đơn vị trên trục bằng $10$ cm= $1$ dm), các cánh hoa tạo bởi các đường parabol có phương trình $ y=\dfrac{x^2}{2}$, $y=-\dfrac{x^2}{2}$, $x=-\dfrac{y^2}{2}$, $x=\dfrac{y^2}{2}$.\\ 
			Diện tích một cánh hoa (nằm trong góc phàn tư thứ nhất) bằng diện tích hình phẳng giới hạn bởi hai đồ thị hàm số $y=\dfrac{x^2}{2}$, $y=\sqrt{2x}$ và hai đường thẳng $ x=0$; $x=2$.
		}
		{\begin{tikzpicture}[scale=1.3,>=stealth, line join = round, line cap = round,font=\footnotesize]
				\draw[->] (-1.5,0)--(0,0)%
				node[below right]{$O$}--(1.5,0) node[below]{$x$};
				\draw[->] (0,-1.5) --(0,1.5) node[right]{$y$};
				\draw (-1,1) rectangle (1,-1);
				\foreach \i in {-1,1}
				\foreach \j in {-1,1}
				\draw[fill=black]  (\i,0) circle (0.5 pt) node [below] {\footnotesize $\i$};
				\draw [fill=gray, smooth, samples=100] plot [domain=0:1] (\x, {(\x)^2})-- plot [domain=1:0] (\x, {sqrt(\x)});
			\end{tikzpicture}
		}\noindent
		Do đó diện tích một cánh hoa bằng 
		\[\displaystyle\int\limits_0^2 \left( \sqrt{2x}-\dfrac{x^2}{2} \right) \mathrm{d}x = \left(\dfrac{2\sqrt{2}}{3}\sqrt{(2x)^3}-\dfrac{x^3}{6} \right) \Big|_0^2 = \dfrac{4}{3} \;\;\text{dm}^2= \dfrac{400}{3}\;\;\text{cm}^2.\]
	}
\end{ex} 
\Closesolutionfile{ans}
\indapan{6}{ans/ans-2-C4B3CD3_1-4-lc}

% \begin{dang}
% 	{Ứng dụng thể tích khối tròn xoay trong bài toán thực tiễn}
% \end{dang}
\Opensolutionfile{ans}[ans/ans-2-C4B3CD3_1-2-lc]
%%==========Câu 7
% \begin{ex}%[2D4V3-4]
% 	\immini{Khi cắt một vật thể hình chiếc niêm bởi mặt phẳng vuông góc với trục $Ox$ tại điểm có hoành độ $x$ ($-2 \le x\le 2$), mặt cắt là tam giác vuông có một góc $45^\circ$ và độ dài một cạnh góc vuông là $\sqrt{14-3x^2}$ (như hình vẽ). Tính thể tích vật thể hình chiếc niêm trên.
% 		\choice
% 		{\True $V=20$}
% 		{$V=20\pi$}
% 		{$V=10$}
% 		{$V=10\pi$}
% 	}
% 	{\begin{tikzpicture}[declare function={r=4;d=3;},scale=0.6]
% 			\path (0,0) coordinate (O)--++(85:r/3) coordinate (A)
% 			($(A)!2!(O)$) coordinate (B)
% 			($(O)+(0:r)$) coordinate (C)
% 			($(C)+(90:d)$) coordinate (D)
% 			;
% 			\draw (A)..controls +(40:1.5) and +(90:1)..(D)..controls +(-90:1) and +(40:1.5)..(B);
% 			\draw (A)--(O)node[midway,left]{} (O)--(B)node[midway,left]{} (C)--(D)
% 			(C)..controls +(-90:1) and +(0:2)..(B)
% 			;
% 			\draw[dashed] 
% 			(A)..controls +(0:2) and +(90:1)..(C)
% 			(D)--(O) (O)--(C)node[pos=0.7,above]{} ;
% 			\path pic[draw,angle radius=17pt,"$\alpha$"]{angle= C--O--D};
% 		\end{tikzpicture}
% 	}
% 	\loigiai{
% 		Diện tích tam giác vuông cân là $S(x)=\dfrac{1}{2}\sqrt{14-3x^2} \cdot \sqrt{14-3x^2} = \dfrac{1}{2}(14-3x^2)$.\\
% 		Vậy thể tích vật thể là $\displaystyle\int\limits_{-2}^2 \dfrac{1}{2}(14-3x^2)\mathrm{\,d}x=20$.
% 	}
% \end{ex}
%%==========Câu 8
% \begin{ex}%[2D4V3-4]
% 	\immini{Trong chương trình nông thôn mới của tỉnh Phú Yên, tại xã Hòa Mỹ Tây có xây một cây cầu bằng bê tông như hình vẽ (đường cong trong hình vẽ là các đường Parabol). Biết $1$ m$^3$ khối bê tông để đổ cây cầu có giá 5 triệu đồng. Tính số tiền mà tỉnh Phú Yên cần bỏ ra để xây cây cầu trên.
% 	}
% 	{\begin{tikzpicture}[scale=1.0, font=\footnotesize, line join=round, line cap=round,>=stealth,samples=100]
% 			\path 
% 			(0,0) coordinate (O)
% 			(-1,0) coordinate (B)
% 			(0,1) coordinate (C)
% 			(1,0) coordinate (D)
% 			(-1.4,0) coordinate (M)
% 			(0,1.96) coordinate (N)
% 			(1.4,0) coordinate (P)
% 			;
% 			\path (-165:4) coordinate (A);
% 			\foreach \x in {B,C,D,M,N,P,O}{\path ($(\x)+(-165:4)$) coordinate (\x_1);}
% 			\fill[gray!40] (N_1)--(N)--plot[domain=0:1.4] (\x,{-(\x)^2+1.96})--(P)--(P_1)--plot[shift={(A)},domain=1.4:0] (\x,{-(\x)^2+1.96});
% 			\fill[gray!40] (M_1)--plot[shift={(A)},domain=-1.4:1.4](\x,{-(\x)^2+1.96})--(P_1)--(D_1)--plot[shift={(A)},domain=1:-1] (\x,{-(\x)^2+1})--(B_1)--(M_1);
% 			\draw[dash pattern=on 2pt off 2pt] plot[domain=-1:1] (\x,{-(\x)^2+1})
% 			plot[domain=-1.4:1.4] (\x,{-(\x)^2+1.96})
% 			(M)--(P) (O)--(N)
% 			;
% 			\draw[shift={(A)}] plot[domain=-1:1] (\x,{-(\x)^2+1});
% 			\draw[shift={(A)}] plot[domain=-1.4:1.4] (\x,{-(\x)^2+1.96});
% 			\draw [dash pattern=on 2pt off 2pt] (B)--(B_1) (C)--(C_1) (D)--(D_1) (M)--(M_1)
% 			(O_1)--(N_1)
% 			;
% 			\draw (N)--(N_1) (P)--(P_1) (M_1)--(P_1);
% 			\foreach \t in {O,B,M,P,O_1,M_1,B_1,D_1,P_1}{
% 				\draw[fill=white] (\t) circle (1pt);}
% 			\path (M_1)--(B_1) node[midway,below]{$0,5$m};
% 			\path (B_1)--(D_1) node[midway,below]{$19$m};
% 			\path (D_1)--(P_1) node[midway,below]{$0,5$m};
% 			\path (O)--(C) node[midway,right]{$2$m};
% 			\path (C)--(N) node[pos=0.2,left]{$0,5$m};
% 		\end{tikzpicture}
% 	}
% 	\choice
% 	{$110$ triệu đồng}
% 	{$250$ triệu đồng}
% 	{$180$ triệu đồng}
% 	{\True $200$ triệu đồng}
% 	\loigiai{
% 		\immini{Chọn hệ trục $Oxy$ như hình vẽ.\\
% 			Gọi $(P_1)\colon y=a_1 x^2+b_1$ là Parabol đi qua hai điểm $A\left(\dfrac{19}{2};0\right)$, $B(0;2)$.\\
% 		}
% 		{\begin{tikzpicture}[scale=0.9, font=\footnotesize, line join=round, line cap=round,>=stealth,samples=100]
% 				\path 
% 				(0,0) coordinate (O)
% 				(-1,0) coordinate (B)
% 				(0,1) coordinate (C)
% 				(1,0) coordinate (D)
% 				(-1.4,0) coordinate (M)
% 				(0,1.96) coordinate (N)
% 				(1.4,0) coordinate (P)
% 				;
% 				\path (-165:4) coordinate (A);
% 				\foreach \x in {B,C,D,M,N,P,O}{\path ($(\x)+(-165:4)$) coordinate (\x_1);}
% 				\fill[gray!40] (N_1)--(N)--plot[domain=0:1.4] (\x,{-(\x)^2+1.96})--(P)--(P_1)--plot[shift={(A)},domain=1.4:0] (\x,{-(\x)^2+1.96});
% 				\fill[gray!40] (M_1)--plot[shift={(A)},domain=-1.4:1.4](\x,{-(\x)^2+1.96})--(P_1)--(D_1)--plot[shift={(A)},domain=1:-1] (\x,{-(\x)^2+1})--(B_1)--(M_1);
% 				\draw[-stealth,dash pattern=on 2pt off 2pt] (-1.4,0)--(0,0)node[below left]{$O$}--(2,0) node[below] {$x$};
% 				\draw[-stealth,dash pattern=on 2pt off 2pt] (0,0)--(0,2.5) node[left] {$y$};
% 				\draw[dash pattern=on 2pt off 2pt] plot[domain=-1:1] (\x,{-(\x)^2+1})
% 				plot[domain=-1.4:1.4] (\x,{-(\x)^2+1.96})
% 				;
% 				\draw[shift={(A)}] plot[domain=-1:1] (\x,{-(\x)^2+1});
% 				\draw[shift={(A)}] plot[domain=-1.4:1.4] (\x,{-(\x)^2+1.96});
% 				\draw [dash pattern=on 2pt off 2pt] (B)--(B_1) (C)--(C_1) (D)--(D_1) (M)--(M_1)
% 				(O_1)--(N_1)
% 				;
% 				\draw (N)--(N_1) (P)--(P_1) (M_1)--(P_1);
% 				\foreach \t in {O,B,M,P,O_1,M_1,B_1,D_1,P_1}{
% 					\draw[fill=white] (\t) circle (1pt);}
% 				\path (M_1)--(B_1) node[midway,below]{$0,5$m};
% 				\path (B_1)--(D_1) node[midway,below]{$19$m};
% 				\path (D_1)--(P_1) node[midway,below]{$0,5$m};
% 				\path (O)--(C) node[midway,right]{$2$m};
% 				\path (C)--(N) node[pos=0.2,left]{$0,5$m};
% 			\end{tikzpicture}
% 		}\noindent
% 		Nên ta có hệ phương trình sau
% 		\[\heva{& 0=a\cdot \left(\dfrac{19}{2}\right)^2+2 \\ & 2=b} \Leftrightarrow \heva{& a_1=-\dfrac{8}{361} \\ & b_1=2 }\Rightarrow (P_1)\colon y=-\dfrac{8}{361}{x^2}+2.\]
% 		Gọi $(P_2)\colon y=a_2 x^2+b_2$ là Parabol đi qua hai điểm $C(10;0)$, $D\left(0;\dfrac{5}{2}\right)$.\\
% 		Nên ta có hệ phương trình sau
% 		\[\heva{& 0=a_2 \cdot 10^2 + \dfrac{5}{2} \\ & \dfrac{5}{2}=b_2 } \Leftrightarrow \heva{& a_2=-\dfrac{1}{40} \\ & b_2=\dfrac{5}{2} } \Rightarrow (P_2)\colon y=-\dfrac{1}{40}{x^2}+\dfrac{5}{2}.\]
% 		Ta có thể tích của bê tông là
% 		\[V=5\cdot 2\left[\displaystyle\int\limits_0^{10}\left(-\dfrac{1}{40}{x^2}+\dfrac{5}{2} \right)\mathrm{\,d}x-\displaystyle\int\limits_0^{\tfrac{19}{2}}\left( -\dfrac{8}{361}x^2+2\right)\mathrm{\,d}x \right]=40\;\;\text{m}^3.\]
% 		Số tiền mà tỉnh Phú Yên cần bỏ ra để xây cây cầu là $5\cdot 40=200$ triệu đồng.
% 	}
% \end{ex}
%%==========Câu 9
\begin{ex}%[2D4V3-4]
	Để kỷ niệm ngày 26-3. Chi đoàn 12A dự định dựng một lều trại có dạng parabol, với kích thước: nền trại là một hình chữ nhật có chiều rộng là $3$ mét, chiều sâu là $6$ mét, đỉnh của parabol cách mặt đất là $3$ mét. Hãy tính thể tích phần không gian phía bên trong trại để lớp 12A cử số lượng người tham dự trại cho phù hợp.
	\choice
	{$30$ m$^3$}
	{\True $36$ m$^3$}
	{$40$ m$^3$}
	{$41$ m$^3$}
	\loigiai{
		Giả sử nền trại là hình chữ nhật $ABCD$ có $AB = 3$ m, $BC = 6$ m, đỉnh của parabol là $I$.\\
		Chọn hệ trục tọa độ $Oxy$ sao cho $O$ là trung điểm của cạnh $AB$, $A$, $B$ và $I$, phương trình của parabol có dạng $y=ax^2+b$, $a \ne 0$.\\
		Do $I$, $A$, $B$ thuộc nên ta có $y=-\dfrac{4}{3}x^2+3$.\\
		Vậy thể tích phần không gian phía trong trại là
		\[V=6 \cdot 2\displaystyle\int\limits_0^{\tfrac{3}{2}}\left(-\dfrac{4}{3}x^2+3\right)\mathrm{\,d}x=36.\]
	}
\end{ex}
%%==========Câu 10
\begin{ex}%[2D4V3-4]
	Cho một vật thể bằng gỗ có dạng hình trụ với chiều cao và bán kính đáy cùng bằng $R$. Cắt khối gỗ đó bởi một mặt phẳng đi qua đường kính của một mặt đáy của khối gỗ và tạo với mặt phẳng đáy của khối gỗ một góc $30^\circ$ ta thu được hai khối gỗ có thể tích là $V_1$ và $V_2$, với $V_1<V_2$. Thể tích $V_1$ bằng
	\choice
	{\True $V_1=\dfrac{2\sqrt{3}R^3}{9}$}
	{$V_1=\dfrac{\sqrt{3}\pi R^3}{27}$}
	{$V_1=\dfrac{\sqrt{3}\pi R^3}{18}$}
	{$V_1=\dfrac{\sqrt{3}R^3}{27}$}
	\loigiai{
		\immini{Khi cắt khối gỗ hình trụ ta được một hình nêm có thể tích $V_1$ như hình vẽ.\\
			Chọn hệ trục tọa độ $Oxy$ như hình vẽ.\\
			Nửa đường tròn đường kính $AB$ có phương trình là
			\[y=\sqrt{R^2-x^2}, x \in [-R;R].\]
		}
		{\begin{tikzpicture}[declare function={r=4;d=3;},scale=0.7]
				\path (0,0) coordinate (O)--++(85:r/3) coordinate (A)
				($(A)!2!(O)$) coordinate (B)
				($(O)+(0:r)$) coordinate (C)
				($(C)+(90:d)$) coordinate (D)
				;
				\draw (A)..controls +(40:1.5) and +(90:1)..(D)..controls +(-90:1) and +(40:1.5)..(B);
				\draw (A)--(O)node[midway,left]{$R$} (O)--(B)node[midway,left]{$R$} (C)--(D)
				(C)..controls +(-90:1) and +(0:2)..(B)
				;
				\draw[dashed] 
				(A)..controls +(0:2) and +(90:1)..(C)
				(D)--(O) (O)--(C)node[pos=0.7,above]{$R$} ;
				\path pic[draw,angle radius=17pt,"$\alpha$"]{angle= C--O--D};
			\end{tikzpicture}
		}\noindent
		Một mặt phẳng vuông góc với trục $Ox$ tại điểm $M$ có hoành độ $x$, cắt hình nêm theo thiết diện là $\triangle MNP$ vuông tại $N$ và có $\widehat{PMN}=30^\circ$.\\
		Ta có $NM=y=\sqrt{R^2-x^2} \Rightarrow NP=MN \cdot \tan 30^\circ = \dfrac{\sqrt{R^2-x^2}}{\sqrt{3}}$.\\
		Do $\triangle MNP$ có diện tích $S(x)=\dfrac{1}{2}NM \cdot NP =\dfrac{1}{2}\cdot \dfrac{R^2-x^2}{\sqrt{3}}$.\\
		Thể tích hình nêm là 
		\[V_1=\displaystyle\int\limits_{-R}^R S(x)\mathrm{\,d}x=\dfrac{1}{2}\displaystyle\int\limits_{-R}^R \dfrac{R^2-x^2}{\sqrt{3}}\mathrm{\,d}x=\dfrac{1}{2\sqrt{3}}\left(R^2 x-\dfrac{1}{3}x^3 \right) \Big|_{-R}^R=\dfrac{2\sqrt{3}{R^3}}{9}.\]
		\textbf{Chú ý:} Có thể ghi nhớ công thức tính thể tích hình nêm
		$V_1=\dfrac{2}{3}{R^2}h=\dfrac{2}{3}{R^3}\tan \alpha $, trong đó $R=\dfrac{AB}{2}$, $\alpha =\widehat{PMN}$.
	}
\end{ex}
%%==========Câu 11
% \begin{ex}%[2D4V3-4]
% 	\immini{Cho một mô hình $3-D$ mô phỏng một đường hầm như hình vẽ bên. Biết rằng đường hầm mô hình có chiều dài $5$ cm; khi cắt hình này bởi mặt phẳng vuông góc với đấy của nó, ta được thiết diện là một hình parabol có độ dài đáy gấp đôi chiều cao parabol. Chiều cao của mỗi thiết diện parobol cho bởi công thức $y=3-\dfrac{2}{5}x$ cm, với $x$ cm là khoảng cách tính từ lối vào lớn hơn của đường hầm mô hình. Tính thể tích (theo đơn vị cm$^3$) không gian bên trong đường hầm mô hình (làm tròn kết quả đến hàng đơn vị).
% 		\choice
% 		{\True $29$}
% 		{$27$}
% 		{$31$}
% 		{$33$}
% 	}
% 	{\begin{tikzpicture}[scale=0.6,declare function={a=0.8;b=0.6;c=0.4;d=0.2;}]
% 			\tikzset{
% 				homothety at/.style args={#1 scaled by #2}{shift={($(#1)!#2!(0,0)$)},scale=#2},
% 			}
% 			\def\mypath{(-120:2)..controls +(90:0.6) and +(-180:0.6)..(0,3)}
% 			\def\mydot{(0,3)..controls +(0:0.25) and +(95:0.05)..(60:2)}
% 			\draw \mypath;
% 			\draw[dashed] \mydot;
% 			\path (7,0) coordinate (c1);
% 			\begin{scope}[homothety at=c1 scaled by a]
% 				\draw \mypath;
% 				\draw[dashed] \mydot;
% 			\end{scope}
% 			\begin{scope}[homothety at=c1 scaled by b]
% 				\draw \mypath;
% 				\draw[dashed] \mydot;
% 			\end{scope}
% 			\begin{scope}[homothety at=c1 scaled by c]
% 				\draw \mypath;
% 				\draw[dashed] \mydot;
% 			\end{scope}
% 			\begin{scope}[homothety at=c1 scaled by d]
% 				\draw \mypath;
% 				\draw \mydot;
% 			\end{scope}
% 			\path 
% 			(-120:2) coordinate (A)
% 			(0,3) coordinate (B)
% 			(60:2) coordinate (C)
% 			(0,0) coordinate (O);
% 			\foreach \x in {A,B,C}{\path ($(c1)!a!(\x)$) coordinate (\x_1);}
% 			\foreach \x in {A,B,C}{\path ($(c1)!b!(\x)$) coordinate (\x_2);}
% 			\foreach \x in {A,B,C}{\path ($(c1)!c!(\x)$) coordinate (\x_3);}
% 			\foreach \x in {A,B,C}{\path ($(c1)!d!(\x)$) coordinate (\x_4);}
% 			\path ($(A_4)!0.5!(B_4)$) coordinate (D);
% 			\draw (A)--(A_4) (B)--(B_4) (A_4)--(C_4)
% 			;
% 			\draw[dashed] (B)node[above]{$3$}--(O)--(D)node[below right]{$5$} (A)--(C) (A_1)--(C_1) (A_2)--(C_2) (A_3)--(C_3)  (C)--(C_4);
% 		\end{tikzpicture}
% 	}
% 	\loigiai{
% 		\immini{Xét một thiết diện Parabol có chiều cao là $h$ và độ dài đáy $2h$ và chọn hệ trục $Oxy$ như hình vẽ trên.\\
% 			Parabol $(P)$ có phương trình $(P)\colon y=ax^2+h$, ($a<0$).\\
% 			Có $B(h;0)\in (P) \Leftrightarrow 0=ah^2+h \Leftrightarrow a=-\dfrac{1}{h}$ (do $h>0$).\\
% 			Diện tích $S$ của thiết diện là
% 			\[S=\displaystyle\int\limits_{-h}^h \left( -\dfrac{1}{h}x^2+h\right)\mathrm{\,d}x=\dfrac{4h^2}{3}, h=3-\dfrac{2}{5}x \Rightarrow S(x)=\dfrac{4}{3}\left(3-\dfrac{2}{5}x\right)^2.\]
% 		}
% 		{\begin{tikzpicture}[scale=0.8,font=\footnotesize]
% 				\path (0,0) coordinate (O)
% 				(2,0) coordinate (A)
% 				(0,2) coordinate (B)
% 				;
% 				\draw[-stealth] (-3.5,0)--(0,0)--(3,0)node[below]{$x$};
% 				\draw[-stealth] (0,-1.5)--(0,4)node[left]{$y$};
% 				\draw[smooth,samples=100] plot[domain=-2:2](\x,{(-1/2)*(\x)^2+2});
% 				\foreach \x in {O,A,B}{\draw[fill=blue!40] (\x) circle (1pt);}
% 				\foreach \x in {-3,-2,-1,1}{\draw (\x,0.05)--(\x,-0.05);}
% 				\foreach \x in {-1,1,3}{\draw (-0.05,\x)--(0.05,\x);}
% 				\node[above left] at (B) {$h$};
% 				\path (O)--(A)node[below]{$h$};
% 				\node at (0,0) [below left]{$O$};
% 			\end{tikzpicture}
% 		}\noindent
% 		Suy ra thể tích không gian bên trong của đường hầm mô hình là
% 		\[V=\displaystyle\int\limits_0^5 S(x)\mathrm{\,d}x=\displaystyle\int\limits_0^5\dfrac{4}{3}\left(3-\dfrac{2}{5}x\right)^2\mathrm{\,d}x\approx 28{,}888 \Rightarrow V\approx 29\;\;\text{cm}^3.\]
% 	}
% \end{ex}
%%==========Câu 12
\begin{ex}%[2D4V3-3]
	\immini{Chuẩn bị cho đêm hội diễn văn nghệ chào đón năm mới, bạn Minh Hiền đã làm một chiếc mũ “cách điệu” cho ông già Noel có dáng một khối tròn xoay. Mặt cắt qua trục của chiếc mũ như hình vẽ bên dưới. Biết rằng $OO'=5$ cm, $OA=10$ cm, $OB=20$ cm, đường cong $AB$ là một phần của parabol có đỉnh là điểm $A$. Thể tích của chiếc mũ bằng
		\choice
		{$\dfrac{2750\pi}{3}$ cm$^3$}
		{\True $\dfrac{2500\pi}{3}$ cm$^3$}
		{$\dfrac{2050\pi}{3}$ cm$^3$}
		{$\dfrac{2250\pi}{3}$ cm$^3$}
	}
	{\begin{tikzpicture}[line join=round, line cap=round,>=stealth,scale=0.2]
			\path (0,0) coordinate (O)
			(10,0) coordinate (A)
			(0,20) coordinate (B)
			(0,-5) coordinate (O')
			(10,-5) coordinate (I)
			(-10,-5) coordinate (C)
			(-10,0) coordinate (D)
			;
			\draw[smooth,samples=100,thick] plot[domain=0:10](\x,{(1/5)*((\x)-10)^2})
			plot[domain=-10:0](\x,{(1/5)*((\x)+10)^2})
			;
			\draw[dash pattern=on 2pt off 2pt] (O')--(O)--(B) (A)--(D);
			\draw[thick] (A)--(I)--(C)--(D);
			\foreach \t/\g in {A/90,B/45,O/-135,O'/-135}{
				\draw[fill=black] (\t) circle (1pt) node[shift={(\g:7pt)},font=\scriptsize]{$ \t $};
			}
		\end{tikzpicture}
	}
	\loigiai{
		\immini{Ta gọi thể tích của chiếc mũ là $V$.\\
			Thể tích của khối trụ có bán kính đáy bằng $OA=10$ cm và đường cao $OO'=5$ cm là $V_1$.\\
			Thể tích của vật thể tròn xoay khi quay hình phẳng giới hạn bởi đường cong $AB$ và hai trục tọa độ quanh trục $Oy$ là $V_2$.\\
			Ta có $V=V_1+V_2$; $V_1=5\cdot 10^2\pi =500\pi $ cm$^3$.\\
			Chọn hệ trục tọa độ như hình vẽ.\\
			Do parabol có đỉnh $A$ nên nó có phương trình dạng $(P)\colon y=a(x-10)^2$.
		}
		{\begin{tikzpicture}[line join=round, line cap=round,>=stealth,font=\scriptsize,scale=0.2]
				\path (0,0) coordinate (O)
				(10,0) coordinate (A)
				(0,20) coordinate (B)
				(0,-5) coordinate (O')
				(-10,0) coordinate (C)
				;
				\draw[-stealth] (-12,0)--(0,0)--(12,0)node[below]{$x$};
				\draw[-stealth] (0,-7)--(0,22)node[left]{$y$};
				\draw[smooth,samples=100,thick] plot[domain=0:10](\x,{(1/5)*((\x)-10)^2})
				plot[domain=-10:0](\x,{(1/5)*((\x)+10)^2})
				;
				\draw[thick] (10,0) rectangle (-10,-5);
				\foreach \t/\g in {O/-135,O'/-135}{
					\draw[fill=black] (\t) circle (1pt) node[shift={(\g:7pt)},font=\scriptsize]{$ \t $};
				}
				\draw[fill=black] (A) circle (1pt)node[shift={(45:10pt)}]{$A(10;0)$};
				\draw[fill=black] (B) circle (1pt)node[right]{$B(0;20)$};
				\path (O')--(O)node[pos=0.4,left]{$5$};
				\node at ($(A)+(100:10)$) {$y=\dfrac{1}{5}(x-10)^2$};
			\end{tikzpicture}
		}\noindent
		Vì $(P)$ qua điểm $B(0;20)$ nên $a=\dfrac{1}{5}$.\\
		Do đó, $(P)\colon y=\dfrac{1}{5}(x-10)^2$. Từ đó suy ra $x=10-\sqrt{5y}$ (do $ x<10$).\\
		Suy ra $V_2=\pi \displaystyle\int\limits_0^{20}\left(10-\sqrt{5y}\right)^2\mathrm{\,d}y=\pi(3000-\dfrac{8000}{3})=\dfrac{1000}{3}\pi$ cm$^3$.\\
		Do đó $V=V_1+V_2=\dfrac{1000}{3}\pi +500\pi =\dfrac{2500}{3}\pi$ cm$^3$.
		
	}
\end{ex}
%%==========Câu 13
\begin{ex}%[2D4V3-4]
	\immini{Một chi tiết máy được thiết kế như hình vẽ bên. Các tứ giác $ABCD$, $CDPQ$ là các hình vuông cạnh $2{,}5$ (cm). Tứ giác $ABEF$ là hình chữ nhật có $BE=3{,}5$ (cm). Mặt bên $PQEF$ được mài nhẵn theo đường parabol $(P)$ có đỉnh parabol nằm trên cạnh $ EF$. Thể tích của chi tiết máy bằng
		
	}
	{\begin{tikzpicture}[scale=0.8, font=\footnotesize, line join=round, line cap=round,>=stealth,declare function={a=3;h=4.5;}]
			\path (0,0) coordinate (A)
			--++(30:a) coordinate (B)
			--++(90:a) coordinate (C)
			($(A)+(C)-(B)$) coordinate (D)
			($(A)+(180:h)$) coordinate (F)
			($(B)+(F)-(A)$) coordinate (E)
			($(D)+(180:a)$) coordinate (P)
			($(C)+(P)-(D)$) coordinate (Q)
			;
			\draw (A)--(B)--(C)--(D)--cycle (F)--(A) (D)--(P)--(Q)--(C);
			\draw[dashed] (F)--(E)--(B);
			\draw (F)..controls +(30:1) and +(-100:1)..(P)node[midway,right]{$c$};
			\draw[dashed] (E)..controls +(30:1) and +(-100:1)..(Q)node[pos=0.3,right]{$c$};
			\foreach \t/\g in {A/-90,B/-90,C/0,D/-45,E/135,F/-90,P/180,Q/135}{
				\draw[fill=black] (\t) circle (1pt) node[shift={(\g:7pt)},font=\scriptsize]{$ \t $};
			}
		\end{tikzpicture}
	}
	\choice
	{$\dfrac{395}{24}$ cm$^3$}
	{$\dfrac{50}{3}$ cm$^3$}
	{$\dfrac{125}{8}$ cm$^3$}
	{\True $\dfrac{425}{24}$ cm$^3$}
	\loigiai{
		\immini{Gọi hình chiếu của $P$, $Q$ trên $AF$ và $BE$ là $S$ và $R$.\\
			Vật thể được chia thành hình lập phương $ABCD.PQRS$ có cạnh $2{,}5$ (cm), thể tích $V_1=\dfrac{125}{8}$ cm$^3$ và phần còn lại có thể tích $V_2$.\\
			Khi đó thể tích vật thể $V=V_1+V_2=\dfrac{125}{8}+V_2$.
		}
		{\begin{tikzpicture}[scale=0.7, font=\footnotesize, line join=round, line cap=round,>=stealth,declare function={a=3;h=6;}]
				\path (0,0) coordinate (A)
				--++(30:a) coordinate (B)
				--++(90:a) coordinate (C)
				($(A)+(C)-(B)$) coordinate (D)
				($(A)+(180:h)$) coordinate (F)
				($(B)+(F)-(A)$) coordinate (E)
				($(D)+(180:a)$) coordinate (P)
				($(C)+(P)-(D)$) coordinate (Q)
				($(A)+(P)-(D)$) coordinate (S)
				($(B)+(Q)-(C)$) coordinate (R)
				($(F)!0.5!(S)$) coordinate (M)
				($(M)+(90:2)$) coordinate (M_1)
				($(E)!0.5!(R)$) coordinate (K)
				($(K)+(90:2)$) coordinate (K_1)
				;
				\path[name path=d1] (F)..controls +(30:1) and +(-100:1)..(P);
				\path[name path=d2] (E)..controls +(30:1) and +(-100:1)..(Q);
				\path[name path=d3] (M)--(M_1);
				\path[name path=d4] (K)--(K_1);
				\path[name intersections={of=d1 and d3,by=N}];
				\path[name intersections={of=d2 and d4,by=H}];
				\fill[gray!30] (N)--(M)--(K)--(H)--cycle;
				\draw (A)--(B)--(C)--(D)--cycle (F)--(A) (D)--(P)--(Q)--(C) (P)--(S) (N)--(M);
				\draw[dashed] (F)--(E)--(B) (Q)--(R)--(S) (M)--(K)--(H) (N)--(H);
				\draw (F)..controls +(30:1) and +(-100:1)..(P);
				\draw[dashed] (E)..controls +(30:1) and +(-100:1)..(Q);
				\draw[-stealth] (A)--++(0:1)node[below]{$x$};
				\draw[-stealth] (F)--++(90:1.5*a)node[left]{$y$};
				\foreach \t/\g in {A/-90,B/-90,C/0,D/-45,E/115,F/-90,P/180,Q/135,S/-90,R/-90,N/90,M/-90,K/-90,H/135}{
					\draw[fill=black] (\t) circle (1pt) node[shift={(\g:7pt)},font=\scriptsize]{$ \t $};
				}
			\end{tikzpicture}
		}\noindent
		Đặt hệ trục $Oxyz$ sao cho $O$ trùng với $F$, $Ox$ trùng với $FA$, $Oy$ trùng với tia $Fy$ song song với $AD$. Khi đó Parabol $(P)$ có phương trình dạng $y=ax^2$, đi qua điểm $P\left(1;\dfrac{5}{2}\right)$ do đó $a=\dfrac{5}{2}\Rightarrow y=\dfrac{5}{2}{x^2}$.\\
		Cắt vật thể bởi mặt phẳng vuông góc với $Ox$ và đi qua điểm $M(x;0;0)$, $0\le x\le 1$ ta được thiết diện là hình chữ nhật $MNHK$ có cạnh là $ MN=\dfrac{5}{2}x^2$ và $ MK=\dfrac{5}{2}$ do đó diện tích $S(x)=\dfrac{25}{4}x^2$.\\
		Áp dụng công thức thể tích vật thể ta có $V_2=\displaystyle\int\limits_0^1\dfrac{25}{4}x^2\mathrm{\,d}x=\dfrac{25}{12}$.\\
		Từ đó $V=\dfrac{125}{8}+\dfrac{25}{12}=\dfrac{425}{24}\approx17{,}7$ cm$^3$.
		
	}
\end{ex}

%%==========Câu 14
\begin{ex}%[2D4V3-4]
	Bổ dọc một quả dưa hấu ta được thiết diện là hình elip có trục lớn $28$ cm, trục nhỏ $25$ cm. Biết cứ $1000$ m$^3$ dưa hấu sẽ làm được cốc sinh tố giá $20000$ đồng. Hỏi từ quả dưa hấu trên có thể thu được bao nhiêu tiền từ việc bán nước sinh tố? Biết rằng bề dày vỏ dưa không đáng kể.
	\choice
	{\True $183000$ đồng} 
	{$180000$ đồng} 
	{$185000$ đồng} 
	{$190000$ đồng}
	\loigiai{ 
		Đường elip có trục lớn $28$ cm, trục nhỏ $25$ cm có phương trình
		\[\dfrac{y^2}{\left(\dfrac{25}{2}\right)^2}=1\Leftrightarrow y^2=\left(\dfrac{25}{2}\right)^2\left(1-\dfrac{x^2}{14^2}\right)\Leftrightarrow y=\pm \dfrac{25}{2}\sqrt{1-\dfrac{x^2}{14^2}}.\]
		Do đó thể tích quả dưa là
		{\allowdisplaybreaks
			\begin{eqnarray*}
				V
				&=& \pi \int\limits_{-14}^{14}\left(\dfrac{25}{2}\sqrt{1-\dfrac{x^2}{14^2}} \right)^2\mathrm{d}x\\
				&=& \pi\left(\dfrac{25}{2}\right)^2\displaystyle\int\limits_{-14}^{14}\left(1-\dfrac{x^2}{14^2}\right)^2\mathrm{d}x\\
				&=&\pi\left(\dfrac{25}{2}\right)^2 \cdot \left(x-\dfrac{x^3}{3\cdot 14^2}\right) \Big|_{-14}^{14}\\
				&=& \pi\left(\dfrac{25}{2}\right)^2 \cdot \dfrac{56}{3}\\
				&=& \dfrac{8750\pi}{3} \;\;\text{cm}^3.
			\end{eqnarray*}
		}
		Do đó tiền bán nước thu được là $\dfrac{8750\pi \cdot 20000}{3\cdot 1000}\approx 183259$ đồng.
	} 
\end{ex} 
%------------------------------------------------------------
% \begin{ex}%[2D4V3-4]%Câu 1.
% 	\immini{Có một cốc nước thủy tinh hình trụ, bán kính trong lòng đáy cốc là $6\,\text{cm}$, chiều cao lòng cốc là $10\,\text{cm}$ đang đựng một lượng nước. Tính thể tích lượng nước trong cốc, biết khi nghiêng cốc nước vừa lúc khi nước chạm miệng cốc thì đáy mực nước trùng với đường kính đáy.
% 		\choice
% 		{\True $240$\,cm$^3$}
% 		{$240\pi$ \,cm$^3$}
% 		{$120$\,cm$^3$}
% 		{$120\pi$ \,cm$^3$}}
% 	{\begin{tikzpicture}[scale=0.7, font=\footnotesize,line join=round, line cap=round, >=stealth]
% 			\begin{scope}[shift={(0,0)}]
% 				\def\a{1.5}
% 				\def\b{.5}
% 				\def\h{4}
% 				\path
% 				(0,0) coordinate (M)
% 				($(M)+(2*\a,0)$) coordinate (N)
% 				($(M)!0.5!(N)$)coordinate (O)
% 				($(M)+(0,\h)$) coordinate (M')
% 				($(N)+(0,\h)$) coordinate (N')
% 				($(O)+(0,\h)$) coordinate (O')
% 				($(M')!0.6!(M)$)coordinate (A)
% 				($(N')!0.6!(N)$)coordinate (B)
% 				;
% 				\fill[black!15] (A) arc (180:0:\a cm and \b cm)--(N)--(M) arc (-180:0:\a cm and \b cm)--(M)--(A);
% 				\draw(M)--(M') (N)--(N');
% 				\draw[dashed,thin] (M) arc (180:0:\a cm and \b cm);
% 				\draw[dashed,thin] (A) arc (180:0:\a cm and \b cm);
% 				\draw (O') ellipse (\a cm and \b cm)	(M) arc (-180:0:\a cm and \b cm) (A) arc (-180:0:\a cm and \b cm);
% 			\end{scope}
% 			\begin{scope}[rotate=-70,shift={(0,6)}]
% 				\def\a{1.5}
% 				\def\b{.5}
% 				\def\h{4}
% 				\path
% 				(0,0) coordinate (M)
% 				($(M)+(2*\a,0)$) coordinate (N)
% 				($(M)!0.5!(N)$)coordinate (O)
% 				($(M)+(0,\h)$) coordinate (M')
% 				($(N)+(0,\h)$) coordinate (N')
% 				($(O)+(0,\h)$) coordinate (O')
% 				($(M')!0.6!(M)$) coordinate (A)
% 				($(N')!0.6!(N)$) coordinate (B)
% 				;
% 				\fill[black!15]  (2.1,-.45)--(.8,.45) .. controls +(70:2) and +(180:0) ..(N') (N').. controls +(-90:.3) and +(180:0) ..(N).. controls +(-90:.3) and +(0:0.3) ..(2.1,-.45);
% 				\draw(M)--(M') (N)--(N');
% 				\draw[dashed,thin] (M) arc (180:0:\a cm and \b cm);
% 				\draw (O') ellipse (\a cm and \b cm)	(M) arc (-180:0:\a cm and \b cm);
				
% 				\draw[dashed] (2.1,-.45)--(.8,.45) .. controls +(70:2) and +(180:0) ..(N') (1.5,0)--(N');
% 				\draw (2.1,-.45) .. controls +(40:1) and +(180:0) ..(N') ;
				
% 			\end{scope}
% 	\end{tikzpicture}} 
	
% 	\loigiai{
% 		\textbf{Cách 1.} 
% 		\begin{center}
% 			\begin{tikzpicture}[scale=1, font=\footnotesize,line join=round, line cap=round, >=stealth]
% 				\path
% 				(0,0) coordinate (A)
% 				(0,3) coordinate (B)
% 				;
				
% 				\draw (A) arc (0: -90: 5 and 2) coordinate (C);
% 				\draw[dashed] (A) arc (0: 90: 4.5 and 2) coordinate (D);
% 				\coordinate (I) at ($(C)!.5!(D)$);
% 				\draw (A)--(B) (C)--(D) (I)--(B);
% 				\draw[dashed] (I)--(A);
% 				\draw 
% 				(C) .. controls +(0:0) and +(-90:1.5) ..(B).. controls +(90:1.5) and +(60:0) ..(D);
% 				\draw pic[draw, angle radius=2mm]{right angle=B--A--I};
% 				\pic[draw,"$\alpha$", angle eccentricity=0.6,angle radius=0.8cm]{angle=A--I--B};
% 				\path (A)--(I) node[above,pos=.7,sloped]{$R$};
% 				\path (C)--(I) node[above,midway,sloped]{$R$};
% 				\path (D)--(I) node[above,midway,sloped]{$R$};
% 			\end{tikzpicture}
% 		\end{center}
% 		Xét thiết diện cắt cốc thủy tinh vuông góc với đường kính tại vị trí bất kỳ có  
% 		$$S(x)=\dfrac{1}{2}\sqrt{R^2-x^2}\cdot \sqrt{R^2-x^2}\cdot \tan \alpha= \dfrac{1}{2}\left( R^2-x^2 \right)\tan \alpha.$$
% 		Thể tích hình cái nêm là: $V=\dfrac{1}{2}\tan \alpha \displaystyle\int\limits_{-R}^{R}{\left( R^2-x^2 \right)}\mathrm{\,d}x=\dfrac{2}{3}R^3\tan \alpha $.\\
% 		Thể tích khối nước tạo thành khi nguyên cốc có hình dạng cái nêm nên $V_{kn}=\dfrac{2}{3}R^3\tan \alpha $. \\
% 		$\Rightarrow V_{kn}=\dfrac{2}{3}R^3\cdot \dfrac{h}{R}=240\,cm^3$.\\
% 		\textbf{Cách 2.} 
% 		\begin{center}
% 			\begin{tikzpicture}[scale=1, font=\footnotesize,line join=round, line cap=round, >=stealth]
% 				\path
% 				(0,0) coordinate (O)
% 				(0,4) coordinate (O')
% 				(7,0) coordinate (J)
% 				(7,4) coordinate (J')
% 				(9,0) coordinate (x)
% 				($(J)!.5!(J')$) coordinate (I)
% 				($(O)!.5!(O')$) coordinate (I')
% 				;
% 				\fill[cyan!20] (J)--(O) .. controls +(82:0.6) and +(180:0.6) ..
% 				(6.15,3.15)--(7.85,.9).. controls +(180:0.05) and +(0:.6) ..
% 				(7,0);
% 				\draw[->] (O)--(x);
% 				\draw[dashed,name path=OB] 
% 				(O) .. controls +(82:0.6) and +(180:0.6) ..
% 				(6.15,3.15)coordinate (B)--(7.85,.9) coordinate (A);
% 				\draw (0,2) ellipse (1 and 2) (O)--(O') (O')--(J') ;
% 				\draw (J) arc (-90: 90: 1 and 2);
% 				\draw[dashed] (J) arc (-90: 90: -1 and 2) (J)--(J') ;
% 				\draw[dashed,name path=II'] (I)--(I');
% 				\path [name intersections={of=OB and II',by=H}];
% 				\coordinate (E) at ($(J)!(H)!(O)$);
% 				\coordinate (N) at ($(O)!.55!(A)$);
				
% 				\path (intersection of H--E and O--I) coordinate (F);
% 				\coordinate (n) at ($(F)!-3!(N)$);
% 				\path[name path=FN] (N)--(n);
% 				\path [name intersections={of=OB and FN,by=M}];
				
% 				\fill[orange!50] (M) .. controls +(-90:1) and +(180:.5) ..(E) .. controls +(0:.5) and +(180:0) ..(N);
% 				\draw[dashed] (H)--(E) (M)--(N) (H)--(N) (O)--(I);
% 				\draw[dashed] (M) .. controls +(-90:1) and +(180:.5) ..(E);
% 				\draw (E) .. controls +(0:.5) and +(180:0) ..(N) (O)--(A);
% 				\draw[<->] ($(O)+(0,-.3)$)--($(E)+(0,-.3)$) node[fill=white,midway,sloped]{$x$};
% 				\draw[<->] ($(O')+(0,.3)$)--($(J')+(0,.3)$) node[fill=white,midway,sloped]{$10$ cm};
% 				\draw[<->] ($(J)+(1.5,0)$)--($(J')+(1.5,0)$) node[fill=white,midway,sloped]{$12$ cm};
% 				\draw[->] ($(E)+(.5,.3)$)--($(E)+(.8,-.5)$) node[below] {$S(x)$};
% 				\pic[draw,"$\alpha$", angle eccentricity=1.1,angle radius=2cm]{angle=J--O--I};
% 				\node[above right] at (F) {$\beta$};
% 				\foreach \x/\g in {H/90,E/-70,F/120,N/-90,M/90,I/0,J/-90,O/-120} \fill[black] (\x) circle (1pt)+(\g:.3) node {$\x$};
% 			\end{tikzpicture}
% 		\end{center}
% 		Dựng hệ trục tọa độ $Oxyz$.\\
% 		Gọi $S\left( x \right)$ là diện tích thiết diện do mặt phẳng có phương vuông góc với trục $Ox$ với khối nước, mặt phẳng này cắt trục $Ox$ tại điểm có hoành độ $h\ge x\ge 0$.\\
% 		Gọi $\widehat{IOJ}=\alpha ,\,\widehat{FHN}=\beta ,\,OE=x$\\
% 		$\tan \alpha =\dfrac{IJ}{OJ}=\dfrac{6}{10}=\dfrac{EF}{OE}\Rightarrow EF=\dfrac{6x}{10}\Rightarrow HF=6-\dfrac{6x}{10}$.\\
% 		$\cos \beta =\dfrac{HF}{HN}=\dfrac{6-\dfrac{6x}{10}}{6}=1-\dfrac{x}{10}\Rightarrow \beta =\arccos \left( 1-\dfrac{x}{10} \right)$\\
% 		%-------------------------------------
% 		$S\left( x \right)=S_{\text{hình quạt}}-S_{HMN}=\dfrac{1}{2}HN^2\cdot 2\beta -\dfrac{1}{2}HM\cdot HN\cdot \sin 2\beta $\\
% 		%-------------------------------------
% 		$\Rightarrow S\left( x \right)=6^2\arccos \left( 1-\dfrac{x}{10} \right)-\dfrac{1}{2}\cdot 6\cdot 6\cdot 2\left( 1-\dfrac{x}{10} \right)\sqrt{1-\left( 1-\dfrac{x}{10} \right)^2}$\\
% 		$\Rightarrow V=\displaystyle\int\limits_{0}^{10}{S\left( x \right) \,\mathrm{d}x}=\displaystyle\int\limits_{0}^{10}{\left( 36\arccos \left( 1-\dfrac{x}{10} \right)-36\left( 1-\dfrac{x}{10} \right)\sqrt{1-\left( 1-\dfrac{x}{10} \right)^2} \right)\,\mathrm{d}x}=240$.}
% \end{ex}
%------------------------------------------------------------
% \begin{ex}%[2D4V3-4]%Câu 2.
% 	\immini{Cho vật thể đáy là hình tròn có bán kính bằng 1 (tham khảo hình vẽ). Khi cắt vật thể bằng mặt phẳng vuông góc với trục $Ox$ tại điểm có hoành độ $x\ \left( -1\le x\le 1 \right)$ thì được thiết diện là một tam giác đều. Thể tích $V$ của vật thể đó là
% 		\choice
% 		{$V=\sqrt{3}$}
% 		{$V=3\sqrt{3}$}
% 		{\True $V=\dfrac{4\sqrt{3}}{3}$}
% 		{$V=\pi $}}
% 	{\includegraphics[scale=0.4]{images/Cau2_C4B3CD3.png} }
% 	\loigiai{
% 		\immini{
% 			Do vật thể có đáy là đường tròn và khi cắt bởi mặt phẳng vuông góc với trục $Ox$ được thiết diện là tam giác đều do đó vật thể đối xứng qua mặt phẳng vuông góc với trục $Oy$ tại điểm $O$.\\
% 			Cạnh của tam giác đều thiết diện là  $a=2\sqrt{1-x^2}$.\\
% 			Diện tích tam giác thiết diện là  
% 			$$S=\dfrac{a^2\sqrt{3}}{4}=\left( 1-x^2 \right)\sqrt{3}.$$
% 		}
% 		{\begin{tikzpicture}[scale=0.7, font=\footnotesize,line join=round, line cap=round, >=stealth]
% 				\path
% 				(0,0) coordinate (O)
% 				(60:3) coordinate (A)
% 				(-60:3) coordinate (B)
% 				($(A)!.5!(B)$) coordinate (x)
% 				;
% 				\draw (O) circle (3) (O)--(A)--(B);
% 				\draw[->] (-4,0) -- (4,0)node[below] {$x$};
% 				\draw[->] (0,-4) -- (0,4)node[right] {$y$};
% 				\path (A)--(B) node[above right,midway]{$\sqrt{1-x^2}$};
% 				\foreach \x/\g in {O/-120,x/-70} \fill[black] (\x) circle (1pt)+(\g:.3) node {$\x$};
% 		\end{tikzpicture}}
% 		\noindent
% 		Thể tích khối cần tìm là 
% 		$$V=2\displaystyle\int\limits_{0}^{1}{Sdx}=2\displaystyle\int\limits_{0}^{1}{\sqrt{3}\left( 1-x^2 \right)=\left. 2\sqrt{3}\left( x-\dfrac{x^3}{3} \right) \right|_{0}^{1}=\dfrac{4\sqrt{3}}{3}}.$$
% 	}
% \end{ex}
% %------------------------------------------------------------
% \begin{ex}%[2D4V3-4]%Câu 3.
% 	Sân vận động Sport Hub (Singapore) là sân có mái vòm kỳ vĩ nhất thế giới. Đây là nơi diễn ra lễ khai mạc Đại hội thể thao Đông Nam Á được tổ chức tại Singapore năm $2015$. Nền sân là một elip $\left( E \right)$ có trục lớn dài $150m$, trục bé dài $90m$ (hình vẽ). Nếu cắt sân vận động theo một mặt phẳng vuông góc với trục lớn của $\left( E \right)$và cắt elip ở $M,N$ (hình vẽ) thì ta được thiết diện luôn là một phần của hình tròn có tâm $I$ (phần tô đậm trong hình 4) với $MN$ là một dây cung và góc $\widehat{MIN}=90^{\circ}.$ Để lắp máy điều hòa không khí thì các kỹ sư cần tính thể tích phần không gian bên dưới mái che và bên trên mặt sân, coi như mặt sân là một mặt phẳng và thể tích vật liệu là mái không đáng kể. Hỏi thể tích xấp xỉ bao nhiêu?
% 	\begin{center}
% 		{\includegraphics[scale=0.8]{images/Cau3_C4B3CD3.png}\\
% 			\begin{tikzpicture}[scale=1, font=\footnotesize,line join=round, line cap=round, >=stealth]
% 				\path
% 				(0,0) coordinate (O)
% 				(-2,0) coordinate (A)
% 				(2,0) coordinate (B)
% 				(70: 2 and 1) ellipse  coordinate (M)
% 				(-70: 2 and 1) ellipse  coordinate (N)
% 				;
% 				\draw (O) ellipse (2 and 1) (M)--(N) (A)--(B);
% 				\fill[black] (B) circle (1pt);
% 				\fill[black] (A) circle (1pt);
% 				\foreach \x/\g in {M/90,N/-90} \fill[black] (\x) circle (1pt)+(\g:.3) node {$\x$};
% 				\begin{scope}[shift={(5,-0)}]
% 					\path (0,0) coordinate (I) (40:2)   coordinate (M)
% 					(140:2)  coordinate (N) ;
% 					\draw (I) circle (2) (M)--(N);
					
					
% 					\clip (-2,1.28) rectangle (2,2);
% 					\fill[black!15] (I) circle (2);
% 					\draw (I) circle (2) (M)--(N);
% 				\end{scope}
% 				\foreach \x/\g in {M/45,N/135,I/-90} \fill[black] (\x) circle (1pt)+(\g:.3) node {$\x$};
% 		\end{tikzpicture} }
% 	\end{center}
% 	\choice
% 	{$57793$ m$^3$}
% 	{\True $115586$ m$^3$}
% 	{$32162$ m$^3$}
% 	{$101793$ m$^3$}
% 	\loigiai{
% 		\begin{center}
% 			\includegraphics[scale=0.8]{images/Cau3_C4B3CD3_g.png}
% 		\end{center}
% 		Chọn hệ trục như hình vẽ\\
% 		Ta cần tìm diện tích của $S\left( x \right)$thiết diện.\\
% 		Gọi $d\left( O,MN \right)=x$\\
% 		$\left( E \right)\colon\dfrac{x^2}{75^2}+\dfrac{y^2}{45^2}=1.$\\
% 		Lúc đó $MN=2y=2\sqrt{45^2\left( 1-\dfrac{x^2}{75^2} \right)}=90\sqrt{1-\dfrac{x^2}{75^2}}$\\
% 		$\Rightarrow R=\dfrac{MN}{\sqrt{2}}=\dfrac{90}{\sqrt{2}}.\sqrt{1-\dfrac{x^2}{75^2}}\Rightarrow R^2=\dfrac{90^2}{2}\cdot \left( 1-\dfrac{x^2}{75^2} \right)$.\\
% 		$S\left( x \right)=\dfrac{1}{4}\pi R^2-\dfrac{1}{2}R^2=\left( \dfrac{1}{4}\pi -\dfrac{1}{2} \right)R^2=\left( \pi -2 \right)\dfrac{2025}{2}.\left( 1-\dfrac{x^2}{75^2} \right).$\\
% 		Thể tích khoảng không cần tìm là
% 		$$V=\displaystyle\int\limits_{-75}^{75}\left( \pi -2 \right)\dfrac{2025}{2}.\left( 1-\dfrac{x^2}{75^2} \right)\approx 115586 \,\text{m}^3.$$
% 	}
% \end{ex}
%------------------------------------------------------------
% \begin{ex}%[2D4V3-4]%Câu 4.
% 	\immini{Gọi $\left( H \right)$ là phần giao của hai khối $\dfrac{1}{4}$ hình trụ có bán kính $a$, hai trục hình trụ vuông góc với nhau như hình vẽ sau. Tính thể tích của khối $\left( H \right)$.
% 		\choice
% 		{${{V}_{\left( H \right)}}=\dfrac{a^3}{2}$}
% 		{${{V}_{\left( H \right)}}=\dfrac{3a^3}{4}$}
% 		{\True $V_{\left( H \right)}=\dfrac{2a^3}{3}$}
% 		{${{V}_{\left( H \right)}}=\dfrac{\pi a^3}{4}$}}
% 	{\includegraphics[scale=0.8]{images/Cau4_C4B3CD3_De.png}}
% 	\loigiai{
% 		\begin{center}
% 			\includegraphics[scale=0.8]{images/Cau4_C4B3CD3.png}
% 		\end{center}
% 		+ Đặt hệ toạ độ $Oxyz$ như hình vẽ, xét mặt cắt song song với mp $\left( Oyz \right)$ cắt trục $Ox$ tại $x$: thiết diện mặt cắt luôn là hình vuông có cạnh $\sqrt{a^2-x^2}$ $\left( 0\le x\le a \right)$.\\
% 		+ Do đó thiết diện mặt cắt có diện tích: $S\left( x \right)=a^2-x^2$.\\
% 		+ Vậy $V_{\left( H \right)}=\displaystyle\int\limits_{0}^{a}{S\left( x \right)\,\mathrm{\,d}x} =\displaystyle\int\limits_{0}^{a}{\left( a^2-x^2 \right)\,\mathrm{d}x} =\left. \left( a^2x-\dfrac{x^3}{3} \right) \right|_{0}^{a}$\\
% 		$=\dfrac{2a^3}{3}$}
% \end{ex}
%------------------------------------------------------------
\begin{ex}%[2D4H3-5]%Câu 5.
	Một bác thợ xây bơm nước vào bể chứa nước. Gọi $h\left( t \right)$ là thể tích nước bơm được sau $t$ giây. Cho ${h}'\left( t \right)=6at^2+2bt$ và ban đầu bể không có nước. Sau 3 giây thì thể tích nước trong bể là $90m^3$, sau $6$ giây thì thể tích nước trong bể là $504m^3$. Tính thể tích nước trong bể sau khi bơm được $9$ giây.
	\choice
	{\True $1458m^3$}
	{$600m^3$}
	{$2200m^3$}
	{$4200m^3$}
	\loigiai{
		$\displaystyle\int\limits_{0}^{3}\left( 6at^2+2bt \right)\,\mathrm{d}t=90\Leftrightarrow \left.\left( 2at^3+bt^2 \right) \right|_{0}^{3}=90\Leftrightarrow 54a+9b=90$\quad (1)\\
		$\displaystyle\int\limits_{0}^{6}\left( 6at^2+2bt \right)\,\mathrm{d}t=504 \Leftrightarrow  \left. \left( 2at^3+bt^2 \right) \right|_{0}^{6}=504\Leftrightarrow 432a+36b=504$\quad (2)\\
		Từ (1), (2) $\Rightarrow  \heva{
			& a=\dfrac{2}{3} \\ 
			& b=6.}$\\
		Sau khi bơm $9$ giây thì thể tích nước trong bể là\\
		$V=\displaystyle\int\limits_{0}^{9}\left(4t^2+12t \right)\,\mathrm{d}t =  \left. \left(\dfrac{4}{3}t^3+6t^2 \right) \right|_{0}^{9}=1458\; \left(m^3 \right)$.}
\end{ex}
%------------------------------------------------------------
\begin{ex}%[2D4H3-5]%Câu 6.
	Người ta thay nước mới cho một bể bơi có dạng hình hộp chữ nhật có độ sâu là $280$cm. Giả sử $h\left( t \right)$là chiều cao (tính bằng cm) của mực nước bơm được tại thời điểm $t$ giây, biết rằng tốc độ tăng của chiều cao mực nước tại giây thứ $t$ là ${h}'(t)=\dfrac{1}{500}\sqrt[3]{t}$ và lúc đầu hồ bơi không có nước. Hỏi sau bao lâu thì bơm được số nước bằng $\dfrac{3}{4}$độ sâu của hồ bơi (làm tròn đến giây)?
	\choice
	{$2$ giờ $36$ giây}
	{$2$ giờ $48$ giây}
	{\True $2$ giờ $38$ giây}
	{$2$ giờ $46$ giây}
	\loigiai{
		Gọi $x$ là thời điểm bơm được số nước bằng $\dfrac{3}{4}$ độ sâu của bể ($x$ tính bằng giây).
		Ta có
		\begin{eqnarray*}
			&&\displaystyle\int\limits_0^x{\dfrac{1}{500}\sqrt[3]{t}\mathrm{\,d}t}=\dfrac{3}{4}\cdot 280\left. \Rightarrow \dfrac{3}{4}t^{\dfrac{4}{3}} \right|_0^x=105000\\
			&\Rightarrow& x\sqrt[3]{x}=140000\Rightarrow \sqrt[3]{x^4}=140000\\
			&\Rightarrow& x=\sqrt[4]{140000^3}\Rightarrow x\approx 7237{,}6242.
		\end{eqnarray*}
		Suy ra $x= 2$ giờ $38$ giây.}
\end{ex}
%------------------------------------------------------------
\Closesolutionfile{ans}
\indapan{6}{ans/ans-2-C4B3CD3_1-2-lc}


%% Ôn KTTX chương IV
% \begin{name}
	{NGUYÊN HÀM - TÍCH PHÂN}
	{KT NGUYÊN HÀM}
	{\tentruong}
	{\thoigian}
\end{name}
\setcounter{ex}{0}\setcounter{bt}{0}
\Opensolutionfile{ans}[ans/ans-2-B11-De1-lc]
\TN
\begin{ex}%Câu 1%[2D4N1-2]
	Họ nguyên hàm của hàm số $f(x)=x^3$ là
	\choice
	{$4x^4+C$}
	{$3x^2+C$}
	{$x^4+C$}
	{\True $\dfrac{1}{4}x^4+C$}
	\loigiai{
Ta có 
	$\displaystyle\int x^3\mathrm{\,d}x=\dfrac{1}{4}x^4+C$.
}
\end{ex}

\begin{ex}%Câu 2%[2D4N1-3]
	Tìm nguyên hàm của hàm số $ f(x)=2\sin x$.
	\choice
	{\True $\displaystyle\int 2\sin x\mathrm{\,d}x=-2\cos x+C$}
	{$\displaystyle\int 2\sin x\mathrm{\,d}x=2\cos x+C$}
	{$\displaystyle\int 2\sin x\mathrm{\,d}x=\sin^2x+C$}
	{$\displaystyle\int 2\sin x\mathrm{\,d}x=\sin 2x+C$}
	\loigiai{
Ta có $\displaystyle\int 2\sin x\mathrm{\,d}x=2\displaystyle\int \sin x \mathrm{\,d}x=-2\cos x+C$.}
\end{ex}

\begin{ex}%Câu 3%[2D4N1-4]
	Họ nguyên hàm của hàm số $f(x)=\mathrm{e}^x+x$ là
	\choice
	{$\mathrm{e}^x+1+C$}
	{$\mathrm{e}^x+x^2+C$}
	{\True $\mathrm{e}^x+\dfrac{1}{2}{x^2}+C$}
	{$\dfrac{1}{x+1}{\mathrm{e}^x}+\dfrac{1}{2}{x^2}+C$}
	\loigiai{
Ta có $\displaystyle\int \left(\mathrm{e}^x+x\right) \mathrm{\,d}x=\displaystyle\int \mathrm{e}^x \mathrm{\,d}x+\displaystyle\int x \mathrm{\,d}x=\mathrm{e}^x+\dfrac{x^2}{2}+C$.}
\end{ex}

\begin{ex}%Câu 4%[2D4N1-2]
	Họ nguyên hàm của hàm số $y=x^2-3x+\dfrac{1}{x}$ là
	\choice
	{$\dfrac{x^3}{3}-\dfrac{3x^2}{2}-\ln\left|x\right|+C$}
	{$\dfrac{x^3}{3}-\dfrac{3x^2}{2}+\ln x+C$}
	{\True $\dfrac{x^3}{3}-\dfrac{3x^2}{2}+\ln\left|x\right|+C$}
	{$\dfrac{x^3}{3}-\dfrac{3x^2}{2}+\dfrac{1}{x^2}+C$}
	\loigiai{
	Ta có 
		$\displaystyle\int \left(x^2-3x+\dfrac{1}{x}\right) \mathrm{\,d}x=\dfrac{x^3}{3}-\dfrac{3x^2}{2}+\ln\left|x\right|+C$.
	
}
\end{ex}

\begin{ex}%Câu 5%[2D4H1-2]
	Tìm nguyên hàm của hàm số $f(x)=\dfrac{x^4+2}{x^2}$.
	\choice
	{$\displaystyle\int f(x)\mathrm{\,d}x=\dfrac{x^3}{3}-\dfrac{1}{x}+C$}
	{$\displaystyle\int f(x)\mathrm{\,d}x=\dfrac{x^3}{3}+\dfrac{2}{x}+C$}
	{$\displaystyle\int f(x)\mathrm{\,d}x=\dfrac{x^3}{3}+\dfrac{1}{x}+C$}
	{\True $\displaystyle\int f(x)\mathrm{\,d}x=\dfrac{x^3}{3}-\dfrac{2}{x}+C$}
	\loigiai{
Ta có
		$\displaystyle\int \dfrac{x^4+2}{x^2} \mathrm{\,d}x=\displaystyle\int \left(x^2+\dfrac{2}{x^2}\right) \mathrm{\,d}x=\dfrac{x^3}{3}-\dfrac{2}{x}+C$.}
\end{ex}

\begin{ex}%Câu 6%[2D4N1-3]
	Cho hàm số $ f(x)=1-\dfrac{1}{\cos^2 x}$. Khẳng định nào dưới đây đúng?
	\choice
	{$\displaystyle\int f(x)\mathrm{\,d}x=x+\tan x+C$}
	{$\displaystyle\int f(x)\mathrm{\,d}x=x+\cot x+C$}
	{\True $\displaystyle\int f(x)\mathrm{\,d}x=x-\tan x+C$}
	{$\displaystyle\int f(x)\mathrm{\,d}x=x-\cot x+C$}
	\loigiai{
Ta có $\displaystyle\int f(x) \mathrm{\,d}x=\displaystyle\int \left(1-\dfrac{1}{\cos^2x}\right)\mathrm{\,d}x=\displaystyle\int  \mathrm{\,d}x-\displaystyle\int \dfrac{\mathrm{\,d}x}{\cos^2x}=x-\tan x+C$.}
\end{ex}

\begin{ex}%Câu 7%[2D4N1-3]
	Họ nguyên hàm của hàm số $f(x)=\cos x+6x$ là
	\choice
	{\True $\sin x+3x^2+C$}
	{$-\sin x+3x^2+C$}
	{$\sin x+6x^2+C$}
	{$-\sin x+C$}
	\loigiai{
Ta có $\displaystyle\int f(x)\mathrm{\,d}x=\displaystyle\int \left(\cos x+6x\right)\mathrm{\,d}x=\sin x+3x^2+C$.}
\end{ex}

\begin{ex}%Câu 8%[2D4N1-2]
	$\displaystyle\int f(x)\mathrm{\,d}x=4x^3+x^2+C$ thì hàm số $f(x)$ bằng
	\choice
	{$f(x)=x^4+\dfrac{x^3}{3}+Cx$}
	{$f(x)=12x^2+2x+C$}
	{\True $f(x)=12x^2+2x$}
	{$f(x)=x^4+\dfrac{x^3}{3}$}
	\loigiai{
		Ta có $ f(x)=F'(x)=\left(4x^3+x^2+C\right)'=12x^2+2x$.}
\end{ex}

\begin{ex}%Câu 9%[2D4N1-4]
	Hàm số $F(x)=2x+3^x-1$ là nguyên hàm của hàm số nào trong các hàm số sau
	\choice
	{\True $f(x)=2+3^x\ln 3$}
	{$f(x)=x^2+\dfrac{3^x}{\ln 3}-x+C$}
	{$f(x)=x^2+\dfrac{3^x}{\ln 3}-x$}
	{$f(x)=2+3^x\ln 3+C$}
	\loigiai{
	Ta có $ f(x)=F'(x)\Rightarrow f(x)=\left(2x+3^x-1\right)'=2+3^x\ln 3$.}
\end{ex}

\begin{ex}%Câu 10%[2D4H1-4]
	Cho$\displaystyle\int \ln x \mathrm{\,d}x=F(x)+C$. Khẳng định nào dưới đây đúng?
	\choice
	{$F'(x)=\dfrac{1}{x}$}
	{$F'(x)=\dfrac{1}{x}+C$} 
	{\True $F'(x)=\ln x$}
	{$F'(x)=\ln x+1$}
	\loigiai{
Ta có $ F'(x)=f(x)=\ln x$.}
\end{ex}

\begin{ex}%Câu 11%[2D4H1-4]
	Cho $F(x)$ là một nguyên hàm của hàm số $f(x)=\mathrm{e}^x+2x$ thỏa mãn $F(0)=\dfrac{3}{2}$. Tìm $F(x)$.
	\choice
	{\True $F(x)=\mathrm{e}^x+x^2+\dfrac{1}{2}$}
	{$F(x)=\mathrm{e}^x+x^2+\dfrac{5}{2}$}
	{$F(x)=\mathrm{e}^x+x^2+\dfrac{3}{2}$}
	{$F(x)=2\mathrm{e}^x+x^2-\dfrac{1}{2}$}
	\loigiai{
Ta có $ F(x)=\displaystyle\int\left(\mathrm{e}^x+2x\right) \mathrm{\,d}x=e^x+x^2+C$.\\
		Theo bài ra ta có $F(0)=1+C=\dfrac{3}{2}\Rightarrow C=\dfrac{1}{2}$.}
\end{ex}

\begin{ex}%Câu 12%[2D4H1-6]
	Một viên đạn được bắn thẳng đứng lên trên từ mặt đất. Giả sử tại thời điểm $t$ giây (coi $t=0$ là thời điểm viên đạn được bắn lên), vận tốc của nó được cho bởi $v(t)=160-9{,}8t$ (m/s). Độ cao của viên đạn (tính từ mặt đất) sau $t=10$ giây là
	\choice
	{$620$ m}
	{$1\,240$ m}
	{$555$ m}
	{\True $1\,110$ m}
	\loigiai{
Gọi $S(t)$ là độ cao của viên đạn sau $t$ giây kể từ lúc bắt đầu bắn.\\
		Ta có $v(t)=S'(t)$. Do đó, $S(t)$ là một nguyên hàm của vận tốc $ v(t)$.\\
		$S(t)=\displaystyle\int v(t) \mathrm{\,d}t=\displaystyle\int \left(160-9{,}8t\right)\mathrm{\,d}t=160t-4{,}9t^2+C$.\\
		Theo giả thiết, $S(0)=0$ nên $C=0$ và ta được $S(t)=160t-4{,}9t^2$ (m).\\
		Độ cao của viên đạn sau $t=10$ giây là
		$S(10)=160\cdot 10-4{,}9\cdot10^2=1\,110$ (m).\\
		Vậy độ cao của viên đạn (tính từ mặt đất) sau $t=10$ giây là $ 1\,110$ (m).\\
}
\end{ex}
 \Closesolutionfile{ans}
\indapan{6}{ans/ans-2-B11-De1-lc}
\Opensolutionfile{ans}[ans/ans-2-B11-De1-ds]
\TNTF
\begin{ex}%Câu 13%[2D4H1-4]
	Cho hàm số $y=h(x)$ có đạo hàm $h'(x)=3x^2$ và hàm số $y=g(x)$ có đạo hàm $g'(x)=\mathrm{e}^x$.
	\choiceTF
	{Hàm số $y=h(x)=6x+C_1$, với $C_1\in \mathbb{R}$}
	{\True Hàm số $y=g(x)=\mathrm{e}^x+C_2$, với $C_2\in \mathbb{R}$}
	{\True $ I=\displaystyle\int \left[xh'(x)+2025\right]\mathrm{\,d}x=\dfrac{3}{4}{x^4}+2025x+C$ với $C\in \mathbb{R}$}
	{\True Cho $ f'(x)=3x^2+\mathrm{e}^x+m-1$. Cho $f(0)=2$; $ f(1)=2\mathrm{e}$ thì giá trị của $m\in (1;2)$}
\loigiai{
\begin{itemchoice}
	\itemch \textbf{Sai}. Ta có $ h(x)=\displaystyle\int h'(x) \mathrm{\,d}x=3\displaystyle\int x^2\mathrm{\,d}x=x^3+C_1$.
	\itemch \textbf{Đúng}. Ta có $ g(x)=\displaystyle\int g'(x)\mathrm{\,d}x=\displaystyle\int \mathrm{e}^x\mathrm{\,d}x=\mathrm{e}^x+C_2$.
	\itemch \textbf{Đúng}. Ta có $ I=\displaystyle\int \left[xh'(x)+2025\right]\mathrm{\,d}x=\displaystyle\int \left[3x^3+2025\right]\mathrm{\,d}x=\dfrac{3}{4}{x^4}+2025x+C$.
	\itemch \textbf{Đúng}. Ta có $ f(x)=\displaystyle\int f'(x)\mathrm{\,d}x=\displaystyle\int \left(3x^2+\mathrm{e}^x+m-1\right)\mathrm{\,d}x=x^3+\mathrm{e}^x+(m-1)x+C$.\\
	Vì $\heva{
		&f(0)=2\\
		&f(1)=2\mathrm{e}}
	\Rightarrow\heva{
		&1+C=2\\
		&1+\mathrm{e}+m-1+C=2\mathrm{e}}
\Rightarrow\heva{
		&C=1\\
		&m=\mathrm{e}-1.}
	$\\
	Vậy $m=\mathrm{e}-1\Rightarrow 1<m<2$.
	\end{itemchoice}
}
\end{ex}

\begin{ex}%Câu 14%[2D4H1-3]
Cho các hàm số $g(x)=\sin x$ , $h(x)=\cos x$.
\choiceTF
{$\displaystyle\int \left[2g(x)-3h(x)\right]\mathrm{\,d}x=3\displaystyle\int g(x)\mathrm{\,d}x-2\displaystyle\int h(x)\mathrm{\,d}x$}
{\True Một nguyên của hàm số $g(x)$ là $-\cos x$}
{Họ nguyên của hàm số $h(x)+2\sqrt{x}$ là $\sin x+\dfrac{3}{2}\sqrt{x^3}+C$}
{\True Họ nguyên hàm của hàm số $f(x)=g(x)\cdot h^2(x)$ là $F(x)=-\dfrac{1}{3}\cos^3 x+C$}
\loigiai{
\begin{itemchoice}
\itemch \textbf{Sai}. Ta có $\displaystyle\int \left[2g(x)-3h(x)\right]\mathrm{\,d}x=2\displaystyle\int g(x)\mathrm{\,d}x-3\displaystyle\int h(x)\mathrm{\,d}x$.
\itemch \textbf{Đúng}. Ta có $\displaystyle\int g(x)\mathrm{\,d}x=\displaystyle\int \sin x \mathrm{\,d}x=-\cos x+C$ nên một nguyên của hàm số $g(x)$ là $-\cos x$.
\itemch \textbf{Sai}. Ta có $\displaystyle\int \left[h(x)+2\sqrt x\right]\mathrm{\,d}x=\displaystyle\int \cos x\mathrm{\,d}x+2\displaystyle\int x^{\tfrac{1}{2}}\mathrm{\,d}x=\sin x+\dfrac{4}{3}\sqrt{x^3}+C$.
\itemch \textbf{Đúng}. Ta có $F'(x)=-\cos^2 x\cdot (\cos x)'=-\cos ^2 x\cdot (-\sin x)=\sin x\cdot \cos^2 x=g(x)\cdot h^2(x)$.
\end{itemchoice}}
\end{ex} 
\begin{ex}%Câu 15%[2D4H1-4]
Cho các hàm số $g(x)=\dfrac{1}{x^2}$, $h(x)=\ln (x+3)$.
\choiceTF
{ Biết $G(x)$ là một nguyên hàm của $g(x)$ và $ G(1)=1$. Khi đó $ G(2)=-\dfrac{1}{2}$}
{\True $ J=\displaystyle\int \left[h(x)+\ln\dfrac{1}{x+3}+2025\right]\mathrm{\,d}x=2025x+C$}
{$I=\displaystyle\int x\cdot h'(x)\mathrm{\,d}x=x-\ln (x+3)+C$ với $C\in \mathbb{R}$}
{\True Giả sử $F(x)$ là một nguyên hàm của $f(x)=\dfrac{x+3}{g(x)}$ và $F(1)=\dfrac{1}{4}$.\\ Khi đó $F(-1)=-\dfrac{7}{4}$}
\loigiai{
\begin{itemchoice}
\itemch \textbf{Sai}. Ta có $ G(x)=\displaystyle\int{g(x)}{\rm{d}}x=\displaystyle\int{\dfrac{1}{x^2}}{\rm{d}}x=\displaystyle\int{x^{-2}}{\rm{d}}x=-\dfrac{1}{x}+C$.\\
Mà $ G(1)=1\Rightarrow C=2$ $\Rightarrow G(x)=-\dfrac{1}{x}+2\Rightarrow G(2)=\dfrac{3}{2}$.
\itemch \textbf{Đúng}. Ta có 
\begin{eqnarray*}
 J&=&\displaystyle\int \left[h(x)+\ln\dfrac{1}{x+3}+2025\right]\mathrm{\,d}x=2025x+C\\
&=&\displaystyle\int \left[\ln (x+3) + \ln\dfrac{1}{x+3}+2025 \right]\mathrm{\,d}x\\
&=&\displaystyle\int \left(\ln 1+2025\right)\mathrm{\,d}x=2025x+C.
\end{eqnarray*}
\itemch \textbf{Sai}. Ta có $\left[x-\ln\left(x+3\right)+C\right]'=1-\dfrac{1}{x+3}=\dfrac{x+2}{x+3}. \quad(1)$ \\
Và $ x\cdot h'(x)=x\cdot \left(\ln (x+3)\right)'=\dfrac{x}{x+3}. \quad(2)$\\
Từ $(1)$ và $(2)$ suy ra $\displaystyle\int xh'(x)\mathrm{\,d}x \ne x-\ln (x+3)+C$.\\
\itemch \textbf{Đúng}. Ta có 
\begin{eqnarray*}
	\displaystyle\int \dfrac{x+3}{g(x)}\mathrm{\,d}x &=&\displaystyle\int \dfrac{x+3}{\dfrac{1}{x^2}}\mathrm{\,d}x\\
	&=&\displaystyle\int x^2(x+3)\mathrm{\,d}x=\displaystyle\int x^3\mathrm{\,d}x+3\displaystyle\int x^2\mathrm{\,d}x\\
	&=&\dfrac{1}{4}{x^4}+x^3+C.
\end{eqnarray*}
Mà $F(1)=\dfrac{1}{4}\Leftrightarrow C=-1$.\\
$F(x)=\dfrac{1}{4}{x^4}+x^3-1\Rightarrow F(-1)=-\dfrac{7}{4}$.
\end{itemchoice}
}
\end{ex}
\begin{ex}%Câu 16%[2D4H1-4]
Cho các hàm số $g(x)=\mathrm{e}^{\tfrac{x}{2}}$, $h(x)=2x^3+5x^2-2x+4$.
\choiceTF
{\True $\displaystyle\int \left[2g(x)+3h(x)\right]\mathrm{\,d}x=2\displaystyle\int g(x)\mathrm{\,d}x+3\displaystyle\int h(x)\mathrm{\,d}x$}
{\True Một nguyên của hàm số $3\cdot g^2(x)$ là $3\mathrm{e}^x$}
{Họ nguyên của hàm số $h(x)$ là $\dfrac{1}{4}{x^3}+\dfrac{5}{3}{x^3}-x^2+C$}
{\True Biết $\displaystyle\int g^4(x)\cdot h(x)\mathrm{\,d}x=(a{x^3}+b{x^2}+cx+d)\mathrm{e}^{2x}+C$. Khi đó $ a+b+c+d=3$}
\loigiai{
\begin{itemchoice}
\itemch \textbf{Đúng}. Ta có $\displaystyle\int \left[2g(x)+3h(x)\right]\mathrm{\,d}x=2\displaystyle\int g(x)\mathrm{\,d}x+3\displaystyle\int h(x)\mathrm{\,d}x$.
\itemch \textbf{Đúng}. Ta có $\displaystyle\int 3g^2(x)=3\displaystyle\int \mathrm{e}^x\mathrm{\,d}x=3\mathrm{e}^x+C$.
\itemch \textbf{Sai}. Ta có $\displaystyle\int h(x)\mathrm{\,d}x=\displaystyle\int \left(2x^3+5x^2-2x+4\right)\mathrm{\,d}x=\dfrac{1}{2}{x^4}+\dfrac{5}{3}{x^3}-x^2+4x+C$.
\itemch \textbf{Đúng}. Ta có 
\begin{eqnarray*}
	&&\displaystyle\int g^4(x)\cdot h(x)\mathrm{\,d}x=(a{x^3}+b{x^2}+cx+d){\mathrm{e}^{2x}}+C\\
	&\Leftrightarrow& \displaystyle\int{\mathrm{e}^{2x}}(2x^3+5x^2-2x+4)\mathrm{\,d}x=(a{x^3}+b{x^2}+cx+d){\mathrm{e}^{2x}}+C.
\end{eqnarray*}
Nên
\begin{eqnarray*}
	\left((ax^3+bx^2+cx+d)\mathrm{e}^{2x}+C\right)'&=&(3ax^2+2bx+c)\mathrm{e}^{2x}+2\mathrm{e}^{2x}\left(ax^3+bx^2+cx+d\right)\\
	&=&\left(2ax^3+(3a+2b)x^2+(2b+2c)x+c+2d\right)\mathrm{e}^{2x}\\
	&=&(2x^3+5x^2-2x+4)\mathrm{e}^{2x}.
\end{eqnarray*}
Do đó $\heva{&2a=2\\&3a+2b=5\\&2b+2c=-2\\
&c+2d=4}\Leftrightarrow\heva{&a=1\\&b=1\\&c=-2\\&d=3.}$\\
Vậy $a+b+c+d=3$.
\end{itemchoice}
}
\end{ex}
\Closesolutionfile{ans}
\indapan{2}{ans/ans-2-B11-De1-ds}
\Opensolutionfile{ans}[ans/ans-2-B11-De1-kq]
\TNSA
\begin{ex}%Câu 17%[2D4H1-3]
Cho $F(x)$ là một nguyên hàm của hàm $f(x)=\dfrac{\cos 2x}{\sin x+\cos x}$ thỏa mãn $F(0)=1$. Tính $F(\pi)$.
\shortans{-1}
\loigiai{
Ta có
\allowdisplaybreaks
\begin{eqnarray*}
	F(x)&=&\displaystyle\int \dfrac{\cos 2x}{\sin x+\cos x}\mathrm{\,d}x 
	=\displaystyle\int \dfrac{\cos^2x-\sin^2x}{\sin x+\cos x}\mathrm{\,d}x\\
	&=&\displaystyle\int \dfrac{(\sin x+\cos x)(\cos x-\sin x)}{\sin x+\cos x} \mathrm{\,d}x\\
	&=&\displaystyle\int (\cos x-\sin x)\mathrm{\,d}x=\sin x+\cos x+C.
\end{eqnarray*}
Do $F(0)=1$ nên $C=0\Rightarrow F(x)=\sin x+\cos x \Rightarrow F(\pi)=-1$.}
\end{ex}

\begin{ex}%Câu 18%[2D4V1-4]
$F(x)$ là một nguyên hàm của hàm số $f(x)=2^x$, thỏa mãn $ F(0)=\dfrac{1}{\ln 2}$. Biểu thức $ F(0)+F(1)+F(2)+\ldots+F(2024)=\dfrac{a^b-c}{\ln a}$ $(a$, $b$, $c\in{N^*})$. Tính $ T=a+b-2c$.
\shortans{2025}
\loigiai{
Ta có $F(x)=\displaystyle\int 2^x \mathrm{\,d}x=\dfrac{2^x}{\ln 2}+C$.\\
Theo giả thiết $F(0)=\dfrac{1}{\ln 2}\Leftrightarrow\dfrac{2^0}{\ln 2}+C=\dfrac{1}{\ln 2}\Leftrightarrow C=0 \Rightarrow F(x)=\dfrac{2^x}{\ln 2}$.\\
Khi đó 
\allowdisplaybreaks
\begin{eqnarray*}
	F(0)+F(1)+F(2)+\ldots+F(2024)&=&\dfrac{2^0}{\ln 2}+\dfrac{2^1}{\ln 2}+\dfrac{2^2}{\ln 2}+\ldots+\dfrac{2^{2024}}{\ln 2}\\
	&=&\dfrac{1}{\ln 2}(2^0+2^1+2^2+\ldots+2^{2024})\\
	&=&\dfrac{1}{\ln 2}\cdot \dfrac{1(1-2^{2025})}{1-2}=\dfrac{2^{2025}-1}{\ln 2}.
\end{eqnarray*}
$\Rightarrow a=2$, $b=2025$, $c=1$.\\
Vậy $ T=a+b-2c=2025$.}
\end{ex}

\begin{ex}%Câu 19%[2D4V1-2]
Gọi $F(x)$ là một nguyên hàm của hàm số $f(x)=\dfrac{(3x-1)^2}{x^2}$, biết đồ thị hàm số $y=F(x)$ đi qua điểm $M(1;-2)$. Tính $F\left(\mathrm{e}^2\right)$ (làm tròn kết quả đến hàng phần chục).
\shortans{44{,}4}
\loigiai{
Ta có
\allowdisplaybreaks
\begin{eqnarray*}
F(x)&=&\displaystyle\int \dfrac{(3x-1)^2}{x^2}\mathrm{\,d}x=\displaystyle\int \left(\dfrac{9x^2-6x+1}{x^2}\right) \mathrm{\,d}x\\
&=&\displaystyle\int \left(9-\dfrac{6}{x}+\dfrac{1}{x^2}\right) \mathrm{\,d}x\\
&=&9\displaystyle\int \mathrm{\,d}x-6\displaystyle\int\dfrac{1}{x} \mathrm{\,d}x+\displaystyle\int x^{-2} \mathrm{\,d}x\\
&=&9x-6\ln\left| x\right|-\dfrac{1}{x}+C.
\end{eqnarray*}
Theo giả thiết, đồ thị hàm số $y=F(x)$ đi qua điểm $M(1;-2)$ nên suy ra
\allowdisplaybreaks
\begin{eqnarray*}
F(1)=-2&\Rightarrow &9-6\ln 1-1+C=-2\\
&\Rightarrow & C=-10\Rightarrow F(x)=9x-6\ln\left|x\right|-\dfrac{1}{x}-10\\
&\Rightarrow& F\left(e^2\right)=9e^2-6\ln\left|\mathrm{e}^2\right|-\dfrac{1}{\mathrm{e}^2}-10\approx 44{,}4.
\end{eqnarray*}
}
\end{ex}

\begin{ex}%Câu 20%[2D4V1-6]
Một xe ô tô đang chạy với tốc độ $90$ km/h thì người lái xe bất ngờ phát hiện chướng ngại vật trên đường cách đó $150$ m. Người lái xe phản ứng $2$ giây sau đó bằng cách đạp phanh cho xe chạy chậm hơn. Kể từ thời điểm này, ô tô chuyển động chậm dần đều với tốc độ $v(t)=-\dfrac{25}{4}t+25$(m/s), trong đó $t$ là thời gian tính bằng giây kể từ lúc đạp phanh. Quãng đường xe ô tô đã di chuyển kể từ lúc người lái xe phát hiện chướng ngại vật trên đường đến khi xe ô tô dừng hẳn là bao nhiêu mét?\\
\shortans{100}
\loigiai{
Gọi $s(t)$ là quãng đường xe ô tô đi được trong $t$ (giây) kể từ lúc đạp phanh.\\ Khi đó
$ s(t)=\displaystyle\int v(t) \mathrm{\,d}t=\displaystyle\int \left(-\dfrac{25}{4}t+25\right) \mathrm{\,d}t=-\dfrac{25}{8}{t^2}+25t+C$.\\
Do $s(0)=0$ nên $C=0$ . Suy ra $s(t)=-\dfrac{25}{8}{t^2}+25t$.\\
Xe ô tô dừng hẳn khi $ v(t)=0\Leftrightarrow-\dfrac{25}{4}t+25=0\Leftrightarrow t=4$.\\
Suy ra quãng đường xe ô tô còn di chuyển được kể từ lúc đạp phanh đến khi xe dừng hẳn là 
$s(4)=-\dfrac{25}{8}{4^2}+25\cdot 4=50$ (m).\\
Ta có tốc độ $90$ km/h cũng là tốc độ $25$ m/s.\\
Do đó, quãng đường xe ô tô đã di chuyển kể từ lúc người lái xe phát hiện chướng ngại vật trên đường đến khi xe ô tô dừng hẳn là: $ 25\cdot 2+50=100$ (m).
}
\end{ex}

\begin{ex}%Câu 21%[2D4V1-6]
Một quần thể vi khuẩn ban đầu gồm $500$ vi khuẩn, sau đó bắt đầu tăng trưởng. Gọi $P(t)$ là số lượng vi khuẩn của quần thể đó tại thời điểm $t$, trong đó $ t$ tính theo ngày $(0\le t\le 10)$. Tốc độ tăng trưởng của quần thể vi khuẩn đó cho bởi hàm số $P'(t)=k\sqrt{t}$, trong đó k là hằng số. Sau 1 ngày, số lượng vi khuẩn của quần thể đó đã tăng lên thành 600 vi khuẩn (Nguồn: R. Larson and
Edwards, Calculus 10e, Cengage 2014). Tính số lượng vi khuẩn của quần thể đó sau $9$ ngày.

\shortans{3200}
\loigiai{
Ta có $P(t)=\displaystyle\int P'(t)\mathrm{\,d}t=\displaystyle\int k\sqrt{t} \mathrm{\,d}t=\displaystyle\int k\cdot t^{\tfrac{1}{2}}\mathrm{\,d}t=k\cdot\dfrac{2}{3}t\sqrt{t}+C$.\\
Từ giả thiết suy ra $\heva{&P(0)=500\\&P(1)=600
}\Rightarrow\heva{&k\cdot\dfrac{2}{3}\cdot0\sqrt 0+C=500\\&k\cdot\dfrac{2}{3}\cdot1\sqrt 1+C=600}\Rightarrow\heva{&C=500\\&
\dfrac{2}{3}k=100}\Rightarrow\heva{&C=500\\&k=150.}$\\
$\Rightarrow P(t)=100t\sqrt t+500$.\\
Do đó, số lượng vi khuẩn của quần thể đó sau $9$ ngày là $P(9)=100\cdot 9\sqrt{9}+500=3\,200$.}
\end{ex}

\begin{ex}%Câu 22%[2D4V1-6]
Cây cà chua khi trồng có chiều cao $5$ cm. Tốc độ tăng chiều cao của cây cà chua sau khi trồng được cho bởi hàm số $ v(t)=-0{,}1t^3+t^2$, trong đó $t$ tính theo tuần, $v(t)$ tính bằng cm/tuần. Gọi $h(t)$ (tính bằng centimét) là độ cao của cây cà chua ở tuần thứ $t$ (Nguồn:A. Bigalke et al., Mathematik, Grundkurs ma-I, Cornelsen 2016). Vào thời điểm cây cà chua đó phát triển nhanh nhất thì cây cà chua sẽ cao bao nhiêu? (làm tròn kết quả đến hàng phần chục).
\shortans{54{,}4}
\loigiai{
Ta có $h(t)=\displaystyle\int v(t) \mathrm{\,d}t=\displaystyle\int \left(-0{,}1t^3+t^2\right) \mathrm{\,d}t=-\dfrac{1}{40}{t^4}+\dfrac{t^3}{3}+C$.\\
Cây cà chua khi trồng có chiều cao $5$ cm nên $h(0)=5\Rightarrow C=5$.\\
Vậy độ cao của cây cà chua ở tuần thứ $t$ được cho bởi hàm số\\ 
\centerline {$h(t)=-\dfrac{1}{40}{t^4}+\dfrac{t^3}{3}+5$ $(t\ge 0)$.}\\
Ta tìm $ t$ $(t\ge 0)$ sao cho $v(t)$ đạt giá trị lớn nhất.\\
$v'(t)=-0{,}3t^2+2t$; $ v'(t)=0\Leftrightarrow-0{,}3t^2+2t=0\Leftrightarrow\hoac{&t=0\\&t=\dfrac{20}{3}.}$\\
Bảng biến thiên
\begin{center}
	\begin{tikzpicture}
	\tkzTabInit[espcl=2.5,lgt=1.5,nocadre]
	{$x$/0.7,$y'$/0.7,$y$/2.1}
	{$-\infty$,$0$,$\tfrac{20}{3}$,$+\infty$}
	\tkzTabLine{,-,0,+,0,-,}
	\tkzTabVar{+/$+\infty$,-/$0$,+/$\dfrac{400}{27}$,-/$-\infty$}
\end{tikzpicture}
\end{center}
Từ đó ta thấy $v(t)$ đạt giá trị lớn nhất tại $t=\dfrac{20}{3}$.\\
Khi đó, cây cà chua sẽ đạt chiều cao là $h\left(\dfrac{20}{3}\right)=\dfrac{4\,405}{81}\approx 54{,}4$ (cm).
}
\end{ex}
\Closesolutionfile{ans}
\indapan{6}{ans/ans-2-B11-De1-kq}
% \begin{name}
	{NGUYÊN HÀM - TÍCH PHÂN}
	{KT NGUYÊN HÀM}
	{\tentruong}
	{\thoigian}
\end{name}
\setcounter{ex}{0}\setcounter{bt}{0}
\TN
\Opensolutionfile{ans}[ans/ans-2C4B11-De2]
\begin{ex}%[2D4N1-2]
	Họ nguyên hàm của hàm số $f(x)=3x^2+1$ là
	\choice
	{$x^3+C$}
	{$\dfrac{x^3}{3}+x+C$}
	{$6x+C$}
	{\True $x^3+x+C$}
	\loigiai{
	$\displaystyle\int(3x^2+1)\mathrm{d}x=x^3+x+C$.
	}
\end{ex} 
\begin{ex}%[2D4N1-2]
	Hàm số nào sau đây là một nguyên hàm của hàm số $y=12x^5$?
	\choice
	{$y=12x^4$}
	{$y=60x^4$}
	{$y=12x^6+5$}
	{\True $y=2x^6+3$}
	\loigiai{		
		Ta có $\displaystyle\int{12x^5\mathrm{d}\,x}=12\cdot\dfrac{x^6}{6}+C=2x^6+C$.}
\end{ex} 
\begin{ex}%[2D4N1-2]
	Tìm họ nguyên hàm $F(x)$ của hàm số $f(x)=\dfrac{1}{x}$.
	\choice
	{\True  $F(x)=\ln \left| x \right|+C$}
	{$F(x)=\ln x+C$}
	{$F(x)=\ln \left| x \right|$}
	{$F(x)=-\dfrac{1}{x^2}+C$}
	\loigiai{		
		Áp dụng công thức nguyên hàm của hàm số ta có $\displaystyle\int{\frac{1}{x}\mathrm{d}\,x}=\ln \left| x \right|+C$.}
\end{ex} 
\begin{ex}%[2D4N1-3]
	Mệnh đề nào \textbf{sai} trong các mệnh đề sau?
	\choice
	{ $\displaystyle\int\cos x\,\mathrm{d}x=\sin x+C$}
	{\True $\displaystyle\int \sin x \, \mathrm{d}x=\cos x+C$}
	{$\displaystyle\int{\dfrac{1}{\cos^2x}\, \mathrm{d}x=\tan x+C}$}
	{$\displaystyle\int{\dfrac{1}{\sin^2x}\, \mathrm{d}x=-\cot x+C}$}
	\loigiai{
		
		Từ bảng nguyên hàm của các hàm cơ bản suy ra $\displaystyle\int \sin x \, \mathrm{d}x=\cos x+C$ sai}
\end{ex} 
\begin{ex}%[2D4N1-4]
	Tìm nguyên hàm của hàm số $f(x)=7^x$.
	\choice
	{$\displaystyle\int 7^x\mathrm{d}\,x=\dfrac{7^{x+1}}{x+1}+C$}
	{$\displaystyle\int 7^x\mathrm{d}\,x=7^x\ln 7+C$}
	{\True $\displaystyle\int 7^x \mathrm{d}\,x=\dfrac{7^x}{\ln 7}+C$}
	{$\displaystyle\int 7^x\mathrm{d}\,x=7^{x+1}+C$}
	\loigiai{		
		Áp dụng công thức nguyên $\displaystyle\int a^x\mathrm{d}\,x=\dfrac{a^x}{\ln a}+C \Rightarrow \displaystyle  \int 7^x\mathrm{d}\,x=\dfrac{7^x}{\ln 7}+C$.}
\end{ex} 
\begin{ex}%[2D4N1-4]
	Nguyên hàm của hàm số $F(x)=2^x+x$ là
	\choice
	{ $2^x+\dfrac{x^2}{2}+C$}
	{$2^x+x^2+C$}
	{$\dfrac{2^x}{\ln 2}+x^2+C$}
	{\True $\dfrac{2^x}{\ln 2}+\dfrac{x^2}{2}+C$}
	\loigiai{		
		Ta có $\displaystyle\int (2^x+x)\,\mathrm{d}\,x=\dfrac{2^x}{\ln 2}+\dfrac{1}{2} x^2+C$.}
\end{ex} 
\begin{ex}%[2D4N1-4]
	$\displaystyle\int (3^x+4^x)\mathrm{d}\,x$ bằng
	\choice
	{\True  $\dfrac{3^x}{\ln 3}+\dfrac{4^x}{\ln 4}+C$}
	{$\dfrac{3^x}{\ln 4}+\dfrac{4^x}{\ln 3}+C$}
	{$\dfrac{4^x}{\ln 3}-\dfrac{3^x}{\ln 4}+C$}
	{$\dfrac{3^x}{\ln 3}-\dfrac{4^x}{\ln 4}+C$}
	\loigiai{
		Áp dụng công thức $\displaystyle\int a^x\,\mathrm{d}x=\frac{a^x}{\ln a}+C$.\\
		Ta có $\displaystyle\int(3^x+4^x)\mathrm{d}\,x
		=\int 3^x\mathrm{d}\,x+\int 4^x\mathrm{d}\,x=\dfrac{3^x}{\ln 3}+\dfrac{4^x}{\ln 4}+C$.}
\end{ex} 
\begin{ex}%[2D4H1-4]
	Họ nguyên hàm của hàm số $f(x)=\mathrm{e}^x+2x$ là
	\choice
	{ $\dfrac{1}{x+1}\mathrm{e}^x+x^2+C$}
	{$\mathrm{e}^x+2x^2+C$}
	{\True $\mathrm{e}^x+x^2+C$}
	{$\mathrm{e}^x+\dfrac{1}{2} x^2+C$}
	\loigiai{
		Ta có $\displaystyle\int(\mathrm{e}^x+2x)\mathrm{d}\,x=\int\mathrm{e}^x\mathrm{d}\,x+\int 2x\mathrm{d}\,x=\mathrm{e}^x+x^2+C$.}
\end{ex} 
\begin{ex}%[2D4H1-4]
	Trong các mệnh đề sau, mệnh đề nào \textbf{sai}?
	\choice
	{\True  $\displaystyle\int\sin x\mathrm{d}x=\cos x+C$}
	{$\displaystyle\int 2x\mathrm{d}x=x^2+C$}
	{$\displaystyle\int \mathrm{e}^x\mathrm{d}x=\mathrm{e}^x+C$}
	{$\displaystyle\int \dfrac{1}{x}\mathrm{d}x=\ln \left| x \right|+C$}
	\loigiai{
		$\displaystyle\int{\sin x\mathrm{d}x}=-\cos x+C$.}
\end{ex} 
\begin{ex}%[2D4N1-1]
	Khẳng định nào sau đây là \textbf{sai}?
	\choice
	{ Mọi hàm số $f(x)$ liên tục trên đoạn $[a;b]$ đều có nguyên hàm trên đoạn $[a;b]$}
	{\True $\displaystyle\int x^\alpha \mathrm{d}x=\dfrac{x^{\alpha +1}}{\alpha +1}+C$ ($C$ là hằng số, $\alpha $ là hằng số)}
	{$\displaystyle\int \mathrm{e}^x\mathrm{d}x=\mathrm{e}^x+C$ ($C$ là hằng số)}
	{$\displaystyle\int{\dfrac{1}{x}\mathrm{d}x=\ln \left| x \right|+C}$ ($C$ là hằng số) với $x\ne 0$}
	\loigiai{		
		$\displaystyle\int x^{\alpha} \mathrm{d}\,x=\dfrac{x^{\alpha +1}}{\alpha +1}+C$ ($C$ là hằng số, $\alpha $ là hằng số và $\alpha \ne -1$).}
\end{ex} 
\begin{ex}%[2D4N1-4]
	Hàm số nào dưới đây là một nguyên hàm của hàm số $f(x)=\sqrt{x}-1$ trên $(0;+\infty)$?
	\choice
	{$F(x)=\dfrac{1}{2\sqrt{x}}$}
	{$F(x)=\dfrac{1}{2\sqrt{x}}-x$}
	{$F(x)=\dfrac{2}{3}\sqrt[3]{x^2}-x+1$}
	{\True $F(x)=\dfrac{2}{3}\sqrt{x^3}-x+2$}
	\loigiai{		
		Ta có : $\displaystyle\int (\sqrt{x}-1)\mathrm{d}x=\frac{2}{3}\sqrt{x^3}-x+C$.}
\end{ex} 
\begin{ex}%[2D4V2-6]
	Một vật chuyển động với gia tốc $a(t)=\dfrac{3}{t+1}$ (m/s$^2$), trong đó $t$ là khoảng thời gian tính từ thời điểm ban đầu. Vận tốc ban đầu của vật là $6$(m/s). Hỏi vận tốc của vật tại giây thứ $8$ là bao nhiêu?
	\choice
	{\True  $12{,}6$ (m/s)}
	{$12{,}2$ (m/s)}
	{$6{,}6$ (m/s)}
	{$12{,}4$ (m/s)}
	\loigiai{
		Vận tốc của vật tại thời điểm t được tính theo công thức
		$$v(t)=\int a(t)\mathrm{d}t=\displaystyle \int \dfrac{3}{t+1}\mathrm{d} t=3\ln (t+1)+C.$$
		Do vận tốc ban đầu của vật bằng $6$ (m/s) nên ta có:
		$$v(0)=3\ln (0+1)+C=6\Rightarrow C=6\Rightarrow v(t)=3\ln (t+1)+6.$$
		Vận tốc chuyển động của vật tại giây thứ $8$ là:
		$$v(8)=3\ln (8+1)+6=3\ln 9+6\approx 12{,}6 \text{ (m/s)}.$$
}
\end{ex} 

\Closesolutionfile{ans}
% \indapan{6}{ans/ans-2C4B11-De2}
\Opensolutionfile{ans}[ans/ans-2C4B11-De2-ds]
\TNTF
\setcounter{ex}{0}
\begin{ex}%[2D4H1-3]
	Cho hàm số $f(x)=\sin \dfrac{x}{2}$ và hàm số $g(x)=\cos \dfrac{x}{2}$ .
	\choiceTF
	{$F(x)=2\cos \dfrac{x}{2}$  là một nguyên hàm của hàm số $f(x)$}
	{\True $G(x)=2\sin \dfrac{x}{2}+\dfrac{1}{2}$  là một nguyên hàm của hàm số $g(x)$}
	{$\displaystyle\int \left[ f(x)-g(x) \right]^2 \mathrm{d}x=x+\cos x+C$ ($C$ là một hằng số)}
	{\True $\displaystyle\int \dfrac{1}{[2f(x)\cdot g(x)]^2}\mathrm{d}x=-\cot x+C$ ($C$ là một hằng số)}
\loigiai{
	\begin{itemchoice}
	\itemch Vì $F'(x)=-\sin \dfrac{x}{2},\forall x\in R$ nên  $F(x)=2\cos \dfrac{x}{2}$  không là một nguyên hàm của hàm số $F(x)$ trên $\mathbb{R}$.  Sai
	\itemch Vì $G'(x)=\cos \dfrac{x}{2}, \forall x\in \mathbb{R}$ nên  $G(x)=2\sin \dfrac{x}{2}+\dfrac{1}{2}$ là một nguyên hàm của hàm số $g(x)$ trên $R$.  Đúng
	\itemch $\displaystyle\int [f(x)-g(x)]^2\mathrm{d}x=\int \left( \sin\dfrac{x}{2}-\cos\frac{x}{2} \right)^2\mathrm{d}x=\int \left(\sin ^2\frac{x}{2}+2\sin\frac{x}{2}\cos\frac{x}{2}+\cos^2\frac{x}{2}\right)\mathrm{d}x=\int( 1+\sin x )\mathrm{d}x=x-\cos x+C$.  Sai
	\itemch $\displaystyle\int \frac{1}{[2f(x)\cdot g(x)]^2}\mathrm{d}x=\int \frac{1}{(2\sin\frac{x}{2}\cos\frac{x}{2})^2}\mathrm{d}x=\int \frac{1}{\sin^2 x}\mathrm{d}x=-\cot x+C$.  Đúng
	\end{itemchoice}
	}
	\end{ex} 
	\begin{ex}%[2D4H1-4]
		Cho hàm số $f(x)=\dfrac{1}{x}$ và $F(x)=\ln x+C_1$, $G(x)=\ln (-x)+C_2$ ($C_1,C_2$ là các hằng số).
		\choiceTF
{\True Trên $(0;+\infty)$, một nguyên hàm của hàm số $f(x)$ là $H(x)=\ln (x)+e$}
{\True Trên $(-\infty ;0)$, nguyên hàm của hàm số $f(x)$ là $G(x)$}
{\True Trên $(0;+\infty)$, nguyên hàm của hàm số $f(x)$ là $F(x)$}
{\True $\displaystyle\int \left[ f(x)+f^2(x) \right]\mathrm{d}x=\ln (3\left| x \right|)-\dfrac{1}{x}+C$ ($C$ là một hằng số)}
\loigiai{
		\begin{itemchoice}
\itemch Vì $H'(x)=\dfrac{1}{x}=F(x),\forall x\in (0;+\infty)$ nên $H(x)$ là một nguyên hàm của hàm số $F(x)$ trên ($0,+\infty $).  Đúng
\itemch $\displaystyle\int f(x)\mathrm{d}x=\int \frac{1}{x}\mathrm{d}x=\ln \left( \left| x \right| \right)+C_2=\ln (-x)+C_2,\forall x\in (-\infty ;0)$.  Đúng
\itemch $\displaystyle\int f(x)\mathrm{d}x=\int \frac{1}{x}\mathrm{d}x=\ln \left( \left| x \right| \right)+C_1=\ln x+C_1,\forall x\in( 0;+\infty)$.  Đúng
\itemch $\displaystyle\int \left[ f(x)+f^2(x) \right]\mathrm{d}x=\int \left( \frac{1}{x}+\frac{1}{x^2} \right)\mathrm{d}x=\ln (\left| x \right|)-\frac{1}{x}+C_3=\ln \left( \left| x \right| \right)-\frac{1}{x}+\ln 3+C=\ln ( 3\left| x \right|)-\frac{1}{x}+C$.  Đúng 
	\end{itemchoice}}
\end{ex} 
\begin{ex}%[2D4V1-4]
	Cho hàm số $f(x)=\cos x$ và hàm số $g(x)=\sin x$.
	\choiceTF
{\True $F(x)=\sin x+\mathrm{e}$ là một nguyên hàm của hàm số $f(x)$ trên $\mathbb{R}$}
{$G(x)={\mathrm{e}^{-\cos x}}+\ln 3$ là một nguyên hàm của hàm số $\mathrm{e}^{g(x)}$ trên $\mathbb{R}$}
{\True $\displaystyle\int \left[ 5f(x)+6g(x) \right]\mathrm{d}x=5\sin x-6\cos x+C$, ($C$ là một hằng số)}
{\True $\displaystyle\int \left[ 2+\left( \frac{g(x)}{f(x)} \right)^2 \right]\mathrm{d}x=x+\tan x+C$, ($C$ là một hằng số)}
\loigiai{
		\begin{itemchoice}
\itemch Vì $F'(x)=\cos x=f(x),\forall x\in \mathbb{R}$ nên $(x)$ là một nguyên hàm của hàm số $f(x)$ trên $\mathbb{R}$.  Đúng
\itemch Vì $G'(x)=\sin x \mathrm{e}^{-\cos x}\ne \mathrm{e}^{\sin x}, \forall x\in \mathbb{R}$ nên $G(x)$ không là một nguyên hàm của hàm số $\mathrm{e}^{g(x)}$ trên $\mathbb{R}$. Sai
\itemch $\displaystyle\int \left[ 5f(x)+6g(x) \right]\mathrm{d}x=\int \left( 5\cos x+6\sin x \right)\mathrm{d}x=5\sin x-6\cos x+C$.  Đúng
\itemch $\displaystyle\int \left[ 2+\left( \frac{g(x)}{f(x)} \right)^2 \right]\mathrm{d}x=\int \left(2+\frac{\sin^2 x}{\cos^2 x}\right)\mathrm{d}x=\int \left( 1+\frac{\sin^2 x+\cos^2 x}{\cos^2 x} \right)\mathrm{d}x\\
=\int \left( 1+\frac{1}{\cos^2 x} \right)\mathrm{d}x=x+\tan x+C$.  Đúng
	\end{itemchoice} }
\end{ex} 
\begin{ex}%[2D4V1-4]
	Cho hàm số $f(x)=3^{2x}$ và hàm số $g(x)=\tan x$.
	\choiceTF
{$F(x)=\dfrac{3^{2x}\ln 3}{2}$  là một nguyên hàm của hàm số $f(x)$ trên $\mathbb{R}$}
{\True $G(x)=-\ln (3\cos x)$ là một nguyên hàm của hàm số $g(x)$ trên $\left( -\dfrac{\pi}{2};\dfrac{\pi}{2} \right)$}
{\True $\displaystyle\int 3f(x)\mathrm{d}x=\dfrac{3^{2x+1}}{\ln 9}+C$, ($C$ là một hằng số)}
{\True $\displaystyle\int [f(x)+g(x)^2]\mathrm{d}x =\dfrac{9^x}{2\ln 3}-x+\tan x+C$, ($C$ là một hằng số)}
\loigiai{
	\begin{itemchoice}
\itemch Vì $F'(x)=\dfrac{2\cdot 3^{2x}\cdot\ln ^23}{2}=3^{2x}\cdot \ln ^2 3\ne f(x),\forall x\in \mathbb{R}$ nên $f(x)$ không là một nguyên hàm của hàm số $F(x)$ trên $\mathbb{R}$.  Sai
\itemch Vì $G'(x)=-\dfrac{-3\sin x}{3\cos x}=\tan x=g(x),\forall x\in \left( -\dfrac{\pi }{2};\dfrac{\pi }{2} \right)$ nên $G(x)$ là một nguyên hàm của hàm số $g(x)$ trên $\left( -\dfrac{\pi }{2};\dfrac{\pi }{2} \right)$.  Đúng
\itemch $\displaystyle\int 3f(x)\mathrm{d}x=\int 3\cdot 3^{2x}\mathrm{d}x=3\cdot 9^x\mathrm{d}x=3\cdot\dfrac{9^x}{\ln 9}+C=\dfrac{3\cdot3^{2x}}{\ln 9}+C=\dfrac{3^{2x+1}}{\ln 9}+C$.  Đúng
\itemch $\displaystyle\int \left[f(x)+g(x)^2\right]\mathrm{d}x=\int \left( 3^{2x}+\tan^2x\right)\mathrm{d}x=\int \left(9^x-1+1+\tan ^2\right)\mathrm{d}x\\
=\int \left(9^x-1+\dfrac{1}{\cos^2 x}\right)\mathrm{d}x
=\dfrac{9^x}{\ln 9}-x+\tan x+C=\dfrac{9^x}{2\ln 3}-x+\tan x+C$.  Đúng
	\end{itemchoice}
}
\end{ex} 

\Closesolutionfile{ans}
% \indapan{2}{ans/ans-2C4B11-De2-ds}
\Opensolutionfile{ans}[ans/ans-2C4B11-De2-kq]
\TNSA
\setcounter{ex}{0}
\begin{ex}%[2D4V2-4]
	Giả sử hàm số $y=f(x)$ liên tục và thỏa mãn: $f(1)=1$ và $f'(x)\sqrt[3]{x^{-1}}=1$, với mọi $x>0$. Tính $4f(8)$.
	\shortans{$47$}
	\loigiai{
		Ta có $f'(x)=\dfrac{1}{\sqrt[3]{x^{-1}}}=\dfrac{1}{x^{-\tfrac{1}{3}}}=x^{\tfrac{1}{3}}$\\
		$\Rightarrow F(x)=\displaystyle\int f'(x)\mathrm{d}x= \int x^{\tfrac{1}{3}}\mathrm{d}x=\dfrac{3}{4}x^{\frac{4}{3}}+C=\frac{3}{4}\sqrt[3]{x^4}+C$.\\
		$f(1)=1\Rightarrow \dfrac{3}{4}+C=1\Rightarrow C=-\dfrac{1}{4}.\\
		\Rightarrow f(x)=\dfrac{3}{4}\sqrt[3]{x^4}-\dfrac{1}{4}$\\
		$\Rightarrow 4f(8)=47$.}
\end{ex} 
\begin{ex}%[2D4V2-6]
Một ô tô đang chạy với vận tốc $10$(m/s) thì người lái xe đạp phanh. Từ thời điểm đó, ô tô chuyển động chậm dần đều với vận tốc $v(t)=10-2t$ (m/s), trong đó $t$ là khoảng thời gian tính bằng giây kể từ lúc đạp phanh. Tính quãng đường ô tô di chuyển được trong $8$ giây cuối cùng.
\shortans{$55$}
\loigiai{
	Chọn mốc thời gian và gốc tọa độ lúc ô tô bắt đầu đạp phanh. Suy ra $t=0;\,s=0$.\\
	$s(t)=\displaystyle \int v(t)\mathrm{d}t=\int (10-2t)\mathrm{d}t=10t-t^2+C$.\\
	$s(0)=0\Rightarrow C=0 \Rightarrow s(t)=10t-t^2$.\\ 
	Ô tô dừng hẳn khi $v(t)=0\Leftrightarrow 10-2t=0\Leftrightarrow t=5$.\\
	Trong $8$ giây cuối:
	\begin{itemize}
		\item ô tô chuyển động đều với vận tốc $10$(m/s) trong $3$ giây đầu.
		\item ô tô chuyển động chậm dần đều trong $5$ giây cuối.
	\end{itemize}
	Quãng đường ô tô di chuyển là: $s=3\cdot 10+10\cdot 5-5^2=55$ m.}

\end{ex} 
\begin{ex}%[2D4V2-4]
	Gọi $F(x)$ là một nguyên hàm của hàm số $f(x)=3^{2x+1} 2^{1+3x}$, biết $F(0)=\dfrac{8}{\ln 72}$. Tính $F(-2)$. (làm tròn kết quả đến hàng phần trăm).
	\shortans{$0{,}47$}
	\loigiai{
		Ta có:\\
		$F(x)=\displaystyle \int{\left(3^{2x+1}\cdot 2^{1+3x} \right)}\mathrm{d}x=\int\left(3\cdot3^{2x}\cdot2\cdot2^{3x}\right)\mathrm{d}x=\int\left(6\cdot9^x\cdot8^x \right)\mathrm{d}x\\
		=6\int 72^x\mathrm{d}x=6\cdot\dfrac{72^x}{\ln 72}+C$.\\
		Theo giả thiết, $F(0)=\dfrac{8}{\ln 72}\Rightarrow 6\cdot\dfrac{72^0}{\ln 72}+C=\dfrac{8}{\ln 72}\Rightarrow C=\dfrac{2}{\ln 72}$\\
		$\Rightarrow F(x)=6\cdot\dfrac{{{72}^{x}}}{\ln 72}+\dfrac{2}{\ln 72}\Rightarrow F\left( -2 \right)=6\cdot \dfrac{72^{-2}}{\ln 72}+\dfrac{2}{\ln 72}\approx 0{,}47$.
		}
\end{ex} 
\begin{ex}%[2D4V2-6]
Một viên đạn được bắn thẳng đứng lên từ độ cao $1{,}5$ mét so với mặt đất. Giả sử tại thời điểm $t$ giây (coi $t=0$ là thời điểm viên đạn được bắn lên), vận tốc của nó được cho bởi $v(t)=170-9{,}8\,t\,\left( \text{m/s} \right)$. Tìm độ cao lớn nhất của viên đạn (làm tròn kết quả đến hàng đơn vị).
	\shortans{$1476$}
	\loigiai{
		Gọi $h(t)$ là độ cao của viên đạn tại thời điểm $t$ giây sau khi bắn. Ta có:\\
		$h(t)=\displaystyle \int v(t)\mathrm{d}t=\int{(170-9{,}8t)}\mathrm{d}t=170t-4{,}9t^2+C$.\\
		Từ giả thiết suy ra: $h\left( 0 \right)=1,5\Rightarrow C=1{,}5\Rightarrow h(t)=170t-4{,}9t^2+1,5$.\\
		Viên đạn đạt độ cao lớn nhất khi $v(t)=0\Leftrightarrow 170-9,8\,t\,=0\Leftrightarrow t=\dfrac{850}{49}$.\\
		Khi đó, độ cao lớn nhất của viên đạn là:\\
		$h\left(\dfrac{850}{49}\right)=170 \cdot\dfrac{850}{49}-4{,}9\left( \dfrac{850}{49} \right)^2+1{,}5=\dfrac{144647}{98}\approx 1476$ (m).}
\end{ex} 
\begin{ex}%[2D4V2-6]
Một chiếc cốc chứa nước ở $95^\circ$ C được đặt trong phòng có nhiệt độ ${{20}^{0}}C$. Theo định luật làm mát của Newton, nhiệt độ của nước trong cốc sau $t$ phút (xem $t=0$ là thời điểm nước ở $95^\circ$ C là một hàm số $(t)$. Tốc độ giảm nhiệt độ của nước trong cốc tại thời điểm t phút được xác định bởi $T'(t)=\left(-\dfrac{3}{2} \mathrm\mathrm{e}^{-\tfrac{t}{50}}\right)^\circ$ C/phút). Tính nhiệt độ của nước tại thời điểm $t=40$ phút (làm tròn kết quả đến hàng phần chục).
	\shortans{$53{,}7$}
	\loigiai{
		Ta có:\\
$\displaystyle T(t)=\int T'(t)\mathrm{d}t
=\int\left( -\frac{3}{2}\mathrm{e}^{-\tfrac{t}{50}} \right)\mathrm{d}t
=-\frac{3}{2}\int\left({\mathrm{e}^{-\tfrac{1}{50}}} \right)^t\mathrm{d}t\\
=-\frac{3}{2}\cdot\frac{\left(\mathrm{e}^{-\tfrac{1}{50}} \right)^t}{\ln \left(\mathrm{e}^{-\tfrac{1}{50}}\right)}+C
=75\left(\mathrm{e}^{-\frac{1}{50}}\right)^t+C$.\\
		Vì $t=0$ là thời điểm nước ở $95^\circ$ C nên $T(0)=95\Rightarrow 75\left(\mathrm{e}^{-\tfrac{1}{50}} \right)^\circ+C=95\Rightarrow C=20$.\\ 
		Suy ra $T(t)=75\left(\mathrm{e}^{-\frac{1}{50}} \right)^t+20$.\\
		Do đó, nhiệt độ của nước tại thời điểm $t=40$ phút là: \\
		$T(40)=75\left(\mathrm{e}^{-\tfrac{1}{50}} \right)^{40}+20\approx 53{,}7 ^\circ$ C.}
\end{ex} 
\begin{ex}%[2D4V2-6]
	Doanh thu bán hàng của một công ty khi bán một loại sản phẩm là số tiền $R(x)$ (triệu đồng) thu được khi $x$ đơn vị sản phẩm được bán ra. Tốc độ biến động (thay đổi) của doanh thu khi $x$ đơn vị sản phẩm đã được bán là hàm số $M_R(x)=R'(x)$. Một công ty công nghệ cho biết, tốc độ biến đổi của doanh thu khi bán một loại con chip của hãng được cho bởi $M_R(x)=40-0{,}1x$, ở đó $x$ là số lượng chip đã bán. Hỏi doanh thu của công ty khi đã bán 500 con chip bằng bao nhiêu tỉ đồng?
	\shortans{$7{,}5$}
	\loigiai{
		Vì $R'(x)=M_R(x)$ nên doanh thu $R(x)$ là một nguyên hàm của $M_R(x)$.\\
		Ta có: $R(x)=\displaystyle \int M_R(x) \mathrm{d}x=\int{(40-0{,}1x)}\mathrm{d}x=40x-0{,}05 x^2+C$.\\
		Khi $x=0$, tức là chưa bán chip nào thì doanh thu sẽ bằng $0$ (triệu đồng), nên $R\left( 0 \right)=0\Rightarrow C=0$.\\
		Suy ra $R(x)=40x-0{,}05 x^2$.\\
		Do đó, doanh thu của công ty khi đã bán 500 con chip là:\\
		$R(500)=40\cdot 500-0{,}05\cdot 500^2=7500$ (triệu đồng) $=7{,}5$ (tỉ đồng).		
	}
\end{ex}
\Closesolutionfile{ans}
% \indapan{6}{ans/ans-2C4B11-De2-kq}
% \begin{name}
	{NGUYÊN HÀM - TÍCH PHÂN}
	{KT TÍCH PHÂN}
	{\tentruong}
	{\thoigian}
\end{name}
\setcounter{ex}{0}\setcounter{bt}{0}
\Opensolutionfile{ans}[ans/ans-2-B12-De1-NLC]
\TN
\begin{ex}%[2D4N2-1]
	Biết $\displaystyle\displaystyle\int\limits f(x) \mathrm{\,d} x=F(x)+C$. Trong các khẳng định sau, khẳng định nào đúng?
	\choice 
		{$\displaystyle\displaystyle\int\limits\limits_a^b f(x) \mathrm{\,d} x=F(b) \cdot F(a)$}
		{$\displaystyle\displaystyle\int\limits\limits_a^b f(x) \mathrm{\,d}x=F(a)-F(b)$}
		{\True $\displaystyle\displaystyle\int\limits\limits_a^b f(x) \mathrm{\,d}x=F(b)-F(a)$}
		{$\displaystyle\displaystyle\int\limits\limits_a^b f(x) \mathrm{\,d} x=F(b)+F(a)$}
	\loigiai{
		Ta có $\displaystyle\displaystyle\int\limits\limits_a^b f(x) \mathrm{\,d}x=F(b)-F(a)$.
	}
\end{ex}
\begin{ex}%[2D4N2-2]
	Tính tích phân $\displaystyle\displaystyle\int\limits\limits_1^2(2a x+b) \mathrm{\,d} x$.
	\choice 
		{\True $3a+b$}
		{$3a+2b$}
		{$a+2 b$}
		{$a+b$}
	\loigiai{
		Ta có $\displaystyle\displaystyle\int\limits\limits_1^2(2 a x+b) \mathrm{\,d} x=\left(a x^2+b x\right)\Big|_1 ^2=4 a+2 b-(a+b)=3 a+b$.
	}
\end{ex}
\begin{ex}%[2D4H2-1]
	Biết $\displaystyle\int\limits_1^8 f(x) \mathrm{\,d} x=-2$, $\displaystyle\int\limits_1^4 f(x) \mathrm{\,d} x=3$ và $\displaystyle\int\limits_1^4 g(x) \mathrm{\,d} x=7$. Mệnh đề nào sau đây \textbf{sai}?
	\choice 
		{$\displaystyle\int\limits_1^4\left[4 f(x)-2 g(x)\right] \mathrm{d} x=-2$}
		{$\displaystyle\int\limits_4^8 f(x) \mathrm{\,d} x=1$}
		{$\displaystyle\int\limits_1^4\left[f(x)+g(x)\right] \mathrm{d} x=10$}
		{\True $\displaystyle\int\limits_4^8 f(x) \mathrm{\,d} x=-5$}
	\loigiai{
		Ta có
		 $\displaystyle\int\limits_4^8 f(x) \mathrm{\,d} x=\displaystyle\int\limits_1^8 f(x) \mathrm{\,d} x-\displaystyle\int\limits_1^4 f(x) \mathrm{\,d} x=-2-3=-5$.
	}
\end{ex}
\begin{ex}%[2D4N2-4]
	Tích phân $I=\displaystyle\int\limits_0^{2018} 2^x \mathrm{\,d} x$ bằng
	 \choice 
		{$\dfrac{2^{2018}}{\ln 2}$}
		{$2^{2018}$}
		{$2^{2018}-1$}
		{\True $\dfrac{2^{2018}-1}{\ln 2}$}
	\loigiai{
		Ta  có
		$I=\displaystyle\int\limits_0^{2018} 2^x \mathrm{\,d} x=\dfrac{2^x}{\ln 2}\,\bigg|_0 ^{2018}=\dfrac{2^{2018}-1}{\ln 2}$.
	}
\end{ex}
\begin{ex}%[2D4N2-3]
	Tích phân $I=\displaystyle\int\limits_{\tfrac{\pi}{4}}^{\tfrac{\pi}{3}} \dfrac{\mathrm{\,d} x}{\sin ^2 x}$ bằng
	\choice 
		 {$\cot \dfrac{\pi}{3}-\cot \dfrac{\pi}{4}$} 
		 {$\cot \dfrac{\pi}{3}+\cot \dfrac{\pi}{4}$}
		 {\True $-\cot \dfrac{\pi}{3}+\cot \dfrac{\pi}{4}$}
	 	 {$-\cot \dfrac{\pi}{3}-\cot \dfrac{\pi}{4}$}
	\loigiai{
		Ta có $I=\displaystyle\int\limits_{\tfrac{\pi}{4}}^{\tfrac{\pi}{3}} \dfrac{\mathrm{\,d} x}{\sin ^2 x}=-\cot x\,\bigg|_{\tfrac{\pi}{4}} ^{\tfrac{\pi}{3}}=-\cot \dfrac{\pi}{3}+\cot \dfrac{\pi}{4}$.
	}
\end{ex}
\begin{ex}%[2D4N2-2]
	Tính tích phân $I=\displaystyle\int\limits_1^2\left(\dfrac{2}{x}-\dfrac{1}{x^2}\right) \mathrm{d} x$.
	\choice 
		{$I=2 \ln 2$}
		{\True $I=2 \ln 2-\dfrac{1}{2}$}
		{$I=2 \mathrm{e}+\dfrac{1}{2}$}
		{$I=0$}
	\loigiai{
		Ta có $I=\displaystyle\int\limits_1^2\left(\dfrac{2}{x}-\dfrac{1}{x^2}\right) \mathrm{d} x=\left(2 \ln |x|+\dfrac{1}{x}\right)\bigg|_1 ^2=\left(2 \ln 2+\dfrac{1}{2}\right)-(2 \ln 1+1)=2 \ln 2-\dfrac{1}{2}$.
	}
\end{ex}
\begin{ex}%[2D4H2-3]
	Tính tích phân $I=\displaystyle\int\limits_0^{\tfrac{\pi}{4}} \tan ^2 x \mathrm{\,d} x$. 
	\choice 
		{$I=2$}
		{$I=\ln 2$}
		{$I=\dfrac{\pi}{12}$}
		{\True $I=1-\dfrac{\pi}{4}$}
	\loigiai{
		Ta có
		\begin{eqnarray*}
			I=\displaystyle\int\limits_0^{\tfrac{\pi}{4}} \tan ^2 x \mathrm{\,d} x=\displaystyle\int\limits_0^{\tfrac{\pi}{4}}\left( \dfrac{1}{\cos^2x}-1\right)  \mathrm{d} x&=&\left( \tan x-x\right) \bigg|_0^{\tfrac{\pi}{4}}\\
			&=&\left( \tan \dfrac{\pi}{4}-\dfrac{\pi}{4}\right)-\left( \tan 0-0\right)=1-\dfrac{\pi}{4}. 
		\end{eqnarray*} 
		
	}
\end{ex}
\begin{ex}%[2D4H2-2]
	Cho $a$, $b$ là các số thực dương thỏa mãn $\sqrt{a}-\sqrt{b}+1=0$. Tính tích phân $I=\displaystyle\int\limits_a^b \dfrac{\mathrm{\,d} x}{\sqrt{x}}$.
	\choice 
		{$I=-2$}
		{$I=1$}
		{$I=\dfrac{1}{2}$}
		{\True $I=2$}
	\loigiai{
		Ta có\\ $I=\displaystyle\int\limits_a^b \dfrac{\mathrm{\,d} x}{\sqrt{x}}=\displaystyle\int\limits_a^b x^{-\tfrac{1}{2}} \mathrm{\,d} x=2 \sqrt{x}\,\bigg|_a ^b=2\left( \sqrt{b}-\sqrt{a}\right) =2\left( 1-\left(\sqrt{a}-\sqrt{b}+1\right) \right) =2\cdot 1=2 $.
	}
\end{ex}
\begin{ex}%[2D4N2-4]
	Cho $\displaystyle\int\limits_2^5 \dfrac{\mathrm{\,d} x}{x}=\ln a$. Tìm $a$. 
	\choice 
		{$2$}
		{$\dfrac{2}{5}$}
		{\True $\dfrac{5}{2}$}
		{$5$}
	\loigiai{ 
		Ta có $\displaystyle\int\limits_2^5 \dfrac{\mathrm{\,d} x}{x}=\ln a \Leftrightarrow \ln |x|\, \bigg|_2^5=\ln a \Leftrightarrow \ln 5-\ln 2=\ln a \Leftrightarrow \ln \dfrac{5}{2}=\ln a \Leftrightarrow a=\dfrac{5}{2}$.
	}
\end{ex}
\begin{ex}%[2D4H2-2]
	Cho hàm số $f(x)$ liên tục trên $\mathbb{R}$ và $\displaystyle\int\limits_0^2\left( f(x)+2 x\right)  \mathrm{d} x=5$. Tính $\displaystyle\int\limits_0^2 f(x) \mathrm{\,d} x$.
	\choice
		{$-9$}
		{$-1$}
		{$9$}
		{\True $1$}
	\loigiai{
		Ta có $\displaystyle\int\limits_0^2\left( f(x)+2 x\right)  \mathrm{d} x=\displaystyle\int\limits_0^2 f(x) \mathrm{\,d} x+\displaystyle\int\limits_0^2 2 x \mathrm{\,d} x=\displaystyle\int\limits_0^2 f(x) \mathrm{\,d} x+4=5$. Do đó $\displaystyle\int\limits_0^2 f(x) \mathrm{\,d} x=1$.
	}
\end{ex}
\begin{ex}%[2D4H2-1]
	Cho hai tích phân $\displaystyle\int\limits_{-2}^5 f(x) \mathrm{\,d} x=8$ và $\displaystyle\int\limits_5^{-2} g(x) \mathrm{\,d} x=3$. Tính $I=\displaystyle\int\limits_{-2}^5\left[ f(x)-4 g(x)-1\right] \mathrm{d} x$.
	\choice 
		{$I=-11$}
		{\True $I=13$}
		{ $I=27$}
		{ $I=3$}
	\loigiai{
		Ta có\\ $I=\displaystyle\int\limits_{-2}^5\left[ f(x)-4 g(x)-1\right]  \mathrm{d} x=\displaystyle\int\limits_{-2}^5 f(x) \mathrm{\,d} x+4 \displaystyle\int\limits_{5}^{-2} g(x) \mathrm{\,d} x-x\,\bigg|_{-2} ^5=8+4\cdot 3-(5+2)=13$.
	}
\end{ex}
\begin{ex}%[2D4H2-2]
	Cho hàm số $y=f(x)=\heva{&3 x^2 & \text { khi } 0 \leq x \leq 1 \\ &4-x & \text { khi } 1 \leq x \leq 2}$. Tính tích phân $\displaystyle\int\limits_0^2 f(x) \mathrm{\,d} x$.
	\choice 
		{\True $\dfrac{7}{2}$}
		{$1$}
		{$\dfrac{5}{2}$}
		{$\dfrac{3}{2}$}
	\loigiai{ 
		Ta có 
		\begin{eqnarray*}
		\displaystyle\int\limits_0^2 f(x) \mathrm{\,d} x&=&\displaystyle\int\limits_0^1 f(x) \mathrm{\,d} x+\displaystyle\int\limits_1^2 f(x) \mathrm{\,d} x=\displaystyle\int\limits_0^1\left(3 x^2\right) \mathrm{d} x+\displaystyle\int\limits_1^2(4-x) \mathrm{\,d} x\\
		&=& x^3\,\bigg|_0 ^1+\left(4 x-\dfrac{x^2}{2}\right)\bigg|_1 ^2=\left( 1^3-0^3\right) +\left[ \left(4\cdot 2-\dfrac{2^2}{2}\right)-\left( 4\cdot1-\dfrac{1^2}{2}\right) \right] = \dfrac{7}{2}.	
		\end{eqnarray*}
	}
\end{ex}
\Closesolutionfile{ans}
% \indapan{6}{ans/ans-2-B12-De1-NLC}
\TNTF
\Opensolutionfile{ans}[ans/ans-2-B12-De1-DS]
\begin{ex}%[2D4V2-2]
	Cho $f(x)$ và $g(x)$ là các hàm số liên tục bất kì trên đoạn $[a;b]$.
	\choiceTF
		{\True $\displaystyle\int\limits_a^b\left(f(x)-g(x)\right) \mathrm{d} x=\displaystyle\int\limits_a^b f(x) \mathrm{\,d} x-\displaystyle\int\limits_a^b g(x) \mathrm{\,d} x$}
		{$\displaystyle\int\limits_a^a\left[f(x)+g(x)\right] \mathrm{d} x=1$}
		{Nếu $\displaystyle\int\limits_a^b f(x) \mathrm{\,d} x=3$ và $\displaystyle\int\limits_a^b\left[3 f(x)-g(x)\right] \mathrm{d} x=10$ thì $\displaystyle\int\limits_a^b g(x) \mathrm{\,d} x=1$}
		{\True Nếu $f(x)+2 f\left(\dfrac{1}{x}\right)=3 x$ với $x \in\left[\dfrac{1}{2}; 2\right]$. Tính $\displaystyle\int\limits_{\tfrac{1}{2}}^2 \dfrac{f(x)}{x} \mathrm{\,d} x=\dfrac{3}{2}$}
	\loigiai{
		\begin{itemchoice}
			\itemch Đúng. Do tính chất tích phân.
			\itemch Sai. Ta có $\displaystyle\int\limits_a^a\left[f(x)+g(x)\right] \mathrm{d} x=0$.
			\itemch Sai. Ta có 
			\begin{eqnarray*}
				\displaystyle\int\limits_a^b\left[ 3 f(x)-g(x)\right]  \mathrm{d} x=10 &\Leftrightarrow& 3 \displaystyle\int\limits_a^b f(x) \mathrm{\,d} x-\displaystyle\int\limits_a^b g(x) \mathrm{\,d} x=10\\ &\Leftrightarrow& 3\cdot 3-\displaystyle\int\limits_a^b g(x) \mathrm{\,d} x=10 \Leftrightarrow\displaystyle\int\limits_a^b g(x) \mathrm{\,d} x=-1.
			\end{eqnarray*}
			\itemch Đúng. Ta có $f(x)+2 f\left(\dfrac{1}{x}\right)=3 x \Rightarrow f\left(\dfrac{1}{x}\right)+2 f(x)=\dfrac{3}{x}$.\\
			Suy ra $\heva{&f(x)+2 f\left(\dfrac{1}{x}\right)=3x \\& 4 f(x)+2 f\left(\dfrac{1}{x}\right)=\dfrac{6}{x}} \Rightarrow f(x)=\dfrac{2}{x}-x\Rightarrow \dfrac{f(x)}{x}=\dfrac{2}{x^2}-1$.\\
			Do đó $\displaystyle\int\limits_{\tfrac{1}{2}}^2 \dfrac{f(x)}{x} \mathrm{\,d} x=\displaystyle\int\limits_{\tfrac{1}{2}}^2\left(\dfrac{2}{x^2}-1\right) \mathrm{d} x=\dfrac{3}{2}$.
		\end{itemchoice}
	}
\end{ex}
\begin{ex}%[2D4V2-2]
	Cho các số thực $a$, $b$ $(a<b)$. Nếu hàm số $y=f(x)$ có đạo hàm là hàm liên tục trên $\mathbb{R}$ và $\displaystyle\displaystyle\int\limits f(x) \mathrm{\,d} x=F(x)+C$.
	\choiceTF
		{$\displaystyle\int\limits_a^b f(x) \mathrm{\,d} x=F(a)-F(b)$}
		{\True $\displaystyle\int\limits_a^b f'(x) \mathrm{\,d} x=f(b)-f(a)$}
		{Nếu $\displaystyle\int\limits_0^2 f(x) \mathrm{\,d} x=2$ thì $\displaystyle\int\limits_0^2\left[ 3 f(x)-2\right]  \mathrm{d} x=4$} 
		{\True Nếu $f(x)+f(2-x)=x^2-2 x+2,\, \forall x \in \mathbb{R}$ và $f(0)=3$ thì $\displaystyle\int\limits_0^2 f'(x) \mathrm{\,d} x=-4$}
	\loigiai{
		\begin{itemchoice}
			\itemch Sai. Ta có $\displaystyle\int\limits_a^b f(x) \mathrm{\,d} x=F(b)-F(a)$.
			\itemch Đúng. Ta có $\displaystyle\int\limits_a^b f'(x) \mathrm{\,d} x=f(x)\,\bigg|_a ^b=f(b)-f(a)$.
			\itemch Sai. Ta có $J=\displaystyle\int\limits_0^2\left[ 3 f(x)-2\right] \mathrm{d} x=3 \displaystyle\int\limits_0^2 f(x) \mathrm{\,d} x-2 \displaystyle\int\limits_0^2 \mathrm{\,d} x=3\cdot 2-2 x\,\bigg|_0 ^2=6-4=2$.
			\itemch Đúng. Ta có $f(x)+f(2-x)=x^2-2 x+2, \forall x \in \mathbb{R}\quad (1)$.\\
			Thay $x=0$ vào (1) ta được
			$f(0)+f(2)=2 \Rightarrow f(2)=2-f(0)=2-3=-1$.\\
			Từ đó có $ \displaystyle\int\limits_0^2 f'(x) \mathrm{\,d} x=f(2)-f(0)=-1-3=-4.$
		\end{itemchoice}
	}
\end{ex}
\begin{ex}%[2D4V2-2]
	Giả sử $f(x)$ và $g(x)$ là hai hàm số bất kỳ có đạo hàm liên tục trên $\mathbb{R}$ và $a$, $b$, $c$ là các số thực.
	\choiceTF
		{\True $\displaystyle\int\limits_a^b f(x) \mathrm{\,d} x=-\displaystyle\int\limits_b^a f(x) \mathrm{\,d} x$} 
		{Nếu $f(x)=\dfrac{1}{x}$ thì $\displaystyle\int\limits_{-3}^{-2} f(x) \mathrm{\,d} x=\ln x\,\bigg|_{-3} ^{-2}$}
		{\True $\displaystyle\int\limits_a^b f(x) \mathrm{\,d} x+\displaystyle\int\limits_b^c f(x) \mathrm{\,d} x+\displaystyle\int\limits_c^a f(x) \mathrm{\,d} x=0$}
		{Nếu $3 f(x)+x f'(x)=x^{2018}$ với mọi $x \in[0 ; 1]$ thì $\displaystyle\int\limits_0^1 f(x) \mathrm{\,d}x=\dfrac{1}{2020\cdot 2019}$}
	\loigiai{
		\begin{itemchoice}
			\itemch Đúng. Theo tính chất của tích phân.
			\itemch Sai. Ta có $\displaystyle\int\limits_{-3}^{-2} \dfrac{1}{x} \mathrm{\,d} x=\left( \ln |x|\right) \bigg|_{-3} ^{-2}$.
			\itemch Đúng. Ta có \\ $\displaystyle\int\limits_a^b f(x) \mathrm{\,d} x+\displaystyle\int\limits_b^c f(x) \mathrm{\,d} x+\displaystyle\int\limits_c^a f(x) \mathrm{\,d} x=\displaystyle\int\limits_a^c f(x) \mathrm{\,d} x+\displaystyle\int\limits_c^a f(x) \mathrm{\,d} x=\displaystyle\int\limits_a^a f(x) \mathrm{\,d} x=0$.
			\itemch Sai. Nhân hai vế của đẳng thức $3 f(x)+x f'(x)=x^{2018}$ với $x^2$ ta được $$3 x^2 f(x)+x^3 f'(x)=x^{2020} \Rightarrow\left[x^3 f(x)\right]'=x^{2020}\Rightarrow x^3 f(x)=\displaystyle\displaystyle\int\limits x^{2020} \mathrm{\,d} x=\dfrac{x^{2021}}{2021}+C\,(*).$$
			Thay $x=0$ vào hai vế $(*)$ ta được $C=0 \Rightarrow f(x)=\dfrac{x^{2018}}{2021}$.\\
			Vậy $\displaystyle\int\limits_0^1 f(x) \mathrm{\,d} x=\displaystyle\int\limits_0^1 \dfrac{1}{2021} x^{2018} \mathrm{\,d} x=\dfrac{1}{2021} \cdot \dfrac{1}{2019} x^{2019}\,\bigg|_0 ^1=\dfrac{1}{2021 \cdot 2019}$.
		\end{itemchoice}
	}
\end{ex}
\begin{ex}%[2D4V2-2]
	Cho $F(x)$ là nguyên hàm của hàm số $f(x)$.
	\choiceTF
		{\True $\displaystyle\int\limits_1^3 f(x) \mathrm{\,d} x=F(3)-F(1)$}
		{\True Nếu $f(x)=\dfrac{2}{x}+\dfrac{3}{x^2}\,(x \neq 0)$, $F(1)=1$ thì $F(3)=2 \ln 3+3$}
		{Nếu $F(-1)=1$ và $F(2)=4$ thì $\displaystyle\int\limits_{-1}^2\left[ f(x)+2 x\right]  \mathrm{d} x=9$}
		{\True Nếu hàm số $y=f(x)$ có đạo hàm liên tục trên $[0;1]$ thỏa $2 f(x)+3 f(1-x)=\sqrt{1-x^2}$ thì $\displaystyle\int\limits_0^1 f'(x) \mathrm{\,d} x=1$}
	\loigiai{
		\begin{itemchoice}
			\itemch  Đúng. Theo định nghĩa tích phân.
			\itemch Đúng. Ta có $\displaystyle\int\limits_1^3 f(x) \mathrm{\,d} x=F(3)-F(1)$. Suy ra $$F(3)=F(1)+\displaystyle\int\limits_1^3\left(\dfrac{2}{x}+\dfrac{3}{x^2}\right) \mathrm{\,d} x=1+\left(2 \ln x-\dfrac{3}{x}\right)\bigg|_1 ^3=2 \ln 3+3.$$
			\itemch Sai. Ta có $I=\displaystyle\int\limits_{-1}^2\left[ f(x)+2 x\right]  \mathrm{d} x=\left[F(x)+x^2\right]\bigg|_{-1} ^2=F(2)+4-F(-1)-1=6$.
			\itemch Đúng. Ta có 
			$\displaystyle\int\limits_0^1 f'(x) \mathrm{\,d} x=f(x)\,\bigg|_0 ^1=f(1)-f(0)$.\\
			Từ $2 f(x)+3 f(1-x)=\sqrt{1-x^2}\Rightarrow\heva{& 2f(0)+3 f(1)=1 \\ &2 f(1)+3 f(0)=0} \Leftrightarrow\heva{&f(0)=-\dfrac{2}{5} \\& f(1)=\dfrac{3}{5}.}$\\	
			Vậy $I=\displaystyle\int\limits_0^1 f'(x) \mathrm{\,d} x=f(1)-f(0)=\dfrac{3}{5}+\dfrac{2}{5}=1$.
		\end{itemchoice}
	}
\end{ex}
\Closesolutionfile{ans}
% \indapan{2}{ans/ans-2-B12-De1-DS}
\Opensolutionfile{ans}[ans/ans-2-B12-De1-KQ]
\TNSA
\begin{ex}%[2D4H2-6]
	Một xe ô tô đang di chuyển với tốc độ $22$ m/s thì gặp chướng ngại vật. Người lái xe phản ứng $3$ giây sau đó và đạp phanh khẩn cấp, kể từ thời điểm đạp phanh, ô tô chuyển động chậm dần đều với tốc độ $v(t)=36-6 t$ m/s, trong đó $t$ là thời gian tính bằng giây kể từ lúc đạp phanh. Hỏi quãng đường ô tô đi được từ lúc phát hiện chướng ngại vật đến khi ô tô dừng hẳn là bao nhiêu mét?
	\shortans{$174$} 
	\loigiai{
		Quãng đường ô tô đi được từ lúc phát hiện chướng ngại vật đến khi đạp phanh là $66$ m.\\
		Xe ô tô dừng hẳn khi $v(t)=0 \Leftrightarrow 36-6 t=0 \Leftrightarrow t=6$.\\
		Quãng đường ô tô đi được từ lúc đạp phanh đến lúc dừng lại là $\displaystyle\int\limits_0^6(36-6 t) \mathrm{\,d} t=108$ m.\\
		Vậy quãng đường ô tô đi được từ lúc phát hiện chướng ngại vật đến khi ô tô dừng hẳn là $66+108=174$ m.
	}
\end{ex}

\begin{ex}%[2D4V2-2]
	Cho hàm số $y=f(x)$ liên tục trên $\mathbb{R}$. Hàm số $y=f'(x)$ có đồ thị $(C)$ như hình vẽ, $(C)$ cắt trục $Ox$ tại ba điểm phân biệt có hoành độ $a<b<c$.
	\begin{center}
		\begin{tikzpicture}[line join = round, line cap = round,>=stealth,x = 1cm,y = .6cm] 
			%Vẽ hệ trục Oxy 
			\draw[->] (-2.5,0)--(0,0) node[below right]{$O$}--(4.5,0) node[below]{$x$}; 
			\draw (-1.1,.3) node {$a$} (1.1,.3) node {$b$}  (3.1,-.3) node{$c$} (2.9,2) node[rotate=70]{$(C)$};
			\draw[->] (0,-4.5)--(0,4.5) node[right]{$y$}; 
			\draw[samples=200,domain=-1.42:3.42,smooth] plot (\x, {(\x)^3-3*(\x)^2-\x+3}); 
		\end{tikzpicture}
	\end{center}
	Biết rằng diện tích hình phẳng giới hạn bởi $(C)\colon y=f'(x)$ và $O x$ bằng $15$, $f(a)=5$, $f(c)=6$. Tính $f(b)$.
	\shortans{$13$}
	\loigiai{
		Diện tích hình phẳng giới hạn bởi $(C)\colon y=f'(x)$ và $Ox$ bằng $15$, do đó
		$$
		15=\displaystyle\int\limits_a^c\left|f'(x)\right| \mathrm{\,d} x=\displaystyle\int\limits_a^b\left|f'(x)\right| \mathrm{\,d} x+\displaystyle\int\limits_b^c\left|f'(x)\right| \mathrm{\,d} x=2 f(b)-f(a)-f(c) .
		$$
		Suy ra $2 f(b)=15+f(a)+f(c) \Rightarrow f(b)=13$.
	}
\end{ex}

\begin{ex}%[2D4H2-2]
	Biết rằng $\displaystyle\int\limits_0^2 \dfrac{x^2}{x+1} \mathrm{\,d} x=a+\ln b$ với $a, b \in \mathbb{Z}$, $b>0$. Tính $2a+b$.
	\shortans{$3$}
	\loigiai{
		Ta có $\displaystyle\int\limits_0^2 \dfrac{x^2}{x+1} \mathrm{~d} x=\displaystyle\int\limits_0^2\left(x-1+\dfrac{1}{x+1}\right) \mathrm{d} x=\left(\dfrac{x^2}{2}-x+\ln |x+1|\right)\bigg|_0 ^2=\ln 3$.\\
		Suy ra $a=0$, $b=3$. Vậy $2 a+b=3$.
	}
\end{ex}

\begin{ex}%[2D4H2-2] 
	Cho $\displaystyle\int\limits_0^1 \dfrac{\mathrm{\,d} x}{\sqrt{x+2}+\sqrt{x+1}}=a \sqrt{b}-\dfrac{8}{3} \sqrt{a}+\dfrac{2}{3},\left(a, b \in \mathbb{N}^*\right)$. Tính $a+2b$.
	\shortans{$8$}
	\loigiai{
		Ta có
		 \begin{eqnarray*}
		 	\displaystyle\int\limits_0^1 \dfrac{\mathrm{\,d} x}{\sqrt{x+2}+\sqrt{x+1}}&=&\displaystyle\int\limits_0^1\left( \sqrt{x+2}-\sqrt{x+1}\right) \mathrm{d} x\\ &=&\dfrac{2}{3}\left(\sqrt{(x+2)^3}-\sqrt{(x+1)^3}\right)\bigg|_0 ^2 
			=2 \sqrt{3}-\dfrac{8}{3} \sqrt{2}+\dfrac{2}{3}.
	\end{eqnarray*}
	Vậy $a=2$, $b=3$, $a+2b=8$.
	}
\end{ex}

\begin{ex}%[2D4H2-6]
	Tại một nơi không có gió, một chiếc khí cầu đang đứng yên ở độ cao $162$ mét so với mặt đất đã được phi công cài đặt cho nó chế độ chuyển động đi xuống. Biết rằng, khí cầu đã chuyển động theo phương thẳng đứng với vận tốc tuân theo quy luật $v(t)=10 t-t^2$, trong đó $t$ phút là thời gian tính từ lúc bắt đầu chuyển động, $v(t)$ được tính theo đơn vị mét/phút. Tìm vận tốc $v$ của khí cầu khi bắt đầu tiếp đất.
	\shortans{$9$}
	\loigiai{
		Gọi thời điểm khí cầu bắt đầu chuyển động là $t=0$, thời điểm khinh khí cầu bắt đầu tiếp đất là $t_1$.
		Quãng đường khí cầu đi được từ thời điểm $t=0$ đến thời điểm khinh khí cầu bắt đầu tiếp đất  $t_1$ là
		\[
		\displaystyle\int\limits_0^{t_1}\left(10 t-t^2\right) \mathrm{d} t=5 t_1^2-\dfrac{t_1^3}{3}=162 \Leftrightarrow \hoac{&t_1 \approx-4{,}93\\& t_1 \approx 10{,}93 \\& t_1=9.}\]		
		Do $v(t) \geq 0$ nên $0 \leq t_1 \leq 10$, suy ra chọn $t_1=9$.\\
		Vậy khi bắt đầu tiếp đất vận tốc $v$ của khí cầu là $v(9)=10\cdot 9-9^2=9$ mét/phút.
	}
\end{ex}

\begin{ex}%[2D4V2-6] 
	Một ô tô chuyển động nhanh dần đều với vận tốc $v(t)=7 t$ m/s. Đi được $5$ s người lái xe phát hiện chướng ngại vật và phanh gấp, ô tô tiếp tục chuyển động chậm dần đều với gia tốc $a=-35 \mathrm{~m} / \mathrm{s}^2$. Tính quãng đường của ô tô đi được từ lúc bắt đầu chuyển bánh cho đến khi dừng hẳn? (quãng đường tính theo đơn vị m).
	\shortans{$105$}
	\loigiai{
		Quãng đường ô tô đi được trong $5$ s đầu là $s_1=\displaystyle\int\limits_0^5 7 t \mathrm{\,d} t=7 \dfrac{t^2}{2}\,\bigg|_0 ^5=87{,}5$.\\
		Phương trình vận tốc của ô tô khi người lái xe phát hiện chướng ngại vật là $v_2(t)=35-35 t$.\\
		Khi xe dừng lại hẳn thì $v_2(t)=0 \Leftrightarrow 35-35 t=0 \Leftrightarrow t=1$.\\
		Quãng đường ô tô đi được từ khi phanh gấp đến khi dừng lại hẳn là
		\[
		s_2=\displaystyle\int\limits_0^1\left(35-35 t\right) \mathrm{d} t=\left(35 t-35 \dfrac{t^2}{2}\right)\bigg|_0 ^1=17{,}5.\]
		Do đó quãng đường của ô tô đi được từ lúc bắt đầu chuyển bánh cho đến khi dừng hẳn là \[s=s_1+s_2=87{,}5+17{,}5=105.\]
	}
\end{ex}
\Closesolutionfile{ans}
% \indapan{6}{ans/ans-2-B12-De1-KQ}
% \begin{name}
	{NGUYÊN HÀM - TÍCH PHÂN}
	{KT TÍCH PHÂN}
	{\tentruong}
	{\thoigian}
\end{name}
\setcounter{ex}{0}\setcounter{bt}{0}
\Opensolutionfile{ans}[ans/ans-2-B12-De2-NLC]
\TN
\begin{ex}%[Cau-1]%[2D4N2-1]
	Cho hàm số $y=f(x)$ liên tục trên khoảng $K$ và $a$, $b$, $c\in K$. Mệnh đề nào sau đây \textbf{sai}?
	\choice
	{$\displaystyle \int\limits_{a}^{a} f(x)\mathrm{\,d}x=0$}
	{$\displaystyle \int\limits_{a}^{b} f(x)\mathrm{\,d}x=\displaystyle \int\limits_{a}^{b} f(t) \mathrm{\,d}t$}
	{$\displaystyle \int\limits_{a}^{b} f(x)\mathrm{\,d}x=-\displaystyle \int\limits_{b}^{a} f(x) \mathrm{\,d}x$}
	{\True $\displaystyle \int\limits_{a}^{b} f(x) \mathrm{\,d}x + \displaystyle \int\limits_{c}^{b} f(x) \mathrm{\,d}x=\displaystyle \int\limits_{a}^{c} f(x) \mathrm{\,d}x$}
	\loigiai{
		Mệnh đề sai là $\displaystyle \int\limits_{a}^{b} f(x)\mathrm{\,d}x + \displaystyle \int\limits_{c}^{b} f(x) \mathrm{\,d}x=\displaystyle \int\limits_{a}^{c} f(x) \mathrm{\,d}x$.
	}
\end{ex}
\begin{ex}%[Cau-2]%[2D4N2-1]
	Cho hàm số $f(x)$ liên tục trên $\mathbb{R}$ và $F(x)$ là nguyên hàm của $f(x)$, biết $\displaystyle \int\limits_{0}^{9} f(x)\mathrm{\,d}x=9$ và $F(0)=3$. Tính $F(9)$.
	\choice
	{$F(9)=-6$}
	{$F(9)=6$}
	{\True $F(9)=12$}
	{$F(9)=-12$}
	\loigiai{
		Ta có $I=\displaystyle\int\limits_{0}^{9} f(x)\mathrm{\,d}x = F(x)\Big|_0^9 = F(9)- F(0)=9\Leftrightarrow F(9)=9 + F(0)=9 + 3= 12$.
	}
\end{ex}
\begin{ex}%[Cau-3]%[2D4N2-2]
	Tính tích phân $I=\displaystyle\int\limits_{0}^{1} x^{2018}(1 + x)\mathrm{\,d}x$?
	\choice
	{$I=\dfrac{1}{2017}+\dfrac{1}{2018}$}
	{$I=\dfrac{1}{2018}+\dfrac{1}{2019}$}
	{$I=\dfrac{1}{2020}+\dfrac{1}{2021}$}
	{\True $I=\dfrac{1}{2019} + \dfrac{1}{2020}$}
	\loigiai{
		Ta có $I=\displaystyle \int\limits_{0}^{1} x^{2018}(1 + x) \mathrm{\,d}x= \displaystyle \int\limits_{0}^{1} \left(x^{2018} + x^{2019}\right) \mathrm{d}x= \left(\dfrac{x^{2019}}{2019} + \dfrac{x^{2020}}{2020}\right)\Bigg|_0^1 = \dfrac{1}{2019}+\dfrac{1}{2020}$.
	}
\end{ex}
\begin{ex}%[Cau-4]%[2D4N2-4]
	Tính $\displaystyle \int\limits_{0}^{1} 2\mathrm{e}^{x} \mathrm{\,d}x$?
	\choice
	{$I=\mathrm{e}^2-2\mathrm{e}$}
	{$I=2\mathrm{e}$}
	{$I=2\mathrm{e}+2$}
	{\True $I=2\mathrm{e}-2$}
	\loigiai{
		Ta có $\displaystyle \int\limits_{0}^{1} 2\mathrm{e}^{x} \mathrm{\,d}x = 2\mathrm{e}^x\Big|_0^1=2\mathrm{e}-2$.
	}
\end{ex}
\begin{ex}%[Cau-5]%[2D4N2-3]
	Cho $a\in \left(0;\dfrac{\pi}{2}\right)$. Tính $J=\displaystyle \int\limits_{0}^{a} \dfrac{29}{\cos^2 x} \mathrm{\,d}x$ theo $a$.
	\choice
	{$J=-29\tan a$}
	{$J=\dfrac{1}{29}\tan a$}
	{$J=29\cot a$}
	{\True $J=29\tan a$}
	\loigiai{
		Ta có $J=\displaystyle \int\limits_{0}^{a} \dfrac{29}{\cos^2 x}\mathrm{\,d}x=29\tan x\Big|_0^a = 29\tan a - 29\tan 0 = 29\tan a$.
	}
\end{ex}
\begin{ex}%[Cau-6]%[2D4H2-5]
	Tích phân $I=\displaystyle \int\limits_{-1}^{2}\left|x^2 - 2x\right|\mathrm{d}x$ có giá trị là
	\choice
	{\True $I=\dfrac{8}{3}$}
	{$I=\dfrac{4}{3}$}
	{$I=0$}
	{$I=-\dfrac{4}{3}$}
	\loigiai{
		Ta có $x^2 - 2x=0 \Leftrightarrow x=0$ hoặc $x=2$.\\
		Bảng xét dấu
		\begin{center}
			\begin{tikzpicture}
				\tkzTabInit[nocadre=false,lgt=2.5,espcl=2.5,deltacl=0.6]
				{$x$/0.7,$x^2-2x$/0.7}
				{$-\infty$,$0$,$2$,$+\infty$}
				\tkzTabLine{,+,0,-,0,+,}   
			\end{tikzpicture}
		\end{center}
		\begin{align*}
			I & = \displaystyle \int\limits_{-1}^{2} \left|x^2 - 2x\right| \mathrm{d}x = \displaystyle \int\limits_{-1}^{0} \left(x^2 - 2x\right) \mathrm{d}x - \displaystyle \int\limits_{0}^{2} \left(x^2 - 2x\right) \mathrm{d}x\\
			& = \left(\dfrac{x^3}{3} - x^2\right)\bigg|_{-1}^0 - \left(\dfrac{x^3}{3} - x^2\right)\bigg|_{0}^2 \\
			& = \left[0 - \left(-\dfrac{1}{3}-1\right)\right] - \left[\left(\dfrac{8}{3}-4\right)-0\right]\\
			& = \dfrac{8}{3}.
		\end{align*}
	}
\end{ex}
\begin{ex}%[Cau-7]%[2D4N2-3]
	Tính tích phân $I=\displaystyle \int\limits_{0}^{\tfrac{\pi}{4}} \sin x \mathrm{\,d}x$?
	\choice
	{$\dfrac{2+\sqrt{2}}{2}$}
	{\True $\dfrac{2-\sqrt{2}}{2}$}
	{$\dfrac{\sqrt{2}}{2}$}
	{$-\dfrac{\sqrt{2}}{2}$}
	\loigiai{
		Ta có $I=\displaystyle \int\limits_{0}^{\tfrac{\pi}{4}} \sin x \mathrm{\,d}x=-\cos x \bigg|_0^{\tfrac{\pi}{4}} = - \cos \left(\dfrac{\pi}{4}\right)+\cos 0= -\dfrac{\sqrt{2}}{2}+1=\dfrac{2-\sqrt{2}}{2}$.
	}
\end{ex}
\begin{ex}%[Cau-8]%[2D4N2-2]
	Tính tích phân $I=\displaystyle \int\limits_{1}^{2}\dfrac{x^2+4x}{x}\mathrm{\,d}x$?
	\choice
	{$I=\dfrac{29}{2}$}
	{$I=-\dfrac{11}{2}$}
	{\True $I=\dfrac{11}{2}$}
	{$I=-\dfrac{29}{2}$}
	\loigiai{
		Ta có $I=\displaystyle \int\limits_{1}^{2} \dfrac{x^2+4x}{x}\mathrm{\,d}x= \int\limits_{1}^{2}(x+4)\mathrm{d}x=\dfrac{11}{2}$.
	}
\end{ex}
\begin{ex}%[Cau-9]%[2D4N2-3]
	Cho tích phân $I=\displaystyle \int\limits_{0}^{\tfrac{\pi}{2}}(4x-1+\cos x)\mathrm{d}x =\pi\left(\dfrac{\pi}{a} -\dfrac{1}{b}\right)+c$, $(a$, $b$, $c\in \mathbb{Q})$. Tính $a-b+c$.
	\choice
	{\True $1$}
	{$-2$}
	{$\dfrac{1}{3}$}
	{$\dfrac{1}{2}$}
	\loigiai{
		Ta có
		\begin{align*}
			I &= \displaystyle \int\limits_{0}^{\tfrac{\pi}{2}} \left(4x - 1 + \cos x\right) \mathrm{d}x \\
			&= \left(2x^2 - x + \sin x\right)\Big|_0^{\tfrac{\pi}{2}} \\
			&= 2\cdot \left(\dfrac{\pi}{2}\right)^2 - \dfrac{\pi}{2} + \sin \left(\dfrac{\pi}{2}\right) \\
			&= \dfrac{\pi^2}{2} - \dfrac{\pi}{2} + 1 \\
			&= \pi\left(\dfrac{\pi}{2}-\dfrac{1}{2}\right)+1.
		\end{align*}
		Suy ra $a=2$, $b=2$, $c=1$. \\
		Vậy $a-b+c=2-2+1=1$.
	}
\end{ex}
\begin{ex}%[Cau-10]%[2D4H2-2]
	Cho $I=\displaystyle \int\limits_{0}^{1}\left(4x-2m^2\right)\mathrm{d}x$. Có bao nhiêu giá trị nguyên của $m$ để $I+6>0$?
	\choice
	{$1$}
	{$5$}
	{$2$}
	{\True $3$}
	\loigiai{
		Ta có $I=\displaystyle \int\limits_{0}^{1}\left(4x-2m^2\right)\mathrm{d}x= \left(2x^2-2m^2 x\right)\bigg|_0^1=-2m^2+2$.\\
		Khi đó $I+6>0\Leftrightarrow-2m^2+2+6>0\Leftrightarrow-m^2+4>0\Leftrightarrow-2<m<2$.\\
		Mà $m$ là số nguyên nên $m \in\{-1;0;1\}$.\\
		Vậy có $3$ giá trị nguyên của $m$ thỏa mãn yêu cầu.
	}
\end{ex}
\begin{ex}%[Cau-11]%[2D4H2-1]
	Cho $\displaystyle \int\limits_{1}^{2} \left[3f(x) + 2g(x)\right]\mathrm{d}x=1$, $\displaystyle \int\limits_{1}^{2} \left[2f(x) - g(x)\right]\mathrm{d}x=-3$. Khi đó $\displaystyle \int\limits_{1}^{2} f(x) \mathrm{\,d}x$ bằng
	\choice
	{$\dfrac{11}{7}$}
	{\True $-\dfrac{5}{7}$}
	{$\dfrac{6}{7}$}
	{$\dfrac{16}{7}$}
	\loigiai{
		Ta có $\displaystyle \int\limits_{1}^{2} \left[3f(x) + 2g(x)\right] \mathrm{d}x = 1 \Leftrightarrow \displaystyle 3\int\limits_{1}^{2} f(x) \mathrm{\,d}x + \displaystyle 2\int\limits_{1}^{2} g(x) \mathrm{\,d}x= 1$.\\
		Và $\displaystyle \int\limits_{1}^{2} \left[2f(x) - g(x)\right] \mathrm{d}x = -3 \Leftrightarrow \displaystyle 2\int\limits_{1}^{2} 2f(x) \mathrm{\,d}x - \displaystyle \int\limits_{1}^{2} g(x) \mathrm{\,d}x = -3$.\\
		Đặt $a = \displaystyle \int\limits_{1}^{2} f(x) \mathrm{\,d}x$, $b = \displaystyle \int\limits_{1}^{2} g(x) \mathrm{\,d}x$ ta có hệ phương trình
		$$\heva{&3a+2b=1\\&2a-b=-3\\} \Leftrightarrow \heva{&a=-\dfrac{5}{7}\\&b=\dfrac{11}{7}.}$$
		Vậy $\displaystyle \int\limits_{1}^{2} f(x) \mathrm{\,d}x=\dfrac{5}{7}$.
	}
\end{ex}
\begin{ex}%[Cau-12]%[2D4V2-6]
	Một vật chuyển động chậm với vận tốc $v(t)=160-10t$ (m/s). Quãng đường mà vật di chuyển được từ thời điểm $t=0$ (s) đến thời điểm mà vật dừng lại là
	\choice
	{$160$ (m)}
	{\True $1280$ (m)}
	{$0$ (m)}
	{$144$ (m)}
	\loigiai{
		Vật dừng lại đồng nghĩa với $v(t)=0 \Leftrightarrow 160-10t=0 \Leftrightarrow t=16$ (s).\\
		Quãng đường vật đi được là $s(t)=\displaystyle \int\limits_{0}^{16}(160-10t) \mathrm{d}t = \left(160t-5t^2\right)\bigg|_0^{16}=1280$ (m).
	}
\end{ex}
\Closesolutionfile{ans}
\indapan{6}{ans/ans-2-B12-De2-NLC}
\Opensolutionfile{ans}[ans/ans-2-B12-De2-DS]
\TNTF
\begin{ex}%[Cau-1]%[2D4H2-1]
	Cho hàm số $y=f(x)$ liên tục trên $\mathbb{R}$ và thỏa mãn $\displaystyle \int\limits_{-1}^{10} f(x)\mathrm{\,d}x=15$,\break $\displaystyle \int\limits_{3}^{5} f(x) \mathrm{\,d}x=-2$, $\displaystyle \int\limits_{-1}^{12} f(x) \mathrm{\,d}x=5$.
	\choiceTF[t]
	{$\displaystyle \int\limits_{10}^{-1} 2f(x)\mathrm{\,d}x=30$}
	{$\displaystyle \int\limits_{10}^{12} \left[f(x)-2\right] \mathrm{d}x=-12$}
	{\True $\displaystyle \int\limits_{-1}^{3} f(x) \mathrm{\,d}x + \int\limits_{5}^{10} f(x) \mathrm{\,d}x=17$}
	{Biết rằng $f(x)>0$, $\forall x>3$; $f(x)<0$, $\forall x<3$ và $\displaystyle \int\limits_{-1}^{12} \left|f(x)\right|\mathrm{d}x=5$. Khi đó $\displaystyle \int\limits_{-1}^{3} f(x)\mathrm{\,d}x - \int\limits_{5}^{12} f(x)\mathrm{\,d}x=3$}
	\loigiai{
		
		\begin{itemchoice}
			\itemch Sai. Vì $ \displaystyle \int\limits_{10}^{-1} 2f(x) \mathrm{\,d}x = -2\int\limits_{-1}^{10} f(x) \mathrm{\,d}x = -2\cdot 15=-30$.
			\itemch Sai. Vì 
			\begin{align*}
				\displaystyle \int\limits_{10}^{12} \left[f(x)-2\right] \mathrm{d}x &= \int\limits_{10}^{12} f(x) \mathrm{\,d}x - \int\limits_{10}^{12} 2 \mathrm{\,d}x \\
				&= \int\limits_{-1}^{12} f(x) \mathrm{\,d}x - \int\limits_{-1}^{10} f(x) \mathrm{\,d}x - 4 \\
				&= 5-15-4 \\
				&= -14.
			\end{align*}
			\itemch Đúng. Vì $\displaystyle \int\limits_{-1}^{3} f(x) \mathrm{\,d}x + \int\limits_{5}^{10} f(x) \mathrm{\,d}x = \int\limits_{-1}^{10} f(x) \mathrm{\,d}x - \int\limits_{3}^{5} f(x) \mathrm{\,d}x = 15 - (-2) = 17$.
			\itemch Sai.\\
			Ta có $f(x)>0$, $\forall x>3$; $f(x)<0$, $\forall x<3$.\\
			Suy ra $\displaystyle \int\limits_{-1}^{12} \left|f(x)\right|\mathrm{d}x= -\int\limits_{-1}^{3} f(x)\mathrm{\,d}x+\int\limits_{3}^{5} f(x)\mathrm{\,d}x + \int\limits_{5}^{12} f(x)\mathrm{\,d}x=5$.\\
			Khi đó\\
			\begin{align*}
				\displaystyle \int\limits_{-1}^{3} f(x) \mathrm{\,d}x -  \int\limits_{5}^{12} f(x) \mathrm{\,d}x &=  \int\limits_{3}^{5} f(x) \mathrm{\,d}x - 5\\
				&= -2-5 \\
				&= -7.
			\end{align*}
		\end{itemchoice}
	}
\end{ex}
\begin{ex}%[Cau-2]%[2D4H2-2]
	Cho hàm số $y = f(x)$ liên tục trên $\mathbb{R}$ thỏa mãn $f(x) = \heva{&\dfrac{4x^2-3}{x}& \text{khi} &x\ge 1\\&ax+b &\text{khi} &-2<x<1\\&x^2+4x-4 &\text{khi} &x \le -2.}$
	\choiceTF[t]
	{\True $\displaystyle \int\limits_{-5}^{-2} f(x) \mathrm{\,d}x = -15$}
	{\True $\displaystyle \int\limits_{3}^{4} f(x) \mathrm{\,d}x = 14 +3\ln 3 - 6\ln 2$}
	{$\displaystyle \int\limits_{0}^{1} f(x) \mathrm{\,d}x = a+b$}
	{\True $\displaystyle \int\limits_{-3}^{0} f(x) \mathrm{\,d}x = \dfrac{-53}{3}$}
	\loigiai{
		\begin{itemchoice}
			\itemch Đúng. Vì $\displaystyle \int\limits_{-5}^{-2} f(x) \mathrm{\,d}x =\int\limits_{-5}^{-2} \left(x^2+4x-4\right) \mathrm{d}x = -15$.
			\itemch Đúng. Vì $\displaystyle \int\limits_{3}^{4} f(x) \mathrm{\,d}x = \int\limits_{3}^{4} \dfrac{4x^2-3}{x} \mathrm{\,d}x = \int\limits_{3}^{4} \left(4x - \dfrac{3}{x} \right) \mathrm{d}x = 14 +3\ln 3 - 6\ln 2$.
			\itemch Sai. Vì $\displaystyle \int\limits_{0}^{1} f(x) \mathrm{\,d}x =\int\limits_{0}^{1}(ax+b)\mathrm{d}x=\left(\dfrac{ax^2}{2} + bx\right)\bigg|_0^1=\dfrac{a}{2}+b\ne a+b$.\\ (Chỉ đúng với $a=0$).
			\itemch Đúng. Vì $\displaystyle \int\limits_{-3}^{0} f(x) \mathrm{\,d}x= \int\limits_{-3}^{-2} \left(x^2+4x-4\right)\mathrm{d}x+\int\limits_{-2}^{0} (ax+b)\mathrm{d}x$.\\
			Do hàm số $y=f(x)$ liên tục trên $\mathbb{R}$ nên 
			$$\heva{&\lim\limits_{x\to 1^-} f(x) = \lim\limits_{x\to 1^+} f(x)\\&\lim\limits_{x\to -2^+} f(x) = \lim\limits_{x\to -2^-} f(x)\\}\Rightarrow \heva{&a+b=1\\&-2a+b=-8\\} \Rightarrow a=3;\, b=-2.$$
			Suy ra $\displaystyle \int\limits_{-3}^{0} f(x) \mathrm{\,d}x = \int\limits_{-3}^{-2} \left(x^2+4x-4\right)\mathrm{d}x+\int\limits_{-2}^{0} (3x-2)\mathrm{d}x=\dfrac{-53}{3}$.
		\end{itemchoice}
	}
\end{ex}
\begin{ex}%[Cau-3]%[2D4H2-4]
	Cho hàm số $f(x)$; $g(x)$ thỏa mãn $\displaystyle \int\limits_{2}^{6} f(x) \mathrm{\,d}x= 3$; $\displaystyle \int\limits_{2}^{6} g(x)\mathrm{\,d}x=-2$.
	\choiceTF[t]
	{\True $\displaystyle \int\limits_{2}^{6} \left[f(x)+g(x)\right]\mathrm{,d}x=1$}
	{$\displaystyle \int\limits_{2}^{6} \left[3f(x)-g(x)-3\right]\mathrm{d}x=10$}
	{\True $\displaystyle \int\limits_{2}^{6} \left[3\mathrm{e}^x-2f(x)\right]\mathrm{d}x= 3\mathrm{e}^6-3\mathrm{e}^2-6$}
	{Biết $\displaystyle \int\limits_{2}^{6}\left[3g(x)-\dfrac{2x-3}{x^2}\right]\mathrm{d}x =a+b\ln 3$, với $a$; $b\in\mathbb{Q}$.
		Khi đó $a^2+12b=-8$}
	\loigiai{
		\begin{itemchoice}
			\itemch Đúng. Vì $\displaystyle \int\limits_{2}^{6}\left[f(x)+g(x)\right] \mathrm{d}x=\displaystyle\int\limits_{2}^{6} f(x)\mathrm{\,d}x+\displaystyle\int\limits_{2}^{6} g(x) \mathrm{\,d}x=3-2=1$.
			\itemch Sai. Vì 
			\begin{align*}
				&\displaystyle \int\limits_{2}^{6}\left[3f(x)-g(x)-3\right]\mathrm{d}x \\
				= & 3\displaystyle\int\limits_{2}^{6} f(x) \mathrm{\,d}x-\displaystyle\int\limits_{2}^{6} g(x) \mathrm{\,d}x - \displaystyle\int\limits_{2}^{6} 3 \mathrm{\,d}x \\
				= & 3\cdot 3-(-2)-12=-1\ne 10.
			\end{align*}
			\itemch Đúng. Vì $\displaystyle \int\limits_{2}^{6} \left[3\mathrm{e}^x-2f(x)\right] \mathrm{\,d}x = 3\displaystyle\int\limits_{2}^{6} \mathrm{e}^x \mathrm{\,d}x - 2\displaystyle\int\limits_{2}^{6} f(x) \mathrm{\,d}x = 3\mathrm{e}^6-3\mathrm{e}^2-6$.
			\itemch Sai. Ta có
			\begin{align*}
				& \displaystyle \int\limits_{2}^{6} \left[3g(x)-\dfrac{2x-3}{x^2}\right] \mathrm{d}x \\
				= &\ 3\displaystyle\int\limits_{2}^{6} g(x)\mathrm{\,d}x- \displaystyle\int\limits_{2}^{6} \dfrac{2x-3}{x^2} \mathrm{\,d}x=-6- \left(2\ln \left|x\right|+\dfrac{3}{x}\right)\bigg|_2^6 \\
				= &-5 + 2\ln 3.
			\end{align*}
			Suy ra $a=-5$; $b=2$.\\
			Vậy $a^2+12b=25+24=49\ne 25$.
		\end{itemchoice}
	}
\end{ex}
\begin{ex}%[Cau-4]%[2D4V2-2]
	Cho hàm số $y=f(x)$ liên tục trên $\mathbb{R}$, đồ thị hàm số $(C)\colon y=f'(x)$ trên đoạn $[-3;6]$ là đường gấp khúc như hình vẽ. Khi đó
	\begin{center}
		\begin{tikzpicture}[scale=0.6,>=stealth, font=\footnotesize, line join=round, line cap=round]  
			%Vẽ hệ trục
			\draw[->] (-4.5,0)--(7.5,0) node[below]{$x$};
			\draw[->] (0,-3.5)--(0,4.5) node[left]{$y$};
			\node at (0,0) [below left]{$O$};
			\path 
				(-3,-2) coordinate (A)
				(-3,0) coordinate (Ax)
				(0,-2) coordinate (Ay)
				(2,3) coordinate (B)
				(2,0) coordinate (Bx)
				(0,3) coordinate (By)
				(6,-1) coordinate (C)
				(6,0) coordinate (Cx)
				(0,-1) coordinate (Cy)
				(5,0) coordinate (Ex)
			;
			\fill[black] (A) circle(1pt) node[below left]{$A$};
			\fill[black] (Ax) circle(1pt) node[above]{$-3$};
			\fill[black] (Ay) circle(1pt) node[right]{$-2$};
			\fill[black] (B) circle(1pt) node[above right]{$B$};
			\fill[black] (Bx) circle(1pt) node[below]{$2$};
			\fill[black] (By) circle(1pt) node[left]{$3$};
			\fill[black] (C) circle(1pt) node[below]{$C$};
			\fill[black] (Cx) circle(1pt) node[above]{$6$};
			\fill[black] (Ex) circle(1pt) node[above right]{$5$};
			\fill[black] (Ex) circle(0pt) node[below left]{$E$};
			\draw[dashed] (Ax)--(A)--(Ay) (Bx)--(B)--(By) (Cx)--(C);
			\draw (A)--(B)--(C);
		\end{tikzpicture}
	\end{center}
	\choiceTF[t]
	{\True $\displaystyle \int\limits_{-3}^{-1} f'(x)\mathrm{\,d}x=-2$}
	{\True $\displaystyle \int\limits_{0}^{1} f'(x)\mathrm{\,d}x=\dfrac{3}{2}$}
	{$f(2)-f(6)=4$}
	{$f(5)+f(-3)-2f(2)=-10$}
	\loigiai{
		\begin{itemchoice}
			\itemch Đúng.\\
			Ta có $A(-3;-2)$, $B(2;3)$ $\Rightarrow AB\colon y=x+1$.\\
			Khi đó $\displaystyle \int\limits_{-3}^{-1} f'(x)\mathrm{\,d}x= \displaystyle\int\limits_{-3}^{-1} (x+1)\mathrm{\,d}x=-2$.
			\itemch Đúng. Vì $\displaystyle \int\limits_{0}^{1} f'(x) \mathrm{\,d}x= \displaystyle\int\limits_{0}^{1}(x+1)\mathrm{\,d}x=\dfrac{3}{2}$.
			\itemch Sai.\\
			Ta có $B(2;3)$, $E(5;0)\Rightarrow BC\colon y=-x+5$.\\
			Khi đó $\displaystyle \int\limits_{2}^{6} f'(x)\mathrm{\,d}x= \displaystyle\int\limits_{2}^{6}(-x+5) \mathrm{\,d}x = 4$.\\
			Vì vậy $f(2)-f(6)=-\displaystyle \int\limits_{2}^{6} f'(x)\mathrm{\,d}x=-4$.
			\itemch Sai.\\
			Ta có $\displaystyle \int\limits_{-3}^{2} f'(x)\mathrm{\,d}x= \displaystyle\int\limits_{-3}^{2} (x+1)\mathrm{\,d}x=\dfrac{5}{2}=f(2)-f(-3)$.\\
			Mặt khác $\displaystyle \int\limits_{2}^{5} f'(x)\mathrm{\,d}x= \displaystyle\int\limits_{2}^{5}(-x+5)\mathrm{\,d}x=\dfrac{9}{2}=f(5)-f(2)$.\\
			Vì vậy $f(5)+f(-3)-2f(2)=\dfrac{9}{2}-\dfrac{5}{2}=2$.
		\end{itemchoice}
	}
\end{ex}
\Closesolutionfile{ans}
\indapan{3}{ans/ans-2-B12-De2-DS}
\Opensolutionfile{ans}[ans/ans-2-B12-De2-KQ]
\TNSA
\begin{ex}%[Cau-1]%[2D4V2-2]
	Cho $\displaystyle \int\limits_{1}^{4}\sqrt{\dfrac{1}{4x}+\dfrac{\sqrt{x}+ \mathrm{e}^x}{\sqrt{x}\cdot \mathrm{e}^{2x}}}\mathrm{\,d}x=a+\mathrm{e}^b-\mathrm{e}^c$ với $a$, $b$, $c$ là các số nguyên. Tính giá trị của biểu thức $S=a+b+c$.
	\shortans{$-4$}
	\loigiai{
		Ta có
		\begin{align*}
			\displaystyle \int\limits_{1}^{4} \sqrt{\dfrac{1}{4x} + \dfrac{\sqrt{x} + \mathrm{e}^x}{\sqrt{x}\cdot \mathrm{e}^{2x}}} \mathrm{\,d}x 
			&= \displaystyle\int\limits_{1}^{4} \sqrt{\left(\dfrac{1}{2\sqrt{x}}\right)^2 + 2\dfrac{1}{2\sqrt{x}\cdot \mathrm{e}^x}+\left(\dfrac{1}{\mathrm{e}^x}\right)^2} \mathrm{\,d}x\\
			&= \displaystyle\int\limits_{1}^{4}\sqrt{\left(\dfrac{1}{2\sqrt{x}} + \dfrac{1}{\mathrm{e}^x}\right)^2}\mathrm{d}x\\
			&=\displaystyle\int\limits_{1}^{4}\left(\dfrac{1}{2\sqrt{x}}+\dfrac{1}{\mathrm{e}^x}\right) \mathrm{\,d}x\\
			&=\left(\sqrt{x}-\mathrm{e}^{-x}\right)\bigg|_1^4\\
			&=1-\mathrm{e}^{-4}+\mathrm{e}^{-1}\\
			&=a+\mathrm{e}^b-\mathrm{e}^c.
		\end{align*}
		$\Rightarrow \heva{&a=1\\&b=-1\\&c=-4.}$\\
		Vậy $a+b+c=1+(-1)+(-4)=-4$.
	}
\end{ex}
\begin{ex}%[Cau-2]%[2D4V2-6]
	Tốc độ chuyển động của thang máy từ tầng $1$ lên tầng cao nhất theo thời gian $t$ (giây) được cho bởi công thức 
	$$v(t) = \heva{&t&\text{khi}&\ 0 \le t \le 2\\&2 &\text{khi}&\ 2 < t \le 20\\&12-0{,}5t &\text{khi}&\ 20 < t \le 24.}$$
	Tính vận tốc trung bình của thang máy.
	\shortans{$1{,}75$}
	\loigiai{
		Quãng đường chuyển động của thang máy là\\
		$s=\displaystyle \int\limits_{0}^{24} v(t)\mathrm{\,d}t = \displaystyle\int\limits_{0}^{2} v(t)\mathrm{\,d}t + \displaystyle\int\limits_{2}^{20} v(t)\mathrm{\,d}t + \displaystyle\int\limits_{20}^{24} v(t)\mathrm{\,d}t = \displaystyle\int\limits_{0}^{2} t\mathrm{\,d}t +\displaystyle\int\limits_{2}^{20} 2 \mathrm{\,d}t+\displaystyle\int\limits_{20}^{24} (12-0{,}5t)\mathrm{d}t = 42$.\\
		Tốc độ trung bình của thang máy là $v_{tb}=\dfrac{s}{t}=\dfrac{42}{24}=1{,}75$ (m/s).
	}
\end{ex}
\begin{ex}%[Cau-3]%[2D4H2-3]
	Biết $\displaystyle \int\limits_{0}^{\tfrac{\pi}{2}} \left(x-1+\sin 2x\right) \mathrm{d}x = \pi \left(\dfrac{\pi}{a} - \dfrac{1}{b}\right) + 1$, $(a$, $b\in \mathbb{Q})$. Tính $a+2b$.
	\shortans{$12$}
	\loigiai{
		Ta có
		\begin{align*}
			\displaystyle \int\limits_{0}^{\tfrac{\pi}{2}} (x-1+\sin 2x) \mathrm{d}x &= \left(\dfrac{1}{2}x^2-x-\dfrac{1}{2}\cos 2x\right)\bigg|_0^{\tfrac{\pi}{2}}\\
			&=\dfrac{1}{2}\cdot \left(\dfrac{\pi}{2}\right)^2 - \dfrac{\pi}{2} - \dfrac{1}{2} \cos \left(2\cdot \dfrac{\pi}{2}\right) + \dfrac{1}{2}\\
			&=\dfrac{\pi^2}{8}-\dfrac{\pi}{2}+1\\
			&=\pi\left(\dfrac{\pi}{8}-\dfrac{1}{2}\right)+1.
		\end{align*}
		Suy ra $a=8$; $b=2$.\\
		Vậy $a+2b=8+2\cdot 2=12$.
	}
\end{ex}
\begin{ex}%[Cau-4]%[2D4V2-3]
	Cho $M$, $N$ là các số thực, xét hàm số $f(x)=M\cdot \sin \pi x + N\cdot \cos \pi x$ thỏa mãn $f(1)=3$ và $\displaystyle \int\limits_{0}^{\tfrac{1}{2}} f(x) \mathrm{\,d}x= -\dfrac{1}{\pi}$. Tính $f'\left(\dfrac{1}{4}\right)$. (Kết quả làm tròn đến hàng phần mười).
	\shortans{$11{,}1$}
	\loigiai{
		Ta có $f(1)=3\Leftrightarrow M\cdot \sin \pi+N\cdot \cos \pi=3\Leftrightarrow N=-3$.\\
		Mặt khác \\
		\begin{align*}
			\displaystyle \int\limits_{0}^{\tfrac{1}{2}} f(x) \mathrm{\,d}x = -\dfrac{1}{\pi} &\Leftrightarrow \int\limits_{0}^{\tfrac{1}{2}} \left(M\cdot \sin \pi x + N \cdot \cos \pi x\right) \mathrm{d}x = -\dfrac{1}{\pi}\\
			&\Leftrightarrow \left(-\dfrac{M}{\pi}\cos \pi x - \dfrac{3}{\pi} \sin \pi x\right)\bigg|_0^{\tfrac{1}{2}} = -\dfrac{1}{\pi}\\
			&\Leftrightarrow -\dfrac{3}{\pi} + \dfrac{M}{\pi} = -\dfrac{1}{\pi}\\
			&\Leftrightarrow M = 2.
		\end{align*}
		Ta được $f(x) = 2\cdot \sin \pi x - 3\cdot \cos \pi x$ nên $f'(x) = 2\pi \cos \pi x + 3\pi \sin \pi x$.\\
		Vậy $f'\left(\dfrac{1}{4}\right) = \dfrac{5\pi \sqrt{2}}{2} \approx 11{,}1$.
	}
\end{ex}
\begin{ex}%[Cau-5]%[2D4V2-6]
	Ba Tí muốn làm cửa sắt được thiết kế như hình bên dưới. Vòm cổng có hình dạng là một Parabol. Giá $1$ m$^2$ cửa sắt là $660\,000$ đồng. Cửa sắt có giá (nghìn đồng) là bao nhiêu?
	\begin{center}
		\begin{tikzpicture}[smooth,samples=300,scale=1,>=stealth, font=\footnotesize]
			%\draw[->] (-3.0,0)--(4.0,0) node[below]{$x$};
			%\draw[->] (0,-1.8)--(0,1.5) node[right]{$y$};
			\path 
			(-2.5,0) coordinate (A)
			(2.5,0) coordinate (B)
			(0,0.5) coordinate (C)
			(2.5,-1.5) coordinate (E)
			(-2.5,-1.5) coordinate (D)
			(0,0) coordinate (O)
			;
			\draw[thick, magenta,domain=-2.5:2.5] plot(\x,{-0.08*(\x)^2+1/2});
			%\draw[thick, magenta,domain=-2.5:2.5] plot(\x,{-0.04*(\x)^2+2-0.2});
			\draw[dashed] (D)--($(D)+(0,-0.5)$)--($(E)+(0,-0.5)$)--(E);
			\draw[dashed] (C)--(3,0.5)--(3,-1.5)--(0,-1.5);
			%\draw[dashed] (A)--(B);
			\draw (A)--(D)--(E)--(B) (0,-1.5)--(0,0.5);
			\node[rotate=90] at (3.2,-0.8) {$2$ m};
			\node[rotate=90] at (2.7,-0.8) {$1{,}5$ m};
			\node[rotate=0] at (0.3,-2.2) {$5$ m};
		\end{tikzpicture}
	\end{center}
	\shortans{$6050$}
	\loigiai{
		\begin{center}
			\begin{tikzpicture}[smooth,samples=300,scale=1,>=stealth, font=\footnotesize]
				\draw[->] (-3.0,0)--(4.0,0) node[below]{$x$};
				\draw[->] (0,-1.8)--(0,1.5) node[right]{$y$};
				\path 
					(-2.5,0) coordinate (A)
					(2.5,0) coordinate (B)
					(0,0.5) coordinate (C)
					(2.5,-1.5) coordinate (E)
					(-2.5,-1.5) coordinate (D)
					(0,0) coordinate (O)
				;
				\draw[thick, magenta,domain=-2.5:2.5] plot(\x,{-0.08*(\x)^2+1/2});
				%\draw[thick, magenta,domain=-2.5:2.5] plot(\x,{-0.04*(\x)^2+2-0.2});
				\draw[dashed] (D)--($(D)+(0,-0.5)$)--($(E)+(0,-0.5)$)--(E);
				\draw[dashed] (C)--(3,0.5)--(3,-1.5)--(0,-1.5);
				\draw[dashed] (A)--(B);
				\draw (A)--(D)--(E)--(B);
				\node[rotate=90] at (3.2,-0.8) {$2$ m};
				\node[rotate=90] at (2.7,-0.8) {$1{,}5$ m};
				\node[rotate=0] at (0.3,-2.2) {$5$ m};
				\draw[fill=black] (O) circle(1pt) node[below right]{$O$};
				\draw[fill=black] (A) circle(1pt) node[below left]{$A$};
				\draw[fill=black] (B) circle(1pt) ($(B)+(60:0.3)$) node {$B$};
				\draw[fill=black] (C) circle(1pt) node[above right]{$C$};
				\draw[fill=black] (D) circle(1pt) node[left]{$D$};
				\draw[fill=black] (E) circle(1pt) node[below right]{$E$};
				\draw[->] (-3.5,0.25) node[left]{Phần 1}--(-1,0.25);
				\draw[->] (-3.5,-1) node[left]{Phần 2}--(-2,-1);
			\end{tikzpicture}
		\end{center}
		Từ hình vẽ ta chia cửa rào sắt ra thành $2$ phần như trên.\\
		Khi đó $S = S_1 + S_2 = S_1 + 5\cdot 1{,}5 = S_1 + 7{,}5$.\\
		Để tính $S_1$ ta vận dụng kiến thức tính diện tích hình phẳng của tích phân.\\
		Gắn hệ trục $Oxy$ trong đó $O$ trung với trung điểm của $AB$, $OB\subset Ox$, $OC \subset Oy$.\\
		Theo đề bài ta có đường cong có dạng hình Parabol. Giả sử $(P)\colon y=ax^2+bx+c$.\\
		Khi đó $\heva{&A\left(-\dfrac{5}{2}; 0\right)\in (P)\\&B\left(\dfrac{5}{2}; 0\right)\in (P)\\&C\left(0;\dfrac{1}{2}\right)\in (P)} \Leftrightarrow\heva{&\dfrac{25}{4}a-\dfrac{5}{2}b+c=0\\&\dfrac{25}{4}a+ \dfrac{5}{2}b+c=0\\&c=\dfrac{1}{2}} \Leftrightarrow \heva{&a= -\dfrac{2}{25}\\ &b = 0\\&c=\dfrac{1}{2}.}$\\
		$\Rightarrow (P) \colon y = -\dfrac{2}{25} x^2 + \dfrac{1}{2}$.\\
		Diện tích $S_2 = 2\displaystyle\int\limits_{0}^{2{,}5} \left(-\dfrac{2}{25} x^2 + \dfrac{1}{2}\right) \mathrm{d}x = \dfrac{10}{6}\ (\mathrm{m}^2)$.\\
		$\Rightarrow S = \dfrac{55}{6}$ (m$^2$).\\
		Vậy  giá tiền cửa sắt là $\dfrac{55}{6} \cdot 660\,000= 6050$ (nghìn đồng).
	}
\end{ex}
\begin{ex}%[Cau-6]%[2D4V2-6]
	Một ô tô đang chạy đều với vận tốc $15$ (m/s) thì phía trước xuất hiện chướng ngại vật nên người lái đạp phanh gấp. Kể từ thời điểm đó, ô tô chuyển động chậm dần đều với gia tốc $-a$ (m/s$^2$). Tìm giá trị của $a$ biết ô tô chuyển động thêm được $20$ (m) thì dừng hẳn. (Kết quả làm tròn đến hàng phần trăm).
	\shortans{$5{,}63$}
	\loigiai{
		Gọi $x(t)$ là hàm biểu diễn quãng đường, $v(t)$ là hàm vận tốc.\\
		Ta có $\displaystyle \int\limits_{0}^{t} (-a) \mathrm{\,d}x = -at \Rightarrow v(t) = -at + 15$.\\
		Mặt khác $x(t)-x(0)=\displaystyle \int\limits_{0}^{t}v(t)\mathrm{\,d}x =\displaystyle \int\limits_{0}^{t}(-at+15)\mathrm{d}x=-\dfrac{1}{2}at^2 + 15t$.\\
		$\Rightarrow x(t)=-\dfrac{1}{2}at^2 + 15t$.\\
		Ta có $\heva{&v(t)=0\\&x(t)=20}\Leftrightarrow\heva{&-at+15=0\\&-\dfrac{1}{2}at^2 + 15t = 20}\Rightarrow -\dfrac{15}{2}t+15t=20 \Rightarrow t=\dfrac{8}{3}$.\\
		$\Rightarrow a=\dfrac{45}{8}\approx 5{,}63$.
	}
\end{ex}
\Closesolutionfile{ans}
\indapan{6}{ans/ans-2-B12-De2-KQ}
% \begin{name}
	{NGUYÊN HÀM - TÍCH PHÂN}
	{KT ỨNG DỤNG NGUYÊN HÀM - TÍCH PHÂN}
	{\tentruong}
	{\thoigian}
\end{name}
\setcounter{ex}{0}\setcounter{bt}{0}

\Opensolutionfile{ans}[ans/ans-2-C4B13-D1]
\TN
\begin{ex}%[Dự án 2025 - đề cấu trúc mới, Hung Doan]%[2D4N3-1]
	Cho hàm số $y=f(x)$ liên tục trên đoạn $[a;b]$. Khi đó, diện tích hình phẳng giới hạn bởi đồ thị của hàm số $y=f(x)$, trục hoành và hai đường thẳng $x=a$, $x=b$ được tính bởi công thức.
	\choice
	{$S=\pi \displaystyle\int \limits_a^b f(x)\mathrm{\,d}x$}
	{$S=\pi \displaystyle\int \limits_a^b |f(x)|\mathrm{\,d}x$}
	{$S=\displaystyle\int \limits_a^b f(x)\mathrm{\,d}x$}
	{\True $S=\displaystyle\int \limits_a^b |f(x)|\mathrm{\,d}x$}
	\loigiai{
		Công thức diện tích hình phẳng giới hạn bởi đồ thị của hàm số $y=f(x)$, trục hoành và hai đường thẳng $x=a$, $x=b$ là $S=\displaystyle\int \limits_a^b |f(x)|\mathrm{\,d}x$.
	}
\end{ex}
\begin{ex}%[Dự án 2025 - đề cấu trúc mới, Hung Doan]%[2D4N3-1]
	Diện tích hình phẳng giới hạn bởi đồ thị của hàm số $y=x^2-4x+3$, trục hoành và hai đường thẳng $x=0$, $x=3$ là
	\choice
	{$S=\pi \displaystyle\int \limits_0^3 \left(x^2-4x+3\right)\mathrm{\,d}x$}
	{$S=\pi \displaystyle\int \limits_0^3 \left|x^2-4x+3\right|\mathrm{\,d}x$}
	{$S=\displaystyle\int \limits_0^3 \left(x^2-4x+3\right)\mathrm{\,d}x$}
	{\True $S=\displaystyle\int \limits_0^3 \left|x^2-4x+3\right|\mathrm{\,d}x$}
	\loigiai{
		Diện tích hình phẳng giới hạn bởi đồ thị của hàm số $y=x^2-4x+3$, trục hoành và hai đường thẳng $x=0$, $x=3$ là
		$$S=\displaystyle\int \limits_0^3 \left|x^2-4x+3\right|\mathrm{\,d}x.$$
	}
\end{ex}
\begin{ex}%[Dự án 2025 - đề cấu trúc mới, Hung Doan]%[2D4N3-1]
	Cho hai hàm số $y=f(x)$, $y=g(x)$ liên tục trên đoạn $[a;b]$. Khi đó, diện tích hình phẳng giới hạn bởi đồ thị của hai hàm số $y=f(x)$, $y=g(x)$ và hai đường thẳng $x=a$, $x=b$ được tính bởi công thức
	\choice
	{$S=\pi \displaystyle\int \limits_a^b [f(x)-g(x)]\mathrm{\,d}x$}
	{$S=\displaystyle\int \limits_a^b \left|f^2(x)-g^2(x)\right|\mathrm{\,d}x$}
	{\True $S=\displaystyle\int \limits_a^b |f(x)-g(x)|\mathrm{\,d}x$}
	{$S=\pi ^2\displaystyle\int \limits_a^b [f(x)-g(x)]\mathrm{\,d}x$}
	\loigiai{
		Diện tích hình phẳng giới hạn bởi đồ thị của hai hàm số $y=f(x)$, $y=g(x)$ và hai đường thẳng $x=a$, $x=b$ là
		$$S=\displaystyle\int \limits_a^b |f(x)-g(x)|\mathrm{\,d}x.$$
	}
\end{ex}
\begin{ex}%[Dự án 2025 - đề cấu trúc mới, Hung Doan]%[2D4N3-1]
	\immini{Cho hàm số $y=f(x)$ có đồ thị như hình vẽ. Diện tích $S$ của hình phẳng trong phần gạch sọc được tính theo công thức
		\choice
		{$S=-\displaystyle\int \limits_a^b f(x)\mathrm{d}x-\displaystyle\int \limits_b^c f(x)\mathrm{d}x$}
		{$S=\displaystyle\int \limits_a^c f(x)\mathrm{d}x$}
		{$S=\displaystyle\int \limits_a^b f(x)\mathrm{d}x+\displaystyle\int \limits_b^c f(x)\mathrm{d}x$}
		{\True $S=-\displaystyle\int \limits_a^b f(x)\mathrm{d}x+\displaystyle\int \limits_b^c f(x)\mathrm{d}x$}}{
		\begin{tikzpicture}[scale=1, font=\footnotesize, line join=round, line cap=round, >=stealth]
			\def\xt{-2.5} \def\xp{3} \def\yt{3.5} \def\yd{-1.5}
			\draw[->, line width=0.8pt](\xt, 0)--(\xp, 0) node[below]{$x$};
			\draw[->, line width=0.8pt](0,\yd)--(0,\yt) node[left]{$y$};
			\node at (0, 0) [below left]{$O$};
			\clip(\xt+0.1,\yd+0.1) rectangle(\xp-0.1,\yt-0.1);
			\draw[smooth,samples=300] plot(\x,{-0.5*(\x)^3+0.2*(\x)^2+2*(\x)+1});
			\node at (-2, 2) [right]{$y=f(x)$};
			\draw[fill=black](-1.423,0) circle (.04)+(-135:.25) node{$a$};
			\draw[fill=black](-0.584,0) circle (.04)+(100:.25) node{$b$};
			\draw[fill=black](2.41,0) circle (.04)+(-135:.25) node{$c$};
			\fill[pattern=north east lines,smooth] (-1.423,0)--plot[domain=-1.423:2.41](\x,{-0.5*(\x)^3+0.2*(\x)^2+2*(\x)+1})--(2.41,0)--cycle;
		\end{tikzpicture}
	}
	\loigiai{
		Diện tích $S$ của hình phẳng trong phần gạch sọc được tính theo công thức là
		$$S=-\displaystyle\int \limits_a^b f(x)\mathrm{d}x+\displaystyle\int \limits_b^c f(x)\mathrm{d}x.$$
	}
\end{ex}
\begin{ex}%[Dự án 2025 - đề cấu trúc mới, Hung Doan]%[2D4N3-3]
	Cho hàm số $y=f(x)$ liên tục trên đoạn $[a;b]$. Gọi $D$ là hình phẳng giới hạn bởi đồ thị hàm số $y=f(x)$, trục hoành và hai đường thẳng $x=a$, $x=b$ $(a<b)$. Thể tích khối tròn xoay tạo thành khi quay $D$ quanh trục $Ox$ được tính theo công thức
	\choice
	{$V=\displaystyle\int \limits_a^b f^2(x)\mathrm{\,d}x$}
	{$V=\pi \displaystyle\int \limits_a^b f(x)\mathrm{\,d}x$}
	{$V=\displaystyle\int \limits_a^b |f(x)|\mathrm{\,d}x$}
	{\True $V=\pi \displaystyle\int \limits_a^b f^2(x)\mathrm{\,d}x$}
	\loigiai{
		Thể tích khối tròn xoay tạo thành khi quay $D$ quanh trục $Ox$ được tính theo công thức là
		$$V=\pi \displaystyle\int \limits_a^b f^2(x)\mathrm{\,d}x.$$
	}
\end{ex}
\begin{ex}%[Dự án 2025 - đề cấu trúc mới, Hung Doan]%[2D4N3-1]
	Diện tích hình phẳng giới hạn bởi đồ thị của hai hàm số $y=x^3-3x$, $y=x$ và hai đường thẳng $x=-1$, $x=3$ được xác định bởi công thức
	\choice
	{$S=\displaystyle\int \limits_{-1}^3 \left(x^3-3x+x\right)\mathrm{\,d}x$}
	{$S=\displaystyle\int \limits_{-1}^3 \left(x^3-3x-x\right)\mathrm{\,d}x$}
	{$S=\displaystyle\int \limits_{-1}^3 \left|x^3-3x+x\right|\mathrm{\,d}x$}
	{\True $S=\displaystyle\int \limits_{-1}^3 \left|x^3-4x\right|\mathrm{\,d}x$}
	\loigiai{
		Diện tích hình phẳng giới hạn bởi $y=x^3-3x$, $y=x$ và $x=-1$, $x=3$ là
		$$S=\displaystyle\int \limits_{-1}^3 \left|x^3-3x-x\right|\mathrm{\,d}x=\displaystyle\int \limits_{-1}^3 \left|x^3-4x\right|\mathrm{\,d}x.$$
	}
\end{ex}
\begin{ex}%[Dự án 2025 - đề cấu trúc mới, Hung Doan]%[2D4N3-1]
	\immini{Cho hàm số $f(x)$ liên tục trên $\mathbb{R}$. Gọi $S$ là diện tích hình phẳng giới hạn bởi các đường $y=f(x)$, $y=0$,$ x=-1$, $x=2$ (như hình vẽ bên). Mệnh đề nào dưới đây đúng?
		\choice
		{$S=-\displaystyle\int \limits_{-1}^1 f(x)\mathrm{\,d}x+\displaystyle\int \limits_1^2 f(x)\mathrm{\,d}x$}
		{$S=\displaystyle\int \limits_{-1}^1 f(x) \mathrm{\,d}x+\displaystyle\int \limits_1^2 f(x) \mathrm{\,d}x$}
		{\True $S=\displaystyle\int \limits_{-1}^1 f(x) \mathrm{\,d}x-\displaystyle\int \limits_1^2 f(x) \mathrm{\,d}x$}
		{$S=-\displaystyle\int \limits_{-1}^1 f(x) \mathrm{\,d}x-\displaystyle\int \limits_1^2 f(x) \mathrm{\,d}x$}}{
		\begin{tikzpicture}[scale=1, font=\footnotesize, line join=round, line cap=round, >=stealth]
			\def\xt{-1.5} \def\xp{3} \def\yt{3} \def\yd{-1.5}
			\draw[->, line width=0.8pt](\xt, 0)--(\xp, 0) node[below]{$x$};
			\draw[->, line width=0.8pt](0,\yd)--(0,\yt) node[left]{$y$};
			\node at (0, 0) [below left]{$O$};
			\clip(\xt+0.1,\yd+0.1) rectangle(\xp-0.1,\yt-0.1);
			\draw[smooth,samples=300] plot(\x,{(\x)^3-2*(\x)^2-(\x)+2});
			\fill[pattern=north east lines,smooth] (-1,0)--plot[domain=-1:2](\x,{(\x)^3-2*(\x)^2-(\x)+2})--(2,0)--cycle;
			\draw[fill=black](-1,0) circle (.04)+(145:.35) node{$-1$};
			\draw[fill=black](1,0) circle (.04)+(-125:.3) node{$1$};
			\draw[fill=black](2,0) circle (.04)+(-45:.3) node{$2$};
		\end{tikzpicture}
	}
	\loigiai{
		Diện tích hình phẳng giới hạn bởi các đường $y=f(x),y=0,x=-1,x=2$ là
		$$S=\displaystyle\int \limits_{-1}^1 f(x) \mathrm{\,d}x-\displaystyle\int \limits_1^2 f(x) \mathrm{\,d}x.$$
	}
\end{ex}
\begin{ex}%[Dự án 2025 - đề cấu trúc mới, Hung Doan]%[2D4N3-1]
	Diện tích hình phẳng giới hạn bởi đồ thị của hai hàm số $y=x^3+2x+1$, $y=x^3+x+3$ và hai đường thẳng $x=1$, $x=3$ được xác định bởi công thức
	\choice
	{$S=\displaystyle\int \limits_1^3 (2x^3+3x+4)\mathrm{\,d}x$}
	{$S=\displaystyle\int \limits_1^3 (x-2)\mathrm{\,d}x$}
	{\True $S=\displaystyle\int \limits_1^3 |x-2|\mathrm{\,d}x$}
	{$S=\displaystyle\int \limits_1^3 |2x^3+3x+4|\mathrm{\,d}x$}
	\loigiai{
		Diện tích hình phẳng giới hạn bởi $y=x^3+2x+1$, $y=x^3+x+3$, $x=1$, $x=3$ là
		$$S=\displaystyle\int \limits_1^3 \left|x^3+2x+1-(x^3+x+3)\right|\mathrm{\,d}x=\displaystyle\int \limits_1^3 |x-2|\mathrm{\,d}x.$$
	}
\end{ex}
\begin{ex}%[Dự án 2025 - đề cấu trúc mới, Hung Doan]%[2D4H3-1]
	Diện tích hình phẳng giới hạn bởi hai đường $y=x^2+2x$ và $y=-x+4$ bằng
	\choice
	{$\dfrac{13}{2}$}
	{$\dfrac{63}{2}$}
	{$\dfrac{205}{6}$}
	{\True $\dfrac{125}{6}$}
	\loigiai{
		\begin{itemize}
			\item Phương trình hoành độ giao điểm của hai đồ thị hàm số $y=x^2+2x$ và $y=-x+4$ là
			$$x^2+2x=-x+4\Leftrightarrow x^2+3x-4=0\Leftrightarrow \hoac{&x=1\\&x=-4.}$$
			\item Diện tích hình phẳng cần tìm là
			\begin{align*}
				S&=\displaystyle\int \limits_{-4}^1 |x^2+2x-(-x+4)|\mathrm{d}x\\
				&=\displaystyle\int \limits_{-4}^1 |x^2+3x-4|\mathrm{d}x\\
				&=\displaystyle\int \limits_{-4}^1 |x^2+3x-4|\mathrm{d}x\\
				&=\displaystyle\int \limits_{-4}^1 (4-3x-x^2)\mathrm{d}x\\
				&=\left(4x-\dfrac{3}{2}x^2-\dfrac{1}{3}x^3\right)\bigg|^1_{-4}=\dfrac{125}{6}.
			\end{align*}
		\end{itemize}
	}
\end{ex}
\begin{ex}%[Dự án 2025 - đề cấu trúc mới, Hung Doan]%[2D4H3-1]
	Diện tích hình phẳng được giới hạn bởi các đường $y=x^2+x-1$ và $y=x^4+x-1$ là
	\choice
	{$\dfrac{8}{15}$}
	{$\dfrac{7}{15}$}
	{$\dfrac{2}{5}$}
	{\True $\dfrac{4}{15}$}
	\loigiai{
		\begin{itemize}
			\item Phương trình hoành độ giao điểm của $y=x^2+x-1$ và $y=x^4+x-1$ là
			$$x^2+x-1=x^4+x-1 \Leftrightarrow x^2-x^4=0\Leftrightarrow \hoac{&x=0\\&x=1\\&x=-1.}$$
			\item Diện tích hình phẳng cần tìm là
			\begin{align*}
				S&=\displaystyle\int \limits_{-1}^1 |x^2-x^4|\mathrm{d}x\\
				&=\displaystyle\int \limits_{-1}^0 |x^2-x^4|\mathrm{d}x+\displaystyle\int \limits_0^1 |x^2-x^4|\mathrm{d}x\\
				&=\left|\displaystyle\int \limits_{-1}^0 (x^2-x^4)\mathrm{d}x\right|+\left|\displaystyle\int \limits_0^1 (x^2-x^4)\mathrm{d}x\right|\\
				&=\left|\left(\dfrac{x^3}{3}-\dfrac{x^5}{5}\right)\bigg|^0_{-1}\right|+\left|\left(\dfrac{x^3}{3}-\dfrac{x^5}{5}\right)\bigg|^1_0 \right|\\
				&=\dfrac{2}{15}+\dfrac{2}{15}=\dfrac{4}{15}.
			\end{align*}
	\end{itemize}}
\end{ex}
\begin{ex}%[Dự án 2025 - đề cấu trúc mới, Hung Doan]%[2D4H3-3]
	Tính thể tích khối tròn xoay được tạo bởi hình phẳng giới hạn bởi đồ thị hàm số $y=3x-x^2$ và trục hoành khi quay quanh trục hoành.
	\choice
	{$\dfrac{85\pi}{7}$}
	{$\dfrac{8\pi}{7}$}
	{\True $\dfrac{81\pi}{10}$}
	{$\dfrac{41\pi}{7}$}
	\loigiai{
		Phương trình hoành độ giao điểm của đồ thị hàm số $y=3x-x^2$ và trục hoành là $$3x-x^2=0\Leftrightarrow \hoac{&x=0\\&x=3.}$$
		Thể tích của khối tròn xoay là $V=\pi \displaystyle\int \limits_0^3 (3x-x^2)^2\mathrm{\,d}x=\dfrac{81\pi}{10}$.}
\end{ex}
\begin{ex}%[Dự án 2025 - đề cấu trúc mới, Hung Doan]%[2D4H3-1]
	Giá trị dương của tham số $m$ sao cho diện tích hình phẳng giới hạn bởi đồ thị của hàm số $y=2x+3$ và các đường thẳng $y=0$, $x=0$, $x=m$ bằng $10$ là
	\choice
	{$m=\dfrac{7}{2}$}
	{$m=5$}
	{\True $m=2$}
	{$m=1$}
	\loigiai{
		Vì $m>0$ nên $2x+3>0$, $\forall x\in [0;m]$.\\
		Diện tích hình phẳng giới hạn bởi đồ thị hàm số $y=2x+3$ và các đường thẳng $y=0$, $x=0$, $x=m$ là
		$$S=\displaystyle\int \limits_0^m (2x+3)\mathrm{d}x=(x^2+3x)\bigg|_0^m=m^2+3m.$$
		Theo giả thiết ta có
		\begin{align*}
			S=10&\Leftrightarrow m^2+3m=10\\
			&\Leftrightarrow m^2+3m-10=0\\
			&\Leftrightarrow \hoac{&m=2\\&m=-5}\\
			&\Leftrightarrow m=2\, (\text{do} m>0).
		\end{align*}
	}
\end{ex}
\Closesolutionfile{ans}
% \indapan{6}{ans/ans-2-C4B13-D1}
\TNTF
\Opensolutionfile{ans}[ans/ans-2-C4B13-D1-DS]
\begin{ex}%[Dự án 2025 - đề cấu trúc mới, Hung Doan]%[2D4H3-1]
	\immini[thm]{Cho đồ thị hàm số $y=f(x)$, và hình phẳng $(H)$ được gạch chéo như hình vẽ. Đặt $a=\displaystyle\int \limits_{-1}^0 f(x)\mathrm{d}x $, $b=\displaystyle\int \limits_0^2 f(x)\mathrm{d}x$.
		\choiceTF
		{Hình phẳng $(H)$ được giới hạn bởi các đường $x=-1$, $x=2$, $y=f(x)$}
		{Hình phẳng $(H)$ có diện tích $S=\left|\displaystyle\int \limits_{-1}^2 f(x)\mathrm{d}x\right|$}
		{\True Hình phẳng $(H)$ có diện tích $S=b-a$}
		{\True $\displaystyle\int \limits_{-1}^2 f(x)\mathrm{d}x>0$}}{
		\begin{tikzpicture}[scale=1, font=\footnotesize, line join=round, line cap=round, >=stealth]
			\def\xt{-2} \def\xp{3} \def\yt{5} \def\yd{-1.5}
			\draw[->, line width=0.8pt](\xt, 0)--(\xp, 0) node[below]{$x$};
			\draw[->, line width=0.8pt](0,\yd)--(0,\yt) node[left]{$y$};
			\node at (0, 0) [below left]{$O$};
			\clip(\xt+0.1,\yd+0.1) rectangle(\xp-0.1,\yt-0.1);
			\draw[smooth,samples=300] plot(\x,{0.4*(\x)^3});
			\fill[pattern=north east lines,smooth] (-1,0)--plot[domain=-1:2](\x,{0.4*(\x)^3})--(2,0)--cycle;
			\draw[fill=black](-1,0) circle (.04)+(145:.35) node{$-1$};
			\draw[fill=black](2,0) circle (.04)+(-90:.3) node{$2$};
			\draw (-1,-0.4)--(-1,0) (2,0)--(2,3.2);
			\node at (2, 3.5) [left]{$f(x)$};
		\end{tikzpicture}
	}
	\loigiai{
		\begin{itemchoice}
			\itemch \textbf{Sai}.\\
			Ta có hình phẳng $(H)$ được giới hạn bởi các đường $x=-1$, $x=2$, $y=f(x)$ và trục $Ox$.
			\itemch \textbf{Sai}.\\
			Hình phẳng $(H)$ có diện tích $S=\displaystyle\int \limits_{-1}^2 \left|f(x)\right|\mathrm{d}x$.
			\itemch \textbf{Đúng}.\\
			Hình phẳng $(H)$ có diện tích $S=b-a$.
			\itemch \textbf{Đúng}.\\
			Ta có $b>a$ nên $S>0$ hay $\displaystyle\int \limits_{-1}^2 f(x)\mathrm{d}x>0$.
		\end{itemchoice}
	}
\end{ex}
\begin{ex}%[Dự án 2025 - đề cấu trúc mới, Hung Doan]%[2D4V3-3]
	\immini{Cho đồ thị của hai hàm số $y=f(x)$, $y=g(x)$ và phần tô màu như hình vẽ.
		\choiceTF
		{Phần hình phẳng tô màu được giới hạn bởi các đường $y=f(x)$, $y=g(x)$, $x=-3$, $x=3$}
		{\True Hình phẳng giới hạn bởi $y=f(x)$, trục $Ox$ có diện tích $S_1=\dfrac{32}{3}$}
		{\True Phần hình phẳng tô màu có diện tích $S_2=\dfrac{9}{2}$}
		{Quay hình phẳng tô màu quanh trục $Ox$ ta được khối tròn xoay có thể tích $V=\dfrac{9}{2}\pi $}}{
		\begin{tikzpicture}[scale=1, font=\footnotesize, line join=round, line cap=round, >=stealth]
			\def\xt{-4} \def\xp{2} \def\yt{4.5} \def\yd{-1}
			\draw[->, line width=0.8pt](\xt, 0)--(\xp, 0) node[below]{$x$};
			\draw[->, line width=0.8pt](0,\yd)--(0,\yt) node[left]{$y$};
			\node at (0, 0) [below left]{$O$};
			\clip(\xt+0.1,\yd+0.1) rectangle(\xp-0.1,\yt-0.1);
			\draw[smooth,samples=300] plot(\x,{-(\x)^2-2*(\x)+3});
			\draw[smooth,samples=300] plot(\x,{(\x)+3});
			\foreach \x/\g in{-3/135,1/45}\draw[fill=black](\x,0) circle (.04)+(\g:.35) node{$\x$};
			\draw[fill=black](0,3) circle (.04)+(0:.35) node{$3$};
			\fill[gray,smooth] (-3,0)--plot[domain=-3:0](\x,{-(\x)^2-2*(\x)+3})--plot[domain=-3:0](\x,{(\x)+3})--cycle;
		\end{tikzpicture}
	}
	\loigiai{
		\begin{itemchoice}
			\itemch \textbf{Sai}.\\
			Phần hình phẳng tô màu được giới hạn bởi các đường $y=f(x)$, $y=g(x)$, $x=-3$, $x=0$.
			\itemch \textbf{Đúng}.\\
			Từ đồ thị suy ra $f(x)=-x^2-2x+3$, $g(x)=x+3$.\\
			Suy ra diện tích $S_1=\displaystyle\int \limits_{-3}^1 \left(-x^2-2x+3\right)\mathrm{d}x=\dfrac{32}{3}$.
			\itemch \textbf{Đúng}.\\
			Phần hình phẳng tô màu có diện tích $S_2=\displaystyle\int \limits_{-3}^0 \left(f(x)-g(x)\right)\mathrm{d}x=\dfrac{9}{2}$.
			\itemch \textbf{Sai}.\\
			Quay hình phẳng tô màu quanh trục $Ox$ ta được khối tròn xoay có thể tích
			$$\begin{aligned}
				V&=\pi \displaystyle\int \limits_{-3}^0 \left[(-x^2-2x+3)^2-(x+3)^2\right]\mathrm{d}x\\
				&=\pi \displaystyle\int \limits_{-3}^0 \left(x^4+4x^3-3x^2-18x\right)\mathrm{d}x\\
				&=\pi \left(\dfrac{x^5}{5}+x^4-x^3-9x^2\right)\bigg|_{-3}^0=\dfrac{108\pi }{5}.
			\end{aligned}$$
		\end{itemchoice}
	}
\end{ex}
\begin{ex}%[Dự án 2025 - đề cấu trúc mới, Hung Doan]%[2D4V3-1]
	Cho hàm số $y=f(x)$ có đồ thị $y=f'(x)$ cắt trục $Ox$ tại ba điểm có hoành độ $a<b<c$ như hình vẽ bên dưới.
	\begin{center}
		\begin{tikzpicture}[scale=1, font=\footnotesize, line join=round, line cap=round, >=stealth]
			\def\xt{-0.5} \def\xp{5} \def\yt{2} \def\yd{-2.5}
			\draw[->](\xt, 0)--(\xp, 0) node[below]{$x$};
			\draw[->](0,\yd)--(0,\yt) node[left]{$y$};
			\node at (0, 0) [below left]{$O$};
			\clip(\xt+0.1,\yd+0.1) rectangle(\xp-0.1,\yt-0.1);
			\draw[smooth,samples=300] plot(\x,{((\x)-1)*((\x)-2)*((\x)-4)});
			\fill (1,0) node[above left]{$a$} circle (1pt);
			\fill (2,0) node[above right]{$b$} circle (1pt);
			\fill (4,0) node[above left]{$c$} circle (1pt);
			\fill[pattern=north east lines,smooth] (1,0)--plot[domain=1:4](\x,{((\x)-1)*((\x)-2)*((\x)-4)})--(4,0)--cycle;
		\end{tikzpicture}
	\end{center}
	\choiceTF
	{\True Hình phẳng gạch sọc được giới hạn bởi các đường $y=f'(x)$ và trục $Ox$}
	{Diện tích hình phẳng gạch sọc $S=\displaystyle\int \limits_a^b f(x)\mathrm{d}x-\displaystyle\int \limits_b^c f(x)\mathrm{d}x$}
	{$\displaystyle\int \limits_a^b f'(x)\mathrm{d}x<\displaystyle\int \limits_b^c f'(x)\mathrm{d}x$}
	{\True $f(b)>f(a)>f(c)$}
	\loigiai{
		\begin{itemchoice}
			\itemch \textbf{Đúng}.\\
			Hình phẳng gạch sọc được giới hạn bởi các đường $y=f'(x)$ và trục $Ox$.
			\itemch \textbf{Sai}.\\
			Diện tích hình phẳng gạch sọc phải là $S=\displaystyle\int \limits_a^b f'(x)\mathrm{d}x-\displaystyle\int \limits_b^c f'(x)\mathrm{d}x$.
			\itemch \textbf{Sai}.
			\begin{center}
				\begin{tikzpicture}[scale=1, font=\footnotesize, line join=round, line cap=round, >=stealth]
					\def\xt{-0.5} \def\xp{5} \def\yt{2} \def\yd{-2.5}
					\draw[->](\xt, 0)--(\xp, 0) node[below]{$x$};
					\draw[->](0,\yd)--(0,\yt) node[left]{$y$};
					\node at (0, 0) [below left]{$O$};
					\clip(\xt+0.1,\yd+0.1) rectangle(\xp-0.1,\yt-0.1);
					\draw[smooth,samples=300] plot(\x,{((\x)-1)*((\x)-2)*((\x)-4)});
					\fill (1,0) node[above left]{$a$} circle (1pt);
					\fill (2,0) node[above right]{$b$} circle (1pt);
					\fill (4,0) node[above left]{$c$} circle (1pt);
					\fill[pattern=north east lines,smooth] (1,0)--plot[domain=1:4](\x,{((\x)-1)*((\x)-2)*((\x)-4)})--(4,0)--cycle;
					\node at (1.5, -0.1) [above]{$S_1$};
					\node at (3, -0.5) [below]{$S_2$};
				\end{tikzpicture}
			\end{center}
			Ta có diện tích $S_1=\displaystyle\int \limits_a^b f'(x)\mathrm{d}x$, diện tích $S_2=-\displaystyle\int \limits_b^c f'(x)\mathrm{d}x$.\\
			Từ hình vẽ ta có $S_1<S_2 \Leftrightarrow \displaystyle\int \limits_a^b f'(x)\mathrm{d}x<-\displaystyle\int \limits_b^c f'(x)\mathrm{d}x$.\\
			\itemch \textbf{Đúng}.\\
			Ta có $\displaystyle\int \limits_a^b f'(x)\mathrm{d}x<-\displaystyle\int \limits_b^c f'(x)\mathrm{d}x\Leftrightarrow f(b)-f(a)<f(b)-f(c) \Leftrightarrow f(a)>f(c)$.\\
			Mặt khác, từ đồ thị hàm $f'(x)$ ta có bảng biến thiên
			\begin{center}
				\begin{tikzpicture}
					\tkzTabInit[nocadre=true,lgt=1.5,espcl=3,deltacl=.55]
					{$x$/0.7, $f'(x)$/0.7, $f(x)$/2}
					{$-\infty$,$a$,$b$,$c$,$+\infty$}
					\tkzTabLine{,-,$0$,+,$0$,-,$0$,+,}
					\tkzTabVar{+/$+\infty$,-/$f(a)$,+/$f(b)$,-/$f(c)$,+/$+\infty$}	
				\end{tikzpicture}
			\end{center}
			Suy ra $f(b)$ lớn hơn $f(a)$ và $f(c)$.\\
			Vậy $f(b)>f(a)>f(c)$.
		\end{itemchoice}
	}
\end{ex}
\begin{ex}%[Dự án 2025 - đề cấu trúc mới, Hung Doan]%[2D4H3-1]
	\immini[thm]{Cho hình vuông $ABCD$ tâm $O$, độ dài cạnh là $4$ cm. Đường cong $BOC$ là một phần của parabol đỉnh $O$ chia hình vuông thành hai hình phẳng có diện tích lần lượt là $S_1$ và $S_2$ (tham khảo hình vẽ).
		\choiceTF
		{Diện tích hình phẳng $S_1=4$}
		{Diện tích hình phẳng $S_2=12$}
		{\True $S_2=2S_1$}
		{$S_2=3S_1$}}{
		\begin{tikzpicture}[scale=1, font=\footnotesize, line join=round, line cap=round, >=stealth]
			\node at (0, 0) [below]{$O$};
			\node at (0, 2.2) {$4$\,cm};
			\node at (2.4, 0) {$4$\,cm};
			\node at (0,1) {$S_1$};
			\node at (0,-1) {$S_2$};
			\fill (-2,-2) node[below left]{$A$};
			\fill (-2,2) node[above left]{$B$};
			\fill (2,2) node[above right]{$C$};
			\fill (2,-2) node[below right]{$D$};
			\clip(-2.1,-2.1) rectangle(2.1,2);
			\draw[smooth,samples=300] plot(\x,{0.5*(\x)^2});
			\draw (2,2)--(2,-2)--(-2,-2)--(-2,2)--(2,2);
		\end{tikzpicture}
		
	}
	\loigiai{
		Gắn hệ trục toạ độ như hình vẽ
		\begin{center}
			\begin{tikzpicture}[scale=1, font=\footnotesize, line join=round, line cap=round, >=stealth]
				\draw[->](-3, 0)--(3, 0) node[below]{$x$};
				\draw[->](0,-3)--(0,3) node[left]{$y$};
				\node at (0, 0) [below left]{$O$};
				\fill (-2,-2) node[below left]{$A$};
				\fill (-2,2) node[above left]{$B$};
				\fill (2,2) node[above right]{$C$};
				\fill (2,-2) node[below right]{$D$};
				\fill (-2,0) node[below left]{$-2$} circle (1pt);
				\fill (0,2) node[above left]{$2$};
				\fill (2,0) node[above right]{$2$};
				\fill (0,-2) node[below right]{$-2$};
				\clip(-2.1,-2.1) rectangle(2.1,2);
				\draw[smooth,samples=300] plot(\x,{0.5*(\x)^2});
				\draw (2,2)--(2,-2)--(-2,-2)--(-2,2)--(2,2);
			\end{tikzpicture}
		\end{center}
		Ta có phương trình parabol $(P)\colon y=\dfrac{1}{2}x^2$.\\
		Suy ra $S_1=\displaystyle\int \limits_{-2}^2 \left(2-\dfrac{1}{2}x^2\right)\mathrm{\,d}x=\dfrac{16}{3}$ (đvdt).\\
		Diện tích hình vuông $ABCD$ là $S_{ABCD}=4^2=16$ (đvdt).\\
		Do đó diện tích $S_2$ là $S_2=S_{ABCD}-S_1=16-\dfrac{16}{3}=\dfrac{32}{3}$ (đvdt).\\
		Vậy tỉ số $\dfrac{S_1}{S_2}=\dfrac{16}{3}\colon \dfrac{32}{3}=\dfrac{1}{2}\Rightarrow S_2=2S_1$.\\
		Khi đó, ta có
		\begin{itemchoice}
			\itemch \textbf{Sai}.
			\itemch \textbf{Sai}.
			\itemch \textbf{Đúng}.
			\itemch \textbf{Sai}.
		\end{itemchoice}
	}
\end{ex}
\Closesolutionfile{ans}
% \indapan{2}{ans/ans-2-C4B13-D1-DS}
\Opensolutionfile{ans}[ans/ans-2-C4B13-D1-KQ]
\TNSA
\begin{ex}%[Dự án 2025 - đề cấu trúc mới, Hung Doan]%[2D4H3-1]
	Tính diện tích hình phẳng giới hạn bởi các đường $y=x^2$, $y=-\dfrac{1}{3}x+\dfrac{4}{3}$ và trục hoành (làm tròn kết quả đến hàng phần trăm).
	\shortans{$1{,}83$}
	\loigiai{
		\begin{center}
			\begin{tikzpicture}[scale=1, font=\footnotesize, line join=round, line cap=round, >=stealth]
				\def\xt{-2} \def\xp{5} \def\yt{3} \def\yd{-1}
				\draw[->, line width=0.8pt](\xt, 0)--(\xp, 0) node[below]{$x$};
				\draw[->, line width=0.8pt](0,\yd)--(0,\yt) node[left]{$y$};
				\node at (0, 0) [below left]{$O$};
				\clip(\xt+0.1,\yd+0.1) rectangle(\xp-0.1,\yt-0.1);
				\draw[ smooth,samples=300] plot(\x,{(\x)^2});
				\draw[ smooth,samples=300] plot(\x,{-1/3*(\x)+4/3});
				\foreach \x in {1,4}
				\draw[shift ={ (\x,0)}]node[below]{$\x$} (0pt,2pt) --(0pt,-2pt);
				\fill[pattern=north east lines,smooth] (0,0)--plot[domain=0:1](\x,{(\x)^2})--(1,0)--cycle;
				\fill[pattern=north east lines,smooth] (1,0)--plot[domain=1:4](\x,{-1/3*(\x)+4/3})--(4,0)--cycle;
				\draw[dashed] (1,0)--(1,1);
			\end{tikzpicture}
		\end{center}
		Phương trình hoành độ giao điểm của các đường là
		\begin{itemize}
			\item $x^2=0\Leftrightarrow x=0$.
			\item $-\dfrac{1}{3}x+\dfrac{4}{3}\Leftrightarrow x=4$.
			\item $x^2=-\dfrac{1}{3}x+\dfrac{4}{3} \Leftrightarrow 3x^2+x-4=0 \Leftrightarrow \hoac{&x=1\\&x=-\dfrac{4}{3}.}$
		\end{itemize}
		Diện tích hình phẳng cần tìm là
		$$S=\displaystyle\int \limits_0^1 x^2\mathrm{d}x+\displaystyle\int \limits_1^4 \left(-\dfrac{1}{3}x+\dfrac{4}{3}\right)\mathrm{d}x =\dfrac{x^3}{3}\bigg|_0^1+\left(-\dfrac{1}{6}x^2+\dfrac{4}{3}x\right)\bigg|_1^4 =\dfrac{11}{6}\approx 1{,}83.$$
	}
\end{ex}
\begin{ex}%[Dự án 2025 - đề cấu trúc mới, Hung Doan]%[2D4V3-1]
	Cho hàm số $y=f(x)$. Hàm số có đồ thị hàm số $y=f'(x)$ như hình vẽ dưới đây.
	\begin{center}
		\begin{tikzpicture}[scale=1, font=\footnotesize, line join=round, line cap=round, >=stealth]
			\def\xt{-2.75} \def\xp{4.75} \def\yt{2.5} \def\yd{-3}
			\draw[->](\xt, 0)--(\xp, 0) node[below]{$x$};
			\draw[->](0,\yd)--(0,\yt) node[left]{$y$};
			\node at (0, 0) [below left]{$O$};
			\node at (-1.5, 1.4) {$y=f'(x)$};
			\foreach \x in {1,4}
			\draw[shift ={ (\x,0)}]node[above]{$\x$} (0pt,2pt) --(0pt,-2pt);
			\draw[shift ={ (-2,0)}]node[above right]{$-2$} (0pt,2pt) --(0pt,-2pt);
			\clip(\xt+0.1,\yd+0.1) rectangle(4,\yt-0.1);
			\draw plot[smooth,tension=0.7] coordinates{(-2.4,2) (-1.2,-1.8) (1,0) (3,-2.5) (4,0)};			
		\end{tikzpicture}
	\end{center}
	Biết diện tích hình phẳng giới hạn bởi trục $Ox$ và đồ thị hàm số $y=f'(x)$ trên đoạn $[-2;1]$ và $[1;4]$ lần lượt bằng $9$ và $12$. Cho biết $f(1)=3$. Tính giá trị biểu thức $P=f(-2)+f(4)$.
	\shortans{$3$}
	\loigiai{
		Ta có $\displaystyle\int \limits_{-2}^1 |f'(x)|\mathrm{d}x=9\Leftrightarrow \displaystyle\int \limits_{-2}^1 f'(x)\mathrm{d}x=-9\Rightarrow f(1)-f(-2)=-9$.\\
		Mà $f(1)=3 \Rightarrow f(-2)=12$.\\
		Ta có $\displaystyle\int \limits_1^4 |f'(x)|\mathrm{d}x=12\Leftrightarrow \displaystyle\int \limits_1^4 f'(x)\mathrm{d}x=-12\Rightarrow f(4)-f(1)=-12$.\\
		Mà $f(1)=3 \Rightarrow f(4)=-9$.\\
		Vậy $P=f(-2)+f(4)=3$.
	}
\end{ex}
\begin{ex}%[Dự án 2025 - đề cấu trúc mới, Hung Doan]%[2D4C3-1]
	Cho hàm số $f(x)=x^3+ax^2+bx+c$ với $a$, $b$, $c$ là các số thực. Biết hàm số $g(x)=f(x)+f'(x)+f''(x)$ có hai giá trị cực trị là $5$ và $2$. Tính diện tích hình phẳng giới hạn bởi đường $y=\dfrac{f(x)}{g(x)+6}$ và $y=1$, kết quả làm tròn đến hàng phần trăm.
	\shortans{$2{,}08$}
	\loigiai{
		Ta có $f'''(x)=6$, khi đó $g'(x)=f'(x)+f''(x)+f'''(x)=f'(x)+f''(x)+6$.\\
		Giả sử $x_1$, $x_2$ ($x_1<x_2$) là hai điểm cực trị của hàm số $g(x)$.\\
		Vì $\lim \limits_{x\to +\infty } g(x)=+\infty $ và $-5$ và $2$ là hai giá trị cực trị của hàm số $g(x)$ nên $\heva{&g(x_1)=2\\&g(x_2)=-5.}$\\
		Phương trình hoành độ giao điểm của $y=\dfrac{f(x)}{g(x)+6}$ và $y=1$ là
		\begin{align*}
			\dfrac{f(x)}{g(x)+6}=1 &\Leftrightarrow g(x)+6=f(x)\\
			&\Leftrightarrow f(x)+f'(x)+f''(x)+6=f(x)\\
			&\Leftrightarrow f'(x)+f''(x)+6=0\\
			&\Leftrightarrow \hoac{&x=x_1\\&x=x_2.} 
		\end{align*}
		Khi đó diện tích hình phẳng cần tìm là
		\begin{align*}
			S&=\displaystyle\int \limits_{x_1}^{x_2} \left|\dfrac{f(x)}{g(x)+6}-1\right|\mathrm{d}x\\
			&=\left|\displaystyle\int \limits_{x_1}^{x_2} \dfrac{f'(x)+f''(x)+6}{g(x)+6}\mathrm{d}x\right|\\
			&=\left|\displaystyle\int \limits_{x_1}^{x_2} \dfrac{g'(x)}{g(x)+6}\mathrm{d}x\right|\\
			&=\left|\ln |g(x)+6|\bigg|_{x_1}^{x_2}\right| \\
			&=\left|\ln |g(x_2)+6|-\ln |g(x_1)+6|\right|\\
			&=\ln 8\approx 2{,}08.
		\end{align*}
	}
\end{ex}
\begin{ex}%[Dự án 2025 - đề cấu trúc mới, Hung Doan]%[2D4V3-2]
	Một viên gạch hoa hình vuông cạnh $40$ cm. Người thiết kế đã sử dụng bốn đường parabol có chung đỉnh tại tâm viên gạch để tạo ra bốn cánh hoa (được tô màu sẫm như hình vẽ bên).
	\begin{center}
		\begin{tikzpicture}[scale=1, font=\footnotesize, line join=round, line cap=round, >=stealth]
			\draw (2,2)--(2,-2)--(-2,-2)--(-2,2)--(2,2);
			\clip(-2,-2) rectangle(2,2);
			\draw[smooth,samples=300] plot(\x,{0.5*(\x)^2});
			\draw[smooth,samples=300] plot(\x,{-0.5*(\x)^2});
			\draw[samples=200,domain=0:2,smooth] plot (\x,{sqrt(2*(\x))});
			\draw[samples=200,domain=0:2,smooth] plot (\x,{-sqrt(2*(\x))});
			\draw[samples=200,domain=-2:0,smooth] plot (\x,{sqrt(-2*(\x))});
			\draw[samples=200,domain=-2:0,smooth] plot (\x,{-sqrt(-2*(\x))});
			\fill[gray,smooth] (-2,0)--plot[domain=-2:0](\x,{-sqrt(-2*(\x))})--(0,0)--cycle;
			\fill[gray,smooth] (-2,0)--plot[domain=-2:0](\x,{sqrt(-2*(\x))})--(0,0)--cycle;
			\fill[gray,smooth] (0,0)--plot[domain=0:2](\x,{-sqrt(2*(\x))})--(2,0)--cycle;
			\fill[gray,smooth] (0,0)--plot[domain=0:2](\x,{sqrt(2*(\x))})--(2,0)--cycle;
			\fill[white,smooth] (-2,0)--plot[domain=-2:2](\x,{-0.5*(\x)^2})--(2,0)--cycle;
			\fill[white,smooth] (-2,0.01)--plot[domain=-2:2](\x,{0.5*(\x)^2})--(2,0.01)--cycle;
		\end{tikzpicture}
	\end{center}
	Diện tích mỗi cánh hoa của viên gạch bằng bằng $\dfrac{a}{b}$ (cm$^2$), với $\dfrac{a}{b}$ là phân số tối giản thì $a$ bằng bao nhiêu?
	\shortans{$400$}
	\loigiai{
		\begin{center}
			\begin{tikzpicture}[scale=1, font=\footnotesize, line join=round, line cap=round, >=stealth]
				\begin{scope}
					\clip(-2,-2) rectangle(2,2);
					\draw[smooth,samples=300] plot(\x,{0.5*(\x)^2});
					\draw[smooth,samples=300] plot(\x,{-0.5*(\x)^2});
					\draw[samples=200,domain=0:2,smooth] plot (\x,{sqrt(2*(\x))});
					\draw[samples=200,domain=0:2,smooth] plot (\x,{-sqrt(2*(\x))});
					\draw[samples=200,domain=-2:0,smooth] plot (\x,{sqrt(-2*(\x))});
					\draw[samples=200,domain=-2:0,smooth] plot (\x,{-sqrt(-2*(\x))});
					\fill[gray,smooth] (-2,0)--plot[domain=-2:0](\x,{-sqrt(-2*(\x))})--(0,0)--cycle;
					\fill[gray,smooth] (-2,0)--plot[domain=-2:0](\x,{sqrt(-2*(\x))})--(0,0)--cycle;
					\fill[gray,smooth] (0,0)--plot[domain=0:2](\x,{-sqrt(2*(\x))})--(2,0)--cycle;
					\fill[gray,smooth] (0,0)--plot[domain=0:2](\x,{sqrt(2*(\x))})--(2,0)--cycle;
					\fill[white,smooth] (-2,0)--plot[domain=-2:2](\x,{-0.5*(\x)^2})--(2,0)--cycle;
					\fill[white,smooth] (-2,0.01)--plot[domain=-2:2](\x,{0.5*(\x)^2})--(2,0.01)--cycle;
				\end{scope}
				\draw[->](-2.75, 0)--(3, 0) node[below]{$x$};
				\draw[->](0,-2.5)--(0,2.75) node[left]{$y$};
				\node at (0, 0) [below left]{$O$};
				\draw[shift ={ (-2,0)}]node[below left]{$-2$} (0pt,2pt) --(0pt,-2pt);
				\draw[shift ={ (2,0)}]node[below right]{$2$} (0pt,2pt) --(0pt,-2pt);
				\draw[shift ={ (0,2)}]node[above right]{$2$} (2pt,0pt) --(-2pt,0pt);
				\draw[shift ={ (0,-2)}]node[below left]{$-2$} (2pt,0pt) --(-2pt,0pt);
				\draw[smooth] (2,2)--(2,-2)--(-2,-2)--(-2,2)--(2,2);
			\end{tikzpicture}
		\end{center}
		Chọn hệ tọa độ như hình vẽ ($1$ đơn vị trên trục bằng $10$ cm=$1$ dm), các cánh hoa tạo bởi các đường parabol có phương trình $y=\dfrac{x^2}{2}$, $y=-\dfrac{x^2}{2}$, $x=-\dfrac{y^2}{2}$, $x=\dfrac{y^2}{2}$.\\
		Diện tích một cánh hoa (nằm trong góc phần tư thứ nhất) bằng diện tích hình phẳng giới hạn bởi hai đồ thị hàm số $y=\dfrac{x^2}{2}$, $y=\sqrt{2x}$ và hai đường thẳng $x=0$; $x=2$.\\
		Do đó diện tích một cánh hoa bằng
		\begin{align*}
			\displaystyle\int \limits_0^2 \left(\sqrt{2x}-\dfrac{x^2}{2}\right)\mathrm{d}x &=\left(\dfrac{2\sqrt{2}}{3}\sqrt{(2x)^3}-\dfrac{x^3}{6}\right)\bigg|\bigg|_0^2\\
			&=\dfrac{4}{3}(\text{dm}^2)=\dfrac{400}{3}\,(\text{cm}^2).
		\end{align*}
		Suy ra $a=400$.
	}
\end{ex}
\begin{ex}%[Dự án 2025 - đề cấu trúc mới, Hung Doan]%[2D4C3-2]
	\immini[thm]{Một bức tường lớn kích thức $8$m $\times$ $8$m trước đại sảnh của một tòa biệt thự được sơn các loại sơn đặc biệt. Người ta vẽ hai nửa đường tròn đường kính $AD$, $AB$ cắt nhau tại $H$; đường tròn tâm $D$, bán kính $AD$, cắt nửa đường tròn đường kính $AB$ tại $K$. Biết tam giác cong $AHK$ được sơn màu xanh và các phần còn lại được sơn màu trắng (như hình vẽ) và một mét vuông sơn trắng, sơn xanh lần lượt có giá là $1$ triệu đồng và $1{,}5$ triệu đồng. Số tiền phải trả là bao nhiêu triệu đồng? (làm tròn đến hàng triệu).}{
		\begin{tikzpicture}[scale=1, font=\footnotesize, line join=round, line cap=round, >=stealth]
			\begin{scope}
				\clip (0,0) rectangle (4,4);
				\fill[pattern=north east lines,smooth] (0,0)--plot[domain=0:4](\x,{sqrt(16-(\x)^2)})--(4,0)--cycle;
				\fill[white,smooth] (0,0)--plot[domain=0:4](\x,{4-sqrt(4*(\x)-(\x)^2)})--(4,0)--cycle;
				\draw[fill=white,samples=200,domain=0:4,smooth] (0,2) circle (2);
				\draw[samples=200,domain=0:4,smooth,variable=\x] plot (\x,{sqrt(16-(\x)^2)});
				\draw[samples=200,domain=0:4,smooth,variable=\x] plot (\x,{4-sqrt(4*(\x)-(\x)^2)});
				%\draw[samples=200,domain=0:4,smooth] (0,2) circle (2);
			\end{scope}
			\draw (0,0)rectangle (4,4);
			\node at (0, 2) [left]{$8$};
			\node at (2,0) [below]{$8$};
			\fill (0,4) node[above left]{$A$} circle(1pt);
			\fill (0,0) node[below left]{$B$} circle(1pt);
			\fill (4,0) node[below right]{$C$} circle(1pt);
			\fill (4,4) node[above right]{$D$} circle(1pt);
			\fill (2,2) node[below left]{$H$} circle(1pt);
			\fill (3.2,2.4) node[below]{$K$} circle(1pt);
			\draw (0,0)rectangle (0.2,0.2);
			\draw (0,4)rectangle (0.2,3.8);
			\draw (4,0)rectangle (3.8,0.2);
			\draw (4,4)rectangle (3.8,3.8);
		\end{tikzpicture}
	}
	\shortans{$67$}
	\loigiai{
		Chọn hệ toạ độ ${Oxy}$ như hình vẽ sau
		\begin{center}
			\begin{tikzpicture}[scale=1, font=\footnotesize, line join=round, line cap=round, >=stealth]
				\begin{scope}
					\clip (0,0) rectangle (4,4);
					\fill[pattern=north east lines,smooth] (0,0)--plot[domain=0:4](\x,{sqrt(16-(\x)^2)})--(4,0)--cycle;
					\fill[white,smooth] (0,0)--plot[domain=0:4](\x,{4-sqrt(4*(\x)-(\x)^2)})--(4,0)--cycle;
					\draw[fill=white,samples=200,domain=0:4,smooth] (0,2) circle (2);
					\draw[samples=200,domain=0:4,smooth,variable=\x] plot (\x,{sqrt(16-(\x)^2)});
					\draw[samples=200,domain=0:4,smooth,variable=\x] plot (\x,{4-sqrt(4*(\x)-(\x)^2)});
					%\draw[samples=200,domain=0:4,smooth] (0,2) circle (2);
				\end{scope}
				\draw (0,0)rectangle (4,4);
				\node at (0, 2) [left]{$8$};
				\node at (2,0) [below]{$8$};
				\fill (0,4) node[above left]{$A$} circle(1pt);
				\fill (0,0) node[below left]{$B$} circle(1pt);
				\fill (4,0) node[below right]{$C$} circle(1pt);
				\fill (4,4) node[above right]{$D$} circle(1pt);
				\fill (2,2) node[below left]{$H$} circle(1pt);
				\fill (3.2,2.4) node[below]{$K$} circle(1pt);
				\draw (0,0)rectangle (0.2,0.2);
				\draw (0,4)rectangle (0.2,3.8);
				\draw (4,0)rectangle (3.8,0.2);
				\draw (4,4)rectangle (3.8,3.8);
				\draw[->](-0.5, 0)--(5, 0) node[below]{$x$};
				\draw[->](0,-0.5)--(0,5) node[left]{$y$};
				\node at (0, 0) [above left]{$O$};
				\draw[dashed] (2,2)--(2,3.4641)node[above]{$E$};
			\end{tikzpicture}
		\end{center}
		Dễ thấy cung $AB$ có phương trình $y=f(x)=8-\sqrt{16-(x-4)^2}$; cung $AH$ có phương trình $y=g(x)=4+\sqrt{16-x^2}$; cung $AC$ có phương trình $y=h(x)=\sqrt{64-x^2}$ và tọa độ các điểm $H(4;4)$ và $K\left(6{,}4;\dfrac{24}{5}\right)$.\\
		Diện tích tam giác $AHK$ là
		\begin{align*}
			S&=S_{AHE}+S_{HEX}\\
			&=\displaystyle\int \limits_0^4 (\sqrt{64-x^2}-4-\sqrt{16-x^2})\mathrm{d}x+\displaystyle\int \limits_4^{6\cdot 4} (\sqrt{64-x^2}-8+\sqrt{16-(x-4)^2})\mathrm{d}x\\
			&\approx 6,25\,5085\,231.
		\end{align*}
		Số tiền cần trả là $S\cdot 1{,}5+(8^2-S)\cdot 1=67{,}12\,754\,262$.\\
		Vậy số tiền cần trả là $67$ (triệu đồng).
	}
\end{ex}
\begin{ex}%[Dự án 2025 - đề cấu trúc mới, Hung Doan]%[2D4V3-5]
	\immini{Một cốc có hình dạng tròn xoay và kích thước như hình vẽ, thiết diện dọc của mặt bên trong cốc (bổ dọc cốc thành $2$ phần bằng nhau) là một đường Parabol. Tính thể tích tối đa mà cốc có thể chứa được (kết quả làm tròn đến chữ số hàng đơn vị).}{
		\definecolor{almond}{rgb}{0.94, 0.87, 0.8}%màu ly
		\definecolor{anti-flashwhite}{rgb}{0.95, 0.95, 0.96}%màu miệng ly
		\begin{tikzpicture}[line join=round, line cap=round,scale=0.5,transform shape,line width=.3mm]
			
			\tikzset{co_vat/.pic={
					\path 
					(1.85,4)coordinate (A)			
					(-1.85,4) coordinate (B)	
					(-3,3)coordinate (C)
					(-3,-1.5)coordinate (D)		
					
					;
					\draw (A)--(B) (C)--(D);
					\foreach\p in {A,B,C,D}
					{\draw[fill=black](\p) circle (1pt);}	
					\node at (-3.8,1) {$10$ cm};
					\node at (0,4.5) {$8$ cm};
					\draw (2,1)--(5,2.5) node[above] {Parabol};
					
					\draw[fill=almond!50!black] (1.75,-4.93)  arc (0:360:1.75 cm and .54cm);
					\draw[fill=almond] (1.75,-4.8)  arc (0:360:1.75 cm and .4cm);
					
					\draw[fill=anti-flashwhite] (1.85,3)  arc (0:360:1.85 cm and .3cm);
					\draw[fill=almond] (1.85,3)
					..controls +(-95:1.3) and +(30:1.7) ..(.4,-1.6) 
					..controls +(-150:.1) and +(90:2) ..(.2,-4.2) 
					..controls +(-150:.1) and +(-30:.1) ..(-.2,-4.2) 
					..controls +(90:2) and +(-30:.1) ..(-.4,-1.6) 
					..controls +(150:1.7) and +(-85:1.3) ..(-1.85,3)
					arc (-180:0:1.85 cm and .3cm);
					;
					\draw[fill=almond] (.2,-4.2) 
					..controls +(-150:.1) and +(-30:.1) ..(-.2,-4.2) 
					..controls +(180:.1) and +(60:.1) ..(-.4,-4.4) 
					..controls +(180:.3) and +(60:.1) ..(-.9,-4.8) 
					..controls +(-30:.4) and +(-150:.4) ..(.9,-4.8) 
					..controls +(120:.1) and +(0:.3) ..(.4,-4.4)
					..controls +(150:.1) and +(-30:.1) ..(.2,-4.2) 
					;
					
			}}
			
			\path
			(0,0)pic[scale=1]{co_vat};
		\end{tikzpicture}
	}
	\shortans{$251$}
	\loigiai{
		\immini{Parabol có phương trình $y=\dfrac{5}{8}x^2\Leftrightarrow x^2=\dfrac{8}{5}y$.\\
			Thể tích tối đa cốc $V=\pi \displaystyle\int \limits_0^{10} \left(\dfrac{8}{5}y\right)\cdot \mathrm{\,d}y\approx 251$.}{
			\begin{tikzpicture}[scale=0.7, font=\footnotesize, line join=round, line cap=round, >=stealth]
				\draw[->, line width=0.8pt](-3.5, 0)--(3.75, 0) node[below]{$x$};
				\draw[->, line width=0.8pt](0,-0.5)--(0,6.5) node[left]{$y$};
				\node at (0, 0) [below left]{$O$};
				\fill (3,0) node[below]{$4$} circle(1pt);
				\fill (-3,0) node[below]{$-4$} circle(1pt);
				\fill (0,5.6125) node[below left]{$10$} circle(1pt);
				\clip(-3,-0.1) rectangle(3,5.75);
				\draw[ smooth,samples=300] plot(\x,{5/8*(\x)^2});
				\draw[dashed] (-3,0)--(-3,5.6125)--(3,5.6125)--(3,0);
			\end{tikzpicture}
		}
	}
\end{ex}
\Closesolutionfile{ans}
% \indapan{6}{ans/ans-2-C4B13-D1-KQ}


% \begin{name}
	{NGUYÊN HÀM - TÍCH PHÂN}
	{KT ỨNG DỤNG NGUYÊN HÀM - TÍCH PHÂN}
	{\tentruong}
	{\thoigian}
\end{name}
\setcounter{ex}{0}\setcounter{bt}{0}
\Opensolutionfile{ans}[ans/ans-2-B13-De2-TN]
\TN
\begin{ex}%[Vovanle]%[2D4N3-1]
Diện tích hình phẳng giới hạn bởi đồ thị hàm số $y=\sin x$, trục hoành và hai đường thẳng $x=0$, $x=2\pi$ được xác định bởi công thức
	\choice
	{$S=\displaystyle\displaystyle\int\limits_0^{2\pi}\sin x\mathrm{\,d}x$}
	{$S=\pi\displaystyle\int\limits_0^{2\pi}\sin x\mathrm{\,d}x$}
	{$S=\pi\displaystyle\int\limits_0^{2\pi}\sin^2 x\mathrm{\,d}x$}
	{\True $S=\displaystyle\int\limits_0^{2\pi}\left| \sin x \right|\mathrm{\,d}x$}
	\loigiai{
Diện tích hình phẳng được tính theo công thức 
$$S=\displaystyle\int\limits_0^{2\pi}\left|\sin x\right|\mathrm{\,d}x.$$
}
\end{ex}
\begin{ex}%[Vovanle]%[2D4N3-1]
Diện tích hình phẳng giới hạn bởi parabol $y=x^2-4$, trục hoành và hai đường thẳng $x=0$, $x=3$ bằng
	\choice
	{\True $\dfrac{23}{3}$}
	{$S=3$}
	{$\dfrac{7}{3}$}
	{$\dfrac{16}{3}$}
	\loigiai{
Diện tích hình phẳng là $$S=\displaystyle\int\limits_0^{3}\left|x^2-4\right|\mathrm{\,d}x=\dfrac{23}{3}.$$
}
\end{ex}
\begin{ex}%[Vovanle]%[2D4N3-3]
Thể tích khối tròn xoay do hình phẳng giới hạn bởi các đường thẳng $y=\sqrt{x}$, trục $Ox$ và hai đường thẳng $x=1$ và $x=2$. Khi quay quanh trục hoành được tính theo công thức nào?
	\choice
	{\True $V=\pi\displaystyle\int\limits_1^2 x\mathrm{\,d}x$}
	{$V=\pi \displaystyle\int\limits_1^2 \sqrt{x}\mathrm{\,d}x$}
	{$V=\pi^2\displaystyle\int\limits_1^2 x\mathrm{\,d}x$}
	{$V=\displaystyle\int\limits_1^2 \left|\sqrt{x}\right|\mathrm{\,d}x$}
	\loigiai{
Thể tích khối tròn xoay do hình phẳng được tính theo công thức
$$V=\pi \displaystyle\int\limits_1^2 \left(\sqrt{x}\right)^2\mathrm{\,d}x=\pi \displaystyle\int\limits_1^2 x\mathrm{\,d}x.$$
}
\end{ex}
\begin{ex}%[Vovanle]%[2D4H3-1]
\immini{Hình phẳng $(H)$ được giới hạn bởi đồ thị hàm số bậc ba và trục hoành được chia thành hai phần có diện tích lần lượt là $S_1$ và $S_2$ (như hình vẽ).\\ 
Biết $\displaystyle\int\limits_{-1}^1f(x)\mathrm{\,d}x=\dfrac{8}{3}$ và $\displaystyle\int\limits_1^4f(x)\mathrm{\,d}x=-\dfrac{63}{8}$. Khi đó diện tích $S$ của hình phẳng $(H)$ bằng
}{
\begin{tikzpicture}[line join=round,line cap=round, font=\footnotesize,scale=0.75,>=stealth]
	\draw[-stealth](-1.5,0)--(4.5,0)node[above]{$x$};
	\draw[-stealth](0,-2.5)--(0,2)node[right]{$y$};			
	\fill (0,0) circle(1pt)node[below left]{$O$}(-0.3,0.3)node[above]{$S_1$}(2.5,-1.2)node[above]{$S_2$};
	\draw[smooth,samples=300,domain=-1.5:4.25] plot(\x,{0.3*((\x)^2-1)*(\x-4)})node[above]{$y=f(x)$};
	\fill[pattern=north east lines]plot[domain=-1:4](\x,{0.3*((\x)^2-1)*(\x-4)})--(-1,0);
	\foreach \x/\g in {-1/140,1/60,4/130}\fill[black] (\x,0) circle (1pt)+(\g:.3)node{$\x$};		
	\end{tikzpicture}
}
	\choice
	{$\dfrac{125}{24}$}
	{$\dfrac{8}3$}
	{\True $\dfrac{253}{24}$}
	{$\dfrac{63}{8}$}
	\loigiai{
Ta có 
$$S_1=\displaystyle\int\limits_{-1}^1f(x)\mathrm{\,d}x=\dfrac{8}{3};
\,S_2=-\displaystyle\int\limits_1^4f(x)\mathrm{\,d}x=\dfrac{63}{8}.$$
Suy ra $S=S_1+S_2=\dfrac{8}{3}+\dfrac{63}{8}=\dfrac{253}{24}$.
}
\end{ex}
\begin{ex}%[Vovanle]%[2D4N3-3]
Hình phẳng giới hạn bởi các đường $y=-x^2+9$, $y=0$, $x=-3$, $x=3$ quay quanh trục $Ox$ tạo thành một khối tròn xoay có thể tích $V$. Khẳng định nào sau đây là đúng?
	\choice
	{$V=\displaystyle\int\limits_{-3}^3\left|-x^2+9\right|\mathrm{\,d}x$}
	{$V=\pi \displaystyle\int\limits_{-3}^3\left|-x^2+9\right|\mathrm{\,d}x$}
	{$V=\displaystyle\int\limits_{-3}^3\left(-x^2+9\right)^2\mathrm{\,d}x$}
	{\True $V=\pi\displaystyle\int\limits_{-3}^3\left(-x^2+9\right)^2\mathrm{\,d}x$}
	\loigiai{
Thể tích khối tròn xoay là
$$V=\pi\displaystyle\int\limits_{-3}^3\left(-x^2+9\right)^2\mathrm{\,d}x.$$
}
\end{ex}
\begin{ex}%[Vovanle]%[2D4N3-1]
Diện tích của hình phẳng giới hạn bởi đồ thị hàm số $y=x^2-4$, trục hoành và hai đường thẳng $x=-2$, $x=2$ bằng
	\choice
	{$S=\pi\displaystyle\int\limits_{-2}^2\left(x^2-4\right)\mathrm{\,d}x$}
	{\True $S=\displaystyle\int\limits_{-2}^2\left|x^2-4\right|\mathrm{\,d}x$}
	{$S=\displaystyle\int\limits_{-2}^2\left(x^2-4\right)\mathrm{\,d}x$}
	{$S=\pi\displaystyle\int\limits_{-2}^2\left(x^2-4\right)^2\mathrm{\,d}x$}
	\loigiai{
Diện tích của hình phẳng là
$$S=\displaystyle\int\limits_{-2}^2\left|x^2-4\right|\mathrm{\,d}x.$$
}
\end{ex}

\begin{ex}%[Vovanle]%[2D4H3-1]
Diện tích hình phẳng giới hạn bởi parabol $y=x^2-4x+5$ và đường thẳng $y=x+1$ được tính theo công thức nào sau đây?
	\choice
	{$S=\displaystyle\int\limits_1^4\left(x^2-5x+4\right)\mathrm{\,d}x$}
	{$S=\displaystyle\int\limits_1^4\left(x^2-5x+4\right)^2\mathrm{\,d}x$}
	{$S=\displaystyle\int\limits_1^4\left|x^2-5x+4\right|\mathrm{\,d}x$}
	{\True $S=\displaystyle\int\limits_1^4\left(x^2+5x+4\right)\mathrm{\,d}x$}
	\loigiai{
Phương trình hoành độ giao điểm của parabol $y=x^2-4x+5$ và đường thẳng $y=x+1$ là
$$x^2-4x+5=x+1\Leftrightarrow x^2-5x+4=0\Leftrightarrow\hoac{&x=1\\&x=4.}$$ 
Diện tích hình phẳng giới hạn bởi parabol $y=x^2-4x+5$ và đường thẳng $y=x+1$ là
$$S=\displaystyle\int\limits_1^4\left|x^2-4x+5-\left(x+1\right)\right|\mathrm{\,d}x=\displaystyle\int\limits_1^4\left|x^2-5x+4\right|\mathrm{\,d}x.$$
}
\end{ex}

\begin{ex}%[Vovanle]%[2D4H3-1]
Diện tích hình phẳng giới hạn bởi đồ thị hàm số $y=x^2$ và đường thẳng $y=2x$ là 
	\choice
	{\True $\dfrac{4}{3}$}
	{$\dfrac{5}{3}$}
	{$\dfrac{3}{2}$}
	{$\dfrac{23}{15}$}
	\loigiai{
Xét phương trình $x^2=2x\Leftrightarrow\hoac{&x=0\\&x=2.}$\\ 
Diện tích hình phẳng giới hạn bởi đồ thị hàm số $y=x^2$ và đường thẳng $y=2x$ là  $$S=\displaystyle\int\limits_0^2\left|x^2-x\right|\mathrm{\,d}x=\left| \displaystyle\int\limits_0^2\left(x^2-x\right)\mathrm{\,d}x\right|=\dfrac{4}{3}.$$
}
\end{ex}

\begin{ex}%[Vovanle]%[2D4H3-1]
\immini{Diện tích phần hình phẳng phần gạch sọc trong hình vẽ được tính theo công thức nào dưới đây? 
	\choice
	{$\displaystyle\int\limits_{-2}^3 \left[f(x)-g(x)\right]\mathrm{\,d}x$}
	{$\displaystyle\int\limits_{-2}^{5} \left[f(x)-g(x)\right]\mathrm{\,d}x+\displaystyle\int\limits_{5}^3 \left[ g(x)-f(x)\right]\mathrm{\,d}x$}
	{\True $\displaystyle\int\limits_{-2}^0 \left[f(x)-g(x)\right]\mathrm{\,d}x+\displaystyle\int\limits_0^3 \left[g(x)-f(x)\right]\mathrm{\,d}x$}
	{$\displaystyle\int\limits_{-2}^0 \left[g(x)-f(x)\right]\mathrm{\,d}x+\displaystyle\int\limits_0^3 \left[f(x)-g(x)\right]\mathrm{\,d}x$}
}{
\begin{tikzpicture}[line join=round,line cap=round, font=\footnotesize,scale=0.5,>=stealth]
	\draw[-stealth](-2.5,0)--(4,0)node[below]{$x$};
	\draw[-stealth](0,-3.7)--(0,7)node[right]{$y$};			
	\fill (0,0) circle(1pt)node[below left]{$O$};
	\draw[smooth,samples=300,domain=-2.2:3.5] plot(\x,{(5/6)*(\x+2)*(\x-1)*(\x-3)})node[above]{$y=f(x)$};
	\draw[smooth,samples=300,domain=-2.4:3.25] plot(\x,{(-5/6)*(\x+2)*(\x-3)})node[below right]{$y=g(x)$};
	\fill[pattern=north east lines]plot[domain=-2:3](\x,{(5/6)*(\x+2)*(\x-1)*(\x-3)})--plot[domain=3:-2](\x,{(-5/6)*(\x+2)*(\x-3)});
	\foreach \x/\g in {-2/150}\fill[black] (\x,0) circle (1pt)+(\g:.6)node{$\x$};
	\foreach \x/\g in {1/-120,3/50}\fill[black] (\x,0) circle (1pt)+(\g:.4)node{$\x$};
	\foreach \x/\g in {5/60}\fill[black] (0,\x) circle (1pt)+(\g:.6)node{$\x$};		
	\end{tikzpicture}
}	
	\loigiai{
Diện tích phần hình phẳng là
$$\displaystyle\int\limits_{-2}^0 \left[f(x)-g(x)\right]\mathrm{\,d}x+\displaystyle\int\limits_0^3 \left[g(x)-f(x)\right]\mathrm{\,d}x.$$
}
\end{ex}

\begin{ex}%[Vovanle]%[2D4V3-1]
Diện tích $S$ của hình phẳng giới hạn bởi đồ thị hai hàm số $y=-x^3$ và $y=x^2-2x$ là
	\choice
	{$S=\dfrac{9}{4}$}
	{$S=\dfrac{7}{3}$}
	{\True $S=\dfrac{37}{12}$}
	{$S=\dfrac{4}{3}$}
	\loigiai{
Hoành độ giao điểm của hai đồ thị là nghiệm của phương trình
$$-x^3=x^2-2x\Leftrightarrow x^3+x^2-2x=0\Leftrightarrow \hoac{&x=-2\\&x=0\\&   x=1.}$$
Diện tích hình phẳng cần tìm là 
\allowdisplaybreaks
\begin{eqnarray*}
S&=&\displaystyle\int\limits_{-2}^0 \left|\left(x^3+x^2-2x\right)\right|\mathrm{\,d}x+\displaystyle\int\limits_0^1 \left|\left(x^3+x^2-2x\right) \right|\mathrm{\,d}x\\
&=&\displaystyle\int\limits_{-2}^0\left(x^3+x^2-2x\right)\mathrm{\,d}x-\displaystyle\int\limits_0^1\left(x^3+x^2-2x \right)\mathrm{\,d}x\\
&=&\left.\left(\dfrac{x^4}4+\dfrac{x^3}3-x^2\right)\right|_{-2}^0-\left.\left(\dfrac{x^4}4+\dfrac{x^3}3-x^2\right)\right|_0^1\\
&=&\dfrac{37}{12}.
\end{eqnarray*}
}
\end{ex}

\begin{ex}%[Vovanle]%[2D4H3-3]
Thể tích vật tròn xoay khi quay hình phẳng $(H)$ xác định bởi các đường $y=\dfrac{1}{3}x^3-x^2$, $y=0$, $x=0$ và $x=3$ quanh trục $Ox$ là
	\choice
	{\True $\dfrac{81\pi}{35}$}
	{$\dfrac{81}{35}$}
	{$\dfrac{71\pi}{35}$}
	{$\dfrac{71}{35}$}
	\loigiai{
Phương trình hoành độ giao điểm 
$$\dfrac{1}{3}x^3-x^2=0\Leftrightarrow \hoac{&x=0\\&x=3.}$$
$$V=\pi\displaystyle\int\limits_0^3\left(\dfrac{1}{3}x^3-x^2\right)^2\mathrm{\,d}x=\pi\displaystyle\int\limits_0^3\left(\dfrac{1}{9}x^6-\dfrac{2}{3}x^5+x^4\right)\mathrm{\,d}x=\dfrac{81\pi}{35}.$$
}
\end{ex}

\begin{ex}%[Vovanle]%[2D4H3-3]
Cho $(H)$ là hình phẳng giới hạn bởi các đường $y=\sqrt{x}$, $y=x-2$ và trục hoành. Biết diện tích của $(H)$ bằng $\dfrac{a}{b}$. Tính giá trị biểu thức $T=a+b$.
	\choice
	{$T=11$}
	{\True $T=13$}
	{$T=10$}
	{$T=19$}
	\loigiai{
\immini{Diện tích của $(H)$ bằng 
$$S=\displaystyle\int\limits_0^2\sqrt{x}\mathrm{\,d}x+\displaystyle\int\limits_2^4\left(\sqrt{x}-x+2\right)\mathrm{\,d}x=\dfrac{10}{3}.$$
Vậy $a=10$; $b=3\Rightarrow a+b=13$.
}{
\begin{tikzpicture}[line join=round,line cap=round, font=\footnotesize,scale=1,>=stealth]
	\draw[-stealth](-0.5,0)--(4.6,0)node[below]{$x$};
	\draw[-stealth](0,-0.5)--(0,2.3)node[right]{$y$};			
	\fill (0,0) circle(1pt)node[below left]{$O$}(4,0) circle(1pt)node[below]{$4$}(0,2) circle(1pt)node[left]{$2$};
	\draw[smooth,samples=300,domain=0:4.5] plot(\x,{sqrt (\x)});
	\draw[smooth,samples=300,domain=4.3:1.4] plot(\x,{\x-2})node[below]{$y=x-2$};
	\draw (1,1)node[above,rotate=30]{$y=\sqrt{x}$};
	\draw [dashed] (4,0)|-(0,2);
	\fill[pattern=north east lines]plot[domain=0:4](\x,{sqrt (\x)})--(2,0)--cycle;			
	\end{tikzpicture}
}	
}
\end{ex}
\Closesolutionfile{ans}
% \indapan{6}{ans/ans-2-B13-De2-TN}

\TNTF
\Opensolutionfile{ans}[ans/ans-2-B13-De2-DS]
\setcounter{ex}{0}
\begin{ex}%[Vovanle]%[2D4H3-1]
\immini{Cho đồ thị hàm số $y=\left(\dfrac{1}{2}\right)^x$, $y=x+1$ và hình phẳng được gạch sọc như hình vẽ.
}{
\begin{tikzpicture}[line join=round,line cap=round, font=\footnotesize,scale=0.75,>=stealth]
	\draw[-stealth](-0.5,0)--(4.6,0)node[below]{$x$};
	\draw[-stealth](0,-0.5)--(0,4)node[right]{$y$};			
	\fill (0,0) circle(1pt)node[below left]{$O$}(2,0) circle(1pt)node[below]{$2$}(0,1) circle(1pt)node[left]{$1$};
	\draw[smooth,samples=300,domain=-0.2:2.5] plot(\x,{\x+1})node[above]{$y=x+1$};
	\draw[smooth,samples=300,domain=-0.2:4] plot(\x,{(0.5)^(\x)})node[above]{$y=\left(\dfrac{1}{2}\right)^x$};	
	\draw (2,3)|-(2,0.25);
	\draw [dashed] (2,0.25)--(2,0);
	\fill[pattern=north east lines]plot[domain=0:2](\x,{\x+1})--(2,0.25)--plot[domain=2:0](\x,{(0.5)^(\x)});			
	\end{tikzpicture}
}
	\choiceTF
	{\True Hình phẳng được gạch sọc giới hạn bởi các đường $x=0$; $x=2$; $y=x+1$; $y=\left(\dfrac{1}{2}\right)^x$}
	{\True Gọi $S_1$ là diện hình phẳng giới hạn bởi trục $Ox$, hai đường thẳng $x=0,\,x=2$ và đồ thị hàm số $y=x+1$. Khi đó $S_1=4$}
	{Gọi $S_2$ là diện hình phẳng giới hạn bởi trục $Ox$, hai đường thẳng $x=0,\,x=2$ và đồ thị hàm số $y=\left(\dfrac{1}{2}\right)^x$. Khi đó $S_2=\dfrac{3}{\ln 2}$}
	{Diện tích hình phẳng được giới hạn bởi các đường $x=0$; $x=2$; $y=x+1$; $y=\left( \dfrac{1}{2}\right)^x$ bằng $4-\dfrac{3}{\ln 2}$}
	\loigiai{

	\begin{itemchoice}
	\itemch Đúng. Hình phẳng được gạch sọc giới hạn bởi các đường $x=0$; $x=2$; $y=x+1$; $y=\left(\dfrac{1}{2}\right)^x$.
	\itemch Đúng. Gọi $S_1$ là diện hình phẳng giới hạn bởi trục $Ox$, hai đường thẳng $x=0,\,x=2$ và đồ thị hàm số $y=x+1$. Khi đó $S_1=\displaystyle\int\limits_0^2\left(x+1\right)\mathrm{\,d}x=\left.\left(\dfrac{x^2}2+x\right)\right|_0^2=2+2=4$.
	\itemch Sai. Gọi $S_2$ là diện hình phẳng giới hạn bởi trục $Ox$, hai đường thẳng $x=0,\,x=2$ và đồ thị hàm số $y=\left( \dfrac{1}{2}\right)^x$. Khi đó $S_2=\displaystyle\int\limits_0^2\left(\dfrac{1}{2}\right)^x\mathrm{\,d}x=\left.\dfrac{\left(\tfrac12\right)^x}{\ln \tfrac12}\right|_0^2=\dfrac{\tfrac{1}{4}-1}{-\ln 2}=\dfrac{3}{4\ln 2}$.
	\itemch Sai. Diện tích hình phẳng được giới hạn bởi các đường $x=0$; $x=2$; $y=x+1$; $y=\left(\dfrac{1}{2}\right)^x$.\\
Ta có $x+1>\left(\dfrac{1}{2}\right)^x$ với mọi $x\in\left[0;2\right]$.\\
Do đó $S=S_1-S_2=4-\dfrac{3}{4\ln 2}$.
	\end{itemchoice}
}
\end{ex}

\begin{ex}%[Vovanle]%[2D4H3-1]
Cho đồ thị các hàm số $y=4-x^2$, $y=x^2$.
\begin{center}
\begin{tikzpicture}[line join=round,line cap=round, font=\footnotesize,scale=1,>=stealth]
\draw[-stealth](-2,0)--(3,0)node[below]{$x$};
	\draw[-stealth](0,-0.5)--(0,4.5)node[right]{$y$};	
\fill[pattern=north east lines]plot[domain=-1:1](\x,{(\x)^2})--plot[domain=1:-1](\x,{-(\x)^2+4});
\draw[smooth,samples=300,domain=-2:2] plot(\x,{(\x)^2})node[above]{$y=x^2$};
\draw[smooth,samples=300,domain=-2.2:2.2] plot(\x,{4-(\x)^2})node[below]{$y=4-x^2$};		
	\fill (0,0) circle(1pt)node[below right]{$O$}(-1,0)circle(1pt)+(-0.1,0) node[below]{$-1$}(1,0)circle(1pt)node[below]{$1$};
	\draw (1,1)--(1,3)(-1,1)--(-1,3);
	\draw[dashed] (1,0)--(1,1)(-1,0)--(-1,1);		
	\end{tikzpicture}
\end{center}
	\choiceTF
	{Hình phẳng được gạch sọc, giới hạn bởi các đường $x=-1$; $x=2$; $y=x^2$; $y=4-x^2$}
	{Gọi $S_1$ là diện hình phẳng giới hạn bởi trục $Ox$, hai đường thẳng $x=-1$, $x=1$ và đồ thị hàm số $y=4-x^2$. Khi đó $S_1=\dfrac{22}{3}$}
	{Gọi $S_2$ là diện hình phẳng giới hạn bởi các đường $y=x^2$; $y=4-x^2$. Khi đó $S_2=16\sqrt2$}
	{Diện tích hình phẳng được giới hạn bởi các đường $x=-1$; $x=1$; $y=x^2$; $y=4-x^2$ là $S=\dfrac{20}3$}
	\loigiai{
	\begin{itemchoice}
	\itemch Sai. Hình phẳng được gạch sọc, giới hạn bởi các đường $x=-1$; $x=1$; $y=x^2$; $y=4-x^2$.	
	\itemch Đúng. $S_1=\displaystyle\int\limits_{-1}^1{\left|4-x^2\right|}\mathrm{\,d}x=\displaystyle\int\limits_{-1}^1{\left(4-x^2\right)}\mathrm{\,d}x=\left.\left( 4x-\dfrac{x^3}3 \right) \right|_{-1}^1=\dfrac{22}3$.
	\itemch Sai. Xét phương trình hoành độ giao điểm $$x^2=4-x^2\Leftrightarrow 2x^2=4\Leftrightarrow x^2=2\Leftrightarrow \hoac{&x=2\\&x=-2.}$$
Do đó 
\allowdisplaybreaks
\begin{eqnarray*}
S_2&=&\displaystyle\int\limits_{-\sqrt2}^{\sqrt2}\left|\left(4-x^2\right)-x^2 \right|\mathrm{\,d}x=\displaystyle\int\limits_{-\sqrt2}^{\sqrt2}\left|4-2x^2\right|\mathrm{\,d}x=\displaystyle\int\limits_{-\sqrt2}^{\sqrt2}\left(4-2x^2\right)\mathrm{\,d}x\\
&=&\left.\left(4x-\dfrac{2x^3}{3}\right)\right|_{-\sqrt2}^{\sqrt2}=\dfrac{16\sqrt2}{3}.
\end{eqnarray*}
	\itemch Đúng. Ta có 
\allowdisplaybreaks
\begin{eqnarray*}	
S&=&\displaystyle\int\limits_{-1}^1\left|\left(4-x^2 \right)-x^2\right|\mathrm{\,d}x=\displaystyle\int\limits_{-1}^1\left|4-2x^2\right|\mathrm{\,d}x=\displaystyle\int\limits_{-1}^1\left(4-2x^2\right)\mathrm{\,d}x\\
&=&\left. \left(4x-\dfrac{2x^3}3\right)\right|_{-1}^1=\dfrac{20}{3}.
\end{eqnarray*}
	\end{itemchoice}
}
\end{ex}

\begin{ex}%[Vovanle]%[2D4H3-3]
Cho đồ thị hàm số $y=5x-x^2$, đường thẳng $y=x$ và phần hình phẳng được gạch sọc như hình vẽ
\begin{center}
\begin{tikzpicture}[line join=round,line cap=round, font=\footnotesize,scale=0.6,>=stealth]
\draw[-stealth](-1,0)--(6,0)node[below]{$x$};
	\draw[-stealth](0,-0.5)--(0,6.5)node[right]{$y$};	
\fill[pattern=north east lines]plot[domain=0:4](\x,\x)--plot[domain=4:0](\x,{-(\x)^2+5*\x});
\draw[smooth,samples=300,domain=-0.5:4.7] plot(\x,\x)node[above]{$y=x$};
\draw[smooth,samples=300,domain=-0.1:5.1] plot(\x,{-(\x)^2+5*\x})node[below]{$y=5x-x^2$};		
	\fill (0,0) circle(1pt)node[below right]{$O$}(4,0)circle(1pt) node[below]{$4$}(0,4)circle(1pt)node[left]{$4$};
	\draw[dashed] (4,0)|-(0,4);		
	\end{tikzpicture}
\end{center}
	\choiceTF
	{\True Diện tích phần hình phẳng được gạch sọc trong hình vẽ là $\dfrac{32}{3}$}
	{Diện tích hình phẳng giới hạn bởi đường cong $y=5x-x^2$, trục hoành và hai đường thẳng $x=0$, $x=5$ là $\dfrac{125}{3}$}
	{\True Thể tích khi quay phần hình phẳng giới hạn bởi đồ thị hàm số $y=5x-x^2$ và đường thẳng $y=x$ quanh trục $Ox$ là $\dfrac{384\pi}{5}$}
	{\True Thể tích khi quay phần hình phẳng giới hạn bởi đường thẳng $y=x$, trục $Ox$, hai đường thẳng $x=2$, $x=5$ quanh trục $Ox$ là $39\pi$}
	\loigiai{
	\begin{itemchoice}
	\itemch Đúng. Xét phương trình hoành độ giao điểm 
	$$5x-x^2=x\Leftrightarrow 4x-x^2=0\Leftrightarrow \hoac{&x=0\\&x=4.}$$
Diện tích hình phẳng được gạch sọc giới hạn bởi hai đường là 
$$\displaystyle\int\limits_0^4{\left(5x-x^2-x \right)}\mathrm{\,d}x=\displaystyle\int\limits_0^4{\left(4x-x^2 \right)}\mathrm{\,d}x=\left.\left( 2x^2-\dfrac{x^3}{3}\right)\right|_0^4=\dfrac{32}3.$$
	\itemch Sai. Diện tích hình phẳng giới hạn bởi đường cong $y=5x-x^2$, trục hoành và hai đường thẳng $x=0$, $x=5$ là 
	$$\displaystyle\int\limits_0^{5}{\left(5x-x^2\right)\mathrm{\,d}x}=\left. \left(\dfrac{5x^2}{2}-\dfrac{x^3}{3}\right)\right|_0^{5}=\dfrac{125}{6}.$$
	\itemch Đúng. Phương trình hoành độ giao điểm hai đường là 
	$$5x-x^2=x\Leftrightarrow 4x-x^2=0\Leftrightarrow \hoac{&x=0\\&x=4.}$$
Thể tích khi quay phần hình phẳng giới hạn bởi đồ thị hàm số $y=5x-x^2$ và đường thẳng $y=x$ quanh trục $Ox$ là
\allowdisplaybreaks
\begin{eqnarray*}
V&=&\pi \displaystyle\int\limits_0^4{{{\left( 5x-x^2 \right)}^2}}\mathrm{\,d}x-\pi \displaystyle\int\limits_0^4{x^2}\mathrm{\,d}x=\pi\displaystyle\int\limits_0^4{\left(x^4-10x^3+24x^2 \right)}\mathrm{\,d}x\\
&=&\pi \left. \left( \dfrac{{x^{5}}}{5}-\dfrac{5x^4}2+8x^3 \right) \right|_0^4=\dfrac{384\pi}{5}.
\end{eqnarray*} 
	\itemch Đúng. Thể tích khi quay phần hình phẳng giới hạn bởi đường thẳng $y=x$, trục $Ox$, hai đường thẳng $x=2$, $x=5$ quanh trục $Ox$ là
$$V_1=\pi \displaystyle\int\limits_2^{5}x^2\mathrm{\,d}x=\pi\left.\dfrac{x^3}{3}\right|_2^{5}=39\pi.$$
	\end{itemchoice}
}
\end{ex}

\begin{ex}%[Vovanle]%[2D4H3-3]
Cho hai đồ thị hàm số $y=x^2-2x-1$ và $y=-x^2+3$ và phần hình phẳng được gạch chéo như hình vẽ.
\begin{center}
\begin{tikzpicture}[line join=round,line cap=round, font=\footnotesize,scale=1,>=stealth]
\draw[-stealth](-2,0)--(3,0)node[below]{$x$};
	\draw[-stealth](0,-2.5)--(0,3.5)node[right]{$y$};	
\fill[pattern=north east lines]plot[domain=2:-1](\x,{(\x)^2-2*\x-1})--plot[domain=-1:2](\x,{-(\x)^2+3});
\draw[smooth,samples=300,domain=-1.2:3] plot(\x,{(\x)^2-2*\x-1})node[above]{$y=x^2-2x-1$};
\draw[smooth,samples=300,domain=-1.8:2.2] plot(\x,{-(\x)^2+3})node[below]{$y=-x^2+3$};		
	\fill (0,0) circle(1pt)node[below right]{$O$}(-1,0)circle(1pt)+(0.1,0) node[below]{$-1$}(2,0)circle(1pt)node[above]{$2$};
	\draw[dashed] (2,0)--(2,-1)(-1,0)--(-1,2);		
	\end{tikzpicture}
\end{center}
	\choiceTF
	{\True Biểu thức diện tích phần hình phẳng gạch chéo trong hình vẽ là 
	$\displaystyle\int\limits_{-1}^2 \left(-2x^2+2x+4 \right)\mathrm{\,d}x$}
	{\True Diện tích hình phẳng giới hạn bởi đường cong $y=x^2-2x-1$, trục hoành và hai đường thẳng $x=0$, $x=1$ là $\dfrac{5}{3}$}
	{Thể tích khi quay phần hình phẳng giới hạn bởi đồ thị hàm số $y=-x^2+3$, trục $Ox$, hai đường thẳng $x=1$, $x=2$ quanh trục $Ox$ là $\dfrac{\pi}{5}$}
	{Thể tích khi quay phần hình phẳng giới hạn bởi đồ thị hàm số $y=x^2-2x-1$, trục $Ox$, hai đường thẳng $x=-1$, $x=2$ quanh trục $Ox$ là $\dfrac{33}{5}$}
	\loigiai{
	\begin{itemchoice}
	\itemch Đúng. Dựa vào hình vẽ ta có diện tích hình phẳng được gạch chéo trong hình vẽ được xác định là biểu thức 
	$$\displaystyle\int_{-1}^2\left[\left(-x^2+2\right)-\left(x^2-2x-2\right)\right]\mathrm{\,d}x=\displaystyle\int_{-1}^2\left(-2x^2+2x+4\right)\mathrm{\,d}x.$$ 	
	\itemch Đúng. Diện tích hình phẳng giới hạn bởi đường cong $y=x^2-2x-1$, trục hoành và hai đường thẳng $x=0,x=1$ là 
	$$\displaystyle\int\limits_0^1\left| x^2-2x-1\right|\mathrm{\,d}x=\dfrac{5}{3}.$$	
	\itemch Sai. Thể tích khi quay phần hình phẳng giới hạn bởi đồ thị hàm số $y=-x^2+3$, trục $Ox$, hai đường thẳng $x=1$, $x=2$ quanh trục $Ox$ là 
	$$\pi\displaystyle\int\limits_1^2(-x^2+3)^2\mathrm{\,d}x=\dfrac{6\pi}{5}.$$	
	\itemch Sai. Thể tích khi quay phần hình phẳng giới hạn bởi đồ thị hàm số $y=x^2-2x-1$, trục $Ox$, hai đường thẳng $x=-1$, $x=2$ quanh trục $Ox$ là 
	$$\pi\displaystyle\int\limits_{-1}^2(x^2-2x-1)^2\mathrm{\,d}x=\dfrac{33\pi }{5}.$$
	\end{itemchoice}
}
\end{ex}
\Closesolutionfile{ans}
\indapan3{ans/ans-2-B13-De2-DS}

\Opensolutionfile{ans}[ans/ans-2-B13-De2-KQ]
\TNSA
\setcounter{ex}{0}
\begin{ex}%[Vovanle]%[2D4H3-1]
Tính diện tích hình phẳng giới hạn bởi đồ thị của hàm số $y=x^3-3x$; $y=x$, hai đường thẳng $x=-1$; $x=2$.
\shortans{$5{,}75$}
	\loigiai{
Diện tích hình phẳng cần tìm là $S=\displaystyle\int\limits_{-1}^2{\left| x^3-3x-x \right|\mathrm{\,d}x=\displaystyle\int\limits_{-1}^2{\left| x^3-4x \right|\mathrm{\,d}x.}}$\\
Ta có $$x^3-3x=x\Leftrightarrow x(x^2-4)=0\Leftrightarrow \hoac{&x=0\\ &x=-2\notin[-1;2]\\&x=2.}$$
Phương trình có hai nghiệm thuộc đoạn $\left[-1;2\right]$ là $x=0$; $x=2$.
\allowdisplaybreaks
\begin{eqnarray*}
  S&=&\displaystyle\int\limits_{-1}^2{\left|x^3-4x\right|}\mathrm{\,d}x=\displaystyle\int\limits_{-1}^0{\left|x^3-4x\right|}\mathrm{\,d}x+\displaystyle\int\limits_0^2{\left| x^3-4x\right|}\mathrm{\,d}x\\
  &=&\left|\displaystyle\int\limits_{-1}^0{(x^3-4x)\mathrm{\,d}x}\right|+\left| \displaystyle\int\limits_0^2{(x^3-4x)\mathrm{\,d}x}\right|\\ 
 &=&\left| \left.\left(\dfrac{x^4}4-2x^2 \right)\right|_0^1\right|+\left|\left. \left(\dfrac{x^4}4-2x^2 \right)\right|_0^2 \right|=\dfrac{23}{4}\approx 5{,}75. 
\end{eqnarray*}
}
\end{ex}

\begin{ex}%[Vovanle]%[2D4H3-3]
Cho hình phẳng giới hạn bởi các đường $y=\sqrt{x}-2$, $y=0$ và $x=9$ quay xung quanh trục $Ox$. Tính thể tích khối tròn xoay tạo thành (làm tròn kết quả thể tích đến hàng phần trăm).
\shortans{$5{,}76$}
	\loigiai{
Phương trình hoành độ giao điểm của đồ thị hàm số $y=\sqrt{x}-2$ và trục hoành 
$$\sqrt{x}-2=0\Leftrightarrow \sqrt{x}=2\Leftrightarrow x=4.$$
Thể tích của khối tròn xoay tạo thành là
\allowdisplaybreaks
\begin{eqnarray*}
V&=&\pi \displaystyle\int\limits_4^{9}{{{\left(\sqrt{x}-2\right)}^2}\mathrm{\,d}x}\\
&=&\pi\displaystyle\int\limits_4^{9}{\left(x-4\sqrt{x}+4\right)}\mathrm{\,d}x\\
&=&\pi\left.\left(\dfrac{x^2}2-\dfrac{8x\sqrt{x}}3+4x\right)\right|_4^{9}\\
&=&\pi\left(\dfrac{81}{2}-72+36\right)-\pi\left(\dfrac{16}{2}-\dfrac{64}{3}+16\right)\\
&=&\dfrac{11\pi}{6}\approx 5{,}76.
\end{eqnarray*}
}
\end{ex}

\begin{ex}%[Vovanle]%[2D4V3-1]
\immini{Cho hàm số $y=ax^4+bx^2+c$ có đồ thị $(C)$, biết rằng $(C)$ đi qua điểm $A(-1;0)$, tiếp tuyến $d$ tại $A$ của $(C)$, cắt $(C)$ tại hai điểm có hoành độ lần lượt là $0$ và $2$. Diện tích hình phẳng giới hạn bởi $d$, đồ thị $(C)$ và hai đường thẳng $x=0$; $x=2$ có diện tích bằng $\dfrac{28}{5}$ (phần gạch sọc trong hình vẽ).
 
Tính diện tích hình phẳng giới hạn bởi $(C)$, trục hoành và hai đường thẳng $x=-1$; $x=0$.
}{
\begin{tikzpicture}[line join=round,line cap=round, font=\footnotesize,scale=0.5,>=stealth]
\draw[-stealth](-2,0)--(2.5,0)node[below]{$x$};
	\draw[-stealth](0,-0.7)--(0,7)node[right]{$y$};	
\fill[pattern=north east lines]plot[domain=0:2](\x,{(\x)^4-3*(\x)^2+2})--cycle;
\draw[smooth,samples=300,domain=-2.02:2.02] plot(\x,{(\x)^4-3*(\x)^2+2});
\draw[smooth,samples=300,domain=-1.5:2.3] plot(\x,{2*(\x+1)});		
	\fill (0,0) circle(1pt)node[below right]{$O$}(-1,0)circle(1pt)+(0.1,0) node[below]{$-1$}(2,0)circle(1pt)node[below]{$2$};
	\draw[dashed] (2,0)--(2,6);		
	\end{tikzpicture}
}	
\shortans{$0{,}2$}
	\loigiai{
Ta có $y'=4ax^3+2bx$ $\Rightarrow d\colon y=\left(-4a-2b\right)\left(x+1\right)$.
Phương trình hoành độ giao điểm của $d$ và $(C)$ là $\left(-4a-2b\right)\left(x+1\right)=ax^4+bx^2+c.\hfill(1)$\\
Phương trình $(1)$ phải cho $2$ nghiệm là $x=0$, $x=2$.
$$\Rightarrow\heva{&-4a-2b=c\\&-12a-6b=16a+4b+c}
\Leftrightarrow \heva{&-4a-2b-c=0&(2)\\&28a+10b+c=0&(3).}$$
Mặt khác, diện tích phần gạch sọc là 
\allowdisplaybreaks
\begin{eqnarray*}
&&\dfrac{28}{5}=\displaystyle\int\limits_0^2{\left[\left(-4a-2b \right)\left( x+1 \right)-ax^4-bx^2-c\right]\mathrm{\,d}x}\\
&\Leftrightarrow& \dfrac{28}{5}=4\left(-4a-2b\right)-\dfrac{32}{5}a-\dfrac{8}3b-2c\\
&\Leftrightarrow& \dfrac{112}{5}a+\dfrac{32}3b+2c=-\dfrac{28}{5}\qquad(4)
\end{eqnarray*}
Giải hệ 3 phương trình $(2)$, $(3)$ và $(4)$ ta được $a=1$, $b=-3$, $c=2$.\\
Khi đó, $(C)\colon y=x^4-3x^2+2$, $d\colon y=2\left(x+1\right)$.\\
Diện tích cần tìm là 
$$S=\displaystyle\int\limits_{-1}^0\left[x^4-3x^2+2-2\left(x+1\right)\right]\mathrm{\,d}x=\displaystyle\int\limits_{-1}^0\left(x^4-3x^2-2x\right)\mathrm{\,d}x=\dfrac1{5}=0{,}2.$$
}
\end{ex}

\begin{ex}%[Vovanle]%[2D4V3-2]
\immini{Một khuôn viên dạng nửa hình tròn có đường kính bằng $4\sqrt{5}$ (m). Trên đó người thiết kế hai phần để trồng hoa có dạng của một cánh hoa hình parabol có đỉnh trùng với tâm nửa hình tròn và hai đầu mút của cánh hoa nằm trên nửa đường 
}{
\begin{tikzpicture}[line join=round,line cap=round, font=\footnotesize,scale=0.5,>=stealth]
\path
({-2*sqrt (5)},0) coordinate (A)
({2*sqrt (5)},0) coordinate (B)
(2,4) coordinate (M)
(-2,4) coordinate (N)
(-2,0) coordinate (C)
(2,0) coordinate (D)
($(M)!0.5!(D)$) coordinate (G)node[right]{$4$ m}
;
\fill[pattern=north east lines]plot[domain=-2:2](\x,{sqrt (20-(\x)^2)})--plot[domain=2:-2](\x,{(\x)^2});	
	\draw (A) arc(180:0:{2*sqrt (5)});
	\draw plot[domain=-2:2](\x,{(\x)^2});
	\draw[dashed] (N)--(C)(M)--(D)(M)--(N);
	\path ($(M)!0.5!(N)$) coordinate (H)node[below]{$4$ m};
	\draw (A)--(B);			
	\end{tikzpicture}
}
\noindent tròn (phần gạch sọc), cách nhau một khoảng bằng $4\,\mathrm{m}$, phần còn lại của khuôn viên (phần không gạch sọc) dành để trang trí cỏ nhân tạo. Biết các kích thước cho như hình vẽ và kinh phí cỏ nhân tạo là $100\,000$ đồng/m$^2$. Hỏi cần bao nhiêu tiền để trang trí cỏ trên phần đất đó? (Số tiền được làm tròn đến hàng nghìn).	
\shortans{$1948$}
	\loigiai{
\immini{Đặt hệ trục tọa độ như hình vẽ. Khi đó phương trình nửa đường tròn là
$$y=\sqrt{R^2-x^2}=\sqrt{\left(2\sqrt{5}\right)^2-x^2}=\sqrt{20-x^2}.$$
Phương trình parabol $(P)$ có đỉnh là gốc $O$ sẽ có dạng $y=ax^2$. Mặt khác $(P)$ qua điểm $M(2;4)$. 
}{
\begin{tikzpicture}[line join=round,line cap=round, font=\footnotesize,scale=0.6,>=stealth]
\path
({-2*sqrt (5)},0) coordinate (A)
({-2*sqrt (5)},0) coordinate (B)
(2,4) coordinate (M)node[above right]{$M(2,4)$}
(-2,4) coordinate (N)
(-2,0) coordinate (C)
(2,0) coordinate (D)
;
\fill[pattern=north east lines]plot[domain=-2:2](\x,{sqrt (20-(\x)^2)})--plot[domain=2:-2](\x,{(\x)^2});
	\draw[-stealth](-5,0)--(5,0)node[below]{$x$};
	\draw[-stealth](0,-0.7)--(0,5)node[right]{$y$};	
	\fill (0,0) circle(1pt)node[below left]{$O$}(-2,0)circle(1pt) node[below]{$-2$}(2,0)circle(1pt)node[below]{$2$}(0,4)circle(1pt)node[below left]{$4$};
	\draw (A) arc(180:0:{2*sqrt (5)});
	\draw plot[domain=-2:2](\x,{(\x)^2});
	\draw[dashed] (N)--(C)(M)--(D)(M)--(N);				
	\end{tikzpicture}
}	
\noindent Do đó $4=a\cdot(-2)^2\Rightarrow a=1$.\\
Phần diện tích của hình phẳng giới hạn bởi $(P)$ và nửa đường tròn.(phần gạch sọc).\\
Ta có công thức $S_1=\displaystyle\int\limits_{-2}^2{\left(\sqrt{20-x^2}-x^2 \right)\mathrm{\,d}x}\approx 11{,}94\,\mathrm{m}^2$.\\
Vậy phần diện tích trồng cỏ là $S_{\text{cỏ}}=\dfrac{1}{2}{{S}_{\text{htron}}}-S_1=\dfrac{1}{2}\cdot 20\cdot \pi-11{,}94\approx 19{,}476\,\mathrm{m}^2$.\\
Số tiền cần có là $S_{\text{cỏ}}\times 100000\approx 1947592\text{ (đồng)}\approx 1948$ (nghìn đồng).
}
\end{ex}

\begin{ex}%[Vovanle]%[2D4V3-4]
Một téc nước hình trụ, đang chứa nước được đặt nằm ngang, có chiều dài $3$ m và đường kính đáy $1$ m. Hiện tại mặt nước trong téc cách phía trên đỉnh của téc $0{,}25$ m (xem hình vẽ). 
\begin{center}
\begin{tikzpicture}[line join=round,line cap=round, font=\footnotesize,scale=1,>=stealth]
\def \x{0.5}
\def \y{1.5}
\def \z{6}
\path
(80:{\x} and {\y}) coordinate (A)
(80:{\x} and {\y})+(\z,0) coordinate (B)
(-80:{\x} and {\y}) coordinate (C)
(-80:{\x} and {\y})+(\z,0) coordinate (D)
(10:{\x} and {\y}) coordinate (E)
(200:{\x} and {\y}) coordinate (F)
($(B)!0.5!(D)$) coordinate (K)
($(A)!0.5!(C)$) coordinate (H)
(40:{\x} and {\y}) coordinate (M)
(160:{\x} and {\y}) coordinate (N)
(40:{\x} and {\y})+(\z,0) coordinate (U)
(160:{\x} and {\y})+(\z,0) coordinate (V)
(C)+(0,-0.7) coordinate (I)
(D)+(0,-0.7) coordinate (J)
(B)+(2,0) coordinate (P)
(D)+(2,0) coordinate (Q)
(intersection of B--D and U--V) coordinate (T)
(B)+(0.6,0) coordinate (G)
(T)+(0.6,0) coordinate (R)
($(P)!0.5!(Q)$) coordinate (m)node[right]{$1$ m}
($(I)!0.5!(J)$) coordinate (X)node[above]{$3$ m}
($(G)!0.5!(R)$) coordinate (Z)node[right]{$0{,}25$ m}
;
\fill[blue!20] plot [domain=-90:-200] (0.5*cos \x,{1.5*sin \x})--(M)--(U)--plot [domain=40:-90] (\z+0.5*cos \x,{1.5*sin \x})--cycle;
\draw  (A) arc (80:280:{\x} and {\y});
\draw[dashed] (C) arc (-80:80:{\x} and {\y});
\draw  (B) arc (80:280:{\x} and {\y});
\draw (D) arc (-80:80:{\x} and {\y});
\draw (A)--(B)(C)--(D)(U)--(V)(N)--(V);
\draw[<->](I)--(J);
\draw[<->](P)--(Q);
\draw[<->](G)--(R);
\draw[dashed](C)--(I)(D)--(J)(B)--(P)(C)--(Q)(M)--(N)(M)--(U)(T)--(R);

\end{tikzpicture}
\end{center}
Tính thể tích của nước trong téc (kết quả làm tròn đến hàng phần trăm)?
\shortans{$1{,}9$}
	\loigiai{
\immini{Thế tích phần dầu còn lại sẽ bằng diện tích hình phẳng gạch sọc trong hình nhân với chiều dài của bồn (chiều cao của trụ).
 
Đường tròn có tâm $O(0;0)$, $R=0{,}5$ có phương trình là 
$$x^2+y^2=0{,}25 \Leftrightarrow y=\pm \sqrt{0{,}25-x^2}.$$
 Diện tích hình gạch sọc chính là diện tích hình phẳng giới hạn bởi các đường 
 $$y=\sqrt{0{,}25-x^2};\,y=-\sqrt{0{,}25-x^2};\,x=-0{,}5;\,x=0{,}25.$$
Do đó 
$$V=Sh=3 \displaystyle\int_{-0{,}5}^{0{,}25}\left|\sqrt{0{,}25-x^2}-\left(-\sqrt{0{,}25-x^2}\right)\right|\mathrm{\,d}x \approx 1{,}896\mathrm{\,m}^3 \approx 1{,}9\mathrm{\,m}^3.$$
}{
\begin{tikzpicture}[line join=round,line cap=round, font=\footnotesize,scale=1,>=stealth]
\def \r{1.5}
\fill[pattern=north east lines]plot[domain=60:300](\r*cos \x,\r*sin \x)--cycle;
	\draw[-stealth](-2,0)--(2.5,0)node[below]{$x$};
	\draw[-stealth](0,-2)--(0,2)node[right]{$y$};	
	\fill (0,0) circle(1pt)node[below left]{$O$}(\r,0)circle(1pt)+(-0.1,0.2) node[right]{$0{,}5$}(-\r,0)circle(1pt)+(0.1,0.2) node[left]{$-0{,}5$}(0,-\r)circle(1pt)+(0.1,-0.2) node[left]{$-0{,}5$}(0,\r)circle(1pt)+(0.1,0.2) node[left]{$0{,}5$}(0.5*\r,0)circle(1pt)+(-0.1,-0.2) node[right]{$0{,}25$};
	\draw (60:\r)--(-60:\r);
	\draw (0,0) circle(\r);			
	\end{tikzpicture}
}	
}
\end{ex}

\begin{ex}%[Vovanle]%[2D4V3-1]
\immini{Cho hai đường tròn $\left(O_1;5\right)$ và $\left(O_2;3\right)$ cắt nhau tại hai điểm $A$, $B$ sao cho $AB$ là một đường kính của đường tròn $\left(O_2\right)$. Gọi $(D)$ là hình thẳng được giới hạn bởi hai đường tròn (phần ở ngoài đường tròn lớn, được gạch chéo như hình vẽ). Một vật trang trí có dạng một khối tròn xoay được tạo thành khi quay miền $(D)$ quanh trục $O_1O_2$. Thể tích của khối tròn xoay được tạo thành có $V=\dfrac{a\pi}{b}$ ($\dfrac{a}{b}$ là phân số tối giản) thì $a^2+b^3$ bằng bao nhiêu?
}{
\begin{tikzpicture}[line join=round,line cap=round, font=\footnotesize,scale=1,>=stealth]
\def \r{0.3}
 \path
    (-4*\r,0) coordinate (O_1)
    (0,0) coordinate (O_2)
    (90:3*\r) coordinate (A)
    (-90:3*\r) coordinate (B)    
    ;
    \pgfmathsetmacro\g{atan (3/4)}
    \fill[pattern=north east lines]plot[domain=-\g:\g](5*\r*cos \x-4*\r,5*\r*sin \x)--plot[domain=90:-90](3*\r*cos \x,3*\r*sin \x);	
	\draw (O_1) circle(5*\r);
	\draw (O_2) circle(3*\r);
	\draw (2*\r,0) coordinate (D)node[above]{$(D)$};
	\draw (A)--(B)(-9*\r,0)--(3*\r,0);
	\foreach \x/\g in {O_1/90,O_2/140,A/60,B/-60}\fill[black] (\x) circle (1pt)+(\g:.3)node{$\x$};		
	\end{tikzpicture}
}
\shortans{$1627$}
	\loigiai{
\immini{Chọn hệ tọa độ $Oxy$ với \\
$O_2\equiv O$, $O_2C\equiv Ox$, $O_2A\equiv Oy$.\\
Đoạn $O_1O_2=\sqrt{O_1A^2-O_2A^2}=\sqrt{5^2-3^2}=4$.\\
Suy ra $\left(O_1\right):{{\left( x+4 \right)}^2}+y^2=25$.\\
Kí hiệu $\left(H_1\right)$ là hình phẳng giới hạn bởi các đường $\left(O_1\right)\colon \left(x+4\right)^2+y^2=25$, $Oy\colon x=0$, $x\geq 0$.\\
Kí hiệu $\left(H_2\right)$ là hình phẳng giới hạn bởi các đường $\left(O_2\right)\colon x^2+y^2=9$, $Oy\colon x=0$, $x\geq 0$.
}{
\begin{tikzpicture}[line join=round,line cap=round, font=\footnotesize,scale=1,>=stealth]
\def \r{0.4}
 \path
    (-4*\r,0) coordinate (O_1)
    (0,0) coordinate (O_2)
    (90:3*\r) coordinate (A)
    (-90:3*\r) coordinate (B)    
    ;
    \pgfmathsetmacro\g{atan (3/4)}
    \fill[pattern=north east lines]plot[domain=-\g:\g](5*\r*cos \x-4*\r,5*\r*sin \x)--plot[domain=90:-90](3*\r*cos \x,3*\r*sin \x);
	\draw[-stealth](-9.5*\r,0)--(4*\r,0)node[below]{$x$};
	\draw[-stealth](0,-5.5*\r)--(0,5.5*\r)node[right]{$y$};		
	\draw (O_1) circle(5*\r);
	\draw (O_2) circle(3*\r);
	\draw (2*\r,0) coordinate (D)node[above]{$(D)$};
	\foreach \x/\g in {O_1/90,O_2/140,A/60,B/-60}\fill[black] (\x) circle (1pt)+(\g:.3)node{$\x$};		
	\end{tikzpicture}
}
\noindent Khi đó thể tích $V$ cần tìm chính bằng thể tích $V_2$ của khối tròn xoay thu được khi quay hình $\left(H_2\right)$ xung quanh trục $Ox$ trừ đi thể tích $V_1$ của khối tròn xoay thu được khi quay hình $\left(H_1\right)$ xung quanh trục $Ox$.\\
Ta có $V_2=\dfrac{1}{2}\cdot \dfrac{4}{3}\pi r^3=\dfrac{2}{3}\pi {\cdot 3^3}=18\pi$.\\
Lại có $V_1=\pi\displaystyle\int\limits_0^1y^2\mathrm{\,d}x=\pi\displaystyle\int\limits_0^1\left[25-\left(x+4\right)^2\right]\mathrm{\,d}x=\left.\pi \left[25x-\dfrac{\left(x+4\right)^3}{3}\right]\right|_0^1
=\dfrac{14\pi}{3}$.\\
Do đó $V=V_2-V_1=18\pi-\dfrac{14\pi}{3}=\dfrac{40\pi}{3}$.\\
Vậy $a^2+b^3=1627$.
}
\end{ex}
\Closesolutionfile{ans}
% \indapan{6}{ans/ans-2-B13-De2-KQ}


%Chương V. Mặt phẳng, đt trong kg
%%Bài 1.
% \chapter{PHƯƠNG PHÁP TỌA ĐỘ TRONG KHÔNG GIAN}
\section{PHƯƠNG TRÌNH MẶT PHẲNG}
% \chude{Xác định các yếu tố cơ bản liên quan đến mặt phẳng}
\begin{dang}{Xác định véctơ pháp tuyến của mặt phẳng. Xác định điểm thuộc và không thuộc mặt phẳng}
	\begin{enumerate}[label=\bf\arabic*.]
		\item \textbf{véctơ pháp tuyến của mặt phẳng:}
		\begin{itemize}
			\item Mặt phẳng $(\alpha)\colon A x+B y+C z+D=0$ có véctơ pháp tuyến $\overrightarrow{n}=(A; B; C)$.
			\item Nếu mặt phẳng $(\alpha)$ có cặp véctơ chỉ phương là $\overrightarrow{a}, \overrightarrow{b}$ thì $(\alpha)$ có véctơ pháp tuyến là $\overrightarrow{n}=\left[\overrightarrow{a}, \overrightarrow{b}\right]$.
			\item véctơ pháp tuyến của mặt phẳng $(\alpha)$ là véctơ có giá vuông góc với $(\alpha)$.
			\item véctơ chỉ phương của mặt phẳng $(\alpha)$ là véctơ có giá song song hoặc trùng với $(\alpha)$.
			\item Nếu $\overrightarrow{n}$ là một véctơ pháp tuyến của $(\alpha)$ thì $k \cdot \overrightarrow{n}$ cũng là một véctơ pháp tuyến của $(\alpha)$.
			\item Nếu $\overrightarrow{a}$ là một véctơ chỉ phương của $(\alpha)$ thì $k \cdot \overrightarrow{a}$ cũng là một véctơ chỉ phương của $(\alpha)$.
			\item[] \textbf{Chú ý:}
			\item Trục $O x$ có véctơ chỉ phương là $\overrightarrow{i}=(1; 0; 0)$.
			\item Trục $O y$ có véctơ chỉ phương là $\overrightarrow{j}=(0; 1; 0)$.
			\item Trục $O z$ có véctơ chỉ phương là $\overrightarrow{k}=(0; 0; 1)$.
			\item Mặt phẳng $(O x y)$ có véctơ pháp tuyến là $\overrightarrow{k}=(0; 0; 1)$.
			\item Mặt phẳng $(O x z)$ có véctơ pháp tuyến là $\overrightarrow{j}=(0; 1; 0)$.
			\item Mặt phẳng $(O y z)$ có véctơ pháp tuyến là $\overrightarrow{i}=(1; 0; 0)$.
		\end{itemize}
		\item \textbf{Điểm thuộc và không thuộc mặt phẳng:}\\
		Cho mặt phẳng $(\alpha)$ có phương trình $A x+B y+C z+D=0$. Khi đó: 
		\begin{itemize}
			\item $N_0\left(x_0; y_0; z_0\right) \in(\alpha) \Leftrightarrow A x_0+B y_0+C z_0+D=0$.
			\item $N_0\left(x_0; y_0; z_0\right) \notin(\alpha) \Leftrightarrow A x_0+B y_0+C z_0+D \neq 0$.	
		\end{itemize}
	\end{enumerate}
\end{dang}

\TN
\Opensolutionfile{ans}[ans/ans2C5B1CD1]
\begin{ex}%[2H2H2-5]
	Trong không gian $O x y z$, tọa độ một véctơ $\overrightarrow{n}$ vuông góc với cả hai véctơ $\overrightarrow{a}=(1; 1;-2), \overrightarrow{b}=(1; 0; 3)$ là
	\choice
	{$(2; 3;-1)$}
	{$(3; 5;-2)$}
	{$(2;-3;-1)$}
	{\True $(3;-5;-1)$}
	\loigiai{
		véctơ $\overrightarrow{n}$ vuông góc với cả hai véctơ $\overrightarrow{a}, \overrightarrow{b}$.\\
		Do đó $\overrightarrow{n}=\left[\overrightarrow{a}, \overrightarrow{b}\right]$.\\
		Ta có $\left[\overrightarrow{a}, \overrightarrow{b}\right]=(3;-5;-1)$.	
	}
\end{ex}

\begin{ex}%[2H2H2-5]
	Trong không gian với hệ tọa độ $O x y z$, cho hai véctơ $\overrightarrow{a}=(2; 1;-2)$ và véctơ $\overrightarrow{b}=(1; 0; 2)$. Tìm tọa độ véctơ $\overrightarrow{c}$ là tích có hướng của $\overrightarrow{a}$ và $\overrightarrow{b}$.
	\choice
	{$\overrightarrow{c}=(2; 6;-1)$}
	{$\overrightarrow{c}=(4; 6;-1)$}
	{$\overrightarrow{c}=(4;-6;-1)$}
	{\True $\overrightarrow{c}=(2;-6;-1)$}
	\loigiai{
		Áp dụng công thức tính tích có hướng trong hệ trục tọa độ $O x y z$, ta được
		$$\overrightarrow{c}=\left[\overrightarrow{a}, \overrightarrow{b}\right]=(2;-6;-1).$$	
	}
\end{ex}

\begin{ex}%[2H2H2-5]
	Trong không gian với hệ trục tọa độ $O x y z$, cho $A(2; 1;-3), B(0;-2; 5)$ và $C(1; 1; 3)$. Tìm tọa độ véctơ $\overrightarrow{n}$ có phương vuông góc với hai véctơ $\overrightarrow{A B}$ và $\overrightarrow{A C}$.
	\choice
	{$\overrightarrow{n}=(8; 4;-3)$}
	{$\overrightarrow{n}=(-18; 0;-3)$}
	{\True $\overrightarrow{n}=(-18; 4;-3)$}
	{$\overrightarrow{n}=(1; 4;-3)$}
	\loigiai{
		Ta có $\overrightarrow{A B}=(-2;-3; 8)$ và $\overrightarrow{A C}=(-1; 0; 6)$. Suy ra $\left[\overrightarrow{A B}, \overrightarrow{A C}\right]=(-18; 4;-3)$.\\
		Vậy $\overrightarrow{n}=\left[\overrightarrow{A B}, \overrightarrow{A C}\right]=(-18; 4;-3)$.
	}
\end{ex}

\begin{ex}%[2H5N1-3]
	Trong không gian $O x y z$, phương trình nào sau đây là phương trình tổng quát của mặt phẳng?
	\choice
	{$x-3 y^2+z-1=0$}
	{$x^2+2 y+4 z-2=0$}
	{\True $2 x-3 y+4 z-2024=0$}
	{$2 x-3 y+4 z^2-2025=0$}
	\loigiai{
		Phương trình tổng quát của mặt phẳng là $2 x-3 y+4 z-2024=0$.	
	}
\end{ex}

\begin{ex}%[2H5H1-3]
	Trong không gian $O x y z$, cho mặt phẳng $(P)\colon 3 x-y+2 z-1=0$. véctơ nào dưới đây \textbf{không phải} là một véctơ pháp tuyến của $(P)$?
	\choice
	{$\overrightarrow{n}=(-3; 1;-2)$}
	{\True $\overrightarrow{n}=(3; 1; 2)$}
	{$\overrightarrow{n}=(3;-1; 2)$}
	{$\overrightarrow{n}=(6;-2; 4)$}
	\loigiai{
		Véctơ pháp tuyến của $(P)$ là $\overrightarrow{n}=(3;-1; 2)$.\\
		$\overrightarrow{n}=(-3; 1;-2)=-1(3;-1; 2)$ là một véctơ pháp tuyến của $(P)$.\\
		$\overrightarrow{n}=(6;-2; 4)=2(3;-1; 2)$ là một véctơ pháp tuyến của $(P)$. 	
	}
\end{ex}

\begin{ex}%[2H5H1-3]
	Trong không gian với hệ tọa độ $O x y z$, véctơ nào dưới đây là một véctơ pháp tuyến của mặt phẳng $(O x y)$?
	\choice
	{$\overrightarrow{i}=(1; 0; 0)$}
	{$\overrightarrow{m}=(1; 1; 1)$}
	{$\overrightarrow{j}=(0; 1; 0)$}
	{\True $\overrightarrow{k}=(0; 0; 1)$}
	\loigiai{
		Do mặt phẳng $(O x y)$ vuông góc với trục $O z$ nên nhận véctơ $\overrightarrow{k}=(0; 0; 1)$ làm một véctơ pháp tuyến.	
	}
\end{ex}

\begin{ex}%[2H5H1-3]
	Trong không gian $O x y z$, véctơ nào dưới đây có giá vuông góc với mặt phẳng $(\alpha)\colon 2 x-3 y+1=0$?
	\choice
	{$\overrightarrow{a}=(2;-3; 1)$}
	{$\overrightarrow{b}=(2; 1;-3)$}
	{\True $\overrightarrow{c}=(2;-3; 0)$}
	{$\overrightarrow{d}=(3; 2; 0)$}
	\loigiai{
		Mặt phẳng $(\alpha)$ có một véctơ pháp tuyến là $\overrightarrow{n}=(2;-3; 0)=\overrightarrow{c}$.	
	}
\end{ex}

\begin{ex}%[2H5H1-3]
	Trong không gian $O x y z$, một véctơ pháp tuyến của mặt phẳng $\dfrac{x}{-2}+\dfrac{y}{-1}+\dfrac{z}{3}=1$ là
	\choice
	{\True $\overrightarrow{n}=(3; 6;-2)$}
	{$\overrightarrow{n}=(2;-1; 3)$}
	{$\overrightarrow{n}=(-3;-6;-2)$}
	{$\overrightarrow{n}=(-2;-1; 3)$}
	\loigiai{
		Phương trình $\dfrac{x}{-2}+\dfrac{y}{-1}+\dfrac{z}{3}=1 \Leftrightarrow-\dfrac{1}{2} x-y+\dfrac{1}{3} z-1=0 \Leftrightarrow 3 x+6 y-2 z+6=0.$\\
		Do đó mặt phẳng đã cho có một véctơ pháp tuyến là $\overrightarrow{n}=(3; 6;-2)$.	
	}
\end{ex}

\begin{ex}%[2H5H1-3]
	Trong không gian $O x y z$, điểm nào dưới đây nằm trên mặt phẳng $(P)\colon 2 x-y+z-2=0$.
	\choice
	{$Q(1;-2; 2)$}
	{$P(2;-1;-1)$}
	{$M(1; 1;-1)$}
	{\True $N(1;-1;-1)$}
	\loigiai{
		Thay toạ độ điểm $Q$ vào phương trình mặt phẳng $(P)$ ta được $2\cdot 1-(-2)+2-2=4 \neq 0$ nên $Q \notin(P)$.\\
		Thay toạ độ điểm $P$ vào phương trình mặt phẳng $(P)$ ta được $2\cdot2-(-1)+(-1)-2=2 \neq 0$ nên $P \notin(P)$.\\
		Thay toạ độ điểm $M$ vào phương trình mặt phẳng $(P)$ ta được $2\cdot1-1+(-1)-2=-2 \neq 0$ nên $M \notin(P)$.\\
		Thay toạ độ điểm $N$ vào phương trình mặt phẳng $(P)$ ta được $2 \cdot 1-(-1)+(-1)-2=0$ nên $N \in(P)$.	
	}
\end{ex}

\begin{ex}%[2H5H1-3]
	Trong không gian với hệ tọa độ $O x y z$, cho mặt phẳng $(\alpha)\colon x+y+z-6=0$. Điểm nào dưới đây \textbf{không thuộc} $(\alpha)$?
	\choice
	{$Q(3; 3; 0)$}
	{$N(2; 2; 2)$}
	{$P(1; 2; 3)$}
	{\True $M(1;-1; 1)$}
	\loigiai{
		\begin{itemize}
			\item  Thay $Q(3; 3; 0)$  vào phương trình mặt phẳng $(\alpha)$, ta được $3+3+0-6=0 \Rightarrow Q \in(\alpha)$.
			\item  Thay $N(2; 2; 2)$ vào phương trình mặt phẳng $(\alpha)$, ta được  $2+2+2-6=0 \Rightarrow N \in(\alpha)$.
			\item Thay $P(1; 2; 3)$ vào phương trình mặt phẳng $(\alpha)$, ta được $1+2+3-6=0 \Rightarrow P \in(\alpha)$.
			\item  Thay $M(1;-1; 1)$ toạ độ vào phương trình mặt phẳng $(\alpha)$, ta được $1-1+1-6\neq 0 \Rightarrow M \notin(\alpha)$.	 
		\end{itemize}
	}
\end{ex}

\begin{ex}%[2H5H1-3]
	Trong không gian với hệ tọa độ $O x y z$, cho mặt phẳng $(P)\colon x-2 y+z-5=0$. Điểm nào dưới đây thuộc $(P)$?
	\choice
	{$P(0; 0;-5)$}
	{\True $M(1; 1; 6)$}
	{$Q(2;-1; 5)$}
	{$N(-5; 0; 0)$}
	\loigiai{
		Ta có $1-2 \cdot 1+6-5=0$ nên $M(1; 1; 6)$ thuộc mặt phẳng $(P)$.	
	}
\end{ex}

\begin{ex}%[2H5H1-3]
	Trong không gian $O x y z$, mặt phẳng $(P)\colon \dfrac{x}{1}+\dfrac{y}{2}+\dfrac{z}{3}=1$ \textbf{không} đi qua điểm nào dưới đây?
	\choice
	{$P(0; 2; 0)$}
	{\True $N(1; 2; 3)$}
	{$M(1; 0; 0)$}
	{$Q(0; 0; 3)$}
	\loigiai{
		Thế tọa độ điểm $N$ vào phương trình mặt phẳng $(P)$ ta có $\dfrac{1}{1}+\dfrac{2}{2}+\dfrac{3}{3}=1$ (sai).\\
		Vậy mặt phẳng $(P)\colon \dfrac{x}{1}+\dfrac{y}{2}+\dfrac{z}{3}=1$ không đi qua điểm $N(1; 2; 3)$.	
	}
\end{ex}

\begin{ex}%[2H5H1-3]
	Trong không gian $O x y z$, mặt phẳng $(\alpha)\colon x-y+2 z-3=0$ đi qua điểm nào dưới đây?
	\choice
	{\True $M\left(1; 1; \dfrac{3}{2}\right)$}
	{$N\left(1;-1;-\dfrac{3}{2}\right)$}
	{$P(1; 6; 1)$}
	{$Q(0; 3; 0)$}
	\loigiai{
		Xét điểm $M\left(1; 1; \dfrac{3}{2}\right)$, ta có $1-1+2 \cdot \dfrac{3}{2}-3=0$ (đúng) nên $M \in(\alpha)$ .\\
		Xét điểm $N\left(1;-1;-\dfrac{3}{2}\right)$, ta có $1+1+2.\left(-\dfrac{3}{2}\right)-3=0$ (sai) nên $N \notin(\alpha)$.\\
		Xét điểm $P(1; 6; 1)$, ta có $1-6+2.1-3=0$ (sai) nên $P \notin(\alpha)$.\\
		Xét điểm $Q(0; 3; 0)$, ta có $0-3+2.0-3=0$ (sai) nên $Q \notin(\alpha)$.	
	}
\end{ex}
\Closesolutionfile{ans}
\indapan{10}{ans/ans2C5B1CD1}
\TNTF
\Opensolutionfile{ans}[ans/ans2C5B1CD1-DS]
\begin{ex}%[2H5H1-2]
	Trong không gian cho hệ tọa độ $O x y z$. Các mệnh đề sau đây đúng hay sai?
	\choiceTF
	{\True Mặt phẳng $(O x y)$ có một véctơ pháp tuyến là $\overrightarrow{n}=(0; 0; 1)$}
	{\True Mặt phẳng $(O x z)$ có véctơ pháp tuyến là $\overrightarrow{n}=(0; 3; 0)$}
	{\True Mặt phẳng $(O y z)$ có véctơ pháp tuyến là $\overrightarrow{n}=(-2; 0; 0)$}
	{\True Trục $O z$ có véctơ chỉ phương là $\overrightarrow{a}=(0; 0;-2024)$}
	\loigiai{
		\begin{itemchoice}
			\itemch Mặt phẳng $(O x y)$ có một véctơ pháp tuyến là $\overrightarrow{n}=(0; 0; 1)$.
			\itemch Mặt phẳng $(O x z)$ có véctơ pháp tuyến là $\overrightarrow{n}=(0; 3; 0)$.
			\itemch Mặt phẳng $(O y z)$ có véctơ pháp tuyến là $\overrightarrow{n}=(-2; 0; 0)$.
			\itemch Trục $O z$ có véctơ chỉ phương là $\overrightarrow{a}=(0; 0;-2024)$.
		\end{itemchoice}
	}
\end{ex}

\begin{ex}%[2H2H2-5]%[2H2H2-1] 
	Trong không gian với hệ toạ độ $O x y z$, cho $\overrightarrow{a}=(1;-2; 3)$ và $\overrightarrow{b}=(1; 1;-1)$. Các mệnh đề sau đây đúng hay sai? 
	\choiceTF
	{\True $\left|\overrightarrow{a}+\overrightarrow{b}\right|=3$}
	{\True $\overrightarrow{a} \cdot \overrightarrow{b}=-4$}
	{\True $\left|\overrightarrow{a}-\overrightarrow{b}\right|=5$}
	{$\left[\overrightarrow{a}, \overrightarrow{b}\right]=(-1;-4; 3)$}
	\loigiai{
		\begin{itemchoice}
			\itemch $\left|\overrightarrow{a}+\overrightarrow{b}\right|=\left|\overrightarrow{a}+\overrightarrow{b}\right|=\sqrt{(1+1)^2+(-2+1)^2+(3-1)^2}=\sqrt{4+1+4}=3$.
			\itemch $\overrightarrow{a} \cdot \overrightarrow{b}=1 \cdot 1+(-2) \cdot 1+3 \cdot(-1)=1-2-3=-4$.
			\itemch $\left|\overrightarrow{a}+\overrightarrow{b}\right|=\left|\overrightarrow{a}+\overrightarrow{b}\right|=\sqrt{(1-1)^2+(-2-1)^2+(3+1)^2}=\sqrt{0+9+16}=5$.
			\itemch 
			$\left[\overrightarrow{a}, \overrightarrow{b}\right]=\left(\left|\begin{array}{cc}-2 & 3 \\ 1 &-1\end{array}\right|;\left|\begin{array}{cc}3 & 1 \\-1 & 1\end{array}\right|;\left|\begin{array}{cc}1 &-2 \\ 1 & 1\end{array}\right|\right)=(-1; 4; 3)$.
		\end{itemchoice}
	}
\end{ex}
\begin{ex}%[2H2H2-4]%[2H2H2-1]
	Trong không gian với hệ trục tọa độ $O x y z$, cho ba véctơ $\overrightarrow{a}=(1; 2;-1), \overrightarrow{b}=(3;-1; 0), \overrightarrow{c}=(1;-5; 2)$. Các mệnh đề sau đây đúng hay sai?
	\choiceTF
	{$\overrightarrow{a}$ cùng phương với $\overrightarrow{b}$}
	{$\left[\overrightarrow{a}, \overrightarrow{b}\right] \cdot \overrightarrow{c}=0$}
	{$\overrightarrow{a}$ không cùng phương với $\overrightarrow{b}$}
	{$\overrightarrow{a}$ vuông góc với $\overrightarrow{b}$}
	\loigiai{   
		\begin{itemchoice}
			\itemch Ta có:
			$\left[\overrightarrow{a}, \overrightarrow{b}\right]=(-1;-3;-7) \neq \overrightarrow{0}$.
			\itemch Hai véctơ $\overrightarrow{a}, \overrightarrow{b}$ không cùng phương.
			\itemch $\left[\overrightarrow{a}, \overrightarrow{b}\right] \cdot \overrightarrow{c}=-1+15-14=0$.
			\itemch Ba véctơ $\overrightarrow{a}, \overrightarrow{b}, \overrightarrow{c}$ đồng phẳng.
		\end{itemchoice}
	}
\end{ex}
\begin{ex}%[2H5H1-2]
	Trong không gian $O x y z$, cho mặt phẳng $(P)\colon 2 x+3 y+z-2024=0$. Các mệnh đề sau đây đúng hay sai?
	\choiceTF
	{\True Mặt phẳng $(P)$ có một véctơ pháp tuyến là $\overrightarrow{n}=(2; 3; 1)$}
	{\True Mặt phẳng $(P)$ có véctơ pháp tuyến là $\overrightarrow{n}=(6; 9; 3)$}
	{\True Mặt phẳng $(P)$ có véctơ pháp tuyến là $\overrightarrow{n}=(-4;-6;-2)$}
	{Điểm $M(0; 0; 2024)$ không thuộc mặt phẳng $(P)$}
	\loigiai{
		\begin{itemchoice}
			\itemch Véctơ pháp tuyến của $(P)$ là $\overrightarrow{n}=(2; 3; 1)$.
			\itemch $\overrightarrow{n}=(6; 9; 3)=3(2; 3; 1).$
			\itemch $\overrightarrow{n}=(-4;-6;-2)=-2(2; 3; 1).$
			\itemch Thay điểm $M(0; 0; 2024)$ vào mặt phẳng $(P)\colon 2\cdot0+3\cdot 0+2024-2024=0 \Rightarrow M \in(P)$.
		\end{itemchoice}
	}
\end{ex}
\begin{ex}%[2H5H1-3] 
	Trong không gian $O x y z$, cho mặt phẳng $(P)\colon x+y+z-3=0$. Các mệnh đề sau đây đúng hay sai?
	\choiceTF
	{\True Điểm $M(-1;-1;-1)$ \textbf{không thuộc} mặt phẳng $(P)$}
	{\True Điểm $N(1; 1; 1)$ \textbf{thuộc} mặt phẳng $(P)$}
	{\True Điểm $K(-3; 0; 0)$ \textbf{không thuộc} mặt phẳng $(P)$}
	{Điểm $Q(0; 0;-3)$ \textbf{thuộc} mặt phẳng $(P)$}
	\loigiai{
		\begin{itemchoice}
			\itemch Điểm $M(-1;-1;-1)$ có tọa độ không thỏa mãn phương trình mặt phẳng $(P)$ nên $M \notin(P)$.
			\itemch Điểm $N(1; 1; 1)$ có tọa độ thỏa mãn phương trình mặt phẳng $(P)$ nên $N \in(P)$.
			\itemch Điểm $K(-3; 0; 0)$ có tọa độ không thỏa mãn phương trình mặt phẳng $(P)$ nên $K \notin(P)$.
			\itemch Điểm $Q(0; 0;-3)$ có tọa độ không thỏa mãn phương trình mặt phẳng $(P)$ nên $Q \notin(P)$.
		\end{itemchoice}
	}
\end{ex}
\Closesolutionfile{ans}
\indapan{2}{ans/ans2C5B1CD1-DS}
\TNSA
\Opensolutionfile{ans}[ans/ans2C5B1CD1-KQ]
\begin{ex}%[2H2H2-5]
	Trong không gian với hệ trục tọa độ $O x y z$, cho $A(0; 1;-1)$, $B(1; 1; 2)$ và $C(1;-1; 0)$. Biết  $\vec{u}=\left[\overrightarrow{B C}, \overrightarrow{B D}\right]$. Khi đó, độ dài của $\vec{u}$ bằng bao nhiêu?
	\shortans[0]{$4$}
	\loigiai{
		Ta có $\overrightarrow{B C}=(0;-2;-2)$ và  $\overrightarrow{B D}=(-1;-1;-1)$.\\
		Khi đó $\vec{u}=\left[\overrightarrow{B C}, \overrightarrow{B D}\right]=(0; 2;-2)$.\\
		Suy ra $\left|\vec{u}\right|=\sqrt{0^2+2^2+(-2)^2}=4$. 	
	}
\end{ex}

\begin{ex}%[2H2V2-5]
	Trong không gian với hệ trục tọa độ $Oxyz$, cho $A(2; 0; 2)$, $B(1;-1;-2)$ và $C(-1; 1; 0)$. Một véctơ $\overrightarrow{n}=(a; b; 2)$ có phương vuông góc với hai véctơ $\overrightarrow{AB}$ và $\overrightarrow{AC}$. Tính giá trị của $a+b$.
	\shortans[0]{$-8$}
	\loigiai{
		Ta có $\overrightarrow{A C}=(-3; 1;-2)$ và $\overrightarrow{A B}=(-1;-1;-4)$.\\
		Vì $\vec{n}$ có phương vuông góc với $\overrightarrow{AB}$ và $\overrightarrow{AC}$ nên $\vec{n}$ cùng phương với vectơ $\left[\overrightarrow{AB},\overrightarrow{AC}\right]=(-6;-10; 4)$.\\
		Suy ra $\overrightarrow{n}=(-3; -5; 2)$
		Vậy $a+b=-3-5=-8$.
	}
\end{ex}

\begin{ex}%[2H2V2-5]
	Hệ trục tọa độ $Oxyz$, cho bốn điểm $A(1;-2; 0)$, $B(2; 0; 3)$, $C(-2; 1; 3)$ và $D(0; 1; 1)$. Tính giá trị của phép tính $\left[\overrightarrow{AB}, \overrightarrow{AC}\right] \cdot \overrightarrow{AD}$.
	\shortans[0]{$-24$}
	\loigiai
	{
		Ta có $\overrightarrow{AB}=(1; 2; 3)$; $\overrightarrow{AC}=(-3; 3; 3)$; $\overrightarrow{A D}=(-1; 3; 1)$.\\
		Khi đó $\left[\overrightarrow{A B}, \overrightarrow{A C}\right]=(-3;-12; 9)$.\\
		Và $\left[\overrightarrow{A B}, \overrightarrow{A C}\right] \cdot \overrightarrow{A D}=(-3) \cdot(-1)+(-12) \cdot 3+9 \cdot 1=-24$.
	}
\end{ex}
\begin{ex}%[2H5H1-2] 
	Trong mặt phẳng tọa độ $O x y z$, mặt phẳng $(P)\colon 2 x-6 y-8 z+1=0$ có một véctơ pháp tuyến $\vec{n}=(1;a;b)$. Khi đó tổng $a+b$ bằng bao nhiêu? 
	\shortans[0]{$-7$}
	\loigiai
	{
		Phương trình tổng quát của mặt phẳng $(P)\colon 2 x-6 y-8 z+1=0$ nên một véctơ pháp tuyến của mặt phẳng $(P)$ có tọa độ là $(2;-6;-8)=2\cdot (1;-3;-4)$.\\
		Suy ra $\vec{n}=(1;-3;-4)$, nên $a+b=-3-4=-7$.	
	}
\end{ex}

\begin{ex}%[2H2V2-5]
	Trong không gian với hệ tọa độ $O x y z$, cho $\overrightarrow{u}=(1; 1; 2), \overrightarrow{v}=(-1; m; m-2)$. Tìm giá trị của $m$ dương sao cho $|[\overrightarrow{u}, \overrightarrow{v}]|=\sqrt{14}$.
	\shortans[0]{$1$}
	\loigiai
	{ Ta có {\allowdisplaybreaks
			\begin{eqnarray*}
				&& [\overrightarrow{u}, \overrightarrow{v}]=(-m-2;-m; m+1)\\ &\Rightarrow& |[\overrightarrow{u}, \overrightarrow{v}]|=\sqrt{(m+2)^2+m^2+(m+1)^2}=\sqrt{3 m^2+6 m+5}.
		\end{eqnarray*}}
		Khi đó $$|[\overrightarrow{u}, \overrightarrow{v}]|=\sqrt{14} \Leftrightarrow 3 m^2+6 m+5=14 \Leftrightarrow 3 m^2+6 m-9=0 \Leftrightarrow \hoac{&m=1 \\&m=-3.}$$
		
	}
\end{ex}

\begin{ex}%[2H2V2-5]
	Trong không gian với hệ tọa độ $O x y z$, cho hai véctơ $\overrightarrow{m}=(4; 3; 1), \overrightarrow{n}=(0; 0; 1)$. Gọi $\overrightarrow{p}=\left(a;b;c\right)$ là véctơ cùng hướng với $[\overrightarrow{m}, \overrightarrow{n}]$ (tích có hướng của hai véctơ $\overrightarrow{m}$ và $\overrightarrow{n}$). Biết $|\overrightarrow{p}|=15$, giá trị của tổng $a+b+c$ bằng bao nhiêu?
	\shortans[0]{$3$}
	\loigiai
	{
		Ta có  $[\overrightarrow{m}; \overrightarrow{n}]=(3;-4; 0)$, suy ra $|[\overrightarrow{m}; \overrightarrow{n}]|=5$.\\
		Do $\overrightarrow{p}$ là véctơ cùng hướng với $[\overrightarrow{m}; \overrightarrow{n}]$ nên $\overrightarrow{p}=k[\overrightarrow{m}; \overrightarrow{n}]$, $k>0$.\\
		Mặt khác $|\overrightarrow{p}|=15 \Leftrightarrow k \cdot|[\overrightarrow{m}, \overrightarrow{n}]| =15 \Leftrightarrow k\cdot 5=15 \Leftrightarrow k=3$.\\
		Suy ra $\overrightarrow{p}=(9;-12; 0)$.	\\
		Vậy $a+b+c=9-12+0=3$.
	}
\end{ex}
\Closesolutionfile{ans}
\indapan{6}{ans/ans2C5B1CD1-KQ}
\begin{dang}{Hai mặt phẳng song song, vuông góc. Khoảng cách một điểm đến mặt phẳng}
	\begin{enumerate}[label=\bf\arabic*.]
		\item \textbf{Điều kiện hai mặt phẳng song song, vuông góc:}\\
		Cho 2 mặt phẳng $\left(\alpha_1\right)\colon A_1 x+B_1 y+C_1 z+D_1=0$ và $\left(\alpha_2\right)\colon A_2 x+B_2 y+C_2 z+D_2=0$ có vectơ pháp tuyến lần lượt là $\overrightarrow{n}_1=\left(A_1; B_1; C_1\right), \overrightarrow{n}_2=\left(A_2; B_2; C_2\right)$. Khi đó:
		\begin{itemize}
			\item $\left(\alpha_1\right) \parallel \left(\alpha_2\right) \Leftrightarrow\heva{&\overrightarrow{n}_1=k \overrightarrow{n}_2 \\ &D_1 \neq k D_2} \quad (k \in \mathbb{R})$.
			\item $\left(\alpha_1\right) \equiv\left(\alpha_2\right) \Leftrightarrow\heva{&\overrightarrow{n}_1=k \overrightarrow{n}_2 \\& D_1=k D_2} \quad (k \in \mathbb{R})$.
			\item $\left(\alpha_1\right)$ cắt $\left(\alpha_2\right) \Leftrightarrow \overrightarrow{n}_1$ và $\overrightarrow{n}_2$ không cùng phương.
			\item $\left(\alpha_1\right) \perp\left(\alpha_2\right) \Leftrightarrow \overrightarrow{n}_1 \cdot \overrightarrow{n}_2=0 \Leftrightarrow A_1 A_2+B_1 B_2+C_1 C_2=0$. 	
		\end{itemize}
		\begin{tikzpicture}[line cap=round,line join=round,>=stealth,x=1.0cm,y=1.0cm,scale=0.6]
			\path
			(1,1) coordinate (A)
			(3,3) coordinate (B)
			(8,3) coordinate (C)
			($(A)+(C)-(B)$) coordinate (D)
			(2,4) coordinate (A')
			(4,6) coordinate (B')
			(9,6) coordinate (C')
			($(A')+(C')-(B')$) coordinate (D')
			(5,5) coordinate (K)
			(6,5) coordinate (M)
			($(K)+(0,1.5)$) coordinate (N)
			(6,2) coordinate (H)
			($(C)!0.5!(D)$) coordinate (H')
			;
			\draw (A)--(B)--(C)--(D)--cycle (A')--(B')--(C')--(D')--cycle ;
			\draw[->] (K)--(N) node[right]{$\vec{n}_1$};
			\draw[->] (H)--($(H)+(0,1.5	 )$) node[right]{$\vec{n}_2$};	
			\draw pic[draw,blue,"$\alpha_1$",angle radius=8mm]{angle=D--A--B};
			\draw pic[draw,blue,"$\alpha_2$",angle radius=8mm]{angle=D'--A'--B'};
			\draw pic[draw,blue,,angle radius=3mm]{right angle=M--K--N};
			\draw pic[draw,blue,,angle radius=3mm]{right angle=M--H--H'};
		\end{tikzpicture}
		\begin{tikzpicture}[line cap=round,line join=round,>=stealth,x=1.0cm,y=1.0cm,scale=0.6]
			\path
			(1,1) coordinate (A)
			(3,3) coordinate (B)
			(8,3) coordinate (C)
			($(A)+(C)-(B)$) coordinate (D)
			($(A)+(-2,3)$) coordinate (E)
			($(B)+(-2,3)$) coordinate (F)
			($(A)!0.5!(C)$) coordinate (G)
			($(G)+(0,1.5)$) coordinate (H)
			($(A)!0.5!(F)$) coordinate (I)
			($(I)+(1.7,1.5)$) coordinate (J)
			($(C)!0.5!(D)$) coordinate (K)
			($(A)!0.5!(B)$) coordinate (L)
			;
			\draw (A)--(B)--(C)--(D)--cycle (A)--(E)--(F)--(B) ;
			\draw[->] (G)--(H) node[right]{$\vec{n}_1$};
			\draw[->] (I)--(J) node[right]{$\vec{n}_2$};	
			\draw pic[draw,blue,"$\alpha_1$",angle radius=8mm]{angle=B--C--D};
			\draw pic[draw,blue,"$\alpha_2$",angle radius=5mm]{angle=A--E--F};
			\draw pic[draw,blue,,angle radius=3mm]{right angle=L--I--J};
			\draw pic[draw,blue,,angle radius=3mm]{right angle=H--G--K};
		\end{tikzpicture}
		\begin{tikzpicture}[line cap=round,line join=round,>=stealth,x=1.0cm,y=1.0cm,scale=0.6]
			\path
			(1,1) coordinate (A)
			(3,3) coordinate (B)
			(8,3) coordinate (C)
			($(A)+(C)-(B)$) coordinate (D)
			($(A)+(0,4)$) coordinate (E)
			($(B)+(0,4.)$) coordinate (F)
			($(A)!0.5!(C)$) coordinate (G)
			($(G)+(0,1.5)$) coordinate (H)
			($(A)!0.5!(F)$) coordinate (I)
			($(I)+(1.7,0)$) coordinate (J)
			($(C)!0.5!(D)$) coordinate (K)
			($(A)!0.5!(B)$) coordinate (L)
			;
			\draw (A)--(B)--(C)--(D)--cycle (A)--(E)--(F)--(B) ;
			\draw[->] (G)--(H) node[right]{$\vec{n}_1$};
			\draw[->] (I)--(J) node[above]{$\vec{n}_2$};	
			\draw pic[draw,blue,"$\alpha_1$",angle radius=8mm]{angle=B--C--D};
			\draw pic[draw,blue,"$\alpha_2$",angle radius=5mm]{angle=A--E--F};
			\draw pic[draw,blue,,angle radius=3mm]{right angle=L--I--J};
			\draw pic[draw,blue,,angle radius=3mm]{right angle=H--G--K};
		\end{tikzpicture}
		\begin{note}
			\textbf{Chú ý:}
			\begin{itemize}
				\item $\overrightarrow{a}$ cùng phương với $\overrightarrow{b} \Leftrightarrow[\overrightarrow{a}, \overrightarrow{b}]=\overrightarrow{0}$.
				\item Nếu $\overrightarrow{n}=[\overrightarrow{a}, \overrightarrow{b}]$ thì vectơ $\overrightarrow{n}$ vuông góc với cả hai vectơ $\overrightarrow{a}$ và $\overrightarrow{b}$.
			\end{itemize}
		\end{note}
		\item \textbf{Khoảng cách từ một điểm đến một mặt phẳng}
		\immini{
			Trong không gian $O x y z$, cho điểm $M_0\left(x_0; y_0; z_0\right)$ và mặt phẳng $(\alpha)\colon A x+B y+C z+D=0$. Khi đó khoảng cách từ điểm $M_0$ đến mặt phẳng $(\alpha)$ được tính: $$d\left(M_0,(\alpha)\right)=\dfrac{\left|A x_0+B y_0+C z_0+D\right|}{\sqrt{A^2+B^2+C^2}}.$$
		}{
			\begin{tikzpicture}[line cap=round,line join=round,>=stealth,x=1.0cm,y=1.0cm,scale=0.6]
				\path
				(1,1) coordinate (A)
				(3,3) coordinate (B)
				(9,3) coordinate (C)
				($(A)+(C)-(B)$) coordinate (D)
				(4,2) coordinate (E)
				(6,2) coordinate (F)
				($(E)+(0,2.5)$) coordinate (G)
				;
				\draw (A)--(B)--(C)--(D)--cycle ;
				\draw[->] (E)--($(E)+(0,2.5)$) node[right]{$M_0$};
				\draw[->] (F)--($(F)+(0,1.5)$) node[right]{$\vec{n}$};	
				\draw pic[draw,blue,"$\alpha$",angle radius=8mm]{angle=D--A--B};
				\draw pic[draw,blue,,angle radius=3mm]{right angle=F--E--G};
				%			\draw pic[draw,blue,,angle radius=3mm]{right angle=H--G--K};
			\end{tikzpicture}
		}
		\begin{note}
			\textbf{Chú ý:}
			\begin{itemize}
				\item Mặt phẳng $(O x y)$ có phương trình: $z=0$.
				\item Mặt phẳng $(O x z)$ có phương trình: $y=0$.
				\item Mặt phẳng $(O y z)$ có phương trình: $x=0$.
			\end{itemize}
		\end{note}
		\item \textbf{Khoảng cách hai mặt phẳng song song}\\
		Khoảng cách giữa mặt phẳng song song là khoảng cách từ một điểm thuộc mặt phẳng này đến mặt phẳng kia (Thực chất là khoảng cách từ một điểm đến mặt phẳng).\\
		Để tính khoảng cách mặt phẳng $\left(\alpha_1\right)$ song song với $\left(\alpha_2\right)$, ta thực hiện như sau:
		\begin{enumerate}
			\item[] \textbf{Bước 1:} Chọn điểm $M \in\left(\alpha_1\right)$.
			\item[] \textbf{Bước 2:} Tính khoảng cách điểm $M$ đến $\left(\alpha_2\right)$.
			\item[] \textbf{Bước 3:} Kết luận: $d\left(\left(\alpha_1\right),\left(\alpha_2\right)\right)=d\left(M,\left(\alpha_2\right)\right)$.
		\end{enumerate}
		\begin{note}
			\textbf{Chú ý:}
			Cho 2 mặt phẳng $\left(\alpha_1\right)\colon A x+B y+C z+D_1=0$ và $\left(\alpha_2\right)\colon A x+B y+C z+D_2=0$ có cùng vectơ pháp tuyến là $\overrightarrow{n}=(A; B; C)$. Khi đó khoảng cách giữa hai mặt phẳng đó là: $$d\left(\left(\alpha_1\right),(\alpha_2)\right)=\dfrac{\left|D_1-D_2\right|}{\sqrt{A^2+B^2+C^2}}.$$ 
		\end{note}
	\end{enumerate}
\end{dang}
\textbf{Khoảng cách hai mặt phẳng song song}
\begin{itemize}
	\item Khoảng cách giữa mặt phẳng song song là khoảng cách từ một điểm thuộc mặt phẳng này đến mặt phẳng kia (Thực chất là khoảng cách từ một điểm đến mặt phẳng).
	\item Để tính khoảng cách mặt phẳng $(\alpha_1)$ song song với $(\alpha_2)$, ta thực hiện như sau:
	\begin{enumEX}[\hspace*{1cm}\bf Bước 1:]{1}
		\item Chọn điểm $M\in (\alpha_1)$
		\item Tính khoảng cách điểm $M$ đến $(\alpha_2)$
		\item Kết luận $\mathrm{d}\left((\alpha_1),(\alpha_2)\right)=\mathrm{d}\left(M,(\alpha_2)\right)$
	\end{enumEX}
	\textbf{Chú ý:} Cho 2 mặt phẳng $(\alpha_1)\colon Ax+By+Cz+D_1=0$ và $(\alpha_2)\colon Ax+By+Cz+D_2=0$ có cùng vectơ pháp tuyến là $\vec{n}=(A;B;C)$.\\
	Khi đó khoảng cách giữa hai mặt phẳng đó là: $\mathrm{d}((\alpha_1),(\alpha))=\dfrac{|D_1-D_2|}{\sqrt{A^2+B^2+C^2}}$.
\end{itemize}
\TN
\Opensolutionfile{ans}[ans/ans2C5B1CD1-D2]
%%==========Câu 27
\begin{ex}%[Câu 2]%[2H5N1-5]
	Khoảng cách từ điểm $M\left(3;2;1\right)$ đến mặt phẳng $(P)\colon Ax+Cz+D=0$, $A.C.D\ne 0$. Chọn khẳng định đúng trong các khẳng định sau:
	\choice
	{\True $\mathrm{d}(M,(P))=\dfrac{\left| 3A+C+D\right|}{\sqrt{A^2+C^2}}$}
	{$\mathrm{d}(M,(P))=\dfrac{\left| A+2B+3C+D\right|}{\sqrt{A^2+B^2+C^2}}$}
	{$\mathrm{d}(M,(P))=\dfrac{\left| 3A+C\right|}{\sqrt{A^2+C^2}}$}
	{$\mathrm{d}(M,(P))=\dfrac{\left| 3A+C+D\right|}{\sqrt{3^2+1^2}}$}
	\loigiai{
		Áp dung công thức $\mathrm{d}(M_0,(\alpha))=\dfrac{\left |Ax_0+By_0+Cz_0+D\right |}{\sqrt{A^2+B^2+C^2}}$.\\
		Ta được: $\mathrm{d}(M,(P))=\dfrac{\left| 3A+C+D\right|}{\sqrt{A^2+C^2}}$.}
\end{ex}

%%==========Câu 28
\begin{ex}%[Câu 3]%[2H5N1-5]
	Trong không gian với hệ tọa độ $Oxyz$, cho mặt phẳng $(P)$ có phương trình: $3x+4y+2z+4=0$ và điểm $A(1;-2;3)$. Tính khoảng cách $\mathrm{d}$ từ $A$ đến $(P)$.
	\choice
	{$\mathrm{d}=\dfrac{5}{9}$}
	{$\mathrm{d}=\dfrac{5}{29}$}
	{\True $\mathrm{d}=\dfrac{5}{\sqrt{29}}$}
	{$\mathrm{d}=\dfrac{\sqrt{5}}{3}$}
	\loigiai{
		Khoảng cách $\mathrm{d}$ từ $A$ đến $(P)$ là $$\mathrm{d}(A,(P))=\dfrac{\left| 3x_A+4y_A+2z_A+4\right|}{\sqrt{3^2+4^2+2^2}}=\dfrac{\left| 3-8+6+4\right|}{\sqrt{29}}=\dfrac{5}{\sqrt{29}}.$$}
\end{ex}

%%==========Câu 29
\begin{ex}%[Câu 4]%[2H5N1-5]
	Trong không gian $Oxyz$, cho mặt phẳng $(P)\colon 2x-2y+z-1=0$. Khoảng cách từ điểm $M\left(-1;2;0\right)$ đến mặt phẳng $(P)$ bằng
	\choice
	{$5$}
	{$2$}
	{\True $\dfrac{5}{3}$}
	{$\dfrac{4}{3}$}
	\loigiai{
		Ta có: $\mathrm{d}\left(M,(P)\right)=\dfrac{\left| 2\cdot\left(-1\right)-2\cdot2+0-1\right|}{\sqrt{2^2+\left(-2\right)^2+1^2}}=\dfrac{5}{3}$.}
\end{ex}

%%==========Câu 30
\begin{ex}%[Câu 5]%[2H5N1-5]
	Trong không gian $Oxyz$, tính khoảng cách từ $M\left(1;2;-3\right)$ đến mặt phẳng $(P)\colon x+2y+2z-10=0$.
	\choice
	{\True $\dfrac{11}{3}$}
	{$3$}
	{$\dfrac{7}{3}$}
	{$\dfrac{4}{3}$}
	\loigiai{
		Ta có: $\mathrm{d}\left(M;(P)\right)=\dfrac{\left| 1+2\cdot 2+2\cdot\left(-3\right)-10\right|}{\sqrt{1^2+2^2+2^2}}=\dfrac{\left| -11\right|}{3}=\dfrac{11}{3}$.}
\end{ex}

%%==========Câu 31
\begin{ex}%[Câu 6]%[2H5H1-5]
	Trong không gian $Oxyz$, cho mặt phẳng $(P)\colon 2x-y+2z-4=0$. Gọi $H$ là hình chiếu vuông góc của điểm $M\left(3;1;-2\right)$ lên mặt phẳng $(P)$. Độ dài đoạn thẳng $MH$ là
	\choice
	{$2$}
	{$\dfrac{1}{3}$}
	{\True $1$}
	{$3$}
	\loigiai{
		Độ dài đoạn thẳng $MH$ là $MH=\mathrm{d}\left(M,(P)\right)=\dfrac{\left| 2\cdot 3-1+2\cdot (-2)-4\right|}{\sqrt{2^2+(-1)^2+2^2}}=1$.}
\end{ex}

%%==========Câu 32
\begin{ex}%[Câu 7]%[2H5H1-5]
	Trong không gian với hệ trục tọa độ $Oxyz$, gọi $H$ là hình chiếu vuông góc của điểm $A(1;-2;3)$ lên mặt phẳng $(P)\colon 2x-y-2z+5=0$. Độ dài đoạn thẳng $AH$ bằng
	\choice
	{$3$}
	{$7$}
	{$4$}
	{$1$}
	\loigiai{
		Độ dài đoạn thẳng $AH$ là $AH=\mathrm{d}\left(A,(P)\right)=\dfrac{\left| 2+2-6+5\right|}{\sqrt{2^2+(-1)^2+(-2)^2}}=1$.}
\end{ex}

%%==========Câu 33
\begin{ex}%[Câu 8]%[2H5H1-5]
	Khoảng cách từ điểm $M(-4;-5;6)$ đến mặt phẳng $(Oxy)$, $(Oyz)$ lần lượt bằng
	\choice
	{\True $6$ và $4$}
	{$6$ và $5$}
	{$5$ và $4$}
	{$4$ và $6$}
	\loigiai{
		Ta có: $\mathrm{d}\left(M,(Oxy)\right)=\left|z_M\right|=6$ và $\mathrm{d}(M,(Oyz))=\left|x_M\right|=4$.}
\end{ex}

%%==========Câu 34
\begin{ex}%[Câu 9]%[2H5H1-5]
	Tính khoảng cách $\mathrm{d}$ từ điểm $B\left(x_0;y_0;z_0\right)$ đến mặt phẳng $(P)\colon y + 1 = 0$ ta được:
	\choice
	{$y_0$}
	{$\left| y_0\right|$}
	{$\dfrac{\left| y_0+1\right|}{\sqrt{2}}$}
	{\True $\left| y_0+1\right|$}
	\loigiai{
		Ta có: $\mathrm{d}\left (M,(P)\right )=\dfrac{\left |y_0+1\right |}{\sqrt{1^2}}=\left| y_0+1\right|$.
	}
\end{ex}

%%==========Câu 35
\begin{ex}%[Câu 10]%[2H5H1-5]
	Khoảng cách từ điểm $C(-2;0;0)$ đến mặt phẳng $(Oxy)$ bằng
	\choice
	{\True $0$}
	{$2$}
	{$1$}
	{$\sqrt{2}$}
	\loigiai{
		Điểm $C$ thuộc mặt phẳng $(Oxy)$ nên $\mathrm{d}\left(C,(Oxy)\right)=0$.}
\end{ex}

%%==========Câu 36
\begin{ex}%[Câu 11]%[2H5H1-5]
	Trong không gian $Oxyz$, khoảng cách giữa hai mặt phẳng $(P)\colon x+2y+2z-10=0$ và $(Q)\colon x+2y+2z-3=0$ bằng
	\choice
	{$\dfrac{4}{3}$}
	{$\dfrac{8}{3}$}
	{\True $\dfrac{7}{3}$}
	{$3$}
	\loigiai{
		Ta có $\dfrac{1}{1}=\dfrac{2}{2}=\dfrac{2}{2}\ne \dfrac{-10}{-3}$ nên $(P)\parallel (Q)$.\\
		Lấy $A\left(2;1;3\right)\in \left(P\right)$. 
		Ta có: $\mathrm{d}\left(\left(P\right),\left(Q\right)\right)=\mathrm{d}\left(A,\left(Q\right)\right)=\dfrac{\left| 2+2\cdot 1+2\cdot3-3\right|}{\sqrt{1^2+2^2+2^2}}=\dfrac{7}{3}$.}
\end{ex}

%%==========Câu 37
\begin{ex}%[Câu 12]%[2H5H1-5]
	Trong không gian $Oxyz$, khoảng cách giữa hai mặt phẳng $(P)\colon x+2y+3z-1=0$ và $(Q)\colon x+2y+3z+6=0$ là
	\choice
	{\True $\dfrac{7}{\sqrt{14}}$}
	{$\dfrac{8}{\sqrt{14}}$}
	{$14$}
	{$\dfrac{5}{\sqrt{14}}$}
	\loigiai{
		Ta có $\dfrac{1}{1}=\dfrac{2}{2}=\dfrac{3}{3}\ne \dfrac{-1}{6}$ nên $(P)\parallel (Q)$.\\
		Khi đó: $\mathrm{d}\left((P);(Q)\right)$ =$\dfrac{\left| D_2-D_1\right|}{\sqrt{A^2+B^2+C^2}}
		=\dfrac{\left| -1-6\right|}{\sqrt{1^2+2^2+3^2}}=\dfrac{7}{\sqrt{14}}$.}
\end{ex}

%%==========Câu 38
\begin{ex}%[Câu 13]%[2H5H1-5]
	Trong không gian $Oxyz$, khoảng cách giữa hai mặt phẳng $(P)\colon x+2y+2z-8=0$ và $(Q)\colon x+2y+2z-4=0$ bằng
	\choice
	{$1$}
	{\True $\dfrac{4}{3}$}
	{$2$}
	{$\dfrac{7}{3}$}
	\loigiai{
		Ta có $\dfrac{1}{1}=\dfrac{2}{2}=\dfrac{2}{2}\ne \dfrac{-8}{-4}$ nên $(P)\parallel (Q)$.\\
		Khi đó: $\mathrm{d}\left((P);(Q)\right)=\dfrac{\left| -8-(-4)\right|}{\sqrt{1^2+2^2+2^2}}=\dfrac{4}{3}$.\\
	}
\end{ex}

%%==========Câu 39
\begin{ex}%[Câu 14]%[2H5H1-4]
	Trong không gian $Oxyz$, mặt phẳng $(P)\colon 2x+y+z-2=0$ vuông góc với mặt phẳng nào dưới đây?
	\choice
	{$2x-y-z-2=0$}
	{\True $x-y-z-2=0$}
	{$x+y+z-2=0$}
	{$2x+y+z-2=0$}
	\loigiai{
		Mặt phẳng $(P)$ có một vectơ pháp tuyến $\overrightarrow{n_P}=\left(2;1;1\right)$.\\
		Mặt phẳng $(Q)\colon x-y-z-2=0$ có một vectơ pháp tuyến $\overrightarrow{n_Q}=\left(1;-1;-1\right)$.\\
		Mà $\overrightarrow{n_P}\cdot\overrightarrow{n_Q}=2-1-1=0\Rightarrow \overrightarrow{n_P}\perp \overrightarrow{n_Q}\Rightarrow (P)\perp (Q)$.\\
		Vậy mặt phẳng $(Q)\colon x-y-z-2=0$ là mặt phẳng cần tìm.}
\end{ex}

%%==========Câu 40
\begin{ex}%[Câu 15]%[2H5H1-4]
	Trong không gian với hệ tọa độ $Oxyz$, cho hai mặt phẳng $(P)\colon 2x+my+3z-5=0$ và $(Q)\colon nx-8y-6z+2=0$, với $m,n\in \mathbb{R}$. Xác định $m,n$ để $(P)$ song song với $(Q)$.
	\choice
	{$m=n=-4$}
	{\True $m=4;n=-4$}
	{$m=- 4;n=4$}
	{$m=n=4$}
	\loigiai{
		Mặt phẳng $(P)$ có véc tơ pháp tuyến $\vec{n_1}=(2;m;3)$.\\
		Mặt phẳng $(Q)$ có véc tơ pháp tuyến $\vec{n_2}=(n;-8;-6)$.\\
		Mặt phẳng $(P)\parallel (Q)\Rightarrow \vec{n_1}=k\cdot \vec{n_2}\, (k\in \mathbb{R})\Leftrightarrow \heva{&2=kn \\&m=- 8k \\&3=- 6k}\Leftrightarrow \heva{&k=-\dfrac{1}{2} \\&m=4 \\&n=- 4.}$}
\end{ex}

%%==========Câu 41
\begin{ex}%[Câu 16]%[2H5H1-4]
	Trong không gian $Oxyz$, cho hai mặt phẳng $(P)\colon x-2y+2z-3=0$ và $(Q)\colon mx+y-2z+1=0$. Với giá trị nào của $m$ thì hai mặt phẳng đó vuông góc với nhau?
	\choice
	{$m=1$}
	{$m=-1$}
	{$m=-6$}
	{\True $m=6$}
	\loigiai{
		Ta có: $(P)\perp (Q)\Leftrightarrow 1\cdot m-2\cdot 1+2\cdot (-2)=0\Leftrightarrow m=6$.}
\end{ex}

%%==========Câu 42
\begin{ex}%[Câu 17]%[2H5V1-4]
	Trong không gian $Oxyz$, cho ba mặt phẳng $(P)\colon x+y+z-1=0$, $(Q)\colon 2x+my+2z+3=0$ và $(R)\colon -x+2y+nz=0$. Tính tổng $m+2n$, biết rằng $(P)\perp (R)$ và $(P)\parallel (Q)$.
	\choice
	{$-6$}
	{$1$}
	{\True $0$}
	{$6$}
	\loigiai{
		$(P)$ có vectơ pháp tuyến $\vec{a}=(1;1;1)$.\\
		$(Q)$ có vectơ pháp tuyến $\vec{b}=(2;m;2)$.\\
		$(R)$ có vectơ pháp tuyến $\vec{c}=(-1;2;n)$.\\
		Ta có: $(P)\perp (R)\Leftrightarrow \vec{a}\cdot \vec{c}=0\Leftrightarrow n=-1$.\\
		$(P)\parallel (Q)\Leftrightarrow \dfrac{2}{1}=\dfrac{m}{1}=\dfrac{2}{1}\Leftrightarrow m=2$.\\
		Vậy $m+2n=2+2\left(-1\right)=0$}
\end{ex}

%%==========Câu 43
\begin{ex}%[Câu 18]%[2H5V1-4]
	Trong không gian $Oxyz$, cho $(P)\colon x+y-2z+5=0$ và $(Q)\colon 4x+(2-m)y+mz-3=0$, $m$ là tham số thực. Tìm tham số $m$ sao cho mặt phẳng $(Q)$ vuông góc với mặt phẳng $(P)$.
	\choice
	{$m=-3$}
	{$m=-2$}
	{$m=3$}
	{\True $m=2$}
	\loigiai{
		Mặt phẳng $(P)$ có véctơ pháp tuyến là $\vec{n_{P}}=(1;1;-2)$.\\
		Mặt phẳng $(Q)$ có véctơ pháp tuyến là $\vec{n_{Q}}=(4;2-m;m)$.\\
		Ta có $(P)\perp (Q)\Leftrightarrow \vec{n_{P}}\perp \vec{n_{Q}}\Leftrightarrow \vec{n_{P}}\cdot \vec{n_{Q}}=0\Leftrightarrow 4\cdot 1+2-m-2m=0\Leftrightarrow m=2$.
	}
\end{ex}

%%==========Câu 44
\begin{ex}%[Câu 19]%[2H5V1-4]
	Trong không gian $Oxyz$ cho hai mặt phẳng $(\alpha)\colon x+2y-z-1=0$ và $(\beta)\colon 2x+4y-mz-2=0$. Tìm $m$ để hai mặt phẳng $(\alpha)$ và $(\beta)$ song song với nhau.
	\choice
	{$m=1$}
	{\True Không tồn tại $m$}
	{$m=-2$}
	{$m=2$}
	\loigiai{
		Ta có vectơ pháp tuyến của $(\alpha)$ là $\overrightarrow{n_1}=(1;2;-1)$, vectơ pháp tuyến của $(\beta)$ là $\overrightarrow{n_2}=(2;4;-m)$.\\
		Hai mặt phẳng $(\alpha)$ và $(\beta)$ song song khi $\dfrac{2}{1}=\dfrac{4}{2}=\dfrac{-m}{-1}\ne \dfrac{-2}{-1}$.\\
		Vậy không có giá trị nào của $m$ thỏa mãn điều kiện trên.}
\end{ex}

%%==========Câu 45
\begin{ex}%[Câu 20]%[2H5V1-4]
	Trong không gian toạ độ $Oxyz$, cho mặt phẳng $(P)\colon x+2y-2z-1=0$, mặt phẳng nào dưới đây song song với $(P)$ và cách $(P)$ một khoảng bằng $3$.
	\choice
	{\True $(Q)\colon x+2y-2z+8=0$}
	{$(Q)\colon x+2y-2z+5=0$}
	{$(Q)\colon x+2y-2z+1=0$}
	{$(Q)\colon x+2y-2z+2=0$}
	\loigiai{
		+ Chọn $A\left(1;0;0\right)\in (P)$.\\
		+ Xét đáp án \textbf{A.}, ta có $\mathrm{d}\left(A;(Q)\right)=\dfrac{\left| 1+8\right|}{\sqrt{1^2+2^2+\left(-2\right)^2}}=3$.
	}
\end{ex}
\Closesolutionfile{ans}
\indapan{10}{ans/ans2C5B1CD1-D2}
\TNTF
\Opensolutionfile{ans}[ans/ans2C5B1CD1-D2-DS]
%%==========Câu 46
\begin{ex}%[Câu 21]%[2H5N1-5]
	Trong không gian toạ độ $Oxyz$, cho điểm $M\left(1;2;0\right)$ và các mặt phẳng $(Oxy)$, $(Oyz)$, $(Oxz)$. Các mệnh đề sau đây đúng hay \textbf{sai}?
	\choiceTF
	{\True $\mathrm{d}\left(M,(Oxz)\right)=2$}
	{\True $\mathrm{d}\left(M,(Oyz)\right)=1$}
	{$\mathrm{d}\left(M,(Oxy)\right)=1$}
	{\True $\mathrm{d}\left(M,(Oxz)\right)>d\left(M,(Oyz)\right)$}
	\loigiai{
		\begin{itemchoice}
			\itemch $\mathrm{d}\left(M,(Oxz)\right)=|2|=2$.	ĐÚNG
			\itemch $\mathrm{d}\left(M,(Oyz)\right)=|1|=1$. ĐÚNG
			\itemch $\mathrm{d}\left(M,(Oxy)\right)=|0|=0$.	SAI
			\itemch $\mathrm{d}\left(M,(Oxz)\right)>d\left(M,(Oyz)\right)$. ĐÚNG
		\end{itemchoice}
	}
\end{ex}
%%==========Câu 48
\begin{ex}%[Câu 23]%[2H5H1-4]
	Trong không gian $Oxyz$, cho hai mặt phẳng $(P)\colon x+2y-2z-6=0$ và $(Q)\colon x+2y-2z+3=0$. Các mệnh đề sau đây đúng hay \textbf{sai}?
	\choiceTF
	{\True Hai mặt phẳng $(P)$ và $(Q)$ song song với nhau}
	{Hai mặt phẳng $(P)$ và $(Q)$ vuông góc với nhau}
	{Khoảng cách giữa hai mặt phẳng $(P)$ và $(Q)$ bằng $2$}
	{\True Khoảng cách giữa hai mặt phẳng $(P)$ và $(Q)$ bằng $3$}
	\loigiai{
		\begin{itemize}
			\item Ta có: $\dfrac{1}{1}=\dfrac{2}{2}=\dfrac{-2}{-2}\ne \dfrac{-6}{3}$ nên $(P)\parallel (Q)$.
			\item $\mathrm{d}\left ((P),(Q)\right )=\dfrac{|-6-3|}{\sqrt{1^2+2^2+(-2)^2}}=3$.
		\end{itemize}
		\begin{itemchoice}
			\itemch Hai mặt phẳng $(P)$ và $(Q)$ song song với nhau. ĐÚNG
			\itemch Hai mặt phẳng $(P)$ và $(Q)$ vuông góc với nhau. SAI
			\itemch Khoảng cách giữa hai mặt phẳng $(P)$ và $(Q)$ bằng $2$. SAI
			\itemch Khoảng cách giữa hai mặt phẳng $(P)$ và $(Q)$ bằng $3$. ĐÚNG
		\end{itemchoice}
	}
\end{ex}

%%==========Câu 47
\begin{ex}%[2H5N1-5]%[2H5H1-5]
	Trong không gian toạ độ $Oxyz$, Biết khoảng cách từ điểm $O$ đến mặt phẳng $(Q)$ bằng 1. Các mệnh đề sau đây đúng hay \textbf{sai}?
	\choiceTF
	{Mặt phẳng $(Q)$ có phương trình là $x + y + z-3 = 0$}
	{\True Mặt phẳng $(Q)$ có phương trình là $2x + y + 2z-3 = 0$}
	{Mặt phẳng $(Q)$ có phương trình là $2x + y- 2z + 6 = 0$}
	{\True Mặt phẳng $(Q)$ có phương trình là $x + 2y + 2z-3= 0$}
	\loigiai{
		\begin{itemchoice}
			\itemch {Ta có $\mathrm{d}(O,(Q))=\dfrac{|-3|}{\sqrt{1^2+1^2+1^2}}=\sqrt{3}\ne 1$. SAI}
			\itemch {Ta có $\mathrm{d}(O,(Q))=\dfrac{|-3|}{\sqrt{2^2+1^2+2^2}}= 1$. ĐÚNG}
			\itemch {Ta có $\mathrm{d}(O,(Q))=\dfrac{|6|}{\sqrt{2^2+1^2+(-2)^2}}=2\ne 1$. SAI}
			\itemch {Ta có $\mathrm{d}(O,(Q))=\dfrac{|-3|}{\sqrt{1^2+2^1+2^2}}=1$. ĐÚNG
			}
		\end{itemchoice}
	}
\end{ex}

%%==========Câu 49
\begin{ex}%[Câu 24]%[2H5H1-4]
	Trong không gian $Oxyz$, cho điểm $N(0;1;0)$ và hai mặt phẳng $(P)\colon 2x-y-2z-9=0$, $(Q)\colon 4x-2y-4z-6=0$. Các mệnh đề sau đây đúng hay \textbf{sai}?
	\choiceTF
	{\True Hai mặt phẳng $(P)$ và $(Q)$ song song với nhau}
	{Khoảng cách từ điểm $N$ đến mặt phẳng $(Q)$ bằng $\dfrac{1}{2}$}
	{\True Khoảng cách giữa hai mặt phẳng $(P)$ và $(Q)$ bằng $2$}
	{Khoảng cách giữa hai mặt phẳng $(P)$ và $(Q)$ bằng $3$}
	\loigiai{
		\begin{itemize}
			\item Ta có $\dfrac{2}{4}=\dfrac{-1}{-2}=\dfrac{-2}{-4}\ne \dfrac{-9}{-6}$ nên $(P)\parallel (Q)$.	
			\item $\mathrm{d}\left(N,\left(Q\right)\right)=\dfrac{\left| -2\cdot 1-6\right|}{\sqrt{4^2+\left(-2\right)^2+\left(-4\right)^2}}=\dfrac{4}{3}$.
			\item $\mathrm{d}\left((P),(Q)\right)=\dfrac{|-9-(-3)|}{\sqrt{2^2+(-1)^2+(-2)^2}}=2$.
		\end{itemize}
		\begin{itemchoice}
			\itemch Hai mặt phẳng $(P)$ và $(Q)$ song song với nhau.	ĐÚNG
			\itemch Khoảng cách điểm đến mặt phẳng $(Q)$ bằng $\dfrac{1}{2}$.	SAI
			\itemch Khoảng cách giữa hai mặt phẳng $(P)$ và $(Q)$ bằng $2$. ĐÚNG
			\itemch Khoảng cách giữa hai mặt phẳng $(P)$ và $(Q)$ bằng $3$. SAI
		\end{itemchoice}
	}
\end{ex}

%%==========Câu 50
\begin{ex}%[Câu 25]%[2H5H1-5]
	Khoảng cách từ điểm $A(2;4;3)$ đến mặt phẳng $(\alpha)\colon 2x+y+2z+1=0$ và $(\beta)\colon x=0$ lần lượt là $\mathrm{d}(A,(\alpha))$, $\mathrm{d}(A,(\beta))$. Các mệnh đề sau đây đúng hay \textbf{sai}?
	\choiceTF
	{$\mathrm{d}\left(A,(\alpha)\right)=3\cdot \mathrm{d}\left(A,(\beta)\right)$}
	{$\mathrm{d}\left(A,(\alpha)\right)>\mathrm{d}\left(A,(\beta)\right)$}
	{$\mathrm{d}\left(A,(\alpha)\right)=\mathrm{d}\left(A,(\beta)\right)$}
	{\True $2\cdot\mathrm{d}\left(A,(\alpha)\right) = \mathrm{d}\left(A,(\beta)\right)$}
	\loigiai{
		Ta có: $\mathrm{d}\left(A,(\alpha)\right)=\dfrac{\left| 2.x_A+y_A+2.z_A+1\right|}{\sqrt{2^2+1^2+2^2}}=1$ và $\mathrm{d}\left(A,(\beta)\right)=\dfrac{\left|x_A\right|}{\sqrt{1^2}}=2$.\\
		Kết luận: $\mathrm{d}\left(A,(\beta)\right)=2\cdot \mathrm{d}\left(A,(\alpha)\right)$.
		\begin{itemchoice}
			\itemch $\mathrm{d}\left(A,(\alpha)\right)=3\cdot \mathrm{d}\left(A,(\beta)\right)$. SAI
			\itemch $\mathrm{d}\left(A,(\alpha)\right)>\mathrm{d}\left(A,(\beta)\right)$. SAI
			\itemch $\mathrm{d}\left(A,(\alpha)\right) =\mathrm{d}\left(A,(\beta)\right)$. SAI
			\itemch $2\cdot \mathrm{d}\left(A,(\alpha)\right)=\mathrm{d}\left(A,(\beta)\right)$. ĐÚNG
		\end{itemchoice}
	}
\end{ex}

%%==========Câu 51
\begin{ex}%Câu 51.%[2H5H1-4]
	Trong không gian $Oxyz$, cho điểm $I(2; 6;-3)$ và các mặt phẳng: $(\alpha)\colon x-2=0$; $(\beta)\colon y-6=0$; $(\gamma): z-3=0$. Các mệnh đề sau đây đúng hay sai?
	\choiceTF
	{\True $(\alpha) \perp(\beta)$}
	{$(\beta) \parallel (Oyz)$}
	{$(\gamma) \parallel Oz$}
	{\True $(\alpha)$ qua $I$}
	\loigiai{
		Ta có:
		\begin{itemize}
			\item $(\alpha): x-2=0$ có véctơ pháp tuyến $\vec{a}=(1 ; 0 ; 0)$.
			\item $(\beta): y-6=0$ có véctơ pháp tuyến $\vec{b}=(0 ; 1 ; 0)$.
			\item $(\gamma): z+3=0$ có véctơ pháp tuyến $\vec{c}=(0 ; 0 ; 1)$.
		\end{itemize}
		\begin{itemchoice}
			\itemch đúng vì ta có $\vec{a} \cdot \vec{b}=1\cdot 0+0\cdot 1+0=0 \Rightarrow(\alpha) \perp(\beta)$.
			\itemch sai vì $(Oyz)$ có véctơ pháp tuyến $\vec{i}=(1 ; 0 ; 0)$ không cùng phương với $\vec{b}=(0 ; 1 ; 0)$ nên $(\beta)$ không song song với mặt phẳng $(Oyz)$. 
			\itemch sai vì trục $Oz$ có vectơ chỉ phương $\vec{k}=(0 ; 0 ; 1)=\vec{c}$ nên $(\gamma) \perp Oz$.
			\itemch đúng vì thay tọa độ điểm $I$ vào $(\alpha)$ ta thấy thỏa thỏa mãn nên $I \in(\alpha)$.	
		\end{itemchoice}
	}
\end{ex}

%%==========Câu 52
\begin{ex}%Câu 52.%[2H5H1-4]
	Trong không gian $Oxyz$, cho hai mặt phẳng $(P)\colon y-9=0$. Xét các mệnh đề sau:
	\begin{multicols}{2}
		\item \hspace*{1cm}(I) $(P) \parallel (Oxz)$.
		\item (II) $(P) \perp Oy$
	\end{multicols}
	\choiceTF
	{Cả (I) và (II) đều sai}
	{(I) đúng, (II) sai}
	{(I) sai, (II) đúng}
	{\True Cả (I) và (II) đều đúng}
	\loigiai{
		Ta có: mặt phẳng $(Oxz)$ có véctơ pháp tuyến $\vec{j}=(0 ; 1 ; 0)$.\\
		Mặt phẳng $(P)$ có véctơ pháp tuyến là $\vec{a}=(0;1;1)=\vec{j}$ nên $(P)\parallel (Oxz)$.\\
		Trục $Oz$ có vectơ chỉ phương là $\vec{j}=(0;1;0)$ nên $(P)\perp Oy$.
		\begin{itemchoice}
			\itemch Cả (I) và (II) đều sai. SAI
			\itemch (I) đúng, (II) sai. SAI
			\itemch (I) sai, (II) đúng. SAI
			\itemch Cả (I) và (II) đều đúng. ĐÚNG
		\end{itemchoice}
	}
\end{ex}

%%==========Câu 53
\begin{ex}%Câu 53.%[2H5H1-4]
	Trong không gian $Oxyz$, Cho ba mặt phẳng $(\alpha)\colon x+y+2z+1=0;(\beta)\colon x+y-z+2=0$; $(\gamma)\colon x-y+5=0$. Các mệnh đề sau đây đúng hay sai?
	\choiceTF
	{$(\alpha) \parallel (\gamma)$}
	{\True $(\alpha) \perp(\beta)$}
	{\True $(\gamma) \perp(\beta)$}
	{\True $(\alpha) \perp(\gamma)$}
	\loigiai{ Ta có:
		\begin{itemize}
			\item Mặt phẳng $(\alpha)$ có véctơ pháp tuyến là $\vec{a}=(1;1;2)$.
			\item Mặt phẳng $(\beta)$ có có véctơ pháp tuyến là $\vec{b}=(1;1;-1)$.
			\item Mặt phẳng $(\gamma)$ có có véctơ pháp tuyến là $\vec{c}=(1;-1;0)$.
			\item $\left [\vec{a},\vec{c}\right ]=(2;2;-2)\ne \vec{0}$ nên $(\alpha)$ và $(\gamma)$ không song song nhau.
			\item $\vec{a} \cdot \vec{b}=0 \Rightarrow(\alpha) \perp(\beta)$.
			\item $\vec{a} \cdot \vec{c}=0 \Rightarrow(\alpha) \perp(\gamma)$.
			\item $\vec{b} \cdot \vec{c}=0 \Rightarrow(\beta) \perp(\gamma)$.
		\end{itemize}
		\begin{itemchoice}
			\itemch $(\alpha)\parallel(\gamma)$. SAI
			\itemch $(\alpha) \perp(\beta)$. ĐÚNG
			\itemch $(\gamma) \perp(\beta)$. ĐÚNG
			\itemch $(\alpha) \perp(\gamma)$. ĐÚNG
		\end{itemchoice}
	}
\end{ex}
\Closesolutionfile{ans}
\indapan{2}{ans/ans2C5B1CD1-D2-DS}
\TNSA
\Opensolutionfile{ans}[ans/ans2C5B1CD1-D2-KQ]
%%==========Câu 54
\begin{ex}%Câu 54.%[2H5H1-5]
	Trong không gian $Oxyz$, cho điểm $M(-1; 2-3)$ và mặt phẳng $(P)\colon 2 x-2 y+z+5=0$. Tính khoảng cách từ điểm $M$ đến mặt phẳng $(P)$ (kết quả viết dưới dạng số thập phân, lấy gần đúng đến hàng phần mười).
	\shortans[0]{$1{,}3$}
	\loigiai{
		Khoảng cách từ điểm $M$ đến mặt phẳng $(P)$ là 
		$$\mathrm{d}\left(M,(P)\right)=\dfrac{\left| 2\cdot (-1)-2\cdot 2+1\cdot (-3)+5\right|}{\sqrt{2^2+(-2)^2+1^2}}=\dfrac{4}{3}.$$
	}
\end{ex}

%%==========Câu 55
\begin{ex}%Câu 55.%[2H5H1-4]
	Trong không gian $Oxyz$, khoảng cách giữa hai mặt phẳng $(P)\colon x+2 y-2 z-16=0$ và $(Q)\colon x+2 y-2 z-1=0$ bằng bao nhiêu?
	\shortans[0]{$5$}
	\loigiai{
		Ta có $\heva{&(P)\parallel (Q) \\ &A(16;0;0)\in (P)}\Rightarrow \mathrm{d}\left((P),(Q)\right)=\mathrm{d}\left(A,(Q)\right)=\dfrac{\left| 16+2\cdot 0-2\cdot 0-1\right|}{\sqrt{1^2+2^2+2^2}}=5$.
	}
\end{ex}
\Closesolutionfile{ans}
\indapan{2}{ans/ans2C5B1CD1-D2-DS}
\TNSA
\Opensolutionfile{ans}[ans/ans2C5B1CD1-D2-KQ]
\begin{ex} %Cau 56D  %[2H5H1-4]
	Trong không gian $Oxyz$, điểm $M \left(0;a;0\right)$ thuộc trục $Oy$ và cách đều hai mặt phẳng: $\left(P\right) \colon x+y-z+1=0$ và $\left(Q\right) \colon x-y+z-5=0$. Khi đó $a$ có giá trị bằng
	\shortans{$-3$}
	\loigiai{
		Ta có $M \in Oy \Rightarrow M\left(0;a;0\right)$.\\
		Theo giả thiết: $\mathrm{d} \left(M,\left(P\right)\right) = \mathrm{d} \left(M,\left(Q\right)\right) \Leftrightarrow \dfrac{\vert a+1 \vert}{\sqrt{3}} = \dfrac{\vert -a-5 \vert}{\sqrt{3}} \Leftrightarrow a = -3$.\\
		Vậy $a = -3$ thì thỏa mãn đề bài.
	}
\end{ex}

\begin{ex} %Cau 57D %[2H5V1-5]
	Trong không gian với hệ trục tọa độ $Oxy$, cho $A \left(1;2;3\right)$, $B \left(3;4;4\right)$. Khi đó giá trị của tham số $m$ bằng bao nhiêu để khoảng cách từ điểm $A$ đến mặt phẳng $\left(P\right)\colon 2x + y + mz -1=0$ bằng độ dài đoạn thẳng $AB$.
	\shortans{$2$}
	\loigiai{
		Ta có $\overrightarrow{AB} = \left(2;2;1\right) \Rightarrow AB = \sqrt{2^2+2^2+1^2} = 3$ \quad(1)\\
		Khoảng cách từ điểm $A$ đến mặt phẳng $\left(P\right)$:\\
		$\mathrm{d} \left(A;\left(P\right)\right) = \dfrac{\vert 2 \cdot 1 + 2 + m \cdot 3 -1 \vert}{\sqrt{2^2+1^2+m^2}} = \dfrac{\vert 3m+3 \vert}{\sqrt{5+m^2}}$ \quad(2).\\
		Để $AB = \mathrm{d} \left(A;\left(P\right)\right) \Rightarrow 3 = \dfrac{\vert 3m+3 \vert}{\sqrt{5+m^2}} \Leftrightarrow 9 \left(5+m^2\right)= 9 \left(m+1\right)^2 \Leftrightarrow m =2$.
	}
\end{ex}

\begin{ex} %Cau 58D %[2H5H1-5]
	Gọi điểm $M \left(0;a;0\right)$ trên trục $Oy$ sao cho khoảng cách từ điểm $M$ đến mặt phẳng $\left(P\right) \colon 2x-y+3z-4=0$ nhỏ nhất. Khi đó giá trị của $a$ là
	\shortans{$-4$}
	\loigiai{
		Khoảng cách từ $M$ đến $\left(P\right)$ nhỏ nhất khi $M$ thuộc $\left(P\right)$. Nên $M$ là giao điểm của trục $Oy$ với mặt phẳng $\left(P\right)$.\\
		Thay $x=0$, $z=0$ vào phương trình ta được $y = -4$. Khi đó $M \left(0;-4;0\right)$\\
		Vậy giá trị của $a = -4$.
	}
\end{ex}

\begin{ex} %Cau 59D %[2H5H1-5]
	Cho điểm $M \left(0;0;m\right)$ thuộc trục $Oz$ sao cho điểm $M$ cách đều điểm $A \left(2;3;4\right)$ và mặt phẳng $\left(P\right) \colon 2x+3y+z-17=0$. Khi đó giá trị của $m$ là
	\shortans{$3$}
	\loigiai{
		Ta có $MA = \sqrt{2^2+3^2+\left(4-m\right)^2}$; $\mathrm{d} \left(M,\left(P\right)\right) = \dfrac{\vert m-17 \vert}{\sqrt{14}}$.\\
		$M$ cách đều điểm $A \left(2;3;4\right)$ và mặt phẳng $\left(P\right) \colon 2x+3y+z-17=0$ khi và chỉ khi\\
		$$\sqrt{2^2+3^2+\left(4-m\right)^2} = \dfrac{\vert m-17 \vert}{\sqrt{14}} \Leftrightarrow 13 \left(m-3\right)^2 = 0 \Leftrightarrow m=3$$
		Vậy $m=3$.
	}
\end{ex}

\begin{ex} %Cau 60D %[2H5V1-5]
	Trong không gian với hệ trục tọa độ $Oxyz$, cho hai điểm $A \left(1;2;3\right)$, $B\left(5;-4;-1\right)$ và mặt phẳng $\left(P\right)$ qua $Ox$ sao cho $\mathrm{d} \left(B;\left(P\right)\right) = 2\mathrm{d} \left(A;\left(P\right)\right)$, $\left(P\right)$ cắt $AB$ tại $I\left(a;b;c\right)$ nằm giữa $AB$. Tính $a+b+c$.
	\shortans{$4$}
	\loigiai{
		Vì $\mathrm{d} \left(B;\left(P\right)\right) = 2\mathrm{d} \left(A;\left(P\right)\right)$ và $\left(P\right)$ cắt đoạn $AB$ tại $I$ nên\\
		$\overrightarrow{BI} = -2 \overrightarrow{AI} \Leftrightarrow \heva{&a-5 = -2\left(a-1\right)\\&b+4 = -2\left(b-2\right)\\&c+1 = -2\left(c-3\right)} \Leftrightarrow \heva{&a=\dfrac{7}{3}\\&b=0\\&c=\dfrac{5}{3}} \Rightarrow a+b+c = 4$.
	}
\end{ex}

\begin{ex} %Cau 61D %[2H5V1-5]
	Trong không gian $Oxyz$, cho mặt phẳng $\left(P\right) \colon 3x+4y-12z+5=0$ và điểm $A \left(2;4;-1\right)$. Trên mặt phẳng $\left(P\right)$ lấy điểm $M$. Gọi $B$ là điểm sao cho $\overrightarrow{AB} = 3\cdot \overrightarrow{AM}$. Tính khoảng cách $\mathrm{d}$ từ $B$ đến mặt phẳng $\left(P\right)$
	\shortans{$6$}
	\loigiai{Ta có: $\overrightarrow{AB} = 3 \cdot \overrightarrow{AM} \Rightarrow BM=2\cdot AM \Rightarrow \dfrac{\mathrm{d} \left(B,\left(P\right)\right)}{\mathrm{d} \left(A,\left(P\right)\right)} = \dfrac{BM}{AM} = 2$
		\immini{
			\begin{eqnarray*}
				&\Rightarrow \mathrm{d} \left(B,\left(P\right)\right) &= 2 \cdot \mathrm{d} \left(A,\left(P\right)\right)\\
				& &= 2 \cdot \dfrac{\vert 3 \cdot 2 + 4 \cdot 4 -12 \cdot \left(-1\right)+5\vert}{\sqrt{3^2+4^2+\left(-12\right)^2}}\\
				& & = 2 \cdot 3 = 6
			\end{eqnarray*}
			Vậy $\mathrm{d} \left(B,\left(P\right)\right) = 6$.}{
			\begin{tikzpicture}[>=stealth,line join=round, line cap=round, scale=0.7]
				\coordinate (A) at (1,3);
				\coordinate (B) at (-1,0);
				\coordinate (C) at (5,0);
				\coordinate (D) at (7,3);
				\coordinate (H) at (1.5,2);
				\coordinate (K) at (4.5,2);
				\coordinate (M) at (4.5,-1.5);
				\coordinate (N) at (1.5,4.5);
				\path[name path=d1] (N)--(M);
				\path[name path=d2] (H)--(K);
				\path[name path=d3] (B)--(C);
				\path[name intersections={of=d1 and d2,by=l}];
				\path[name intersections={of=d1 and d3,by=z}];
				\draw (A)--(B)--(C)--(D)--(A); \draw[dashed] (K)--(4.5,0); \draw (4.5,0)--(M)--(z);
				\draw (l) node[below left]{$M$} circle (1pt)--(N) node[above]{$A$}  circle (1pt)--(H) node[below left]{$H$}  circle (1pt)--(K) node[below right]{$K$} circle (1pt); \draw (M) node[below]{$B$} circle (1pt); \draw [dashed] (l)--(z);
				\begin{scope}
					\clip (A)--(B)--(C);
					\draw (B) circle (1.1);
					\draw (-0.8,0) node[above right]{$P$} ;
				\end{scope}
			\end{tikzpicture}
	}}
\end{ex}

\begin{ex} %Cau 62D %[2H5H1-4]
	Trong không gian $Oxyz$, cho hai mặt phẳng $\left(P\right) \colon 2x+my+2mz-9=0$ và $\left(Q\right) \colon 6x-y-z-10=0$. Tìm $m$ để $\left(P\right) \perp \left(Q\right)$
	\shortans{$4$}
	\loigiai{
		$\left(P\right) \colon 2x+my+2mz-9=0$ có véc-tơ pháp tuyến là $\overrightarrow{a} = \left(2;m;2m\right)$\\
		$\left(Q\right) \colon 6x-y-z-10=0$ có véc-tơ pháp tuyến là $\overrightarrow{b} = \left(6;-1;-1\right)$\\
		Khi đó $\left(P\right) \perp \left(Q\right) \Leftrightarrow \overrightarrow{a} \cdot \overrightarrow{b} =0 \Leftrightarrow 2 \cdot 6 + m \cdot \left(-1\right)+2m \cdot \left(-1\right) =0 \Leftrightarrow m=4$.
	}
\end{ex}

\begin{ex} %Cau 63D %[2H5H1-4]
	Trong không gian $Oxyz$, cho hai mặt phẳng $\left(P\right) \colon 5x+my+z-5=0$ và $\left(Q\right) \colon nx-3y-2z+7=0$. Để $\left(P\right) \parallel \left(Q\right)$ thì giá trị của $m+n$ là (làm tròn đến chữ số thập phân thứ nhất)
	\shortans{$-8{,}5$}
	\loigiai{
		$\left(P\right) \colon 5x+my+z-5=0$ có véc-tơ pháp tuyến là $\overrightarrow{a} = \left(5;m;1\right)$\\
		$\left(Q\right) \colon nx-3y-2z+7=0$ có véc-tơ pháp tuyến là $\overrightarrow{b} = \left(n;-3;-2\right)$\\
		Để $\left(P\right) \parallel \left(Q\right) \Leftrightarrow \left[a;b\right] = \overrightarrow{0} \Leftrightarrow \heva{&-2m+3=0\\&n+10=0\\&-15-mn=0} \Leftrightarrow \heva{&m=\dfrac{3}{2}\\&n=-10}$\\
		Khi đó $m+n = \dfrac{3}{2} + \left(-10\right)= -8,5$.
	}
\end{ex}

\begin{ex} %Cau 64D %[2H5V1-4]
	Trong không gian $Oxyz$, cho hai mặt phẳng $\left(P\right) \colon 2x-my-4z-6+m=0$ và $\left(Q\right) \colon \left(m+3\right)x +y+\left(5m+1\right)z -7=0$. Tìm $m$ để $\left(P\right) \equiv \left(Q\right)$.
	\shortans{$-1$}
	\loigiai{
		$\left(P\right) \colon 2x-my-4z-6+m=0$ có véc-tơ pháp tuyến là $\overrightarrow{a} = \left(2;-m;-4\right)$\\
		$\left(Q\right) \colon \left(m+3\right)x +y+\left(5m+1\right)z -7=0$ có véc-tơ pháp tuyến là $\overrightarrow{b} = \left(m+3;1;5m+1\right)$\\
		Khi đó với $m \neq -3$, $m \neq -\dfrac{1}{5}$ ta có $\left(P\right) \equiv \left(Q\right) \Leftrightarrow \dfrac{2}{m+3} = \dfrac{-m}{1} = \dfrac{-4}{5m+1} \Leftrightarrow m = -1$.
	}
\end{ex}

\begin{ex} %Cau 65D %[2H5H1-3]
	Trong không gian $Oxyz$, cho hai mặt phẳng $\left(P\right) \colon x-2y-z+3=0$ và $\left(Q\right) \colon 2x +y+z -1=0$. Mặt phẳng $\left(R\right)$ đi qua điểm $M\left(1;1;1\right)$ chứa giao tuyến của $\left(P\right)$ và $\left(Q\right)$; phương trình của $\left(R\right) \colon m\left(x-2y-z+3\right) + \left(2x+y+z-1\right)=0$. Khi đó giá trị của $m$ là bao nhiêu?
	\shortans{$-3$}
	\loigiai{
		Vì $\left(R\right) \colon m\left(x-2y-z+3\right) + \left(2x+y+z-1\right)=0$ đi qua điểm $M \left(1;1;1\right)$ nên ta có:\\
		$m\left(1-2 \cdot 1-1+3\right) + \left(2 \cdot 1+1+1-1\right)=0 \Leftrightarrow m = -3$\\
		Vậy $m = -3$.
	}
\end{ex}

\begin{ex} %Cau 66D %[2H5V1-4]
	Trong không gian $Oxyz$, cho $3$ điểm $A\left(1;0;0\right)$,$B\left(0;b;0\right)$,$C\left(0;0;c\right)$ trong đó $b \cdot c \neq 0$ và mặt phẳng $\left(P\right) \colon y-z+1=0$. Giá trị của $\dfrac{2b}{c}$ bằng bao nhiêu để mặt phẳng $\left(ABC\right)$ vuông góc với mặt phẳng $\left(P\right)$.
	\shortans{$2$}
	\loigiai{
		Phương trình mặt phẳng $\left(ABC\right) \colon \dfrac{x}{1}+\dfrac{y}{b}+\dfrac{z}{c}=1$ có véc-tơ pháp tuyến là $\overrightarrow{n} = \left(1;\dfrac{1}{b};\dfrac{1}{c}\right)$.\\
		Phương trình mặt phẳng $\left(P\right) \colon y-z+1=0$ có véc-tơ pháp tuyến là $\overrightarrow{n'} = \left(0;1;-1\right)$.\\
		Do đó $\left(ABC\right) \perp \left(P\right) \Leftrightarrow \overrightarrow{n} \cdot \overrightarrow{n'} = 0 \Leftrightarrow \dfrac{1}{b}-\dfrac{1}{c} = 0 \Leftrightarrow b = c$.\\
		Vậy $\dfrac{2b}{c} = 2$.
	}
\end{ex}

\begin{ex} %Cau 67D %[2H5V1-3]
	Trong không gian $Oxyz$, cho mặt phẳng $\left(\alpha\right) \colon ax-y+2z+b=0$ đi qua giao tuyến của hai mặt phẳng $\left(P\right) \colon x-y-z+1=0$ và $\left(Q\right) \colon x+2y+z-1=0$. Tính $a+4b$
	\shortans{$-16$}
	\loigiai{
		Trên giao tuyến $\Delta$ của hai mặt phẳng $\left(P\right)$, $\left(Q\right)$ ta lấy lần lượt hai điểm $A$, $B$ như sau\\
		Lấy $A \left(x;y;1\right) \in \Delta$, ta có hệ phương trình $\heva{&x-y=0\\&x+2y=0} \Rightarrow x=y=0 \Rightarrow A \left(0;0;1\right)$.\\
		Lấy $B \left(-1;y;z\right) \in \Delta$, ta có hệ phương trình $\heva{&y+z=0\\&2y+z=0} \Rightarrow \heva{&y=2\\&z=2} \Rightarrow B \left(-1;2;-2\right)$.\\
		Vì $\Delta \subset \left(\alpha\right)$ nên $A$, $B \in \left(\alpha\right)$. Do đó ta có: $\heva{&2+b=0\\&-a+b-6=0} \Rightarrow \heva{&a=-8\\&b=-2}$.\\
		Vậy $a+4b = -8 + 2 \cdot \left(-2\right) = -16$.
	}
\end{ex}

\begin{ex} %Cau 68D %[2H5V1-4]
	Gọi $m$, $n$ là hai giá trị thực thỏa mãn giao tuyến của hai mặt phẳng $\left(P_m\right) \colon mx+2y+nz+1=0$ và $\left(Q_m\right) \colon x-my + nz + 2=0$ vuông góc với mặt phẳng $\left(\alpha\right) \colon 4x -y -6z +3=0$. Tính $m+n$
	\shortans{$3$}
	\loigiai{
		\begin{description}
			\item[+] $\left(P_m\right) \colon mx+2y+nz+1=0$ có véc-tơ pháp tuyến $\overrightarrow{n}_1 = \left(m;2;n\right)$
			\item[+] $\left(Q_m\right) \colon x-my+nz+2=0$ có véc-tơ pháp tuyến $\overrightarrow{n}_2 = \left(1;-m;n\right)$
			\item[+] $\left(\alpha \right) \colon 4x-y-6z+3=0$ có véc-tơ pháp tuyến $\overrightarrow{n}_{\alpha} = \left(4;-1;-6\right)$.
			\item[+] Giao tuyến của hai mặt phẳng $\left(P_m\right)$ và $\left(Q_m\right)$ vuông góc với mặt phẳng $\left(\alpha\right)$ nên
			$$\heva{&\left(P_m\right) \perp \left(\alpha\right)\\&\left(Q_m\right) \perp \left(\alpha\right)} \Leftrightarrow \heva{&\overrightarrow{n}_1 \perp \overrightarrow{n}_{\alpha}\\&\overrightarrow{n}_2 \perp \overrightarrow{n}_{\alpha}} \Leftrightarrow \heva{&\overrightarrow{n}_1 \cdot \overrightarrow{n}_{\alpha} = 0\\&\overrightarrow{n}_2 \cdot \overrightarrow{n}_{\alpha} = 0} \Leftrightarrow \heva{&4m-2-6n=0\\&4+m-6n=0} \Leftrightarrow \heva{&m=2\\&n=1}$$
		\end{description}
		Vậy $m+n = 3$.
	}
\end{ex}

\begin{ex} %Cau 69D %[2H5V1-4]
	Trong không gian với hệ tọa độ $Oxyz$ có bao nhiêu mặt phẳng song song với mặt phẳng $\left(Q\right) \colon x+y+z+3=0$, cách điểm $M \left(3;2;1\right)$ một khoảng bằng $3\sqrt{3}$ biết rằng tồn tại một điểm $X\left(a;b;c\right)$ trên mặt phẳng đó, khi đó $a+b+c$ có giá trị bằng
	\shortans{$15$}
	\loigiai{
		Ta có mặt phẳng cần tìm là $\left(P\right) \colon x+y+z+d=0$ với $d \neq 3$.\\
		Mặt phẳng $\left(P\right)$ cách điểm $M \left(3;2;1\right)$ một khoảng bằng $3\sqrt{3}$ nên\\
		$\mathrm{d} \left(M,\left(P\right)\right) = \dfrac{\vert 6+d \vert}{\sqrt{3}} = 3\sqrt{3} \Leftrightarrow \hoac{&d=3 \quad(L)\\&d=-15} \Rightarrow d = -15$.\\
		Suy ra $\left(P\right) \colon x+y+z-15=0$.\\
		Theo giả thiết $X\left(a;b;c\right) \in \left(P\right) \Leftrightarrow a+b+c = 15$.
	}
\end{ex}

\begin{ex} %Cau 70D %[2H5C1-4]
	Biết rằng trong không gian với hệ tọa độ $Oxyz$ có hai mặt phẳng $\left(P\right)$ và $\left(Q\right)$ cùng thỏa mãn các điều kiện sau: đi qua hai điểm $A\left(1;1;1\right)$ và $B\left(0;-2;2\right)$, đồng thời cắt các trục tọa độ $Ox$, $Oy$ tại hai điểm cách đều $O$. Giả sử $\left(P\right) \colon x+b_{1}y+c_{1}z+d_1=0$ và $\left(Q\right) \colon x+b_2 y+c_2 z+d_2=0$. Tính giá trị biểu thức $b_1b_2 + c_1c_2$
	\shortans{$-9$}
	\loigiai{
		$\textbf{Cách 1}$\\
		Xét mặt phẳng $\left(\alpha\right) \colon x+by+cz+d=0$ thỏa mãn các điều kiện: đi qua hai điểm $A \left(1;1;1\right)$ và $B\left(0;-2;2\right)$, đồng thời cắt các trục tọa độ $Ox$, $Oy$ tại hai điểm cách đều $O$.\\
		Vì $\left(\alpha\right)$ đi qua $A \left(1;1;1\right)$ và $B \left(0;-2;2\right)$ nên ta có hệ phương trình:
		$$\heva{&1+b+c+d=0\\&-2b+2c+d=0} \quad(*)$$
		Mặt phẳng $\left(\alpha\right)$ cắt các trục tọa độ $Ox$, $Oy$ lần lượt tại $M \left(-d;0;0\right)$, $N \left(0;\dfrac{-d}{c};0\right)$.\\
		Vì $M$, $N$ cách đều $O$ nên $OM = ON$. Suy ra: $\vert d \vert = \left\vert \dfrac{d}{b} \right\vert$.\\
		Nếu $d=0$ thì chỉ tồn tại duy nhất một mặt phẳng thỏa mãn yêu cầu bài toán (mặt phẳng này sẽ đi qua điểm $O$).\\
		Do đó để tồn tại hai mặt phẳng thỏa mãn yêu cầu bài toán thì $\vert d \vert = \left\vert \dfrac{d}{b} \right\vert \Leftrightarrow b = \pm 1$.
		\begin{description}
			\item[$\bullet$] Với $b=1$, $\left(*\right) \Leftrightarrow \heva{&c+d=-2\\&2c+d=2} \Leftrightarrow \heva{&c=4\\&d=-6}$. Ta được mặt phẳng $\left(P\right) \colon x+y+4z-6=0$.
			\item[$\bullet$] Với $b=-1$, $\left(*\right) \Leftrightarrow \heva{&c+d=0\\&2c+d=-2} \Leftrightarrow \heva{&c=-2\\&d=2}$. Ta có mặt phẳng $\left(P\right) \colon x-y-2z+2=0$.
		\end{description}
		Vậy $b_1b_2 + c_1c_2 = 1 \cdot \left(-1\right) + 4 \cdot \left(-2\right) = -9$.\\
		$\textbf{Cách 2}$\\
		Ta có $\overrightarrow{AB} = \left(-1;-3;1\right)$.\\
		Xét mặt phẳng $\left(\alpha\right) \colon x+by+cz+d=0$ thõa mãn các điều kiện: đi qua hai điểm $A \left(1;1;1\right)$ và $B\left(0;-2;2\right)$, đồng thời cắt các trục tọa độ $Ox$, $Oy$ tại hai điểm cách đều $O$ lần lượt tại $M$, $N$. Vì $M$, $N$ cách đều $O$ nên ta có hai trường hợp sau
		\begin{description}
			\item[TH1] $M \left(a;0;0\right)$, $N \left(0;a;0\right)$ với $a \neq 0$ khi đó $\left(\alpha\right)$ chính là $\left(P\right)$. Ta có $\overrightarrow{MN} = \left(-a;a;0\right)$, chọn $\overrightarrow{u}_1 = \left(-1;1;0\right)$ là một véc-tơ cùng phương với $\overrightarrow{MN}$.\\
			Khi đó $\overrightarrow{n}_P = \left[\overrightarrow{AB},\overrightarrow{u}_1\right] = \left(-1;-1;-4\right)$
			suy ra $\left(P\right) \colon x+y+4z+d_1 = 0$.
			\item[TH2] $M \left(-a;0;0\right)$, $N \left(0;a;0\right)$ với $a \neq 0$ khi đó $\left(\alpha\right)$ chính là $\left(Q\right)$. Ta có $\overrightarrow{MN} = \left(a;a;0\right)$, chọn $\overrightarrow{u}_2 = \left(1;1;0\right)$ là một véc-tơ cùng phương với $\overrightarrow{MN}$.\\
			Khi đó $\overrightarrow{n}_Q = \left[\overrightarrow{AB},\overrightarrow{u}_2\right] = \left(-1;1;2\right)$
			suy ra $\left(Q\right) \colon x-y-2z+d_2 = 0$.
		\end{description}
		Vậy $b_1b_2 + c_1c_2 = 1 \cdot \left(-1\right) + 4 \cdot \left(-2\right) = -9$.
	}
\end{ex}
%-------------HetCD1---------------
\Closesolutionfile{ans}
\indapan{6}{ans/ans2C5B1CD1-D2-KQ}
% % \begin{dang}{LẬP PHƯƠNG TRÌNH TỔNG QUÁT MẶT PHẲNG}
% Để lập phương trình tổng quát của mặt phẳng $\left(\alpha\right)$ thông thường ta có 3 trường hợp cơ bản sau:\\
% $\textbf{Trường hợp 1:}$ Khi bài toán cho biết mặt phẳng $\left(\alpha\right)$ đi qua điểm $M_0 \left(x_0;y_0;z_0\right)$ và có một vectơ pháp tuyến $\overrightarrow{n} = \left(A;B;C\right)$ hoặc có hai vectơ chỉ phương $\overrightarrow{a}$, $\overrightarrow{b}$ (với $\overrightarrow{n} = \left[\overrightarrow{a},\overrightarrow{b}\right]$) thì viết dưới dạng sau:
% $$\left(\alpha\right) \colon A\left(x-x_0\right)+B\left(y-y_0\right)+C\left(z-z_0\right)=0$$
% $\textbf{Trường hợp 2:}$ Khi bài toán cho biết mặt phẳng $\left(\alpha\right)$ có một vectơ pháp tuyến $\overrightarrow{n} = \left(A;B;C\right)$ hoặc có hai vectơ chỉ phương $\overrightarrow{a}$, $\overrightarrow{b}$ (với $\overrightarrow{n} = \left[\overrightarrow{a},\overrightarrow{b}\right]$) và không tìm được điểm $M_0 \left(x_0;y_0;z_0\right) \in \left(\alpha\right)$ thì ta thực hiện các bước sau:
% \begin{itemize}
% 	\item $\textbf{Bước 1:}$ Viết phương trình mặt phẳng $\left(\alpha\right)$ dưới dạng:
% 	$$Ax+By+Cz+D=0$$
% 	\item $\textbf{Bước 2:}$ Sau đó dựa vào giả thiết bài toán để tìm giá trị $D$.
% \end{itemize}
% \begin{note}
% 	Dạng này, giả thiết có liên quan đến khoảng cách và góc liên quan đến mặt phẳng.
% \end{note}
% $\textbf{Trường hợp 3:}$ Khi bài toán cho biết mặt phẳng $\left(\alpha\right)$ đi qua điểm $M_0 \left(x_0;y_0;z_0\right)$ và giả thiết bài toán không cho vectơ pháp tuyến $\overrightarrow{n}$ hoặc không cho hai vectơ chỉ phương $\overrightarrow{a}$, $\overrightarrow{b}$ thì ta thực hiện các bước sau:
% \begin{itemize}
% 	\item $\textbf{Bước 1:}$ Gọi vectơ pháp tuyến của mặt phẳng $\left(\alpha\right)$ là $\overrightarrow{n} = \left(A;B;C\right)$ với $A^2+B^2+C^2 \neq 0$
% 	\item $\textbf{Bước 2:}$ Viết phương trình mặt phẳng $\left(\alpha\right)$ dưới dạng:
% 	$$\left(\alpha\right) \colon A\left(x-x_0\right)+B\left(y-y_0\right)+C\left(z-z_0\right)=0$$
% 	\item $\textbf{Bước 3:}$ Sau đó dựa vào giả thiết bài toán để tìm hai phương trình chứa $3$ ẩn $A$, $B$, $C$.
% 	\begin{note}
% 		\begin{itemize}
% 			\item Dạng này, giả thiết có liên quan đến khoảng cách và góc liên quan đến mặt phẳng
% 			\item Để giải tìm vectơ pháp tuyến của mặt phẳng đơn giản hơn thì gọi vectơ pháp tuyến của mặt phẳng là $\overrightarrow{n} = \left(1;B;C\right)$.
% 		\end{itemize}
% 	\end{note}
% \end{itemize}

\begin{dang}{Viết PTTQ MP khi biết điểm đi qua và một VTPT hoặc hai VTCP}
	\textbf{1. Lập phương trình tổng quát} của mặt phẳng đi qua điểm $M_0 \left(x_0;y_0;z_0\right)$ và biết một vectơ pháp tuyến $\overrightarrow{n} = \left(A;B;C\right)$\\
	Trong KG $Oxyz$, phương trình tổng quát của mặt phẳng đi qua điểm $M_0 \left(x_0;y_0;z_0\right)$ và có vectơ pháp tuyến $\overrightarrow{n}= \left(A;B;C\right)$ là:\\
	$$A\left(x-x_0\right) + B\left(y-y_0\right)+C\left(z-z_0\right) = 0$$
	hay $Ax+By+Cz+D=0$ với $D= -Ax_0-By_0-Cz_0$
	\begin{center}
		\begin{tikzpicture}[>=stealth,line join=round, line cap=round, scale=0.8]
			\coordinate (A) at (1.5,3);
			\coordinate (B) at (-1,0);
			\coordinate (C) at (5,0);
			\coordinate (D) at (7.5,3);
			\coordinate (I) at (1.9,1.5); \coordinate (J) at (1.9,4.5); \coordinate (K) at (1.9,-0.8);
			\coordinate (M) at (3.5,1.5);
			\coordinate (N) at (5.5,2);
			\begin{scope}
				\clip (A)--(B)--(C);
				\draw (B) circle (1);
			\end{scope}
			\draw (A)--(B)--(C)--(D)--(A);
			\draw (-0.8,0) node[above right]{$\alpha$};
			\draw (K)--(1.9,0); \draw [dashed] (1.9,0)--(I) circle (.8pt);  \draw (I)--(J);
			\draw (M) node[below]{$M$} circle (.8pt)--(N) node[below]{$N$} circle (.8pt);
			\draw[->,line width=1] (M)--(N);
			\draw[->,line width=1.4] (1.9,2.5)--(1.9,3.8) node[right]{$\overrightarrow{n}$};
		\end{tikzpicture}
	\end{center}
	$\textbf{Chú ý:}$
	\begin{enumerate}[a.]
		\item Mặt phẳng $\left(\alpha\right)$ có cặp vectơ chỉ phương $\overrightarrow{a}$, $\overrightarrow{b}$ ($\overrightarrow{a}$, $\overrightarrow{b}$ không cùng phương) thì mặt phẳng $\left(\alpha\right)$ có vectơ pháp tuyến $\overrightarrow{n}= \left[\overrightarrow{a},\overrightarrow{b}\right]$.
		\item Mặt phẳng $\left(\alpha\right)$ đi qua ba điểm $A$, $B$, $C$ không thẳng hàng thì có cặp vectơ chỉ phương $\overrightarrow{AB}$, $\overrightarrow{AC}$ nên mặt phẳng $\left(\alpha\right)$ có vectơ pháp tuyến $\overrightarrow{n} = \left[\overrightarrow{AB}, \overrightarrow{AC}\right]$.
		\item Dựa vào tính chất vuông góc, song song giữa mặt phẳng với mặt phẳng, giữa đường thẳng với mặt phẳng trong không gian để tìm vectơ chỉ phương, vectơ pháp tuyến của mặt phẳng cần lập.
		\begin{itemize}
			\item Hai mặt phẳng song song thì có cùng vectơ pháp tuyến.
			\item Hai mặt phẳng vuông góc thì vectơ chỉ phương của mặt phẳng này là vectơ pháp tuyến của mặt phẳng kia.
			\item Đường thẳng song song mặt phẳng thì vectơ chỉ phương của đường thẳng là vectơ chỉ phương của mặt phẳng.
			\item Đường thẳng vuông góc mặt phẳng thì vectơ chỉ phương của đường thẳng là vectơ pháp tuyến của mặt phẳng.
		\end{itemize}
	\end{enumerate}
	\textbf{2. Các trường hợp đặc biệt của mặt phẳng}
	\begin{enumerate}[a.]
		\item Phương trình mặt phẳng theo đoạn chắn\\
		Mặt phẳng $\left(\alpha\right)$ không đi qua gốc tọa độ $O$ và lần lượt cắt trục $Ox$ tại $A \left(a;0;0\right)$, cắt trục $Oy$ tại $B \left(0;b;0\right)$, cắt trục $Oz$ tại $C \left(0;0;c\right)$ có $\textbf{phương trình mặt phẳng theo đoạn chắn}$ là: $\dfrac{x}{a}+\dfrac{y}{b}+\dfrac{z}{c}=1$ với $a \cdot b \cdot c \neq 0$
		\begin{center}
			\begin{tikzpicture}[>=stealth,line join=round, line cap=round, scale=0.7]
				\coordinate (O) at (0,0); 
				\coordinate (A) at (-1.9,-1.9); \coordinate (C) at (0,3); \coordinate (B) at (3,0);
				\path[name path=trucy] (O)--(5,0);
				\path[name path=d1] (A)--(C);
				\path[name path=d2] (B)--(C);
				\path[name path=d3] (A)--(B);
				\path[name path=trucz] (O)--(0,5); \path[name path=trucx] (O)--(-4,-4);
				\path[name intersections={of=d1 and trucz, by=I}];  
				\path[name intersections={of=d1 and trucx, by=J}];
				\path[name intersections={of=d2 and trucy, by=K}]; 
				\draw [fill = blue!5] (I)--(J)--(K)--(I);
				\draw [->] (I)--(0,5) node[right]{$z$};\draw [->] (K)--(4,0) node[above right]{$y$}; \draw [->] (J)--(-3,-3) node[right]{$x$};
				\draw [dashed] (O) node[below]{$O$} circle (.8pt)--(I) node[left]{$C(0;0;c)$};
				\draw [dashed] (O)--(J) node[above left]{$A(a;0;0)$};
				\draw [dashed] (O)--(K) node[below right]{$B(0;b;0)$};
			\end{tikzpicture} 
		\end{center}
		\item Phương trình mặt phẳng đặc biệt\\
		Xét phương trình mặt phẳng $\left(\alpha\right) \colon Ax+By+Cz+D=0$ với $A^2+B^2+C^2 \neq 0$
		\begin{itemize}
			\item Nếu $D = 0$ thì mặt phẳng $\left(\alpha\right)$ đi qua gốc tọa độ $O$ và có dạng $\left(\alpha\right) \colon Ax+By+Cz=0$.
			\begin{center}
				\begin{tikzpicture}[>=stealth,line join=round, line cap=round, scale=0.8]
					%p1
					\coordinate (O) at (0,0); 
					\coordinate (x) at (-2,-2); 
					\coordinate (y) at (4,0); 
					\coordinate (z) at (0,3.5); 
					%p2
					\draw [->] (O) node[below]{$O$}--(x) node[below right]{$x$}; 
					\draw [->] (O)--(y) node[below]{$y$}; 
					\coordinate (A) at (3.3,2.5); \coordinate (B) at (-3,2);
					\draw [fill = blue!5] (A)--(O)--(B)--cycle;
					\path[name path=d1] (A)--(B);
					\path[name path=d2] (O)--(z);
					\path[name intersections={of=d1 and d2, by=M}];
					\draw [->] (M)--(z) node[right]{$z$}; \draw [dashed] (O)--(M);
					\draw (-2.5,2.1) node[below right]{$\left(\alpha\right)$};
					\draw (0.2,-2) node[right]{$Ax+By+Cz=0$};
				\end{tikzpicture}
			\end{center}
			\item Nếu $A=0$, $B \neq 0$, $C \neq 0$ thì mặt phẳng $\left(\alpha\right)$ song song hoặc chứa trục $Ox$.
			\item[+] Mặt phẳng $\left(\alpha\right)$ song song $Ox$ thì có dạng $\left(\alpha\right) \colon By+Cz+D=0$.(Hình 1)
			\item[+] Mặt phẳng $\left(\alpha\right)$ chứa trục $Ox$ thì có dạng $\left(\alpha\right) \colon By+Cz=0$.
			\item Nếu $A \neq 0$, $B = 0$, $C \neq 0$ thì mặt phẳng $\left(\alpha\right)$ song song hoặc chứa trục $Oy$.
			\item[+] Mặt phẳng $\left(\alpha\right)$ song song $Oy$ thì có dạng $\left(\alpha\right) \colon Ax+Cz+D=0$.(Hình 2)
			\item[+] Mặt phẳng $\left(\alpha\right)$ chứa trục $Oy$ thì có dạng $\left(\alpha\right) \colon Ax+Cz=0$.
			\item Nếu $A\neq 0$, $B \neq 0$, $C = 0$ thì mặt phẳng $\left(\alpha\right)$ song song hoặc chứa trục $Oz$.
			\item[+] Mặt phẳng $\left(\alpha\right)$ song song $Oz$ thì có dạng $\left(\alpha\right) \colon Ax+By+D=0$.(Hình 3)
			\item[+] Mặt phẳng $\left(\alpha\right)$ chứa trục $Oz$ thì có dạng $\left(\alpha\right) \colon Ax+By=0$.
			\item Nếu $A=B= 0$, $C \neq 0$ thì mặt phẳng $\left(\alpha\right)$ song song hoặc trùng với $\left(Oxy\right)$.
			\item[+] Mặt phẳng $\left(\alpha\right)$ song song $\left(Oxy\right)$ thì có dạng $\left(\alpha\right) \colon Cz+D=0$.(Hình 4)
			\item[+] Mặt phẳng $\left(\alpha\right)$ chứa $\left(Oxy\right)$ thì có dạng $\left(\alpha\right) \colon z=0$.
			\item Nếu $A=C= 0$, $B \neq 0$ thì mặt phẳng $\left(\alpha\right)$ song song hoặc trùng với $\left(Oxz\right)$.
			\item[+] Mặt phẳng $\left(\alpha\right)$ song song $\left(Oxz\right)$ thì có dạng $\left(\alpha\right) \colon By+D=0$.(Hình 5)
			\item[+] Mặt phẳng $\left(\alpha\right)$ chứa $\left(Oxz\right)$ thì có dạng $\left(\alpha\right) \colon y=0$.
			\item Nếu $B=C= 0$, $A \neq 0$ thì mặt phẳng $\left(\alpha\right)$ song song hoặc trùng với $\left(Oyz\right)$.
			\item[+] Mặt phẳng $\left(\alpha\right)$ song song $\left(Oyz\right)$ thì có dạng $\left(\alpha\right) \colon Ax+D=0$.(Hình 6)
			\item[+] Mặt phẳng $\left(\alpha\right)$ chứa $\left(Oyz\right)$ thì có dạng $\left(\alpha\right) \colon x=0$.
		\end{itemize}
		
		\begin{tabular}{*{2}{c}}
			\begin{tikzpicture}[>=stealth,line join=round, line cap=round, scale=0.9]
				%p1
				\coordinate (O) at (0,0); 
				\coordinate (x) at (-2.6,-2.6); 
				\coordinate (y) at (4,0); 
				\coordinate (z) at (0,3); 
				%p2
				\coordinate (M) at ($(O)!0.5!(x)$);
				\coordinate (N) at ($(O)!0.6!(y)$);
				\coordinate (P) at ($(O)!0.7!(z)$);
				\coordinate (P') at ($(P)-(2,2)$);\coordinate (N') at ($(N)-(2,2)$);
				\coordinate (L) at ($(P')-(P)$);
				%p3
				\path[name path=d1] (N')--(P');
				\path[name path=d2] (O)--(x);
				\path[name intersections={of=d1 and d2, by=S}];
				\draw [fill = blue!5, line width = .7pt] (N)--(P)--(P')--(N')--cycle;
				\draw [->] (N)--(y) node[below right]{$y$}; \draw [->] (P)--(z) node[below right]{$z$};
				\draw [->] (S)--(x) node[below right]{$x$};
				\draw [dashed] (N)--(O) node[below right]{$O$} circle (1pt)--(P);
				\draw [dashed] (S)--(O);\draw [dashed] (P') node[right]{$(\alpha)$}--(L)--(N');
				\draw (3,3) node [below]{$By+Cz+D=0$};
				%p4
				\draw [->,dashed,line width = 1.2pt] (O)--(-0.5,-0.5) node[above]{$\overrightarrow{i}$};
			\end{tikzpicture} & \begin{tikzpicture}[>=stealth,line join=round, line cap=round, scale=0.9]
				%p1
				\coordinate (O) at (0,0); 
				\coordinate (x) at (-2,-2); 
				\coordinate (y) at (4,0); 
				\coordinate (z) at (0,4); 
				%p2
				\coordinate (M) at ($(O)!0.65!(x)$);
				\coordinate (N) at ($(O)!0.6!(y)$);
				\coordinate (P) at ($(O)!0.5!(z)$);
				\coordinate (M') at ($(M)+(3,0)$);\coordinate (P') at ($(P)+(3,0)$);
				\coordinate (L) at ($(P')-(P)$);
				%p3
				\path[name path=d1] (P')--(M');
				\path[name path=d2] (O)--(y);
				\path[name intersections={of=d1 and d2, by=S}];
				\draw [fill = blue!5, line width = .7pt] (P)--(M)--(M')--(P')--cycle;
				\draw [->] (P)--(z) node[below right]{$z$}; \draw [->] (M)--(x) node[below right]{$x$};
				\draw [->] (S)--(y) node[below right]{$y$};
				\draw [dashed] (M)--(O) node[below right]{$O$} circle (1pt)--(P);
				\draw [dashed] (S)--(O); \draw [dashed] (P') node[below left]{$(\alpha)$}--(L)--(M');
				%datten
				\draw (3.3,3.3) node [below]{$Ax+Cz+D=0$};
				%p4
				\draw [->,dashed,line width = 1.2pt] (O)--(0.8,0) node[above]{$\overrightarrow{j}$};
			\end{tikzpicture}\\
			$\textbf{Hình 1}$    & $\textbf{Hình 2}$    \\
			\begin{tikzpicture}[>=stealth,line join=round, line cap=round, scale=0.7]
				%p1
				\coordinate (O) at (0,0); 
				\coordinate (x) at (-2,-2); 
				\coordinate (y) at (4,0); 
				\coordinate (z) at (0,4); 
				%p2
				\coordinate (M) at ($(O)!0.65!(x)$);
				\coordinate (N) at ($(O)!0.6!(y)$);
				\coordinate (P) at ($(O)!0.5!(z)$);
				\coordinate (M') at ($(M)+(0,3)$);\coordinate (N') at ($(N)+(0,3)$);
				\coordinate (L) at ($(M')-(M)$);
				%p3
				\path[name path=d1] (N')--(M');
				\path[name path=d2] (O)--(z);
				\path[name intersections={of=d1 and d2, by=S}];
				\draw [fill = blue!5, line width = .7pt] (N)--(M)--(M')--(N')--cycle;
				\draw [->] (N)--(y) node[below right]{$y$}; \draw [->] (M)--(x) node[below right]{$x$};
				\draw [->] (S)--(z) node[below right]{$z$};
				\draw [dashed] (M)--(O) node[below right]{$O$} circle (1pt)--(N);
				\draw [dashed] (S)--(O); \draw [dashed] (N') node[below]{$(\alpha)$}--(L)--(M');
				%datten
				\draw (2,-1.5) node [below]{$Ax+By+D=0$};
				%p4
				\draw [->,dashed,line width = 1.2pt] (O)--(0,0.8) node[right]{$\overrightarrow{k}$};
			\end{tikzpicture}   & \begin{tikzpicture}[>=stealth,line join=round, line cap=round, scale=0.9]
				%p1
				\coordinate (O) at (0,0); 
				\coordinate (x) at (-3,-3); 
				\coordinate (y) at (2.6,0); 
				\coordinate (z) at (0,2.6); 
				%p2
				\coordinate (A) at ($(O)!0.4!(z)$);
				\coordinate (B) at ($(A)+(2,0)$);
				\coordinate (D) at ($(A)-(2,2)$);
				\coordinate (C) at ($(B)-(2,2)$);
				\coordinate (B') at ($(B)-(A)$);
				\coordinate (C') at ($(C)-(A)$);
				\coordinate (D') at ($(D)-(A)$);
				%p3
				\path[name path=d1] (B)--(C);
				\path[name path=d2] (D)--(C);
				\path[name path=d3] (O)--(y);
				\path[name path=d4] (O)--(x);
				\path[name intersections={of=d2 and d4, by=M}];
				\path[name intersections={of=d1 and d3, by=N}];
				%p5
				\draw [fill = green!15, line width = .7pt] (A)--(D)--(C)--(B)--cycle;
				\draw [->] (A)--(z) node[below right]{$z$}; \draw [->] (N)--(y) node[below right]{$y$};
				\draw [->] (M)--(x) node[below right]{$x$};
				\draw [dashed] (M)--(O) node[below right]{$O$} circle (1pt)--(N); \draw [dashed] (O)--(A);
				\draw [dashed] (D)--(D')--(C')--(C); \draw [dashed] (C')--(B')--(B) node[below left]{$(\alpha)$};
				%datten
				\draw (2,2.4) node [below]{$Cz+D=0$};
			\end{tikzpicture}     \\
			$\textbf{Hình 3}$  & $\textbf{Hình 4}$   \\
			\begin{tikzpicture}[>=stealth,line join=round, line cap=round, scale=0.9]
				%p1
				\coordinate (O) at (0,0); 
				\coordinate (x) at (-2.5,-2.5); 
				\coordinate (y) at (2.6,0); 
				\coordinate (z) at (0,2.6); 
				%p2
				\coordinate (A) at ($(O)!0.45!(y)$);
				\coordinate (B) at ($(A)+(0,2)$);
				\coordinate (D) at ($(A)-(2,2)$);
				\coordinate (C) at ($(B)-(2,2)$);
				\coordinate (B') at ($(B)-(A)$);
				\coordinate (C') at ($(C)-(A)$);
				\coordinate (D') at ($(D)-(A)$);
				%p3
				\path[name path=d1] (B)--(C);
				\path[name path=d2] (D)--(C);
				\path[name path=d3] (O)--(z);
				\path[name path=d4] (O)--(x);
				\path[name intersections={of=d2 and d4, by=M}];
				\path[name intersections={of=d1 and d3, by=N}];
				%p5
				\draw [fill = green!15, line width = .7pt] (A)--(D)--(C)--(B)--cycle;
				\draw [->] (A)--(y) node[below right]{$y$}; \draw [->] (M)--(x) node[below right]{$x$};
				\draw [->] (N)--(z) node[below right]{$z$};
				\draw [dashed] (M)--(O) node[below right]{$O$} circle (1pt)--(N); \draw [dashed] (O)--(A);
				\draw [dashed] (D)--(D')--(C')--(C); \draw [dashed] (C')--(B')--(B);
				%datten
				\draw (2,-1.7) node [below]{$By+D=0$};
				\draw (B) node[below]{$(\alpha)$};
			\end{tikzpicture}   & \begin{tikzpicture}[>=stealth,line join=round, line cap=round, scale=0.9]
				%p1
				\coordinate (O) at (0,0); 
				\coordinate (x) at (-2,-2); 
				\coordinate (y) at (2.6,0); 
				\coordinate (z) at (0,2.6); 
				%p2
				\coordinate (A) at ($(O)!0.6!(x)$);
				\coordinate (B) at ($(A)+(2.2,0)$);
				\coordinate (D) at ($(A)+(0,2)$);
				\coordinate (C) at ($(B)+(0,2)$);
				\coordinate (B') at ($(B)-(A)$);
				\coordinate (C') at ($(C)-(A)$);
				\coordinate (D') at ($(D)-(A)$);
				%p3
				\path[name path=d1] (B)--(C);
				\path[name path=d2] (D)--(C);
				\path[name path=d3] (O)--(y);
				\path[name path=d4] (O)--(z);
				\path[name intersections={of=d1 and d3, by=M}];
				\path[name intersections={of=d2 and d4, by=N}];
				%p5
				\draw [fill = green!15, line width = .7pt] (A)--(D)--(C)--(B)--cycle;
				\draw [->] (A)--(x) node[right]{$x$}; \draw [->] (M)--(y) node[below]{$y$};
				\draw [->] (N)--(z) node[right]{$z$};
				\draw [dashed] (M)--(O) node[above left]{$O$} circle (1pt)--(N); \draw [dashed] (O)--(A);
				\draw [dashed] (D)--(D')--(C')--(C); \draw [dashed] (C')--(B')--(B);
				%datten
				\draw (2,-1.7) node [below]{$Ax+D=0$};
				\draw (B) node[above left]{$(\alpha)$};
			\end{tikzpicture}   \\
			$\textbf{Hình 5}$    &$\textbf{Hình 6}$ \\
		\end{tabular}
		
	\end{enumerate}
	$\textbf{Nhận xét:}$
	\begin{itemize}
		\item Để nhớ các phương trình mặt phẳng đặc biệt thì lấy phương trình $\left(\alpha\right) \colon Ax+By+Cz+D=0$  làm chuẩn.
		\item[+] Mặt phẳng $\left(\alpha\right)$ chứa gốc tọa độ $O\left(0;0;0\right)$ thì $D=0$.
		\item[+] Mặt phẳng $\left(\alpha\right)$ chứa trục tương ứng nào (trục $Ox$, $Oy$, $Oz$) thì ẩn đó không có (không chứa $Ax$, $By$, $Cz$) và $D=0$.
		\item[+] Mặt phẳng $\left(\alpha\right)$ song song với trục tương ứng nào (trục $Ox$, $Oy$, $Oz$) thì ẩn đó không có (không chứa $Ax$, $By$, $Cz$) và $D \neq 0$.
		\item Nếu không nhớ các phương trình mặt phẳng đặc biệt thì nhớ vec-tơ chỉ phương của các trục $Ox$, $Oy$, $Oz$ và vectơ pháp tuyến các mặt phẳng tọa độ $\left(Oxy\right)$, $\left(Oxz\right)$, $\left(Oyz\right)$ để chuyển bài toán lập phương trình mặt phẳng khi biết một điểm và một vectơ pháp tuyến.
		\item[+] Trục $Ox$ có vectơ chỉ phương là $\overrightarrow{i} = \left(1;0;0\right)$.
		\item[+] Trục $Oy$ có vectơ chỉ phương là $\overrightarrow{j} = \left(0;1;0\right)$.
		\item[+] Trục $Ox$ có vectơ chỉ phương là $\overrightarrow{k} = \left(0;0;1\right)$.
		\item[+] Mặt phẳng $\left(Oxy\right)$ có vectơ pháp tuyến là $\overrightarrow{k} = \left(0;0;1\right)$.
		\item[+] Mặt phẳng $\left(Oxz\right)$ có vectơ pháp tuyến là $\overrightarrow{j} = \left(0;1;0\right)$.
		\item[+] Mặt phẳng $\left(Oyz\right)$ có vectơ pháp tuyến là $\overrightarrow{i} = \left(1;0;0\right)$.
	\end{itemize}
\end{dang}

\Opensolutionfile{ans}[ans/CD3_17-23]
\TN

\begin{ex}%[2H5H1-3]
	Trong không gian với hệ tọa độ $O x y z$, phương trình nào dưới đây là phương trình mặt phẳng đi qua điểm $M(1 ; 2 ;-3)$ và có một vectơ pháp tuyến $\vec{n}=(1 ;-2 ; 3)$.
	\choice
	{\True $x-2 y+3 z+12=0$}
	{$x-2 y-3 z-6=0$}
	{$x-2 y+3 z-12=0$}
	{$x-2 y-3 z+6=0$}
	\loigiai{
		Phương trình mặt phẳng đi qua điểm $M(1 ; 2 ;-3)$ và có một vectơ pháp tuyến $\vec{n}=(1 ;-2 ; 3)$ là $$1(x-1)-2(y-2)+3(z+3)=0 \Leftrightarrow x-2y+3 z+12=0.$$
	}
\end{ex}
\begin{ex}%[2H5H1-3] 
	Trong không gian với hệ trục tọa độ $Oxyz$, phương trình mặt phẳng đi qua điểm $A(1 ; 2 ;-3)$ có vectơ pháp tuyến $\vec{n}=(2 ;-1 ; 3)$ là
	\choice
	{\True $2 x-y+3 z+9=0$}
	{$2 x-y+3 z-4=0$}
	{$x-2 y-4=0$}
	{$2 x-y+3 z+4=0$}
	\loigiai{Phương trình mặt phẳng đi qua điểm $A(1 ; 2 ;-3)$ có vectơ pháp tuyến $\vec{n}=(2 ;-1 ; 3)$ là
		\allowdisplaybreaks
		\begin{eqnarray*}
			&&2(x-1)-1 (y-2)+3 (z+3)=0\\
			&\Leftrightarrow& 2 x-2-y+2+3 z+9=0\\
			&\Leftrightarrow& 2 x-y+3 z+9=0.
		\end{eqnarray*}
	}
\end{ex}

\begin{ex}%[2H5H1-3] 
	Trong không gian $O x y z$, phương trình của mặt phẳng đi qua điểm $A(3 ; 0 ;-1)$ và có vectơ pháp tuyến $\vec{n}=(4 ;-2 ;-3)$ là
	\choice
	{$4x-2 y+3z-9=0$}
	{\True $4x-2y-3z-15=0$}
	{$3x-z-15=0$}
	{$4x-2y-3z+15=0$}
	\loigiai{
		Mặt phẳng đi qua điểm $A(3 ; 0 ;-1)$ và có vectơ  pháp tuyến $\vec{n}=(4 ;-2 ;-3)$ có phương trình:
		$$4(x-3)-2(y-0)-3(z+1)=0 \Leftrightarrow 4 x-2 y-3 z-15=0.$$
	}
\end{ex}

\begin{ex}%[2H5H1-3]
	Trong KG $Oxyz$, phương trình mặt phẳng qua $A(-1 ; 1 ;-2)$ và có vectơ  pháp tuyến $\vec{n}=(1 ;-2 ;-2)$ là
	\choice
	{\True $x-2 y-2 z-1=0$}
	{$-x+y-2z-1=0$}
	{$x-2y-2z+7=0$}
	{$-x+y-2z+1=0$}
	\loigiai{
		Mặt phẳng $(P)$ đi qua $A(-1 ; 1 ;-2)$ và có vectơ  pháp tuyến $\vec{n}=(1 ;-2 ;-2)$ nên có phương trình
		$$1(x+1)-2(y-1)-2(z+2)=0 \Leftrightarrow x-2y-2z-1=0.$$
	}
\end{ex}

\begin{ex}%[2H5N1-1] 
	Trong KG $Oxyz$, phương trình mặt phẳng $(Oyz)$ là
	\choice
	{$z=0$}
	{\True $x=0$}
	{$x+y+z=0$}
	{$y=0$}
	\loigiai{
		Mặt phẳng $(Oyz)$ nhận $\vec{i}=(1;0;0)$ làm vectơ  pháp tuyến và đi qua gốc tọa độ $O(0;0;0)$ có phương trình là $x=0$.
	}
\end{ex}

\begin{ex}%[2H5N1-1]
	Trong KG $Oxyz$, phương trình của mặt phẳng $(Oxy)$ là
	\choice
	{\True $z=0$}
	{$x=0$}
	{$y=0$}
	{$x+y=0$}
	\loigiai{
		Phương trình của mặt phẳng $(Oxy)$ là $z=0$.
	}
\end{ex}

\begin{ex}%[2H5N1-1] 
	Trong không gian với hệ toạ độ $Oxyz$, phương trình nào dưới đây là phương trình của mặt phẳng $(Oyz)$?
	\choice
	{$y=0$}
	{\True $x=0$}
	{$y-z=0$}
	{$z=0$}
	\loigiai{
		Mặt phẳng $(Oyz)$ đi qua điểm $O(0 ; 0 ; 0)$ và có vectơ  pháp tuyến là $\vec{i}=(1 ; 0 ; 0)$ nên ta có phương trình mặt phẳng $(O y z)$ là  $1(x-0)+0(y-0)+0(z-0)=0 \Leftrightarrow x=0$.
	}
\end{ex}
\begin{ex}%[2H5N1-1] 
	Trong không gian với hệ tọa độ $O x y z$, phương trình nào sau đây là phương trình của mặt phẳng $O z x$ ?
	\choice
	{$x=0$}
	{$y-1=0$}
	{\True $y=0$}
	{$z=0$}
	\loigiai{
		Ta có mặt phẳng $(Oxz)$ đi qua điểm $O(0 ; 0 ; 0)$ và vuông góc với trục $O y$ nên có VTPT $\vec{n}=(0 ; 1 ; 0)$.\\
		Do đó phương trình của mặt phẳng $(Oxz)$ là $y=0$.
	}
\end{ex}

\begin{ex}%[2H5H1-3] 
	Trong không gian với hệ tọa độ $O x y z$, phương trình mặt phẳng $(P)$ qua $M(0 ;-2 ; 1)$ và có cặp vectơ  chỉ phương $\vec{a}=(1 ; 1 ;-2),$ $ \vec{b}=(1 ; 0 ; 3)$ là
	\choice
	{\True $3 x-5 y-z-6=0$}
	{$3 x-5 y-z+6=0$}
	{$3 x+5 y-z+6=0$}
	{$3 x-5 y+z-6=0$}
	\loigiai{
		Ta có $\vec{n}=[\vec{a}, \vec{b}]=(3 ;-5 ;-1)$.\\
		Mặt phẳng $(P)$ đi qua $M(0 ;-2 ; 1)$ và có vectơ  pháp tuyến $\vec{n}=(3 ;-5 ;-1)$ nên có phương trình $$3(x-0)-5(y+2)-(z-1)=0 \Leftrightarrow 3 x-5 y-z-6=0.$$
	}
\end{ex}
\begin{ex}%[2H5H1-3] 
	Trong không gian với hệ tọa độ $O x y z$, cặp vectơ  $\vec{a}=(2 ; 1 ;-2), $ $\vec{b}=(1 ; 0 ; 2)$ có giá song song với mặt phẳng $(P)$. Phương trình mặt phẳng $(P)$ qua $C(1 ; 1 ; 3)$ là
	\choice
	{$2 x+6 y-z-7=0$}
	{$2 x-6 y-z+5=0$}
	{$2 x+6 y+z+5=0$}
	{\True $2 x-6 y-z+7=0$}
	\loigiai{
		Ta có $\vec{n}=[\vec{a}, \vec{b}]=(2 ;-6 ;-1)$.\\
		Mặt phẳng $(P)$ đi qua $C(1 ; 1 ; 3)$ và có vectơ  pháp tuyến $\vec{n}=(2 ;-6 ;-1)$ nên có phương trình $$2(x-1)-6(y-1)-1(z-3)=0 \Leftrightarrow 2 x-6 y-z+7=0.$$
	}
\end{ex}

\begin{ex}%[2H5H1-3] 
	Trong không gian $O x y z$, cho ba điểm $A(3 ; 0 ; 0),$ $ B(0 ; 1 ; 0)$ và $C(0 ; 0 ;-2)$. Mặt phẳng $(A B C)$ có phương trình là
	\choice
	{$\dfrac{x}{3}+\dfrac{y}{-1}+\dfrac{z}{2}=1$}
	{\True $\dfrac{x}{3}+\dfrac{y}{1}+\dfrac{z}{-2}=1$}
	{$\dfrac{x}{3}+\dfrac{y}{1}+\dfrac{z}{2}=1$}
	{$\dfrac{x}{-3}+\dfrac{y}{1}+\dfrac{z}{2}=1$}
	\loigiai{Theo công thức phương trình mặt chắn, ta có
		$(A B C)\colon  \dfrac{x}{3}+\dfrac{y}{1}+\dfrac{z}{-2}=1$.}
\end{ex}
\begin{ex}%[2H5H1-3] 
	Trong không gian với hệ tọa độ $O x y z$, cho ba điểm $A(0 ; 1 ; 2), $ $B(2 ;-2 ; 1),$ $ C(-2 ; 1 ; 0)$. Khi đó, phương trình mặt phẳng $(A B C)$ là $a x+y-z+d=0$. Hãy xác định $a$ và $d$.
	\choice
	{\True $a=1,$ $ d=1$}
	{$a=6, $ $d=-6$}
	{$a=-1, $ $d=-6$}
	{$a=-6, $ $d=6$}
	\loigiai{
		Ta có $\overrightarrow{A B}=(2 ;-3 ;-1) ; \overrightarrow{A C}=(-2 ; 0 ;-2)$.
		
		$$[\overrightarrow{A B}, \overrightarrow{A C}]=\left(\left|\begin{array}{cc}-3 & -1 \\ 0 & -2\end{array}\right| ;\left|\begin{array}{cc}-1 & 2 \\ -2 & -2\end{array}\right| ;\left|\begin{array}{cc}2 & -3 \\ -2 & 0\end{array}\right|\right)=(6 ; 6 ;-6).$$
		Chọn $\vec{n}=\dfrac{1}{6}[\overrightarrow{A B} ; \overrightarrow{A C}]=(1 ; 1 ;-1)$ là một VTPT của mp$(A B C)$. Ta có 
		$$(A B C)\colon x+y-1-z+2=0 \Leftrightarrow x+y-z+1=0.$$ Vậy $a=1,$ $ d=1$.
	}
\end{ex}

\begin{ex}%[2H5H1-3] 
	Trong không gian $O x y z$, cho điểm $A(0 ;-3 ; 2)$ và mặt phẳng $(P)\colon 2 x-y+3 z+5=0$. Mặt phẳng đi qua $A$ và song song với $(P)$ có phương trình là
	\choice
	{$2 x-y+3 z+9=0$}
	{$2 x+y+3 z-3=0$}
	{$2 x+y+3 z+3=0$}
	{\True $2 x-y+3 z-9=0$}
	\loigiai{
		Gọi $(Q)$ là mặt phẳng cần tìm.\\
		Theo bài $(Q) \parallel (P) \Rightarrow(Q)\colon 2 x-y+3 z+m=0\,(m \neq 5)$.\\
		Mà $(Q)$ qua $A \Leftrightarrow 2\cdot 0-(-3)+3\cdot 2+m=0 \Leftrightarrow m=-9$.\\
		Vậy $(Q)\colon 2 x-y+3 z-9=0$.
	}
\end{ex}
\begin{ex}%[2H5H1-3] 
	Trong không gian $O x y z$, cho hai điểm $A(0 ; 0 ; 1)$ và $B(1 ; 2 ; 3)$. Mặt phẳng đi qua $A$ và vuông góc với $A B$ có phương trình là
	\choice
	{$x+2 y+2 z-11=0$}
	{\True $x+2 y+2 z-2=0$}
	{$x+2 y+4 z-4=0$}
	{$x+2 y+4 z-17=0$}
	\loigiai{
		Ta có $\overrightarrow{A B}=(1 ; 2 ; 2)$.\\
		Mặt phẳng đi qua $A$ và vuông góc với $A B$ nên nhận $\overrightarrow{A B}=(1 ; 2 ; 2)$ làm vectơ pháp tuyến có phương trình $$1(x-0)+2(y-0)+2(z-1)=0 \Leftrightarrow x+2 y+2 z-2=0.$$
	}
\end{ex}

\begin{ex}%[2H5H1-3] 
	Trong mặt phẳng $O x y z$, cho hai điểm $A(1 ; 0 ; 0)$ và $B(3 ; 2 ; 1)$. Mặt phẳng đi qua $A$ và vuông góc với $A B$ có phương trình là
	\choice
	{\True $2 x+2 y+z-2=0$}
	{$4 x+2 y+z-17=0$}
	{$4 x+2 y+z-4=0$}
	{$2 x+2 y+z-11=0$}
	\loigiai{
		Mặt phẳng đi qua $A$ và vuông góc với $A B$ nên nhận $\overrightarrow{A B}=(2 ; 2 ; 1)$ làm vectơ pháp tuyến.\\
		Vậy phương trình mặt phẳng cần tìm là $$2(x-1)+2 y+z=0 \Leftrightarrow 2 x+2 y+z-2=0.$$
	}
\end{ex}
\begin{ex}%[2H5H1-3] 
	Trong KG $Oxyz$, cho hai điểm $A(0 ; 1 ; 1)$  và $B(1 ; 2 ; 3)$. Viết phương trình của mặt phẳng $(P)$ đi qua $A$ và vuông góc với đường thẳng $A B$.
	\choice
	{\True $x+y+2 z-3=0$}
	{$x+y+2z-6=0$}
	{$x+3y+4z-7=0$}
	{$x+3y+4z-26=0$}
	\loigiai{
		Mặt phẳng $(P)$ đi qua $A(0 ; 1 ; 1)$ và nhận vectơ $\overrightarrow{A B}=(1 ; 1 ; 2)$ là vectơ pháp tuyến
		$$(P)\colon 1(x-0)+1(y-1)+2(z-1)=0 \Leftrightarrow x+y+2 z-3=0.$$
	}
\end{ex}

\begin{ex}%[2H5H1-3] 
	Trong không gian $O x y z$, cho ba điểm $A(-1 ; 1 ; 1),$ $ B(2 ; 1 ; 0),$ $ C(1 ;-1 ; 2)$. Mặt phẳng đi qua $A$ và vuông góc với đường thẳng $B C$ có phương trình là
	\choice
	{$3 x+2 z+1=0$}
	{\True $x+2 y-2 z+1=0$}
	{$x+2 y-2 z-1=0$}
	{$3 x+2 z-1=0$}
	\loigiai{
		Ta có $\overrightarrow{B C}=(-1 ;-2 ; 2)$ là một vectơ  pháp tuyến của mặt phẳng $(P)$ cần tìm.\\
		$\vec{n}=-\overrightarrow{B C}=(1 ; 2 ;-2)$ cũng là một vectơ  pháp tuyến của mặt phẳng $(P)$.\\
		Vậy phương trình mặt phẳng $(P)$ là $x+2 y-2 z+1=0$.
	}
\end{ex}
\begin{ex}%[2H5H1-3] 
	Trong không gian với hệ tọa độ $O x y z$, cho các điểm $A(0 ; 1 ; 2), $ $B(2 ;-2 ; 1)$, $C(-2 ; 0 ; 1)$. Phương trình mặt phẳng đi qua $A$ và vuông góc với $B C$ là
	\choice
	{$y+2 z-5=0$}
	{$2 x-y-1=0$}
	{\True $2 x-y+1=0$}
	{$-y+2 z-5=0$}
	\loigiai{
		Ta có vectơ  pháp tuyến của mặt phẳng $(P)$ là $\overrightarrow{B C}=(-4 ; 2 ; 0)$.\\
		Phương trình mặt phẳng $(P)$ là
		$$-4(x-0)+2(y-1)+0(z-2)=0 \Leftrightarrow-4 x+2 y-2=0 \Leftrightarrow 2 x-y+1=0.$$}
\end{ex}

\begin{ex}%[2H5H1-3] 
	Trong không gian $O x y z$, mặt phẳng $(P)$ đi qua hai điểm $A(0 ; 1 ; 0)$, $B(2 ; 3 ; 1)$ và vuông góc với mặt phẳng $(Q)\colon x+2 y-z=0$ có phương trình là
	\choice
	{$4x-3y+2z+3=0$}
	{\True $4 x-3 y-2 z+3=0$}
	{$2 x+y-3 z-1=0$}
	{$4 x+y-2 z-1=0$}
	\loigiai{
		Ta có $\overrightarrow{A B}=(2 ; 2 ; 1)$, vectơ  pháp tuyến mặt phẳng $(Q)\colon \overrightarrow{n}_Q=(1 ; 2 ;-1)$.\\
		Theo đề bài ta có vectơ  pháp tuyến mặt phẳng $(P)\colon \overrightarrow{n}_P=\left[\overrightarrow{n}_Q , \overrightarrow{A B}\right]=(4 ;-3 ;-2)$.\\
		Phương trình mặt phẳng $(P)$ có dạng $4 x-3 y-2 z+C=0$.\\
		Mặt phẳng $(P)$ đi qua $A(0 ; 1 ; 0)$ nên $-3+C=0 \Leftrightarrow C=3$.\\
		Vậy phương trình mặt phẳng $(P)$ là $4 x-3 y-2 z+3=0$.
	}
\end{ex}
\begin{ex}%[2H5H1-3] 
	Cho hai mặt phẳng $(\alpha)\colon  3 x-2 y+2 z+7=0,$ $(\beta)\colon 5 x-4 y+3 z+1=0$. Phương trình mặt phẳng đi qua gốc tọa độ $O$ đồng thời vuông góc với cả $(\alpha)$ và $(\beta)$ là
	\choice
	{$2 x-y-2 z=0$}
	{$2 x-y+2 z=0$}
	{\True $2 x+y-2 z=0$}
	{$2 x+y-2 z+1=0$}
	\loigiai{
		vectơ pháp tuyến của hai mặt phẳng lần lượt là $\overrightarrow{n}_\alpha=(3 ;-2 ; 2), \overrightarrow{n}_\beta=(5 ;-4 ; 3)$.\\
		Suy ra $\left[\overrightarrow{n}_\alpha ; \overrightarrow{n}_\beta\right]=(2 ; 1 ;-2)$ là vectơ pháp tuyến của mặt phẳng cần tìm.\\
		Phương trình mặt phẳng đi qua gốc tọa độ $O, $ có vectơ pháp tuyến $\vec{n}=(2 ; 1 ;-2)$ là $2 x+y-2 z=0$.
	}
\end{ex}

\begin{ex}%[2H5H1-3] 
	Trong không gian với hệ tọa độ $O x y z$, cho điểm $A(2 ; 4 ; 1) ;$ $ B(-1 ; 1 ; 3)$ và mặt phẳng $(P)\colon x-3 y+2 z-5=0$. Một mặt phẳng $(Q)$ đi qua hai điểm $A, B$ và vuông góc với mặt phẳng $(P)$ có dạng $a x+b y+c z-11=0$. Khẳng định nào sau đây là đúng?
	\choice
	{\True $a+b+c=5$}
	{$a+b+c=15$}
	{$a+b+c=-5$}
	{$a+b+c=-15$}
	\loigiai{Vì $(Q)$ vuông góc với $(P)$ nên $(Q)$ nhận vectơ pháp tuyến $\vec{n}=(1 ;-3 ; 2)$ của $(P)$ làm vectơ chỉ phương.\\
		Mặt khác $(Q)$ đi qua $A$ và $B$ nên $(Q)$ nhận $\overrightarrow{A B}=(-3 ;-3 ; 2)$ làm vectơ chỉ phương.\\
		$(Q)$ nhận $\overrightarrow{n}_Q=[\vec{n}, \overrightarrow{A B}]=(0 ; 8 ; 12)$ làm vectơ pháp tuyến.\\
		Vậy phương trình mặt phẳng $(Q)\colon  0(x+1)+8(y-1)+12(z-3)=0\Leftrightarrow 2 y+3 z-11=0$.\\
		Vậy $a+b+c=5$.}
\end{ex}

\begin{ex}%[2H5V1-3]
	Trong không gian $O x y z$, cho hai mặt phẳng $(P)\colon  x-3 y+2 z-1=0$, 
	$(Q)\colon  x-z+2=0$. Mặt phẳng $(\alpha)$ vuông góc với cả $(P)$ và $(Q)$ đồng thời cắt trục $O x$ tại điểm có hoành độ bằng 3 . Phương trình của $(\alpha)$ là
	\choice
	{\True $x+y+z-3=0$}
	{$x+y+z+3=0$}
	{$-2 x+z+6=0$}
	{$-2 x+z-6=0$}
	\loigiai{
		$(P)$ có vectơ pháp tuyến $\overrightarrow{n}_P=(1 ;-3 ; 2),(Q)$ có vectơ pháp tuyến $\overrightarrow{n}_Q=(1 ; 0 ;-1)$.\\
		Vì mặt phẳng $(\alpha)$ vuông góc với cả $(P)$ và $(Q)$ nên $(\alpha)$ có một vectơ pháp tuyến là $\left[\overrightarrow{n}_P, \overrightarrow{n}_Q\right]=(3 ; 3 ; 3)=3(1 ; 1 ; 1)$.\\
		Vì mặt phẳng $(\alpha)$ cắt trục $O x$ tại điểm có hoành độ bằng $3$ nên $(\alpha)$ đi qua điểm $M(3 ; 0 ; 0)$.\\
		Vậy $(\alpha)$ đi qua điểm $M(3 ; 0 ; 0)$ và có vectơ pháp tuyến $\overrightarrow{n}_\alpha=(1 ; 1 ; 1)$ nên $(\alpha)$ có phương trình: $x+y+z-3=0$.
	}
\end{ex}

\begin{ex}%[2H5H1-3] 
	Trong không gian với hệ trục tọa độ $O x y z$, cho mặt phẳng $(P)\colon  a x+b y+c z-9=0$ chứa hai điểm $A(3 ; 2 ; 1),$ $ B(-3 ; 5 ; 2)$ và vuông góc với mặt phẳng $(Q)\colon  3 x+y+z+4=0$. Tính tổng $S=a+b+c$?
	\choice
	{$S=-12$}
	{$S=2$}
	{\True $S=-4$}
	{$S=-2$}
	\loigiai{
		$\overrightarrow{A B}=(-6 ; 3 ; 1)$.\\
		$\overrightarrow{n}_{(Q)}=(3 ; 1 ; 1)$ là vectơ pháp tuyến  của $(Q)$.\\
		Mặt phẳng $(P)$ chứa hai điểm $A(3 ; 2 ; 1),$ $ B(-3 ; 5 ; 2)$ và vuông góc với mặt phẳng $(Q)$.\\
		Suy ra $ \overrightarrow{n}_{(P)}=\left[\overrightarrow{A B}, \overrightarrow{n}_{(Q)}\right]=(2 ; 9 ;-15)$ là vectơ pháp tuyến  của $(P)$.\\
		$A(3 ; 2 ; 1) \in(P)\Rightarrow(P)\colon 2 x+9 y-15 z-9=0$ hoặc $(P)\colon -2 x-9 y+15 z+9=0$.\\
		Mặt khác $(P)\colon a x+b y+c z-9=0 \Rightarrow a=2 ; $ $b=9 ;$ $ c=-15$.\\
		Vậy $S=a+b+c=2+9+(-15)=-4$.
	}
\end{ex}
\begin{ex}%[2H5H1-3] 
	Trong không gian $O x y z$, phương trình của mặt phẳng $(P)$ đi qua điểm $B(2 ; 1 ;-3)$, đồng thời vuông góc với hai mặt phẳng $(Q)\colon x+y+3 z=0,$ $(R)\colon 2 x-y+z=0$ là
	\choice
	{$4 x+5 y-3 z+22=0$}
	{$4 x-5 y-3 z-12=0$}
	{$2 x+y-3 z-14=0$}
	{\True $4 x+5 y-3 z-22=0$}
	\loigiai{
		Mặt phẳng $(Q)\colon x+y+3 z=0,$ $(R)\colon 2 x-y+z=0$ có các vectơ pháp tuyến lần lượt là $\overrightarrow{n}_1=(1 ; 1 ; 3)$ và $\overrightarrow{n}_2=(2 ;-1 ; 1)$.\\
		Vì $(P)$ vuông góc với hai mặt phẳng $(Q),$ $(R)$ nên $(P)$ có vectơ pháp tuyến là $\vec{n}=\left[\overrightarrow{n}_1, \overrightarrow{n}_2\right]=(4 ; 5 ;-3)$.\\
		Ta lại có $(P)$ đi qua điểm $B(2 ; 1 ;-3)$ nên $$(P)\colon 4(x-2)+5(y-1)-3(z+3)=0\Leftrightarrow 4 x+5 y-3 z-22=0.$$
	}
\end{ex}
\Closesolutionfile{ans}
\indapan{10}{ans/CD3_17-23}

\Opensolutionfile{ans}[ans/CD3_17-23DS]
\TNTF

\begin{ex}%[2H5H1-3]
	Trong KG $Oxyz$, cho điểm $A(1; -2; 3)$ và hai vectơ  $\overrightarrow{v}=(-1; 2; 3)$, $\overrightarrow{u}=(-2; 0; 1)$.
	\choiceTF
	{\True $\overrightarrow{v}=-\overrightarrow{i}+2\overrightarrow{j}+3\overrightarrow{k}$}
	{$\overrightarrow{u}\perp \overrightarrow{v}$}
	{\True Phương trình mặt phẳng đi qua điểm $A(1; -2; 3)$ và vuông góc với giá của vectơ  $\overrightarrow{v}=(-1; 2; 3)$ là $x-2y-3z+4=0$}
	{Phương trình mặt phẳng đi qua điểm $A(1; -2; 3)$ và vuông góc với giá của vectơ $\overrightarrow{u}=(-2; 0; 1)$ là $2x-y+1=0$}
	\loigiai{
		\begin{itemchoice}
			\itemch Đúng. \\Ta có $\overrightarrow{v}=(-1; 2; 3) \Leftrightarrow \overrightarrow{v}=-\overrightarrow{i}+2\overrightarrow{j}+3\overrightarrow{k}$.
			\itemch Sai.\\ Ta có $\overrightarrow{u}\cdot \overrightarrow{v} = 2 + 0 + 3 = 5\neq 0 \Rightarrow \overrightarrow{u}\not \perp \overrightarrow{v}$.
			\itemch Đúng.\\ Mặt phẳng đi qua điểm $A(1; -2; 3)$ và vuông góc với giá của vectơ  $\overrightarrow{v}=(-1; 2; 3)$ có phương trình
			\[-1(x-1)+2(y+2)+3(z-3)=0\Leftrightarrow x-2y-3z+4=0.\]
			\itemch Sai.\\ Mặt phẳng đi qua điểm $A(1; -2; 3)$ và vuông góc với giá của vectơ $\overrightarrow{u}=(-2; 0; 1)$ có phương trình
			\[ -2(x-1)+0(y+2)+1(z-3)=0\Leftrightarrow 2x -z +1=0.\]
		\end{itemchoice}
	}
\end{ex}
\begin{ex}%[2H5H1-3]
	Trong KG $Oxyz$, cho ba điểm $A(1;1;4)$, $B(2;7;9)$, $C(0;9;13)$.
	\choiceTF
	{\True $\overrightarrow{AB}=\overrightarrow{i}+6\overrightarrow{j}+5\overrightarrow{k}$}
	{$\overrightarrow{AB}\perp \overrightarrow{AC}$}
	{\True Phương trình mặt phẳng đi qua ba điểm $A$, $B$, $C$ là $x-y+z-4=0$}
	{Phương trình mặt phẳng đi qua ba điểm $A$, $B$, $C$ là $2x+y-z-2=0$}
	\loigiai{
		\begin{itemchoice}
			\itemch $\overrightarrow{AB}=(1; 6; 5) \Rightarrow \overrightarrow{AB}=\overrightarrow{i}+6\overrightarrow{j}+5\overrightarrow{k}$.
			\itemch Ta có $\overrightarrow{AC}=(-1; 8; 9)$, khi đó $\overrightarrow{AB}\cdot \overrightarrow{AC}= -1 + 48 + 45 = 92 \neq 0 \Rightarrow \overrightarrow{AB}\not\perp \overrightarrow{AC}$.
			\itemch Ta có $[\overrightarrow{AB}, \overrightarrow{AC}]= (14;-14;14)=14(1;-1;1)$.\\
			Mặt phẳng $(ABC)$ đi qua điểm $A$ và có vectơ pháp tuyến $\overrightarrow{n}=(1;-1;1)$ là $x - y + z - 4 = 0$.
			\itemch Phương trình mặt phẳng đi qua ba điểm $A,B,C$ là $x - y + z - 4 = 0$.			
		\end{itemchoice}
	}
\end{ex}
\begin{ex}%[2H5H1-3]
	Trong KG $Oxyz$, cho điểm $M(2; -1; 4)$ và mặt phẳng $(P)\colon 3x - 2y+z+1=0$.
	\choiceTF
	{\True Mặt phẳng $(P)$ có một vec-tơ pháp tuyến là $\overrightarrow{n}=(-3; 2; -1)$}
	{Mặt phẳng $(P)$ đi qua điểm $B(-1; 1; 2)$}
	{\True Phương trình của mặt phẳng $(Q)$ đi qua điểm $M$ và song song với mặt phẳng $(P)$ là $3x-2y+z-12=0$}
	{Phương trình của mặt phẳng $(R)$ đi qua điểm $O$, $M$ và vuông góc với mặt phẳng $(P)$ là $7x+my+nz=0$. Khi đó $m+n=8$}
	\loigiai{
		\begin{itemchoice}
			\itemch Mặt phẳng $(P)$ có vec-tơ pháp tuyến là $\overrightarrow{n}=(3;-2;1)=-(-3;2;-1)$.
			\itemch Ta có $3\cdot (-1)-2\cdot(1)+2+1=-2\neq 0$. Suy ra mặt phẳng $(P)$ không đi qua điểm $B$.
			\itemch Mặt phẳng $(Q)$ song song với mặt phẳng $(P)$ có dạng $3x-2y+z+d=0$.\\
			Vì $M\in (Q) \Rightarrow d = -12$. Vậy phương trình mặt phẳng $(Q)\colon 3x-2y+z-12=0$.
			\itemch Ta có mặt phẳng $(R)$ đi qua điểm $O$, $M$ và vuông góc với mặt phẳng $(P)$ cho nên mặt phẳng $(R)$ có vec-tơ pháp tuyến là $\overrightarrow{n}_R=\left[\overrightarrow{OM}, \overrightarrow{n}_P\right] =(7;10;-1)$.\\
			Mặt phẳng $(R)$ đi qua điểm $O$ và có vec-tơ pháp tuyến $\overrightarrow{n}_R=(7;10;-1)$ có phương trình $7x+10y-z=0$. Khi đó $m+n=9$.
	\end{itemchoice}}
\end{ex}
\begin{ex}%[2H5H1-3]
	Trong KG $Oxyz$, cho hai điểm $A(1;0;0)$, $B(4;1;2)$.
	\choiceTF
	{$\overrightarrow{AB}=(5;1;2)$}
	{\True Nếu $I$ là trung điểm đoạn thẳng $AB$ thì $I\left(\dfrac{5}{2};\dfrac{1}{2};1\right)$}
	{\True Mặt phẳng $(\alpha) $ đi qua $A$ và vuông góc với $AB$ có phương trình là $3x+y+2z-3=0$}
	{Mặt phẳng trung trực của đoạn thẳng $AB$ có phương trình là $3x+y+2z-12=0$}
	\loigiai{
		\begin{itemchoice}			
			\itemch Ta có $\overrightarrow{AB}=(3;1;2)$.
			\itemch Nếu $I$ là trung điểm đoạn thẳng $AB$ thì $I\left(\dfrac{5}{2};\dfrac{1}{2};1\right)$.
			\itemch Mặt phẳng $(\alpha) $ vuông góc với $AB$ cho nên mặt phẳng $(\alpha)$ có vec-tơ pháp tuyến $\overrightarrow{n}=\overrightarrow{AB}=(3;1;2)$.\\
			Mặt phẳng $(\alpha)$ đi qua $A$ và có vec-tơ pháp tuyến $\overrightarrow{n}=(3;1;2)$ có phương trình là $3x+y+2z-3=0$.
			\itemch Mặt phẳng trung trực của đoạn thẳng $AB$ là mặt phẳng đi qua điểm $I$ và vuông góc $AB$ nên có phương trình là
			\[\begin{array}{l} {3\left(x-\dfrac{5}{2} \right)+y-\dfrac{1}{2}+2\left(z-1\right)=0} \\ {\Leftrightarrow 3x+y+2z-10=0} \end{array}\]
	\end{itemchoice}}
\end{ex}
\begin{ex}%[2H5H1-3]
	Trong không gian với hệ trục tọa độ $Oxyz$, cho điểm $M(1;2;3)$. Gọi $A$, $B$, $C$ lần lượt là hình chiếu vuông góc của $M$ trên các trục $Ox$, $Oy$, $Oz$.
	\choiceTF
	{\True Điểm $A$ có tọa độ là $A\left(1;0;0\right)$}
	{Điểm $B$ có tọa độ là $B\left(1;2;0\right)$}
	{$\overrightarrow{BC}=(-1;-2;3)$}
	{Phương trình mặt phẳng $(ABC)$ là $\dfrac{x}{1}+\dfrac{y}{2}+\dfrac{z}{3}=0$}
	\loigiai{
		\begin{itemchoice}
			\itemch Điểm $A$ có tọa độ là $A\left(1;0;0\right)$.
			\itemch Điểm $B$ có tọa độ là $B\left(0;2;0\right)$.
			\itemch Ta có $C(0;0;3)$. Suy ra $\overrightarrow{BC}=(0;-2;3)$.
			\itemch Mặt phẳng $(ABC)$ là $\dfrac{x}{1}+\dfrac{y}{2}+\dfrac{z}{3}=1$.
		\end{itemchoice}
	}
\end{ex}
\begin{ex}%[2H5H2-3]%Câu 28
	Trong KG $Oxyz$, cho hai điểm $A(1 ; 0 ; 0), B(4 ; 1 ; 2)$. Mệnh đề nào sau đây đúng hay sai?
	\choiceTF
	{\True $\overrightarrow{AB}=(3 ; 1 ; 2)$}
	{\True Mặt phẳng đi qua $\mathrm{A}$ và vuông góc với $AB$ có phương trình là $3x+y+2z-3=0$}
	{\True Nếu $I$ là trung điểm đoạn thẳng $AB$ thì $I\left(\dfrac{5}{2} ; \dfrac{1}{2} ; 1\right)$}
	{\True Mặt phẳng trung trực đoạn thẳng $AB$ có phương trình là $3x+y+2z-12=0$}
	\loigiai{
		\begin{itemchoice}
			\itemch Đúng.\\Do $A(1 ; 0 ; 0), B(4 ; 1 ; 2)$ nên ta có $\overrightarrow{A B}=(3 ; 1 ; 2)$.
			\itemch Đúng.\\Gọi $(Q)$ là mặt phẳng đi qua $A(1 ; 0 ; 0)$ và vuông góc với $A B$ suy ra mặt phẳng $(Q)$ nhận vectơ $\overrightarrow{A B}=(3 ; 1 ; 2)$ làm vectơ pháp tuyến.\\ 
			Vậy phương trình mặt phẳng $(Q)$ cần tìm có dạng: $3(x-1)+y+2 z=0 \Leftrightarrow 3 x+y+2 z-3=0$.
			\itemch Đúng.\\$I$ là trung điểm đoạn thẳng $A B$ nên $I\left(\dfrac{5}{2} ; \dfrac{1}{2} ; 1\right)$.
			\itemch Đúng. \\Mặt phẳng trung trực đoạn thẳng $AB$ là mặt phẳng đi qua $\mathrm{I}$ và vuông góc $AB$ nên có phương trình là
			$$3\left(x-\dfrac{5}{2}\right)+y-\dfrac{1}{2}+2(z-2)=0 \\
			\Leftrightarrow 3x+y+2z-12=0..$$
		\end{itemchoice}
	}
\end{ex}
\begin{ex} %[2H5H2-3]%Cau 29
	Trong không gian với hệ trục tọa độ $Oxyz$, cho điểm $M(1 ; 2 ; 3)$. Gọi $A, B, C$ lần lượt là hình chiếu vuông góc của $M$ trên các trục $Ox, Oy, Oz$. Mệnh đề nào sau đây đúng hay sai?
	\choiceTF
	{\True Điểm $A$ có tọa độ là $A(1 ; 0 ; 0)$}
	{Điểm $B$ có tọa độ là $B(1 ; 2 ; 0)$}
	{Phương trình mặt phẳng $(A B C)$ là $\dfrac{x}{1}+\dfrac{y}{2}+\dfrac{z}{3}=0$}
	{\True Phương trình mặt phẳng $(A B C)$ là $\dfrac{x}{1}+\dfrac{y}{2}+\dfrac{z}{3}=1$}
	\loigiai{
		\begin{itemchoice}
			\itemch Đúng.\\Do $A$ là hình chiếu vuông góc của $M$ trên trục $Ox \Rightarrow A(1 ; 0 ; 0)$.
			\itemch Sai.\\Do $B$ là hình chiếu vuông góc của $M$ trên trục $Oy \Rightarrow B(0 ; 2 ; 0)$.
			\itemch Sai.\\$C$ là hình chiếu vuông góc của $M$ trên trục $Oz \Rightarrow C(0 ; 0 ; 3)$.
			\itemch Đúng.\\Vì 3 điểm $A(1;0;0);B(0;2;0);C(0;0;3)$ thuộc $Ox;Oy;Oz$ nên phương trình mặt phẳng $(A B C)$ là $\dfrac{x}{1}+\dfrac{y}{2}+\dfrac{z}{3}=1$.
		\end{itemchoice}
	}
\end{ex}	

\begin{ex}%[2H5H2-3]%Cau 30
	Trong KG $Oxyz$, cho điểm $A(3 ; 5 ; 2)$.  Gọi $A_{1}, A_{2}, A_{3}$ lần lượt là hình chiếu của điểm $A$ lên các mặt phẳng $(Oxy),(Oyz),(Oxz)$. Mệnh đề nào sau đây đúng hay sai?
	\choiceTF
	{\True Điểm $A_{1}$ có tọa độ là $(3 ; 5 ; 0)$}
	{\True Phương trình mặt phẳng đi qua các điểm $A_{1}, A_{2}, A_{3}$ là $10 x+6 y+15z-60=0$}
	{Phương trình mặt phẳng đi qua các điểm $A_{1}, A_{2}, A_{3}$ là $10 x+6 y+15 z-90=0$}
	{Phương trình mặt phẳng đi qua các điểm $A_{1}, A_{2}, A_{3}$ là $\dfrac{x}{3}+\dfrac{y}{5}+\dfrac{z}{2}=1$}
	\loigiai{
		\begin{enumerate}
			\itemch Đúng.\\Vì $A_1$ là hình chiếu của $A$ trên mặt phẳng $(Oxy)$ nên $A_1$ có tọa độ là $(3;5;0)$.
			\itemch Đúng.\\Mặt phẳng đi qua $A_1(3;5;0);A_2(0;5;2),A_3(3;0;2)$ có vectơ pháp tuyến được tính từ tích có hướng của hai vectơ
			$$\overrightarrow{A_1A_2}=(-3;0;2)$$
			$$\overrightarrow{A_1A_3}=(0;-5;2).$$
			Tích có hướng của hai vectơ này là
			$$\overrightarrow{n}=\left[ \overrightarrow{A_1A_2}, \overrightarrow{A_1A_3}\right]=(10;6;15).$$
			Phương trình mặt phẳng là
			$10(x-3)+6(y-5)+15(z-10)=0$\\$\Rightarrow 10x+6y+15-60=0$.
			\itemch Sai.\\Vì phương trình mặt phẳng là $10(x-3)+6(y-5)+15(z-10)=0$\\ $\Rightarrow 10x+6y+15-60=0$.
			\itemch Sai.\\Phương trình mặt phẳng đi qua các điểm $A_{1}, A_{2}, A_{3}$ là $\dfrac{x}{3}+\dfrac{y}{5}+\dfrac{z}{2}=1$\\
			Để kiểm tra phương trình này, ta nhân cả hai vế phương trình $\dfrac{x}{3}+\dfrac{y}{5}+\dfrac{z}{2}=1$ với 30 ta được
			$$10x+6y+15z-30=0 \neq 10x+6y+15-60=0.$$
		\end{enumerate}
	}
\end{ex}
\begin{ex} %[2H5H2-3]%Cau 31
	Trong KG $Oxyz$, cho hai điểm $A(4 ; 0 ; 1)$ và $B(-2 ; 2 ; 3)$. Mệnh đề nào sau đây đúng hay sai?
	\choiceTF
	{\True $\overrightarrow{AB}=(-6 ; 2 ; 2)$}
	{\True Nếu $I$ là trung điểm đoạn thẳng $AB$ thì $I(1 ; 1 ; 2)$}
	{ Mặt phẳng trung trực của đoạn thẳng $AB$ có phương trình là $x+y+2z-6=0$}
	{\True Mặt phẳng trung trực của đoạn thẳng $AB$ có phương trình là $3x-y-z=0$}
	
	\loigiai{
		\begin{enumerate}
			\itemch Đúng.\\Vì $\overrightarrow{AB}=(-6;2;2)$.
			\itemch Đúng.\\Vì tọa độ trung điểm $I=\left(\dfrac{4-2}{2};\dfrac{0+2}{2};\dfrac{1+3}{2}\right)=\left(1;1;2\right)$.
			\itemch Sai.\\
			Mặt phẳng trung trực của đoạn thẳng $AB$ là mặt phẳng đi qua trung điểm $I$ và vuông góc với $\overrightarrow{AB}$.\\
			Phương trình mặt phẳng có dạng
			$$a(x-1)+b(y-1)+c(z-2)=0.$$
			Với $\overrightarrow{n}=(a;b;c)$ là các vectơ pháp tuyến của mặt phẳng trung trực.\\
			Vì mặt phẳng trung trực vuông góc với $\overrightarrow{AB}=(-6;2;2)$ nên ta chọn vectơ pháp tuyến là $(-6;2;2)$.\\
			Do đó phương trình mặt phẳng là
			$$-6(x-1)+2(y-1)+2(z-2)=0 \Leftrightarrow 3x-y-z=0.$$
			\itemch Đúng.\\Vì phương trình mặt phẳng là
			$-6(x-1)+2(y-1)+2(z-2)=0 \Leftrightarrow 3x-y-z=0$.
		\end{enumerate}
	}
\end{ex}
\begin{ex} %[2H5H2-6]%Cau 32
	Trong không gian hệ tọa độ $Oxyz$, cho $A(1 ; 2 ;-1) ; B(-1 ; 0 ; 1)$ và mặt phẳng $(P)\colon x+2y-z+1=0$. Mệnh đề nào sau đây đúng hay sai?
	\choiceTF
	{\True $\overrightarrow{AB}=(1 ; 1;-1)$}
	{\True Phương trình mặt phẳng $(Q)$ qua $A,B$ và vuông góc với $(P)$ là $x+z=0$}
	{\True Khoảng cách từ điểm $A$ đến mặt phẳng $(P)$ là: $\mathrm{d}(A,(P))=\dfrac{7 \sqrt{6}}{6}$}
	{ Phương trình mặt phẳng $(Q)$ qua $A, B$ và vuông góc với $(P)$ là $3x-y+z=0$}
	\loigiai{
		\begin{enumerate}
			\itemch Đúng.\\Vì $\overrightarrow{AB}=(-2;-2;2)=-\dfrac{1}{2}(-2;-2;2)=(1 ; 1;-1)$.
			\itemch Đúng.\\
			vectơ pháp tuyến của mặt phẳng $(P)$ là $(1;2;-1)$\\
			Mặt phẳng $(Q$ chứa $\overrightarrow{AB}$ và vuông góc với $(P)$ nên vectơ pháp tuyến của $(Q)$ là tích có hướng của $\overrightarrow{AB}$ và vectơ pháp tuyến của $(P)$
			$$\overrightarrow{n_Q}=\left[ \overrightarrow{AB}, \overrightarrow{n_P}\right] =(-2;0;-2)=(1;0;1).$$
			Vậy phương trình mặt phẳng $(Q)$ qua $A,B$ và vuông góc với $(P)$ là $1(x-1)+0+1(z+1)=x+z=0$.
			\itemch Đúng.\\Khoảng cách từ điểm $A(x_1;y_1;z_1)$ đến mặt phẳng $(P)=ax+by+cz+d=0$ là
			$$\mathrm{d}(A,P)=\dfrac{\left|1\cdot 1+2\cdot 2-(-1)+1\right|}{\sqrt{1^2+2^2+(-1)^2}}=\dfrac{7}{\sqrt{6}}=\dfrac{7\sqrt{6}}{6}.$$
			\itemch Sai.\\Vì phương trình mặt phẳng $(Q)$ qua $A,B$ và vuông góc với $(P)$ là $x+z=0$.
		\end{enumerate}
	}
\end{ex}
\Closesolutionfile{ans}
\indapan{3}{ans/CD3_17-23DS}

\Opensolutionfile{ans}[ans/CD3-14-25-KQ]
\TNSA

\begin{ex} %[2H5H2-3]%câu 33
	Trong KG $Oxyz$, phương trình tổng quát mặt phẳng $(P)\colon ax+by+cz+d=0$ đi qua điểm $M(3 ;-1 ; 4)$ đồng thời vuông góc với giá của vectơ $\overrightarrow{a}=(1 ;-1 ; 2)$. Tính $a+b+c$.
	\shortans{$2$}
	\loigiai{
		Mặt phẳng $(P)$ đi qua điểm $M(3 ;-1 ; 4)$ đồng thời vuông góc với giá của $\overrightarrow{a}=(1 ;-1 ; 2)$ nên nhận $\overrightarrow{a}=(1 ;-1 ; 2)$ làm vectơ pháp tuyến. \\Do đó, $(P)$ có phương trình là
		$$1(x-3)-1(y+1)+2(z-4)=0 \Leftrightarrow x-y+2 z-12=0.$$
		Suy ra $a+b+c=2$.
	}
\end{ex}
\begin{ex} %[2H5H2-3]%Câu 34.
	Trong KG $Oxyz$, phương trình mặt phẳng $(P)\colon ax+by+cz+d=0$ qua $M(0 ;-2 ; 1)$ và có cặp vectơ chỉ phương $\overrightarrow{a}=(-2 ;-3 ; 8), \overrightarrow{b}=(-1 ; 0 ; 6)$. Tính $a+b+c$.
	\shortans{$17$} 
	\loigiai{
		Ta có $\overrightarrow{n}=\left[\overrightarrow{a}, \overrightarrow{b}\right]=(-18 ; 4 ;-3)$. \\
		Mặt phẳng $(P)$ đi qua $M(0 ;-2 ; 1)$ và có vectơ pháp tuyến $\overrightarrow{n}=(-18 ; 4 ;-3)$ nên có phương trình $-18(x-0)+4(y+2)-3(z-1)=0 \Leftrightarrow 18 x-4 y+3 z-11=0$.\\
		Vậy mặt phẳng cần tìm có phương trình: $18x-4y+3z-11=0$.\\
		Suy ra $a+b+c=17$.
	}
\end{ex}
\begin{ex} %[2H5H2-3] %Câu 35
	Trong KG $Oxyz$, cho $A(1 ; 1 ; 0), B(0 ; 2 ; 1), C(1 ; 0 ; 2), D(1 ; 1 ; 1)$. Mặt phẳng $(\alpha)\colon ax+by+cz+d=0$ đi qua $A(1 ; 1 ; 0), B(0 ; 2 ; 1),(\alpha)$ song song với đường thẳng $CD$. Tính $a+b+c$.
	\shortans{$4$}
	\loigiai{
		$\overrightarrow{AB}=(-1 ; 1 ; 1), \overrightarrow{CD}=(0 ; 1 ;-1) \Rightarrow \left[ \overrightarrow{ B}, \overrightarrow{D}\right] =(-2 ;-1 ;-1)$.\\
		$(\alpha)$ đi qua $A(1 ; 1 ; 0)$ và có một VTPT là $\overrightarrow{n}=(2 ; 1 ; 1) \Rightarrow(\alpha)\colon 2 x+y+z-3=0$.\\
		Suy ra $a+b+c=4$.
	}
\end{ex}
\begin{ex} %[2H5H2-3]%Câu 36 
	Trong KG $Oxyz$, cho điểm $M(2 ; 1 ;-3)$ và mặt phẳng $(P)\colon 3 x-2 y+z-3=0$. Phương trình của mặt phẳng đi qua $M$ và song song với $(P)$ có dạng $(Q)\colon ax+by+cz+d=0$. Tính $a+b+c$.
	\shortans{$2$}
	\loigiai{
		Mặt phẳng $(Q)$ cần tìm song song với mặt phẳng $(P)\colon 3 x-2 y+z-3=0$ nên có phương trình dạng
		$$(Q)\colon 3x-2y+z+m=0, m \neq -3.$$
		Vì $M$ $\in(Q)$ nên $(Q)\colon 3\cdot2-2\cdot1+(-3)+m=0 \Leftrightarrow m=-1$.\\
		Vậy $(Q)\colon 3x-2y+z-1=0$.\\
		Suy ra $a+b+c=2$.
	}
\end{ex}
\begin{ex} %[2H5H2-3]%Câu 37.
	Trong KG $Oxyz$, cho ba điểm $A(3 ;-2 ;-2), B(3 ; 2 ; 0), C(0 ; 2 ; 1)$. Phương trình mặt phẳng $(ABC)$ có dạng $=ax+by+cz+d=0$. Tính $a+b+c$.
	\shortans{$5$}
	\loigiai{
		Ta có $\overrightarrow{AB}=(0 ; 4 ; 2), \overrightarrow{AC}=(-3 ; 4 ; 3), \overrightarrow{n}=\left[ \overrightarrow{B} ; \overrightarrow{C}\right]=(4 ;-6 ; 12)$.\\
		Ta có $\overrightarrow{n}=(4 ;-6 ; 12)$ cùng phương $\overrightarrow{n}_{1}=(2 ;-3 ; 6)$.\\
		Mặt phẳng $(ABC)$ đi qua điểm $C(0 ; 2 ; 1)$ và có một vectơ pháp tuyến $\overrightarrow{n}_{1}=(2 ;-3 ; 6)$ nên $(ABC)$ có phương trình là
		$$2(x-0)-3(y-2)+6(z-1)=0 \Leftrightarrow 2 x-3 y+6 z=0.$$
		Vậy phương trình mặt phẳng cần tìm là $2x-3y+6z=0$.\\
		Suy ra $a+b+c=5$. 
	}
\end{ex}
\begin{ex}  %[2H5H2-4]%Câu 38
	Trong không gian, cho hai điểm $A(0 ; 0 ; 1)$ và $B(2 ; 1 ; 3)$. Phương trình mặt phẳng đi qua $A$ và vuông góc với $ABC\colon ax+by+cz+d=0$. Tính $a+b+c$.
	\shortans{$5$}
	\loigiai{
		Mặt phẳng đi qua $A(0 ; 0 ; 1)$ và nhận vectơ $\overrightarrow{AB}=(2 ; 1 ; 2)$ làm vectơ pháp tuyến nên có phương trình là
		$$2(x-0)+(y-0)+2(z-1)=0 \Leftrightarrow 2x+y+2z-2=0.$$
		Suy ra $a+b+c=5$.}
\end{ex}
\begin{ex} %[2H5H2-4]%Câu 39
	Trong KG $Oxyz$, cho hai điểm $A(2 ; 4 ; 1), B(-1 ; 1 ; 3)$ và mặt phẳng $(P)\colon x-3y+2z-5=0$. Lập phương trình mặt phẳng $(Q)$ đi qua hai điểm $A, B$ và vuông góc với mặt phẳng $(P)\colon ax+by+cz+d=0$. Tính $a+b+c$.
	\shortans{$5$}
	\loigiai{
		Ta có: $\overrightarrow{AB}=(-3 ;-3 ; 2)$, vectơ pháp tuyến của $(P)$ là $\overrightarrow{n}_{P}=(1 ;-3 ; 2)$.\\
		Từ giả thiết suy ra $\overrightarrow{n}=\left[\overrightarrow{AB}, \overrightarrow{n}_{P}\right]=(0 ; 8 ; 12)$ là vectơ pháp tuyến của $(Q)$. \\
		$(Q)$ đi qua điểm $A(2 ; 4 ; 1)$ suy ra phương trình tổng quát của $(Q)$ là
		$$0(x-2)+8(y-4)+12(z-1)=0 \Leftrightarrow 2y+3z-11=0.$$
		Suy ra $a+b+c=5$.
	}
\end{ex}
\begin{ex}%[2H5H2-3]%Câu 40
	Trong KG $Oxyz$, gọi $M, N, P$ lần lượt là hình chiếu vuông góc của $A(2 ;-3 ; 1)$ lên các mặt phẳng tọa độ. Tính $a+b+c$ của phương trình mặt phẳng $(MNP)\colon ax+by+cz+d=0$. 
	\shortans{$7$}
	\loigiai{
		Không mất tính tổng quát, ta giả sử $M, N, P$ lần lượt là hình chiếu vuông góc của $A(2 ;-3 ; 1)$ lên các mặt phẳng tọa độ $(Oxy),(Oxz),(Oyz)$. \\
		Khi đó $M(2 ;-3 ; 0), N(2 ; 0 ; 1)$ và $P(0 ;-3 ; 1).$\\
		$\overrightarrow{MN}=(0 ; 3 ; 1)$ và $\overrightarrow{MP}=(-2 ; 0 ; 1)$. \\
		Ta có $\overrightarrow{MN}$ và $\overrightarrow{MP}$ là cặp vectơ không cùng phương và có giá nằm trong $(MNP)$.\\
		Do đó $(MNP)$ có một vectơ pháp tuyến là $\overrightarrow{n}=\left[\overrightarrow{M N}, \overrightarrow{MP}\right]=(3 ;-2 ; 6)$.\\
		Mặt khác $(MNP)$ đi qua $M(2 ;-3 ; 0)$ nên có phương trình là
		$$3(x-2)-2(y+3)+6(z-0)=0 \Leftrightarrow 3x-2y+6z-12=0.$$
		Suy ra $a+b+c=7$.
	}
\end{ex}
\Closesolutionfile{ans}
\indapan{8}{ans/CD3-14-25-KQ}
\begin{dang}{Viết PTTQ MP khi biết VTPT, VTCP nhưng không biết điểm đi qua}
	\begin{itemize}
		\item Viết phương trình mặt phẳng $(\alpha)$ dưới dạng
		$$
		Ax+By+Cz+D=0
		.$$
		\item Sau đó dựa vào giả thiết bài toán để tìm giá trị $D$.\\
		Chú ý: Dạng này giả thiết có liên quan đến khoảng cách và góc liên quan đến mặt phẳng.
	\end{itemize}
\end{dang}

\Opensolutionfile{ans}[ans/CD3-B46-B49-KQ]
\TN
\begin{ex}%[2H5V2-5]%Câu 41
	Trong KG $Oxyz$, cho mặt phẳng $(P)\colon 2 x+2y-z-1=0$ Mặt phẳng nào sau đây song song với $(P)$ và cách $(P)$ một khoảng bằng $3$?
	\choice
	{$(Q)\colon 2x+2y-z+10=0$}
	{$(Q)\colon 2x+2y-z+4=0$}
	{\True $(Q)\colon 2x+2y-z+8=0$}
	{$(Q)\colon 2x+2y-z-8=0$}
	\loigiai{
		Mặt phẳng $(P)$ đi qua điểm $M(0 ; 0 ;-1)$ và có một vectơ pháp tuyến $\overrightarrow{n}=(2 ; 2 ;-1)$.\\
		Mặt phẳng $(Q)$ song song với $(P)$ và cách $(P)$ một khoảng bằng $3$ nên có dạng
		$$(Q)\colon 2x+2y-z+d=0,\quad(d \neq -1).$$
		Mặt khác ta có $\mathrm{d}(M,(Q))=3$ 
		\begin{align*}
			\Leftrightarrow & \dfrac{|1+d|}{\sqrt{4+4+1}}=3\\
			\Leftrightarrow &|d+1|=9\\
			\Leftrightarrow &\hoac{d&=8\\d&=-10} \text{(thỏa mãn)}.
		\end{align*}
		Do đó $(Q)\colon 2x+2y-z+8=0$ hoặc $(Q)\colon 2x+2y-z-10=0$. 
	}
\end{ex}
\begin{ex} %[2H5V2-5]%Câu 42
	Trong KG $Oxyz$, cho ba điểm $A(2 ; 0 ; 0), B(0 ; 3 ; 0), C(0 ; 0 ;-1)$. Phương trình của mặt phẳng $(P)$ qua $D(1 ; 1 ; 1)$ và song song với mặt phẳng $(ABC)$ là
	\choice
	{$2x+3y-6z+1=0$}
	{\True $3x+2y-6z+1=0$}
	{$3x+2y-5z=0$}
	{$6x+2y-3z-5=0$}
	\loigiai{
		Phương trình đoạn chắn của mặt phẳng $(ABC)$ là $\dfrac{x}{2}+\dfrac{y}{3}+\dfrac{z}{-1}=1$.\\
		Mặt phẳng $(P)$ song song với mặt phẳng $(ABC)$ nên\\
		$(P)\colon \dfrac{1}{2} x+\dfrac{1}{3} y-z+m=0\quad(m \neq-1)$.\\
		Do $D(1 ; 1 ; 1) \in(P)$ có $\dfrac{1}{2}\cdot 1+\dfrac{1}{3} \cdot 1-1+m=0 \Leftrightarrow m-\dfrac{1}{6}=0 \Leftrightarrow m=\dfrac{1}{6}$.\\
		Vậy $(P)\colon \dfrac{1}{2}x+\dfrac{1}{3}y-z+\dfrac{1}{6}=0 \Leftrightarrow (P)\colon 3x+2y-6z+1=0$.
	}
\end{ex}
\begin{ex} %[2H5V2-5]%Câu 43.
	Trong KG $Oxyz$ cho $A(2 ; 0 ; 0), B(0 ; 4 ; 0), C(0 ; 0 ; 6), D(2 ; 4 ; 6)$. Gọi $(P)$ là mặt phẳng song song với mặt phẳng $(A B C),(P)$ cách đều $D$ và mặt phẳng $(ABC)$. Phương trình của $(P)$ là
	\choice
	{\True $6x+3y+2z-24=0$}
	{$6x+3y+2z-12=0$}
	{$6x+3y+2z=0$}
	{$6x+3y+2z-36=0$}
	\loigiai{
		$(ABC)\colon \dfrac{x}{2}+\dfrac{y}{4}+\dfrac{z}{6}=1 \Leftrightarrow 6x+3y+2z-12=0$.\\
		$(P)\parallel(ABC) \Rightarrow(P)\colon 6x+3y+2z+m=0\quad(m\neq-12)$.\\
		$(P)$ cách đều $D$ và mặt phẳng $(ABC) \Rightarrow \mathrm{d}(D,(P))=\mathrm{d}(A,(P))$.
		\begin{align*}
			\Leftrightarrow& \dfrac{|6\cdot 2+3\cdot 4+2\cdot 6+m|}{\sqrt{6^{2}+3^{2}+2^{2}}}=\dfrac{|6\cdot 2+3\cdot 0+2\cdot 0+m|}{\sqrt{6^{2}+3^{2}+2^{2}}}\\
			\Leftrightarrow&|36+m|=|12+m|\\ \Leftrightarrow& \hoac{36+m=12+m \\ 36+m=-12-m}\\
			\Leftrightarrow& m=-24 \text{(cách)  (nhận).}
		\end{align*}
		Vậy phương trình của $(P)$ là $6x+3y+2z-24=0$.
	}
\end{ex}
\begin{ex} %[2H5V2-5]%Cau 44 
	Trong không gian với hệ trục tọa độ $Oxyz$, cho mặt phẳng $(Q)\colon x+2y+2z-3=0$, mặt phẳng $(P)$ không qua $O$, song song với mặt phẳng $(Q)$ và $\mathrm{d}((P),(Q))=1$. Phương trình mặt phẳng $(P)$ là
	\choice
	{$x+2y+2z+1=0$}
	{$ x+2y+2z=0$}
	{\True $ x+2y+2z-6=0$}
	{$ x+2y+2z+3=0$}
	\loigiai{
		Vì mặt phẳng $(P)$ song song với mặt phẳng $(Q)$.\\
		$\Rightarrow$ vtpt $\overrightarrow{n}_{P}=$ vtpt $\overrightarrow{n}_{Q}=(1 ; 2 ; 2)$.\\
		Phương trình mặt phẳng $(P)$ có dạng $x+2y+2z+d=0\quad(d \ne 0).$\\
		Gọi $A(3 ; 0 ; 0) \in (Q)$\\
		$\Rightarrow \mathrm{d}((P),(Q))=\mathrm{d}(A,(P))=1$\\
		$\Leftrightarrow \dfrac{|3+D|}{3}=1 \Leftrightarrow\hoac{3+d&=3 \\ 3+d&=-3} \Leftrightarrow\hoac{d&=0 &(\text{loại})&O\\ d&=-6 &(\text{nhận})&}.$
	}
\end{ex}
\begin{ex} %[2H5V2-5]%Cau 45
	Trong KG $Oxyz$, cho mặt phẳng $(P)\colon 2 x-2 y+z-5=0$.  Viết phương trình mặt phẳng $(Q)$ song song với mặt phẳng $(P)$, cách $(P)$ một khoảng bằng $3$ và cắt trục $Ox$ tại điểm có hoành độ dương. 
	\choice
	{$(Q)\colon 2x-2y+z+4=0$}
	{\True $(Q)\colon 2x-2y+z-14=0$}
	{$(Q)\colon 2x-2y+z-19=0$}
	{$(Q)\colon 2x-2y+z-8=0$}
	\loigiai{
		Ta có, $(Q)$ song song $(P)$ nên phương trình mặt phẳng $(Q)\colon 2x-2y+z+d=0$; $d\ne -5$.\\
		Chọn $M(0 ; 0 ; 5)\in(P)$.\\
		Ta có $\mathrm{d}((P),(Q))=\mathrm{d}(M),(Q))=\dfrac{|5+d|}{\sqrt{2^{2}+(-2)^{2}+1^{2}}}=3 \Leftrightarrow\hoac{d&=4 \\ d&=-14.}\\
		\\d=4 \Rightarrow(Q)\colon 2x-2y+z+4=0$ khi đó $(Q)$ cắt $Ox$ tại điểm $M_{1}(-2 ; 0 ; 0)$ có hoành độ âm nên trường hợp này $(Q)$ không thỏa đề bài.\\
		$d=-14 \Rightarrow(Q)\colon 2x-2y+z-14=0$ khi đó $(Q)$ cắt $Ox$ tại điểm $M_{2}(7 ; 0 ; 0)$ có hoành độ dương do đó $(Q)\colon 2x-2y+z-14=0$ thỏa đề bài.\\
		Vậy phương trình mặt phẳng $(Q)\colon 2x-2y+z-14=0$.
	}
\end{ex}
\Closesolutionfile{ans}
\indapan{10}{ans/CD3-B46-B49-KQ}

\TNSA
\Opensolutionfile{ans}[ans/CD3-14-25-KQ2]
\begin{ex} %[2H5V2-5]%Câu 46
	Trong không gian hệ toạ độ $Oxyz$, lập phương trình các mặt phẳng song song với mặt phẳng $(\beta)\colon x+y-z+3=0$ và cách $(\beta)$ một khoảng bằng $\sqrt{3}$ có dạng $ax+by+cz+d=0\quad (d\neq 0)$. Tính $a+b+c$.
	\shortans{$1$}
	\loigiai{
		Gọi mặt phẳng $(\alpha)$ cần tìm.\\
		Vì $(\alpha)\parallel(\beta)$ nên phương trình $(\alpha)$ có dạng: $x+y-z+c=0$ với $c$ khác $\backslash\{3\}$.\\
		Lấy điểm $I(-1 ;-1 ; 1) \in(\beta)$.\\
		Vì khoảng cách từ $(\alpha)$ đến $(\beta)$ bằng $\sqrt{3}$ nên ta có
		$$\mathrm{d}(I,(\alpha))=\sqrt{3} \Leftrightarrow \dfrac{|-1-1-1+c|}{\sqrt{3}}=\sqrt{3} \Leftrightarrow \dfrac{|c-3|}{\sqrt{3}}=\sqrt{3} \Leftrightarrow\hoac{c=0 \\ c=6}. (\text{thỏa điều kiện } c \in \mathbb{R} \backslash\{3\} ).$$
		Vậy phương trình $(\alpha)\colon x+y-z+6=0$ hoặc $(\alpha)\colon x+y-z=0$.\\
		Suy ra $a+b+c=1$.
	}
\end{ex}
\begin{ex} %[2H5V2-5]%Câu 47.
	Trong không gian với hệ trục tọa độ $Oxyz$, cho hai mặt phẳng $\left(Q_{1}\right)\colon 3x-y+4z+2=0$ và $\left(Q_{2}\right)\colon 3x-y+4z+8=0$. Viết phương trình mặt phẳng $(P)\colon ax+by+cz=0$ song song và cách đều hai mặt phẳng $\left(Q_{1}\right)$ và $\left(Q_{2}\right)$. Tính $a+b+c$.
	\shortans{$6$}
	\loigiai{
		Mặt phẳng $(P)$ có dạng $3x-y+4z+d=0$. \\
		Lấy $M(0 ; 2 ; 0) \in\left(Q_{1}\right)$ và $N(0 ; 8 ; 0) \in\left(Q_{2}\right)$. Do $\left(Q_{1}\right)\parallel\left(Q_{2}\right)$ trung điểm $I(0 ; 5 ; 0)$ của $MN$ phải thuộc vào $(P)$ nên ta tìm được $D=5$. Vậy $(P)\colon 3x-y+4z+5=0$.\\
		Suy ra $a+b+c=6$.
	}
\end{ex}
\begin{ex}  %[2H5V2-5]%Câu 48
	Trong KG $Oxyz$, gọi $(\gamma)$ là mặt phẳng cách đều hai mặt phẳng sau đây:
	$4x-y-2z-3=0$, $4x-y-2z-5=0$. lập mặt phẳng $(\gamma)$ có dạng $ax+by+cz=0$. Tính $a+b+c+d$.
	\shortans{$-3$}
	\loigiai{
		Gọi điểm $A(0;-3; 0) \in (\alpha)\colon4x-y-2z-3=0$ và $B(0 ;-5 ; 0) \in (\beta)\colon4x-y-2z-5=0$.\\
		Mặt phẳng cách đều hai mặt phẳng trên có dạng: $(\gamma)\colon 4x-y-2z+m=0$.\\
		Để mặt phẳng $(\gamma)$ cách đều hai mặt phẳng trên thì
		$$\mathrm{d}(A\colon(\beta))=2 \mathrm{d}(A\colon(\gamma))
		\Leftrightarrow|m+3|=1 \Leftrightarrow\hoac{m=-2 \\ m=-4.}$$ 
		Mặt khác điểm hai điểm $A, B$ phải nằm về hai phía của mặt phẳng $(\gamma)$.\\
		Do đó:
		\begin{itemize}
			\item Với $m=-2$ ta có: $(4\cdot0+3-2\cdot0-2)(4\cdot0+5-2\cdot0-2)>0$ nên $A, B$ cùng phía.
			\item Với $ m=-4$ ta có: $(4\cdot0+3-2\cdot0-4)(4\cdot 0+5-2\cdot 0-4)<0$ nên $A, B$ khác phía.
		\end{itemize}
		Vậy phương trình mặt phẳng cần tìm là $(\gamma)\colon 4x-y-2z-4=0$.\\
		Suy ra $a+b+c+d=-3$.
	}
\end{ex}
\begin{ex} %[2H5V2-5]%Câu 49.
	Trong KG $Oxyz$ cho các điểm $A(2 ; 0 ; 0), B(0 ; 4 ; 0), C(0 ; 0 ; 6), D(2 ; 4 ; 6)$. Gọi $(P)$ là mặt phẳng song song với mặt phẳng $(A B C),(P)$ cách đều $D$ và mặt phẳng $(A B C)$. Viêt phương trình của mặt phẳng $(P)\colon ax+by+cz+d=0$. Tính $a+b+c$.
	\shortans{$11$}
	\loigiai{
		Phương trình mặt phẳng $(ABC)$ là $\dfrac{x}{2}+\dfrac{y}{4}+\dfrac{z}{6}=1 \Leftrightarrow 6x+3y+2z-12=0$
		\begin{itemize}
			\item $(P)$ song song với mặt phẳng $(ABC)$ nên $(P)$ có dạng $$6x+3y+2z+d=0\quad(d \ne q-12).$$
			\item Khoảng cách từ $D$ đến mặt phẳng $(P)$ là
			\begin{align*}
				&\mathrm{d}(D),(P))=\mathrm{d}((ABC),(P))\\
				&\Leftrightarrow \mathrm{d}(D),(P))=\mathrm{d}(A,(P))\\
				&\Leftrightarrow|36+d|=|12+d|\\
				&\Leftrightarrow d=-24.
			\end{align*}
		\end{itemize}
		Vậy $(P)\colon 6x+3y+2z-24=0$.\\
		Suy ra $a+b+c=11$.
	}
\end{ex}
\Closesolutionfile{ans}
\indapan{4}{ans/CD3-14-25-KQ2}

\begin{dang}{Viết PTTQ khi biết điểm đi qua nhưng không biết vectơ}
\end{dang}
\begin{tomtat}
	Khi bài toán cho biết mặt phẳng $(\alpha)$ đi qua điềm $M_0\left(x_0 ; y_0 ; z_0\right)$ và giả thiết bài toán không cho vectơ pháp tuyến $\overrightarrow{n}$ hoặc không cho hai vectơ chỉ phương $\overrightarrow{a}, \overrightarrow{b}$ thì ta thực hiện các bước sau:
	\begin{itemize}
		\item Gọi vectơ pháp tuyến của mặt phẳng $(\alpha)$ là $\overrightarrow{n}=(A ; B ; C)$ với $A^2+B^2+C^2 \neq 0$.
		\item Viết phương trình mặt phẳng $(\alpha)$ dưới dạng:
		$$
		(\alpha)\colon A\left(x-x_0\right)+B\left(y-y_0\right)+C\left(z-z_0\right)=0.
		$$
		\item Sau đó dựa vào giả thiết bài toán để tìm \textbf{hai} phương trình chứa 3 ẩn $A, B, C$.
	\end{itemize}
	Chú ý:
	\begin{itemize}
		\item Dạng này, giả thiết có liên quan đến khoảng cách và góc liên quan đến mặt phẳng.
		\item Để giải tìm vectơ pháp tuyến của mặt phẳng đơn giàn hơn thì gọi vectơ pháp tuyến của mặt phẳng là $\overrightarrow{n}=(1 ; B ; C)$.
	\end{itemize}
\end{tomtat}

\Opensolutionfile{ans}[ans/CD3-50-50]
\TN

\begin{ex} %[2H5V2-5]%Câu 50
	Trong KG $Oxyz$, cho $3$ điểm $A(1 ; 0 ; 0), B(0 ;-2 ; 3), C(1 ; 1 ; 1)$. Gọi $(P)$ là mặt phẳng chứa $A, B$ sao cho khoảng cách từ $C$ tới mặt phẳng $(P)$ bằng $\dfrac{2}{\sqrt{3}}$.  Phương trình mặt phẳng $(P)$ là
	\choice
	{$\hoac{&2x+3y+z-1=0 \\& 3x+y+7z+6=0}$}
	{$\hoac{&x+2y+z-1=0 \\ &-2x+3y+6z+13=0}$}
	{$\hoac{&x+y+2z-1=0 \\& -2x+3y+7z+23=0}$}
	{\True $\hoac{&x+y+z-1=0 \\& -23x+37y+17z+23=0}$} 
	\loigiai{
		Gọi $(P)\colon \heva{&\text{ qua } A(1 ; 0 ; 0)\\ &\text{ VTPT } \overrightarrow{n}=(A ; B ; C) \neq \overrightarrow{0}}$\\
		$(P)\colon A \cdot(x-1)+By+Cz=0$.\\
		$B\in(P)\colon -A-2B+3=0 \Leftrightarrow A=-2B+3C$.\\
		$ \mathrm{d}(C\colon(P))=\dfrac{2}{\sqrt{3}} \Leftrightarrow \dfrac{|B+C|}{\sqrt{A^{2}+B^{2}+C^{2}}}=\dfrac{2}{\sqrt{3}}$\\
		$\Leftrightarrow 3\left(B^{2}+C^{2}+2BC\right)=4\left(A^{2}+B^{2}+C^{2}\right)$\\
		$\Leftrightarrow B^{2}+C^{2}-6BC+4A^{2}=0$.\\
		Thay $A=-2B+3C$ vào $B^{2}+C^{2}-6BC+4A^{2}=0$\\ 
		Ta có: $B^{2}+C^{2}-6BC+4(-2B+3C)^{2}=0 \Leftrightarrow 17B^{2}-54BC+37C^{2}=0$\\
		Cho $C=1$ từ đó suy ra $17 B^{2}-54 B+37=0 \Leftrightarrow\hoac{&B=1& &\Rightarrow& &A=1&\\ &B=\dfrac{37}{17}& &\Rightarrow& &A=\dfrac{-23}{17}.&}$\\
		Suy ra $\hoac{&(P)\colon x+y+x-1=0\\&(P)\colon-23x+37y+17z+23=0.}$
	}
\end{ex}
\begin{ex}%[2H5H1-3]
	Trong hệ trục tọa độ $O x y z$ cho $3$ điểm $M(4 ; 2 ; 1)$, $N(0 ; 0 ; 3)$, $Q(2 ; 0 ; 1)$. Viết phương trình mặt phẳng chứa $O Q$ và cách đều $2$ điểm $M$, $N$.
	\choice
	{$x-2 y-2 z=0$ hoặc $x+4 y-2 z=0$}
	{$x+2 y+2 z=0$ hoặc $x-4 y-2 z=0$}
	{$x+2 y-2 z=0$ hoặc $x+4 y-2 z=0$}
	{\True $x+2 y-2 z=0$ hoặc $x-4 y-2 z=0$}
	\loigiai{
		Gọi $(\alpha)\colon A x+B y+C z+D=0$ $\left(A^2+B^2+C^2 \neq 0\right)$.\\
		$O \in(\alpha)$ nên ta có $D=0$, $Q \in(\alpha)$ nên ta có $2 A+C=0 \Rightarrow C=-2 A$.\\
		Theo đề bài
		$$\mathrm{d}(M,(\alpha))=\mathrm{d}(N,(\alpha))	\Leftrightarrow|2 A+2 B|=|-6 A| \Leftrightarrow \hoac{&2 A + 2B = 6 A\\&2 A + 2B = - 6 A}\Leftrightarrow \hoac{&B=2 A& (*)\\&B=-4 A	 & (* *).}$$
		Từ $(*)$ chọn $A=1 \Rightarrow B=2$, $C=-2 \Rightarrow(\alpha)\colon x+2 y-2 z=0$.\\
		Từ $(**)$ chọn $A=1 \Rightarrow B=-4$, $C=-2 \Rightarrow(\alpha)\colon x-4 y-2 z=0$.
	}
\end{ex}
\Closesolutionfile{ans}
\indapan{10}{ans/CD3-50-50}
\TNSA
\begin{ex}%[2H5H1-3]
	Trong không gian với hệ toạ độ $O x y z$, biết mặt phẳng $(P)\colon Ax+By+Cz+D=0$ ($A$, $B$, $C \in \mathbb{Z}$, $A$ và $C$ trái dấu) qua $O$, vuông góc với mặt phẳng $(Q)\colon x+y+z=0$ và cách điểm $M(1 ; 2 ;-1)$ một khoảng bằng $\sqrt{2}$. Tính giá trị của $A+B+C$.
	\shortans{$0$}
	\loigiai{
		$(P)$ qua $O$ nên phương trình có dạng $A x+B y+C z=0$ (với $A^2+B^2+C^2 \neq 0$ ).\\
		Vì $({P}) \perp({Q})$ nên $1 \cdot A+1 \cdot B+1 \cdot C=0 \Leftrightarrow C=-A-B \quad (1)$.\\
		Do $\mathrm{d}(M,(P))=\sqrt{2} \Leftrightarrow \dfrac{|A+2 B-C|}{\sqrt{A^2+B^2+C^2}}=\sqrt{2} \Leftrightarrow(A+2 B-C)^2=2\left(A^2+B^2+C^2\right) \quad (2)$.\\
		Từ $(1)$ và $(2)$ ta được $8 A B+5 B^2=0 \Leftrightarrow\hoac{&B=0 &(3)\\ &8 A+5 B=0&(4).}$\\
		Từ $(3)$, ta có ${B}=0 \Rightarrow {C}=-{A}$ (nhận do $A$ và $C$ trái dấu). \\
		Chọn ${A}=1$, ${C}=-1 \Rightarrow({P})\colon  x-z=0$.\\
		Khi đó $A+B+C=0$.\\
		Từ $(4)$, ta có $8 {A}+5 {B}=0$. \\
		Chọn ${A}=5$, ${B}=-8 \Rightarrow {C}=3 \Rightarrow({P})\colon 5 x-8 y+3 z=0$. (loại  do $A$ và $C$ cùng dấu).
	}
\end{ex}

\begin{ex}%[2H5H1-3]
	Trong không gian với hệ toạ độ $Oxyz$, cho các điểm $M(-1 ; 1 ; 0)$, $N(0 ; 0 ;-2)$, $I(1 ; 1 ; 1)$. Biết mặt phẳng $({P})$ qua ${A}$ và ${B}$, đồng thời khoảng cách từ ${I}$ đến $({P})$ bằng $\sqrt{3}$. Giả sử phương trình mặt phẳng $(P)$ có dạng $ax+by+z+d=0$ với $b>0$. Tính $\dfrac{a}{b}$ viết dưới dạng số thập phân.
	\shortans{$1{,}4$}
	\loigiai{
		Phương trình mặt phẳng $({P})$ có dạng $a x+b y+ z+d=0$ $\left(a^2+b^2+1 \neq 0\right)$.\\
		Ta có $\heva{ &M \in(P) \\ &N \in(P) \\ &\mathrm{d}(I,(P))=\sqrt{3}} \Leftrightarrow \hoac{&a=-b,\ 2 =a-b,\ d=a-b &(1)\\ &5 a=7 b,\ 2 =a-b,\ d=a-b &(2).}$
		\begin{itemize}
			\item Với $(1) \Rightarrow $ Phương trình mặt phẳng $(P)\colon x-y+z+2=0$ (loại do $b<0$).
			\item Với $(2) \Rightarrow $ Phương trình mặt phẳng $(P)\colon 7 x+5 y+z+2=0$ (nhận do $b=5>0$).\\
			Khi đó $\dfrac{a}{b}=\dfrac{7}{5}=1{,}4$.
		\end{itemize}
	}
\end{ex}

\begin{ex}%[2H5H1-3]
	Trong không gian với hệ toạ độ $O x y z$, cho tứ diện $ABCD$ với $A(1 ;-1 ; 2)$, $B(1 ; 3 ; 0)$, $C(-3 ; 4 ; 1)$, $D(1 ; 2 ; 1)$. Mặt phẳng $({P})$ đi qua ${A}$, ${B}$ sao cho khoảng cách từ ${C}$ đến $({P})$ bằng khoảng cách từ ${D}$ đến $({P})$. Biết có hai mặt phẳng $(P)$ thỏa yêu cầu đề bài là $x+b_1y+c_1z+d_1=0$ và $x+b_2y+c_2z+d_2=0$. Tính $S=b_1+c_1+b_2+c_2$.
	\shortans{$9$}
	\loigiai{
		Phương trình mặt phẳng $({P})$ có dạng $a x+b y+c z+d=0$ với $\left(a^2+b^2+c^2 \neq 0\right)$.\\
		Ta có $\heva{&A \in(P) \\ &B \in(P) \\ &\mathrm{d}(C,(P))=\mathrm{d}(D,(P))} \Leftrightarrow\heva{&a-b+2 1+d=0 \\&a+3 b+d=0 \\&\dfrac{|-3 a+4 b+1+d|}{\sqrt{a^2+b^2+1^2}}=\dfrac{|a+2 b+1+d|}{\sqrt{a^2+b^2+1^2}}}$\\
		$ \Leftrightarrow\hoac{&b=2 a,\ c=4 a,\ d=-7 a \\& c=2 a,\ b=a,\ d=-4 a}$
		\begin{itemize}
			\item Với $b=2 a$, $c=4 a$, $d=-7 a$ và ta đã có $a=1$ nên $({P}) \colon x+2 y+4 z-7=0$.\\
			Khi đó $b_1=2$, $c_1=4$.
			\item Với $c=2 a$, $b=a$, $d=-4 a$ và ta đã có $a=1$ nên $({P})\colon x+y+2 z-4=0$.\\
			Khi đó $b_2=1$, $c_2=2$.
		\end{itemize}
		Vậy $S=2+4+1+2=9$.
	}
\end{ex}

\begin{ex}%[2H5H1-3]
	Trong không gian với hệ trục tọa độ $O x y z$, cho các điểm $A(1 ; 2 ; 3)$, $B(0 ;-1 ; 2)$, $C(1 ; 1 ; 1)$. Mặt phẳng $(P)$ đi qua $A$ và gốc tọa độ $O$ sao cho khoảng cách từ $B$ đến $(P)$ bằng khoảng cách từ $C$ đến $(P)$. Biết phương trình mặt phẳng $(P)$ có dạng $ax+by-4z+d=0$. Hỏi $a$ có bao nhiêu ước nguyên?
	\shortans{$12$}
	\loigiai{
		Vì $O \in(P)$ nên $(P)\colon a x+by-4 z=0$, với $a^2+b^2+16 \neq 0$.\\
		Do $A \in(P) \Rightarrow a+2 b -12=0$ $(1)$\\
		Và $\mathrm{d}(B,(P))=\mathrm{d}(C,(P)) \Leftrightarrow|-b-8|=|a+b-4|$ $(2)$.\\
		Từ $(1)$ và $(2) \Rightarrow b=0$.	Khi đó ta được $a=-3 \cdot (-4) =12$.\\
		Các ước nguyên của $12$ là $\{\pm 1;\pm 2; \pm 3; \pm 4; \pm 6; \pm 12\}$ có $12$ ước nguyên.
	}
\end{ex}

\begin{ex}%[2H5H1-3]
	Trong không gian với hệ trục tọa độ $O x y z$, cho ba điểm $A(1 ; 1 ;-1)$, $B(1 ; 1 ; 2)$, $C(-1 ; 2 ;-2)$ và mặt phẳng $({P})\colon x-2 y+2 z+1=0$. Mặt phẳng $(\alpha)$ đi qua ${A}$, vuông góc với mặt phẳng $({P})$, cắt đường thẳng ${BC}$ tại ${I}$ sao cho $I B=2 I C$. Biết có hai mặt phẳng $(\alpha)$ thỏa yêu cầu đề bài có phương trình lần lượt là $4x+b_1y+c_1+d_1=0$ và $2x+b_2y+c_2+d_2=0$ với $b_1<b_2$. Hỏi có bao nhiêu giá trị nguyên thuộc tập $(b_1;b_2)$?
	\shortans{$4$}
	\loigiai{
		Phương trình mặt phẳng $(\alpha)$ có dạng $a x+b y+c z+d=0$, với $a^2+b^2+c^2 \neq 0$.\\
		Do $A(1 ; 1 ;-1) \in(\alpha)$ nên $a+b-c+d=0$. $\quad (1)$;\\
		$(\alpha) \perp(P)$ nên $a-2 b+2 c=0\quad (2)$.
		\begin{eqnarray*}
			&I B=2 I C &\Rightarrow \mathrm{d}(B,(\alpha))=2 \mathrm{d}(C ;(\alpha))\\  & &\Rightarrow \dfrac{|a+b+2 c+d|}{\sqrt{a^2+b^2+c^2}}=2 \dfrac{|-a+2 b-2 c+d|}{\sqrt{a^2+b^2+c^2}}\\
			& &\Leftrightarrow\hoac{&3 a-3 b+6 c-d=0 \\&-a+5 b-2 c+3 d=0.}
		\end{eqnarray*}
		Từ $(1)$, $(2)$, $(3)$ ta có 2 trường hợp sau
		\begin{itemize}
			\item $\heva{&a+b-c+d=0 \\&a-2 b+2 c=0 \\& 3 a-3 b+6 c-d=0} \Leftrightarrow \heva{&b=\dfrac{-1}{2} a \\& c=-a \\& d=\dfrac{-3}{2} a.}$
			\item $\heva{&a+b-c+d=0 \\ &a-2 b+2 c=0 \\ &-a+5 b-2 c+3 d=0} \Leftrightarrow \heva{&b=\dfrac{3}{2} a \\ &c=a \\ &d=\dfrac{-3}{2} a.} \quad (3)$
		\end{itemize}
		Do theo đề bài, ta có $a>0$ nên ta có thể có được $4x+b_1y+c_1+d_1=0$ là mặt phẳng ở trường hợp $1$ và $2x+b_2y+c_2+d_2=0$ là mặt phẳng ở trường hợp $2$.\\
		Khi đó
		\begin{itemize}
			\item Chọn $a=4 \Rightarrow b_1=-2$; $c_1=-4$; $d_1=-6 \Rightarrow(\alpha)\colon 4 x-2y-4 z-6=0$.
			\item Với $a=2 \Rightarrow b=3 $; $c=2 $; $d=-3 \Rightarrow(\alpha)\colon 2 x+3 y+2 z-3=0$.
		\end{itemize}
		Vậy ta có tập $(-2;3)$ có tất cả $4$ giá trị nguyên là $-1$, $0$, $1$, $2$.
	}
\end{ex}
\Closesolutionfile{ans}
\indapan{6}{ans/ans-2-C5B1CD2-D3}
\begin{dang}{Một số dạng khác}
	
\end{dang}
\Opensolutionfile{ans}[ans/ans-2-C5B1CD2-D4]
\TN

\begin{ex}%[2H5H1-3]
	Trong không gian $O x y z$ cho điểm $M(1 ; 2 ; 3)$. Viết phương trình mặt phẳng $(P)$ đi qua điểm $M$ và cắt các trục tọa độ $O x$, $O y$, $O z$ lần lượt tại $A$, $B$, $C$ sao cho $M$ là trọng tâm của tam giác $A B C$.
	\choice
	{$(P)\colon 6 x+3 y+2 z+18=0$}
	{$(P)\colon 6 x+3 y+2 z+6=0$}
	{\True $(P)\colon 6 x+3 y+2 z-18=0$}
	{$(P)\colon 6 x+3 y+2 z-6=0$}
	\loigiai{
		Theo giả thiết $A \in O x$, $B \in O y$, $C \in O z$ nên ta có thể đặt $A(a ; 0 ; 0)$, $B(0 ; b ; 0)$, $C(0 ; 0 ; c)$.\\
		Vì $M(1 ; 2 ; 3)$ là trọng tâm tam giác $A B C$ nên $\heva{&a=3 \\ &b=6 \\ &c=9.}$\\
		Từ đó ta có phương trình mặt phẳng theo đoạn chắn là
		$$	(P)\colon \dfrac{x}{3}+\dfrac{y}{6}+\dfrac{z}{9}=1 \Leftrightarrow 6 x+3 y+2 z-18=0.		$$
	}
\end{ex}

\begin{ex}%[2H5H1-3]
	Trong không gian với hệ trục tọa độ $O x y z$, cho điểm $G(1 ; 4 ; 3)$. Mặt phẳng nào sau đây cắt các trục $O x$, $O y$, $O z$ lần lượt tại $A$, $B$, $C$ sao cho $G$ là trọng tâm tứ diện $O A B C$?
	\choice
	{$\dfrac{x}{3}+\dfrac{y}{12}+\dfrac{z}{9}=1$}
	{\True $12 x+3 y+4 z-48=0$}
	{$\dfrac{x}{4}+\dfrac{y}{16}+\dfrac{z}{12}=0$}
	{$12 x+3 y+4 z=0$}
	\loigiai{
		Mặt phẳng $(P)$ cắt các trục $O x$, $O y$, $O z$ lần lượt tại $A$, $B$, $C$ nên $A(a ; 0 ; 0)$, $B(0 ; b ; 0)$, $C(0 ; 0 ; c)$.
		Vì $G$ là trọng tâm tứ diện $O A B C$ nên $$\heva{&x_G=\dfrac{x_A+x_B+x_C+x_O}{4}=\dfrac{a}{4} \\& y_G=\dfrac{y_A+y_B+y_C+y_O}{4}=\dfrac{b}{4} \\& z_G=\dfrac{z_A+z_B+z_C+z_O}{4}=\dfrac{c}{4}} \Rightarrow\heva{&a=4 \\ &b=16 \\ &c=12.}$$
		Khi đó mặt phẳng $(P)$ có phương trình là $\dfrac{x}{4}+\dfrac{y}{16}+\dfrac{z}{12}=1$ hay $12 x+3 y+4 z-48=0$.\\
		Vậy mặt phẳng $(P)$ thỏa mãn là $12 x+3 y+4 z-48=0$.
	}
\end{ex}

\begin{ex}%[2H5V1-3]
	Viết phương trình mặt phẳng $(\alpha)$ đi qua $M(2 ; 1 ;-3)$, biết $(\alpha)$ cắt trục $O x$, $O y$, $O z$ lần lượt tại $A$, $B$, $C$ sao cho tam giác $A B C$ nhận $M$ làm trực tâm.
	\choice
	{$2 x+5 y+z-6=0$}
	{$2 x+y-6 z-23=0$}
	{\True $2 x+y-3 z-14=0$}
	{ $3 x+4 y+3 z-1=0$}
	\loigiai{
		Giả sử $A(a ; 0 ; 0)$, $B(0 ; b ; 0)$, $C(0 ; 0 ; c)$, $a b c \neq 0$.\\
		Khi đó mặt phẳng $(\alpha)$ có dạng $\dfrac{x}{a}+\dfrac{y}{b}+\dfrac{z}{c}=1$.\\
		Do $M \in(\alpha) \Rightarrow \dfrac{2}{a}+\dfrac{1}{b}-\dfrac{3}{c}=1$.\\
		Ta có $\overrightarrow{A M}=(2-a ; 1 ;-3)$, $\overrightarrow{B M}=(2 ; 1-b ;-3)$, $\overrightarrow{B C}=(0 ;-b ; c)$, $\overrightarrow{A C}=(-a ; 0 ; c)$.\\
		Do $M$ là trực tâm tam giác $A B C$ nên $\heva{&\overrightarrow{A M} \cdot \overrightarrow{B C}=0 \\ &\overrightarrow{B M} \cdot \overrightarrow{A C}=0}\Leftrightarrow\heva{&-b-3 c=0 \\ &-2 a-3 c=0} \Leftrightarrow\heva{&b=-3 c \\ &a=-\dfrac{3 c}{2}.}$\\
		Thay $(2)$ vào $(1)$ ta có $-\dfrac{4}{3 c}-\dfrac{1}{3 c}-\dfrac{3}{c}=1 \Leftrightarrow c=-\dfrac{14}{3} \Rightarrow a=7$, $b=14$.\\
		Do đó $(\alpha)\colon \dfrac{x}{7}+\dfrac{y}{14}-\dfrac{3 z}{14}=1 \Leftrightarrow 2 x+y-3 z-14=0$.
	}
\end{ex}

\begin{ex}%[2H5V1-3]
	Trong không gian với hệ trục toạ độ $Oxyz,$ điểm $M\left(a,b,c\right)$ thuộc mặt phẳng $(P)\colon x+y+z-6=0$ và cách đều các điểm $A\left(1;6;0\right)$, $B\left(-2;2;-1\right)$, $C\left(5;-1;3\right).$ Tích $abc$ bằng
	\choice
	{\True $6$}
	{$-6$}
	{$0$}
	{$5$}
	\loigiai{
		Ta có
		\begin{eqnarray*}
			&\heva{&a+b+c=6\\&MA^2=MB^2\\&MA^2=MC^2}&\Leftrightarrow\heva{&a+b+c=6\\&
				\left(a-1\right)^2+\left(b-6\right)^2+b^2=\left(a+2\right)^2+\left(b-2\right)^2+\left(c+1\right)^2\\&
				\left(a-1\right)^2+\left(b-6\right)^2+c^2=\left(a-5\right)^2+\left(b+1\right)^2+\left(c-3\right)^2}\\
			&	&\Leftrightarrow\heva{&	a+b+c=6\\&3a+4b+c=14\\&4a-7b+3b=-1}\\ &&\Leftrightarrow\heva{&a=1\\&b=2\\&c=3}\Rightarrow abc=6.
	\end{eqnarray*}}
\end{ex}

\begin{ex}%[2H5V1-3]
	Trong không gian với hệ tọa độ $Oxyz,$ cho điểm $M\left(3;2;1\right)$. Mặt phẳng $(P)$ đi qua $M$ và cắt các trục tọa độ $Ox$, $Oy$, $Oz$ lần lượt tại các điểm $A$, $B$, $C$ không trùng với gốc tọa độ sao cho $M$ là trực tâm tam giác $ABC$. Trong các mặt phẳng sau, tìm mặt phẳng song song với mặt phẳng $(P)$.
	\choice
	{\True $3x+2y+z+14=0$}
	{$2x+y+3z+9=0$}
	{$3x+2y+z-14=0$}
	{$2x+y+z-9=0$}
	\loigiai{
		Gọi $A\left(a;0;0\right);B\left(0;b;0\right);C\left(0;0;c\right)$.\\
		Phương trình mặt phẳng $(P)$ có dạng $\dfrac{x}{a}+\dfrac{y}{b}+\dfrac{z}{c}=1$ $\left(abc\ne 0\right)$.\\
		Vì $(P)$ qua $M$ nên $\dfrac{3}{a}+\dfrac{2}{b}+\dfrac{1}{c}=1\quad(1)$.\\
		Ta có $\overrightarrow{MA}=\left(a-3;-2;-1\right)$; $\overrightarrow{MB}=\left(-3;b-2;-1\right)$; $\overrightarrow{BC}=\left(0;-b;c\right)$; $\overrightarrow{AC}=\left(-a;0;c\right)$.\\
		Vì $M$ là trực tâm của tam giác $ABC$ nên $$\heva{&				\overrightarrow{MA}\cdot \overrightarrow{BC}=0\\
			&\overrightarrow{MB}\cdot \overrightarrow{AC}=0}\Leftrightarrow\heva{&		2b=c\\&	3a=c} \quad (2).$$
		Từ $(1)$ và $(2)$ suy ra $a=\dfrac{14}{3}$; $b=\dfrac{14}{2}$; $c=14$.\\
		Khi đó phương trình $(P)\colon 3x+2y+z-14=0$.\\
		Vậy mặt phẳng song song với $(P)$ là $3x+2y+z+14=0$.}
\end{ex}

\begin{ex}%[2H5V1-3]
	Trong không gian với hệ tọa độ $O x y z$, cho các điểm $A(0 ; 1 ; 2)$, $B(2 ;-2 ; 0)$, $C(-2 ; 0 ; 1)$. Mặt phẳng $(P)$ đi qua $A$, trực tâm $H$ của tam giác $A B C$ và vuông góc với mặt phẳng $(A B C)$ có phương trình là
	\choice
	{\True $4 x-2 y-z+4=0$}
	{$4 x-2 y+z+4=0$}
	{$4 x+2 y+z-4=0$}
	{$4 x+2 y-z+4=0$}
	\loigiai{
		Ta có $\overrightarrow{A B}=(2 ;-3 ;-2)$, $\overrightarrow{A C}=(-2 ;-1 ;-1)$ nên $\left[\overrightarrow{A B}, \overrightarrow{A C}\right]=(1 ; 6 ;-8)$.\\
		Phương trình mặt phẳng $(A B C)$ là $x+6 y-8 z+10=0$.\\
		Phương trình mặt phẳng qua $B$ và vuông góc với $A C$ là $2 x+y+z-2=0$.\\
		Phương trình mặt phẳng qua $C$ và vuông góc với $A B$ là $2 x-3 y-2 z+6=0$.\\
		Giao điểm của ba mặt phẳng trên là trực tâm $H$ của tam giác $A B C$ nên $H\left(-\dfrac{22}{101} ; \dfrac{70}{101} ; \dfrac{176}{101}\right)$.\\
		Mặt phẳng $(P)$ đi qua $A$, $H$ nên $\overrightarrow{n_P} \perp \overrightarrow{A H}=\left(-\dfrac{22}{101} ;-\dfrac{31}{101} ;-\dfrac{26}{101}\right)=-\dfrac{1}{101}(22 ; 31 ; 26)$.\\
		Mặt phẳng $(P) \perp(A B C)$ nên $\overrightarrow{n_P} \perp \overrightarrow{n}_{(A B C)}=(1 ; 6 ;-8)$.\\
		Vậy $\left[\overrightarrow{n}_{(A B C)} ; \overrightarrow{u}_{A H}\right]=(404 ;-202 ;-101)$ là một vectơ pháp tuyến của $(P)$.\\
		Chọn $\overrightarrow{n}_P=(4 ;-2 ;-1)$ nên phương trình mặt phẳng $(P)$ là $4 x-2 y-z+4=0$.
	}
\end{ex}

\begin{ex}%[2H5V1-3]
	Trong không gian với hệ tọa độ $O x y z$, viết phương trình mặt phẳng $(P)$ đi qua $A(1 ; 1 ; 1)$ và $B(0 ; 2 ; 2)$ đồng thời cắt các tia $O x$, $O y$ lần lượt tại hai điểm $M$, $N$ ( không trùng với gốc tọa độ $O$ ) sao cho $O M=2 O N$.
	\choice
	{$(P)\colon 3x+y+2z-6=0$}
	{$(P)\colon 2x+3y-z-4=0$}
	{$(P)\colon 2x+y+z-4=0$}
	{\True $(P)\colon x+2 y-z-2=0$}
	\loigiai{
		Giả sử $(P)$ đi qua 3 điểm $M(a ; 0 ; 0)$, $N(0 ; b ; 0)$, $P(0 ; 0 ; c)$.\\
		Suy ra $(P)\colon \dfrac{x}{a}+\dfrac{y}{b}+\dfrac{z}{c}=1$.\\
		Mà $(P)$ đi qua $A(1 ; 1 ; 1)$ và $B(0 ; 2 ; 2)$ nên ta có hệ $\heva{&\dfrac{1}{a}+\dfrac{1}{b}+\dfrac{1}{c}=1 \\ &\dfrac{2}{b}+\dfrac{2}{c}=1} \Leftrightarrow\heva{&a=2 \\ &\dfrac{2}{b}+\dfrac{2}{c}=1.}$\\
		Theo giả thuyết ta có $O M=2 O N \Leftrightarrow|a|=2|b| \Leftrightarrow|b|=1$.
		\begin{itemize}
			\item \textbf{TH1.} $b=1 \Rightarrow c=-2$ suy ra $(P)\colon x+2 y-z-2=0$.
			\item \textbf{TH2.} $b=-1 \Rightarrow c=-\dfrac{2}{3}$ suy ra $(P)\colon x-2 y+3 z-2=0$.
		\end{itemize}
	}
\end{ex}

\begin{ex}%[2H5V1-3]
	Trong không gian $O x y z$, cho mặt phẳng $(\alpha)$ đi qua điểm $M(1 ; 2 ; 3)$ và cắt các trục $O x$, $O y$, $O z$ lần lượt tại $A$, $B$, $C$ (khác gốc tọa độ $O$ ) sao cho $M$ là trực tâm tam giác $A B C$. Mặt phẳng $(\alpha)$ có phương trình dạng $a x+b y+c z-14=0$. Tính tổng $T=a+b+c$.
	\choice
	{$8$}
	{$14$}
	{\True $6$}
	{$11$}
	\loigiai{
		Do $M$ là trực tâm tam giác $ABC$, nên ta có
		\begin{itemize}
			\item $OA \perp BC$ và $AM \perp BC$ nên $(OAM) \perp BC \Rightarrow OM \perp BC$.
			\item $OB \perp AC$ và $BM \perp AC$ nên $(OBM) \perp AC \Rightarrow OM \perp AC$.
		\end{itemize}
		Từ đó ta được $OM \perp (ABC)$ nên $\overrightarrow{OM}=(1;2;3)$ là vectơ pháp tuyến của $(ABC)$.\\
		Vậy phương trình mặt phẳng $(ABC)$ là $$1\cdot (x-1)+2\cdot (y-2)+3\cdot (x-3)=0 \Leftrightarrow x+2y+3z-14=0.$$
		Dẫn đến $a=1$, $b=2$, $c=3$ nên $T=1+2+3=6$.
	}
\end{ex}

\begin{ex}%[2H5V1-3]
	Trong không gian $O x y z$, cho hai mặt phẳng $(P)\colon x+4 y-2 z-6=0$, $(Q)\colon x-2 y+4 z-6=0$. Mặt phẳng $(\alpha)$ chứa giao tuyến của $(P)$, $(Q)$ và cắt các trục tọa độ tại các điểm $A$, $B$, $C$ sao cho hình chóp $O. A B C$ là hình chóp đều. Phương trình mặt phẳng $(\alpha)$ là
	\choice
	{\True $x+y+z-6=0$}
	{$x+y+z+6=0$}
	{$x+y+z-3=0$}
	{$x+y-z-6=0$}
	\loigiai{
		Mặt phẳng $(P)\colon x+4 y-2 z-6=0$ có vectơ pháp tuyến $\overrightarrow{n_P}=(1 ; 4 ;-2)$.\\
		Mặt phẳng $(Q)\colon x-2 y+4 z-6=0$ có vectơ pháp tuyến $\overrightarrow{n_Q}=(1 ;-2 ; 4)$.\\
		Ta có $\left[\overrightarrow{n}_P ; \overrightarrow{n}_Q\right]=(12 ;-6 ;-6)$, cùng phương với $\overrightarrow{u}=(2 ;-1 ;-1)$.\\
		Gọi $d=(P) \cap(Q)$. Ta có đường thẳng $d$ có vectơ chỉ phương là $\overrightarrow{u}=(2 ;-1 ;-1)$ và đi qua điểm $M(6 ; 0 ; 0)$.\\
		Mặt phẳng $(\alpha)$ cắt các trục tọa độ tại các điểm $A(a ; 0 ; 0)$, $B(0 ; b ; 0)$, $C(0 ; 0 ; c)$ với $a b c \neq 0$.\\
		Phương trình mặt phẳng $(\alpha)\colon \dfrac{x}{a}+\dfrac{y}{b}+\dfrac{z}{c}=1$.\\
		Mặt phẳng $(\alpha)$ có vectơ pháp tuyến $\vec{n}=\left(\dfrac{1}{a} ; \dfrac{1}{b} ; \dfrac{1}{c}\right)$.\\
		Mặt phẳng $(\alpha)$ chứa $d$ nên $$\heva{&\vec{n} \perp \vec{u} \\ &M \in(\alpha)} \Leftrightarrow\heva{&\dfrac{2}{a}-\dfrac{1}{b}-\dfrac{1}{c}=0 \\ &\dfrac{6}{a}=1} \Leftrightarrow\heva{&a=6 \\ &\dfrac{1}{b}+\dfrac{1}{c}=\dfrac{1}{3}.\quad (*)}$$
		Ta lại có hình chóp $O.ABC$ là hình chóp đều $$\Leftrightarrow O A=O B=O C \Leftrightarrow|a|=|b|=|c| \Leftrightarrow|b|=|c|=6.$$
		Kêt hợp với điều kiện $(*)$ ta được $b=c=6$.\\
		Vậy phương trình của mặt phẳng $(\alpha)\colon \dfrac{x}{6}+\dfrac{y}{6}+\dfrac{z}{6}=1 \Leftrightarrow x+y+z-6=0$.
	}
\end{ex}
\begin{ex}%[2H5V1-3]
	Trong không gian tọa độ $O x y z$, cho mặt phẳng $(\alpha)$ đi qua $M(1 ;-3 ; 8)$ và chắn trên $O z$ một đoạn dài gấp đôi các đoạn chắn trên các tia $O x$, $O y$. Giả sử $(\alpha)\colon a x+b y+c z+d=0$ ($a$, $b$, $c$, $d$ là các số nguyên). Tính $S=\dfrac{a+b+c}{d}$.
	\choice
	{$3$}
	{$-3$}
	{$\dfrac{5}{4}$}
	{\True $-\dfrac{5}{4}$}
	\loigiai{
		Giả sử mặt phẳng $(\alpha)$ cắt các tia $Ox$, $Oy$, $Oz$ lần lượt tại $A(m ; 0 ; 0)$, $B(0 ; n ; 0)$, $C(0 ; 0 ; p)$ (với $m$, $n$, $p>0$).\\
		Theo giả thiết có $O C=2 O A=2 O B \Rightarrow p=2 m=2 n. \quad(1)$\\
		Phương trình mặt phẳng $(\alpha)$ có dạng $\dfrac{x}{m}+\dfrac{y}{n}+\dfrac{z}{p}=1. \quad (2)$\\
		Do mặt phẳng $(\alpha)$ đi qua $M(1 ;-3 ; 8)$ nên $\dfrac{1}{m}-\dfrac{3}{n}+\dfrac{8}{p}=1$.\\
		Thay $(1)$ vào $(2)$ ta được $\dfrac{1}{m}-\dfrac{3}{m}+\dfrac{8}{2 m}=1 \Leftrightarrow \dfrac{2}{m}=1 \Leftrightarrow m=2 \Rightarrow m=n=2,\ p=4$.
		Phương trình mặt phẳng $(\alpha)$ có dạng $\dfrac{x}{2}+\dfrac{y}{2}+\dfrac{z}{4}=1 \Leftrightarrow 2 x+2 y+z-4=0$.\\
		Từ đó suy ra $a=2 t$, $b=2 t$, $c=t$, $d=-4 t \quad(t \neq 0)$.\\
		Vậy $S=\dfrac{a+b+c}{d}=-\dfrac{5}{4}$.
	}
\end{ex}
\Closesolutionfile{ans}
\indapan{10}{ans/ans-2-C5B1CD2-D4}

\Opensolutionfile{ans}[ans/ans-2-C5B1CD2-D4]
\TNSA
\begin{ex}%[2H5V1-3]
	Trong không gian với hệ trục tọa độ $O x y z$, cho hai điểm $A(3 ; 1 ; 7)$, $B(5 ; 5 ; 1)$ và mặt phẳng $(P)\colon 2 x-y-z+4=0$. Điểm $M$ thuộc $(P)$ sao cho $M A=M B=\sqrt{35}$. Biết $M$ có hoành độ nguyên, tính $O M$ (làm tròn đến chữ số hàng phần trăm).
	\shortans{$2{,}83$}
	\loigiai{
		Gọi $M(a ; b ; c)$ với $a \in \mathbb{Z}$, $b \in \mathbb{R}$, $c \in \mathbb{R}$.\\
		Ta có $\overrightarrow{A M}=(a-3 ; b-1 ; c-7)$ và $\overrightarrow{B M}=(a-5 ; b-5 ; c-1)$.\\
		Vì $\heva{&M \in ( P )\\&M A = M B = \sqrt { 3 5 }}\Leftrightarrow \heva{&M \in(P) \\&M A^2=M B^2\\&M A^2=35}$ nên ta có hệ phương trình sau
		\begin{eqnarray*}
			\allowdisplaybreaks
			& &\heva{&2a - b - c + 4 = 0\\&(a-3)^ {2} + ( b - 1) ^ {2} + (c-7)^{2} = (a-5)^{2} + ( b - 5 ) ^ { 2 } + ( c - 1 ) ^ { 2 }\\&( a - 3 ) ^ { 2 } + ( b - 1 ) ^ { 2 } + ( c - 7 ) ^ { 2 } = 3 5 }\\
			&\Leftrightarrow &\heva{&2 a-b-c=-4 \\&4 a+8 b-12 c=-8 \\&(a-3)^2+(b-1)^2+(c-7)^2=35}\\
			&\Leftrightarrow &\heva{&b=c\\&c=a+2 \\&(a-3)^2+(b-1)^2+(c-7)^2=35}\Leftrightarrow \heva{&b=a+2 \\&c=a+2 \\&3a^2-14=0}
			\Leftrightarrow \heva{&a=0 \\&b=2 \ (\text{do }a \in \mathbb{Z})\\&c=2.}
		\end{eqnarray*}
		Ta có $M(2 ; 2 ; 0)$. Suy ra $O M=2 \sqrt{2}\approx 2{,}83$.\\
	}
\end{ex}
\begin{ex}%[2H5V1-3]
	Trong không gian với hệ tọa độ $O x y z$, mặt phẳng $(P)$ chứa điểm $M(1 ; 3 ;-2)$, cắt các tia $O x$, $O y$, $O z$ lần lượt tại $A$, $B$, $C$ sao cho $\dfrac{O A}{1}=\dfrac{O B}{2}=\dfrac{O C}{4}$. Biết phương trình mặt phẳng $(P)$ có dạng $ax+by+cz-8=0$. Tính $P=\dfrac{a+c}{2b}$ (kết quả được viết dưới dạng số thập phân).
	\shortans{$1{,}25$}
	\loigiai{
		Phương trình mặt chắn cắt tia $O x$ tại $A(a ; 0 ; 0)$, cắt tia $O y$ tại $B(0 ; b ; 0)$, cắt tia $O z$ tại $C(0 ; 0 ; c)$ có dạng là $(P)\colon \dfrac{x}{a}+\dfrac{y}{b}+\dfrac{z}{c}=1$ (với $a>0, b>0, c>0$).\\
		Theo đề $\dfrac{O A}{1}=\dfrac{O B}{2}=\dfrac{O C}{4} \Leftrightarrow \dfrac{a}{1}=\dfrac{b}{2}=\dfrac{c}{4} \Rightarrow\heva{&a=\dfrac{b}{2} \\ &c=2 b.}$\\
		Vì $M(1 ; 3 ;-2)$ nằm trên mặt phẳng $(P)$ nên ta có $$\dfrac{1}{\frac{b}{2}}+\dfrac{3}{b}+\dfrac{-2}{2 b}=1 \Leftrightarrow \dfrac{4}{b}=1 \Leftrightarrow b=4.$$
		Khi đó $a=2$, $c=8$.\\
		Vậy phương trình mặt phẳng $(P)$ là $\dfrac{x}{2}+\dfrac{y}{4}+\dfrac{z}{8}=1 \Leftrightarrow 4 x+2 y+z-8=0$.\\
		Khi đó $=\dfrac{a+c}{2b}=\dfrac{4+1}{2\cdot 2}=1{,}25$
	}
\end{ex}
\begin{ex}%[2H5V1-3]
	Trong không gian với hệ tọa độ $O x y z$ cho mặt phẳng $(P)$ đi qua điểm $M(9 ; 1 ; 1)$ cắt các tia $O x$, $O y$, $O z$ tại $A$, $B$, $C$ ($A$, $B$, $C$ không trùng với gốc tọa độ ). Thể tích tứ diện $O A B C$ đạt giá trị nhỏ nhất là bao nhiêu  (kết quả được viết dưới dạng số thập phân)?
	\shortans{$40{,}5$}
	\loigiai{
		Giả sử $A(a ; 0 ; 0)$, $B(0 ; b ; 0)$, $C(0 ; 0 ; c)$ với $a$, $b$, $c>0$.\\
		Mặt phẳng $(P)$ có phương trình ( theo đoạn chắn) $$\dfrac{x}{a}+\dfrac{y}{b}+\dfrac{z}{c}=1.$$
		Vì mặt phẳng $(P)$ đi qua điểm $M(9 ; 1 ; 1)$ nên $$\dfrac{9}{a}+\dfrac{1}{b}+\dfrac{1}{c}=1.$$
		Ta có $1=\dfrac{9}{a}+\dfrac{1}{b}+\dfrac{1}{c} \geq 3 \sqrt[3]{\dfrac{9}{abc}} \Rightarrow abc\geq 243$.\\
		$$
		V_{O A B C}=\dfrac{1}{6} abc \geq \dfrac{243}{6}=\dfrac{81}{2}.$$
		Vậy thể tích tứ diện $O A B C$ đạt giá trị nhỏ nhất là $\dfrac{81}{2}=40{,}5$.
	}
\end{ex}

%Câu 70
\begin{ex}%[2H5H1-3]
	Trong không gian với hệ trục tọa độ $Oxyz$, cho ba điểm $A(a;0;0)$, $B(0;b;0)$, $C(0;0;c)$ với $a$, $b$, $c$ là ba số thực dương thay đổi, thỏa mãn điều kiện $\dfrac{1}{a}+\dfrac{1}{b}+\dfrac{1}{c}=2017$. Khi đó, mặt phẳng $(ABC)$ luôn đi qua một điểm cố định có tọa độ là $M(m;m;m)$. Tính giá trị $P=2017m+2$.
	\shortans[\kindSA]{$3$}	
	\loigiai{
		Phương trình mặt phẳng đi qua ba điểm $A(a;0;0)$, $B(0;b;0)$, $C(0;0;c)$ có dạng  $$(ABC) \colon \dfrac{x}{a}+\dfrac{y}{b}+\dfrac{z}{c}=1.$$
		Giả sử $M(m;m;m)$ là một điểm cố định nằm trên $(ABC)$. Khi đó ta có $$ M \in (ABC) \Leftrightarrow \dfrac{m}{a}+\dfrac{m}{b}+\dfrac{m}{c}=1 \Leftrightarrow m \left(\dfrac{1}{a}+\dfrac{1}{b}+\dfrac{1}{c}\right)=1 \Leftrightarrow m \cdot 2017 =1 \Leftrightarrow m =\dfrac{1}{2017}.$$
		Vậy $P=2017m+2=2017 \cdot \dfrac{1}{2017}+2=3$.
	}
\end{ex}
%Câu 71
\begin{ex}%[2H5V1-3]
	Trong không gian với hệ trục tọa độ $Oxyz$, cho ba điểm $M(1;2;5)$. Tính số mặt phẳng $(\alpha)$ đi qua $M$ và cắt các trục $Ox$, $Oy$, $Oz$ lần lượt tại $A$, $B$, $C$ sao cho $OA=OB=OC \neq 0$.
	\shortans[\kindSA]{$4$}	
	\loigiai{
		Gọi $A(a;0;0)$, $B(0;b;b)$, $C(0;0;c)$ lần lượt là giao điểm của mặt phẳng $(\alpha)$ với các trục $Ox$, $Oy$ và $Oz$ (với $abc \neq 0$). \\
		Khi đó $(\alpha) \equiv (ABC) \colon \dfrac{x}{a}+\dfrac{y}{b}+\dfrac{z}{c}=1$.\\
		Ta có $OA=\sqrt{a^2+0^2+0^2}=|a|$. Tương tự $OB=|b|$, $OC=|c|$.\\
		Vì $OA=OB=OC$ nên $\heva{&OA=OB\\ &OC=OB} \Leftrightarrow \heva{&|a|=|b|\\&|c|=|b|} \Leftrightarrow \heva{&a= \pm b \\ &c = \pm b.}$
		\begin{itemize}
			\item Trường hợp $1$: $a=b$, $c=b$. \\
			Khi đó $\dfrac{x}{b}+\dfrac{y}{b}+\dfrac{z}{b}=1$ mà $M(1;2;5) \in (ABC)$ nên $\dfrac{1}{b}+\dfrac{2}{b}+\dfrac{5}{b}=1 \Leftrightarrow b=8.$
			\item Trường hợp $2$: $a=b$, $c=-b$. \\
			Khi đó $\dfrac{x}{b}+\dfrac{y}{b}-\dfrac{z}{b}=1$ mà $M(1;2;5) \in (ABC)$ nên $\dfrac{1}{b}+\dfrac{2}{b}-\dfrac{5}{b}=1 \Leftrightarrow b=-2.$
			\item Trường hợp $3$: $a=-b$, $c=b$. \\
			Khi đó $\dfrac{x}{b}-\dfrac{y}{b}+\dfrac{z}{b}=1$ mà $M(1;2;5) \in (ABC)$ nên $\dfrac{1}{b}-\dfrac{2}{b}+\dfrac{5}{b}=1 \Leftrightarrow b=4.$
			\item Trường hợp $4$: $a=-b$, $c=-b$. \\
			Khi đó $\dfrac{x}{b}-\dfrac{y}{b}+\dfrac{z}{b}=1$ mà $M(1;2;5) \in (ABC)$ nên $\dfrac{1}{b}-\dfrac{2}{b}-\dfrac{5}{b}=1 \Leftrightarrow b=-6.$
		\end{itemize}
		Vậy có bốn mặt phẳng $(\alpha)$ thỏa yêu cầu bài toán.
	}
\end{ex}
\begin{ex}%[2H5V1-3]
	Trong không gian với hệ trục tọa độ $Oxyz$, có bao nhiêu mặt phẳng $(P)$ đi qua ba điểm $M(2;1;3)$, $A(0;0;4)$ và cắt hai trục $Ox$, $Oy$ lần lượt tại $B$, $C$ khác $O$ thỏa mãn diện tích tam giác $OBC$ bằng $1$?	
	\shortans[\kindSA]{$2$}	
	\loigiai{
		Gọi $B(b;0;0)$ và $C(0;c;0)$ lần lượt là giao điểm của $(P)$ với các trục $Ox$, $Oy$.\\
		Khi đó ta có phương trình mặt phẳng $(P) \colon \dfrac{x}{b}+\dfrac{y}{c}+\dfrac{z}{4}=1$.\\
		Vì $M(2;1;3) \in (P)$ nên ta có $\dfrac{2}{b}+\dfrac{1}{c}+\dfrac{3}{4}=1 \Leftrightarrow \dfrac{2}{b}+\dfrac{1}{c} = \dfrac{1}{4} \Leftrightarrow 4b+8c=bc$. \quad (1)\\
		Diện tích tam giác $OBC$ bằng $1$ nên $\dfrac{1}{2} \cdot OB \cdot OC =1 \Leftrightarrow |b| \cdot |c|=2 \Leftrightarrow |bc|=2.$ \quad (2)\\
		Từ (1) và (2), ta có hệ phương trình $\heva{&4b+8c=bc\\&|bc|=2.} \quad (I)$\\
		\begin{itemize}
			\item Xét trường hợp $bc>0$. \\
			Khi đó
			$$(I) \Leftrightarrow \heva{&4b+8c=bc\\&bc=2} \Leftrightarrow \heva{&4a+8b=2\\&2bc=4} \Leftrightarrow \heva{&2b=1-4c\\ &(1-4c)c=4}
			\Leftrightarrow \heva{&2b=1-4c\\ &4c^2-c+4=0  \; (\text{pt vô nghiệm}).}$$
			\item Xét trường hợp $bc<0$. \\
			Khi đó
			\begin{align*}
				(I) \Leftrightarrow \heva{&4b+8c=bc\\&bc=-2} \Leftrightarrow \heva{&4a+8b=2\\&2bc=-4} &\Leftrightarrow  \heva{&2b=1-4c\\ &(1-4c)c=-4}\\
				&\Leftrightarrow \heva{&2b=1-4c\\ &4c^2-c-4=0.} \\
				&\Leftrightarrow \heva {&2b=1-4c\\&c=\dfrac{1 \pm \sqrt{65}}{2}}\\
				&\Leftrightarrow \heva{&c=\dfrac{1 + \sqrt{65}}{2} \\ &b=\dfrac{-1-2\sqrt{65}}{2}} \; \text{hay} \;\heva{&c=\dfrac{1 - \sqrt{65}}{2} \\ &b=\dfrac{-1+2\sqrt{65}}{2}.} 
			\end{align*}
		\end{itemize}
		Vậy có $2$ cặp số $(b;c)$ thỏa yêu cầu bài toán nên có $2$ mặt phẳng $(P)$ thỏa yêu cầu bài tán.
	}
\end{ex}
\Closesolutionfile{ans}
\indapan{6}{ans/ans-2-C5B1CD2-D4}
% \begin{dang}{Bài toán thực tế}
Gắn hệ trục toạ độ vào mô hình. Đặt gốc toạ độ tại vị trí có "3 góc vuông"
% \newcommand{\gv}[4][black]{\draw[#1,thick] ($(#3)!8pt!(#2)$)--($(#3)!2!($($(#3)!8pt!(#2)$)!.5!($(#3)!8pt!(#4)$)$)$)--($(#3)!8pt!(#4)$);}
% \begin{longtable}{|>{\raggedright\arraybackslash}p{5.2cm}|>{\raggedright\arraybackslash}p{5.4cm}|>{\raggedright\arraybackslash}p{5.7cm}|}
% 		\hline
% 	    \multicolumn{3}{|>{\centering\arraybackslash}p{16.5cm}|}{\textbf{I. Gắn trục tọa độ đối với hình chóp}} \\ \hline    
% 	    \multicolumn{3}{|>{\centering\arraybackslash}p{16.5cm}|}{\textbf{1. Hình chóp có cạnh bên (SA) vuông góc với mặt đáy}} \\ \hline                                                                                                                                                                               
% 		\multicolumn{1}{|>{\raggedright\arraybackslash}p{5.2cm}|}{\begin{tabular}[l]{>{\raggedright\arraybackslash}p{5.2cm}} \textbf{Đáy là tam giác đều}
% 							\begin{tikzpicture}[>=stealth,font=\footnotesize]
% 							\def\a{3}
% 							\def\b{2}
% 							\def\h{2}
% 							\path (0:0) coordinate (A)
% 							++(0:\a) coordinate (C)
% 							++(-130:\b) coordinate (B)
% 							($(A)+(90:\h)$) coordinate (S)
% 							($(B)!1/2!(C)$) coordinate (O)
% 							($(O)+(90:3.5)$) coordinate (O1)
% 							($(S)+(O)-(A)$) coordinate (H);
% 							\draw[dashed,thick] (A)--(C);
% 							\draw[thick] (S)--(A)--(B)--(C)--(S)--(B);
% 							\draw[dashed,thick](A)--(O);
% 							\draw[thick](S)--(H);
% 							%Ve truc Ox,Oy, Oz
% 							\draw[thick,->](C)--($(O)!2!(C)$) node [pos=0.9,above ]{$x$};
% 							\draw[thick,->](A)--($(O)!1.2!(A)$) node [pos=0.9,above right]{$y$};
% 							\draw[thick,->](O)--(O1) node [pos=0.9,above right]{$z$};
% 							%Các góc vuông
% 							\gv{S}{H}{O}
% 							\gv{C}{O}{H}
% 							\gv{A}{O}{H}
% 							\gv{O}{A}{S}
% 							\foreach \x/\g in {A/-90,B/0,C/0,S/180,O/-10,H/-10}
% 							\fill[black] (\x) circle (1pt) ($(\g:4mm)+(\x)$) node {$\x$};	
% 						\end{tikzpicture}
				 
% 				 - Gọi $O$ là trung điểm $BC$. Chọn hệ trục tọa độ như hình vẽ, $AB=a=1$.
				
% 				- Tọa độ các điểm là:
						
% 						$O(0;0;0)$, $A \left(0;\dfrac{\sqrt{3}}{2};0\right)$, $B \left(\dfrac{-1}{2};0;0\right)$, $C \left(\dfrac{1}{2};0;0\right)$, $S \left(0;\dfrac{\sqrt{3}}{2};\underbrace {OH}_{ = SA}\right)$.
% 				\end{tabular}} &\multicolumn{1}{l|}{\begin{tabular}[l]{>{\raggedright\arraybackslash}p{5.2cm}}\textbf{Đáy là tam giác cân tại A}
				
% 					\begin{tikzpicture}[>=stealth,font=\footnotesize]
% 						\def\a{3.5}
% 						\def\b{2.5}
% 						\def\h{3.5}
% 						\path (0:0) coordinate (B)
% 						++(0:\a) coordinate (C)
% 						++(-150:\b) coordinate (A)
% 						($(B)!1/2!(C)$) coordinate (O)
% 						($(A)+(90:\h)$) coordinate (S)
% 						($(O)+(90:3.8)$) coordinate (O1)
% 						($(S)+(O)-(A)$) coordinate (H);
% 						\draw[dashed,thick] (B)--(C);
% 						\draw[thick] (S)--(B)--(A)--(C)--(S)--(A);
% 						\draw[dashed,thick](A)--(O);
% 						\draw[thick](S)--(H);
% 						%Ve truc Ox,Oy, Oz
% 						\draw[thick,->](C)--($(O)!1.4!(C)$) node [pos=0.9,below]{$x$};
% 						\draw[thick,->](A)--($(O)!1.4!(A)$) node [pos=0.9,below left]{$y$};
% 						\draw[thick,->](O)--(O1) node [pos=0.9,above right]{$z$};
% 						%Các góc vuông
% 						\gv{B}{A}{S}
% 						\gv{A}{O}{C}
% 						\foreach \x/\g in {A/-20,B/120,C/-50,S/180,O/40,H/180}
% 						\fill[black] (\x) circle (1pt) ($(\g:4mm)+(\x)$) node {$\x$};	
% 					\end{tikzpicture}
					
% 			- Gọi $O$ là trung điểm $BC$. Chọn hệ trục tọa độ như hình vẽ, $a=1$.
			
% 			- Tọa độ các điểm là:
				
% 				$O(0;0;0)$, $A \left(0;OA;0\right)$, $B \left(-OB;0;0\right)$, $C \left(OC;0;0\right)$, $S \left(0;OA;\underbrace {OH}_{ = SA}\right)$.
% 			\end{tabular}} & \begin{tabular}[l]{>{\raggedright\arraybackslash}p{5.6cm}}\textbf{ Đáy là tam giác cân tại B}
			
% 				\begin{tikzpicture}[>=stealth,font=\footnotesize]
% 					\def\a{3}
% 					\def\b{2}
% 					\def\h{2}
% 					\path (0:0) coordinate (A)
% 					++(0:\a) coordinate (C)
% 					++(-150:\b) coordinate (B)
% 					($(A)!1/2!(C)$) coordinate (O)
% 					($(A)+(90:\h)$) coordinate (S)
% 					($(O)+(90:2.5)$) coordinate (O1)
% 					($(S)+(O)-(A)$) coordinate (H);
% 					\draw[dashed,thick] (A)--(C);
% 					\draw[thick] (S)--(A)--(B)--(C)--(S)--(B);
% 					\draw[dashed,thick](B)--(O) ;
% 					\draw[thick](S)--(H);
% 					%Ve truc Ox,Oy, Oz
% 					\draw[thick,->](C)--($(O)!1.4!(C)$) node [pos=0.9,below]{$x$};
% 					\draw[thick,->](B)--($(O)!1.2!(B)$) node [pos=0.9,below left]{$y$};
% 					\draw[dashed,thick](O)--($(O)!1/2!(H)$);
% 					\draw[thick,->]($(O)!1/2!(H)$)--(O1) node [pos=0.9,above right]{$z$};
% 					%Các góc vuông
% 					\gv{S}{A}{C}
% 					\gv{B}{O}{C}
% 					\foreach \x/\g in {B/-20,A/120,C/-50,S/180,O/40,H/-10}
% 					\fill[black] (\x) circle (1pt) ($(\g:4mm)+(\x)$) node {$\x$};	
% 				\end{tikzpicture}
				
% 			- Gọi $O$ là trung điểm $BC$. Chọn hệ trục tọa độ như hình vẽ, $AB=a=1$.
			
% 			- Tọa độ các điểm là:
			
% 			$O(0;0;0)$, $A \left(\dfrac{-1}{2};0;0\right)$, $B \left(0;\dfrac{\sqrt{3}}{2};0\right)$, $C \left(\dfrac{1}{2};0;0\right)$, $S \left(0;\dfrac{\sqrt{3}}{2};\underbrace {OH}_{ = SA}\right)$.
% 	\end{tabular} \\ \hline
% 	\multicolumn{1}{|>{\raggedright\arraybackslash}p{5.2cm}|}{\begin{tabular}[l]{>{\raggedright\arraybackslash}p{5.2cm}} \textbf{Đáy là tam giác vuông tại $B$}
			
% 			\begin{tikzpicture}[>=stealth,font=\footnotesize,scale=1]
% 				\def\a{4}
% 				\def\b{3}
% 				\def\h{2.6}
% 				\path (0:0) coordinate (A)
% 				++(0:\a) coordinate (C)
% 				++(-150:\b) coordinate (B)
% 				($(C)!1.01!(B)$) coordinate (O)
% 				($(A)+(90:\h)$) coordinate (S)
% 				($(O)+(90:3.5)$) coordinate (O1)
% 				($(S)+(O)-(A)$) coordinate (H);
% 				\draw[dashed,thick] (A)--(C);
% 				\draw[thick] (S)--(A)--(B)--(C)--(S)--(B);
% 				\draw[thick](S)--(H);
% 				%Ve truc Ox,Oy, Oz
% 				\draw[thick,->](C)--($(O)!1.1!(C)$) node [pos=0.9,above ]{$x$};
% 				\draw[thick,->](A)--($(O)!1.2!(A)$) node [pos=0.9, above]{$y$};
% 				\draw[thick,->](O)--(O1) node [above]{$z$};
% 				%Các góc vuông
% 				\gv{H}{O}{A}
% 				\gv{C}{B}{A}
% 				\gv{C}{A}{S}
% 				\gv{S}{H}{B}
% 				\foreach \x/\g in {A/-90,B/0,C/-40,S/90,O/-110,H/-10}
% 				\fill[black] (\x) circle (1pt) ($(\g:4mm)+(\x)$) node {$\x$};	
% 			\end{tikzpicture}
% 	\end{tabular}} &\multicolumn{1}{l|}{\begin{tabular}[l]{>{\raggedright\arraybackslash}p{5.2cm}}\textbf{Đáy là tam giác vuông tại $A$}
	
% 	\begin{tikzpicture}[>=stealth,font=\footnotesize,scale=1]
% 		\def\a{3.5}
% 		\def\b{4}
% 		\def\h{3}
% 		\path (0:0) coordinate (O)
% 		++(0:\a) coordinate (C)
% 		++(-165:\b) coordinate (B)
% 		($(A)+(90:\h)$) coordinate (S)
% 		($(O)+(90:3.3)$) coordinate (O1)
% 		($(C)!1!(O)$) coordinate (A);
% 		\draw[dashed,thick] (B)--(A)--(C) (O)--(S);
% 		\draw[thick] (S)--(B)--(C)--(S);
% 		%Ve truc Ox,Oy, Oz
% 		\draw[thick,->](C)--($(O)!1.2!(C)$) node [pos=0.9,below]{$x$};
% 		\draw[thick,->](B)--($(O)!1.25!(B)$) node [pos=0.9,right]{$y$};
% 		\draw[thick,->](S)--(O1) node [pos=0.9,above right]{$z$};
% 		%Các góc vuông
% 		\gv{B}{O}{C}
% 		\gv{A}{O}{C}
% 		\foreach \x/\g in {A/-45,B/120,C/-50,S/180,O/145}
% 		\fill[black] (\x) circle (1pt) ($(\g:4mm)+(\x)$) node {$\x$};	
% 	\end{tikzpicture}
% 	\end{tabular}} & \begin{tabular}[l]{>{\raggedright\arraybackslash}p{5.6cm}}\textbf{Đáy là tam giác thường}
% 	\begin{tikzpicture}[>=stealth,font=\footnotesize,scale=1]
% 	\def\a{3.8}
% 	\def\b{3}
% 	\def\h{2}
% 	\path (0:0) coordinate (A)
% 	++(0:\a) coordinate (C)
% 	++(-150:\b) coordinate (B)
% 	($(A)!1/2!(C)$) coordinate (O)
% 	($(A)+(90:\h)$) coordinate (S)
% 	($(O)+(90:2.5)$) coordinate (O1)
% 	($(S)+(O)-(A)$) coordinate (H);
% 	\draw[dashed,thick] (A)--(C);
% 	\draw[thick] (S)--(A)--(B)--(C)--(S)--(B);
% 	\draw[dashed,thick](B)--(O) ;
% 	\draw[thick](S)--(H);
% 	%Ve truc Ox,Oy, Oz
% 	\draw[thick,->](C)--($(O)!1.4!(C)$) node [pos=0.9,below]{$x$};
% 	\draw[thick,->](B)--($(O)!1.2!(B)$) node [pos=0.9,below left]{$y$};
% 	\draw[dashed,thick](O)--($(O)!1/2!(H)$);
% 	\draw[thick,->]($(O)!1/2!(H)$)--(O1) node [pos=0.9,above right]{$z$};
% 	%Các góc vuông
% 	\gv{S}{A}{C}
% 	\gv{B}{O}{C}
% 	\foreach \x/\g in {B/-20,A/120,C/-50,S/180,O/40,H/-10}
% 	\fill[black] (\x) circle (1pt) ($(\g:4mm)+(\x)$) node {$\x$};	
% 	\end{tikzpicture}
% 	\end{tabular} \\ \hline
% 	\end{longtable}
% \begin{longtable}{|>{\raggedright\arraybackslash}p{5.4cm}|>{\raggedright\arraybackslash}p{5.2cm}|>{\raggedright\arraybackslash}p{5.8cm}|}
% 	\hline
% 	{\begin{tabular}[l]{>{\raggedright\arraybackslash}p{4.8cm}} \textbf{Đáy là tam giác vuông tại $B$}
			
% 			\begin{tikzpicture}[>=stealth,font=\footnotesize,scale=1]
% 				\def\a{4}
% 				\def\b{3}
% 				\def\h{2.6}
% 				\path (0:0) coordinate (A)
% 				++(0:\a) coordinate (C)
% 				++(-150:\b) coordinate (B)
% 				($(C)!1.01!(B)$) coordinate (O)
% 				($(A)+(90:\h)$) coordinate (S)
% 				($(O)+(90:3.5)$) coordinate (O1)
% 				($(S)+(O)-(A)$) coordinate (H);
% 				\draw[dashed,thick] (A)--(C);
% 				\draw[thick] (S)--(A)--(B)--(C)--(S)--(B);
% 				\draw[thick](S)--(H);
% 				%Ve truc Ox,Oy, Oz
% 				\draw[thick,->](C)--($(O)!1.1!(C)$) node [pos=0.9,above ]{$x$};
% 				\draw[thick,->](A)--($(O)!1.2!(A)$) node [pos=0.9, above]{$y$};
% 				\draw[thick,->](O)--(O1) node [above]{$z$};
% 				%Các góc vuông
% 				\gv{H}{O}{A}
% 				\gv{C}{B}{A}
% 				\gv{C}{A}{S}
% 				\gv{S}{H}{B}
% 				\foreach \x/\g in {A/-90,B/0,C/-40,S/90,O/-110,H/-10}
% 				\fill[black] (\x) circle (1pt) ($(\g:4mm)+(\x)$) node {$\x$};	
% 			\end{tikzpicture}
		
% 			- Chọn hệ trục tọa độ như hình vẽ, $a=1$.
			
% 			- Tọa độ các điểm là:
				
% 				$B \equiv O(0;0;0)$, $A \left(0;AB;0\right)$, $C \left(BC;0;0\right)$, $S \left(0;AB;\underbrace {BH}_{ = SA}\right)$.
	
% 	\end{tabular}}&{\begin{tabular}[l]{>{\raggedright\arraybackslash}p{5cm}}\textbf{Đáy là tam giác vuông tại $A$}
			
% 			\begin{tikzpicture}[>=stealth,font=\footnotesize,scale=1]
% 				\def\a{3.5}
% 				\def\b{4}
% 				\def\h{3}
% 				\path (0:0) coordinate (O)
% 				++(0:\a) coordinate (C)
% 				++(-165:\b) coordinate (B)
% 				($(A)+(90:\h)$) coordinate (S)
% 				($(O)+(90:3.3)$) coordinate (O1)
% 				($(C)!1!(O)$) coordinate (A);
% 				\draw[dashed,thick] (B)--(A)--(C) (O)--(S);
% 				\draw[thick] (S)--(B)--(C)--(S);
% 				%Ve truc Ox,Oy, Oz
% 				\draw[thick,->](C)--($(O)!1.2!(C)$) node [pos=0.9,below]{$x$};
% 				\draw[thick,->](B)--($(O)!1.25!(B)$) node [pos=0.9,right]{$y$};
% 				\draw[thick,->](S)--(O1) node [pos=0.9,above right]{$z$};
% 				%Các góc vuông
% 				\gv{B}{O}{C}
% 				\gv{A}{O}{C}
% 				\foreach \x/\g in {A/-45,B/120,C/-50,S/180,O/145}
% 				\fill[black] (\x) circle (1pt) ($(\g:4mm)+(\x)$) node {$\x$};	
% 			\end{tikzpicture}

% 			- Chọn hệ trục tọa độ như hình vẽ, $a=1$.
			
% 			- Tọa độ các điểm là:
				
% 				$A \equiv O(0;0;0)$, $B \left(0;OB;0\right)$, $C \left(AC;0;0\right)$, $S \left(0;0;SA \right)$.
	
% 	\end{tabular}}&{\begin{tabular}[l]{>{\raggedright\arraybackslash}p{5cm}} \textbf{Đáy là tam giác thường}
% 			\begin{tikzpicture}[>=stealth,font=\footnotesize,scale=1]
% 				\def\a{3.8}
% 				\def\b{3}
% 				\def\h{2}
% 				\path (0:0) coordinate (A)
% 				++(0:\a) coordinate (C)
% 				++(-150:\b) coordinate (B)
% 				($(A)!1/2!(C)$) coordinate (O)
% 				($(A)+(90:\h)$) coordinate (S)
% 				($(O)+(90:2.5)$) coordinate (O1)
% 				($(S)+(O)-(A)$) coordinate (H);
% 				\draw[dashed,thick] (A)--(C);
% 				\draw[thick] (S)--(A)--(B)--(C)--(S)--(B);
% 				\draw[dashed,thick](B)--(O) ;
% 				\draw[thick](S)--(H);
% 				%Ve truc Ox,Oy, Oz
% 				\draw[thick,->](C)--($(O)!1.4!(C)$) node [pos=0.9,below]{$x$};
% 				\draw[thick,->](B)--($(O)!1.2!(B)$) node [pos=0.9,below left]{$y$};
% 				\draw[dashed,thick](O)--($(O)!1/2!(H)$);
% 				\draw[thick,->]($(O)!1/2!(H)$)--(O1) node [pos=0.9,above right]{$z$};
% 				%Các góc vuông
% 				\gv{S}{A}{C}
% 				\gv{B}{O}{C}
% 				\foreach \x/\g in {B/-20,A/120,C/-50,S/180,O/40,H/-10}
% 				\fill[black] (\x) circle (1pt) ($(\g:4mm)+(\x)$) node {$\x$};	
% 			\end{tikzpicture}
			
% 				- Dựng đường cao $BO$ của $\triangle ABC$. Chọn hệ trục tọa độ như hình vẽ, $a=1$.\\
% 				- Tọa độ các điểm là:
			
% 				$O(0;0;0)$, $A \left(-OA;0;0\right)$, $B \left(0;OB;0\right)$, $C \left(OC;0;0\right)$, $S \left(-OA;0;\underbrace {OH}_{ = SA}\right)$.
% 	\end{tabular}}\\ \hline
% 	{\begin{tabular}[l]{>{\raggedright\arraybackslash}p{5cm}} \textbf{Đáy là hình vuông, hình chữ nhật}
			
% 			\begin{tikzpicture}[>=stealth,font=\footnotesize,scale=1]
% 				\def\a{3}
% 				\def\b{2}
% 				\def\h{2}
% 				\path 	(0:0) coordinate (A)
% 				++(0:\a) coordinate (D)
% 				++(-130:\b) coordinate (C)
% 				($(A)+(C)-(D)$) coordinate (B)
% 				($(A)+(90:\h)$) coordinate (S);
% 				\draw[dashed,thick] (S)--(A)--(B) (D)--(A);
% 				\draw[thick] (S)--(B)--(C)--(D)--(S)--(C);
% 				%Ve truc Ox,Oy, Oz
% 				\draw[thick,->](D)--($(A)!1.2!(D)$) node [pos=0.9,above ]{$x$};
% 				\draw[thick,->](B)--($(A)!1.2!(B)$) node [pos=0.9, above]{$y$};
% 				\draw[thick,->](S)--($(A)!1.2!(S)$) node [pos=0.9,above ]{$z$};
% 				%Các góc vuông
% 				\gv{D}{A}{S}
% 				\gv{D}{A}{B}
% 				\foreach \x/\g in {A/-90,B/-40,C/-40,D/-90,S/180}
% 				\fill[black] (\x) circle (1pt) ($(\g:4mm)+(\x)$) node {$\x$};	
% 			\end{tikzpicture}
		
% 			- Chọn hệ trục tọa độ như hình vẽ, $a=1$.
			
% 			- Tọa độ các điểm là:
			
% 			$A \equiv O(0;0;0)$, $B \left(0;AB;0\right)$, $C \left(AD;AB;0\right)$, $D(AD;0;0)$, $S \left(0;0;SA\right)$.
% \end{tabular}}&{\begin{tabular}[l]{>{\raggedright\arraybackslash}p{5cm}}\textbf{Đáy là hình thoi}
		
% 		\begin{tikzpicture}[>=stealth,font=\footnotesize,scale=1]
% 			\def\a{3}
% 			\def\b{2}
% 			\def\h{2}
% 			\path 	(0:0) coordinate (A)
% 			++(0:\a) coordinate (D)
% 			++(-130:\b) coordinate (C)
% 			($(A)+(C)-(D)$) coordinate (B)
% 			($(A)+(90:\h)$) coordinate (S)
% 			($(A)!1/2!(C)$) coordinate (O)
% 			($(S)+(O)-(A)$) coordinate (H);
% 			\draw[dashed,thick] (S)--(A)--(B) (D)--(A);
% 			\draw[thick] (S)--(B)--(C)--(D)--(S)--(C) (S)--(H);
% 			\draw[dashed,thick] (A)--(C) (B)--(D);
% 			%Ve truc Ox,Oy, Oz
% 			\draw[thick,->](A)--($(A)!-1/5!(C)$) node [pos=0.9,above ]{$x$};
% 			\draw[thick,->](B)--($(B)!-1/10!(D)$) node [pos=0.9, above]{$y$};
% 			\draw[thick,dashed](O)--($(O)!1/2!(H)$);
% 			\draw[thick,->]($(O)!1/2!(H)$)--($(O)!3/2!(H)$) node [pos=0.9,right ]{$z$};
	
% 			%Các góc vuông
% 			\gv{D}{A}{S}
% 			\gv{S}{H}{O}
% 			\gv{A}{O}{D}
% 			\foreach \x/\g in {A/-90,B/-40,C/-40,D/-90,S/180,O/-90,H/-40}
% 			\fill[black] (\x) circle (1pt) ($(\g:4mm)+(\x)$) node {$\x$};	
% 		\end{tikzpicture}
	
% 			- Chọn hệ trục tọa độ như hình vẽ, $a=1$.
			
% 			- Tọa độ các điểm là:
			
% 			$O(0;0;0)$, $A(OA;0;0)$, $B \left(0;OB;0\right)$, $C \left(-OC;0;0\right)$, $D(0;-OD;0)$, $S \left(OA;0;\underbrace {OH}_{ = SA} \right)$.
		
% \end{tabular}}&{\begin{tabular}[l]{>{\raggedright\arraybackslash}p{5.8cm}} \textbf{Đáy là hình thang vuông}
% 	\begin{tikzpicture}[>=stealth,font=\footnotesize,scale=1]
% 		\def\a{3}
% 		\def\b{2.1}
% 		\def\h{2.2}
% 		\path 	(0:0) coordinate (A)
% 		++(0:\a) coordinate (D)
% 		($(A)+(-140:\b)$) coordinate (B)
% 		($(A)+(90:\h)$) coordinate (S)
% 		($(A)!0.78!(D)$) coordinate (H)
% 		($(H)+(B)-(A)$) coordinate (C);
% 		\draw[dashed,thick] (S)--(A)--(B) (D)--(A) (C)--(H);
% 		\draw[thick] (S)--(B)--(C)--(D)--(S)--(C);
% 		%Ve truc Ox,Oy, Oz
% 		\draw[thick,->](D)--($(A)!1.2!(D)$) node [pos=0.9,above ]{$x$};
% 		\draw[thick,->](B)--($(A)!1.2!(B)$) node [pos=0.9, above]{$y$};
% 		\draw[thick,->](S)--($(A)!1.2!(S)$) node [pos=0.9,above ]{$z$};
% 		%Các góc vuông
% 		\gv{D}{A}{S}
% 		\gv{D}{A}{B}
% 		\gv{C}{H}{A}
% 		\foreach \x/\g in {A/-90,B/-40,C/-40,D/-90,S/180,H/120}
% 		\fill[black] (\x) circle (1pt) ($(\g:4mm)+(\x)$) node {$\x$};	
% 	\end{tikzpicture}
	
% 	 	- Chọn hệ trục tọa độ như hình vẽ, $a=1$.
	 
% 	 - Tọa độ các điểm là:
	 
% 	 $A \equiv O(0;0;0)$, $B \left(0;AB;0\right)$, $C \left(AH;AB;0\right)$, $D(AD;0;0)$, $S \left(0;0;SA\right)$.
% \end{tabular}}\\ \hline
% \end{longtable}
% \newpage
% 	\begin{longtable}{|>{\raggedright\arraybackslash}p{5cm}|>{\raggedright\arraybackslash}p{5cm}|>{\raggedright\arraybackslash}p{5cm}|}
% 	\hline
% 	\multicolumn{3}{|>{\centering\arraybackslash}p{16.5cm}|}{\textbf{2. Hình chóp có cạnh mặt bên $(SAB)$ vuông góc với mặt đáy}}                                                                                                                                                                                 \\ \hline
% 	\multicolumn{1}{|>{\raggedright\arraybackslash}p{5cm}|}{\begin{tabular}[l]{>{\raggedright\arraybackslash}p{4.5cm}} \textbf{Đáy là tam giác, mặt bên là tam giác thường}
% 			\begin{tikzpicture}[>=stealth,font=\footnotesize,scale=1]
% 				\def\a{4}
% 				\def\b{3}
% 				\def\h{3}
% 				\path (0:0) coordinate (A)
% 				++(0:\a) coordinate (C)
% 				++(-140:\b) coordinate (B)
% 				($(A)!0.55!(B)$) coordinate (O)
% 				($(A)!1/3!(B)$) coordinate (H)
% 				($(O)+(90:3.7)$) coordinate (O1)
% 				($(H)+(90:\h)$) coordinate (S)
% 				($(S)+(O)-(H)$) coordinate (K);
% 				\draw[dashed,thick] (A)--(C) (C)--(O);
% 				\draw[thick] (S)--(A)--(B)--(C)--(S)--(B);
% 				\draw[thick](S)--(H) (S)--(K);
% 				%Ve truc Ox,Oy, Oz
% 				\draw[thick,->](C)--($(O)!1.1!(C)$) node [pos=0.9,above ]{$x$};
% 				\draw[thick,->](A)--($(O)!1.3!(A)$) node [pos=0.9, above]{$y$};
% 				\draw[thick,->](O)--(O1) node [above]{$z$};
% 				%Các góc vuông
% 				\gv{C}{O}{B}
% 				\gv{S}{H}{B}
% 				\foreach \x/\g in {A/-90,B/0,C/-40,S/90,O/-110,H/-110,K/-45}
% 				\fill[black] (\x) circle (1pt) ($(\g:4mm)+(\x)$) node {$\x$};	
% 			\end{tikzpicture}
			
% 		- Vẽ đường cao $CO$ trong $\triangle ABC$. Chọn hệ trục như hình vẽ, $a=1$.
		
% 		- Tọa độ các điểm là:
				
% 				$O(0;0;0)$, $A \left(0;OA;0\right)$, $B \left(0;-OB;0\right)$, $C \left(OC;0;0\right)$, $S \left(0;OH;\underbrace {OK}_{ =SH}\right)$.
% 	\end{tabular}} &\multicolumn{1}{l|}{\begin{tabular}[l]{>{\raggedright\arraybackslash}p{5cm}}\textbf{Đáy là tam giác cân tại $C$ (hoặc đều), mặt bên là tam giác cân tại $S$ (hoặc đều)}
% 			\begin{tikzpicture}[>=stealth,font=\footnotesize,scale=1]
% 				\def\a{4}
% 				\def\b{3}
% 				\def\h{3.3}
% 				\path (0:0) coordinate (A)
% 				++(0:\a) coordinate (C)
% 				++(-140:\b) coordinate (B)
% 				($(A)!1/2!(B)$) coordinate (O)
% 				($(O)+(90:3.7)$) coordinate (O1)
% 				($(O)+(90:\h)$) coordinate (S);
% 				\draw[dashed,thick] (A)--(C) (C)--(O);
% 				\draw[thick] (S)--(A)--(B)--(C)--(S)--(B);
% 				\draw[thick](S)--(O);
% 				%Ve truc Ox,Oy, Oz
% 				\draw[thick,->](C)--($(O)!1.1!(C)$) node [pos=0.9,above ]{$x$};
% 				\draw[thick,->](A)--($(O)!1.3!(A)$) node [pos=0.9, above]{$y$};
% 				\draw[thick,->](O)--(O1) node [above]{$z$};
% 				%Các góc vuông
% 				\gv{C}{O}{B}
% 				\gv{S}{O}{A}
% 				\foreach \x/\g in {A/-90,B/0,C/-40,S/180,O/-110}
% 				\fill[black] (\x) circle (1pt) ($(\g:4mm)+(\x)$) node {$\x$};	
% 			\end{tikzpicture}
% 			- Gọi $O$ là trung điểm $BC$. Chọn hệ trục như hình vẽ, $a=1$.
			
% 			- Tọa độ các điểm là:
			
% 			$O(0;0;0)$, $A \left(0;OA;0\right)$, $B \left(0;-OB;0\right)$, $C \left(OC;0;0\right)$, $S \left(0;0;SO\right)$.
% 	\end{tabular}} & \begin{tabular}[l]{>{\raggedright\arraybackslash}p{5.4cm}} \textbf{Đáy là hình chữ nhật, hình vuông, mặt bên là tam giác thường}
% 			\begin{tikzpicture}[>=stealth,font=\footnotesize,scale=1]
% 			\def\a{3}
% 			\def\b{2}
% 			\def\h{2.5}
% 			\path (0:0) coordinate (A)
% 			++(0:\a) coordinate (B)
% 			++(-130:\b) coordinate (C)
% 			($(A)+(C)-(B)$) coordinate (D)
% 			($(A)!1/3!(B)$) coordinate (H)
% 			($(H)+(90:\h)$) coordinate (S)
% 			($(A)+(S)-(H)$) coordinate (K);
% 			\draw[dashed,thick] (S)--(A)--(D) (B)--(A) (S)--(H);
% 			\draw[thick] (S)--(D)--(C)--(B)--(S)--(C) (K)--(S);
% 			%Ve truc Ox,Oy, Oz
% 			\draw[thick,->](B)--($(A)!1.2!(B)$) node [pos=0.9,above ]{$x$};
% 			\draw[thick,->](D)--($(A)!1.2!(D)$) node [pos=0.9, above]{$y$};
% 			\draw[thick,->](A)--($(A)!1.2!(K)$) node [right]{$z$};
% 			%Các góc vuông
% 			\gv{S}{H}{B}
% 			\gv{D}{A}{B}
% 			\foreach \x/\g in {A/-90,D/-40,C/-40,B/-90,S/90,H/-90,K/180}
% 			\fill[black] (\x) circle (1pt) ($(\g:4mm)+(\x)$) node {$\x$};	
% 		\end{tikzpicture}
% 		- Chọn hệ trục tọa độ như hình vẽ, $a=1$.
		
% 		- Tọa độ các điểm là:
			
% 			$A \equiv O(0;0;0)$, $B \left(AB;0;0\right)$, $C \left(AB;AD;0\right)$, $D \left(0;AD;0\right)$, $S \left(AH;0;\underbrace {AK}_{ = SH}\right)$.
% 	\end{tabular} \\\hline
% \end{longtable}
% \newpage
% \begin{longtable}{|>{\raggedright\arraybackslash}p{8.5cm}|>{\raggedright\arraybackslash}p{8.5cm}|}
% 	\hline
% 	{\begin{tabular}[l]{>{\raggedright\arraybackslash}p{8.5cm}} \textbf{Hình chóp tam giác đều}
% 		\begin{tikzpicture}[>=stealth,font=\footnotesize,scale=1]
% 			\def\a{4}
% 			\def\b{3}
% 			\def\h{4}
% 			\path (0:0) coordinate (A)
% 			++(0:\a) coordinate (C)
% 			++(-150:\b) coordinate (B)
% 			($(B)!1/2!(C)$) coordinate (O)
% 			($(A)!2/3!(O)$) coordinate (H)
% 			($(H)+(90:\h)$) coordinate (S)
% 			($(S)+(O)-(H)$) coordinate (K);
% 			\draw[dashed,thick] (A)--(C) (A)--(O) (S)--(H);
% 			\draw[thick] (S)--(A)--(B)--(C)--(S)--(B) (S)--(K);
% 			\foreach \x/\g in {A/180,B/-90,C/0,S/180,H/-120}
% 				%Ve truc Ox,Oy, Oz
% 				\draw[thick,->](C)--($(C)!-0.2!(B)$) node [pos=0.9,above ]{$x$};
% 				\draw[thick,->](A)--($(A)!-0.2!(O)$) node [pos=0.9, above]{$y$};
% 				\draw[thick,->](O)--($(O)!1.15!(K)$) node [above]{$z$};
% 				%Các góc vuông
% 				\gv{S}{H}{O}
% 				\gv{A}{O}{K}
% 				\gv{K}{O}{C}
% 				\foreach \x/\g in {A/-90,B/-90,C/-90,S/180,H/-90,O/0,K/0}
% 				\fill[black] (\x) circle (1pt) ($(\g:4mm)+(\x)$) node {$\x$};	
% 			\end{tikzpicture}
			
% 			Gọi $O$ là trung điểm $BC$. Chọn hệ trục như hình vẽ, $a=1$.
			
% 			- Tọa độ các điểm là:
			
% 			$O(0;0;0)$, $A \left(0;\dfrac{AB \sqrt{3}}{2};0\right)$, $B \left(-\dfrac{BC}{2};0;0\right)$, $C \left(0;0;OC\right)$, $S \left(0;\underbrace {\dfrac{AB \sqrt{3}}{6}}_{ =SH};\underbrace {OK}_{ =SH}\right)$.
% 	\end{tabular}} &{\begin{tabular}[l]{>{\raggedright\arraybackslash}p{8.5cm}}\textbf{Hình chóp tứ giác đều}
% 			\begin{tikzpicture}[>=stealth,font=\footnotesize,scale=1]
% 					\def\a{3}
% 				\def\b{2.5}
% 				\def\h{3.4}
% 				\path (0:0) coordinate (D)
% 				++(0:\a) coordinate (A)
% 				++(-150:\b) coordinate (B)
% 				($(D)+(B)-(A)$) coordinate (C)
% 				($(D)!1/2!(B)$) coordinate (O)
% 				($(O)+(90:\h)$) coordinate (S);
% 				\draw[dashed,thick] (C)--(D)--(A) (D)--(S) (D)--(B) (A)--(C);
% 				\draw[thick] (S)--(C)--(B)--(A)--(S) (B)--(S);
% 				%Ve truc Ox,Oy, Oz
% 				\draw[thick,->](A)--($(C)!1.2!(A)$) node [pos=0.9,above ]{$x$};
% 				\draw[thick,->](B)--($(D)!1.4!(B)$) node [pos=0.9, above right]{$y$};
% 				\draw[dashed,thick] (S)--(O);
% 				\draw[thick,->](S)--($(O)!1.15!(S)$) node [above]{$z$};
% 				%Các góc vuông
% 				\gv{D}{O}{S}
% 				\gv{B}{O}{A}
% 				\foreach \x/\g in {A/-40,C/180,D/180,B/-140,S/180,O/-120}
% 				\fill[black] (\x) circle (1pt) ($(\g:4mm)+(\x)$) node {$\x$};	
% 			\end{tikzpicture}
			
% 			- Chọn hệ trục như hình vẽ, $a=1$.
			
% 			- Tọa độ các điểm là:
			
% 			$O(0;0;0)$, $A \left(\underbrace{\dfrac{AB \sqrt{2}}{2}}_{ =OA};0;0\right)$, $B\left(0;\underbrace{\dfrac{AB \sqrt{2}}{2}}_{ =OB};0\right)$, $C \left(\underbrace{-\dfrac{AB \sqrt{2}}{2}}_{ =-OA};0;0 \right)$,  $D \left(0;\underbrace{-\dfrac{AB \sqrt{2}}{2}}_{ =-OA};0 \right)$, $S \left(0;0;SO\right)$.
% 	\end{tabular}}\\ \hline
% \multicolumn{2}{|>{\centering\arraybackslash}p{17cm}|}{\textbf{II. Gắn tọa độ đối với hình lăng trụ}}                                                                                                                                                                                 \\ \hline
% \multicolumn{2}{|>{\centering\arraybackslash}p{17cm}|}{\textbf{1. Hình lăng trụ đứng}}                                                                                                                                                            \\ \hline
% {\begin{tabular}[l]{>{\raggedright\arraybackslash}p{8.5cm}} \textbf{Hình lập phương, hình hộp chữ nhật}
% 		\begin{tikzpicture}[>=stealth,font=\footnotesize,scale=0.9]
% 		\def\a{4} 
% 		\def\b{1.8}
% 		\def\h{2}
% 		\path 	(0:0) coordinate (A)
% 		++(0:\a) coordinate (D)
% 		++(-130:\b) coordinate (C)
% 		($(A)+(C)-(D)$) coordinate (B)
% 		($(A)+(90:\h)$) coordinate (A')
% 		($(B)+(90:\h)$) coordinate (B')
% 		($(C)+(90:\h)$) coordinate (C')
% 		($(D)+(90:\h)$) coordinate (D');
% 		\draw[dashed,thick] 	(B)--(A)--(D)	(A)--(A');
% 		\draw[thick] (C)--(C') 	(D)--(D') 	(B)--(B') 	(B)--(C)--(D) (A')--(B')--(C')--(D')--cycle;
% 			%Ve truc Ox,Oy, Oz
% 			\draw[thick,->](D)--($(A)!1.2!(D)$) node [pos=0.9,above ]{$x$};
% 			\draw[thick,->](B)--($(A)!1.4!(B)$) node [pos=0.9,right]{$y$};
% 			\draw[thick,->](A')--($(A)!1.2!(A')$) node [right]{$z$};
% 			%Các góc vuông
% 			\gv{B}{A}{D}
% 			\gv{D}{A}{A'}
% 			\foreach \x/\g in  {A/180,B/180,C/0,D/-85,A'/180,B'/180,C'/0,D'/0}
% 			\fill[black] (\x) circle (1pt) ($(\g:4mm)+(\x)$) node {$\x$};	
% 		\end{tikzpicture}
		
% 		- Chọn hệ trục như hình vẽ, $a=1$.
		
% 		- Tọa độ các điểm là:
		
% 	$A \equiv O \left(0;0;0\right)$, $B \left(0;AB;0\right)$, $C \left(AD;AB;0\right)$,  $D \left(AD;0;0\right)$, $A' \left(0;0;AA'\right)$, $B' \left(0;AB;AA'\right)$, $C' \left(AD;AB;AA'\right)$, $D' \left(AD;0;AA'\right)$.
% \end{tabular}} &{\begin{tabular}[l]{>{\raggedright\arraybackslash}p{8.5cm}}\textbf{\textbf{Hình lăng trụ đứng đáy là hình thoi}}
% 		\begin{tikzpicture}[>=stealth,font=\footnotesize,scale=0.9]
% 			\def\a{4} 
% 			\def\b{1.5}
% 			\def\h{1.8}
% 			\path 	(0:0) coordinate (A)
% 			++(0:\a) coordinate (D)
% 			++(-130:\b) coordinate (C)
% 			($(A)+(C)-(D)$) coordinate (B)
% 			($(A)+(90:\h)$) coordinate (A')
% 			($(B)+(90:\h)$) coordinate (B')
% 			($(C)+(90:\h)$) coordinate (C')
% 			($(D)+(90:\h)$) coordinate (D')
% 			($(A)!1/2!(C)$) coordinate (O)
% 			($(A')!1/2!(C')$) coordinate (O');
% 			\draw[dashed,thick] 	(B)--(A)--(D)	(A)--(A') (A)--(C) (B)--(D);
% 			\draw[thick] (C)--(C') 	(D)--(D') 	(B)--(B') 	(B)--(C)--(D) (A')--(B')--(C')--(D')--cycle (A')--(C') (B')--(D');
% 			%Ve truc Ox,Oy, Oz
% 			\draw[thick,->](C)--($(A)!1.2!(C)$) node [pos=0.9,below]{$x$};
% 			\draw[thick,->](B)--($(D)!1.1!(B)$) node [below right]{$y$};
% 			\draw[thick,dashed](O)--(O');
% 			\draw[thick,->](O')--($(O)!1.6!(O')$) node [right]{$z$};
% 			%Các góc vuông
% 			\gv{A'}{A}{D}
% 			\gv{B}{A}{D}
% 			\gv{A}{O}{D}
% 			\foreach \x/\g in  {A/180,B/159,C/15,D/-85,A'/180,B'/180,C'/0,D'/0,O/-90}
% 			\fill[black] (\x) circle (1pt) ($(\g:4mm)+(\x)$) node {$\x$};	
% 		\end{tikzpicture}
		
% 	- Chọn hệ trục như hình vẽ, $a=1$.
	
% 	- Tọa độ các điểm là:
	
% 	$O \left(0;0;0\right)$, $A \left(-OA;0;0\right)$, $B \left(0;OB;0\right)$, $C \left(OC;0;0\right)$,  $D \left(0;-OD:0\right)$, $A' \left(-OA;0;AA'\right)$, $B' \left(0;OB;AA'\right)$, $C' \left(OC;0;CC'\right)$, $D' \left(0;-OD;DD'\right)$.
% \end{tabular}}\\ \hline
% {\begin{tabular}[l]{>{\raggedright\arraybackslash}p{8.5cm}} \textbf{Lăng trụ tam giác đều}
% 		\begin{tikzpicture}[>=stealth,font=\footnotesize,scale=0.9]
% 			\def\a{4.3} 
% 			\def\b{2}
% 			\def\h{2.5}
% 			\path 	(0:0) coordinate (A)
% 			++(0:\a) coordinate (C)
% 			(A)	++(-50:\b) coordinate (B)
% 			($(A)+(90:\h)$) coordinate (A')
% 			($(B)+(90:\h)$) coordinate (B')
% 			($(C)+(90:\h)$) coordinate (C')
% 			($(A)!1/2!(C)$) coordinate (O)
% 			($(A')!1/2!(C')$) coordinate (O');
% 			\draw[dashed,thick] 	(A)--(C) (B)--(O) (O)--(O');
% 			\draw[thick] (C)--(C')--(B')--(A') (B)--(B') (B')--(O')	(B)--(C) (C')--(A')--(A)--(B) ;
% 			%Ve truc Ox,Oy, Oz
% 			\draw[thick,->](C)--($(A)!1.2!(C)$) node [pos=0.9,above ]{$x$};
% 			\draw[thick,->](B)--($(O)!1.4!(B)$) node [pos=0.9,right]{$y$};
% 			\draw[thick,->](O')--($(O)!1.2!(O')$) node [right]{$z$};
% 			%Các góc vuông
% 			\gv{B}{O}{C}
% 			\gv{C}{O}{O'}
% 				\foreach \x/\g in {A/180,B/180,C/-50,A'/180,B'/180,C'/0,O/130}
% 			\fill[black] (\x) circle (1pt) ($(\g:4mm)+(\x)$) node {$\x$};	
% 		\end{tikzpicture}
		
% 		- Chọn hệ trục như hình vẽ, $a=1$.
		
% 		- Tọa độ các điểm là:
		
% 		$O \left(0;0;0\right)$, $A \left(-\dfrac{AC}{2};0;0\right)$, $B \left(0;OB;0\right)$,  $C \left(\dfrac{AC}{2};0;0\right)$, $A' \left(-\dfrac{AC}{2};0;AA' \right)$, $B' \left(0;OB;AA'\right)$, $C' \left(\dfrac{AC}{2};0;AA'\right)$.
% \end{tabular}} &{\begin{tabular}[l]{>{\raggedright\arraybackslash}p{8.4cm}}\textbf{\textbf{Lăng trụ đứng có đáy là tam giác thường}}

% \begin{tikzpicture}[>=stealth,font=\footnotesize,scale=0.9]
% 	\def\a{4.3} 
% 	\def\b{2}
% 	\def\h{2.5}
% 	\path 	(0:0) coordinate (A)
% 	++(0:\a) coordinate (C)
% 	(A)	++(-50:\b) coordinate (B)
% 	($(A)+(90:\h)$) coordinate (A')
% 	($(B)+(90:\h)$) coordinate (B')
% 	($(C)+(90:\h)$) coordinate (C')
% 	($(A)!0.4!(C)$) coordinate (O)
% 	($(A')!0.4!(C')$) coordinate (O');
% 	\draw[dashed,thick] 	(A)--(C) (B)--(O) (O)--(O');
% 	\draw[thick] (C)--(C')--(B')--(A') (B)--(B') (B')--(O')	(B)--(C) (C')--(A')--(A)--(B) ;
% 	%Ve truc Ox,Oy, Oz
% 	\draw[thick,->](C)--($(A)!1.2!(C)$) node [pos=0.9,above ]{$x$};
% 	\draw[thick,->](B)--($(O)!1.4!(B)$) node [pos=0.9,right]{$y$};
% 	\draw[thick,->](O')--($(O)!1.2!(O')$) node [right]{$z$};
% 	%Các góc vuông
% 	\gv{B}{O}{C}
% 	\gv{C}{O}{O'}
% 	\foreach \x/\g in {A/180,B/180,C/-50,A'/180,B'/180,C'/0,O/130}
% 	\fill[black] (\x) circle (1pt) ($(\g:4mm)+(\x)$) node {$\x$};	
% \end{tikzpicture}
		
% 		- Vẽ đường cao $CO$ của $\triangle ABC$. Chọn hệ trục như hình vẽ, $a=1$.
		
% 		- Tọa độ các điểm là:
		
% 		$O \left(0;0;0\right)$, $A \left(-OA;0;0\right)$, $B \left(0;OB;0\right)$, $C \left(OC;0;0\right)$, $A' \left(-OA;0;AA'\right)$, $B' \left(0;OB;AA'\right)$, $C' \left(OC;0;AA'\right)$.
% \end{tabular}} \\ \hline
% \multicolumn{2}{|>{\centering\arraybackslash}p{17cm}|}{\textbf{2. Hình lăng trụ xiên}}                                                                          \\ \hline
% {\begin{tabular}[l]{>{\raggedright\arraybackslash}p{8.4cm}} \textbf{Lăng trụ có đáy là tam giác đều, hình chiếu của các đỉnh trên mặt đối diện là trung điểm của một cạnh tam giác đáy}
% 		\begin{tikzpicture}[>=stealth,font=\footnotesize,scale=0.9]
% 			\def\a{4.3} 
% 			\def\b{2}
% 			\def\h{2.5}
% 			\path 	(0:0) coordinate (A)
% 			++(0:\a) coordinate (C)
% 			(A)	++(-50:\b) coordinate (B)
% 			($(A)!1/2!(C)$) coordinate (O)
% 			($(O)+(90:\h)$) coordinate (A')
% 			($(A')+(B)-(A)$) coordinate (B')
% 			($(A')+(C)-(A)$) coordinate (C')
% 			($(A')!1/2!(C')$) coordinate (O');
% 			\draw[dashed,thick] 	(A)--(C) (B)--(O) (O)--(A');
% 			\draw[thick] (C)--(C')--(B')--(A') (B)--(B') (B')--(O')	(B)--(C) (C')--(A')--(A)--(B) ;
% 			%Ve truc Ox,Oy, Oz
% 			\draw[thick,->](C)--($(A)!1.2!(C)$) node [pos=0.9,above ]{$x$};
% 			\draw[thick,->](B)--($(O)!1.4!(B)$) node [pos=0.9,right]{$y$};
% 			\draw[thick,->](A')--($(O)!1.25!(A')$) node [right]{$z$};
% 			%Các góc vuông
% 			\gv{A}{O}{A'}
% 			\gv{A}{O}{B}
% 			\foreach \x/\g in {A/180,B/180,C/-50,A'/180,B'/180,C'/0,O/30}
% 			\fill[black] (\x) circle (1pt) ($(\g:4mm)+(\x)$) node {$\x$};	
% 		\end{tikzpicture}
		
% 		- Chọn hệ trục như hình vẽ, ta dễ xác định tọa đọ các điểm $O$, $A$, $B$, $C$, $A'$.
		
% 		- Tìm tọa độ các điểm còn lại thông qua $\overrightarrow{AA'}=\overrightarrow{BB'}=\overrightarrow{CC'}$.
% \end{tabular}} &{\begin{tabular}[l]{>{\raggedright\arraybackslash}p{8.4cm}}\textbf{\textbf{Lăng trụ xiên có đáy là hình vuông hoặc hình chữ nhật, hình chiếu của một đỉnh là một điểm thuộc cạnh đáy không chứa đỉnh đó}}
		
% 		\begin{tikzpicture}[>=stealth,font=\footnotesize,scale=0.9]
% 			\def\a{4} 
% 			\def\b{1.5}
% 			\def\h{2.8}
% 			\path (0:0) coordinate (A)
% 			++(0:\a) coordinate (D)
% 			++(-150:\b) coordinate (C)
% 			($(A)+(C)-(D)$) coordinate (B)
% 			($(B)!1/2!(C)$) coordinate (O)
% 			($(O)+(90:\h)$) coordinate (A')
% 			($(A')+(B)-(A)$) coordinate (B')
% 			($(A')+(C)-(A)$) coordinate (C')
% 			($(A')+(D)-(A)$) coordinate (D')
% 			($(A)!1/2!(D)$) coordinate (O');
% 			\draw[dashed,thick] 	(B)--(A)--(D)	(A)--(A') (A')--(O) (O)--(O');
% 			\draw[thick] (C)--(C') 	(D)--(D') 	(B)--(B') 	(B)--(C)--(D) (A')--(B')--(C')--(D')--cycle;
% 			%Ve truc Ox,Oy, Oz
% 			\draw[thick,->](C)--($(C)!-0.3!(B)$) node [pos=0.9,below]{$x$};
% 			\draw[thick,->,dashed](O')--($(O)!1.5!(O')$) node [right]{$y$};
% 			\draw[thick,->](A')--($(O)!1.3!(A')$) node [pos=0.9,right]{$z$};
% 			%Các góc vuông
% 			\gv{C}{O}{A'}
% 			\gv{C}{B}{A}
% 			\foreach \x/\g in  {A/180,B/180,C/-90,D/-85,A'/180,B'/180,C'/0,D'/0,O/-90}
% 			\fill[black] (\x) circle (1pt) ($(\g:4mm)+(\x)$) node {$\x$};		
% 		\end{tikzpicture}
		
% 		- Chọn hệ trục như hình vẽ, ta dễ xác định tọa đọ các điểm $O$, $A$, $B$, $C$, $A'$.
	
% 		- Tìm tọa độ các điểm còn lại thông qua $\overrightarrow{AA'}=\overrightarrow{BB'}=\overrightarrow{CC'}=\overrightarrow{DD'}$.
% \end{tabular}} \\ \hline
% \end{longtable}

\end{dang}
\TN
\Opensolutionfile{ans}[ans/ans-2C5B1CD3]
\begin{ex}%[2H5H1-5]
	Cho tứ diện $O.ABC$, có $OA$, $OB$, $OC$ đôi một vuông góc và $OA=5$, $OB=2$, $OC=4$. Gọi $M$, $N$ lần lượt là trung điểm của $OB$ và $OC$. Gọi $G$ là trọng tâm của tam giác $ABC$. Khoảng cách từ $G$ đến mặt phẳng $(AMN)$ là
	\choice
	{\True $\dfrac{20}{3\sqrt{129}}$}
	{$\dfrac{20}{\sqrt{129}}$}
	{$\dfrac{1}{4}$}
	{$\dfrac{1}{2}$}
	\loigiai{
		\begin{center}
			\begin{tikzpicture}[>=stealth,font=\footnotesize,scale=1]
				\def\a{3.5}
				\def\b{4.5}
				\def\h{3}
				\path (0:0) coordinate (O)
				++(0:\a) coordinate (C)
				++(-165:\b) coordinate (B)
				($(O)+(90:\h)$) coordinate (A)
				($(O)!1/2!(B)$) coordinate (M)
				($(O)!1/2!(C)$) coordinate (N) ;
				\draw[dashed,thick] (B)--(O)--(C) (O)--(A) (A)--(M)--(N)--(A);
				\draw[thick] (A)--(B)--(C)--(A);
				%Ve truc Ox,Oy, Oz
				\draw[thick,->](B)--($(O)!1.6!(B)$) node [pos=0.9,right]{$x$};
				\draw[thick,->](C)--($(O)!1.2!(C)$) node [pos=0.9,below]{$y$};
				\draw[thick,->](A)--($(O)!1.2!(A)$) node [pos=0.9,above right]{$z$};
				\foreach \x/\g in {B/120,C/-90,A/180,O/45,M/-30,N/60}
				\fill[black] (\x) circle (1pt) ($(\g:4mm)+(\x)$) node {$\x$};	
			\end{tikzpicture}
		\end{center}
	Chọn hệ trục tọa độ $Oxyz$ như hình vẽ.\\
	Ta có $O(0;0;0)$, $A \in Oz$, $B \in Ox$, $C \in Oy$ sao cho $OA=5$, $OB=2$, $OC=4$.\\
	Do đó $A(0;0;5)$, $B(2;0;0)$, $C(0;4;0)$.\\
	Khi đó $G$ là trọng tâm tam giác $ABC$ nên $G\left(\dfrac{2}{3};\dfrac{4}{3};\dfrac{5}{3}\right)$.\\
	Vì $M$ là trung điểm $OB$ nên $M(1;0;0)$.\\
	Vì $N$ là trung điểm $OC$ nên $N(0;2;0)$.\\
	Phương trình mặt phẳng $(AMN)$ là $\dfrac{x}{1}+\dfrac{y}{2}+\dfrac{z}{5}=1$ hay $10x+5y+2z-10=0$.\\
	Vậy khoảng cách từ $G$ đến mặt phẳng $(AMN)$ là
	$$\mathrm{d}(G,(AMN))=\dfrac{\left|\dfrac{20}{3}+\dfrac{20}{3}+\dfrac{10}{3}-10 \right|}{\sqrt{100+25+4}}=\dfrac{20}{3\sqrt{129}}.$$
	}
\end{ex}
\begin{ex}%[2H5V1-5]
Cho hình chóp $S.ABCD$ có đáy là hình thang vuông tại $A$ và $D$, $SA \perp (ABCD)$. Góc giữa $SB$ và mặt phẳng đáy bằng $45^\circ$, $E$ là trung điểm của $SD$, $AB=2a$, $AD=DC=a$. Tính khoảng cách từ điểm $B$ đến mặt phẳng $(ACE)$.
	\choice
	{ $\dfrac{2a}{2}$}
	{\True$\dfrac{4a}{3}$}
	{$a$}
	{$\dfrac{3a}{4}$}
	\loigiai{
\begin{center}
		\begin{tikzpicture}[>=stealth,font=\footnotesize,scale=1]
		\def\a{6.5}
		\def\b{3}
		\def\h{3.4}
		\path (0:0) coordinate (A)
		++(0:\a) coordinate (B)
		($(A)+(-145:\b)$) coordinate (D)
		($(A)+(90:\h)$) coordinate (S)
		($(A)!0.5!(B)$) coordinate (H)
		($(S)!0.5!(D)$) coordinate (E)
		($(H)+(D)-(A)$) coordinate (C);
		\draw[dashed,thick] (S)--(A)--(D) (B)--(A) (C)--(A)--(E);
		\draw[thick] (S)--(D)--(C)--(B)--(S)--(C) (C)--(E);
		\draw [thick]($(A)!7/8!(B)$) arc (180:135:0.5) node [pos=0.5,left]{$45^\circ$};
		%Ve truc Ox,Oy, Oz
		\draw[thick,->](B)--($(A)!1.2!(B)$) node [pos=0.9,above ]{$x$};
		\draw[thick,->](D)--($(A)!1.2!(D)$) node [pos=0.9, above]{$y$};
		\draw[thick,->](S)--($(A)!1.2!(S)$) node [pos=0.9,above ]{$z$};
		\foreach \x/\g in {A/40,D/-40,C/-40,B/-90,S/180,E/180}
		\fill[black] (\x) circle (1pt) ($(\g:4mm)+(\x)$) node {$\x$};	
	\end{tikzpicture}
\end{center}		
Hình chiếu của $SB$ trên mặt phẳng $(ABCD)$ là $AB$ nên góc giữa $SB$ và mặt đáy là góc giữa $SB$ và $AB$ bằng $\widehat{SBA}=45^\circ$.\\
Vì tam giác $SAB$ vuông cân tại $A$ nên $SA=2a$.\\
Chọn hệ trục tọa độ như hình vẽ, ta có $A(0;0;0)$, $B(0;2a;0)$, $C(a;a;0)$, $D(a;0;0)$, $S(0;0;2a)$, $E \left(\dfrac{a}{2}\;0;a \right)$.\\
Ta có $\overrightarrow{AC}=(a;a;0)$, $\overrightarrow{AE}= \left(\dfrac{a}{2};0;a\right)$. Do đó $\left[\overrightarrow{AC},\overrightarrow{AE}\right]=\left(a^2;-a^2;-\dfrac{a^2}{2}\right)$.\\
Mặt phẳng $(ACE)$ có véc-tơ pháp tuyến là $\overrightarrow{n}=(2;-2;-1)$ nên $(ACE) \colon 2x-2y-z=0$.\\
Vậy $\mathrm{d}(B,(ACE))=\dfrac{|2 \cdot 2a|}{\sqrt{4+4+1}}=\dfrac{4a}{3}$.
	}
\end{ex}
\begin{ex}%[2H5V1-5]
Trong KG $Oxyz$, cho hình chóp $SABCD$ có đáy $ABCD$ là hình chữ nhật. Biết $A(0;0;0)$, $D(2;0;0)$, $B(0;4;0)$, $S(0;0;4)$. Gọi $M$ là trung điểm của $SB$. Tính khoảng cách từ $B$ đến mặt phẳng $(CDM)$.
	\choice
	{$\mathrm{d}(B,(CDM))=2$}
	{$\mathrm{d}(B,(CDM))=2 \sqrt{2}$}
	{$\mathrm{d}(B,(CDM))=\dfrac{1}{\sqrt{2}}$}
	{\True$\mathrm{d}(B,(CDM))=\sqrt{2}$}
	\loigiai{
		\begin{center}
		\begin{tikzpicture}[>=stealth,font=\footnotesize,scale=1]
			\def\a{5}
			\def\b{2.5}
			\def\h{4}
			\path 	(0:0) coordinate (A)
			++(0:\a) coordinate (D)
			++(-130:\b) coordinate (C)
			($(A)+(C)-(D)$) coordinate (B)
			($(A)+(90:\h)$) coordinate (S)
			($(S)!1/2!(B)$) coordinate (M);
			\draw[dashed,thick] (S)--(A)--(B) (D)--(A) (C)--(D)--(M)--(C);
			\draw[thick] (S)--(B)--(C)--(D)--(S)--(C);
			%Ve truc Ox,Oy, Oz
			\draw[thick,->](D)--($(A)!1.2!(D)$) node [pos=0.9,above ]{$x$};
			\draw[thick,->](B)--($(A)!1.4!(B)$) node [pos=0.9, above]{$y$};
			\draw[thick,->](S)--($(A)!1.2!(S)$) node [pos=0.9,above ]{$z$};
			
			\foreach \x/\g in {A/-90,B/-40,C/-40,D/-90,S/180,M/180}
			\fill[black] (\x) circle (1pt) ($(\g:4mm)+(\x)$) node {$\x$};	
		\end{tikzpicture}		
		\end{center}
	Tứ giác $ABCD$ là hình chữ nhật nên $\heva{&x_A+x_C=x_B+x_D\\&y_A+y_C=y_B+y_D\\&z_A+z_C=z_B+z_D} \Leftrightarrow \heva{&x_C=2\\&y_C=4\\&z_C=0} \Leftrightarrow C(2;4;0)$.\\
	Vì $M$ là trung điểm $SB$ nên $M(0;2;2)$.\\
	Ta có $\overrightarrow{CD}=(0;-4;0)$, $\overrightarrow{CM}=(-2;-2;2)$. Do đó $\left[\overrightarrow{CD},\overrightarrow{CM}\right]=(-8;0;-8)$.\\
	Mặt phẳng $(CDM)$ có véc-tơ pháp tuyến là $\overrightarrow{n}=(1;0;1)$.\\
	Suy ra $(CDM)$ có phương trình $x+z-2=0$.\\
	Vậy $\mathrm{d}(B,(CDM))=\dfrac{|0+0-2|}{\sqrt{1^2+0^2+1^2}}=\sqrt{2}$.
	}
\end{ex}
\begin{ex}%[2H5V1-5]
Một phần sân trường được định vị bởi các điểm $A$, $B$, $C$, $D$ như hình vẽ.
\begin{center}
	\begin{tikzpicture}[>=stealth,font=\footnotesize,scale=1]
		\def\a{5}
		\def\b{2.5}
		\def\h{4}
		\path 	(0:0) coordinate (A)
		++(0:\a) coordinate (B)
		++(-90:\b*1.4) coordinate (C)
		($(A)+(-90:\b)$) coordinate (D);
		\draw[thick] (A)--(B) node[pos=0.5,above]{$2500$cm};
		\draw[thick] (B)--(C) node[pos=0.5,right]{$1500$cm};
		\draw[thick] (A)--(D) node[pos=0.5,left]{$1600$cm};
		\draw[thick] (D)--(C);
		%Ve truc Ox,Oy, Oz
		%\draw[thick,->](D)--($(A)!1.2!(D)$) node [pos=0.9,above ]{$x$};
		%\draw[thick,->](B)--($(A)!1.4!(B)$) node [pos=0.9, above]{$y$};
		%\draw[thick,->](S)--($(A)!1.2!(S)$) node [pos=0.9,above ]{$z$};
		
		\foreach \x/\g in {A/180,B/-40,C/-40,D/-90}
		\fill[black] (\x) circle (1pt) ($(\g:4mm)+(\x)$) node {$\x$};	
	\end{tikzpicture}
\end{center}
Bước đầu chúng được lấy "thăng bằng" để có cùng độ cao, biết $ABCD$ là hình thang vuông ở $A$ và $B$ với độ dài $AB=25$ m, $AD=15$ m, $BC=18$ m. Do yêu cầu kĩ thuật, khi lát phẳng phần sân trường phải thoát nước về góc sân ở $C$ nên người ta lấy độ cao ở các điểm $B$, $C$, $D$ xuống thấp hơn so với độ cao ở $A$ là $10$ cm, $a$ cm, $6$ cm tương ứng. Giá trị của $a$ là số nào sau đây?
	\choice
	{$15{,}7$ cm}
	{\True $17{,}2$ cm}
	{$18{,}1$ cm}
	{$17{,}5$ cm}
	\loigiai{
	Chọn hệ trục tọa độ $Oxyz$ sao cho $O \equiv A$, tia $Ox \equiv AD$, tia $Oy \equiv AB$.
	\begin{center}
		\begin{tikzpicture}[>=stealth,font=\footnotesize,scale=0.7]
			\def\a{5}
			\def\b{2.5}
			\def\h{4}
			\path 	(0:0) coordinate (A)
			++(0:\a) coordinate (B)
			++(-120:\b*1.4) coordinate (C)
			($(A)+(-123:\b)$) coordinate (D)
			($(B)+(-90:1.2)$) coordinate (B')
			($(C)+(-90:2.2)$) coordinate (C')
			($(D)+(-90:1.8)$) coordinate (D');
			\draw[thick] (A)--(B)--(C)--(D)--(A);
			\draw[dashed] (B)--(B')--(C')--(C)--(B);
			\draw[dashed] (D)--(D')--(C');
			%Ve truc Ox,Oy, Oz
			\draw[thick,->](D)--($(A)!1.8!(D)$) node [pos=0.9, above left]{$x$};
			\draw[thick,->](B)--($(A)!1.4!(B)$) node [pos=0.9,above ]{$y$};
			\draw[thick,->](A)--($(A)+(90:\h)$) node [pos=0.9,right]{$z$};
			
			\foreach \x/\g in {A/180,B/-40,C/140,D/180,B'/0,C'/0,D'/-90}
			\fill[black] (\x) circle (1pt) ($(\g:4mm)+(\x)$) node {$\x$};	
		\end{tikzpicture}
	\end{center}
	Khi đó $A(0;0;0)$, $B(0;2500;0)$, $C(1800;2500;0)$, $D(1500;0;0)$.\\
	Khi hạ độ cao các điểm ở các điểm $B$, $C$, $D$ xuống thấp hơn so với độ cao ở $A$ là $10$ cm, $a$ cm, $6$ cm tương ứng ta có các điểm mới $B'(0;2500;-10)$, $C(1800;2500;-a)$, $D'(1500;0;-6)$.\\
	Theo bài ta có bốn điểm $A$, $B'$, $C'$, $D'$ đồng phẳng.\\
	Phương trình mặt phẳng $(AB'D') \colon x+y+250z=0$.\\
	Do $C'(1800;2500;-a) \in (AB'D')$ nên có $1800+2500-250a=0 \Leftrightarrow a=17{,}2$.\\
	Vậy $a=17{,}2$ cm.
	}
\end{ex}
\Closesolutionfile{ans}
\indapan{10}{ans/ans-2C5B1CD3}
\Opensolutionfile{ans}[ans/ans-0-B15-KQ]
\TNSA
\begin{ex}%[2H2V2-6]
	Một sân vận động được xây dựng theo mô hình là hình chóp cụt $OAGD.BCFE$ có hai đáy song song với nhau. Mặt sân $OAGD$ là hình chữ nhật và được gắn hệ trục $Oxyz$ như hình vẽ dưới (đơn vị trên mỗi trục tọa độ là mét). Mặt sân $OAGD$ có chiều dài $OA=100 \mathrm{~m}$, chiều rộng $OD=60 \mathrm{~m}$ và tọa độ điểm $B(10;10;8)$. Tính khoảng cách từ điểm $G$ đến mặt phẳng $(OBED)$ (kết quả làm tròn đến hàng phần chục).
	\begin{center}
		\tikzset{every picture/.style={line width=0.75pt}}         
		\begin{tikzpicture}[x=0.75pt,y=0.75pt,yscale=-1,xscale=1]
			\draw (250,250) -- (250,77) ;
			\draw [shift={(250,75)}, rotate = 90] [color={rgb, 255:red, 0; green, 0; blue, 0 }  ][line width=0.75]    (10.93,-3.29) .. controls (6.95,-1.4) and (3.31,-0.3) .. (0,0) .. controls (3.31,0.3) and (6.95,1.4) .. (10.93,3.29)   ;
			\draw    (250,250) -- (398.1,200.63) ;
			\draw [shift={(400,200)}, rotate = 161.57] [color={rgb, 255:red, 0; green, 0; blue, 0 }  ][line width=0.75]    (10.93,-3.29) .. controls (6.95,-1.4) and (3.31,-0.3) .. (0,0) .. controls (3.31,0.3) and (6.95,1.4) .. (10.93,3.29)   ;
			\draw    (250,250) -- (126.86,200.74) ;
			\draw [shift={(125,200)}, rotate = 21.8] [color={rgb, 255:red, 0; green, 0; blue, 0 }  ][line width=0.75]    (10.93,-3.29) .. controls (6.95,-1.4) and (3.31,-0.3) .. (0,0) .. controls (3.31,0.3) and (6.95,1.4) .. (10.93,3.29)   ;
			\draw  [dash pattern={on 4.5pt off 4.5pt}]  (152,212) -- (275,175) ;
			\draw  [dash pattern={on 4.5pt off 4.5pt}]  (275,175) -- (375,207) ;
			\draw    (175,175) -- (152,212) ;
			\draw    (175,175) -- (275,212) ;
			\draw    (250,250) -- (275,212) ;
			\draw    (275,212) -- (350,186) ;
			\draw    (350,186) -- (375,207) ;
			\draw    (175,175) -- (275,150) ;
			\draw    (275,150) -- (350,186) ;
			\draw  [dash pattern={on 4.5pt off 4.5pt}]  (275,150) -- (275,175) ;
			
			\draw (402,203) node [anchor=north west][inner sep=0.75pt]   [align=left] {x};
			
			\draw (129,215) node [anchor=north west][inner sep=0.75pt]   [align=left] {y};
			
			\draw (252,78) node [anchor=north west][inner sep=0.75pt]   [align=left] {z};
			
			\draw (252,253) node [anchor=north west][inner sep=0.75pt]   [align=left] {O};
			
			\draw (377,210) node [anchor=north west][inner sep=0.75pt]   [align=left] {A};
			
			\draw (154,215) node [anchor=north west][inner sep=0.75pt]   [align=left] {D};
			
			\draw (264,177) node [anchor=north west][inner sep=0.75pt]   [align=left] {G};
			
			\draw (177,178) node [anchor=north west][inner sep=0.75pt]   [ below] {E};
			
			\draw (277,215) node [anchor=north west][inner sep=0.75pt]   [align=left] {B};
			
			\draw (351,163) node [anchor=north west][inner sep=0.75pt]   [align=left] {C};
			
			\draw (282,133) node [anchor=north west][inner sep=0.75pt]   [align=left] {F};		
		\end{tikzpicture}
	\end{center}
	\shortans{$62{,}5$}
	\loigiai{Gắn hình chóp cụt $OAGD.BCFE$ vào hệ trục $Oxyz$, ta có: $O(0;0;0), A(100;0;0), G(100; 60;0), \\D(0;60;0), B(10;10;8)$, $\overrightarrow{OD}=(0 ; 60 ; 0), \overrightarrow{OB}=(10 ; 10 ; 8)$.\\
		Véc-tơ pháp tuyến của mặt phẳng $(OBED)$ là $\vec{n}=[\overrightarrow{OD}, \overrightarrow{OB}]=(480 ; 0 ;-600) = 120(4 ; 0 ;-5)$.\\
		Phương trình mặt phẳng $(OBED)$ đi qua điểm $O(0 ; 0 ; 0)$ và có véc-tơ pháp tuyến $\vec{n}=(4 ; 0 ;-5)$ là  $4x-5z=0$.\\
		Khoảng cách từ điểm $G$ đến mặt phẳng $(OBED)$ là $$\mathrm{d}(G,(O B E D))=\dfrac{|4\cdot 100-5\cdot 0|}{\sqrt{16+25}}=\dfrac{400 \sqrt{41}}{41} \approx 62{,}5.$$
	}
\end{ex}
\begin{ex}%[2H2V2-6]
	Một công trình đang xây dựng được gắn hệ trục $Oxyz$ như hình vẽ dưới (đơn vị trên mỗi trục tọa độ là mét). Mỗi cột bê tông có dạng hình lăng trụ tứ giác đều và có tâm của mặt đáy trên lần lợt là $A(3 ; 2 ; 3), B(6 ; 3 ; 3), C(9 ; 4 ; 2), D\left(6 ; 0 ; \dfrac{5}{2}\right)$. Tính khoảng cách từ điểm $D$ đến mặt phẳng $(ABC)$ (kết quả làm tròn tới hàng phần trăm).
	\begin{center}
		\includegraphics[width=0.5\linewidth]{image/h3.png}
	\end{center}
	\shortans{$2,85$}
	\loigiai{Ta có $\overrightarrow{AB}=(3;1;0); \overrightarrow{AC}=(6;2;-1)$. Phương trình mặt phẳng $(ABC)$ qua $A$ và có véc-tơ pháp tuyến $[\overrightarrow{AB},\overrightarrow{AC}]=(-1;3;0)$ là $-x + 3y - 3 = 0$. \\
		Khoảng cách từ $D$ tới mặt phẳng $(ABC)$ là $\mathrm{d}(D,(ABC))=\dfrac{\left|-6-3\right|}{\sqrt{1^2 + 3^2}}=\dfrac{9\sqrt{10}}{10} \approx 2{,}85$.
	}
\end{ex}

\begin{ex}%[2H2V2-6]
	Một công trình đang xây dựng được gắn hệ trục $Oxyz$ (đơn vị trên mỗi trục tọa độ là mét). Ba bức tường $(P),(Q),(R)$ (như hình vẽ) của tòa nhà lần lượt có phương trình $(P)\colon x+2y-2z+1=0$, $(Q)\colon 2x+y+2z-3=0,(R)\colon 2x+4y-4z-19=0$. Tính khoảng cách giữa hai bức tường $(P)$ và $(R)$ của tòa nhà.
	\begin{center}
		\includegraphics[width=0.5\linewidth]{image/h1.png}
	\end{center}
	\shortans{$3{,}5$}
	\loigiai{Tính khoảng cách giữa hai bức tường $(P)$ và $(R)$ của tòa nhà.\\
		Chọn điểm $M(-1 ; 0 ; 0) \in(P)$. Do hai bức tường $(P)$ và $(R)$ song song nhau nên \\
		$$\mathrm{d}((P),(R))=\mathrm{d}(M,(R))=\dfrac{|2\cdot (-1)+4\cdot 0-4\cdot 0-19|}{\sqrt{4+16+16}}=\dfrac{21}{6}=3{,}5 \text{m.}$$}
\end{ex}


\begin{ex}%[2H2V2-6]
	Một công trình đang xây dựng được gắn hệ trục $O x y z$ (đơn vị trên mỗi trục tọa độ là mét). Ba bức tường $(P),(Q),(R),(T)$ (như hình vẽ) của tòa nhà lần lượt có phương trình $(P)\colon 2x-y-z+1=0$, $(Q)\colon x+3y-z-2=0,(R)\colon 4x-2y-2 z+9=0,(T)\colon 2x+6y-2z+15=0$. Tính chiều rộng bức tường $(Q)$ của tòa nhà (kết quả làm tròn đến hàng phần chục).
	\begin{figure}[!ht]
		\centering
		\includegraphics[width=0.5\linewidth]{image/h2.png}
	\end{figure}
	\shortans{$2{,}9$}
	\loigiai{
		Do hai bức tường $(P)$ và $(R)$ song song nhau nên chiều rộng bức tường $(Q)$ là khoảng cách giữa hai bức tường $(P)$ và $(R)$. Chọn điểm $N(0 ; 0 ; 1) \in(P)$.\\
		Do hai bức tường $(P)$ và $(R)$ song song nhau nên $$\mathrm{d}((P),(R))=\mathrm{d}(N,(R))=\dfrac{|4\cdot 0-2\cdot 0-2\cdot 1+9|}{\sqrt{4+1+1}}=\dfrac{7}{\sqrt{6}} \approx 2{,}9.$$}
\end{ex}
\begin{ex}%[2H2V2-6]
	Cho hình lập phương $ABCD.A'B'C'D'$ có độ dài cạnh bằng 1. Gọi $M, N, P, Q$ lần lượt là trung điểm của $AB, BC, C'D',DD'$. Chọn hệ tọa độ $Oxyz$ như hình vẽ, xác định tọa độ các điểm $M, N, P, Q$. 
	Tính khoảng cách từ điểm $Q$ đến mặt phẳng $(MNP)$. Kết quả làm tròn đến hàng phần chục.
	\begin{center}
		\tikzset{every picture/.style={line width=0.75pt}} %set default line width to 0.75pt        
		\begin{tikzpicture}[x=0.75pt,y=0.75pt,yscale=-1,xscale=1]
			%uncomment if require: \path (0,797); %set diagram left start at 0, and has height of 797
			%Straight Lines [id:da8382470567419189] 
			\draw    (250,400) -- (250,327) ;
			\draw [shift={(250,325)}, rotate = 90] [color={rgb, 255:red, 0; green, 0; blue, 0 }  ][line width=0.75]    (10.93,-3.29) .. controls (6.95,-1.4) and (3.31,-0.3) .. (0,0) .. controls (3.31,0.3) and (6.95,1.4) .. (10.93,3.29)   ;
			%Straight Lines [id:da22849660996364696] 
			\draw    (175,575) -- (375,575) ;
			%Straight Lines [id:da3429340193043935] 
			\draw    (375,575) -- (450,500) ;
			%Straight Lines [id:da6171158479105385] 
			\draw    (250,400) -- (175,475) ;
			%Straight Lines [id:da20159315767924957] 
			\draw    (175,475) -- (175,575) ;
			%Straight Lines [id:da09157475506424606] 
			\draw    (175,475) -- (375,475) ;
			%Straight Lines [id:da7794866428713987] 
			\draw    (375,475) -- (375,575) ;
			%Straight Lines [id:da5730802129402581] 
			\draw    (375,475) -- (450,400) ;
			%Straight Lines [id:da4069170044141508] 
			\draw    (250,400) -- (450,400) ;
			%Straight Lines [id:da7895384060713708] 
			\draw    (450,400) -- (450,500) ;
			%Straight Lines [id:da3822968918446086] 
			\draw    (212.5,437.5) -- (350,400) ;
			%Straight Lines [id:da5114427996681619] 
			\draw  [dash pattern={on 4.5pt off 4.5pt}]  (350,400) -- (412.5,537.5) ;
			%Straight Lines [id:da5639353335968631] 
			\draw  [dash pattern={on 4.5pt off 4.5pt}]  (212.5,437.5) -- (412.5,537.5) ;
			%Straight Lines [id:da7141193916967588] 
			\draw  [dash pattern={on 4.5pt off 4.5pt}]  (350,400) -- (375,525) ;
			%Straight Lines [id:da6592626042182406] 
			\draw  [dash pattern={on 4.5pt off 4.5pt}]  (250,500) -- (175,575) ;
			%Straight Lines [id:da4180570193860802] 
			\draw    (175,575) -- (126.41,623.59) ;
			\draw [shift={(125,625)}, rotate = 315] [color={rgb, 255:red, 0; green, 0; blue, 0 }  ][line width=0.75]    (10.93,-3.29) .. controls (6.95,-1.4) and (3.31,-0.3) .. (0,0) .. controls (3.31,0.3) and (6.95,1.4) .. (10.93,3.29)   ;
			%Straight Lines [id:da9167552187295651] 
			\draw  [dash pattern={on 4.5pt off 4.5pt}]  (250,500) -- (450,500) ;
			%Straight Lines [id:da8527980278686933] 
			\draw    (450,500) -- (523,500) ;
			\draw [shift={(525,500)}, rotate = 180] [color={rgb, 255:red, 0; green, 0; blue, 0 }  ][line width=0.75]    (10.93,-3.29) .. controls (6.95,-1.4) and (3.31,-0.3) .. (0,0) .. controls (3.31,0.3) and (6.95,1.4) .. (10.93,3.29)   ;
			%Straight Lines [id:da8000503760713495] 
			\draw  [dash pattern={on 4.5pt off 4.5pt}]  (250,400) -- (250,500) ;
			\draw (177,478) node [anchor=north west][inner sep=0.75pt]   [align=left] {A};
			
			\draw (252,403) node [anchor=north west][inner sep=0.75pt]   [align=left] {B};
			
			\draw (452,403) node [anchor=north west][inner sep=0.75pt]   [align=left] {C};
			
			\draw (376,481) node [anchor=north west][inner sep=0.75pt]   [align=left] {D};
			
			\draw (177,578) node [anchor=north west][inner sep=0.75pt]   [align=left] {A'};
			
			\draw (252,503) node [anchor=north west][inner sep=0.75pt]   [align=left] {B'};
			
			\draw (451,506) node [anchor=north west][inner sep=0.75pt]   [align=left] {C'};
			
			\draw (377,578) node [anchor=north west][inner sep=0.75pt]   [align=left] {D'};
			
			\draw (195,419) node [anchor=north west][inner sep=0.75pt]   [align=left] {M};
			
			\draw (351,381) node [anchor=north west][inner sep=0.75pt]   [align=left] {N};
			
			\draw (414.5,540.5) node [anchor=north west][inner sep=0.75pt]   [align=left] {P};
			
			\draw (351,527) node [anchor=north west][inner sep=0.75pt]   [align=left] {Q};
			
			\draw (513,518) node [anchor=north west][inner sep=0.75pt]   [align=left] {x};
			
			\draw (140,620) node [anchor=north west][inner sep=0.75pt]   [align=left] {y};
			
			\draw (263,331) node [anchor=north west][inner sep=0.75pt]   [align=left] {z};
		\end{tikzpicture}
	\end{center}
	\shortans{$1,4$}
	\loigiai{Thiết lập hệ tọa độ $O x y z$ như hình vẽ, gốc $O \equiv B'$. Khi đó $M\left(0 ; \dfrac{1}{2} ; 1\right), N\left(\dfrac{1}{2} ; 0 ; 1\right), P\left(1 ; \dfrac{1}{2} ; 0\right),\\
		Q\left(1 ; 1 ; \dfrac{1}{2}\right)$. Phương trình mặt phẳng $(MNP)$ đi qua $M\left(0;\dfrac{1}{2};1\right)$ và có véc-tơ pháp tuyến $[\overrightarrow {MN} ,\overrightarrow {MP}]=\left( \dfrac{1}{{2}};\dfrac{1}{{2}};\dfrac{1}{{2}}\right)$ là $2x + 2y + 2z - 3 = 0$.\\
		Khoảng cách từ điểm $Q$ đến mặt phẳng $(MNP)$ là $$\mathrm{d}(Q;(MNP)=\dfrac{\left|2\cdot 1+2\cdot 1+2\cdot \dfrac{1}{2}\right|}{\sqrt{2^2 +2^2 +2^2}}= \dfrac{5\sqrt{3}}{6}\approx 1{,}4.$$
	}
\end{ex}
\begin{ex}%[2H2V2-6]
	Cho hình chóp $S.ABCD$ có đáy $ABCD$ là hình vuông cạnh $a, SAD$ là tam giác đều và nằm trong mặt phẳng với đáy. Gọi $M$ và $N$ lần lượt là trung điểm của $BC$ và $CD$. Chọn hệ tọa độ $Oxyz$ như hình vẽ dưới. Gọi $Q$ là trung điểm $S D$. Tính khoảng cách giữa hai mặt phẳng $(SAC)$ và mặt phẳng $(ONQ)$ (kết quả làm tròn đến hàng phần chục).
	\begin{center}
		% Gradient Info
		\tikzset {_0pmvymc3o/.code = {\pgfsetadditionalshadetransform{ \pgftransformshift{\pgfpoint{-198 bp } { 158.4 bp }  }  \pgftransformscale{1.32 }  }}}
		\pgfdeclareradialshading{_ffechrcqo}{\pgfpoint{160bp}{-128bp}}{rgb(0bp)=(0,0,0);
			rgb(0bp)=(0,0,0);
			rgb(6.785714285714286bp)=(0,0,0);
			rgb(14.017857142857142bp)=(0,0,0);
			rgb(20bp)=(0,0,0);
			rgb(25bp)=(1,1,1);
			rgb(400bp)=(1,1,1)}
		\tikzset{every picture/.style={line width=0.75pt}} %set default line width to 0.75pt        
		\begin{tikzpicture}[x=0.75pt,y=0.75pt,yscale=-1,xscale=1]
			%uncomment if require: \path (0,300); %set diagram left start at 0, and has height of 300
			%Straight Lines [id:da1986885226481041] 
			\draw    (108,216) -- (297,216) ;
			\draw [shift={(202.5,216)}, rotate = 0] [color={rgb, 255:red, 0; green, 0; blue, 0 }  ][line width=0.75]      (0, 0) circle [x radius= 1.34, y radius= 1.34]   ;
			%Straight Lines [id:da8495468285480916] 
			\draw [color={rgb, 255:red, 0; green, 0; blue, 0 }  ,draw opacity=1 ][shading=_ffechrcqo,_0pmvymc3o]   (135,54) -- (171,63) -- (207,72) -- (243,81) -- (279,90) -- (315,99) -- (351,108) ;
			%Straight Lines [id:da7064838767550801] 
			\draw  [dash pattern={on 4.5pt off 4.5pt}]  (135,162) -- (324,162) ;
			%Straight Lines [id:da7232875579325917] 
			\draw  [dash pattern={on 4.5pt off 4.5pt}]  (162,108) -- (108,216) ;
			%Straight Lines [id:da6308494004055998] 
			\draw  [dash pattern={on 4.5pt off 4.5pt}]  (162,108) -- (351,108) ;
			%Straight Lines [id:da6682204451648537] 
			\draw    (351,108) -- (297,216) ;
			%Straight Lines [id:da3749747303473572] 
			\draw  [dash pattern={on 4.5pt off 4.5pt}]  (135,54) -- (135,162) ;
			%Straight Lines [id:da8031053998975075] 
			\draw    (135,54) -- (108,216) ;
			\draw [shift={(121.5,135)}, rotate = 99.46] [color={rgb, 255:red, 0; green, 0; blue, 0 }  ][line width=0.75]      (0, 0) circle [x radius= 1.34, y radius= 1.34]   ;
			%Straight Lines [id:da6847944951979155] 
			\draw    (135,54) -- (297,216) ;
			%Straight Lines [id:da6134510126128467] 
			\draw  [dash pattern={on 4.5pt off 4.5pt}]  (135,54) -- (162,108) ;
			%Straight Lines [id:da8173261646582135] 
			\draw    (324,162) -- (376,162) ;
			\draw [shift={(378,162)}, rotate = 180] [color={rgb, 255:red, 0; green, 0; blue, 0 }  ][line width=0.75]    (10.93,-3.29) .. controls (6.95,-1.4) and (3.31,-0.3) .. (0,0) .. controls (3.31,0.3) and (6.95,1.4) .. (10.93,3.29)   ;
			%Straight Lines [id:da8552794618253725] 
			\draw    (108,216) -- (81.89,268.21) ;
			\draw [shift={(81,270)}, rotate = 296.57] [color={rgb, 255:red, 0; green, 0; blue, 0 }  ][line width=0.75]    (10.93,-3.29) .. controls (6.95,-1.4) and (3.31,-0.3) .. (0,0) .. controls (3.31,0.3) and (6.95,1.4) .. (10.93,3.29)   ;
			%Straight Lines [id:da44029097674743123] 
			\draw    (135,54) -- (135,2) ;
			\draw [shift={(135,0)}, rotate = 90] [color={rgb, 255:red, 0; green, 0; blue, 0 }  ][line width=0.75]    (10.93,-3.29) .. controls (6.95,-1.4) and (3.31,-0.3) .. (0,0) .. controls (3.31,0.3) and (6.95,1.4) .. (10.93,3.29)   ;
			\draw (164,111) node [anchor=north west][inner sep=0.75pt]   [align=left] {A};
			
			\draw (353,111) node [anchor=north west][inner sep=0.75pt]   [align=left] {B};
			
			\draw (299,219) node [anchor=north west][inner sep=0.75pt]   [align=left] {C};
			
			\draw (110,219) node [anchor=north west][inner sep=0.75pt]   [align=left] {D};
			
			\draw (326,165) node [anchor=north west][inner sep=0.75pt]   [align=left] {M};
			
			\draw (137,165) node [anchor=north west][inner sep=0.75pt]   [align=left] {O};
			
			\draw (204.5,219) node [anchor=north west][inner sep=0.75pt]   [align=left] {N};
			
			\draw (366,164) node [anchor=north west][inner sep=0.75pt]   [align=left] {x};
			
			\draw (96,251) node [anchor=north west][inner sep=0.75pt]   [align=left] {y};
			
			\draw (136,8) node [anchor=north west][inner sep=0.75pt]   [align=left] {z};
			
			\draw (122,35) node [anchor=north west][inner sep=0.75pt]   [align=left] {S};
			
			\draw (97.5,137) node [anchor=north west][inner sep=0.75pt]   [align=left] {Q};
		\end{tikzpicture}
	\end{center}
	\shortans{$0{,}3$}
	\loigiai{
		Với hệ trục toạ độ như hình vẽ ta có $S\left(0 ; 0 ; \dfrac{a \sqrt{3}}{2}\right) ; M(a ; 0 ; 0) ; N\left(\dfrac{a}{2} ; \dfrac{a}{2} ; 0\right); A(0;-\dfrac{a}{2};0);$\\
		$B\left(a;-\dfrac{-a}{2};0\right); C\left(a; \dfrac{a}{2};0\right); D\left(0;\dfrac{a}{2};0\right);Q\left(0;\dfrac{a}{4};\dfrac{a\sqrt{3}}{4}\right)$. \\
		Lấy $a=1.$ Mặt phẳng $(SAC)$ qua $A$ và có véc-tơ pháp tuyến $[\overrightarrow{SA},\overrightarrow{AC}]$ là
		$2\sqrt3 x - 2\sqrt3 y + 2z - \sqrt3 = 0.$\\
		Khoảng cách cần tìm 
		$$\mathrm{d}((SAC);(OQN))=\mathrm{d}(O;(SAC))= \dfrac{\sqrt{21}}{14}\approx 0{,}3.$$
	}
\end{ex}
\begin{ex}%[2H2V2-6]
	Cho tứ diện $OABC$, có $OA, OB, OC$ đôi một vuông góc và $OA=5, OB=2, OC=4$. Gọi $M, N$ lần lượt là trung điểm của $OB$ và $OC$. Chọn hệ tọa độ $Oxyz$ như hình vẽ dưới. Tính khoảng cách từ điểm $B$ đến mặt phẳng $(AMN)$. Kết quả làm tròn đến hàng phần chục.
	\begin{center}		
		
		\tikzset{every picture/.style={line width=0.75pt}} %set default line width to 0.75pt        
		
		\begin{tikzpicture}[x=0.75pt,y=0.75pt,yscale=-1,xscale=1]
			%uncomment if require: \path (0,300); %set diagram left start at 0, and has height of 300
			
			%Straight Lines [id:da3679447504076472] 
			\draw  [dash pattern={on 4.5pt off 4.5pt}]  (252,140) -- (364,140) ;
			%Straight Lines [id:da13962597702284096] 
			\draw  [dash pattern={on 4.5pt off 4.5pt}]  (252,140) -- (196,196) ;
			%Straight Lines [id:da8580092790638094] 
			\draw  [dash pattern={on 4.5pt off 4.5pt}]  (252,56) -- (252,140) ;
			%Straight Lines [id:da7130062908773143] 
			\draw    (252,56) -- (196,196) ;
			%Straight Lines [id:da012778923343735649] 
			\draw    (252,56) -- (364,140) ;
			%Straight Lines [id:da3173437648870021] 
			\draw    (196,196) -- (364,140) ;
			%Straight Lines [id:da528126976168177] 
			\draw  [dash pattern={on 4.5pt off 4.5pt}]  (224,168) -- (308,140) ;
			%Straight Lines [id:da44405621585220234] 
			\draw  [dash pattern={on 4.5pt off 4.5pt}]  (252,56) -- (308,140) ;
			%Straight Lines [id:da4474952587057264] 
			\draw  [dash pattern={on 4.5pt off 4.5pt}]  (252,56) -- (224,168) ;
			%Straight Lines [id:da6666103459826491] 
			\draw    (252,56) -- (252,30) ;
			\draw [shift={(252,28)}, rotate = 90] [color={rgb, 255:red, 0; green, 0; blue, 0 }  ][line width=0.75]    (10.93,-3.29) .. controls (6.95,-1.4) and (3.31,-0.3) .. (0,0) .. controls (3.31,0.3) and (6.95,1.4) .. (10.93,3.29)   ;
			%Straight Lines [id:da6450164393512425] 
			\draw    (364,140) -- (390,140) ;
			\draw [shift={(392,140)}, rotate = 180] [color={rgb, 255:red, 0; green, 0; blue, 0 }  ][line width=0.75]    (10.93,-3.29) .. controls (6.95,-1.4) and (3.31,-0.3) .. (0,0) .. controls (3.31,0.3) and (6.95,1.4) .. (10.93,3.29)   ;
			%Straight Lines [id:da9897161761780247] 
			\draw    (196,196) -- (169.41,222.59) ;
			\draw [shift={(168,224)}, rotate = 315] [color={rgb, 255:red, 0; green, 0; blue, 0 }  ][line width=0.75]    (10.93,-3.29) .. controls (6.95,-1.4) and (3.31,-0.3) .. (0,0) .. controls (3.31,0.3) and (6.95,1.4) .. (10.93,3.29)   ;
			
			
			\draw (169,226) node [anchor=north west][inner sep=0.75pt]   [align=left] {x};
			
			\draw (380,142) node [anchor=north west][inner sep=0.75pt]   [align=left] {y};
			
			\draw (268,20) node [anchor=north west][inner sep=0.75pt]   [align=left] {z};
			
			\draw (226.8,46.8) node [anchor=north west][inner sep=0.75pt]   [align=left] {A};
			
			\draw (198,199) node [anchor=north west][inner sep=0.75pt]   [align=left] {B};
			
			\draw (360.2,149) node [anchor=north west][inner sep=0.75pt]   [align=left] {C};
			
			\draw (253,124.6) node [anchor=north west][inner sep=0.75pt]   [align=left] {O};
			
			\draw (209.8,155.6) node [anchor=north west][inner sep=0.75pt]   [align=left] {M};
			
			\draw (309,121) node [anchor=north west][inner sep=0.75pt]   [align=left] {N};
			
			
		\end{tikzpicture}
	\end{center}	
	\shortans{$0{,}9$}
	\loigiai{Chọn hệ trục tọa độ Oxyz như hình vẽ.
		
		\begin{center}
			
			\tikzset{every picture/.style={line width=0.75pt}} %set default line width to 0.75pt        
			
			\begin{tikzpicture}[x=0.75pt,y=0.75pt,yscale=-1,xscale=1]
				
				
				%Straight Lines [id:da3679447504076472] 
				\draw  [dash pattern={on 4.5pt off 4.5pt}]  (252,140) -- (364,140) ;
				%Straight Lines [id:da13962597702284096] 
				\draw  [dash pattern={on 4.5pt off 4.5pt}]  (252,140) -- (196,196) ;
				%Straight Lines [id:da8580092790638094] 
				\draw  [dash pattern={on 4.5pt off 4.5pt}]  (252,56) -- (252,140) ;
				%Straight Lines [id:da7130062908773143] 
				\draw    (252,56) -- (196,196) ;
				%Straight Lines [id:da012778923343735649] 
				\draw    (252,56) -- (364,140) ;
				%Straight Lines [id:da3173437648870021] 
				\draw    (196,196) -- (364,140) ;
				%Straight Lines [id:da528126976168177] 
				\draw  [dash pattern={on 4.5pt off 4.5pt}]  (224,168) -- (308,140) ;
				%Straight Lines [id:da44405621585220234] 
				\draw  [dash pattern={on 4.5pt off 4.5pt}]  (252,56) -- (308,140) ;
				%Straight Lines [id:da4474952587057264] 
				\draw  [dash pattern={on 4.5pt off 4.5pt}]  (252,56) -- (224,168) ;
				%Straight Lines [id:da6666103459826491] 
				\draw    (252,56) -- (252,30) ;
				\draw [shift={(252,28)}, rotate = 90] [color={rgb, 255:red, 0; green, 0; blue, 0 }  ][line width=0.75]    (10.93,-3.29) .. controls (6.95,-1.4) and (3.31,-0.3) .. (0,0) .. controls (3.31,0.3) and (6.95,1.4) .. (10.93,3.29)   ;
				%Straight Lines [id:da6450164393512425] 
				\draw    (364,140) -- (390,140) ;
				\draw [shift={(392,140)}, rotate = 180] [color={rgb, 255:red, 0; green, 0; blue, 0 }  ][line width=0.75]    (10.93,-3.29) .. controls (6.95,-1.4) and (3.31,-0.3) .. (0,0) .. controls (3.31,0.3) and (6.95,1.4) .. (10.93,3.29)   ;
				%Straight Lines [id:da9897161761780247] 
				\draw    (196,196) -- (169.41,222.59) ;
				\draw [shift={(168,224)}, rotate = 315] [color={rgb, 255:red, 0; green, 0; blue, 0 }  ][line width=0.75]    (10.93,-3.29) .. controls (6.95,-1.4) and (3.31,-0.3) .. (0,0) .. controls (3.31,0.3) and (6.95,1.4) .. (10.93,3.29)   ;
				
				
				\draw (169,226) node [anchor=north west][inner sep=0.75pt]   [align=left] {x};
				
				\draw (380,142) node [anchor=north west][inner sep=0.75pt]   [align=left] {y};
				
				\draw (268,20) node [anchor=north west][inner sep=0.75pt]   [align=left] {z};
				
				\draw (226.8,46.8) node [anchor=north west][inner sep=0.75pt]   [align=left] {A};
				
				\draw (198,199) node [anchor=north west][inner sep=0.75pt]   [align=left] {B};
				
				\draw (360.2,149) node [anchor=north west][inner sep=0.75pt]   [align=left] {C};
				
				\draw (253,124.6) node [anchor=north west][inner sep=0.75pt]   [align=left] {O};
				
				\draw (209.8,155.6) node [anchor=north west][inner sep=0.75pt]   [align=left] {M};
				
				\draw (309,121) node [anchor=north west][inner sep=0.75pt]   [align=left] {N};
				
				
			\end{tikzpicture}
		\end{center}
		Ta có $O(0 ; 0 ; 0), A \in {Oz}, B \in O x, C \in O y$ sao cho $A O=5, OB=2, OC=4\Rightarrow A(0 ; 0 ; 5), B(2 ; 0 ; 0), C(0 ; 4 ; 0)$. $M$ là trung điểm $OB$ nên $M(1 ; 0 ; 0)$. $N$ là trung điểm $OC$ nên $N(0 ; 2 ; 0)$.\\
		Phương trình mặt phẳng $(AMN)$ qua $A$ và có véc-tơ pháp tuyến $[\overrightarrow{AM},\overrightarrow{AN}]=(10;5;2)$ là $10x + 5y + 2z - 10 = 0$.\\
		Ta có $\mathrm{d}(B;(AMN))=\mathrm{d}(O;(AMN))=\dfrac{10}{\sqrt{129}}\approx 0{,}9$.
		
	}
\end{ex}

\begin{ex}%[2H2V2-6]
	Cho hình chóp $S.ABCD$ đáy là hình thang vuông tại $A$ và $D, SA \perp(ABCD)$. Góc giữa $SB$ và mặt phẳng đáy bằng $45^{\circ}, E$ là trung điểm của $SD, AB=2a, AD = DC = a$. Chọn hệ tọa độ $Oxyz$ như hình vẽ dưới. Tính khoảng cách từ điểm $B$ đến mặt phẳng $(AEC)$ (kết quả làm tròn đến hàng phần chục).
	\begin{center}	
		
		\tikzset{every picture/.style={line width=0.75pt}} %set default line width to 0.75pt        
		
		\begin{tikzpicture}[x=0.75pt,y=0.75pt,yscale=-1,xscale=1]
			%uncomment if require: \path (0,300); %set diagram left start at 0, and has height of 300
			
			%Straight Lines [id:da8913682824234928] 
			\draw  [dash pattern={on 4.5pt off 4.5pt}]  (252,140) -- (364,140) ;
			%Straight Lines [id:da039948733048584595] 
			\draw  [dash pattern={on 4.5pt off 4.5pt}]  (252,140) -- (196,196) ;
			%Straight Lines [id:da6718633500675624] 
			\draw  [dash pattern={on 4.5pt off 4.5pt}]  (252,56) -- (252,140) ;
			%Straight Lines [id:da9759242944770992] 
			\draw    (252,56) -- (196,196) ;
			\draw [shift={(224,126)}, rotate = 111.8] [color={rgb, 255:red, 0; green, 0; blue, 0 }  ][line width=0.75]      (0, 0) circle [x radius= 1.34, y radius= 1.34]   ;
			%Straight Lines [id:da563221306551871] 
			\draw    (252,56) -- (364,140) ;
			%Straight Lines [id:da15516892739193677] 
			\draw    (280,196) -- (364,140) ;
			%Straight Lines [id:da7881469638756575] 
			\draw    (252,56) -- (252,30) ;
			\draw [shift={(252,28)}, rotate = 90] [color={rgb, 255:red, 0; green, 0; blue, 0 }  ][line width=0.75]    (10.93,-3.29) .. controls (6.95,-1.4) and (3.31,-0.3) .. (0,0) .. controls (3.31,0.3) and (6.95,1.4) .. (10.93,3.29)   ;
			%Straight Lines [id:da6714011604516419] 
			\draw    (364,140) -- (390,140) ;
			\draw [shift={(392,140)}, rotate = 180] [color={rgb, 255:red, 0; green, 0; blue, 0 }  ][line width=0.75]    (10.93,-3.29) .. controls (6.95,-1.4) and (3.31,-0.3) .. (0,0) .. controls (3.31,0.3) and (6.95,1.4) .. (10.93,3.29)   ;
			%Straight Lines [id:da5385347511589023] 
			\draw    (196,196) -- (169.41,222.59) ;
			\draw [shift={(168,224)}, rotate = 315] [color={rgb, 255:red, 0; green, 0; blue, 0 }  ][line width=0.75]    (10.93,-3.29) .. controls (6.95,-1.4) and (3.31,-0.3) .. (0,0) .. controls (3.31,0.3) and (6.95,1.4) .. (10.93,3.29)   ;
			%Straight Lines [id:da05923631932446871] 
			\draw    (196,196) -- (280,196) ;
			%Straight Lines [id:da8592722804571848] 
			\draw    (252,56) -- (280,196) ;
			%Straight Lines [id:da802931795948812] 
			\draw  [dash pattern={on 4.5pt off 4.5pt}]  (252,140) -- (280,196) ;
			%Straight Lines [id:da4994838123991385] 
			\draw  [dash pattern={on 4.5pt off 4.5pt}]  (224,126) -- (252,140) ;
			%Straight Lines [id:da23640159060964017] 
			\draw  [dash pattern={on 4.5pt off 4.5pt}]  (224,126) -- (280,196) ;
			
			
			\draw (169,226) node [anchor=north west][inner sep=0.75pt]   [align=left] {x};
			
			\draw (380,142) node [anchor=north west][inner sep=0.75pt]   [align=left] {y};
			
			\draw (268,20) node [anchor=north west][inner sep=0.75pt]   [align=left] {z};
			
			\draw (238,44.4) node [anchor=north west][inner sep=0.75pt]   [align=left] {S};
			
			\draw (198,199) node [anchor=north west][inner sep=0.75pt]   [align=left] {D};
			
			\draw (360.2,149) node [anchor=north west][inner sep=0.75pt]   [align=left] {B};
			
			\draw (253,124.6) node [anchor=north west][inner sep=0.75pt]   [align=left] {A};
			
			\draw (281,198) node [anchor=north west][inner sep=0.75pt]   [align=left] {C};
			
			\draw (211,114) node [anchor=north west][inner sep=0.75pt]   [align=left] {E};
			
			
		\end{tikzpicture}
	\end{center}
	\shortans{$1{,}3$}
	\loigiai{Lấy $a=1$. Ta có $(SB,(ABCD))=\widehat{SBA}=45^0 \Rightarrow \triangle ASB$ vuông cân tại $A.$ Suy ra $SA=AB=2$.\\
		Ta có $A(0;0;0); S(0;0;2); C(1;1;0); B(0;2;0); D(1;0;0); E\left(\dfrac{1}{2};0;1\right)$.\\
		Phương trình mặt phẳng $(AEC)$ qua $A$ và có véc-tơ pháp tuyến $[\overrightarrow{AE}, \overrightarrow{AC}]=\left(-1;1;\dfrac{1}{2}\right)$ là $-2x + 2y + z = 0$.\\
		Khoảng cách từ điểm $B$ đến mặt phẳng $(AEC)$ là 
		$$\mathrm{d}(B,(AEC))=\dfrac{|2\cdot 2|}{\sqrt{2^2 +2^2 +1^2}}=\dfrac{4}{3}\approx 1{,}3.$$}
\end{ex}

\begin{ex}%[2H2V2-6]
	Trong không gian với hệ trục tọa độ $Oxyz$, cho bốn điểm $S(-1 ; 6 ; 2), A(0 ; 0 ; 6),\\ B(0 ; 3 ; 0)$, $C(-2 ; 0 ; 0)$. Gọi $H$ là chân đường cao vẽ từ $S$ của tứ diện $S.ABC$. Giả sử phương trình mặt phẳng đi qua ba điểm $S,B,H$ có dạng $x+by+cz+d=0$ với $b,c,d \in \mathbb{Z}$. Tính $b+c+d.$
	\shortans{$-17$}
	\loigiai{Phương trình mặt phẳng $(ABC): \dfrac{x}{-2}+\dfrac{y}{3}+\dfrac{z}{6}=1 \Leftrightarrow-3 x+2 y+z-6=0$. \\
		$H$ là chân đường cao vẽ từ $S$ của tứ diện $S.ABC$ nên $H$ là hình chiếu vuông góc của $S$ lên mặt phẳng $(A B C) \Rightarrow H\left(\dfrac{19}{14} ; \dfrac{31}{7} ; \dfrac{17}{14}\right)$.\\
		Mặt phẳng $(SBH)$ qua $B(0;3;0)$ và có véc-tơ pháp tuyến $$[\overrightarrow{BH}, \overrightarrow{SB}]=\left(\dfrac{11}{14} ; \dfrac{55}{14} ;-\dfrac{11}{2}\right)=\dfrac{11}{14}(1 ; 5 ;-7).$$
		Phương trình mặt phẳng $(SBH)$ là $ x+5(y-3)-7 z=0\\ \Leftrightarrow x+5 y-7 z-15=0$. Ta có $b+c+d =-17.$
	}
\end{ex}
\begin{ex}%[2H2V2-6]
	Trong KG $Oxyz$, cho hình chóp $S.ABCD$, đáy $ABCD$ là hình chữ nhật. Biết $A(0 ; 0 ; 0), D(2 ; 0 ; 0), B(0 ; 4 ; 0), S(0 ; 0 ; 4)$. Gọi $M$ là trung điểm của $SB$ và $G$ là trọng tâm của tam giác $SCD$. Tính khoảng cách từ điểm $B$ đến mặt phẳng $(AMG)$. Kết quả làm tròn đến hàng phần chục.
	\shortans{$2{,}8$}
	\loigiai{
		\begin{center}
			\tikzset{every picture/.style={line width=0.75pt}} %set default line width to 0.75pt        
			\begin{tikzpicture}[x=0.75pt,y=0.75pt,yscale=-1,xscale=1]
				%uncomment if require: \path (0,300); %set diagram left start at 0, and has height of 300
				
				%Straight Lines [id:da8697087570722153] 
				\draw  [dash pattern={on 4.5pt off 4.5pt}]  (252,140) -- (364,140) ;
				%Straight Lines [id:da4072889759548881] 
				\draw  [dash pattern={on 4.5pt off 4.5pt}]  (252,140) -- (196,196) ;
				%Straight Lines [id:da5044623873055065] 
				\draw  [dash pattern={on 4.5pt off 4.5pt}]  (252,56) -- (252,140) ;
				%Straight Lines [id:da6764918883827946] 
				\draw    (252,56) -- (196,196) ;
				%Straight Lines [id:da4329647923634008] 
				\draw    (252,56) -- (308,98) ;
				%Straight Lines [id:da011038461362765872] 
				\draw    (308.27,98.2) -- (364,140) ;
				\draw [shift={(308,98)}, rotate = 36.87] [color={rgb, 255:red, 0; green, 0; blue, 0 }  ][line width=0.75]      (0, 0) circle [x radius= 1.34, y radius= 1.34]   ;
				%Straight Lines [id:da5196781175343534] 
				\draw    (308,196) -- (364,140) ;
				%Straight Lines [id:da3160409919917737] 
				\draw    (252,56) -- (252,30) ;
				\draw [shift={(252,28)}, rotate = 90] [color={rgb, 255:red, 0; green, 0; blue, 0 }  ][line width=0.75]    (10.93,-3.29) .. controls (6.95,-1.4) and (3.31,-0.3) .. (0,0) .. controls (3.31,0.3) and (6.95,1.4) .. (10.93,3.29)   ;
				%Straight Lines [id:da9102255389540188] 
				\draw    (364,140) -- (390,140) ;
				\draw [shift={(392,140)}, rotate = 180] [color={rgb, 255:red, 0; green, 0; blue, 0 }  ][line width=0.75]    (10.93,-3.29) .. controls (6.95,-1.4) and (3.31,-0.3) .. (0,0) .. controls (3.31,0.3) and (6.95,1.4) .. (10.93,3.29)   ;
				%Straight Lines [id:da3969763351140396] 
				\draw    (196,196) -- (169.41,222.59) ;
				\draw [shift={(168,224)}, rotate = 315] [color={rgb, 255:red, 0; green, 0; blue, 0 }  ][line width=0.75]    (10.93,-3.29) .. controls (6.95,-1.4) and (3.31,-0.3) .. (0,0) .. controls (3.31,0.3) and (6.95,1.4) .. (10.93,3.29)   ;
				%Straight Lines [id:da06997294598351322] 
				\draw    (196,196) -- (308,196) ;
				%Straight Lines [id:da06948427828035464] 
				\draw    (252,56) -- (308,196) ;
				
				
				\draw (169,226) node [anchor=north west][inner sep=0.75pt]   [align=left] {x};
				
				\draw (380,142) node [anchor=north west][inner sep=0.75pt]   [align=left] {y};
				
				\draw (268,20) node [anchor=north west][inner sep=0.75pt]   [align=left] {z};
				
				\draw (238,44.4) node [anchor=north west][inner sep=0.75pt]   [align=left] {S};
				
				\draw (198,199) node [anchor=north west][inner sep=0.75pt]   [align=left] {D};
				
				\draw (360.2,149) node [anchor=north west][inner sep=0.75pt]   [align=left] {B};
				
				\draw (253,124.6) node [anchor=north west][inner sep=0.75pt]   [align=left] {A};
				
				\draw (309,198) node [anchor=north west][inner sep=0.75pt]   [align=left] {C};
				
				\draw (314.6,82) node [anchor=north west][inner sep=0.75pt]   [align=left] {M};
			\end{tikzpicture}
		\end{center}
		Chọn hệ trục tọa độ như hình vẽ. Ta có $A(0 ; 0 ; 0), D(2 ; 0 ; 0), B(0 ; 4 ; 0), S(0 ; 0 ; 4)$.\\
		$M$ là trung điểm của $S B \Rightarrow M(0 ; 2 ; 2)$.\\
		Tứ giác $A B C D$ là hình chữ nhật nên $\left\{\begin{array}{l}x_{A}+x_{C}=x_{B}+x_{D} \\ y_{A}+y_{C}=y_{B}+y_{D} \\ z_{A}+z_{C}=z_{B}+z_{D}\end{array} \Rightarrow\left\{\begin{array}{l}x_{C}=2 \\ y_{C}=4 \\ z_{C}=0\end{array} \Rightarrow C(2 ; 4 ; 0)\right.\right.$.\\
		$G$ là trọng tâm của tam giác $S C D \Rightarrow G\left(\dfrac{4}{3}; \dfrac{4}{3} ; \dfrac{4}{3}\right)$.\\
		Phương trình mặt phẳng $(AMG)$ qua $A$ và có véc-tơ pháp tuyến $[\overrightarrow{AM},\overrightarrow{AG}]=\left(0;\dfrac{-8}{3};\dfrac{8}{3} \right)$ là $y-z=0.$\\
		Khoảng cách từ điểm B đến mặt phẳng $(AMG)$ là $\mathrm{d}(B,(AMG))= \dfrac{|4|}{\sqrt{1^2+1^2}}=\dfrac{4}{\sqrt{2}}\approx 2{,}8$.
	}
\end{ex}

\begin{ex}%[2H2V2-6]
	Cho hình hộp chữ nhật $ABCD \cdot A'B'C'D'$ có các kích thước $AB=4, AD=3, AA'=5$. Gọi $G$ là trọng tâm của tam giác $ACB'$. Gọi $m$ là khoảng cách từ điểm $G$ đến mặt phẳng $\left(AB'C\right)$ và $n$ là khoảng cách giữa hai mặt phẳng $(AB'D')$ và $(CB'D')$. Tính $m+n$.
	\shortans{$0$}
	\loigiai{
		\begin{center}
			\tikzset{every picture/.style={line width=0.75pt}} %set default line width to 0.75pt        
			\begin{tikzpicture}[x=0.75pt,y=0.75pt,yscale=-1,xscale=1]
				%Straight Lines [id:da09989011048457908] 
				\draw    (250,400) -- (250,327) ;
				\draw [shift={(250,325)}, rotate = 90] [color={rgb, 255:red, 0; green, 0; blue, 0 }  ][line width=0.75]    (10.93,-3.29) .. controls (6.95,-1.4) and (3.31,-0.3) .. (0,0) .. controls (3.31,0.3) and (6.95,1.4) .. (10.93,3.29)   ;
				%Straight Lines [id:da5887885003342179] 
				\draw    (175,575) -- (375,575) ;
				%Straight Lines [id:da38181530618442316] 
				\draw    (375,575) -- (450,500) ;
				%Straight Lines [id:da5520368192051237] 
				\draw    (250,400) -- (175,475) ;
				%Straight Lines [id:da857881333421624] 
				\draw    (175,475) -- (175,575) ;
				%Straight Lines [id:da1055531864002952] 
				\draw    (175,475) -- (375,475) ;
				%Straight Lines [id:da9921967082690255] 
				\draw    (375,475) -- (375,575) ;
				%Straight Lines [id:da830188535856651] 
				\draw    (375,475) -- (450,400) ;
				%Straight Lines [id:da8543854930333936] 
				\draw    (250,400) -- (450,400) ;
				%Straight Lines [id:da010540648916782303] 
				\draw    (450,400) -- (450,500) ;
				%Straight Lines [id:da5916479203423954] 
				\draw  [dash pattern={on 4.5pt off 4.5pt}]  (250,500) -- (175,575) ;
				%Straight Lines [id:da3953089475850997] 
				\draw    (175,575) -- (126.41,623.59) ;
				\draw [shift={(125,625)}, rotate = 315] [color={rgb, 255:red, 0; green, 0; blue, 0 }  ][line width=0.75]    (10.93,-3.29) .. controls (6.95,-1.4) and (3.31,-0.3) .. (0,0) .. controls (3.31,0.3) and (6.95,1.4) .. (10.93,3.29)   ;
				%Straight Lines [id:da9157272689802225] 
				\draw  [dash pattern={on 4.5pt off 4.5pt}]  (250,500) -- (450,500) ;
				%Straight Lines [id:da9303391433588712] 
				\draw    (450,500) -- (523,500) ;
				\draw [shift={(525,500)}, rotate = 180] [color={rgb, 255:red, 0; green, 0; blue, 0 }  ][line width=0.75]    (10.93,-3.29) .. controls (6.95,-1.4) and (3.31,-0.3) .. (0,0) .. controls (3.31,0.3) and (6.95,1.4) .. (10.93,3.29)   ;
				%Straight Lines [id:da1786416964548232] 
				\draw  [dash pattern={on 4.5pt off 4.5pt}]  (250,400) -- (250,500) ;
				\draw (177,478) node [anchor=north west][inner sep=0.75pt]   [align=left] {D'};
				
				\draw (252,403) node [anchor=north west][inner sep=0.75pt]   [align=left] {A'};
				
				\draw (452,403) node [anchor=north west][inner sep=0.75pt]   [align=left] {B'};
				
				\draw (376,481) node [anchor=north west][inner sep=0.75pt]   [align=left] {C'};
				
				\draw (177,578) node [anchor=north west][inner sep=0.75pt]   [align=left] {D};
				
				\draw (252,503) node [anchor=north west][inner sep=0.75pt]   [align=left] {A};
				
				\draw (452,503) node [anchor=north west][inner sep=0.75pt]   [align=left] {B};
				
				\draw (377,578) node [anchor=north west][inner sep=0.75pt]   [align=left] {C};
				
				\draw (513,518) node [anchor=north west][inner sep=0.75pt]   [align=left] {x};
				
				\draw (140,620) node [anchor=north west][inner sep=0.75pt]   [align=left] {y};
				
				\draw (263,331) node [anchor=north west][inner sep=0.75pt]   [align=left] {z};
			\end{tikzpicture}
		\end{center}			
		Chọn hệ trục tọa độ như hình vẽ. Ta có $A(0;0 ; 0), C(4 ; 3 ; 0), B^{\prime}(4 ; 0 ; 5), B(4 ; 0 ; 0); D'(0;3;5).$ $G$ là trọng tâm của tam giác $ACB' \Rightarrow G\left(\dfrac{8}{3} ; 1 ; \dfrac{5}{3}\right).$\\
		Vì $G \in (ACB')$ nên $\mathrm{d}(G, (ACB'))=0.$\\
		Vì hai mặt phẳng $(AB'D')$ và $(CB'D')$ cắt nhau nên khoảng cách của chúng  bằng $0.$ \\
		Vậy $m+n=0$.}
\end{ex}
\Closesolutionfile{ans}
\indapan{6}{ans/ans-0-B15-KQ}

%%%==============Bai_BT1==============%%%
\begin{ex}%[2H5C1-5]
	Cho hình chóp $S.ABCD$ có đáy $ABCD$ là hình vuông cạnh $a$, cạnh bên $SA=a$ và vuông góc với mặt phẳng đáy. Gọi $M$, $N$ lần lượt là trung điểm của $SB$ và $SD$ và $G$ là trọng tâm của tam giác $AMN$.  Biết độ dài đoạn $BG$ có dạng $x\cdot a$. Hỏi giá trị $x$ bằng bao nhiêu? (Kết quả được làm tròn đến hàng phần trăm).
	
	\shortans{$0{,}87$}
	\loigiai{
		\immini{Đặt hệ trục tọa độ $Oxyz$ như hình vẽ. Khi đó\\ 
			$A\equiv O (0;0;0)$, $B\left(a;0;0\right)$, $D\left(0;a;0\right)$, $S\left(0;0;a\right)$.\\ 
			Suy ra $M\left(\dfrac{a}{2};0;\dfrac{a}{2} \right)$ và $N\left(0;\dfrac{a}{2};\dfrac{a}{2} \right)$.\\
			Vì $G$ là trọng tâm của tam giác $AMN$ nên $G\left(\dfrac{a}{6};\dfrac{a}{6};\dfrac{a}{3} \right)$.\\
			Khi đó độ dài đoạn $BG$ là
			$$BG=\sqrt{\left(\dfrac{5a}{6}\right)^2+\left(\dfrac{a}{6}\right)^2+\left(\dfrac{a}{6}\right)^2}=\dfrac{\sqrt{3}}{2}a\approx 0{,}87 a.$$}{\begin{tikzpicture}[scale=0.9]
				\def\a{3.5}
				\def\h{3.5}
				\path 	(0:0) coordinate (A)
				++(0:\a) coordinate (D)
				++(-130:\a/2) coordinate (C)
				($(A)+(C)-(D)$) coordinate (B)
				($(A)+(90:\h)$) coordinate (S)
				(intersection of A--C and B--D) coordinate (O)
				($(S)!0.5!(B)$) coordinate (M)
				($(S)!0.5!(D)$) coordinate (N)
				($(M)!0.5!(N)$) coordinate (I)
				($(A)!2/3!(I)$) coordinate (G);
				\draw[dashed,thick] 	(B)--(A)--(D)	(A)--(S) (A)--(M)--(N)--(A) (B)--(D);
				\draw [-stealth,thick]  (B) -- ($(B)!-1/2!(A)$)node[left, below]{$x$};	
				\draw [-stealth,thick]  (S) -- ($(S)!-1/3!(A)$)node[above]{$z$};
				\draw [-stealth,thick]  (D) -- ($(D)!-1/4!(A)$)node[right]{$y$};
				\draw[thick] 			(B)--(C)--(D)
				(B)--(S)	(C)--(S)	(D)--(S);
				\foreach \x/\g in {A/-65,B/150,C/-45,D/45,S/180,M/150,N/20,G/0}
				\fill[black] 	(\x) circle (1.5pt)
				($(\g:3mm)+(\x)$) node {$\x$};
		\end{tikzpicture}}
	}
\end{ex}
%%%==============HetBai_BT1==============%%%

%%%==============Bai_BT2==============%%%
\begin{ex}%[2H5C1-5]
	Cho hình chóp $S.ABCD$ có đáy $ABCD$ là hình vuông cạnh $a$, cạnh bên $SA=a$ và vuông góc với mặt phẳng đáy. Gọi $M$, $N$ lần lượt là trung điểm của $SB$ và $SD$ và $G$ là trọng tâm của tam giác $AMN$. Khoảng cách từ điểm $G$ đến mặt phẳng $\left(SBC\right)$ là bao nhiêu nếu $a=6\sqrt{3}$?
	
	\shortans{$2$}
	\loigiai{
		\immini{Chọn hệ trục tọa độ $Oxyz$ thỏa mãn:\\ 
			$A\equiv O(0;0;0)$, $B\left(a;0;0\right)$, $D\left(0;a;0\right)$, $S\left(0;0;a\right)$.\\ 
			Do đó $C(a;a;0)$.\\ 
			Suy ra $M\left(\dfrac{a}{2};0;\dfrac{a}{2} \right)$ và $N\left(0;\dfrac{a}{2};\dfrac{a}{2} \right)$.\\
			Vì $G$ là trọng tâm của tam giác $AMN$ nên $G\left(\dfrac{a}{6};\dfrac{a}{6};\dfrac{a}{3} \right)$.\\
			Phương trình mặt phẳng $(SBD)$ là 
			$$\dfrac{x}{a}+\dfrac{y}{a}+\dfrac{z}{a}=1.$$
			Do đó khoảng cách từ $G$ đến mặt phẳng $\left(SBC\right)$ là
			$$\mathrm{d}\left(G,(SBD)\right)=\dfrac{\left| \dfrac{1}{a}\cdot\dfrac{a}{6}+\dfrac{1}{a}\cdot\dfrac{a}{6}+\dfrac{1}{a}\cdot\dfrac{a}{3}-1\right|}{\sqrt{\left(\dfrac{1}{a}\right)^2+\left(\dfrac{1}{a}\right)^2+\left(\dfrac{1}{a}\right)^2}}=\dfrac{a}{3\sqrt{3}}=2.$$}{\begin{tikzpicture}[scale=0.9]
				\def\a{3.5}
				\def\h{3.5}
				\path 	(0:0) coordinate (A)
				++(0:\a) coordinate (D)
				++(-130:\a/2) coordinate (C)
				($(A)+(C)-(D)$) coordinate (B)
				($(A)+(90:\h)$) coordinate (S)
				(intersection of A--C and B--D) coordinate (O)
				($(S)!0.5!(B)$) coordinate (M)
				($(S)!0.5!(D)$) coordinate (N)
				($(M)!0.5!(N)$) coordinate (I)
				($(A)!2/3!(I)$) coordinate (G);
				\draw[dashed,thick] 	(B)--(A)--(D)	(A)--(S) (A)--(M)--(N)--(A) (B)--(D);
				\draw [-stealth,thick]  (B) -- ($(B)!-1/2!(A)$)node[left, below]{$x$};	
				\draw [-stealth,thick]  (S) -- ($(S)!-1/3!(A)$)node[above]{$z$};
				\draw [-stealth,thick]  (D) -- ($(D)!-1/4!(A)$)node[right]{$y$};
				\draw[thick] 			(B)--(C)--(D)
				(B)--(S)	(C)--(S)	(D)--(S);
				\foreach \x/\g in {A/-65,B/150,C/-45,D/45,S/180,M/150,N/20,G/10}
				\fill[black] 	(\x) circle (1.5pt)
				($(\g:3mm)+(\x)$) node {$\x$};
		\end{tikzpicture}}
	}
\end{ex}
%%%==============HetBai_BT2==============%%%

%%%==============Bai_BT3==============%%%
\begin{ex}%[2H5C1-5]
	Cho hình chóp $S.ABCD$ có đáy $ABCD$ là hình vuông cạnh $a$, cạnh bên $SA=a$ và vuông góc với mặt phẳng đáy. Gọi $M$, $N$ lần lượt là trung điểm của $SB$ và $SD$ và $G$ là trọng tâm của tam giác $AMN$. Tính khoảng cách từ điểm $C$ đến mặt phẳng $\left(AMN\right)$ biết $a=\sqrt{3}$.
	
	\shortans{$2$}
	\loigiai{
		\immini{Chọn hệ trục tọa độ $Oxyz$ thỏa mãn: $A\equiv O$, $B\left(a;0;0\right)$, $D\left(0;a;0\right)$, $S\left(0;0;a\right)$. Do đó $C(a;a;0)$.\\ 
			Suy ra $M\left(\dfrac{a}{2};0;\dfrac{a}{2} \right)$ và $N\left(0;\dfrac{a}{2};\dfrac{a}{2} \right)$.\\
			Vì $G$ là trọng tâm của tam giác $AMN$ nên $G\left(\dfrac{a}{6};\dfrac{a}{6};\dfrac{a}{3} \right)$.\\
			Ta có $AC$ là hình chiếu vuông góc của $SC$ lên mặt phẳng $(ABCD)$. Mà $AC \perp BD$ nên $SC \perp BD$.\\
			Hơn nữa vì $MN \parallel BD$ (tính chất đường trung bình) nên $SC \perp MN$. \quad(1)\\
			Lại có do $\triangle SAB$ cân tại $A$ có $M$ là trung điểm $SB$ nên $AM \perp SB$.\\
			Hơn nữa vì $BC \perp (SAB)$ nên $BC \perp AM$.\\ 
			Do đó $AM \perp (SBC)$.\\
			Suy ra $AM \perp SC$. \quad(2)}{\begin{tikzpicture}[scale=1]
				\def\a{3.5}
				\def\h{3.5}
				\path 	(0:0) coordinate (A)
				++(0:\a) coordinate (D)
				++(-130:\a/2) coordinate (C)
				($(A)+(C)-(D)$) coordinate (B)
				($(A)+(90:\h)$) coordinate (S)
				(intersection of A--C and B--D) coordinate (O)
				($(S)!0.5!(B)$) coordinate (M)
				($(S)!0.5!(D)$) coordinate (N)
				($(M)!0.5!(N)$) coordinate (I)
				($(A)!2/3!(I)$) coordinate (G);
				\draw[dashed,thick] 	(B)--(A)--(D)	(A)--(S) (A)--(M)--(N)--(A) (B)--(D);
				\draw [-stealth,thick]  (B) -- ($(B)!-1/2!(A)$)node[left, below]{$x$};	
				\draw [-stealth,thick]  (S) -- ($(S)!-1/3!(A)$)node[above]{$z$};
				\draw [-stealth,thick]  (D) -- ($(D)!-1/4!(A)$)node[right]{$y$};
				\draw[thick] 			(B)--(C)--(D)
				(B)--(S)	(C)--(S)	(D)--(S);
				\foreach \x/\g in {A/-65,B/150,C/-45,D/45,S/180,M/150,N/20,G/0}
				\fill[black] 	(\x) circle (1.5pt)
				($(\g:3mm)+(\x)$) node {$\x$};
		\end{tikzpicture}}
		\noindent Từ (1) và (2) ta có $SC \perp (AMN)$, hay $\overrightarrow{SC}$ là véc-tơ pháp tuyến của mặt phẳng $(AMN)$.\\ 
		Hay mặt phẳng $(AMN)$ có một véc-tơ pháp tuyến $\overrightarrow{n} = (1;1;-1)$.\\
		Phương trình mặt phẳng $(AMN)$ là 
		$$x+y-z=0.$$
		Do đó khoảng cách từ $C$ đến mặt phẳng $\left(AMN\right)$ là
		$$\mathrm{d}\left(C,(AMN)\right)=\dfrac{\left| a+a-0\right|}{\sqrt{1^2+1^2+\left(-1\right)^2}}=\dfrac{2a}{\sqrt{3}}=2.$$
	}
\end{ex}
%%%==============HetBai_BT3==============%%%

%%%==============Bai_BT4==============%%%
\begin{ex}%[2H5C1-5]
	Cho hình chóp $S.ABCD$ có đáy $ABCD$ là hình chữ nhật, $AB=a$, $BC=a\sqrt{3} $, $SA=a$ và $SA$ vuông góc với đáy $ABCD$. Tính khoảng cách từ điểm $C$ đến mặt phẳng $\left(SBD\right)$ biết $a=\sqrt{21}$.
	
	\shortans{$6$}
	\loigiai{
		\immini{
			Đặt hệ trục tọa độ $Oxyz$ như hình vẽ.\\ 
			Khi đó, ta có
			\[A\left(0;0;0\right), B\left(a;0;0\right), C\left(a;a\sqrt{3} ;0\right), D\left(0;a\sqrt{3} ;0\right), S\left(0;0;a\right).\] 
			Phương trình mặt phẳng $(SBD)$ là
			$$\dfrac{x}{a}+\dfrac{y}{a\sqrt{3}} + \dfrac{z}{a}=1.$$
			Do đó khoảng cách từ $C$ đến mặt phẳng $\left(SBD\right)$ là
			$$\mathrm{d}\left(C,(SBD)\right)=\dfrac{\left| \dfrac{1}{a}\cdot a+\dfrac{1}{a\sqrt{3}}\cdot a\sqrt{3}+\dfrac{1}{a}\cdot0\right|}{\sqrt{\left(\dfrac{1}{a}\right)^2+\left(\dfrac{1}{a\sqrt{3}}\right)^2+\left(\dfrac{1}{a}\right)^2}}=\dfrac{2\sqrt{21}a}{7}=6.$$}{\begin{tikzpicture}[scale=0.85]
				\def\a{3.5}
				\def\h{\a}
				\path 	(0:0) coordinate (A)
				++(0:\a) coordinate (D)
				++(-130:\a/2) coordinate (C)
				($(A)+(C)-(D)$) coordinate (B)
				($(A)+(90:\h)$) coordinate (S)
				(intersection of A--C and B--D) coordinate (O)
				($(S)!2/3!(O)$) coordinate (G);
				\draw[dashed,thick] 	(B)--(A)--(D)	(A)--(S) (B)--(D) (A)--(C);
				\draw [-stealth,thick]  (B) -- ($(B)!-1/2!(A)$)node[left]{$x$};	
				\draw [-stealth,thick]  (S) -- ($(S)!-1/3!(A)$)node[above]{$z$};
				\draw [-stealth,thick]  (D) -- ($(D)!-1/4!(A)$)node[right]{$y$};
				\draw[thick] 			(B)--(C)--(D) (B)--(S)	(C)--(S)	(D)--(S);
				\foreach \x/\g in {A/180,B/150,C/-45,D/45,S/180}
				\fill[black] 	(\x) circle (1.5pt) ($(\g:3mm)+(\x)$) node {$\x$};
		\end{tikzpicture}}
	}
\end{ex}
%%%==============HetBai_BT4==============%%%

%%%==============Bai_BT5==============%%%
\begin{ex}%[2H5C1-5]
	Cho hình chóp $S.ABCD$ có đáy $ABCD$ là hình chữ nhật, $AB=a$, $BC=a\sqrt{3} $, $SA=a$ và $SA$ vuông góc với đáy $ABCD$. Gọi $G$ là trọng tâm của tam giác $SBD$. Tính khoảng cách từ điểm $G$ đến mặt phẳng $\left(SCD\right)$ biết $a=\sqrt{3}$.
	
	\shortans{$0{,}5$}
	\loigiai{
		\immini{ Đặt hệ trục tọa độ $Oxyz$ như hình vẽ.\\ 
			Khi đó, ta có
			\[A\left(0;0;0\right), B\left(a;0;0\right), C\left(a;a\sqrt{3} ;0\right), D\left(0;a\sqrt{3} ;0\right), S\left(0;0;a\right).\] 
			$G$ là trọng tâm của tam giác $SBD$ $\Rightarrow G\left(\dfrac{a}{3} ;\dfrac{a\sqrt{3} }{3} ;\dfrac{a}{3} \right)$.\\
			Gọi phương trình mặt phẳng $(SCD)$ có dạng
			$$Ax+By+Cz+D=0.$$
			Vì $S,C,D\in (SCD)$ nên ta có hệ
		}{\begin{tikzpicture}[scale=0.9]
				\def\a{3.5}
				\def\h{\a}
				\path 	(0:0) coordinate (A)
				++(0:\a) coordinate (D)
				++(-130:\a/2) coordinate (C)
				($(A)+(C)-(D)$) coordinate (B)
				($(A)+(90:\h)$) coordinate (S)
				(intersection of A--C and B--D) coordinate (O)
				($(S)!2/3!(O)$) coordinate (G);
				\draw[dashed,thick] 	(B)--(A)--(D)	(A)--(S) (B)--(D) (A)--(C) (S)--(O);
				\draw [-stealth,thick]  (B) -- ($(B)!-1/2!(A)$)node[left, below]{$x$};	
				\draw [-stealth,thick]  (S) -- ($(S)!-1/3!(A)$)node[above]{$z$};
				\draw [-stealth,thick]  (D) -- ($(D)!-1/4!(A)$)node[right]{$y$};
				\draw[thick] 			(B)--(C)--(D) (B)--(S)	(C)--(S)	(D)--(S);
				\foreach \x/\g in {A/180,B/150,C/-45,D/45,S/180,G/180,O/-90}
				\fill[black] 	(\x) circle (1.5pt) ($(\g:3mm)+(\x)$) node {$\x$};
		\end{tikzpicture}}
		$$\heva{
			& 	Ca		+D	=0\\
			&Aa 	+a\sqrt{3}B +D =0\\
			&a\sqrt{3}B +D=0
		} \Leftrightarrow \heva{&A=0\\&C=B\sqrt{3}\\
			&Ca+D=0.}$$
		Vì vậy phương trình mặt phẳng $(SCD)$ là
		$$y+\sqrt{3}z-a\sqrt{3}=0.$$
		Vậy khoảng cách từ $G$ đến mặt phẳng $\left(SCD\right)$ là
		$$\mathrm{d}\left(C,(SBD)\right)=\dfrac{\left|\dfrac{a\sqrt{3}}{3}+ \dfrac{a\sqrt{3}}{3}-a\sqrt{3}\right|}{\sqrt{1^2+\left(\sqrt{3}\right)^2}}=\dfrac{a\sqrt{3}}{6}=0{,}5.$$
	}
\end{ex}
%%%==============HetBai_BT5==============%%%

%%%==============Bai_BT6==============%%%
\begin{ex}%[2H5C1-5]
	Cho hình chóp $S.ABCD$ có đáy $ABCD$ là hình vuông tâm $I$, có độ dài đường chéo bằng $a\sqrt{2} $ và $SA$ vuông góc với mặt phẳng $\left(ABCD\right)$. Gọi $\alpha $ là góc giữa hai mặt phẳng $\left(SBD\right)$ và $\left(ABCD\right)$ và $\tan \alpha =\sqrt{2} $. Khoảng cách từ điểm $I$ đến mặt phẳng $\left(SAB\right)$ có dạng $x\cdot a$. Tìm giá trị của $x$.
	
	\shortans{$0{,}5$}
	\loigiai{
		\immini{ Hình vuông $ABCD$ có độ dài đường chéo bằng $a\sqrt{2} $ suy ra hình vuông đó có cạnh bằng $a$.\\
			Ta có 
			$\heva{&\left(SBD\right)\cap \left(ABCD\right)=BD \\ &{SI\bot BD} \\ &{AI\bot BD} } $\\
			$\Rightarrow {\left(\left(SBD\right); \left(ABCD\right)\right)}={\left(SI; AI\right)}=\widehat{SIA}$
			Ta có $\tan \alpha =\tan \widehat{SIA}=\dfrac{SA}{AI} \Leftrightarrow SA=a$.\\
			Ta xét hệ trục tọa độ $Oxyz$ như hình vẽ với\\ 
			$A\left(0; 0; 0\right)$, $B\left(a; 0; 0\right)$, $C\left(a; a; 0\right)$, $D(0;a;0)$, $S\left(0; 0; a\right)$.\\
			Suy ra $I\left(\dfrac{a}{2} ;\dfrac{a}{2};0 \right)$.\\
			Phương trình mặt phẳng $(SAB)$ là $y=0$.\\
			Vì vậy khoảng cách từ $I$ đến mặt phẳng $\left(SAB\right)$ là $\dfrac{a}{2}=0{,}5a$.}{\begin{tikzpicture}[scale=1]
				\def\a{3.5}
				\def\h{\a}
				\path 	(0:0) coordinate (A)
				++(0:\a) coordinate (D)
				++(-130:\a/2) coordinate (C)
				($(A)+(C)-(D)$) coordinate (B)
				($(A)+(90:\h)$) coordinate (S)
				(intersection of A--C and B--D) coordinate (I);
				\draw[dashed,thick] 	(B)--(A)--(D)	(A)--(S) (B)--(D) (A)--(C) (S)--(I);
				\draw [-stealth,thick]  (B) -- ($(B)!-1/2!(A)$)node[left, below]{$x$};	
				\draw [-stealth,thick]  (S) -- ($(S)!-1/3!(A)$)node[above]{$z$};
				\draw [-stealth,thick]  (D) -- ($(D)!-1/4!(A)$)node[right]{$y$};
				\draw[thick] 			(B)--(C)--(D) (B)--(S)	(C)--(S)	(D)--(S);
				\foreach \x/\g in {A/180,B/150,C/-45,D/45,S/180,I/-90}
				\fill[black] 	(\x) circle (1.5pt) ($(\g:3mm)+(\x)$) node {$\x$};
		\end{tikzpicture}}
	}
\end{ex}
%%%==============HetBai_BT6==============%%%

%%%==============Bai_BT7==============%%%
\begin{ex}%[2H5C1-5]
	Cho hình chóp $S.ABCD$ có đáy $ABCD$ là hình vuông tâm $I$, có độ dài đường chéo bằng $a\sqrt{2} $ và $SA$ vuông góc với mặt phẳng $\left(ABCD\right)$. Gọi $\alpha $ là góc giữa hai mặt phẳng $\left(SBD\right)$ và $\left(ABCD\right)$ và $\tan \alpha =\sqrt{2} $. Tính khoảng cách từ điểm $I$ đến mặt phẳng $\left(SCD\right)$ biết $a=2\sqrt{2}$.
	
	\shortans{$1$}
	\loigiai{
		\immini{ Hình vuông $ABCD$ có độ dài đường chéo bằng $a\sqrt{2} $ suy ra hình vuông đó có cạnh bằng $a$.\\
			Ta có 
			$\heva{&\left(SBD\right)\cap \left(ABCD\right)=BD \\ &{SI\bot BD} \\ &{AI\bot BD} }$\\
			$ \Rightarrow {\left(\left(SBD\right); \left(ABCD\right)\right)}={\left(SI; AI\right)}=\widehat{SIA}.$
			Ta có $\tan \alpha =\tan \widehat{SIA}=\dfrac{SA}{AI} \Leftrightarrow SA=a$.\\
			Ta có $A\left(0; 0; 0\right)$, $B\left(a; 0; 0\right)$, $C\left(a; a; 0\right)$, $D(0;a;0)$, $S\left(0; 0; a\right)\Rightarrow I\left(\dfrac{a}{2} ;\dfrac{a}{2};0 \right)$.\\
			Phương trình mặt phẳng $(SCD)$ có dạng
			$$Ax+By+Cz+D=0.$$}{\begin{tikzpicture}[scale=1]
				\def\a{3.5}
				\def\h{\a}
				\path 	(0:0) coordinate (A)
				++(0:\a) coordinate (D)
				++(-130:\a/2) coordinate (C)
				($(A)+(C)-(D)$) coordinate (B)
				($(A)+(90:\h)$) coordinate (S)
				(intersection of A--C and B--D) coordinate (I);
				\draw[dashed,thick] 	(B)--(A)--(D)	(A)--(S) (B)--(D) (A)--(C) (S)--(I);
				\draw [-stealth,thick]  (B) -- ($(B)!-1/2!(A)$)node[left, below]{$x$};	
				\draw [-stealth,thick]  (S) -- ($(S)!-1/3!(A)$)node[above]{$z$};
				\draw [-stealth,thick]  (D) -- ($(D)!-1/4!(A)$)node[right]{$y$};
				\draw[thick] 			(B)--(C)--(D) (B)--(S)	(C)--(S)	(D)--(S);
				\foreach \x/\g in {A/180,B/150,C/-45,D/45,S/180,I/-90}
				\fill[black] 	(\x) circle (1.5pt) ($(\g:3mm)+(\x)$) node {$\x$};
		\end{tikzpicture}}
		\noindent Vì $S,C,D\in (SCD)$ nên ta có hệ
		$$\heva{
			& 	Ca		+D	=0\\
			&aA 	aB+D =0\\
			&aB +D=0
		} \Leftrightarrow \heva{&A=0\\&C=B\\
			&Ca+D=0.}$$
		Vì vậy phương trình mặt phẳng $(SCD)$ là
		$$y+z-a=0.$$
		Vậy khoảng cách từ $I$ đến mặt phẳng $\left(SCD\right)$ là
		$$\mathrm{d}\left(C,(SCD)\right)=\dfrac{\left|\dfrac{a}{2}+0-a\right|}{\sqrt{1^2+1^2}}=\dfrac{a\sqrt{2}}{4}=1.$$
	}
\end{ex}
%%%==============HetBai_BT7==============%%%

%%%==============Bai_BT8==============%%%
\begin{ex}%[2H5C1-5]
	Cho hình chóp $S.ABCD$ có đáy $ABCD$ là hình vuông cạnh $a$, mặt bên $SAB$ là tam giác đều và nằm trong mặt phẳng vuông góc với mặt phẳng $\left(ABCD\right)$. Tính khoảng cách từ điểm $A$ đến mặt phẳng $\left(SBD\right)$ biết $a=\sqrt{21}$.
	
	\shortans{$3$}
	\loigiai{
		\begin{center}
			\begin{tikzpicture}
				\def\a{4}
				\def\h{4.5}
				\path 	(0:0) coordinate (A)
				++(0:\a) coordinate (D)
				++(-130:\a/2) coordinate (C)
				($(A)+(C)-(D)$) coordinate (B)
				($(A)!0.5!(B)$) coordinate (H)
				($(C)!0.5!(D)$) coordinate (K)
				($(H)+(90:\h)$) coordinate (S)
				(intersection of A--C and B--D) coordinate (O);%giao điểm O
				\draw[dashed,thick] 	(B)--(A)--(D)	(A)--(S) (S)--(H)	(H)--(K);
				\draw [-stealth,thick]  (B) -- ($(B)!-1/2!(A)$)node[left, below]{$x$};	
				\draw [-stealth,thick]  (S) -- ($(S)!-1/3!(H)$)node[above]{$z$};
				\draw [-stealth,thick]  (K) -- ($(K)!-1/3!(H)$)node[right]{$y$};
				\draw[thick] 			(B)--(C)--(D)
				(B)--(S)	(C)--(S)	(D)--(S);
				\foreach \x/\g in {A/45,B/185,C/-45,D/45,S/180}
				\fill[black] 	(\x) circle (1.5pt)
				($(\g:4mm)+(\x)$) node {$\x$};
				\fill[black] 	(H) circle (1.5pt)
				($(-30:5mm)+(H)$) node {\footnotesize{$H\equiv O$}};
			\end{tikzpicture}
		\end{center}
		Chọn hệ trục tọa độ $Oxyz$ như hình vẽ. Khi đó
		\[S\left(0; 0; \dfrac{a\sqrt{3} }{2} \right); A\left(\dfrac{-a}{2} ;0;0\right); B\left(\dfrac{a}{2} ;0; 0\right);C\left(\dfrac{a}{2} ;a; 0\right); D\left(\dfrac{-a}{2} ;a; 0\right).\] 
		Phương trình mặt phẳng $(SBD)$ có dạng
		$$Ax+By+Cz+D=0.$$
		Vì $S,B,D\in (SBD)$ nên ta có hệ
		$$\heva{
			& \dfrac{a\sqrt{3} }{2}C		+D	=0\\
			&\dfrac{a}{2} A 	  +D=0\\
			&-\dfrac{a}{2} A  +aB +D=0
		} \Leftrightarrow \heva{&A=-\dfrac{2}{a}D\\ &B=-\dfrac{2}{a} D\\
			&C=-\dfrac{2\sqrt{3}}{3a}D.}$$
		Vì vậy phương trình mặt phẳng $(SBD)$ là
		$$x+y+\dfrac{\sqrt{3}}{3}z-\dfrac{a}{2}=0.$$
		Vậy khoảng cách từ $A$ đến mặt phẳng $\left(SBD\right)$ là
		$$\mathrm{d}\left(A,(SBD)\right)=\dfrac{\left|-\dfrac{a}{2}-\dfrac{a}{2}\right|}{\sqrt{1^2+1^2+\left(\dfrac{\sqrt{3}}{3}\right)^2 }}=\dfrac{a\sqrt{21}}{7}=3.$$
	}
\end{ex}
%%%==============HetBai_BT8==============%%%

%%%==============Bai_BT9==============%%%
\begin{ex}%[2H5C1-5]  
	Cho hình chóp $S.ABCD$ có đáy $ABCD$ là hình vuông cạnh $a$, mặt bên $SAB$ là tam giác đều và nằm trong mặt phẳng vuông góc với mặt phẳng $\left(ABCD\right)$. Gọi $G$ là trọng tâm của tam giác $SAB$ và $M$, $N$ lần lượt là trung điểm của $SC$, $SD$. Tính khoảng cách từ điểm $S$ đến mặt phẳng $\left(GMN\right)$ biết $a=\sqrt{14}$.
	
	\shortans{$2$}
	\loigiai{\;
		\begin{center}
			\begin{tikzpicture}[scale=1]
				\def\a{4}
				\def\h{4.5}
				\path 	(0:0) coordinate (A)
				++(0:\a) coordinate (D)
				++(-130:\a/2) coordinate (C)
				($(A)+(C)-(D)$) coordinate (B)
				($(A)!0.5!(B)$) coordinate (H)
				($(C)!0.5!(D)$) coordinate (K)
				($(H)+(90:\h)$) coordinate (S)
				(intersection of A--C and B--D) coordinate (O)
				($(S)!2/3!(H)$) coordinate (G)
				($(S)!1/2!(D)$) coordinate (N)
				($(S)!1/2!(C)$) coordinate (M);
				\draw[dashed,thick] 	(B)--(A)--(D)	(A)--(S) (S)--(H)	(H)--(K) (G)--(N);
				\draw [-stealth,thick]  (B) -- ($(B)!-1/2!(A)$)node[left, below]{$x$};	
				\draw [-stealth,thick]  (S) -- ($(S)!-1/3!(H)$)node[above]{$z$};
				\draw [-stealth,thick]  (K) -- ($(K)!-1/3!(H)$)node[right]{$y$};
				\draw[thick] 			(B)--(C)--(D)
				(B)--(S)	(C)--(S)	(D)--(S) (G)--(M)--(N);
				\foreach \x/\g in {A/45,B/185,C/-45,D/45,S/180,M/-10,N/0,G/-45}
				\fill[black] 	(\x) circle (1.5pt)
				($(\g:4mm)+(\x)$) node {$\x$};
				\fill[black] 	(H) circle (1.5pt)
				($(-30:5mm)+(H)$) node {\footnotesize{$H\equiv O$}};
			\end{tikzpicture}
		\end{center}
		Chọn hệ trục tọa độ $Oxyz$ như hình vẽ.\\ 
		Khi đó
		$S\left(0; 0; \dfrac{a\sqrt{3} }{2} \right)$, $A\left(\dfrac{-a}{2} ;0;0\right)$, $ B\left(\dfrac{a}{2} ;0; 0\right)$, $C\left(\dfrac{a}{2} ;a; 0\right)$ và $D\left(\dfrac{-a}{2} ;a; 0\right)$.\\
		Suy ra $G\left(0; 0; \dfrac{a\sqrt{3} }{6} \right)$, $M\left(\dfrac{a}{4} ;\dfrac{a}{2} ; \dfrac{a\sqrt{3} }{4} \right)$, $N\left(-\dfrac{a}{4} ;\dfrac{a}{2} ; \dfrac{a\sqrt{3} }{4} \right)$.\\
		Phương trình mặt phẳng $(GMN)$ có dạng
		$$Ax+By+Cz+D=0.$$
		Vì $G,M,N\in (GMN)$ nên ta có hệ
		$$\heva{
			& 	\dfrac{a\sqrt{3} }{6}C		+D	=0\\
			&\dfrac{a}{4} A 	+\dfrac{a}{2}B + \dfrac{a\sqrt{3} }{4}C +D=0\\
			&-\dfrac{a}{4} A  +\dfrac{a}{2} B+\dfrac{a\sqrt{3} }{4}C +D=0
		} \Leftrightarrow \heva{&A=0\\&B=\dfrac{1}{a} D\\
			&C=-\dfrac{2\sqrt{3}}{a} D.}$$
		Vì vậy phương trình mặt phẳng $(GMN)$ là
		$$y-2\sqrt{3}z+a=0.$$
		Vậy khoảng cách từ $S$ đến mặt phẳng $\left(GMN\right)$ là
		$$\mathrm{d}\left(S,(GMN)\right)=\dfrac{\left|-2a\right|}{\sqrt{1^2+1^2+\left(-2\sqrt{3}\right)^2 }}=\dfrac{2a\sqrt{14}}{14}=2.$$
	}
\end{ex}
\Closesolutionfile{ans}
\indapan{6}{ans/ans-0-B15-KQ}
%%%==============HetBai_BT9==============%%%
%%Bài 2.
% \foreach \i in {1,2,...,7} {\input{data/12/C5B2/C5B2CD\i.tex}}
%%Bài 3.
% \foreach \i in {1,2,3,4,7} {\input{data/12/C5B3/C5B3CD\i.tex}}
%%%%%12C
%%Bài 1.
\setcounter{section}{13}
\setcounter{dang}{0}
\section{PHƯƠNG TRÌNH MẶT PHẲNG}
\subsection{LÝ THUYẾT CẦN NHỚ}
\subsubsection{Vectơ pháp tuyến của mặt phẳng}
\begin{itemize}
	\immini{\item [\iconMT] \indam{Định nghĩa:} Vectơ pháp tuyến $\vec{n}$ của mặt phẳng $(P)$ là những vectơ khác $\vec{0}$ và có giá vuông góc với $(P)$. 
		\item [\iconMT] \indam{Chú ý:} 
		\begin{boxdn}
			\begin{itemize}
				\item [$\bullet$] $\vec{n} \ne \vec{0}$ và có giá vuông với $(P)$;
				\item [$\bullet$] Nếu $\vec{n}$ và $\vec{n'}$ cùng là vectơ pháp tuyến của $(P)$ thì $\vec{n'} = k \cdot \vec{n}$ (tọa độ tỉ lệ nhau).
			\end{itemize}
		\end{boxdn}
	}{
		\begin{tikzpicture}[scale=0.8, line join=round, line cap=round,>=stealth]
			\tkzDefPoints{0/0/A,4/0/B,5/2/C}
			\coordinate (D) at ($(A)+(C)-(B)$);
			\tkzDrawPolygon(A,B,C,D)
			\tkzMarkAngles[size=0.7cm,arc=l](B,A,D)
			\tkzLabelAngles[pos=0.5,rotate=10](B,A,D){\scriptsize$P$}
			\draw[->] (2,1)--(2,2.5)node[right]{\scriptsize$\vec{n}$};
			\draw[->] (3,1.5)--(3,3)node[right]{\scriptsize$\vec{n'}$};
	\end{tikzpicture}}
\end{itemize}

\subsubsection{Cặp vectơ chỉ phương của mặt phẳng}
\begin{itemize}
	\item [\iconMT] \indam{Định nghĩa:} Trong không gian $Oxyz$, cho hai vectơ $\vec u$, $\vec v$ được gọi là cặp vectơ chỉ phương của mặt phẳng $(P)$ nếu chúng không cùng phương và có giá nằm trong hoặc song song với mặt phẳng $(P)$.
	\item [\iconMT] \indam{Chú ý:} 
	\begin{boxdn}
\immini{		\begin{itemize}
			\item [$\bullet$] Cho hai vectơ $\vec u = (a; b; c)$ và $\vec v = (a'; b'; c')$. Khi đó 
			$$\vec n = (bc' - b'c;ca' - c'a; ab' - a'b)$$
			vuông góc với cả hai vectơ $\vec u$ và $\vec v$, được gọi là tích có hướng của $\vec u$ và $\vec v$, ký hiệu là $[\vec u, \vec v]$.
			\item [$\bullet$] Nếu $\vec u$, $\vec v$ là cặp vectơ chỉ phương của $(P)$ thì $[\vec u,\vec v]$ là một vectơ pháp tuyến của $(P)$.
		\end{itemize}}{
	\begin{tikzpicture}[>=stealth, line join=round, line cap = round,scale=0.8]
	\def\d{4}
	\def\r{3}
	\path (0:0) coordinate (B)
	++(0:\d) coordinate (C)
	++(50:\r) coordinate (D)
	($(B)+(D)-(C)$) coordinate (A)
	(2.5,2) coordinate (M)
	(3.5,2) coordinate (N)
	(3.,1.6) coordinate (P)
	;
	\draw[->] (M)--($(M)+(-130:1.5)$) node[pos=0.4,left] {$\vec u$};
	\draw[->] (N)--($(N)+(0:1.8)$) node[pos=0.4,below] {$\vec v$};
	\draw[->] (P)--($(P)+(90:1.8)$) node[pos=0.9,right] {${[\vec u,\vec v]}$};
	\draw (A)--(B)--(C)--(D)--cycle;
	\begin{scope}
		\clip (A)--(B)--(C);
		\draw[opacity=0.7] (B) circle(0.8cm)node[black,shift={(25:4mm)}]{$P$};
	\end{scope}
	%		\foreach \x/ \goc in {A/180,B/180,C/0,D/0} 
	%		\fill (\x) circle (1pt) ($(\x)+(\goc:3mm)$) node {$\x$};
\end{tikzpicture}}
	\end{boxdn}

\end{itemize}
\subsubsection{Phương trình tổng quát của mặt phẳng}
\begin{itemize}
	\item [\iconMT] \indam{Công thức:} Mặt phẳng $(P)$ đi qua điểm $M(x_0;y_0;z_0)$ và nhận $\vec{n}=(a;b;c)$ làm vectơ pháp tuyến có phương trình là 
	\boxmini{$a(x-x_0)+b(y-y_0)+c(z-z_0)=0$}
	Thu gọn ta được dạng 
	$$ax+by+cz+d=0$$
	\item [\iconMT] \indam{Chú ý:}
	\begin{boxdn}
		\begin{itemize}
			\item [\ding{172}] Phương trình các mặt phẳng tọa độ: 
			\begin{listEX}[2]
				\item [$\bullet$] $(Oxy) \colon z=0$.
				\item [$\bullet$] $(Oxz) \colon y=0$.
				\item [$\bullet$] $(Oyz) \colon x=0$.
			\end{listEX}
			\item [\ding{173}] Phương trình mặt phẳng $(\alpha)$ song song với mặt phẳng tọa độ: 
			\begin{listEX}[2]
				\item [$\bullet$] $(\alpha) \parallel (Oxy) \Rightarrow z=a \quad a \ne 0$.
				\item [$\bullet$] $(\alpha) \parallel (Oxz) \Rightarrow y=b \quad b \ne 0 $.
				\item [$\bullet$] $(\alpha) \parallel (Oyz) \Rightarrow x=c \quad c \ne 0$.
			\end{listEX}
		\end{itemize}
	\end{boxdn}
\end{itemize}

\subsubsection{Vị trị tương đối giữa hai mặt phẳng}
\begin{itemize}
	\item [] Cho hai mặt phẳng $(P) \colon a_1x + b_1y + c_1z + d_1=0$ và $(Q) \colon a_2x + b_2y + c_2z + d_2=0$. \\
	Gọi $\vec{n_1}=(a_1;b_1;c_1)$, $\vec{n_2}=(a_2;b_2;c_2)$ lần lượt là vectơ pháp tuyến của $(P)$ và $(Q)$.\\
	\begin{boxdn}
		\begin{listEX}[1]
			\item [\ding{172}] Nếu $\heva{&\vec{n_1}= k \cdot \vec{n_2}\\& d_1 =k\cdot d_2}$ thì $(P)$ trùng $(Q)$.
			\item [\ding{173}] Nếu $\heva{&\vec{n_1}= k \cdot \vec{n_2}\\& d_1 \ne k\cdot d_2}$ thì $(P)$ song song $(Q)$.
			\item [\ding{174}] Nếu $\vec{n_1}$ không cùng phương với $\vec{n_2}$ thì $(P)$ cắt $(Q)$.
			\item [\ding{175}] Nếu $\vec{n_1} \perp \vec{n_2}$ hay $a_1a_2+b_1b_2+c_1c_2=0$ thì $(P) \perp (Q)$.
		\end{listEX}
	\end{boxdn}  
\end{itemize}

\subsubsection{Khoảng cách từ một điểm đến mặt phẳng}
\begin{itemize}
	\immini{\item [\iconMT] \indam{Định nghĩa:} Cho điểm $M(x_0;y_0;z_0)$ và mặt phẳng $(P) \colon ax+by+cz+d=0$. Gọi $H$ là hình chiếu vuông góc của điểm $M$ lên mặt phẳng $(P)$. Khi đó độ dài đoạn $MH$ được gọi là khoảng cách từ điểm $M$ đến $(P)$. Kí hiệu $\mathrm{d}\left(M,(P) \right)$.
		\item [\iconMT] \indam{Công thức tính:}
		\boxmini{$\mathrm{d}\left(M,(P) \right)=\dfrac{\bigg|ax_0+by_0+cz_0+d\bigg|}{\sqrt{a^2+b^2+c^2}}$}
	}{
		\begin{tikzpicture}[scale=0.8, line join=round, line cap=round]
			\tkzDefPoints{0/0/A,4/0/B,5/2/C}
			\coordinate (D) at ($(A)+(C)-(B)$);
			\tkzDrawPolygon(A,B,C,D)
			\tkzMarkAngles[size=0.7cm,arc=l](B,A,D)
			\tkzLabelAngles[pos=0.5,rotate=10](B,A,D){$P$}
			\draw (2,1)node[right]{$H$}--(2,3)node[above]{$M$};
			\draw[fill=black] (2,1) circle(1.5pt) (2,3) circle(1.5pt);
	\end{tikzpicture}}
	\item [\iconMT] \indam{Đặc biệt:} 
	\begin{listEX}[3]
		\item [\ding{172}] $\mathrm{d}\left(M,(Oxy) \right)=\big|z_M\big|$.
		\item [\ding{173}]  $\mathrm{d}\left(M,(Oxz) \right)=\big|y_M\big|$.
		\item [\ding{174}]  $\mathrm{d}\left(M,(Oyz) \right)=\big|x_M\big|$.
	\end{listEX}
\end{itemize}
\subsection{PHÂN LOẠI, PHƯƠNG PHÁP GIẢI TOÁN}
\begin{dang}{Xác định vectơ pháp tuyến và điểm thuộc mặt phẳng}
	Cho mặt phẳng $(\alpha)$.
	\begin{itemize}
		\item  [\ding{172}] Nếu véctơ $\overrightarrow{n}$ khác $\overrightarrow{0}$ và có giá vuông góc với $(\alpha)$ thì $\overrightarrow{n}$ được gọi là véctơ pháp tuyến của $(\alpha)$.
		\item  [\ding{173}] Nếu hai véctơ $\overrightarrow{a}, \overrightarrow{b}$ không cùng phương, có giá song song hoặc nằm trong $(\alpha)$ thì $\overrightarrow{a}, \overrightarrow{b}$ được gọi là cặp véctơ chỉ phương của $(\alpha)$. Khi đó, nếu $\vec{a}=(a_1;a_2;a_3)$, $\vec{b}=(b_1;b_2;b_3)$ thì
		$$\vec{n}= [\vec{a}, \vec{b}]=\left(\left|\begin{array}{ll}a_2 & a_3 \\ b_2 & b_3\end{array}\right| ;\left|\begin{array}{ll}a_3 & a_1 \\ b_3 & b_1\end{array}\right| ;\left|\begin{array}{ll}a_1 & a_2 \\ b_1 & b_2\end{array}\right|\right)$$ là một vectơ pháp tuyến của mặt phẳng $(P)$.
		\item [\ding{174}] Nếu $(\alpha) \colon ax + by + cz + d = 0$ thì vectơ pháp tuyến của $(\alpha)$ là $\vec{n}=(a;b;c)$.
	\end{itemize}
\end{dang}
\boxmini{BÀI TẬP TỰ LUẬN}

\begin{vd}
	\immini{Cho hình lập phương $ABCD.A'B'C'D'$. 
		\begin{tasks}
			\task Xác định vectơ pháp tuyến của các mặt phẳng $(ABCD)$,  $(ABB'A')$,  $(ACC'A')$,  $(ADD'A')$.
			\task Chứng minh $\vec{AB'}$ là một vectơ pháp tuyến của $(BCD'A')$.
		\end{tasks}
	}{
	\begin{tikzpicture}[scale=0.7, font=\footnotesize, line join=round, line cap=round, >=stealth]
		\def\bc{4} % cạnh BC
		\def\ba{2} % cạnh BA
		\def\h{3} % đường cao
		\def\gocB{35} % góc B của đáy
		\coordinate[label=below left:$B$] (B) at (0,0);
		\coordinate[label=above left:$A$] (A) at (\gocB:\ba);
		\coordinate[label=below:$C$] (C) at (\bc,0);
		\coordinate[label=right:$D$] (D) at ($(C)-(B)+(A)$);
		\coordinate[label=above left:$A'$] (A') at ($(A)+(90:\h)$);
		\coordinate[label=left:$B'$] (B') at ($(B)-(A)+(A')$);
		\coordinate[label=below right:$C'$] (C') at ($(C)-(A)+(A')$);
		\coordinate[label=right:$D'$] (D') at ($(D)-(A)+(A')$);
		\draw (B')--(B)--(C)--(D)--(D')--(A')--(B')--(C')--(D') (C)--(C');
		\draw[dashed] (A')--(A)--(D) (A)--(B);
		\foreach \diem in {A,B,C,D,A',B',C',D'}	\fill (\diem)circle(1.5pt);
\end{tikzpicture}}
\end{vd}

\dongcham{9}
\begin{vd}
	Cho mặt phẳng $(P): 2x-3y+4z+5=0$. Hãy chỉ ra một vectơ pháp tuyến của $(P)$ và hai điểm thuộc $(P)$. 

	\loigiai{vectơ $\overrightarrow{n}=(2;-3;4) $ là một vectơ pháp tuyến của mặt phẳng $(P)$. 
		
	}
\end{vd}
\dongcham{4}

\begin{vd}
	Cho $(P)$ là mặt phẳng trung trực của $MN$ với $M(1;-2;3)$, $N(1;4;1)$. Hãy chỉ ra một vectơ pháp tuyến của $(P)$ và một điểm thuộc $(P)$.
	\loigiai{
		Mặt phẳng trung trực của $MN$ đi qua trung điểm của $MN$ và vuông góc với $MN$.\\
		Vậy $\vec{MN}=(0; 6; -2)$ là một vectơ pháp tuyến của mặt phẳng $(P)$.
	}
\end{vd}
\dongcham{4}
\begin{vd}
	Chỉ ra một vectơ pháp tuyến của mặt phẳng $(\alpha)$ biết
	\begin{listEX}[1]
		\item $(\alpha)$ đi qua $A(-1; 3; 5)$, $B(3;2;-2)$ và $C(0; 3; 0)$
		\item $(\alpha)$ đi qua $M(0; 3; 1)$, $N(-3;2;5)$ và $P(-2; 0; 0)$
	\end{listEX}
	\loigiai{
		\begin{listEX}[1]
			\item 	Ta có $\vec{AB} = (4; -1;-7)$, $\vec{AC} = (1; 0; -5)$.\\
			Xét vectơ $\vec{n}=[\vec{AB}, \vec{AC}] = \left(\left|\begin{array}{cc}-1 & -7 \\ 0 & -5\end{array}\right| ;\left|\begin{array}{cc}-7 & 4 \\ -5 & 1\end{array}\right| ;\left|\begin{array}{cc}4 & -1 \\ 1 & 0\end{array}\right|\right)=(5 ; 13 ; 1)$.\\
			Vậy $\vec{n} = (5 ; 13 ; 1)$ là một vectơ pháp tuyến của mặt phẳng $(\alpha)$.
			\item Ta có $\vec{MN} = (-3; -1;4)$, $\vec{MP} = (-2; -3; -1)$.\\
			Xét vectơ $\vec{n}=[\vec{MN}, \vec{MP}] = \left(\left|\begin{array}{cc}-1 & 4 \\ -3 & -1\end{array}\right| ;\left|\begin{array}{cc}4 & -3 \\ -1 & -2\end{array}\right| ;\left|\begin{array}{cc}-3 & -1 \\ -2 & -3\end{array}\right|\right)=(13; -11; 7)$.\\
			Vậy $\vec{n} = (13 ; -11 ; 7)$ là một vectơ pháp tuyến của mặt phẳng $(\alpha)$.
		\end{listEX}
	}
\end{vd}
\dongcham{6}


\begin{vd}
	Cho tứ diện $ABCD$ có các đỉnh là $A(5 ; 1 ; 3)$, $B(1 ; 6 ; 2)$, $C(5 ; 0 ; 4)$ và $D(4 ; 0 ; 6)$. Gọi $(\alpha)$ là mặt phẳng chứa cạnh $AB$ và song song với cạnh $CD$. Hãy tìm một điểm thuộc $(\alpha)$ và một vectơ pháp tuyến của $(\alpha)$.
	\loigiai{
		Ta có $\overrightarrow{AB}=(-4 ; 5 ;-1)$, $\overrightarrow{CD}=(-1 ; 0 ; 2)$ nên $\overrightarrow{AB}, \overrightarrow{CD}$ không cùng phương.\\
		Mà giá của $\overrightarrow{AB}$ nằm trong mặt phẳng $(\alpha)$ và giá của $\overrightarrow{CD}$ song song với mặt phẳng $(\alpha)$ nên $\overrightarrow{AB}$, $\overrightarrow{CD}$ là một cặp vectơ chỉ phương của mặt phẳng $(\alpha)$.
		Vậy một vectơ pháp tuyến của $(\alpha)$ là
		$$
		\begin{aligned}
			\vec{n}=\left[\overrightarrow{AB}, \overrightarrow{CD}\right]&=
			\left(\left|\begin{array}{cc}
				5 & -1 \\
				0 & 2
			\end{array}\right| ;
			\left|\begin{array}{cc}
				-1 & -4 \\
				2 & -1
			\end{array}\right| ;
			\left|\begin{array}{cc}
				-4 & 5 \\
				-1 & 0
			\end{array}\right|\right) \\
			&=\left(5 \cdot 2-0 \cdot(-1) ; (-1)\cdot (-1) - 2 \cdot (-4) ; (-4) \cdot 0-5 \cdot (-1)\right) \\
			&=(10 ; 9 ; 5).
		\end{aligned}
		$$
	}
\end{vd}
\dongcham{4}
\boxmini{BÀI TẬP TRẮC NGHIỆM}
\setcounter{ex}{0}

\begin{ex}
	Cho mặt phẳng $(\alpha) \colon 2x-y+3z-2=0$. Điểm nào sau đây thuộc mặt phẳng $(\alpha)$?
	\choice
	{$A(1;-3;1)$}
	{\True $B(2;-1;-1)$}
	{$C(2;-1;1)$}
	{$D(1;2;3)$}
	\loigiai{
	Thay tọa độ các điểm vào phương trình $(\alpha)$, tọa độ $B(2;-1;-1)$ thỏa mãn.}
\end{ex}

\begin{ex}%[2H3Y2-7]
	Cho mặt phẳng $(\alpha) \colon x+y+z-6=0$. Điểm nào dưới đây \textbf{không} thuộc $(\alpha)$?
	\choice
	{\True $M(1;-1;1)$}
	{$N(2;2;2)$}
	{$P(1;2;3)$}
	{$Q(3;3;0)$}
	\loigiai
	{
		Ta có $1-1+1-6=-5 \neq 0$ nên $M(1;-1;1)$ không thuộc $(\alpha)$.
	}
\end{ex}

\begin{ex}
	Cho $(\alpha)$ vuông góc với giá của $\vec{a}=(2;-1;3)$. Vectơ nào dưới đây là vectơ pháp tuyến của $(\alpha)$?
	\choice
	{$\vec{n_1}=(-2;1;3)$}
	{\True $\vec{n_2}=(-2;1;-3)$}
	{$\vec{n_3}=(4;2;6)$}
	{$\vec{n_4}=(4;-2;-6)$}
	\loigiai{
		$(\alpha)$ vuông góc với giá của $\vec{a}=(2;-1;3)$ nên $\vec{a}$ là một vectơ pháp tuyến của $(\alpha)$.\\
		Do đó $\vec{n_2}=-\vec{a}$ cũng là một vectơ pháp tuyến của $(\alpha)$.
	}
\end{ex}

\begin{ex}%[2H3B2-2]
	vectơ nào sau đây \textbf{không} phải là vectơ pháp tuyến của mặt phẳng $(P):x+3y-5z+2=0$.
	\choice
	{$\overrightarrow{n}_1=(-1;-3;5)$}
	{\True $\overrightarrow{n}_2=(-2;-6;-10)$}
	{$\overrightarrow{n}_3=(-3;-9;15)$}
	{$\overrightarrow{n}_4=(2;6;-10)$}
	\loigiai{Mặt phẳng $(P)$ nhận vectơ $\overrightarrow{a}=(1;3;-5)$ làm vectơ pháp tuyến.\\
		Xét $\overrightarrow{n}_2=(-2;-6;-10)$ có $\dfrac{-2}{1}\ne\dfrac{-6}{3}\ne\dfrac{-10}{-5}$ nên $\overrightarrow{n}_2$ không cùng phương với $\overrightarrow{a}$.\\
		Suy ra $\overrightarrow{n}_2$ không là vectơ pháp tuyến của $(P)$.}
\end{ex}

\begin{ex}
	Trong không gian  $Oxyz$, mặt phẳng tọa độ $(Oxy)$ có một vectơ pháp tuyến là
	\choice
	{$\overrightarrow{n}=(0;1;0)$}
	{\True $\overrightarrow{n}=(0;0;1)$}
	{$\overrightarrow{n}=(1;0;0)$}
	{$\overrightarrow{n}=(1;1;0)$}
	\loigiai{
		Mặt phẳng tọa độ $(Oxy)\colon x=0\Rightarrow$ 1 vectơ pháp tuyến là $\overrightarrow{n}=(0;0;1)$.}
\end{ex}

\begin{ex}
	Trong không gian $Oxyz$, cho điểm $A(4;-3;7)$ và $B(2;1;3)$. Một vectơ pháp tuyến của mặt phẳng trung trực của đoạn $AB$ là
	\choice
	{\True $\vec{n}=(1;-2;2)$}
	{$\vec{n}=(2;4;4)$}
	{$\vec{n}=(6;-2;10)$}
	{$\vec{n}=(-2;-4;4)$}
	\loigiai{
		Ta có $\overrightarrow{AB}(-2;4;-4)$ cùng phương với  $\vec{n}=(1;-2;2)$. Suy ra  $\vec{n}=(1;-2;2)$ là một vectơ pháp tuyến.
	}
\end{ex}

\begin{ex}%[HK1 - Chuyên Huỳnh Mẫn Đạt - Kiên Giang - 20-21]%[Phan Quốc Trí-EX5]%[2H3B2-2]%
	Trong không gian $Oxyz$, $(P)$ là mặt phẳng trung trực của đoạn $AB$, biết $A(1;3;0)$, $B(-2;1;-1)$. vectơ nào sau đây là vectơ pháp tuyến của $(P)$?
	\choice
	{$\overrightarrow{n}_4=(3;-2;-1)$}
	{$\overrightarrow{n}_2=(-3;2;-1)$}
	{$\overrightarrow{n}_3=(-3;4;1)$}
	{\True $\overrightarrow{n}_1=(3;2;1)$}
	\loigiai{
		Mặt phẳng $(P)$ có vectơ pháp tuyến là  $\overrightarrow{BA}= (3;2;1)$.
	}
\end{ex}

\begin{ex}%[Trần Bình Thuận - DA2]%[2H3B2-2]% câu 2
	Trong không gian $Oxyz$, vectơ nào sau đây là một vectơ pháp tuyến của $(P)$. Biết $\vec{u}=(1;-2;0)$, $\vec{v}=(0;2;-1)$ là cặp vectơ chỉ phương của $(P)$.
	\choice
	{$\vec{n}=(1;2;0)$}
	{\True $\vec{n}=(2;1;2)$}
	{$\vec{n}=(2;-1;2)$}
	{$\vec{n}=(0;1;2)$}
	\loigiai{
		Ta có $(P)$ có một vectơ pháp tuyến là $\vec{n}=\left[\vec{u},\vec{v}\right]=\left(
		\begin{vmatrix}
			-2&0\\
			2&-1
		\end{vmatrix};
		\begin{vmatrix}
			0&1\\
			-1&0
		\end{vmatrix};
		\begin{vmatrix}
			1&-2\\
			0&2
		\end{vmatrix}\right)=(2;1;2)$.
	}
\end{ex}

\begin{ex}
	Trong không gian  $Oxyz$, cho $(\alpha)$ song song với giá của $\vec{a}=(1;-2;-3)$, $\vec{b}=(-4;2;0)$. Vectơ nào dưới đây \textbf{không phải} là vectơ pháp tuyến của $(\alpha)$?
	\choice
	{$\vec{n_1}=(6;12;-6)$}
	{$\vec{n_2}=(1;2;-1)$}
	{$\vec{n_3}=(-2;-4;2)$}
	{\True $\vec{n_4}=(-3;-6;-3)$}
	\loigiai{
		$(\alpha)$ song song với giá của $\vec{a}=(1;-2;-3)$, $\vec{b}=(-4;2;0)$ nên $\vec{a}$, $\vec{b}$ là cặp vectơ chỉ phương của $(\alpha)$.\\
		Một vectơ pháp tuyến của mặt phẳng $(\alpha)$ là
		$$
		\begin{aligned}
			\vec{n}=[\vec{a}, \vec{b}] & =\left(\left|\begin{array}{cc}
				-2 & -3 \\ 2 & 0
			\end{array}\right| ;\left|\begin{array}{cc}
				-3 & 1 \\ 0 & -4
			\end{array}\right| ;\left|\begin{array}{cc}
				1 & -2 \\ -4 & 2
			\end{array}\right|\right) \\
			& =(6 ;12 ; -6) .
		\end{aligned}
		$$
		Ta có $\vec{n_1}=\vec{n}$; $\vec{n_2}=\dfrac{1}{6}\vec{n}$; $\vec{n_3}=-\dfrac{1}{3}\vec{n}$ là các vectơ pháp tuyến của $(\alpha)$.\\
		Vậy $\vec{n_4}=(-3;-6;-3)$ không phải là vectơ pháp tuyến của $(\alpha)$.
	}
\end{ex}


\begin{ex}%[Thi thử, Sở GD và ĐT - Hậu Giang, 2020]%[Trần Thành Thống, 12EX10]%[2H3B2-2]%
	Trong không gian $Oxyz$, cho ba điểm $A(2;0;0)$, $B(0;-3;0)$, $C(0;0;6)$. Tọa độ một vectơ pháp tuyến của mặt phẳng $(ABC)$ là
	\choice
	{$\overrightarrow{n}=(1;-2;3)$}
	{$\overrightarrow{n}=(3;2;1)$}
	{\True $\overrightarrow{n}=(3;-2;1)$}
	{$\overrightarrow{n}=(2;-3;6)$}
	\loigiai{
		Ta có $\overrightarrow{AB}=\left(-2;-3;0\right)\; \overrightarrow{AC}=\left(0;3;6\right)$.\\
		$\Rightarrow$ vectơ pháp tuyến của mặt phẳng $\left(ABC\right)$ là $\overrightarrow{v}=\left[\overrightarrow{AB};\overrightarrow{AC}\right]=\left(-18;12;-6\right)$.\\
		Ta có $\overrightarrow{v}=\left(-18;12;-6\right)$ cùng phương với $\overrightarrow{n}=\left(3;-2;1\right)$.
	}
\end{ex}

\begin{ex}
	Trong không gian $Oxyz$, cho ba điểm $A(2;-1;3)$, $B(4;0;1)$ và $C(-10;5;3)$. vectơ nào dưới đây là vectơ pháp tuyến của mặt phẳng $(ABC)$?
	\choice
	{$\overrightarrow{n}=(1;2;0)$}
	{$\overrightarrow{n}=(1;-2;2)$}
	{$\overrightarrow{n}=(1;8;2)$}
	{$\True \overrightarrow{n}=(1;2;2)$}
	\loigiai{
		Ta có $\overrightarrow{AB}=(2;1;-2)$, $\overrightarrow{AC}=(-12;6;0)$, $\left[\overrightarrow{AB},\overrightarrow{AC}\right]=(12;24;24)$.\\
		$\Rightarrow (ABC)$ có một vectơ pháp tuyến là $\overrightarrow{n}=\dfrac{1}{12}\left[\overrightarrow{AB},\overrightarrow{AC}\right]=(1;2;2)$.}
\end{ex}


\begin{ex}%[2H3B2-2]%[50 dạng toán đề minh họa 2020-Nguyễn Tâm Phục]%Câu 6.
	Trong không gian $Oxyz$, cho hai điểm $A(2;-1;5)$, $B(1;-2;3)$. Mặt phẳng $(\alpha)$ đi qua hai điểm $A$, $B$ và song song với trục $Ox$ có vectơ pháp tuyến $\overrightarrow{n}=(0;a;b)$. Khi đó tỉ số $\dfrac{a}{b}$ bằng
	\choice
	{\True $-2$}
	{$-\dfrac{3}{2}$}
	{$\dfrac{3}{2}$}
	{$2$}
	\loigiai{
		$\overrightarrow{BA}=(1;1;2)$; $\overrightarrow{i}=(1;0;0)$ là vectơ đơn vị của trục $Ox$.\\
		Vì $(\alpha)$ đi qua hai điểm $A$, $B$ và song song với trục $Ox$ nên $\left[\overrightarrow{BA},\overrightarrow{i}\right]=(0;2;-1)$ là một vectơ pháp tuyến của $(\alpha)$. Do đó $\dfrac{a}{b}=-2$.}
\end{ex}

\begin{dang}{Lập phương trình mặt phẳng khi biết các yếu tố liên quan}
	\begin{itemize}
		\item [\iconCV] \indamm{Công thức:} Cho $(P)$ qua điểm $M(x_0,y_0,z_0)$ và một vectơ pháp tuyến $\overrightarrow{n_P}=(a,b,c)$. Khi đó, phương trình $(P)$ là
		\begin{align*}
			\boxed{(P):a(x-x_0)+b(y-y_0)+c(z-z_0)=0}
		\end{align*}
		\item [\iconCV] \indamm{Một số cách xác định vectơ pháp tuyến thường gặp:}
		\begin{listEX}[1]
			\item [\ding{172}] Nếu $(P)\bot AB$ thì $\overrightarrow{n_P}=\overrightarrow{AB}$;
			\item [\ding{173}] Nếu $(P)$ là mặt phẳng trung trực của đoạn $AB$ thì $(P)$ qua trung điểm $I$ của $AB$ và $\overrightarrow{n_P}=\overrightarrow{AB}$;
			\item [\ding{174}] Nếu $(P)$ có cặp vectơ chỉ phương $\vec u$, $\vec v$ thì $\overrightarrow{n_P}=[\vec u,\vec v]$ là một vectơ pháp tuyến của $(P)$.
			\item [\ding{175}] Nếu $(P)$ qua ba điểm $A,B,C$ phân biệt và không thẳng hàng thì $\overrightarrow{n_P}=\left[ \overrightarrow{AB},\overrightarrow{AC} \right]$;
			\item [\ding{176}] Nếu $(P)$ qua hai điểm $A,B$ phân biệt và song song với $d$ thì $\overrightarrow{n_P}=\left[ \overrightarrow{AB},\overrightarrow{u_d} \right]$;
			\item [\ding{177}] Nếu $(P)$ qua điểm $A$ và chứa $d$ thì $\overrightarrow{n_P}=\left[ \overrightarrow{AM},\overrightarrow{u_d} \right]$, với $M \in d$.
		\end{listEX}
	\item [\iconCV] \indamm{Phương trình theo đoạn chắn:}
		Cho $(P)$ đi qua $A(a;0;0),\,B(0;b;0),\,C(0;0;c)$ với $abc \neq 0$ thì $(P):\dfrac{x}{a}+\dfrac{y}{b}+\dfrac{z}{c}=1$ (phương trình theo đoạn chắn)
	\end{itemize}
\end{dang}
\boxmini{BÀI TẬP TỰ LUẬN}
\setcounter{vd}{0}
\begin{vd}
	Trong không gian $Oxyz$, cho ba điểm $A(3;-2;-2)$, $B(3;2;0)$, $C(0;2;1)$.
	\begin{tasks}
		\task Lập phương trình mặt phẳng qua $A$ và vuông góc với $BC$.
		\task Lập phương trình mặt phẳng trung trực của đoạn $AB$.
		\task Lập phương trình mặt phẳng $(ABC)$.
	\end{tasks} 
	\loigiai{
		Ta có
		$\vec{AB}=(0;4;2)$, $\vec{AC}=(-3;4;3)$ là cặp vectơ chỉ phương của $(ABC)$.\\
		$\vec{n}=\left[\vec{AB},\vec{AC}\right]=(4;-6;12)$.\\
		Chọn $\vec{n}_1=\dfrac{1}{2} \vec{n}=(2;-3;6)$ là một vectơ pháp tuyến của $(ABC)$.\\
		Mặt phẳng $(ABC)$ đi qua điểm $C(0;2;1)$ và có một vectơ pháp tuyến $\vec{n}_1=(2;-3;6)$ nên $(ABC)$ có phương trình là
		$$2(x-0)-3(y-2)+6(z-1)=0\Leftrightarrow 2x-3y+6z=0.$$
		Vậy phương trình mặt phẳng cần tìm là $2x-3y+6z=0$.
	}
\end{vd}
\dongcham{11}

\begin{vd}
	Cho tứ diện $ ABCD $ có các đỉnh $ A(5;1;3)$, $B(1;6;2)$, $C(5;0;4),D(4;0;6) $.
	\begin{listEX}
		\item Hãy viết phương trình của các mặt phẳng $ (ACD) $ và $ (BCD) $;
		\item  Hãy viết phương trình mặt phẳng $ (\alpha) $ chứa cạnh $ AB $ và song song với cạnh $ CD $;
		\item Gọi $A'$, $B'$, $C'$ lần lượt là hình chiếu vuông góc của $A$, $B$, $C$ lên các trục $Ox$, $Oy$, $Oz$. Hãy viết phương trình mặt phẳng $(A'B'C')$.
	\end{listEX}
	\loigiai{
		\begin{listEX}
			\item Ta có $ \vec{AC}=(0;-1;1),\vec{AD}=(-1;-1;3),\vec{BC}=(4;-6;2),\vec{BD}=(4;-6;4) $.
			\begin{itemize}
				\item 	Mặt phẳng $ (ACD) $ qua $ A(5;1;3) $ và có vectơ pháp tuyến $ \vec{n}=\left[\vec{AC},\vec{AD}\right]=(-2;-1;-1) $ có phương trình là 
				$$ -2(x-5)-(y-1)-(z-3)=0\Leftrightarrow 2x+y+z-14=0 .$$
				\item 	Mặt phẳng $ (BCD) $ qua $ B(1;6;2) $ và có vectơ pháp tuyến $ \vec{n}=\left[\vec{BC},\vec{BD}\right]=(-12;-8;0) $ có phương trình là 
				$$ -12(x-1)-8(y-6)-0(z-3)=0\Leftrightarrow 3x+2y-15=0 .$$
			\end{itemize}
			\item Ta có $ \vec{AB}=(-4;5;-1) $, $\vec{CD}=(-1;0;2)$.\\
			Mặt phẳng $ (\alpha) $ chứa cạnh $ AB $ và song song với cạnh $ CD $ qua $ A(5;1;3) $ và có vectơ pháp tuyến $ \vec{n}=\left[\vec{AB},\vec{CD}\right]=(10;9;5) $ có phương trình là 
			$$ 10(x-5)+9(y-1)+5(z-3)=0\Leftrightarrow 10x+9y+5z-74=0 .$$
			\item Ta có $A'(5;0;0)$, $B'(0;6;0)$, $C'(0;0;4)$. Phương trình mặt phẳng $(A'B'C')$ là
			$$\dfrac{x}{5}+\dfrac{y}{6}+\dfrac{z}{4}=1.$$
		\end{listEX}
	}
\end{vd}
\dongcham{17}
\begin{vd}
	Viết phương trình của mặt phẳng
	\begin{tasks}(2)
		\task Chứa trục $ Ox $ và điểm $ M(-4;1;2) $;
		\task Chứa trục $ Oz $ và điểm $ P(3;0;-7) $.
	\end{tasks}
	\loigiai{
		\begin{listEX}
			\item Mặt phẳng $ (P) $ chứa trục $ Ox $ và điểm $ M(-4;1;2) $ nên có vectơ pháp tuyến \\$ \vec{n}=\left[\vec{i},\vec{OM}\right]=(0;-2;1) $, phương trình của $ (P) $ là $ -2y+z=0 $.
			\item Mặt phẳng $ (R) $ chứa trục $ Oz $ và điểm $ P(3;0;-7) $ nên có vectơ pháp tuyến \\$ \vec{n}=\left[\vec{k},\vec{OP}\right]=(0;3;0) $, phương trình của $ (R) $ là $ y=0 $.
		\end{listEX}
	}
\end{vd}
\dongcham{12}

\begin{vd}
	Một phần sân nhà bác An có dạng hình thang $ABCD$ vuông tại $A$ và $B$ với độ dài $AB=9$ m, $AD=5$ m và $BC=6$ m như Hình bên dưới. Theo thiết kế ban đầu thì mặt sân bằng phẳng và $A$, $B$, $C$, $D$ có độ cao như nhau. Sau đó bác An thay đổi thiết kế để nước có thể thoát về phía góc sân ở vị trí $C$ bằng cách giữ nguyên độ cao ở $A$, giảm độ cao của sân ở vị trí $B$ và $D$ xuống thấp hơn độ cao ở $A$ lần lượt là $6$ cm và $3{,}6$ cm. Để mặt sân sau khi lát gạch vẫn là bề mặt phẳng thì bác An cần phải giảm độ cao ở $C$ xuống bao nhiêu centimét so với độ cao ở $A$?
	\begin{center}
		% \includegraphics[scale=.4]{images/2P5-1-H5-9}
		% \hspace{0.5cm}
		\begin{tikzpicture}[scale=0.5, font=\footnotesize,line join=round, line cap=round, >=stealth]
			\path
			(0,0) coordinate (A) 
			(9,0) coordinate(B)
			(9,-6) coordinate(C)
			(0,-5) coordinate(D)
			;
			\draw[thick] (A)--(B)--(C)--(D)--cycle;
			\node [above] at ($(A)!0.5!(B)$) {$9$ m};
			\node [right] at ($(B)!0.5!(C)$) {$6$ m};
			\node [left] at ($(A)!0.5!(D)$) {$5$ m};
			\foreach \i/\g in {A/90,B/90,C/-90,D/-90}{\draw[fill=black](\i) circle (0pt) ($(\i)+(\g:4mm)$) node[scale=1]{$\i$};}
		\end{tikzpicture}
	\end{center}
	\loigiai{
		Tại vị trí ban đầu $A$, $B$, $C$, $D$ có độ cao như nhau, chọn hệ trục tọa độ có gốc tọa độ là điểm $A$ và các trục tọa độ lần lượt là $AD$, $AB$ và $Az$, với $Az \perp (ABCD)$.\\
		Khi đó $A(0 ; 0 ; 0)$, $D(5; 0 ; 0)$, $B(0; 9 ; 0)$, $C(6; 9 ; 0)$.\\
		Sau đó bác An thay đổi thiết kế để nước có thể thoát về phía góc sân ở vị trí $C$ bằng cách giữ nguyên độ cao ở $A$, giảm độ cao của sân ở vị trí $B$ và $D$ xuống thấp hơn độ cao ở $A$ lần lượt là $6$ cm và $3{,}6$ cm.\\
		Khi đó, $A(0 ; 0 ; 0)$, $D(5; 0 ; -3{,}6)$, $B(0; 9 ; -6)$.\\
		Ta có $\overrightarrow{AB}=(0 ; 9 ; -6)$, $ \overrightarrow{AD}=(5 ; 0 ; -3{,}6)$ là cặp vectơ chỉ phương của mặt phẳng $(ABD)$ nên một vectơ pháp tuyến của $(ABD)$ là $\left[\overrightarrow{AB}, \overrightarrow{AD}\right]=(-32{,}4 ; -30 ; -45)$.\\
		Vậy mặt phẳng $(ABD)$ qua $A(0 ; 0 ; 0)$ và có vectơ pháp tuyến $\vec{n}=(-32{,}4 ; -30 ; -45)$ nên có phương trình là
		\allowdisplaybreaks
		\begin{eqnarray*}
			-32{,}4 (x-2)-30(y+1)-45(z-3)=0 \qquad \text{hay } -32{,}4 x -30y -45z=0.
		\end{eqnarray*}
		Để mặt sân sau khi lát gạch vẫn là bề mặt phẳng thì bác An cần phải giảm độ cao ở $C$ xuống $k$ centimét so với độ cao ở $A$ nên suy ra $C(6; 9 ; -k)$.\\
		Ta có $A$, $B$, $C$, $D$ đồng phẳng\\
		$\Leftrightarrow C \in (ABD)$\\
		$\Leftrightarrow -32{,}4\cdot 6 -30 \cdot 9 -45\cdot (-k)=0$\\
		$\Leftrightarrow k=10{,}32$.\\
		Vậy bác An cần phải giảm độ cao ở $C$ xuống $10{,}32$ centimét so với độ cao ở $A$.
	}
\end{vd}
\dongcham{14}
\boxmini{BÀI TẬP TRẮC NGHIỆM}
\setcounter{ex}{0}
\begin{ex}%[2H3Y2-3]
	Phương trình mặt phẳng đi qua điểm $A(1;2;3)$ và có vectơ pháp tuyến $\overrightarrow{n}=(-2;0;1)$ là
	\choice
	{$-2x+z+1=0$}
	{$-2y+z-1=0$}
	{\True $-2x+z-1=0$}
	{$-2x+y-1=0$}
	\loigiai
	{
		Phương trình của mặt phẳng cần tìm là $-2(x-1)+0(y-2)+1(z-3)=0 \Leftrightarrow -2x+z-1=0$.
	}
\end{ex}
\cham{2}

\begin{ex}%[2H3B2-3]
	Phương trình nào được cho dưới đây là phương trình mặt phẳng $(Oyz)$?
	\choice
	{$x=y+z$}
	{$y-z=0$}
	{$y+z=0$}
	{\True $x=0$}
	\loigiai{
		Trong không gian với hệ tọa độ $Oxyz$, phương trình của mặt phẳng $(Oyz)$ là $x=0$.
	}
\end{ex}
\cham{2}

\begin{ex}%[Lê Quý Đôn, Hà Nội, lần 1, 2018]%[2H3B2-3]%[Nguyễn Bình Nguyên-12Ex7]
	Cho các điểm $A(0;1;2)$, $B(2;- 2;1)$, $C(- 2;0;1)$. Phương trình mặt phẳng đi qua $A$ và vuông góc với $BC$ là
	\choice
	{$2x - y - 1 = 0$}
	{$ - y + 2z - 3 = 0$}
	{\True $2x - y + 1 = 0$}
	{$y + 2z - 5 = 0$}
	\loigiai{
		Ta có $\overrightarrow{n}=\dfrac{1}{2}\overrightarrow{BC}=(-2;1;0)$.\\
		Vậy phương trình mặt phẳng đi qua $A$ và vuông góc với $BC$ có dạng
		$ - 2(x - 0) + 1(y - 1) = 0 \Leftrightarrow  - 2x + y - 1 = 0$ $ \Leftrightarrow 2x - y + 1 = 0$.}
\end{ex}
\cham{3}

\begin{ex}%[2H3B2-3]
	Cho hai điểm $A(4;0;1)$ và $B(-2;2;3)$. Phương trình nào dưới đây là phương trình mặt phẳng trung trực của đoạn thẳng $AB$?
	\choice
	{$3x-y-z+1=0$}
	{$3x+y+z-6=0$}
	{\True $3x-y-z=0$}
	{$6x-2y-2z-1=0$}
	\loigiai
	{
		Gọi $(\alpha)$ là mặt phẳng trung trực của đoạn thẳng $AB$. Khi đó $(\alpha)$ đi qua điểm $M(1;1;2)$, là trung điểm của $AB$, và nhận $\overrightarrow{AB}=(-6;2;2)$ làm vectơ pháp tuyến. Phương trình của mặt phẳng $(\alpha)$ là
		$$-6(x-1)+2(y-1)+2(z-2)=0 \Leftrightarrow -6x+2y+2z=0 \Leftrightarrow 3x-y-z=0.$$
	}
\end{ex}
\cham{3}
\begin{ex}
	Trong không gian $Oxyz$, cho hai điểm $A(1;1;1)$ và $B(1;3;5)$. Viết phương trình mặt phẳng trung trực của đoạn $AB$.
	\choice
	{$y-2z-6=0$}
	{$y-2z+2=0$}
	{$y-3z+4=0$}
	{\True $y+2z-8=0$}
	\loigiai{
		Ta có $I(1;2;3)$ là trung điểm của đoạn $AB$.\\
		Mặt phẳng trung trực của đoạn thẳng $AB$ đi qua $I$ và có vectơ pháp tuyến $\overrightarrow{AB}=(0;2;4)=2(0;1;2)$, suy ra phương trình mặt phẳng trung trực cần tìm là
		\begin{center}
			$0(x-1)+1(y-2)+2(z-3)=0\Leftrightarrow y+2z-8=0$.
		\end{center}
	}
\end{ex}

\begin{ex}%[Thi thử lần 1, THPT Văn Giang - Hưng Yên, 2019]%[Đỗ Đường Hiếu, 12EX-8-2019]%[2H3B2-3]%
	Trông không gian $Oxyz$, phương trình mặt phẳng $(P)$ đi qua $A(0;-1;4)$ và song song với giá của hai vectơ $\vec{u}=(3;2;1)$, $\vec{v}=(-3;0;1)$ là
	\choice
	{\True $x-3y+3z-15=0$}
	{$x-2y+3z-14=0$}
	{$x-y-z+3=0$}
	{$x-3y+3z-9=0$}
	\loigiai{
		Mặt phẳng $(P)$ có vectơ pháp tuyến là $\left[ \vec{u}; \vec{v}\right] =(2;-6;6)$. Hay $(P)$ có vectơ pháp tuyến là $\vec n=(1;-3;3)$.\\
		Phương trình mặt phẳng $(P)$ là
		$$1\cdot(x-0)-3\cdot (y+1)+3\cdot (z-4)=0\;\text{hay}\; (P)\colon x-3y+3z-15=0.$$
	}
\end{ex}

\begin{ex}
	Trong không gian $Oxyz$, cho ba điểm $A(3;-2;-2)$, $B(3;2;0)$, $C(0;2;1)$. Phương trình mặt phẳng $(ABC)$ là
	\choice
	{$2x-3y+6z+12=0$}
	{$2x+3y-6z-12=0$}
	{\True $2x-3y+6z=0$}
	{$2x+3y+6z+12=0$}
	\loigiai{
		Ta có
		$\vec{AB}=(0;4;2)$, $\vec{AC}=(-3;4;3)$ là cặp vectơ chỉ phương của $(ABC)$.\\
		$\vec{n}=\left[\vec{AB},\vec{AC}\right]=(4;-6;12)$.\\
		Chọn $\vec{n}_1=\dfrac{1}{2} \vec{n}=(2;-3;6)$ là một vectơ pháp tuyến của $(ABC)$.\\
		Mặt phẳng $(ABC)$ đi qua điểm $C(0;2;1)$ và có một vectơ pháp tuyến $\vec{n}_1=(2;-3;6)$ nên $(ABC)$ có phương trình là
		$$2(x-0)-3(y-2)+6(z-1)=0\Leftrightarrow 2x-3y+6z=0.$$
		Vậy phương trình mặt phẳng cần tìm là $2x-3y+6z=0$.
	}
\end{ex}
\cham{4}

\begin{ex}%[12-TN-BGD-3]%[Nguyễn Thành Khang, dự án 12-TN-BGD-3]%[2H3B2-3]%
	Trong không gian với hệ trục toạ độ $Oxyz$, cho ba điểm $A(1;0;0),B(0;-1;-1),C(5;-1;1)$. Mặt phẳng $(ABC)$ có phương trình là
	\choice
	{$2x+3y+5z-2=0$}
	{$2x-3y-5z-2=0$}
	{$2x-3y-5z+2=0$}
	{\True $2x+3y-5z-2=0$}
	\loigiai{
		Ta có $\vec{AB}=(-1;-1;-1), \vec{AC}=(4;-1;1)$ nên vectơ pháp tuyến của mặt phẳng $(ABC)$ là $\vec{n}=\left[\vec{AC},\vec{AB}\right]=(2;3;-5)$, mà mặt phẳng $(ABC)$ đi qua $A(1;0;0)$ nên có phương trình là $2x+3y-5z-2=0$.
	}
\end{ex}
\cham{4}
\begin{ex}
	Mặt phẳng $(\alpha)$ đi qua $A(-1; 4; -6)$ và chứa trục $Oy$ có phương trình là
	\choice
	{$-2x+y+z=0$}
	{$6x+z=0$}
	{$3x-y-6z+1=0$}
	{\True $6x-z=0$}
	\loigiai{
		Ta có $\vec{OA} = (-1; 4; -6)$, $\vec{j} = (0; 1; 0)$ song song hoặc trùng với $(\alpha)$. Nên $\vec{OA}$ và $\vec{j}$ là cặp vectơ chỉ phương của $(\alpha)$.\\
		Xét vectơ $\vec{n}=[\vec{OA}, \vec{j}] = \left(\left|\begin{array}{cc}4 & -6 \\ 1 & 0\end{array}\right| ;\left|\begin{array}{cc}-6 & -1 \\ 0 & 0\end{array}\right| ;\left|\begin{array}{cc}-1 & 4 \\ 0 & 1\end{array}\right|\right)=(6 ; 0 ; -1)$.\\
		Do đó $\vec{n} = (6; 0 ; -1)$ là một vectơ pháp tuyến của mặt phẳng $(\alpha)$.\\
		Phương trình mặt phẳng $(\alpha)$ là $6x-z=0$.
	}
\end{ex}
\cham{6}
\begin{ex}%[Thi thử L1, Cụm chuyên môn, Sở GDDT Hải Phòng, 2019]%[Nguyễn Quang Tân, dự án 12-EX-7-2019]%[2H3B2-3]%
	Trong không gian ${Oxyz}$, mặt phẳng chứa trục ${Ox}$ và đi qua điểm $A(1;1;-1)$ có phương trình là
	\choice
	{\True $ y+z=0$}
	{$ z+1=0$}
	{$ x+z=0$}
	{$ x-y=0$}
	\loigiai{
		Gọi $\vec n$ là vectơ pháp tuyến của mặt phẳng $(P)$ chứa trục ${Ox}$ và đi qua điểm $A(1;1;-1)$.\\
		Ta có $\heva{&\vec n \bot \overrightarrow {OA}  = \left(1;1; - 1\right)\\& \vec n \perp \vec i = \left( 1;0;0 \right).}$\\
		Chọn một vectơ pháp tuyến của mặt phẳng $(P)$ là $\vec{n} = \left[ \vec i, \vec{OA} \right] = \left(0;1;1\right)$.\\
		Vậy phương trình mặt phẳng là $y + z = 0$.
	}
\end{ex}
\cham{6}
\begin{ex}%[Đề thi thử - Trường THPT chuyên Lương Thế Vinh - Đồng Nai - Lần 1 - 2018]%[2H3B2-3]%[Kim Minh Bui - 12EX8]%
	Trong không gian $Oxyz,$ cho ba điểm $A(2;1;1),\ B(3;0;-1),\ C(2;0;3)$. Mặt phẳng $(\alpha)$ đi qua hai điểm $A,\ B$ và song song với đường thẳng $OC$ có phương trình là
	\choice
	{$3x+y-2z-5=0$}
	{$4x+2y+z-11=0$}
	{$x-y+z-2=0$}
	{\True $3x+7y-2z-11=0$}
	\loigiai{
		Gọi $\overrightarrow{n}$ là vtpt của mặt phẳng $(\alpha)$.\\
		Ta có $\begin{cases} AB \subset (\alpha) \\ OC \parallel (\alpha) \end{cases} \Rightarrow
		\begin{cases} \overrightarrow{n} \perp \overrightarrow{AB} \\ \overrightarrow{n} \perp \overrightarrow{OC} \end{cases}$ nên $\overrightarrow{n}$ cùng phương với $\overrightarrow{AB} \wedge \overrightarrow{OC}$.\\
		Ta có $\overrightarrow{AB}=(1;-1;-2),\ \overrightarrow{OC}=(2;0;3) \Rightarrow \overrightarrow{AB} \wedge \overrightarrow{OC}=(-3;-7;2) = (-1) \cdot (3;7;-2).$ Ta chọn $\overrightarrow{n} = (3;7;-2)$. \\ Phương trình mặt phẳng $(\alpha)$ là: $3x+7y-2z-11=0.$
	}
\end{ex}
\cham{6}
\begin{ex}
	Mặt phẳng đi qua hai điểm $A(1;2;-1)$, $B(0;4;3)$ và song song với trục $Oz$ có phương trình là
	\choice
	{\True $2x + y -4 =0$}
	{$4x - 4y +3 z+7 =0$}
	{$x + 2y -5=0$}
	{$2x + y+z -3 =0$}
	\loigiai{
		$\vec{AB}=(-1;2;4)$, $\vec{k}=(0;0;1)$. vectơ pháp tuyến của mặt phẳng cần tìm là $\vec{n}=\big[\vec{AB}, \vec{k}\big]=(2;1;0)$.\\
		Mặt phẳng qua $A(1;2;-1)$, nhận $\vec{n}=(2;1;0)$ làm vectơ pháp tuyến có phương trình là
		$$2(x-1)+1(y-2)+0(z+1)=0 \Leftrightarrow 2x+y-4=0.$$
		
	}
\end{ex}
\cham{6}

\begin{ex}%[2H3B2-3]
	Cho điểm $M(1;2;-3)$. Gọi $M_{1}$, $M_{2}$, $M_{3}$ lần lượt là hình chiếu vuông góc của $M$ lên trục $Ox$, $Oy$, $Oz$. Phương trình mặt phẳng đi qua ba điểm $M_{1}$, $M_{2}$, $M_{3}$ là
	\choice
	{\True $x+\dfrac{y}{2}-\dfrac{z}{3}=1$}
	{$\dfrac{x}{3}+\dfrac{y}{2}+\dfrac{z}{1}=1$}
	{$x+\dfrac{y}{2}+\dfrac{z}{3}=1$}
	{$x+\dfrac{y}{2}+\dfrac{z}{3}=-1$}
	\loigiai{
		Ta có $M_{1}(1;0;0)$, $M_{2}(0;2;0)$, $M_{3}(0;0;-3)$.\\
		Phương trình mặt phẳng đi qua $M_{1}$, $M_{2}$, $M_{3}$ là $x+\dfrac{y}{2}-\dfrac{z}{3}=1$.
	}
\end{ex}
\cham{4}

\begin{ex}%[2H3K2] 
	Mặt phẳng nào sau đây cắt các trục $Ox$, $Oy$, $Oz$ lần lượt tại các điểm $A$, $B$, $C$ sao cho tam giác $ABC$ nhận điểm $G\big(1; 2; 1\big)$ là trọng tâm?
	\choice{$x + 2y + 2z  - 6 = 0$}
	{\True $2x + y + 2z  - 6 = 0$}
	{$2x + 2y + z  - 6 = 0$}
	{$2x + 2y + 6z - 6 = 0$} 
\end{ex}
\cham{6}
\begin{ex}%[2H3B2-3]
	Cho mặt phẳng  $\left(P\right)$ đi qua điểm $M\left(2; - 4; 1\right)$  và chắn trên các trục tọa độ $Ox$, $Oy$, $Oz$ theo ba đoạn có độ dài đại số lần lượt là $a$, $b$, $c$. Phương trình tổng quát của mặt phẳng $\left(P\right)$ khi $a$, $b$, $c$ theo thứ tự tạo thành một cấp số nhân có công bội bằng $2$ là 
	\choice
	{$4x + 2y - z - 1 = 0$}
	{$4x -  2y + z +  1 = 0$}
	{$16x + 4y - 4z - 1 = 0$}
	{\True $4x + 2y +  z - 1 = 0$}
	\loigiai{Do giả thiết suy ra $a, b, c \neq 0$ và $b = 2a$, $c = 2b$. Giả sử $A\left(a; 0;0\right)$, $B\left(0; b;0\right)$ và $C\left(0; 0;c\right)$ khi đó phương trình mặt phẳng $\left(P\right)\colon \dfrac{x}{a} + \dfrac{y}{b} + \dfrac{z}{c} = 1$.  Do $M$ thuộc $\left(P\right)$  nên
		$$\dfrac{2}{a} - \dfrac{4}{b} + \dfrac{1}{c} = 1\Leftrightarrow \dfrac{2}{a} - \dfrac{4}{2a} + \dfrac{1}{4a} = 1\Leftrightarrow a = \dfrac{1}{4}.$$
		Suy ra $b = \dfrac{1}{2}$ và $c = 1$ do đó phương trình mặt phẳng $\left(P\right)\colon 4x + 2y + z - 1 = 0$.
	}
\end{ex}
\cham{8}

\begin{dang}{Vị trí tương đối của hai mặt phẳng}
	Cho hai mặt phẳng $(P) \colon a_1x+b_1y+c_1z+d_1=0$ và $(Q) \colon a_2x+b_2y+c_2z+d_2=0$.
		\begin{listEX}[1]
		\item [\ding{172}] Nếu $\heva{&\vec{n_1}= k \cdot \vec{n_2}\\& d_1 =k\cdot d_2}$ thì $(P)$ trùng $(Q)$.
		\item [\ding{173}] Nếu $\heva{&\vec{n_1}= k \cdot \vec{n_2}\\& d_1 \ne k\cdot d_2}$ thì $(P)$ song song $(Q)$.
		\item [\ding{174}] Nếu $\vec{n_1}$ không cùng phương với $\vec{n_2}$ thì $(P)$ cắt $(Q)$.
		\item [\ding{175}] Nếu $\vec{n_1} \perp \vec{n_2}$ hay $a_1a_2+b_1b_2+c_1c_2=0$ thì $(P) \perp (Q)$.
	\end{listEX}
\end{dang}
\setcounter{ex}{0}
\setcounter{vd}{0}
\boxmini{BÀI TẬP TỰ LUẬN}

\begin{vd}%[2H5H1-4]	
	Tìm các cặp mặt phẳng song song hoặc vuông góc trong các mặt phẳng sau
	\begin{listEX}[2]
		\item [] $(P)\colon 2x+3y-2z+7=0$
		\item [] $(Q)\colon 3x-2y-11=0$
		\item [] $(R)\colon 4x+6y-4z-9=0$
		\item [] $(T)\colon 7x+y-z+1=0$
	\end{listEX}
	\loigiai{
		Các mặt phẳng $(P)$, $(Q)$, $(R)$, $(T)$ có các vectơ pháp tuyến lần lượt là $\overrightarrow{n}_1=(2;3;-2)$, $\overrightarrow{n}_2=(3;-2;0)$, $\overrightarrow{n}_3=(4;6;-4)$, $\overrightarrow{n}_4=(7;1;-1)$.\\
		Ta có $\overrightarrow{n}_1\cdot \overrightarrow{n}_2=2 \cdot3 +3\cdot (-2)+(-2)\cdot0 =0$, suy ra $(P)\perp (Q)$.\\
		Vì $\dfrac{4}{2}=\dfrac{6}{3}=\dfrac{-4}{-2}\ne \dfrac{-9}{7}$ nên $(P) \parallel (R)$.\\
		Ta lại có $ (P)\perp (Q)$ và $(P) \parallel (R)$, suy ra $(Q) \perp (R)$.\\
		Ta có $\dfrac{2}{7}\ne \dfrac{3}{1}$ suy ra $\overrightarrow{n}_2$ và $\overrightarrow{n}_4$ không cùng phương.\\
		Mặt khác, $\overrightarrow{n}_1\cdot \overrightarrow{n}_4=2\cdot7+3\cdot1+(-2)\cdot(-1)=19\ne 0$. Suy ra $(P)$ và $(T)$ cắt nhau nhưng không vuông góc. Tương tự, ta cũng có $(Q)$ và $(T)$ cắt nhau nhưng không vuông góc.}
\end{vd}
\dongcham{11}
\begin{vd}
	Trong không gian $Oxyz$, cho mặt phẳng $(\alpha)\colon 2x-3y+z+5=0$.
	\begin{listEX}[1]
		\item Chứng minh rằng mặt phẳng $\left(\alpha'\right)\colon-4 x+6 y-2 z+7=0$ song song với $(\alpha)$.
		\item Viết phương trình mặt phẳng $(\beta)$ đi qua điểm $M(1 ; -2 ; 3)$ và song song với $(\alpha)$.
	\end{listEX}
	\loigiai{
		\begin{listEX}[1]
			\item Xét $(\alpha)\colon 2x-3y+z+5=0$ và $\left(\alpha'\right)\colon -4x+6y-2z+7=0$.\\
			Ta có $\dfrac{2}{-4}=\dfrac{-3}{6}=\dfrac{1}{-2} \neq \dfrac{5}{7}$ nên $(\alpha) \parallel \left(\alpha'\right)$.
			\item Mặt phẳng $(\alpha)$ có vectơ pháp tuyến $\vec{n}=(2 ;-3 ; 1)$.\\
			Vì $(\beta) \parallel (\alpha)$ nên $(\beta)$ có vectơ pháp tuyến $\vec{n}=(2 ;-3 ; 1)$.\\
			Vậy mặt phẳng $(\beta)$ đi qua điểm $M(1 ;-2 ; 3)$ và có vectơ pháp tuyến $\vec{n}=(2 ;-3 ; 1)$ nên có phương trình là
			\allowdisplaybreaks
			\begin{eqnarray*}
				2(x-1)-3(y+2)+(z-3)=0 \qquad \text{hay } 2x-3y+z-11=0.
			\end{eqnarray*}
		\end{listEX}
		
	}
\end{vd}

\dongcham{10}

\begin{vd}%[Thi thử, Sở GD và ĐT-THANH HÓA, 2020]%[Nguyễn Hữu Tính]%[2H3B2-3]%
	Trong không gian $Oxyz$, cho  hai mặt phẳng $(Q) \colon x+y+3z=0$, $(R) \colon  2x-y+z=0$.
	\begin{enumEX}[a)]{1}
		\item Xét vị trí tương đối của $(Q)$ và $(R)$;
		\item Viết trình của mặt phẳng $(P)$ đi qua điểm $B(2;1;-3)$, đồng thời vuông góc với $(Q)$ và $(R)$.
	\end{enumEX}
	\loigiai{
		vectơ pháp tuyến $(P)$ là $n_{\overrightarrow{P}}= \left[ n_{\overrightarrow{Q}},n_{\overrightarrow{R}} \right]= (4;5;-3)$.\\
		Phương trình mặt phẳng $(P)$ là $4(x-2)+5(y-1)-3(z+3)=0 \Leftrightarrow 4x+5y-3z-22=0$.
	}
\end{vd}
\dongcham{10}
\begin{vd}%[Đề tập huấn, Sở GD - ĐT tỉnh Quảng Bình, 2019]%[Nguyễn Tiến, dự án 12EX5]%[2H3K2-3]%
	Trong không gian với hệ tọa độ $Oxyz$, cho hai điểm $A(-2;4;-1)$, $B(1;1;3)$ và mặt phẳng $(P)$ có phương trình $x-3y+2z-5=0$. Viết phương trình mặt phẳng $(Q)$ đi qua hai điểm $A$, $B$ và vuông góc với mặt phẳng $(P)$.
	\loigiai{
		$\left.\begin{array}{l} \overrightarrow{AB}=(3;-3;4)\\ \overrightarrow{n}_{(P)}=(1;-3;2)\end{array}\right\}\Rightarrow\left[\overrightarrow{AB},\overrightarrow{n}_{(P)}\right]=(6;-2;-6)=2(3;-1;-3)$.\\
		Mặt phẳng $(Q)$ đi qua điểm $A$ và có vectơ pháp tuyến $\overrightarrow{n}_{(Q)}=(3;-1;-3)$ có phương trình
		\begin{eqnarray*}
			& & 3(x+2)-(y-4)-3(z+1)=0\\
			&\Leftrightarrow & 3x-y-3z+7=0.
		\end{eqnarray*}
	}
\end{vd}
\dongcham{10}

\boxmini{BÀI TẬP TRẮC NGHIỆM}

\begin{ex}%[2H3B2-7]
	Cho mặt phẳng $(P)\colon -x+y+3z+1=0$. Mặt phẳng song song với mặt phẳng $(P)$ có phương trình nào sau đây?
	\choice
	{\True $2x-2y-6z+7=0$}
	{$-2x+2y+3z+5=0$}
	{$x-y+3z-3=0$}
	{$-x-y+3z+1=0$}
	\loigiai{
		vectơ pháp tuyến của mặt phẳng $ (P) $ là $ \overrightarrow{n}=(-1;1;3)$ cùng phương với vectơ $\overrightarrow{n}=(2;-2;-6) $. Vì $ \dfrac{2}{-1}\neq \dfrac{7}{1} $ nên phương trình mặt phẳng song song với $ (P) $ là $2x-2y-6z+7=0$.
	}
\end{ex}
\cham{2}

\begin{ex}%[2H3B2-7]
	Cho hai mặt phẳng $(P) \colon 2x+4y+3z-5=0$ và $(Q) \colon mx-ny-6z+2-0$. Giá trị của $m,n$ sao cho $(P) \parallel (Q)$ là
	\choice
	{$m=4;n=-8$}
	{$m=n=4$}
	{\True $m=-4;n=8$}
	{$m=n=-4$}
	\loigiai{
		$(P)$ có vectơ chỉ phương $\overrightarrow{u}_{(P)}=(2;4;3)$, $(Q)$ có vectơ chỉ phương $\overrightarrow{u}_{(Q)}=(m;-n;-6)$.\\ Để hai mặt phẳng trên song song thì $\overrightarrow{u}_{(Q)}=k\overrightarrow{u}_{(P)}\,(k \neq 0) \Leftrightarrow \heva{&m=2k\\&-n=4k\\&-6=3k} \Rightarrow \heva{&k=-2\\&m=-4\\&n=8.}$
	}
	
\end{ex}
\cham{3}

\begin{ex}
	Cho hai mặt phẳng $(P)\colon x+my+(m-1)z+1=0$ và $(Q)\colon x+y+2z=0$. Tập hợp tất cả các giá trị $m$ để hai mặt phẳng này \textbf{không} song song là
	\choice
	{$(0;+\infty)$}
	{$\mathbb{R}\setminus\{-1;1;2\}$}
	{$(-\infty;3)$}
	{\True $\mathbb{R}$}
	\loigiai{
		Ta có $A(0;0;0)\in (Q)$.\\
		$(P)\parallel (Q)\Leftrightarrow \heva{&\dfrac{1}{1}=\dfrac{m}{1}=\dfrac{m-1}{2}\\&A(0;0;0)\notin (P)}$. Hệ này vô nghiệm. Do đó $(P)$ không song song với $(Q)$, với mọi giá trị của $m$. 
	} 
\end{ex}
\cham{4}

\begin{ex}
	Cho mặt phẳng $(\alpha)\colon x+y+z-1=0$. Trong các mặt phẳng sau, tìm mặt phẳng vuông góc với mặt phẳng $(\alpha)$.
	\choice
	{$2x-y+z+1=0$}
	{\True $2x-y-z+1=0$}
	{$2x+2y+2z-1=0$}
	{$x-y-z+1=0$}
	\loigiai{
		Mặt phẳng $(\alpha)$ có $\overrightarrow{n}_{(\alpha)}=(1;1;1)$.\\
		Mặt phẳng $2x-y-z+1=0$ có vectơ pháp tuyến $\overrightarrow{n}_1=(2;-1;-1)\Rightarrow\overrightarrow{n}_{(\alpha)}\cdot\overrightarrow{n}_1=0$ nên mặt phẳng $(\alpha)$ vuông góc với mặt phẳng $2x-y-z+1=0$.
	}
\end{ex}

\begin{ex}%[2H3B2-7]
	Cho mặt phẳng $(P)\colon 2x-y+2z-3=0$ và $(Q) \colon x+my+z-1=0$. Tìm tham số $m$ để hai mặt phẳng $P$ và $Q$ vuông góc với nhau.
	\choice
	{$m=-4$}
	{$m=- \dfrac{1}{2}$}
	{$m=\dfrac{1}{2}$}
	{\True $m=4$}
	\loigiai
	{ Mặt phẳng $(P)$ và $(Q)$ có vectơ pháp tuyến lần lượt là $\overrightarrow{n}_1=(2;-1;2)$ và $\overrightarrow{n}_2=(1;m;1)$. \\
		Do đó $(P) \perp (Q) \Leftrightarrow \overrightarrow{n}_1 \cdot \overrightarrow{n}_2=0 \Leftrightarrow 2-m+2=0 \Leftrightarrow m=4$.
		
	}
\end{ex}
\cham{2}

\begin{ex}
	Cho hai mặt phẳng $(P)\colon x+2y-z-1=0$, $(Q)\colon 3x-(m+2)y+(2m-1)z+3=0$. Tìm $m$ để hai mặt phẳng $(P)$ và $(Q)$ vuông góc với nhau.
	\choice
	{\True $m=0$}
	{$m=2$}
	{$m=-2$}
	{$m=-1$}
	\loigiai{
		vectơ pháp tuyến của $(P)$, $(Q)$ lần lượt là $\overrightarrow{n}_P=(1;2;-1)$ và $\overrightarrow{n}_Q=(3;-m-2;2m-1)$.\\
		$(P)\perp (Q)\Leftrightarrow \overrightarrow{n}_P\cdot\overrightarrow{n}_Q=0\Leftrightarrow 3-2(m+2)-2m+1=0\Leftrightarrow m=0$.
	}
\end{ex}

\begin{ex}%[2H3B2-3]%
	Mặt phẳng đi qua $A(1;3;-2)$ và song song với mặt phẳng $(P) \colon 2x-y+3z+4=0$ có phương trình là
	\choice
	{\True $2x-y+3z+7=0$}
	{$2x-y+3z-7=0$}
	{$2x+y-3z+7=0$}
	{$2x+y+3z+7=0$}
	\loigiai{
		Ta có $\vec{n}=\vec{n_{(P)}}=(2;-1;3)$. Khi đó phương trình mặt phẳng qua $A(1;3;-2)$ và song song $(P)$ là
		$$2(x-1)-1(y-3)+3(z+2)=0\Leftrightarrow 2x-y+3z+7=0.$$
	}
\end{ex}


\begin{ex}%[KSCL giữa HK2 Cụm trường THPT TP Nam Định]%[Nguyễn Tiến, 12EX7]%[2H3B2-3]%
	Cho điểm $A(2;-1;-3)$ và mặt phẳng $(P)\colon 3x-2y+4z-5=0$. Mặt phẳng $(Q)$ đi qua $A$ và song song với mặt phẳng $(P)$ có phương trình là
	\choice
	{\True $(Q)\colon 3x-2y+4z+4=0$}
	{$(Q)\colon 3x+2y+4z+8=0$}
	{$(Q)\colon 3x-2y+4z+5=0$}
	{$(Q)\colon 3x-2y+4z-4=0$}
	\loigiai{
		Do mặt phẳng $(Q)$ song song với mặt phẳng $(P)$ nên có vectơ pháp tuyến là $\overrightarrow{n}=(3;-2;4)$.\\
		Phương trình mặt phẳng $(Q)\colon 3(x-2)-2(y+1)+4(z+3)=0 \Leftrightarrow 3x-2y+4z+4=0$.
	}
\end{ex}

\begin{ex}
	Cho mặt phẳng $(P)$ đi qua các điểm $A(-2; 0; 0)$, $B(0; 3; 0)$, $C(0; 0; -3)$. Mặt phẳng $(P)$ vuông góc với mặt phẳng nào trong các mặt phẳng sau?
	\choice
	{\True $2x+2y-z-1=0$}
	{$x+y+z+1=0$}
	{$3x-2y+2z+6=0$}
	{$x-2y-z-3=0$}
	\loigiai{
		Mặt phẳng $(P)\colon \dfrac{x}{-2}+\dfrac{y}{3}+\dfrac{z}{-3}=1$ hay $(P)\colon 3x-2y+2z+6=0$ có vectơ pháp tuyến $\vec{n}=(3;-2;2)$.\\
		Ta có $3\cdot 2 -2\cdot 2-2\cdot 1=0$ nên $(P)$ vuông góc với mặt phẳng $2x+2y-z-1=0$.
	}
\end{ex}

\begin{ex}
	Mặt phẳng qua $A(1;2;-1)$ và vuông góc với các mặt phẳng $(P) \colon 2x-y+3z-2=0$; $(Q) \colon x+y+z-1=0$ có phương trình là
	\choice
	{$x-y+z+2=0$}
	{$4x-y+z-1=0$}
	{$x+y+2z-1=0$}
	{\True $4x-y-3z-5=0$}
	\loigiai{
		vectơ pháp tuyến của mặt phẳng $(P)$ và $(Q)$ lần lượt là $\overrightarrow{n_1}=(2;-1;3)$ và $\overrightarrow{n_2}=(1;1;1)$.
		\\
		Ta có $\left[\overrightarrow{n_1};\overrightarrow{n_2}\right]=(-4;1;3)$.
		Mặt phẳng cần tìm qua $A(1;2;-1)$ và có vectơ pháp tuyến là $\overrightarrow{n}=(-4;1;3)$ nên có phương trình là
		$$-4 \cdot (x-1)+1 \cdot (y-2)+3 \cdot (z+1)=0\Leftrightarrow4x-y-3z-5=0.$$}
\end{ex}

\begin{ex}
	Cho hai mặt phẳng $(P)$, $(Q)$ lần lượt có phương trình là $x+y-z=0$, $x-2y+3z=4$ và cho điểm $M(1;-2;5)$. Tìm phương trình mặt phẳng $(\alpha)$ đi qua điểm $M$ và đồng thời vuông góc với hai mặt phẳng $(P)$, $(Q)$.
	\choice
	{$5x+2y-z+14=0$}
	{\True $x-4y-3z+6=0$}
	{$x-4y-3z-6=0$}
	{$5x+2y-z+4=0$}
	\loigiai{
		Ta có $\vec{n}_{(P)} = (1;1;-1)$ và
		$\vec{n}_{(Q)} = (1;-2;3).$\\
		Suy ra $\left[\vec{n}_{(P)},\vec{n}_{(Q)}\right]= (1;-4;-3).$\\
		Do $(\alpha)$ vuông góc với $(P)$ và $(Q)$ nên $\heva{&\vec{n}_{(\alpha)} \perp \vec{n}_{(P)}\\&\vec{n}_{(\alpha)} \perp \vec{n}_{(Q)}}$. \\
		Chọn $\vec{n}_{(\alpha)} = \left[\vec{n}_{(P)},\vec{n}_{(Q)}\right]=(1;-4;-3)$. Hơn nữa, $(\alpha)$ đi qua $M(1;-2;5)$ nên có phương trình là $$(x-1)-4(y+2)-3(z-5)=0\Leftrightarrow x-4y-3z+6=0.$$
	}
\end{ex}

\begin{ex}%[Thi thử, THPT chuyên Quang Trung, 2020]%[Phạm Doãn Lê Bình, 12EX2-2020]%[2H3B2-3]%
	Cho điểm $A(-4;1;1)$ và mặt phẳng $(P)\colon x-2y-z+4=0$. Mặt phẳng $(Q)$ đi qua điểm $A$ và song song với mặt phẳng $(P)$ có phương trình là
	\choice
	{$(Q)\colon x-2y-z+7=0$}
	{\True $(Q)\colon x-2y-z-7=0$}
	{$(Q)\colon x-2y+z+5=0$}
	{$(Q)\colon x-2y+z-5=0$}
	\loigiai{
		Do $(Q)\parallel (P)$ nên phương trình của $(Q)$ có dạng $x-2y-z+c=0$ ($c\ne 4$).\\
		Do $A \in (Q)$ nên $-4-2\cdot 1 - 1 + c = 0 \Leftrightarrow c = 7$ (thỏa).\\
		Vậy $(Q)\colon x-2y-z+7=0$.
	}
\end{ex}

\begin{ex}%[Thi thử L1, THPT Chuyên ĐH Vinh, Nghệ An, 2019]%[Nguyễn Đắc Giáp, dự án 12EX6]%[2H3B2-3]%
	Cho hai mặt phẳng $(P)\colon x-3y+2z-1=0$, $(Q)\colon x-z+2=0$. Mặt phẳng $\left(\alpha\right)$ vuông góc với hai mặt phẳng $(P),(Q)$ đồng thời cắt trục $Ox$ tại điểm có hoành độ bằng $3$. Phương trình của $\left(\alpha\right)$ là
	\choice
	{$-2x+z+6=0$}
	{$-2x+z-6=0$}
	{\True $x+y+z-3=0$}
	{$x+y+z+3=0$}
	\loigiai{
		Mặt phẳng $(P)$ có một vectơ pháp tuyến là $\overrightarrow{n}_P=(1;-3;2)$.\\
		Mặt phẳng $(Q)$ có một vectơ pháp tuyến là $\overrightarrow{n}_Q=(1;0;-1)$.\\
		Vì mặt phẳng $(\alpha)$ vuông góc với hai mặt phẳng $(P)$ và $(Q)$ nên $(\alpha)$ có một vectơ pháp tuyến là	 $\overrightarrow{n}_{\alpha}=\left[\overrightarrow{n}_P,\overrightarrow{n}_Q\right]=3(1;1;1)$.\\
		Mà mặt phẳng $(\alpha)$ đi qua $A(3;0;0)$, nên suy ra phương trình là $\left(\alpha\right)\colon x+y+z-3=0$.
	}
\end{ex}

\begin{ex}%[2H3K2-3]%
	Cho $A\left( 1;-1;2 \right);\ B\left( 2;1;1 \right)$ và mặt phẳng $\left( P \right):x+y+z+1=0$. Mặt phẳng $\left( Q \right)$ chứa $A,\ B$ và vuông góc với mặt phẳng $\left( P \right)$. Mặt phẳng $\left( Q \right)$ có phương trình là
	\choice
	{$3x-2y-z+3=0$}
	{\True $3x-2y-z-3=0$}
	{$-x+y=0$}
	{$x+y+z-2=0$}
	\loigiai
	{
		Ta có $\vec{A B}=(1 ; 2 ;-1)$ và vectơ pháp tuyến của $(P)$ là $\vec{n}_{P}=(1 ; 1 ; 1)$. Gọi vectơ pháp tuyến của $(Q)$ là $\vec{n}_{Q}$.\\
		Vì $(Q)$ chứa $A,B$ nên $\vec{n_{Q}} \perp \vec{A B}$, mặt khác $(Q) \perp(P)$ nên $\vec{n_{Q}} \perp \vec{n_{P}}$. \\
		Từ đó suy ra $\vec{n_{Q}}=\left[\vec{A B}, \vec{n_{P}}\right]=(3 ;-2 ;-1)$.\\
		$(Q)$ đi qua $A(1;-1;2)$ và có vectơ pháp tuyến $\vec{n_{Q}}=(3 ;-2 ;-1)$ nên $(Q)$ có phương trình là
		$$(Q):3(x-1)-2(y+1)-(z-2)=0 \Leftrightarrow 3 x-2 y-z-3=0.$$
	}
\end{ex}

\begin{ex}%[2H3K2-3]%[Đề thi thử L2, THPT Nguyễn Quang Diêu, 2018]%[Đỗ Đường Hiếu, 12EX-9]%
	Cho hai điểm $A(2;4;1)$, $B(-1;1;3)$ và mặt phẳng $(P)\colon x-3y+2z-5=0$. Một mặt phẳng $(Q)$ đi qua hai điểm $A$, $B$ và vuông góc với mặt phẳng $(P)$ có dạng là $ax+by+cz-11=0$. Tính $a+b+c$.
	\choice
	{$a+b+c=-7$}
	{$a+b+c=10$}
	{\True $a+b+c=5$}
	{$a+b+c=3$}
	\loigiai{
		Ta có $\overrightarrow{AB}=\left(-3;-3;2\right)$ và vectơ pháp tuyến của mặt phẳng $(P)$ là $\overrightarrow{n}_P=\left(1;-3;2\right)$.\\
		Mặt phẳng $(Q)$ đi qua hai điểm $A$, $B$ và vuông góc với mặt phẳng $(P)$ có một vectơ chỉ phương là
		$$\overrightarrow{n}_Q=\left[\overrightarrow{AB}, \overrightarrow{n}_P\right]=\left(0;8;12\right) =4\left(0;2;3\right).$$
		Phương trình mặt phẳng $(Q)$ là
		$$0\cdot (x-2)+2\cdot (y-4)+3\cdot (z-1)=0.$$
		Hay $(Q)\colon 2y+3z-11=0$.
		Từ đó suy ra $a=0$, $b=2$, $c=3$. Do đó $a+b+c=0+2+3=5$.
	}
\end{ex}

\begin{dang}{Khoảng cách từ một điểm đến mặt phẳng, khoảng cách giữa hai mặt phẳng song song}
	\begin{itemize}
		\item [\iconMT] \indam{Khoảng cách từ một điểm đến mặt phẳng:} Cho điểm $M(x_0;y_0;z_0)$ và mặt phẳng $(P) \colon ax+by+cz+d=0$. Khi đó
		\boxmini{$\mathrm{d}\left(M,(P) \right)=\dfrac{\bigg|ax_0+by_0+cz_0+d\bigg|}{\sqrt{a^2+b^2+c^2}}$}
		\item [\iconMT] \indam{Khoảng cách giữa hai mặt phẳng song song:} 	Cho hai mặt phẳng $(P) \colon ax + by + cz + d_1=0$ và $(Q) \colon ax + by + cz + d_2=0$ song song nhau. 
		Khi đó
		\boxmini{$\mathrm{d}\left((P),(Q) \right)=\dfrac{\bigg|d_1-d_2\bigg|}{\sqrt{a^2+b^2+c^2}}$}
	\end{itemize}
\end{dang}
\setcounter{ex}{0}
\setcounter{vd}{0}
\boxmini{BÀI TẬP TỰ LUẬN}
\begin{vd}%[2H5H1-5] %[Dang]
	Tính khoảng cách từ điểm $A(1;2;3)$ đến các mặt phẳng sau
	\begin{enumEX}[a)]{3}
		\item $\left(P\right) \colon 3x + 4z + 10 = 0$;
		\item $\left(Q\right) \colon 2x - 10 = 0$;
		\item $\left(R\right) \colon 2x + 2y + z - 3 = 0$.
	\end{enumEX}
	
	\loigiai{
		\begin{listEX}
			\item $\mathrm{d}\left( A;\left( P \right) \right) = \dfrac{\left| {3 \cdot 1 + 0 \cdot 2 + 4 \cdot 3 + 10} \right|}{\sqrt {3^2 + 4^2}} = 5$.
			\item $\mathrm{d}\left(A;\left( Q \right) \right) = \dfrac{\left| {2 \cdot 1 - 10} \right|}{\sqrt {2^2} } = 2$.
			\item $\mathrm{d}\left( A;\left( R \right) \right) = \dfrac{\left| {2 \cdot 1 + 2 \cdot 2 + 1 \cdot 3 - 3} \right|}{\sqrt {2^2 + 2^2 + 1^2} } = 2$.
		\end{listEX}
	}
\end{vd}
\dongcham{3}
\begin{vd}%[2H5N1-4]%[2H5H1-5]%[Dự án tex hóa sách bài tập Toán 12 CTST]%[Lê Thị Thúy Hằng]
	Cho hai mặt phẳng $(P) \colon 2x+y+2z+12=0$, $(Q) \colon 4x+2y+4z-6=0$.
	\begin{enumerate}
		\item Chứng minh $(P) \parallel (Q)$.
		\item Tính khoảng cách giữa hai mặt phẳng $(P)$ và $(Q)$.
	\end{enumerate}
	\loigiai{
		\begin{enumerate}
			\item Xét hai mặt phẳng $(P) \colon 2x+y+2z+12=0$ và $(Q) \colon 4x+2y+4z-6=0$, ta có
			$\dfrac{4}{2} = \dfrac{2}{1} = \dfrac{4}{2} \ne \dfrac{-6}{12}$, suy ra $(P) \parallel (Q)$.
			\item Trên mặt phẳng $(Q)$, lấy điểm $M(0;1;1)$.\\
			Ta có
			$$ \mathrm{d}((P),(Q)) = \mathrm{d} (M, (P)) = \dfrac{\left| 2 \cdot 0 + 1 \cdot 1 + 2 \cdot 1 + 12 \right|}{\sqrt{2^2+1^2+2^2}}=\dfrac{15}{3}=5.$$
		\end{enumerate}
	}
\end{vd}
\dongcham{7}
\begin{vd}
	\immini{
		Một kĩ sư xây dựng thiết kế khung một ngôi nhà trong không gian $Oxyz$ như Hình 9 nhờ một phần mềm đồ họa máy tính.
		\begin{enumerate}
			\item Viết phương trình mặt phẳng mái nhà $(DEMN)$.
			\item Tính khoảng cách từ điểm $B$ đến mái nhà $(DEMN)$.
		\end{enumerate}
	}
	{
		\begin{tikzpicture}[scale=.7, font=\footnotesize, line join=round, line cap=round, >=stealth]
			\path 
			(0:0) coordinate (O)
			(-135:2) coordinate (A)
			(0:4) coordinate (C)
			($(A)+(-135:0.5)$) coordinate (x)
			($(A)+(45:2)$) coordinate (O)
			($(A)+(C)-(O)$) coordinate (B)
			($(O)+(90:4)$) coordinate (D)
			($(D)+(90:0.8)$) coordinate (z)
			($(C)+(0:0.5)$) coordinate (y)
			($(A)+(90:4)$) coordinate (E)
			($(B)+(90:4)$) coordinate (F)
			($(C)+(90:4)$) coordinate (H)
			($(D)+(20:2)$) coordinate (N)
			($(E)+(20:2)$) coordinate (M)
			($(O)+(-50:4.5)$) coordinate (h)
			;
			\draw[dashed] (A)--(O)--(C) (O)--(D)--(H);
			\draw[->] (A)--(x) node[below]{$x$};
			\draw[->](C)--(y) node[above]{$y$};
			\draw[->](D)--(z) node[right]{$z$};
			\draw 	(B)--(F)--(E)--(A)--cycle
			(D)--(E)--(M)--(N)--cycle
			(B)--(C)--(H)--(F)--(M) (N)--(H) 
			;
			\draw (A) node[left]{$A(6;0;0)$}
			(O) node[below right]{$O(0;0;0)$}
			(C) node[below right]{$C(0;4;0)$}
			(D) node[left]{$D(0;0;4)$}
			(M) node[right]{$M(6;2;6)$}
			(N) node[above right]{$N(0;2;6)$}
			(E) node[left]{$E(6;0;4)$}
			(B) node[right]{$B$}
			(H) node[right]{$H$}
			(h) node[above]{Hình 9}
			;
			%pic[draw,angle radius=2mm]{right angle=C--O--O'}
			%pic[draw,angle radius=2mm]{right angle=A--O--C}
			%pic[draw,angle radius=2mm]{right angle=O'--O--A}
			%;
			%\foreach \x/\g in {A/170,B/-15,C/-30,O/180,E/170,F/-15,H/0,D/180}
			%\draw[fill=black] 	(\x) circle (.5pt)
			%($(\g:.5)+(\x)$) node {$\x$};	
		\end{tikzpicture}
	}
	\loigiai{
		\begin{enumerate}
			\item Mặt phẳng $(DEMN)$ có cặp vectơ chỉ phương là $\overrightarrow{DE} =(6;0;0)$, $\overrightarrow{DN} = (0;2;2)$. Ta có $\left[ \overrightarrow{DE}, \overrightarrow{DN} \right] =(0;-12;12)$, suy ra $(DEMN)$ có vectơ pháp tuyến là $$\overrightarrow{n} = -\dfrac{1}{12} \left[ \overrightarrow{DE}, \overrightarrow{DN} \right] = (0;1;-1).$$
			Phương trình của mặt phẳng $(DEMN)$ là $y-z+4=0$.
			\item $B(6;4;0)$, suy ra $\mathrm{d} (B,(DEMN)) = \dfrac{\left| 4+4 \right|}{\sqrt{0^2+1^2+(-1)^2}} = \dfrac{8}{\sqrt{2}} = 4\sqrt{2}$.
		\end{enumerate}
	}
\end{vd}
\dongcham{7}
\begin{vd}
	Cho hình hộp chữ nhật $ABCD.A'B'C'D'$ có $DA=2$, $DC=3$, $DD'=2$. Tính khoảng cách từ đỉnh $B'$ đến mặt phẳng $(BA'C')$.
	\loigiai{
		\begin{center}
			\begin{tikzpicture}[scale=1, font=\footnotesize, line join=round, line cap=round, >=stealth]
				\path 
				(0,0) coordinate (D)
				($(D)+(-135:2)$) coordinate (A)
				($(A)+(-135:1)$) coordinate (x)
				(0:4.5) coordinate (C)
				($(C)+(0:1)$) coordinate (y)
				($(A)+(C)-(D)$) coordinate (B)
				($(D)+(90:3)$) coordinate (D')
				($(D')+(90:1)$) coordinate (z)
				($(A)+(90:3)$) coordinate (A')
				($(B)+(90:3)$) coordinate (B')
				($(C)+(90:3)$) coordinate (C')
				;
				\draw[dashed] (A)--(D)--(C)--cycle (D)--(D');
				\draw (A')--(A)--(B)--(B')--(A')--(B')--(C')--(D') (B)--(C)--(C')--(B') (D')--(A')--(C')--(B);
				\draw[->] (A')--(B);
				\draw[->] (A)--(x) node[below]{$x$};
				\draw[->] (C)--(y) node[below]{$y$};
				\draw[->] (D')--(z) node[left]{$z$};
				\foreach \x/\g in {A/170,B/-15,C/-60,D/180,D'/180,A'/180,B'/90,C'/12}
				\draw	(\x)
				($(\g:.2)+(\x)$) node {$\x$};	
			\end{tikzpicture}
		\end{center}
		Chọn hệ tọa độ $Oxyz$ sao cho gốc tọa độ $O$ trùng với điểm $D$.\\
		Khi đó, tọa độ các đỉnh của hình hộp chữ nhật $ABCD.A'B'C'D'$ là 
		$D(0,0,0)$, $A(2,0,0)$, $C(0,3,0)$, $B(2,3,0)$, $D'(0,0,2)$, $A'(2,0,2)$, $B'(2,3,2)$, $C'(0,3,2)$.
		Mặt phẳng $(BA'C')$ có cặp vectơ chỉ phương là
		$\overrightarrow{BA'}=(0;-3;2)$, $\overrightarrow{BC'}=(-2;0;2)$. \\
		Ta có $\left[ \overrightarrow{BA'}, \overrightarrow{BC'} \right] =(-6;-4;-6)$, suy ra $(BA'C')$ có vectơ pháp tuyến là 
		$$\overrightarrow{n} = -\dfrac{1}{2} \left[ \overrightarrow{BA'}, \overrightarrow{BC'} \right] = (3;2;3).$$\\
		Phương trình của $(BA'C')$ là
		$$3(x-2)+2(y-3)+3z=0 \text{ hay } 3x+2y+3z-12=0.$$
		Khoảng cách từ đỉnh $B'$ đến mặt phẳng $(BA'C')$ là
		$$\mathrm{d} (B', (BA'C')) = \dfrac{\left| 3 \cdot 2 + 2 \cdot 3 + 3 \cdot 2 - 12 \right|}{\sqrt{3^2+2^2+3^2}}=\dfrac{6}{\sqrt{22}}=\dfrac{3 \sqrt{22}}{11}.$$
	} 
\end{vd}
\dongcham{10}
\boxmini{BÀI TẬP TRẮC NGHIỆM}

\begin{ex}
	Khoảng cách từ $ A(-2;1;-6) $ đến mặt phẳng $ (Oxy) $ là 
	\choice
	{\True $ 6 $}
	{$ 2 $}
	{$ 1 $}
	{$ \dfrac{7}{\sqrt{41}} $}
	\loigiai{
		Ta có $ (Oxy) \colon z=0 $. Ta được $ d(A,(Oxy)) = \dfrac{|-6|}{1} = 6 $.
	}	
\end{ex}
\cham{3}
\begin{ex}
	Cho hai điểm $A(-2;1;3)$, $B(4;1;-1)$. Khoảng cách từ trung điểm $I$ của đoạn $AB$ đến mặt phẳng $(Oyz)$ là
	\choice
	{$0$}
	{$2$}
	{$4$}
	{\True $1$}
	\loigiai{
		Ta có trung điểm của đoạn $AB$ là $I(1;1;1)$ nên $\mathrm{d}(I,(Oyz))=|x_I|=1$.
	}
\end{ex}


\begin{ex}%[2H3Y2-6]%
	Cho mặt phẳng $(P)\colon 2x+3y+4z-5=0$ và điểm $A(1;-3;1)$. Khoảng cách từ điểm $A$ đến mặt phẳng $(P)$ bằng
	\choice
	{\True $\dfrac{8}{\sqrt{29}}$}
	{$\dfrac{8}{9}$}
	{$\dfrac{3}{\sqrt{29}}$}
	{$\dfrac{8}{29}$}
	\loigiai{
		Ta có
		\[\mathrm{d}(A, (P))=\dfrac{|2\cdot 1+3\cdot (-3)+4\cdot 1-5|}{\sqrt{2^2+3^2+4^2}}=\dfrac{8}{\sqrt{29}}.\]
	}
\end{ex}

\begin{ex}
	Gọi $H$ là hình chiếu vuông góc của điểm $A(2;3;-1)$ trên mặt phẳng $(\alpha)\colon 16x+12y-15z+7=0$. Tính độ dài đoạn thẳng $AH$.
	\choice
	{$\dfrac{19}{25}$}
	{\True $\dfrac{12}{25}$}
	{$\dfrac{19}{625}$}
	{$\dfrac{12}{625}$}
	\loigiai{
		Độ dài đoạn thẳng $AH$ bằng $\mathrm{d}\left(A;(\alpha)\right)=\dfrac{|16\cdot 2+12\cdot (-3)-15\cdot 1+7|}{\sqrt{16^2+12^2+(-15)^2}}=\dfrac{12}{25}$.
	}
\end{ex}

\begin{ex}%[2H3B2-6]
	Cho hai mặt phẳng $(P)\colon x+2y-2z+3=0$ và $(Q)\colon x+2y-2z-1=0$. Khoảng cách giữa hai mặt phẳng $(P)$ và $(Q)$ là 
	\choice 
	{$\dfrac{4}{9}$}
	{$\dfrac{2}{3}$}
	{\True $\dfrac{4}{3}$}
	{$-\dfrac{4}{3}$}
	
	\loigiai{
		Lấy $M(-3;0;0)\in (P)$. Vì $(P)\parallel (Q)$ nên khoảng cách giữa hai mặt phẳng $(P)$ và $(Q)$ bằng khoảng cách từ điểm $M$ đến mặt phẳng $(Q)$.\\
		Ta có $\mathrm{d}(M,(Q))=\dfrac{|x_M+2y_M-2z_M-1|}{\sqrt{1^2+2^2+(-2)^2}}=\dfrac{4}{3}$.
	}
\end{ex}
\cham{4}

\begin{ex}
	Biết rằng hai mặt phẳng $4x-4y+2z-7=0$ và $2x-2y+z+4=0$ chứa hai mặt của hình lập phương. Thể tích khối lập phương đó bằng
	\choice
	{$V=\dfrac{9 \sqrt{3}}{2}$}
	{$V=\dfrac{27}{8}$}
	{$V=\dfrac{81 \sqrt{3}}{8}$}
	{\True $V=\dfrac{125}{8}$}
	\loigiai{
		Khoảng cách giữa hai mặt phẳng trên bằng độ dài cạnh của hình lập phương.\\
		Gọi $(P)\colon 4x-4y+2z-7=0$ và $(Q)\colon 2x-2y+z+4=0$.\\
		Lây $M(0;0;-4) \in (Q)$ và $\mathrm{d}(M,(P))=\dfrac{5}{2}$.\\
		Vậy $V=\dfrac{125}{8}$.
	}
\end{ex}

\begin{ex}
	Cho hai điểm $A(2;2;-2)$ và $B(3;-1;0)$. Đường thẳng $AB$ cắt mặt phẳng $(P)\colon x+y-z+2=0$ tại điểm $I$. Tỉ số $\dfrac{IA}{IB}$ bằng
	\choice
	{\True $2$}
	{$4$}
	{$6$}
	{$3$}
	\loigiai{
		Ta có $\dfrac{IA}{IB}=\dfrac{d(A,(P))}{d(B,(P))}=\dfrac{8}{\sqrt{3}} : \dfrac{4}{\sqrt{3}}=2$.
	}
\end{ex}
\cham{4}

\begin{ex}%[2H3B2-6]
	Cho hai mặt phẳng $(P) \colon x+y-z+1=0$ và $(Q) \colon x-y+z-5=0.$ Có bao nhiêu điểm $M$ trên trục $Oy$ thỏa mãn $M$ cách đều hai mặt phẳng $(P)$ và $(Q)$?
	\choice
	{$0$}
	{\True $1$}
	{$2$}
	{$3$}
	\loigiai{
		Vì $M\in Oy$ nên $M(0;y;0).$\\
		Ta có $\mathrm{d}(M;(P))=\dfrac{|y+1|}{\sqrt{3}}$ và $\mathrm{d}(M;(Q))=\dfrac{|-y-5|}{\sqrt{3}}.$\\
		Theo giả thiết $\mathrm{d}(M;(P))=\mathrm{d}(M;(Q))\Leftrightarrow |y+1|=|-y-5|\Leftrightarrow \hoac{&y+1=-y-5\\&y+1=y=5}\Leftrightarrow \hoac{&y=-3\\& 0y=4 \,(\text{vô nghiệm})}$\\
		$\Rightarrow M(0;-3;0).$\\
		Vậy có $1$ điểm $M$ thỏa mãn bài.
	}
\end{ex}
\cham{6}
\begin{ex}%[2H3B2-3]
	Cho điểm $A(1;2;3)$ và mặt phẳng $(P)\colon x+y+z-2=0$. Mặt phẳng $(Q)$ song song với mặt phẳng $(P)$ và $(Q$) cách điểm $A$ một khoảng bằng $3\sqrt{3}$. Phương trình mặt phẳng $(Q)$ là
	\choice
	{$x+y+z+3=0$ và $x+y+z-3=0$}
	{$x+y+z+3=0$ và $x+y+z+15=0$}
	{\True $x+y+z+3=0$ và $x+y+z-15=0$}
	{$x+y+z+3=0$ và $x+y-z-15=0$}
	\loigiai{
		Do $(Q)\parallel (P)\Rightarrow (Q)\colon x+y+z+d=0,\quad d\neq -2$.\\
		Mà $d\left(A,(Q)\right)=3\sqrt{3}\Leftrightarrow |6+d|=9 \Leftrightarrow \hoac{&d=3\\&d=-15.}$\\
		Vậy $(Q_1)\colon x+y+z+3=0$ và $(Q_2)\colon x+y+z-15=0$.
		
	} 
\end{ex}
\cham{6}
\begin{ex}
	Cho mặt phẳng $ (P)\colon x+2y+z-4=0 $ và điểm $ D(1;0;3) $. Mặt phẳng $ (Q) $ song song với $ (P) $ và cách $ D $ một khoảng bằng $ \sqrt{6} $ có phương trình là
	\choice
	{$ \hoac{&x+2y-z-10=0\\&x+2y-z+2=0} $}
	{$ x+2y+z+2=0 $}
	{\True $ \hoac{&x+2y+z+2=0\\&x+2y+z-10=0} $}
	{$ x+2y+z-10=0 $}
	\loigiai{
		Vì $(Q)\parallel (P)$ nên $(Q)$ có phương trình dạng $(Q)\colon x+2y+z+D=0$ $(D\neq -4)$.\\
		Lại có $\mathrm{d}(D,(Q))=\sqrt{6}\Leftrightarrow \dfrac{|1+3+D|}{\sqrt{1+1+4}}=\sqrt{6}\Leftrightarrow |D+4|=6\Leftrightarrow \hoac{&D=2\\&D=-10}$.\\
		Vậy $(Q)\colon x+2y+z+2=0$ hoặc $(Q)\colon x+2y+z-10=0$.
	}
\end{ex}

\begin{ex}
	Cho hình chóp $S.ABCD$ có đáy $ABCD$ là hình chữ nhật. Biết $A(0; 0; 0), D(2; 0; 0), B(0; 4; 0), S(0; 0; 4)$. Gọi $M$ là trung điểm của $SB$. Tính khoảng cách từ $B$ đến mặt phẳng $(CDM)$.
	\choice
	{$d(B,(CDM))=\sqrt{2}$}
	{$d(B,(CDM))=2$}
	{$d(B,(CDM))=\dfrac{1}{\sqrt{2}}$}
	{\True $d(B,(CDM))=2\sqrt{2}$}
	\loigiai
	{
		Do $ABCD$ là hình chữ nhật nên $C\left(2;4;0\right)$. Và $M$ là trung điểm $SB$ nên $M\left(0;2;2\right)$.\\
		Phương trình mặt phẳng $\left(CDM\right)$ đi qua $M$ và nhận $\vec{n}=\left[\overrightarrow{MC},\overrightarrow{MD}\right]=\left(-8;0;-8\right)$ làm vectơ pháp tuyến là $x+z-8=0$.\\
		Khi đó $\mathrm{d}\left(B,\left(MCD\right)\right)=\dfrac{|4-8|}{\sqrt{1^2+0^2+1^2}}=2\sqrt{2}$.
	}
\end{ex}

\begin{ex}%[Thi thử, Chuyên Phan Bội Châu - Nghệ An, 2019-L1]%[Duong Xuan Loi, 12-EX-5-2019]%[2H3B4-1]
	Cho hình lập phương $ABCD.A'B'C'D'$ có cạnh bằng $2$. Khoảng cách giữa hai mặt phẳng $(AB'D')$ và $(BC'D)$ bằng
	\choice
	{$\dfrac{\sqrt{3}}{3}$}
	{\True $\dfrac{2\sqrt{3}}{3}$}
	{$\dfrac{\sqrt{3}}{2}$}
	{$\sqrt{3}$}
	\loigiai{
		\immini{
			Chọn hệ trục toạ độ như hình vẽ.\\
			Ta có $A(0;0;0),B(2;0;0),C(2;2;0),D(0;2;0)$.\\
			$A'(0;0;2),B'(2;0;2),C'(2;2;2),D'(0;2;2)$.\\
			Mặt phẳng $(AB'D')$ qua $A$ và có một vectơ pháp tuyến là $-\dfrac{1}{4}\left[\overrightarrow{AB'}, \overrightarrow{AD'}\right]=(1;1;-1)$ nên có phương trình $x+y-z=0.$				
		}{
			\begin{tikzpicture}[scale=0.8, font=\footnotesize, line join=round, line cap=round, >=stealth]
				\tkzDefPoints{0/0/A,-1.1/-1.1/B,2/-1.1/C}
				\coordinate (D) at ($(A)+(C)-(B)$);
				\coordinate (A') at ($(A)+(0,2.5)$);
				\coordinate (x) at ($(A)!1.5!(B)$);
				\coordinate (y) at ($(A)!1.3!(D)$);
				\coordinate (z) at ($(A)!1.4!(A')$);
				\tkzDefPointsBy[translation=from A to A'](B,C,D){B'}{C'}{D'}
				\tkzDrawPolygon(A',B',B,C,D,D')
				\tkzDrawSegments(B',C' C',D' C,C' B',D' C',B C',D)
				\tkzDrawSegments[dashed](A,B A,D A,A' B,D A,B' A,D')
				\tkzDrawPoints[fill=black](A,B,D,C,A',B',C',D')
				\tkzLabelPoints[above](D')
				\tkzLabelPoints[below](C,D)
				\tkzLabelPoints[above left](A')
				\tkzLabelPoints[left](A,B',B)
				\tkzLabelPoints[right](C')
				\draw[->] (B)--(x)node[right]{$x$};
				\draw[->] (D)--(y)node[above]{$y$};
				\draw[->] (A')--(z)node[right]{$z$};
			\end{tikzpicture}
		}\noindent
		Mặt phẳng $\left(BC'D\right)$ qua $B$ và có một vectơ pháp tuyến là $-\dfrac{1}{4}\left[\overrightarrow{BC'}, \overrightarrow{BD}\right]=(1;1;-1)$ nên có phương trình $x+y-z-2=0$.\\
		Ta có $(AB'D')\parallel (BC'D)$ nên
		$$\mathrm{d}((AB'D'),(BC'D))=\mathrm{d}(A,(BC'D))=\dfrac{|-2|}{\sqrt{1^2+1^2+t(-1)^2}}=\dfrac{2\sqrt{3}}{3}.$$
	}
\end{ex}

\begin{ex}%[Thi thử L1, Star Education HCM, 2019]%[Nguyễn Ngọc Dũng, dự án 12EX6]%[2H3K4-1]
	Cho hình hộp chữ nhật $ABCD.A’B’C’D’$ có $AB=a$, $AD=2a$, $AA'=3a$. Gọi $M$, $N$, $P$ lần lượt là trung điểm của $BC$, $C’D’$ và $DD’$. Tính khoảng cách từ $A$ đến $(MNP)$.
	\choice
	{\True $ \dfrac{15}{11}a $}
	{$\dfrac{15}{22}a$}
	{$\dfrac{9}{11}a$}
	{$\dfrac{3}{4}a$}
	\loigiai{
		\immini{
			Gán hệ trục tọa độ như hình vẽ với độ dài đơn vị trên trục là $a$. Khi đó, ta tính được tọa độ các điểm như sau
			$$A(0;0;0), M(1;1;0), N\left(2;\dfrac{1}{2}; 3\right), P\left( 2;0;\dfrac{3}{2} \right).$$
			Ta có $\overrightarrow{MN} = \left( 1;-\dfrac{1}{2};3\right)$ và $\overrightarrow{MP} = \left( 1;-1;\dfrac{3}{2}\right)$. \\
			Chọn $\left[ \overrightarrow{MN}, \overrightarrow{MP}\right] = \left( \dfrac{9}{4}; \dfrac{3}{2}; -\dfrac{1}{2}\right)$ là vtpt của $(MNP)$.\\
			Suy ra $(MNP)\colon 9x + 6y - 2z - 15 = 0$.\\
			Do đó $\mathrm{d}(A,(MNP)) = \dfrac{15}{11}$.\\
			Vậy $\mathrm{d}(A,(MNP)) = \dfrac{15a}{11}$.	
		}{
			\begin{tikzpicture}[scale=0.8, font=\footnotesize, line join=round, line cap=round, >=stealth]
				\tikzset{label style/.style={font=\footnotesize}}
				\def\h{4} \def\r{5} \def\x{2.2} \def\y{1.5}
				\coordinate[label={below}:$B$] (A) at (-3,-3);
				\coordinate[label={below,xshift=2mm}:{$A\equiv O$}] (B) at ($(A)+(\x,\y)$);
				\coordinate[label={below right}:$D$] (C) at ($(B)+(\r,0)$);
				\coordinate[label={below right}:$C$] (D) at ($(A)+(\r,0)$);
				\coordinate[label={above left}:$B'$] (A') at ($(A)+(0,\h)$);
				\coordinate[label={above left}:$A'$] (B') at ($(A')+(\x,\y)$);
				\coordinate[label={above right}:$D'$] (C') at ($(B')+(\r,0)$);
				\coordinate[label={above}:$C'$] (D') at ($(A')+(\r,0)$);
				\coordinate[label={above}:{$x$}] (x) at ($(B)!1.2!(C)$);
				\coordinate[label={below}:{$y$}] (y) at ($(B)!1.2!(A)$);
				\coordinate[label={right}:{$z$}] (z) at ($(B)!1.2!(B')$);
				
				\coordinate[label={below}:{$M$}] (M) at ($(A)!0.5!(D)$);
				\coordinate[label={above}:{$N$}] (N) at ($(C')!0.5!(D')$);
				\coordinate[label={right}:{$P$}] (P) at ($(C)!0.5!(C')$);
				
				\draw (A)--(A') (C)--(C') (D)--(D') (A)--(D)--(C) (A')--(B')--(C')--(D')--(A') (P)--(N);
				\draw[dashed] (B)--(B') (A)--(B)--(C) (P)--(M)--(N);
				\draw[->] (C)--(x);
				\draw[->] (B')--(z);
				\draw[->] (A)--(y);
				
				\tkzLabelSegment[below](B,C){$2a$}
				\tkzLabelSegment[right](D,C){$a$}
				\tkzLabelSegment[left](A,A'){$3a$}
				\tkzDrawPoints[fill=black](A,B,C,D,A',B',C',D',M,N,P)
			\end{tikzpicture}
		}
	}
\end{ex}

\begin{ex}%[1H3B5-2]
	Cho hình chóp $S.ABCD$ có đáy $ABCD$ là hình vuông cạnh $a$, $SA\perp (ABCD)$ và $SA=a\sqrt{3}$. Tính khoảng cách từ điểm $B$ đến mặt phẳng $(SCD)$.
	\choice
	{\True $\dfrac{a\sqrt{3}}{2}$}
	{$\dfrac{a\sqrt{2}}{4}$}
	{$\dfrac{a\sqrt{2}}{3}$}
	{$\dfrac{a}{2}$}
	\loigiai{
	\immini{
		Chuẩn hóa $a=1$. Với hệ trục đã chọn như hình vẽ thì $B(1;0;0)$, $S(0;0;\sqrt{3})$, $C(1;1;0)$, $D(0;1;0)$.\\
		Ta có 
		\begin{enumEX}[]{1}
			\item $\vec{CD}=(-1;0;0)$, $\vec{CS}=(-1;-1;\sqrt{3})$,
			\item $\vec{CB}=(0;-1;0)$; $[\vec{CD},\vec{CS}]=(0;\sqrt{3};1)$
		\end{enumEX}
			Khoảng cách từ điểm $B$ đến $(SCD)$ được tính theo công thức:
		$$d=\dfrac{\big|[\vec{CD},\vec{CS}]\cdot \vec{CB}\big|}{\big|[\vec{CD},\vec{CS}]\big|}=\dfrac{\sqrt{3}}{2}$$
		}{
		\begin{tikzpicture}[scale=0.6, font=\footnotesize,>=stealth]
			\path
			(0,0) coordinate (A)
			(-2,-2) coordinate (B)
			(5,0) coordinate (D)
			($(B)+(D)-(A)$)coordinate (C)
			%($(A)!0.5!(C)$)coordinate (I)
			($(A)+(0,3)$)coordinate (S)
			;
			\draw[->] (D)--(7,0) node[below]{$y$};
			\draw[->] (B)--(-3,-3) node[below]{$x$};
			\draw[->] (S)--(0,4) node[left]{$z$};
			\draw (C)--(D)--(S)--(C)--(B)--(S);
			\draw[dashed] (S)--(A)--(D) (B)--(A);
			\pic[draw,thin,angle radius=2mm] {right angle = B--A--D};
			\foreach \x/\g in {A/180,B/-90,C/-100,D/-80,S/170}\draw[fill=black] (\x) circle (.04) +(\g:.5)node{\footnotesize$\x$};
	\end{tikzpicture}}	
	}
\end{ex}

\begin{ex}
	Cho hình chóp $S. ABCD$ có đáy $ABCD$ là hình vuông cạnh $a$, $SD=\dfrac{3a}{2}$, hình chiếu vuông góc của $S$ lên mặt phẳng $(ABCD)$ là trung điểm của cạnh $AB$. Tính khoảng cách $d$ từ $A$ đến mặt phẳng $(SBD)$. 
	\choice
	{$d=\dfrac{a}{3}$}
	{$d=\dfrac{a}{6}$}
	{$d=\dfrac{3a}{2}$}
	{\True $d=\dfrac{2a}{3}$}
	\loigiai{
		
	}
\end{ex}

\subsection{BÀI TẬP TRẮC NGHIỆM TỰ LUYỆN}
\TN
	\setcounter{ex}{0}
	\Opensolutionfile{ans}[ans/B1-De2-1]

\begin{ex}%[2H5N1-4]
	Trong không gian với hệ tọa độ $Oxyz$, cho mặt phẳng $(P)\colon x-z+3=0$. Mặt phẳng nào sau đây vuông góc với mặt phẳng $(P)$?
	\choice
	{\True $(\alpha)\colon 2x-y+2z=0$}
	{$(\beta)\colon 2x-y-2z=0$}
	{$(Q)\colon -2x-y+2z=0$}
	{$(R)\colon 2x+y-2z=0$}
	\loigiai{
		Mặt phẳng $(P)$ có véc-tơ pháp tuyến là $\overrightarrow{n}_{(P)}=(1;0;-1)$.\\
		Mặt phẳng $(\alpha)$ có véc-tơ pháp tuyến là $\overrightarrow{n}_{(\alpha)}=(2;-1;2)$.\\
		Ta có $\overrightarrow{n}_{(P)}\cdot \overrightarrow{n}_{(\alpha)}=1\cdot 2+0\cdot (-1)+(-1)\cdot 2=0$. Do đó, $(P)\perp (\alpha)$.
	}
\end{ex}

%G:\My Drive\CODE12-2024\DE-ON-THEO BAI\2H5-TACH DE\Bai1-De2.tex
\begin{ex}%[2H5N1-5]
	Trong không gian với hệ tọa độ $Oxyz$, cho điểm $A(1;3;-2)$ và mặt phẳng $(P)\colon 2x+y-2z-3=0$. Khoảng cách từ điểm $A$ đến mặt phẳng $(P)$ bằng
	\choice
	{\True $2$}
	{$1$}
	{$\dfrac{2}{3}$}
	{$3$}
	\loigiai{
		Ta có $\mathrm{d}\big(A,(P)\big)=\dfrac{\left|2\cdot 1+3-2\cdot (-2)-3\right|}{\sqrt{2^2+1^2+(-2)^2}}=2$.
	}
\end{ex}

%G:\My Drive\CODE12-2024\DE-ON-THEO BAI\2H5-TACH DE\Bai1-De2.tex
\begin{ex}%[2H5N1-2]
	Trong không gian với hệ tọa độ $Oxyz$, cho mặt phẳng $(P)\colon x-y+3=0$. Véc-tơ nào sau đây \textbf{không phải} là véc-tơ pháp tuyến của mặt phẳng $(P)$?
	\choice
	{$\overrightarrow{a}=(3;-3;0)$}
	{$\overrightarrow{a}=(1;-1;0)$}
	{\True $\overrightarrow{a}=(1;-1;3)$}
	{$\overrightarrow{a}=(-1;1;0)$}
	\loigiai{
		Véc-tơ pháp tuyến của mặt phẳng $(P)$ là $\overrightarrow{n}=(1;-1;0)$.\\
		Ta có $\overrightarrow{a}=(-1;1;0)=-(1;-1;0)=-\overrightarrow{n}$. Vậy $\overrightarrow{a}=(-1;1;0)$ là một véc-tơ pháp tuyến của mặt phẳng $(P)$.\\
		Tương tự $\overrightarrow{a}=(3;-3;0)=3(1;-1;0)=3\overrightarrow{n}$. Vậy $\overrightarrow{a}=(3;-3;0)$ là một véc-tơ pháp tuyến của mặt phẳng $(P)$.\\
		Do véc-tơ $\overrightarrow{a}=(1;-1;3)$ không cùng phương với véc-tơ $\overrightarrow{n}=(1;-1;0)$. Nên $\overrightarrow{a}=(1;-1;3)$ không là véc-tơ pháp tuyến của mặt phẳng $(P)$.
	}
\end{ex}

%G:\My Drive\CODE12-2024\DE-ON-THEO BAI\2H5-TACH DE\Bai1-De2.tex
\begin{ex}%[2H5H1-3]
	Trong không gian với hệ tọa độ $Oxyz$ với điểm $M(-3;1;4)$ và gọi $A$, $B$, $C$ lần lượt là hình chiếu của $M$ lên các trục $Ox$, $Oy$, $Oz$. Phương trình nào dưới đây là phương trình mặt phẳng song song với mặt phẳng $(ABC)$?
	\choice
	{$4x-12y+3z-12=0$}
	{$4x-12y-3z+12=0$}
	{$4x+12y-3z-12=0$}
	{\True $4x-12y-3z-12=0$}
	\loigiai{
		Vì $A$, $B$, $C$ lần lượt là hình chiếu của $M(-3;1;4)$ các trục $Ox$, $Oy$, $Oz$ nên $A(-3;0;0)$, $B(0;1;0)$, $C(0;0;4)$.\\
		Phương trình mặt phẳng $(ABC)\colon \dfrac{x}{-3}+\dfrac{y}{1}+\dfrac{z}{4}=1$ $\Leftrightarrow$ $4x-12y-3z+12=0$.\\
		Vậy mặt phẳng $4x-12y-3z-12=0$ song song với mặt phẳng $(ABC)$.
	}
\end{ex}

%G:\My Drive\CODE12-2024\DE-ON-THEO BAI\2H5-TACH DE\Bai1-De2.tex
\begin{ex}%[2H5H1-2]
	Trong không gian với hệ tọa độ $Oxyz$, cho ba điểm $A(2;0;0)$, $B(0;-3;0)$, $C(0;0;1)$. Một véc-tơ pháp tuyến của mặt phẳng $(ABC)$ là
	\choice
	{\True $\overrightarrow{n}=(3;-2;6)$}
	{$\overrightarrow{n}=(2;-3;-1)$}
	{$\overrightarrow{n}=(2;3;1)$}
	{$\overrightarrow{n}=(2;-3;1)$}
	\loigiai{
		Phương trình mặt phẳng $(ABC)\colon \dfrac{x}{2}+\dfrac{y}{-3}+\dfrac{z}{1}=1$ $\Leftrightarrow$ $3x-2y+6z-6=0$.\\
		Vậy mặt phẳng $(ABC)$ có một véc-tơ pháp tuyến $\overrightarrow{n}=(3;-2;6)$.
	}
\end{ex}

%G:\My Drive\CODE12-2024\DE-ON-THEO BAI\2H5-TACH DE\Bai1-De2.tex
\begin{ex}%[2H2N2-2]
	Trong không gian với hệ tọa độ $Oxyz$, cho điểm $M(2024;0;-1)$. Mệnh đề nào dưới đây đúng?
	\choice
	{\True $M\in(Oxz)$}
	{$M\in Oy$}
	{$M\in(Oxy)$}
	{$M\in(Oyz)$}
	\loigiai{
		Do tung độ của điểm $M(2024;0;-1)$ bằng $0$ nên $M\in(Oxz)$.
	}
\end{ex}

%G:\My Drive\CODE12-2024\DE-ON-THEO BAI\2H5-TACH DE\Bai1-De2.tex
\begin{ex}%[2H5H1-3]
	Trong không gian với hệ tọa độ $Oxyz$, cho ba điểm $A(-2;0;0)$, $B(0;3;0)$ và $C(0;0;4)$. Mặt phẳng $(ABC)$ có phương trình là
	\choice
	{\True $\dfrac{x}{-2}+\dfrac{y}{3}+\dfrac{z}{4}=1$}
	{$\dfrac{x}{2}+\dfrac{y}{3}+\dfrac{z}{-4}=1$}
	{$\dfrac{x}{2}+\dfrac{y}{-3}+\dfrac{z}{4}=1$}
	{$\dfrac{x}{2}+\dfrac{y}{3}+\dfrac{z}{4}=1$}
	\loigiai{
		Vì ba điểm $A(-2;0;0)$, $B(0;3;0)$ và $C(0;0;4)$ lần lượt nằm trên các trục tọa độ $Ox$, $Oy$, $Oz$ và không trùng với gốc tọa độ $O$ nên mặt phẳng $(ABC)$ có phương trình là
		\begin{align*}
			\dfrac{x}{-2}+\dfrac{y}{3}+\dfrac{z}{4}=1.
		\end{align*}
	}
\end{ex}

%G:\My Drive\CODE12-2024\DE-ON-THEO BAI\2H5-TACH DE\Bai1-De2.tex
\begin{ex}%[2H5H1-5]
	Trong không gian với hệ tọa độ $Oxyz$, cho mặt phẳng $(\alpha)\colon 2x+2y-z+m=0$ ($m$ là tham số). Tìm giá trị của tham số $m$ dương để khoảng cách từ gốc tọa độ đến mặt phẳng $(\alpha)$ bằng $1$.
	\choice
	{$-6$}
	{$-3$}
	{\True $3$}
	{$6$}
	\loigiai{
		Ta có $\mathrm{d}(O,(\alpha))=\dfrac{\left| m\right|}{3}=1$ $\Leftrightarrow$ $\left| m\right|=3$ $\Leftrightarrow$ $m=\pm 3$.\\
		Do $m>0$ nên $m=3$.
	}
\end{ex}

%G:\My Drive\CODE12-2024\DE-ON-THEO BAI\2H5-TACH DE\Bai1-De2.tex
\begin{ex}%[2H2N2-2]
	Trong không gian với hệ tọa độ $Oxyz$, hình chiếu vuông góc của điểm $M(2;3;-4)$ trên mặt phẳng $(Oyz)$ có tọa độ là
	\choice
	{$(2;0;-4)$}
	{\True $(0;3;-4)$}
	{$(2;3;0)$}
	{$(0;3;0)$}
	\loigiai{
		Trong không gian với hệ tọa độ $Oxyz$, hình chiếu vuông góc của điểm $M_0(x_0;y_0;z_0)$ trên mặt phẳng $(Oyz)$ có tọa độ là $(0;y_0;z_0)$.\\
		Do đó, hình chiếu vuông góc của điểm $M(2;3;-4)$ trên mặt phẳng $(Oyz)$ có tọa độ là $(0;3;-4)$.
	}
\end{ex}

%G:\My Drive\CODE12-2024\DE-ON-THEO BAI\2H5-TACH DE\Bai1-De2.tex
\begin{ex}%[2H5H1-3]
	Trong không gian với hệ tọa độ $Oxyz$, cho hai điểm $A(1;2;-1)$, $B(-1;0;1)$ và mặt phẳng $(P)\colon x+2y-z+1=0$. Viết phương trình mặt phẳng $(Q)$ qua $A$, $B$ và vuông góc với $(P)$.
	\choice
	{$(Q)\colon 3x-y+z=0$}
	{\True $(Q)\colon x+z=0$}
	{$(Q)\colon 2x-y+3=0$}
	{$(Q)\colon -x+y+z=0$}
	\loigiai{
		Ta có $\overrightarrow{AB}=(-2;-2;2)$, $\overrightarrow{n}_P=(1;2;-1)$.\\
		$\left[\overrightarrow{AB},\overrightarrow{n}_P\right]=(-2;0;-2)$.\\
		Mặt phẳng $(Q)$ đi qua $A$, $B$ và vuông góc với $(P)$ nhận véc-tơ $\overrightarrow{n}_Q=\left[\overrightarrow{AB},\overrightarrow{n}_P\right]$ là véc-tơ pháp tuyến.\\
		Mặt phẳng $(Q)$ có phương trình là
		\begin{align*}
			-2(x-1)+0(x-2)-2(z+1)=0 \Leftrightarrow -2x-2z=0 \Leftrightarrow x+z=0.
		\end{align*}
	}
\end{ex}

%G:\My Drive\CODE12-2024\DE-ON-THEO BAI\2H5-TACH DE\Bai1-De2.tex
\begin{ex}%[2H5N1-2]
	Trong không gian với hệ tọa độ $Oxyz$, cho hai véc-tơ $\overrightarrow{u}=(1;2;3)$, $\overrightarrow{v}=(0;-1;1)$. Mặt phẳng $(\alpha)$ đi qua điểm $A(1;2;5)$ và song song với giá của hai véc-tơ $\overrightarrow{u}$ và $\overrightarrow{v}$. Véc-tơ nào dưới đây là một véc-tơ pháp tuyến của mặt phẳng $(\alpha)$?
	\choice
	{$\overrightarrow{n}_3=(-1;-1;-1)$}
	{\True $\overrightarrow{n}_2=(5;-1;-1)$}
	{$\overrightarrow{n}_1=(5;1;-1)$}
	{$\overrightarrow{n}_4=(-1;-1;5)$}
	\loigiai{
		Vì mặt phẳng $(\alpha)$ song song với giá của hai véc-tơ $\overrightarrow{u}$ và $\overrightarrow{v}$ nên có một véc-tơ pháp tuyến là $\overrightarrow{n}_2=\left[\overrightarrow{u},\overrightarrow{v}\right]=(5;-1;-1)$.
	}
\end{ex}

%G:\My Drive\CODE12-2024\DE-ON-THEO BAI\2H5-TACH DE\Bai1-De2.tex
\begin{ex}%[2H5H1-3]
	Trong không gian với hệ tọa độ $Oxyz$, cho ba điểm $A(1;-1;0)$, $B(-1;0;1)$, $C(2;1;-1)$. Phương trình mặt phẳng $(ABC)$ là
	\choice
	{\True $3x+y+5z-2=0$}
	{$x+3y+z+2=0$}
	{$3x-y+5z-2=0$}
	{$3x+y+5z+2=0$}
	\loigiai{
		Mặt phẳng $(ABC)$ đi qua $A(1;-1;0)$ và nhận $\overrightarrow{n}=\left[\overrightarrow{AB},\overrightarrow{AC}\right]=(-3;-1;-5)=-(3;1;5)$ làm véc-tơ pháp tuyến có phương trình: $3(x-1)+1(y+1)+5(z-0)=0$.\\
		Do đó, $(ABC)\colon 3x+y+5z-2=0$.
	}
\end{ex}
	\Closesolutionfile{ans}

\TNTF
	\setcounter{ex}{0}
	\Opensolutionfile{ans}[ans/B1-De2-2]
\begin{ex}%[2H5V1-5]
	Trong không gian với hệ trục tọa độ $Oxyz$, cho hai điểm $A(1;2;3)$, $B(3;4;4)$ và mặt phẳng $(\alpha)\colon 2x+y+mz-1=0$. Các mệnh đề sau đúng hay sai?
	\choiceTF
	{\True Mặt phẳng đi qua $3$ điểm là hình chiếu vuông góc của $A(1;2;3)$ lên ba trục tọa độ có phương trình là $6x+3y+2z-6=0$}
	{Điểm $A$ cách đều mặt phẳng $(\gamma)\colon 2x+y+mz-1=0$ và điểm $B$ khi $m=-2$}
	{\True Biết mặt phẳng $(\beta)\colon 4x+(n-2)y+z-3=0$ song song với mặt phẳng $(\alpha)$. Khi đó, $2m+n=5$}
	{Khi $B\in (\alpha)\colon 2x+y+mz-1=0$ thì $m=-2$}
	\loigiai{
		\begin{itemchoice}
			\itemch \textbf{Đúng}.\\
			Ta có hình chiếu vuông góc của $A(1;2;3)$ lên ba trục tọa độ lần lượt $M(1;0;0)$, $N(0;2;0)$, $P(0;0;3)$.\\
			Do đó, phương trình mặt phẳng cần tìm là $\dfrac{x}{1}+\dfrac{y}{2}+\dfrac{z}{3}=1$ $\Leftrightarrow$ $6x+3y+2z-6=0$.
			\itemch \textbf{Sai}.\\
			Ta có $\overrightarrow{AB}=(2;2;1)$ $\Rightarrow$ $AB=\sqrt{2^2+2^2+1^2}=3$. \hfill$(1)$\\
			Khoảng cách từ $A$ đến mặt phẳng $(P)$ là
			\begin{align*}
				\mathrm{d}\big(A,(P)\big)=\dfrac{\left|2\cdot 1+2+m\cdot 3-1\right|}{\sqrt{2^2+1^2+ m^2}}=\dfrac{\left|3m+3\right|}{\sqrt{5+m^2}}. \tag{2}
			\end{align*}
			Do đó, $AB=\mathrm{d}\big(A,(P)\big)$ khi và chỉ khi
			\begin{align*}
				3=\dfrac{\left|3m+3\right|}{\sqrt{5+ m^2}} \Leftrightarrow 9(5+m^2)=9(m+1)^2 \Leftrightarrow m=2.
			\end{align*}
			\itemch \textbf{Đúng}.\\
			Vì $(\alpha)\parallel (\beta)$ nên $\dfrac{2}{4}=\dfrac{1}{n-2}=\dfrac{m}{1}\ne\dfrac{-1}{-3}$ $\Leftrightarrow$ $\heva{
				& m=\dfrac{1}{2}\\
				& n=4
			}$ $\Rightarrow$ $2m+n=5$.
			\itemch \textbf{Sai}.\\
			Ta có $B\in (\alpha)\colon 2x+y+mz-1=0$ $\Leftrightarrow$ $2\cdot 3+4+m\cdot 4-1=0$ $\Leftrightarrow$ $m=-\dfrac{9}{4}$.
		\end{itemchoice}
	}
\end{ex}

%G:\My Drive\CODE12-2024\DE-ON-THEO BAI\2H5-TACH DE\Bai1-De2.tex
\begin{ex}%[2H5H1-5]
	Trong không gian với hệ tọa độ $Oxyz$, cho $A(1;2;-1)$, $B(-1;0;1)$ và mặt phẳng $(P)\colon x+2y-z+1=0$. Các mệnh đề sau đúng hay sai?
	\choiceTF
	{\True Biết điểm $M$ nằm trên tia $Ox$ mà khoảng cách từ $M$ đến mặt phẳng $(P)$ bằng $\sqrt{6}$. Khi đó, hoành độ điểm $M$ là $x_M=5$}
	{\True Mặt phẳng $(Q)$ qua $A$, $B$ và vuông góc với $(P)$ có phương trình là $x+z=0$}
	{\True Mặt phẳng $(P)$ có một véc-tơ pháp tuyến là $(1;2;-1)$}
	{Khi $m=-4$ thì mặt phẳng $(R)\colon 2x-my+3=0$ vuông góc với mặt phẳng $(P)$}
	\loigiai{
		\begin{itemchoice}
			\itemch \textbf{Đúng}.\\
			$M$ nằm trên tia $Ox$ $\Rightarrow$ $M(x;0;0)$, $x>0$.\\
			Khi đó, khoảng cách từ $M$ đến mặt phẳng $(P)$ bằng $\sqrt{6}$ khi và chỉ khi
			\begin{align*}
				\dfrac{\left| x+1\right|}{\sqrt{6}}=\sqrt{6} \Leftrightarrow x=5, (x>0).
			\end{align*}
			\itemch \textbf{Đúng}.\\
			Ta có $\overrightarrow{AB}=(-2;-2;2)$, $\overrightarrow{u}=-\dfrac{1}{2}\overrightarrow{AB}=(1;1;-1)$; $\overrightarrow{n}_P=(1;2;-1)$.\\
			Do đó, $\overrightarrow{n}_Q=\left[\overrightarrow{u},\overrightarrow{n}_{(P)}\right]=(1;0;1)$ là một véc-tơ pháp tuyến của mặt phẳng $(Q)$.\\
			Vậy $(Q)\colon 1\cdot(x-1)+0\cdot(y-2)+1\cdot(z+1)=0 \Leftrightarrow x+z=0$.
			\itemch \textbf{Đúng}.\\
			Mặt phẳng $(P)\colon x+2y-z+1=0$ có một véc-tơ pháp tuyến là $(1;2;-1)$.
			\itemch \textbf{Sai}.\\
			$(P)$ có một véc-tơ pháp tuyến là $\overrightarrow{n}_P=(1;2;-1)$; $(R)$ có một véc-tơ pháp tuyến là $\overrightarrow{n}_R=(2;-m;0)$.\\
			Hai mặt phẳng $(P)$ và $(R)$ vuông góc với nhau khi và chỉ khi
			\begin{align*}
				\overrightarrow{n}_P\cdot \overrightarrow{n}_R=0 \Leftrightarrow 1\cdot 2+2\cdot (-m)+(-1)\cdot 0=0 \Leftrightarrow m=1.
			\end{align*}
		\end{itemchoice}
	}
\end{ex}

%G:\My Drive\CODE12-2024\DE-ON-THEO BAI\2H5-TACH DE\Bai1-De2.tex
\begin{ex}%[2H5V1-3]
	Trong không gian với hệ tọa độ $Oxyz$, cho hai điểm $A(1;2;1)$ và $B(3;-1;5)$. Các mệnh đề sau đúng hay sai?
	\choiceTF
	{\True Phương trình mặt phẳng trung trực của đoạn thẳng $AB$ là $2x-3y+4z-\dfrac{29}{2}=0$}
	{Điểm $N(1;2;-1)$ đối xứng với $A(1;2;1)$ qua mặt phẳng $(Oyz)$}
	{\True Mặt phẳng $(P)$ vuông góc với đường thẳng $AB$ và cắt các trục $Ox$, $Oy$ và $Oz$ lần lượt tại các điểm $D$, $E$ và $F$. Khi thể tích của tứ diện $ODEF$ bằng $\dfrac{3}{2}$, phương trình mặt phẳng $(P)$ là $2x-3y+4z\pm 6=0$}
	{Véc-tơ $\overrightarrow{AB}$ là một véc-tơ pháp tuyến của mặt phẳng $(\alpha)\colon 2x+3y+4z-2=0$}
	\loigiai{
		\begin{itemchoice}
			\itemch \textbf{Đúng}.\\
			Trung điểm của đoạn thẳng $AB$ là điểm $I\left(2;\dfrac{1}{2};3\right)$.\\
			Vậy mặt phẳng trung trực của $AB$ đi qua $I\left(2;\dfrac{1}{2};3\right)$ và nhận $\overrightarrow{AB}=(2;-3;4)$ làm véc-tơ pháp tuyến nên có phương trình là
			\begin{align*}
				2(x-2)-3\left(y-\dfrac{1}{2}\right)+4(z-3)=0 \Leftrightarrow 2x-3y+4z-\dfrac{29}{2}=0.
			\end{align*}
			\itemch \textbf{Sai}.\\
			Điểm đối xứng với $A(1;2;1)$ qua mặt phẳng $(Oyz)$ có tọa độ là $(-1;2;1)$.
			\itemch \textbf{Đúng}.\\
			Vì $AB\perp (P)$ nên mặt phẳng $(P)$ có một véc-tơ pháp tuyến là $\overrightarrow{AB}=(2;-3;4)$. Do đó, phương trình mặt phẳng $(P)$ có dạng $2x-3y+4z+d=0$.\\
			Từ đây tìm được $D\left(-\dfrac{d}{2};0;0\right)$, $E\left(0;\dfrac{d}{3};0\right)$, $F\left(0;0;-\dfrac{d}{4}\right)$ suy ra $OD=\dfrac{\left|d\right|}{2}$, $OE=\dfrac{\left|d\right|}{3}$, $OF=\dfrac{\left|d\right|}{4}$.\\
			Mặt khác, tứ diện $ODEF$ có $OD$, $OE$, $OF$ đôi một vuông góc nên thể tích của tứ diện $ODEF$ là $V_{ODEF}=\dfrac{1}{6}OD. OE. OF=\dfrac{(\left|d\right|)^3}{144}$.\\
			Do đó, $V_{ODEF}=\dfrac{3}{2}$ $\Leftrightarrow$ $\dfrac{(\left|d\right|)^3}{144}=\dfrac{3}{2}$ $\Leftrightarrow$ $\left|d\right|=6$ $\Leftrightarrow$ $d=\pm 6$.\\
			Vậy phương trình mặt phẳng $(P)$ là $2x-3y+4z\pm 6=0$.
			\itemch \textbf{Sai}.\\
			Véc-tơ $\overrightarrow{AB}=(2;-3;4)$ không cùng phương với $\overrightarrow{n}=(2;3;4)$ là véc-tơ pháp tuyến của mặt phẳng $(\alpha)$ nên $\overrightarrow{AB}$ không là một véc-tơ pháp tuyến của mặt phẳng $(\alpha)$.
		\end{itemchoice}
	}
\end{ex}

%G:\My Drive\CODE12-2024\DE-ON-THEO BAI\2H5-TACH DE\Bai1-De2.tex
\begin{ex}%[2H5V1-4]
	Trong không gian với hệ tọa độ $Oxyz$, cho hai mặt phẳng $(\alpha)\colon 3x-2y+2z+7=0$ và $(\beta)\colon 5x-4y+3z+1=0$. Các mệnh đề sau đúng hay sai?
	\choiceTF
	{Hai mặt phẳng $(\alpha)$, $(\beta)$ song song với nhau}
	{Điểm $A(1;2;-1)$ nằm trên mặt phẳng $(\alpha)\colon 3x-2y+2z+7=0$}
	{\True Phương trình mặt phẳng qua $O$, đồng thời vuông góc với cả $(\alpha)$ và $(\beta)$ có phương trình là $2x+y-2z=0$}
	{\True Mặt phẳng $(\gamma)$ đi qua điểm $I(1;0;-1)$ và song song với $(\alpha)\colon 3x-2y+2z+7=0$ có phương trình là $(\gamma)\colon 3x-2y+2z-1=0$}
	\loigiai{
		\begin{itemchoice}
			\itemch \textbf{Sai}.\\
			Ta có $\dfrac{3}{5}\ne\dfrac{-2}{-4}$ nên hai mp $(\alpha)$, $(\beta)$ cắt nhau. Vậy chúng không song song.
			\itemch \textbf{Sai}.\\
			Ta có $3\cdot 1-2\cdot 2+2\cdot (-1)+7\ne 0$ nên $A(1;2;-1)$ không nằm trên mặt phẳng $(\alpha)$.
			\itemch \textbf{Đúng}.\\
			Mặt phẳng $(\alpha)$ có một véc-tơ pháp tuyến là $\overrightarrow{n}_1=(3;-2;2)$.\\
			Mặt phẳng $(\beta)$ có một véc-tơ pháp tuyến là $\overrightarrow{n}_2=(5;-4;3)$.\\
			Do mặt phẳng $(Q)$ vuông góc với cả $(\alpha)$ và $(\beta)$ nên $\overrightarrow{n}=\left[\overrightarrow{n}_1,\overrightarrow{n}_2\right]=(2;1;-2)$ là một véc-tơ pháp tuyến của mặt phẳng $(Q)$.\\
			Mặt phẳng $(Q)$ đi qua $O(0;0;0)$ và có véc-tơ pháp tuyến $\overrightarrow{n}=(2;1;-2)$ có phương trình là $2x+y-2z=0$.
			\itemch \textbf{Đúng}.\\
			Mặt phẳng $(\gamma)$ song song với mặt phẳng $(\alpha)\colon 3x-2y+2z+7=0$ nên phương trình của mặt phẳng $(\gamma)$ có dạng $3x-2y+2z+d=0$, với $d\ne7$.\\
			Vì $I(1;0;-1)\in(\gamma)$ nên $3\cdot 1-2\cdot 0+2\cdot(-1)+d=0$ $\Leftrightarrow$ $d=-1$ (thỏa mãn điều kiện $d\ne7$).\\
			Vậy $(\gamma)\colon 3x-2y+2z-1=0$.
		\end{itemchoice}
	}
\end{ex}
	\Closesolutionfile{ans}

\TNSA
	\setcounter{ex}{0}
	\Opensolutionfile{ans}[ans/B1-De2-3]
\begin{ex}%[2H5V1-7]
	Từ mặt nước trong một bể nước, tại ba vị trí đôi một cách nhau $2$ m, người ta lần lượt thả dây dọi để quả dọi chạm đáy bể. Phần dây dọi (thẳng) nằm trong nước tại ba vị trí đó lần lượt có độ dài $4$ m; $4{,}4$ m; $4{,}8$ m. Biết đáy bể là phẳng. Hỏi đáy bể nghiêng so với mặt phẳng nằm ngang một góc bao nhiêu độ (làm tròn đến hàng phần chục)?\\
	\shortans[oly]{$21{,}8$}
	\loigiai{
		Gọi ba vị trí trên mặt nước là $A$, $B$, $C$ thì tam giác $ABC$ là tam giác đều cạnh bằng $2$ m. Gọi dây dọi lần lượt là $AA'$, $BB'$, $CC'$ có độ dài lần lượt là $4$ m; $4{,}4$ m; $4{,}8$ m.\\
		Chọn hệ trục toạ độ $Oxyz$ sao cho $O$ là trung điểm của $BC$, tia $Ox$ chứa điểm $A$, tia $Oy$ chứa điểm $B$, tia $Oz$ đi qua trung điểm của $B'C'$ và đơn vị trên các trục là mét.\\
		Ta có $OB=OC=1$, $OA=\sqrt{3}$ $\Rightarrow$ $A'\left(\sqrt{3};0;4\right)$, $B'(0;1;4{,}4)$, $C'(0;-1;4{,}8)$.\\
		Khi đó, $\overrightarrow{A'B'}=\left(-\sqrt{3};1;0{,}4\right)$, $\overrightarrow{A'C'}=\left(-\sqrt{3};-1;0{,}8\right)$.\\
		Mặt phẳng $(A'B'C')$ có một véc-tơ pháp tuyến là $\overrightarrow{n}=\left[\overrightarrow{A'B'},\overrightarrow{A'C'}\right]=0{,}4\sqrt{3}\left(\sqrt{3};1;5\right)$.\\
		Mặt phẳng $(ABC)$ có một véc-tơ pháp tuyến là $\overrightarrow{k}=(0;0;1)$.\\
		Do đó, $\cos\big((ABC),(A'B'C')\big)=\left|\cos\left(\overrightarrow{n},\overrightarrow{k}\right)\right|=\dfrac{5}{\sqrt{29}}$. Góc cần tìm gần bằng $21{,}8^\circ$.
	}
\end{ex}

\begin{ex}%[2H5V1-3]
	Trong không gian với hệ tọa độ $Oxyz$, cho mặt cầu $(S)\colon (x-1)^2+(y+1)^2+z^2=11$ và hai véc-tơ $\overrightarrow{u}_1=(1;1;2)$, $\overrightarrow{u}_2=(1;2;1)$. Gọi $(P)$ là mặt phẳng tiếp xúc với mặt cầu $(S)$ đồng thời song song với giá của hai véc-tơ $\overrightarrow{u}_1$, $\overrightarrow{u}_2$. Phương trình mặt phẳng $(P)$ có dạng $3x+by+cz+d=0$, với $b,\,c,\,d\in\mathbb{Z}$ và $d\ne -15$. Khi đó, $b+c+d$ bằng bao nhiêu?\\
	\shortans[oly]{$5$}
	\loigiai{
		Mặt cầu $(S)$ có tâm $I(1;-1;0)$, bán kính $R=\sqrt{11}$.\\
		Mặt phẳng $(P)$ song song với của hai véc-tơ $\overrightarrow{u}_1$, $\overrightarrow{u}_2$ nên $(P)$ có véc-tơ pháp tuyến là $\overrightarrow{n}=\left[\overrightarrow{u}_1,\overrightarrow{u}_2\right]=(-3;1;1)$.\\
		Phương trình mặt phẳng $(P)$ có dạng $-3x+y+z+d=0$$\Leftrightarrow$ $3x-y-z-d=0$, $d\ne 15$.\\
		Mặt khác, mặt phẳng $(P)$ tiếp xúc với mặt cầu $(S)$ nên ta có
		\begin{align*}
			\mathrm{d}\big(I,(P)\big)=R \Leftrightarrow \dfrac{\left|3+1-0-d\right|}{\sqrt{9+1+1}}=\sqrt{11} \Leftrightarrow \left|-d+4\right|=11 \Leftrightarrow \hoac{&d=15& (\text{loại})\\ &d=-7.&}
		\end{align*}
		Với $d=-7$, ta có phương trình mặt phẳng $(P)$ là $-3x+y+z-7=0$ $\Leftrightarrow$ $3x-y-z+7=0$.\\
		Vậy $b+c+d=-1-1+7=5$.
	}
\end{ex}

\begin{ex}%[2H5C1-3]
	Trong không gian với hệ tọa độ $Oxyz$, có hai mặt phẳng $(P)$ và $(Q)$ cùng thỏa mãn các điều kiện sau: đi qua hai điểm $A(1;1;1)$ và $B(0;-2;2)$, đồng thời cắt các trục tọa độ $Ox$, $Oy$ tại hai điểm cách đều $O$. Giả sử $(P)$ có phương trình $x+b_1y+c_1z+d_1=0$ và $(Q)$ có phương trình $x+b_2y+c_2z+d_2=0$. Tính giá trị biểu thức $b_1b_2+c_1c_2$.\\
	\shortans[oly]{$-9$}
	\loigiai{
		Xét mặt phẳng $(\alpha)$ có phương trình $x+by+cz+d=0$ thỏa mãn các điều kiện: đi qua hai điểm $A(1;1;1)$ và $B(0;-2;2)$, đồng thời cắt các trục tọa độ $Ox,Oy$ tại hai điểm cách đều $O$.\\
		Vì $(\alpha)$ đi qua $A(1;1;1)$ và $B(0;-2;2)$ nên ta có hệ phương trình
		\begin{align*}
			\heva{& 1+b+c+d=0\\ & -2b+2c+d=0.} \tag{*}
		\end{align*}
		Mặt phẳng $(\alpha)$ cắt các trục tọa độ $Ox$, $Oy$ lần lượt tại $M(-d;0;0)$, $N\left(0;\dfrac{-d}{b};0\right)$.\\
		Vì $M$, $N$ cách đều $O$ nên $OM=ON$. Suy ra $\left|d\right|=\left|\dfrac{d}{b}\right|$.\\
		Nếu $d=0$ thì chỉ tồn tại duy nhất một mặt phẳng thỏa mãn yêu cầu bài toán (mặt phẳng này sẽ đi qua điểm $O$).\\
		Do đó, để tồn tại hai mặt phẳng thỏa mãn yêu cầu bài toán thì $\left|d\right|=\left|\dfrac{d}{b}\right|$ $\Leftrightarrow$ $b=\pm 1$.
		\begin{itemize}
			\item Với $b=1$, $(*)$ $\Leftrightarrow$ $\heva{
				& c+d=-2\\
				& 2c+d=2
			}$ $\Leftrightarrow$ $\heva{
				& c=4\\
				& d=-6
			}$. Ta được $(P)\colon x+y+4z-6=0$.
			\item Với $b=-1$, $(*)$ $\Leftrightarrow$ $\heva{
				& c+d=0\\
				& 2c+d=-2
			}$ $\Leftrightarrow$ $\heva{
				& c=-2\\
				& d=2
			}$. Ta được $(Q)\colon x-y-2z+2=0$.
		\end{itemize}
		Vậy $b_1 b_2+ c_1 c_2=1\cdot (-1)+4\cdot (-2)=-9$.
	}
\end{ex}

\begin{ex}%[2H5V1-3]
	Trong không gian với hệ tọa độ $Oxyz$, cho mặt cầu $(S)\colon (x+1)^2+(y-2)^2+(z-3)^2=8$ và điểm $A(1;3;2)$.
	Mặt phẳng $(P)$ đi qua $A$ và cắt $(S)$ theo giao tuyến là đường tròn có bán kính nhỏ nhất. Biết phương trình của $(P)$ có dạng $ax+by+cz+6=0$. Tính $a+b+c$.\\
	\shortans[oly]{$-4$}
	\loigiai{
		Mặt cầu $(S)$ có tâm $I(-1;2;3)$, bán kính $R=2\sqrt{2}$.\\
		Ta có $\overrightarrow{IA}=(2;1;-1)$; $AI=\sqrt{6}<R$, suy ra điểm $A$ nằm trong mặt cầu $(S)$.\\
		Gọi $H$ là hình chiếu vuông góc của $I$ trên mặt phẳng $(P)$. Khi đó mặt phẳng $(P)$ đi qua $A$ và cắt $(S)$ theo giao tuyến là đường tròn có bán kính $r=\sqrt{R^2-IH^2}$. Do đó, $r$ nhỏ nhất khi và chỉ khi $IH$ lớn nhất.\\
		Mặt khác, ta luôn có $IH\le IA$, dấu bằng xảy ra khi và chỉ khi $H$ trùng với $A$, hay $(P)\perp IA$.\\
		Mặt phẳng $(P)$ có véc-tơ pháp tuyến $\overrightarrow{IA}=(2;1;-1)$ và qua $A(1;3;2)$ có phương trình $2(x-1)+(y-3)-1(z-2)=0$ $\Leftrightarrow$ $2x+y-z-3=0$ $\Leftrightarrow$ $-4x-2y+2z+6=0$.\\
		Vậy $a+b+c=-4$.
	}
\end{ex}

\begin{ex}%[2H5V1-3]
	Trong không gian với hệ tọa độ $Oxyz$, cho hai điểm $A(2;-3;1)$, $B(-1;1;0)$ và mặt phẳng $(P)\colon x-y+z-2=0$. Một mặt phẳng $(Q)$ đi qua hai điểm $A$, $B$ và vuông góc với $(P)$ có dạng là $ax+by+cz+2=0$. Tính $a^2+ b^2+ c^2$.\\
	\shortans[oly]{$56$}
	\loigiai{
		$\overrightarrow{AB}=(-3;4;-1)$, $(P)$ có một véc-tơ pháp tuyến là $\overrightarrow{n}_{(P)}=(1;-1;1)$.\\
		$\left[\overrightarrow{AB},\overrightarrow{n}_{(P)}\right]=(3;2;-1)$.\\
		$(Q)$ đi qua $B(-1;1;0)$ và có một véc-tơ pháp tuyến $\overrightarrow{n}_{(Q)}=(3;2;-1)$ nên có phương trình
		\begin{align*}
			3(x+1)+2(y-1)-z=0 \Leftrightarrow 3x+2y-z+1=0 \Leftrightarrow 6x+4y-2z+2=0.
		\end{align*}
		Suy ra $a=6$, $b=4$, $c=-2$ hay $a^2+b^2+c^2=56$.
	}
\end{ex}

\begin{ex}%[2H5H1-3]
	Trong không gian với hệ trục $Oxyz$, cho ba điểm $A(1;2;1)$, $B(2;-1;0)$, $C(1;1;3)$. Phương trình mặt phẳng đi qua ba điểm $A$, $B$, $C$ có dạng $ax+by+cz-12=0$. Khi đó, $a-b-2c$ bằng\\
	\shortans[oly]{$3$}
	\loigiai{
		Ta có $\overrightarrow{AB}=(1;-3;-1)$, $\overrightarrow{AC}=(0;-1;2)$ suy ra $\left[\overrightarrow{AB},\overrightarrow{AC}\right]=(-7;-2;-1)=-1(7;2;1)$.\\
		Mặt phẳng $(ABC)$ đi qua điểm $A(1;2;1)$ có véc-tơ pháp tuyến $\overrightarrow{n}=(7;2;1)$ có phương trình là $7x+2y+z-12=0$. Khi đó, $a-b-2c=3$.
	}
\end{ex}

\centerline{---HẾT---}
\Closesolutionfile{ans}
%\newpage
%%=====================
%\begin{center}
%\textbf{\large BẢNG ĐÁP ÁN}
%\end{center}
%\noindent\textbf{ĐÁP ÁN PHẦN I}
%\inputansbox{10}{ans/B1-De2-1}
	
%\noindent\textbf{ĐÁP ÁN PHẦN II}
%\inputansbox[2]{2}{ans/B1-De2-2}
	
%\noindent\textbf{ĐÁP ÁN PHẦN III}
%\inputansbox[3]{6}{ans/B1-De2-3}



%%Bài 2.
\setcounter{dang}{0}
\newpage
\section{PHƯƠNG TRÌNH ĐƯỜNG THẲNG}
\subsection{LÝ THUYẾT CẦN NHỚ}
\subsubsection{Vectơ chỉ phương của đường thẳng}
\begin{itemize}
	\immini{\item [\iconMT] \indam{Định nghĩa:} Vectơ chỉ phương  $\vec{u}$ của đường thẳng $d$ là những vectơ khác $\vec{0}$ và có giá song song hoặc trùng với $d$. 
		\item [\iconMT] \indam{Chú ý:} 
		\begin{boxdn}
			\begin{itemize}
				\item [$\bullet$] $\vec{u} \ne \vec{0}$ và có giá song song hoặc trùng với $d$. 
				\item [$\bullet$] Nếu $\vec{u}$ và $\vec{u'}$ cùng là vectơ chỉ phương của $d$ thì $\vec{u'} = k \cdot \vec{u}$ (\textit{tọa độ tỉ lệ nhau}).
			\end{itemize}
		\end{boxdn}
	}{
		\begin{tikzpicture}[scale=0.8, line join=round, line cap=round,>=stealth]
			\draw[thick] (0,0)--(4,0)node[below right]{$d$};
			\draw[->,blue] (1,1)--(3,1)node[below]{\scriptsize$\vec{u}$};
			\draw[->,violet] (2.3,1.5)--(3.7,1.5)node[above]{\scriptsize$\vec{u'}$};
	\end{tikzpicture}}
\end{itemize}
\subsubsection{Phương trình tham số của đường thẳng}
\begin{itemize}
	\item [\iconMT] \indam{Công thức:} Đường thẳng $d$ đi qua điểm $M(x_0;y_0;z_0)$ và nhận $\vec{u}=(u_1;u_2;u_3)$ làm vectơ chỉ phương có phương trình là 
	\boxmini{$\heva{&x=x_0+u_1t\\&y=y_0+u_2t\\&z=z_0+u_3t} \quad \left( t \in \mathbb{R}\right) \quad (1) $}
	\item [\iconMT] \indam{Chú ý:}
	\begin{boxdn}
		\begin{itemize}
			\item [\ding{172}] Phương trình các trục tọa độ: 
			\begin{listEX}[3]
				\item [$\bullet$] $Ox \colon \heva{&x=t\\&y=0\\&z=0}$ .
				\item [$\bullet$] $Oy \colon \heva{&x=0\\&y=t\\&z=0}$ .
				\item [$\bullet$] $Oz \colon \heva{&x=0\\&y=0\\&z=t}$ .
			\end{listEX}
			\item [\ding{173}] Nếu $u_1$, $u_2$ và $u_3$ đều khác $0$ thì $(1)$ có thể được viết dưới dạng
			\boxmini{$\dfrac{x-x_0}{u_1}=\dfrac{y-y_0}{u_2}=\dfrac{z-z_0}{u_3} \quad (2)$}
			$(2)$ được gọi là phương trình chính tắc của đường thẳng $d$.
		\end{itemize}
	\end{boxdn}
\end{itemize}
\subsubsection{Vị trị tương đối giữa hai đường thẳng}
Cho hai đường thẳng 
\begin{itemize}
	\item [$\bullet$] $\Delta_1$ qua điểm $M(x_0;y_0;z_0)$, vectơ chỉ phương $\vec{u}=(u_1;u_2;u_3)$;
	\item [$\bullet$] $\Delta_2$ qua điểm $N(x_0';y_0';z_0')$, vectơ chỉ phương $\vec{v}=(v_1;v_2;v_3)$.
\end{itemize}
	\begin{listEX}[1]
		\item [] \indamm{Trường hợp 1:} Nếu $\bigg[\vec{u},\vec{v}\bigg] = \vec{0}$ và 
		\begin{itemize}
			\item [$\bullet$] $\bigg[\vec{u},\vec{MN}\bigg]\ne \vec{0}$  thì $\Delta_1$ song song $\Delta_2$; 
			\item [$\bullet$] $\bigg[\vec{u},\vec{MN}\bigg]  =\vec{0}$  thì $\Delta_1$ trùng $\Delta_2$.
		\end{itemize}
		\item [] \indamm{Trường hợp 2:} Nếu $\bigg[\vec{u},\vec{v}\bigg] \ne \vec{0}$ và 
		\begin{itemize}
			\item [$\bullet$] $\bigg[\vec{u},\vec{v}\bigg] \cdot \vec{MN} \ne 0$  thì $\Delta_1$ chéo $\Delta_2$; 
			\item [$\bullet$] $\bigg[\vec{u},\vec{v}\bigg] \cdot \vec{MN} =0$  thì $\Delta_1$ cắt $\Delta_2$.
		\end{itemize}
	\end{listEX}
\subsection{PHÂN LOẠI, PHƯƠNG PHÁP GIẢI TOÁN}
\begin{dang}{Xác định điểm thuộc và vectơ chỉ phương của đường thẳng}
	Cho đường thẳng $d$.
	\begin{listEX}[1]
		\item [\ding{172}] Nếu $\vec{u} \ne \vec{0}$ và có giá song song hoặc trùng với $d$ thì $\vec{u}$ là vectơ chỉ phương của $d$.
		\item [\ding{173}] Nếu $d$ qua hai điểm $AB$ thì $d$ có một vectơ chỉ phương là $\vec{AB}=\left(x_B-x_A; y_B-y_A;z_B-z_A \right)$. 
		\item [\ding{174}] Nếu $d$ vuông góc với giá của hai vectơ $\vec{a}$, $\vec{b}$ không cùng phương thì $d$ có một vectơ chỉ phương là $\vec{u}=[\vec{a},\vec{b}]$.
		\item [\ding{175}] Cho đường thẳng  $d \colon \heva{&x=x_0+u_1t\\&y=y_0+u_2t\\&z=z_0+u_3t} \quad \left( t \in \mathbb{R}\right)$ thì
		\begin{itemize}
			\item [$\bullet$] Một vectơ chỉ phương của $d$ là $\vec{u}=(u_1;u_2;u_3)$ (hệ số của $t$).
			\item [$\bullet$] Muốn xác định tọa độ một điểm thuộc $d$, ta chỉ cần cho trước giá trị cụ thể của tham số $t$, thay vào hệ phương trình tính $x$, $y$ và $z$.
		\end{itemize}
	\end{listEX}
\end{dang}
\boxmini{BÀI TẬP TỰ LUẬN}
\setcounter{vd}{0}

\begin{vd}
	Cho đường thẳng $d:\heva{x=1-t\\y=2+3t\\z=2+t}\quad (t\in\mathbb{R})$. Tìm một vectơ chỉ phương và hai điểm thuộc đường thẳng $d$.
	\loigiai{
	}
\end{vd}
\dongcham{2}

\begin{vd}
	Trong không gian $Oxyz$, cho hình chóp $O.ABC$ có $A\left( 2;0;0 \right),B\left( 0;4;0 \right)$ và $C\left( 0;0;7 \right)$.
	\begin{enumerate}
		\item Tìm tọa độ một vectơ chỉ phương của đường thẳng $AB$, $AC$.
		\item Vectơ $\overrightarrow{v}=\left( -1;2;0\right)$ có là vectơ chỉ phương của đường thẳng $AB$ không?
	\end{enumerate}
	\loigiai{
		\immini{\begin{enumerate}
				\item Ta có $\overrightarrow{AB}=\left( -2;4;0 \right)$ là một vectơ chỉ phương của đường thẳng $AB$; $\overrightarrow{AC}=\left( -2;0;7 \right)$ là một vectơ chỉ phương của đường thẳng $AC$.
				\item Vì $\overrightarrow{v}=\left( -1;2;0 \right)=\dfrac{1}{2}\overrightarrow{AB}$ nên $\overrightarrow{v}$ là một vectơ chỉ phương của đường thẳng $AB$.
		\end{enumerate}}{\begin{tikzpicture}[xscale=0.8,yscale=0.8]
				\path
				(0,0) coordinate (A)
				(2,-1.5) coordinate (B)
				(5,0) coordinate (C)
				(2.5,3) coordinate (O)
				;
				\draw[dashed] (A)--(C)
				;
				\draw (A)--(B) (A)--(O) (C)--(B) (B)--(O) (C)--(O)
				;
				\foreach \x/\g in {O/90, A/180, B/-90, C/0} \path (\x) circle (.05) +(\g:.3) node {$\x$}
				;
				\draw (2.5,-2.5) node{\textit{Hình 2}}
				;
				\foreach \x/\g in {O/90, A/90, B/90, C/90} \draw[fill=black] (\x) circle (.05) +(\g:.3) node {}
				;
		\end{tikzpicture}}
	}
\end{vd}
\dongcham{5}

\begin{vd}%[2H3H3-1][Mức độ 2]
	Trong không gian $Oxyz$, cho hai mặt phẳng $(P)\colon 2x+y-z-1=0$ và $(Q)\colon x-2y+z-5=0$. Gọi $\Delta$ là giao tuyến của $(P)$ và $(Q)$. Tìm một điểm thuộc $\Delta$ và một vectơ chỉ phương của $\Delta$.
	\loigiai{
		Mặt phẳng $(P)$ và $(Q)$ có VTPT lần lượt là $\overrightarrow{n}_P=(2;1;-1)$ và $\overrightarrow{n}_Q=(1;-2;1)$.\\
		Vậy vectơ chỉ phương của đường thẳng $d$ là giao tuyến của $(P)$ và $(Q)$ là
		$\overrightarrow{u}=\left[\overrightarrow{n}_P,\overrightarrow{n}_Q\right]=(1;3;5)$.}
\end{vd}
\dongcham{5}
\boxmini{BÀI TẬP TRẮC NGHIỆM}
\setcounter{ex}{0}

\begin{ex}
	Cho đường thẳng $d\colon \heva{&x = 1 + 2t\\&y = - t\\&z = 4 + 5t}$. Đường thẳng $d$ có một vectơ chỉ phương là
	\choice
	{\True $\overrightarrow{u_2} = \left(2;-1;5\right)$}
	{$\overrightarrow{u_4} = \left(1;-1;4\right)$}
	{$\overrightarrow{u_3} = \left(1;-1;5\right)$}
	{$\overrightarrow{u_1} = \left(1;0;4\right)$}
	\loigiai{
		Đường thẳng $d$ có một vectơ chỉ phương là $\overrightarrow{u} = \left(2;-1;5\right)$.}
\end{ex}

\begin{ex}
	Cho đường thẳng $d\colon \dfrac{x-2}{-1}=\dfrac{y-1}{2}=\dfrac{z}{1}$. Đường thẳng $d$ có một vectơ chỉ phương là
	\choice
	{\True $\vec{u}=(-1;2;1)$}
	{ $\vec{u}=(2;1;0)$}
	{ $\vec{u}=(-1;2;0)$}
	{ $\vec{u}=(2;1;1)$}
	\loigiai{
		\textbf{Cần nhớ}: Đường thẳng $d\colon \dfrac{x-x_0}{a} =\dfrac{y-y_0}{b}= \dfrac{z-z_0}{c}$ có một VTCP là $\vec{u}=(a;b;c)$ và đi qua điểm $M(x_0;y_0;z_0)$.\\
		Đường thẳng $d\colon \dfrac{x-2}{-1}=\dfrac{y-1}{2}=\dfrac{z}{1}$ có một vectơ chỉ phương là $\vec{u}=(-1;2;1)$.
	}
\end{ex}

\begin{ex}
	Cho đường thẳng $d: \dfrac{x-1}{2}=\dfrac{y+1}{3}=\dfrac{z}{2}$. Điểm nào trong các điểm dưới đây nằm trên đường thẳng $d$?
	\choice
	{$P(5;2;5)$}
	{$Q(1;0;0)$}
	{\True $M(3;2;2)$}
	{$N(1;-1;2)$}
	\loigiai{
	}
\end{ex}

\begin{ex}
	Cho đường thẳng $d: \begin{cases}
		x=1+2t \\
		y=2+3t \\
		z=5-t
	\end{cases}(t\in \mathbb{R})$. Đường thẳng $d$ \textbf{không} đi qua điểm nào sau đây?
	\choice{$M(1;2;5)$}
	{\True $N(2;3;-1)$}
	{$P(3;5;4)$}
	{$Q(-1;-1;6)$} 
	\loigiai{
	}
\end{ex}

\begin{ex}%[Phát triển đề minh họa, 2021]%[Đoàn Minh Tân]%[2H3Y3-1]%
	Cho hai điểm $A(2;-1;4)$ và $B(-1;3;2)$. Đường thẳng $AB$ có một vectơ chỉ phương là
	\choice
	{$\overrightarrow{u}_1=(1;2;2)$}
	{$\overrightarrow{u}_3=(1;2;6)$}
	{\True $\overrightarrow{u}_2=(3;-4;2)$}
	{$\overrightarrow{u}_4=(1;-4;2)$}
	\loigiai{
		Đường thẳng $AB$ nhận $\overrightarrow{AB}=(-3;4;-2)$ làm một vectơ chỉ phương.\\
		Do đó $\overrightarrow{u}_2=(3;-4;2)=-\overrightarrow{AB}$ cũng là một vectơ chỉ phương của $AB$.
	}
\end{ex}

\begin{ex}%[Paul Hieu Nguyen]%[2H3B3-1]%Câu 35.10
	Cho tam giác $ABC$ với $A(1;0;-2)$, $B(2;-3;-4)$, $C(3;0;-3)$. Gọi $G$ là trọng tâm tam giác $ABC$. vectơ nào sau đây là một vectơ chỉ phương của đường thẳng $OG$?
	\choice
	{\True $(-2;1;3)$}
	{$(3;-2;1)$}
	{$(2;1;3)$}
	{$(-1;-3;2)$}
	\loigiai{
		$G$ là trọng tâm tam giác $ABC \Rightarrow G(2;-1;-3)\Rightarrow \overrightarrow{OG}=(2;-1;-3)$.\\
		Đường thẳng $OG$ nhận $\vec{u}=-\overrightarrow{OG}=(-2;1;3)$ làm một vectơ chỉ phương.
	}
\end{ex}

\begin{ex}
	Cho đường thẳng $d$ song song với trục $Oy$. Đường thẳng $d$ có một vectơ chỉ phương là
	\choice
	{$\overrightarrow{u}_4=(2019;0;2019)$}
	{$\overrightarrow{u}_1=(2019;0;0)$}
	{\True $\overrightarrow{u}_2=(0;2019;0)$}
	{$\overrightarrow{u}_3=(0;0;2019)$}
	\loigiai{
		Trục $Oy$ có vectơ chỉ phương $\overrightarrow{j}=(0;1;0)$, mà $d\parallel Oy$ nên $d$ có một vectơ chỉ phương là \[\overrightarrow{u}_2=2019\overrightarrow{j}=(0;2019;0)\]
	}
\end{ex}

\begin{ex}%[2-TT-41-Vted-lan6-2019]%[Trần Nhân Kiệt, dự án tex đề W-T-B]%[2H3Y3-1]%
	Cho đường thẳng $\Delta$ vuông góc với mặt phẳng $(\alpha) \colon x+2z+3=0$. Một vectơ chỉ phương của $\Delta$ là
	\choice
	{$\overrightarrow{v}=(1;2;3)$}
	{\True $\overrightarrow{a}=(1;0;2)$}
	{$\overrightarrow{u}=(2;0;-1)$}
	{$\overrightarrow{b}=(2;-1;0)$}
	\loigiai{
		Ta có $\Delta$ vuông góc với $(\alpha) \Rightarrow \overrightarrow{a}=(1;0;2)$ là một vectơ chỉ phương của $\Delta$.}
\end{ex}

\begin{ex}%[2H3V3-1][Mức độ 3]
	vectơ chỉ phương của đường thẳng vuông góc với mặt phẳng đi qua ba điểm $A(1;2;4)$, $B(-2;3;5)$, $C(-9;7;6)$ có toạ độ là
	\choice
	{$(3;4;-5)$}
	{$(3;-4;5)$}
	{$(-3;4;-5)$}
	{\True $(3;4;5)$}
	\loigiai{
		Gọi $(P)$ là mặt phẳng đi qua ba điểm $A$, $B$, $C$. \\
		Gọi $\overrightarrow{a}$ là vectơ chỉ phương của đường thẳng $d$ là đường thẳng vuông góc với $(P)$.\\
		Ta có $\overrightarrow{AB}=(-3;1;1),\overrightarrow{AC}=(-10;5;2)$.\\
		Vì $d$ vuông góc với $(P)$ nên $d$ có vectơ chỉ phương là
		$\overrightarrow{a}=\left[\overrightarrow{AB},\overrightarrow{AC}\right]=(-3;-4;-5)=-1(3;4;5)$.}
\end{ex}

\begin{ex}%[2H3B3-1]%
	Cho hai mặt phẳng $(P): 3x-2y+2z-5=0$, $(Q): 4x+5y-z+1=0$. Các điểm $A, B$ phân biệt thuộc giao tuyến của hai mặt phẳng $(P)$ và $(Q)$. Khi đó $\overrightarrow{AB}$ cùng phương với vectơ nào sau đây?
	\choice
	{\True $\overrightarrow{u}=(8;-11;-23)$}
	{$\overrightarrow{k}=(4;5;-1)$}
	{$\overrightarrow{w}=(3;-2;2)$}
	{$\overrightarrow{v}=(-8;11;-23)$}
	\loigiai{
		Ta có $\overrightarrow{n}=(3;-2;2)$ và $\overrightarrow{n'}=(4;5;-1)$ lần lượt là các vectơ pháp tuyến của các mặt phẳng $(P), (Q)$. Do đó $\left[\overrightarrow{n}, \overrightarrow{n'}\right]=(-8;11;23)$ là một vectơ chỉ phương của giao tuyến của $(P)$ và $(Q)$.\\
		Từ đó suy ra $\overrightarrow{AB}$ cùng phương với vectơ $\overrightarrow{u}=(8;-11;-23)$.
	}
\end{ex}

\begin{dang}{Viết phương trình đường thẳng $d$ khi biết vài yếu tố liên quan}
	\begin{itemize}
		\item [\iconCH] \indamm{Phương pháp chung:} Ta cần xác định vectơ chỉ phương $\vec{u}$  và một điểm $M$ thuộc đường thẳng.
		\item [\iconCH] \indamm{Một số kiểu xác định vectơ $\vec{u}$ thường gặp:} 
		\begin{listEX}[1]
			\item [\ding{172}] $d$ qua hai điểm $A$, $B$ thì $\vec{u}=\vec{AB}=(x_B-x_A;y_B-y_A;z_B-z_A)$.
			\item [\ding{173}] $d$ song song với $\Delta$ thì $\vec{u}=\vec{u_{\Delta}}$.
			\item [\ding{174}] $d$ vuông góc với $(P)$ thì $\vec{u}=\vec{n}_{P}$.
			\item [\ding{175}]  $d$ vuông góc với giá của hai vectơ $\vec{a}$ và $\vec{b}$ (không cùng phương) thì $\vec{u}=[\vec{a},\vec{b}]$.
		\end{listEX}
	\end{itemize}
\end{dang}
\boxmini{BÀI TẬP TỰ LUẬN}
\setcounter{vd}{0}

\begin{vd}%[2H5H2-3]%[Dự án tex hóa sách bài tập Toán 12 CTST]%[Lê Thị Thúy Hằng]
	Lập phương trình chính tắc của đường thẳng $d$ trong mỗi trường hợp sau
	\begin{enumerate}
		\item $d$ đi qua điểm $A(4;-2;5)$ và có vectơ chỉ phương $\overrightarrow{a}=(7;3;-9)$.
		\item $d$ đi qua hai điểm $M(0;0;1)$, $N(3;3;6)$.
		\item $d$ có phương trình tham số là $\heva{&x=8+5t\\&y=7+4t\\&z=11+9t.}$
	\end{enumerate}
	\loigiai{
		\begin{enumerate}
			\item Đường thẳng $d$ đi qua điểm $A(4;-2;5)$ và có vectơ chỉ phương $\overrightarrow{a}=(7;3;-9)$ nên $d$ có phương trình chính tắc là
			$\dfrac{x-4}{7}=\dfrac{y+2}{3}=\dfrac{z-5}{-9}$.
			\item Đường thẳng $d$ đi qua hai điểm $M(0;0;1)$, $N(3;3;6)$ nên $d$ có vectơ chỉ phương là $\overrightarrow{MN}=(3;3;5)$.\\
			Suy ra phương trình chính tắc của đường thẳng $d$ là $\dfrac{x}{3}=\dfrac{y}{3}=\dfrac{z-1}{5}$.
			\item Đường thẳng $d$ có phương trình tham số là $\heva{&x=8+5t\\&y=7+4t\\&z=11+9t}$, suy ra $d$ có phương trình chính tắc là $\dfrac{x-8}{5}=\dfrac{y-7}{4}=\dfrac{z-11}{9}$.
		\end{enumerate}
	}
\end{vd}
\dongcham{8}
\begin{vd}
\immini{Trong một khu du lịch, người ta cho du khách trải nghiệm thiên nhiên bằng cách đu theo đường trượt zipline từ vị trí $A$ cao 15 m của tháp 1 này sang vị trí $B$ cao 10 m của tháp 2 trong khung cảnh tuyệt đẹp xung quanh. Với hệ trucuc toạ độ $O x y z$ cho trưóc (đơn vị: mét), toạ độ của $A$ và $B$ lần lượt là $(3 ; 2,5 ; 15)$ và $(21 ; 27,5 ; 10)$.
\begin{enumEX}[a)]{1}
	\item Viết phương trình đường thẳng chứa đường trượt zipline này.
	\item Xác định toạ độ của du khách khi ở độ cao 12 mét.
\end{enumEX}}{
\includegraphics[scale=0.6]{images/2P5-B2-NguyenHuuTinh-H5-23}
}
\end{vd}
\dongcham{5}
\begin{vd}
	Trong không gian $Oxyz$, Lập phương trình tham số và phương trình chính tắc (nếu có) của đường thẳng $d$ trong các trường hợp sau:
	\begin{enumEX}[a)]{1}
		\item $d$ đi qua điểm $M$ và song song với đường thẳng $\Delta \colon \dfrac{x-1}{2}=\dfrac{y+1}{1}=\dfrac{z}{-1}$
		\item $d$ qua điểm $M(3;2;-1)$ và vuông góc với mặt phẳng $(P)\colon x+z-2=0$.
		\item $d$ đi qua điểm $M(1; 2; 1)$, đồng thời vuông góc với cả hai đường thẳng $\Delta_{1}\colon \dfrac{x-2}{1}=\dfrac{y+1}{-1}=\dfrac{z-1}{1}$ và $\Delta_{2}\colon \dfrac{x+1}{1}=\dfrac{y-3}{2}=\dfrac{z-1}{-1}$.
	\end{enumEX}

	\loigiai
	{
		\begin{enumerate}[a)]
			\item Đường thẳng $\Delta$ có vectơ chỉ phương là $\vec{u}=(2;1;-1)$.\\
			Đường thẳng $d$ qua $M(2;1;0)$ và song song với đường thẳng $\Delta$ cũng nhận $\vec{u}=(2;1;-1)$ làm vectơ chỉ phương của nó.\\
			Vậy phương trình đường thẳng cần tìm là
			\[\dfrac{x-2}{2}=\dfrac{y-1}{1}=\dfrac{z}{-1} \Leftrightarrow\dfrac{x-2}{4}=\dfrac{y-1}{2}=\dfrac{z}{-2}.\]
			\item Mặt phẳng $(P)\colon x+z-2=0$ có vectơ pháp tuyến là $\overrightarrow{n}_{(P)}=(1;0;1)$.\\
			Đường thẳng $\Delta$ đi qua $M$ và vuông góc với $(P)$ nhận $\overrightarrow{n}_{(P)}$ làm vectơ chỉ phương có phương trình là
			$$\heva{& x=3+t \\ & y=2\\& z=-1+t.}$$
			\item Đường thẳng $\Delta_{1}$ và $\Delta_{2}$ có vectơ chỉ phương là $\vec{u}_1 =(1;-1;1)$ và $\vec{u}_2=(1;2;-1)$.\\
			Vì $d$ vuông góc với cả hai đường thẳng $\Delta_{1}$ và $\Delta_{2}$ nên $d$ có vectơ chỉ phương là $\vec{u}=\left[\vec{u}_{1}, \vec{u}_{2}\right]=(-1; 2; 3)$.\\
			Vậy phương trình đường thẳng $d$ là $\heva{&x=1-t \\&y=2+2 t\\&z=1+3 t}.$
		\end{enumerate}
		
	}
\end{vd}
\dongcham{18}
\begin{vd}%[2H3V3-2]
	Trong không gian $Oxyz$, cho điểm $A(1;-2;0)$, mặt phẳng $(P)\colon 2x-3y+z+5=0$ và đường thẳng $d\colon \dfrac{x-1}{2}=\dfrac{y}{-1}=\dfrac{z+1}{1}$. Viết phương trình đường thẳng $\Delta$ đi qua $A$, cắt $d$ và song song với mặt phẳng $(P)$.
	\loigiai
	{
		Mặt phẳng $(P)$ có VTPT $\overrightarrow{n}=(2 ;-3 ; 1)$.\\
		Gọi $M$ là giao điểm của $\Delta$ và $d\colon \heva{&x=1+2t\\&y=-t\\&z=-1+t}$ là $M(1+2t ;-t ;-1+t)$.\\
		Đường thẳng $\Delta$ nhận $\overrightarrow{AM}=(2 t ;-t+2 ; t-1)$ làm VTCP.\\
		Đường thẳng $\Delta$ song song với mặt phẳng $(P)$ nên
		$$\overrightarrow{AM} \cdot \overrightarrow{n}=0 \Leftrightarrow 2t \cdot 2+(-t+2)\cdot(-3)+(t-1) \cdot 1=0 \Leftrightarrow t=\dfrac{7}{8}.$$
		Suy ra $\overrightarrow{AM}=\left(\dfrac{7}{4} ; \dfrac{9}{8} ;-\dfrac{1}{8}\right)=\dfrac{1}{8}(14 ; 9 ;-1)$.\\
		Đường thẳng $\Delta$ qua $A$ và nhận $\overrightarrow{AM}=(14 ; 9 ;-1)$ làm VTCP nên phương trình $\Delta\colon \heva{&x=1+14t\\&y=-2+9t\\&z=-t.}$	
	}
\end{vd}
\dongcham{10}

\begin{vd}%[2H3K3-2]%Câu 27%[Đình Phúc ]%
	Trong Không gian với hệ tọa độ $Oxyz$, cho hai đường thẳng $d_1 \colon \dfrac{x-2}{1}=\dfrac{y-1}{-1}=\dfrac{z-2}{-1}$ và $d_2 \colon \heva{&x=t\\&y=3\\&z=-2+t}$. Viết phương trình đường vuông góc chung của hai đường thẳng $d_1, d_2$.
	\loigiai{
		Gọi $d$ là đường thẳng cần tìm.
		Gọi $A=d \cap d_1,B=d \cap d_2$\\
		$\begin{aligned}
			& A \in d_1 \Rightarrow A(2+a;1-a;2-a)\\
			& B \in d_2 \Rightarrow B(b;3;-2+b)\\
			& \overrightarrow{AB}=(-a+b-2;a+2;a+b-4)
		\end{aligned}$\\
		$d_1$ có vectơ chỉ phương $\overrightarrow{a}_1=(1;-1;-1)$,
		$d_2$ có vectơ chỉ phương $\overrightarrow{a}_2=(1;0;1)$\\
		$\heva{&d \perp d_1\\&d \perp d_2} \Leftrightarrow \heva{&\overrightarrow{AB} \perp \overrightarrow{a}_1\\&\overrightarrow{AB} \perp \overrightarrow{a}_2} \Leftrightarrow \heva{&\overrightarrow{AB} \cdot \overrightarrow{a}_1=0\\&\overrightarrow{AB} \cdot \overrightarrow{a}_2=0} \Leftrightarrow \heva{&a=0\\&b=3} \Rightarrow A(2;1;2);B(3;3;1)$\\
		$d$ đi qua điểm $A(2;1;2)$ và có vectơ chỉ phương $\overrightarrow{a}_d=\overrightarrow{AB}=(1;2;-1)$\\
		Vậy phương trình của $d$ là $\heva{&x=2+t\\&y=1+2t\\&z=2-t}$}
\end{vd}
\dongcham{10}
\boxmini{BÀI TẬP TRẮC NGHIỆM}
\setcounter{ex}{0}

\begin{ex}
	Cho đường thẳng $\Delta$ đi qua điểm $M\left(2;0;-1\right)$ và có vectơ chỉ phương $\overrightarrow{a}=\left(4;-6;2\right)$. Phương trình tham số của đường thẳng $\Delta$ là
	\choice
	{$\heva{&x=-2+2t\\& y=-3t\\& z=1+t}$}
	{$\heva{&x=2+2t\\ &y=-3t\\& z=-1+t}$}
	{$\heva{&x=-2+4t\\& y=-6t\\ &z=1+2t}$}
	{\True $\heva{&x=4+2t\\ &y=-3t\\ &z=2+t}$}
	\loigiai{
	}
\end{ex}
\cham{3}

\begin{ex}
	Cho hai điểm $A(2;-1;3), B(3;2;-1)$. Phương trình nào sau đây là phương trình đường thẳng $AB$?
	\choice
	{$\heva{&x=1+2t\\&y=3-t\\&z=-4+3t}$}
	{\True $\heva{&x=2+t\\&y=-1+3t\\&z=3-4t}$}
	{$\heva{&x=2+t\\&y=-1+t\\&z=3-4t}$}
	{$\heva{&x=1+2t\\&y=1-t\\&z=-4+3t}$}
	\loigiai{
	}
\end{ex}
\cham{3}

\begin{ex}
	Cho đường thẳng $\Delta: \dfrac{2x-1}{2}=\dfrac{y}{1}=\dfrac{z+1}{-1}$, điểm $A(2;-3;4)$. Đường thẳng qua $A$ và song song với $\Delta$ có phương trình là
	\choice
	{ $\heva{&x=2+t  \\&	y=-3+t  \\&	z=4-t}$}
	{\True $\heva{&x=2-2t  \\&y=-3-t  \\&	z=4+t}$}
	{$\heva{&x=2+2t  \\&y=-3+t  \\&	z=4+t}$}
	{ $\heva{&x=2+2t  \\&y=1-3t  \\&z=-1+4t}$}
	\loigiai{
	}
\end{ex}
\cham{3}

\begin{ex}
	Viết phương trình đường thẳng đi qua điểm $N(2;-3;-5)$ và vuông góc với mặt phẳng $(P): 2x-3y-z+2=0$.
	\choice
	{\True $\dfrac{x-2}{2}=\dfrac{y+3}{-3}=\dfrac{z+5}{-1}$}
	{$\dfrac{x+2}{2}=\dfrac{y-3}{-3}=\dfrac{z-5}{-1}$}
	{$\dfrac{x+2}{2}=\dfrac{y-3}{-3}=\dfrac{z-1}{-5}$}
	{$\dfrac{x-2}{2}=\dfrac{y+3}{-3}=\dfrac{z+1}{-5}$}
	\loigiai{
	}
\end{ex}
\cham{3}

\begin{ex}
	Cho tam giác $ABC$ có $A(3;2;-4), B(4;1;1)$ và $C(2;6;-3).$ Viết phương trình đường thẳng $d$ đi qua trọng tâm $G$ của tam giác $ABC$ và vuông góc với mặt phẳng $(ABC)$.
	\choice
	{$d:\dfrac{x-3}{3}=\dfrac{y-3}{2}=\dfrac{z+2}{-1}$}
	{$d:\dfrac{x+12}{3}=\dfrac{y+7}{2}=\dfrac{z-3}{-1}$}
	{\True $d:\dfrac{x-3}{7}=\dfrac{y-3}{2}=\dfrac{z+2}{-1}$}
	{$d:\dfrac{x+7}{3}=\dfrac{y+3}{2}=\dfrac{z-2}{-1}$}
	\loigiai{
	}
\end{ex}
\cham{3}

\begin{ex}%[2H3V3-2]
	Cho hai điểm $A(1;-1;1)$ và $B(-1;2;3)$ và đường thẳng $\Delta\colon\dfrac{x+1}{-2}=\dfrac{y-2}{1}=\dfrac{z-3}{3}$. Phương trình đường thẳng đi qua điểm $A$, đồng thời vuông góc với hai đường thẳng $AB$ và $\Delta$ là
	\choice
	{$\dfrac{x+1}{7}=\dfrac{y-1}{-2}=\dfrac{z+1}{4}$}
	{$\dfrac{x+1}{7}=\dfrac{y-1}{2}=\dfrac{z+1}{4}$}
	{$\dfrac{x-7}{1}=\dfrac{y-2}{-1}=\dfrac{z-4}{1}$}
	{\True $\dfrac{x-1}{7}=\dfrac{y+1}{2}=\dfrac{z-1}{4}$}
	\loigiai{
		Ta có: $\overrightarrow{AB}=(-2;3;2)$.\\
		Đường thẳng $\Delta$ có vectơ chỉ phương là $\overrightarrow{u}_{\Delta}=(-2;1;3)\Rightarrow\left[\overrightarrow{AB},\overrightarrow{u}_{\Delta}\right]=(7;2;4)$.\\
		Đường thẳng đi qua điểm $A$, đồng thời vuông góc với hai đường thẳng $AB$ và $\Delta$ có vectơ chỉ phương $\left[\overrightarrow{AB},\overrightarrow{u}_{\Delta}\right]$ có phương trình là $\dfrac{x-1}{7}=\dfrac{y+1}{2}=\dfrac{z-1}{4}$.}
\end{ex}
\cham{4}

\begin{ex}
	Cho điểm $A(1;2;3)$ và đường thẳng $d: \dfrac{x+1}{2}=\dfrac{y}{1}=\dfrac{z-3}{-2}$. Gọi $\Delta$ là đường thẳng đi qua điểm $A$, vuông góc với đường thẳng $d$ và cắt trục hoành. Tìm một vectơ chỉ phương $\overrightarrow{u}$ của đường thẳng $\Delta$.
	\choice
	{$\overrightarrow{u}=(0;2;1)$}
	{$\overrightarrow{u}=(1;0;1)$}
	{$\overrightarrow{u}=(1;-2;0)$}
	{\True $\overrightarrow{u}=(2;2;3)$}
	\loigiai{
	}
\end{ex}
\cham{5}

\begin{ex}
	Cho hai đường thẳng $d_1: \dfrac{x-1}{1}=\dfrac{y+1}{2}=\dfrac{z}{-1}$ và $d_2: \dfrac{x-2}{1}=\dfrac{y}{2}=\dfrac{z+3}{2}$. Viết phương trình đường thẳng $\Delta$ đi qua điểm $A(1;0;2)$, cắt $d_1$ và vuông góc với $d_2$.
	\choice
	{$\dfrac{x-1}{-2}=\dfrac{y}{3}=\dfrac{z-2}{4}$}
	{\True $\dfrac{x-3}{2}=\dfrac{y-3}{3}=\dfrac{z+2}{-4}$}
	{$\dfrac{x-5}{-2}=\dfrac{y-6}{-3}=\dfrac{z-2}{4}$}
	{$\dfrac{x-1}{-2}=\dfrac{y}{3}=\dfrac{z-2}{-4}$}
	\loigiai{
		Gọi $B(1+t;-1+2t;-t) \in d_1$. Do đó $\overrightarrow{AB}=(t;2t-1;-t-2)$.\\
		Xét $\overrightarrow{AB}.\overrightarrow{u}_{d_2}=0 \Longleftrightarrow 1.t +2(2t-1)+2(-t-2)=0\Longleftrightarrow t=2$.\\
		Ta có $\overrightarrow{AB}=(2;3;-4)$ chính là vec-tơ chỉ phương của đường thẳng $\Delta$.\\
		Vậy phương trình đường thẳng $\Delta$ là $\heva{x=1+2t\\y=3t\\z=2-4t}$.\\
		Với $t=1$ thì đường thẳng $\Delta$ đi qua điểm $C(3;3;-2)$.\\
		Vậy phương trình đường thẳng $\Delta$ là $\dfrac{x-3}{2}=\dfrac{y-3}{3}=\dfrac{z+2}{-4}$.
	}
\end{ex}
\cham{5}
\begin{ex}%[Paul Hieu Nguyen]%[2H3K3-2]%Câu 35.16
	Cho đường thẳng $\Delta$ đi qua $M(1;2;2)$, song song với mặt phẳng
	$(P)\colon x-y+z+3=0$ đồng thời cắt đường thẳng $d\colon \dfrac{x-1}{1}=\dfrac{y-2}{1}=\dfrac{z-3}{1}$ có phương trình là
	\def\dotEX{}
	\choice
	{\True $\heva{&x=1-t\\&y=2-t\\&z=2.}$}
	{$\heva{&x=1\\&y=2-t\\&z=2-t.}$}
	{$\heva{&x=1-t\\&y=2+t\\&z=2.}$}
	{$\heva{&x=-1+t\\&y=-1+2t\\&z=2t.}$}
	\loigiai{
		Đường thẳng $d$ có phương trình tham số là $\heva{&x=1+t\\&y=2+t\\&z=3+t.}$\\
		Mặt phẳng $(P)$ có một VTPT là $\vec{n}_P=(1;-1;1)$.\\
		Giả sử $\Delta$ cắt $d$ tại $A$
		$\Rightarrow A(1+t;2+t;3+t)$ và $\overrightarrow{MA}=(t;t;1+t)$.\\
		Vì $\Delta$ song song $(P)$ nên $\overrightarrow{MA}\cdot\vec{n}=0\Leftrightarrow t=-1\Rightarrow \overrightarrow{MA}=(-1;-1;0)$.\\
		Chọn một VTCP của $\Delta$ là $\vec{u}_{\Delta}=\overrightarrow{MA}=(-1;-1;0)$.\\
		Đường thẳng $\Delta$ có phương trình tham số là $\heva{&x=1-t\\&y=2-t\\&z=2.}$
	}
\end{ex}
\begin{ex}%%[THPT Việt Đức_Hà Nội_HK2] %[2H3K3]
	Cho đường thẳng $d: x=y=z$. Viết phương trình đường thẳng $d'$ là hình chiếu vuông góc của $d$ lên mặt phẳng tọa độ $(Oyz)$. 
	\choice
	{$\heva{&x=0\\&	y=t \\&	z=2t}$}
	{$\heva{&x=t\\&y=t \\&z=2t}$}
	{$\heva{&x=0\\&y=2+t \\&z=1+t}$}
	{\True $\heva{&x=0\\&y=t \\&z=t}$}
	\loigiai{
	}
\end{ex}
\cham{5}

\begin{ex}
	Cho đường thẳng $d:\dfrac{x+1}{2}=\dfrac{y}{1}=\dfrac{z-2}{1},$ mặt phẳng $(P):x+y-2z+5=0$ và điểm $A(1;-1;2).$ Viết phương trình đường thẳng $\Delta$ cắt $d$ và $(P)$ lần lượt tại $M$ và $N$ sao cho $A$ là trung điểm của đoạn thẳng $MN$.
	\choice
	{\True $\Delta:\dfrac{x-3}{2}=\dfrac{y-2}{3}=\dfrac{z-4}{2}$}
	{$\Delta:\dfrac{x-1}{6}=\dfrac{y+1}{1}=\dfrac{z-2}{2}$}
	{$\Delta:\dfrac{x+5}{6}=\dfrac{y+2}{1}=\dfrac{z}{2}$}
	{$\Delta:\dfrac{x+1}{2}=\dfrac{y+4}{3}=\dfrac{z-3}{2}$}
	\loigiai{
		Ta loại ngay được $1$ phương án vì điểm A không thuộc đường thẳng này.\\
		Đường thẳng $d$ có phương trình tham số $d:\left\{\begin{aligned}x&=-1+2t\\y&=t\\z&=2+t\end{aligned}\right.$\\
		$M\in d\Rightarrow M(-1+2t;t;2+t), A$ là trung điểm $MN\Rightarrow N(3-2t;-2-t;2-t)$\\
		$N\in (P)\Rightarrow 3-2t-2-t-2(2-t)+5=0\Leftrightarrow t=2\Rightarrow N(-1;-4;0)\Rightarrow \overrightarrow{NA}=(2;3;2).\Rightarrow $ Loại được hai phương án không thỏa mãn điều kiện này.\\
		Còn duy nhất $1$ phương án cần chọn.
	}
\end{ex}
\cham{5}
\begin{ex}%[Đề thi HK2, môn Toán THPT Yên Lạc, Vĩnh phúc]%[Nguyễn Quang Dũng, dự án 12-EX-6-2020]%[2H3K3-2]%
	Trong không gian $Oxyz$, đường vuông góc chung của hai đường thẳng chéo nhau $d_1\colon \dfrac{x-2}{2}=\dfrac{y-3}{3}=\dfrac{z+4}{-5}$ và $d_2\colon \dfrac{x+1}{3}=\dfrac{y-4}{-2}=\dfrac{z-4}{-1}$ có phương trình là
	\choice
	{$\dfrac{x-2}{2}=\dfrac{y+2}{2}$}
	{$\dfrac{x-2}{2}=\dfrac{y-2}{3}=\dfrac{z-3}{4}$}
	{\True $\dfrac{x}{1}=\dfrac{y}{1}=\dfrac{z-1}{1}$}
	{$\dfrac{x}{2}=\dfrac{y-2}{3}=\dfrac{z-3}{-1}$}
	\loigiai{
		Giả sử $M\in d_1$, $N\in d_2$ và $MN$ là đoạn vuông góc chung.\\
		Ta có $M(2+2t;3+3t;-4-5t)\,\,N(-1+3s;4-2s;4-s)$, $\overrightarrow{MN}=(3s-2t-3;-2s-3t+1;-s+5t+8)$.\\
		vectơ chỉ phương của $d_1$, $d_2$ lần lượt là $\overrightarrow{u}_1=(2;3;-5)$, $\overrightarrow{u}_2=(3;-2;-1)$. \\
		Ta có $MN$ là đoạn vuông góc chung của $d_1$ và $d_2$ khi và chỉ khi
		\[\heva{&\overrightarrow{MN}\cdot\overrightarrow{u}_1=0\\&\overrightarrow{MN}\cdot\overrightarrow{u}_2=0}\Leftrightarrow\heva{&5s-38t-43=0\\&14s-5t-19=0}\Leftrightarrow\heva{&t=-1\\&s=1.}\]
		Suy ra $M(0;0;1)$, $\overrightarrow{MN}=(2;2;2)$. Ta có phương trình của đường vuông góc chung $MN$ là
		\[\dfrac{x}{1}=\dfrac{y}{1}=\dfrac{z-1}{1}.\]
	}
\end{ex}
\begin{dang}{Vị trí tương đối của hai đường thẳng}
	Cho $d$ qua điểm $M$ và có vectơ chỉ phương $\vec{u}$; $d'$ qua điểm $N$ và có vectơ chỉ phương $\vec{v}$.
	\begin{listEX}[1]
		\item [\ding{172}] Nếu $\vec{u}$ cùng phương $\vec{v}$ ($\vec{u}=k\vec{v}$) và $M \notin d'$ thì $d \parallel d'$.
		\item [\ding{173}] Nếu $\vec{u}$ cùng phương $\vec{v}$ ($\vec{u}=k\vec{v}$) và $M \in d'$ thì $d$ trùng với $d'$.
		\item [\ding{174}] Nếu $\left[\vec{u},\vec{v}\right] \cdot \vec{MN} \ne 0$ thì $d$ và $d'$ chéo nhau.
		\item [\ding{175}] Nếu $\left[\vec{u},\vec{v}\right] \cdot \vec{MN} = 0$ thì $d$ và $d'$ cắt nhau.
		\item [\ding{176}] Nếu $\vec{u} \cdot \vec{v} =0$ thì $d$ và $d'$ vuông góc nhau.
	\end{listEX}
\end{dang}
\boxmini{BÀI TẬP TỰ LUẬN}
\begin{vd}%[2H5V2-4]
	\immini{
		Trên phần mềm mô phỏng 3D một máy khoan trong không gian $Oxyz$, cho biết phương trình trục $a$ của mũi khoan và một đường rãnh $b$ trên vật cần khoan (tham khảo hình vẽ bên) lần lượt là $$a\colon\heva{& x=1\\&y=2\\&z=3t }\text{ và }b\colon\heva{& x=1+4t'\\&y=2+2t'\\&z=6. }$$
		\begin{enumerate}
			\item Chứng minh $a$, $b$ vuông góc và cắt nhau.
			\item Tìm giao điểm của $a$ và $b$.
		\end{enumerate}
	}{
		\includegraphics[scale=0.3]{images/12-SGK-CTST-5-2-13}
	}
	\loigiai
	{
		\begin{enumerate}
			\item $a$ có vectơ chỉ phương $\overrightarrow{m}=(0;0;3)$ và $b$ có vectơ chỉ phương $\overrightarrow{n}=(4;2;0)$.\\
			Ta có $\overrightarrow{m}\cdot\overrightarrow{n}=0\cdot 4+0\cdot 2+3\cdot 0=0$.\\
			Do đó $\overrightarrow{m}\perp\overrightarrow{n}$ hay $a\perp b$.
			\item Gọi $M$ là giao điểm của $a$ và $b$, ta có $\heva{& M\in a\\&M\in b }\Rightarrow\heva{& M(1;2;3t)\\&M(1+4t';2+2t';6)}$. Khi đó ta có $$\heva{& 1=1+4t'\\&2=2+2t'\\&3t=6 }\Rightarrow\heva{& t=2\\&t'=0 }\Rightarrow M(1;2;6).$$
			Vậy giao điểm của $a$ và $b$ là $M(1;2;6)$.
		\end{enumerate}
	}
\end{vd}
\dongcham{6}

\setcounter{vd}{0}
\begin{vd}
	Trong khôn gian $Oxyz$, xét vị trí tương đối giữa hai đường thẳng $d$ và $d'$ trong mỗi trường hợp sau. Nếu chúng cắt nhau, hãy xác định tọa độ giao điểm.
	\begin{enumerate}
		\item $d \colon \heva{&x=2+3t\\&y=3+2t\\&z=4+2t}$ và $d' \colon \heva{&x=8+9t'\\&y=7+6t'\\&z=8+6t';}$
		\item $d \colon \dfrac{x}{4}=\dfrac{y-3}{3}=\dfrac{z-1}{2}$ và $d' \colon \dfrac{x-5}{8}=\dfrac{y-5}{6}=\dfrac{z-3}{4}$;
		\item $d \colon \heva{&x=2\\&y=3+2t\\&z=1-t}$ và $d' \colon \dfrac{x-4}{3}=\dfrac{y-1}{4}=\dfrac{z-5}{5}$;
		\item $d \colon \dfrac{x-2}{3}=\dfrac{y-3}{4}=\dfrac{z-2}{3}$ và $d' \colon \heva{&x=5\\&y=7+2t\\&z=5-t.}$
	\end{enumerate}
	\loigiai{
		\begin{enumerate}
			\item Đường thẳng $d$ đi qua điểm $M(2;3;4)$ và nhận $\overrightarrow{a}=(3;2;2)$ làm vectơ chỉ phương.\\
			Đường thẳng $d'$ đi qua điểm $M'(8;7;8)$ và nhận $\overrightarrow{a'}=(9;6;6)$ làm vectơ chỉ phương.\\
			Ta có $\overrightarrow{MM'}=(6;4;4)$, $\overrightarrow{a'} = 3 \overrightarrow{a} = \dfrac{3}{2} \overrightarrow{MM'}$, suy ra ba vectơ $\overrightarrow{a}$, $\overrightarrow{a'}$, $\overrightarrow{MM'}$ cùng phương. Do đó, $d \equiv d'$.
			\item Đường thẳng $d$ đi qua điểm $M(0;3;1)$ và nhận $\overrightarrow{a}=(4;3;2)$ làm vectơ chỉ phương.\\
			Đường thẳng $d'$ đi qua điểm $M'(5;5;3)$ và nhận $\overrightarrow{a'}=(8;6;4)$ làm vectơ chỉ phương.\\
			Ta có $\overrightarrow{MM'}=(5;2;2)$, $\overrightarrow{a'} = 2\overrightarrow{a}$, suy ra $\overrightarrow{a}$, $\overrightarrow{a'}$ cùng phương. Mặt khác, $\dfrac{4}{5} \ne \dfrac{3}{2}$, suy ra $\overrightarrow{a}$, $\overrightarrow{MM'}$ không cùng phương. Do đó $d \parallel d'$.
			\item Đường thẳng $d$ đi qua điểm $M(2;3;1)$ và nhận $\overrightarrow{a}=(0;2;-1)$ làm vectơ chỉ phương.\\
			Đường thẳng $d'$ đi qua điểm $M'(4;1;5)$ và nhận $\overrightarrow{a'}=(3;4;5)$ làm vectơ chỉ phương.\\
			Ta có $\overrightarrow{MM'}=(2;-2;4)$, $\left[ \overrightarrow{a}, \overrightarrow{a'} \right] = (14;-3;-6) \ne \overrightarrow{0}$, $\left[ \overrightarrow{a}, \overrightarrow{a'} \right] \cdot \overrightarrow{MM'} =10 \ne 0$, suy ra $d$ và $d'$ chéo nhau.
			item Đường thẳng $d$ đi qua điểm $M(2;3;2)$ và nhận $\overrightarrow{a}=(3;4;3)$ làm vectơ chỉ phương.\\
			Đường thẳng $d'$ đi qua điểm $M'(5;7;5)$ và nhận $\overrightarrow{a'}=(0;2;-1)$ làm vectơ chỉ phương.\\
			Ta có $\overrightarrow{MM'}=(3;4;3)$, $\left[ \overrightarrow{a}, \overrightarrow{a'} \right] = (-10;3;6) \ne \overrightarrow{0}$, $\left[ \overrightarrow{a}, \overrightarrow{a'} \right] \cdot \overrightarrow{MM'} =0 \ne 0$, suy ra $d$ và $d'$ cắt nhau.
		\end{enumerate}
	}
\end{vd}
\dongcham{29}
\boxmini{BÀI TẬP TRẮC NGHIỆM}
\setcounter{ex}{0}

\begin{ex} %[2H3B3-6]%Câu 9:
	Cho hai đường thẳng $d\colon \heva{&x=1+t \\& y=2 t \\& z=2-t} \text{ và } d'\colon \heva{&x=2+2t' \\& y=3+4t' \\& z=5-2t'.}$ Mệnh đề nào sau đây đúng?
	\choice
	{$d$ và $d'$ chéo nhau}
	{$d$ trùng $d'$}
	{\True $d$ song song $d'$}
	{$d$ cắt $d'$}
	\loigiai{
		$d$ đi qua $A=(1;0;2)$, có vectơ chỉ phương $\vec{a}=(1;2;-1).$\\
		$d'$ đi qua $B=(2;3;5)$, có vectơ chỉ phương $\vec{a'}=(2;4;-2).$\\
		Ta có $\vec{a}$ cùng phương $\vec{a'}$ nên loại B. D.\\
		$A \notin d'$ nên $d$ song song $d'$.
	}
\end{ex}
\cham{5}

\begin{ex} %[2H3B3-6]%Câu 11:
	Cho hai đường thẳng $d_1 \colon \dfrac{x-1}{1}=\dfrac{y+3}{-2}=\dfrac{z+3}{-3}$ và $d_2\colon \heva{&x=3t \\& y=-1+2t \\& z=0}$. Mệnh đề nào đưới đây đúng?
	\choice
	{\True $d_1$ cắt và không vuông góc với $d_2$}
	{$d_1$ cắt và vuông góc với $d_2$}
	{$d_1$ song song $d_2$}
	{$d_1$ chéo $d_2$}
	\loigiai{
		$d_1$ đi qua $A=(1;-3;-3)$, có vectơ chỉ phương $\vec{a}_1=(1;-2;-3).$\\
		$d_2$ đi qua $B=(0;-1;0)$, có vectơ chỉ phương $\vec{a}_2=(3;2;0).$\\
		Ta có $\vec{a}_1$ không cùng phương $\vec{a}_2$.\\
		$\vec{a}_1 \cdot \vec{a}_2 \ne 0 $.\\
		$[\vec{a}_1 , \vec{a}_2]\cdot \vec{AB} = 0$ nên $d_1$ cắt và không vuông góc với $d_2$.
	}
\end{ex}
\cham{5}
\begin{ex}
	Cho hai đường thẳng $d_1\colon\heva{&x=1-2t\\&y=1+t\\&z=1-t}$ và $d_2\colon\dfrac{x+1}{2}=\dfrac{y-2}{-1}=\dfrac{z}{1}$. Chọn khẳng định đúng.
	\choice
	{$d_1\parallel d_2$}
	{\True $d_1\equiv d_2$}
	{$d_1$, $d_2$ chéo nhau}
	{$d_1$, $d_2$ cắt nhau}
	\loigiai{
		Ta có $d_1$, $d_2$ có vectơ chỉ phương lần lượt là $\overrightarrow{u}_1=(-2;1;-1)$, $\overrightarrow{u}_2=(2;-1;1)$. \\
		Và $A(1;1;1)\in d_1$, $B(-1;2;0)\in d_2\Rightarrow \overrightarrow{AB}=(-2;1;-1)$.\\
		Khi đó $\overrightarrow{u}_1$, $\overrightarrow{u}_2$, $\overrightarrow{AB}$ cùng phương nên $d_1\equiv d_2$.
	}
\end{ex}

\begin{ex}
	Vị trí tương đối của hai đường thẳng $\Delta _1 \colon \dfrac{x-1}{3}=y=\dfrac{z+1}{2}$ và $\Delta _2 \colon \dfrac{x}{2}=\dfrac{y-1}{-1}=\dfrac{z}{1}$,
	\choice
	{Trùng nhau }
	{\True Chéo nhau}
	{Song song}
	{Cắt nhau}
	\loigiai{
		Đường thẳng	$\Delta _1$ đi qua điểm $M_1\left(1;0;-1\right)$ và có vectơ chỉ phương $\overrightarrow{u_1}=\left(3;1;2\right)$.\\
		Đường thẳng $\Delta _2$ đi qua điểm $M_2\left(0;1;0\right)$ và có vectơ chỉ phương $\overrightarrow{u_2}=\left(2;-1;1\right)$.\\
		Ta có $\heva{& \overrightarrow{u_1}\wedge \overrightarrow{u_2}=\left(3;1;-5\right)\\& \overrightarrow{M_1M_2}=(-1;1;1)} \Rightarrow \left(\overrightarrow{u_1}\wedge \overrightarrow{u_2}\right)\cdot \overrightarrow{M_1M_2}=-3+1-5=-7\ne 0$.\\
		Do đó $\Delta _1$ và $\Delta _2$ chéo nhau.
	}
\end{ex}
\begin{ex}
	Cho hai đường thẳng $d_1:\dfrac{x+1}{2}=\dfrac{y-1}{-m}=\dfrac{z-2}{-3}$ và $d_2: \dfrac{x-3}{1}=\dfrac{y}{1}=\dfrac{z-1}{1}$. Tìm tất cả các giá trị thực của $m$ để $d_1$ vuông góc $d_2$.
	\choice
	{$m=5$}
	{$m=1$}
	{$m=-5$}
	{\True$m=-1$}
	\loigiai{
	}
\end{ex}
\cham{5}

\begin{ex}%[Kiểm tra, Sở GD và ĐT - Khánh Hòa, 2019]%[Thanh Ta, 12EX6]%[2H3B3-6]%
	Cho hai đường thẳng $\Delta_1\colon \dfrac{x-1}{2}=\dfrac{y-2}{3}=\dfrac{z-3}{4}$ và $\Delta_2\colon \dfrac{x-4}{1}=\dfrac{y-3}{-2}=\dfrac{z-5}{-2}$. Tọa độ giao điểm $M$ của hai đường thẳng đã cho là
	\choice
	{$M(5;1;3)$}
	{$M(0;-1;-1)$}
	{\True $M(3;5;7)$}
	{$M(2;3;7)$}
	\loigiai{
		Phương trình tham số của đường thẳng $\Delta_1$ là $\Delta_1\colon\heva{& x=1+2t \\ & y=2+3t\\ &z=3+4t}$, thay vào phương trình $\Delta_2$ ta được
		$$\dfrac{2t-3}{1}=\dfrac{3t-1}{-2}=\dfrac{4t-2}{-2}\Rightarrow t=1.$$
		Vậy giao điểm của $\Delta_1$ và $\Delta_2$ là $M(3;5;7)$.
	}
\end{ex}
\begin{ex}%[Thi Thử Lần 2, THPT Lương Thế Vinh - Hà Nội, 2019]%[Dương BùiĐức, dự án 12EX6]%[2H3B3-6]%
	Cho hai đường thẳng $d_1\colon\heva{
		&x=1+t\\ &y=2-t\\ &z=3+2t
	}$ và $d_2\colon\dfrac{x-1}{2}=\dfrac{y-m}{1}=\dfrac{z+2}{-1}$ (với $m$ là tham số). Tìm $m$ để hai đường thẳng $d_1,\ d_2$ cắt nhau.
	\choice
	{\True $m=5$}
	{$m=7$}
	{$m=9$}
	{$m=4$}
	\loigiai{
		Ta có $ M_{1}(1;2;3)\in d_{1} $ và vectơ chỉ phương của $ d_{1} $ là $ \overrightarrow{u}_{1}=(1;-1;2) $, $ M_{2}(1;m;-2)\in d_{2} $ và vectơ chỉ phương của $ d_{2} $ là $ \overrightarrow{u}_{2}=(2;1;-2) $. Suy ra $ \overrightarrow{M_{1}M_{2}}=(0;m-2;-5) $ và $ [\overrightarrow{u}_{1},\overrightarrow{u}_{2}]=(0;6;3) $.\\
		Để $ d_{1} $ cắt $ d_{2} $ thì $ [\overrightarrow{u}_{1},\overrightarrow{u}_{2}]\cdot \overrightarrow{M_{1}M_{2}}=0\Leftrightarrow 6(m-2)-15=0\Leftrightarrow m=5 $.
	}
\end{ex}
\cham{6}
\begin{ex}
	Cho hai đường thẳng $d: \left \lbrace \begin{aligned} &x=1+mt\\ &y=t  \\&z=-1+2t   \end{aligned} \right. (t \in \mathbb{R})$ và $d': \left \lbrace \begin{aligned} &x=1-t'\\ &y=2+2t'\\&z=3-t'   \end{aligned} \right. (t' \in \mathbb{R})$. Giá trị của $m$ để hai đường thẳng $d$ và $d'$ cắt nhau là
	\choice
	{\True $m=0$}
	{$m=1$}
	{$m=-1$}
	{$m=2$}
	\loigiai{
		Đường thẳng $d$ đi qua $A(1;0;-1)$, có vectơ chỉ phương $\overrightarrow{u_1}=(m;1;2)$.\\
		Đường thẳng $d'$ đi qua $B(1;2;3)$, có vectơ chỉ phương $\overrightarrow{u_2}=(-1;2;-1)$.\\
		Ta có $[\overrightarrow{u_1}, \overrightarrow{u_2}]=(-5;m-2;2m+1)$ và $\overrightarrow{AB}=(0;2;4)$.\\
		Hai đường thẳng $d$ và $d'$ cắt nhau $\Leftrightarrow [\overrightarrow{u_1}, \overrightarrow{u_2}]\cdot AB=0\Leftrightarrow m=0$.
	}
\end{ex}
\begin{dang}{Vị trí tương đối của đường thẳng và mặt phẳng}
	Xét đường thẳng $d \colon \heva{&x=x_0+u_1t\\&y=y_0+u_2t\\&z=z_0+u_3t}$ và mặt phẳng $(P) \colon Ax + By + Cz + D =0$.
	\begin{itemize}
		\item [\iconMT] \indam{Phương pháp:} Xét $d \cap (P) \Rightarrow A(x_0+u_1t)+B(y_0+u_2t)+C(z_0+u_3t)+D=0 \quad (*)$
		\begin{boxdn}
			\begin{itemize}
				\item [$\bullet$] Nếu (*) có đúng 1 nghiệm $t$ thì $d$ cắt $(P)$;
				\item [$\bullet$] Nếu (*) vô nghiệm  thì $d$ song song $(P)$;
				\item [$\bullet$] Nếu (*) nghiệm đúng với mọi $t$ thì $d$ nằm trong $(P)$.
			\end{itemize}
		\end{boxdn}
		\item [\iconMT] \indam{Đặc biệt:} Với $\vec{u}$ là vectơ chỉ phương của $d$ và $\vec{n}$ là vectơ pháp tuyến của $(P)$ thì
		$$d \perp (P) \Leftrightarrow \vec{u} \text{ cùng phương với } \vec{n} \text{ hay } \vec{u}=k \cdot\vec{n}$$
	\end{itemize}
\end{dang}
\boxmini{BÀI TẬP TỰ LUẬN}
\setcounter{vd}{0}

\begin{vd}
	Xét vị trí tương đối giữa đường thẳng và mặt phẳng được chỉ ra ở các câu sau:
	\begin{enumEX}[a)]{1}
		\item $(\alpha) \colon y+2z=0$ và $d \colon \heva{& x=2-t \\ & y=4+2t \\ &z=1} $.
		\item $(P)\colon3x-3y+2z-5=0$ và $d\colon\heva{&x=-1+2t\\&y=3+4t\\&z=3t}~(t\in\mathbb{R})$.
		\item $(P)\colon 3x-3y+2z+1=0$ và $d\colon \dfrac{x+1}{1}=\dfrac{y}{-1}=\dfrac{z-1}{-3}$.
	\end{enumEX}
	\loigiai{
			\begin{enumEX}[a)]{1}
			\item Gọi $M(2-t;4+2t;1) \in d$, thay tọa độ $M$ vào phương trình của $(\alpha)$ ta được $4+2t+2=0 \Leftrightarrow t=-3$. Từ đó tìm được $M(5;-2;1)$. Suy ra $d$ cắt $(\alpha)$.
			\item Xét phương trình $3(-1+2t)-3(3+4t)+2\cdot3t-5=0\Leftrightarrow 0\cdot t-17=0$ (vô nghiệm).\\
			Vậy $d\parallel (P)$.
			\item Viết lại đường thẳng $d$ ở dạng tham số $\heva{&x=-1+t\\&y=-t\\&z=1-3t}$.\\
			Xét phương trình $3\cdot (-1+t)-3\cdot (-t)+2\cdot (1-3t)+1=0 \Leftrightarrow 0=0$. Kết luận phương trình có vô số nghiệm $\Rightarrow d \subset (P)$.
		\end{enumEX}
	}
\end{vd}
\dongcham{15}

\begin{vd}
	Tìm điều kiện của tham số $m$ để
	\begin{enumEX}[a)]{1}
		\item $ \Delta:\dfrac{x-10}{5}=\dfrac{y-2}{1}=\dfrac{z+2}{1}$ vuông góc với $(P):10x+2y+mz+11=0$.
		\item $d\colon \dfrac{x-1}{2}=\dfrac{y+1}{3}=\dfrac{z-1}{-1}$ song song với $(\alpha)\colon -x+m^2y+mz+1=0$.
	\end{enumEX}
	\loigiai{
	\begin{enumEX}[a)]{1}
		\item Đường thẳng $ \Delta $ có VTCP ${\overrightarrow{u}}_{\Delta}=(5;1;1)$. \\
		Mặt phẳng $(P)$ có VTPT ${\overrightarrow{n}}_P=(10;2;m)$. \\
		Để $ \Delta \perp (P) \Leftrightarrow {\overrightarrow{u}}_{\Delta}\parallel {\overrightarrow{n}}_P \Leftrightarrow \dfrac{10}{5}=\dfrac{2}{1}=\dfrac{m}{1} \Leftrightarrow m=2$.
		\item Đường thẳng $d$ có phương trình tham số $\heva{&x=1+2t\\&y=-1+3t\\&z=1-t.}$\\
		Xét phương trình $-(1+2t)+m^2(-1+3t)+m(1-t)+1=0\Leftrightarrow \left(3m^2-m-2\right)t-m^2+m=0.\quad (1)$\\
		Ta có $d\parallel (\alpha)$ khi và chỉ khi $(1)$ vô nghiệm $\Leftrightarrow \heva{&3m^2-m-2=0\\&-m^2+m\neq 0}\Leftrightarrow m=-\dfrac{2}{3}$.
	\end{enumEX}}
\end{vd}
\dongcham{12}
\boxmini{BÀI TẬP TRẮC NGHIỆM}
\setcounter{ex}{0}
\begin{ex}
	Cho đường thẳng $d:\,\displaystyle\frac{x-1}{2}=\frac{y}{-2}=\dfrac{z-1}{1}$. Tìm tọa độ giao điểm $ M $ của đường thẳng $ d $ với mặt phẳng ($ Oxy $).
	\choice
	{\True$M(-1;2;0) $}
	{$ M(1;0;0)$}
	{$ M(2;-1;0)$}
	{$ M(3;-2;0)$}
	\loigiai{
	}
\end{ex}
\cham{3}

\begin{ex}
	Cho đường thẳng $d: \dfrac{x-1}{-1}=\dfrac{y+3}{2}=\dfrac{z-3}{1}$ và mặt phẳng $(P): 2x+y-2z+9=0$. Tìm toạ độ giao điểm của $d$ và $(P)$.
	\choice
	{$(2; 1; 1)$}
	{\True $(0;-1;4)$}
	{$(1; -3; 3)$}
	{$(2; -5; 1)$} 
	\loigiai{
	}
\end{ex}
\cham{4}
\begin{ex}
	Cho mặt phẳng $(\alpha)\colon x+2y+3z-6=0$ và đường thẳng $\Delta\colon\dfrac{x+1}{-1}=\dfrac{y+1}{-1}=\dfrac{z-3}{1}$. Mệnh đề nào sau đây đúng?
	\choice
	{$\Delta$ cắt và không vuông góc với $(\alpha)$}
	{$\Delta\parallel (\alpha)$}
	{\True $\Delta\subset (\alpha)$}
	{$\Delta \perp (\alpha)$}
	\loigiai{
		Mặt phẳng $(\alpha)$ có vectơ pháp tuyến $\overrightarrow{n}=(1;2;3)$.\\
		Đường thẳng $\Delta$ đi qua $M(-1;-1;3)$ và có vectơ chỉ phương $\overrightarrow{u}=(-1;-1;1)$.\\
		Ta có $\overrightarrow{n}\cdot\overrightarrow{u}=1\cdot (-1)+2\cdot (-1)+3\cdot 1=0$ và $M\in (\alpha)$.\\
		Vậy $\Delta\subset (\alpha)$.
	}
\end{ex}

\begin{ex}
	Cho đường thẳng $d:\dfrac{x-1}{1}=\dfrac{y-1}{4}=\dfrac{z-m}{-1}$ và mặt phẳng $(P): 2x+my-(m^2+1)z+m-2m^2=0$. Có bao nhiêu giá trị của $m$ để đường thẳng $d$ nằm trên $(P)$?
	\choice{$0$}{\True $1$}{$2$}{Vô số}
	\loigiai{
	}
\end{ex}
\cham{6}

\begin{ex}
	Cho mặt phẳng $(\alpha): x+y+z-6=0$ và đường thẳng $\Delta:\heva{&x=m+t\\&y=-1+nt\\&z=4+2t}$. Tìm điều kiện của $m$ và $n$ để đường thẳng $\Delta$ song song với mặt phẳng $(\alpha)$.
	\choice{\True $\heva{&m\neq 3\\&n=-3}$}
	{$\heva{&m=3\\&n\neq -3}$}
	{$\heva{&m=3\\&n=-3}$}
	{$\heva{&m\neq 3\\&n\neq -3}$}
	\loigiai{
	}
\end{ex}
\cham{6}
\begin{ex}
	Cho đường thẳng $d\colon \dfrac{x-1}{2}=\dfrac{y+2}{-1}=\dfrac{z+1}{1}.$ Trong các mặt phẳng dưới đây mặt phẳng nào vuông góc với đường thẳng $d$?
	\choice
	{$2x-2y+2z+4=0$}
	{$4x-2y-2z-4=0$}
	{$4x+2y+2z+4=0$}
	{\True $4x-2y+2z+4=0$}
	\loigiai{
		Đường thẳng $d$ có vec-tơ chỉ phương $\overrightarrow{u}=(2;-1;1)$.\\
		Xét mặt phẳng $4x-2y+2z+4=0$ có vec-tơ pháp tuyến $\overrightarrow{n}=(4;-2;2)\Rightarrow\overrightarrow{n}=2\overrightarrow{v}.$\\
		$\Rightarrow d$ vuông góc với mặt phẳng có phương trình $4x-2y+2z+4=0$.
	}
\end{ex}

\begin{ex}
	Cho đường thẳng $d\colon\heva{&x=3+2t\\&y=5-3mt\\&z=-1+t.}$ và mặt phẳng $(P)\colon4x-4y+2z-5=0$. Giá trị nào của $m$ để đường thẳng $d$ vuông góc với mặt phẳng $(P)$.
	\choice
	{$m=-\dfrac{5}{6}$}
	{\True $m=\dfrac{2}{3}$}
	{$m=\dfrac{3}{2}$}
	{$m=\dfrac{5}{6}$}
	\loigiai{
		\begin{itemize}
			\item Mặt phẳng $(P)$ có vectơ pháp tuyến là $\overrightarrow{n}=(4;-4;2)$.
			\item Đường thẳng $d$ có vectơ chỉ phương là $\overrightarrow{u}=(2;-3m;1)$.
		\end{itemize}
		Đường thẳng $d$ vuông góc với mặt phẳng $(P)$ khi và chỉ khi $\overrightarrow{n}$ cùng phương với $\overrightarrow{u}$\\
		$\Leftrightarrow\dfrac{2}{4}=\dfrac{-3m}{-4}=\dfrac{1}{2}\Leftrightarrow3m=2\Leftrightarrow m=\dfrac{2}{3}$.}
\end{ex}

\begin{ex}
	Cho điểm $A(1;2;3)$ và đường thẳng $d\colon\dfrac{x-2}{2}=\dfrac{y+2}{-1}=\dfrac{z-3}{1}$. Phương trình mặt phẳng $(P)$ đi qua $A$ và vuông góc với đường thẳng $d$ là
	\choice
	{\True $2x-y+z-3=0$}
	{$x+2y+3z-7=0$}
	{$x+2y+3z-1=0$}
	{$2x-y+z=0$}
	\loigiai{
		Đường thẳng $d$ có một vectơ chỉ phương $\overrightarrow{u}_d=(2;-1;1)$.\\
		Mặt phẳng $(P)$ vuông góc với đường thẳng $d$ nên có một vectơ pháp tuyến $\overrightarrow{n}_P=\overrightarrow{u}_d=(2;-1;1)$.\\
		Mặt phẳng $(P)$ đi qua $A(1;2;3)$ và có một vectơ pháp tuyến $\overrightarrow{n}_P=(2;-1;1)$ có phương trình là
		\[2(x-1)-1(y-2)+1(z-3)=0\Leftrightarrow 2x-y+z-3=0.\]
	}
\end{ex}

\begin{ex}%[2H3B3-6]%
	Cho hai đường thẳng chéo nhau  $d_1:\dfrac{x-2}{2}=\dfrac{y+2}{1}=\dfrac{z-6}{-2}$; $d_2:\dfrac{x-4}{1}=\dfrac{y+2}{-2}=\dfrac{z+1}{3}$. Phương trình mặt phẳng $(P)$ chứa $d_1$ và song song với $d_2$ là
	\choice
	{$(P):x+8y+5z+16=0$}
	{$(P):x+4y+3z-12=0$}
	{$(P):2x+y-6=0$}
	{\True $(P):x+8y+5z-16=0$}
	\loigiai
	{
		$d_1$ đi qua điểm $M(2;-2;6)$ và có vectơ chỉ phương $\vec{u}_1=(2;1;-2).$ \\
		$d_2$ đi qua điểm $N(4;-2;-1)$ và có vectơ chỉ phương $\vec{u}_2=(1;-2;3)$.\\
		Vì mặt phẳng $(P)$ chứa $d_1$ và song song với $d_2$ nên chọn một vectơ pháp tuyến của $(P)$ là $ \vec{n}_{(P)}=\left[\vec{u}_1,\vec{u}_2\right]=(-1;-8;-5)$.\\
		Mặt phẳng $(P)$ đi qua $M(2;-2;6)$ và nhận $\vec{n}_{(P)}=(-1;-8;-5)$ làm vectơ pháp tuyến nên có phương trình $$-1\cdot(x-2)-8\cdot(y+2)-5\cdot(z-6)=0  \Leftrightarrow x+8y+5z-16=0.$$
	}
\end{ex}

\begin{ex}%[HK2 (2017-2018),Sở GD Quảng Trị]%[Phan Minh Tâm ex9]%[2H3B2-3]%
	Cho hai đường thẳng $ d_1 \colon \dfrac{x-1}{2}=\dfrac{y+1}{3}=\dfrac{z-3}{-5} $ và $ d_2\colon \heva{&x=-1+t\\&y=4+3t\\&z=1+t} $. Tìm phương trình mặt phẳng chứa đường thẳng $ d_1 $ và song song với đường thẳng $ d_2. $
	\choice
	{$ 18x-7y+3z+34=0 $}
	{$ 18x+7y+3z-20=0 $}
	{$ 18x+7y+3z+20=0 $}
	{\True $ 18x-7y+3z-34=0 $}
	\loigiai
	{Đường thẳng $ d_1 $ qua $ M(1;-1;3) $ và nhận $ \vec{u_1} =(2;3;-5) $ làm vectơ chỉ phương; $ d_2 $ có vectơ chỉ phương $ \vec{u_2}=(1;3;1) $.\\
		Mặt phẳng $ (P) $ chứa $ d_1 $ và song song $ d_2 $ nên nhận vectơ $ \vec{n} = \left[\vec{u_1},\vec{u_2}\right]=(18;-7;3) $ làm vectơ pháp tuyến.\\
		Vậy phương trình tổng quát của $ (P)$ là \begin{eqnarray*}
			&&18(x-1)-7(y+1)+3(z-3)=0\\ &\Leftrightarrow& 18x-7y+3z-34=0.
		\end{eqnarray*}
	}
\end{ex}
\begin{dang}{Hình chiếu, đối xứng}
\begin{enumerate}[\iconCH]
	\item \indamm{Bài toán 1: Tìm hình chiếu vuông góc của điểm $M$ trên $(P)$:}
	\immini{\begin{itemize}
			\item [$\bullet$] Viết phương trình đường thẳng $MH$ qua $M$ và nhận $\overrightarrow{n_P}$ làm vectơ chỉ phương;
			\item [$\bullet$] Giải hệ giữa đường $MH$ với mặt phẳng $(P)$, tìm $t$. Từ đó, suy ra tọa độ $H$.
		\end{itemize}
		\begin{note}
			Gọi $M'$ đối xứng với $M$ qua mặt phẳng $(P)$ thì
			$$\heva{&x_M'=2x_M-x_H\\&y_M'=2y_M-y_H\\&z_M'=2z_M-z_H}.$$
	\end{note}}{
		\begin{tikzpicture}[scale=1, line join=round, line cap=round]
			\tkzDefPoints{0/0/A,4/0/B,5/1.5/C,2/1/H,2/3/M}
			\coordinate (D) at ($(A)+(C)-(B)$);
			\coordinate (M') at ($(H)+(H)-(M)$);
			\tkzInterLL(M,M')(A,B)\tkzGetPoint{I}
			\tkzDrawPolygon(A,B,C,D)
			\tkzDrawSegments(M,H I,M')
			\tkzDrawSegments[dashed](H,I)
			\draw[fill=black] (2,1) circle (1.5pt) (2,3) circle (1.5pt) (2,-1) circle (1.5pt);
			\draw[->] (3.5,1.1)--(3.5,2.5) node[above right]{$\vec{n_P}$};
			\tkzMarkAngles[size=0.7cm,arc=l](B,A,D)
			\tkzLabelAngles[pos=0.45,rotate=10](B,A,D){$P$}
			\tkzLabelPoints[right](M,M',H)
	\end{tikzpicture}}
	\item \indamm{Bài toán 2: Tìm hình chiếu vuông góc của điểm $M$ trên $d$:}
		\immini{
		\begin{itemize}
			\item [$\bullet$] Tham số điểm $H \in d$ theo ẩn $t$;
			\item [$\bullet$] Giải $\overrightarrow{MH} \cdot \overrightarrow{u_d}=0$, tìm $t$. Từ đó, suy ra tọa độ $H$.
		\end{itemize}
		\begin{note}
			Gọi $M'$ đối xứng với $M$ qua mặt phẳng $d$ thì
			$$\heva{&x_M'=2x_M-x_H\\&y_M'=2y_M-y_H\\&z_M'=2z_M-z_H}.$$
	\end{note}}{
		\begin{tikzpicture}[scale=1, line join=round, line cap=round]
			\tkzDefPoints{0/0/A,5/0/B,2/0/H,2/1.5/M}
			\coordinate (M') at ($(H)+(H)-(M)$);
			\tkzDrawSegments(A,B M,M')
			\draw[fill=black] (2,0) circle (1.5pt) (2,1.5) circle (1.5pt) (2,-1.5) circle (1.5pt);
			\draw[->] (2.5,0.3)--(4,0.3) node[above right]{$\vec{u_d}$};
			\tkzLabelPoints[right](M,M')
			\tkzLabelPoints[below right](H)
			\tkzLabelSegment[pos=0.9,below right](A,B){$d$}
	\end{tikzpicture}}
\end{enumerate}
\end{dang}
\boxmini{BÀI TẬP TỰ LUẬN}
\setcounter{vd}{0}

\begin{vd}
	Trong hệ tọa độ $Oxyz$, cho điểm $M(2;-3;1)$ và đường thẳng $d\colon\dfrac{x+1}{2}=\dfrac{y+2}{-1}=\dfrac{z}{2}$. 
	\begin{enumEX}[a)]{1}
		\item Tìm tọa độ hình chiếu vuông góc của điểm $M$ lên $d$.
		\item Tìm tọa độ điểm $M'$ đối xứng với điểm $M$ qua $d$.
	\end{enumEX}
	\loigiai{
		Gọi $H$ là điểm thuộc đường thẳng $d$, suy ra $H(-1+2t;-2-t;2t)$ với $t\in\mathbb{R}$.\\
		Ta có $\overrightarrow{MH}=(2t-3;1-t;2t-1)$ và một vectơ chỉ phương của đường thẳng là $\overrightarrow{u}=(2;-1;2)$.\\
		Điểm $H$ là hình chiếu của $M$ lên đường thẳng $d$ khi $2(2t-3)-(1-t)+2(2t-1)=0\Leftrightarrow 9t-9=0\Leftrightarrow t=1$.\\
		Suy ra $H(1;-3;2)$, do đó tọa độ điểm $M'$ đối xứng với $M$ qua $d$ là $M'(0;-3;3)$.
	}
\end{vd}
\dongcham{13}
\begin{vd}
	Trong không gian với hệ tọa độ $Oxyz$, cho điểm $M(2;7;-9)$ và mặt phẳng $(P)\colon x+2y-3z-1=0$. 
	\begin{enumEX}[a)]{1}
		\item Tìm tọa độ hình chiếu vuông góc của $M$ trên mặt phẳng $(P)$.
		\item Tìm tọa độ điểm $M'$ đối xứng với điểm $M$ qua $(P)$.
	\end{enumEX}
	\loigiai{
		Đường thẳng $d$ đi qua $M$ vuông góc với $(P)$ có phương trình $\heva{&x=2+t \\&y=7+2t \\&z=-9-3t} (t\in \mathbb{R})$. \\
		Gọi $H$ là hình chiếu vuông góc của điểm $M$ trên $(P)$ thì $H=d\cap(P)$.\\
		Xét phương trình: $2+t+2(7+2t)-3(-9-3t)-1=0 \Leftrightarrow 14t+42=0 \Leftrightarrow t=-3$. \\
		Với $t=-3 \Rightarrow \heva{&x=-1\\&y=1\\&z=0}$. Vậy $H(-1;1;0)$.\\
		Từ đây, suy ra $M'(-4;-5;9)$}
\end{vd}
\dongcham{15}

\boxmini{BÀI TẬP TRẮC NGHIỆM}
\setcounter{ex}{0}

\begin{ex}
	Hình chiếu vuông góc của điểm $A(3;-4;5)$ trên mặt phẳng $(Oxz)$ là điểm
	\choice
	{$M(3;0;0)$}
	{$M(0;-4;5)$}
	{$M(0;0;5)$}
	{\True $M(3;0;5)$}
	\loigiai{
		Hình chiếu vuông góc của điểm $A(3;-4;5)$ trên mặt phẳng $(Oxz)$ là điểm $M(3;0;5)$.}
\end{ex} \cham{3}

\begin{ex}
	Hình chiếu vuông góc của điểm $A(1;2;3)$ trên mặt phẳng $(Oxy)$ là điểm
	\choice
	{$M(0;0;3)$}
	{\True $N(1;2;0)$}
	{$Q(0;2;0)$}
	{$P(1;0;0)$}
	\loigiai
	{
		Hình chiếu vuông góc của điểm $A(1;2;3)$ trên mặt phẳng $(Oxy)$ là điểm $N(1;2;0)$.
	}
\end{ex} \cham{3}

\begin{ex}
	Hình chiếu vuông góc của điểm $M(2;1;-3)$ lên mặt phẳng $(Oyz)$ có tọa độ là
	\choice
	{$(2;0;0)$}
	{$(2;1;0)$}
	{\True $(0;1;-3)$}
	{$(2;0;-3)$}
	\loigiai{
		Điểm thuộc $(Oyz)$ có tọa độ $(0;y;z)$ nên hình chiếu của $M$ lên $(Oyz)$ có tọa độ là $(0;-1;3)$.
	}
\end{ex} \cham{3}

\begin{ex}
	Hình chiếu vuông góc của điểm $A(3;2;1)$ trên trục $Ox$ có tọa độ là
	\choice
	{$(0;2;1)$}
	{$(0;2;0)$}
	{\True $(3;0;0)$}
	{$(0;0;1)$}
	\loigiai{
		Hình chiếu vuông góc của điểm $A(3;2;1)$ lên trục $Ox$ là $A'(3;0;0)$.
	}
\end{ex} \cham{3}

\begin{ex}
	Hình chiếu của điểm $M(2;3;-2)$ trên trục $Oy$ có tọa độ là
	\choice
	{$ (2;0;0) $}
	{\True $ (0;3;0) $}
	{$ (0;0;-2) $}
	{$ (2;0;-2) $}
	\loigiai{
		Hình chiếu của điểm $M(2;3;-2)$ trên trục $Oy$ có tọa độ là $(0;3;0)$.
	}
\end{ex} \cham{3}

\begin{ex}
	Cho điểm $M(3;2;-1)$, điểm $M'(a;b;c)$ đối xứng của M qua trục $Oy$, khi đó $a+b+c$ bằng
	\choice
	{$6$}
	{$2$}
	{$4$}
	{\True $0$}
	\loigiai{
		Với $M(a;b;c)\Rightarrow$ điểm đối xứng của $M$ qua trục $Oy$ là $M'(-a;b;-c)$ \\
		$ \Rightarrow M'(-3;2;1)\Rightarrow a+b+c=0 $.}
\end{ex}

\begin{ex}%[Thi thử L2, Cụm-NBHL,2020]%[Võ Minh Tâm, EX-11-2020]%[2H3B1-1]%
	Điểm đối xứng với điểm $A(-2;7;5)$ qua mặt phẳng $(Oxz)$ là điểm $B$ có tọa độ là
	\choice
	{$B(2;7;-5)$}
	{\True$B(-2;-7;5)$}
	{$B(-2;7;-5)$}
	{$B(2;-7;-5)$}
	\loigiai{
		Hình chiếu vuông góc của điểm $A(-2;7;5)$ trên mặt phẳng $(Oxz)$ là điểm $H(-2;0;5)$.\\
		Điểm $B(-2;-7;5)$ đối xứng với điểm $A$ qua mặt phẳng $(Oxz)$ nên $H$ là trung điểm của $AB$. Vậy điểm đối xứng với điểm $A(-2;7;5)$ qua mặt phẳng $(Oxz)$ là điểm $B(-2;-7;5)$.
	}
\end{ex}

\begin{ex}
	Tọa độ hình chiếu vuông góc của điểm $A(2; -1; 0)$ lên mặt phẳng $(P): 3x - 2y + z + 6 = 0$ là
	\choice
	{$(5; -3; 1)$}
	{\True $(-1; 1; -1)$}
	{$(1; 1; 1)$}
	{$(3; -2; 1)$}
	\loigiai{
		Gọi $H (x; y; -6 -3x + 2y)$ là hình chiếu của $A$ lên mặt phẳng $P$.\\ Ta có $\overrightarrow{AH} = (x - 2; y + 1; -6 - 3x + 2y)$. \\Do $\overrightarrow{AH} \perp (P)$ nên hai vectơ $\overrightarrow{AH}$ và $\overrightarrow{n}_P$ cùng phương.\\ Suy ra ta có hệ phương trình \begin{center}$\dfrac{x - 2}{3} = \dfrac{y + 1}{-2} = \dfrac{-6-3x+2y}{1}.$\end{center}
		Giải hệ ta thu được một nghiệm là $(-1; 1; -1)$.
	}
\end{ex}
\cham{5}
\begin{ex}
	Gọi hình chiếu vuông góc của điểm $A(3; -1; -4)$ lên mặt phẳng $(P): 2x - 2y - z -3 = 0$ là điểm
	$H(a; b; c).$ Khi đó khẳng định nào sau đây đúng?
	\choice
	{\True $a+b+c=-1$}
	{$a+b+c=3$}
	{$a+b+c=5$}
	{$a+b+c=-\dfrac{5}{3}$}
	\loigiai{
		\begin{itemize}
			\item [$\bullet$] Đường thẳng $AH$ qua $A(3;-1;-4)$ và nhận $\vec{n}=(2;-2;-1)$ làm vectơ pháp tuyến nên có phương trình là
			$$\heva{&x=3+2t\\&y=-1-2t\\&z=-4-t}$$
			\item [$\bullet$] $H\left(3+2t;-1-2t;-4-t \right) =AH \cap (P)$. Phương trình để xác định $t$ (\textit{thay vào phương trình của} $(P)$) là
			$$2(3+2t)-2(-1-2t)-(-4-t)-3=0 \Leftrightarrow t=-1.$$
			Với $t=-1$ thì $H(1;1;-3)$. Suy ra $a+b+c=1+1-3=-1.$
		\end{itemize}
	}
\end{ex}
\cham{5}
\begin{ex}
	Cho mặt phẳng $(P): 2x+2y-z+9=0$ và điểm $A(-7; -6; 1)$. Tìm tọa độ điểm $A'$ đối xứng với điểm $A$ qua mặt phẳng $(P)$.
	\choice
	{\True $A'(1; 2; -3)$}
	{$A'(1; 2; 1)$}
	{$A'(5; 4; 9)$}
	{$A'(9; 0; 9)$}
	\loigiai{
		Họi $H$ là hình chiếu vuông góc của điểm $A$ lên $(P)$.
		\begin{itemize}
			\item [$\bullet$] Đường thẳng $AH$ qua $A(-7;-6;1)$ và nhận $\vec{n}=(2;2;-1)$ làm vectơ pháp tuyến nên có phương trình là
			$$\heva{&x=-7+2t\\&y=-6+2t\\&z=1-t}$$
			\item [$\bullet$] $H\left(-7+2t;-6+2t;1-t \right) =AH \cap (P)$. Phương trình để xác định $t$ (\textit{thay vào phương trình của} $(P)$) là
			$$2(-7+2t)+2(-6+2t)-(1-t)+9=0 \Leftrightarrow t=2.$$
			Với $t=2$ thì $H(-3;-2;-1)$ và $H$ là trung điểm của đoạn $AA'$ nên
			$$\heva{&x_{A'}=2x_H-x_A=1\\&y_{A'}=2y_H-y_A=2\\&z_{A'}=2z_H-z_A=-3} \Rightarrow A'(1;2;-3).$$
		\end{itemize}
	}
\end{ex}
\cham{5}


\begin{ex}
	Cho điểm $A\left(4; - 3; 2\right)$ và đường thẳng $d: \dfrac{x + 2}{3}=\dfrac{y + 2}{2}=\dfrac{z}{- 1}$. Gọi điểm $H$ là hình chiếu vuông góc của điểm $A$ lên đường thẳng $d$. Tọa độ điểm $H$ là
	\choice
	{$H\left(5; 4; - 1\right)$}
	{\True $H\left(1; 0; - 1\right)$}
	{$H\left(- 5; - 4; 1\right)$}
	{$H\left(- 2; - 2; 0\right)$}
	\loigiai{
		$d$ có vectơ chỉ phương $\vec{u}=(3;2;-1)$.
		\begin{itemize}
			\item [$\bullet$]  Gọi $H(-2+3t;-2+2t;-t) \in d$, ta có $\vec{AH}=(3t-6;2t+1;-t-2)$.
			\item [$\bullet$] $\vec{AH}$ vuông góc $\vec{u}$, suy ra $\vec{AH} \cdot \vec{u}=0$ hay
			$$ (3t-6) \cdot 3 + (2t+1) \cdot 2 + (-t-2) \cdot (-1)=0 \Leftrightarrow t=1.$$
		\end{itemize}
		Với $t=1$ thì $H(1;0;-1)$.
	}
\end{ex}
\cham{7}
\begin{ex}%[2-TT-ChuyenBacGiang-thang4-2019]%[Duong Xuan Loi, dự án tex đề W-T-B]%[2H3K3-8]%Câu 33
	Cho đường thẳng $d \colon \dfrac{x-1}{2}=\dfrac{y+1}{1}=\dfrac{z}{-1}$, $M(2;1;0)$. Gọi $H(a;b;c)$ là điểm thuộc $d$ sao cho $MH$ có độ dài nhỏ nhất. Tính $T=a^2+b^2+c^2$.
	\choice
	{$T=\sqrt{5}$}
	{$T=12$}
	{$T=21$}
	{\True $T=6$}
	\loigiai{
		Phương trình tham số của đường thẳng $d$ là $\left\{\begin{aligned}
			&x=1+2t\\
			&y=-1+t\\
			&z=-t.\\
		\end{aligned}\right. $\\
		Lấy $H \in d \Rightarrow H(1+2t;-1+t;-t)$ và $\overrightarrow{MH}=(2t-1;t-2;-t)$.\\
		$MH$ nhỏ nhất khi và chỉ khi $H$ là hình chiếu của $M$ xuống $d$, do đó
		$$\overrightarrow{MH} \cdot \overrightarrow{u}_d=0\Leftrightarrow 2(2t-1)+(t-2)+t=0 \Leftrightarrow t=\dfrac{2}{3}.$$
		Vậy $H\left(\dfrac{7}{3};-\dfrac{1}{3};-\dfrac{2}{3}\right) \Rightarrow \left\{\begin{aligned}
			&a=\dfrac{7}{3}\\
			&b=-\dfrac{1}{3}\\
			&c=-\dfrac{2}{3}\\
		\end{aligned}\right. \Rightarrow T=a^2+b^2+c^2=6$.}
\end{ex}
\begin{ex}
	Cho điểm $M\left(1; 2; - 6\right)$ và đường thẳng $d: \heva{ & x=2 + 2t \\ & y=1 - t \\  & z= - 3 + t} \left(t\in \mathbb{R}\right)$. Điểm $N$ là điểm đối xứng của $M$ qua đường thẳng $d$ có tọa độ là
	\choice
	{$N\left(0; 2; - 4\right)$}
	{\True $N\left(- 1; 2; - 2\right)$}
	{$N\left(1; - 2; 2\right)$}
	{$N\left(- 1; 0; 2\right)$}
	\loigiai{
		$d$ có vectơ chỉ phương $\vec{u}=(2;-1;1)$. Gọi $H$ là hình chiếu vuông góc của $m$ trên $d$. 
		\begin{itemize}
			\item [$\bullet$] Ta có $H(2+2t;1-t;-3+t) \in d$ và $\vec{MH}=(2t+1;-t-1;t+3)$.
			\item [$\bullet$] $\vec{MH}$ vuông góc $\vec{u}$, suy ra $\vec{MH} \cdot \vec{u}=0$ hay
			$$ (2t+1) \cdot 2 + (-t-1) \cdot (-1) + (t+3) \cdot (1)=0 \Leftrightarrow t=-1.$$
		\end{itemize}
		Với $t=-1$ thì $H(0;2;-4)$ và $H$ là trung điểm của đoạn $MN$ nên 
		$$\heva{&x_{N}=2x_H-x_M=-1\\&y_{M}=2y_H-y_M=2\\&z_{N}=2z_H-z_M=-2} \Rightarrow N(-1;2;-2).$$
	}
\end{ex}
\cham{7}

\begin{ex}
	Cho đường thẳng $\Delta: \dfrac{x}{2}=\dfrac{y+1}{1}=\dfrac{z-1}{-1}$ và hai điểm $A(1;0;1)$, $B(-1;1;2)$. Biết điểm $M(a;b;c)$ thuộc $\Delta $ sao cho $\left| \overrightarrow{MA}-3\overrightarrow{MB} \right|$ đạt giá trị nhỏ nhất. Khi đó, tổng $a+2b+4c$ bằng bao nhiêu?
	\choice
	{$0$}
	{$-1$}
	{\True $2$}
	{$1$}
	\loigiai{
		\begin{itemize}
			\item [\iconMT] \indam{Cách 1:} Gọi $I$ là điểm thỏa $\overrightarrow{IA}-3\overrightarrow{IB}=\vec{0}$, suy ra $I\left(-2;\dfrac{3}{2};\dfrac{5}{2}\right)$.\\
			Theo kết quả của \indamm{Bài toán 5} thì  $\left| \overrightarrow{MA}-3\overrightarrow{MB} \right|$ nhỏ nhất khi $M$ là hình chiếu vuông góc của điểm $I$ lên $\Delta$.
			\begin{itemize}
				\item Gọi $M(2t;-1+t;1-t) \in \Delta$ là hình chiếu vuông góc của $I$ lên $\Delta$. Ta có
				$$\vec{IM} \cdot \vec{u}_\Delta = 0 \Leftrightarrow t=-\dfrac{1}{2}.$$
				\item Với $t=-\dfrac{1}{2}$ thì $M\left( -1;-\dfrac{3}{2};\dfrac{3}{2}\right)$. Suy ra $a+2b+4c=2$.
			\end{itemize}
			\item [\iconMT] \indam{ Cách 2:} Ta tham số tọa độ điểm $M$, sau đó dùng khảo sát hàm để xứ lý max - min.\\
			Gọi $M(2m;-1+m;1-m)\in\Delta$. Ta có $$\overrightarrow{MA}-3\overrightarrow{MB}=(4+4m;-5+2m;-3-2m)$$
			Suy ra $$\left|\overrightarrow{MA}-3\overrightarrow{MB}\right|=\sqrt{24m^2+24m+50}=\sqrt{24\left(m+\dfrac{1}{2}\right)^2 +44}.$$
			Nhận xét $\left|\overrightarrow{MA}-3\overrightarrow{MB}\right|$ nhỏ nhất khi $m=-\dfrac{1}{2}$. Từ đó suy ra $M\left( -1;-\dfrac{3}{2};\dfrac{3}{2}\right)$.\\
			Vậy $a+2b+4c=2$.
		\end{itemize}
	}
\end{ex}

\begin{ex}
	Cho ba điểm $A(0;-2;-1)$, $B(-2;-4;3)$, $C(1;3;-1)$ và mặt phẳng $(P)\colon x+y-2z-3=0$. Gọi $M(a;b;c)\in (P)$ sao cho $\left|\vec{MA}+\vec{MB}+2\vec{MC}\right|$ đạt giá trị nhỏ nhất. Tính $a-b+2c$.
	\choice
	{$3$}
	{$-1$}
	{$4$}
	{\True $-2$}
	\loigiai{
		Gọi $I$ là điểm sao cho $\vec{IA}+\vec{IB}+2\vec{IC}=\vec{0}\Rightarrow I(0;0;0)$. Từ đó ta có
		\begin{eqnarray*}
			\left|\vec{MA}+\vec{MB}+2\vec{MC}\right|
			&=&\left|\left(\vec{IA}-\vec{IM}\right) +\left(\vec{IB}-\vec{IM}\right)+2\cdot\left(\vec{IC}-\vec{IM}\right)\right|\\
			&=&\left|\left( \vec{IA}+\vec{IB}+2\vec{IC}\right) -4\vec{IM}\right|\\
			&=&\left|\vec{0} -4\vec{IM}\right|=4IM.
		\end{eqnarray*}
		Bởi vậy $\left|\vec{MA}+\vec{MB}+2\vec{MC}\right|$ đạt giá trị nhỏ nhất $\Leftrightarrow IM$ đạt giá trị nhỏ nhất $\Leftrightarrow M$ là hình chiếu vuông góc của $I$ trên mặt phẳng $(P)$ $\Leftrightarrow M$ là giao điểm của đường thẳng $d$ đi qua $I$ và vuông góc với mặt phẳng $(P)$.\\
		Phương trình đường thẳng $d$ là $d\colon \heva{&x=t\\&y=t\\&z=-2t.}$\\
		Tọa độ giao điểm của $d$ và $(P)$ ứng với $t$ là nghiệm phương trình
		$$(t)+(t)-2\cdot(-2t)-3=0 \Leftrightarrow t=\dfrac{1}{2}.$$
		Tọa độ điểm $M$ cần tìm là $M\left(\dfrac{1}{2};\dfrac{1}{2};-1\right)$. Suy ra $a-b+2c=-2$
	}
\end{ex}
\Closesolutionfile{ans}
\subsection{BÀI TẬP TRẮC NGHIỆM TỰ LUYỆN}
\subsection*{\indam{PHẦN I. Câu trắc nghiệm nhiều phương án lựa chọn. Thí sinh trả lời từ câu 1 đến câu 12. Mỗi câu hỏi thí sinh chỉ chọn một phương án.}} 
	\setcounter{ex}{0}
	\Opensolutionfile{ans}[ans/B2-De2-1]
\begin{ex}%[2H5N2-2]
	Trong không gian với hệ trục tọa độ $O x y z$, cho hai điểm $A(0 ;-1 ;-2)$ và $B(2 ; 2 ; 2)$. Véc-tơ $\overrightarrow{a}$ nào dưới đây là một véc-tơ chỉ phương của đường thẳng $A B$?
	\choice
	{$\overrightarrow{a}=(-2 ; 1 ; 0)$}
	{$\overrightarrow{a}=(2 ; 3 ; 0)$}
	{$\overrightarrow{a}=(2 ; 1 ; 0)$}
	{\True $\overrightarrow{a}=(2 ; 3 ; 4)$}
	\loigiai{
		Ta có $\overrightarrow{A B}=(2 ; 3 ; 4)$ nên đường thẳng $A B$ có một véc-tơ chỉ phương là $\overrightarrow{a}=(2 ; 3 ; 4)$.}
\end{ex}

%G:\My Drive\CODE12-2024\DE-ON-THEO BAI\2H5-TACH DE\Bai2-De2.tex
\begin{ex}%[2H5H2-1]
	Đường thẳng $(\Delta)\colon \dfrac{x-1}{2}=\dfrac{y+2}{1}=\dfrac{z}{-1}$ \textbf{không} đi qua điểm nào dưới đây?
	\choice
	{$C(3 ;-1 ;-1)$}
	{$D(1 ;-2 ; 0)$}
	{\True $A(-1 ; 2 ; 0)$}
	{$B(-1 ;-3 ; 1)$}
	\loigiai{
		Ta có $\dfrac{-1-1}{2} \neq \dfrac{2+2}{1} \neq \dfrac{0}{-1}$ nên điểm $A(-1 ; 2 ; 0)$ không thuộc đường thẳng $(\Delta)$.
	}
\end{ex}

%G:\My Drive\CODE12-2024\DE-ON-THEO BAI\2H5-TACH DE\Bai2-De2.tex
\begin{ex}%[2H5H2-3]
	Cho đường thẳng $\Delta$ đi qua điểm $M(2 ; 0 ;-1)$ và có một véc-tơ chỉ phương $\overrightarrow{a}=(4 ;-6 ; 2)$. Phương trình tham số của đường thẳng $\Delta$ là
	\choice
	{\True $\heva{&x=2+2 t \\& y=-3 t \\& z=-1+t}$}
	{$\heva{&x=-2+4 t \\& y=-6 t \\& z=1+2 t}$}
	{$\heva{&x=4+2 t \\& y=-3 t \\& z=2+t}$}
	{$\heva{&x=-2+2 t \\& y=-3 t \\& z=1+t}$}
	\loigiai{
		Véc-tơ chỉ phương $\overrightarrow{a}=(4 ;-6 ; 2)=2(2 ;-3 ; 1)$ nên đường thẳng $\Delta$ có phương trình tham số là $\heva{&x=2+2 t \\& y=-3 t \\& z=-1+t.}$
	}
\end{ex}

%G:\My Drive\CODE12-2024\DE-ON-THEO BAI\2H5-TACH DE\Bai2-De2.tex
\begin{ex}%[2H5N2-2]
	Trong không gian $Oxyz$, cho đường thẳng $d\colon \dfrac{x-1}{2}=\dfrac{y}{-1}=\dfrac{z-1}{-3}$. Một véc-tơ chỉ phương của đường thẳng $d$ là
	\choice
	{\True $\overrightarrow{u}_3=(2 ; -1 ; -3)$}
	{$\overrightarrow{u}_4=(-2 ;-1 ; 3)$}
	{$\overrightarrow{u}_1=(2 ;-1 ; 3)$}
	{ $\overrightarrow{u}_2=(1 ; 0 ; 1)$}
	\loigiai{
		Một véc-tơ chỉ phương của đường thẳng $d$ là $\overrightarrow{u}_3=(2 ;-1 ;-3)$.
	}
\end{ex}

%G:\My Drive\CODE12-2024\DE-ON-THEO BAI\2H5-TACH DE\Bai2-De2.tex
\begin{ex}%[2H5H2-3]
	Trong không gian với hệ trục tọa độ $O x y z$, cho mặt phẳng $(P)$ có phương trình là $2 x+y-5 z+6=0$ . Phương trình đường thẳng $d$ đi qua điểm $M(1 ;-2 ; 7)$ và  vuông góc với $(P)$ là
	\choice
	{$d\colon  \dfrac{x+1}{2}=\dfrac{y-2}{-1}=\dfrac{z+7}{-5}$}
	{$d\colon  \dfrac{x-1}{2}=\dfrac{y-2}{1}=\dfrac{z-7}{-5}$}
	{\True $d\colon  \dfrac{x-1}{2}=\dfrac{y+2}{1}=\dfrac{z-7}{-5}$}
	{$d\colon  \dfrac{x-2}{1}=\dfrac{y-1}{-2}=\dfrac{z+5}{7}$}
	\loigiai{
		Ta có $d$ vuông góc với $(P)$ nên có véc-tơ chỉ phương là $\overrightarrow{u}=(2 ; 1 ;-5)$.\\
		Kết hợp với $d$ đi qua điểm $M(1 ;-2 ; 7)$ nên $d\colon  \dfrac{x-1}{2}=\dfrac{y+2}{1}=\dfrac{z-7}{-5}$.
	}
\end{ex}

%G:\My Drive\CODE12-2024\DE-ON-THEO BAI\2H5-TACH DE\Bai2-De2.tex
\begin{ex}%[2H5H2-4]
	Trong không gian với hệ trục tọa độ $O x y z$, cho hai đường thẳng $d_1\colon \dfrac{x-1}{2}=\dfrac{y-2}{3}=\dfrac{z-3}{4}$ và $d_2\colon\heva{&x=1+t \\& y=2+2 t \\& z=3-2 t}$. Mệnh đề nào sau đây đúng?
	\choice
	{\True $d_1$ và $d_2$  vừa cắt nhau vừa vuông góc}
	{$d_1$ và $d_2$ không vuông góc và không cắt nhau}
	{$d_1$ và $d_2$ cắt nhau nhưng không vuông góc}
	{$d_1$ và $d_2$ vuông góc nhưng không cắt nhau}
	\loigiai{
		Chọn $M(1 ; 2 ; 3), $ $N(0 ; 0 ; 5)$ là hai điểm lần lượt thuộc đường thẳng $d_1$ và $d_2$.\\
		Ta có $\overrightarrow{u}_{d_1}=(2 ; 3 ; 4)$ và $\overrightarrow{u}_{d_2}=(1 ; 2 ;-2)$ nên $\overrightarrow{u}_{d_1} \cdot \overrightarrow{u}_{d_2}=0$ nên $d_1 \perp d_2$.\\
		Mặt khác, ta có $\left[\overrightarrow{u}_{d_1} ; \overrightarrow{u}_{d_2}\right] \overrightarrow{M N}=0$ nên $d_1$ cắt $d_2$.\\
		Vậy hai đường thẳng vừa vuông góc, vừa cắt nhau.
	}
\end{ex}

%G:\My Drive\CODE12-2024\DE-ON-THEO BAI\2H5-TACH DE\Bai2-De2.tex
\begin{ex}%[2H5N2-2]
	Trong không gian với hệ trục tọa độ $Oxyz$, véc-tơ nào dưới đây là véc-tơ chỉ phương của trục $Oz$?
	\choice
	{$\overrightarrow{m}=(1 ; 1 ; 1)$}
	{\True $\overrightarrow{k}=(0 ; 0 ; 1)$}
	{$\overrightarrow{i}=(1 ; 0 ; 0)$}
	{$\overrightarrow{j}=(0 ; 1 ; 0)$}
	\loigiai{
		Trục $O z$ có một vectơ chỉ phương là $\overrightarrow{k}=(0 ; 0 ; 1)$.}
\end{ex}

%G:\My Drive\CODE12-2024\DE-ON-THEO BAI\2H5-TACH DE\Bai2-De2.tex
\begin{ex}%[2H5N2-1]
	Đường thẳng $(\Delta)\colon \dfrac{x-1}{2}=\dfrac{y+2}{1}=\dfrac{z}{-1}$ đi qua điểm nào dưới đây?
	\choice
	{$Q(-1 ;-2 ; 0)$}
	{$N(-1 ; 2 ; 0)$}
	{$P(3 ; 1 ;-1)$}
	{\True $M(1 ;-2 ; 0)$}
	\loigiai{
		Ta có $\dfrac{1-1}{2}=\dfrac{2-2}{1}=\dfrac{0}{-1}$ nên điểm $M(1 ;-2 ; 0)$ thuộc đường thẳng $(\Delta)$.
	}
\end{ex}

%G:\My Drive\CODE12-2024\DE-ON-THEO BAI\2H5-TACH DE\Bai2-De2.tex
\begin{ex}%[2H5H2-7]
	Đường thẳng $d\colon \dfrac{x-1}{2}=\dfrac{y+1}{-1}=\dfrac{z+3}{-1}$ vuông góc với đường thẳng nào dưới đây?
	\choice
	{$d_1\colon \heva{&x=2-3 t \\& y=-2 t \\& z=1+5 t}$}
	{\True $d_4\colon \heva{&x=1-3 t \\& y=2-t \\& z=5-5 t}$}
	{$d_3\colon \heva{&x=2+3 t \\& y=3-t \\& z=5 t}$}
	{$d_2\colon \heva{&x=2 \\& y=3-3 t \\& z=1+t}$}
	\loigiai{
		Đường thẳng $d$ có véc-tơ chỉ phương $\overrightarrow{u}=(2 ;-1 ;-1)$.\\
		Các đường thẳng $d_1,$ $ d_2, $ $d_3,$ $ d_4$ lần lượt có véc-tơ chỉ phương là
		$\overrightarrow{u}_1=(-3 ;-2 ; 5),$ $ \overrightarrow{u}_2=(0 ;-3 ; 1),$ $ \overrightarrow{u}_3=(3 ;-1 ; 5)$ và $\overrightarrow{u}_4=(-3 ;-1 ;-5)$.\\
		Vì $\overrightarrow{u} \cdot \overrightarrow{u}_4=0$ nên $d \perp d_4$.
	}
\end{ex}

%G:\My Drive\CODE12-2024\DE-ON-THEO BAI\2H5-TACH DE\Bai2-De2.tex
\begin{ex}%[2H5H2-5]
	Trong không gian với hệ trục tọa độ $Oxyz$, cho đường thẳng $d$ có véc-tơ chỉ phương $\overrightarrow{u}$ và mặt phẳng $(P)$ có véc-tơ pháp tuyến $\overrightarrow{n}$. Mệnh đề nào dưới đây đúng?
	\choice
	{$d$ song song với $(P)$ thì $\overrightarrow{u}$ cùng phương với $\overrightarrow{n}$}
	{$\overrightarrow{u}$ vuông góc với $\overrightarrow{n}$ thì $d$ song song với $(P)$}
	{\True $\overrightarrow{u}$ không vuông góc với $\overrightarrow{n}$ thì $d$ cắt $(P)$}
	{$d$ vuông góc với $(P)$ thì $\overrightarrow{u}$ vuông góc với $\overrightarrow{n}$}
	\loigiai{
		Ta có $\overrightarrow{u}$ không vuông góc với $\overrightarrow{n}$ thì $d$ cắt $(P)$.}
\end{ex}

%G:\My Drive\CODE12-2024\DE-ON-THEO BAI\2H5-TACH DE\Bai2-De2.tex
\begin{ex}%[2H5H2-3]
	Cho đường thẳng $d$ có phương trình tham số $\heva{&x=1+2 t \\& y=2-t \\& z=-3+t}$. Viết phương trình chính tắc của đường thẳng $d$.
	\choice
	{$d\colon  \dfrac{x-1}{2}=\dfrac{y-2}{-1}=\dfrac{z-3}{1}$}
	{$d\colon  \dfrac{x+1}{2}=\dfrac{y+2}{-1}=\dfrac{z-3}{1}$}
	{$d\colon \dfrac{x-1}{2}=\dfrac{y-2}{1}=\dfrac{z+3}{1}$}
	{\True $d\colon  \dfrac{x-1}{2}=\dfrac{y-2}{-1}=\dfrac{z+3}{1}$}
	\loigiai{
		Từ phương trình tham số ta thấy đường thẳng $d$ đi qua điểm tọa độ $(1 ; 2 ;-3)$ và có véc-tơ chỉ phương  $\overrightarrow{u}=(2 ;-1 ; 1)$.\\
		Suy ra phương trình chính tắc của $d$ là $ \dfrac{x-1}{2}=\dfrac{y-2}{-1}=\dfrac{z+3}{1}$.
	}
\end{ex}

%G:\My Drive\CODE12-2024\DE-ON-THEO BAI\2H5-TACH DE\Bai2-De2.tex
\begin{ex}%[2H5V2-3]
	Trong không gian $O x y z$, cho đường thẳng $d\colon \dfrac{x+3}{2}=\dfrac{y+1}{1}=\dfrac{z}{-1}$ và mặt phẳng $(P)\colon x+y-3 z-2=0$. Gọi $d'$ là đường thẳng nằm trong mặt phẳng $(P)$, cắt và vuông góc với $d$. Đường thẳng $d'$ có phương trình là $\dfrac{x+1}{a}=\dfrac{y}{5}=\dfrac{z+1}{c}$. Tính $S=a-c$.
	\choice
	{$3$}
	{$-7 $}
	{\True $-3$}
	{$4$}
	\loigiai{
		Phương trình tham số của $d\colon\heva{&x=-3+2 t \\& y=-1+t \\& z=-t.}$\\
		Tọa độ giao điểm của $d$ và $(P)$ là nghiệm của hệ
		\allowdisplaybreaks
		\begin{eqnarray*}
			&&\heva{&x=-3+2 t \\& y=-1+t \\& z=-t \\& x+y-3 z-2=0} \Leftrightarrow \heva{&x=-3+2 t \\& y=-1+t \\& z=-t \\& -3+2 t-1+t+3 t-2=0} \\&\Leftrightarrow&\heva{&t=1 \\& x=-1 \\& y=0 \\& z=-1} \Rightarrow d \cap(P)=M(-1 ; 0 ;-1).
		\end{eqnarray*}
		Theo đề bài, đường thẳng $d$ có véc-tơ chỉ phương $u_d=(2;1;-1)$, mặt phẳng $(P)$ có véc-tơ pháp tuyến $n_{(P)}=(1;1;-3)$.\\
		Vì $d'$ nằm trong mặt phẳng $(P)$, cắt và vuông góc với $d$ nên $d'$ đi qua $M$ và có véc-tơ chỉ phương $\overrightarrow{u}_{d'}=\left[\overrightarrow{n}_{(P)}, \overrightarrow{u}_d\right]=(2 ;-5 ;-1)$ hay $d'$ nhận véc-tơ $\overrightarrow{v}=(-2 ; 5 ; 1)$ làm véc tơ chỉ phương.\\
		Phương trình của $d'\colon  \dfrac{x+1}{-2}=\dfrac{y}{5}=\dfrac{z+1}{1}$.\\
		Do đó $S=a-c=-2-1=-3$.
} \end{ex}

	\Closesolutionfile{ans}

\subsection*{\indam{PHẦN II. Câu trắc nghiệm đúng sai. Thí sinh trả lời từ câu 1 đến câu 4. Trong mỗi ý a), b), c), d) ở mỗi câu, thí sinh chọn đúng hoặc sai.}}
	\setcounter{ex}{0}
	\Opensolutionfile{ans}[ans/B2-De2-2]

\begin{ex}%[2H5H2-5]
	Trong không gian $O x y z$ cho đường thẳng $d$ có phương trình tham số $\heva{&x=-1+2 t \\& y=1+t \\& z=3-2 t.}$
	\choiceTF
	{Giao điểm của đt $d$ và mặt phẳng $(P)\colon x+2 y-3 z+2=0$  là $I(0 ; 1 ; 2)$}
	{Véc-tơ $\overrightarrow{a}=(4 ; 2 ;-3)$ là một véc-tơ chỉ phương   của đường thẳng $d$}
	{\True  Đường thẳng $d$ đi qua điểm $A(-1 ; 1 ; 3)$}
	{\True Phương trình chính tắc của đường thẳng $d$ là $\dfrac{x+1}{2}=\dfrac{y-1}{1}=\dfrac{z-3}{-2}$}
	\loigiai{
		\begin{itemchoice}
			\itemch \textbf{Sai.} Vì  ta có $-1+2 t+2(1+t)-3(3-2 t)+2=0$ $\Leftrightarrow 10 t-6=0 \Leftrightarrow t=\dfrac{3}{5}$.\\
			Giao điểm của $d$ và $(P)$ là $B\left(\dfrac{1}{5} ; \dfrac{8}{5} ; \dfrac{9}{5}\right)$.\itemch \textbf{Sai.} Vì véc-tơ chỉ phương của đường thẳng $d$ là $\overrightarrow{u}_d=(2 ; 1 ;-2)$. \\Xét hai véc-tơ $\overrightarrow{a}=(4 ; 2 ;-3)$ và $\overrightarrow{u}_d=(2 ; 1 ;-2)$.\\
			Vì $\dfrac{1}{2} \neq \dfrac{-2}{-3}$ nên $\overrightarrow{a}=(4 ; 2 ;-3)$ và $\overrightarrow{u}_d$ không cùng phương. \\Do đó $\overrightarrow{a}$ không là véc-tơ chỉ phương của đường thẳng $d$.
			\itemch \textbf{Đúng.} Vì theo phương trình tham số của đường thẳng $d$ thì $d$ đi qua điểm $A(-1 ; 1 ; 3)$.
			\itemch \textbf{Đúng.} Vì  đường thẳng $d$ đi qua điểm $M(-1 ; 1 ; 3)$ và có một  véc-tơ chỉ phương   là $\overrightarrow{u}_d=(2 ; 1 ;-2)$ nên có phương trình chính tắc là $\dfrac{x+1}{2}=\dfrac{y-1}{1}=\dfrac{z-3}{-2}$.
		\end{itemchoice}
	}
\end{ex}

%G:\My Drive\CODE12-2024\DE-ON-THEO BAI\2H5-TACH DE\Bai2-De2.tex
\begin{ex}%[2H5H2-7]
	Trong không gian $O x y z$ cho đường thẳng $d\colon \dfrac{x-1}{2}=\dfrac{y+1}{-1}=\dfrac{z}{1}$ và mặt phẳng $(P)\colon x+y+2 z-3=0$.
	\choiceTF
	{\True Đường thẳng $d'$ đi qua điểm $A(1 ; 0 ;-1)$ và vuông góc với mặt phẳng $(P)$. Phương trình tham số của đường thẳng $d'$ là $\heva{&x=1+t \\& y=t \\& z=-1+2 t}$}
	{ Đường thẳng $d$ có một véc-tơ chỉ phương là $\overrightarrow{a}=(1 ;-1 ; 0)$}
	{ Đường thẳng $d$ đi qua điểm $M(2 ;-1 ; 1)$}
	{\True Góc giữa đường thẳng $d$ và mặt phẳng $(P)$ bằng $30^{\circ}$}
	\loigiai{
		\begin{itemchoice}
			\itemch \textbf{Đúng.} Vì đường thẳng $d'$ vuông góc với mặt phẳng $(P)$ nên có một véc-tơ chỉ phương là $\overrightarrow{u}_{d'}=\overrightarrow{n}_P=(1 ; 1 ; 2)$.\\
			Kết hợp với $d'$
			đi qua $A(1;0;-1)$ nên phương trình tham số của đường thẳng $d'$ là $\heva{&x=1+t \\& y=t \\& z=-1+2 t.}$
			\itemch \textbf{Sai.} Vì đường thẳng $d$ có một véc-tơ chỉ phương là $\overrightarrow{u}=(2 ;-1 ; 1)$. Mà $\overrightarrow{a},$ $ \overrightarrow{u}$ không cùng phương nên $\overrightarrow{a}$ không là véc-tơ chỉ phương  của $d$.
			\itemch \textbf{Sai.} Vì lấy $M\in d \Rightarrow M(1+2t; -1- t;  t).$ Không có giá trị nào của $t$ thỏa hệ $\heva{&1+2t=2 \\& -1- t=-1 \\&  t=1}$ nên đường thẳng $d$ không đi qua điểm $M(2 ;-1 ; 1)$.
			\itemch \textbf{Đúng.} Vì gọi $\alpha$ là góc giữa đường thẳng $d$ và mặt phẳng $(P)$. Ta có
			\allowdisplaybreaks
			\begin{eqnarray*}
				\sin \alpha&=&\bigg|\cos \left(\overrightarrow{u}_d, \overrightarrow{n}_P\right)\bigg|=\dfrac{\left|\overrightarrow{u}_d \cdot \overrightarrow{n}_P\right|}{\left|\overrightarrow{u}_d\right|\left|\overrightarrow{n}_P\right|}\\&=&\dfrac{|2 \cdot 1+(-1) \cdot 1+1 \cdot 2|}{\sqrt{2^{2}+(-1)^{2}+1^{2}} \cdot \sqrt{1^{2}+1^{2}+2^{2}}}=\dfrac{1}{2} \Rightarrow \alpha=30^{\circ}.
			\end{eqnarray*}
		\end{itemchoice}
	}
\end{ex}

%G:\My Drive\CODE12-2024\DE-ON-THEO BAI\2H5-TACH DE\Bai2-De2.tex
\begin{ex}%[2H5H2-6] 
	Trong không gian $Oxyz$ cho đường thẳng $d$ có phương trình tham số $\heva{&x=2+t \\& y=3-2 t \\& z=3 t.}$
	\choiceTF
	{\True Đường thẳng $d$ có một véc-tơ chỉ phương là $\overrightarrow{u}=(1 ;-2 ; 3)$}
	{ Đường thẳng $d$ đi qua điểm $M(1 ;-2 ; 3)$}
	{\True  Đường thẳng $d'$ đi qua điểm $A(1 ; 2 ;-2)$ và song song với đường thẳng $d$. Phương trình tham số của đường thẳng $d'$ là $\heva{&x=1+t \\& y=2-2 t \\& z=-2+3 t}$}
	{ Khoảng cách từ điểm $B(0 ; 1 ; 2)$ đến đường thẳng $d$ bằng $3$}
	\loigiai{
		\begin{itemchoice}
			\itemch \textbf{Đúng.} Vì  đường thẳng $d$ có một véc-tơ chỉ phương là $\overrightarrow{u}=(1 ;-2 ; 3)$.
			\itemch \textbf{Sai.} Vì lấy $M\in d \Rightarrow M(2+t; 3-2 t; 3 t).$ Không có giá trị nào của $t$ thỏa hệ $\heva{&2+t=1 \\& 3-2 t=-2 \\& 3 t=3}$ nên đường thẳng $d$ không đi qua điểm $M(1 ;-2 ; 3)$.
			\itemch \textbf{Đúng.} Vì $d'\parallel d$ nên $d'$ có một véc-tơ chỉ phương  là $\overrightarrow{u}_{d'}=\overrightarrow{u}_d=(1 ;-2 ; 3)$. Vậy đường thẳng $d'$ có phương trình tham số là $\heva{&x=1+t \\& y=2-2 t \\& z=-2+3 t.}$
			\itemch \textbf{Sai.} Vì lấy điểm $C (3 ; 1 ; 3)\in d,$ ta có $\overrightarrow{u}_d=(1 ;-2 ; 3),$ $ \overrightarrow{B C}=(3 ; 0 ; 1)$.\\
			Khoảng cách từ điểm $B$ đến đường thẳng $d$ là $h=\dfrac{\left[\overrightarrow{B C} , \overrightarrow{u}_d\right]}{\left|\overrightarrow{u}_d\right|}=\dfrac{\sqrt{364}}{7}$.\end{itemchoice}
	}
\end{ex}

%G:\My Drive\CODE12-2024\DE-ON-THEO BAI\2H5-TACH DE\Bai2-De2.tex
\begin{ex}%[2H5V2-5] 
	Trong không gian $O x y z$ cho đường thẳng $d$ có phương trình tham số $\heva{&x=2+2 t \\& y=1-t \\& z=1+2 t.}$
	\choiceTF
	{ Đường thẳng $d$ có một  véc-tơ chỉ phương là $\overrightarrow{a}=(2 ; 1 ; 1)$}
	{\True Điểm $B(4 ; 0 ; 3)$ thuộc đường thẳng $d$}
	{ Khoảng cách giữa đường thẳng $d$ và mặt phẳng $(P)\colon x+2 y-3=0$ bằng $1$}
	{\True  Đường thẳng $d$ và đường thẳng $d'\colon \dfrac{x}{4}=\dfrac{y-2}{-2}=\dfrac{z+1}{4}$ trùng nhau}
	\loigiai{
		\begin{itemchoice}
			\itemch \textbf{Sai.} Vì  véc-tơ chỉ phương của đường thẳng $d$ là $\overrightarrow{u}_d=(2 ;-1 ; 2)$.\\ Xét hai  véc-tơ $\overrightarrow{a}=(2 ; 1 ; 1)$ và $\overrightarrow{u}_d=(2 ;-1 ; 2)$.\\
			Vì $\dfrac{2}{2} \neq \dfrac{1}{-1}$ nên $\overrightarrow{a}=(2 ; 1 ; 1)$ và $\overrightarrow{u}_d$ không cùng phương. \\Do đó $\overrightarrow{a}$ không là  véc-tơ chỉ phương của đường thẳng $d$.
			\itemch \textbf{Đúng.} Vì thế tọa độ điểm $B(4 ; 0 ; 3)$ vào phương trình của đường thẳng $d$, ta có $\heva{&4=2+2 t \\& 0=1-t \\& 3=1+2 t}\Leftrightarrow t=1$ nên điểm $B(4 ; 0 ; 3)$ thuộc đường thẳng $d$. 
			\itemch \textbf{Sai.} Vì  ta có $\overrightarrow{u}_d=(2 ;-1 ; 2),$ $\overrightarrow{n}_{(P)}=(1 ; 2 ; 0)$ thỏa mãn $\overrightarrow{u}_d \cdot \overrightarrow{n}_{(P)}=0$. \\Suy ra $\overrightarrow{u}_d \perp \overrightarrow{n}_{(P)} \Rightarrow \hoac{&d \parallel (P) \\& d \subset(P).}$\\
			Mặt khác điểm $A(2 ; 1 ; 1) \in d$ nhưng $A \notin(P)$ nên $d \parallel(P)$.\\Theo câu \textbf{b)}, ta có $$B(4 ; 0 ; 3)\in d\Rightarrow \mathrm{d}(d,(P))=\mathrm{d}(B,(P))=\dfrac{4+2 \cdot 0-3}{\sqrt{1^2+2^2}}=\dfrac{\sqrt{5}}{5}.$$ 
			\itemch \textbf{Đúng.} Vì  ta có $\overrightarrow{u}_d=(2 ;-1 ; 2),$ $ \overrightarrow{u}_d'=(4 ;-2 ; 4)$ cùng phương.\hfill $(1)$\\
			Điểm $A(2 ; 1 ; 1) \in d$ và $A(2 ; 1 ; 1) \in d'$. \hfill $(2)$\\
			Từ $(1)$ và $(2)$ chứng tỏ $d,$ $ d'$ trùng nhau.
		\end{itemchoice}
	}
\end{ex}
\Closesolutionfile{ans}
\subsection*{\indam{PHẦN III. Câu trắc nghiệm trả lời ngắn. Thí sinh trả lời từ câu 1 đến câu 6 vào ô kết quả.}}
\setcounter{ex}{0}
\Opensolutionfile{ans}[ans/B2-De2-3]
\begin{ex}%[2H5V2-5] 
	Trong không gian $O x y z$, cho ba điểm $A(1 ;-2 ; 1),$ $ B(5 ; 0 ;-1),$ $ C(3 ; 1 ; 2)$ và mặt phẳng $(Q)\colon 3 x+y-z+3=0$. Gọi $M(a ; b ; c)$ là điểm thuộc $(Q)$ thỏa mãn $M A^2+M B^2+2 M C^2$ nhỏ nhất. Khi đó tổng $a+b+3 c$ bằng bao nhiêu?\\
	\shortans[oly]{$5$}
	\loigiai{
		Gọi $E$ là điểm thỏa mãn $\overrightarrow{E A}+\overrightarrow{E B}+2 \overrightarrow{E C}=\overrightarrow{0} \Rightarrow E(3 ; 0 ; 1)$.\\
		Ta có 
		\allowdisplaybreaks
		\begin{eqnarray*}
			S&=&M A^2+M B^2+2 M C^2=\overrightarrow{M A}^2+\overrightarrow{M B}^2+2 \overrightarrow{M C}^2\\&=&\left(\overrightarrow{M E}+\overrightarrow{E A}\right)^2+\left(\overrightarrow{M E}+\overrightarrow{E B}\right)^2+2\left(\overrightarrow{M E}+\overrightarrow{E C}\right)^2\\&=&4 M E^2+E A^2+E B^2+2 E C^2.
		\end{eqnarray*}
		Vì $E A^2+E B^2+2 E C^2$ không đổi nên $S$ nhỏ nhất khi và chỉ khi $M E$ nhỏ nhất.\\
		Suy ra $ M$ là hình chiếu vuông góc của $E$ lên $(Q)$. \\Do đó $ME\perp (Q)$ nên $\overrightarrow{u}_{ME}=\overrightarrow{n}_{(Q)}=(3;1;-1)$, và $E(3 ; 0 ; 1)$.\\
		Suy ra phương trình đường thẳng $M E\colon \heva{&x=3+3 t \\& y=t \\& z=1-t.}$\\
		Tọa độ điểm $M$ là nghiệm của hệ phương trình $\heva{&x=3+3 t \\& y=t \\& z=1-t \\& 3 x+y-z+3=0} \Leftrightarrow\heva{&x=0 \\& y=-1 \\& z=2 \\& t=-1.}$\\
		Vậy $M(0 ;-1 ; 2) \Rightarrow a=0,$ $ b=-1,$ $ c=2 \Rightarrow a+b+3 c=5$.
	}
\end{ex}

\begin{ex}%[2H5H2-8] 
	Trong không gian $O x y z$, một viên đạn được bắn ra từ điểm $A(3 ; 4 ; 2)$ và trong $4$ giây đầu đạn đi với vận tốc không đổi, véc-tơ vận tốc (trên giây) là $\overrightarrow{v}=(4 ; 5 ; 1)$. Biết viên đạn trúng mục tiêu tại điểm $M(13 ; b ; c)$, tính $b+2 c$.\\
	\shortans[oly]{$25{,}5$}
	\loigiai{
		Phương trình đường đi của viên đạn $\heva{&x=3+4 t \\& y=4+5 t \\& z=2+t}$ với $0 \leq t \leq 4$.\\
		Viên đạn trúng mục tiêu tại điểm $M(13 ; b ; c)$ khi $M$ nằm trên đường đi của viên đạn
		$$
		\Rightarrow\heva{& 1 3 = 3 + 4 t \\&
			b = 4 + 5 t  \\&
			c = 2 + t } \Leftrightarrow \heva{&
			t=\dfrac{5}{2} \\&
			b=\dfrac{33}{2} \\&
			c=\dfrac{9}{2}} \Rightarrow b+2 c=\dfrac{33}{2}+9=\dfrac{51}{2}=25{,}5.
		$$}
\end{ex}

\begin{ex}%[2H5V2-4]
	Trong không gian với hệ trục tọa độ $O x y z$, cho điểm $M(3 ; 3 ;-2)$ và hai đường thẳng $d_1\colon \dfrac{x-1}{1}=\dfrac{y-2}{3}=\dfrac{z}{1} ;$ $ d_2\colon \dfrac{x+1}{-1}=\dfrac{y-1}{2}=\dfrac{z-2}{4}$. Đường thẳng $d$ đi qua $M$ cắt $d_1,$ $ d_2$ lần lượt tại $A$ và $B$. Khi đó độ dài đoạn thẳng $A B$ bằng bao nhiêu?\\
	\shortans[oly]{$3$}
	\loigiai{
		Ta có
		\begin{itemize}
			\item Phương trình tham số của $d_1\colon \heva{&x=1+t_1 \\& y=2+3 t_1 \\& z=t_1} ;$ $ t_1 \in \mathbb{R},$\\ Do $A \in d_1$ nên $ A\left(1+t_1 ; 2+3 t_1 ; t_1\right).$
			\item Phương trình tham số của $d_2\colon \heva{&x=-1-t_2 \\& y=1+2 t_2 \\& z=2+4 t_2} ;$ $t_2 \in \mathbb{R}.$ \\Do $ B \in d_2$ nên $ B\left(-1-t_2 ; 1+2 t_2 ; 2+4 t_2\right)$.
		\end{itemize}
		$\overrightarrow{M A}=\left(t_1-2 ; 3 t_1-1 ; t_1+2\right) ;$ $ \overrightarrow{M B}=\left(-4-4 t_2 ;-2+2 t_2 ; 4+4 t_2\right)$.\\
		Vì $A,$ $ B,$ $ M$ thẳng hàng nên 
		\allowdisplaybreaks
		\begin{eqnarray*}
			&&\overrightarrow{M A}=k \overrightarrow{M B}, k \in \mathbb{R}
			\\&\Leftrightarrow&\heva{&t_1-2=-4 k-k t_2 \\& 3 t_1-1=-2 k+2 k t_2 \\& t_1+2=4 k+4 k t_2} \Leftrightarrow\heva{&t_1+4 k+k t_2=2 \\& 3 t_1+2 k-2 k t_2=1 \\& t_1-4 k-4 k t_2=-2}\\& \Leftrightarrow&\heva{&t_1=0 \\& k=\dfrac{1}{2} \\& k t_2=0} \Leftrightarrow\heva{&t_1=0 \\& k=\dfrac{1}{2} \\& t_2=0.}
		\end{eqnarray*}
		Vậy $A(1 ; 2 ; 0)$ và $B(-1 ; 1 ; 2) \Rightarrow \overrightarrow{A B}=(-2 ;-1 ; 2)$.\\ Độ dài đoạn thẳng $A B=\left|\overrightarrow{A B}\right|=3$.
	}
\end{ex}

\begin{ex}%[2H5V2-8] 
	Hình vẽ dưới đây là hình ảnh Cầu Cổng Vàng (The Golden Gate Bridge) ở Mỹ. Xét hệ trục toạ độ $O x y z$ với $O$ là bệ của chân cột trụ tại mặt nước, trục $O z$ trùng với cột trụ, mặt phẳng $O x y$ là mặt nước và xem như trục $O y$ cùng phương với cầu như hình vẽ. Dây cáp $A D$ (xem như là một đoạn thẳng) đi qua đỉnh $D$ thuộc trục $O z$ và điểm $A$ thuộc mặt phẳng $O y z$, trong đó điểm $D$ là đỉnh cột trụ cách mặt nước $227$ m, điểm $A$ cách mặt nước $75$ m và cách trục $O z$ khoảng $343$ m. \begin{flushright}
		\textit{(Nguồn: https://www.goldengate.org/assets/1/6/ggb-exhibit-chapter-statistics.pdf)}
	\end{flushright}
	\begin{center}
		\includegraphics[scale=0.7]{image/Cau-cong-vang}
	\end{center}
	Giả sử ta dùng một đoạn dây nối điểm $N$ trên dây cáp $A D$ và điểm $M$ trên thành cầu, biết $M$ cách mặt nước $75$ m và $M N$ song song với cột trụ. Tính độ dài $M N$ (đơn vị mét) biết điểm $M$ cách trục $O z$ một khoảng bằng $230$ m (kết quả làm tròn đến hàng phần mười).\\
	\shortans[oly]{$50{,}1$}
	\loigiai{
		Chọn một đơn vị trên các trục bằng $1$ m.\\
		Ta có $D(0 ; 0 ; 227),$ $ A(0 ;-343 ; 75),$ $ M(0 ;-230 ; 75)$, $\overrightarrow{A D}=(0 ; 343 ; 152)$.\\ Phương trình đường thẳng $A D\colon \heva{&x=0 \\& y=343 t \\& z=227+152 t} \Rightarrow N(0 ; 343 t ; 227+152 t)$.\\
		Ta có $\overrightarrow{M N}=(0 ; 343 t+230 ; 152+152 t)$, $M N$ song song với trục $O z$, suy ra $$ 343 t+230=0 \Rightarrow t=-\dfrac{230}{343} \Rightarrow M N=152+152 \cdot \left(-\dfrac{230}{343}\right) \approx 50,1(m).$$
	}
\end{ex}

\begin{ex}%[2H5V2-5] 
	Trong không gian với hệ trục tọa độ $O x y z$, cho hai điểm $A(-2 ;-1 ; 2)$ và $B(5 ;-1 ; 1)$. Đường thẳng $d'$ là hình chiếu của đường thẳng $A B$ lên mặt phẳng $(P)\colon x+2 y+z+2=0$ có một véc-tơ chỉ phương $\overrightarrow{u}=(a ; b ; 2)$. Tính $S=a+b$.\\
	\shortans[oly]{$-4$}
	\loigiai{
		Gọi $(Q)$ là mặt phẳng chứa đường thẳng $A B$ và vuông góc $(P)$. 
		\\Khi đó, đường thẳng $d'=(P) \cap(Q)$.\\
		Có $\heva{&\overrightarrow{n}_{(Q)} \perp \overrightarrow{A B}=(7 ; 0 ;-1) \\& \overrightarrow{n}_{(Q)} \perp \overrightarrow{n}_{(P)}=(1 ; 2 ; 1)}$. Suy ra chọn $\overrightarrow{n}_{(Q)}=\left[\overrightarrow{A B} ; \overrightarrow{n}_{(P)}\right]=(2 ;-8 ; 14)$.\\
		Mặt khác $\heva{&\overrightarrow{u}_{\left(d'\right)} \perp \overrightarrow{n}_{(P)} \\& \overrightarrow{u}_{\left(d'\right)} \perp \overrightarrow{n}_{(Q)}} $. Suy ra chọn $\overrightarrow{u}_{\left(d'\right)}=\left[\overrightarrow{n}_{(P)} ; \overrightarrow{n}_{(Q)}\right]=(36 ;-12 ;-12)$ cùng phương với $\overrightarrow{u}(-6 ; 2 ; 2)$.\\
		Như vậy $a=-6,$ $ b=2 \Rightarrow a+b=-4$.
	}
\end{ex}

\begin{ex}%[2H5V2-4]
	Trong không gian $O x y z$, cho điểm $A(1 ; 0 ; 2)$ và đường thẳng $d\colon \dfrac{x-1}{1}=\dfrac{y}{1}=\dfrac{z+1}{2}$. Đường thẳng $\Delta$ đi qua $A$, vuông góc và cắt $d$ đi qua điểm $M(a ; b ; 0)$. Tính $\dfrac{a}{b}$.\\
	\shortans[oly]{$1{,}5$}
	\loigiai{
		Đường thẳng $d$ có véc-tơ chỉ phương là $\overrightarrow{u}_d=(1 ; 1 ; 2)$.\\
		Gọi giao điểm của đường thẳng $\Delta$ và $d$ là $B$.\\
		Vì $B\in d$ nên $B(1+t ; t ;-1+2 t)\Rightarrow \overrightarrow{A B}=(t ; t ;-3+2 t)$.\\
		Vì đường thẳng $\Delta$ vuông góc với đường thẳng $d$ nên $$\overrightarrow{A B} \perp \overrightarrow{u}_d \Leftrightarrow \overrightarrow{A B} \cdot \overrightarrow{u}_d=0 \Leftrightarrow 1 \cdot t+1  \cdot  t+2  \cdot (-3+2 t)=0 \Leftrightarrow t=1.$$
		Do đó $B(2 ; 1 ; 1),$ $ \overrightarrow{A B}=(1 ; 1 ;-1)$.\\
		Đường thẳng $\Delta$ đi qua điểm $A(1 ; 0 ; 2)$ và có vectơ chỉ phương là $\overrightarrow{A B}=(1 ; 1 ;-1)$ nên có phương trình tham số $\heva{&x=1+t \\& y=0+t  \\& z=2-t.}$\\ $M(a ; b ; 0) \in \Delta \Rightarrow M(3 ; 2 ; 0) \Rightarrow a=3 ;$ $ b=2 ;$ $ \dfrac{a}{b}=1{,}5.$
	}
\end{ex}
\centerline{---HẾT---}
\Closesolutionfile{ans}
%\newpage
%%=====================
%\begin{center}
%\textbf{\large BẢNG ĐÁP ÁN}
%\end{center}
%\noindent\textbf{ĐÁP ÁN PHẦN I}
%\inputansbox{10}{ans/B2-De2-1}
	
%\noindent\textbf{ĐÁP ÁN PHẦN II}
%\inputansbox[2]{2}{ans/B2-De2-2}
	
%\noindent\textbf{ĐÁP ÁN PHẦN III}
%\inputansbox[3]{6}{ans/B2-De2-3}




%%Bài 3.
\setcounter{dang}{0}
\newpage
\section{CÔNG THỨC TÍNH GÓC TRONG KHÔNG GIAN}
\subsection{LÝ THUYẾT CẦN NHỚ}
\subsubsection{Góc giữa hai mặt phẳng}
\begin{itemize}
	\item [\iconMT] \indam{Công thức:} Gọi $\vec{n_1}=(a_1;b_1;c_1)$, $\vec{n_2}=(a_2;b_2;c_2)$ lần lượt là vectơ pháp tuyến của $(P)$ và $(Q)$; $\varphi$ là góc giữa hai mặt phẳng $(P)$ và $(Q)$, với $0^\circ \leq \varphi \leq 90^\circ$.
	Khi đó
	\boxmini{$\cos \varphi =\bigg|\cos\left(\vec{n_1}, \vec{n_2}\right) \bigg| =\dfrac{\bigg|a_1a_2+b_1b_2+c_1c_2\bigg|}{\sqrt{a_1^2+b_1^2+c_1^2} \cdot \sqrt{a_2^2+b_2^2+c_2^2}}$}
	\item [\iconMT] \indam{Chú ý:}
	\begin{itemize}
		\item [$\bullet$] Nếu $(P)$ song song hoặc trùng $(Q)$ thì $\varphi =0^\circ$.
		\item [$\bullet$] Nếu $(P)\perp (Q)$ thì $\varphi =90^\circ$. Khi đó $\vec{n_1}\cdot \vec{n_2}=0 \Leftrightarrow a_1a_2+b_1b_2+c_1c_2=0$.
	\end{itemize}
\end{itemize}
\subsubsection{Góc giữa hai đường thẳng}
\begin{itemize}
	\item [\iconMT] \indam{Công thức:}  Gọi $\vec{u}=(u_1;u_2;u_3)$, $\vec{v}=(v_1;v_2;v_3)$ lần lượt là vectơ chỉ phương của  $d_1$ và $d_2$; $\varphi$ là góc giữa hai đường thẳng $d_1$ và $d_2$, với $0^\circ \leq \varphi \leq 90^\circ$.
	Khi đó
	\boxmini{$\cos \varphi =\bigg|\cos\left(\vec{u}, \vec{v}\right) \bigg| =\dfrac{\bigg|u_1v_1+u_2v_2+u_3v_3\bigg|}{\sqrt{u_1^2+u_2^2+u_3^2} \cdot \sqrt{v_1^2+v_2^2+v_3^2}}$}
	\item [\iconMT] \indam{Chú ý:}
	\begin{itemize}
		\item [$\bullet$] Nếu $d_1$ song song hoặc trùng $d_2$ thì $\varphi =0^\circ$.
		\item [$\bullet$] Nếu $d_1\perp d_2$ thì $\varphi =90^\circ$. Khi đó $\vec{u} \cdot\vec{u} =0 \Leftrightarrow u_1v_1+u_2v_2+u_3v_3=0$.
	\end{itemize}
\end{itemize}

\subsubsection{Góc giữa đường thẳng và mặt phẳng}
\begin{itemize}
	\item [\iconMT] \indam{Công thức:}  Gọi $\vec{u}=(u_1;u_2;u_3)$, $\vec{n}=(A;B;C)$ lần lượt là vectơ chỉ phương của  $d$ và vectơ pháp tuyến của $(P)$; $\varphi$ là góc giữa đường thẳng $d$ và mặt phẳng $(P)$, với $0^\circ \leq \varphi \leq 90^\circ$.
	Khi đó
	\boxmini{$\sin \varphi =\bigg|\cos\left(\vec{u}, \vec{n}\right) \bigg| =\dfrac{\bigg|u_1A+u_2B+u_3C\bigg|}{\sqrt{u_1^2+u_2^2+u_3^2} \cdot \sqrt{A^2+B^2+C^2}}$}
	\item [\iconMT] \indam{Chú ý:}
	\begin{itemize}
		\item [$\bullet$] Nếu $d$ song song hoặc trùng $(P)$ thì $\varphi =0^\circ$, khi đó $\vec{u} \perp \vec{n}$
		\item [$\bullet$] Nếu $d$ vuông góc với $(P)$ thì $\varphi =90^\circ$, khi đó $\vec{u} =k \cdot \vec{n}$.
	\end{itemize}
\end{itemize}
\subsection{PHÂN LOẠI, PHƯƠNG PHÁP GIẢI TOÁN}
\begin{dang}{Tính góc trong không gian Oxyz}
	\begin{itemize}
		\item [$\bullet$] Xác định vectơ chỉ phương (vectơ pháp tuyến);
		\item [$\bullet$] Áp dụng đúng công thức.
	\end{itemize}
\end{dang}
\boxmini{BÀI TẬP TỰ LUẬN}
\setcounter{vd}{0}
\begin{vd}
	Trong không gian $Oxyz$, tính góc giữa hai mặt phẳng sau:
	\begin{enumEX}[a)]{1}
		\item $(P)\colon x+y+4z-2=0$ và $(Q)\colon 2x-2z+7=0$.
		\item $(P)\colon 2x - y-2z-9=0$ và $(Q)\colon  x - y - 6=0$.
	\end{enumEX}
	\loigiai
	{
	\begin{enumEX}[a)]{1}
		\item Ta có $\vec{n}_P=(1;1;4)$, $\vec{n}_Q=(2;0;-2)$ lần lượt là vectơ pháp tuyến của $(P)$ và $(Q)$.\\
		Suy ra $\cos\left((P),(Q)\right)=|\cos\left(\vec{n}_P,\vec{n}_Q\right)|=\dfrac{|\vec{n}_P\cdot \vec{n}_Q|}{|\vec{n}_P|\cdot |\vec{n}_Q|}=\dfrac{|2+0-8|}{\sqrt{18}\cdot \sqrt{8}}=\dfrac{1}{2}$.\\
		Vậy góc giữa $(P)$ và $(Q)$ bằng $60^\circ$.
		\item $(P)\colon 2x - y-2z-9=0$ có $1$ vectơ pháp tuyến là $\overrightarrow{n_1} = (2;-1;-2)$.\\
		$(Q)\colon  x - y - 6=0$ có $1$ vectơ pháp tuyến là $\overrightarrow{n_2} = (1;-1;0)$.\\
		$\cos \left(\left( P \right);\left( Q \right)\right) = \dfrac{\left|\overrightarrow{n_1}, \overrightarrow{n_2}\right|}{\left|\overrightarrow{n_1}\right|\left|\overrightarrow{n_2}\right|} $
		$= \dfrac{\left|2\cdot 1 + \left(- 1\right)\left(- 1\right) + 0\right|}{\sqrt{2^2 + 1^2 + 2^2}\cdot \sqrt{1^2+ 1^2 + 0}}= \dfrac{1}{\sqrt{2}} \Rightarrow \left(\left( P \right);\left( Q \right)\right) = 45^\circ$.
	\end{enumEX}
	}
\end{vd}
\dongcham{10}
\begin{vd}%[Phan Quốc Trí]%[2H3B3-4]%
	Trong không gian $Oxyz$, tính góc giữa hai đường thẳng sau:
	\begin{enumEX}[a)]{1}
		\item $d:\heva{&x=1-t\\&y=t \\&z=0}$ và $d': \dfrac{x}{-2}=\dfrac{y}{1}=\dfrac{z-1}{-2}$.
		\item $d_1\colon \heva{&x=2+t\\&y=-1+t\\&z=3}$ và $ d_2\colon \heva{&x=1-t'\\&y=2\\&z=-2+t'}$. 
	\end{enumEX} 
	\loigiai{
		\begin{enumEX}[a)]{1}
			\item Gọi $\varphi$ là góc giữa hai đường thẳng $d$ và $d'$. Ta có
			$$\cos \varphi = \dfrac{\left| (-1)\cdot (-2) + 1 \cdot 1+0\cdot (-2) \right|}{\sqrt{(-1)^2+1^2+0^2}\cdot \sqrt{(-2)^2+1^2+(-2)^2}} = \dfrac{\sqrt{2}}{2} \Rightarrow \varphi = 45^{\circ}.$$
			\item $d_1$ có VTCP $\overrightarrow{v}_1=(1;1;0)$ và $d_2$ có VTCP $\overrightarrow{v}_2=(-1;0;1)$, $\left|v_1 \right|=\sqrt{2} $, $\left|v_2 \right|=\sqrt{2}$.\\
			Khi đó góc giữa hai đường thẳng $d_1$ và $d_2$ là
			\[\cos \left(d_1, d_2 \right)=\dfrac{\left|\overrightarrow{v}_1\cdot \overrightarrow{v}_2 \right|}{\left|\overrightarrow{v}_1 \right|\cdot\left|\overrightarrow{v}_2 \right|}=\dfrac{1}{2}. \]
			Vậy góc giữa hai đường thẳng $d_1$ và $d_2$ là $60^\circ$.
		\end{enumEX}
	}
\end{vd}
\dongcham{10}
\begin{vd}%[2H3B3-4]%
	Trong không gian $Oxyz$, tính góc giữa đường thẳng và mặt phẳng sau:
	\begin{enumEX}[a)]{1}
		\item $d\colon\dfrac{x-1}{1}=\dfrac{y}{2}=\dfrac{z+1}{-1}$ và $(P)\colon x-y+2z+1=0$.
		\item $d\colon \dfrac{x-1}{4}=\dfrac{y-6}{3}=\dfrac{z+4}{1}$ và $(P)\colon 4x+3y-z+1=0$
	\end{enumEX}
	\loigiai{
		\begin{enumEX}[a)]{1}
			\item Ta có $\overrightarrow{n}_{(P)}=(1;-1; 2)$ và $\overrightarrow{u}_d=(1; 2;-1)$.\\
			Vậy $\sin\left(d;(P)\right)=\dfrac{\left|\overrightarrow{n}_{(P)}\cdot\overrightarrow{u}_d\right|}{\left|\overrightarrow{n}_{(P)}\right|\cdot\left|\overrightarrow{u}_d\right|} =\dfrac{|1-2-2|}{\sqrt{6}\cdot\sqrt{6}}=\dfrac{1}{2}\Rightarrow \left(d;(P)\right)=30^{\circ}$.
			\item Mặt phẳng $(P)$ có vectơ pháp tuyến $\overrightarrow{n}=(4;3;-1)$.\\
			Đường thẳng $d$ có vectơ chỉ phương $\overrightarrow{u}=(4;3;1)$.\\
			Gọi $\alpha$ là góc giữa $d$ và $(P)$, ta có $\sin\alpha=\dfrac{\left|\overrightarrow{n}\cdot \overrightarrow{u}\right|}{\left|\overrightarrow{n}\right|\cdot \left|\overrightarrow{u}\right|}=\dfrac{|16+9-1|}{\sqrt{16+9+1}\cdot \sqrt{16+9+1}}=\dfrac{12}{13} \Rightarrow \alpha \approx 67,38^\circ$.
		\end{enumEX}
		}
\end{vd}
\dongcham{17}
\boxmini{BÀI TẬP TRẮC NGHIỆM}
\setcounter{ex}{0}
\Opensolutionfile{ans}[ans/2H5-B3-d1]
\begin{ex}
	Cho mặt phẳng $(P):x+2y-2z+3=0$, mặt phẳng $(Q):x-3y+5z-2=0$. Cosin của góc giữa hai mặt phẳng $(P)$, $(Q)$ là
	\choice
	{$-\dfrac{\sqrt{35}}{7}$}
	{$\dfrac{5}{7}$}
	{\True $\dfrac{\sqrt{35}}{7}$}
	{$-\dfrac{5}{7}$}
	\loigiai{
		Mặt phẳng $(P)$ có ve-tơ pháp tuyến là $\overrightarrow{n}_1=(1;2;-2)$\\
		Mặt phẳng $(Q)$ có vec-tơ pháp tuyến là $\overrightarrow{n}_2=(1;-3;5)$.\\
		Ta có $\cos[(P),(Q)]=\left|\cos(\overrightarrow{n}_1,\overrightarrow{n}_2)\right|=\dfrac{|\overrightarrow{n}_1\cdot\overrightarrow{n}_2|}{|\overrightarrow{n}_1|\cdot|\overrightarrow{n}_2|}=\left|\dfrac{-15}{3\sqrt{35}}\right|=\dfrac{\sqrt{35}}{7}$.}
\end{ex}

\begin{ex}
	Góc giữa hai mặt phẳng $(P):x+2y+z+4=0$ và $(Q):-x+y+2z+3=0$ bằng
	\choice
	{$45^{\circ}$}
	{$90^{\circ}$}
	{$30^{\circ}$}
	{\True $60^{\circ}$}
	\loigiai{
		Gọi $\varphi$ là góc giữa $(P)$ và $(Q)$. Ta có
		$$\cos \varphi = \dfrac{\left| 1\cdot (-1)+2\cdot 1+1\cdot 2 \right|}{\sqrt{1^2+2^2+1^2} \cdot \sqrt{(-1)^2 +1^2+2^2}}=\dfrac{1}{2} \Rightarrow \varphi = 60^{\circ}.$$
	}
\end{ex}

\begin{ex}
	Tính góc $\alpha$ giữa mặt $(P)\colon x+z-4=0$ và mặt phẳng $(Oxy)$.
	\choice
	{\True $45^\circ$}
	{$30^\circ$}
	{$90^\circ$}
	{$60^\circ$}
	\loigiai{
		Ta có $\vec{n}_{(P)}=(1;0;1)$, $\vec{n}_{(Oxy)}=(0;0;1)$.\\
		Suy ra $\cos \alpha =\dfrac{\left| 1\cdot 0+0\cdot 0+1\cdot 1 \right|}{\sqrt{2}\cdot \sqrt{1}}=\dfrac{\sqrt{2}}{2}$.\\
		Vậy $\widehat{((P);(Q))}=45^\circ$.
	}
\end{ex}

\begin{ex}
	Cho điểm $H\left(2;1;2 \right)$, điểm $H$ là hình chiếu vuông góc của gốc tọa độ $O$ xuống mặt phẳng $\left(P \right)$, số đo góc giữa mặt phẳng $\left(P \right)$ và mặt phẳng $\left(Q \right):x+y-11=0$ là
	\choice
	{\True $45^\circ$}
	{$30^\circ$}
	{$60^\circ$}
	{$90^\circ$}
	\loigiai
	{
		Vì điểm $H$ là hình chiếu vuông góc của gốc tọa độ $O$ xuống mặt phẳng $(P)$ nên ta chọn
		$\vec{OH}=\vec{n}_{(P)}=(2;1;2)$.\\
		Phương trình mặt phẳng $(P)$ có dạng
		$$2\left(x-2\right)+\left(y-1\right)+2\left(z-2\right)=0\Leftrightarrow 2x+y+2z-9=0.$$
		Do đó, góc giữa 2 mặt phẳng $(P),(Q)$ tính như sau
		$$\cos \left((P),(Q)\right)=\dfrac{\left|{\vec{n}_{(P)}\cdot\vec{n}_{(Q)}}\right|}{\left|{\vec{n}_{(P)}}\right|\left|{\vec{n}_Q}\right|}=\dfrac{|2\cdot1+1\cdot1+2\cdot0|}{\sqrt{9}\cdot\sqrt{2}}=\dfrac{3}{3\sqrt{2}}=\dfrac{\sqrt{2}}{2}.$$
		Do đó góc giữa mặt phẳng $(P)$ và mặt phẳng $(Q)$ bằng $\cos 45^\circ=\dfrac{\sqrt{2}}{2}$.
	}
\end{ex}


\begin{ex}
	Cho hai đường thẳng $d_1\colon \dfrac{x}{-1}=\dfrac{y+1}{2}=\dfrac{z}{2}$, $d_2\colon\heva{& x=2t \\ & y=1\\ & z=1-t}$. Gọi $\varphi$ là góc
	giữa hai đường thẳng $d_1$, $d_2$. Tính $\cos\varphi$.
	\choice
	{$\cos \varphi=-\dfrac{4\sqrt{5}}{15}$}
	{\True $\cos \varphi=\dfrac{4\sqrt{5}}{15}$}
	{$\cos \varphi=\dfrac{\sqrt{6}}{9}$}
	{$\cos \varphi=-\dfrac{\sqrt{6}}{9}$}
	\loigiai
	{Đường thẳng $d_1$, $d_2$ lần lượt có vectơ chỉ phương là $\overrightarrow{u}_1=(-1; 2; 2)$ và $\overrightarrow{u}_2=(2; 0; -1)$.\\
		Vậy $\cos\varphi=\left|\cos\left(\overrightarrow{u}_1, \overrightarrow{u}_2\right)\right|=\dfrac{|(-1)\times 2 +2\times 0+ 2\times(-1)|}{\sqrt{(-1)^2+2^2+2^2}\cdot\sqrt{2^2+0^2+(-1)^2}}=\dfrac{|-4|}{3\sqrt{5}}=\dfrac{4\sqrt{5}}{15}$.
	}
\end{ex}

\begin{ex}%[2H3B3-4]%
	Cho đường thẳng $d_1\colon \dfrac{x}{-1}=\dfrac{y+1}{1}=\dfrac{z-1}{-2}$ và $d_2\colon \dfrac{x+1}{-1}=\dfrac{y}{1}=\dfrac{z-3}{1}$. Góc giữa hai đường thẳng bằng
	\choice
	{\True $90^\circ$}
	{$30^\circ$}
	{$60^\circ$}
	{$45^\circ$}
	\loigiai{
		Đường thẳng $d_1$ có VTCP là $\vec{a}=(-1;1;-2)$, đường thẳng $d_2$ có VTCP là $\vec{b}=(-1;1;1)$.\\
		Ta có $\vec{a}\cdot \vec{b}= 0\Rightarrow d_1\perp d_2\Rightarrow \left( d_1, d_2\right) =90^\circ$.
	}
\end{ex}

\begin{ex}
	Cho đường thẳng $d$ là giao tuyến của hai mặt phẳng\\ $(P) \colon x-z \cdot \sin \alpha +\cos \alpha =0$ và $(Q) \colon y-z \cdot \cos \alpha -\sin \alpha =0$, $\alpha \in \left(0;\dfrac{\pi}{2}\right)$. Góc giữa $d$ và trục $Oz$ là
	\choice
	{$90^{\circ}$}
	{$30^{\circ}$}
	{\True $45^{\circ}$}
	{$60^{\circ}$}
	\loigiai{
		Xét hệ phương trình $\left\{\begin{aligned}
			&x-z \cdot \sin \alpha +\cos \alpha =0\\
			&y-z \cdot \cos \alpha -\sin \alpha =0\\
		\end{aligned}\right. $. \\
		Ta có $\vec{n}_P=\left(1;0;-\sin \alpha \right)$ và $\vec{n}_Q=\left(0;1;-\cos \alpha \right)$.\\
		vectơ chỉ phương của $d$ là $\vec{u}_d=\left[\vec{n}_P;\vec{n}_Q \right]=\left(\sin \alpha;\cos \alpha;1\right)$.\\
		vectơ chỉ phương của trục $Oz$ là $\vec{k}=(0;0;1)$.\\
		Gọi $\varphi $ là góc giữa đường thẳng $d$ và trục $Oz$.\\
		Ta có $\cos \varphi =\dfrac{\left|\vec{u}_d \cdot \vec{k}\right|}{\left|\vec{u}_d\right| \cdot \left|\vec{k}\right|}=\dfrac{1}{\sqrt{2}}$. Suy ra $\varphi =45^{\circ}$.}
\end{ex}

\begin{ex}%[2-TT-5- Đề thi tháng 2-2019, Toán 12 trường THPT chuyên Bắc Giang- 2019]%[Nguyễn Thế Anh-EX6]%[2H3B3-4]%
	Cho đường thẳng $d\colon \heva{& x=1-t \\ & y=2+2t \\ & z=3+t}$ và mặt phẳng $(P)\colon x-y+3=0$. Tính số đo góc giữa đường thẳng $d$ và mặt phẳng $(P)$.
	\choice
	{$45^\circ$}
	{$120^\circ$}
	{\True $60^\circ$}
	{$30^\circ$}
	\loigiai
	{
		Ta có $\vec {u}=(-1;2;1)$ là vectơ chỉ phương của đường thẳng $d$ và $\vec{n}=(1;-1;0)$ là vectơ pháp tuyến của mặt phẳng $(P)$.\\
		Suy ra $\sin \left(d;(P)\right)=\left|\cos \left(\vec{u};\vec{n}\right)\right|=\dfrac{\left|\vec{u} \cdot \vec{n}\right|}{|\vec{u}|\cdot |\vec{n}|}=\dfrac{\sqrt{3}}{2}$. Vậy $(d;(P))=60^\circ$.
	}
\end{ex}

\begin{ex}
	Cho mặt phẳng $(P)\colon 3x+4y+5z-8=0$ và đường thẳng $d\colon\heva{&x=2-3t\\&y=-1-4t\\&z=5-5t}$. Góc giữa đường thẳng $d$ và mặt phẳng $(P)$ là
	\choice
	{\True $90^{\circ}$}
	{$45^{\circ}$}
	{$30^{\circ}$}
	{$60^{\circ}$}
	\loigiai{
		Mặt phẳng $(P)$ có một vectơ pháp tuyến là $\overrightarrow{n}=(3;4;5)$.\\
		Đường thẳng $d$ có một vectơ chỉ phương là $\overrightarrow{u}=(-3;-4;-5)$.\\
		Ta có $\overrightarrow{n}=-\overrightarrow{u}\Rightarrow d\perp(P)$ nên góc giữa đường thẳng $d$ và mặt phẳng $(P)$ là $90^{\circ}$.
	}
\end{ex}

\begin{ex}
	Cho mặt phẳng $(P)\colon x+y-\sqrt{2}z+5=0$. Tính góc $\varphi$ giữa mặt phẳng $(P)$ và trục $Oy$.
	\choice
	{$\varphi=60^{\circ}$}
	{$\varphi=45^{\circ}$}
	{$\varphi=90^{\circ}$}
	{\True $\varphi=30^{\circ}$}
	\loigiai{
		Ta có vectơ pháp tuyến của mặt phẳng $(P)$ là $\overrightarrow{n}=(1;1;-\sqrt{2})$ và vectơ chỉ phương của trục $Oy$ là $\overrightarrow{u}=(0;1;0)$. Suy ra $\sin\varphi=\dfrac{|\overrightarrow{u}\cdot \overrightarrow{n}|}{|\overrightarrow{u}|\cdot|\overrightarrow{n}|}=\dfrac{|1|}{\sqrt{4}\cdot 1}=\dfrac{1}{2}\Rightarrow\varphi=30^{\circ}$.
	}
\end{ex}


\begin{ex}
	Cho hai mặt phẳng $(P)\colon (m-1)x+y-2z+m=0$ và $(Q)\colon 2x-z+3=0.$ Tìm $m$ để $(P)$ vuông góc với $(Q).$
	\choice
	{\True $m=0$}
	{$m=\dfrac32$}
	{$m=5$}
	{$m=-1$}
	\loigiai{
		$(P)$ vuông góc với $(Q)$ khi và chỉ khi các vectơ pháp tuyến của chúng vuông góc với nhau, tức là $$(m-1;1;-2)\cdot(2;0;-1)=0\Leftrightarrow m=0.$$
	}
\end{ex}

\begin{ex}
	Cho mặt phẳng $(P) \colon x-3y+2z+1=0$ và $(Q) \colon (2m-1)x+m(1-2m)y+(2m-4)z+14=0$ với $m$ là tham số thực. Tổng các giá trị của $m$ để $(P)$ và $(Q)$ vuông góc nhau bằng
	\choice
	{$-\dfrac{3}{2}$}
	{\True $-\dfrac{1}{2}$}
	{$-\dfrac{5}{2}$}
	{$-\dfrac{7}{2}$}
	\loigiai{
		$(P)$ có vectơ pháp tuyến $\overrightarrow{n}_P =(1;-3;2)$.
		\\ $(Q)$ có vectơ pháp tuyến $\overrightarrow{n}_{Q} =\left( 2m-1;m(1-2m);2m-4 \right)$. \\
		$(P)$ và $(Q)$ vuông góc với nhau khi và chỉ khi $\overrightarrow{n}_P \perp \overrightarrow{n}_{Q}$.
		\\ Điều này tương đương với $\overrightarrow{n}_P \cdot \overrightarrow{n}_Q =0 \Leftrightarrow 6m^2+3m -9=0 \Leftrightarrow \hoac{&m=1 \\ &m=-\dfrac{3}{2}.}$
		\\ Tổng các giá trị của $m$ để $(P)$ và $(Q)$ vuông góc nhau bằng $1 -\dfrac{3}{2} = -\dfrac{1}{2}$.
	}
\end{ex}

\begin{ex}%[Trần Bình Thuận - DA2]%[2H3K2-5]% Câu 7
	Cho hai mặt phẳng $ (P)\colon x+2y-z+2=0 $ và $ (Q) \colon x-my+(m+1)z+m-2=0 $, với $ m $ là tham số. Gọi $ S $ là tập hơn tất cả các giá trị của $ m $ sao cho góc giữa $ (P) $ và $ (Q) $ bằng $ 60^{\circ} $. Tính tổng các phần tử của $ S $.
	\choice
	{$ 1 $}
	{$ -\dfrac{1 }{2} $}
	{$ \dfrac{3 }{2} $}
	{\True $ \dfrac{1 }{2} $}
	\loigiai{
		vectơ pháp tuyến của mặt phẳng $ (P): \vec{n}_{(P)} = (1;2;-1) $.\\
		vectơ pháp tuyến của mặt phẳng $ (Q): \vec{n}_{(Q)} = (1;-m;m+1) $.\\
		Góc giữa hai mặt phẳng $ (P) $ và $ (Q) $ bằng $ 60^{\circ} $ nên
		{\allowdisplaybreaks
			\begin{eqnarray*}
				&&\dfrac{|1-2m-m-1|}{\sqrt{6}\cdot\sqrt{2m^2+2m+2}}= \cos 60^{\circ}\\
				&\Leftrightarrow& |3m| = \dfrac{1}{2}\sqrt{6}\cdot\sqrt{2m^2+2m+2}\\
				&\Leftrightarrow& 9m^2 = 3(m^2+m+1)\\
				&\Leftrightarrow& 2m^2-m-1=0 \Leftrightarrow \hoac{&m=1\\&m=-\dfrac{1}{2}.}
		\end{eqnarray*}}
		Do đó $ S=1+\left( -\dfrac{1}{2} \right) =\dfrac{1}{2} $.
	}
\end{ex}

\begin{ex}
	Hãy tìm tham số thực $m$ để góc giữa hai đường thẳng  sau bằng $60^\circ$.
	$$d\colon \heva{&x=1+t\\&y=-\sqrt{2}t\\&z=1+t}\, ,t\in \mathbb{R} \text {và } d'\colon \heva{&x=1+t'\\&y=1+\sqrt{2}t'\\&z=1+mt'}\, , t'\in \mathbb{R}$$ 
	\choice
	{$\dfrac{1}{2}$}
	{$-1$}
	{\True $-\dfrac{1}{2}$}
	{$1$}
	\loigiai{
		Ta có $\heva{&\vec{u}_d=(1;-\sqrt{2};1)\\&\vec{u}_{d'}=(1;\sqrt{2};m)}\Rightarrow	\cos\alpha = \dfrac{\left|-1+m\right| }{\sqrt{4}\cdot \sqrt{3+m^2}}$
		\begin{eqnarray*}
			\cos\alpha = \cos 60^\circ &\Leftrightarrow& \dfrac{\left|-1+m\right| }{\sqrt{4}\cdot \sqrt{3+m^2}}=\dfrac{1}{2}\\
			&\Leftrightarrow& |-1+m|=\sqrt{3+m^2}\\
			&\Leftrightarrow& -2m+2=0\\
			&\Leftrightarrow& m=1
		\end{eqnarray*}
	}
\end{ex}

\begin{ex}
	Cho các điểm $A(-1;\sqrt{3};0)$, $B(1;\sqrt{3};0)$, $C(0;0;\sqrt{3})$ và điểm $M$ thuộc trục $Oz$ sao cho hai mặt phẳng $(MAB)$ và $(ABC)$ vuông góc với nhau. Tính góc giữa hai mặt phẳng $(MAB)$ và $(OAB)$.
	\choice
	{\True $45^\circ$}
	{$60^\circ$}
	{$15^\circ$}
	{$30^\circ$}
	\loigiai{
		$M(0;0;m)$ thuộc trục $Oz$.\\
		Ta có $\vv{AM}=(1;-\sqrt{3};m)$, $\vv{AB}=(2;0;0)$, $\vv{AC}=(1;-\sqrt{3};\sqrt{3})$.\\
		$\Rightarrow \vv{n}_1 =\left[ \vv{AB}, \vv{AC} \right] =(0;-2\sqrt{3};-2\sqrt{3})$,
		$\vv{n}_2 =\left[ \vv{AB}, \vv{AM} \right] =(0;-2m;-2\sqrt{3})$.\\
		Mặt phẳng $(ABC)$ có một vectơ pháp tuyến là $\vv{n}_1$, mặt phẳng $(MAB)$ có một vectơ pháp tuyến là $\vv{n}_2$.\\
		Hai mặt phẳng $(MAB)$ và $(ABC)$ vuông góc với nhau khi và chỉ khi
		$$\vv{n}_1 \perp \vv{n}_2
		\Leftrightarrow 0\cdot 0  +(-2\sqrt{3})\cdot (-2m) + (-2\sqrt{3})\cdot (-2\sqrt{3}) =0
		\Leftrightarrow m=-\sqrt{3}.$$
		Mặt phẳng $(OAB)$ có một vectơ pháp tuyến là $\vv{n}_3 = \left[ \vv{OA}, \vv{OB} \right] =(0;0;-2\sqrt{3})$.\\
		Gọi $\varphi$ là góc giữa hai mặt phẳng $(MAB)$ và $(OAB)$. Khi đó
		$$\cos \varphi =\left| \cos \left( \vv{n}_2, \vv{n}_3 \right) \right|
		= \dfrac{\left| \vv{n}_2 \cdot \vv{n}_3 \right|}{\left| \vv{n}_2 \right| \cdot \left| \vv{n}_3 \right|}
		=\dfrac{12}{2\sqrt{6} \cdot 2\sqrt{3}} =\dfrac{1}{\sqrt{2}}.$$
		Vậy góc giữa hai mặt phẳng $(MAB)$ và $(OAB)$ là $45^\circ$.
	}
\end{ex}



\Closesolutionfile{ans}
\begin{dang}{Tọa độ hóa một số bài toán hình không gian}
\end{dang}

\boxmini{BÀI TẬP TỰ LUẬN}
\setcounter{vd}{0}

\begin{vd}%[2H5V1-6]
	Cho hình lăng trụ đứng $OBC.O'B'C'$ có đáy là tam giác $OBC$ vuông tại $O$ và có $OB=3a$, $OC=a$, $OO'=2a$. Tính góc giữa
	\begin{enumerate}
		\item hai đường thẳng $BO'$ và $B'C$;
		\item hai mặt phẳng $(O'BC)$ và $(OBC)$;
		\item đường thẳng $B'C$ và mặt phẳng $(O'BC)$.
	\end{enumerate}
	\loigiai
	{
		\immini{
			Xét hệ trục tọa độ $Oxyz$ sao cho các điểm có tọa độ như sau: $O(0;0;0)$, $O'(2a;0;0)$, $B(0;3a;0)$, $C(0;0;1a)$.\\
			Trong không gian $Oxyz$ vừa chọn, ta có $B'(2a;3a;0)$, $C'(2a;0;1a)$, $\overrightarrow{BO}'=(0;-3a;0)$, $\overrightarrow{CB}'=(2a;3a;-a)$.
			\begin{enumerate}
				\item Hai đường thẳng $BO'$ và $B'C$ có vectơ chỉ phương lần lượt là $\overrightarrow{u}=(0;3;0)$, $\overrightarrow{v}=(2;3;-1)$.\\
				Ta có 
				\begin{eqnarray*}
					\cos (BO',B'C)&=&\dfrac{\left |\overrightarrow{u}\cdot\overrightarrow{v} \right |}{\left |\overrightarrow{u} \right |\cdot\left |\overrightarrow{v} \right |}\\&=&\dfrac{\left |0\cdot 2+3\cdot 3+0\cdot (-1) \right |}{\sqrt{3^2}\cdot\sqrt{2^2+3^2+(-1)^2}}=\dfrac{3}{\sqrt{14}}.
				\end{eqnarray*}
				Suy ra $(BO',B'C)\approx 36^{\circ}42'$.
				\item Ta có phương trình mặt phẳng $(O'BC)$ theo đoạn chắn là $\dfrac{x}{2a}+\dfrac{y}{3a}+\dfrac{z}{a}=1$ hay $3x+2y+6z-6a=0$.
			\end{enumerate}
		}{
			\begin{tikzpicture}[line cap=round,line join=round,scale=.8,>=stealth,font=\footnotesize ]
				\coordinate (O) at (0,0);
				\coordinate (O') at (0,4.5);
				\coordinate (B) at (5.5,0);
				\coordinate (C) at (2,-2);
				\coordinate (C') at ($(C)+(0,4.5)$);
				\coordinate (B') at ($(B)+(0,4.5)$);
				\coordinate (y) at ($(O)!1.3!(B)$);
				\coordinate (x) at ($(O)!1.3!(O')$);%trung điểm
				\coordinate (z) at ($(O)!1.5!(C)$);%trung điểm
				\draw (B')--(C)--(O')--(B')--(C')--(C)--(B)--(B')(B)--(C')--(O')--(O)--(C)(O)--(C');
				\draw[dashed](O)--(B)--(O');
				\foreach \x/\g in {O'/180,B'/75,C'/180,O/180,B/-80,C/-100}
				\fill[black](\x) circle (1pt)
				($(\x)+(\g:3mm)$) node{\x};
				\draw[->](B)--(y)node[below]{$y$};
				\draw[->](O')--(x)node[above]{$x$};
				\draw[->](C)--(z)node[below right]{$z$};
			\end{tikzpicture}
		}
		Mặt phẳng $(O'BC)$ có vectơ pháp tuyến $\overrightarrow{n}=(3;2;6)$, mặt đáy $(OBC)$ có vectơ pháp tuyến $\overrightarrow{k}=(0;0;1)$. Gọi $\alpha$ là góc giữa mặt phẳng $(O'BC)$ và mặt đáy.\\
		Ta có $\cos\alpha =\dfrac{\left |\overrightarrow{n}\cdot\overrightarrow{k} \right |}{\left |\overrightarrow{n} \right |\cdot\left |\overrightarrow{k} \right |}=\dfrac{\left |3\cdot 0+2\cdot 0+6\cdot 1 \right |}{\sqrt{3^2+2^2+6^2}\cdot\sqrt{1^2}}=\dfrac{6}{7}$.\\
		Suy ra $\left ((O'BC),(OBC) \right )\approx 31^{\circ}1'$.
		\begin{enumerate}
			\setcounter{enumi}{2}
			\item Gọi $\beta$ là góc giữa đường thẳng $B'C$ và mặt phẳng $(O'BC)$.\\
			Ta có $\sin\beta =\dfrac{\left |\overrightarrow{v}\cdot\overrightarrow{n} \right |}{\left |\overrightarrow{v} \right |\cdot\left |\overrightarrow{n} \right |}=\dfrac{\left |2\cdot 3+3\cdot 2+(-1)\cdot 6 \right |}{\sqrt{2^2+3^2+(-1)^2}\cdot\sqrt{3^2+2^2+6^2}}=\dfrac{3\sqrt{14}}{49}$.\\
			Suy ra $(B'C,(O'BC))\approx 13^{\circ}15'$.
		\end{enumerate}
	}
\end{vd}
\dongcham{31}
\begin{vd}%[2H5H2-7]
	Cho hình chóp $S.ABCD$ có đáy $ABCD$ là hình vuông cạnh bằng $4$. Mặt bên $SAB$ là tam giác cân tại $S$ có chiều cao bằng $6$ và nằm trong mặt phẳng vuông góc với đáy.
	\begin{listEX}  
		\item Tính góc $\alpha$ giữa hai đường thẳng $SD$ và $BC$;
		\item Tính góc $\beta$ giữa hai mặt phẳng $(SAD)$ và $(SCD)$.
	\end{listEX}
	\loigiai{
		\immini{Gọi $O$ là trung điểm của $AB$ suy ra $SO\perp(ABCD)$.\\
			Chọn hệ trục $Oxyz$ như hình bên. Ta có: $S(0;0;6)$, $A(2; 0; 0)$, $B(-2; 0; 0)$, $C(-2; 4; 0)$, $D(2; 4; 0)$.}
		{\begin{tikzpicture}[scale=.7,>=stealth, font=\footnotesize, line join=round, line cap=round]
				\coordinate (O) at (0, 0);
				\coordinate (I) at (4, 0);
				\coordinate (S) at (0,6);
				\coordinate (A) at (-1, -1);
				\coordinate (B) at (1, 1);
				\coordinate (C) at (5, 1);
				\coordinate (D) at ($(A)+(C)-(B)$);
				%\draw[color=gray!50,dashed] (\xmin,\ymin) grid (\xmax,\ymax);
				\draw[dashed](A)node[below]{$A$}--(B)node[above right]{$B$}--(C)node[right]{$C$}(O)--(S)node[right]{$S$}--(B)(O)node[left]{$O$}--(I)node[below ]{$I$};
				\draw(S)node[left]{$6$}--(A)node[above left]{$2$}--(D)--(C)--(S)--(D)node[below]{$D$};
				\draw[->](S)--++(0,1)node[left]{$z$};
				\draw[->](I)node[above]{$4$}--++(1,0)node[above]{$y$};
				\draw[->](A)--++(-1,-1)node[left]{$x$};
		\end{tikzpicture}}
		\begin{listEX}
			\item Ta có $\overrightarrow{SD}=(2;4;-6)$, $\overrightarrow{BC}=(0; 4; 0)$, suy ra
			\begin{eqnarray*}
				&\cos \alpha & =\dfrac{|\overrightarrow{S D} \cdot \overrightarrow{B C}|}{|\overrightarrow{S D}| \cdot|\overrightarrow{B C}|}=\dfrac{|2 \cdot 0+4 \cdot 4-6 \cdot 0|}{\sqrt{2^2+4^2+(-6)^2} \cdot \sqrt{4^2}} \\
				&& =\dfrac{\sqrt{14}}{7} \Rightarrow \alpha=57{,}7^{\circ}.
			\end{eqnarray*}
			\item Mặt phẳng $(S A D)$ có cặp vectơ chỉ phương là $\overrightarrow{S D}=(2; 4;-6)$,\\        
			$\overrightarrow{S A}=(2; 0;-6)$ nên có vectơ pháp tuyến $\overrightarrow{n}=-\dfrac{1}{8}[\overrightarrow{S D}, \overrightarrow{SA}]=(3; 0; 1)$.\\     
			Mặt phẳng $(SCD)$ có cặp vectơ chỉ phương là $\overrightarrow{SD}=(2; 4;-6)$,   $\overrightarrow{D C}=(-4; 0; 0)$ nên có vectơ pháp tuyến $\overrightarrow{u}=\dfrac{1}{8}[\overrightarrow{S D}, \overrightarrow{D C}]=(0; 3; 2)$.\\
			Suy ra $\cos \beta=\dfrac{\left|\overrightarrow{n} \cdot \overrightarrow{u}\right|}{|\overrightarrow{n}| \cdot\left|\overrightarrow{u}\right|}=\dfrac{|3 \cdot 0+0\cdot 3+1 \cdot 2|}{\sqrt{3^2+1^2} \cdot \sqrt{3^2+2^2}}=\dfrac{2}{\sqrt{130}} \Rightarrow \beta \approx 79{,}9^{\circ}$.
		\end{listEX}
	}
\end{vd}
\dongcham{15}

\begin{vd}%[2H5V2-8]
	\immini{Người ta muốn dựng một cột ăng-ten trên một sườn đồi. Ăng-ten được dựng thẳng đứng trong không gian $Oxyz$ với độ dài đơn vị trên mỗi trục bằng $1$ m. Gọi $O$ là gốc cột, $A$ là điểm buộc dây cáp vào cột ăng-ten và $M$, $N$ là hai điểm neo dây cáp xuống mặt sườn đồi (hình vẽ). Cho biết toạ độ các điểm nói trên lần lượt là $O(0;0;0)$, $A(0;0;6)$, $M(3;-4;3)$, $N(-5;-2;2)$.
	}{\begin{tikzpicture}[scale=.4,>=stealth, font=\footnotesize, line join=round, line cap=round]
			\filldraw[green!40!brown!50] 
			(-5,0) .. controls (-3,3) and (-1.5,2.5) .. (0,2) .. controls (1.5,2.5) and (3,2) .. (5,0) -- cycle;
			\filldraw[green!70!black] 
			(-6,0) .. controls (-4,2.5) and (-2,2) .. (0,0) -- cycle;
			\filldraw[green!80!black] 
			(-4,0) .. controls (0,4) and (3,3.5) .. (6,0) -- cycle;
			\draw[line width=.5, double](5,0.6)node[right]{$O$}circle(2pt)--(5,4.2);
			\draw[line width=1,red](4,5)--(6,3) (3.5,4.8)--(4.5,5.2) (4.5,3.8)--(5.5,4.2) (5.5,2.8)--(6.5,3.2);
			\draw(0,1)node[below]{$N$}circle(2pt)--(5,2.5)node[right]{$A$}circle(2pt)--(1,2)node[above]{$M$}circle(2pt);
			% Vẽ các ngọn đồi đậm hơn phía trước cùng
			\filldraw[green!90!black] 
			(-5,0) .. controls (-3.5,1.5) and (-1.5,1.5) .. (0,0) -- cycle;
			
	\end{tikzpicture}}
	\begin{listEX}
		\item Tính độ dài các đoạn dây cáp $MA$ và $NA$.
		\item Tính góc tạo bởi các sợi dây cáp $MA$, $NA$ với mặt phẳng sườn đồi.
	\end{listEX} 
	\loigiai{
		\begin{listEX}
			\item Ta có $\overrightarrow{MA}=(-3; 4; 3), \overrightarrow{NA}=(5; 2; 4)$, suy ra
			\begin{itemize}
				\item $MA=\sqrt{(-3)^2+4^2+3^2}=\sqrt{34} \approx 5{,}8$ m. 
				\item $NA=\sqrt{5^2+2^2+4^2}=\sqrt{45} \approx 6{,}7$ m.
			\end{itemize}
			\item  Mặt phẳng $(OMN)$ có cặp vectơ chỉ phương là $\overrightarrow{OM}=(3;-4; 3)$,  $\overrightarrow{ON}=(-5;-2; 2)$ nên có vectơ pháp tuyến $\overrightarrow{n}=[\overrightarrow{O M}, \overrightarrow{O N}]=(-2;-21;-26)$.\\        
			Gọi $\alpha$, $\beta$ lần lượt là góc tạo bởi $MA$, $NA$ với mặt phẳng $(AMN)$.\\
			Ta có
			\begin{eqnarray*}
				&\sin \alpha=\dfrac{|\overrightarrow{M A} \cdot \overrightarrow{n}|}{|\overrightarrow{M A}| \cdot|\overrightarrow{n}|} & =\dfrac{|-3 \cdot(-2)+4 \cdot(-21)+3 \cdot(-26)|}{\sqrt{(-3)^2+4^2+3^2} \cdot \sqrt{(-2)^2+(-21)^2+(-26)^2}} \\
				&&=\dfrac{156}{\sqrt{38\,114}} \Rightarrow \alpha \approx 53^{\circ}.
			\end{eqnarray*}
			Và  
			\begin{eqnarray*}
				&\sin \beta=\dfrac{|\overrightarrow{NA} \cdot \overrightarrow{n}|}{|\overrightarrow{N A}| \cdot|\overrightarrow{n}|} & =\frac{|5 \cdot(-2)+2 \cdot(-21)+4 \cdot(-26)|}{\sqrt{5^2+2^2+4^2} \cdot \sqrt{(-2)^2+(-21)^2+(-26)^2}} \\
				&&=\dfrac{156}{\sqrt{50\,445}} \Rightarrow \beta \approx 44^{\circ}. 
			\end{eqnarray*}
		\end{listEX}    
	}
\end{vd}
\dongcham{13}

% \begin{vd}
% 	Một khuôn nướng bánh mì được mô phỏng trong không gian $Oxyz$ như hình vẽ với  $S(0; 0; 0)$, $P(8; 0; 0)$, $Q(8; 18; 0)$, $T(-1;-1; 7)$, $R(9; 19; 7)$. Tính góc giữa hai cạnh kề nhau, giữa cạnh bên và mặt đáy, giữa mặt bên và mặt đáy của khuôn.
% 	\begin{center}
% 		% \includegraphics[width=6cm]{images/KhuonBanh.jpg}~
% 		\begin{tikzpicture}[>=stealth,scale=.8]
% 			\path (0,0) coordinate (S)
% 			(-20:0.5) coordinate (ex)
% 			(15:0.5) coordinate (ey)
% 			(0,0.65) coordinate (ez)
% 			($4*(ex)$) coordinate (P)
% 			($4*(ex)+9*(ey)$) coordinate (Q)
% 			($9*(ey)$) coordinate (H)
% 			($-0.5*(ex)-0.5*(ey)+3.5*(ez)$) coordinate (T)
% 			($4.5*(ex)+9.5*(ey)+3.5*(ez)$) coordinate (R)
% 			($4.5*(ex)-.5*(ey)+3.5*(ez)$) coordinate (E)
% 			($-.5*(ex)+9.5*(ey)+3.5*(ez)$) coordinate (K)
% 			;
			
% 			\foreach \x in {-3,...,12}{
% 				\draw[dashed,gray!65] ($-3*(ex)+\x*(ey)$)--($9*(ex)+\x*(ey)$);
% 				\ifnum\x>0
% 				\pgfmathsetmacro{\xt}{int(2*\x)}
% 				\draw ($\x*(ey)+(0,-1pt)$)--($\x*(ey)+(0,1pt)$) node[above,font=\tiny]{$\xt$};
% 				\ifnum \x<9
% 				\draw ($\x*(ex)+(0,-1pt)$)--($\x*(ex)+(0,1pt)$) node[below,font=\tiny]{$\xt$};
% 				\draw[dashed,gray!65] ($-3*(ey)+\x*(ex)$)--($12*(ey)+\x*(ex)$);
% 				\fi
% 				\else
% 				\pgfmathsetmacro{\xt}{int(2*\x)}
% 				\draw ($\x*(ey)+(0,-1pt)$)--($\x*(ey)+(0,1pt)$)
% 				+(-0.1,0) node[below,font=\tiny]{$\xt$};
% 				\draw ($\x*(ex)+(0,-1pt)$)--($\x*(ex)+(0,1pt)$) +(-0.1,0) node[below,font=\tiny]{$\xt$};
% 				\draw[dashed,gray!65] ($-3*(ey)+\x*(ex)$)--($12*(ey)+\x*(ex)$);
% 				\fi
% 			}
% 			\foreach \x in {-1,1,2,...,5}{
% 				\pgfmathsetmacro{\xt}{int(2*\x)}
% 				\draw ($\x*(ez)+(1pt,0)$)--($\x*(ez)+(-1pt,0)$) +(0.5pt,0) node[anchor=east,font=\tiny]{$\xt$};
% 			}
% 			\draw[->,cyan!65] ($-3*(ex)$)--($9*(ex)$);
% 			\draw[->,cyan!65] ($-3*(ey)$)--($12.5*(ey)$);
% 			\draw[->,cyan!65] ($-1.5*(ez)$)--($5.5*(ez)$);
% 			\draw[dashed](S)--(H)--(Q) (H)--(K);
% 			\draw (S)--(P)--(E)--(T)--cycle
% 			(P)--(Q)--(R)--(E)
% 			(R)--(K)--(T);
% 			\foreach \t/\g in {S/-65,P/45,Q/-60,H/-90,R/20,K/75,T/110,E/90}{
% 				\shade[ball color=blue] (\t) circle (1pt) node[shift={(\g:7pt)},font=\scriptsize]{$ \t $};
% 			}
% 		\end{tikzpicture}
% 	\end{center}
% 	\loigiai{
% 		\begin{listEX}[1]
% 			\item Tính góc giữa hai cạnh kề nhau\\
% 			$\overrightarrow{SP}=\left(8;0;0\right);\overrightarrow{SH}=\left(0;18;0\right);\overrightarrow{ST}=\left(-1;-1;7\right)$.\\
% 			$\overrightarrow{SP}\cdot\overrightarrow{SH}=0\Rightarrow \left(SP,SH\right)=90^\circ$.\\
% 			$\cos\left(SP,ST\right)=\dfrac{\left|\overrightarrow{SP}\cdot\overrightarrow{ST}\right|}{\left|\overrightarrow{SP}\right|\cdot\left|\overrightarrow{ST}\right|}=\dfrac{1}{\sqrt{51}}\Rightarrow \left(SP,ST\right)\approx 82^\circ$.\\
% 			$\cos\left(SH,ST\right)=\dfrac{\left|\overrightarrow{SH}\cdot\overrightarrow{ST}\right|}
% 			{\left|\overrightarrow{SH}\right|\cdot\left|\overrightarrow{ST}\right|}
% 			=\dfrac{1}{\sqrt{51}}\Rightarrow \left(SH,ST\right)\approx 82^\circ$.
% 			\item Tính góc giữa cạnh bên và mặt đáy\\
% 			Gọi $\alpha$ là góc giữa cạnh bên $ST$ và mặt phẳng đáy.\\
% 			$\sin\alpha=\dfrac{\left|\overrightarrow{ST}\cdot\vec{k}\right|}{\left|\overrightarrow{ST}\right|\cdot\left|\vec{k}\right|}=\dfrac{7}{\sqrt{51}}\Rightarrow \alpha\approx 78^\circ$.
% 			\item Tính góc giữa mặt bên và mặt đáy\\
% 			Gọi $\beta$ là góc giữa mặt bên và mặt phẳng đáy.\\
% 			$\vec{n}=\left[\overrightarrow{ST},\overrightarrow{SP}\right]=\left(0;56;8\right)$.\\
% 			$\cos\beta=\dfrac{\left|\overrightarrow{n}\cdot\vec{k}\right|}{\left|\overrightarrow{n}\right|\cdot\left|\vec{k}\right|}=\dfrac{\sqrt{2}}{10}\Rightarrow \beta\approx 82^\circ$.
% 		\end{listEX}
% 	}
% \end{vd}
% \dongcham{28}
\boxmini{BÀI TẬP TRẮC NGHIỆM}
\setcounter{ex}{0}
\Opensolutionfile{ans}[ans/2H5-B3-d2]

\begin{ex}
		Trong hệ trục toạ độ $Oxyz$, với mặt phẳng $(Ox y)$ là mặt đất, một máy bay cất cánh từ vị trí $A(0; 10; 0)$ với vận tốc $\vec{v}=(150; 150; 40)$. Tính góc nâng của máy bay (góc giữa hướng chuyển động bay lên của máy bay với đường băng và làm tròn kết quả đến hàng đơn vị).
		\choice
		{$10^{\circ}$}
		{$12^{\circ}$}
		{\True $11^{\circ}$}
		{$9^{\circ}$}
	\loigiai{Gọi $\alpha$ là góc nâng của máy bay.\\
		$\sin\alpha=\dfrac{\left|\overrightarrow{v}\cdot\vec{k}\right|}{\left|\overrightarrow{v}\right|\cdot\left|\vec{k}\right|}\Rightarrow \alpha\approx 10^\circ 40'$.
	}
\end{ex}


\begin{ex}
	Cho hình lập phương $ABCD.A'B'C'D'$có cạnh bằng $a$. Tính số đo góc giữa hai mặt phẳng $(BA'C)$ và $(DA'C)$.
	\choice
	{$30^\circ$}
	{$120^\circ$}
	{$90^\circ$}
	{\True $60^\circ$}
	\loigiai{
		\immini
		{ Chọn hệ tọa độ $Oxyz$ có $A\equiv O$, $ \overrightarrow{AB}$, $\overrightarrow{AD}$, $\overrightarrow{AA'}$ lần lượt cùng hướng với các vectơ đơn vị $\overrightarrow{i}$, $\overrightarrow{j}$, $\overrightarrow{k}$.\\
			Lấy $a=1$, suy ra $B(1;0;0)$, $D(0;1;0)$, $A'(0;0;1)$, $C(1;1;0)$.\\
			Mặt phẳng $(BA'C)$ có vectơ pháp tuyến là \\ $\overrightarrow{n_1}=\overrightarrow{BA'}\wedge \overrightarrow{BC}=(-1;0;-1)$.\\
			Mặt phẳng $(DA'C)$ có vectơ pháp tuyến là \\ $\overrightarrow{n_2}=\overrightarrow{DA'}\wedge \overrightarrow{DC}=(0;1;1)$.\\
			Gọi $\varphi $ là góc giữa hai mặt phẳng $(BA'C)$ và $(DA'C)$, ta có
			$ \cos\varphi =\left|{\cos\left(\overrightarrow{n_1},\overrightarrow{n_2}\right)}\right|=\dfrac{\left|{\overrightarrow{n_1} \cdot \overrightarrow{n_2}}\right|}{\left|{\overrightarrow{n_1}}\right|\cdot \left|{\overrightarrow{n_2}}\right|}=\dfrac{1}{\sqrt{2}\cdot \sqrt{2}}=\dfrac{1}{2}$.\\ Suy ra $\varphi =60^\circ$.
		}
		{ \begin{tikzpicture}[scale=1,>=stealth, font=\footnotesize, line join=round, line cap=round]
				\tkzDefPoints{0/0/A,-1.3/-1.1/B,2/-1.1/C}
				\coordinate (D) at ($(A)+(C)-(B)$);
				\coordinate (A') at ($(A)+(0,3.3)$);
				\coordinate (x) at ($(A)!1.6!(B)$);
				\coordinate (y) at ($(A)!1.2!(D)$);
				\coordinate (z) at ($(A)!1.3!(A')$);
				\tkzDrawSegments[vector style](B,x D,y A',z)
				\tkzDefPointsBy[translation=from A to A'](B,C,D){B'}{C'}{D'}
				\tkzDrawPolygon(A',B',B,C,D,D')
				\tkzDrawSegments(B',C' C',D' C,C')
				\tkzDrawSegments[dashed](A,B A,D A,A' A',C A,C B,D A',B A',D)
				\tkzDrawPoints[fill=black,size=4](A,B,D,C,A',B',C',D')
				\tkzLabelPoints[below](B,C,x)
				\tkzLabelPoints[left](B')
				\tkzLabelPoints[right](C',z)
				\tkzLabelPoints[above right](A,D,A',D',y)
			\end{tikzpicture}
		}
	}
\end{ex}

\begin{ex}
	Cho hình lập phương $MNPQ.M'N'P'Q'$ có $E$, $F$, $G$ lần lượt là trung điểm của $NN'$, $PQ$, $M'Q'$ Tính góc giữa hai đường thẳng $EG$ và $P'F$.
	\choice
	{$60^{\circ} $}
	{\True $90^{\circ} $}
	{$30^{\circ} $}
	{$45^{\circ} $}
	\loigiai{
		\immini{Chọn hệ trục tọa độ $Oxyz$ sao cho $M(0;0;0)$, $N(1;0;0)$, $Q(0;1;0)$ và $M'(0;0;1)$. Lúc đó $P(1;1;0), N'(1;0;1), Q'(0;1;1)$ và $P'(1;1;1)$.\\
			Vì $E, F, G$ lần lượt là trung điểm $NN', PQ$ và $M'Q'$ nên $E\left(1;0;\dfrac{1}{2}\right)$, $F\left(\dfrac{1}{2};1;0\right)$ và $G\left(0;\dfrac{1}{2};1\right)$.\\
			Suy ra $\vec{EG}=\left(-1;\dfrac{1}{2};\dfrac{1}{2}\right)$, $\vec{P'F}=\left(-\dfrac{1}{2};0;-1\right)$, do đó $\vec{EG}\cdot \vec{P'F}=0$ hay $(EG,P'F)=90^{\circ}$.}{\begin{tikzpicture}[line cap=round,line join=round,scale=1.5]%[Mai Hà Lan] %1.20
				%-------------- Đáy ABCD
				\tkzDefPoints{0/0/A, -0.5/-1/B, 2/0/D}
				\coordinate (C) at ($(B)+(D)-(A)$);
				%-------------- Đáy A'B'C'D'
				\tkzDefPointBy[rotation = center A angle 90](D) \tkzGetPoint{A'} %Phép quay tâm A, góc quay 90 độ, biến D thành A'
				\coordinate (B') at ($(B)+(A')-(A)$);
				\coordinate (D') at ($(D)+(A')-(A)$);
				\coordinate (C') at ($(B')+(D')-(A')$);
				%---------------
				\tkzDrawSegments[dashed](A,B A,D A,A')
				\tkzDrawPolygon(A',B',C',D')
				\tkzDrawPolygon(B,C,C',B')
				\tkzDrawSegments(D,D' C,D)
				\tkzDrawPoints(A,B,C,D,A',B',C',D')
				\tkzLabelPoint(A){$M$}
				\tkzLabelPoint(B){$N$}
				\tkzLabelPoint(C){$P$}
				\tkzLabelPoint(D){$Q$}
				\tkzLabelPoint(A'){$M'$}
				\tkzLabelPoint(B'){$N'$}
				\tkzLabelPoint(C'){$P'$}
				\tkzLabelPoint(D'){$Q'$}
				%\tkzLabelPoints(A,B,C,D,A',B',C',D')
				\tkzDefMidPoint(B,B') \tkzGetPoint{E}
				\tkzDefMidPoint(C,D) \tkzGetPoint{F}
				\tkzDefMidPoint(A',D') \tkzGetPoint{G}
				\tkzLabelPoints[above](G)
				\tkzLabelPoints[right](E,F)
				\tkzDrawSegments[dashed](E,G)
				\tkzDrawSegments(C',F)
				\tkzDrawPoints(G,E)
		\end{tikzpicture}}
	}
\end{ex}

\begin{ex}
	Cho hình hộp chữ nhật $ABCD.A'B'C'D'$ có các cạnh $AB=2,AD=3,AA'=4$. Góc giữa hai mặt phẳng $(AB'D')$ và $(A'C'D)$ là $\alpha$. Tính giá trị gần đúng của góc $\alpha$.
	\choice
	{$45,2^{\circ}$}
	{$38,1^{\circ}$}
	{\True $61,6^{\circ}$}
	{$53,4^{\circ}$}
	\loigiai{
		\immini{Gắn hình hộp chữ nhật $ABCD.A'B'C'D'$ vào hệ trục tọa độ $Oxyz$. Khi đó $A(0,0,0)$, $B(0;2;0)$, $C(3;2;0)$, $D(3;0;0)$, $A'(0;0;4)$, $B'(0;2;4)$, $C'(3;2;4)$, $D'(3;0;4)$.\\
			$\vec{AB'}=(0;2;4)$, $\vec{AD'}=(3;0;4)$, $\vec{A'C'}=(3;2;0),\vec{A'D}=(3;0;-4)$.\\
			Gọi $\overrightarrow{n}_{1}$ là vectơ pháp tuyến của $(AB'D')$. Ta có $\overrightarrow{n}_{1}=[\vec{AB'},\vec{AD'}]=(8;12;-6)$.\\
			Gọi $\overrightarrow{n}_{2}$ là vectơ pháp tuyến của $(A'C'D)$. Ta có $\overrightarrow{n}_{2}=[\vec{A'C'},\vec{A'D}]=(-8;12;-6)$.\\
			$\alpha$ là góc giữa hai mặt phẳng $(AB'D')$ và $(A'C'D)$, ta có \\
			$\cos \alpha=\left|\dfrac{\overrightarrow{n}_{1}\cdot \overrightarrow{n}_{2}}{|\overrightarrow{n}_{1}||\overrightarrow{n}_{2}|} \right|=\dfrac{29}{61}$.\\
			Vậy giá trị gần đúng của góc $\alpha$ là $61,6^{\circ}$.
		}{
	\begin{tikzpicture}
		\tkzDefPoints{0/0/A,-3/-2/D,2/-2/C,5/0/B,0/5/A',-3/3/D',2/3/C',5/5/B'}
		\tkzDrawPoints(A,B,C,D,A',B',C',D')
		\tkzLabelPoints[above right](A')
		\tkzLabelPoints[above left](B')
		\tkzLabelPoints[above](C')
		\tkzLabelPoints[above left](B,D',D)
		\tkzLabelPoints[below right](A,C)
		\tkzDrawSegments(B,C C,D D,D' C,C' B,B' A',B' B',C' C',D' D',A' B',D' A',C' C',D)
		\tkzDrawSegment[style=dashed](A,B)
		\tkzDrawSegment[style=dashed](A,D)
		\tkzDrawSegment[style=dashed](A,A')
		\tkzDrawSegment[style=dashed](A,B')
		\tkzDrawSegment[style=dashed](A,D)
		\tkzDrawSegment[style=dashed](A,D')
		\tkzDrawSegment[style=dashed](A',D)
\end{tikzpicture}}
	}
\end{ex}

\begin{ex}
		Cho hình chóp $SABCD$ có $ABCD$ là hình vuông cạnh $a$, $SA$ vuông góc $\left(ABCD\right)$, $SA=a$. Gọi $E$ và $F$ lần lượt là trung điểm $SB,SD$. Cô-sin của góc hợp bởi hai mặt phẳng $\left(AEF\right)$ và $\left(ABCD\right)$ là
		\choice
		{$\sqrt{3}$}
		{$\dfrac{\sqrt{3}}{2}$}
		{$\dfrac{1}{2}$}
		{\True $\dfrac{\sqrt{3}}{3}$}
	\loigiai{
		\immini{
			Chọn hệ trục tọa độ như hình vẽ. Ta có:
			$A\left(0;0;0\right)$, $B\left(a;0;0\right)$, $D\left(0;a;0\right)$, $S\left(0;0;a\right)$, $E\left(\dfrac{a}{2};0;\dfrac{a}{2}\right)$, $F\left(0;\dfrac{a}{2};\dfrac{a}{2}\right)$ và
			$\overrightarrow{AE}\left(\dfrac{a}{2};0;\dfrac{a}{2}\right)$, $\overrightarrow{AF}\left(0;\dfrac{a}{2};\dfrac{a}{2}\right)$.\\
			$\Rightarrow \left[\overrightarrow{AE},\overrightarrow{AF}\right]=\left(-\dfrac{a^2}{4};-\dfrac{a^2}{4};\dfrac{a^2}{4}\right)$.\\
			$\Rightarrow $ Một VTPT của mặt phẳng $\left(AEF\right)$ là $\left(1;1;-1\right)$.
			Phương trình mặt phẳng $\left(AEF\right)\colon $ $x+y-z=0$.\\
			Phương trình mặt phẳng $\left(ABCD\right)\colon z=0$.
		}{
			\begin{tikzpicture}[scale=0.8, font=\footnotesize, line join=round, line cap=round, >=stealth]
				\def \xa{-2}
				\def \xb{-1}
				\def \y{4}
				\def \z{3}
				\coordinate (A) at (0,0);
				\coordinate (B) at ($(A)+(\xa,\xb)$);
				\coordinate (Bx) at ($(B)+(\xa,\xb)$);
				\coordinate (D) at ($(A)+(\y,0)$);
				\coordinate (Dy) at ($(D)+(2,0)$);
				\coordinate (C) at ($ (B)+(D)-(A) $);
				\coordinate (S) at ($ (A)+(0,\z) $);
				\coordinate (Sz) at ($ (S)+(0,1.5) $);
				\tkzDefMidPoint(S,B)    \tkzGetPoint{E}
				\tkzDefMidPoint(S,D)    \tkzGetPoint{F}
				\draw [dashed] (B)--(A)--(D) (A)--(S) (A)--(E)--(F)--(A);
				\draw (S)--(B)--(C)--(D)--(S)--(C);
				\draw[->] (B)--(Bx) node [below right] {$x$};
				\draw[->] (D)--(Dy) node [below] {$y$};
				\draw[->] (S)--(Sz) node [right] {$z$};
				\tkzDrawPoints(S,A,B,C,D,E,F)
				\tkzLabelPoints[above right](D)
				\tkzLabelPoints[below right](C)
				\tkzLabelPoints[above left](S)
				\tkzLabelPoints[left](A)
				\tkzLabelPoints[below](B)
				\tkzLabelPoints[above right](F)
				\tkzLabelPoints[above left](E)
		\end{tikzpicture}}
		\noindent Góc hợp bởi hai mặt phẳng $\left(AEF\right)$ và $\left(ABCD\right)$ là $\alpha $, ta có $\cos \alpha =\left| \dfrac{-1}{\sqrt{3}\cdot  1}\right|=\dfrac{1}{\sqrt{3}}.$
	}
\end{ex}
\begin{ex}
	Cho hình chóp tam giác $O.ABC$ có $OA$, $OB$, $OC$ đôi một vuông góc và $OA=OB=OC$. Lấy $M$, $N$ lần lượt là trung điểm của $AB$, $OC$. Gọi $\alpha$ là góc tạo bởi $OA$ và $MN$. Tính $\cos\alpha$.
	\choice
	{\True $\dfrac{\sqrt{3}}{3}$}
	{$\dfrac{1}{3}$}
	{$\dfrac{\sqrt{3}}{4}$}
	{$\dfrac{\sqrt{3}}{2}$}
	\loigiai{
		\immini{
			Do $OA$, $OB$, $OC$ đôi một vuông góc nên chọn hệ trục toạ độ như hình vẽ. Khi đó, ta có hệ toạ độ của các điểm $O(0;0;0)$, $A(a;0;0)$, $B(0;a;0)$, $C(0;0;a)$.\\
			Suy ra $M\left(\dfrac{a}{2};\dfrac{a}{2};0\right)$, $N\left(0;0;\dfrac{a}{2}\right)$ nên $\overrightarrow{N M}=\left(\dfrac{a}{2} ; \dfrac{a}{2} ;-\dfrac{a}{2}\right)$.\\
			Suy ra $\cos \alpha=\dfrac{\dfrac{a^{2}}{2}}{\sqrt{0^{2}+0^{2}+a^{2}} \sqrt{\left(\dfrac{a}{2}\right)^{2}+\left(\dfrac{a}{2}\right)^{2}+\left(-\dfrac{a}{2}\right)^{2}}}=\dfrac{\sqrt{3}}{3}$.
		}{
			\begin{tikzpicture}[scale=0.8,>=stealth, font=\footnotesize, line join=round, line cap=round]
				%%\draw[color=gray!50,dashed] (-6,-6) grid (6,6);
				\tkzSetUpPoint[fill=black]
				\tkzDefPoints{0/0/O,3/0/C,0/3/A,-2/-2/B}
				\tkzDefMidPoint(A,B)\tkzGetPoint{M}
				\tkzDefMidPoint(O,C)\tkzGetPoint{N}
				\coordinate (x) at ($(O)!1.3!(B)$);
				\coordinate (y) at ($(O)!1.2!(C)$);
				\coordinate (z) at ($(O)!1.2!(A)$);
				\draw[line width = 0.6pt,->](B) -- (x);
				\draw[line width = 0.6pt,->](C) -- (y);
				\draw[line width = 0.6pt,->](A) -- (z);
				\tkzDrawPoints[fill=black,size=4](O,A,B,C,M,N)
				\tkzLabelPoints[above left](A)
				\tkzLabelPoints[left](B,O)
				\tkzLabelPoints[left](M)
				\tkzLabelPoints[above](N)
				\tkzLabelPoints[right](z,x)
				\tkzLabelPoints[below](y)
				\tkzLabelPoints[above right](C)
				\tkzDrawSegments[dashed](O,A O,B O,C M,N)
				\tkzDrawSegments(A,B B,C A,C)
				\tkzMarkRightAngles[size=0.16](B,O,C A,O,C B,O,A)
			\end{tikzpicture}
		}
	}
\end{ex}
\begin{ex}%[2H3K4-1]%
	Hình chóp $S.ABC$ có đáy là tam giác vuông tại $B$ có $AB=a$,  $AC =2a$. $SA$ vuông góc với mặt phẳng đáy, $SA = 2a$. Gọi $\psi$ là góc tạo bởi hai mặt phẳng $(SAC)$ và $(SBC)$. Tính $\cos\psi$.
	\choice
	{$\dfrac{1}{2}$}
	{$\dfrac{\sqrt{3}}{5}$}
	{$\dfrac{\sqrt{3}}{2}$}
	{\True $\dfrac{\sqrt{15}}{5}$}
	\loigiai{
		\immini{Chọn hệ trục tọa độ $Bxyz$ như hình vẽ.\\
			Ta tính được $B(0;0;0)$, $A(a;0;0)$, $C(0;a \sqrt{3};0)$, $S(a;0;2a)$,\\
			$\vv{SA}=(0;0;-2a)$, $\vv{SB}=(-a;0;-2a)$, $\vv{SC}=(-a;a\sqrt{3};-2a)$,\\
			$\vv{n_1}= \left[ \vv{SA} ; \vv{SC} \right] = (2a^2\sqrt{3};2a^2;0)$ là VTPT của $(SAC)$,\\
			$\vv{n_2}= \left[ \vv{SB} ; \vv{SC} \right] = (2a^2\sqrt{3};0;-a^2\sqrt{3})$ là VTPT của $(SBC)$.\\
			Ta có $$\cos \psi = \big|\cos \left[\vv{n_1};\vv{n_2}\right]\big|=\dfrac{|\vv{n_1}\vv{n_2}|}{|\vv{n_1}|\cdot |\vv{n_2}|}=\dfrac{\sqrt{15}}{5}.$$
		}{\begin{tikzpicture}[>=stealth,line join=round,line cap=round,font=\footnotesize,scale=0.8]
				\tkzDefPoints{0/0/A, 2/-2/B, 7/0/C}
				\coordinate (S) at ($(A)+(0,5)$);
				\tkzDefPointBy[homothety = center B ratio 1.7](A)
				\tkzGetPoint{E}
				\tkzDefPointBy[homothety = center B ratio 1.4](C)
				\tkzGetPoint{F}
				\coordinate (G) at ($(B)+(0,7)$);
				\tkzDrawSegments[dashed](A,C)
				\tkzDrawSegments(S,A S,B B,C S,C A,B)
				\tkzLabelPoints[below left](A,B)
				\tkzLabelPoints[below right](C)
				\tkzLabelPoints[above](S)
				\tkzMarkRightAngle(A,B,C)
				\draw[->](B)--(E);
				\draw[->](B)--(F);
				\draw[->](B)--(G);
				\draw (-1.5,1.5)node[below left]{$x$} (9,0)node[above]{$y$} (2.5,5)node[right]{$z$};
				\draw (0,2)node[above left]{$2a$};
				\draw (1.3,-1.5)node[above left]{$a$};
				\draw (3,0) node[above]{$2a$};
			\end{tikzpicture}
	}}
\end{ex}

\begin{ex}
	Cho hình chóp $S.ABCD$ có đáy $ ABCD $ là hình chữ nhật, $ AB=a $, $ BC=a\sqrt{3} $, $ SA=a $ và $ SA $ vuông góc với đáy $ ABCD $. Tính $ \sin \alpha $, với $ \alpha $ là góc tạo bởi giữa đường thẳng $ BD $ và mặt phẳng $ (SBC) $.
	\choice
	{\True $\dfrac{\sqrt{2}}{4} $}
	{$ \dfrac{\sqrt{3}}{3} $}
	{$ \dfrac{\sqrt{3}}{4} $}
	{$\dfrac{\sqrt{2}}{2} $}
	\loigiai{
		\immini{
			Đặt hệ trục tọa độ $ Oxyz $ như hình vẽ. Khi đó $A(0;0;0)$, $B(a;0;0)$, $D(0;a\sqrt{3};0)$, $S(0;0;a)$.\\
			Ta có $ \overrightarrow {BD}  = (-a;a\sqrt{3};0) = a(-1;\sqrt{3};0) $ nên đường thẳng $ BD $ có vectơ chỉ phương là $ \overrightarrow {u} = (-1;\sqrt{3};0) $.\\
			Ta có  $\overrightarrow {SB} = (a;0;-a)$, $ \overrightarrow {BC} = (0;a\sqrt{3};0) \Rightarrow \left[\overrightarrow {SB} ,\overrightarrow {BC} \right] = \left( {a^2 \sqrt{3}; 0; a^2\sqrt{3}} \right) = a^2 \sqrt{3}  (1;0;1)$. Nên mặt phẳng $ (SBC) $ có vectơ pháp tuyến là $ \overrightarrow {n} = (1;0;1) $.\\
			$ \alpha $ là góc tạo bởi giữa đường thẳng $ BD $ và mặt phẳng $ (SBC) $ thì \\
			$
			\sin \alpha = \dfrac{\left|\overrightarrow u \cdot \overrightarrow{n} \right|}{\left| {\overrightarrow{u}} \right| \cdot \left| {\overrightarrow {n}} \right|}
			= \dfrac{|( - 1)\cdot 1 + \sqrt 3 \cdot 0 + 0 \cdot1 |}{\sqrt {(- 1)^2  + (\sqrt 3)^2  + 0^2 } \cdot \sqrt {1^2  + 0^2  + 1^2}}
			= \dfrac{\sqrt 2}{4}.
			$
		}{
			\begin{tikzpicture}[scale=0.5,>=stealth]
				\tkzDefPoints{0/0/A, -3/-2/x, 7/0/y, 0/7/z}
				\coordinate (B) at ($(A)+(x)$);
				\coordinate (D) at ($(A)+(y)$);
				\coordinate (C) at ($(B)+(D)$);
				\coordinate (S) at ($(A)+(z)$);
				\coordinate (a) at ($(A)!1.4!(B)$);
				\coordinate (b) at ($(A)!1.2!(D)$);
				\coordinate (c) at ($(S)+(0,1)$);
				\tkzDrawPoints[fill=black](A,B,C,D,S)
				\draw (S)--(B)--(C)--(D)--(S) (S)--(C);
				\draw[dashed] (S)--(A)--(B)--(D)--(A);
				\draw[->] (B)--(a);
				\draw[->] (D)--(b);
				\draw[->] (S)--(c);
				\tkzLabelPoints[left](S,A,B)
				\tkzLabelPoints[below right](C)
				\tkzLabelPoints[above right](D)
				\node[above] at (a) {$y$};
				\node[below] at (b) {$x$};
				\node[left] at (c) {$z$};
			\end{tikzpicture}
		}
	}
\end{ex}

\begin{ex}
	Cho hình chóp tứ giác đều $S.ABCD$ có cạnh đáy bằng $a$, tâm $O$. Gọi $M$ và $N$ lần lượt là trung điểm $SA$ và $BC$. Biết góc giữa $MN$ và $\left(ABCD\right)$ bằng $60^{\circ}$, côsin góc giữa $MN$ và mặt phẳng $(SBD)$ bằng
	\choice
	{$\dfrac{\sqrt{41}}{41}$}
	{$\dfrac{2\sqrt{41}}{41}$}
	{$\dfrac{\sqrt{5}}{5}$}
	{\True $\dfrac{2\sqrt{5}}{5}$}
	\loigiai{
		\begin{center}
			\begin{tikzpicture}[scale=0.8, font=\footnotesize, line join=round, line cap=round, >=stealth]
				\coordinate (A) at (0,0);
				\coordinate (B) at ($(A) +(5,0)$);
				\coordinate (C) at ($(B) +(-140:2.5)$);
				\coordinate (D) at ($(A) +(-140:2.5)$);
				\coordinate (O) at ($(A)!0.5!(C)$);
				\coordinate (S) at ($(O)+(0,4)$);
				\coordinate (M) at ($(S)!0.5!(A)$);
				\coordinate (N) at ($(B)!0.5!(C)$);
				\coordinate (I) at ($(A)!0.5!(O)$);
				\coordinate (K) at ($(S)!0.5!(O)$);
				\coordinate (x) at ($(A)!1.4!(C)$);
				\coordinate (y) at ($(D)!1.4!(B)$);
				\coordinate (z) at ($(O)!1.4!(S)$);
				\foreach \x in {A,B,C,D,O,M,N}{
					\draw[fill] (\x) circle(1pt);
				}
				\draw (D)--(C)--(B) (S)--(B) (S)--(C) (S)--(D);
				\draw[dashed] (A)--(B) (A)--(D) (S)--(A) (S)--(O) (M)--(N) (B)--(D) (A)--(C);
				\draw[->]  (C)--(x) node[right]{$x$};
				\draw[->]  (B)--(y) node[above]{$y$};
				\draw[->]  (S)--(z) node[right]{$z$};
				\path
				node at (A) [left]{$A$}
				node at (B) [below]{$B$}
				node at (C) [below]{$C$}
				node at (D) [below left]{$D$}
				node at (S) [left]{$S$}
				node at (O) [below]{$O$}
				node at (M) [below left]{$M$}
				node at (N) [below right]{$N$};
			\end{tikzpicture}
		\end{center}
		Chọn hệ trục tọa độ như hình vẽ, đặt $OS=k$ $(k>0)$.\\
		Ta có $A\left(-\dfrac{a\sqrt{2}}{2};0;0\right)$, $B\left(0;\dfrac{a\sqrt{2}}{2};0\right)$, $C\left(\dfrac{a\sqrt{2}}{2};0;0\right)$, $D\left(0;-\dfrac{a\sqrt{2}}{2};0\right)$, $S\left(0;0;k\right)$,\\ $M\left(-\dfrac{a\sqrt{2}}{4};0;\dfrac{k}{2}\right)$ và $N\left(\dfrac{a\sqrt{2}}{4};\dfrac{a\sqrt{2}}{4};0\right)$.\\
		Ta có $\vec{u}_{MN}= \vec{MN}=\left(\dfrac{a\sqrt{2}}{2}; \dfrac{a\sqrt{2}}{4};-\dfrac{k}{2}\right)$ và $\vec{n}_{ABCD} = (0;0;1)$.\\
		Ta có \begin{eqnarray*}
			\sin \left(MN;(ABCD)\right)= \dfrac{\left|\vec{u}_{MN} \cdot \vec{n}_{ABCD}\right|}{\left|\vec{u}_{MN}\right| \cdot \left|\vec{n}_{ABCD}\right|} & \Leftrightarrow & \sin 60^{\circ} =\dfrac{\dfrac{k}{2}}{\sqrt{\dfrac{9a^2}{16}+\dfrac{k^2}{4}}} \\
			&\Leftrightarrow & k =  \dfrac{a\sqrt{30}}{2}.
		\end{eqnarray*}
		Ta có $\vec{n}_{SBD}=(1;0;0)$, do đó
		\begin{eqnarray*}
			\sin \left(MN,(SBD)\right)= \dfrac{\left|\vec{u}_{MN} \cdot \vec{n}_{SBD}\right|}{\left|\vec{u}_{MN}\right| \cdot \left|\vec{n}_{SBD}\right|} & \Leftrightarrow & \sin \left(MN,(SBD)\right)= \dfrac{\dfrac{a\sqrt{2}}{2}}{\sqrt{\dfrac{5a^2}{2}}} = \dfrac{\sqrt{5}}{5}.
		\end{eqnarray*}
		$\Rightarrow  \cos \left(MN,(SBD)\right) = \dfrac{2\sqrt{5}}{5}$. (do góc nhọn nên $cos$ dương)
	}
\end{ex}
\begin{ex}
	Cho hình chóp $S.ABCD$ có đáy là hình thang vuông tại $A$ và $B$, $AB=BC=a$, $AD=2a, \ \ SA$ vuông góc với mặt đáy $(ABCD)$, $SA=a$. Gọi $M,N$ lần lượt là trung điểm của $SB$ và $CD$. Tính cosin của góc giữa $MN$ và $(SAC)$.
	\choice
	{$\dfrac{2}{\sqrt{5}}$}
	{$\dfrac{1}{\sqrt{5}}$}
	{$\dfrac{3 \sqrt{5}}{10}$}
	{\True $\dfrac{\sqrt{55}}{10}$}
	\loigiai{
		\immini{
			Chọn hệ trục $Oxyz$ như hình vẽ, với $O \equiv A$. \\
			Khi đó ta có: $A(0;0;0)$, $B(a;0;0)$, $C(a;a;0)$, $D(0;2a;0)$, $S(0;0;a)$. \\
			Khi đó: $M \left(\dfrac{a}{2};0; \dfrac{a}{2} \right)$, $N \left(\dfrac{a}{2}; \dfrac{3a}{2};0 \right)$. \\
			Ta có: $- \dfrac{1}{a} \vec{SA}=(0;0;1)= \vec{u}$; $\dfrac{1}{a} \vec{SC}=(1;1;-1)= \vec{v}$. \\
			Gọi $\vec{n}$ là vectơ pháp tuyến của mặt phẳng $(SAC)$ ta có $\vec{n}= \left[\vec{u}, \vec{v} \right]=(-1;-1;0)$. \\
			Lại có: $\dfrac{2}{a} \vec{MN}=(0;3;-1)= \vec{w}$. \\
			Gọi $\alpha $ là góc giữa $MN$ và $(SAC)$ ta có: $\sin \alpha = \dfrac{\left| \vec{n}. \vec{w} \right|}{\left| \vec{n} \right|. \left| \vec{w} \right|}= \dfrac{3}{2 \sqrt{5}} \Rightarrow \cos \alpha = \dfrac{\sqrt{55}}{10}.$
		}
		{
			\begin{tikzpicture}[scale=0.7,>=stealth]
				\tkzDefPoints{0/0/A, 5/0/D, -2/-2/B, 1/-2/C}
				\tkzDefShiftPoint[A](90:4){S}
				\draw [thick] [->] (0,4)--(0,5) node[right] {$z$};
				\draw [thick] [->] (5,0)--(6,0) node[above] {$y$};
				\draw [thick] [->] (-2,-2)--(-3,-3) node[above] {$x$};
				\tkzDefMidPoint(S,B)\tkzGetPoint{M}
				\tkzDefMidPoint(C,D)\tkzGetPoint{N}
				\tkzDrawSegments(S,B S,C S,D B,C C,D)
				\tkzDrawSegments[dashed](B,A A,D S,A M,N)
				\tkzDrawPoints(A,B,C,D,S,M,N)
				\tkzLabelPoints[below](A, B, C)
				\tkzLabelPoints[above right](D,N)
				\tkzLabelPoints[left](S,M)
				\tkzLabelSegments[right](A,B S,A){$a$}
				\tkzLabelSegment[below](C,B){$a$}
				\tkzLabelSegment[above](A,D){$2a$}
			\end{tikzpicture}
		}
	}
\end{ex}
\Closesolutionfile{ans}
\subsection{BÀI TẬP TRẮC NGHIỆM TỰ LUYỆN}
\TN
	\setcounter{ex}{0}
	\Opensolutionfile{ans}[ans/B3-De2-1]

\begin{ex}%[2H2N2-4]
	Trong không gian $Oxyz$, cho đường thẳng $d\colon\heva{
		& x=6+5t \\
		& y=2+t \\
		& z=1
	}$ và mặt phẳng $\left( P \right)\colon3x-2y+1=0$. Tính góc hợp bởi đường thẳng $d$ và mặt phẳng $\left( P \right)$.
	\choice
	{$30^\circ $}
	{\True $45^\circ $}
	{$60^\circ $}
	{$90^\circ $}
	\loigiai{
		Đường thẳng $d\colon\heva{
			& x=6+5t \\
			& y=2+t \\
			& z=1 \\
		}$ có vectơ chỉ phương $\overrightarrow{u}=\left( 5;1;0 \right)$.\\
		Mặt phẳng $\left( P \right)\colon3x-2y+1=0$ có vectơ pháp tuyến $\overrightarrow{n}=\left( 3;-2;0 \right)$.\\
		Gọi $\alpha $ là góc hợp bởi đường thẳng $d$ và mặt phẳng $\left( P \right)$.\\
		Khi đó: $\sin \alpha =\dfrac{\left| \overrightarrow{u}\cdot\overrightarrow{n} \right|}{\left| {\overrightarrow{u}} \right|\cdot\left| {\overrightarrow{n}} \right|}=\dfrac{\left| 5\cdot3+1\cdot\left( -2 \right)+0\cdot0 \right|}{\sqrt{5^2+1^2}\cdot\sqrt{3^2+{{\left( -2 \right)}^2}}}=\dfrac{\sqrt{2}}{2}$.\\ Suy ra $\alpha =45^\circ $.}
\end{ex}

%G:\My Drive\CODE12-2024\DE-ON-THEO BAI\2H5-TACH DE\Bai3-De2.tex
\begin{ex}%[2H2N2-4]
	Trong không gian $Oxyz$, cho ba điểm $M\left( 2; 3; -1 \right)$, $N\left( -1; 1; 1 \right)$ và $P\left( 1; m-1; 2 \right)$. Tìm $m$ để tam giác $MNP$ vuông tại $N$.
	\choice
	{\True $m=0$}
	{$m=-4$}
	{$m=2$}
	{$m=-6$}
	\loigiai{
		Ta có\\
		$\overrightarrow{NM}=\left( 3; 2; -2 \right)$, $\overrightarrow{NP}=\left( 2; m-2; 1 \right)$.\\
		Tam giác $MNP$ vuông tại $N$ khi và chỉ khi
		\begin{eqnarray*}
			& & \overrightarrow{NM}\cdot\overrightarrow{NP}=0\\
			&\Leftrightarrow & 3\cdot2+2\cdot\left( m-2 \right)-2\cdot1=0\\
			&\Leftrightarrow & m=0.
		\end{eqnarray*}
		Vậy giá trị cần tìm của $m$ là $m=0$.}
\end{ex}

%G:\My Drive\CODE12-2024\DE-ON-THEO BAI\2H5-TACH DE\Bai3-De2.tex
\begin{ex}%[2H5N2-7]
	Trong không gian $Oxyz$, tính góc giữa hai đường thẳng $d_1\colon\dfrac{x}{1}=\dfrac{y+1}{-1}=\dfrac{z-1}{2}$ và $d_2\colon\dfrac{x+1}{-1}=\dfrac{y}{1}=\dfrac{z-3}{1}$.
	\choice
	{$60^\circ $}
	{$30^\circ $}
	{$45^\circ $}
	{\True $90^\circ $}
	\loigiai{
		Ta có $\overrightarrow{u}_{d_1}=\left( 1;-1;2 \right)$ và
		$\overrightarrow{u}_{d_2}=\left( -1;1;1 \right)$ lần lượt là véc tơ chỉ phương của $d_1$ và $d_2$.\\
		$\overrightarrow{u}_{d_1}\cdot\overrightarrow{u}_{d_2}=1\cdot\left( -1 \right)+\left( -1 \right)\cdot1+2\cdot1=0\Rightarrow {d_1}\bot {d_2}\Rightarrow \left( \widehat{d_1,d_2} \right)=90^\circ $.}
\end{ex}

%G:\My Drive\CODE12-2024\DE-ON-THEO BAI\2H5-TACH DE\Bai3-De2.tex
\begin{ex}%[2H5N2-7]
	Trong không gian $Oxyz$, cho đường thẳng $d\colon\dfrac{x-4}{1}=\dfrac{y-5}{2}=\dfrac{z}{3}$ mặt phẳng $\left( \alpha \right)$ chứa đường thẳng $d$ sao cho khoảng cách từ $O$ đến $\left( \alpha \right)$ đạt giá trị lớn nhất. Khi đó góc giữa mặt phẳng $\left( \alpha \right)$ và trục $Ox$ là $\varphi $ thỏa mãn.
	\choice
	{$\sin \varphi =\dfrac{2}{3\sqrt{3}}$}
	{$\sin \varphi =\dfrac{1}{3\sqrt{3}}$}
	{$\sin \varphi =\dfrac{1}{2\sqrt{3}}$}
	{\True $\sin \varphi =\dfrac{1}{\sqrt{3}}$}
	\loigiai{
		Đường thẳng $d$ có véc tơ chỉ phương $\overrightarrow{u}=\left( 1; 2; 3 \right)$.\\
		Gọi $H$ là hình chiếu của $O$ lên $d$, $K$ là hình chiếu của $O$ lên $\left( \alpha \right)$.\\
		Ta có
		$d\left( O,\left( \alpha \right) \right)=OK\le OH\Rightarrow d\left( O,\left( \alpha \right) \right)$ lớn nhất bằng $OH$ khi $K\equiv H$.\\
		Khi đó $\left( \alpha \right)$ chứa $d$ và nhận $\overrightarrow{n}=\overrightarrow{OH}$ làm véc tơ pháp tuyến.\\
		$H\in d\Rightarrow H\left( 4+t; 5+2t; 3t \right)\Rightarrow \overrightarrow{OH}=\left( 4+t; 5+2t; 3t \right)$.\\
		Vì $OH\bot d\Rightarrow \overrightarrow{OH}\cdot\overrightarrow{u}=0\Leftrightarrow 4+t+2\left( 5+2t \right)+3\cdot3t=0\Leftrightarrow 14t+14=0\Leftrightarrow t=-1$.\\
		$\Rightarrow H\left( 3; 3; -3 \right)$, $\overrightarrow{OH}=\left( 3; 3; -3 \right)$.\\
		Trục $Ox$ có véc tơ chỉ phương  $\overrightarrow{i}=\left( 1; 0; 0 \right)$.\\
		$\sin \varphi =\dfrac{\left| \overrightarrow{i}\cdot\overrightarrow{n} \right|}{\left| \overrightarrow{i} \right|\cdot\left| \overrightarrow{n} \right|}=\dfrac{\left| 3 \right|}{\sqrt{1}\cdot\sqrt{3^2+3^2+{{\left( -3 \right)}^2}}}=\dfrac{1}{\sqrt{3}}$.}
\end{ex}

%G:\My Drive\CODE12-2024\DE-ON-THEO BAI\2H5-TACH DE\Bai3-De2.tex
\begin{ex}%[2H5N1-3]
	Trong không gian $Oxyz$, gọi $\left( P \right)$ là mặt phẳng chứa trục $Oy$ và tạo với mặt phẳng $y+z+1=0$ một góc $60^\circ$. Phương trình mặt phẳng $\left( P \right)$ là
	\choice
	{$\hoac{
			& x-z-1=0 \\
			& x-z=0 \\
		}$}
	{$\hoac{
			& x-2z=0 \\
			& x+z=0 \\
		}$}
	{$\hoac{
			& x-y=0 \\
			& x+y=0 \\
		}$}
	{\True $\hoac{
			& x-z=0 \\
			& x+z=0 \\
		}$}
	\loigiai{
		+) Do $\left( P \right)$ chứa trục $Oy$ nên phương trình $\left( P \right)$ có dạng $\left( P \right)\colon ax+cz=0$, $\left( {a^2}+c^2>0 \right)$.\\
		+) Gọi $\left(Q\right)\colon y+z+1=0$.\\ Ta có $\cos \left( \left( P \right),\,\left( Q \right) \right)=\cos 60^\circ \Leftrightarrow \dfrac{\left| c \right|}{\sqrt{a^2+c^2}\cdot\sqrt{2}}=\dfrac{1}{2}\Leftrightarrow c=\pm a$.\\
		Khi đó $\hoac{
			& \left( P \right)\colon ax+az=0 \\
			& \left( P \right)\colon ax-az=0 \\
		}\Leftrightarrow \hoac{
			& \left( P \right)\colon x+z=0 \\
			& \left( P \right)\colon x-z=0 \\
		}$ .}
\end{ex}

%G:\My Drive\CODE12-2024\DE-ON-THEO BAI\2H5-TACH DE\Bai3-De2.tex
\begin{ex}%[2H5N1-4]
	Với giá trị nào của $m$ thì đường thẳng $\left( D \right)\colon\dfrac{x+1}{2}=\dfrac{y-3}{m}=\dfrac{z-1}{m-2}$ vuông góc với mặt phẳng $\left( P \right)\colon x+3y+2z=2$.
	\choice
	{\True $6$}
	{$5$}
	{$-7$}
	{$1$}
	\loigiai{
		Vectơ chỉ phương của $\left( D \right)$ là $\overrightarrow{a}=\left( 2;m;m-2 \right)$.\\
		Vectơ pháp tuyến của $\left( P \right)$ là $\overrightarrow{n}=\left( 1;3;2 \right)$.\\
		$\left( D \right)\bot \left( P \right)\Leftrightarrow $ $\overrightarrow{a}$ và $\overrightarrow{n}$ cùng phương.\\ $2=\dfrac{m}{3}=\dfrac{m-2}{2}\Leftrightarrow m=6$.}
\end{ex}

%G:\My Drive\CODE12-2024\DE-ON-THEO BAI\2H5-TACH DE\Bai3-De2.tex
\begin{ex}%[2H5N1-4]
	Trong không gian $Oxyz$, cho mặt phẳng $\left( P \right)\colon mx+ny-2z-1=0$ và đường thẳng $\dfrac{x}{n+1}=\dfrac{y}{m}=\dfrac{1-z}{1}$ với $m\ne 0$, $m\ne -1$. Khi $\left( P \right)\bot d$ thì tổng $m+n$ bằng bao nhiêu?
	\choice
	{$m+n=2$}
	{\True $m+n=-2$}
	{$m+n=-\dfrac{1}{2}$}
	{$m+n=-\dfrac{2}{3}$}
	\loigiai{
		Sử dụng tỷ lệ thức $\dfrac{m}{n+1}=\dfrac{n}{m}=\dfrac{-2}{-1}\Rightarrow \dfrac{m+n}{n+1+m}=2\Rightarrow m+n=-2$.}
\end{ex}

%G:\My Drive\CODE12-2024\DE-ON-THEO BAI\2H5-TACH DE\Bai3-De2.tex
\begin{ex}%[2H5N2-7]
	Trong không gian $Oxyz$, cho hai đường thẳng $d_1:\dfrac{x}{1}=\dfrac{y+1}{-1}=\dfrac{z-1}{2}$ và $d_2:\dfrac{x+1}{-1}=\dfrac{y}{1}=\dfrac{z-3}{1}$. Góc giữa hai đường thẳng đó bằng
	\choice
	{\True $90^\circ $}
	{$60^\circ $}
	{$30^\circ $}
	{$45^\circ $}
	\loigiai{
		Đường thẳng $d_1$ có véctơ chỉ phương $\overrightarrow{u}_1=\left( 1;-1;2 \right)$.\\
		Đường thẳng $d_2$ có véctơ chỉ phương $\overrightarrow{u}_2=\left( -1;1;1 \right)$.\\
		Gọi $\alpha$ là góc giữa hai đường thẳng trên.\\
		Khi đó ta có $\cos \alpha =\left| \cos \left( \overrightarrow{u}_1,\overrightarrow{u}_2 \right) \right|=\dfrac{\left| 1\cdot\left( -1 \right)+\left( -1 \right)\cdot1+2\cdot1 \right|}{\sqrt{1^2+{{\left( -1 \right)}^2}+2^2}\cdot\sqrt{{{\left( -1 \right)}^2}+1^2+1^2}}=0$.\\$\Rightarrow \left( \widehat{d_1,d_2} \right)=90^\circ $.}
\end{ex}

%G:\My Drive\CODE12-2024\DE-ON-THEO BAI\2H5-TACH DE\Bai3-De2.tex
\begin{ex}%[2H5N2-7]
	Trong không gian $Oxyz$, cho đường thẳng $\Delta \colon x=\dfrac{y}{2}=\dfrac{z-1}{3}$ và mặt phẳng $\left( P \right)\colon4x+2y+z-1=0$. Khi đó khẳng định nào sau đây là đúng?
	\choice
	{\True Góc tạo bởi $\left( \Delta \right)$ và $\left( P \right)$ lớn hơn $30^\circ $}
	{$ \Delta \parallel\left( P \right)$}
	{$ \Delta\bot \left( P \right)$}
	{$ \Delta \subset \left( P \right)$}
	\loigiai{
		Ta có $\sin \left( \widehat{\Delta ,\left( P \right)} \right)=\dfrac{11}{7\sqrt{6}}>\dfrac{1}{2}$.\\ Suy ra góc tạo bởi $ \Delta $ và $\left( P \right)$ lớn hơn $30^\circ $.}
\end{ex}

%G:\My Drive\CODE12-2024\DE-ON-THEO BAI\2H5-TACH DE\Bai3-De2.tex
\begin{ex}%[2H2N2-4]
	Trong không gian $Oxyz$, cho mặt phẳng $\left( P \right)\colon3x+4y+5z-8=0$ và đường thẳng $d\colon\heva{
		& x=2-3t \\
		& y=-1-4t \\
		& z=5-5t \\
	}$. Góc giữa đường thẳng $d$ và mặt phẳng $\left( P \right)$ là
	\choice
	{\True $90^\circ $}
	{$45^\circ $}
	{$60^\circ $}
	{$30^\circ $}
	\loigiai{
		Mặt phẳng $\left( P \right)$ có một véc tơ pháp tuyến là $\overrightarrow{n}=\left( 3;4;5 \right)$.\\
		Đường thẳng $d$ có một véc tơ chỉ phương là $\overrightarrow{u}=\left( -3;-4;-5 \right)$.\\
		Ta có $\overrightarrow{n}=-\overrightarrow{u}\Rightarrow d\bot \left( P \right)$ nên góc $90^\circ $}
\end{ex}

%G:\My Drive\CODE12-2024\DE-ON-THEO BAI\2H5-TACH DE\Bai3-De2.tex
\begin{ex}%[2H5N2-7]
	Trong không gian $Oxyz$, cho hai đường thẳng $d_1\colon\heva{
		& x=-1-t \\
		& y=3+4t \\
		& z=3+3t \\
	}$ và $d_2\colon\dfrac{x}{1}=\dfrac{y+8}{-4}=\dfrac{z+3}{-3}$. Tính góc hợp bởi đường thẳng $d_1$ và $d_2$.\\
	\choice
	{$90^\circ $}
	{$60^\circ $}
	{\True $0^\circ $}
	{$30^\circ $}
	\loigiai{
		Ta có đường thẳng $d_1\colon \heva{
			& x=-1-t \\
			& y=3+4t \\
			& z=3+3t \\
		}$ có vectơ chỉ phương là $\overrightarrow{u_1}=\left( -1; 4; 3 \right)$,\\
		đường thẳng $d_2\colon\dfrac{x}{1}=\dfrac{y+8}{-4}=\dfrac{z+3}{-3}$ có vectơ chỉ phương là $\overrightarrow{u_2}=\left( 1;-4;-3 \right)$.\\
		Vì $\overrightarrow{u_1}=\left( -1;4;3 \right)$ và $\overrightarrow{u_2}=\left( 1;-4;-3 \right)$ cùng phương với nhau nên góc hợp bởi đường thẳng $d_1$ và $d_2$ bằng $0^\circ $.}
\end{ex}

%G:\My Drive\CODE12-2024\DE-ON-THEO BAI\2H5-TACH DE\Bai3-De2.tex
\begin{ex}%[2H5N2-7]
	Trong không gian $Oxyz$, cho mặt phẳng $(P)\colon -\sqrt{3}x+y+1=0$. Tính góc tạo bởi $(P)$ với trục $Ox$?
	\choice
	{$120^\circ$}
	{$30^\circ$}
	{$150^\circ$}
	{\True $60^\circ$}
	\loigiai{
		Mặt phẳng $(P)$ có véc tơ pháp tuyến $\overrightarrow{n}=(-\sqrt{3};1;0)$\\
		Trục $Ox$ có có véc tơ pháp tuyến $\overrightarrow{i}=(1;0;0)$.\\
		Góc tạo bởi $(P)$ với trục $Ox$\\
		$\sin((P),Ox)=\left| \cos((P), Ox) \right|=\dfrac{\left| \overrightarrow{n}\cdot\overrightarrow{i} \right|}{\left| \overrightarrow{n} \right|\cdot\left| \overrightarrow{i} \right|}=\dfrac{\left| -\sqrt{3}\cdot1+1\cdot0+0\cdot0 \right|}{\sqrt{3+1}\cdot\sqrt{1}}=\dfrac{\sqrt{3}}{2}$.\\
		Vậy góc tạo bởi $(P)$ với trục $Ox$ bằng $60^\circ$.}
\end{ex}
	\Closesolutionfile{ans}

\TNTF
	\setcounter{ex}{0}
	\Opensolutionfile{ans}[ans/B3-De2-2]

\begin{ex}%[2H5N1-3]
	Trong không gian $Oxyz$, cho hai đường thẳng $d_1\colon\heva{
		& x=2+t \\
		& y=-1+t \\
		& z=3 \\
	}$ và $d_2\colon\heva{
		& x=1-t \\
		& y=2 \\
		& z=-2+t \\
	}$. Xét tính đúng sai của các khẳng định sau
	\choiceTF
	{Đường thẳng $d_1$ có một vectơ chỉ phương là $\overrightarrow{u_1}=\left( 1; 1; 3 \right)$}
	{\True Góc giữa hai đường thẳng $d_1$ và $d_2$ bằng $60^\circ $}
	{\True Đường thẳng $d_2$ có một vectơ chỉ phương là $\overrightarrow{u_2}=\left( -1; 0; 1 \right)$}
	{Giá trị cosin của góc giữa hai đường thẳng $d_1$ và $d_2$ bằng $-\dfrac{1}{2}$}
	
	\loigiai{
		\begin{itemchoice}
			\itemch Sai. Đường thẳng $d_1$ có một vectơ chỉ phương là $\overrightarrow{u_1}=\left( 1; 1; 0 \right)$.
			\itemch Đúng. Ta có $\cos \left( {d_1}, d_2 \right)=\dfrac{\left| 1\cdot\left( -1 \right)+0+0 \right|}{\sqrt{2}\cdot\sqrt{2}}=\dfrac{1}{2}\Rightarrow \left( d_1, d_2 \right)=60^\circ $.
			\itemch Đúng. Đường thẳng $d_2$ có một vectơ chỉ phương là $\overrightarrow{u_2}=\left( -1; 0; 1 \right)$.
			\itemch Sai. Ta có $\cos \left( {d_1},\,d_2 \right)=\dfrac{\left| 1\cdot\left( -1 \right)+0+0 \right|}{\sqrt{2}\cdot\sqrt{2}}=\dfrac{1}{2}$.
		\end{itemchoice}
	}
\end{ex}

%G:\My Drive\CODE12-2024\DE-ON-THEO BAI\2H5-TACH DE\Bai3-De2.tex
\begin{ex}%%[2H5N2-7]
	Trong không gian $Oxyz$, cho mặt phẳng $\left( P \right)\colon2x+y+2z-1=0$ và hai điểm $A\left( 1; -1; 2 \right)$, $B\left( 0; 1; -1 \right)$. Xét tính đúng sai của các khẳng định sau
	\choiceTF
	{Giá trị cosin của góc giữa đường thẳng $AB$ và mặt phẳng $\left( P \right)$ bằng $\dfrac{2}{\sqrt{21}}$}
	{Đường thẳng $AB$ vuông góc với mặt phẳng $\left( P \right)$}
	{\True Mặt phẳng $\left( OAB \right)$ có một vectơ pháp tuyến là $\overrightarrow{n}=\left( -1; 1; 1 \right)$}
	{\True Giá trị cosin của góc giữa mặt phẳng $\left( OAB \right)$ và mặt phẳng $\left( P \right)$ bằng $\dfrac{\sqrt{3}}{9}$}
	\loigiai{
		\begin{itemchoice}
			\itemch Sai. Ta có $\sin \left( AB, \left( P \right) \right)=\dfrac{\left| -1\cdot2+2\cdot1+\left( -3 \right)\cdot2 \right|}{\sqrt{14}\cdot3}=\dfrac{\sqrt{7}}{7}$.\\ $\Rightarrow \cos \left( d, \left( P \right) \right)=\sqrt{1-\dfrac{1}{7}}=\dfrac{\sqrt{42}}{7}$.
			\itemch Sai. Đường thẳng $AB$ có một véctơ chỉ phương $\overrightarrow{u}=\overrightarrow{AB}=\left( -1; 2; -3 \right)$, ta thấy véctơ $\overrightarrow{u}$ không cùng phương với véctơ $\overrightarrow{n}$ nên $AB$ không vuông góc với mặt phẳng $\left( P \right)$.
			\itemch Đúng. Ta có $A\left( 1; -1; 2 \right)\Rightarrow \overrightarrow{OA}=\left( 1; -1; 2 \right)$; $B\left( 0; 1; -1 \right)\Rightarrow \overrightarrow{OB}=\left( 0; 1; -1 \right).$\\
			Do đó, mặt phẳng $\left( OAB \right)$ có một vectơ pháp tuyến là $\overrightarrow{n}=\left[ \overrightarrow{OA},\overrightarrow{OB} \right]=\left( -1; 1;  \right)$.
			\itemch Đúng. Mặt phẳng $\left( OAB \right)$ có một vectơ pháp tuyến là $\overrightarrow{n}=\left( -1; 1 ; 1 \right)$ và mặt phẳng $\left( P \right)$ có một vectơ pháp tuyến là $\overrightarrow{n_1}=\left( 2; 1; 2 \right)$.\\ $\Rightarrow \cos \left( \left( OAB \right),\left( P \right) \right)=\dfrac{\left| 2\cdot\left( -1 \right)+1\cdot1+2\cdot1 \right|}{3\cdot\sqrt{3}}=\dfrac{\sqrt{3}}{9}.$
		\end{itemchoice}
	}
\end{ex}

%G:\My Drive\CODE12-2024\DE-ON-THEO BAI\2H5-TACH DE\Bai3-De2.tex
\begin{ex}%[2H5N2-7]
	Trong không gian $Oxyz$, cho đường thẳng $d\colon\dfrac{x}{1}=\dfrac{y}{-2}=\dfrac{z}{1}$ và mặt phẳng $\left( P \right)\colon5x+11y+2z-4=0$. Xét tính đúng sai của các khẳng định sau
	\choiceTF
	{\True Đường thẳng $d$ có một vectơ chỉ phương là $\overrightarrow{u}=\left( 1; -2; 1 \right)$}
	{Mặt phẳng $\left( P \right)$ có một vectơ pháp tuyến là $\overrightarrow{n}=\left( -5; -11; 2 \right)$}
	{Giá trị cosin của góc giữa đường thẳng $d$ và mặt phẳng $\left( P \right)$ bằng $\dfrac{1}{2}$}
	{\True Góc giữa đường thẳng $d$ và mặt phẳng bằng $30^\circ $}
	\loigiai{
		\begin{itemchoice}
			\itemch Đúng. Đường thẳng $d$ có một vectơ chỉ phương là $\overrightarrow{u}=\left( 1; -2; 1 \right)$.
			\itemch Sai. Mặt phẳng $\left( P \right)$ có một vectơ pháp tuyến là $\overrightarrow{n}=\left( 5; 11; 2 \right)$.
			\itemch Sai. Ta có $\sin \left( d, \left( P \right) \right)=\dfrac{\left| 1\cdot5+\left( -2 \right)\cdot11+1\cdot2 \right|}{\sqrt{6}\cdot5\sqrt{6}}=\dfrac{1}{2}\Rightarrow \cos \left( d, \left( P \right) \right)=\dfrac{\sqrt{3}}{2}$.
			\itemch Đúng. Ta có $\sin \left( d, \left( P \right) \right)=\dfrac{\left| 1\cdot5+\left( -2 \right)\cdot11+1\cdot2 \right|}{\sqrt{6}\cdot5\sqrt{6}}=\dfrac{1}{2}\Rightarrow \left( d, \left( P \right) \right)=30^\circ $.
		\end{itemchoice}
	}
\end{ex}

%G:\My Drive\CODE12-2024\DE-ON-THEO BAI\2H5-TACH DE\Bai3-De2.tex
\begin{ex}%[2H5N2-7]
	Trong không gian $Oxyz$, cho mặt phẳng $\left( P \right)\colon3x+4y+5z+2=0$ và đường thẳng $d$ là giao tuyến của hai mặt phẳng $\left( \alpha \right)\colon x-2y+1=0$ và $\left( \beta \right)\colon x-2z-3=0$. Xét tính đúng sai của các khẳng định sau
	\choiceTF
	{\True Mặt phẳng $\left( P \right)$ có một vectơ pháp tuyến là $\overrightarrow{n}=\left( 3; 4; 5 \right)$}
	{Góc giữa đường thẳng $d$ và mặt phẳng $\left( P \right)$ bằng $30^\circ $}
	{Đường thẳng $d$ có một vectơ chỉ phương là $\overrightarrow{u}=\left( 2\,;-1\,;1 \right)$}
	{\True Đường thẳng $d$ cắt mặt phẳng $\left( P \right)$ tại $A\left( \dfrac{7}{15}; \dfrac{11}{15}; -\dfrac{19}{15} \right)$}
	\loigiai{
		\begin{itemchoice}
			\itemch Đúng. Mặt phẳng $\left( P \right)$ có một vectơ pháp tuyến là $\overrightarrow{n}=\left( 3; 4; 5 \right)$.
			\itemch Sai. Ta có $\sin \left( d, \left( P \right) \right)=\dfrac{\left| 4\cdot3+2\cdot4+2\cdot5 \right|}{2\sqrt{6}\cdot5\sqrt{2}}=\dfrac{\sqrt{3}}{2}\Rightarrow \sin \left( d, \left( P \right) \right)=60^\circ $. 
			\itemch Sai.  Mặt phẳng $\left( \alpha \right)$ có một vectơ pháp tuyến là $\overrightarrow{n_1}=\left( 1; -2; 0 \right)$.\\
			Mặt phẳng $\left( \beta \right)$ có một vectơ pháp tuyến là $\overrightarrow{n_2}=\left( 1; 0; -2 \right)$.\\
			$\Rightarrow d$ có một vectơ chỉ phương là $\overrightarrow{u}=\left[ \overrightarrow{n_1}, \overrightarrow{n_2} \right]=\left( 4; 2; 2 \right)$.
			\itemch Đúng. Ta có\\
			$\heva{
				& x-2y+1=0 \\
				& x-2z-3=0 \\
				& 3x+4y+5z+2=0 \\
			}\Leftrightarrow \heva{
				& x=\dfrac{7}{15} \\
				& y=\dfrac{11}{15} \\
				& z=-\dfrac{19}{15} \\
			}$.\\ Suy ra đường thẳng $d$ cắt mặt phẳng $\left( P \right)$ tại $A\left( \dfrac{7}{15}; \dfrac{11}{15}; -\dfrac{19}{15} \right)$.
		\end{itemchoice}
	}
\end{ex}

\Closesolutionfile{ans}
\TNSA

	\setcounter{ex}{0}
	\Opensolutionfile{ans}[ans/B3-De2-3]
	
\begin{ex}%[2H5H2-7]
	Trong hệ tọa độ $Oxyz$, một vật chuyển động theo quĩ đạo là một đường thẳng. Tại thời điểm ban đầu, vật ở vị trí điểm $A(1;5;0)$, sau $10$ phút vật ở vị trí điểm $B(101;205;1250)$. Hỏi vật chuyển động theo phương hợp với mặt đất góc bao nhiêu độ ( giả sử mặt đất là mặt phẳng $Oxy$, kết quả làm tròn đến hàng phần chục).\\
	\shortans[oly]{$79{,}9$}
	\loigiai{
		Ta có $\overrightarrow{AB}=(100;200;1250)$.\\
		$\sin\left( AB,\left( Oxy \right) \right)=\dfrac{\left| \overrightarrow{AB}\cdot{{\overrightarrow{n}}_{Oxy}} \right|}{\left| \overrightarrow{AB} \right|\cdot\left| {{\overrightarrow{n}}_{Oxy}} \right|}=\dfrac{1250}{\sqrt{{{100}^2}+{{200}^2}+{{1250}^2}}}$ nên $\widehat{\left( AB;(Oxy) \right)}\approx 79,9^0$.}
\end{ex}

\begin{ex}%[2H5H2-7]
	Cho hình lăng trụ tam giác đều $ABC.A^\prime B^\prime C^\prime $ có cạnh bên $2a$, góc tạo bởi $A^\prime B$ và mặt đáy là $60^\circ$. Gọi $M$ là trung điểm $BC$. Ta có $\cos\left( A^\prime C, AM \right)=\dfrac{\sqrt{a}}{b}$ với $\dfrac{a}{b}$ là phân số tối giản, $a,b\in N$. Tổng $a+b$ bằng bao nhiêu?\\
	\shortans[oly]{$7$}
	\loigiai{
		\immini{	Chọn hệ trục tọa độ như hình vẽ.\\
			Ta có $AB=AC=BC=\dfrac{2a}{\tan 60^\circ}=\dfrac{2a}{\sqrt{3}}$.\\$\Rightarrow MC=\dfrac{BC}{2}=\dfrac{a}{\sqrt{3}}$.\\
			$AM=\dfrac{AB\sqrt{3}}{2}=a$.\\Khi đó: $M\left( 0; 0; 0 \right)$, $A\left( 0; a; 0 \right)$, $C\left( \dfrac{a}{\sqrt{3}}; 0; 0 \right)$, $A^\prime \left( 0; a; 2a \right)$.\\
			Ta có $\overrightarrow{A^\prime C}=\left( \dfrac{a}{\sqrt{3}} ;-a; -2a \right)$ $\Rightarrow A^\prime C=\dfrac{4a}{\sqrt{3}}$.\\
			$\overrightarrow{AM}=\left( 0; -a; 0 \right)$ $\Rightarrow AM=a$.\\
			Khi đó có $\cos \left( A^\prime C, AM \right)=\dfrac{\left| \overrightarrow{A^\prime C}\cdot\overrightarrow{AM} \right|}{\left| \overrightarrow{A^\prime C} \right|\cdot\left| \overrightarrow{AM} \right|}=\dfrac{\sqrt{3}}{4}$.\\ Vậy $a+b=7$.}{	\begin{tikzpicture}[scale=0.7, font=\footnotesize,line join=round, line cap=round, >=stealth]
				\path
				(0:0) coordinate (B)
				(-45:3) coordinate (C)
				(0:5) coordinate (A)
				\foreach \x in {A,B,C}{(\x)+(90:5.5) coordinate (\x')}
				($(B)!0.5!(C)$) coordinate (M)
				($(B')!0.5!(C')$) coordinate (M')
				($(M)!1.5!(M')$) coordinate (T1)
				($(M)!1.5!(C)$) coordinate (T2)
				($(M)!1.5!(A)$) coordinate (T3)
				;
				\draw[dashed] (A)--(B) (A)--(M) (C)--(A')--(B);
				\draw 
				(A)--(C) (B')--(B)--(C)
				(A')--(C')--(B')--cycle  (C)--(C') (A)--(A') 
				;
				\draw[thick,->] (M)--(T1) ;
				\draw[thick,->] (A)--(T3) ;
				\draw[thick,->] (C)--(T2) ;
				\foreach \x/\g in {A/-10,B/-90,C/180,A'/0,B'/180,C'/90,M/180}\draw[fill=white] (\x) circle (.03) +(\g:.3) node{$\x$};
				\draw[right]  (T1) node{$z$} (T3) node{$y$} (T2) node{$x$};
		\end{tikzpicture}}
	}
\end{ex}

\begin{ex}%[2H5H2-7]
	Trong không gian $Oxyz$, cho đường thẳng $d\colon\dfrac{x+1}{2}=\dfrac{y-1}{2}=\dfrac{z+2}{1}$ và mặt phẳng $(P)\colon3x+my-1=0$ ($m$ là tham số ).Tìm $m$ để đường thẳng $d$ tạo với mặt phẳng $\left( P \right)$ góc $\alpha $ thỏa mãn $\sin\alpha =\dfrac{2}{3}$.\\
	\shortans[oly]{$0$}
	\loigiai{
		Ta có đường thẳng $d$ có véc tơ chỉ phương $\overrightarrow{u_d}=(2; 2; 1)$.\\Mặt phẳng $\left(P\right)$ có véc tơ pháp tuyến  $\overrightarrow{n}_P=(3; m; 0)$.\\ Suy ra $\sin\left( d,\left( P \right) \right)=\dfrac{\left| 6+2m \right|}{3\cdot\sqrt{9+m^2}}$.\\
		Theo giả thiết ta có
		\begin{eqnarray*}
			& & \sin\left( d,\left( P \right) \right)=\dfrac{2}{3}\\
			&\Leftrightarrow & \dfrac{\left| 6+2m \right|}{3\cdot\sqrt{9+m^2}}=\dfrac{2}{3}\\
			&\Leftrightarrow & 4m^2+24m+36=4m^2+36 \\
			&\Leftrightarrow & m=0.
		\end{eqnarray*}
	}
\end{ex}

\begin{ex}%[2H5H2-7]
	Trong không gian, cho mặt phẳng $(P)$ có phương trình $ax+by+cz-1=0$ với $c<0$ đi qua hai điểm $A(0; 1; 0)$, $B(1; 0; 0)$ tạo với $\left( Oyz \right)$ một góc $60^\circ$. Khi đó $a+b-\sqrt{2}c$ bằng\\
	\shortans[oly]{$4$}
	\loigiai{
		Mặt phẳng $\left( P \right)$ đi qua $A$, $B$ nên $a=b=1\quad (1)$ .\\
		Ta có $\cos ((P),(Oyz))=\dfrac{\left| a \right|}{\sqrt{a^2+b^2+c^2}\cdot\sqrt{1}}=\dfrac{1}{2}\quad(2)$.\\
		Thay (1) vào (2) ta được:
		\begin{eqnarray*}
			& & \dfrac{1}{\sqrt{2+c^2}}=\dfrac{1}{2}\\
			&\Leftrightarrow & \sqrt{2+c^2}=2\\
			&\Leftrightarrow & c^2=2\\
			&\Leftrightarrow & \hoac{&c=\sqrt{2} \quad\left(\text{Loại}\right)\\&c=-\sqrt{2}}\\
			&\Leftrightarrow & c=-\sqrt{2}
		\end{eqnarray*}
		Vậy $a+b-\sqrt{2}c=4$.}
\end{ex}

\begin{ex}%[2H5H2-7]
	Có hai bức tường hình vuông cạnh $5m$, vuông góc với nhau và cùng vuông góc với mặt đất, hai mặt tường giao nhau tại cột $d$. Trên cột $d$ có một điểm $A$ cách mặt đất $2m$. Có một chiếc cột cao $1m$ đặt vuông góc với mặt đất, khoảng cách từ chân cột đến mỗi bức tường là $1$m. Người ta muốn căng một chiếc bạt phẳng hình tam giác đi qua điểm $A$ và đầu cột, hai đầu mút $M$, $N$ thuộc hai chân tường sao cho diện tích bạt bé nhất. Hỏi phải căng chiếc bạt hợp với mặt đất góc bao nhiêu độ ( Kết quả làm tròn đến hàng phần chục).\\
	\shortans[oly]{$54{,}7$}
	\loigiai{
		\immini{	Đặt hệ trục $Oxyz$ như hình vẽ.\\
			Ta có $A(0;0;2)$, $I(1;1;1)$. Gọi $M(m;0;0)$; $N(0;n;0)$ với $m>0$, $n>0$.\\
			Phương trình mặt phẳng $(AMN)\colon\dfrac{x}{m}+\dfrac{y}{n}+\dfrac{z}{2}=1$.\\
			Vì mặt phẳng $\left( AMN \right)$ đi qua điểm $I(1;1;1)$ nên ta có $\dfrac{1}{m}+\dfrac{1}{n}+\dfrac{1}{2}=1\Rightarrow \dfrac{1}{m}+\dfrac{1}{n}=\dfrac{1}{2}$.\\
			Ta có mặt phẳng $\left(AMN\right)$ có véc tơ pháp tuyến ${\overrightarrow{n}}_{\left( AMN \right)}=\left( \dfrac{1}{m};\dfrac{1}{n};\dfrac{1}{2} \right)$,\\mặt phẳng $\left(OMN\right)$ có véc tơ pháp tuyến${{\overrightarrow{n}}_{\left( OMN \right)}}=\left( 0;0;1 \right)$.\\$\Rightarrow \cos\left( \left( AMN \right),\left( OMN \right) \right)=\dfrac{\dfrac{1}{2}}{\sqrt{\dfrac{1}{m^2}+\dfrac{1}{n^2}+\dfrac{1}{4}}}$.}{	\begin{tikzpicture}[scale=0.7, font=\footnotesize,line join=round, line cap=round, >=stealth]
				\path
				(0:0) coordinate (D)
				(0:4) coordinate (C)
				(-135:4) coordinate (A)
				+(C) coordinate (B)
				\foreach \x in {A,B,C,D}{(\x)+(90:5.5) coordinate (\x')}
				($(D)!0.8!(C)$) coordinate (N)
				($(D)!0.5!(C)$) coordinate (T1)
				($(D)!0.8!(A)$) coordinate (M)
				($(D)!0.5!(A)$) coordinate (T2)+(T1) coordinate (E)
				(E)+(90:2.5) coordinate (I) (D)+(90:4) coordinate (T3)
				;
				\foreach \x/\g in {M/135,N/90,I/90}\draw[fill=white] (\x) circle (.03) +(\g:.4) node{$\x$};
				\draw (T1)--(E)--(T2) (I)--(E);
				\draw [->](D)--(C);
				\draw [->](D)--(A);
				\draw [->](D)--(D');
				\draw[right] (C) node{$y$} (D') node{$z$} (A) node{$x$} ;
				\draw[fill=white] (T3) circle (.03) +(180:.4) node{$A$};
		\end{tikzpicture}			}
		Mà ${S_{AMN}}\cdot \cos\left( \left( AMN \right),\left( OMN \right) \right)={S_{OMN}}\Rightarrow {S_{AMN}}=mn\sqrt{\dfrac{1}{m^2}+\dfrac{1}{n^2}+\dfrac{1}{4}}=\dfrac{\sqrt{\dfrac{1}{m^2}+\dfrac{1}{n^2}+\dfrac{1}{4}}}{\dfrac{1}{m}\cdot\dfrac{1}{n}}$.
		Đặt $\dfrac{1}{m}=a$, $\dfrac{1}{n}=b$ thì $a+b=\dfrac{1}{2}$ và ${S_{AMN}}=\sqrt{\dfrac{a^2+b^2+\dfrac{1}{4}}{a^2b^2}}=\sqrt{\dfrac{2a^2+2b^2+2ab}{a^2{b^2}}}$.\\
		Theo bất đẳng thức Cosi ta có $2a^2+2b^2+2ab\ge 3\sqrt[3]{8a^3{b^3}}=6ab$.\\
		Suy ra ${S_{AMN}}\ge \sqrt{\dfrac{6}{ab}}$.\\ Mà $a+b\ge 2\sqrt{ab}\Rightarrow \dfrac{1}{2}\ge 2\sqrt{ab}\Rightarrow ab\le \dfrac{1}{16}$ nên ${S_{AMN}}\ge \sqrt{96}$.\\
		Dấu bằng xảy ra khi $a=b=\dfrac{1}{4}$.\\ Thay vào (*) tc có $\cos\left( \left( AMN \right),\left( OMN \right) \right)=\dfrac{1}{\sqrt{3}}$ nên $\widehat{\left( \left( AMN \right),\left( OMN \right) \right)}\approx 54,7^\circ$\\
	}
\end{ex}

\begin{ex}%[2H5H2-7]
	Trong không gian, tìm $m$ để số đo góc giữa hai đường thẳng $d_1$, $d_2$ bằng $60^\circ$ biết $d_1\colon\heva{
		& x=1+t \\
		& y=1-t \\
		& z=-3+\sqrt{2}t \\
	}$, $d_2 \colon\heva{
		& x=2+mt \\
		& y=3+t \\
		& z=\sqrt{2}t \\
	}$.\\
	\shortans[oly]{$1$}
	\loigiai{
		Ta có $\cos \alpha =\cos 60^\circ=\dfrac{\left| 1\cdot m-1\cdot1+2 \right|}{\sqrt{1+1+2}\sqrt{m^2+1+2}}\Leftrightarrow \left| m-1 \right|=\sqrt{m^2+1}\Leftrightarrow m=1$.}
\end{ex}

\centerline{---HẾT---}
\Closesolutionfile{ans}
%\newpage
%%=====================
%\begin{center}
%\textbf{\large BẢNG ĐÁP ÁN}
%\end{center}
%\noindent\textbf{ĐÁP ÁN PHẦN I}
%\inputansbox{10}{ans/B3-De2-1}
	
%\noindent\textbf{ĐÁP ÁN PHẦN II}
%\inputansbox[2]{2}{ans/B3-De2-2}
	
%\noindent\textbf{ĐÁP ÁN PHẦN III}
%\inputansbox[3]{6}{ans/B3-De2-3}




%%Bài 4.
\setcounter{dang}{0}
\newpage
\section{PHƯƠNG TRÌNH MẶT CẦU}
\subsection{LÝ THUYẾT CẦN NHỚ}
\subsubsection{Định nghĩa}
\begin{itemize}
	\immini{	\item [\iconCH] Trong không gian, tập hợp tất cả các điểm $M$  cách điểm $I$ cố định một khoảng không đổi $r$ $(r>0)$  cho trước được gọi là mặt cầu tâm $I$ bán kính $R$. Kí hiệu $S(I;r)$ hay viết tắt là $(S)$. Vậy $S(I;R)=\{M|IM=r\}.$
		\item [\iconCH] Nhận xét: 
		\begin{itemize}
			\item Nếu $IM=r$ thì $M$ nằm trên mặt cầu.
			\item 	Nếu $IM<r$ thì $M$ nằm trong mặt cầu.
			\item 	Nếu $I M>r$ thì $M$ nằm ngoài mặt cầu.
		\end{itemize}
	}{
		\begin{tikzpicture}
			\draw[fill=black] (0,0) circle(1pt) node[left]{$I$};
			\draw (0,0) circle(1.5);
			\draw (-1.5,0)..controls (-1.45,-0.8) and (1.45,-0.8)..(1.5,0);
			\draw[dashed] (-1.5,0)..controls (-1.45,0.8) and (1.45,0.8)..(1.5,0);
			\draw[dashed] (0,0)--(1,-0.45);
			\draw[fill=black] (1,-0.45) circle(1pt) node[below]{$M$};
			\node at (0.6,0) {$r$};
	\end{tikzpicture}}
\end{itemize}
\subsubsection{Phương trình mặt cầu}
\begin{itemize}
	\item [\iconCH] Trong không gian $Oxyz$, mặt cầu $(S)$ tâm $I(a;b;c)$ bán kính $r$ có phương trình là $$(x-a)^2+(y-b)^2+(z-c)^2=r^2.$$
	\item [\iconCH] Dạng khai triển: $x^2+y^2+z^2-2ax-2by-2cz+d=0, \text{ với } d=a^2+b^2+c^2-r^2>0.$
\end{itemize}
\subsection{PHÂN LOẠI, PHƯƠNG PHÁP GIẢI TOÁN}
\begin{dang}{Xác định tâm $I$, bán kính $r$ của mặt cầu cho trước}
	\begin{itemize}
		\item [\iconCH] \indamm{Loại 1.} Cho $(S) \colon (x-a)^2+(y-b)^2+(z-c)^2=r^2$ . Khi đó
		\begin{listEX}[1]
			\item [\ding{172}] Tâm $I\left(a;b;c\right)$ (đổi dấu số trong dấu ngoặc);
			\item [\ding{173}] Bán kính $r$ (Rút căn vế phải).
		\end{listEX}
		\item [\iconCH] \indamm{Loại 2.} Cho $(S)\colon x^2+y^2+z^2-2ax-2by-2cz+d=0$. Khi đó
		\begin{listEX}[1]
			\item [\ding{172}] Điều kiện để (*) là mặt cầu là $a^2+b^2+c^2-d > 0$;
			\item [\ding{173}] Tâm $I\left(a,b,c\right)$ (đổi dấu hệ số của $x$, $y$, $z$ và chia đôi);
			\item [\ding{174}]  Bán kính $R=\sqrt{a^2+b^2+c^2-d}$ .
		\end{listEX}
	\end{itemize}
\end{dang}
\boxmini{BÀI TẬP TỰ LUẬN}
\setcounter{vd}{0}

\begin{vd}
	Trong các phương trình sau, phương trình nào là phương trình mặt cầu? Hãy xác định tâm và bán kính (nếu là phương trình mặt cầu).
	\begin{enumEX}[a)]{2}
		\item $(x-2)^2 + y^2 + (z+1)^2 = 4$.
		\item $x^2+y^2+z^2-2x-4y+6z-2=0$.
		\item $x^2 + y^2 + z^2 - 2x + 4y + 3z + 8 = 0$.
		\item $3x^2+3y^2+3z^2+6x+12y-9z+1=0$
	\end{enumEX}
	\loigiai{
		\begin{enumEX}[a)]{1}
			\item 
			\item Mặt cầu $(S)$ có tâm $I(1;2;-3)$ và bán kính $R=\sqrt{1^2+2^2+(-3)^2+2}=4$
			\item 
			\item Ta có 
			\begin{eqnarray*}
				&&3x^2+3y^2+3z^2+6x+12y-9z+1=0 \\
				&\Leftrightarrow& x^2+y^2+z^2+2x+4y-3z+\dfrac{1}{3}=0\\
				&\Leftrightarrow& (x^2 + 2x) + (y^2 +2\cdot2\cdot y) + \left(z^2-2\cdot\dfrac{3}{2}\cdot z\right) = \dfrac{-1}{3} \\
				&\Leftrightarrow& (x + 1)^2 + (y + 2)^2 + \left(z-\dfrac{3}{2}\right)^2 = 1+4+\dfrac{9}{4}-\dfrac{1}{3}\\
				&\Leftrightarrow& (x + 1)^2 + (y + 2)^2 + \left(z-\dfrac{3}{2}\right)^2 = \dfrac{83}{12}.
			\end{eqnarray*}
			Vậy đây là phương trình mặt cầu $(S)$ tâm $I\left(-1;-2;\dfrac{3}{2}\right)$, bán kính $r=\sqrt{\dfrac{83}{12}}=\dfrac{\sqrt{249}}{6}$.
		\end{enumEX}
	}
\end{vd}

\begin{vd}
	Trong không gian $Oxyz$, tìm tất cả giá trị của tham số $m$ để các phương trình sau là phương trình mặt cầu.
	\begin{enumEX}[a)]{1}
		\item  $x^2 + y^2 + z^2 - 2(m + 2)x + 4my - 2mz + 5m^2 + 9 = 0$;
		\item  ${x^{2}+y^{2}+z^{2}+2(m+2) x-2(m-1) z+3 m^{2}-5=0}$.
	\end{enumEX}
	\loigiai{
		\begin{enumEX}[a)]{2}
			\item Gọi phương trình đã cho có dạng $x^2 + y^2 + z^2 - 2ax - 2by - 2cz + d = 0$ với $a = m + 2$, $b = -2m$, $c = m$, $d = 5m^2 + 9$.\\
			Để phương trình đã cho là phương trình mặt cầu thì
			$$a^2 + b^2 + c^2 - d > 0 \Leftrightarrow m^2 + 4m + 4 + 4m^2 + m^2 - 5m^2 - 9 > 0 \Leftrightarrow m^2 + 4m - 5 > 0 \Leftrightarrow \hoac{&m < -5\\&m > 1.}$$
			\item Phương trình đã cho là phương trình của một mặt cầu khi và chỉ khi
			\[(m+2)^{2}+(m-1)^{2}-3 m^{2}+5>0 \Leftrightarrow m^{2}-2 m-10<0 \Leftrightarrow 1-\sqrt{11}<m<1+\sqrt{11}.\]
			Do $m \in \mathbb{Z}$ nên $m \in\{-2 ;-1 ; 0 ; 1 ; 2 ; 3 ; 4\}$. Vậy có $7$ giá trị nguyên của $m$ thỏa yêu cầu bài toán.
		\end{enumEX}
	}
\end{vd}

\dongcham{10}
\boxmini{BÀI TẬP TRẮC NGHIỆM}
\setcounter{ex}{0}
\Opensolutionfile{ans}[ans/2H5-B4-d1]

\begin{ex}
	Cho mặt cầu $(S)\colon (x+1)^2+(y-2)^2+(z-1)^2=9$. Tìm tọa độ tâm $I$ và tính bán kính $R$ của $(S)$.
	\choice
	{$I(1;-2;-1)$ và $R=3$}
	{$I(1;-2;-1)$ và $R=9$}
	{\True $I(-1;2;1)$ và $R=3$}
	{$I(-1;2;1)$ và $R=9$}
	\loigiai{
		Ta có mặt cầu $(S)$ có tâm $I(-1;2;1)$ và bán kính $R=3$.
	}
\end{ex}

\begin{ex}%[2H3Y1-3]%
	Cho mặt cầu $(S)\colon (x-1)^2+(y+2)^2+z^2=9$. Mặt cầu $(S)$ có thể tích bằng
	\choice
	{\True $V=36\pi$}
	{$V=14\pi$}
	{$V=\dfrac{4}{36}\pi$}
	{$V=16\pi$}
	\loigiai{
		Mặt cầu $(S)\colon (x-1)^2+(y+2)^2+z^2=9$ có tâm là $(1;-2;0)$, bán kính $R=3$.\\
		Thể tích mặt cầu $V=\dfrac{4}{3}\pi R^3=36\pi$.}
\end{ex}

\begin{ex}
	Cho mặt cầu $(S)\colon x^2+y^2+z^2-4x-6y+8z-7=0$. Tọa độ tâm và bán kính mặt cầu $(S)$ lần lượt là
	\choice
	{$I(-2;-3;4)$, $R=6$}
	{$I(-2;-3;4)$, $R=36$}
	{$I(2;3;-4)$, $R=36$}
	{\True $I(2;3;-4)$, $R=6$}
	\loigiai{
		Ta có
		$$x^2+y^2+z^2-4x-6y+8z-7=0\Leftrightarrow(x-2)^2+(y-3)^2+(z+4)^2=36.$$
		Nên mặt cầu $(S)$ có tâm $I(2;3;-4)$ và bán kính $R=6$.}
\end{ex}

\begin{ex}
	Cho mặt cầu $(S)\colon x^2+y^2+z^2-8x+2y+1=0$. Tìm tọa độ tâm và bán kính của mặt cầu $(S)$.
	\choice
	{\True $I(4;-1;0)$, $R=4$}
	{$I(-4;1;0)$, $R=4$}
	{$I(-4;1;0)$, $R=2$}
	{$I(4;-1;0)$, $R=2$}
	\loigiai{
		Mặt cầu $(S)$ có tâm $I(4;-1;0)$ và bán kính $R=\sqrt{4^2+(-1)^2+0^2-1}=4$.
	}
\end{ex}

\begin{ex} %[Word to LaTeX 3.2]
	Cho mặt cầu $(S)\colon 2x^2+2y^2+2z^2+12x-4y+4=0$. Mặt cầu $(S)$ có đường kính $AB$. Biết điểm $A(-1;-1;0)$ thuộc mặt cầu $(S)$. Tọa độ điểm $B$ là
	\choice
	{$B(-5;3;-2)$}
	{$B(-11;5;0)$}
	{$B(-11;5;-4)$}
	{\True$B(-5;3;0)$}
	\loigiai{
		\begin{itemize}
			\item [$\bullet$] Viết lại phương trình $(S) \colon x^2+y^2+z^2+6x-2y+2=0$. Khi đó tâm của mặt cầu là $I(-3;1;0)$.
			\item [$\bullet$] Vì $AB$ là đường kính nên $I$ là trung điểm của $AB$, suy ra $B(-5;3;0)$.
	\end{itemize}}
\end{ex}

\begin{ex}%[Thi HK2, Sở GD Bình Dương, 2018]%[2H3Y1-3]%[Trần Bá Huy, 12EX-8-2018]
	Phương trình nào dưới đây là phương trình mặt cầu?
	\choice
	{$x^2+y^2-z^2+4x-2y+6z+5=0$}
	{$x^2+y^2+z^2+4x-2y+6z+15=0$}
	{\True $x^2+y^2+z^2+4x-2y+z-1=0$}
	{$x^2+y^2+z^2-2x+2xy+6z-5=0$}
	\loigiai{
		Phương trình $x^2+y^2+z^2+4x-2y+z-1=0$ là phương trình mặt cầu vì có dạng là $x^2+y^2+z^2-2ax-2by-2cz+d=0$ và thỏa $a^2+b^2+c^2-d>0$ (dễ nhận biết vì $d=-1<0$).
	}
\end{ex}

\begin{ex}%[2H3K1-3]
	Cho phương trình $x^2+y^2+z^2-2mx-2(m+2)y-2(m+3)z+16m+13=0$. Tìm tất cả các giá trị thực của $m$ để phương trình trên là phương trình của một mặt cầu.
	\choice 
	{\True $m<0$ hay $m>2$}
	{$m \leq -2$ hay $m \geq 0$}
	{$m<-2$ hay $m>0$}
	{$m \leq 0$ hay $m \geq 2$}  
	\loigiai{ 
		Phương trình đã cho là phương trình của một mặt cầu khi và chỉ khi 
		$$\begin{aligned} &\, m^2+(m+2)^2+(m+3)^2-16m-13>0 \\ 
			\Leftrightarrow &\, 3m^3-6m>0\\
			\Leftrightarrow &\, \left[\begin{aligned} &m<0\\ &m>2  \\   \end{aligned}. \right. 
		\end{aligned}$$ 
	}
\end{ex}

\begin{ex}%[TT, TTLTĐH Diệu Hiền, Cần Thơ tháng 10, 2017]%[Thọ Bùi, dự án 12EX6]%[2H3B1-3]%
	Có tất cả bao nhiêu giá trị của tham số $m$ (biết $m \in \mathbb{N}$) để phương trình $x^2 + y^2 + z^2 + 2(m-2)y - 2(m+3)z + 3m^2 + 7 = 0$ là phương trình của một mặt cầu?
	\choice
	{$2$}
	{$3$}
	{\True $4$}
	{$5$}
	\loigiai
	{
		Đồng nhất hệ số của phương trình $x^2 + y^2 + z^2 + 2(m-2)y - 2(m+3)z + 3m^2 + 7 = 0$ (*) với phương trình $x^2 + y^2 + z^2 - 2ax - 2by - 2cz + d = 0$ ta được $a = 0$, $b = 2 - m$, $c = m + 3$ và $d = 3m^2 + 7$.\\
		Phương trình (*) là phương trình của một mặt cầu khi
		\begin{align*}
			a^2 + b^2 + c^2 - d > 0 & \Leftrightarrow (2-m)^2 + (m+3)^2 - (3m^2 + 7) > 0 \\
			& \Leftrightarrow -m^2 + 2m + 6 > 0 \\
			& \Leftrightarrow 1 - \sqrt{7} < m < 1 + \sqrt{7}.
		\end{align*}
		Do $1 - \sqrt{7} < m < 1 + \sqrt{7}$ và $m \in \mathbb{N}$ nên $m \in \{ 0; 1; 2; 3 \}$.
	}
\end{ex}

\begin{ex}%[HK2 (2017-2018), THPT LÊ QUÝ ĐÔN, HÀ NỘI]%[Trần Hòa, dự án EX9]%[2H3B1-3]
	Cho mặt cầu $(S)\colon x^2 + y^2+ z^2 -2x - 4y + 4z - m = 0$ ($m$ là tham số ). Biết mặt cầu có  bán kính bằng $5$. Tìm $m$.
	\choice%36
	{$m=25$}
	{$m=11$}
	{\True $m=16$}
	{$m=-16$}
	\loigiai{
		\begin{itemize}
			\item [$\bullet$] Công thức bán kính mặt cầu là $R=\sqrt{a^2+b^2+c^2-d}=\sqrt{1+4+4+m}$.
			\item [$\bullet$] Theo giả thiết $R=5\Leftrightarrow \sqrt{1+4+4+m}=5\Leftrightarrow m=16$.
		\end{itemize}
		
	}
\end{ex}

\begin{ex} 
	Mặt cầu $(S):x^2+y^2+z^2-4mx+4y+2mz+m^2+4m=0$ có bán kính nhỏ nhất khi $m$ bằng
	\choice
	{\True$\dfrac{1}{2}$}
	{$\dfrac{1}{3}$}
	{$\dfrac{\sqrt{3}}{2}$}
	{$0$}
	\loigiai{
		\begin{itemize}
			\item [$\bullet$] Công thức bán kính mặt cầu là
			\begin{eqnarray*}
				R &=&\sqrt{a^2+b^2+c^2-d}\\
				&=& \sqrt{4m^2+4+m^2-(m^2+4m)}\\
				&=&\sqrt{4m^2-4m+4}\\
				&=& \sqrt{(2m-1)^2+3}\quad(1).
			\end{eqnarray*} 
			\item [$\bullet$] Biểu thức (1) đạt giá trị nhỏ nhất khi $m=\dfrac{1}{2}$.
	\end{itemize}}
\end{ex}
\Closesolutionfile{ans}

\begin{dang}{Lập phương trình mặt cầu và ứng dụng thực tiễn}
	\begin{itemize}
		\item [\iconCH] \indamm{Phương pháp chung:} Cần xác định được tọa độ tâm $I\left(a;b;c\right)$ và độ dài bán kính $r$.
		\item [\iconCH] \indamm{Các bài toán cơ bản:}
		\begin{listEX}[1]
			\item [\ding{172}] Mặt cầu có tâm $I\left(a;b;c\right)$ và đi qua điểm $A\left(x_A;{y_A};{z_A}\right)$ thì bán kính $$r=IA=\sqrt{\left(x_A-x_I\right)^2+\left(y_A-y_I\right)^2+\left(z_A-z_I\right)^2}.$$
			\item [\ding{173}] Mặt cầu (S) có đường kính $AB$ thì
			\begin{itemize}
				\item [$\bullet$] Tâm $I\left(a;b;c\right)$ là trung điểm của $AB$ hay $I\left(\dfrac{x_A+x_B}{2};\dfrac{y_A+y_B}{2};\dfrac{z_A+z_B}{2}\right)$.
				\item [$\bullet$] Bán kính $r=\dfrac{AB}{2}=\dfrac{\sqrt{\left(x_B-x_A\right)^2+\left(y_B-y_A\right)^2+\left(z_B-z_A\right)^2}}{2}$.
			\end{itemize}
			\item [\ding{174}] Mặt cầu có tâm $I(a;b;c)$ và tiếp xúc với $(\alpha) \colon Ax+By+Cz+D=0$ thì bán kính $$r=\mathrm{d}\left(I,(\alpha) \right)= \dfrac{\big|Aa+Bb+Cc+D\big|}{\sqrt{A^2+B^2+C^2}}.$$
			\item [\ding{175}] Mặt cầu qua bốn điểm $A$, $B$, $C$, $D$ không đồng phẳng (ngoại tiếp tứ diện $ABCD$)\\
			Gọi $(S)$ có dạng $x^2+y^2+z^2-2ax-2by-2cz+d=0$ (*)\\
			Thay tọa độ 4 điểm $A$, $B$, $C$, $D$ vào (*), ta được hệ phương trình 4 ẩn số $a$, $b$, $c$, $d$;\\
			Giải tìm $a$, $b$, $c$, $d$. Suy ra tâm $I\left(a,b,c\right)$ , bán kính $R=\sqrt{a^2+b^2+c^2-d}$.
		\end{listEX}
	\end{itemize}
\end{dang}
\boxmini{BÀI TẬP TỰ LUẬN}
\setcounter{vd}{0}

\begin{vd}
	Trong không gian $Oxyz$, viết phương trình mặt cầu $(S)$
	\begin{enumEX}[a)]{1}
		\item Có tâm $I(2;-1;0)$ và đi qua điểm $M(4;1;-2)$;
		\item Có đường kính $AB$ với $A(0;1;3)$, $B(4;-5;-1)$;
		\item Có tâm $I(1;-2;3)$ và tiếp xúc với trục $Oy$;
		\item Có tâm $I(1;2;-1)$ và tiếp xúc với $(P)\colon x-2y-2z-8=0$.
	\end{enumEX}
	\loigiai{
		\begin{itemize}
			\item[a)] Bán kính mặt cầu là $r=IM=\sqrt{(4-2)^2+(1+1)^2+(-2-0)^2}=\sqrt{12}$.
			Phương trình mặt cầu tâm $I(2;-1;0)$, bán kính $r=\sqrt{12}$ là
			$$(x-2)^2+(y+1)^2+z^2=12.$$
			\item[b)] Tâm của mặt cầu $(S)$ là trung điểm $I$ của đoạn thẳng $AB$, suy ra $I(2;-2;1)$. Bán kính mặt cầu $(S)$ là $R=\dfrac{AB}{2}=\dfrac{\sqrt{4^2+(-6)^2+(-4)^2}}{2}=\sqrt{17}$. Vậy phương trình mặt cầu $(S)$ là
			$$(x-2)^2+(y+2)^2+(z-1)^2=17. $$
			\item[c)] Gọi $M$ là hình chiếu của $I\left( 1;-2;3 \right)$ lên $Oy$, ta có: $M\left( 0;-2;0 \right)$.\\
			$\overrightarrow{IM}=\left( -1;0;-3 \right)\Rightarrow R=d\left( I,Oy \right)=IM=\sqrt{10}$ là bán kính mặt cầu cần tìm.\\
			Phương trình mặt cầu là: ${{\left( x-1 \right)}^{2}}+{{\left( y+2 \right)}^{2}}+{{\left( z-3 \right)}^{2}}=10.$
			\item [d)] 
			Mặt cầu có tâm $I(1;2;-1)$ và tiếp xúc với $(P)\colon x-2y-2z-8=0$ sẽ có bán kính là
			$$R= \mathrm{d}(I,(P))
			=\dfrac{|1-2\cdot 2 -2 \cdot (-1) -8|}{\sqrt{1^2+(-2)^2+(-2)^2}}
			=3.$$
			Vậy phương trình mặt cầu là $(x-1)^2+(y-2)^2+(z+1)^2=9$.
		\end{itemize}
	}
\end{vd}
\dongcham{22}
\begin{vd}
Viết phương trình mặt cầu ngoại tiếp tứ diện $ABCD$, biết
\begin{enumEX}[a)]{1}
	\item $A(1 ; 1 ; 0)$, $B(1 ; 0 ; 1)$, $C(0 ; 1 ; 1)$, $D(1 ; 2 ; 3)$.
	\item $A (1; 2; -4)$; $B (1; -3; 1)$, $C (2; 2; 3)$, $D (1; 0; 4)$.
\end{enumEX}
\loigiai{
	\begin{enumEX}[a)]{1}
		\item Giả sử phương trình mặt cầu có dạng: $(S): x^2+y^2+z^2-2 a x-2 b y-2 c z+d=0$ + Với $a^2+b^2+c^2-d>0$, ta có $\mathrm{A}(1 ; 1 ; 0), \mathrm{B}(1 ; 0 ; 1), \mathrm{C}(0 ; 1 ; 1), \mathrm{D}(1 ; 2 ; 3) \in(S)$ :
		$\Rightarrow\left\{\begin{array}{l}1+1+0-a-2 b-0+d=0 \\ 1+0+1-2 a-0-2 c+d=0 \\ 0+1+1-0-2 b-2 c+d=0 \\ 1+2^2+3^2-2 a-4 b-6 c+d=0\end{array}\right.$
		$\Leftrightarrow\left\{\begin{array}{l}-2 a-2 b+d=-2 \\ -2 c-2 c+d=-2 \\ -2 b-2 c+d=-2 \\ -2 a-4 b-6 c+d=-14\end{array} \Leftrightarrow\left\{\begin{array}{l}a=\frac{3}{2} \\ b=\frac{3}{2} \\ c=\frac{3}{2} \\ d=4\end{array}\right.\right.$
		
		$$
		\begin{aligned}
			& \Rightarrow(S): x^2+y^2+z^2-2 \cdot \frac{3}{2} x-2 \cdot \frac{3}{2} y-2 \cdot \frac{3}{2} z+4=0 \\
			& \Leftrightarrow x^2+y^2+z^2-3 x-3 y-3 z+4=0 .
		\end{aligned}
		$$
		
		\item Gọi phương trình mặt cầu $(S)$ :
		
		$$
		x^2+y^2+z^2-2 a x-2 b y-2 c z+d=0\left(a^2+b^2+c^2-d>0\right)
		$$
		
		
		Do mặt cầu đi qua 4 điểm $A, B, C, D$ nên tọa độ của 4 điểm thỏa mãn phương trình mặt cầu
		
		$$
		\begin{aligned}
			& \Leftrightarrow\left\{\begin{array}{l}
				-2 a-4 b+8 c+d=-21 \\
				-2 a+6 b-2 c+d=-11 \\
				-4 a-4 b-6 c+d=-17 \\
				-2 a-8 c+d=-17
			\end{array}\right. \\
			& \Leftrightarrow\left\{\begin{array}{c}
				a=-2 \\
				b=1 \\
				c=0 \\
				d=-21
			\end{array}\right.
		\end{aligned}
		$$
		
		
		Vậy phương trình mặt cầu cần tìm là
		
		$$
		x^2+y^2+z^2+4 x-2 y-21=0
		$$
		
	\end{enumEX}
}
\end{vd}
\dongcham{17}
\begin{vd}%GV:Phan Phú Quý
	Giả sử người ta biểu diễn mô phỏng của tòa nhà Ericsson Globe ở phần Khởi động trong hệ trục tọa độ $Oxyz$ bởi một mặt cầu có tâm $I$, đường kính $110$ m và $OA=85$ m như hình vẽ (đơn vị trên trục là mét). Hãy viết phương trình của mặt cầu này.\\
	\includegraphics[width=7cm]{image/2P5-B4-PhuQuy-H2}
	%\includegraphics[width=7cm]{image/5372.pdf}\\
	\begin{tikzpicture}[scale=0.5]
		\def\r{5}
		\pgfmathsetmacro\a{\r *sin(60)}
		\pgfmathsetmacro\h{\r *cos(60)}
		\def\b{1}
		%		\draw[step=1,gray,very thin]
		%		(-7,-7) grid (7,5);
		\path
		(0,0) coordinate (O)
		(O)++(90:\h) coordinate (I)
		(I)++(90:\r) coordinate (A)
		(O)++(0:\a) coordinate (M)
		(O)++(180:\a) coordinate (N);
		\draw (M) arc (-30:210:\r);
		\draw[dashed] (M) arc (0:180:\a cm and \b cm);
		\draw (M) arc (0:-180:\a cm and \b cm);
		\coordinate (B) at ($(O)+({\a*cos(-120)},{\b*sin(-120)})$);
		\coordinate (R) at ($(I)+({\r*cos(-165)},{\r*sin(-165)})$);
		\coordinate (S) at ($(I)+({\r*cos(-15)},{\r*sin(-15)})$);
		\foreach \i/\j in {O/-50,I/180,A/150}
		\draw[fill=black] (\i) node[shift=(\j:.32)]{$\i$}circle (1.2pt);
		\draw[dashed] (O)--(B) (O)--(M) (O)--(A) (R)--(S);
		\draw[->] (B)--($ (B)+({1.7*\a*cos(-120)},{1.7*\b*sin(-120)}) $) node[below right]{$ x $};
		\draw[->] (M)--($ (M)+(0:2.5) $) node[below right]{$ y $};
		\draw[->] (A)--($ (A)+(90:2) $) node[right]{$ z $};
		\draw (-8,-2)--(5,-2)--($ (S)+(0:2) $)--(S)
		(R)--($ (R)+(180:1) $)--(-8,-2);
		\draw (-5.5,-0.5) node[below]{$ \text{Mặt đất} $};
	\end{tikzpicture}
	% \centering{\textit{Hình 5.37}}
	\loigiai{
		\begin{itemize}
			\item Bán kính của mặt cầu tâm $I$ là $R=IA=\dfrac{110}{2}=55$ m.
			\item Ta có $OA=OI+IA\Rightarrow OI=OA-IA=85-55=30$ m.\\
			Vì $I\in Oz$ nên toạ độ điểm $I(0;0;30)$.
			\item Phương trình mặt cầu tâm $I(0;0;30)$ có bán kính $R=55$ m là
			$$ x^2+y^2+(z-30)^2=55^2 \text{ hay } x^2+y^2+(z-30)^2=3025.  $$ 
		\end{itemize}
	}
\end{vd}
\dongcham{7}

\begin{vd}
	Bạn Bình đố bạn Nam tìm được đường kính của quả bóng rổ, biết rằng nếu đặt quả bóng ở một góc căn phòng hình hộp chữ nhật, sao cho quả bóng chạm (tiếp xúc) với hai bức tường và nền nhà của căn phòng đó (khi đó khoảng cách từ tâm quả bóng đến hai bức tường và nền nhà đều bằng bán kính của quả bóng) thì có một điểm $M$ trên quả bóng với khoảng cách lần lượt đến hai bức tường và nền nhà là $17$ cm, $18$ cm và $21$ cm (Hình bên dưới). Hãy giúp Nam xác định đường kính của quả bóng rổ đó. Biết rằng loại bóng rổ tiêu chuẩn có đường kính từ $23$ cm đến $24{,}5$ cm.
	\begin{center}
		\includegraphics[height=4cm]{image/2P5-B4-TheUt-H3}
	\end{center}
	\loigiai{
		\immini{Xét quả bóng tiếp xúc với các bức tường và chọn hệ trục $ Oxyz $ như hình vẽ bên.\\
			Gọi $ I(a;a;a) $ là tâm của mặt cầu và $ r=a>0 $.\\
			Phương trình mặt cầu của quả bóng là $$ (S)\colon (x-a)^2+(y-a)^2+(z-a)^2=a^2 .$$
			Giả sử $ M(x;y;z) $ nằm trên mặt cầu (bề mặt của quả bóng) sao cho $ \mathrm{d}\left(M,(Oxy) \right)=21  $, $ \mathrm{d}\left(M,(Oxz) \right)=18  $, $ \mathrm{d}\left(M,(Oyz) \right)=17  $. Khi đó $ z=21, y=18, x=17 $. Khi đó ta có phương trình
			\begin{eqnarray*}
				&&(17-a)^2+(18-a)^2+(21-a)^2=a^2\\
				&\Leftrightarrow & 2a^2-112a+1054=0\\
				&\Leftrightarrow & \hoac{&a\approx 11{,}97(\text{nhận})\\&a\approx 44{,}03(\text{loại})}
			\end{eqnarray*}
			Vậy đường kính của quả bóng rổ là $ 2a\approx 23{,}94 $ cm.
		}{
			\begin{tikzpicture}[font=\fontsize{8}{8},scale=1.2]
				\def\a{-135} \def\ra{1.8}
				\def\b{-13} \def\rb{2.4}
				\def\c{90}	\def\rc{2.4}
				\def\m{40}
				\def\rM{1.6}
				\path 
				(0,0) coordinate (O)
				(\a:2.8) coordinate (A) (\a:\ra) coordinate (A')
				(\b:3.2) coordinate (B) (\b:\rb) coordinate (B')
				(\c:3.1) coordinate (C) (\c:\rc) coordinate (C')
				($(A')+(\b:\rb)$) coordinate (D)
				($(D)+(90:\rc)$) coordinate (M)
				($(M)+({180+\b}:\rb)$) coordinate (E)
				($(M)+({180+\a}:\ra)$) coordinate (F);
				\fill (O) circle(.6pt) node[shift={(138:6pt)}]{$O$};
				\draw (M)--(D)node[midway, right]{$ 21 $} (M)--(E)node[midway, below]{$ 18 $} (M)--(F)node[midway, below]{$ 17 $};
				\begin{scope}[-stealth,line width=.5pt]
					\draw (O)--(A) node[shift={(160:4pt)}]{$x$};
					\draw (O)--(B) node[above]{$y$};
					\draw (O)--(C) node[shift={(140:4pt)}]{$z$};
				\end{scope}
				\fill[red] (M) circle(1pt) node[right]{$M(x,y,z)$};
				\fill[blue] (E) circle(1pt) node[left]{$B$};
				\fill[magenta!70!blue] (F) circle(1pt) node[right]{$A$};
				\fill[cyan!50!blue] (D) circle(1pt) node[right]{$C$};
			\end{tikzpicture}
		}
	}
\end{vd}
\dongcham{8}
\boxmini{BÀI TẬP TRẮC NGHIỆM}
\setcounter{ex}{0}
\Opensolutionfile{ans}[ans/2H5-B4-d2]

\begin{ex}%[2H3Y1-3]
	Mặt cầu tâm $I(3;-1;0)$, bán kính $R=5$ có phương trình là
	\choice
	{$(x+3)^2+(y-1)^2+z^2=5$}
	{$(x-3)^2+(y+1)^2+z^2=5$}
	{\True $(x-3)^2+(y+1)^2+z^2=25$}
	{$(x+3)^2+(y-1)^2+z^2=25$}
	\loigiai{
		Mặt cầu tâm $I(3;-1;0)$, bán kính $R=5$ có phương trình là $(x-3)^2+(y+1)^2+z^2=25$.
	}
\end{ex}

\begin{ex}%[2H3B1-3]
	Phương trình mặt cầu tâm $I(2; -3; -4)$, bán kính bằng $4$ là
	\choice
	{$(x+2)^2+(y-3)^2+(z-4)^2=16$}
	{\True $(x-2)^2+(y+3)^2+(z+4)^2=16$}
	{$(x+2)^2+(y-3)^2+(z-4)^2=4$}
	{$(x-2)^2+(y+3)^2+(z+4)^2=4$}
	\loigiai
	{
	}
\end{ex}

\begin{ex}%[2H3B1-3]
	Viết phương trình mặt cầu $(S)$ có tâm $I(-1;1;-2)$ và đi qua điểm $A(2;;1;2)$.
	\choice%34
	{$(S)\colon (x-1)^2 +(y+1)^2 +(z-2)^2=5$}
	{$(S)\colon (x-2)^2 +(y-1)^2 +(z-2)^2=25$}
	{\True $(S)\colon (x+1)^2 +(y-1)^2 +(z+2)^2=25$}
	{$(S)\colon x^2 +y^2 +z^2 +2x -2y +4z + 1 = 0$}
	\loigiai{
		Bán kính mặt cầu là $R=IA=\sqrt{9+0+16}=5$.\\ Vậy phương trình mặt cầu là $(x+1)^2 +(y-1)^2 +(z+2)^2=25$.
	}
\end{ex}

\begin{ex}%[2H3B1-3]
	Mặt cầu tâm $I(-3; 0; 4)$ và đi qua điểm $A(-3; 0; 0)$ có phương trình là
	\choice
	{$(x-3)^2+y^2+(z+4)^2=4$}
	{$(x-3)^2+ y^2 + (z+4)^2=16$}
	{\True $(x+3)^2+y^2+(z-4)^2=16$}
	{$(x+3)^2+y^2+(z-4)^2=4$}
	\loigiai{
		Bán kính mặt cầu $R=IA=4.$}
\end{ex}

\begin{ex}%[2H3Y1-3]
	Phương trình mặt cầu $\left(S\right)$ đường kính $AB$ với $A\left(4; -3; 5\right)$, $B\left(2; 1; 3\right)$ là
	\choice
	{$x^2 + y^2 + z^2 + 6x + 2y - 8z - 26 = 0$}
	{\True $x^2 + y^2 + z^2 - 6x + 2y - 8z + 20 = 0$}
	{$x^2 + y^2 + z^2 + 6x -  2y + 8z - 20 = 0$}
	{$x^2 + y^2 + z^2 - 6x + 2y - 8z +  26 = 0$}
	\loigiai{ Ta có $AB = \sqrt{\left(2 - 4\right)^2 + \left(1 + 4\right)^2 + \left(3 - 5\right)^2} = 2\sqrt{6}$.\\
		Gọi $I$, $R$ là tâm và bán kính của mặt cầu $\left(S\right)$ suy ra $R = \dfrac{AB}{2} = \sqrt{6}$ và $I\left(3; - 1; 4\right)$.\\
		Khi đó phương trình mặt cầu $\left(S\right)$ là
		$$\left(x - 3\right)^2 + \left(y + 1\right)^2 + \left(z - 4\right)^2 = 6\Leftrightarrow x^2 + y^2 + z^2 - 6x + 2y - 8z + 20 = 0$$
	}
\end{ex}

\begin{ex}%[2H3B1-3]
	Cho hai điểm $A(2; 4; 1)$ và $B(-2; 2; -3)$. Phương trình mặt cầu đường kính $AB$ là
	\choice
	{$x^2+(y-3)^2+(z-1)^2=9$}
	{$x^2+(y+3)^2+(z-1)^2=9$}
	{$x^2+(y-3)^2+(z+1)^2=3$}
	{\True $x^2+(y-3)^2+(z+1)^2=9$}
	\loigiai{
		Mặt cầu đường kính $AB$ có tâm là trung điểm của đoạn thẳng $AB.$\\
		Suy ra tọa độ tâm mặt cầu là $I\left(0;3;-1\right).$ Bán kính mặt cầu: $R=\dfrac{AB}{2}=3.$}
\end{ex}



\begin{ex}%[2H3B1-3]
	Viết phương trình mặt cầu $(S)$ có tâm $I(-1;4;2)$, biết thể tích khối cầu tương ứng là $V=972\pi$.
	\choice
	{\True $(x+1)^2+(y-4)^2+(z-2)^2=81$}
	{$(x+1)^2+(y-4)^2+(z-2)^2=9$}
	{$(x-1)^2+(y+4)^2+(z-2)^2=9$}
	{$(x-1)^2+(y+4)^2+(z+2)^2=81$}
	\loigiai{
		Thể tích khối cầu $V=\dfrac{4}{3}\pi R^3=972\pi\Leftrightarrow R=9$.\\
		Phương trình mặt cầu $(S)\colon (x+1)^2+(y-4)^2+(z-2)^2=81$.
	}
\end{ex}

\begin{ex}%[2H3B1-3]
	Mặt cầu $(S)$ có tâm $I(2; 1; -1)$, tiếp xúc với mặt phẳng tọa độ $(Oyz)$. Phương trình của mặt cầu $(S)$ là
	\choice
	{$(x+2)^2+(y+1)^2+(z-1)^2=4$}
	{$(x-2)^2+(y-1)^2+(z+1)^2=1$}
	{\True $(x-2)^2+(y-1)^2+(z+1)^2=4$}
	{$(x+2)^2+(y-1)^2+(z+1)^2=2$}
	\loigiai{
		Bán kính mặt cầu: $R=d\left[I,\left(Oyz\right)\right]=\left|x_I\right|=2.$}
\end{ex}

\begin{ex}%[2H3B1-3]
	Mặt cầu có tâm $I(1; 2; -3)$ và tiếp xúc với trục $Oy$ có bán kính bằng
	\choice
	{$2$}
	{$\sqrt{5}$}
	{\True $\sqrt{10}$}
	{$\sqrt{13}$}
	\loigiai{
		Bán kính mặt cầu: $R=d\left[I,Oy\right]=\sqrt{x_I^2+z_I^2}=\sqrt{10}.$}
\end{ex}

\begin{ex}
	Trong không gian $Oxyz$, mặt cầu tâm $I(-1;0;3)$ tiếp xúc với mặt phẳng $(\alpha) \colon 4y-3z+19=0$ có phương trình là
	\choice
	{$(x-1)^2+y^2+(z+3)^2=4$}
	{$(x+1)^2+y^2+(z-3)^2=2$}
	{\True $(x+1)^2+y^2+(z-3)^2=4$}
	{$(x-1)^2+y^2+(z+3)^2=2$}
	\loigiai{
		Mặt cầu tiếp xúc với mặt phẳng $\Leftrightarrow R=\mathrm{d}\left(I,(\alpha)\right)=\dfrac{|-3 \cdot 3+19|}{\sqrt{4^2+(-3)^2}}=2$.\\
		Vậy phương trình mặt cầu là $(x+1)^2+y^2+(z-3)^2=4$.
	}
\end{ex}

\begin{ex}%[2H3K1-3]
	Viết phương trình mặt cầu $(S)$ đi qua $A(-1;2;0)$, $B(-2;1;1)$ và có tâm nằm trên trục $Oz$.
	\choice 
	{\True $x^2+y^2+z^2-z-5=0$}
	{$x^2+y^2+z^2+5=0$}
	{$x^2+y^2+z^2-x-5=0$}
	{$x^2+y^2+z^2-y-5=0$}  
	\loigiai{ 
		Tâm $I$ của mặt cầu trên trục $Oz$ có tọa độ $I(0;0;c)$.\\
		Hai điểm $A, B$ nằm trên mặt cầu nên
		$$\begin{aligned} &\,IA^2=IB^2\\ 
			\Leftrightarrow &\,1^2+2^2+c^2=2^2+1^2+(1-c)^2  \\   
			\Leftrightarrow &\,c=\frac{1}{2}.
		\end{aligned}$$ 
		Từ đó, phương trình mặt cầu là $x^2+y^2+\left(z-\dfrac{1}{2}\right)^2=1^2+2^2+\left(\dfrac{1}{2}\right)^2$ \\
		hay $x^2+y^2+z^2-z-5=0$.
	} 
\end{ex}


\begin{ex}%[HK2 (2017-2018), THPT LÊ QUÝ ĐÔN, HÀ NỘI]%[Trần Hòa, dự án EX9]%[2H3B1-3]
	Cho mặt cầu $(S)$ tâm $I$ nằm trên mặt phẳng $(Oxy)$ đi qua ba điểm $A(1;2;-4)$, $B(1;-3;1)$, $C(2;2;3)$. Tìm tọa độ điểm $I$.
	\choice%8
	{$I(2;-1;0)$}
	{$I(0;0;1)$}
	{$I(0;0;-2)$}
	{\True $I(-2;1;0)$}
	\loigiai{
		Vì $I\in (Oxy)\Rightarrow I(a;b;0)$. Ta  có $\overrightarrow{AI}=(a-1;b-2;4);\overrightarrow{BI}=(a-1;b+3;-1);\overrightarrow{CI}=(a-2;b-2;-3)$.\\
		Do $I$ là tâm cầu nên
		\begin{eqnarray*}
			& & \heva{&IA = IB\\&IA = IC}\\
			&\Leftrightarrow& 
			\heva{&(a-1)^2+(b-2)^2+4^2=(a-1)^2+(b+3)^2+1\\&(a-1)^2+(b-2)^2+4^2=(a-2)^2+(b-2)^2+9}\\
			&\Leftrightarrow& \heva{&-4b+20=6b+10\\&-2a+17=-4a+13}\\
			&\Leftrightarrow& \heva{&b=1\\&a=-2}\\
			&\Rightarrow& I(-2;1;0).
		\end{eqnarray*}
	}
\end{ex}


\begin{ex}%[HK2 (2017-2018), THPT Tân Hiệp, Kiên Giang]%[Bùi Mạnh Tiến, dự án (12EX-9)]%[2H3B1-3]
	Cho $3$ điểm $A(2;3;0)$, $B(0;-4;1)$, $C(3;1;1)$. Mặt cầu đi qua ba điểm $A,B,C$ và có tâm $I$ thuộc mặt phẳng $(Oxz)$, biết $I(a;b;c)$. Tính tổng $T=a+b+c$.
	\choice
	{$T=3$}
	{$T=-3$}
	{\True $T=-1$}
	{$T=2$}
	\loigiai{
		Gọi phương trình mặt cầu có dạng $(S)\colon x^2+y^2+z^2-2ax-2by-2cz+d=0$.\\
		Mặt cầu có tâm $I(a;b;c)$. Vì $I\in (Oxz)$ và $A,B,C\in (S)$ nên ta có hệ
		\begin{align*}
			\heva{&13-4a-6b+d=0\\&17+8b-2c+d=0\\&11-6a-2b-2c+d=0\\&b=0}\Leftrightarrow \heva{&13-4a-6b+d=0\\&4a+14b-2c=-4\\&-2a+4b-2c=2\\&b=0}\Leftrightarrow \heva{&a=-1\\&b=0\\&c=0\\&d=-17.}
		\end{align*}
		Vậy $T=a+b+c=-1+0+0=-1$.
	}
\end{ex}

\begin{ex}%[TT lần 2, Chuyên KHTN, Hà Nội 2018]%[2H3B1-3]%[Nguyện Ngô và Hồ Như Vương,12EX7]
	Cho các điểm $A(1; 0; 0)$, $B(0; 2; 0)$, $C(0; 0; -2)$. Bán kính mặt cầu ngoại tiếp hình chóp $OABC$ là
	\choice
	{$\dfrac{7}{2}$}
	{$\dfrac{1}{2}$}
	{\True $\dfrac{3}{2}$}
	{$\dfrac{5}{2}$}
	\loigiai{
		\begin{enumerate}[\faCheckSquareO]
			\item \textbf{Cách 1.}	Gọi $I(x; y; z)$ là tâm mặt cầu ngoại tiếp hình chóp $OABC$. Khi đó
			$$\heva{&OI^2=AI^2\\&OI^2=BI^2\\&OI^2=CI^2}\Leftrightarrow
			\heva{&x^2+y^2+z^2=(x-1)^2+y^2+z^2\\&x^2+y^2+z^2=x^2+(y-2)^2+z^2\\&x^2+y^2+z^2=x^2+y^2+(z+2)^2}\Leftrightarrow\heva{&x=\dfrac{1}{2}\\&y=1\\&z=-1.}$$
			Suy ra bán kính $R=OI=\dfrac{3}{2}$.
			\item \textbf{Cách 2.} 
			\begin{itemize}
				\item [$\bullet$] Gọi mặt cầu $(S)$ có dạng $x^2+y^2+z^2-2ax-2by-2cz+d=0 \quad(1)$.
				\item [$\bullet$] Thay lần lượt tọa độ 4 điểm $O$, $A$, $B$, $C$ vào (1) và giải hệ, ta tìm được $a$, $b$, $c$, $d$.
				\item [$\bullet$] Tính $R=\sqrt{a^2+b^2+c^2-d}$.
			\end{itemize}
		\end{enumerate}
		
	}
\end{ex}


\begin{ex}%[2H3K1]
	Cho điểm $D(3; 4; -2).$ Gọi $A, B, C$ lần lượt là hình chiếu vuông góc của $D$ trên các trục tọa độ $Ox, Oy, Oz.$ Gọi $(S)$ là mặt cầu ngoại tiếp tứ diện $ABCD.$ Tính diện tích mặt cầu $(S).$
	\choice
	{$\dfrac{4\sqrt{29}\pi}{3}$}
	{$\dfrac{29\sqrt{29}\pi}{6}$}
	{$116\pi$}
	{\True $29\pi$}
	\loigiai{
		Nhận xét rằng, bốn điểm $A$, $B$, $C$, $D$ là 4 trong 8 đỉnh của một hình hộp chữ nhật có đường chéo là $OD$. Suy ra $R=\dfrac{OD}{2}=\dfrac{\sqrt{29}}{2}$.\\
		Diện tích mặt cầu $S=4 \pi R^2=29\pi$.
	}
\end{ex}


\begin{dang}{Vị trí tương đối của điểm, của mặt phẳng với mặt cầu}
	\begin{enumerate}[\iconCH]
		\item \indamm{Bài toán 1:} Xét điểm $M(x_0;y_0;z_0)$ và mặt cầu $S \colon (x-a)^2+(y-b)^2+(z-c)^2-r^2=0 \quad (1)$. Thay tọa độ điểm $M$ vào vế trái của (1), nếu
		\begin{listEX}[1]
			\item [\ding{172}] Kết quả bằng $0$ thì $M \in (S)$.
			\item [\ding{173}] Kết quả ra số âm thì $M$ nằm trong $(S)$.
			\item [\ding{174}] Kết quả ra số dương thì $M$ nằm trong $(S)$.
		\end{listEX}
		\item \indamm{Bài toán 2:} Cho mặt cầu $(S)$ có tâm $I(a;b;c)$, bán kính $r$ và mặt phẳng $(P)\colon Ax+By+Cz+D=0$.
		\begin{tcolorbox}[colframe=orange!3,colback=red!3!white,boxrule=0.2mm]
			\begin{listEX}[1]
				\item [\ding{172}] Nếu $\mathrm{d}\left(I,(P) \right)=\dfrac{\bigg|Aa+Bb+Cc+D\bigg|}{\sqrt{A^2+B^2+C^2}}>r$ thì $(P)$ và $(S)$ không có điểm chung.
				\item [\ding{173}] Nếu $\mathrm{d}\left(I,(P) \right)=\dfrac{\bigg|Aa+Bb+Cc+D\bigg|}{\sqrt{A^2+B^2+C^2}}=r$ thì $(P)$ tiếp xúc $(S)$.
				\item [\ding{174}] Nếu $\mathrm{d}\left(I,(P) \right)=\dfrac{\bigg|Aa+Bb+Cc+D\bigg|}{\sqrt{A^2+B^2+C^2}}<r$ thì $(P)$ cắt $(S)$.
			\end{listEX}
		\end{tcolorbox}
	\end{enumerate}
\end{dang}
\boxmini{BÀI TẬP TỰ LUẬN}
\setcounter{vd}{0}

\begin{vd}%[2H5N3-1]
	Cho mặt cầu $(S)$ có tâm $I(2;-1;4)$ và bán kính $R=5$. Các điểm $A(3;1;5)$, $B(-1;11;14)$, $C(6;2;4)$ nằm trong, nằm trên hay nằm ngoài mặt cầu $(S)$?
	\loigiai{
		\begin{itemize}
			\item $IA=\sqrt{1^2+2^2+1^2}=\sqrt{6} \approx 2{,}45<R=5$, suy ra $I A<R$. Do đó, điểm $A$ nằm trong mặt cầu $(S)$.
			\item 
			$IB=\sqrt{(-3)^2+12^2+10^2}=\sqrt{253} \approx 15,91>5$, suy ra $I B>R$. Do đó, điểm $B$ nằm ngoài mặt cầu $(S)$.
			\item $I C=\sqrt{4^2+3^2+0^2}=5=R$. Do đó, điểm $C$ nằm trên mặt cầu $(S)$.
		\end{itemize}
	}
\end{vd}
\dongcham{5}
\begin{vd}%[2H5H3-4]   
	\immini{Trong không gian $Oxyz$ (đơn vị trên mỗi trục là mét), một router phát sóng wifi có đầu thu phát được đặt tại điểm $I(4;2;10)$.
		\begin{listEX}[1]
			\item Cho biết bán kính phủ sóng wifi là $40$ m. Viết phương trình mặt cầu $(S)$ biểu diễn ranh giới của vùng phủ sóng.
			\item Một người sử dụng máy tính tại điểm $M(6;12;0)$. Hãy cho biết điểm $M$ nằm trong hay nằm ngoài mặt cầu $(S)$ và người đó có thể sử dụng được sóng wifi của router nói trên hay không?
			\item Câu hỏi tương tự đối với người sử dụng máy tính ở điểm $N(14;6;50)$.
	\end{listEX}}
	{\begin{tikzpicture}[scale=.8,line cap=round,line join=round,font=\footnotesize,>=stealth]
			\tikzset{laptop/.pic={
					% Manhinh
					\fill[cyan!20] (-4, 2.5) -- (4, 2.5) -- (3.5, -1.5) -- (-3.5, -1.5) -- cycle;
					\draw[thick] (-4, 2.5) -- (4, 2.5) -- (3.5, -1.5) -- (-3.5, -1.5) -- cycle;
					
					% vien
					\draw[thick, black] (-4.1, 2.6) -- (4.1, 2.6) -- (3.6, -1.6) -- (-3.6, -1.6) -- cycle;
					
					% Camera
					\fill[red] (0, 2.6) circle (0.1);
					
					% Keyboard 
					\fill[gray!30] (-3.5, -1.5) -- (3.5, -1.5) -- (4, -3) -- (-4, -3) -- cycle;
					\draw[thick] (-3.5, -1.5) -- (3.5, -1.5) -- (4, -3) -- (-4, -3) -- cycle;
					
					% Base
					\fill[blue!20] (-4, -3) rectangle (4, -3.2);
					\draw[thick] (-4, -3) -- (4, -3) -- (4, -3.2) -- (-4, -3.2) -- cycle;
					
					% Touchpad
					\fill[gray!40] (-1.5, -2.5) rectangle (1.5, -2);
					\draw[thick] (-1.5, -2.5) rectangle (1.5, -2);
					
					% Keys
					\foreach \x in {-3.2,-2.7,...,3.2} {
						\foreach \y in {-2.8,-2.5,...,-1.8} {
							\fill[gray!50] (\x,\y) rectangle ++(0.4,0.3);
						}
					}
					
			}}
			\tikzset{router/.pic={
					\fill[black!80] (-3,1,0) -- (3,1,0) -- (3,-2,-1) -- (-3,-2,-1) -- cycle;
					\fill[black!60] (-3,1,0) -- (3,1,0) -- (3,1,-0.2) -- (-3,1,-0.2) -- cycle;
					\fill[black!50] (3,1,0) -- (3,-2,-1) -- (3,-2.2,-1) -- (3,1,-0.2) -- cycle;
					\fill[black!50] (-3,-2,-1) -- (3,-2,-1) -- (3,-2.2,-1) -- (-3,-2.2,-1) -- cycle;
					% Mặt trên
					\foreach \i in {-2.7,-2.4,...,2.7}
					\draw[gray!90] (\i, 1, -0.05) -- (\i, -1.8, -1.05);
					% Anten
					\draw[thick, fill=cyan] (-2.5,1,0) -- (-2.3,1,0) -- (-2.3,3,0.2) -- (-2.5,3,0.2) -- cycle;
					\draw[thick, fill=cyan] (2.5,1,0) -- (2.3,1,0) -- (2.3,3,0.2) -- (2.5,3,0.2) -- cycle;
					% LED
					\foreach \i in {-1.8,-1.4,-1.0,-0.6,-0.2,0.2,0.6,1.0,1.4,1.8}
					\fill[yellow] (\i,0.5,-0.8) circle (0.05);
					% Mặt trước
					\foreach \i in {-2.2,-1.8,-1.4,-1.0,-0.6,-0.2,0.2,0.6,1.0,1.4,1.8}
					\fill[white] (\i,-2.1,-1.05) circle (0.05);
			}}
			\path (0:0) coordinate(O) (-135:3) coordinate(x) (0:6) coordinate(y) (90:3) coordinate(z)
			(-70:2.2) coordinate(B) (-27:4.4) coordinate(A) (50:1.7) coordinate(I);
			%\fill[cyan!15] (O)--(x)--($(x)+(y)-(O)$)--(y)--cycle;
			\draw[->] (O)--(x) node[shift={(150:.2)}]{$x$};
			\draw[->] (O)--(y) node[shift={(80:.2)}]{$y$};
			\draw[->] (O)--(z) node[shift={(160:.2)}]{$z$};
			\path (1.3,-.5) pic[rotate=15,scale=.15,shift=(-150:3.5)]{laptop};
			\path (4.7,-.8) pic[rotate=-15,scale=.15,shift=(-150:3.5)]{laptop};
			\path (1,2) pic[rotate=5, scale=.2,shift=(10:.5)]{router};
			\foreach \d/\g in {B/170, A/180, O/180, I/180 } \fill (\d) circle(1pt) node[shift={(\g:.3)}]{$\d$};
			\path pic[draw,angle radius=.15cm]{right angle=x--O--y} pic[draw,angle radius=.15cm]{right angle=z--O--y} pic[draw,angle radius=.15cm]{right angle=x--O--z};
	\end{tikzpicture}}
	\loigiai{
		\begin{listEX}[1]
			\item Mặt cầu $(S)$ có tâm $I(4;2;10)$, bán kính $R=40$ nên có phương trình là
			$$(x-4)^2+(y-2)^2+(z-10)^2=1600.$$
			\item  Ta có $I M=\sqrt{2^2+10^2+(-10)^2}=\sqrt{204} \approx 14{,}3<40$, suy ra $IM<R$. Do đó, điểm $M$ nằm trong mặt cầu $(S)$.\\
			Vậy người đó có thể sử dụng được sóng wifi của router nói trên.
			\item  Ta có $I N=\sqrt{10^2+4^2+40^2}=\sqrt{1716} \approx 41{,}4>40$, suy ra $IN>R$. Do đó, điểm $N$ nằm ngoài mặt cầu $(S)$.\\
			Vậy người đó \textbf{không} thể sử dụng được sóng wifi của router nói trên.
	\end{listEX}}
\end{vd}
\dongcham{10}

\begin{vd}
	Cho mặt cầu $(S)\colon (x - 2)^2 + y^2 + (z + 1)^2=9$ và mặt phẳng $(P)\colon 2x - y - 2z - 3=0$. 
	\begin{enumEX}[a)]{1}
		\item Chứng minh rằng mặt phẳng $(P)$ cắt mặt cầu $(S)$.
		\item Biết mặt cầu $(S)$ cắt $(P)$ theo giao tuyến là đường tròn $(C)$. Tính bán kính $r$ của đường tròn $(C)$.
	\end{enumEX}
	\loigiai{
		\immini
		{
			Mặt cầu $(S)$ có tâm $I(2;0;-1)$, bán kính $R=3$.\\
			Khoảng cách từ $I$ đến mặt phẳng $(P)$ là
			$$h=\mathrm{d}(I;(P))=\dfrac{|4-0+2-3|}{3}=1.$$
			Ta có $h^2+r^2=R^2\Leftrightarrow 1+r^2=9\Leftrightarrow r=2\sqrt{2}$.
		}
		{
			\begin{tikzpicture}[scale=0.7, font=\footnotesize, line join=round, line cap=round, >=stealth]
				\tikzset{label style/.style={font=\footnotesize}}
				\coordinate (A) at (2,0);
				\coordinate (I) at (0,2);
				\coordinate (K) at (0,0);
				\coordinate (B) at (-2,0);
				\draw[dashed] (A) arc (0:180:2cm and 0.35cm);
				\draw (A) arc (0:-180:2cm and 0.35cm);
				\draw (I) circle (2.828cm);
				\filldraw (I) node[above] {$I$} circle (1pt);
				%\filldraw (K) node[above right] {$K$} circle (1pt);
				%\filldraw (A) node[right] {$A$} circle (1pt);
				%\filldraw (B) node[left] {$B$} circle (1pt);
				\draw[dashed] (I)--(A)--(B)--(I)--(K);
				\tkzLabelSegment[above](I,A){$R$}
				\tkzLabelSegment[above=-0.1](K,A){$r$}
				\tkzLabelSegment[right](I,K){$h$}
			\end{tikzpicture}
		}
	}
\end{vd}
\dongcham{14}
\boxmini{BÀI TẬP TRẮC NGHIỆM}
\setcounter{ex}{0}
\Opensolutionfile{ans}[ans/2H5-B4-d3]
\begin{ex}
	Cho điểm $M\left(1;-1;3\right)$ và mặt cầu $(S)$ có phương trình$\left(x-1\right)^2+(y+2)^2+z^2=9$. Khẳng định đúng là
	\choice
	{\True $M$ nằm ngoài $(S)$}
	{$M$ nằm trong $(S)$}
	{$M$ nằm trên$(S)$}
	{$M$ trùng với tâm của $(S)$}
	\loigiai{
		Thay tọa độ $M$ vào vế trái của phương trình mặt cầu, ta được $(1-1)^2+(-1+2)^2+(3)^2=10>9$. Suy ra $M$ nằm ngoài $(S)$}
\end{ex}

\begin{ex}%[BG-12-New-4in1, Nguyễn Cường]%[2H5H3-3]
	Cho mặt cầu $(S)\colon x^2+y^2+z^2-2x-4y-6z = 0$ và ba điểm $O(0; 0; 0)$, $A(1; 2; 3)$, $B(2; -1; -1)$. Trong số ba điểm trên số điểm nằm trên mặt cầu là
	\choice
	{$2$}
	{$0$}
	{$3$}
	{\True $1$}
	\loigiai{Lần lượt thay tọa độ các điểm $O, A, B$ vào phương trình mặt cầu $(S)$ ta chỉ thấy duy nhất điểm $O$ thuộc mặt cầu $(S)$.
	}
\end{ex}

\begin{ex}%[BG-12-New-4in1, Nguyễn Cường]%[2H5H3-3]
	Cho mặt cầu $(S): x^2+y^2+z^2-4y+6z-2=0$ và mặt phẳng $(P)\colon x+y-z+4=0$. Trong các mệnh đề sau, mệnh đề nào đúng?
	\choice
	{ $(P)$ tiếp xúc $(S)$}
	{\True $(P)$ và $(S)$ không có điểm chung}
	{$(P)$ đi qua tâm của $(S)$}
	{$(P)$ cắt $(S)$}
	\loigiai{
		$(S)$ có tâm $I(0;2;-3)$ và bán kính $R=\sqrt{15}$.\\
		Ta có $\mathrm{d}[I,(P)]=\dfrac{\left|2+3+4 \right| }{\sqrt{1+1+1}}=3\sqrt{3} > \sqrt{15}=R$ nên $(P)$ và $(S)$ không có điểm chung.
	}
\end{ex}

\begin{ex}%[2H3B2-7]%Câu 9.
	Cho mặt phẳng $(P)$ và mặt cầu $(S)$ có phương trình lần lượt là $(P)\colon 2x+2y+z-m^2+4m-5=0$; $(S)\colon x^2+y^2+z^2-2x+2y-2z-6=0$. Giá trị của $m$ để $(P)$ tiếp xúc $(S)$ là
	\choice
	{$m=5$}
	{$m=-1$}
	{\True $m=-1$ hoặc $m=5$}
	{$m=1$ hoặc $m=-5$}
	\loigiai{
		Mặt cầu $(S)\colon x^2+y^2+z^2-2x+2y-2z-6=0$ có tâm $I(1;-1;1)$ và bán kính $R=3$.\\
		$(P)$ tiếp xúc $(S)\Leftrightarrow\mathrm{d}\left(I,(P)\right)=R$
		\begin{eqnarray*}
			&\Leftrightarrow&\dfrac{\left|2\cdot 1+2\cdot (-1)+1-m^2+4m-5\right|}{\sqrt{2^2+2^2+1^2}}=3\\
			&\Leftrightarrow&\left|m^2-4m+4\right|=9\\
			&\Leftrightarrow&\hoac{&m^2-4m+4=9\\&m^2-4m+4=-9\quad \text{(vô nghiệm)}}\\
			&\Leftrightarrow& m^2-4m-5=0\Leftrightarrow\hoac{&m=-1\\&m=5.}
		\end{eqnarray*}
	}
\end{ex}

\begin{ex}%[Thi thử L3, Yên Lạc 2, Vĩnh Phúc 2018]%[Hoàng Trình,12EX6]%[2H3B2-7]%
	Trong không gian với hệ tọa độ $Oxyz$, cho mặt phẳng $(P)\colon 2x+2y+z-2=0$ và mặt cầu $(S)$ tâm $I\left(2;1;-1\right)$ bán kính $R=2$. Bán kính đường tròn giao của mặt phẳng $(P)$ và mặt cầu $(S)$ là
	\choice
	{\True $r=\sqrt{3}$}
	{$r=\sqrt{5}$}
	{$r=1$}
	{$r=3$}
	\loigiai{
		\immini{
			Gọi bán kính đường tròn giao của mặt phẳng $(P)$\\
			và mặt cầu $(S)$ là $r$.\\
			Ta có $h=\mathrm{d}(I,(P))=\dfrac{\left|2\cdot 2+2\cdot (-1)-1-2 \right| }{\sqrt{2^2+2^2+1^2}}=1$.\\
			Suy ra $r=\sqrt{2^2-1^2}=\sqrt{3}$.
		}
		{
			\begin{tikzpicture}[line width=0.6pt]
				\draw[fill=black] (0,0)coordinate(I) circle(1pt) node[above left]{$I$};
				\draw[fill=black] (0,-0.9)coordinate(H) circle(1pt);
				\node at (0,-0.9) [left]{$H$};
				\draw[dashed] (0,0)--(0,-0.9)--(1,-1.3)coordinate(M)--(0,0);
				\tkzMarkRightAngle(I,H,M)
				\node at (0.83,-0.7) {$R$};
				\node at (0.4,-1.23) {$r$};
				\draw[line width=0.6pt] (1.61,-1.01) arc [radius=1.9, start angle=-32.1, end angle=212];
				\draw[line width=0.6pt] (-1.164,-1.5) arc [radius=1.9, start angle=-127.9, end angle=-52];
				\draw[dashed,line width=0.6pt] (-1.61,-1.01) arc [radius=1.9, start angle=212, end angle=232.1];
				\draw[dashed,line width=0.6pt] (1.61,-1.01) arc [radius=1.9, start angle=-32.1, end angle=-52];
				\coordinate (A) at (-2.7,-1.5);
				\coordinate (B) at (-1.7,-0.3);
				\coordinate (C) at (2.6,-0.3);
				\coordinate (D) at ($(A)+(C)-(B)$);
				\draw (-1.84,-0.47)--(A)--(D)--(C)--(1.87,-0.3);
				\draw[line width=0.5pt,dashed] (-1.84,-0.47)--(B)--(1.87,-0.3);
				\draw[dashed] (-1.64,-0.9)..controls (-1.57,-0.2) and (1.57,-0.2)..(1.64,-0.9);
				\draw (-1.64,-0.9)..controls (-1.57,-1.6) and (1.57,-1.6)..(1.64,-0.9);
				\node at (-2.25,-1.3) {$(P)$};
				\node at (-1.9,1.3) {$(S)$};
				\draw (-2.1,-1.5)..controls (-1.95,-1.1)..(-2.34,-1.05);
			\end{tikzpicture}
		}
	}
\end{ex}

\begin{ex}%[Dự án EX-7-2019]%[Phạm Tuấn]%[2H3B2-7]%
	Trong không gian $Oxyz $, mặt cầu có phương trình $x^2+y^2+z^2-2x+2y-6z+2=0$
	cắt mặt phẳng $(Oxz)$ theo một đường tròn có bán kính bằng
	\choice
	{$3\sqrt{2}$}
	{\True $2\sqrt{2}$}
	{$5$}
	{$4\sqrt{2}$}
	\loigiai{
		Mặt cầu đã cho có tâm $I(1;-1;3)$ và bán kính $R=3$. \\
		Khoảng cách từ $I$ đến mặt phẳng $(Oxz)$ bằng $1$, do đó bán kính của đường tròn bằng \[\sqrt{3^2-1^2}=2\sqrt{2}.\]
	}
\end{ex}

\begin{ex}%[Thi thử L1, Chuyên Ngoại Ngữ, Hà Nội, 2018]%[2H3B2-7]%[Nguyễn Bình Nguyên-12Ex8]%
	Trong không gian với hệ trục tọa độ $Oxyz$, cho mặt cầu $(S) \colon (x-1)^2+(y-2)^2+(z-2)^2=9$ và mặt phẳng $(P) \colon 2x-y-2z+1=0$. Biết $(P)$ cắt $(S)$ theo giao tuyến là đường tròn có bán kính $r$. Tính $r$.
	\choice
	{$r=3$}
	{$r=2$}
	{\True $r=2\sqrt{2}$}
	{$r=\sqrt{3}$}
	\loigiai
	{Ta có $(S) \colon (x-1)^2+(y-2)^2+(z-2)^2=9$ $\Rightarrow \heva{&I(1;2;2)\\&R=3}.$\\ $d=\mathrm{d}\left(I,(\alpha)\right)=\dfrac{|2\cdot1-2-2\cdot2+1|}{\sqrt{2^2+(-1)^2+(-2)^2}}=1.$\\
		Vậy	$r=\sqrt{R^2-d^2}=2\sqrt{2}$.}
\end{ex}

\begin{ex}%[Thi thử L2, Lương Thế Vinh, Hà Nội, 2018]%[Phạm Toàn, Dự án (12EX-9)]%[2H3B2-7]%
	Trong không gian $Oxyz$ cho mặt cầu $(S)\colon x^2+y^2+z^2-6x-4y-12z=0$ và mặt phẳng $(P)\colon 2x+y-z-2=0$. Tính diện tích thiết diện của mặt cầu $(S)$ cắt bởi mặt phẳng $(P)$.
	\choice
	{$50\pi $}
	{\True $S=49\pi $}
	{$25\pi $}
	{$36\pi $}
	\loigiai{
		Mặt cầu $(S)$ có tâm $I(3;2;6)$ và bán kính $R=\sqrt{3^2+2^2+6^2}=7$. Vì $I$ thuộc $(P)$ nên $(P)$ cắt $(S)$ theo thiết diện là đường tròn có bán kính bằng $7$. Diện tích thiết diện bằng $49\pi$.
	}
\end{ex}

\begin{ex}%[Đề thi thử Tốt nghiệp THPT lần 1, Sở GD&ĐT Hà Nội, 2020]%[Nguyễn Đắc Giáp, 12EX8]%[2H3B2-7]%
	Trong không gian với hệ tọa độ $Oxyz$, cho điểm $I(2;1;1)$ và mặt phẳng $(P)\colon 2x+y+2z-1=0$. Mặt cầu $(S)$ có tâm $I$, cắt $(P)$ theo một đường tròn có bán kính $r=4$. Mặt cầu $(S)$ có phương trình là
	\choice
	{$(x-2)^2 + (y-1)^2 + (z-1)^2 =18$}
	{$(x-2)^2 + (y-1)^2 +(z-1)^2 = 2\sqrt{5}$}
	{\True $(x-2)^2 + (y-1)^2 + (z-1)^2 =20$}
	{$(x+2)^2 + (y+1)^2 + (z+1)^2 =20$}
	\loigiai{
		Ta có $$\mathrm{d}(I;(P)) = \dfrac{\left | 2\cdot 2+1+2\cdot 1-1  \right |}{\sqrt{2^2+1^2+2^2}} =2.$$
		Vì mặt cầu $(S)$ có tâm $I$, cắt $(P)$ theo một đường tròn có bán kính $r=4$ nên mặt cầu $(S)$ có bán kính
		$$R=\sqrt{r^2+\mathrm{d}^2(I,(P))} = \sqrt{4^2+2^2} = 2\sqrt{5}.$$
		Vậy phương trình mặt cầu $(S)$ là $(x-2)^2 + (y-1)^2 + (z-1)^2 =20$.
	}
\end{ex}

\begin{ex}%[HK2, Sở GD&ĐT Đà Nẵng, 2019]%[Trần Chiến, 12EX6-2020]%[2H3B2-7]%
	Trong không gian $Oxyz$, cho mặt cầu $(S)$ có tâm $I(1;2;1)$ và cắt mặt phẳng $(P)\colon 2x-y+2z+7 =0$ theo một đường tròn có đường kính bằng $8$. Phương trình mặt cầu $(S)$ là
	\choice
	{\True $(x-1)^2+(y-2)^2 +(z-1)^2=25$}
	{$(x-1)^2+(y-2)^2 +(z-1)^2=81$}
	{$(x+1)^2+(y+2)^2 +(z+1)^2=9$}
	{$(x-1)^2+(y-2)^2 +(z-1)^2=5$}
	\loigiai{
		\immini{
			Gọi $R$, $r$, $d$ lần lượt là bán kính của mặt cầu, bán kính của đường tròn và khoảng cách từ tâm $I$ đến mp$(P)$. Khi đó $R = \sqrt{r^2+d^2}$.\\
			Ta có $d= \mathrm{d}(I,(P)) =\dfrac{|2\cdot 1 -2+2\cdot 1+7|}{\sqrt{2^2+ (-1)^2+2^2}} = 3$ và $r=4$. \\
			Suy ra $R =\sqrt{3^2+4^2}=5$.\\
			Vậy phương trình mặt cầu $(S)$ là $(x-1)^2+(y-2)^2 +(z-1)^2=25$.
		}{
			\begin{tikzpicture}[scale=0.7, font=\footnotesize, line join=round, line cap=round, >=stealth]
				\tkzDefPoints{0/0/O}
				\def\a{3}
				\def\b{0.6}
				\coordinate (M) at (0:\a cm and \b cm);
				\coordinate (I) at ($(O)+(0,1.5)$);
				\tkzDrawCircle(I,M)
				\draw (M) arc (0:-180:\a cm and \b cm);
				\draw[dashed] (M) arc (0:180:\a cm and \b cm);
				\tkzDrawSegments[dashed](I,O I,M M,O)
				\tkzDrawPoints[fill=black](I,O,M)
				\tkzLabelSegment[left](I,O){$d$}
				\tkzLabelSegment[above right](I,M){$R$}
				\tkzLabelSegment[below](M,O){$r$}
			\end{tikzpicture}
		}
	}
\end{ex}
\Closesolutionfile{ans}


\subsection{BÀI TẬP TRẮC NGHIỆM TỰ LUYỆN}
\subsection*{\indam{PHẦN I. Câu trắc nghiệm nhiều phương án lựa chọn. Thí sinh trả lời từ câu 1 đến câu 12. Mỗi câu hỏi thí sinh chỉ chọn một phương án.}} 
	\setcounter{ex}{0}
	\Opensolutionfile{ans}[ans/B4-De2-1]

\begin{ex}%[2H5H3-1]
	Trong không gian $Oxyz$, điểm nào sau đây nằm trong mặt cầu $(S)\colon \left(x-1\right)^2+\left(y-4\right)^2+\left(z-3\right)^2=16$?
	\choice
	{$M\left(0;7;-3\right)$}
	{$P\left(1;0;0\right)$}
	{\True $N\left(0;4;3\right)$}
	{$Q\left(1;0;3\right)$}
	\loigiai{
		Mặt cầu $(S)$ có tâm $I\left(1;4;3\right)$ và bán kính $R=4$.\\
		$IM=\sqrt{(-1)^2+3^2+(-6)^2}=\sqrt{46} > R\Rightarrow M$ nằm ngoài mặt cầu.\\
		$IN=\sqrt{(-1)^2+0^2+0^2}=1< R\Rightarrow N$ nằm trong mặt cầu.\\
		$IP=\sqrt{0^2+\left(-4\right)^2+\left(-3\right)^2}=5> R\Rightarrow P$ nằm ngoài mặt cầu.\\
		$IQ=\sqrt{0^2+\left(-4\right)^2+0^2}=4=R\Rightarrow Q$ thuộc mặt cầu.\\
	}
\end{ex}

%G:\My Drive\CODE12-2024\DE-ON-THEO BAI\2H5-TACH DE\Bai4-De2.tex
\begin{ex}%[2H5H3-3]
	Trong không gian $Oxyz$, cho mặt cầu $(S)$ có tâm $A\left(3;-1;1\right)$ và đi qua $M\left(2;-2;4\right)$. Phương trình mặt cầu $(S)$ là
	\choice
	{\True $\left(x-3\right)^2+\left(y+1\right)^2+\left(z-1\right)^2=11$}
	{$\left(x+3\right)^2+\left(y-1\right)^2+\left(z+1\right)^2=11$}
	{$\left(x+3\right)^2+\left(y-1\right)^2+\left(z+1\right)^2=\sqrt{11}$}
	{$\left(x-3\right)^2+\left(y+1\right)^2+\left(z-1\right)^2=\sqrt{11}$}
	\loigiai{
		Bán kính $R=AM=\sqrt{(-1)^2+(-1)^2+3^2}=\sqrt{11}$.\\
		Phương trình mặt cầu  là  $(S)\colon \left(x-3\right)^2+\left(y+1\right)^2+\left(z-1\right)^2=11$.
	}
\end{ex}

%G:\My Drive\CODE12-2024\DE-ON-THEO BAI\2H5-TACH DE\Bai4-De2.tex
\begin{ex}%[2H5N3-2]
	Trong không gian $Oxyz$, cho mặt cầu $(S):\left(x-2\right)^2+\left(y+2\right)^2+\left(z-1\right)^2=18$. Bán kính của $(S)$ bằng
	\choice
	{$9$}
	{$18$}
	{$6\sqrt{2}$}
	{\True $3\sqrt{2}$}
	\loigiai{
		Bán kính $R=\sqrt{18}=3\sqrt{2}$.
	}
\end{ex}

%G:\My Drive\CODE12-2024\DE-ON-THEO BAI\2H5-TACH DE\Bai4-De2.tex
\begin{ex}%[2H5H3-3]
	Trong không gian $Oxyz$, cho hai điểm $M\left(3;-2;5\right)$, $N\left(-1;6;-3\right)$. Mặt cầu đường kính $MN$ có phương trình là
	\choice
	{$\left(x+1\right)^2+\left(y+2\right)^2+\left(z+1\right)^2=6$}
	{$\left(x-1\right)^2+\left(y-2\right)^2+\left(z-1\right)^2=6$}
	{\True $\left(x-1\right)^2+\left(y-2\right)^2+\left(z-1\right)^2=36$}
	{$\left(x+1\right)^2+\left(y+2\right)^2+\left(z+1\right)^2=36$}
	\loigiai{
		Tâm $I$ của mặt cầu là trung điểm đoạn $MN$ $\Rightarrow$ $I\left(1;2;1\right)$.\\
		Bán kính mặt cầu $R=\dfrac{MN}{2}=\dfrac{\sqrt{\left(-1-3\right)^2+\left(6+2\right)^2+\left(-3-5\right)^2}}{2}=6$.\\
		Vậy phương trình mặt cầu là $\left(x-1\right)^2+\left(y-2\right)^2+\left(z-1\right)^2=36$.
	}
\end{ex}

%G:\My Drive\CODE12-2024\DE-ON-THEO BAI\2H5-TACH DE\Bai4-De2.tex
\begin{ex}%[2H5H3-2]
	Trong không gian $Oxyz$, cho mặt cầu $(S)\colon \left(x-5\right)^2+\left(y-1\right)^2+\left(z+2\right)^2=9$. Đường kính của mặt cầu $(S)$ là
	\choice
	{$9$}
	{$3$}
	{\True $6$}
	{$18$}
	\loigiai{
		Bán kính $R=\sqrt{9}=3\Rightarrow$ đường kính bằng $2R=2\cdot 3=6$.
	}
\end{ex}

%G:\My Drive\CODE12-2024\DE-ON-THEO BAI\2H5-TACH DE\Bai4-De2.tex
\begin{ex}%[2H5H3-2]
	Trong không gian $Oxyz$, phương trình nào sau đây là phương trình của mặt cầu?
	\choice
	{$x^2+z^2+3x-2y+4z-1=0$}
	{$x^2+y^2+z^2-2x+2y-4z+8=0$}
	{\True $x^2+y^2+z^2-2x+4z-1=0$}
	{$x^2+y^2+z^2+2xy-4y+4z-1=0$}
	\loigiai{
		Điều kiện để phương trình $x^2+y^2+z^2-2ax-2by-2cz+d=0$ là phương trình mặt cầu là
		\[
		a^2+b^2+c^2-d > 0.
		\]
		Xét phương trình  $x^2+y^2+z^2-2x+4z-1=0$ có  $1^2+0^2+(-2)^2-(-1)=6> 0$. \\
		Suy ra phương trình  $x^2+y^2+z^2-2x+4z-1=0$  là phương trình của một mặt cầu.
	}
\end{ex}

%G:\My Drive\CODE12-2024\DE-ON-THEO BAI\2H5-TACH DE\Bai4-De2.tex
\begin{ex}%[2H5H3-2]
	Trong không gian với hệ toạ độ $Oxyz$,  cho mặt cầu $(S):(x+2)^2+y^2+\left(z-3\right)^2=4$. Tâm của $(S)$ có toạ độ là
	\choice
	{$\left(-1;0;\dfrac{3}{2} \right)$}
	{$\left(2;0;-3\right)$}
	{$\left(1;0;\dfrac{3}{2} \right)$}
	{\True $\left(-2;0;3\right)$}
	\loigiai{
		Phương trình mặt cầu $\left(x-a\right)^2+\left(y-b\right)^2+\left(z-c\right)^2=R^2$ có tâm $I\ (a;b;c)$.\\
		Suy ra $(S)\colon (x+2)^2+y^2+\left(z-3\right)^2=4$ có tâm $\left(-2;0;3\right)$.
	}
\end{ex}

%G:\My Drive\CODE12-2024\DE-ON-THEO BAI\2H5-TACH DE\Bai4-De2.tex
\begin{ex}%[2H5H3-4]
	Trong không gian $Oxyz$, một thiết bị phát sóng đặt tại vị trí $A(2;0;0)$. Vùng phủ sóng của thiết bị có bán kính bằng $1$. Điểm nào sau đây thuộc vùng phủ sóng của thiết bị nói trên?
	\choice
	{\True $P\left(1;0;0\right)$}
	{$O\left(0;0;0\right)$}
	{$N\left(0;1;1\right)$}
	{$M\left(1;0;3\right)$}
	\loigiai{
		Ta có $AM=\sqrt{(-1)^2+0^2+3^2}=\sqrt{10} > R$ nên $M$ không thuộc vùng phủ sóng.\\
		$AN=\sqrt{0^2+1^2+1^2}=\sqrt{2} > R$ nên $N$ không thuộc vùng phủ sóng.\\
		$AP=\sqrt{(-1)^2+0^2+0^2}=1=R$ nên $P$ thuộc vùng phủ sóng.\\
		$AO=\sqrt{(-2)^2+0^2+0^2}=2> R$ nên $O$ không thuộc vùng phủ sóng.
	}
\end{ex}

%G:\My Drive\CODE12-2024\DE-ON-THEO BAI\2H5-TACH DE\Bai4-De2.tex
\begin{ex}%[2H5H3-2]
	Trong không gian $Oxyz$, cho mặt cầu $(S)\colon x^2+y^2+z^2-2x+4y+2z-3=0$. Bán kính $R$ của mặt cầu $(S)$ bằng
	\choice
	{$\sqrt{3}$}
	{$9$}
	{$3\sqrt{3}$}
	{\True $3$}
	\loigiai{
		Mặt cầu có dạng $x^2+y^2+z^2-2ax-2by-2cz+d=0$ có bán kính là $R=\sqrt{a^2+b^2+c^2-d}$.\\
		Suy ra $R=\sqrt{1^2+(-2)^2+1^2-(-3)}=3$.
	}
\end{ex}

%G:\My Drive\CODE12-2024\DE-ON-THEO BAI\2H5-TACH DE\Bai4-De2.tex
\begin{ex}%[2H5H3-3]
	Trong không gian $Oxyz$, phương trình mặt cầu tâm $I\left(1;2;3\right)$, bán kính $R=2$ có dạng
	\choice
	{\True $\left(x-1\right)^2+\left(y-2\right)^2+\left(z-3\right)^2=4$}
	{$\left(x-1\right)^2+\left(y-2\right)^2+\left(z-3\right)^2=2$}
	{$\left(x+1\right)^2+\left(y+2\right)^2+\left(z+3\right)^2=2$}
	{$\left(x+1\right)^2+\left(y+2\right)^2+\left(z+3\right)^2=4$}
	\loigiai{
		Phương trình mặt cầu tâm $I\left(1;2;3\right)$, bán kính $R=2$ có dạng
		\[
		\left(x-1\right)^2+\left(y-2\right)^2+\left(z-3\right)^2=4.
		\]
	}
\end{ex}

%G:\My Drive\CODE12-2024\DE-ON-THEO BAI\2H5-TACH DE\Bai4-De2.tex
\begin{ex}%[2H5H3-2]
	Trong không gian $Oxyz$, tìm $m$ để phương trình $x^2+y^2+z^2-2x-y+4z-m=0$ là phương trình của mặt cầu.
	\choice
	{$m\le \dfrac{21}{4}$}
	{\True $m > \dfrac{21}{4}$}
	{$m\ge \dfrac{21}{4}$}
	{$m < \dfrac{21}{4}$}
	\loigiai{
		Phương trình đã cho là phương trình mặt cầu khi và chỉ khi
		\[
		a^2+b^2+c^2-d > 0 \Leftrightarrow 1^2+\left(\dfrac{1}{2} \right)^2+(-2)^2+m > 0 \Leftrightarrow m > \dfrac{21}{4}.
		\]
	}
\end{ex}

%G:\My Drive\CODE12-2024\DE-ON-THEO BAI\2H5-TACH DE\Bai4-De2.tex
\begin{ex}%[2H5N3-1]
	Trong không gian với hệ toạ độ $Oxyz$, mặt cầu  $(S)\colon x^2+y^2+z^2-2ax-2by-2cz+d=0$ có bán kính $R$ bằng
	\choice
	{$a^2+b^2+c^2+d$}
	{$\sqrt{a^2+b^2+c^2+d}$}
	{\True $\sqrt{a^2+b^2+c^2-d}$}
	{$a^2+b^2+c^2-d$}
	\loigiai{
		Mặt cầu  $(S)\colon x^2+y^2+z^2-2ax-2by-2cz+d=0$ có bán kính $R = \sqrt{a^2+b^2+c^2-d}$.
	}
\end{ex}
	\Closesolutionfile{ans}

\subsection*{\indam{PHẦN II. Câu trắc nghiệm đúng sai. Thí sinh trả lời từ câu 1 đến câu 4. Trong mỗi ý a), b), c), d) ở mỗi câu, thí sinh chọn đúng hoặc sai.}}
	\setcounter{ex}{0}
	\Opensolutionfile{ans}[ans/B4-De2-2]
\begin{ex}%[2H5H3-3]
	Trong không gian $Oxyz$, cho mặt cầu $(S): \left(x-2\right)^2+y^2+\left(z+1\right)^2=1$ và mặt phẳng $(P)\colon  x+2y-z+1=0$. Các mệnh đề sau đúng hay sai?
	\choiceTF
	{Khoảng cách từ tâm $I$ đến mặt phẳng $(P)$ bằng $\dfrac{\sqrt{6}}{3}$}
	{\True Mặt cầu $(S)$ có tâm $I\left(2;0;-1\right)$ và bán kính $R=1$}
	{Mặt phẳng $(P)$ tiếp xúc mặt cầu $(S)$}
	{\True Phương trình mặt cầu tâm $I\left(2;0;-1\right)$ và tiếp xúc mặt phẳng $(P)$ là: $\left(S'\right): x^2+y^2+z^2-4x+2z+\dfrac{7}{3}=0$}
	\loigiai{
		\begin{itemchoice}
			\itemch \textbf{Sai}.\\
			$\mathrm{d}\left(I,(P)\right)=\dfrac{\left|2+1+1\right|}{\sqrt{1+4+1}}=\dfrac{2\sqrt{6}}{3}$.
			\itemch \textbf{Đúng}.\\
			Mặt cầu $(S)\colon \left(x-2\right)^2+y^2+\left(z+1\right)^2=1$ có tâm $I\left(2;0;-1\right)$ và bán kính $R=1$.
			\itemch \textbf{Sai}.\\
			Vì $\mathrm{d}\left(I,(P)\right)=\dfrac{2\sqrt{6}}{3} > R$ nên $(P)$ không cắt mặt cầu $(S)$.
			\itemch \textbf{Đúng}.\\
			Phương trình mặt cầu cần tìm có bán kính $R=\mathrm{d}\left(I,(P)\right)=\dfrac{2\sqrt{6}}{3}$, tâm $I\left(2;0;-1\right)$ nên có phương trình 
			\[
			\left(x-2\right)^2+y^2+\left(z+1\right)^2=\left(\dfrac{2\sqrt{6}}{3} \right)^2 \Leftrightarrow x^2+y^2+z^2-4x+2z+\dfrac{7}{3}=0.
			\]
		\end{itemchoice}
	}
\end{ex}

%G:\My Drive\CODE12-2024\DE-ON-THEO BAI\2H5-TACH DE\Bai4-De2.tex
\begin{ex}%[2H5H3-3]
	Trong không gian $Oxyz$, cho mặt cầu $(S)\colon \left(x-1\right)^2+(y+3)^2+\left(z-2\right)^2=49$. 
	\choiceTF
	{\True Mặt cầu $(S)$ có bán kính $R=7$}
	{\True Điểm $A\left(1;4;2\right)$ nằm trên mặt cầu $(S)$}
	{Mặt cầu $(S)$ có tâm $I\left(1;3;2\right)$}
	{Mặt cầu $(S)$ còn có phương trình: $x^2+y^2+z^2-2x+6y-4z-49=0$}
	\loigiai{
		\begin{itemchoice}
			\itemch \textbf{Đúng}.\\
			Mặt cầu $(S)\colon \left(x-1\right)^2+(y+3)^2+\left(z-2\right)^2=49=7^2$ nên mặt cầu $(S)$ có bán kính $R=7$.
			\itemch \textbf{Đúng}.\\
			Thay toạ độ điểm $A\left(1;4;2\right)$ vào vế trái của mặt cầu $(S)$ được: $VT=7^2=49$.
			\itemch \textbf{Sai}.\\
			Mặt cầu $(S)$ có tâm là $I\left(1;-3;2\right)$.
			\itemch \textbf{Sai}.\\
			Ta có
			\begin{eqnarray*}
				& & (S) \colon \left(x-1\right)^2+(y+3)^2+\left(z-2\right)^2=49\\
				& \Leftrightarrow &  x^2+y^2+z^2-2x+6y-4z+1+9+4=49\\
				& \Leftrightarrow & x^2+y^2+z^2-2x+6y-4z-35=0.
			\end{eqnarray*}
		\end{itemchoice}
	}
\end{ex}

%G:\My Drive\CODE12-2024\DE-ON-THEO BAI\2H5-TACH DE\Bai4-De2.tex
\begin{ex}%[2H5H3-3]
	Trong không gian $Oxyz$, cho mặt cầu $(S): x^2+y^2+z^2+2x+8y+1=0$. Các mệnh đề sau đúng hay sai?
	\choiceTF
	{Mặt cầu $(S)$ có tâm $I\left(1;4;0\right)$}
	{\True Mặt cầu $(S)$ còn có phương trình: $(S): \left(x+1\right)^2+\left(y+4\right)^2+z^2=16$}
	{\True Điểm $M\left(0;3;4\right)$ nằm bên ngoài mặt cầu $(S)$}
	{\True Mặt cầu $(S)$ có bán kính $R=4$}
	\loigiai{
		\begin{itemchoice}
			\itemch \textbf{Sai}.\\
			Mặt cầu $(S)$ có tâm là $I\left(-1;-4;0\right)$.
			\itemch \textbf{Đúng}.\\
			Vì mặt cầu $(S)$ có tâm $I\left(-1;-4;0\right)$, bán kính $R=4$ nên có phương trình là
			\[
			(S)\colon  \left(x+1\right)^2+\left(y+4\right)^2+z^2=16.
			\]
			\itemch \textbf{Đúng}.\\
			Ta có $IM=\sqrt{1^2+7^2+4^2} > R$ nên điểm $M\left(0;3;4\right)$ nằm bên ngoài mặt cầu $(S)$.
			\itemch \textbf{Đúng}.\\
			Mặt cầu $(S)$ có bán kính $R=\sqrt{a^2+b^2+c^2-d}=\sqrt{1+16-1}=4$.
		\end{itemchoice}
	}
\end{ex}

%G:\My Drive\CODE12-2024\DE-ON-THEO BAI\2H5-TACH DE\Bai4-De2.tex
\begin{ex}%[2H5H3-2]
	Trong không gian $Oxyz$, cho ba điểm $A(1; 2; -4)$, $B(1; -3; 1)$, $C(2; 2; 3)$.
	\choiceTF
	{\True Bán kính của mặt cầu $(S_4)$ đi qua ba điểm $A$, $B$, $C$  và có tâm nằm trên mặt phẳng $(Oxy)$ là $R = \sqrt{26}$}
	{Mặt cầu $\left(S_1 \right)$ tâm $A$, bán kính $R=1$ có phương trình là: $\left(x-1\right)^2+\left(y-2\right)^2+\left(z-4\right)^2=1$}
	{\True Bán kính của mặt cầu $\left(S_2 \right)$ có tâm là $A$ và đi qua điểm $C$ là $\sqrt{50}$}
	{\True Mặt cầu $\left(S_3 \right)$ nhận $AB$ làm đường kính có phương trình là: $\left(x-1\right)^2+\left(y+\dfrac{1}{2} \right)^2+\left(z+\dfrac{3}{2} \right)^2=\dfrac{25}{2}$}
	\loigiai{
		\begin{itemchoice}
			\itemch \textbf{Đúng}.\\
			Gọi phương trình mặt cầu $(S_4)$ có dạng $x^2+y^2+z^2-2ax-2by-2cz+d=0$, với tâm $I(a; b; c)$.\\
			Ta có $I(a;b;c)\in (Oxy) \Rightarrow c=0$. \\
			Vì $ \heva{&{A\in (S)} \\&{B\in (S)} \\&{C\in (S)}}
			\Rightarrow	 \heva{&{-2a-4b+d=-21} \\ &{-2a+6b+d=-11} \\&{-4a-4b+d=-17}}
			\Leftrightarrow \heva{&{a=-2} \\&{b=1} \\&{d=-21}}$\\
			Suy ra $R=\sqrt{a^2+b^2+c^2-d}=\sqrt{4+1+0+21}=\sqrt{26}$.
			\itemch \textbf{Sai}.\\
			Mặt cầu $\left(S_1 \right)$ tâm $A$, bán kính $R=1$ có phương trình là $\left(x-1\right)^2+\left(y-2\right)^2+\left(z+4\right)^2=1$.
			\itemch \textbf{Đúng}.\\
			Ta có $\overrightarrow{AC}=\left(1;0;7\right) \Rightarrow AC=\sqrt{50}$.\\
			Suy ra bán kính của mặt cầu $\left(S_2 \right)$ có tâm là $A$ và đi qua điểm $C$ là $AC=\sqrt{50}$.
			\itemch \textbf{Đúng}.\\
			Ta có $\overrightarrow{AB}=\left(0;-5;5\right) \Rightarrow AB=5\sqrt{2}$.\\
			Mặt cầu $\left(S_3 \right)$ nhận $AB$ làm đường kính nên có tâm $I\left(1;-\dfrac{1}{2};-\dfrac{3}{2} \right)$, $R=\dfrac{AB}{2}=\dfrac{5\sqrt{2}}{2}$, có phương trình là
			\[
			\left(x-1\right)^2+\left(y+\dfrac{1}{2} \right)^2+\left(z+\dfrac{3}{2} \right)^2=\dfrac{25}{2}.
			\]
		\end{itemchoice}
	}
\end{ex}
\Closesolutionfile{ans}

\subsection*{\indam{PHẦN III. Câu trắc nghiệm trả lời ngắn. Thí sinh trả lời từ câu 1 đến câu 6 vào ô kết quả.}}
	\setcounter{ex}{0}
	\Opensolutionfile{ans}[ans/B4-De2-3]

\begin{ex}%[2H5H3-4]
	Trong không gian $Oxyz$, cho phương trình $x^2+y^2+z^2-4x+2my+3m^2-2m=0$ với $m$ là tham số. Tính tổng tất cả các giá trị nguyên của $m$ để phương trình đã cho là phương trình mặt cầu.\\
	\shortans[oly]{$1$}
	\loigiai{
		Phương trình  $x^2+y^2+z^2-4x+2my+3m^2-2m=0$ là phương trình mặt cầu khi và chỉ khi 
		\[
		-2m^2+2m+4> 0\Leftrightarrow m\in \left(-1; 2\right).
		\]
		Do $m\in \mathbb{Z}\Rightarrow m\in \left\{0; 1\right\}$.
		Vậy tổng tất cả các giá trị nguyên của $m$ bằng $1$.
	}
\end{ex}

\begin{ex}%[2H5V3-4]
	Trong không gian $Oxyz$ (đơn vị của các trục tọa độ là ki--lô-mét), đài kiểm soát không lưu sân bay có tọa độ $\left(-64;128;64\right)$. Máy bay bay trong phạm vi cách đài kiểm soát $500$ km thì sẽ hiển thị trên màn hình ra đa. Một máy bay $N$ xuất hiện trên màn hình ra đa và một máy bay $M$ nằm trong mặt phẳng $(P)\colon x-2y+2z-1458=0$ sao cho hai máy bay $M$, $N$ thuộc đường thẳng có vectơ chỉ phương là $\overrightarrow{u}=\left(1;1;1\right)$. Khoảng cách nhỏ nhất giữa hai máy bay $M$, $N$ là bao nhiêu km? (kết quả làm tròn đến hàng đơn vị)\\
	\shortans[oly]{$260$}
	\loigiai{
		\immini{
			Máy bay $N$ xuất hiện trên màn hình ra đa nên $N$ thuộc mặt cầu $(S)$ có tâm $I\left(-64;128;64\right)$, bán kính $R=500$.\\
			Ta có 
			\[	
			\mathrm{d}\left(I,(P)\right)=\dfrac{\left|x_I-2y_I+2z_I-1458\right|}{\sqrt{1^2+2^2+2^2}}=550> R
			\] nên $(P)$ và $(S)$ không giao nhau.\\	
			Gọi $\alpha$ là góc tạo bởi $MN$ và mặt phẳng $(P)$.\\
			Gọi $H$ là hình chiếu của $N$ lên mặt phẳng $(P)$.
		}{\begin{tikzpicture}[scale=1, font=\footnotesize,line join=round, line cap=round, >=stealth, every node/.style={scale=0.8}]
				\coordinate (I) at (0,0);
				\coordinate (N) at (0,-1);
				\coordinate (H) at (0,-3);
				\coordinate (M) at (-1,-3);
				\path let \p1=(I),\p2=(N),\n1={veclen(\x2-\x1,\y2-\y1)} in \pgfextra{\xdef\Temp{\n1}};
				\draw (I) circle (\Temp);
				\coordinate (A) at (-2,-3);
				\coordinate (B) at (4,-3);
				\coordinate (C) at (1.5,-2);
				\coordinate (D) at ($(N)+(C)-(M)$);
				\draw[->] (C)--(D) node[midway,right]{$\vec{u}$};
				\draw(A)--(B) (I)--(H) (M)--(N);
				\foreach \i/\g in {M/-90,H/-90,N/-50,I/180} \fill (\i) circle (1.2pt) node[shift={(\g:3mm)},scale=0.8]{$\i$};
			\end{tikzpicture}
		}
		\noindent
		Mặt khác $\overrightarrow{MN}$ cùng phương với véc-tơ $\overrightarrow{u}=\left(1;1;1\right)$ và $\overrightarrow{n_P}=(1;-2;2)$ suy ra 
		\[
		\sin \alpha=\dfrac{\left|\overrightarrow{u}\cdot \overrightarrow{n_P}\right|}{\left|\overrightarrow{u}\right|\cdot \left|\overrightarrow{n_P}\right|}=\dfrac{1}{3\sqrt{3}}.
		\]
		Khi đó $MN=\dfrac{NH}{\sin \alpha} \ge \dfrac{NH_{\min}}{\sin \alpha}=\dfrac{\mathrm{d}\left(I,(P)\right)-R}{\sin \alpha}=\dfrac{550-500}{\dfrac{1}{3\sqrt{3}}}=150\sqrt{3} \approx 260$.
	}
\end{ex}

\begin{ex}%[2H5V3-2]
	Một vỏ kem ốc quế là một loại bánh khô, hình nón $(N)$ trong không gian $Oxyz$, thường được làm bằng một chiếc bánh xốp dùng để đặt kem vào và cầm ăn mà không cần bát hoặc muỗng. Người ta thả vào vỏ kem $(N)$ một viên kem vani hình cầu có đính hai viên socola nhỏ tại hai vị trí $A(2; 1; 3)$ và $B(6; 5; 5)$ sao cho đường kính $AB$ có $B$ là tâm đường tròn đáy khối nón. Khi thể tích của khối nón $(N)$ nhỏ nhất thì mặt phẳng qua đỉnh $S$ của khối nón $(N)$ và song song với mặt phẳng chứa đường tròn đáy của $(N)$ có phương trình $2x+by+cz+d=0$. Tính giá trị của biểu thức $T=b+c+d$.\\
	\shortans[oly]{$12$}
	\loigiai{
		\immini{
			Gọi chiều cao khối nón $SB=h \left(h > 0\right)$ và bán kính đường tròn đáy $BC=R$.\\
			Ta có $V=\dfrac{1}{3} \pi R^2 h $. \hfill (1)\\
			Ta có $	\overrightarrow{AB}=(4; 4; 2)\Rightarrow AB=6$.\\
			Xét mặt cầu có đường kính $AB$.\\
			Bán kính là $r=\dfrac{AB}{2}=3$ và tâm $I(4; 3; 4)$.\\
			Vì $\triangle SHI$ đồng dạng với $\triangle SBC$ nên
			\[ 
			\dfrac{SI}{SC}=\dfrac{IH}{BC} \Leftrightarrow \dfrac{h-3}{\sqrt{h^2+R^2}}=\dfrac{3}{R}.
			\]
			\[\Leftrightarrow \dfrac{\left(h-3\right)^2}{h^2+R^2}=\dfrac{9}{R^2} \Leftrightarrow R^2\left[\left(h-3\right)^2-9\right]=9h^2\Leftrightarrow R^2=\dfrac{9h^2}{h^2-6h} \tag{2}.
			\]
			Thay $(2)$ vào $(1)$ ta có\\
			$V=\dfrac{1}{3} \pi\cdot \dfrac{9h^2}{h^2-6h}\cdot h=3\pi\cdot \dfrac{h^2}{h-6}$ với $h > 6$.
		}{
			\begin{tikzpicture}[scale=0.8, font=\footnotesize,line join=round, line cap=round, >=stealth, every node/.style={scale=0.8}]
				\def\a{3}
				\def\b{1.0}
				\pgfmathsetmacro\h{\a*sqrt(3)}
				\pgfmathsetmacro\r{\h/3}
				\pgfmathsetmacro\g{asin(\b/\h)}
				\pgfmathsetmacro\xo{\a *cos(\g)}
				\pgfmathsetmacro\yo{\b *sin(\g)}
				\begin{scope}[rotate = 180]
					\path 
					(\xo,\yo) coordinate (Mr) (0,0) coordinate (B) (90:\h) coordinate (S) (0:\a) coordinate (C)  (-\xo,\yo) coordinate (Ml) ($(B)!1/3!(S)$) coordinate (I) ($(B)!2!(I)$) coordinate (A);
					\coordinate (H) at ($(S)!(I)!(C)$);
					\draw[dash pattern=on 2pt off 1.5pt] (S)--(B) (I)--(H)
					(I) circle (\r) (I) ellipse ({\r} and {\r/3});
					\draw (Mr) arc (\g:180-\g:{\a} and {\b}) (S)--(Ml) arc(180-\g:360+\g:{\a} and {\b})--cycle (B)--(C);
					\foreach \x /\gN in {B/-90,S/90,C/0,I/180,A/-40,H/60}
					\fill[black] (\x) circle (1.2pt)($(\x)+(\gN:4mm)$) node {$\x$};
				\end{scope}
			\end{tikzpicture}
		}
		Xét $V'=3\pi\cdot \dfrac{2h\left(h-6\right)-h^2}{\left(h-6\right)^2}=3\pi\cdot \dfrac{h^2-12h}{\left(h-6\right)^2}$.\\
		Ta được BBT như sau:
		\begin{center}
			\begin{tikzpicture}[scale=0.8, font=\footnotesize,line join=round, line cap=round, >=stealth, every node/.style={scale=0.8}]
				\tkzTabInit[nocadre=true,lgt=1.2,espcl=2.5,deltacl=0.6]
				{$h$ /0.6,$V'$ /0.6,$V$ /2.5}
				{$-\infty$,$0$,$6$,$12$,$+\infty$}
				\tkzTabLine{,+,$0$,-,d,-,$0$,+,}
				\tkzTabVar{-/$-\infty$,+/$V(0)$,-D+/$-\infty$/$+\infty$,-/$V(12)$,+/$+\infty$}
				
			\end{tikzpicture}
		\end{center}
		Do $h > 6$ nên $V_{\min}$ khi $SB=h=12$ $\Rightarrow A$ là trung điểm của $SB$ $\Rightarrow S(-2; -3; 1)$.\\
		Khi đó $(P)$ đi qua $S$, vuông góc với $AB$ nên có một VTPT $\overrightarrow{n}=\overrightarrow{AB}=(4; 4; 2)$ hay $\overrightarrow{n}=(2; 2;1)$.\\
		Suy ra $(P)\colon  2(x+2)+2(y+3)+z-1=0\Leftrightarrow (P)\colon 2x+2y+z+9=0$.\\
		Vậy $T=b+c+d=2+1+9=12$.
	}
\end{ex}

\begin{ex}%[2H5V3-2]
	Trong không gian $Oxyz$, cho mặt cầu $(S)\colon x^2+y^2+z^2-2x-4y+6z-13=0$ và đường thẳng $d\colon \heva{& x=-1+t \\ & y=-2+t \\ & z=1+t}$. Gọi $M(a; b; c)$ với $a < 0$ là điểm thuộc đường thẳng $d$ sao cho từ $M$ kẻ được ba tiếp tuyến $MA$, $MB$, $MC$ đến mặc cầu $(S)$ ($A,B,C$ là các tiếp điểm) thỏa mãn $\widehat{AMB}=60^{\circ}$; $\widehat{BMC}=90^{\circ}$; $\widehat{CMA}=120^{\circ}$. Tính giá trị của biểu thức $P=a+b+c$.\\
	\shortans[oly]{$-2$}
	\loigiai{
		\immini{
			Ta có mặt cầu $(S)$ có tâm $I\left(1;2;-3\right)$, bán kính $R=3\sqrt{3}$.\\
			Đặt $MA=MB=MC=a$.\\
			Tam giác $MAB$ đều nên $AB=a$.\\
			Tam giác $MBC$ vuông cân tại $M$ nên $BC=a\sqrt{2}$.\\
			Tam giác $MCA$ có $\widehat{CMA}=120^{\circ}$ nên $CA=a\sqrt{3}$.\\
			Xét tam giác $ABC$ có $AB^2+BC^2=AC^2$ nên tam giác $ABC$ vuông tại $B$ hay tam giác $ABC$ nội tiếp đường tròn đường kính $AC$ và $AH=\dfrac{1}{2} AC=\dfrac{a\sqrt{3}}{2}$.
		}{
			\begin{tikzpicture}[scale=0.7, line join=round, line cap=round,>=stealth]
				\coordinate (H) at (0,0);
				\coordinate (I) at (0,-3);
				\coordinate (M) at (0,5);
				%\draw (H) ellipse (4 cm and 1.5 cm);
				\coordinate (A) at ($(H) + (110:4 and 1.5)$);
				\coordinate (C) at ($(H) + (-30:4 and 1.5)$);
				\coordinate (B) at ($(H) + (-150:4 and 1.5)$);
				\coordinate (E) at ($(H) + (-90:4 and 1.5)$);
				\draw pic[draw,angle radius=0.3cm]{right angle=I--B--M};
				\draw pic[draw,angle radius=0.3cm]{right angle=I--A--M};
				\draw pic[draw,angle radius=0.3cm]{right angle=M--C--I};
				\draw  (B)--(M)--(C) (I)--(B) (I)--(C) (B)--(C);
				\draw[dashed] (A)--(B) (C)--(A) (H)--(A) (H)--(B) (H)--(C) (A)--(M)--(H) (I)--(A) (H)--(I);
				\foreach \i/\g in {A/130,B/-130,C/0,M/90,I/-90,H/45}
				\fill[black] (\i) circle(1pt)+(\g:3mm)node[scale=0.7]{$\i$};
			\end{tikzpicture}
		}	
		\noindent
		Xét tam giác vuông $IAM$ có
		\[
		\dfrac{1}{AH^2}=\dfrac{1}{AM^2}+\dfrac{1}{AI^2} \Rightarrow MA=a=3\Rightarrow IM^2=AM^2+AI^2=36.
		\]
		Mặt khác $M\in (d)\Rightarrow M\left(-1+t;-2+t;1+t\right)$.\\
		Hay $\left(t-2\right)^2+\left(t-4\right)^2+\left(t+4\right)^2=36$ $\Leftrightarrow 3t^2-4t=0\Leftrightarrow \hoac{ & t=0 \\ & t=\dfrac{4}{3}.}$\\
		Suy ra $M\left(-1;-2;1\right)$ hay $M\left(\dfrac{1}{3};-\dfrac{2}{3};\dfrac{7}{3} \right)$.\\
		Vì giả thiết cho $a < 0$ nên $a=-1;b=-2;c=1$. Khi đó $P=a+b+c=-2$.
	}
\end{ex}

\begin{ex}%[2H5V3-4]
	Trong không gian $Oxyz$ (đơn vị của các trục tọa độ là ki--lô-mét), một trạm thu phát sóng điện thoại di động có đầu thu đặt tại điểm $I\left(1;2;2\right)$ biết rằng bán kính phủ sóng của trạm là $3$ km. Hai người sử dụng điện thoại lần lượt tại $M\left(4;-4;2\right)$ và $N\left(6;0;6\right)$. Gọi $E(a; b; c)$ với $a < 0$ là một điểm thuộc ranh giới vùng phủ sóng của trạm sao cho tổng khoảng cách từ $E$ đến vị trí $M$ và $N$ lớn nhất. Tính $T=a+b+c$.\\
	\shortans[oly]{$4$}
	\loigiai{
		\immini{
			Xét mặt cầu $(S)$ có tâm $I\left(1;2;2\right)$, bán kính $R=3$.\\
			Ta có $IM=IN=3\sqrt{3} > R$ nên cả hai điểm $M$, $N$ đều nằm ngoài mặt cầu $(S)$.\\
			Gọi $H$ là trung điểm của $MN$ suy ra $H\left(5;-2;4\right)$ và 
			\[
			EH^2=\dfrac{EM^2+EN^2}{2}-\dfrac{MN^2}{4}.
			\]
			Ta có $\left(EM+EN\right)^2\le 2\left(EM^2+EN^2\right)=2\left(EH^2+\dfrac{MN^2}{4} \right)$.
		}{
			\begin{tikzpicture}[scale=0.6, font=\footnotesize,line join=round, line cap=round, >=stealth, every node/.style={scale=0.8}]
				\def\r{2.0}
				\def\x{2.0}
				\def\y{0.8}
				\path
				(0,0) coordinate (I)
				(-4,0) coordinate (M)
				(-\r,0) coordinate (A)
				(\r,0) coordinate (B)
				($(I)+({\x*cos(-120)},{\y*sin(-120)})$) coordinate (E)
				($(I)+({2.5*\x*cos(-120)},{2.5*\y*sin(-120)})$) coordinate (H)
				($(E)!2!(I)$) coordinate (E')
				($(M)!2!(H)$) coordinate (N)
				;
				\draw (I) circle(\r)
				(B) arc (0:-180:\x cm and \y cm)
				(E)--(H) (M)--(N)
				;
				\draw[dashed] (B) arc (0:180:\x cm and \y cm)
				(I)--(B)node[below,midway]{$R$} (E')--(I)--(E);
				\foreach \i/\g in {I/90,E/-90,E'/90,H/-120,M/90,N/-90}
				\fill[black] (\i) circle(1pt)+(\g:3mm)node[scale=1]{$\i$};
			\end{tikzpicture}
		}
		\noindent
		Khi đó $\left(EM+EN\right)$ lớn nhất khi $EH$ lớn nhất khi và chỉ khi $E$ là giao điểm của $IH$ và mặt cầu $(S)$.\\
		$IH$ có phương trình là $\heva{& x=1+2t \\ & y=2-2t \\ & z=2+t.}$ \hfill (1)\\
		Mặt cầu $(S)$ có phương trình $\left(x-1\right)^2+\left(y-2\right)^2+\left(z-2\right)^2=9.$ \hfill (2)\\
		Thay (1) vào (2), ta được
		\[(2t)^2+\left(-2t\right)^2+t^2=9\Leftrightarrow t=\pm 1.
		\]
		Suy ra có hai điểm $E'\left(3;0;3\right)$ hoặc $E\left(-1;4;1\right)$. \\
		Vì $a < 0$ nên $E\left(-1;4;1\right)$ hay $T=a+b+c=4$.
	}
\end{ex}

\begin{ex}%[2H5H3-4]
	\immini[thm]{Người ta muốn thiết kế một bồn chứa khí hoá lỏng hình cầu bằng phần mềm 3D (tham khảo hình vẽ). Cho biết phương trình bề mặt của bồn chứa là $(S)\colon  \left(x-2\right)^2+y^2+\left(z+1\right)^2=1$. Phương trình mặt phẳng chứa nắp là $(P)\colon  z-6=0$. Tính khoảng cách từ tâm bồn chứa đến mặt phẳng chứa nắp.\\
	\shortans[oly]{$7$}
}{
\includegraphics[scale=0.7]{image/bonkhi}}
	\loigiai{
		Tâm của bồn chứa $I\left(2;0;-1\right)$.\\
		Khoảng cách từ tâm bồn chứa đến mặt phẳng chứa nắp là
		\[
		\mathrm{d}\left(I,(P)\right)=\dfrac{\left|-1-6\right|}{\sqrt{1^2}}=7.
		\]
	}
\end{ex}

\centerline{---HẾT---}
\Closesolutionfile{ans}
%\newpage
%%=====================
%\begin{center}
%\textbf{\large BẢNG ĐÁP ÁN}
%\end{center}
%\noindent\textbf{ĐÁP ÁN PHẦN I}
%\inputansbox{10}{ans/B4-De2-1}
	
%\noindent\textbf{ĐÁP ÁN PHẦN II}
%\inputansbox[2]{2}{ans/B4-De2-2}
	
%\noindent\textbf{ĐÁP ÁN PHẦN III}
%\inputansbox[3]{6}{ans/B4-De2-3}




%Chuong VI. Xs
% \setcounter{section}{0}
\section{XÁC SUẤT CÓ ĐIỀU KIỆN}
%%%%%%%%%%%%%%%%
\subsection{Trọng tâm kiến thức}
\begin{tomtat}
	\subsubsection{Xác suất có điều kiện}
	\begin{boxdn}
	\begin{itemize}
	\item
	Cho hai biến cố $A$ và $B$. Xác suất của biến cố $A$, tính trong điều kiện biết rằng biến cố $B$ đã xảy ra, được gọi là xác suất của $A$ với điều kiện $B$ và kí hiệu là $\mathrm{P}(A\mid B)$.
	\item
	Cho hai biến cố $A$ và $B$ bất kì, với $\mathrm{P}(B)>0$. Khi đó
	$$\mathrm{P}(A \mid B)=\dfrac{\mathrm{P}(A B)}{\mathrm{P}(B)}.$$
	\end{itemize}
	\end{boxdn}
	\subsubsection{Công thức nhân xác suất}
	\begin{boxdn}
	Vậy với hai biến cố $A$ và $B$ bất kì, ta có
	$$\mathrm{P}(A B)=\mathrm{P}(B) \cdot \mathrm{P}(A \mid B).$$
	Công thức trên được gọi là \textbf{\textit{công thức nhân xác suất}}.
	\end{boxdn}
	\begin{note}
	\begin{itemize}
	\item Vì $AB=BA$ nên với hai biến cố $A$ và $B$ bất kì, ta cũng có
	$$\mathrm{P}(A B)=\mathrm{P}(A) \cdot \mathrm{P}(B \mid A) \text{. }$$
	\item Nếu $A$ và $B$ là hai biến cố độc lập thì
	$$\mathrm{P}(A B)=\mathrm{P}(A) \cdot \mathrm{P}(B).$$
	\end{itemize}
	\end{note}
\end{tomtat}
%%%%%%%%%%%%%%
\subsection{Các dạng bài tập}
%============================
\begin{dang}{Tính xác suất có điều kiện theo định nghĩa}
%	\begin{itemize}
%	\item Cho hai biến cố $A$ và $B$. Xác suất của biến cố $B$ khi biến cố $A$ đã xảy ra được gọi là \textbf{xác suất của $B$ với điều kiện $A$}, kí hiệu là $\mathrm{P}(B \mid A)$.
%	\item Sử dụng định nghĩa để tính xác suất có điều kiện (áp dụng với các bài có thể tìm được số phần tử của các biến cố).
%	\end{itemize}
\end{dang}
%----------------------------
\subsubsection{Ví dụ minh hoạ}
\begin{vd}%[2D5H1-2]
	Có hai hộp chứa các thẻ được đánh số. Hộp thứ nhất có các thẻ được đánh số từ $1$ đến $4$, hộp thứ hai có các thẻ được đánh số từ $5$ đến $6$. Các thẻ có cùng kích thước và khối lượng. Bạn Phương lấy ngẫu nhiên một thẻ từ hộp thứ nhất bỏ vào hộp thứ hai. Sau đó bạn lại lấy ngẫu nhiên một thẻ từ hộp thứ hai. Liệt kê các kết quả của phép thử biết lần thứ nhất bạn Phương lấy được một thẻ đánh số chẵn.
	\loigiai{
	Vì đã biết lần thứ nhất bạn Phương lấy được một thẻ đánh số chẵn. Nghĩa là lúc đó bạn Phương có thể lấy được thẻ đánh số $2$ hoặc $4$.\\
	Nếu bạn Phương lấy được thẻ đánh số $2$ và bỏ vào hộp thứ hai, thì lúc này trong hộp thứ hai có các thẻ đánh số từ $5$ đến $6$ và $2$. Do đó ta có các khả năng $(2;5),(2;6),(2;7),(2;2)$.\\
	Tương tự như vậy nếu bạn Phương lấy được thẻ đánh số $4$, ta có các khả năng $(4;5),(4;6),(4,7),(4,4)$.\\
	Vậy các kết quả của phép thử biết lần thứ nhất bạn Phương lấy được một thẻ đánh số chẵn là 
	$$(2;5),(2;6),(2;7),(2;2),(4;5),(4;6),(4,7),(4,4).$$
	}
\end{vd}
\begin{vd}%[2D5H1-2]
	Một hộp có $5$ viên bi cùng kích thước và khối lượng, trong đó có $3$ viên bi màu đỏ và $2$ viên bi màu xanh. Lấy ngẫu nhiên lần lượt $2$ viên bi và không hoàn lại. Tính xác suất để lấy được viên bi thứ hai có màu xanh, biết rằng viên bi thứ nhất có màu đỏ.
	\loigiai{
	Gọi
	\begin{itemize}
	\item $A$ là biến cố \lq\lq  Lấy được viên bi thứ hai có màu xanh\rq\rq;
	\item $B$ là biến cố \lq\lq  Lấy được viên bi thứ nhất có màu đỏ\rq\rq.
	\end{itemize}
	Khi đó xác suất để lấy được viên bi thứ hai có màu xanh, biết rằng viên bi thứ nhất có màu đỏ chính là xác suất của $A$ với điều kiện $B$.\\
	Vì một viên bi đỏ đã được lấy ra ở lần thứ nhất nên trong hộp còn lại $4$ viên bi, trong đó có $2$ viên bi xanh.\\
	Từ đó ta có $\mathrm{P}(A \mid B)=\dfrac{2}{4}=0{,}5$.\\
	Vậy xác suất để lấy được viên bi thứ hai có màu xanh, biết rằng viên bi thứ nhất có màu đỏ là $0{,}5$.
	}	
\end{vd}
\begin{vd}%[2D5H2-2]
	Một hộp có $20$ viên bi trắng và $10$ viên bi đen, các viên bi có cùng kích thước và khối lượng. Bạn Bình lấy ngẫu nhiên một viên bi trong hộp, không trả lại. Sau đó bạn An lấy ngẫu nhiên một viên bi trong hộp đó.\\
	Gọi $A$ là biến cố: ``An lấy được viên bi trắng''; $B$ là biến cố: ``Bình lấy được viên bi trắng''. Tính $\mathrm{P}(A\mid B)$, $P\left(A\mid\overline{B}\right)$.
	\loigiai{
		Nếu $B$ xảy ra tức là Bình lấy được viên bi trắng. \\
		Khi đó, trong hộp còn lại $29$ viên bi với $19$ viên bi trắng và $10$ viên bi đen.\\ 
		Vậy $\mathrm{P}(A\mid B)=\dfrac{19}{29}\approx 0{,}67$.\\
		Nếu $\overline{B}$ xảy ra tức là Bình lấy được viên bi đen.\\
	Khi đó trong hộp còn lại $29$ viên bi với $20$ viên bi trắng và $9$ viên bi đen.\\
	Vậy $\mathrm{P}(A\mid\overline{B})=\dfrac{20}{29}$.
	}
\end{vd}
\begin{vd}%[2D5H2-2]
	Câu lạc bộ cờ của nhà trường gồm $35$ thành viên, mỗi thành viên biết chơi ít nhất một trong hai môn cờ vua hoặc cờ tướng. Biết rằng có $25$ thành viên biết chơi cờ vua và $20$ thành viên biết chơi cờ tướng. Chọn ngẫu nhiên $1$ thành viên của câu lạc bộ. Tính xác suất thành viên được chọn biết chơi cờ vua, biết rằng thành viên đó biết chơi cờ tướng.
	\loigiai{
	Gọi $A$ là biến cố \lq\lq  Thành viên được chọn biết chơi cờ tướng\rq\rq \,và $B$ là biến cố \lq\lq  Thành viên được chọn biết chơi cờ vua\rq\rq.\\
	Số thành viên của câu lạc bộ biết chơi cả hai môn cờ là $20+25-35=10$.\\
	Do đó, trong số $20$ thành viên biết chơi cờ tướng, có đúng $10$ thành viên biết chơi cờ vua.\\ 
	Vậy nên xác suất thành viên được chọn biết chơi cờ vua, biết rằng thành viên đó biết chơi cờ tướng là 
	$$\mathrm{P}(B \mid A)=\dfrac{10}{20}=0{.}5.$$
	}
\end{vd}
\begin{vd}%[2D5H1-2]
	Hộp thứ nhất chứa $4$ viên bi xanh và $3$ viên bi đỏ. Hộp thứ hai chứa $3$ viên bi xanh và $5$ viên bi đỏ. Các viên bi có cùng kích thước và khối lượng. Bạn Thanh lấy ra ngẫu nhiên $1$ viên bi từ hộp thứ nhất bỏ vào hộp thứ hai, sau đó lại lấy ra ngẫu nhiên $1$ viên bi từ hộp thứ hai. Tính xác suất để viên bi lấy ra ở lần thứ hai là viên bi đỏ, biết viên bi lấy ra ở lần thứ nhất là viên bi đỏ.
	\loigiai{
	Gọi $A$ là biến cố \lq\lq  viên bi lấy ra lần thứ hai là viên bi đỏ\rq\rq; $B$ là biến cố \lq\lq  viên bi lấy ra lần thứ hai là viên bi đỏ\rq\rq.\\
	Biến cố $B$ xảy ra, nghĩa là lần thứ nhất lấy được viên bi đỏ và bỏ vào hộp thứ hai. Khi đó trong hộp thứ hai sẽ có $3$ viên bi xanh và $6$ viên bi đỏ.\\
	Vậy $\mathrm{P}(A \mid B)= \dfrac{6}{9}=\dfrac{2}{3}$.
	}
\end{vd}
%=========================
% \setcounter{subsubsection}{1}
\begin{dang}{Tính xác suất có điều kiện theo công thức}
	Nếu $\mathrm{P}(B)>0$ thì xác suất của biến cố $A$ với điều kiện $B$ được xác định bởi công thức
	$$\mathrm{P}(A \mid B)=\dfrac{\mathrm{P}(A B)}{\mathrm{P}(B)}.$$
\end{dang}
%----------------------------
\subsubsection{Ví dụ minh hoạ}
\begin{vd}%[2D5H1-2]
	Cho hai biến cố $A$, $B$ có $\mathrm{P}(A)=0,4$; $\mathrm{P}(B)=0,6$; $\mathrm{P}(AB)=0,2$. Tính các xác suất $\mathrm{P}(A\mid B)$ và $\mathrm{P}(B\mid A)$.
	\loigiai{
	Ta có
	\begin{itemize}
	\item $\mathrm{P}(A\mid B)=\dfrac{\mathrm{P}(AB)}{\mathrm{P}(B)}=\dfrac{0{,}2}{0{,}6}=\dfrac{1}{3}$.
	\item $\mathrm{P}(B|A)=\dfrac{\mathrm{P}(AB)}{\mathrm{P}(A)}=\dfrac{0{,}2}{0{,}4}=0{,}5$.
	\end{itemize}
	}
\end{vd}
\begin{vd}%[2D5H1-2]
	Cho hai biến độc lập $A,B$ với $\mathrm{P}(A)=0{,}8$. Tính $\mathrm{P}(A\mid B)$.
	\loigiai{
	Vì $A$ và $B$ là hai biến cố độc lập, do đó
	\[\mathrm{P}(A\mid B)=\dfrac{\mathrm{P}(AB)}{\mathrm{P}(B)}=\dfrac{\mathrm{P}(A)\cdot \mathrm{P}(B)}{\mathrm{P}(B)}=\mathrm{P}(A)=0{,}8.\]
	\begin{note} Ta có thể dùng công thức $\mathrm{P}\left(A\mid B\right)=\mathrm{P}(A)$ với $A$ và $B$ là hai biến cố độc lập.
	\end{note}
	}
\end{vd}
\begin{vd}%[2D5H2-2]
	Một lô sản phẩm có $20$ sản phẩm, trong đó có $5$ sản phẩm chất lượng thấp. Lấy liên tiếp $2$ sản phẩm trong lô sản phẩm trên, trong đó sản phẩm lấy ra ở lần thứ nhất không được bỏ lại vào lô sản phẩm. Tính xác suất để cả hai sản phẩm được lấy ra đều có chất lượng thấp.
	\loigiai{
	Gọi $A$ là biến cố \lq\lq  sản phẩm thứ nhất có chất lượng thấp\rq\rq, và $B$ là biến cố \lq\lq  sản phẩm thứ hai có chất lượng thấp\rq\rq.\\
	Xác suất của $A$ là xác suất để lấy ra một sản phẩm chất lượng thấp trong lần đầu tiên:
	$$\mathrm{P}(A)=\dfrac{n(A)}{n(\Omega)}=\dfrac{5}{20}=\dfrac{1}{4}.$$
	Sau khi lấy một sản phẩm chất lượng thấp, số sản phẩm chất lượng thấp giảm còn $4$ trong tổng số $19$ sản phẩm.\\
	Xác suất của $B$ khi đã xảy ra $A$ là xác suất để lấy ra một sản phẩm chất lượng thấp trong lần thứ hai:
	$$\mathrm{P}(B \mid A)=\dfrac{4}{19}.$$
	Áp dụng quy tắc nhân xác suất:
	$$\mathrm{P}(AB)=\mathrm{P}(A)\cdot \mathrm{P}(B \mid A)=\dfrac{1}{4}\cdot\dfrac{4}{19}=\dfrac{1}{19}.$$
	}
\end{vd}
\begin{vd}%[2D5H2-2]
	Trong cuộc khảo sát $300$ gia đình ở một khu vực, người ta nhận thấy rằng có $90\%$ gia đình có tivi và $60\%$ gia đình có máy tính bàn. Mỗi gia đình đều có ít nhất một trong hai thiết bị này. Chọn ngẫu nhiên một gia đình. Tính xác suất gia đình có máy tính bàn trong nhóm các gia đình có tivi.
	\loigiai{
	Gọi $A$ là biến cố \lq\lq  Gia đình được chọn có máy tính bàn\rq\rq; $B$ là biến cố \lq\lq  Gia đình được chọn có tivi\rq\rq. Khi đó $AB$ là biến cố \lq\lq  Gia đình được chọn có cả máy tính bàn và tivi\rq\rq. \\
	Ta có $n(B)=0{,}9\cdot300=270$ và $n(AB)=0{,}9\cdot300+0{,}6\cdot300-300=150$. \\
	Do đó $\mathrm{P}(A\mid B)=\dfrac{n\left(A\cap B\right)}{n(B)}=\dfrac{150}{270}=\dfrac{5}{9}$.
	}
\end{vd}
%----------------------------
\subsubsection{Bài tập áp dụng}

\begin{bt}%[2D5H2-2]
	Một phòng nghiên cứu dược học cho $500$ người bị bệnh $H$ dùng hai loại thuốc $X, Y$ để điều trị. Một số người được điều trị bằng thuốc $X$ và số người còn lại được điều trị bằng thuốc $Y$. Kết quả nghiên cứu được trình bày ở bảng $6.2$.
	\begin{center}
	\begin{tikzpicture}
	\begin{scope}[xscale=4.4,yscale=0.85]
	\path
	(0,0) foreach \i[count=\k] in {$X$,$Y$} {++(1,0)node(1\k){\i}}
	(0,-1) node {Khỏi bệnh} foreach \i[count=\k] in {$180$,$190$} {++(1,0)node(2\k){\i}}
	(0,-2) node{Không khỏi bệnh} foreach \i[count=\k] in {$60$,$70$} {++(1,0)node(3\k){\i}}
	%(0,-3) node{tiện} foreach \i[count=\k] in {,$O_1$,,$O_4$,} {++(1,0)node(4\k){\i}}
	%(0,-4) node{điện} foreach \i[count=\k] in {,$O_1$,,$O_4$,} {++(1,0)node(5\k){\i}}
	%(0,-5) node{nước} foreach \i[count=\k] in {,$O_1$,,$O_4$,} {++(1,0)node(6\k){\i}}
	;
	%\path
	%(23.south east) node{$\times$}
	%;
	\draw[shift={(-0.5,.5)}] (0,0) grid (3.,-3)
	(0,0)--(1.,-1)
	(0,-1) node[above right]{Tình trạng}
	(1,0) node[below left]{Loại thuốc}
	;
	\end{scope}
	\end{tikzpicture}
	\end{center}
	Chọn ngẫu nhiên một người trong số này. Gọi $A$ là biến cố \lq\lq  Người được chọn khỏi bệnh\rq\rq, $B$ là biến cố \lq\lq  Người được chọn điều trị bằng thuốc $X$\rq\rq, $C$ là biến cố \lq\lq  Người được chọn điều trị bằng thuốc $Y$\rq\rq.
	\begin{listEX}
	\item Tính và giải thích ý nghĩa của $\mathrm{P}(A \mid B)$ và $\mathrm{P}(A \mid C)$.
	\item Có thể nói loại thuốc nào có hiệu quả hơn trong việc điều trị bệnh $H$?
	\end{listEX}
	\loigiai{
	\begin{listEX}
	\item $\mathrm{P}(A \mid B)=\dfrac{n\left(AB\right)}{n(B)}=\dfrac{180}{240}=\dfrac{3}{4}$ và $\mathrm{P}(A \mid C)=\dfrac{n\left(A\cap C\right)}{n(C)}=\dfrac{190}{260}=\dfrac{19}{26}$. \\
	Theo các kết quả trên, xác suất để một người khỏi bệnh khi được chọn điều trị bằng thuốc $X$ là $\dfrac{3}{4}$ và xác suất để một người khỏi bệnh khi được chọn điều trị bằng thuốc $Y$ là $\dfrac{19}{26}$.
	\item Do $\dfrac{3}{4}>\dfrac{19}{26}$ nên loại thuốc $X$ có hiệu quả hơn loại thuốc $Y$ trong việc điều trị bệnh $H$.
	\end{listEX}
	}
\end{bt}
\begin{bt}%[2D5H2-2]
	Một xí nghiệp dệt may có những dải của một loại vải đang sản xuất theo một quy trình đặc biệt. Những dải này có thể bị lỗi theo hai hướng: lỗi chiều dài và lỗi kết cấu. Thông qua đợt kiểm tra quy trình sản xuất, người ta thấy rằng có $10\%$ dải không đạt yêu cầu về chiều dài, $5\%$ dải không đạt yêu cầu và kết cấu và chỉ có $0{,}8\%$ dải không đạt yêu cầu về cả chiều dài và kết cấu.
	\begin{listEX}
	\item Nếu chọn ngẫu nhiên một dải từ quy trình này thì xác suất không đạt yêu cầu về kết cấu là bao nhiêu?
	\item Nếu một dải được chọn ngẫu nhiên từ quy trình này và phép đo nhanh xác định dải đó không đạt yêu cầu về chiều dài, tính xác suất để dải đó không đạt yêu cầu về kết cấu.
	\end{listEX}
	\loigiai{
	\begin{listEX}
	\item Nếu chọn ngẫu nhiên một dải từ quy trình này thì xác suất không đạt yêu cầu về kết cấu là $\dfrac{5}{100}+\dfrac{0{,}8}{100}=\dfrac{29}{500}$.
	\item Gọi $A$ là biến cố \lq\lq  Dải được chọn từ quy trình không đạt yêu cầu về kết cấu\rq\rq;\\
	$B$ là biến cố \lq\lq  Dải được chọn từ quy trình không đạt yêu cầu về chiều dài\rq\rq.\\
	Khi đó $AB$ là biến cố \lq\lq  Một dải từ quy trình không đạt yêu cầu về cả kết cấu và chiều dài\rq\rq. Ta có $\mathrm{P}(AB)=0{,}8\%=0{,}008$ và $\mathrm{P}(B)=0{,}1$. \\
	Do đó $\mathrm{P}(A\mid B)=\dfrac{\mathrm{P}\left(AB\right)}{\mathrm{P}(B)}=\dfrac{0{,}008}{0{,}1}=0{,}08$.
	\end{listEX}
	}
\end{bt}
\begin{bt}%[2D5H2-2]
	Trong một lọ có chứa bi đen và bi trắng cùng kích thước và khối lượng, lấy ngẫu nhiên lần lượt hai viên bi ra ngoài và không bỏ vào lại. Biết rằng xác suất để lần đầu lấy được bi đen là $0{,}47$; xác suất để lần đầu lấy được bi đen và lần thứ hai lấy được bi trắng là $0{,34}$. Tính xác suất để lấy được bi trắng ở lần thứ hai với điều kiện lần đầu lấy được bi đen.
	\loigiai{
	Gọi $A$ là biến cố \lq\lq  Lấy được bi trắng ở lần thứ hai\rq\rq; $B$ là biến cố \lq\lq  Lấy được bi đen ở lần đầu\rq\rq. \\
	Do đó $\mathrm{P}(A \mid B)=\dfrac{\mathrm{P}(AB)}{\mathrm{P}(B)}=\dfrac{0{,34}}{0{,47}}=\dfrac{34}{47}$.
	}
\end{bt}
%=======================
\begin{dang}{Tính xác suất có điều kiện nhờ sơ đồ hình cây}
	Trên sơ đồ hình cây:
	\begin{itemize}
	\item Xác suất của các nhánh trong sơ đồ hình cây từ đỉnh thứ hai là xác suất có điều kiện.
	\item Xác suất xảy ra của mỗi kết quả bằng tích các xác suất trên các nhánh của cây đi đến kết quả đó.
	\end{itemize}
\end{dang}
%----------------------------
\subsubsection{Ví dụ minh hoạ}
\begin{vd}%[2D5H2-3]
	Một hộp có $8$ bi màu đỏ và $5$ viên bi màu vàng; các viên bi có kích thước và khối lượng như nhau. Có $5$ viên bi trong hộp được đánh số, trong đó có $3$ viên bi màu đỏ và $2$ viên bi màu vàng. Lấy ngẫu nhiên một viên bi trong hộp. Dùng sơ đồ hình cây, tính xác suất để viên bi được lấy ra, có màu đỏ, biết rằng viên bi đó được đánh số.
	\loigiai{
	Xét các biến cố sau:
	\begin{itemize}
	\item $A \colon$ \lq\lq  Viên bi được lấy ra có đánh số\rq\rq.
	\item $B \colon$ \lq\lq  Viên bi được lấy ra có màu đỏ \rq\rq.
	\end{itemize}
	Khi đó, xác suất để viên bi được lấy ra có màu đỏ, biết ràng viên bi đó được đánh số, chính là xác suất có điều kiện $\mathrm{P}(B\mid A)$.\\
	Sơ đồ hình cây biểu thị cách tính xác suất có điều kiện $\mathrm{P}(B\mid A)$, được vẽ như sau:
	\begin{center}
	\begin{tikzpicture}[scale=0.8]
	\def\gocm{20}
	\def\gocn{10}
	\def\r{4}
	\tikzset{s/.style={outer sep=0.5 mm,draw=magenta,rectangle,minimum width=2.75cm,rounded corners=1mm}}
	\path(0,0)node(O){}++(\gocm:\r)node[s](A1){A}++(\gocn:\r)node[s](A2){$B$}++(0:\r)node[s](A3){$AB$};
	\path(A1)++({-\gocn}:\r)node[s](a2){$\overline{B}$}++(0:\r)node[s](a3){$A\overline{B}$};
	\path(O)++(-\gocm:\r)node[s](B1){$\overline{A}$}++(\gocn:\r)node[s](B2){$B$}++(0:\r)node[s](B3){$\overline{A}B$};
	\path(B1)++({-\gocn}:\r)node[s](b2){$\overline{B}$}++(0:\r)node[s](b3){$\overline{A}\overline{B}$};
	\foreach \x/\y in {
	O/A1,A1/A2,
	O/B1,B1/B2,
	A1/a2,
	B1/b2}
	\draw[-stealth](\x.east)--(\y.west);
	\path(O)--(A1.west)node[pos=0.5,above,sloped]{$\frac{5}{13}$}(O)--(B1.west)node[pos=0.5,below,sloped]{$\frac{8}{13}$}(B1.east)--(B2.west)node[pos=0.5,above,sloped]{$\frac{5}{8}$}(A1.east)--(A2.west)node[pos=0.5,above,sloped]{$\frac{3}{5}$}
	(A1.east)--(a2.west)node[pos=0.5,below,sloped]{$\frac{2}{5}$}
	(B1.east)--(b2.west)node[pos=0.5,below,sloped]{$\frac{3}{8}$};
	\end{tikzpicture}
	\end{center}
	Vậy xác suất để viên bi được lấy ra có màu đỏ, biết rằng viên bi đó có đánh số, là $ 0{,}6$.
	}
\end{vd}
\begin{vd}%[2D5H2-3]
	Ở một sân bay, người ta sử dụng một loại máy soi tự động phát hiện hàng cấm trong hành lí kí gửi. Máy phát chuông cảnh báo với $95 \%$ các kiện hành lí có chứa hàng cấm và $2 \%$ các kiện hành lí không chứa hàng cấm. Tỉ lệ các kiện hành lí có chứa hàng cấm là $1 \%$.\\
	Chọn ngẫu nhiên một kiện hành lí để soi bằng máy trên. Sử dụng sơ đồ hình cây, tính xác suất của các biến cố:\\
	M: \lq\lq  Kiện hành lí có chứa hàng cấm và máy phát chuông cảnh báo \rq\rq;
	N : \lq\lq  Kiện hành lí không chứa hàng cấm và máy phát chuông cảnh báo\rq\rq.
	\loigiai
	{
	Gọi $A$ là biến cố \lq\lq  Kiện hành lí có chứa hàng cấm\rq\rq và $B$ là biến cố \lq\lq  Máy phát chuông cành báo\rq\rq. Ta có
	$$
	\mathrm{P}(B \mid A)=0{,}95 ; \mathrm{P}(B \mid \overline{A})=0{,}02 ; \mathrm{P}(A)=0{,}01.
	$$
	Do đó $\mathrm{P}(\overline{A})=1-\mathrm{P}(A)=0{,}99 ; \mathrm{P}(\overline{B} \mid A)=1-\mathrm{P}(B \mid A)=0{,}05 ; \mathrm{P}(\overline{B} \mid \overline{A})=1-\mathrm{P}(B \mid \overline{A})=0{,}98$.
	Ta có sơ đồ hình cây như sau:
	\begin{center}
	\begin{tikzpicture}[yscale=0.7]
	\def\gocm{20}
	\def\gocn{10}
	\def\r{3.5}
	\tikzset{s/.style={outer sep=0.5 mm,draw=magenta,rectangle,minimum width=3cm,rounded corners=1mm}}
	\path(0,0)node(O){}++(\gocm:\r)node[s](A1){A}++(\gocn:\r)node[s](A2){$B$}++(0:\r)node[s](A3){$AB$}++(0:\r)node[s](A4){$0{,}0095$};
	\path(A1)++({-\gocn}:\r)node[s](a2){$\overline{B}$}++(0:\r)node[s](a3){$A\overline{B}$}++(0:\r)node[s](a4){$0{,}0005$};
	\path(O)++(-\gocm:\r)node[s](B1){$\overline{A}$}++(\gocn:\r)node[s](B2){$B$}++(0:\r)node[s](B3){$\overline{A}B$}++(0:\r)node[s](B4){$0{,}0198$};
	\path(B1)++({-\gocn}:\r)node[s](b2){$\overline{B}$}++(0:\r)node[s](b3){$\overline{A}\overline{B}$}++(0:\r)node[s](b4){$0{,}9702$};
	\foreach \x/\y in {
	O/A1,A1/A2,
	O/B1,B1/B2,
	A1/a2,
	B1/b2}
	\draw[-stealth](\x.east)--(\y.west);
	\path(O)--(A1.west)node[pos=0.5,above,sloped]{$0{,}01$}(O)--(B1.west)node[pos=0.5,below,sloped]{$0{,}99$}(B1.east)--(B2.west)node[pos=0.5,above,sloped]{\tiny$0{,}02$}(A1.east)--(A2.west)node[pos=0.5,above,sloped]{\tiny$0{,}95$}
	(A1.east)--(a2.west)node[pos=0.5,below,sloped]{\tiny$0{,}05$}
	(B1.east)--(b2.west)node[pos=0.5,below,sloped]{\tiny$0{,}98$};
	\end{tikzpicture}
	\end{center}
	Do $M=A B$ nên $\mathrm{P}(M)=\mathrm{P}(A B)=0{,}0095$.\\
	Do $N=\overline{A} B$ nên $\mathrm{P}(N)=\mathrm{P}(\overline{A} B)=0{,}0198$.
	}
\end{vd}
\begin{vd}%[2D5H2-3]
	Theo kết quả từ trạm nghiên cứu khí hậu tại địa phương $T$, xác suất để một ngày có gió là $0{,}6$; nếu ngày có gió thì xác suất có mưa là $0{,}4$; nếu ngày không có gió thì xác suất có mưa là $0{,}2$. Gọi $G$ là biến cố \lq\lq  Ngày có gió\rq\rq~ và $M$ là biến cố \la\la Ngày có mưa\rq\rq.
	\begin{listEX}
	\item Vẽ lại sơ đồ hình cây sau và điền vào ô? các giá trị xác suất tương ứng.
	\begin{center}
	\begin{tikzpicture}[node distance=1.5cm, every node/.style={fill=none}, align=center,scale=0.6]
	\definecolor{diagram_bg_green}{HTML}{d5e8d4}
	\definecolor{diagram_bg_blue}{HTML}{dae8fc}
	\definecolor{diagram_bg_pink}{HTML}{f8cecc}
	\definecolor{diagram_bd_green}{HTML}{82b366}
	\definecolor{diagram_bd_blue}{HTML}{7494c2}
	\definecolor{diagram_bd_pink}{HTML}{b85450}
	\tikzset{>={Latex[width=2mm,length=2mm]}, base/.style = {rectangle, rounded corners, draw=black, text centered, drop shadow={shadow xshift=0.6mm, shadow yshift=-0.6mm}},
	Style1/.style = {base, fill=diagram_bg_blue, draw=diagram_bd_blue},
	Style2/.style = {base, fill=diagram_bg_pink, draw=diagram_bd_pink},
	Style3/.style = {base, fill=diagram_bg_green, draw=diagram_bd_green},
	Style4/.style = {base, minimum width=2.5cm, fill=orange!15, draw=orange},
	}
	\node (B0) [Style1, text width=1cm] {Ngày};
	\node (B1) [Style2, right of=B0, xshift=3cm, yshift=1cm, text width=3.5cm] {Có gió $(G)$};
	\node (B3) [Style2, right of=B0, xshift=3cm, yshift=-1cm, text width=3.5cm] {Không có gió $(\overline{G})$};
	\node (B11) [Style3, right of=B1, xshift=6cm, yshift=0.5cm, text width=3cm] {Có mưa};
	\node (B13) [Style3, right of=B1, xshift=6cm, yshift=-0.5cm, text width=3cm] {Không có mưa};
	\node (B31) [Style3, right of=B3, xshift=6cm, yshift=0.5cm, text width=3cm] {Có mưa};
	\node (B33) [Style3, right of=B3, xshift=6cm, yshift=-0.5cm, text width=3cm] {Không có mưa};
	\draw[->] (B0) -- (B1.west)node[sloped,above,pos=0.5]{$\mathrm{P}(G)=?$};
	\draw[->] (B0) -- (B3.west)node[sloped,below,pos=0.5]{$\mathrm{P}(\overline{G})=?$};
	\draw[->] (B1) -- (B11.west)node[sloped,above,pos=0.5]{$\mathrm{P}(M\mid G)=?$};
	\draw[->] (B1) -- (B13.west)node[sloped,below,pos=0.5]{$\mathrm{P}(\overline{M}\mid G)=?$};
	\draw[->] (B3) -- (B31.west)node[sloped,above,pos=0.5]{$\mathrm{P}(M\mid\overline{G})=?$};
	\draw[->] (B3) -- (B33.west)node[sloped,below,pos=0.5]{$\mathrm{P}(\overline{M}\mid\overline{G})=?$};
	\end{tikzpicture}
	\end{center}
	\item Tính xác suất $\mathrm{P}(G M)$ và $\mathrm{P}(G \overline{M})$. Nêu ý nghĩa của các xác suất này.
	\end{listEX}
	\loigiai{
	\begin{listEX}
	\item Theo đề bài, nếu ngày có gió thì xác suất có mưa là 0{,}4 nên $\mathrm{P}(M \mid G)=0{,}4$.\\
	Suy ra $\mathrm{P}(\overline{M} \mid G)=1-0{,}4=0{,}6$.
	Ngày không có gió thì xác suất có mưa là $0{,}2$ nên $\mathrm{P}(M \mid \overline{G})=0{,}2$.\\
	Suy ra $\mathrm{P}(\overline{M} \mid \overline{G})=1-0{,}2=0{,}8$.
	\begin{center}
	\begin{tikzpicture}[node distance=1.5cm, every node/.style={fill=none}, align=center,scale=0.6]
	\definecolor{diagram_bg_green}{HTML}{d5e8d4}
	\definecolor{diagram_bg_blue}{HTML}{dae8fc}
	\definecolor{diagram_bg_pink}{HTML}{f8cecc}
	\definecolor{diagram_bd_green}{HTML}{82b366}
	\definecolor{diagram_bd_blue}{HTML}{7494c2}
	\definecolor{diagram_bd_pink}{HTML}{b85450}
	\tikzset{>={Latex[width=2mm,length=2mm]}, base/.style = {rectangle, rounded corners, draw=black, minimum width=1cm, text centered, drop shadow={shadow xshift=0.6mm, shadow yshift=-0.6mm}},
	Style1/.style = {base, fill=diagram_bg_blue, draw=diagram_bd_blue},
	Style2/.style = {base, fill=diagram_bg_pink, draw=diagram_bd_pink},
	Style3/.style = {base, fill=diagram_bg_green, draw=diagram_bd_green},
	Style4/.style = {base, minimum width=2.5cm, fill=orange!15, draw=orange},
	}
	\node (B0) [Style1, text width=1cm] {Ngày};
	\node (B1) [Style2, right of=B0, xshift=3cm, yshift=1cm, text width=3cm] {Có gió $(G)$};
	\node (B3) [Style2, right of=B0, xshift=3cm, yshift=-1cm, text width=3cm] {Không có gió $(\overline{G})$};
	\node (B11) [Style3, right of=B1, xshift=6cm, yshift=0.5cm, text width=4cm] {Có mưa};
	\node (B13) [Style3, right of=B1, xshift=6cm, yshift=-0.5cm, text width=4cm] {Không có mưa};
	\node (B31) [Style3, right of=B3, xshift=6cm, yshift=0.5cm, text width=4cm] {Có mưa};
	\node (B33) [Style3, right of=B3, xshift=6cm, yshift=-0.5cm, text width=4cm] {Không có mưa};
	\draw[->] (B0) -- (B1.west)node[sloped,above,pos=0.5]{$\mathrm{P}(G)=0{,}6$};
	\draw[->] (B0) -- (B3.west)node[sloped,below,pos=0.5]{$\mathrm{P}(\overline{G})=0{,}4$};
	\draw[->] (B1) -- (B11.west)node[sloped,above,pos=0.5]{$\mathrm{P}(M\mid G)=0{,}4$};
	\draw[->] (B1) -- (B13.west)node[sloped,below,pos=0.5]{$\mathrm{P}(\overline{M}\mid G)=0{,}6$};
	\draw[->] (B3) -- (B31.west)node[sloped,above,pos=0.5]{$\mathrm{P}(M\mid\overline{G})=0{,}2$};
	\draw[->] (B3) -- (B33.west)node[sloped,below,pos=0.5]{$\mathrm{P}(\overline{M}\mid\overline{G})=0{,}8$};
	\end{tikzpicture}
	\end{center}
	\item $\mathrm{P}(G M)=\mathrm{P}(G) \cdot \mathrm{P}(M \mid G)=0{,}6 \cdot 0{,}4=0{,}24 $.\\
	$\mathrm{P}(G \overline{M})=\mathrm{P}(G) \cdot \mathrm{P}(\overline{M} \mid G)=0{,}6 \cdot 0{,}6=0{,}36$.\\
	Điểu này có nghĩa là tại địa phương $T$, trong một ngày, xác suất để trời vừa có gió và vừa có mưa là $0{,}24$; xác suất để trời có gió nhưng không có mưa là $0{,}36$.
	\end{listEX}
	}
\end{vd}
%----------------------------
\subsubsection{Bài tập áp dụng}
\begin{bt}%[2D5H2-3]
	Bạn Việt chuẩn bị đi tham quan một hòn đảo trong hai ngày thứ Bảy và Chủ nhật. Ở hòn đảo đó, mỗi ngày chỉ có nắng hoặc mưa, nếu một ngày là nắng thì khả năng xảy ra mưa ở ngày tiếp theo là $20 \%$, còn nếu một ngày là mưa thì khả năng ngày hôm sau vẫn mưa là $30 \%$. Theo dự báo thời tiết, xác suất trời sẽ nắng vào thứ Bảy là $0{,}7$.
	Hãy tìm các giá trị thích hợp thay vào \mbox{?} ở sơ đồ hình cây sau:
	\begin{center}
	\begin{tikzpicture}[yscale=0.7]
	\def\gocm{20}
	\def\gocn{10}
	\def\r{4}
	\tikzset{s/.style={outer sep=0.5 mm,draw=magenta,rectangle,minimum width=2.75cm,rounded corners=1mm}}
	\path(0,0)node(O){}++(\gocm:\r)node[s](A1){Nắng}++(\gocn:\r)node[s](A2){Nắng};
	\path(A1)++({-\gocn}:\r)node[s](a2){Mưa};
	\path(O)++(-\gocm:\r)node[s](B1){Mưa}++(\gocn:\r)node[s](B2){Nắng};
	\path(B1)++({-\gocn}:\r)node[s](b2){Mưa};
	\foreach \x/\y in {
	O/A1,A1/A2,
	O/B1,B1/B2,
	A1/a2,
	B1/b2}
	\draw[-stealth](\x.east)--(\y.west);
	\path(O)--(A1.west)node[pos=0.5,above,sloped]{$\mbox{0{,}7}$}(O)--(B1.west)node[pos=0.5,below]{$\mbox{?}$}(B1.east)--(B2.west)node[pos=0.5,above]{$\mbox{?}$}(A1.east)--(A2.west)node[pos=0.5,above]{$\mbox{?}$}
	(A1.east)--(a2.west)node[pos=0.5,below,sloped]{$\mbox{0{,}2}$}
	(B1.east)--(b2.west)node[pos=0.5,below,sloped]{$\mbox{0{,}3}$};
	%%Node dòng trên
	\path(A2)++(0,1)node{\textbf{Chủ nhật}}++(180:4)node{\textbf{Thứ bảy}};
	\end{tikzpicture}
	\end{center}
	\loigiai{
	Gọi $A$ là biến cố \lq\lq  Ngày thứ Bảy trời nắng\rq\rq và $B$ là biến cố \lq\lq  Ngày Chủ nhật trời mưa\rq\rq.\\
	Ta có $\mathrm{P}(A)=0{,}7 ; \mathrm{P}(B \mid A)=0{,}2 ; \mathrm{P}(B \mid \overline{A})=0{,}3$.\\
	Do đó $\mathrm{P}(\overline{A})=1-\mathrm{P}(A)=0{,}3 ;\, \mathrm{P}(\overline{B} \mid A)=1-\mathrm{P}(B \mid A)=0{,}8 ;\, \mathrm{P}(\overline{B} \mid \overline{A})=1-\mathrm{P}(B \mid \overline{A})=0{,}7$.
	Áp dụng công thức nhân xác suất, ta có xác suất trời nắng vào thứ Bảy và trời mưa vào Chủ nhật là
	$$
	\mathrm{P}(A B)=\mathrm{P}(A) \mathrm{P}(B \mid A)=0{,}7\cdot 0{,}2=0{,}14 .
	$$
	Tương tự, ta có
	\allowdisplaybreaks
	\begin{eqnarray*}
	&&\mathrm{P}(A \overline{B})=\mathrm{P}(A) \mathrm{P}(\overline{B} \mid A)=0{,}7\cdot 0{,}8=0{,}56 ; \\
	&&\mathrm{P}(\overline{A} B)=\mathrm{P}(\overline{A}) \mathrm{P}(B \mid \overline{A})=0{,}3\cdot 0{,}3=0{,}09 ; \\
	&&\mathrm{P}(\overline{A} \overline{B})=\mathrm{P}(\overline{A}) \mathrm{P}(\overline{B} \mid \overline{A})=0{,}3\cdot0{,}7=0{,}21.
	\end{eqnarray*}
	Ta có thể biểu diễn các kết quả trên theo sơ đồ hình cây như sau:
	\begin{center}
	\begin{tikzpicture}[yscale=0.7]
	\def\gocm{20}
	\def\gocn{10}
	\def\r{4}
	\tikzset{s/.style={outer sep=0.5 mm,draw=magenta,rectangle,minimum width=2.75cm,rounded corners=1mm}}
	\path(0,0)node(O){}++(\gocm:\r)node[s](A1){A}++(\gocn:\r)node[s](A2){$\overline{B}$}++(0:\r)node[s](A3){$A\overline{B}$}++(0:\r)node[s](A4){$0{,}56$};
	\path(A1)++({-\gocn}:\r)node[s](a2){B}++(0:\r)node[s](a3){$AB$}++(0:\r)node[s](a4){$0{,}14$};
	\path(O)++(-\gocm:\r)node[s](B1){$\overline{A}$}++(\gocn:\r)node[s](B2){$\overline{B}$}++(0:\r)node[s](B3){$\overline{A}\overline{B}$}++(0:\r)node[s](B4){$0{,}21$};
	\path(B1)++({-\gocn}:\r)node[s](b2){$B$}++(0:\r)node[s](b3){$\overline{A}B$}++(0:\r)node[s](b4){$0{,}09$};
	\foreach \x/\y in {
	O/A1,A1/A2,
	O/B1,B1/B2,
	A1/a2,
	B1/b2}
	\draw[-stealth](\x.east)--(\y.west);
	\path(O)--(A1.west)node[pos=0.5,above,sloped]{$0{,}7$}(O)--(B1.west)node[pos=0.5,below,sloped]{$0{,}3$}(B1.east)--(B2.west)node[pos=0.5,above,sloped]{$0{,}7$}(A1.east)--(A2.west)node[pos=0.5,above,sloped]{$0{,}8$}
	(A1.east)--(a2.west)node[pos=0.5,below,sloped]{$0{,}2$}
	(B1.east)--(b2.west)node[pos=0.5,below,sloped]{$0{,}3$};
	%%Node dòng trên
	\path(A2)++(0,1)node{\textbf{Chủ nhật}}++(180:4)node{\textbf{Thứ bảy}}(A3)++(0,1)node{\textbf{Kết quả}}(A4)++(0,1)node{\textbf{Xác suất}};
	\end{tikzpicture}
	\end{center}}
\end{bt}
\begin{bt}%[2D5H2-3]
	Hộp thứ nhất có $4$ viên bi xanh và $6$ viên bi đỏ. Hộp thứ hai có $5$ viên bi xanh và $4$ viên bi đỏ. Các viên bi có cùng kích thước và khối lượng. Lấy ra ngẫu nhiên $1$ viên bi từ hộp thứ nhất chuyển sang hộp thứ hai. Sau đó lại lấy ra ngẫu nhiên $1$ viên bi từ hộp thứ hai.
	Sử dụng sơ đồ hình cây, tính xác suất của các biến cố:
	$A$ : \lq\lq  Viên bi lấy ra từ hộp thứ nhất có màu xanh và viên bi lấy ra từ hộp thứ hai có màu đỏ\rq\rq ;
	$B$ : \lq\lq  Hai viên bi lấy ra có cùng màu\rq\rq.
	\loigiai{
	Gọi $X$ là biến cố: \lq\lq  Viên bi lấy ra từ hộp thứ nhất có màu xanh\rq\rq.\\
	$Y$ là biến cố: \lq\lq  Viên bi lấy ra từ hộp thứ hai có màu đỏ\rq\rq.\\
	Ta có
	$
	\mathrm{P}(Y|X)=0{,}4; \mathrm{P}(Y \mid \overline{X})=0{,}5 ; \mathrm{P}(X)=0{,}4
	$.\\
	Do đó $\mathrm{P}(\overline{X})=1-\mathrm{P}(X)=0{,}6 ; \mathrm{P}(\overline{Y} |X)=1-\mathrm{P}(Y|X)=0{,}6 ; \\\mathrm{P}(\overline{Y} \mid \overline{X})=1-\mathrm{P}(Y \mid \overline{X})=0{,}5$.\\
	Ta có sơ đồ hình cây như sau
	\begin{center}
	\begin{tikzpicture}[yscale=0.7]
	\def\gocm{20}
	\def\gocn{10}
	\def\r{4}
	\tikzset{s/.style={outer sep=0.5 mm,draw=magenta,rectangle,minimum width=2.75cm,rounded corners=1mm}}
	\path(0,0)node(O){}++(\gocm:\r)node[s](A1){X}++(\gocn:\r)node[s](A2){$Y$};
	\path(A1)++({-\gocn}:\r)node[s](a2){$\overline{Y}$};
	\path(O)++(-\gocm:\r)node[s](B1){$\overline{X}$}++(\gocn:\r)node[s](B2){$Y$};
	\path(B1)++({-\gocn}:\r)node[s](b2){$\overline{Y}$};
	\foreach \x/\y in {
	O/A1,A1/A2,
	O/B1,B1/B2,
	A1/a2,
	B1/b2}
	\draw[-stealth](\x.east)--(\y.west);
	\path(O)--(A1.west)node[pos=0.5,above,sloped]{$0{,}4$}(O)--(B1.west)node[pos=0.5,below,sloped]{$0{,}6$}(B1.east)--(B2.west)node[pos=0.5,above,sloped]{$0{,}5$}(A1.east)--(A2.west)node[pos=0.5,above,sloped]{$0{,}4$}
	(A1.east)--(a2.west)node[pos=0.5,below,sloped]{$0{,}6$}
	(B1.east)--(b2.west)node[pos=0.5,below,sloped]{$0{,}5$};
	\end{tikzpicture}
	\end{center}
	Khi đó $\mathrm{P}(A)=\mathrm{P}(XY)=0{,}4\cdot 0{,}4=0{,}16$;\\ $\mathrm{P}(B)=\mathrm{P}(X\overline{Y})+\mathrm{P}(\overline{X}Y)=0{,}4\cdot 0{,}6+0{,}6\cdot 0{,}5=0{,}54$.
	}
\end{bt}
\begin{bt}%[2D5V2-3]
	Một trường đại học tiến hành khảo sát tình trạng việc làm sau khi tốt nghiệp của sinh viên. Kết quả khảo sát cho thấy tỉ lệ người tìm được việc làm đúng chuyên ngành là $85 \%$ đối với sinh viên tốt nghiệp loại giỏi và $70 \%$ đối với sinh viên tốt nghiệp loại khác.
	Tỉ lệ sinh viên tốt nghiệp loại giỏi là $30 \%$. Gặp ngẫu nhiên một sinh viên đã tốt nghiệp của trường.
	Sử dụng sơ đồ hình cây, tính xác suất của các biến cố:
	C: \lq\lq  Sinh viên tốt nghiệp loại giỏi và tìm được việc làm đúng chuyên ngành\rq\rq;
	$D$ : \lq\lq  Sinh viên không tốt nghiệp loại giỏi và tìm được việc làm đúng chuyên ngành\rq\rq.
	\loigiai{
	Gọi $X$ là biến cố: \lq\lq  Sinh viên tốt nghiệp loại Giỏi\rq\rq.\\
	$Y$ là biến cố: \lq\lq  Sinh viên tìm được việc làm đúng chuyên ngành\rq\rq.\\
	Ta có
	$
	\mathrm{P}(Y|X)=0{,}85; \mathrm{P}(Y \mid \overline{X})=0{,}7 ; \mathrm{P}(X)=0{,}3
	$.\\
	Do đó $\mathrm{P}(\overline{X})=1-\mathrm{P}(X)=0{,}7 ; \mathrm{P}(\overline{Y} |X)=1-\mathrm{P}(Y|X)=0{,}15 ; \\\mathrm{P}(\overline{Y} \mid \overline{X})=1-\mathrm{P}(Y \mid \overline{X})=0{,}3$.\\
	Ta có sơ đồ hình cây như sau
	\begin{center}
	\begin{tikzpicture}[yscale=0.7]
	\def\gocm{20}
	\def\gocn{10}
	\def\r{4}
	\tikzset{s/.style={outer sep=0.5 mm,draw=magenta,rectangle,minimum width=2.75cm,rounded corners=1mm}}
	\path(0,0)node(O){}++(\gocm:\r)node[s](A1){X}++(\gocn:\r)node[s](A2){$Y$};
	\path(A1)++({-\gocn}:\r)node[s](a2){$\overline{Y}$};
	\path(O)++(-\gocm:\r)node[s](B1){$\overline{X}$}++(\gocn:\r)node[s](B2){$Y$};
	\path(B1)++({-\gocn}:\r)node[s](b2){$\overline{Y}$};
	\foreach \x/\y in {
	O/A1,A1/A2,
	O/B1,B1/B2,
	A1/a2,
	B1/b2}
	\draw[-stealth](\x.east)--(\y.west);
	\path(O)--(A1.west)node[pos=0.5,above,sloped]{$0{,}3$}(O)--(B1.west)node[pos=0.5,below,sloped]{$0{,}7$}(B1.east)--(B2.west)node[pos=0.5,above,sloped]{$0{,}7$}(A1.east)--(A2.west)node[pos=0.5,above,sloped]{$0{,}85$}
	(A1.east)--(a2.west)node[pos=0.5,below,sloped]{$0{,}15$}
	(B1.east)--(b2.west)node[pos=0.5,below,sloped]{$0{,}3$};
	\end{tikzpicture}
	\end{center}
	Khi đó $\mathrm{P}(C)=\mathrm{P}(XY)=0{,}3\cdot 0{,}85=0{,}255$; $\mathrm{P}(B)=\mathrm{P}(\overline{X}Y)=0{,}7\cdot 0{,}7=0{,}49$.
	}
\end{bt}
%=====================
\begin{dang}{Công thức nhân xác suất}
	Với hai biến cố $A$ và $B$ bất kì, ta có
	$$\mathrm{P}(A B)=\mathrm{P}(B) \cdot \mathrm{P}(A \mid B).$$
\end{dang}
%----------------------------
\subsubsection{Ví dụ minh hoạ}
\begin{vd}%[2D5H2-4]
	Trong một hộp kín có 7 chiếc bút bi xanh và 5 chiếc bút bi đen, các chiếc bút có cùng kích thước và khối lượng. Bạn Sơn lấy ngẫu nhiên một chiếc bút bi từ trong hộp, không trả lại. Sau đó bạn Tùng lấy ngẫu nhiên một trong 11 chiếc bút còn lại. Tính xác suất để Sơn lấy được bút bi đen và Tùng lấy được bút bi xanh.
	\loigiai{
	Gọi $A$ là biến cố: ``Bạn Sơn lấy được bút bi đen'';\\
	$B$ là biến cố: ``Bạn Tùng lấy được bút bi xanh''.\\
	Ta cần tính $\mathrm{P}(AB)$.\\
	Vì $n(A)=5$ nên $\mathrm{P}(A)=\dfrac{5}{12}$.\\
	Nếu $A$ xảy ra tức là bạn Sơn lấy được bút bi đen thì trong hộp có 11 bút bi với 7 bút bi xanh.\\
	Vậy $\mathrm{P}(B\mid A)=\dfrac{7}{11}$.\\
	Theo công thức nhân xác suất: $\mathrm{P}(AB)=\mathrm{P}(A)\cdot \mathrm{P}(B\mid A)=\dfrac{5}{12}\cdot\dfrac{7}{11}=\dfrac{35}{132}$.\\}
\end{vd}
\begin{vd}
	Cho hai biến cố $A$ và $B$ có $\mathrm{P}(A)=0{,}3 ; \mathrm{P}(B)=0{,}5$ và $\mathrm{P}(A \mid B)=0{,}4$. Tính $\mathrm{P}(\overline{A} B)$ và $\mathrm{P}(\overline{A} \mid B)$.
	\loigiai
	{
	\immini
	{
	Theo công thức nhân xác suất, ta có $\mathrm{P}(A B)=\mathrm{P}(B) \cdot \mathrm{P}(A \mid B)=0{,}2$.\\
	Vì $\overline{A} B$ và $A B$ là hai biến cố xung khắc và $\overline{A} B \cup A B=B$ nên theo tính chất của xác suất, ta có $\mathrm{P}(\overline{A} B)=\mathrm{P}(B)-\mathrm{P}(A B)=0{,}3$.\\
	Theo công thức tính xác suất có điều kiện,
	$$
	\mathrm{P}(\overline{A} \mid B)=\dfrac{\mathrm{P}(\overline{A} B)}{\mathrm{P}(B)}=\dfrac{0{,}3}{0{,}5}=0{,}6.$$
	}
	{
	\begin{tikzpicture}[scale=0.54]
	\def\firstven{(0,0) ellipse (3cm and 2cm)}
	\def\secondven{(2.5,1) ellipse (2.8cm and 2cm)}
	\begin{scope}
	\clip \firstven;
	\fill[gray!50,opacity=0.85] \secondven;
	\end{scope}
	\draw \firstven \secondven;
	\node at (-2.2,2) {$A$};
	\node at (5.6,2.2){$B$};
	\node at (1.3,0.5){$AB$};
	\node at (4,1){$\overline{A} B$};
	\end{tikzpicture}
	}
	}
\end{vd}
\begin{vd}%[2D5H2-4]
	Ô cửa bí mật (Let's Make a Deal) là một trò chơi trên truyền hình nổi tiếng ở Mỹ, đã được mua bản quyền và phát sóng ở nhiều nước trên thế giới. Nội dung trò chơi như sau:
	\begin{itemize}
	\item Người chơi được mời lên sân khấu và đứng trước ba cánh cửa đóng kín. Sau một cánh cửa có chiếc ô tô, sau mỗi cánh cửa còn lại là một con lừa. Người chơi được yêu cầu chọn ngẫu nhiên một cánh cửa, nhưng không được mở ra.
	\item Tiếp đó người quản trò tuyên bố sẽ mở ngẫu nhiên một trong hai cánh cửa người chơi không chọn mà sau cửa đó là con lừa. Người quản trò hỏi người chơi muốn giữ nguyên sự lựa chọn ban đầu của mình hay muốn chuyển sang cửa chưa mở còn lại.
	\end{itemize}
	Giả sử người chơi chọn cửa số 1 và người quản trò mở cửa số 3. Kí hiệu $E_1$; $E_2$; $E_3$ tương ứng là các biến cố: ``Sau ở cửa số 1 có ô tô''; ``Sau ở cửa số 2 có ô tô''; ``Sau ở cửa số 3 có ô tô'' và $H$ là biến cố: ``Người quản trò mở ở cửa số 3 thấy con lừa''.
	Sau khi người quản trò mở cánh cửa số 3 thấy con lừa, tức là khi $H$ xảy ra. Để quyết định thay đổi lựa chọn hay không, người chơi cần so sánh hai xác suất có điều kiện: $P\left(E_1\mid H\right)$ và $P\left(E_2\mid H\right)$
	\begin{listEX}
	\item Chứng minh rằng:
	\begin{enumEX}[\itemCI]{2}
	\item $\mathrm{P}(E_1)=\mathrm{P}(E_2)=\mathrm{P}(E_3)=\dfrac{1}{3}$;
	\item $P\left(H \mid E_1\right)=\dfrac{1}{2} ~\text{và}~P\left(H \mid E_2\right)=1$.
	\end{enumEX}
	\item Sử dụng công thức tính xác suất có điều kiện và công thức nhân xác suất, chứng minh rằng:
	$$ \mathrm{P}\left(E_1 \mid H\right)=\dfrac{\mathrm{P}(E_1) \cdot P\left(H \mid E_1\right)}{\mathrm{P}(H)}$$
	%	\item $P\left(E_2 \mid H\right)=\dfrac{\mathrm{P}(E_2) \cdot P\left(H \mid E_2\right)}{\mathrm{P}(H)$.
	\item Từ các kết quả trên hãy suy ra:
	$$P\left(E_2 \mid H\right)=2 P\left(E_1 \mid H\right)$$
	Từ đó hãy đưa ra lời khuyên cho người chơi: Nên giữ nguyên sự lựa chọn ban đầu hay chuyển sang cửa chưa mở còn lại?\\
	\end{listEX}
	\loigiai{
	\begin{listEX}
	\item Không gian mẫu $\Omega$ là tập hợp gồm 3 phần thưởng (1 ô tô + 2 con lừa) $\Rightarrow n(\Omega)=3$.
	Ta có $n(E_1)=n(E_2)=n(E_3)=1$.\\
	Suy ra $\mathrm{P}(E_1)=\mathrm{P}(E_2)=\mathrm{P}(E_3)=\dfrac{1}{3}$.\\
	Nếu $E_1$ xảy ra, tức là sau cửa số 1 có ô tô. Khi đó, sau cửa số 2 và 3 là con lừa. Người quản trò chọn ngẫu nhiên một trong hai cửa số 2 và 3 để mở ra. Do đó, việc chọn cửa số 2 hay cửa số 3 có khả năng như nhau. \\
	Vậy $P\left(H\mid E_1\right)=\dfrac{1}{2}$.\\
	Nếu $E_2$ xảy ra, tức là cửa số 2 có ô tô. Khi đó, người quản trò chắc chắn phải mở cửa số 3.\\
	Do đó $P\left(H\mid E_2\right)=1$.
	\item Ta có $$\begin{aligned} &P\left(E_1\mid H\right)=\dfrac{\mathrm{P}(E_1H)}{\mathrm{P}(H)}\\\Leftrightarrow &\mathrm{P}(E_1H)=\mathrm{P}(E_1)\cdot P\left(H\mid E_1\right)\\\Leftrightarrow &P\left(E_1\mid H\right)=\dfrac{\mathrm{P}(E_1)\cdot P\left(H\mid E_1\right)}{\mathrm{P}(H)}.\end{aligned}$$
	\item Ta có $$P\left(E_2 \mid H\right)=\dfrac{\mathrm{P}(E_2) \cdot P\left(H\mid E_2\right)}{\mathrm{P}(H)}.$$
	Suy ra
	$$\begin{aligned}
	\dfrac{P\left(E_2 \mid H\right)}{P\left(E_1 \mid H\right)}&=\dfrac{\mathrm{P}(E_2) \cdot P\left(H \mid E_2\right)}{\mathrm{P}(E_1) \cdot P\left(H \mid E_1\right)}\\
	&=\dfrac{\dfrac{1}{3} \cdot 1}{\dfrac{1}{3} \cdot\dfrac{1}{2}}=2.
	\end{aligned}$$
	Suy ra $P\left(E_2\mid H\right)=2\cdot P\left(E_1\mid H\right)$.\\
	\textbf{\textit{Nhận xét: }}Từ kết quả trên ta thấy người chơi nên chuyển sang cửa chưa mở còn lại để tăng gấp đôi khả năng trúng thưởng chiếc ô tô.
	\end{listEX}
	}
\end{vd}
\begin{vd}%[2D5H2-4]
	Một nhóm $5$ học sinh nam và $4$ học sinh nữ tham gia lao động trên sân trường. Cô giáo chọn ngẫu nhiên đồng thời $2$ bạn trong nhóm đi tưới cây. Tính xác suất để hai bạn được chọn có cùng giới tính, biết rằng có ít nhất $1$ bạn nam được chọn.
	\loigiai{
	Số phần tử của không gian mẫu là $n(\Omega)=C^2_9=36$.\\
	Gọi A là biến cố \lq\lq  Hai bạn được chọn có cùng giới tính\rq\rq.\\
	B là biến cố \lq\lq  Có ít nhất một bạn nam được chọn\rq\rq.\\
	Ta có $n(B)=C^2_5+C^1_5\cdot C^1_4=30$ suy ra $\mathrm{P}(B)=\dfrac{30}{36}$.\\
	Ta có $n(AB)=C^2_5=10$ suy ra $\mathrm{P}(AB)=\dfrac{10}{36}$.\\
	Vậy $\mathrm{P}(A|B)=\dfrac{\mathrm{P}(AB)}{\mathrm{P}(B)}=\dfrac{10}{30}=\dfrac{1}{3}$.
	}
\end{vd}
%----------------------------
\subsubsection{Bài tập áp dụng}
\begin{bt}%[2D5H1-2]
	Cho hai biến cố $A$ và $B$ có $\mathrm{P}(A)=0{,}4; \mathrm{P}(B)=0{,}8$ và $\mathrm{P}(A|\overline{B})=0{,}5$. Tính $\mathrm{P}(A\overline{B})$ và $\mathrm{P}(A|B)$.
	\loigiai{
	Ta có $\mathrm{P}(A\overline{B})=\mathrm{P}(A|\overline{B})\cdot \mathrm{P}(\overline{B})=0{,}5\cdot 0{,}2=0{,}1$.\\
	Vì $AB$ và $A\overline{B}$ là hai biến cố xung khắc và $AB\cup A\overline{B}=A$ nên theo tính chất của xác suất, ta có\\ $\mathrm{P}(AB)=\mathrm{P}(A)-\mathrm{P}(A\overline{B})=0{,}4-0{,}1=0{,}3$.\\
	Khi đó: $\mathrm{P}(A|B)=\dfrac{\mathrm{P}(AB)}{\mathrm{P}(B)}=\dfrac{0{,}3}{0{,}8}=0{,}375$.
	}
\end{bt}
\begin{bt}%[2D5H2-3]
	\immini{Máy tính và thiết bị lưu điện (UPS) được kết nối như hình 5. Khi xảy ra sự cố điện, UPS bị hỏng với xác suất $0{,}02$. Nếu UPS bị hỏng khi xảy ra sự cố điện, máy tính sẽ bị hỏng với xác suất $0{,}1$; ngược lại, nếu UPS không bị hỏng, máy tính sẽ không bị bỏng.
	\begin{listEX}
	\item Tính xác suất để cả UPS và máy tính đều không bị hỏng khi xảy ra sự cố điện.
	\item Tính xác suất để cả UPS và máy tính đều bị hỏng khi xảy ra sự cố điện.
	\end{listEX}
	}
	{\includegraphics[width=6.5cm,height=4cm]{images/12-SGK-CTST-6-1-5}}
	\loigiai{
	Gọi $A$ là biến cố \lq\lq  UPS bình thường\rq\rq\, và $B$ là biến cố: \lq\lq  Máy tính bình thường\rq\rq.\\
	Ta có
	$
	\mathrm{P}(B|A)=1; \mathrm{P}(B\mid \overline{A})=0{,}9 ; \mathrm{P}(A)=0{,}98
	$.\\
	Do đó $\mathrm{P}(\overline{A})=1-\mathrm{P}(A)=0{,}02 ; \mathrm{P}(\overline{B} \mid \overline{A})=1-\mathrm{P}(B \mid \overline{A})=0{,}1$.\\
	Ta có sơ đồ hình cây như sau
	\begin{center}
	\begin{tikzpicture}[yscale=0.6]
	\def\gocm{20}
	\def\gocn{10}
	\def\r{4}
	\tikzset{s/.style={outer sep=0.5 mm,draw=magenta,rectangle,minimum width=2.75cm,rounded corners=1mm}}
	\path(0,0)node(O){}++(\gocm:\r)node[s](A1){A}++(\gocn:\r)node[s](A2){$B$};
	\path(O)++(-\gocm:\r)node[s](B1){$\overline{A}$}++(\gocn:\r)node[s](B2){$B$};
	\path(B1)++({-\gocn}:\r)node[s](b2){$\overline{B}$};
	\foreach \x/\y in {
	O/A1,A1/A2,
	O/B1,B1/B2,
	B1/b2}
	\draw[-stealth](\x.east)--(\y.west);
	\path(O)--(A1.west)node[pos=0.5,above,sloped]{$0{,}98$}(O)--(B1.west)node[pos=0.5,below,sloped]{$0.02$}(B1.east)--(B2.west)node[pos=0.5,above,sloped]{$0{,}9$}(A1.east)--(A2.west)node[pos=0.5,above,sloped]{$1$}
	(B1.east)--(b2.west)node[pos=0.5,below,sloped]{$0{,}1$};
	\end{tikzpicture}
	\end{center}
	\begin{listEX}
	\item $\mathrm{P}(AB)=0{,}98\cdot 1=0{,}98$
	\item $\mathrm{P}(\overline{A}.\overline{B})=0{,}02\cdot 0{,}1=0{,}002$.
	\end{listEX}
	}
\end{bt}
\begin{bt}%[2D5H2-4]
	Công ty nước giải khát $X$ tổ chức một chương trình khuyến mại như sau: Trong mỗi thùng 24 chai nước giải khát đều có hai chai trúng thưởng (giải thưởng được viết ở dưới nắp chai), người tham gia chương trình được mở nắp một cách ngẫu nhiên lần lượt hai chai trong một thùng. Tính xác suất để một người tham gia chương trình mở được cả hai chai đều trúng thưởng.
	\loigiai{
		Gọi $A$ là biến cố \lq\lq  chai thứ nhất có trúng thưởng\rq\rq, và $B$ là biến cố \lq\lq  chai thứ hai có trúng thưởng\rq\rq.\\
		Xác suất của $A$ là xác suất để lấy ra một chai có trúng thưởng lần đầu tiên là $$\mathrm{P}(A)=\dfrac{n(A)}{n(\Omega)}=\dfrac{2}{24}=\dfrac{1}{12}.$$
		Sau khi lấy một chai trúng thưởng, số chai trúng trưởng còn $1$ trong tổng số $23$ chai nước.
		Xác suất của $B$ khi đã xảy ra $A$ là xác suất để lấy ra một chai trúng thưởng lần thứ hai là $$\mathrm{P}(B \mid A)=\dfrac{1}{23}.$$
		Áp dụng quy tắc nhân xác suất, ta có
		$$\mathrm{P}(A \cap B)=\mathrm{P}(A) \cdot \mathrm{P}(B \mid A)=\dfrac{1}{2} \cdot \dfrac{1}{23}=\dfrac{1}{46}.$$
		Vậy, xác suất để cả hai chai nước đều trúng thưởng là $\dfrac{1}{46}$.
	}
\end{bt}


% %%%%%%%
% \newpage
\subsection{Bài tập tự luận}
% \BTTL
\setcounter{bt}{0}
%%==========Bài 1
\begin{bt}
	Cho $P(A)=0{,}2;P(B)=0{,}51;P(B\mid A)=0{,}8$. Tính $P(A\mid B)$.
	\loigiai{
		Ta có
		$P(AB)=P(A) \cdot P(B \mid A)=0{,}2\cdot 0{,}8=0{,}16$.\\
		$P(AB)=P(B) \cdot P(A \mid B) \Rightarrow P(A \mid B)=\dfrac{P(AB)}{P(B)}=\dfrac{0{,}16}{0{,}51}\approx 0{,}314$.
	}
\end{bt}

%%==========Bài 2
\begin{bt}
	Cho hai biến độc lập $A,B$ với $\mathrm{P}(A)=0{,}8$, $\mathrm{P}(B)=0{,}25$. Tính $\mathrm{P}(A\mid B)$.
	\loigiai{
		Vì $A$ và $B$ là hai biến cố độc lập, do đó
		\[\mathrm{P}(A\mid B)=\dfrac{\mathrm{P}(A\cap B)}{\mathrm{P}(B)}=\dfrac{\mathrm{P}(A)\cdot\mathrm{P}(B)}{\mathrm{P}(B)}=\mathrm{P}(A)=0{,}8.\]	
	}
\end{bt}
%%==========Bài 3
\begin{bt}
	Cho hai biến cố $A$, $B$ có $\mathrm{P}(A)=0{,}6;\mathrm{P}(B)=0{,}8;\mathrm{P}(A\cap B)=0{,}4$. Tính các xác suất sau:
	\begin{listEX}[2]
		\item $\mathrm{P}(B\mid A);\mathrm{P}(\overline{B}\mid A)$.
		\item $\mathrm{P}(A \cap \overline{B})$.
	\end{listEX}
	\loigiai{
		\begin{enumerate}
			\item $\mathrm{P}(B\mid A)=\dfrac{\mathrm{P}(A \cap B)}{\mathrm{P}(A)}=\dfrac{0{,}4}{0{,}6}=\dfrac{2}{3} 
			\Rightarrow \mathrm{P}(\overline{B}\mid A)=1-\mathrm{P}(B\mid A) = 1 -\dfrac{2}{3}=\dfrac{1}{3}$.
			\item Ta có 
			\begin{eqnarray*}
				\mathrm{P}(A \cap \overline{B}) &=& \mathrm{P}\left(\overline{B}\mid A\right)\cdot \mathrm{P} (A)=\dfrac{1}{3}\cdot 0{,}6=0{,}2.
			\end{eqnarray*}
	\end{enumerate}}
\end{bt}
%%==========Bài 5
\begin{bt}%[2D5H1-2]
	Cho hai biến cố $A$ và $B$ có $P(A)=0{.}4; P(B)=0{.}8$ và $P(A|B)=0{.}5$. Tính $P(A\overline{B})$ và $P(A|B)$.
	\loigiai{
		Ta có $P(A\overline{B})=P(A|\overline{B})\cdot P(\overline{B})=0{.}5\cdot 0{.}2=0{.}1$.\\
		Vì $AB$ và $A\overline{B}$ là hai biến cố xung khắc và $AB\cup A\overline{B}=A$ nên theo tính chất của xác suất, ta có\\ $P(AB)=P(A)-P(A\overline{B})=0{.}4-0{.}1=0{.}3$.\\
		Khi đó: $P(A|B)=\dfrac{P(AB)}{P(B)}=\dfrac{0{.}3}{0{.}8}=0{.}375$.
	}
\end{bt}
%%==========Bài 4
\begin{bt}%[2D5B1-2]
	Một thư viện có $35\%$ tổng số sách là sách khoa học, $14\%$ tổng số sách là sách khoa học tự nhiên. Chọn ngẫu nhiên một quyển sách của thư viện. Tính xác suất để quyển sách được chọn là sách khoa học tự nhiên, biết rằng đó là quyển sách về khoa học.
	\loigiai
	{Gọi $A$ là biến cố \lq\lq  Sách được chọn là sách khoa học tự nhiên\rq\rq.\\
	Gọi $B$ là biến cố \lq\lq  Sách được chọn là sách khoa học\rq\rq.\\
	Do có $35\%$ tổng số sách là sách khoa học nên $P(B)=0{.}35$.\\
	Do có $14\%$ tổng số sách là sách khoa học tự nhiên nên $P(AB)=0{.}14$.\\
	Vậy $P(A|B)=\dfrac{P(AB)}{P(B)}=\dfrac{0{.}14}{0{.}35}=0{.}4$.
	}
\end{bt}


%%==========Bài 7
\begin{bt}
	Một hộp kín đựng 20 tấm thẻ giống hệt nhau đánh số từ 1 đến 20. Một người rút ngẫu nhiên ra một tấm thẻ từ trong hộp. Người đó được thông báo rằng thẻ rút ra mang số chẵn. Tính xác suất để người đó rút được thẻ số 10.
	\loigiai{
	Gọi $A$ là biến cố: ``Người đó rút được thẻ số 10''.\\
	Gọi $B$ là biến cố: ``Người đó rút được thẻ mang số chẵn''.\\
	Không gian mẫu mà 20 tấm thẻ đánh số từ 1 đến 20 $\Rightarrow$ $n(\Omega)=20$.\\
	Ta cần tính $P\left(A\mid B\right)$.\\
	Ta có, từ 1 đến 20 có 10 số chẵn nên $n(B)=10$.\\
	Vậy $P(B)=\dfrac{n(B)}{n(\Omega)}=\dfrac{10}{20}=\dfrac{1}{2}$.\\
	Trong số 10 số chẵn có một số 10 nên $n(AB)=1$.\\
	Vậy $P(AB)=\dfrac{n(AB)}{n(\Omega)}=\dfrac{1}{20}$.\\
	Do đó $P(A\mid B)=\dfrac{P(AB)}{P(B)}=\dfrac{1}{10}\approx 0{,}1$.
	}
\end{bt}

%%==========Bài 8
\begin{bt}
	Gieo hai con xúc xắc cân đối, đồng chất. Tính xác suất để:
	\begin{enumerate}[a)]
	\item Tổng số chấm xuất hiện trên hai con xúc xắc bằng 7 nếu biết rằng ít nhất có một con xúc xắc xuất hiện mặt 5 chấm;
	\item Có ít nhất có một con xúc xắc xuất hiện mặt 5 chấm nếu biết rằng tổng số chấm xuất hiện trên hai con xúc xắc bằng 7. 
	\end{enumerate}
	\loigiai{
	\begin{enumerate}[a)]
	\item Không gian mẫu $n(\Omega)=6\cdot 6=36$.\\
	Gọi $A$ là biến cố: ``Tổng số chấm xuất hiện trên hai con xúc xắc bằng 7''.\\
	Gọi $B$ là biến cố: ``Có ít nhất một con xúc xắc xuất hiện mặt 5 chấm''.\\
	Ta có $n(B)=6+6+1=13$ ứng với các trường hợp $(5;x)$; $(x;5)$; $(5;5)$.\\
	Vậy $P(B)=\dfrac{n(B)}{n(\Omega)}=\dfrac{13}{36}$.\\
	Ta có tổng số chấm xuất hiện trên hai con xúc xắc bằng 7 trong đó có ít nhất một con xúc xắc xuất hiện mặt 5 chấm ứng với các trường hợp $(5;2)$ và $(2;5)$. \\
	Suy ra $n(AB)=2$.\\
	Vậy $P(AB)=\dfrac{n(AB)}{n(\Omega)}=\dfrac{2}{36}=\dfrac{1}{18}$.\\
	Do đó 
	$P(A\mid B)=\dfrac{P(AB)}{P(B)}=\dfrac{2}{13}$.
	\item Ta tính $P(B\mid A)$.\\
	Biến cố tổng hai mặt là $7: A=\{(1;6);(2;5);(3;4);(4;3);(5;2);(6;1)\}$ nên $n(A)=6$.\\
	Vậy $P(A)=\dfrac{n(A)}{n(\Omega)}=\dfrac{6}{36}$.\\
	Ta có $P(BA)=P(AB)=\dfrac{1}{18}$.\\
	Do đó 
	$P(B\mid A)=\dfrac{P(BA)}{P(A)}=\dfrac{2}{6}=\dfrac{1}{3}$.
	\end{enumerate}
	}
\end{bt}
%%==========Bài 9
\begin{bt}
	Gieo hai con xúc xắc cân đối, đồng chất. Tính xác suất để tổng số chấm xuất hiện trên hai con xúc xắc đó không nhỏ hơn 10 nếu biết rằng có ít nhất một con xúc xắc xuất hiện mặt 5 chấm.
	\loigiai{
	Không gian mẫu $n(\Omega)=6\cdot 6=36$.\\
	Gọi $A$ là biến cố: ``Tổng số chấm xuất hiện trên hai con xúc xắc không nhỏ hơn 10''.\\
	Gọi $B$ là biến cố: ``Có ít nhất một con xúc xắc xuất hiện mặt 5 chấm''.\\
	Ta có $n(B)=6+6+1=13$ ứng với các trường hợp $(5;x)$; $(x;5)$; $(5;5)$.\\
	Vậy $P(B)=\dfrac{n(B)}{n(\Omega)}=\dfrac{13}{36}$.\\
	Ta có tổng số chấm xuất hiện trên hai con xúc xắc không nhỏ hơn 10 trong đó có ít nhất một con xúc xắc xuất hiện mặt 5 chấm ứng với các trường hợp $(5;5);(5;6);(6;5)$.\\
	Suy ra $n(AB)=3$.\\
	Vậy $P(AB)=\dfrac{n(AB)}{n(\Omega)}=\dfrac{3}{36}=\dfrac{1}{12}$.\\
	Do đó 
	$P(A\mid B)=\dfrac{P(AB)}{P(B)}=\dfrac{3}{13}$.
	}
\end{bt}



%%==========Bài 12
\begin{bt}
	Một hộp có $3$ quả bóng màu xanh, $4$ quả bóng màu đỏ; các quả bóng có kích thước và khối lượng như nhau. Lấy bóng ngẫu nhiên hai lần liên tiếp, trong đó mỗi lần lấy ngẫu nhiên một quả bóng trong hộp, ghi lại màu của quả bóng lấy ra và bỏ lại quả bóng đó vào hộp. Xét các biến cố:
	\begin{enumerate}
	\item $A$ \lq\lq  \,Quả bóng màu xanh được lấy ra ở lần thứ nhất\rq\rq;
	\item $B$ \lq\lq  \,Quả bóng màu đỏ được lấy ra ở lần thứ hai \rq\rq.
	\end{enumerate}
	Chứng minh rằng $A$, $B$ là hai biến cố độc lập.
	\loigiai{
	Gọi $A_i$ là biến cố \lq\lq  Lần thứ $i$ lấy được bóng màu xanh\rq\rq:\\
	$A=A_1A_2 \cup \overline{A_1}~\overline{A_2} \Rightarrow P\left(A\right)=\dfrac{3}{7}\cdot\dfrac{3}{7}+\dfrac{3}{7}\cdot\dfrac{4}{7}=\dfrac{3}{7}$.\\
	Gọi $B_i$ là biến cố \lq\lq  Lần thứ $i$ lấy được bóng màu đỏ\rq\rq:\\
	$B=B_1B_2 \cup \overline{B_1}B_2 \Rightarrow P\left(B\right)=\dfrac{4}{7}\cdot\dfrac{4}{7}+\dfrac{4}{7}\cdot\dfrac{3}{7}=\dfrac{4}{7}$.\\
	Ta có, $\mathrm{P}(A\mid B)=\dfrac{n\left(A \cap B\right)}{n\left(B\right)}=\dfrac{4\cdot 3}{7\cdot 4}=\dfrac{3}{7}=P\left(A\right) \Rightarrow A, B$ là hai biến cố độc lập.}
\end{bt}
%%==========Bài 13
\begin{bt}
	Cho hai con xúc xắc cân đối và đồng chất. Gieo lần lượt từng xúc xắc trong hai xúc xắc đó. Tính xác suất để tổng số chấm xuất hiện trên hai xúc xắc bằng $6$, biết rằng xúc xắc thứ nhất xuất hiện mặt $4$ chấm.
	\loigiai{
	Gọi $A$ là biến cố \lq\lq  xúc xắc thứ nhất xuất hiện mặt $4$ chấm\rq\rq ~và $B$ là biến cố \lq\lq  tổng số chấm xuất hiện trên hai xúc xắc bằng $6$\rq\rq.\\
	Xác suất của $A$ là $\mathrm{P}(A)$ là xác suất để xúc xắc thứ nhất xuất hiện mặt $4$ chấm. Vì xúc xắc cân đối và đồng chất, nên \[\mathrm{P}(A)=\dfrac{1}{6}.\]
	Xác suất của $B$ khi biết $A$ đã xảy ra là $\mathrm{P}(B \mid A)$. Trong trường hợp này, để tổng số chấm là $6$, xúc xắc thứ hai phải xuất hiện mặt $2$ chấm. Do đó, $\mathrm{P}(B \mid A)=\dfrac{1}{6}$.\\
	Vậy, theo quy tắc xác suất điều kiện, ta có:\\
	$$\mathrm{P}(B \mid A)=\dfrac{\mathrm{P}(A \cap B)}{\mathrm{P}(A)} \Rightarrow \mathrm{P}(A \cap B)=\mathrm{P}(B \mid A)\cdot \mathrm{P}(A)=\dfrac{1}{6}\cdot \dfrac{1}{6}=\dfrac{1}{36}.$$
	}
\end{bt}
%%==========Bài 14
\begin{bt}
	Một doanh nghiệp trước khi xuất khẩu áo sơ mi phải qua hai lần kiểm tra chất lượng sản phẩm, nếu cả hai lần đều đạt thì chiếc áo đó mới đủ tiêu chuẩn xuất khẩu. Biết rằng bình quân $98\%$ sản phẩm làm ra qua được lần kiểm tra thứ nhất và $95\%$ sản phẩm qua được lần kiểm tra thứ nhất sẽ tiếp tục qua được lần kiểm tra thứ hai. Tính xác suất để một chiếc áo sơ mi đủ tiêu chuẩn xuất khẩu.
	\loigiai{
	$A$ là biến cố \lq\lq  sản phẩm qua được lần kiểm tra thứ nhất\rq\rq.\\
	$B$ là biến cố \lq\lq  sản phẩm qua được lần kiểm tra thứ hai\rq\rq.\\
	Bài toán yêu cầu tính xác suất của biến cố $A\cap B$, tức là sản phẩm vừa qua được lần kiểm tra thứ nhất, và sau đó qua được lần kiểm tra thứ hai.\\
	Xác suất của $A$ là $\mathrm{P}(A)=0{,}98$.\\
	Xác suất của $B$ khi đã qua được $A$ là $\mathrm{P}(B \mid A)=0{,}95$.\\
	Áp dụng công thức xác suất có điều kiện, ta có:
	$$\mathrm{P}(A \cap B)=\mathrm{P}(A)\cdot \mathrm{P}(B \mid A)=0{,}98\cdot 0{,}95=0{,}931.$$
	Vậy, xác suất để một chiếc áo sơ mi đủ tiêu chuẩn xuất khẩu là $93{,}1\%$
	}
\end{bt}
%%==========Bài 17
\begin{bt}
	Một nhóm $50$ học sinh có $23$ bạn biết chơi cầu lông mà không biết chơi bóng đá và $21$ bạn biết chơi bóng đá mà không biết chơi cầu lông. Biết rằng mỗi học sinh trong nhóm này biết chơi bóng đá hoặc cầu lông. Chọn ngẫu nhiên một học sinh trong nhóm. Tính xác suất học sinh này biết chơi bóng đá, biết rằng bạn ấy biết chơi cầu lông. 
	\loigiai{
		Gọi $A$ là biến cố "Học sinh được chọn biết chơi bóng đá"; $B$ là biến cố "Học sinh được chọn biết chơi cầu lông".\\
		Ta có $n\left(A\cap B\right)=50-(23+21)=6$ và $n(B)=23+6=29$. Do đó 
		$\mathrm{P}(A|B)=\dfrac{n\left(A\cap B\right)}{n(B)}=\dfrac{6}{29}$.
	}
\end{bt}
%%==========Bài 16
\begin{bt}
	Có $2$ linh kiện điện tử, xác suất để mỗi linh kiện hỏng trong một thời điểm bất kì lần lượt là: $0{,}01$; $0{,}02$. Hai linh kiện đó được lắp vào một mạch điện theo sơ đồ ở \textit{Hình a, b}. Trong mỗi trường hợp, hãy tính xác suất để trong mạch điện có dòng điện chạy qua
	\begin{center}
	\tikzset{noratorW/.style={voosource,
	bipoles/oosource/circlesize=0.5,
	bipoles/oosource/circleoffset=0.5},
	nullatorW/.style={esource, sources/scale=0.8}}
	\begin{tikzpicture}
	\draw (0,0) to[nullatorW] ++(2,0) to[nullatorW] ++(2, 0)--(4,-2)to[battery1](0,-2)--(0,0);
	\draw (2,-2.5) node[below] {\textit{a)}};
	\end{tikzpicture}
	\quad
	\begin{tikzpicture}
	\draw (0,0) to[battery1](4,0)--(4,-2)to[nullatorW](0,-2)--(0,0);
	\draw (4,-1) to[nullatorW](0,-1);
	\draw (2,-2.5) node[below] {\textit{b)}};
	\end{tikzpicture}
	\end{center}
	\loigiai{
	Gọi
	\begin{itemize}
	\item	$A$: \lq\lq  Linh kiện thứ nhất không hỏng\rq\rq .
	\item 	$B$: \lq\lq  Linh kiện thứ hai không hỏng\rq\rq.
	\end{itemize}
	\begin{enumerate}
	\item \textbf{Hai linh kiện mắc nối tiếp.}\\
	Xác suất để cả hai linh kiện đều không hỏng là:
	$$\mathrm{P}(A \cap B)=\mathrm{P}(A)\cdot \mathrm{P}(B \mid A)$$
	Ta có\\
	$\mathrm{P}(A)=1-P(\overline{A})=1-0{,}01=0{,}99$.\\
	$\mathrm{P}(B \mid A)=1-P(\overline{B}\mid A)=1-0{,}02=0{,}98$.\\
	$\Rightarrow \mathrm{P}(A \cap B)=\mathrm{P}(A)\cdot \mathrm{P}(B \mid A)=0{,}99\cdot0{,}98=0{,}9702$.
	\item \textbf{Hai linh kiện mắc song song.}\\
	Xác suất để ít nhất một linh kiện không hỏng là:
	$$\mathrm{P}	(A \cup B)=\mathrm{P}(A)+\mathrm{P}(B)-\mathrm{P}(A \cap B) = 0{,}99+0{,}98-0{,}9702=0{,}9998.$$
	\end{enumerate}
	}
\end{bt}
%%==========Bài 6
\begin{bt}%[2D5V1-3]
	Mỗi bạn học sinh trong lớp của Minh lựa chọn một trong hai ngoại ngữ là tiếng Anh hoặc tiếng Nhật. Xác suất chọn tiếng Anh của mỗi bạn học sinh nữ là $0{.}6$ và của mỗi bạn học sinh nam là $0{.}7$. Lớp của Minh có $25$ bạn nữ và $20$ bạn nam. Chọn ra ngẫu nhiên một bạn trong lớp.
	Sử dụng sơ đồ hình cây, tính xác suất của các biến cố 
	A:\lq\lq  Bạn được chọn là nam và học tiếng Nhật\rq\rq.\\
	B:\lq\lq  Bạn được chọn là nữ và học tiếng Anh\rq\rq .
	\loigiai{ 
		Gọi $A$ là biến cố \lq\lq  Bạn được chọn là nữ\rq\rq và $B$ là biến cố: \lq\lq  Bạn được chọn học tiếng Anh\rq\rq. \\
		Ta có
		$
		P(Y|X)=0{.}6; P(Y \mid \bar{X})=0{.}7 ; P(X)=\dfrac{5}{9}
		$.\\
		Do đó $P(\bar{X})=1-P(X)=\dfrac{4}{9} ; P(\bar{Y} |X)=1-P(Y|X)=0{.}4; \\P(\bar{Y} \mid \bar{X})=1-P(Y \mid \bar{X})=0{.}3$.\\
		Ta có sơ đồ hình cây như sau
		\begin{center}
			\begin{tikzpicture}[yscale=0.9]
				\def\gocm{20}
				\def\gocn{10}
				\def\r{4}
				\tikzset{s/.style={outer sep=0.5 mm,draw=magenta,rectangle,minimum width=2.75cm,rounded corners=1mm}}
				\path(0,0)node(O){}++(\gocm:\r)node[s](A1){X}++(\gocn:\r)node[s](A2){$Y$};
				\path(A1)++({-\gocn}:\r)node[s](a2){$\bar{Y}$};
				\path(O)++(-\gocm:\r)node[s](B1){$\bar{X}$}++(\gocn:\r)node[s](B2){$Y$};
				\path(B1)++({-\gocn}:\r)node[s](b2){$\bar{Y}$};
				\foreach \x/\y in {
					O/A1,A1/A2,
					O/B1,B1/B2,
					A1/a2,
					B1/b2}
				\draw[-stealth](\x.east)--(\y.west);
				\path(O)--(A1.west)node[pos=0.5,above]{$\dfrac{5}{9}$}(O)--(B1.west)node[pos=0.5,below]{$\dfrac{4}{9}$}(B1.east)--(B2.west)node[pos=0.5,above,sloped]{$0{.}7$}(A1.east)--(A2.west)node[pos=0.5,above,sloped]{$0{.}6$}
				(A1.east)--(a2.west)node[pos=0.5,below,sloped]{$0{.}4$}
				(B1.east)--(b2.west)node[pos=0.5,below,sloped]{$0{.}3$};
			\end{tikzpicture}
		\end{center}
		Do $A=\overline{X}\cdot\overline{Y}$ nên $P(\overline{X}\cdot\overline{Y})=\dfrac{4}{9}\cdot 0{.}3=\dfrac{2}{15}$.\\
		Do $B=XY$ nên $P(XY)=\dfrac{5}{9}\cdot 0{.}6=\dfrac{1}{3}$.\\
	}
\end{bt}
%%==========Bài 10
\begin{bt}
	Bạn An phải thực hiện hai thí nghiệm liên tiếp. Thí nghiệm thứ nhất có xác suất thành công là 0,7. Nếu thí nghiệm thứ nhất thành công thì xác suất thành công của thí nghiệm thứ hai là 0,9. Nếu thí nghiệm thứ nhất không thành công thì xác suất thành công của thí nghiệm thứ hai chỉ là 0,4. Tính xác suất để:
	\begin{enumerate}[a)]
		\item Cả hai thí nghiệm đều thành công;
		\item Cả hai thí nghiệm đều không thành công;
		\item Thí nghiệm thứ nhất thành công và thí nghiệm thứ hai không thành công. 
	\end{enumerate}
	\loigiai{
		Ta có sơ đồ hình cây
		\begin{center}
			\begin{tikzpicture}[line join = round,line cap = round, >=stealth, thick, font = \small, yscale = 0.85]
				%	\draw[gray!50,xstep = 1, ystep = 1] (-5,-5) grid (5,5);
				\path
				(0,0) coordinate (0) node[above=3mm] {TN}
				+(-2,-3) coordinate (11) node[left=3mm] {TB1}
				+(2,-3) coordinate (12) node[right=3mm] {TC1}
				(11)+(-1,-3) coordinate (21) node[below=3mm] {TB2} node[below=9mm] {TB1-TB2}
				+(1,-3) coordinate (22) node[below=3mm] {TC2} node[below=9mm] {TB1-TC2}
				(12)+(-1,-3) coordinate (23) node[below=3mm] {TB2} node[below=9mm] {TC1-TB2}
				+(1,-3) coordinate (24) node[below=3mm] {TC2} node[below=9mm] {TC1-TC2}
				(0)--(11) node[midway,left=3mm] {$\dfrac{3}{10}$}
				(0)--(12) node[midway,right=3mm] {$\dfrac{7}{10}$}
				(11)--(21) node[midway,left=2mm] {$\dfrac{6}{10}$}
				(11)--(22) node[midway,right=2mm] {$\dfrac{4}{10}$}
				(12)--(23) node[midway,left=2mm] {$\dfrac{1}{10}$}
				(12)--(24) node[midway,right=2mm] {$\dfrac{9}{10}$}
				;
				\draw (11)--(0)--(12)
				(21)--(11)--(22)
				(23)--(12)--(24)
				;
				\node[circle, line width = .2 mm, draw = black, text = black, fill = yellow!60, anchor = center, outer sep = 0pt, minimum size = .2cm] (c) at (0) {};
				\node[circle, line width = .2 mm, draw = black, text = black, fill = black, anchor = center, outer sep = 0pt, minimum size = .2cm] (c) at (11) {};
				\node[circle, line width = .2 mm, draw = black, text = black, fill = cyan, anchor = center, outer sep = 0pt, minimum size = .2cm] (c) at (12) {};
				\node[circle, line width = .2 mm, draw = black, text = black, fill = black, anchor = center, outer sep = 0pt, minimum size = .2cm] (c) at (21) {};
				\node[circle, line width = .2 mm, draw = black, text = black, fill = cyan, anchor = center, outer sep = 0pt, minimum size = .2cm] (c) at (22) {};
				\node[circle, line width = .2 mm, draw = black, text = black, fill = black, anchor = center, outer sep = 0pt, minimum size = .2cm] (c) at (23) {};
				\node[circle, line width = .2 mm, draw = black, text = black, fill = cyan, anchor = center, outer sep = 0pt, minimum size = .2cm] (c) at (24) {};
			\end{tikzpicture}
		\end{center}
		\begin{enumerate}[a)]
			\item Xác suất cả hai thí nghiệm đều thành công là $$\dfrac{7}{10}\cdot \dfrac{9}{10}=\dfrac{63}{100}.$$
			\item Xác suất cả hai thí nghiệm đều không thành công là $$\dfrac{3}{10}\cdot \dfrac{6}{10}=\dfrac{18}{100}.$$
			\item Xác suất thí nghiệm thứ nhất thành công và thí nghiệm thứ hai không thành công là $$\dfrac{7}{10}\cdot \dfrac{1}{10}=\dfrac{7}{100}.$$
		\end{enumerate}
	}
\end{bt}
%%==========Bài 11
\begin{bt}
	Trong một túi có một số chiếc kẹo cùng loại, chỉ khác màu, trong đó có 6 cái kẹo màu cam, còn lại là kẹo màu vàng. Hà lấy ngẫu nhiên một cái kẹo từ trong túi, không trả lại.
	Sau đó Hà lại lấy ngẫu nhiên thêm một cái kẹo khác từ trong túi. Biết rằng xác suất Hà lấy được cả hai cái kẹo màu cam là $\dfrac{1}{3}$. Hỏi ban đầu trong túi có bao nhiêu cái kẹo?
	\loigiai{
		Gọi số kẹo là $x~(x>6)$.\\
		Số kẹo màu vàng là $x-6$.\\
		Khi Hà lấy được chiếc kẹo màu cam thì số kẹo trong túi là $x-1$ và số kẹo cam còn lại trong túi là 5 cái.\\
		Ta có sơ đồ cây
		\begin{center}
			\begin{tikzpicture}[line join = round,line cap = round, >=stealth, thick, font = \small, yscale = 0.85]
				%	\draw[gray!50,xstep = 1, ystep = 1] (-5,-5) grid (5,5);
				\path
				(0,0) coordinate (0) node[above=3mm] {Túi kẹo}
				+(-2,-3) coordinate (11) node[left=3mm] {C}
				+(2,-3) coordinate (12) node[right=3mm] {V}
				(11)+(-1,-3) coordinate (21) node[below=3mm] {C} node[below=9mm] {CC}
				+(1,-3) coordinate (22) node[below=3mm] {V} node[below=9mm] {CV}
				(12)+(-1,-3) coordinate (23) node[below=3mm] {C} node[below=9mm] {VC}
				+(1,-3) coordinate (24) node[below=3mm] {V} node[below=9mm] {VV}
				(0)--(11) node[midway,left=3mm] {$\dfrac{6}{x}$}
				(0)--(12) node[midway,right=3mm] {$\dfrac{x-6}{x}$}
				(11)--(21) node[midway,left=2mm] {$\dfrac{5}{x-1}$}
				(11)--(22) node[midway,right=2mm] {$\dfrac{x-6}{x-1}$}
				(12)--(23) node[midway,left=2mm] {$\dfrac{6}{x-1}$}
				(12)--(24) node[midway,right=2mm] {$\dfrac{x-7}{x-1}$}
				;
				\draw (11)--(0)--(12)
				(21)--(11)--(22)
				(23)--(12)--(24)
				;
				\node[circle, line width = .2 mm, draw = black, text = black, fill = yellow!50!orange, anchor = center, outer sep = 0pt, minimum size = .2cm] (c) at (0) {};
				\node[circle, line width = .2 mm, draw = black, text = black, fill=orange, anchor = center, outer sep = 0pt, minimum size = .2cm] (c) at (11) {};
				\node[circle, line width = .2 mm, draw = black, text = black, fill = yellow, anchor = center, outer sep = 0pt, minimum size = .2cm] (c) at (12) {};
				\node[circle, line width = .2 mm, draw = black, text = black, fill = orange, anchor = center, outer sep = 0pt, minimum size = .2cm] (c) at (21) {};
				\node[circle, line width = .2 mm, draw = black, text = black, fill = yellow, anchor = center, outer sep = 0pt, minimum size = .2cm] (c) at (22) {};
				\node[circle, line width = .2 mm, draw = black, text = black, fill = orange, anchor = center, outer sep = 0pt, minimum size = .2cm] (c) at (23) {};
				\node[circle, line width = .2 mm, draw = black, text = black, fill = yellow, anchor = center, outer sep = 0pt, minimum size = .2cm] (c) at (24) {};
			\end{tikzpicture}
		\end{center}
		Xác suất để Hà lấy được cả hai cái kẹo màu cam là
		$$
		\dfrac{6}{x}\cdot \dfrac{5}{x-1}=\dfrac{1}{3} \Rightarrow x^2-x-90=0 \Rightarrow \hoac{&x=-9~\text{(loại)}\\&x=10~\text{thỏa mãn}.}
		$$
		Vậy ban đầu trong túi có 10 cái kẹo.
	}
\end{bt}











%%==========Bài 15
\begin{bt}
	Trên giá sách có $10$ quyển sách Khoa học và $15$ quyển sách Nghệ thuật. Có $9$ quyển sách viết bằng tiếng Anh, trong đó $3$ quyển sách Khoa học có $6$ quyển sách Nghệ thuật, các quyển sách còn lại viết bằng tiếng Việt. Lấy ngẫu nhiên một quyển sách. Dùng sơ đồ hình cây, tính xác suất để quyển sách được lấy ra là sách viết bằng tiếng Việt, biết rằng quyển sách đó là sách Khoa học.
	\loigiai{
		Gọi các biến cố:\\
		$A$: \lq\lq  Sách lấy ra là sách tiếng Việt\rq\rq.\\
		$B$: \lq\lq  Sách lấy ra là sách khoa học\rq\rq.\\
		Khi đó, xác suất để cuốn sách được lấy ra là tiếng Việt, biết rằng cuốn sách đó là sách khoa học là $\mathrm{P}\left(A\mid B\right)$.
		Ta có sơ đồ cây
		\begin{center}
			\begin{tikzpicture}[scale=.2,>=stealth,every node/.style={shape=rectangle,draw,rounded corners, color=blue, fill=blue!10}]
			%-------------
			\tikzstyle{block} = [rectangle, draw, fill=blue!10, rounded corners, text centered, text width = 10em, minimum height = 2em]
			%-------------
			\node (c1) {Cuốn sách được lấy};
			\node (c2) [block, above right = 4cm of c1]{$B\colon$ \,\lq\lq  Sách lấy ra là sách khoa học \rq\rq};
			\node at (7,7.5){$\mathrm{P}(B)=2/5$};
			\node at (7,-7.5){$\mathrm{P}(\overline{B})=3/5$};
			\node (c3) [block, below right= 4cm of c1]{$\overline{B}\colon$\,\lq\lq  Sách lấy ra là sách nghệ thuật \rq\rq};
			\node at (37.8,25){$\mathrm{P}(A\mid B)=7/10$};
			\node (c4) [above right = 2cm of c2]{$A\colon $\,\lq\lq  Sách lấy ra là sách tiếng Việt \rq\rq };
			\node (c5) [below right = 2cm of c2]{$\overline{A}\colon $\,\lq\lq  Sách lấy ra là sách tiếng Anh \rq\rq};
			\node at (39,9.5){$\mathrm{P}(\overline{A}\mid B)=3/10$};
			\node (c6) [block, above right =2cm of c3]{$A\colon $\,\lq\lq  Sách lấy ra là sách tiếng Việt \rq\rq };
			\node at (39,-8.5){$\mathrm{P}(A\mid \overline{B})=3/5$};
			\node (c7) [block, below right = 2cm of c3]{$\overline{A}\colon $\,\lq\lq  Sách lấy ra là sách tiếng Anh \rq\rq};
			\node at (39,-27.8){$\mathrm{P}(\overline{A}\mid \overline{B})=2/5$};
			%--------------
			\draw[->] (c1.east) -- (c2.west);
			\draw[->] (c1.east) -- (c3.west);
			\draw[->] (c2.east) -- (c4.west);
			\draw[->] (c2.east) -- (c5.west);
			\draw[->] (c3.east) -- (c6.west);
			\draw[->] (c3.east) -- (c7.west);
		\end{tikzpicture}
		\end{center}
		Từ đó ta có xác suất để cuốn sách lấy ra là tiếng Việt, biết rằng cuốn sách đó là sách khoa học là $\mathrm{P}(A\mid B)=\dfrac{7}{10}=0{,}7$.
	}
\end{bt}
% %%%%%%%%%%%%%%%%%
% \newpage
\subsection{Bài tập trắc nghiệm}%\BTTN
%
\Opensolutionfile{ans}[ans/ans-2-B18]
%\TN
\begin{ex}%[2D5N1-2]
	Cho hai biến cố $A$, $B$ có xác suất $\mathrm{P}(A)=0{,}4$, $\mathrm{P}(B)=0{,}6$, $\mathrm{P}(AB)=0{,}2$. Tính xác suất $\mathrm{P}(A|B)$.
	\choice
	{\True$\dfrac{1}{3}$}
	{$\dfrac{1}{2}$}
	{$0{,}3$}
	{$0{,}25$}
	\loigiai{
	Theo định nghĩa xác suất có điều kiện, ta có
	$$\mathrm{P}(A \mid B)=\dfrac{\mathrm{P}(A B)}{\mathrm{P}(B)}=\dfrac{0{,}2}{0{,}6}=\dfrac{1}{3}.$$	}
\end{ex}
\begin{ex}%[2D5N1-2]
	Cho hai biến cố $A$, $B$ có xác suất $\mathrm{P}(A)=0{,}4$, $\mathrm{P}(B)=0{,}3$, $\mathrm{P}(A\mid B)=0{,}25$. Tính xác suất $\mathrm{P}(B\mid A)$.
	\choice
	{\True $0{,}1875$}
	{$0{,}48$}
	{$0{,}333$}
	{$0{,}95$}
	\loigiai{
	Ta có
	\allowdisplaybreaks
	\begin{eqnarray*}
	\mathrm{P}(A \mid B)&=&\dfrac{\mathrm{P}(A B)}{\mathrm{P}(B)}\\
	\Rightarrow \mathrm{P}(AB)&=&\mathrm{P}(B)\cdot 	\mathrm{P}(A \mid B)=0{,}3\cdot 0{,}25=0{,}075.
	\end{eqnarray*}
	Từ đó suy ra
	$$\mathrm{P}(B\mid A)=\dfrac{\mathrm{P}(AB)}{\mathrm{P}(A)}=\dfrac{0{,}075}{0{,}4}=0{,}1875.$$	
	}
\end{ex}
\begin{ex}%[2D5H1-2]
	Cho hai biến độc lập $A,B$ với $\mathrm{P}(A)=0{,}8$, $\mathrm{P}(B)=0{,}25$. Khi đó, $\mathrm{P}(B\mid A)$ bằng
	\choice
	{$0{,}2$}
	{$0{,}8$}
	{\True $0{,}25$}
	{$0{,}75$}
	\loigiai{
	Vì $A$ và $B$ là hai biến cố độc lập, do đó
	\[\mathrm{P}(B\mid A)=\mathrm{P}(B)=0{,}25.\]	
	}
\end{ex}	
\begin{ex}%[2D5H1-2]
	Cho một hộp kín có $6$ thẻ ATM của ACB và 4 thẻ ATM của Vietcombank. Lấy ngẫu nhiên lần lượt $2$ thẻ (lấy không hoàn lại). Tìm xác suất để lần thứ hai lấy được thẻ ATM của Vietcombank nếu biết lần thứ nhất đã lấy được thẻ ATM của ACB.
	\choice
	{ $\dfrac{1}{3}$}
	{$\dfrac{2}{3}$}
	{$\dfrac{2}{9}$}
	{\True$\dfrac{4}{9}$}
	\loigiai{
	Gọi $A$ là biến cố \lq\lq  lần thứ hai lấy được thẻ ATM Vietcombank\rq\rq, $B$ là biến cố \lq\lq  lần thứ nhất lấy được thẻ ATM của ACB\rq\rq. Ta cần tìm $\mathrm{P}(A\mid B)$.\\	
	Sau khi lấy lần thứ nhất (biến cố $B$ đã xảy ra) trong hộp còn lại $9$ thẻ (trong đó $4$ thẻ Vietcombank) nên $\mathrm{P}(A\mid B)=\dfrac{4}{9}$.}
\end{ex}
\begin{ex}%[2D5H1-2]
	Một nhóm $50$ học sinh có $23$ bạn biết chơi cầu lông mà không biết chơi bóng đá và $21$ bạn biết chơi bóng đá mà không biết chơi cầu lông. Biết rằng mỗi học sinh trong nhóm này biết chơi bóng đá hoặc cầu lông. Chọn ngẫu nhiên một học sinh trong nhóm. Tính xác suất học sinh này biết chơi bóng đá, biết rằng bạn ấy biết chơi cầu lông. 
	\choice
	{$\dfrac{23}{29}$}
	{\True$\dfrac{6}{29}$}
	{$\dfrac{21}{29}$}
	{$\dfrac{6}{23}$}
	\loigiai{
	Gọi $A$ là biến cố \lq\lq  Học sinh được chọn biết chơi bóng đá\rq\rq; $B$ là biến cố \lq\lq  Học sinh được chọn biết chơi cầu lông\rq\rq.\\
	Ta có $n\left(A\cap B\right)=50-(23+21)=6$ và $n(B)=23+6=29$. Do đó 
	$$\mathrm{P}(A\mid B)=\dfrac{P\left(A B\right)}{\mathrm{P}(B)}=\dfrac{6}{29}.$$
	}
\end{ex}
\begin{ex}%[2D5V1-2]
	Một bình đựng $3$ bi xanh và $2$ bi trắng. Lấy ngẫu nhiên lần $1$ một viên bi (không bỏ vào lại), rồi lần $2$ một viên bi. Tính xác suất để lần $1$ lấy một viên bi xanh, lần $2$ lấy một viên bi trắng.
	\choice
	{$\dfrac{1}{5}$}
	{$\dfrac{1}{10}$}
	{$\dfrac{1}{3}$}
	{\True $\dfrac{3}{10}$}
	\loigiai{
	Gọi $A$ là biến cố lấy một bi xanh lần thứ nhất thì $\mathrm{P}(A)=\dfrac{3}{5}$.\\
	Gọi $B$ là biến cố lấy một bi trắng lần thứ hai.
	Gọi $C$ là biến cố lấy lần $1$ lấy một viên bi xanh, lần $2$ lấy một viên bi trắng.
	Nếu $A$ đã xảy ra thì trong bình chi còn $2$ bi xanh, $2$ bi trằng. Khi đó $\mathrm{P}(B\mid A)=\dfrac{2}{4}$.\\
	Mà $C=AB$, do đó theo công thức nhân ta có:
	$$\mathrm{P}(C)=\mathrm{P}(AB)=\mathrm{P}(A)\cdot \mathrm{P}(B\mid A)=\dfrac{3}{5} \cdot \dfrac{1}{2}=\dfrac{3}{10}.
	$$}
\end{ex}	
\begin{ex}%[2D5H1-2]
	Một công ty đấu thầu $2$ dự án. Khả năng thắng thầu của các dự án $I$ và $II$ lần lượt là $0{,}4$ và $0{,}5$. Khả năng thắng thầu của hai dự án là $0{,}3$. Gọi $A$, $B$ lần lượt là biến cố thắng thầu dự án $I$ và dự án $II$. Biết công ty thắng thầu dự án $I$, tìm xác suất công ty thắng thầu dự án $II$.
	\choice
	{$0{,}25$}
	{$0{,}5$}
	{\True$0{,}75$}
	{$0{,}125$}
	\loigiai{
	Gọi $C$ là biến cố công ty thắng dự $II$ biết công ty thắng dự án $I$. Ta có
	$$\mathrm{P}(C)=\mathrm{P}(B|A)=\dfrac{\mathrm{P}(AB)}{\mathrm{P}(A)}=\dfrac{0{,}3}{0{,}4}=0{,}75.$$
	}
\end{ex}
\begin{ex}%[2D5H1-2]
	Một sinh viên làm $2$ bài tập kế tiếp. Xác suất làm đúng bài thứ nhất là $0{,}7$. Nếu làm đúng bài thứ nhất thì khả năng làm đúng bài thứ $2$ là $0{,}8$, nhưng nếu làm sai bài thứ $1$ thì khả năng làm đúng bài thứ $2$ là $0{,}2$. Tính xác suất sinh viên làm ít nhất một bài.
	\choice
	{$ 0{,}903$}
	{$0{,}737$}
	{\True $0{,}76$}
	{$0{,}62$}
	\loigiai{
	Gọi $A$, $B$ lần lượt là biến cố sinh viên làm đúng bài $1$, bài $2$.\\ Theo đề bài ta có $\mathrm{P}(A)=0{,}7$; $\mathrm{P}(B\mid A)=0{,}8$	và $\mathrm{P}(B\mid \overline{A})=0{,}2$.\\
	Suy ra 
	$$ \mathrm{P}(\overline{B}\mid \overline{A})=1-\mathrm{P}(B\mid \overline{A})=1-0{,}2=0{,}8.$$
	Ta có
	\allowdisplaybreaks
	\begin{eqnarray*}
	\mathrm{P}(A\cup B)&=&1-\mathrm{P}(\overline{A\cup B})\\
	&=&1-\mathrm{P}(\overline{A}\cdot \overline{B})\\
	&=&1- \mathrm{P}(\overline{A})\cdot \mathrm{P}(\overline{B}\mid \overline{A})\\
	&=& 1- 0{,}3 \cdot 0{,}8=0{,}76.
	\end{eqnarray*}
	}
\end{ex}
\begin{ex}%[2D5H1-2]
	Một thư viện có $35\%$ tổng số sách là sách khoa học, $14\%$ tổng số sách là sách khoa học tự nhiên. Chọn ngẫu nhiên một quyển sách của thư viện. Tính xác suất để quyển sách được chọn là sách khoa học tự nhiên, biết rằng đó là quyển sách về khoa học.
	\choice
	{$0{,}2$}
	{\True $0{,}4$}
	{$0,{4}$}
	{$0{,}8$}
	\loigiai
	{Gọi $A$ là biến cố \lq\lq  Sách được chọn là sách khoa học tự nhiên\rq\rq.\\
	Gọi $B$ là biến cố \lq\lq  Sách được chọn là sách khoa học\rq\rq.\\
	Do có $35\%$ tổng số sách là sách khoa học nên $\mathrm{P}(B)=0{,}35$.\\
	Do có $14\%$ tổng số sách là sách khoa học tự nhiên nên $\mathrm{P}(AB)=0{,}14$.\\
	Vậy $\mathrm{P}(A\mid B)=\dfrac{\mathrm{P}(AB)}{\mathrm{P}(B)}=\dfrac{0{,}14}{0{,}35}=0{,}4$.
	}
\end{ex}
\begin{ex}%[2D5V1-2]
	Trong một kì thi, thí sinh được phép thi $3$ lần. Xác suất lần đầu vượt qua kì thi là $0{,}9$. Nếu trượt lần đầu thì xác suất vượt qua kì thi lần hai là $0{,}7$. Nếu trượt cả hai lần thì xác suất vượt qua kì thi ở lần thứ ba là $0{,}3$. Tính xác suất để thí sinh thi đậu.
	\choice
	{\True $0{,}979$}
	{$ 0{,97}$ }
	{$ 0{,79}$ }
	{$ 0{,797}$ }
	\loigiai{
	Gọi $A$, $B$, $C$ lần lượt là biến cố thí sinh thi đậu lần thứ $1$, lần thứ $2$, lần thứ $3$.
	Gọi $D$ là biến cố để thí sinh thi đậu. Ta có
	\allowdisplaybreaks
	\begin{eqnarray*}
	D&=&A\cup (\overline{A}B)\cup (\overline{A}\,\overline{B}C)\\
	\Rightarrow D&=&\mathrm{P}(A)+P (\overline{A}B)+P (\overline{A}\,\overline{B}C).
	\end{eqnarray*}
	Trong đó ta có
	\begin{itemize}
	\item $\mathrm{P}(A)=0{,}9$.
	\item $P (\overline{A}B)=\mathrm{P}(\overline{A})\cdot P (B\mid \overline{A})=0{,}1\cdot 0{,}7=0{,}07$.
	\item $P (\overline{A}\,\overline{B}C)=\mathrm{P}(\overline{A}\,\overline{B})\cdot \mathrm{P}(C\mid \overline{A}\,\overline{B})=\mathrm{P}(\overline{A})\cdot P (\overline{B}\mid \overline{A}) \cdot \mathrm{P}(C\mid \overline{A}\,\overline{B})=0{,}1\cdot (1-0{,}7)\cdot 0{,}3=0{,}009$.
	\end{itemize}
	Vậy $\mathrm{P}(D)=0{,}9+0{,}07+0{,}009=0{,}979$.
	}
\end{ex}
\begin{ex}%[2D5V1-3]
	Trong một hộp kín có $7$ chiếc bút bi xanh và $5$ chiếc bút bi đen, các chiếc bút có cùng kích thước và khối lượng. Bạn Sơn lấy ngẫu nhiên một chiếc bút bi từ trong hộp, không trả lại. Sau đó bạn Tùng lấy ngẫu nhiên một trong $11$ chiếc bút còn lại. Tính xác suất để Sơn lấy được bút bi đen và Tùng lấy được bút bi xanh.
	\choice
	{$\dfrac{139}{132}$}
	{\True$\dfrac{35}{132}$}
	{$\dfrac{25}{132}$}
	{$\dfrac{49}{132}$}
	\loigiai{
	Gọi $A$ là biến cố: \lq\lq  Bạn Sơn lấy được bút bi đen\rq\rq ;\\
	$B$ là biến cố: ``Bạn Tùng lấy được bút bi xanh''.\\
	Ta cần tính $\mathrm{P}(AB)$.\\
	Vì $n(A)=5$ nên $\mathrm{P}(A)=\dfrac{5}{12}$.\\
	Nếu $A$ xảy ra tức là bạn Sơn lấy được bút bi đen thì trong hộp có 11 bút bi với 7 bút bi xanh.\\
	Vậy $\mathrm{P}(B\mid A)=\dfrac{7}{11}$.\\
	Theo công thức nhân xác suất: $\mathrm{P}(AB)=\mathrm{P}(A)\cdot \mathrm{P}(B\mid A)=\dfrac{5}{12}\cdot\dfrac{7}{11}=\dfrac{35}{132}$.\\
	Một phương pháp mô tả trực quan lời giải trên là dùng sơ đồ hình cây
	\begin{center}
	\begin{tikzpicture}[line join = round,line cap = round, >=stealth, thick, font = \small, scale = 1,yscale=0.85]
	%	\draw[gray!50,xstep = 1, ystep = 1] (-5,-5) grid (5,5);
	\path
	(0,0) coordinate (0) node[above=3mm] {$O$}
	+(-2,-3) coordinate (11) node[left=3mm] {Đ}
	+(2,-3) coordinate (12) node[right=3mm] {X}
	(11)+(-1,-3) coordinate (21) node[below=3mm] {Đ} node[below=9mm] {ĐĐ}
	+(1,-3) coordinate (22) node[below=3mm] {X} node[below=9mm] {ĐX}
	(12)+(-1,-3) coordinate (23) node[below=3mm] {Đ} node[below=9mm] {XĐ}
	+(1,-3) coordinate (24) node[below=3mm] {X} node[below=9mm] {XX}
	(0)--(11) node[midway,left=3mm] {$\dfrac{5}{12}$}
	(0)--(12) node[midway,right=3mm] {$\dfrac{7}{12}$}
	(11)--(21) node[midway,left=2mm] {$\dfrac{4}{11}$}
	(11)--(22) node[midway,right=2mm] {$\dfrac{7}{11}$}
	(12)--(23) node[midway,left=2mm] {$\dfrac{5}{11}$}
	(12)--(24) node[midway,right=2mm] {$\dfrac{6}{11}$}
	;
	\draw (11)--(0)--(12)
	(21)--(11)--(22)
	(23)--(12)--(24)
	;
	\node[circle, line width = .2 mm, draw = black, text = black, fill = yellow!60, anchor = center, outer sep = 0pt, minimum size = .2cm] (c) at (0) {};
	\node[circle, line width = .2 mm, draw = black, text = black, fill = black, anchor = center, outer sep = 0pt, minimum size = .2cm] (c) at (11) {};
	\node[circle, line width = .2 mm, draw = black, text = black, fill = cyan, anchor = center, outer sep = 0pt, minimum size = .2cm] (c) at (12) {};
	\node[circle, line width = .2 mm, draw = black, text = black, fill = black, anchor = center, outer sep = 0pt, minimum size = .2cm] (c) at (21) {};
	\node[circle, line width = .2 mm, draw = black, text = black, fill = cyan, anchor = center, outer sep = 0pt, minimum size = .2cm] (c) at (22) {};
	\node[circle, line width = .2 mm, draw = black, text = black, fill = black, anchor = center, outer sep = 0pt, minimum size = .2cm] (c) at (23) {};
	\node[circle, line width = .2 mm, draw = black, text = black, fill = cyan, anchor = center, outer sep = 0pt, minimum size = .2cm] (c) at (24) {};
	\end{tikzpicture}
	\end{center}
	Trên nhánh OĐ và OX tương ứng ghi xác suất lấy được bút đen và bút xanh.\\
	Trên nhánh ĐĐ, ĐX tương ứng ghi xác suất lấy được bút đen, bút xanh với điều kiện đã lấy được bút đen.\\
	Trên nhánh XĐ, XX tương ứng ghi xác suất lấy được bút đen, bút xanh với điều kiện đã lấy được bút xanh.\\
	Vậy xác suất cần tính là $\dfrac{5}{12}\cdot\dfrac{7}{11}=\dfrac{35}{132}$.
	}
\end{ex}
\begin{ex}%[2D5H1-2]
	Một hộp kín đựng $20$ tấm thẻ giống hệt nhau đánh số từ $1$ đến $20$. Một người rút ngẫu nhiên ra một tấm thẻ từ trong hộp. Người đó được thông báo rằng thẻ rút ra mang số chẵn. Tính xác suất để người đó rút được thẻ số $10$.
	\choice
	{$0{,}4$}
	{$0{,}3$}
	{$0{,}2$}
	{\True$0{,}1$}
	\loigiai
	{	Gọi $A$ là biến cố: \lq\lq  Người đó rút được thẻ số 10\rq\rq.\\
	Gọi $B$ là biến cố: \lq\lq  Người đó rút được thẻ mang số chẵn\rq\rq .\\
	Không gian mẫu mà 20 tấm thẻ đánh số từ 1 đến 20 $\Rightarrow$ $n(\Omega)=20$.\\
	Ta cần tính $P\left(A\mid B\right)$.\\
	Ta có, từ 1 đến 20 có 10 số chẵn nên $n(B)=10$.\\
	Vậy $\mathrm{P}(B)=\dfrac{n(B)}{n(\Omega)}=\dfrac{10}{20}=\dfrac{1}{2}$.\\
	Trong số 10 số chẵn có một số 10 nên $n(AB)=1$.\\
	Vậy $\mathrm{P}(AB)=\dfrac{n(AB)}{n(\Omega)}=\dfrac{1}{20}$.\\
	Do đó $\mathrm{P}(A\mid B)=\dfrac{\mathrm{P}(AB)}{\mathrm{P}(B)}=\dfrac{1}{10}= 0{,}1$.
	}
\end{ex}
\begin{ex}%[2D6N1-2]
	Cho hai biến cố $A$, $B$ xung khắc với nhau thỏa $\mathrm{P}(A)=0{,}2$; $\mathrm{P}(B)=0{,}4$. Khi đó $\mathrm{P}\left(A|B\right)$ bằng
	\choice
	{$0{,}5$}
	{$0{,}2$}
	{$0{,}4$}
	{\True $0$}
	\loigiai{
	Do $A$, $B$ xung khắc nên $\mathrm{P}\left(A\cap B\right)=0$. \\
	Vậy $\mathrm{P}\left(A|B\right)=\dfrac{\mathrm{P}\left(A\cap B\right)}{\mathrm{P}(B)}=0$.
	}
\end{ex}	
\begin{ex}%[2D6H1-2] 
	Trong một kì thi học sinh giỏi môn Toán của một tỉnh có $200$ học sinh tham gia, trong đó có $95$ học sinh nữ và $105$ học sinh nam. Kết quả của kì thi cho biết có $50$ học sinh đạt giải (bao gồm nhất, nhì và ba), trong đó có $24$ học sinh nữ và $26$ học sinh nam. Chọn ngẫu nhiên một học sinh trong số $200$ học sinh đó. Tính xác suất để học sinh được chọn có giải, biết rằng học sinh đó là nam.
	\choice
	{$\dfrac{8}{35}$}
	{$\dfrac{24}{95}$}
	{$\dfrac{26}{95}$}
	{\True $\dfrac{26}{105}$}
	\loigiai{
	Xét các biến cố:\\
	$A$: "Học sinh được chọn đạt giải."\\
	$B$: "Học sinh được chọn là nam."\\
	Ta có: $\mathrm{P}(A\cap B)=\dfrac{26}{200}=0{,}13$ và $\mathrm{P}(B)=\dfrac{105}{200}=0{,}525$.\\
	Xác xuất để học sinh được chọn đạt giải, biết rằng học sinh đó là nam, là
	$$\mathrm{P}(A|B)=\dfrac{\mathrm{P}(A\cap B)}{\mathrm{P}(B)}=\dfrac{0{,}13}{0{,}525}=\dfrac{26}{105}.$$
	}
\end{ex}
\begin{ex}%[2D6H1-2]
	Trong một lô hàng áo thun trơn gồm $1000$ chiếc của một công ty dệt may có $100$ áo thun có màu đỏ. Các áo thun đỏ đó gồm có các kích thước là $X$, $M$ và $L$, trong đó có $30$ áo kích thước $M$. Chọn ngẫu nhiên một chiếc áo trong lô hàng đó. Giả sử chiếc áo được chọn là màu đỏ, tính xác suất để chiếc áo đó có kích thước $M$.
	\choice
	{$0{,}03$}
	{\True $0{,}3$}
	{$0{,}01$}
	{$0{,}3$}
	\loigiai{
	Xét các biến cố:\\
	$A$: "Áo được chọ có kích thước $M$.";\\
	$B$: "Áo được chọn là màu đỏ."\\
	Khi đó, xác suất để chiếc áo được chọn có kích thước $M$, biết rằng áo thun đó có màu đỏ, là xác suất có điều kiện $\mathrm{P}(A|B)$, ta có:
	$$\mathrm{P}(A|B)=\dfrac{n(A\cap B)}{n(B)}=\dfrac{30}{100}=0,3$$
	}
\end{ex}
\begin{ex}%[2D6H1-2]
	Trong một hộp đựng $500$ chiếc thẻ cùng loại có $200$ chiếc thẻ màu vàng. Trên mỗi chiếc thẻ màu vàng có ghi một trong năm số: $1$, $2$, $3$, $4$, $5$. Có $40$ chiếc thẻ màu vàng ghi số $5$. Chọn ra ngẫu nhiên một chiếc thẻ trong hộp đựng thẻ. Giả sử chiếc thẻ chọn ra có màu vàng. Tính xác suất để chiếc thẻ đó ghi số $5$.
	\choice
	{\True $0{,}2$}
	{$\dfrac{2}{15}$}
	{$0{,}4$}
	{$\dfrac{4}{15}$}
	\loigiai{
	Gọi biến cố $A$: \lq\lq  Thẻ được chọn ghi số $5$\rq\rq. \\
	Gọi biến cố $B$: \lq\lq  Thẻ được chọn có màu vàng\rq\rq. \\
	Có $\mathrm{P}\left(A|B\right)=\dfrac{n\left(A\cap B\right)}{n(B)}=\dfrac{40}{200}=0{,}2$. 
	}
\end{ex}	
\begin{ex}%[2D6H1-2]
	Một hộp đựng $8$ viên bi màu đỏ và $5$ viên bi màu vàng (các viên bi có kích thước và khối lượng như nhau). Có $5$ viên bi trong hộp được đánh số, trong đó có $3$ viên bi màu đỏ và $2$ viên bi màu vàng. Lấy ngẫu nhiên một viên bi trong hộp. Tính xác suất để viên bi được lấy ra có màu đỏ, biết rằng viên bi đó được đánh số.
	\choice
	{$0{,}4$}
	{\True $0{,}6$}
	{$0{,}2$}
	{$0{,}3$}
	\loigiai{
	Gọi biến cố $A$: \lq\lq  Viên bi được lấy ra có màu đỏ \rq\rq. \\
	Gọi biến cố $B$: \lq\lq  Viên bi được lấy ra có đánh số \rq\rq. \\
	Có $\mathrm{P}\left(A|B\right)=\dfrac{n\left(A\cap B\right)}{n(B)}=\dfrac{3}{5}=0{,}6$. 
	}
\end{ex}	
\begin{ex}%[2D6H1-2]
	Gieo lần lượt hai con xúc xắc cân đối và đồng chất. Tính xác suất để tổng số chấm xuất hiện trên hai con xúc xắc không nhỏ hơn $6$, biết rằng xúc xắc thứ nhất xuất hiện mặt $4$ chấm.
	\choice
	{$\dfrac{5}{24}$}
	{$\dfrac{5}{12}$}
	{$\dfrac{5}{36}$}
	{\True $\dfrac{5}{6}$}
	\loigiai{
	Gọi biến cố $A$: \lq\lq  Tổng số chấm xuất hiện trên hai con xúc xắc bằng $6$ \rq\rq. \\
	Gọi biến cố $B$: \lq\lq  Xúc xắc thứ nhất xuất hiện mặt $4$ chấm\rq\rq. \\
	Có $\mathrm{P}\left(A|B\right)=\dfrac{n\left(A\cap B\right)}{n(B)}=\dfrac{5}{6}$. 
	}
\end{ex}
\begin{ex}%[2D6H2-2] 
	Cho hai biến cố $A$ và $B$ thỏa $\mathrm{P}(A)=0{,}4$; $\mathrm{P}(B)=0{,}8$; $\mathrm{P}\left(A| \overline{B}\right)=0{,}5$. Tính $\mathrm{P}(A|B)$.
	\choice
	{$0{,}4$}
	{\True $0{,}375$}
	{$0{,}5$}
	{$0{,}625$}
	\loigiai{
	Ta có
	\begin{eqnarray*}
	& & \mathrm{P}\left(A| \overline{B}\right) = \dfrac{\mathrm{P}(A \overline{B})}{\mathrm{P}(\overline{B})}\\
	&\Leftrightarrow & 0{,}5 = \dfrac{\mathrm{P}(A)-\mathrm{P}(AB)}{1-P(B)}\\
	&\Leftrightarrow & \mathrm{P}(AB)=\mathrm{P}(A) - 0{,}5\cdot[1-P(B)]\\
	&\Leftrightarrow & \mathrm{P}(AB)=0{,}3.
	\end{eqnarray*}
	Khi đó 
	$$\mathrm{P}\left(A|B\right)=\dfrac{\mathrm{P}\left(AB\right)}{\mathrm{P}(B)}=0{,}375.$$ }
\end{ex}
\begin{ex}%[2D6V1-4]
	Một công ty vừa ra mắt sản phẩm $X$ và tổ chức ngày trải nghiệm sản phẩm. Họ thống kê được trong $200$ người đến tham quan ngày trải nghiệm có $60$ người là nam giới và $140$ người là nữ giới. Trong số những người được thống kê này, có $120$ người mua sản phẩm $X$, gồm $40$ khách hàng nam và $80$ khách hàng nữ, còn lại là không mua sản phẩm $X$. Chọn ngẫu nhiên một người trong số $200$ người được thống kê. Tính xác suất để người này mua sản phẩm $X$, biết rằng người được chọn là nữ giới.
	\choice
	{$\dfrac{2}{3}$}
	{$\dfrac{7}{10}$}
	{$\dfrac{2}{5}$}
	{\True $\dfrac{4}{7}$}
	\loigiai{
	Xét các biến cố:\\
	$A$: "Người được chọn mua sản phẩm $X$.";\\
	$B$: "Người được chọn là nữ giới."\\
	Khi đó xác suất để người này mua sản phẩm $X$, biết rằng người được chọn là nữ giới là xác suất của $A$ với điều kiện $B$.\\
	Có $80$ người mua sản phẩm $X$ là nữ giới nên $\mathrm{P}(AB)=\dfrac{80}{200}=0{,}4$.\\
	Có $140$ người là nữ giới trong số lượng thống kê nên $\mathrm{P}(B)=\dfrac{140}{200}=0{,}7$.\\
	Vậy $\mathrm{P}(A|B)=\dfrac{\mathrm{P}(AB)}{\mathrm{P}(B)}=\dfrac{4}{7}$.
	}
\end{ex}
\begin{ex}%[2D6V2-4] 
	Theo kết quả từ trạm nghiên cứu khí hậu tại một địa phương, xác suất để một ngày có gió là $0{,}6$. Nếu ngày đó có gió thì xác suất có mưa là $0{,}4$. Tính xác suất để trời có gió nhưng không có mưa ở địa phương đó trong một ngày.
	\choice
	{$0{,}6$}
	{\True $0{,}36$}
	{$0{,}24$}
	{$0{,}16$}
	\loigiai{
	Xét các biến cố:\\
	$A$:"Ngày có gió" và $B$: "Ngày có mưa".\\
	Xác suất để trời có gió nhưng không có mưa ở địa phương đó trong một ngày là $P(A\overline{B})$.\\
	Theo đề bài, nếu ngày có gió thì xác suất có mưa là $0{,}4$ nên $\mathrm{P}(B|A)=0{,}4$.\\
	Suy ra $\mathrm{P}(\overline{B}|A)=1-0{,}4=0{,}6$.\\
	Ta có
	$$ \mathrm{P}(\overline{B}|A)=\dfrac{\mathrm{P}(A\overline{B})}{\mathrm{P}(A)} \Rightarrow \mathrm{P}(A\overline{B})=\mathrm{P}(A)\cdot \mathrm{P}(\overline{B}|A) = 0{,}6 \cdot 0{,}6 =0{,}36 $$}
\end{ex}
\Closesolutionfile{ans}
\indapan{6}{ans/ans-2-B18}
%\TNTF
\Opensolutionfile{ans}[ans/ans-2-B18-DS]
\begin{ex}%[2D5H1-2]
	Hai xạ thủ An và Bình bắn vào cùng một mục tiêu ở hai thời điểm khác nhau với xác suất bắn trúng mục tiêu lần lượt là $0{,}6$ và $0{,}7$. Xét các biến cố\\ 
	$A$: \lq\lq  Xạ thủ An bắn trúng mục tiêu\rq\rq;\\
	$B$: \lq\lq  Xạ thủ Bình bắn trúng mục tiêu \rq\rq.\\
	Xét tính đúng sai của các khẳng định sau?
	\choiceTF
	{$\mathrm{P}(\overline{A})=0{,}6$; $\mathrm{P}(\overline{B})=0{,}7$}
	{\True Hai biến cố $\overline{A}, \overline{B}$ là độc lập}
	{Xác suất cả hai xạ thủ đều không bắn trúng mục tiêu là $0{,}42$}
	{Xác suất cả hai xạ thủ đều bắn trúng mục tiêu là $0{,}58$}
	\loigiai{
	\begin{itemchoice}
	\itemch Sai. Do $\mathrm{P}(A)=0{,}6$; $\mathrm{P}(B)=0{,}7$.
	\itemch Đúng. Vì hai xạ thủ bắn ở hai thời điểm khác nhau nên các cặp biến cố $A$ và $B, \overline{A}$ và $\overline{B}$ là độc lập.
	\itemch Sai. Vì $\mathrm{P}(\overline{A} \cap \overline{B})=0{,}4 \cdot 0{,}3=0{,}12$.
	\itemch Sai. Vì $\mathrm{P}(A \cap B)=0{,}6 \cdot 0{,}7=0{,}42$.
	\end{itemchoice}
	}
\end{ex}
\begin{ex}%[2D5H1-2]
	Một xạ thủ bắn vào bia số $1$ và bia số $2$. Xác suất để xạ thủ đó bắn trúng bia số $1$, bia số $2$ lần lượt là $0{,}8$; $0{,}9$. Xác suất để xạ thủ đó bắn trúng cả hai bia là $0{,}8$. Xét hai biến cố
	\begin{itemize}
	\item $A$: \lq\lq  Xạ thủ đó bắn trúng bia số $1$\rq\rq;
	\item $B$: \lq\lq  Xạ thủ đó bắn trúng bia số $2$\rq\rq.
	\end{itemize}
	Xét tính đúng sai của các khẳng định sau?
	\choiceTF
	{Hai biến cố $A$ và $B$ có độc lập}
	{Biết xạ thủ đó bắn trúng bia số $1$ thì xác suất xạ thủ đó bắn trúng bia số $2$ là $0{,}72$}
	{\True Biết xạ thủ đó không bắn trúng bia số $1$, thì xác suất xạ thủ đó bắn trúng bia số $2$ bằng $0{,}9$}
	{Biết xạ thủ đó không bắn trúng bia số $1$ thì xác suất xạ thủ đó bắn không trúng bia số $2$ bằng $0{,}9$}
	\loigiai{
	\begin{itemchoice}
	\itemch Đúng. $A$ và $B$ là hai biến cố độc lập.
	\itemch Sai. Xác suất xạ thủ đó bắn trúng bia số $2$ và bia số $1$ là $$\mathrm{P}(B|A)=\dfrac{\mathrm{P}(B\cap A)}{\mathrm{P}(A)}=\dfrac{\mathrm{P}(B) \cdot \mathrm{P}(A)}{\mathrm{P}(A)}=\mathrm{P}(B)=0{,}9.$$
	\itemch Đúng. Xác suất xạ thủ đó bắn không bắn trứng bia số $1$ mà trúng bia số $2$ là $$\mathrm{P}(B|\overline{A})=\dfrac{\mathrm{P}(B\cap \overline{A})}{\mathrm{P}(\overline{A})}=\dfrac{\mathrm{P}(B) \cdot \mathrm{P}(\overline{A})}{\mathrm{P}(\overline{A})}=\mathrm{P}(B)=0{,}9.$$
	\itemch Sai. Xác suất xạ thủ đó bắn không bắn trúng bia số $1$ và không trúng bia số $2$ là
	$$\mathrm{P}(\overline{B}|\overline{A})=\dfrac{\mathrm{P}(\overline{B}\cap \overline{A})}{\mathrm{P}(\overline{A})}=\dfrac{\mathrm{P}(\overline{B}) \cdot \mathrm{P}(\overline{A})}{\mathrm{P}(\overline{A})}=\mathrm{P}(\overline{B})=1-\mathrm{P}(B)=1-0{,}9=0{,}1.$$
	\end{itemchoice}
	}
\end{ex}
\begin{ex}%[2D6C1-2] 
	Cho $A$ và $B$ là hai biến cố độc lập với $\mathrm{P}(A)=0{,}7$ và $\mathrm{P}(B)=0{,}4$. Xét tính đúng sai của các khẳng định sau:
	\choiceTF[4]
	{$\mathrm{P}(A|B)=0{,}6$}
	{\True $\mathrm{P}(B|\overline{A})=0{,}4$}
	{\True $\mathrm{P}(\overline{A}|B)=0{,}3$}
	{\True $\mathrm{P}(\overline{B}|\overline{A})=0{,}6$}
	\loigiai{
	\begin{itemchoice}
	\itemch $A$ và $B$ độc lập nên $\mathrm{P}(A|B)= \dfrac{\mathrm{P}(A)\cdot \mathrm{P}(B)}{\mathrm{P}(B)} =\mathrm{P}(A)=0{,}7$.
	\itemch $\overline{A}$ và $B$ độc lập nên $\mathrm{P}(B|\overline{A})=\dfrac{\mathrm{P}(B)\cdot \mathrm{P}(\overline{A})}{\mathrm{P}(\overline{A})}=\mathrm{P}(B)=0{,}4$.
	\itemch $\overline{A}$ và $B$ độc lập nên $\mathrm{P}(\overline{A}|B)= \dfrac{\mathrm{P}(B)\cdot \mathrm{P}(\overline{A})}{\mathrm{P}(B)}= \mathrm{P}(\overline{A})=1-\mathrm{P}(A)=0{,}3$.
	\itemch $\overline{A}$ và $\overline{B}$ độc lập nên $\mathrm{P}(\overline{B}|\overline{A})=\dfrac{\mathrm{P}(\overline{B})\cdot \mathrm{P}(\overline{A})}{\mathrm{P}(\overline{A})}=\mathrm{P}(\overline{B})=0{,}6$.
	\end{itemchoice}
	}
\end{ex}
\begin{ex}%[2D6C1-2] 
	Cho $A$ và $B$ với $\mathrm{P}(\overline{A})=0{,}4$; $\mathrm{P}(B)=0{,}8$ và $\mathrm{P}(AB)=0{,}4$. Xét tính đúng sai của các khẳng định sau:
	\choiceTF
	{\True $\mathrm{P}(A|B)=\dfrac{1}{2}$}
	{$\mathrm{P}(B|A)=\dfrac{1}{2}$}
	{\True $\mathrm{P}(\overline{B}|A)=\dfrac{1}{3}$}
	{$\mathrm{P}(\overline{A}B)=\dfrac{3}{5}$}
	\loigiai{
	\begin{itemchoice}
	\itemch $\mathrm{P}(A|B)=\dfrac{\mathrm{P}(AB)}{\mathrm{P}(B)}=\dfrac{1}{2}$.
	\itemch $\mathrm{P}(B|A)=\dfrac{P(AB)}{P(A)}=\dfrac{P(AB)}{1-P(\overline{A})}=\dfrac{2}{3}$.
	\itemch $\mathrm{P}(\overline{B}|A)=\dfrac{\mathrm{P}(A\overline{B})}{\mathrm{P}(A)}=\dfrac{\mathrm{P}(A)-\mathrm{P}(AB)}{\mathrm{P}(A)}=\dfrac{1}{3}$.
	\itemch $\mathrm{P}(\overline{A}B)=\mathrm{P}(\overline{A}|B) \cdot \mathrm{P}(B)= \dfrac{\mathrm{P}(\overline{A}B)}{\mathrm{P}(B)} \cdot \mathrm{P}(B) = \mathrm{P}(\overline{A}B) = \mathrm{P}P(B)-\mathrm{P}(AB) =\dfrac{2}{5}$.
	\end{itemchoice}
	}
\end{ex}
\begin{ex}%[2D6C1-4] 
	Một công ty truyền thông đấu thầy $2$ dự án. Khả năng thắng thầu của dự án $1$ là $0{,}5$ và dự án $2$ là $0{,}6$. Khả năng thắng thầu cả $2$ dự án là $0{,}4$. Gọi $A$, $B$ lần lượt là biến cố thắng thầu dự án $1$ và $2$. Xét tính đúng sai của các khẳng định sau:
	\choiceTF
	{$A$, $B$ là hai biến cố độc lập.}
	{\True Xác xuất công ty thắng thầu đúng $1$ dự án là $0{,}3$}
	{\True Biết công ty thắng thầu dự án $1$, xác suất công ty thắng thầu dự án $2$ là $0{,}8$}
	{Biết công ty không thắng thầu dự án $1$, xác suất công ty thắng thầu dự án $2$ là $0{,}6$}
	\loigiai{
	\begin{itemchoice}
	\itemch Theo đề bài: $\mathrm{P}(A)=0{,}5$; $\mathrm{P}(B)=0{,}6$ và $\mathrm{P}(AB)=0{,}4$.\\
	Vì $\mathrm{P}(AB) \ne \mathrm{P}(A)\cdot \mathrm{P}(B)$ nên $A$, $B$ không độc lập.
	\itemch Gọi biến cố $C$: "Thắng thầu đúng một dự án."\\
	Khi đó
	\begin{eqnarray*}
	& \mathrm{P}(C) & =\mathrm{P}(\overline{A}B)+\mathrm{P}(A\overline{B})\\
	& & = [\mathrm{P}(B)-\mathrm{P}(AB)] + [ \mathrm{P}(A)-\mathrm{P}(AB) ]\\
	& & = \mathrm{P}(A) + \mathrm{P}(B) - 2\mathrm{P}(AB)\\
	&& = 0{,}3.
	\end{eqnarray*}
	\itemch Biết công ty thắng thầu dự án $1$, xác suất công ty thắng thầu dự án $2$ là
	$$\mathrm{P}(B|A)=\dfrac{\mathrm{P}(AB)}{\mathrm{P}(A)}=\dfrac{0{,}4}{0{,}5}=0{,}8.$$
	\itemch Biết công ty không thắng thầu dự án $1$, xác suất công ty thắng thầu dự án $2$ là
	$$\mathrm{P}(B|\overline{A})=\dfrac{\mathrm{P}(\overline{A}B)}{\mathrm{P}(\overline{A})}=\dfrac{\mathrm{P}(B)-\mathrm{P}(AB)}{\mathrm{P}(\overline{A})}=\dfrac{0{,}6-0{,}4}{0{,}5}=0{,}4.$$
	\end{itemchoice}
	}
\end{ex}
\begin{ex}%[2D6C1-4] 
	Ở một sân bay, người ta sử dụng một loại máy soi tự động phát hiện hàng cấm trong hành lí kí gửi. Máy phát chuông cảnh báo với $95 \%$ các kiện hành lí có chứa hàng cấm và $2\%$ các kiện hành lí không chứa hàng cấm. Tỉ lệ các kiện hành lí có chứa hàng cấm là $0{,1}\%$. Chọn ngẫu nhiên một kiện hành lí để soi bằng máy trên. Xét tính đúng sai của các mệnh đề sau:
	\choiceTF
	{\True Máy không phát chuông cảnh báo với $5\%$ các kiện hành lí có chứa hàng cấm}
	{\True Máy không phát chuông cảnh báo với $98\%$ các kiện hành lí không chứa hàng cấm}
	{Xác suất chọn được kiện hành lí có chứa hàng cấm và máy phát chuông cảnh báo là $0{,}0095$}
	{\True Xác suất chọn được kiện hành lí không chứa hàng cấm và máy phát chuông cảnh báo là $0{,}01998$}
	\loigiai{
	Xét các biến cố:\\
	$A$: "Kiện hành lí có chứa hàng cấm."\\
	$B$: "Máy phát chuông cảnh báo."\\
	Theo đề, ta có $\mathrm{P}(B|A)=0{,}95$; $\mathrm{P}(B| \overline{A})=0{,}02$ và $\mathrm{P}(A)=0{,}001$.
	\begin{itemchoice}
	\itemch Xác suất máy không phát chuông cảnh báo với các kiện hành lí có chứa hàng cấm là
	$$P(\overline{B}|A)=1-P(B|A)=0{,}05=5\%.$$
	\itemch Xác suất máy không phát chuông cảnh báo với các kiện hành lí không chứa hàng cấm là
	$$\mathrm{P}(\overline{B}|\overline{A})=1-\mathrm{P}(B|\overline{A})=0{,}05=1-0{,}02=98\%.$$
	\itemch 
	Xác suất chọn được kiện hành lí có chứa hàng cấm và máy phát chuông cảnh báo là
	$$ \mathrm{P}(AB)=P(B|A) \cdot \mathrm{P}(A) =0{,}00095 = 0{,}095\% .$$
	\itemch Ta có
	\begin{eqnarray*}
	& & \mathrm{P}(B| \overline{A})=0,{,}02\\
	&\Leftrightarrow & \dfrac{\mathrm{P}(B)-\mathrm{P}(AB)}{\mathrm{P}(\overline{A})}=0{,}02\\
	&\Leftrightarrow & \mathrm{P}(B)= 0{,}02 \cdot \mathrm{P}(\overline{A}) + \mathrm{P}(AB)\\
	&\Leftrightarrow & \mathrm{P}(B)= 0{,}02093.
	\end{eqnarray*}
	Ta có $\mathrm{P}(AB)=\mathrm{P}(B|A) \cdot \mathrm{P}(A) =0{,}00095 = 0{,}095\% $.\\
	Xác suất chọn được kiện hành lí không chứa hàng cấm và máy phát chuông cảnh báo là
	$$\mathrm{P}(\overline{AB})=\mathrm{P}(B)-\mathrm{P}(AB)= 0{,}02093 - 0{,}00095= 0{,}01998.$$
	\end{itemchoice}
	}
\end{ex}
\begin{ex}%[2D5H1-2]
	Một lớp học có $17$ học sinh nam và $24$ học sinh nữ. Cô giáo gọi ngẫu nhiên lần lượt $2$ học sinh (có thứ tự) lên trả lời câu hỏi. Xét các biến cố\\
	$A$: \lq\lq  Lần thứ nhất cô giáo gọi $1$ học sinh nam\rq\rq;\\
	$B$: \lq\lq  Lần thứ hai cô giáo gọi $1$ học sinh nữ\rq\rq.\\
	Xét tính đúng sai của các khẳng định sau?
	\choiceTF
	{$\mathrm{P}(B \mid A)=0{,}575$}
	{$\mathrm{P}(B \mid \overline{A})=0{,}6$}
	{$\mathrm{P}(\overline{B} \mid A)=0{,}425$}
	{$\mathrm{P}(\overline{B} \mid \overline{A})=0{,}4$}
	\loigiai{
	Nếu lần thứ nhất gọi $1$ học sinh nam thì số học sinh còn lại là $40$ , số học sinh nam còn lại là $16$, số học sinh nữ giữ nguyên; nếu lần thứ nhất gọi $1$ học sinh nữ thì số học sinh còn lại là $40$, số học sinh nam giữ nguyên, số học sinh nữ còn lại là $23$.
	\begin{itemchoice}
	\itemch Sai. Vì $\mathrm{P}(B \mid A)=\dfrac{24}{40}=0{,}6$.
	\itemch	Sai. Vì $\mathrm{P}(B \mid \overline{A})=\dfrac{23}{40}=0{,}575$.
	\itemch Sai. Vì $\mathrm{P}(\overline{B} \mid A)=\dfrac{16}{40}=0{,}4$.
	\itemch Sai. Vì $\mathrm{P}(\overline{B} \mid \overline{A})=\dfrac{17}{40}=0{,}425$.
	\end{itemchoice}
	}
\end{ex}
\begin{ex}%[2D5H1-2]
	Gieo một xúc xắc cân đối và đồng chất $1$ lần. Xét các biến cố:\\
	$A$: \lq\lq  Mặt xuất hiện của xúc xắc ghi số $5$\rq\rq;\\
	$B$: \lq\lq  Mặt xuất hiện của xúc xắc ghi số lẻ\rq\rq.\\
	Xét tính đúng sai của các khẳng định sau?
	\choiceTF
	{$\mathrm{P}(A)=\dfrac{5}{6}$}
	{\True $\mathrm{P}(A \cap B)=\dfrac{1}{6}$}
	{\True $\mathrm{P}(B \mid A)=1$}
	{$\mathrm{P}(A \mid B)=\dfrac{1}{2}$}
	\loigiai{
	\begin{itemchoice}
	\itemch Sai. Vì $\mathrm{P}(A)=\dfrac{1}{6}$ .
	\itemch	Đúng. Vì $\mathrm{P}(A \cap B)=\dfrac{1}{6}$ .
	\itemch Đúng. Vì $\mathrm{P}(B \mid A)=1$.
	\itemch Sai. Vì $\mathrm{P}(B)=\dfrac{1}{2}$. Khi đó, $\mathrm{P}(A \mid B)=\dfrac{\frac{1}{6}}{\frac{1}{2}}=\dfrac{1}{3}$.
	\end{itemchoice}
	}
\end{ex}
\begin{ex}%[2D5V1-2]
	Một cửa hàng kinh doanh tổ chức rút thăm trúng thưởng cho hai loại sản phẩm. Tỉ lệ trúng thưởng của các loại sản phẩm I, II lần lượt là $6 \%$; $4 \%$. Trong một hộp kín gồm các thăm cùng loại, người ta để lẫn lộn $200$ chiếc thăm cho sản phẩm loại I và $300$ chiếc thăm cho sản phẩm loại II. Một khách hàng lấy ngẫu nhiên $1$ chiếc thăm từ chiếc hộp đó. Xét tính đúng sai của các khẳng định sau
	\choiceTF
	{Xác suất để chiếc thăm được lấy ra là trúng thưởng bằng $10\%$}
	{Xác suất thăm được lấy ra trúng thưởng là thăm cho sản phẩm loại I bằng $6\%$ }
	{\True Xác suất thăm được lấy ra trúng thưởng là thăm cho sản phẩm loại II bằng $50\%$}
	{Khả năng lấy ra được thăm trúng thưởng là thăm sản phẩm loại II cao hơn khả năng lấy ra được thăm trúng thưởng là thăm sản phẩm loại I}
	\loigiai{
	\begin{itemchoice}
	\itemch Sai. Xét biến cố $A$: \lq\lq  Chiếc thăm được lấy ra là trúng thưởng\rq\rq.\\	
	Khi đó, ta có\\	
	$\mathrm{P}(A)=\dfrac{6\% \cdot 200 + 4\% \cdot 300}{200+300}=0{,}048$.	
	\itemch Sai. Xét biến cố $B$: \lq\lq  Chiếc thăm được lấy ra là thăm cho sản phẩm loại I\rq\rq.\\
	Khi đó, ta có:\\
	$\mathrm{P}(B|A)=\dfrac{n(B\cap A)}{n(A)}=\dfrac{6\% \cdot 200}{6\% \cdot 200 + 4\% \cdot 300}=0{,}5$.
	\itemch	Đúng. Xét biến cố $C$: \lq\lq  Chiếc thăm được lấy ra là thăm cho sản phẩm loại II\rq\rq.\\
	Khi đó, ta có:\\
	$\mathrm{P}(C|A)=\dfrac{n(C\cap A)}{n(A)}=\dfrac{4\% \cdot 300}{6\% \cdot 200 + 4\% \cdot 300}=0{,}5$.
	\itemch Sai. Do $\mathrm{P}(B|A) = \mathrm{P}(C|A)$ nên xác suất hai chiếc thăm lấy được là như nhau.
	\end{itemchoice}	
	}
\end{ex}
\begin{ex}%[2D5V1-2]
	Một phòng nghiên cứu dược học cho $500$ người bị bệnh $H$ dùng hai loại thuốc $X$, $Y$ để điều trị. Một số người được điều trị bằng thuốc $X$ và số người còn lại được điều trị bằng thuốc $Y$. Kết quả nghiên cứu được trình bày ở bảng sau 
	\begin{center}
	\begin{tikzpicture}
	\begin{scope}[xscale=4.4]
	\path
	(0,0) foreach \i[count=\k] in {$X$,$Y$} {++(1,0)node(1\k){\i}}
	(0,-1) node {Khỏi bệnh} foreach \i[count=\k] in {$180$,$190$} {++(1,0)node(2\k){\i}}
	(0,-2) node{Không khỏi bệnh} foreach \i[count=\k] in {$60$,$70$} {++(1,0)node(3\k){\i}};
	\draw[shift={(-0.5,.5)}] (0,0) grid (3.,-3)
	(0,0)--(1.,-1)
	(0,-1) node[above right]{Tình trạng}
	(1,0) node[below left]{Loại thuốc}
	;
	\end{scope}
	\end{tikzpicture}
	\end{center}
	Chọn ngẫu nhiên một người trong số này. Gọi $A$ là biến cố "Người được chọn khỏi bệnh", $B$ là biến cố "Người được chọn điều trị bằng thuốc $X$", $C$ là biến cố "Người được chọn điều trị bằng thuốc $Y$". \\ Xét tính đúng sai của các khẳng định sau?
	\choiceTF
	{Xác suất để một người khỏi bệnh khi điều trị bằng thuốc $Y$ có kí hiệu là $\mathrm{P}(A|B)$ }
	{\True $\mathrm{P}(A|B)=\dfrac{3}{4}$}
	{Xác xuất để một người khỏi bênh khi điều trị bằng thuốc $Y$ bằng $\dfrac{7}{26}$}
	{\True Thuốc $X$ có hiệu quả hơn thuốc $Y$ trong điều trị bệnh}
	\loigiai{
	\begin{itemchoice}
	\itemch Sai. Vì xác suất để một người khỏi bệnh khi điều trị bằng thuốc $X$ có kí hiệu là $\mathrm{P}(A|B)$. 
	\itemch Đúng. Vì $\mathrm{P}(A|B)=\dfrac{n\left(AB\right)}{n(B)}=\dfrac{180}{240}=\dfrac{3}{4}$.
	\itemch Sai. Vì $\mathrm{P}(A|C)=\dfrac{n\left(A C\right)}{n(C)}=\dfrac{190}{260}=\dfrac{19}{26}$. \\
	\itemch Đúng. Vì xác suất để một người khỏi bệnh khi được chọn điều trị bằng thuốc $X$ là $\dfrac{3}{4}$ và xác suất để một người khỏi bệnh khi được chọn điều trị bằng thuốc $Y$ là $\dfrac{19}{26}$. Do $\dfrac{3}{4}>\dfrac{19}{26}$ nên loại thuốc $X$ có hiệu quả hơn loại thuốc $Y$ trong việc điều trị bệnh $H$. 
	\end{itemchoice}
	}
\end{ex}
\Closesolutionfile{ans}
\indapan{3}{ans/ans-2-B18-DS}
\Opensolutionfile{ans}[ans/ans-2-B18-KQ]
%\TNSA
\begin{ex}%[2D5N1-2]
	Cho hai biến cố $A$, $B$ có xác suất $\mathrm{P}(A)=0{,}4$, $\mathrm{P}(B)=0{,}6$; $\mathrm{P}(AB)=0{,}2$. Xác suất $\mathrm{P}(\overline{A}\mid B)=\dfrac{a}{b}$ với $\dfrac{a}{b}$ là phân số tối giản. Tính $M=a^2+b^2$.
	\shortans{$13$}
	\loigiai{
	Theo định nghĩa và tính chất xác suất có điều kiện, ta có
	$$\mathrm{P}(\overline{A}\mid B)=1-\mathrm{P}(A \mid B)=1-\dfrac{\mathrm{P}(A B)}{\mathrm{P}(B)}=1-\dfrac{0{,}2}{0{,}6}=1-\dfrac{1}{3}=\dfrac{2}{3}.$$
	Từ đó ta có $a=2$, $b=3$. Vậy $M=a^2+b^2=2^2+3^2=13$.
	}
\end{ex}
\begin{ex}%[2D5H1-2]
	Một công ty bảo hiểm nhận thấy có $48 \%$ số người mua bảo hiểm ô tô là phụ nữ và có $36 \%$ số người mua bảo hiểm ô tô là phụ nữ trên $45$ tuổi. Biết một người mua bảo hiểm ô tô là phụ nữ, tính xác suất người đó trên $45$ tuổi.
	\shortans{$0{,}75$}
	\loigiai
	{
	Gọi $A$ là biến cố \lq\lq  Người mua bảo hiểm ô tô là phụ nữ\rq\rq, $B$ là biến cố \lq\lq  Người mua bảo hiểm ô tô trên 45 tuổi\rq\rq. Ta cần tính $\mathrm{P}(B \mid A)$.\\
	Do có $48 \%$ người mua bảo hiểm ô tô là phụ nữ nên $\mathrm{P}(A)=0{,}48$.\\
	Do có $36 \%$ số người mua bảo hiểm ô tô là phụ nữ trên $45$ tuổi nên $\mathrm{P}(A B)=0{,}36$.\\
	Vậy $\mathrm{P}(B \mid A)=\dfrac{\mathrm{P}(A B)}{\mathrm{P}(A)}=\dfrac{0{,}36}{0{,}48}=0{,}75$. 
	}
\end{ex}
\begin{ex}%[2D5V1-2]
	Một nhóm $5$ học sinh nam và $4$ học sinh nữ tham gia lao động trên sân trường. Cô giáo chọn ngẫu nhiên đồng thời $2$ bạn trong nhóm đi tưới cây. Tính xác suất để hai bạn được chọn có cùng giới tính, biết rằng có ít nhất $1$ bạn nam được chọn.	(Kết quả làm tròn đến hai chữ số thập phân)
	\shortans{$0{,}67$}
	\loigiai{
	Số phần tử của không gian mẫu là $n(\Omega)=\mathrm{C}^2_9=36$.\\
	Gọi A là biến cố \lq\lq  Hai bạn được chọn có cùng giới tính\rq\rq.\\
	B là biến cố \lq\lq  Có ít nhất một bạn nam được chọn\rq\rq.\\
	Ta có $n(B)=\mathrm{C}^2_5+\mathrm{C}^1_5=15$ suy ra $\mathrm{P}(B)=\dfrac{15}{36}$.\\
	Ta có $n(AB)=\mathrm{C}^2_5=10$ suy ra $\mathrm{P}(AB)=\dfrac{10}{36}$.\\
	Vậy $\mathrm{P}(A|B)=\dfrac{\mathrm{P}(AB)}{\mathrm{P}(B)}=\dfrac{10}{15}=\dfrac{2}{3}\approx 0{,}67$.
	} 
\end{ex}
\begin{ex}%[2D5V1-2]
	Kết quả khảo sát những bệnh nhân bị tai nạn xe máy về mối liên hệ giữa việc đội mũ bảo hiểm và khả năng bị chấn thương vùng đầu cho thấy:\\
	- Tỉ lệ bệnh nhân bị chấn thương vùng đầu khi gặp tai nạn là $80 \%$;\\
	- Tỉ lệ bệnh nhân đội mũ bảo hiểm đúng cách khi gặp tai nạn là $90 \%$;\\
	- Tỉ lệ bệnh nhân đội mũ bảo hiểm đúng cách bị chấn thương vùng đầu là $18 \%$.\\
	Hỏi theo kết quả điều tra trên, việc đội mũ bảo hiểm đúng cách sẽ làm giảm khả năng bị chấn thương vùng đầu bao nhiêu lần? (Kết quả làm tròn đến hàng phần mười).
	\shortans{$4{,}6$}
	\loigiai{
	Gọi A là biến cố: \lq\lq  Bệnh nhân bị chấn thương vùng đầu khi gặp tai nạn\rq\rq\, và B là biến cố: \lq\lq  Bệnh nhân đội mũ bảo hiểm đúng cách khi gặp tai nạn\rq\rq.\\
	Theo đề bài ta có $\mathrm{P}(A)=0{,}8, \mathrm{P}(B)=0{,}9, \mathrm{P}(B|A)=0{,}18$.\\
	Suy ra $\mathrm{P}(B\mid A)=\dfrac{\mathrm{P}(AB)}{\mathrm{P}(A)}\Rightarrow \mathrm{P}(AB)=\mathrm{P}(A)\cdot \mathrm{P}(B\mid A)=0{,}8\cdot 0{,}18=0{,}144$.\\
	Vì $A\bar{B}$ và $AB$ là hai biến cố xung khắc nên $A\overline{B}\cup AB=A$.\\
	Suy ra $\mathrm{P}(\overline{B}A)=\mathrm{P}(A)-\mathrm{P}(AB)=0{,}8-0{,}144=0{,}656$.\\
	Ta có $\mathrm{P}(\overline{B}\mid A)=\dfrac{\mathrm{P}(\overline{B}A)}{\mathrm{P}(A)}=\dfrac{0{,}656}{0{,}8}=0{,}82$.
	Khi đó $\dfrac{\mathrm{P}(\overline{B}\mid A)}{\mathrm{P}(B|A)}=\dfrac{0{,}82}{0{,18}}\approx 4{,}6$.\\
	Như vậy việc đội mũ bảo hiểm đúng cách sẽ làm giảm khả năng chấn thương vùng đầu xuống $4{,}6$ lần.
	}
\end{ex} 
\begin{ex}%[2D5H1-2]
	Một công ty đấu thầu $2$ dự án. Khả năng thắng thầu của các dự án $I$ và $II$ lần lượt là $0{,}4$ và $0{,}5$. Khả năng thắng thầu của hai dự án là $0{,}3$. Gọi $A$, $B$ lần lượt là biến cố thắng thầu dự án $I$ và dự án $II$. Biết công ty không thắng thầu dự án $I$, tìm xác suất công ty thắng thầu dự án $II$.(Kết quả làm tròn đến hai chữ số thập phân)
	\shortans{$0{,}33$}
	\loigiai{
	Gọi $D$ là biến cố công ty thắng dự $II$ khi công ty đã không thắng dự án $I$. Ta có
	$$\mathrm{P}(D)=\mathrm{P}(B\mid\overline{A})=\dfrac{\mathrm{P}(\overline{A}B)}{\mathrm{P}(\overline{A})}=\dfrac{\mathrm{P}(B)-\mathrm{P}(AB)}{1-\mathrm{P}(\overline{A})}=\dfrac{0{,}5-0{,}3}{1-0{,}4}=\dfrac{1}{3}\approx 0{,}33.$$
	}
\end{ex}
\begin{ex}%[2D5V1-3]
	Hộp thứ nhất có $4$ viên bi xanh và $6$ viên bi đỏ. Hộp thứ hai có $5$ viên bi xanh và $4$ viên bi đỏ. Các viên bi có cùng kích thước và khối lượng. Lấy ra ngẫu nhiên $1$ viên bi từ hộp thứ nhất chuyển sang hộp thứ hai. Sau đó lại lấy ra ngẫu nhiên $1$ viên bi từ hộp thứ hai.
	Sử dụng sơ đồ hình cây, tính xác suất của biến cố
	$B\colon$ \lq\lq  Hai viên bi lấy ra có cùng màu\rq\rq.
	\shortans{$0{,}54$}
	\loigiai{
	Gọi $X$ là biến cố: \lq\lq  Viên bi lấy ra từ hộp thứ nhất có màu xanh\rq\rq.\\
	$Y$ là biến cố: \lq\lq  Viên bi lấy ra từ hộp thứ hai có màu đỏ\rq\rq.\\
	Ta có
	$
	\mathrm{P}(Y|X)=0{,}4$; $\mathrm{P}(Y \mid \overline{X})=0{,}5 $; $\mathrm{P}(X)=0{,}4
	$.\\
	Do đó $\mathrm{P}(\overline{X})=1-\mathrm{P}(X)=0{,}6$; $\mathrm{P}(\overline{Y} |X)=1-\mathrm{P}(Y|X)=0{,}6$; \\
	$\mathrm{P}(\overline{Y} \mid \overline{X})=1-\mathrm{P}(Y \mid \overline{X})=0{,}5$.\\
	Ta có sơ đồ hình cây như sau
	\begin{center}
	\begin{tikzpicture}
	\def\gocm{20}
	\def\gocn{10}
	\def\r{4}
	\tikzset{s/.style={outer sep=0.5 mm,draw=magenta,rectangle,minimum width=2.75cm,rounded corners=1mm}}
	\path(0,0)node(O){}++(\gocm:\r)node[s](A1){X}++(\gocn:\r)node[s](A2){$Y$};
	\path(A1)++({-\gocn}:\r)node[s](a2){$\overline{Y}$};
	\path(O)++(-\gocm:\r)node[s](B1){$\overline{X}$}++(\gocn:\r)node[s](B2){$Y$};
	\path(B1)++({-\gocn}:\r)node[s](b2){$\overline{Y}$};
	\foreach \x/\y in {
	O/A1,A1/A2,
	O/B1,B1/B2,
	A1/a2,
	B1/b2}
	\draw[-stealth](\x.east)--(\y.west);
	\path(O)--(A1.west)node[pos=0.5,above,sloped]{$0{,}4$}(O)--(B1.west)node[pos=0.5,below,sloped]{$0{,}6$}(B1.east)--(B2.west)node[pos=0.5,above,sloped]{$0{,}5$}(A1.east)--(A2.west)node[pos=0.5,above,sloped]{$0{,}4$}
	(A1.east)--(a2.west)node[pos=0.5,below,sloped]{$0{,}6$}
	(B1.east)--(b2.west)node[pos=0.5,below,sloped]{$0{,}5$};
	\end{tikzpicture}
	\end{center}
	Khi đó $\mathrm{P}(B)=\mathrm{P}(X\overline{Y})+\mathrm{P}(\overline{X}Y)=0{,}4\cdot 0{,}6+0{,}6\cdot 0{,}5=0{,}54$.
	}
\end{ex}
\begin{ex}%[2D6H1-2] 
	Cho hai biến cố $A$, $B$ có $P(A)=0{,}51$; $P(B)=0{,}2$; $P(A | B)=0{,}8$. Tính $P(B | A)$, làm tròn kết quả đến hàng phần trăm.
	\shortans{$0{,}31$}
	\loigiai{
	Ta có
	\begin{eqnarray*}
	&&P(A | B)=0{,}8\Leftrightarrow \dfrac{P(A\cap B)}{P(B)}=0{,}8\\
	&\Leftrightarrow& P(A \cap B)=0{,}8\cdot P(B)=0{,}8\cdot 0{,}2=0{,}16\\
	&\Rightarrow& P(B|A)=\dfrac{P(A\cap B)}{P(A)}=\dfrac{0{,}16}{0{,}51} \approx 0{,}31.
	\end{eqnarray*}
	}
\end{ex}
\begin{ex}%[2D6H1-2] 
	Cho hai biến cố $A$, $B$ có $P(A)=0{,}8$; $P(B)=0{,}5$; $P(A \cap B)=0{,}2$. Tính xác suất biến cố $B$ không xảy ra với điều kiện biến cố $A$ xảy ra.
	\shortans{$0{,}75$}
	\loigiai{
	Xác suất biến cố $B$ không xảy ra với điều kiện biến cố $A$ xảy ra là
	$$P(\overline{B} |A)=\dfrac{P(A\overline{B})}{P(A)}=\dfrac{P(A)-P(A\cap B)}{P(A)}=\dfrac{3}{4}=0{,}75.$$
	}
\end{ex}
\begin{ex}%[2D6V1-4] 
	Gieo một con xúc xắc cân đối và đồng chất hai lần. Tính xác suất để tổng số chấm xuất hiện trong hai lần gieo nhỏ hơn $8$. Biết rằng con lần gieo thứ nhất xuất hiện mặt $4$ chấm.
	\shortans{$0{,}5$}
	\loigiai{
	Xét các biến cố:\\
	$A$: "Lần thứ nhất xuất hiện mặt 4 chấm."\\
	$B$: "Tổng số chấm trong hai lần gieo nhỏ hơn $8$."\\
	Khi đó $$A=\{(4;1), (4;2), (4;3), (4;4), (4;5), (4;6)\};$$
	$$A\cap B= \{(4;1), (4;2), (4;3)\}.$$
	Suy ra $n(A)=6$ và $n(A\cap B)=3$.\\
	Vậy xác suất để tổng số chấm xuất hiện trong hai lần gieo bằng 6, biết rằng con lần gieo thứ nhất xuất hiện mặt 4 chấm là
	$$P(B|A)=\dfrac{n(A\cap B)}{n(A)}=\dfrac{3}{6}=0{,}5.$$
	}
\end{ex}
\begin{ex}%[2D6V1-4] 
	Một công ty bảo hiểm nhận thấy có $56 \%$ số người mua bảo hiểm sức khỏe là phụ nữ và có $42 \%$ số người mua bảo hiểm sức khỏe là phụ nữ trên 50 tuổi. Tính tỉ lệ người trên $50$ tuổi trong số những người phụ nữ mua bảo hiểm sức khỏe.
	\shortans{ $0{,}75$}
	\loigiai{
	Xét các biến cố:\\
	A: "Người mua bảo hiểm sức khỏe là phụ nữ."\\
	B: "Người mua bảo hiểm sức khỏe là phụ nữ trên $50$ tuổi."\\
	Khi đó $P(A)=0{,}56$ và $P(A \cap B)=0{,}42$.\\
	Xác suất người mua bảo hiểmTỉ lệ người trên $45$ tuổi trong số những người phụ nữ mua bảo hiểm sức khỏe là
	$$P(B|A)=\dfrac{P(A\cap B)}{P(A)}=\dfrac{0{,}42}{0{,}56}=0{,}75.$$}
\end{ex}
\begin{ex}%[2D6C1-4] 
	Tại một khu phố có $100$ căn nhà, trong đó có $40$ căn nhà gắn biển số lẻ. Biết rằng có $25$ căn nhà gắn biển số lẻ và $15$ nhà gắn biển số chẵn có ô tô. Chọn ngẫu nhiên một nhà trong khu phố đó. Tính xác suất nhà được chọn gắn biển số lẻ, biết rằng nhà đó không có ô tô.
	\shortans{$0{,}25$}
	\loigiai{
	Xét các biến cố:\\
	$A$: "Nhà được chọn gắn số lẻ."\\
	$B$: "Nhà được chọn có ô tô."\\
	Khi đó xác suất nhà được chọn gắn biển số lẻ, biết rằng nhà đó không có ô tô, là $P(A | \overline{B})$.\\
	Số căn nhà gắn số lẻ và không có ô tô là $n(A\cap \overline{B})=40-25=15$.\\
	Số căn nhà không có ô tô là $n(\overline{B})=100-(25+15)=60$.\\
	Ta có
	$$P(A | \overline{B})=\dfrac{n(A\cap \overline{B})}{n(\overline{B})}=0{,}25.$$
	}
\end{ex}
\begin{ex}%[2D6C1-4] 
	Kết quả một cuộc khảo sát các vụ tai nạn giao thông ô tô về mối quan hệ giữa việc thắt dây an toàn của người lái xe khi xảy ra tai nạn giao thông và nguy cơ tử vong của người lái xe khi xảy ra tai nạn giao thông cho thấy:
	\begin{itemize}
	\item Tỉ lệ người lái xe tử vong khi xảy ra tai nạn giao thông là $0{,4}\%$.
	\item Tỉ lệ người lái xe không thắt dây an toàn giao thông khi xảy ra tai nạn giao thông là $28 \%$.
	\item Tỉ lệ người lái xe tử vong khi xảy ra tai nạn giao thông trong trường hợp không thắt dây an toàn là $0{,}3 \%$.
	\end{itemize}
	Hỏi theo kết quả khảo sát trên, việc thắt dây an toàn của người lái xe ô tô sẽ làm giảm khả năng tử vong là bao nhiêu lần? (làm tròn đến hàng phần mười).
	\shortans{$7{,}7$}
	\loigiai{
	Chọn ngẫu nhiên một một vụ tai nạn giao thông của cuộc khảo sát trên. Xét các biến cố:\\
	$A$: "Người lái xe đó tử vong khi xảy ra tai nạn giao thông."\\
	$B$: "Người lái xe đó không thắt dây an toàn khi xảy ra tai nạn giao thông."\\
	Ta có $P(A)= 0{,4}\%$; $P(B)= 28\%$; $P(A\cap B)=0{,}3\%$.\\
	Xác suất người lái xe đó tử vong khi xảy ra tai nạn giao thông trong trường hợp không thắt dây an toàn là
	$$P(A|B)=\dfrac{P(A\cap B)}{P(B)}=\dfrac{3}{280}.$$
	Xác suất người lái xe đó có thắt dây an toàn giao thông là $P(\overline{B})=72\%$.\\
	Xác suất người lái xe đó tử vong khi xảy ra tai nạn giao thông trong trường hợp có thắt dây an toàn là
	$$P(A|\overline{B})=\dfrac{P(A\cap \overline{B})}{P(\overline{B})}=\dfrac{P(A)-P(A\cap B)}{P(\overline{B}}=\dfrac{1}{720}.$$
	Ta có
	$$\dfrac{P(A|B)}{P(A|\overline{B})}=\dfrac{54}{7}\approx7{,}7.$$
	Vậy theo khảo sát trên, việc thắt dây an toàn của người lái xe ô tô sẽ làm giảm khả năng tử vong khoảng $7{,}7$ lần.
	}
\end{ex}
\Closesolutionfile{ans}
\indapan{6}{ans/ans-2-B18-KQ}
% \setcounter{section}{1}
\section{CÔNG THỨC XÁC SUẤT TOÀN PHẦN VÀ CÔNG THỨC BAYES}
%%%%%%%%%%%%%%%%
\subsection{Trọng tâm kiến thức}
\begin{tomtat}
\subsubsection{Công thức xác suất toàn phần}
\begin{boxdn}
Cho hai biến cố $A$ và $B$ là hai biến cố tùy ý. Khi đó
$$\mathrm{P}(A)=\mathrm{P}(B)\cdot \mathrm{P}(A|B)+\mathrm{P}(\overline{B})\cdot \mathrm{P}(A|\overline{B}).$$
Công thức trên được gọi là công thức xác suất toàn phần.
\end{boxdn}
\subsubsection{Công thức Bayes}
\begin{boxdn}
Cho hai biến cố $A$ và $B$ với $\mathrm{P}(A)>0$. Khi đó
$$\mathrm{P}\left(B|A\right)=\dfrac{\mathrm{P}(B)\cdot\mathrm{P}\left(A|B\right)}{\mathrm{P}(A)}.$$
\end{boxdn}
\begin{note}
Công thức Bayes còn được viết dưới dạng 
$$\mathrm{P}(B | A)=\dfrac{\mathrm{P}(B) \cdot \mathrm{P}(A \mid B)}{\mathrm{P}(B)\cdot \mathrm{P}(A | B)+\mathrm{P}\left(\overline{B}\right)\cdot \mathrm{P}\left(A | \overline{B}\right)}.$$
\end{note}
\end{tomtat}
%%%%%%%%%%%%%%
\subsection{Các dạng bài tập}
\setcounter{dang}{0}
\begin{dang}{Tính xác suất theo công thức xác suất toàn phần}
	Với hai biến cố $A$ và $B$ tùy ý thì
	$\mathrm{P}(A)=\mathrm{P}(B)\cdot\mathrm{P}(A|B)+\mathrm{P}(\overline{B})\cdot\mathrm{P}(A|\overline{B}).$
\end{dang}
%----------------------------
\subsubsection{Ví dụ minh hoạ}
\begin{vd}%[2D5H2-2]
	Cho hai biến cố $A$, $B$ với $\mathrm{P}(B)=0{,}6$; $\mathrm{P}(A|B) =0{,}7$ và $\mathrm{P}\left(A|\overline{B}\right)=0{,}4$. Tính $\mathrm{P}(A)$
	\loigiai{
		Ta có $\mathrm{P}\left(\overline{B}\right)= 1- \mathrm{P}(B) = 1-0{,}6 = 0{,}4$.\\
		Áp dụng công thức xác suất toàn phần, ta có
		\[\mathrm{P}(A) = \mathrm{P}(A|B)\cdot \mathrm{P}(B) + \mathrm{P}\left(A|\overline{B}\right)\cdot \mathrm{P}\left(\overline{B}\right)= 0{,}7\cdot 0{,}6 + 0{,}4\cdot 0{,}4=0{,}58.\]
	}
\end{vd}
\begin{vd}%[2D5H2-2]
	Trong một kì thi tốt nghiệp trung học phổ thông, một tỉnh X có $80 \%$ học sinh lựa chọn tổ hợp A00 (gồm các môn Toán, Vật lí, Hoá học). Biết rằng, nếu một học sinh chọn tổ hợp A00 thì xác suất để học sinh đó đỗ đại học là 0{,}6; còn nếu một học sinh không chọn tổ hợp A00 thì xác suất để học sinh đó đỗ đại học là 0{,}7. Chọn ngẫu nhiên một học sinh của tỉnh X đã tốt nghiệp trung học phổ thông trong kì thi trên. Tính xác suất để học sinh đó đỗ đại học.
	\loigiai{Gọi $A$ là biến cố: \lq \lq Học sinh đó chọn tổ hợp A00\rq \rq~; $B$ là biến cố: \lq \lq Học sinh đó đỗ đại học\rq \rq.\\
		Ta cần tính $\mathrm{P}(B)$. Theo công thức xác suất toàn phần, ta cần biết: $\mathrm{P}(A), \mathrm{P}(\overline{A}), \mathrm{P}(B \mid A)$ và $\mathrm{P}(B \mid \overline{A})$.\\
		Ta có: $\mathrm{P}(A)=0{,}8; \mathrm{P}(\overline{A})=1-\mathrm{P}(A)=1-0{,}8=0{,}2$.\\
		$\mathrm{P}(B \mid A)$ là xác suất để một học sinh đỗ đại học với điều kiện học sinh đó chọn tổ hợp $A 00$ \\$\Rightarrow \mathrm{P}(B \mid A)=0{,}6$.\\
		$\mathrm{P}(B \mid \overline{A})$ là xác suất để một học sinh đỗ đại học với điều kiện học sinh đó không chọn tổ hợp $\mathrm{A} 00$\\$ \Rightarrow \mathrm{P}(B \mid \overline{A})=0{,}7$.\\
		Thay vào công thức xác suất toàn phần ta được:$$\mathrm{P}( B)={\mathrm{P}(A) \cdot \mathrm{P}(B \mid A)+\mathrm{P}(\overline{A}) \cdot \mathrm{P}(B \mid \overline{A})}={0{,}8 \cdot 0{,}6+0{,}2 \cdot 0{,}7} = 0{,}62.$$}
\end{vd}
\begin{vd}%[2D5H2-2]
	\immini{
		Số khán giả đến xem buổi biểu diễn ca nhạc ngoài trời phụ thuộc vào thời tiết. Giả sử, nếu trời không mưa thì xác suất để bán hết vé là $0{,}9$; còn nếu trời mưa thì xác suất để bán hết vé chỉ là $0{,}4$. Dự báo thời tiết cho thấy xác suất để trời mưa vào buổi biểu diễn là $0{,}75$. Nhà tổ chức sự kiện quan tâm đến xác suất để bán được hết vé là bao nhiêu.\\
			Gọi $A$ là biến cố \lq \lq Trời mưa\rq \rq~và $B$ là biến cố \lq \lq Bán hết vé\rq \rq~trong tình huống.
		\begin{listEX}
			\item Tính $\mathrm{P}(A), \mathrm{P}(\overline{A}), \mathrm{P}(B \mid A), \mathrm{P}(B \mid \overline{A})$.
			\item Tính xác suất để nhà tổ chức sự kiện bán hết vé.
		\end{listEX}
	}{	\includegraphics[width=8cm,height=6cm]{images/im2D5-2-1.png}}
	\loigiai{
		\begin{listEX}
			\item $\mathrm{P}(A)=0{,}75$; $\mathrm{P}(\overline{A})=1-\mathrm{P}(A)=0{,}25$; $\mathrm{P}(B\mid A)=0{,}4$; $\mathrm{P}(B\mid \overline{A})=0{,}9$.
			\item 	Ta có $\mathrm{P}(B)= \mathrm{P}(A) \cdot \mathrm{P}(B \mid A)+\mathrm{P}(\overline{A}) \cdot \mathrm{P}(B \mid \overline{A}) = 0{,}75\cdot 0{,}4 +0{,}25\cdot 0{,}9=0{,}525$.
	\end{listEX}}
\end{vd}
\begin{vd}%[2D5H2-2]
	Một hộp có $60$ viên bi màu xanh và $40$ viên bi màu đỏ; các viên bi có kích thước và khối lượng như nhau. Sau khi thống kê, người ta thấy: có $50\%$ số viên bi màu xanh có dán nhãn và $75\%$ số viên bi màu đỏ có dán nhãn; những viên bi còn lại không có dán nhãn.
	\begin{listEX}
		\item Chọn số thích hợp cho $\boxed{?}$ trong bảng (đơn vị: viên bi).
		\begin{center}
			\begin{tabular}{|c|C{3cm}|C{3cm}|}
				\hline
				\diaghead{Dán nhãn Màu bi}{\normalsize Màu bi}{\normalsize Dán nhãn} & Có dán nhãn & Không dán nhãn \\
				\hline
				Đỏ & $\boxed{?}$ & $\boxed{?}$ \\
				\hline
				Xanh & $\boxed{?}$ & $\boxed{?}$\\
				\hline
			\end{tabular}
		\end{center}
		\item Lấy ra ngẫu nhiên một viên bi trong hộp. Sử dụng công thức xác suất toàn phần, tính xác suất để viên bi được lấy ra có dán nhãn.
	\end{listEX}
	\loigiai{
		\begin{listEX}
			\item Số viên bi màu đỏ có dán nhãn là $75\%\cdot 40 = 30$ (viên bi).\\
			Số viên bi màu xanh có dãn nhẫn là $50\%\cdot 60 = 30$ (viên bi).\\
			\begin{center}
				\begin{tabular}{|c|C{2.75cm}|C{2.75cm}|}
					\hline
					\diaghead{Dán nhãn Màu bi}{\normalsize Dán nhãn}{\normalsize Màu bi} & Có dán nhãn & Không dán nhãn \\
					\hline
					Đỏ & $30$ & $10$ \\
					\hline
					Xanh & $30$ & $30$\\
					\hline
				\end{tabular}
			\end{center}
			Sau khi hoàn thiện bảng $3$ ta được bảng $4$ (đơn vị: viên bi).
			\item Xét hai biến cố sau
			\begin{itemize}
				\item $A$: \lq\lq  Viên bi được chọn ra có dãn nhãn\rq\rq.
				\item $B$: \lq\lq  Viên bi được chọn ra có màu đỏ\rq\rq.
			\end{itemize}
			Khi đó, ta có
		$$\mathrm{P}(B)=\dfrac{40}{100} = \dfrac{2}{5};
		\quad \mathrm{P}\left(\overline{B}\right)= 1 - \mathrm{P}(B)=1-\dfrac{2}{5}=\dfrac{3}{5};
		\quad \mathrm{P}(A|B) = \dfrac{30}{40}=\dfrac{3}{4};
		\quad \mathrm{P}\left(A|\overline{B}\right)=\dfrac{30}{60}=\dfrac{1}{2}.$$
			Áp dụng công thức tính xác suất toàn phần, ta có
			\[\mathrm{P}(A) = \mathrm{P}(B)\cdot\mathrm{P}\left(A|B\right) + \mathrm{P}\left(\overline{B}\right)\cdot\mathrm{P}\left(A|\overline{B}\right) = \dfrac{2}{3}\cdot \dfrac{3}{4} + \dfrac{3}{5}\cdot \dfrac{1}{2}=\dfrac{4}{5}.\]
			Vậy xác suất để viên bi được lấy ra có dãn nhãn bằng $\dfrac{4}{5}$.
		\end{listEX}
	}
\end{vd}
%Ví dụ 3
\begin{vd}%[2D5H2-2]
	Trong trò chơi hái hoa có thưởng của lớp 12A, cô giáo treo $10$ bông hoa trên cành cây, trong đó có $5$ bông hoa chưa phiếu có thưởng. Bạn Bình hái bông hoa đầu tiên, sau đó bạn An hái bông hoa thứ hai.
	\begin{listEX}
		\item Vẽ sơ đồ cây biểu thị tình huống trên.
		\item Từ đó, tính xác suất bạn An hái được bông hoa chứa phiếu có thưởng.
	\end{listEX}
	\loigiai{
		Xét hai biến cố
		\begin{itemize}
			\item $A$: \lq\lq  Bông hoa bạn An hái được chứa phiếu có thưởng\rq\rq.
			\item $B$: \lq\lq  Bông hoa bạn Bình hái được chứa phiếu có thưởng\rq\rq.
		\end{itemize}
		Khi đó, ta có
		$$\mathrm{P}(B)=\dfrac{5}{10}=\dfrac{1}{2}; 
		\quad \mathrm{P}\left(\overline{B}\right) = 1 - \mathrm{P}(B)=1-\dfrac{1}{2}=\dfrac{1}{2}; 
		\quad \mathrm{P}(A|B)=\dfrac{4}{9};
		\quad \mathrm{P}\left(A|\overline{B}\right)=\dfrac{5}{9}.$$
		\begin{listEX}
			\item Sơ đồ hình cây biểu thị tình huống đã cho là
			\begin{center}
				\begin{tikzpicture}[->,>=stealth,line join=round,line cap=round,font=\footnotesize,scale=1]
					\def\xmot{3}
					\def\xhai{8}
					\node (O) at (0,0){};
					\node (B) at (\xmot,1){$B$};
					\node (B1) at (\xmot,-1){$\overline{B}$};
					\node (BA) at (\xhai,2){$A$};
					\node (BA1) at (\xhai,0.3){$\overline{A}$};
					\node (B1A) at (\xhai,-0.3){$A$};
					\node (B1A1) at (\xhai,-1.75){$\overline{A}$};
					\foreach \x/\y/\p/\l in
					{
						O/B/above/$\mathrm{P}(B)=\dfrac{1}{2}$,
						B/BA/above/$\mathrm{P}(A|B)=\dfrac{4}{9}$,
						B/BA1//,
						O/B1/below/$\mathrm{P}\left(\overline{B}\right)=\dfrac{1}{2}$,
						B1/B1A/above/$\mathrm{P}\left(A|\overline{B}\right)=\dfrac{5}{9}$,
						B1/B1A1//
					}
					{
						\draw[->] (\x)--(\y)node[midway,\p,scale=0.8,sloped]{\l};
					}
					\node (mot) at (\xmot,4) {Bông hoa bạn Bình hái ra};
					\node (hai) at (\xhai,4) {Bông hoa bạn An hái ra};
					\draw[->,thick,orange] (mot)--(B);
					\draw[->,thick,orange] (hai)--(BA);
				\end{tikzpicture}
			\end{center}
			\item Áp dụng công thức tính xác suất toàn phần, ta có:
			\[\mathrm{P}(A) = \mathrm{P}(B)\cdot\mathrm{P}\left(A|B\right) + \mathrm{P}\left(\overline{B}\right)\cdot\mathrm{P}\left(A|\overline{B}\right) = \dfrac{1}{2}\cdot \dfrac{4}{9} + \dfrac{1}{2}\cdot\dfrac{5}{9}=\dfrac{1}{2}.\]
			Vậy xác suất bạn An hái được bông hoa chứa phiếu có thưởng bằng $\dfrac{1}{2}$.
		\end{listEX}
	}
\end{vd}
\begin{vd}%[2D5H2-2]
	Một hộp có $5$ quả cầu trắng và $10$ quả cầu đen cùng kích thước và khối lượng. Lấy ngẫu nhiên lần lượt hai quả cầu (không hoàn lại) từ hộp. Xác suất để lần thứ hai lấy được quả cầu trắng là
	\loigiai{
		Xét phép thử lấy ngẫu nhiên lần lượt hai quả cầu (không hoàn lại) từ hộp. Gọi:
		\begin{itemize}
			\item $A$ là biến cố \lq \lq Lần thứ hai lấy được quả cầu trắng\rq \rq~;
			\item $B$ là biến cố \lq \lq Lần thứ nhất lấy được quả cầu trắng\rq \rq~;
			\item$\overline{B}$ là biến cố \lq \lq Lần thứ nhất lấy được quả cầu đen\rq \rq~.
		\end{itemize}
		Ta có:
		$$\mathrm{P}(B)=\dfrac{5}{15}=\dfrac{1}{3};\quad \mathrm{P}(\overline{B})=\dfrac{10}{15}=\dfrac{2}{3}.$$
		Nếu lần thứ nhất lấy được quả cầu trắng thì trong hộp còn $4$ quả cầu trắng và $10$ quả cầu đen. Do đó $\mathrm{P}(A|B)=\dfrac{4}{14}=\dfrac{2}{7}$.\\
		Nếu lần thứ nhất lấy được quả cầu đen thì trong hộp còn $5$ quả cầu trắng và $9$ quả cầu đen. Do đó $\mathrm{P}(A|\overline{B})=\dfrac{5}{14}$.\\
		Áp dụng công thức xác suất toàn phần, ta có:
		$$\mathrm{P}(A) = \mathrm{P}(B)\cdot \mathrm{P}(A|B) + \mathrm{P}(\overline{B})\cdot \mathrm{P}(A|\overline{B}) =\dfrac{1}{3}\cdot \dfrac{2}{7}+\dfrac{2}{3}\cdot \dfrac{5}{14}=\dfrac{1}{3}.$$
		Vậy xác suất để lần thứ hai lấy được quả cầu trắng bằng $\dfrac{1}{3}$.
	}
\end{vd}
%----------------------------
\subsubsection{Bài tập áp dụng}
\begin{bt}
	Cho hai biến cố $A$, $B$ sao cho $\mathrm{P}(A) = 0{,}6$; $\mathrm{P}(B)=0{,}4$; $\mathrm{P}(B\mid\overline{A}) = 0{,}3$. Tính~$\mathrm{P}(B|A)$.
	\loigiai{
		Áp dụng công thức xác suất toàn phần ta có	$$\mathrm{P}(B)=\mathrm{P}(A)\cdot \mathrm{P}(B \mid A)+\mathrm{P}(\overline{A}) \cdot \mathrm{P}(B \mid \overline{A})
		\Leftrightarrow 0{,}4=0{,}6\cdot\mathrm{P}(B \mid A) +0{,}4 \cdot 0{,}3.$$
		Suy ra $\mathrm{P}(B|A)=\dfrac{7}{15} . $}
\end{bt}
\begin{bt}%[2D5H2-2]
	Một loại linh kiện do hai nhà máy số I, số II cùng sản xuất. Tỉ lệ phế phẩm của các nhà máy I, II lần lượt là $4\%$; $3\%$. Trong một lô linh kiện để lẫn lộn $80$ sản phẩm của nhà máy số I và $120$ sản phẩm của nhà máy số II. Một khách hàng lấy ngẫu nhiên một linh liện từ lô hàng đó.
	Tính xác suất để linh kiện được lấy ra là linh kiện tốt.
	\loigiai{
		Xét các biến cố
		\begin{itemize}
			\item $A$: \lq\lq  Linh kiện lấy ra là linh kiện tốt\rq\rq.
			\item $B$: \lq\lq  Linh kiện lấy ra là linh kiện từ nhà máy số I\rq\rq.
			\item $\overline{B}$: \lq\lq  Linh kiện lấy ra là linh kiện từ nhà máy số II\rq\rq.
		\end{itemize}
		Theo đề bài, ta có
		$$\mathrm{P}(A|B) =1 - 0{,}04 = 0{,}96;
		\quad\quad \mathrm{P}\left(A|\overline{B}\right) = 1-0{,}03 = 0{,}97;
		\quad\quad \mathrm{P}(B)=\dfrac{80}{200}=0{,}4;
		\quad\quad \mathrm{P}\left(\overline{B}\right) = \dfrac{120}{200}=0{,}6.
		$$
		Khi đó áp dụng công thức xác suất toàn phần, ta có
		\[\mathrm{P}(A) = \mathrm{P}(A|B)\cdot \mathrm{P}(B) + \mathrm{P}\left(A|\overline{B}\right)\cdot\mathrm{P}\left(\overline{B}\right)=0{,}96\cdot 0{,}4 + 0{,}97\cdot 0{,}6=0{,}966.\]
	}
\end{bt}
\begin{bt}%[2D5H2-2]
	Người ta khảo sát khả năng chơi nhạc cụ của một nhóm học sinh tại trường X. Nhóm này có $60\%$ học sinh là nam. Kết quả khảo sát cho thấy có $20\%$ học sinh nam và $15\%$ học sinh nữ biết chơi ít nhất một nhạc cụ. Chọn ngẫu nhiên một học sinh trong nhóm này. Tính xác suất để chọn được học sinh biết chơi ít nhất một nhạc cụ.
	\loigiai{
		Xét phép thử chọn ngẫu nhiên một học sinh trong nhóm.\\
		Gọi $A$ là biến cố \lq \lq Chọn được một học sinh biết chơi ít nhất một nhạc cụ\rq \rq~ và $B$, $\overline{B}$ lần lượt là các biến cố \lq \lq Chọn được một học sinh nam\rq \rq~ và \lq \lq Chọn được một học sinh nữ\rq \rq~.\\
		Theo đề bài:
		$$\mathrm{P}(B) = 60\% = 0{,}6;\quad \mathrm{P}(\overline{B}) = 1 - 0{,}6 = 0{,}4;$$
		$$\mathrm{P}(A|B) = 20\% = 0{,}2;\quad \mathrm{P}(A|\overline{B}) = 15\% = 0{,}15.$$
		Áp dụng công thức xác suất toàn phần, ta có:
		$$\mathrm{P}(A) = \mathrm{P}(B)\cdot \mathrm{P}(A|B) + \mathrm{P}(\overline{B})\cdot \mathrm{P}(A|\overline{B}) = 0{,}6\cdot 0{,}2 + 0{,}4\cdot 0{,}15 = 0{,}18.$$
		Vậy xác suất để chọn được một học sinh biết chơi nhạc cụ là $0{,}18$.
	}
\end{bt}
\begin{bt}%[2D5V2-2]
	Một doanh nghiệp có $45 \%$ nhân viên là nữ. Tỉ lệ nhân viên nữ và tỉ lệ nhân viên nam mua bảo hiểm nhân thọ lần lượt là $7 \%$ và $5 \%$. Gặp ngẫu nhiên một nhân viên của doanh nghiệp. Tính xác suất nhân viên đó có mua bảo hiểm nhân thọ.
	\loigiai{
	\begin{itemize}
		\item \textbf{Cách 1:} Giả sử doanh nghiệp có $100$ nhân viên, trong đó có $45$ nhân viên là nữ và $55$ nhân viên là nam. Tỉ lệ nhân viên nữ mua bảo hiểm nhân thọ là $7 \%$, tức là có $3{,}15$ nhân viên nữ mua bảo hiểm nhân thọ. Tỉ lệ nhân viên nam mua bảo hiểm nhân thọ là $5 \%$, tức là có $2{,}75$ nhân viên nam mua bảo hiểm nhân thọ. Tổng số nhân viên mua bảo hiểm nhân thọ là $3{,}15+2{,}75=5{,}9$.\\
		Xác suất để ngẫu nhiên chọn được một nhân viên mua bảo hiểm nhân thọ trong doanh nghiệp là $\dfrac{5{,}9}{100}=0{,}059$.
		\item \textbf{Cách 2:} Xét các biến cố
		\begin{itemize}
			\item $A$: \lq\lq  Nhân viên có mua bảo hiểm nhân thọ\rq\rq.
			\item $B$: \lq\lq  Nhân viên là nữ \rq\rq.
		\end{itemize}
		Do doanh nghiệp có $45\%$ nhân viên là nữ cho nên \[\mathrm{P}(B)=0{,}45 ~\text{và}~ \mathrm{P}(\overline{B})=0{,}55.\]
		Mặt khác tỉ lệ nhân viên nữ và tỉ lệ nhân viên nam mua bảo hiểm nhân thọ lần lượt là $7 \%$ và $5 \%$ cho nên ta có \[\mathrm{P}(A\mid B)=0{,}07~ \text{và}~ \mathrm{P}(A\mid \overline{B})=0{,}05.\]
		Xác suất nhân viên có mua bảo hiểm nhân thọ là
		\[ \mathrm{P}(A)=\mathrm{P}(B)\mathrm{P}(A\mid B)+\mathrm{P}(\overline{B})\mathrm{P}(A\mid \overline{B})=0{,}45\cdot 0{,}07
		+ 0{,}55\cdot 0{,}05 = 0{,}059.\]
	\end{itemize}
	}
\end{bt}
\begin{bt}%[2D5H2-2]%[2D5V2-2]%[2D5V2-3]
	Có hai đội thi đấu môn Bắn súng. Đội I có 5 vận động viên, đội II có 7 vận động viên. Xác suất đạt huy chương vàng của mỗi vận động viên đội I và đội II tương ứng là 0{,}65 và 0{,}55. Chọn ngẫu nhiên một vận động viên.
	Tính xác suất để vận động viên này đạt huy chương vàng;
	\loigiai{
		Xét biến cố $A$: \lq \lq Vận động viên này thuộc đội I\rq \rq~. Xét biến cố $B$: \lq \lq Vận động viên này đạt huy chương vàng\rq \rq~.
		Ta có $\mathrm{P}(B)=\mathrm{P}(A)\cdot \mathrm{P}(B \mid A)+\mathrm{P}(\overline{A}) \cdot \mathrm{P}(B \mid \overline{A})$.
		\begin{itemize}
			\item Tính $\mathrm{P}(A)$: Đây là xác suất để vận động viên đó thuộc đội I. Vậy $\mathrm{P}(A)=\dfrac{5}{12}$.
			\item Tính $\mathrm{P}(\overline{A})$: $\mathrm{P}(\overline{A})=1-\mathrm{P}(A)=\dfrac{7}{12}$.
			\item Tính $\mathrm{P}(B\mid A)$: Đây là xác suất để vận động viên thuộc đội I đạt huy chương vàng.\\ Vậy $\mathrm{P}(B\mid A)=0{,}65$.
			\item Tính $\mathrm{P}(B\mid \overline{A})$: Đây là xác suất để vận động viên thuộc đội I đạt huy chương vàng. \\ Vậy $\mathrm{P}(B\mid \overline{A})=0{,}55$.
		\end{itemize}
		Vậy $\mathrm{P}(B)=\mathrm{P}(A)\cdot \mathrm{P}(B \mid A)+\mathrm{P}(\overline{A}) \cdot \mathrm{P}(B \mid \overline{A})=\dfrac{5}{12}\cdot 0{,}65+\dfrac{7}{12}\cdot 0{,}55=\dfrac{71}{120}\approx 0{,}59$. \\
		Vậy xác suất để vận động viên này đạt huy chương vàng là khoảng $0{,}59$.
	}
\end{bt}
\begin{bt}%[2D5H2-2]
	Chuồng I có 5 con gà mái, 2 con gà trống. Chuồng II có 3 con gà mái, 5 con gà trống. Bác Mai bắt một con gà trong số đó theo cách sau: Bác tung một con xúc xắc cân đối, đồng chất. Nếu số chấm chia hết cho 3 thì bác chọn chuồng I, nếu số chấm không chia hết cho 3 thì bác chọn chuồng II. Sau đó, từ chuồng đã chọn bác bắt ngẫu nhiên một con gà. Tính xác suất để bác Mai bắt được con gà mái.
	\loigiai{
		Gọi $A$ là biến cố: \lq\lq  Bác Mai bắt được con gà mái\rq\rq.\\
		Gọi $B$ là biến cố: \lq\lq  tung con xúc xắc được số chấm chia hết cho $3$\rq\rq\, suy ra $\mathrm{P}(B)=\dfrac{2}{6}=\dfrac{1}{3}$.\\
		Gọi $\overline{B}$ là biến cố: \lq\lq  tung con xúc xắc được số chấm không chia hết cho $3$\rq\rq\, suy ra $\mathrm{P}(\overline{B})=\dfrac{4}{6}=\dfrac{2}{3}$.\\
		Ta có xác suất bắt được gà mái từ chuồng I là $\mathrm{P}\left(A\mid B\right)=\dfrac{5}{7}$.\\
		Ta có xác suất bắt được gà mái từ chuồng II là $\mathrm{P}\left(A\mid \overline{B}\right)=\dfrac{3}{8}$.\\
		Áp dụng công thức xác suất toàn phần ta có
		$$\mathrm{P}(A)=\mathrm{P}(B)\cdot \mathrm{P}(A\mid B)+\mathrm{P}(\overline{B})\cdot \mathrm{P}(A\mid \overline{B})=\dfrac{1}{3}\cdot \dfrac{5}{7}+\dfrac{2}{3}\cdot \dfrac{3}{8}=\dfrac{41}{84}.$$
		Vậy xác suất để bác Mai bắt được con gà mái là $\dfrac{41}{84}$.
	}
\end{bt}
\begin{bt}%[2D5H2-2]%[2D5H2-2]
	Tại nhà máy X sản xuất linh kiện điện tử tỉ lệ sản phẩm đạt tiêu chuẩn là $80 \%$. Trước khi xuất xưởng ra thị trường, các linh kiện điện tử đều phải qua khâu kiểm tra chất lượng để đóng dấu OTK. Vì sự kiểm tra không tuyệt đối hoàn hảo nên nếu một linh kiện điện tử đạt tiêu chuẩn thì nó có xác suất 0{,}99 được đóng dấu OTK; nếu một linh kiện điện tử không đạt tiêu chuẩn thì nó có xác suất 0{,}95 không được đóng dấu OTK. Chọn ngẫu nhiên một linh kiện điện tử của nhà máy X trên thị trường.
	\begin{listEX}
		\item Tính xác suất để linh kiện điện tử đó được đóng dấu OTK.
		\item Dùng sơ đồ hình cây, hãy mô tả cách tính xác suất để linh kiện điện tử được chọn không được đóng dấu OTK.
	\end{listEX}
	\loigiai{Gọi $A$ là biến cố: \lq \lq Linh kiện điện tử đó đạt tiêu chuẩn\rq \rq~. Gọi $B$ là biến cố: \lq \lq Linh kiện điện tử đó được đóng dấu OTK\rq \rq~.
		\begin{listEX}
			\item Ta có $\mathrm{P}(B)=\mathrm{P}(A)\cdot \mathrm{P}(B \mid A)+\mathrm{P}(\overline{A}) \cdot \mathrm{P}(B \mid \overline{A})$.
			\begin{itemize}
				\item Tính $\mathrm{P}(A)$: Đây là xác suất để linh kiện đó đạt tiêu chuẩn. Vậy $\mathrm{P}(A)=0{,}8$.
				\item Tính $\mathrm{P}(\overline{A})$: $\mathrm{P}(\overline{A})=1-\mathrm{P}(A)=0{,}2$.
				\item Tính $\mathrm{P}(B\mid A)$: Đây là xác suất để linh kiện điện tử đó được đóng dấu OTK với điều kiện nó đạt tiêu chuẩn. Vậy $\mathrm{P}(B\mid A)=0{,}99$.
				\item Tính $\mathrm{P}(B\mid \overline{A})$: Đây là xác suất để linh kiện điện tử đó được đóng dấu OTK với điều kiện nó không đạt tiêu chuẩn. Vậy $\mathrm{P}(B\mid \overline{A})=1-0{,}95=0{,}05$.
			\end{itemize}
			Vậy $\mathrm{P}(B)=\mathrm{P}(A)\cdot \mathrm{P}(B \mid A)+\mathrm{P}(\overline{A}) \cdot \mathrm{P}(B \mid \overline{A})=0{,}8\cdot 0{,}99+0{,}2\cdot 0{,}05=0{,}802$. \\
			Vậy xác suất để linh kiện điện tử đó được đóng dấu OTK là $0{,}802$.
			\item Ta có sơ đồ hình cây
			\begin{center}
				\begin{tikzpicture}[yscale=0.65, font=\footnotesize, line join=round, line cap=round, >=stealth]
					\draw (0,0)--(-2,-2.5) node [midway, shift=(140:4mm)] {$0{,}8$} --(-3,-5) node [midway, shift=(160:4mm)] {$0{,}99$};
					\draw (0,0)--(2,-2.5) node [midway, shift=(40:4mm)] {$0{,}2$} --(3,-5) node [midway, shift=(20:4mm)] {$0{,}95$};
					\draw (-2,-2.5)--(-1,-5) node [midway, shift=(20:4mm)] {$0{,}01$};
					\draw (2,-2.5)--(1,-5) node [midway, shift=(160:4mm)] {$0{,}05$};
					\draw[draw=none,fill=yellow] (0,0) circle [radius=7pt] node[shift=(90:6mm)] {Gốc $O$};
					\draw[draw=none,fill=cyan] (-2,-2.5) circle [radius=7pt] node[shift=(140:6mm)] {$A$};
					\draw[draw=none,fill=green] (2,-2.5) circle [radius=7pt] node[shift=(40:6mm)] {$\overline{A}$};
					\draw[draw=none,fill=magenta] (1,-5) circle [radius=7pt] node[shift=(-90:6mm)] {$B$};
					\draw[draw=none,fill=magenta!40!black!30] (-1,-5) circle [radius=7pt] node[shift=(-90:6mm)] {$\overline{B}$};
					\draw[draw=none,fill=magenta] (-3,-5) circle [radius=7pt] node[shift=(-90:6mm)] {$B$};
					\draw[draw=none,fill=magenta!40!black!30] (3,-5) circle [radius=7pt] node[shift=(-90:6mm)] {$\overline{B}$};
				\end{tikzpicture}
			\end{center}
			Có hai nhánh cây đi từ $O$ tới $\overline{B}$ là $OA\overline{B}$ và $O\overline{A}\cdot\overline{B}$. Vậy xác suất để linh kiện điện tử được chọn không được đóng dấu OTK là $$\mathrm{P}(\overline{B})=0{,}8\cdot 0{,}01+0{,}2\cdot 0{,}95=0{,}198.$$
		\end{listEX}
	}
\end{bt}
\begin{bt}%[2D5H2-2]
	Trong một cuộc khảo sát tình trạng công việc trên $900$ người đã có bằng tốt nghiệp trung học phổ thông ở một địa phương cho cả nam lẫn nữ, người ta thu được số liệu thống kê trong bảng sau.
	\begin{center}
		\begin{tabular}{|l|c|c|}
			\hline
			\diagbox{Giới tính}{Tình trạng} & Có việc làm & Thất nghiệp\\\hline
			Nam & $460$ & $40$\\\hline
			Nữ & $140$ & $260$\\\hline
		\end{tabular}
	\end{center}
	Chọn ngẫu nhiên một người trong nhóm này. Gọi $A$ là biến cố \lq \lq Người được chọn là nữ\rq \rq , $B$ là biến cố \lq \lq Người được chọn có việc làm\rq \rq .
	\begin{listEX}
		\item Vẽ lại sơ đồ hình cây sau đây và hoàn thành kết quả ở các ô \fbox{?}.
		\begin{center}
			\begin{tikzpicture}[line join = round, line cap = round, >=stealth, font=\footnotesize, yscale=0.7]
				\begin{scope}[every node/.style={draw, rounded corners=5pt}]
					\node (A) at (0,0){Chọn một người};
					\def \gocA{30}
					\def \kcA{4}
					\node (B1) at ($(A)+(\gocA:\kcA)$){$A$};
					\node (B2) at ($(A)+(-\gocA:\kcA)$){$\overline{A}$};
					\def \gocB{15}
					\def \kcB{4}
					\node (B11) at ($(B1)+(\gocB:\kcB)$){$B$};
					\node (B12) at ($(B1)+(-\gocB:\kcB)$){$\overline{B}$};
					\node (B21) at ($(B2)+(\gocB:\kcB)$){$B$};
					\node (B22) at ($(B2)+(-\gocB:\kcB)$){$\overline{B}$};
				\end{scope}
				\begin{scope}[every node/.style={midway,sloped},every path/.style={->}]
					\draw (A)--(B1) node[above]{$\mathrm{P}(A)=$\fbox{?}};
					\draw (A)--(B2) node[below]{$\mathrm{P}(\overline{A})=$\fbox{?}};
					\draw (B1)--(B11) node[above]{$\mathrm{P}(B|A)=$\fbox{?}};
					\draw (B1)--(B12) node[below]{$\mathrm{P}(\overline{B}|A)=$\fbox{?}};
					\draw (B2)--(B21) node[above]{$\mathrm{P}(B|\overline{A})=$\fbox{?}};
					\draw (B2)--(B22) node[below]{$\mathrm{P}(\overline{B}|\overline{A})=$\fbox{?}};
				\end{scope}
				\def \kcC{1.7}
				\foreach \i/\j in {B11/AB,B12/{A\overline{B}},B21/{\overline{A}B},B22/{\overline{A}\,\overline{B}}}% Tạo nội dung lặp
				{
					\node at ($(\i)+(\kcC,0)$)[]{$\j$};
					\node at ($(\i)+({2*\kcC},0)$)[]{\fbox{?}};
				}
				\node (B) at ($(B11)+(\kcC,0.7)$){\textbf{Kết quả}};
				\node (C) at ($(B)+(\kcC,0)$){\textbf{Xác suất}};
			\end{tikzpicture}\\
			$A \colon$ nữ; $\overline{A} \colon$ nam; $B \colon$ có việc; $\overline{B} \colon$ thất nghiệp.
		\end{center}
		\item Tính xác suất để chọn được một người có việc làm.
	\end{listEX}
	\loigiai{
		\begin{listEX}
			\item Theo đề bài xác suất để chọn được một người nữ là $\mathrm{P}(A)=\dfrac{4}{9}$, suy ra $\mathrm{P}(\overline{A})=\dfrac{5}{9}$.\\
			Xác suất chọn được người có việc làm nếu người đó là nữ $\mathrm{P}(B|A)=\dfrac{140}{400}=\dfrac{7}{20}$. Suy ra $\mathrm{P}(\overline{B}|A)=\dfrac{13}{20}$.\\
			Xác suất chọn được người có việc làm nếu người đó không là nữ $\mathrm{P}(B|\overline{A})=\dfrac{460}{500}=\dfrac{23}{25}$.\\
			Suy ra $\mathrm{P}(\overline{B}|\overline{A})=\dfrac{2}{25}.$\\
			\begin{center}
				\begin{tikzpicture}[line join = round, line cap = round, >=stealth, font=\footnotesize, scale=1]
					\begin{scope}[every node/.style={draw, rounded corners=5pt}]
						\node (A) at (0,0){Chọn một người};
						\def \gocA{30}
						\def \kcA{4}
						\node (B1) at ($(A)+(\gocA:\kcA)$){$A$};
						\node (B2) at ($(A)+(-\gocA:\kcA)$){$\overline{A}$};
						\def \gocB{15}
						\def \kcB{4}
						\node (B11) at ($(B1)+(\gocB:\kcB)$){$B$};
						\node (B12) at ($(B1)+(-\gocB:\kcB)$){$\overline{B}$};
						\node (B21) at ($(B2)+(\gocB:\kcB)$){$B$};
						\node (B22) at ($(B2)+(-\gocB:\kcB)$){$\overline{B}$};
					\end{scope}
					\begin{scope}[every node/.style={midway,sloped},every path/.style={->}]
						\draw (A)--(B1) node[above]{$\mathrm{P}(A)=$\fbox{$\dfrac{4}{9}$}};
						\draw (A)--(B2) node[below]{$\mathrm{P}(\overline{A})=$\fbox{$\dfrac{5}{9}$}};
						\draw (B1)--(B11) node[above]{$\mathrm{P}(B|A)=$\fbox{$\dfrac{7}{20}$}};
						\draw (B1)--(B12) node[below]{$\mathrm{P}(\overline{B}|A)=$\fbox{$\dfrac{13}{20}$}};
						\draw (B2)--(B21) node[above]{$\mathrm{P}(B|\overline{A})=$\fbox{$\dfrac{23}{25}$}};
						\draw (B2)--(B22) node[below]{$\mathrm{P}(\overline{B}|\overline{A})=$\fbox{$\dfrac{2}{25}$}};
					\end{scope}
					\def \kcC{1.7}
					\foreach \i/\j/\k in {B11/AB/{\dfrac{7}{45}},B12/{A\overline{B}}/{\dfrac{13}{45}},B21/{\overline{A}B}/{\dfrac{23}{45}},B22/{\overline{A}\,\overline{B}}/{\dfrac{2}{45}}}% Tạo nội dung lặp
					{
						\node at ($(\i)+(\kcC,0)$)[]{$\j$};
						\node at ($(\i)+({2*\kcC},0)$)[]{$\k$};
					}
					\node (B) at ($(B11)+(\kcC,0.7)$){\textbf{Kết quả}};
					\node (C) at ($(B)+(\kcC,0)$){\textbf{Xác suất}};
				\end{tikzpicture}
			\end{center}
			\item Xác suất để chọn được một người có việc làm $$\mathrm{P}(B)=\mathrm{P}(A)\mathrm{P}(B|A)+\mathrm{P}(\overline{A})\mathrm{P}(B|\overline{A})=\dfrac{4}{9}\cdot \dfrac{7}{20}+\dfrac{5}{9}\cdot \dfrac{23}{25}=\dfrac{2}{3}.$$
		\end{listEX}
	}
\end{bt}
\begin{bt}%[2D5V2-2]%[2D5V2-4]
	Hộp thứ nhất có $1$ viên bi xanh và $5$ viên bi đỏ. Hộp thứ hai có $3$ viên bi xanh và $5$ viên bi đỏ. Các viên bi có cùng kích thước và khối lượng. Lấy ra ngẫu nhiên đồng thời $2$ viên bi từ hộp thứ nhất chuyển sang hộp thứ hai. Sau đó lại lấy ra ngẫu nhiên $2$ viên bi từ hộp thứ hai.
	Tính xác suất để hai viên bi lấy ra từ hộp thứ hai là bi đỏ.
	\loigiai{
		Xét các biến cố
		\begin{itemize}
			\item $A$: \lq\lq  Hai viên bi lấy ra từ hộp thứ hai là bi đỏ\rq\rq.
			\item $B_1$: \lq\lq  Hai viên bi lấy ra từ hộp thứ nhất có cả màu xanh và màu đỏ\rq\rq.
			\item $B_2$: \lq\lq  Hai viên bi lấy ra từ hộp thứ nhất có màu đỏ\rq\rq.
		\end{itemize}
		Ta có
		\[\mathrm{P}(B_1) = \dfrac{\mathrm{C}^1_{5}}{\mathrm{C}^2_{6}}=\dfrac{1}{3};\quad
		\mathrm{P}(B_2) = \dfrac{\mathrm{C}^2_5}{\mathrm{C}^2_{6}}=\dfrac{2}{3};\quad
		\mathrm{P}(A\mid B_1) = \dfrac{\mathrm{C}^2_6}{\mathrm{C}^2_{10}}=\dfrac{1}{3};\quad
		\mathrm{P}(A\mid B_2) = \dfrac{\mathrm{C}^2_7}{\mathrm{C}^2_{10}}=\dfrac{7}{15}.
		\]
		Áp dụng công thức xác suất toàn phần, ta có
		\allowdisplaybreaks
		\begin{eqnarray*}
			\mathrm{P}(A)
			&=& \mathrm{P}(A|B_1)\cdot \mathrm{P}(B_1) + \mathrm{P}(A|B_2)\cdot\mathrm{P}(B_2)\\
			&=& \dfrac{1}{3}\cdot \dfrac{1}{3} + \dfrac{7}{15}\cdot \dfrac{2}{3}\\
			&=& \dfrac{19}{45}.
		\end{eqnarray*}
}\end{bt}
%Bài 2

%========================
\begin{dang}{Công thức Bayes tính xác suất}
	\begin{itemize}
		\item Giả sử $A$ và $B$ là hai biến cố ngẫu nhiên thoả mãn $\mathrm{P}(A) > 0$ và $0 < \mathrm{P}(B) < 1$. Khi đó
		$$\mathrm{P}(B|A)=\dfrac{\mathrm{P}(B)\mathrm{P}(A|B)}{\mathrm{P}(B)\mathrm{P}(A|B)+\mathrm{P}(\overline{B})\mathrm{P}(A|\overline{B})}$$ gọi là \textbf{\textit{công thức Bayes}}.
	\end{itemize}
	\begin{note}
		Với $\mathrm{P}(A) > 0$, công thức $\mathrm{P}(B|A)=\dfrac{\mathrm{P}(B)\mathrm{P}(A|B)}{\mathrm{P}(A)}$ cũng được gọi là công thức Bayes.
	\end{note}
\end{dang}
%----------------------------
\subsubsection{Ví dụ minh hoạ}
\begin{vd}%[2D5N2-3]
	Cho hai biến cố $A$, $B$ sao cho $\mathrm{P}(A) = 0{,}6$; $\mathrm{P}(B)=0{,}4$; $\mathrm{P}(A|B) = 0{,}3$. Tính $\mathrm{P}(B|A)$.
	\loigiai{
		Áp dụng công thức Bayes, ta có
		\[\mathrm{P}(B|A)=\dfrac{\mathrm{P}(B)\cdot \mathrm{P}(A|B)}{\mathrm{P}(A)}=\dfrac{0{,}4\cdot 0{,}3}{0{,}6}=0{,}2.\]
	}
\end{vd}
\begin{vd}%[2D5N2-3]
	Cho $\mathrm{P}(A)=\dfrac{2}{5}$; $\mathrm{P}\left( B\mid A\right)=\dfrac{1}{3}$; $\mathrm{P}\left(B\mid \overline{A}\right)=\dfrac{1}{4}$. Giá trị của $\mathrm{P}(A\mid B)$ là
	\loigiai{Áp dụng công thức Bayes ta có
		$$\mathrm{P}(A\mid B)=\dfrac{\mathrm{P}(A)\cdot \mathrm{P}(B\mid A)}{\mathrm{P}(A)\cdot \mathrm{P}(B\mid A)+\mathrm{P}\left( \overline{A}\right) \cdot \mathrm{P}\left( B\mid \overline{A}\right) }=\dfrac{\dfrac{2}{5}\cdot \dfrac{1}{3}}{\dfrac{2}{5}\cdot \dfrac{1}{3}+\dfrac{3}{5}\cdot \dfrac{1}{4}}=\dfrac{8}{17}.$$
	}
\end{vd}
\begin{vd}%[2D5H2-3]
	\immini{Cho sơ đồ hình cây như hình bên. Biết $\mathrm{P}(A)=0{,}6$. Tính $\mathrm{P}(A \mid B)$.}
	{\begin{tikzpicture}[xscale=.2,yscale=0.15,,>=stealth]
			\tikzstyle{block} = [rectangle, draw, fill=blue!10\text{,} rounded corners, text centered, text width = 10em, minimum height = 2em]
			\node (c1) {};
			\node (c2)[above right = 1.5cm of c1] {$A$};
			%			\node at (0.5,5){\fbox{$0\text{,}7$}};
			%			\node at (0.5,-5){\fbox{$0\text{,}3$}};
			\node (c3) [below right= 1.5cm of c1]{$\overline{A}$};
			\node at (12,11.5){$0\text{,}9$};
			\node (c4) at (21.5, 12){$B$};
			\node (c5) at (21.5, 2){$\overline{B}$};
			\node at (12,3){$0\text{,}1$};
			\node (c6) at (21.5, -4){$B$};
			\node at (12,-4){$0\text{,}3$};
			\node (c7) at (21.5, -14){$\overline{B}$};
			\node at (12,-13){$0\text{,}7$};
			\draw[->] (c1.east) -- (c2.west);
			\draw[->] (c1.east) -- (c3.west);
			\draw[->] (c2.east) -- (c4.west);
			\draw[->] (c2.east) -- (c5.west);
			\draw[->] (c3.east) -- (c6.west);
			\draw[->] (c3.east) -- (c7.west);
	\end{tikzpicture}}
	\loigiai{
		Theo công thức Bayes ta có\begin{eqnarray*}
			\mathrm{P}\left( {A|B} \right) &=& \dfrac{{\mathrm{P}\left( A \right) \mathrm{P}\left( {B|A} \right)}}{{P\left( B \right)}} \\
			&=& \dfrac{{\mathrm{P}\left( A \right)\mathrm{P}\left( {B|A} \right)}}{\mathrm{P}(A) \cdot P\mathrm{P}(B \mid A)+\mathrm{P}(\overline{A}) \cdot \mathrm{P}(B \mid \overline{A})} \\
			&=& \dfrac{0{,}6 \cdot 0{,}9}{0{,}6 \cdot 0{,}9+0{,}4 \cdot 0{,}3} = \dfrac{9}{11}.
		\end{eqnarray*}
	}
\end{vd}
\begin{vd}%[2D5H2-3]
	Một bệnh viện có hai phòng khám là phòng A và phòng B với khả năng lựa chọn của bệnh nhân là như nhau. Tỉ lệ bệnh nhân nam có ở phòng A và phòng B lần lượt là $60\%$ và $40\%$. Một người bệnh được chọn ngẫu nhiêu từ hai phòng khám và biết người này là nam, xác suất để người bệnh được chọn đến từ phòng A là
	\loigiai{Một người bệnh được chọn ngẫu nhiên từ hai phòng khám.\\
		Gọi $X$ là biến cố \lq \lq Người đó đến từ phòng khám A\rq \rq \, và $Y$, $\overline{Y}$ lần lượt là biến cố \lq \lq Người đó là nam\rq \rq \; và \lq \lq Người đó không là nam\rq \rq.\\
		Ta có sơ đồ hình cây sau
		\begin{center}
			\begin{tikzpicture}[>=stealth,xscale=0.8,yscale=0.5]
				%Khung 1
				\draw (-3.5,-1) rectangle (2.2,0);
				\draw (-0.8,-0.5) node{Bệnh nhân được chọn} ;
				%Mui ten 1,2
				\draw [->] (2.2,-0.5)--(3.8,1.6) node[pos=0.5,sloped,above]{$0{,}5$};
				\draw [->] (2.2,-0.5)--(3.8,-2.6) node[pos=0.5,sloped,below]{$0{,}5$};
				%Khung 2.1
				\draw (3.8,1.1) rectangle (5.1,2.1);
				\draw (8.9/2,1.6) node{$X$} ;
				%Khung 2.2
				\draw (3.8,-2.1) rectangle (5.1,-3.1);
				\draw (8.9/2,-2.6) node{$\overline{X}$} ;
				%Mui ten 3,4
				\draw [->] (5.1,1.6)--(6.5,2.6) node[pos=0.5,sloped,above]{$0{,}6$};
				\draw [->] (5.1,1.6)--(6.5,0.6) node[pos=0.5,sloped,below]{$0{,}4$};
				%Mui ten 5,6
				\draw [->] (5.1,-2.6)--(6.5,-1.6) node[pos=0.5,sloped,above]{$0{,}4$};
				\draw [->] (5.1,-2.6)--(6.5,-3.6) node[pos=0.5,sloped,below]{$0{,}6$};
				%Khung 3.1
				\draw (6.5,2.2) rectangle (7.7,3.2);
				\draw (7.1,5.4/2) node{$Y$} ;
				%Khung 3.2
				\draw (6.5,1.2) rectangle (7.7,0.2);
				\draw (7.1,1.4/2) node{$\overline{Y}$} ;
				%Khung 3.3
				\draw (6.5,-1.1) rectangle (7.7,-2.1);
				\draw (7.1,-3.2/2) node{$Y$} ;
				%Khung 3.3
				\draw (6.5,-2.9) rectangle (7.7,-3.9);
				\draw (7.1,-3.4) node{$\overline{Y}$} ;
				%Kết quả
				\draw (9.5,3.7) node{\textbf{Kết quả}};
				\draw (9.5,2.7) node{$XY$};
				\draw (9.5,0.7) node{$X \overline{Y}$};
				\draw (9.5,-1.6) node{$\overline{X}Y$};
				\draw (9.5,-3.4) node{$\overline{X} \,\,\overline{Y}$};
				%Xác suất
				\draw (12.5,3.7) node{\textbf{Xác suất}};
				\draw (12.5,2.7) node{$0{,}3$};
				\draw (12.5,0.7) node{$0{,}2$};
				\draw (12.5,-1.6) node{$0{,}2$};
				\draw (12.5,-3.4) node{$0{,}3$};
			\end{tikzpicture}
		\end{center}
		Theo công thức Bayes, ta có $$\mathrm{P}(X|Y)=\dfrac{\mathrm{P}(X)\mathrm{P}(Y|X)}{\mathrm{P}(X)\mathrm{P}(Y|X)+\mathrm{P}(\overline{X})\mathrm{P}(Y|\overline{X})}=\dfrac{0{,}3}{0{,}3+0{,}2}=0{,}6.$$
		Vậy với một người bệnh được chọn ngẫu nhiêu từ hai phòng khám và biết người này là nam, xác suất để người đó đến từ phòng A là $0{,}6$.}
\end{vd}
\begin{vd}%[2D5H2-3]
	Trong một kì thi tốt nghiệp trung học phổ thông, một tỉnh X có $80 \%$ học sinh lựa chọn tổ hợp A00 (gồm các môn Toán, Vật lí, Hoá học). Biết rằng, nếu một học sinh chọn tổ hợp A00 thì xác suất để học sinh đó đỗ đại học là 0{,}6; còn nếu một học sinh không chọn tổ hợp A00 thì xác suất để học sinh đó đỗ đại học là 0{,}7. Chọn ngẫu nhiên một học sinh của tỉnh X đã tốt nghiệp trung học phổ thông trong kì thi trên. Biết rằng học sinh này đã đỗ đại học. Tính xác suất để học sinh đó chọn tổ hợp A00. (làm tròn hai chữ số thập phân)
	\loigiai{Gọi $A$ là biến cố: ``Học sinh đó chọn tổ hợp A00''; $B$ là biến cố: ``Học sinh đó đỗ đại học''.\\
		Ta cần tính $\mathrm{P}(A \mid B)$. Theo công thức Bayes, ta cần biết: $\mathrm{P}(A), \mathrm{P}(\overline{A}), \mathrm{P}(B \mid A)$ và $\mathrm{P}(B \mid \overline{A})$.\\
		Ta có:
		\begin{itemize}
			\item $\mathrm{P}(A)=0{,}8; \mathrm{P}(\overline{A})=1-\mathrm{P}(A)=1-0{,}8=0{,}2$.\\
			\item $\mathrm{P}(B \mid A)$ là xác suất để một học sinh đỗ đại học với điều kiện học sinh đó chọn tổ hợp $A 00$.\\
			Suy ra $\mathrm{P}(B \mid A)=0{,}6$.
			\item $\mathrm{P}(B \mid \overline{A})$ là xác suất để một học sinh đỗ đại học với điều kiện học sinh đó không chọn tổ hợp $\mathrm{A} 00$. Suy ra $\mathrm{P}(B \mid \overline{A})=0{,}7$.
		\end{itemize}
		Thay vào công thức Bayes ta được:
		$$\mathrm{P}(A \mid B)=\frac{\mathrm{P}(A) \cdot \mathrm{P}(B \mid A)}{\mathrm{P}(A) \cdot \mathrm{P}(B \mid A)+\mathrm{P}(\overline{A}) \cdot \mathrm{P}(B \mid \overline{A})}=\frac{0{,}8 \cdot 0{,}6}{0{,}8 \cdot 0{,}6+0{,}2 \cdot 0{,}7} \approx 0{,}77.$$}
\end{vd}
\begin{vd}%[2D5H2-3]
	Kết quả khảo sát tại một xã cho thấy có $20 \%$ cư dân hút thuốc lá. Tỉ lệ cư dân thường xuyên gặp các vấn đề sức khoẻ về đường hô hấp trong số những người hút thuốc lá và không hút thuốc lá lần lượt là $70\%$, $15\%$. Giả sử ta gặp một cư dân của xã, gọi $A$ là biến cố \lq\lq  Người đó có hút thuốc lá\rq\rq\, và $B$ là biến cố \lq\lq  Người đó thường xuyên gặp các vấn đề sức khoẻ về đường hô hấp\rq\rq.
	\begin{listEX}
		\item Vẽ sơ đồ hình cây.
		\item Nếu ta gặp một cư dân của xã thì xác suất người đó thường xuyên gặp các vấn đề sức khoẻ về đường hô hấp là bao nhiêu?
		\item  Nếu ta gặp một cư dân của xã thường xuyên gặp các vấn đề sức khoẻ về đường hô hấp thì xác suất người đó có hút thuốc lá là bao nhiêu?
	\end{listEX}
	\loigiai{
		\begin{listEX}
			\item Ta có sơ đồ hình cây sau.
			\begin{center}
				\begin{tikzpicture}[scale=.3,>=stealth]
					%-------------
					\tikzstyle{block} = [rectangle, draw, fill=none, rounded corners, minimum height = 2em]
					%-------------
					\node[block] (c1) {Gặp một cư dân};
					\node[block] (c2) [above right = 2cm of c1]{$A$};
					\node[block] (c3) [ below right= 2cm of c1]{$\overline{A}$};
					\node[block] (c4) [above right = 1cm of c2]{$B$} ;
					\node[block] (c5) [below right = 1cm of c2]{$\overline{B}$};
					\node[block] (c6) [ above right =1cm of c3]{$B$};
					\node[block] (c7) [ below right = 1cm of c3]{$\overline{B}$};
					%--------------
					\draw
					(17.5,15) node[right] {\text{Kết quả}}
					(20,11.3) node[right] {$AB$}
					(20,2.5) node[right] {$A\overline{B}$}
					(20,-2.5) node[right] {$\overline{A}B$}
					(20,-11.5) node[right] {$\overline{A} \,\,\overline{B}$};
					%--------------
					\draw
					(25,15) node[right] {\text{Xác suất}}
					(25,11.3) node[right] {$0{,}14$}
					(25,2.5) node[right] {$0{,}06$}
					(25,-2.5) node[right] {$0{,}12$}
					(25,-11.5) node[right] {$0{,}68$};
					%------------
					\draw[->] (c1.east) --node[above left]{$0{,}2$} (c2.west);
					\draw[->] (c1.east) --node[below left]{$0{,}8$} (c3.west);
					\draw[->] (c2.east) --node[above left]{$0{,}7$} (c4.west);
					\draw[->] (c2.east) --node[below left]{$0{,}3$} (c5.west);
					\draw[->] (c3.east) --node[above left]{$0{,}15$} (c6.west);
					\draw[->] (c3.east) -- node[below left]{$0{,}85$} (c7.west);
				\end{tikzpicture}
			\end{center}
			\item Ta có $\mathrm{P}(B)=\mathrm{P}(A) \cdot \mathrm{P}(B | A)+\mathrm{P}(\overline{A}) \cdot \mathrm{P}\left(B | \overline{A}\right)=0{,}14+0{,}12=0{,}26$.\\
			Vậy nếu ta gặp một cư dân của xã thì xác suất người đó thường xuyên gặp các vấn đề sức khoẻ về đường hô hấp là $26\%$.
			\item Theo công thức Bayes, ta có $\mathrm{P}\left(A | B\right)=\dfrac{\mathrm{P}(A)\cdot \mathrm{P}\left(B| A\right)}{\mathrm{P}(B)}=\dfrac{0{,}14}{0{,}26} \approx 0{,}54$.\\
			Vậy nếu ta gặp một cư dân của xã thường xuyên gặp các vấn đề sức khoẻ về đường hô hấp thì xác suất người đó có hút thuốc lá là khoảng $54\%$.
		\end{listEX}
	}
\end{vd}
\begin{vd}%[2D5H2-3]
	Một nhà máy có hai phân xưởng I và II. Phân xưởng I sản xuất $40\%$ số sản phẩm
	và phân xưởng II sản xuất $60\%$ số sản phẩm. Tỉ lệ sản phẩm bị lỗi của phần xưởng I
	là $2\%$ và của phân xưởng II là $1\%$.
	\begin{listEX}
		\item  Kiểm tra ngẫu nhiên 1 sản phẩm của nhà máy và tính xác suất để sản phẩm đó bị lỗi.
		\item Biết rằng sản phẩm được chọn bị lỗi. Hỏi xác suất sản phẩm đó do phân xưởng nào
		sản xuất cao hơn?
	\end{listEX}
	\loigiai{\begin{listEX}
			\item  Gọi $A$ là biến cố “Sản phẩm bị lỗi” và $B$ là biến cố “Sản phẩm lấy ra do phân xưởng I
			sản xuất”.\\
			Do phân xưởng I sản xuất $40\%$ số sản phẩm và phân xưởng II sản xuất $60\%$ số sản phẩm nên
			$$\mathrm{P}(B)=0{,}4 \text{ và } \mathrm{P}(\overline{B})=1-0{,}4=0{,}6.$$
			Do tỉ lệ sản phẩm bị lỗi của phân xưởng I là $2\%$ và của phân xưởng II là $1\%$ nên
			$$\mathrm{P}(A|B)=0{,}02 \text{ và } \mathrm{P}(A|\overline{B})=0{,}01.$$
			Xác suất để sản phẩm lấy ra bị lỗi là
			$$\mathrm{P}(A)=\mathrm{P}(B)\mathrm{P}(A|B)+\mathrm{P}(\overline{B})\mathrm{P}(A|\overline{B})=0{,}4\cdot0{,}02+0{,}6 \cdot 0{,}01=0{,}014.$$
			\item Nếu sản phẩm lấy ra bị lỗi thì xác suất sản phẩm đó do phân xưởng I sản xuất là
			$$\mathrm{P}(B|A)=\dfrac{\mathrm{P}(B)\mathrm{P}(A|B)}{\mathrm{P}(A)}=\dfrac{0{,}4 \cdot 0{,}02}{0{,}014}=\dfrac{4}{7}.$$
			Nếu sản phẩm lấy ra bị lỗi thì xác suất sản phẩm đó do phân xưởng II sản xuất là
			$$\mathrm{P}(\overline{B}|A)=1-\mathrm{P}(B|A)=\dfrac{3}{7}.$$
			Vậy nếu sản phẩm lấy ra bị lỗi thì xác suất sản phẩm đó do phân xưởng I sản xuất cao hơn
			xác suất sản phẩm đó do phân xưởng II sản xuất.
	\end{listEX}}
\end{vd}
\begin{vd}%[2D5H2-3]
	Giả sử có một loại bệnh mà tỉ lệ người mắc bệnh là $0{,}1\%$. Giả sử có một loại xét nghiệm, mà ai mắc bệnh khi xét nghiệm cũng có phản ứng dương tính, nhưng tỉ lệ phản ứng dương tính giả là $5\%$ (tức là trong số những người không bị bệnh có $5\%$ số người xét nghiệm lại có phản ứng dương tính).
	\begin{listEX}
		\item Vẽ sơ đồ cây biểu thị tình huống trên.
		\item Khi một người xét nghiệm có phản ứng dương tính thì khả năng mắc bệnh của người đó là bao nhiêu phần trăm (làm tròn kết quả đến hàng phần trăm).
	\end{listEX}
	\loigiai{
		\begin{listEX}
			\item Xét hai biến cố
			\begin{itemize}
				\item $K$: \lq\lq  Người được chọn ra không mắc bệnh\rq\rq.
				\item $D$: \lq\lq  Người được chọn ra có phản ứng dương tính\rq\rq.
			\end{itemize}
			Do tỉ lệ người mắc bệnh là là $0,1\%=0{,}001$ nên $\mathrm{P}(K)=1-0{,}001 = 0{,}999$.\\
			Trong số những người không mắc bệnh có $5\%$ số người có phản ứng dương tính nên \break $\mathrm{P}(D|K) = 5\% = 0{,}05$. Vì ai mắc bệnh khi xét nghiệm cũng có phản ứng dương tính nên $\mathrm{P}\left(\overline{K}\right)=1$.\\
			Ta có sơ đồ cây biểu thị tình huống đã cho
			\begin{center}
				\begin{tikzpicture}[->,>=stealth,line join=round,line cap=round,font=\footnotesize,scale=1]
					\def\xmot{4}
					\def\xhai{8}
					\node (O) at (0,0){};
					\node (B) at (\xmot,1){$K$};
					\node (B1) at (\xmot,-1){$\overline{K}$};
					\node (BA) at (\xhai,2){$D$};
					\node (BA1) at (\xhai,0.3){$\overline{D}$};
					\node (B1A) at (\xhai,-0.3){$D$};
					\node (B1A1) at (\xhai,-1.75){$\overline{D}$};
					\foreach \x/\y/\p/\l in
					{
						O/B/above/$\mathrm{P}(K)=0{,}999$,
						B/BA/above/$\mathrm{P}(D|K)=0{,}05$,
						B/BA1//,
						O/B1/below/$\mathrm{P}\left(\overline{K}\right)=0{,}001$,
						B1/B1A/above/$\mathrm{P}\left(D|\overline{K}\right)=1$,
						B1/B1A1//
					}
					{
						\draw[->] (\x)--(\y)node[midway,\p,scale=0.8,sloped]{\l};
					}
				\end{tikzpicture}
			\end{center}
			\item Ta thấy rằng khả năng mắc bệnh của một người xét nghiệm có phản ứng dương tính chính là $\mathrm{P}\left(\overline{K}|D\right)$. Áp dụng công thức Bayes, ta có:
			\[\mathrm{P}\left(\overline{K}|D\right)=\dfrac{\mathrm{P}\left(\overline{K}\right)\cdot \mathrm{P}\left(D|\overline{K}\right)}{\mathrm{P}\left(\overline{K}\right)\cdot \mathrm{P}\left(D|\overline{K}\right) + \mathrm{P}(K)\cdot\mathrm{P}(D|K)}=\dfrac{0{,}001\cdot 1}{0{,}001\cdot 1 + 0{,}999 \cdot 0{,}05}\approx 1{,}96\%.\]
			Vậy xác suất mắc bệnh của một người xét nghiệm có phản ứng dương tính là $1{,}96\%$.
		\end{listEX}
	}
\end{vd}

\begin{vd}%[2D5H2-3]
	\immini{Một loại xét nghiệm nhanh SARS-CoV-2 cho
		kết quả dương tính với $76,2\%$ các ca thực sự
		nhiễm virus và kết quả âm tính với $99,1\%$ các
		ca thực sự không nhiễm virus (nguồn: https://
		tapchiyhocvietnam.vn/index.php/vmj/article/
		view/2124/1921). Giả sử tỉ lệ người nhiễm virus
		SARS-CoV-2 trong một cộng đồng là $1\%$.\\
		Một người làm xét nghiệm và nhận được kết quả dương tính.
		Tính xác suất người đó thực sự nhiễm virus (kết quả làm tròn đến hàng phần nghìn).
	}{\includegraphics[scale=0.5]{images/12-SGK-CTST-6-2-1.png}}
	\loigiai{
		Gọi $A$ là biến cố “Người làm xét nghiệm có kết quả dương tính” và $B$ là biến cố
		“Người làm xét nghiệm thực sự nhiễm virus”.\\
		Do xét nghiệm cho kết quả dương tính với $76,2\%$ các ca thực sự nhiễm virus nên $\mathrm{P}(A|B)=0{,}762$.\\
		Do xét nghiệm cho kết quả âm tính với $99,1\%$ các ca thực sự không nhiễm virus nên
		$\mathrm{P}(\overline{A}|\overline{B})=0{,}991$. Suy ra $$\mathrm{P}(A|\overline{B})=1-0{,}991=0{,}009.$$
		Do tỉ lệ người nhiễm virus trong cộng đồng là $1\%$ nên $\mathrm{P}(B)=0,01$ và $\mathrm{P}(\overline{B}) = 0,99$.
		Áp dụng công thức xác suất toàn phần, ta có xác suất người làm xét nghiệm có kết quả
		dương tính là $$\mathrm{P}(A)=\mathrm{P}(B)\mathrm{P}(A|B)+\mathrm{P}(\overline{B})\mathrm{P}(A|\overline{B})=0{,}01\cdot0{,}762+0{,}99\cdot0{,}009=0{,}01653.$$
		Xác suất một người thực sự nhiễm virus khi người đó có kết quả xét nghiệm dương tính
		là $\mathrm{P}(B|A)$. Ta có $$\mathrm{P}(B|A)=\dfrac{\mathrm{P}(B)\mathrm{P}(A|B)}{\mathrm{P}(A)}=\dfrac{0{,}01\cdot0{,}762}{0{,}01653}\approx0{,}461.$$}
\end{vd}
%----------------------------
\subsubsection{Bài tập áp dụng}
\begin{bt}%[2D5H2-3]
	Trong một kì sát hạch lái xe có $65 \%$ thí sinh nam. Biết rằng $80 \%$ thí sinh nam và $70\% $ thí sinh nữ đỗ kì sát hạch này.
	\begin{listEX}
		\item Tính tỉ lệ thí sinh đỗ kì sát hạch này.
		\item Chọn ngẫu nhiên một thí sinh đã đỗ kì sát hạch. Tính xác suất thí sinh đó là nữ.
	\end{listEX}
	\loigiai{
		\begin{listEX}
			\item Xét các biến cố sau
			\begin{itemize}
				\item $D$: \lq\lq  Thí sinh đỗ kì sát hạch\rq\rq
				\item $M$: \lq\lq  Thí sinh là nam giới\rq\rq\,
				\item $\overline{M}$: \lq\lq  Thí sinh là nữ giới\rq\rq\,
				\item $D|M$: \lq\lq  Thí sinh nam đỗ kì sát hạch\rq\rq\,
				\item $D|\overline{M}$: \lq\lq  Thí sinh nữ đỗ kì sát hạch\rq\rq\,.
			\end{itemize}
			Theo đề bài ta có
			\begin{center}
				$\mathrm{P}(M) =0{,}65$;  $\mathrm{P}(D|M) =0{,}8$; $\mathrm{P}(\overline{M}) =0{,}35$; $\mathrm{P}(D|\overline{M}) =0{,}7$.
			\end{center}
			Theo công thức xác suất toàn phần, ta có
			\begin{align*}
				\mathrm{P}(D) &= \mathrm{P}(M) \cdot \mathrm{P}(D|M) + \mathrm{P}(\overline{M}) \cdot \mathrm{P}(D|\overline{M}).
			\end{align*}
			Thay vào giá trị đã cho
			\begin{align*}
				\mathrm{P}(D) &= 0{,}65 \cdot 0{,}8 + 0{,}35 \cdot 0{,}7 \\
				&= 0{,}765.
			\end{align*}
			\item Xác suất một thí sinh đỗ là nữ\\
			Để tính xác suất này, ta sử dụng công thức Bayes
			\begin{align*}
				\mathrm{P}(\overline{M}|D) &= \dfrac{\mathrm{P}(D|\overline{M}) \cdot \mathrm{P}(\overline{M})}{\mathrm{P}(D)}.
			\end{align*}
			Thay vào giá trị đã cho và kết quả tính được ở trên
			\begin{align*}
				\mathrm{P}(\overline{M}|D) &= \dfrac{0{,}7 \cdot 0{,}35}{0{,}765} \\
				&\approx 0{,}321.
			\end{align*}
			Vậy, xác suất một thí sinh đỗ kì sát hạch là nữ là khoảng $0{,}321$, hay chấp nhận được là khoảng $32{,}1\%$.
		\end{listEX}
	}
\end{bt}
\begin{bt}%[2D5H2-3]
	Bạn Nam tham gia một gian hàng trò chơi dân gian trong hội xuân của trường. Trò chơi có hai lượt chơi. Xác suất để Nam thắng ở lượt chơi thứ nhất là $0{,}6$. Nếu Nam thắng ở lượt chơi thứ nhất thì xác suất Nam thắng ở lượt chơi thứ hai là $0{,}8 $. Ngược lại, nếu Nam thua ở lượt chơi thứ nhất thì xác suất Nam thắng ở lượt chơi thứ hai là $0{,}3$.
	\begin{listEX}
		\item Vẽ sơ đồ hình cây mô tả các khả năng xảy ra và xác suất tương ứng khi Nam tham gia trò chơi này.
		\item 	Biết Nam đã thắng ở lượt chơi thứ hai, tính xác suất Nam thắng ở lượt chơi thứ nhất.
	\end{listEX}
	\loigiai{
		\begin{listEX}
			\item Sơ đồ hình cây
			\begin{center}
				\begin{tikzpicture}[scale=.2,>=stealth]
					%-------------
					\tikzstyle{block} = [rectangle, draw, fill=cyan!20, rounded corners, text centered, text width = 10em, minimum height = 2em]
					%-------------
					\node (c1) [block] {Trò chơi};
					\node (c2) [block, above right = 3cm of c1]{Nam thắng lượt chơi thứ nhất};
					\node (c3) [block, below right= 3cm of c1]{Nam thua lượt chơi thứ nhất };
					\node (c4) [block,above right = 1.5cm of c2]{Nam thắng lượt chơi thứ hai};
					\node (c5) [block,below right = 1.5cm of c2]{Nam thua lượt chơi thứ hai};
					\node (c6) [block, above right =1.5cm of c3]{Nam thắng lượt chơi thứ hai};
					\node (c7) [block, below right = 1.5cm of c3]{Nam thua lượt chơi thứ hai};
					%--------------
					\draw[->] (c1.east) -- (c2.west);
					\draw[->] (c1.east) -- (c3.west);
					\draw[->] (c2.east) -- (c4.west);
					\draw[->] (c2.east) -- (c5.west);
					\draw[->] (c3.east) -- (c6.west);
					\draw[->] (c3.east) -- (c7.west);
					\draw ($(c1.east)!.5!(c2.west)$) node [below right=-.1]{\color{red}$\mathrm{P}(A) =0{,}6$};
					\draw ($(c1.east)!.5!(c3.west)$) node [above right=-.1]{\color{red}$\mathrm{P}(\overline{A}) =0{,}4$};
					\draw ($(c2.east)!.5!(c4.west)$) node [below right=-.1]{\color{red}$\mathrm{P}(B|A) =0{,}8$};
					\draw ($(c2.east)!.5!(c5.west)$) node [above right=-.1]{\color{red}$\mathrm{P}(\overline{B}|A) =0{,}2$};
					\draw ($(c3.east)!.5!(c6.west)$) node [below right=-.1]{\color{red}$\mathrm{P}(B|\overline{A}) =0{,}3$};
					\draw ($(c3.east)!.5!(c7.west)$) node [above right=-.1]{\color{red}$\mathrm{P}(\overline{B}|\overline{A}) =0{,}7$};
				\end{tikzpicture}
			\end{center}
			\item
			Công thức Bayes cho sự kiện $A$ và $B$ là $\mathrm{P}(A | B)=\dfrac{\mathrm{P}(B | A) \cdot \mathrm{P}(A)}{\mathrm{P}(B)}	$.\\
			Trong đó
			\begin{itemize}
				\item $A$ là Nam thắng ở lượt chơi thứ nhất.
				\item $B$ là Nam thắng ở lượt chơi thứ hai.
				\item $\mathrm{P}(A)$: Xác suất Nam thắng ở lượt chơi thứ nhất $0{,}6$.
				\item $\mathrm{P}(B | A)$: Xác suất Nam thắng ở lượt chơi thứ hai khi đã thắng ở lượt chơi thứ nhất $0{,}8$.
				\item $\mathrm{P}(B | \overline{A})$: Xác suất Nam thắng ở lượt chơi thứ hai khi đã thua ở lượt chơi thứ nhất $0.3$.
			\end{itemize}
			Công thức Bayes:
			$$
			\begin{aligned}[t]
				\mathrm{P}(A | B)&=\dfrac{\mathrm{P}(B | A) \cdot \mathrm{P}(A)}{\mathrm{P}(B | A) \cdot \mathrm{P}(A)+\mathrm{P}(B | \overline{A}) \cdot \mathrm{P}(\overline{A})} \\
				& =\dfrac{0{,}8 \cdot 0{,}6}{0{,}8 \cdot 0{,}6+0{,}3 \cdot 0{,}4} \\
				& = \dfrac{0{,}48}{0{,}48+0{,}12} \\
				& = \dfrac{0{,}48}{0{,}6} \\
				& = 0{,}8.
			\end{aligned}
			$$
			Vậy, xác suất Nam thắng ở lượt chơi thứ nhất khi đã thắng ở lượt chơi thứ hai là khoảng $0{,}8$ hoặc $80 \%$.
		\end{listEX}
	}
\end{bt}

\begin{bt}%[2D5H2-3]
	Năm $2001$, Cộng đồng châu Âu có làm một đợt kiểm tra rất rộng rãi các con bò để phát hiện những con bị bệnh bò điên. Không có xét nghiệm nào cho kết quả chính xác $100 \%$. Một loại xét nghiệm, mà ở đây ta gọi là xét nghiệm $A$ cho kết quả như sau: khi con bò bị bệnh bò điên thì xác suất để có phản ứng dương tính trong xét nghiệm A là $70 \%$ còn khi con bò không bị bệnh thì xác suất để có phản ứng dương tính trong xét nghiệm $A$ là $10 \%$. Biết rằng tỉ lệ bò bị mắc bệnh bò điên ở Hà Lan là $13$ con trên $1~000~000$ con \textit{(Nguồn: F. M. Dekking et al., Amodern introduction to probability and statistics Understanding why and how, Springer, $2005$)}. Hỏi khi một con bò ở Hà Lan có phản ứng dương tính với xét nghiệm $A$ thì xác suất để nó bị mắc bệnh bò điên là bao nhiêu?
	\loigiai{
		Xét hai biến cố\\
		$N$: \lq\lq  Con bò được chọn bị nhiễm bệnh\rq\rq.\\
		$D$: \lq\lq  Con bò được chọn có phản ứng dương tính\rq\rq.\\
		Khi đó, ta có\\
		\[\mathrm{P}(N)=\dfrac{13}{1 000 000}=0{,}000013; \qquad \mathrm{P}(\overline{N})=1-\mathrm{P}(N)=0{,}999987;\]
		\[\mathrm{P}(D|N)=70\%=0{,}7; \qquad \mathrm{P}(D|\overline{N})=10\%=0{,}1.\]
		Áp dụng công thức Bayes, ta có\\
		$\mathrm{P}(N|D)=\dfrac{\mathrm{P}(D|N) \cdot \mathrm{P}(N)}{\mathrm{P}(N) \cdot \mathrm{P}(D|N)+\mathrm{P}(\overline{N})\mathrm{P}(D|\overline{N})}=\dfrac{0{,}7 \cdot 0{,}000013}{0{,}7 \cdot 0{,}000013+0{,}1 \cdot 0{,}999987}\approx 0{,}009\%$.
	}
\end{bt}
\begin{bt}%[2D5H2-3]
	\immini{Khi phát hiện một vật thể bay, xác suất một hệ thống radar phát
		cảnh báo là $0,9$ nếu vật thể bay đó là mục tiêu thật và là $0,05$
		nếu đó là mục tiêu giả. Có $99\%$ các vật thể bay là mục tiêu giả.
		Biết rằng hệ thống radar đang phát cảnh báo khi phát hiện một
		vật thể bay. Tính xác suất vật thể đó là mục tiêu thật.}{\includegraphics[scale=0.7]{images/12-SGK-CTST-6-2-4.png}}	\loigiai{Gọi $ A $ là biến cố \lq\lq  Hệ thống radar phát cảnh báo\rq\rq \, và $ B $ là biến cố \lq\lq  Vật thể bay là mục tiêu thật\rq\rq.\\
		Do xác suất một hệ thống radar cảnh báo nếu vật thể bay là mục tiêu thật là $ 0,9 $ nên $ \mathrm{P}(A|B)=0,9 $. \\
		Do xác suất một hệ thống radar cảnh báo nếu vật thể bay là mục tiêu giả là $ 0,05 $ nên $ \mathrm{P}(A|\overline{B})=0,05 $. \\
		Do có $ 99\% $ các vật thể bay là mục tiêu giả nên $ \mathrm{P}(\overline{B})=0,99 $ và $ \mathrm{P}(B)= 0,01$.\\
		Áp dụng công thức xác suất toàn phần, ta có xác suất để hệ thống radar phát cảnh báo là
		$$ \mathrm{P}(A)=\mathrm{P}(B)\mathrm{P}(A|B)+\mathrm{P}(\overline{B})\mathrm{P}(A|\overline{B})=0,01 \cdot 0,9+0,99 \cdot 0,05= 0,0585$$
		Xác suất vật bay là mục tiêu thật khi hệ thống radar đang phát cảnh báo là $ \mathrm{P}(B|A) $. Ta có
		$$ \mathrm{P}(B|A)=\dfrac{\mathrm{P}(B)\mathrm{P}(A|B)}{\mathrm{P}(A)}=\dfrac{0,01 \cdot 0,9}{0,0585}=\dfrac{2}{13}. $$}
\end{bt}
\begin{bt}%[2D5H2-3]
	Trong một kho rượu có 30\% là rượu loại I. Chọn ngẫu nhiên một chai rượu đưa cho ông Tùng, một người sành rượu, để nếm thử. Biết rằng, một chai rượu loại I có xác suất 0{,}9 để ông Tùng xác nhận là loại I; một chai rượu không phải loại I có xác suất 0{,}95 để ông Tùng xác nhận đây không phải là loại I. Sau khi nếm, ông Tùng xác nhận đây là rượu loại I. Tính xác suất để chai rượu đúng là rượu loại I.
	\loigiai{Xét biến cố $A$: ``Chai rượu đó là chai rượu loại I''. Xét biến cố $B$: ``Ông Tùng xác nhận chai rượu đó là rượu loại I''.\\
		Ta cần tính $\mathrm{P}(A\mid B)$.
		Áp dụng công thức Bayes
		$$\mathrm{P}(A\mid B) = \dfrac{\mathrm{P}(A)\cdot \mathrm{P}(B\mid A)}{\mathrm{P}(A)\cdot \mathrm{P}(B\mid A) + \mathrm{P}(\overline{A})\cdot \mathrm{P}(B \mid \overline{A})}.$$
		\begin{itemize}
			\item Tính $P\mathrm{P}(A)$: Đây là chai rượu đó là chai rượu loại I. Vậy $\mathrm{P}(A)=0{,}3$.
			\item Tính $\mathrm{P}(\overline{A})$: $\mathrm{P}(\overline{A})=1-\mathrm{P}(A)=0{,}7$.
			\item Tính $\mathrm{P}(B \mid A)$: Đây là xác suất ông Tùng xác nhận đúng một chai rượu loại I là một chai rượu loại I. Vậy $\mathrm{P}(B \mid A)=0{,}9$.
			\item Tính $\mathrm{P}(B \mid \overline{A})$: Đây là xác suất ông Tùng xác nhận sai một chai rượu không phải loại I là một chai rượu loại I. Vậy $\mathrm{P}(B \mid \overline{A})=1-0{,}95=0{,}05$.
		\end{itemize}
		Vậy ta có
		$$\mathrm{P}(A\mid B) = \dfrac{\mathrm{P}(A)\cdot \mathrm{P}(B\mid A)}{\mathrm{P}(A)\cdot \mathrm{P}(B\mid A) + \mathrm{P}(\overline{A})\cdot \mathrm{P}(B \mid \overline{A})}=\dfrac{0{,}3\cdot 0{,}9}{0{,}3\cdot 0{,}9+0{,}7\cdot 0{,}05}=\dfrac{270}{277}\approx 0{,}8852.$$
		Vậy xác suất để chai rượu mà ông Tùng xác nhận là rượu loại I đúng là rượu loại I là khoảng $0{,}8852$.}
\end{bt}
\begin{bt}%[2D5H2-3]
	Một loại linh kiện do hai nhà máy số I, số II cùng sản xuất. Tỉ lệ phế phẩm của các nhà máy I, II lần lượt là $4\%$; $3\%$. Trong một lô linh kiện để lẫn lộn $80$ sản phẩm của nhà máy số I và $120$ sản phẩm của nhà máy số II. Một khách hàng lấy ngẫu nhiên một linh liện từ lô hàng đó.
	\begin{listEX}
		\item Tính xác suất để linh kiện được lấy ra là linh kiện tốt.
		\item Giả sử linh kiện được lấy ra là linh kiện phế phẩm. Xác suất linh kiện đó do nhà máy nào sản xuất là cao nhất?
	\end{listEX}
	\loigiai{
		\begin{listEX}
			\item Xét các biến cố
			\begin{itemize}
				\item $A$: \lq\lq  Linh kiện lấy ra là linh kiện tốt\rq\rq.
				\item $B_1$: \lq\lq  Linh kiện lấy ra là linh kiện từ nhà máy số I\rq\rq.
				\item $B_2$: \lq\lq  Linh kiện lấy ra là linh kiện từ nhà máy số II\rq\rq.
			\end{itemize}
			Theo đề bài, ta có
			\begin{listEX}[2]
				\item[] $\mathrm{P}(A|B_1) =1 - 0{,}04 = 0{,}96$.
				\item[] $\mathrm{P}\left(A|B_2\right) = 1-0{,}03 = 0{,}97$.
				\item[] $\mathrm{P}(B_1)=\dfrac{80}{200}=0{,}4$.
				\item[] $\mathrm{P}\left(B_2\right) = \dfrac{120}{200}=0{,}6$.
			\end{listEX}
			Khi đó áp dụng công thức xác suất toàn phần, ta có
			\[\mathrm{P}(A) = \mathrm{P}(A|B_1)\cdot \mathrm{P}(B_1) + \mathrm{P}\left(A|B_2\right)\cdot\mathrm{P}\left(B_2\right)=0{,}96\cdot 0{,}4 + 0{,}97\cdot 0{,}6=0{,}966.\]
			\item Ta có $\mathrm{P}\left(\overline{A}\right)=1-\mathrm{P}(A) = 0{,}034$.\\
			Áp dụng công thức Bayes, ta có
			\begin{itemize}
				\item $\mathrm{P}\left(B_1|\overline{A}\right)=\dfrac{\mathrm{P}\left(\overline{A}|B_1\right)\cdot \mathrm{P}(B_1)}{\mathrm{P}\left(\overline{A}\right)} = \dfrac{0{,}04\cdot 0{,}4}{0{,}034} = \dfrac{8}{17}\approx 0{,}048$.
				\item $\mathrm{P}\left(B_2|\overline{A}\right)=\dfrac{\mathrm{P}\left(\overline{A}|B_2\right)\cdot \mathrm{P}(B_2)}{\mathrm{P}\left(\overline{A}\right)} = \dfrac{0{,}03\cdot 0{,}6}{0{,}034} = \dfrac{8}{167}\approx 0{,}054$.
			\end{itemize}
			Vậy với điều kiện linh kiện lấy ra là linh kiện phế phẩm thì xác suất linh kiện đó do nhà máy II sản xuất là cao nhất.
		\end{listEX}
	}
\end{bt}
\begin{bt}%[2D5V2-3]
	Một nghiên cứu đã chỉ ra rằng tỉ lệ người bị lao phổi trong nhóm $X$ những người mắc phải hội chứng suy giảm miễn dịch $ {H}$ là $15{,}2 \%$. Kết quả nghiên cứu về một số triệu chứng lâm sàng như có ho trong vòng bốn tuần, hoặc có bị sốt trong vòng bốn tuần, hoặc ra mồ hôi ban đêm từ ba tuần trở lên của nhóm $X$ cho thấy.
	\begin{itemize}
		\item Trong số những người mắc bệnh lao phổi, có $93{,}2 \%$ trường hợp có ít nhất một triệu chứng;
		\item Trong số những người không mắc bệnh lao phổi, có $35{,}8 \%$ trường hợp không có triệu chứng nào.
	\end{itemize}
	Nếu bác sĩ gặp một bệnh nhân thuộc nhóm $X$ và bệnh nhân đó có ít nhất một triệu chứng trên thì xác suất bệnh nhân này mắc bệnh lao phổi là bao nhiêu?
	\loigiai{
		Gọi $A$ là biến cố \lq\lq  Người bị lao phổi\lq\lq .\\
		$\overline{A}$ là biến cố \lq\lq  Người không mắc lao phổi\lq\lq .\\
		$B$ là biến cố \lq\lq  Những người có ít nhất một triệu chứng\rq\rq.\\
		$\overline{B}$ là biến cố \lq\lq  Những người không có triệu chứng\rq\rq.\\
		Ta có $\mathrm{P}(A)=0{,}152$. \\
		Khi đó, xác suất những người không mắc lao phổi là $$\mathrm{P}\left(\overline{A}\right)=1-0{,}152=0{,}848.$$
		Ta có xác suất những người có ít một triệu chứng trong những người mắc lao phổi là $$\mathrm{P}\left(B|A\right)=0{,}932.$$
		Khi đó, xác suất những người không có triệu chứng trong những người mắc lao phổi là  $$\mathrm{P}\left(\overline{B}|A\right)=1-0{,}932=0{,}068.$$
		Mặt khác, ta có xác suất những người không có triệu chứng  $$\mathrm{P}\left(\overline{B}|\overline{A}\right)=0{,}358.$$
		Khi đó, xác suất những người có ít nhất một triệu chứng trong những người không mắc bệnh lao phổi là $$\mathrm{P}\left(B|\overline{A}\right)=1-0{,}358=0{,}642.$$
		Ta có $\mathrm{P}(B)=\mathrm{P}\left(B|A\right)\cdot \mathrm{P}(A)+\mathrm{P}\left(B|\overline{A}\right)\cdot \mathrm{P}\left(\overline{A}\right)=0{,}932\cdot 0{,}152+0{,}642\cdot 0{,}848=0{,}68608$.\\
		Theo công thức Bayes, ta có $\mathrm{P}(A|B)=\dfrac{\mathrm{P}(A)\cdot\mathrm{P}\left(B|A\right)}{\mathrm{P}(B)}=\dfrac{0{,}152\cdot 0{,}932}{0{,}68608}\approx 0{,}2065$.
	}
\end{bt}
\begin{bt}%[2D5V2-3]
	Có hai hộp đựng các viên bi cùng kích thước và khối lượng. Hộp thứ nhất chứa $5$ viên bi đỏ và $5$ viên bi xanh, hộp thứ hai chứa $6$ viên bi đỏ và $4$ viên bi xanh. Lấy ngẫu nhiên một viên bi từ hộp thứ nhất chuyển sang hộp thứ hai, sau đó lấy ra ngẫu nhiên một viên bi từ hộp thứ hai (giả sử viên bi được lấy ra từ hộp thứ hai là bi đỏ). Tính xác suất viên bi đỏ đó là của hộp thứ nhất.
	\loigiai{
		Gọi $A$ là biến cố \lq\lq  Viên bi được lấy ra từ hộp thứ hai là bi đỏ\rq\rq.\\
		$B$ là biến cố \lq\lq  Viên bi được lấy ra từ hộp thứ nhất chuyển sang hộp thứ hai là viên bi đỏ\rq\rq.\\
		$\overline{B}$ là biến cố \lq\lq  Viên bi được lấy ra từ hộp thứ nhất chuyển sang hộp thứ hai là bi xanh\rq\rq.\\
		Ta có $\mathrm{P}(B)=\dfrac{5}{10}=\dfrac{1}{2}$; $\mathrm{P}(\overline{B})=\dfrac{5}{10}=\dfrac{1}{2}$.\\
		Nếu viên bi được lấy ra từ hộp thứ nhất chuyển sang hộp thứ hai là bi đỏ thì sau khi chuyển, hộp thứ hai có $7$ bi đỏ và $4$ bi xanh. Do đó $\mathrm{P}(A\mid B) =\dfrac{7}{11}$.\\
		Nếu viên bi được lấy ra từ hộp thứ nhất chuyển sang hộp thứ hai là bi xanh thì sau khi chuyển, hộp thứ hai có $6$ bi đỏ và $5$ bi xanh. Do đó $\mathrm{P}(A\mid \overline{B})=\dfrac{6}{11}$.\\
		Áp dụng công thức xác suất toàn phần ta có
		$$\mathrm{P})(A) = \mathrm{P}(B) \cdot \mathrm{P}(A\mid B) + \mathrm{P}(\overline{B}) \cdot \mathrm{P}(A\mid \overline{B}) = \dfrac{1}{2} \cdot \dfrac{7}{11} + \dfrac{1}{2}\cdot \dfrac{6}{11} = \dfrac{13}{22}.$$
		Xác suất viên bi được lấy ra từ hộp thứ nhất chuyển sang hộp thứ hai là viên bi đỏ khi biết viên bi lấy ra từ hộp thứ hai là bi đỏ là
		$$\mathrm{P}(B\mid A) =\dfrac{\mathrm{P}(B) \cdot \mathrm{P}(A\mid B)}{\mathrm{P}(A)} = \dfrac{\dfrac{1}{2} \cdot \dfrac{7}{11}}{\dfrac{13}{22}} = \dfrac{7}{13}.$$
		Vì khi viên bi lấy sang hộp thứ II là bi đỏ thì hộp II có $7$ bi đỏ, do vậy xác suất viên bi đỏ lấy ra là của hộp thứ nhất là $\dfrac{1}{7}\cdot\dfrac{7}{13}=\dfrac{1}{13}$.
	}
\end{bt}
\begin{bt}%[2D5V2-3]
	Có hai chiếc hộp, hộp I có $5$ viên bi màu trắng và $5$ viên bi màu đen; hộp II có $6$ viên bi màu trắng và $4$ viên bi màu đen. Các viên bi có cùng kích thước và khối lượng. Lấy ngẫu nhiên đồng thời hai viên bi từ hộp I bỏ sang hộp II. Sau đó lấy ngẫu nhiên một viên bi từ hộp II.
	\begin{listEX}
		\item Tính xác suất để viên bi được lấy ra là viên bi màu trắng.
		\item Giả sử viên bi được lấy ra là viên bi màu trắng. Tính xác suất viên bi màu trắng đó thuộc hộp I.
	\end{listEX}
	\loigiai{
		\begin{listEX}
			\item Xét các biến cố
			\begin{itemize}
				\item $A$: \lq\lq  Viên bi lấy ra là viên màu trắng\rq\rq.
				\item $B_1$: \lq\lq  2 viên bi lấy ra từ hộp I có màu trắng\rq\rq.
				\item $B_2$: \lq\lq  2 viên bi lấy ra từ hộp I có màu đen\rq\rq.
				\item $B_3$: \lq\lq  2 viên bi lấy ra từ hộp I có cả hai màu đen trắng\rq\rq.
			\end{itemize}
			Ta có
			\[\mathrm{P}(B_1) = \dfrac{\mathrm{C}^2_{5}}{\mathrm{C}^2_{10}}=\dfrac{2}{9};\quad
			\mathrm{P}(B_2) = \dfrac{\mathrm{C}^2_5}{\mathrm{C}^2_{10}}=\dfrac{2}{9};\quad
			\mathrm{P}(B_3) = \dfrac{\mathrm{C}^1_5\cdot\mathrm{C}^1_5}{\mathrm{C}^2_{10}}=\dfrac{5}{9}.
			\]
			Áp dụng công thức xác suất toàn phần, ta có
			\allowdisplaybreaks
			\begin{eqnarray*}
				\mathrm{P}(A)
				&=& \mathrm{P}(A|B_1)\cdot \mathrm{P}(B_1) + \mathrm{P}(A|B_2)\cdot\mathrm{P}(B_2) + \mathrm{P}(A|B_3)\cdot\mathrm{P}(B_3)\\
				&=& \dfrac{8}{12}\cdot \dfrac{2}{9} + \dfrac{6}{12}\cdot \dfrac{2}{9} + \dfrac{7}{12}\cdot \dfrac{5}{9}\\
				&=& \dfrac{7}{12}.
			\end{eqnarray*}
			\item Gọi $C$ là biến cố \lq\lq  Viên bi được lấy ra là viên màu trắng thuộc hộp I\rq\rq.\\
			Ta cần tính $\mathrm{P}(C|A)$. \\
			Khi $B_1$ xảy ra, hộp II có $2$ viên bi trắng là từ hộp I nên $P(C|B_1)=\dfrac{2}{12}$.\\
			Khi $B_2$ xảy ra, hộp II không có viên bi trắng nào là từ hộp I nên $P(C|B_2)=0$.\\
			Khi $B_3$ xảy ra, hộp II có $1$ viên bi trắng là từ hộp I nên $P(C|B_3)=\dfrac{1}{12}$.\\
			Do đó, ta có
			\allowdisplaybreaks
			\begin{eqnarray*}
				\mathrm{P}(C) &=& \mathrm{P}(C|B_1)\cdot \mathrm{P}(B_1) + \mathrm{P}(C|B_2)\cdot \mathrm{P}(B_2) + \mathrm{P}(C|B_3)\cdot \mathrm{P}(B_3)\\
				&=& \dfrac{2}{12}\cdot \dfrac{2}{9} + 0 + \dfrac{1}{12}\cdot \dfrac{5}{9}\\
				&=& \dfrac{1}{12}.
			\end{eqnarray*}
			Đồng thời vì khi $C$ xảy ra thì $A$ cũng xảy ra nên $\mathrm{P}(A|C)=1$.\\
			Áp dụng công thức Bayes, ta có
			$$\mathrm{P}(C|A) = \dfrac{\mathrm{P}(C)\cdot \mathrm{P}(A|C)}{\mathrm{P}(A)} 
				= \dfrac{\dfrac{1}{12}\cdot 1}{\dfrac{7}{12}} = \dfrac{1}{7}.$$
		\end{listEX}
	}
\end{bt}
%===================

% %%%%%%%
\setcounter{bt} {0}
\subsection{Bài tập tự luận}
%%==========Bài 1
\begin{bt}%%%%[2D5H2-2]
	Trong quân sự, một máy bay chiến đấu của đối phương có thể xuất hiện ở vị trí X với xác suất 0{,}55. Nếu máy bay đó không xuất hiện ở vị trí X thì nó xuât hiện ở vị trí Y. Để phòng thủ, các bệ phóng tên lửa được bố trí tại các vị trí X và Y. Khi máy bay đối phương xuất hiện ở vị trí X hoặc Y thì tên lửa sẽ được phóng để hạ máy bay đó.\\
	Xét phương án tác chiến sau: Nếu máy bay xuất hiện tại X thì bắn 2 quả tên lửa và nếu máy bay xuất hiện tại Y thì bắn 1 quả tên lửa.\\
	Biết rằng, xác suất bắn trúng máy bay của mỗi quả tên lửa là 0{,}8 và các bệ phóng tên lửa hoạt động độc lập. Máy bay bị bắn hạ nếu nó trúng ít nhất 1 quả tên lửa. Tính xác suất bắn hạ máy bay đối phương trong phương án tác chiến nêu trên.
	\loigiai{Xét biến cố $A$: ``Máy bay xuất hiện ở vị trí X'', điều đó có nghĩa là biến cố $\overline{A}$: ``Máy bay xuất hiện ở vị trí Y''. Xét biến cố $B$: ``Máy bay bị bắn hạ''. \\
	Ta có $P(B)=P(A)\cdot P(B \mid A)+P(\overline{A}) \cdot P(B \mid \overline{A})$.
	\begin{itemize}
	\item Tính $P(A)$, $P(\overline{A})$: $P(A)=0{,}55$ và $P(\overline{A})=0{,}35$.
	\item Tính $P(B\mid A)$: Đây là xác suất để máy bay bị bắn hạ tại vị trí X. Máy bay bị bắn hạ nếu nó trúng ít nhất một 1 quả tên lửa (trong 2 quả tên lửa đối với máy bay ở vị trí X), mà xác suất bắn trúng máy bay của mỗi quả tên lửa là $0{,}8$, vậy $P(B\mid A)=1-\left(1-0{,}8\right)\left(1-0{,}8\right)=0{,}96$.
	\item Tính $P(B\mid \overline{A})$: Đây là xác suất để máy bay bị bắn hạ tại vị trí Y. Máy bay bị bắn hạ nếu nó trúng ít nhất một 1 quả tên lửa (trong 1 quả tên lửa đối với máy bay ở vị trí Y), mà xác suất bắn trúng máy bay của mỗi quả tên lửa là $0{,}8$, vậy $P(B\mid \overline{A})=0{,}8$.
	\end{itemize}
	Vậy $P(B)=P(A)\cdot P(B \mid A)+P(\overline{A}) \cdot P(B \mid \overline{A})=0{,}55\cdot 0{,}96+0{,}35\cdot 0{,}8=0{,}808$. \\
	Vậy xác suất để máy bay bị bắn hạ là $0{,}808$.
	}
\end{bt}
%%==========Bài 2
\begin{bt}%%%%[2D5H2-2]
	Có hai chuồng thỏ. Chuồng I có 5 con thỏ đen và 10 con thỏ trắng. Chuồng II có 7 con thỏ đen và 3 con thỏ trắng. Trước tiên, từ chuồng II lấy ra ngẫu nhiên 1 con thỏ rồi cho vào chuồng I. Sau đó, từ chuồng I lấy ra ngẫu nhiên 1 con thỏ. Tính xác suất để con thỏ được lấy ra là con thỏ trắng.
	\loigiai{Xét biến cố $A$: ``Con thỏ được lấy ra từ chuồng II để cho vào chuồng I là con thỏ trắng''. 	Xét biến cố $B$: ``Con thỏ được lấy ra từ chuồng I là con thỏ trắng''. \\	
	Ta có $P(B)=P(A)\cdot P(B \mid A)+P(\overline{A}) \cdot P(B \mid \overline{A})$.
	\begin{itemize}
	\item Tính $P(A)$: Đây là xác suất để lấy ra ngẫu nhiên 1 con thỏ trắng từ chuồng II rồi cho vào chuồng I. Có $n\left(\Omega\right)=\mathrm{C}^1_{10}$, $n\left(A\right)=\mathrm{C}^1_3$. Vậy $P(A)=\dfrac{3}{10}$.
	\item Tính $P(\overline{A})$: $P(\overline{A})=1-P(A)=\dfrac{7}{10}$.
	\item Tính $P(B\mid A)$: Đây là xác suất để lấy ra ngẫu nhiên 1 con thỏ trắng từ chuồng I với điều kiện đã chọn ra 1 con thỏ trắng từ chuồng II rồi cho vào chuồng I, tức là có 5 con thỏ đen và 11 con thỏ trắng ở trong chuồng I. Tương tự như trên ta có $P(B\mid A)=\dfrac{11}{16}$.
	\item Tính $P(B\mid \overline{A})$: Đây là để lấy ra ngẫu nhiên 1 con thỏ trắng từ chuồng I với điều kiện đã chọn ra 1 con thỏ đen từ chuồng II rồi cho vào chuồng I, tức là có 6 con thỏ đen và 10 con thỏ trắng ở trong chuồng I. Tương tự như trên ta có $P(B\mid \overline{A})=\dfrac{10}{16}$.
	\end{itemize}
	Vậy $P(B)=P(A)\cdot P(B \mid A)+P(\overline{A}) \cdot P(B \mid \overline{A})=\dfrac{3}{10}\cdot \dfrac{11}{16}+\dfrac{7}{10}\cdot \dfrac{10}{16}=\dfrac{103}{160}=0{,}64375$. \\
	Vậy xác suất để con thỏ được lấy ra là con thỏ trắng là $0{,}64375$.
	}
\end{bt}
%%==========Bài 3
\begin{bt}%%%%[2D5V2-3]
	Một bộ lọc được sử dụng để chặn thư rác trong các tài khoản thư điện tử. Tuy nhiên, vì bộ lọc không tuyệt đối hoàn hảo nên một thư rác bị chặn với xác suất 0{,}95 và một thư đúng (không phải là thư rác) bị chặn với xác suất 0{,}01. Thống kê cho thấy tỉ lệ thư rác là $3 \%$.
	\begin{listEX}
	\item Chọn ngẫu nhiên một thư bị chặn. Tính xác suất để đó là thư rác.
	\item Chọn ngẫu nhiên một thư không bị chặn. Tính xác suất để đó là thư đúng.
	\item Trong số các thư bị chặn, có bao nhiêu phần trăm là thư đúng? Trong số các thư không bị chặn, có bao nhiêu phần trăm là thư rác?
	\end{listEX}
	\loigiai{Xét biến cố $A$: ``Thư đó là thư rác''. Xét biến cố $B$: ``Thư đó bị chặn''.
	\begin{listEX}
	\item Ta cần tính $P(A\mid B)$.
	Áp dụng công thức Bayes
	$$P(A\mid B) = \dfrac{P(A)\cdot P(B\mid A)}{P(A)\cdot P(B\mid A) + P(\overline{A})\cdot P(B \mid \overline{A})}.$$
	\begin{itemize}
	\item Tính $P(A)$: Đây là xác suất thư đó là thư rác. Vậy $P(A)=0{,}03$.
	\item Tính $P(\overline{A})$: $P(\overline{A})=1-P(A)=0{,}97$.
	\item Tính $P(B \mid A)$: Đây là xác suất thư đó là thư rác bị chặn. Vậy $P(B \mid A)=0{,}95$.
	\item Tính $P(B \mid \overline{A})$: Đây là xác suất thư đó là thư đúng bị chặn. Vậy $P(B \mid \overline{A})=0{,}01$.
	\end{itemize}
	Vậy $P(A\mid B) = \dfrac{P(A)\cdot P(B\mid A)}{P(A)\cdot P(B\mid A) + P(\overline{A})\cdot P(B \mid \overline{A})}=\dfrac{0{,}03\cdot 0{,}95}{0{,}03\cdot 0{,}95+0{,}97\cdot 0{,}01}=\dfrac{285}{382}\approx 0{,}746$.\\
	Vậy xác suất để chọn ngẫu nhiên một thư bị chặn mà thư đó là thư rác là khoảng $0{,}746$.
	\item Ta cần tính $P(\overline{A}\mid \overline{B})$.
	Áp dụng công thức Bayes
	$$P(\overline{A}\mid \overline{B}) = \dfrac{P(\overline{A})\cdot P(\overline{B}\mid \overline{A})}{P(\overline{A})\cdot P(\overline{B}\mid \overline{A}) + P(A)\cdot P(\overline{B} \mid A)}.$$
	\begin{itemize}
	\item Tính $P(\overline{B} \mid A)$: Ta có $(\overline{B} \mid A)=1-P(B \mid A)=0{,}05$.
	\item Tính $P(\overline{B} \mid \overline{A})$: Ta có $P(\overline{B} \mid \overline{A})=1-P(B \mid \overline{A})=0{,}99$.
	\end{itemize}
	Vậy $P(\overline{A}\mid \overline{B}) = \dfrac{P(\overline{A})\cdot P(\overline{B}\mid \overline{A})}{P(\overline{A})\cdot P(\overline{B}\mid \overline{A}) + P(A)\cdot P(\overline{B} \mid A)}=\dfrac{0{,}97\cdot 0{,}99}{0{,}97\cdot 0{,}99+0{,}03\cdot 0{,}05}=\dfrac{3201}{3206}\approx 0{,}998$.\\
	Vậy xác suất để chọn ngẫu nhiên một thư không bị chặn mà thư đó là thư đúng là khoảng $0{,}998$.
	\item Trong số các thư bị chặn, có $74{,}6 \%$ là thư rác, có $25{,}4 \%$ là thư đúng. \\
	Trong số các thư không bị chặn, có $0{,}2 \%$ là thư rác, có $99{,}8 \%$ là thư đúng.
	\end{listEX}
	}
\end{bt}
%%==========Bài 4
\begin{bt}%%%%[2D5N2-2]
	Trong một trường học, tỉ lệ học sinh nữ là $52\%$. Tỉ lệ học sinh nữ và tỉ lệ học sinh nam
	tham gia câu lạc bộ nghệ thuật lần lượt là $18\%$ và $15\%$. Gặp ngẫu nhiên 1 học sinh của trường. 
	\begin{listEX}
	\item Tính xác suất học sinh đó có tham gia câu lạc bộ nghệ thuật. 
	\item Biết rằng học sinh có tham gia câu lạc bộ nghệ thuật. Tính xác suất học sinh đó là nam.
	\end{listEX}
	\loigiai{
	\begin{listEX}
	\item Gọi $A$ là biến cố "Học sinh đó là nữ" và $B$ là biến cố "Học sinh đó tham gia câu lạc bộ nghệ thuật".\\
	Do tỉ lệ học sinh nữ là $52\%$ nên
	\begin{center}
	$P(A)=0{,}52$ và $P(\overline{A})=1-0{,}52=0{,}48$.
	\end{center}
	Do tỉ lệ học sinh nữ và tỉ lệ học sinh nam tham gia câu lạc bộ nghệ thuật lần lượt là $18\%$ và $15\%$ nên
	\begin{center}
	$P(B|A)=0{,}18$ và $P(B|\overline{A})=0{,}15$.
	\end{center}
	Xác suất để học sinh đó có tham gia câu lạc bộ nghệ thuật là
	$$P(B)=P(A)P(B|A)+P(\overline{A})P(B|\overline{A})=0{,}52\cdot0{,}18+0{,}48\cdot0{,}15=0{,}1656.$$
	\item Do học sinh có tham gia câu lạc bộ nghệ thuật nên xác suất học sinh đó là nam là
	$$P(\overline{A}|B)=\dfrac{P(\overline{A})P(B|\overline{A})}{P(B)}=\dfrac{0{,}48\cdot0{,}15}{0{,}1656}=\dfrac{10}{23}.$$
	\end{listEX}}
\end{bt}
%%==========Bài 5
\begin{bt}%%%%[2D5Y2-3]
	Tỉ lệ người dân đã tiêm vắc xin phòng bệnh A ở một địa phương là $65\%$. Trong số những 
	người đã tiêm phòng, tỉ lệ mắc bệnh A là $5\%$ còn trong số những người chưa tiêm, tỉ lệ 
	mắc bệnh A là $17\%$. Gặp ngẫu nhiên một người ở địa phương đó.
	\begin{listEX}
	\item Tính xác suất người đó mắc bệnh A.
	\item Biết rằng người đó mắc bệnh A. Tính xác suất người đó không tiêm vắc xin phòng bệnh A.
	\end{listEX}
	\loigiai{Gọi $H_1$ là biến cố "Gặp được người đã tiêm vắc xin phòng bệnh $A$", $H_2$ là biến cố "Gặp được người chưa tiêm vắc xin phòng bệnh $A$", $A$ là biến cố" người đó mắc bệnh A".
	\begin{listEX}
	\item Theo công thức Bayes, ta có:
	$$\mathrm{P}(A)=\mathrm{P}(H_1).\mathrm{P}(A|H_1)+\mathrm{P}(H_2).\mathrm{P}(A|H_2)$$
	$$ \Leftrightarrow \mathrm{P}(A)= 0,65.0,05+0,35.0,17=0,092. $$
	\item $$\mathrm{P}(H_2|A)=\dfrac{\mathrm{P}(AH_2)}{\mathrm{P}(A)}=\dfrac{\mathrm
	P(H_2)\mathrm{P}(A|H_2)}{\mathrm{P}(A)}$$
	$$\Leftrightarrow \mathrm{P}(H_2|A)= \dfrac{0,35.0,17}{0,092}=\dfrac{119}{184}\approx 0,65.
	$$
	\end{listEX}}
\end{bt}
%%==========Bài 6
\begin{bt}%%%%[2D5N2-3]
	Ở một khu rừng nọ có 7 chú lùn, trong đó có 4 chú luôn nói thật, 3 chú còn lại nói thật với 
	xác suất 0,5. Bạn Tuyết gặp ngẫu nhiên một chú lùn. Gọi $A$ là biến cố “Chú lùn đó luôn
	nói thật” và $B$ là biến cố “Chú lùn đó tự nhận mình luôn nói thật”.
	\begin{listEX}
	\item Tính xác suất của các biến cố $A$ và $B$.
	\item Biết rằng chú lùn mà bạn Tuyết gặp tự nhận mình là người luôn nói thật. Tính xác suất để 
	chú lùn đó luôn nói thật.
	\end{listEX}
	\loigiai{Gọi $C$ là biến cố "bạn Tuyết gặp được chú lùn nói thật với xác suất $0,5$".
	\begin{listEX}
	\item Ta có
	$\mathrm{P}(A)=\dfrac{4}{7}$
	Theo công thức Bayes, ta có
	$\mathrm{P}(B)=\mathrm{P}(A).\mathrm{P}(B|A)+\mathrm{P}(C).\mathrm{P}(B|C)$
	$\Leftrightarrow \mathrm{P}(B)=\dfrac{4}{7}.1+\dfrac{3}{7}.0,5=\dfrac{11}{14}\approx 0,79 $
	\item $$\mathrm{P}(A|B)=\dfrac{\mathrm{P}(AB)}{\mathrm{P(B)}}=\dfrac{\mathrm{P}(A).\mathrm{P}(B|A)}{\mathrm{P}(B)}=\dfrac{\dfrac{4}{7}.1}{\dfrac{11}{14}}=\dfrac{8}{11}\approx 0,73.$$
	\end{listEX}}
\end{bt}
%%==========Bài 7
\begin{bt}%%%
	Giả sử có khoảng $40 \%$ thư điện tử (email) gửi đến một địa chỉ là thư rác. Người ta sử dụng một thuật toán để phân loại thư rác, biết rằng thuật toán này có thể phân loại đến $99 \%$ thư rác và tỉ lệ sai sót khi phân loại thư bình thường thành thư rác là $5 \%$. Tính xác suất một thư điện tử là thư bình thường nếu thư này đã được phân loại đúng.
	\loigiai{
	Ta có công thức
	$$	P(A | B)=\dfrac{P(B | A) \cdot P(A)}{P(B)}$$
	Trong đó
	\begin{itemize}
	\item $A$: Thư điện tử là thư bình thường.
	\item $B$: Thư đã được phân loại đúng.
	\item 	$\mathrm{P}(A)$: Xác suất một thư điện tử là thư bình thường ban đầu.\\
	Vì có $40 \%$ thư rác, nên $\mathrm{P}(A)=$ $1-0{,}4=0{,}6$.
	\item $\mathrm{P}(B | A)$: Xác suất một thư bình thường được phân loại đúng.\\
	Do tỉ lệ sai sót là $5 \%$, nên $\mathrm{P}(B | A)=1-0{,}05=0{,}95$.
	\item $P(B)$: Xác suất một thư nào đó được phân loại đúng, tính bằng tổng xác suất một thư rác được phân loại đúng và xác suất một thư bình thường được phân loại đúng
	\end{itemize}
	$$\begin{aligned}[t]
	\mathrm{P}(B)&=\mathrm{P}(B | A) \cdot \mathrm{P}(A)+\mathrm{P}(B |\overline{A}) \cdot \mathrm{P}(\overline{A}) \\&
	=0{,}95 \cdot 0{,}6+0{,}99 \cdot 0{,}4=0{,}97.
	\end{aligned}$$
	Áp dụng định lý Bayes: $\mathrm{P}(A | B)=\dfrac{\mathrm{P}(B | A) \cdot \mathrm{P}(A)}{\mathrm{P}(B)}=\dfrac{0{,}95 \cdot 0{,}6}{0{,}97}=	\dfrac{57}{97}$.
	}
\end{bt}
%%==========Bài 8
\begin{bt}%%%
	Một chiếc hộp có $20$ chiếc thẻ cùng loại, trong đó có $2$ chiếc thẻ màu xanh và $18$ chiếc thẻ màu trắng. Bạn Châu rút thẻ hai lần một cách ngẫu nhiên, mỗi lần rút một thẻ và thẻ được rút ra không bỏ lại hộp. Tính xác suất để cả hai lần bạn Châu đều rút được thẻ màu xanh.
	\loigiai{
	Xét hai biến cố\\
	$A$: \lq\lq  Thẻ thứ nhất rút được màu xanh\rq\rq.\\
	$B$: \lq\lq  Thẻ thứ hai rút được màu xanh\rq\rq.\\
	Khi đó, ta có xác suất để cả hai lần bạn Châu đều rút được thẻ màu xanh là \[\mathrm{P}(A\cap B)=\dfrac{C_2^2}{C_{20}^2}=\dfrac{1}{190}.\]
	}
\end{bt}
% \subsection{Bài tập trắc nghiệm}
% \BTTN
% \PHIEUTRACNGHIEM
\Opensolutionfile{ans}[ans/ans-2-B19]
\begin{ex}%[2D6N2-1]Câu 1
Cho hai biến cố $A$ và $B$ với $0<P(B)<1$. Khi đó công thức xác suất toàn phần cho biến cố $A$ là
	\choice
	{\True $\mathrm{P}(A)=\mathrm{P}(B)\mathrm{P}(A|B)+\mathrm{P}(\overline{B})\mathrm{P}(A|\overline{B})$}
	{$\mathrm{P}(A)=\mathrm{P}(A)\mathrm{P}(A|B)+\mathrm{P}(\overline{A})\mathrm{P}(A|\overline{B})$}
	{$\mathrm{P}(A)=\mathrm{P}(\overline{B})\mathrm{P}(A|B)+\mathrm{P}(B)\mathrm{P}(A|\overline{B})$}
	{$\mathrm{P}(B)=\mathrm{P}(\overline{B})\mathrm{P}(A|B)+\mathrm{P}(B)\mathrm{P}(B|\overline{B})$}
	\loigiai{Cho hai biến cố $A$ và $B$ với $0<P(B)<1$. Khi đó 
	$\mathrm{P}(A)=\mathrm{P}(B)P(A|B)+\mathrm{P}(\overline{B})\mathrm{P}(A|\overline{B})$ 
	gọi là {\bf công thức xác suất toàn phần}.
	}
\end{ex}

\begin{ex}%[2D6H2-1]Câu 2
Cho hai biến cố $A=A_1+A_2$ và biến cố $B=B_1+B_2$ biểu diễn theo đồ ven như sau
\begin{center}
\tikzset{every picture/.style={line width=0.75pt}} %set default line width to 0.75pt 
\begin{tikzpicture}[x=0.75pt,y=0.75pt,yscale=-1,xscale=1]
	%uncomment if require: \path (0,300); %set diagram left start at 0, and has height of 300
	%Shape: Rectangle [id:dp41068169978107316] 
	\draw [fill=red!30] (94,104) -- (354,104) -- (354,176) -- (94,176) -- cycle ;
	%Shape: Ellipse [id:dp24667066386593728] 
	\draw [fill=green!30] (144,140) .. controls (144,128.95) and (176.24,120) .. (216,120) .. controls (255.76,120) and (288,128.95) .. (288,140) .. controls (288,151.05) and (255.76,160) .. (216,160) .. controls (176.24,160) and (144,151.05) .. (144,140) -- cycle ;
	%Straight Lines [id:da2324528203840277] 
	\draw (192,104) -- (192,176) ;
	% Text Node
	\draw (302,115) node [anchor=north west][inner sep=0.75pt] [align=left] {A\textsubscript{2}};
	% Text Node
	\draw (121,114) node [anchor=north west][inner sep=0.75pt] [align=left] {A\textsubscript{1}};
	% Text Node
	\draw (169,131) node [anchor=north west][inner sep=0.75pt] [align=left] {B\textsubscript{1}};
	% Text Node
	\draw (230,130) node [anchor=north west][inner sep=0.75pt] [align=left] {B\textsubscript{2}};
\end{tikzpicture}
\end{center}
Tính xác xuất của $P(A)$.
	\choice
	{$\mathrm{P}(A)=\mathrm{P}(B_1)\mathrm{P}(A_1|B_1)+\mathrm{P}(B_2)\mathrm{P}(A_1|B_2)$}
	{\True $\mathrm{P}(A)=\mathrm{P}(B_1)\mathrm{P}(A|B_1)+\mathrm{P}(B_2)\mathrm{P}(A|B_2)$}
	{$\mathrm{P}(A)=\mathrm{P}(B)\mathrm{P}(A_1|B_1)+\mathrm{P}(B)\mathrm{P}(A_2|B_2)$}
	{$\mathrm{P}(A)=\mathrm{P}(A_1)\mathrm{P}(A|B_1)+\mathrm{P}(B_2)\mathrm{P}(A|B_2)$}
	\loigiai{ 
	Cho hai biến cố $A$ và $B$ với $0<P(B)<1$. Khi đó 
	$\mathrm{P}(A)=\mathrm{P}(B)\mathrm{P}(A|B)+\mathrm{P}(\overline{B})\mathrm{P}(A|\overline{B})$ 
	gọi là {\bf công thức xác suất toàn phần}.\\
	 $\Rightarrow \mathrm{P}(A)=\mathrm{P}(B_1)\mathrm{P}(A|B_1)+\mathrm{P}(B_2)\mathrm{P}(A|B_2)$}
\end{ex}

\begin{ex}%[2D6N2-1]Câu 3
Giả sử hai biến cố $A$ và $B$ ngẫu nhiên thỏa mãn $\mathrm{P}(A)>0$ với $0<\mathrm{P}(B)<1$. Khi đó công thức {\bf Bayes} là
	\choice
	{$\mathrm{P}(A|B)=\dfrac{\mathrm{P}(B)\mathrm{P}(A|B)}{\mathrm{P}(B)\mathrm{P}(A|B)+\mathrm{P}(\overline{B})\mathrm{P}(A|\overline{B})}$}
	{$\mathrm{P}(B|A)=\dfrac{\mathrm{P}(A)P(A|B)}{\mathrm{P}(B)\mathrm{P}(A|B)+\mathrm{P}(\overline{B})\mathrm{P}(A|\overline{B})}$}
	{\True $\mathrm{P}(B|A)=\dfrac{\mathrm{P}(B)P(A|B)}{\mathrm{P}(B)\mathrm{P}(A|B)+\mathrm{P}(\overline{B})\mathrm{P}(A|\overline{B})}$}
	{$\mathrm{P}(B|A)=\dfrac{\mathrm{P}(B)\mathrm{P}(B|A)}{\mathrm{P}(B)\mathrm{P}(A|B)+\mathrm{P}(\overline{B})\mathrm{P}(A|\overline{B})}$}
	\loigiai{Giả sử hai biến cố $A$ và $B$ ngẫu nhiên thỏa mãn $P(A)>0$ với $0<P(B)<1$. Khi đó\\
	$\mathrm{P}(B|A)=\dfrac{\mathrm{P}(B)\mathrm{P}(A|B)}{\mathrm{P}(B)P(A|B)+\mathrm{P}(\overline{B})\mathrm{P}(A|\overline{B})}$
	 gọi là công thức {\bf Bayes}. 
	}
\end{ex}

\begin{ex}%[2D6N2-2]
	Cho $\mathrm{P}(A)=\dfrac{2}{5}$; $\mathrm{P}(B \mid A)=\dfrac{1}{3}$; $\mathrm{P}(B \mid \overline{A})=\dfrac{1}{4}$. Giá trị của $\mathrm{P}(B)$ là 
	\choice
	{$\dfrac{19}{60}$}
	{\True $\dfrac{17}{60}$}
	{$\dfrac{9}{20}$}
	{$\dfrac{7}{30}$}
	\loigiai{
	Áp dụng công thức xác suất toàn phần ta có
	$$
	\mathrm{P}(B)=\mathrm{P}(A) \cdot \mathrm{P}(B \mid A)+\mathrm{P}\left(\overline{A}\right) \cdot \mathrm{P}(B \mid \overline{A})=\dfrac{2}{5} \cdot \dfrac{1}{3}+\dfrac{3}{5} \cdot \dfrac{1}{4}=\dfrac{17}{60}.
	$$
	}
\end{ex}

\begin{ex}%[2D6N2-2]
	Cho hai biến cố $A$, $B$ với $\mathrm{P}(B)=0{,}6$; $\mathrm{P}(A \mid B)=0{,}7$ và $\mathrm{P}(A \mid \overline{B})=0{,}4$. Khi đó, $\mathrm{P}(A)$ bằng 
	\choice
	{$0{,}7$}
	{$0{,}4$}
	{\True $0{,}58$}
	{$0{,}52$}
	\loigiai{
	Ta có $\mathrm{P}(\overline{B})=1-\mathrm{P}(B)=1-0{,}6=0{,}4$.\\
	Áp dụng công thức xác suất toàn phần, ta có
	$$
	\mathrm{P}(A)=\mathrm{P}(A \mid B) \cdot \mathrm{P}(B)+\mathrm{P}(A \mid \overline{B}) \cdot \mathrm{P}(\overline{B})=0{,}7 \cdot 0{,}6+0{,}4 \cdot 0{,}4=0{,}58.
	$$
	}
\end{ex}

\begin{ex}%[2D6N2-1]
	Theo công thức Bayes ta có 
	\choice
	{$\mathrm{P}(A \mid B)=\dfrac{\mathrm{P}(B)\cdot \mathrm{P}(B \mid A)}{\mathrm{P}(A)}$}
	{$\mathrm{P}(B \mid A)=\dfrac{\mathrm{P}(A)\cdot \mathrm{P}(B \mid A)}{\mathrm{P}(B)}$}
	{\True $\mathrm{P}(A \mid B)=\dfrac{\mathrm{P}(A)\cdot \mathrm{P}(B \mid A)}{\mathrm{P}(B)}$}
	{$\mathrm{P}(A \mid B)=\dfrac{\mathrm{P}(B)\cdot \mathrm{P}(A \mid B)}{\mathrm{P}(A)}$}
	\loigiai{
	Theo công thức Bayes ta có $\mathrm{P}(A \mid B)=\dfrac{\mathrm{P}(A)\cdot \mathrm{P}(B \mid A)}{\mathrm{P}(B)}$
	}
\end{ex}

\begin{ex}%[2D6N2-3]
	Cho $\mathrm{P}(A)=\dfrac{4}{5}$; $\mathrm{P}(B \mid A)=\dfrac{2}{3}$; $\mathrm{P}(B \mid \overline{A})=\dfrac{1}{4}$. Giá trị của $\mathrm{P}(A \mid B)$ là 
	\choice
	{$\dfrac{33}{35}$}
	{\True $\dfrac{32}{35}$}
	{$\dfrac{9}{35}$}
	{$\dfrac{26}{35}$}
	\loigiai{
	Áp dụng công thức Bayes ta có
	$$
	\mathrm{P}(A \mid B)=\dfrac{\mathrm{P}(A) \cdot \mathrm{P}(B \mid A)}{\mathrm{P}(A) \cdot \mathrm{P}(B \mid A)+\mathrm{P}(\overline{A}) \cdot \mathrm{P}(B \mid \overline{A})}=\dfrac{\dfrac{4}{5} \cdot \dfrac{2}{3}}{\dfrac{4}{5} \cdot \dfrac{2}{3}+\dfrac{1}{5} \cdot \dfrac{1}{4}}=\dfrac{32}{35}.
	$$
	}
\end{ex}

\begin{ex}%[2D6H2-2]
	Tỉ lệ người dân đã tiêm vắc xin phòng bệnh A ở một địa phương là $65 \%$. Trong số những người đã tiêm phòng, tỉ lệ mắc bệnh A là $5 \%$ còn trong số những người chưa tiêm, tỉ lệ mắc bệnh A là $17 \%$. Gặp ngẫu nhiên một người ở địa phương đó. Xác suất người đó mắc bệnh A là 
	\choice
	{$0{,}0325$}
	{$0{,}018$}
	{\True $0{,}092$}
	{$0{,}0525$}
	\loigiai{
	Gọi $H_1$ là biến cố \lq\lq  Gặp được người đã tiêm vắc xin phòng bệnh A\rq\rq, $H_2$ là biến cố \lq\lq  Gặp được người chưa tiêm vắc xin phòng bệnh A\rq\rq, $K$ là biến cố \lq\lq  Người đó mắc bệnh A\rq\rq. \\
	Theo công thức xác suất toàn phần, ta có
	\begin{eqnarray*}
	\mathrm{P}(K)&=&\mathrm{P}\left(H_1\right) \cdot \mathrm{P}\left(K \mid H_1\right)+\mathrm{P}\left(H_2\right) \cdot \mathrm{P}\left(K \mid H_2\right) \\
	&=& 0{,}65 \cdot 0{,}05+0{,}35 \cdot 0{,}17=0{,}092.
	\end{eqnarray*}
	}
\end{ex}

\begin{ex}%[2D6H2-2]
	Một nhà máy có hai phân xưởng I và II. Phân xưởng I sản xuất $40 \%$ số sản phẩm và phân xưởng II sản xuất $60 \%$ số sản phẩm. Tỉ lệ sản phẩm bị lỗi của phần xưởng I là $2 \%$ và của phân xưởng II là $1 \%$. Kiểm tra ngẫu nhiên $1$ sản phẩm của nhà máy và xác suất để sản phẩm đó bị lỗi là 
	\choice
	{$0{,}02$}
	{$0{,}6$}
	{\True $0{,}014$}
	{$0{,}01$}
	\loigiai{
	Gọi $A$ là biến cố \lq\lq  Sản phẩm bị lỗi\rq\rq, $B$ là biến cố \lq\lq  Sản phẩm lấy ra do phân xưởng $I$ sản xuất\rq\rq.\\
	Do phân xưởng I sản xuất $40 \%$ số sản phẩm và phân xưởng II sản xuất $60 \%$ số sản phẩm nên
	$$
	\mathrm{P}(B)=0{,}4 \,\,\text{và}\,\, \mathrm{P}(\overline{B})=1-0{,}4=0{,}6.
	$$
	Do tỉ lệ sản phẩm bị lỗi của phân xưởng I là $2 \%$ và của phân xưởng II là $1 \%$ nên
	$$
	\mathrm{P}(A \mid B)=0{,}02 \,\, \text{và} \,\, \mathrm{P}(A \mid \overline{B})=1-0{,}01.
	$$
	Xác suất để sản phẩm lấy ra bị lỗi là
	$$
	\mathrm{P}(A)=\mathrm{P}(B) \mathrm{P}(A \mid B)+\mathrm{P}(\overline{B}) \mathrm{P}(A \mid \overline{B})=0{,}4 \cdot 0{,}02+0{,}6 \cdot 0{,}01=0{,}014.
	$$
	}
\end{ex}

\begin{ex}%[2D6V2-4]
	Cho $A$, $B$ là các biến cố thỏa mãn $\mathrm{P}(\overline{A}\cap\overline{B})=0{,}35$; $\mathrm{P}(A)=0{,}25$; $\mathrm{P}(B)=0{,}6$. Giá trị của $\mathrm{P}(A|B)$ bằng
	\choice
	{$\dfrac{1}{5}$}
	{\True$\dfrac{1}{3}$}
	{$\dfrac{7}{15}$}
	{$\dfrac{2}{3}$}
	\loigiai{Ta có $\mathrm{P}(\overline{A}\cap\overline{B}) = \mathrm{P}\left( \overline A \right)\mathrm{P}\left( \overline B |\overline A \right) \Rightarrow \mathrm{P}\left( \overline B |\overline A \right) = \dfrac{\mathrm{P}(\overline{A}\cap\overline{B})}{\mathrm{P}\left( \overline A \right)} = \dfrac{0{,}35}{0{,}75} = \dfrac{7}{15}.$ \\
	Suy ra $\mathrm{P}\left( B|\overline A \right) = 1 - \dfrac{7}{{15}} = \dfrac{8}{{15}}$.\\
	Theo công thức xác suất toàn phần, ta có
	\begin{eqnarray*}
	&&\mathrm{P}\left( B \right) = \mathrm{P}\left( B|A \right)\mathrm{P}\left( A \right) + \mathrm{P}\left( B|\overline A \right)\mathrm{P}\left( \overline A\right)\\
	&\Rightarrow& \mathrm{P}\left(B|A\right) = \dfrac{\mathrm{P}\left( B \right) - \mathrm{P}\left( {B|\overline A } \right)\mathrm{P}\left( {\overline A } \right)}{\mathrm{P}\left( A \right)} = \dfrac{0{,}6 - \dfrac{8}{{15}} \cdot 0{,}75}{0{,}25} = 0{,}8.
	\end{eqnarray*}
	Theo công thức Bayes, ta được
	$$\mathrm{P}\left( A|B\right) = \dfrac{\mathrm{P}\left( A \right)\mathrm{P}\left( {B|A} \right)}{\mathrm{P}\left( B \right)} = \dfrac{0{,}25 \cdot 0{,}8}{0{,}6} = \dfrac{1}{3}.$$}
\end{ex}

\begin{ex}%[2D6H2-3]
	Một bệnh viện có hai phòng khám là phòng A và phòng B với khả năng lựa chọn của bệnh nhân là như nhau. Tỉ lệ bệnh nhân nam có ở phòng A và phòng B lần lượt là $60\%$ và $40\%$. Một người bệnh được chọn ngẫu nhiêu từ hai phòng khám và biết người này là nam, xác suất để người bệnh được chọn đến từ phòng A là
	\choice
	{\True $0{,}6$}
	{$0{,}5$}
	{$0{,}4$}
	{$0{,}3$}
	\loigiai{Một người bệnh được chọn ngẫu nhiên từ hai phòng khám.\\
	Gọi $X$ là biến cố \lq \lq Người đó đến từ phòng khám A\rq \rq \, và $Y$, $\overline{Y}$ lần lượt là biến cố \lq \lq Người đó là nam\rq \rq \; và \lq \lq Người đó không là nam\rq \rq.\\
	Ta có sơ đồ hình cây sau
	\begin{center}
	\begin{tikzpicture}[>=stealth,scale=0.6]
	%Khung 1
	\draw (-2,-1) rectangle (2.2,0);
	\draw (0.1,-0.5) node{Bệnh nhân được chọn} ;
	%Mui ten 1,2
	\draw [->] (2.2,-0.5)--(3.8,1.6) node[pos=0.5,sloped,above]{$0{,}5$};
	\draw [->] (2.2,-0.5)--(3.8,-2.6) node[pos=0.5,sloped,below]{$0{,}5$};
	%Khung 2.1
	\draw (3.8,1.1) rectangle (5.1,2.1);
	\draw (8.9/2,1.6) node{$X$} ;
	%Khung 2.2
	\draw (3.8,-2.1) rectangle (5.1,-3.1);
	\draw (8.9/2,-2.6) node{$\overline{X}$} ;
	%Mui ten 3,4
	\draw [->] (5.1,1.6)--(6.5,2.6) node[pos=0.5,sloped,above]{$0{,}6$};
	\draw [->] (5.1,1.6)--(6.5,0.6) node[pos=0.5,sloped,below]{$0{,}4$};
	%Mui ten 5,6
	\draw [->] (5.1,-2.6)--(6.5,-1.6) node[pos=0.5,sloped,above]{$0{,}4$};
	\draw [->] (5.1,-2.6)--(6.5,-3.6) node[pos=0.5,sloped,below]{$0{,}6$};
	%Khung 3.1
	\draw (6.5,2.2) rectangle (7.7,3.2);
	\draw (7.1,5.4/2) node{$Y$} ;
	%Khung 3.2
	\draw (6.5,1.2) rectangle (7.7,0.2);
	\draw (7.1,1.4/2) node{$\overline{Y}$} ;
	%Khung 3.3
	\draw (6.5,-1.1) rectangle (7.7,-2.1);
	\draw (7.1,-3.2/2) node{$Y$} ;
	%Khung 3.3
	\draw (6.5,-2.9) rectangle (7.7,-3.9);
	\draw (7.1,-3.4) node{$\overline{Y}$} ;
	%Kết quả
	\draw (9.5,3.7) node{\textbf{Kết quả}};	
	\draw (9.5,2.7) node{$XY$};
	\draw (9.5,0.7) node{$X \overline{Y}$};
	\draw (9.5,-1.6) node{$\overline{X}Y$};
	\draw (9.5,-3.4) node{$\overline{X}\overline{Y}$};
	%Xác suất
	\draw (12.5,3.7) node{\textbf{Xác suất}};	
	\draw (12.5,2.7) node{$0{,}3$};
	\draw (12.5,0.7) node{$0{,}2$};
	\draw (12.5,-1.6) node{$0{,}2$};
	\draw (12.5,-3.4) node{$0{,}3$};	
	\end{tikzpicture}
	\end{center}
	Theo công thức Bayes, ta có $$\mathrm{P}(X|Y)=\dfrac{\mathrm{P}(X)\mathrm{P}(Y|X)}{\mathrm{P}(X)\mathrm{P}(Y|X)+\mathrm{P}(\overline{X})\mathrm{P}(Y|\overline{X})}=\dfrac{0{,}3}{0{,}3+0{,}2}=0{,}6.$$
	Vậy với một người bệnh được chọn ngẫu nhiêu từ hai phòng khám và biết người này là nam, xác suất để người đó đến từ phòng A là $0{,}6$.}
\end{ex}

\begin{ex}%[2D6V2-3]
	Một bệnh viện đang xét nghiệm cho một số bệnh nhân để xác định liệu họ có nhiễm virus $X$ hay không. Xác suất để một bệnh nhân bị nhiễm virus $X$ là $0{,}05$. Khi xét nghiệm, nếu một bệnh nhân bị nhiễm thì xác suất để kết quả xét nghiệm dương tính là $0{,}95$. Nếu một bệnh nhân không bị nhiễm thì xác suất để kết quả xét nghiệm âm tính là $0{,}98$. Một bệnh nhân được chọn ngẫu nhiên và có kết quả xét nghiệm dương tính. Xác suất để bệnh nhân đó thực sự bị nhiễm virus $X$ là
	\choice
	{$\dfrac{133}{2000}$}
	{$\dfrac{19}{400}$}
	{\True $\dfrac{5}{7}$}
	{$\dfrac{2}{7}$}
	\loigiai{Một bệnh nhân đến một bệnh viên để xét nghiệm.\\
	Gọi $A$ là biến cố \lq \lq Bệnh nhân bị nhiễm virus $X$\rq \rq \, và $B$, $\overline{B}$ lần lượt là biến cố \lq \lq Kết quả xét nghiệm dương tính\rq \rq \; và \lq \lq Kết quả xét nghiệm âm tính\rq \rq.\\
	Ta xét sơ đồ hình cây như sau
	\begin{center}
	\begin{tikzpicture}[>=stealth,scale=0.6]
	%Khung 1
	\draw (-3.5,-1) rectangle (2.2,0);
	\draw (-1.3/2,-0.5) node{Bệnh nhân được xét nghiệm} ;
	%Mui ten 1,2
	\draw [->] (2.2,-0.5)--(3.8,1.6) node[pos=0.5,sloped,above]{$0{,}05$};
	\draw [->] (2.2,-0.5)--(3.8,-2.6) node[pos=0.5,sloped,below]{$0{,}95$};
	%Khung 2.1
	\draw (3.8,1.1) rectangle (5.1,2.1);
	\draw (8.9/2,1.6) node{$A$} ;
	%Khung 2.2
	\draw (3.8,-2.1) rectangle (5.1,-3.1);
	\draw (8.9/2,-2.6) node{$\overline{A}$} ;
	%Mui ten 3,4
	\draw [->] (5.1,1.6)--(6.5,2.6) node[pos=0.5,sloped,above]{$0{,}95$};
	\draw [->] (5.1,1.6)--(6.5,0.6) node[pos=0.5,sloped,below]{$0{,}05$};
	%Mui ten 5,6
	\draw [->] (5.1,-2.6)--(6.5,-1.6) node[pos=0.5,sloped,above]{$0{,}02$};
	\draw [->] (5.1,-2.6)--(6.5,-3.6) node[pos=0.5,sloped,below]{$0{,}98$};
	%Khung 3.1
	\draw (6.5,2.2) rectangle (7.7,3.2);
	\draw (7.1,5.4/2) node{$B$} ;
	%Khung 3.2
	\draw (6.5,1.2) rectangle (7.7,0.2);
	\draw (7.1,1.4/2) node{$\overline{B}$} ;
	%Khung 3.3
	\draw (6.5,-1.1) rectangle (7.7,-2.1);
	\draw (7.1,-3.2/2) node{$B$} ;
	%Khung 3.3
	\draw (6.5,-2.9) rectangle (7.7,-3.9);
	\draw (7.1,-3.4) node{$\overline{B}$} ;
	%Kết quả
	\draw (9.5,3.7) node{\textbf{Kết quả}};	
	\draw (9.5,2.7) node{$AB$};
	\draw (9.5,0.7) node{$A\overline{B}$};
	\draw (9.5,-1.6) node{$\overline{A}B$};
	\draw (9.5,-3.4) node{$\overline{A}\cap\overline{B}$};
	%Xác suất
	\draw (12.5,3.7) node{\textbf{Xác suất}};	
	\draw (12.5,2.7) node{$0{,}0475$};
	\draw (12.5,0.7) node{$0{,}0025$};
	\draw (12.5,-1.6) node{$0{,}019$};
	\draw (12.5,-3.4) node{$0{,}931$};
	\end{tikzpicture}
	\end{center}
	Theo công thức Bayes, ta có $$\mathrm{P}(A|B)=\dfrac{\mathrm{P}(A)\mathrm{P}(B|A)}{\mathrm{P}(A)\mathrm{P}(B|A)+\mathrm{P}(\overline{A})\mathrm{P}(B|\overline{A})}=\dfrac{0{,}0475}{0{,}0475+0{,}019}=\dfrac{5}{7}.$$	 
	Vậy với một bệnh nhân có kết quả xét nghiệm dương tính, xác suất để bệnh nhân đó thực sự bị nhiễm virus $X$ là $\dfrac{5}{7}$.}
\end{ex}

\begin{ex}%[2D6V2-2]
	Kết quả khảo sát tại một xã cho thấy có $20 \%$ cư dân hút thuốc lá. Tỉ lệ cư dân thường xuyên gặp các vấn đề sức khoẻ về đường hô hấp trong số những người hút thuốc lá và không hút thuốc lá lần lượt là $70 \%$, $15 \%$. Tỉ lệ gặp một cư dân của xã thì xác suất người đó thường xuyên gặp các vấn đề sức khoẻ về đường hô hấp là bao nhiêu phần trăm? 
	\choice
	{\True $26 \%$}
	{$12 \%$}
	{$68 \%$}
	{$24 \%$}
	\loigiai{
	Giả sử ta gặp một cư dân của xã, gọi $A$ là biến cố \lq\lq  Người đó có hút thuốc lá\rq\rq\, và $B$ là biến cố \lq\lq  Người đó thường xuyên gặp các vấn đề sức khoẻ về đường hô hấp\rq\rq. Ta có sơ đồ hình cây sau.
	\begin{center}
	\begin{tikzpicture}[scale=.3,>=stealth,every node/.style={shape=rectangle,draw,rounded corners, color=blue, fill=blue!10}]
	%-------------
	\tikzstyle{block} = [rectangle, draw, fill=blue!10, rounded corners, text centered, text width = 10em, minimum height = 2em]
	%-------------
	\node (c1) {Gặp một cư dân};
	\node (c2) [above right = 2cm of c1]{$A$};
	\node (c3) [ below right= 2cm of c1]{$\overline{A}$};
	\node (c4) [above right = 1cm of c2]{$B$} ;
	\node (c5) [below right = 1cm of c2]{$\overline{B}$};
	\node (c6) [ above right =1cm of c3]{$B$};
	\node (c7) [ below right = 1cm of c3]{$\overline{B}$};
	%--------------
	\draw 
	(17.5,15) node[right] {\text{Kết quả}} 
	(20,11.3) node[right] {$AB$} 
	(20,2.5) node[right] {$A\overline{B}$} 
	(20,-2.5) node[right] {$\overline{A}B$} 
	(20,-11.5) node[right] {$\overline{A}\cap \overline{B}$};
	%--------------
	\draw 
	(25,15) node[right] {\text{Xác suất}} 
	(25,11.3) node[right] {$0{,}14$} 
	(25,2.5) node[right] {$0{,}06$} 
	(25,-2.5) node[right] {$0{,}12$} 
	(25,-11.5) node[right] {$0{,}68$};
	%------------
	\draw[->] (c1.east) --node[above left]{$0{,}2$} (c2.west);
	\draw[->] (c1.east) --node[below left]{$0{,}8$} (c3.west);
	\draw[->] (c2.east) --node[above left]{$0{,}7$} (c4.west);
	\draw[->] (c2.east) --node[below left]{$0{,}3$} (c5.west);
	\draw[->] (c3.east) --node[above left]{$0{,}15$} (c6.west);
	\draw[->] (c3.east) -- node[below left]{$0{,}85$} (c7.west);
	\end{tikzpicture}
	\end{center}
	Ta có $\mathrm{P}(B)=\mathrm{P}(A) \cdot \mathrm{P}(B | A)+\mathrm{P}(\overline{A}) \cdot \mathrm{P}\left(B | \overline{A}\right)=0{,}14+0{,}12=0{,}26$.\\
	Vậy nếu ta gặp một cư dân của xã thì xác suất người đó thường xuyên gặp các vấn đề sức khoẻ về đường hô hấp là $26\%$.
	}
\end{ex}

\begin{ex}%[2D6V2-3]
	Ở một địa phương $X$, xác suất để một người lớn trên $40$ tuổi mắc bệnh ung thư là $0{,}05$. Xác suất bác sĩ chẩn đoán đúng một người mắc bệnh ung thư là $0{,}78$ và chẩn đoán sai (không bị ung thư nhưng được chẩn đoán mắc bệnh) là $0{,}06$. Xác suất để một người thật sự mắc bệnh ung thư khi nhận được kết quả chẩn đoán bị ung thư bằng
	\choice
	{\True$0{,}40625$}
	{$0{,}096$}
	{$0{,}904$}
	{$0{,}59375$}
	\loigiai{Một bệnh nhân trên 40 tuổi ở địa phương X đến bác sĩ để khám bệnh ung thư.\\
	Gọi $A$ là biến cố \lq \lq Người đó mắc bệnh ung thư\rq \rq \, và $B$, $\overline{B}$ lần lượt là biến cố \lq \lq Bác sĩ chẩn đoán người đó bị ung thư\rq \rq \;và \lq \lq Bác sĩ chẩn đoán người đó không bị ung thư\rq \rq.\\
	Ta xét sơ đồ hình cây như sau
	\begin{center}
	\begin{tikzpicture}[>=stealth,scale=0.7]
	%Khung 1
	\draw (-3.5,-1) rectangle (2.2,0);
	\draw (-1.3/2,-0.5) node{Bệnh nhân được chẩn đoán} ;
	%Mui ten 1,2
	\draw [->] (2.2,-0.5)--(3.8,1.6) node[pos=0.5,sloped,above]{$0{,}05$};
	\draw [->] (2.2,-0.5)--(3.8,-2.6) node[pos=0.5,sloped,below]{$0{,}95$};
	%Khung 2.1
	\draw (3.8,1.1) rectangle (5.1,2.1);
	\draw (8.9/2,1.6) node{$A$} ;
	%Khung 2.2
	\draw (3.8,-2.1) rectangle (5.1,-3.1);
	\draw (8.9/2,-2.6) node{$\overline{A}$} ;
	%Mui ten 3,4
	\draw [->] (5.1,1.6)--(6.5,2.6) node[pos=0.5,sloped,above]{$0{,}78$};
	\draw [->] (5.1,1.6)--(6.5,0.6) node[pos=0.5,sloped,below]{$0{,}22$};
	%Mui ten 5,6
	\draw [->] (5.1,-2.6)--(6.5,-1.6) node[pos=0.5,sloped,above]{$0{,}06$};
	\draw [->] (5.1,-2.6)--(6.5,-3.6) node[pos=0.5,sloped,below]{$0{,}94$};
	%Khung 3.1
	\draw (6.5,2.2) rectangle (7.7,3.2);
	\draw (7.1,5.4/2) node{$B$} ;
	%Khung 3.2
	\draw (6.5,1.2) rectangle (7.7,0.2);
	\draw (7.1,1.4/2) node{$\overline{B}$} ;
	%Khung 3.3
	\draw (6.5,-1.1) rectangle (7.7,-2.1);
	\draw (7.1,-3.2/2) node{$B$} ;
	%Khung 3.3
	\draw (6.5,-2.9) rectangle (7.7,-3.9);
	\draw (7.1,-3.4) node{$\overline{B}$} ;
	%Kết quả
	\draw (9.5,3.7) node{\textbf{Kết quả}};	
	\draw (9.5,2.7) node{$AB$};
	\draw (9.5,0.7) node{$A\overline{B}$};
	\draw (9.5,-1.6) node{$\overline{A}B$};
	\draw (9.5,-3.4) node{$\overline{A}\cap\overline{B}$};
	%Xác suất
	\draw (12.5,3.7) node{\textbf{Xác suất}};	
	\draw (12.5,2.7) node{$0{,}039$};
	\draw (12.5,0.7) node{$0{,}011$};
	\draw (12.5,-1.6) node{$0{,}057$};
	\draw (12.5,-3.4) node{$0{,}893$};
	\end{tikzpicture}
	\end{center}
	Theo công thức Bayes, ta có $$\mathrm{P}(A|B)=\dfrac{\mathrm{P}(A)\mathrm{P}(B|A)}{\mathrm{P}(A)\mathrm{P}(B|A)+\mathrm{P}(\overline{A})\mathrm{P}(B|\overline{A})}=\dfrac{0{,}039}{0{,}039+0{,}057}=0{,}40625.$$	 
	Vậy xác suất để một người thật sự mắc bệnh ung thư khi nhận được kết quả chẩn đoán bị ung thư bằng $0{,}40625$.}
\end{ex}

\begin{ex}%[2D6V2-2]
	Trung tâm kiểm soát và phòng ngừa dịch bệnh Hoa Kỳ (Centers for Disease Control and Prevention, viết tắt là (CDC) thống kê vào thời điểm năm $2020 - 2021$ về số lượng sốc phản vệ sau khi tiêm vaccine ở một số nơi tại Hoa Kỳ và châu Âu như sau: Trong $360{,}19$ triệu liều vaccine $P$ được sử dụng có $581$ ca sốc phản vệ (có khả năng gây tử vong) và $4\,259$ ca phản ứng phụ (không sốc phản vệ, không gây tử vong); trong $67{,}72$ triệu liều vaccine $A$ được sử dụng có $195$ ca sốc phản vệ và $1\,118$ ca phản ứng phụ.\\
	\textit{(Nguồn: https://www.ncbi.nlm.nih.gov/pmc/articles/PMC8626274/)}
	\choice
	{$1{,}9 \cdot 10^{-6}$}
	{$2{,}8 \cdot 10^{-6}$}
	{\True $1{,}81 \cdot 10^{-6}$}
	{$2{,}81 \cdot 10^{-6}$}
	\loigiai{
	Xét ngẫu nhiên một người trong số được thống kê ở trên. Tính xác suất để người đó thuộc trường hợp sốc phản vệ (có khả năng gây tử vong).\\
	Gọi $X$ là biến cố “Người được chọn tiêm vaccine $P$”, khi đó $\overline X $ là biến cố “Người được chọn tiêm vaccine $A$”.\\
	$Y$ là biến cố “Người được chọn thuộc trường hợp sốc phản vệ”.\\
	Khi đó, xác suất chọn được người tiêm vaccine $P$ là $\mathrm{P}(X)=\dfrac{360{,}19 \cdot 10^6}{360{,}19 \cdot 10^6+67{,}72 \cdot 10^6}$.\\
	Xác suất chọn được người tiêm vaccine $A$ là $\mathrm{P}(\overline X)=\dfrac{67{,}72 \cdot 10^6}{360{,}19 \cdot 10^6+67{,}72 \cdot 10^6}$.\\
	Xác suất chọn được người bị sốc phản vệ, nếu người đó tiêm vaccine $P$ là $\mathrm{P}(Y|X)=\dfrac{581}{360{,}19 \cdot 10^6}$.\\
	Xác suất chọn được người bị sốc phản vệ, nếu người đó tiêm vaccine $A$ là $\mathrm{P}(Y|\overline X)=\dfrac{195}{67{,}72 \cdot 10^6}$.\\
	Áp dụng công thức tính xác suất toàn phần, ta có:
	$$\mathrm{P}(Y)=\mathrm{P}(X)\cdot \mathrm{P}(Y|X)+\mathrm{P}(\overline{X})\cdot \mathrm{P}(Y|\overline{X}) \approx 1{,}81 \cdot 10^{-6}.$$
	}
\end{ex}

\begin{ex}%[2D6V2-2]Câu 4
	Trên bàn có hai hộp $B_1$ và $B_2$ đều đựng đá cẩm thạch. Hộp $B_1$ chứa 7 viên xanh và 4 viên trắng. Hộp $B_2$ chứa 3 viên xanh và 10 viên vàng. Các hộp được sắp xếp sao cho xác suất chọn hộp $B_1$ là $\dfrac{1}{3}$ và xác suất chọn hộp $B_2$ là $\dfrac{2}{3}$. Kathy bị bịt mắt và dược yêu cầu chọn một viên đá cẩm thạch. Cô ấy sẽ được thưởng một chiếc TV nếu chọn được một viên màu xanh. Xác suất cô ấy được thưởng một chiếc TV màu xanh là bao nhiêu
	\choice
	{\True $\dfrac{157}{429}$}
	{$\dfrac{158}{429}$}
	{$\dfrac{159}{429}$}
	{$\dfrac{59}{429}$}
	\loigiai{Gọi $A$ là biến cố \lq\lq  Kathy được thưởng một chiếc TV\rq\rq.\\
	Gọi $B_j$ là biến cố \lq\lq  hộp $B_j$ được chọn\rq\rq, $j=1,2$.\\
	Xác suất $\mathrm{P}(B_1)=\dfrac{1}{3}, P(B_2)=\dfrac{2}{3}$.\\
	Xác suất cần tìm là \\ $\mathrm{P}(A)=\mathrm{P}(B_1)\cdot \mathrm{P}(A|B_1)+\mathrm{P}(B_2)\cdot \mathrm{P}(A|B_2)=\dfrac{1}{3}\cdot\dfrac{7}{11}+\dfrac{2}{3}\cdot\dfrac{3}{13}=\dfrac{157}{429}$.
	}
\end{ex}

\begin{ex}%[2D6V2-2]Câu 5
Có $0{,}5\%$ dân số mắc bệnh $X$. Có một xét nghiệm để phát hiện bệnh $X$. Đối với những người mắc bệnh $X$, Xác suất xét nghiệm này không dương tính là $2\%$. Đối vơi những người không mắc bệnh $X$, xác suất xét nghiệm này dương tính là $3\%$. Xác suất một người được chọn ngẫu nhiên có kết quả dương tính với xét nghiệm phát hiện bệnh $X$ là bao nhiêu?
	\choice
	{$\dfrac{13}{400}$}
	{\True $\dfrac{139}{4000}$}
	{$\dfrac{137}{4000}$}
	{$\dfrac{13}{200}$}
	\loigiai{ Gọi $A$ là biến cố \lq\lq  người được chọn có kết quả dương tính với xét nghiệm phát hiện bệnh $X$ \rq\rq .\\
	Gọi $B_1$, $B_2$ lần lượt là biến cố \lq\lq  người được chọn mắc bệnh $X$\rq\rq. \\
	Xác suất $\mathrm{P}(B_1)=\dfrac{5}{1000}$, $\mathrm{P}(B_2)=\dfrac{995}{1000}$.\\
	Xác suất cần tìm là \\ $\mathrm{P}(A)=\mathrm{P}(B_1)\cdot \mathrm{P}(A|B_1)+\mathrm{P}(B_2)\cdot\mathrm{P}(A|B_2)=\dfrac{5}{1000}\cdot\dfrac{98}{1000}+\dfrac{995}{1000}\cdot\dfrac{3}{1000}=\dfrac{139}{4000}$
	}
\end{ex}

\begin{ex}%[2D6V2-2]Câu 6
Một cặp sinh đôi có thể do cùng một trứng sinh ra (sinh đôi thật), hoặc do hai trứng
khác nhau sinh ra (sinh đôi giả). Các cặp sinh đôi thật luôn cùng giới tính. Đối với
cặp sinh đôi giả thì khả năng cùng giới tính và khác giới tính là như nhau. Thống kê
cho thấy $34\%$ cặp sinh đôi đều là trai, $30\%$ cặp sinh đôi đều là gái, và $36\%$ cặp sinh đôi có giới tính khác nhau. Tìm xác suất sinh đôi thật?
	\choice
	{$0{,}29$}
	{$0{,}27$}
	{\True$0{,}28$}
	{$0{,}3$}
	\loigiai{Gọi $B_1$, $B_2$, $A$ lần lượt là biến cố \lq\lq  cặp sinh đôi là thật\rq\rq,\, \lq\lq  cặp sinh đôi là giả\rq\rq,\, \lq\lq  cặp sinh đôi cùng giới\rq\rq.
	\begin{eqnarray*}
	\mathrm{P}(A)& =& \mathrm{P}(B_1)\cdot \mathrm{P}(A|B_1)+\mathrm{P}(B_2)\mathrm{P}(A|B_2)\\
\Leftrightarrow	\, \, \, 0{,}34+0{,}3	&= & P(B_1)\cdot1+[1-P(B_1)]\cdot0{,}5\\
	\Rightarrow \quad \quad P(B_1) &= & 0{,28}
	\end{eqnarray*}
	}
\end{ex}

\begin{ex}%[2D6V2-2]Câu 7
	Một lô hạt giống được chia làm $2$ loại: loại $1$ chiếm $\dfrac{2}{3}$ số hạt của lô, còn lại là loại $2$. Hạt trong loại $1$ có tỉ lệ nảy mầm là $80\%$, loại $2$ có tỉ lệ nảy mầm là $60\%$. Hỏi tỉ lệ nảy mầm chung của lô hạt giống này là bao nhiêu? (Hay nói cách khác: Ta lấy ngẫu nhiên từ lô ra $1$ hạt giống. Tìm xác suất để chọn được hạt nảy mầm).
	\choice
	{$\dfrac{11}{16}$}
	{$\dfrac{11}{14}$}
	{$\dfrac{11}{11}$}
	{\True$\dfrac{11}{15}$}
	\loigiai{ Gọi $A_i$ là biến cố \lq\lq  hạt giống được lấy ra từ lô thứ $i$\rq\rq \, (với $i=1,2$).\\
	Gọi $A$ là biến cố \lq\lq  hạt giống được lấy ra là hạt nảy mầm\rq\rq .
	Dễ thấy rằng $A_1$, $A_2$, $A_3$ tạo thành một hệ đầy đủ các biến cố.\\
	Theo đề bài: $\mathrm{P}(A_1)=\dfrac{2}{3}$, $\mathrm{P}(A_2)=\dfrac{1}{3}$ và $\mathrm{P}(A|A_1)=0{,}8$, $\mathrm{P}(A|A_2)=0{,}6$.\\
	Áp dụng công thức xác suất đầy đủ
	$$\mathrm{P}(A)=\mathrm{P}(A_1)\cdot \mathrm{P}(A|A_1)+\mathrm{P}(A_2)\mathrm{P}(A|A_2)=\dfrac{2}{3}\cdot0{,}8+\dfrac{1}{3}\cdot0{,}6=\dfrac{11}{15}.$$
	}
\end{ex}

\begin{ex}%[2D6V2-2]Câu 8
Một trạm cấp cứu có $80\%$ nạn nhân bỏng nóng và $20\%$ bỏng do hóa chất. Loại bỏng do nóng có $30\%$ biến chứng và bỏng do hóa chất có $50\%$ biến chứng. Tìm xác suất của nạn nhân bị biến chứng.
	\choice
	{$0{,}35$}
	{$0{,}36$}
	{\True $0{,}34$}
	{$0{,}37$}
	\loigiai{ Gọi $A$ là biến cố \lq\lq  nạn nhân bị biến chứng\rq\rq.
	Gọi $A_1$ là biến cố \lq\lq  nạn nhân bị bỏng do nóng\rq\rq.
	Gọi $A_2$ là biến cố \lq\lq  nạn nhân bị bỏng do hóa chất\rq\rq .
	Ta thấy rằng $A_1$ và $A_2$ lập thành một hệ đầy đủ.\\
	Theo đề bài	$\mathrm{P}(A_1)=0{,}8$;\, $\mathrm{P}(A_2)=0{,}2$; \,$\mathrm{P}(A|A_1)=0{,}3$; \,$\mathrm{P}(A|A_2)=0{,}5$.\\
	Sử dụng công thức xác suất đầy đủ
	$$\mathrm{P}(A)=\mathrm{P}(A_1)\mathrm{P}(A|A_1)+P(A_2)\mathrm{P}(A|A_2)=0{,}8\cdot0{,}3+0{,}2\cdot0{,}5=0{,}34.$$
	}
\end{ex}

\begin{ex}%[2D6C2-2]Câu 9
Chuồng gà thứ nhất có $9$ con mái và $1$ con trống. Chuồng gà thứ hai có $1$ con mái và $5$ con trống. Từ mỗi chuồng gà bắt ra ngẫu nhiên một con. Các con gà còn lại được dồn vào chuồng thứ ba. Bắt ngẫu nhiên một con gà trong chuồng thứ ba. Tính xác suất để bắt được gà trống.
	\choice
	{$\dfrac{38}{107}$}
	{\True$\dfrac{38}{105}$}
	{$\dfrac{38}{109}$}
	{$\dfrac{39}{105}$}
	\loigiai{ 
	$B_1$ là biến cố $2$ con gà được bắt đều là trống.\\
	$B_2$ là biến cố $2$ con gà được bắt đều là mái.\\
	$B_3$ là biến cố $2$ con gà được bắt gồm $1$ trống và $1$ mái.\\
	$A$ là biến cố bắt được gà trống ở chuồng thứ ba.\\
	 $B_1$ xảy ra $\Rightarrow$ Chuồng thứ ba gồm: $10$ gà mái và $4$ gà trống.\\
	 $B_2$ xảy ra $\Rightarrow$ Chuồng thứ ba gồm: $8$ gà mái và $6$ gà trống.\\
	 $B_3$ xảy ra $\Rightarrow$ Chuồng thứ ba gồm: $9$ gà mái và $5$ gà trống.\\
	$\mathrm{P}(B_1)=\dfrac{1}{10}\cdot \dfrac{\mathrm{C}_{5}^{1}}{6}=\dfrac{1}{12}$;
	$\mathrm{P}(B_2)=\dfrac{\mathrm{C}_{9}^{1}}{10}\cdot \dfrac{1}{6}=\dfrac{3}{20}$;
	$\mathrm{P}(B_3)=1-\mathrm{P}(B_1)-\mathrm{P}(B_2)=\dfrac{23}{30}$.\\
	Xác suất để bắt được gà trống ở chuồng thứ ba là
	\begin{eqnarray*}
	\mathrm{P}(A)&=& \mathrm{P}(B_1)\mathrm{P}(A|B_1)+\mathrm{P}(B_2)\mathrm{P}(A|B_2)+\mathrm{P}(B_3)\mathrm{P}(A|B_3)\\
	&=& \dfrac{1}{12}\cdot\dfrac{4}{14}+\dfrac{3}{20}\cdot\dfrac{6}{14}+\dfrac{23}{30}\cdot\dfrac{5}{14}\\
	&= & \dfrac{38}{105}.
	\end{eqnarray*}
	}
\end{ex}

\begin{ex}%[2D6V2-3]Câu 10
Dây chuyền lắp ráp được các chi tiết do hai máy sản xuất. Trung bình máy thứ nhất cung cấp $60\%$ chi tiết, máy thứ hai cung cấp $40\%$ chi tiết. Khoảng $90\%$ chi tiết do máy thứ nhất sản xuất là đạt tiêu chuẩn, còn $85\%$ chi tiết do máy thứ hai sản xuất là đạt tiêu chuẩn. Lấy ngẫu nhiên từ dây chuyền một sản phẩm, lấy nó đạt tiêu chuẩn. Tìm xác suất để sản phẩm đó do máy thứ nhất sản xuất.
	\choice
	{$0{,}713$}
	{$0{,}715$}
	{$0{,}814$}
	{\True$0{,}614$}
	\loigiai{Gọi $A$ là biến cố \lq\lq  Chi tiết lấy từ dây chuyền đạt tiêu chuẩn\rq\rq.\\ $B_1$ là biến cố \lq\lq  Chi tiết do máy thứ nhất sản xuất \rq\rq.\\
	$B_2$ là biến cố \lq\lq  Chi tiết do máy thứ 2 sản xuất \rq\rq.\\
	Ta cần tính xác suất $\mathrm{P}(B_1|A)$.\\
	Theo điều kiện bài toán $\mathrm{P}(B_1)=0{,}6$; $\mathrm{P}(B_2)=0{,}4$; $\mathrm{P}(A|B_1)=0{,}9$; $\mathrm{P}(A|B_2)=0{,}85$. \\
	 Theo công thức Bayes
	 $$\mathrm{P}(B_1|A)=\dfrac{\mathrm{P}(B_1)\mathrm{P}(A|B_1)}{\mathrm{P}(B_1)\mathrm{P}(A|B_1)+\mathrm{P}(B_2)\mathrm{P}(A|B_2)}=\dfrac{0{,}6\cdot0{,9}}{0{,}6\cdot0{,9}+0{,}4\cdot0{,}85}=0{,}614.$$
	}
\end{ex}

\begin{ex}%[2D6C2-3]Câu 11
 Tan giờ học buổi chiều một sinh viên có $60\%$ về nhà ngay, nhưng do giờ cao điểm nên có $30\%$ ngày bị tắc đường nên bị về nhà muộn (từ $30$ phút trở lên) còn $20\%$ số ngày sinh viên đó vào quán Internet cạnh trường để chơi game, những ngày này xác suất về nhà muộn là $80\%$. Còn lại những ngày khác sinh viên đó đi chơi với bạn bè có xác suất về muộn là $90\%$. Tính xác suất sinh viên đó đi chơi với bạn và về muộn.
	\choice
	{$0{,}275$}
	{$0{,}575$}
	{\True $0{,}375$}
	{$0{,}475$}
	\loigiai{ 
	Gọi $B$ là biến cố sinh viên đi học về muộn \\
	Gọi $\overline{B}$ là biến cố sinh viên đó đi học không về muộn\\
	$E_1$ là biến cố tan học về nhà ngay $\Rightarrow \mathrm{P}(E_1)=0{,}6$; $\mathrm{P}(B|E_1)=0{,}3$.\\
 	$E_2$ là biến cố tan học đi chơi game $\Rightarrow \mathrm{P}(E_2)=0{,}2$; $\mathrm{P}(B|E_2)=0{,}8$.\\
	 $E_3$ là biến cố tan học về đi chơi với bạn $\Rightarrow \mathrm{P}(E_3)=0{,}2$; $\mathrm{P}(B|E_3)=0{,}9$.\\
	$B$ có thể xảy ra một trong 3 biến cố
	 \begin{eqnarray*}
	 	\mathrm{P}(B)&= & \mathrm{P}(E_1)\cdot \mathrm{P}(B|E_1)+\mathrm{P}(E_2)\cdot \mathrm{P}(B|E_2)+\mathrm{P}(E_3)\cdot \mathrm{P}(B|E_3)\\
	 	&= & 0{,}6\cdot0{,}3+0{,}2\cdot0{,}8+0{,}2\cdot0{,}9\\
	 	&= & 0{,}52.
	 \end{eqnarray*}
	Xác suất để sinh viên đó đi chơi với bạn và về muộn là
	$\mathrm{P}(E_3|B)=\dfrac{\mathrm{P}(E_3)\cdot \mathrm{P}(B|E_3)}{\mathrm{P}(B)}=0{,}375$.
	}
\end{ex}

\begin{ex}%[2D6V2-3]Câu 12
	Dây chuyền lắp ráp máy vô tuyến điện gồm các linh kiện là sản phẩm từ $2$ nhà máy sản xuất ra. Số linh kiện nhà máy $1$ sản xuất chiếm $55\%$, số linh kiện nhà máy $2$ sản xuất chiếm $45\%$; tỷ lệ sản phẩm đạt tiêu chuẩn của nhà máy $1$ là $90\%$, nhà máy $2$ là $87\%$. Lấy ngẫu nhiên ra $1$ linh kiện từ dây chuyền lắp ráp đó ra kiểm tra thì được kết quả linh kiện đạt chuẩn. Tìm xác suất để linh kiện đó do nhà máy $1$ sản xuất?
	\choice
	{$0{,}6683$}
	{\True$0{,}5583$}
	{$0{,}7583$}
	{$0{,}4583$}
	\loigiai{ Gọi $A_i$ là biến cố \lq\lq  linh kiện do nhà máy thứ $i$ sản xuất\rq\rq \, $i=1,2$.\\
	Gọi $B$ là biến cố \lq\lq  linh kiện đạt chuẩn\rq\rq, ta cần tìm $P(A_1|B)$\\
	Ta có\\ $\mathrm{P}(B)=\mathrm{P}(A_1)\mathrm{P}(B|A_1)+\mathrm{P}(A_2)\mathrm{P}(B|A_2)=0{,}55\cdot0{,}9+0{,}45\cdot0{,}87=0{,}8865$.\\
	$\mathrm{P}(A_1|B)=\dfrac{\mathrm{P}(A_1)\mathrm{P}(B|A_1)}{\mathrm{P}(B)}=\dfrac{0{,}55\cdot0{,}9}{0{,}8865}=0{,}5583$.
	}
\end{ex}

\begin{ex}%[2D6H2-2]
	Người ta khảo sát khả năng chơi nhạc cụ của một nhóm học sinh tại trường X. Nhóm này có $60\%$ học sinh là nam. Kết quả khảo sát cho thấy có $20\%$ học sinh nam và $15\%$ học sinh nữ biết chơi ít nhất một nhạc cụ. Chọn ngẫu nhiên một học sinh trong nhóm này. Gọi $A$ là biến cố \lq\lq  Chọn được một học sinh biết chơi ít nhất một nhạc cụ\rq\rq\,và $B$, $\overline{B}$ lần lượt là các biến cố \lq\lq  Chọn được một học sinh nam\rq\rq\,và \lq\lq  Chọn được một học sinh nữ\rq\rq. 
	\choiceTF
	{\True Xác suất $\mathrm{P}(B) = 60\% = 0{,}6$}
	{$\mathrm{P}(A|B) = 0{,}8$}
	{\True $\mathrm{P}(A|\overline{B}) = 0{,}15$}
	{\True Xác suất để chọn được học sinh biết chơi ít nhất một nhạc cụ là $18\%$}
	\loigiai{
	Xét phép thử chọn ngẫu nhiên một học sinh trong nhóm.\\
	Gọi $A$ là biến cố \lq\lq  Chọn được một học sinh biết chơi ít nhất một nhạc cụ\rq\rq \,và $B$, $\overline{B}$ lần lượt là các biến cố \lq\lq  Chọn được một học sinh nam\rq\rq\,và \lq\lq  Chọn được một học sinh nữ\rq\rq.
	Theo đề bài 
	\begin{itemchoice}
	\itemch $\mathrm{P}(B) = 60\% = 0{,}6$.
	$\mathrm{P}(\overline{B}) = 1-0{,}6 = 0{,}4$.
	\itemch $\mathrm{P}(A|B) = 20\% = 0{,}2$.
	\itemch $\mathrm{P}(A|\overline{B}) = 15\% = 0{,}15$. 
	\itemch Áp dụng công thức xác suất toàn phần, ta có
	$$\mathrm{P}(A) = \mathrm{P}(B)\cdot \mathrm{P}(A|B) + \mathrm{P}(\overline{B})\cdot \mathrm{P}(A|\overline{B}) = 0{,}6\cdot0{,}2 + 0{,}4\cdot 0{,}15 = 0{,}18.$$
	Vậy xác suất để chọn được một học sinh biết chơi nhạc cụ là $0{,}18$ hay $18\%$.
	\end{itemchoice}
	}
\end{ex}

\begin{ex}%[2D6H2-4]
	Kết quả khảo sát tại một xã cho thấy có $20\%$ cư dân hút thuốc lá. Tỉ lệ cư dân thường xuyên gặp các vấn đề sức khoẻ về đường hô hấp trong số những người hút thuốc lá và không hút thuốc lá lần lượt là $70\%$, $15\%$. Giả sử ta gặp một cư dân của xã, gọi $A$ là biến cố \lq\lq  Người đó có hút thuốc lá\rq\rq\,và $B$ là biến cố \lq\lq  Người đó thường xuyên gặp các vấn đề sức khoẻ về đường hô hấp\rq\rq.
	\choiceTF
	{$\mathrm{P}(AB)=0{,}13$}
	{$\mathrm{P}(\overline{A}B)=0{,}14$}
	{\True Nếu ta gặp một cư dân của xã thì xác suất người đó thường xuyên gặp các vấn đề sức khoẻ về đường hô hấp là $0{,}26$}
	{\True Nếu ta gặp một cư dân của xã thường xuyên gặp các vấn đề sức khoẻ về đường hô hấp thì xác suất người đó có hút thuốc lá xấp xỉ $54\%$}
	\loigiai{
	Giả sử ta gặp một cư dân của xã, gọi $A$ là biến cố \lq\lq  Người đó có hút thuốc lá\rq\rq\,và $B$ là biến cố \lq\lq  Người đó thường xuyên gặp các vấn đề sức khoẻ về đường hô hấp\rq\rq.\\
	Khi đó, ta có $\mathrm{P}(A)=0{,}2$; $\mathrm{P}(\overline{A})=0{,}8$; $\mathrm{P}(B|A)=0{,}7$, $\mathrm{P}(B|\overline{A})=0{,}15$.
	\begin{itemchoice}
	\itemch $\mathrm{P}(AB)=\mathrm{P}(A)\cdot \mathrm{P}(B|A)=0{,}2\cdot 0{,}7=0{,}14$.
	\itemch $\mathrm{P}(\overline{A}B)=\mathrm{P}(\overline{A})\cdot \mathrm{P}(B|\overline{A})=0{,}15\cdot0{,}8=0{,}12$.
	\itemch	
	Ta có $\mathrm{P}(B) = \mathrm{P}(A)\cdot \mathrm{P}(B|A) + \mathrm{P}(\overline{A})\cdot \mathrm{P}(B|\overline{A}) = 0{,}14 + 0{,}12 = 0{,}26$.\\
	Vậy nếu ta gặp một cư dân của xã thì xác suất người đó thường xuyên gặp các vấn đề sức khoẻ về đường hô hấp là $26\%$.
	\itemch Theo công thức Bayes, ta có $\mathrm{P}(A|B) = \dfrac{\mathrm{P}(A)\mathrm{P}(B|A)}{\mathrm{P}(B)} =\dfrac{ 0{,}2\cdot 0{,}7}{0{,} 26}\approx 0{,}54$.\\
	Vậy nếu ta gặp một cư dân của xã thường xuyên gặp các vấn đề sức khoẻ về đường hô hấp thì xác suất người đó có hút thuốc lá là khoảng $54\%$.
	\end{itemchoice}
	}
\end{ex}

\begin{ex}%[2D6H2-3]%1
	Tỉ lệ người dân đã tiêm vắc xin phòng bệnh $A$ ở một địa phương là $75\%$. Trong số những người đã tiêm phòng, tỉ lệ mắc bệnh $A$ là $10\%$; trong số những người chưa tiêm phòng, tỉ lệ mắc bệnh $A$ là $32\%$. Chọn ngẫu nhiên một người ở địa phương đó. Gọi $A$ là biến cố: \lq\lq  Người được chọn đã tiêm vắc xin phòng bệnh\rq\rq \, và $B$ là biến cố: \lq\lq  Người được chọn mắc bệnh $A$\rq\rq.
	\choiceTF
	{ $\mathrm{P}(A)=0{,}25$}
	{\True $\mathrm{P}\left(B|A\right)=0{,}1$}
	{\True $\mathrm{P}\left(B|\overline{A}\right)=0{,}32$}
	{ \True $\mathrm{P}\left(\overline{A}|B\right)=\dfrac{16}{31}$}
	\loigiai{
	Vì tỉ lệ người dân đã tiêm vắc xin phòng bệnh $A$ ở địa phương là $75\%$ nên $\mathrm{P}(A)=0{,}75$ và $\mathrm{P}\left(\overline{A}\right)=0{,}25$.\\
	Vì tỉ lệ mắc bệnh $A$ trong số những người đã tiêm phòng là $10\%$ và trong số những người chưa tiêm phòng là $32\%$ nên $\mathrm{P}(B|A)=0{,}1$ và $\mathrm{P}\left(B|\overline{A}\right)=0{,}32$.\\
	Theo công thức Bayes
	\[
	\mathrm{P}\left(\overline{A}|B\right)=\dfrac{\mathrm{P}\left(\overline{A}\right)\cdot \mathrm{P}\left(B|\overline{A}\right)}{\mathrm{P}(A) \cdot \mathrm{P}(B|A)+\mathrm{P}(\overline{A}) \cdot \mathrm{P}\left(B|\overline{A}\right)}=\dfrac{0{,}25\cdot 0{,}32}{0{,}75\cdot 0{,}1+0{,}25\cdot 0{,}32}=\dfrac{16}{31}.
	\]
	}
\end{ex}

\begin{ex}%[2D6H2-2]%2
	Ở một địa phương, tỉ lệ nam và nữ là $2 : 3$. Số người mắc bệnh bạch tạng của địa phương này chiếm tỉ lệ $0{,}45\%$ dân cư. Biết tỉ lệ nữ giới mắc bệnh bạch tạng là $0{,}35\%$. Xét phép thử chọn ngẫu nhiên một người ở địa phương, gọi
	\begin{itemize}
	\item $A$ là biến cố \lq\lq  Người được chọn mắc bệnh bạch tạng\rq\rq;
	\item $B$ là biến cố \lq\lq  Người được chọn là nam\rq\rq.
	\end{itemize}
	\choiceTF
	{ $\mathrm{P}(B)=\dfrac{3}{5}$}
	{$\mathrm{P}(A)=0{,}35\%$}
	{\True $\mathrm{P}(A) = \mathrm{P}(B)\cdot \mathrm{P}(A|B) + \mathrm{P}\left(\overline{B}\right)\cdot \mathrm{P}\left(A|\overline{B}\right)$}
	{\True Tỉ lệ nam giới mặc bệnh bạch tạng bằng $0{,}65\%$}
	\loigiai{
	Ta có
	\[
	\mathrm{P}(B)=\dfrac{2}{5}; \quad \mathrm{P}\left(\overline{B}\right) =\dfrac{3}{5};
	\]
	\[
	\mathrm{P}(A)=0{,}45\%;\quad \mathrm{P}\left(A|\overline{B}\right)=0{,}35\%.
	\]
	Áp dụng công thức xác suất toàn phần, ta có
	\[
	\mathrm{P}(A) = \mathrm{P}(B)\cdot \mathrm{P}(A|B) + \mathrm{P}\left(\overline{B}\right)\cdot \mathrm{P}\left(A|\overline{B}\right).
	\] 
	Từ đó suy ra
	$$\mathrm{P}(A|B) =\dfrac{\mathrm{P}(A) - \mathrm{P}\left(\overline{B}\right)\cdot \mathrm{P}\left(A|\overline{B}\right)}{\mathrm{P}(B)}=\dfrac{0{,}45\%-\dfrac{3}{5}\cdot 0{,}35\%}{\dfrac{2}{5}}=0{,}65\%.$$
	Vậy tỉ lệ nam giới mắc bệnh bạch tạng của địa phương đó là $0{,}65\%$.
	}
\end{ex}

\begin{ex}%[2D6H2-3]%3
	Bạn Nam tham gia một gian hàng trò chơi dân gian trong hội xuân của trường. Trò chơi có hai lượt chơi. Xác suất để Nam thắng ở lượt chơi thứ nhất là $0{,}6$. Nếu Nam thắng ở lượt chơi thứ nhất thì xác suất Nam thắng ở lượt chơi thứ hai là $0{,}8 $. Ngược lại, nếu Nam thua ở lượt chơi thứ nhất thì xác suất Nam thắng ở lượt chơi thứ hai là $0{,}3$. Xét các biến cố
	\begin{itemize}
	\item $A$: \lq\lq  Nam thắng ở lượt chơi thứ nhất\rq\rq.
	\item $B$: \lq\lq  Nam thắng ở lượt chơi thứ hai\rq\rq.
	\end{itemize}
	\choiceTF
	{ $\mathrm{P}(A)=0{,}8$}
	{ $\mathrm{P}\left(B|A\right) = 0{,}6$}
	{\True $\mathrm{P}\left(B|\overline{A}\right) = 0{,}3$}
	{\True Xác suất Nam thắng ở lượt chơi thứ nhất khi đã thắng ở lượt chơi thứ hai là khoảng $80 \%$}
	\loigiai{
	Theo đề bài, ta có
	\begin{listEX}[3]
	\item $\mathrm{P}(A)=0{,}6$
	\item $\mathrm{P}\left(B|A\right) = 0{,}8$.
	\item $\mathrm{P}\left(B|\overline{A}\right) = 0{,}3$.
	\end{listEX}
	Áp dụng công thức Bayes, ta có
	\begin{eqnarray*}
	\mathrm{P}(A | B) &= & \dfrac{\mathrm{P}(B | A) \cdot \mathrm{P}(A)}{\mathrm{P}(B | A) \cdot \mathrm{P}(A)+\mathrm{P}(B | \overline{A}) \cdot \mathrm{P}(\overline{A})} \\
	& = & \dfrac{0{,}8 \cdot 0{,}6}{ 0{,}8 \cdot 0{,}6 + 0{,}3 \cdot 0{,}4 } \\
	& \approx & 0{,}8.
	\end{eqnarray*}
	Vậy xác suất Nam thắng ở lượt chơi thứ nhất khi đã thắng ở lượt chơi thứ hai là khoảng $0{,}8$ hay $80 \%$.
	}
\end{ex}

\begin{ex}%[2D6V2-4]%4
	Một loại linh kiện do hai nhà máy số I, số II cùng sản xuất. Tỉ lệ phế phẩm của các nhà máy I, II lần lượt là $4\%$; $3\%$. Trong một lô linh kiện để lẫn lộn $80$ sản phẩm của nhà máy số I và $120$ sản phẩm của nhà máy số II. Một khách hàng lấy ngẫu nhiên một linh liện từ lô hàng đó. Xét các biến cố sau
	\begin{itemize}
	\item $A$: \lq\lq  Linh kiện lấy ra là linh kiện tốt\rq\rq.
	\item $B_1$: \lq\lq  Linh kiện lấy ra là linh kiện từ nhà máy số I\rq\rq.
	\item $B_2$: \lq\lq  Linh kiện lấy ra là linh kiện từ nhà máy số II\rq\rq.
	\end{itemize}
	\choiceTF
	{\True $\mathrm{P}(B_1)=0{,}4$}
	{\True $\mathrm{P}\left(A|B_2\right) = 0{,}97$}
	{\True $\mathrm{P}(A) = 0{,}966$}
	{Nếu linh kiện được lấy ra là linh kiện phế phẩm thì xác suất linh kiện đó do nhà máy II sản xuất là cao nhất}
	\loigiai{
	Theo đề bài, ta có
	\begin{listEX}[2]
	\item $\mathrm{P}(A|B_1) =1 - 0{,}04 = 0{,}96$.
	\item $\mathrm{P}\left(A|B_2\right) = 1-0{,}03 = 0{,}97$.
	\item $\mathrm{P}(B_1)=\dfrac{80}{200}=0{,}4$.
	\item $\mathrm{P}\left(B_2\right) = \dfrac{120}{200}=0{,}6$.
	\end{listEX}
	Khi đó áp dụng công thức xác suất toàn phần, ta có
	\[\mathrm{P}(A) = \mathrm{P}(A|B_1)\cdot \mathrm{P}(B_1) + \mathrm{P}\left(A|B_2\right)\cdot\mathrm{P}\left(B_2\right)=0{,}96\cdot 0{,}4 + 0{,}97\cdot 0{,}6=0{,}966.\]
	Ta có $\mathrm{P}\left(\overline{A}\right)=1-\mathrm{P}(A) = 0{,}034$.\\
	Áp dụng công thức Bayes, ta có
	\begin{itemize}
	\item $\mathrm{P}\left(B_1|\overline{A}\right)=\dfrac{\mathrm{P}\left(\overline{A}|B_1\right)\cdot \mathrm{P}(B_1)}{\mathrm{P}\left(\overline{A}\right)} = \dfrac{0{,}04\cdot 0{,}4}{0{,}034} = \dfrac{8}{17}\approx 0{,}048$.
	\item $\mathrm{P}\left(B_2|\overline{A}\right)=\dfrac{\mathrm{P}\left(\overline{A}|B_2\right)\cdot \mathrm{P}(B_2)}{\mathrm{P}\left(\overline{A}\right)} = \dfrac{0{,}03\cdot 0{,}6}{0{,}034} = \dfrac{8}{167}\approx 0{,}054$.
	\end{itemize}
	Vậy với điều kiện linh kiện lấy ra là linh kiện phế phẩm thì xác suất linh kiện đó do nhà máy I sản xuất là cao nhất.
	}
\end{ex}

\begin{ex}%[2D6H2-4]
	Xác suất để một chuyến bay khởi hành đúng giờ là $\mathrm{P}(D)=0{,}83$; xác suất để nó đến đúng giờ là $\mathrm{P}(A)=0{,}82$; xác suất để nó khởi hành và đến đều đúng giờ là $\mathrm{P}(D \cap A)=0{,}78$. 
	\choiceTF
	{\True Xác suất để một máy bay đến đúng giờ biết rằng nó đã khởi hành đúng giờ là $0{,}94$}
	{Xác suất để một máy bay khởi hành đúng giờ biết rằng nó sẽ đến đúng giờ là $0{,}85$}
	{\True Xác suất để một máy bay đến đúng giờ biết rằng nó khởi hành không đúng giờ là $0{,}24$}
	{Xác suất để một máy bay khởi hành đúng giờ biết rằng nó sẽ đến không đúng giờ là $0{,}95$}
	\loigiai{
	Ta có $\mathrm{P}(A \cap \overline{D})=\mathrm{P}(A)-\mathrm{P}(A \cap D)=0{,}82-0{,}78=0{,}04$.\\
	$\mathrm{P}(D \cap \overline{A})=\mathrm{P}(D)-\mathrm{P}(D \cap A)=0{,}83-0{,}78=0{,}05$.
	\begin{itemchoice}
	\itemch Xác suất để một máy bay đến đúng giờ biết rằng nó đã khởi hành đúng giờ là
	$$
	\mathrm{P}(A \mid D)=\frac{\mathrm{P}(D \cap A)}{\mathrm{P}(D)}=\frac{0{,}78}{0{,}83}=0{,}94.
	$$
	\itemch Xác suất để một máy bay khởi hành đúng giờ biết rằng nó đã đến đúng giờ là
	$$
	\mathrm{P}(D \mid A)=\frac{\mathrm{P}(D \cap A)}{\mathrm{P}(A)}=\frac{0{,}78}{0{,}82}=0{,}95.
	$$
	\itemch Xác suất để máy bay đến đúng giờ khi nó khởi hành không đúng giờ là
	$$
	\mathrm{P}(A \mid \overline{D})=\dfrac{\mathrm{P}(A \cap \overline{D})}{\mathrm{P}(\overline{D})}=\dfrac{0{,}82-0{,}78}{0{,}17}=0{,}24.
	$$
	\itemch Xác suất để một máy bay khởi hành đúng giờ biết rằng nó sẽ đến đúng giờ là $$
	\mathrm{P}(D \mid \overline{A})=\dfrac{\mathrm{P}(D \cap \overline{A})}{\mathrm{P}(\overline{A})}=\frac{0{,}83-0{,}78}{1-0{,}82}=0{,}28.
	$$
	\end{itemchoice}
	}
\end{ex}

\begin{ex}%[2D6H2-4]
	Cho hai biến cố $A$, $B$ sao cho $\mathrm{P}(A)=0{,}6$; $\mathrm{P}(B)=0{,}4$; $\mathrm{P}(A \mid B)=0{,}3$. 
	\choiceTF
	{\True $\mathrm{P}(B \mid A)=0{,}2$}
	{\True $\mathrm{P}(A \mid \overline{B})=0{,}8$}
	{$\mathrm{P}(B \mid \overline{A})=0{,}8$}
	{$\mathrm{P}(B \cap A)=0{,}24$}
	\loigiai{\begin{itemchoice}
	\itemch	Áp dụng công thức Bayes, ta có
	$$
	\mathrm{P}(B \mid A)=\dfrac{\mathrm{P}(B) \cdot \mathrm{P}(A \mid B)}{\mathrm{P}(A)}=\frac{0{,}4 \cdot 0{,}3}{0{,}6}=0{,}2.
	$$
	\itemch Ta có
	$\mathrm{P}(A)=\mathrm{P}(B)\cdot \mathrm{P}(A \mid B) + \mathrm{P}(\overline{B})\cdot \mathrm{P}(A \mid \overline{B})$.\\
	Suy ra $\mathrm{P}(A \mid \overline{B})=\dfrac{0{,}6-0{,}4\cdot0{,}3}{1-0{,}4}=0{,}8$.
	\itemch Ta có
	$\mathrm{P}(B)=\mathrm{P}(A)\cdot \mathrm{P}(B \mid A) + \mathrm{P}(\overline{A})\cdot \mathrm{P}(B \mid \overline{A})$.\\
	Suy ra $\mathrm{P}(B \mid \overline{A})=\dfrac{0{,}4-0{,}6\cdot0{,}2}{1-0{,}6}=0{,}7$.
	\itemch Ta có $\mathrm{P}(B \cap A)=\mathrm{P}(A\mid B)\cdot \mathrm{P}(B)=0{,}3\cdot 0{,}4=0{,}12$.
	\end{itemchoice}
	}
\end{ex}

\begin{ex}%%[2D6V2-3]Câu 1
Có $7$ hộp bi, trong đó có $4$ hộp loại $1$, $3$ hộp loại $2$. Mỗi hộp loại $1$ có $3$ bi trắng và $5$ bi đỏ, mỗi hộp loại $2$ có $4$ bi trắng và $6$ bi đỏ. Chọn ngẫu nhiên $1$ hộp và từ đó lấy ra $1$ bi thì được bi trắng. Tìm xác suất để bi lấy ra này thuộc hộp loại $2$. (Làm tròn đến kết quả hàng phần trăm).
	\shortans{$0{,}44$}
	\loigiai{ 
	Gọi $B$ là biến cố \lq\lq  lấy được bi trắng\rq\rq.\\
	$A_1$ là biến có lấy hộp loại 1.\\
	$A_2$ là biến có lấy hộp loại 2.\\
	$\mathrm{P}(A_1)=\dfrac{ \mathrm{C} _{4}^{1}}{ \mathrm{C} _{7}^{1}}=\dfrac{4}{7}$;
	$\mathrm{P}(A_2)=\dfrac{ \mathrm{C} _{3}^{1}}{ \mathrm{C} _{7}^{1}}=\dfrac{3}{7}$;
	$\mathrm{P}(B|A_1)=\dfrac{ \mathrm{C} _{3}^{1}}{ \mathrm{C} _{8}^{1}}=\dfrac{3}{8}$;
	$\mathrm{P}(B|A_2)=\dfrac{ \mathrm{C} _{4}^{1}}{ \mathrm{C} _{10}^{1}}=\dfrac{4}{10}=\dfrac{2}{5}$.\\
	$\Rightarrow \mathrm{P}(B)=\mathrm{P}(A_1)\mathrm{P}(B|A_1)+\mathrm{P}(A_2)\mathrm{P}(B|A_2)=\dfrac{4}{7}\cdot\dfrac{3}{8}+\dfrac{3}{7}\cdot\dfrac{2}{5}=\dfrac{27}{70}$.\\
	Xác suất để bi lấy ra này thuộc hộp loại $2$ là\\
	$\mathrm{P}(A_2|B)=\dfrac{\mathrm{P}(A_2)\mathrm{P}(B|A_2)}{\mathrm{P}(B)}=\dfrac{\dfrac{3}{7}\cdot\dfrac{2}{5}}{\dfrac{27}{70}}=\dfrac{4}{9}\approx 0{,}44$.
	}
\end{ex}

\begin{ex}%[2D6V2-2]Câu 2
Trong $12$ xạ thủ có $5$ người bắn trúng hồng tâm với xác suất $0{,}8$; $7$ người bắn trúng hồng tâm với xác suất $0{,}7$. Chọn ngẫu nhiên $1$ xạ thủ. Tìm xác suất để người này bắn trúng hồng tâm. (Làm tròn đến kết quả hàng phần trăm).
	\shortans{$0{,}74$}
	\loigiai{ Gọi $A$ là biến cố \lq\lq  Xạ thủ bắn trúng hồng tâm\rq\rq.\\
	$A_1$ là biến cố \lq\lq  nhóm $5$ người bắn trúng hồng tâm\rq\rq.\\
	$A_2$ là biến cố \lq\lq  nhóm $7$ người bắn trúng hồng tâm\rq\rq.\\
	$\mathrm{P}(A_1)=\dfrac{5}{12}$; $\mathrm{P}(A_2)=\dfrac{7}{12}$;
	$\mathrm{P}(A|A_1)=0{,}8$; $P(A|A_2)=0{,}7$.\\
	Xác suất để người này bắn trúng hồng tâm
	$$\mathrm{P}(A)=\mathrm{P}(A_1)\mathrm{P}(A|A_1)+\mathrm{P}(A_1)P(A|A_2)=\dfrac{5}{12}\cdot0{,}8+\dfrac{7}{12}\cdot0{,}7=\dfrac{89}{120}\approx 0{,}74.$$
	}
\end{ex}

\begin{ex}%[2D6V2-2]Câu 3
 Có $3$ hộp phấn. Hộp thứ nhất có $7$ viên trắng và $3$ viên vàng; hộp thứ hai có $16$ viên trắng và $4$ viên vàng; hộp thứ $3$ có $22$ viên trắng và $8$ viên vàng. Ta tung đồng thời $3$ đồng xu cân đối và đồng chất: nếu được cả $3$ mặt sấp thì chọn hộp thứ nhất; nếu được $1$ mặt sấp và $2$ mặt ngửa thì chọn hộp thứ hai; trường hợp còn lại thì chọn hộp thứ ba. Từ hộp đã chọn ta lấy ngẫu nhiên ra $1$ viên phấn. Tính xác suất để lấy được viên phấn trắng. (Làm tròn đến kết quả hàng phần trăm).
	\shortans{$0{,}75$}
	\loigiai{ 
	Gọi $A$ là biến cố \lq\lq  Lấy được một viên phấn trắng\rq\rq.\\
	 $A_1$ là biến cố \lq\lq  Chọn hộp 1\rq\rq.\\
	 $A_2$ là biến cố \lq\lq  Chọn hộp 2\rq\rq.\\
	 $A_3$ là biến cố \lq\lq  Chọn hộp 3\rq\rq.\\
	 $\mathrm{P}(A_1)=\left ( \dfrac{1}{2} \right )^3$.\\ 
	 $\mathrm{P}(A_2)=3\cdot\dfrac{1}{2}\left ( \dfrac{1}{2} \right )^2=\dfrac{3}{8}$.\\ 
 $\mathrm{P}(A_3)=1-\dfrac{3}{8}-\dfrac{1}{8}=\dfrac{1}{2}$.\\	 
	 	 	$\mathrm{P}(A|A_1)=\dfrac{7}{10}$; $\mathrm{P}(A|A_2)=\dfrac{16}{20}=\dfrac{4}{5}$;$\mathrm{P}(A|A_3)=\dfrac{22}{30}=\dfrac{11}{15}$.\\
	 	 	Xác suất để lấy được viên phấn trắng
	 	 	\begin{eqnarray*}
	 	 	P(A)	& =& \mathrm{P}(A_1)\mathrm{P}(A|A_1)+\mathrm{P}(A_2)\mathrm{P}(A|A_2)+\mathrm{P}(A_3)\mathrm{P}(A|A_3)\\
	 	 	&= & \dfrac{1}{8}\cdot\dfrac{7}{10}+\dfrac{3}{8}\cdot\dfrac{4}{5}+\dfrac{1}{2}\cdot\dfrac{11}{5}\\
	 	 	& =& \dfrac{181}{240}\\
	 	 	&\approx &0{,75}.
	 	 	\end{eqnarray*}
	}
\end{ex}

\begin{ex}%[2D6C2-3]Câu 5
Một loại linh kiện do $3$ nhà máy số $I$, số $II$, số $III$ cùng sản xuất. Tỷ lệ phế phẩm của các nhà máy lần lượt là: $I$: $0{,}04$; $II$: $0{,}03$ và $III$: $0{,}05$. Trong $1$ lô linh kiện để lẫn lộn $80$ sản phẩm của nhà máy số $I$, $120$ của nhà máy số $II$ và $100$ của nhà máy số $III$. Khách hàng lấy phải một linh kiện loại phế phẩm từ lô hàng đó. Khả năng linh kiện đó do nhà máy nào sản xuất là cao nhất? (Làm tròn đến kết quả hàng phần trăm).
\shortans{$3$}
\loigiai{
	Gọi $E_1$ là biến cố \lq\lq  phế phẩm máy số $I$\rq\rq $\Rightarrow \mathrm{P}(E_1)=0{,}04,\mathrm{P}(\overline{E_1})=0{,}96$.\\
	Gọi $E_2$ là biến cố \lq\lq  phế phẩm máy số $II$\rq\rq $\Rightarrow \mathrm{P}(E_2)=0{,}03,\mathrm{P}(\overline{E_2})=0{,}97$.\\
	Gọi $E_3$ là biến cố \lq\lq  phế phẩm máy số $III$\rq\rq $\Rightarrow \mathrm{P}(E_1)=0{,}05,\mathrm{P}(\overline{E_3})=0{,}95$\\
	Gọi $B$ là biến cố khách hàng lấy được $1$ linh kiện tốt\\
	$\mathrm{P}(B|\overline{E_1})=\dfrac{\mathrm{C}_{80}^{1}}{\mathrm{C}_{300}^{1}}=\dfrac{4}{15}$;
	$\mathrm{P}(B|\overline{E_2})=\dfrac{\mathrm{C}_{120}^{1}}{\mathrm{C}_{300}^{1}}=\dfrac{2}{5}$;
	$\mathrm{P}(B|\overline{E_3})=\dfrac{\mathrm{C}_{100}^{1}}{\mathrm{C}_{300}^{1}}=\dfrac{1}{3}$.\\
	Xác suất để khách hàng lấy được linh kiện tốt là\allowdisplaybreaks
	\begin{eqnarray*}
	\mathrm{P}(B)&=&\mathrm{P}(\overline{E_1})\mathrm{P}(B|\overline{E_1})+P(\overline{E_2})\mathrm{P}(B|\overline{E_2})+\mathrm{P}(\overline{E_3})\mathrm{P}(B|\overline{E_3})\\
	&=&0{,}96\cdot\dfrac{4}{15}+0{,}97\cdot\dfrac{2}{5}+0{,}95\cdot\dfrac{1}{3}\approx 0{,}96.
	\end{eqnarray*}
	Gọi $\overline{B}$ là biến cố khách hàng lấy $1$ linh kiện loại không tốt\\
	 $\mathrm{P}(\overline{B})=1-\mathrm{P}(B)=0{,}04$.\\
	$\mathrm{P}(E_1|\overline{B})=\dfrac{\mathrm{P}(E_1)P(\overline{B}|E_1)}{\mathrm{P}(\overline{B})}=\dfrac{0{,}04 \cdot \dfrac{\mathrm{C}_{80}^{1}}{\mathrm{C}_{300}^{1}}}{0{,}04}=0{,}26$.\\
	$\mathrm{P}(E_2|\overline{B})=\dfrac{\mathrm{P}(E_2)\mathrm{P}(\overline{B}|E_2)}{\mathrm{P}(\overline{B})}=\dfrac{0{,}03 \cdot \dfrac{\mathrm{C}_{120}^{1}}{\mathrm{C}_{300}^{1}}}{0{,}04}=0{,}3$.\\
	$\mathrm{P}(E_3|\overline{B})=\dfrac{\mathrm{P}(E_3)P(\overline{B}|E_3)}{\mathrm{P}(\overline{B})}=\dfrac{0{,}05 \cdot \dfrac{\mathrm{C}_{100}^{1}}{\mathrm{C}_{300}^{1}}}{0{,}04}=0{,}41$.\\
	Vậy linh kiện đó do máy $III$ là cao nhất.
	}
\end{ex}

\begin{ex}%[2D6V2-3]Câu 6
	Trong $1$ đám đông, số người nam bằng số người nữ. Xác suất mắc cận thị của nam là $0{,}4$ và nữ là $0{,}6$. Chọn ngẫu nhiên $1$ người. Xác suất chọn được nam không cận thị. (Làm tròn đến kết quả hàng phần trăm).
	\par\shortans{$0{,}6$}
	\loigiai{ Goi tiêp $H_1$ là biến cố chọn nam; $H_2$ là biến cố chọn được nữ.\\
	$C$ là biến cố người được chọn là không cận thị.\\
	Ta có $\mathrm{P}(H_1)=\mathrm{P}(H_2)=\dfrac{1}{2}$\\
	 $\mathrm{P}(C)=\mathrm{P}(H_1)\mathrm{P}(C|H_1)+\mathrm{P}(H_2)\mathrm{P}(C|H_2)=\dfrac{1}{2}\cdot0{,}6+\dfrac{1}{2}\cdot0{,}4=0{,}5$.\\
	Xác suất để chọn ngẫu nhiên ra $1$ người mà người đó là nam không cận thị là\\ (Áp dụng định lý Bayes)\\
	$\mathrm{P}(H_1|C)=\dfrac{\mathrm{P}(H_1)\mathrm{P}(C|H_1)}{\mathrm{P}(C)}=\dfrac{\dfrac{1}{2}\cdot0{,}6}{0{,}5}=0{,}6$.}
	\end{ex}

\begin{ex}%[2D6H2-2]
	Khi phát hiện một vật thể bay, xác suất một hệ thống radar phát 
	cảnh báo là $0{,}9$ nếu vật thể bay đó là mục tiêu thật và là $0{,}05$ 
	nếu đó là mục tiêu giả. Có $99\%$ các vật thể bay là mục tiêu giả. 
	Tính xác suất để radar phát hiện cảnh báo khi phát hiện một vật thể bay (Làm tròn đến hàng phần trăm).
	\shortans{$0{,}96$}
	\loigiai{
	Gọi $ A $ là biến cố \lq\lq  Hệ thống radar phát cảnh báo\rq\rq\, và $ B $ là biến cố \lq\lq  Vật thể bay là mục tiêu thật\rq\rq.\\
	Do xác suất một hệ thống radar cảnh báo nếu vật thể bay là mục tiêu thật là $ 0{,}9 $ nên $ \mathrm{P}(A|B)=0{,}9 $. \\
	Do xác suất một hệ thống radar cảnh báo nếu vật thể bay là mục tiêu giả là $ 0{,}05 $ nên $ \mathrm{P}(A|\overline{B})=0{,}05 $. \\
	Do có $ 99\% $ các vật thể bay là mục tiêu giả nên $ \mathrm{P}(\overline{B})=0{,}99 $ và $ \mathrm{P}(B)= 0{,}01$.\\
	Áp dụng công thức xác suất toàn phần, ta có xác suất để hệ thống radar phát cảnh báo là
	$$ \mathrm{P}(A)=\mathrm{P}(B)\mathrm{P}(A|B)+\mathrm{P}(\overline{B})\mathrm{P}(A|\overline{B})=0{,}01 \cdot 0{,}9+0{,}99 \cdot 0{,}05= 0{,}9595.$$
	}
\end{ex}

\begin{ex}%[2D6H2-2]
	Một loại vaccine được tiêm ở địa phương X. Người có bệnh nền thì với xác suất $0{,}35$ có phản ứng phụ sau tiêm, người không có bệnh nền thì chỉ có phản ứng phụ sau tiêm với xác suất $0{,}16$. Chọn ngẫu nhiên một người được tiêm vaccine. Tính xác suất người này có phản ứng phụ, biết rằng tỉ lệ người có bệnh nền ở địa phương X là $18\%$. (Làm tròn đến hàng phần trăm).
	\shortans{$0{,}19$}
	\loigiai{
	Gọi $A$ là biến cố \lq\lq  Người được chọn có bệnh nền\rq\rq\, và $B$ là biến cố \lq\lq  Người này có phản ứng phụ sau tiêm\rq\rq.\\
	Ta có $\mathrm{P}(A)=0{,}18$; $\mathrm{P}(\overline{A})=0{,}82$.\\
	$\mathrm{P}(B\mid A)$ là xác suất để một người bệnh có phản ứng sau tiêm với điều kiện có bệnh nền, suy ra $\mathrm{P}(B\mid A)=0{,}35$.\\
	$\mathrm{P}(B\mid \overline{A})$ là xác suất để một người bệnh có phản ứng sau tiêm với điều kiện không có bệnh nền, suy ra $\mathrm{P}(B\mid \overline{A})=0{,}16$.\\
	Theo công thức xác suất toàn phần, ta được 
	$$\mathrm{P}(B)=\mathrm{P}(A)\cdot \mathrm{P}(B\mid A)+\mathrm{P}(\overline{A})\cdot \mathrm{P}(B\mid \overline{A})=0{,}18\cdot 0{,}35+0{,}82\cdot 0{,}16=\dfrac{971}{5\,000}\approx 0{,}19.$$
	}
\end{ex}

\begin{ex}%[2D5V2-2]
	Có hai chuồng thỏ. Chuồng I có $5$ con thỏ đen và $10$ con thỏ trắng. Chuồng II có $7$ con thỏ đen và $3$ con thỏ trắng. Trước tiên, từ chuồng II lấy ra ngẫu nhiên $1$ con thỏ rồi cho vào chuồng I. Sau đó, từ chuồng I lấy ra ngẫu nhiên $1$ con thỏ. Tính xác suất để con thỏ được lấy ra là con thỏ trắng (Làm tròn đến hàng phần trăm).
	\par\shortans{$0{,}64$}
	\loigiai{Xét biến cố $A$ \lq\lq  Con thỏ được lấy ra từ chuồng II để cho vào chuồng I là con thỏ trắng\rq\rq.\\
	Xét biến cố $B$ \lq\lq  Con thỏ được lấy ra từ chuồng I là con thỏ trắng\rq\rq. \\	
	Ta có $\mathrm{P}(B)=\mathrm{P}(A)\cdot \mathrm{P}(B \mid A)+\mathrm{P}(\overline{A}) \cdot \mathrm{P}(B \mid \overline{A})$.
	\begin{itemize}
	\item Tính $\mathrm{P}(A)$: Đây là xác suất để lấy ra ngẫu nhiên 1 con thỏ trắng từ chuồng II rồi cho vào chuồng I. Có $n\left(\Omega\right)=\mathrm{C}^1_{10}$, $n\left(A\right)=\mathrm{C}^1_3$. Vậy $\mathrm{P}(A)=\dfrac{3}{10}$.
	\item Tính $\mathrm{P}(\overline{A})$: $\mathrm{P}(\overline{A})=1-\mathrm{P}(A)=\dfrac{7}{10}$.
	\item Tính $\mathrm{P}(B\mid A)$: Đây là xác suất để lấy ra ngẫu nhiên 1 con thỏ trắng từ chuồng I với điều kiện đã chọn ra 1 con thỏ trắng từ chuồng II rồi cho vào chuồng I, tức là có 5 con thỏ đen và 11 con thỏ trắng ở trong chuồng I. Tương tự như trên ta có $\mathrm{P}(B\mid A)=\dfrac{11}{16}$.
	\item Tính $\mathrm{P}(B\mid \overline{A})$: Đây là để lấy ra ngẫu nhiên 1 con thỏ trắng từ chuồng I với điều kiện đã chọn ra 1 con thỏ đen từ chuồng II rồi cho vào chuồng I, tức là có 6 con thỏ đen và 10 con thỏ trắng ở trong chuồng I. Tương tự như trên ta có $\mathrm{P}(B\mid \overline{A})=\dfrac{10}{16}$.
	\end{itemize}
	Vậy $\mathrm{P}(B)=\mathrm{P}(A)\cdot \mathrm{P}(B \mid A)+\mathrm{P}(\overline{A}) \cdot \mathrm{P}(B \mid \overline{A})=\dfrac{3}{10}\cdot \dfrac{11}{16}+\dfrac{7}{10}\cdot \dfrac{10}{16}=\dfrac{103}{160}=0{,}64375$. \\
	Vậy xác suất để con thỏ được lấy ra là con thỏ trắng xấp xỉ là $0{,}64$.
	}
\end{ex}

\begin{ex}%[2D6V2-3]
	Người ta điều tra thấy ở một địa phương nọ có $2\%$ tài xế sử dụng điện thoại di động khi lái xe. Trong các vụ tai nạn ở địa phương đó, người ta nhận thấy có $10\%$ là do tài xế có sử dụng điện thoại khi lái xe gây ra. Hỏi việc sử dụng điện thoại di động khi lái xe làm tăng xác suất gây tai nạn lên bao nhiêu lần?
	\par\shortans{$5$}
	\loigiai{
	Gọi $ A $ là biến cố \lq\lq  Tài xế sử dụng điện thoại di động khi lái xe\rq\rq\,và $ B $ là biến cố \lq\lq  Địa phương có tai nạn\rq\rq.\\
	Do có $ 2\% $ tài xế sử dụng điện thoại di động khi lái xe nên $ \mathrm{P}(A)=0{,}02 $.\\
	Do trong các vụ tai nạn ở địa phương, có $ 10\% $ là do tài xế có sử dụng điện thoại khi lái xe nên $ \mathrm{P}(A|B)=0{,}1 $.
	Xác suất gây tai nạn do tài xế sử dụng điện thoại di động khi lái xe là
	$$ \mathrm{P}(B|A)= \dfrac{\mathrm{P}(B)\mathrm{P}(A|B)}{\mathrm{P}(A)}$$
	Do đó $ \dfrac{\mathrm{P}(B|A)}{\mathrm{P}(B)}=\dfrac{\mathrm{P}(A|B)}{\mathrm{P}(A)}=\dfrac{0{,}1}{0{,}02}= 5$. Suy ra $ \mathrm{P}(B|A)=5\mathrm{P}(B) $.\\
	Vậy việc sử dụng điện thoại di động khi lái xe làm tăng xác suất gây tai nạn lên $5$ lần.
	}
\end{ex}

\begin{ex}%[2D6C2-4]
	Tỉ lệ người dân đã tiêm vắc xin phòng bệnh A ở một địa phương là $65\%$. Trong số những người đã tiêm phòng, tỉ lệ mắc bệnh A là $5\%$ còn trong số những người chưa tiêm, tỉ lệ mắc bệnh A là $17\%$. Gặp ngẫu nhiên một người ở địa phương đó. Biết rằng người đó mắc bệnh X. Khi đó xác suất người đó không tiêm vắc xin phòng bệnh X có dạng $\dfrac{a}{b}$. Giá trị $b-a$ là?
	\shortans{$65$}
	\loigiai{
	Gọi $A$ là biến cố \lq\lq  người đó mắc bệnh $X$\rq\rq\,và $B$ là biến cố \lq\lq  Gặp được người đã tiêm vắc xin phòng bệnh X\rq\rq.\\
	Theo công thức xác suất toàn phần, ta có
	\begin{eqnarray*}
	\mathrm{P}(A) & = &\mathrm{P}(B) \cdot \mathrm{P}(A \mid B)+\mathrm{P}(\overline{B}) \cdot \mathrm{P}(A \mid \overline{B}) \\
	& = &0,65 \cdot 0{,}05+0{,}35 \cdot 0{,}17=0{,}092.
	\end{eqnarray*}
	Suy ra
	\begin{eqnarray*}
	\mathrm{P}(\overline{B} \mid A) & =& \dfrac{\mathrm{P}(A \overline{B})}{\mathrm{P}(A)}=\dfrac{\mathrm{P}(\overline{B}) \mathrm{P}(A \mid \overline{B})}{\mathrm{P}(A)} \\
	& =& \dfrac{0,35 \cdot 0{,}17}{0{,}092}=\dfrac{119}{184}.
	\end{eqnarray*}
	Khi đó $a=119$ và $b=184$, suy ra $b-a=65$.
	}
\end{ex}

\begin{ex}%[2D6C2-4]
	Ở một khu rừng nọ có $7$ chú lùn, trong đó có $4$ chú luôn nói thật, $3$ chú còn lại nói thật với xác suất $0{,}5$. Bạn Tuyết gặp ngẫu nhiên một chú lùn. Gọi $A$ là biến cố \lq\lq  Chú lùn đó luôn nói thật\rq\rq\,và $B$ là biến cố \lq\lq  Chú lùn đó tự nhận mình luôn nói thật\rq\rq. Biết rằng chú lùn mà bạn Tuyết gặp tự nhận mình là người luôn nói thật. Tính xác suất để chú lùn đó luôn nói thật (làm tròn hai chữ số thập phân).
	\shortans{$0{,}73$}
	\loigiai{
	Ta có $\mathrm{P}(A)=\dfrac{4}{7}$; $\mathrm{P}(\overline{A})=\dfrac{3}{7}$; $\mathrm{P}(B \mid A)=1$; $\mathrm{P}(B \mid \overline{A})=0{,}5$.\\
	Theo công thức xác suất toàn phần, ta có
	\begin{eqnarray*}
	\mathrm{P}(B) & =& \mathrm{P}(A) \cdot \mathrm{P}(B \mid A)+\mathrm{P}(\overline{A}) \cdot \mathrm{P}(B \mid \overline{A}) \\
	& = & \dfrac{4}{7} \cdot 1+\dfrac{3}{7} \cdot 0{,}5=\dfrac{11}{14}.
	\end{eqnarray*}
	Khi đó
	$$
	\mathrm{P}(A \mid B)=\dfrac{\mathrm{P}(AB)}{\mathrm{P}(B)}=\dfrac{\mathrm{P}(A) \cdot \mathrm{P}(B \mid A)}{\mathrm{P}(B)}=\dfrac{\dfrac{4}{7} \cdot 1}{\dfrac{11}{14}}=\dfrac{8}{11} \approx 0{,}73.
	$$
	}
\end{ex}
\Closesolutionfile{ans}
% \section*{ÔN TẬP CHƯƠNG VI}
\subsubsection{Bài tập tự luận}
\setcounter{bt}{0}
\begin{bt}%[2D5H2-1]
	Một khu dân cư có $85 \%$ các hộ gia đình sử dụng điện để đun nấu. Hơn nữa, có $21 \%$ các hộ gia đình sử dụng bếp từ để đun nấu. Chọn ngẫu nhiên một hộ gia đình, tính xác suất hộ đó sử dụng bếp từ để đun nấu, biết hộ đó sử dụng điện để đun nấu.
	\loigiai{
	Gọi $A$ là biến cố \lq\lq  Hộ gia đình sử dụng điện để đun nấu\rq\rq, $B$ là biến cố \lq\lq  Hộ gia đình sử dụng bếp từ để đun nấu\rq\rq.\\
	Ta cần tính $\mathrm{P} (B \mid A)$.
	\begin{itemize}
	\item Do có $85 \%$ các hộ gia đình sử dụng điện để đun nấu nên $\mathrm{P} (A)=0{,}85$.
	\item Do trong các hộ gia đình sử dụng điện để đun nấu, có $21 \%$ các hộ gia đình sử dụng bếp từ để đun nấu nên $\mathrm{P} (AB)=0{,}21$.
	\item Vậy
	$\mathrm{P} (B \mid A)= \dfrac{\mathrm{P}(AB)}{\mathrm{P}(A)}=\dfrac{0{,}21}{0{,}85}=\dfrac{21}{85}.$
	\end{itemize}
	}
\end{bt}

\begin{bt}%[2D5H2-1]
	Cho hai biến cố ngẫu nhiên $A$ và $B$. Biết rằng $\mathrm{P} (A \mid B)=2 \mathrm{P}(B \mid A)$ và $\mathrm{P}(A B) \neq 0$.
	Tính tỉ số $\dfrac{\mathrm{P}(A)}{\mathrm{P}(B)}$.
	\loigiai{Ta có $$\mathrm{P} (A \mid B)=2 \mathrm{P}(B \mid A)\Leftrightarrow \dfrac{\mathrm{P}(AB)}{\mathrm{P}(B)}=2\dfrac{\mathrm{P}(AB)}{\mathrm{P}(A)}\Leftrightarrow \dfrac{\mathrm{P} (A)}{\mathrm{P}(B)} =2 .$$
}
\end{bt}

\begin{bt}%[2D5V2-3]
	Phòng công nghệ của một công ty có $4$ kĩ sư và $6$ kĩ thuật viên. Chọn ra ngẫu nhiên đồng thời $3$ người từ phòng. Tính xác suất để cả $3$ người được chọn đều là kĩ sư, biết rằng trong $3$ người được chọn có ít nhất $2$ kĩ sư.
	\loigiai{
	Để giải bài toán này, ta sử dụng công thức Bayes:
	$$
	\mathrm{P}(A \mid B)=\frac{\mathrm{P}(B \mid A) \mathrm{P}(A)}{\mathrm{P}(B)}
	$$
	Trong đó
	\begin{itemize}
	\item $A$ là biến cố \lq\lq  Cả 3 người được chọn đều là kĩ sư\rq\rq .
	\item $B$ là là biến cố \lq\lq  Trong 3 người được chọn có ít nhất 2 kĩ sư\rq\rq .
	\item $\mathrm{P}(A \mid B)$ là xác suất cần tìm.
	\end{itemize}
	Ta có
	\begin{itemize}
	\item $\mathrm{P}(A)=\dfrac{\mathrm{C} _4^3}{\mathrm{C}_{10}^3}=\dfrac{1}{30}$
	\item $\mathrm{P}(B)=\dfrac{\mathrm{C}_4^2\cdot \mathrm{C}_6^1+\mathrm{C}_4^3}{\mathrm{C}_{10}^3}=\dfrac{1}{3}$.
	\item $\mathrm{P}(B \mid A)=1$.
	\end{itemize}
	Áp dụng công thức Bayes, ta có
	$$
	\mathrm{P}(A \mid B)=\dfrac{\mathrm{P}(B \mid A) \mathrm{P}(A)}{\mathrm{P}(B)}=\dfrac{1 \cdot \dfrac{1}{30}}{\dfrac{1}{3}} =\dfrac{1}{10}.
	$$
	Vậy xác suất để cả $3$ người được chọn đều là kĩ sư, biết rằng trong $3$ người được chọn có ít nhất $2$ kĩ sư, là khoảng $10\%$.
	}
\end{bt}

\begin{bt}%[2D5V2-4]
	Có hai cái hộp giống nhau, hộp thứ nhất chứa $5$ quả bóng bàn màu trắng và $3$ quả bóng bàn màu vàng, hộp thứ hai chứa $4$ quả bóng bàn màu trắng và $6$ quả bóng bàn màu vàng. Minh lấy ra ngẫu nhiên $1$ quả bóng từ hộp thứ nhất. Nếu quả bóng đó là bóng vàng thì Minh lấy ra ngẫu nhiên đồng thời $2$ quả bóng từ hộp thứ hai, còn nếu quả bóng đó màu trắng thì Minh lấy ra ngẫu nhiên đồng thời $3$ quả bóng từ hộp thứ hai.
	\begin{enumEX}[\hspace*{.5cm}a)]{1}
	\item Sử dụng sơ đồ hình cây, tính xác suất để có đúng $1$ quả bóng màu vàng trong các quả bóng lấy ra từ hộp thứ hai.
	\item Biết rằng các quả bóng lấy ra từ hộp thứ hai đều có màu trắng. Tính xác suất để quả bóng lấy ra từ hộp thứ nhất có màu vàng.
	\end{enumEX}
	\loigiai{\begin{enumEX}[\hspace*{.5cm}a)]{1}
	\item\, \begin{center}
	\begin{tikzpicture}[scale=0.7, line join=round, line cap=round, >=stealth]
	\draw [->](0,0)--(4,4) node[right]{$\text{Vàng}$};\draw (2,2) node[above]{$\dfrac{3}{8}$};
	\draw [->](0,0)--(4,-4) node[right]{$\text{Trắng}$};\draw (2,-2) node[below]{$\dfrac{5}{8}$};
	\draw [->](6,4)--(9,6) node[right]{$\text{Vàng,\, Trắng}$};\draw (7.5,5) node[above]{$\frac{8}{15}$};\draw [->](15,6)--(17,6) node[right]{$\dfrac{1}{5}$};
	\draw [->](6,4)--(9,4) node[right]{$\text{Vàng,\, Vàng}$};\draw (7.5,4) node[above]{$\frac{1}{3}$};\draw [->](15,4)--(17,4) node[right]{$\dfrac{1}{8}$};
	\draw [->](6,4)--(9,2) node[right]{$\text{Trắng,\, Trắng}$};\draw (7.5,3) node[above]{$\frac{2}{15}$};\draw [->](15,2)--(17,2) node[right]{$\dfrac{1}{20}$};
	\draw [->](6,-4)--(9,-1) node[right]{$\text{Vàng,\, Trắng,\, Trắng}$};\draw (7.5,-2.5) node[above]{$\frac{3}{10}$};\draw [->](15,-1)--(17,-1) node[right]{$\dfrac{3}{16}$};
	\draw [->](6,-4)--(9,-3) node[right]{$\text{Vàng,\,Vàng,\,Trắng}$};\draw (7.5,-3.5) node[above]{$\frac{1}{2}$};\draw [->](15,-3)--(17,-3) node[right]{$\dfrac{5}{16}$};
	\draw [->](6,-4)--(9,-5) node[right]{$\text{Vàng,\, Vàng,\,Vàng}$};\draw (7.5,-4.5) node[above]{$\frac{1}{6}$};\draw [->](15,-5)--(17,-5) node[right]{$\dfrac{5}{48}$};
	\draw [->](6,-4)--(9,-7) node[right]{$\text{Trắng,\, Trắng,\,Trắng}$};\draw (7.5,-5.5) node[above]{$\frac{1}{30}$};\draw [->](15,-7)--(17,-7) node[right]{$\dfrac{1}{48}$};
	\end{tikzpicture}
	\end{center}
	Vậy xác suất để có đúng $1$ quả bóng màu vàng trong các quả bóng lấy ra từ hộp thứ hai là $$ \dfrac{1}{5} + \dfrac{3}{16}=\dfrac{31}{80}.$$
	\item Xét các biến cố
	\begin{itemize}
	\item $A$: \lq\lq  Quả bóng lấy ra từ hộp thứ nhất có màu vàng\rq\rq.
	\item $B$: \lq\lq  Các quả bóng lấy ra từ hộp thứ hai đều có màu trắng\rq\rq.
	\end{itemize}
	Ta có
	\[\mathrm{P}(A) = \dfrac{3}{8};\quad
	\mathrm{P}(B) = \dfrac{2}{15};\quad
	\mathrm{P}(B\mid A) = \dfrac{1}{20} \]
	Áp dụng công thức Bayes, ta có
	\[
	\mathrm{P} (A \mid B)=\dfrac{\mathrm{P} (B \mid A) \mathrm{P} (A)}{\mathrm{P} (B)}=\dfrac{\dfrac{3}{8} \cdot \dfrac{1}{20}}{\dfrac{2}{15}} =\dfrac{9}{64}.\]
	\end{enumEX}
}\end{bt}

\begin{bt}%[2D5V2-2]%[2D5V2-4]
	Hộp thứ nhất có $1$ viên bi xanh và $5$ viên bi đỏ. Hộp thứ hai có $3$ viên bi xanh và $5$ viên bi đỏ. Các viên bi có cùng kích thước và khối lượng. Lấy ra ngẫu nhiên đồng thời $2$ viên bi từ hộp thứ nhất chuyển sang hộp thứ hai. Sau đó lại lấy ra ngẫu nhiên $2$ viên bi từ hộp thứ hai.
	\begin{enumEX}[\hspace*{.5cm}a)]{1}
	\item Tính xác suất để hai viên bi lấy ra từ hộp thứ hai là bi đỏ.
	\item Biết rằng $2$ viên bi lấy ra từ hộp thứ hai là bi đỏ. Tính xác suất để $2$ viên bi lấy ra từ hộp thứ nhất cũng là bi đỏ.
	\end{enumEX}
	\loigiai{
	\begin{enumEX}[\hspace*{.5cm}a)]{1}
	\item Xét các biến cố
	\begin{itemize}
	\item $A$: \lq\lq  Hai viên bi lấy ra từ hộp thứ hai là bi đỏ\rq\rq.
	\item $B_1$: \lq\lq  Hai viên bi lấy ra từ hộp thứ nhất có cả màu xanh và màu đỏ\rq\rq.
	\item $B_2$: \lq\lq  Hai viên bi lấy ra từ hộp thứ nhất có màu đỏ\rq\rq.
	\end{itemize}
	Ta có
	\[\mathrm{P}(B_1) = \dfrac{\mathrm{C}^1_{5}}{\mathrm{C}^2_{6}}=\dfrac{1}{3};\quad
	\mathrm{P}(B_2) = \dfrac{\mathrm{C}^2_5}{\mathrm{C}^2_{6}}=\dfrac{2}{3};\quad
	\mathrm{P}(A\mid B_1) = \dfrac{\mathrm{C}^2_6}{\mathrm{C}^2_{10}}=\dfrac{1}{3};\quad
	\mathrm{P}(A\mid B_2) = \dfrac{\mathrm{C}^2_7}{\mathrm{C}^2_{10}}=\dfrac{7}{15}.
	\]
	Áp dụng công thức xác suất toàn phần, ta có
	\allowdisplaybreaks
	\begin{eqnarray*}
	\mathrm{P}(A)
	&=& \mathrm{P}(A|B_1)\cdot \mathrm{P}(B_1) + \mathrm{P}(A|B_2)\cdot\mathrm{P}(B_2)\\
	&=& \dfrac{1}{3}\cdot \dfrac{1}{3} + \dfrac{7}{15}\cdot \dfrac{2}{3}\\
	&=& \dfrac{19}{45}.
	\end{eqnarray*}
	\item Từ yêu cầu bài toán, ta cần tìm $\mathrm{P}(B_2|A)$.\\
	Áp dụng công thức Bayes, ta có
	\allowdisplaybreaks
	\begin{eqnarray*}
	\mathrm{P}(B_2|A)
	&=& \dfrac{\mathrm{P}(A|B_2)\cdot \mathrm{P}(B_2)}{\mathrm{P}(A)}\\
	&=& \dfrac{\dfrac{7}{15}\cdot \dfrac{2}{3}}{\dfrac{19}{45}}\\
	&=& \dfrac{14}{19}.
	\end{eqnarray*}
	\end{enumEX}
}\end{bt}

\begin{bt}
	Một cửa hàng kinh doanh tổ chức rút thăm trúng thưởng cho hai loại sản phẩm. Tỉ lệ trúng thưởng của các loại sản phẩm I, II lần lượt là $6 \%$; $4 \%$. Trong một hộp kín gồm các thăm cùng loại, người ta để lẫn lộn $200$ chiếc thăm cho sản phẩm loại I và $300$ chiếc thăm cho sản phẩm loại II. Một khách hàng lấy ngẫu nhiên $1$ chiếc thăm từ chiếc hộp đó.
	\begin{enumerate}
	\item Tính xác suất để chiếc thăm được lấy ra là trúng thưởng.
	\item Giả sử chiếc thăm được lấy ra là trúng thưởng. Xác suất chiếc thăm đó thuộc loại sản phẩm nào là cao nhất?
	\end{enumerate}
	\loigiai{
	\begin{enumerate}
	\item
	Xét biến cố $A$: \lq\lq  Chiếc thăm được lấy ra là trúng thưởng\rq\rq.\\
	Khi đó, ta có
	$$\mathrm{P}(A)=\dfrac{6\% \cdot 200 + 4\% \cdot 300}{200+300}=0{,}048.$$
	\item
	Xét hai biến cố:\\
	$B$: \lq\lq  Chiếc thăm được lấy ra là thăm cho sản phẩm loại I\rq\rq.\\
	$C$: \lq\lq  Chiếc thăm được lấy ra là thăm cho sản phẩm loại II\rq\rq.\\
	Khi đó, ta có:
	$$\mathrm{P}(B|A)=\dfrac{n(B\cap A)}{n(A)}=\dfrac{6\% \cdot 200}{6\% \cdot 200 + 4\% \cdot 300}=0{,}5.$$
	$$\mathrm{P}(C|A)=\dfrac{n(C\cap A)}{n(A)}=\dfrac{4\% \cdot 300}{6\% \cdot 200 + 4\% \cdot 300}=0{,}5.$$
	Vậy xác suất hai chiếc thăm lấy được là như nhau.
	\end{enumerate}
	}
\end{bt}

\begin{bt}
	Một xạ thủ bắn vào bia số $1$ và bia số $2$. Xác suất để xạ thủ đó bắn trúng bia số $1$, bia số $2$ lần lượt là $0{,}8$; $0{,}9$. Xác suất để xạ thủ đó bắn trúng cả hai bia là $0{,}8$. Xét hai biến cố sau:
	\begin{itemize}
	\item $A$: \lq\lq  Xạ thủ đó bắn trúng bia số $1$\rq\rq;
	\item $B$: \lq\lq  Xạ thủ đó bắn trúng bia số $2$\rq\rq.
	\end{itemize}
	\begin{enumerate}
	\item Hai biến cố $A$ và $B$ có độc lập hay không?
	\item Biết xạ thủ đó bắn trúng bia số $1$, tính xác suất xạ thủ đó bắn trúng bia số $2$.
	\item Biết xạ thủ đó không bắn trúng bia số $1$, tính xác suất xạ thủ đó bắn trúng bia số $2$.
	\end{enumerate}
	\loigiai{
	\begin{enumerate}
	\item $A$ và $B$ là hai biến cố độc lập.
	\item Xác suất xạ thủ đó bắn trúng bia số $2$ và bia số $1$ là $$\mathrm{P}(B|A)=\dfrac{\mathrm{P}(B\cap A)}{\mathrm{P}(A)}=\dfrac{\mathrm{P}(B) \cdot \mathrm{P}(A)}{\mathrm{P}(A)}=\mathrm{P}(B)=0{,}9.$$
	\item Xác suất xạ thủ đó bắn trúng bia số $2$ và không bắn trứng bia số $1$ là $$\mathrm{P}(B|\overline{A})=\dfrac{\mathrm{P}(B\cap \overline{A})}{\mathrm{P}(\overline{A})}=\dfrac{\mathrm{P}(B) \cdot \mathrm{P}(\overline{A})}{\mathrm{P}(\overline{A})}=\mathrm{P}(B)=0{,}9.$$
	\end{enumerate}
	}
\end{bt}

\begin{bt}
	Một chiếc hộp có $40$ viên bi, trong đó có $12$ viên bi màu đỏ và $28$ viên bi màu vàng; các viên bi có kích thước và khối lượng như nhau. Bạn Ngân lấy ngẫu nhiên viên bi từ chiếc hộp đó hai lần, mỗi lần lấy ra một viên bi và viên bi được lấy ra không bỏ lại hộp. Tính xác suất để cả hai lần bạn Ngân đều lấy ra được viên bi màu vàng.
	\loigiai{
	Xét hai biến cố\\
	$A$: \lq\lq  Viên bi thứ nhất màu vàng\rq\rq.\\
	$B$: \lq\lq  Viên bi thứ hai màu vàng\rq\rq.\\
	Khi đó, ta có xác suất để cả hai lần bạn Ngân đều lấy ra được viên bi màu vàng là\\
	$\mathrm{P}(A\cap B)=\dfrac{\mathrm{C}_{28}^2}{\mathrm{C}_{40}^2}=\dfrac{63}{130}$.
	}
\end{bt}

\begin{bt}
	Giả sử trong một nhóm người có $2$ người nhiễm bệnh, $58$ người còn lại là không nhiễm bệnh. Để phát hiện ra người nhiễm bệnh, người ta tiến hành xét nghiệm tất cả mọi người của nhóm đó. Biết rằng đối với người nhiễm bệnh, xác suất xét nghiệm có kết quả dương tính là $85 \%$ nhưng đối với người không nhiễm bệnh thì xác suất để bị xét nghiệm có phản ứng dương tính là $7 \%$.
	\begin{enumerate}
	\item Vẽ sơ đồ hình cây biểu thị tình huống trên.
	\item Giả sử $X$ là một người trong nhóm bị xét nghiệm có kết quả dương tính. Tính xác suất để $X$ là người nhiễm bệnh.
	\end{enumerate}
	\loigiai{
	\begin{enumerate}
	\item Xét hai biến cố\\
	$N$: \lq\lq  Người được chọn bị nhiễm bệnh\rq\rq.\\
	$D$: \lq\lq  Người được chọn có phản ứng dương tính\rq\rq.\\
	Khi đó, ta có
	\[\mathrm{P}(N)=\dfrac{2}{60}=\dfrac{1}{30}; \qquad \mathrm{P}(\overline{N})=\dfrac{58}{60}=\dfrac{29}{30}.\]
	\[\mathrm{P}(D|N)=85\%=0{,}85; \qquad \mathrm{P}(D|\overline{N})=7\%=0{,}07.\]
	Ta có sơ đồ cây biểu thị tình huống đã cho là
	\begin{center}
	\begin{tikzpicture}[->,>=stealth,line join=round,line cap=round,font=\footnotesize,scale=1]
	\def\xmot{4}
	\def\xhai{8}
	\node (O) at (0,0){};
	\node (B) at (\xmot,1){$N$};
	\node (B1) at (\xmot,-1){$\overline{N}$};
	\node (BA) at (\xhai,2){$D$};
	\node (BA1) at (\xhai,0.3){$\overline{D}$};
	\node (B1A) at (\xhai,-0.3){$D$};
	\node (B1A1) at (\xhai,-1.75){$\overline{D}$};
	\foreach \x/\y/\p/\l in
	{
	O/B/above/$\mathrm{P}(N)=\dfrac{1}{30}$,
	B/BA/above/$\mathrm{P}(D|N)=0{,}85$,
	B/BA1//,
	O/B1/below/$\mathrm{P}\left(\overline{N}\right)=\dfrac{29}{30}$,
	B1/B1A/above/$\mathrm{P}\left(D|\overline{N}\right)=0{,}07$,
	B1/B1A1//
	}
	{
	\draw[->] (\x)--(\y)node[midway,\p,scale=0.8,sloped]{\l};
	}
	\end{tikzpicture}
	\end{center}
	\item $\mathrm{P}(N|D)=\dfrac{\mathrm{P}(D|N) \cdot \mathrm{P}(N)}{\mathrm{P}(N) \cdot \mathrm{P}(D|N)+\mathrm{P}(\overline{N})\mathrm{P}(D|\overline{N})}=\dfrac{0{,}85 \cdot \dfrac{1}{30}}{0{,}85 \cdot \dfrac{1}{30}+0{,}07 \cdot \dfrac{29}{30}}\approx 29{,}5\%$.
	\end{enumerate}
	}
\end{bt}

\begin{bt}
	Trong một cuộc khảo sát trên một nhóm gồm $50$ học sinh chơi cầu lông có cả các bạn nam và các bạn nữ, số liệu thống kê các bạn thuận tay trái và thuận tay phải được cho như bảng sau:
	\begin{center}
	\begin{tabular}{|l|c|c|}
	\hline
	\diagbox{Giới tính}{Tay thuận}& Tay trái & Tay phải\\\hline
	Nam& $5$ & $32$\\\hline
	Nữ & $2$ & $11$\\\hline
	\end{tabular}
	\end{center}
	Chọn ngẫu nhiên một bạn học sinh trong nhóm này. Gọi $A$ là biến cố "Người được chọn là bạn nam", $B$ là biến cố "Chọn được người thuận tay trái", $C$ là biến cố "Chọn được người thuận tay phải."\\
	Tính và giải thích ý nghĩa của $P(A|B)$ và $P(A|C)$
	\loigiai{
	Ta có $P(AB)=\dfrac{5}{50}=\dfrac{1}{10}$, $P(B)=\dfrac{7}{50}.$\\
	Vậy $P(A|B)=\dfrac{P(AB)}{P(B)}=\dfrac{\dfrac{1}{10}}{\dfrac{7}{50}}=\dfrac{5}{7}.$\\
	Do đó xác suất để chọn ra một bạn nam với điều kiện bạn đó thuận tay trái là $\dfrac{5}{7}.$\\
	Ta có $P(AC)=\dfrac{32}{50}=\dfrac{16}{25}$, $P(C)=\dfrac{43}{50}.$\\
	Vậy $P(A|C)=\dfrac{P(AC)}{P(C)}=\dfrac{\dfrac{16}{25}}{\dfrac{43}{50}}=\dfrac{32}{43}.$\\
	Do đó xác suất để chọn ra một bạn nam với điều kiện bạn thuận tay phải là $\dfrac{32}{43}.$
	}
\end{bt}

\begin{bt}
	Một hãng hàng không sau khi nghiên cứ các chuyến bay cho kết quả như sau$\colon$ Xác suất để một chuyến bay khởi hành đúng giờ là $0,83$; xác suất để một chuyến bay đến nơi đúng giờ là $0,82$; xác suất để chuyến bay khởi hành đúng giờ và đến nơi đúng giờ là $0,78$. Gọi $A$ là biến cố "Chuyến bay khởi hành đúng giờ" và $B$ là biến cố "Chuyến bay đến nơi đúng giờ".
	\begin{enumerate}
	\item Tính và giải thích ý nghĩa của $P(A|B)$.
	\item Tính và giải thích ý nghĩa của $P(B|A)$.
	\item Tính $P(B|\overline{A})$ và cho biết xác suất chuyến bay đến nơi đúng giờ là tăng hay giảm khi có thêm thông tin chuyến bay khở hành không đúng giờ.
	\end{enumerate}
	\loigiai{
	\begin{enumerate}
	\item Ta có $P(A)=0,83,P(B)=0,82$ và $P(AB)=0,78$.\\
	Vậy $P(A|B)=\dfrac{P(AB)}{P(B)}=\dfrac{0,78}{0,82}=\dfrac{39}{41}=\approx 0,95$.\\
	Do đó, xác suất để chuyến bay khởi hành đúng giờ, biết rằng chuyến bay đến nơi đúng giờ là $0,95$
	\item Ta có $P(B|A)=\dfrac{P(AB)}{P(A)}=\dfrac{0.78}{0,83}=\dfrac{78}{83} \approx 0,94$.\\
	Vậy xác suất để chuyến bay đến nơi đúng giờ với điều kiện chuyến bay đã khởi hành đúng giờ là $0,94$.
	\item Ta có $P(\overline{A})=1-P(A)=1-0,83=0,17$.\\
	Ta có $P(\overline{A}|B)=1-P(A|B)=1-\dfrac{39}{41}=\dfrac{2}{41}$.\\
	Theo công thức Bayes, ta có\\
	$P(B|\overline{A})=\dfrac{P(B)\cdot P(\overline{A}|B)}{P(\overline{A})}=\dfrac{0,82\cdot \dfrac{2}{41}}{0,17}=\dfrac{4}{17}\approx 0,235$.\\
	Vì $0,235<0,82$ nên xác suất chuyến bay đến nơi đúng giờ là giảm khi có thêm thông tin chuyến bay khở hành không đúng giờ.
	\end{enumerate}
	}
\end{bt}

\begin{bt}
	Trong một cuộc khảo sát tình trạng công việc trên $900$ người đã có bằng tốt nghiệp trung học phổ thông ở một địa phương cho cả nam lẫn nữ, người ta thu được số liệu như bảng sau:
	\begin{center}
	\begin{tabular}{|l|c|c|}
	\hline
	\diagbox{Giới tính}{Tình trạng} & Có việc làm & Thất nghiệp\\\hline
	Nam & $460$ & $40$\\\hline
	Nữ & $140$ & $260$\\\hline
	\end{tabular}
	\end{center}
	Chọn ngẫu nhiên một người trong nhóm này. Gọi $A$ là biến cố "Người được chọn là nữ", $B$ là biến cố "Người được chọn có việc làm".
	\begin{enumerate}
	\item Vẽ lại sơ đồ hình cây sau đây và hoàn thành kết quả ở các ô \fbox{?}.
	\begin{center}
	\begin{tikzpicture}[line join = round, line cap = round, >=stealth, font=\footnotesize, scale=1]
	\begin{scope}[every node/.style={draw, rounded corners=5pt}]
	\node (A) at (0,0){Chọn một người};
	\def \gocA{30}
	\def \kcA{4}
	\node (B1) at ($(A)+(\gocA:\kcA)$){$A$};
	\node (B2) at ($(A)+(-\gocA:\kcA)$){$\overline{A}$};
	\def \gocB{15}
	\def \kcB{4}
	\node (B11) at ($(B1)+(\gocB:\kcB)$){$B$};
	\node (B12) at ($(B1)+(-\gocB:\kcB)$){$\overline{B}$};
	\node (B21) at ($(B2)+(\gocB:\kcB)$){$B$};
	\node (B22) at ($(B2)+(-\gocB:\kcB)$){$\overline{B}$};
	\end{scope}
	\begin{scope}[every node/.style={midway,sloped},every path/.style={->}]
	\draw (A)--(B1) node[above]{$\mathrm{P}(A)=$\fbox{?}};
	\draw (A)--(B2) node[below]{$\mathrm{P}(\overline{A})=$\fbox{?}};
	\draw (B1)--(B11) node[above]{$\mathrm{P}(B|A)=$\fbox{?}};
	\draw (B1)--(B12) node[below]{$\mathrm{P}(\overline{B}|A)=$\fbox{?}};
	\draw (B2)--(B21) node[above]{$\mathrm{P}(B|\overline{A})=$\fbox{?}};
	\draw (B2)--(B22) node[below]{$\mathrm{P}(\overline{B}|\overline{A})=$\fbox{?}};
	\end{scope}
	\def \kcC{1.7}
	\foreach \i/\j in {B11/AB,B12/{A\overline{B}},B21/{\overline{A}B},B22/{\overline{A}\,\overline{B}}}% Tạo nội dung lặp
	{
	\node at ($(\i)+(\kcC,0)$)[]{$\j$};
	\node at ($(\i)+({2*\kcC},0)$)[]{\fbox{?}};
	}
	\node (B) at ($(B11)+(\kcC,0.7)$){\textbf{Kết quả}};
	\node (C) at ($(B)+(\kcC,0)$){\textbf{Xác suất}};
	\end{tikzpicture}\\
	$A \colon$ nữ; $\overline{A} \colon$ nam; $B \colon$ có việc; $\overline{B} \colon$ thất nghiệp.
	\end{center}
	\item Tính xác suất để chọn được một người có việc làm.
	\item Biết rằng đã chọn được một người có việc làm, tính xác suất để người này là nữ.
	\end{enumerate}
	\loigiai{
	\begin{enumerate}
	\item Theo đề bài xác suất để chọn được một người nữ là $P(A)=\dfrac{4}{9}$, suy ra $P(\overline{A})=\dfrac{5}{9}$.\\
	Xác suất chọn được người có việc làm nếu người đó là nữ $P(B|A)=\dfrac{140}{400}=\dfrac{7}{20}$. Suy ra $P(\overline{B}|A)=\dfrac{13}{20}$.\\
	Xác suất chọn được người có việc làm nếu người đó không là nữ $P(B|\overline{A})=\dfrac{460}{500}=\dfrac{23}{25}$.\\
	Suy ra $P(\overline{B}|\overline{A})=\dfrac{2}{25}.$\\
	\begin{center}
	\begin{tikzpicture}[line join = round, line cap = round, >=stealth, font=\footnotesize, scale=1]
	\begin{scope}[every node/.style={draw, rounded corners=5pt}]
	\node (A) at (0,0){Chọn một người};
	\def \gocA{30}
	\def \kcA{4}
	\node (B1) at ($(A)+(\gocA:\kcA)$){$A$};
	\node (B2) at ($(A)+(-\gocA:\kcA)$){$\overline{A}$};
	\def \gocB{15}
	\def \kcB{4}
	\node (B11) at ($(B1)+(\gocB:\kcB)$){$B$};
	\node (B12) at ($(B1)+(-\gocB:\kcB)$){$\overline{B}$};
	\node (B21) at ($(B2)+(\gocB:\kcB)$){$B$};
	\node (B22) at ($(B2)+(-\gocB:\kcB)$){$\overline{B}$};
	\end{scope}
	\begin{scope}[every node/.style={midway,sloped},every path/.style={->}]
	\draw (A)--(B1) node[above]{$\mathrm{P}(A)=$\fbox{$\dfrac{4}{9}$}};
	\draw (A)--(B2) node[below]{$\mathrm{P}(\overline{A})=$\fbox{$\dfrac{5}{9}$}};
	\draw (B1)--(B11) node[above]{$\mathrm{P}(B|A)=$\fbox{$\dfrac{7}{20}$}};
	\draw (B1)--(B12) node[below]{$\mathrm{P}(\overline{B}|A)=$\fbox{$\dfrac{13}{20}$}};
	\draw (B2)--(B21) node[above]{$\mathrm{P}(B|\overline{A})=$\fbox{$\dfrac{23}{25}$}};
	\draw (B2)--(B22) node[below]{$\mathrm{P}(\overline{B}|\overline{A})=$\fbox{$\dfrac{2}{25}$}};
	\end{scope}
	\def \kcC{1.7}
	\foreach \i/\j/\k in {B11/AB/{\dfrac{7}{45}},B12/{A\overline{B}}/{\dfrac{13}{45}},B21/{\overline{A}B}/{\dfrac{23}{45}},B22/{\overline{A}\,\overline{B}}/{\dfrac{2}{45}}}% Tạo nội dung lặp
	{
	\node at ($(\i)+(\kcC,0)$)[]{$\j$};
	\node at ($(\i)+({2*\kcC},0)$)[]{$\k$};
	}
	\node (B) at ($(B11)+(\kcC,0.7)$){\textbf{Kết quả}};
	\node (C) at ($(B)+(\kcC,0)$){\textbf{Xác suất}};
	\end{tikzpicture}
	\end{center}
	\item Xác suất để chọn được một người có việc làm $$P(B)=P(A)P(B|A)+P(\overline{A})P(B|\overline{A})=\dfrac{4}{9}\cdot \dfrac{7}{20}+\dfrac{5}{9}\cdot \dfrac{23}{25}=\dfrac{2}{3}.$$
	\item Theo công thức Bayes, ta có\\
	$$P(A|B)=\dfrac{P(A)P(B|A)}{P(B)}=\dfrac{\dfrac{4}{9}\cdot \dfrac{7}{20}}{\dfrac{2}{3}}=\dfrac{7}{30}.$$
	\end{enumerate}
	}
\end{bt}

\begin{bt}
	Theo thống kê, tỉ lệ khách hàng thân thiết của một siêu thị là $35\%$. Trong nhóm khách hàng thân thiết này, có $74\%$ khách hàng mua rau sạch. Trong nhóm khách hàng còn lại, tỉ lệ mua rau sạch là $28\%$
	\begin{enumerate}
	\item Tính tỉ lệ khách hàng mua rau sạch của siêu thị đó.
	\item Trong một dịp đặc biệt, người ta đã chọn được một khách hàng mua rau sạch. Tính xác suất người này là khách hàng thân thiết.
	\end{enumerate}
	\loigiai{
	Gọi $A$ là biến cố "Khách hàng được chọn là khách hàng thân thiết của siêu thị".\\
	Gọi $B$ là biến cố "Khách hàng được chọn là khách hàng mua rau sạch".\\
	\begin{enumerate}
	\item Ta có $P(A)=0,35$ và $P(\overline{A})=1-P(A)=1-0,35=0,65$.\\
	Vì trong nhóm khách hàng thân thiết này, có $74\%$ khách hàng mua rau sạch nên $P(B|A)=0,74$.\\
	Vì trong nhóm khách hàng còn lại, tỉ lệ mua rau sạch là $28\%$ nên $P(B|\overline{A})=0,28$.\\
	Xác suất khách hàng mua rau sạch $$P(B)=P(A)P(B|A)+P(\overline{A})P(B|\overline{A})=0,35\cdot 0,74+0,65\cdot 0,28=0,441.$$
	Vậy tỉ lệ khách hàng mua rau sạch của siêu thị đó là $44,1\%$
	\item Khi chọn được một khách hàng mua rau sạch thì xác suất người này là khách hàng thân thiết là$\colon$
	$$P(A|B)=\dfrac{P(A)P(B|A)}{P(B)}=\dfrac{0,35\cdot 0,74 }{0,441}=\dfrac{37}{63}.$$
	\end{enumerate}
	}
\end{bt}

\begin{bt}
	Trung tâm kiểm soát và phòng ngừa dịch bệnh Hoa Kỳ (Centers for Disease Control and Prevention, viết tắt là (CDC) thống kê vào thời điểm năm $2020 - 2021$ về số lượng sốc phản vệ sau khi tiêm vaccine ở một số nơi tại Hoa Kỳ và châu Âu như sau: Trong $360{,}19$ triệu liều vaccine $P$ được sử dụng có $581$ ca sốc phản vệ (có khả năng gây tử vong) và $4\,259$ ca phản ứng phụ (không sốc phản vệ, không gây tử vong); trong $67{,}72$ triệu liều vaccine $A$ được sử dụng có $195$ ca sốc phản vệ và $1\,118$ ca phản ứng phụ.\\
	\textit{(Nguồn: https://www.ncbi.nlm.nih.gov/pmc/articles/PMC8626274/)}
	\begin{enumerate}
	\item Xét ngẫu nhiên một người trong số được thống kê ở trên. Tính xác suất để người đó thuộc trường hợp sốc phản vệ (có khả năng gây tử vong).
	\item Nếu gặp một người có biểu hiện sốc phản vệ (có khả năng gây tử vong) trong số này thì có thể nói khả năng cao là người đó đã tiêm vaccine $P$ hay $A$ ?
	\end{enumerate}
	\loigiai{
	%Bài này thực chất không cần dùng đến công thức xác suất toàn phần và công thức Bayes, tuy nhiên trong chương nên mình chọn giải theo lý thuyết của chương
	\begin{enumerate}
	\item Xét ngẫu nhiên một người trong số được thống kê ở trên. Tính xác suất để người đó thuộc trường hợp sốc phản vệ (có khả năng gây tử vong).\\
	Gọi $X$ là biến cố “Người được chọn tiêm vaccine $P$”, khi đó $\overline X $ là biến cố “Người được chọn tiêm vaccine $A$”.\\
	$Y$ là biến cố “Người được chọn thuộc trường hợp sốc phản vệ”.\\
	Khi đó, xác suất chọn được người tiêm vaccine $P$ là $P(X)=\dfrac{360{,}19 \cdot 10^6}{360{,}19 \cdot 10^6+67{,}72 \cdot 10^6}$.\\
	Xác suất chọn được người tiêm vaccine $A$ là $P(\overline X)=\dfrac{67{,}72 \cdot 10^6}{360{,}19 \cdot 10^6+67{,}72 \cdot 10^6}$.\\
	Xác suất chọn được người bị sốc phản vệ, nếu người đó tiêm vaccine $P$ là $P(Y|X)=\dfrac{581}{360{,}19 \cdot 10^6}$.\\
	Xác suất chọn được người bị sốc phản vệ, nếu người đó tiêm vaccine $A$ là $P(Y|\overline X)=\dfrac{195}{67{,}72 \cdot 10^6}$.\\
	Áp dụng công thức tính xác suất toàn phần, ta có:
	$$P(Y)=P(X)\cdot P(Y|X)+P(\overline{X})\cdot P(Y|\overline{X}) \approx 1{,}81 \cdot 10^{-6}$$
	\item So sánh khả năng người đó đã tiêm vaccine $P$ hay $A$\\
	Theo công thức Bayes, ta có xác suất người được chọn bị sốc phản vệ tiêm vaccine $P$ là:
	$$P(X|Y)=\dfrac{P(X)\cdot P(Y|X)}{P(Y)} \approx 0{,}75$$
	Vậy khả năng người đó đã tiêm vaccine $P$ cao hơn so với khả năng tiêm vaccine $A$.
	\end{enumerate}
	}
\end{bt}

\begin{bt}
	Một nhà máy có hai phân xưởng cùng sản xuất một loại sản phẩm. Phân xưởng thứ nhất sản xuất $60 \%$ và phân xưởng thứ hai sản xuất $40 \%$ tổng số sản phẩm của cả nhà máy. Tỉ lệ phế phẩm của từng phân xưởng lần lượt là $16 \%$ và $20 \%$. Lấy ngẫu nhiên một sản phẩm trong kho hàng của nhà máy.
	\begin{enumerate}
	\item Tính xác suất để lấy được phế phẩm.
	\item Giả sử đã lấy được phế phẩm, tính xác suất phế phẩm đó do phân xưởng thứ nhất sản xuất.
	\item Nếu lấy được sản phẩm tốt, khả năng sản phẩm đó do phân xưởng nào sản xuất là cao hơn?
	\end{enumerate}
	\loigiai{
	\begin{enumerate}
	\item Gọi $A$ là biến cố “Chọn được sản phẩm từ phân xưởng thứ nhất”, khi đó $\overline A $ là biến cố “Chọn được sản phẩm từ phân xưởng thứ hai”.\\
	$B$ là biến cố “Chọn được sản phẩm là phế phẩm”.\\
	Khi đó: $P(A)=60 \% = 0{,}6$; $P(\overline{A})=40 \% = 0{,}4$; $P(B|A)=16 \%=0{,}16$; $P(\overline B|A)=0{,}84$; $P(B|\overline A)=20 \%=0{,}2$ .\\
	Áp dụng công thức tính xác suất tính xác suất toàn phần, ta có:
	$$P(B)=P(A)\cdot P(B|A)+P(\overline{A})\cdot P(B|\overline{A})=0{,}6 \cdot 0{,}16+0{,}4 \cdot 0{,}2=0,176.$$
	Vậy xác suất lấy được phế phẩm là $0{,}176$.
	\item Chọn được phế phẩm, biến cố phế phẩm đó do phân xưởng thứ nhất sản xuất là $A|B$, áp dụng công thức Bayes, ta được:
	$$P(A|B)=\dfrac{P(A)\cdot P(B|A)}{P(B)}=\dfrac{0{,}6\cdot 0{,}16}{0{,}176}=\dfrac{6}{11} \approx 0{,}55.$$
	\item Khi lấy được sản phẩm tốt, để so sánh khả năng sản phẩm thuộc phân xưởng, ta tính xác suất để sản phẩm tốt được chọn ấy thuộc phân xưởng thứ nhất \\
	Từ ý a suy ra $P(\overline{B})=1-0{,}176=0,824$.
	Theo công thức Bayes, ta có:
	$$P(A|\overline{B})=\dfrac{P(A)\cdot P(\overline{B}|A)}{P(\overline{B})}=\dfrac{0{,}6 \cdot 0{,}84}{0{,}824} \approx 0{,}61.$$
	Vậy khả năng sản phẩm tốt được chọn từ phân xưởng thứ nhất cao hơn.
	\end{enumerate}
	}
\end{bt}

\begin{bt}
Để thử nghiệm tác dụng điều trị bệnh mất ngủ của hai loại thuốc $X$ và $Y$, người ta tiến hành thử nghiệm trên 4000 người bệnh tình nguyện. Kết quả được cho trong bảng thống kê sau:
	\begin{center}
	\begin{tabular}{|c|c|c|}
	\hline \diagbox{Kết quả}{Dùng thuốc}&$X$ & $Y$\\
	\hline Khỏi bệnh & $1600$ & $1200$\\
	\hline Không khỏi bệnh & $800$ & $400$\\
	\hline
	\end{tabular}
	\end{center}
	Chọn ngẫu nhiên 1 người bệnh tham gia tình nguyện thử nghiệm thuốc.
	\begin{enumEX}{1}
	\item Tính xác suất để người đó khỏi bệnh nếu biết người đó uống thuốc $X$.
	\item Tính xác suất để người bệnh đó uống thuốc $Y$, biết rằng người đó khỏi bệnh.
	\end{enumEX}
	\loigiai{
	\begin{enumEX}{1}
	\item Không gian mẫu là tập hợp gồm $4000$ người điều trị bệnh nên $n(\Omega)=4000$.\\
	Gọi $A$ là biến cố: \lq\lq  Người đó khỏi bệnh\rq\rq\, và $B$ là biến cố: \lq\lq  Người đó uống thuốc $X$\rq\rq.\\
	Khi đó $AB$ là biến cố: \lq\lq  Người đó khỏi bệnh và uống thuốc $X$\rq\rq.\\
	Ta có số người uống thuốc $X$ là $1600+800=2400$ nên $n(B)=2400\Rightarrow \mathrm{P}(B)=\dfrac{2400}{4000}=\dfrac{3}{5}$.\\
	Trong những người khỏi bệnh và uống thuốc $X$ có $1600$ người nên $n(AB)=1600\Rightarrow \mathrm{P}(AB)=\dfrac{1600}{4000}=\dfrac{2}{5}$.\\
	Ta có $\mathrm{P}(A\mid B)=\dfrac{\mathrm{P}(AB)}{\mathrm{P}(B)}=\dfrac{2}{5}:\dfrac{3}{5}=\dfrac{2}{3}$.\\
	Vậy xác suất để người đó khỏi bệnh nếu biết người đó uống thuốc $X$ là $\dfrac{2}{3}$.
	\item Gọi $C$ là biến cố: \lq\lq  Người đó uống thuốc $Y$\rq\rq\, và $D$ là biến cố: \lq\lq  Người đó khỏi bệnh\rq\rq.\\
	Khi đó $CD$ là biến cố: \lq\lq  Người đó uống thuốc $Y$ và khỏi bệnh\rq\rq.\\
	Ta có số người khỏi bệnh là $1600+1200=2800$ nên $$n(D)=2800\Rightarrow \mathrm{P}(D)=\dfrac{2800}{4000}=\dfrac{7}{10}.$$\\
	Có $1200$ người uống thuốc $Y$ và khỏi bệnh nên $n(CD)=1200\Rightarrow \mathrm{P}(CD)=\dfrac{1200}{4000}=\dfrac{3}{10}$.\\
	Ta có $\mathrm{P}(C\mid D)=\dfrac{\mathrm{P}(CD)}{\mathrm{P}(D)}=\dfrac{3}{10}:\dfrac{7}{10}=\dfrac{3}{7}$.\\
	Vậy xác suất để người bệnh đó uống thuốc $Y$, biết rằng người đó khỏi bệnh là $\dfrac{3}{7}$.
	\end{enumEX}
	}
\end{bt}

\begin{bt}
Một nhóm có $25$ học sinh, trong đó $14$ học sinh học khá môn Toán, $16$ học sinh học khá môn Vật lí, $1$ em không học khá cả hai môn Toán và Vật lí. Chọn ngẫu nhiên một học sinh trong số đó, tính xác suất để học sinh đó
	\begin{enumEX}{1}
	\item Học khá môn Toán, đồng thời học khá môn Vật lí.
	\item Học khá môn Toán, nhưng không học khá môn Vật lí.
	\item Học khá môn Toán, biết rằng học sinh đó học khá môn Vật lí.
	\end{enumEX}
	\loigiai{
	Chọn bất kì một học sinh trong $25$ học sinh nên số phần tử không gian mẫu là $n(\Omega)=25$.
	\begin{enumEX}{1}
	\item Gọi $C$ là biến cố: \lq\lq  Học sinh được chọn học khá môn Toán, đồng thời học khá môn Vật lí\rq\rq.\\
	Ta có số học sinh học khá cả Toán và Vật lí là $14+16-24=6$ nên $n(C)=6$.\\
	Xác suất để chọn được học sinh học khá môn Toán, đồng thời học khá môn Vật lí là $$\mathrm{P}(C)=\dfrac{n(C)}{n(\Omega)}=\dfrac{6}{25}.$$
	\item
	Gọi $D$ là biến cố: \lq\lq  Học sinh được chọn học khá môn Toán, nhưng không học khá môn Vật lí\rq\rq.\\
	Vì số học sinh chỉ học khá môn Toán là $14-6=8$ nên $n(D)=8$.\\
	Xác suất để chọn được học sinh học khá môn Toán, nhưng không học khá môn Vật lí là $$\mathrm{P}(D)=\dfrac{8}{25}.$$
	\item Gọi $A$ là biến cố: \lq\lq  Học sinh được chọn học khá môn Toán\rq\rq\, và $B$ là biến cố: \lq\lq  Học sinh được chọn học khá môn Vật lí\rq\rq\, nên $AB$ là biến cố: \lq\lq  Học sinh được chọn học khá môn Toán và Vật lí\rq\rq.\\
	Có $16$ học sinh học khá môn Vật lí nên $n(B)=16\Rightarrow \mathrm{P}(B)=\dfrac{16}{25}$.\\
	Ta có $\mathrm{P}(AB)=\dfrac{6}{25}$.\\
	Khi đó xác suất để chọn được học sinh học khá môn Toán, biết rằng học sinh đó học khá môn Vật lí là
	$$\mathrm{P}(A\mid B)=\dfrac{\mathrm{P}(AB)}{\mathrm{P}(B)}=\dfrac{\frac{6}{25}}{\frac{16}{25}}=\dfrac{3}{8}.$$
	\end{enumEX}
	}
\end{bt}

\begin{bt}
Chuồng I có 5 con gà mái, 2 con gà trống. Chuồng II có 3 con gà mái, 5 con gà trống. Bác Mai bắt một con gà trong số đó theo cách sau: Bác tung một con xúc xắc cân đối, đồng chất. Nếu số chấm chia hết cho 3 thì bác chọn chuồng I, nếu số chấm không chia hết cho 3 thì bác chọn chuồng II. Sau đó, từ chuồng đã chọn bác bắt ngẫu nhiên một con gà. Tính xác suất để bác Mai bắt được con gà mái.
	\loigiai{
	Gọi $A$ là biến cố: \lq\lq  Bác Mai bắt được con gà mái\rq\rq.\\
	Gọi $B_1$ là biến cố: \lq\lq  tung con xúc xắc được số chấm chia hết cho $3$\rq\rq\, suy ra $\mathrm{P}(B_1)=\dfrac{2}{6}=\dfrac{1}{3}$.\\
	Gọi $B_2$ là biến cố: \lq\lq  tung con xúc xắc được số chấm không chia hết cho $3$\rq\rq\, suy ra $\mathrm{P}(B_2)=\dfrac{4}{6}=\dfrac{2}{3}$.\\
	Ta có xác suất bắt được gà mái từ chuồng I là $\mathrm{P}\left(A\mid B_1\right)=\dfrac{5}{7}$.\\
	Ta có xác suất bắt được gà mái từ chuồng II là $\mathrm{P}\left(A\mid B_2\right)=\dfrac{3}{8}$.\\
	Áp dụng công thức xác suất toàn phần ta có
	$$\mathrm{P}(A)=\mathrm{P}(B_1)\cdot \mathrm{P}(A\mid B_1)+\mathrm{P}(B_2)\cdot \mathrm{P}(A\mid B_2)=\dfrac{1}{3}\cdot \dfrac{5}{7}+\dfrac{2}{3}\cdot \dfrac{3}{8}=\dfrac{41}{84}.$$
	Vậy xác suất để bác Mai bắt được con gà mái là $\dfrac{41}{84}$.
	}
\end{bt}

\begin{bt}
	Một loại vaccine được tiêm ở địa phương $X$. Người có bệnh nền thì với xác suất $0{,}35$ có phản ứng phụ sau tiêm, người không có bệnh nền thì chỉ có phản ứng phụ sau tiêm với xác suất $0{,}16$. Chọn ngẫu nhiên một người được tiêm vaccine và người này có phản ứng phụ. Tính xác suất để người này có bệnh nền, biết rằng tỉ lệ người có bệnh nền ở địa phương $X$ là $18\%$.
	\loigiai{
	Gọi $A$ là biến cố: \lq\lq  Người được chọn có bệnh nền\rq\rq\, và $B$ là biến cố: \lq\lq  Người này có phản ứng phụ sau tiêm\rq\rq.\\
	Ta có $\mathrm{P}(A)=0{,}18$; $\mathrm{P}(\overline{A})=0{,}82$.\\
	$\mathrm{P}(B\mid A)$ là xác suất để một người bệnh có phản ứng sau tiêm với điều kiện có bệnh nền, suy ra $\mathrm{P}(B\mid A)=0{,}35$.\\
	$\mathrm{P}(B\mid \overline{A})$ là xác suất để một người bệnh có phản ứng sau tiêm với điều kiện không có bệnh nền, suy ra $\mathrm{P}(B\mid \overline{A})=0{,}16$.\\
	Theo công thức Bayes, ta được
	$$\mathrm{P}(A\mid B)=\dfrac{\mathrm{P}(A)\cdot \mathrm{P}(B\mid A)}{\mathrm{P}(A)\cdot \mathrm{P}(B\mid A)+\mathrm{P}(\overline{A})\cdot \mathrm{P}(B\mid \overline{A})}=\dfrac{0{,}18\cdot 0{,}35}{0{,}18\cdot 0{,}35+0{,}82\cdot 0{,}16}=\dfrac{315}{971}.$$
	}
\end{bt}

\subsubsection{Bài tập trắc nghiệm}
\setcounter{ex}{0}
\Opensolutionfile{ans}[ans12]
\begin{ex}%[2D5N2-1]
	Cho hai biến cố $A$ và $B$ có $\mathrm{P}(A)=0\text{,}8$; $\mathrm{P}(B)=0\text{,}5$ và $\mathrm{P}(A B)=0\text{,}2$. Xác suất của biến cố $A$ với điều kiện $B$ là
	\choice
	{\True$0\text{,}4$}
	{$0\text{,}5$}
	{$0\text{,}25$}
	{$0\text{,}625$}
	\loigiai{Ta có
	$\mathrm{P}(A\mid B)=\dfrac{\mathrm{P}(A B)}{\mathrm{P}(B)}=\dfrac{0\text{,}2}{0\text{,}5}=0\text{,}4$.}
\end{ex}

\begin{ex}%[2D5N2-1]
	Cho hai biến cố $A$ và $B$ có $\mathrm{P}(A)=0\text{,}8$; $\mathrm{P}(B)=0\text{,}5$ và $\mathrm{P}(A B)=0\text{,}2$. Xác suất biến cố $B$ không xảy ra với điều kiện biến cố $A$ xảy ra là
	\choice
	{$0\text{,}6$}
	{$0\text{,}5$}
	{\True$0\text{,}75$}
	{$0\text{,}25$}
	\loigiai{Ta có $\mathrm{P}(B\mid A)=\dfrac{\mathrm{P}(A B)}{\mathrm{P}(A)}=\dfrac{0\text{,}2}{0\text{,}8}=0\text{,}25$.\\
	Mặt khác $\mathrm{P}(A)>0\Rightarrow \mathrm{P}(\overline{B}\mid A)=1-\mathrm{P}(B\mid A)=1-0\text{,}25=0\text{,}75$.}
\end{ex}

\begin{ex}%[2D5H2-1]
	Cho hai biến cố $A$ và $B$ có $\mathrm{P}(A)=0\text{,}8$; $\mathrm{P}(B)=0\text{,}5$ và $\mathrm{P}(A B)=0\text{,}2$. Giá trị của biểu thức $
	\dfrac{\mathrm{P}(A \mid B)}{\mathrm{P}(A)}-\dfrac{\mathrm{P}(B \mid A)}{\mathrm{P}(B)}$ là
	\choice
	{$-0\text{,}5$}
	{\True$0$}
	{$0\text{,}5$}
	{$1$}
	\loigiai{Ta có $$
	\dfrac{\mathrm{P}(A \mid B)}{\mathrm{P}(A)}-\dfrac{\mathrm{P}(B \mid A)}{\mathrm{P}(B)}=\dfrac{\mathrm{P}(AB)}{\mathrm{P}(A)\cdot \mathrm{P}(B)}- \dfrac{\mathrm{P}(BA)}{\mathrm{P}(B)\cdot \mathrm{P}(A)}=\dfrac{0\text{,}2}{0\text{,}8\cdot 0\text{,}5}-\dfrac{0\text{,}2}{0\text{,}8\cdot 0\text{,}5}=0.$$}
\end{ex}

\begin{ex}
	Cho hai biến cố xung khắc $A$, $B$ với $\mathrm{P}(A)=0{,}2$; $\mathrm{P}(B)=0{,}4$. Khi đó, $\mathrm{P}(A\mid B)$ bằng
	\choice{$0{,}5$}
	{$0{,}2$}
	{$0{,}4$}
	{$0$}
	\loigiai{
	Vì $A$ và $B$ là 2 biến cố xung khắc nên $\mathrm{P}(A\cap B)=0$.\\
	$\Rightarrow \mathrm{P}(A|B)=\dfrac{\mathrm{P}(A\cap B)}{\mathrm{P}(B)}=\dfrac{0}{0{,}4}=0$.
	}
\end{ex}

\begin{ex}
	Cho $\mathrm{P}(A)=\dfrac{2}{5}$; $\mathrm{P}\left( B\mid A\right)=\dfrac{1}{3}$. Giá trị của $\mathrm{P}(AB)$ là
	\choice
	{\True $\dfrac{2}{15} $}
	{$ \dfrac{3}{16}$}
	{$ \dfrac{1}{5}$}
	{$ \dfrac{4}{15}$}
	\loigiai{
	$\mathrm{P}(AB)=\mathrm{P}(A)\cdot \mathrm{P}\left( B\mid A\right)=\dfrac{2}{5}\cdot \dfrac{1}{3}=\dfrac{2}{15}$.
	}
\end{ex}

\begin{ex}
	Cho $\mathrm{P}(A)=\dfrac{2}{5}$; $\mathrm{P}\left(B\mid \overline{A}\right)=\dfrac{1}{4}$. Giá trị của $\mathrm{P}\left(B\overline{A}\right)$ là
	\choice
	{$\dfrac{1}{7} $}
	{$ \dfrac{4}{19}$}
	{$ \dfrac{4}{21}$}
	{\True $ \dfrac{3}{20}$}
	\loigiai{
	$\mathrm{P}\left(B\overline{A}\right)=\mathrm{P}\left( B\mid A\right)\cdot \mathrm{P}\left( \overline{A}\right) =\dfrac{1}{4}\cdot \dfrac{3}{5}=\dfrac{3}{20}$.
	}
\end{ex}

\begin{ex}
	Cho $\mathrm{P}(A)=\dfrac{2}{5}$; $\mathrm{P}\left( B\mid A\right)=\dfrac{1}{3}$; $\mathrm{P}\left(B\mid \overline{A}\right)=\dfrac{1}{4}$. Giá trị của $\mathrm{P}(B)$ là
	\choice
	{$\dfrac{19}{60} $}
	{\True $ \dfrac{17}{60}$}
	{$ \dfrac{9}{20}$}
	{$ \dfrac{7}{30}$}
	\loigiai{Áp dụng công thức Bayes ta có
	$$\mathrm{P}(A\mid B)=\dfrac{\mathrm{P}(A)\cdot \mathrm{P}(B\mid A)}{\mathrm{P}(A)\cdot \mathrm{P}(B\mid A)+\mathrm{P}\left( \overline{A}\right) \cdot \mathrm{P}\left( B\mid \overline{A}\right) }=\dfrac{\frac{2}{5}\cdot \frac{1}{3}}{\frac{2}{5}\cdot \frac{1}{3}+\frac{3}{5}\cdot \frac{1}{4}}=\dfrac{8}{17}.$$
	Khi đó
	$\mathrm{P}(B)=\dfrac{\mathrm{P}(AB)}{\mathrm{P}(A\mid B)}=\dfrac{2}{15}:\dfrac{8}{17}=\dfrac{17}{60}$.
	}
\end{ex}

\begin{ex}
	Cho $A$, $B$ là các biến cố của một phép thử $T$. Biết rằng $P(B)>0$, xác suất của biến cố $A$ với điều kiện biến cố $B$ đã xảy ra được tính theo công thức nào sau đây?
	\choice
	{$P(A|B)=\dfrac{P(A)}{P(B)}$}
	{$P(A|B)=\dfrac{P(A)}{P(A B)}$}
	{\True $P(A| B)=\dfrac{P(A B)}{P(B)}$}
	{$P(A|B)=\dfrac{P(A B)}{P(A) \cdot P(B)}$}
	\loigiai{ Dựa theo công thức tính xác suất biến cố $A$ với điều kiện $B$ thì $P(A| B)=\dfrac{P(A B)}{P(B)}$ là đáp án đúng.
	}
\end{ex}

\begin{ex}
	Cho $A$, $B$ là các biến cố thỏa mãn $P(\overline{AB})=0{,}35$, $P(A)=0{,}25$, $P(B)=0{,}6$. Giá trị của $P(A|B)$ bằng
	\choice
	{$\dfrac{1}{5}$}
	{\True$\dfrac{1}{3}$}
	{$\dfrac{7}{15}$}
	{$\dfrac{2}{3}$}
	\loigiai{Ta có $P\left( {\overline {AB} } \right) = P\left( {\overline A } \right)P\left( {\overline B |\overline A } \right) \Rightarrow P\left( {\overline B |\overline A } \right) = \dfrac{{P\left( {\overline {AB} } \right)}}{{P\left( {\overline A } \right)}} = \dfrac{{0{,}35}}{{0{,}75}} = \dfrac{7}{{15}}.$ \\
	Suy ra $P\left( {B|\overline A } \right) = 1 - \dfrac{7}{{15}} = \dfrac{8}{{15}}$.\\
	Theo công thức xác suất toàn phần, ta có
	$$\begin{array}{l}
	P\left( B \right) = P\left( {B|A} \right)P\left( A \right) + P\left( {B|\overline A } \right)P\left( {\overline A } \right)\\
	\Rightarrow P\left( {B|A} \right) = \dfrac{{P\left( B \right) - P\left( {B|\overline A } \right)P\left( {\overline A } \right)}}{{P\left( A \right)}} = \dfrac{{0{,}6 - \dfrac{8}{{15}} \cdot 0{,}75}}{{0{,}25}} = 0{,}8.
	\end{array}.$$
	Theo công thức Bayes, ta được
	$$P\left( {A|B} \right) = \dfrac{{P\left( A \right)P\left( {B|A} \right)}}{{P\left( B \right)}} = \dfrac{{0{,}25 \cdot 0{,}8}}{{0{,}6}} = \dfrac{1}{3}.$$}
\end{ex}

\begin{ex}%[2D5V2-1]
	Một nhà máy thực hiện khảo sát toàn bộ công nhân về sự hài lòng của họ về điều kiện làm việc tại phân xưởng. Kết quả khảo sát như sau:
	\begin{center}
	\begin{tabular}{|c|c|c|}
	\hline
	\diagbox {Khảo sát công nhân}{Kết quả khảo sát}	& Hài lòng & Không hài lòng \\
	\hline
	\textbf{Số công nhân phân xưởng I}	& $37$ & $13$ \\
	\hline
	\textbf{Số công nhân phân xưởng II}	& $63$ & $27$\\
	\hline
	\end{tabular}
	\end{center}
	Gặp ngẫu nhiên một công nhân của nhà máy. Gọi $A$ là biến cố \lq\lq  Công nhân đó làm việc tại phân xưởng I \rq\rq.
	Xác suất của biến cố $A$ là
	\choice
	{$\dfrac{37}{140}$}
	{$\dfrac{37}{50}$}
	{\True$\dfrac{5}{14}$}
	{$\dfrac{1}{2}$}
	\loigiai{
	\begin{itemize}
	\item Không gian mẫu $n(\Omega)=50+90=140$.
	\item Ta có $n(A)=37+13=50$.
	\item Xác suất của biến cố $A$ là $\mathrm{P}(A)=\dfrac{n(A)}{n(\Omega)}=\dfrac{50}{140}=\dfrac{5}{14}$.
	\end{itemize}
	}
\end{ex}

\begin{ex}%[2D5V2-1]
	Một nhà máy thực hiện khảo sát toàn bộ công nhân về sự hài lòng của họ về điều kiện làm việc tại phân xưởng. Kết quả khảo sát như sau:
	\begin{center}
	\begin{tabular}{|c|c|c|}
	\hline
	\diagbox {Khảo sát công nhân}{Kết quả khảo sát}	& Hài lòng & Không hài lòng \\
	\hline
	\textbf{Số công nhân phân xưởng I}	& $37$ & $13$ \\
	\hline
	\textbf{Số công nhân phân xưởng II}	& $63$ & $27$\\
	\hline
	\end{tabular}
	\end{center}
	Gặp ngẫu nhiên một công nhân của nhà máy. Gọi $A$ là biến cố \lq\lq  Công nhân đó làm việc tại phân xưởng I \rq\rq \, và $B$ là biến cố \lq\lq  Công nhân đó hài lòng với điều kiện làm việc tại phân xưởng\rq\rq.
	Xác suất của biến cố $B$ với điều kiện $A$ không xảy ra là
	\choice
	{$\dfrac{2}{7}$}
	{$0\text{,}9$}
	{\True$0\text{,}7$}
	{$\dfrac{9}{20}$}
	\loigiai{
	\begin{itemize}
	\item Ta có $n(B\cap \overline{A})=63$ và $n(\overline{A})=63+27=90$.
	\item $\mathrm{P}(B\mid \overline{A})=\dfrac{n(B\cap \overline{A})}{n(\overline{A})}=\dfrac{63}{90}=0\text{,}7$.
	\end{itemize}
	}
\end{ex}

\begin{ex}
	Người ta nhập hai lô hàng vào kho. Lô thứ nhất chứa $10$ sản phẩm, trong đó có $3$ phế phẩm. Lô thứ hai có $4$ phế phẩm và $8$ sản phẩm tốt. Chọn ngẫu nhiên một sản phẩm. Xác suất chọn được một sản phẩm tốt là
	\choice
	{\True $\dfrac{15}{22}$}
	{$\dfrac{7}{15}$}
	{$\dfrac{7}{22}$}
	{$\dfrac{83}{242}$}
	\loigiai{
	%Bài này thực sự không cần thiết phải dùng xác suất có điều kiện vì:
	Gọi $A$ là biến cố “Chọn được sản phẩm tốt”, theo đề ra, kho có $22$ sản phẩm, trong đó có $15$ sản phẩm tốt nên:\\
	$n(A)=15$, $n(\Omega)=22$. Vậy $P(A)=\dfrac{n(A)}{n(\Omega)}=\dfrac{15}{22}$.
	}
\end{ex}

\begin{ex}
	An có một túi gồm một số chiếc kẹo cùng loại, chỉ khác màu, trong đó có $6$ chiếc kẹo sô-cô-la đen, còn lại là $4$ chiếc kẹo sô-cô-la trắng. An lấy ngẫu nhiên $1$ chiếc kẹo trong túi để cho Bình, rồi lại lấy ngẫu nhiên tiếp 1 chiếc kẹo nữa trong túi và cũng đưa cho Bình. Xác suất để Bình nhận được $2$ chiếc kẹo sô-cô-la đen là
	\choice
	{\True $\dfrac{1}{3} $}
	{$ \dfrac{1}{4}$}
	{$ \dfrac{2}{5}$}
	{$ \dfrac{3}{7}$}
	\loigiai{
	Gọi $A$ là biến cố: \lq\lq  An lấy lần 1 được 1 chiếc kẹo sô-cô-la đen\rq\rq\, và $B$ là biến cố: \lq\lq  An lấy lần 2 được 1 chiếc kẹo sô-cô-la đen\rq\rq.\\
	Khi đó $AB$ là biến cố: \lq\lq  Cả hai lần đều lấy được kẹo sô-cô-la đen\rq\rq.\\
	Ta có $\mathrm{P}(A)=\dfrac{n(A)}{n(\Omega)}=\dfrac{6}{10}$.\\
	Sau khi lấy 1 chiếc kẹo sô-cô-la đen thì xác suất để chọn 1 chiếc kẹo sô-cô-la đen trong hộp đựng $5$ chiếc kẹo sô-cô-la đen, còn lại là $4$ chiếc kẹo sô-cô-la trắng là $\mathrm{P}(B\mid A)=\dfrac{5}{9}$.\\
	Khi đó $\mathrm{P}(AB)=\mathrm{P}(A)\cdot \mathrm{P}(B\mid A)=\dfrac{6}{10}\cdot \dfrac{5}{9}=\dfrac{1}{3}$.\\
	Xác suất để Bình nhận được $2$ chiếc kẹo sô-cô-la đen là $\dfrac{1}{3}$.
	}
\end{ex}

\begin{ex}
	An có một túi gồm một số chiếc kẹo cùng loại, chỉ khác màu, trong đó có $6$ chiếc kẹo sô-cô-la đen, còn lại là $4$ chiếc kẹo sô-cô-la trắng. An lấy ngẫu nhiên $1$ chiếc kẹo trong túi để cho Bình, rồi lại lấy ngẫu nhiên tiếp 1 chiếc kẹo nữa trong túi và cũng đưa cho Bình. Xác suất để Bình nhận được chiếc kẹo sô-cô-la đen ở lần thứ nhất và chiếc kẹo sô-cô-la trắng ở lần thứ hai là
	\choice
	{$\dfrac{1}{5} $}
	{$ \dfrac{3}{16}$}
	{$ \dfrac{1}{4}$}
	{$ \dfrac{4}{17}$}
	\loigiai{
	Gọi $A$ là biến cố: \lq\lq  Bình nhận được chiếc kẹo ở lần đầu tiên là đen\rq\rq\, và $B$ là là biến cố: \lq\lq  Bình nhận được chiếc kẹo ở lần 2 là trắng \rq\rq.\\
	Ta có $\mathrm{P}(A)=\dfrac{3}{5}$ và $\mathrm{P}(B\mid A)=\dfrac{4}{9}$.\\
	Khi đó $\mathrm{P}(B\mid A)=\dfrac{\mathrm{P}(AB)}{\mathrm{P}(A)}\Rightarrow \mathrm{P}(AB)=\mathrm{P}(B\mid A)\cdot \mathrm{P}(A)=\dfrac{3}{5}\cdot \dfrac{4}{9}=\dfrac{4}{15}$.
	}
\end{ex}

\begin{ex}
	Một bệnh viện có hai phòng khám là phòng A và phòng B với khả năng lựa chọn của bệnh nhân là như nhau. Tỉ lệ bệnh nhân nam có ở phòng A và phòng B lần lượt là $60\%$ và $40\%$. Một người bệnh được chọn ngẫu nhiêu từ hai phòng khám và biết người này là nam, xác suất để người bệnh được chọn đến từ phòng A là
	\choice
	{\True $0{,}6$}
	{$0{,}5$}
	{$0{,}4$}
	{$0{,}3$}
	\loigiai{Một người bệnh được chọn ngẫu nhiên từ hai phòng khám.\\
	Gọi $X$ là biến cố \lq \lq Người đó đến từ phòng khám A\rq \rq \, và $Y$, $\overline{Y}$ lần lượt là biến cố \lq \lq Người đó là nam\rq \rq \; và \lq \lq Người đó không là nam\rq \rq.\\
	Ta có sơ đồ hình cây sau
\begin{center}
	\begin{tikzpicture}[>=stealth,scale=0.7]
	%Khung 1
	\draw (-3.8,-1) rectangle (2.2,0);
	\draw (-0.8,-0.5) node{Bệnh nhân được chọn} ;
	%Mui ten 1,2
	\draw [->] (2.2,-0.5)--(3.8,1.6) node[pos=0.5,sloped,above]{$0{,}5$};
	\draw [->] (2.2,-0.5)--(3.8,-2.6) node[pos=0.5,sloped,below]{$0{,}5$};
	%Khung 2.1
	\draw (3.8,1.1) rectangle (5.1,2.1);
	\draw (8.9/2,1.6) node{$X$} ;
	%Khung 2.2
	\draw (3.8,-2.1) rectangle (5.1,-3.1);
	\draw (8.9/2,-2.6) node{$\overline{X}$} ;
	%Mui ten 3,4
	\draw [->] (5.1,1.6)--(6.5,2.6) node[pos=0.5,sloped,above]{$0{,}6$};
	\draw [->] (5.1,1.6)--(6.5,0.6) node[pos=0.5,sloped,below]{$0{,}4$};
	%Mui ten 5,6
	\draw [->] (5.1,-2.6)--(6.5,-1.6) node[pos=0.5,sloped,above]{$0{,}4$};
	\draw [->] (5.1,-2.6)--(6.5,-3.6) node[pos=0.5,sloped,below]{$0{,}6$};
	%Khung 3.1
	\draw (6.5,2.2) rectangle (7.7,3.2);
	\draw (7.1,5.4/2) node{$Y$} ;
	%Khung 3.2
	\draw (6.5,1.2) rectangle (7.7,0.2);
	\draw (7.1,1.4/2) node{$\overline{Y}$} ;
	%Khung 3.3
	\draw (6.5,-1.1) rectangle (7.7,-2.1);
	\draw (7.1,-3.2/2) node{$Y$} ;
	%Khung 3.3
	\draw (6.5,-2.9) rectangle (7.7,-3.9);
	\draw (7.1,-3.4) node{$\overline{Y}$} ;
	%Kết quả
	\draw (9.5,3.7) node{\textbf{Kết quả}};
	\draw (9.5,2.7) node{$XY$};
	\draw (9.5,0.7) node{$X \overline{Y}$};
	\draw (9.5,-1.6) node{$\overline{X}Y$};
	\draw (9.5,-3.4) node{$\overline{X}\overline{Y}$};
	%Xác suất
	\draw (12.5,3.7) node{\textbf{Xác suất}};
	\draw (12.5,2.7) node{$0{,}3$};
	\draw (12.5,0.7) node{$0{,}2$};
	\draw (12.5,-1.6) node{$0{,}2$};
	\draw (12.5,-3.4) node{$0{,}3$};
	\end{tikzpicture}
\end{center}
	Theo công thức Bayes, ta có $$P(X|Y)=\dfrac{P(X)P(Y|X)}{P(X)P(Y|X)+P(\overline{X})P(Y|\overline{X})}=\dfrac{0{,}3}{0{,}3+0{,}2}=0{,}6.$$
	Vậy với một người bệnh được chọn ngẫu nhiêu từ hai phòng khám và biết người này là nam, xác suất để người đó đến từ phòng A là $0{,}6$.}
\end{ex}

\begin{ex}
	Một bệnh viện đang xét nghiệm cho một số bệnh nhân để xác định liệu họ có nhiễm virus $X$ hay không. Xác suất để một bệnh nhân bị nhiễm virus $X$ là $0{,}05$. Khi xét nghiệm, nếu một bệnh nhân bị nhiễm thì xác suất để kết quả xét nghiệm dương tính là $0{,}95$. Nếu một bệnh nhân không bị nhiễm thì xác suất để kết quả xét nghiệm âm tính là $0{,}98$. Một bệnh nhân được chọn ngẫu nhiên và có kết quả xét nghiệm dương tính. Xác suất để bệnh nhân đó thực sự bị nhiễm virus $X$ là
	\choice
	{$\dfrac{133}{2000}$}
	{$\dfrac{19}{400}$}
	{\True $\dfrac{5}{7}$}
	{$\dfrac{2}{7}$}
	\loigiai{Một bệnh nhân đến một bệnh viên để xét nghiệm.\\
	Gọi $A$ là biến cố \lq \lq Bệnh nhân bị nhiễm virus $X$\rq \rq \, và $B$, $\overline{B}$ lần lượt là biến cố \lq \lq Kết quả xét nghiệm dương tính\rq \rq \; và \lq \lq Kết quả xét nghiệm âm tính\rq \rq.\\
	Ta xét sơ đồ hình cây như sau
\begin{center}
		\begin{tikzpicture}[>=stealth,scale=0.7]
	%Khung 1
	\draw (-4.8,-1) rectangle (2.2,0);
	\draw (-1.3,-0.5) node{Bệnh nhân được xét nghiệm} ;
	%Mui ten 1,2
	\draw [->] (2.2,-0.5)--(3.8,1.6) node[pos=0.5,sloped,above]{$0{,}05$};
	\draw [->] (2.2,-0.5)--(3.8,-2.6) node[pos=0.5,sloped,below]{$0{,}95$};
	%Khung 2.1
	\draw (3.8,1.1) rectangle (5.1,2.1);
	\draw (8.9/2,1.6) node{$A$} ;
	%Khung 2.2
	\draw (3.8,-2.1) rectangle (5.1,-3.1);
	\draw (8.9/2,-2.6) node{$\overline{A}$} ;
	%Mui ten 3,4
	\draw [->] (5.1,1.6)--(6.5,2.6) node[pos=0.5,sloped,above]{$0{,}95$};
	\draw [->] (5.1,1.6)--(6.5,0.6) node[pos=0.5,sloped,below]{$0{,}05$};
	%Mui ten 5,6
	\draw [->] (5.1,-2.6)--(6.5,-1.6) node[pos=0.5,sloped,above]{$0{,}02$};
	\draw [->] (5.1,-2.6)--(6.5,-3.6) node[pos=0.5,sloped,below]{$0{,}98$};
	%Khung 3.1
	\draw (6.5,2.2) rectangle (7.7,3.2);
	\draw (7.1,5.4/2) node{$B$} ;
	%Khung 3.2
	\draw (6.5,1.2) rectangle (7.7,0.2);
	\draw (7.1,1.4/2) node{$\overline{B}$} ;
	%Khung 3.3
	\draw (6.5,-1.1) rectangle (7.7,-2.1);
	\draw (7.1,-3.2/2) node{$B$} ;
	%Khung 3.3
	\draw (6.5,-2.9) rectangle (7.7,-3.9);
	\draw (7.1,-3.4) node{$\overline{B}$} ;
	%Kết quả
	\draw (9.5,3.7) node{\textbf{Kết quả}};
	\draw (9.5,2.7) node{$AB$};
	\draw (9.5,0.7) node{$A\overline{B}$};
	\draw (9.5,-1.6) node{$\overline{A}B$};
	\draw (9.5,-3.4) node{$\overline{A}\overline{B}$};
	%Xác suất
	\draw (12.5,3.7) node{\textbf{Xác suất}};
	\draw (12.5,2.7) node{$0{,}0475$};
	\draw (12.5,0.7) node{$0{,}0025$};
	\draw (12.5,-1.6) node{$0{,}019$};
	\draw (12.5,-3.4) node{$0{,}931$};
	\end{tikzpicture}
\end{center}
	Theo công thức Bayes, ta có $$P(A|B)=\dfrac{P(A)P(B|A)}{P(A)P(B|A)+P(\overline{A})P(B|\overline{A})}=\dfrac{0{,}0475}{0{,}0475+0{,}019}=\dfrac{5}{7}.$$
	Vậy với một bệnh nhân có kết quả xét nghiệm dương tính, xác suất để bệnh nhân đó thực sự bị nhiễm virus $X$ là $\dfrac{5}{7}$.}
\end{ex}

\begin{ex}
	Ở một địa phương $X$, xác suất để một người lớn trên $40$ tuổi mắc bệnh ung thư là $0{,}05$. Xác suất bác sĩ chẩn đoán đúng một người mắc bệnh ung thư là $0{,}78$ và chẩn đoán sai (không bị ung thư nhưng được chẩn đoán mắc bệnh) là $0{,}06$. Xác suất để một người thật sự mắc bệnh ung thư khi nhận được kết quả chẩn đoán bị ung thư bằng
	\choice
	{\True$0{,}40625$}
	{$0{,}096$}
	{$0{,}904$}
	{$0{,}59375$}
	\loigiai{Một bệnh nhân trên 40 tuổi ở địa phương X đến bác sĩ để khám bệnh ung thư.\\
	Gọi $A$ là biến cố \lq \lq Người đó mắc bệnh ung thư\rq \rq \, và $B$, $\overline{B}$ lần lượt là biến cố \lq \lq Bác sĩ chẩn đoán người đó bị ung thư\rq \rq \;và \lq \lq Bác sĩ chẩn đoán người đó không bị ung thư\rq \rq.\\
	Ta xét sơ đồ hình cây như sau
\begin{center}
		\begin{tikzpicture}[>=stealth,scale=0.7]
	%Khung 1
	\draw (-4.7,-1) rectangle (2.2,0);
	\draw (-1.3,-0.5) node{Bệnh nhân được chẩn đoán} ;
	%Mui ten 1,2
	\draw [->] (2.2,-0.5)--(3.8,1.6) node[pos=0.5,sloped,above]{$0{,}05$};
	\draw [->] (2.2,-0.5)--(3.8,-2.6) node[pos=0.5,sloped,below]{$0{,}95$};
	%Khung 2.1
	\draw (3.8,1.1) rectangle (5.1,2.1);
	\draw (8.9/2,1.6) node{$A$} ;
	%Khung 2.2
	\draw (3.8,-2.1) rectangle (5.1,-3.1);
	\draw (8.9/2,-2.6) node{$\overline{A}$} ;
	%Mui ten 3,4
	\draw [->] (5.1,1.6)--(6.5,2.6) node[pos=0.5,sloped,above]{$0{,}78$};
	\draw [->] (5.1,1.6)--(6.5,0.6) node[pos=0.5,sloped,below]{$0{,}22$};
	%Mui ten 5,6
	\draw [->] (5.1,-2.6)--(6.5,-1.6) node[pos=0.5,sloped,above]{$0{,}06$};
	\draw [->] (5.1,-2.6)--(6.5,-3.6) node[pos=0.5,sloped,below]{$0{,}94$};
	%Khung 3.1
	\draw (6.5,2.2) rectangle (7.7,3.2);
	\draw (7.1,5.4/2) node{$B$} ;
	%Khung 3.2
	\draw (6.5,1.2) rectangle (7.7,0.2);
	\draw (7.1,1.4/2) node{$\overline{B}$} ;
	%Khung 3.3
	\draw (6.5,-1.1) rectangle (7.7,-2.1);
	\draw (7.1,-3.2/2) node{$B$} ;
	%Khung 3.3
	\draw (6.5,-2.9) rectangle (7.7,-3.9);
	\draw (7.1,-3.4) node{$\overline{B}$} ;
	%Kết quả
	\draw (9.5,3.7) node{\textbf{Kết quả}};
	\draw (9.5,2.7) node{$AB$};
	\draw (9.5,0.7) node{$A\overline{B}$};
	\draw (9.5,-1.6) node{$\overline{A}B$};
	\draw (9.5,-3.4) node{$\overline{A}\overline{B}$};
	%Xác suất
	\draw (12.5,3.7) node{\textbf{Xác suất}};
	\draw (12.5,2.7) node{$0{,}039$};
	\draw (12.5,0.7) node{$0{,}011$};
	\draw (12.5,-1.6) node{$0{,}057$};
	\draw (12.5,-3.4) node{$0{,}893$};
	\end{tikzpicture}
\end{center}
	Theo công thức Bayes, ta có $$P(A|B)=\dfrac{P(A)P(B|A)}{P(A)P(B|A)+P(\overline{A})P(B|\overline{A})}=\dfrac{0{,}039}{0{,}039+0{,}057}=0{,}40625.$$
	Vậy xác suất để một người thật sự mắc bệnh ung thư khi nhận được kết quả chẩn đoán bị ung thư bằng $0{,}40625$.}
\end{ex}
\Closesolutionfile{ans}
% % \newpage
% \setcounter{ex}{0}
\Opensolutionfile{ans}[ans/ans-0-B15]
%\TN
%%==========PHẦN 1=============================================
%%==========Câu 1
\begin{ex}%[2D6H1-2]
	Cho $A$, $B$ là các biến cố của một phép thử $T$. Biết rằng $\mathrm{P}(B)>0$, xác suất của biến cố $A$ với điều kiện biến cố $B$ đã xảy ra được tính theo công thức nào sau đây?
	\choice
	{$\mathrm{P}(A\mid B)=\dfrac{\mathrm{P}(A)}{\mathrm{P}(B)}$}
	{$\mathrm{P}(A\mid B)=\dfrac{\mathrm{P}(A)}{\mathrm{P}(A B)}$}
	{\True $\mathrm{P}(A\mid B)=\dfrac{\mathrm{P}(A B)}{\mathrm{P}(B)}$}
	{$\mathrm{P}(A\mid B)=\dfrac{\mathrm{P}(A B)}{\mathrm{P}(A) \cdot \mathrm{P}(B)}$}
	\loigiai{ Dựa theo công thức tính xác suất biến cố $A$ với điều kiện $B$ thì $\mathrm{P}(A\mid B)=\dfrac{\mathrm{P}(A B)}{\mathrm{P}(B)}$ là đáp án đúng.
	}
\end{ex}
%%==========Câu 2
\begin{ex}%[2D6H1-2]
	Cho hai biến cố độc lập $A$, $B$ với $\mathrm{P}(A)=0{,}3$; $\mathrm{P}(B)=0{,}4$. Khi đó, $\mathrm{P}(A\mid B)$ bằng
	\choice{$0{,}7$}
	{$0{,}12$}
	{$0{,}4$}
	{\True $0{,}3$}
	\loigiai{
	Vì $A$ và $B$ là $2$ biến cố độc lập nên $\mathrm{P}(A\cap B)=\mathrm{P}(A)\cdot \mathrm{P}(B)=0{,}12$.\\
	$$\Rightarrow \mathrm{P}(A \mid B)=\dfrac{\mathrm{P}(A\cap B)}{\mathrm{P}(B)}=\mathrm{P}(A)=\dfrac{0{,}12}{0{,}4}=0{,}3.$$
	}
\end{ex}
%%==========Câu 3
\begin{ex}%[2D6H1-2]
	Cho hai biến cố xung khắc $A$, $B$ với $\mathrm{P}(A)=0{,}2$; $\mathrm{P}(B)=0{,}4$. Khi đó, $\mathrm{P}(A\mid B)$ bằng
	\choice{$0{,}5$}
	{$0{,}2$}
	{$0{,}4$}
	{\True $0$}
	\loigiai{
	Vì $A$ và $B$ là $2$ biến cố xung khắc nên $\mathrm{P}(A\cap B)=0$.\\
	$$\Rightarrow \mathrm{P}(A\mid B)=\dfrac{\mathrm{P}(A\cap B)}{\mathrm{P}(B)}=\dfrac{0}{0{,}4}=0.$$
	}
\end{ex}
%%==========Câu 4
\begin{ex}%[2D6H1-2]
	Cho $\mathrm{P}(A)=\dfrac{2}{5}$; $\mathrm{P}\left( B\mid A\right)=\dfrac{1}{3}$. Giá trị của $\mathrm{P}(AB)$ là 
	\choice
	{\True $\dfrac{2}{15} $}
	{$ \dfrac{3}{16}$}
	{$ \dfrac{1}{5}$}
	{$ \dfrac{4}{15}$}
	\loigiai{
	Ta có $\mathrm{P}(AB)=\mathrm{P}(A)\cdot \mathrm{P}\left( B\mid A\right)=\dfrac{2}{5}\cdot \dfrac{1}{3}=\dfrac{2}{15}$.
	}
\end{ex}
%%==========Câu 5
\begin{ex}%[2D6H1-2]
	Cho $\mathrm{P}(A)=\dfrac{2}{5}$; $\mathrm{P}\left(B\mid \overline{A}\right)=\dfrac{1}{4}$. Giá trị của $\mathrm{P}\left(B\overline{A}\right)$ là 
	\choice
	{$\dfrac{1}{7} $}
	{$ \dfrac{4}{19}$}
	{$ \dfrac{4}{21}$}
	{\True $ \dfrac{3}{20}$}
	\loigiai{
	Ta có $\mathrm{P}\left(B\overline{A}\right)=\mathrm{P}\left( B\mid \overline{A} \right)\cdot \mathrm{P}\left( \overline{A}\right) =\dfrac{1}{4}\cdot \dfrac{3}{5}=\dfrac{3}{20}$.
	}
\end{ex}
%%%%%----------Câu 1
\begin{ex}%[2D6N1-1]%[Võ Thanh Hiệp]
	Cho hai biến cố $A$ và $B$. Xác suất của biến cố $B$, tính trong điều kiện biết rằng biến cố $A$ đã xảy ra, được gọi là xác suất của $B$ với điều kiện $A$ kí hiệu là 
	\choice
	{$\mathrm{P}\left(A\mid B\right)$}
	{\True $\mathrm{P}\left(B\mid A\right)$}
	{$\mathrm{P}\left(AB\right)$}
	{$\mathrm{P}\left(B\right)$}
	\loigiai{
	Theo định nghĩa xác suất có điều kiện. Xác suất của $B$ với điều kiện $A$ kí hiệu là $\mathrm{P}\left(B\mid A\right)$.
	}	
\end{ex}
%%%%%----------Câu 2
\begin{ex}%[2D6N1-2]%[Võ Thanh Hiệp]
	Cho hai biến cố $A$ và $B$. Biết rằng xác suất của biến cố $A$ bằng $0{,}6$; xác suất của biến cố biến cố $B$ trong điều kiện biến cố $A$ đã xảy ra bằng $0{,}2$. Tính xác suất của $A$ và $B$ đều xảy ra. 
	\choice
	{\True $\dfrac{3}{25}$}
	{$\dfrac{3}{10}$}
	{$\dfrac{1}{3}$}
	{$\dfrac{2}{3}$}
	\loigiai{
	Ta có $\mathrm{P}\left(A\right) = 0{,}6$; $\mathrm{P}\left(B\mid A\right) = 0{,}2$.\\
	Suy ra 
	$\mathrm{P}\left(AB\right)=
	\mathrm{P}\left(A\right)\cdot \mathrm{P}\left(B\mid A\right)=0{,}6\cdot 0{,}2 =
	0{,}12 = \dfrac{3}{25} $.\\
	Vậy xác suất của $A$ và $B$ đều xảy ra là $\mathrm{P}\left(AB\right)= \dfrac{3}{25}$.
	}	
\end{ex}
%%%%%----------Câu 3
\begin{ex}%[2D6H1-2]%[Võ Thanh Hiệp]
	Gieo hai con xúc xắc cân đối đồng chất. Tính xác suất để tổng số chấm trên hai con xúc xắc bằng bằng $8$ nếu biết rằng ít nhất có một con xúc xắc xuất hiện mặt $3$ chấm.
	\choice
	{$\dfrac{5}{11}$}
	{$\dfrac{2}{5}$}
	{$\dfrac{11}{5}$}
	{\True $\dfrac{2}{11}$}
	\loigiai{
	Gọi $A$ là biến cố: \lq\lq  Tổng số chấm trên hai con xúc xắc bằng bằng $8$\rq\rq.\\
	Gọi $B$ là biến cố: \lq\lq  Có ít nhất có một con xúc xắc xuất hiện mặt $3$ chấm\rq\rq.\\
	Gieo hai con xúc xắc cân đối đồng chất, ta có $n\left(\Omega\right)=6\cdot 6 = 36$.\\
	$A=\{(2,6); (6,2); (3,5); (5,3); (4;4)\}$	$\Rightarrow n\left(A\right) = 5
	\Rightarrow \mathrm{P}\left(A\right)= \dfrac{5}{36}$.\\	
	$B=\{(3,1);(3,2); (3,3); (3,4); (3,5); (3,6); (1,3); (2,3); (4,3); (5,3); (6,3)\}$.\\
	$\Rightarrow n\left(B\right) =11 \Rightarrow \mathrm{P}\left(B\right)= \dfrac{11}{36}$.\\
	$AB=\{(3,5); (5,3)\}$
	$\Rightarrow n\left(AB\right) =2\Rightarrow \mathrm{P}\left(AB\right)=\dfrac{2}{36}=\dfrac{1}{18}$.\\
	Xác suất cần tính là 	
	$\mathrm{P}\left(A\mid B\right) =\dfrac{\mathrm{P}\left(AB\right) }{\mathrm{P}\left(B\right)}= \dfrac{2}{11}$.
	}	
\end{ex}
%%%%%----------Câu 4
\begin{ex}%[2D6H1-1]%[Võ Thanh Hiệp]
	Cho $A$ và $B$ là hai biến cố. Trong các mệnh đề sau, mệnh đề nào {\bf sai}?
	\choice
	{\True Với $\mathrm{P}\left(B\right)>0$. Khi đó $\mathrm{P}\left(A\mid B\right)=\mathrm{P}\left(B\right)\cdot \mathrm{P}\left(AB\right)$}
	{Nếu $A$ và $B$ là hai biến cố độc lập thì $\mathrm{P}\left(B\right)=\mathrm{P}\left(B \mid A\right)$}
	{Với $\mathrm{P}\left(B\right)>0$. Khi đó $\mathrm{P}\left(\overline{A}\mid B\right)= 1-\mathrm{P}\left(A\mid B\right)$}
	{Nếu $A$ và $B$ là hai biến cố độc lập thì $\mathrm{P}\left(A\mid B\right)=\mathrm{P}\left(A \mid \overline{B}\right)$}
	\loigiai{
	\begin{itemize}[\color{blue}\checkmark]
	\item Theo công thức nhân xác suất ta có $\mathrm{P}\left(AB \right)= \mathrm{P}\left(B\right)\cdot \mathrm{P}\left(A \mid B\right)$.\\
	Vậy $\mathrm{P}\left(A\mid B\right)=\mathrm{P}\left(B\right)\cdot \mathrm{P}\left(AB\right)$ sai.
	\item Nếu $A$ và $B$ là hai biến cố độc lập thì $\mathrm{P}\left(AB\right) =\mathrm{P}\left(A\right) \cdot \mathrm{P}\left(B\right)$.\\
	Suy ra $\mathrm{P}\left(B \mid A\right)=\dfrac{\mathrm{P}\left(BA\right)}{\mathrm{P}\left(A\right) }
	=	\dfrac{\mathrm{P}\left(B\right) \cdot \mathrm{P}\left(A\right) }{\mathrm{P}\left(A\right) }
	=\mathrm{P}\left(B\right)$.
	\item 
	\immini{Vì $\overline{A}B$ và $AB$ là hai biến cố xung khắc và $\overline{A}B \cup AB = B$ nên\\ $\mathrm{P}\left(\overline{A}B\right)=\mathrm{P}\left(B\right)-\mathrm{P}\left(AB\right)$.\\
	Với $\mathrm{P}\left(B\right)>0$.\\
	Ta có $\begin{aligned}[t]
	\mathrm{P}\left(\overline{A}\mid B\right)&=
	\dfrac{\mathrm{P}\left(\overline{A}B\right)}{\mathrm{P}\left(B\right) }\\
	&=\dfrac{\mathrm{P}\left(B\right)-\mathrm{P}\left(AB\right)}{\mathrm{P}\left(B\right) }\\
	&= 1- \dfrac{\mathrm{P}\left(AB\right)}{\mathrm{P}\left(B\right) }\\
	&=
	1-\mathrm{P}\left(A\mid B\right)
	\end{aligned}$.
	}{
	\begin{tikzpicture}[scale=1.5,>=stealth, line join=round, line cap=round]
	\draw[ pattern=north east lines]
	(0,0) to [bend left=90] (2,2) to [bend left=90] (0,0) (0,-.5) node {$A$} ;
	\draw[ pattern=north west lines]
	(1,0) to [bend left=90] (3,2) to [bend left=90] (1,0)
	(2.5,-.5) node {$B$};
	\def\miena{(0,-3.5) to [bend left=90] (2,-1.5) to [bend left=90] (0,-3.5)};
	\def\mienb{(1,-3.5) to [bend left=90] (3,-1.5) to [bend left=90] (1,-3.5)};	
	\node [circle,draw,fill=white] at (2.7,1.5){$\small \overline{A} B$};
	\node [circle,draw,fill=white] at (1.6,1.2){$\small AB$};
	\end{tikzpicture}
	}
	\item Nếu $A$ và $B$ là hai biến cố độc lập thì $A$ và $\overline{B}$ cũng là biến cố độc lập. Do đó\\
	$\mathrm{P}\left(A \mid B\right)=
	\dfrac{\mathrm{P}\left(AB\right)}{\mathrm{P}\left(B\right) }=
	\dfrac{\mathrm{P}\left(A\right) \cdot \mathrm{P}\left(B\right) }{\mathrm{P}\left(B\right) }
	=\mathrm{P}\left(A\right)$.\\
	$\mathrm{P}\left(A \mid \overline{B}\right)=
	\dfrac{\mathrm{P}\left(A\overline{B}\right)}{\mathrm{P}\left(\overline{B}\right) }=
	\dfrac{\mathrm{P}\left(A\right) \cdot \mathrm{P}\left(\overline{B}\right) }{\mathrm{P}\left(\overline{B}\right) }
	=\mathrm{P}\left(A\right)$.\\
	$\Rightarrow \mathrm{P}\left(A\mid B\right)=
	\mathrm{P}\left(A \mid \overline{B}\right)=\mathrm{P}\left(A\right)$.
	\end{itemize}
	}
\end{ex}
%%%%%----------Câu 5
\begin{ex}%[2D6N2-1]%[Võ Thanh Hiệp]
	Cho $A$ và $B$ là hai biến cố. Công thức nào dưới đây là công thức tính xác suất toàn phần?
	\choice
	{$\mathrm{P}\left(A\right)= \mathrm{P}\left(B\right)\cdot \mathrm{P}\left(A\mid B\right) + 
	\mathrm{P}\left(\overline{B}\right)\cdot \mathrm{P}\left(A\mid B\right)$}
	{$\mathrm{P}\left(A\right)= \mathrm{P}\left(B\right)\cdot \mathrm{P}\left(A\mid B\right) + 
	\mathrm{P}\left(\overline{B}\right)\cdot \mathrm{P}\left(\overline{A}\mid \overline{B}\right)$}
	{$\mathrm{P}\left(A\right)= \mathrm{P}\left(B\right)\cdot \mathrm{P}\left(A\mid B\right) + 
	\mathrm{P}\left(\overline{B}\right)\cdot \mathrm{P}\left(\overline{A}\mid B\right)$}
	{\True $\mathrm{P}\left(A\right)= \mathrm{P}\left(B\right)\cdot \mathrm{P}\left(A\mid B\right) + 
	\mathrm{P}\left(\overline{B}\right)\cdot \mathrm{P}\left(A\mid \overline{B}\right)$}
	\loigiai{
	Công thức tính xác suất toàn phần là 	
	$$\mathrm{P}\left(A\right)= \mathrm{P}\left(B\right)\cdot \mathrm{P}\left(A\mid B\right) + 
	\mathrm{P}\left(\overline{B}\right)\cdot \mathrm{P}\left(A\mid \overline{B}\right).$$
	}
\end{ex}
%%%%%----------Câu 6
\begin{ex}%[2D6N2-2]%[Võ Thanh Hiệp]
	Cho hai biến cố $A$ và $B$ có $\mathrm{P}\left(A\right) = 0{,}3$; $\mathrm{P}\left(B\right) = 0{,}5$ và
	$\mathrm{P}\left(B\mid A\right)=0{,}4$. Tính $\mathrm{P}\left(A\mid B\right)$. 
	\choice
	{$0{,}6$}
	{\True $0{,}24$}
	{$0{,}15$}
	{$0{,}5$}
	\loigiai{
	Áp dụng công thức Bayes	ta có
	\begin{eqnarray*}
	\mathrm{P}\left(A\mid B\right)&=&
	\dfrac{\mathrm{P}\left(A\right) \cdot \mathrm{P}\left(B\mid A\right) }{\mathrm{P}\left(A\right)\cdot \mathrm{P}\left(B\mid A\right)
	+ \mathrm{P}\left(\overline{A}\right)\cdot \mathrm{P}\left(B\mid \overline{A}\right)}\\
	&=& \dfrac{\mathrm{P}\left(A\right) \cdot \mathrm{P}\left(B\mid A\right) }{\mathrm{P}\left(B\right)}\\
	&=&\dfrac{0{,}3\cdot 0{,}4}{0{,}5}=0{,}24.
	\end{eqnarray*}
	}
\end{ex}
%%%%%----------Câu 7
\begin{ex}%[2D6N1-2]%[Võ Thanh Hiệp]
	Cho hai biến cố $A$ và $B$ có $\mathrm{P}\left(A\right) = 0{,}7$; $\mathrm{P}\left(B\right) = 0{,}4$ và
	$\mathrm{P}\left(AB\right)=0{,}2$. Tính xác suất biến cố $B$ với điều kiện $A$. 
	\choice
	{$\dfrac{1}{2}$}
	{$\dfrac{4}{7}$}
	{\True $\dfrac{2}{7}$}
	{$\dfrac{7}{10}$}
	\loigiai{
	Xác suất của biến cố $B$ với điều kiện $A$ là $\mathrm{P}\left(B\mid A\right)$.\\
	Ta có $\mathrm{P}\left(B\mid A\right) 
	= \dfrac{\mathrm{P}\left(AB\right) }{\mathrm{P}\left(A\right)}
	=\dfrac{0{,}2}{0{,}7}=\dfrac{2}{7}$.
	}
\end{ex}
%%%%%----------Câu 8
\begin{ex}%[2D6H1-2]%[Võ Thanh Hiệp]
	Cho hai biến cố $A$ và $B$ có $\mathrm{P}\left(A\right) = 0{,}6$; $\mathrm{P}\left(B\right) = 0{,}4$ và $\mathrm{P}\left(AB\right)=0{,}3$. Xác suất biến cố $A$ không xảy ra với điều kiện $B$ là 
	\choice
	{$\dfrac{7}{10}$}
	{$\dfrac{3}{4}$}
	{$\dfrac{1}{2}$}
	{\True $\dfrac{1}{4}$}
	\loigiai{
	Xác suất biến cố $A$ không xảy ra với điều kiện $B$ là 
	$\mathrm{P}\left(\overline{A}\mid B\right)$.\\
	Ta có $\mathrm{P}\left(\overline{A}B\right)= 1- \mathrm{P}\left(A \mid B\right)
	=1-\dfrac{\mathrm{P}\left(AB\right)}{\mathrm{P}\left(B\right)}=1-\dfrac{0{,}3}{0{,}4}=\dfrac{1}{4}$.
	}
\end{ex}
%%%%%----------Câu 9
\begin{ex}%[2D6H1-2]%[Võ Thanh Hiệp]
	Trong hộp có $7$ viên bi màu xanh, $5$ viên bi màu đỏ, các viên bi có cùng kích thước và khối lượng. An lấy ngẫu nhiên một viên bi từ hộp không trả lại, sau đó Bình lấy một viên bi trong các bi còn lại. Tính xác suất để An lấy được bi màu xanh và Bình lấy được bi màu đỏ.
	\choice
	{\True $\dfrac{35}{132}$}
	{$\dfrac{35}{144}$}
	{$\dfrac{1}{11}$}
	{$\dfrac{3}{132}$}
	\loigiai{
	Gọi $A$ là biến cố: \lq\lq  An lấy được bi màu xanh\rq\rq.\\
	$B$ là biến cố: \lq\lq  Bình lấy được bi màu đỏ\rq\rq.\\
	Ta cần tính $\mathrm{P}\left(AB\right)$.\\
	Ta có $\mathrm{P}\left(A\right) = \dfrac{7}{12}$, 
	$\mathrm{P}\left(B\mid A\right) = \dfrac{5}{11}$.\\
	Theo công thức nhân xác suất
	$\mathrm{P}\left(AB\right)= \mathrm{P}\left(A\right)\cdot \mathrm{P}\left(B\mid A\right)
	=\dfrac{7}{12} \cdot \dfrac{5}{11}=\dfrac{35}{132}$.
	}
\end{ex}
%%%%%----------Câu 10
\begin{ex}%[2D6H1-3]%[Võ Thanh Hiệp]
	\immini{ Cho sơ đồ cây như hình vẽ bên. Khẳng định nào sau đây {\bf sai}?
	\choice
	{$\mathrm{P}\left(A\right)= 0{,}3$}
	{$\mathrm{P}\left(B\right)= 0{,}54$}
	{\True $\mathrm{P}\left(A\mid B\right)=0{,}6$}
	{$\mathrm{P}\left(\overline{B}\mid \overline{A}\right)=0{,}2$}
	}{
	\tikzstyle{xs} = [rectangle ,fill=white,draw=black,rounded corners,align=center]
	\tikzstyle{bc} = [circle ,fill=white,draw=black,rounded corners,align=center]
	\begin{tikzpicture}[scale=0.7,>=stealth, font=\footnotesize, line join=round, line cap=round]
	\node[bc,text width=2.5mm] (O) at(0,0) { };
	\node[bc] (A) at ($(O)+(30:3)$) {$A$};
	\node[bc] (A1) at ($(O)+(-30:3)$){$\overline{A}$};
	\node[bc] (B) at ($(A)+(20:3)$) {$B$};
	\node[bc] (B1) at ($(A)+(-20:3)$) {$\overline{B}$};
	\node[bc] (B2) at ($(A1)+(20:3)$) {$B$};
	\node[bc] (B3) at ($(A1)+(-20:3)$) {$\overline{B}$};
	\draw[->] {(O)--node [above,xs,sloped] {$0{,}3$}(A)} ;
	\draw[->] {(O)--node [below,xs,sloped] {$0{,}7$}(A1)} ;
	\draw[->] {(A)--node [above,xs,sloped] {$0{,}4$}(B)} ;
	\draw[->] {(A)--node [below,xs,sloped] {$0{,}6$}(B1)} ;
	\draw[->] {(A1)--node [above,xs,sloped] {$0{,}8$}(B2)} ;
	\draw[->] {(A1)--node [below,xs,sloped] {$0{,}2$}(B3)} ;
	\end{tikzpicture}
	}
	\loigiai{
	\begin{center}	
	\tikzstyle{xs} = [rectangle ,fill=white,draw=black,rounded corners,align=center]
	\tikzstyle{bc} = [circle ,fill=white,draw=black,rounded corners,align=center]
	\begin{tikzpicture}[scale=0.7,>=stealth, font=\footnotesize, line join=round, line cap=round]
	\node[bc,text width=2.5mm] (O) at(0,0) { };
	\node[bc] (A) at ($(O)+(32:3)$) {$A$};
	\node[bc] (A1) at ($(O)+(-32:3)$){$\overline{A}$};
	\node[bc] (B) at ($(A)+(20:3)$) {$B$};
	\node[bc] (B1) at ($(A)+(-20:3)$) {$\overline{B}$};
	\node[bc] (B2) at ($(A1)+(20:3)$) {$B$};
	\node[bc] (B3) at ($(A1)+(-20:3)$) {$\overline{B}$};
	\draw[->] {(O)--node [above,xs,sloped] {$0{,}3$}(A)} ;
	\draw[->] {(O)--node [below,xs,sloped] {$0{,}7$}(A1)} ;
	\draw[->] {(A)--node [above,xs,sloped] {$0{,}4$}(B)} ;
	\draw[->] {(A)--node [below,xs,sloped] {$0{,}6$}(B1)} ;
	\draw[->] {(A1)--node [above,xs,sloped] {$0{,}8$}(B2)} ;
	\draw[->] {(A1)--node [below,xs,sloped] {$0{,}2$}(B3)} ;
	\node[bc] (AB) at ($(B)+(0:2)$) {$AB$};
	\node[bc] (AB1) at ($(B1)+(0:2)$) {$A\overline{B}$};
	\node[bc] (AB2) at ($(B2)+(0:2)$) {$\overline{A}B$};
	\node[bc] (AB3) at ($(B3)+(0:2)$) {$\overline{A}\,\overline{B}$};
	\end{tikzpicture}
	\end{center}
	Từ sơ đồ cây suy ra\\
	$\mathrm{P}\left(A\right)= 0{,}3$; $\mathrm{P}\left(\overline{A}\right)= 0{,}7$;
	$\mathrm{P}\left(A B\right)= 0{,}12$;\\
	$\mathrm{P}\left(B\mid A\right)= 0{,}4$; $\mathrm{P}\left(B\mid \overline{A}\right)= 0{,}8$;
	$\mathrm{P}\left(\overline{B}\mid \overline{A}\right)=0{,}2$.\\
	Suy ra
	$\mathrm{P}\left(B\right)= \mathrm{P}\left(A\right)\cdot \mathrm{P}(B\mid A)
	+ \mathrm{P}\left(\overline{A}\right)\cdot \mathrm{P}(B\mid \overline{A})
	= 0{,}3\cdot 0{,}4 + 0{,}7\cdot 0{,}6 = 0{,}54 $.\\
	$\mathrm{P}\left(A\mid B\right)= \dfrac{\mathrm{P}\left(AB\right)}{\mathrm{P}\left(B\right)}
	=\dfrac{0{,}12}{0{,}54}=\dfrac{2}{9}$.\\
	Vậy $\mathrm{P}\left(A\mid B\right)=0{,}6$ sai.
	}
\end{ex}
%%%%%----------Câu 11
\begin{ex}%[2D6H2-3]%[Võ Thanh Hiệp]
	Cho hai biến cố ngẫu nhiên $A$ và $B$. Biết rằng $\mathrm{P}\left(A\mid B\right)= \dfrac{2}{3} \mathrm{P}\left(B\mid A\right)$ và $P\left(AB\right) \ne 0$. Tính tỉ số $\dfrac{\mathrm{P}\left(A\right)}{\mathrm{P}\left(B\right)}$.
	\choice
	{$\dfrac{3}{2}$}
	{\True $\dfrac{2}{3}$}
	{$\dfrac{1}{3}$}
	{$\dfrac{1}{2}$}
	\loigiai{
	Theo công thức Bayes ta có\\
	$\mathrm{P}\left(B\mid A\right)=
	\dfrac{\mathrm{P}\left(B\right)\cdot \mathrm{P}\left(A\mid B\right)}{\mathrm{P}\left(A\right)}
	\Leftrightarrow 
	\dfrac{\mathrm{P}\left(A\mid B\right)}{\mathrm{P}\left(B\mid A\right)}
	=\dfrac{\mathrm{P}\left(A\right) }{\mathrm{P}\left(B\right)}=\dfrac{2}{3}$. 
	}
\end{ex}
%%%%%----------Câu 12
\begin{ex}%[2D6H2-2]%[Võ Thanh Hiệp]
	Cho hai biến cố ngẫu nhiên $A$ và $B$. Biết rằng $\mathrm{P}\left(A\right)= \dfrac{3}{5}$; $\mathrm{P}\left(B\mid A\right)= \dfrac{1}{4}$ và $\mathrm{P}\left(B\mid \overline{A}\right)= \dfrac{1}{3}$. Tính $\mathrm{P}\left(B\overline{A}\right)$.
	\choice
	{$\dfrac{43}{180}$}
	{\True$\dfrac{2}{15}$}
	{$\dfrac{3}{15}$}
	{$\dfrac{17}{180}$}
	\loigiai{
	Ta có $\mathrm{P}\left(A\right)= \dfrac{3}{5}\Rightarrow \mathrm{P}\left(\overline{A}\right)=\dfrac{2}{5}$.\\
	Áp dụng công thức xác suất toàn phần ta có\\
	$\mathrm{P}\left(B\right)=\mathrm{P}\left(A\right)\cdot \mathrm{P}\left(B\mid A\right)
	+\mathrm{P}\left(\overline{A}\right)\cdot \mathrm{P}\left(B\mid \overline{A}\right)
	=\dfrac{3}{5}\cdot\dfrac{1}{4}+\dfrac{2}{5}\cdot\dfrac{1}{3}=\dfrac{17}{60}$.\\
	Áp dụng công thức nhân xác suất ta có\\
	$\mathrm{P}\left(B\overline{A}\right)=\mathrm{P}\left(\overline{A}\right)\cdot \mathrm{P}\left(B\mid \overline{A}\right)
	=\dfrac{2}{5}\cdot\dfrac{1}{3}=\dfrac{2}{15}$.
	}
\end{ex}
%%==========Câu 6
\begin{ex}%[2D6H1-2]
	Cho $\mathrm{P}(A)=\dfrac{2}{5}$; $\mathrm{P}\left( B\mid A\right)=\dfrac{1}{3}$; $\mathrm{P}\left(B\mid \overline{A}\right)=\dfrac{1}{4}$. Giá trị của $\mathrm{P}(B)$ là 
	\choice
	{$\dfrac{19}{60} $}
	{\True $ \dfrac{17}{60}$}
	{$ \dfrac{9}{20}$}
	{$ \dfrac{7}{30}$}
	\loigiai{Áp dụng công thức Bayes ta có
	$$\mathrm{P}(A\mid B)=\dfrac{\mathrm{P}(A)\cdot \mathrm{P}(B\mid A)}{\mathrm{P}(A)\cdot \mathrm{P}(B\mid A)+\mathrm{P}\left( \overline{A}\right) \cdot \mathrm{P}\left( B\mid \overline{A}\right) }=\dfrac{\dfrac{2}{5}\cdot \dfrac{1}{3}}{\dfrac{2}{5}\cdot \dfrac{1}{3}+\dfrac{3}{5}\cdot \dfrac{1}{4}}=\dfrac{8}{17}.$$
	Khi đó 
	$\mathrm{P}(B)=\dfrac{\mathrm{P}(AB)}{\mathrm{P}(A\mid B)}=\dfrac{2}{15}:\dfrac{8}{17}=\dfrac{17}{60}$.
	}
\end{ex}
%%==========Câu 7
\begin{ex}%[2D6H1-2]
	An có một túi gồm một số chiếc kẹo cùng loại, chỉ khác màu, trong đó có $6$ chiếc kẹo sô-cô-la đen, còn lại là $4$ chiếc kẹo sô-cô-la trắng. An lấy ngẫu nhiên $1$ chiếc kẹo trong túi để cho Bình, rồi lại lấy ngẫu nhiên tiếp $1$ chiếc kẹo nữa trong túi và cũng đưa cho Bình. Xác suất để Bình nhận được $2$ chiếc kẹo sô-cô-la đen là 
	\choice
	{\True $\dfrac{1}{3} $}
	{$ \dfrac{1}{4}$}
	{$ \dfrac{2}{5}$}
	{$ \dfrac{3}{7}$}
	\loigiai{
	Gọi $A$ là biến cố \lq\lq  An lấy lần $1$ được $1$ chiếc kẹo sô-cô-la đen\rq\rq\, và $B$ là biến cố \lq\lq  An lấy lần $2$ được $1$ chiếc kẹo sô-cô-la đen\rq\rq.\\
	Khi đó $AB$ là biến cố \lq\lq  Cả hai lần đều lấy được kẹo sô-cô-la đen\rq\rq.\\
	Ta có $\mathrm{P}(A)=\dfrac{n(A)}{n(\Omega)}=\dfrac{3}{5}$.\\
	Sau khi lấy $1$ chiếc kẹo sô-cô-la đen thì xác suất để chọn $1$ chiếc kẹo sô-cô-la đen trong hộp đựng $5$ chiếc kẹo sô-cô-la đen, còn lại là $4$ chiếc kẹo sô-cô-la trắng là $\mathrm{P}(B\mid A)=\dfrac{5}{9}$.\\
	Khi đó $\mathrm{P}(AB)=\mathrm{P}(A)\cdot \mathrm{P}(B\mid A)=\dfrac{3}{5}\cdot \dfrac{5}{9}=\dfrac{1}{3}$.\\
	Xác suất để Bình nhận được $2$ chiếc kẹo sô-cô-la đen là $\dfrac{1}{3}$.
	}
\end{ex}
%%==========Câu 8
\begin{ex}%[0D0H2-9]
	Người ta nhập hai lô hàng vào kho. Lô thứ nhất chứa $10$ sản phẩm, trong đó có $3$ phế phẩm. Lô thứ hai có $4$ phế phẩm và $8$ sản phẩm tốt. Chọn ngẫu nhiên một sản phẩm. Xác suất chọn được một sản phẩm tốt là
	\choice
	{\True $\dfrac{15}{22}$}
	{$\dfrac{7}{15}$}
	{$\dfrac{7}{22}$}
	{$\dfrac{83}{242}$}
	\loigiai{
	Gọi $A$ là biến cố “chọn được sản phẩm tốt”, theo đề bầi, kho có $22$ sản phẩm, trong đó có $15$ sản phẩm tốt nên $n(A)=15$, $n(\Omega)=22.$\\
	Vậy $\mathrm{P}(A)=\dfrac{n(A)}{n(\Omega)}=\dfrac{15}{22}$.
	}
\end{ex}
%%==========Câu 9
\begin{ex}%[2D6V2-2]
	Cho $A$, $B$ là các biến cố thỏa mãn $\mathrm{P}(\overline{A}\cap \overline{B})=0{,}35$; $\mathrm{P}(A)=0{,}25$; $\mathrm{P}(B)=0{,}6$. Giá trị của $\mathrm{P}(A\mid B)$ bằng
	\choice
	{$\dfrac{1}{5}$}
	{\True$\dfrac{1}{3}$}
	{$\dfrac{7}{15}$}
	{$\dfrac{2}{3}$}
	\loigiai{
	Ta có $$\mathrm{P}(\overline{A}\cap \overline{B}) = \mathrm{P} \left(\overline A \right) \mathrm{P} \left( \overline B \mid \overline A \right) \Rightarrow \mathrm{P} \left(\overline B \mid \overline A \right) = \dfrac{\mathrm{P}(\overline{A}\cap \overline{B})}{\mathrm{P} \left( \overline A\right)} = \dfrac{0{,}35}{0{,}75} = \dfrac{7}{15}.$$
	Suy ra $$\mathrm{P} \left(B\mid \overline A\right) = 1 - \dfrac{7}{15} = \dfrac{8}{15}.$$
	Theo công thức xác suất toàn phần, ta có
	\begin{eqnarray*}
	&&	\mathrm{P}\left( B \right) = \mathrm{P}\left(B\mid A\right)\mathrm{P}\left( A \right) + \mathrm{P}\left( B\mid \overline A \right)\mathrm{P}\left( \overline A \right)\\
	&\Rightarrow& \mathrm{P}\left( B\mid A \right) = \dfrac{\mathrm{P}\left( B \right) - \mathrm{P} \left( {B\mid \overline A } \right)\mathrm{P}\left( {\overline A } \right)}{\mathrm{P}\left( A \right)} = \dfrac{0{,}6 - \dfrac{8}{{15}} \cdot 0{,}75}{0{,}25} = 0{,}8.	
	\end{eqnarray*}
	Theo công thức Bayes, ta được
	$$\mathrm{P}\left( {A\mid B} \right) = \dfrac{{\mathrm{P}\left( A \right)\mathrm{P}\left( {B\mid A} \right)}}{{\mathrm{P}\left( B \right)}} = \dfrac{{0{,}25 \cdot 0{,}8}}{{0{,}6}} = \dfrac{1}{3}.$$}
\end{ex}
%%==========Câu 10
\begin{ex}%[2D6V1-3]
	Một bệnh viện có hai phòng khám là phòng A và phòng B với khả năng lựa chọn của bệnh nhân là như nhau. Tỉ lệ bệnh nhân nam có ở phòng A và phòng B lần lượt là $60\%$ và $40\%$. Một người bệnh được chọn ngẫu nhiêu từ hai phòng khám và biết người này là nam, xác suất để người bệnh được chọn đến từ phòng A là
	\choice
	{\True $0{,}6$}
	{$0{,}5$}
	{$0{,}4$}
	{$0{,}3$}
	\loigiai{Một người bệnh được chọn ngẫu nhiên từ hai phòng khám.\\
	Gọi $X$ là biến cố \lq \lq Người đó đến từ phòng khám A\rq \rq \, và $Y$, $\overline{Y}$ lần lượt là biến cố \lq \lq Người đó là nam\rq \rq \; và \lq \lq Người đó không là nam\rq \rq.\\
	Ta có sơ đồ hình cây sau
	\begin{center}
		\begin{tikzpicture}[>=stealth,scale=0.8]
	%Khung 1
	\draw (-2,-1) rectangle (2.2,0);
	\draw (0.1,-0.5) node{Bệnh nhân được chọn} ;
	%Mui ten 1,2
	\draw [->] (2.2,-0.5)--(3.8,1.6) node[pos=0.5,sloped,above]{$0{,}5$};
	\draw [->] (2.2,-0.5)--(3.8,-2.6) node[pos=0.5,sloped,below]{$0{,}5$};
	%Khung 2.1
	\draw (3.8,1.1) rectangle (5.1,2.1);
	\draw (8.9/2,1.6) node{$X$} ;
	%Khung 2.2
	\draw (3.8,-2.1) rectangle (5.1,-3.1);
	\draw (8.9/2,-2.6) node{$\overline{X}$} ;
	%Mui ten 3,4
	\draw [->] (5.1,1.6)--(6.5,2.6) node[pos=0.5,sloped,above]{$0{,}6$};
	\draw [->] (5.1,1.6)--(6.5,0.6) node[pos=0.5,sloped,below]{$0{,}4$};
	%Mui ten 5,6
	\draw [->] (5.1,-2.6)--(6.5,-1.6) node[pos=0.5,sloped,above]{$0{,}4$};
	\draw [->] (5.1,-2.6)--(6.5,-3.6) node[pos=0.5,sloped,below]{$0{,}6$};
	%Khung 3.1
	\draw (6.5,2.2) rectangle (7.7,3.2);
	\draw (7.1,5.4/2) node{$Y$} ;
	%Khung 3.2
	\draw (6.5,1.2) rectangle (7.7,0.2);
	\draw (7.1,1.4/2) node{$\overline{Y}$} ;
	%Khung 3.3
	\draw (6.5,-1.1) rectangle (7.7,-2.1);
	\draw (7.1,-3.2/2) node{$Y$} ;
	%Khung 3.3
	\draw (6.5,-2.9) rectangle (7.7,-3.9);
	\draw (7.1,-3.4) node{$\overline{Y}$} ;
	%Kết quả
	\draw (9.5,3.7) node{\textbf{Kết quả}};	
	\draw (9.5,2.7) node{$XY$};
	\draw (9.5,0.7) node{$X \overline{Y}$};
	\draw (9.5,-1.6) node{$\overline{X}Y$};
	\draw (9.5,-3.4) node{$\overline{X}\overline{Y}$};
	%Xác suất
	\draw (12.5,3.7) node{\textbf{Xác suất}};	
	\draw (12.5,2.7) node{$0{,}3$};
	\draw (12.5,0.7) node{$0{,}2$};
	\draw (12.5,-1.6) node{$0{,}2$};
	\draw (12.5,-3.4) node{$0{,}3$};	
	\end{tikzpicture}
	\end{center}
	\noindent Theo công thức Bayes, ta có $$\mathrm{P}(X\mid Y)=\dfrac{\mathrm{P}(X)\mathrm{P}(Y\mid X)}{\mathrm{P}(X)\mathrm{P}(Y\mid X)+\mathrm{P}(\overline{X})\mathrm{P}(Y\mid \overline{X})}=\dfrac{0{,}3}{0{,}3+0{,}2}=0{,}6.$$
	Vậy với một người bệnh được chọn ngẫu nhiêu từ hai phòng khám và biết người này là nam, xác suất để người đó đến từ phòng A là $0{,}6$.}
\end{ex}
%%==========Câu 11
\begin{ex}%[2D6V1-3]
	Ở một địa phương $X$, xác suất để một người lớn trên $40$ tuổi mắc bệnh ung thư là $0{,}05$. Xác suất bác sĩ chẩn đoán đúng một người mắc bệnh ung thư là $0{,}78$ và chẩn đoán sai (không bị ung thư nhưng được chẩn đoán mắc bệnh) là $0{,}06$. Xác suất để một người thật sự mắc bệnh ung thư khi nhận được kết quả chẩn đoán bị ung thư bằng
	\choice
	{\True$0{,}40625$}
	{$0{,}096$}
	{$0{,}904$}
	{$0{,}59375$}
	\loigiai{Một bệnh nhân trên 40 tuổi ở địa phương X đến bác sĩ để khám bệnh ung thư.\\
	Gọi $A$ là biến cố \lq \lq Người đó mắc bệnh ung thư\rq \rq \, và $B$, $\overline{B}$ lần lượt là biến cố \lq \lq Bác sĩ chẩn đoán người đó bị ung thư\rq \rq \;và \lq \lq Bác sĩ chẩn đoán người đó không bị ung thư\rq \rq.\\
	Ta xét sơ đồ hình cây như sau
	\begin{center}
		\begin{tikzpicture}[>=stealth,scale=0.8]
	%Khung 1
	\draw (-3.5,-1) rectangle (2.2,0);
	\draw (-1.3/2,-0.5) node{Bệnh nhân được chẩn đoán} ;
	%Mui ten 1,2
	\draw [->] (2.2,-0.5)--(3.8,1.6) node[pos=0.5,sloped,above]{$0{,}05$};
	\draw [->] (2.2,-0.5)--(3.8,-2.6) node[pos=0.5,sloped,below]{$0{,}95$};
	%Khung 2.1
	\draw (3.8,1.1) rectangle (5.1,2.1);
	\draw (8.9/2,1.6) node{$A$} ;
	%Khung 2.2
	\draw (3.8,-2.1) rectangle (5.1,-3.1);
	\draw (8.9/2,-2.6) node{$\overline{A}$} ;
	%Mui ten 3,4
	\draw [->] (5.1,1.6)--(6.5,2.6) node[pos=0.5,sloped,above]{$0{,}78$};
	\draw [->] (5.1,1.6)--(6.5,0.6) node[pos=0.5,sloped,below]{$0{,}22$};
	%Mui ten 5,6
	\draw [->] (5.1,-2.6)--(6.5,-1.6) node[pos=0.5,sloped,above]{$0{,}06$};
	\draw [->] (5.1,-2.6)--(6.5,-3.6) node[pos=0.5,sloped,below]{$0{,}94$};
	%Khung 3.1
	\draw (6.5,2.2) rectangle (7.7,3.2);
	\draw (7.1,5.4/2) node{$B$} ;
	%Khung 3.2
	\draw (6.5,1.2) rectangle (7.7,0.2);
	\draw (7.1,1.4/2) node{$\overline{B}$} ;
	%Khung 3.3
	\draw (6.5,-1.1) rectangle (7.7,-2.1);
	\draw (7.1,-3.2/2) node{$B$} ;
	%Khung 3.3
	\draw (6.5,-2.9) rectangle (7.7,-3.9);
	\draw (7.1,-3.4) node{$\overline{B}$} ;
	%Kết quả
	\draw (9.5,3.7) node{\textbf{Kết quả}};	
	\draw (9.5,2.7) node{$AB$};
	\draw (9.5,0.7) node{$A\overline{B}$};
	\draw (9.5,-1.6) node{$\overline{A}B$};
	\draw (9.5,-3.4) node{$\overline{A}\overline{B}$};
	%Xác suất
	\draw (12.5,3.7) node{\textbf{Xác suất}};	
	\draw (12.5,2.7) node{$0{,}039$};
	\draw (12.5,0.7) node{$0{,}011$};
	\draw (12.5,-1.6) node{$0{,}057$};
	\draw (12.5,-3.4) node{$0{,}893$};
	\end{tikzpicture}
	\end{center}
	Theo công thức Bayes, ta có $$\mathrm{P}(A\mid B)=\dfrac{\mathrm{P}(A)\mathrm{P}(B\mid A)}{\mathrm{P}(A)\mathrm{P}(B\mid A)+\mathrm{P}(\overline{A})\mathrm{P}(B\mid \overline{A})}=\dfrac{0{,}039}{0{,}039+0{,}057}=0{,}40625.$$	 
	Vậy xác suất để một người thật sự mắc bệnh ung thư khi nhận được kết quả chẩn đoán bị ung thư bằng $0{,}40625$.}
\end{ex} 
%%==========Câu 12
\begin{ex}%[2D6V2-3]
	Một loại vaccine được tiêm ở địa phương $X$. Người có bệnh nền thì với xác suất $0{,}35$ có phản ứng phụ sau tiêm, người không có bệnh nền thì chỉ có phản ứng phụ sau tiêm với xác suất $0{,}16$. Chọn ngẫu nhiên một người được tiêm vaccine và người này có phản ứng phụ. Tính xác suất để người này có bệnh nền, biết rằng tỉ lệ người có bệnh nền ở địa phương $X$ là $18\%$.
	\choice
	{\True $\dfrac{315}{971}$ }
	{ $\dfrac{31}{971}$}
	{$0{,}16$} 
	{$0{,}063$}
	\loigiai{
	Gọi $A$ là biến cố \lq\lq  Người được chọn có bệnh nền\rq\rq\, và $B$ là biến cố \lq\lq  Người này có phản ứng phụ sau tiêm\rq\rq.\\
	Ta có $\mathrm{P}(A)=0{,}18$; $\mathrm{P}(\overline{A})=0{,}82$.\\
	$\mathrm{P}(B\mid A)$ là xác suất để một người bệnh có phản ứng sau tiêm với điều kiện có bệnh nền, suy ra $$\mathrm{P}(B\mid A)=0{,}35.$$
	$\mathrm{P}(B\mid \overline{A})$ là xác suất để một người bệnh có phản ứng sau tiêm với điều kiện không có bệnh nền, suy ra $$\mathrm{P}(B\mid \overline{A})=0{,}16.$$
	Theo công thức Bayes, ta được 
	$$\mathrm{P}(A\mid B)=\dfrac{\mathrm{P}(A)\cdot \mathrm{P}(B\mid A)}{\mathrm{P}(A)\cdot \mathrm{P}(B\mid A)+\mathrm{P}(\overline{A})\cdot \mathrm{P}(B\mid \overline{A})}=\dfrac{0{,}18\cdot 0{,}35}{0{,}18\cdot 0{,}35+0{,}82\cdot 0{,}16}=\dfrac{315}{971}.$$
	}
\end{ex}
%==============================================================
\Closesolutionfile{ans}
\indapan{6}{ans/ans-0-B15}
%%==========HẾT PHẦN 1=========================================
\Opensolutionfile{ans}[ans/ans-0-B15-DS]

%%%%%----------Câu 13
\begin{ex}%[2D6V1-2]%[Nguyễn Khánh Trọng]
	Một nhà máy thực hiện khảo sát toàn bộ công nhân về sự hài lòng của họ về điều kiện làm việc tại phân xưởng. Kết quả khảo sát như sau:
	\begin{center}
	\begin{tabular}{|c|ccll|}
	\hline
	\multirow{2}{*}{Khảo sát công nhân} & \multicolumn{4}{c|}{Kết quả khảo sát} \\ \cline{2-5} 
	& \multicolumn{1}{c|}{Hài lòng} & \multicolumn{3}{c|}{Không hài lòng} \\ \hline
	Số công nhân xưởng I & \multicolumn{1}{c|}{23} & \multicolumn{3}{c|}{12} \\ \hline
	Số công nhân xưởng II & \multicolumn{1}{c|}{25} & \multicolumn{3}{c|}{15} \\ \hline
	\end{tabular}
	\end{center}
	Gặp ngẫu nhiên một công nhân của nhà máy. Gọi $A$ là biến cố \lq\lq  Công nhân đó làm việc tại phân xưởng I\rq\rq \, và $B$ là biến cố \lq\lq  Công nhân đó hài lòng với điều kiện làm việc tại phân xưởng\rq\rq. Xét tính đúng sai của các phát biểu sau.
	\choiceTF
	{\True Xác suất của biến cố $A$ là $\dfrac{7}{15}$}
	{Xác suất của biến cố $B$ là $0{,}65$}
	{\True Xác suất gặp được công nhân không hài lòng với điều kiện làm việc tại phân xưởng biết công nhân đó thuộc xưởng I là $\dfrac{12}{35}$}
	{Xác suất gặp được công nhân thuộc xưởng II biết công nhân đó hài lòng với điều kiện làm việc tại phân xưởng là $0{,}52$}
	\loigiai{
	Gặp ngẫu nhiên một công nhân của nhà máy, ta có $n(\Omega)=23+12+25+15=75$. 
	\begin{itemchoice}
	\itemch {\bf Đúng.}\\
	Ta có $n(A)=23+12=35$.\\
	Xác suất của biến cố $A$ là 
	$\mathrm{P}(A)=\dfrac{35}{75}=\dfrac{7}{15}$.
	\itemch {\bf Sai.} \\
	Ta có $n(B)=23+25=48$. \\
	Xác suất của biến cố $B$ là 
	$\mathrm{P}(B)=\dfrac{48}{75}=\dfrac{16}{25}$.
	\itemch Đúng. Ta có $n(\overline{B}A)=12\Rightarrow \mathrm{P}(\overline{B}A)=\dfrac{12}{75}=\dfrac{4}{25}$.\\
	Do đó
	$\mathrm{P}(\overline{B}\mid A)=\dfrac{\mathrm{P}(\overline{B}A)}{\mathrm{P}(A)}=\dfrac{\dfrac{4}{25}}{\dfrac{7}{15}}=\dfrac{12}{35}$.
	\itemch {\bf Sai.}\\
	Ta có $n(\overline{A}B)=25\Rightarrow\mathrm{P}(\overline{A}B)
	=\dfrac{25}{75}=\dfrac{1}{3}$.\\
	Do đó
	$\mathrm{P}(\overline{A}\mid B)=\dfrac{\mathrm{P}(\overline{A}B)}{\mathrm{P}(B)}
	=\dfrac{\dfrac{1}{3}}{\dfrac{16}{25}}=\dfrac{25}{48}$.
	\end{itemchoice}
	}
\end{ex}
%%%%%----------Câu 14
\begin{ex}%[2D6H1-2]%[Nguyễn Khánh Trọng]
	Cho hai biến cố $A$ và $B$ có $P(A) = 0{,}35$; $P(B \mid A)= 0{,}6$ và $P(B \mid \overline{A})= 0{,}2$. Mỗi phát biểu dưới đây đúng hay sai?
	\choiceTF
	{\True $\mathrm{P}(\overline{A})=0{,}65$}
	{$\mathrm{P}(AB)=0{,}2$}
	{\True $\mathrm{P}(\overline{B}\mid A)=0{,}4$}
	{\True $\mathrm{P}(B)=0{,}34$}
	\loigiai{
	\begin{itemchoice}
	\itemch {\bf Đúng.}\\
	Vì $\mathrm{P}(\overline{A})=1-\mathrm{P}(A)=1-0{,}35=0{,}65$.
	\itemch {\bf Sai.}\\
	Vì $\mathrm{P}(AB)=\mathrm{P}(B\mid A)\cdot\mathrm{P}(A)
	= 0{,}35\cdot0{,}6=0{,}21$.
	\itemch {\bf Đúng.}\\
	Vì $\mathrm{P}(\overline{B}\mid A)
	=1-\mathrm{P}(B\mid A)=1-0{,}6=0{,}4$.
	\itemch {\bf Đúng.} \\
	Vì $\mathrm{P}(B)=\mathrm{P}(A)\cdot\mathrm{P}(B\mid A)
	+\mathrm{P}(\overline{A})\cdot\mathrm{P}(B\mid\overline{A})
	=0{,}35\cdot 0{,}6+0{,}65\cdot0{,}2=0{,}34$.
	\end{itemchoice}
	}
\end{ex}
%%%%%----------Câu 15
\begin{ex}%[2D6H1-3]%[Nguyễn Khánh Trọng]
	\immini{Cho sơ đồ cây như hình vẽ. Xét tính đúng sai của các phát biểu sau.
	\choiceTF
	{\True $ \mathrm{P}\left(A\right)= 0{,}25$}
	{$\mathrm{P}\left(A\overline{B}\right)=\dfrac{1}{8}$}
	{$\mathrm{P}\left(B\right)= 0{,}65$}
	{\True $\mathrm{P}\left(A\mid B\right)=0{,}16$}
	}{
	\tikzstyle{xs} = [rectangle ,fill=white,draw=black,rounded corners,align=center]
	\tikzstyle{bc} = [circle ,fill=white,draw=black,rounded corners,align=center]
	\begin{tikzpicture}[scale=1.2,>=stealth, font=\footnotesize, line join=round, line cap=round]
	\node[bc,text width=2.5mm] (O) at(0,0) { };
	\node[bc] (A) at ($(O)+(30:3)$) {$A$};
	\node[bc] (A1) at ($(O)+(-30:3)$){$\overline{A}$};
	\node[bc] (B) at ($(A)+(20:3)$) {$B$};
	\node[bc] (B1) at ($(A)+(-20:3)$) {$\overline{B}$};
	\node[bc] (B2) at ($(A1)+(20:3)$) {$B$};
	\node[bc] (B3) at ($(A1)+(-20:3)$) {$\overline{B}$};
	\draw[->] {(O)--node [above,xs,sloped] {$?$}(A)} ;
	\draw[->] {(O)--node [below,xs,sloped] {$\mathrm{P}(\overline{A})=0{,}75$}(A1)} ;
	\draw[->] {(A)--node [above,xs,sloped] { $\mathrm{P}(B\mid A)=0{,}4$}(B)} ;
	\draw[->] {(A)--node [below,xs,sloped] {$?$}(B1)} ;
	\draw[->] {(A1)--node [above,xs,sloped] {$?$}(B2)} ;
	\draw[->] {(A1)--node [below,xs,sloped] {$\mathrm{P}(\overline{B}\mid \overline{A})=0{,}3$}(B3)} ;
	\end{tikzpicture}
	}
	\loigiai{
	\begin{center}	
	\tikzstyle{xs} = [rectangle ,fill=white,draw=black,rounded corners,align=center]
	\tikzstyle{bc} = [circle ,fill=white,draw=black,rounded corners,align=center]
	\begin{tikzpicture}[scale=1,>=stealth, font=\footnotesize, line join=round, line cap=round]
	\node[bc,text width=2.5mm] (O) at(0,0) { };
	\node[bc] (A) at ($(O)+(32:3)$) {$A$};
	\node[bc] (A1) at ($(O)+(-32:3)$){$\overline{A}$};
	\node[bc] (B) at ($(A)+(20:3)$) {$B$};
	\node[bc] (B1) at ($(A)+(-20:3)$) {$\overline{B}$};
	\node[bc] (B2) at ($(A1)+(20:3)$) {$B$};
	\node[bc] (B3) at ($(A1)+(-20:3)$) {$\overline{B}$};
	\draw[->] {(O)--node [above,xs,sloped] {$0{,}25$}(A)} ;
	\draw[->] {(O)--node [below,xs,sloped] {$0{,}75$}(A1)} ;
	\draw[->] {(A)--node [above,xs,sloped] {$0{,}4$}(B)} ;
	\draw[->] {(A)--node [below,xs,sloped] {$0{,}6$}(B1)} ;
	\draw[->] {(A1)--node [above,xs,sloped] {$0{,}7$}(B2)} ;
	\draw[->] {(A1)--node [below,xs,sloped] {$0{,}3$}(B3)} ;
	\node[bc] (AB) at ($(B)+(0:2)$) {$AB$};
	\node[bc] (AB1) at ($(B1)+(0:2)$) {$A\overline{B}$};
	\node[bc] (AB2) at ($(B2)+(0:2)$) {$\overline{A}B$};
	\node[bc] (AB3) at ($(B3)+(0:2)$) {$\overline{A}\,\overline{B}$};
	\end{tikzpicture}
	\end{center}
	Từ sơ đồ cây suy ra\\
	\begin{itemchoice}
	\itemch {\bf Đúng.}\\
	$\mathrm{P}\left(A\right) = 1 - \mathrm{P}\left(\overline{A}\right) = 1- 0{,}75 = 0{,}25$.
	\itemch {\bf Sai.}\\
	Vì $\mathrm{P}\left(A \overline{B}\right)
	=\mathrm{P}\left(A\right) \cdot \mathrm{P}\left(\overline{B}\mid A\right)
	= 0{,}25\cdot 0{,}6=0{,}15$.
	\itemch {\bf Sai.}\\
	Ta có $\mathrm{P}\left(B\right)= \mathrm{P}\left(A\right)\cdot \mathrm{P}(B\mid A)
	+ \mathrm{P}\left(\overline{A}\right)\cdot \mathrm{P}(B\mid \overline{A})
	= 0{,}25\cdot 0{,}4 + 0{,}75\cdot 0{,}7 = 0{,}625$.
	\itemch {\bf Đúng.}\\
	Ta có $\mathrm{P}\left(A B\right)= \mathrm{P}\left(A\right)\cdot \mathrm{P}\left(B\mid A\right)
	=0{,}25\cdot0{,}4=0{,}1$.\\
	Suy ra $\mathrm{P}\left(A\mid B\right)= \dfrac{\mathrm{P}\left(AB\right)}{\mathrm{P}\left(B\right)}
	=\dfrac{0{,}1}{0{,}625}=0{,}16$.
	\end{itemchoice}
	}
\end{ex}
%%%%%----------Câu 16
\begin{ex}%[2D6V2-2]%[Nguyễn Khánh Trọng]
	Trong một trường học, tỉ lệ học sinh nữ là $55\%$. Tỉ lệ học sinh nữ và tỉ lệ học sinh nam tham gia câu lạc bộ tiếng anh lần lượt là $20\%$ và $15\%$. Gặp ngẫu nhiên $1$ học sinh của trường. 
	Gọi $A$ là biến cố \lq\lq  Học sinh đó là nữ\rq\rq\, và $B$ là biến cố \lq\lq  Học sinh đó tham gia câu lạc bộ tiếng Anh\rq\rq. Xét tính đúng sai của các phát biểu sau.
	\choiceTF
	{\True $\mathrm{P}(\overline{A})=0{,}45$}
	{$\mathrm{P}(B\mid \overline{A}) = 0{,}15$ và $\mathrm{P}(\overline{B}\mid A) = 0{,}2$}
	{Xác suất để học sinh đó có tham gia câu lạc bộ tiếng Anh là $0{,}1675$}
	{\True Biết rằng học sinh có tham gia câu lạc bộ tiếng Anh. Xác suất học sinh đó là nam bằng $\dfrac{27}{71}$}
	\loigiai{
	\begin{itemchoice}
	\itemch {\bf Đúng.}\\
	Do tỉ lệ học sinh nữ là $55\%$ nên
	$\mathrm{P}(A) = 0{,}55$ và $\mathrm{P}(\overline{A}) = 1 - 0{,}55 = 0{,}45$.
	\itemch {\bf Sai.}\\
	Do tỉ lệ học sinh nữ và tỉ lệ học sinh nam tham gia câu lạc bộ tiếng Anh lần lượt là $20\%$ và $15\%$ nên $\mathrm{P}(B\mid A) = 0{,}2$ và $\mathrm{P}(B\mid \overline{A}) = 0{,}15$.\\
	Suy ra $\mathrm{P}(\overline{B}\mid A) =1-\mathrm{P}(B\mid A)=1-0{,}2=0{,}8$.
	\itemch {\bf Sai.}\\
	Xác suất để học sinh đó có tham gia câu lạc bộ tiếng Anh là
	$$\mathrm{P}(B) = \mathrm{P}(A)\cdot\mathrm{P}(B\mid A) + \mathrm{P}(\overline{A})\cdot\mathrm{P}(B\mid\overline{A}) = 0{,}55 \cdot 0{,}2 + 0{,}45 \cdot 0{,}15 = 0{,}1775.$$
	\itemch {\bf Đúng.}\\
	Do học sinh có tham gia câu lạc bộ tiếng Anh nên xác suất học sinh đó là nam là
	$$\mathrm{P}(\overline{A}\mid B) = \dfrac{\mathrm{P}(\overline{A})\cdot \mathrm{P}(B\mid \overline{A})}{\mathrm{P}(B)}=
	\dfrac{0{,}45\cdot 0{,}15}{0{,}1775}=\dfrac{27}{71}.$$
	\end{itemchoice}
	}
\end{ex}
%%==========Câu 13
\begin{ex}%[2D6H1-2]
	Lớp $12A$ có $40$ học sinh, trong đó có $25$ học sinh tham gia câu lạc bộ Tiếng Anh, $16$ học sinh tham gia câu lạc bộ Toán, $12$ học sinh vừa tham gia câu lạc bộ tiếng Anh vừa tham gia câu lạc bộ Toán. Chọn ngẫu nhiên $1$ học sinh. Xét các biến cố sau\\
	$A\colon$ \lq\lq  Học sinh được chọn tham gia câu lạc bộ Tiếng Anh\rq\rq;\\
	$B\colon$ \lq\lq  Học sinh được chọn tham gia câu lạc bộ Toán\rq\rq.\\
	Xét tính đúng, sai của các khẳng định sau
	\choiceTF
	{$\mathrm{P}(A)=0{,}4$}
	{$\mathrm{P}(B)=0{,}625$}
	{\True $\mathrm{P}(A | B)=0{,}75$}
	{\True $\mathrm{P}(B | A)=0{,}48$}
	\loigiai{
	\begin{itemchoice}
	\itemch Sai, vì xác suất của biến cố $A$ là $\mathrm{P}(A)=\dfrac{25}{40}=0{,}625$.
	\itemch Sai, vì xác suất của biến cố $B$ là $\mathrm{P}(B)=\dfrac{16}{40}=0{,}4$.
	\itemch Đúng, vì số học sinh vừa tham gia câu lạc bộ tiếng Anh vừa tham gia câu lạc bộ Toán là $12$, số học sinh tham gia câu lạc bộ Toán là $16$ nên $\mathrm{P}(A | B)=\dfrac{12}{16}=0{,}75$.
	\itemch Đúng, vì số học sinh vừa tham gia câu lạc bộ tiếng Anh vừa tham gia câu lạc bộ Toán là $12$, số học sinh tham gia câu lạc bộ Tiếng Anh là $25$ nên $\mathrm{P}(B | A)=\dfrac{12}{25}=0{,}48$.
	\end{itemchoice}
	}
\end{ex}
%%==========Câu 14
\begin{ex}%[2D6V1-3]
	\immini{Cho sơ đồ hình cây như hình bên. Xét tính đúng, sai của các khẳng định sau
	\choiceTF
	{$\mathrm{P(AB)}=0{,}48$}
	{ $\mathrm{P(A| B)}=0{,}5$}
	{$\mathrm{P(\overline{A}| B)}=0{,}3$}
	{\True $\dfrac{\mathrm{P}(B) \mathrm{P}(\overline{A} | B)}{\mathrm{P}(\overline{A})}=0{,}6$}}
	{\begin{tikzpicture}[scale=.2,>=stealth]
	\tikzstyle{block} = [rectangle, draw, fill=blue!10\text{,} rounded corners, text centered, text width = 10em, minimum height = 2em]
	\node (c1) {};
	\node (c2)[above right = 1.5cm of c1] {$A$};
	\node at (0.5,5){\fbox{$0\text{,}2$}};
	\node at (0.5,-5){\fbox{$0\text{,}8$}};
	\node (c3) [below right= 1.5cm of c1]{$\overline{A}$};
	\node at (12,11.5){\fbox{$0\text{,}7$}};
	\node (c4) at (21.5, 12){$B$};
	\node (c5) at (21.5, 2){$\overline{B}$};
	\node at (12,3){\fbox{$0\text{,}3$}};
	\node (c6) at (21.5, -4){$B$};
	\node at (12,-4){\fbox{$0\text{,}6$}};
	\node (c7) at (21.5, -14){$\overline{B}$};
	\node at (12,-13){\fbox{$0\text{,}4$}};
	\draw[->] (c1.east) -- (c2.west);
	\draw[->] (c1.east) -- (c3.west);
	\draw[->] (c2.east) -- (c4.west);
	\draw[->] (c2.east) -- (c5.west);
	\draw[->] (c3.east) -- (c6.west);
	\draw[->] (c3.east) -- (c7.west);
	\end{tikzpicture}}
	\loigiai{
	\begin{itemchoice}
	\itemch Sai, vì $\mathrm{P}(B)=0\text{,}2\cdot 0\text{,}7+0\text{,}8\cdot 0\text{,}6=0\text{,}62$, $\mathrm{P}(\overline{A})=0\text{,}8$ và $\mathrm{P}(AB)=0\text{,}2\cdot 0\text{,}7=0\text{,}14$.
	\itemch Sai, vì $\mathrm{P}(A| B)=\dfrac{\mathrm{P}(AB)}{\mathrm{P}(B)}=\dfrac{0\text{,}14}{0\text{,}62}=\dfrac{7}{31}$.
	\itemch Sai, vì $\mathrm{P}(\overline{A} | B)=1-\mathrm{P}(A| B)=1-\dfrac{7}{31}=\dfrac{24}{31}$.
	\itemch Đúng, vì $\dfrac{\mathrm{P}(B) \mathrm{P}(\overline{A} | B)}{\mathrm{P}(\overline{A})}=\dfrac{0\text{,}62\cdot \dfrac{24}{31}}{0\text{,}8}=0\text{,}6$.
	\end{itemchoice}	
	}
\end{ex}
%%==========Câu 15
\begin{ex}%[2D6V1-3]
	Trong một hộp có $18$ quả bóng bàn loại I và $2$ quả bóng bàn loại II, các quả bóng bàn có hình dạng và kích thước như nhau. Một học sinh lấy ngẫu nhiên lần lượt $2$ quả bóng bàn (lấy không hoàn lại) trong hộp.\\ Xét tính đúng, sai của các khẳng định sau
	\choiceTF
	{Xác suất để lần thứ nhất lấy được quả bóng bàn loại II là $\dfrac{9}{10}$}
	{\True Xác suất để lần thứ hai lấy được quả bóng bàn loại II, biết lần thứ nhất lấy được quả bóng bàn loại II, là $\dfrac{1}{19}$}
	{Xác suất để cả hai lần đều lấy được quả bóng bàn loại II là $\dfrac{9}{190}$}
	{\True Xác suất để ít nhất $1$ lần lấy được quả bóng bàn loại I là $\dfrac{189}{190}$}
	\loigiai{
	Xét các biến cố\\
	$A\colon$ \lq\lq  Lần thứ nhất lấy được quả bóng bàn loại II\rq\rq;\\
	$B\colon$ \lq\lq  Lần thứ hai lấy được quả bóng bàn loại II\rq\rq. 
	\begin{itemchoice}
	\itemch Sai, vì Xác suất để lần thứ nhất lấy được quả bóng bàn loại II là $\mathrm{P}(A)=\dfrac{2}{20}=\dfrac{1}{10}$.
	\itemch Đúng, vì sau khi lấy $1$ quả bóng bàn loại II thì chỉ còn $1$ quả bóng bàn loại II trong hộp.\\ Suy ra xác suất để lần thứ hai lấy được quả bóng bàn loại II, biết lần thứ nhất lấy được quả bóng bàn loại II là $\mathrm{P}(B | A)=\dfrac{1}{19}$.
	\itemch Sai, vì xác suất để cả hai lần đều lấy được quả bóng bàn loại II là
	$$
	\mathrm{P}(C)=\mathrm{P}(A \cap B)=\mathrm{P}(A) \cdot \mathrm{P}(B | A)=\dfrac{1}{10} \cdot \dfrac{1}{19}=\dfrac{1}{190}.
	$$
	\itemch Đúng, vì để ít nhất $1$ lần lấy được quả bóng bàn loại I là
	$$
	\mathrm{P}(\overline {C})=1-\mathrm{P}(C)=1-\frac{1}{190}=\frac{189}{190} \text {. }
	$$
	\end{itemchoice}
	}
\end{ex}
%%==========Câu 16
\begin{ex}%[2D6V2-3]
	Một xưởng máy sử dụng một loại linh kiện được sản xuất từ hai cơ sở I và II. Số linh kiện do cơ sở I sản xuất chiếm $61 \%$, số linh kiện do cơ sở II sản xuất chiếm $39 \%$. Tỉ lệ linh kiện đạt tiêu chuẩn của cơ sở I, cơ sở II lần lượt là $93 \%$, $82 \%$. Kiểm tra ngẫu nhiên $1$ linh kiện ở xường máy. Xét các biến cố\\
	$A_1\colon$ \lq\lq  Linh kiện được kiểm tra do cơ sở I sản xuất\rq\rq;\\
	$A_2\colon$\lq\lq  Linh kiện được kiểm tra do cơ sở II sản xuất\rq\rq;\\
	$B\colon$ \lq\lq  Linh kiện được kiểm tra đạt tiêu chuẩn\rq\rq.\\
	Xét tính đúng, sai của các khẳng định sau
	\choiceTF
	{$\mathrm{P}\left(A_1\right)=0{,}39$} 
	{\True $\mathrm{P}\left(B | A_2\right)=0{,}82$}
	{$\mathrm{P}(B)=0{,}89$}
	{\True $\mathrm{P}\left(A_1 | B\right)=0{,}55$}
	\loigiai{
	\begin{itemchoice}
	\itemch Sai, vì $\mathrm{P}\left(A_1\right)=0{,}61$.
	\itemch Đúng, vì $\mathrm{P}\left(A_2\right)=0{,}39$;~ $\mathrm{P}\left(B | A_1\right)=0{,}93 ;~\mathrm{P}\left(B | A_2\right)=0{,}82$.
	\itemch Đúng, vì theo công thức xác suất toàn phần, ta có
	$$
	\mathrm{P}(B)=\mathrm{P}\left(A_1\right) \cdot \mathrm{P}\left(B | A_1\right)+\mathrm{P}\left(A_2\right) \cdot \mathrm{P}\left(B | A_2\right)=0{,}61\cdot 0{,}93+0{,}39\cdot 0{,}82=0{,}8871.
	$$
	\itemch Sai, vì theo công thức Bayes, ta có $$\mathrm{P}\left(A_1 | B\right)=\dfrac{\mathrm{P}\left(A_1\right) \cdot \mathrm{P}\left(B | A_1\right)}{\mathrm{P}(B)}=\dfrac{0{,}61 \cdot 0{,}93}{0{,}8871} \approx 0{,}64.$$
	\end{itemchoice}
	}
\end{ex}
%==============================================================
\Closesolutionfile{ans}
\indapan{3}{ans/ans-0-B15-DS}
%%==========HẾT PHẦN 2=========================================
\Opensolutionfile{ans}[ans/ans-0-B15-KQ]
%\TNSA
%%==========PHẦN 3=============================================
%%==========Câu 17
\begin{ex}%[2D6V2-2]
	Có hai hộp đựng các viên bi cùng kích thước và khối lượng. Hộp thứ nhất chứa $5$ viên bi đỏ và $5$ viên bi xanh, hộp thứ hai chứa $6$ viên bi đỏ và $4$ viên bi xanh. Lấy ngẫu nhiên một viên bi từ hộp thứ nhất chuyển sang hộp thứ hai, sau đó lấy ra ngẫu nhiên một viên bi từ hộp thứ hai. Tính xác suất để viên bi được lấy ra từ hộp thứ hai là viên bi đỏ (làm tròn đến hàng phần trăm).
	\shortans{$0{,}59$}
	\loigiai{
	Xét phép thử lấy ngẫu nhiên một viên bi từ hộp thứ nhất chuyển sang hộp thứ hai, sau đó lấy ra ngẫu nhiên một viên bi từ hộp thứ hai. Xét các biến cố sau
	\begin{itemize}
	\item $A$ là biến cố \lq \lq Viên bi được lấy ra từ hộp thứ hai là bi đỏ \rq \rq;
	\item $C$ là biến cố \lq \lq Viên bi được lấy ra từ hộp thứ hai là bi của hộp thứ nhất \rq \rq;
	\item $\overline{C}$ là biến cố \lq \lq Viên bi được lấy ra từ hộp thứ hai là bi của hộp thứ hai\rq \rq.
	\end{itemize}
	Sau khi chuyển một viên bi từ hộp thứ nhất sang hộp thứ hai thì hộp thứ hai có $11$ viên bi. Ta có 
	$$\mathrm{P}(C)=\dfrac{1}{11};\, \mathrm{P}(\overline{C})=\dfrac{10}{11}.$$
	Xác suất để viên bi được lấy ra từ hộp thứ hai là bi đỏ của hộp thứ nhất 
	$$\mathrm{P}(A\mid C)=\dfrac{5}{10}=\dfrac{1}{2}.$$
	Xác suất để viên bi được lấy ra từ hộp thứ hai là bi đỏ của hộp thứ hai 
	$$\mathrm{P}(A\mid \overline{C})=\dfrac{6}{10}=\dfrac{3}{5}.$$
	Áp dụng công thức xác suất toàn phần, ta có
	$$\mathrm{P}(A) = \mathrm{P}(C)\cdot \mathrm{P}(A\mid C) + \mathrm{P}(\overline{C})\cdot P(A\mid \overline{C}) =\dfrac{1}{11}\cdot \dfrac{1}{2}+\dfrac{10}{11}\cdot \dfrac{3}{5}=\dfrac{13}{22}\approx 0{,}59.$$
	}
\end{ex}
%%==========Câu 18
\begin{ex} %[2D6H1-4]
	Trong số $40$ học sinh lớp $12$A, có $22$ em đăng kí thi ngành Kinh tế, $25$ em đăng kí thi ngành Luật, $3$ em không đăng kí thi cả hai ngành này. Chọn ngẫu nhiên một học sinh, biết rằng em đó đăng kí thi ngành luật. Tính xác suất để em đó đăng kí thi ngành kinh tế. 	
	\shortans{$0{,}4$}
	\loigiai{	Gọi $A$ và $B$ lần lượt là tập hợp các học sinh đăng kí thi ngành kinh tế và ngành luật.
	Ta có $|A\cup B| =40-3 = 37$.\\
	Số sinh viên đăng kí cả hai ngành là $ |A\cap B| = |A|+|B|-|A\cup B|=10$.\\
	Vậy chọn ngẫu nhiên một học sinh, biết rằng em đó đăng kí thi ngành luật thì xác suất để em đó đăng kí thi ngành kinh tế là	 $\dfrac{10}{25}=\dfrac{2}{5}=0{,}4$.
	}
\end{ex}
%%==========Câu 19
\begin{ex} %[2D6V2-2]
	Trong một tuần, Sơn chọn ngẫu nhiên ba ngày chạy bộ buổi sáng. Nếu chạy bộ thì xác suất Sơn ăn thêm một quả trứng vào bữa sáng hôm đó là $0{,}7$ . Nếu không chạy bộ thì xác suất Sơn ăn thêm một quả trứng vào bữa sáng hôm đó là $0{,}25$. Chọn ngẫu nhiên một ngày trong tuần của Sơn. Tính xác suất để hôm đó Sơn chạy bộ nếu biết rằng bữa sáng hôm đó Sơn có ăn thêm một quả trứng (làm tròn đến hàng phần trăm).
	\par\shortans{$0{,}68$}	
	\loigiai{
	Gọi $A$	là biến cố \lq \lq Chọn ngày Sơn ăn trứng \rq \rq .\\
	Gọi $B_1$ là biến cố \lq \lq Ngày Sơn chạy bộ \rq \rq và $B_2$ là biến cố \lq \lq Ngày Sơn không chạy bộ \rq \rq.\\
	Ta có $\mathrm{P}(B_1) = \dfrac{3}{7}$ và $\mathrm{P}(B_2)=\dfrac{4}{7}$.\\
	Xác suất ngày ăn trứng và chạy bộ là
	$\mathrm{P}(A\mid B_1) = 0{,}7$.\\
	Xác suất ngày ăn trứng và không chạy bộ là
	$\mathrm{P}(A\mid B_2) = 0{,}25$.\\
	Khi đó ta có $\mathrm{P}(A) =\mathrm{P}(B_1)\cdot \mathrm{P}(A\mid B_1)+ \mathrm{P}(B_2)\cdot \mathrm{P}(A\mid B_2) = \dfrac{3}{7}\cdot 0{,}7+ \dfrac{4}{7}\cdot 0{,}25 = \dfrac{31}{70}$.\\
	Theo công thức Bayes, ta có xác suất hôm chọn Sơn chạy bộ mà trong bữa sáng có ăn một quả trứng là 
	$$\mathrm{P}(B_1\mid A) = \dfrac{\mathrm{P}(B_1A)}{P(A)} = \dfrac{\mathrm{P}(B_1)\cdot \mathrm{P}(A\mid B_1)}{\mathrm{P}(A)} = \dfrac{\dfrac{3}{7}\cdot 0{,}7}{ \dfrac{31}{70}}=\dfrac{21}{31}\approx 0{,}68.$$	
	}
\end{ex}
%%==========Câu 20
\begin{ex}%[2D6V2-2]
	Giả sử có khoảng $40 \%$ thư điện tử (email) gửi đến một địa chỉ là thư rác. Người ta sử dụng một thuật toán để phân loại thư rác, biết rằng thuật toán này có thể phân loại đến $99 \%$ thư rác và tỉ lệ sai sót khi phân loại thư bình thường thành thư rác là $5 \%$. Tính xác suất một thư điện tử là thư bình thường nếu thư này đã được phân loại đúng (làm tròn đến hàng phần trăm).
	\shortans{$0{,}59$}
	\loigiai{
	Ta có công thức
	$$	\mathrm{P}(A \mid B)=\dfrac{\mathrm{P}(B \mid A) \cdot \mathrm{P}(A)}{\mathrm{P}(B)}$$
	Trong đó
	\begin{itemize}
	\item $A$: Thư điện tử là thư bình thường.
	\item $B$: Thư đã được phân loại đúng.
	\item 	$\mathrm{P}(A)$: Xác suất một thư điện tử là thư bình thường ban đầu.\\
	Vì có $40 \%$ thư rác, nên $\mathrm{P}(A)=$ $1-0{,}4=0{,}6$.
	\item $\mathrm{P}(B \mid A)$: Xác suất một thư bình thường được phân loại đúng.\\
	Do tỉ lệ sai sót là $5 \%$, nên $\mathrm{P}(B \mid A)=1-0{,}05=0{,}95$.
	\item $\mathrm{P}(B)$: Xác suất một thư nào đó được phân loại đúng, tính bằng tổng xác suất một thư rác được phân loại đúng và xác suất một thư bình thường được phân loại đúng
	\end{itemize}
	$$\begin{aligned}[t]
	\mathrm{P}(B)&=\mathrm{P}(B \mid A) \cdot \mathrm{P}(A)+\mathrm{P}(B \mid \overline{A}) \cdot \mathrm{P}(\overline{A}) \\&
	=0{,}95 \cdot 0{,}6+0{,}99 \cdot 0{,}4=0{,}97.
	\end{aligned}$$
	Áp dụng định lý Bayes, ta có
	$$\mathrm{P}(A \mid B)=\dfrac{\mathrm{P}(B \mid A) \cdot \mathrm{P}(A)}{\mathrm{P}(B)}=\dfrac{0{,}95 \cdot 0{,}6}{0{,}97}=\dfrac{57}{97}\approx0{,}59.$$
	}
\end{ex}
% \begin{ex}%[2D6V2-1]
% 	\immini{Cho sơ đồ hình cây như hình bên. Tính giá trị của biểu thức $\dfrac{\mathrm{P}(B) \mathrm{P}(\overline{A} \mid B)}{\mathrm{P}(\overline{A})}$.
% 	\par\shortans{$0{,}6$
% 	}
% 	}
% 	{\begin{tikzpicture}[scale=.2,>=stealth]
% 	\tikzstyle{block} = [rectangle, draw, fill=blue!10\text{,} rounded corners, text centered, text width = 10em, minimum height = 2em]
% 	\node (c1) {};
% 	\node (c2)[above right = 1.5cm of c1] {$A$};
% 	\node at (0.5,5){\fbox{$0\text{,}2$}};
% 	\node at (0.5,-5){\fbox{$0\text{,}8$}};
% 	\node (c3) [below right= 1.5cm of c1]{$\overline{A}$};
% 	\node at (12,11.5){\fbox{$0\text{,}7$}};
% 	\node (c4) at (21.5, 12){$B$};
% 	\node (c5) at (21.5, 2){$\overline{B}$};
% 	\node at (12,3){\fbox{$0\text{,}3$}};
% 	\node (c6) at (21.5, -4){$B$};
% 	\node at (12,-4){\fbox{$0\text{,}6$}};
% 	\node (c7) at (21.5, -14){$\overline{B}$};
% 	\node at (12,-13){\fbox{$0\text{,}4$}};
% 	\draw[->] (c1.east) -- (c2.west);
% 	\draw[->] (c1.east) -- (c3.west);
% 	\draw[->] (c2.east) -- (c4.west);
% 	\draw[->] (c2.east) -- (c5.west);
% 	\draw[->] (c3.east) -- (c6.west);
% 	\draw[->] (c3.east) -- (c7.west);
% 	\end{tikzpicture}}
% 	\loigiai{
% 	\begin{itemize}
% 	\item Ta có $\mathrm{P}(B)=0{,}2\cdot 0{,}7+0{,}8\cdot 0{,}6=0{,}62$; $\mathrm{P}(\overline{A})=0{,}8$ và $\mathrm{P}(AB)=0{,}2\cdot 0{,}7=0{,}14$.
% 	\item $\mathrm{P}(A\mid B)=\dfrac{\mathrm{P}(AB)}{\mathrm{P}(B)}=\dfrac{0{,}14}{0{,}62}=\dfrac{7}{31}$.
% 	\item $\mathrm{P}(\overline{A} \mid B)=1-\mathrm{P}(A\mid B)=1-\dfrac{7}{31}=\dfrac{24}{31}$.
% 	\item $\dfrac{\mathrm{P}(B) \mathrm{P}(\overline{A} \mid B)}{\mathrm{P}(\overline{A})}=\dfrac{0{,}62\cdot \dfrac{24}{31}}{0{,}8}=0{,}6$.
% 	\end{itemize}	
% 	}
% \end{ex}
\begin{ex}%[2D6V2-4]
	Năm $2001$, Cộng đồng châu Âu có làm một đợt kiểm tra rất rộng rãi các con bò để phát hiện những con bị bệnh bò điên. Không có xét nghiệm nào cho kết quả chính xác $100 \%$. Một loại xét nghiệm, mà ở đây ta gọi là xét nghiệm A cho kết quả như sau: khi con bò bị bệnh bò điên thì xác suất để có phản ứng dương tính trong xét nghiệm A là $70 \%$ còn khi con bò không bị bệnh thì xác suất để có phản ứng dương tính trong xét nghiệm A là $10 \%$. Biết rằng tỉ lệ bò bị mắc bệnh bò điên ở Hà Lan là $13$ con trên $1~000~000$ con \textit{(Nguồn: F. M. Dekking et al., Amodern introduction to probability and statistics Understanding why and how, Springer, $2005$)}. Khi con bò ở Hà Lan có phản ứng dương tính với xét nghiệm A thì xác suất để nó bị mắc bệnh bò điên là $\mathrm{P}$, tính $1000P$ (lấy gần đúng đến hàng phần trăm).
	\shortans{$ 0{,}09$}
	\loigiai{
	Xét hai biến cố\\	
	$N$: \lq\lq  Con bò được chọn bị nhiễm bệnh\rq\rq.\\	
	$D$: \lq\lq  Con bò được chọn có phản ứng dương tính\rq\rq.\\	
	Khi đó, ta có\\	
	\[\mathrm{P}(N)=\dfrac{13}{1 000 000}=0{,}000013; \qquad \mathrm{P}(\overline{N})=1-\mathrm{P}(N)=0{,}999987;\]	
	\[\mathrm{P}(D|N)=70\%=0{,}7; \qquad \mathrm{P}(D|\overline{N})=10\%=0{,}1.\]
	Áp dụng công thức Bayes, ta có\\
	$\mathrm{P}(N|D)=\dfrac{\mathrm{P}(D|N) \cdot \mathrm{P}(N)}{\mathrm{P}(N) \cdot \mathrm{P}(D|N)+\mathrm{P}(\overline{N})\mathrm{P}(D|\overline{N})}=\dfrac{0{,}7 \cdot 0{,}000013}{0{,}7 \cdot 0{,}000013+0{,}1 \cdot 0{,}999987}\approx 0{,}009\%$.\\
	Do đó $1000\mathrm{P}\approx1000\cdot 0{,}009\%\approx 0{,}09$.
	}
\end{ex}

%\TNSA
%%%%%----------Câu 17
\begin{ex}%[2D6V1-2]%[Võ Thanh Hiệp]
	Lớp 12A có $40$ học sinh, trong đó có $22$ bạn nữ và $18$ bạn nam. Có $3$ tên Hiền gồm hai bạn nam và một bạn nữ. Thầy giáo chọn ngẫu nhiên một bạn lên bảng làm bài tập. Tính xác suất để chọn đúng bạn tên Hiền là bạn nam ({\it kết quả làm tròn đến hàng phần trăm}).
	\shortans{$0{,}25$}
	\loigiai{
	Gọi $A$ là biến cố: \lq\lq  Chọn bạn tên Hiền\rq\rq.\\
	Gọi $B$ là biến cố: \lq\lq  Chọn bạn nam\rq\rq.\\
	Ta có $\mathrm{P}\left(A\right) = \dfrac{3}{40}$,
	$\mathrm{P}\left(B\right) = \dfrac{18}{40}=\dfrac{9}{20}$, 
	$\mathrm{P}\left(AB\right) = \dfrac{2}{18}=\dfrac{1}{9}$.\\
	Xác suất chọn đúng bạn tên Hiền với điều kiện là bạn nam $\mathrm{P}\left(A\mid B\right)$.\\
	Ta có $\mathrm{P}\left(A\mid B\right)=\dfrac{\mathrm{P}\left(AB\right)}{\mathrm{P}\left(B\right)}=
	\dfrac{20}{81} \approx 0{,}25$.
	}
\end{ex}
%%%%%----------Câu 18
\begin{ex}%[2D6V1-2]%[Võ Thanh Hiệp]
	Một hộp đựng $30$ viên bi kích thước, chất liệu như nhau, trong đó có $20$ viên bi xanh và $10$ viên bi trắng. Lấy ngẫu nhiên ra một viên bi không bỏ lại trong hộp, rồi lại lấy ngẫu nhiên ra một viên bi nữa. Tính xác suất để lấy được một viên bi trắng ở lần thứ nhất và một viên bi xanh ở lần thứ hai ({\it kết quả làm tròn đến hàng phần trăm}).
	\shortans{$0{,}23$}
	\loigiai{
	Gọi $A$ là biến cố: \lq\lq  Lấy được một viên bi trắng ở lần thứ nhất\rq\rq.\\
	Gọi $B$ là biến cố: \lq\lq  Lấy được một viên bi xanh ở lần thứ hai\rq\rq.\\
	Ta có $P\left(A\right) = \dfrac{10}{30}=\dfrac{1}{3}$.\\
	Nếu $A$ đã xảy ra, tức là một viên bi trắng đã được lấy ra ở lần thứ nhất, thì còn lại trong hộp $29$ viên bi trong đó số viên bi xanh là 20, do đó $P\left(B\mid A\right)=\dfrac{20}{29}$.\\
	Xác suất để lấy được một viên bi trắng ở lần thứ nhất và một viên bi xanh ở lần thứ hai là $P\left(A B\right)$.\\
	Ta có $P\left(A B\right) = P\left(A\right) \cdot P\left(B\mid A\right)=
	\dfrac{1}{3} \cdot \dfrac{20}{29}=\dfrac{20}{87}\approx 0{,}23$.
	}
\end{ex}
%%%%%----------Câu 19
\begin{ex}%[2D6V2-2]%[Võ Thanh Hiệp]
	Khảo sát tỉ lệ người dân trong một xã nghiện thuốc lá là $20\%$; tỉ lệ người bị bệnh phổi trong số người nghiện thuốc lá là $70\%$, trong số người không nghiện thuốc lá là $15\%$. Hỏi khi ta gặp ngẫu nhiên một người dân của của xã đó thì khả năng mà người đó bị bệnh phổi là bao nhiêu $\%$?
	\shortans{$26$}
	\loigiai{
	Gọi $A$ là biến cố: \lq\lq  Người nghiện thuốc lá\rq\rq.\\
	Suy ra $\overline{A}$ là biến cố: \lq\lq  Người không nghiện thuốc lá\rq\rq. \\
	Gọi $B$ là biến cố: \lq\lq  Người bị bệnh phổi\rq\rq.\\
	Xác suất người nghiện thuốc lá là $\mathrm{P}\left(A\right) = 20\% = 0{,}2$.\\
	Xác suất người không nghiện thuốc lá là $\mathrm{P}\left(\overline{A}\right) =1- \mathrm{P}\left(A\right) = 0{,}8$.\\
	Xác suất người bị bệnh phổi trong số người nghiện thuốc lá là $70\% \Rightarrow \mathrm{P}\left(B\mid A\right) = 0{,}7$.\\
	Xác suất người bị bệnh phổi không nghiện thuốc lá là $15\% \Rightarrow \mathrm{P}\left(B\mid \overline{A}\right) = 0{,}15$.\\
	Xác suất người bị bệnh phổi là\\
	$\mathrm{P}\left(B\right) = \mathrm{P}\left(A\right) \cdot \mathrm{P}\left(B\mid A\right)+
	\mathrm{P}\left(\overline{A}\right) \cdot \mathrm{P}\left(B\mid \overline{A}\right)
	= 0{,}2\cdot 0{,}7+0{,}8\cdot 0{,}15=0{,}26=26\%$.
	}
\end{ex}
%%%%%----------Câu 20
\begin{ex}%[2D6V2-2]%[Võ Thanh Hiệp]
	Có $2$ xạ thủ loại I và $8$ xạ thủ loại II, xác suất bắn trúng đích của các loại xạ thủ loại I là $0{,}9$ và loại II là $0{,}7$. Chọn ngẫu nhiên ra một xạ thủ và xạ thủ đó bắn một viên đạn. Tìm xác suất để viên đạn đó trúng đích.
	\par\shortans{$0{,}74$}
	\loigiai{
	Gọi $A$ là biến cố: \lq\lq  Viên đạn bắn trúng đích\rq\rq.\\	
	Gọi $B$ là biến cố: \lq\lq  Chọn xạ thủ loại I\rq\rq.\\
	Gọi $C$ là biến cố: \lq\lq  Chọn xạ thủ loại II\rq\rq.\\	
	Xác suất biến cố $B$ là $\mathrm{P}\left(B\right)=\dfrac{2}{10}=0{,}2$.
	Xác suất biến cố $C$ là $\mathrm{P}\left(C\right)=\dfrac{8}{10}=0{,}8$.
	Xác suất biến cố viên đạn đó trúng đích với điều kiện là xạ thủ loại I là
	$\mathrm{P}\left(A\mid B\right)= 0{,}9$.\\
	Xác suất biến cố viên đạn đó trúng đích với điều kiện là xạ thủ loại II là
	$\mathrm{P}\left(A\mid C\right)= 0{,}7$.\\
	$\mathrm{P}\left(A\right) = \mathrm{P}\left(B\right) \cdot \mathrm{P}\left(A\mid B\right)+
	\mathrm{P}\left(C\right) \cdot \mathrm{P}\left(A\mid C\right)
	= 0{,}2\cdot 0{,}9+0{,}8\cdot 0{,}7=0{,}74$.
	}
\end{ex}
%%%%%----------Câu 21
\begin{ex}%[2D6V2-3]%[Võ Thanh Hiệp]
	Khảo sát sự yêu thích môn Toán của hai lớp 12 của một trường. Lớp 12A1 có 40 học sinh và có $80\%$ học sinh thích môn Toán, lớp 12A2 có 32 học sinh và có $75\%$ học sinh thích môn Toán. Chọn ngẫu nhiên một học sinh. Biết rằng bạn đó yêu thích môn Toán, tính xác suất bạn đó học lớp 12A1 ({\it kết quả làm tròn đến hàng phần trăm}).
	\shortans{$0{,}57$}
	\loigiai{
	Gọi $A$ là biến cố: \lq\lq  Học sinh yêu thích môn Toán\rq\rq.\\
	Gọi $B$ là biến cố: \lq\lq  Học sinh lớp 12A1\rq\rq.\\
	Gọi $C$ là biến cố: \lq\lq  Học sinh lớp 12A2\rq\rq.\\	
	Theo đề bài ta có\\
	$\mathrm{P}\left(B\right) = \dfrac{40}{72} =\dfrac{5}{9}$;
	$\mathrm{P}\left(C\right) = \dfrac{32}{72} =\dfrac{4}{9}$;
	$\mathrm{P}\left(A\mid B\right) = 80\% =\dfrac{4}{5}$;
	$\mathrm{P}\left(A\mid C\right) = 75\% =\dfrac{3}{4}$.\\ 
	Áp dụng công thức xác suất toàn phần, ta có\\
	$\mathrm{P}\left(A\right) = \mathrm{P}\left(B\right) \cdot \mathrm{P}\left(A\mid B\right)+
	\mathrm{P}\left(C\right) \cdot \mathrm{P}\left(A\mid C\right)
	= \dfrac{5}{9}\cdot \dfrac{4}{5}+ \dfrac{4}{9}\cdot \dfrac{3}{4}=\dfrac{7}{9}$.\\
	Xác suất cần tìm là 
	$\mathrm{P}\left(B\mid A\right)=
	\dfrac{\mathrm{P}\left(A\mid B\right)\cdot \mathrm{P}\left(B\right) }{\mathrm{P}\left(A\right)}=
	\dfrac{\dfrac{4}{5}\cdot\dfrac{5}{9}}{\dfrac{7}{9}}=\dfrac{4}{7}\approx 0{,}57$.
	}
\end{ex}
%%%%%----------Câu 22
\begin{ex}%[2D6V2-3]%[Võ Thanh Hiệp]
	Hộp thứ nhất có 6 viên bi đỏ và 4 viên bi xanh, hộp thứ hai có 4 viên bi đỏ và 6 viên bi xanh, các viên bi có cùng khối lượng và kích thước. Lấy ngẫu nhiên 1 viên bi từ hộp thứ nhất bỏ sang hộp thứ hai. Sau đó từ hộp thứ hai lấy ngẫu nhiên ra một viên bi. Biết rằng viên bi lấy ra từ hộp hai là viên bi màu đỏ. Tính xác suất viên bi bỏ từ hộp thứ nhất sang hộp thứ hai là màu xanh ({\it kết quả làm tròn đến hàng phần trăm}).
	\shortans{$0{,}35$}
	\loigiai{
	Gọi $A$ là biến cố: \lq\lq  Bi bỏ từ hộp thứ nhất sang hộp thứ hai là bi màu xanh \rq\rq.\\	
	Suy ra $\overline{A}$ là biến cố: \lq\lq  Bi bỏ từ hộp thứ nhất sang hộp thứ hai là bi màu đỏ \rq\rq.\\
	Gọi $B$ là biến cố: \lq\lq  Bi lấy từ hộp thứ hai là bi màu đỏ \rq\rq.\\	
	Theo đề bài ta có\\
	$\mathrm{P}\left(A\right)= \dfrac{4}{10}$;
	$\mathrm{P}\left(\overline{A}\right)= \dfrac{6}{10}$;
	$\mathrm{P}\left(B\mid A\right)=\dfrac{4}{11}$;
	$\mathrm{P}\left(B\mid \overline{A}\right)=\dfrac{5}{11}$.\\
	Áp dụng công thức xác suất toàn phần, ta có\\
	$\mathrm{P}\left(B\right) =\mathrm{P}\left(A\right) \cdot \mathrm{P}\left(B\mid A\right)+
	\mathrm{P}\left(\overline{A}\right) \cdot \mathrm{P}\left(B\mid \overline{A}\right)=
	\dfrac{4}{10}\cdot\dfrac{4}{11}+ \dfrac{6}{10}\cdot\dfrac{5}{11}=\dfrac{23}{55}$.\\
	Xác suất cần tìm là 
	$\mathrm{P}\left(A\mid B\right)=
	\dfrac{\mathrm{P}\left(B\mid A\right)\cdot \mathrm{P}\left(A\right) }{\mathrm{P}\left(B\right)}=
	\dfrac{\dfrac{4}{11}\cdot\dfrac{4}{10}}{\dfrac{23}{55}}=\dfrac{8}{23}\approx 0{,}35$.
	}
\end{ex}
\Closesolutionfile{ans}
\indapan{3}{ans/ans-2-B6-De2-TLN}

%HK1
% \foreach \i in {1,...,5} {\input{data/12/CK1/De_\i.tex}}

%C1
% \begin{name}
	{\tenchude}
	{ĐỀ ÔN TẬP CHƯƠNG I}
	{LỚP TOÁN THẦY PHÁT}
	{\thoigian}
\end{name}
\TN
\Opensolutionfile{ans}[ans/ans\currfilebase-Phan-I]
\begin{ex}%[2-D1B5-SO-13-2425]%[VN-MT-7, Lê Hải Phụng]%[2D1N1-2]
Cho hàm số $y=f(x)$ có bảng biến thiên như sau:
\begin{center}
\begin{tikzpicture}
\tkzTabInit[nocadre=true,lgt=1.2,espcl=2.5,deltacl=0.5]
{$x$/0.7,$f'(x)$/0.7,$f(x)$/2}
{$-\infty$,$-1$,$0$,$1$,$+\infty$}
\tkzTabLine{,-,0,+,0,-,0,+,}
\tkzTabVar{+/$+\infty$,-/$-1$,+/$4$,-/$-1$,+/$+\infty$}
\end{tikzpicture}
\end{center}
Hàm số đã cho đồng biến trên khoảng nào dưới đây?
\choice
{$(-\infty;-1)$}
{$(-1;1)$}
{$(0;1)$}
{\True $(-1;0)$}

\loigiai{
Dựa vào bảng biến thiên, ta thấy hàm số đã cho đồng biến trên $(-1;0)$ và $(1;+\infty)$
}
\end{ex}

\begin{ex}%[2-D1B5-SO-13-2425]%[VN-MT-7, Lê Hải Phụng]%%[2D1N2-2]
 Cho hàm số $f(x)$ có bảng biến thiên như sau:
 \begin{center}
 \begin{tikzpicture}
 \tkzTabInit[nocadre=true,lgt=1,espcl=3,deltacl=0.5]
 {$x$/0.7,$y'$/0.7,$y$/2}
 {$-\infty$,$1$,$3$,$+\infty$}
 \tkzTabLine{,+,0,-,0,+,} 
 \tkzTabVar{-/$-\infty$,+/$3$,-/$-2$,+/$+\infty$}
 \end{tikzpicture}
 \end{center}
 Hàm số $f(x)$ đạt cực đại tại 
 \choice
 {$x=-2$}
 {$x=3$}
 {\True $x=1$}
 {$x=2$}
 
 \loigiai{
 Hàm số $f(x)$ đạt cực đại tại $x=1$.
 }
\end{ex}

\begin{ex}%[2-D1B5-SO-13-2425]%[VN-MT-7, Lê Hải Phụng]%[2D1H5-1]
 Hàm số $y = f(x)$ có bảng biến thiên như sau:
 \begin{center}
 \begin{tikzpicture}
 \tkzTabInit[nocadre=true,lgt=1,espcl=3,deltacl=0.5]
 {$x$/0.7,$y'$/0.7,$y$/2}{$-\infty$,$1$,$+\infty$}
 \tkzTabLine{,+,d,+,}
 \tkzTabVar{-/$2$,+D-/$+\infty$/$-\infty$,+/$2$}
 \end{tikzpicture}
 \end{center}
 Hàm số đồng biến trên
 \choice
 {\True $(1;+\infty)$}
 {$(-\infty;2)$}
 {$\mathbb{R}$}
 {$\mathbb{R} \setminus \{1\}$}
 \loigiai{
 Dựa vào bảng biến thiên và bốn đáp án, hàm số đồng biến chỉ đúng với đáp án là khoảng $(1;+\infty)$.
 }
\end{ex}

\begin{ex}%[2-D1B5-SO-13-2425]%[VN-MT-7, Lê Hải Phụng]%[2D1N3-2]
 Giá trị lớn nhất của hàm số $y=x+\dfrac{4}{x}$ trên $(-4;0)$ là
 \choice
 {\True $-4$}
 {$4$}
 {$-5$}
 {$5$}
 \loigiai{Tập xác định $\mathscr{D}=\mathbb{R}\setminus \{0\}$.\\
 Đạo hàm $y'=1-\dfrac{4}{x^2}=\dfrac{x^2-4}{x^2}$.\\
 Xét $y'=0\Leftrightarrow \dfrac{x^2-4}{x^2}=0\Leftrightarrow x^2-4=0\Leftrightarrow\hoac{&x=2\\&x=-2.}$\\
 Suy ra $x=-2$ vì $x\in (-4;0)$.\\
 Ta có $y(-2)=-4$.
\begin{center}
 \begin{tikzpicture}
 \tkzTabInit[nocadre=true,lgt=1.2,espcl=2.5,deltacl=0.5]
 {$x$/0.7,$y'$/0.7,$y$/2}
 {$-4$,$-2$,$0$}
 \tkzTabLine{,+,0,-,d,}
 \tkzTabVar{-/$-5$,+/$-4$,-D/$-\infty$}
 \end{tikzpicture}
 \end{center}
 Vậy giá trị lớn nhất của hàm số $y=x+\dfrac{4}{x}$ trên $(-4;0)$ là $-4$.}
\end{ex}

\begin{ex}%[2-D1B5-SO-13-2425]%[VN-MT-7, Lê Hải Phụng]%[2D1H3-1]
\immini{ Cho hàm số $f(x)$ có đồ thị trên $[-3;3]$ như hình vẽ.
 Giá trị lớn nhất $M$ và giá trị nhỏ nhất $m$ của hàm số $f(x)$ trên $[-3;3]$ lần lượt là
 \choice[2]
 {$M=3$; $m=-1$}
 {\True $M=4$; $m=-2$}
 {$M=3$; $m=-3$}
 {$M=-1$; $m=1$}
 }{ \begin{tikzpicture}[scale=0.7, font=\normalsize, line join=round, line cap=round,>=stealth]
 \def\xmin{-4} \def\xmax{4}
 \def\ymin{-3} \def\ymax{5}
 \draw[->] (\xmin,0)--(0,0)node[above right]{$O$}--(\xmax,0)node[below]{$x$};
 \draw[->] (0,\ymin)--(0,\ymax) node [left]{$y$};
 \clip (\xmin+0.1,\ymin+0.1) rectangle (\xmax-0.1,\ymax-0.1);
 \draw[smooth,samples=100]plot[domain=-3:-1](\x,{-1.25*(\x)^2-2.5*(\x)+1.75});
 \draw[smooth,samples=100]plot[domain=-1:1](\x,{-2*(\x)^3+1});
 \draw[smooth,samples=100]plot[domain=1:3](\x,{-3.5+2.5*(\x)});
 \draw[dashed] (-3,0)--(-3,-2)--(0,-2) (-1,0)--(-1,3)--(0,3) (0,-1)--(1,-1)--(1,0) (3,0)--(3,4)--(0,4);
 \path 
 (-3,0) node[below left]{$-3$}
 (-1,0) node[below]{$-1$}
 (1,0) node[above]{$1$}
 (3,0) node[below]{$3$}
 (0,-2) node[right]{$-2$}
 (0,-1) node[left]{$-1$}
 (0,3) node[right]{$3$}
 (0,4) node[left]{$4$};
 \fill (-3,-2) circle (1pt);
 \fill (-1,3) circle (1pt);
 \fill (1,-1) circle (1pt);
 \fill (3,4) circle (1pt);
 \end{tikzpicture}}
 \loigiai{Từ đồ thị, ta có giá trị lớn nhất $M=4$ và giá trị nhỏ nhất $m=-2$ của hàm số $f(x)$ trên $[-3;3]$.}
\end{ex}

\begin{ex}%[2-D1B5-SO-13-2425]%[VN-MT-7, Lê Hải Phụng]%[2D1H4-1]
 Đồ thị hàm số $y=\dfrac{x+1}{x^2+x-2}$ có bao nhiêu đường tiệm cận đứng?
 \choice
 {$1$}
 {\True $2$}
 {$3$}
 {$4$}
 \loigiai{Ta có $y=\dfrac{x+1}{x^2+x-2}=\dfrac{x+1}{(x-1)(x+2)}$.\\
 Hàm số đã cho có tập xác định là $\mathbb{R}\setminus\{1;-2\}$.\\
 Ta có \begin{itemize}
 \item $\lim\limits_{x\to1^-}f(x)=\lim\limits_{x\to1^-}\dfrac{x+1}{(x-1)(x+2)}=-\infty$.
 \item $\lim\limits_{x\to1^+}f(x)=\lim\limits_{x\to1^-}\dfrac{x+1}{(x-1)(x+2)}=+\infty$.
 \end{itemize} 
Suy ra $x=1$ là tiệm cận đứng của đồ thị hàm số.\\
Ta có \begin{itemize}
 \item $\lim\limits_{x\to-2^-}f(x)=\lim\limits_{x\to-2^-}\dfrac{x+1}{(x-1)(x+2)}=-\infty$.
 \item $\lim\limits_{x\to-2^+}f(x)=\lim\limits_{x\to-2^-}\dfrac{x+1}{(x-1)(x+2)}=+\infty$.
\end{itemize} 
Suy ra $x=-2$ là tiệm cận đứng của đồ thị hàm số.\\
Vậy đồ thị hàm số đã cho có $2$ đường tiệm cận đứng.}
\end{ex}

\begin{ex}%[2-D1B5-SO-13-2425]%[VN-MT-7, Lê Hải Phụng]%[2D1H4-1]
 Cho hàm số $y=f(x)=\dfrac{6x^2+7x-2023}{2x^2+3x+2024}$. Đồ thị hàm số có tiệm cận ngang là
 \choice
 {\True $y=3$}
 {$y=0$}
 {$y=1$}
 {$y=2$}
 \loigiai{Tập xác định của hàm số đã cho là $\mathbb{R}$.\\
 Ta có \begin{itemize}
 \item $\lim\limits_{x\to+\infty}f(x)=\lim\limits_{x\to+\infty}\dfrac{6x^2+7x-2023}{2x^2+3x+2024}=\dfrac{6}{2}=3$.
 \item $\lim\limits_{x\to-\infty}f(x)=\lim\limits_{x\to-\infty}\dfrac{6x^2+7x-2023}{2x^2+3x+2024}=\dfrac{6}{2}=3$.
 \end{itemize} 
 Suy ra $y=3$ là tiệm cận ngang của đồ thị hàm số.}
\end{ex}

\begin{ex}%[2-D1B5-SO-13-2425]%[VN-MT-7, Lê Hải Phụng]%[2D1H4-1]
 Tiệm cận xiên của đồ thị hàm số $y=\dfrac{x^3+x^2-2x-1}{x^2-2}$ là đường thẳng có phương trình
 \choice
 {$y=2x+1$}
 {\True $y=x+1$}
 {$y=-x+1$}
 {$y=x$}
 \loigiai{Tập xác định của hàm số $\mathscr{D}=\mathbb{R}\setminus\left\lbrace \pm\sqrt{2}\right\rbrace $.\\
 Phương trình đường tiệm cận xiên có dạng $y=ax+b$.\\
 Trong đó
 \begin{itemize}
 \item $a=\lim\limits_{x\to +\infty} \dfrac{f(x)}{x}=\lim\limits_{x\to +\infty}\dfrac{x^3+x^2-2x-1}{x^3-2x}=1$.
 \item $b=\lim\limits_{x\to +\infty} \left[f(x)-ax\right]=\lim\limits_{x\to +\infty} \left(\dfrac{x^3+x^2-2x-1}{x^2-2}-x\right)=\lim\limits_{x\to +\infty}\dfrac{x^2+1}{x^2-2}=1$.
 \end{itemize}
 Ta cũng có \begin{itemize}
 \item $a=\lim\limits_{x\to -\infty} \dfrac{f(x)}{x}=\lim\limits_{x\to -\infty}\dfrac{x^3+x^2-2x-1}{x^3-2x}=1$.
 \item $b=\lim\limits_{x\to -\infty} \left[f(x)-ax\right]=\lim\limits_{x\to -\infty} \left(\dfrac{x^3+x^2-2x-1}{x^2-2}-x\right)=\lim\limits_{x\to -\infty}\dfrac{x^2+1}{x^2-2}=1$.
 \end{itemize}
 Và $\lim\limits_{x\to \pm\infty} \left[f(x)-(ax+b)\right]=\lim\limits_{x\to \pm\infty} \left[\dfrac{x^3+x^2-2x-1}{x^2-2}-(x+1)\right]=0$.\\
 Do đó, đồ thị hàm số có tiệm cận xiên là đường thẳng $y=x+1$. }
\end{ex}

\begin{ex}%[2-D1B5-SO-13-2425]%[VN-MT-7, Lê Hải Phụng]%[2D1H5-1]
\immini{Đồ thị của hàm số nào dưới đây có dạng như đường cong trong hình bên?
 \choice[2]
 {\True $y=x^3-2024x$}
 {$y=-x^3+3x$}
 {$y=x^3-3x^2+2024$}
 {$y=-x^3+3x^2-2$}
}{ \begin{tikzpicture}[scale=0.7, line join=round, line cap=round,>=stealth]
 \tikzset{every node/.style={scale=0.7}}
 \draw[->] (-2.5,0)--(2.5,0) node[below left] {$x$};
 \draw[->] (0,-2.5)--(0,2.5) node[below left] {$y$};
 \draw (0,0) node [below left] {$O$};
 \fill (0,0) circle (1pt);
 \begin{scope}
 \clip (-2.5,-2.5) rectangle (2.5,2.5);
 \draw[samples=200,domain=-2:2,smooth,variable=\x] plot (\x,{1*((\x)^3)+0*((\x)^2)+-3*(\x)+0});
 \end{scope}
\end{tikzpicture} }
 \loigiai{Đường cong có dạng của đồ thị hàm số bậc ba với hệ số $a>0$ và đi qua $O(0;0)$. Do đó đồ thị trên của hàm số $y=x^3-2024x$.}
\end{ex}

\begin{ex}%[2-D1B5-SO-13-2425]%[VN-MT-7, Lê Hải Phụng]%[2D1H5-1]
 \immini{Cho hàm số $y=\dfrac{ax+b}{cx+d}$ có đồ thị là đường cong trong hình vẽ bên. Tọa độ giao điểm của đồ thị
 hàm số đã cho và trục tung là
 \choice[2]
 {\True $(0;-2)$}
 {$(2;0)$}
 {$(-2;0)$}
 {$(0;2)$}
 }{ \begin{tikzpicture}[scale=.5, font=\normalsize, line join=round, line cap=round,>=stealth]
 \def\xmin{-7} \def\xmax{6}
 \def\ymin{-5} \def\ymax{6}
 \draw[->] (\xmin,0)--(0,0)node[below left]{$O$}--(\xmax,0)node[below]{$x$};
 \draw[->] (0,\ymin)--(0,\ymax) node [left]{$y$};
 \clip (\xmin+0.1,\ymin+0.1) rectangle (\xmax-0.1,\ymax-0.1);
 \draw[smooth,samples=100]plot[domain=\xmin:\xmax](\x,{(\x-2)/(\x+1)}) (\xmin,1)--(\xmax,1);
 \path 
 (-1,0) node[below left]{$-1$}
 (2,0) node[below]{$2$}
 (0,1) node[above right]{$1$}
 (0,-2) node[right]{$-2$};
 \fill (0,0) circle (1pt);
 \fill (-1,1) circle (1pt);
 \fill (0,1) circle (1pt);
 \fill (-1,0) circle (1pt);
 \fill (2,0) circle (1pt);
 \fill (0,-2) circle (1pt);
 \end{tikzpicture}}
 \loigiai{Từ đồ thị hàm số đã cho, ta có tọa độ giao điểm của đồ thị hàm số đã cho và trục tung là $(0;-2)$.}
\end{ex}

\begin{ex}%[2-D1B5-SO-13-2425]%[VN-MT-7, Lê Hải Phụng]%[2D1H5-1]
 \immini{Đồ thị của hàm số nào dưới đây có dạng như đường cong trong hình bên dưới?
 \choice
 {$y=x+2$}
 {$y=\dfrac{x^2-2x+2}{x+1}$}
 {$y=x^2-2x+2$}
 {\True $\dfrac{x^2+2x+2}{x+1}$}
 }{ \begin{tikzpicture}[scale=.6, font=\normalsize, line join=round, line cap=round,>=stealth]
 \def\xmin{-5} \def\xmax{5}
 \def\ymin{-4} \def\ymax{5}
 \draw[->] (\xmin,0)--(0,0)node[below right]{$O$}--(\xmax,0)node[below]{$x$};
 \draw[->] (0,\ymin)--(0,\ymax) node [left]{$y$};
 \clip (\xmin+0.1,\ymin+0.1) rectangle (\xmax-0.1,\ymax-0.1);
 \draw[smooth,samples=125]plot[domain=\xmin:\xmax](\x,{((\x)^2+2*(\x)+2)/(\x+1)});
 \draw[smooth,samples=125]plot[domain=\xmin:\xmax](\x,{(\x)+1});
 \draw[dashed] (-2,0)--(-2,-2)--(0,-2);
 \path 
 (-1,0) node[below right]{$-1$}
 (2,0) node[below]{$2$}
 (-2,0) node[above]{$-2$}
 (0,1) node[right]{$1$}
 (0,-2) node[right]{$-2$};
 \fill (0,0) circle (1pt);
 \fill (0,1) circle (1pt);
 \fill (0,-2) circle (1pt);
 \fill (2,0) circle (1pt);
 \fill (-2,0) circle (1pt);
 \fill (-2,-2) circle (1pt);
 \end{tikzpicture}}
 \loigiai{Từ đồ thị ta thấy $x=-1$ là tiệm cận đứng của đồ thị hàm số đã cho và đi qua $(-2;-2)$.\\
 Vậy đồ thị trên của hàm số $y=\dfrac{x^2+2x+2}{x+1}$.}
\end{ex}

\begin{ex}%[2-D1B5-SO-13-2425]%[VN-MT-7, Lê Hải Phụng]%[2D1H5-1]
 \immini{Cho hàm số $y=\dfrac{x^2+a}{x+b}$ có đồ thị là đường cong trong hình vẽ bên. Giá trị của $T=a+b$ bằng
 \choice
 {$T=0$}
 {$T=-2$}
 {\True $T=-1$}
 {$T=2$}
 }{ \begin{tikzpicture}[scale=.5, font=\normalsize, line join=round, line cap=round,>=stealth]
 \def\xmin{-4} \def\xmax{7}
 \def\ymin{-4} \def\ymax{7}
 \draw[->] (\xmin,0)--(0,0)node[above right]{$O$}--(\xmax,0)node[below]{$x$};
 \draw[->] (0,\ymin)--(0,\ymax) node [left]{$y$};
 \clip (\xmin+0.1,\ymin+0.1) rectangle (\xmax-0.1,\ymax-0.1);
 \draw[smooth,samples=125]plot[domain=\xmin:\xmax](\x,{((\x)^2)/(\x-1)});
 \draw[smooth,samples=100]plot[domain=\xmin:\xmax](\x,{(\x)+1});
 \draw[dashed] (0,2)--(1,2);
 \path 
 (0,1) node[left]{$1$}
 (0,2) node[left]{$2$}
 (1,\ymin) node[above right]{$x=1$};
 \fill (0,0) circle (1pt);
 \fill (0,1) circle (1pt);
 \fill (0,2) circle (1pt);
 \end{tikzpicture}}
 \loigiai{Từ đồ thị ta thấy $x=1$ là tiệm cận đứng của đồ thị hàm số nên $b=-1$. Suy ra $y=\dfrac{x^2+a}{x-1}$.\\
 Hàm số đi qua $(0;0)$ nên $\dfrac{0^2+a}{0-1}=0\Leftrightarrow a=0$.\\
 Vậy $T=a+b=0+(-1)=-1$.}
\end{ex}
\Closesolutionfile{ans}

\TNTF
\Opensolutionfile{ans}[ans/ans\currfilebase-Phan-II]
\begin{ex}%[2-D1B5-SO-13-2425]%[VN-MT-7, Lê Hải Phụng]%[2D1H2-1]
Cho hàm số $y=f(x)=x^4-2x^2-5$. Các khẳng định sau là đúng hay sai?
\choiceTF
{\True Hàm số có $3$ điểm cực trị}
{Hàm số đồng biến trên $(0;+\infty)$}
{Điểm $M(0;1)$ là điểm cực đại của đồ thị hàm số $y=f(x)$}
{Hàm số $y=f(x)$ và $y=f(2x)$ có cùng điểm cực đại}
\loigiai{
 Tập xác định $\mathscr{D}=\mathbb{R}$.\\
 Đạo hàm $y'=4x^3-4x$.\\
 Xét $y'=0\Leftrightarrow 4x^3-4x=0\Leftrightarrow\hoac{&x=0\\&x=1\\&x=-1.}$\\
 Bảng biến thiên
 \begin{center}
 \begin{tikzpicture}
 \tkzTabInit[nocadre=true,lgt=1.2,espcl=2.5,deltacl=0.5]
 {$x$/0.7,$f'(x)$/0.7,$f(x)$/2}
 {$-\infty$,$-1$,$0$,$1$,$+\infty$}
 \tkzTabLine{,-,0,+,0,-,0,+,}
 \tkzTabVar{+/$+\infty$,-/$-6$,+/$-5$,-/$-6$,+/$+\infty$}
 \end{tikzpicture}
 \end{center}
\begin{itemchoice}
\itemch \textbf{Đúng}.\\
Hàm số có $3$ điểm cực trị là $x=-1$, $x=0$, $x=1$.
\itemch \textbf{Sai}.\\
Hàm số đồng biến trên $(-1;0)$ và $(1;+\infty)$.
\itemch \textbf{Sai}.\\
Điểm $M(0;-5)$ là điểm cực đại của đồ thị hàm số $y=f(x)$
\itemch \textbf{Đúng}.\\
Xét $y=f(2x)$, ta có $y'=2f'(2x)$.\\
Xét $y'=0\Leftrightarrow 2f'(2x)=0\Leftrightarrow f'(2x)=0\Leftrightarrow \hoac{&2x=-1\\&2x=0\\&2x=1}\Leftrightarrow\hoac{&x=-\dfrac{1}{2}\\&x=0\\&x=\dfrac{1}{2}.}$\\
Bảng biến thiên
\begin{center}
 \begin{tikzpicture}
 \tkzTabInit[nocadre=true,lgt=1.2,espcl=2.5,deltacl=0.5]
 {$x$/0.7,$y'$/0.7,$y$/2}
 {$-\infty$,$-\tfrac{1}{2}$,$0$,$\tfrac{1}{2}$,$+\infty$}
 \tkzTabLine{,-,0,+,0,-,0,+,}
 \tkzTabVar{+/$+\infty$,-/$f(-1)$,+/$f(0)$,-/$f(1)$,+/$+\infty$}
 \end{tikzpicture}
\end{center}
Suy ra $x=0$ là điểm cực đại của hàm số $y=f(2x)$.\\
Vậy hàm số $y=f(x)$ và $y=f(2x)$ có cùng điểm cực đại.
\end{itemchoice}
}
\end{ex}

\begin{ex}%[2-D1B5-SO-13-2425]%[VN-MT-7, Lê Hải Phụng]%[2D1H3-1]
 Cho hàm số $y=f(x)=x^3-3x+2$. Các khẳng định sau là đúng hay sai?
 \choiceTF
 {\True $\min\limits_{[0;1]} y=0$}
 {\True $\min\limits_{[0;2]} y=y(0)$}
 {$\min\limits_{[-1;0]} y+\max\limits_{[0;1]} y=4$}
 {$\min\limits_{\left[-\frac{3}{2};0 \right] } \dfrac{1}{y}=\dfrac{8}{25}$}
 \loigiai{ Tập xác định $\mathscr{D}=\mathbb{R}$.\\
 Đạo hàm $y'=3x^2-3$.\\
 Xét $y'=0\Leftrightarrow 3x^2-3=0\Leftrightarrow\hoac{&x=1\\&x=-1.}$\\
 Bảng biến thiên
 \begin{center}
 \begin{tikzpicture}
 \tkzTabInit[nocadre=true,lgt=1.2,espcl=2.5,deltacl=0.5]
 {$x$/0.7,$f'(x)$/0.7,$f(x)$/2}
 {$-\infty$,$-1$,$1$,$+\infty$}
 \tkzTabLine{,+,0,-,0,+,}
 \tkzTabVar{-/$+\infty$,+/$4$,-/$0$,+/$-\infty$}
 \end{tikzpicture}
 \end{center}
 \begin{itemchoice}
 \itemch \textbf{Đúng}.\\
 Ta xét trên $[0;1]$, ta có $y(0)=2$ và $y(1)=0$. Vậy $\min\limits_{[0;1]} y=0$.
 \itemch \textbf{Đúng}.\\
 Ta xét trên $[0;2]$, ta có $y(0)=2$, $y(1)=0$ và $y(2)=4$. Vậy $\min\limits_{[0;2]} y=0=y(0)$.
 \itemch \textbf{Sai}.\\
 Ta xét trên $[0;1]$, ta có $y(0)=2$ và $y(1)=0$. Vậy $\min\limits_{[0;1]} y=0$ và $\max\limits_{[0;1]} y=2$, khi đó tổng bằng $0+2=2$.
 \itemch \textbf{Sai}.\\
 Ta có $g(x)=\dfrac{1}{y}=\dfrac{1}{x^3-3x+2}$.\\
 Tập xác định $\mathscr{D}=\mathbb{R}\setminus\{-2;1\}$.\\
 $g'(x)=\left(\dfrac{1}{y} \right)'=\dfrac{-3x^2-3}{x^3-3x+2}$.\\
 Bảng biến thiên
 \begin{center}
 \begin{tikzpicture}
 \tkzTabInit[nocadre=true,lgt=1.2,espcl=2.5,deltacl=0.5]
 {$x$/0.7,$g'(x)$/0.7,$g(x)$/2}
 {$-\tfrac{3}{2}$,$-1$,$0$}
 \tkzTabLine{,-,0,+,}
 \tkzTabVar{+/$\dfrac{8}{25}$,-/$\dfrac{1}{4}$,+/$\dfrac{1}{2}$}
 \end{tikzpicture}
 \end{center}
 Vậy $\min\limits_{\left[-\frac{3}{2};0 \right] } \dfrac{1}{y}=\dfrac{1}{4}$ khi $x=-1$.
 \end{itemchoice}
 }
\end{ex}

\begin{ex}%[2-D1B5-SO-13-2425]%[VN-MT-7, Lê Hải Phụng]%[2D1H4-1]
 Hàm số $y = f(x)$ có bảng biến thiên như sau
 \begin{center}
 \begin{tikzpicture}
 \tkzTabInit[nocadre=true,lgt=1.2,espcl=2.5,deltacl=0.5]
 {$x$/0.7,$y'$/0.7,$y$/2}{$-\infty$,$2$,$+\infty$}
 \tkzTabLine{,+,d,+,}
 \tkzTabVar{-/$1$,+D-/$+\infty$/$-\infty$,+/$1$}
 \end{tikzpicture}
 \end{center}
 \choiceTF
 {\True Tập xác định của hàm số là $\mathscr{D}=\mathbb{R}\setminus\{2\}$}
 {Hàm số đồng biến trên $\mathbb{R}$}
 {\True Tiệm cận ngang của hàm số là $y = 1$}
 {Hàm số đạt cực đại tại $x = 2$}
 \loigiai{
 \begin{itemchoice}
 \itemch \textbf{Đúng}.\\
 Tập xác định của hàm số là $\mathscr{D}=\mathbb{R}\setminus\{2\}$.
 \itemch \textbf{Sai}.\\
 Hàm số đống biến trên $(-\infty; 2)$ và $(2; +\infty)$.
 \itemch \textbf{Đúng}.\\
 Vì $\lim\limits_{x \to -\infty} f(x) = 1$ nên tiệm cận ngang của hàm số là $y = 1$.
 \itemch \textbf{Sai}.\\
 Hàm số không có cực trị.
 \end{itemchoice}
 }
\end{ex}

\begin{ex}%[2-D1B5-SO-13-2425]%[VN-MT-7, Lê Hải Phụng]%[2D1H5-7]
 \immini{
 Cho hàm số $y = f(x)$ có đồ thị như sau. Các mệnh đề sau đúng hay sai?
 \choiceTF
 {Hàm số đồng biến trên $(-\infty; -1)$}
 {Hàm số đạt cực đại tại $x = -2$}
 {\True Giá trị nhỏ nhất của hàm số $y = f(x)$ trên $(-\infty;-1)$ là $\dfrac{3}{2}$}
 {\True Điểm cực tiểu của hàm số là $x = -2$}}{ \begin{tikzpicture}[scale=0.7, font=\footnotesize, line join=round, line cap=round, >=stealth]
 \def \a{1}
 \def \b{1}
 \def \c{1}
 \def \u{-2}
 \def \v{-2}
 \def \f{((\a)*(\x)^2+(\b)*(\x)+(\c))/((\u)*(\x)+(\v))} %%Hàm số
 \def \tcx{-(\x)/2} %Tiệm cận xiên
 \def \xo{-{\v/\u}} %Tiệm cận đứng
 \def \yo{0.5} %Tiệm cận ngang
 \def \kx{4.5} %độ rộng của đồ thị theo x
 \draw[->] (\xo-\kx,0)--(\xo+\kx,0) node[below left] {$x$};
 \draw[->] (0,\yo-\kx)--(0,\yo+\kx) node[below left] {$y$};
 \draw (0,0) node [above right] {\scriptsize $O$};
 \draw[dashed] (-1,1pt)--(-1,-1pt) + (-30:6mm) node {\scriptsize $-1$};
 \draw[dashed] (-2,1pt)--(-2,-1pt) + (-90:6mm) node {\scriptsize $-2$};
 \draw[dashed] (1pt,1.5)--(-1pt,1.5) + (0:6mm) node {\scriptsize $\dfrac{3}{2}$};
 \draw[dashed] (1pt,-0.5)--(-1pt,-0.5) + (-45:10mm) node {\scriptsize $-\dfrac{1}{2}$};
 \begin{scope}
 \clip (\xo-\kx,\yo-\kx) rectangle (\xo+\kx,\yo+\kx);
 \draw[smooth,samples=200,domain=\xo-\kx:\xo-0.1,smooth,variable=\x] plot (\x,{\f});
 \draw[smooth,samples=200,domain=\xo+0.1:\xo+\kx,smooth,variable=\x] plot (\x,{\f});
 %Tiệm cận
 \draw[thin] (\xo,\yo-\kx)--(\xo,\yo+\kx);
 \draw[smooth,samples=200,domain=\xo-\kx:\xo+\kx,variable=\x] plot (\x,{\tcx});
 \end{scope}
 %Vẽ đường dóng
 \draw[dashed, thin] (-2,0) -- (-2,1.5) -- (0,1.5);
 \fill (0,0) circle (1pt);
 \fill (-2,0) circle (1pt);
 \fill (-1,0) circle (1pt);
 \fill (0,-0.5) circle (1pt);
 \fill (0,1.5) circle (1pt);
 \fill (-2,1.5) circle (1pt); 
 \end{tikzpicture}}
 \loigiai{
 \begin{itemchoice}
 \itemch \textbf{Sai}.\\
 Hàm số đồng biến trên $(-2;-1)$, $(-1;0)$ và nghịch biến trên $(-\infty;-2)$, $(0;+\infty)$.
 \itemch \textbf{Sai}.\\
 Hàm số đạt cực tiểu tại $x = -2$.
 \itemch \textbf{Đúng}.\\
 Giá trị nhỏ nhất của hàm số $y = f(x)$ trên $(-\infty;-1)$ là $\dfrac{3}{2}$.
 \itemch \textbf{Đúng}.\\
 Điểm cực tiểu của hàm số là $x = -2$.
 \end{itemchoice}
 }
\end{ex}
\Closesolutionfile{ans}

\TNSA
\Opensolutionfile{ans}[ans/ans\currfilebase-Phan-III]
\begin{ex}%[2-D1B5-SO-13-2425]%[VN-MT-7, Lê Hải Phụng]%[2D1N1-1]
 Cho hàm số $y=x^3-3x^2+1$. Tính tổng của tất cả các giá trị cực đại và giá trị cực tiểu của hàm số trên. 
 
 \shortans{-2}
 
 \loigiai{Tập xác định của hàm số $\mathscr{D}=\mathbb{R}$.\\
 Ta có đạo hàm $y'=3x^2-6x$.\\
 Xét $y'=0\Leftrightarrow3x^2-6x=0\Leftrightarrow\hoac{&x=0\\&x=2.}$\\
 Bảng biến thiên:\\
 \begin{center}
 \begin{tikzpicture}
 \tkzTabInit[nocadre=true,lgt=1,espcl=3,deltacl=0.5]
 {$x$/0.7,$y'$/0.7,$y$/2}
 {$-\infty$,$0$,$2$,$+\infty$}
 \tkzTabLine{,+,0,-,0,+,} 
 \tkzTabVar{-/$-\infty$,+/$1$,-/$-3$,+/$+\infty$}
 \end{tikzpicture}
 \end{center}
 Suy ra giá trị cực đại và cực tiểu lần lượt là $y=1$ và $y=-3$. Khi đó $1+(-3)=-2$.}
\end{ex}

\begin{ex}%[2-D1B5-SO-13-2425]%[VN-MT-7, Lê Hải Phụng]%[2D1H4-1]
 Tiệm cận xiên của đồ thị hàm số $y=f(x)=\dfrac{3x-x^2}{2x-1}$ là đường thẳng $y=ax+b$. Tính giá trị của biểu thức $P=a^2-b$.
 
 \shortans{-1}
 
 \loigiai{Tập xác định của hàm số $\mathscr{D}=\mathbb{R}\setminus\left\lbrace \dfrac{1}{2}\right\rbrace $.\\
 Phương trình đường tiệm cận xiên có dạng $y=ax+b$.\\
 Trong đó
 \begin{itemize}
 \item $a=\lim\limits_{x\to +\infty} \dfrac{f(x)}{x}=\lim\limits_{x\to +\infty}\dfrac{3x-x^2}{2x^2-x}=-\dfrac{1}{2}$.
 \item $b=\lim\limits_{x\to +\infty} \left[f(x)-ax\right]=\lim\limits_{x\to +\infty} \left(\dfrac{3x-x^2}{2x-1}+\dfrac{1}{2}x\right)=\dfrac{5}{4}$.
 \end{itemize}
 Ta cũng có \begin{itemize}
 \item $a=\lim\limits_{x\to -\infty} \dfrac{f(x)}{x}=\lim\limits_{x\to -\infty}\dfrac{3x-x^2}{2x^2-x}=-\dfrac{1}{2}$.
 \item $b=\lim\limits_{x\to -\infty} \left[f(x)-ax\right]=\lim\limits_{x\to -\infty} \left(\dfrac{3x-x^2}{2x-1}+\dfrac{1}{2}x\right)=\dfrac{5}{4}$.
 \end{itemize}
 Và $\lim\limits_{x\to \pm\infty} \left[f(x)-(ax+b)\right]=\lim\limits_{x\to \pm\infty} \left[\dfrac{3x-x^2}{2x-1}-\left(-\dfrac{1}{2}x+\dfrac{5}{4} \right) \right]=0$.\\
 Do đó, đồ thị hàm số có tiệm cận xiên là đường thẳng $y=-\dfrac{1}{2}x+\dfrac{5}{4}$.\\
 Do đó $a=-\dfrac{1}{2}$ và $b=\dfrac{5}{4}$. Vậy $P=-1$.}
\end{ex}

\begin{ex}%[2-D1B5-SO-13-2425]%[VN-MT-7, Lê Hải Phụng]%[2D1N5-4]
 Hàm số $y=f(x)=-x^3+2x^2-x+1$ có đồ thị $(C)$ và hàm số $y=g(x)=1$ có đồ thị là $(d)$. Số giao điểm của $(C)$ và $(d)$ là
 
 \shortans{2}
 \loigiai{Ta xét phương trình hoành độ giao điểm:
 \[-x^3+2x^2-x+1=1\Leftrightarrow-x^3+2x^2-x=0\Leftrightarrow\hoac{&x=0\\&x=1.}\]
 Suy ra giao điểm của $(C)$ và $(d)$ là $(0;1)$ và $(1;1)$.\\
 Vậy số giao điểm của $(C)$ và $(d)$ là $2$.}
\end{ex}

\begin{ex}%[2-D1B5-SO-13-2425]%[VN-MT-7, Lê Hải Phụng]%[2D1V2-7]
 Giả sử doanh số (tính bằng sản phẩm) của một sản phẩm mới (trong một năm nhất định) tuân theo quy luật logistic được mô hình hóa bằng hàm số \[f(t)=\dfrac{5000}{1+5\mathrm{e}^{-t}},\, t\ge0,\]
 trong đó thời gian $t$ được tính bằng năm, kể từ khi phát hành sản phẩm mới. Khi đó đạo hàm $f'(t)$ biểu thị tốc độ bán hàng. Hỏi sau khi phát hành bao nhiêu năm thì tốc độ bán hàng là cực đại? (làm tròn đến chữ số thập phân thứ nhất)
 
 \shortans{1{,}6}
 
 \loigiai{
 Gọi $g(t)$ là hàm tốc độ bán hàng.\\
 Khi đó $g(t)=f'(t)=\dfrac{25\,000\mathrm{e}^{-t}}{(1+5\mathrm{e}^{-t})^2}$, $t\ge0$.\\
 Ta có $g'(t)=\dfrac{25\,000\mathrm{e}^{-t}(1+5\mathrm{e}^{-t})(5\mathrm{e}^{-t}-1)}{(1+5\mathrm{e}^{-t})^4}$; $g'(t)=0\Leftrightarrow t=-\ln{\dfrac{1}{5}}$.\\
 Bảng biến thiên hàm số
 \begin{center}
 \begin{tikzpicture}
 \tkzTabInit[nocadre=true,lgt=1.3,espcl=2.5,deltacl=0.6]
 {$t$ /0.6, $g'(t)$ /0.6, $g(t)$ /2.5}
 {$0$, $-\ln{\tfrac{1}{5}}$, $+\infty$}
 \tkzTabLine{,+,0,-,}
 \tkzTabVar{-/ $694{,}4$,+/$1250$,-/$0$}
 \end{tikzpicture}
 \end{center}
 Hàm số đạt cực đại tại $t=-\ln{\dfrac{1}{5}}\approx1{,}6$.\\
 Vậy sau khi phát hành $1{,}6$ năm thì tốc độ bán hàng là cực đại.
 }
\end{ex}

\begin{ex}%[2-D1B5-SO-13-2425]%[VN-MT-7, Lê Hải Phụng]%[2D1C5-8]
 Một tàu đổ bộ tiếp cận Mặt Trăng theo cách tiếp cận thẳng đứng và đốt cháy các tên lửa hãm ở độ cao $677{,}6$ km so với bề mặt của Mặt Trăng được tính (gần đúng) bởi hàm
 \[h(t) = 0{,}01t^3 - 1{,}16t^2 + 34{,}52t - 46{,}4\]
 Trong khoảng thời gian $t$ ở $50$ giây đầu $(0 \le t \le 50)$. Khoảng cách con tàu lớn nhất so với bề mặt của Mặt Trăng là bao nhiêu?
 
 \shortans{260}
 
 \loigiai{
 Hàm số $h(t) = 0{,}01t^3 - 1{,}16t^2 + 34{,}52t - 46{,}4$.\\
 + Tập xác định $\mathscr{D} = \mathbb{R}$.\\
 + Đạo hàm $h'(t) = 0{,}03t^2 - 2{,}32t + 34{,}52 = 0 \Leftrightarrow \hoac{&x = \dfrac{-10\sqrt{31}+116}{3} \quad\in (0;50)\\&x = \dfrac{10\sqrt{31}+116}{3} \quad \not\in (0;50).}$\\
 + $\heva{&h(0) = 34{,}52\\&h(50) = 29{,}6\\&h\left(\dfrac{-10\sqrt{31}+116}{3}\right) = 260} \Rightarrow \max\limits_{0\le t \le 50} = 260$.\\
 Vậy trong khoảng thời gian $t$ ở $50$ giây đầu $(0 \le t \le 50)$. Khoảng cách con tàu lớn nhất so với bề mặt của Mặt Trăng là $260$ km.
 }
\end{ex}

\begin{ex}%[2-D1B5-SO-13-2425]%[VN-MT-7, Lê Hải Phụng]%[2D1V3-6]
 \immini{
 Sự phân huỷ của rác thải hữu cơ có trong nước sẽ làm tiêu hao oxygen hoà tan trong nước. Nồng độ oxygen (mg/l) trong một hồ nước sau $t$ giờ $(t \geq 0)$ khi một lượng rác thải hữu cơ bị xả vào hồ được xấp xỉ bởi hàm số (có đồ thị như đường cong ở hình bên)
 \[
 y(t)=5-\dfrac{15t}{9t^2+1}.
 \]
 }{
 \begin{tikzpicture}[>=stealth,x=1cm,y=0.3cm,scale=2,font=\footnotesize]
 \draw[->] (-0.5,0) -- (4,0) node[below] {$t$};
 \draw[->] (0,-1) -- (0,6) node[left] {$y$};
 \filldraw (0,0) circle (1pt)node[below left]{$O$};
 \draw[domain=0:4,samples=200,red] plot (\x,{5-(15*(\x))/(9*(\x)^2+1)});
 \draw[dashed] (0,5) node [left] {$5$}--(4,5);
 \foreach \x/\g in {1/-90,2/-90,3/-90}
 \draw[thin] (\x,2pt)--(\x,-2pt) + (\g:3mm) node {$\x$};
 \end{tikzpicture}
 }
%  \noindent
%  (Theo: https://www.researchgate.net/publication/264903978$\_$Microrespirometric$\_$ characterization$\_$\\of$\_$activated$\_$sludge$\_$inhibition$\_$by$\_$copper$\_$and$\_$zinc)\\
 Trong đó, đạo hàm $y'(t)$ biểu thị tốc độ thay đổi nồng độ oxigen trong nước. Tốc độ thay đổi nồng độ oxigen lớn nhất khi $t=\dfrac{\sqrt{a}}{b}$ giờ. Tính giá trị của $a-b$ biết $a$ và $b$ là các số nguyên tố.
 
 \shortans{0}
 
 \loigiai{
 Ta có $y'(t)=\dfrac{135t^2-15}{(9t^2+1)^2}$.\\
 Suy ra $y''(t)=\dfrac{-2430t^3+810t}{(9t^2+1)^3}$.\\
 Cho $y''(t)=0\Leftrightarrow -2430t^3+810t=0\Leftrightarrow \hoac{&t=\pm\dfrac{\sqrt{3}}{3}\\&t=0.}$
 \begin{center}
 \begin{tikzpicture}[>=stealth]
 \tkzTabInit[nocadre=true,lgt=1.5,espcl=3,deltacl=0.6]{$t$/1 ,$y''(t)$/.6,$y'(t)$/2}
 {$0$ , $\tfrac{\sqrt{3}}{3}$ , $+\infty$}
 \tkzTabLine{ $0$, + , $0$ , - , }
 \tkzTabVar{-/$-15$ , +/$\dfrac{15}{8}$ , -/$0$}
 \end{tikzpicture}
 \end{center}
 Từ bảng biến thiên ta có $\max\limits_{t\in[0;+\infty)}y'(t)=y'\left(\dfrac{\sqrt{3}}{3}\right)=\dfrac{15}{8}$. \\
 Vậy tốc độ thay đổi nồng độ oxigen lớn nhất khi $t=\dfrac{\sqrt{3}}{3}$ giờ.\\
 Vậy $a=b=3$. Khi đó $a-b=0$.
 }
\end{ex}
\Closesolutionfile{ans}
% \begin{indapan}
% 	{ans/ans\currfilebase}
% \end{indapan}


% \begin{name}
 {Biên soạn: Đỗ Minh Phúc \\ Phản biện: Đoàn Thị Lý}
 {Đề ôn tập chương I}
\end{name}

\caulc
\Opensolutionfile{ans}[ans/ans\currfilebase-Phan-I]
\begin{ex}%[2-D1B5-SO-14-2425]%[VN-MT-7, Đỗ Minh Phúc]%[2D1H1-1]
 Cho hàm số $y=f(x)$ có đạo hàm $f'(x)=-x^2-4$, $\forall x \in \mathbb{R}$. Mệnh đề nào dưới đây đúng?
\choice
{Hàm số đồng biến trên khoảng $(2;+\infty)$}
{Hàm số đồng biến trên khoảng $(-2;2)$}
{\True Hàm số nghịch biến trên khoảng $(-\infty;+\infty)$}
{Hàm số đồng biến trên khoảng $(-\infty;-2)$}
\loigiai{
Do hàm số $y=f(x)$ có đạo hàm $f'(x)=-x^2-4<0$, $\forall x \in \mathbb{R}$ nên hàm số nghịch biến trên khoảng $(-\infty;+\infty)$.
}
\end{ex}

\begin{ex}%[2-D1B5-SO-14-2425]%[VN-MT-7, Đỗ Minh Phúc]%[2D1N2-2]
 Cho hàm số $y=f(x)$ có bảng biến thiên như sau.
 \begin{center}
 \begin{tikzpicture}
 \tkzTabInit[nocadre,lgt=1.2,espcl=2.5,deltacl=0.6]
 {$x$/0.6,$f'(x)$/0.6,$f(x)$/2}
 {$-\infty$,$-1$,$2$,$+\infty$}
 \tkzTabLine{,+,0,-,0,+,}
 \tkzTabVar{-/$-\infty$,+/$1$,-/$-2$,+/$+\infty$}
 \end{tikzpicture}
 \end{center}
 Giá trị cực tiểu của hàm số đã cho bằng
 \choice
 {$-1$}
 {$2$}
 {\True $-2$}
 {$1$}
\loigiai{
 Dựa vào bảng biến thiên, ta có giá trị cực tiểu của hàm số đã cho bằng $-2$.}
\end{ex}

\begin{ex}%[2-D1B5-SO-14-2425]%[VN-MT-7, Đỗ Minh Phúc]%[2D1N3-2]
 Cho hàm số $y=f(x)$ liên tục trên $\mathbb{R}$ và có bảng biến thiên như sau.
 \begin{center}
 \begin{tikzpicture}
 \tkzTabInit[nocadre,lgt=1.2,espcl=2.5,deltacl=0.6]
 {$x$/0.6,$y'$/0.6,$y$/2}
 {$-\infty$,$0$,$2$,$+\infty$}
 \tkzTabLine{,+,0,-,0,+,}
 \tkzTabVar{-/$2$,+/$4$,-/$-5$,+/$2$}
 \end{tikzpicture}
\end{center}
 Tổng giá trị lớn nhất và giá trị nhỏ nhất của hàm số $y=f(x)$ trên $\mathbb{R}$ bằng
 \choice
 {$6$}
 {$9$}
 {$-3$}
 {\True $-1$}
 \loigiai{
 Trên $\mathbb{R}$, ta có giá trị lớn nhất của hàm số $y=f(x)$ bằng $4$ tại $x=0$ và giá trị nhỏ nhất bằng $-5$ tại $x=2$.\\
 Khi đó tổng giá trị lớn nhất và giá trị nhỏ nhất của hàm số $y=f(x)$ trên $\mathbb{R}$ bằng $-1$.
}
\end{ex}

\begin{ex}%[2-D1B5-SO-14-2425]%[VN-MT-7, Đỗ Minh Phúc]%[2D1N3-1]
 \immini[thm]{Cho hàm số $y=f(x)$ có đồ thị như hình vẽ. Giá trị lớn nhất của hàm số trên đoạn $[0;3]$ bằng
 \choice
 {\True $4$}
 {$2$}
 {$3$}
 {$0$}}{\begin{tikzpicture}[scale=0.6, font=\footnotesize, line join=round, line cap=round, >=stealth]
 \def\xmin{-2}\def\xmax{4}\def\ymin{-1}\def\ymax{5}
 \draw[->] (\xmin-0.2,0)--(\xmax+0.2,0) node[below] {$x$};
 \draw[->] (0,\ymin-0.2)--(0,\ymax+0.2) node[right] {$y$};
 \draw (0,0) node [below left] {$O$};
 \foreach \x in {1,2}
 \fill (\x,0)circle (1pt) node [below] {$\x$};
 \foreach \x in {3}
 \fill (\x,0)circle (1pt) node [above right] {$\x$};
 \foreach \y in {2,4}
 \fill (0,\y)circle (1pt) node [left] {$\y$};
 \clip (\xmin,\ymin) rectangle (\xmax,\ymax);
 \draw[smooth,samples=200,domain=\xmin:\xmax] plot (\x,{-1*((\x)^3)+3*((\x)^2)+0*(\x)+0});
 \draw[dashed] (1,0)--(1,2)--(0,2);\fill (1,2) circle (1pt);
 \draw[dashed] (0,0)--(0,0)--(0,0);\fill (0,0) circle (1pt);
 \draw[dashed] (2,0)--(2,4)--(0,4);\fill (2,4) circle (1pt);
\end{tikzpicture}}
 \loigiai{
 Từ đồ thị hàm số $f(x)$ ta có $\max_{[0;3]}f(x)=4$ tại $x=2$.
 }
\end{ex}

\begin{ex}%[2-D1B5-SO-14-2425]%[VN-MT-7, Đỗ Minh Phúc]%[2D1H4-1]
 Đường tiệm cận ngang của đồ thi hàm số $y=\dfrac{2024x+2025}{x-5}$ là
 \choice
 {$y=2025$}
 {\True $y=2024$}
 {$y=1$}
 {$y=-5$}
 \loigiai{
 Ta có $\lim\limits_{x \to +\infty} \dfrac{2024x+2025}{x-5}=2024$ và $\lim\limits_{x \to-\infty} \dfrac{2024x+2025}{x-5}=2024$ nên đồ thị hàm số có đường tiệm cận ngang là $y=2024$. 
 }
\end{ex}

\begin{ex}%[2-D1B5-SO-14-2425]%[VN-MT-7, Đỗ Minh Phúc]%[2D1N4-1]
 Đường tiệm cận đứng của đồ thị hàm số $y=\dfrac{15x-6}{10x+5}$ là
 \choice
 {$x=\dfrac{3}{2}$}
 {$x=-\dfrac{6}{5}$}
 {\True $x=-\dfrac{1}{2}$}
 {$x=\dfrac{2}{5}$}
 \loigiai{
 Điều kiện xác định $x \neq-\dfrac{1}{2}$.\\
 Ta có $\lim\limits_{x \to\left(-\tfrac{1}{2}\right)^+} \dfrac{15x-6}{10x+5}=-\infty$ và $\lim\limits_{x \to\left(-\tfrac{1}{2}\right)^-} \dfrac{15x-6}{10x+5}=+\infty$ nên đồ thị hàm số có đường tiệm cận đứng là $x=-\dfrac{1}{2}$. 
 }
\end{ex}

\begin{ex}%[2-D1B5-SO-14-2425]%[VN-MT-7, Đỗ Minh Phúc]%[2D1H4-1]
 Tiệm cận xiên của đồ thị hàm số $y=\dfrac{-x^2-3x+4}{x}$ là đường thẳng có phương trình nào sau đây?
 \choice
 {\True $y=-x-1$}
 {$y=x-1$}
 {$y=-x+1$}
 {$y=x+1$}
 \loigiai{
 Ta có $a=\lim\limits_{x \to+\infty}\left(\dfrac{-x^2-3x+4}{x+2}:x\right)=\lim\limits_{x \to+\infty} \dfrac{-x^2-3x+4}{x^2+2x}=-1$.\\
 Lại có $b=\lim\limits_{x \to+\infty}\left[\dfrac{-x^2-3x+4}{x+2}-(-1)x\right]=\lim\limits_{x \to+\infty} \dfrac{-x+4}{x+2}=-1$.\\
 (Tương tự, $\lim\limits_{x \to-\infty}\left(\dfrac{-x^2-3x+4}{x+2}:x\right)=-1$, $\lim\limits_{x \to-\infty}\left[\dfrac{-x^2-3x+4}{x+2}-(-1)x\right]=-1$).\\
 Tiệm cận xiên của đồ thị hàm số $y=\dfrac{-x^2-3x+4}{x+2}$ là đường thẳng có phương trình $y=-x-1$.
 }
\end{ex}

\begin{ex}%[2-D1B5-SO-14-2425]%[VN-MT-7, Đỗ Minh Phúc]%[2D1H5-1]
 \immini[thm]{Đường cong ở hình sau là đồ thi của hàm số nào?
 \choice
 {\True $y=-x^3+3x^2-4$}
 {$y=x^3-4$}
 {$y=x^2-4$}
 {$y=-x^2-4$}}{\begin{tikzpicture}[scale=0.6, font=\footnotesize, line join=round, line cap=round, >=stealth]
 \def\xmin{-2}\def\xmax{4}\def\ymin{-4.5}\def\ymax{1}
 \draw[->] (\xmin-0.2,0)--(\xmax+0.2,0) node[below] {$x$};
 \draw[->] (0,\ymin-0.2)--(0,\ymax+0.2) node[right] {$y$};
 \draw (0,0) node [below left] {$O$};
 \foreach \x in {-1}
 \fill (\x,0)circle (1pt) node [below left] {$\x$};
 \foreach \x in {2}
 \fill (\x,0)circle (1pt) node [below] {$\x$};
 \foreach \y in {-4}
 \fill (0,\y)circle (1pt) node [below left] {$\y$};
 \clip (\xmin,\ymin) rectangle (\xmax,\ymax);
 \draw[smooth,samples=200,domain=\xmin:\xmax] plot (\x,{-1*((\x)^3)+3*((\x)^2)+0*(\x)+-4});
 \fill (0,0)circle (1pt);
\end{tikzpicture}}
 \loigiai{
 Xét dáng hình của đồ thị, ta loại được hàm số $y=x^2-4$ và $y=-x^2-4$.\\
 Do $\lim\limits_{x \to+\infty} y=-\infty$ nên ta loại hàm số $y=x^3-4$ và nhận hàm số $y=-x^3+3x^2-4$. 
 }
\end{ex}

\begin{ex}%[2-D1B5-SO-14-2425]%[VN-MT-7, Đỗ Minh Phúc]%[2D1H5-1]
 Hàm số nào sau đây có bảng biến thiên như hình bên dưới?
 \begin{center}
 \begin{tikzpicture}
 \tkzTabInit[nocadre,lgt=1.2,espcl=2.5,deltacl=0.6]
 {$x$/0.6,$y'$/0.6,$y$/2}
 {$-\infty$,$2$,$+\infty$}
 \tkzTabLine{,-,d,-,}
 \tkzTabVar{+/$2$,-D+/$-\infty$/$+\infty$,-/$2$}
 \end{tikzpicture}
 \end{center}
 \choice
 {\True $y=\dfrac{2x+1}{x-2}$}
 {$y=\dfrac{2x-5}{x-2}$}
 {$y=\dfrac{2x+1}{x+2}$}
 {$y=\dfrac{2x-1}{x+2}$}
 \loigiai{
 Từ bảng biến thiên, ta nhận thấy đồ thị hàm số có tiệm cận đứng là $x=2$ là nên loại hàm số $y=\dfrac{2x+1}{x+2}$ và $y=\dfrac{2x-1}{x+2}$.\\
 Ta nhận thấy hàm số nghịch biến trên từng khoảng xác định nên loại hàm số $y=\dfrac{2x-5}{x-2}$ và nhận hàm số $y=\dfrac{2x+1}{x-2}$.
 }
\end{ex}

\begin{ex}%[2-D1B5-SO-14-2425]%[VN-MT-7, Đỗ Minh Phúc]%[2D1H5-1]
 \immini[thm]{Đường cong ở hình bên là đồ thị của hàm số nào sau đây?
 \choice
 {$y=-x^3+x^2-2x+1$}
 {$y=\dfrac{x^2-x+3}{x-1}$}
 {\True $y=\dfrac{x^2-3x+6}{x-1}$}
 {$y=\dfrac{2x+3}{x-1}$}}{\begin{tikzpicture}[scale=0.5, font=\footnotesize, line join=round, line cap=round, >=stealth]
 \def\xmin{-3}\def\xmax{5}\def\ymin{-8}\def\ymax{6}
 \draw[->] (\xmin-0.2,0)--(\xmax+0.2,0) node[below] {$x$};
 \draw[->] (0,\ymin-0.2)--(0,\ymax+0.2) node[right] {$y$};
 \draw (0,0) node [below left] {$O$};
 \foreach \x in {-2,2,3,4}
 \fill (\x,0)circle (1pt) node [below] {$\x$};
 \foreach \y in {-6,-4,-2,2,4,6}
 \fill (0,\y)circle (1pt) node [left] {$\y$};
 \clip (\xmin,\ymin) rectangle (\xmax,\ymax);
 \draw (1,\ymin)--(1,\ymax);
 \draw[domain=\xmin:\xmax] plot (\x,{1*(\x)+-2});
 \draw[smooth,samples=200,domain=\xmin:0.9] plot (\x,{(1*((\x)^2)+-3*(\x)+6)/(1*(\x)+-1)});
 \draw[smooth,samples=200,domain=1.1:\xmax] plot (\x,{(1*((\x)^2)+-3*(\x)+6)/(1*(\x)+-1)});
 \draw[dashed] (-1,0)--(-1,-5)--(0,-5);\fill (-1,-5) circle (1pt);
 \draw[dashed] (3,0)--(3,3)--(0,3);\fill (3,3) circle (1pt);
 \fill (0,0)circle (1pt);
\end{tikzpicture}}
 \loigiai{
 \begin{itemize}
 \item Xét hàm số $y=-x^3+x^2-2x+1$. Vì đồ thị hàm số $y=-x^3+x^2-2x+1$ không có đường tiệm cận. Suy ra phương án $y=-x^3+x^2-2x+1$ sai.
 \item Xét hàm số $y=\dfrac{x^2-x+3}{x-1}=x+\dfrac{3}{x-1}$.\\
 Ta có
 $\lim\limits_{x \to+\infty}[y-x]=\lim\limits_{x \to+\infty} \dfrac{3}{x-1}=0$ và $\lim\limits_{x \to-\infty}[y-x]=\lim\limits_{x \to-\infty} \dfrac{3}{x-1}=0$.\\
 Do đó đường thẳng $y=x$ là đường tiệm cận xiên của đồ thị hàm số. Suy ra phương án $y=\dfrac{x^2-x+3}{x-1}$ sai.
 \item Xét hàm số $y=\dfrac{x^2-3x+6}{x-1}=x-2+\dfrac{4}{x-1}$.\\
 Ta có $\lim\limits_{x \to 1^-} y=-\infty$ và $\lim\limits_{x \to 1^+} y=+\infty$.\\
 Do đó đường thẳng $x=1$ là đường tiệm cận đứng của đồ thị hàm số.\\
 Lại có $\lim\limits_{x \to+\infty}[y-(x-2)]=\lim\limits_{x \to+\infty} \dfrac{4}{x-1}=0$ và $\lim\limits_{x \to-\infty}[y-(x-2)]=\lim\limits_{x \to-\infty} \dfrac{4}{x-1}=0$.\\
 Do đó đường thẳng $y=x-2$ là đường tiệm cận xiên của đồ thị hàm số.\\
 Hơn nữa, đồ thị hàm số cắt trục tung tại điểm có tung độ bằng $-6$ nên suy ra phương án $y=\dfrac{x^2-3x+6}{x-1}$ đúng.
 \item Xét hàm số $y=\dfrac{2x+3}{x-1}$. Vì đồ thị hàm số $y=\dfrac{2x+3}{x-1}$ không có đường tiệm cận xiên nên phương án $y=\dfrac{2x+3}{x-1}$ sai.
 \end{itemize}
 }
\end{ex}

\begin{ex}%[2-D1B5-SO-14-2425]%[VN-MT-7, Đỗ Minh Phúc]%[2D1V3-6]
 Khi nuôi cá thí nghiệm trong hồ, một nhà khoa học đã nhận thấy rằng: nếu trên mỗi đơn vị diện tích của mặt hồ có $n$ con cá thì trung bình mỗi con cá sau một vụ cân nặng là $P(n)=800-20n$ (g). Hỏi phải thả bao nhiêu con cá trên một đơn vị diện tích của mặt hồ để sau một vụ thu hoạch được nhiều cá nhất?
 \choice
 {$19$}
 {\True $20$}
 {$21$}
 {$22$}
 \loigiai{
 Gọi $F(n)$ là hàm cân nặng của $n$ con cá sau vụ thu hoạch trên một đơn vị diện tích.\\
 Ta có $F(n)=(800-20n) \cdot n=800n-20n^2$.\\
 Để sau một vụ thu hoạch được nhiều cá nhất thì cân nặng của $n$ con cá trên một đơn vị điện tích của mặt hồ là lớn nhất.\\
 Bài toán trở thành tìm $n\in \mathbb{N}^*$ sao cho $F(n)$ đạt giá trị lớn nhất.\\
 Ta có $F'(n)=800-40n$.\\
 Cho $F'(n)=0 \Leftrightarrow 800-40n=0 \Leftrightarrow n=20$.\\
 Ta có bảng biến thiên
 \begin{center}
 \begin{tikzpicture}
 \tkzTabInit[nocadre,lgt=1.2,espcl=2.5,deltacl=0.6]
 {$n$/0.6,$F'(n)$/0.6,$F(n)$/2}{$-\infty$,$20$,$+\infty$}
 \tkzTabLine{,+,0,-,}
 \tkzTabVar{-/$-\infty$,+/$8\,000$,-/$-\infty$}
 \end{tikzpicture}
 \end{center}
 Vậy phải thả $20$ con cá trên một đơn vị diện tích của mặt hồ để sau một vụ thu hoạch được nhiều cá nhất.
 }
\end{ex}

\begin{ex}%[2-D1B5-SO-14-2425]%[VN-MT-7, Đỗ Minh Phúc]%[2D1H2-1]
 Hàm số $f(x)=x^3-3x^2-9x+1$ đạt cực đại tại điểm
 \choice
 {\True $x=-1$}
 {$x=1$}
 {$x=3$}
 {$x=-3$}
 \loigiai{
 Ta có $f'(x)=3x^2-6x-9$.\\
 Cho $f'(x)=0\Leftrightarrow 3x^2-6x-9=0\Leftrightarrow\hoac{&x=-1\\&x=3.}$\\
 Ta có bảng biến thiên
 \begin{center}
 \begin{tikzpicture}
 \tkzTabInit[nocadre,lgt=1.2,espcl=2.5,deltacl=0.6]
 {$x$/0.6,$y'$/0.6,$y$/2}
 {$-\infty$,$-1$,$3$,$+\infty$}
 \tkzTabLine{,+,0,-,0,+,}
 \tkzTabVar{-/$-\infty$,+/$6$,-/$-26$,+/$+\infty$}
 \end{tikzpicture}
 \end{center}
 Dựa vào bảng biến thiên, ta thấy hàm số đạt cực đại tại $x=-1$. 
 }
\end{ex}
\Closesolutionfile{ans}

\cauds
\Opensolutionfile{ans}[ans/ans\currfilebase-Phan-II]
\begin{ex}%[2-D1B5-SO-14-2425]%[VN-MT-7, Đỗ Minh Phúc]%[2D1H5-3]
 Cho hàm số $y=2x^{3}+x^{2}-\dfrac{1}{2}x-3$ có đồ thị $(C)$.
 \choiceTF
 {\True Hàm số xác định trên $\mathbb{R}$}
 {Hàm số đồng biến trên $(-\infty;+\infty)$}
 {Hàm số không có cực trị}
 {\True Đồ thị hàm số cắt đường thẳng $y=m$ tại $3$ điểm khi và chỉ khi $-\dfrac{329}{108}<m<-\dfrac{11}{4}$}
 \loigiai{
 Ta có $y'=6x^{2}+2x-\dfrac{1}{2}$.\\
 Cho $y'=0 \Leftrightarrow 6x^{2}+2x-\dfrac{1}{2}=0 \Leftrightarrow \hoac{&x=-\dfrac{1}{2}\\&x=\dfrac{1}{6}.}$\\
 Bảng biến thiên
 \begin{center}
 \begin{tikzpicture}[>=stealth]
 \tkzTabInit[nocadre,lgt=1,espcl=2,deltacl=0.5]{$x$/1,$y'$/.7,$y$/2}
 {$-\infty$,$-\dfrac{1}{2}$,$\dfrac{1}{6}$, $+\infty$}
 \tkzTabLine{,+,$0$,-,$0$,+,}
 \tkzTabVar{-/$-\infty$,+/$-\dfrac{11}{4}$,-/$-\dfrac{329}{108}$,+/$+\infty$}
 \end{tikzpicture}
 \end{center}
 \begin{itemchoice}
 \itemch {\bf Đúng}.\\
 Tập xác định $\mathbb{R}$.
 \itemch {\bf Sai}.
 \begin{itemize}
 \item Hàm số đồng biến trên $\left(-\infty; -\dfrac{1}{2}\right)$ và $\left(\dfrac{1}{6}; +\infty\right)$.
 \item Hàm số nghịch biến trên $ \left(-\dfrac{1}{2}; \dfrac{1}{6}\right)$.
 \end{itemize}
 \itemch {\bf Sai}.\\
 Hàm số đạt cực đại tại $x_{\text{CĐ}}=-\dfrac{1}{2}$, $y_{\text{CĐ}}=-\dfrac{11}{4}$; hàm số đạt cực tiểu tại $x_{\text{CT}}=\dfrac{1}{6}$, $y_{\mathrm{CT}}=-\dfrac{329}{108}$.
 \itemch {\bf Đúng}.\\
 Dựa vào bảng biến thiên, đồ thị hàm số cắt đường thẳng $y=m$ tại $3$ điểm khi và chỉ khi $-\dfrac{329}{108}<m<-\dfrac{11}{4}$.
 \end{itemchoice} 
 }
\end{ex}

\begin{ex}%[2-D1B5-SO-14-2425]%[VN-MT-7, Đỗ Minh Phúc]%[2D1V5-3]
 \immini[thm]{Cho hàm số $y=f(x)$ xác định trên $\mathbb{R}$. Đồ thị hàm số $y=f'(x)$ cắt trục hoành tại $3$ điểm phân biệt $a$, $b$, $c$ ($a<b<c)$ như hình bên.
 \choiceTF
 {Hàm số $y=f(x)$ đồng biến trên $(-\infty;a)$}
 {Hàm số có $2$ điểm cực trị}
 {\True Giá trị cực đại của hàm số là $f(b)$}
 {\True Biết $ f(b) < 0$. Đồ thị hàm số $ y = f (x)$ cắt trục hoành tại hai điểm phân biệt}
 }
 {\begin{tikzpicture}[scale=0.7, font=\footnotesize, line join=round, line cap=round, >=stealth]
 \draw[->,black] (-2.5,0) -- (3,0)node[above left] {$x$};
 \draw[->,black] (0,-2.5) -- (0,3.1)node[below right] {$y$};
 \node at (-1.62,0) [above left] {$a$};
 \node at (0.41,0) [above] {$b$};
 \node at (1.82,0) [above] {$c$};
 \node at (0,0) [below left] {$O$};
 \draw[smooth,samples=100,domain=-2.1:2.5] plot(\x,{0.6*(\x)^3-0.4*(\x)^2-1.92*(\x)+0.8});
 \fill %vẽ các điểm rỗng ruột
 (-1.68,0) circle (1pt)
 (0.41,0) circle (1pt)
 (1.94,0) circle (1pt)
 (0,0) circle (1pt);
 \end{tikzpicture}}
 \loigiai{
 Ta có $f'(x)=0\Leftrightarrow\hoac{&x=a\\&x=b\\&x=c.} $
 \begin{center}
 \begin{tikzpicture}
 \tkzTabInit[nocadre,lgt=1.5,espcl=2.5,deltacl=0.6]
 {$x$/1,$y'$/1,$y$/2}
 {$-\infty$,$a$,$b$,$c$,$+\infty$}
 \tkzTabLine{,-,$0$,+,$0$,-,$0$,+}
 \tkzTabVar{+/ $+\infty$ ,-/$f\left(a\right)$,+/$f\left(b\right)$,-/$f\left(c\right)$,+/ $+\infty$}
 \end{tikzpicture}
 \end{center}
 \begin{itemchoice}
 \itemch {\bf Sai}.\\
 Theo bảng biến thiên, hàm số $y=f(x)$ nghịch biến trên $(-\infty;a)$.
 \itemch {\bf Sai}.\\
 Theo bảng biến thiên, hàm số $y=f(x)$ có $3$ điểm cực trị.
 \itemch {\bf Đúng}.\\
 Theo bảng biến thiên, giá trị cực đại của hàm số là $f(b)$.
 \itemch {\bf Đúng}.\\
 Do $ f(b)<0$ nên đồ thị hàm số cắt trục hoành tại hai điểm phân biệt.
 \end{itemchoice}
 }
\end{ex}

\begin{ex}%%[2-D1B5-SO-14-2425]%[VN-MT-7, Đỗ Minh Phúc]%[2D1H5-4]
 \immini[thm]{Cho hàm số $y=f(x)=\dfrac{ax+b}{cx-1}$ có đồ thị như hình vẽ bên. Các khẳng định sau là đúng hay sai?
 \choiceTF
 {$b=-2$}
 {\True $a+b+c=2$}
 {\True Phương trình $f(x)=1$ có duy nhất một nghiệm}
 {\True Đồ thị hàm số nhận điểm $I(1;-1)$ là tâm đối xứng}
 }
 {\begin{tikzpicture}[scale=0.7,>=stealth, font=\footnotesize, line join=round, line cap=round]
 \def\a{-1} \def\b{2} \def\c{1} \def\d{-1} % Hệ số
 \def\xmin{-3} \def\xmax{5}
 \def\ymin{-4} \def\ymax{4}
 \draw[->] (\xmin,0)--(\xmax,0) node [below]{$x$};
 \draw[->] (0,\ymin)--(0,\ymax) node [left]{$y$};
 \node at (0,0) [above left]{$O$};
 \clip (\xmin+0.1,\ymin+0.1) rectangle (\xmax-0.1,\ymax-0.1);
 \draw[smooth,samples=300,domain=\xmin:(-\d/\c-0.1)] plot(\x,{(\a*(\x)+\b)/(\c*(\x)+\d)});
 \draw[smooth,samples=300,domain=(-\d/\c+0.1:\xmax)] plot(\x,{(\a*(\x)+\b)/(\c*(\x)+\d)});
 \draw (-\d/\c,\ymin)--(-\d/\c,\ymax);
 \draw (\xmin,\a/\c)--(\xmax,\a/\c);
 \foreach \d/\g in{1/135,2/60}
 \draw[fill=black](\d,0)circle(1pt)node[shift={(\g:0.35)}]{$\d$};
 \foreach \d/\g in{-2/180,-1/135}
 \draw[fill=black](0,\d)circle(1pt)node[shift={(\g:0.35)}]{$\d$};
 \fill (0,0) circle (1pt);
 \end{tikzpicture}}
 \loigiai{
 \begin{itemchoice}
 \itemch {\bf Sai}.\\
 Vì điểm $(0;-2)$ thuộc đồ thị hàm số $y=f(x)$ nên ta có $\dfrac{b}{-1}=-2\Leftrightarrow b=2$.
 \itemch {\bf Đúng}.\\
 Vì điểm $(0;-2)$ thuộc đồ thị hàm số $y=f(x)$ nên ta có $\dfrac{b}{-1}=-2\Leftrightarrow b=2$.\\
 Đồ thị hàm số $y=f(x)$ có tiệm cận ngang $y=\dfrac{a}{c}$ và tiệm cận đứng $x=\dfrac{1}{c}$, do đó
 \[\heva{&\dfrac{a}{c}=-1\\&\dfrac{1}{c}=1}\Leftrightarrow \heva{&a=-1\\&c=1.}\]
 Vậy $a+b+c=2$.
 \itemch {\bf Đúng}.\\
 Vẽ đường thẳng $y=1$ trên mặt phẳng tọa độ, ta thấy đường thẳng $y=1$ cắt đồ thị hàm số $y=f(x)$ tại duy nhất một điểm.
 \itemch {\bf Đúng}.\\
 Đồ thị hàm số nhận đường thẳng $y=-1$ làm tiệm cận ngang và $x=1$ là tiệm cận đứng, do đó điểm $(1;-1)$ là tâm đối xứng của đồ thị.
 \end{itemchoice}
 }
\end{ex}

\begin{ex}%[2-D1B5-SO-14-2425]%[VN-MT-7, Đỗ Minh Phúc]%[2D1H4-1]
 Cho hàm số $y=f(x)=\dfrac{x^2+mx-1}{x-1}$.
 \choiceTF
 {Hàm số có cực trị khi và chỉ khi $m\geq 0$}
 {\True Tiệm cận xiên của đồ thị hàm số là đường thẳng $y=x+m+1$}
 {Với $m=1$, hàm số nghịch biến trên khoảng $(0;2)$}
 {Tổng các giá trị nguyên dương của tham số $m$ để hàm số đồng biến trên khoảng $(3;5)$ bằng $6$}
 \loigiai{\begin{itemchoice}
 \itemch {\bf Sai}.\\
 Có $y'=\dfrac{x^2-2x-m+1}{(x-1)^2}$.\\
 Hàm số có hai cực trị khi và chỉ khi phương trình $x^2-2x-m+1=0 $ có hai nghiệm phân biệt khác $1\Leftrightarrow \heva{&\Delta' >0\\&1^2-2\cdot 1-m+1\ne 0} \Leftrightarrow \heva{&m>0\\&m \ne 0} \Leftrightarrow m>0$. 
 \itemch {\bf Đúng}.\\
 Ta có $y=x+m+1+\dfrac{m}{x-1}$.\\
 $\lim\limits_{x\to \pm \infty}\left[y-(x+m+1)\right]=\lim\limits_{x\to\pm \infty}\dfrac{m}{x-1}=0$.\\
 Vậy đồ thị hàm số có tiệm cận xiên $y=x+m+1$.
 \itemch {\bf Sai}.\\
 Với $m=1$, hàm số trở thành $y=\dfrac{x^2+x-1}{x-1}$ không xác định trên khoảng $(0;2)$ nên không nghịch biến trên khoảng $(0;2)$.
 \itemch {\bf Sai}.\\
 Hàm số đồng biến trên khoảng $(3;5)$
 \begin{eqnarray*}
 \Leftrightarrow x^2-2x-m+1\geq 0,\ \forall x\in (3;5)
 &\Leftrightarrow& m\leq x^2-2x+1,\ \forall x\in (3;5)\\
 &\Leftrightarrow& m\leq \min\limits_{[3;5]} \left(x^2-2x+1\right).
 \end{eqnarray*}
 Xét hàm $g(x)=x^2-2x+1 $ có bảng biến thiên
 \begin{center}
 \begin{tikzpicture}
 \tkzTabInit[nocadre,lgt=1.2,espcl=2.5,deltacl=0.6]
 {$x$/.7 ,$g'(x)$/.7,$g(x)$/2}
 {$-\infty$,$1$,$+\infty$}
 \tkzTabLine{,-,0,+,}
 \tkzTabVar{+/,-/$g(1)$,+/ /}
 \end{tikzpicture}
 \end{center}
 Từ bảng biến thiên suy ra $m\leq g(3) \Leftrightarrow m\leq 4$.\\
 Vì $m$ nguyên dương nên $m \in \{1;2;3;4\}$.\\
 Vậy tổng các giá trị $m$ thỏa mãn yêu cầu đề bài bằng $10$.
 \end{itemchoice}}
\end{ex}
\Closesolutionfile{ans}

\caukq
\Opensolutionfile{ans}[ans/ans\currfilebase-Phan-III]
\begin{ex}%[2-D1B5-SO-14-2425]%[VN-MT-7, Đỗ Minh Phúc]%[2D1V1-3]
 Có bao nhiêu giá trị nguyên của tham số $m$ để hàm số $y=(m-1)x^3-(m-1)x^2+3x+2024$ đồng biến trên tập xác định?
\shortans{10}
\loigiai{
Tập xác định $\mathscr{D}=\mathbb{R}$.\\
Ta có $y'=3(m-1)x^2-2(m-1)x+3$.\\
Hàm số đồng biến trên $\mathbb{R}$ khi $y' \geq 0$, $\forall x \in \mathbb{R} \Leftrightarrow 3(m-1)x^2-2(m-1)x+3 \geq 0$, $\forall x \in \mathbb{R}$.
\begin{itemize}
 \item Nếu $m-1=0 \Leftrightarrow m=1$. Khi đó $y' \geq 0 \Leftrightarrow 3 \geq 0$ luôn đúng $\forall x \in \mathbb{R}$.\\
 Suy ra $m=1$ thoả mãn yêu cầu bài toán.
 \item Nếu $m-1 \neq 0 \Leftrightarrow m \neq 1$.\\
 Khi đó 
 \begin{eqnarray*}
 3(m-1) x^2-2(m-1) x+3 \geq 0, \forall x \in \mathbb{R}
 &\Leftrightarrow&\heva{&\Delta'=(m-1)^2-9(m-1) \leq 0 \\ &a=m-1>0}\\
 &\Leftrightarrow&\heva{&1 \leq m \leq 10 \\ &m>1} \Leftrightarrow 1<m \leq 10\text{ (thỏa mãn)}.
 \end{eqnarray*}
 Mà $m \in \mathbb{Z} \Rightarrow m \in\{2;3;4;5;6;7;8;9;10\}$.
\end{itemize}
Vậy có tất cả $10$ giá trị nguyên của tham số $m$ thoả mãn yêu cầu bài toán.
}
\end{ex}

\begin{ex}%[2-D1B5-SO-14-2425]%[VN-MT-7, Đỗ Minh Phúc]%[2D1V1-3]
 Cho hàm số $y=f(x)$ liên tục trên $\mathbb{R}$ thoả mãn $f'(x)=x(x-1)^2(x-2)^3$. Hàm số $g(x)=f\left(x^2-2x+2\right)$ có bao nhiêu điểm cực trị?
 \shortans{3}
 \loigiai{
 Ta có $f'(x)=0 \Leftrightarrow\hoac{&x=0\\&x=1\\&x=2.}$\\
 Bảng biến thiên
 \begin{center}
 \begin{center}
 \begin{tikzpicture}[scale=1, font=\footnotesize, line join=round, line cap=round, >=stealth]
 \tkzTabInit[nocadre,lgt=1.2,espcl=2.5,deltacl=0.6]
 {$x$/.6,$f'(x)$/.6,$f(x)$/2}
 {$-\infty$,$0$,$1$,$2$,$+\infty$}
 \tkzTabLine{,+,$0$,-,$0$,-,$0$,+}
 \tkzTabVar{-/$-\infty$,+/$ $,R,-/$ $,+/$+\infty$}
 \end{tikzpicture}
 \end{center}
 \end{center}
 Ta có $g'(x)=(2x-2)f'\left(x^2-2x+2\right)$.\\
 Cho $g'(x)=0 \Leftrightarrow\hoac{&2x-2=0\\&f'\left( x^2-2x+2\right)=0}\Leftrightarrow\hoac{&x=1\\&x^2-2x+2=0\\&x^2-2x+2=1\\&x^2-2x+2=2}\Leftrightarrow\hoac{&x=1\\&x=0\\&x=2.}$\\
 Bảng biến thiên
 \begin{center}
 \begin{tikzpicture}
 \tkzTabInit[nocadre,lgt=1.2,espcl=2.5,deltacl=0.6]
 {$x$/0.6,$g'(x)$/0.6,$g(x)$/2}
 {$-\infty$,$0$,$1$,$2$,$+\infty$}
 \tkzTabLine{,-,0,+,0,-,0,+,}
 \tkzTabVar{+/$+\infty$,-/,+/,-/,+/$+\infty$}
 \end{tikzpicture}
 \end{center}
 Vây hàm số $g(x)=f\left(x^2-2x+2\right)$ có $3$ điểm cực trị.
 }
\end{ex}

\begin{ex}%[2-D1B5-SO-14-2425]%[VN-MT-7, Đỗ Minh Phúc]%[2D1V3-1]
 Tìm $m$ để giá trị lớn nhất của hàm số $y=\dfrac{x-m}{x+1}$ trên đoạn $[1;3]$ bằng $2$.
 \shortans{-3}
 \loigiai{
 Ta có $y'=\dfrac{1+m}{(x+1)^2}$.
 \begin{itemize}
 \item Trường hợp 1: $1+m>0 \Leftrightarrow m>-1$.\\
 Khi đó $y'>0$, $\forall x \in[1;3]$ nên hàm số $y=\dfrac{x-m}{x+1}$ đồng biến trên đoạn $[1;3]$.\\
 Suy ra $\max _{[1;3]}y=y(3)=\dfrac{3-m}{4}=2 \Leftrightarrow m=-5$ (loại).
 \item Trường hợp 2: $1+m<0 \Leftrightarrow m<-1$.\\
 Khi đó $y'<0$, $\forall x \in[1;3]$ nên hàm số $y=\dfrac{x-m}{x+1}$ nghịch biến trên đoạn $[1;3]$.\\
 Suy ra $\max _{[1;3]}y=y(1)=\dfrac{1-m}{2}=2 \Leftrightarrow m=-3$ (thỏa mãn).
 \end{itemize}
 Vậy $m=-3$ là giá trị cần tìm.
 }
\end{ex}

\begin{ex}%[2-D1B5-SO-14-2425]%[VN-MT-7, Đỗ Minh Phúc]%[2D1V3-6]
 Chị Hà dự định sử dụng hết $4$ m$^2$ kính để làm một bể cá bằng kính có dạng hình hộp chữ nhật không nắp, chiều dài gấp đôi chiều rộng (các mối ghép có kích thước không đáng kể). Bể cá có dung tích lớn nhất bằng bao nhiêu mét khối (kết quả làm tròn đến hàng phần trăm)?
 \shortans{0{,}73}
 \loigiai{
 \begin{center}
 \begin{tikzpicture}[scale=0.8, font=\footnotesize, line join=round, line cap=round, >=stealth]
 \coordinate (B) at (0,0); 
 \coordinate (A) at (2,2);
 \coordinate (C) at (4,0);
 \coordinate (D) at ($(C)-(B)+(A)$);
 \coordinate (B') at ($(B)+(90:3)$);
 \coordinate (C') at ($(C)+(90:3)$);
 \coordinate (D') at ($(D)+(90:3)$);
 \coordinate (A') at ($(A)+(90:3)$);
 \draw (A')--(B')--(C')--(D')--(A') (B)--(B') (C)--(C') (D)--(D')node[right,midway]{$h$} (B)--(C)node[below,midway,sloped]{$2x$}--(D)node[right,midway]{$x$};
 \draw [dashed] (B)--(A)--(D) (A')--(A);
 \end{tikzpicture}
 \end{center}
 Giả sử bể cá có kích thước như hình vẽ, với $x$, $h>0$.\\
 Theo đề bài ta có $2x^2+2xh+4xh=4 \Leftrightarrow h=\dfrac{4-2x^2}{6x}$.\\
 Do $x>0$, $h>0$ nên $4-2x^2>0 \Leftrightarrow 0<x<\sqrt{2}$.\\
 Thể tích của bể cá là $V=2x^2h=\dfrac{4x-2x^3}{3}=f(x)$, với $x \in\left(0;\sqrt{2}\right)$.\\
 Ta có $f'(x)=\dfrac{4}{3}-2x^2$.\\
 Cho $f'(x)=0 \Leftrightarrow \dfrac{4}{3}-2x^2=0\Leftrightarrow x=\dfrac{\sqrt{6}}{3}$ (vì $x>0$).\\
 Bảng biến thiên
 \begin{center}
 \begin{tikzpicture}
 \tkzTabInit[nocadre,lgt=1.2,espcl=2.5,deltacl=0.6]
 {$x$/0.6,$f'(x)$/0.6,$f(x)$/2}{$0$,$\tfrac{\sqrt{6}}{3}$,$\sqrt{2}$}
 \tkzTabLine{,+,0,-,}
 \tkzTabVar{-/$0$,+/$\tfrac{8\sqrt{6}}{27}$,-/$0$}
 \end{tikzpicture}
 \end{center}
 Vậy bể cá có dung tích lớn nhất bằng $\dfrac{8\sqrt{6}}{27} \mathrm{~m}^3 \approx 0{,}73 \mathrm{~m}^3$.
}
\end{ex}

\begin{ex}%[2-D1B5-SO-14-2425]%[VN-MT-7, Đỗ Minh Phúc]%[2D1V4-2]
 Có bao nhiêu giá trị thực của tham số $m$ để đồ thị hàm số $y=\dfrac{x^2-1}{x^2+(2-m)x+2m+1}$ có đúng hai đường tiệm cận?
 \shortans{3}
 \loigiai{
 Ta có $\lim\limits_{x \to \pm \infty} y=\lim\limits_{x \to \pm \infty} \dfrac{x^2-1}{x^2+(2-m) x+2 m+1}=\lim\limits_{x \to \pm \infty} \dfrac{1-\dfrac{1}{x^2}}{1+(2-m) \dfrac{1}{x}+(2 m+1) \dfrac{1}{x^2}}=1$.\\
 Suy ra đồ thị của hàm số đã cho có đường tiệm cận ngang $y=1$, do vậy đồ thị đó có đúng hai đường tiệm cận khi và chỉ khi đồ thị hàm số có đúng một đường tiệm cận đứng
 $\Leftrightarrow$ phương trình $x^2+(2-m) x+2 m+1=0$ (*) có nghiệm kép hoặc có một nghiệm $x=-1$ và một nghiệm khác $1$ hoặc có một nghiệm $x=1$ và một nghiệm khác $-1$.
 \begin{itemize}
 \item Trường hợp 1: Phương trình (*) có nghiệm kép
 \[\Leftrightarrow \Delta=0 \Leftrightarrow(2-m)^2-4(2m+1)=0 \Leftrightarrow m^2-12 m=0 \Leftrightarrow\hoac{&m=0\\&m=12.}\]
 \item Trường hợp 2: Phương trình (*) một có nghiệm $x=1$ và một nghiệm khác $-1$
 \[\Leftrightarrow\heva{&m=-4\\&m \neq 0}\Leftrightarrow m=-4.\]
 \item Trường hợp 3: Phương trình (*) một có nghiệm $x=-1$ và một nghiệm khác $1$ \[\Leftrightarrow\heva{&m=0\\&m\neq-4}\Leftrightarrow m=0.\]
 \end{itemize}
Vậy có $3$ giá trị của $m$ thỏa mãn yêu cầu bài toán là $m=-4$, $m=0$, $m=12$.
}
\end{ex}

\begin{ex}%[2-D1B5-SO-14-2425]%[VN-MT-7, Đỗ Minh Phúc]%[2D1V1-2]
 \immini[thm]{Cho hàm số $f(x)$ liên tục trên $\mathbb{R}$ và có đồ thị hàm số $y=f'(x)$ như hình vẽ bên. Hàm số $y=f(x)-\dfrac{1}{3}x^3+6x$ đồng biến trên khoảng $(a;b)$. Khi đó giá trị của biểu thức $b-a$ bằng bao nhiêu?}{\begin{tikzpicture}[scale=0.6, font=\footnotesize, line join=round,line cap=round,>=stealth]
 \def\xmin{-5}\def\xmax{3}\def\ymin{-3}\def\ymax{3}
 \draw[->] (\xmin-0.2,0)--(\xmax+0.2,0) node[above] {$x$};
 \draw[->] (0,\ymin-0.2)--(0,\ymax+0.2) node[right] {$y$};
 \draw (0,0) node [below left] {$O$};
 \foreach \x in {-2,2}
 \fill (\x,0)circle (1pt) node [above] {$\x$};
 \foreach \y in {2}
 \fill (0,\y)circle (1pt) node [left] {$\y$};
 \clip (\xmin,\ymin) rectangle (\xmax,\ymax);
 \draw[smooth,samples=200,domain=-4:0] plot (\x,{1*((\x)^2)+4*\x+2});
 \draw[smooth,samples=200,domain=0:3] plot (\x,{-1*((\x)^2)+0*\x+2});
 \draw[dashed] (-2,0)--(-2,-2)--(2,-2)--(2,0);
 \fill (-2,-2) circle (1pt);
 \fill (2,-2) circle (1pt);
 \fill (0,0) circle (1pt);
 \end{tikzpicture}}
 \shortans{4}
 \loigiai{
 Ta có $y=f(x)-\dfrac{1}{3} x^3+6 x$ nên $y'=f'(x)-x^2+6$.\\
 Quan sát đồ thị hàm số $y=f'(x)$ và parabol $(P)\colon y=x^2-6$ trên cùng một hệ trục tọa độ như hình vẽ.
 \begin{center}
 \begin{tikzpicture}[scale=0.8, font=\footnotesize, line join=round, line cap=round, >=stealth]
 \def\xmin{-5}\def\xmax{3}\def\ymin{-7}\def\ymax{3}
 \draw[->] (\xmin-0.2,0)--(\xmax+0.2,0) node[above] {$x$};
 \draw[->] (0,\ymin-0.2)--(0,\ymax+0.2) node[right] {$y$};
 \draw (0,0) node [below left] {$O$};
 \foreach \x in {-2,2}
 \fill (\x,0)circle (1pt) node [above] {$\x$};
 \foreach \y in {2}
 \fill (0,\y)circle (1pt) node [left] {$\y$};
 \foreach \y in {-6,-2}
 \fill (0,\y)circle (1pt) node [below right] {$\y$};
 \clip (\xmin,\ymin) rectangle (\xmax,\ymax);
 \draw[smooth,samples=200,domain=-4:0] plot (\x,{1*((\x)^2)+4*\x+2});
 \draw[smooth,samples=200,domain=0:3] plot (\x,{-1*((\x)^2)+0*\x+2});
 \draw[smooth,samples=200,domain=-3:3] plot (\x,{1*((\x)^2)+0*\x+-6});
 \draw[dashed] (-2,0)--(-2,-2)--(2,-2)--(2,0);
 \fill (-2,-2) circle (1pt);
 \fill (2,-2) circle (1pt);
 \fill (0,0) circle (1pt);
 \end{tikzpicture}
 \end{center}
 Từ đồ thị ta có $y'=f'(x)-x^2+6>0 \Leftrightarrow f'(x)>x^2-6 \Leftrightarrow-2<x<2$.\\
 Vậy hàm số $y=f(x)-\dfrac{1}{3}x^3+6x$ đồng biến trên khoảng $(-2;2)$.
}
\end{ex}
\Closesolutionfile{ans}
\begin{indapan}
	{ans/ans\currfilebase}
\end{indapan}


% \begin{name}
	{\tenchude}
	{ĐỀ ÔN TẬP CHƯƠNG I}
	{LỚP TOÁN THẦY PHÁT}
	{\thoigian}
\end{name}
\TN
\Opensolutionfile{ans}[ans/ans\currfilebase-Phan-I]
\begin{ex}%[2-D1B5-SO-15-2425]%[VN-MT-7, Đoàn Thị Lý]%[2D1N1-1]
 Cho hàm số $y=f(x)$ có đạo hàm $f'(x)=x(x-2)^3$, với mọi $x \in \mathbb{R}$. Hàm số đã cho nghịch biến trên khoảng nào dưới đây?
 \choice 
 {$(1;3)$}
 {$(-1;0)$}
 {\True $(0;1)$}
 {$(-2;0)$}
 \loigiai{
 Ta có $f'(x)=0 \Leftrightarrow\hoac{&x=0 \\&x=2.}$ \\
 Bảng xét dấu $f'(x)$
 \begin{center}
 \begin{tikzpicture}
 \tkzTabInit[nocadre=true,lgt=1.2,espcl=2.5,deltacl=0.5]
 {$x$ /.7, $f'(x)$ /.7}
 {$-\infty$,$0$,$2$,$+\infty$}
 \tkzTabLine{ ,+,0,-,0, +, }
 \end{tikzpicture}
 \end{center}
 Dựa vào bảng xét dấu $f'(x)$ ta thấy hàm số đã cho nghịch biến trên $(0;1)$.
 }
\end{ex}

\begin{ex}%[2-D1B5-SO-15-2425]%[VN-MT-7, Đoàn Thị Lý]%[2D1N1-2] 
 Cho hàm số $y=f(x)$ có tập xác định là $\mathscr{D}=\mathbb{R}\setminus\big\{-1\big\}$ và có bảng xét dấu của đạo hàm như hình sau:
 \begin{center}
 \begin{tikzpicture}
 \tkzTabInit[nocadre=true,lgt=1.2,espcl=2.5,deltacl=0.5]
 {$x$ /.7, $f'(x)$ /.7}
 {$-\infty$,$-1$,$1$,$+\infty$}
 \tkzTabLine{ ,-,d,-,0, +, }
 \end{tikzpicture}
 \end{center}
 Hàm số đã cho nghịch biến trên khoảng nào dưới đây? 
 \choice 
 {$(1 ;+\infty)$}
 {$(-\infty; 1)$}
 {$(-1;+\infty)$}
 {\True $(-\infty;-1)$}
 \loigiai{ 
 Từ bảng xét dấu ta thấy hàm số đã cho nghịch biến trên khoảng $(-\infty;-1)$.
 }
\end{ex}

\begin{ex}%[2-D1B5-SO-15-2425]%[VN-MT-7, Đoàn Thị Lý]%[2D1N3-1]
 \immini{
 Cho hàm số $y=f(x)$ liên tục trên đoạn $[-2;2]$ và có đồ thị như hình vẽ bên. Gọi $M$ và $m$ lần lượt là giá trị lớn nhất và giá trị nhỏ nhất của hàm số đã cho trên đoạn $[-2;2]$. Giá trị của $M+m$ bằng
 \choice[2]
 {$0$}
 {$1$}
 {$4$}
 {\True $3$}
 }
 {\begin{tikzpicture} [scale=.85, font=\footnotesize, line join=round, line cap=round, >=stealth]
 \draw[->] (-3,0)--(0,0) node[below right]{$O$}--(3,0) node[below]{$x$};
 \draw[->] (0,-1)--(0,4) node[right]{$y$};
 \draw (-2,3) .. controls ++(-80:1) and ++(170:.5) .. (-1.4,0).. controls (-1.2,.2) and (-1.1,1).. (-1,1).. controls (-.8,1.1) and (-.6,0) .. (-.4,0) .. controls (-.3,0.1) .. (0,1).. controls (-.1,.75) and (.6,3) .. (1,3).. controls (1.15,2.9).. (2,1);
 \draw[dashed] (-2,0) node[below]{$-2$}--(-2,3) --(0,3)node[above left]{$3$}--(1,3)--(1,0)node[below]{$1$} (-1,0)node[below]{$-1$}--(-1,1)--(0,1)node[above left]{$1$}--(2,1)--(2,0)node[below]{$2$};
 \foreach \i in {(0,3),(-2,3),(1,3),(-1,1),(0,1),(-2,0), (-1,0),(1,0),(2,0),(2,1),(-1.4,0),(-.4,0),(0,0)}\fill \i circle (1pt); 
 \end{tikzpicture}
 }
 \loigiai{ 
 Quan sát đồ thị ta thấy $M=\max\limits_{[-2;2]} f(x)=3$ và $m=\min\limits_{[-2 ; 2]} f(x)=0$. Vậy $M+m=3+0=3$.
 }
\end{ex}

\begin{ex}%[2-D1B5-SO-15-2425]%[VN-MT-7, Đoàn Thị Lý]%[2D1N4-1]
 Cho hàm số $y=f(x)$ có bảng biến thiên như sau:
 \begin{center}
 \begin{tikzpicture}
 \tkzTabInit[nocadre=true,lgt=1.2,espcl=3, deltacl=0.5]
 {$x$/0.7,$f’(x)$/0.7,$f(x)$/2}
 {$-\infty$,$0$,$1$,$+\infty$}
 \tkzTabLine{,+,0,-,d,+,}
 \tkzTabVar{-/$0$,+/$2$,-D-/$-\infty$/$3$,+/$5$}
 \end{tikzpicture}
 \end{center}
 Tổng số tiệm cận ngang và tiệm cận đứng của đồ thị hàm số đã cho là 
 \choice 
 {$4$}
 {\True $3$}
 {$2$}
 {$1$}
 \loigiai{
 Từ bảng biến thiên ta có
 \begin{itemize}
 \item $\lim\limits _{x \to-\infty} y=0$, suy ra đồ thị hàm số có tiệm cận ngang là $y=0$.
 \item $\lim\limits _{x\to+\infty} y=5$, suy ra đồ thị hàm số có tiệm cận ngang là $y=5$.
 \item $\lim\limits _{x \to 1^-} y=+\infty$, suy ra đồ thị hàm số có tiệm cận đứng là $x=1$.
 \end{itemize}
 Vậy tổng số tiệm cận ngang và tiệm cận đứng của đồ thị hàm số đã cho là $3$.
 }
\end{ex}

\begin{ex}%[2-D1B5-SO-15-2425]%[VN-MT-7, Đoàn Thị Lý]%[2D1N4-1]
 Cho hàm số $y=\dfrac{2 x+1}{x-3}$. Tiệm cận ngang của đồ thị hàm số đã cho là
 \choice 
 {$x=3$}
 {$x=2$}
 {$x=-\dfrac{1}{2}$}
 {\True $y=2$}
 \loigiai{
 Ta có $\lim\limits _{x \to \pm \infty} y=\lim\limits _{x \to \pm \infty} \dfrac{2 x+1}{x-3}=\lim\limits _{x \to \pm \infty} \dfrac{2+\dfrac{1}{x}}{1-\dfrac{3}{x}}=2$, suy ra đồ thị hàm số có tiệm cận ngang $y=2$.
 }
\end{ex}

\begin{ex}%[2-D1B5-SO-15-2425]%[VN-MT-7, Đoàn Thị Lý]%[2D1N5-1]
 \immini{
 Biết hàm số $y=\dfrac{ax+b}{cx+d}$ có đồ thị như trong hình vẽ bên. Mệnh đề nào dưới đây đúng?
 \choice
 {\True $y'>0$, $\forall x \neq 1$}
 {$y'>0$, $\forall x \in \mathbb{R}$}
 {$y'<0$, $\forall x \in \mathbb{R}$}
 {$y'<0$, $\forall x \neq 1$}
 }{
 \begin{tikzpicture} [scale=.8, font=\footnotesize, line join=round, line cap=round, >=stealth]
 \draw[->] (-2,0)--(0,0) node[below right]{$O$}--(1,0) node[above right]{$1$}--(4,0) node[below]{$x$}; 
 \draw[->] (0,-2)--(0,4) node[left]{$y$}; 
 \fill (0,0) circle (1.0pt) (1,0) circle (1.0pt); 
 \begin{scope}
 \clip (-2,-2) rectangle (4,4);
 \draw[samples=300,domain=-2:4,smooth] plot (\x, {(\x-2)/(\x-1)}); 
 \end{scope}
 \draw (-2,1)--(4,1); 
 \end{tikzpicture}
 }
 \loigiai{ 
 Tập xác định của hàm số đã cho là $\mathscr{D}=\mathbb{R}\setminus\{1\}$.\\
 Từ đồ thị của hàm số suy ra hàm số đã cho đồng biến trên mỗi khoảng xác định vì vậy $y'>0, \forall x \neq 1$.
 }
\end{ex}

\begin{ex}%[2-D1B5-SO-15-2425]%[VN-MT-7, Đoàn Thị Lý]%[2D1N5-1]
 \immini{Đồ thị của hàm số nào dưới đây có dạng như đường cong trong hình bên?
 \choice 
 {\True $y=x^3-3 x$}
 {$y=-x^3+3 x$}
 {$y=x^4-2 x^2$}
 {$y=-x^4+2 x^2$}
 }
 {
 \begin{tikzpicture} [scale=.8, font=\footnotesize, line join=round, line cap=round, >=stealth]
 \draw[->] (-2.5,0)--(0,0) node[below left]{$O$}--(2.5,0) node[below]{$x$};
 \draw[->] (0,-2.5)--(0,2.5) node[right]{$y$};
 \draw[samples=100,domain=-2.05:2.05,smooth] plot (\x, {(\x)^3-3*(\x)});
 \fill (0,0) circle (1.0pt);
 \end{tikzpicture} 
 }
 \loigiai{
 Đường cong có dạng của đồ thị hàm số bậc ba với hệ số $a>0$ nên chỉ có hàm số $y=x^3-3 x$ thỏa yêu cầu bài toán.
 }
\end{ex}

\begin{ex}%[2-D1B5-SO-15-2425]%[VN-MT-7, Đoàn Thị Lý]%[2D1H1-3]
 Tập hợp tất cả các giá trị thực của tham số $m$ để hàm số $y=x^3+(m+1) x^2+3 x+2$ đồng biến trên $\mathbb{R}$ là
 \choice 
 {\True $[-4;2]$}
 {$(-4;2)$}
 {$(-\infty;-4] \cup[2;+\infty)$}
 {$(-\infty;-4)\cup(2;+\infty)$}
 \loigiai{ 
 Tập xác định $\mathscr{D}=\mathbb{R}$.\\
 Ta có $y'=3 x^2+2(m+1) x+3$.\\
 Hàm số $y=x^3+(m+1) x^2+3 x+2$ đồng biến trên $\mathbb{R}$ khi và chỉ khi 
 \[y' \geq 0, \forall x \in \mathbb{R}
 \Leftrightarrow \Delta'=(m+1)^2-9 \leq 0 \Leftrightarrow m^2+2 m-8 \leq 0 \Leftrightarrow-4 \leq m \leq 2.\] 
 Vậy $m \in[-4 ; 2]$.
 }
\end{ex}

\begin{ex}%[2-D1B5-SO-15-2425]%[VN-MT-7, Đoàn Thị Lý]%[2D1H3-1]
 Gọi $M$ và $m$ lần lượt là giá trị lớn nhất và giá trị nhỏ nhất của hàm số $y=f(x)=\dfrac{2x+1}{x-2}$ trên đoạn $[3;7]$ . Tính giá trị của $M^2+m$.
 \choice 
 {\True $52$}
 {$58$}
 {$6$}
 {$10$}
 \loigiai{
 Hàm số $y=f(x)=\dfrac{2 x+1}{x-2}$ liên tục trên $[3;7]$.\\
 Ta có 
 $y'=\dfrac{-5}{(x-2)^2}<0, \forall x \in[3 ; 7]$ nên hàm số $y=f(x)$ nghịch biến trên $[3;7]$.\\ Lúc đó
 \[M=\max\limits_{[3;7]} f(x) = f(3)=7,m=\min\limits_{[3;7]}f(x)=f(7)=3.\]
 Vậy $M^2+m=52$.
 }
\end{ex}

\begin{ex}%[2-D1B5-SO-15-2425]%[VN-MT-7, Đoàn Thị Lý]%[2D1H4-1]
 Đồ thị hàm số $y=\dfrac{x^2-3 x-4}{x^2-16}$ có bao nhiêu đường tiệm cận đứng? 
 \choice 
 {\True $1$}
 {$0$}
 {$2$}
 {$3$}
 \loigiai{
 Tập xác định $\mathscr{D}=\mathbb{R} \setminus\big\{- 4;4\big\}$. Ta có
 \begin{itemize}
 \item $\lim\limits _{x \to (-4)^-} y=\lim\limits _{x \to (-4)^-} \dfrac{x^2-3 x-4}{x^2-16}=\lim\limits _{x \to (-4)^-} \dfrac{(x+1)(x-4)}{(x+4)(x-4)}=\lim\limits _{x \to (-4)^-} \dfrac{x+1}{x+4}=+\infty$, suy ra $x=-4$ là tiệm cận đứng của đồ thị hàm số.
 \item $\lim\limits_{x \to 4} y=\lim\limits _{x \to 4} \dfrac{x^2-3 x-4}{x^2-16}=\lim\limits _{x \to 4} \dfrac{(x+1)(x-4)}{(x+4)(x-4)}=\lim\limits _{x \to 4} \dfrac{x+1}{x+4}=\dfrac{5}{8}$, suy ra $x=4$ không là tiệm cận đứng của đồ thị hàm số.
 \end{itemize}
 Vậy đồ thị hàm số có $1$ đường tiệm cận đứng.
 }
\end{ex}

\begin{ex}%[2-D1B5-SO-15-2425]%[VN-MT-7, Đoàn Thị Lý]%[2D1H5-4] 
 Đồ thị của hàm số $y=x^3-3 x+2$ cắt trục tung tại điểm có tung độ bằng 
 \choice
 {$0$}
 {$1$}
 {\True $2$}
 {$3$}
 \loigiai{
 Gọi $M\left(x_0;y_0\right)$ là giao điểm của đồ thị hàm số với trục tung. Ta có $x_0=0 \Rightarrow y_0=2$.
 }
\end{ex}

\begin{ex}%[2-D1B5-SO-15-2425]%[VN-MT-7, Đoàn Thị Lý]%[2D1H5-4]
 Số giao điểm của đồ thị hàm số $y=x^3-3 x+1$ và trục hoành là \choice 
 {\True $3$}
 {$0$}
 {$2$}
 {$1$}
 \loigiai{
 Tập xác định $\mathscr{D}=\mathbb{R}$.\\
 Ta có $y'=3 x^2-3=3\left(x^2-1\right)$, $y'=0 \Leftrightarrow x= \pm 1$.\\
 Bảng biến thiên của hàm số
 \begin{center}
 \begin{tikzpicture}
 \tkzTabInit[nocadre=true,lgt=1.2,espcl=3, deltacl=0.5]
 {$x$/0.7,$y'$/0.7,$y$/2}
 {$-\infty$,$-1$,$1$,$+\infty$}
 \tkzTabLine{,-,0,+,0,-,}
 \tkzTabVar{-/$-\infty$,+/$3$,-/$-1$,+/$+\infty$}
 \end{tikzpicture}
 \end{center}
 Từ bảng biến thiên ta thấy đồ thị hàm số cắt trục hoành tại $3$ điểm phân biệt.
 }
\end{ex}
\Closesolutionfile{ans}

\TNTF
\Opensolutionfile{ans}[ans/ans\currfilebase-Phan-II]
\begin{ex}%[2-D1B5-SO-15-2425]%[VN-MT-7, Đoàn Thị Lý]%[2D1N3-1]
 \immini{
 Cho hàm số $y=f(x)$ liên tục trên $\mathbb{R}$ và có đồ thị như hình vẽ bên. 
 \choiceTF
 {\True Hàm số đồng biến trên khoảng $(0;2)$}
 {Hàm số đạt cực đại tại $x=0$}
 {\True Giá trị nhỏ nhất của hàm số trên $[-1;1]$ bằng $-4$} 
 {Hàm số $g(x)=f(3-x)$ nghịch biến trên $(2;5)$}
 }
 {
 \begin{tikzpicture} [scale=.8, font=\footnotesize, line join=round, line cap=round, >=stealth]
 \draw[->] (-1.8,0)--(-1,0) node[below left]{$-1$}--(0,0) node[below left]{$O$}--(2,0) node[above]{$2$}--(3.2,0) node[below]{$x$};
 \draw[->] (0,-4.8)--(0,-4) node[below left]{$-4$}--(0,1.2) node[right]{$y$};
 \draw[samples=100,domain=-1.1:3.05,smooth] plot (\x, {-(\x)^3+3*(\x)^2-4});
 \foreach \x in {(0,0),(-1,0),(0,-4),(2,0)}
 \fill \x circle (1pt);
 \end{tikzpicture}
 }
 \loigiai{
 \begin{itemchoice}
 \itemch \textbf{Đúng}.\\
 Hàm số đồng biến trên khoảng $(0;2)$.
 \itemch \textbf{Sai}.\\
 Hàm số đạt cực đại tại $x=2$.
 \itemch \textbf{Đúng}.\\
 Ta có $\min\limits_{[-1; 1]} f(x)=-4$.
 \itemch \textbf{Sai}.\\
 Xét hàm số $g(x)=f(3-x)$. Vì $f(x)$ liên tục trên $\mathbb{R}$ nên $g(x)$ liên tục trên $\mathbb{R}$.\\
 Từ đồ thị của hàm số ta có bảng xét dấu của $f'(x)$ như sau:
 \begin{center}
 \begin{tikzpicture}
 \tkzTabInit[nocadre=true,lgt=1.2,espcl=2.5,deltacl=0.5]
 {$x$ /.7, $f'(x)$ /.7}
 {$-\infty$,$0$,$2$,$+\infty$}
 \tkzTabLine{ ,-,0,+,0, -, }
 \end{tikzpicture}
 \end{center}
 Ta có $g'(x)=(3-x)' f'(3-x)=-f'(3-x)$.\\
 Cho $g'(x)=0 \Leftrightarrow-f'(3-x)=0 \Leftrightarrow\hoac{&3-x=0 \\ &3-x=2} \Leftrightarrow\hoac{&x=3 \\& x=1.}$\\
 Từ bảng xét dấu của $f'(x)$ suy ra được bảng xét dấu của $g'(x)$ như sau:
 \begin{center}
 \begin{tikzpicture}
 \tkzTabInit[nocadre=true,lgt=1.2,espcl=2.5,deltacl=0.5]
 {$x$ /.7, $g'(x)$ /.7}
 {$-\infty$,$1$,$3$,$+\infty$}
 \tkzTabLine{ ,+,0,-,0, +, }
 \end{tikzpicture}
 \end{center}
 Vậy hàm số $g(x)$ không nghịch biến trên $(2;5)$.
 \end{itemchoice}
 }
\end{ex}

\begin{ex}%[2-D1B5-SO-15-2425]%[VN-MT-7, Đoàn Thị Lý]%[2D1H2-2]
 Cho hàm số $y=f(x)$ liên tục trên $\mathbb{R}$ và có bảng xét dấu của đạo hàm $f'(x)$ như hình sau: 
 \begin{center}
 \begin{tikzpicture}
 \tkzTabInit[nocadre=true,lgt=1.2,espcl=2.5,deltacl=0.5]
 {$x$ /.7, $f'(x)$ /.7}
 {$-\infty$,$-2$,$0$,$1$,$2$,$+\infty$}
 \tkzTabLine{ ,+,0,-,d,+,0,-,0,-, }
 \end{tikzpicture}
 \end{center}
 \choiceTF
 {Hàm số đã cho đồng biến trên các khoảng $(-\infty;0)$ và $(0;1)$}
 {\True Hàm số đã cho nghịch biến trên khoảng $(3 ;+\infty)$}
 {Hàm số đã cho có $2$ điểm cực trị}
 {\True Hàm số đã cho đạt cực tiểu tại điểm $x=0$}
 \loigiai{
 Dựa vào bảng xét dấu của $f'(x)$ ta có
 \begin{itemchoice}
 \itemch \textbf{Sai}.\\
 Hàm số không đồng biến trên khoảng $(-\infty;0)$.
 \itemch \textbf{Đúng}.\\
 Hàm số nghịch biến trên khoảng $(1;+\infty)$ chứa khoảng $(3;+\infty)$.
 \itemch \textbf{Sai}.\\
 Hàm số đã cho liên tục trên $\mathbb{R}$ và $f'(x)$ đổi dấu ba lần nên hàm số đã cho có $3$ điểm cực trị.
 \itemch \textbf{Đúng}.\\
 Hàm số đã cho liên tục trên $\mathbb{R}$ và tại điểm $x=0$, $f'(x)$ đổi dấu từ âm sang dương nên hàm số đã cho đạt cực tiểu tại $x=0$.
 \end{itemchoice} 
 }
\end{ex}

\begin{ex}%[2-D1B5-SO-15-2425]%[VN-MT-7, Đoàn Thị Lý]%[2D1H3-1]
 Cho hàm số $y=f(x)=\dfrac{3 x-1}{x-3}$. Gọi $M$ và $m$ lần lượt là giá trị lớn nhất và nhỏ nhất của hàm số $y=f(x)$ trên đoạn $[0;2]$.
 \choiceTF
 {Hàm số đã cho đồng biến trên khoảng $(0;2)$} 
 {$M=f(1)=\dfrac{1}{3}$}
 {\True $m=f(2)=-5$}
 {\True Có $5$ giá trị nguyên dương bé hơn $10$ của $t$ sao cho $f(x)\le t,\,\forall x\in [0;2]$}
 \loigiai{
 Hàm số đã cho liên tục trên đoạn $[0;2]$.\\
 Ta có $y'=-\dfrac{8}{(x-3)^2}<0$, $ \forall x \neq 3$, suy ra hàm số nghịch biến trên đoạn $[0 ; 2]$.\\
 Vậy $M=\max\limits_{[0 ; 2]}f(x)=f(0)=\dfrac{1}{3}$ và $m=\min\limits_{[0 ; 2]}f(x)=f(2)=-5$.
 \begin{itemchoice}
 \itemch \textbf{Sai}.\\
 Vì hàm số nghịch biến trên các khoảng $\left( -\infty;3\right)$ và $\left(3;+\infty\right)$ nên hàm số nghịch biến trên $(0;2)~\subset~ \left( -\infty;3\right)$.
 \itemch \textbf{Sai}.\\
 Vì $M=\max\limits_{[0;2]}f(x)=f(0)=\dfrac{1}{3}$.
 \itemch \textbf{Đúng}.\\
 Vì $m=\min\limits_{[0;2]}f(x)=f(2)=-5$.
 \itemch \textbf{Sai}.\\
 Ta có $f(x)\le t,\,\forall x\in [0;2]\Leftrightarrow \max\limits_{[0;2]}f(x)\le t\Leftrightarrow t\ge \dfrac{1}{3}$.\\
 Vì $t$ nguyên dương và bé hơn $6$ nên $t\in\left\lbrace 1;2;3;4;5\right\rbrace$.\\ Vậy có $5$ giá trị của $t$ thỏa mãn.
 \end{itemchoice} 
 }
\end{ex}

\begin{ex}%[2-D1B5-SO-15-2425]%[VN-MT-7, Đoàn Thị Lý]%[2D1N4-1]
 Cho hàm số $y=f(x)$ xác định và liên tục trên $\mathbb{R}\setminus\{-1\}$, có bảng biến thiên như sau:
 \begin{center}
 \begin{tikzpicture}
 \tkzTabInit[nocadre=true, lgt=1, espcl=4, deltacl=0.5]
 {$x$/0.7,$y’$/0.7,$y$/2}
 {$-\infty$,$-1$,$+\infty$}
 \tkzTabLine{,+,d,+,}
 \tkzTabVar{-/$-2$,+D-/$+\infty$/$-\infty$,+/$-2$}
 \end{tikzpicture}
 \end{center}
 \choiceTF
 {Hàm số có $2$ cực trị}
 {\True Đồ thị hàm số cắt đường thẳng $y=1$ tại đúng $1$ điểm}
 {Hàm số đồng biến trên $\left(-2;3\right)$}
 {\True Đồ thị hàm số có tiệm cận đứng $x=-1$ và tiệm cận ngang $y=-2$}
 \loigiai{ 
 Từ bảng biến thiên, ta có
 \begin{itemchoice}
 \itemch \textbf{Sai}.\\
 Vì $f'(x)$ không đổi dấu trên $\mathbb{R}\setminus\{-1\}$ nên hàm số không có cực trị.
 \itemch \textbf{Đúng}.\\
 Đồ thị hàm số cắt đường thẳng $y=1$ tại đúng $1$ điểm.
 \itemch \textbf{Sai}.\\
 Vì hàm số không xác định trên $\left(-2;3\right)$ nên hàm số không đồng biến trên $\left(-2;3\right)$.
 \itemch \textbf{Đúng}.
 \begin{itemize}
 \item $\lim\limits _{x \to(-1)^-} f(x)=+\infty$ và $\lim\limits _{x \to(-1)^+} f(x)=-\infty$, suy ra $x=-1$ là tiệm cận đứng.
 \item $\lim\limits _{x \to-\infty} y=-2$ và $\lim\limits_{x\to \infty} y=-2$, suy ra $y=-2$ là tiệm cận ngang. 
 \end{itemize}
 Vậy đồ thị hàm số có tiệm cận đứng là $x=-1$ và tiệm cận ngang là $y=-2$.
 \end{itemchoice}
 }
\end{ex}
\Closesolutionfile{ans}

\TNSA
\Opensolutionfile{ans}[ans/ans\currfilebase-Phan-III]
\begin{ex}%[2-D1B5-SO-15-2425]%[VN-MT-7, Đoàn Thị Lý]%[2D1H2-1]
 Cho hàm số $y=\dfrac{2 x^2+5 x+4}{x+2}$. Độ dài của đoạn thẳng nối hai điểm cực trị của đồ thị hàm số bằng bao nhiêu (kết quả làm tròn đến hàng phần trăm)?
 \shortans{8{,}24}
 \loigiai{ 
 Điều kiện $x \neq-2$.\\
 Ta có $y'=\dfrac{2 x^2+8 x+6}{(x+2)^2}$ $(x \neq-2)$.
 Cho $y'=0 \Rightarrow 2 x^2+8 x+6=0 \Rightarrow\hoac{&x=-1 \\& x=-3.}$\\
 Với $x=-1\Rightarrow y=1$.\\
 Với $x=-3\Rightarrow y=-7$.\\
 Đồ thị hàm số có hai điểm cực trị $A(-1;1)$ và $B(-3;-7)$. Suy ra $AB=2 \sqrt{17}\approx 8{,}24$.
 }
\end{ex}

\begin{ex}%[2-D1B5-SO-15-2425]%[VN-MT-7, Đoàn Thị Lý]%[2D1H2-2]
 \immini{
 Cho hàm số $y=f(x)$ có đồ thị $y=f'(x)$ như hình vẽ bên. Hàm số $y=f(x)$ có bao nhiêu điểm cực trị?
 }
 {
 \begin{tikzpicture} [scale=.8, font=\footnotesize, line join=round, line cap=round, >=stealth]
\def \xmin{-3}\def \xmax{4}\def \ymin{-1.3}\def \ymax{3.5} 
\draw[->] (\xmin,0)--(\xmax,0) node[shift=(-110:0.2)] {$x$};
\draw[->] (0,\ymin)--(0,\ymax) node[shift=(-150:0.2)] {$y$};
\fill (0,0) circle(1pt) node[shift=(-135:0.25)]{$O$}
 (-1,0) circle(1pt) node[shift=(-90:0.2)]{$-1$}
 (2,0) circle(1pt) node[shift=(-90:0.2)]{$2$};
\clip (\xmin,\ymin) rectangle (\xmax,\ymax); 
\draw[yscale=0.2, smooth,samples=100,domain=\xmin:\xmax] plot(\x, {(\x)^3-1.5*(\x)^2-6*(\x)+10});
 \draw[dashed] (-1,0)--(-1,2.7)--(0,2.7);
 \draw (3,2.1) node[rotate=73]{$y=f'(x)$};
 \end{tikzpicture}
 }
 \shortans{1}
 \loigiai{
 Số điểm cực trị của hàm số đã cho là $1$. Vì dựa vào đồ thị của $f'(x)$ ta thấy $f'(x)$ đổi dấu một lần từ âm sang dương nên hàm số đã cho có một cực trị (một cực tiểu).
 }
\end{ex}

\begin{ex}%[2-D1B5-SO-15-2425]%[VN-MT-7, Đoàn Thị Lý]%[2D1H4-1]
 Tiệm cận xiên của đồ thị hàm số $y=f(x)=\dfrac{x^2-x+1}{x-1}$ có dạng $y=a x+b,(a, b \in \mathbb{Z})$. Tính giá trị biểu thức $P=5 a+2024 b$.
 \shortans{5}
 \loigiai{
 Giả sử hàm số có đồ thị là $(C)$. Ta có $y=\dfrac{x^2-x+1}{x-1}=x+\dfrac{1}{x-1}$. Từ đó có
 \[\lim\limits _{x \to \pm \infty}[f(x)-x]=\lim\limits _{x \to \pm \infty}\left[\dfrac{x^2-x+1}{x-1}-x\right]=\lim\limits _{x \to \pm \infty} \dfrac{1}{x-1}=\lim\limits _{x \to \pm \infty} \dfrac{\dfrac{1}{x}}{1-\dfrac{1}{x}}=0.\] 
 Suy ra $(C)$ có tiệm cận xiên là đường thẳng $y=x$.\\
 Vậy $a=1$, $b=0$, $P=5a+2024 b=5\cdot 1+2024\cdot 0=5$.
 }
\end{ex}

\begin{ex}%[2-D1B5-SO-15-2425]%[VN-MT-7, Đoàn Thị Lý]%[2D1V1-3]
 Cho hàm số $y=f(x)$ có đạo hàm trên $\mathbb{R}$ và bảng xét dấu đạo hàm như hình vẽ sau:
 \begin{center}
 \begin{tikzpicture}
 \tkzTabInit[nocadre=true,lgt=1.2,espcl=2.5,deltacl=0.5]
 {$x$ /.7, $f'(x)$ /.7}
 {$-\infty$,$-10$,$-2$,$3$,$8$,$+\infty$}
 \tkzTabLine{ ,+,0,+,0,-,0,-,0,+, }
 \end{tikzpicture}
 \end{center} 
 Tìm $m$ để hàm số $y=f\left( x^3+4x+m \right)$ nghịch biến trên khoảng $(-1;1)$?
 \shortans{3}
 \loigiai{
 Đặt $t=x^3+4x+m\Rightarrow t'=3x^2+4>0,\,\forall x\in (-1;1)$ nên $t$ đồng biến trên $(-1;1)$ và $t\in \left(m-5;m+5\right)$.\\
 Yêu cầu bài toán trở thành tìm $m$ để hàm số $f(t)$ nghịch biến trên khoảng $\left( m-5;m+5 \right)$.\\
 Dựa vào bảng xét dấu ta được $\heva{& m-5\ge -2 \\ 
 & m+5\le 8}\Leftrightarrow \heva{& m\ge 3 \\ 
 & m\le 3}\Leftrightarrow m=3$.
 }
\end{ex}

\begin{ex}%[2-D1B5-SO-15-2425]%[VN-MT-7, Đoàn Thị Lý]%[2D1V1-3]
 \immini{
 Cho hàm số $y=f(x)$ có đạo hàm liên tục trên $\mathbb{R}$ và có đồ thị $y=f'(x)$ như hình vẽ bên. Đặt $g(x)=f(x-m)-\dfrac{1}{2}{(x-m-1)^2}+2019$, với $m$ là tham số thực. Gọi $S$ là tập hợp các giá trị nguyên dương của $m$ để hàm số $y=g(x)$ đồng biến trên khoảng $(5;6)$. Tính tổng tất cả các phần tử thuộc $S$.
 }
 {
 \begin{tikzpicture} [scale=.8, font=\footnotesize, line join=round, line cap=round, >=stealth]
 \draw[->] (-1.8,0)--(0,0) node[below left]{$O$}--(4,0) node[below]{$x$};
 \draw[->] (0,-3)--(0,3) node[right]{$y$};
 \draw[samples=100,domain=-1.1:3.1,smooth] plot (\x, {(\x)^3-3*(\x)^2+2});
 \draw[dashed] (-1,0) node[below left]{$-1$}--(-1,-2)--(0,-2)node[below left]{$-2$}--(2,-2)--(2,0)node[above]{$2$} (0,2)node[above left]{$2$}--(3,2)--(3,0) node[below]{$3$};
 \draw (1,0) node[below] {$1$};
 \foreach \i in {(0,0),(3,2),(2,-2),(0,2),(-1,-2),(-1,0),(2,0),(0,-2),(1,0),(3,0)}\fill \i circle (1pt);
 \draw (3.3,2.2) node[rotate=82]{$y=f'(x)$};
 \end{tikzpicture}
}
 \shortans{14}
 \loigiai{
 Ta có 
 $g'(x)=f'(x-m)-(x-m-1)$.\\
 Xét phương trình $g'(x)=0\Leftrightarrow f'(x-m)-(x-m-1)=0\qquad (1)$.\\
 Đặt $x-m=t$, phương trình (1) trở thành $f'(t)-(t-1)=0\Leftrightarrow f'(t)=t-1\qquad(2)$.\\
 Nghiệm của phương trình $(2)$ là hoành độ giao điểm của hai đồ thị hàm số $y=f'(t)$ và $y=t-1$.\\
 Ta có đồ thị các hàm số $y=f'( t)$ và $y=t-1$ như sau: 
 \begin{center}
 \begin{tikzpicture} [scale=.8, font=\footnotesize, line join=round, line cap=round, >=stealth]
 \draw[->] (-1.8,0)--(0,0) node[below left]{$O$}--(4,0) node[below]{$x$};
 \draw[->] (0,-3)--(0,3) node[right]{$y$};
 \draw[samples=100,domain=-1.1:3.1,smooth] plot (\x, {(\x)^3-3*(\x)^2+2});
 \draw[samples=100,domain=-1.4:3.2,smooth] plot (\x, {(\x)-1});
 \draw[dashed] (-1,0) node[below left]{$-1$}--(-1,-2)--(0,-2)node[below left]{$-2$}--(2,-2)--(2,0)node[above]{$2$} (0,2)node[above left]{$2$}--(3,2)--(3,0) node[below]{$3$};
 \draw (1,0) node[below]{$1$};
 \foreach \i in {(0,0),(1,0),(3,2),(2,-2),(0,2),(-1,-2),(-1,0),(2,0),(0,-2),(1,0),(3,0)}\fill \i circle (1pt);
 \draw (3.3,2.2) node[rotate=82]{$y=f'(x)$};
 \end{tikzpicture}
 \end{center} 
 Căn cứ đồ thị các hàm số ta có phương trình (2) có nghiệm là $\hoac{& t=-1 \\ 
 & t=1 \\ 
 & t=3 }
 \Rightarrow\hoac{&x=m-1 \\ 
 & x=m+1 \\ 
 & x=m+3.}$\\
 Ta có bảng biến thiên của $y=g(x)$ như sau:
 \begin{center}
 \begin{tikzpicture}
 \tkzTabInit[nocadre=true,lgt=1.2,espcl=2.5,deltacl=0.5]
 {$x$/0.7,$g’(x)$/0.7,$g(x)$/2}
 {$-\infty$,$m-1$,$m+1$,$m+3$,$+\infty$}
 \tkzTabLine{,-,0,+,0,-,0,+,}
 \tkzTabVar{+/$+\infty$,-/,+/,-/,+/$+\infty$}
 \end{tikzpicture}
 \end{center}
 Để hàm số $y=g(x)$ đồng biến trên khoảng $(5;6)$ cần 
 $\hoac{
 & \heva{
 & m-1\le 5 \\ 
 & m+1\ge 6} \\ 
 & m+3\le 5 \\ }\Leftrightarrow \hoac{
 & 5\le m\le 6 \\ 
 & m\le 2.}$\\
 Vì $m\in \mathbb{N}^*$ nên $m\in \big\{1;2;5;6\big\}$. Suy 
 $S=14$.
 }
\end{ex}

\begin{ex}%[2-D1B5-SO-15-2425]%[VN-MT-7, Đoàn Thị Lý]%[2D1V3-6]
 Một hộp sữa dung tích $1 \ell$ (lít) có dạng hình hộp chữ nhật với đáy là hình vuông cạnh bằng $x$ cm và chiều cao $h$ cm. Tìm giá trị của $x$ để diện tích toàn phần của hình hộp là nhỏ nhất.
 \shortans[]{10}
 \loigiai{
 \begin{center}
 \begin{tikzpicture}
 \def\a{3}
 \def\b{2}
 \def\g{30}
 \def\h{2}
 \path
 (0:0) coordinate (A)--++(\g:\b) coordinate (B)--++(0:\a) coordinate (C)--++(\g-180:\b) coordinate (D)
 \foreach \x in {A,B,C,D}{
 ($(\x)+(90:\h)$) coordinate (\x')}
 ;
 \foreach \x/\g in {A/180,B/150,C/0,D/-30,A'/180,B'/150,C'/0,D'/-30}
 \fill[black](\x) circle (1pt);
 \draw[dashed] (B')--(B)--(A)
 (B)--(C);
 \draw
 (A)--(D)--(D')--(A')--cycle
 (A')--(B')--(C')--(D')
 (D)--(C)--(C');
 \draw (1.6,0.15) node {$x$} (4,.4) node {$x$} (5,2) node {$h$}
 ;
 \end{tikzpicture} 
 \end{center}
 Thể tích của hộp sữa là $V=x^2h$ $\mathrm{cm}^3$.\\
 Theo bài ra, ta có $V=1\mathrm{\ell}=1000 \mathrm{\,cm}^3$. Suy ra $x^2h=1000\Rightarrow h=\dfrac{1000}{x^2}$.\\
 Ta có diện tích toàn phần của hộp sữa là \[S_{\rm{tp}}=S_{\rm{xq}}+S_{\rm{d}}=4hx+2x^2=4\cdot\dfrac{1000}{x^2}\cdot x+2x^2=2x^2+\dfrac{4000}{x}.\]
 Đặt $y=f(x)=2x^2+\dfrac{4000}{x}$, suy ra $f'(x)=4x-\dfrac{4000}{x^2}$.\\
 Xét $f'(x)=0\Leftrightarrow 4x-\dfrac{4000}{x^2}=0\Leftrightarrow 4x^3-4000=0\Leftrightarrow x=10$.\\
 Ta có bảng biến thiên của hàm số $y=f(x)$ như sau:
 \begin{center}
 \begin{tikzpicture}
 \tkzTabInit[nocadre=true,lgt=1.2,espcl=2.5,deltacl=0.5]
 {$x$/0.7,$f’(x)$/0.7,$f(x)$/2}
 {$0$,$10$,$+\infty$}
 \tkzTabLine{d,-,0,+,}
 \tkzTabVar{-D+//$+\infty$,-/$600$,+/$+\infty$}
 \end{tikzpicture}
 \end{center}
 Vậy để hộp sữa có diện tích toàn phần nhỏ nhất thì $x=10$.
 } 
\end{ex}
\Closesolutionfile{ans}
% \begin{indapan}
% 	{ans/ans\currfilebase}
% \end{indapan}


% \begin{name}
{Biên soạn: Dương Phước Sang \\ Phản biện: Dương Công Tạo}
{Đề ôn tập chương I}
\end{name}

\caulc
\Opensolutionfile{ans}[ans/ans\currfilebase-Phan-I]
\begin{ex}%[2-D1B5-SO-16-2425]%[VN-MT-7, Dương Phước Sang]%[2D1H1-1]
Hàm số $y=x^3-3x^2+2$ đồng biến trên khoảng nào dưới đây?
\choice
{\True $(-2;0)$}
{$(0;+\infty)$}
{$(-\infty;2)$}
{$(0;2)$}
\loigiai{
Ta có $y'=3x^2-6x$.\\
$y' \geq 0 \Leftrightarrow 3x^2-6x \geq 0 \Leftrightarrow \hoac{&x \geq 2\\&x \leq 0.}$\\
Suy ra hàm số đồng biến trên các khoảng $(-\infty;0)$ và $(2;+\infty)$ nên đồng biến trên khoảng $(-2;0)$.
}
\end{ex}

\begin{ex}%[2-D1B5-SO-16-2425]%[VN-MT-7, Dương Phước Sang]%[2D1H2-1]
Cho hàm số $y=27x^3+108x^2-81x+189$. Điểm cực tiểu của hàm số là
\choice
{$-3$}
{\True $\dfrac{1}{3}$}
{$175$}
{$675$}
\loigiai{
Ta có $y'=81x^2+216x-81$.\\
$y'=0 \Leftrightarrow \hoac{&x=\dfrac{1}{3}\\&x=-3.}$
\begin{center}
\begin{tikzpicture}
\tkzTabInit[nocadre=true,lgt=1.2,espcl=2.5,deltacl=0.6]
{$x$/0.7, $y'$/0.6, $y$/2}
{$-\infty$,$-3$,$\tfrac{1}{3}$,$+\infty$}
\tkzTabLine{,+,0,-,0,+,}
\tkzTabVar{-/,+/,-/,+/}
\end{tikzpicture}
\end{center}
Vậy điểm cực tiểu của hàm số là $x_{_\text{CT}}=\dfrac{1}{3}$.
}
\end{ex}

\begin{ex}%[2-D1B5-SO-16-2425]%[VN-MT-7, Dương Phước Sang]%[2D1H3-1]
Giá trị lớn nhất của hàm số $f(x)=x^3-8x^2+16x-9$ trên đoạn $[1;3]$ là
\choice
{$\max\limits_{[1; 3]} f(x)=0$}
{\True $\max\limits_{[1; 3]} f(x)=\dfrac{13}{27}$}
{$\max\limits_{[1; 3]} f(x)=-6$}
{$\max\limits_{[1; 3]} f(x)=5$}
\loigiai{
Hàm số $f(x)$ liên tục trên $[1;3]$.\\
Ta có $f'(x)=3x^2-16x+16$; $f'(x)=0 \Leftrightarrow \hoac{&x=4 \notin (1;3)\\&x=\dfrac{4}{3}\in (1;3).}$\\
$f(1)=0$; $f\left( \dfrac{4}{3} \right)=\dfrac{13}{27}$; $f(3)=-6$.\\
Do đó $\max\limits_{x \in [1;3]} f(x)=f\left( \dfrac{4}{3} \right)=\dfrac{13}{27}$.
}
\end{ex}

\begin{ex}%[2-D1B5-SO-16-2425]%[VN-MT-7, Dương Phước Sang]%[2D1H3-1]
Giá trị lớn nhất của hàm số $f(x)=x^4-4x^2+1$ trên đoạn $[1;3]$ bằng
\choice
{\True $46$}
{$64$}
{$3$}
{$\sqrt{2}$}
\loigiai{
Ta có hàm số $f(x)=x^4-4x^2+1$ liên tục trên đoạn $[1;3]$.\\
$f'(x)=4x^3-8x$.\\
$f'(x)=0 \Leftrightarrow 4x^3-8x=0 \Leftrightarrow \hoac{&x=0 \notin (1;3)\\&x=\sqrt{2} \in (1;3)\\&x=-\sqrt{2} \notin (1;3).}$\\
$f(1)=-2$; $f\left( \sqrt{2} \right)=-3$; $f(3)=46$.\\
Vậy giá trị lớn nhất của hàm đã cho trên đoạn $[1;3]$ bằng $46$.
}
\end{ex}

\begin{ex}%[2-D1B5-SO-16-2425]%[VN-MT-7, Dương Phước Sang]%[2D1N4-1]
Tiệm cận ngang của đồ thị hàm số $y=\dfrac{2x+3}{x-1}$ là
\choice
{$y=1$}
{\True $y=2$}
{$x=1$}
{$x=2$}
\loigiai{
Tập xác định của hàm số là $\mathscr{D}=\mathbb{R} \setminus \{1\}$.\\
Ta có 
$\lim\limits_{x \to \pm\infty} y=\lim\limits_{x \to \pm\infty} \dfrac{2x+3}{x-1}=\lim\limits_{x \to \pm\infty} \dfrac{2+\dfrac{3}{x}}{1-\dfrac{1}{x}}=2$.\\
Vậy đồ thị của hàm số có tiệm cận ngang là đường thẳng $y=2$.
}
\end{ex}

\begin{ex}%[2-D1B5-SO-16-2425]%[VN-MT-7, Dương Phước Sang]%[2D1N4-1]
Tiệm cận đứng của đồ thị hàm số $y=\dfrac{2x+1}{x-1}$ là
\choice
{\True $x=1$}
{$y=2$}
{$x=2$}
{$x=-1$}
\loigiai{
Tập xác định của hàm số $\mathscr{D}=\mathbb{R} \setminus \{1\}$.\\
Ta có 
$\lim\limits_{x \to 1^+} y=\lim\limits_{x \to 1^+} \dfrac{2x+1}{x-1}=\lim\limits_{x \to 1^+} \left(2+\dfrac{3}{x-1}\right)=+\infty$.\\
Vậy đồ thị của hàm số có tiệm cận đứng là đường thẳng $x=1$.
}
\end{ex}

\begin{ex}%[2-D1B5-SO-16-2425]%[VN-MT-7, Dương Phước Sang]%[2D1H4-1]
Đường thẳng nào sau đây là tiệm cận xiên của đồ thị hàm số $y=\dfrac{2x^2-3x+1}{x+2}$?
\choice
{$y=2x$}
{$y=2$}
{\True $y=2x-7$}
{$x=-2$}
\loigiai{
Ta có $y=f(x)=\dfrac{2x^2-3x+1}{x+2}=2x-7+\dfrac{15}{x+2}$.\\
$\lim\limits_{x \to \pm\infty} [f(x)-(2x-7)]=\lim\limits_{x \to \pm\infty} \dfrac{15}{x+2}=0$.\\
Vậy đồ thị hàm số có tiệm cận xiên là $y=2x-7$.
}
\end{ex}

\begin{ex}%[2-D1B5-SO-16-2425]%[VN-MT-7, Dương Phước Sang]%[2D1N1-2]
Cho hàm số $y=f(x)$ có bảng biến thiên như sau:
\begin{center}
\begin{tikzpicture}
\tkzTabInit[nocadre=true,lgt=1.2,espcl=2.5,deltacl=0.6]
{$x$/0.6, $y'$/0.6, $y$/2}
{$-\infty$,$-2$,$0$,$+\infty$}
\tkzTabLine{,+,0,-,0,+,}
\tkzTabVar{-/$-\infty$,+/$1$,-/$-3$,+/$+\infty$}
\end{tikzpicture} 
\end{center}
Hàm số $y=f(x)$ nghịch biến trên khoảng nào dưới đây?
\choice
{$(-\infty;-2)$}
{$(0;+\infty)$}
{$(-3;1)$}
{\True $(-2;0)$}
\loigiai{
Dựa vào bảng biến thiên, ta thấy $f'(x)<0 \Leftrightarrow x \in (-2;0)$ nên hàm số nghịch biến trên khoảng $(-2;0)$.
}
\end{ex}

\begin{ex}%[2-D1B5-SO-16-2425]%[VN-MT-7, Dương Phước Sang]%[2D1H5-1]
Cho bảng biến thiên của hàm số $y=f(x)$ như sau:
\begin{center}
\begin{tikzpicture}
\tikzset{double style/.append style={double distance=1.5pt}}
\tkzTabInit[nocadre=true,lgt=1.2,espcl=4,deltacl=0.6]
{$x$/0.6,$y'$/0.6,$y$/2}
{$-\infty$,$1$,$+\infty$}
\tkzTabLine{,-,d,-,}
\tkzTabVar{+/$1$,-D+/$-\infty$/$+\infty$,-/$1$}
\end{tikzpicture} 
\end{center}
Hỏi đây là bảng biến thiên của hàm số nào trong các hàm số sau?
\choice
{$y=\dfrac{x-3}{x-1}$}
{$y=\dfrac{-x+2}{x-1}$}
{$y=\dfrac{x+2}{x+1}$}
{\True $y=\dfrac{x+2}{x-1}$}
\loigiai{
Bảng biến thiên được cung cấp có đặc điểm:
\begin{itemize}
\item Đồ thị hàm số có đường tiệm cận đứng là $x=1$, loại $y=\dfrac{x+2}{x+1}$.
\item Đồ thị hàm số có đường tiệm cận ngang là $y=1$, loại $y=\dfrac{-x+2}{x-1}$.
\item $y'<0,\,\forall x \ne 1$, trong khi $\left(\dfrac{x-3}{x-1}\right)'=\dfrac{2}{(x-1)^2}>0,\,\forall x \neq 1$, loại $y=\dfrac{x-3}{x-1}$.
\end{itemize}
Chỉ có hàm số $y=\dfrac{x+2}{x-1}$ thỏa mãn các đặc điểm trên.
}
\end{ex}

\begin{ex}%[2-D1B5-SO-16-2425]%[VN-MT-7, Dương Phước Sang]%[2D1H1-1]
Cho hàm số $y=\dfrac{x^2-2x}{1-x}$. Khẳng định nào sau đây đúng?
\choice
{Hàm số đồng biến trên $\mathbb{R}$}
{\True Hàm số nghịch biến trên các khoảng $(-\infty;1)$ và $(1;+\infty)$}
{Hàm số nghịch biến trên $\mathbb{R}$}
{Hàm số đồng biến trên các khoảng $(-\infty;1)$ và $(1;+\infty)$}
\loigiai{
Tập xác định $\mathscr{D}=\mathbb{R} \setminus \big\{1\big\}$.\\
$y'=\dfrac{-x^2+2x-2}{(1-x)^2}=\dfrac{-(x-1)^2-1}{(1-x)^2}<0,\,\forall x \in \mathscr{D}$.\\
Vậy hàm số nghịch biến trên các khoảng $(-\infty;1)$ và $(1;+\infty)$.
}
\end{ex}

\begin{ex}%[2-D1B5-SO-16-2425]%[VN-MT-7, Dương Phước Sang]%[1D7H2-8]
Cho chuyển động được xác định bởi phương trình $s(t)=3t^3+4t^2-t$, trong đó $t$ được tính bằng giây (s) và $s(t)$ được tính bằng mét. Vận tốc của chuyển động khi $t=4$\,s bằng
\choice
{\True $175$ m/s}
{$41$ m/s}
{$176$ m/s}
{$20$ m/s}
\loigiai{
Ta có $v(t)=s'(t)=9t^2+8t-1$.\\
Vận tốc của chuyển động khi $t=4$\,s bằng $v(4)=9 \cdot 4^2+8 \cdot 4-1=175$ m/s.
}
\end{ex}

\begin{ex}%[2-D1B5-SO-16-2425]%[VN-MT-7, Dương Phước Sang]%[2D1H2-1]
Cho hàm số $y=f(x)$ có đạo hàm $f'(x)=x^2(x-1)$ với mọi số thực $x$. Số điểm cực tiểu của hàm số $f(x)$ là
\choice
{$0$}
{\True $1$}
{$2$}
{$3$}
\loigiai{
Ta có $f'(x)=x^2(x-1)=0 \Leftrightarrow \hoac{&x=0&&\text{(nghiệm kép)}\\&x=1.}$\\
Bảng biến thiên:
\begin{center}
\begin{tikzpicture}
\tkzTabInit[nocadre=true,lgt=1.2,espcl=2.5,deltacl=0.6]
{$x$/0.6, $f'(x)$/0.6, $f(x)$/2}
{$-\infty$,$0$,$1$,$+\infty$}
\tkzTabLine{,-,0,-,0,+,}
\tkzTabVar{+/,R,-/,+/}
\end{tikzpicture} 
\end{center}
Từ bảng biến thiên ta thấy hàm số có một điểm cực tiểu duy nhất là $x=1$.
}
\end{ex}
\Closesolutionfile{ans}

\cauds
\Opensolutionfile{ans}[ans/ans\currfilebase-Phan-II]
\begin{ex}%[2-D1B5-SO-16-2425]%[VN-MT-7, Dương Phước Sang]%[2D1V2-1]
Cho các hàm số $f(x)=x^3-3x^2+2025$ và $g(x)=\dfrac{x^2-2x+1}{x-2}$.
\choiceTF
{\True Hàm số $y=f(x)$ nghịch biến trên khoảng $(0;2)$}
{Hàm số $y=g(x)$ nghịch biến trên khoảng $(1;3)$}
{\True Điểm cực đại của hàm số $y=f(x)$ là $x=0$}
{\True Đường thẳng đi qua $2$ điểm cực trị của đồ thị hàm số $y=g(x)$ cũng đi qua điểm $N(2;2)$}
\loigiai{
\begin{itemchoice}
\itemch \textbf{Đúng}.\\
Với $f(x)=x^3-3x^2+2025$, ta có $f'(x)=3x^2-6x$ và $f'(x)=0 \Leftrightarrow \hoac{&x=0\\&x=2.}$\\
Từ đó, ta có bảng xét dấu của $f'(x)$ như sau:
\begin{center}
\begin{tikzpicture}
\tkzTabInit[nocadre=true,lgt=1.2,espcl=2,deltacl=0.6]
{$x$/0.6,$f'(x)$/0.6}
{$-\infty$,$0$,$2$,$+\infty$}
\tkzTabLine{,+,0,-,0,+,}
\end{tikzpicture}
\end{center}
Suy ra hàm số nghịch biến trên khoảng $(0;2)$.
\itemch \textbf{Sai}.\\
Hàm số $y=g(x)=\dfrac{x^2-2x+1}{x-2}$ có tập xác định: $\mathscr{D}=\mathbb{R} \setminus \big\{2\big\}$.\\
Ta có $y'=\dfrac{x^2-4x+3}{(x-2)^2}$ và $y'=0 \Leftrightarrow x^2-4x+3=0 \Leftrightarrow \hoac{&x=1\\&x=3.}$\\
Bảng xét dấu của $g'(x)$:
\begin{center}
\begin{tikzpicture}
\tikzset{double style/.append style={double distance=1.5pt}}
\tkzTabInit[nocadre=true,lgt=1.2,espcl=2.5,deltacl=0.6]
{$x$/0.6,$g'(x)$/0.6}
{$-\infty$,$1$,$2$,$3$,$+\infty$}
\tkzTabLine{,+,0,-,d,-,0,+,}
\end{tikzpicture}
\end{center}
Suy ra hàm số nghịch biến trên khoảng $(1;2)$ và $(2;3)$.
\itemch \textbf{Đúng}.\\
Với $f(x)=x^3-3x^2+2025$, ta có $f'(x)=3x^2-6x$ và $f'(x)=0 \Leftrightarrow \hoac{&x=0\\&x=2.}$\\
Bảng biến thiên của hàm số $y=f(x)$:
\begin{center}
\begin{tikzpicture}
\tkzTabInit[nocadre=true,lgt=1.2,espcl=2.5,deltacl=0.6]
{$x$/0.6, $f'(x)$/0.6, $f(x)$/2}
{$-\infty$,$0$,$2$,$+\infty$}
\tkzTabLine{,+,0,-,0,+,}
\tkzTabVar{-/,+/,-/,+/}
\end{tikzpicture}
\end{center}
Suy ra điểm cực đại của hàm số là $x=0$.
\itemch \textbf{Đúng}.\\
Hàm số $y=g(x)=\dfrac{x^2-2x+1}{x-2}$ có tập xác định: $\mathscr{D}=\mathbb{R} \setminus \big\{2\big\}$.\\
Ta có $y'=\dfrac{x^2-4x+3}{(x-2)^2}$ và $y'=0 \Leftrightarrow x^2-4x+3=0 \Leftrightarrow \hoac{&x=1\\&x=3.}$\\
Bảng biến thiên của $g(x)$:
\begin{center}
\begin{tikzpicture}
\tikzset{double style/.append style={double distance=1.5pt}}
\tkzTabInit[nocadre=true,lgt=1.2,espcl=2.5,deltacl=0.6]
{$x$/0.6,$y'$/0.6,$y$/2}
{$-\infty$,$1$,$2$,$3$,$+\infty$}
\tkzTabLine{,+,0,-,d,-,0,+,}
\tkzTabVar{-/$-\infty$,+/$0$,-D+/$-\infty$/$+\infty$,-/$4$,+/$+\infty$}
\end{tikzpicture}
\end{center}
Hai điểm cực trị của đồ thị hàm số $y=g(x)$ là $A(1;0)$ và $B(3;4)$, cùng thuộc $AB\colon y=2x-2$.\\
Đường thẳng $AB$ đó đi qua điểm $N(2;2)$.
\end{itemchoice}
}
\end{ex}

\begin{ex}%[2-D1B5-SO-16-2425]%[VN-MT-7, Dương Phước Sang]%[2D1V3-1]
Cho các hàm số $f(x)=x^3-8x^2+16x-9$ và $h(x)=\dfrac{x^2-x+1}{x-1}$.
\choiceTF
{\True Giá trị lớn nhất của hàm số $y=f(x)$ trên đoạn $[-1;1]$ là $0$}
{Gọi giá trị lớn nhất và giá trị nhỏ nhất của hàm số $y=f(x)$ trên đoạn $[1;3]$ lần lượt là $a$, $b$. Khi đó giá trị của $27a-b$ bằng $13$}
{\True Giá trị nhỏ nhất của hàm số $y=h(x)$ trên khoảng $(1;+\infty)$ là $3$}
{\True Giá trị nhỏ nhất của hàm số $y=f\big(h(x)\big)$ trên khoảng $(1;3)$ là $-9$}
\loigiai{
\begin{itemchoice}
\itemch \textbf{Đúng}.\\
Với $f(x)=x^3-8x^2+16x-9$, ta có $f'(x)=3x^2-16x+16$.\\
Cho $f'(x)=0 \Leftrightarrow 3x^2-16x+16=0 \Leftrightarrow \hoac{&x=4 \notin [-1;1]\\&x=\dfrac{4}{3} \notin [-1;1].}$\\
Do $\heva{&f(-1)=-34\\&f(1)=0}$ nên $\max\limits_{x \in [-1;1]} f(x) =0$.
\itemch \textbf{Sai}.\\
Với $f(x)=x^3-8x^2+16x-9$, ta có $f'(x)=3x^2-16x+16$.\\
Cho $f'(x)=0 \Leftrightarrow 3x^2-16x+16=0 \Leftrightarrow \hoac{&x=4 \notin [1;3]\\&x=\dfrac{4}{3} \in [1;3].}$\\
Vì $\heva{&f(1)=0\\&f(\dfrac{4}{3})=\dfrac{13}{27}\\&f(3)=-6}$ nên $\heva{&\max\limits_{x \in [1;3]} f(x) =\dfrac{13}{27}=a\\&\min\limits_{x \in [1;3]} f(x) =-6=b}$. Từ đó $27a-b=19$.
\itemch \textbf{Đúng}.\\
Với $h(x)=\dfrac{x^2-x+1}{x-1}$, ta có $h'(x)=\dfrac{x^2-2x}{(x-1)^2}$.\\
Cho $h'(x)=0 \Rightarrow x^2-2x=0 \Leftrightarrow \hoac{&x=0 \notin (1;+\infty)\\&x=2 \in (1;+\infty).}$
\begin{center}
\begin{tikzpicture}
\tkzTabInit[nocadre=true,lgt=1.2,espcl=2.5,deltacl=0.6]
{$x$/0.6,$h'(x)$/0.6,$h(x)$/2}
{$1$,$2$,$+\infty$}
\tkzTabLine{,-,0,+,}
\tkzTabVar{+/$+\infty$,-/$3$,+/$+\infty$}
\end{tikzpicture}
\end{center}
Từ bảng biến thiên, suy ra giá trị nhỏ nhất của hàm số $y=h(x)$ trên khoảng $(1;+\infty)$ là $3$.
\itemch \textbf{Đúng}.\\
Với $t=h(x)=\dfrac{x^2-x+1}{x-1}$, ta có $h'(x)=\dfrac{x^2-2x}{(x-1)^2}$.\\
Cho $h'(x)=0 \Rightarrow x^2-2x=0 \Leftrightarrow \hoac{&x=0 \notin (1;3)\\&x=2 \in (1;3).}$
\begin{center}
 \begin{tikzpicture}
 \tkzTabInit[nocadre=true,lgt=1.2,espcl=2.5,deltacl=0.6]
 {$x$/0.6,$h'(x)$/0.6,$h(x)$/2}
 {$1$,$2$,$3$}
 \tkzTabLine{,-,0,+,}
 \tkzTabVar{+/$+\infty$,-/$3$,+/$\tfrac{7}{2}$}
 \end{tikzpicture}
\end{center}
Như thế đặt $t=h(x)$, $x\in (1;3)$ thì $t \in [3;+\infty)$ và $y=f(t)=t^3-8t^2+16t-9$.\\
Ta có $f'(t)=3t^2-16t+16$.\\
Cho $f'(t)=0 \Leftrightarrow 3t^2-16t+16=0 \Leftrightarrow \hoac{&t=4 \in [3;+\infty)\\&x=\dfrac{4}{3} \notin [3;+\infty).}$
\begin{center}
 \begin{tikzpicture}
 \tkzTabInit[nocadre=true,lgt=1.2,espcl=2.5,deltacl=0.6]
 {$t$/0.6,$f'(t)$/0.6,$f(t)$/2}
 {$3$,$4$,$+\infty$}
 \tkzTabLine{,-,0,+,}
 \tkzTabVar{+/$-6$,-/$-9$,+/$+\infty$}
 \end{tikzpicture}
\end{center}
Vậy $\min\limits_{1<x<3} f\big(h(x)\big)=\min\limits_{t \geq 3} f(t)=f(4)=-9$.
\end{itemchoice}
}
\end{ex}

\begin{ex}%[2-D1B5-SO-16-2425]%[VN-MT-7, Dương Phước Sang]%[2D1V4-1]
Cho các hàm số $f(x)=\dfrac{x-2}{x+3}$ và $g(x)=\dfrac{x^2-3x}{x+1}$. 
\choiceTF
{\True Đồ thị hàm số $y=f(x)$ có đường tiệm cận ngang là đường thẳng $y=1$}
{\True Đồ thị hàm số $y=g(x)$ có đường tiệm cận đứng là đường thẳng $x=-1$}
{\True Đồ thị hàm số $y=g(x)$ có đường tiệm cận xiên là đường thẳng $y=x-4$}
{\True Đồ thị hàm số $y=g\big(f(x)\big)$ không có đường tiệm cận xiên nào cả}
\loigiai{
\begin{itemchoice}
\itemch \textbf{Đúng}.\\
Ta có $\lim\limits_{x \to \pm\infty} \dfrac{x-2}{x+3}=\lim\limits_{x \to \pm\infty} \dfrac{1-\dfrac{2}{x}}{1+\dfrac{3}{x}}=1$ nên đồ thị hàm số $y=\dfrac{x-2}{x+3}$ có đường tiệm cận ngang là đường thẳng $y=1$.
\itemch \textbf{Đúng}.\\
Ta có $\lim\limits_{x \to -1^+} \dfrac{x^2-3x}{x+1}=\lim\limits_{x \to -1^+} \left(x-4+\dfrac{4}{x+1}\right)=+\infty$ nên đồ thị hàm số $y=\dfrac{x^2-3x}{x+1}$ có đường tiệm cận đứng là đường thẳng $x=-1$.
\itemch \textbf{Đúng}.\\
Ta có $y=\dfrac{x^2-3x}{x+1}=x-4+\dfrac{4}{x+1}$ nên $\lim\limits_{x \to \pm\infty} \left(\dfrac{x^2-3x}{x+1}-(x-4)\right)=\lim\limits_{x \to \pm\infty} \dfrac{4}{x+1}=0$.\\
Vậy đường thẳng $y=x-4$ là tiệm cận xiên của đồ thị hàm số $y=\dfrac{x^2-3x}{x+1}$.
\itemch \textbf{Đúng}.\\
Ta có $y=g\big(f(x)\big)=\dfrac{\big(f(x)\big)^2-3\big(f(x)\big)}{f(x)+1}$ nên $\lim\limits_{x \to \pm\infty} g\big(f(x)\big)=\lim\limits_{x \to \pm\infty} \dfrac{\big(f(x)\big)^2-3\big(f(x)\big)}{f(x)+1}$.\\
Mà $\lim\limits_{x \to \pm\infty} f(x)=\lim\limits_{x \to \pm\infty} \dfrac{x-2}{x+3}=\lim\limits_{x \to \pm\infty} \dfrac{1-\dfrac{2}{x}}{1+\dfrac{3}{x}}=1$ nên
$\lim\limits_{x \to \pm\infty} g\big(f(x)\big)=\dfrac{1^2-3\cdot 1}{1+1}=-1$.\\
Vậy $(C)\colon y=g\big(f(x)\big)$ có tiệm cận ngang $y=-1$ mà không có tiệm cận xiên.
\end{itemchoice}
}
\end{ex}

\begin{ex}%[2D1C4-3]
Cho hàm số $y=f(x)$ xác định trên tập $\mathscr{D}=\mathbb{R} \setminus \{2\}$, có bảng biến thiên như sau:
\begin{center}
\begin{tikzpicture}
\tikzset{double style/.append style={double distance=1.5pt}}
\tkzTabInit[nocadre=true,lgt=1.2,espcl=2]
{$x$/0.6,$f'(x)$/0.6,$f(x)$/2}
{$-\infty$,$1$,$2$,$3$,$+\infty$}
\tkzTabLine{,+,0,-,d,-,0,+,}
\tkzTabVar{-/$-\infty$,+/$1$,-D+/$-\infty$/$+\infty$,-/$5$,+/$+\infty$}
\end{tikzpicture}
\end{center}
\choiceTF
{\True Hàm số $y=f(x)$ có cực đại nhỏ hơn cực tiểu}
{\True Hàm số $f(x)=\dfrac{x^2-x-1}{x-2}$ có bảng biến thiên như trên}
{\True Đồ thị hàm số $y=f(x)$ luôn có đúng $1$ tiệm cận đứng}
{Đồ thị hàm số $y=f(x)$ luôn có $1$ hoặc $2$ tiệm cận xiên}
\loigiai{
\begin{itemchoice}
\itemch \textbf{Đúng}.\\
Hàm số $y=f(x)$ có cực đại bằng $1$ và cực tiểu bằng $5$ nên cực đại nhỏ hơn cực tiểu.\itemch \textbf{Đúng}.\\
Xét hàm số $y=\dfrac{x^2-x-1}{x-2}$ có tập xác định $\mathscr{D}=\mathbb{R} \setminus \big\{2\big\}$.\\
Ta có $y'=\dfrac{x^2-4x+3}{(x-2)^2}$ và
$y'=0 \Leftrightarrow x^2-4x+3=0 \Leftrightarrow \hoac{&x=1\\&x=3.}$\\
Bảng biến thiên của hàm số $y=\dfrac{x^2-x-1}{x-2}$ đúng như bảng biến thiên được cung cấp.
\begin{center}
\begin{tikzpicture}
\tikzset{double style/.append style={double distance=1.5pt}}
\tkzTabInit[nocadre=true,lgt=1.2,espcl=2,deltacl=0.6]
{$x$/0.6,$y'$/0.6,$y$/1.8}
{$-\infty$,$1$,$2$,$3$,$+\infty$}
\tkzTabLine{,+,0,-,d,-,0,+,}
\tkzTabVar{-/$-\infty$,+/$1$,-D+/$-\infty$/$+\infty$,-/$5$,+/$+\infty$}
\end{tikzpicture}
\end{center}
\itemch \textbf{Đúng}.\\
Tại mọi $x_0 \neq 2$, bảng biến thiên hàm số thể hiện $\lim\limits_{x \to x_0} f(x)=f(x_0)$ nên $x=x_0$ không là tiệm cận của đồ thị hàm số.\\
Và chỉ có $\lim\limits_{x \to 2^-} f(x)=-\infty$ nên chỉ có $x=2$ là tiệm cận đứng duy nhất của đồ thị hàm số.
\itemch \textbf{Sai}.\\
Không có đủ cơ sở nào để khẳng định được hàm số $y=f(x)$ có $1$ tiệm cận xiên.\\
Ít nhất có hàm số $f(x)=\dfrac{(x-2)^6-5(x-2)^4+15(x-2)^2+24(x-2)+5}{8(x-2)}$
\[\text{Có }
\begin{aligned}[t]
f'(x)&=\dfrac{1}{8}\left[5(x-2)^4-15(x-2)^2+15-\dfrac{5}{(x-2)^2}\right]\\
&=\dfrac{5}{8}\cdot\dfrac{\left(x^2-4x+3\right)^3}{(x-2)^2}.
\end{aligned}\]
Bảng biến thiên của hàm số $f(x)$ này đúng như bảng biến thiên được cung cấp nhưng đồ thị hàm số không hề có tiệm cận xiên do $\lim\limits_{x \to \pm\infty} \dfrac{f(x)}{x}=+\infty$.
\end{itemchoice}
}
\end{ex}
\Closesolutionfile{ans}

\caukq
\Opensolutionfile{ans}[ans/ans\currfilebase-Phan-III]
\begin{ex}%[2-D1B5-SO-16-2425]%[VN-MT-7, Dương Phước Sang]%[2D1C5-5]
\immini[thm]{
Cho hàm số $y=f(x)$ có đạo hàm trên $\mathbb{R}$, thỏa mãn $f(-1)=f(3)=0$ và đồ thị của hàm số 
$y=f'(x)$ có dạng như hình bên đây. Có tất cả bao nhiêu cặp số nguyên $\big\{a;b\}$ thuộc đoạn $[-10;10]$ để hàm số $y=\big[f(x)\big]^2$ nghịch biến trên khoảng $(a;b)$?
\shortans{48}}
{\begin{tikzpicture}[xscale=0.6, yscale=0.3, font=\footnotesize, samples=200, >=stealth]
\draw[->] (-2,0)--(4.1,0) node[below]{$x$};
\draw[->] (0,-4.6)--(0,4.6) node[left]{$y$};
\node at (0,0) [below left=-2pt]{$O$};
\draw plot[domain=-1.42:3.42] (\x,{-(\x)^3+3*(\x)^2+\x-3});
\fill[yscale=2] 
(-1,0) circle(1.0pt) node[below,xshift=-7]{$-1$}
(1,0) circle(1.0pt) node[below,xshift=2]{$1$}
(3,0) circle(1.0pt) node[below,xshift=-4]{$3$};
\end{tikzpicture}}
\loigiai{
Từ đồ thị và giả thiết, ta có bảng biến thiên của hàm số $y=f(x)$ như sau:
\begin{center}
\begin{tikzpicture}
\tkzTabInit[nocadre=true,lgt=1.2,espcl=2.5,deltacl=0.6]
{$x$/0.6,$y'$/0.6,$y$/2}
{$-\infty$,$-1$,$1$,$3$,$+\infty$}
\tkzTabLine{,+,0,-,0,+,0,-,}
\tkzTabVar{-/,+/$0$,-/,+/$0$,-/}
\end{tikzpicture}
\end{center}
Như thế $f(x) \leq 0$ với mọi $x \in \mathbb{R}$.\\
Với $y=\big[f(x)\big]^2$, ta có $y'=\left[\big(f(x)\big)^2\right]'=2f(x) \cdot f'(x)$.\\
Bảng xét dấu của $y'=\left[\big(f(x)\big)^2\right]'$:
\begin{center}
\begin{tikzpicture}
\tkzTabInit[nocadre=true,lgt=2.3,espcl=2,deltacl=0.6]
{$x$ /0.6, $f'(x)$ /0.6, $f(x)$/0.6, $\left[\big(f(x)\big)^2\right]'$ /0.9}
{$-\infty$,$-1$,$1$,$3$,$+\infty$}
\tkzTabLine{,+,0,-,0,+,0,-,}
\tkzTabLine{,-,0,-,t,-,0,-,}
\tkzTabLine{,-,0,+,0,-,0,+,}
\end{tikzpicture}
\end{center}
Như vậy hàm số $y=\big[f(x)\big]^2$ nghịch biến trên các khoảng $(-\infty;-1)$ và $(1;3)$.\\
Từ đó, số cặp số nguyên $\{a;b\}$ là số cách chọn $2$ từ $3$ số $\{1;2;3\}$ hoặc từ $10$ số $\{-10;-9;\ldots;-1\}$.\\
Số cặp số $\{a;b\}$ là $\mathrm{C}_3^2+\mathrm{C}_{10}^2=48$.
}
\end{ex}

\begin{ex}%[2-D1B5-SO-16-2425]%[VN-MT-7, Dương Phước Sang]%[2D1V5-4]
Đồ thị hàm số $y=x^3-3x^2-9x+5$ có điểm cực đại và điểm cực tiểu lần lượt là $A$ và $B$. 
Gọi $I$ là giao điểm của $AB$ với trục $Ox$. Đặt tỷ số $\dfrac{IA}{IB}=\dfrac{b}{c}$ tối giản ($b,c \in \mathbb{N}$). Tính $T=b+c$.
\par\shortans{16}
\loigiai{
Hàm số $y=x^3-3x^2-9x+5$ có tập xác định $\mathscr{D}=\mathbb{R}$.\\
$y'=3x^2-6x-9$. Cho $y'=0 \Leftrightarrow 3x^2-6x-9=0 \Leftrightarrow \hoac{&x=-1\\&x=3.}$\\
Với $x=-1$ ta có $y=y(-1)=10$. Đặt $A(-1;10)$.\\
Với $x=3$ ta có $y=y(3)=-22$. Đặt $B(3;-22)$.\\
Vì $AB$ cắt $Ox$ tại $I$ nên $\dfrac{IA}{IB}=\dfrac{\mathrm{d}(A,Ox)}{\mathrm{d}(B,Ox)}=\dfrac{\left|y_A\right|}{\left|y_B\right|}=\dfrac{10}{22}=\dfrac{5}{11}$.\\
Như vậy $b=5$ và $c=11$ nên $T=b+c=16$.
}
\end{ex}

\begin{ex}%[2-D1B5-SO-16-2425]%[VN-MT-7, Dương Phước Sang]%[2D1V3-1]
Gọi $M$, $m$ lần lượt là giá trị lớn nhất và giá trị nhỏ nhất của hàm số $y=\dfrac{3\sin x+2}{\sin x+1}$ trên đoạn $\left[ 0;\dfrac{\pi}{2} \right]$. Xác định giá trị làm tròn đến hàng phần mười của biểu thức $M^2+m^2$.
\shortans{10{,}3}
\loigiai{
Đặt $t=\sin x$, ta có $x \in \left[ 0;\dfrac{\pi}{2} \right]$ nên $t \in [0;1]$.\\
Xét hàm $f(t)=\dfrac{3t+2}{t+1}$ trên đoạn $[0;1]$ có $f'(t)=\dfrac{1}{(t+1)^2}>0,\, \forall t \in [0;1]$.\\
Suy ra hàm số $f(t)$ đồng biến trên $[0;1]$.\\
Từ đó ta có
$M=\max\limits_{[0;1]} f(t)=f(1)=\dfrac{5}{2}$
và 
$m=\min\limits_{[0;1]} f(t)=f(0)=2$.\\
Khi đó, $M^2+m^2=\left( \dfrac{5}{2} \right)^2+2^2=\dfrac{41}{4}=10{,}25 \approx 10{,}3$.
}
\end{ex}

\begin{ex}%[2-D1B5-SO-16-2425]%[VN-MT-7, Dương Phước Sang]%[2D1V3-6]
Vận tốc của một tàu con thoi từ lúc cất cánh tại thời điểm $t=0$\,s cho đến thời điểm $t=126$\,s được cho bởi công thức $v(t)=0{,}001302t^3-0{,}09029t^2+83$ (vận tốc được tính bằng đơn vị ft/s). Gọi $v_{\min}$ là vận tốc nhỏ nhất của tàu con thoi. Xác định kết quả làm tròn đến hàng phần mười của $v_{\min}$.
\shortans{18{,}7}
\loigiai{
Hàm số $v(t)=0{,}001302t^3-0{,}09029t^2+83$ liên tục trên đoạn $[0;126]$.\\
Ta có $v'(t)=0{,}003906t^2-0{,}18058t$.\\
Cho $v'(t)=0 \Leftrightarrow 0{,}003906t^2-0{,}18058t=0 \Leftrightarrow \hoac{&t=0\\&t=\dfrac{0{,}18058}{0{,}003906}.}$\\
Trên đoạn $[0;126]$, ta có 
$v(0)=83$; $v\left(\dfrac{0{,}18058}{0{,}003906}\right) \approx 18{,}67301185$; $v(126) \approx 1254{,}045512$.\\
Tàu con thoi đạt vận tốc nhỏ nhất bằng $v\left(\dfrac{0{,}18058}{0{,}003906}\right) \approx 18{,}7$ ft/s.
}
\end{ex}

\begin{ex}%[2-D1B5-SO-16-2425]%[VN-MT-7, Dương Phước Sang]%[2D1H4-1]
Một mảnh vườn hình chữ nhật có diện tích bằng $900$\,m$^2$. Biết chiều dài của mảnh vườn là $x$\,(m). Gọi biểu thức tính chu vi của mảnh vườn là $P(x)$\,(m). Biết rằng phương trình tiệm cận xiên của đồ thị hàm số $P(x)$ là $y=ax+b$. Tính giá trị biểu thức $T=10^a+b$.
\shortans{100}
\loigiai{
Mảnh vườn có chiều dài $x$\,(m) nên có chiều rộng là $\dfrac{900}{x}$\,(m).\\
Điều kiện: $x \geq \dfrac{900}{x} \Leftrightarrow x \geq 30$.\\
Ta có $P(x)=2\left( x+\dfrac{900}{x} \right)=2x+\dfrac{1800}{x}$.\\
Vì $\lim\limits_{x \to +\infty} [P(x)-2x]=\lim\limits_{x \to +\infty} \dfrac{1800}{x}=0$ nên đồ thị hàm số $P(x)$ có tiệm cận xiên là đường thẳng $y=2x$.\\
Suy ra $a=2$, $b=0$. Do vậy, $T=100$.
}
\end{ex}

\begin{ex}%[2-D1B5-SO-16-2425]%[VN-MT-7, Dương Phước Sang]%[2D1V5-4]
Biết rằng đồ thị hàm số $y=\dfrac{x+1}{x-1}$ cắt đường thẳng $y=2x-1$ tại hai điểm phân biệt $A$, $B$. Tính diện tích tam giác $OAB$.
\shortans{1}
\loigiai{
Phương trình hoành độ giao điểm của đồ thị hai hàm số $y=\dfrac{x+1}{x-1}$ và $y=2x-1$ là
\[\dfrac{x+1}{x-1}=2x-1 \Leftrightarrow 2x^2-4x=0 \Leftrightarrow \hoac{&x=0\\&x=2.}\]
Suy ra toạ độ các giao điểm của đồ thị hai hàm số đó là $A(0;-1)$, $B(2;3)$.\\
Diện tích cần tìm là 
$S=\dfrac{1}{2}|0 \cdot 3-(-1) \cdot 2|=1$.
}
\end{ex}
\Closesolutionfile{ans}
\begin{indapan}
	{ans/ans\currfilebase}
\end{indapan}


% \setcounter{deso}{4}
\begin{name}
	{\tenchude}
	{ĐỀ ÔN TẬP CHƯƠNG I}
	{LỚP TOÁN THẦY PHÁT}
	{\thoigian}
\end{name}

\TN
\Opensolutionfile{ans}[ans/ansc1l4-Phan-I]
\begin{ex}%[2-D1B5-SO-17-2425]%[VN-MT-7, VM031]%[2D1N5-1]
 \immini{Đường cong cho trong hình bên là đồ thị của hàm số nào trong các hàm số dưới đây?
 \choice[2]
 {$y=-x^3+2x-1$}
 {$y=-x^3+3x+1$}
 {$y=2x^3-6x+1$}
 {\True $y=x^3-3x+1$}}
 {\begin{tikzpicture}[font=\footnotesize,line join=round, line cap=round, >=stealth,scale=0.6] 
 \def \xmin{-3}\def \xmax{3}\def \ymin{-2}\def \ymax{4} 
 \draw[->] (\xmin,0)--(\xmax,0) node[shift=(-110:0.2)] {$x$};
 \draw[->] (0,\ymin)--(0,\ymax) node[shift=(-150:0.2)] {$y$};
 \fill (0,0) circle(1pt) node[shift=(135:0.25)]{$O$}
 (-1,0) circle(1pt) node[shift=(-90:0.2)]{$-1$}
 (1,0) circle(1pt) node[shift=(90:0.2)]{$1$}
 (0,-1) circle(1pt) node[shift=(-150:0.28)]{$-1$}
 (0,1) circle(1pt) node[shift=(0:0.2)]{$1$}
 (0,3) circle(1pt) node[shift=(0:0.2)]{$3$}
 (-1,3) circle(1pt) (1,-1) circle(1pt); 
 \draw[dashed] (-1,0)|-(0,3) (1,0)|-(0,-1);
 \clip (\xmin,\ymin) rectangle (\xmax,\ymax);
 \draw[smooth,samples=100,domain=\xmin:\xmax] plot(\x,{(\x)^3-3*(\x)+1}); 
 \end{tikzpicture}}
 \loigiai{
 Quan sát đồ thị, ta thấy
 \begin{itemize}
 \item Đây là đồ thị của hàm số $y=ax^3+bx^2+cx+d$ $ (a\ne 0)$ có $a > 0$.
 \item Đồ thị hàm số có hai điểm cực trị $(-1;3)$ và $(1;-1)$.
 \end{itemize}
 Vậy đường cong trong hình vẽ là đồ thị hàm số $y=x^3-3x+1$.
 }
\end{ex}

\begin{ex}%[2-D1B5-SO-17-2425]%[VN-MT-7, VM031]%[2D1H5-1]
 \immini{Cho hàm số $y=\dfrac{ax+b}{cx-1}$ có đồ thị như hình vẽ bên dưới. Trong các hệ số $a$, $b$, $c$ có bao nhiêu số dương?
 \choice[2]
 {$0$}
 {\True $2$}
 {$1$}
 {$3$}}
 {\begin{tikzpicture}[font=\footnotesize,line join=round, line cap=round, >=stealth,scale=0.6] 
 \def \xmin{-3}\def \xmax{4.5}\def \ymin{-4}\def \ymax{3} 
 \draw[->] (\xmin,0)--(\xmax,0) node[shift=(-110:0.2)] {$x$};
 \draw[->] (0,\ymin)--(0,\ymax) node[shift=(-150:0.2)] {$y$};
 \fill (0,0) circle(1pt) node[shift=(-135:0.25)]{$O$}
 (1,0) circle(1pt) node[shift=(-135:0.25)]{$1$}
 (0,-1) circle(1pt) node[shift=(-145:0.28)]{$-1$}
 (2,0) circle(1pt) node[shift=(-135:0.25)]{$2$}; 
 \draw (\xmin,-1)--(\xmax,-1);
 \clip (\xmin,\ymin) rectangle (\xmax,\ymax); 
 \draw[smooth,samples=100,domain=\xmin:0.99] plot(\x,{(-1*(\x)+2)/((\x)-1)});
 \draw[smooth,samples=100,domain=1.01:\xmax] plot(\x,{(-1*(\x)+2)/((\x)-1)}); 
 \draw (1,\ymin)--(1,\ymax);
 \end{tikzpicture}}
 \loigiai{
 \begin{itemize}
 \item Tiệm cận đứng $x=\dfrac{1}{c}=1\Leftrightarrow c=1$.
 \item Tiệm cận ngang $y=\dfrac{a}{c}=-1\Leftrightarrow a=-c\Rightarrow a=-1$.
 \item Đồ thị cắt trục hoành tại $x=2$ nên $2a+b=0$ hay $b=-2a=2$.
 \end{itemize}
 Vậy có hai số dương.
 }
\end{ex}

\begin{ex}%[2-D1B5-SO-17-2425]%[VN-MT-7, VM031]%[2D1H5-1]
 \immini{Đường cong cho trong hình bên là đồ thị của hàm số nào trong các hàm số dưới đây?
 \choice[2]
 {$y=\dfrac{x^2-2x+2}{x+1}$}
 {$y=\dfrac{-x^2+x+2}{x-1}$}
 {\True $y=\dfrac{x^2-x+1}{-x+1}$}
 {$y=\dfrac{-x^2-x+1}{x-1}$}}
 {\begin{tikzpicture}[font=\footnotesize,line join=round, line cap=round, >=stealth,scale=0.6] 
 \def \xmin{-3}\def \xmax{5}\def \ymin{-5}\def \ymax{3} 
 \draw[->] (\xmin,0)--(\xmax,0) node[shift=(-110:0.2)] {$x$};
 \draw[->] (0,\ymin)--(0,\ymax) node[shift=(-150:0.2)] {$y$};
 \fill (0,0) circle(1pt) node[shift=(-135:0.25)]{$O$}
 (1,0) circle(1pt) node[shift=(-120:0.22)]{$1$}
 (2,0) circle(1pt) node[shift=(90:0.2)]{$2$}
 (0,1) circle(1pt) node[shift=(-45:0.21)]{$1$}
 (0,-3) circle(1pt) node[shift=(180:0.25)]{$-3$}
 (2,-3) circle(1pt);
 \draw[dashed] (2,0)|-(0,-3);
 \clip (\xmin,\ymin) rectangle (\xmax,\ymax); 
 \clip (\xmin,\ymin) rectangle (\xmax,\ymax);
 \draw[smooth,samples=100,domain=\xmin:0.99] plot(\x,{((\x)^2-1*(\x)+1)/(-1*(\x)+1)}); 
 \draw[smooth,samples=100,domain=1.01:\xmax] plot(\x,{((\x)^2-1*(\x)+1)/(-1*(\x)+1)}); 
 \draw[smooth,samples=100,domain=\xmin:\xmax] plot(\x,{-(\x)}); 
 \draw (1,\ymin)--(1,\ymax);
 \end{tikzpicture}}
 \loigiai{
 \begin{itemize}
 \item Đồ thị hàm số có tiệm cận đứng $x=1$.
 \item Đồ thị hàm số có tiệm cận xiên $y=-x$.
 \item Đồ thị hàm số đi qua điểm $\left(2;-3\right)$.
 \end{itemize}
 Vậy đường cong trong hình vẽ là đồ thị hàm số $y=\dfrac{x^2-x+1}{-x+1}$.
 }
\end{ex}

\begin{ex}%[2-D1B5-SO-17-2425]%[VN-MT-7, VM031]%[2D1H4-1]
 \immini{Cho hàm số $y=\dfrac{ax^2+bx+1}{cx+2}$ có đồ thị như hình vẽ bên dưới. Tính giá trị biểu thức $T=2a+3b-c$.
 \choice[2]
 {$9$}
 {\True $10$}
 {$8$}
 {$11$}}
 {\begin{tikzpicture}[font=\scriptsize, line join=round, line cap=round, >=stealth,scale=0.6] 
 \def \xmin{-6}\def \xmax{2.5}\def \ymin{-4}\def \ymax{2.5} 
 \draw[->] (\xmin,0)--(\xmax,0) node[shift=(-110:0.2)] {$x$};
 \draw[->] (0,\ymin)--(0,\ymax) node[shift=(-150:0.2)] {$y$};
 \fill (0,0) circle(1pt) node[shift=(-45:0.25)]{$O$}
 (-2,0) circle(1pt) node[shift=(-140:0.25)]{$-2$}
 (-1,0) circle(1pt) node[shift=(110:0.2)]{$-1$}
 (0,1) circle(1pt) node[shift=(150:0.19)]{$1$};
 \clip (\xmin,\ymin) rectangle (\xmax,\ymax); 
 \draw[smooth,samples=100,domain=\xmin:-2.01] plot(\x,{((\x)^2+3*(\x)+1)/((\x)+2)}); 
 \draw[smooth,samples=100,domain=-1.99:\xmax] plot(\x,{((\x)^2+3*(\x)+1)/((\x)+2)}); 
 \draw[smooth,samples=100,domain=\xmin:\xmax] plot(\x,{(\x)+1}); 
 \draw(-2,\ymin)--(-2,\ymax);
 \end{tikzpicture}}
 \loigiai{
 \begin{center}
 
 \end{center}
 \begin{itemize}
 \item Đồ thị có tiệm cận đứng $x=-2$. Suy ra $-\dfrac{2}{c}=-2\Leftrightarrow c=1$.
 \item Đồ thị có tiệm cận xiên đi qua hai điểm $(0;1)$ và $(-1;0)$ nên có phương trình \[\dfrac{x}{-1}+\dfrac{y}{1}=1\Leftrightarrow y=x+1.\]
 \end{itemize}
 Khi đó ta có
 \begin{itemize}
 \item $\lim\limits_{x\to+\infty} \dfrac{ax^2+bx+1}{x\left(x+2\right)}=1\Leftrightarrow a=1$;
 \item $\lim\limits_{x\to+\infty} \left(\dfrac{x^2+bx+1}{x+2}-x\right)=\lim\limits_{x\to+\infty}\dfrac{\left(b-2\right)x+1}{x+2}=b-2=1\Leftrightarrow b=3$. 
 \end{itemize}
 Vậy $T=2a+3b-c=2+9-1=10$.
 }
\end{ex}

\begin{ex}%[2-D1B5-SO-17-2425]%[VN-MT-7, VM031]%[2D1N1-2]
 Cho hàm số $f(x)$ có bảng biến thiên như sau
 \begin{center}
 \begin{tikzpicture}
 \tkzTabInit[nocadre=true,lgt=1.2,espcl=2.5,deltacl=0.5]
 {$x$/0.7,$f'(x)$/0.7,$f(x)$/2}
 {$-\infty$,$0$,$2$,$+\infty$}
 \tkzTabLine{,+,0,-,0,+,}
 \tkzTabVar{-/$-\infty$,+/$1$,-/$-3$,+/$+\infty$}
 \end{tikzpicture}
 \end{center}
 Hàm số đã cho nghịch biến trên khoảng nào dưới đây?
 \choice
 {$(2;+\infty)$}
 {\True $(0;2)$}
 {$(-3;1)$}
 {$(-\infty;1)$}
 \loigiai{
 Dựa vào bảng biến thiên của hàm số, ta có hàm số đã cho nghịch biến trên khoảng $(0;2)$. 
 }
\end{ex}

\begin{ex}%[2-D1B5-SO-17-2425]%[VN-MT-7, VM031]%[2D1H1-1]
 \immini{Hàm số $y=-x^3+3x^2$ đồng biến trên khoảng nào dưới đây?
 \choice
 {$(0;4)$}
 {$(-\infty;0)$}
 {$(2;+\infty)$}
 {\True $(0;2)$} }
 
 \loigiai{
 Ta có $y'=-3x^2+6x=0\Leftrightarrow \hoac{&x=0\\&x=2.}$\\
 Hàm số đồng biến khi $y' > 0$ $\Leftrightarrow 0< x < 2$.
 }
\end{ex}

\begin{ex}%[2-D1B5-SO-17-2425]%[VN-MT-7, VM031]%[2D1N2-2]
 \immini{Cho hàm số $y=f(x)$ xác định và liên tục trên đoạn $[-2; 2]$ và có đồ thị là đường cong trong hình vẽ sau.
 Điểm cực tiểu của đồ thị hàm số $y=f(x)$ là
 \choice[2]
 {$x=1$}
 {$x=-2$}
 {\True $M(1;-2)$}
 {$M(-2;-4)$} }
 {\begin{tikzpicture}[font=\footnotesize,line join=round, line cap=round, >=stealth,scale=0.6] 
 \def \xmin{-2.8}\def \xmax{3}\def \ymin{-4.7}\def \ymax{5} 
 \draw[->] (\xmin,0)--(\xmax,0) node[shift=(-110:0.2)] {$x$};
 \draw[->] (0,\ymin)--(0,\ymax) node[shift=(-150:0.2)] {$y$};
 \fill (0,0) circle(1pt) node[shift=(45:0.25)]{$O$}
 (-2,0) circle(1pt) node[shift=(90:0.2)]{$-2$}
 (-1,0) circle(1pt) node[shift=(-90:0.2)]{$-1$}
 (1,0) circle(1pt) node[shift=(90:0.2)]{$1$}
 (2,0) circle(1pt) node[shift=(-90:0.2)]{$2$}
 (0,-4) circle(1pt) node[shift=(0:0.25)]{$-4$}
 (0,-2) circle(1pt) node[shift=(180:0.25)]{$-2$}
 (0,2) circle(1pt) node[shift=(0:0.2)]{$2$}
 (0,4) circle(1pt) node[shift=(180:0.2)]{$4$}
 (-1,2) circle(1pt) (1,-2) circle(1pt) (-2,-4) circle(1pt) (2,4) circle(1pt);
 \clip (\xmin,\ymin) rectangle (\xmax,\ymax); 
 \draw[smooth,samples=100,domain=-2:2] plot(\x,{4/3*(\x)^3-10/3*(\x)}); 
 \draw[dashed] (-2,0)|-(0,-4) (-1,0)|-(0,2) (1,0)|-(0,-2) (2,0)|-(0,4);
 \end{tikzpicture}}
 \loigiai{
 Dựa vào đồ thị hàm số ta thấy điểm cực tiểu của đồ thị hàm số $y=f(x)$ là $M(1;-2)$.
 }
\end{ex}

\begin{ex}%[2-D1B5-SO-17-2425]%[VN-MT-7, VM031]%[2D1N3-1]
 \immini{Cho hàm số $f(x)$ liên tục trên đoạn $[-2;2]$ có đồ thị như hình vẽ.
 Giá trị nhỏ nhất của hàm số trên đoạn $[-2;2]$ là
 \choice[2]
 {$1$}
 {\True $-1$}
 {$-2$}
 {$3$}}
 {\begin{tikzpicture}[font=\footnotesize,line join=round, line cap=round, >=stealth,scale=0.6] 
 \def \xmin{-3}\def \xmax{3}\def \ymin{-2}\def \ymax{4} 
 \draw[->] (\xmin,0)--(\xmax,0) node[shift=(-110:0.2)] {$x$};
 \draw[->] (0,\ymin)--(0,\ymax) node[shift=(-150:0.2)] {$y$};
 \fill (0,0) circle(1pt) node[shift=(135:0.25)]{$O$}
 (-2,0) circle(1pt) node[shift=(135:0.25)]{$-2$}
 (-1,0) circle(1pt) node[shift=(-90:0.2)]{$-1$}
 (1,0) circle(1pt) node[shift=(90:0.2)]{$1$}
 (2,0) circle(1pt) node[shift=(-90:0.2)]{$2$}
 (0,-1) circle(1pt) node[shift=(-145:0.28)]{$-1$}
 (0,3) circle(1pt) node[shift=(135:0.25)]{$3$}
 (-2,-1) circle(1pt) (1,-1) circle(1pt) (-1,3) circle(1pt) (2,3) circle(1pt);
 \clip (\xmin,\ymin) rectangle (\xmax,\ymax); 
 \draw[smooth,samples=100,domain=\xmin:\xmax] plot(\x,{(\x)^3-3*(\x)+1}); 
 \draw[dashed] (-2,0)--(-2,-1)--(1,-1)--(1,0) (-1,0)--(-1,3)--(2,3)--(2,0);
 \end{tikzpicture}}
 \loigiai{
 Từ đồ thị ta thấy $\min\limits_{[-2;2]} f(x)=f(1)=-1$.
 }
\end{ex}

\begin{ex}%[2-D1B5-SO-17-2425]%[VN-MT-7, VM031]%[2D1H3-1]
 Giá trị nhỏ nhất của hàm số $y=x^2-2x+3$ trên đoạn $[2;4]$ là
 \choice
 {\True $3$}
 {$-1$}
 {$0$}
 {$1$}
 \loigiai{
 \begin{itemize}
 \item $y'=\left(x^2-2x+3\right)'=2x-2$;
 \item $y'=0\Leftrightarrow 2x-2=0\Leftrightarrow x=1\notin [2;4]$.
 \end{itemize}
 Ta có $y(2)=3$; $y(4)=11$.\\
 Vậy $\min\limits_{[2;4]}y=y(2)=3$.
 }
\end{ex}

\begin{ex}%[2-D1B5-SO-17-2425]%[VN-MT-7, VM031]%[2D1N4-1]
 Đồ thị hàm số $y=\dfrac{1+2x}{x-1}$ có đường tiệm cận ngang là
 \choice
 {$x=1$}
 {$y=1$}
 {$x=2$}
 {\True $y=2$}
 \loigiai{
 Ta có $\lim\limits_{x\to \pm \infty} \dfrac{1+2x}{x-1}=\lim\limits_{x\to \pm \infty}\dfrac{\tfrac{1}{x}+2}{1-\tfrac{1}{x}}=2$.\\
 Nên $y=2$ là tiệm cận ngang của đồ thị hàm số.
 }
\end{ex}

\begin{ex}%[2-D1B5-SO-17-2425]%[VN-MT-7, VM031]%[2D1N4-1]
 Đường tiệm cận xiên của đồ thị hàm số $y=\dfrac{x^2-2x+3}{x+1}$ là
 \choice
 {\True $y=x-3$}
 {$y=x+1$}
 {$y=-3x+1$}
 {$x=-3y+1$}
 \loigiai{
 Tập xác định $\mathscr{D}=\mathbb{R}\setminus \{-1\}$.\\
 Phương trình đường tiệm cận xiên có dạng $y=ax+b$.\\
 Trong đó
 \begin{itemize}
 \item $a=\lim\limits_{x\to +\infty} \dfrac{f(x)}{x}=\lim\limits_{x\to +\infty}\dfrac{x^2-2x+3}{x^2+x}=1$;
 \item $b=\lim\limits_{x\to +\infty} \left[f(x)-ax\right]=\lim\limits_{x\to +\infty} \left(\dfrac{x^2-2x+3}{x+1}-x\right)=\lim\limits_{x\to +\infty}\dfrac{-3x+3}{x+1}=-3$.
 \end{itemize}
 Ta cũng có 
 \begin{itemize}
 \item $\lim\limits_{x\to -\infty}\dfrac{f(x)}{x}=1$;
 \item $\lim\limits_{x\to -\infty}\left[f(x)-x\right]=-3$.
 \end{itemize}
 Khi đó $\lim\limits_{x\to \pm\infty} \left[f(x)-(x-3)\right]=\lim\limits_{x\to \pm\infty} \dfrac{6}{x+1}=0$.\\
 Do đó, đồ thị hàm số có tiệm cận xiên là đường thẳng $y=x-3$.
 }
\end{ex}

\begin{ex}%[2-D1B5-SO-17-2425]%[VN-MT-7, VM031]%[2D1H4-1]
 Tổng số đường tiệm cận của đồ thị hàm số $y=\dfrac{\sqrt{x}+1}{3x-9\sqrt{x}+6}$ là
 \choice
 {\True $3$}
 {$4$}
 {$2$}
 {$1$}
 \loigiai{
 Tập xác định $\mathscr{D}=[0;+\infty)\setminus \left\{1;4\right\}$.\\
 Ta có $\lim\limits_{x\to +\infty}\dfrac{\sqrt{x}+1}{3x-9\sqrt{x}+6}=0$.\\
 Nên đồ thị hàm số có $1$ đường tiệm cận ngang là $y=0$.
 \begin{itemize}
 \item $\lim\limits_{x\to 1^+}\dfrac{\sqrt{x}+1}{3x-9\sqrt{x}+6}=\lim\limits_{x\to 1^+} \dfrac{\sqrt{x}+1}{3\left(\sqrt{x}-1\right)\left(\sqrt{x}-2\right)}=+\infty$;
 \item $\lim\limits_{x\to 1^-}\dfrac{\sqrt{x}+1}{3x-9\sqrt{x}+6}=-\infty$.
 \end{itemize}
 Suy ra đường thẳng $x=1$ là $1$ tiệm cận đứng của đồ thị hàm số.
 \begin{itemize}
 \item $\lim\limits_{x\to 4^+} \dfrac{\sqrt{x}+1}{3\left(\sqrt{x}-1\right)\left(\sqrt{x}-2\right)}=+\infty$;
 \item $\lim\limits_{x\to 4^-} \dfrac{\sqrt{x}+1}{3x-9\sqrt{x}+6}=-\infty$.
 \end{itemize}
 Suy ra đường thẳng $x=4$ là $1$ tiệm cận đứng của đồ thị hàm số.\\
 Đồ thị hàm số không có tiệm cận xiên.\\
 Vậy tổng số đường tiệm cận của đồ thị hàm số là $3$.
 }
 \end{ex}
\Closesolutionfile{ans}

\TNTF
\Opensolutionfile{ans}[ans/ansc1l4-Phan-II]
\begin{ex}%[2-D1B5-SO-17-2425]%[VN-MT-7, VM031]%[2D1V5-5]
 \immini{Cho hàm số $y=f(x)$ có đạo hàm trên $\mathbb{R}$ và hàm số $y=f'(x)$ là hàm số bậc ba có đồ thị là đường cong trong hình vẽ.
 \choiceTF
 {Hàm số $y=f(x)$ đồng biến trên khoảng $(-\infty;-2)$}
 {Hàm số $y=f(x)$ có hai điểm cực trị}
 {$f'(2)=4$}
 {\True Hàm số $g(x)=f(x)-\dfrac{1}{2} x^2+x+2024$ đồng biến trên khoảng $\left(-\dfrac{5}{2};-\dfrac{3}{2} \right)$}}
 {\begin{tikzpicture}[font=\footnotesize,line join=round, line cap=round, >=stealth,scale=0.8] 
 \def \xmin{-4}\def \xmax{2}\def \ymin{-5}\def \ymax{1} 
 \draw[->] (\xmin,0)--(\xmax,0) node[shift=(-110:0.2)] {$x$};
 \draw[->] (0,\ymin)--(0,\ymax) node[shift=(-150:0.2)] {$y$};
 \fill (0,0) circle(1pt) node[shift=(-135:0.25)]{$O$}
 (-3,0) circle(1pt) node[shift=(90:0.2)]{$-3$}
 (-2,0) circle(1pt) node[shift=(90:0.2)]{$-2$}
 (-1,0) circle(1pt) node[shift=(90:0.2)]{$-1$}
 (1,0) circle(1pt) node[shift=(110:0.22)]{$1$}
 (0,-4) circle(1pt) node[shift=(-145:0.3)]{$-4$}
 (0,-2) circle(1pt) node[shift=(-145:0.3)]{$-2$}
 (-1,-2) circle(1pt) (-3,-4) circle(1pt); 
 \clip (\xmin,\ymin) rectangle (\xmax,\ymax); 
 \draw[smooth,samples=100,domain=\xmin:\xmax] plot(\x,{(\x)^3+3*(\x)^2-4}); 
 \draw[dashed] (-3,0)|-(0,-4) (-1,0)|-(0,-2);
 \end{tikzpicture}}
 \loigiai{
 \begin{itemchoice}
 \itemch \textbf{Sai}.\\
 Vì từ đồ thị của hàm số $y=f'(x)$ ta thấy $f'(x)\ge 0$ với $\forall x\ge 1$ nên hàm số đồng biến trên khoảng $(1;+\infty)$.
 \itemch \textbf{Sai}.\\
 Vì từ đồ thị của hàm số $y=f'(x)$ ta thấy $f'(x)$ chỉ đổi dấu một lần qua $x=1$ nên hàm số có một điểm cực trị.
 \itemch \textbf{Sai}.\\
 Từ đồ thị ta có hàm số $f'(x)$ có dạng: $f'(x)=a(x+2)^2 (x-1)$.\\
 Đồ thị hàm số $y=f'(x)$ đi qua $(0;-4)$ nên $-4=a(0+2)^2 (0-1)\Leftrightarrow a=1$.\\
 Vậy $f'(x)=(x+2)^2(x-1)\Rightarrow f'(2)=(2+2)^2 (2-1)=16$.
 \itemch \textbf{Đúng}.\\
 Ta có $g'(x)=f'(x)-x+1=0\Leftrightarrow f'(x)=x-1$.\\
 Vẽ đường thẳng $y=x-1$ trên cùng hệ trục tọa độ với đồ thị hàm số $y=f'(x)$.
 \begin{center}
 \begin{tikzpicture}[line cap=butt,line join=miter,>=stealth,scale=.6,font=\footnotesize]
 \tikzset{declare function={xmin=-4;xmax=2;ymin=-5;ymax=1.5;
 f(\x)=(\x)^3+3*(\x)^2-4;
 g(\x)=(\x)-1;
 },
 smooth,samples=450
 }
 \draw[->] (xmin,0)--(xmax,0) node[shift={(-100:7pt)}]{$ x $};
 \draw[->] (0,ymin)--(0,ymax) node[shift={(190:7pt)}]{$ y $};
 \fill (0,0) node[shift={(225:9pt)}]{$ O $};
 \draw[dashed](-3,0)node[above]{$-3$}|-(0,-4)node[below left]{$-4$}
 (-1,0)node[above]{$-1$}|-(0,-2)node[below left]{$-2$}
 (-2,0)node[above]{$-2$} (1,0)node[above left]{$1$};
 \fill(-2,0) circle (1pt) node[above]{$-2$} 
 (1,0)circle (1pt) node[above left]{$1$}
 (-3,-4) circle (1pt) (-1,-2) circle (1pt) 
 (1,0) circle (1pt) (0,0) circle (1pt) (0,-4) circle (1pt) (-3,0) circle (1pt) (-1,0) circle (1pt)
 (0,-2) circle (1pt);
 \clip (xmin,ymin) rectangle (xmax,ymax);
 \draw plot[domain=xmin:xmax] (\x, {f(\x)});
 \draw plot[domain=xmin:xmax] (\x, {g(\x)});
 \end{tikzpicture}
 \end{center}
 Khi đó $f'(x)=x-1\Leftrightarrow \hoac{&x=-3\\&x=-1\\&x=1.}$\\
 Bảng biến thiên của hàm số $g(x)$
 \begin{center}
 \begin{tikzpicture}
 \tkzTabInit[nocadre=true,lgt=1.2,espcl=2.5,deltacl=0.5]
 {$x$/0.7,$f'(x)$/0.7,$f(x)$/2}
 {$-\infty$,$-3$,$-1$,$1$,$+\infty$}
 \tkzTabLine{,-,$0$,+,$0$,-,$0$,+,}
 \tkzTabVar{+/$+\infty$,-/$g(-3)$,+/$g(-1)$,-/$g(1)$,+/$+\infty$}
 \end{tikzpicture}
 \end{center}
 Ta có hàm số $g(x)$ đồng biến trên khoảng $(-3;-1)$ nên $g(x)$ đồng biến trên khoảng $\left(-\dfrac{5}{2};-\dfrac{3}{2} \right)$.
 \end{itemchoice}
 }
\end{ex}

\begin{ex}%[2-D1B5-SO-17-2425]%[VN-MT-7, VM031]%[2D1V2-1]
 Cho hàm số $y=x^3-3x+1$.
 \choiceTF
 {\True Điểm cực tiểu của hàm số là $x=1$}
 {Hàm số đồng biến trên khoảng $(-1;1)$}
 {\True Giả sử hàm số đã cho có hai điểm cực trị là $x_1$; $x_2$. Khi đó giá trị $x_1 \cdot x_2=-1$}
 {Gọi $A$, $B$ lần lượt là điểm cực đại và điểm cực tiểu của đồ thị hàm số. Khi đó, diện tích tam giác $ABC$ là $12$ với $C(-1;2)$}
 \loigiai{
 \begin{itemchoice}
 \itemch \textbf{Đúng}.\\
 Ta có $y'=3x^2-3$\\
 $y'=0\Leftrightarrow 3x^2-3=0\Leftrightarrow \hoac{&x=-1\\&x=1} \Leftrightarrow \hoac{&y(-1)=3\\&y(1)=-1.}$\\
 Ta có bảng biến thiên
 \begin{center}
 \begin{tikzpicture}
 \tkzTabInit[nocadre=true,lgt=1.2,espcl=2.5,deltacl=0.5]
 {$x$/0.7,$f'(x)$/0.7,$f(x)$/2}
 {$-\infty$,$-1$,$1$,$+\infty$}
 \tkzTabLine{,+,0,-,0,+,}
 \tkzTabVar{-/$-\infty$,+/$3$,-/$-1$,+/$+\infty$}
 \end{tikzpicture}
 \end{center}
 Từ bảng biến thiên ta có điểm cực tiểu của hàm số là $x=1$.
 \itemch \textbf{Sai}.\\
 Vì từ bảng biến thiên ta có hàm số nghịch biến trên khoảng $(-1;1)$.
 \itemch \textbf{Đúng}.\\
 Vì $x_1 \cdot x_2=1\cdot (-1)=-1$.
 \itemch \textbf{Sai}.\\
 Vì $A(-1;3)$, $B(1;-1)$, $C(-1;2)$ nên
 \begin{itemize}
 \item $\left|\overrightarrow{AB}\right|=\sqrt{2^2+(-4)^2}=2\sqrt{5}$;
 \item $\left|\overrightarrow{AC}\right|=\sqrt{0^2+(-1)^2}=1$;
 \item $\cos \widehat{BAC}=\cos\big(\overrightarrow{AB},\overrightarrow{AC}\big)=\dfrac{x_1 x_2+y_1 y_2}{\sqrt{x_1^2+y_1^2} \sqrt{x_2^2+y_2^2}}=\dfrac{2\cdot 0+(-4)(-1)}{\sqrt{2^2+(-4)^2} \sqrt{0^2+(-1)^2}}=\dfrac{2\sqrt{5}}{5}$.
 \item $\sin \widehat{BAC}=\sqrt{1-\cos^2 \widehat{BAC}}=\dfrac{\sqrt{5}}{5}$.
 \item $S_{\triangle ABC}=\dfrac{1}{2}\cdot AB\cdot AC\cdot\sin \widehat{BAC}=\dfrac{1}{2}\cdot 2\sqrt{5}\cdot 1\cdot \dfrac{\sqrt{5}}{5}=1$.
 \end{itemize}
 \end{itemchoice}
 }
\end{ex}

\begin{ex}%[2-D1B5-SO-17-2425]%[VN-MT-7, VM031]%[2D1V3-1]
 Cho hàm số $y=\dfrac{x+m}{x-1}$ ($m$ là tham số thực).
 \choiceTF
 {\True Khi $m=2$ thì giá trị lớn nhất của hàm số trên đoạn $[2;5]$ là $4$}
 {\True Khi $m=2$ thì giá trị nhỏ nhất của hàm số trên đoạn $[2;5]$ là $\dfrac{7}{4}$}
 {Khi $m <-1$ thì giá trị nhỏ nhất của hàm số trên đoạn $[2;4]$ là $y(4)$}
 {Khi $\min\limits_{[2;4]}y=3$ thì giá trị của tham số $m$ là $1\le m < 3$}
 \loigiai{
 Tập xác định $\mathscr{D}=\mathbb{R}\setminus \{1\}$.\\
 Ta có $y'=\dfrac{-1-m}{(x-1)^2}$.
 \begin{itemchoice}
 \itemch \textbf{Đúng}.\\
 Khi $m=2$ thì $y'=\dfrac{-1-2}{\left(x-1\right)^2}=\dfrac{-3}{\left(x-1\right)^2} < 0\; \forall x\in \mathscr{D}\Rightarrow$ hàm số nghịch biến trên từng khoảng xác định, do đó hàm số cũng nghịch biến trên $[2;5]$.\\
 Vậy $\max\limits_{[2;5]}y=y(2)=4$.
 \itemch \textbf{Đúng}.\\
 Khi $m=2$ thì $y'=\dfrac{-1-2}{\left(x-1\right)^2}=\dfrac{-3}{\left(x-1\right)^2} < 0\; \forall x\in \mathscr{D}\Rightarrow$ hàm số nghịch biến trên từng khoảng xác định, do đó hàm số cũng nghịch biến trên $[2;5]$.\\
 Vậy $\min\limits_{[2;5]} y=y(5)=\dfrac{7}{4}$.
 \itemch \textbf{Sai}.\\
 Với $m <-1$ $\Rightarrow-1-m > 0\Rightarrow y' > 0$ nên hàm số đã cho đồng biến trên trên từng khoảng xác định, do đó hàm số cũng đồng biến trên $\left[2;4\right]$ suy ra $\min\limits_{[2;4]}y=y(2)$.
 \itemch \textbf{Sai}.
 \begin{itemize}
 \item Trường hợp 1.\\
 $-1-m > 0\Leftrightarrow m <-1$ $\Rightarrow y' > 0$ nên hàm số đã cho đồng biến trên $[2;4]$.\\
 Khi đó $\min\limits_{[2;4]} y=y(2)\Leftrightarrow 3=2+m\Leftrightarrow m=1\text{ (không thoả mãn)}$.
 \item Trường hợp 2.\\
 $-1-m < 0\Leftrightarrow m >-1$ $\Rightarrow y' < 0$ nên hàm số đã cho nghịch biến trên $[2;4]$.
 \end{itemize}
 Khi đó $\min\limits_{[2;4]} y=y(4)\Leftrightarrow 3=\dfrac{4+m}{3} \Leftrightarrow m=5\text{ (thoả mãn)}$.\\
 Suy ra $m\notin [1;3)$.
 \end{itemchoice}
 }
\end{ex}

\begin{ex}%[2-D1B5-SO-17-2425]%[VN-MT-7, VM031]%[2D1V4-1]
 \immini{Cho hàm số $y=\dfrac{ax+b}{x+c}.(a,b,c\in\mathbb{R})$ có đồ thị như hình vẽ. Khi đó 
 \choiceTF
 {\True Đồ thị hàm số có tiệm cận ngang là $y=-1$}
 {\True Đồ thị hàm số có tiệm cận đứng là $x=1$}
 {$a+b+c=1$}
 {Hàm số đồng biến trên các khoảng xác định}}
 {\begin{tikzpicture}[>=stealth,scale=.6]
 \draw[->] (-4,0) --(4,0);
 \draw[->](0,-4)--(0,4);
 \draw (0,0)circle (1pt) node[above left]{$O$};
 \draw (4,0) node[below]{$x$};
 \draw (0,4) node[left]{$y$};
 \draw (1,0)circle (1pt) node[above left]{$1$};
 \draw (0,-1)circle (1pt) node[above left]{$-1$};
 \draw (2,0)circle (1pt) node[above]{$2$};
 \clip (-4,-4) rectangle(4,4);
 \draw[thick,samples=100] plot[domain=-4:4]
 (\x,{(-(\x)+2)/((\x)-1)});
 \draw[thick,samples=100] plot[domain=-4:4]
 (\x,-1);
 \draw (2,-2) node
 {$x=1$};
 \end{tikzpicture}} 
 \loigiai{
 \begin{itemchoice}
 \itemch \textbf{Đúng}.\\
 Dựa vào đồ thị ta thấy đường thẳng $y=-1$ là tiệm cận ngang.
 \itemch \textbf{Đúng}.\\
 Dựa vào đồ thị ta thấy đường thẳng $x=1$ là tiệm cận đứng.
 \itemch \textbf{Đúng}.\\
 Dựa vào đồ thị hàm số ta có 
 \begin{itemize}
 \item Tiệm cận ngang $y=-1\Rightarrow a=-1$.
 \item Tiệm cận đứng $x=1\Rightarrow c=-1$.
 \item Đồ thị hàm số đi qua điểm $(2;0)$ nên $0=\dfrac{-2+b}{2-1}\Rightarrow b=2$. 
 \end{itemize}
 Vậy $a+b+c=-1+2-1=0$.
 \itemch \textbf{Sai}.\\
 Ta có $y=\dfrac{-x+2}{x-1}\Rightarrow y'=\dfrac{-1}{(x-1)^2}< 0$, $\forall x\ne 1$.\\
 Vậy hàm số nghịch biến trên các khoảng xác định.
 \end{itemchoice}
 }
\end{ex}
\Closesolutionfile{ans}

\TNSA
\Opensolutionfile{ans}[ans/ansc1l4-Phan-III]
\begin{ex}%[Đề ôn tập chương1-3]%[Nguyễn Tấn Linh]%[2D1C1-3]
	\immini
	{
	Cho hàm số $y=f(x)$ có đạo hàm liên tục trên $\mathbb{R}$ và có đồ thị $y=f'\left(x\right)$ như hình vẽ. Đặt $g\left(x\right)=f\left(x-m\right)-\dfrac{1}{2}\left(x-m-1\right)^2+2019$, với $m$ là tham số thực. Gọi $S$ là tập hợp các giá trị nguyên dương của $m$ để hàm số $y=g\left(x\right)$ đồng biến trên khoảng $\left(5;6\right)$. Tính tổng tất cả các phần tử trong $S$.\shortans{$14$}
	}
	{
	\begin{tikzpicture}[line join = round, line cap = round,>=stealth,font=\footnotesize,scale=.75,declare function={f(\x)=\a*(\x)^3+(\b)*(\x)^2+(\c)*\x+(\d);}]
	\def\xt{-2} \def\xp{4} \def\yd{-3} \def\yt{3}
	\def\a{1}
	\def\b{-3}
	\def\c{0}
	\def\d{2}
	\draw[->] (\xt,0)--(\xp,0) node[below]{$x$};
	\draw[->] (0,\yd)--(0,\yt) node[left]{$y$};
	\fill (0,0) circle (1.5pt) node[below left]{$O$};
	\begin{scope}
	\clip (\xt+0.1,\yd+0.1) rectangle (\xp-0.1,\yt-0.1);
	\draw[samples=150,smooth,domain=\xt:\xp] plot(\x,{f(\x)});
	\end{scope}
	\draw[dashed] (-1,0) node[above]{$-1$}|-(2,-2)--(2,0) node[above]{$2$} (3,0) node[below]{$3$}|-(0,2) node[above left]{$2$};
	\node at (1,0)[below]{$1$};
	\end{tikzpicture}
	}
	\loigiai{
	Xét hàm số $g\left(x\right)=f\left(x-m\right)-\dfrac{1}{2}\left(x-m-1\right)^2+2019$.\\
	$g'\left(x\right)=f'\left(x-m\right)-\left(x-m-1\right)$.\\
	Xét phương trình $g'\left(x\right)=0$.\hfill$\left(1\right)$\\
	Đặt $x-m=t$, phương trình $\left(1\right)$ trở thành $f'\left(t\right)-\left(t-1\right)=0\Leftrightarrow f'\left(t\right)=t-1$.\hfill$\left(2\right)$\\
	Nghiệm của phương trình $\left(2\right)$ là hoành độ giao điểm của hai đồ thị hàm số $y=f'\left(t\right)$ và $y=t-1$.\\
	Ta có đồ thị các hàm số $y=f'\left(t\right)$ và $y=t-1$ như sau
	\begin{center}
	\begin{tikzpicture}[line join = round, line cap = round,>=stealth,font=\footnotesize,scale=.75,declare function={f(\x)=\a*(\x)^3+(\b)*(\x)^2+(\c)*\x+(\d);}]
	\def\xt{-2} \def\xp{4} \def\yd{-3} \def\yt{3}
	\def\a{1}
	\def\b{-3}
	\def\c{0}
	\def\d{2}
	\draw[->] (\xt,0)--(\xp,0) node[below]{$x$};
	\draw[->] (0,\yd)--(0,\yt) node[left]{$y$};
	\fill (0,0) circle (1.5pt) node[below left]{$O$};
	\begin{scope}
	\clip (\xt+0.1,\yd+0.1) rectangle (\xp-0.1,\yt-0.1);
	\draw[samples=150,smooth,domain=\xt:\xp] plot(\x,{f(\x)});
	\draw[samples=150,smooth,domain=\xt:\xp] plot(\x,{\x-1});
	\end{scope}
	\draw[dashed] (-1,0) node[above]{$-1$}|-(2,-2)--(2,0) node[above]{$2$} (3,0) node[below]{$3$}|-(0,2) node[above left]{$2$};
	\node at (1,0)[below]{$1$};
	\end{tikzpicture}
	\end{center}
	Căn cứ đồ thị các hàm số ta có phương trình $\left(2\right)$ có nghiệm là $\hoac{&t=-1 \\&t=1 \\&t=3}\Rightarrow \hoac{&x=m-1 \\&x=m+1 \\&x=m+3.}$\\
	Ta có bảng biến thiên của $y=g\left(x\right)$
	\begin{center}
	\begin{tikzpicture}
	\tkzTabInit[lgt=2,espcl=2.5]{$x$/1, $y'$/1, $y$/2}{$-\infty$, $m-1$, $m+1$, $m+3$, $+\infty$}
	\tkzTabLine{,-,$0$,+,$0$,-,$0$,+,}
	\tkzTabVar{+/ $+\infty$ , -/ , +/ , -/ , +/ $+\infty$}
	\end{tikzpicture}
	\end{center}
	Để hàm số $y=g\left(x\right)$ đồng biến trên khoảng $\left(5;6\right)$ cần $\hoac{&\heva{&m-1\le 5 \\&m+1\ge 6} \\&m+3\le 5}\Leftrightarrow \hoac{&5\le m\le 6 \\&m\le 2.}$\\
	Vì $m\in \mathbb{N}^*\Rightarrow S=\{1;2;5;6\}\Rightarrow$ Tổng các phần tử trong $S$ bằng $14$.}
	\end{ex}

\begin{ex}%[2-D1B5-SO-17-2425]%[VN-MT-7, VM031]%[2D1V3-6]
 Trong một trò chơi, mỗi đội chơi được phát một tấm bìa hình chữ nhật kích thước 21 cm, 29,5 cm. Nhiệm vụ của mỗi đội là cắt ở bốn góc của tấm bìa này bốn hình vuông bằng nhau, rồi gập tấm bìa lại và dán keo để được một cái hộp không nắp có dạng hình hộp chữ nhật như hình vẽ. 
 \begin{center}
 \begin{tabular}{p{8cm}p{8cm}}
 \begin{tikzpicture}[scale=0.6, font=\footnotesize, line join=round, line cap=round, >=stealth]
 \foreach \x/\y/\pos in {0/0/A, 1/0/B, 1/1/C, 0/1/D, 6/0/A', 0/4/M, 1/4/N, 1/5/P, 0/5/Q, 6/4/M'} \path ($(\x,\y)$) coordinate (\pos); 
 \coordinate (B') at ($(B)+(A')-(A)$);
 \coordinate (C') at ($(C)+(A')-(A)$);
 \coordinate (D') at ($(D)+(A')-(A)$);
 \coordinate (N') at ($(N)+(M')-(M)$);
 \coordinate (P') at ($(P)+(M')-(M)$);
 \coordinate (Q') at ($(Q)+(M')-(M)$);
 \fill[gray!40] (A)--(B)--(C)--(D)--cycle;
 \fill[gray!40] (A')--(B')--(C')--(D')--cycle;
 \fill[gray!40] (M)--(N)--(P)--(Q)--cycle;
 \fill[gray!40] (M')--(N')--(P')--(Q')--cycle;
 \draw (A)--(B')--(P')--(Q)--(A) (D)--(C') (M)--(N') (B)--(P) (A')--(Q');
 \draw[|<->|] ([xshift=-3mm]A)--([xshift=-3mm]Q)node[pos=.5,left]{$21$\,cm};
 \draw[|<->|] ([yshift=3mm]Q)--([yshift=3mm]P')node[pos=.5,above]{$29{,}5$\,cm};
 \end{tikzpicture} & 
 \begin{tikzpicture}[scale=0.6, font=\footnotesize, line join=round, line cap=round, >=stealth]
 \foreach \x/\y/\pos in {0/0/A, -1.8/-1.8/B, 3/-1.8/C} \path ($(\x,\y)$) coordinate (\pos); 
 \coordinate (D) at ($(A)+(C)-(B)$);
 \coordinate (A') at ($(A)+(0,1)$);
 \coordinate (B') at ($(B)+(A')-(A)$);
 \coordinate (C') at ($(C)+(A')-(A)$);
 \coordinate (D') at ($(D)+(A')-(A)$); 
 \draw (A')--(B')--(B)--(C)--(D)--(D')--(A');
 \draw (B')--(C')--(D') (C)--(C');
 \draw [dashed] (B)--(A)--(D) (A)--(A');
 \end{tikzpicture}
 \end{tabular}
 \end{center}
%\begin{center}
% \hspace{3cm}
% 
%\end{center}
 Đội nào thiết kế được chiếc hộp có thể tích lớn nhất sẽ dành chiến thắng. Hãy xác định cạnh của hình vuông bị cắt để thu được hộp có thể tích lớn nhất. (Coi mép dán không đáng kể, kết quả làm tròn đến hàng phần trăm).
 \shortans{4{,}03}
 \loigiai{
 Gọi cạnh của hình vuông bị cắt ở bốn góc là $x$.\\
 Điều kiện $0< 2x < 21\Leftrightarrow 0< x < 10,5$, đơn vị cm.\\
 Ta có kích thước của khối hộp chữ nhật là $x$; $21-2x$;\ $29{,}5-2x$.\\
 Thể tích của khối hộp là $V=(21-2x)\cdot (29{,}5-2x)\cdot x=619{,}5x-101x^2+4x^3=f(x)$.\\
 Thể tích khối hộp lớn nhất khi hàm số $f(x)$ đạt giá trị lớn nhất.\\
 Xét hàm số $f(x)=619{,}5x-101x^2+4x^3$ trên khoảng $(0;10{,}5)$
 \begin{eqnarray*}
 &&f'(x)=12x^2-202x+619{,}5=0\\
 &\Leftrightarrow& \hoac{&x_1 \approx 4{,}03\\&x_2 \approx 12{,}80.}
 \end{eqnarray*}
 Ta có bảng biến thiên
 \begin{center}
 \begin{tikzpicture}
 \tkzTabInit[nocadre=true,lgt=1.2,espcl=2.5,deltacl=0.5]
 {$x$/0.7,$f'(x)$/0.7,$f(x)$/2}
 {$0$,$x_1$,$10{,}5$}
 \tkzTabLine{,+,0,-,}
 \tkzTabVar{-/$0$,+/$f(x_1)$,-/$f(10{,}5)$}
 \end{tikzpicture}
 \end{center}
 Suy ra $\max\limits_{(0;10{,}5)} f(x)=f(x_1)$.\\
 Vậy cạnh của hình vuông xấp xỉ $4{,}03$ cm.
 }
\end{ex}

\begin{ex}%[2-D1B5-SO-17-2425]%[VN-MT-7, VM031]%[2D1H2-1]
 Điểm cực tiểu $x_{\text{CT}}$ của hàm số $y=x^3+3x^2-9x$ là
 \shortans{1}
 \loigiai{
 Ta có $y'=3x^2+6x-9=0$.\\
 $y'=0 \Leftrightarrow \hoac{&x=1\\&x=-3} \Rightarrow \hoac{&y(1)=-5\\&y(-3)=27.}$\\
 Bảng biến thiên
 \begin{center}
 \begin{tikzpicture}
 \tkzTabInit[nocadre=true,lgt=1.2,espcl=2.5,deltacl=0.5]
 {$x$/0.7,$f'(x)$/0.7,$f(x)$/2}
 {$-\infty$,$-3$,$1$,$+\infty$}
 \tkzTabLine{,+,0,-,0,+,}
 \tkzTabVar{-/$-\infty$,+/$27$,-/$-5$,+/$+\infty$}
 \end{tikzpicture}
 \end{center}
 Vậy $x=1$ là điểm cực tiểu.
 }
\end{ex}

\begin{ex}%[Vovanle]%[2D1V5-4]
	Một đường thẳng cắt đồ thị hàm số $y=3x^4-4x^2$ tại bốn điểm phân biệt có hoành độ $0;1;a;b$. Tính $S=ab-a-b$. (làm tròn 2 chữ số thậm phân)
	\shortans{$0{,}67$}
	\loigiai{
		Đường thẳng $d$ cắt đồ thị $(C)$ của hàm số $y=f(x)=3x^4-4x^2$ lần lượt tại các điểm $A$, $B$ có hoành độ $0;1$ nên
		$y_A=f(0)=0$; $y_B=f(1)=-1$.\\
		$\Rightarrow A(0;0),\,B(1;-1)$.\\
		Suy ra PTĐT $d$ là $y=-x$.\\
		Phương trình hoành độ giao điểm của $d$ và $(C)$ là
		\allowdisplaybreaks
		\begin{eqnarray*}
			&&3x^4-4x^2=-x\\
			& \Leftrightarrow& 3x^4-4x^2+x=0 \\ 
			& \Leftrightarrow& x\left(3x^3-4x+1\right)=0 \\ 
			& \Leftrightarrow& x(x-1)\left(3x^2+3x-1\right)=0 \\ 
			&\Leftrightarrow& \hoac{& x=0\\& x-1=0 \\& 3x^2+3x-1=0}\Leftrightarrow \hoac{& x=0 \\& x=1\\& x=\dfrac{-3-\sqrt{21}}{6}\\& x=\dfrac{-3+\sqrt{21}}{6}.}
		\end{eqnarray*}
		Từ đó suy ra $a=\dfrac{-3-\sqrt{21}}{6};b=\dfrac{-3+\sqrt{21}}{6}\Rightarrow S=ab-a-b=\dfrac{2}{3}$.\\
		\textbf{Nhận xét:} Do biểu thức $S$ đối xứng nên ta có thể áp dụng định lí Vi-ét để tính nhanh hơn\\
		Cụ thể $a,\,b$ là nghiệm của phương trình $3x^2+3x-1=0$ nên $ab=-\dfrac{1}{3};\,a+b=-1$.\\
		Từ đó suy ra $S=ab-a-b=ab-(a+b)=-\dfrac{1}{3}-(-1)=\dfrac{2}{3}\approx0{,}67$.
	}
\end{ex}

\begin{ex}%[2-D1B5-SO-17-2425]%[VN-MT-7, VM031]%[2D1V3-1]
 Cho hàm số $y=\dfrac{x-m^2-1}{x-m}$ có bao nhiêu giá trị nguyên $m$ thỏa mãn $\max\limits_{[0;4]}y=-6$.
 \shortans{1}
 \loigiai{
 Tập xác định $\mathscr{D}=\mathbb{R}\setminus \{m\}$.\\
 Ta có $y'=\dfrac{m^2-m+1}{\left(x-m\right)^2} > 0$, $\forall x\in \mathscr{D}$ (do $m^2-m+1=\left(m-\dfrac{1}{2} \right)^2+\dfrac{3}{4} > 0$, $\forall m\in \mathbb{R}$).\\
 Do đó hàm số đồng biến trên các khoảng $(-\infty; m)$ và $(m;+\infty)$.\\
 Khi đó $\max\limits_{[0;4]}y=y(4)$.\\
 Để hàm số đã cho có giá trị lớn nhất trên $[0;4]$ bằng $-6$ thì
 \[\heva{&m\notin [0;4]\\&y(4)=-6} \Leftrightarrow \heva{&m\notin [0;4]\\
 &\dfrac{3-m^2}{4-m}=-6} \Leftrightarrow \heva{&m\notin [0;4]\\
 &m^2+6m-27=0} \Leftrightarrow \heva{&m\notin [0;4]\\&\hoac{&m=3\\&m=-9.}} \Leftrightarrow m=-9.
 \]
 Vậy có $1$ giá trị của $m$ thỏa mãn yêu cầu bài toán.
 }
\end{ex}

\begin{ex}%[2-D1B5-SO-17-2425]%[VN-MT-7, VM031]%[2D1V4-2]
 Biết tích các giá trị của tham số $m$ để đồ thị của hàm số $y=\dfrac{2x-4}{x^2+2(m-2)x+m^2+1}$ có đúng $2$ đường tiệm cận là $\dfrac{a}{b}$, $\dfrac{a}{b}$ là phân số tối giản. Tính $P=a^2+b^2$.
 \shortans{85}
 \loigiai{
 Đặt $f(x)=x^2+2(m-2)x+m^2+1$.\\
 Dễ thấy đồ thị không có tiệm cận xiên.\\
 Đồ thị có $1$ tiệm cận ngang là $y=0$ do $\lim\limits_{x\to+\infty} \dfrac{2x-4}{x^2+2(m-2)x+m^2+1}=0$.\\
 Do đó, để đồ thị hàm số có đúng hai đường tiệm cận thì đồ thị hàm số chỉ có đúng $1$ đường tiệm cận đứng.\\
 Khi đó, $f(x)=0$ có $2$ nghiệm phân biệt trong đó có $1$ nghiệm $x=2$ hoặc $f(x)=0$ có nghiệm kép
 \begin{eqnarray*}
 &\Leftrightarrow& \hoac{&\heva{&\Delta' > 0\\&f(2)=0}\\&\Delta '=0} \Leftrightarrow\hoac{&\heva{&(m-2)^2-m^2-1> 0\\&4+2(m-2)\cdot 2+m^2+1=0}\\&(m-2)^2-m^2-1=0}\\
 &\Leftrightarrow&\hoac{&\heva{&-4m+3> 0\\&m^2+4m-3=0} \\&-4m+3=0} \Leftrightarrow \hoac{&\heva{&m < \dfrac{3}{4} \\&m=-2\pm \sqrt{7}} \\&m=\dfrac{3}{4}} \Leftrightarrow \hoac{&m=-2\pm \sqrt{7} \\&m=\dfrac{3}{4}.}
 \end{eqnarray*}
 Vậy tích tất cả các giá trị thực của tham số $m$ là $P=\left(-2+\sqrt{7} \right)\cdot\left(-2-\sqrt{7} \right)\cdot\dfrac{3}{4}=-3\cdot\dfrac{3}{4}=\dfrac{-9}{4}$.\\
 Do đó $a=-9$, $b=4$ nên $P=a^2+b^2=81+4=85$.
 }
\end{ex}
\Closesolutionfile{ans}
% \begin{indapan}
% 	{ans/ansc1l4}
% \end{indapan}


% \begin{name}
	{\tenchude}
	{ĐỀ ÔN TẬP CHƯƠNG I}
	{LỚP TOÁN THẦY PHÁT}
	{\thoigian}
\end{name}

\TN
\Opensolutionfile{ans}[ans/ansc101]
\Opensolutionfile{ans}[ans/ansDe1-TN1]
\begin{ex}%[2D1H5-3]
	Cho hàm số $y=f(x)$ có đồ thị như hình. Tìm số nghiệm của phương trình $2f(x)+3=0$.
	\begin{center}
		\begin{tikzpicture}[line join=round, line cap = round, >=stealth, scale=1,font=\footnotesize,transform shape]
			\pgfmathsetmacro\a{sqrt(2)}
			\draw[->] (-2.5,0) -- (2.5,0)node[above]{$x$};
			\draw[->] (0,-2.5) -- (0,2.5)node[right]{$y$};
			%\draw[-] (-2.1,-1.5) -- (2.2,-1.5) node[right]{$y=-\frac{3}{2}$};
			\draw[fill=black]
			(0,0) circle(1pt) node[below right]{$O$}
			(0,2) circle(1pt) node[above left]{$2$}
			(0,-2) circle(1pt) node[below left]{$-2$}
			(-\a,0) circle(1pt) node[above]{$-\sqrt{2}$}
			(\a,0) circle(1pt) node[above]{$\sqrt{2}$}
			;
			\draw[dashed]
			(-\a,0)--(-\a,-2)--(\a,-2)--(\a,0)
			;
			\draw[smooth,samples=100,domain=-2.005:2.005] plot(\x,{(\x)^4-4*(\x)^2+2});
		\end{tikzpicture}
	\end{center}
	\choice
	{\True $4$}
	{$2$}
	{$0$}
	{$3$}
	\loigiai{
		Ta có $2f(x)+3=0 \Leftrightarrow f(x)=-\dfrac{3}{2}.\quad (*)$
		\begin{center}
			\begin{tikzpicture}[line join=round, line cap = round, >=stealth, scale=1,font=\footnotesize,transform shape]
				\pgfmathsetmacro\a{sqrt(2)}
				\draw[->] (-2.5,0) -- (2.5,0)node[above]{$x$};
				\draw[->] (0,-2.5) -- (0,2.5)node[right]{$y$};
				\draw[-] (-2.1,-1.5) -- (2.2,-1.5) node[right]{$y=-\frac{3}{2}$};
				\draw[fill=black]
				(0,0) circle(1pt) node[below right]{$O$}
				(0,2) circle(1pt) node[above left]{$2$}
				(0,-2) circle(1pt) node[below left]{$-2$}
				(-\a,0) circle(1pt) node[above]{$-\sqrt{2}$}
				(\a,0) circle(1pt) node[above]{$\sqrt{2}$}
				;
				\draw[dashed]
				(-\a,0)--(-\a,-2)--(\a,-2)--(\a,0)
				;
				\draw[smooth,samples=100,domain=-2.005:2.005] plot(\x,{(\x)^4-4*(\x)^2+2});
			\end{tikzpicture}
		\end{center}
		Số nghiệm của phương trình $(*)$ là số giao điểm của đồ thị hàm số $f(x)$ và đường thẳng nằm ngang $y=-\dfrac{3}{2}$. Quan sát hình vẽ, nhận thấy số giao điểm là $4$. Suy ra số nghiệm của phương trình là $4$.
	}
\end{ex}

\begin{ex}%[2D1N5-3]
	\immini
	{
		Cho hàm số có đồ thị là đường cong trong hình bên. Tọa độ giao điểm của đồ thị hàm số đã cho và trục tung là
		\choice
		{$(0;-2)$}
		{$(-1;0)$}
		{\True $(0;-1)$}
		{$(-2;0)$}
	}
	{
		\begin{tikzpicture}[>=stealth,x=1cm,y=1cm,scale=1,font=\footnotesize,line cap=round,line join=round]
			\draw[->] (-2.5,0)--(0,0)%
			node[below right]{$O$}--(2.5,0) node[below]{$x$};
			\draw[->] (0,-3) --(0,2) node[right]{$y$};
			\foreach \x in {-2,2}{
					\draw[fill=black] (\x,0) node[above]{$\x$} circle (1pt);%Ox
				}
			\foreach \y in {-2}{
					\draw[fill=black] (0,\y) node[below left]{$\y$} circle (1pt);%Oy
				}
			\draw[fill=black] (0,-1) node[above left]{$-1$} circle (1pt) (0,1) node[left] {$1$} circle (1pt);
			\draw [domain=-1.7:1.7, samples=100]%
			plot (\x, {(\x)^4-2*(\x)^2-1});
			\draw [dashed] (1,0) node[above]{$1$}%
			--(1,-2)--(-1,-2)--(-1,0)
			node[above]{$-1$};
			\draw[fill=black] (0,0) circle(1pt);
		\end{tikzpicture}
	}
	\loigiai{
		Từ đồ thị ta thấy đồ thị hàm số cắt trục tung tại điểm có tọa độ $(0;-1)$.
	}
\end{ex}

\begin{ex}%[2D1V5-4]
	Cho hàm số $y=x^3-3mx^2+\left(3m-1\right)x+6m$ có đồ thị là $(C)$. Tìm tất cả các giá trị thực của tham số $m$ để $(C)$ cắt trục hoành tại ba điểm phân biệt có hoành độ $x_1, x_2, x_3$ thỏa mãn điều kiện $x_1^2+x_2^2+x_3^2+x_1x_2x_3=20$.
	\choice
	{$m=\dfrac{3\pm \sqrt{33}}{3}$}
	{$m=\dfrac{2\pm \sqrt{3}}{3}$}
	{$m=\dfrac{5\pm \sqrt{5}}{3}$}
	{\True $m=\dfrac{2\pm \sqrt{22}}{3}$}
	\loigiai{
		Phương trình hoành độ giao điểm của $(C)$ và trục hoành là
		\begin{eqnarray*}
			& & x^3-3mx^2+\left(3m-1\right)x+6m=0\\
			&\Leftrightarrow & \left(x+1\right)\left(x^2-\left(3m+1\right)x+6m\right)=0\\
			&\Leftrightarrow & \hoac{& x=-1=x_3 \\
				& g(x)=x^2-\left(3m+1\right)x+6m=0.\quad (*)
			}
		\end{eqnarray*}
		Điều kiện để $(C)$ cắt trục hoành tại ba điểm phân biệt có hoành độ $x_1, x_2, x_3$ là $(*)$ có $2$ nghiệm phân biệt khác $-1$. Khi đó ta có
		\[ \heva{& \Delta >0 \\
				& g\left(-1\right)\ne 0}\Leftrightarrow \heva{& 9m^2-18m+1>0 \\
				& 9m+2\ne 0}\Leftrightarrow \heva{& \hoac{& m<\dfrac{3-2\sqrt{2}}{3} \\
					& m>\dfrac{3+2\sqrt{2}}{3}} \\
				& m\ne -\dfrac{2}{9}
				.}\]
		Khi đó
		\begin{eqnarray*}
			& & x_1^2+x_2^2+x_3^2+x_1x_2x_3=20\\
			&\Leftrightarrow & x_1^2+x_2^2-x_1x_2=19\\
			&\Leftrightarrow & \left(x_1+x_2\right)^2-3x_1x_2-19=0\\
			&\Leftrightarrow & \left(3m+1\right)^2-18m-19=0\\
			&\Leftrightarrow &9m^2-12m-18=0\\
			&\Leftrightarrow & m=\dfrac{2\pm \sqrt{22}}{3} \text{ (thỏa mãn điều kiện)}.
		\end{eqnarray*}
	}
\end{ex}

\begin{ex}%[BG-12NEW-4in1, Nguyen Huynh]%[2D1H4-1]
	Đồ thị của hàm số nào dưới đây \textbf{không} có tiệm cận ngang?
	\choice
	{$y=3^x$}
	{$y=\dfrac{\sqrt{x^2+1}}{2x+3}$}
	{\True $y=\log_3x$}
	{$y=\dfrac{1}{1+x}$}
	\loigiai{
		Hàm số $y=\log_3x$ có tập xác định $(0; +\infty)$ và $\displaystyle\lim\limits_{x\to +\infty}y=+\infty$ nên đồ thị hàm số không có tiệm cận ngang.\\
		Hàm số $y=3^x$ có tập xác định $(-\infty; +\infty)$ và $\displaystyle\lim\limits_{x\to -\infty}y=0$ nên đồ thị hàm số có tiệm cận ngang là đường thẳng $y=0$.\\
		Hàm số $y=\dfrac{1}{1+x}$ có tập xác định $(-\infty;-1)\cup (-1; +\infty)$ và $\displaystyle\lim\limits_{x\to +\infty}y=\displaystyle\lim\limits_{x\to -\infty}y=0$ nên đồ thị hàm số có tiệm cận ngang là đường thẳng $y=0$.\\
		Hàm số $y=\dfrac{\sqrt{x^2+1}}{2x+3}$ có tập xác định $\left(-\infty;-\dfrac{3}{2}\right)\cup \left(-\dfrac{3}{2}; +\infty\right)$ và $\displaystyle\lim\limits_{x\to +\infty}y=\dfrac{1}{2}$ và $\displaystyle\lim\limits_{x\to -\infty}y=-\dfrac{1}{2}$ nên đồ thị hàm số có $2$ đường tiệm cận ngang là đường thẳng $y=-\dfrac{1}{2}$ và $y=\dfrac{1}{2}$.
	}
\end{ex}

\begin{ex}%[MĐ2]%[2D1H2-6]
	Hàm số $y=\ln \left(x^3-3x^2+1\right)$ có bao nhiêu điểm cực trị?
	\choice
	{$ 2 $}
	{$ 3 $}
	{$ 0 $}
	{\True $ 1 $}
	\loigiai
	{
		Điều kiện xác định $ x^3-3x^2+1>0 $\\
		Ta có $ y'=\dfrac{3x^2-6x}{x^3-3x^2+1} $, $ y'=0\Leftrightarrow \hoac{& x=0 \\ & x=2 \text{ (không thỏa mãn)}.} $\\
		Ta có $ y''=\dfrac{-3x^4+12x^3-18x^2+6x-6}{\left(x^3-3x^2+1\right)^2} $,
		nên $ y''(0)=-6<0 $ do đó hàm số đạt cực đại tại $ x=0 $.\\
		Hàm số đã cho có một điểm cực trị.
	}
\end{ex}

\begin{ex}%[Mức độ N]%[2D1N1-1]
	Cho hàm số $y=f(x)$ có đạo hàm $f'(x)=x^2+1\text{, } \forall x \in \mathbb{R}$. Mệnh đề nào dưới đây đúng?
	\choice{Hàm số nghịch biến trên khoảng $(1;+\infty)$}{Hàm số nghịch biến trên khoảng $(-1;1)$}{\True Hàm số đồng biến trên khoảng $(-\infty;+\infty)$}{Hàm số nghịch biến trên khoảng $(-\infty;0)$}
	\loigiai{Vì $f'(x)=x^2+1>0,$ $\forall x \in \mathbb{R}$ nên hàm số đồng biến trên khoảng $(-\infty;+\infty)$.}
\end{ex}

\begin{ex}%[SGK 12 - CTST, Mức độ 2]%[BG12-4IN1, Nguyễn Khánh Trọng]%[2D1H3-6]
	Khi làm nhà kho, bác An muốn cửa sổ có dạng hình chữ nhật với chu vi bằng $4 \mathrm{~m}$. Tìm kích thước khung cửa sổ sao cho diện tích cửa sổ lớn nhất (để hứng được nhiều ánh sáng nhất)?
	\choice
	{$3$ m}
	{\True $1$ m}
	{$2$ m}
	{$1{,}5$ m}
	\loigiai{
		Gọi chiều dài của khung cửa sổ là $x$ (mét). Điều kiện $0<x<2$.\\
		Suy ra chiều rộng của khung cửa sổ là $2-x$ (mét).\\
		Khi đó diện tích của khung cửa sổ là $x\left(2-x\right)=-x^2+2x$.\\
		Đặt $f(x)=-x^2+2x\Rightarrow f'(x)=-2x+2=0\Leftrightarrow x=1$. Ta có bảng biến thiên như sau
		\begin{center}
			\begin{tikzpicture}[font=\normalsize,t style/.style={style=solid}]
				%dòng khai báo
				\tkzTabInit[lgt=1.2,espcl=2.5,deltacl=0.5]
				{$x$ /0.75, $f'(x)$/0.75, $f(x)$/2}
				{$ 0$,$ 1 $,$ 2$}
				%dòng xét dấu
				\tkzTabLine{  , +,0 , -,  }  % z, t, d;
				%dòng biến thiên
				\tkzTabVar{-/$0$,+/$1$,-/$0$} %+ hoac-
			\end{tikzpicture}
		\end{center}
		Như bảng biến thiên ta thấy được diện tích khung của sổ lớn nhất khi $x=1$ hay khung cửa có dạng hình vuông cạnh $1$ mét.}
\end{ex}

\begin{ex}%[Mức độ 2]%[2D1H1-5]
	Sau khi phát hiện một bệnh dịch, các chuyên gia y tế ước tính số người nhiễm bệnh kể từ ngày xuất hiện bệnh nhân đầu tiên đến ngày thứ $t$ là $f(t)=45t^2-t^3$ (kết quả khảo sát được trong 8 tháng vừa qua). Xem $f'(t)$ là tốc độ truyền bệnh (người/ngày) tại thời điểm $t$.
	\choice
	{\True Từ ngày đầu tiên đến ngày thứ 10 tốc độ truyền bệnh tăng dần}
	{Từ ngày thứ 10 đến ngày thứ 20 tốc độ truyền bệnh giảm dần}
	{Từ ngày thứ 15 đến ngày thứ 20 tốc độ truyền bệnh tăng dần}
	{Từ ngày thứ 15 đến ngày thứ 20 tốc độ truyền bệnh tăng dần rồi giảm dần kể từ ngày thứ 21}
	\loigiai
	{
		$f'(t)=90t-3t^2 \ge 0 \Rightarrow 0\le t \le 30$.\\
		$
			f''(t)=90-6t=0 \Rightarrow t=15.
		$\\
		Bảng biến thiên
		\begin{center}
			\begin{tikzpicture}[scale=1, font=\footnotesize]%<DTools>
				\tkzTabInit[nocadre=false, lgt=1.2, espcl=4, deltacl=0.6]
				{$t$/0.8,$f'(t)$/0.6,$f(t)$/2}
				{$0$,$15$,$30$};
				\tkzTabLine{,+,$0$,-,};
				\tkzTabVar{-/$0$,+/$675$,-/$0$};
			\end{tikzpicture}
		\end{center}
		Từ bảng biến thiên ta thấy từ ngày đầu tiên đến ngày thứ 10 tốc độ truyền bệnh tăng dần.
	}
\end{ex}

\begin{ex}%[CKP]giảng 12-4in1, Nhật Thiện]%[2D1H2-7]
	Một công ty tiến hành khai thác $17$ giếng dầu trong khu vực được chỉ định. Trung bình mỗi giếng dầu chiết xuất được $245$ thùng dầu mỗi ngày. Công ty	có thể khai thác nhiều hơn $17$ giếng dầu nhưng	cứ khai thác thêm một giếng thì lượng dầu mỗi giếng chiết xuất được hằng ngày sẽ giảm $9$ thùng.	Để giám đốc công ty có thể quyết định số giếng cần thêm cho phù hợp với tài chính, hãy chỉ ra số giếng công ty có thể khai thác thêm để sản lượng
	dầu chiết xuất đạt cực đại.
	\choice
	{\True $5$}
	{$3$}
	{$4$}
	{$6$}
	\loigiai{
		Gọi $x$ ($x>0$) là số giếng dầu khai thác thêm.\\
		Sản lượng dầu khi khai thác thêm $x$ giếng là $(17+x)\cdot (245-9\cdot x)$ (thùng).\\
		Xét hàm số $f(x)=(17+x)(245-9x)=-9x^2+92x+4\,165$ mô tả sản lượng dầu.\\
		Ta có $f'(x)=0\Leftrightarrow -18x+92=0\Leftrightarrow x=\dfrac{46}{9}$.\\
		Bảng biến thiên
		\begin{center}
			\begin{tikzpicture}
				\tkzTabInit[nocadre=false,lgt=1.2,espcl=2.5,deltacl=0.6]
				{$x$ /0.6,$f’(x)$ /0.6,$f(x)$ /2}
				{$0$,$\tfrac{46}{9}$,$+\infty$}
				\tkzTabLine{,+,0,-,}
				\tkzTabVar{-/,+/$\dfrac{39\,601}{9}$,-/}
			\end{tikzpicture}
		\end{center}
		Dựa vào bảng biến thiên, để sản lượng dầu chiết suất đạt cực đại, công ty có thể khai thác thêm $5$ giếng dầu.
	}
\end{ex}

\begin{ex}%[Mức độ 3]giảng 12, Phạm Tiến Long]%[2D1V4-3]
	Gọi $d$ là tiệm cận xiên của đồ thị hàm số $f(x)=\dfrac{mx^2+nx+1}{x-1}$, với $m$, $n$ là tham số. Biết rằng $d$ song song với đường thẳng $\Delta \colon y=3x+2$ và đi qua điểm $M(-1;4)$. Khi đó $m+n$ bằng
	\choice
	{$5$}
	{$6$}
	{\True $7$}
	{$8$}
	\loigiai{	Hàm số đã cho có tập xác định $\mathscr{D}=\mathbb{R}\backslash\{1\}$.\\
		Ta có $\begin{aligned}[t]
				a & =\lim\limits_{x \rightarrow+\infty} \dfrac{f(x)}{x}=\lim\limits_{x \rightarrow+\infty} \dfrac{mx^2+nx+1}{x^2-x}=m;                                                                  \\
				b & =\lim\limits_{x \rightarrow+\infty}[f(x)-ax]=\lim\limits_{x \rightarrow+\infty}\left(\dfrac{mx^2+nx+1}{x-1}-mx\right)=\lim\limits_{x \rightarrow+\infty} \dfrac{(m+n)x+1}{x-1}=m+n.
			\end{aligned}$\\
		Ta cũng có $\lim\limits_{x \rightarrow-\infty} \dfrac{f(x)}{x}=m$; $\lim\limits_{x \rightarrow-\infty}[f(x)-x]=m+n$.\\
		Do đó, tiệm cận xiên của đồ thị hàm số là đường thẳng $d\colon y=mx+m+n$.\\
		Vì $d$ song song với đường thẳng $\Delta \colon y=3x+2$ và đi qua điểm $M(-1;4)$ nên ta có
		\[\heva{&m=3\\&-m+m+n=4\\&m+n\ne 2}\Leftrightarrow \heva{&m=3\\&n=4.}\]
		Vậy $m+n=7$.
	}
\end{ex}

\begin{ex}%[Mức độ 1]%[BG12-4IN1, Nguyễn Khánh Trọng]%[2D1N3-4]
	\immini[thm]{
		Cho hàm số $f(x)$ liên tục trên đoạn $[-1;3]$ và có đồ thị như hình vẽ bên. Có bao nhiêu giá trị nguyên dương của tham số $m$ để bất phương trình $f(x)\ge m$ có nghiệm trên $[-1;2]$.
		\choice
		{$3$}
		{\True $2$}
		{$1$}
		{$0$}
	}
	{\begin{tikzpicture}[scale=0.8, font=\footnotesize, line join=round, line cap=round, >=stealth]
			\draw[->] (-2.1,0)--(3.5,0) node[above left] {$x$};
			\draw[->] (0,-2.5)--(0,4.0) node[below right] {$y$};
			\draw (0,0) node [below right] {$O$};
			\foreach \x in {-2,-1,1,2,3}
			\draw[thin] (\x,1pt)--(\x,-1pt) node [below] {$\x$};
			\foreach \y in {-2,-1,1,2,3}
			\draw[thin] (1pt,\y)--(-1pt,\y) node [above left] {$\y$};
			\draw[dashed,thin](2,0)--(2,-2)--(0,-2);
			\draw[dashed,thin](-1,0)--(-1,1)--(0,1);
			\draw[dashed,thin](3,0)--(3,3)--(0,3);
			\draw[line width = 0.5pt] (2,-2)--(3,3);
			\begin{scope}
				\clip (-3,-3) rectangle (4,3.5);
				\draw[samples=200,domain=-1:2,smooth,variable=\x] plot (\x,{-1*(\x)^2+0*(\x)+2});
			\end{scope}
		\end{tikzpicture}
	}
	\loigiai{
		Dựa vào đồ thị ta có $\max \limits_{[-1; 2]} f(x)=f(0)=2$.\\
		Bất phương trình $f(x)\ge m$ có nghiệm trên $[-1;2]$ khi và chỉ khi
		\[\max\limits_{[-1;2]}f(x)\ge m\Leftrightarrow 2\ge m.\]
		Suy ra $m\in\{1;2\}$. Vậy có $2$ giá trị nguyên dương của $m$ thỏa mãn.}
\end{ex}

\begin{ex}%[Dự án BG 4in1, Nguyễn Văn Nay]%[2D1H3-2]
	\immini
	{Cho hàm số $f(x)$ có đạo hàm là $f'(x)$. Đồ thị của hàm số $y=f'(x)$ cắt $Ox$ tại các điểm có hoành độ bằng $0,2$ như hình vẽ. Biết $f(2)+f(4)=f(3)+f(0)$. Giá trị nhỏ nhất của $f(x)$ trên $[0;4]$ là
		\choice
		{ $f(1)$}
		{\True $f(4)$}
		{ $f(2)$}
		{ $f(0)$}
	}
	{\begin{tikzpicture}[scale=0.75, font=\footnotesize,line join=round, line cap=round,>=stealth]
			\draw[->] (-2.5,0)--(0,0)node[below right]{$O$}--(5.5,0) node[above]{$x$};
			\draw[->] (0,-1.6)--(0,2.5) node [left]{$y$};
			\begin{scope}
				\clip (-3.5,-1.5) rectangle (7.5,2.5);
				\draw[smooth,samples=100] plot[domain=-3.5:7.5](\x,{-(0.5*\x-1)^2+1});
			\end{scope}
			\draw (2,0)node[below]{$1$}(4,0)node[below]{$2$}(4,0)node[below]{$2$};
			\fill[black] (2,0) circle (2pt);
			\fill[black] (4,0) circle (2pt);
			\fill[black] (0,0) circle (2pt);
		\end{tikzpicture}}
	\loigiai{
		Ta có bảng biến thiên của hàm số
		\begin{center}
			\begin{tikzpicture}[scale=0.7, font=\footnotesize, line join=round, line cap=round, >=stealth]
				\tkzTabInit[nocadre=false,lgt=1.5,espcl=2.5,deltacl=0.6]
				{$x$ /0.6,$y'$ /0.6,$y$ /1.8}
				{$-\infty$,$0$,$2$,$+\infty$}
				\tkzTabLine{,-,$0$,+,$0$,-,}
				\tkzTabVar{+/$+\infty$ , -/$f(0)$ , +/$f(2)$ , -/$-\infty$}
			\end{tikzpicture}
		\end{center}
		Từ bảng biến thiên ta thấy hàm số đồng biến trên $[0;2]$, hàm số nghịch biến trên $[2;4]$ \[do \]vậy ta có
		\[\heva{&f(0)<f(2) \\&f(2)>f(3)>f(4)}\Rightarrow\heva{&f(3)-f(2)<0\\&f(4)-f(0)=f(3)-f(2)<0 }\Rightarrow f(4)<f(0)\Rightarrow\heva{&f(2)>f(3)>f(4)\\&f(2)>f(0)>f(4).}\]
		Vậy $\max\limits_{[0;4]}f(x)=f(4)$.}
\end{ex}

\begin{ex}%[Mức độ 2]%[2D1H1-5]
	\immini{
		Một vật được ném từ mặt đất lên trời xiên góc $\alpha$ so với phương nằm ngang với vận tốc ban đầu $v_0=9$ m/s (Hình vẽ). Khi đó quỹ đạo chuyển động của vật tuân theo phương trình $y=\dfrac{-g}{2v_0^2\cos^2\alpha}x^2+x\tan \alpha$, ở đó $x$ (mét) là khoảng cách vật bay được theo phương ngang từ điểm ném, $y$ (mét) là độ cao so với mặt đất của vật trong quá trình bay, $g$ là gia tốc trọng trường (theo Vật lí đại cương, Nhà xuất bản Giáo dục Việt Nam, $2016$).
	}{
		\begin{tikzpicture}[scale=1.1, font=\footnotesize, line join=round, line cap=round, >=stealth]
			\draw[->](-0.5,0)--(3.5,0) node[below]{$x$};
			\draw[->](0,-0.5)--(0,3) node[left]{$y$};
			\draw[fill=black] (0,0) circle(1pt) node[above left]{$O$};
			\draw[black,samples=200,domain=0:3,smooth,variable=\x] plot (\x,{-1*((\x)^2)+3*(\x)});
			\path
			(0,0) coordinate (O)
			(1,0) coordinate (A)
			(0.5,1.5) coordinate (B)
			;
			\draw[->] (O)--(B) node[above]{$\overrightarrow{v}$};
			\draw[black] pic["$\alpha$", draw=black, angle eccentricity=0.5, angle radius=0.6cm]
				{angle=A--O--B};
			%					\draw(1.5,-0.5) node[below]{Hình 2.10};
		\end{tikzpicture}
	}
	Khi góc $\alpha=60^\circ$, thì $y$ đồng biến trên khoảng nào? (giả sử gia tốc trọng trường là $g=9{,}8$ m/s$^2$).
	\choice
	{\True $(0;3{,}58)$}
	{$(3{,}58;5)$}
	{$(0;4)$}
	{$(0;+\infty)$}
	\loigiai{
		Đồ thị là đường parabol có đỉnh tại $x=-\dfrac{b}{2a}=-\dfrac{\tan \alpha}{\dfrac{-g}{v_0^2\cos^2\alpha}}=\dfrac{v_0^2\cos^2\alpha \tan \alpha}{g}\approx 3{,}58$.

	}
\end{ex}

\begin{ex}%[BG12new-4in1, Trần Hoà]%[2D1H1-1]
	Cho hàm số $y=\dfrac{3-x}{x+1}$. Mệnh đề nào sau đây đúng?
	\choice
	{\True Hàm số nghịch biến trên khoảng $(-\infty;-1)$}
	{Hàm số nghịch biến trên $\mathbb{R}$}
	{Hàm số đồng biến trên khoảng $(-\infty;-1)$}
	{Hàm số đồng biến trên $\mathbb{R}$}
	\loigiai{
		Tập xác định của hàm số là $\mathscr D =\mathbb{R} \setminus \{-1\}$. Ta có $y'=-\dfrac{4}{(x+1)^2}<0,~\forall x \neq -1$.\\ Do đó, hàm số đã cho nghịch biến trên mỗi khoảng $(-\infty;-1)$, $(-1;+\infty)$.
	}
\end{ex}

\begin{ex}%[BG-12NEW-4in1, Nguyen Huynh]%[2D1N4-1]
	Tiệm cận đứng của đồ thị hàm số $y=\dfrac{x+2}{x+1}$ là
	\choice
	{\True $x=-1$}
	{$x=-2$}
	{$x=1$}
	{$x=2$}
	\loigiai{
		Tập xác định của hàm số là $\mathbb{R}\setminus\{-1\}$.\\
		Ta có
		\begin{itemize}
			\item $\lim\limits_{x\to (-1)^+}\dfrac{x+2}{x+1}=+\infty$;\\
			\item $\lim\limits_{x\to (-1)^-}\dfrac{x+2}{x+1}=-\infty$.
		\end{itemize}
		Vậy $=x-1$, là tiệm cận đứng của đồ thị.}
\end{ex}

\begin{ex}%[Dự án Giảng 12 Nhóm Toán & LaTex, Lê Minh Thiện Anh]%[2D1H5-1]
	\immini{Cho hàm số $y=a{x^3}+b{x^2}+cx+d$ $(a,\,b,\,c,\,d \in \mathbb{R})$ có bảng biến thiên như hình bên.
	Có bao nhiêu số dương trong các số $a,\,b,\,c,\,d$ ?
	\choice
	{\True $2$}
	{$4$}
	{$1$}
	{$3$}}
	{\begin{tikzpicture}[scale=1]
		\tkzTabInit[nocadre=false,lgt=1.2,espcl=2.5,deltacl=0.6]
		{$x$ /.6,$y'$/.6,$y$/2.5}{$-\infty$,$0$,$4$,$+\infty$}
		\tkzTabLine{,+,0,-,0,+,}
		\tkzTabVar{-/$-\infty$,+/$3$,-/$-5$,+/$+\infty$}
		%\tkzTabVar{+/$+\infty$,-/$-3$,+/$2$,-/$-\infty$}
	\end{tikzpicture}}
	\loigiai{
		Từ bảng biến thiên, ta có\\
		$\heva{
				&f(0)=3\\
				&f(4)=-5\\
				&f'(0)=0\\
				&f'(4)=0}\Leftrightarrow\heva{
				&d=3\\
				&64a+16b+4c+d=-5\\
				&c=0\\
				&48a+8b+c=0}\Leftrightarrow\heva{
				&a=\dfrac{1}{4}\\
				&b=-\dfrac{3}{2}\\
				&c=0\\
				&d=3.}$\\
		Vậy trong các số $ a,b,c,d$ có 2 số dương.
	}
\end{ex}

\begin{ex}%[TEX NBV, Phạm Hoài]%[2D1N1-2]
	\immini[thm]{
		Biết hàm số $y=\dfrac{x+a}{x+1}$ ($a$ là số thực cho trước, $a\neq 1$ có đồ thị như hình bên). Mệnh đề nào dưới đây đúng?
		\choice
		{$y'<0, \,\forall x\neq -1$}
		{\True  $y'>0, \,\forall x\neq -1$}
		{$y'<0, \,\forall x\in \mathbb{R}$}
		{$y'>0, \,\forall x\in \mathbb{R}$}
	}{\begin{tikzpicture}[line join=round, line cap=round,>=stealth,thick,scale=0.75]
			\tikzset{every node/.style={scale=0.9}}
			\draw[->] (-4.1,0)--(4.2,0) node[below left] {$x$};
			\draw[->] (0,-4.1)--(0,5.2) node[below left] {$y$};
			\draw (0,0) node [below left] {$O$};
			\foreach \x/\nx in {1/1,2/2,3/3}
			\draw[thin] (\x,1pt)--(\x,-1pt) node [below] {$\nx$};
			\foreach \x/\nx in {-1/-1,-2/-2}
			\draw[thin] (\x,1pt)--(\x,-1pt) node [above right] {$\nx$};
			\foreach \x/\nx in {-4/-4,-3/-3}
			\draw[thin] (\x,1pt)--(\x,-1pt) node [above] {$\nx$};
			\foreach \y/\ny in {-1/-1,1/1,2/2,3/3,4/4}
			\draw[thin] (1pt,\y)--(-1pt,\y) node [left] {$\ny$};
			\foreach \y/\ny in {-4/-4,-3/-3,-2/-2}
			\draw[thin] (1pt,\y)--(-1pt,\y) node [right] {$\ny$};
			%\draw[dashed,thin](2,0)--(2,-6)--(0,-6);
			%\draw[dashed,thin] (1.01,-10)--(1.01,2);
			\begin{scope}
				\clip (-4,-4) rectangle (4,5);
				\draw[samples=200,domain=-5:-1.1,smooth,variable=\x] plot (\x,{(-1*(\x)-3)/(2*(\x)+2)});
				\draw[samples=200,domain=-.7:5,smooth,variable=\x] plot (\x,{(-1*(\x)-3)/(2*(\x)+2)});
				\draw (-1,-4)--(-1,5) (-5,-0.5)--(5,-0.5);
			\end{scope}
		\end{tikzpicture}}
	\loigiai{Dựa vào đồ thị, hàm số đồng biến trên từng khoảng xác định. Do đó $y'>0\, \forall x\ne -1$ suy ra $1-a>0\Rightarrow a<1$.
	}
\end{ex}

\begin{ex}%[BG12, Tran Tony]%[2D1H2-2]
	\immini{
		Cho hàm số bậc ba $y=f(x)$ có đồ thị như hình vẽ. Số điểm cực trị của hàm số $y=|f(x)|$ là
		\choice
		{$3$}
		{$2$}
		{$4$}
		{\True $5$}
	}
	{
		\begin{tikzpicture}[scale=0.5, font=\footnotesize, line join=round, line cap=round, >=stealth,yscale=0.7]
			\draw[->] (-3.5,0)--(5,0) node[below]{$x$} ;
			\draw[->] (0,-3)--(0,4) node[left]{$y$};
			\draw[fill=black] (0,0) circle(1pt) node[below right=-2pt] {$O$} ;
			\clip (-3.5,-3) rectangle (5,4) ;
			\draw[smooth, samples=100, domain=-3.5:5] plot(\x,{(\x-1.3)*(\x-1.3)*(\x-1.3) - 3*(\x-1.3)}) ;
		\end{tikzpicture}
	}
	\loigiai{
		\immini{
			Từ đồ thị hàm số $y=f(x)$, ta suy ra đồ thị hàm số $y=|f(x)|$ như hình vẽ bên. Dễ thấy hàm số $y=|f(x)|$ có $5$ điểm cực trị.
		}
		{
			\begin{tikzpicture}[scale=0.5, font=\footnotesize, line join=round, line cap=round, >=stealth]
				\draw[->] (-3.5,0)--(5,0) node[below]{$x$} ;
				\draw[->] (0,-1)--(0,4) node[left]{$y$};
				\draw[fill=black] (0,0) circle(1pt) node[below right=-2pt] {$O$} ;
				\clip (-3.5,-1.5) rectangle (5,4) ;
				\draw[smooth, samples=100, domain=-3.5:-0.432051] plot(\x,{-(\x-1.3)*(\x-1.3)*(\x-1.3) + 3*(\x-1.3)}) ;
				\draw[smooth, samples=100, domain=-0.432051:1.3] plot(\x,{(\x-1.3)*(\x-1.3)*(\x-1.3) - 3*(\x-1.3)}) ;
				\draw[smooth, samples=100, domain=1.3:3.03205] plot(\x,{-(\x-1.3)*(\x-1.3)*(\x-1.3) + 3*(\x-1.3)}) ;
				\draw[smooth, samples=100, domain=3.03205:5] plot(\x,{(\x-1.3)*(\x-1.3)*(\x-1.3) - 3*(\x-1.3)}) ;
			\end{tikzpicture}
		}
	}
\end{ex}

\begin{ex}%[Dự án TL12New-4in1-NCT]%[2D1V4-2]
	Cho hàm số $y=\dfrac{2mx+m}{x-1}$. Tìm tất cả các giá trị của tham số $m$ để đường tiệm cận đứng, tiệm cận ngang của đồ thị hàm số cùng hai trục tọa độ tạo thành một hình chữ nhật có diện tích bằng $8$.
	\choice
	{$m\neq\pm 2$}
	{$m=\pm\dfrac{1}{2}$}
	{$m=2$}
	{\True $m=\pm 4$}
	\loigiai{
		Đồ thị hàm số có đường TCĐ là $x=1$ và đường TCN là $y=2m$.\\
		Diện tích hình chữ nhật tạo bởi hai đường tiện cận và hai trục tọa độ có diện tích bằng $8$ khi và chỉ khi \[1\cdot|2m|=8\Leftrightarrow m=\pm 4.\]}
\end{ex}

\begin{ex}%[2D1H5-7]
	Khoảng cách giữa hai điểm cực trị của đồ thị hàm số $y=\dfrac{x^2+x+1}{x+1}$ bằng
	\choice
	{\True $2\sqrt{5}$}
	{$2\sqrt{3}$}
	{$3\sqrt{2}$}
	{$5\sqrt{2}$}
	\loigiai{
		Tập xác định $\mathscr{D}=\mathbb{R}\setminus\{-1\}$.\\
		Ta có $y'=\dfrac{x^2+2x}{(x+1)^2}$ và $y'=0\Leftrightarrow x^2+2x=0\Leftrightarrow\hoac{& x=0 \\ & x=-2.}$
		\begin{center}
			\begin{tikzpicture}
				\tkzTabInit[nocadre=false, lgt=1.2, espcl=2.5, deltacl=0.6]{$x$/0.6,$y'$/0.6,$y$/2}
				{$-\infty$, $-2$, $-1$, $0$, $+\infty$}
				\tkzTabLine {,+,0,-,d,-,0,+,}
				\tkzTabVar{-/$-\infty$, +/$-3$, -D+/$-\infty$ /$+\infty$,-/$1$, +/$+\infty$}
			\end{tikzpicture}
		\end{center}
		Từ bảng biến thiên ta có tọa độ hai điểm cực trị là $A(-2;-3)$ và $B(0;1)$.\\
		Vậy khoảng cách giữa hai điểm cực trị là $AB=2\sqrt{5}$.
	}
\end{ex}

\begin{ex}%[Mức độ N]%[2D1N4-1]
	Cho hàm số $y=f(x)$ có bảng biến thiên như sau
	\begin{center}
		\begin{tikzpicture}
			\tkzTabInit[nocadre=false, lgt=1.5,espcl=4.5]
			{$x$/1,$f'(x)$/1,$f(x)$/2}
			{$-\infty$,$1$, $+\infty$}
			\tkzTabLine{,,d,,}
			\tkzTabVar{-/$3$, +D-/$+\infty$/$\--\infty$/,+/$3$/}
		\end{tikzpicture}
	\end{center}
	Tiệm cận đứng của đồ thị hàm số đã cho có phương trình là
	\choice
	{$x=-1$}
	{$x=-3$}
	{$x=3$}
	{\True $x=1$}
	\loigiai{Ta có $\lim\limits_{x \to 1^-}y=+\infty$; $\lim\limits_{x \to 1^+}y=-\infty$.\\
		Vậy tiệm cận đứng của đồ thị hàm số đã cho có phương trình là $x=1$.}
\end{ex}

\begin{ex}%[Mức độ 1]%[2D1N1-5]
	Đồ thị dưới mô tả sự thay đổi độ cao của một máy bay. Độ cao của máy bay giảm trong khoảng thời gian nào?
	\begin{center}
		\begin{tikzpicture}
			\pgfplotsset{/pgf/number format/use comma}
			\begin{axis}[
					title={Sự thay đổi độ cao của máy bay theo thời gian},
					title style={at={(1,1.2)},anchor=north east},
					xlabel={Thời gian (phút)},
					ylabel={Độ cao (mét)},
					xmin=0, xmax=100,
					ymin=0, ymax=12500,
					xtick={0,20,40,60,80,100},
					ytick={0,2500,5000,7500,10000,12500},
					%			yticklabel style={/pgf/number format/sci},
					legend pos=north west,
					ymajorgrids=true,
					grid style={dashed,black},
				]
				\addplot[
					color=black,
					domain=0:100,
					smooth
				]
				{500*x - 5*x^2};
			\end{axis}
		\end{tikzpicture}
	\end{center}
	\choice
	{$(0;50)$}
	{\True $(50;100)$}
	{$(0;100)$}
	{$(40;60)$}
	\loigiai{
		Từ đồ thị ta thấy độ cao máy bay giảm trong khoảng thời gian $(50;100)$ phút.
	}
\end{ex}

\begin{ex}%[2D1H5-3]
	\immini{Cho hàm số $y=f(x)$ có đồ thị như hình vẽ bên cạnh. Tìm $m$ để phương trình $f(x)=m$ có bốn nghiệm phân biệt.
		\choice
		{$-4<m\le -3$}
		{\True $-4<m<-3$}
		{$-4\le m<-3$}
		{$m>-4$}
	}{\begin{tikzpicture}[line cap=round,line join=round,>=stealth,x=1cm,
				y=1cm]
			% Vẽ 2 trục, điền các số lên trục
			\draw[->] (-3.08,0) -- (3.06,0); %Vẽ trục Ox
			\foreach \x in {1,-1} %Đánh số trên trục
			\draw[shift={(\x,0)},color=black] (0pt,2pt) -- (0pt,-2pt)
			node[above] { $\x$};
			\draw[->,color=black] (0,-5.06) -- (0,0.98); %Vẽ trục Oy
			\foreach \y in {-3,-4} %đánh số trên trục
			\draw[shift={(0,\y)},color=black] (2pt,0pt) -- (-2pt,0pt)
			node[above left] {\normalsize $\y$};
			\draw[color=black] (3,.2) node[right] {\normalsize $x$}; %đặt tên trục
			\draw[color=black] (.1,0.8) node[right] {\normalsize $y$}; %đặt tên trục
			\draw[color=black] (0pt,-8pt) node[right] {\normalsize $O$}; %gốc tọa độ
			\clip(-3.08,-4.06) rectangle (2.06,0.98); %cắt khung đồ thị
			%Vẽ đồ thị
			\draw[smooth,samples=100,domain=-2.08:2.06]
			plot(\x,{(\x)^4-2*(\x)^2-3}); %Vẽ đồ thị
			% Vẽ thêm mấy cái râu ria
			\draw[dashed] (-1,0)--(-1,-4)--(1,-4)--(1,0);
			%Vẽ dấu chấm tròn
			\fill (0cm,0cm) circle (1.5pt);
		\end{tikzpicture}}
	\loigiai{
		Từ đồ thị ta thấy  phương trình $f(x)=m$ có bốn ngiệm phân biệt khi $-4<m<-3$.}
\end{ex}

\begin{ex}%[2D1H2-7]
	Giả sử chi phí tiền xăng C (đồng) phụ thuộc tốc độ trung bình  $v\left( {{\rm{\;km}}/{\rm{h}}} \right)$ theo công thức
	\[C(v) = \dfrac{16000}{v} + \dfrac{5}{2}v \quad (0 < v \le 120)\]
	Tính tốc độ trung bình để chi phí tiền xăng đạt cực tiểu.
	\choice
	{$60$ km/h}
	{$70$ km/h}
	{$50$ km/h}
	{\True $80$ km/h}
	\loigiai{
		Tập xác định: $D=(0; 120]$.\\
		Đạo hàm $C'(v)=-\dfrac{16000}{v^2}+\dfrac{5}{2}=\dfrac{5(v-80)(v+80)}{2v^2}$; $C'(v)=0\Leftrightarrow v=-80$ (loại) hoặc $v=80$.\\
		Bảng biến thiên
		\begin{center}
			\begin{tikzpicture}[>=stealth]
				\tkzTabInit[nocadre=false,lgt=1.5,espcl=2,deltacl=0.5]{$v$/.7 ,$C'(v)$/.7,$C(v)$/2}
				{$0$ , $80$ , $120$}
				\tkzTabLine{ d, - , $0$ , + , }
				\tkzTabVar{+D+/$ $/$+\infty$ , -/$400$ , +/$\dfrac{1300}{3}$}
			\end{tikzpicture}
		\end{center}
		Quan sát bảng biến thiên, ta nhận thấy hàm số đạt cực tiểu khi $v=80$.\\
		Như vậy, để chi phí tiền xăng đạt cực tiểu, tài xế nên chạy xe với tốc độ trung bình là $80$ km/h.
	}
\end{ex}

\begin{ex}%[Mức độ 3]%[BG12-4IN1, Nguyễn Khánh Trọng]%[2D1V3-6]
	Ông An dự định làm một cái bể chứa nước hình trụ bằng inox có nắp đậy với thể tích là $k$ m$^3$ $(k>0)$. Chi phí mỗi m$^2$ đáy là $600$ nghìn đồng, mỗi m$^2$ nắp là $200$ nghìn đồng và mỗi m$^2$ mặt bên là $400$ nghìn đồng. Hỏi ông An cần chọn bán kính đáy của bể là bao nhiêu để chi phí làm bể là ít nhất? (Biết bề dày vỏ inox không đáng kể)
	\choice
	{$\sqrt[3]{\dfrac{k}{\pi}}$}
	{$\sqrt[3]{\dfrac{2\pi}{k}}$}
	{\True $\sqrt[3]{\dfrac{k}{2\pi}}$}
	{$\sqrt[3]{\dfrac{k}{2}}$}
	\loigiai{
		\immini{
			Gọi $r, h$ $(r,h>0)$ lần lượt là bán kính đáy và chiều cao của hình trụ.\\
			Thể tích khối trụ $V=\pi r^2h=k\Rightarrow h=\dfrac{k}{\pi r^2}$.\\
			Diện tích đáy và nắp là $S_d=S_n=\pi r^2$; diện tích xung quanh là $S_{xq}=2\pi rh$.\\
			Khi đó chi phí làm bể là\\
			\[C=(600+200)\pi r^2+400\cdot 2\pi rh=800\pi r^2+800\pi r\dfrac{k}{\pi r^2}=800\left(\pi r^2+\dfrac{k}{r}\right).\]
		}{
			\begin{tikzpicture}[scale=0.7, font=\footnotesize, line join=round, line cap=round, >=stealth]
				\def \x{1.8} %bán kính trục lớn elip
				\def \y{0.8} %bán kính trục bé elip
				\def \h{3.5} %chiều cao hình trụ
				\coordinate (A) at (0,0);
				\coordinate (B) at (2*\x,0);
				\coordinate (O) at ($(A)!0.5!(B)$);
				\coordinate (O') at ($(O)+(0,\h)$);
				\coordinate (A') at ($(A)+(0,\h)$);
				\coordinate (B') at ($(B)+(0,\h)$);
				\draw[dashed] (B) arc(0:180:\x cm and \y cm);
				\draw (B) arc(0:-180:\x cm and \y cm);
				\draw (O') ellipse (\x cm and \y cm);
				\coordinate (I) at ($(O)+(-30:\x cm and \y cm)$);
				\tkzDrawSegments(A,A' B,B')
				\tkzDrawSegments[dashed](O,I)
				\tkzDrawPoints[fill=black,size=3](O)
				\tkzLabelSegment[above](O,I){$r$}
				\draw[<->] (-0.7,0)--(-0.7,\h);
				\node at (-0.7,\h/2) [left]{$h$};
			\end{tikzpicture}
		}
		\noindent
		Đặt $f(r)=\pi r^2+\dfrac{k}{r}$, $r>0\Rightarrow f'(r)=2\pi r-\dfrac{k}{r^2}=\dfrac{2\pi r^3-k}{r^2}$;\\
		Ta có $f'(r)=0\Leftrightarrow r=\sqrt[3]{\dfrac{k}{2\pi}}$, $(k>0)$.\\
		Lập bảng biến thiên, ta thấy $f(r)$ đạt giá trị nhỏ nhất khi $r=\sqrt[3]{\dfrac{k}{2\pi}}$.\\
		Vậy với bán kính đáy là $r=\sqrt[3]{\dfrac{k}{2\pi}}$ thì chi phí làm bể là ít nhất.}
\end{ex}

\begin{ex}%[Dự Án Giảng 12 4 in 1, Lê Văn Toàn]%[2D1CV5-6]%[2D1V5-6]
	Cho hàm số $y=\dfrac{x+2}{x+1}$ có đồ thị $(C)$. Gọi $d$ là khoảng cách từ giao điểm hai tiệm cận của đồ thị $(C)$ đến một tiếp tuyến của $(C)$. Giá trị lớn nhất của $d$ có thể đạt được là
	\choice
	{$\sqrt{3}$}
	{\True $\sqrt{2}$}
	{$3\sqrt{3}$}
	{$2\sqrt{2}$}
	\loigiai{
		Ta có $y=\dfrac{x+2}{x+1}\Rightarrow y'=\dfrac{-1}{(x+1)^2}$.\\
		Đồ thị $(C)$ của hàm số $y=\dfrac{x+2}{x+1}$ có đường tiệm cận đứng là $x=-1$ và đường tiệm cận ngang là $y=1$.\\
		Suy ra giao điểm hai đường tiệm cận là $I(-1;1)$.\\
		Lấy $M(x_0;y_0)\in (C)$ tùy ý với $x_0\ne -1$, $y_0=\dfrac{x_0+2}{x_0+1}$.\\
		Ta có tiếp tuyến của đồ thị $(C)$ tại điểm $M(x_0;y_0)$ là
		\[\Delta\colon y=\dfrac{-1}{(x_0+1)^2}(x-x_0)+y_0\Leftrightarrow \Delta\colon x+(x_0+1)^2y-x_0^2-4x_0-2=0.\]
		Khoảng cách từ điểm $I$ đến tiếp tuyến của đồ thị $(C)$ tại điểm $M(x_0;y_0)$ là
		\allowdisplaybreaks
		\begin{eqnarray*}
			d=\mathrm{d}(I,\Delta)&=&\dfrac{\left|-1+(x_0+1)^2-x_0^2-4x_0-2\right|}{\sqrt{1+(x_0+1)^4}}\\
			&=&\dfrac{2|x_0+1|}{\sqrt{1+(x_0+1)^4}}\\
			&=&\dfrac{2}{\sqrt{\dfrac{1}{(x_0+1)^2}+(x_0+1)^2}}\\
			&\le& \dfrac{2}{\sqrt{2\sqrt{\dfrac{1}{(x_0+1)^2}\cdot (x_0+1)^2}}}=\sqrt{2}.
		\end{eqnarray*}
		Dấu ``$=$'' xảy ra khi và chỉ khi $\dfrac{1}{(x_0+1)^2=(x_0+1)^2}\Leftrightarrow (x_0+1)^4=1\Leftrightarrow \hoac{&x_0=0\\&x_0=-2}$ (nhận).\\
		Vậy giá trị lớn nhất của $d$ có thể đạt được là $\sqrt{2}$.
	}
\end{ex}

\begin{ex}%[TEX NBV, Trương Đăng Khoa]%[2D1V3-6]
	\immini{Cho nửa đường tròn đường kính $AB=2$ và hai điểm $C$, $D$ thay đổi trên nửa đường tròn đó sao cho $ABCD$ là hình thang. Diện tích lớn nhất của hình thang $ABCD$ bằng
		\choice
		{ $\dfrac{1}{2}$}
		{ \True $\dfrac{3\sqrt{3}}{4}$}
		{ $1$}
		{ $\dfrac{3\sqrt{3}}{2}$
		}}{
		\begin{tikzpicture}[scale=1, font=\footnotesize, line join=round, line cap=round,>=stealth]
			\coordinate (O) at (2.5,0);
			\coordinate (A) at (0,0);
			\coordinate (B) at (5,0);
			\coordinate (D) at ($(O) + (120:2.5)$);
			\coordinate (C) at ($(O) + (60:2.5)$);
			\coordinate (I) at ($(D)!0.5!(C)$);
			\coordinate (H) at ($(A)!(D)!(O)$);
			\draw (5,0) arc (0:180:2.5);
			\draw (A)--(B)--(C)--(D)--(A) (O)--(I) (O)--node[above]{$1$}(D) (D)--node[left]{$x$}(H);
			\draw pic[draw,angle radius=0.3cm]{right angle=D--H--A};
			\draw pic[draw,angle radius=0.3cm]{right angle=I--O--B};
			\foreach \x/\y in {A/-90, B/-90, D/90,C/90, I/-45,O/-90, H/-90}{\fill (\x) circle(1.2pt) ($(\x)+(\y:0.3cm)$) node{$\x$};}
		\end{tikzpicture}
	}
	\loigiai{
		Gọi $H$ là hình chiếu vuông góc của $D$ lên $AB$, $I$ là trung điểm của đoạn $CD$ và $O$ là trung điểm của $AB$.\\
		Đặt $DH=x$, $0<x<1$.\\
		Ta có $DC=2DI=2OH=2\sqrt{OD^2-DH^2}=2\sqrt{1-x^2}$.\\
		Diện tích của hình thang $ABCD$ là $S=f( x )=\dfrac{( AB+CD )\cdot DH}{2}=\left( 1+\sqrt{1-x^2}\right)x$.\\
		Ta có $f'( x )=\dfrac{\sqrt{1-x^2}+1-2x^2}{\sqrt{1-x^2}}$.\\
		$f'( x )=0\Leftrightarrow \sqrt{1-x^2}+1-2x^2=0$. ($\ast$)\\
		Đặt $t=\sqrt{1-x^2}$, ($t\ge 0$) khi đó phương trình ($\ast$) trở thành $2t^2+t-1=0\Leftrightarrow \hoac{
				& t=-1 \text{ (loại)}\\
				& t=\dfrac{1}{2}.
			}$\\
		Khi đó  $\sqrt{1-x^2}=\dfrac{1}{2}\Leftrightarrow x^2=\dfrac{3}{4}\Leftrightarrow x=\pm \dfrac{\sqrt{3}}{2}$.\\
		Bảng biến thiên
		\begin{center}
			\begin{tikzpicture}[scale=1, font=\footnotesize]
				\tkzTabInit[nocadre=false, lgt=1.2, espcl=2, deltacl=0.6]
				{$x$/0.8,$f'(x)$/0.6,$f(x)$/2}
				{$0$,$\dfrac{\sqrt{3}}{2}$,$1$};
				\tkzTabLine{,+,$0$,-,};
				\tkzTabVar{-/$0$,+/$\dfrac{3\sqrt{3}}{4}$,-/$1$};
			\end{tikzpicture}
		\end{center}
		Vậy diện tích lớn nhất của hình thang $ABCD$ bằng $\dfrac{3\sqrt{3}}{4}$.}
\end{ex}

\begin{ex}%[BG-12NEW-4in1, Nguyen Huynh]%[2D1H4-1]
	Trong mặt phẳng $Oxy$, tổng khoảng cách từ gốc tọa độ đến tất cả các đường tiệm cận của đồ thị hàm số $y=\log_2\dfrac{2x+3}{x-1}$ bằng
	\choice
	{$2$}
	{$3$}
	{$\dfrac{5}{2}$}
	{\True $\dfrac{7}{2}$}
	\loigiai{
		Điều kiện $\dfrac{2x+3}{x-1}>0\Leftrightarrow\hoac{&x>1\\&x<-\dfrac{3}{2}.}$\\
		Ta xét các giới hạn sau
		\begin{itemize}
			\item $\lim\limits_{x\to{1^+}}\left(\log_2\dfrac{2x+3}{x-1}\right)=+\infty$.
			\item $\lim\limits_{x\to{\left(-\tfrac{3}{2}\right)^-}}\left(\log_2\dfrac{2x+3}{x-1}\right)=-\infty$.
		\end{itemize}
		Từ đó suy ra tiệm cận đứng là $d_1\colon x=-\dfrac{3}{2}$; $d_2\colon x=1$.\\
		Mặt khác $\lim\limits_{x\to+\infty}\left(\log_2\dfrac{2x+3}{x-1}\right)=\lim\limits_{x\to-\infty}\left(\log_2\dfrac{2x+3}{x-1}\right)=1$.\\
		Từ đó suy ra tiệm cận ngang là $\left(d_3\right)\colon y=1$.\\
		Ta có $T=\mathrm{d}\left(O,d_1\right)+\mathrm{d}\left(O,d_2\right)+\mathrm{d}\left(O,d_3\right)=\dfrac{3}{2}+1+1=\dfrac{7}{2}$.}
\end{ex}

\begin{ex}%[Dự án Giảng 12 Nhóm Toán & LaTex, Lê Minh Thiện Anh]%[2D1N5-1]
	\immini{Đường cong bên là đồ thị của một trong bốn hàm số đã cho sau đây. Hỏi đó là hàm số nào?
		\choice
		{$y=-x^3+x^2-2$}
		{\True $y=x^3+3x^2-2$}
		{$y=x^3-3x+2$}
		{$y=x^2-3x-2$}
	}{
		\begin{tikzpicture}[scale=0.6, font=\footnotesize,line join=round, line cap=round,>=stealth]
			\draw[->] (-3.7,0.) -- (2,0) node[below]{$x$};
			\draw[->] (0,-3) -- (0,3) node[right]{$y$};
			\fill (0,0) node[above left]{$O$};
			\fill (0,-2) circle(2pt) node[below left]{$-2$};
			\draw[smooth,samples=300,domain=-3.1:1.1] plot(\x,{(\x+2)^3-3*(\x+2)^2+2});
		\end{tikzpicture}
	}
	\loigiai{
		Dựa vào hình dáng đồ thị, ta thấy đây là đồ thị của hàm số bậc ba $y=ax^3+bx^2+cx+d$ với $a>0$ nên loại các hàm $y=x^2-3x-2$, $y=-x^3+x^2-2$.\\
		Mặt khác, đồ thị đi qua điểm $(0;-2)$ nên loại hàm $y=x^3-3x+2$.
	}
\end{ex}

\begin{ex}%[2D1N5-1]
	Bảng biến thiên sau là của hàm số nào dưới đây?
	\begin{center}
		\begin{tikzpicture}
			\tkzTabInit[nocadre=true,espcl=2.5,lgt=1.2,deltacl=0.5]
			{$x$/0.7,$y'$/0.7,$y$/2}
			{$-\infty$,$0$,$1$,$2$,$+\infty$}
			\tkzTabLine{,+,$0$,-,d,-,$0$,+,}
			\tkzTabVar{-/$-\infty$,+/$2$,-D+/$-\infty$/$+\infty$,-/$6$,+/$+\infty$}
		\end{tikzpicture}
	\end{center}
	\choice
	{$y=\dfrac{x^2-4x+2}{x-1}$}
	{\True $y=\dfrac{x^2+2x-2}{x-1}$}
	{$y=\dfrac{x^2+2x-2}{x+1}$}
	{$y=\dfrac{x^2+2}{x-1}$}
	\loigiai{
		Từ bảng biến thiên ta thấy $\heva{&\lim\limits_{x\to 1^-}y=-\infty\\&\lim\limits_{x\to 1^+}y=+\infty}\Rightarrow x=1$ là đường tiệm cận đứng nên loại đáp án \textbf{C}.\\
		Đồ thị hàm số có điểm cực đại $(0;2)$ nên loại đáp án \textbf{D}.\\
		Đồ thị hàm số có điểm cực tiểu $(2;6)$ nên loại đáp án \textbf{A}.
	}
\end{ex}

\begin{ex}%[BG-12NEW-4in1, Nguyen Huynh]%[2D1N4-1]
	Tiệm cận xiên của đồ thị hàm số $y=\dfrac{x^2+3x+5}{x+2}$ là
	\choice
	{$y=x$}
	{$y=x+1$}
	{\True $y=x+2$}
	{$y=x+3$}
	\loigiai{
		Tập xác định của hàm số là $\mathbb{R}\setminus\{-2\}$.
		Ta thấy
		\begin{itemize}
			\item $\lim\limits_{x\to +\infty}\dfrac{x^2+3x+5}{x(x+2)}=\lim\limits_{x\to +\infty}\dfrac{1+\tfrac{3}{x}+\tfrac{5}{x^2}}{1+\tfrac{2}{x}}=1$.\\
			\item $\lim\limits_{x\to+\infty}(y-x)=\lim\limits_{x\to +\infty}\dfrac{2x+3}{x+2}=2$.
		\end{itemize}
		Vậy $y=x+2$ là tiệm cận xiên của đồ thị hàm số.\\
		Tương tự, ta thấy $y=x+2$ là tiệm cận xiên của đồ thị hàm số.\\
		Vậy $y=x+2$ là tiệm cận xiên của đồ thị hàm số.

	}
\end{ex}

\begin{ex}%[BG-12NEW-4in1, Nguyen Huynh]%[2D1H4-1]
	Cho hàm số $y=a^x$ với $0<a\ne1$. Mệnh đề nào sau đây \textbf{sai}?
	\choice
	{Đồ thị hàm số $y=a^x$ và đồ thị hàm số $y=\log_ax$ đối xứng nhau qua đường thẳng $y=x$}
	{Hàm số $y=a^x$ có tập xác định là $\mathbb{R}$ và tập giá trị là $(0;+\infty)$}
	{Hàm số $y=a^x$ đồng biến trên tập xác định của nó khi $a>1$}
	{\True Đồ thị hàm số $y=a^x$ có tiệm cận đứng là trục tung}
	\loigiai{Theo lý thuyết, ta có $\lim\limits_{x\to0^+}a^x=1$ và $\lim\limits_{x\to0^-}a^x=1$ nên không nhận trục tung làm tiệm cận đứng.}
\end{ex}

\begin{ex}%[Khảo sát chất lượng L1 - THPT Nguyễn Viết Xuân - Vĩnh Phúc - 2019, Mức độ H]%[Dự án giảng 12 - Trung Anh]%[2D1H2-5]
	Biết đồ thị hàm số $y = x^4 - 2mx^2 + 2$ có ba điểm cực trị là ba đỉnh của một tam giác vuông cân. Tính giá trị của biểu thức $P = m^2 + 2m + 1$.
	\choice
	{$P = 1$}
	{\True $P = 4$}
	{$P = 2$}
	{$P = 0$}
	\loigiai{
		Tập xác định: $\mathscr{D} = \mathbb{R}$.
		$y' = 4x^3 - 4mx$.\\
		$y' = 0 \Leftrightarrow \hoac{&x = 0 \\	&x^2 = m.}$\\
		Hàm số có ba điểm cực trị $\Leftrightarrow m > 0$.\\
		Khi đó ba điểm cực trị của hàm số là $x_1 = 0$, $x_2 = \sqrt{m}$, $x_3 = -\sqrt{m}$.\\
		Vậy ba điểm cực trị của đồ thị hàm số là $A(0;2)$, $B\left(\sqrt{m}; 2 - m^2 \right)$, $C\left(-\sqrt{m}; 2 - m^2 \right)$. Ba điểm này luôn tạo thành tam giác cân tại $A$. Vậy tam giác này vuông cân khi và chỉ khi $\widehat{BAC} = 90^\circ$.\\
		Tương đương $\overrightarrow{AB} \cdot \overrightarrow{AC} = 0$, hay $\sqrt{m} \cdot \left(-\sqrt{m}\right) + (-m^2) \cdot (-m^2) = 0$.\\
		Giải phương trình này ta có $m = 1$ là nghiệm duy nhất. Do đó $P = 4$.
	}
\end{ex}
\begin{ex}%[SGK12-KNTT ]%[2D1H5-8]
	Khi máu di chuyển từ tim qua các động mạch chính rồi đến các mao mạch và quay trở lại qua các tĩnh mạch, huyết áp tâm thu (tức là áp lực của máu lên động mạch khi tim co bóp) liên tục giảm xuống. Giả sử một người có huyết áp tâm thu $P$ (tính bằng mmHg) được cho bởi hàm số
	\[
		P(t)=\dfrac{25 t^2+125}{t^2+1}, 0 \leq t \leq 10,
	\]
	trong đó thời gian $t$ được tính bằng giây. Tính tốc độ thay đổi của huyết áp sau $5$ giây kể từ khi máu rời tim.
	\choice
	{$-\dfrac{20}{17}$}
	{\True $-\dfrac{250}{169}$}
	{$-\dfrac{120}{163}$}
	{$-\dfrac{19}{132}$}
	\loigiai{
		Ta có tốc độ thay đổi của huyết áp là $P'(t)=\dfrac{-200t}{(t^2+1)^2}$.\\
		Do đó tốc độ thay đổi huyết áp sau $5$ giây là $P'(5)=-\dfrac{250}{169}$.
	}
\end{ex}
\begin{ex}%[KNTT]giảng 12 New-4in1, Toàn Phan]%[2D1H5-8]KNTT
	Khi bỏ qua sức cản của không khí, độ cao (mét) của một vật được phóng thẳng đứng lên trên từ điểm cách mặt đất $2$ m với vận tốc ban đầu $24{,}5$ m/s là $h(t)=2+24{,}5t-4{,}9t^2$ (theo Vật lí đại cương, NXB Giáo dục Việt Nam, $2016$). Tìm vận tốc của vật sau $2$ giây.
	\choice
	{\True $4{,}9$}
	{$3{,}2$}
	{$1{,}3$}
	{$5{,}5$}
	\loigiai{
	Theo ý nghĩa cơ học của đạo hàm, vận tốc của vật là $v=h'(t)=24{,}5-9{,}8t$ m/s.\\
	Do đó, vận tốc của vật sau $2$ giây là $v(2)=24{,}5-9{,}8\cdot 2=4{,}9$ m/s.
	}
\end{ex}

\TL

\begin{ex}%[SGK12-CTST, Mức độ 2]%[2D1H2-1]
	Tìm cực trị của hàm số $g(x)=\dfrac{x^2+x+4}{x+1}$.
	\loigiai{
		Tập xác định $\mathscr{D}=\mathbb{R}\setminus \{-1\}$.\\
		Ta có $g(x)=x+\dfrac{4}{x+1} \Rightarrow g'(x)=1-\dfrac{4}{(x+1)^2}=\dfrac{x^2+2x-3}{(x+1)^2}$;\\
		$g'(x)=0 \Leftrightarrow x^2+2x-3=0 \Leftrightarrow \hoac{& x=-3\\& x=1.}$\\
		Bảng biến thiên
		\begin{center}
			\begin{tikzpicture}
				\tkzTabInit[nocadre=false,lgt=1.2,espcl=2.5,deltacl=0.6]
				{$x$ /0.6,$g'(x)$ /0.6,$g(x)$ /2}
				{$-\infty$,$-3$,$-1$,$1$,$+\infty$}
				\tkzTabLine{,+,$0$,-,d,-,$0$,+,}
				\tkzTabVar{-/$-\infty$,+/$-5$,-D+/$-\infty$/$+\infty$,-/$3$,+/$+\infty$}
			\end{tikzpicture}
		\end{center}
		Vậy hàm số đạt cực đại tại $x=-3$, $y_{\text{CĐ}}=g(-3)=-5$; và hàm số đạt cực tiểu tại $x=1$, $y_{\text{CT}}=g(1)=3$.
	}
\end{ex}
\begin{ex}%[SGK12-CTST, Mức độ 2]%[2D1H1-5]
	Kim ngạch xuất khẩu rau quả của Việt Nam trong các năm từ $2010$ đến $2017$ có thể được tính xấp xỉ bằng công thức $f(x)=0{,}01x^3-0{,}04x^2+0{,}25x+0{,}44$ (tỉ USD) với $x$ là số năm tính từ $2010$ đến $2017$ $(0\leq x\leq 7)$.
	\begin{flushright}
		(\textit{Theo:} https://infographics.vn/interactive-xuat-khau-rau-qua-
		du-bao-bung-no-dat-4-ty-usd-trong-nam-2023/116220.vna)
	\end{flushright}
	\begin{enumerate}
		\item Tính đạo hàm của hàm số $y=f(x)$.
		\item Chứng minh rằng kim ngạch xuất khẩu rau quả của Việt Nam tăng liên tục trong các năm từ $2010$ đến $2017$.
	\end{enumerate}
	\loigiai{
		\begin{enumerate}
			\item Ta có $f'(x)=0{,}03x^2-0{,}08x+0{,}25$.
			\item Xét $f'(x)=0 \Leftrightarrow 0{,}03x^2-0{,}08x+0{,}25=0$ (vô nghiệm).\\
			      Bảng biến thiên
			      \begin{center}
				      \begin{tikzpicture}
					      \tkzTabInit[nocadre=false,lgt=1.2,espcl=5,deltacl=0.6]
					      {$x$ /0.6,$f'(x)$ /0.6,$f(x)$ /2}
					      {$0$,$7$}
					      \tkzTabLine{,+,}
					      \tkzTabVar{-/$0{,}44$,+/$3{,}66$}
				      \end{tikzpicture}
			      \end{center}
			      Từ bảng biến thiên trên, ta thấy $f'(x)>0$, $\forall x\in [0;7]$.\\
			      Vậy kim ngạch xuất khẩu rau quả của Việt Nam tăng liên tục trong các năm từ $2010$ đến $2017$.
		\end{enumerate}
	}
\end{ex}
\begin{ex}%[Mức độ 4]%[BG12-4IN1, Nguyễn Khánh Trọng]%[2D1C3-6]
	\immini[thm]{
		Cho một tấm gỗ hình vuông cạnh $200$ cm. Người ta cắt một tấm gỗ có hình một tam giác vuông $ABC$ từ tấm gỗ hình vuông đã cho như hình vẽ bên. Biết $AB=x$ ($0<x<60$ cm) là một cạnh góc vuông của tam giác $ABC$ và tổng độ dài cạnh góc vuông $AB$ với cạnh huyền $BC$ bằng $120$ cm. Tìm $x$ để tam giác $ABC$ có diện tích lớn nhất.
		% \shortans{$40$}
	}{
		\begin{tikzpicture}[scale=0.72, font=\footnotesize, line join=round, line cap=round, >=stealth]
			\draw[dashed] (0,0)--(4,0)--(0,1)--(0,0);
			\draw (4,0)--(5,0)--(5,5)--(0,5)--(0,1);
			\node at (0,0.5)[below left] {$x$}; \node at (2,0.5)[above,rotate=-13] {$120-x$}; \node at (2.5,5)[above] {$200$};
			\fill (0,0) circle (1.5pt) node[below left]{$A$} (4,0) circle (1.5pt) node[below]{$C$} (0,1) circle (1.5pt) node[left]{$B$};
		\end{tikzpicture}
	}

	\loigiai{
		Độ dài cạnh huyền $BC$ là $120-x$.\\
		Khi đó độ dài cạnh $AC=\sqrt{BC^2-AB^2}=\sqrt{(120-x)^2-x^2}=\sqrt{14400-240x}$.\\
		Diện tích tam giác $ABC$ là $S=\dfrac{1}{2}AB\cdot AC=\dfrac{1}{2}x\sqrt{14400-240x}$.\\
		Xét hàm số $f(x)=x\sqrt{14400-240x}$ với $0<x<60$.\\
		Ta có $f'(x)=\sqrt{14400-240x}-\dfrac{120x}{\sqrt{14400-240x}}=\dfrac{14400-360x}{\sqrt{14400-240x}}$;\\
		$f'(x)=0\Leftrightarrow x=40\in(0;60)$.\\
		Bảng biến thiên
		\begin{center}
			\begin{tikzpicture}
				\tkzTabInit[nocadre=false,lgt=1.2,espcl=2.5,deltacl=0.6]
				{$x$ /0.6,$f'(x)$ /0.6,$f(x)$ /2}
				{$0$,$40$,$60$}
				\tkzTabLine{,+,$0$,-,}
				\tkzTabVar{-/, +/,-/}
			\end{tikzpicture}
		\end{center}
		Vậy tam giác $ABC$ có diện tích lớn nhất khi $AB=40$ cm.
	}
\end{ex}
\begin{ex}%[Dự án đề ôn tập GK1, Mui Doan]%[2D1V3-6]
	Có hai xã $A$, $B$ cùng ở một bên bờ sông Lam, khoảng cách từ hai xã đó đến bờ sông lần lượt là $AA'=500$  m, $BB'=600$  m và người ta đo được $A'B'=2\,200$  m. Các kĩ sư muốn xây một trạm cung cấp nước sạch nằm bên bờ sông Lam cho dân hai xã. Để tiết kiệm chi phí, các kĩ sư cần phải chọn vị trí $M$ của trạm cung cấp nước sạch đó trên đoạn $A'B'$ sao cho tổng khoảng cách từ hai xã đến vị trí $M$ là nhỏ nhất. Hãy tìm vị trí tối ưu đó.
	% \shortans{$2460$}
	\begin{center}
		\begin{tikzpicture}
			\path
			(0:0) coordinate (A')
			(0:6) coordinate (B')
			(0:2) coordinate (M)
			($(A')+(90:2.5)$) coordinate (A)
			($(B')+(90:3)$) coordinate (B)
			;
			\fill[cyan!50] (-1.5,-1) rectangle (7.5,0);
			\draw[thick] (A')--node[left]{$500$ m}(A)--(M)--(B)--node[right]{$600$ m}(B');

			\foreach \i/\j in{A'/-100,B'/-80,A/100,B/80,M/-90}{\fill [black](\i) circle (1pt) ($(\i)+(\j:3mm)$) node {$\i$};}

			\draw [dashed,<->]	(0,.6)--(6,.6) node[pos=0.75,sloped,above]{$2\,200$ m};
		\end{tikzpicture}
	\end{center}
	\loigiai{
		\begin{center}
			\begin{tikzpicture}
				\path
				(0:0) coordinate (A')
				(0:6) coordinate (B')
				(0:2) coordinate (M)
				($(A')+(90:2.5)$) coordinate (A)
				($(B')+(90:3)$) coordinate (B)
				;
				\draw[thick] (A')--node[left]{$500 \text{(m)}$}(A)--(M)--(B)--node[right]{$600 \text{(m)}$}(B') (A')--(B');

				\foreach \i/\j in{A'/-100,B'/-80,A/100,B/80,M/-90}{\fill [black](\i) circle (1pt) ($(\i)+(\j:3mm)$) node {$\i$};}

				\draw [dashed,<->]	(0,.6)--(6,.6) node[pos=0.75,sloped,above]{$2\,200\text{(m)}$}; %Tùy chọn sloped,above,below
				\node at (3,-1.5){\it Hình 37};
			\end{tikzpicture}
		\end{center}
		Đặt $A'M=x$, $(0<x<2200)$,  $B'M=2200-x$.\\
		Ta có  $AM=\sqrt{x^2+500^2}$, $BM=\sqrt{(2200-x)^2+600^2}$.\\
		Khi đó tổng khoảng cách từ hai xã đến vị trí $M$ là $AM+BM= \sqrt{x^2+500^2}+\sqrt{(2200-x)^2+600^2} $.\\
		Xét hàm số $f(x)= \sqrt{x^2+500^2}+\sqrt{(2200-x)^2+600^2}$ trên khoảng $(0<x<2200)$.\\
		%$f(x)=\sqrt{x^2+500}+\sqrt{x^2-4400x+4840600}$ .\\
		$f'(x)=\dfrac{x}{\sqrt{x^2+500^2}}-\dfrac{2200-x}{\sqrt{(2200-x)^2+600^2}}$,
		\allowdisplaybreaks
		\begin{eqnarray*}
			f'(x)=0&\Leftrightarrow&\dfrac{x}{\sqrt{x^2+500^2}}=\dfrac{2200-x}{\sqrt{(2200-x)^2+600^2}}
			\\
			&\Leftrightarrow&\dfrac{x^2}{x^2+500^2}=\dfrac{(2200-x)^2}{(2200-x)^2+600^2}\\
			&\Leftrightarrow&\dfrac{x^2+500^2}{x^2}=\dfrac{(2200-x)^2+600^2}{(2200-x)^2}\\
			&\Leftrightarrow& 1+\dfrac{500^2}{x^2}=1+\dfrac{600^2}{(2200-x)^2}
			\\
			&\Leftrightarrow& \dfrac{25}{x^2}=\dfrac{36}{(2200-x)^2}
			\\
			&\Leftrightarrow& \dfrac{5}{x}=\dfrac{6}{2200-x}
			\\
			&\Leftrightarrow& x=1\,000~ \text{vì}~  x>0.
		\end{eqnarray*}
		Bảng biến thiên hàm số $f(x)$ trên khoảng $( 0;2\,200)$.
		\begin{center}
			\begin{tikzpicture}
				\tkzTabInit[lgt=1.2,espcl=4.5,deltacl=0.6]
				{$x$/1,$f'(x)$/1,$f(x)$/3} {$0$,$1000$,$2200$}
				\tkzTabLine{,-,0,+,}
				\tkzTabVar{+/$2780$,-/$2460$,+/$2856$}
			\end{tikzpicture}
		\end{center}
		Vậy giá trị nhỏ nhất của tổng khoảng cách từ hai xã đó đến bờ sông  là khoảng $2\,460$  m, tại vị trí $M$ cách điểm $A'$  là $1\,000$  m.
	}
\end{ex}
\Closesolutionfile{ans}

% \begin{indapan}
% 	{ans/ansc1l4}
% \end{indapan}


% \begin{name}
	{\tenchude}
	{ĐỀ ÔN TẬP CHƯƠNG I}
	{LỚP TOÁN THẦY PHÁT}
	{\thoigian}
\end{name}

\TN
\Opensolutionfile{ans}[ans/ansc101]
\begin{ex}%[Mức độ N]%[2D1N2-2]
	\immini{	Cho hàm số $y=f(x)$ có đồ thị như hình bên. Hàm số đạt cực đại tại
		\choice
		{$x=-3$}
		{\True $x=-2$}
		{$y=0$}
		{$y=4$}}{\begin{tikzpicture}[>=stealth]
			\draw [->] (-4,0)--(1.5,0);
			\draw [->] (0,-1)--(0,5);
			\draw (0,0) node[below right]{$O$};
			\draw (1.5,0) node[below]{$x$};
			\draw (0,5) node[below left]{$y$};
			\foreach \x in {-3,-2}{\draw (\x,-.1)--(\x,.1) node[below left,black]{$\x$};}
			\foreach \y in {4}{\draw [-] (-.1,\y)--(.1,\y) node[right,black]{$\y$};}
			\clip (-4,-1) rectangle (5,5);
			\draw [thick,samples=100] plot[domain=-5:5](\x,{(\x)^3+3*(\x)^2});
			\draw[dashed] (0,4) -- (-2,4) --(-2,0);
			\fill[black] (-2,4) circle(2pt);
			\draw (1,5) node[below right]{$f(x)$};
		\end{tikzpicture}}
	\loigiai{Từ hình vẽ, ta thấy hàm số $y=f(x)$ đạt cực đại tại điểm $x=-2$.}
\end{ex}

\begin{ex}%[2-D1B2-SO-5-2425]%[VN-MT-7, VM026]%[2D1N3-1]
	Cho hàm số $y=f\left( x \right)$ xác định và liên tục trên đoạn $\left[ -4;5 \right]$, có bảng biến thiên như hình sau
	\begin{center}
		\begin{tikzpicture}
			\tkzTabInit[nocadre=true,lgt=1.0,espcl=3, deltacl=0.5]
			{$x$/0.7,$y’$/0.7,$y$/2}
			{$-4$,$-2$,$4$,$5$}
			\tkzTabLine{,+,0,-,0,+,}
			\tkzTabVar{-/$\dfrac{2}{3}$,+/$\dfrac{46}{3}$,-/$-\dfrac{62}{3}$,+/$-\dfrac{52}{3}$}
		\end{tikzpicture}
	\end{center}
	Gọi $M$, $N$ lần lượt là giá trị lớn nhất, giá trị nhỏ nhất của hàm số $y=f(x)$ xác định trên đoạn $[-4;5]$. Tính $M+N$?
	\choice
	{\True $-\frac{16}{3}$}
	{$-\frac{50}{3}$}
	{$2$}
	{$-20$}
	\loigiai{
		Dựa vào bảng biến thiên, ta có $M+N=\dfrac{46}{3}-\dfrac{62}{3}=-\dfrac{16}{3}$.
	}
\end{ex}

\begin{ex}%[2D1N5-5]
	\immini{Cho hàm số $f(x)$ có đạo hàm $f’(x)$ xác định, liên tục trên $\mathbb{R}$ và $f’(x)$ có đồ thị như hình vẽ bên. Khẳng định nào sau đây là đúng?
		\choice
		{Hàm số $f(x)$ đồng biến trên $(-\infty;1)$}
		{Hàm số $f(x)$ đồng biến trên $(-\infty;1)$ và $(1;+\infty)$}
		{\True Hàm số $f(x)$ đồng biến trên $(1;+\infty)$}
		{Hàm số $f(x)$ đồng biến trên $\mathbb{R}$}}
	{\begin{tikzpicture}[>=stealth,line join=round,line cap=round,font=\normalsize,scale=0.6]
			\draw[-stealth] (-1,0)--(0,0)node[above left]{$O$}--(3,0)node[below]{$x$};
			\draw[-stealth] (0,-3)--(0,3)node[left]{$y$};
			\draw (0,-2.5)--(1,0)--(2.5,2.5);
			\draw
			(1,0) node[below right]{$1$}
			;
		\end{tikzpicture}}
	\loigiai{
		Dựa vào đồ thị hàm số $f’(x)$, ta thấy $f’(x)>0,\forall x\in(1;+\infty)$ suy ra hàm số $f(x)$ đồng biến trên $(1;+\infty)$.}
\end{ex}

\begin{ex}%[2D1N5-1]
	\immini[thm]{Đồ thị hình bên là của một trong bốn hàm số sau. Hỏi đó là hàm số nào?
		\choice
		{$y=\dfrac{x^2+x-1}{x-1}$}
		{\True $y=\dfrac{x^{2}-x+1}{x-1}$}
		{$y=\dfrac{x^2-4x-1}{-x+1}$}
		{$y=\dfrac{x^2-3x-1}{-x+1}$}}{
		\begin{tikzpicture}[line cap=butt,line join=miter,>=stealth,scale=0.57,font=\footnotesize]
			\tikzset{declare function={xmin=-3.5;xmax=4.7;ymin=-3.5;ymax=6;},
				smooth,samples=450}
			\draw[->] (xmin,0)--(xmax,0) node[shift={(0:7pt)}]{$ x $};
			\draw[->] (0,ymin-.2)--(0,ymax) node[shift={(90:7pt)}]{$ y $};
			\fill (0,0) node[shift={(140:6pt)}]{$ O $};
			\clip (xmin,ymin) rectangle (xmax,ymax);
			\foreach \i in {-3,-2,2,3,4}{
					\draw(\i,1.5pt)--(\i,-1.5pt)node[below]{$\i$};}
			\foreach \j in {-2,1,2,3,4,5}{
					\draw(-1.5pt,\j)--(1.5pt,\j) node[left]{$\j$};}
			\draw(-1.5pt,-1)--(1.5pt,-1)node[shift={(160:6.5pt)}]{$-1$};
			\draw(1,-1.5pt)--(1,1.5pt)node[shift={(-75:7pt)}]{$1$};
			\draw(-1,-1.5pt)--(-1,1.5pt)node[shift={(100:5pt)}]{$-1$};
			\def\f(#1){((#1)^2-(#1)+1)/((#1)-1)}
			\def\a{-1}
			\def\b{0}
			\def\c{0.5}
			\def\d{1.5}
			\def\e{2}
			\def\g{3}
			\pgfmathsetmacro\fa{\f(\a)}
			\pgfmathsetmacro\fb{\f(\b)}
			\pgfmathsetmacro\fc{\f(\c)}
			\pgfmathsetmacro\fd{\f(\d)}
			\pgfmathsetmacro\fe{\f(\e)}
			\pgfmathsetmacro\fg{\f(\g)}
			\draw[samples=100] plot[domain=-5.3:0.9] (\x,{\f(\x)});
			\draw[samples=100] plot[domain=1.05:5.2] (\x,{\f(\x)});
			\draw[] (1,ymin)--(1,ymax) node [pos=0.95,sloped, above]{$x=1$};
			\draw[] (xmin,ymin)--(6,ymax) node [pos=0.08,sloped, above]{$y=x$};
		\end{tikzpicture}
	}
	\loigiai{
	}
\end{ex}

\begin{ex}%[2-D1B5-SO-16-2425]%[VN-MT-7, Dương Phước Sang]%[2D1H5-1]
	Cho bảng biến thiên của hàm số $y=f(x)$ như sau:
	\begin{center}
		\begin{tikzpicture}
			\tikzset{double style/.append style={double distance=1.5pt}}
			\tkzTabInit[nocadre=true,lgt=1.2,espcl=4,deltacl=0.6]
			{$x$/0.6,$y'$/0.6,$y$/2}
			{$-\infty$,$1$,$+\infty$}
			\tkzTabLine{,-,d,-,}
			\tkzTabVar{+/$1$,-D+/$-\infty$/$+\infty$,-/$1$}
		\end{tikzpicture}
	\end{center}
	Hỏi đây là bảng biến thiên của hàm số nào trong các hàm số sau?
	\choice
	{$y=\dfrac{x-3}{x-1}$}
	{$y=\dfrac{-x+2}{x-1}$}
	{$y=\dfrac{x+2}{x+1}$}
	{\True $y=\dfrac{x+2}{x-1}$}
	\loigiai{
		Bảng biến thiên được cung cấp có đặc điểm:
		\begin{itemize}
			\item Đồ thị hàm số có đường tiệm cận đứng là $x=1$, loại $y=\dfrac{x+2}{x+1}$.
			\item Đồ thị hàm số có đường tiệm cận ngang là $y=1$, loại $y=\dfrac{-x+2}{x-1}$.
			\item $y'<0,\,\forall x \ne 1$, trong khi $\left(\dfrac{x-3}{x-1}\right)'=\dfrac{2}{(x-1)^2}>0,\,\forall x \neq 1$, loại $y=\dfrac{x-3}{x-1}$.
		\end{itemize}
		Chỉ có hàm số $y=\dfrac{x+2}{x-1}$ thỏa mãn các đặc điểm trên.
	}
\end{ex}

\begin{ex}%[BG-12NEW-4in1, Nguyen Huynh]%[2D1N4-1]
	\immini{
		Cho hàm số $y=\dfrac{2x^2}{x^2-1}$ có đồ thị là đường cong như hình vẽ bên. Số các đường tiệm cận đứng, tiệm cận ngang và tiệm cận xiên (nếu có) của đồ thị hàm số đã cho là	\choice{\True $4$}{ $2$}{ $3$}{$5$}}
	{	\begin{tikzpicture}[x=1cm,y=1cm,scale=.5]
			\draw[->] (-4,0)--(4,0)node[below right]{$x$};
			\draw[->] (0,-4)--(0,4)node[left]{$y$};
			\fill (0,0)node[above left]{ $O$};
			\draw (-4,2)--(4,2) node[below]{ $y=2$};
			\node at (-1,-4)[above left]{ $x=-1$};
			\node at (1,-4)[above right]{ $x=1$};
			\clip (-4,-4)rectangle(4,4);
			\draw[black,samples=150,smooth,domain=-3.85:3.85] plot(\x,{2*(\x)^(2)/((\x)^(2)-1)});
		\end{tikzpicture}
	}
	\loigiai{Đồ thị hàm số đã cho có tiệm cận đứng là các đường thẳng $x=-1$, $x=1$, tiệm cận ngang là đường thẳng $y=2$ và không có tiệm cận xiên}
\end{ex}

\begin{ex}%[Mức độ 3]%[2D1H2-1]
	Đồ thị của hàm số $y=x^3-3x^2-9x+1$ có hai điểm cực trị $A$ và $B$. Điểm nào dưới đây thuộc đường thẳng $AB$?
	\choice
	{$P (1;0)$}
	{$M (0;1)$}
	{\True $N(1;-10)$}
	{$Q (-1;10)$}
	\loigiai
	{
		Ta có $y'=3x^2-6x-9$.\\
		$y'=0\Leftrightarrow\hoac{&x=-1\Rightarrow y=6\\&x=3\Rightarrow y=-26.}$\\
		$\Rightarrow AB: y=-8x-2$
	}
\end{ex}

\begin{ex}%[De-chuan-hoa-so-1]%[Dương Quang]%[2D1N4-1]
	Đường tiệm cận ngang của đồ thị hàm số $y=\dfrac{2x-4}{-x+2}$ là
	\choice
	{ $y=2$}
	{ $x=2$}
	{$x=-2$}
	{\True$y=-2$}
	\loigiai{$\underset{x\to -\infty }{\mathop{\lim }}\,\dfrac{2x-4}{x+2}=-2$ và $\underset{x\to +\infty }{\mathop{\lim }}\,\dfrac{2x-4}{x+2}=-2$ nên đồ thị hàm số có tiệm cận ngang là $y=-2$.
	}
\end{ex}

\begin{ex}%[Mức độ H]%[2D1H3-4]
	Gọi $m$, $M$ lần lượt là giá trị nhỏ nhất, giá trị lớn nhất của hàm số $y=\sqrt{4-x^2}$. Tổng $m+M$ bằng
	\choice{\True $2$}{$0$}{$4$}{$1$}
	\loigiai{Tập xác định $\mathscr{D}=[-2;2]$.\\
	Ta có $y'=\dfrac{-x}{\sqrt{4-x^2} }\Rightarrow y'=0 \Leftrightarrow x = 0 \in [-2;2]$.\\
	Bảng biến thiên
	\begin{center}
		\begin{tikzpicture}
			\tkzTabInit[nocadre=false, lgt=1.5,espcl=3.5]
			{$x$/1,$y'$/1,$y$/2}
			{$-2$,$0$,$2$}
			\tkzTabLine{,+,0,-, }
			\tkzTabVar{-/$0$,+/$2$,-/$0$/}
		\end{tikzpicture}
	\end{center}
	Dựa vào bảng biến thiên ta thấy $ \underset{[-2;2]}{\text{max}}y=2; \underset{[-2;2]}{\text{min}}y=0$.\\
	Vậy $m+M=0+2=2$.}
\end{ex}

\begin{ex}%[2D1H5-3]
	Cho hàm số $y=f(x)$ xác định, liên tục trên $\mathbb{R}$ và có bảng biến thiên sau
	\begin{center}
		\begin{tikzpicture}
			\tkzTabInit[nocadre=false,lgt=0.7,espcl=2.1]
			{$x$ /0.6,$y'$ /0.6,$y$ /2}
			{$-\infty$,$-1$,$0$,$1$,$+\infty$}
			\tkzTabLine{,-,$0$,+,$0$,-,$0$,+,}
			\tkzTabVar{+/$-\infty$, -/$-1$,+/$0$,-/$-1$,+/$-\infty$}
		\end{tikzpicture}
	\end{center}
	Tìm tất cả các giá trị của tham số $m$ để phương trình $f(x)-1=m$ có đúng hai nghiệm.
	\choice
	{\True $m=-2,m>-1$}
	{$m=-2,m\ge -1$}
	{$-2<m<-1$}
	{$m>0,m=-1$}
	\loigiai{
	$f( x )-1=m\Leftrightarrow f( x )=m+1$. \\
	Dựa vào bảng biến thiên, để phương trình $f\left( x \right)-1=m$ có đúng hai nghiệm thì\\
	$\left[ \begin{aligned}
			 & m+1>0  \\
			 & m+1=-1 \\
		\end{aligned} \right.\Leftrightarrow \left[ \begin{aligned}
			 & m>-1  \\
			 & m=-2.
		\end{aligned} \right.$}
\end{ex}

\begin{ex}%[2-D1B3-SO-8-2425]%[VN-MT-7, Nguyễn Hồng Thạch]%[2D1H4-2]
	Tìm tất cả các giá trị thực của tham số $m$ để đồ thị hàm số $y=\dfrac{mx-8}{x+2}$ có hai đường tiệm cận.
	\choice
	{$m\neq 4$}
	{\True $m\neq -4$}
	{$m=4$}
	{$m=-4$}
	\loigiai{
		Ta có $x+2=0\Leftrightarrow x=-2$.\\
		Đồ thị hàm số đã cho có hai đường tiệm cận $\Leftrightarrow m\cdot(-2)-8\ne 0\Leftrightarrow m\neq -4$.}
\end{ex}

\begin{ex}%[De-chuan-hoa-so-15]%[Nguyễn Cường]%[2D1H3-6]
	Tại trường THPT Y, để giảm nhiệt độ trong các phòng học từ nhiệt độ ban đầu là $28^\circ C$, một hệ thống điều hòa làm mát được phép hoạt động trong $10$ phút. Gọi $T$ (đơn vị $^\circ C$) là nhiệt độ phòng ở phút thứ $t$ (tính từ thời điểm bật máy) được cho bởi công thức $T=-0{,}008t^3-0{,}16t+28$ $\left(t \in \left[0;10\right]\right)$. Nhiệt độ thấp nhất trong phòng có thể đạt được trong khoảng thời gian $10$ phút đó gần đúng là
	\choice
	{$27{,}832^\circ C$}
	{\True $18{,}4^\circ C$}
	{$26{,}2^\circ C$}
	{$25{,}312^\circ C$}
	\loigiai{
	Ta có $T'=-0{,}024t^2-0{,}16 < 0 \quad \forall t \in [0;10]$.\\
	$\Rightarrow T \ge T(10)=18{,}4^\circ C$.\\
	Do đó, nhiệt độ thấp nhất phòng có thể đạt được trong khoảng thời gian $10$ phút đó là $18{,}4^\circ C$.
	}
\end{ex}

\begin{ex}%[2D1V5-5]
	\immini{
		Cho hàm số $y=f(x)$ xác định trên $\mathbb{R}$, hàm số $y=f'(x)$ liên tục trên $\mathbb{R}$ và có đồ thị như hình vẽ bên. Hàm số $g(x)=f\left(3-\mathrm{e}^{x}\right)$ đồng biến trên khoảng nào dưới đây.
		\choice
		{\True$(2; 5)$}
		{$(-1; 0)$}
		{$(0; 1)$}
		{$(1; 2)$}
	}
	{
		\begin{tikzpicture}[scale=0.7, line join=round, line cap=round, >=stealth]
			\tikzset{every node/.style={scale=1}}
			\def\xmin{-2}\def\xmax{4}\def\ymin{-2}\def\ymax{3}
			\draw[->] (\xmin-0.2,0)--(\xmax+0.2,0) node[below]{$x$};
			\draw[->] (0,\ymin-0.2)--(0,\ymax+0.2) node[right]{$y$};
			\draw (0,0) node[below left]{$O$};
			\foreach \x in {1}\draw (\x,0.1)--(\x,-0.1) node[below]{$\x$};
			\foreach \x in {-1}\draw (\x,-0.1)--(\x,0.1) node[above left]{$\x$};
			\foreach \x in {2}\draw (\x,0.1)--(\x,-0.1) node[below right]{$\x$};
			\foreach \y in {-1,1}\draw (0.1,\y)--(-0.1,\y) node[left]{$\y$};
			\foreach \y in {2}\draw (0.1,\y)--(-0.1,\y) node[above left]{$\y$};
			\clip (\xmin,\ymin) rectangle (\xmax,\ymax);
			\draw[thick,smooth,samples=200,domain=\xmin:\xmax] plot(\x,{0.98*((\x)^2-1)*((\x)-2)});
		\end{tikzpicture}
	}
	\loigiai{
		Ta có $g'(x)=-\mathrm{e}^{x} \cdot f'\left(3-\mathrm{e}^{x}\right)$.\\
		Để hàm số đồng biến thì \[-\mathrm{e}^{x} \cdot f'\left(3-\mathrm{e}^{x}\right)>0 \Leftrightarrow f'\left(3-\mathrm{e}^{x}\right)<0.\]
		Dựa vào đồ thị hàm số ta được
		\[
			f'\left(3-\mathrm{e}^{x}\right)<0 \Leftrightarrow \hoac{&3-\mathrm{e}^{x}<-1\\&1<3-\mathrm{e}^{x}<2} \Leftrightarrow \hoac{&\mathrm{e}^{x}>4\\&1<\mathrm{e}^{x}<2} \Leftrightarrow \hoac{&x>\ln 4\\&0<x<\ln 2.}
		\]
		Suy ra hàm số đồng biến trên $(0;\ln 2)$ và $(\ln 4;+\infty)$.\\
		Mà $(2;5)\subset (\ln 4;+\infty)$ nên hàm số đồng biến trên $(2;5)$.
	}

\end{ex}

\begin{ex}%[Mức độ 2]%[2D1H1-5]
	Cho chuyển động thẳng xác định bởi phương trình $S=t^3-3t^2+4t$, trong đó $t$ tính bằng giây $(s)$ và $S$ được tính bằng mét $~\mathrm{(m)}$. Gia tốc của chất điểm từ thời điểm $t=1$ đến thời điểm $t=2$ giây thay đổi như thế nào?
	\choice
	{Gia tốc tăng rồi giảm}
	{Gia tốc giảm}
	{\True Gia tốc tăng}
	{Gia tốc không thay đổi}
	\loigiai
	{
		Vận tốc của chất điểm là $v=s'=3t^2-6t+4$.\\
		Gia tốc của chất điểm là $a=v'=6t-6$.\\
		Vậy từ thời điểm $t=1$ giây đến $t=2$ giây, gia tốc của vật luôn tăng.
	}
\end{ex}

\begin{ex}%[BG12, Tran Tony]%[2D1H2-2]
	\immini{
		Cho hàm số $y=f(x)$ có bảng biến thiên như hình vẽ bên. Điểm cực tiểu của hàm số $y=f(3x)$ là
		\choice
		{\True $x=\dfrac{2}{3}$}
		{$x=2$}
		{$y=-3$}
		{$x=-\dfrac{2}{3}$}
	}{
		\begin{tikzpicture}[font=\footnotesize, line join=round, line cap=round, >=stealth]
			\tkzTabInit[nocadre=false,lgt=1.2,espcl=2.2,deltacl=0.6]{$x$ /0.6,$f'(x)$ /0.6,$f(x)$ /2}{$-\infty$,$-1$,$2$,$+\infty$}
			\tkzTabLine{,+,0,-,+,}
			\tkzTabVar{-/$-\infty$,+/$4$,-/$-3$,+/$+\infty$}
		\end{tikzpicture}
	}
	\loigiai{
		Ta có $y'=3f'(3x)$, $f'(3x)=0\Leftrightarrow \hoac{&3x=-1\\&3x=2}\Leftrightarrow\hoac{&x=-\dfrac{1}{3}\\&x=\dfrac{2}{3}.}$\\
		Do $f'(x)$ và $3f'(3x)$ cùng dấu nên hàm số $y=f(3x)$ có điểm cực tiểu là $x=\dfrac{2}{3}$.
	}
\end{ex}

\begin{ex}%[2-D1B5-SO-14-2425]%[VN-MT-7, Đỗ Minh Phúc]%[2D1V3-6]
	Khi nuôi cá thí nghiệm trong hồ, một nhà khoa học đã nhận thấy rằng: nếu trên mỗi đơn vị diện tích của mặt hồ có $n$ con cá thì trung bình mỗi con cá sau một vụ cân nặng là $P(n)=800-20n$ (g). Hỏi phải thả bao nhiêu con cá trên một đơn vị diện tích của mặt hồ để sau một vụ thu hoạch được nhiều cá nhất?
	\choice
	{$19$}
	{\True $20$}
	{$21$}
	{$22$}
	\loigiai{
		Gọi $F(n)$ là hàm cân nặng của $n$ con cá sau vụ thu hoạch trên một đơn vị diện tích.\\
		Ta có $F(n)=(800-20n) \cdot n=800n-20n^2$.\\
		Để sau một vụ thu hoạch được nhiều cá nhất thì cân nặng của $n$ con cá trên một đơn vị điện tích của mặt hồ là lớn nhất.\\
		Bài toán trở thành tìm $n\in \mathbb{N}^*$ sao cho $F(n)$ đạt giá trị lớn nhất.\\
		Ta có $F'(n)=800-40n$.\\
		Cho $F'(n)=0 \Leftrightarrow 800-40n=0 \Leftrightarrow n=20$.\\
		Ta có bảng biến thiên
		\begin{center}
			\begin{tikzpicture}
				\tkzTabInit[nocadre,lgt=1.2,espcl=2.5,deltacl=0.6]
				{$n$/0.6,$F'(n)$/0.6,$F(n)$/2}{$-\infty$,$20$,$+\infty$}
				\tkzTabLine{,+,0,-,}
				\tkzTabVar{-/$-\infty$,+/$8\,000$,-/$-\infty$}
			\end{tikzpicture}
		\end{center}
		Vậy phải thả $20$ con cá trên một đơn vị diện tích của mặt hồ để sau một vụ thu hoạch được nhiều cá nhất.
	}
\end{ex}

\begin{ex}%[BG12new-4in1, Trần Hoà]%[2D1H1-1]
	Cho hàm số $y=f(x)$ có  $f'(x)=(x+1)^2(x-1)^3(2-x), \forall x\in \mathbb{R}$. Hàm số $y=f(x)$ đồng biến trên khoảng nào dưới đây?
	\choice
	{\True $(1;2)$}
	{$(-\infty;-1)$}
	{$(-1;1)$}
	{$(2;+\infty)$}
	\loigiai
	{
		Dựa vào bảng xét dấu của $f'(x)$:
		\begin{center}
			\begin{tikzpicture}
				\tkzTabInit[lgt=1.2,espcl=1.2]
				{$x$ /1, $f'(x)$ /1}
				{$-\infty$, $-1$,$1$, $2$, $+\infty$}
				\tkzTabLine{ ,-,z,-,z,+,z,-, }
			\end{tikzpicture}
		\end{center}
		ta suy ra hàm số $y=f(x)$ đồng biến trên $(1;2)$.
	}
\end{ex}

\begin{ex}%[Dự Án Giảng 12 4 in 1, Lê Văn Toàn]%[2D1N5-7]
	Đồ thị hàm số $y=\dfrac{2x+1}{x-1}$ cắt trục tung tại điểm có tung độ bằng
	\choice
	{$1$}
	{$-\dfrac{1}{2}$}
	{\True $-1$}
	{$2$}
	\loigiai{
		Gọi $M$ là giao điểm của đồ thị với trục tung, suy ra $x_M=0$.\\
		Thay vào biểu thức của đồ thị hàm số ta được $y_M=-1$.}
\end{ex}

\begin{ex}%[2D1H2-7]
	\immini{Đường dây điện $110KV$ kéo từ trạm phát (điểm $A$) trong đất liền ra Côn Đảo (điểm $C$). Biết khoảng cách ngắn nhất từ $C$ đến $B$ là $60$ km, khoảng cách từ $A$ đến $B$ là $100$ km, mỗi km dây điện dưới nước chi phí là $5000$ USD, chi phí cho mỗi km dây điện trên bờ là $3000$ USD. Hỏi điểm $D$ cách điểm $A$ bao nhiêu để mắc dây điện từ $A$ đến $D$ rồi từ $D$ đến $B$ chi phí đạt cực tiểu? (hình vẽ bên)
		\choice
		{$40$ km}
		{$50$ km}
		{\True $55$ km}
		{$45$ km}}{
		\begin{tikzpicture}[scale=1, font=\footnotesize, line join=round, line cap=round, >=stealth]
			\path
			(0,0) coordinate (A)
			(3,0) coordinate (D)
			(5,0) coordinate (B)
			(B)+(90:4) coordinate (C)
			;

			\draw (A)--(B)--(C)--(D) (A)--(C);

			\foreach \p/\r in {A/90,B/-120,D/-90,C/90}
			\fill (\p) circle (1.5pt) node[shift={(\r:3mm)}]{$\p$};
		\end{tikzpicture}
	}
	\loigiai{
		\immini{
			Đặt khoảng cách từ $D$ đến $B$ là $x$, $0\le x\le 100$. Khi đó khoảng cách từ $A$ đến $D$ là $100 - x$ km và khoảng cách từ $D$ đến $C$ là $\sqrt{x^2 + 3600}$ km. \\
			Chi thí cho việc kéo đường dây điện từ $A$ đến $D$ rồi đến $C$ được tính theo công thức
			\[f(x)=3000\left(100 - x\right) + 5000\sqrt{x^2 + 3600}=300000 - 3000x + 5000\sqrt{x^2 + 3600}.\]
			Ta xác định $x$ sao cho $f$ đạt cực tiểu. \\
			Ta có $f'(x)= - 3000 + \dfrac{5000x}{\sqrt{x^2 + 3600}}=0\Leftrightarrow x=45$.
		}
		{
			\begin{tikzpicture}[scale=1, font=\footnotesize, line join=round, line cap=round, >=stealth]
				\path
				(0,0) coordinate (A)
				(3,0) coordinate (D)
				(5,0) coordinate (B)
				(B)+(90:4) coordinate (C)
				;

				\draw (A)--node[midway,below]{$100-x$}(D)--node[midway,below]{$x$}(B)--(C)--(D) (A)--(C);

				\foreach \p/\r in {A/90,B/-120,D/-90,C/90}
				\fill (\p) circle (1.5pt) node[shift={(\r:3mm)}]{$\p$};
			\end{tikzpicture}
		}
		Bảng biến thiên
		\begin{center}
			\begin{tikzpicture}
				\tkzTabInit[nocadre=false,lgt=1.2,espcl=2.5,deltacl=0.6]
				{$x$ /0.6,$f'(x)$ /0.6,$f(x)$ /2}
				{$0$,$45$,$100$}
				\tkzTabLine{,+,0,-,}
				\tkzTabVar{-/$f(0)$,+/$f(45)$,-/$f(100)$}
			\end{tikzpicture}
		\end{center}
		\noindent
		Dựa vào bảng biến thiên, hàm số $f(x)$ đạt cực tiểu khi $x=45$. Vậy khoảng cách cần tính để chi phí kéo dây là $55$ km.
	}
\end{ex}

\begin{ex}%[2D1H5-1]
	\immini{Người ta muốn chế tạo một chiếc hộp hình hộp chữ nhật có thể tích $500$ $\mathrm{cm}^3$. Chiều cao hộp phải là $2$ cm, các kích thước khác là $x, y$ với $x > 0$ và $y > 0$.
		Công thức xác định diện tích toàn phần $S(x)$ của chiếc hộp theo $x$ là
		\choice
		{$S(x)= 500+4 x-\dfrac{1000}{x}$}
		{\True $S(x)= 500+4 x+\dfrac{1000}{x}$}
		{$S(x)= 250+4 x+\dfrac{1000}{x}$}
		{$S(x)= 500+2 x+\dfrac{1000}{x}$}
	}{\begin{tikzpicture}[declare function={a=2;b=4;h=2;},line join=round,scale =0.8]
			\path (0,0) coordinate (B)
			(35:a) coordinate (A)
			(b,-1) coordinate (C)
			($(C)-(B)+(A)$) coordinate (D)
			($(A)+(90:h)$) coordinate (A')
			($(B)-(A)+(A')$) coordinate (B')
			($(C)-(A)+(A')$) coordinate (C')
			($(D)-(A)+(A')$) coordinate (D');
			\fill[orange!10] (B)--(B')--(A')--(D')--(D)--(C)--cycle;
			\draw ( B')--(B)--(C)--(D)--(D')--(A')--(B')--(C')--(D')  (C)--(C');
			\path (B)--(C)node[pos=0.5,sloped,black,below]{$x$};
			\path (C)--(D)node[pos=0.5,sloped,black,below]{$y$};
			\path (D)--(D')node[pos=0.5,sloped,black,below,scale=0.8]{$2$ cm};
			\draw[dashed]  (A')--(A)--(D)  (A)--(B);
		\end{tikzpicture}}
	\loigiai{
		\begin{itemize}
			\item Biểu thị $y$ theo $x$.\\
			      Ta có $500=x\cdot y \cdot 2 \Rightarrow y=\dfrac{250}{x}$.
			\item Diện tích toàn phần của chiếc hộp là
			      \[S(x)= 2\cdot 2\cdot x+2\cdot 2\cdot y+2\cdot x\cdot y= 500+4 x+\dfrac{1000}{x}.\]
		\end{itemize}
	}
\end{ex}

\begin{ex}%[2-D1B5-SO-16-2425]%[VN-MT-7, Dương Phước Sang]%[2D1H3-1]
	Giá trị lớn nhất của hàm số $f(x)=x^3-8x^2+16x-9$ trên đoạn $[1;3]$ là
	\choice
	{$\max\limits_{[1; 3]} f(x)=0$}
	{\True $\max\limits_{[1; 3]} f(x)=\dfrac{13}{27}$}
	{$\max\limits_{[1; 3]} f(x)=-6$}
	{$\max\limits_{[1; 3]} f(x)=5$}
	\loigiai{
		Hàm số $f(x)$ liên tục trên $[1;3]$.\\
		Ta có $f'(x)=3x^2-16x+16$; $f'(x)=0 \Leftrightarrow \hoac{&x=4 \notin (1;3)\\&x=\dfrac{4}{3}\in (1;3).}$\\
		$f(1)=0$; $f\left( \dfrac{4}{3} \right)=\dfrac{13}{27}$; $f(3)=-6$.\\
		Do đó $\max\limits_{x \in [1;3]} f(x)=f\left( \dfrac{4}{3} \right)=\dfrac{13}{27}$.
	}
\end{ex}

\begin{ex}%[2-D1B3-SO-7-2425]%[VN-MT-7, VM024]%[2D1N4-1]
	Cho hàm số $ y=f(x)$ có $\lim\limits_{x\to +\infty}f(x)=2$, $\lim\limits_{x\to -\infty}f(x)=+\infty$.
	\choice
	{Đồ thị hàm số đã cho có hai đường tiệm cận ngang}
	{Đồ thị hàm số đã cho có đúng một tiệm cận ngang là đường thẳng $x=2$}
	{\True Đồ thị hàm số đã cho có đúng một tiệm cận ngang}
	{Đồ thị hàm số đã cho không có tiệm cận ngang}
	\loigiai{
		Ta có $\lim\limits_{x\to +\infty}f(x)=2$. Do đó, đường thẳng $y=2$ là tiệm cận ngang của đồ thị hàm số $y=f(x)$.
	}
\end{ex}

\begin{ex}%[Mức độ N]%[2D1N3-1]
	Tìm giá trị lớn nhất của hàm số $y=\dfrac{3\sin x+2}{\sin x+1}$ trên đoạn $\left[0;\dfrac{\pi}{2}\right]$.
	\choice{\True $\dfrac{5}{2}$}{$\dfrac{11}{2}$}{$\dfrac{31}{2}$}{$2$}
	\loigiai{Đặt $t=\sin x$, $t \in [0;1]$.\\
		Xét hàm số $f(t)=\dfrac{3t+2}{t+1}$, $t \in [0;1].$\\Ta có $f'(t)=\dfrac{1}{(t+1)^2}>0$, $t\in [0;1]$.\\
		Vậy	$\underset{[0;1]}{\text{max}}f(t)=f(1)=\dfrac{5}{2}$.}
\end{ex}

\begin{ex}%[BG12new-4in1, Trần Hoà]%[2D1H1-2]
	\immini{Cho hàm số $y=f(x)$ xác định, liên tục trên $\mathbb{R}$ và có đạo hàm $f'(x)$. Biết rằng $f'(x)$ có đồ thị như hình vẽ bên. Mệnh đề nào sau đây đúng?
		\choice
		{\True Hàm số $y=f(x)$ nghịch biến trên khoảng $\left(0; + \infty\right)$}
		{Hàm số $y=f(x)$ nghịch biến trên khoảng $(- 3;-2)$}
		{Hàm số $y=f(x)$ đồng biến trên khoảng $\left(- \infty; 3\right)$}
		{Hàm số $y=f(x)$ đồng biến trên khoảng $(- 2; 0)$}}{\begin{tikzpicture}[>=stealth,line join=round,line cap=round,font=\footnotesize,scale=1]
			\draw[->] (-4.1,0)--(2.1,0) node[below left] {$x$};
			\draw[->] (0,-2.6)--(0,2.6) node[below left] {$y$};
			\draw[fill=black] (0,0) circle (1pt) node[above left] {$O$};
			\foreach \x in {-3,-2}
			\draw[thin] (\x,1pt)--(\x,-1pt) node [below left] {$\x$};
			\begin{scope}
				\clip (-4,-2.6) rectangle (2,2.6);
				\draw[samples=200,domain=-4:2,smooth,variable=\x] plot (\x,{(-(\x)-3)*((\x)+2)*((\x)^2)});
			\end{scope}
		\end{tikzpicture}}
	\loigiai{
		Từ đồ thị của hàm số, ta nhận thấy
		Với $\forall x\in \left(- 3; - 2\right)$, $f'(x)>0$ nên hàm số đồng biến.
		Với $\forall x\in \left(- \infty; - 3\right)$ và $(- 2; 0)$ và $\left(0; + \infty\right)$, $f'(x)<0$ nên hàm số nghịch biến.
		Vậy hàm số nghịch biến trên $\left(0; + \infty\right)$.}
\end{ex}

\begin{ex}%[Mức độ N]%[2D1N2-1]
	Hàm số nào dưới đây không có cực trị?
	\choice{$y=\dfrac{x^2+1}{x}$}{\True $y=\dfrac{2x-2}{x+1}$}{$y=x^2-2x+1$}{$y=-x^3+x+1$}
	\loigiai{Xét hàm số $y=\dfrac{2x-2}{x+1}\text{, } \forall x \ne-1$.\\ Ta có $y'=\dfrac{4}{(x+1)^2}>0\text{, } \forall x \ne -1$.\\
		Vậy hàm số $y=\dfrac{2x-2}{x+1}$ không cực trị.}
\end{ex}

\begin{ex}%[2D1N5-1]
	\immini[thm]{Bảng biến thiên ở hình bên là của một trong bốn hàm số sau đây. Hỏi đó là hàm số nào?
		\choice
		{$y=-x^3-2x^2+5$}
		{\True $y=x^3-3x^2+5$}
		{$y=-x^3-3x+5$}
		{$y=x^3+3x^2+5$}}{
		\begin{tikzpicture}
			\tkzTabInit[nocadre=false, lgt=1.2, espcl=1.6]{$x$ /0.6,$f'(x)$ /0.6,$f(x)$ /1.5}{$-\infty$,$0$,$2$,$+\infty$}
			\tkzTabLine{,+,$0$,-,$0$,+,}
			\tkzTabVar{-/ $-\infty$/, +/$5$ , -/$1$  , +/$+\infty$/}
		\end{tikzpicture}}
	\loigiai{

	}
\end{ex}

\begin{ex}%[Mức độ 2]%[Dự án giảng new 4in1, Trần Quang Thạnh]%[2D1H1-4]
	Cho hàm số $y=f(x)$ có bảng biến thiên sau
	\begin{center}
		\begin{tikzpicture}
			\tkzTabInit[nocadre=false,lgt=1.2,espcl=2.5]
			{$x$ /0.7,$f(x)$ /2}{$-\infty$,$0$,$4$,$+\infty$}
			\tkzTabVar{+/$+\infty$,-/$-5$,+/$-1$,-/$0$}
		\end{tikzpicture}
	\end{center}
	Bất phương trình $f(8x) < f(3x-185)$ có bao nhiêu nghiệm nguyên âm?
	\choice
	{$39$}
	{$38$}
	{$37$}
	{\True $36$}
	\loigiai{
		Ta thấy $f(x)$ nghịch biến trên khoảng $(0;+\infty)$.\\
		Với $x<0$, ta có $8x<0$ và $3x-185<0$, do đó
		\[f(8x) < f(3x-185) \Leftrightarrow 8x>3x-185 \Leftrightarrow x>-37.\]
		Vì $x$ nguyên âm nên $x\in\{-36;-35;\ldots;-1\}$.
	}
\end{ex}

\begin{ex}%[TEX NBV, Phạm Hoài]%[2D1N1-2]
	\immini[thm]{
		Biết hàm số $y=\dfrac{x+a}{x+1}$ ($a$ là số thực cho trước, $a\neq 1$ có đồ thị như hình bên). Mệnh đề nào dưới đây đúng?
		\choice
		{$y'<0, \,\forall x\neq -1$}
		{\True  $y'>0, \,\forall x\neq -1$}
		{$y'<0, \,\forall x\in \mathbb{R}$}
		{$y'>0, \,\forall x\in \mathbb{R}$}
	}{\begin{tikzpicture}[line join=round, line cap=round,>=stealth,thick,scale=0.75]
			\tikzset{every node/.style={scale=0.9}}
			\draw[->] (-4.1,0)--(4.2,0) node[below left] {$x$};
			\draw[->] (0,-4.1)--(0,5.2) node[below left] {$y$};
			\draw (0,0) node [below left] {$O$};
			\foreach \x/\nx in {1/1,2/2,3/3}
			\draw[thin] (\x,1pt)--(\x,-1pt) node [below] {$\nx$};
			\foreach \x/\nx in {-1/-1,-2/-2}
			\draw[thin] (\x,1pt)--(\x,-1pt) node [above right] {$\nx$};
			\foreach \x/\nx in {-4/-4,-3/-3}
			\draw[thin] (\x,1pt)--(\x,-1pt) node [above] {$\nx$};
			\foreach \y/\ny in {-1/-1,1/1,2/2,3/3,4/4}
			\draw[thin] (1pt,\y)--(-1pt,\y) node [left] {$\ny$};
			\foreach \y/\ny in {-4/-4,-3/-3,-2/-2}
			\draw[thin] (1pt,\y)--(-1pt,\y) node [right] {$\ny$};
			%\draw[dashed,thin](2,0)--(2,-6)--(0,-6);
			%\draw[dashed,thin] (1.01,-10)--(1.01,2);
			\begin{scope}
				\clip (-4,-4) rectangle (4,5);
				\draw[samples=200,domain=-5:-1.1,smooth,variable=\x] plot (\x,{(-1*(\x)-3)/(2*(\x)+2)});
				\draw[samples=200,domain=-.7:5,smooth,variable=\x] plot (\x,{(-1*(\x)-3)/(2*(\x)+2)});
				\draw (-1,-4)--(-1,5) (-5,-0.5)--(5,-0.5);
			\end{scope}
		\end{tikzpicture}}
	\loigiai{Dựa vào đồ thị, hàm số đồng biến trên từng khoảng xác định. Do đó $y'>0\, \forall x\ne -1$ suy ra $1-a>0\Rightarrow a<1$.
	}
\end{ex}

\begin{ex}%[Sách tham khảo, Mức độ H]%[Dự án giảng 12 - Trung Anh]%[2D1H2-4]
	Tìm tất cả các giá trị thực của tham số $m$ để hàm số $y=x^3-3(m+1)x^2+12mx+2019$ có hai điểm cực trị $x_1,\ x_2$ thỏa mãn $x_1+x_2+2x_1x_2=-8$.
	\choice
	{\True $m=-1$}
	{$m=2$}
	{$m=1$}
	{$m=-2$}
	\loigiai{
		Ta có $y'=3x^2-6(m+1)x+12m,\ y'=0\Leftrightarrow 3x^2-6(m+1)x+12m=0$. \\
		Hàm số có hai điểm cực trị $\Leftrightarrow \Delta '=9m^2-18m+9>0\Leftrightarrow m\ne 1$.\tagEX{1}
		Giả sử $x_1,\ x_2$ là hai nghiệm của phương trình $y'=0$, theo định lí Vi-ét ta có
		\[\heva{&x_1+x_2=-\dfrac{b}{a}=2(m+1)\\&x_1\cdot x_2=\dfrac{c}{a}=4m.}\]
		Do đó $x_1+x_2+2x_1\cdot x_2=-8\Leftrightarrow 2(m+1)+8m=-8\Leftrightarrow 10m=-10\Leftrightarrow m=-1$ thỏa mãn $(1)$.\\
		Vậy $m=-1$ là giá trị cần tìm của $m$.}
\end{ex}

\begin{ex}%[ĐỀ CHUẨN HÓA CHƯƠNG 1-GIẢI TÍCH 12]%[Huỳnh Đức Vũ]%[2D1V5-6]
	Cho hàm số $y=-x^3+3x^2+2$ có đồ thị $(C)$. Biết rằng, tại điểm $M$ thuộc $(C)$ tiếp tuyến của $(C)$ có hệ số góc lớn nhất. Tìm phương trình tiếp tuyến đó.
	\choice
	{\True $y=3x+1$}
	{$y=-3x+1$}
	{$y=-3x-1$}
	{$y=3x-1$}
	\loigiai{
		$y'=-3x^2+6x=-3(x-1)^2+3\leq 3$. \\
		Tiếp tuyến của $(C)$ có hệ số góc lớn nhất bằng $3$ tại điểm $M(1;4)$ có phương trình là
		$y=3(x-1)+4=3x+1$.}
\end{ex}

\begin{ex}%[Mức độ 3]giảng 12, Phạm Tiến Long]%[2D1V4-3]
	Tìm tham số $m$ để đồ thị hàm số $f(x)=\dfrac{x^2-mx+1}{x-2}$ có tiệm cận xiên cắt hai trục tọa độ $Ox$, $Oy$ tại hai điểm $A$, $B$ sao cho tam giác $OAB$  có diện tích bằng $8$.
	\choice
	{$m=2$ hoặc $m=6$}
	{\True $m=-2$ hoặc $m=6$}
	{$m=2$ hoặc $m=-6$}
	{$m=-2$ hoặc $m=-6$}
	\loigiai{	Hàm số đã cho có tập xác định $\mathscr{D}=\mathbb{R}\backslash\{-1\}$.\\
		Ta có $\begin{aligned}[t]
				a & =\lim\limits_{x \rightarrow+\infty} \dfrac{f(x)}{x}=\lim\limits_{x \rightarrow+\infty} \dfrac{x^2-mx+1}{x^2-2x}=1;                                                                \\
				b & =\lim\limits_{x \rightarrow+\infty}[f(x)-ax]=\lim\limits_{x \rightarrow+\infty}\left(\dfrac{x^2-mx+1}{x-2}-x\right)=\lim\limits_{x \rightarrow+\infty} \dfrac{(2-m)x+1}{x-2}=2-m.
			\end{aligned}$\\
		Ta cũng có $\lim\limits_{x \rightarrow-\infty} \dfrac{f(x)}{x}=1$; $\lim\limits_{x \rightarrow-\infty}[f(x)-x]=2-m$.\\
		Do đó, tiệm cận xiên của đồ thị hàm số là đường thẳng $d\colon y=x+2-m$.\\
		Đường thẳng $d$ cắt hai trục tọa độ tại hai điểm $A(0;2-m)$ và $B(m-2;0)$.\\
		Dễ thấy tam giác $OAB$ vuông cân tại $O$.\\
		Với điều kiện $m\ne 2$, ta có
		\begin{eqnarray*}
			& & \dfrac{1}{2}\cdot OA^2=8\\
			&\Leftrightarrow & \dfrac{1}{2} \cdot (2-m)^2=8\\
			&\Leftrightarrow & (2-m)^2=16\\
			&\Leftrightarrow & \hoac{&2-m=4\\&2-m=-4}\\
			&\Leftrightarrow & \hoac{&m=-2\text{ (thỏa điều kiện)}\\&m=6.\text{ (thỏa điều kiện)}}
		\end{eqnarray*}
		Vậy $m=-2$ hoặc $m=6$.
	}
\end{ex}

\begin{ex}%[Dự án TL12New-4in1-NCT]%[2D1V4-1]
	Cho hàm số $y=f(x)$ có bảng biến thiên như sau
	\begin{center}
		\begin{tikzpicture}[>=stealth]
			\tkzTabInit[nocadre=false,lgt=1,espcl=3,deltacl=0.5]{$x$/.7 ,$y'$/.7,$y$/2}
			{$-\infty$ , $-2$ , $2$ , $+\infty$}
			\tkzTabLine{, + , $0$ , - , $0$ , + ,}
			\tkzTabVar{-/$-\infty$ , +/$3$ , -/$0$ , +/$+\infty$}
		\end{tikzpicture}
	\end{center}
	Đồ thị hàm số $y=\dfrac{1}{f(3-x)-2}$ có bao nhiêu tiệm cận đứng?
	\choice
	{$0$}
	{$2$}
	{\True $3$}
	{$1$}
	\loigiai{
		Ta thấy $f(x)=2$ có $3$ nghiệm $\Rightarrow$ đồ thị hàm số $y=\dfrac{1}{f(3-x)-2}$ có $3$ tiệm cận đứng.}
\end{ex}

\begin{ex}%[Dự án TL12New-4in1-NCT]%[2D1H4-2]
	Tìm tất cả các giá trị thực của tham số $m$ để đồ thị hàm số $y=\dfrac{mx+3}{\sqrt{mx^2-5}}$ có hai đường tiệm cận ngang.
	\choice
	{$m\geq 0$}
	{$m>\sqrt{5}$}
	{$m<0$}
	{\True $m>0$}
	\loigiai{
		Ta có $\lim\limits_{x\to+\infty}\dfrac{mx+3}{\sqrt{mx^2-5}}=\lim\limits_{x\to+\infty}\dfrac{m+\dfrac{3}{x}}{\sqrt{m-\dfrac{5}{x^2}}}$ và $\lim\limits_{x\to-\infty}\dfrac{mx+3}{\sqrt{mx^2-5}}=\lim\limits_{x\to-\infty}\dfrac{m+\dfrac{3}{x}}{-\sqrt{m-\dfrac{5}{x^2}}}$.\\
		Để đồ thị hàm số $y=\dfrac{mx+3}{\sqrt{mx^2-5}}$ có hai đường tiệm cận ngang thì $m>0$.\\
		Khi đó hai đường tiệm cận ngang là $y=\pm\sqrt{m}$.}
\end{ex}

\begin{ex}%[Mức độ 2]giảng 12, Phạm Tiến Long]%[2D1H4-3]
	Gọi $d$ là tiệm cận xiên của đồ thị hàm số $f(x)=2x-4+\dfrac{1}{3x+4}$. Giao điểm của $d$ với trục tung là
	\choice
	{$M(2;0)$}
	{$N(-2;0)$}
	{$P(0;4)$}
	{\True $Q(0;-4)$}
	\loigiai{
		Hàm số đã cho có tập xác định là $\mathbb{R}\backslash \left\{-\dfrac{4}{3}\right\}$.\\
		Ta có $\lim\limits_{x\to +\infty}[f(x)-(2x-4)]=0$ và $\lim\limits_{x\to -\infty}[f(x)-(2x-4)]=0$.\\
		Do đó, đồ thị hàm số có tiệm cận xiên là đường thẳng $d\colon y=2x-4$.\\
		Giao điểm của $d$ với trục tung là $Q(0;-4)$.
	}
\end{ex}
\begin{ex}%[CD12 - CTST, Mức độ 2] %[2D1H1-5]
	Chi phí sản xuất $x$ sản phẩm mỗi tháng của một công ty cho bởi hàm $\overline{C(x)}$ có bảng biến thiên như sau
	\begin{center}
		\begin{tikzpicture}
			\tikzset{double style/.append style={double distance=2pt}}
			\tkzTabInit[lgt=1.2, espcl=2]
			{$x$/0.6,$\overline{C'}(x)$/0.6,$\overline{C}(x)$/1.5}{$0$,$1000$,$+\infty$}
			\tkzTabLine{,-,0,+,}
			\tkzTabVar{+/,-/$60$,+/$+\infty$}
		\end{tikzpicture}
	\end{center}
	Hỏi khi số sản phẩm mỗi tháng vượt qua giá trị bao nhiêu thì chi phí sản xuất bắt đầu tăng.
	\choice
	{\True $1\,000$}
	{$60$}
	{$500$}
	{$3\,60$}
	\loigiai{
		Từ bảng biến thiên ta thấy khi mỗi tháng xưởng sản xuất vượt quá $1\,000$ sản phẩm thì chi phí trung bình  sản xuất một sản phẩm thấp bắt đầu tăng.
	}
\end{ex}
\Closesolutionfile{ans}
\TL
\begin{ex}%[SGK12-CTST, Mức độ 2]%[2D1H2-1]
	Xét sự biến thiên và các điểm cực trị của hàm số $g(x)=\dfrac{x^2+x+4}{x+1}$.
	\loigiai{
		Tập xác định $\mathscr{D}=\mathbb{R}\setminus \{-1\}$.\\
		Ta có $g(x)=x+\dfrac{4}{x+1} \Rightarrow g'(x)=1-\dfrac{4}{(x+1)^2}=\dfrac{x^2+2x-3}{(x+1)^2}$;\\
		$g'(x)=0 \Leftrightarrow x^2+2x-3=0 \Leftrightarrow \hoac{& x=-3\\& x=1.}$\\
		Bảng biến thiên
		\begin{center}
			\begin{tikzpicture}
				\tkzTabInit[nocadre=false,lgt=1.2,espcl=2.5,deltacl=0.6]
				{$x$ /0.6,$g'(x)$ /0.6,$g(x)$ /2}
				{$-\infty$,$-3$,$-1$,$1$,$+\infty$}
				\tkzTabLine{,+,$0$,-,d,-,$0$,+,}
				\tkzTabVar{-/$-\infty$,+/$-5$,-D+/$-\infty$/$+\infty$,-/$3$,+/$+\infty$}
			\end{tikzpicture}
		\end{center}
		Vậy hàm số đồng biến trên mỗi khoảng $(-\infty;-3)$ và $(1;+\infty)$; nghịch biến trên mỗi khoảng $(-3;-1)$ và $(-1;1)$. Hàm số đạt cực đại tại $x=-3$, $y_{\text{CĐ}}=g(-3)=-5$; và hàm số đạt cực tiểu tại $x=1$, $y_{\text{CT}}=g(1)=3$.
	}
\end{ex}
\begin{ex}%[2-D1B4-SO-10-2425]%[VN-MT-7, VM012]%[2D1V5-8]
	Giả sử chi phí cho xuất bản $x$ cuốn tạp chí (gồm: lương cán bộ, công nhân viên, giấy in,\ldots) được cho bởi công thức
	$C(x)=0{,}0\,001x^2-0{,}2x+10\,000$,
	trong đó $C(x)$ được tính theo đơn vị là vạn đồng ($1$ vạn đồng $=$ 10\,000 đồng). Chi phí phát hành cho mỗi cuốn là $4$ nghìn đồng. Tỉ số $M(x)=\dfrac{T(x)}{x}$ được gọi là chi phí trung bình cho một cuốn tạp chí khi xuất bản $x$ cuốn và tổng chi phí $T(x)$ (xuất bản và phát hành) cho $x$ cuốn tạp chí. Tìm chi phí trung bình thấp nhất cho một cuốn tạp chí là bao nhiêu vạn đồng, biết rằng nhu cầu hiện tại xuất bản không quá 30\,000 cuốn?
	% \shortans{2{,}2}
	\loigiai{
	Chi phí phát hành cho mỗi cuốn là $4$ nghìn đồng, tức là $0{,}4$ vạn đồng.\\
	Suy ra chi phí phát hành cho $x$ cuốn là $0{,}4x$ (vạn đồng).\\
	Theo đề bài, ta có tổng chi phí xuất bản và phát hành cho $x$ cuốn tạp chí là\\
	$T(x)=C(x)+0{,}4x=0{,}0\,001x^2+0{,}2x+10\,000$, với $x > 0$.\\
	Ta có $f(x)=M(x)=\dfrac{T(x)}{x}=0{,}0\,001x+0{,}2+\dfrac{10\,000}{x}$.\\
	Xét hàm số $f(x)=0{,}0\,001x+0{,}2+\dfrac{10\,000}{x}$, với $0< x\le 30\,000$.\\
	$f'(x)=0{,}0\,001-\dfrac{10\,000}{x^2}=\dfrac{0{,}0\,001x^2-10\,000}{x^2}$, $f'(x)=0\Leftrightarrow x=10\,000$ (do $x>0$).\\
	$\lim\limits_{x\to 0^+} f(x)=+\infty$.\\
	Bảng biến thiên:
	\begin{center}
		\begin{tikzpicture}[>=stealth]
			\tkzTabInit[nocadre=true,lgt=1.2,espcl=2.5,deltacl=0.6]{$x$/.7 ,$f'(x)$/.7,$f(x)$/2}
			{$0$ , $10\,000$ , $30\,000$}
			\tkzTabLine{ , - , $0$ , + , }
			\tkzTabVar{+/$+\infty$ , -/$f(10\,000)$ , +/$f(30\,000)$}
		\end{tikzpicture}
	\end{center}
	Dựa vào bảng biến thiên, ta thấy giá trị của $M(x)$ nhỏ nhất khi $x=10\,000$.\\
	Do đó, số lượng tạp chí cần xuất bản sao cho chi phí trung bình thấp nhất là $x=10\,000$ (cuốn).\\
	Vậy chi phí trung bình cho một cuốn tạp chí khi xuất bản $10\,000$ cuốn là $M(10\,000)=2{,}2$ (vạn đồng).
	}
\end{ex}
\begin{ex}%[2-D1B1-SO-1-2425]%[VN-MT-7, Nguyễn Cao Cường]%[2D1C2-7]
	\immini[thm]{Người ta muốn thiết kế một lồng nuôi cá có bề mặt hình chữ nhật bao gồm phần mặt nước có diện tích bằng $54$ m$^2$ và phần đường đi xung quanh có thiết kế như hình vẽ (đơn vị: mét). Khi kích thước $a$ thay đổi trong khoảng $(3;+\infty)$ thì giá trị hàm số mô tả diện tích lối đi theo kích thước $a$ sẽ giảm đến giá trị $S_0$ rồi tăng lên. Xác định giá trị $S_0$.
	}
	{\begin{tikzpicture}[>=stealth,line join=round,line cap=round,font=\footnotesize,scale=0.75]
			\path
			(0,0) coordinate (A)
			(5,0) coordinate (B)
			(0,4) coordinate (D)
			($(D)+(B)-(A)$) coordinate (C)
			($(B)+(-0.5,0)$) coordinate (M)
			($(B)+(0,0.5)$) coordinate (N)
			($(A)+(1,0)$) coordinate (H)
			($(A)+(0,0.5)$) coordinate (T)
			($(C)+(-0.5,0)$) coordinate (Q)
			($(C)+(0,-0.5)$) coordinate (P)
			($(D)+(1,0)$) coordinate (R)
			($(D)+(0,-0.5)$) coordinate (S)
			($(T)+(H)-(A)$) coordinate (A')
			($(M)+(N)-(B)$) coordinate (B')
			($(Q)+(P)-(C)$) coordinate (C')
			($(R)+(S)-(D)$) coordinate (D')
			($(D)+(0,0.5)$) coordinate (x)
			($(C)+(0,0.5)$) coordinate (y)
			($(D)+(-0.5,0)$) coordinate (u)
			($(A)+(-0.5,0)$) coordinate (v)
			($(A)+(0,-0.5)$) coordinate (x')
			($(H)+(0,-0.5)$) coordinate (y')
			($(M)+(0,-0.5)$) coordinate (u')
			($(B)+(0,-0.5)$) coordinate (v')
			($(C)+(0.5,0)$) coordinate (x'')
			($(P)+(0.5,0)$) coordinate (y'')
			($(N)+(0.5,0)$) coordinate (u'')
			($(B)+(0.5,0)$) coordinate (v'')
			;
			\draw[fill=cyan!20!brown](A)--(B)--(C)--(D)--(A);
			\draw[fill=cyan!90!blue](A')--(B')--(C')--(D')--(A');
			\draw(D)--(x) (C)--(y)(D)--(u)(A)--(v)(A)--(x')(H)--(y')(M)--(u')(B)--(v')(C)--(x'') (P)--(y'') (N)--(u'') (B)--(v'');
			\draw[<->] (x)--(y)node[pos=0.5,above]{$a$};
			\draw[<->] (u)--(v)node[pos=0.5,left]{$b$};
			\draw[<->] (x')--(y')node[pos=0.5,below]{$2$};
			\draw[<->] (u')--(v')node[pos=0.5,below]{$1$};
			\draw[<->] (u'')--(v'')node[pos=0.5,right]{$1$};
			\draw[<->] (x'')--(y'')node[pos=0.5,right]{$1$};
		\end{tikzpicture}}
	% \shortans{42}
	\loigiai{
		Gọi $x$, $y$ lần lượt là độ dài, rộng của mặt nước. Điều kiện $x$, $y>0$.\\
		Phần mặt nước có diện tích bằng $54$ m$^2$ nên ta có $xy=54$. \quad\quad $(*)$\\
		Theo đề bài ta có $x=a-3$, $y=b-2$.\\
		Từ $(*)$ suy ra \[(a-3)(b-2)=54\Rightarrow b=\dfrac{54}{a-3}+2=\dfrac{2a+48}{a-3}.\]
		Diện tích lối đi là
		\allowdisplaybreaks
		\begin{eqnarray*}
			S(a)&=&a\cdot b-x\cdot y\\
			&=&ab-54\\
			&=&a\cdot \dfrac{2a+48}{a-3}-54\\
			&=&\dfrac{2a^2+48a}{a-3}-54.
		\end{eqnarray*}
		$S'(a)=\dfrac{2a^2-12a-144}{\left(a-3\right)^2}$.\\
		Xét $S'(a)=0\Leftrightarrow \hoac{&a=-6\\&a=12.}$\\
		Bảng biến thiên
		\begin{center}
			\begin{tikzpicture}
				\tkzTabInit[nocadre=true,lgt=1.2,espcl=4,deltacl=0.5]
				{$a$ /0.7,$S'(a)$ /0.7,$S(a)$ /2}
				{$3$,$12$,$+\infty$}
				\tkzTabLine{,-,$0$,+,}
				\tkzTabVar{+/$+\infty$,-/$42$,+/$+\infty$}
			\end{tikzpicture}
		\end{center}
		Vậy $S_0=42$.
	}
\end{ex}
\begin{ex}%[Dự Án Giảng 12 4 in 1, Lê Văn Toàn]%[2D1C5-6]
	Cho hàm số $y=\dfrac{1}{4}x^4-\dfrac{7}{2}x^2$ có đồ thị $(C)$. Tiếp tuyến tại điểm $A$ thuộc $(C)$ cắt $(C)$ tại hai điểm phân biệt $M\left(x_1;y_1\right)$, $N\left(x_2;y_2\right)$ ($M$, $N$ khác $A)$ thỏa mãn $y_1-y_2=6\left(x_1-x_2\right)$. Các điểm $A$ thỏa mãn có tổng các hoành độ là
	% \shortans{$-3$}
	\loigiai{
		Gọi $A\left(x_0;y_0\right)\in\,(C)$ là tọa độ tiếp điểm của phương trình tiếp tuyến.\\
		Ta có hệ số góc $k=y'\left(x_0\right)=x_0^3-7x_0$.\\
		Phương trình tiếp tuyến $y=k\left(x-x_0\right)+y_0=\left(x_0^3-7x_0\right)\left(x-x_0\right)+y_0$.\\
		Ta có
		\begin{eqnarray*}
			&&y_1-y_2=6\left(x_1-x_2\right)\\
			&\Leftrightarrow& k\left(x_1-x_0\right)+y_0-\left[k\left(x_2-x_0\right)+y_0\right]=6\left(x_1-x_2\right)\\
			&\Leftrightarrow& k\left(x_1-x_2\right)=6\left(x_1-x_2\right)\\
			&\Leftrightarrow& k=6\\
			&\Leftrightarrow& x_0^3-7x_0=6\\
			&\Leftrightarrow& x_0^3-7x_0-6=0\\
			&\Leftrightarrow& \hoac{&x_0=3\Rightarrow y_0=-\dfrac{45}{4}\\&x_0=-1\Rightarrow y_0=-\dfrac{13}{4}\\&x_0=-2\Rightarrow y_0=-10.}
		\end{eqnarray*}
		Khi đó các phương trình tiếp tuyến tương ứng là
		\[\hoac{&d_1\colon y=6(x-3)-\dfrac{45}{4}=6x-\dfrac{117}{4}\\&d_2\colon y=6(x+1)-\dfrac{13}{4}=6x+\dfrac{11}{4}\\&d_3\colon y=6(x+2)-10=6x+2.}\]
		Phương trình hoành độ giao điểm của $(C)$ với các tiếp tuyến là
		\[\hoac{&\dfrac{1}{4}x^4-\dfrac{7}{2}x^2-6x+\dfrac{117}{4}=0\text{ (có 1 nghiệm nên không thỏa)}\\&\dfrac{1}{4}x^4-\dfrac{7}{2}x^2-6x-\dfrac{11}{4}=0\text{ (có 3 nghiệm nên thỏa mãn)}\\&\dfrac{1}{4}x^4-\dfrac{7}{2}x^2-6x-2=0\text{ (có 3 nghiệm nên thỏa mãn).}}\]
		Do đó tổng các hoành độ điểm các tiếp điểm là $-1-2=-3$.
	}
\end{ex}
% \begin{indapan}
% 	{ans/ansc1l4}
% \end{indapan}


%C2
% \begin{name}
	{\tenchude}
	{ĐỀ ÔN TẬP CHƯƠNG II}
	{LỚP TOÁN THẦY PHÁT}
	{\thoigian}
\end{name}

\TN
\Opensolutionfile{ans}[ans/ans\currfilebase-Phan-I]

\begin{ex}%[2-H2B4-SO-10-2425 (Nguồn: Bài 4 - Đề 1 - Ôn Tập Chương II)]%[VN-MT-7, Trần Bảo Hiên]%[2H2H1-2]
Cho tứ diện $ABCD$. Gọi $G$ là trọng tâm tam giác $BCD$ và điểm $M$ thuộc cạnh $AB$ sao cho $AM=2BM$. Đẳng thức nào sau đây là đúng?
\choice
{$\overrightarrow{MG}=\overrightarrow{AB}+\overrightarrow{AC}+\overrightarrow{AD}$}
{$\overrightarrow{MG}=\dfrac{1}{3}\overrightarrow{AB}-\dfrac{1}{3}\overrightarrow{AC}-\dfrac{1}{3}\overrightarrow{AD}$}
{\True $\overrightarrow{MG}=-\dfrac{1}{3}\overrightarrow{AB}+\dfrac{1}{3}\overrightarrow{AC}+\dfrac{1}{3}\overrightarrow{AD}$}
{$\overrightarrow{MG}=\dfrac{4}{3}\overrightarrow{AB}-\dfrac{1}{3}\overrightarrow{AC}-\dfrac{1}{3}\overrightarrow{AD}$}
\loigiai{
\begin{center}
\begin{tikzpicture}[scale=0.8,font=\footnotesize,line join=round,line cap=round,>=stealth]
\coordinate (A) at (-1,4);
\coordinate (B) at (-3,0);
\coordinate (C) at (1,-2);
\coordinate (D) at (3,0);
\coordinate (I) at ($(C)!1/2!(D)$);
\coordinate (G) at ($(B)!2/3!(I)$);
\coordinate (M) at ($(A)!2/3!(B)$);
\draw(A)--(B)--(C)--(D)--(A)--(C);
\draw[dashed](I)--(B)--(D) (M)--(G);
\foreach \i/\g in {A/90,B/180,C/-90,D/0,G/-90,M/135}{\fill (\i) circle (1.0pt)($(\i)+(\g:3mm)$) node[scale=1]{$\i$};}
\end{tikzpicture}
\end{center}
Ta có $M$ thuộc cạnh $AB$ và $AM=2BM$ nên $\overrightarrow{AM}=\dfrac{2}{3}\overrightarrow{AB}$.\\
Do $G$ là trọng tâm tam giác $BCD$ nên $3\overrightarrow{AG}=\overrightarrow{AB}+\overrightarrow{AC}+\overrightarrow{AD}$ hay $\overrightarrow{AG}=\dfrac{1}{3}\left(\overrightarrow{AB}+\overrightarrow{AC}+\overrightarrow{AD}\right)$.\\
Mà $\overrightarrow{MG}=\overrightarrow{AG}-\overrightarrow{AM}$ nên $\overrightarrow{MG}=\dfrac{1}{3}\left(\overrightarrow{AB}+\overrightarrow{AC}+\overrightarrow{AD}\right)-\dfrac{2}{3}\overrightarrow{AB}=-\dfrac{1}{3}\overrightarrow{AB}+\dfrac{1}{3}\overrightarrow{AC}+\dfrac{1}{3}\overrightarrow{AD}$.
}
\end{ex}

\begin{ex}%[2-H2B4-SO-10-2425 (Nguồn: Bài 4 - Đề 1 - Ôn Tập Chương II)]%[VN-MT-7, Trần Bảo Hiên]%[2H2H1-2]
Cho hình lập phương $ABCD.EFGH$. Hãy xác định góc giữa cặp vectơ $\overrightarrow{AB}$ và $\overrightarrow{EG}$?
\choice
{$60^\circ$}
{\True $45^\circ$}
{$90^\circ$}
{$120^\circ$}
\loigiai{
\begin{center}
\begin{tikzpicture}[scale=0.8, font=\footnotesize, line join=round, line cap=round, >=stealth]
\path
(0:0) coordinate (B)
(0:4) coordinate (C)
($(B)+(45:2)$) coordinate (A)
($(A)+(C)-(B)$) coordinate (D)
($(A)+(90:3)$) coordinate (E)
($(B)+(90:3)$) coordinate (F)
($(C)+(90:3)$) coordinate (G)
($(D)+(90:3)$) coordinate (H);
\draw[dashed] (B)--(A)--(E) (A)--(D)--(B) (A)--(C);
\draw (E)--(F)--(G)--(H)--(D)--(C)--(B)--(F) (E)--(H) (E)--(G)--(C) (F)--(H);
\foreach \x/\g in {A/170,B/170,C/0,D/0,E/170,F/170,G/0,H/0}
\draw[fill=black] (\x) circle (.5pt)($(\g:.3)+(\x)$) node {$\x$};
\end{tikzpicture}
\end{center}
Ta có $\overrightarrow{EG}=\overrightarrow{AC}$ (do $ACGE$ là hình chữ nhật)
$\Rightarrow\left(\overrightarrow{AB},\overrightarrow{EG}\right)=\left(\overrightarrow{AB},\overrightarrow{AC}\right)=\widehat{BAC}=45^\circ$.
}
\end{ex}

\begin{ex}%[2-H2B4-SO-10-2425 (Nguồn: Bài 4 - Đề 1 - Ôn Tập Chương II)]%[VN-MT-7, Trần Bảo Hiên]%[2H2N2-2]
Trong không gian với hệ tọa độ $Oxyz$, cho điểm $A(2;3;-2)$. Gọi $A_1$ là hình chiếu vuông góc của điểm $A$ lên mặt phẳng $(Oyz)$. Khi đó tọa độ của điểm $A_1$ là
\choice
{$(2;3;0)$}
{$(2;0;0)$}
{$(-2;3;-2)$}
{\True $(0;3;-2)$}
\loigiai{
Hình chiếu vuông góc của điểm $A$ lên mặt phẳng $(Oyz)$ là $A_1(0;3;-2)$.
}
\end{ex}

\begin{ex}%[2-H2B4-SO-10-2425 (Nguồn: Bài 4 - Đề 1 - Ôn Tập Chương II)]%[VN-MT-7, Trần Bảo Hiên]%[2H2H2-2]
Trong không gian với hệ tọa độ $Oxyz$, cho vectơ $\overrightarrow{a}=\left(2;\dfrac{1}{3};-5\right)$ và điểm $M(2;3;4)$. Tọa độ điểm $N$ thỏa mãn $\overrightarrow{MN}=\overrightarrow{a}$ là
\choice
{$\left(2;\dfrac{5}{3};-\dfrac{1}{2}\right)$}
{$\left(0;\dfrac{8}{3};9\right)$}
{\True $\left(4;\dfrac{10}{3};-1\right)$}
{$\left(0;-\dfrac{8}{3};-9\right)$}
\loigiai{
Gọi tọa độ điểm $N$ là $\left(x_N;y_N;z_N\right)$, ta có $\overrightarrow{MN}=\left(x_N-2;y_N-3;z_N-4\right)$.\\
Ta có $\overrightarrow{MN}=\overrightarrow{a}\Leftrightarrow\heva{&x_N-2=2\\&y_N-3=\dfrac{1}{3}\\&z_N-4=-5}\Leftrightarrow\heva{&x_N=2+2\\&y_N=\dfrac{1}{3}+3\\&z_N=-5+4}\Leftrightarrow\heva{&x_N=4\\&y_N=\dfrac{10}{3}\\&z_N=-1.}$\\
Vậy $N\left(4;\dfrac{10}{3};-1\right)$.
}
\end{ex}

\begin{ex}%[2-H2B4-SO-10-2425 (Nguồn: Bài 4 - Đề 1 - Ôn Tập Chương II)]%[VN-MT-7, Trần Bảo Hiên]%[2H2N2-3]
Trong không gian với hệ toạ độ $Oxyz$, cho các vectơ $\overrightarrow{a}=(1;1;2)$ và $\overrightarrow{b}=(-2;0;1)$. Tọa độ của vectơ $\overrightarrow{u}=\overrightarrow{a}-\overrightarrow{b}$ là
\choice
{\True $\overrightarrow{u}=(3;1;1)$}
{$\overrightarrow{u}=(-1;1;1)$}
{$\overrightarrow{u}=(3;1;-3)$}
{$\overrightarrow{u}=(1;3;3)$}
\loigiai{
Ta có $\overrightarrow{u}=\overrightarrow{a}-\overrightarrow{b} \Rightarrow \overrightarrow{u}=\left(1-(-2);1-0;2-1\right) \Rightarrow \overrightarrow{u}=(3;1;1)$.
}
\end{ex}

\begin{ex}%[2-H2B4-SO-10-2425 (Nguồn: Bài 4 - Đề 1 - Ôn Tập Chương II)]%[VN-MT-7, Trần Bảo Hiên]%[2H2H2-2]
Trong không gian với hệ tọa độ $Oxyz$, cho điểm $M(4;1;-2)$ và vectơ $\overrightarrow{u}=(4;-2;6)$. Tìm tọa độ điểm $N$ biết rằng $\overrightarrow{MN}=-\dfrac{1}{2}\overrightarrow{u}$.
\choice
{$(3;3;3)$}
{$(3;-3;3)$}
{\True $(2;2;-5)$}
{$(-3;-3;3)$}
\loigiai{
Ta có $-\dfrac{1}{2}\overrightarrow{u}=(-2;1;-3)$.\\
Gọi tọa độ điểm $N$ là $\left(x_N;y_N;z_N\right)$, ta có $\overrightarrow{MN}=\left(x_N-4;y_N-1;z_N+2\right)$.\\
Ta có $\overrightarrow{MN}=-\dfrac{1}{2}\overrightarrow{u}\Leftrightarrow\heva{&x_N-4=-2\\&y_N-1=1\\&z_N+2=-3}\Leftrightarrow\heva{&x_N=2\\&y_N=2\\&z_N=-5.}$\\
Vậy $N(2;2;-5)$.
}
\end{ex}

\begin{ex}%[2-H2B4-SO-10-2425 (Nguồn: Bài 4 - Đề 1 - Ôn Tập Chương II)]%[VN-MT-7, Trần Bảo Hiên]%[2H2N2-3]
Trong không gian với hệ tọa độ $Oxyz$, cho hai điểm $A(2;-1;4)$, $B(5;3;-8)$. Độ dài của vectơ $\overrightarrow{AB}$ là
\choice
{$5$}
{$8$}
{$9$}
{\True $13$}
\loigiai{
Ta có $\overrightarrow{AB}=(3;4;-12)$.\\
Độ dài của vectơ $\overrightarrow{AB}$ là $\left\vert\overrightarrow{AB}\right\vert=\sqrt{3^2+4^2+(-12)^2}=13$.
}
\end{ex}

\begin{ex}%[2-H2B4-SO-10-2425 (Nguồn: Bài 4 - Đề 1 - Ôn Tập Chương II)]%[VN-MT-7, Trần Bảo Hiên]%[2H2H2-3]
Trong không gian với hệ tọa độ $Oxyz$, cho hai vectơ $\overrightarrow{a}=(1;-2;-3)$, $\overrightarrow{b}=(-2;m-1;2)$. Tìm tham số $m$ để vectơ $\overrightarrow{a}$ vuông góc với vectơ $\overrightarrow{b}$.
\choice
{\True $m=-3$}
{$m=1$}
{$m=5$}
{$m=0$}
\loigiai{
Ta có $\overrightarrow{a}\perp\overrightarrow{b}\Leftrightarrow\overrightarrow{a}\cdot\overrightarrow{b}=0\Leftrightarrow 1\cdot(-2)+(-2)\cdot(m-1)+(-3)\cdot2=0\Leftrightarrow-2-2m+2-6=0\Leftrightarrow m=-3$.
}
\end{ex}

\begin{ex}%[2-H2B4-SO-10-2425 (Nguồn: Bài 4 - Đề 1 - Ôn Tập Chương II)]%[VN-MT-7, Trần Bảo Hiên]%[2H2N2-2]
Trong không gian với hệ tọa độ $Oxyz$, cho điểm $A(4;0;0)$, $B(0;2;0)$. Tâm đường tròn ngoại tiếp tam giác $OAB$ là
\choice
{$I(2;-1;0)$}
{$I\left(\dfrac{4}{3};\dfrac{2}{3};0\right)$}
{$I(-2;1;0)$}
{\True $I(2;1;0)$}
\loigiai{
Ta có $A(4;0;0)\in Ox$, $B(0;2;0)\in Oy$ nên tam giác $OAB$ vuông tại $O$.\\
Do đó, tâm đường tròn ngoại tiếp tam giác $OAB$ là trung điểm $I$ của cạnh $AB$.\\
Vậy $I(2;1;0)$.
}
\end{ex}

\begin{ex}%[2-H2B4-SO-10-2425 (Nguồn: Bài 4 - Đề 1 - Ôn Tập Chương II)]%[VN-MT-7, Trần Bảo Hiên]%[2H2H2-2]
Cho hai điểm $A(1;2;3)$ và $B(3;0;-5)$. Gọi $M$ là điểm đối xứng của $A$ qua $B$. Tọa độ của điểm $M$ là
\choice
{$(2;-2;-8)$}
{\True $(5;-2;-13)$}
{$(2;1;-1)$}
{$(7;2;-7)$}
\loigiai{
Vì $M$ là điểm đối xứng của $A$ qua $B$ nên $B$ là trung điểm của $AM$.
Gọi $M(x_M;y_M;z_M)$, ta có
\begin{center}
$\heva{&x_B=\dfrac{x_A+x_M}{2}\\&y_B=\dfrac{y_A+y_M}{2}\\&z_B=\dfrac{z_A+z_M}{2}}\Leftrightarrow\heva{&x_M=2x_B-x_A\\&y_M=2y_B-y_A\\&z_M=2z_B-z_A}\Leftrightarrow\heva{&x_M=2\cdot3-1=5\\&y_M=2\cdot0-2=-2\\&z_M=2\cdot(-5)-3=-13.}$
\end{center}
Vậy $M(5;-2;-13)$.
}
\end{ex}

\begin{ex}%[2-H2B4-SO-10-2425 (Nguồn: Bài 4 - Đề 1 - Ôn Tập Chương II)]%[VN-MT-7, Trần Bảo Hiên]%[2H2H2-2]
Cho tam giác $MNP$ có $M(-1;3;2)$, $N(2;2;0)$ và $P(-1;1;1)$. Biết $N$ là trọng tâm của tam giác $MNQ$. Điểm $Q$ có tọa độ là
\choice
{\True $(8;2;-3)$}
{$(4;-2;0)$}
{$(2;0;-2)$}
{$(0;-2;-2)$}
\loigiai{
Do $N$ là trọng tâm của của tam giác $MPQ$ nên $\heva{&2=\dfrac{-1-1+x_Q}{3}\\&2=\dfrac{3+1+y_Q}{3}\\&0=\dfrac{2+1+z_Q}{3}}\Leftrightarrow\heva{&x_Q=8\\&y_Q=2\\&z_Q=-3.}$\\
Vậy $N(8;2;-3)$.
}
\end{ex}

\begin{ex}%[2-H2B4-SO-10-2425 (Nguồn: Bài 4 - Đề 1 - Ôn Tập Chương II)]%[VN-MT-7, Trần Bảo Hiên]%[2H2C2-6]
Một chiếc máy ảnh được đặt trên giá đỡ ba chân với điểm đặt $E(0;0;8)$ và các điểm tiếp xúc với mặt đất của ba chân lần lượt là $A_1(0;1;0)$, $A_2\left(\dfrac{\sqrt{3}}{2};-\dfrac{1}{2};0\right)$, $A_3\left(-\dfrac{\sqrt{3}}{2};-\dfrac{1}{2};0\right)$.\\
\begin{center}
\begin{tikzpicture}[line join = round, line cap=round,>=stealth,font=\footnotesize,scale=1]
\draw[dashed,orange] (0,0) ellipse (2cm and 1cm);
\path
(0,0) coordinate (O)
(170:2cm and 1cm) coordinate (Q1)
(-10:2cm and 1cm) coordinate (Q1')
(60:2cm and 1cm) coordinate (Q4)
(-120:2cm and 1cm) coordinate (Q4')
(-70:2cm and 1cm) coordinate (Q2)
(10:2cm and 1cm) coordinate (Q3)
($(0,0)+(0,4)$) coordinate (S)
($(Q1)!-0.5!(Q1')$) coordinate (L1)
($(Q4)!1.5!(Q4')$)coordinate (L2)
($(L1)+(L2)-(O)$) coordinate (L3)
($2*(O)-(L3)$) coordinate (L4)
($2*(L1)-(L3)$) coordinate (L5)
($2*(L2)-(L3)$) coordinate (L6)
(-150:2cm and 1cm) coordinate (A2)
(80:2cm and 1cm) coordinate (A3)
(-70:2cm and 1cm) coordinate (Q2)
(-7:2cm and 1cm) coordinate (A1) ;
\draw[fill=cyan,opacity=0.2] (L3)--(L5)--(L4)--(L6)--cycle;;
\draw[->] ($(Q1)!-0.5!(Q1')$)--($(Q1)!1.5!(Q1')$) node[right]{$y$};
\draw[->] ($(Q4)!-0.5!(Q4')$)--($(Q4)!1.5!(Q4')$) node[below]{$x$};
\fill
(A1) circle(2pt) node[above right]{$A_1$}
(A3) circle(2pt) node[above right]{$A_3$}
(A2) circle(2pt) node[below left]{$A_2$}
(S) circle(2pt) node[below right,xshift=0.4cm]{$E\left(0;0;8\right)$}
(O) circle(2pt) node[below right]{$O$};
\draw[line width=1pt] (S)--(A1) (S)--(A2) (S)--(A3) ;
\draw[->,line width=1.5pt,black] (0,0)--($(O)!1.5!(S)$)node[right]{$z$};
\draw[->,line width=1.5pt,red] (S)--($(S)!0.6!(A1)$) node[above right]{$\overrightarrow{F}_1$};
\draw[->,line width=1.5pt,red] (S)--($(S)!0.6!(A2)$) node[left]{$\overrightarrow{F}_2$};
\draw[->,line width=1.5pt,red] (S)--($(S)!0.5!(A3)$) node[below right]{$\overrightarrow{F}_3$};
\draw ($(S)+(-0.2,0.5)$) node[]{
\begin{tikzpicture}[line join = round, line cap=round,>=stealth,font=\footnotesize,scale=1]
\draw[fill=black] (-0.75,-0.5)rectangle (0.5,0.5);
\draw[fill=cyan] (-0.65,-0.35)rectangle (0.35,-0.25);
\draw[fill=white] (-0.65,0.35)rectangle (0.35,0.25);
\draw[fill=black] (-0.65,0.5)rectangle (-0.5,0.75);
\draw[fill=black] (-0.65,0.65)rectangle (0.5,0.85);
\draw[fill=black] (-0.4,0)--(-1,0.5)--(-1,-0.5)--cycle;
\end{tikzpicture}
};
\end{tikzpicture}
\end{center}
Biết rằng trọng lượng của chiếc máy là $240N$. Tọa độ của lực $\overrightarrow{F_1}$ là
\choice
{\True $\overrightarrow{F_1}=(0;10;-80)$}
{$\overrightarrow{F_1}=(0;10;80)$}
{$\overrightarrow{F_1}=(0;-10;-80)$}
{$\overrightarrow{F_1}=(10;0;-80)$}
\loigiai{
Ta có $\overrightarrow{EA_1}=(0;1;-8)$, $\overrightarrow{EA_2}=\left(\dfrac{\sqrt{3}}{2};-\dfrac{1}{2};-8\right)$, $\overrightarrow{EA_3}=\left(-\dfrac{\sqrt{3}}{2};-\dfrac{1}{2};-8\right)$.\\
Nên $EA_1=EA_2=EA_3=\sqrt{65}$.\\
Mặt khác, $\left\vert\overrightarrow{F_1}\right\vert=\left\vert\overrightarrow{F_2}\right\vert=\left\vert\overrightarrow{F_3}\right\vert$ vì đèn cân bằng và trọng lực của đèn tác dụng đều lên 3 chân của giá đỡ.\\
Do đó $\overrightarrow{F_1}=k\overrightarrow{EA_1}=(0;k;-8k)$, $\overrightarrow{F_2}=k\overrightarrow{EA_2}=\left(\dfrac{\sqrt{3}}{2}k;-\dfrac{1}{2}k;-8k\right)$, $\overrightarrow{F_3}=k\overrightarrow{EA_3}=\left(-\dfrac{\sqrt{3}}{2}k;-\dfrac{1}{2}k;-8k\right)$.\\ $\Rightarrow\overrightarrow{F_1}+\overrightarrow{F_2}+\overrightarrow{F_3}=(0;0;-24k)$.\\
Mà $\overrightarrow{F_1}+\overrightarrow{F_2}+\overrightarrow{F_3}=\overrightarrow{P}=(0;0;-240)\Rightarrow -24k=-240\Rightarrow k=10$.\\
Vậy $\overrightarrow{F_1}=(0;10;-80)$.
}
\end{ex}
\Closesolutionfile{ans}

\TNTF
\Opensolutionfile{ans}[ans/ans\currfilebase-Phan-II]

\begin{ex}%[2-H2B4-SO-10-2425 (Nguồn: Bài 4 - Đề 1 - Ôn Tập Chương II)]%[VN-MT-7, Trần Bảo Hiên]%[2H2V1-2]
Cho hình lập phương $ABCD.A'B'C'D'$ có cạnh bằng $a$. Trên các cạnh $CD$ và $BB'$ ta lần lượt lấy các điểm $M$ và $N$ sao cho $DM=BN=x$ với $0\le x\le a$.
\choiceTF
{\True $\overrightarrow{AC'}=\overrightarrow{AA'}+\overrightarrow{AB}+\overrightarrow{AD}$}
{\True Gọi $K$ là trung điểm $AD$. Khi đó $\overrightarrow{C'K}=\overrightarrow{C'C}+\overrightarrow{C'D'}+\dfrac{1}{2}\overrightarrow{C'B'}$}
{$\overrightarrow{AB}\cdot\overrightarrow{B'D'}=a^2$}
{\True Góc giữa vectơ $\overrightarrow{AC'}$ và $\overrightarrow{MN}$ bằng $90^\circ$}
\loigiai{
\begin{center}
\begin{tikzpicture}[scale=0.85, font=\footnotesize, line join=round, line cap=round, >=stealth]
\path
(0:0) coordinate (B)
(0:4) coordinate (C)
($(B)+(45:2)$) coordinate (A)
($(A)+(C)-(B)$) coordinate (D)
($(A)+(90:4)$) coordinate (A')
($(B)+(90:4)$) coordinate (B')
($(C)+(90:4)$) coordinate (C')
($(D)+(90:4)$) coordinate (D')
($(A)!1/2!(D)$) coordinate (K)
($(D)!2/3!(C)$) coordinate (M)
($(B)!2/3!(B')$) coordinate (N);
\draw[dashed] (B)--(A)--(A') (D)--(A)--(C')--(K) (M)--(N);
\draw (D')--(A')--(B')--(C')--(D')--(D)--(C)--(B)--(B') (C')--(C);
\foreach \x/\g in {A/170,B/170,C/0,D/0,A'/170,B'/170,C'/0,D'/0,K/120,M/0,N/180}
\draw[fill=black] (\x) circle (.5pt)($(\g:.3)+(\x)$) node {$\x$};
\end{tikzpicture}
\end{center}
\begin{itemchoice}
\itemch \textbf{Đúng}.\\
Ta có $\overrightarrow{AC'}=\overrightarrow{AA'}+\overrightarrow{AC}$ (do $A'C'CA$ là hình bình hành).\\
Ngoài ra, ta có $\overrightarrow{AC}=\overrightarrow{AB}+\overrightarrow{AD}$ (do $ABCD$ là hình bình hành).\\
Suy ra $\overrightarrow{AC'}=\overrightarrow{AA'}+\overrightarrow{AB}+\overrightarrow{AD}$.
\itemch \textbf{Đúng}.\\
Ta có
\begin{align*}
\overrightarrow{C'K}=\overrightarrow{C'C}+\overrightarrow{CK}&=\overrightarrow{C'C}+\dfrac{1}{2}\left(\overrightarrow{CA}+\overrightarrow{CD}\right)\\ &=\overrightarrow{C'C}+\dfrac{1}{2}\left(\overrightarrow{C'A'}+\overrightarrow{C'D'}\right)\\
&=\overrightarrow{C'C}+\dfrac{1}{2}\left(\overrightarrow{C'B'}+\overrightarrow{C'D'}+\overrightarrow{C'D'}\right)\\
&=\overrightarrow{C'C}+\dfrac{1}{2}\overrightarrow{C'B'}+\overrightarrow{C'D'}.
\end{align*}
\itemch \textbf{Sai}.\\
Ta có
\begin{align*}
\overrightarrow{AB}\cdot\overrightarrow{B'D'}&=\overrightarrow{AB}\cdot\left(\overrightarrow{B'A'}+\overrightarrow{B'C'}\right)\\
&=\overrightarrow{AB}\cdot\left(-\overrightarrow{AB}+\overrightarrow{AD}\right)\\
&=-\overrightarrow{AB}\cdot\overrightarrow{AB}+\overrightarrow{AB}\cdot\overrightarrow{AD}\\
&=-{\overrightarrow{AB}}^2=-a^2.
\end{align*}
\itemch \textbf{Đúng}.\\
Ta đặt $\overrightarrow{AA'}=\overrightarrow{a}$, $\overrightarrow{AB}=\overrightarrow{b}$, $\overrightarrow{AD}=\overrightarrow{c}$. Ta có $\left\vert\overrightarrow{a}\right\vert=\left\vert\overrightarrow{b}\right\vert=\left\vert\overrightarrow{c}\right\vert=a$.\\
$\overrightarrow{AC'}=\overrightarrow{AA'}+\overrightarrow{AB}+\overrightarrow{AD}$ hay $\overrightarrow{AC'}=\overrightarrow{a}+\overrightarrow{b}+\overrightarrow{c}$.\\
Mặt khác $\overrightarrow{MN}=\overrightarrow{AN}-\overrightarrow{AM}=\left(\overrightarrow{AB}+\overrightarrow{BN}\right)-\left(\overrightarrow{AD}+\overrightarrow{DM}\right)$ với $\overrightarrow{BN}=\dfrac{x}{a}\overrightarrow{a}$ và $\overrightarrow{DM}=\dfrac{x}{a}\overrightarrow{b}$.\\
Do đó $\overrightarrow{MN}=\left(\overrightarrow{b}+\dfrac{x}{a}\overrightarrow{a}\right)-\left(\overrightarrow{c}+\dfrac{x}{a}\overrightarrow{b}\right)=\dfrac{x}{a}\overrightarrow{a}+\left(1-\dfrac{x}{a}\right)\overrightarrow{b}-\overrightarrow{c}$.\\
Ta có $\overrightarrow{AC'}\cdot\overrightarrow{MN}=\left(\overrightarrow{a}+\overrightarrow{b}+\overrightarrow{c}\right)\left[\dfrac{x}{a}\overrightarrow{a}+\left(1-\dfrac{x}{a}\right)\overrightarrow{b}-\overrightarrow{c}\right]$.\\
Vì $\overrightarrow{a}\cdot\overrightarrow{b}=\overrightarrow{a}\cdot\overrightarrow{c}=\overrightarrow{b}\cdot\overrightarrow{c}=0$ nên ta có
\begin{center}
$\overrightarrow{AC'}\cdot\overrightarrow{MN}=\dfrac{x}{a}{\overrightarrow{a}}^2+\left(1-\dfrac{x}{a}\right){\overrightarrow{b}}^2-{\overrightarrow{c}}^2=x\cdot a+\left(1-\dfrac{x}{a}\right)a^2-a^2=0$.
\end{center}
Vậy góc giữa vectơ $\overrightarrow{AC'}$ và $\overrightarrow{MN}$ bằng $90^\circ$.
\end{itemchoice}
}
\end{ex}

\begin{ex}%[2-H2B4-SO-10-2425 (Nguồn: Bài 4 - Đề 1 - Ôn Tập Chương II)]%[VN-MT-7, Trần Bảo Hiên]%[2H2V2-5]
Trong không gian với hệ tọa độ $Oxyz$, cho hình bình hành $ABCD$ có $A(-3;4;2)$, $B(-5;6;2)$, $C(-10;17;-7)$.
\choiceTF
{\True Tọa độ trung điểm của $AB$ là $I(-4;5;2)$}
{\True Tọa độ trọng tâm của tam giác $ABC$ là $G(-6;9;-1)$}
{$\overrightarrow{AB}\cdot\overrightarrow{AD}=10$}
{Tọa độ trực tâm của tam giác $ABD$ là $H(-5;12;4)$}
\loigiai{
\begin{itemchoice}
\itemch \textbf{Đúng}.\\
Gọi $I$ là trung điểm của $AB$. Khi đó $\heva{&x_I=\dfrac{x_A+x_B}{2}=\dfrac{-3+(-5)}{2}=-4\\&y_I=\dfrac{y_A+y_B}{2}=\dfrac{4+6}{2}=5\\&z_I=\dfrac{z_A+z_B}{2}=\dfrac{2+2}{2}=2.}$\\
Vậy $I(-4;5;2)$.
\itemch \textbf{Đúng}.\\
Gọi $G$ là trọng tâm của tam giác $ABC$.\\
Khi đó $\heva{&x_G=\dfrac{x_A+x_B+x_C}{3}=\dfrac{-3+(-5)+(-10)}{3}=-6\\&y_I=\dfrac{y_A+y_B+y_C}{3}=\dfrac{4+6+17}{3}=9\\&z_I=\dfrac{z_A+z_B+z_C}{3}=\dfrac{2+2+(-7)}{3}=-1.}$\\
Vậy $G(-6;9;-1)$.
\itemch \textbf{Sai}.\\
Ta có $\overrightarrow{AB}=(-2;2;0)$, $\overrightarrow{DC}=(-10-x_D;17-y_D;-7-z_D)$.
Vì $ABCD$ là hình bình hành nên $\overrightarrow{AB}=\overrightarrow{DC}\Leftrightarrow\heva{&-10-x_D=-2\\&17-y_D=2\\&-7-z_D=0}\Leftrightarrow\heva{&x_D=-8\\&y_D=15\\&z_D=-7}\Rightarrow D(-8;15;-7)$.\\
$\overrightarrow{AD}=(-5;11;-9)$. Do đó $\overrightarrow{AB}\cdot\overrightarrow{AD}=-2\cdot(-5)+2\cdot11+0\cdot(-9)=32$.
\itemch \textbf{Sai}.\\
Gọi $H(a;b;c)$ là trực tâm tam giác $ABD$. Do $\left[\overrightarrow{AB},\overrightarrow{AD}\right]$ có giá vuông góc với $(ABC)$ nên nó vuông góc với vectơ $\overrightarrow{AH}$.\\
Do đó $\heva{&\overrightarrow{AH}\perp\overrightarrow{BD}\\&\overrightarrow{BH}\perp\overrightarrow{AD}\\&\left[\overrightarrow{AB},\overrightarrow{AD}\right]\cdot\overrightarrow{AH}=0.}$\\
Ta có 
$\overrightarrow{AH}=(a+3;b-4;c-2)$, $\overrightarrow{BH}=(a+5;b-6;c-2)$, $\overrightarrow{BD}=(-3;9;-9)$, $\overrightarrow{AD}=(-5;11;-9)$, $\overrightarrow{AB}=(-2;2;0)$, 
$\left[\overrightarrow{AB},\overrightarrow{AD}\right]=(-18;-18;-12)$.
Suy ra 
\begin{center}
$\heva{&-3(a+3)+9(b-4)-9(c-2)=0\\&-5(a+5)+11(b-6)-9(c-2)=0\\&-18(a+3)-18(b-4)-12(c-2)=0}\Leftrightarrow\heva{&-3a+9b-9c=27\\&-5a+11b-9c=73\\&-18a-18b-12c=-42}\Leftrightarrow\heva{&a=-\dfrac{153}{11}\\&b=\dfrac{100}{11}\\&c=\dfrac{118}{11}.}$
\end{center}
Vậy $H\left(-\dfrac{153}{11};\dfrac{100}{11};\dfrac{118}{11}\right)$.
\end{itemchoice}
}
\end{ex}

\begin{ex}%[2-H2B4-SO-10-2425 (Nguồn: Bài 4 - Đề 1 - Ôn Tập Chương II)]%[VN-MT-7, Trần Bảo Hiên]%[2H2H2-4]
Trong không gian với hệ tọa độ $Oxyz$, cho các điểm $A(2;1;-1)$, $B(3;1;0)$, $C(-1;1;3)$.
\choiceTF
{\True Ba điểm $A$, $B$, $C$ không thẳng hàng}
{\True Ba điểm $A$, $B$, $D(4;1;1)$ thẳng hàng}
{Góc $\widehat{ABC}=45^\circ$}
{\True $\left[\overrightarrow{AB},\overrightarrow{AC}\right]=(0;-7;0)$}
\loigiai{
\begin{itemchoice}
\itemch \textbf{Đúng}.\\
Ta có $\overrightarrow{AB}=(1;0;1)$, $\overrightarrow{AC}=(-3;0;4)$, $\overrightarrow{AB}\ne k\overrightarrow{AC}=(-3k;0;4k)$ với mọi $k$ nên hai vectơ $\overrightarrow{AB}$ và $\overrightarrow{AC}$ không cùng phương. Do đó ba điểm $A$, $B$, $C$ không thẳng hàng.
\itemch \textbf{Đúng}.\\
Ta có $\overrightarrow{AB}=(1;0;1)$, $\overrightarrow{AD}=(2;0;2)\Rightarrow\overrightarrow{AD}=2\overrightarrow{AB}$. Do đó ba điểm $A$, $B$, $D$ thẳng hàng.
\itemch \textbf{Sai}.\\
Ta có $\overrightarrow{BA}=(-1;0;-1)$, $\overrightarrow{BC}=(-4;0;3)$, suy ra
\begin{center}
$\cos\widehat{ABC}=\cos\left(\overrightarrow{BA},\overrightarrow{BC}\right)=\dfrac{\overrightarrow{BA}\cdot\overrightarrow{BC}}{\left\vert\overrightarrow{BA}\right\vert\cdot\left\vert\overrightarrow{BC}\right\vert}=\dfrac{1}{5\sqrt{2}}\Rightarrow\widehat{ABC}\approx 82^\circ$.\\
\end{center}

\itemch \textbf{Đúng}.\\
Ta có $\overrightarrow{AB}=(1;0;1)$, $\overrightarrow{AC}=(-3;0;4)\Rightarrow\left[\overrightarrow{AB},\overrightarrow{AC}\right]=(0;-7;0)$.
\end{itemchoice}
}
\end{ex}

\begin{ex}%[2-H2B4-SO-10-2425 (Nguồn: Bài 4 - Đề 1 - Ôn Tập Chương II)]%[VN-MT-7, Trần Bảo Hiên]%[2H2V2-5]
Trong không gian với hệ tọa độ $Oxyz$, cho các điểm $A(1;1;2)$, $B(3;-1;2)$, $C(2;0;1)$.
\choiceTF
{\True Ba điểm $A$, $B$, $C$ không thẳng hàng}
{\True Điểm $M(a;b;3)$ thỏa mãn ba điểm $A$, $C$, $M$ thẳng hàng thì $a+b=2$}
{Góc $\alpha$ là góc tạo bởi hai vectơ $\overrightarrow{AB}$, $\overrightarrow{BC}$ thì $\cos\alpha=-1$}
{Gọi điểm $M(a;b;3)$ thỏa mãn ba điểm $A$, $C$, $M$ thẳng hàng. Khi đó $\left[\overrightarrow{AB},\overrightarrow{AM}\right]=(1;1;2)$}
\loigiai{
\begin{itemchoice}
\itemch \textbf{Đúng}.\\
Ta có $\overrightarrow{AB}=(2;-2;0)$, $\overrightarrow{BC}=(-1;1;-1)$. Suy ra $\overrightarrow{AB}\ne k\cdot\overrightarrow{BC}$ với mọi $k\in\mathbb{R}$ nên ba điểm $A$, $B$, $C$ không thẳng hàng.
\itemch \textbf{Đúng}.\\
Ta có $\overrightarrow{AC}=(1;-1;-1)$, $\overrightarrow{AM}=(a-1;b-1;1)$.\\
Ba điểm $A$, $C$, $M$ thẳng hàng khi và chỉ khi $\overrightarrow{AB}=k\overrightarrow{AM}\Leftrightarrow\heva{&1=k(a-1)\\&-1=k(b-1)\\&-1=k}\Leftrightarrow\heva{&a=0\\&b=2\\&k=-1.}$\\
Vậy $a+b=2$.
\itemch \textbf{Sai}.\\
\begin{center}
$\cos\alpha=\cos\left(\overrightarrow{AB},\overrightarrow{BC}\right)=\dfrac{\overrightarrow{AB}\cdot\overrightarrow{BC}}{\left\vert\overrightarrow{AB}\right\vert\cdot\left\vert\overrightarrow{BC}\right\vert}=\dfrac{2\cdot(-1)+(-2)\cdot1+0\cdot(-1)}{2\sqrt{2}\cdot\sqrt{3}}=-\dfrac{\sqrt{6}}{3}$.
\end{center}
\itemch \textbf{Sai}.\\
$\overrightarrow{AB}=(2;-2;0)$, $\overrightarrow{AM}=(-1;1;1)$.\\
$\left[\overrightarrow{AB},\overrightarrow{AM}\right]=(-2;-2;0)$.
\end{itemchoice}
}
\end{ex}
\Closesolutionfile{ans}

\TNSA
\Opensolutionfile{ans}[ans/ans\currfilebase-Phan-III]

\begin{ex}%[2-H2B4-SO-10-2425 (Nguồn: Bài 4 - Đề 1 - Ôn Tập Chương II)]%[VN-MT-7, Trần Bảo Hiên]%[2H2V2-4]
Cho hình chóp $S.ABC$ có $SA=SB=SC=AB=AC=a$, $BC=a\sqrt{2}$. Góc giữa hai véctơ $\overrightarrow{AB}$ và $\overrightarrow{SC}$ bằng bao nhiêu độ?
\shortans{120}
\loigiai{
\begin{center}
\begin{tikzpicture}[scale=0.8,font=\footnotesize,line join=round,line cap=round,>=stealth]
\coordinate (S) at ($(A)+(50:3)$);
\coordinate (A) at (0,0);
\coordinate (B) at ($(A)+(-60:3)$);
\coordinate (C) at ($(A)+(0:5)$);
\draw(S)--(A)--(B)--(C)--(S)--(B);
\draw[dashed](A)--(C);
\foreach \i/\g in {S/90,A/180,B/-90,C/0}{\fill (\i) circle (1.0pt) ($(\i)+(\g:3mm)$)node[scale=1]{$\i$};}
\end{tikzpicture}
\end{center}
Tam giác $ABC$ có $AB=AC=a$, $BC=a\sqrt{2}\Rightarrow\triangle ABC$ vuông tại $A\Rightarrow\overrightarrow{AB}\cdot\overrightarrow{AC}=0$.\\
\begin{align*}
\cos\left(\overrightarrow{SC},\overrightarrow{AB}\right)&=\dfrac{\overrightarrow{SC}\cdot\overrightarrow{AB}}{\left\vert\overrightarrow{SC}\right\vert\cdot\left\vert\overrightarrow{AB}\right\vert}=\dfrac{\left(\overrightarrow{SA}+\overrightarrow{AC}\right)\cdot\overrightarrow{AB}}{SC\cdot AB}\\
&=\dfrac{\overrightarrow{SA}\cdot\overrightarrow{AB}+\overrightarrow{AC}\cdot\overrightarrow{AB}}{SC\cdot AB}=\dfrac{SA\cdot AB\cdot \cos120^\circ}{SC\cdot AB}=-\dfrac{1}{2}.
\end{align*}
Suy ra $\left(\overrightarrow{SC},\overrightarrow{AB}\right)=120^\circ$.
}
\end{ex}

\begin{ex}%[2-H2B4-SO-10-2425 (Nguồn: Bài 4 - Đề 1 - Ôn Tập Chương II)]%[VN-MT-7, Trần Bảo Hiên]%[2H2H2-2]
Trong không gian với hệ tọa độ $Oxyz$, cho hình hộp $ABCD.A'B'C'D'$ có $A(1;0;1)$, $B(2;1;2)$, $D(1;-1;1)$, $C'(4;5;-5)$. Giả sử $A'(x;y;z)$, tính $x+y+z$.
\shortans{2}
\loigiai{
\begin{center}
\begin{tikzpicture}[scale=0.8, font=\footnotesize, line join=round, line cap=round, >=stealth]
\path
(0:0) coordinate (B)
(0:4) coordinate (C)
($(B)+(45:2)$) coordinate (A)
($(A)+(C)-(B)$) coordinate (D)
($(A)+(80:3)$) coordinate (A')
($(B)+(80:3)$) coordinate (B')
($(C)+(80:3)$) coordinate (C')
($(D)+(80:3)$) coordinate (D');
\draw[dashed] (B)--(A)--(A') (A)--(D);
\draw (D')--(A')--(B')--(C')--(D')--(D)--(C)--(B)--(B') (C)--(C');
\foreach \x/\g in {A/170,B/170,C/0,D/0,A'/170,B'/170,C'/0,D'/0}
\draw[fill=black] (\x) circle (.5pt)
($(\g:.3)+(\x)$) node {$\x$};
\end{tikzpicture}
\end{center}
Ta có $\overrightarrow{AC'}=(3;5;-6)$, $\overrightarrow{AB}=(1;1;1)$, $\overrightarrow{AD}=(0;-1;0)$.\\
Theo quy tắc hình bình hành ta có $\overrightarrow{AB}+\overrightarrow{AD}+\overrightarrow{AA'}=\overrightarrow{AC'}$, suy ra
\begin{center}
$\overrightarrow{AA'}=\overrightarrow{AC'}-\overrightarrow{AB}-\overrightarrow{AD}\Rightarrow\overrightarrow{AA'}=(2;5;-7)$.
\end{center}
Ta có $A'(x;y;z)\Rightarrow\overrightarrow{AA'}=(2;5;-7)\Leftrightarrow\heva{&x-1=2\\&y=5\\&z-1=-7}\Leftrightarrow\heva{&x=3\\&y=5\\&z=-6}\Rightarrow A'(3;5;-6)$.\\
Vậy $x+y+z=3+5+(-6)=2$.
}
\end{ex}

\begin{ex}%[2-H2B4-SO-10-2425 (Nguồn: Bài 4 - Đề 1 - Ôn Tập Chương II)]%[VN-MT-7, Trần Bảo Hiên]%[2H2V2-5]
Cho hình hộp $ABCD.A'B'C'D'$ có $A(1;0;1)$, $B(2;1;2)$, $D(1;-1;1)$, $C'(4;5;-5)$. Biết rằng có một vectơ $\overrightarrow{v}=(a;b;6)$ vuông góc với cả hai vectơ $\overrightarrow{CC'}$ và $\overrightarrow{C'D'}$. Tính $a+b$.
\shortans{-6}
\loigiai{
\begin{center}
\begin{tikzpicture}[scale=0.8, font=\footnotesize, line join=round, line cap=round, >=stealth]
\path
(0:0) coordinate (B)
(0:4) coordinate (C)
($(B)+(45:2)$) coordinate (A)
($(A)+(C)-(B)$) coordinate (D)
($(A)+(80:3)$) coordinate (A')
($(B)+(80:3)$) coordinate (B')
($(C)+(80:3)$) coordinate (C')
($(D)+(80:3)$) coordinate (D');
\draw[dashed] (B)--(A)--(A') (A)--(D);
\draw (D')--(A')--(B')--(C')--(D')--(D)--(C)--(B)--(B') (C)--(C');
\foreach \x/\g in {A/170,B/170,C/0,D/0,A'/170,B'/170,C'/0,D'/0}
\draw[fill=black] (\x) circle (.5pt)($(\g:.3)+(\x)$) node {$\x$};
\end{tikzpicture}
\end{center}
Ta có $\overrightarrow{AB}=(1;1;1)$, $\overrightarrow{AD}=(0;-1;0)$.\\
Gọi $C(x;y;z)\Rightarrow\overrightarrow{AC}=(x-1;y;z-1)$. Theo quy tắc hình bình hành ta có
\begin{center}
$\overrightarrow{AC}=\overrightarrow{AB}+\overrightarrow{AD}\Leftrightarrow\heva{&x-1=1\\&y=0\\&z-1=1}\Leftrightarrow\heva{&x=2\\&y=0\\&z=2}\Rightarrow\overrightarrow{CC'}=(2;5;-7)$.
\end{center}
Mặt khác $\overrightarrow{C'D'}=\overrightarrow{BA}=(-1;-1;-1)$.\\
Suy ra $\left[\overrightarrow{CC'},\overrightarrow{C'D'}\right]=(-12;9;3)$ là một vectơ thoả mãn yêu cầu bài toán.\\
Ta có $\left[\overrightarrow{CC'},\overrightarrow{C'D'}\right]$ và $\overrightarrow{v}$ cùng phương nên có số thực $k$ để $\left[\overrightarrow{CC'};\overrightarrow{C'D'}\right]=k\cdot\overrightarrow{v}$.\\
Suy ra $\heva{&-12=k\cdot a\\&9=k\cdot b\\&3=k\cdot6}\Leftrightarrow\heva{&a = -24\\&b = 18\\&k=\dfrac{1}{2}.}$\\
Vậy $a+b=-6$.
}
\end{ex}

\begin{ex}%[2-H2B4-SO-10-2425 (Nguồn: Bài 4 - Đề 1 - Ôn Tập Chương II)]%[VN-MT-7, Trần Bảo Hiên]%[2H2V2-6]
Trong một căn phòng dạng hình hộp chữ nhật với chiều dài $8$ m, rộng $6$ m và cao $4$ m có cây quạt treo tường. Cây quạt $A$ treo chính gữa bức tường $8$ m và cách trần $1$ m, cây quạt $B$ treo chính giữa bức tường $6$ m và cách trần $1{,}5$ m. Chọn hệ trục tọa độ $Oxyz$ như hình vẽ bên dưới (đơn vị: mét). Hãy tính độ dài vectơ $\overrightarrow{AB}$ (làm tròn đến hàng đơn vị).
\begin{center}
\definecolor{amber}{rgb}{1.0, 0.75, 0.0}%mau non
\definecolor{antiquebrass}{rgb}{0.8, 0.58, 0.46}%mau da
\definecolor{antiquewhite}{rgb}{0.98, 0.92, 0.84}%mau ao
\definecolor{cadmiumgreen}{rgb}{0.0, 0.42, 0.24}%mau quan
\definecolor{cadetblue}{rgb}{0.37, 0.62, 0.63}%mau but
\definecolor{brown(traditional)}{rgb}{0.59, 0.29, 0.0}%mau giay
\definecolor{brilliantlavender}{rgb}{0.96, 0.73, 1.0}%màu sơn tím
\definecolor{brightube}{rgb}{0.82, 0.62, 0.91}%màu sơn tím đậm
%---------------màu quạt
\definecolor{burntorange}{rgb}{0.8, 0.33, 0.0}
\definecolor{arsenic}{rgb}{0.23, 0.27, 0.29}
\definecolor{battleshipgrey}{rgb}{0.52, 0.52, 0.51}
\begin{tikzpicture}[line join=round, line cap=round,scale=0.7,transform shape,font=\large]
\clip (1,-1) rectangle (16,13);
%\draw[gray!50] (-3,-3) grid (3,4);
\definecolor{burntsienna}{rgb}{0.91, 0.45, 0.32}
\tikzset{mai/.pic={
\def\mainha{
(.5,2)
foreach \n in {1,2,...,22} { -- ++ (0,0) -- ++ (0,1) -- ++ (1,0) -- ++ (0,-1) } -- cycle
(.5,1)
foreach \n in {1,2,...,22} { -- ++ (0,0) -- ++ (0,1) -- ++ (1,0) -- ++ (0,-1) } -- cycle
(.5,0)
foreach \n in {1,2,...,22} { -- ++ (0,0) -- ++ (0,1) -- ++ (1,0) -- ++ (0,-1) } -- cycle
(.5,-1)
foreach \n in {1,2,...,22} { -- ++ (0,0) -- ++ (0,1) -- ++ (1,0) -- ++ (0,-1) } -- cycle
(.5,-2)
foreach \n in {1,2,...,22} { -- ++ (0,0) -- ++ (0,1) -- ++ (1,0) -- ++ (0,-1) } -- cycle
(.5,-3)
foreach \n in {1,2,...,22} { -- ++ (0,0) -- ++ (0,1) -- ++ (1,0) -- ++ (0,-1) } -- cycle
(.5,-4)
foreach \n in {1,2,...,22} { -- ++ (0,0) -- ++ (0,1) -- ++ (1,0) -- ++ (0,-1) } -- cycle;
}
\clip (-4,-3)--(17.5,-3)--(17.5,3)--(.5,3)--cycle;
\draw[white,fill=burntsienna!70] \mainha;
\draw (17.5,-3)--(17.5,3)--(17.5,9);}}
\fill[brilliantlavender] (18,-3)--(18,13)--(12,13)--(12,-3)--cycle;
\fill[brightube] (12,13)--(12,3)--(-18,3)--(-18,13)--cycle;
\path(-2.5,0)pic[scale=1,xslant=-1]{mai};
\path
(12,3) coordinate (O)
(3,3) coordinate (x)
(12,13) coordinate (z)
(15,0) coordinate (y);
\foreach\p/\g in {y/160,x/45, z/-45,O/-90}
{
\node at (\p) [shift=(\g:4mm)] {$\p$};
}
\draw[line width=.5mm,->] (O)--(z) ;
\draw[line width=.5mm,->] (O)--(x);
\draw[line width=.5mm,->] (O)--(y);
\node at ($(O)+(-6,4.5)$) {$8$ m};
\draw[red,line width=.5mm,<->] ($(O)+(0,4)$)--($(x)+(-2,4)$);
%===========================A FAN
\tikzset{fan/.pic={
%chân quạt
\draw[fill=battleshipgrey](-.2,.8)--(.2,.8)--(.4,-2.2)
..controls +(-120:.3) and +(-60:.3) ..(-.4,-2.2)--cycle;
%---Nút bấm
\foreach \i in{-1.95,-1.7}{%-1.45
\draw[fill=arsenic](-.15,\i) rectangle (.15,\i+.15);
}
%-----------------------------------------------------
\draw[black](0,.8) circle (2.25cm);
\draw[black](0,.8) circle (2.15cm);
\draw[black](0,.8) circle (1.42cm);
\draw[black](0,.8) circle (1.48cm);
\draw[fill=black](0,.8) circle (6mm);
\draw[fill=arsenic](0,.8) circle (5mm);
\def\N{
(0,.8)
..controls +(145:1.3) and +(170:1) ..(0,2.8)
..controls +(-10:1.4) and +(-20:1) ..(.6,1.76)
..controls +(160:.4) and +(100:.4) ..(.2,.9)--cycle;
}
\foreach \i/\j/\k in {0/0/0,120/.7/-1.2,240/-.7/-1.2}
{
\draw[black,rotate=\i,shift={(\j,\k)}]\N;
\fill[burntorange,rotate=\i,shift={(\j,\k)}] \N;
}
%lồng quạt
\def\r{2.15}
\foreach \i in {0,15,25,35,...,365}
\draw[double] ($(\i:\r)+(0,.8)$)--(0,.8);
\draw[fill=arsenic](0,.8) circle (3.5mm);
}}
\path
(6,11)pic[scale=.6]{fan}
(14,9)pic[scale=.55,yslant=-.3]{fan};
\end{tikzpicture}
\end{center}
\shortans{5}
\loigiai{
Từ hình vẽ $A\in(Oxz)$ nên $A(x;0;z)$ và $B\in(Oyz)$ nên $B(0;y;z)$.\\
Cây quạt $A$ treo chính giữa bức tường $8$ m và cách trần $1$ m nên $A(4;0;3)$.\\
Cây quạt $B$ treo chính giữa bức tường $6$ m và cách trần $1{,}5$ m nên $B\left(0;3;\dfrac{5}{2}\right)$.\\
Khi đó $\overrightarrow{AB}=\left(-4;3;-\dfrac{1}{2}\right)\Rightarrow\left\vert\overrightarrow{AB}\right\vert=\sqrt{(-4)^2+3^2+\left(-\dfrac{1}{2}\right)^2}\approx 5$ m.
}
\end{ex}

\begin{ex}%[2-H2B4-SO-10-2425 (Nguồn: Bài 4 - Đề 1 - Ôn Tập Chương II)]%[VN-MT-7, Trần Bảo Hiên]%[2H2V2-4]
Một chi tiết trong bộ trang sức được gắn hệ trục tọa độ $Oxyz$ như hình vẽ. Các hình chóp $S.ABCD$ và $I.ABCD$ là các hình chóp đều cạnh $1$ cm. Tính số đo góc nhị diện $[S,CD,I]$ theo đơn vị độ, làm tròn đến hàng đơn vị.
\shortans{109}
\loigiai{
\begin{center}
\begin{tikzpicture}[scale=0.8,font=\footnotesize,line join=round,line cap=round,>=stealth]
\def \a{6}
\def \h{5}
\path (0:0) coordinate (A)
++(0:\a) coordinate (D)
++(-138:0.45*\a) coordinate (C)
($(A)+(C)-(D)$) coordinate (B)
($(A)!0.5!(C)$) coordinate (O)
($(O)+(90:\h)$) coordinate (S)
($(S)!-0.2!(O)$) coordinate (H)
($(C)!-0.4!(A)$) coordinate (K)
($(D)!-0.2!(B)$) coordinate (L)
($(D)!0.5!(C)$) coordinate (M)
($(O)!1!180:(S)$) coordinate (I);
\draw [dashed] (B)--(A)--(D) (A)--(S) (A)--(C) (B)--(D) (S)--(O)--(I)--(A);
\draw (B)--(C)--(D) (B)--(S) (C)--(S) (D)--(S) (C)--(I)--(B) (I)--(D);
\draw[->] (S)--(H)node[right]{$z$};
\draw[->] (C)--(K)node[above]{$x$};
\draw[->] (D)--(L)node[above]{$y$};
\foreach \x/\g in {A/135,B/180,C/80,D/70,S/40,O/50,I/-90,M/150}
\fill (\x) circle (1pt)
($(\g:3mm)+(\x)$) node {$\x$};
\end{tikzpicture}
\end{center}
Ta có $ABCD$ là hình vuông cạnh $1$ cm nên $OC=OD=\dfrac{\sqrt{2}}{2}$.\\
Xét $\triangle SOC$ vuông tại $O$, ta có $OS=\sqrt{SC^2-OC^2}=\sqrt{1^2-\left(\dfrac{\sqrt{2}}{2}\right)^2}=\dfrac{\sqrt{2}}{2}$.\\
Xét $\triangle IOC$ vuông tại $O$, ta có $OI=\sqrt{IC^2-OC^2}=\sqrt{1^2-\left(\dfrac{\sqrt{2}}{2}\right)^2}=\dfrac{\sqrt{2}}{2}$.\\
Vậy $C\left(\dfrac{\sqrt{2}}{2};0;0\right)$, $D\left(0;\dfrac{\sqrt{2}}{2};0\right)$, $S\left(0;0;\dfrac{\sqrt{2}}{2}\right)$, $I\left(0;0;-\dfrac{\sqrt{2}}{2}\right)$.\\
Gọi $M$ là trung điểm của $CD$ thì $M\left(\dfrac{\sqrt{2}}{4};\dfrac{\sqrt{2}}{4};0\right)$.\\
Ta có $\heva{&CD\perp MI\\&CD\perp MS}\Rightarrow[S,CD,I]=\widehat{SMI}$.\\
Ta có $\overrightarrow{MS}=\left(-\dfrac{\sqrt{2}}{4};-\dfrac{\sqrt{2}}{4};\dfrac{\sqrt{2}}{2}\right)$, $\overrightarrow{MI}=\left(-\dfrac{\sqrt{2}}{4};-\dfrac{\sqrt{2}}{4};-\dfrac{\sqrt{2}}{2}\right)$.\\
$\Rightarrow\overrightarrow{MS}\cdot\overrightarrow{MI}=-\dfrac{1}{4}$, $\left\vert\overrightarrow{MS}\right\vert=\dfrac{\sqrt{3}}{2}$, $\left\vert\overrightarrow{MI}\right\vert=\dfrac{\sqrt{3}}{2}$.
\begin{align*}
\cos\widehat{SMI}=\cos\left(\overrightarrow{MS},\overrightarrow{MI}\right)=\dfrac{\overrightarrow{MS}\cdot\overrightarrow{MI}}{\left\vert\overrightarrow{MS}\right\vert\cdot\left\vert\overrightarrow{MI}\right\vert}=\dfrac{-\dfrac{1}{4}}{\dfrac{\sqrt{3}}{2}\cdot\dfrac{\sqrt{3}}{2}}=-\dfrac{1}{3}\Rightarrow\widehat{SMI}\approx109^\circ.
\end{align*}
}
\end{ex}

\begin{ex}%[2-H2B4-SO-10-2425 (Nguồn: Bài 4 - Đề 1 - Ôn Tập Chương II)]%[VN-MT-7, Trần Bảo Hiên]%[2H2C1-4]
\immini[thm]{Một chiếc ô tô được đặt trên mặt đáy dưới của một khung sắt có dạng hình hộp chữ nhật với đáy trên là hình chữ nhật $ABCD$, mặt phẳng $(ABCD)$ song song với mặt phẳng nằm ngang. Khung sắt đó được buộc vào móc $E$ của chiếc cần cẩu sao cho các đoạn dây cáp $EA$, $EB$, $EC$ và $ED$ có độ dài bằng nhau và cùng tạo với mặt phẳng $(ABCD)$ một góc bằng $60^\circ$ (hình minh họa). Chiếc cần cẩu đang kéo khung sắt lên theo phương thẳng đứng. Biết rằng các lực căng $\overrightarrow{F_1}$, $\overrightarrow{F_2}$, $\overrightarrow{F_3}$, $\overrightarrow{F_4}$ đều có cường độ là $4{,}7$ kN và trọng lượng của khung sắt là $3$ kN. Tính trọng lượng của chiếc xe ô tô (làm tròn đến hàng phần chục)?
}
{
\begin{tikzpicture}[>=latex,line join=round, line cap=round,scale=0.8,transform shape]
\definecolor{bostonuniversityred}{rgb}{0.8, 0.0, 0.0}
\definecolor{charcoal}{rgb}{0.21, 0.27, 0.31}
\definecolor{bananayellow}{rgb}{1.0, 0.88, 0.21}
\definecolor{anti-flashwhite}{rgb}{0.95, 0.95, 0.96}
% \clip (-6,-3) rectangle (6,3);
\tikzset{%
xeoto/.pic={%
%--------------------------
\tikzset{xe/.pic={
\def\N{
(-2.7,.56)--(-2.5,.56)
..controls +(50:1.5) and +(165:1.5) .. (2.1,1.88)--(2.05,2)
..controls +(-10:.1) and +(130:.1) .. (3.25,1.75)--(3.15,1.65)
..controls +(-4:.2) and +(130:.15) .. (4.05,.7)--(4.25,.75)
..controls +(-40:.2) and +(130:.15) .. (4.55,.35)--(4.35,.26)
..controls +(-40:.2) and +(130:.15) .. (4.8,-.45)--(4.92,-.4)
..controls +(-40:.25) and +(73:.17) .. (4.8,-1.8)--(-4.4,-1.8)
..controls +(175:.7) and +(-175:3.2) ..cycle;
}
\fill[bostonuniversityred] \N;
\draw \N;
\def\K{
(-2.2,.56)--(3.3,.7)
..controls +(100:1.18) and +(43:3) .. cycle;
}
\fill[bottom color=charcoal,top color=charcoal!20!white, middle color=charcoal!80!white] \K;
\draw \K;
\def\K1{
(-2.2,.56)..controls +(43:.2) and +(43:.2) .. (-1.58,1.05)--(-1.53,.57)--cycle;
}
\draw \K1;
\fill[charcoal] \K1;
\def\K2{
(1.2,1.85)..controls +(-10:.1) and +(160:.1) .. (1.58,1.8)--(1.8,.65)--(1.25,.65)--cycle;
}
\draw \K2;
\fill[charcoal] \K2;
\def\Kt{
(-2.5,.56)..controls +(50:1.5) and +(165:1.5) .. (2.1,1.88)--(2.05,2)
..controls +(170:2.2) and +(45:1.5) .. (-2.7,.56)--cycle;
}
\fill[charcoal!50] \Kt;
\draw \Kt;
\def\Ks{
(3.25,1.75)--(3.15,1.65)
..controls +(-4:.2) and +(130:.15) .. (4.05,.7)--(4.22,.75)
..controls +(120:.3) and +(-35:.3) .. cycle;
}
\fill[charcoal!50] \Ks;
\draw \Ks;
%Đèn sau
\def\D{
(4.55,.35)--(4.35,.26)
..controls +(-40:.2) and +(130:.15) .. (4.8,-.45)--(4.92,-.4)
..controls +(110:.2) and +(-40:.15) ..cycle;
}
\fill[bananayellow] \D;
\draw \D;
\def\M{
(2.2,-1.3)--(-1.8,-1.4)--(-1.78,-1.7)
..controls +(-5:.3) and +(-90:.6) ..cycle;
}
\draw \M;
\fill[charcoal!90] \M;
\draw (-1.6,.55)..controls +(-170:.5) and +(95:.4) .. (-1.78,-1.7)
(1.6,.65)..controls +(-30:.5) and +(35:.3) .. (1.7,-1.3);
%gương
\def\G{ (-1.5,.45)--(-1.4,.6)..controls +(85:1) and +(20:.6) .. (-1.25,.5)--(-1.4,.33);
}
\draw \G;
\fill[bostonuniversityred] \G;
%Đèn trước
\def\Dt{
(-4.85,-.7)..controls +(75:1) and +(65:.8) .. (-4.5,-.7)
..controls +(-115:.6) and +(-105:.4) .. cycle;
}
\fill[bananayellow] \Dt;
\draw \Dt;
\def\Dt2{
(-4.85,-.7)
..controls +(75:.6) and +(65:.4) .. (-4.7,-.7)
..controls +(-115:.3) and +(-105:.2) .. cycle;
}
\fill[anti-flashwhite] \Dt2;
\draw \Dt2;
\draw[fill=anti-flashwhite] (-4.86,-1.45)--(-4.82,-1.5)--(-4.55,-1.3)
..controls +(90:.3) and +(45:.2) .. cycle;
}}
\tikzset{banh_xe/.pic={
\draw[fill=charcoal] (-3.25,-1.65) circle (1) ;
\draw[fill=anti-flashwhite] (-3.25,-1.65) circle (.7) ;
\draw[fill=charcoal] (-3.25,-1.65) circle (.4) ;
}}
%----------------
\path
(0,0)pic[scale=1]{xe}(0,0)pic[scale=1]{banh_xe}(6.9,0)pic[scale=1]{banh_xe};
}}
\def\bc{4.25} % cạnh BC
\def\ba{1.5} % cạnh BA
\def\h{4} % đường cao
\def\gocnghieng{90} % góc nghiêng
\def\gocB{160} % góc B của đáy
\coordinate (B1) at (0,0);
\coordinate (A1) at (\gocB:\ba);
\coordinate (C1) at (\bc,0.25);
\coordinate (D1) at ($(C1)-(B1)+(A1)$);
\coordinate[label=above left:$A$] (A) at ($(A1)+(\gocnghieng:\h)$);
\coordinate[label=below left:$B$] (B) at ($(B1)-(A1)+(A)$);
\coordinate[label=right:$C$] (C) at ($(C1)-(A1)+(A)$);
\coordinate[label=above right:$D$] (D) at ($(D1)-(A1)+(A)$);
\coordinate (E) at ($(A)!0.5!(C)+(\gocnghieng:\h)$);
%------------
\draw[->,blue,very thick] (E)--($(E)!0.4!(A)$) node[above left]{$\overrightarrow{F_1}$};
\draw[->,blue,very thick] (E)--($(E)!0.4!(B)$) node[right]{$\overrightarrow{F_2}$};
\draw[->,blue,very thick] (E)--($(E)!0.4!(C)$) node[above right]{$\overrightarrow{F_3}$}; \draw[->,blue,very thick] (E)--($(E)!0.4!(D)$) node[left=2pt]{$\overrightarrow{F_4}$};
%------------
\path (E) node[left=1mm]{$E$};
\draw[blue,very thick] (A)--(B)--(C)--(D)--cycle
(A1)--(A) (D1)--(D) (C1)--(C)
(A)--(E)--(B) (C)--(E)--(D);
\draw[fill=teal] (A1)--(B1)--(C1)--(D1)--cycle;
\draw[fill=teal!30] (A1)--(B1)--(C1)--++(0,-0.3)--([yshift=-0.3cm]B1)--([yshift=-0.3cm]A1)--cycle;
\foreach \diem in {A1,B1,C1,D1,A,B,C,D,E} \fill (\diem)circle(1.5pt);
%phần móc và dây
\def\r{0.3}\def\rr{0.25}
\coordinate (tam) at ([yshift=6mm]E);
\draw[brown,fill=brown,line width=1pt] (tam) circle (\r cm);
\fill (tam) circle (2pt);
\draw[brown,line width=1pt] (tam)++(\r,0)--++(0,0.7)(tam)++(-\r,0)--++(0,0.7);
\draw[line width=1.5pt] (tam)--++(0,-1.35*\r) arc(90:370:1mm);
%%%%%%%%%%%%%%%%%%%
\pic[scale=0.45,rotate=4] at (1.6,1.3) [pic type = xeoto];
%--------
\draw[blue,very thick] (B)--(B1);
\end{tikzpicture}
}
\shortans{13{,}3}
\loigiai{
\begin{center}
\begin{tikzpicture}[scale=1,font=\footnotesize,line join=round,line cap=round,>=stealth]
\def \a{4.5}
\def \h{4}
\path (0:0) coordinate (A_1)
++(0:\a) coordinate (D_1)
++(-138:0.45*\a) coordinate (C_1)
($(A_1)+(C_1)-(D_1)$) coordinate (B_1)
($(A_1)!0.5!(C_1)$) coordinate (O)
($(O)+(90:\h)$) coordinate (E)
($(A_1)!0.5!(E)$) coordinate (F_1)
($(B_1)!0.5!(E)$) coordinate (F_2)
($(C_1)!0.5!(E)$) coordinate (F_3)
($(D_1)!0.5!(E)$) coordinate (F_4);
\draw [dashed] (B_1)--(A_1)--(D_1) (A_1)--(C_1) (B_1)--(D_1) (E)--(O);
\draw (B_1)--(C_1)--(D_1);
\draw[dashed,->] (E)--(A_1);
\draw[->] (E)--(B_1);
\draw[->] (E)--(C_1);
\draw[->] (E)--(D_1);
\foreach \x/\g in {A_1/-90,B_1/-135,C_1/-45,D_1/45,E/90,O/-90}
\fill (\x) circle (1pt)($(\g:3mm)+(\x)$) node {$\x$};
\foreach \y/\m in {F_1/20,F_2/135,F_3/45,F_4/45}
\draw ($(\m:3mm)+(\y)$) node {$\overrightarrow{\y}$};
\end{tikzpicture}
\end{center}
Gọi $A_1$, $B_1$, $C_1$, $D_1$ lần lượt là các điểm sao cho $\overrightarrow{EA_1}=\overrightarrow{F_1}$, $\overrightarrow{EA_2}=\overrightarrow{F_2}$, $\overrightarrow{EA_3}=\overrightarrow{F_3}$, $\overrightarrow{EA_4}=\overrightarrow{F_4}$.\\
Vì $EA=EB=EC=ED$ và cùng tạo với mặt phẳng $(ABCD)$ một góc bằng $60^\circ$, các lực căng $\overrightarrow{F_1}$, $\overrightarrow{F_2}$, $\overrightarrow{F_3}$, $\overrightarrow{F_4}$ đều có cường độ là $4{,}7$ kN nên $EA_1=EB_1=EC_1=ED_1$ và cùng tạo với mặt phẳng $\left(A_1B_1C_1D_1\right)$ một góc bằng $60^\circ$.\\
Vì $ABCD$ là hình chữ nhật nên $A_1B_1C_1D_1$ cũng là hình chữ nhật.\\
Gọi $O$ là tâm của hình chữ nhật $A_1B_1C_1D_1$.\\
Ta suy ra $EO\perp \left(A_1B_1C_1D_1\right)$.\\
Do đó, $\left(EA_1,\left(A_1B_1C_1D_1\right)\right)=\widehat{EA_1O}=60^\circ$.\\
Ta có $\left\vert\overrightarrow{F_1}\right\vert=\left\vert\overrightarrow{F_2}\right\vert=\left\vert\overrightarrow{F_3}\right\vert=\left\vert\overrightarrow{F_4}\right\vert=4{,}7$ kN nên $EA_1=EB_1=EC_1=ED_1=4{,}7$.\\
Tam giác $EA_1O$ vuông tại $O$ nên $EO=EA_1\cdot\sin\widehat{EA_1O}=2{,}35\sqrt{3}$.\\
Ta có
\begin{align*}
\overrightarrow{F_1}+\overrightarrow{F_2}+\overrightarrow{F_3}+\overrightarrow{F_4}&=\overrightarrow{EA_1}+\overrightarrow{EA_2}+\overrightarrow{EA_3}+\overrightarrow{EA_4}\\
&=\overrightarrow{EO}+\overrightarrow{OA_1}+\overrightarrow{EO}+\overrightarrow{OA_2}+\overrightarrow{EO}+\overrightarrow{OA_3}+\overrightarrow{EO}+\overrightarrow{OA_4}\\
&=4\overrightarrow{EO}+\left(\overrightarrow{OA_1}+\overrightarrow{OC_1}\right)+\left(\overrightarrow{OB_1}+\overrightarrow{OD_1}\right)\\
&=4\overrightarrow{EO}.
\end{align*}
Gọi $\overrightarrow{P}$ là trọng lực của khung sắt có chứa chiếc ô tô. Khi đó ta có 
\begin{align*}
 \overrightarrow{P} = \overrightarrow{F}_1 + \overrightarrow{F}_2 + \overrightarrow{F}_3 + \overrightarrow{F}_4 = 4\overrightarrow{EO}.
\end{align*}
Suy ra trọng lượng của khung sắt có chứa ô tô là $\left|\overrightarrow{P}\right| = 4\left\vert\overrightarrow{EO}\right\vert=4\cdot2{,}35\sqrt{3}=9{,}4\sqrt{3}$ kN.\\
Vì trọng lượng của khung sắt là $3$ kN nên trọng lượng của chiếc xe ô tô là $9{,}4\sqrt{3}-3\approx13{,}3$ kN.
}
\end{ex}
\Closesolutionfile{ans}
 
% \begin{indapan}
% 	{ans/ans\currfilebase}
% \end{indapan}


% \begin{name}
	{\tenchude}
	{ĐỀ ÔN TẬP CHƯƠNG II}
	{LỚP TOÁN THẦY PHÁT}
	{\thoigian}
\end{name}

\TN
\Opensolutionfile{ans}[ans/ans\currfilebase-Phan-I]
\begin{ex}%[2-H2B4-SO-11-2425]%[VN-MT-7, Đào Trung Kiên]%[2H2N1-2]
Cho tứ diện $ABCD$. Có bao nhiêu vectơ có điểm đầu là $A$ và điểm cuối là một trong các đỉnh còn lại của tứ diện?
\choice{$1$}
{$2$}
{\True $3$}
{$4$}
\loigiai{
\begin{center}
\begin{tikzpicture}[font=\footnotesize, line join=round, line cap=round, >=stealth, scale=0.9]
 \def\a{4}
 \path (0:0) coordinate (B)
 ++(0:\a) coordinate (D)
 ++(-120:\a/2) coordinate (C)
 ($(B)+(60:\a)$) coordinate (A);
 \draw[dashed] (B)--(D);
 \draw[->] (A)--(C);
 \draw[->] (A)--(D);
 \draw[->] (A)--(B);
 \draw (B)--(C)--(D);
 \foreach \x/\g in {A/90,B/180,C/-45,D/0}
 \fill (\x) circle (1pt)
 ($(\g:3mm)+(\x)$) node {$\x$};
\end{tikzpicture}
\end{center}
Ba vectơ $\overrightarrow{AB}$, $\overrightarrow{AC}$, $\overrightarrow{AD}$.
}
\end{ex}

\begin{ex}%[2-H2B4-SO-11-2425]%[VN-MT-7, Đào Trung Kiên]%[2H2N1-1]
Cho hình lập phương $ABCD.A'B'C'D'$. Hai vectơ nào dưới đây có giá cùng nằm trong mặt phẳng $(ABCD)$?
 \choice{$\overrightarrow{DD'}$, $\overrightarrow{AC}$}
 {$\overrightarrow{AD'}$, $\overrightarrow{AD}$}
 {$\overrightarrow{AD'}$, $\overrightarrow{AC}$}
 {\True $\overrightarrow{AC}$, $\overrightarrow{AD}$}
 \loigiai{
\begin{center}
\begin{tikzpicture}[font=\footnotesize, line join=round, line cap=round, >=stealth, scale=1]
 \def\a{3.5}
 \path (0:0) coordinate (A)
 ++(0:\a) coordinate (D)
 ++(-130:\a/2) coordinate (C)
 ($(A)+(C)-(D)$) coordinate (B)
 ($(A)+(90:\a)$) coordinate (A')
 ($(B)+(90:\a)$) coordinate (B')
 ($(C)+(90:\a)$) coordinate (C')
 ($(D)+(90:\a)$) coordinate (D');
 \draw[dashed] (B)--(A) (A)--(A');
 \draw[dashed,->] (A)--(C);
 \draw[dashed,->] (A)--(D);
 \draw (C)--(C') (D)--(D') (B)--(B') (B)--(C)--(D) (A')--(B')--(C')--(D')--cycle;
 \foreach \x/\g in {A/180,B/180,C/0,D/0,A'/180,B'/180,C'/0,D'/0}
 \fill (\x) circle (1pt)
 ($(\g:4mm)+(\x)$) node {$\x$}; 
\end{tikzpicture}
\end{center}
Hai vectơ $\overrightarrow{AC}$, $\overrightarrow{AD}$ có giá cùng nằm trong mặt phẳng $(ABCD)$.
 }
\end{ex}

\begin{ex}%[2-H2B4-SO-11-2425]%[VN-MT-7, Đào Trung Kiên]%[2H2N1-1]
Cho hình lập phương $ABCD.A'B'C'D'$ có cạnh là $a$. Hai vectơ nào dưới đây có cùng độ dài?
 \choice{$\overrightarrow{DD'}$, $\overrightarrow{AC}$}
 {$\overrightarrow{AD'}$, $\overrightarrow{AD}$}
 {\True $\overrightarrow{AD'}$, $\overrightarrow{AC}$}
 {$\overrightarrow{AC}$, $\overrightarrow{AD}$}
 \loigiai{
\begin{center}
 \begin{tikzpicture}[font=\footnotesize, line join=round, line cap=round, >=stealth, scale=1]
 \def\a{3.5}
 \path (0:0) coordinate (A)
 ++(0:\a) coordinate (B)
 ++(-130:\a/2) coordinate (C)
 ($(A)+(C)-(B)$) coordinate (D)
 ($(A)+(-90:\a)$) coordinate (A')
 ($(B)+(-90:\a)$) coordinate (B')
 ($(C)+(-90:\a)$) coordinate (C')
 ($(D)+(-90:\a)$) coordinate (D');
 \draw[dashed] (B)--(A) (A)--(A')--(B') (A')--(D');
 \draw[->] (A)--(C);
 \draw[->] (A)--(D);
 \draw[dashed,->] (A)--(D');
 \draw (C)--(C')--(B') (D)--(D')--(C') (A)--(B)--(B') (B)--(C)--(D) ;
 \foreach \x/\g in {A/180,D/180,C/0,B/0,A'/180,D'/180,C'/0,B'/0}
 \fill (\x) circle (1pt)
 ($(\g:4mm)+(\x)$) node {$\x$}; 
 \end{tikzpicture}
\end{center}
Vì $\left|\overrightarrow{AD'}\right|=\left|\overrightarrow{AC}\right|=a\sqrt{2}$ nên hai vectơ $\overrightarrow{AD'}$ và $\overrightarrow{AC}$ có cùng độ dài.
 }
\end{ex}

\begin{ex}%[2-H2B4-SO-11-2425]%[VN-MT-7, Đào Trung Kiên]%[2H2N1-1]
Cho hình lập phương $ABCD.A'B'C'D'$ có cạnh là $a$. Vectơ nào bằng vectơ $\overrightarrow{D'C'}$?
 \choice{$\overrightarrow{DD'}$}
 {$\overrightarrow{AD}$}
 {\True $\overrightarrow{AB}$}
 {$\overrightarrow{CD}$}
 \loigiai{
\begin{center}
 \begin{tikzpicture}[font=\footnotesize, line join=round, line cap=round, >=stealth, scale=1]
 \def\a{3.5}
 \path (0:0) coordinate (A)
 ++(0:\a) coordinate (B)
 ++(-130:\a/2) coordinate (C)
 ($(A)+(C)-(B)$) coordinate (D)
 ($(A)+(-90:\a)$) coordinate (A')
 ($(B)+(-90:\a)$) coordinate (B')
 ($(C)+(-90:\a)$) coordinate (C')
 ($(D)+(-90:\a)$) coordinate (D');
 \draw[dashed] (B)--(A) (A)--(A')--(B') (A')--(D');
 \draw[->] (A)--(B);
 \draw[->] (D')--(C');
 \draw (C)--(C')--(B') (A)--(D)--(D') (A)--(B)--(B') (B)--(C)--(D) ;
 \foreach \x/\g in {A/180,D/180,C/0,B/0,A'/180,D'/180,C'/0,B'/0}
 \fill (\x) circle (1pt)
 ($(\g:4mm)+(\x)$) node {$\x$}; 
 \end{tikzpicture}
\end{center}
Vì hai vectơ $\overrightarrow{AB}$ và $\overrightarrow{D'C'}$ có cùng hướng và cùng độ dài nên $\overrightarrow{AB}=\overrightarrow{D'C'}$.
 }
\end{ex}

\begin{ex}%[2-H2B4-SO-11-2425]%[VN-MT-7, Đào Trung Kiên]%[2H2N2-1]
Trong không gian với hệ tọa độ $Oxyz$, cho hai điểm $A(1; -1; 2)$ và $B(2; 1; -4)$. Vectơ $\overrightarrow{AB}$ có tọa độ là
 \choice{\True $(1;2;-6)$}
 {$(1; 0; -6)$}
 {$(-1; -2; 6)$}
 {$(3; 0; -2)$}
 \loigiai{
Ta có $\overrightarrow{AB}=(2-1; 1-(-1); -4-2)\Rightarrow \overrightarrow{AB}=(1; 2; -6)$.
 }
\end{ex}

\begin{ex}%[2-H2B4-SO-11-2425]%[VN-MT-7, Đào Trung Kiên]%[2H2N2-3]
Trong không gian với hệ tọa độ $Oxyz$, cho biểu diễn của vectơ $\overrightarrow{a}$ qua các vectơ đơn vị là $\overrightarrow{a}=2\overrightarrow{i}+\overrightarrow{k}-3\overrightarrow{j}$. Tọa độ của vectơ $\overrightarrow{a}$ là
 \choice{\True $(2; -3; 1)$}
 {$(1; -3; 2)$}
 {$(2; 1; -3)$}
 {$(1; 2; -3)$}
 \loigiai{ Ta có $\overrightarrow{a}=2\cdot\overrightarrow{i}-3\cdot\overrightarrow{j}+1\cdot \overrightarrow{k}$ nên $\overrightarrow{a}=(2; -3; 1)$.
 }
\end{ex}

\begin{ex}%[2-H2B4-SO-11-2425]%[VN-MT-7, Đào Trung Kiên]%[2H2H2-2]
Trong không gian với hệ tọa độ $Oxyz$, cho hình bình hành $ABCD$ với các đỉnh có tọa độ là $A(3; 1; 2)$, $B(1; 0; 1)$, $C(2; 3; 0)$. Tọa độ đỉnh $D$ là
 \choice{$D(1; 1; 0)$}
 {$D(0; 2; -1)$}
 {\True $D(4; 4; 1)$}
 {$D(1; 3; -1)$}
 \loigiai{
Ta có $ABCD$ là hình bình hành nên $\overrightarrow{AD}=\overrightarrow{BC}\Leftrightarrow\heva{&x_D-3=1\\&y_D-1=3\\&z_D-2=-1}\Leftrightarrow\heva{&x_D=4\\&y_D=4\\&z_D=1}\Rightarrow D(4; 4; 1)$.
 }
\end{ex}

\begin{ex}%[2-H2B4-SO-11-2425]%[VN-MT-7, Đào Trung Kiên]%[2H2H2-2]
Trong không gian với hệ tọa độ $Oxyz$, cho vectơ $\overrightarrow{a}=(-3; 2; 1)$ và điểm $A(4; 6; -3)$. Tọa độ điểm $B$ thỏa mãn $\overrightarrow{AB}=\overrightarrow{a}$ là
 \choice{$(-1; -8; 2)$}
 {$(7; 4; -4)$}
 {\True $(1; 8; -2)$}
 {$(-7; -4; 4)$}
 \loigiai{
Đặt $B(x; y; z)$. Ta có $\overrightarrow{AB}=(x-4; y-6; z+3)$.\\
Khi đó $\overrightarrow{AB}=\overrightarrow{a}\Leftrightarrow \heva{&x-4=-3\\&y-6=2\\&z+3=1}\Leftrightarrow \heva{&x=1\\&y=8\\&z=-2.}$\\
Vậy $B(1; 8; -2)$.
 }
\end{ex}

\begin{ex}%[2-H2B4-SO-11-2425]%[VN-MT-7, Đào Trung Kiên]%[2H2H2-3]
Trong không gian với hệ tọa độ $Oxyz$, cho ba vectơ $\overrightarrow{a}=(1; 2; 3)$, $\overrightarrow{b}=(-2; 0; 1)$,  $\overrightarrow{c}=(-1; 0; 1)$. Tìm tọa độ của vectơ $\overrightarrow{n}=\overrightarrow{a}+\overrightarrow{b}+2\overrightarrow{c}-3\overrightarrow{i}$.
 \choice{$\overrightarrow{n}=(6; 2; 6)$}
 {$\overrightarrow{n}=(6; 2; -6)$}
 {$\overrightarrow{n}=(0; 2; 6)$}
 {\True $\overrightarrow{n}=(-6; 2; 6)$}
 \loigiai{Vì $2\overrightarrow{c}=(-2; 0; 2)$ và $-3\overrightarrow{i}=(-3; 0; 0)$ nên $\overrightarrow{n}=\overrightarrow{a}+\overrightarrow{b}+2\overrightarrow{c}-3\overrightarrow{i}$ có tọa độ $(-6; 2; 6)$.
 }
\end{ex}

\begin{ex}%[2-H2B4-SO-11-2425]%[VN-MT-7, Đào Trung Kiên]%[2H2H2-2]
Trong không gian với hệ tọa độ $Oxyz$, cho ba điểm $A(3; 5; -1)$, $B(7; x; 1)$ và $C(9; 2; y)$. Để $A$, $B$, $C$ thẳng hàng thì $x+y$ bằng
 \choice{\True $5$}
 {$6$}
 {$4$}
 {$7$}
 \loigiai{
Ta có $\overrightarrow{AB}=(4; x-5; 2)$, $\overrightarrow{AC}=(6; -3; y+1)$.\\
Vì $\overrightarrow{AB}\neq \overrightarrow{0}$ nên $A$, $B$, $C$ thẳng hàng khi $\overrightarrow{AB}$, $\overrightarrow{AC}$ cùng phương\\
\[\Leftrightarrow \overrightarrow{AB}=k\overrightarrow{AC}\Leftrightarrow \heva{&4=k\cdot 6\\&x-5=k\cdot(-3)\\&2=k(y+1)}\Leftrightarrow \heva{&k=\dfrac{2}{3}\\&x=3\\&y=2.}\]
Vậy $x+y=5$.
 }
\end{ex}

\begin{ex}%[2-H2B4-SO-11-2425]%[VN-MT-7, Đào Trung Kiên]%[2H2H2-4]
Trong không gian với hệ tọa độ $Oxyz$, điểm $M$ thuộc trục $Ox$ và cách đều hai điểm $A(4; 2; -1)$ và $B(2; 1; 0)$ là
 \choice{$M(-4; 0; 0)$}
 {$M(5; 0; 0)$}
 {\True $M(4; 0; 0)$}
 {$M(-5; 0; 0)$}
 \loigiai{$M\in Ox\Rightarrow M(x; 0; 0)$. Ta có $\overrightarrow{MA}=(4-x; 2; -1)$, $\overrightarrow{MB}=(2-x; 1; 0)$.\\
$M$ cách đều hai điểm $A$, $B$ khi \[MA=MB\Leftrightarrow \sqrt{(4-x)^2+2^2+(-1)^2}=\sqrt{(2-x)^2+1^2+0^2}\Leftrightarrow x=4\]
 }
\end{ex}

\begin{ex}%[2-H2B4-SO-11-2425]%[VN-MT-7, Đào Trung Kiên]%[2H2H2-2]
Trong không gian với hệ tọa độ $Oxyz$, cho ba điểm $A(1; 3; 4)$, $B(1; 0; -2)$ và $C(4; 0; 1)$. Tọa độ trọng tâm $G$ của tam giác $ABC$ là
 \choice{$G(3; 0; 2)$}
 {\True $G(2; 1; 1)$}
 {$G(1; 1; 3)$}
 {$G(3; 0; -1)$}
 \loigiai{
Tọa độ trọng tâm của tam giác $ABC$ là $G(2; 1; 1)$.
 }
\end{ex}
\Closesolutionfile{ans}

\TNTF
\Opensolutionfile{ans}[ans/ans\currfilebase-Phan-II]
\begin{ex}%[2-H2B4-SO-11-2425]%[VN-MT-7, Đào Trung Kiên]%[2H2V1-4]
\immini{Một chất điểm ở vị trí $A$ của hình lập phương $ABCD.A'B'C'D'$. Chất điểm chịu tác động bởi ba lực $\overrightarrow{a}$, $\overrightarrow{b}$, $\overrightarrow{c}$ lần lượt cùng hướng với $\overrightarrow{AD}$, $\overrightarrow{AB}$, $\overrightarrow{AC'}$ như hình vẽ bên. Độ lớn của lực $\overrightarrow{a}$, $\overrightarrow{b}$ và $\overrightarrow{c}$ tương ứng là $10$ N, $10$ N và $10\sqrt{3}$ N.
\choiceTF
{$\overrightarrow{a}+\overrightarrow{b}=\overrightarrow{c}$}
{$\left|\overrightarrow{a}+\overrightarrow{b}\right|=20$ (N)}
{\True $\left|\overrightarrow{a}+\overrightarrow{c}\right|=\left|\overrightarrow{b}+\overrightarrow{c}\right|$}
{\True $\left|\overrightarrow{a}+\overrightarrow{b}+\overrightarrow{c}\right|=30$ (N)}
}{ \begin{tikzpicture}[font=\footnotesize, line join=round, line cap=round, >=stealth, scale=0.7]
 \def\a{3.5}
 \path (0:0) coordinate (A)
 ++(0:\a) coordinate (B)
 ++(-130:\a/2) coordinate (C)
 ($(A)+(C)-(B)$) coordinate (D)
 ($(A)+(-90:\a)$) coordinate (A')
 ($(B)+(-90:\a)$) coordinate (B')
 ($(C)+(-90:\a)$) coordinate (C')
 ($(D)+(-90:\a)$) coordinate (D')
 ($(B)!0.5!(A)$) coordinate (M)
 ($(C')!0.5!(A)$) coordinate (N)
 ($(D)!0.5!(A)$) coordinate (P);
 \draw[dashed] (B)--(A)--(C') (A)--(A')--(B') (A')--(D');
 \draw[->,thick] (A)--(M)node[midway, above] {$\overrightarrow{b}$};
 \draw[->] (D')--(C');
 \draw[->,thick] (A)--(N)node[midway, above right] {$\overrightarrow{c}$};
 \draw[->,thick] (A)--(P)node[midway, above left] {$\overrightarrow{a}$};
 \draw (C)--(C')--(B') (B)--(A)--(D)--(D') (A)--(B)--(B') (B)--(C)--(D) ;
 \foreach \x/\g in {A/90,D/180,C/0,B/0,A'/180,D'/180,C'/0,B'/0}
 \fill (\x) circle (1pt)
 ($(\g:4mm)+(\x)$) node {$\x$}; 
 \end{tikzpicture}
}
\loigiai{
Xét hình lập phương $ABCD.A'B'C'D'$ với cạnh bằng $x>0$, ta có $AC'=\sqrt{AB^2+AD^2+AA'^2}=x\sqrt{3}$.\\
Vì $\triangle ADC'$ vuông tại $D$ nên
$\cos\left(\overrightarrow{a},\overrightarrow{c}\right)=\cos \widehat{DAC'}=\dfrac{AD}{AC'}=\dfrac{1}{\sqrt{3}}$.\\
Tương tự, $\triangle ABC'$ vuông tại $B$ nên $\cos\left(\overrightarrow{b},\overrightarrow{c}\right)=\cos\widehat{BAC'}=\dfrac{AB}{AC'}=\dfrac{1}{\sqrt{3}}$.
 \begin{itemchoice}
 \itemch \textbf{Sai}.\\
Giả sử $\overrightarrow{a}+\overrightarrow{b}=\overrightarrow{d}$. Theo quy tắc hình bình hành thì $\overrightarrow{d}$ cùng hướng với $\overrightarrow{AC}$ nên $\overrightarrow{d}$ không cùng phương với $\overrightarrow{AC'}$. Suy ra $\overrightarrow{a}+\overrightarrow{b}=\overrightarrow{c}$ là sai.
 \itemch \textbf{Sai}.\\
Ta có $\left( \overrightarrow{a} + \overrightarrow{b}\right)^2 = {\overrightarrow{a}}^2 + {\overrightarrow{b}}^2 + 2\overrightarrow{a}\cdot \overrightarrow{b} = 10^2+10^2+0 =200$, suy ra $\left|\overrightarrow{a}+\overrightarrow{b}\right|=10\sqrt{2}$.
 \itemch \textbf{Đúng}.\\
Ta có \begin{itemize}
\item $\left(\overrightarrow{a}+\overrightarrow{c}\right)^2=\left|\overrightarrow{a}\right|^2+2\overrightarrow{a}\cdot\overrightarrow{c}+\left|\overrightarrow{c}\right|^2=10^2+2\cdot 10\cdot 10\sqrt{3}\cdot\dfrac{1}{\sqrt{3}}+\left(10\sqrt{3}\right)^2=600$.\\
Suy ra $\left|\overrightarrow{a}+\overrightarrow{c}\right|=\sqrt{600}$.
\item $\left(\overrightarrow{b}+\overrightarrow{c}\right)^2=\left|\overrightarrow{b}\right|^2+2\overrightarrow{b}\cdot\overrightarrow{c}+\left|\overrightarrow{c}\right|^2=10^2+2\cdot 10\cdot 10\sqrt{3}\cdot\dfrac{1}{\sqrt{3}}+\left(10\sqrt{3}\right)^2=600$.\\
Suy ra $\left|\overrightarrow{a}+\overrightarrow{c}\right|=\sqrt{600}$.
 \end{itemize}
Vậy $\left|\overrightarrow{a}+\overrightarrow{c}\right|=\left|\overrightarrow{b}+\overrightarrow{c}\right|$.
 \itemch \textbf{Đúng}.\\
Giả sử lực tổng hợp là $\overrightarrow{m}$, tức là $\overrightarrow{m}=\overrightarrow{a}+\overrightarrow{b}+\overrightarrow{c}$. Do đó
\begin{eqnarray*}
&& \big|\overrightarrow{m}\big|^2=\left(\overrightarrow{a}+\overrightarrow{b}+\overrightarrow{c}\right)^2\\
&\Leftrightarrow& \left|\overrightarrow{m}\right|^2={\overrightarrow{a}}^2+{\overrightarrow{b}}^2+{\overrightarrow{c}}^2+2\overrightarrow{a}\cdot\overrightarrow{b}+2\overrightarrow{b}\cdot\overrightarrow{c}+2\overrightarrow{c}\cdot\overrightarrow{a}\\
&\Leftrightarrow& \left|\overrightarrow{m}\right|^2=10^2+10^2+(10\sqrt{3})^2+2\cdot 10\cdot10\sqrt{3}\cdot\dfrac{1}{\sqrt{3}}+2\cdot 10\cdot10\sqrt{3}\cdot\dfrac{1}{\sqrt{3}}\\
&\Leftrightarrow& \left|\overrightarrow{m}\right|^2=900.
\end{eqnarray*}
Suy ra cường độ lực tổng hợp $\overrightarrow{a}+\overrightarrow{b}+\overrightarrow{c}$ bằng $30$ N.
\end{itemchoice}
}
\end{ex}

\begin{ex}%[2-H2B4-SO-11-2425]%[VN-MT-7, Đào Trung Kiên]%[2H2H2-4]
Trong không gian với hệ tọa độ $Oxyz$, cho ba điểm $A(2; 3; 1)$, $B(-1; 2; 0)$, $C(1; 1; -2)$.
 \choiceTF
 {\True $\overrightarrow{OA}=2\overrightarrow{i}+3\overrightarrow{j}+\overrightarrow{k}$}
 {$\overrightarrow{AB}=(3; -1; -1)$}
 {\True Gọi $D$ là đỉnh của hình bình hành $ABCD$, khi đó $D(4; 2; -1)$}
 {\True $G$ là trọng tâm của tam giác $ABC$, khi đó $OG=\dfrac{\sqrt{41}}{3}$}
 \loigiai{
 \begin{itemchoice}
 \itemch \textbf{Đúng}.\\
Vì $A(2; 3; 1)$ nên $\overrightarrow{OA}=2\overrightarrow{i}+3\overrightarrow{j}+\overrightarrow{k}$.
 \itemch \textbf{Sai}.\\
$\overrightarrow{AB}=(-3; -1; -1)$.
 \itemch \textbf{Đúng}.\\
Gọi $D(x; y; z)$. Khi đó $\overrightarrow{AB}=(-3; -1; -1)$ và $\overrightarrow{DC}=(1-x; 1-y; -2-z)$.\\
Vì $ABCD$ là hình bình hành nên $\overrightarrow{AB}=\overrightarrow{DC}\Leftrightarrow \heva{&-3=1-x\\&-1=1-y\\&-1=-2-z}\Leftrightarrow \heva{&x=4\\&y=2\\&z=-1.}$\\
Vậy $D(4; 2; -1)$.
 \itemch \textbf{Đúng}.\\
Gọi $G(x; y; z)$ là trọng tâm của tam giác $ABC$. Khi đó $\heva{&x=\dfrac{2-1+1}{3}=\dfrac{2}{3}\\&y=\dfrac{3+2+1}{3}=2\\&z=\dfrac{1+0-2}{3}=-\dfrac{1}{3}.}$\\
Vậy $G\left(\dfrac{2}{3}; 2; -\dfrac{1}{3}\right)$ nên $OG=\sqrt{\left(\dfrac{2}{3}\right)^2+2^2+\left(-\dfrac{1}{3}\right)^2}=\dfrac{\sqrt{41}}{3}$.

\end{itemchoice}
 }
\end{ex}

\begin{ex}%[2-H2B4-SO-11-2425]%[VN-MT-7, Đào Trung Kiên]%[2H2H2-4]
Trong không gian với hệ tọa độ $Oxyz$.
 \choiceTF
 {\True Cho hai vectơ $\overrightarrow{u}=m\overrightarrow{i}+2\overrightarrow{j}-3\overrightarrow{k}$, $\overrightarrow{v}=m\overrightarrow{j}+2\overrightarrow{i}+4\overrightarrow{k}$. Biết rằng $\overrightarrow{u}\cdot \overrightarrow{v}=8$, khi đó $m=5$}
 {Góc giữa hai vectơ $\overrightarrow{u}=(1; -2;1)$ và $\overrightarrow{v}=(-2; 1; 1)$ bằng $60^\circ$}
 {\True Cho lăng trụ đứng $ABC.A'B'C'$ có $A(0; 0; 0)$, $B(2; 0; 0)$, $C(0; 2; 0)$ và $A'(0; 0; 2)$. Góc giữa $BC'$ và $A'C$ bằng $90^\circ$}
 {Gọi $\varphi$ là góc giữa hai vectơ $\overrightarrow{a}$ và $\overrightarrow{b}$ (với $\overrightarrow{a}$ và $\overrightarrow{b}$ khác $\overrightarrow{0}$), khi đó $\cos\varphi=\dfrac{|\overrightarrow{a}|\cdot|\overrightarrow{b}|}{\overrightarrow{a}\cdot\overrightarrow{b}}$}
\loigiai{
 \begin{itemchoice}
 \itemch \textbf{Đúng}.\\
Từ giả thiết ta có $\overrightarrow{u}=(m; 2; -3)$, $\overrightarrow{v}=(2; m; 4)$.\\
Do đó $\overrightarrow{u}\cdot\overrightarrow{v}=8\Leftrightarrow 2m+2m-3\cdot 4=8\Leftrightarrow m=5$.
 \itemch \textbf{Sai}.\\
Ta có $\cos(\overrightarrow{u},\overrightarrow{v})=\dfrac{\overrightarrow{u}\cdot\overrightarrow{v}}{|\overrightarrow{u}|\cdot|\overrightarrow{v}|}=\dfrac{-3}{\sqrt{6}\cdot\sqrt{6}}=-\dfrac{1}{2}\Rightarrow \left(\overrightarrow{u},\overrightarrow{v}\right)=120^\circ$.
 \itemch \textbf{Đúng}.\\
Gọi $C'(x; y; z)$, vì $ABC.A'B'C'$ là hình lăng trụ đứng nên $\overrightarrow{AA'}=\overrightarrow{CC'}\Leftrightarrow \heva{&x-0=0\\&y-2=0\\&z-0=2}$.\\
Từ đó ta có $B(2; 0; 0)$, $C'(0; 2; 2)$ nên $\overrightarrow{BC'}=(-2; 2; 2)$.\\
Vì $A'(0; 0; 2)$ và $C(0; 2; 0)$ nên $\overrightarrow{A'C}=(0; 2; -2)$.\\
Từ đó suy ra $\overrightarrow{BC'}\cdot\overrightarrow{A'C}=0$ nên góc giữa $BC'$ và $A'C$ bằng $90^\circ$.
\itemch \textbf{Sai}.\\
Công thức tính côsin của góc giữa hai vectơ $\overrightarrow{a}$ và $\overrightarrow{b}$ (với $\overrightarrow{a}$ và $\overrightarrow{b}$ khác $\overrightarrow{0}$) là  $\cos\left(\overrightarrow{a},\overrightarrow{b}\right)=\dfrac{\overrightarrow{a}\cdot\overrightarrow{b}}{|\overrightarrow{a}|\cdot|\overrightarrow{b}|}$.
\end{itemchoice}
 }
\end{ex}

\begin{ex}%[2-H2B4-SO-11-2425]%[VN-MT-7, Đào Trung Kiên]%[2H2V2-6]
Hình minh họa sơ đồ ngôi nhà Trong không gian với hệ tọa độ $Oxyz$, trong đó nền nhà, bốn bức tường và hai mái nhà đều là hình chữ nhật.
\begin{center}
\begin{tikzpicture}[font=\footnotesize, line join=round, line cap=round, >=stealth, scale=1.2]
 \def\a{3}
 \def\b{5}
 \def\h{3}
 \path (0:0) coordinate (C)
 ++(0:\a) coordinate (B)
 ++(-160:\b) coordinate (O)
 ($(O)+(B)-(C)$) coordinate (A)
 ($(O)+(90:\h)$) coordinate (E)
 ($(B)+(90:\h)$) coordinate (G)
 ($(C)+(90:\h)$) coordinate (H)
 ($(A)+(90:\h)$) coordinate (F)
 ($(A)+(0:1)$) coordinate (x)
 ($(H)+(35:2)$) coordinate (Q)
 ($(E)+(35:2)$) coordinate (P)
 ($(E)+(90:1)$) coordinate (z)
 ($(O)!1.3!(C)$) coordinate (y);
 \draw[dashed] (G)--(H)--(C)--(B) (C)--(O);
 \draw (G)--(Q)--(H)--(E)--(F)--(G)--(B)--(A)--(O)--(E) (F)--(A) (F)--(P)--(E) (P)--(Q);
 \draw [->] (A)--(x);
 \draw [->] (E)--(z);
 \draw [->,dashed] (C)--(y);
 \draw (Q)node[above]{$Q(2; 5; 4)$} (G)node[right]{$G(4; 5; 3)$} (B)node[right]{$B(4; 5; 0)$} (P)node[right]{$P(2; 0; 4)$} (O)node[below]{$O(0; 0; 0)$} (E)node[left]{$E(0; 0; 3)$} (x)node[below]{$x$} (y)node[above]{$y$} (z)node[left]{$z$};
 \foreach \x/\g in {A/-90,C/180,F/0,H/90}
 \fill (\x) circle (1pt)
 ($(\g:4mm)+(\x)$) node {$\x$}; 
 \fill (E) circle (1pt) (Q) circle (1pt) (O) circle (1pt) (G) circle (1pt) (P) circle (1pt) (B) circle (1pt);
\end{tikzpicture}
\end{center}
 \choiceTF
 {\True Tọa độ điểm $F(4; 0; 3)$}
 {Tọa độ vectơ $\overrightarrow{AH}=(4; 5; 3)$}
 {$\overrightarrow{AH}\cdot\overrightarrow{AF}=3$}
 {\True Góc đốc của mái nhà, tức là số đo của góc nhị diện có cạnh là đường thẳng $FG$, hai mặt lần lượt là $(FGQP)$ và $(FGHE)$ bằng $26{,}6^\circ$ (làm tròn đến hàng phần mười của đơn vị độ)}
 \loigiai{
 \begin{itemchoice}
 \itemch \textbf{Đúng}.\\
Vì nền nhà là hình chữ nhật nên $OACB$ là hình chữ nhật, suy ra $x_A=x_B=4, y_C=y_B=5$.\\
Do điểm $A$ nằm trên trục $O x$ nên tọa độ điểm $A(4; 0; 0)$; điểm $C$ nằm trên trục $Oy$ nên tọa độ điểm $C(0; 5; 0)$.\\
Tường nhà là hình chữ nhật nên $OCHE$ là hình chữ nhật, suy ra $y_H=y_C=5$, $z_H=z_E=3$.\\
Do $H$ nằm trên mặt phẳng $(Oyz)$ nên tọa độ điểm $H(0; 5; 3)$.\\
Tứ giác $OAFE$ là hình chữ nhật nên $x_F=x_A=4, z_F=z_E=3$.\\
Do $F$ nằm trên mặt phẳng $(Oxz)$ nên tọa độ điểm $F(4; 0; 3)$.
 \itemch \textbf{Sai}.\\
Ta có toạ độ vectơ $\overrightarrow{AH}=(-4; 5; 3)$.
 \itemch \textbf{Sai}.\\
Ta có $\overrightarrow{AF}=(0; 0; 3)$. Suy ra $\overrightarrow{AH}\cdot\overrightarrow{AF}=0+0+9=9$.
 \itemch \textbf{Đúng}.\\
Để tính góc đốc của mái nhà, ta tính số đo của góc nhị diện có cạnh là đường thẳng $FG$, hai mặt lần lượt là $(FGQP)$ và $(FGHE)$.\\
Do mặt phẳng $(O z x)$ vuông góc với hai mặt phẳng $(FGQP)$ và $(F G H E)$ nên $\widehat{PFE}$ là góc phẳng nhị diện cần tìm.\\
Ta có $\overrightarrow{FP}=(-2; 0; 1), \overrightarrow{FE}=(-4; 0; 0)$ suy ra 
\[\cos\widehat{PFE}=\cos \left(\overrightarrow{FP}, \overrightarrow{FE}\right)=\dfrac{\overrightarrow{FP} \cdot \overrightarrow{FE}}{\left|\overrightarrow{FP}\right|\cdot\left|\overrightarrow{FE}\right|}= \dfrac{(-2)(-4)+0\cdot 0+1\cdot 0}{\sqrt{(-2)^2+0^2+1^2} \cdot \sqrt{(-4)^2+0^2+0^2}}=\dfrac{2 \sqrt{5}}{5}.\]
Do đó, $\widehat{PFE} \approx 26{,}6^{\circ}$.\\
Vậy góc đốc mái nhà khoảng $26{,}6^{\circ}$.
\end{itemchoice}
 }
\end{ex}
\Closesolutionfile{ans}

\TNSA
\Opensolutionfile{ans}[ans/ans\currfilebase-Phan-III]
\begin{ex}%[2-H2B4-SO-11-2425]%[VN-MT-7, Đào Trung Kiên]%[2H2H1-3]
Cho hai vectơ $\overrightarrow{a}$, $\overrightarrow{b}$ thỏa mãn $\left|\overrightarrow{a}\right|=3$, $\left|\overrightarrow{b}\right|=4$, $\left|\overrightarrow{a}+\overrightarrow{b}\right|=6$. Tính $\left|\overrightarrow{a}-\overrightarrow{b}\right|$ (làm tròn kết quả đến hàng phần trăm).
\shortans{3{,}74}
\loigiai{
Ta có $\left|\overrightarrow{a}+\overrightarrow{b}\right|^2=\left(\overrightarrow{a}+\overrightarrow{b}\right)^2=\left|\overrightarrow{a}\right|^2+2\overrightarrow{a}\overrightarrow{b}+\left|\overrightarrow{b}\right|^2\Rightarrow 2\overrightarrow{a}\overrightarrow{b}=\left|\overrightarrow{a}+\overrightarrow{b}\right|^2-\left|\overrightarrow{a}\right|^2-\left|\overrightarrow{b}\right|^2=11$.\\
$\left|\overrightarrow{a}-\overrightarrow{b}\right|^2=\left(\overrightarrow{a}-\overrightarrow{b}\right)^2=\left|\overrightarrow{a}\right|^2-2\overrightarrow{a}\overrightarrow{b}+\left|\overrightarrow{b}\right|^2=9-11+16=14\Rightarrow \left|\overrightarrow{a}-\overrightarrow{b}\right|=\sqrt{14}\approx 3{,}74$.
}
\end{ex}

\begin{ex}%[2-H2B4-SO-11-2425]%[VN-MT-7, Đào Trung Kiên]%[2H2V1-4]
Một chiếc đèn trang trí hình tròn được treo song song với mặt phẳng trần nhà nằm ngang bởi ba sợi dây không giãn $OA$, $OB$, $OC$ đôi một vuông góc (như hình vẽ dưới đây). Biết lực căng của sợi dây tương ứng trên mỗi dây $OA$, $OB$, $OC$ lần lượt là $\overrightarrow{F_1}$, $\overrightarrow{F_2}$, $\overrightarrow{F_3}$ thỏa mãn $\left|\overrightarrow{F_1}\right|=\left|\overrightarrow{F_2}\right|=\left|\overrightarrow{F_3}\right|=16$ (N). Tính trọng lượng (đơn vị: N) của chiếc đèn đó (làm tròn kết quả đến hàng phần mười).
\begin{center}
\begin{tikzpicture}[line join=round, line cap=round,>=stealth,xscale=1,yscale=0.4]
 \path (0:0) coordinate (O')
 ++(0:2) coordinate (C)
 ($(O')+(150:2)$) coordinate (B)
 ($(O')+(220:2)$) coordinate (A)
 ($(O')+(90:8)$) coordinate (O)
 ($(O)!0.5!(A)$) coordinate (A1)
 ($(O)!0.5!(B)$) coordinate (B1)
 ($(O)!0.5!(C)$) coordinate (C1);
 \draw[fill=blue,opacity=0.15] (O') circle (2 cm);
 \draw (O') circle (2 cm);
 \draw (O)--(A) (O)--(B) (O)--(C) (2,0)--(2,-1) (-2,0)--(-2,-1);
 \draw (2,-1) arc (0:-180:2);
 \draw[color=red,dashed] (C)--(O');
 \draw[dashed] (O')--(0,-3);
 \draw[fill=blue,opacity=0.4] (-2,8) rectangle (2,8.6);
 \draw [->] (0,-3)--(0,-7)node[midway, right]{$\overrightarrow{P}$};
 \draw [->] (O)--(A1)node[above right]{$\overrightarrow{F_1}$};
 \draw [->] (O)--(B1)node[above left]{$\overrightarrow{F_2}$};
 \draw [->] (O)--(C1)node[above right]{$\overrightarrow{F_3}$};
 \draw (0,7.5) node[below] {$O$};
 \foreach \x/\g in {A/60,B/100,C/0}
 \fill (\x) circle (1pt)
 ($(\g:5mm)+(\x)$) node {$\x$};
\end{tikzpicture}
\end{center}
 \shortans{27{,}7}
 \loigiai{
\begin{center}
\begin{tikzpicture}[line join=round, line cap=round,>=stealth,scale=1]
 \def\a{3.5}
 \path (0:0) coordinate (O)
 ++(0:\a) coordinate (B)
 ++(-160:\a*1.3) coordinate (A)
 ($(B)+(A)-(O)$) coordinate (E)
 ($(O)+(90:\a)$) coordinate (C)
 ($(E)+(90:\a)$) coordinate (F)
 ($(A)+(90:\a)$) coordinate (A')
 ($(B)+(90:\a)$) coordinate (B');
 \draw[dashed] (O)--(A) (O)--(B) (O)--(C) (O)--(F);
 \draw[thick] (C)--(F) (C)--(A') (C)--(B') (F)--(E) (A)--(E)--(B)--(B')--(F)--(A')--cycle;
 \foreach \x/\g in {O/170,A/-90,E/-90,B/-60,C/100,F/0}
 \fill[black] (\x) circle (1pt)
 ($(\g:4mm)+(\x)$) node {$\x$}; 
\end{tikzpicture}
\end{center}
Gọi $P$ là trọng lượng của đèn, ta có $P=\left|\overrightarrow{F_1}+\overrightarrow{F_2}+\overrightarrow{F_3}\right|=\left|\overrightarrow{OA}+\overrightarrow{OB}+\overrightarrow{OC}\right|$.\\
Vẽ hình vuông $OAEB$, ta có $\overrightarrow{OA}+\overrightarrow{OB}=\overrightarrow{OE}$ (quy tắc hình bình hành).\\
Vẽ hình chữ nhật $OCFE$, ta có $\overrightarrow{OC}+\overrightarrow{OE}=\overrightarrow{OF}$ (quy tắc hình bình hành).\\
Suy ra $P=\left|\overrightarrow{OF}\right|=OF$.\\
Xét hình vuông $OAEB$, cạnh bằng $16$ và có đường chéo $OE=16\sqrt{2}$.\\
Xét tam giác vuông $OEF$, vuông tại $E$, có $OF=\sqrt{OE^2+EF^2}=\sqrt{\left(16\sqrt{2}\right)^2+16^2}=16\sqrt{3}\approx 27{,}7$.\\
Vậy $P\approx 27{,}7$ N.
 }
\end{ex}

\begin{ex}%[2-H2B4-SO-11-2425]%[VN-MT-7, Đào Trung Kiên]%[2H2H2-4]
Trong không gian với hệ tọa độ $Oxyz$, cho hai điểm $B(2; 1; 0)$, $C(1; 4; 5)$. Điểm $M(x; y; z)$ thuộc trục hoành sao cho $MB=MC$. Khi đó giá trị $2x+y+z$ bằng bao nhiêu?
 \shortans{-37}
 \loigiai{
Do điểm $M \in Ox$ nên $M(x; 0; 0)$, ta có 
\begin{eqnarray*}
MB=MC&\Leftrightarrow& MB^2=MC^2\Leftrightarrow (2-x)^2+1^2+0^2=(1-x)^2+4^2+5^2\\
&\Leftrightarrow&x^2-4x+5=x^2-2x+42\Leftrightarrow x=-\dfrac{37}{2}. 
\end{eqnarray*}
Vậy $M\left(-\dfrac{37}{2};0;0\right)\Rightarrow 2x+y+z=-37$.
 }
\end{ex}

\begin{ex}%[2-H2B4-SO-11-2425]%[VN-MT-7, Đào Trung Kiên]%[2H2H1-4]
Trong không gian tọa độ $Oxyz$ cho $\overrightarrow{a}$ và $\overrightarrow{b}$ tạo với nhau một góc $120^{\circ}$. Biết rằng $|\overrightarrow{a}|=4$; $|\overrightarrow{b}|=3$, tính giá trị của biểu thức $A=|\overrightarrow{a}-\overrightarrow{b}|+|\overrightarrow{a}+\overrightarrow{b}|$ ( làm tròn kết quả đến hàng phần trăm).

\shortans{9{,}69}
 \loigiai{
Ta có $|\overrightarrow{a}-\overrightarrow{b}|^2=\left(\overrightarrow{a}-\overrightarrow{b}\right)^2=|\overrightarrow{a}|^2-2 \overrightarrow{a} \cdot \overrightarrow{b}+|\overrightarrow{b}|^2=16-2|\overrightarrow{a}| \cdot|\overrightarrow{b}|\cdot\cos 120^{\circ}+9=37$.\\
Tương tự $|\overrightarrow{a}+\overrightarrow{b}|^2=\left(\overrightarrow{a}+\overrightarrow{b}\right)^2=|\overrightarrow{a}|^2+2 \overrightarrow{a} \cdot \overrightarrow{b}+|\overrightarrow{b}|^2=16+2|\overrightarrow{a}| \cdot|\overrightarrow{b}|\cdot\cos 120^{\circ}+9=13$.\\
Do đó $A=|\overrightarrow{a}-\overrightarrow{b}|+|\overrightarrow{a}+\overrightarrow{b}|=\sqrt{37}+\sqrt{13} \approx 9{,}69$.
 }
\end{ex}

\begin{ex}%[2-H2B4-SO-11-2425]%[VN-MT-7, Đào Trung Kiên]%[2H2V2-6]
Người ta cần lắp một camera phía trên sân bóng để phát sóng truyền hình một trận bóng đá, camera có thể di động để luôn thu được hình ảnh rõ nét về diễn biến trên sân. Các kĩ sư dự định trồng bốn chiếc cột cao 30 m và sử dụng hệ thống cáp gắn vào bốn đầu cột để giữ camera ở vị trí mong muốn.\\
Mô hình thiết kế được xây dựng như sau\\
Trong hệ trục toạ độ $Oxyz$ (đơn vị độ dài trên mỗi trục là $1$ m), các đỉnh của bốn chiếc cột lần lượt là các điểm $M(90; 0; 30)$, $N(90; 120; 30)$, $P(0; 120; 30)$, $Q(0; 0; 30)$.\\
Giả sử $K_0$ là vị trí ban đầu của camera có cao độ bằng $25$ và $K_0M=K_0N=K_0P=K_0Q$. Để theo dõi quả bóng đến vị trí $A$, camera được hạ thấp theo phương thẳng đứng xuống điểm $K_1$ cao độ bằng $19$.\\
Tọa độ của vectơ $\overrightarrow{K_0K_1}=(a; b; c)$ với $a$, $b$, $c$ là các số thực. Tính $P=a+b-c$.
\begin{center}
 \begin{tikzpicture}
 \def\a{5.5}
 \def\b{2}
 \def\h{3.5}
 \path (0:0) coordinate (O)
 ++(0:\a) coordinate (P')
 ($(O)+(220:\b)$) coordinate (M')
 ($(O)+(0:\a+1)$) coordinate (y)
 ($(P')+(M')-(O)$) coordinate (F)
 ($(O)+(90:\h)$) coordinate (Q)
 ($(M')+(90:\h)$) coordinate (M)
 ($(F)+(90:\h)$) coordinate (N)
 ($(P')+(90:\h)$) coordinate (P)
 ($(O)!1.3!(M')$) coordinate (x)
 ($(O)!1.3!(Q)$) coordinate (z)
 ($(O)!0.5!(Q)$) coordinate (Q_1)
 ($(O)!0.7!(Q)$) coordinate (Q_0)
 ($(F)!0.5!(N)$) coordinate (N_1)
 ($(F)!0.7!(N)$) coordinate (N_0)
 ($(O)!0.1!(F)$) coordinate (A)
 ($(O)!0.9!(F)$) coordinate (C)
 ($(M')!0.1!(P')$) coordinate (B)
 ($(M')!0.9!(P')$) coordinate (D);
 \coordinate (H) at (intersection of Q--N and M--P);
 \coordinate (H') at (intersection of F--O and M'--P');
 \coordinate (K_0) at (intersection of Q_0--N_0 and H--H');
 \coordinate (K_1) at (intersection of Q_1--N_1 and H--H');
 \draw [->] (O)--(y) node[below]{$y$};
 \draw [->] (O)--(x) node[below]{$x$};
 \draw [->] (O)--(z) node[left]{$z$};
 \draw [fill=green] (A)--(B)--(C)--(D)--(A);
 \draw[dashed] (M')--(P') (F)--(O) (K_1)--(Q) (K_1)--(M) (K_1)--(P) (K_1)--(N);
 \draw (Q)--(K_0)--(N) (M')--(M)--(K_0)--(P) (N)--(F) (P)--(P');
 \foreach \x/\g in {O/45,F/-90,P/40,Q/60,M/160,K_0/100,K_1/-90,N/0}
 \fill (\x) circle (1pt)
 ($(\g:4mm)+(\x)$) node {$\x$}; 
 \fill[black](4.3,-0.4) node [below right]{$A$} circle (1.5pt);
 \end{tikzpicture}
\end{center} 
\shortans{6}
 \loigiai{
\begin{center}
 \begin{tikzpicture}
 \def\a{5.5}
 \def\b{2}
 \def\h{3.5}
 \path (0:0) coordinate (O)
 ++(0:\a) coordinate (P')
 ($(O)+(220:\b)$) coordinate (M')
 ($(O)+(0:\a+1)$) coordinate (y)
 ($(P')+(M')-(O)$) coordinate (F)
 ($(O)+(90:\h)$) coordinate (Q)
 ($(M')+(90:\h)$) coordinate (M)
 ($(F)+(90:\h)$) coordinate (N)
 ($(P')+(90:\h)$) coordinate (P)
 ($(O)!1.3!(M')$) coordinate (x)
 ($(O)!1.3!(Q)$) coordinate (z)
 ($(O)!0.5!(Q)$) coordinate (Q_1)
 ($(O)!0.7!(Q)$) coordinate (Q_0)
 ($(F)!0.5!(N)$) coordinate (N_1)
 ($(F)!0.7!(N)$) coordinate (N_0)
 ($(O)!0.1!(F)$) coordinate (A)
 ($(O)!0.9!(F)$) coordinate (C)
 ($(M')!0.1!(P')$) coordinate (B)
 ($(M')!0.9!(P')$) coordinate (D);
 \coordinate (H) at (intersection of Q--N and M--P);
 \coordinate (H') at (intersection of F--O and M'--P');
 \coordinate (K_0) at (intersection of Q_0--N_0 and H--H');
 \coordinate (K_1) at (intersection of Q_1--N_1 and H--H');
 \draw [->] (O)--(y) node[below]{$y$};
 \draw [->] (O)--(x) node[below]{$x$};
 \draw [->] (O)--(z) node[left]{$z$};
 \draw [fill=green] (A)--(B)--(C)--(D)--(A);
 \draw[dashed] (M')--(P') (F)--(O) (K_1)--(Q) (K_1)--(M) (K_1)--(P) (K_1)--(N);
 \draw (Q)--(K_0)--(N) (M')--(M)--(K_0)--(P) (N)--(F) (P)--(P');
 \foreach \x/\g in {O/45,F/-90,P/40,Q/60,M/160,K_0/100,K_1/-90,N/0}
 \fill (\x) circle (1pt)
 ($(\g:4mm)+(\x)$) node {$\x$}; 
 \fill[black](4.3,-0.4) node [below right]{$A$} circle (1.5pt);
\end{tikzpicture}

\end{center} 
Gọi $K_0(x; y; 25)$ và $K_1(x; y; 19)$ suy ra $\overrightarrow{K_0K_1}=(0; 0; -6)$.\\
Vậy $a =0$, $b=0$, $c=-6$ nên $P=a+b-c=6$.}
\end{ex}

\begin{ex}%[2-H2B4-SO-11-2425]%[VN-MT-7, Đào Trung Kiên]%[2H2H1-3]
\immini{Cho tứ diện $OABC$ có các cạnh $OA$, $OB$, $OC$ đôi một vuông góc và $OA=OB=OC=1$. Gọi $M$ là trung điểm của cạnh $AB$. Côsin của góc giữa hai vectơ $\overrightarrow{OM}$ và $\overrightarrow{AC}$ bằng $-\dfrac{a}{b}$ với $\dfrac{a}{b}$ là phân số tối giản. Tính $Q = a\cdot b$.}{
\begin{tikzpicture}[line join=round, line cap=round,>=stealth,scale=0.8]
 \def\a{4} %Khai báo cạnh
 \def\h{3}
 \path (0:0) coordinate (O)
 ++(0:\a) coordinate (B)
 ($(O)+(-50:2.4)$) coordinate (A)
 ($(O)+(90:\h)$) coordinate (C)
 ($(A)!0.5!(B)$) coordinate (M);
 \draw (C)--(O)--(A)--(C)--(B)--(A);
 \draw[dashed,->] (O)--(M);
 \draw[dashed] (O)--(B) ;
 \draw[->] (A)--(C);
 \foreach \x /\goc in {A/-90,B/0,C/170,M/-40,O/180}
 \fill (\x) circle (1pt)
 ($(\x)+(\goc:3mm)$) node {$\x$};
% \draw pic[draw,angle radius=2mm]{right angle=B--A--S};%Theo chiều dương
\end{tikzpicture}
}
 \shortans{2}
 \loigiai{
Đặt $\overrightarrow{OA}=\overrightarrow{a}$, $\overrightarrow{OB}=\overrightarrow{b}$, $\overrightarrow{OC}=\overrightarrow{c}$.\\
Khi đó, $\left|\overrightarrow{a}\right|=\left|\overrightarrow{b}\right|=\left|\overrightarrow{c}\right|=1$ và $\overrightarrow{a}\cdot\overrightarrow{b}=\overrightarrow{a}\cdot\overrightarrow{c}=\overrightarrow{b}\cdot\overrightarrow{c}=0$.\\
Ta có $\cos\left(\overrightarrow{OM},\overrightarrow{AC}\right)=\dfrac{\overrightarrow{OM}\cdot\overrightarrow{AC}}{\left|\overrightarrow{OM}\right|\cdot\left|\overrightarrow{AC}\right|}$.\\
Mặt khác, do $\overrightarrow{OM}=\dfrac{1}{2}\left(\overrightarrow{OA}+\overrightarrow{OB}\right)=\dfrac{1}{2}\left(\overrightarrow{a}+\overrightarrow{b}\right)$ và $\overrightarrow{AC}=\overrightarrow{OC}-\overrightarrow{OA}=\overrightarrow{c}-\overrightarrow{a}$ nên 
\[\overrightarrow{OM}\cdot\overrightarrow{AC}=\dfrac{1}{2}\left(\overrightarrow{a}+\overrightarrow{b}\right)\cdot\left(\overrightarrow{c}-\overrightarrow{a}\right)=\dfrac{1}{2}\left(\overrightarrow{a}\cdot\overrightarrow{c}-{\overrightarrow{a}}^2+\overrightarrow{b}\cdot\overrightarrow{c}-\overrightarrow{b}\cdot\overrightarrow{a}\right)=-\dfrac{1}{2}.\]
Ta có $AC=\sqrt{OA^2+OC^2}=\sqrt{2}$, $OM=\dfrac{1}{2}AB=\dfrac{1}{2}\sqrt{OA^2+OC^2}=\dfrac{\sqrt{2}}{2}$.\\
Từ đó $\cos\left(\overrightarrow{OM},\overrightarrow{AC}\right)=-\dfrac{1}{2}$ nên $a=1$ và $b=2$.\\
Vậy $Q=a\cdot b=2$.
 }
\end{ex}
\Closesolutionfile{ans}
 
% \begin{indapan}
% 	{ans/ans\currfilebase}
% \end{indapan}


% \begin{name}
 {Biên soạn: Vũ Hồng Toàn \\ Phản biện: Lê Văn Hiếu}
{Đề ôn tập chương II}
\end{name}

\TN
\Opensolutionfile{ans}[ans/ans\currfilebase-Phan-I]

\begin{ex}%[2-H2B4-SO-12-2425 (Nguồn Đề 3 - Bài 4)]%[VN-MT-7, Vũ Hồng Toàn]%[2H2H1-2]
Cho hình chóp $ S.ABCD$ có đáy là hình bình hành tâm $O$. Đẳng thức nào sau đây \textbf{sai}?
\choice
{$\overrightarrow{BC}+\overrightarrow{DA}=\overrightarrow{BA}+\overrightarrow{DC}$}
{$\overrightarrow{SA}+\overrightarrow{SC}=\overrightarrow{SB}+\overrightarrow{SD}$}
{$\overrightarrow{SA}+\overrightarrow{SB}+\overrightarrow{SC}+\overrightarrow{SD}=4\overrightarrow{SO}$}
{\True $\overrightarrow{AB}+\overrightarrow{CD}=\overrightarrow{AC}+\overrightarrow{BD}$}
\loigiai{
\begin{center}
\begin{tikzpicture}[scale=1, font=\footnotesize, line join=round, line cap=round, >=stealth]
\def\a{4} \def\b{2.5} \def\h{4.5} \def\g{30}
\path
(0,0) coordinate (A)
(0:\a) coordinate (D)++
(\g-180:\b) coordinate (C)++(180:\a) coordinate (B)
($(A)!.5!(C)$) coordinate (O)++ (90:\h) coordinate (S)
;
\draw[dashed](B)--(A)--(D)--cycle (A)--(C);
\draw[->,dashed] (S)--(O);
\draw[->,dashed] (S)--(A);
\draw[->] (S)--(B);
\draw[->] (S)--(C);
\draw[->] (S)--(D);
\draw (B)--(C)--(D);
\foreach \x /\gN in {A/160,B/180,C/-20,S/90,O/-90,D/0}
\fill(\x) circle (1pt)($(\x)+(\gN:3mm)$) node {$\x$};
\end{tikzpicture}
\end{center}
Với $O$ là trung điểm của $AC$, ta có $\overrightarrow{SA}+\overrightarrow{SC}=2\overrightarrow{SO}$.\\
Với $O$ là trung điểm của $BD$, ta có $\overrightarrow{SB}+\overrightarrow{SD}=2\overrightarrow{SO}$.\\
Từ đó suy ra
\begin{itemize}
\item $\overrightarrow{SA}+\overrightarrow{SC}=\overrightarrow{SB}+\overrightarrow{SD}$.
\item $\overrightarrow{SA}+\overrightarrow{SB}+\overrightarrow{SC}+\overrightarrow{SD}=4\overrightarrow{SO}$.
\item
$\overrightarrow{BC}+\overrightarrow{DA}=\overrightarrow{BA}+\overrightarrow{AC}+\overrightarrow{DC}+\overrightarrow{CA}=\overrightarrow{BA}+\overrightarrow{DC}+\left(\overrightarrow{AC}+\overrightarrow{CA}\right)=\overrightarrow{BA}+\overrightarrow{DC}$.
\item $\overrightarrow{AB}+\overrightarrow{CD}=\overrightarrow{AC}+\overrightarrow{CB}+\overrightarrow{CB}+\overrightarrow{BD}=\overrightarrow{AC}+\overrightarrow{BD}+2\overrightarrow{CB}$.
\end{itemize}
}
\end{ex}


\begin{ex}%[2-H2B4-SO-12-2425 (Nguồn Đề 3 - Bài 4)]%[VN-MT-7, Vũ Hồng Toàn]%[2H2H1-3]
Cho hình lập phương $ABCD.A_1B_1C_1D_1$ có cạnh $a$. Gọi $M$ là trung điểm $AD$. Giá trị $\overrightarrow{B_1M}\cdot\overrightarrow{BD_1}$ bằng
\choice
{$a^2$}
{\True $\dfrac{1}{2} a^2$}
{$\dfrac{3}{2} a^2$}
{$\dfrac{3}{4} a^2$}
\loigiai{
\begin{center}
\begin{tikzpicture}[scale=1, font=\footnotesize, line join=round, line cap=round, >=stealth]
\def\a{4} \def\b{2.5} \def\h{3.5} \def\g{30}
\path
(0,0) coordinate (A)
(0:\a) coordinate (D)++
(\g-180:\b) coordinate (C)++(180:\a) coordinate (B)
(90:\h) coordinate (A_1) ++(0:\a) coordinate (D_1)++
(\g-180:\b) coordinate (C_1)++(180:\a) coordinate (B_1)
($(A)!.5!(D)$) coordinate (M)
;
\draw[dashed](B)--(A)--(D)--cycle (A_1)--(A)--(C);
\draw[dashed,->] (B_1)--(M);
\draw[dashed,->] (B)--(D_1);
\draw (A_1)--(B_1)--(C_1)--(D_1) --cycle (B_1)--(B)--(C)--(C_1)--(A_1) (C)--(D)--(D_1)--(B_1) ;
\foreach \x /\gN in {A/40,B/180,C/-20,M/60,A_1/90,D_1/0,C_1/90,B_1/180,D/0}
\fill(\x) circle (1pt)($(\x)+(\gN:3mm)$) node {$\x$};
\end{tikzpicture}
\end{center}
Áp dụng quy tắc cộng ta có
\begin{itemize}
 \item $\overrightarrow{B_1M}=\overrightarrow{B_1B}+\overrightarrow{BA }+\overrightarrow{AM}$;
 \item $\overrightarrow{BD_1}=\overrightarrow{BA}+\overrightarrow{AD}+\overrightarrow{DD_1}$.
\end{itemize}
Khi đó
\allowdisplaybreaks
\begin{eqnarray*}
\overrightarrow{B_1M}\cdot \overrightarrow{BD_1}&=&\left(\overrightarrow{B_1 B}+\overrightarrow{BA}+\overrightarrow{AM}\right)\cdot\left(\overrightarrow{BA}+\overrightarrow{AD}+\overrightarrow{DD_1}\right)\\
&=&\overrightarrow{B_1B}\cdot\overrightarrow{BA}+\overrightarrow{B_1B}\cdot\overrightarrow{AD}+\overrightarrow{B_1B}\cdot\overrightarrow{DD_1}\\&&+{\overrightarrow{BA}}^2+\overrightarrow{BA}\cdot \overrightarrow{AD}+\overrightarrow{BA}\cdot \overrightarrow{DD_1}\\&&+\overrightarrow{AM}\cdot\overrightarrow{BA}+\overrightarrow{AM}\cdot\overrightarrow{AD}+\overrightarrow{AM}\cdot\overrightarrow{DD_1}\\
&=&\overrightarrow{B_1B}\cdot\overrightarrow{DD_1}+{\overrightarrow{BA}}^2+\overrightarrow{AM}\cdot\overrightarrow{AD}\\
&=&-{\overrightarrow{B_1B}}^2+{\overrightarrow{BA}}^2+\dfrac{1}{2}{\overrightarrow{AD}}^2\\
&=&-a^2+a^2+\dfrac{a^2}{2} \\
&=&\dfrac{a^2}{2}.
\end{eqnarray*}
}
\end{ex}


\begin{ex}%[2-H2B4-SO-12-2425 (Nguồn Đề 3 - Bài 4)]%[VN-MT-7, Vũ Hồng Toàn]%[2H2N2-3]
Trong KG $Oxyz$, cho điểm $A(3;-1;5)$. Tọa độ của vectơ $\overrightarrow{OA}$ là
\choice
{$(3;1;5)$}
{\True $(3;-1;5)$}
{$(-3;-1;5)$}
{$(-3;1;-5)$}
\loigiai{Ta có
$A(3;-1;5)\Rightarrow \overrightarrow{OA}=(3;-1;5)$.
}
\end{ex}


\begin{ex}%[2-H2B4-SO-12-2425 (Nguồn Đề 3 - Bài 4)]%[VN-MT-7, Vũ Hồng Toàn]%[2H2N2-3]
Trong KG $Oxyz$, cho hai điểm $M\left(\dfrac{1}{2};1;-3\right)$ và $N\left(\dfrac{1}{2};-2;4\right)$. Tọa độ của vectơ $\overrightarrow{MN}$ là
\choice
{$(1;-1;1)$}
{\True $(0;-3;7)$}
{$(0;3;-7)$}
{$\left(\dfrac{1}{2};-\dfrac{1}{2};\dfrac{1}{2} \right)$}
\loigiai{
Ta có $\overrightarrow{MN}=\left(\dfrac{1}{2}-\dfrac{1}{2};(-2)-1;4-(-3)\right)=(0;-3;7)$.
}
\end{ex}


\begin{ex}%[2-H2B4-SO-12-2425 (Nguồn Đề 3 - Bài 4)]%[VN-MT-7, Vũ Hồng Toàn]%[2H2H2-3]
Trong KG $Oxyz$, cho các vectơ $\overrightarrow{a}=(1;2;3)$, $\overrightarrow{b}=(2;1;-3)$ và $\overrightarrow{c}=(-1;1;5)$. Vectơ $\overrightarrow{x}=\overrightarrow{a}-4\overrightarrow{b}+2\overrightarrow{c}$ có tọa độ là
\choice
{$\overrightarrow{x}=(9;0;25)$}
{$\overrightarrow{x}=(-9;0;-25)$}
{$\overrightarrow{x}=(9;0;5)$}
{\True $\overrightarrow{x}=(-9;0;25)$}
\loigiai{
Ta có
\begin{itemize}
\item $-4\overrightarrow{b}=(-8;-4;12)$.
\item $2\overrightarrow{c}=(-2;2;10)$.
\end{itemize}
Vậy $ \overrightarrow{x}=\overrightarrow{a}-4\overrightarrow{b}+2\overrightarrow{c}=(-9;0;25) $.
}
\end{ex}


\begin{ex}%[2-H2B4-SO-12-2425 (Nguồn Đề 3 - Bài 4)]%[VN-MT-7, Vũ Hồng Toàn]%[2H2N2-3]
Trong không gian với hệ toạ độ $Oxyz$, cho hai điểm $A(0;-1;2)$ và $B(1;-2;3)$. Toạ độ của vectơ $3\overrightarrow{AB}$ là
\choice
{$(3;3;3)$}
{\True $(3;-3;3)$}
{$(-3;3;3)$}
{$(-3;-3;3)$}
\loigiai{
Ta có $\overrightarrow{AB}=(1;-1;1)$.\\
Vậy $3\overrightarrow{AB}=(3;-3;3)$.
}
\end{ex}


\begin{ex}%[2-H2B4-SO-12-2425 (Nguồn Đề 3 - Bài 4)]%[VN-MT-7, Vũ Hồng Toàn]%[2H2H2-3]
Trong KG $Oxyz$, cho hai vectơ $\overrightarrow{u}=(3;-1;1)$ và $\overrightarrow{v}=(1;2;-2)$. Độ dài của vectơ $\overrightarrow{u}+\overrightarrow{v}$ là
\choice
{$\sqrt{10}$}
{$\sqrt{11}+3$}
{\True $3\sqrt{2}$}
{$5$}
\loigiai{
Có $\overrightarrow{u}+\overrightarrow{v}=(4;1;-1)$.\\
Độ dài của vectơ $\overrightarrow{u}+\overrightarrow{v}$ là $\left|\overrightarrow{u}+\overrightarrow{v}\right|=\sqrt{4^2+1^2+(-1)^2}=3\sqrt{2}$.
}
\end{ex}


\begin{ex}%[2-H2B4-SO-12-2425 (Nguồn Đề 3 - Bài 4)]%[VN-MT-7, Vũ Hồng Toàn]%[2H2H2-4]
Trong KG $Oxyz$, cho ba điểm $A(-2; 1; 0)$, $B(0;-2; 5)$, $C(6;-2;1)$. Tích vô hướng của hai vectơ $\overrightarrow{AB}$ và $\overrightarrow{BC}$ là
\choice
{$\sqrt{38}\cdot\sqrt{52}$}
{$-\sqrt{38}\cdot\sqrt{52}$}
{$8$}
{\True $-8$}
\loigiai{
Ta có $\overrightarrow{AB}=(2;-3; 5)$ và $\overrightarrow{BC}=(6; 0;-4)$.\\
Tích vô hướng của hai vectơ $\overrightarrow{AB}$ và $\overrightarrow{BC}$ là $\overrightarrow{AB}\cdot\overrightarrow{BC}=2\cdot 6+(-3)\cdot 0+5\cdot (-4)=-8$.
}
\end{ex}


\begin{ex}%[2-H2B4-SO-12-2425 (Nguồn Đề 3 - Bài 4)]%[VN-MT-7, Vũ Hồng Toàn]%[2H2H2-2]
Trong không gian với hệ trục tọa độ $Oxyz$, cho hai điểm $A(2;1;1)$ và $B(-1;2;1)$. Tìm tọa độ $A'$ đối xứng với $A$ qua $B$.
\choice
{$A'(3;4;-3)$}
{\True $A'(-4;3;1)$}
{$A'(4;-3;3)$}
{$A'(4;33)$}
\loigiai{
Vì $A'$ đối xứng với $A$ qua $B$ nên $B$ là trung điểm của $AA'$.\\
Do đó $\heva{&x_B=\dfrac{x_A+x_{A'}}{2}\\&y_B=\dfrac{y_A+y_{A'}}{2}\\&z_B=\dfrac{z_A+z_{A'}}{2}}\Rightarrow\heva{&x_{A'}=2x_B-x_A=2\cdot(-1)-2=-4\\&y_{A'}=2y_B-y_A=2\cdot 2-1=3\\&z_{A'}=2z_B-z_A=2\cdot 1-1=1.}$\\
Vậy $A'(-4;3;1)$.
}
\end{ex}


\begin{ex}%[2-H2B4-SO-12-2425 (Nguồn Đề 3 - Bài 4)]%[VN-MT-7, Vũ Hồng Toàn]%[2H2H2-2]
Trong KG $Oxyz$, cho hai điểm $M(0;0;2)$ và $N(4;-2;6)$. Tìm tọa độ điểm $P$ sao cho $N$ là trung điểm của $MP$.
\choice
{$P(2;-1;4)$}
{$(4;-2;4)$}
{$(2;-1;2)$}
{\True $P(8;-4; 10)$}
\loigiai{
Vì $N$ là trung điểm của $MP$ nên\\
\centerline{$\heva{&x_N=\dfrac{x_M+x_P}{2}\\&y_N=\dfrac{y_M+y_P}{2}\\&z_N=\dfrac{z_M+z_P}{2}}\Rightarrow\heva{&4=\dfrac{0+x_P}{2}\\&-2=\dfrac{0+y_P}{2}\\&6=\dfrac{2+z_P}{2}}\Rightarrow\heva{&x_P=8\\&y_P=-4\\&z_P=10.}$}
Vậy $P(8;-4; 10)$.
}
\end{ex}


\begin{ex}%[2-H2B4-SO-12-2425 (Nguồn Đề 3 - Bài 4)]%[VN-MT-7, Vũ Hồng Toàn]%[2H2H2-2]
Trong KG $Oxyz$, cho tam giác $MNP$ có $M(-1;3;0)$, $N(2;2;1)$, $P(-1;1;2)$. Trọng tâm $G$ của tam giác $MNP$ có tọa độ là
\choice
{\True $(0;2;1)$}
{$(0;6;3)$}
{$(2;0;1)$}
{$(0;-2;1)$}
\loigiai{
Theo công thức tính tọa độ trong tâm của tam giác ta có\\
\centerline{$\heva{&x_G=\dfrac{-1+2-1}{3}\\&y_G=\dfrac{3+2+1}{3}\\&z_G=\dfrac{0+1+2}{3}}\Rightarrow\heva{&x_G=0\\&y_G=2\\&z_G=1.}$}
Vậy $G(0;2;1)$.
}
\end{ex}


\begin{ex}%[2-H2B4-SO-12-2425 (Nguồn Đề 3 - Bài 4)]%[VN-MT-7, Vũ Hồng Toàn]%[2H2H2-6]
Trong không gian chọn hệ trục tọa độ cho trước, đơn vị đo là kilômét, rađa phát hiện một máy bay chiến đấu của Nga di chuyển với vận tốc và hướng không đổi từ điểm $M(600;400;20)$ đến điểm $N(800;500;30)$ trong $30$ phút. Nếu máy bay tiếp tục giữ nguyên vận tốc và hướng bay thì tọa độ của máy bay sau $15$ phút tiếp theo bằng bao nhiêu?
\choice
{$(700;500;30)$}
{$(900;650;55)$}
{\True $(900;550;35)$}
{$(800;540;30)$}
\loigiai{
Gọi $Q(x;y;z)$ là tọa độ của máy bay sau $15$ phút tiếp theo.\\
Ta có $\overrightarrow{MN}=(200;100;10)$ và $\overrightarrow{NQ}=(x-800;y-500;z-30)$.\\
Vì máy bay giữ nguyên hướng bay nên $\overrightarrow{MN}$ và $\overrightarrow{NQ}$ cùng hướng.
Do máy bay tiếp tục giữ nguyên vận tốc và thời gian bay từ $M\to N$ gấp $2$ lần thời gian bay từ $N\to Q$ nên\\
\centerline{$MN=2NQ
\Rightarrow\overrightarrow{MN}=2\overrightarrow{NQ}
\Rightarrow\heva{&200=2(x-800)\\&100=2(y-500)\\&10=2(z-30)}
\Rightarrow\heva{&x=900\\&y=550\\&z=35.}
\Rightarrow Q(900;550;35)$.}
Tọa độ của máy bay sau $15$ phút tiếp theo là $(900;550;35)$ .
}
\end{ex}

\Closesolutionfile{ans}

\TNTF
\Opensolutionfile{ans}[ans/ans\currfilebase-Phan-II]

\begin{ex}%[2-H2B4-SO-12-2425 (Nguồn Đề 3 - Bài 4)]%[VN-MT-7, Vũ Hồng Toàn]%[2H2H2-4]
Cho hình lăng trụ tam giác đều $ABC.A'B'C'$ có $AB=a$ và $AA'=a\sqrt{2}$. Gọi $M$ là trung điểm $BC$.
\choiceTF
{\True $\overrightarrow{AC}=\overrightarrow{AB}+\overrightarrow{BC}$}
{\True $\overrightarrow{A'M}=\overrightarrow{A'A}+\overrightarrow{A'B'}-\overrightarrow{CM}$}
{$\overrightarrow{A'M}\cdot\overrightarrow{AC}=\dfrac{a^2\sqrt{3}}{4}$}
{\True Góc giữa vectơ $\overrightarrow{AB'}$ và $\overrightarrow{BC'}$ bằng $60^\circ$}
\loigiai{
\begin{center}
\begin{tikzpicture}[scale=1, font=\footnotesize, line join=round, line cap=round, >=stealth]
\def\a{4} \def\b{2.2} \def\h{3.5} \def\g{30}
\path
(0,0) coordinate (A)
(0:\a) coordinate (C)++
(\g-180:\b) coordinate (B)
(90:\h) coordinate (A') ++(0:\a) coordinate (C')++
(\g-180:\b) coordinate (B')
($(C)!.5!(B)$) coordinate (M);
\draw[dashed](C)--(A)--(M)--(A');
\draw (A')--(A)--(B)--(B')--cycle (B)--(C)--(C')--(B') (A')--(C');
\foreach \x /\gN in {A/160,B/-90,C/-20,M/-30,A'/90,C'/90,B'/90}
\fill(\x) circle (1pt)($(\x)+(\gN:3mm)$) node {$\x$};
\end{tikzpicture}
\end{center}
Do $ABC.A'B'C'$ là lăng trụ tam giác đều cạnh $a$ nên $\triangle ABC$ đều cạnh $a$ và $AA'\perp (ABC)$.\\
Ta có $M$ là trung điểm của $BC$ nên $AM=\dfrac{a\sqrt{3}}{2}$.
\begin{itemchoice}
\itemch \textbf{Đúng}.\\ Áp dụng quy tắc cộng ta có $\overrightarrow{AC}=\overrightarrow{AB}+\overrightarrow{BC}$.
\itemch \textbf{Đúng}.\\Ta có $\overrightarrow{A'A}+\overrightarrow{A'B'}-\overrightarrow{CM}=\overrightarrow{A'A}+\overrightarrow{AB}+\overrightarrow{BM}=\overrightarrow{A'B}+\overrightarrow{BM}=\overrightarrow{A'M}$.
\itemch \textbf{Sai}.\\Ta có
\allowdisplaybreaks
\begin{eqnarray*}
\overrightarrow{A'M}\cdot \overrightarrow{AC}&=&\left(\overrightarrow{A'A}+\overrightarrow{AM}\right)\cdot\overrightarrow{AC}\\&=&\overrightarrow{A'A}\cdot\overrightarrow{AC}+\overrightarrow{AM}\cdot\overrightarrow{AC}\\&=&\overrightarrow{AM}\cdot\overrightarrow{AC}\\&=&\dfrac{a\sqrt{3}}{2}\cdot a\cdot \cos 30^\circ\\&=&\dfrac{3a^2}{4}.
\end{eqnarray*}
\itemch \textbf{Đúng}.\\Ta có
\begin{itemize}
 \item $\triangle ABB'$ vuông tại $B$ nên $AB'=\sqrt{AB^2+BB'^2}=\sqrt{a^2+2a^2}=a\sqrt{3}$.
 \item $\triangle BCC'$ vuông tại $C$ nên $BC'=\sqrt{BC^2+CC'^2}=\sqrt{a^2+2a^2}=a\sqrt{3}$.
 \item $\overrightarrow{AB}\cdot\overrightarrow{BC}=\left|\overrightarrow{AB}\right|\cdot \left|\overrightarrow{BC}\right|\cdot \cos 120^\circ=a\cdot a\cdot\dfrac{-1}{2}=-\dfrac{a^2}{2}$.
 \item $\overrightarrow{BB'}\cdot\overrightarrow{CC'}=\overrightarrow{BB'}^2=2a^2$.
\end{itemize}
Khi đó
\allowdisplaybreaks
\begin{eqnarray*}
\overrightarrow{AB'}\cdot\overrightarrow{BC'}&=&\left(\overrightarrow{AB}+\overrightarrow{BB'}\right)\cdot\left(\overrightarrow{BC}+\overrightarrow{CC'}\right)\\ &=&\overrightarrow{AB}\cdot\overrightarrow{BC}+\overrightarrow{AB}\cdot\overrightarrow{CC'}+\overrightarrow{BB'}\cdot\overrightarrow{BC}+\overrightarrow{BB'}\cdot\overrightarrow{CC'}\\
&=&-\dfrac{a^2}{2}+0+0+2a^2\\&=&\dfrac{3a^2}{2}.
\end{eqnarray*}
Suy ra $\cos \left(\overrightarrow{AB'},\overrightarrow{BC'}\right)=\dfrac{\overrightarrow{AB'}\cdot\overrightarrow{BC'}}{\left|\overrightarrow{AB'}\right|\cdot\left|\overrightarrow{BC'}\right|}=\dfrac{\dfrac{3a^2}{2}}{a\sqrt{3}\cdot a\sqrt{3}}=\dfrac{1}{2} \Rightarrow \left(\overrightarrow{AB'}, \overrightarrow{BC'}\right)=60^\circ$.
\end{itemchoice}
}
\end{ex}


\begin{ex}%[2-H2B4-SO-12-2425 (Nguồn Đề 3 - Bài 4)]%[VN-MT-7, Vũ Hồng Toàn]%[2H2H2-3]
Trong KG $Oxyz$, cho tam giác $ABC$ có các đỉnh $A(1;-2;0)$, $B(2;1;-2)$, $C(0;3;4)$. 
\choiceTF
{\True Tọa độ của vectơ $\overrightarrow{AB}$ là $(1;3;-2)$}
{\True Tọa độ trọng tâm của tam giác $ABC$ là $G\left(1;\dfrac{2}{3};\dfrac{2}{3} \right)$}
{Tọa độ hình chiếu của điểm $B$ trên mặt phẳng $(Oxy)$ là $H(0;0;-2)$}
{$\overrightarrow{x}=2\overrightarrow{AB}-3\overrightarrow{BC}$. Tọa độ của vectơ $\overrightarrow{x}=(-4;12;14)$}
\loigiai{
\begin{itemchoice}
\itemch \textbf{Đúng}.\\ Ta có $\overrightarrow{AB}=(1;3;-2)$.
\itemch \textbf{Đúng}.\\ Ta có $\heva{&x_G=\dfrac{x_A+x_B+x_C}{3}=1\\&y_G=\dfrac{y_A+y_B+y_C}{3}=\dfrac{2}{3}\\&z_G=\dfrac{z_A+z_B+z_C}{3}=\dfrac{2}{3}}\Rightarrow G\left(1;\dfrac{2}{3};\dfrac{2}{3} \right)$.
\itemch \textbf{Sai}.\\Tọa độ hình chiếu của điểm $B(2;1;-2)$ trên mặt phẳng $(Oxy)$ là $H(2;1;0)$.
\itemch \textbf{Sai}.\\Ta có
\begin{itemize}
\item $\overrightarrow{AB}=(1;3;-2)\Rightarrow 2\overrightarrow{AB}=(2;6;-4)$.
\item $\overrightarrow{BC}=(-2;2;6)\Rightarrow-3\overrightarrow{BC}=(6;-6;-18)$.
\end{itemize}
Vậy $\overrightarrow{x}=2\overrightarrow{AB}-3\overrightarrow{BC}=(8;0;-22)$.
\end{itemchoice}
}
\end{ex}


\begin{ex}%[2-H2B4-SO-12-2425 (Nguồn Đề 3 - Bài 4)]%[VN-MT-7, Vũ Hồng Toàn]%[2H2V2-5]
Trong không gian với hệ trục tọa độ $Oxyz$, cho bốn điểm $A(0;-1; 1)$, $B(-2; 1;-1)$, $C(-1; 3; 2)$, $D(-1; 0; 0)$. 
\choiceTF
{\True Ba điểm $A$, $B$, $C$ không thẳng hàng}
{\True Ba điểm $A$, $B$, $D$ thẳng hàng}
{Côsin của góc giữa $\overrightarrow{AB}$ và $\overrightarrow{CB}$ bằng $-\dfrac{\sqrt{42}}{21}$}
{Bốn điểm $A$, $B$, $C$, $D$ không đồng phẳng}
\loigiai{
\begin{itemchoice}
\itemch \textbf{Đúng}.\\Ta có $\overrightarrow{AB}=(-2; 2;-2)$, $ \overrightarrow{BC}=(1; 2; 3)$.\\
Giả sử tồn tại số $k\ne 0$ sao cho $\overrightarrow{AB}=k\overrightarrow{BC}\Rightarrow \heva{&-2=k\\&2=2k\\&-2=3k.}
$\\Hệ vô nghiệm suy ra không tồn tại $k$.
Vậy ba điểm $A$, $B$, $C$ không thẳng hàng.
\itemch \textbf{Đúng}.\\ Ta có $\overrightarrow{AB}=(-2; 2;-2)$, $ \overrightarrow{BD}=(1;-1; 1)$.\\
Vì $\overrightarrow{AB}=-2\overrightarrow{BD}$.
Suy ra điểm $A$, $B$, $D$ thẳng hàng
\itemch \textbf{Sai}.\\Ta có $\overrightarrow{AB}=(-2; 2;-2)$, $\overrightarrow{CB}=(-1;-2;-3)$.\\
Khi đó $\cos\left (\overrightarrow{AB},\overrightarrow{CB}\right)=\dfrac{\overrightarrow{AB}\cdot\overrightarrow{CB}}{\left|\overrightarrow{AB}\right|\cdot\left|\overrightarrow{CB}\right|} =\dfrac{(-2)\cdot(-1)+2\cdot(-2)+(-2)\cdot(-3)}{\sqrt{12}\cdot\sqrt{14}}=\dfrac{\sqrt{42}}{21}$.
\itemch \textbf{Sai}.\\Ta có $\overrightarrow{AB}=(-2; 2;-2)$, $ \overrightarrow{BD}=(1;-1; 1)$.\\
Vì $\overrightarrow{AB}=-2\overrightarrow{BD}$.
Suy ra điểm $A$, $B$, $D$ thẳng hàng.\\
Khi đó luôn tồn tại một mặt phẳng qua $C$ và chứa đường thẳng đi qua ba điểm $A$, $B$, $D$.\\
Vậy bốn điểm $A$, $B$, $C$, $D$ đồng phẳng.
\end{itemchoice}
}
\end{ex}


\begin{ex}%[2-H2B4-SO-12-2425 (Nguồn Đề 3 - Bài 4)]%[VN-MT-7, Vũ Hồng Toàn]%[2H2V2-5]
Trong không gian với hệ trục tọa độ $Oxyz$, cho ba điểm $A(-1; 2; 1)$; $B(2;-2;4)$; $C(0;-4;1)$. 
\choiceTF
{\True Ba điểm $A$, $B$, $C$ không thẳng hàng}
{\True Biết điểm $D(5;-6;7)$. Khi đó ba điểm $A$, $B$, $D$ thẳng hàng}
{$\cos\left(\overrightarrow{AB},\overrightarrow{AC}\right)=\dfrac{37}{\sqrt{1258}}$}
{Cho $\overrightarrow{u}=(x-1;2y+1;3z-5)$ thoả mãn $\overrightarrow{u}\perp \overrightarrow{AB}$ và $\overrightarrow{u}\perp \overrightarrow{AC}$. Khi đó $x^2+y^2+z^2=2024$}
\loigiai{
\begin{itemchoice}
\itemch \textbf{Đúng}.\\Ta có $\overrightarrow{AB}=(3;-4;3)$, $\overrightarrow{AC}=(1;-6;0)$.\\ Giả sử tồn tại số $k\ne 0$ sao cho $\overrightarrow{AB}=k\overrightarrow{AC}\Rightarrow \heva{&3=k\\&-4=-6k\\&3=0k.}
$\\
Hệ vô nghiệm suy ra không tồn tại $k$. Vậy ba điểm $A$, $B$, $C$ không thẳng hàng.
\itemch \textbf{Đúng}.\\Ta có $\overrightarrow{AB}=(3;-4;3)$, $\overrightarrow{AD}=(6;-8;6)\Rightarrow \overrightarrow{AD}=2\overrightarrow{AB}$.\\ 
Vậy ba điểm $A$, $B$, $D$ thẳng hàng.
\itemch \textbf{Sai}.\\
Ta có $\cos\left(\overrightarrow{AB},\overrightarrow{AC}\right)=\dfrac{\overrightarrow{AB}\cdot\overrightarrow{AC}}{\left|\overrightarrow{AB}\right|\cdot\left|\overrightarrow{AC}\right|}=\dfrac{3\cdot 1+(-4)\cdot (-6)+3\cdot 0}{\sqrt{9+16+9}\cdot\sqrt{1+36+0}}=\dfrac{27}{\sqrt{1\,258}}$.
\itemch \textbf{Sai}.\\Ta có $\overrightarrow{u}\perp \overrightarrow{AB}$ và $\overrightarrow{u}\perp \overrightarrow{AC}$ suy ra $\overrightarrow{u}$ cùng phương với $\left[\overrightarrow{AB},\overrightarrow{AC}\right]=(18;3;-14)$.\\
Xét trường hợp $\overrightarrow{u}=(18;3;-14)$ ta có\\
\centerline{$\heva{&x-1=18\\&2y+1=3\\&3z-5=-14}\Rightarrow\heva{&x=19\\&y=1\\&z=-3.}$}\\
Vậy $x^2+y^2+z^2=19^2+1+9=371$.
\end{itemchoice}
}
\end{ex}

\Closesolutionfile{ans}

\TNSA
\Opensolutionfile{ans}[ans/ans\currfilebase-Phan-III]


\begin{ex}%[2-H2B4-SO-12-2425 (Nguồn Đề 3 - Bài 4)]%[VN-MT-7, Vũ Hồng Toàn]%[2H2V2-4]
Cho hình lập phương $ABCD.A'B'C'D'$ có cạnh là $a$. Gọi $G$ là trọng tâm tam giác $B'C'D'$, $I$ là trung điểm của $AB'$. Tính $\cos\left(\overrightarrow{A'D}, \overrightarrow{IG}\right)$ (làm tròn kết quả đến hàng phần trăm).
\shortans{0{,}14}
\loigiai{
\begin{center}
\begin{tikzpicture}[scale=1, font=\footnotesize, line join=round, line cap=round, >=stealth]
\def\a{4} \def\b{2.5} \def\h{3.5} \def\g{30}
\path
(0,0) coordinate (A)
(0:\a) coordinate (D)++
(\g-180:\b) coordinate (C)++(180:\a) coordinate (B)
(90:\h) coordinate (A') ++(0:\a) coordinate (D')++
(\g-180:\b) coordinate (C')++(180:\a) coordinate (B')
($(A')!.5!(C')$) coordinate (M)
($(C')!2/3!(M)$) coordinate (G)
($(A)!.5!(B')$) coordinate (I)
;
\draw[dashed](B)--(A)--(D) (A')--(A)--(B') (I)--(G) (A')--(D);
\draw (A')--(B')--(C')--(D') --cycle (B')--(B)--(C)--(C')--(A') (C)--(D)--(D')--(B') ;
\foreach \x /\gN in {A/60,B/180,C/-20,G/0,A'/90,D'/0,C'/-40,B'/180,D/0,I/180}
\fill(\x) circle (1pt)($(\x)+(\gN:3mm)$) node {$\x$};
\end{tikzpicture}
\end{center}
Ta có cạnh hình lập phương là $a\Rightarrow A'D=a\sqrt{2}$ và $\overrightarrow{A'D}=\overrightarrow{AD}-\overrightarrow{AA'}$.
\allowdisplaybreaks
\begin{eqnarray*}
\overrightarrow{IG}&=&\overrightarrow{IB'}+\overrightarrow{B'G}\\&=&\dfrac{1}{2} \left(\overrightarrow{AA'}+\overrightarrow{AB}\right)+\dfrac{1}{3} \left(\overrightarrow{B'D'}+\overrightarrow{B'C'}\right)\\&=&\dfrac{1}{2} \left(\overrightarrow{AA'}+\overrightarrow{AB}\right)+\dfrac{1}{3} \left(\overrightarrow{BD}+\overrightarrow{AD}\right) \\ &=&\dfrac{1}{2} \left(\overrightarrow{AA'}+\overrightarrow{AB}\right)+\dfrac{1}{3} \left(2\overrightarrow{AD}-\overrightarrow{AB}\right)\\&=&\dfrac{1}{2} \overrightarrow{AA'}+\dfrac{1}{6} \overrightarrow{AB}+\dfrac{2}{3} \overrightarrow{AD} \\
\Rightarrow{\overrightarrow{IG}}^2&=&\left(\dfrac{1}{2} \overrightarrow{AA'}+\dfrac{1}{6} \overrightarrow{AB}+\dfrac{2}{3} \overrightarrow{AD}\right)^2=\dfrac{13a^2}{18}\\
\Rightarrow IG&=&\dfrac{a\sqrt{26}}{6}.
\end{eqnarray*}
Mà\\
\centerline{$\overrightarrow{A'D}\cdot\overrightarrow{IG}=\left(\dfrac{1}{2} \overrightarrow{AA'}+\dfrac{1}{6} \overrightarrow{AB}+\dfrac{2}{3} \overrightarrow{AD}\right)\cdot\left(\overrightarrow{AD}-\overrightarrow{AA'}\right)=\dfrac{a^2}{6} $.}
Vậy\\
\centerline{$ \cos\left(\overrightarrow{A'D}, \overrightarrow{IG}\right)=\dfrac{\overrightarrow{A'D}\cdot\overrightarrow{IG}}{A'D\cdot IG}=\dfrac{\dfrac{a^2}{6}}{\dfrac{a\sqrt{26}}{6}\cdot a\sqrt{2}}=\dfrac{\sqrt{13}}{26}\approx 0{,}14 $.}
}
\end{ex}


\begin{ex}%[2-H2B4-SO-12-2425 (Nguồn Đề 3 - Bài 4)]%[VN-MT-7, Vũ Hồng Toàn]%[2H2H2-2]
Trong KG $Oxyz$, cho hình hộp $ABCD.A'B'C'D'$. Biết $A(2; 4; 0)$, $B(4; 0; 0)$, $C(-1; 4;-7)$ và $D'(6; 8; 10)$. Tọa độ đỉnh $B'$ của hình hộp có dạng $B'(a;b;c)$. Tính $a+b+c$.
\shortans{30}
\loigiai{
\begin{center}
\begin{tikzpicture}[scale=1, font=\footnotesize, line join=round, line cap=round, >=stealth]
\def\a{3} \def\b{2} \def\h{3} \def\g{30}
\path
(0,0) coordinate (A)
(0:\a) coordinate (D)++
(\g-180:\b) coordinate (C)++(180:\a) coordinate (B)
(80:\h) coordinate (A') ++(0:\a) coordinate (D')++
(\g-180:\b) coordinate (C')++(180:\a) coordinate (B')
;
\draw[dashed](B)--(A)--(D) (A')--(A);
\draw (A')--(B')--(C')--(D') --cycle (B')--(B)--(C)--(C') (C)--(D)--(D') ;
\foreach \x /\gN in {A/60,B/180,C/-20,A'/180,D'/0,C'/-40,B'/180,D/0}
\fill(\x) circle (1pt)($(\x)+(\gN:3mm)$) node {$\x$};
\end{tikzpicture}
\end{center}
Ta có $\overrightarrow{BC}=(-5; 4;-7)$. Gọi $D(x; y; z)\Rightarrow \overrightarrow{AD}=(x-2; y-4; z)$.\\
Vì $ABCD$ là hình bình hành nên\\
\centerline{$\overrightarrow{AD}=\overrightarrow{BC}\Rightarrow\heva{&x-2=-5\\&y-4=4\\&z=-7}\Rightarrow\heva{&x=-3\\&y=8\\&z=-7}\Rightarrow D(-3;8;-7)$.}
Ta có $\overrightarrow{DD'}=(9; 0; 17)$ và $\overrightarrow{BB'}=(a-4; b; c)$.\\
Vì $BB'D'D$ là hình bình hành nên\\
\centerline{$\overrightarrow{BB'}=\overrightarrow{DD'}\Rightarrow \heva{&a-4=9\\&b=0\\&c=17}\Rightarrow\heva{&a=13\\&b=0\\&c=17}\Rightarrow B'(13; 0; 17)$.}
Vậy $a+b+c=13+0+17=30$.
}
\end{ex}


\begin{ex}%[2-H2B4-SO-12-2425 (Nguồn Đề 3 - Bài 4)]%[VN-MT-7, Vũ Hồng Toàn]%[2H2H2-4]
Trong KG $Oxyz$, cho ba điểm $A(1;6;2)$, $B(5; 1; 3)$ và $C(4; 0; 6)$. Biết $\overrightarrow{u}=(14;a;b)$ vuông góc với với cả hai vectơ $\overrightarrow{AB}$ và $\overrightarrow{AC}$. Tính $a-b$.
\shortans{4}
\loigiai{
Ta có $\overrightarrow{AB}=(4;-5; 1)$ và $\overrightarrow{AC}=(3;-6; 4)$.\\
Vì $\overrightarrow{u}$ vuông góc với cả hai vectơ $\overrightarrow{AB}$ và $\overrightarrow{AC}$ nên $\overrightarrow{u}$ cùng phương với vectơ \linebreak $\left[\overrightarrow{AB}, \overrightarrow{AC}\right]=(-14;-13;-9)$.
Do đó tồn tại số thực $k$ để $\overrightarrow{u}=k\left[\overrightarrow{AB}, \overrightarrow{AC}\right]$.\\
Khi đó ta có $k=-1$, $\overrightarrow{u}=(14;13;9)$.\\
Suy ra $a=13$ và $b=9$. Vậy $a-b=13-9=4$.
}
\end{ex}


\begin{ex}%[2-H2B4-SO-12-2425 (Nguồn Đề 3 - Bài 4)]%[VN-MT-7, Vũ Hồng Toàn]%[2H2V2-6]
Hình minh họa sơ đồ ngôi nhà Trong KG $Oxyz$, trong đó nền nhà, bốn bức tường và hai mái nhà đều là hình chữ nhật. Biết tọa độ của vectơ $\overrightarrow{AH}=(a;b;c)$. Tìm $a+b+c$.
\begin{center}
 \begin{tikzpicture}[font=\footnotesize, line join=round, line cap=round, >=stealth, scale=1.2]
 \def\a{3}
 \def\b{5}
 \def\h{3}
 \path (0:0) coordinate (C)
 ++(0:\a) coordinate (B)
 ++(-160:\b) coordinate (O)
 ($(O)+(B)-(C)$) coordinate (A)
 ($(O)+(90:\h)$) coordinate (E)
 ($(B)+(90:\h)$) coordinate (G)
 ($(C)+(90:\h)$) coordinate (H)
 ($(A)+(90:\h)$) coordinate (F)
 ($(A)+(0:1)$) coordinate (x)
 ($(H)+(35:2)$) coordinate (Q)
 ($(E)+(35:2)$) coordinate (P)
 ($(E)+(90:1)$) coordinate (z)
 ($(O)!1.3!(C)$) coordinate (y);
 \draw[dashed] (G)--(H)--(C)--(B) (C)--(O);
 \draw[] (G)--(Q)--(H)--(E)--(F)--(G)--(B)--(A)--(O)--(E) (F)--(A) (F)--(P)--(E) (P)--(Q);
 \draw [->] (A)--(x);
 \draw [->] (E)--(z);
 \draw [->,dashed] (C)--(y);
 \draw [] (Q)node[above]{$Q(2; 5; 4)$} (G)node[right]{$G(4; 5; 3)$} (B)node[right]{$B(4; 5; 0)$} (P)node[right]{$P(2; 0; 4)$} (O)node[below]{$O(0; 0; 0)$} (E)node[left]{$E(0; 0; 3)$} (x)node[below]{$x$} (y)node[above]{$y$} (z)node[left]{$z$};
 \foreach \x/\g in {A/-90,C/180,F/0,H/90}
 \fill[black] (\x) circle (1pt)
 ($(\g:4mm)+(\x)$) node {$\x$}; 
 \end{tikzpicture}
\end{center}
\shortans{4}
\loigiai{
Vì nền nhà là hình chữ nhật nên $OABC$ là hình chữ nhật, suy ra $x_A=x_B=4$, $y_C=y_B=5$.\\
Do điểm $A$ nằm trên trục $Ox$ nên tọa độ điểm $A(4;0;0)$; điểm $C$ nằm trên trục $Oy$ nên tọa độ điểm $C(0;5;0)$.\\
Tường nhà là hình chữ nhật nên $OCHE$ là hình chữ nhật, suy ra $y_H=y_C=5$.\\
Do $H$ nằm trên mặt phẳng $(Oyz)$ nên tọa độ điểm $H(0;5;3)$.\\
Khi đó $\overrightarrow{AH}=(0-4;5-0;3-0)\Rightarrow \overrightarrow{AH}=(-4;5;3)$.
Suy ra $a=-4$, $b=5$, $c=3$.\\
Vậy $a+b+c=-4+5+3=4$.
}
\end{ex}


\begin{ex}%[2-H2B4-SO-12-2425 (Nguồn Đề 3 - Bài 4)]%[VN-MT-7, Vũ Hồng Toàn]%[2H2V2-6]
 \immini{
 Một chiếc ô tô được đặt trên mặt đáy dưới một khung sắt có dạng hình hộp chữ nhật với đáy trên là hình chữ nhật $ABCD$, mặt phẳng $(ABCD)$ song song với mặt mặt phẳng nằm ngang. Khung sắt đó được buộc vào móc $E$ của chiếc cần cẩu sao cho các đoạn dây cáp $EA$, $EB$, $EC$, $ED$ có độ dài bằng nhau và cùng tạo với mặt phẳng $(ABCD)$ một góc $60^\circ$ như hình vẽ. Chiếc cần cẩu kéo khung sắt lên theo phương thẳng đứng. Biết lực căng $\overrightarrow{F_1}$, $\overrightarrow{F_2}$, $\overrightarrow{F_3}$, $\overrightarrow{F_4}$ đều có cường độ $5\, 000$ N và trọng lượng khung sắt là $2\, 000$ N. Biết trọng lượng của chiếc xe ô tô bằng $m\times 9{,}81$ N. Giá trị của $m$ làm tròn đến hàng đơn vị bằng bao nhiêu?
 }
 {
 \begin{tikzpicture}[scale=0.65, font=\footnotesize, line join=round, line cap=round, >=stealth,transform shape]
 \definecolor{bostonuniversityred}{rgb}{0.8, 0.0, 0.0}
 \definecolor{charcoal}{rgb}{0.21, 0.27, 0.31}
 \definecolor{bananayellow}{rgb}{1.0, 0.88, 0.21}
 \definecolor{anti-flashwhite}{rgb}{0.95, 0.95, 0.96}
 % \clip (-6,-3) rectangle (6,3);
 \tikzset{%
 xeoto/.pic={%
 %--------------------------
 \tikzset{xe/.pic={
 \def\N{
 (-2.7,.56)--(-2.5,.56)
 ..controls +(50:1.5) and +(165:1.5) .. (2.1,1.88)--(2.05,2)
 ..controls +(-10:.1) and +(130:.1) .. (3.25,1.75)--(3.15,1.65)
 ..controls +(-4:.2) and +(130:.15) .. (4.05,.7)--(4.25,.75)
 ..controls +(-40:.2) and +(130:.15) .. (4.55,.35)--(4.35,.26)
 ..controls +(-40:.2) and +(130:.15) .. (4.8,-.45)--(4.92,-.4)
 ..controls +(-40:.25) and +(73:.17) .. (4.8,-1.8)--(-4.4,-1.8)
 ..controls +(175:.7) and +(-175:3.2) ..cycle
 ;
 }
 \fill[bostonuniversityred] \N;
 \draw \N;
 \def\K{
 (-2.2,.56)--(3.3,.7)
 ..controls +(100:1.18) and +(43:3) .. cycle
 ;
 }
 \fill[bottom color=charcoal,top color=charcoal!20!white, middle color=charcoal!80!white] \K;
 \draw \K;
 \def\K1{
 (-2.2,.56)
 ..controls +(43:.2) and +(43:.2) .. (-1.58,1.05)--(-1.53,.57)--cycle
 ;
 }
 \draw \K1;
 \fill[charcoal] \K1;
 \def\K2{
 (1.2,1.85)
 ..controls +(-10:.1) and +(160:.1) .. (1.58,1.8)--(1.8,.65)--(1.25,.65)--cycle
 ;
 }
 \draw \K2;
 \fill[charcoal] \K2;
 \def\Kt{
 (-2.5,.56)
 ..controls +(50:1.5) and +(165:1.5) .. (2.1,1.88)--(2.05,2)
 ..controls +(170:2.2) and +(45:1.5) .. (-2.7,.56)--cycle
 ;
 }
 \fill[charcoal!50] \Kt;
 \draw \Kt;
 \def\Ks{
 (3.25,1.75)--(3.15,1.65)
 ..controls +(-4:.2) and +(130:.15) .. (4.05,.7)--(4.22,.75)
 ..controls +(120:.3) and +(-35:.3) .. cycle
 ;
 }
 \fill[charcoal!50] \Ks;
 \draw \Ks;
 %Đèn sau
 \def\D{
 (4.55,.35)--(4.35,.26)
 ..controls +(-40:.2) and +(130:.15) .. (4.8,-.45)--(4.92,-.4)
 ..controls +(110:.2) and +(-40:.15) ..cycle
 ;
 }
 \fill[bananayellow] \D;
 \draw \D;
 \def\M{
 (2.2,-1.3)--(-1.8,-1.4)--(-1.78,-1.7)
 ..controls +(-5:.3) and +(-90:.6) ..cycle
 ;
 }
 \draw \M;
 \fill[charcoal!90] \M;
 \draw (-1.6,.55)
 ..controls +(-170:.5) and +(95:.4) .. (-1.78,-1.7)
 (1.6,.65)
 ..controls +(-30:.5) and +(35:.3) .. (1.7,-1.3)
 ;
 %gương
 \def\G{
 (-1.5,.45)--(-1.4,.6)
 ..controls +(85:1) and +(20:.6) .. (-1.25,.5)--(-1.4,.33)
 ;
 }
 \draw \G;
 \fill[bostonuniversityred] \G;
 %Đèn trước
 \def\Dt{
 (-4.85,-.7)
 ..controls +(75:1) and +(65:.8) .. (-4.5,-.7)
 ..controls +(-115:.6) and +(-105:.4) .. cycle
 ;
 }
 \fill[bananayellow] \Dt;
 \draw \Dt;
 \def\Dt2{
 (-4.85,-.7)
 ..controls +(75:.6) and +(65:.4) .. (-4.7,-.7)
 ..controls +(-115:.3) and +(-105:.2) .. cycle
 ;
 }
 \fill[anti-flashwhite] \Dt2;
 \draw \Dt2;
 \draw[fill=anti-flashwhite] (-4.86,-1.45)--(-4.82,-1.5)--(-4.55,-1.3)
 ..controls +(90:.3) and +(45:.2) .. cycle;
 }}
 \tikzset{banh_xe/.pic={
 \draw[fill=charcoal] (-3.25,-1.65) circle (1) ;
 \draw[fill=anti-flashwhite] (-3.25,-1.65) circle (.7) ;
 \draw[fill=charcoal] (-3.25,-1.65) circle (.4) ;
 }}
 %----------------
 \path
 (0,0)pic[scale=1]{xe}(0,0)pic[scale=1]{banh_xe}(6.9,0)pic[scale=1]{banh_xe};
 %--------------------------------
 }}
 %%%%%%%%%%%%%%%%%%%
 \def\bc{4.25} % cạnh BC
 \def\ba{2} % cạnh BA
 \def\h{3.5} % đường cao
 \def\gocnghieng{90} % góc nghiêng
 \def\gocB{160} % góc B của đáy
 \coordinate (B1) at (0,0);
 \coordinate (A1) at (\gocB:\ba);
 \coordinate (C1) at (\bc,0.25);
 \coordinate (D1) at ($(C1)-(B1)+(A1)$);
 \coordinate[label=above left:$A$] (A) at ($(A1)+(\gocnghieng:\h)$);
 \coordinate[label=below left:$B$] (B) at ($(B1)-(A1)+(A)$);
 \coordinate[label=right:$C$] (C) at ($(C1)-(A1)+(A)$);
 \coordinate[label=above right:$D$] (D) at ($(D1)-(A1)+(A)$);
 \coordinate (E) at ($(A)!0.5!(C)+(\gocnghieng:\h)$);
 %------------
 \draw[->,blue,very thick] (E)--($(E)!0.65!(A)$) node[above left]{$\overrightarrow{F_1}$};
 \draw[->,blue,very thick] (E)--($(E)!0.65!(B)$) node[right]{$\overrightarrow{F_2}$};
 \draw[->,blue,very thick] (E)--($(E)!0.65!(C)$) node[above right]{$\overrightarrow{F_3}$};
 \draw[->,blue,very thick] (E)--($(E)!0.65!(D)$) node[left=2pt]{$\overrightarrow{F_4}$};
 %------------
 \path (E) node[left=1mm]{$E$};
 \draw[blue,very thick] (A)--(B)--(C)--(D)--cycle
 (A1)--(A) (D1)--(D) (C1)--(C)
 (A)--(E)--(B) (C)--(E)--(D);
 \draw[fill=teal] (A1)--(B1)--(C1)--(D1)--cycle;
 \draw[fill=teal!30] (A1)--(B1)--(C1)--++(0,-0.3)--([yshift=-0.3cm]B1)--([yshift=-0.3cm]A1)--cycle;
 \foreach \diem in {A1,B1,C1,D1,A,B,C,D,E} \fill (\diem)circle(1.5pt);
 %phần móc và dây
 \def\r{0.3}\def\rr{0.25}
 \coordinate (tam) at ([yshift=6mm]E);
 \draw[brown,fill=brown,line width=1pt] (tam) circle (\r cm);
 \fill (tam) circle (2pt);
 \draw[brown,line width=1pt] (tam)++(\r,0)--++(0,0.7)
 (tam)++(-\r,0)--++(0,0.7);
 \draw[line width=1.5pt] (tam)--++(0,-1.35*\r) arc(90:370:1mm);
 %%%%%%%%%%%%%%%%%%%
 \pic[scale=0.45,rotate=4] at (1.6,1.3) [pic type = xeoto];
 %--------
 \draw[blue,very thick] (B)--(B1);
 \end{tikzpicture}
 }
 \shortans{1562}
 \loigiai{
 \begin{center}
 \begin{tikzpicture}[scale=1, font=\footnotesize, line join=round, line cap=round, >=stealth]
 \def\a{4} \def\b{2.2} \def\h{4.5} \def\g{40}
 \path
 (0,0) coordinate (A)
 (0:\a) coordinate (D)++
 (-\g:\b) coordinate (C)++(180:\a) coordinate (B)
 ($(A)!.5!(C)$) coordinate (O)++ (90:\h) coordinate (E)
 ;
 \draw[dashed] (E)--(D)--(A)--(C)--(D)--(B) (E)--(O); 
 \draw (E)--(A)--(B)--(C)--(E)--(B);
 \foreach \i/\j [count =\k from 1] in {A/180,B/210,C/-10}{\draw[->] (E)-- ($(E)!0.65!(\i)$) coordinate(\i') node[midway, shift={(\j:4mm)}]{$\overrightarrow{F}_\k$};}
 \draw[->, dashed] (E)-- ($(E)!0.65!(D)$) coordinate(D') node[midway, shift={(250:4mm)}]{$\overrightarrow{F}_4$};
 \draw (A')--(B')--(C');
 \draw[dashed] (B')--(D')--(A')--(C')--(D');
 \draw[->, dashed] (E)--($(E)!0.65!(O)$) coordinate(O');
 \foreach \x /\gN in {A/160,B/-90,C/-20,E/90,O/-90,D/30, A'/160, B'/220, C'/10, D'/40, O'/-55}
 \fill(\x) circle (1pt)($(\x)+(\gN:3mm)$) node {$\x$};
 \end{tikzpicture}
 \end{center}
 Gọi $O$ là hình chiếu vuông góc của $E$ trên $(ABCD)$. Ta có $EA=EB=EC=ED$ nên các tam giác vuông $EOA$, $EOB$, $EOC$, $EOD$ bằng nhau. Suy ra $OA=OB=OC=OD$ hay $O$ là tâm hình chữ nhật $ABCD$.\\ 
 Gọi $A'$, $B'$, $C'$, $D'$ lần lượt là các điểm sao cho $\overrightarrow{EA'}=\overrightarrow{F}_1$, $\overrightarrow{EB'}=\overrightarrow{F}_2$, $\overrightarrow{EC'}=\overrightarrow{F}_3$ và $\overrightarrow{ED'}=\overrightarrow{F}_4$.\\
 Vì $\left|\overrightarrow{F}_1\right| = \left|\overrightarrow{F}_2\right| = \left|\overrightarrow{F}_3\right| = \left|\overrightarrow{F}_4\right| = 5\,000$ N nên $EA'=EB'=EC'=ED'=5\,000$. Do đó $E.A'B'C'D'$ là hình chóp có đáy $A'B'C'D'$ là hình chữ nhật.\\
 Gọi $O'$ là tâm hình chữ nhật $A'B'C'D'$, ta có $O'$ thuộc $EO$.\\ 
 Theo quy tắc hình bình hành $\overrightarrow{F_1}+\overrightarrow{F_3}=2\overrightarrow{EO'}$; $\overrightarrow{F_2}+\overrightarrow{F_4}=2\overrightarrow{EO'}$.\\
 Khi đó
 $\overrightarrow{F_1}+\overrightarrow{F_3}+\overrightarrow{F_2}+\overrightarrow{F_4}=4\overrightarrow{EO'}$.\\
 Các dây cáp $EA$, $EB$, $EC$, $ED$ có độ dài bằng nhau và cùng tạo với mặt phẳng $(ABCD)$ một góc $60^\circ$ nên $\widehat{EA'O'}=60^\circ$. Do đó\\
 \[EO'=EA'\cdot\sin 60^\circ=5\, 000\cdot\dfrac{\sqrt{3}}{2}=2\, 500\sqrt{3}.\]
 Gọi trọng lực của xe và khung sắt là $\overrightarrow{P}$. Vì chiếc xe ô tô và khung sắt ở vị trí cân bằng nên 
 \[\overrightarrow{P}=\overrightarrow{F_1}+\overrightarrow{F_2}+\overrightarrow{F_3}+\overrightarrow{F_4} = 4\overrightarrow{EO'}.\]
 Suy ra trọng lượng của xe và khung sắt là $\left|\overrightarrow{P}\right| = 4\left|\overrightarrow{EO'}\right| = 4\cdot 2\,500\cdot \sqrt{3} = 10\,000\sqrt{3}$ N.\\
 Vì khung sắt có trọng lượng bằng $2\,000$ N nên trọng lượng của xe ô tô là $10\,000\sqrt{3}-2\,000$ N.\\
 Vậy $m=\dfrac{10\,000\sqrt{3}-2\,000}{9{,}81}\approx 1\,562$.
 }
\end{ex}


\begin{ex}%[2-H2B4-SO-12-2425 (Nguồn Đề 3 - Bài 4)]%[VN-MT-7, Vũ Hồng Toàn]%[2H2V2-6]
 \immini{
 Một vật nặng có trọng lượng là $400$ N được đặt trên một khung sắt hình tròn như hình bên. Biết $ABCD$ là hình chữ nhật, mặt phẳng $(ABCD)$ song song với mặt phẳng nằm ngang. Khung sắt được móc vào điểm $S$ sao cho các đoạn dây cáp $SA$, $ SB$, $SC$, $SD$ có độ dài bằng nhau và cùng tạo với mặt phẳng $(ABCD)$ một góc bằng $45^\circ$. Chiếc cần cẩu kéo khung sắt lên theo phương thẳng đứng. Biết trọng lượng của khung sắt là $200$ N; cường độ các lực căng $\overrightarrow{F}_1$, $\overrightarrow{F}_2$, $\overrightarrow{F}_3$, $\overrightarrow{F}_4$ là bằng nhau. Tính cường độ của lực căng $\overrightarrow{F}_1$ (làm tròn đến hàng đơn vị).
 }
 {
 \begin{tikzpicture}[scale=.7,>=stealth, font=\footnotesize, line join=round, line cap=round]
 \tikzset{day/.pic=
 {\draw[shade,bottom color=brown!30,top color= white!30,rounded corners=0.5ex,line width=1.5pt,gray!80]
 (0,0) ellipse ({2.5pt} and {8pt});}
 }
 \def\h{6}
 \def\a{3}
 \def\b{1.5}
 \path
 (0,0) coordinate (O)
 ($(O)+(0,\h)$) coordinate (S)
 ($(O)+(10:\a cm and \b cm)$)coordinate (M)
 ($(O)+(180:\a cm and .8*\b cm)$)coordinate (A)
 ($(O)+(0:\a cm and .8*\b cm)$)coordinate (C)
 ($(O)+(60:\a cm and .8*\b cm)$)coordinate (B)
 ($(O)+(-120:\a cm and .8*\b cm)$)coordinate (D)
 ;
 
 \draw[fill=brown] (M) arc (10:-190:\a cm and \b cm);
 \draw[fill=white] (M) arc (10:-190:\a cm and .8*\b cm);
 \draw [fill=white] (M) arc (10:190:\a cm and .8*\b cm);
 \draw[fill,bottom color=black!30,top color= brown!70, ,left color=black!50] ($(S)-(.5,.3)$) rectangle ($(S)+(.5,.3)$);
 \foreach \m in {0,1,2,...,12}{\pic[rotate=-20] at ($(A)+(.25*\m,.5*\m)$) {day};}
 \foreach \m in {0,1,2,...,14}{\pic[rotate=-15] at ($(D)+(.11*\m,.5*\m)$) {day};}
 \foreach \m in {0,1,2,...,10}{\pic[rotate=5] at ($(B)+(-.15*\m,.5*\m)$) {day};}
 \foreach \m in {0,1,2,...,12}{\pic[rotate=18] at ($(C)+(-.25*\m,.5*\m)$) {day};}
 \foreach \x/\y in {A/180,B/150,C/-45,D/-90,S/90}
 \fill[black] (\x) circle (4pt) ($(\y:7mm)+(\x)$) node {$\x$};
 \foreach \i/\j [count =\k from 1] in {A/180,B/245,C/0,D/-65}{\draw[->] (S)-- ($(S)!0.6!(\i)$) coordinate(\i') node[midway, shift={(\j:5mm)}]{$\overrightarrow{F}_\k$};}
 \end{tikzpicture}
 }
 \shortans{212}
 \loigiai{
 \begin{center}
 \begin{tikzpicture}[scale=1, font=\footnotesize, line join=round, line cap=round, >=stealth]
 \def\a{4.5} \def\b{2.2} \def\h{4.5} \def\g{40}
 \path
 (0,0) coordinate (A)
 (0:\a) coordinate (B)++
 (-\g:\b) coordinate (C)++(180:\a) coordinate (D)
 ($(A)!.5!(C)$) coordinate (O)++ (90:\h) coordinate (S)
 ;
 \draw[dashed] (S)--(B)--(A)--(C)--(B)--(D) (S)--(O); 
 \draw (S)--(A)--(D)--(C)--(S)--(D);
 \foreach \i/\j [count =\k from 1] in {A/180,D/210,C/-10}{\draw[->] (S)-- ($(S)!0.65!(\i)$) coordinate(\i') node[midway, shift={(\j:4mm)}]{$\overrightarrow{F}_\k$};}
 \draw[->, dashed] (S)-- ($(S)!0.65!(B)$) coordinate(B') node[midway, shift={(250:4mm)}]{$\overrightarrow{F}_4$};
 \draw (A')--(D')--(C');
 \draw[dashed] (D')--(B')--(A')--(C')--(B');
 \draw[->, dashed] (S)--($(S)!0.65!(O)$) coordinate(O');
 \foreach \x /\g in {A/160,D/-90,C/-20,S/90,O/-90,B/30, A'/160, D'/220, C'/10, B'/40, O'/-55}
 \fill(\x) circle (1pt)($(\x)+(\g:3mm)$) node {$\x$};
 \end{tikzpicture}
 \end{center}
 Gọi $O$ là hình chiếu vuông góc của $S$ trên $(ABCD)$. Ta có $SA=SB=SC=SD$ nên các tam giác vuông $SOA$, $SOB$, $SOC$, $SOD$ bằng nhau. Suy ra $OA=OB=OC=OD$ hay $O$ là tâm hình chữ nhật $ABCD$.\\ 
 Gọi $A'$, $B'$, $C'$, $D'$ lần lượt là các điểm sao cho $\overrightarrow{SA'}=\overrightarrow{F}_1$, $\overrightarrow{SB'}=\overrightarrow{F}_2$, $\overrightarrow{SC'}=\overrightarrow{F}_3$ và $\overrightarrow{SD'}=\overrightarrow{F}_4$.\\
 Vì $\left|\overrightarrow{F}_1\right| = \left|\overrightarrow{F}_2\right| = \left|\overrightarrow{F}_3\right| = \left|\overrightarrow{F}_4\right|$ nên $SA'=SB'=SC'=SD'$. Do đó $S.A'B'C'D'$ là hình chóp có đáy $A'B'C'D'$ là hình chữ nhật.\\
 Gọi $O'$ là tâm hình chữ nhật $A'B'C'D'$, ta có $O'$ thuộc $SO$.\\
 Ta có
 \[\overrightarrow{F}_1 + \overrightarrow{F}_2 + \overrightarrow{F}_3 + \overrightarrow{F}_4 = \overrightarrow{SA'} + \overrightarrow{SB'} + \overrightarrow{SC'} + \overrightarrow{SD'} = \left(\overrightarrow{SA'} + \overrightarrow{SC'}\right) + \left(\overrightarrow{SB'} + \overrightarrow{SD'}\right) =4\overrightarrow{SO'}.\]
 Gọi $\overrightarrow{P}$ là trọng lực của vật nặng và khung sắt. Do vật và khung sắt ở vị trí cân bằng nên
 \[\overrightarrow{P} = \overrightarrow{F}_1 + \overrightarrow{F}_2 + \overrightarrow{F}_3 + \overrightarrow{F}_4 = 4\overrightarrow{SO'}.\]
 Theo giả thiết ta có $\left|\overrightarrow{P}\right|=400+200=600$ N nên $4\left|\overrightarrow{SO'}\right| =600 \Leftrightarrow SO'=150$.\\
 Lại có $\bigl(SA,(ABCD)\bigr)=\widehat{SAO}=\widehat{SA'O'}=45^\circ$.\\
 Vì $\triangle SO'A'$ vuông tại $O'$ nên $SA'=\dfrac{SO'}{\sin 45^{\circ}}=150\sqrt{2}$.\\
 Vậy cường độ của lực căng $\overrightarrow{F}_1$ là $\left|\overrightarrow{F}_1\right| = 150\sqrt{2}\approx 212$ N.
 }
\end{ex}


\Closesolutionfile{ans}
 
\begin{indapan}
	{ans/ans\currfilebase}
\end{indapan}


% \begin{name}
 {Biên soạn: Lê Văn Hiếu\\Phản biện: Bùi Lương Phúc}
{Đề ôn tập chương II}
\end{name}

\caulc
\Opensolutionfile{ans}[ans/ans\currfilebase-Phan-I]
\begin{ex}%[2-H2B4-SO-13-2425 (Nguồn Đề 13 - Bài 4)]%[VN-MT-7, Lê Văn Hiếu]%[2H2N1-2]
 Cho hình hộp chữ nhật $ABCD.A'B'C'D'$. Khẳng định nào sau đây đúng?
 \choice
 {$\overrightarrow{AC'}=\overrightarrow{AB}+\overrightarrow{AC}+\overrightarrow{AB}$}
 {\True $\overrightarrow{AC'}=\overrightarrow{AB}+\overrightarrow{AD}+\overrightarrow{AA'}$}
 {$\overrightarrow{AC'}=\overrightarrow{AB'}+\overrightarrow{AC}+\overrightarrow{AD'}$}
 {$\overrightarrow{AC'}=\overrightarrow{AB'}+\overrightarrow{AD'}+\overrightarrow{AA'}$}
 \loigiai{Theo quy tắc hình hộp ta có $\overrightarrow{AC'}=\overrightarrow{AB}+\overrightarrow{AD}+\overrightarrow{AA'}$.}
\end{ex}

\begin{ex}%[2-H2B4-SO-13-2425 (Nguồn Đề 13 - Bài 4)]%[VN-MT-7, Lê Văn Hiếu]%[2H2N1-3]
 Nếu một vật có khối lượng $m$ (kg) thì lực hấp dẫn $\overrightarrow{P}$ của trái đất tác dụng lên vật được xác định theo công thức $\overrightarrow P=m\overrightarrow g$, trong đó $\overrightarrow g$ là vectơ gia tốc rơi tự do có độ lớn $g=9{,}8$ (m/s$^2$). Độ lớn của lực hấp dẫn trái đất tác dụng lên một quả lê có khối lượng $105$ g là
 \choice
 {$102{,}9$ N}
 {$1029$ N}
 {\True $1{,}029$ N}
 {$10{,}29$ N}
 \loigiai{Đổi $105$ g$=0{,}105$ kg.\\
 Độ lớn của lực hấp dẫn của trái đất tác dụng lên quả lê là $\left|\overrightarrow P\right|=m\left|\overrightarrow g\right|=0{,}105\cdot9{,}8=1{,}029$ N.
 }
\end{ex}

\begin{ex}%[2-H2B4-SO-13-2425 (Nguồn Đề 13 - Bài 4)]%[VN-MT-7, Lê Văn Hiếu]%[2H2N2-3]
 Cho biết máy bay $A$ đang bay với vectơ vận tốc $\overrightarrow u=( 300;200;400)$ (đơn vị: km/h). Máy bay $B$ bay ngược hướng và có tốc độ gấp $2$ lần tốc độ của máy bay $A$. Tọa độ vectơ vận tốc $\overrightarrow v$ của máy bay $B$ là
 \choice
 {$\overrightarrow v=(600;400;800)$}
 {$\overrightarrow v=(150;100;200)$}
 {\True $\overrightarrow v=(-600;-400;-800)$}
 {$\overrightarrow v=(-150;-100;-200)$}
 \loigiai{Máy bay $B$ bay ngược hướng và có tốc độ gấp $2$ lần tốc độ của máy bay $A$ nên vectơ vận tốc $\overrightarrow{v}$ ngược hướng với vectơ vận tốc $\overrightarrow{u}$ và $|\overrightarrow{v}|=2\left|\overrightarrow{u}\right|$, do đó $\overrightarrow v=-2\overrightarrow u\Rightarrow\overrightarrow v=(-600;-400;-800)$.
}\end{ex}

\begin{ex}%[2-H2B4-SO-13-2425 (Nguồn Đề 13 - Bài 4)]%[VN-MT-7, Lê Văn Hiếu]%[2H2N2-2]
 Trong không gian $Oxyz$, cho hai điểm $A(-3;2;-1)$, $B(-1;0;5)$. Tọa độ trung điểm $I$ của đoạn thẳng $AB$ là
 \choice
 {$I(-1;1;2)$}
 {$I(2;1;-2)$}
 {$I(-2;-1;2)$}
 {\True $I(-2;1;2)$}
 \loigiai{Tọa độ trung điểm $I$ của đoạn thẳng $AB$ là $I(-2;1;2)$.
}\end{ex}

\begin{ex}%[2-H2B4-SO-13-2425 (Nguồn Đề 13 - Bài 4)]%[VN-MT-7, Lê Văn Hiếu]%[2H2N2-2]
 Trong không gian tọa độ $Oxyz$, biết $\overrightarrow{OM}=2\overrightarrow i-3\overrightarrow j+\overrightarrow k$. Toạ độ của điểm $M$ là
 \choice
 {$(-2;3;-1)$}
 {\True $(2;-3;1)$}
 {$(-3;2;1)$}
 {$(2;1;-3)$}
 \loigiai{Vì $\overrightarrow{OM}=2\overrightarrow i-3\overrightarrow j+\overrightarrow k$ nên $M(2;-3;1)$.
}\end{ex}

\begin{ex}%[2-H2B4-SO-13-2425 (Nguồn Đề 13 - Bài 4)]%[VN-MT-7, Lê Văn Hiếu]%[2H2N2-3]
 Trong không gian với hệ trục tọa độ $Oxyz$, cho hai điểm $A(-2;2;1)$, $B( 0;1;3)$. Toạ độ của vectơ $\overrightarrow{AB}$ là
 \choice
 {\True $\overrightarrow{AB}=(2;-1;2)$}
 {$\overrightarrow{AB}=(-2;3;4)$}
 {$\overrightarrow{AB}=(-2;1;-2)$}
 {$\overrightarrow{AB}=(-2;2;3)$}
 \loigiai{Ta có $\overrightarrow{AB}=(2;-1;2)$.
}\end{ex}

\begin{ex}%[2-H2B4-SO-13-2425 (Nguồn Đề 13 - Bài 4)]%[VN-MT-7, Lê Văn Hiếu]%[2H2N1-3]
 Cho tứ diện đều $ABCD$ có cạnh bằng $2$. Tính $\overrightarrow{AB}\cdot\overrightarrow{CD}$.
 \choice
 {$\overrightarrow{AB}\cdot\overrightarrow{CD}=-4$}
 {$\overrightarrow{AB}\cdot\overrightarrow{CD}=2$}
 {$\overrightarrow{AB}\cdot\overrightarrow{CD}=1$}
 {\True $\overrightarrow{AB}\cdot\overrightarrow{CD}=0$}
 \loigiai{
 Vì $ABCD$ là tứ diện đều nên các tam giác $ABC$ và $ABD$ là các tam giác đều.\\
 Khi đó
 \begin{eqnarray*}
 \overrightarrow{AB}\cdot\overrightarrow{CD}&=&\overrightarrow{AB}\cdot(\overrightarrow{AD}-\overrightarrow{AC})\\
 &=&\overrightarrow{AB}\cdot\overrightarrow{AD}-\overrightarrow{AB}\cdot\overrightarrow{AC}\\
 &=&2\cdot2\cdot\cos60^\circ-2\cdot2\cdot\cos60^\circ=0.
 \end{eqnarray*}
}\end{ex}

\begin{ex}%[2-H2B4-SO-13-2425 (Nguồn Đề 13 - Bài 4)]%[VN-MT-7, Lê Văn Hiếu]%[2H2N2-5]
 Trong không gian với hệ trục tọa độ $Oxyz$, cho hai điểm $M(-5;2;3)$, $I(2;3;1)$. Gọi $N$ là điểm đối xứng với $M$ qua $I$. Tính độ dài đoạn $ON$.
 \choice
 {$ON=6\sqrt2$}
 {$ON=5\sqrt2$}
 {\True $ON=7\sqrt2$}
 {$ON=3\sqrt2$}
 \loigiai{Vì $N$ là điểm đối xứng với $M$ qua $I$ nên $I$ là trung điểm của đoạn $MN$, do đó $N( 9;4;-1)$.\\
 Vậy $ON=\sqrt{9^2+4^2+(-1)^2}=7\sqrt2$.
}\end{ex}

\begin{ex}%[2-H2B4-SO-13-2425 (Nguồn Đề 13 - Bài 4)]%[VN-MT-7, Lê Văn Hiếu]%[2H2N2-5]
 Trong không gian với hệ trục tọa độ $Oxyz$, cho hai vectơ $\overrightarrow a=(1;-2;0)$ và $\overrightarrow b=(-2;3;1)$. Cho các mệnh đề sau.
 \begin{enumEX}{2}
 \item $\overrightarrow a\cdot\overrightarrow b=-8$.
 \item $2\overrightarrow a=(2;-4;1)$.
 \item $\overrightarrow a+\overrightarrow b=(-1;0;-1)$.
 \item $\left|\overrightarrow b\right|=14$.
 \end{enumEX}
 Số mệnh đề đúng là
 \choice
 {\True $1$}
 {$3$}
 {$2$}
 {$4$}
 \loigiai{Ta có
 \begin{enumerate}
 \item $\overrightarrow a\cdot\overrightarrow b=-8$.
 \item $2\overrightarrow a=(2;-4;0)$.
 \item $\overrightarrow a+\overrightarrow b=(-1;1;1)$.
 \item $\left|\overrightarrow b\right|=\sqrt{14}$.
 \end{enumerate}
 Vậy số mệnh đề đúng là $1$.
}\end{ex}

\begin{ex}%[2-H2B4-SO-13-2425 (Nguồn Đề 13 - Bài 4)]%[VN-MT-7, Lê Văn Hiếu]%[2H2N2-5]
 Trong không gian với hệ trục tọa độ $Oxyz$, cho $\overrightarrow a=(1;-2;3)$ và $\overrightarrow b=(2;-1;-1)$. Mệnh đề nào là mệnh đề \textbf{sai}?
 \choice
 {Vectơ $\overrightarrow{u}=(-5;-7;-3)$ cùng vuông góc với vectơ $\overrightarrow a$ và $\overrightarrow b$}
 {Vectơ $\overrightarrow a$ không cùng phương với vectơ $\overrightarrow b$}
 {Vectơ $\overrightarrow a$ không vuông góc với vectơ $\overrightarrow b$}
 {\True $\left|\overrightarrow a\right|=14$}
 \loigiai{
 Ta có $\left[\overrightarrow a,\overrightarrow b\right]=( 5;7;3)$, suy ra vectơ $\overrightarrow{u}=(-5;-7;-3)$ cùng phương với $\left[\overrightarrow a,\overrightarrow b\right]$ nên $\overrightarrow{u}$ vuông góc với hai vectơ $\overrightarrow{a}$ và $\overrightarrow{b}$.\\
 Do $\dfrac12\ne\dfrac{-2}{-1}$ nên vectơ $\overrightarrow a$ không cùng phương với vectơ $\overrightarrow b$.\\
 Do $\overrightarrow a\cdot\overrightarrow b=1\cdot2+(-2)(-1)+3(-1)=1$ nên vectơ $\overrightarrow a$ không vuông góc với vectơ $\overrightarrow b$.\\
 Ta có $\left|\overrightarrow a\right|=\sqrt{1^2+(-2)^2+3^2}=\sqrt{14}$.
}\end{ex}

\begin{ex}%[2-H2B4-SO-13-2425 (Nguồn Đề 13 - Bài 4)]%[VN-MT-7, Lê Văn Hiếu]%[2H2H1-2]
 Cho hình chóp $S.ABCD$ có đáy $ABCD$ là hình bình hành tâm $O$. Trong các mệnh đề sau mệnh đề nào là mệnh đề \textbf{sai}?
 \choice
 {$\overrightarrow{SA}+\overrightarrow{SB}+\overrightarrow{SC}+\overrightarrow{SD}=4\overrightarrow{SO}$}
 {$\overrightarrow{SA}-\overrightarrow{SB}+\overrightarrow{SC}-\overrightarrow{SD}=\overrightarrow0$}
 {\True $\overrightarrow{SA}+\overrightarrow{SB}+\overrightarrow{SC}+\overrightarrow{SD}=\overrightarrow0$}
 {$\overrightarrow{OA}+\overrightarrow{OB}+\overrightarrow{OC}+\overrightarrow{OD}=\overrightarrow0$}
 \loigiai{
 \begin{center}
 \begin{tikzpicture}[scale=1, font=\footnotesize, line join=round, line cap=round, >=stealth]
 \coordinate (A) at (0,0);
 \coordinate (B) at (-2,-1);
 \coordinate (C) at (3,-1);
 \coordinate (D) at ($(A)+(C)-(B)$);
 \coordinate (O) at ($(A)!1/2!(C)$);
 \coordinate (S) at ($(O)+(0,5)$);
 \draw(S)--(B)--(C)--(D)--(S)--(C);
 \draw[dashed](B)--(A)--(D)--(B)(C)--(A)--(S)--(O);
 \foreach \p/\g in {A/150, B/-90, C/-90, D/0, S/90, O/-90}\draw[fill=black] (\p) circle (1pt)node[shift={(\g:.3)},scale=1]{$\p$};
 \end{tikzpicture}
 \end{center}
 \begin{itemize}
 \item Ta có $O$ là trung điểm của $AC$ nên $\overrightarrow{SA}+\overrightarrow{SC}=2\overrightarrow{SO}$.\\
 $O$ là trung điểm của $BD$ nên $\overrightarrow{SB}+\overrightarrow{SD}=2\overrightarrow{SO}$.\\
 Do đó $\overrightarrow{SA}+\overrightarrow{SB}+\overrightarrow{SC}+\overrightarrow{SD}=4\overrightarrow{SO}$ là khẳng định đúng.
 \item $\overrightarrow{SA}-\overrightarrow{SB}+\overrightarrow{SC}-\overrightarrow{SD}=\overrightarrow{BA}+\overrightarrow{DC}=\overrightarrow0$ là khẳng định đúng.
 \item Ta có $\overrightarrow{SA}+\overrightarrow{SB}+\overrightarrow{SC}+\overrightarrow{SD}=4\overrightarrow{SO}$ như chứng minh trên.\\
 Do đó $\overrightarrow{SA}+\overrightarrow{SB}+\overrightarrow{SC}+\overrightarrow{SD}=\overrightarrow0$ là khẳng định sai.
 \item Ta có $O$ là trung điểm của $AC$ nên $\overrightarrow{OA}+\overrightarrow{OC}=\overrightarrow{O}$.\\
 $O$ là trung điểm của $BD$ nên $\overrightarrow{OB}+\overrightarrow{OD}=\overrightarrow0$.\\
 Do đó $\overrightarrow{OA}+\overrightarrow{OB}+\overrightarrow{OC}+\overrightarrow{OD}=\overrightarrow0$ là khẳng định đúng.
 \end{itemize}
 }
\end{ex}

\begin{ex}%[2-H2B4-SO-13-2425 (Nguồn Đề 13 - Bài 4)]%[VN-MT-7, Lê Văn Hiếu]%[2H2V1-4]
 \immini[thm]{
 Một chiếc đèn chùm treo có khối lượng $m=5$ kg được thiết kế với đĩa đèn được giữ bởi bốn đoạn xích $SA$, $SB$, $SC$, $SD$ sao cho $S.ABCD$ là hình chóp tứ giác đều có $\widehat{ASC}=60^\circ$ (Hình bên).\\
 Biết $\overrightarrow{P}=m\overrightarrow g$ trong đó $\overrightarrow g$ là vectơ gia tốc rơi tự do có độ lớn $10$ m/s$^2$, $\overrightarrow{P}$ là trọng lực tác động lên vật có đơn vị là N, $m$ là khối lượng của vật có đơn vị kg. Cho các kết luận dưới đây.
 \begin{enumerate}
 \item $SA$, $SB$ là hai vectơ cùng phương.
 \item $\left|\overrightarrow{SA}\right|=\left|\overrightarrow{SB}\right|=\left|\overrightarrow{SC}\right|=\left|\overrightarrow{SD}\right|$.
 \item Độ lớn của trọng lực $\overrightarrow{P}$ tác động lên chiếc đèn chùm bằng $50$ N.
 \item Độ lớn của lực căng cho mỗi sợi xích bằng $\dfrac{25\sqrt3}6$ N.
 \end{enumerate}
 }{
 \begin{tikzpicture}[scale=.65,>=stealth, font=\footnotesize, line join=round, line cap=round]
 \tikzset{day/.pic=
 {\draw[shade,bottom color=brown!30,top color= white!30,rounded corners=0.5ex,line width=1.5pt,gray!80]
 (0,0) ellipse ({2.5pt} and {8pt});}
 }
 \def\h{6}
 \def\a{3}
 \def\b{1.5}
 \path
 (0,0) coordinate (O)
 ($(O)+(0,\h)$) coordinate (S)
 ($(O)+(10:\a cm and \b cm)$)coordinate (M)
 ($(O)+(180:\a cm and .8*\b cm)$)coordinate (A)
 ($(O)+(0:\a cm and .8*\b cm)$)coordinate (C)
 ($(O)+(60:\a cm and .8*\b cm)$)coordinate (B)
 ($(O)+(-120:\a cm and .8*\b cm)$)coordinate (D)
 ;
 \draw[fill=brown] (M) arc (10:-190:\a cm and \b cm);
 \draw[fill=white] (M) arc (10:-190:\a cm and .8*\b cm);
 \draw [fill=white] (M) arc (10:190:\a cm and .8*\b cm);
 \draw[fill,bottom color=black!30,top color= brown!70, ,left color=black!50] ($(S)-(.5,.3)$) rectangle ($(S)+(.5,.3)$);
 \foreach \m in {0,1,2,...,12}{\pic[rotate=-20] at ($(A)+(.25*\m,.5*\m)$) {day};}
 \foreach \m in {0,1,2,...,14}{\pic[rotate=-15] at ($(D)+(.11*\m,.5*\m)$) {day};}
 \foreach \m in {0,1,2,...,10}{\pic[rotate=5] at ($(B)+(-.15*\m,.5*\m)$) {day};}
 \foreach \m in {0,1,2,...,12}{\pic[rotate=18] at ($(C)+(-.25*\m,.5*\m)$) {day};}
 \foreach \x/\y in {A/180,B/220,C/-45,D/-90,S/90}
 \fill[black] (\x) circle (4pt) ($(\y:7mm)+(\x)$) node {$\x$};
 \end{tikzpicture}
 }
 \noindent
 Số kết luận đúng là
 \choice
 {$1$}
 {\True $2$}
 {$3$}
 {$0$}
 \loigiai{
 \begin{enumerate}
 \item $SA$, $SB$ là hai vectơ cùng phương. \textbf{Sai}
 \item $\left|\overrightarrow{SA}\right|=\left|\overrightarrow{SB}\right|=\left|\overrightarrow{SC}\right|=\left|\overrightarrow{SD}\right|$. \textbf{Đúng}
 \item Độ lớn của trọng lực $\overrightarrow{P}$ tác động lên chiếc đèn chùm là $\left|\overrightarrow{P}\right|=m\cdot\left|\overrightarrow{g}\right|=5\cdot10=50$ N. \textbf{Đúng}
 \item Độ lớn của lực căng cho mỗi sợi xích bằng $\dfrac{25\sqrt3}6$ N. \textbf{Sai}\\
 Ta có $S.ABCD$ là hình chóp tứ giác đều $\Rightarrow SA=SB=SC=SD$.\\
 Mà $\widehat{ASC}=60^\circ\Rightarrow$ tam giác $SAC$ đều.\\
 Gọi $O$ là trung điểm $AC$.\\
 Ta có hợp lực của $4$ lực căng của $4$ sợi xích
 \[\overrightarrow{F}=\overrightarrow{SA}+\overrightarrow{SC}+\overrightarrow{SB}+\overrightarrow{SD}=2\overrightarrow{SO}+2\overrightarrow{SO}=4\overrightarrow{SO}.\]
 Để đèn chùm đứng yên thì hợp lực của các sợi xích phải cân bằng với trọng lực hay $4\overrightarrow{SO}=\overrightarrow{P}$ hay $4SO=50\Leftrightarrow SO=12{,}5$.\\
 Xét tam giác đều $SAC$ có $SA=\dfrac{\sqrt3}2SO=\dfrac{25\sqrt3}4$.\\
 Vậy độ lớn của lực căng cho mỗi sợi xích là $\dfrac{25\sqrt3}4$ N.
 \end{enumerate}
}\end{ex}
\Closesolutionfile{ans}

\cauds
\Opensolutionfile{ans}[ans/ans\currfilebase-Phan-II]
\begin{ex}%[2-H2B4-SO-13-2425 (Nguồn Đề 13 - Bài 4)]%[VN-MT-7, Lê Văn Hiếu]%[2H2H2-4]
 Trong không gian với hệ tọa độ $Oxyz$, cho hai vectơ $\overrightarrow a=(1;2;-2)$ và $\overrightarrow b=(-1;-1;0)$.
 \choiceTF
 {$\left|\overrightarrow a\right|=9$}
 {\True $\overrightarrow a+\overrightarrow b=( 0;1;-2)$}
 {$\overrightarrow a$ và $\overrightarrow b$ cùng phương}
 {\True $\left(\overrightarrow a,\overrightarrow b\right)=135^\circ$}
 \loigiai{
 \begin{itemchoice}
 \itemch \textbf{Sai}.\\
 Ta có $\left|\overrightarrow a\right|=\sqrt{1^2+2^2+(-2)^2}=3$.
 \itemch \textbf{Đúng}.\\
 Ta có $\overrightarrow a+\overrightarrow b=(1-1;2-1;-2+0)\Rightarrow \overrightarrow a+\overrightarrow b=( 0;1;-2)$.
 \itemch \textbf{Sai}.\\
 Ta có $\dfrac1{-1}\ne\dfrac2{-1}$ nên $\overrightarrow a$ và $\overrightarrow b$ không cùng phương.
 \itemch \textbf{Đúng}.\\
 Áp dụng công thức:
 \begin{align*}
 \cos(\overrightarrow a,\overrightarrow b)&=\dfrac{\overrightarrow a\cdot\overrightarrow b}{\left|\overrightarrow a\right|\left|\overrightarrow b\right|}\\
 &=\dfrac{1\cdot(-1)+2\cdot(-1)+(-2)\cdot0}{\sqrt{1^2+2^2+(-2)^2}\cdot\sqrt{(-1)^2+(-1)^2+0^2}}=\dfrac{-3}{3\sqrt2}=-\dfrac1{\sqrt2}.
 \end{align*}
 Suy ra $(\overrightarrow a,\overrightarrow b)=135^\circ$.
 \end{itemchoice}
}\end{ex}

\begin{ex}%[2-H2B4-SO-13-2425 (Nguồn Đề 13 - Bài 4)]%[VN-MT-7, Lê Văn Hiếu]%[2H2H2-4]
 Cho $4$ điểm $A(1;2;0)$, $B(5;1;4)$, $C(7;-2;-2)$, $D(3;m;2)$.
 \choiceTF
 {Độ dài đoạn $AB$ lớn hơn độ dài đoạn $AC$}
 {\True $ m=\dfrac32$ thì $D$ là trung điểm của $AB$}
 {$ m=5$ thì $AB\perp AD$}
 {$ m=-1$ thì $AB\parallel CD$}
 \loigiai{
 \begin{itemchoice}
 \itemch \textbf{Sai}.\\
 Ta có $AB=\sqrt{4^2+(-1)^2+4^2}=\sqrt{33}$ và $AC=\sqrt{6^2+(-4)^2+(-2)^2}=\sqrt{56}$.
 \itemch \textbf{Đúng}.\\
 Tọa độ trung điểm của đoạn $AB$ là $\left(3;\dfrac32;2\right)\Rightarrow m=\dfrac32$.
 \itemch \textbf{Sai}. Với $m=5$, ta có $D(3;5;2)$.\\
 Ta có $AB\perp AD\Leftrightarrow\overrightarrow{AB}\cdot\overrightarrow{AD}=0$.\\
 Vì $\heva{&\overrightarrow{AB}=(4;-1;4)\\
 &\overrightarrow{AD}=(2;3;2)}$ nên $\overrightarrow{AB}\cdot\overrightarrow{AD}=4\cdot2+(-1)\cdot3+4\cdot2=13\ne0$.
 \itemch \textbf{Sai}. Với $m=-1$, ta có $D(3;-1;2)$.\\
 Ta có $\heva{&\overrightarrow{AB}=(4;-1;4)\\
 &\overrightarrow{CD}=(-4;-1;4)} \Rightarrow\overrightarrow{AB}\ne k\overrightarrow{CD}$ (với mọi số thực $k$) $\Rightarrow AB$ không song song với $CD$.
 \end{itemchoice}
}\end{ex}

\begin{ex}%[2-H2B4-SO-13-2425 (Nguồn Đề 13 - Bài 4)]%[VN-MT-7, Lê Văn Hiếu]%[2H2H2-4]
 Trong không gian $Oxyz$, cho các điểm $A( 8;9;2)$, $B( 3;5;1)$ và $C(11;10;4)$.
 \choiceTF
 {Điểm $D$ thỏa mãn $ABCD$ là hình bình hành có tọa độ là $D( 6;6;3)$}
 {\True Độ dài trung tuyến $AM$ bằng $\dfrac{\sqrt{14}}2$}
 {$\widehat{BAC}=30^\circ$}
 {\True Điểm $N$ thuộc mp$( Oxy)$ sao cho ba điểm $A$, $B$, $N$ thẳng hàng có tọa độ là $N(-2;1;0)$}
 \loigiai{
 \begin{itemchoice}
 \itemch \textbf{Sai}. Giả sử $D(x;y;z)$.\\
 Ta có $\overrightarrow{AB}=(-5;-4;-1)$, $\overrightarrow{DC}=(11-x;10-y;4-z)$.\\
 Tứ giác $ABCD$ là hình bình hành $\Leftrightarrow \overrightarrow{AB}=\overrightarrow{DC}\Leftrightarrow\heva{
 &-5=11-x\\
 &-4=10-y\\
 &-1=4-z\\
 }\Leftrightarrow\heva{
 &x=16\\
 &y=14\\
 &z=5.}$\\
 Vậy $D(16;14;5)$.
 \itemch \textbf{Đúng}.\\
 Tọa độ trung điểm $M$ của $BC$ là $M\left(7;\dfrac{15}2;\dfrac52\right)$.\\
 Ta có $\overrightarrow{AM}=\left(-1;-\dfrac32;\dfrac12\right)$. Suy ra $AM=\sqrt{(-1)^2+\left(-\dfrac32\right)^2+\left(\dfrac12\right)^2}=\dfrac{\sqrt{14}}2$.
 \itemch \textbf{Sai}.\\
 Ta có $\overrightarrow{AB}=(-5;-4;-1)$; $\overrightarrow{AC}=( 3;1;2)$.\\
 Do đó
 \begin{align*}
 \cos\widehat{BAC}&=\cos\left(\overrightarrow{AB},\overrightarrow{AC}\right)=\dfrac{\overrightarrow{AB}\cdot\overrightarrow{AC}}{AB\cdot AC}\\
 &=\dfrac{(-5)\cdot3+(-4)\cdot1+(-1)\cdot2}{\sqrt{(-5)^2+(-4)^2+(-1)^2}\cdot\sqrt{3^2+1^2+2^2}}=-\dfrac{\sqrt3}2.
 \end{align*}
 Suy ra $\widehat{BAC}=150^\circ$.
 \itemch \textbf{Đúng}.\\
 Vì $N\in( Oxy)$ nên $N(x;y;0)$. Ta có $\overrightarrow{AB}=(-5;-4;-1)$; $\overrightarrow{AN}=(x-8;y-9;-2)$.\\
 Vì $3$ điểm $A$, $B$, $N$ thẳng hàng nên $\overrightarrow{AB}$ cùng phương với $\overrightarrow{AN}$. Khi đó
 \[\heva{
 &\dfrac{x-8}{-5}=2\\
 &\dfrac{y-9}{-4}=2\\
 }\Leftrightarrow\heva{
 &x-8=-10\\
 &y-9=-8\\
 }\Leftrightarrow\heva{
 &x=-2\\
 &y=1.\\
 }\]
 Vậy $N(-2;1;0)$.
 \end{itemchoice}
}\end{ex}

\begin{ex}%[2-H2B4-SO-13-2425 (Nguồn Đề 13 - Bài 4)]%[VN-MT-7, Lê Văn Hiếu]%[2H2V1-3]
 \immini[thm]{
 Cho hình hộp $ABCD.A'B'C'D'$. Gọi $M$, $N$ là các điểm lần lượt thuộc các đường thẳng $CA$ và $DC'$ sao cho $\overrightarrow{MC}=m\overrightarrow{MA}$, $\overrightarrow{ND}=m\overrightarrow{NC'}$. Đặt $\overrightarrow{BA}=\overrightarrow a$, $\overrightarrow{BB'}=\overrightarrow b$, $\overrightarrow{BC}=\overrightarrow c$.
 \choiceTF
 {$\overrightarrow{BD'}=\overrightarrow a+\overrightarrow b-\overrightarrow c$}
 {\True $\overrightarrow{BM}=\dfrac1{1-m}\overrightarrow c-\dfrac m{1-m}\overrightarrow a$}
 {\True $\overrightarrow{BN}=\dfrac1{1-m}\overrightarrow a-\dfrac m{1-m}\overrightarrow b+\overrightarrow c$}
 {$ m=\dfrac12$ thì $MN\parallel BD'$}
 }{
 \begin{tikzpicture}[scale=0.85,>=stealth, font=\footnotesize, line join=round, line cap=round]
 \coordinate (A) at (0,0);
 \coordinate (B) at (-2,-1);
 \coordinate (C) at ($(B)+(4,0)$);
 \coordinate (D) at ($(A)+(C)-(B)$);
 \coordinate (A') at ($(A)+(0,4)$);
 \coordinate (B') at ($(B)+(A')-(A)$);
 \coordinate (C') at ($(C)+(A')-(A)$);
 \coordinate (D') at ($(D)+(A')-(A)$);
 \coordinate (M) at ($(C)!1/3!(A)$);
 \coordinate (N) at ($(D)!1/3!(C')$);
 \draw(A')--(B')--(C')--(D')--(A')(B)--(B')(C)--(C')(D)--(D')(B)--(C)--(D)--(C');
 \draw[dashed](A)--(A')(A)--(B)(A)--(D)(C)--(A);
 \foreach \p/\g in {A/160, B/-90, C/-90, D/-90, A'/90, B'/90, C'/90, D'/90, M/90, N/90}\draw[fill=black] (\p) circle (1pt)node[shift={(\g:.3)},scale=1]{$\p$};
 \end{tikzpicture}
 }
 \loigiai{
 Dễ thấy $m\ne1$ vì nếu $m=1$, khi đó $\overrightarrow{MC}=\overrightarrow{MA}\Leftrightarrow \overrightarrow{AC}=\overrightarrow{0}\Leftrightarrow A\equiv C$ (vô lý).
 \begin{itemchoice}
 \itemch \textbf{Sai}.\\
 Theo quy tắc hình hộp ta có $\overrightarrow{BD'}=\overrightarrow a+\overrightarrow b+\overrightarrow c$.
 \itemch \textbf{Đúng}.\\
 Ta có
 \allowdisplaybreaks
 \begin{eqnarray*}
 &&\overrightarrow{MC}=m\overrightarrow{MA}\\
 &\Rightarrow&\overrightarrow{BC}-\overrightarrow{BM}=m\overrightarrow{BA}-m\overrightarrow{BM}\\
 &\Rightarrow&(1-m)\overrightarrow{BM}=\overrightarrow{BC}-m\overrightarrow{BA}\\
 &\Rightarrow&\overrightarrow{BM}=\dfrac1{1-m}\overrightarrow{BC}-\dfrac m{1-m}\overrightarrow{BA}=\dfrac1{1-m}\overrightarrow c-\dfrac m{1-m}\overrightarrow a.
 \end{eqnarray*}
 \itemch \textbf{Đúng}.\\
 Tương tự ta có
 \allowdisplaybreaks
 \begin{align*}
 \overrightarrow{BN}&=\dfrac1{1-m}\overrightarrow{BD}-\dfrac m{1-m}\overrightarrow{BC'}\\
 &=\dfrac1{1-m}\overrightarrow a+\dfrac1{1-m}\overrightarrow c-\dfrac m{1-m}(\overrightarrow b+\overrightarrow c)\\
 &=\dfrac1{1-m}\overrightarrow a-\dfrac m{1-m}\overrightarrow b+\overrightarrow c
 \end{align*}
 \itemch \textbf{Sai}.\\
 Ta có
 \allowdisplaybreaks
 \begin{eqnarray*}
 &&\overrightarrow{MN}=\overrightarrow{BN}-\overrightarrow{BM}\\
 &\Rightarrow&\overrightarrow{MN}=\dfrac{1+m}{1-m}\overrightarrow a-\dfrac m{1-m}\overrightarrow b-\dfrac m{1-m}\overrightarrow c.
 \end{eqnarray*}
 Vì $MN\parallel BD'$ nên $\overrightarrow{MN}$ cùng phương $\overrightarrow{BD'}$. Từ đó ta có
 \allowdisplaybreaks
 \begin{eqnarray*}
 &&\overrightarrow{MN}=k\overrightarrow{BD'}\\
 &\Rightarrow&\heva{
 &\dfrac{1+m}{1-m}=k\\
 &-\dfrac m{1-m}=k\\
 &-\dfrac m{1-m}=k\\
 }\\
 &\Rightarrow&m=-\dfrac12.
 \end{eqnarray*}
 \end{itemchoice}
}\end{ex}
\Closesolutionfile{ans}

\caukq
\Opensolutionfile{ans}[ans/ans\currfilebase-Phan-III]
\begin{ex}%[2-H2B4-SO-13-2425 (Nguồn Đề 13 - Bài 4)]%[VN-MT-7, Lê Văn Hiếu]%[2H2H2-3]
 Trong không gian với hệ tọa độ $Oxyz$, cho các điểm $A(1;0;3)$, $B(2;3;-4)$, $C(-3;1;2)$. Gọi $D(x;y;z)$ là điểm sao cho $ABCD$ là hình bình hành. Tính tổng $T=x+y+z$.
 \shortans[]{3}
 \loigiai{
 Ta có $\overrightarrow{AB}=(1;3;-7)$, $\overrightarrow{DC}=(-3-x;1-y;2-z)$.\\
 Tứ giác $ABCD$ là hình bình hành khi
 \[\overrightarrow{AB}=\overrightarrow{DC}\Leftrightarrow \heva{&1=-3-x\\ &3=1-y\\ &-7=2-z} \Leftrightarrow \heva{&x=-4\\ &y=-2\\ &z=9.}\]
 Vậy, $D(-4;-2;9)$.
 Khi đó $T=-4-2+9=3$.
}\end{ex}

\begin{ex}%[2-H2B4-SO-13-2425 (Nguồn Đề 13 - Bài 4)]%[VN-MT-7, Lê Văn Hiếu]%[2H2H2-2]
 Trong không gian với hệ tọa độ $Oxyz$, cho hình vuông $ABCD$ có $B(3;0;8)$, $D(-5;-4;0)$. Tính $\left|\overrightarrow{CA}+\overrightarrow{CB}\right|$ (kết quả làm tròn đến hàng đơn vị).
 \shortans[]{19}
 \loigiai{
 \begin{center}
 \begin{tikzpicture}[scale=1,>=stealth, font=\footnotesize, line join=round, line cap=round]
 \coordinate (A) at (0,0);
 \coordinate (B) at (-4,0);
 \coordinate (C) at (-4,-4);
 \coordinate (D) at ($(C)+(A)-(B)$);
 \coordinate (M) at ($(A)!1/2!(B)$);
 \draw(A)--(B)--(C)--(D)--(A)(B)--(D)(C)--(M);
 \foreach \p/\g in {A/90,B/90,C/180,D/0, M/90}\draw[fill=black] (\p) circle (1pt)node[shift={(\g:.3)},scale=1]{$\p$};
 \end{tikzpicture}
 \end{center}
 Ta có $\overrightarrow{BD}=(-8;-4;-8)$ $\Rightarrow BD=12$ $\Rightarrow AB=\dfrac{12}{\sqrt2}$ $=6\sqrt2$.\\
 Gọi $M$ là trung điểm $AB$ ta có $BM=\dfrac12AB=3\sqrt2$.\\
 Áp dụng định lí Pythagore ta có $MC=\sqrt{BC^2+BM^2}=\sqrt{72+18}=3\sqrt{10}$.\\
 Từ đó $\left|\overrightarrow{CA}+\overrightarrow{CB}\right|$ $=\left| 2\overrightarrow{CM}\right|$ $=2CM$ $=6\sqrt{10}\approx19$.
}\end{ex}

\begin{ex}%[2-H2B4-SO-13-2425 (Nguồn Đề 13 - Bài 4)]%[VN-MT-7, Lê Văn Hiếu]%[2H2H2-4]
 Trong không gian với hệ trục tọa độ $Oxyz$, cho hai vectơ $\overrightarrow a=(1;-2;0)$, $\overrightarrow b=(1;3;-2)$. Tính góc giữa hai vectơ $\overrightarrow a$ và $\overrightarrow b$ (tính theo độ làm tròn đến hàng đơn vị).
 \shortans[]{127}
 \loigiai{Ta có
 \[\cos(\overrightarrow a,\overrightarrow b)=\dfrac{\overrightarrow a\cdot\overrightarrow b}{\left|\overrightarrow a\right|\cdot\left|\overrightarrow b\right|}=\dfrac{1-6}{\sqrt{1^2+(-2)^2+0}\cdot\sqrt{1^2+3^2+(-2)^2}}=\dfrac{-5}{\sqrt5\cdot\sqrt{14}}.\]
 Vậy $(\overrightarrow a,\overrightarrow b)\approx 127^\circ$.
}\end{ex}

\begin{ex}%[2-H2B4-SO-13-2425 (Nguồn Đề 13 - Bài 4)]%[VN-MT-7, Lê Văn Hiếu]%[2H2H2-6]
 \immini[thm]{Trong một phòng học dạng hình hộp chữ nhật, với chiều dài $8$ m, chiều rộng $6$ m và chiều cao $3$ m. Hai bạn An và Bình làm nhiệm vụ trực nhật, mạng nhện cần quét ở góc ngoài cùng trên trần nhà, An bảo không nên đứng ngay vị trí đó ở nền nhà quét vì bụi sẽ rơi xuống người mình. An lại đố Bình ``nếu mình đứng ở giữa nhà quét thì chổi quét nhà dài mấy mét để quét được vị trí mạng nhện, biết đầu cán chổi (vị trí $B$ trên hình vẽ minh họa) cao $1{,}5$ m so với sàn nhà''. Bình trả lời đứng vị trí đó chổi dài $5$ m cũng không tới. Hỏi Bình đã tính được chổi cần dài bao nhiêu mét (làm tròn kết quả đến hàng phần trăm)?
 }{
 \begin{tikzpicture}[scale=.5,>=stealth, font=\footnotesize, line join=round, line cap=round]
 \coordinate (O) at (0,0);
 \coordinate (A') at (-140:3);
 \coordinate (A) at ($(A')+(0,5)$);
 \coordinate (B') at (8,0);
 \coordinate (C') at ($(A')+(8,0)$);
 \coordinate (b) at ($(B')+(0,5)$);
 \coordinate (c) at ($(C')+(0,5)$);
 \coordinate (o') at ($(O)+(0,5)$);
 \coordinate (x) at ($(O)!1.5!(A')$);
 \coordinate (y) at ($(O)!1.25!(B')$);
 \coordinate (z) at ($(O)!1.5!(o')$);
 \coordinate (b') at ($(A')!1/2!(B')$);
 \coordinate (B) at ($(b')+(0,2.5)$);
 \draw[dashed](O)--(A')(O)--(B')(O)--(o');
 \draw[-stealth](A')--(x);
 \draw[-stealth](B')--(y);
 \draw[-stealth](o')--(z);
 \draw[dashed](b')--(B)--(A);
 \draw(A')--(C') node[midway, below]{$8$m}--(B') node[midway, right, shift={(-45:.3)}]{$6$m}--(b)--(c)--(A)--(o')--(b)(C')--(c)(A)--(A');
 \foreach \p/\g in {A/90,x/-90,y/-90,z/0, O/-90, B/0}\draw[fill=black] (\p) circle (1pt)node[shift={(\g:.3)},scale=1]{$\p$};
 \end{tikzpicture}
 }
\shortans[]{5{,}22}
 \loigiai{
 Xét hệ tọa độ $Oxyz$ như hình vẽ, ta có vị trí mạng nhện ở $A(6;0;3)$ vị trí cầm chổi $B\left(3;4;\dfrac32\right)$.\\
 Vậy chổi phải có độ dài $AB=\sqrt{(3-6)^2+(4-0)^2+\left(\dfrac32-3\right)^2}=\dfrac{\sqrt{109}}2\approx 5{,}22$ m.
}\end{ex}

\begin{ex}%[2-H2B4-SO-13-2425 (Nguồn Đề 13 - Bài 4)]%[VN-MT-7, Lê Văn Hiếu]%[2H2V2-6]
 \immini[thm]{
 Với hệ trục tọa độ $Oxyz$ sao cho $O$ nằm trên mặt nước, mặt phẳng $(Oxy)$ là mặt nước, trục $Oz$ hướng lên trên (đơn vị đo: mét), một con chim bói cá đang ở vị trí $C$ cách mặt nước $2$ m, cách mặt phẳng $(Oxz)$, $(Oyz)$ lần lượt là $3$ m và $1$ m phóng thẳng xuống vị trí con cá, biết con cá cách mặt nước $50$ cm, cách mặt phẳng $(Oxz)$, $(Oyz)$ lần lượt là $1$ m và $1{,}5$ m. Gọi $B(a;b;0)$ là điểm lúc chim bói cá vừa tiếp xúc với mặt nước. Tính $T=a+b$.
 }{
 \begin{tikzpicture}[line join=round, line cap=round,scale=1,transform shape, >=stealth]
 \definecolor{columbiablue}{rgb}{0.61, 0.87, 1.0}%màu nước
 \definecolor{arsenic}{rgb}{0.23, 0.27, 0.29}%màu mỏ
 \definecolor{antiquewhite}{rgb}{0.98, 0.92, 0.84}%màu trắng
 \definecolor{cadmiumorange}{rgb}{0.93, 0.53, 0.18}%lông cam
 \definecolor{coolblack}{rgb}{0.0, 0.18, 0.39}%cánh đậm
 \definecolor{brandeisblue}{rgb}{0.0, 0.44, 1.0}%màu xanh đầu
 \definecolor{darkcoral}{rgb}{0.8, 0.36, 0.27}%màu chân
 
 %---------màu vẽ cá
 \definecolor{amber}{rgb}{1.0, 0.49, 0.0}
% \clip (-3,-3.5) rectangle (3.5,3);

 \tikzset{san/.pic={ 
 \path
 (-1.3,-1.5) coordinate (O)
 ($(O)+(-142:2)$) coordinate (y)
 ($(O)+(0:4.7)$) coordinate (x)
 ($(O)+(90:3)$) coordinate (z)
 ($(x)+(y)-(O)$) coordinate (t)
 
 (-.8,-2.2)coordinate (A)
 (1.6,0.8) coordinate (C)
 ($(A)!.13!(C)$) coordinate (B)
 ;
 \fill[columbiablue] (O)--(x)--(t)--(y)--cycle;
 
 \foreach\p/\g/\t in {x/-90/y, y/-90/x, z/0/z}
 {
 \node at (\p) [shift=(\g:2mm)] {\tiny $\t$};
 }
 
 \foreach\p/\g in {A/180,B/0,C/-50,O/-90}
 %\node at (\p) [shift=(\g:2mm)] {\tiny $\p$};
 {
 \draw[fill=black](\p) circle (.5pt) +(\g:2mm)node{\tiny $\p$};
 }
 
 \draw[->] (O)--(x) ;
 \draw[->] (O)--(y);
 \draw[->] (O)--(z);
 \draw[dashed] (A)--(B);
 \draw (B)--(C);
 %---------nước
 \draw (-1,-1.7)
 ..controls +(-120:.5) and +(-160:.5) ..(0,-2)
 (-.9,-1.9)
 ..controls +(-70:.2) and +(-160:.2) ..(-.2,-1.9)
 (-.95,-1.8)
 ..controls +(70:.5) and +(30:.5) ..(0,-1.9)
 (.1,-1.6)
 ..controls +(-20:.2) and +(30:.2) ..(.2,-1.9)
 (-.7,-1.75)
 ..controls +(-170:.2) and +(-160:.3) ..(-.5,-1.9)
 ;
 }}
 
 \path
 (0,0)pic[scale=1]{san}
 ;
 
 \tikzset{chim_boi_ca/.pic={
 %==============cánh trái
 \draw[fill=coolblack] %(-1,1.1)..controls +(90:.3) and +(170:.3) ..
 (-.55,1.4)
 ..controls +(110:.7) and +(100:.3) ..(-.9,1.8)
 ..controls +(135:.3) and +(120:.3) ..(-1.2,1.8)
 ..controls +(145:.25) and +(110:.25) ..(-1.45,1.7)
 ..controls +(165:.15) and +(85:.15) ..(-1.75,1.63)
 ..controls +(-165:.1) and +(85:.1) ..(-1.9,1.5)
 ..controls +(165:.15) and +(85:.1) ..(-2.1,1.3)
 ..controls +(-165:.1) and +(95:.1) ..(-2.2,1.1)
 ..controls +(-160:.1) and +(95:.1) ..(-2.35,1)--(-1,1.1)
 ;
 %======--------------------
 %Tô lông đầu
 \def\L{
 (2,.84)
 ..controls +(170:.2) and +(25:.3) ..(1,.65)
 ..controls +(-145:.5) and +(40:.6) ..(.3,.4)
 ..controls +(-140:.3) and +(-60:.3) ..(0,.6)
 ..controls +(120:.3) and +(-50:.7) ..(-1,1.1)
 ..controls +(90:.3) and +(170:.3) ..(-.55,1.4)%1
 ..controls +(-40:.2) and +(140:.1) ..(-.25,1.2)
 ..controls +(-70:.2) and +(-160:.35) ..(.15,.85)
 ..controls +(75:.7) and +(135:.8) ..(1.9,1.3)--(2.1,1)--cycle
 ;
 }
 %\draw[red]\L;
 \fill[brandeisblue] \L;
 %==============================
 \draw[fill=antiquewhite] (2,1.05)
 ..controls +(165:.2) and +(-35:.2)..(1.7,1.15)
 ..controls +(145:.2) and +(65:.2)..(1.2,1.15)
 ..controls +(-60:.1) and +(165:.1)..(1.4,1)
 ..controls +(-15:.2) and +(-145:.3)..cycle
 ;
 \draw[fill=arsenic] (1.6,1.2)
 ..controls +(155:.18) and +(55:.15)..(1.23,1.17)
 ..controls +(-75:.2) and +(-95:.2)..cycle
 ;
 \fill (1.44,1.14) circle(1mm);
 
 \fill[cadmiumorange] (2,1.05)
 ..controls +(165:.2) and +(-35:.2)..(1.7,1.15)
 ..controls +(145:.2) and +(95:.3)..cycle
 ;
 \fill[cadmiumorange] (2,.84)
 ..controls +(150:.2) and +(-25:.1) ..(1.7,1)
 ..controls +(155:.2) and +(-50:.3) ..(1.2,1.08)
 ..controls +(130:.2) and +(50:.2) ..(.75,1)
 ..controls +(-130:.2) and +(-10:.2) ..(.21,1)
 ..controls +(-120:.1) and +(70:.1) ..(.15,.84)
 ..controls +(-20:.3) and +(-150:.3) ..(.8,.85)
 ..controls +(-10:.3) and +(150:.5) ..(1.7,.85)
 ..controls +(-10:.1) and +(150:.1) ..cycle
 ;
 %==================
 %viền đen đầu
 \def\X{
 (0,.6)
 ..controls +(120:.3) and +(-50:.7) ..(-1,1.1)
 
 (-.55,1.4)%1
 ..controls +(-40:.2) and +(140:.1) ..(-.25,1.2)
 ..controls +(-70:.2) and +(-160:.35) ..(.15,.85)
 ..controls +(75:.7) and +(135:.8) ..(1.9,1.3)
 ;
 }
 \draw[black]\X;
 %====================
 %Tô mỏ
 \def\N{
 (1.9,1.3)
 ..controls +(-35:.4) and +(140:.3) ..(3.4,.78)
 ..controls +(-175:.2) and +(-10:.3) ..(2,.84)
 ..controls +(150:.2) and +(-25:.1) ..(1.7,1)
 ..controls +(20:.2) and +(175:.1) ..(2,1.05)
 ..controls +(-35:.2) and +(155:.1) ..cycle
 ;
 }
 %\draw[red]\N;
 \fill[arsenic] \N;
 %Mỏ
 \def\M{
 (1.9,1.3)
 ..controls +(-35:.4) and +(140:.3) ..(3.4,.78)
 ..controls +(-175:.2) and +(-10:.3) ..(2,.84)
 ..controls +(170:.2) and +(25:.3) ..(1,.65)
 ;
 }
 \draw[black]\M;
 %==============================
 
 %================Cánh phải
 \draw[fill=coolblack]
 %(-2.6,.95)
 %..controls +(-160:.2) and +(145:.3) ..(-1.6,.5)
 %..controls +(-35:.2) and +(145:.3) ..(-1,-.5)
 %..controls +(-35:.2) and +(-145:.2) ..
 (-.6,-.3)%3
 ..controls +(-130:.5) and +(-10:.2) ..(-1,-.95)
 ..controls +(170:.1) and +(-10:.15) ..(-1.2,-1.02)
 ..controls +(170:.1) and +(-40:.15) ..(-1.45,-.95)
 ..controls +(160:.1) and +(-80:.15) ..(-1.6,-.8)
 ..controls +(140:.1) and +(-70:.15) ..(-1.75,-.65)
 ..controls +(140:.1) and +(-70:.15) ..(-1.9,-.5)
 ..controls +(160:.1) and +(-70:.15) ..(-2.05,-.35)
 ..controls +(150:.1) and +(-70:.15) ..(-2.25,-.2)
 ..controls +(-160:.1) and +(-20:.15) ..(-2.5,-.18)
 ..controls +(160:.1) and +(-30:.15) ..(-2.75,-.16)
 ..controls +(160:.1) and +(-60:.15) ..(-2.95,-.05)
 ..controls +(160:.1) and +(-80:.15) ..(-3.2,.05)
 ..controls +(160:.1) and +(-90:.15) ..(-3.35,.15)
 ..controls +(160:.1) and +(-95:.15) ..(-3.55,.25)
 ..controls +(-160:.2) and +(-145:.2) ..(-3.7,.35)
 ..controls +(-160:.2) and +(-160:.4) ..(-3.7,.53)
 ..controls +(-160:.2) and +(-160:.3) ..(-3.85,.65)
 ..controls +(160:.5) and +(-170:.3) ..(-2.6,.95)
 ..controls +(0:.5) and +(120:.3) ..(-.6,-.3)%3
 ;
 %===================
 %===================
 \fill[brandeisblue]
 (-.8,-1.1)%lông đuôi xanh
 ..controls +(-80:.2) and +(140:.2) ..(-.5,-1.5)
 ..controls +(-40:.2) and +(140:.4) ..(-.3,-2.5)
 ..controls +(135:.7) and +(-85:.6) ..cycle
 ;
 \draw (-.3,-2.5)
 ..controls +(135:.7) and +(-85:.6) ..(-.8,-1.1)%lông đuôi xanh
 ;
 %=============đuôi
 \draw[fill=brandeisblue] (-.45,-2.3)
 ..controls +(-95:.2) and +(95:.2) ..(-.4,-2.8)
 --(-.32,-2.8)--(-.25,-2.2)
 (-.25,-2.2)--(-.32,-2.78)--(-.27,-2.78)--(-.16,-2.2)
 ;
 %================
 %Tô lông cam
 \def\C{
 (2,.84)
 ..controls +(170:.2) and +(25:.3) ..(1,.65)
 ..controls +(-145:.5) and +(40:.6) ..(.3,.4)
 ..controls +(-140:.3) and +(-60:.3) ..(0,.6)
 ..controls +(120:.3) and +(-50:.7) ..(-1,1.1)%2
 ..controls +(130:.2) and +(20:.3) ..(-2.6,.95)
 ..controls +(-160:.2) and +(145:.3) ..(-1.6,.5)
 ..controls +(-35:.2) and +(145:.3) ..(-1,-.5)
 ..controls +(-35:.2) and +(-145:.2) ..(-.6,-.3)%3
 ..controls +(-135:.2) and +(100:.2) ..(-.8,-1.1)%lông đuôi xanh
 ..controls +(-80:.2) and +(140:.2) ..(-.5,-1.5)
 ..controls +(-40:.2) and +(140:.4) ..(-.3,-2.5)%đuôi dưới
 ..controls +(40:.4) and +(-150:.5) ..(.5,-1.1)%chân
 ..controls +(-20:.2) and +(160:.2) ..(.8,-1.15)
 ..controls +(150:.1) and +(-70:.1) ..(.65,-.9)
 ..controls +(110:.2) and +(-95:.8) ..(1.26,.6)
 ..controls +(75:.1) and +(-160:.2) ..cycle
 ;
 }
 
 \fill[cadmiumorange] \C;
 \draw[black]\C;

 \draw[black] (-1,1.1)
 ..controls +(130:.2) and +(20:.3) ..(-2.6,.95)
 
 (-.6,-.3)%3
 ..controls +(-135:.2) and +(100:.2) ..(-.8,-1.1)
 ;
 
 %======================chân
 \draw[fill=darkcoral]
 (.5,-1.1)
 ..controls +(-20:.1) and +(170:.1) ..(1.5,-1.1)%chân
 ..controls +(-10:.1) and +(-10:.3) ..(1.2,-1.2)
 ;
 %móng 2
 \draw (1.47,-1.15)%chân
 ..controls +(-10:.1) and +(120:.1) ..(1.63,-1.25)
 ;
 %---------------------
 \draw[fill=darkcoral]
 (.7,-1.15)
 ..controls +(40:.1) and +(170:.1) ..(1.3,-1.08)%chân
 ..controls +(-10:.1) and +(-10:.1) ..(1.2,-1.2)
 ..controls +(170:.1) and +(30:.1) ..(1,-1.2)
 ;
 %móng 2
 \draw (1.27,-1.15)%chân
 ..controls +(-10:.1) and +(120:.1) ..(1.43,-1.25)
 ;
 %------------------------
 \draw[fill=darkcoral]
 (.25,-1.1)
 ..controls +(-20:.2) and +(-170:.1) ..(.5,-1.1)%chân
 ..controls +(-20:.2) and +(160:.2) ..(.8,-1.15)
 ..controls +(-20:.2) and +(160:.2) ..(1.1,-1.2)%móng 1
 ..controls +(-20:.1) and +(-50:.1) ..(1.03,-1.25)
 ..controls +(130:.05) and +(-10:.05) ..(.8,-1.26)
 ..controls +(170:.05) and +(10:.05) ..(.5,-1.28)
 ..controls +(-150:.1) and +(-160:.15) ..(.3,-1.25)
 ..controls +(160:.1) and +(-80:.1) ..(.1,-1.15)
 ;
 %móng 1
 \draw (1.1,-1.25)%móng 1
 ..controls +(-20:.1) and +(95:.1) ..(1.2,-1.35)
 ;
 }}
 %===========Vẽ cá
 \tikzset{ca/.pic={
 %vây
 \def\V{
 (-.35,.74)
 ..controls +(120:.12) and +(40:.22) ..(-.7,.72)--cycle
 (-.7,.32)
 ..controls +(-170:.1) and +(10:.1) ..(-.95,.3)
 ..controls +(60:.1) and +(-140:.1) ..(-.85,.45)--(-.65,.4)--cycle
 (-.3,.32)
 ..controls +(-170:.1) and +(10:.1) ..(-.45,.1)
 ..controls +(-40:.1) and +(-110:.1) ..(-.1,.37)--cycle
 ;
 }
 \fill[amber] \V;
 \draw\V;
 
 %-----------------
 \def\C{
 (-1.25,.83)
 ..controls +(-45:.2) and +(130:.2) ..(-1,.58)
 ..controls +(35:.2) and +(130:.52) ..(.05,.52)
 ..controls +(-90:.1) and +(-110:.1) ..(.05,.52)--(.04,.44)
 ..controls +(-150:.5) and +(-40:.2) ..(-1,.53)
 ..controls +(-140:.1) and +(40:.1) ..(-1.3,.42)
 ..controls +(60:.2) and +(-55:.2) ..cycle
 ;
 }
 
 \fill[amber] \C;
 \draw\C;
 %-----------------
 \def\Cn{
 (-1,.57)
 ..controls +(35:.1) and +(130:.4) ..(.01,.52)
 ..controls +(-140:.4) and +(-40:.2) ..cycle
 ;
 }
 %\draw[ecru!70!black]\Cn;
 \fill[amber!70] \Cn;
 \draw (-.3,.7)
 ..controls +(-120:.1) and +(120:.2) ..(-.3,.34);
 
 \draw[fill=white] (-.22,.55) circle (.08);
 \draw[fill=black] (-.22,.55) circle (.048);
 
 }}

 \path (-1,-2.5)pic[xscale=-.35,yscale=.35]{ca}
 (1.6,1)pic[xscale=-.15,yscale=.15,rotate=-40]{chim_boi_ca}; 
 \end{tikzpicture}
 }
\shortans[]{2{,}8}
 \loigiai{Ta có $A(1{,}5;1;-0{,}5)$ và $C(1;3;2)$ suy ra $\overrightarrow{AC}(-0{,}5;2;2{,}5)$ và $\overrightarrow{AB}=(a-1{,}5;b-1;0{,}5)$.\\
 Vì $A$, $B$, $C$ thẳng hàng nên ta có $\overrightarrow{AB}=k\overrightarrow{AC}$. Suy ra
 \[\heva{&a-1{,}5=k(-0{,}5)\\
 &b-1=2k\\
 &0{,}5=2{,}5k}
 \Leftrightarrow\heva{&k=\dfrac15\\
 &a=1{,}5-\dfrac{0{,}5}{5}=\dfrac75\\
 &b=1+\dfrac{2}{5}=\dfrac75.}
 \]
 Suy ra $B\left(\dfrac75;\dfrac75;0\right)$.\\
 Vậy $T=\dfrac75+\dfrac75=2{,}8$.
}\end{ex}

\begin{ex}%[2-H2B4-SO-13-2425 (Nguồn Đề 13 - Bài 4)]%[VN-MT-7, Lê Văn Hiếu]%[2H2H2-6]
 Một căn phòng dạng hình hộp chữ nhật với chiều dài $8$ m, rộng $6$ m và cao $4$ m có hai chiếc quạt treo tường. Chiếc quạt $A$ treo chính giữa bức tường $8$ m và cách trần $1$ m, chiếc quạt $B$ treo chính giữa bức tường $6$ m và cách trần $1{,}5$ m. (Tham khảo hình vẽ minh họa).
 \begin{center}
 \begin{tikzpicture}[line join=round, line cap=round,scale=.6,transform shape,>=stealth]
 \definecolor{amber}{rgb}{1.0, 0.75, 0.0}%mau non
 \definecolor{antiquebrass}{rgb}{0.8, 0.58, 0.46}%mau da
 \definecolor{antiquewhite}{rgb}{0.98, 0.92, 0.84}%mau ao
 \definecolor{cadmiumgreen}{rgb}{0.0, 0.42, 0.24}%mau quan
 \definecolor{cadetblue}{rgb}{0.37, 0.62, 0.63}%mau but
 \definecolor{brown(traditional)}{rgb}{0.59, 0.29, 0.0}%mau giay
 \definecolor{brilliantlavender}{rgb}{0.96, 0.73, 1.0}%màu sơn tím
 \definecolor{brightube}{rgb}{0.82, 0.62, 0.91}%màu sơn tím đậm
 %---------------màu quạt
 \definecolor{burntorange}{rgb}{0.8, 0.33, 0.0}
 \definecolor{arsenic}{rgb}{0.23, 0.27, 0.29}
 \definecolor{battleshipgrey}{rgb}{0.52, 0.52, 0.51}
 \clip (1,-1) rectangle (16,12);
 %\draw[gray!50] (-3,-3) grid (3,4);
 
 \definecolor{burntsienna}{rgb}{0.91, 0.45, 0.32}
 \tikzset{mai/.pic={
 \def\mainha{
 (.5,2)
 foreach \n in {1,2,...,22} { -- ++ (0,0) -- ++ (0,1) -- ++ (1,0) -- ++ (0,-1) } -- cycle
 
 (.5,1)
 foreach \n in {1,2,...,22} { -- ++ (0,0) -- ++ (0,1) -- ++ (1,0) -- ++ (0,-1) } -- cycle
 
 (.5,0)
 foreach \n in {1,2,...,22} { -- ++ (0,0) -- ++ (0,1) -- ++ (1,0) -- ++ (0,-1) } -- cycle
 
 (.5,-1)
 foreach \n in {1,2,...,22} { -- ++ (0,0) -- ++ (0,1) -- ++ (1,0) -- ++ (0,-1) } -- cycle
 
 (.5,-2)
 foreach \n in {1,2,...,22} { -- ++ (0,0) -- ++ (0,1) -- ++ (1,0) -- ++ (0,-1) } -- cycle
 
 (.5,-3)
 foreach \n in {1,2,...,22} { -- ++ (0,0) -- ++ (0,1) -- ++ (1,0) -- ++ (0,-1) } -- cycle
 
 (.5,-4)
 foreach \n in {1,2,...,22} { -- ++ (0,0) -- ++ (0,1) -- ++ (1,0) -- ++ (0,-1) } -- cycle
 ;
 }
 \clip (0.5,-3)--(17.5,-3)--(17.5,3)--(.5,3)--cycle;
 \draw[white,fill=burntsienna!70] \mainha;
 \draw (17.5,-3)--(17.5,3)--(17.5,9);
 }}
 
 \fill[brilliantlavender] (16,-1)--(16,9)--(12,12)--(12,3)--cycle;
 \fill[brightube] (12,13)--(12,3)--(-18,3)--(-18,13)--cycle;
 \begin{scope}
 \clip[draw] (1,3)--(12,3)--(16,-1)--(5,-1)--cycle;
 \path
 (-2.5,0)pic[scale=1,xslant=-1]{mai}%
 ;
 \end{scope}
 
 \path
 (12,3) coordinate (O)
 (3,3) coordinate (x)
 (12,12) coordinate (z)
 (15,0) coordinate (y)
 ;
 
 \foreach\p/\g in {y/70,x/80, z/-30,O/-120}
 {
 \node at (\p) [shift=(\g:3mm)] {$\p$};
 }
 
 \draw[line width=.5mm,->] (O)--(z) ;
 \draw[line width=.5mm,->] (O)--(x);
 \draw[line width=.5mm,->] (O)--(y); 
 \draw[red,line width=.5mm,<->] ($(O)+(0,4)$)--($(x)+(-2,4)$) node[midway, above]{$8$ m};
 
 %===========================A FAN
 \tikzset{fan/.pic={
 %chân quạt
 \draw[fill=battleshipgrey](-.2,.8)--(.2,.8)--(.4,-2.2)
 ..controls +(-120:.3) and +(-60:.3) ..(-.4,-2.2)--cycle;
 %---Nút bấm
 \foreach \i in{-1.95,-1.7}{%-1.45
 \draw[fill=arsenic](-.15,\i) rectangle (.15,\i+.15);
 }
 %-----------------------------------------------------
 \draw[black](0,.8) circle (2.25cm);
 \draw[black](0,.8) circle (2.15cm);
 
 \draw[black](0,.8) circle (1.42cm);
 \draw[black](0,.8) circle (1.48cm);
 
 \draw[fill=black](0,.8) circle (6mm);
 \draw[fill=arsenic](0,.8) circle (5mm);
 \def\N{
 (0,.8)
 ..controls +(145:1.3) and +(170:1) ..(0,2.8)
 ..controls +(-10:1.4) and +(-20:1) ..(.6,1.76)
 ..controls +(160:.4) and +(100:.4) ..(.2,.9)--cycle
 ;
 }
 \foreach \i/\j/\k in {0/0/0,120/.7/-1.2,240/-.7/-1.2}
 {
 \draw[black,rotate=\i,shift={(\j,\k)}]\N;
 \fill[burntorange,rotate=\i,shift={(\j,\k)}] \N;
 }
 
 %lồng quạt
 \def\r{2.15}
 \foreach \i in {0,15,25,35,...,365}
 \draw[double] ($(\i:\r)+(0,.8)$)--(0,.8);
 
 \draw[fill=arsenic](0,.8) circle (3.5mm);

 }}
 \path
 (6.5,9.5)pic[scale=.6]{fan}
 (14,7.5)pic[scale=.55,yslant=-.3]{fan};
 \end{tikzpicture}
 \end{center}
 Hỏi khoảng cách giữa hai chiếc quạt $A$, $B$ cách nhau bao nhiêu mét (làm tròn đến hàng phần trăm).
 
 \shortans[]{5{,}02}
 \loigiai{
 Chọn hệ trục tọa độ như hình vẽ, khi đó ta có tọa độ quạt $A$ là $A(4;0;3)$ và tọa độ quạt $B$ là $B\left(0;3;\dfrac{5}{2}\right)$.\\
 Khi đó $\overrightarrow{AB}=\left(-4;3;-\dfrac{1}{2}\right)$.\\
 Vậy khoảng cách giữa hai quạt $A$, $B$ là $AB= \sqrt{(-4)^2+3^2+\left(-\dfrac{1}{2}\right)^2} \approx 5{,}02$.}
\end{ex}
\Closesolutionfile{ans}
 
\begin{indapan}
	{ans/ans\currfilebase}
\end{indapan}


% \begin{name}
 {Biên soạn: Bùi Lương Phúc \\ Phản biện: Trần Bảo Hiên}
{Đề ôn tập chương II}
\end{name}


\TN
\Opensolutionfile{ans}[ans/ans\currfilebase-Phan-I]

\begin{ex}%[2-H2B4-SO-14-2425 (Nguồn Đề 5 - Bài 4- Ôn tập chương II)]%[VN-MT-7, Bùi Lương Phúc]%[2H2H1-1]
\immini{Cho hình hộp $ABCD.A'B'C'D'$ (tham khảo hình bên).
Vectơ $\overrightarrow{u}=\overrightarrow{BB'}+\overrightarrow{BA}+\overrightarrow{BC}$ bằng vectơ nào dưới đây?
\choice[2]
{$\overrightarrow{BD}$}
{\True $\overrightarrow{BD'}$}
{$\overrightarrow{BC}$}
{$\overrightarrow{BA'}$}
}
{\begin{tikzpicture}
[scale=0.8, font=\footnotesize, line join=round, line cap=round, >=stealth]
\coordinate (A) at (0,0);
\coordinate (B) at (0.8,1.3);
\coordinate (C) at (4.5,1.3);
\coordinate (D) at ($(A)+(C)-(B)$);
\coordinate (A') at ($(A)+(0.3,3)$);
\coordinate (B') at ($(A')+(B)-(A)$);
\coordinate (C') at ($(B')+(C)-(B)$);
\coordinate (D') at ($(A')+(D)-(A)$);
\draw (A')--(A)--(D)--(D')--(A')--(B')--(C')--(C)--(D) (C')--(D');
\draw[dashed,->](B)--(B');
\draw[dashed,->] (B)--(A);
\draw[dashed,->] (B)--(C);
\foreach \x/\y in {A/-180,B/180,C/0,D/0,A'/180,B'/180,C'/0,D'/0} \fill[black](\x) circle (1pt) ($(\x)+(\y:3mm)$) node{$\x$};
\end{tikzpicture}}
\loigiai{
Ta có $\overrightarrow{u}=\overrightarrow{BB'}+\overrightarrow{BA}+\overrightarrow{BC}=\overrightarrow{BB'}+\left(\overrightarrow{BA}+\overrightarrow{BC}\right)=\overrightarrow{BB'}+\overrightarrow{BD}=\overrightarrow{BD'}$.
}
\end{ex}

\begin{ex}%[2-H2B4-SO-14-2425 (Nguồn Đề 5 - Bài 4- Ôn tập chương II)]%[VN-MT-7, Bùi Lương Phúc]%[2H2H1-3]
\immini{Cho tứ diện $ABCD$ có $AB=AC=AD$ và $\widehat{BAC}=\widehat{BAD}=60^\circ$. Góc giữa hai vectơ $\overrightarrow{AB}$ và $\overrightarrow{CD}$ có số đo bằng
\choice[2]
{$60^\circ$}
{$45^\circ$}
{\True $90^\circ$}
{$120^\circ$}
}
{ \begin{tikzpicture}
[scale=0.8, font=\footnotesize, line join=round, line cap=round, >=stealth]
\coordinate (A) at (0.4,4);
\coordinate (B) at (-1.5,1.5);
\coordinate (D) at (3,1.5);
\coordinate (C) at (0,0);
\draw (A)--(B)--(C)--(D)--cycle (A)--(C);
\draw pic[draw,angle radius=5mm] {angle = B--A--C};
\draw[thick] pic[draw,angle radius=3.5mm] {angle = B--A--D};
\draw[dashed](B)--(D);
\foreach \x/\y in {A/90,B/-120,C/-90,D/-90} \fill[black](\x) circle (1pt) ($(\x)+(\y:3mm)$) node{$\x$};
\draw(-0.5,2.7)--(-0.6,2.8);
\draw (0.13,2) -- (0.27,2);
\draw (1.65,2.7) -- (1.75,2.8);
\end{tikzpicture}}
\loigiai{
\begin{center}
 \begin{tikzpicture}
 [scale=0.8, font=\footnotesize, line join=round, line cap=round, >=stealth]
 \coordinate (A) at (0.4,4);
 \coordinate (B) at (-1.5,1.5);
 \coordinate (D) at (3,1.5);
 \coordinate (C) at (0,0);
 \draw (A)--(B)--(C)--(D)--cycle (A)--(C);
 \draw pic[draw,angle radius=5mm] {angle = B--A--C};
 \draw[thick] pic[draw,angle radius=3.5mm] {angle = B--A--D};
 \coordinate (m) at ($(A)+(0.3,-0.8)$);
 \coordinate (n) at ($(A)+(-0.4,-1)$);
 \node [shift=(m)]{$60^\circ$};
 \node [shift=(n)]{$60^\circ$};
 \draw[dashed](B)--(D);
 \foreach \x/\y in {A/90,B/-120,C/-90,D/-90} \fill[black](\x) circle (1pt) ($(\x)+(\y:3mm)$) node{$\x$};
 \draw(-0.5,2.7)--(-0.6,2.8);
 \draw (0.13,2) -- (0.27,2);
 \draw (1.65,2.7) -- (1.75,2.8);
 \end{tikzpicture}
\end{center}
Ta có
\begin{align*}
 \overrightarrow{AB}\cdot \overrightarrow{CD}=&\, \overrightarrow{AB}\cdot (\overrightarrow{AD}-\overrightarrow{AC})\\
 =&\, \overrightarrow{AB}\cdot \overrightarrow{AD}-\overrightarrow{AB}\cdot \overrightarrow{AC} \\
 =&\, AB\cdot AD\cdot \cos 60{}^\circ -AB\cdot AC\cdot \cos 60^\circ =0 
\end{align*}
Suy ra $\overrightarrow{AB}\perp\overrightarrow{CD}\Rightarrow \left(\overrightarrow{AB},\overrightarrow{CD}\right)=90^\circ$.
}
\end{ex}

\begin{ex}%[2-H2B4-SO-14-2425 (Nguồn Đề 5 - Bài 4- Ôn tập chương II)]%[VN-MT-7, Bùi Lương Phúc]%[2H2N2-2]
Trong KG $Oxyz$, cho điểm $M$ thỏa mãn $\overrightarrow{OM}=2\overrightarrow{i}+3\overrightarrow{j}-\overrightarrow{k}$. Tọa độ của điểm $M$ là
\choice
{$(2;3;1)$}
{$(-2;-3;1)$}
{$(2;-1;3)$}
{\True $(2;3;-1)$}
\loigiai{
Ta có $\overrightarrow{OM}=2\overrightarrow{i}+3\overrightarrow{j}-\overrightarrow{k}\Rightarrow \overrightarrow{OM}=(2;3;-1)\Rightarrow M=(2;3;-1)$.
}
\end{ex}

\begin{ex}%[2-H2B4-SO-14-2425 (Nguồn Đề 5 - Bài 4- Ôn tập chương II)]%[VN-MT-7, Bùi Lương Phúc]%[2H2N2-3]
Trong KG $Oxyz$, cho vectơ $\overrightarrow{u}=2\overrightarrow{i}-\dfrac{1}{2}\overrightarrow{j}+4\overrightarrow{k}$. Tọa độ của vectơ $\overrightarrow{u}$ là
\choice
{\True $\left(2;-\dfrac{1}{2};4 \right)$}
{$\left(2;\dfrac{1}{2};4 \right)$}
{$(2;1;4)$}
{$\left(\dfrac{1}{2};-2;\dfrac{1}{4} \right)$}
\loigiai{
Áp dụng định lí: $\overrightarrow{u}=(a;b;c)\Leftrightarrow \overrightarrow{u}=a\overrightarrow{i}+b\overrightarrow{j}+c\overrightarrow{k}$. 
Ta có $\overrightarrow{u}=2\overrightarrow{i}-\dfrac{1}{2}\overrightarrow{j}+4\overrightarrow{k}=2\overrightarrow{i}+\left(-\dfrac{1}{2}\right)\overrightarrow{j}+4\overrightarrow{k}$, suy ra tọa độ của vectơ $\overrightarrow{u}$ là $\left(2;-\dfrac{1}{2};4\right)$.
}
\end{ex}

\begin{ex}%[2-H2B4-SO-14-2425 (Nguồn Đề 5 - Bài 4- Ôn tập chương II)]%[VN-MT-7, Bùi Lương Phúc]%[2H2H2-5]
Trong KG $Oxyz$, cho hai vectơ $\overrightarrow{u}=(-1;3;-2)$ và $\overrightarrow{v}=(2;5;-1)$. Vectơ nào dưới đây vuông góc với cả hai vectơ $\overrightarrow{u}$ và $\overrightarrow{v}$?
\choice
{${\overrightarrow{w}_4}=(-8;-9;-1)$}
{\True ${\overrightarrow{w}_2}=(7;-5;-15)$}
{${\overrightarrow{w}_1}=(7;5;-15)$}
{${\overrightarrow{w}_3}=(1;8;-3)$}
\loigiai{
Ta có $\left[ \overrightarrow{u},\overrightarrow{v} \right]=\left(\left| \begin{matrix}
3 & -2 \\
5 & -1 \\
\end{matrix} \right|;\left| \begin{matrix}
-2 & -1 \\
-1 & 2 \\
\end{matrix} \right|;\left| \begin{matrix}
-1 & 3 \\
2 & 5 \\
\end{matrix} \right|\right) \Rightarrow \left[\overrightarrow{u},\overrightarrow{v} \right]= (7;-5;-15)$.\\
Vì $\left[\overrightarrow{u},\overrightarrow{v} \right]$ vuông góc với cả hai vectơ $\overrightarrow{u}$ và $\overrightarrow{v}$ nên vectơ $\overrightarrow{w}_2 =(7;-5;-15)$ vuông góc với cả hai vectơ $\overrightarrow{u}$ và $\overrightarrow{v}$.
}
\end{ex}

\begin{ex}%[2-H2B4-SO-14-2425 (Nguồn Đề 5 - Bài 4- Ôn tập chương II)]%[VN-MT-7, Bùi Lương Phúc]%[2H2N2-3]
Trong KG $Oxyz$, cho vectơ $\overrightarrow{a}=(-1;4;2)$. Toạ độ của vectơ $-2\overrightarrow{a}$ là
\choice
{$(-2;8;4)$}
{\True $(2;-8;-4)$}
{$(-2;-8;-4)$}
{$(2;8;4)$}
\loigiai{
Ta có $-2\overrightarrow{a}=\left((-2)\cdot (-1);(-2)\cdot 4;(-2)\cdot 2\right) \Rightarrow -2\overrightarrow{a}=(2;-8;-4)$.
}
\end{ex}

\begin{ex}%[2-H2B4-SO-14-2425 (Nguồn Đề 5 - Bài 4- Ôn tập chương II)]%[VN-MT-7, Bùi Lương Phúc]%[2H2N2-4]
Trong KG $Oxyz$, cho vectơ $\overrightarrow{u}=(2;-1;2)$. Độ dài của vectơ $\overrightarrow{u}$ bằng
\choice
{$\sqrt{7}$}
{\True $3$}
{$9$}
{$2$}
\loigiai{
Độ dài của vectơ $\overrightarrow{u}$ là
$\left| \overrightarrow{u} \right|=\sqrt{2^2+(-1)^2+2^2}=3$.
}
\end{ex}

\begin{ex}%[2-H2B4-SO-14-2425 (Nguồn Đề 5 - Bài 4- Ôn tập chương II)]%[VN-MT-7, Bùi Lương Phúc]%[2H2N2-4]
Trong KG $Oxyz$, cho hai vectơ $\overrightarrow{u}=(-2;1;5)$ và $\overrightarrow{v}=(0;-3;1)$. Tích vô hướng của hai vectơ $\overrightarrow{u}$ và $\overrightarrow{v}$ bằng
\choice
{$10\sqrt{3}$}
{$0$}
{\True $2$}
{$-2$}
\loigiai{
Tích vô hướng của hai vectơ $\overrightarrow{u}$ và $\overrightarrow{v}$ là\\
\[\overrightarrow{u}\cdot \overrightarrow{v}=-2\cdot 0+1\cdot (-3)+5\cdot 1=2.\]
}
\end{ex}

\begin{ex}%[2-H2B4-SO-14-2425 (Nguồn Đề 5 - Bài 4- Ôn tập chương II)]%[VN-MT-7, Bùi Lương Phúc]%[2H2H2-2]
Trong KG $Oxyz$, cho điểm $A(2;-4;3)$ và vectơ $\overrightarrow{u}=(2;2;7)$. Biết rằng $\overrightarrow{u}=\overrightarrow{AB}$, tính tọa độ điểm $B$.
\choice
{$(2;6;4)$}
{$(1;3;2)$}
{\True $(4;-2;10)$}
{$(2;-1;5)$}
\loigiai{
Gọi $B(x;y;z)$, ta có $\overrightarrow{AB}=(x-2;y+4;z-3)$.\\
Vì $\overrightarrow{u}=\overrightarrow{AB}$ nên ta có \begin{center}
$\heva{&x-2=2 \\&y+4=2 \\&z-3=7}\Rightarrow \heva{&x=4 \\&y=-2 \\&z=10}\Rightarrow B(4;-2;10)$.
\end{center}
}
\end{ex}

\begin{ex}%[2-H2B4-SO-14-2425 (Nguồn Đề 5 - Bài 4- Ôn tập chương II)]%[VN-MT-7, Bùi Lương Phúc]%[2H2N2-2]
Trong KG $Oxyz$, cho hai điểm $A(2;3;-4)$ và $B(0;1;6)$. Trung điểm $M$ của đoạn thẳng $AB$ có tọa độ là
\choice
{$(-2;-2;10)$}
{$(1;2;2)$}
{\True $(1;2;1)$}
{$(2;2;-10)$}
\loigiai{
Gọi trung điểm của đoạn thẳng $AB$ là $M(x;y;z)$, ta có
\[x=\dfrac{1+0}{2}=1, 
y=\dfrac{3+1}{2}=2, 
z=\dfrac{-4+6}{2}=1.\]
Vậy $M=(1;2;1)$.
}
\end{ex}

\begin{ex}%[2-H2B4-SO-14-2425 (Nguồn Đề 5 - Bài 4- Ôn tập chương II)]%[VN-MT-7, Bùi Lương Phúc]%[2H2N2-2]
Trong KG $Oxyz$, cho tam giác $MNP$ có $M(1;1;2)$, $N(-1;0;2)$, $P(3;-7;5)$. Trọng tâm $G$ của tam giác $MNP$ có tọa độ là
\choice
{\True $(1;-2;3)$}
{$(-1;2;3)$}
{$(1;2;3)$}
{$(3;-6;7)$}
\loigiai{
Theo công thức tính tọa độ trọng tâm của tam giác ta có\\
\begin{center}
$\heva{& x_G=\dfrac{1-1+3}{3}=1 \\& y_G=\dfrac{1+0-7}{3}=-2 \\& z_G=\dfrac{2+2+5}{3}=3}\Rightarrow G(1;-2;3)$.
\end{center}
}
\end{ex}

\begin{ex}%[2-H2B4-SO-14-2425 (Nguồn Đề 5 - Bài 4- Ôn tập chương II)]%[VN-MT-7, Bùi Lương Phúc]%[2H2H2-2]
Trong KG $Oxyz$, cho hình hộp $ABCD.A'B'C'D'$ có $A(0;0;0)$, $B(3;0;0)$, $D(0;3;0)$ và $D'(0;3;-3)$. Tìm tọa độ đỉnh $A'$ của hình hộp.
\choice
{\True $(0;0;-3)$}
{$(0;3;0)$}
{$(3;3;0)$}
{$(0;0;3)$}
\loigiai{
\begin{center}
\begin{tikzpicture}
[scale=0.8, font=\footnotesize, line join=round, line cap=round, >=stealth]
\coordinate (A) at (0,0);
\coordinate (B) at (0.8,1.3);
\coordinate (C) at (4.5,1.3);
\coordinate (D) at ($(A)+(C)-(B)$);
\coordinate (A') at ($(A)+(0.3,3)$);
\coordinate (B') at ($(A')+(B)-(A)$);
\coordinate (C') at ($(B')+(C)-(B)$);
\coordinate (D') at ($(A')+(D)-(A)$);
\draw (A')--(A)--(D)--(D')--(A')--(B')--(C')--(C)--(D) (C')--(D');
\draw[dashed] (A)--(B)--(B');
\draw[dashed] (B)--(C);
\foreach \x/\y in {A/-180,B/180,C/0,D/0,A'/180,B'/180,C'/0,D'/0} \fill[black](\x) circle (1pt) ($(\x)+(\y:3mm)$) node{$\x$};
\end{tikzpicture}
\end{center}
Ta có $\overrightarrow{AD}=(0;3;0)$.\\
Gọi $A'(x_2;y_2;z_2)\Rightarrow \overrightarrow{A'D'}=(0-x_2;3-y_2;-3-z_2)$.\\
Vì $ADD'A'$ là hình bình hành $\Rightarrow \overrightarrow{A'D'}=\overrightarrow{AD}\Leftrightarrow
\heva{& 0-x_2=0 \\& 3-y_2=3 \\& -3-z_2=0}
\Leftrightarrow \heva{& x_2=0 \\& y_2=0 \\& z_2=-3}
\Rightarrow {A}'(0;0;\,-3)$.
}
\end{ex}
\Closesolutionfile{ans}


\TNTF
\Opensolutionfile{ans}[ans/ans\currfilebase-Phan-II]

\begin{ex}%[2-H2B4-SO-14-2425 (Nguồn Đề 5 - Bài 4- Ôn tập chương II)]%[VN-MT-7, Bùi Lương Phúc]%[2H2V1-3]
\immini{
Cho tứ diện $ABCD$ có $AB$, $AC$ và $AD$ đôi một vuông góc. Gọi $M$ là trung điểm của cạnh $BC$, $H$ là trung điểm của đoạn thẳng $MD$. Cho $AB=AC=a$.
\choiceTF
{\True $\overrightarrow{DM}=\dfrac{1}{2}\overrightarrow{DB}+\dfrac{1}{2}\overrightarrow{DC}$}
{$\overrightarrow{AH}=\dfrac{1}{4}\overrightarrow{AB}+\dfrac{1}{2}\overrightarrow{AC}+\dfrac{1}{2}\overrightarrow{AD}$}
{\True $\overrightarrow{AB}\cdot \overrightarrow{AH}=\dfrac{1}{4}a^2$}
{Góc giữa hai vectơ $\overrightarrow{AH}$ và $\overrightarrow{BC}$ bằng $60^\circ $}
}{\begin{tikzpicture}
[scale=0.7, font=\footnotesize, line join=round, line cap=round, >=stealth]
\coordinate (A) at (0,0);
\coordinate (B) at (-1,-2);
\coordinate (C) at (6,0);
\coordinate (D) at (0,4);
\draw (B)--(D)--(C);
\draw[->] (B)--(C);
\draw[dashed] (B)--(A)--(C) (A)--(D);
\coordinate (M) at ($(B)!0.5!(C)$);
\coordinate (H) at ($(M)!0.5!(D)$);
\draw (D)--(M);
\draw[->, dashed] (A)--(H);
\foreach \x/\g in {A/140,B/-90,C/0,D/90,M/-90,H/0} \fill[black](\x) circle (1pt) ($(\x)+(\g:5mm)$) node{$\x$};
\end{tikzpicture}}
\loigiai{
\begin{itemchoice}
\itemch \textbf{Đúng}.\\
Vì $M$ là trung điểm của cạnh $BC$ nên \[\overrightarrow{DM}=\dfrac{1}{2}\left(\overrightarrow{DB}+\overrightarrow{DC} \right)=\dfrac{1}{2}\overrightarrow{DB}+\dfrac{1}{2}\overrightarrow{DC}.\]
\itemch \textbf{Sai}.\\
Ta có
$\overrightarrow{AH}
=\dfrac{1}{2}\left(\overrightarrow{AM}+\overrightarrow{AD} \right), \overrightarrow{AM}
=\dfrac{1}{2}\left(\overrightarrow{AB}+\overrightarrow{AC} \right)$
\begin{align*}
\Rightarrow \overrightarrow{AH}
=&\, \dfrac{1}{2}\left[ \dfrac{1}{2}\left(\overrightarrow{AB}+\overrightarrow{AC} \right)+\overrightarrow{AD} \right]\\
=&\, \dfrac{1}{2}\cdot \dfrac{1}{2}\left(\overrightarrow{AB}+\overrightarrow{AC}\right)+\dfrac{1}{2}\overrightarrow{AD}\\
=&\, \dfrac{1}{4}\overrightarrow{AB}+\dfrac{1}{4}\overrightarrow{AC}+\dfrac{1}{2}\overrightarrow{AD}
\end{align*}
Rõ ràng $\overrightarrow{AH}=\dfrac{1}{4}\overrightarrow{AB}+\dfrac{1}{2}\overrightarrow{AC}+\dfrac{1}{2}\overrightarrow{AD}$ sai vì $\overrightarrow{AC} \ne\overrightarrow{0}$.
\itemch \textbf{Đúng}.\\
Ta có
\begin{align*}
\overrightarrow{AB}\cdot \overrightarrow{AH}
=&\, \overrightarrow{AB}\cdot \left[ \dfrac{1}{4}\overrightarrow{AB}+\dfrac{1}{4}\overrightarrow{AC}+\dfrac{1}{2}\overrightarrow{AD} \right]\\
=&\, \overrightarrow{AB}\cdot \left(\dfrac{1}{4}\overrightarrow{AB} \right)+\overrightarrow{AB}\cdot \left(\dfrac{1}{4}\overrightarrow{AC} \right)+\overrightarrow{AB}\cdot \left(\dfrac{1}{2}\overrightarrow{AD} \right)\\ =&\, \dfrac{1}{4}{{\overrightarrow{AB}}^2}+\dfrac{1}{4}\overrightarrow{AB}\cdot \overrightarrow{AC}+\dfrac{1}{2}\overrightarrow{AB}\cdot \overrightarrow{AD}.
\end{align*}
Vì $AB\perp AC$ nên $\overrightarrow{AB}\cdot \overrightarrow{AC}=0$, $AB\perp AD$ nên $\overrightarrow{AB}\cdot \overrightarrow{AD}=0$.\\
Vậy $\overrightarrow{AB}\cdot \overrightarrow{AH}=\dfrac{1}{4}{{\overrightarrow{AB}}^2}=\dfrac{1}{4}{\left| \overrightarrow{AB} \right|^2}=\dfrac{1}{4}a^2$.
\itemch \textbf{Sai}.\\
Ta có $\overrightarrow{BC}=\overrightarrow{AC}-\overrightarrow{AB}$, $\overrightarrow{AH}=\dfrac{1}{4}\overrightarrow{AB}+\dfrac{1}{4}\overrightarrow{AC}+\dfrac{1}{2}\overrightarrow{AD}$.\\
\begin{align*}
\overrightarrow{BC}\cdot \overrightarrow{AH}=&\, \left(\overrightarrow{AC}-\overrightarrow{AB} \right)\cdot \left(\dfrac{1}{4}\overrightarrow{AB}+\dfrac{1}{4}\overrightarrow{AC}+\dfrac{1}{2}\overrightarrow{AD} \right)\\
=&\, \dfrac{1}{4}\overrightarrow{AC}\cdot \overrightarrow{AB}+\dfrac{1}{4}{\overrightarrow{AC}}^2+\dfrac{1}{2}\overrightarrow{AC}\cdot \overrightarrow{AD}-\dfrac{1}{4}{\overrightarrow{AB}}^2-\dfrac{1}{4}\overrightarrow{AB}\cdot \overrightarrow{AC}-\dfrac{1}{2}\overrightarrow{AB}\cdot \overrightarrow{AD} \\
=&\, 0+\dfrac{1}{4}\left| \overrightarrow{AC} \right|^2+0-\dfrac{1}{4}\left| \overrightarrow{AB} \right|^2-0-0\\
=&\, \dfrac{1}{4}a^2-\dfrac{1}{4}a^2=0.
\end{align*} 
Vậy góc giữa hai vectơ $\overrightarrow{AH}$ và $\overrightarrow{BC}$ bằng $90^\circ $.
\end{itemchoice}
}
\end{ex}

\begin{ex}%[2-H2B4-SO-14-2425 (Nguồn Đề 5 - Bài 4- Ôn tập chương II)]%[VN-MT-7, Bùi Lương Phúc]%[2H2H2-3]
Trong KG $Oxyz$, cho hai điểm $M(-4;3;-1)$ và $N(2;-1;-3)$.
\choiceTF
{\True Tọa độ vectơ $\overrightarrow{OM}$ bằng $(-4;3;-1)$}
{\True Điểm đối xứng của $M$ qua trục cao có tọa độ là $(4;-3;-1)$}
{Gọi $G$ là trọng tâm của tam giác $OMN$. \\
Hình chiếu của điểm $G$ trên mặt phẳng $\left(Oxy \right)$ có tọa độ là $\left(0;0;-\dfrac{4}{3} \right)$}
{\True Nếu $\overrightarrow{v}=3\overrightarrow{i}-2\overrightarrow{j}-\overrightarrow{k}$ thì hai vectơ $\overrightarrow{MN}$ và $\overrightarrow{v}$ cùng hướng}
\loigiai{
\begin{itemchoice}

\itemch \textbf{Đúng}.\\
Tọa độ vectơ $\overrightarrow{OM}$ cũng là tọa độ điểm $M(-4;3;-1)$.
\itemch \textbf{Đúng}.\\
Điểm đối xứng của $M$ qua trục cao có tọa độ là $(4;-3;-1)$.
\itemch \textbf{Sai}.\\
Ta có trọng tâm tam giác $OMN$ là $G\left(-\dfrac{2}{3};\dfrac{2}{3};-\dfrac{4}{3} \right)$.\\
Hình chiếu của $G$ trên $\left(Oxy \right)$ có tọa độ là $\left(-\dfrac{2}{3};\dfrac{2}{3};0 \right)$.
\itemch \textbf{Đúng}.\\
$\overrightarrow{MN}=(6;-4;-2)$.\\
Vì $\overrightarrow{v}(3;-2;-1)$ nên $\overrightarrow{MN}=2\overrightarrow{v}$.\\
Vậy $\overrightarrow{MN}$ và $\overrightarrow{v}$ cùng hướng.
\end{itemchoice}
}
\end{ex}

\begin{ex}%[2-H2B4-SO-14-2425 (Nguồn Đề 5 - Bài 4- Ôn tập chương II)]%[VN-MT-7, Bùi Lương Phúc]%[2H2H2-4]
Trong KG $Oxyz$, cho vectơ $\overrightarrow{u}=(1;2;3)$ và điểm $A(2;7;4)$.
\choiceTF
{\True Hình chiếu của điểm $A$ trên trục tung có tọa độ là $(0;7;0)$}
{$\left| 2\overrightarrow{u}-\overrightarrow{j} \right|=8$}
{\True Một điểm $B$ nằm trên mặt phẳng $\left(Oxy \right)$ sao cho $\overrightarrow{AB}$ cùng phương với $\overrightarrow{u}=(1;2;3)$. Khi đó điểm $B$ có tọa độ là $\left(\dfrac{2}{3};\dfrac{13}{3};0 \right)$}
{$\sin^2\left(\overrightarrow{u},\overrightarrow{i} \right) +\sin^2\left(\overrightarrow{u},\overrightarrow{j} \right)+\sin ^2\left(\overrightarrow{u},\overrightarrow{k} \right)=1$}
\loigiai{
\begin{itemchoice}
\itemch \textbf{Đúng}.\\
Hình chiếu của điểm $A(2;7;4)$ trên trục tung có tọa độ là $(0;7;0)$.
\itemch \textbf{Sai}.\\
Ta có $2\overrightarrow{u}=(2;4;6);\overrightarrow{j}=(0;1;0)\Rightarrow 2\overrightarrow{u}-\overrightarrow{j}=(2;3;6)$.\\
Suy ra $\left| 2\overrightarrow{u}-\overrightarrow{j} \right|=\sqrt{2^2+3^2+6^2}=7$.
\itemch \textbf{Đúng}.\\
Vì điểm $B$ nằm trên mặt phẳng $\left(Oxy \right)$ nên ta gọi $B(x;y;0)$\\
$\Rightarrow \overrightarrow{AB}=(x-2;y-7;-4)$.\\
Vì $\overrightarrow{AB}$ cùng phương với $\overrightarrow{u}=(1;2;3)$ nên có số thực $k$ sao cho $\overrightarrow{AB}=k\overrightarrow{u}$
hay
\begin{align*}
\heva{& x-2=k \\& y-7=2k \\& -4=3k} \Leftrightarrow \heva{& k=-\dfrac{4}{3} \\& x=\dfrac{2}{3} \\& y=\dfrac{13}{3}.}
\end{align*}
Vậy $B\left(\dfrac{2}{3};\dfrac{13}{3};0 \right)$.
\itemch \textbf{Sai}.\\
Ta có
$\cos \left(\overrightarrow{u},\overrightarrow{i} \right)=\dfrac{\overrightarrow{u}\cdot \overrightarrow{i}}{\left| \overrightarrow{u} \right|\cdot\left| \overrightarrow{i} \right|}=\dfrac{1\cdot 1+2\cdot 0+3\cdot0}{\sqrt{1^2+2^2+3^2}\cdot 1}=\dfrac{1}{\sqrt{14}}$;\\
Tương tự, $\cos \left(\overrightarrow{u},\overrightarrow{j} \right)=\dfrac{2}{\sqrt{14}}$; $\cos \left(\overrightarrow{u},\overrightarrow{k} \right)=\dfrac{3}{\sqrt{14}}$\\
Vậy 
\begin{align*}
&\sin ^2\left(\overrightarrow{u},\overrightarrow{i} \right)+\sin ^2\left(\overrightarrow{u},\overrightarrow{j} \right)+\sin^2\left(\overrightarrow{u},\overrightarrow{k} \right)\\
=&\, 3-\left[ \cos^2\left(\overrightarrow{u},\overrightarrow{i} \right)+\cos ^2\left(\overrightarrow{u},\overrightarrow{j} \right)+\cos^2\left(\overrightarrow{u},\overrightarrow{k} \right) \right]\\
=&\, 3-\left(\dfrac{1}{14}+\dfrac{4}{14}+\dfrac{9}{14} \right)\\
=&\, 2.
\end{align*}
\end{itemchoice}
}
\end{ex}

\begin{ex}%[2-H2B4-SO-14-2425 (Nguồn Đề 5 - Bài 4- Ôn tập chương II)]%[VN-MT-7, Bùi Lương Phúc]%[2H2H2-4]
Trong KG $Oxyz$, cho bốn điểm $A(0;-2;1)$, $B(1;0;-2)$, $C(3;1;-2)$ và $D(-2;-2;-1)$.
\choiceTF
{Ba điểm $A$, $B$, $D$ thẳng hàng}
{\True Tam giác $ACD$ là tam giác vuông tại $A$}
{\True Góc giữa hai vectơ $\overrightarrow{AB}$ và $\overrightarrow{CD}$ là góc tù}
{Khoảng cách từ điểm $A$ đến đường thẳng $CD$ bằng $\dfrac{3\sqrt{210}}{35}$}
\loigiai{
\begin{itemchoice}
\itemch \textbf{Sai}.\\
Ta có $\overrightarrow{AB}=(1;2;-3)$; $\overrightarrow{AD}=(-2;0;-2)$.\\
Vì $\dfrac{1}{-2}\ne \dfrac{-3}{-2}$ nên
hai vectơ $\overrightarrow{AB}$ và $\overrightarrow{AD}$ không cùng phương. \\
Suy ra ba điểm $A$, $B$, $D$ không thẳng hàng.
\itemch \textbf{Đúng}.\\
Ta có $\overrightarrow{AC}=(3;3;-3)$, $\overrightarrow{AD}=(-2;0;-2)$.\\
$\overrightarrow{AC}\cdot\overrightarrow{AD}=3\cdot (-2)+3\cdot 0+(-3) \cdot (-2)=0\Rightarrow AC\perp AD$.\\
Suy ra tam giác $ACD$ là tam giác vuông tại $A$.
\itemch \textbf{Đúng}.\\
Ta có $\overrightarrow{AB}=(1;2;-3)$, $\overrightarrow{CD}=(-5;-3;1)$.\\
Vì $\overrightarrow{AB}\cdot \overrightarrow{CD}=1\cdot (-5)+2\cdot (-3)+(-3) \cdot 1=-14<0$ nên $\cos\left(\overrightarrow{AB},\overrightarrow{CD} \right)<0\Rightarrow \left(\overrightarrow{AB},\overrightarrow{CD} \right)$ là góc tù.
\itemch \textbf{Sai}.\\
Ta có $AC=\sqrt{3^2+3^2+(-3)^2}=3\sqrt{3}$;\\ $AD=\sqrt{(-2)^2+0^2+2^2}=2\sqrt{2}$;\\
$CD=\sqrt{(-5)^2+(-3)^2+1^2}=\sqrt{35}$.\\
Tam giác $ACD$ là tam giác vuông tại $A$ nên
$S_{ACD}=\dfrac{1}{2}AC\cdot AD=3\sqrt{6}$.\\
Khoảng cách từ điểm $A$ đến đường thẳng $CD$ chính là chiều cao kẻ từ $A$ của tam giác $ACD$.\\
\[\mathrm{d}(A,CD)=\dfrac{2S_{ACD}}{CD}=\dfrac{6\sqrt{6}}{\sqrt{35}}=\dfrac{6\sqrt{210}}{35}.\]
\end{itemchoice}
}
\end{ex}
\Closesolutionfile{ans}


\TNSA
\Opensolutionfile{ans}[ans/ans\currfilebase-Phan-III]

\begin{ex}%[2-H2B4-SO-14-2425 (Nguồn Đề 5 - Bài 4- Ôn tập chương II)]%[VN-MT-7, Bùi Lương Phúc]%[2H2V2-4]
Cho tứ diện $ABCD$ có $AB=AC=AD=a$, $\widehat{BAC}=\widehat{BAD}=60^\circ $ và $\widehat{CAD}=90^\circ $. Gọi $I$ là điểm trên cạnh $AB$ sao cho $AI=3IB$ và $J$ là trung điểm của $CD$. Gọi $\alpha$ là số đo của góc giữa hai vectơ $\overrightarrow{AB}$ và $\overrightarrow{IJ}$, hãy tính $\cos \alpha$ (kết quả làm tròn đến hàng phần mười).
\shortans[4]{-0{,}4}

\loigiai{
\begin{center}
\begin{tikzpicture}[line join = round, line cap=round,>=stealth,font=\footnotesize,scale=1]
\coordinate (0,0) at (A);
\coordinate (B) at (-1.5,-3);
\coordinate (C) at (-0.5,-4);
\coordinate (D) at (3,-3);
\coordinate (I) at ($(A)!0.75!(B)$);
\coordinate (J) at ($(C)!0.5!(D)$);
\fill (A) circle (1pt) node[above] {$A$};
\fill (B) circle (1pt) node[left] {$B$};
\fill (C) circle (1pt) node[below] {$C$};
\fill (D) circle (1pt) node[right] {$D$};
\fill (I) circle (1pt) node[left] {$I$};
\fill (J) circle (1pt) node[below] {$J$};
\draw (A) -- (B)--(C)--(D)--(A)--(C);
\draw[dashed] (B) -- (D) (I)--(J);
\draw (0.5,-3.6)--(0.5,-3.85) (2,-3.15)--(2,-3.4);
\end{tikzpicture}
\end{center}
Ta có 
$\overrightarrow{IJ}=\overrightarrow{IA}+\overrightarrow{AJ}=-\dfrac{3}{4}\overrightarrow{AB}+\dfrac{1}{2}\left(\overrightarrow{AC}+\overrightarrow{AD} \right)=-\dfrac{3}{4}\overrightarrow{AB}+\dfrac{1}{2}\overrightarrow{AC}+\dfrac{1}{2}\overrightarrow{AD}$ nên
\[\overrightarrow{IJ}\cdot \overrightarrow{AB}=\left(-\dfrac{3}{4}\overrightarrow{AB}+\dfrac{1}{2}\overrightarrow{AC}+\dfrac{1}{2}\overrightarrow{AD} \right)\cdot \overrightarrow{AB}=\dfrac{1}{2}\left(\overrightarrow{AC}\cdot \overrightarrow{AB}+\overrightarrow{AD}\cdot \overrightarrow{AB}-\dfrac{3}{2}{\overrightarrow{AB}}^2 \right).\]
Lại có $\overrightarrow{AB}\cdot \overrightarrow{AD}=AB\cdot AD\cdot \cos60^\circ =\dfrac{a^2}{2}$; 
$\overrightarrow{AC}\cdot \overrightarrow{AB}=AC\cdot AB\cdot \cos60^\circ =\dfrac{a^2}{2}$; ${\overrightarrow{AB}}^2=AB^2=a^2$.\\
Suy ra, $\overrightarrow{IJ}\cdot \overrightarrow{AB}=\dfrac{1}{2}\left(\dfrac{a^2}{2}+\dfrac{a^2}{2}-\dfrac{3}{2}a^2 \right)=-\dfrac{a^2}{4}$. \\
Ta có $\widehat{CAD}=90^\circ \Rightarrow \overrightarrow{AC}\cdot \overrightarrow{AD}=0$.
\begin{align*}
IJ^2=&\, {\overrightarrow{IJ}}^2=\dfrac{1}{4}\left(\overrightarrow{AC}+\overrightarrow{AD}-\dfrac{3}{2}\overrightarrow{AB} \right)^2 \\
=&\, \dfrac{1}{4}\left({\overrightarrow{AB}}^2+{\overrightarrow{AC}}^2+\dfrac{9}{4}{\overrightarrow{AB}}^2+2\overrightarrow{AC}\cdot \overrightarrow{AD}-3\overrightarrow{AC}\cdot \overrightarrow{AB}-3\overrightarrow{AB}\cdot \overrightarrow{AD} \right) \\
=&\, \dfrac{1}{4}\left(a^2+a^2+\dfrac{9}{4}a^2+2\cdot 0-3\cdot \dfrac{1}{2}a^2-3\cdot \dfrac{1}{2}a^2 \right)=\dfrac{5a^2}{16}.
\end{align*}
$\Rightarrow IJ=\dfrac{a\sqrt{5}}{4}$.\\
Vậy $\cos \alpha=\dfrac{\overrightarrow{AB}\cdot \overrightarrow{IJ}}{AB\cdot IJ}=\dfrac{-\dfrac{a^2}{4}}{\dfrac{a\sqrt{5}}{4}\cdot a}=-\dfrac{\sqrt{5}}{5}\approx -0{,}4$.
}
\end{ex}

\begin{ex}%[2-H2B4-SO-14-2425 (Nguồn Đề 5 - Bài 4- Ôn tập chương II)]%[VN-MT-7, Bùi Lương Phúc]%[2H2H2-4]
\immini{Trong không gian, xét hệ tọa độ $Oxyz$ có gốc $O$ trùng với vị trí của một giàn khoan trên biển, mặt phẳng $\left(Oxy \right)$ trùng với mặt biển (được coi là phẳng) với trục $Ox$ hướng về phía tây, trục $Oy$ hướng về phía nam và trục $Oz$ hướng thẳng đứng lên trời. Đơn vị đo Trong KG $Oxyz$ lấy theo kilômét. Một chiếc ra đa đặt tại giàn khoan và một chiếc tàu thám hiểm có tọa độ là $(30;25;-15)$ (tham khảo hình bên).\\
Tính khoảng cách theo đơn vị kilômét từ chiếc ra đa đến chiếc tàu thám hiểm (kết quả làm tròn đến hàng phần mười).}{
\begin{tikzpicture}[line join = round, line cap=round,>=stealth,font=\footnotesize,scale=0.7]
\definecolor{xanhdatroi}{RGB}{0,127,255}
\tikzset{icon-cot/.pic={
\path[black!80]
(0,0)coordinate (A)
($(A)+(0,8)$)coordinate (B)
($(B)+(-2,-1)$)coordinate (C)
($(A)+(C)-(B)$)coordinate (D)
($(C)!(A)!(D)$) coordinate (A1)
($2*(A1)-(A)+(0,1)$)coordinate (E)
($(B)+(E)-(A)$) coordinate (F)
;
%%%%%%%%%%%%%%%%%%%%%%%%%%%%%%%%%%%%
\draw (A)--(B)--(C)--(D)--cycle
(D)--(E)--(F)--(C);
\foreach \x/\y in {0/0.1,0.1/0.2,0.2/0.3,0.3/0.4,0.4/0.5,0.5/0.6,0.6/0.7,0.7/0.8,0.8/0.9,0.9/1}
\draw
($(A)!\x!(B)$)--($(D)!\y!(C)$)
($(D)!\x!(C)$)--($(A)!\y!(B)$)
($(E)!\x!(F)$)--($(D)!\y!(C)$)
($(D)!\x!(C)$)--($(E)!\y!(F)$)
($(A)!\x!(B)$)--($(E)!\y!(F)$)
($(E)!\x!(F)$)--($(A)!\y!(B)$)
(B)--(F) (A)--(E);
}}
%%%%%%%%%%%%%%%%%%%%%%%%%%%%%%%%%%%%
\fill[bottom color=blue!80!green!15!black!60, top color=blue!20!green!20, middle color=blue!80!green]
(-5,-3.5) rectangle (6,2);
\fill[inner color=xanhdatroi, outer color=xanhdatroi!50]
(-5,4) rectangle (6,2);

%%%%%%%%%%%%%%%%%%%%%%%%%%%%%%%%
%De
\draw[fill=orange!10,scale=1,yshift=-1.25cm,xshift=-0.3cm,opacity=0.2,decorate, decoration={snake, amplitude=0.4mm, segment length=2mm},scale=1.1]
(-1.5,0.5)--(0.5,-0.5)--(4.5,-0.5) --($(-1.5,0.5)+(4.5,-1)-(0.5,-1)$)--cycle
;
%%%%%%%%%%%%%%%%%%%%%%%%%%%%%%%%
%Chande
\path[black!80]
(0.75,-1.2) pic[scale=0.15]{icon-cot}
(0.75,-1.5) pic[scale=0.15]{icon-cot}
(1.8,-1.2) pic[scale=0.15]{icon-cot}
(1.8,-1.5) pic[scale=0.15]{icon-cot}
(2.8,-1.2) pic[scale=0.15]{icon-cot}
(2.8,-1.5) pic[scale=0.15]{icon-cot}
;
%%%%%%%%%%%%%%%%%%%%%%%%%%%%%%%%
%San
\draw[fill=orange!80!gray!60,scale=1]
(-1.5,0.5)--(0.5,-0.5)--(4.5,-0.5) --($(-1.5,0.5)+(4.5,-1)-(0.5,-1)$)--cycle
;
%%%%%%%%%%%%%%%%%%%%%%%%%%%%%%%%%%%%
%Giankhoan
\path[black!80]
(0.75,0) pic[scale=0.15]{icon-cot}
(0.75,0.2) pic[scale=0.15]{icon-cot}
(1.8,0) pic[scale=0.15]{icon-cot}
(1.8,0.2) pic[scale=0.15]{icon-cot}
(2.8,0) pic[scale=0.15]{icon-cot}
(2.8,0.2) pic[scale=0.15]{icon-cot}
;
%%%%%%%%%%%%%%%%%%%%%%%%%%%%%%%%%%%%
%Hetoado Oxyz
\begin{scope}
\draw[->,line width=1pt,red] (0,0)--(4,-2) node [below]{y};
\draw[->,line width=1pt,red] (0,0)--(-4,0) node [below]{x};
\draw[->,line width=1pt,red] (0,0)--(0,3.5) node [right]{z};
\fill (0,0) circle(1pt) node [above left, red]{O};
\end{scope}
\end{tikzpicture}
}
\shortans[4]{41{,}8}
\loigiai{
Theo đề bài ta có tọa độ của ra đa là $(0;0;0)$, tọa độ của tàu thám hiểm là $(30;25;-15)$.\\
Khi đó khoảng cách giữa ra đa và tàu thám hiểm là
\[\mathrm{d}=\sqrt{(30-0)^2+(25-0)^2+(-15-0)^2}=5\sqrt{70}\approx 41{,}8.\]
Vậy khoảng khoảng cách giữa ra đa và tàu thám hiểm là $41{,}8$\,(km).
}
\end{ex}

\begin{ex}%[2-H2B4-SO-14-2425 (Nguồn Đề 5 - Bài 4- Ôn tập chương II)]%[VN-MT-7, Bùi Lương Phúc]%[2H2V2-5]
Trong KG $Oxyz$, cho tứ diện $ABCD$ có $A(0;1;-1)$, $B(1;1;2)$, $C(1;-1;0)$, $D(0;0;1)$. Biết rằng có một vectơ $\overrightarrow{v}=(a;b;2)$ vuông góc với cả hai vectơ $\overrightarrow{BC}$ và $\overrightarrow{BD}$. Tính $3a+b$.
\shortans[4]{-2}
\loigiai{
Ta có $\overrightarrow{BC}=(0;-2;-2)$, $\overrightarrow{BD}=(-1;-1;-1)$.
\begin{align*}
\left[ \overrightarrow{BC},\overrightarrow{BD} \right]=\left(\left| \begin{matrix}
-2 & -2 \\
-1 & -1 \\
\end{matrix} \right|;\left| \begin{matrix}
-2 & 0 \\
-1 & -1 \\
\end{matrix} \right|;\left| \begin{matrix}
0 & -2 \\
-1 & -1 \\
\end{matrix} \right| \right) \Rightarrow \left[ \overrightarrow{BC},\overrightarrow{BD} \right]=(0;2;-2).
\end{align*}
Khi đó $\left[ \overrightarrow{BC},\overrightarrow{BD} \right]=(0;2;-2)$ là một vectơ vuông góc với cả hai vectơ $\overrightarrow{BC}$, $\overrightarrow{BD}$. \\
Ta có $\left[ \overrightarrow{BC},\overrightarrow{BD} \right]$ và $\overrightarrow{v}=(a;b;2)$ cùng phương nên có số thực $k$ để $\overrightarrow{v}= k\cdot \left[ \overrightarrow{BC},\overrightarrow{BD} \right]$.\\
Suy ra
$\heva{&a= k\cdot 0 \\&b=k\cdot 2 \\&2=k \cdot(-2)}$. \\
Giải ra ta được $a=0$, $b=-2$.\\
Vậy $3a+b=-2$.
}
\end{ex}

\begin{ex}%[2-H2B4-SO-14-2425 (Nguồn Đề 5 - Bài 4- Ôn tập chương II)]%[VN-MT-7, Bùi Lương Phúc]%[2H2V2-4]
Trong không gian với một hệ trục toạ độ cho trước (đơn vị đo lấy theo kilômét), ra đa phát hiện một chiếc máy bay di chuyển với vận tốc và hướng không đổi từ điểm $A(800;500;7)$ đến điểm $B(940;550;9)$ trong $10$ phút. Nếu máy bay tiếp tục giữ nguyên vận tốc và hướng bay thì toạ độ của máy bay sau $5$ phút tiếp theo là $C(x;y;z)$. Tính $x+y+z$.
\shortans[4]{1595}
\loigiai{
Vị trí của máy bay sau $5$ phút tiếp theo là $C(x;y;z)$.\\
Vì hướng của máy bay không đổi nên $\overrightarrow{AB}$ và $\overrightarrow{BC}$ cùng hướng.\\
Do vận tốc của máy bay không đổi và thời gian bay từ $A$ đến $B$ gấp đôi thời gian bay từ $B$ đến $C$ nên $AB=2BC$.\\
Do đó $\overrightarrow{BC}=\dfrac{1}{2}\overrightarrow{AB}$.\\
Mặt khác, $\dfrac{1}{2}\overrightarrow{AB}=\left(\dfrac{940-800}{2};\dfrac{550-500}{2};\dfrac{9-7}{2} \right) \Rightarrow \dfrac{1}{2}\overrightarrow{AB}=(70;25;1)$;\\
Mà $\overrightarrow{BC}=(x-940 ; y-550 ; z-9)$
nên
$\heva{&x-940=70 \\&y-550=25 \\&z-9=1} \Rightarrow \heva{&x=1010 \\&y=575 \\&z=10} \Rightarrow x+y+z=1\,595$.
}
\end{ex}

\begin{ex}%[2-H2B4-SO-14-2425 (Nguồn Đề 5 - Bài 4- Ôn tập chương II)]%[VN-MT-7, Bùi Lương Phúc]%[2H2V2-4]
\immini{Một tấm gỗ tròn được treo song song với mặt phẳng nằm ngang bởi ba sợi dây không dãn xuất phát từ điểm $O$ trên trần nhà và lần lượt buộc vào ba điểm $A$, $B$, $C$ trên tấm gỗ tròn sao cho các lực căng $\overrightarrow{F}_1$, $\overrightarrow{F}_2$, $\overrightarrow{F}_3$ lần lượt trên mỗi dây $OA$, $OB$, $OC$ đôi một vuông góc với nhau và có độ lớn $\left| \overrightarrow{F}_1 \right|=\left| \overrightarrow{F}_2 \right|=\left| \overrightarrow{F}_3 \right|=10$\,N (xem hình vẽ).\\
Tính trọng lượng $P$ của tấm gỗ tròn đó (kết quả làm tròn đến hàng phần mười).}{
\begin{tikzpicture}[line join = round, line cap=round,>=stealth,font=\footnotesize,scale=0.9]
\path
(0,-1) coordinate (O)
(0,-4) coordinate (O')
(0,-8) coordinate (O'')
($(100:2cm and 1cm)+(0,-4)$) coordinate (B)
($(220:2cm and 1cm)+(0,-4)$) coordinate (A)
($(-20:2cm and 1cm)+(0,-4)$) coordinate (C) ;
\filldraw[fill=gray!80, draw=gray] (0,-4.4) ellipse (2cm and 1.1cm);
\filldraw[fill=gray!40, draw=gray] (O') ellipse (2cm and 1cm);
\draw[]
(O)--(A)
(O)--(B) (O)--(C)
;
\filldraw[gray!80, line width=1pt]
($($(O')+(2,0)$)+(0,-0.5)$)--($(O')+(2,0)$)
($($(O')-(2,0)$)+(0,-0.45)$)--($(O')-(2,0)$)
;
\draw[->,line width=1pt] (O)--($(O)!1/2!(A)$)node[left]{$\overrightarrow{F}_1$};
\draw[->,line width=1pt] (O)--($(O)!0.5!(B)$)node[right]{$\overrightarrow{F}_2$};
\draw[->,line width=1pt] (O)--($(O)!1/2!(C)$)
node[right]{$\overrightarrow{F}_3$}
;
\draw[->]
($(270:2cm and 1cm)+(0,-4.52)$)--(O'')node[pos=0.2,right]{$\overrightarrow{P}$}
;
\draw[dashed] ($(270:2cm and 1cm)+(0,-3.9)$)--(O')--(C);
\foreach \p/\r in {O'/180,B/-90,O/90,A/-90,C/-100}
\fill (\p) circle (1.2pt) node[shift={(\r:3mm)}]{$\p$};
\filldraw[fill=gray!80, draw=gray] (-2,-1) rectangle (2,-0.9);
\end{tikzpicture}
}
\shortans[4]{17{,}3}
\loigiai{
\begin{center}
\begin{tikzpicture}
[scale=0.9, font=\footnotesize, line join=round, line cap=round, >=stealth]
\coordinate (O) at (0,0);
\coordinate (A_1) at (-1.5,-2);
\coordinate (M) at (0.5,-4);
\coordinate (C_1) at ($(O)+(M)-(A_1)$);
\coordinate (B_1) at ($(O)+(-0.5,-1.7)$);
\coordinate (D_1) at ($(A_1)+(B_1)-(O)$);
\coordinate (N) at ($(B_1)+(C_1)-(O)$);
\coordinate (Q) at ($(D_1)+(M)-(A_1)$);
\draw (M)--(A_1)--(D_1)--(Q)--(N)--(C_1)--(M)--(Q);
\draw[->,line width=1pt] (O)--(A_1) node[midway, above,xshift=-1mm] {$\overrightarrow{F}_1$};;
\draw[->,line width=1pt] (O) -- (C_1) node[midway, above] {$\overrightarrow{F}_3$};
\draw[dashed] (D_1)--(B_1)--(N);
\draw[dashed,->,line width=1pt] (O)--(B_1) node[midway, below, xshift=2mm] {$\overrightarrow{F}_2$};
\draw[dashed,red,->] (O)--(Q);
\foreach \x/\y in {O/90,B_1/-90,C_1/0,D_1/-90,A_1/120,M/0,N/0,Q/-90} \fill[black](\x) circle (1pt) ($(\x)+(\y:4mm)$) node{$\x$};
\end{tikzpicture}
\end{center}
Gọi $A_1$, $B_1$, $C_1$ lần lượt là các điểm sao cho $\overrightarrow{OA}_1=\overrightarrow{F}_1$, $\overrightarrow{OB}_1=\overrightarrow{F}_2$, $\overrightarrow{OC}_1=\overrightarrow{F}_3$.\\
Lấy các điểm $D_1$, $M$, $N$, $Q$ sao cho $OA_1D_1B_1.C_1MQN$ là hình hộp.\\
Theo quy tắc hình hộp ta có $\overrightarrow{OA}_1+\overrightarrow{OB}_1+\overrightarrow{OC}_1=\overrightarrow{OQ}$.\\
Do các lực căng $\overrightarrow{F}_1$, $\overrightarrow{F}_2$, $\overrightarrow{F}_3$ đôi một vuông góc với nhau và có độ lớn $\left| \overrightarrow{F}_1 \right|=\left| \overrightarrow{F}_2 \right|=\left| \overrightarrow{F}_3 \right|=10$\,N nên hình hộp $OA_1D_1B_1.C_1MQN$ có ba cạnh $OA_1$, $OB_1$, $OC_1$ đôi một vuông góc và đều có độ dài bằng $10$.\\
Vì thế $OA_1D_1B_1.C_1MQN$ là hình lập phương có độ dài cạnh bằng $10$.\\
Suy ra độ dài đường chéo bằng $10\sqrt{3}$.\\
Gọi $\overrightarrow{P}$ là trọng lượng tác dụng lên tấm gỗ.\\
Do tấm gỗ ở vị trí cân bằng nên $\overrightarrow{F}_1+\overrightarrow{F}_2+\overrightarrow{F}_3=\overrightarrow{P}$.\\
Suy ra $\left| \overrightarrow{P} \right|=\left| \overrightarrow{OQ} \right|=10\sqrt{3}\approx 17{,}3$.\\
Vậy trọng lượng của tấm gỗ tròn là $P=\left| \overrightarrow{P} \right|=10\sqrt{3}\approx 17{,}3$\,(N).}
\end{ex}

\begin{ex}%[2-H2B4-SO-14-2425 (Nguồn Đề 5 - Bài 4- Ôn tập chương II)]%[VN-MT-7, Bùi Lương Phúc]%[2H2V2-4]
 \immini{Một chậu cây được đặt trên một giá đỡ có bốn chân với điểm đặt $S(0;0;40)$ và các điểm chạm mặt đất của bốn chân lần lượt là $A(40;0;0)$, $B(0;40;0)$, $C(-40;0;0)$, $D(0;-40;0)$ (đơn vị là cm). Cho biết trọng lực tác dụng lên chậu cây có độ lớn $60$\,N và được phân bố thành bốn lực $\overrightarrow{F}_1,\overrightarrow{F}_2,\overrightarrow{F}_3,\overrightarrow{F}_4$ có độ lớn bằng nhau như hình vẽ. \\
 Tính $\left| \overrightarrow{F}_1+\overrightarrow{F}_2+\overrightarrow{F}_3+3\overrightarrow{F}_4 \right|$ (kết quả làm tròn đến hàng đơn vị).}{
 \hspace*{0.5cm}\begin{tikzpicture}[line join = round, line cap=round,>=stealth,font=\footnotesize,scale=1]
 \tikzset{chauhoa/.pic={\fill[orange!80!black!90] (-1,-1.1) -- (1,-1.1) -- (0.8,-1.8) -- (-0.8,-1.8) -- cycle;
 \draw[thick] (-1,-1.1) -- (1,-1.1) -- (0.8,-1.8) -- (-0.8,-1.8) -- cycle;
 \fill[brown] (-0.3,-1) -- (0.3,-1) -- (0.2,0) -- (-0.2,0) -- cycle;
 \draw[brown, very thick] (-0.3,-1) -- (0.3,-1) -- (0.2,-0.2) -- (-0.2,-0.2) -- cycle;
 \fill[brown!60!black!90] (-0.1,-0.7) circle (0.1);
 \fill[brown!60!black!90] (0.1,-0.9) circle (0.1);
 \fill[brown!60!black!90] (-0.2,-0.5) circle (0.1);
 \fill[brown!50!black!80] (0,-0.3) circle (0.1);
 \fill[green] (0,0.3) ellipse (0.6 and 0.3);
 \fill[green] (-0.5,0.3) ellipse (0.6 and 0.3);
 \fill[green] (0.5,0.3) ellipse (0.6 and 0.3);
 \fill[green] (0,-0.2) ellipse (0.6 and 0.3);
 \draw[green, very thick] (0,0.3) to[out=60,in=120] (0.5,1.5);
 \draw[green, very thick] (0,0.3) to[out=120,in=60] (-0.5,1.5);
 \draw[green, very thick] (0,-0.2) to[out=60,in=120] (0.5,1);
 \draw[green, very thick] (0,-0.2) to[out=120,in=60] (-0.5,1);
 \fill[green] (0.5,1.2) ellipse (0.4 and 0.2);
 \fill[green] (-0.5,1.2) ellipse (0.4 and 0.2);
 \fill[green] (0.5,0.8) ellipse (0.4 and 0.2);
 \fill[green] (-0.5,0.8) ellipse (0.4 and 0.2);}}
 \path
 (0,0) coordinate (O)
 (0,4) coordinate (S)
 (-115:2.3cm and 1.2cm) coordinate (A)
 (-10:2.3cm and 1.2cm) coordinate (B)
 ($2*(O)-(A)$) coordinate (C)
 ($2*(O)-(B)$) coordinate (D)
 ($(S)!0.55!(A)$) coordinate(F1)
 ($(S)!0.55!(B)$) coordinate(F2)
 ($(S)!0.55!(C)$) coordinate(F3)
 ($(S)!0.55!(D)$) coordinate(F4);
 \draw[->, shorten >=-0.5cm, shorten <=-0.5cm](C)--(A) node[below left=10pt]{$x$};
 \draw[->, shorten >=-0.5cm, shorten <=-0.5cm](D)--(B) node[below right=5pt]{$y$};
 \foreach \p in {A,B,C,D} {\draw (S)--(\p);}
 \foreach \f/\g [count =\i from 1] in {F1/200,F2/10,F3/-120,F4/160} {\draw[->] (S)--(\f)node [shift={(\g:0.4)}]{$\overrightarrow{F}_\i$};}
 \draw[->] (O)--(90:5.5) node[left]{$z$};
 \foreach \i/\j in {O/-90, A/-70, B/-90, C/-70, D/-90} \fill (\i) node[shift={(\j:0.3)}]{$\i$} circle(1pt);
 \fill (S) node[shift={(210:0.4)}]{$S$} circle(1pt);
 \draw ($(S)+(0,0.7)$) pic[scale=0.4]{chauhoa};
 \end{tikzpicture}}
 \shortans[4]{95}
 \loigiai{
 \begin{center}
 \begin{tikzpicture}[line join = round, line cap=round,>=stealth,font=\footnotesize,scale=1.2]
 \path
 (0,0) coordinate (O)
 (0,4) coordinate (S)
 (-115:2.5cm and 1.2cm) coordinate (A)
 (-10:2.5cm and 1.2cm) coordinate (B)
 ($2*(O)-(A)$) coordinate (C)
 ($2*(O)-(B)$) coordinate (D)
 ($(S)!0.55!(A)$) coordinate(A')
 ($(S)!0.55!(B)$) coordinate(B')
 ($(S)!0.55!(C)$) coordinate(C')
 ($(S)!0.55!(D)$) coordinate(D')
 ($(A')!1/2!(C')$) coordinate(O');
 \draw (S)--(D)--(A)--(B)--(S)--(A) (B')--(A')--(D');
 \draw[dashed] (S)--(C)--(D)--(B)--(C)--(A) (S)--(O) (D')--(C')--(B') (A')--(C') (B')--(D');
 \foreach \i/\j in {O/-70, A/-90, B/-60, C/0, D/200, S/90, A'/200, B'/20, C'/160, D'/160, O'/-50} \fill (\i) node[shift={(\j:0.3)}]{$\i$} circle(1pt);
 \draw[->, red] (S)--(A') node[midway, left=-4pt] {$\overrightarrow{F}_1$};
 \draw[->, red] (S)--(B') node[midway, right] {$\overrightarrow{F}_2$};
 \draw[->, red] (S)--(C') node[midway, shift={(250:0.3)}] {$\overrightarrow{F}_3$};
 \draw[->, red] (S)--(D') node[midway, left] {$\overrightarrow{F}_4$};
 \end{tikzpicture}
 \end{center}
 Tứ giác $ABCD$ có hai đường chéo bằng nhau và vuông góc với nhau tại trung điểm của mỗi đường nên là hình vuông.\\
 Ta có $\overrightarrow{SA}=(40;0;-40)$, $\overrightarrow{SB}=(0;40;-40)$, $\overrightarrow{SC}=(-40;0;-40)$, $\overrightarrow{SD}=(0;-40;-40)$\\
 $\Rightarrow SA=SB=SC=SD=40\sqrt{2}$. \\
 Do đó $S.ABCD$ là hình chóp tứ giác đều.\\
 Các vectơ $\overrightarrow{F}_1$, $\overrightarrow{F}_2$, $\overrightarrow{F}_3$, $\overrightarrow{F}_4$ có điểm đầu tại $S$ và điểm cuối lần lượt là $A'$ ,$B'$, $C'$, $D'$.\\
 Ta có $SA'=SB'=SC'=SD'$ nên $S.A'B'C'D'$ cũng là hình chóp tứ giác đều.\\
 Gọi $\overrightarrow{F}$ là trọng lực tác dụng lên chậu cây và ${O}'$ là tâm của hình vuông $A'B'C'D'$.\\
 Ta có 
 $\overrightarrow{F}=\overrightarrow{F}_1+\overrightarrow{F}_2+\overrightarrow{F}_3+\overrightarrow{F}_4=\overrightarrow{SA'}+\overrightarrow{SB'}+\overrightarrow{SC'}+\overrightarrow{SD'}=4\overrightarrow{SO'}$.\\
 Mà $\left| \overrightarrow{F} \right|=60\Rightarrow \left| \overrightarrow{SO'} \right|=15$ và $\overrightarrow{SO}=(0;0;-40)$ nên $\left| \overrightarrow{SO} \right|=40$.\\
 Vậy $\overrightarrow{SO'}=\dfrac{15}{40}\overrightarrow{SO}=\dfrac{3}{8}\overrightarrow{SO}$.\\
 Suy ra
 $\overrightarrow{SA'}=\dfrac{3}{8}\overrightarrow{SA}$, 
 $\overrightarrow{SB'}=\dfrac{3}{8}\overrightarrow{SB}$, 
 $\overrightarrow{SC'}=\dfrac{3}{8}\overrightarrow{SC}$ và 
 $\overrightarrow{SD'}=\dfrac{3}{8}\overrightarrow{SD}$.\\
 Do đó $\overrightarrow{F}_1=(15;0;-15)$, $\overrightarrow{F}_2=(0;15;-15)$, $\overrightarrow{F}_3=(-15;0;-15)$, $\overrightarrow{F}_4=(0;-15;-15)$.\\
 Suy ra $\overrightarrow{F}_1+\overrightarrow{F}_2+\overrightarrow{F}_3+3\overrightarrow{F}_4=(0;-30;-90)$. \\
 Vậy $\left| \overrightarrow{F}_1+\overrightarrow{F}_2+\overrightarrow{F}_3+3\overrightarrow{F}_4 \right|=\sqrt{0^2+(-30)^2+(-90)^2}=30\sqrt{10}\approx 95$\,(N).
 }
\end{ex}
\Closesolutionfile{ans}
 
\begin{indapan}
	{ans/ans\currfilebase}
\end{indapan}


%C3
% \begin{name}
	{\tenchude}
	{ĐỀ ÔN TẬP CHƯƠNG III}
	{LỚP TOÁN THẦY PHÁT}
	{\thoigian}
\end{name}
\TN
\Opensolutionfile{ans}[ans/ans\currfilebase-Phan-I]
\begin{ex}%[2-D3B3-SO-7-2425]%[VN-MT-7, Nguyễn Kiều Nhã Tú]%[2D3N1-1]
 Cho mẫu số liệu ghép nhóm
 \begin{center}
 \begin{tabular}{|c|c|c|c|c|c|}
 \hline
 Nhóm & $[a_1;a_2)$ & $\ldots$ & $[a_j;a_{j+1})$ & $\ldots$ &$[a_k;a_{k+1})$ \\
 \hline
 Tần số & $m_1$ & $\ldots$ & $m_i$ & $\ldots$ & $m_k$ \\
 \hline
 \end{tabular}
 \end{center}
 trong đó các tần số $m_1 > 0$, $m_k > 0$ và $n = m_1 + \cdots + m_k$ là cỡ mẫu.\\
 Khoảng biến thiên của mẫu số liệu ghép nhóm trên là
 \choice
 {\True $R = a_{k+1} - a_1$}
 {$R = a_k - a_{k+1}$}
 {$R = a_{k+1} + a_1$}
 {$R = a_k + a_{k+1}$}
 \loigiai{
 Khoảng biến thiên của mẫu số liệu ghép nhóm trên là $R = a_{k+1} - a_1$.
 }
\end{ex}

\begin{ex}%[2-D3B3-SO-7-2425]%[VN-MT-7, Nguyễn Kiều Nhã Tú]%[2D3N1-1]
 Cho mẫu số liệu ghép nhóm có tứ phân vị thứ nhất, thứ hai, thứ ba lần lượt là $Q_1$, $Q_2$ và $Q_3$. Khoảng tứ phân vị của mẫu số liệu ghép nhóm đó bằng
 \choice
 {$Q_2 - Q_1$}
 {$Q_1 - Q_3$}
 {\True $Q_3 - Q_1$}
 {$Q_1 - Q_2$}
 \loigiai{
 Khoảng tứ phân vị của mẫu số liệu ghép nhóm là $Q_3 - Q_1$.
 }
\end{ex}

\begin{ex}%[2-D3B3-SO-7-2425]%[VN-MT-7, Nguyễn Kiều Nhã Tú]%[2D3H1-3]
 Một người ghi lại thời gian đàm thoại của một số cuộc gọi cho kết quả như bảng sau:
 \begin{center}
 \begin{tabular}{|c|c|c|c|c|c|}
 \hline
 Thời gian $t$ (phút) & $[0;1)$ & $[1;2)$ & $[2;3)$ & $[3;4)$ & $[4;5)$ \\
 \hline
 Số cuộc gọi & $8$ & $17$ & $25$ & $20$ & $10$ \\
 \hline
 \end{tabular}
 \end{center}
 Khoảng tứ phân vị của mẫu số liệu trên có giá trị bằng
 \choice
 {\True $\dfrac{61}{34}$}
 {$\dfrac{7}{2}$}
 {$\dfrac{29}{17}$}
 {$\dfrac{177}{34}$}
 \loigiai{
 Cỡ mẫu $n = 80$.\\
 Giả sử $x_1$, $x_2$, $\ldots$, $x_{80}$ là thời gian đàm thoại của $80$ cuộc gọi đã được sắp xếp theo thứ tự không giảm.
 \begin{itemize}
 \item Vì $\dfrac{n}{4} = 20$ và $8 < 20 < 8 + 17$ nên tứ phân vị thứ nhất thuộc nhóm $[1;2)$ và có giá trị là
 \[Q_1 = 1 + \dfrac{20-8}{17}\cdot 1 = \dfrac{29}{17}.\]
 \item Lại có $\dfrac{3n}{4} = 60$ và $8+17+25 < 60 < 8 + 17+25+20$ nên tứ phân vị thứ ba thuộc nhóm $[3;4)$ và có giá trị là
 \[Q_3 = 3 + \dfrac{60-(8+17+25)}{20}\cdot 1 = \dfrac{7}{2}.\]
 \end{itemize}
 Vậy khoảng tứ phân vị là $\Delta_{Q}=Q_3-Q_1=\dfrac{7}{2}-\dfrac{29}{17}=\dfrac{61}{34}$.
 }
\end{ex}

\begin{ex}%[2-D3B3-SO-7-2425]%[VN-MT-7, Nguyễn Kiều Nhã Tú]%[1D5H2-3]
 Sau khi kiểm tra sức khoẻ tổng quát, kết quả số cân nặng của học sinh lớp 12A sĩ số $40$ học sinh được thể hiện trong bảng số liệu sau:
 \begin{center}
 \begin{tabular}{|c|c|c|c|c|c|}
 \hline
 Cân nặng (kg)& $[40;50)$ & $[50;60)$ & $[60;70)$ & $[70;80)$ & $[80;90)$ \\
 \hline
 Số học sinh & $7$ & $12$ & $12$ & $7$ & $2$ \\
 \hline
 \end{tabular}
 \end{center}
 Tứ phân vị thứ nhất của mẫu số liệu trên bằng
 \choice
 {$50$}
 {$50{,}5$}
 {\True $52{,}5$}
 {$55{,}5$}
 \loigiai{
 Cỡ mẫu $n = 40$.\\
 Giả sử $x_1$, $x_2$, $\ldots$, $x_{40}$ là cân nặng của $40$ học sinh đã được sắp xếp theo thứ tự không giảm.\\
 Vì $\dfrac{n}{4} = 10$ và $7 < 10 < 7 + 12$ nên tứ phân vị thứ nhất thuộc nhóm $[50;60)$ và có giá trị là \[Q_1 = 50 + \dfrac{10-7}{12}\cdot 10 = 52{,}5.\]
 }
\end{ex}

\begin{ex}%[2-D3B3-SO-7-2425]%[VN-MT-7, Nguyễn Kiều Nhã Tú]%[1D5H2-3]
 Chỉ số ô nhiễm không khí (AQI) tại thủ đô Hà Nội trong tháng $6/2024$ được thống kê vào $10$h$30$ sáng các ngày trong tháng thể hiện trong bảng số liệu sau:
 \begin{center}
 \begin{tabular}{|c|c|c|c|c|c|}
 \hline
 Chỉ số (AQI) & $[130;145)$ & $[145;160)$ & $[160;175)$ &$[175;190)$ & $[190;205)$ \\
 \hline
 Số ngày & $8$ & $7$ & $6$ & $7$ & $2$ \\
 \hline
 \end{tabular}
 \end{center}
 Tứ phân vị thứ ba của mẫu số liệu trên gần nhất với giá trị nào trong các giá trị sau?
 \choice
 {$175$}
 {$176{,}5$}
 {$180{,}2$}
 {\True $178{,}2$}
 \loigiai{
 Cỡ mẫu $n = 30$.\\
 Giả sử $x_1$, $x_2$, $\ldots$, $x_{30}$ là chỉ số (AQI) của $30$ ngày trong tháng $6/2024$ đã được sắp xếp theo thứ tự không giảm.\\
 Vì $\dfrac{3n}{4} =22{,}5 $ và $8+7+6< 22{,}5 < 8+7+6+7$ nên tứ phân vị thứ ba thuộc nhóm $[175;190)$ và có giá trị là
 \[Q_3 = 175 + \dfrac{22,5-(8+7+6)}{7}\cdot 15 \approx 178{,}2.\]
 }
\end{ex}

\begin{ex}%[2-D3B3-SO-7-2425]%[VN-MT-7, Nguyễn Kiều Nhã Tú]%[2D3N1-2]
 Trong kì thi chọn học sinh giỏi ở cụm trường THPT A, môn Toán có $25$ học sinh tham gia kết quả điểm bài thi của học sinh được thể hiện trong bảng sau:
 \begin{center}
 \begin{tabular}{|c|c|c|c|c|c|}
 \hline
 Điểm bài thi & $[10;12)$ & $[12;14)$ & $[14;16)$ & $[16;18)$ & $[18;20)$ \\
 \hline
 Số lần & $4$ & $6$ & $8$ & $4$ & $3$ \\
 \hline
 \end{tabular}
 \end{center}
 Khoảng biến thiên của mẫu số liệu ghép nhóm nhận giá trị nào trong các giá trị dưới đây?
 \choice
 {$18,5$}
 {$10,5$}
 {$8$}
 {\True $10$}
 \loigiai{
 Khoảng biến thiên của mẫu số liệu là $20 - 10 = 10$.
 }
\end{ex}

\begin{ex}%[2-D3B3-SO-7-2425]%[VN-MT-7, Nguyễn Kiều Nhã Tú]%[1D5H2-3]
 Đo cân nặng $40$ học sinh lớp 12A ta được bảng số liệu như sau:
 \begin{center}
 \begin{tabular}{|c|c|c|c|c|c|c|c|}
 \hline
 Khối lượng (kg) & $[40;45)$ & $[45;50)$ & $[50;55)$ & $[55;60)$ & $[60;65)$ & $[65;70)$ & $[70;75)$ \\
 \hline
 Số học sinh & $4$ & $13$ & $7$ & $5$ & $6$ & $2$ & $1$ \\
 \hline
 \end{tabular}
 \end{center}
 Tứ phân vị thứ nhất của mẫu số liệu ghép nhóm thuộc khoảng nào sau đây?
 \choice
 {$[40;45)$}
 {\True $[45;50)$}
 {$[50;55)$}
 {$[55;60)$}
 \loigiai{
 Cỡ mẫu $n = 40$.\\
 Giả sử $x_1$, $x_2$, $\ldots$, $x_{40}$ là cân nặng của $40$ học sinh lớp 12A đã được sắp xếp theo thứ tự không giảm.\\
 Vì $\dfrac{n}{4} =10$ và $4< 10 < 4+13$ nên tứ phân vị thứ nhất thuộc nhóm $[45;50)$.
 }
\end{ex}

\begin{ex}%[2-D3B3-SO-7-2425]%[VN-MT-7, Nguyễn Kiều Nhã Tú]%[1D5H1-3]
 Thống kê điểm thi đánh giá năng lực của một trường THPT qua thang điểm $120$ môn Toán như sau:
 \begin{center}
 \begin{tabular}{|c|c|c|c|c|c|}
 \hline
 Điểm & $[0;20)$ & $[20;40)$ & $[40;60)$ & $[60;80)$ & $[80;100)$ \\
 \hline
 Số học sinh & $25$ & $35$ & $37$ & $15$ & $8$ \\
 \hline
 \end{tabular}
 \end{center}
 Điểm trung bình của tất cả các học sinh tham gia dự thi thuộc khoảng nào sau đây?
 \choice
 {\True $(40;45)$}
 {$(45;50)$}
 {$(50;55)$}
 {$(55;60)$}
 \loigiai{
 Ta có bảng sau:
 \begin{center}
 \begin{tabular}{|c|c|c|c|c|c|}
 \hline
 Điểm & $[0;20)$ & $[20;40)$ & $[40;60)$ & $[60;80)$ & $[80;100)$ \\
 \hline
 Giá trị đại diện & $10$ & $30$ & $50$ & $70$ & $90$ \\
 \hline
 Số học sinh & $25$ & $35$ & $37$ & $15$ & $8$ \\
 \hline
 \end{tabular}
 \end{center}
 Điểm trung bình của các thí sinh dự thi là
 \[\overline{x} = \dfrac{25 \cdot 10 + 35 \cdot 30 + 37 \cdot 50 + 15 \cdot 70 + 8 \cdot 90}{120} = 41.\]
 Do đó, điểm trung bình của tất cả các học sinh tham gia dự thi thuộc khoảng $(40;45)$.
 }
\end{ex}

\begin{ex}%[2-D3B3-SO-7-2425]%[VN-MT-7, Nguyễn Kiều Nhã Tú]%[2D3H2-2]
 Đo chiều cao các em học sinh khối $10$ ta thu được kết quả trong bảng sau:
 \begin{center}
 \begin{tabular}{|c|c|c|c|c|c|c|}
 \hline
 Chiều cao (cm) & $[150;152)$ & $[152;154)$ & $[154;156)$ & $[156;158)$ & $[158;160)$ & $[160;162]$ \\
 \hline
 Số học sinh & $5$ & $18$ & $40$& $26$ & $8$ & $3$ \\
 \hline
 \end{tabular}
 \end{center}
 Tính phương sai của mẫu số liệu ghép nhóm trên (làm tròn kết quả đến hàng phần mười).
 \choice
 {$4{,}5$}
 {$5{,}6$}
 {\True $4{,}7$}
 {$4{,}8$}
 \loigiai{
 Ta có bảng sau:
 \begin{center}
 \begin{tabular}{|c|c|c|c|c|c|c|}
 \hline
 Chiều cao (cm) & $[150;152)$ & $[152;154)$ & $[154;156)$ & $[156;158)$ & $[158;160)$ & $[160;162]$ \\
 \hline
 Giá trị đại diện & $151$ &$153$ & $155$& $157$ & $159$ & $161$ \\
 \hline
 Số học sinh & $5$ & $18$ & $40$& $26$ & $8$ & $3$ \\
 \hline
 \end{tabular}
 \end{center}
 Giá trị trung bình của mẫu số liệu là
 \[\overline{x} = \dfrac{5 \cdot 151 + 18 \cdot 153 + 40 \cdot 155 + 26 \cdot 157 + 8 \cdot 159 + 3 \cdot 161}{100} = 155{,}46.\]
 Khi đó phương sai của mẫu số liệu là
% \[s^2 =\dfrac{5(151-155{,}46)^2 + 18(153-155{,}46)^2 + \ldots + 3(161-155{,}46)^2}{100}=4{,}7084.\]
 \[s^2 = \dfrac{5 \cdot 151^2 + 18 \cdot 153^2 + 40 \cdot 155^2 + 26 \cdot 157^2 + 8 \cdot 159^2 + 3 \cdot 161^2}{100} -(155{,}46)^2 = 4{,}7084.\]
 }
\end{ex}

\begin{ex}%[2-D3B3-SO-7-2425]%[VN-MT-7, Nguyễn Kiều Nhã Tú]%[2D3N2-1]
 Số đặc trưng nào sau đây \textbf{không sử dụng} để đo mức độ phân tán của mẫu số liệu ghép nhóm?
 \choice
 {Khoảng biến thiên}
 {\True Trung vị}
 {Phương sai}
 {Khoảng tứ phân vị}
 \loigiai{
 \begin{itemize}
 \item Khoảng biến thiên của mẫu số liệu ghép nhóm xấp xỉ cho khoảng biến thiên của mẫu số liệu gốc. Khoảng biến thiên được dùng để đo mức độ phân tán của mẫu số liệu ghép nhóm. Khoảng biến thiên càng lớn thì mẫu số liệu càng phân tán.
 \item Trung vị của mẫu số liệu ghép nhóm xấp xỉ cho trung vị của mẫu số liệu gốc, nó chia mẫu số liệu thành hai phần, mỗi phần chứa $50$\% giá trị. Vậy trung vị không thể hiện mức độ phân tán.
 \item Phương sai của mẫu số liệu ghép nhóm xấp xỉ cho phương sai của mẫu số liệu gốc. Phương sai được dùng để đo mức độ phân tán của mẫu số liệu ghép nhóm xung quanh số trung bình của mẫu số liệu đó. Phương sai càng lớn thì mẫu số liệu càng phân tán.
 \item Khoảng tứ phân vị của mẫu số liệu ghép nhóm xấp xỉ cho khoảng tứ phân vị của mẫu số liệu gốc. Khoảng tứ phân vị được dùng để đo mức độ phân tán của mẫu số liệu ghép nhóm. Khoảng tứ phân vị càng lớn thì mẫu số liệu càng phân tán.
 \end{itemize}
 }
\end{ex}

\begin{ex}%[2-D3B3-SO-7-2425]%[VN-MT-7, Nguyễn Kiều Nhã Tú]%[2D3N2-1]
 Ý nghĩa của độ lệch chuẩn đối với mẫu số liệu ghép nhóm là
 \choice
 {\True dùng độ lệch chuẩn của mẫu số liệu để ước lượng độ lệch chuẩn xung quanh số trung bình của mẫu số liệu đó}
 {cho biết về ý nghĩa trung tâm của mẫu số liệu và cả về độ tán xạ dữ liệu của mẫu số liệu}
 {chia mẫu số liệu thành hai phần, mỗi phần chứa $50$\% giá trị}
 {chia mẫu số liệu thành bốn phần, mỗi phần chứa $25$\% giá trị}
 \loigiai{
 Ý nghĩa độ lệch chuẩn của mẫu số liệu ghép nhóm: Độ lệch chuẩn của mẫu số liệu ghép nhóm xấp xỉ cho độ lệch chuẩn của mẫu số liệu gốc. Độ lệch chuẩn được dùng để đo mức độ phân tán của mẫu số liệu ghép nhóm xung quanh số trung bình của mẫu số liệu đó. Độ lệch chuẩn càng lớn thì mẫu số liệu càng phân tán.
 }
\end{ex}

\begin{ex}%[2-D3B3-SO-7-2425]%[VN-MT-7, Nguyễn Kiều Nhã Tú]%[2D3H2-2]
 Quãng đường đi bộ tập thể dục mỗi ngày (đơn vị: km) của bác An trong $20$ ngày được thống kê lại ở bảng sau:
 \begin{center}
 \begin{tabular}{|c|c|c|c|c|c|}
 \hline
 Quãng đường (km) & $[2{,}2;2{,}6)$ & $[2{,}6;3{,}0)$ & $[3{,}0;3{,}4)$ & $[3{,}4;3{,}8)$ & $[3{,}8;4{,}2)$ \\
 \hline
 Tần số & $3$ & $6$ & $5$ & $5$ & $1$ \\
 \hline
 \end{tabular}
 \end{center}
 Độ lệch chuẩn của mẫu số liệu trên có giá trị xấp xỉ bằng
 \choice
 {$3{,}1$}
 {$0{,}042$}
 {$0{,}206$}
 {\True $0{,}45$}
 \loigiai{
 Ta có bảng sau:
 \begin{center}
 \begin{tabular}{|c|c|c|c|c|c|}
 \hline
 Quãng đường (km) & $[2{,}2;2{,}6)$ & $[2{,}6;3{,}0)$ & $[3{,}0;3{,}4)$ & $[3{,}4;3{,}8)$ & $[3{,}8;4{,}2)$ \\
 \hline
 Giá trị đại diện &$2{,}4$&$2{,}8$&$3{,}2$&$3{,}6$&$4{,}0$\\
 \hline
 Tần số & $3$ & $6$ & $5$ & $5$ & $1$ \\
 \hline
 \end{tabular}
 \end{center}
 Số trung bình của mẫu số liệu ghép nhóm là
 \[\overline{x} = \dfrac{3 \cdot 2{,}4 + 6 \cdot 2{,}8 + 5 \cdot 3{,}2 + 5 \cdot 3{,}6 + 1 \cdot 4{,}0}{20} = 3{,}1.\]
 Phương sai của mẫu số liệu ghép nhóm là
 \[s^2 = \dfrac{3 \cdot (2{,}4 - 3{,}1)^2 + 6 \cdot (2{,}8 - 3{,}1)^2 + 5 \cdot (3{,}2 - 3{,}1)^2 + 5 \cdot (3{,}6 - 3{,}1)^2 + 1 \cdot (4{,}0 - 3{,}1)^2}{20} = 0{,}206.\]
 Độ lệch chuẩn của mẫu số liệu ghép nhóm là
 \[s = \sqrt{0{,}206} \approx 0{,}45.\]
 }
\end{ex}
\Closesolutionfile{ans}

\TNTF
\Opensolutionfile{ans}[ans/ans\currfilebase-Phan-II]

\begin{ex}%[2-D3B3-SO-7-2425]%[VN-MT-7, Nguyễn Kiều Nhã Tú]%[2D3H1-4]
 Thành tích chạy $50$\,m của $30$ em học sinh lớp $10$ trường THPT A (đơn vị: giây) được thống kê như bảng sau:
 \begin{center}
 \begin{tabular}{*{6}{p{2cm}}}
 $6{,}3$ & $6{,}2$ & $6{,}5$ & $6{,}8$ & $6{,}9$ & $8{,}2$ \\
 $6{,}6$ & $6{,}7$ & $7{,}0$ & $7{,}1$ & $7{,}2$ & $8{,}3$ \\
 $7{,}4$ & $7{,}3$ & $7{,}2$ & $7{,}1$ & $7{,}0$ & $8{,}4$ \\
 $7{,}1$ & $7{,}3$ & $7{,}5$ & $7{,}5$ & $7{,}6$ & $8{,}7$ \\
 $7{,}6$ & $7{,}7$ & $7{,}8$ & $7{,}5$ & $7{,}7$ & $7{,}8$. \\
 \end{tabular}
 \end{center}
 \choiceTF
 {\True Tần số của nhóm $[7,0;7,5)$ là $10$}
 {Trung bình mỗi em chạy $50$\,m hết số thời gian là $7{,}5$ (giây)}
 {Khoảng biến thiên của mẫu số liệu ghép nhóm trên là $R=3{,}1$}
 {\True Khoảng tứ phân vị của mẫu số liệu ghép nhóm trên là $\Delta_{Q}=0{,}781$}
 \loigiai{
 \begin{itemchoice}
 \itemch \textbf{Đúng}.\\
 Bảng tần số ghép nhóm của mẫu số liệu trên là
 \begin{center}
 \begin{tabular}{|c|c|c|c|c|c|c|}
 \hline
 Thời gian chạy (giây) & $[6{,}0;6{,}5)$ & $[6{,}5;7{,}0)$ & $[7{,}0;7{,}5)$ & $[7{,}5;8{,}0)$ & $[8{,}0;8{,}5)$ & $[8{,}5;9{,}0)$ \\
 \hline
 Tần số & $2$ & $5$ & $10$ & $9$ & $3$ & $1$ \\
 \hline
 \end{tabular}
 \end{center}
 \itemch \textbf{Sai}.\\
 Tổng số học sinh là $n=30$.\\
 Trung bình mỗi em chạy $50$\,m hết số thời gian là
 \begin{center}
 $\overline{x}=\dfrac{6{,}25 \cdot 2+6{,}75 \cdot 5+7{,}25 \cdot 10+7{,}75 \cdot 9+8{,}25 \cdot 3+8{,}75 \cdot 1}{30}=7{,}4$ (giây).
 \end{center}
 \itemch \textbf{Sai}.\\
 Khoảng biến thiên của mẫu số liệu ghép nhóm trên là $R=9{,}0-6{,}0=3{,}0$.
 \itemch \textbf{Đúng}.
 \begin{itemize}
 \item Cỡ mẫu là $n=30$.\\
 Gọi $x_{1}$,$\ldots$, $x_{30}$ là thời gian chạy $50$\,m của $30$ học sinh và giả sử dãy này đã được sắp xếp theo thứ tự không giảm.\\
 Khi đó, trung vị là $\dfrac{x_{15}+x_{16}}{2}$. Do $2$ giá trị $x_{15}$, $x_{16}$ thuộc nhóm $[7{,}0;7{,}5)$ nên nhóm này chứa trung vị. Do đó
 \[
 M_{\text{e}}=7{,}0+\dfrac{\dfrac{30}{2}-(2+5)}{10} \cdot 0{,}5=7{,}4.
 \]
 Tứ phân vị thứ hai $Q_{2}$ chính là trung vị $M_{\text{e}}$.
 \item Tứ phân vị thứ nhất $Q_{1}$\\
 Nhóm chứa $Q_{1}$ là nhóm $[7{,}0;7{,}5)$. Khi đó $Q_{1}=7{,}0+\dfrac{\dfrac{30}{4}-(2+5)}{10} \cdot 0{,}5=7{,}025$.
 \item Tứ phân vị thứ ba $Q_{3}$\\
 Nhóm chứa $Q_{3}$ là nhóm $[7{,}5;8{,}0)$.\\
 Khi đó $Q_{3}=7{,}5+\dfrac{\dfrac{3\cdot30}{4}-(2+5+10)}{9} \cdot 0{,}5=7{,}80(5) \approx 7{,}806$.
 \item Khoảng tứ phân vị của mẫu số liệu ghép nhóm trên là $\Delta_{Q}=Q_{3}-Q_{1}=7{,}806-7{,}025=0{,}781$.
 \end{itemize}
 \end{itemchoice}
 }
\end{ex}

\begin{ex}%[2-D3B3-SO-7-2425]%[VN-MT-7, Nguyễn Kiều Nhã Tú]%[2D3H2-3]
 Khảo sát thời gian xem điện thoại trong một ngày của một số học sinh khối $12$ thu được mẫu số liệu ghép nhóm sau:
 \begin{center}
 \begin{tabular}{|c|c|c|c|c|c|}
 \hline
 Thời gian (phút) & $[0;20)$ & $[20;40)$ & $[40;60)$ & $[60;80)$ & $[80;100)$ \\
 \hline
 Số học sinh & $4$ & $8$ & $12$ & $10$ & $8$ \\
 \hline
 \end{tabular}
 \end{center}
 \choiceTF
 {\True Tổng số học sinh được khảo sát là $42$}
 {Mốt của mẫu số liệu lớn hơn $54$}
 {\True Khoảng tứ phân vị của mẫu số liệu lớn hơn $38$}
 {\True Phương sai của mẫu số liệu nhỏ hơn $610$}
 \loigiai{
 \begin{itemchoice}
 \itemch \textbf{Đúng}.\\
 Tổng số học sinh được khảo sát là $n=4+8+12+10+8=42$.
 \itemch \textbf{Sai}.\\
 Nhóm có tần số lớn nhất là $[40;60)$.\\
 Mốt của mẫu số liệu là
 \[M_{\text{0}}=40+\dfrac{12-8}{(12-8)+(12-10)} \cdot(60-40) \approx 53{,}3.\]
 \itemch \textbf{Đúng}.\\
 Gọi $x_{1}$, $x_{2}$,$\ldots$, $x_{42}$ là thời gian xem điện thoại trong ngày của $42$ học sinh khối $12$ và giả sử dãy này đã sắp xếp theo thứ tự không giảm.\\
 Khi đó tứ phân vị thứ nhất của mẫu gốc là trung vị của dãy $x_{1}$, $x_{2}$,$\ldots$, $x_{21}$, tức là $x_{11}$. Do đó $Q_{1}$ thuộc nhóm $[20;40)$.\\
 Tứ phân vị thứ ba của mẫu gốc là trung vị của dãy $x_{22}$, $x_{2}$,$\ldots$, $x_{42}$, tức là $x_{32}$. Do đó $Q_{3}$ thuộc nhóm $[60;80)$.\\
 Suy ra $Q_{1}=20+\dfrac{\dfrac{42}{4}-4}{8} \cdot(40-20)=36{,}25$ và
 $Q_{3}=60+\dfrac{\dfrac{3 \cdot 42}{4}-24}{10} \cdot(80-60)=75$.\\
 Khoảng tứ phân vị của mẫu số liệu là $\Delta Q=Q_{3}-Q_{1}=75-36{,}25=38{,}75$.
 \itemch \textbf{Đúng}.\\
 Số trung bình của mẫu số liệu là
 \[
 \overline{x}=\dfrac{4 \cdot 10+8 \cdot 30+12 \cdot 50+10 \cdot 70+8 \cdot 90}{42} \approx 54{,}76.
 \]
 Phương sai của mẫu số liệu là
 \begin{eqnarray*}
 s^{2} & = & \dfrac{4 (10-54{,}76)^{2}+8 (30-54{,}76)^{2}+12 (50-54{,}76)^{2}+10 (70-54{,}76)^{2}+8 (90-54{,}76)^{2}}{42}\\
 & \approx & 605{,}9.
 \end{eqnarray*}
 \end{itemchoice}
 }
\end{ex}

\begin{ex}%[2-D3B3-SO-7-2425]%[VN-MT-7, Nguyễn Kiều Nhã Tú]%[2D3H2-3]
 Một trang trại phân $1\ 000$ quả trứng thành $5$ loại, tùy theo khối lượng (đã được làm tròn) của chúng được thống kê bởi bảng dưới đây:
 \begin{center}
 \begin{tabular}{|c|c|c|c|c|c|}
 \hline
 Khối lượng (gam) & $[30;36)$ & $[36;42)$ & $[42;48)$ & $[48;54)$ & $[54;60)$ \\
 \hline
 Số trứng & $45$ & $190$ & $500$ & $250$ & $15$ \\
 \hline
 \end{tabular}
 \end{center}
 \choiceTF
 {\True Tần suất của khối lượng trứng $[30;36)$ là $19 \%$}
 {Số trung vị của mẫu số liệu là $43$}
 {Khoảng biến thiên của mẫu số liệu $39{,}18$}
 {\True Độ lệch chuẩn của mẫu số liệu là $\dfrac{6 \sqrt{17}}{5}$}
 \loigiai{
 \begin{itemchoice}
 \itemch \textbf{Đúng}.\\
 Tần suất của khối lượng trứng $[30;36)$ là $\dfrac{190}{1\ 000} \cdot 100=19 \%$.
 \itemch \textbf{Sai}.\\
 Nhóm chứa trung vị là nhóm $[42;48)$.
 \[M_{\text{e}}=42+\dfrac{\dfrac{1\ 000}{2}-235}{500} \cdot(48-42)=\dfrac{2\ 259}{50}.\]
 \itemch \textbf{Sai}.\\
 Khoảng biến thiên của mẫu số liệu là $60-30=30$.
 \itemch \textbf{Đúng}.\\
 Ta có bảng sau:
 \begin{center}
 \begin{tabular}{|c|c|c|c|c|c|}
 \hline
 Khối lượng (gam) & $[30 ; 36)$ & $[36 ; 42)$ & $[42 ; 48)$ & $[48 ; 54)$ & $[54 ; 60)$ \\
 \hline
 Số trứng & $45$ & $190$ & $500$ & $250$ & $15$ \\
 \hline
 Giá trị đại diện & $33$ & $39$ & $45$ & $51$ & $57$ \\
 \hline
 \end{tabular}
 \end{center}
 Phương sai là
 \[
 s^{2}=\dfrac{33^{2} \cdot 45+39^{2} \cdot 190+45^{2} \cdot 500+51^{2} \cdot 250+57^{2} \cdot 15}{1000}-45^{2}=24{,}48.
 \]
 Vậy độ lệch chuẩn của mẫu số liệu là $s=\sqrt{24{,}48}=\dfrac{6 \sqrt{17}}{5}$.
 \end{itemchoice}
 }
\end{ex}

\begin{ex}%[2-D3B3-SO-7-2425]%[VN-MT-7, Nguyễn Kiều Nhã Tú]%[2D3V2-3]
 Bảng sau thống kê lại tổng số giờ nắng trong tháng $6$ của các năm từ $2002$ đến $2021$ tại hai trạm quan trắc đặt ở Nha Trang và Quy Nhơn.
 \begin{center}
 \begin{tabular}{|c|c|c|c|c|c|c|}
 \hline
 Số giờ nắng & $[130 ; 160)$ & $[160 ; 190)$ & $[190 ; 220)$ & $[220 ; 250)$ & $[250 ; 280)$ & $[280 ; 310)$ \\
 \hline
 Số liệu ở Nha Trang & $1$ & $1$ & $1$ & $8$ & $7$ & $2$ \\
 \hline
 Số liệu ở Quy Nhơn & $0$ & $1$ & $2$ & $4$ & $10$ & $3$ \\
 \hline
 \end{tabular}
 \end{center}
 \begin{flushright}
 (Nguồn: Tổng cục Thống kê)
 \end{flushright}
 \choiceTF
 {Xét số liệu ở Nha Trang thì khoảng tứ phân vị của mẫu số liệu ghép nhóm là $32{,}64$}
 {\True Nếu so sánh theo khoảng tứ phân vị thì số giờ nắng trong tháng $6$ của Quy Nhơn đồng đều hơn}
 {\True Xét số liệu của Quy Nhơn ta có độ lệch chuẩn của mẫu số liệu ghép nhóm (làm tròn kết quả đến hàng phần trăm) là $30{,}59$}
 {Nếu so sánh theo độ lệch chuẩn thì số giờ nắng trong tháng $6$ của Nha Trang đồng đều hơn}
 \loigiai{
 \begin{itemchoice}
 \itemch \textbf{Sai}.\\
 Cỡ mẫu $n=20$.\\
 Gọi $x_{1}$, $x_{2}$,$\ldots$, $x_{20}$ là mẫu số liệu gốc về số giờ nắng trong tháng $6$ trong $20$ năm của Nha Trang được xếp theo thứ tự không giảm.\\
 Ta có $x_{1} \in[130;160)$; $x_{2} \in[160;190)$; $x_{3} \in[190;220)$; $x_{4}$,$\ldots$, $x_{11} \in[220;250)$; $x_{12}$,$\ldots$,\\
 $x_{18} \in[250;280)$; $x_{19}$, $x_{20} \in[280;310)$.\\
 Tứ phân vị thứ nhất của mẫu số liệu gốc là $\dfrac{1}{2}\left(x_{5}+x_{6}\right) \in[220;250)$. Do đó, tứ phân vị thứ nhất của mẫu số liệu ghép nhóm là \[Q_{1}=220+\dfrac{\dfrac{20}{4}-(1+1+1)}{8}\cdot(250-220)=227{,}5.\]
 Tứ phân vị thứ ba của mẫu số liệu gốc là $\dfrac{1}{2}\left(x_{15}+x_{16}\right) \in[250;280)$. Do đó, tứ phân vị thứ ba của mẫu số liệu ghép nhóm là \[Q_{3}=250+\dfrac{\dfrac{3\cdot20}{4}-(1+1+1+8)}{7}\cdot(280-250)=\dfrac{1870}{7}.\]
 Khoảng tứ phân vị của mẫu số liệu ghép nhóm là $\Delta_{Q}=Q_{3}-Q_{1}\approx 39{,}64$.
 \itemch \textbf{Đúng}.\\
 Gọi $y_{1}$, $y_{2}$,$\ldots$, $y_{50}$ là mẫu số liệu gốc về số giờ nắng trong tháng $6$ trong $20$ năm của Quy Nhơn được xếp theo thứ tự không giảm.\\
 Ta có $y_{1} \in[160;190)$; $y_{2}$, $y_{3} \in[190;220)$; $y_{4}$,$\ldots$, $y_{7} \in[220;250)$; $y_{8} $,$\ldots$, $y_{17} \in[250;280)$;
 $y_{18}$,$\ldots$, $y_{20} \in[280;310)$.\\
 Tứ phân vị thứ nhất của mẫu số liệu gốc là $\dfrac{1}{2}\left(y_{5}+y_{6}\right) \in[220;250)$. Do đó, tứ phân vị thứ nhất của mẫu số liệu ghép nhóm là \[Q_{1}'=220+\dfrac{\dfrac{20}{4}-(1+2)}{4}\cdot(250-220)=235.\]
 Tứ phân vị thứ ba của mẫu số liệu gốc là $\dfrac{1}{2}\left(y_{15}+y_{16}\right) \in[250;280)$. Do đó, tứ phân vị thứ ba của mẫu số liệu ghép nhóm là \[Q_{3}'=250+\dfrac{\dfrac{3\cdot 20}{4}-(1+2+4)}{10}\cdot(280-250)=274.\]
 Khoảng tứ phân vị của mẫu số liệu ghép nhóm là $\Delta_{Q}'=Q_{3}'-Q_{1}'=39$.\\
 Vậy nếu so sánh theo khoảng tứ phân vị thì số giờ nắng trong tháng $6$ của Quy Nhơn đồng đều hơn.
 \itemch \textbf{Đúng}.\\
 Xét số liệu của Nha Trang
 \begin{itemize}
 \item Số trung bình 
 \[\overline{x}_{X}=\dfrac{1\cdot 145+1\cdot 175+1\cdot 205+8\cdot 235+7\cdot 265+2\cdot 295}{20}=242{,}5.\]
 \item Độ lệch chuẩn \[s_{X}=\sqrt{\dfrac{1\cdot 145^{2}+1\cdot 175^{2}+1\cdot 205^{2}+8\cdot 235^{2}+7\cdot 265^{2}+2\cdot 295^{2}}{20}-242{,}5^{2}} \approx 35{,}34.\]
 \end{itemize}
 \itemch \textbf{Sai}.\\
 Xét số liệu của Quy Nhơn
 \begin{itemize}
 \item Số trung bình 
 \[\overline{x}_{Y}=\dfrac{1\cdot 175+2\cdot 205+4\cdot 235+10\cdot 265+3\cdot 295}{20}=253.\]
 \item Độ lệch chuẩn 
 \[s_{Y}=\sqrt{\dfrac{1\cdot 175^{2}+2\cdot 205^{2}+4\cdot 235^{2}+10\cdot 265^{2}+3\cdot 295^{2}}{20}-253^{2}} \approx 30{,}59.\]
 \end{itemize}
 Vậy nếu so sánh theo độ lệch chuẩn thì số giờ nắng trong tháng $6$ của Quy Nhơn đồng đều hơn.
 \end{itemchoice}
 }
\end{ex}
\Closesolutionfile{ans}

\TNSA
\Opensolutionfile{ans}[ans/ans\currfilebase-Phan-III]

\begin{ex}%[2-D3B3-SO-7-2425]%[VN-MT-7, Nguyễn Kiều Nhã Tú]%[2D3N1-2]
 Chỉ số AQI là chỉ số thể hiện chất lượng không khí. Có $5$ thông số ảnh hưởng đến chỉ số AQI là Ozone mặt đất, ô nhiễm phân tử (bụi min PM$2.5$ và PM$10$), CO, NO$_2$, SO$_2$ (với NO$_2$, SO$_2$ là tác nhân gây ra mưa axit). Chỉ số AQI từ $0-50$ là mức tốt, từ $51-100$ là trung bình, từ $101-150$ là không tốt cho các nhóm nhạy cảm, từ $151-200$ là không lành mạnh, từ $201-300$ là rất không tốt, và trên $301$ là rất nguy hiểm. Hà Nội của chúng ta là một trong những thành phố ô nhiễm nhất thế giới. Ngày 5/3/2024 chỉ số AQI của Hà Nội đạt mức $241$ và là thành phố ô nhiễm nhất thế giới ngày hôm đó. Chỉ số AQI của một số các thành phố ngày $24/6/2024$ được cho trong bảng sau:
 \begin{center}
 \begin{tabular}{|c|c|c|c|c|}
 \hline
 Chỉ số AQI& $[0;50)$& $[50;100)$& $[100;150)$& $[150;200)$\\\hline
 Số thành phố& $73$& $47$& $7$& $2$\\\hline
 \end{tabular}
 \end{center}
 Khoảng biến thiên của mẫu số liệu trên là bao nhiêu?
 \par
 \shortans{200}
 \loigiai{Khoảng biến thiên của mẫu số liệu trên là $200-0=200$.}
\end{ex}

\begin{ex}%[2-D3B3-SO-7-2425]%[VN-MT-7, Nguyễn Kiều Nhã Tú]%[2D3V1-3]
 Thống kê lượng khách du lịch đến tỉnh Quảng Ninh từ năm $2007$ đến năm $2023$ (đơn vị: triệu người) cho kết quả như sau:
 \begin{center}
 \begin{tabular}{|c|c|c|c|c|c|c|c|c|}
 \hline
 $3{,}4$& $4{,}2$& $5{,}0$& $5{,}4$& $6{,}2$& $7$& $7{,}5$& $7{,}5$& $7{,}8$\\\hline
 $8{,}3$& $9{,}87$& $12{,}2$& $14$& $8{,}8$& $4{,}4$& $9{,}5$& $15{,}5$&\\\hline
 \end{tabular}
 \end{center}
 Ghép nhóm dãy số liệu trên thành các nhóm có độ dài bằng nhau với nhóm đầu tiên là $[1;5)$ rồi cho biết khoảng tứ phân vị của mẫu số liệu ghép nhóm trên.
 \par
 \shortans[]{4{,}44}
 \loigiai{
 Số lượng khách du lịch đến tỉnh Quảng Ninh được cho dưới bảng sau:
 \begin{center}
 \begin{tabular}{|c|c|c|c|c|}
 \hline
 Lượng khách (triệu người)& $[1;5)$& $[5;9)$& $[9;13)$& $[13;17)$\\\hline
 Số năm& $3$& $9$& $3$& $2$\\\hline
 \end{tabular}
 \end{center}
 Cỡ mẫu là $ n=3+9+3+2=17$.\\
 Gọi $x_1$, $x_2$, $\ldots$, $x_{17}$ là số khách đến Quảng Ninh du lịch và giả sử rằng dãy số liệu gốc này đã được sắp xếp theo thứ tự không giảm. \\
 Tứ phân vị thứ nhất của mẫu số liệu gốc này là $\dfrac{1}{2}\left(x_4+x_5\right)$ nên nhóm chứa tứ phân vị thứ nhất là nhóm $[5;9)$ và ta có
 \[Q_1=5+\dfrac{\dfrac{17}{4}-3}{9}\cdot 4\approx 5{,}56.\]
 Tứ phân vị thứ ba của mẫu số liệu gốc là $\dfrac{1}{2}\left(x_{13}+x_{14}\right)$ nên nhóm chứa tứ phân vị thứ ba là nhóm $[9;13)$ và ta có
 \[Q_3=9+\dfrac{\dfrac{3\cdot 17}{4}-12}{3} \cdot 4=10\]
 Vậy khoảng tứ phân vị của mẫu số liệu ghép nhóm là $\Delta Q=Q_3-Q_1=10-5{,}56=4{,}44$.
 }
\end{ex}

\begin{ex}%[2-D3B3-SO-7-2425]%[VN-MT-7, Nguyễn Kiều Nhã Tú]%[2D3H2-2]
 Chiều dài của $40$ bé sơ sinh $12$ ngày tuổi được chọn ngẫu nhiên ở viện nhi trung ương được nghiên cứu thống kê ở bảng dưới đây:
 \begin{center}
 \begin{tabular}{|c|c|c|c|c|c|c|c|}
 \hline
 Chiều dài (cm)& $[44;46)$& $[46;48)$& $[48;50)$&$[50;52)$& $[52;54)$& $[54;56)$& $[56;58)$\\
 \hline
 Số trẻ& $3$& $3$& $10$&$0$& $15$& $7$& $2$\\
 \hline
 \end{tabular}
 \end{center}
 Tìm độ lệch chuẩn (làm tròn đến hàng phần trăm) của $40$ bé sơ sinh ở bảng thống kê trên.
 \par
 \shortans[]{3{,}28}
 \loigiai{
 Ta có bảng phân bố của mẫu ghép nhóm $40$ bé sơ sinh:
 \begin{center}
 \begin{tabular}{|c|c|c|c|c|c|c|c|}
 \hline
 Chiều dài (cm)& $[44;46)$& $[46;48)$& $[48;50)$&$[50;52)$& $[52;54)$& $[54;56)$& $[56;58)$\\\hline
 Số trẻ& $3$& $3$& $10$&$0$& $15$& $7$& $2$\\\hline
 Chiều dài
 đại diện
 (cm)& $45$& $47$& $49$& $51$& $53$& $55$& $57$\\\hline
 \end{tabular}
 \end{center}
 Chiều dài trung bình của $40$ trẻ là
 \[\overline{x}=\dfrac{45\cdot3+47\cdot3+49\cdot10+51\cdot0+53\cdot15+55\cdot7+57\cdot2}{40}=51{,}5 \text{ (cm)}.\]
 Phương sai của $40$ bé sơ sinh ở bảng thống kê trên là
% \[s^2=\frac{3\cdot {{( 51{,}5-45 )}^2}+3\cdot {{( 51{,}5-47 )}^2}+ +2\cdot {{( 57-51{,}5 )}^2}}{40}=10{,}75.\]
 \[s^2=\dfrac{45^2\cdot3+47^2\cdot3+49^2\cdot10+51^2\cdot0+53^2\cdot15+55^2\cdot7+57^2\cdot2}{40}-51{,}5()^2=10{,}75.\]
 Độ lệch chuẩn của $40$ bé sơ sinh ở bảng thống kê trên là $s=\sqrt{s^2}\approx 3{,}28$.
 }
\end{ex}

\begin{ex}%[2-D3B3-SO-7-2425]%[VN-MT-7, Nguyễn Kiều Nhã Tú]%[2D3H2-2]
 Một công ty bất động sản Đất Vàng thực hiện cuộc khảo sát khách hàng xem họ có nhu cầu mua nhà ở mức giá nào để tiến hành dự án xây nhà ở Thăng Long group sắp tới. Kết quả khảo sát $500$ khách hàng được ghi lại ở bảng sau:
 \begin{center}
 \begin{tabular}{|c|c|c|c|c|c|}
 \hline
 Mức giá
 (triệu đồng)& $[10;14)$& $[14;18)$& $[18;22)$& $[22;26)$& $[26;30)$\\\hline
 Số khách hàng& $75$& $105$& $179$& $96$& $45$\\\hline
 \end{tabular}
 \end{center}
 Độ lệch chuẩn (làm tròn đến hàng phần trăm) của mức giá đất là bao nhiêu?
 \par
 \shortans[]{4{,}64}
 \loigiai{
 Bảng phân bố tần số tần suất của bảng số liệu của công ty bất động sản Đất Vàng như sau:
 \begin{center}
 \begin{tabular}{|c|c|c|c|c|c|}
 \hline
 Mức giá
 (triệu đồng)& $[10;14)$& $[14;18)$& $[18;22)$& $[22;26)$& $[26;30)$\\\hline
 Số khách hàng& $75$& $105$& $179$& $96$& $45$\\\hline
 Mức giá đại diện& $12$& $16$& $20$& $24$& $28$\\\hline
 \end{tabular}
 \end{center}
 Mức giá trung bình của công ty là 
 \[\overline{x}=\dfrac{75 \cdot 12 + 105 \cdot 16 + 179 \cdot 20 + 96 \cdot 24 + 45 \cdot 28}{500}\approx 19{,}45\, \text{(triệu đồng)}.\]
 Phương sai của mức giá là 
 \begin{eqnarray*}
 s^2&=&\dfrac{75 (12-19{,}45)^2+105(16-19{,}45)^2+179 (20-19{,}45)^2 + 96(24-19{,}45)^2 + 45 (28-19{,}45)^2}{500}\\
 &\approx& 21{,}49.
 \end{eqnarray*}
 Độ lệch chuẩn của mức giá $\sqrt {s^2} \approx 4{,}64$.
 }
\end{ex}

\begin{ex}%[2-D3B3-SO-7-2425]%[VN-MT-7, Nguyễn Kiều Nhã Tú]%[2D3H2-2]
 Bạn Minh Nhàn sử dụng điện thoại thông minh để chơi game trong một ngày. Số lần bạn sử dụng điện thoại được thống kê như sau:
 \begin{center}
 \begin{tabular}{|c|c|c|c|c|c|}
 \hline
 Thời gian (đơn vị: h)& $[3; 5)$ & $[5; 7)$ & $[7; 9)$ & $[9; 11)$ & $[11; 13)$ \\
 \hline
 Số lần sử dụng & $2$ & $5$ & $13$ & $8$ & $2$ \\
 \hline
 \end{tabular}
 \end{center}
 Hãy tính tỉ số phần trăm (làm tròn 1 chữ số thập phân) giữa độ lệch chuẩn và giá trị trung bình.
 \par
 \shortans[]{23{,}9}
 \loigiai{
 Ta có bảng sau:
 \begin{center}
 \begin{tabular}{|c|c|c|c|c|c|}
 \hline
 Thời gian (đơn vị: h) & $[3; 5)$ & $[5; 7)$ & $[7; 9)$ & $[9; 11)$ & $[11; 13)$ \\
 \hline
 Giá trị đại diện & $4$ & $6$ & $8$ & $10$ & $12$ \\
 \hline
 Số lần sử dụng & $2$ & $5$ & $13$ & $8$ & $2$ \\
 \hline
 \end{tabular}
 \end{center}
 Xét mẫu số liệu của Minh Nhàn $n=2+5+13+8+2=30$.\\
 Số trung bình của mẫu số liệu ghép nhóm là
 \[
 \overline{x}=\dfrac{2\cdot 4+5\cdot 6+13\cdot 8+8\cdot 10+2\cdot 12}{30}=8{,}2.
 \]
 Phương sai của mẫu số liệu ghép nhóm là
 \[
 s^2=\dfrac{1}{30}\left(2\cdot 4^2+5\cdot 6^2+13\cdot 8^2+8\cdot 10^2+2\cdot 12^2\right)-(8{,}2)^2=3{,}83.
 \]
 Độ lệch chuẩn của mẫu số liệu ghép nhóm là $s=\sqrt{s^2} \approx \sqrt{3{,}83} \approx 1{,}96$.\\
 Vậy tỉ số phần trăm giữa độ lệch chuẩn và giá trị trung bình là
 $\dfrac{1{,}96}{8{,}2}\cdot 100\%\approx 23{,}9\%$.
 }
\end{ex}

\begin{ex}%[2-D3B3-SO-7-2425]%[VN-MT-7, Nguyễn Kiều Nhã Tú]%[2D3H2-2]
 Điều tra chi phí thuê nhà ở hằng tháng của một số nhân viên độc thân, công ty $X$ thu được số liệu dưới đây:
 \begin{center}
 \begin{tabular}{|l|c|c|c|c|c|}
 \hline \begin{tabular}{l}
 {Tiền thuê nhà} \\
 {(trăm nghìn đồng)}
 \end{tabular}
 & $[3 ; 6)$ & $[6 ; 9)$ & $[9 ; 12)$ & $[12 ; 15)$ & $[15 ; 18)$ \\
 \hline {Số nhân viên} & $64$ & $40$ & $84$ & $56$ & $16$ \\
 \hline
 \end{tabular}
 \end{center}
 Tính độ lệch chuẩn chi phí thuê nhà hằng tháng của những nhân viên được điều tra (kết quả làm tròn đến hàng phần chục).
 \par
 \shortans[]{3{,}68}
 \loigiai{
 Bổ sung thêm các giá trị đại diện, ta có bảng sau:
 \begin{center}
 \begin{tabular}{|l|c|c|c|c|c|}
 \hline \begin{tabular}{l}
 {Tiền thuê nhà} \\
 {(trăm nghìn đồng)}
 \end{tabular}
 & $[3 ; 6)$ & $[6 ; 9)$ & $[9 ; 12)$ & $[12 ; 15)$ & $[15 ; 18)$ \\
 \hline { Giá trị đại diện} & $4{,}5$ & $7{,}5$ & $10{,5}$ & $13{,}5$ & $16{,}5$ \\
 \hline { Số nhân viên} & $64$ & $40$ & $84$ & $56$ & $16$ \\
 \hline
 \end{tabular}
 \end{center}
 Từ mẫu số liệu đã cho, ta tính được số trung bình là
 \[\overline{x}=\dfrac{4{,}5\cdot 64 + 7{,}5\cdot 40 + 10{,}5\cdot 84 + 13{,}5 \cdot 56 + 16{,}5\cdot 16}{64+40+84+56+16}\approx9{,}58.\]
 Từ đó ta có phương sai chi phí thuê nhà hàng tháng của những nhân viên được điều tra là
 \begin{eqnarray*}
 s^2&=&\dfrac{64 \left(4{,}5-\overline{x}\right)^2+40\left(7{,}5-\overline{x}\right)^2+84\left(10{,}5-\overline{x}\right)^2+56\left(13{,}5-\overline{x}\right)^2+16 \left(16{,}5-\overline{x}\right)^2}{64+40+84+56+16}\\
 &\approx& 13{,}55.
 \end{eqnarray*}
 Vậy độ lệch chuẩn chi phí thuê nhà hàng tháng của những nhân viên được điều tra là
 \[s=\sqrt{s^2}\approx 3{,}68.\]
 }
\end{ex}
\Closesolutionfile{ans}
% \begin{indapan}
% 	{ans/ans\currfilebase}
% \end{indapan}


% \begin{name}
 {Biên soạn: Lại Thị Hảo \\ Phản biện: Nguyễn Tài Tuệ}
 {Đề ôn tập chương III}
\end{name}

\caulc
\Opensolutionfile{ans}[ans/ans\currfilebase-Phan-I]

\begin{ex}%[2-D3B3-SO-8-2425]%[VN-MT-7, Lại Thị Hảo]%[2D3N1-2]
Xét mẫu số liệu ghép nhóm cho bởi bảng sau:
\begin{center}
\begin{tabular}{|c|c|c|c|c|c|c|}
 \hline
 Nhóm & $[40; 45)$ & $[45; 50)$ & $[50; 55)$ & $[55; 60)$ & $[60; 65)$ & \\
 \hline
 Tần số & $4$ & $11$ & $9$ & $8$ & $8$ & $n=40$ \\
 \hline
\end{tabular}
\end{center}
Khoảng biến thiên của mẫu số liệu ghép nhóm đã cho bằng
\choice
{$5$}
{$40$}
{$6$}
{\True $25$}
\loigiai{
 Ta có đầu mút trái của nhóm $1$ là $a_1=40$, đầu mút phải của nhóm $5$ là $a_6=65$.\\
 Vậy khoảng biến thiên của mẫu số liệu ghép nhóm đó là
 \[R=a_6-a_1=65-40=25.\] 
 }
\end{ex}

\begin{ex}%[2-D3B3-SO-8-2425]%[VN-MT-7, Lại Thị Hảo]%[2D3N1-1]
 Xét mẫu số liệu ghép nhóm cho bởi bảng sau:
\begin{center}
\begin{tabular}{|c|c|c|c|c|c|c|}
 \hline
 Nhóm & $[3; 13)$ & $[13; 23)$ & $[23; 33)$ & $[33; 43)$ & $[43; 53)$ & \\
 \hline
 Tần số & $8$ & $7$ & $10$ & $6$ & $9$ & $n=40$ \\
 \hline
\end{tabular}
\end{center}
 Tần số của nhóm $2$ của mẫu số liệu ghép nhóm đã cho bằng
 \choice
 {$6$}
 {\True $7$}
 {$9$}
 {$40$}
 \loigiai{Tần số của nhóm $2$ của mẫu số liệu ghép nhóm đã cho là $n_2=7$.}
\end{ex}

\begin{ex}%[2-D3B3-SO-8-2425]%[VN-MT-7, Lại Thị Hảo]%[2D3N1-4]
Xét mẫu số liệu ghép nhóm cho bởi bảng như hình bên.
 Tần số tích lũy $c f_2$ của nhóm $2$ của mẫu số liệu ghép nhóm đã cho bằng
 \choice
 {$4$}
 {$11$}
 {\True $15$}
 {$40$}
 \begin{center}
 \begin{tabular}{|c|c|c|c|c|c|c|}
 \hline
 Nhóm & $[17; 21)$ & $[21; 25)$ & $[25; 29)$ & $[29; 33)$ & $[33; 37)$ & \\
 \hline
 Tần số & $5$ & $10$ & $6$ & $7$ & $12$ & $n=40$ \\
 \hline
 \end{tabular}
 \end{center}
\loigiai{
 Ta có $c f_2=n_1+n_2=15$.
}
\end{ex}

\begin{ex}%[2-D3B3-SO-8-2425]%[VN-MT-7, Lại Thị Hảo]%[2D3N1-1]
 Khảo sát thời gian tập thể dục trong ngày của một số học sinh khối $11$ thu được mẫu số liệu ghép nhóm sau:
 \begin{center}
 \begin{tabular}{|c|c|c|c|c|c|}
 \hline
 Thời gian (phút) & $[0; 20)$ & $[20; 40)$ & $[40; 60)$ & $[60; 80)$ & $[80; 100)$\\
 \hline
 Số học sinh & $5$ & $9$ & $12$ & $10$ & $6$ \\
 \hline
 \end{tabular}
 \end{center}
 Giá trị đại diện của nhóm $[60; 80)$ là
 \choice
 {$10$}
 {$20$}
 {\True $70$}
 {$40$}
 \loigiai{
 Giá trị đại diện của nhóm $[60; 80)$ là $\dfrac{60+80}{2}=70$.
 }
\end{ex}

\begin{ex}%[2-D3B3-SO-8-2425]%[VN-MT-7, Lại Thị Hảo]%[2D3N1-1]
 Mẫu số liệu dưới đây ghi lại tốc độ của $40$ ô tô khi đi qua một trạm đo tốc độ (đơn vị: km/h):
 \begin{center}
 \begin{tabular}{|l|c|c|c|c|c|c|}
 \hline
 Tốc độ (km/h) & $[40; 45)$ & $[45; 50)$ & $[50; 55)$ & $[55; 60)$ & $[60; 65)$ & $[65; 70)$\\
 \hline
 Số ô tô & $4$ & $11$ & $7$ & $8$ & $8$ & $2$ \\
 \hline
 \end{tabular}
 \end{center}
 Độ dài của nhóm $[55; 60)$ là
 \choice
 {$10$}
 {$55$}
 {\True $5$}
 {$60$}
 \loigiai{
 Độ dài của nhóm $[55; 60)$ là $60-55=5$.
 }
\end{ex}

\begin{ex}%[2-D3B3-SO-8-2425]%[VN-MT-7, Lại Thị Hảo]%[1D5H1-3]
 Người ta đếm số xe ô tô đi qua một trạm thu phí mỗi phút trong khoảng thời gian từ $9$ giờ đến $9$ giờ $30$ phút sáng. Kết quả được ghi lại ở bảng sau:
 \begin{center}
 \begin{tabular}{|c|c|c|c|c|c|}
 \hline
 Số xe & $[6; 10]$ & $[11; 15]$ & $[16; 20]$ & $[21; 25]$ & $[26; 30]$\\
 \hline
 Số lần & $5$ & $9$ & $3$ & $9$ & $4$ \\
 \hline
 Giá trị đại diện & $8$ & $13$ & $18$ & $23$ & $28$ \\
 \hline
 \end{tabular}
 \end{center}
 Tính số trung bình cộng của mẫu số liệu ghép nhóm trên.
 \choice
 {$18{,}4$}
 {$18{,}7$}
 {$17{,}4$}
 {\True $17{,}7$}
 \loigiai{
 Số xe trung bình đi qua trạm trong mỗi phút xấp xỉ bằng
 \[\overline{x}=\dfrac{5 \cdot 8+9 \cdot 13+3 \cdot 18+9 \cdot 23+4 \cdot 28}{30}=17{,}7.\]
 }
\end{ex}

\begin{ex}%[2-D3B3-SO-8-2425]%[VN-MT-7, Lại Thị Hảo]%[1D5H2-3]
 Xét mẫu số liệu ghép nhóm cho bởi bảng sau:
\begin{center}
 \begin{tabular}{|c|c|c|}
 \hline
 Nhóm & Tần số & Tần số tích lũy\\
 \hline
 $[160; 163)$ & $6$ & $6$ \\
 $[163; 166)$ & $11$ & $17$ \\
 $[166; 169)$ & $9$ & $26$ \\
 $[169; 172)$ & $7$ & $33$ \\
 $[172; 175)$ & $3$ & $36$ \\
 \hline
 & $n=36$ & \\
 \hline
\end{tabular}
\end{center}
Tứ phân vị thứ nhất của mẫu số liệu ghép nhóm đã cho bằng
\choice
{\True $\dfrac{1802}{11}$}
{$163$}
{$9$}
{$\dfrac{329}{2}$}
 \loigiai{
Số phần tử của mẫu là $n=36$.\\
Ta có $\dfrac{n}{4}=\dfrac{36}{4}=9$ mà $6< 9< 17$. Suy ra nhóm $2$ là nhóm đầu tiên có tần số tích lũy lớn hơn hoặc bằng $9$.\\
Xét nhóm $2$ là nhóm $[163; 166)$ có $s=163$, $h=3$, $n_2=11$ và nhóm $1$ là nhóm $[160; 163)$ có $c f_1=6$.\\
Tứ phân vị thứ nhất là
\[Q_1=s+\left(\dfrac{9-c f_1}{n_2}\right) \cdot h=163+\left(\dfrac{9-6}{11}\right) \cdot 3=\dfrac{1802}{11}.\]
 }
\end{ex}

\begin{ex}%[2-D3B3-SO-8-2425]%[VN-MT-7, Lại Thị Hảo]%[1D5H2-3]
 Xét mẫu số liệu ghép nhóm cho bởi bảng sau:
\begin{center}
 \begin{tabular}{|c|c|c|}
\hline
Nhóm & Tần số & Tần số tích lũy\\
\hline
$[40; 45)$ & $5$ & $5$ \\
$[45; 50)$ & $10$ & $15$ \\
$[50; 55)$ & $7$ & $22$ \\
$[55; 60)$ & $9$ & $31$ \\
$[60; 65)$ & $7$ & $38$ \\
$[65; 70)$ & $4$ & $42$ \\
\hline
& $n=42$ & \\
\hline
 \end{tabular}
\end{center}
 Tứ phân vị thứ hai của mẫu số liệu ghép nhóm đã cho bằng
 \choice
 {\True $\dfrac{380}{7}$}
 {$50$}
 {$\dfrac{42}{7}$}
 {$\dfrac{105}{2}$}
\loigiai{
 Số phần tử của mẫu là $n=42$.\\
 Ta có $\dfrac{n}{2}=\dfrac{42}{2}=21$ mà $15< 21< 22$. Suy ra nhóm $3$ là nhóm đầu tiên có tần số tích lũy lớn hơn hoặc bằng $21$.\\
 Xét nhóm $3$ là nhóm $[50; 55)$ có $r=50$, $d=5$, $n_3=7$ và nhóm $2$ là nhóm $[45; 50)$ có $c f_2=15$.\\
 Tứ phân vị thứ hai là
 \[Q_2=r+\left(\dfrac{21-c f_2}{n_3}\right) \cdot d=50+\left(\dfrac{21-15}{7}\right) \cdot 5=\dfrac{380}{7}.\]
}
\end{ex}

\begin{ex}%[2-D3B3-SO-8-2425]%[VN-MT-7, Lại Thị Hảo]%[1D5H2-3]
Doanh thu bán hàng trong $20$ ngày được lựa chọn ngẫu nhiên của một của hàng được ghi lại ở bảng sau (đơn vị: triệu đồng):
\begin{center}
 \begin{tabular}{|c|c|c|c|c|c|}
 \hline
 Doanh thu & $[5; 7)$ & $[7; 9)$ & $[9; 11)$ & $[11; 13)$ & $[13; 15)$\\
 \hline
 Số ngày & $2$ & $7$ & $7$ & $3$ & $1$ \\
 \hline
\end{tabular}
\end{center}
Tứ phân vị thứ ba của mẫu số liệu gần nhất với giá trị nào trong các giá trị dưới đây?
\choice
{$10$}
{\True $11$}
{$12$}
{$13$}
 \loigiai{
Gọi $x_1$, $x_2$, $\ldots$, $x_{20}$ là doanh thu bán hàng trong $20$ ngày xếp theo thứ tự không giảm.\\
Khi đó $x_1$, $x_2 \in[5; 7)$; $x_3, \ldots, x_9 \in[7; 9)$; $x_9, \ldots, x_{16} \in[9; 11)$; $x_{17}, \ldots, x_{19} \in[11; 13)$; $x_{20} \in[13; 15)$.\\
Do đó, tứ phân vị thứ ba của mẫu số liệu thuộc nhóm $[9; 11)$.\\
Ta có $n=20$, $n_m=7$, $C=9$, $u_m=9$, $u_{m+1}=11$. Khi đó
\[Q_3=9+\dfrac{\dfrac{3 \cdot 20}{4}-9}{7}\cdot(11-9) =\dfrac{75}{7}\approx 11.\]
}
\end{ex}

\begin{ex}%[2-D3B3-SO-8-2425]%[VN-MT-7, Lại Thị Hảo]%[2D3H1-3]
Mẫu số liệu đây ghi lại tốc độ của $40$ ô tô khi đi qua một trạm đo tốc độ (đơn vị: km/h) được lập bảng tần số ghép nhóm như sau:
\begin{center}
\begin{tabular}{|c|c|c|c|c|c|c|}
 \hline
 Nhóm & $[40; 45)$ & $[45; 50)$ & $[50; 55)$ & $[55; 60)$ & $[60; 65)$ & $[65; 70)$ \\
 \hline
 Giá trị đại diện & $42{,}5$ & $47{,}5$ & $52{,}5$ & $57{,}5$ & $62{,}5$ & $67{,}5$ \\
 \hline
 Tần số & $4$ & $11$ & $7$ & $8$ & $8$ & $2$ \\
 \hline
\end{tabular}
\end{center}
 Khoảng tứ phân vị của mẫu số liệu trên gần bằng số nào dưới đây?
 \choice
 {$11{,}5$}
 {\True $12{,}3$}
 {$14{,}6$}
 {$23$}
\loigiai{
 Số phần tử của mẫu là $n=40$.\\
 Ta có $\dfrac{n}{4}=\dfrac{40}{4}=10$. Suy ra nhóm $2$ là nhóm đầu tiên có tần số tích lũy lớn hơn hoặc bằng $10$.\\
 Xét nhóm $2$ là nhóm $[45; 50)$ có $r=45$; $d=5$; $n_2=11$ và nhóm $1$ là nhóm $[40; 45)$ có $c f_1=4$.\\
 Áp dụng công thức, ta có $Q_1$ của mẫu số liệu là 
 \[Q_1=45+\left(\dfrac{10-4}{11}\right) \cdot 5\approx 47{,}7\, (\mathrm{km/h}).\]
 Ta có $\dfrac{3n}{4}=30$. Suy ra nhóm $4$ là nhóm đầu tiên có tần số tích lũy lớn hơn hoặc bằng $30$.\\ 
 Xét nhóm $4$ là nhóm $[55; 60)$ có $r=55$; $d=5$; $n_4=8$ và nhóm $3$ là nhóm $\left[50; 55\right)$ có $c f_3=22$.\\
 Áp dụng công thức, ta có $Q_3$ của mẫu số liệu là
 \[Q_3=55+\left(\dfrac{30-22}{8}\right) \cdot 5=60\,(\mathrm{km/h}).\]
 Do đó $\Delta_Q=Q_3-Q_1=60-\dfrac{525}{11}=\dfrac{135}{11} \approx 12{,}3$.
}
\end{ex}

\begin{ex}%[2-D3B3-SO-8-2425]%[VN-MT-7, Lại Thị Hảo]%[2D3H2-2]
Mỗi ngày bác An đều đi bộ để rèn luyện sức khỏe. Quãng đường đi bộ mỗi ngày (đơn vị: km) của bác An trong $20$ ngày được thống kê lại ở bảng sau:
\begin{center}
 \begin{tabular}{|c|c|c|c|c|c|}
\hline
Quãng đường (km) & $[2{,}7; 3{,}0)$ & $[3{,}0; 3{,}3)$ & $[3{,}3; 3{,}6)$ & $[3{,}6; 3{,}9)$ & $[3{,}9; 4{,}2)$\\
\hline
Số ngày & $3$ & $6$ & $5$ & $4$ & $2$ \\
\hline
 \end{tabular}
\end{center}
 Phương sai của mẫu số liệu ghép nhóm là
 \choice
 {$3{,}39$}
 {$11{,}62$}
 {\True $0{,}1314$}
 {$0{,}36$}
\loigiai{
 Ta có bảng sau:
\begin{center}
 \begin{tabular}{|c|c|c|c|c|c|}
\hline
Quãng đường (km) & $[2{,}7; 3{,}0)$ & $[3{,}0; 3{,}3)$ & $[3{,}3; 3{,}6)$ & $[3{,}6; 3{,}9)$ & $[3{,}9; 4{,}2)$\\
\hline
Giá trị đại diện & $2{,}85$ & $3{,}15$ & $3{,}45$ & $3{,}75$ & $4{,}05$ \\
\hline
Số ngày & $3$ & $6$ & $5$ & $4$ & $2$ \\
\hline
 \end{tabular}
\end{center}
 Số trung bình của mẫu số liệu ghép nhóm là
 \[\overline{x}=\dfrac{3 \cdot 2{,}85+6 \cdot 3{,}15+5 \cdot 3{,}45+4 \cdot 3{,}75+2 \cdot 4{,}05}{20}=3{,}39.\]
 Phương sai của mẫu số liệu ghép nhóm là 
 \[s^2=\dfrac{3 (2{,}85-3{,}39)^2+6(3{,}15-3{,}39)^2+5(3{,}45-3{,}39)^2+4(3{,}75-3{,}39)^2+2 (4{,}05-3{,}39)^2}{20}=0{,}1314.\]
}
\end{ex}

\begin{ex}%[2-D3B3-SO-8-2425]%[VN-MT-7, Lại Thị Hảo]%[2D3H2-2]
 Một bác tài xế thống kê lại độ dài quãng đường (đơn vị: km) bác đã lái xe mỗi ngày trong một tháng ở bảng sau:
\begin{center}
 \begin{tabular}{|c|c|c|c|c|c|}
 \hline
 Độ dài quãng đường (km) & $[50; 100)$ & $[100; 150)$ & $[150; 200)$ & $[200; 250)$ & $[250; 300)$\\
 \hline
 Số ngày & $5$ & $10$ & $9$ & $4$ & $2$ \\
 \hline
\end{tabular}
\end{center}
Độ lệch chuẩn của mẫu số liệu ghép nhóm gần bằng
\choice
{$33{,}91$}
{$155{,}15$}
{\True $55{,}68$}
{$36{,}54$}
 \loigiai{
Ta có bảng sau:
\begin{center}
 \begin{tabular}{|c|c|c|c|c|c|}
 \hline
Độ dài quãng đường (km)& $[50; 100)$ & $[100; 150)$ & $[150; 200)$ & $[200; 250)$ & $[250; 300)$\\
 \hline
 Giá trị đại diện & $75$ & $125$ & $175$ & $225$ & $275$ \\
 \hline
 Số ngày & $3$ & $6$ & $5$ & $4$ & $2$\\
 \hline
\end{tabular}
\end{center}
Số trung bình của mẫu số liệu ghép nhóm là
\[\overline{x}=\dfrac{5 \cdot 75+10 \cdot 125+9 \cdot 175+4 \cdot 225+2 \cdot 275}{30}=155.\]
Phương sai của mẫu số liệu ghép nhóm là
\[s^2=\dfrac{5 \cdot (75-155)^2+10 \cdot (125-155)^2+9 \cdot (175-155)^2+4 \cdot (225-155)^2+2 \cdot (275-155)^2}{30}=3\,100.\]
Độ lệch chuẩn của mẫu số liệu ghép nhóm là 
\[s=\sqrt{s^2}=\sqrt{3\,100} \approx 55{,}68.\]
 }
\end{ex}
\Closesolutionfile{ans}

\cauds
\Opensolutionfile{ans}[ans/ans\currfilebase-Phan-II]

\begin{ex}%[2-D3B3-SO-8-2425]%[VN-MT-7, Lại Thị Hảo]%[2D3N1-3]
 Cho bảng số liệu sau:
\begin{center}
 \begin{tabular}{|c|c|c|c|c|c|}
 \hline
 Nhóm & $[20; 25)$ & $[25; 30)$ & $[30; 35)$ & $[35; 40)$ & $[40; 45)$\\
 \hline
 Tần số & $6$ & $6$ & $4$ & $1$ & $1$ \\
 \hline
 \end{tabular}
\end{center}
 \choiceTF
 {\True Khoảng biến thiên của mẫu số liệu ghép nhóm là $25$}
 {\True Tần số của nhóm hai là $6$}
 {Tần số tích lũy của nhóm ba là $4$}
 {Khoảng tứ phân vị của mẫu số liệu ghép nhóm là hiệu giữa tứ phân vị thứ ba và tứ phân vị thứ hai của mẫu số liệu ghép nhóm}
 \loigiai{
\begin{itemchoice}
\itemch \textbf{Đúng}.\\
Khoảng biến thiên của mẫu số liệu ghép nhóm là $R=45-20=25$.
\itemch \textbf{Đúng}.\\
Tần số của nhóm hai (nhóm $[25;30)$) là $6$.
\itemch \textbf{Sai}.\\
Tần số tích lũy của nhóm ba là $6+6+4=16$.
\itemch \textbf{Sai}.\\
Khoảng tứ phân vị của mẫu số liệu ghép nhóm là hiệu giữa tứ phân vị thứ ba và tứ phân vị thứ nhất của mẫu số liệu ghép nhóm.
\end{itemchoice}}
\end{ex}

\begin{ex}%[2-D3B3-SO-8-2425]%[VN-MT-7, Lại Thị Hảo]%[2D3H1-3]
 Một vườn thú ghi lại tuổi thọ (đơn vị: năm) của $20$ con hổ và thu được kết quả như sau:
\begin{center}
 \begin{tabular}{|c|c|c|c|c|c|}
\hline
Tuổi thọ & $[14; 15)$ & $[15; 16)$ & $[16; 17)$ & $[17; 18)$ & $[18; 19)$\\
\hline
Số con hổ & $1$ & $3$ & $8$ & $6$ & $2$ \\
\hline
 \end{tabular}
\end{center}
 \choiceTF
 {\True Khoảng biến thiên của mẫu số liệu ghép nhóm này là $5$}
 {\True Nhóm chứa tứ phân vị thứ nhất là $[16; 17)$}
 {Nhóm chứa tứ phân vị thứ ba là $[18; 19)$}
 {\True Tần số tích lũy của nhóm $[17; 18)$ là $18$}
\loigiai{
\begin{itemchoice}
\itemch \textbf{Đúng}.\\
Khoảng biến thiên $R=19-14=5$.
\itemch \textbf{Đúng}.\\
Cỡ mẫu là $1+3+8+6+2=20$.\\
Gọi $x_1, x_2, \ldots, x_{20}$ là tuổi thọ của $20$ con hổ được sắp xếp theo thứ tự không giảm.\\
Tứ phân vị thứ nhất của mẫu số liệu gốc là $\dfrac{x_5+x_6}{2} \in[16; 17)$ nên nhóm chứa tứ phân vị thứ nhất là $[16; 17)$.
\itemch \textbf{Sai}.\\
Tứ phân vị thứ ba của mẫu số liệu gốc là $\dfrac{x_{15}+x_{16}}{2} \in[17; 18)$. Do đó nhóm chứa tứ phân vị thứ ba là $[17; 18)$.
\itemch \textbf{Đúng}.\\
Tần số tích lũy của nhóm $[17; 18)$ là $1+3+8+6=18$.
\end{itemchoice}
}
\end{ex}

\begin{ex}%[2-D3B3-SO-8-2425]%[VN-MT-7, Lại Thị Hảo]%[1D5H2-2]
Cho mẫu số liệu ghép nhóm về lương của nhân viên trong phòng kế toán tổng hợp một công ty X như sau:
\begin{center}
 \begin{tabular}{|l|c|c|c|c|c|}
\hline
 Lương (triệu đồng) & $[6; 9)$ & $[9; 12)$ & $[12; 15)$ & $[15; 18)$ & $[18; 21)$\\
\hline
Số nhân viên & $6$ & $5$ & $3$ & $2$ & $1$ \\
\hline
\end{tabular}
\end{center}
\choiceTF
{\True Giá trị đại diện của nhóm $[6; 9)$ là $7{,}5$}
{\True Trung bình lương các nhân viên là $11{,}2$ triệu đồng}
{Nhóm chứa trung vị là $[12; 15)$}
{\True Độ dài nhóm $[15; 18)$ là $3$}
\loigiai{
 \begin{itemchoice}
 \itemch \textbf{Đúng}.\\
 Giá trị đại diện của nhóm $[6; 9)$ là $\dfrac{6+9}{2}=7{,}5$.
 \itemch \textbf{Đúng}.\\
 Trung bình lương các nhân viên là
 \[\overline{x}=\dfrac{1}{17}(6\cdot 7{,}5+5\cdot 10{,}5+3\cdot 13{,}5+2\cdot 16{,}5+19{,}5)=11{,}2\, \text{(triệu đồng)}.\]
 \itemch \textbf{Sai}.\\
 Phòng kế toán có $17$ nhân viên. Vì $x_9 \in[9; 12)$ nên nhóm này chứa trung vị.
 \itemch \textbf{Đúng}.\\
 Độ dài nhóm $[15; 18)$ là $18-15=3$.
 \end{itemchoice}
 }
\end{ex}

\begin{ex}%[2-D3B3-SO-8-2425]%[VN-MT-7, Lại Thị Hảo]%[2D3V2-2]
 Cho mẫu số liệu ghép nhóm thống kê chiều cao (đơn vị: cm) của $45$ học sinh lớp 9A như sau:
 \begin{center}
 \begin{tabular}{|c|c|c|c|c|c|}
 \hline
 Nhóm & $[145; 150)$ & $[150; 155)$ & $[155; 160)$ & $[160; 165)$ & $[165; 170)$ \\
 \hline
 Tần số & $8$ & $12$ & $15$ & $6$ & $4$ \\
 \hline
 \end{tabular}
 \end{center}
 \choiceTF
 {Giá trị đại diện của nhóm $[150; 155)$ là $152$\,cm}
 {\True Chiều cao trung bình của học sinh là $155{,}94$\,cm}
 {Phương sai của mẫu số liệu (làm tròn đến hàng phần trăm) là $36{,}04$}
 {\True Độ lệch chuẩn của mẫu số liệu (làm tròn đến hàng phần trăm) là $5{,}85$}
\loigiai{
\begin{itemchoice}
 \itemch \textbf{Sai}.\\
 Giá trị đại diện của nhóm $[150; 155)$ là $\dfrac{150+155}{2}=152{,}5$.
 \itemch \textbf{Đúng}.\\
 Ta có bảng giá trị đại diện như sau:
 \begin{center}
 \begin{tabular}{|l|c|c|}
 \hline
 Nhóm & Giá trị đại diện & Tần số \\
 \hline
 $[145; 150)$ & $147{,}5$ & $8$ \\
 \hline
 $[150; 155)$ & $152{,}5$ & $12$ \\
 \hline
 $[155; 160)$ & $157{,}5$ & $15$ \\
 \hline
 $[160; 165)$ & $162{,}5$ & $6$ \\
 \hline
 $[165; 170)$ & $167{,}5$ & $4$ \\
 \hline
 \end{tabular}
 \end{center}
 Chiều cao trung bình của học sinh là
 \[\overline{x}=\dfrac{147{,}5 \cdot 8+152{,}5\cdot 12+157{,}5\cdot 15+162{,}5\cdot 6+167{,}5\cdot 4}{45}=\dfrac{2\,807}{18} \approx 155{,}94.\]
 \itemch \textbf{Sai}.\\
 Phương sai của mẫu số liệu là 
 \[s^2=\dfrac{8(147{,}5-155{,}94)^2+12(152{,}5-155{,}94)^2+\cdots+4(167{,}5-155{,}94)^2}{45}=\dfrac{2\,774}{81}\approx 34{,}25.\]
 \itemch \textbf{Đúng}.\\
 Độ lệch chuẩn $s=\sqrt{s^2}\approx 5{,}85$.
\end{itemchoice}
 }
\end{ex}
\Closesolutionfile{ans}

\caukq
\Opensolutionfile{ans}[ans/ans\currfilebase-Phan-III]

\begin{ex}%[2-D3B3-SO-8-2425]%[VN-MT-7, Lại Thị Hảo]%[2D3N1-2]
 Cho mẫu số liệu ghép nhóm số tiền điện phải trả trong một tháng của các hộ gia đình ở một khu phố (đơn vị: ngàn đồng) như sau:
\begin{center}
 \begin{tabular}{|l|c|c|c|c|c|c|}
 \hline
 Nhóm & $[375; 450)$ & $[450; 525)$ & $[525; 600)$ & $[600; 675)$ & $[675; 750)$ & $[750; 825]$\\
 \hline
 Tần số & $6$ & $15$ & $10$ & $6$ & $9$ & $4$ \\
 \hline
 \end{tabular}
\end{center}
 Tìm khoảng biến thiên của mẫu số liệu ghép nhóm trên.
 
 \shortans[]{450}
 \loigiai{
 Khoảng biến thiên của mẫu số liệu ghép nhóm trên là $R=a_7-a_1=825-375=450$. 
 }
\end{ex}

\begin{ex}%[2-D3B3-SO-8-2425]%[VN-MT-7, Lại Thị Hảo]%[2D3N1-2]
 Cho mẫu số liệu ghép nhóm về tuổi thọ (đơn vị tính là năm) của một loại bóng đèn mới như sau:
 \begin{center}
 \begin{tabular}{|l|c|c|c|c|}
 \hline
 Tuổi thọ & $[2; 3{,}5)$ & $[3{,}5; 5)$ & $[5; 6{,}5)$ & $[6{,}5; 8)$\\
 \hline
 Số bóng đèn & $8$ & $22$ & $35$ & $15$ \\
 \hline
 \end{tabular}
 \end{center}
 Tìm khoảng biến thiên của mẫu số liệu trên.
 
 \shortans[]{6}
 \loigiai{
 Khoảng biến thiên của mẫu số liệu trên là $8-2=6$. 
 }
\end{ex}

\begin{ex}%[2-D3B3-SO-8-2425]%[VN-MT-7, Lại Thị Hảo]%[2D3N1-4]
Cho bảng tần số ghép nhóm số liệu thống kê chiều cao của $38$ mẫu cây ở một vườn thực vật (đơn vị: centimét) như sau:
\begin{center}
 \begin{tabular}{|c|c|c|c|c|c|c|}
 \hline
 Nhóm & $[30; 40)$ & $[40; 50)$ & $[50; 60)$ & $[60; 70)$ & $[70; 80)$ & \\
 \hline
 Tần số & $4$ & $10$ & $14$ & $6$ & $4$ & $n=38$ \\
 \hline
 \end{tabular}
\end{center}
 Tần số tích luỹ của nhóm $4$ bằng bao nhiêu?
 \par
\shortans[]{34}
 \loigiai{
 Ta có bảng số liệu ghép nhóm như sau:
\begin{center}
 \begin{tabular}{|l|c|c|}
 \hline
 Nhóm & Tần số & Tần số tích lũy \\
 \hline
 $[30; 40)$ & $4$ & $4$ \\
 \hline
 $[40; 50)$ & $10$ & $14$ \\
 \hline
 $[50; 60)$ & $14$ & $28$ \\
 \hline
 $[60; 70)$ & $6$ & $34$ \\
 \hline
 $[70; 80)$ & $4$ & $38$ \\
 \hline
 \end{tabular}
\end{center} 
 Vậy tần số tích luỹ của nhóm $4$ là $34$. 
 }
\end{ex}

\begin{ex}%[2-D3B3-SO-8-2425]%[VN-MT-7, Lại Thị Hảo]%[1D5H1-3]
 Cân nặng của một số quả mít trong một khu vườn được thống kê ở bảng sau:
\begin{center}
 \begin{tabular}{|c|c|c|c|c|c|}
 \hline
 Cân nặng (kg) & $[4; 6)$ & $[6; 8)$ & $[8; 10)$ & $[10; 12)$ & $[12; 14)$\\
 \hline
 Số quả mít & $6$ & $12$ & $19$ & $9$ & $4$ \\
 \hline
 \end{tabular}
\end{center} 
 Tính cân nặng trung bình của một quả mít.
 
 \shortans[]{8{,}72}
 \loigiai{
 Số trung bình cộng của mẫu số liệu ghép nhóm là
 \[\overline{x}=\dfrac{6 \cdot 5+12 \cdot 7+19 \cdot 9+9 \cdot 11+4\cdot 13}{50}=8{,}72.\]
 Vậy cân nặng trung bình của một quả mít là $8{,}72$ kg.
 }
\end{ex}

\begin{ex}%[2-D3B3-SO-8-2425]%[VN-MT-7, Lại Thị Hảo]%[1D5H2-3]
 Để đánh giá chất lượng dịch vụ tài xế công nghệ của hãng X, người ta ghi lại thời gian chờ của các khách hàng được thể hiện trong bảng sau:
\begin{center}
 \begin{tabular}{|c|c|c|c|c|c|}
 \hline
 Thời gian chờ (phút)& $[1; 2{,}5)$ & $[2{,}5; 4)$ & $[4; 5{,}5)$ & $[5{,}5; 7)$ & $[7; 8{,}5)$\\
 \hline
 Lượng khách hàng (tần số) & $10$ & $5$ & $23$ & $6$ & $3$ \\
 \hline
 \end{tabular}
\end{center}
 Tìm tứ phân vị thứ nhất của mẫu số liệu trên (kết quả làm tròn đến hàng phần trăm).
 
 \shortans[]{3{,}03}
 \loigiai{
 Cỡ mẫu là $n=10+5+23+6+3=47$.\\
 Gọi $x_1, \ldots, x_{47}$ là thời gian chờ của $47$ khách hàng và giả sử số liệu gốc này đã được sắp xếp theo thứ tự không giảm.\\
 Tứ phân vị thứ nhất của mẫu số liệu gốc là $x_{12}$ nên nhóm chứa $Q_1$ là nhóm $[2{,}5; 4)$.\\
 Khi đó $Q_1=2{,}5+\dfrac{\dfrac{1 \cdot 47}{4}-10}{5} \cdot 1{,}5=3{,}025\approx 3{,}03$.
 
 }
\end{ex}

\begin{ex}%[2-D3B3-SO-8-2425]%[VN-MT-7, Lại Thị Hảo]%[2D3H2-2]
 Tìm hiểu thời gian sử dụng điện thoại trong một ngày của các bạn học sinh lớp 12A được ghi lại trong bảng sau:
\begin{center}
 \begin{tabular}{|l|c|c|c|c|}
 \hline
 Thời gian (giờ) & $[0; 1{,}5)$ & $[1{,}5; 3)$ & $[3; 4{,}5)$ & $[4{,}5; 6)$\\
 \hline
 Số học sinh & $8$ & $12$ & $6$ & $4$ \\
 \hline
 \end{tabular}
\end{center}
 Tìm phương sai của mẫu số liệu trên.
 
 \shortans[]{2{,}16}
 \loigiai{
 Chọn giá trị đại diện cho các nhóm số liệu, ta có:
\begin{center}
 \begin{tabular}{|l|c|c|c|c|}
 \hline
 Thời gian (giờ) & $[0; 1{,}5)$ & $[1{,}5; 3)$ & $[3; 4{,}5)$ & $[4{,}5; 6)$\\
 \hline
 Giá trị đại diện & $0{,}75$ & $2{,}25$ & $3{,}75$ & $5{,}25$ \\
 \hline
 Số học sinh & $8$ & $12$ & $6$ & $4$ \\ 
 \hline
 \end{tabular}
\end{center}
 Thời gian sử dụng điện thoại trung bình của các bạn lớp 12A là
 \[\overline{x}=\dfrac{1}{30}(8 \cdot 0{,}75+12 \cdot 2{,}25+6 \cdot 3{,}75+4 \cdot 5{,}25)=2{,}55.\]
 Phương sai của mẫu số liệu trên là \[s^2=\dfrac{1}{30}\left(8 \cdot 0{,}75^2+12 \cdot 2{,}25^2+6 \cdot 3{,}75^2+4 \cdot 5{,}25^2\right)-2{,}55^2=2{,}16.\]
 }
\end{ex}
\Closesolutionfile{ans}
\begin{indapan}
	{ans/ans\currfilebase}
\end{indapan}


% \begin{name}
 {Biên soạn:Nguyễn Tài Tuệ \\ Phản biện: Bùi Văn Lợi}
 {Đề ôn tập chương III}
\end{name}

\TN
\Opensolutionfile{ans}[ans/ans\currfilebase-Phan-I]
\begin{ex}%[2-D3B3-SO-9-2425]%[VN-MT-7, Nguyễn Tài Tuệ]%[2D3N1-1]
\immini{Cho mẫu số liệu ghép nhóm được cho trong bảng bên. 
Gọi $ Q_1$, $Q_2$, $Q_3$ lần lượt là tứ phân vị thứ nhất, tứ phân vị thứ hai và tứ phân vị thứ ba của mẫu số liệu. Khoảng tứ phân vị của mẫu số liệu trên là
\choice
{\True $\Delta_Q=Q_3-Q_1$}
{$\Delta_Q=Q_3-Q_2$}
{$\Delta_Q=Q_2-Q_1$}
{$\Delta_Q=Q_3-\dfrac{3}{2}{Q_1}$}}{
\begin{tabular}{|c|c|}
 \hline Nhóm & Tần số \\
 \hline $[a_1; a_2)$ & $n_1$ \\
 \hline$[a_2; a_3)$ & $n_2$ \\
 \hline$\ldots$ & $\ldots$ \\
 \hline$[a_m; a_{m+1})$ & $n_m$\\
 \hline & $n$ \\
 \hline
\end{tabular}}
\loigiai{
Theo định nghĩa $\Delta_Q=Q_3-Q_1$.}
\end{ex}

\begin{ex}%[2-D3B3-SO-9-2425]%[VN-MT-7, Nguyễn Tài Tuệ]%[2D3N1-1]
\immini{Cho mẫu số liệu ghép nhóm được cho trong bảng bên. 
Khoảng biến thiên của mẫu số liệu trên là
\choice
{$R=a_m-a_1$}
{$R=a_{m+1}-a_m$}
{$R=a_{m+1}-a_2$}
{\True $R=a_{m+1}-a_1$}}{

\begin{tabular}{|c|c|}
 \hline Nhóm & Tần số \\
 \hline $[a_1 ; a_2)$ & $n_1$ \\
 \hline $[a_2 ; a_3)$ & $n_2$ \\
 \hline $\ldots$ & $\ldots$ \\
 \hline $[a_m ; a_{m+1})$ & $n_m$\\
 \hline & $n$ \\
 \hline
\end{tabular}}
\loigiai{
Ta có $R=a_{m+1}-a_1$.
}
\end{ex}

\begin{ex}%[2-D3B3-SO-9-2425]%[VN-MT-7, Nguyễn Tài Tuệ]%[2D3N1-2]
\immini{ Bảng thống kê chiều cao của $40$ mẫu cây ở một vườn thực vật (đơn vị: centimét) được cho trong bảng như hình bên. 
Khoảng biến thiên của mẫu số liệu trên bằng
\choice
{\True $R=60$}
{$R=50$}
{$R=70$}
{$R=10$}}{\begin{tabular}{|c|c|c|}
 \hline Nhóm & Tần số & Tần số tích lūy \\
 \hline$[30 ; 40)$ & $4$ & $4$ \\
 $[40 ; 50)$ & $10$ & $14$ \\
 $[50 ; 60)$ & $14$ & $28$ \\
 $[60 ; 70)$ & $6$ & $34$ \\
 $[70 ; 80)$ & $4$ & $38$ \\
 $[80 ; 90)$ & $2$ & $40$ \\
 \hline & $n=40$ & \\
 \hline
\end{tabular}}
\loigiai{
Khoảng biến thiên của mẫu số liệu là $R=90-30=60$.}
\end{ex}

\begin{ex}%[2-D3B3-SO-9-2425]%[VN-MT-7, Nguyễn Tài Tuệ]%[1D5H2-3]
Thời gian (phút) truy bài trước mỗi buổi học của một số học sinh trong một tuần được ghi lại ở bảng sau:
 \begin{center}
\begin{tabular}{|c|c|c|c|c|c|}
\hline
Thời gian & $[9{,}5 ; 12{,}5)$ & $[12{,}5 ; 15{,}5)$ & $[15{,}5 ; 18{,}5)$ & $[18{,}5 ; 21{,}5)$ & $[21{,}5 ; 24{,}5)$ \\
\hline
Số học sinh & $3$ & $12$ & $15$ & $24$ & $2$ \\
\hline
\end{tabular}
\end{center}
Nhóm chứa tứ phân vị thứ nhất là
\choice
{$[9{,}5;12{,}5)$}
{\True $[12{,}5;15{,}5)$}
{$[15{,}5;18{,}5)$}
{$[18{,}5;21{,}5)$}
\loigiai{
Cỡ mẫu $n=56$.\\
Gọi $x_1$, $x_2,\ldots,x_{56} $ là mẫu số liệu gốc về thời gian truy bài trước mỗi buổi học của $56$ số học sinh trong một tuần được xếp theo thứ tự không giảm.\\
Ta có $\dfrac{x_{14}+x_{15}}{2}\in [12{,}5;15{,}5)$, nên nhóm chứa tứ phân vị thứ nhất là $[12{,}5;15{,}5) $.}
\end{ex}

\begin{ex}%[2-D3B3-SO-9-2425]%[VN-MT-7, Nguyễn Tài Tuệ]%[1D5H2-3]
Khảo sát thời gian tập thể dục trong ngày của một số học sinh khối $11$ thu được mẫu số liệu ghép nhóm sau:
\begin{center}
\begin{tabular}{|c|c|c|c|c|c|}
\hline
Thời gian (phút) & $[0 ; 20)$ & $[20 ; 40)$ & $[40 ; 60)$ & $[60 ; 80)$ & $[80 ; 100)$ \\
\hline
Số học sinh & $5$ & $9$ & $12$ & $10$ & $6$ \\
\hline
\end{tabular}
\end{center}
Nhóm chứa tứ phân vị thứ ba là
\choice
{$[20 ;40)$}
{$[40 ;60)$}
{\True $[60 ;80)$}
{$[80 ;100)$}
\loigiai{
Cỡ mẫu $n=42$.\\
Gọi $x_1$, $x_2,\ldots, x_{42} $ là mẫu số liệu gốc về thời gian tập thể dục trong ngày của $42$ học sinh khối $11$ được xếp theo thứ tự không giảm.\\
Tứ phân vị thứ ba của mẫu số liệu gốc là $x_{32}$ thuộc nhóm $[60 ;80)$.}
\end{ex}

\begin{ex}%[2-D3B3-SO-9-2425]%[VN-MT-7, Nguyễn Tài Tuệ]%[1D5H2-3]
Cho mẫu số liệu ghép nhóm về thời gian (phút) đi từ nhà đến nơi làm việc của các nhân viên của một công ty như sau:
\begin{center}
\begin{tabular}{|c|c|c|c|c|c|c|c|}
\hline
Thời gian & $[15 ; 20)$ & $[20 ; 25)$ & $[25 ; 30)$ & $[30 ; 35)$ & $[35 ; 40)$ & $[40 ; 45)$ & $[45 ; 50)$ \\
\hline
Số nhân viên & $7 $& $14$ & $25$ & $37$ & $21$ & $14$ & $10$ \\
\hline
\end{tabular}
\end{center}
Tứ phân vị thứ nhất $Q_1$ và tứ phân vị thứ ba $Q_3$ của mẫu số liệu ghép nhóm này là
\choice
{\True $Q_1=\dfrac{136}{5}$, $Q_3=\dfrac{800}{21}$}
{$Q_1=\dfrac{1360}{37}$, $Q_3=\dfrac{800}{21}$}
{$Q_1=\dfrac{1360}{37}$, $Q_3=\dfrac{3280}{83}$}
{$Q_1=\dfrac{136}{5}$, $Q_3=\dfrac{3280}{83}$}
\loigiai{
Cỡ mẫu $n=128$.\\
Gọi $x_1$, $x_2,\ldots, x_{128} $ là mẫu số liệu gốc về thời gian đi từ nhà đến nơi làm việc của các nhân viên của một công ty được xếp theo thứ tự không giảm.\\
Tứ phân vị thứ nhất của mẫu số liệu gốc là $\dfrac{x_{32}+x_{33}}{2} \in [25;30)$.\\
Do đó, tứ phân vị thứ nhất của mẫu số liệu ghép nhóm là
\[Q_1= 25+ \dfrac{\dfrac{128}{4} - (7+14)}{25} \cdot (30-25)
=\dfrac{136}{5}.\]
Tứ phân vị thứ ba của mẫu số liệu gốc là $\dfrac{x_{96}+x_{97}}{2}\in[35 ;40)$.\\
Do đó, tứ phân vị thứ ba của mẫu số liệu ghép nhóm là
\[ Q_3=35+\dfrac{\dfrac{3\cdot 128}{4}-(7+14+25+37)}{21}\cdot (40-35)
=\dfrac{800}{21}.\]
}
\end{ex}

\begin{ex}%[2-D3B3-SO-9-2425]%[VN-MT-7, Nguyễn Tài Tuệ]%[1D5H2-3]
Thời gian (phút) truy cập Internet mỗi buổi tối của một số học sinh được cho trong bảng sau:
\begin{center}
\begin{tabular}{|l|c|c|c|c|c|}
\hline
Thời gian (phút) & $[9{,}5 ; 12{,}5)$ & $[12{,}5 ; 15{,}5)$ & $[15{,}5 ; 18{,}5)$ & $[18{,}5 ; 21{,}5)$ & $[21{,}5 ; 24{,}5)$ \\
\hline Số học sinh & $3$ & $12$ & $15$ & $24$ & $2$ \\
\hline
\end{tabular}
\end{center}
Tìm tứ phân vị thứ nhất $Q_1$.
\choice
{$Q_1=15$}
{$Q_1=15{,}5$}
{$Q_1=15{,}2$}
{\True $Q_1=15{,}25$}
\loigiai{
Cỡ mẫu là $n=56$.\\
Tứ phân vị thứ nhất $Q_1$ là $\dfrac{x_{14}+x_{15}}{2}$. Do $x_{14}$, $x_{15}$ đều thuộc nhóm $[12{,}5;15{,}5)$ nên nhóm này chứa $Q_1$ .\\
Do đó, $p=2$; ${a_2}=12{,}5$; ${m_2}=12$; ${m_1}=3$, $a_3-a_2=3$ và ta có
\[ Q_1=12{,}5+\dfrac{\dfrac{56}{4}-3}{12}\cdot 3=15{,}25.\]
}
\end{ex}

\begin{ex}%[2-D3B3-SO-9-2425]%[VN-MT-7, Nguyễn Tài Tuệ]%[1D5H1-3]
Thống kê cân nặng của học sinh lớp 11A cho trong bảng dưới đây:
\begin{center}
\begin{tabular}{|c|c|c|c|c|c|c|}
\hline
Cân nặng & $[40{,}5 ; 45{,}5)$ & $[45{,}5 ; 50{,}5)$ & $[50{,}5 ; 55{,}5)$ & $[55{,}5 ; 60{,}5)$ & $[60{,}5 ; 65{,}5)$ & $[65{,}5 ; 70{,}5)$\\
\hline
Số học sinh & $10$ & $7$ & $16$ & $4$ & $2$ & $3$ \\
\hline
\end{tabular}
\end{center}
Tính cân nặng trung bình của học sinh lớp 11A (kết quả làm tròn đến hàng phần trăm).
\choice
{$50{,}1$}
{$52{,}83$}
{$50{,}81$}
{\True $51{,}81$}
\loigiai{
Trong mỗi khoảng cân nặng, giá trị đại diện là trung bình cộng của giá trị hai đầu mút nên ta có bảng sau:
\begin{center}
\begin{tabular}{|c|c|c|c|c|c|c|}
\hline
Cân nặng (kg) & $43$ & $48$ & $53$ & $58$ & $63$ & $68$ \\
\hline
Số học sinh & $10$ & $7$ & $16$ & $4$ & $2$ & $3$ \\
\hline
\end{tabular}
\end{center}
Tổng số học sinh là $n=42 $.\\
Cân nặng trung bình của học sinh lớp 11A là \\
\[ \overline{x}=\dfrac{10\cdot 43+7\cdot 48+16\cdot 53+4\cdot 58+2\cdot 63+3\cdot 68}{42}\approx 51{,}81~\text{(kg).}\]
}
\end{ex}

\begin{ex}%[2-D3B3-SO-9-2425]%[VN-MT-7, Nguyễn Tài Tuệ]%[2D3H2-2]
\immini{Tìm phương sai của một mẫu số liệu ghép nhóm cho bởi bảng thống kê như hình bên.
\choice
{$ 13{,}24$}
{$15{,}74$}
{$18{,}84$}
{\True $14{,}84$}}{\begin{tabular}{|c|c|c|}
 \hline Lớp chiều cao & Giá trị đại diện & Tần số \\
 \hline$[150 ; 154)$ & $152$ & $25$ \\
 \hline$[154 ; 158)$ & $156$ & $50$ \\
 \hline$[158 ; 162)$ & $160$ & $200$ \\
 \hline$[162 ; 166)$ & $164$ & $175$ \\
 \hline$[166 ; 170)$ & $168$ & $50$ \\
 \hline
\end{tabular}}
\loigiai{
Ta có chiều cao trung bình \\
\[\overline x=\dfrac{1}{500}(152\cdot 25+156\cdot 50+160\cdot 200+164\cdot 175+168\cdot 50)=161{,}4.\] 
Phương sai của mẫu số liệu ghép nhóm là
\begin{align*}
 s^2=\dfrac{1}{500}& \left[25(152-161{,}4)^2+50(156-161{,}4)^2+200(160-161{,}4)^2+175(164-161{,}4)^2\right. \\
 &\left.+50(168-161{,}4)^2\right] = 14{,}84.
\end{align*}
}
\end{ex}

\begin{ex}%[2-D3B3-SO-9-2425]%[VN-MT-7, Nguyễn Tài Tuệ]%[2D3H2-2]
Kết quả khảo sát thời gian sử dụng liên tục (đơn vị: giờ) từ lúc sạc đầy cho đến khi hết pin của một số máy vi tính cùng loại được thống kê ở bảng sau:
\begin{center}
\begin{tabular}{|c|c|c|c|c|}
\hline
Thời gian sử dụng & $[7{,}2 ; 7{,}4)$ & $[7{,}4 ; 7{,}6)$ & $[7{,}6 ; 7{,}8)$& $[7{,}8 ; 8{,}0)$ \\
\hline
Số máy & $2$ & $4$ & $7$ & $6$ \\
\hline
\end{tabular}
\end{center}
Tính độ lệch chuẩn của mẫu số liệu ghép nhóm (kết quả làm tròn đến hàng phần nghìn).
\choice
{$0{,}192$}
{\True $0{,}194$}
{$0{,}037$}
{$0{,}2$}
\loigiai{
Từ bảng thống kê ta có
\begin{center}
\begin{tabular}{|c|c|c|c|c|}
\hline
Thời gian sử dụng & $[7{,}2 ; 7{,}4)$ & $[7{,}4 ; 7{,}6)$ & $[7{,}6 ; 7{,}8)$ & $[7{,}8 ; 8{,}0)$ \\
\hline
Giá trị đại diện & $7{,}3$ & $7{,}5$ & $7{,}7$ & $7{,}9$ \\
\hline
Số máy & $2$ & $4$ & $7$ & $6$ \\
\hline
\end{tabular}
\end{center}
\noindent
Tổng số máy $n=2+4+7+6=19$.\\
Thời gian sử dụng trung bình của pin là $\overline x=\dfrac{2\cdot 7{,}3+4\cdot 7{,}5+7\cdot 7{,}7+6\cdot 7{,}9}{19}=\dfrac{1459}{190}$.\\
Phương sai của mẫu số liệu là $s^2=\dfrac{1}{19}(2\cdot 7{,}3^2+4\cdot 7{,}5^2+7\cdot 7{,}7^2+6\cdot 7{,}9^2)-\left (\dfrac{1459}{190}\right )^2=\dfrac{338}{9025}$.\\
Độ lệch chuẩn của mẫu số liệu là $s=\sqrt{s^2}=\sqrt{\dfrac{ 338}{9025}} =\dfrac{13 \sqrt{2}}{95}\approx 0,194 $.}
\end{ex}

\begin{ex}%[2-D3B3-SO-9-2425]%[VN-MT-7, Nguyễn Tài Tuệ]%[2D3N1-1]
Đại lượng nào đo độ phân tán của nửa giữa của mẫu số liệu, không bị ảnh hưởng nhiều bởi các giá trị ngoại lệ trong mẫu số liệu?
\choice
{Khoảng biến thiên}
{\True Khoảng tứ phân vị}
{Phương sai}
{Độ lệch chuẩn}
\loigiai{
Khoảng tứ phân vị dùng để đo độ phân tán của nửa giữa của mẫu số liệu, không bị ảnh hưởng nhiều bởi các giá trị ngoại lệ trong mẫu số liệu.}
\end{ex}

\begin{ex}%[2-D3B3-SO-9-2425]%[VN-MT-7, Nguyễn Tài Tuệ]%[2D3N2-1]
Để so sánh mức độ phân tán của các mẫu số liệu ghép nhóm có cùng số trung bình ta dùng đại lượng nào?
\choice
{Khoảng biến thiên}
{Khoảng tứ phân vị}
{Trung vị}
{\True Độ lệch chuẩn}
\loigiai{
Để so sánh mức độ phân tán của các mẫu số liệu ghép nhóm có cùng số trung bình ta dùng phương sai và độ lệch chuẩn.
}
\end{ex}
\Closesolutionfile{ans}

\TNTF
\Opensolutionfile{ans}[ans/ans\currfilebase-Phan-II]
\begin{ex}%[2-D3B3-SO-9-2425]%[VN-MT-7, Nguyễn Tài Tuệ]%[2D3H1-2]
Mẫu số liệu dưới đây ghi lại tốc độ của $40$ ô tô khi đi qua một trạm đo tốc độ (đơn vị: km/h):
\begin{center}
\begin{tabular}{llllllllll}
$48{,}5$ & $43$ & $50$ & $55$ & $45$ & $60$ & $53$ & $55{,}5$ & $44$ & $65$ \\
$51$ & $62{,}5$ & $41$ & $44{,}5$ & $57$ & $57$ & $68$ & $49$ & $46{,}5$ & $53{,}5$ \\
$61$ & $49{,}5$ & $54$ & $62$ & $59$ & $56$ & $47$ & $50$ & $60$ & $61$ \\
$49{,}5$ & $52{,}5$ & $57$ & $47$ & $60$ & $55$ & $45$ & $47{,}5$ & $48$ & $61{,}5$
\end{tabular}
\end{center}
\choiceTF
{\True Bảng tần số ghép nhóm cho mẫu số liệu trên có sáu nhóm ứng với sáu nửa khoảng là: 
\centerline{
\begin{tabular}{|c|c|c|c|c|c|c|c|}
 \hline
 Nhóm & $[40 ; 45)$ & $[45 ; 50)$ & $[50 ; 55)$ & $[55 ; 60)$ & $[60 ; 65)$ & $[65 ; 70)$ & \\
 \hline
 Tần số & $4$ & $11$ & $7$ & $8$ & $8$ & $2$ & $n=40$ \\
 \hline
\end{tabular}
}}
{Mẫu số liệu trên có số trung bình là $54{,}875$}
{\True Tứ phân vị của mẫu số liệu trên là $Q_1=47{,}8~\text{(km/h)}$; ${Q_2}=53{,}6 ~\text{(km/h)}$; ${Q_3}=60~\text{(km/h)}$}
{Khoảng biến thiên của mẫu số liệu trên là $25$}
\loigiai{
\begin{itemchoice}
\itemch {\bf Đúng}.\\
Bảng tần số ghép nhóm: 
\begin{center}
\begin{tabular}{|c|c|c|c|c|c|c|c|}
 \hline
 Nhóm & $[40 ; 45)$ & $[45 ; 50)$ & $[50 ; 55)$ & $[55 ; 60)$ & $[60 ; 65)$ & $[65 ; 70)$ & \\
 \hline
 Tần số & $4$ & $11$ & $7$ & $8$ & $8$ & $2$ & $n=40$ \\
 \hline
\end{tabular}
\end{center}
Vậy bảng tần số đã cho đúng.
\itemch {\bf Sai}.\\
Số trung bình
\[
\overline x= \dfrac{4\cdot 42{,}5+11\cdot 47{,}5+7\cdot 52{,}5+8\cdot 57{,}5+8\cdot 62{,}5+2\cdot 67 \cdot 5}{40}= 53{,}875~\text{(km/h).}
\]
\itemch \textbf{Đúng}.\\
Cỡ mẫu là $n=40$.\\
Ta có $\dfrac{n}{2}=20$ nên nhóm $3$ là nhóm đầu tiên có tần số tích lũy lớn hơn hoặc bằng $20$.
Xét nhóm $3$ là nhóm $[50;55)$ có $r=50$, $d=5$, $n_3=7$ và $c{f_2}=15$.
Số trung vị của mẫu số liệu là
\[
M_e=50+\dfrac{20-15}{7}\cdot 5\approx 53{,}6 ~\text{(km/h)}.
\]
Ta có $\dfrac{n}{4}=10$ nên nhóm $2$ là nhóm đầu tiên có tần số tích lũy lớn hơn hoặc bằng $10$.\\
Xét nhóm $2$ là nhóm $[45;50)$ có $r=45$, $d=5$, $n_2=11$ và $cf_1=4$.\\
Tứ phân vị thứ nhất là 
\[
Q_1=45+\dfrac{10-4}{11}\cdot 5\approx 47{,}8~\text{(km/h).}\]
Ta có $\dfrac{3n}{4}=30$ nên nhóm $4$ là nhóm đầu tiên có tần số tích lũy lớn hơn hoặc bằng $30$.\\
Xét nhóm $4$ là nhóm $[(60)$ có $r=55$, $d=5$, $n_4=8$ và $c{f_3}=22$.\\
Tứ phân vị thứ ba là
\[ Q_3=55+\dfrac{30-22}{8}\cdot 5=60~\text{(km/h).} \]
Vậy các tứ phân vị của mẫu số liệu trên là
$Q_1=47{,}8$ (km/h); ${Q_2}=53{,}6$ (km/h); $ Q_3 =60$ (km/h).
\itemch {\bf Sai}.\\
Khoảng biến thiên của mẫu số liệu là
$R=70-40=30$.
\end{itemchoice}
}
\end{ex}

\begin{ex}%[2-D3B3-SO-9-2425]%[VN-MT-7, Nguyễn Tài Tuệ]%[2D3H2-2]
\immini{Bảng bên cho ta bảng tần số ghép nhóm số liệu thống kê cân nặng của $40$ học sinh lớp 12B trong một trường trung học phổ thông (đơn vị: kilôgam). 
Các mệnh đề sau \textbf{đúng} hay \textbf{sai}?
\choiceTF
{\True Số học sinh nặng dưới $50$ (kg) là $12$}
{\True Mốt của mẫu số liệu ghép nhóm trên xấp xỉ bằng $54{,}29$ (kg)}
{Khoảng tứ phân vị của mẫu số liệu ghép nhóm trên là $\dfrac{39}{2}$}
{Phương sai của mẫu số liệu ghép nhóm là $128$}}{
\begin{tabular}{|c|c|}
 \hline
 Nhóm & Số học sinh\\
 \hline $[30 ;40)$ & $2$\\
 \hline $[40 ;50)$ & $10$\\
 \hline $[50 ;60)$ & $16$\\
 \hline $[60 ;70)$ & $8$\\
 \hline $[70 ;80)$ & $2$\\
 \hline $[80 ;90)$ & $2$\\
 \hline & $n=40$\\
 \hline
\end{tabular}}
\loigiai{
\begin{itemchoice}
\itemch {\bf Đúng}.\\
Số học sinh nặng dưới $50$ kg là $2+10=12$. 
\itemch {\bf Đúng}.\\
Nhóm chứa mốt của mẫu số liệu là $[50 ;60)$.\\
Do đó $u_m=50$, $n_m=16$, $n_{m-1}=10$, $n_{m+1}=8$, $u_{m+1}-u_m=60-50=10$.\\
Mốt của mẫu số liệu ghép nhóm là 
\[
M_\text{o}=50+\dfrac{16-10}{(16-10)+(16-8)}\cdot 10=\dfrac{380}{7}\approx 54{,}29~\text{(kg).}
\]
Mốt của mẫu số liệu ghép nhóm trên xấp xỉ bằng $54{,}29$ (kg). 
\itemch \textbf{Sai}.\\
Cỡ mẫu $n=40$.\\
Gọi $x_1, x_2 \in [30 ;40)$; $x_3,\,\ldots,x_{12} \in [40 ;50)$; $x_{13},\,...,x_{28}\in [50 ;60);$ $x_{29},\,...,x_{36}\in [60 ;70);$ $x_{37},x_{38}\in [70 ;80);$ $x_{39},x_{40}\in [80 ;90) $.\\
Tứ phân vị thứ nhất của mẫu số liệu gốc là
$ \dfrac{1}{2}(x_{10}+x_{11})\in[40 ;50)$. 
Do đó, tứ phân vị thứ nhất của mẫu số liệu ghép nhóm là 
\[
Q_1=40+\dfrac{\dfrac{40}{4}-2}{10} \cdot (50-40)=48.
\]
Tứ phân vị thứ ba của mẫu số liệu gốc là
$\dfrac{1}{2}(x_{30}+x_{31})\in[60 ;70)$.
Do đó, tứ phân vị thứ ba của mẫu số liệu ghép nhóm là 
\[
Q_3=60+\dfrac{\dfrac{3\cdot 40}{4}-(2+10+16)}{8}\cdot (70-60)=\dfrac{125}{2}.
\]
Vậy khoảng tứ phân vị của mẫu số liệu ghép nhóm là 
\[
\Delta_Q=\dfrac{125}{2}-48=\dfrac{29}{2}.
\]
\itemch {\bf Sai}. \\ Ta có bảng cân nặng của các em học sinh theo giá trị đại diện:
\begin{center}
\begin{tabular}{|c|c|c|}
\hline
Nhóm & Giá trị đại diện & Tần số\\
\hline $[30 ;40)$ & $35$ & $2$\\
\hline $[40 ;50)$ & $45$ & $10$\\
\hline $[50 ;60)$ & $55$ & $16$\\
\hline $[60 ;70)$ & $65$ & $8$\\
\hline $[70 ;80)$ & $75$ & $2$\\
\hline $[80 ;90)$ & $85$ & $2$\\
\hline & & $n=40$\\
\hline
\end{tabular}
\end{center}
Cỡ mẫu $n=2+10+16+8+2+2=40 $.\\
Số trung bình của mẫu số liệu ghép nhóm là 
\[
\dfrac{35\cdot 2+45\cdot 10+55\cdot 16+65\cdot 8+75\cdot 2+85\cdot 2}{40}=\dfrac{2240}{40}=56 \text{ (kg).}
\]
Phương sai của mẫu số liệu ghép nhóm là 
\[
s^2=\dfrac{1}{40}\left(2\cdot 35^2+10\cdot 45^2+16\cdot 55^2+8\cdot 65^2+2\cdot 75^2+2\cdot 85^2\right)-56^2=3265-3136=129.
\]
\end{itemchoice}
}
\end{ex}

\begin{ex}%[2-D3B3-SO-9-2425]%[VN-MT-7, Nguyễn Tài Tuệ]%[2D3H2-2]
Trong một hội thao, thời gian chạy $200$\,m của một nhóm các vận động viên được ghi lại ở bảng sau:
\begin{center}
\begin{tabular}{|c|c|c|c|c|c|}
\hline
Thời gian (giây) & $[21 ; 21{,}5)$ & $[21{,}5 ; 22)$ & $[22 ; 22{,}5)$ & $[22{,}5 ; 23)$ & $[23 ; 23{,}5)$\\
\hline
Số vận động viên & $5$ & $10$ & $30$ & $45$ & $30$\\
\hline
\end{tabular}
\end{center}
\choiceTF 
{Tần suất của nhóm vận động viên chạy trong khoảng thời gian từ $22$ giây đến dưới $22{,}5$ giây bằng $30\% $}
{\True Số trung vị của mẫu số liệu (làm tròn đến chữ số thập phân thứ $2$) bằng $22{,}67$}
{Khoảng biến thiên của mẫu số liệu bằng $R=2$}
{Độ lệch chuẩn của mẫu số liệu (làm tròn đến chữ số thập phân thứ $2$) bằng $0{,}28$}
\loigiai{
\begin{itemchoice}
\itemch {\bf Sai}.\\ 
Cỡ mẫu $n=5+10+30+45+30=120$.\\
Tần suất của nhóm vận động viên chạy trong khoảng thời gian từ $22$ giây đến dưới $22{,}5$ giây bằng $f_3=\dfrac{n_3}{n}=\dfrac{30}{120}=25\% $.
\itemch {\bf Đúng}.\\
Gọi $x_1$, $x_2$, $ \ldots$, $x_{120}$ là thời gian chạy của $120$ vận động viên và dãy này là một dãy không giảm.\\
Khi đó trung vị là
$\dfrac{x_{60}+x_{61}}{2}$. Do $x_{60},\,x_{61}\in[22{,}5;\,23)$ nên nhóm này chứa trung vị.\\
Ta có 
$M_\text{e}=22{,}5+\dfrac{\dfrac{120}{2}-(5+10+30)}{45}\cdot(23-22{,}5)\approx 22{,}67$.
\itemch {\bf Sai}. \\
Khoảng biến thiên của mẫu số liệu bằng $R=23{,}5-21=2{,}5$.
\itemch {\bf Sai}. \\
Giá trị trung bình của mẫu số liệu là
\[
\overline x=\dfrac{5\cdot 21{,}25+10\cdot 21{,}75+30\cdot 22{,}25+45\cdot 22{,}75+30\cdot 23{,}25}{120}\approx 22{,}60.
\]
Độ lệch chuẩn của mẫu số liệu (làm tròn đến chữ số thập phân thứ $2$) bằng
\[
s=\sqrt{\dfrac{5(-1{,}35)^2+10(-0{,}85)^2+30(-0{,}35)^2+45(0{,}15)^2+30(0{,}65)^2}{120}}\approx 0{,}53.
\]
\end{itemchoice}
}
\end{ex}

\begin{ex}%[2-D3B3-SO-9-2425]%[VN-MT-7, Nguyễn Tài Tuệ]%[2D3V1-3]
Giả sử kết quả khảo sát hai khu vực A và B về độ tuổi kết hôn của một số phụ nữ vừa lập gia đình được cho ở bảng sau:
\begin{center}
\begin{tabular}{|c|c|c|c|c|c|}
\hline
Tuổi kết hôn & $[19;22)$ & $[22;25)$ & $[25;28)$ & $[28;31)$ & $[31;34)$\\
\hline
Số phụ nữ khu vực A & $10$ & $27$ & $31$ & $25$ & $7$\\
\hline
Số phụ nữ khu vực B & $47$ & $40$ & $11$ & $2$ & $0$\\
\hline
\end{tabular}
\end{center}
\choiceTF
{Khoảng biến thiên của mẫu số liệu ghép nhóm ứng với khu vực A là $15$ (tuổi)}
{Khoảng biến thiên của mẫu số liệu ghép nhóm ứng với khu vực B là $12$ (tuổi)}
{Khoảng tứ phân vị của mẫu số liệu ghép nhóm ứng với khu vực A là $\dfrac{61}{3}$ (tuổi)}
{Nếu so sánh theo khoảng tứ phân vị thì phụ nữ ở khu vực B có độ tuổi kết hôn đồng đều hơn}
\loigiai{
\begin{itemchoice}
\itemch \textbf{Đúng}.\\ 
Khoảng biến thiên của mẫu số liệu ghép nhóm ứng với khu vực A là $34-19=15$ (tuổi). 
\itemch \textbf{Đúng}. \\
Khoảng biến thiên của mẫu số liệu ghép nhóm ứng với khu vực B là $31-19=12$ (tuổi)
\itemch \textbf{Sai}. \\ 
Cỡ mẫu $n=100$.\\
Gọi $x_1$, $x_2$, $\ldots$, $x_{100} $ là mẫu số liệu gốc về độ tuổi kết hôn của phụ nữ ở khu vực A được xếp theo thứ tự không giảm.\\
Ta có $x_1$, $x_2$, $\ldots$, $x_{10}\in[19;22)$; ${x_{11}}$, $\ldots$, $ x_{37} \in[22;25)$; $x_{38},\ldots, x_{68} \in[25;28)$; $x_{69},\ldots,x_{93} \in[28;31)$; $x_{94},\ldots, x_{100}\in[31;34)$.\\
Tứ phân vị thứ nhất của mẫu số liệu gốc là
$\dfrac{1}{2}(x_{25}+x_{26})\in[22;25)$. Do đó, tứ phân vị thứ nhất của mẫu số liệu ghép nhóm là $Q_1=22+\dfrac{\dfrac{100}{4}-10}{27}(25-22)=\dfrac{71}{3}$.\\
Tứ phân vị thứ ba của mẫu số liệu gốc là
$\dfrac{1}{2}(x_{75}+x_{76})\in[28;31)$. Do đó, tứ phân vị thứ ba của mẫu số liệu ghép nhóm là $Q_3=28+\dfrac{\dfrac{3\cdot 100}{4}-(10+27+31)}{25}(31-28)=\dfrac{721}{25}$.\\
Khoảng tứ phân vị của mẫu số liệu ghép nhóm là $\Delta_Q=Q_3-Q_1=\dfrac{388}{75}$.
\itemch \textbf{Đúng}.\\
Gọi $y_1$, $y_2,\ldots,y_{100} $ là mẫu số liệu gốc về độ tuổi kết hôn của phụ nữ ở khu vực B được xếp theo thứ tự không giảm.\\
Ta có $y_1$, $y_2,\ldots,y_{47} \in [19;22)$; $ y_{48},\ldots, y_{87} \in[22;25)$; $y_{88}, \ldots, y_{98} \in[25;30)$;$y_{99}$, ${y_{100}}\in[28;31)$.\\
Tứ phân vị thứ nhất của mẫu số liệu gốc là $\dfrac{1}{2}(y_{25}+y_{26})\in[19;22)$.
Do đó, tứ phân vị thứ nhất của mẫu số liệu ghép nhóm là
\[
Q_1^\prime=19+\dfrac{\dfrac{100}{4}}{47}(22-19)=\dfrac{968}{47}
\]
Tứ phân vị thứ ba của mẫu số liệu gốc là $\dfrac{1}{2}(y_{75}+y_{76})\in[22;25)$. Do đó, tứ phân vị thứ ba của mẫu số liệu ghép nhóm là
\[
Q_3^\prime=22+\dfrac{\dfrac{3\cdot 100}{4}-47}{40}(25-22)=\dfrac{241}{10}
\]
Có $\Delta_Q^\prime < \Delta_Q$ nên phụ nữ ở khu vực B có độ tuổi kết hôn đồng đều hơn.
\end{itemchoice}
}
\end{ex}
\Closesolutionfile{ans}

\TNSA
\Opensolutionfile{ans}[ans/ans\currfilebase-Phan-III]
\begin{ex}%[2-D3B3-SO-9-2425]%[VN-MT-7, Nguyễn Tài Tuệ]%[2D3N1-2]
Thời gian tập luyện trong một ngày (tính theo giờ) của một số vận động viên được ghi lại ở bảng sau:
\begin{center}
\begin{tabular}{|c|c|c|c|c|c|}
\hline
Thời gian tập luyện &$[0; 2)$ &$[2; 4)$ &$[4; 6)$ &$[6; 8)$ &$[8; 10)$\\
\hline
Số vận động viên & $3$ & $8$ & $12$ & $12$ & $4$\\
\hline
\end{tabular}
\end{center}
Hãy tìm khoảng biến thiên cho thời gian tập luyện của các vận động viên.
\shortans[]{10}
\loigiai{
Gọi $R$ là khoảng biến thiên của mẫu số liệu ghép nhóm về thời gian tập luyện trong ngày của các vận động viên. Ta có $R=10-0=10$.}
\end{ex}

\begin{ex}%[2-D3B3-SO-9-2425]%[VN-MT-7, Nguyễn Tài Tuệ]%[2D3H1-3]
Một trang báo điện tử thống kê thời gian người sử dụng đọc thông tin trên trang trong mỗi lần truy cập ở bảng sau:\\
\begin{center}
\begin{tabular}{|c|c|c|c|c|c|}
\hline
Thời gian đọc (phút) & $[0; 2)$ &$[2; 4)$ &$[4; 6)$ &$[6; 8)$ &$[8; 10)$\\
\hline
Số lượt truy cập & $45$ & $34$ & $23$ & $18$ & $5$\\
\hline
\end{tabular}
\end{center}
Hãy tìm khoảng tứ phân vị của mẫu số liệu ghép nhóm trên.
\shortans[]{3{,}89}
\loigiai{
Cỡ mẫu là $n=45+34+23+18+5=125$.\\
Gọi $x_1$, $x_2$,\ldots, $x_{125}$ là thời gian đọc thông tin trên trang báo điện tử của $125$ lượt truy cập và giả sử rằng dãy số liệu gốc này đã được sắp xếp theo thứ tự tăng dần.\\
Tứ phân vị thứ nhất của mẫu số liệu gốc là $\dfrac{1}{2}(x_{31}+x_{32})$ nên nhóm chứa tứ phân vị thứ nhất là nhóm $[0; 2)$.\\
Tứ phân vị thứ nhất của mẫu số liệu ghép nhóm là
\[
Q_1 = 0+\dfrac{\dfrac{1\cdot 125}{4}-0}{45}\cdot (2-0)\approx 1{,}39.
\]
Tứ phân vị thứ ba của mẫu số liệu gốc là $\dfrac{1}{2}(x_{94}+x_{95})$ nên nhóm chứa tứ phân vị thứ nhất là nhóm $[4; 6)$.\\
Tứ phân vị thứ ba của mẫu số liệu ghép nhóm là
\[ Q_3=4+\dfrac{\dfrac{3\cdot 125}{4}-(45+34)}{23}\cdot (6-4) \approx 5{,}28.
\]
Vậy khoảng tứ phân vị của mẫu số liệu ghép nhóm là
\[
\Delta_Q=Q_3-Q_1\approx 5{,}28-1{,}39=3{,}89.
\]
}
\end{ex}

\begin{ex}%[2-D3B3-SO-9-2425]%[VN-MT-7, Nguyễn Tài Tuệ]%[2D3H2-2]
Người ta ghi lại tiền lãi (đơn vị: triệu đồng) của một số nhà đầu tư (với số tiền đầu tư như nhau), khi đầu tư vào hai lĩnh vực A, B cho kết quả như sau:
\begin{center}
\begin{tabular}{|c|c|c|c|c|c|}
\hline
Tiền lãi & $[5;10)$ & $[10;15)$ & $[15;20)$ & $[20;25)$ & $[25;30)$\\
\hline
Số nhà đầu tư vào lĩnh vực A & $2$ & $5$ & $8$ & $6$ & $4$\\
\hline
Số nhà đầu tư vào lĩnh vực B & $8$ & $4$ & $2$ & $5$ & $6$\\
\hline
\end{tabular}
\end{center} 
Tính hiệu phương sai $s_B^2-s_A^2$ cho các mẫu số liệu về tiền lãi của các nhà đầu tư ở hai lĩnh vực này. \par
\shortans[]{47{,}7}
\loigiai{
Ta có mẫu số liệu ghép nhóm với giá trị đại diện là:
\begin{center}
\begin{tabular}{|c|c|c|c|c|c|}
\hline
Tiền lãi & $[5;10)$ & $[10;15)$ & $[15;20)$ & $[20;25)$ & $[25;30)$\\
\hline
Giá trị đại diện & $7{,}5$ & $12{,}5$ & $17{,}5$ & $22{,}5$ & $27{,}5$\\
\hline
Số nhà đầu tư vào lĩnh vực A & $2$ & $5$ & $8$ & $6$ & $4$\\
\hline
Số nhà đầu tư vào lĩnh vực B & $8$ & $4$ & $2$ & $5$ & $6$\\
\hline
\end{tabular}
\end{center}
Tiền lãi trung bình khi đầu tư vào lĩnh vực A là
\[
\overline{x}_A=\dfrac{7{,}5\cdot 2+12{,}5\cdot 5+17{,}5\cdot 8+22{,}5\cdot 6+27{,}5\cdot 4}{2+5+8+6+4}
=18{,}5.
\]
Tiền lãi trung bình khi đầu tư vào lĩnh vực B là
\[
\overline{x}_B=\dfrac{7{,}5\cdot 8+12{,}5\cdot 4+17{,}5\cdot 2+22{,}5\cdot 5+27{,}5\cdot 6}{8+4+2+5+6}
=16{,}9.
\]
Phương sai của mẫu số liệu về tiền lãi khi đầu tư vào lĩnh vực A là
\[
s_A^2=\dfrac{1}{25}\left(7{,}5^2\cdot 2+12{,}5^2\cdot 5+17{,}5^2\cdot 8+22{,}5^2\cdot 6+27{,}5^2\cdot 4\right)-18{,}5^2
=34.
\]
Phương sai của mẫu số liệu về tiền lãi khi đầu tư vào lĩnh vực B là \\
\[
s_B^2=\dfrac{1}{25}\left(7{,}5^2\cdot 8+12{,}5^2\cdot 4+17{,}5^2\cdot 2+22{,}5^2\cdot 5+27{,}5^2\cdot 6\right)-16{,}9^2
=64{,}64.
\]
Do đó $s_B^2-s_A^2 = 47{,}7$.
}
\end{ex}

\begin{ex}%[2-D3B3-SO-9-2425]%[VN-MT-7, Nguyễn Tài Tuệ]%[2D3H2-2]
Thời gian hoàn thành một bài kiểm tra trắc nghiệm của một số học sinh lớp $10$ của hai lớp 10A và 10B được ghi lại ở bảng sau:
\begin{center}
\begin{tabular}{|c|c|c|c|c|c|}
\hline
Thời gian (phút) & $[6;7)$ & $[7;8)$ & $[8;9)$ & $[9;10)$ & $[10;11)$\\
\hline
Học sinh lớp $10A$ & $8$ & $10$ & $13$ & $10$ & $9$\\
\hline
Học sinh lớp $10B$ & $4$ & $12$ & $17$ & $14$ & $3$\\
\hline
\end{tabular}
\end{center} 
Tính hiệu độ lệch chuẩn $s_{10A}-s_{10B}$ (kết quả làm tròn đến hàng phần trăm).
\shortans[]{0{,}29}
\loigiai{
Lập lại mẫu số liệu ghép nhóm theo giá trị đại diện, ta được:
\begin{center}
\begin{tabular}{|c|c|c|c|c|c|}
\hline
Giá trị đại diện & $6{,}5$ & $7{,}5$ & $8{,}5$ & $9{,}5$ & $10{,}5$\\
\hline
Học sinh lớp $10A$ & $8$ & $10$ & $13$ & $10$ & $9$\\
\hline
Học sinh lớp $10B$ & $4$ & $12$ & $17$ & $14$ & $3$\\
\hline
\end{tabular}
\end{center}
Cỡ mẫu $n=50$.
\begin{enumerate}
\item Xét số liệu của lớp $10A$.\\
Số trung bình là
$\overline x_{10A}=\dfrac{8\cdot 6{,}5+10\cdot 7{,}5+13\cdot 8{,}5+10\cdot 9{,}5+9\cdot 10{,}5}{50}=8{,}54$.\\
Độ lệch chuẩn là
$s_{10A}=\sqrt{\dfrac{8\cdot 6{,}5^2+10\cdot 7{,}5^2+13\cdot 8{,}5^2+10\cdot 9{,}5^2+9\cdot 10{,}5^2}{50}-8{,}54^2}\approx 1{,}33$.
\item Xét số liệu của lớp $10B$.\\
Số trung bình là
$\overline x_{10B}=\dfrac{4\cdot 6{,}5+12\cdot 7{,}5+17\cdot 8{,}5+14\cdot 9{,}5+3\cdot 10{,}5}{50}=8{,}5$.\\
Độ lệch chuẩn là $s_{10B}=\sqrt{\dfrac{4\cdot 6{,}5^2+12\cdot 7{,}5^2+17\cdot 8{,}5^2+14\cdot 9{,}5^2+3\cdot 10{,}5^2}{50}-8{,}5^2}\approx 1{,}04$.
\end{enumerate}
Do đó $s_{10A}-s_{10B}\approx 0{,}29$.
}
\end{ex}

\begin{ex}%[2-D3B3-SO-9-2425]%[VN-MT-7, Nguyễn Tài Tuệ]%[2D3H2-2]
Giá đóng cửa của một cổ phiếu là giá của cổ phiếu đó cuối một phiên giao dịch. Bảng sau thống kê giá đóng cửa (đơn vị: nghìn đồng) của hai mã cổ phiếu A và B trong $50$ ngày giao dịch liên tiếp:
\begin{center}
\begin{tabular}{|c|c|c|c|c|c|}
\hline
Giá đóng cửa &$[120;122)$ &$[122;124)$ &$[124;126)$ &$[126;128)$ &$[128;130)$\\
\hline
Số ngày giao dịch của cổ phiếu A & $8$ & $9$ & $12$ & $10$ & $11$\\
\hline
Số ngày giao dịch của cổ phiếu B & $16$ & $4$ & $3$ & $6$ & $21$\\
\hline
\end{tabular} 
\end{center} 
Tính tỉ số $\dfrac{s_B^2}{s_A^2}$ (kết quả làm tròn đến hàng phần trăm).
\shortans[]{1{,}65}
\loigiai{
Ta có bảng thống kê giá đóng cửa theo giá trị đại diện như sau:
\begin{center}
\begin{tabular}{|c|c|c|c|c|c|}
\hline
Giá trị đại diện & $121$ & $123$ & $125$ & $127$ & $129$\\
\hline
Số ngày giao dịch của cổ phiếu A & $8$ & $9$ & $12$ & $10$ & $11$\\
\hline
Số ngày giao dịch của cổ phiếu B & $16$ & $4$ & $3$ & $6$ & $21$\\
\hline
\end{tabular}
\end{center}
\begin{enumerate}
\item Xét mẫu số liệu của cổ phiếu A
\begin{itemize}
\item Số trung bình của mẫu số liệu ghép nhóm là
\[
\overline{x}_A=\dfrac{8\cdot 121+9\cdot 123+12\cdot 125+10\cdot 127+11\cdot 129}{50}=125{,}28.
\]
\item Phương sai của mẫu số liệu ghép nhóm là
\[
s_A^2=\dfrac{1}{50}\left(8\cdot 121^2+9\cdot 123^2+12\cdot 125^2+10\cdot 127^2+11\cdot 129^2)-(125{,}28\right)^2
=7{,}5216.
\]
\end{itemize}
\item Xét mẫu số liệu của cổ phiếu B
\begin{itemize}
\item Số trung bình của mẫu số liệu ghép nhóm là
\[
\overline{x}_B=\dfrac{16\cdot 121+4\cdot 123+3\cdot 125+6\cdot 127+21\cdot 129}{50}
=125{,}48.
\]
\item Phương sai của mẫu số liệu ghép nhóm là\\
\[
s_B^2=\dfrac{1}{50}\left(16\cdot 121^2+4\cdot 123^2+3\cdot 125^2+6\cdot 127^2+21\cdot 129^2\right)-(125{,}48)^2
=12{,}4096.
\]
\end{itemize}
\end{enumerate}
Do đó $\dfrac{s_B^2}{s_A^2}\approx 1{,}65$.
}
\end{ex}

\begin{ex}%[2-D3B3-SO-9-2425]%[VN-MT-7, Nguyễn Tài Tuệ]%[2D3H2-2]
Thầy Niên thống kê lại điểm trung bình cuối năm của các học sinh lớp 10A và 10B ở bảng sau:
\begin{center}
\begin{tabular}{|c|c|c|c|c|c|}
\hline
Điểm trung bình &$[5;6)$ &$[6;7)$ &$[7;8)$ &$[8;9)$ &$[9;10)$\\
\hline
Số học sinh lớp 10A & $1$ & $0$ & $11$ & $22$ & $6$\\
\hline
Số học sinh lớp 10B & $0$ & $6$ & $8$ & $14$ & $12$\\
\hline
\end{tabular}
\end{center}
Tính $s_A-s_B$ (kết quả làm tròn đến hàng phần chục).
\shortans[]{-0{,}2}
\loigiai{
Ta có bảng thống kê điểm trung bình theo giá trị đại diện
\begin{center}
\begin{tabular}{|c|c|c|c|c|c|}
\hline
Giá trị đại diện & $5{,}5$ & $6{,}5$ & $7{,}5$ & $8{,}5$ & $9{,}5$\\
\hline
Số học sinh lớp 10A & $1$ & $0$ & $11$ & $22$ & $6$\\
\hline
Số học sinh lớp 10B & $0$ & $6$ & $8$ & $14$ & $12$\\
\hline
\end{tabular}
\end{center}
\begin{enumerate}
\item Xét mẫu số liệu của lớp 10A
\begin{itemize}
\item Số trung bình của mẫu số liệu ghép nhóm là
\[
\overline{x}_A=\dfrac{1\cdot 5{,}5+0.6{,}5+11\cdot 7{,}5+22\cdot 8{,}5+6\cdot 9{,}5}{40}=8{,}3.
\]
\item Phương sai của mẫu số liệu ghép nhóm là 
\[
s_A^2=\dfrac{1}{40}\left(1\cdot 5{,}5^2+0\cdot 6{,}5^2+11\cdot 7{,}5^2+22\cdot 8{,}5^2+6\cdot 9{,}5^2\right)-(8{,}3)^2
=0{,}61.
\]
\item Độ lệch chuẩn của mẫu số liệu ghép nhóm là $s_A=\sqrt{0{,}61}$.
\end{itemize}
\item Xét mẫu số liệu của lớp 10B
\begin{itemize}
\item Số trung bình của mẫu số liệu ghép nhóm là\\
\[
\overline{x}_B=\dfrac{0\cdot 5{,}5+6\cdot 6{,}5+8\cdot 7{,}5+14\cdot 8{,}5+12\cdot 9{,}5}{40}
=8{,}3.
\]
\item Phương sai của mẫu số liệu ghép nhóm là\\
\[s_B^2=\dfrac{1}{40}\left(0\cdot 5{,}5^2+6\cdot 6{,}5^2+8\cdot 7{,}5^2+14\cdot 8{,}5^2+12\cdot 9{,}5^2\right)-(8{,}3)^2
=1{,}06.\]
\item Độ lệch chuẩn của mẫu số liệu ghép nhóm là $s_B=\sqrt{1{,}06}$.
\end{itemize}
\end{enumerate}
Do đó $s_A-s_B \approx -0{,}2$.
}
\end{ex}
\Closesolutionfile{ans}
\begin{indapan}
	{ans/ans\currfilebase}
\end{indapan}


% \begin{name}
 {Biên soạn: Bùi Văn Lợi \\ Phản biện: HP Minh Nguyen}
 {Đề ôn tập chương III}
\end{name}

\TN
\Opensolutionfile{ans}[ans/ans\currfilebase-Phan-I]

\begin{ex}%[2-D3B3-SO-10-2425]%[VN-MT-7, Bùi Văn Lợi]%[2D3N1-2]
Xét mẫu dữ liệu cho bởi bảng sau:
\begin{center}
\begin{tabular}{|c|c|c|c|c|c|c|}
\hline 
Nhóm & $[14;15)$ & $[15;16)$ & $[16;17)$ & $[17;18)$ & $[18;19)$ & \\ 
\hline 
Tần số & $1$ & $3$ & $8$ & $6$ & $2$ & $n=20$ \\ 
\hline 
\end{tabular} 
\end{center}
Khoảng biến thiên của mẫu số liệu ghép nhóm bằng
\choice
{$3$}
{$4$}
{\True $5$}
{$6$}

\loigiai{
Khoảng biến thiên của mẫu số liệu ghép nhóm là $R=19-14=5$.
}
\end{ex}

\begin{ex}%[2-D3B3-SO-10-2425]%[VN-MT-7, Bùi Văn Lợi]%[2D3N1-1]
Xét mẫu số liệu cho bởi bảng sau:
\begin{center}
\begin{tabular}{|c|c|c|c|c|c|c|}
\hline
Nhóm & $[40;45)$ & $[45;50)$ & $[50;55)$ & $[55;60)$ & $[60;65)$ & \\
\hline
Tần số & $4$ & $11$ & $9$ & $n_4$ & $8$ & $n=40$ \\
\hline
\end{tabular}
\end{center}
Tần số $n_4$ của nhóm $4$ trong mẫu số liệu trên bằng
\choice
{$7$}
{\True $8$}
{$9$}
{$10$}

\loigiai{
Ta có $n_4 = 40-(4+11+9+8)=8$.
}
\end{ex}

\begin{ex}%[2-D3B3-SO-10-2425]%[VN-MT-7, Bùi Văn Lợi]%[2D3H1-4]
Xét mẫu số liệu cho bởi bảng sau:
\begin{center}
\begin{tabular}{|c|c|c|c|c|c|c|c|}
\hline
Nhóm & $[0;2)$ & $[2;4)$ & $[4;6)$ & $[6;8)$ & $[8;10)$ & $[10;12)$ &\\
\hline
Tần số & $3$ & $8$ & $12$ & $10$ & $7$ & $5$ & $n=45$\\
\hline
Tần số tích lũy & $3$ & $11$ & $23$ & $33$ & $40$ & $n_6$ &\\
\hline
\end{tabular}
\end{center}
Tần số tích lũy của nhóm $6$ bằng
\choice
{$40$}
{$42$}
{\True $45$}
{$41$}

\loigiai{
Tần số tích lũy của nhóm $6$ bằng $n_1+n_2+n_3+n_4+n_5+n_6=3+8+12+10+7+5=45$.
}
\end{ex}

\begin{ex}%[2-D3B3-SO-10-2425]%[VN-MT-7, Bùi Văn Lợi]%[1D5H2-3]
Xét mẫu số liệu cho bởi bảng sau:
\begin{center}
\begin{tabular}{|c|c|c|c|c|c|c|c|}
\hline Nhóm & $[40;45)$ & $[45;50)$ & $[50;55)$ & $[55;60)$ & $[60;65)$ & $[65;70)$ & \\
\hline Tần số & $5$ & $10$ & $7$ & $9$ & $7$ & $4$ & $n=42$ \\
\hline Tần số tích lũy & $5$ & $15$ & $22$ & $31$ & $38$ & $42$ & \\
\hline
\end{tabular}
\end{center}
Tứ phân vị thứ nhất của mẫu số liệu trên bằng
\choice
{$47{,}5$}
{\True $47{,}75$}
{$48$}
{$48{,}25$}

\loigiai{
Cỡ mẫu là $n=42$.\\
Gọi $x_1,x_2,\cdots,x_{42}$ là mẫu số liệu gốc được xếp theo thứ tự không giảm.\\
Tứ phân vị thứ nhất của mẫu số liệu gốc là $x_{11} \in [45;50)$.\\
Xét nhóm $[45;50)$ có đầu mút trái $s=45$, độ dài $h=5$, tần số $n_2=10$ và $cf_1=5$.\\
Tứ phân vị thứ nhất là $Q_1 = 45 + \left(\dfrac{10{,}5-5}{10}\right) \cdot 5=47{,}75$.
}
\end{ex}

\begin{ex}%[2-D3B3-SO-10-2425]%[VN-MT-7, Bùi Văn Lợi]%[1D5H2-3]
Xét mẫu số liệu cho bởi bảng sau:
\begin{center}
\begin{tabular}{|c|c|c|c|c|c|c|c|}
\hline
Nhóm & $[0;2)$ & $[2;4)$ & $[4;6)$ & $[6;8)$ & $[8;10)$\\
\hline
Tần số & $3$ & $8$ & $12$ & $12$ & $4$\\
\hline
Tần số tích lũy & $3$ & $11$ & $23$ & $35$ & $39$\\
\hline
\end{tabular}
\end{center}
Tứ phân vị thứ hai của mẫu số liệu trên gần nhất với kết quả nào dưới đây?
\choice
{$5{,}52$}
{\True $5{,}42$}
{$4{,}5$}
{$4{,}75$}

\loigiai{
Cỡ mẫu là $n=39$.\\
Gọi $x_1,x_2,\cdots,x_{39}$ là mẫu số liệu gốc được xếp theo thứ tự không giảm.\\
Tứ phân vị thứ hai của mẫu số liệu gốc là $x_{20} \in [4;6)$.\\
Xét nhóm $[4;6)$ có đầu mút trái $r=11$, độ dài $d=2$, tần số $n_3=12$ và $cf_2=11$.\\
Tứ phân vị thứ hai là $Q_2 =4+\left(\dfrac{19{,}5-11}{12}\right) \cdot 2 = \dfrac{65}{12} \approx 5{,}42$.
}
\end{ex}

\begin{ex}%[2-D3B3-SO-10-2425]%[VN-MT-7, Bùi Văn Lợi]%[1D5H2-3]
Cho mẫu số liệu ghép nhóm về tuổi thọ (đơn vị tính là năm) của một loại bóng đèn mới như sau:
\begin{center}
\begin{tabular}{|c|c|c|c|c|c|c|c|}
\hline
Tuổi thọ & $[2;3{,}5)$ & $[3{,}5;5)$ & $[5;6{,}5)$ & $[6{,}5;8)$\\
\hline
Số bóng đèn & $8$ & $22$ & $35$ & $15$\\
\hline
\end{tabular}
\end{center}
Nhóm chứa tứ phân vị thứ ba của mẫu số liệu là
\choice
{$[2;3{,}5)$}
{$[3{,}5;5)$}
{\True $[5;6{,}5)$}
{$[6{,}5;8)$}

\loigiai{
Lập bảng tần số tích lũy
\begin{center}
\begin{tabular}{|c|c|c|c|c|c|c|c|}
\hline
Nhóm & $[2;3{,}5)$ & $[3{,}5;5)$ & $[5;6{,}5)$ & $[6{,}5;8)$\\
\hline
Tần số & $8$ & $22$ & $35$ & $15$\\
\hline
Tần số tích lũy & $8$ & $30$ & $65$ & $80$\\
\hline
\end{tabular}
\end{center}
Cỡ mẫu là $n=8+22+35+15=80$.\\
Gọi $x_1,x_2,\cdots,x_{80}$ là mẫu số liệu gốc về tuổi thọ của $80$ bóng đèn được xếp theo thứ tự không giảm.\\
Tứ phân vị thứ ba của mẫu số liệu gốc là $\dfrac{x_{60}+x_{61}}{2} \in [5;6{,}5)$.\\
Khi đó nhóm chứa tứ phân vị thứ ba là $[5;6{,}5)$.
}
\end{ex}

\begin{ex}%[2-D3B3-SO-10-2425]%[VN-MT-7, Bùi Văn Lợi]%[2D3H1-3]
Bạn Thu rất thích nhảy hiện đại. Thời gian tập nhảy mỗi ngày trong thời gian gần đây của bạn Thu được thống kê lại ở bảng sau:
\begin{center}
\begin{tabular}{|c|c|c|c|c|c|c|c|}
\hline
Thời gian (phút) & $[20;25)$ & $[25;30)$ & $[30;35)$ & $[35;40)$ & $[40;45)$\\
\hline
Số ngày & $6$ & $6$ & $4$ & $1$ & $1$\\
\hline
\end{tabular}
\end{center}
Khoảng tứ phân vị của mẫu số liệu ghép nhóm là
\choice
{\True $8{,}125$}
{$8{,}5$}
{$13{,}5$}
{$4{,}5$}

\loigiai{
Ta lập bảng tần số tích lũy
\begin{center}
\begin{tabular}{|c|c|c|c|c|c|c|c|}
\hline Nhóm & $[20;25)$ & $[25;30)$ & $[30;35)$ & $[35;40)$ & $[40;45)$\\
\hline Tần số & $6$ & $6$ & $4$ & $1$ & $1$\\
\hline Tần số tích lũy & $6$ & $12$ & $16$ & $17$ & $18$\\
\hline
\end{tabular}
\end{center}
Cỡ mẫu là $n=18$.\\
Gọi $x_1,x_2,\cdots,x_{18}$ là mẫu số liệu gốc được xếp theo thứ tự không giảm.\\
Tứ phân vị thứ nhất của mẫu số liệu gốc là $x_5 \in [20;25)$. Khi đó\\
\[Q_1 = 20+\dfrac{4{,}5}{6} \cdot 5 = 23{,}75.\]
Tứ phân vị thứ nhất của mẫu số liệu gốc là $x_{14} \in [30;35)$. Khi đó\\
\[Q_3 = 30+\dfrac{13{,}5-12}{4} \cdot 5 = 31{,}875.\]
Vậy khoảng tứ phân vị của mẫu số liệu ghép nhóm là $\Delta_Q = Q_3-Q_1 = 8{,}125$.
}
\end{ex}

\begin{ex}%[2-D3B3-SO-10-2425]%[VN-MT-7, Bùi Văn Lợi]%[1D5N1-2]
Điều tra về chiều cao của học sinh khối 12 của một trường THPT, ta có kết quả sau:
\begin{center}
\begin{tabular}{|c|c|c|}
\hline
Nhóm & Chiều cao (cm) & Số học sinh\\
\hline
$1$ & $[160;163)$ & $6$\\
\hline
$2$ & $[163;166)$ & $10$\\
\hline
$3$ & $[166;169)$ & $12$\\
\hline
$4$ & $[169;172)$ & $20$\\
\hline
$5$ & $[172;175)$ & $40$\\
\hline
$6$ & $[175;178)$ & $12$\\
\hline
& & $n=100$\\
\hline
\end{tabular}
\end{center}
Giá trị đại diện của nhóm thứ năm là
\choice
{$173$}
{\True $173{,}5$}
{$174{,}5$}
{$175$}

\loigiai{
Giá trị đại diện của nhóm thứ năm là $\dfrac{172+175}{2}=173{,}5$.
}
\end{ex}

\begin{ex}%[2-D3B3-SO-10-2425]%[VN-MT-7, Bùi Văn Lợi]%[1D5N1-2]
Các bạn học sinh lớp 12A trả lời $40$ câu hỏi trong một bài kiểm tra. Kết quả được thống kê ở bảng sau. Hãy tính độ dài mỗi nhóm.
\begin{center}
\begin{tabular}{|c|c|c|c|c|c|}
\hline
Số câu trả lời đúng & $[20;24)$ & $[24;28)$ & $[28;32)$ & $[32;36)$ & $[36;40)$ \\
\hline
Số học sinh & $4$ & $7$ & $5$ & $10$ & $6$ \\
\hline
\end{tabular}
\end{center}
\choice
{$2$}
{$3$}
{\True $4$}
{$5$}

\loigiai{
Độ dài mỗi nhóm là $4$.
}
\end{ex}

\begin{ex}%[2-D3B3-SO-10-2425]%[VN-MT-7, Bùi Văn Lợi]%[1D5N1-3]
Tìm cân nặng trung bình của học sinh lớp 12A cho trong bảng sau, làm tròn đến hàng phần trăm.
\begin{center}
\begin{tabular}{|c|c|c|c|c|c|c|}
\hline
Cân nặng ($\mathrm{kg}$) & $[40{,}5;45{,}5)$ & $[45{,}5;50{,}5)$ & $[50{,}5;55{,}5)$ & $[55{,}5;60{,}5)$ & $[60{,}5;65{,}5)$ & $[65{,}5;70{,}5)$ \\
\hline
Số học sinh & $8$ & $10$ & $5$ & $12$ & $6$ & $3$ \\
\hline
\end{tabular}
\end{center}
\choice
{$54{,}5$}
{$53{,}31$}
{\True $53{,}79$}
{$54{,}45$}

\loigiai{
Trong mỗi khoảng cân nặng, giá trị đại diện là trung bình cộng của hai giá trị đầu mút nên ta có bảng sau
\begin{center}
\begin{tabular}{|c|c|c|c|c|c|c|}
\hline
Giá trị đại diện & $43$ & $48$ & $53$ & $58$ & $63$ & $68$ \\
\hline
Số học sinh & $8$ & $10$ & $5$ & $12$ & $6$ & $3$ \\
\hline
\end{tabular}
\end{center}
Tổng số học sinh là $n=44$. Cân nặng trung bình của học sinh lớp 12A là
\[\overline{x} = \dfrac{8\cdot 43 +10\cdot 48 +5\cdot 53 +12 \cdot 58+6 \cdot 63+ 3\cdot 68}{44} = 53{,}79.\]
}
\end{ex}

\begin{ex}%[2-D3B3-SO-10-2425]%[VN-MT-7, Bùi Văn Lợi]%[2D3H2-2]
Tìm phương sai của mẫu số liệu ghép nhóm được cho ở bảng sau (làm tròn kết quả đến hàng phần mười).
\begin{center}
\begin{tabular}{|c|c|c|c|c|c|c|}
\hline
Nhóm & $[1{,}5;2)$ & $[2;2{,}5)$ & $[2{,}5;3)$ & $[3;3{,}5)$ & $[3{,}5;4)$\\
\hline
Tần số & $4$ & $9$ & $13$ & $8$ & $6$\\
\hline
\end{tabular}
\end{center}
\choice
{\True $0{,}4$}
{$0{,}7$}
{$1{,}2$}
{$0{,}8$}

\loigiai{
Lập bảng giá trị đại diện cho mẫu số liệu như sau:
\begin{center}
\begin{tabular}{|c|c|c|c|c|c|c|}
\hline
Nhóm & $[1{,}5;2)$ & $[2;2{,}5)$ & $[2{,}5;3)$ & $[3;3{,}5)$ & $[3{,}5;4)$\\
\hline
Giá trị đại diện & $1{,}75$ & $2{,}25$ & $2{,}75$ & $3{,}25$ & $3{,}75$\\
\hline
Tần số & $4$ & $9$ & $13$ & $8$ & $6$\\
\hline
\end{tabular}
\end{center}
Ta có $n=40$, số trung bình cộng của mẫu số liệu ghép nhóm là
\[\overline{x} = \dfrac{4\cdot 1{,}75+9\cdot 2{,}25+13\cdot 2{,}75+8\cdot 3{,}25+6\cdot 3{,}75}{40} = 2{,}7875.\]
Phương sai của mẫu số liệu là
\[s^2 = \dfrac{4\cdot 1{,}75^2+9\cdot 2{,}25^2+13\cdot 2{,}75^2+8\cdot 3{,}25^2+6\cdot 3{,}75^2}{40} - (2{,}7875)^2 \approx 0{,}4.\]
}
\end{ex}

\begin{ex}%[2-D3B3-SO-10-2425]%[VN-MT-7, Bùi Văn Lợi]%[2D3H2-2]
Tìm độ lệch chuẩn của mẫu số liệu ghép nhóm được cho ở bảng sau (làm tròn kết quả đến hàng phần mười).
\begin{center}
\begin{tabular}{|c|c|c|c|c|c|c|}
\hline
Nhóm & $[25;35)$ & $[35;45)$ & $[45;55)$ & $[55;65)$ & $[65;75)$ & \\
\hline
Tần số & $9$ & $7$ & $5$ & $10$ & $9$ & $n=40$ \\
\hline
\end{tabular}
\end{center}
\choice
{$15{,}1$}
{$15{,}0$}
{$14{,}8$}
{\True $14{,}9$}

\loigiai{
Ta có bảng thống kê sau:
\begin{center}
\begin{tabular}{|c|c|c|c|c|c|c|}
\hline Nhóm & $[25;35)$ & $[35;45)$ & $[45;55)$ & $[55;65)$ & $[65;75)$ & \\
\hline Giá trị đại diện & $30$ & $40$ & $50$ & $60$ & $70$ &\\
\hline Tần số & $9$ & $7$ & $5$ & $10$ & $9$ & $n=40$ \\
\hline
\end{tabular}
\end{center}
Số trung bình cộng của mẫu số liệu ghép nhóm là
\[
\overline{x} = \dfrac{30\cdot 9+40\cdot 7+50\cdot 5+60\cdot 10+70\cdot 9}{40} = 50{,}75.
\]
Phương sai của mẫu số liệu là
\[
s^2 = \dfrac{30^2\cdot 9+40^2\cdot 7+50^2\cdot 5+60^2\cdot 10+70^2\cdot 9}{40} - (50{,}75)^2=221{,}9375.
\]
Vậy độ lệch chuẩn của mẫu số liệu trên là $s = \sqrt{221{,}9375} \approx 14{,}9$.
}
\end{ex}
\Closesolutionfile{ans}

\TNTF
\Opensolutionfile{ans}[ans/ans\currfilebase-Phan-II]

\begin{ex}%[2-D3B3-SO-10-2425]%[VN-MT-7, Bùi Văn Lợi]%[2D3H1-4]
Cho bảng số liệu sau:
\begin{center}
\begin{tabular}{|c|c|c|c|c|c|c|}
\hline
Nhóm & $[40;45)$ & $[45;50)$ & $[50;55)$ & $[55;60)$ & $[60;65)$ & \\
\hline
Tần số & $9$ & $5$ & $5$ & $4$ & $7$ & $n=30$ \\
\hline
\end{tabular}
\end{center}
\choiceTF
{Khoảng biến thiên của mẫu số liệu ghép trên là $R = 65$}
{\True Tần số của nhóm $5$ là $7$}
{Tần số tích lũy của nhóm $3$ là $15$}
{\True Tần số tích lũy của nhóm $5$ hơn nhóm $2$ là $16$}

\loigiai{
\begin{itemchoice}
\itemch \textbf{Sai}.\\
Khoảng biến thiên của mẫu số liệu ghép nhóm trên là $R=65-40=25$.
\itemch \textbf{Đúng}.\\
Tần số của nhóm $5$ là $7$.
\itemch \textbf{Sai}.\\
Tần số tích lũy của nhóm $3$ là $cf_3 = n_1+n_2+n_3 = 9+5+5 = 19$.
\itemch \textbf{Đúng}.\\
Tần số tích lũy của nhóm $2$ là $cf_2 = n_1+n_2 = 9+5 = 14$.\\
Tần số tích lũy của nhóm $5$ là $cf_5 = n_1+n_2+n_3+n_4+n_5 = 9+5+5+4+7 = 30$.\\
Vậy tần số tích lũy của nhóm $5$ hơn tần số tích lũy của nhóm $2$ là $16$.
\end{itemchoice}
}
\end{ex}

\begin{ex}%[2-D3B3-SO-10-2425]%[VN-MT-7, Bùi Văn Lợi]%[2D3H1-3]
Cho bảng số liệu sau:
\begin{center}
\begin{tabular}{|c|c|c|c|c|c|c|}
\hline Nhóm & $[6;8)$ & $[8;10)$ & $[10;12)$ & $[12;14)$ & $[14;16)$ & \\
\hline Tần số & $3$ & $5$ & $8$ & $10$ & $4$ & $n=30$ \\
\hline
\end{tabular}
\end{center}
\choiceTF
{Tứ phân vị thứ nhất của mẫu số liệu trên là $Q_1 = 9$}
{\True Tứ phân vị thứ hai của mẫu số liệu có giá trị nhỏ hơn $12$}
{\True Tứ phân vị thứ ba của mẫu số liệu có giá trị nằm trong khoảng $[12;14)$}
{Khoảng tứ phân vị của mẫu số liệu ghép nhóm trên là $\Delta_Q = 3{,}75$}

\loigiai{
Ta có bảng số liệu ghép nhóm:
\begin{center}
\begin{tabular}{|c|c|c|c|c|c|c|}
\hline Nhóm & $[6;8)$ & $[8;10)$ & $[10;12)$ & $[12;14)$ & $[14;16)$ & \\
\hline Tần số & $3$ & $5$ & $8$ & $10$ & $4$ & $n=30$ \\
\hline Tần số tích lũy & $3$ & $8$ & $16$ & $26$ & $30$ & \\
\hline
\end{tabular}
\end{center}
\begin{itemchoice}
\itemch \textbf{Sai}.\\
Ta có $\dfrac{n}{4} = 7{,}5$ mà $3<7{,}5<8$ nên nhóm $2$ là nhóm đầu tiên có tần số tích lũy lớn hơn hoặc bằng $7{,}5$.\\
Xét nhóm $[8;10)$ có đầu mút trái $s=8$, độ dài $h=2$, tần số $n_2 = 5$ và $cf_1 =3$.\\
Vậy, tứ phân vị thứ nhất là
$Q_1 = s+\left(\dfrac{7{,}5-cf_1}{n_2}\right) \cdot h = 8+\left(\dfrac{7{,}5-3}{5}\right) \cdot 2 = 9{,}8$.
\itemch \textbf{Đúng}.\\
Ta có $\dfrac{n}{2} =15$ mà $8<15<16$ nên nhóm $3$ là nhóm đầu tiên có tần số tích lũy lớn hơn hoặc bằng $15$.\\
Xét nhóm $[10;12)$ có đầu mút trái $r=10$, độ dài $d=2$, tần số $n_3 = 8$ và $cf_2 =8$.\\
Vậy, tứ phân vị thứ hai là
$Q_2 = r+\left(\dfrac{15-cf_2}{n_3}\right) \cdot d = 10+\left(\dfrac{15-8}{8}\right) \cdot 2 = 11{,}75 < 12$.
\itemch \textbf{Đúng}.\\
Ta có $\dfrac{3n}{4} =22{,}5$ mà $16<22{,}5<26$ nên nhóm $4$ là nhóm đầu tiên có tần số tích lũy lớn hơn hoặc bằng $22{,}5$.\\
Xét nhóm $[12;14)$ có đầu mút trái $t=12$, độ dài $l=2$, tần số $n_4 = 10$ và $cf_3 =16$.\\
Vậy, tứ phân vị thứ ba là
$Q_3 = t+\left(\dfrac{22{,}5-cf_3}{n_4}\right) \cdot l = 12+\left(\dfrac{22{,}5-16}{10}\right) \cdot 2 = 13{,}3$.
\itemch \textbf{Sai}.\\
Khoảng tứ phân vị của mẫu số liệu ghép nhóm trên là $\Delta_Q = Q_3-Q_1 = 3{,}5$.
\end{itemchoice}
}
\end{ex}

\begin{ex}%[2-D3B3-SO-10-2425]%[VN-MT-7, Bùi Văn Lợi]%[1D5H1-3]
Một mẫu số liệu được cho ở dạng bảng tần số ghép nhóm như sau:
\begin{center}
\begin{tabular}{|c|c|c|c|c|c|}
\hline
Nhóm & $[0{,}5;2{,}5)$ & $[2{,}5;4{,}5)$ & $[4{,}5;6{,}5)$ & $[6{,}5;8{,}5)$ & $[8{,}5;10{,}5)$ \\ 
\hline
Tần số & $4$ & $7$ & $16$ & $8$ & $5$ \\
\hline
\end{tabular}
\end{center}
\choiceTF
{\True Nhóm $[4{,}5;6{,}5)$ có giá trị đại diện là $5{,}5$}
{Nhóm $[8{,}5;10{,}5)$ có giá trị đại diện là $9$}
{\True Các nhóm trong mẫu số liệu đều độ dài bằng nhau}
{Số trung bình của mẫu số liệu trên lớn hơn $5{,}5$ (làm tròn kết quả đến hàng phần trăm)}

\loigiai{
Cỡ mẫu của số liệu là $n = 4+7+16+8+5 = 40$.\\
Bảng sau cho biết giá trị đại diện và độ dài của mỗi nhóm
\begin{center}
\begin{tabular}{|c|c|c|c|c|c|}
\hline
Nhóm & $[0{,}5;2{,}5)$ & $[2{,}5;4{,}5)$ & $[4{,}5;6{,}5)$ & $[6{,}5;8{,}5)$ & $[8{,}5;10{,}5)$ \\
\hline
Giá trị đại diện & $1{,}5$ & $3{,}5$ & $5{,}5$ & $7{,}5$ & $9{,}5$ \\
\hline
Độ dài nhóm & $2$ & $2$ & $2$ & $2$ & $2$ \\
\hline
\end{tabular}
\end{center}
\begin{itemchoice}
\itemch \textbf{Đúng}.\\
Nhóm $[4{,}5;6{,}5)$ có giá trị đại diện là $5{,}5$.
\itemch \textbf{Sai}.\\
Nhóm $[8{,}5;10{,}5)$ có giá trị đại diện là $9{,}5$.
\itemch \textbf{Đúng}.\\
Các nhóm trong mẫu số liệu có độ dài bằng nhau.
\itemch \textbf{Sai}.\\
Số trung bình cộng của mẫu số liệu ghép nhóm là
\[ \overline{x} = \dfrac{4\cdot 1{,}5+7\cdot 3{,}5+16\cdot 5{,}5+8\cdot 7{,}5+5\cdot 9{,}5}{40} = \dfrac{113}{20} = 5{,}65.\]
\end{itemchoice}
}
\end{ex}

\begin{ex}%[2-D3B3-SO-10-2425]%[VN-MT-7, Bùi Văn Lợi]%[2D3H2-2]
Mỗi ngày bác Nam đều đi bộ để rèn luyện sức khỏe. Quãng đường đi bộ mỗi ngày (đơn vị: km) của bác Nam trong $25$ ngày được thống kê ở bảng sau:
\begin{center}
\begin{tabular}{|c|c|c|c|c|c|c|}
\hline
Quãng đường (km) & $[1{,}5;2{,}5)$ & $[2{,}5;3{,}5)$ & $[3{,}5;4{,}5)$ & $[4{,}5;5{,}5)$ & $[5{,}5;6{,}5)$ & \\
\hline
Số ngày & $4$ & $2$ & $10$ & $7$ & $2$ & $n=25$\\
\hline
\end{tabular}
\end{center}
Các mệnh đề sau đúng hay sai? Kết quả làm tròn đến hàng phần trăm.
\choiceTF
{\True Độ dài nhóm của mẫu số liệu trên là $1$}
{\True Số trung bình của mẫu số liệu trên là $4{,}04$}
{Phương sai của mẫu số liệu trên là $1{,}6$}
{\True Độ lệch chuẩn của mẫu số liệu trên là $1{,}15$}

\loigiai{
Ta có bảng sau:
\begin{center}
\begin{tabular}{|c|c|c|c|c|c|}
\hline
Nhóm & $[1{,}5;2{,}5)$ & $[2{,}5;3{,}5)$ & $[3{,}5;4{,}5)$ & $[4{,}5;5{,}5)$ & $[5{,}5;6{,}5)$\\
\hline
Giá trị đại diện & $2$ & $3$ & $4$ & $5$ & $6$\\
\hline
Tần số & $4$ & $2$ & $10$ & $7$ & $2$\\
\hline
\end{tabular}
\end{center}
\begin{itemchoice}
\itemch \textbf{Đúng}.\\
Độ dài nhóm của mẫu số liệu là $1$.
\itemch \textbf{Đúng}.\\
Số trung bình của mẫu số liệu là
\[\overline{x} = \dfrac{2\cdot 4+3\cdot 2 +4\cdot 10+5\cdot 7+6\cdot 2}{25} = 4{,}04.\]
\itemch \textbf{Sai}.\\
Phương sai của mẫu số liệu là
\[
s^2 = \dfrac{(2-4{,}04)^2 \cdot 4+ (3-4{,}04)^2 \cdot 2 + (4-4{,}04)^2\cdot 10+(5-4{,}04)^2\cdot 7+(6-4{,}04)^2\cdot 2}{25} \approx 1{,}32.
\]
\itemch \textbf{Đúng}.\\
Độ lệch chuẩn của mẫu số liệu là $s = \sqrt{s^2} \approx 1{,}15$.
\end{itemchoice}
}
\end{ex}
\Closesolutionfile{ans}

\TNSA
\Opensolutionfile{ans}[ans/ans\currfilebase-Phan-III]

\begin{ex}%[2-D3B3-SO-10-2425]%[VN-MT-7, Bùi Văn Lợi]%[2D3N1-2]
Kết quả đo chiều cao của $250$ cây dừa đột biến $3$ năm tuổi ở một viện nghiên cứu được tổng hợp ở bảng sau:
\begin{center}
\begin{tabular}{|c|c|c|c|c|c|}
\hline 
Chiều cao (m$^2$) & {$[8{,}5 ; 8{,}8)$} & {$[8{,}8 ; 9{,}1)$} & {$[9{,}1 ; 9{,}4)$} & {$[9{,}4 ; 9{,}7)$} & {$[9{,}7 ; 10)$} \\
\hline Số cây & $36$ & $45$ & $83$ & $65$ & $21$ \\
\hline
\end{tabular}
\end{center}
Tìm khoảng biến thiên của mẫu số liệu ghép nhóm trên.

\shortans[]{1{,}5}

\loigiai{
Khoảng biến thiên của mẫu số liệu ghép nhóm là
$R=10-8{,}5=1{,}5$.
}
\end{ex}

\begin{ex}%[2-D3B3-SO-10-2425]%[VN-MT-7, Bùi Văn Lợi]%[2D3H1-4]
Một cộng ty bất động sản Đất Vàng thực hiện cuộc khảo sát khách hàng xem họ có
nhu cầu mua nhà ở mức giá nào đã tiến hành dự án xây nhà ở Thăng Long group sắp tới. Kết quả khảo sát $500$ khách hàng được ghi lại ở bảng sau:
\begin{center}
\begin{tabular}{|l|c|c|c|c|c|}
\hline 
Mức giá (triệu đồng) & $[10 ; 14)$ & $[14 ; 18)$ & $[18 ; 22)$ & $[22 ; 26)$ & $[26 ; 30)$ \\
\hline Số khách hàng & $75$ & $105$ & $179$ & $96$ & $45$ \\
\hline
\end{tabular}
\end{center}
Tìm tần số tích lũy của nhóm $[18;22)$.

\shortans[]{359}

\loigiai{
Tần số tích lũy của nhóm $[18;22)$ là $cf_3 = 75+105+179 = 359$.
}
\end{ex}

\begin{ex}%[2-D3B3-SO-10-2425]%[VN-MT-7, Bùi Văn Lợi]%[1D5H2-3]
Bảng dưới đây biểu diễn mẫu số liệu ghép nhóm về chiều cao (đơn vị: cm) của $43$ học sinh trong một lớp học khối $11$ của một trường phổ thông.
\begin{center}
\begin{tabular}{|c|c|c|c|c|c|c|c|}
\hline 
Nhóm & $[150 ; 155)$ & $[155 ; 160)$ & $[160 ; 165)$ & $[165 ; 170)$ & $[170 ; 175)$ & $[175 ; 180)$ & \\
\hline 
Tần số & $5$ & $10$ & $12$ & $9$ & $4$ & $3$ & $n=43$ \\ 
\hline 
\end{tabular} 
\end{center}
Tứ phân vị thứ hai của mẫu số liệu ghép nhóm trên bằng bao nhiêu (kết quả làm tròn đến hàng đơn vị)?

\shortans[]{163}

\loigiai{
Ta có bảng sau:
\begin{center}
\begin{tabular}{|c|c|c|}
\hline Nhóm & Tần số & Tần số tích lũy\\
\hline$[150 ; 155)$ & $5$ & $5$\\
\hline$[155 ; 160)$ & $10$& $15$\\
\hline$[160 ; 165)$ & $12$& $27$\\
\hline$[165 ; 170)$ & $9$ & $36$\\
\hline$[170 ; 175)$ & $4$ & $40$\\
\hline$[175 ; 180)$ & $3$ & $43$\\
\hline & $n=43$ & \\
\hline
\end{tabular}
\end{center}
Số phần tử của mẫu là $n=43$.\\
Ta có $\dfrac{n}{2}=21{,}5$ mà $15<21{,}5<27$ nên nhóm $3$ là nhóm đầu tiên có tần số tích lũy lớn hơn hoặc bằng $21{,}5$.\\
Xét nhóm $3$ là nhóm $[160;165)$ có đầu mút trái $r=160$, độ dài $d=5$, tần số $n_3=12$ và $cf_2=15$.\\
Từ đó ta có tứ phân vị thứ $2$ là
\[Q_2=160+\left(\dfrac{21{,}5-15}{12}\right)\cdot 5 \approx 163~\text{(cm)}.\]
}
\end{ex}

\begin{ex}%[2-D3B3-SO-10-2425]%[VN-MT-7, Bùi Văn Lợi]%[2D3H1-3] 
Bảng sau thống kê khối lượng (đơn vị: gam) một số quả măng cụt được lựa chọn ngẫu nhiên trong một thùng hàng
\begin{center}
\begin{tabular}{|c|c|c|c|c|c|c|}
\hline 
Nhóm & $[80 ; 82)$ & $[82 ; 84)$ & $[84 ; 86)$ & $[86 ; 88)$ & $[88 ; 90)$ & \\ 
\hline 
Tần số & $17$ & $22$ & $26$ & $19$ & $16$ & $n=100$ \\ 
\hline 
\end{tabular} 
\end{center}
Tính khoảng tứ phân vị của mẫu số liệu ghép nhóm trên (kết quả làm tròn đến hàng phần mười).

\shortans[]{4{,}3}

\loigiai{
Ta có bảng sau:
\begin{center}
\begin{tabular}{|c|c|c|}
\hline Nhóm & Tần số & Tần số tích lũy\\
\hline$[80 ; 82)$ & $17$ & $17$\\
\hline$[82 ; 84)$ & $22$ & $39$\\
\hline$[84 ; 86)$ & $26$ & $65$\\
\hline$[86 ; 88)$ & $19$ & $84$\\
\hline$[88 ; 90)$ & $16$ & $100$\\
\hline & $n=100$ & \\
\hline
\end{tabular}
\end{center}
Số phần tử của mẫu là $n=100$.\\
Ta có $\dfrac{n}{4}=25$ mà $17<25<39$.\\ 
Suy ra nhóm $2$ là nhóm đầu tiên có tần số tích lũy lớn hơn hoặc bằng $25$.\\ 
Xét nhóm $2$ là nhóm $[82 ; 84)$ có đầu mút trái $s=82$, độ dài $h=2$, tần số $n_2=22$ và $cf_1=17$.\\
Từ đó ta có tứ phân vị thứ nhất là 
\[Q_1=82+\left(\dfrac{25-17}{22}\right)\cdot 2 = \dfrac{910}{11}.\]
Tương tự, ta có $\frac{3n}{4}=75$ mà $65<75<84$.
Suy ra nhóm $4$ là nhóm có tần số tích lũy lớn hơn hoặc bằng $75$.\\
Khi đó tứ phân vị thứ ba là 
\[Q_3=86+ \left(\dfrac{75-65}{19}\right) \cdot 2 = \dfrac{1654}{19}.\]
Vậy khoảng tứ phân vị của mẫu số liệu ghép nhóm đã cho là 
\[\Delta_Q=Q_3-Q_1 =\dfrac{1654}{19}-\dfrac{910}{11} \approx 4{,}3~\text{(gam)}.\]
}
\end{ex}

\begin{ex}%[2-D3B3-SO-10-2425]%[VN-MT-7, Bùi Văn Lợi]%[1D5H1-3]
Mẫu số liệu cân nặng (đơn vị: gam) của $30$ trái xoài trong một thùng xoài chuẩn bị đem ra thị trường được biểu thị ở bảng sau:
\begin{center}
\begin{tabular}{|c|c|c|c|c|c|}
\hline
Cân nặng (g) & $[400;480)$ & $[480;560)$ & $[560;640)$ & $[640;720)$ & $[720;800)$ \\
\hline
Số quả xoài & $5$ & $3$ & $13$ & $7$ & $2$ \\
\hline
\end{tabular}
\end{center}
Cân nặng trung bình của mỗi quả xoài có dạng $\dfrac{a}{b}$ (gam) với $\dfrac{a}{b}$ là phân số tối giản ($a$, $b \in \mathbb{N}$). Khi đó giá trị của biểu thức $M = 2a-3b$ là bao nhiêu?

\shortans[]{3559}

\loigiai{
Ta có bảng sau:
\begin{center}
\begin{tabular}{|c|c|c|c|c|c|}
\hline
Nhóm & $[400;480)$ & $[480;560)$ & $[560;640)$ & $[640;720)$ & $[720;800)$ \\
\hline
Giá trị đại diện & $440$ & $520$ & $600$ & $680$ & $760$ \\
\hline
Tần số & $5$ & $3$ & $13$ & $7$ & $2$ \\
\hline
\end{tabular}
\end{center}
Số trung bình cộng của mẫu số liệu ghép nhóm trên là
\[
\overline{x}=\dfrac{5 \cdot 440+3\cdot 520+13 \cdot 600+7\cdot 680+2 \cdot 760}{30} = \dfrac{1784}{3}~\text{(gam)}.
\]
Khi đó $\dfrac{a}{b} = \dfrac{1784}{3}$ nên $M =2a-3b=3559$.
}
\end{ex}

\begin{ex}%[2-D3B3-SO-10-2425]%[VN-MT-7, Bùi Văn Lợi]%[2D3H2-2] 
Kiểm tra khối lượng của $40$ bao xi măng (đơn vị: kg) được chọn ngẫu nhiêm trước khi xuất xưởng, ta được mẫu số liệu ghép nhóm sau:
\begin{center}
\begin{tabular}{|c|c|c|c|c|c|c|}
\hline
Khối lượng (kg) & $[48{,}5;49)$ & $[49;49{,}5)$ & $[49{,}5;50)$ & $[50;50{,}5)$ & $[50{,}5;51)$ & $[51;51{,}5)$ \\
\hline
Số bao xi măng & $7$ & $3$ & $8$ & $6$ & $7$ & $9$ \\
\hline
\end{tabular}
\end{center}
Phương sai của mẫu số liệu ghép nhóm trên bằng bao nhiêu (kết quả làm tròn đến hàng phần trăm)?

\shortans[]{0{,}77}

\loigiai{
Ta có bảng sau:
\begin{center}
\begin{tabular}{|c|c|c|c|c|c|c|}
\hline
Nhóm & $[48{,}5;49)$ & $[49;49{,}5)$ & $[49{,}5;50)$ & $[50;50{,}5)$ & $[50{,}5;51)$ & $[51;51{,}5)$ \\
\hline
Giá trị đại diện & $48{,}75$ & $49{,}25$ & $49{,}75$ & $50{,}25$ & $50{,}75$ & $51{,}25$ \\
\hline
Tần số & $7$ & $3$ & $8$ & $6$ & $7$ & $9$ \\
\hline
\end{tabular}
\end{center}
Số trung bình cộng của mẫu số liệu ghép nhóm trên là
\[
\overline{x}=\dfrac{7\cdot 48{,}75+3\cdot 49{,}25+8\cdot 49{,}75+6\cdot 50{,}25+7\cdot 50{,}75+9\cdot 51{,}25}{40} = 50{,}125~\text{(kg)}.
\]
Phương sai của mẫu số liệu ghép nhóm trên là
\[
s^2=\dfrac{7\cdot 48{,}75^2+3\cdot 49{,}25^2+8\cdot 49{,}75^2+6\cdot 50{,}25^2+7\cdot 50{,}75^2+9\cdot 51{,}25^2}{40} - (50{,}125)^2 \approx 0{,}77.
\]
}
\end{ex}
\Closesolutionfile{ans}

\begin{indapan}
	{ans/ans\currfilebase}
\end{indapan}


% \begin{name}
 {Biên soạn: Nguyen Huynh \\ Phản biện: Nguyễn Kiều Nhã Tú}
 {Đề ôn tập chương III}
\end{name}

\caulc
\Opensolutionfile{ans}[ans/ans\currfilebase-Phan-I]

\begin{ex}%[2-D3B3-SO-11-2425]%[VN-MT-7, Nguyen Huynh]%[1D5H2-3]
 Khảo sát thời gian xem ti vi trong một ngày của một số học sinh khối $11$ thu được mẫu số liệu ghép nhóm sau:
 \begin{center}
 \begin{tabular}{|l|c|c|c|c|c|}
 \hline Thời gian (phút)&$[0 ; 20)$&$[20 ; 40)$&$[40 ; 60)$&$[60 ; 80)$&$[80 ; 100)$\\
 \hline Số học sinh & $5$ & $9$ & $12$ & $10$ & $6$ \\
 \hline
 \end{tabular}
 \end{center}
 Nhóm chứa tứ phân vị thứ nhất là
 \choice
 {$[0 ; 20)$}
 {\True $[20 ; 40)$}
 {$[40 ; 60)$}
 {$[60 ; 80)$}
 \loigiai{
 Cỡ mẫu $n=5+9+12+10+6=42$.\\ 
 Gọi $x_1$, $x_2$, $\ldots$, $x_{42}$ là mẫu số liệu về thời gian xem ti vi trong một ngày của một số học sinh khối $11$ được xếp theo thứ tự không giảm.\\
 Ta có
 $x_1$, $\ldots$, $x_5 \in[0 ; 20)$; $x_6$, $\ldots$, $x_{14} \in[20 ; 40)$; $x_{15}$, $\ldots$, $x_{26} \in[40 ; 60)$; $x_{27}$, $\ldots$, $x_{36} \in[60 ; 80)$; $x_{37}; \ldots$; $x_{42} \in[40 ; 45)$.\\
 Tứ phân vị thứ nhất của mẫu số liệu $x_1$, $x_2$, $\ldots$, $x_{42}$ là $\dfrac{1}{2}(x_{10}+x_{11})$.\\ 
 Do $x_{10}\in [20 ; 40)$ và $x_{11}\in [20 ; 40)$ nên nhóm chứa tứ phân vị thứ nhất là $[20 ; 40)$.}
\end{ex}

\begin{ex}%[2-D3B3-SO-11-2425]%[VN-MT-7, Nguyen Huynh]%[2D3N1-2]
 Cô Hà thống kê lại đường kính thân gỗ của một số cây xoan đào $6$ năm tuổi được trồng ở một lâm trường ở bảng sau:
 \begin{center}
 \begin{tabular}{|c|c|c|c|c|c|}
 \hline Đường kính $(\mathrm{cm})$ &$[40 ; 45)$&$[45 ; 50)$&$[50 ; 55)$&$[55 ; 60)$&$[60 ; 65)$\\
 \hline Tần số & $5$ & $20$ & $18$ & $7$ & $3$ \\
 \hline
 \end{tabular}
 \end{center}
 Hãy tìm khoảng biến thiên của mẫu số liệu ghép nhóm trên.
 \choice
 {\True $25$}
 {$30$}
 {$6$}
 {$69{,}8$}
 \loigiai{Khoảng biến thiên của mẫu số liệu ghép nhóm trên là $65-40=25$.}
\end{ex}

\begin{ex}%[2-D3B3-SO-11-2425]%[VN-MT-7, Nguyen Huynh]%[2D3H1-3]
 Bạn Chi rất thích nhảy hiện đại. Thời gian tập nhảy mỗi ngày trong thời gian gần đây của bạn Chi được thống kê lại ở bảng sau:
 \begin{center}
 \begin{tabular}{|c|c|c|c|c|c|}
 \hline Thời gian(phút)&{$[20 ; 25)$}&{$[25 ; 30)$}&{$[30 ; 35)$}&{$[35 ; 40)$}&{$[40 ; 45)$}\\
 \hline Số ngày & $6$ & $6$ & $4$ & $1$ & $1$ \\
 \hline
 \end{tabular}
 \end{center}
 Khoảng tứ phân vị của mẫu số liệu ghép nhóm là
 \choice
 {$23{,}75$}
 {$27{,}5$}
 {$31{,}88$}
 {\True $8{,}125$}
 \loigiai{Cỡ mẫu $n=18$. \\
 Gọi $x_1$, $x_2$, $\ldots$, $x_{18}$ là mẫu số liệu về thời gian tập nhảy mỗi ngày của bạn Chi được xếp theo thứ tự không giảm.\\
 Ta có
 $x_1$, $\ldots$, $x_6 \in[20 ; 25)$; $x_7$, $\ldots$, $x_{12} \in[25 ; 30)$; $x_{13}$, $\ldots$, $x_{16} \in[30 ; 35)$; $x_{17}$, $\in[35 ; 40)$; $x_{18} \in[40 ; 45)$.\\
 Tứ phân vị thứ nhất của mẫu số liệu gốc là $x_5 \in[20 ; 25)$. Do đó, tứ phân vị thứ nhất của mẫu số liệu ghép nhóm là \[Q_1=20+\dfrac{\dfrac{18}{4}}{6}(25-20)=23{,}75.\]
 Tứ phân vị thứ ba của mẫu số liệu góc là $x_{14} \in[30 ; 35)$. Do đó, tứ phân vị thứ ba của mẫu số liệu ghép nhóm là \[Q_3=30+\dfrac{\dfrac{3\cdot18}{4}-(6+6)}{4}(35-30)=31{,}875.\]
 Khoảng tứ phân vị của mẫu số liệu ghép nhóm là $\Delta_Q=Q_3-Q_1=8{,}125$.}
\end{ex}

\begin{ex}%[2-D3B3-SO-11-2425]%[VN-MT-7, Nguyen Huynh]%[1D5H2-3]
 Trong dịp nghỉ hè bạn Lan rất thích đi bơi. Thời gian đi bơi mỗi ngày trong thời gian gần đây của bạn Lan được thống kê lại ở bảng sau:
 \begin{center}
 \begin{tabular}{|c|c|c|c|c|c|}
 \hline Thời gian (phút) & {$[30 ; 35)$} & {$[35 ; 40)$} & {$[45 ; 50)$} & {$[50 ; 55)$} & {$[55 ; 60)$} \\
 \hline Số ngày & $3$ & $6$ & $4$ & $8$ & $4$ \\
 \hline
 \end{tabular}
 \end{center}
 Nhóm chứa tứ phân vị thứ nhất $Q_1$ là
 \choice
 {$[30 ; 35)$}
 {\True $[35 ; 40)$}
 {$[45 ; 50)$}
 {$[50 ; 55)$}
 \loigiai{Cỡ mẫu là $n=25$.\\
 Gọi $x_1$; $x_2$; $\ldots$; $x_{25}$ là mẫu số liệu về thời gian đi bơi mỗi ngày trong thời gian gần đây của bạn Lan được xếp theo thứ tự không giảm.\\
 Ta có
 $x_1$; $\ldots$; $x_3 \in[30 ; 35)$; $x_4$; $\ldots$; $x_{9} \in[35 ; 40)$; $x_{10}$; $\ldots$; $x_{13} \in[45 ; 50)$; $x_{14}$; $\ldots$; $x_{21}\in[50 ; 55)$; $x_{22}$; $\ldots$; $x_{25} \in[55 ; 60)$.\\ 
 Tứ phân vị thứ nhất của mẫu số liệu gốc là $\frac{x_6+x_7}{2}$. Do $x_6$, $x_7$ đều thuộc nhóm $[35 ; 40)$ nên nhóm này chứa $Q_1$.}
\end{ex}

\begin{ex}%[2-D3B3-SO-11-2425]%[VN-MT-7, Nguyen Huynh]%[1D5H2-3]
 Khảo sát thời gian tập nghe nhạc trong ngày của học sinh lớp 12B thu được mẫu số liệu ghép nhóm sau:
 \begin{center}
 \begin{tabular}{|c|c|c|c|c|c|}
 \hline Thời gian(phút)&{$[0 ; 20)$}&{$[20 ; 40)$}&{$[40 ; 60)$}&{$[60 ; 80)$}&{$[80 ; 100)$}\\
 \hline Số học sinh & $5$ & $10$ & $12$ & $9$ & $4$ \\
 \hline
 \end{tabular}
 \end{center}
 Nhóm chứa tứ phân vị thứ ba $Q_3$ là
 \choice
 {$[20 ; 40)$}
 {$[40 ; 60)$}
 {\True $[60 ; 80)$}
 {$[80 ; 100)$}
 \loigiai{
 Cỡ mẫu là $n=40$.\\
 Gọi $x_1$; $x_2$; $\ldots$; $x_{40}$ là mẫu số liệu về tập nghe nhạc trong ngày của học sinh lớp 12B được xếp theo thứ tự không giảm.\\
 Ta có
 $x_1$; $\ldots$; $x_5 \in[0 ; 20)$; $x_6$; $\ldots$; $x_{15} \in[20 ; 40)$; $x_{16}$; $\ldots$; $x_{27} \in[40 ; 60)$; $x_{28}$; $\ldots$; $x_{36}\in[60 ; 80)$; $x_{37}$; $\ldots$; $x_{40} \in[80 ; 100)$.\\ 
 Tứ phân vị thứ ba của mẫu số liệu gốc là $\dfrac{x_{30}+x_{31}}{2}$. Do $x_{30}$, $x_{31}$ đều thuộc nhóm $[60 ; 80)$ nên nhóm này chứa $Q_3$.}
\end{ex}

\begin{ex}%[2-D3B3-SO-11-2425]%[VN-MT-7, Nguyen Huynh]%[1D5H2-3]
 Một nhóm học sinh thi nhau giải khối rubik $4 \times 4$. Thời gian (đơn vị: giây) hoàn thành của nhóm học sinh được thống kê trong bảng sau:
 \begin{center}
 \begin{tabular}{|c|c|c|c|c|c|}
 \hline Thời gian giải rubik &{$[8 ; 10)$}&{$[10 ; 12)$}&{$[12 ; 14)$}&{$[14 ; 16)$}&{$[16 ; 18)$}\\
 \hline Số học sinh & $4$ & $6$ & $8$ & $4$ & $3$ \\
 \hline
 \end{tabular}
 \end{center}
 Tìm tứ phân vị thứ nhất và tứ phân vị thứ ba của mẫu số liệu.
 \choice
 {\True $Q_1=10{,}75$, $Q_3=14{,}375$}
 {$Q_1=11{,}0625$, $Q_3=14{,}375$}
 {$Q_1=10{,}75$, $Q_3=13{,}83$}
 {$Q_1=10{,}85$, $Q_3=14{,}75$}
 \loigiai{Cỡ mẫu là $n=25$.\\
 Tứ phân vị thứ nhất của mẫu số liệu gốc là $\dfrac{x_6+x_7}{2}\in[10 ; 12)$. Do đó, tứ phân vị thứ nhất của mẫu số liệu ghép nhóm là \[Q_1=10+\dfrac{\dfrac{25}{4}-4}{6}(12-10)=10{,}75.\]
 Tứ phân vị thứ ba của mẫu số liệu là $\dfrac{x_{19}+x_{20}}{2}\in[14 ; 16)$. Do đó, tứ phân vị thứ ba của mẫu số liệu ghép nhóm là \[Q_3=14+\dfrac{\dfrac{3\cdot25}{4}-(4+6+8)}{4}(16-14)=14{,}375.\]}
\end{ex}

\begin{ex}%[2-D3B3-SO-11-2425]%[VN-MT-7, Nguyen Huynh]%[2D3H1-3]
 Mỗi ngày bác Hương đều đi bộ để rèn luyện sức khoẻ. Quãng đường đi bộ mỗi ngày (đơn vị: km) của bác Hương trong $20$ ngày được thống kê lại ở bảng sau:
 \begin{center}
 \begin{tabular}{|c|c|c|c|c|c|}
 \hline
 Quãng đường & $[2{,}7;3{,}0)$ & $[3{,}0;3{,}3)$ & $[3{,}3;3{,}6)$ & $[3{,}6;3{,}9)$ & $[3{,}9;4{,}2)$ \\
 \hline
 Số ngày & $3$ & $6$ & $5$ & $4$ & $2$ \\
 \hline
 \end{tabular}
 \end{center}
 Khoảng tứ phân vị của mẫu số liệu ghép nhóm là
 \choice
 {$0{,}9$}
 {$0{,}975$}
 {$0{,}5$}
 {\True $0{,}575$}
 \loigiai{
 Cỡ mẫu $n=20$.\\
 Gọi $x_1$, $x_2$, $\ldots$, $x_{20}$ là mẫu số liệu về quãng đường đi bộ mỗi ngày của bác Hương trong $20$ ngày được xếp theo thứ tự không giảm.\\
 Ta có $x_1,\ldots,x_3 \in [2{,}7;3{,}0)$; $x_4,\ldots,x_9 \in [3{,}0;3{,}3)$; $x_{10},\ldots,x_{14} \in [3{,}3;3{,}6)$; $x_{15},\ldots,x_{18} \in [3{,}6;3{,}9)$; $x_{19},x_{20} \in [3{,}9;4{,}2)$.\\
 Tứ phân vị thứ nhất của mẫu số liệu gốc là $\dfrac{1}{2} \left(x_5+x_6 \right)\in [3{,}0;3{,}3)$.
 Do đó, tứ phân vị thứ nhất của mẫu số liệu ghép nhóm là \[Q_1=3{,}0+\dfrac{\dfrac{20}{4}-3}{6} (3{,}3-3{,}0)=3{,}1\]
 Tứ phân vị thứ ba của mẫu số liệu là $\dfrac{1}{2} \left(x_{15}+x_{16} \right)\in [3{,}6;3{,}9)$.
 Do đó, tứ phân vị thứ ba của mẫu số liệu ghép nhóm là
 \[Q_3=3{,}6+\dfrac{\dfrac{3\cdot 20}{4}-(3+6+5)}{4} (3{,}9-3{,}6)=3{,}675.
 \]
 Khoảng tứ phân vị của mẫu số liệu ghép nhóm là
 \[\Delta_Q=Q_3-Q_1=0{,}575.
 \]
 }
\end{ex}


\begin{ex}%[2-D3B3-SO-11-2425]%[VN-MT-7, Nguyen Huynh]%[1D5H1-3]
 Doanh thu bán hàng trong $20$ ngày được lựa chọn ngẫu nhiên của một của hàng được ghi lại ở bảng sau (đơn vị: triệu đồng):
 \begin{center}
 \begin{tabular}{|c|c|c|c|c|c|}
 \hline
 Doanh thu & $[5;7)$ & $[7;9)$ & $[9;11)$ & $[11;13)$ & $[13;15)$ \\
 \hline
 Số ngày & $2$ & $7$ & $7$ & $3$ & $1$ \\
 \hline
 \end{tabular}
 \end{center}
 Số trung bình của mẫu số liệu trên thuộc khoảng nào trong các khoảng dưới đây?
 \choice
 {$[7;9)$}
 {\True $[9;11)$}
 {$[11;13)$}
 {$[13;15)$}
 \loigiai{
 Bảng tần số ghép nhóm theo giá trị đại diện là
 \begin{center}
 \begin{tabular}{|c|c|c|c|c|c|}
 \hline
 Doanh thu & $[5;7)$ & $[7;9)$ & $[9;11)$ & $[11;13)$ & $[13;15)$ \\
 \hline
 Giá trị đại diện & $6$ & $8$ & $10$ & $12$ & $14$ \\
 \hline
 Số ngày & $2$ & $7$ & $7$ & $3$ & $1$ \\
 \hline
 \end{tabular}
 \end{center}
 Số trung bình \[\overline{x}=\dfrac{2\cdot6+7\cdot8+7\cdot10+3\cdot12+1\cdot14}{20}=9{,}4.\]
 }
\end{ex}


\begin{ex}%[2-D3B3-SO-11-2425]%[VN-MT-7, Nguyen Huynh]%[2D3H2-2]
 Một siêu thị thống kê số tiền (đơn vị: chục nghìn đồng) mà $44$ khách hàng mua hàng ở siêu thị đó trong một ngày. Số liệu được ghi lại trong bảng sau:
 \begin{center}
 \begin{tabular}{|c|c|c|}
 \hline
 Nhóm & Giá trị đại diện & Tần số \\
 \hline
 $[40;45)$ & $42{,}5$ & $4$ \\
 \hline
 $[45;50)$ & $47{,}5$ & $14$ \\
 \hline
 $[50;55)$ & $52{,}5$ & $8$ \\
 \hline
 $[55;60)$ & $57{,}5$ & $10$ \\
 \hline
 $[60;65)$ & $62{,}5$ & $6$ \\
 \hline
 $[65;70)$ & $67{,}5$ & $2$ \\
 \hline
 & & $n=44$ \\
 \hline
 \end{tabular}
 \end{center}
 Phương sai của mẫu số liệu ghép nhóm trên là
 \choice
 {$53{,}2$}
 {\True $46{,}1$}
 {$30$}
 {$11$} 
 \loigiai{Số trung bình cộng của mẫu số liệu ghép nhóm là
 \[\overline{x}=\dfrac{4\cdot42{,}5+14\cdot47{,}5+8\cdot52{,}5+10\cdot57{,}5+6\cdot62{,}5+2\cdot67{,}5}{44}=\dfrac{585}{11}.\]
 Phương sai của mẫu số liệu ghép nhóm là
 \begin{align*}
 \allowdisplaybreaks
 s^2&=\dfrac{4\left(42{,}5-\dfrac{585}{11} \right)^2+14\left(47{,}5-\dfrac{585}{11} \right)^2+8\left(52{,}5-\dfrac{585}{11} \right)^2+10\left(57{,}5-\dfrac{585}{11} \right)^2}{44} \\
 &+\dfrac{6\left(62{,}5-\dfrac{585}{11} \right)^2+2\cdot \left(67{,}5-\dfrac{585}{11} \right)^2}{44}\approx 46{,}1.
 \end{align*}
 }
\end{ex}


\begin{ex}%[2-D3B3-SO-11-2425]%[VN-MT-7, Nguyen Huynh]%[2D3H2-2]
 Khảo sát chiều cao (đơn vị cm) của học sinh lớp 12A, ta thu được kết quả như sau: 
 \begin{center}
 \begin{tabular}{|c|c|c|c|c|c|}
 \hline
 Kết quả đo & $[150;155)$ & $[155;160)$ & $[160;165)$ & $[165;170)$ & $[170;175)$ \\
 \hline
 Số học sinh & $6$ & $10$ & $14$ & $5$ & $5$ \\
 \hline
 \end{tabular}
 \end{center}
 Độ lệch chuẩn của mẫu số liệu ghép nhóm trên thuộc khoảng nào sau đây
 \choice
 {$\left(5{,}5;6\right)$}
 {\True $\left(6;6{,}5\right)$}
 {$\left(6{,}5;7\right)$}
 {$\left(7;7{,}5\right)$}
 \loigiai{
 Chọn giá trị đại diện cho các nhóm số liệu, ta có
 \begin{center}
 \begin{tabular}{|c|c|c|c|c|c|}
 \hline
 Giá trị đại diện & $152{,}5$ & $157{,}5$ & $162{,}5$ & $167{,}5$ & $172{,}5$ \\
 \hline
 Số học sinh & $6$ & $10$ & $14$ & $5$ & $5$ \\
 \hline
 \end{tabular}
 \end{center}
 Tổng số học sinh tham gia khảo sát là $n=6+10+14+5+5=40$.\\
 Chiều cao trung bình của học sinh trong lớp là \[\overline{x}=\dfrac{152{,}5\cdot 6+157{,}5\cdot10+162{,}5\cdot14+167{,}5\cdot5+172{,}5\cdot5}{40}=161{,}625\approx 161{,}6.\]
 Phương sai của mẫu số liệu trên là
 \begin{align*}
 s^2&=\dfrac{m_1 \left(x_1-\overline{x}\right)^2+\cdots+m_k \left(x_k-\overline{x}\right)^2}{n}\\
 &=\dfrac{6\left(152{,}5-161{,}6\right)^2+10\left(157{,}5-161{,}6\right)^2+14\left(162{,}5-161{,}6\right)^2}{40}\\
 &+\dfrac{5\left(167{,}5-161{,}6\right)^2+6\left(172{,}5-161{,}6\right)^2}{40}\\
 &\approx 36{,}1.
 \end{align*}
 Độ lệch chuẩn của mẫu số liệu trên là $s=\sqrt{s^2}=\sqrt{36{,}1} \approx 6{,}01\in(6;6{,}5)$.
 }
\end{ex}


\begin{ex}%[2-D3B3-SO-11-2425]%[VN-MT-7, Nguyen Huynh]%[2D3N1-1]
 Có bao nhiêu nhận xét đúng trong các nhận xét sau:
 \begin{enumerate}
 \item Khoảng biến thiên của mẫu số liệu ghép nhóm luôn luôn bằng khoảng biến thiên của mẫu số liệu.
 \item Khoảng biến thiên của mẫu số liệu ghép nhóm được dùng để đo mức độ phân tán của mẫu số liệu ghép nhóm.
 \item Khoảng biến thiên của mẫu số liệu ghép nhóm càng lớn thì mẫu số liệu càng phân tán.
 \end{enumerate}
 \choice
 {$0$}
 {$1$}
 {\True $2$}
 {$3$}
 \loigiai{
 \begin{itemize}
 \item Nhận xét ``Khoảng biến thiên của mẫu số liệu ghép nhóm luôn luôn bằng khoảng biến thiên của mẫu số liệu''\, sai vì khoảng biến thiên của mẫu số liệu ghép nhóm xấp xỉ cho khoảng biến thiên của mẫu số liệu. 
 \item Nhận xét ``Khoảng biến thiên của mẫu số liệu ghép nhóm được dùng để đo mức độ phân tán của mẫu số liệu ghép nhóm''\, đúng.
 \item Nhận xét ``Khoảng biến thiên của mẫu số liệu ghép nhóm càng lớn thì mẫu số liệu càng phân tán''\, đúng.
 \end{itemize}
 
 }
\end{ex}


\begin{ex}%[2-D3B3-SO-11-2425]%[VN-MT-7, Nguyen Huynh]%[2D3N1-1]
 Nhận xét nào \textbf{sai} trong các nhận xét sau?
 \begin{enumerate}
 \item Khoảng tứ phân vị của mẫu số liệu ghép nhóm bị ảnh hưởng bởi các giá trị bất thường trong mẫu số liệu.
 \item Khoảng tứ phân vị của mẫu số liệu ghép nhóm xấp xỉ cho khoảng tứ phân vị của mẫu số liệu.
 \item Khoảng tứ phân vị càng lớn thì mẫu số liệu càng phân tán.
 \item Khoảng tứ phân vị được dùng để đo mức độ phân tán của mẫu số liệu ghép nhóm.
 \end{enumerate}
 \choice
 {\True Nhận xét a)}
 {Nhận xét b)}
 {Nhận xét c)}
 {Nhận xét d)}
 \loigiai{\begin{itemize}
 \item Nhận xét ``Khoảng tứ phân vị của mẫu số liệu ghép nhóm bị ảnh hưởng bởi các giá trị bất thường trong mẫu số liệu''\, sai vì khoảng tứ phân vị của mẫu số liệu ghép nhóm chỉ phụ thuộc vào nửa giữa của mẫu số liệu, nên không bị ảnh hưởng bởi các giá trị bất thường và có thể dùng đại lượng này để loại giá trị bất thường.
 \item Nhận xét ``Khoảng tứ phân vị của mẫu số liệu ghép nhóm xấp xỉ cho khoảng tứ phân vị của mẫu số liệu''\, đúng.
 \item Nhận xét ``Khoảng tứ phân vị càng lớn thì mẫu số liệu càng phân tán''\, đúng.
 \item Nhận xét ``Khoảng tứ phân vị được dùng để đo mức độ phân tán của mẫu số liệu ghép nhóm''\, đúng.
 \end{itemize}
 }
\end{ex}

\Closesolutionfile{ans}

\cauds
\Opensolutionfile{ans}[ans/ans\currfilebase-Phan-II]

\begin{ex}%[2-D3B3-SO-11-2425]%[VN-MT-7, Nguyen Huynh]%[2D3H2-2]
 Bảng bảng biểu diễn mẫu số liệu ghép nhóm về nhiệt độ ($^\circ$C) của tỉnh Nghệ An tháng $5$ năm $2024$.
 \begin{center}
 \begin{tabular}{|c|c|c|c|}
 \hline
 Nhóm & Giá trị đại diện & Tần số & Tần số tích lũy \\
 \hline
 $[29;31)$ & $30$ & $1$ & $1$ \\
 \hline
 $[31;33)$ & $32$ & $4$ & $5$ \\
 \hline
 $[33;35)$ & $34$ & $5$ & $10$ \\
 \hline
 $[35;37)$ & $36$ & $13$ & $26$ \\
 \hline
 $[37;39]$ & $38$ & $7$ & $33$ \\
 \hline
 & & $n=30$ & \\
 \hline
 \end{tabular}
 \end{center}
 \choiceTF
 {\True Nhóm $[31;33)$ có tần số bằng $4$}
 {Mốt của mẫu số liệu ghép nhóm đã cho là $13$ (làm tròn đến hàng phần trăm)}
 {\True Khoảng tứ phân vị của mẫu số liệu ghép nhóm trên bằng $2{,}92$ (làm tròn đến hàng phần trăm)}
 {\True Phương sai của mẫu số liệu ghép nhóm trên bằng $4{,}57$ (làm tròn đến hàng phần trăm)} 
 \loigiai{
 \begin{itemchoice}
 \itemch \textbf{Đúng}. 
 \\Nhóm $[31;33)$ có tần số bằng $4$.
 
 \itemch \textbf{Sai}. 
 \\Ta có nhóm $[35;37)$ có tần số lớn nhất nên mốt của mẫu số liệu trên là \[M_{\text{o}}=u+\dfrac{n_i-n_{i-1}}{2n_i-n_{i-1}-n_{i+1}}\cdot g=35+\dfrac{13-5}{2\cdot 13-5-7}\cdot2=36{,}14.\]
 
 \itemch \textbf{Đúng}. 
 \\Ta có số phần tử của mẫu là $n=30$.\\
 Ta có $\dfrac{n}{4}=7{,}5$ nên nhóm $[33;35)$ là nhóm đầu tiên có tần số tích lũy lớn hơn hoặc bằng $7{,}5$.\\
 Nhóm $[33;35)$ có $s=33;h=2; n=5$ và nhóm $2$ là nhóm $[31;33)$ có $cf_1=5$.\\
 Áp dụng công thức ta có tứ phân vị thứ nhất là $Q_1=33+\dfrac{7{,}5-5}{5}\cdot2=34$ ($^\circ$C).\\
 Ta có $\dfrac{3n}{4}=22{,}5$ nên nhóm $[35;37)$ là nhóm đầu tiên có tần số tích lũy lớn hơn hoặc bằng $22{,}5$.\\
 Xét nhóm $4$ là nhóm $[35;37)$ có $t=35$; $l=2$; $n_4=13$ và nhóm $3$ có tần số tích lũy $cf_4=10$.\\
 Áp dụng công thức, ta có tứ phân vị thứ $3$ là $Q_3=35+\dfrac{22{,}5-10}{13}\cdot2=36{,}92$ ($^\circ$C).\\
 Vậy khoảng tứ phân vị của mẫu số liệu ghép nhóm đã cho là \[\Delta Q=Q_3-Q_1=36{,}92-34=2{,}92.\]
 
 \itemch \textbf{Đúng}. 
 \\Ta có số trụng bình cộng của mẫu số liệu ghép nhóm trên là\\
 \[\overline{x}=\dfrac{1}{30} \left(1\cdot30+32\cdot4+34\cdot5+36\cdot13+38\cdot7\right)=35{,}4\,(^\circ \mathrm C).\]
 Vậy phương sai của mẫu số liệu ghép nhóm trên là 
% \begin{eqnarray*}
% s^2&=&\dfrac{1}{30} \left[1\cdot(30-35{,}4)^2+4\cdot(32-35{,}4)^2+5\cdot(34-35{,}4)^2+13\cdot(36-35{,}4)^2+7\cdot(38-35{,}4)^2\right]\\&\approx&4{,}57.
% \end{eqnarray*}
 \[s^2=\dfrac{1}{30} \left(30^2\cdot1+32^2\cdot4+34^2\cdot5+36^2\cdot13+38^2\cdot7\right)-(35{,}4)^2\approx 4{,}57.\]
 \end{itemchoice}
 }
\end{ex}

\begin{ex}%[2-D3B3-SO-11-2425]%[VN-MT-7, Nguyen Huynh]%[2D3H2-2]
 Cho bảng phân bố tần số ghép lớp cân nặng (đơn vị: kg) của các công nhân trong một công ty như sau 
 \begin{center}
 \begin{tabular}{|c|c|c|c|c|c|c|}
 \hline Cân nặng &$[50;52)$&$[52;54)$&$[54;56)$&$[56;58)$&$[58;60)$& Cộng \\
 \hline Tần số & $15$ & $20$ & $45$ & $15$ & $5$ & $100$ \\
 \hline
 \end{tabular}
 \end{center}
 \choiceTF
 {\True Tần suất của nhóm $[52;54)$ là $20$}
 {Số trung vị của mẫu số liệu lớn hơn $54{,}9$}
 {\True Khoảng biến thiên của mẫu số liệu trên là $10$}
 {Độ lệch chuẩn của mẫu số liệu trên là $4{,}35$}
 \loigiai{
 \begin{itemchoice}
 \itemch \textbf{Đúng}. 
 \\Tần số của nhóm $[52;54)$ là $20$.\\
 Tần suất của nhóm $[52;54)$ là $\dfrac{20}{100}\cdot100\%=20\%$.
 
 \itemch \textbf{Sai}. 
 \\Trung vị của mẫu số liệu là $x_3 \in [54;56)$.\\
 Do đó, trung vị của mẫu số liệu ghép nhóm là\\ $M_{\text{e}}=Q_2=54+\dfrac{\dfrac{2\cdot100}{4}-(15+20)}{45} (56-54)=\dfrac{164}{3}\approx54{,}667$.
 \\Do đó trung vị của mẫu số liệu bé hơn $54{,}9$
 \itemch\textbf{Đúng}. 
 \\Khoảng biến thiên của mẫu số liệu là $R=60-50=10$.
 
 \itemch \textbf{Sai}. 
 \\Số trung bình cộng của mẫu số liệu ghép nhóm của công ty là \[\overline{x}=\dfrac{51\cdot15+53\cdot20+55\cdot45+57\cdot15+59\cdot5}{100}=54{,}5.\]
 Phương sai của mẫu số liệu ghép nhóm của công ty là
% \begin{eqnarray*}
% s^2&=&\dfrac{15\cdot \left(51-54{,}5\right)^2+20\cdot \left(53-54{,}5\right)^2+45\cdot \left(55-54{,}5\right)^2+15\cdot \left(57-54{,}5\right)^2+5\cdot(59-54{,}4)^2}{100}\\&=&4{,}35.
% \end{eqnarray*}
 \[s^2=\dfrac{51^2\cdot15+53^2\cdot20+55^2\cdot45+57^2\cdot15+59^2\cdot5}{100}-(54{,}5)^2=4{,}35.\]
 Độ lệch chuẩn của mẫu số liệu ghép nhóm của công ty là $s=\sqrt{s^2}=\sqrt{4{,}35} \approx 2{,}09$.
 \end{itemchoice}
 }
\end{ex}

\begin{ex}%[2-D3B3-SO-11-2425]%[VN-MT-7, Nguyen Huynh]%[2D3V2-3]
 Cho bảng số liệu dưới đây về thời gian (phút) tập thể dục buổi sáng của hai bạn Bình và Chi trong $30$ ngày.
 \begin{center}
 \begin{tabular}{|c|c|c|c|c|c|}
 \hline
 Thời gian & $[15;20)$ & $[20;25)$ & $[25;30)$ & $[30;35)$ & $[35;40)$ \\
 \hline
 Bạn Bình & $5$ & $8$ & $10$ & $4$ & $3$ \\
 \hline
 Bạn Chi & $10$ & $10$ & $5$ & $3$ & $2$ \\
 \hline
 \end{tabular}
 \end{center}
 \choiceTF
 {\True Khoảng biến thiên của mẫu số liệu ghép nhóm về thời gian tập thể dục của Chi là $25$ (phút)}
 {Tứ phân vị thứ nhất của mẫu số liệu ghép nhóm về thời gian tập thể dục buổi sáng của bạn Bình là $Q_1=\dfrac{354}{16}$}
 {\True Khoảng tứ phân vị của mẫu số liệu ghép nhóm về thời gian tập thể dục buổi sáng của bạn Chi là $28{,}75$}
 {\True Phương sai của mẫu số liệu ghép nhóm về thời gian tập thể dục buổi sáng của bạn Bình là $\dfrac{314}{9}$} 
 \loigiai{
 Ta có \begin{center}
 \begin{tabular}{|c|c|c|c|c|c|}
 \hline
 Thời gian & $\left[15;20\right)$ & $\left[20;25\right)$ & $\left[25;30\right)$ & $\left[30;35\right)$ & $\left[35;40\right)$ \\
 \hline
 Giá trị đại diện & $17{,}5$ & $22{,}5$ & $27{,}5$ & $32{,}5$ & $37{,}5$ \\
 \hline
 Bạn Bình & $5$ & $8$ & $10$ & $10$ & $10$ \\
 \hline
 Bạn Chi & $10$ & $10$ & $5$ & $3$ & $2$ \\
 \hline
 \end{tabular}
 \end{center}
 
 \begin{itemchoice}
 \itemch \textbf{Đúng}. 
 \\Khoảng biến thiên của mẫu số liệu ghép nhóm là $40-15=25$ (phút).
 \itemch \textbf{Sai}. 
 \\Xét số liệu của bạn Bình.\\
 Ta có cỡ mẫu $n=30$.\\
 Vì $\dfrac{n}{4}=\dfrac{30}{4}=7{,}5$ và $5< 7{,}5< 5+8$ nên tứ phân vị thứ nhất thuộc nhóm $[20;25)$.\\
 Tứ phân vị thứ nhất của mẫu số liệu về thời gian tập thể dục buổi sáng của bạn Bình là \[Q_1=20+\dfrac{\dfrac{30}{4}-5}{8}\cdot5=\dfrac{345}{16}.\]
 \itemch \textbf{Đúng}. 
 \\Xét số liệu của bạn Chi.\\
 Ta có cỡ mẫu $n=30$.\\
 Vì $\dfrac{n}{4}=\dfrac{30}{4}=7{,}5$ và $7{,}5< 10$ nên tứ phân vị thứ nhất thuộc nhóm $[15;20)$.\\
 Tứ phân vị thứ nhất của mẫu số liệu về thời gian tập thể dục buổi sáng của bạn Chi là\\ \[Q'_1=15+\dfrac{\dfrac{30}{4}-0}{10} \cdot 5=18{,}75.\]
 Vì $\dfrac{3n}{4}=\dfrac{3 \cdot 30}{4}=22{,}5$ và $10+10< 22{,}5< 10+10+5$. Do đó tứ phân vị thứ ba thuộc nhóm $[25;30)$.\\
 Tứ phân vị thứ ba của mẫu số liệu về thời gian tập thể dục buổi sáng của bạn Chi là \[Q'_3=25+\dfrac{\dfrac{3\cdot30}{4}-(10+10)}{5}\cdot5=27{,}5.\]
 Vậy khoảng tứ phân vị là $\Delta_Q'=Q'_3-Q'_1=28{,}75$.
 
 \itemch \textbf{Đúng}. 
 \\Thời gian trung bình bạn Bình tập thể dục buổi sáng là
 \[\overline{x}=\dfrac{17{,}5\cdot5+22{,}5\cdot8+27{,}5\cdot10+32{,}5\cdot4+37{,}5\cdot3}{30}=\dfrac{157}{6} \approx 26{,}17.\]
 Phương sai của mẫu số liệu ghép nhóm về thời gian tập thể dục buổi sáng của bạn Bình là
 % \begin{align*}
 % s_B^2&=\dfrac{5\left(17{,}5-\dfrac{157}{6} \right)^2+8\left(22{,}5-\dfrac{157}{6} \right)^2+10\left(27{,}5-\dfrac{157}{6} \right)^2+4\left(32{,}5-\dfrac{157}{6} \right)^2}{30}\\
 % &+\dfrac{3\left(37,5-\dfrac{157}{6} \right)^2}{30}\\
 % &=\dfrac{314}{9}.
 % \end{align*}
 \[s_B^2=\dfrac{17{,}5^2\cdot5+22{,}5^2\cdot8+27{,}5^2\cdot10+32{,}5^2\cdot4+37{,}5^2\cdot3}{30}-\left(\dfrac{157}{6}\right)^2 =\dfrac{314}{9}.\]
 \end{itemchoice}
 }
\end{ex}

\begin{ex}%[2-D3B3-SO-11-2425]%[VN-MT-7, Nguyen Huynh]%[2D3V2-3]
 Bảng sau đây cho biết chiều cao của các em học sinh lớp 12A và 12B:
 \begin{center}
 \begin{tabular}{|c|c|c|c|c|c|c|}
 \hline
 Chiều cao (cm) & $[145; 150)$ & $[150; 155)$ & $[155; 160)$ & $[160; 165)$ & $[165; 170)$ & $[170; 175)$ \\
 \hline
 Số học sinh của lớp 12A& $2$ & $1$ & $15$ & $11$ & $9$ & $3$ \\
 \hline
 Số học sinh của lớp 12B & $0$ & $1$ & $16$ & $11$ & $10$ & $4$ \\
 \hline 
 \end{tabular}
 \end{center}
 \choiceTF
 {\True Dựa vào khoảng biến thiên của mẫu số liệu ghép nhóm thì chiều cao của học sinh lớp 12A phân tán hơn lớp 12B}
 {Dựa vào khoảng tứ phân vị của mẫu số liệu ghép nhóm thì học sinh lớp 12A có chiều cao phân tán hơn học sinh lớp 12B}
 {Dựa vào phương sai của mẫu số liệu ghép nhóm thì chiều cao của học sinh lớp 12A ít phân tán hơn học sinh lớp 12B}
 {\True Học sinh lớp 12B có chiều cao đồng đều hơn học sinh lớp 12A vì có độ lệch chuẩn nhỏ hơn} 
 \loigiai{
 \begin{itemchoice}
 \itemch \textbf{Đúng}.\\ Với số liệu của học sinh lớp 12A, có khoảng biến thiên $R_A=175-145=30$.\\
 Với số liệu của học sinh lớp 12B, có khoảng biến thiên $R_B=175-150=25$.\\
 Do $R_A > R_B$ nên chiều cao của học sinh lớp 12A phân tán hơn lớp 12B.
 \itemch \textbf{Sai}.\\Với mẫu số liệu ghép nhóm của học sinh lớp 12A.\\
 Cỡ mẫu là $n=2+1+15+11+9+3=41$.\\
 Gọi $x_1$, $x_2$, $x_3$, $\ldots$, $x_{41}$ là mẫu số liệu gồm chiều cao của học sinh lớp 12A được sắp xếp theo thứ tự không giảm.\\
 Tứ phân vị thứ nhất của mẫu số liệu gốc là $\dfrac{1}{2} (x_{10}+x_{11})\in [155; 160)$ nên nhóm chứa tứ phân vị thứ nhất là nhóm $[155; 160)$. Do đó \[Q_1=155+\dfrac{\dfrac{41}{4}-3}{15}\cdot5\approx 157{,}42.\]
 Tứ phân vị thứ ba của mẫu số liệu gốc là $\dfrac{1}{2} (x_{31}+x_{32})\in [165; 170)$ nên nhóm chứa tứ phân vị thứ ba là nhóm $[165; 170)$. Do đó \[Q_3=165+\dfrac{\dfrac{3\cdot41}{4}-29}{9}\cdot5\approx 165{,}97.\]
 Suy ra $\Delta_{Q_A}=Q_3-Q_1 \approx 8{,}55$.\\
 Với mẫu số liệu ghép nhóm của học sinh lớp 12B\\
 Cỡ mẫu là $n=1+16+11+10+4=42$.
 \\Gọi $x_1$, $x_2$, $x_3$, $\ldots$, $x_{42}$ là mẫu số liệu gồm chiều cao của học sinh lớp 12B được sắp xếp theo thứ tự không giảm.\\
 Tứ phân vị thứ nhất của mẫu số liệu gốc là $x_{11} \in [155; 160)$ nên nhóm chứa tứ phân vị thứ nhất là nhóm $[155; 160)$. Do đó \[Q'_1=155+\dfrac{\dfrac{42}{4}-1}{16}\cdot5\approx 157{,}97.\]
 Tứ phân vị thứ ba của mẫu số liệu gốc là $x_{32} \in [165; 170)$ nên nhóm chứa tứ phân vị thứ ba là nhóm $[165; 170)$. Do đó \[Q'_3=165+\dfrac{\dfrac{3\cdot42}{4}-28}{10}\cdot5=166{,}75.\]
 Suy ra $\Delta_{Q_B}=Q'_3-Q'_1 \approx 8{,}78$.\\
 Do $\Delta_{Q_A} < \Delta_{Q_B}$ nên học sinh lớp 12A có chiều cao phân tán ít hơn học sinh lớp 12B.
 
 \itemch \textbf{Sai}.\\ Chọn giá trị đại diện cho các nhóm số liệu ta được bảng
 \begin{center}
 \begin{tabular}{|c|c|c|c|c|c|c|}
 \hline
 Chiều cao (cm) & $[145; 150)$ & $[150; 155)$ & $[155; 160)$ & $[160; 165)$ & $[165; 170)$ & $[170; 175)$ \\
 \hline
 Giá trị đại diện & $147{,}5$ & $152{,}5$ & $157{,}5$ & $162{,}5$ & $167{,}5$ & $172{,}5$ \\
 \hline
 Số học sinh của lớp 12A & $2$ & $1$ & $15$ & $11$ & $9$ & $3$ \\
 \hline
 Số học sinh của lớp 12B & $0$ & $1$ & $16$ & $11$ & $10$ & $4$ \\
 \hline 
 \end{tabular}
 \end{center}
 Chiều cao trung bình của học sinh lớp 12A là \[\overline{x}_A=\dfrac{1}{41} (2\cdot 147{,}5+1\cdot 152{,}5+15\cdot 157{,}5+11\cdot 162{,}5+9\cdot 167{,}5+3\cdot 172{,}5)\approx 161{,}52.\]
 Chiều cao trung bình của học sinh lớp 12B là \[\overline{x}_B=\dfrac{1}{42} (0\cdot 147{,}5+1\cdot 152{,}5+16\cdot 157{,}5+11\cdot 162{,}5+10\cdot 167{,}5+4\cdot 172{,}5)=162{,}5.\]
 Phương sai của mẫu số liệu lớp 12A là
 \begin{eqnarray*}
 s_A^2&=&\dfrac{1}{41} \left(2\cdot 147{,}5^2+1\cdot 152{,}5^2+15\cdot 157{,}5^2+11\cdot 162{,}5^2+9\cdot 167{,}5^2+3\cdot 172{,}5^2\right)-\left(161{,}52\right)^2\\&\approx& 34{,}41.
 \end{eqnarray*}
 
 Phương sai của mẫu số liệu lớp 12B là
 \begin{eqnarray*}
 s_B^2&=&\dfrac{1}{42} \left(0\cdot 147{,}5^2+1\cdot 152{,}5^2+16\cdot 157{,}5^2+11\cdot 162{,}5^2+10\cdot 167{,}5^2+4\cdot 172{,}5^2\right)-\left(162{,}50\right)^2\\&\approx& 27{,}38.
 \end{eqnarray*}
 Vậy dựa vào phương sai của mẫu số liệu ghép nhóm thì chiều cao của học sinh lớp 12A phân tán hơn học sinh lớp 12B.
 
 \itemch \textbf{Đúng}. 
 \\Độ lệch chuẩn của mẫu số liệu lớp 12A là $s_A \approx \sqrt{35{,}83} \approx 5{,}99$.\\
 Độ lệch chuẩn của mẫu số liệu lớp 12B là $s_B \approx \sqrt{27{,}38} \approx 5{,}23$.\\
 Vậy học sinh lớp 12B có chiều cao đồng đều hơn học sinh lớp 12A vì có độ lệch chuẩn nhỏ hơn.
 \end{itemchoice}
 }
\end{ex}
\Closesolutionfile{ans}

\caukq
\Opensolutionfile{ans}[ans/ans\currfilebase-Phan-III]
\begin{ex}%[2-D3B3-SO-11-2425]%[VN-MT-7, Nguyen Huynh]%[2D3N1-2]%[Câu 1]
 Bảng sau đây cho biết chiều cao (đơn vị: cm) của học sinh lớp 5A:
 \begin{center}
 \begin{tabular}{|c|c|c|c|c|c|c|}
 \hline
 Chiều cao & $[85; 90)$ & $[90; 95)$ & $[95; 100)$ & $[100; 105)$ & $[105; 110)$& $[110; 115)$\\
 \hline Tần số & $1$ & $4$ & $8$ & $12$ & $3$ & $2$ \\ \hline
 \end{tabular}
 \end{center}
 Tìm khoảng biến thiên của mẫu số liệu ghép nhóm về chiều cao của học sinh lớp 5A.
 
 \shortans[]{30}
 \loigiai{
 Khoảng biến thiên của mẫu số liệu ghép nhóm trên là $R=115-85=30$.
 }
\end{ex}

\begin{ex}%[2-D3B3-SO-11-2425]%[VN-MT-7, Nguyen Huynh]%[2D3H1-3]
 Bảng sau đây cho biết chiều cao (đơn vị: cm) của học sinh lớp 5A:
 \begin{center}
 \begin{tabular}{|c|c|c|c|c|c|c|}
 \hline
 Chiều cao & $[85; 90)$ & $[90; 95)$ & $[95; 100)$ & $[100; 105)$ & $[105; 110)$& $[110; 115)$\\
 \hline 
 Tần số & $1$ & $4$ & $8$ & $12$ & $3$ & $2$ \\ 
 \hline
 \end{tabular}
 \end{center}
 Tìm khoảng tứ phân vị của mẫu số liệu ghép nhóm về chiều cao của học sinh lớp 5A (làm tròn đến hàng phần mười).
 
 \shortans[]{7{,}4}
 \loigiai{
 Cỡ mẫu là $n=1+4+8+12+3+2=30$.\\
 Gọi $x_1$; $x_2$; $\ldots$; $x_{30}$ là mẫu số liệu gốc về chiều cao của học sinh lớp 5A được sắp xếp theo thứ tự không giảm.\\
 Khi đó 
 $x_1 \in [85;90)$;
 $x_2$, $x_3$, $x_4$, $x_5\in [90;95)$;
 $x_6$, $\ldots$, $x_{13} \in [95;100)$;
 $x_{14}$, $\ldots$, $x_{25} \in [100;105)$;
 $x_{26}$, $x_{27}$, $x_{28} \in [105;110)$;
 $x_{29}$, $x_{30} \in [110;115)$;\\
 Tứ phân vị thứ nhất của mẫu số liệu gốc là $x_{8} \in [95;100)$.\\
 Do đó tứ phân vị thứ nhất của mẫu số liệu ghép nhóm là 
 \[Q_1=95+\dfrac{\dfrac{30}{4}-5}{8} \cdot (100-95)\approx 96{,}56.\]
 Tứ phân vị thứ ba của mẫu số liệu gốc là $x_{23} \in [100;105)$.\\ 
 Do đó tứ phân vị thứ ba của mẫu số liệu ghép nhóm là 
 \[Q_3=100+\dfrac{\dfrac{3\cdot 30}{4}-13}{12} \cdot (105-100)\approx 103{,}96.\]
 Vậy khoảng tứ phân vị của mẫu số liệu trên là $\Delta_Q=Q_3-Q_1\approx 103{,}96-96{,}56=7{,}4$.}
\end{ex}

\begin{ex}%[2-D3B3-SO-11-2425]%[VN-MT-7, Nguyen Huynh]%[2D3H2-2]
 Số người xem trong $60$ buổi chiếu phim của một rạp chiếu phim nhỏ.
 \begin{center}
 \begin{tabular}{|c|c|c|c|c|c|c|c|}
 \hline
 Lớp người xem & $[0; 10)$&$[10; 20)$&$[20; 30)$&$[30; 40)$&$[40; 50)$&$[50; 60]$&Cộng\\
 \hline
 Tần số & $5$&$9$&$11$&$15$&$12$&$8$&$60$\\
 \hline
 \end{tabular}
 \end{center}
 Hãy tính phương sai của mẫu số liệu ghép nhóm trên (kết quả được làm tròn đến hàng đơn vị). 
 \par\shortans[]{220}
 \loigiai{
 \begin{center}
 \begin{tabular}{|c|c|c|c|c|c|c|c|}
 \hline
 Lớp người xem & $[0; 10)$&$[10; 20)$&$[20; 30)$&$[30; 40)$&$[40; 50)$&$[50; 60]$&Cộng\\
 \hline
 Giá trị đại diện & $5$ & $15$ & $25$ & $35$ & $45$ & $55$ & \\
 \hline
 Tần số & $5$&$9$&$11$&$15$&$12$&$8$&$60$\\
 \hline
 \end{tabular}
 \end{center}
 Số trung bình của mẫu số liệu ghép nhóm là
 \begin{eqnarray*}
 \overline{x}&=&\dfrac{n_1c_1+n_2c_2+n_3c_3+n_4c_4+n_5c_5+n_6c_6}{n}\\
 &=&\dfrac{5 \cdot 5+9 \cdot 15+11 \cdot 25+15 \cdot 35+12 \cdot 45+8 \cdot 55}{60}\\
 &=&\dfrac{97}{3}\approx 32{,}3.
 \end{eqnarray*}
 Phương sai của mẫu số liệu ghép nhóm là
% \begin{eqnarray*}
% s_x^2&=&\dfrac{n_1(c_1-\overline{x})^2+n_2(c_2-\overline{x})^2+n_3(c_3-\overline{x})^2+n_4(c_4-\overline{x})^2+n_5(c_5-\overline{x})^2+n_6(c_6-\overline{x})^2}{n}\\
% &=&\dfrac{5(5-32{,}3)^2+9(15-32{,}3)^2+11(25-32{,}3)^2+15(35-32{,}3)^2}{60}+\\
% && +\dfrac{12(45-32{,}3)^2+8(55-32{,}3)^2}{60}\\
% &\approx& 220.
% \end{eqnarray*} 
 \[s_x^2=\dfrac{5 \cdot 5^2+9 \cdot 15^2+11 \cdot 25^2+15 \cdot 35^2+12 \cdot 45^2+8 \cdot 55^2}{60}-\left(\dfrac{97}{3}\right)^2 \approx 220.\] 
 }
\end{ex}

\begin{ex}%[2-D3B3-SO-11-2425]%[VN-MT-7, Nguyen Huynh]%[2D3H2-2]
 Bảng thống kê cự li ném tạ của một vận động viên như sau:
 \begin{center}
 \begin{tabular}{|c|c|c|c|c|c|}
 \hline
 Cự li (m) & $[19; 19{,}5)$&$[19{,}5; 20)$&$[20; 20{,}5)$&$[20{,}5; 21)$&$[21; 21{,}5)$\\
 \hline
 Tần số & $13$ & $45$ & $24$ & $12$ & $6$\\
 \hline
 \end{tabular}
 \end{center} 
 Hãy tính độ lệch chuẩn của mẫu số liệu ghép nhóm trên (kết quả được làm tròn đến hàng phần trăm).
 
 \shortans[]{0{,}53}
 \loigiai{
 \begin{center}
 \begin{tabular}{|c|c|c|c|c|c|}
 \hline
 Cự li (m) & $[19; 19{,}5)$&$[19{,}5; 20)$&$[20; 20{,}5)$&$[20{,}5; 21)$&$[21; 21{,}5)$\\
 \hline
 Giá trị đại diện & $19{,}25$ & $19{,}75$ & $20{,}25$ &$20{,}75$ &$21{,}25$ \\
 \hline
 Tần số & $13$ & $45$ & $24$ & $12$ & $6$\\
 \hline
 \end{tabular}
 \end{center}
 Cỡ mẫu $n=100$.
 \\Số trung bình \[\overline{x}=\dfrac{13 \cdot 19{,}25+45 \cdot 19{,}75+24 \cdot 20{,}25+12 \cdot 20{,}75+6 \cdot 21{,}25}{100}=20{,}015\]
 Phương sai
 \begin{eqnarray*}
 s^2&=&\dfrac{13 \cdot \left(19{,}25-20{,}015\right)^2+45 \cdot \left(19{,}75-20{,}015\right)^2+24 \cdot \left(20{,}25-20{,}015\right)^2}{100}\\
 && +\dfrac{12 \cdot \left(20{,}75-20{,}015\right)^2+6 \cdot \left(21{,}25-20{,}015\right)^2}{100} \\
 &\approx& 0{,}28. 
 \end{eqnarray*}
 Độ lệch chuẩn $s=\sqrt{0{,}28} \approx 0{,}53$.
 }
\end{ex}

\begin{ex}%[2-D3B3-SO-11-2425]%[VN-MT-7, Nguyen Huynh]%[2D3H2-3]
 Thành tích môn nhảy cao của các vận động viên tại một giải điền kinh dành cho học sinh trung học phổ thông như sau:
 \begin{center}
 \begin{tabular}{|c|c|c|c|c|}
 \hline
 Mức xà (cm) & $[170;172)$ & $[172;174)$ & $[174;176)$ & $[176;180)$ \\ \hline
 Số vận động viên & $3$ & $10$ & $6$ & $1$ \\ \hline
 \end{tabular}
 \end{center}
 Tính khoảng tứ phân vị $\Delta_Q$ và độ lệch chuẩn $s$ của mẫu số liệu ghép nhóm trên. Khi đó $\Delta_Q+s$ bằng bao nhiêu (kết quả làm tròn đến hàng phần trăm)?
 \par\shortans[]{3{,}96}
 \loigiai{
 Cỡ mẫu là $n=3+10+6+1=20$.\\
 Gọi $x_1$, $x_2$, $\ldots$, $x_{20}$ là mức xà của $20$ vận động viên được sắp xếp theo thứ tự tăng dần.\\
 Tứ phân vị thứ nhất của mẫu số liệu gốc là $\dfrac{x_5+x_6}{2}\in [172; 174)$. \\
 Do đó, tứ phân vị thứ nhất của mẫu số liệu ghép nhóm là \[Q_1=172+\dfrac{\dfrac{20}{4}-3}{10}\cdot(174-172)=172{,}4.\]
 Tứ phân vị thứ ba của mẫu số liệu gốc là $\dfrac{x_{15}+x_{16}}{2}\in [174; 176)$. \\
 Do đó, tứ phân vị thứ ba của mẫu số liệu ghép nhóm là \[Q_3=174+\dfrac{\dfrac{3\cdot20}{4}-13}{6}\cdot(176-174)\approx 174{,}7.\]
 Do đó khoảng tứ phân vị là $\Delta_Q=Q_3-Q_1\approx 174{,}7-172{,}4\approx 2{,}3$.\\
 Chọn giá trị đại diện cho mẫu số liệu ta có
 \begin{center}
 \begin{tabular}{|c|c|c|c|c|}
 \hline
 Mức xà (cm) &$[170; 172)$ &$[172; 174)$ &$[174; 176)$ &$[176; 180)$\\
 \hline
 Giá trị đại diện & $171$ & $173$ & $175$ & $178$\\
 \hline
 Số vận động viên & $3$ & $10$ & $6$ & $1$\\
 \hline
 \end{tabular}
 \end{center}
 Mức xà trung bình là $\overline{x}=\dfrac{3\cdot171+10\cdot173+6\cdot175+1\cdot178}{20}=173{,}55$.\\
 Phương sai và độ lệch chuẩn
 \[S^2=\dfrac{1}{20}\left(3\cdot171^2+10\cdot173^2+6\cdot175^2+1\cdot178^2\right)-(173{,}55)^2\approx 2{,}75.\]
 Suy ra $s=\sqrt{s^2}=\sqrt{2{,}75}\approx 1{,}66$. 
 Khi đó $\Delta_Q+s\approx 3{,}96$.
 }
\end{ex}

\begin{ex}%[2-D3B3-SO-11-2425]%[VN-MT-7, Nguyen Huynh]%[2D3H1-3]
 Một người ghi lại thời gian đàm thoại của một số cuộc gọi cho kết quả như bảng sau:
 \begin{center}
 \begin{tabular}{|c|c|}
 \hline
 Thời gian (phút) & Số cuộc gọi \\
 \hline
 $0\le t<1$ & $8$ \\
 \hline
 $1\le t<2$ & $17$ \\
 \hline
 $2\le t<3$ & $25$ \\
 \hline
 $3\le t<4$ & $20$ \\
 \hline
 $4\le t<5$ & $10$ \\
 \hline
 \end{tabular}
 \end{center}
 Tìm khoảng tứ phân vị của mẫu số liệu ghép nhóm trên (làm tròn đến hàng phần mười).
 \par\shortans[]{1{,}8}
 \loigiai{ 
 Ta có bảng mẫu số liệu ghép nhóm được viết lại như sau:
 \begin{center}
 \begin{tabular}{|c|c|c|c|c|c|}
 \hline
 Thời gian t (phút) & $[0;1)$ & $[1; 2)$ & $[2; 3)$ & $[3; 4)$ & $[4; 5)$ \\
 \hline
 Số cuộc gọi & $8$ & $17$ & $25$ & $20$ & $10$ \\
 \hline
 \end{tabular}
 \end{center}
 Có cỡ mẫu $n=8+17+25+20+10=80$.\\
 Giả sử $x_1$, $x_2$, $\ldots$, $x_{80}$ là thời gian đàm thoại của $80$ cuộc gọi được sắp xếp theo thứ tự tăng dần.\\
 Tứ phân vị thứ nhất của mẫu số liệu gốc là $\dfrac{x_{20}+x_{21}}{2} \in [1; 2)$ nên nhóm chứa tứ phân vị thứ nhất là $[1; 2)$.
 \[Q_1=1+\dfrac{\dfrac{80}{4}-8}{17} \left(2-1\right)\approx 1{,}7.\]
 Tứ phân vị thứ ba của mẫu số liệu gốc là $\dfrac{x_{60}+x_{61}}{2} \in [3; 4)$ nên nhóm chứa tứ phân vị thứ ba là $[3; 4)$.
 \[Q_3=3+\dfrac{\dfrac{3\cdot 80}{4}-50}{20} \left(4-3\right)=3{,}5.\]
 Khoảng tứ phân vị của mẫu số liệu ghép nhóm là $\Delta_Q=Q_3-Q_1 \approx 1{,}8$. 
 }
\end{ex}
\Closesolutionfile{ans}
\begin{indapan}
	{ans/ans\currfilebase}
\end{indapan}



%%GK2
% \begin{name}
	{\tenchude}
	{TOÁN 12}
	{LỚP TOÁN THẦY PHÁT}
	{Thời gian: 90 phút - Không kể thời gian phát đề}
\end{name}
\TN
\Opensolutionfile{ans}[ans/ansDe1-TN1]
\begin{ex}%[2D4N1-1]%[Dự án EX-TF-TLN lần 3 -Mui Doan]
	Nguyên hàm của hàm số $f(x)=x^n$ (với $n\neq -1$) là
	\choice
	{\True $\dfrac{x^{n+1}}{n+1}+C$}
	{$x^{n+1}+C$}
	{$\dfrac{n+1}{x^{n+1}}+C$}
	{$\dfrac{1}{x^{n+1}}+C$}
	\loigiai{
		Nguyên hàm của hàm số $f(x)=x^n$ (với $n\neq -1$) là	$\dfrac{x^{n+1}}{n+1}+C$.
	}
\end{ex}

\begin{ex}%[2D4N1-1]%[Tổ 19 - Đợt 17 - Chương 4 - - CD - Đề 2]%[Bình]
	Hàm số $y=F(x)$ là một nguyên hàm của hàm số $y=f(x)$. Hãy chọn khẳng định \textbf{đúng}.
	\choice
	{$F(x)=f'(x)$}
	{\True $F'(x)=f(x)$}
	{$F(x)=f'(x)+C$}
	{$F'(x)+C=f(x)$}
	\loigiai
	{
		Khẳng định đúng là: $F'(x)=f(x)$.
	}
\end{ex}

\begin{ex}%[2D4N1-2]
	Họ nguyên hàm của hàm số $f(x)=3x^2+1$ là
	\choice
	{$x^3+C$}
	{$\dfrac{x^3}{3}+x+C$}
	{$6x+C$}
	{\True $x^3+x+C$}
	\loigiai{
		$\displaystyle\int{(3x^2+1)\mathrm{\,d}x=x^3+x+C}$.}
\end{ex}

\begin{ex}%[12-MH-2-MH2025]%[MH-2025,Chu Hà]%[2D4N1-3]
	Họ nguyên hàm của hàm số $f(x)= \dfrac{1}{\sqrt{x}} $ là
	\choice
	{\True $2\sqrt{x}+C$}
	{$\sqrt{x} + C$}
	{$- \sqrt{x} +C$}
	{$-2\sqrt{x}+C$}
	\loigiai
	{
	Ta có $\displaystyle \int\dfrac{1}{\sqrt{x}}\mathrm{\,d}x=\int x^{-\tfrac{1}{2}}\mathrm{\,d}x=2x^{\tfrac{1}{2}}+C=2\sqrt{x}+C$.
	}
\end{ex}

\begin{ex}%[2D4N1-4]
	Phát biểu nào sau đây là đúng?
	\choice
	{$\displaystyle\int e^{-3 x} \mathrm{~d} x=e^{-3 x}+C$}
	{\True $\displaystyle\int e^{-3 x} \mathrm{~d} x=-\frac{1}{3} e^{-3 x}+C$}
	{$\displaystyle\int e^{-3 x} \mathrm{~d} x=\frac{1}{3} e^{-3 x}+C$}
	{$\displaystyle\int e^{-3 x} \mathrm{~d} x=-\frac{1}{3} e^{-3 x}$}
	\loigiai{$\displaystyle\int e^{-3 x} \mathrm{~d} x=-\frac{1}{3} e^{-3 x}+C$.}
\end{ex}

\begin{ex}%[Dự án 2025 - đề cấu trúc mới, Nguyễn Kiều Nhã Tú]%[2D4N1-4]
	Họ nguyên hàm của hàm số $f(x)=\dfrac{1}{x^2}+2^x$ là
	\choice
	{$\ln x^2+2^x \cdot \ln 2+C$}
	{$\ln x^2+\dfrac{2^x}{\ln 2}+C$}
	{\True $-\dfrac{1}{x}+\dfrac{2^x}{\ln 2}+C$}
	{$\dfrac{1}{x}+2^x \cdot \ln 2+C$}
	\loigiai{
		Ta có: $\displaystyle\int\left(\dfrac{1}{x^2}+2^x\right)\mathrm{\,d} x=-\dfrac{1}{x}+\dfrac{2^x}{\ln 2}+C$.
	}
\end{ex}

\begin{ex}%[12-MH-2-MH2025]%[MH-2025, Nguyễn Trần Phong]%[2D4H1-2]
	Tìm một nguyên hàm $F(x)$ của hàm số $ f(x)=3 x^{2} + 5$ biết $F(6) =241$.
	\choice
	{ $ F(x)=x^{3} + 10 x + 15$ }
	{ \True $ F(x)=x^{3} + 5 x - 5$ }
	{ $ F(x)=x^{3} + 5 x - 25$ }
	{ $ F(x)=x^{3} + 8 x^{2} + 5 x - 5$ }
	\loigiai{
	$\displaystyle \int {\left(3 x^{2} + 5\right)\mathrm{\,d}x} = x^{3} + 5 x + C$.\\
	Mà $F(6) =241 \Leftrightarrow 6^3 + 5 \cdot 6 + C =241 \Leftrightarrow C = -5 $.\\
	Vậy $ F(x)=x^{3} + 5 x - 5$.
	}
\end{ex}

\begin{ex}%[2D4H1-3]
	$\displaystyle \int \left(\sin \dfrac{x}{2}+\cos\dfrac{x}{2} \right)^2 \mathrm{d}x$ bằng
	\choice
	{\True $x-\cos x+C$}
	{$\left(-\cos \dfrac{x}{2}+\sin\dfrac{x}{2} \right)^2$}
	{$\dfrac{1}{3} \left(\sin \dfrac{x}{2}+\cos\dfrac{x}{2}\right)^3$}
	{$x+\cos x+C$}
	\loigiai{
		Ta có
		\allowdisplaybreaks
		\begin{eqnarray*}
			\displaystyle \int \left(\sin \dfrac{x}{2}+\cos\dfrac{x}{2} \right)^2 \mathrm{d}x &=& \displaystyle \int \left(\sin^2 \dfrac{x}{2}+\cos^2\dfrac{x}{2}+2\sin \dfrac{x}{2}\cos \dfrac{x}{2} \right) \mathrm{d}x\\
			&=& \displaystyle \int \left(1+\sin x \right) \mathrm{d}x\\
			&=& \displaystyle \int \mathrm{d}x+\displaystyle \int \sin x \mathrm{d}x\\
			&=& x-\cos x+C.
		\end{eqnarray*}
	}
\end{ex}

\begin{ex}%[Cau-2]%[2D4N2-1]
	Cho hàm số $f(x)$ liên tục trên $\mathbb{R}$ và $F(x)$ là nguyên hàm của $f(x)$, biết $\displaystyle \int\limits_{0}^{9} f(x)\mathrm{\,d}x=9$ và $F(0)=3$. Tính $F(9)$.
	\choice
	{$F(9)=-6$}
	{$F(9)=6$}
	{\True $F(9)=12$}
	{$F(9)=-12$}
	\loigiai{
		Ta có $I=\displaystyle\int\limits_{0}^{9} f(x)\mathrm{\,d}x = F(x)\Big|_0^9 = F(9)- F(0)=9\Leftrightarrow F(9)=9 + F(0)=9 + 3= 12$.
	}
\end{ex}

\begin{ex}%[2D4N2-2]
	Tích phân $\displaystyle\int\limits_{1}^{2} x^3\mathrm{\,d}x$ bằng
	\choice{$\dfrac{15}{3}$}{$\dfrac{17}{4}$}{$\dfrac{7}{4}$}{\True$\dfrac{15}{4}$}
	\loigiai{
		Ta có $\displaystyle\int\limits_{1}^{2} x^3\mathrm{\,d}x=\dfrac{x^4}{4}\Big|_1^2=\dfrac{15}{4}$.
	}
\end{ex}

\begin{ex}%[2D4N2-3]
	Tính tích phân $\displaystyle\int\limits_0^\pi \sin 3 x \mathrm{\,d} x$.
	\choice
	{$-\dfrac{1}{3}$}
	{$\dfrac{1}{3}$}
	{$-\dfrac{2}{3}$}
	{\True $\dfrac{2}{3}$}
	\loigiai{Ta có $\displaystyle\int\limits_0^\pi \sin 3 x \mathrm{\,d} x=-\left.\dfrac{1}{3} \cos 3 x\right|_0 ^\pi=-\dfrac{1}{3}(-1-1)=\dfrac{2}{3}$.}
\end{ex}

\begin{ex}%[Tổ 20 - Chương 4 - - CD]%[Nguyễn Văn Sang]%[2D4N2-4]
	Tính giá trị tích phân $\displaystyle\int\limits_1^3 3\cdot 5^x \mathrm{~d}x$.
	\choice
	{$\dfrac{\mathrm{e}^2}{\ln 5}$ }
	{\True $\dfrac{360}{\ln 5}$ }
	{$\dfrac{\mathrm{e}^3-\mathrm{e}}{\ln 5}$ }
	{$\dfrac{320}{\ln 5}$ }
	\loigiai{
		Ta có $\displaystyle\int\limits_1^3 3\cdot 5^x \mathrm{~d}x=3 \displaystyle\int\limits_1^3 5^x \mathrm{~d}x=\dfrac{3\cdot 5^x}{\ln 5}\bigg|_1 ^3=\dfrac{360}{\ln 5}$.
	}
\end{ex}
\Closesolutionfile{ans}

\TNTF
\Opensolutionfile{ans}[ans/ansDe1-TN2]
\begin{ex}%[Dự án EX - TF - TLN 2024]%[Doan Hung]%[2D4H1-2]
	Cho hàm số $F(x)=x^3-2x+1$, $x \in \mathbb{R}$ là một nguyên hàm của hàm số $f(x)$.
	\choiceTF
	{Nếu hàm số $G(x)$ cũng là một nguyên hàm của hàm số $f(x)$ và $G(-1)=3$ thì $G(x)=F(x)-1, x \in \mathbb{R}$}
	{\True Nếu hàm số $H(x)$ cũng là một nguyên hàm của hàm số $f(x)$ và $H(1)=-3$ thì $H(x)=F(x)-3, x \in \mathbb{R}$}
	{Nếu hàm số $K(x)$ cũng là một nguyên hàm của hàm số $f(x)$ và $K(0)=0$ thì $K(x)=F(x)+1, x \in \mathbb{R}$}
	{\True Nếu hàm số $M(x)$ cũng là một nguyên hàm của hàm số $f(x)$ và $M(2)=4$ thì $M(x)=F(x)-1, x \in \mathbb{R}$}
	\loigiai{
	\begin{itemchoice}
	\itemch Vì hàm số $G(x)$ cũng là một nguyên hàm của hàm số $f(x)$ nên $G(x)=F(x)+C$.\\
	Vì $G(-1)=3\Leftrightarrow F(-1)+C=3\Leftrightarrow C=3-F(-1)\Leftrightarrow C=3-2=1$.\\
	Vậy $G(x)=F(x)+1$.
	\itemch Vì hàm số $H(x)$ cũng là một nguyên hàm của hàm số $f(x)$ nên $H(x)=F(x)+C$.\\
	Vì $H(1)=-3\Leftrightarrow F(1)+C=-3\Leftrightarrow C=-3-F(1)\Leftrightarrow C=-3-0=-3$.\\
	Vậy $H(x)=F(x)-3$.
	\itemch Vì hàm số $K(x)$ cũng là một nguyên hàm của hàm số $f(x)$ nên $K(x)=F(x)+C$.\\
	Vì $K(0)=0\Leftrightarrow F(0)+C=0\Leftrightarrow C=-F(0)\Leftrightarrow C=-1$.\\
	Vậy $H(x)=F(x)-1$.
	\itemch Vì hàm số $M(x)$ cũng là một nguyên hàm của hàm số $f(x)$ nên $M(x)=F(x)+C$.\\
	Vì $M(2)=4\Leftrightarrow F(2)+C=4\Leftrightarrow C=4-F(2)\Leftrightarrow C=-1$.\\
	Vậy $H(x)=F(x)-1$.
	\end{itemchoice}
	}
	\end{ex}

\begin{ex}%[2D4N2-2]
	Cho hàm số $f(x)=x^2$.
	\choiceTF
	{\True $\displaystyle\int f(x)\mathrm{\,d}x=\dfrac{x^3}{3}+C$}
	{$\displaystyle\int\limits_0^2 f(x)\mathrm{\,d}x=\dfrac{7}{3}$}
	{Giả sử $F(x)$ là một nguyên hàm của $f(x)$. Khi đó $f'(x)=F(x)$}
	{\True Gọi $F(x)$ là một nguyên hàm của $f(x)$. Nếu đồ thị hàm số của $F(x)$ đi qua điểm $(3;1)$ thì $F(x)=\dfrac{x^3}{3}-8$.}
	\loigiai{
		\begin{itemchoice}
			\itemch  $\displaystyle\int x^2\mathrm{\,d}x=\dfrac{x^3}{3}+C$.\\
			\itemch $\displaystyle\int\limits_0^2 x^2\mathrm{\,d}x=\dfrac{x^3}{3}\Big|_0^2=\dfrac{8}{3}-0=\dfrac{8}{3}$.\\
			\itemch  $F(x)$ là nguyên hàm của $f(x)$ thì $F'(x)=f(x)$.\\
			\itemch Nguyên hàm $F(x)=\dfrac{x^3}{3}+C$. Mà $F(x)$ đi qua  $(3;1)$ nên $C=-8$.
		\end{itemchoice}
	}
\end{ex}
\Closesolutionfile{ans}

\TNSA
\Opensolutionfile{ans}[ans/ansDe1-TN3]
\begin{ex}%[2D4H1-1]%[Đào Trung Kiên]
	Biết $ F(x) $ là một nguyên hàm của hàm số $ f(x) = \mathrm{e}^{2x} $ và $ F(0) = 0$. Tính giá trị của $F(\ln 3)$.
	\shortans[]{$4$}
	\loigiai{
		Ta có $ \heva{& F(0) = 0 \\ & F(x) = \dfrac{1}{2} \cdot \mathrm{e}^{2x} + C } \Rightarrow F(x) = \dfrac{1}{2} \cdot \mathrm{e}^{2x} - \dfrac{1}{2} \Rightarrow F(\ln 3) =  \dfrac{1}{2}  \cdot \left  (  \mathrm{e}^{ 2 \cdot \ln 3 } - 1 \right ) = 4$.
	}
\end{ex}

\begin{ex}%[2D4H2-2]%[Tổ 20 - Đợt 17 - Chương 4 - - CD - Đề 7]%[Lê Thị Thanh Tuyền]
	Biết rằng $\displaystyle\int\limits_{-1}^3[2f(x)-3g(x)] \mathrm{\,d} x=10$ và $\displaystyle\int\limits_{-1}^3[3f(x)+g(x)] \mathrm{\,d} x=4$.
	Tích phân $\displaystyle\int\limits_{-1}^3[10f(x)+7g(x)] \mathrm{\,d} x$ bằng
	\shortans{$6$}

	\loigiai{

	Đặt $a=\displaystyle\int\limits_{-1}^3f(x) \mathrm{\,d} x,\quad b=\displaystyle\int\limits_{-1}^3g(x) \mathrm{\,d} x$.\\
	Ta có $\displaystyle\int\limits_{-1}^3[2f(x)-3g(x)] \mathrm{\,d} x=10$ và $\displaystyle\int\limits_{-1}^3[3f(x)+g(x)]\mathrm{\,d} x=4$.\\
	Suy ra  $\heva{&2a-3b=10\\& 3a+b=4} \Leftrightarrow\heva{&a=2\\&b=-2.}$
	\\
	Vậy $\displaystyle\int\limits_{-1}^3[10f(x)+7g(x)] \mathrm{\,d} x=10a+7b=6$.
	}
\end{ex}

\begin{ex}%[ST12-dot1-Trần xuân Hòa]%[2D4H3-1]
	Tính diện tích hình phẳng giới hạn bởi đồ thị các hàm số $y=-x^2+2x$ và $y=-3x$. (Kết quả làm tròn đến chữ số thập phân thứ nhất).
	\shortans{$20{,}8$}
	\loigiai
	{Phương trình hoành độ giao điểm $-x^2+2x=-3x\Leftrightarrow\hoac{&x=0\\&x=5.}$\\
		Khi đó diện tích $S$ của hình phẳng được xác định bởi
		\begin{eqnarray*}
			&S&=\displaystyle\int \limits_0^5|-x^2+2x+3x|\mathrm{\, d}x\\
			&&=\displaystyle\int\limits_0^5|-x^2+5x|\mathrm{\, d}x\\
			&&=\left| \displaystyle\int\limits_0^5(-x^2+5x)\mathrm{\, d}x\right|\\
			&&=\left| \left(-\dfrac{x^3}{3}+\dfrac{5x^2}{2}\right)\Bigg|_0^5\right| =\dfrac{125}{6}\approx 20{,}8.
		\end{eqnarray*}
	}
\end{ex}

\begin{ex}%[Vovanle]%[2D4H3-3]
	Cho hình phẳng giới hạn bởi các đường $y=\sqrt{x}-2$, $y=0$ và $x=9$ quay xung quanh trục $Ox$. Tính thể tích khối tròn xoay tạo thành (làm tròn kết quả thể tích đến hàng phần trăm).
	\shortans{$5{,}76$}
	\loigiai{
		Phương trình hoành độ giao điểm của đồ thị hàm số $y=\sqrt{x}-2$ và trục hoành
		\[\sqrt{x}-2=0\Leftrightarrow \sqrt{x}=2\Leftrightarrow x=4.\]
		Thể tích của khối tròn xoay tạo thành là
		\allowdisplaybreaks
		\begin{eqnarray*}
			V&=&\pi \displaystyle\int\limits_4^{9}{{{\left(\sqrt{x}-2\right)}^2}\mathrm{\,d}x}\\
			&=&\pi\displaystyle\int\limits_4^{9}{\left(x-4\sqrt{x}+4\right)}\mathrm{\,d}x\\
			&=&\pi\left.\left(\dfrac{x^2}2-\dfrac{8x\sqrt{x}}3+4x\right)\right|_4^{9}\\
			&=&\pi\left(\dfrac{81}{2}-72+36\right)-\pi\left(\dfrac{16}{2}-\dfrac{64}{3}+16\right)\\
			&=&\dfrac{11\pi}{6}\approx 5{,}76.
		\end{eqnarray*}
	}
\end{ex}

\Closesolutionfile{ans}

\TL
\begin{ex}%[2D4H2-2]
	Cho số thực $a>1$, tính tích phân
	$\displaystyle\int\limits_0^a {|x-1|}\mathrm{\,d}x$ theo $ a$.
	\loigiai{
		Ta có hàm số $ f(x)=x-1$ có một nguyên hàm $F(x)=\dfrac{x^2}{2}-x$.\\
		$f(x)=|x-1|=\heva{x-1 & \text{ nếu } x\ge 1\\ -(x-1) & \text{ nếu } x<1.}$\\
		\begin{eqnarray*}
			\text{Ta lại có } \displaystyle\int\limits_0^a {|x-1|}\mathrm{\,d}x
			&=& \displaystyle\int\limits_0^1 {|x-1|}\mathrm{\,d}x+
			\displaystyle\int\limits_1^a {|x-1|}\mathrm{\,d}x\\
			&=& \displaystyle\int\limits_0^1 {-(x-1)}\mathrm{\,d}x+
			\displaystyle\int\limits_1^a {(x-1)}\mathrm{\,d}x\\
			&=& F(0) - F(1) + F(a) - F(1) = \dfrac{a^2}{2}-a+1.
		\end{eqnarray*}
	}
\end{ex}

\begin{ex}%[2D4V1-4]
	$F(x)$ là một nguyên hàm của hàm số $f(x)=2^x$, thỏa mãn $ F(0)=\dfrac{1}{\ln 2}$. Biểu thức $ F(0)+F(1)+F(2)+\ldots+F(2024)=\dfrac{a^b-c}{\ln a}$ $(a$, $b$, $c\in{N^*})$. Tính $ T=a+b-2c$.
	% \shortans{2025}
	\loigiai{
		Ta có $F(x)=\displaystyle\int 2^x \mathrm{\,d}x=\dfrac{2^x}{\ln 2}+C$.\\
		Theo giả thiết $F(0)=\dfrac{1}{\ln 2}\Leftrightarrow\dfrac{2^0}{\ln 2}+C=\dfrac{1}{\ln 2}\Leftrightarrow C=0 \Rightarrow F(x)=\dfrac{2^x}{\ln 2}$.\\
		Khi đó
		\allowdisplaybreaks
		\begin{eqnarray*}
			F(0)+F(1)+F(2)+\ldots+F(2024)&=&\dfrac{2^0}{\ln 2}+\dfrac{2^1}{\ln 2}+\dfrac{2^2}{\ln 2}+\ldots+\dfrac{2^{2024}}{\ln 2}\\
			&=&\dfrac{1}{\ln 2}(2^0+2^1+2^2+\ldots+2^{2024})\\
			&=&\dfrac{1}{\ln 2}\cdot \dfrac{1(1-2^{2025})}{1-2}=\dfrac{2^{2025}-1}{\ln 2}.
		\end{eqnarray*}
		$\Rightarrow a=2$, $b=2025$, $c=1$.\\
		Vậy $ T=a+b-2c=2025$.}
\end{ex}

\begin{ex}%[2D4C3-2]
	\immini
	{Ông An xây dựng một sân bóng đá mini hình chữ nhật có chiều rộng $30$m và chiều dài $50$m. Để giảm bớt chi phí cho việc trồng cỏ nhân tạo, ông An chia sân bóng ra làm hai phần (tô đen và không tô đen) như hình bên. Phần tô đen gồm hai phần diện tích bằng nhau và đường cong $AIB$ là một parabol đỉnh $I$ được trồng cỏ nhân tạo với giá $130\,000$ đồng/m$^2$ và phần còn lại được trồng với giá $90\,000$ đồng/m$^2$.
	}
	{\begin{tikzpicture}[scale=0.9, font=\footnotesize,line join=round, line cap=round, >=stealth]
			\coordinate (I) at (0,0);
			\coordinate (A) at (1.5,2.25);
			\coordinate (B) at ($(A)+(0,-4.5)$);
			\coordinate (C) at ($(B)+(-7.5,0)$);
			\coordinate (D) at ($(A)-(B)+(C)$);
			\coordinate (M) at ($(A)!1/2!(D)$);
			\coordinate (N) at ($(B)!1/2!(C)$);
			\coordinate (P) at ($(A)+(-1.5,0)$);
			\coordinate (Q) at ($(B)+(-1.5,0)$);
			\coordinate (R) at ($(D)+(1.5,0)$);
			\coordinate (S) at ($(C)+(1.5,0)$);
			\coordinate (td) at ($(D)+(0,0.3)$);
			\coordinate (dt) at ($(D)+(-0.3,0)$);
			\coordinate (ct) at ($(C)+(-0.3,0)$);
			\coordinate (tr) at ($(R)+(0,0.3)$);
			\coordinate (rp) at ($(R)+(0.3,0)$);
			\coordinate (I') at ($(R)!1/2!(S)$);
			\coordinate (g) at ($(I')+(0.3,0)$);
			\fill[gray]plot[domain=0:1.5](\x,{sqrt(3.375*(\x))})--(A)--plot[domain=1.5:0](\x,{-sqrt(3.375*(\x))})--cycle;
			\fill[gray](C)--(D)--plot[domain=-6:-4.5](\x,{sqrt(3.375*(-\x-4.5))})--plot[domain=-4.5:-6](\x,{-sqrt(3.375*(-\x-4.5))})--cycle;
			\draw (A)--(B)--(C)--(D)--cycle (M)--(N);
			\draw[dashed] (P)--(Q) (R)--(S);
			\draw[<->](td)--(tr);
			\node at ($(td)!1/2!(tr)$)[above]{$10$ m};
			\draw[<->](dt)--(ct);
			\node at ($(dt)!1/2!(ct)$)[above,rotate=90]{$30$ m};
			\draw[<->](rp)--(g);
			\node at ($(rp)!1/2!(g)$)[above,rotate=-90]{$15$ m};
			\foreach \x/\g in {A/90,B/-90,I/180,I'/-40}\draw[fill=black] (\x) circle (.05) +(\g:.5)node{\footnotesize$\x$};
		\end{tikzpicture}}
	\noindent
	Hỏi ông An phải trả bao nhiêu tiền (triệu đồng) để trồng cỏ nhân tạo cho sân bóng.
	% \shortans{$151$}
	\loigiai{
		\immini{
			Chọn hệ trục tọa độ như hình vẽ ($I$ là gốc tọa độ). Khi đó đường cong $IAB$ là một parabol có phương trình dạng $y=ax^2$.\\
			Parabol đi qua điểm $\left(15;10 \right)$, suy ra
			\[a \cdot 15^2=10 \Rightarrow a=\dfrac{2}{45}.\]
		}
		{
			\begin{tikzpicture}[smooth,samples=300,scale=0.6,>=stealth]
				\fill[gray!30] (-4,2)--(4,2)--plot[domain=4:-4](\x,{0.125*(\x)^2});
				\draw[->] (-5,0)--(5,0) node[below]{$x$};
				\draw[->] (0,-1)--(0,3) node[right]{$y$};
				\draw (0,0) node[below left]{$I$};
				\draw[domain=-4:4] plot(\x,{0.125*(\x)^2});
				\draw[fill=black] (4,2) circle(1.5pt) (-4,2) circle(1.5pt);
				\draw[dashed] (4,0)node[below]{$15$}--(4,2)node[right]{$B$}--(-4,2)node[left]{$A$}--(-4,0)node[below]{$-15$};

				\node[right] at (0,2.4) {$10$};
			\end{tikzpicture}}
		\noindent
		Vậy $y=\dfrac{2}{45}x^2$. Diện tích phần tô đen là $S=2 \cdot \displaystyle\int\limits_{-15}^{15} \left(10-\dfrac{2}{45}x^2 \right) \mathrm{\,d}x=400 \, (\text{m}^2)$.\\
		Diện tích phần còn lại của sân bóng là $S_2=30 \cdot 50-400=1100\,(\text{m}^2).$\\
		Số tiền Ông An phải trả để trồng cỏ nhân tạo cho sân bóng là
		\[130000\times 400+90000\times 1100=151000000\text{ đồng}=151\text{ triệu đồng.}\]}
\end{ex}

% \Closesolutionfile{ansbook}
% \HetDe
% \label{De1}
% %
% \cleardoublepage
% \setcounter{page}{1}
% \rfoot{Trang \thepage/\pageref{DA1} - Đáp án trắc nghiệm Mã đề 1}
% \begin{center}
% 	\bfseries ĐÁP ÁN TRẮC NGHIỆM MÃ ĐỀ 1
% \end{center}

% \inputansbox{10}{ans/ansDe1-TN1}
% \inputansbox[3]{2}{ans/ansDe1-TN2}
% \inputansbox{3}{ans/ansDe1-TN3}
% \label{DA1}
% %

% \begin{name}
	{\tenchude}
	{TOÁN 12}
	{LỚP TOÁN THẦY PHÁT}
	{Thời gian: 90 phút - Không kể thời gian phát đề}
\end{name}
\TN
\Opensolutionfile{ans}[ans/ansDe2-TN1]
\begin{ex}%[EX-Ôn tập TN 2025, Nguyễn Tiến]%[2D4N1-1]
	Phát biểu nào sau đây là đúng?
	\choice
	{\True $\displaystyle\int\limits F'(x)\mathrm{\,d}x=F(x)+C$}
	{$\displaystyle\int\limits F(x)\mathrm{\,d}x=F'(x)+C$}
	{$\displaystyle\int\limits F(x)\mathrm{\,d}x=F(x)+C$}
	{$\displaystyle\int\limits F'(x)\mathrm{\,d}x=F'(x)+C$}
	\loigiai{
		Phát biểu đúng là $\displaystyle\int\limits F'(x)\mathrm{\,d}x=F(x)+C$.
	}
\end{ex}

\begin{ex}%[Đề 12 - 2025 - Nguyen Son]%[2D4H1-1]
	Cho hàm số $ f(x) $ có đạo hàm $ f'(x) $ liên tục và có một nguyên hàm là hàm số $ F(x) $. Tìm nguyên hàm $ I=\displaystyle\int \left[2f(x)+f'(x)+1\right]\mathrm{d}x $.
	\choice
	{\True $ I=2F(x)+f(x)+x+C $}
	{$ I=2F(x)+xf(x)+C $}
	{$ I=2xF(x)+f(x)+x+1 $}
	{$ I=2xF(x)+f(x)+x+C $}
	\loigiai{
		Ta có $ I=\displaystyle\int \left[2f(x)+f'(x)+1\right]\mathrm{d}x=2F(x)+f(x)+x+C $.
	}
\end{ex}

\begin{ex}%[2D4N1-2]
	Nguyên hàm của hàm số $f(x)=x^3+x$ là
	\choice
	{\True $\dfrac{1}{4}x^4+\dfrac{1}{2}x^2+C$}
	{$3x^2+1+C$}
	{$x^3+x+C$}
	{$x^4+x^2+C$}
	\loigiai{
		$\displaystyle\int{(x^3+x^2)\mathrm{\,d}x}=\dfrac{1}{4}x^4+\dfrac{1}{2}x^2+C$.}
\end{ex}

\begin{ex}%[2D4N1-3]%[Dự án EX-TF-TLN lần 3 -Mui Doan]
	Họ tất cả các nguyên hàm của hàm số $f(x)=x+\sin x$ là
	\choice
	{$\dfrac{x^2}{2}+\cos x+C$}
	{$x^2+\cos x+C$}
	{$x^2-\cos x+C$}
	{\True $\dfrac{x^2}{2}-\cos x+C$}
	\loigiai{
		Ta có $\displaystyle\int f(x)\mathrm{\,d}x=\displaystyle\int\left(x+\sin x\right)\mathrm{\,d}x=\dfrac{x^2}{2}-\cos x+C$.}
\end{ex}

\begin{ex}%[Mui Doan, dự án 12EX-OTTN2025]%[2D4N1-4]
	Hàm số nào sau đây là một nguyên hàm của hàm số $y=10^x$?
	\choice
	{$y=10^x \ln 10$}
	{$y=10^x$}
	{$y=\dfrac{10^{x+1}}{x+1}$}
	{\True $y=\dfrac{10^x}{\ln 10}$}
	\loigiai{
		Theo công thức nguyên hàm, một nguyên hàm của $y=10^x$ là $y=\dfrac{10^x}{\ln 10}$.
	}
\end{ex}

\begin{ex}%[2D4N1-4]
	$\displaystyle\int (3^x+4^x)\mathrm{d}\,x$ bằng
	\choice
	{\True  $\dfrac{3^x}{\ln 3}+\dfrac{4^x}{\ln 4}+C$}
	{$\dfrac{3^x}{\ln 4}+\dfrac{4^x}{\ln 3}+C$}
	{$\dfrac{4^x}{\ln 3}-\dfrac{3^x}{\ln 4}+C$}
	{$\dfrac{3^x}{\ln 3}-\dfrac{4^x}{\ln 4}+C$}
	\loigiai{
		Áp dụng công thức $\displaystyle\int a^x\,\mathrm{d}x=\frac{a^x}{\ln a}+C$.\\
		Ta có $\displaystyle\int(3^x+4^x)\mathrm{d}\,x
			=\int 3^x\mathrm{d}\,x+\int 4^x\mathrm{d}\,x=\dfrac{3^x}{\ln 3}+\dfrac{4^x}{\ln 4}+C$.}
\end{ex}

\begin{ex}%[2D4H1-2]
	Tìm nguyên hàm của hàm số $ f(x)=\dfrac{{x^4}+2}{x^2}$.
	\choice
	{$\displaystyle\int{f(x)\mathrm{d}x=}\dfrac{{x^3}}{3}-\dfrac{1}{x}+C$}
	{\True $\displaystyle\int{f(x)\mathrm{d}x=}\dfrac{{x^3}}{3}+\dfrac{2}{x}+C$}
	{$\displaystyle\int{f(x)\mathrm{d}x=}\dfrac{{x^3}}{3}+\dfrac{1}{x}+C$}
	{\True $\displaystyle\int{f(x)\mathrm{d}x=}\dfrac{{x^3}}{3}-\dfrac{2}{x}+C$}
	\loigiai{
		Ta có: $\displaystyle\int{f(x)\mathrm{d}x=}\displaystyle\int{\dfrac{{x^4}+2}{{x^2}}}\mathrm{d}x=\displaystyle\int{( {x^2}+\dfrac{2}{{x^2}} )}\mathrm{d}x=\dfrac{{x^3}}{3}-\dfrac{2}{x}+C$.
	}
\end{ex}

\begin{ex}%[2D4H1-3]
	$\displaystyle \int \left(\cos \dfrac{x}{2} \right)^2 \mathrm{d}x$ bằng
	\choice
	{$x+\sin x+C$}
	{$\dfrac{1}{3}\left(\cos \dfrac{x}{2} \right)^3+C$}
	{$\left(\sin \dfrac{x}{2} \right)^2+C$}
	{\True $\dfrac{1}{2}x+ \dfrac{1}{2}\sin x+C$}
	\loigiai{
		Ta có
		\allowdisplaybreaks
		\begin{eqnarray*}
			\displaystyle \int \left(\cos \dfrac{x}{2} \right)^2 \mathrm{d}x &=& \displaystyle \int \dfrac{1+\cos x}{2} \mathrm{d}x\\
			&=& \dfrac{1}{2} \displaystyle \int \mathrm{d}x+ \dfrac{1}{2}\displaystyle \int \cos x \mathrm{d}x \\
			&=& \dfrac{1}{2}x+ \dfrac{1}{2}\sin x+C.
		\end{eqnarray*}
	}
\end{ex}

\begin{ex}%[Dự án 2025 - đề cấu trúc mới, Nguyễn Kiều Nhã Tú]%[2D4N2-1]
	Cho hai hàm số $f(x)$ và $g(x)$ liên tục trên $K$, $a$, $b\in K$. Khẳng định nào sau đây là \textbf{sai}?
	\choice
	{$\displaystyle\int\limits_a^b [f(x)+g(x)]\mathrm{\,d} x=\displaystyle\int\limits_a^b f(x)\mathrm{\,d} x+\displaystyle\int\limits_a^b g(x)\mathrm{\,d}x$}
	{$\displaystyle\int\limits_a^b 2f(x)\mathrm{\,d}x=2 \displaystyle\int\limits_a^b f(x)\mathrm{\,d}x$}
	{\True $\displaystyle\int\limits_a^b f(x)g(x)\mathrm{\,d} x=\displaystyle\int\limits_a^b f(x)\mathrm{\,d}x \cdot \displaystyle\int\limits_a^b g(x)\mathrm{\,d}x$}
	{$\displaystyle\int\limits_a^b[f(x)-g(x)]\mathrm{\,d} x=\displaystyle\int\limits_a^b f(x)\mathrm{\,d} x-\displaystyle\int\limits_a^b g(x)\mathrm{\,d}x$}
	\loigiai{
		Theo tính chất của tích phân, không có tính chất $\displaystyle\int\limits_a^b f(x)g(x)\mathrm{\,d} x=\displaystyle\int\limits_a^b f(x)\mathrm{\,d}x \cdot \displaystyle\int\limits_a^b g(x)\mathrm{\,d}x$.
	}
\end{ex}

\begin{ex}%[2D4N2-2]%[To 20 - Dot 17 - Chuong 4 - Bai 3 - CD - De 3]%[Đình Nguyên]
	Tính $\displaystyle\int\limits_{3}^{5}\left(x^{2}+2 x\right) \mathrm{\,d}x$.
	\choice
	{\True $\dfrac{146}{3}$}
	{$146$}
	{$3$}
	{$\dfrac{143}{6}$}
	\loigiai
	{
		Ta có
		\begin{align*}
			\int\limits_{3}^{5}\left(x^{2}+2 x\right) \mathrm{\,d}x & =\int\limits_{3}^{5} x^{2} \mathrm{\,d}x+2 \int\limits_{3}^{5} x \mathrm{\,d}x             \\
			                                                        & =\left.\dfrac{1}{3} x^{3}\right|_{3} ^{5}+\left.2 \cdot \dfrac{1}{2} x^{2}\right|_{3} ^{5} \\
			                                                        & =\left.\dfrac{1}{3} x^{3}\right|_{3} ^{5}+\left.x^{2}\right|_{3} ^{5}                      \\
			                                                        & =\dfrac{1}{3}\left(5^{3}-3^{3}\right)+\left(5^{2}-3^{2}\right)=\dfrac{146}{3}.
		\end{align*}
	}
\end{ex}

\begin{ex}%[2D4N2-3]
	Tích phân $I=\displaystyle\int\limits_{\frac{\pi}{4}}^{\frac{\pi}{3}} \dfrac{\mathrm{\,d} x}{\sin ^2 x}$ bằng
	\choice
	{$\cot \dfrac{\pi}{3}-\cot \dfrac{\pi}{4}$}
	{$\cot \dfrac{\pi}{3}+\cot \dfrac{\pi}{4}$}
	{\True $-\cot \dfrac{\pi}{3}+\cot \dfrac{\pi}{4}$}
	{$-\cot \dfrac{\pi}{3}-\cot \dfrac{\pi}{4}$}
	\loigiai{Ta có
	$I=\displaystyle\int\limits_{\frac{\pi}{4}}^{\frac{\pi}{3}} \dfrac{\mathrm{\,d} x}{\sin ^2 x}=-\left.\cot x\right|_{\frac{\pi}{4}} ^{\frac{\pi}{3}}=-\cot \dfrac{\pi}{3}+\cot \dfrac{\pi}{4}$.
	}
\end{ex}

\begin{ex}%[Tổ 20 - Chương 4 - - CD]%[Nguyễn Văn Sang]%[2D4N2-4]
	Tính $I=\displaystyle\int\limits_0^1\left(2 \cdot  \mathrm{e}^{-x}+4^x\right) \mathrm{\,d}x$
	\choice
	{\True $I=\dfrac{-2}{\mathrm{e}}+2+\dfrac{3}{\ln 4}$ }
	{$I=\dfrac{2}{\mathrm{e}}+2+\dfrac{3}{\ln 4}$ }
	{$I=\dfrac{-2}{\mathrm{e}}+2-\dfrac{3}{\ln 4}$ }
	{$I=\dfrac{-2}{\mathrm{e}}+2+\dfrac{4}{\ln 4}$ }
	\loigiai{
		\[I=\displaystyle\int\limits_0^1\left(2 \cdot \mathrm{e}^{-x}+4^x\right) \mathrm{\,d} x=\left(-2 \mathrm{e}^{-x}+\dfrac{4^x}{\ln 4}\right)\bigg|_0 ^1=\dfrac{-2}{\mathrm{e}}+\dfrac{4}{\ln 4}+2-\dfrac{1}{\ln 4}=\dfrac{-2}{\mathrm{e}}+2+\dfrac{3}{\ln 4}.\]
	}
\end{ex}
\Closesolutionfile{ans}

\TNTF
\Opensolutionfile{ans}[ans/ansDe2-TN2]
\begin{ex}%[BG-12-4in1, Phạm Đức]%[2D4N1-1]
	Cho hàm số $f(x)$ có một nguyên hàm là $F(x)$, $k$ là số thực bất kỳ. Mỗi khẳng định sau đây đúng hay sai?
	\choiceTF[t]
	{\True $\displaystyle\int f(x)\mathrm{\,d}x=F(x)+C$ với $C$ là hằng số}
	{$f'(x)=F(x)$}
	{$\displaystyle\int kf(x)\mathrm{\,d}x=k\int f(x)\mathrm{\,d}x$}
	{\True $\displaystyle\int \left[f(x)+F'(x)\right]\mathrm{\,d}x=2\left[F(x)+2025\right]+C$ với $C$ là hằng số}
	\loigiai{
		\begin{itemchoice}
			\itemch Theo định nghĩa nguyên hàm ta có $\displaystyle\int f(x)\mathrm{\,d}x=F(x)+C$ với $C$ là hằng số.
			\itemch Tổng quát ta chỉ có kết quả $F'(x)=f(x)$.
			\itemch Tính chất $\displaystyle\int kf(x)\mathrm{\,d}x=k\int f(x)\mathrm{\,d}x$ chỉ đúng với $k\ne 0$.
			\itemch Ta có $\left(2\left[F(x)+2025\right]+C\right)'=2F'(x)=2f(x)=f(x)+f(x)=f(x)+F'(x)$.

			Vậy $\displaystyle\int \left[f(x)+F'(x)\right]\mathrm{\,d}x=2\left[F(x)+2025\right]+C$ với $C$ là hằng số.
		\end{itemchoice}
	}
\end{ex}

\begin{ex}%[2D4H2-1]
	Cho $\displaystyle\int\limits_0^{\frac{\pi}{2}} \left[\sin x +f(x)\right] \mathrm{\, d}x=6$. Xét tính đúng, sai cho mỗi khẳng định sau.
	\choiceTF[t]
	{\True $\displaystyle\int\limits_0^{\tfrac{\pi}{2}} \left[\sin x +f(x)\right] \mathrm{\, d}x= \displaystyle\int\limits_0^{\tfrac{\pi}{2}} \sin x \mathrm{\, d}x + \displaystyle\int\limits_0^{\tfrac{\pi}{2}} f(x) \mathrm{\, d}x$}
	{\True  $\displaystyle\int\limits_0^{\tfrac{\pi}{2}} \sin x\mathrm{\,d}x=1$}
	{\True $\displaystyle\int\limits_0^{\tfrac{\pi}{2}} f(x)\mathrm{\, d}x=5$}
	{$\displaystyle\int\limits_0^{\tfrac{\pi}{2}}\left[3+2f(x)\right]\mathrm{\, d}x=\dfrac{3\pi}{2}+5$}
	\loigiai{
		\begin{itemchoice}
			\itemch  Ta có $\displaystyle\int\limits_0^{\tfrac{\pi}{2}} \left[\sin x +f(x)\right] \mathrm{\, d}x= \displaystyle\int\limits_0^{\tfrac{\pi}{2}} \sin x \mathrm{\, d}x + \displaystyle\int\limits_0^{\tfrac{\pi}{2}} f(x) \mathrm{\, d}x$.
			\itemch  Ta có $\displaystyle\int\limits_0^{\tfrac{\pi}{2}} \sin x\mathrm{\,d}x=-\cos x\bigg|_0^{\tfrac{\pi}{2}}=-\cos \dfrac{\pi}{2}+\cos 0=1$.
			\itemch  Ta có $\displaystyle\int\limits_0^{\tfrac{\pi}{2}} \left[\sin x +f(x)\right] \mathrm{\, d}x= \displaystyle\int\limits_0^{\tfrac{\pi}{2}} \sin x \mathrm{\, d}x + \displaystyle\int\limits_0^{\tfrac{\pi}{2}} f(x) \mathrm{\, d}x= 1 + \displaystyle\int\limits_0^{\tfrac{\pi}{2}} f(x) \mathrm{\, d}x=6$.\\
			Suy ra $\displaystyle\int\limits_0^{\tfrac{\pi}{2}} f(x)\mathrm{\, d}x=6-1=5$.
			\itemch  Ta có $\displaystyle\int\limits_0^{\tfrac{\pi}{2}}\left[3+2f(x)\right]\mathrm{\, d}x=\displaystyle\int\limits_0^{\tfrac{\pi}{2}}3\mathrm{\, d}x+2\displaystyle\int\limits_0^{\tfrac{\pi}{2}} f(x)\mathrm{\, d}x=3x\bigg|_0^{\tfrac{\pi}{2}}+2\cdot5 =\dfrac{3\pi}{2}+10\neq\dfrac{3\pi}{2}+5$.
		\end{itemchoice}
	}
\end{ex}
\Closesolutionfile{ans}

\TNSA
\Opensolutionfile{ans}[ans/ansDe2-TN3]
\begin{ex}%[2D4H1-1]%[Đào Trung Kiên]
	Biết $F(x)$ là một nguyên hàm của hàm số $f(x)=\sin x$ và đồ thị hàm số $y=F(x)$ đi qua điểm $M\left(0;1\right)$. Tính $F\left(\dfrac{\pi}{2}\right)$ (làm tròn kết quả tới hàng đơn vị).
	\shortans[]{$2$}
	\loigiai{
		Ta có $F(x)=\displaystyle\int f(x)\mathrm{\,d}x=-\cos x+C$.\\
		Mà đồ thị hàm số $y=F(x)$ đi qua $M(0;1)$ nên $F(0)=1\Leftrightarrow -1+C=1\Leftrightarrow C=2$.\\
		Suy ra $F(x)=-\cos x+2$ nên $F\left(\dfrac{\pi}{2}\right)=2$.}
\end{ex}

\begin{ex}%[2D4H2-2]%[Tổ 20 - Đợt 17 - Chương 4 - - CD]%[Phạm Hà Giang]
	Tích phân $\displaystyle\int\limits_2^8 \dfrac{1}{x} \mathrm{\,d}x=a\ln 2$. Giá trị của $a$ bằng bao nhiêu?
	\shortans{$2$}
	\loigiai
	{ $\displaystyle\int\limits_2^8 \dfrac{1}{x} \mathrm{\,d}x =\left. \ln \left| x \right| \right| _2^8=\ln 8 -\ln 2 =2\ln 2.$\\
		Suy ra $a=2$.
	}
\end{ex}

\begin{ex}%[2D4H3-1]
	\immini{Cho hàm số $f(x)$ liên tục trên $\mathbb{R}$. Đồ thị hàm số $y=f'(x)$ được cho như hình bên. Diện tích các hình phẳng $(K)$, $(H)$ lần lượt là $\dfrac{5}{12}$ và $\dfrac{8}{3}$. Biết $f(-1)=\dfrac{19}{12}$, tính $f(2)$(kết quả làm tròn đến hàng phần mười).
	}
	{
		\begin{tikzpicture}[scale=1, font=\footnotesize, line join=round, line cap=round, >=stealth]
			\tikzset{label style/.style={font=\footnotesize}}
			%Nhập giới hạn đồ thị và hàm số cần vẽ
			\def \xmin{-1.5}
			\def \xmax{2.5}
			\def \ymin{-2.6}
			\def \ymax{1}
			\def \hamso{sin((2*(\x))*180/pi)}
			%\def \tiemcanxien{\x+1}
			%Tự động
			\draw[->] (\xmin,0)--(\xmax,0) node[below left] {$x$};
			\draw[->] (0,\ymin)--(0,\ymax) node[below left] {$y$};
			\draw[fill=black] (0,0) circle(1pt) node [below left] {$O$};
			%Vẽ các điểm trên 2 hệ trục
			\foreach \x in {-1,2}
			\fill[black] (\x,0) circle(1pt) node [above left] {$\x$};
			%Tự động
			\begin{scope}
				\clip (\xmin+0.01,\ymin+0.01) rectangle (\xmax-0.01,\ymax-0.01);
				\draw[samples=350,domain=\xmin+0.01:\xmax-0.01,smooth,variable=\x] plot (\x,{(\x+1)*(\x)*(\x-2)});
			\end{scope}
			\draw[pattern = north east lines] (-1,0)--plot[domain=-1:2] (\x,{(\x+1)*(\x)*(\x-2)});
			\draw (-.5,-.5)node{$(K)$};
			\draw (1,.5)node{$(H)$};
		\end{tikzpicture}
	}
	\shortans[]{$-0{,}7$}
	\loigiai{
	Ta có $S_K=\displaystyle \int\limits_{-1}^0 f'(x)\mathrm{\,d}x
		=f(0)-f(-1)	=\dfrac{5}{12}
		\Rightarrow f(0)=\dfrac{5}{12}+f(-1)=\dfrac{5}{12}+\dfrac{19}{12}=2$.\\
	Lại có: $S_H=\displaystyle \int\limits_0^2 \left| f'(x) \right|\mathrm{\,d}x
		=-\displaystyle \int\limits_0^2 f'(x)\mathrm{\,d}x
		=f(0)-f(2)=\dfrac{8}{3}
		\Rightarrow f(2)=f(0)-\dfrac{8}{3}=2-\dfrac{8}{3}=\dfrac{-2}{3}\approx -0{,}7$.
	}
\end{ex}

\begin{ex} %[2D4H3-3]
	Cho hình phẳng $(D)$ được giới hạn bởi các đường $ y={x^2}+2x+2$; $y=6-x$; $y=2$ và $(D)$ nằm ngoài Parabol $y={x^2}+2x+2$. Khi cho $(D)$ quay quanh trục $Ox$, ta nhận được vật thể tròn xoay có thể tích $V=\dfrac{a\pi }{b}$, trong đó $a$, $b$ là các số nguyên dương. Giá trị biểu thức $P=a-2{b^2}$ bằng bao nhiêu.
	\shortans[1]{$73$}
	\loigiai{
	Vẽ các đường $ y={x^2}+2x+2$; $y=6-x$; $y=2$.
	\begin{center}
		\begin{tikzpicture}[scale=0.7,line join=round,line cap=round,font=\footnotesize,>=stealth]
			\def\f(#1){1*((#1)^2)+2*(#1)+2}
			\def\g(#1){6-1*(#1)}
			\draw[->] (-3,0)--(7,0) node[below] { $x$};
			\draw[->] (0,-1)--(0,7) node[left] { $y$};
			\draw (0,0) node [below left] { $O$};
			\foreach \x in {1,4}		\draw (\x,0.1)--(\x,-0.1) node [below] { $\x$};
			\foreach \y in {5}		\draw (0.1,\y)--(-0.1,\y) node [left] { $\y$};
			\node[left] at (0,2.2) {$2$};
			\clip (-4,-1) rectangle (6,6);
			\draw[thick,smooth,samples=200] plot[domain=-3:3] (\x,{\f(\x)});
			\draw[thick,smooth,samples=200] plot[domain=-2:7] (\x,{\g(\x)});
			\draw[thick,smooth,samples=200] plot[domain=-3:3] (-3,2)--(6,2);
			\draw[dashed] (4,0)--(4,2) (1,0)-- (1,5)--(0,5);
			\fill[pattern=north east lines,pattern color=red]plot[domain=0:2](\x,{(\x)^2+2*(\x)+2})-- (1,5)--(1,2) -- cycle;
			\fill[pattern=north east lines,pattern color=black] (1,2)--(1,5)--plot[domain=0:2](\x,{6-1*(\x)})-- (4,2) -- cycle;
		\end{tikzpicture}
	\end{center}
	Dựa vào hình vẽ ta có\\
	$V=\pi \int\limits_0^1{| {{( {x^2}+2x+2 )}^2}-{2^2} |\mathrm{\,d}x}+\pi \int\limits_1^4{| {{( 6-x )}^2}-{2^2} |\mathrm{\,d}x}=\dfrac{523\pi }{15}\\
		\Rightarrow a=523,b=15$.\\
	Vậy $P=523-{{2\cdot 15}^2}=73$.
	}
\end{ex}

\Closesolutionfile{ans}

\TL
\begin{ex}%[2D4H2-2] 
	Tính tích phân $I=\displaystyle\int\limits_{-1}^2{\left|x^3-3x+2\right|}\mathrm{\,d}x$.
	\loigiai{
	$C=\displaystyle\int\limits_{-1}^2 {\left|x^3-3x+2\right|}\mathrm{\,d}x$.\\
	Xét $f(x) = x^3-3x-2$ trên $[-1; 2]$.\\
	Cho $f(x) = 0
		\Leftrightarrow
		x^3-3x-2 = 0
		\Leftrightarrow
		\hoac{& x=2\in [-1; 2] \\ & x=-1\in [-1; 2].}$\\
	Bảng xét dấu $f(x)$ trên đoạn $[-1; 2]$.
	\begin{center}
		\begin{center}
			\begin{tikzpicture}
				\tkzTabInit[nocadre=false,lgt=3,espcl=2.5,deltacl=0]
				{$x$ /0.6,$x^3-3x+2$ /0.6}
				{,$-1$,$2$,}
				\tkzTabLine{,h,$0$,-,$0$,h,}
			\end{tikzpicture}
		\end{center}
	\end{center}
	Do đó
	$C = -\displaystyle\int\limits_{-1}^2 {(x^3-3x-2)}\mathrm{\,d}x	=
		-\left.\left(\dfrac{1}{4} x^4-\dfrac{3}{2} x^2 -2x\right)\right|_{-1}^2=
		6- \left( -\dfrac{3}{4} \right)=\dfrac{27}{4}$.
	}
\end{ex}


\begin{ex}%[2D4V1-4]
	Cho hàm số $f(x)$ thỏa mãn $ f(0)=1-\ln 2$ và $\mathrm{e}^ xf'(x)=2^x\left[f(x)\right]^2$ với mọi $x\in\mathbb{R}$. Giá trị của $f(1)$ bằng bao nhiêu? (\textit{Kết quả làm tròn đến hàng phần trăm}).
	% \shortans{$0{,}42$}
	\loigiai{
		Từ giả thiết ta có $f'(x)=\dfrac{2^x}{\mathrm{e}^ x}{\left[f(x)\right]^2}$ với mọi $ x\in\left(1;2\right]$.\\
		Do đó $ f(x)\ge f(1)=1>0$ với mọi $ x\in\left[1;2\right]$.\\
		Xét với mọi $ x\in [1 ; 2]$ ta có
		\begin{align*}
			\mathrm{e}^ x{f}'(x)=2^x{\left[f(x)\right]^2} & \Rightarrow{f}'(x)=\dfrac{2^x}{\mathrm{e}^ x}{\left[f(x)\right]^2} \\&\Rightarrow\dfrac{f'(x)}{\left[f(x)\right]^2}=\left(\dfrac{2}{\mathrm{e}}\right)^x\\&\Rightarrow-\left(\dfrac{1}{f(x)}\right)'=\left(\dfrac{2}{\mathrm{e}}\right)^x \\&\Rightarrow{\left(\dfrac{1}{f(x)}\right)'}=-\left(\dfrac{2}{\mathrm{e}}\right)^x\\&\Rightarrow\dfrac{1}{f(x)}=-\displaystyle\int\left(\dfrac{2}{\mathrm{e}}\right)^x\mathrm{\,d} x\\&\Rightarrow\dfrac{1}{f(x)}=-\dfrac{\left(\dfrac{2}{\mathrm{e}}\right)^x}{\ln \dfrac{2}{\mathrm{e}}}+C\\&\Rightarrow\dfrac{1}{f(x)}=\dfrac{\left(\dfrac{2}{\mathrm{e}}\right)^x}{1-\ln 2}+C.
		\end{align*}
		Mà $ f(0)=1-\ln 2\Rightarrow C=0$. \\Do đó
		$\dfrac{1}{f(x)}=\dfrac{\left(\dfrac{2}{\mathrm{e}}\right)^x}{1-\ln 2}$
		$\Rightarrow f(x)=\dfrac{1-\ln 2}{\left(\dfrac{2}{\mathrm{e}}\right)^x}=\dfrac{(1-\ln 2)\mathrm{e}^x}{2^x}$.\\
		Vậy $ f(1)=\dfrac{\mathrm{e}-\mathrm{e}\ln 2}{2}\approx 0{,}42$.}
\end{ex}

\begin{ex}%[12-MH-2-MH2025]%[MH-2025, Nguyễn Trần Phong]%[2D4C3-2]
	\immini{Chướng ngại vật \lq\lq  tường cong\rq\rq trong một sân thi đấu X-Game là một khối bê tông có chiều cao từ mặt đất lên là $3$ m. Giao của mặt tường cong và mặt đất là đoạn thẳng $AB = 2$ m. Thiết diện của khối tường cong cắt bởi mặt phẳng vuông góc với $AB$ tại $A$ là một hình tam giác vuông cong $ACE$ với $AC = 4$ m, $CE = 3$ m và cạnh cong $AE$ nằm trên một đường Parabol có trục đối xứng vuông góc với mặt đất. Tại vị trí $M$ là trung điểm của $AC$ thì tường cong có độ cao $1$ m. Thể tích bê tông cần sử dụng để tạo nên khối tường cong đó gần nhất với số nào dưới đây?
		% \shortans{$9{,}3$}
	}{\begin{tikzpicture}[>=stealth,x=0.8cm,y=0.8cm,scale=0.7]
			\coordinate[label=below:$A$] (A) at (0,0);
			\coordinate[label=left:$B$] (B) at (-2,2);
			\coordinate[label=below:$C$] (C) at (6,0);
			\coordinate[label=right:$E$] (E) at (6,6);
			\coordinate (G) at (2,4);
			\coordinate (H) at (4,1.8);
			\coordinate (D) at ($(C)+(B)-(A)$);
			\coordinate (F) at ($(E)+(D)-(C)$);
			\coordinate[label=below:$M$] (M) at ($(A)!0.5!(C)$);
			\coordinate (K) at ($(A)!0.5!(B)$);
			\coordinate (N) at ($(M)+(0,1.1)$);
			\draw (D)--(C)--(A)--(B) (C)--(E)--(F) (M)--(N);
			\draw[dashed] (B)--(D)--(F);
			\foreach \diem in {A,B,C,D,E,F,M,F}	\fill (\diem)circle(1.5pt);
			%\tkzLabelPoints[above left](D)
			%\tkzLabelSegment[right](M,N){\footnotesize$1$ m}
			%\tkzLabelSegment[left](A,B){\footnotesize$2$ m}
			%\tkzLabelSegment[right](C,E){\footnotesize$3{,}5$ m}
			\draw(-1,.8) node[left]{\footnotesize $2$ m} (3,0.8) node[right]{\footnotesize$1$ m} (6,3) node[right]{\footnotesize$3$ m};

			\draw plot[smooth,tension=.65] coordinates{(B) (G) (F)};
			\draw plot[smooth,tension=.65] coordinates{(A) (H) (E)};
			\fill [pattern = north east lines] plot[smooth,tension=.65] coordinates{(A) (H) (E)} (0,0) --(-2,2)--(4,8)--(6,6)--cycle;
			\fill [draw=none, pattern = north east lines, color=white] (0,0) plot[smooth,tension=.65] coordinates{(B) (G) (F)} (-2,2)--(4,2)--cycle;
		\end{tikzpicture}
	}
	\loigiai{
		\immini{Chọn hệ trục tọa độ như hình vẽ.\\
			Gọi $AE \colon y = ax^2 + bx + c$.\\
			Do $AE$ đi qua $A(-4; 0)$ nên ta có $16a - 4b + c = 0$.\\
			Do $E (0; 3)$ thuộc cạnh cong $AE$ nên $c = 3$ (2).\\
			Do $N(-2; 1)$ thuộc cạnh cong $AE$ nên $4a - 2b + c = 1$ (3).\\
			Từ (1), (2), (3) suy ra $a = \dfrac{1}{8}$, $b = \dfrac{5}{4}$, $c = 3 \Rightarrow AE \colon y = \dfrac{1}{8}x^2 + \dfrac{5}{4}x + 3$.\\
			Khi đó $S_{AEC} = \displaystyle\int_{-4}^0\left(\dfrac{1}{8} x^2 + \dfrac{5}{4}x + 3\right) dx = \dfrac{14}{3}\left(m^2\right)$.
		}{\begin{tikzpicture}[>=stealth,x=0.8cm,y=0.8cm,scale=0.7]
				\coordinate[label=below:$A$] (A) at (0,0);
				\coordinate[label=left:$B$] (B) at (-2,2);
				\coordinate[label=below:$C$] (C) at (6,0);
				\coordinate[label=right:$E$] (E) at (6,6);
				\coordinate (G) at (2,4);
				\coordinate (H) at (4,1.8);
				\coordinate[label = above left:$D$] (D) at ($(C)+(B)-(A)$);
				\coordinate[label = above:$F$] (F) at ($(E)+(D)-(C)$);
				\coordinate[label=below:$M$] (M) at ($(A)!0.5!(C)$);
				\coordinate (K) at ($(A)!0.5!(B)$);
				\coordinate[label=above left:$N$] (N) at ($(M)+(0,1.1)$);
				\draw (D)--(C)--(A)--(B) (C)--(E)--(F) (M)--(N);
				\draw[dashed] (B)--(D) (D)--(F);
				\foreach \diem in {A,B,C,D,E,F,M,F,N}	\fill (\diem)circle(1.5pt);
				\coordinate (x) at ($(M)!1.5!(C)$);
				\draw[->](C)--(x); \draw (x) node[right]{$x$};
				\coordinate (y) at ($(C)!1.4!(E)$);
				\draw[->](E)--(y); \draw (y) node[right]{$y$};
				%\tkzLabelPoints[above left](D)
				%\tkzLabelSegment[right](M,N){\footnotesize$1$ m}
				%\tkzLabelSegment[left](A,B){\footnotesize$2$ m}
				%\tkzLabelSegment[right](C,E){\footnotesize$3{,}5$ m}
				\draw plot[smooth,tension=.65] coordinates{(B) (G) (F)};
				\draw plot[smooth,tension=.65] coordinates{(A) (H) (E)};
				\fill [pattern = north east lines] plot[smooth,tension=.65] coordinates{(A) (H) (E)} (0,0) --(-2,2)--(4,8)--(6,6)--cycle;
				\fill [draw=none, pattern = north east lines, color=white] (0,0) plot[smooth,tension=.65] coordinates{(B) (G) (F)} (-2,2)--(4,2)--cycle;
			\end{tikzpicture}
		}
		\noindent Thể tích khối tường cong là $V = S_{AEC} \cdot AB = \frac{14}{3} \cdot 2 = \dfrac{28}{3} = 9{,}3\left(\mathrm{~m}^3\right)$.	}
\end{ex}

% \Closesolutionfile{ansbook}
% \HetDe
% \label{De2}
% %
% \cleardoublepage
% \setcounter{page}{1}
% \rfoot{Trang \thepage/\pageref{DA2} - Đáp án trắc nghiệm Mã đề 2}
% \begin{center}
% 	\bfseries ĐÁP ÁN TRẮC NGHIỆM MÃ ĐỀ 2
% \end{center}

% \inputansbox{10}{ans/ansDe2-TN1}
% \inputansbox[3]{2}{ans/ansDe2-TN2}
% \inputansbox{3}{ans/ansDe2-TN3}
% \label{DA2}
% %

% \begin{name}
	{\tenchude}
	{TOÁN 12}
	{LỚP TOÁN THẦY PHÁT}
	{Thời gian: 90 phút - Không kể thời gian phát đề}
\end{name}
\TN
\Opensolutionfile{ans}[ans/ansDe3-TN1]
\begin{ex}%[EX-Ôn Tập TN 2025,  Đỗ Vũ Minh Thắng]%[2D4N1-1]
	Hàm số nào sau đây là một nguyên hàm của hàm số $f(x) = \sin x$?
	\choice
	{$F_{1}(x) = \sin x$}
	{$F_{2}(x) = -\sin x$}
	{$F_{3}(x) = \cos x$}
	{\True $F_{4}(x) = -\cos x$}
	\loigiai{
		Vì $(-\cos x)'=\sin x$ nên $F_{4}(x)$ là một nguyên hàm của hàm số $f(x)$.
	}
\end{ex}

\begin{ex}%[2D4N1-1]%[To 20 - Dot 17 - Chuong 4 - Bai 3 - CD - De 1 - TN]%[Nguyễn Hữu Duy]
	Nếu $F(x)$ là nguyên hàm của hàm số $f(x)$, thì tích phân của $f(x)$ trên đoạn $[a;b]$ được tính như thế nào?
	\choice
	{\True $F(b)-F(a)$}
	{$F(a)-F(b)$}
	{$\dfrac{F(b)}{F(a)}$}
	{$\dfrac{F(a)}{F(b)}$}
	\loigiai
	{ Ta có
		$\displaystyle\int_{a}^{b} f(x) \mathrm{\,d}x = F(b) - F(a)$.
	}
\end{ex}

\begin{ex}%[2D4N1-2]
	Nguyên hàm của hàm số $f(x)=x^4+x^2$ là
	\choice
	{\True $\dfrac{1}{5}x^5+\dfrac{1}{3}x^3+C$}
	{$x^4+x^2+C$}
	{$x^5+x^3+C$}
	{$4x^3+2x+C$}
	\loigiai{
		$\displaystyle\int{f(x)\mathrm{\,d}x}=\displaystyle\int{(x^4+x^2)\mathrm{\,d}x}$ $=\dfrac{1}{5}x^5+\dfrac{1}{3}x^3+C$.}
\end{ex}

\begin{ex}%[2D4N1-3]%[Dự án EX-TF-TLN lần 3 -Mui Doan]
	Họ tất cả các nguyên hàm của hàm số $f(x)=\sin x$ là
	\choice
	{$-\sin x+C$}
	{$\cos x+C$}
	{$\dfrac{1}{2}\sin^2x+C$}
	{\True $-\cos x+C$}
	\loigiai{
		Ta có $\displaystyle\int f(x)\mathrm{\,d}x=\displaystyle\int\sin x\mathrm{\,d}x=-\cos x+C$.}
\end{ex}

\begin{ex}%[2D4N1-4]
	Hàm số nào dưới đây là một nguyên hàm của hàm số $f(x)=\sqrt{x}-1$ trên $(0;+\infty)$?
	\choice
	{$F(x)=\dfrac{1}{2\sqrt{x}}$}
	{$F(x)=\dfrac{1}{2\sqrt{x}}-x$}
	{$F(x)=\dfrac{2}{3}\sqrt[3]{x^2}-x+1$}
	{\True $F(x)=\dfrac{2}{3}\sqrt{x^3}-x+2$}
	\loigiai{
		Ta có : $\displaystyle\int (\sqrt{x}-1)\mathrm{d}x=\frac{2}{3}\sqrt{x^3}-x+C$.}
\end{ex}

\begin{ex}%[Mức độ 1]%[BG-12-New-4in1, Phạm Đức Thiệu]%[2D4N1-4]
	Tìm họ nguyên hàm của hàm số $f(x)=2^x$.
	\choice
	{$\displaystyle\int f(x)\mathrm{\,d}x=2^x+C$}
	{\True $\displaystyle\int f(x)\mathrm{\,d}x=\dfrac{2^x}{\ln 2}+C$}
	{$\displaystyle\int f(x)\mathrm{\,d}x=2^x\ln 2+C$}
	{$\displaystyle\int f(x)\mathrm{\,d}x=\dfrac{2^{x+1}}{x+1}+C$}
	\loigiai{
		Ta có $\displaystyle\int f(x)\mathrm{\,d}x=\displaystyle\int 2^x\mathrm{\,d}x=\dfrac{2^x}{\ln 2}+C$.
	}
\end{ex}

\begin{ex}%[2D4H1-2]
	Khẳng định nào sau đây \textbf{đúng}?
	\choice
	{\True $\displaystyle\int{\left( x-\dfrac{1}{x} \right)^2\mathrm{d}x} =\dfrac{x^3}{3}-2x-\dfrac{1}{x}+C$ }
	{ $\displaystyle\int{\left( x-\dfrac{1}{x} \right)^2\mathrm{d}x}=\dfrac{x^3}{3}-2x+\dfrac{1}{x}+C$ }
	{ $\displaystyle\int{\left( x-\dfrac{1}{x} \right)^2\mathrm{d}x} =\dfrac{1}{3}\left( x-\dfrac{1}{x} \right)^3+C$ }
	{ $\displaystyle\int{\left( x-\dfrac{1}{x} \right)^2\mathrm{d}x}=\dfrac{1}{3}\left( x-\dfrac{1}{x} \right)^3\left( 1+\dfrac{1}{x^2} \right)+C$ }
	\loigiai{
		Ta có $\displaystyle\int{\left( x-\dfrac{1}{x} \right)^2\mathrm{d}x}=\int{\left( x^2-2+\dfrac{1}{x^2} \right)\mathrm{d}x}=\dfrac{x^3}{3}-2x-\dfrac{1}{x}+C$.
	}
\end{ex}

\begin{ex}%[2D4H1-3]
	Tìm nguyên hàm của hàm số $f(x)=\sin x +6x$ là
	\choice
	{\True $-\cos x+3x^{2}+C$}
	{$\cos x +6x^{2}+C$}
	{$\cos x +C$}
	{$\cos x +3x^{2}+C$}
	\loigiai{ Ta có $\displaystyle\int\limits (\sin x +6x)\mathrm{\,d}x=-\cos x+3x^{2}+C$.
	}
\end{ex}

\begin{ex}%[Dự án 2025 - đề cấu trúc mới, Nguyễn Kiều Nhã Tú]%[2D4N2-1]
	Cho hàm số $y=f(x)$ liên tục trên đoạn $[a;b]$. Mệnh đề nào dưới đây là \textbf{sai}?
	\choice
	{$\displaystyle\int\limits_a^b f(x)\mathrm{\,d} x=-\displaystyle\int\limits_b^a f(x)\mathrm{\,d}x$}
	{\True $\displaystyle\int\limits_a^b f(x)\mathrm{\,d} x=\displaystyle\int\limits_a^c f(x)\mathrm{\,d} x+\displaystyle\int\limits_c^b f(x)\mathrm{\,d}x$, $\forall c \in \mathbb{R}$}
	{$\displaystyle\int\limits_a^b f(x)\mathrm{\,d} x=\displaystyle\int\limits_a^b f(t)\mathrm{\,d}t$}
	{$\displaystyle\int\limits_a^a f(x)\mathrm{\,d}x=0$}
	\loigiai{
		Theo tính chất của tích phân,  $\displaystyle\int\limits_a^b f(x) \mathrm{\,d}x=\displaystyle\int\limits_a^c f(x)\mathrm{\,d} x+\displaystyle\int\limits_c^b f(x) \mathrm{\,d}x$, $\forall c\in \mathbb{R}$ chỉ đúng khi $c\in[a;b]$.
	}
\end{ex}

\begin{ex}%[2D4N2-2]%[To 20 - Dot 17 - Chuong 4 - Bai 3 - CD - De 3]%[Đình Nguyên]
	Tích phân $\displaystyle\int\limits_{0}^{1}\left(2 x^{2}-1\right) \mathrm{\,d}x$ có giá trị bằng
	\choice
	{$1$}
	{$2$}
	{$\dfrac{1}{3}$}
	{\True $\dfrac{-1}{3}$}
	\loigiai
	{
	Ta có $ \displaystyle
		\int\limits_{0}^{1}\left(2 x^{2}-1\right) \mathrm{\,d}x=\left.\left(\dfrac{2 x^{3}}{3}-x\right)\right|_{0} ^{1}=\dfrac{-1}{3}
	$.
	}
\end{ex}

\begin{ex}%[2D4N2-3]
	Giá trị của $\displaystyle\int\limits_0^{\frac{\pi}{2}}{\sin x\mathrm{\,d}x}$ bằng
	\choice
	{0}
	{\True 1}
	{$-1$}
	{$\dfrac{\pi}{2}$}
	\loigiai{
	Tính được $\displaystyle\int\limits_0^{\frac{\pi}{2}}{\sin x\mathrm{\,d}x}=-\cos x\Big|_0^{\frac{\pi}{2}}=1$.}
\end{ex}

\begin{ex}%[Tổ 20 - Chương 4 - - CD]%[Nguyễn Văn Sang]%[2D4N2-4]
	Biết $I=\displaystyle\int\limits_0^1 3^x \cdot 7^{x+1} \cdot \mathrm{\,d} x=\dfrac{a}{\ln b+\ln c}$, trong đó $a, b, c \in \mathbb{Z}$ và $b, c$ là số nguyên tố.  Khi đó $a+b+c$ bằng
	\choice
	{\True $150$}
	{$147$}
	{$157$}
	{$140$}
	\loigiai{
		Ta có \[
			I=\displaystyle\int\limits_0^1 3^x \cdot 7^{x+1} \cdot \mathrm{\,d} x=7 \cdot \displaystyle\int\limits_0^1 21^x \cdot \mathrm{\,d} x=7 \cdot \dfrac{21^x}{\ln 21}\bigg|_0 ^1=\dfrac{140}{\ln 21}=\dfrac{140}{\ln 3+\ln 7}.
		\]
		Suy ra $a=140$, $b=3$, $c=7$ và $a+b+c=150$.
	}
\end{ex}
\Closesolutionfile{ans}

\TNTF
\Opensolutionfile{ans}[ans/ansDe3-TN2]
\begin{ex}%[2D4H1-1]%[Đào Trung Kiên]
	Xét hàm số $f(x)=\left(ax+b\right)^n$ với $a\neq 0$, $n \in \mathbb{R}\setminus \{0,1\}$ thì
	\choiceTF{\True $\displaystyle\int f(x)\mathrm{\,d}x=\dfrac{1}{a(n+1)}(ax+b)^{n+1}+C$}
	{$f'(x)=\dfrac{1}{a(n+1)}(ax+b)^{n+1}+C$}
	{\True Nếu $F(x)$ là một nguyên hàm của $f(x)$ và thỏa mãn $F\left(-\dfrac{b}{a}\right)=0$ thì $F(0)=\dfrac{b^{n+1}}{a(n+1)}$}
	{$\displaystyle\int f(x)\mathrm{\,d}x=\dfrac{1}{(n+1)}(ax+b)^{n+1}+C$}
	\loigiai{Với $f(x)=\left(ax+b\right)^n$ với $a\neq 0$, $n \in \mathbb{R}\setminus \{0,1\}$ thì ta có
		\begin{itemchoice}
			\itemch $\displaystyle\int f(x)\mathrm{\,d}x=\dfrac{1}{a(n+1)}(ax+b)^{n+1}+C$.
			\itemch Ta có $f'(x)=n(ax+b)^{n-1}$ nên khẳng định trong đề bài là sai.
			\itemch Nếu $F(x)$ là một nguyên hàm của $f(x)$ suy ra $F(x)=\dfrac{1}{a(n+1)}(ax+b)^{n+1}+C$ và thỏa mãn $F\left(-\dfrac{b}{a}\right)=0$ thì $C=0$ nên $F(0)=\dfrac{b^{n+1}}{a(n+1)}$.
			\itemch $\displaystyle\int f(x)\mathrm{\,d}x=\dfrac{1}{a(n+1)}(ax+b)^{n+1}+C$ là đúng nên $\displaystyle\int f(x)\mathrm{\,d}x=\dfrac{1}{(n+1)}(ax+b)^{n+1}+C$ là sai.
		\end{itemchoice}

	}
\end{ex}

\begin{ex}%[2D4H2-2]%[Tổ 20 - Đợt 17 - Chương 4 - - CD - Đề 7]%[Bình]
	Xét tính đúng sai của các mệnh đề sau:
	\choiceTF
	{\True $\displaystyle\int\limits_1^2 \dfrac{1}{x}\mathrm{\,d}x=\ln 2$}
	{\True $\displaystyle\int\limits_0^a \left(\dfrac{1}{\cos^2 x}+3\right)\mathrm{\,d}x=\tan a+3a$}
	{$\displaystyle\int\limits_m^n \dfrac{1}{x\sqrt{x}}\mathrm{\,d}x=2\left(\sqrt{m}-\sqrt{n}\right)$}
	{Có $2$ giá trị $a$ để $\displaystyle\int\limits_1^a 3x(x-2)\mathrm{\,d}x=0$}
	\loigiai
	{
		\begin{itemchoice}
			\itemch Đúng. $\displaystyle\int\limits_1^2 \dfrac{1}{x}\mathrm{\,d}x=\ln |x|\bigg\rvert_1^2=\ln 2$.
			\itemch Đúng. $\displaystyle\int\limits_0^a \left(\dfrac{1}{\cos^2 x}+3\right)\mathrm{\,d}x=(\tan x+3x)\bigg\rvert_0^a=\tan a+3a$.
			\itemch Sai. $\displaystyle\int\limits_m^n \dfrac{1}{x\sqrt{x}}\mathrm{\,d}x=\displaystyle\int\limits_m^n x^{\tfrac{-3}{2}}\mathrm{\,d}x=\dfrac{-2}{\sqrt{x}}\bigg\rvert_m^n=-2\left(\dfrac{1}{\sqrt{n}}-\dfrac{1}{\sqrt{m}}\right)$.
			\itemch Sai.
			\begin{eqnarray*}
				\displaystyle\int\limits_1^a 3x(x-2)\mathrm{\,d}x=\displaystyle\int\limits_1^a \left(3x^2-6x\right)\mathrm{\,d}x=\left(x^3-3x^2\right)\bigg\rvert_1^a&=&a^3-3a^2+2=0\\
				&\Leftrightarrow&\hoac{&a=1+\sqrt{3}\\&a=1-\sqrt{3}\\&a=1.}
			\end{eqnarray*}
			Vậy có $3$ giá trị $a$ để $\displaystyle\int\limits_1^a 3x(x-2)\mathrm{\,d}x=0$.
		\end{itemchoice}
	}
\end{ex}
\Closesolutionfile{ans}

\TNSA
\Opensolutionfile{ans}[ans/ansDe3-TN3]
\begin{ex}%[2D4H1-1]%[Đào Trung Kiên]
	Cho $F(x)$ là một nguyên hàm của hàm số $f(x) = ax + \dfrac{b}{x^2}$ $(x \neq 0)$. Biết $F(-1) = 1$, $F(1) = 4$, $f(1) = 0$. Tính giá trị của $M = 2a - b$ (làm tròn tới hàng phần mười).
	\shortans[]{$4,5$}
	\loigiai{
	Ta có $\displaystyle\int f(x)\mathrm{\,d}x = \displaystyle\int \left(ax + \dfrac{b}{x^2}\right)\mathrm{\,d}x = \dfrac{ax^2}{2} - \dfrac{b}{x} + C$.\\
	Theo giả thiết, ta có hệ phương trình $\heva{&F(-1) = 1 \\ &F(1) = 4 \\ &f(1) = 0} \Leftrightarrow \heva{&a + b + C = 1 \\ &a - b + C = 4 \\ &a + b = 0} \Rightarrow \heva{&a = \dfrac{3}{2} \\ &b = -\dfrac{3}{2}\cdot}$\\
	Vậy $M = 2a - b = 3 + \dfrac{3}{2} = \dfrac{9}{2}=4{,}5$
	}
\end{ex}

\begin{ex}%[2D4H2-2]%[Tổ 20 - Đợt 17 - Chương 4 - - CD - Đề 7]%[Lê Thị Thanh Tuyền]
	Cho hàm số $f(x)=a x^2+b x+c$ thỏa mãn $\displaystyle\int\limits_0^1f(x) \mathrm{\,d}x=28, \displaystyle\int\limits_{-1}^1f(x) \mathrm{\,d} x=52$ và $\displaystyle\int\limits_0^2f(x) \mathrm{\,d} x=66$. Giá trị của biểu thức $P=a^b. c$ là
	\shortans{$2025$}

	\loigiai{
		\begin{itemize}
			\item Ta có: $\displaystyle\int f(x) \mathrm{\,d} x=\displaystyle\int\left(a x^2+b x+c\right) d x=\dfrac{a}{3} x^3+\dfrac{b}{2} x^2+c x+C$.
			\item  $\displaystyle\int\limits_0^1f(x)\mathrm{\,d} x=\left.28\Rightarrow\left(\dfrac{a}{3} x^3+\dfrac{b}{2} x^2+c x\right)\right|_0 ^1=28\Leftrightarrow \dfrac{1}{3} a+\dfrac{1}{2} b+c=28 \quad (1)$.
			\item 	$\displaystyle\int\limits_{-1}^1f(x) \mathrm{\,d} x=\left.52\Rightarrow\left(\dfrac{a}{3} x^3+\dfrac{b}{2} x^2+c x\right)\right|_{-1} ^1=52\Leftrightarrow \dfrac{2}{3} a+2c=52 \quad (2)$.
			\item 	$\displaystyle\int\limits_0^2f(x)\mathrm{\,d} x=\left.66\Rightarrow\left(\dfrac{a}{3} x^3+\dfrac{b}{2} x^2+c x\right)\right|_0 ^2=66\Leftrightarrow \dfrac{8}{3} a+2b+2c=66 \quad (3)$.
			\item 	Từ $(1), (2)$ và $(3)$ ta có hệ phương trình: $\heva{&\dfrac{1}{3} a+\dfrac{1}{2} b+c=28\\& \dfrac{2}{3} a+2c=52\\& \dfrac{8}{3} a+2b+2c=66}\Leftrightarrow\heva{&a=3\\&b=4\\&c=25.}$
			\item Vậy $P=3^4.25=2025$.
		\end{itemize}
	}
\end{ex}

\begin{ex}%[2D4H3-1]%[Tổ 21 - Đợt 17 - Chương 4 - - Cánh Diều - Đề 2]
	Cho hàm số $f(x)=\heva{&7-4 x^{3} \text { khi } 0 \leq x \leq 1 \\ &4-x^{2} \text { khi } x>1}$. Tính diện tích hình phẳng giới hạn bởi đồ thị hàm số $f(x)$ và các đường thẳng $x=0$, $x=3$, $y=0$.
	\shortans{$10$}
	\loigiai
	{\immini{Ta có
			\begin{align*}
				S & =\displaystyle\int_{0}^{1} (7-4 x^{3}) \mathrm{d}x+\displaystyle\int_{1}^{2} (4-x^{2}) \mathrm{d}x+\displaystyle\int_{2}^{3}( x^{2}-4) \mathrm{d}x \\
				  & =7 x-\left.x^{4}\right|_{0} ^{1}+\left.\left(4 x-\dfrac{x^{3}}{3}\right)\right|_{1} ^{2}+\left.\left(\dfrac{x^{3}}{3}-4 x\right)\right|_{2} ^{3}   \\
				  & =6+4-\dfrac{7}{3}-3-\dfrac{8}{3}+8=10.
			\end{align*}}{
			\begin{tikzpicture}[scale=.5,font=\footnotesize, line join=round,line cap=round,>=stealth]
				\draw[->] (-1,0)--(4,0) node[below] {$x$};
				\draw[->] (0,-6)--(0,8) node[left] {$y$};
				% Tô vùng giữa x=2 và x=3, trục Ox và đồ thị 4-x^2
				\fill[fill=orange, opacity=0.5]
				(2,0) -- plot[domain=2:3, samples=100] (\x,{4-(\x)^2}) -- (3,0) -- cycle;
				\fill[fill=orange, opacity=0.5]
				(1,0) -- plot[domain=1:2, samples=100] (\x,{4-(\x)^2}) -- (2,0) -- cycle;
				\fill[fill=orange, opacity=0.5]
				(0,0) -- plot[domain=0:1, samples=100] (\x,{7-4*(\x)^3}) -- (1,0) -- cycle;
				% Vẽ các đồ thị hàm số
				\draw[color=blue, samples=100, domain=0:1] plot(\x,{7-4*(\x)^3});
				\draw[color=blue, samples=100, domain=1:3] plot(\x,{4-(\x)^2});
				% Đường nét đứt và chú thích các điểm
				\draw[dashed] (0,3)--(1,3)--(1,0) (3,0)--(3,-5)--(0,-5);
				\node[left] at (0,3) {$3$};
				\node[left] at (0,7) {$7$};
				\node[left] at (0,-5) {$-5$};
				\node[below left] at (0,0) {$O$};
				\node[below] at (1,0) {$1$};
				\node[below left] at (2,0) {$2$};
				\node[below left] at (3,0) {$3$};
			\end{tikzpicture}
		}
	}
\end{ex}

\begin{ex}%[Nguyễn Tuấn, dự án sáng tác đề 12]%[2D4H3-3]
	\immini
	{
		Một vật trang trí có dạng khối tròn xoay tạo thành khi quay miền $(R)$ được giới hạn bởi đường gấp khúc $DABFE$ và cung tròn $ED$ (phần gạch chéo trong hình bên) xung quanh trục $AB$. Biết $ABCD$ là hình chữ nhật cạnh $AB =3$ cm, $AD=2$ cm; $F$ là trung điểm của $BC$; điểm $E$ cách $AD$ một đoạn bằng $1$ cm. Tính thể tích của vật trang trí đó, làm tròn kết quả đến hàng phần mười, đơn vị cm$^3$.
		\shortans{$16{,}5$}
	}
	{
		\begin{tikzpicture}[scale=1.2, line join=round, line cap=round,>=stealth,font=\footnotesize]
			\path (0,0) coordinate (B)	(2,0) coordinate (C) (2,3) coordinate (D) (0,3) coordinate (A) ($(B)!.5!(C)$) coordinate (F) ($(F)+(0,2)$) coordinate (E) ($(A)!.5!(D)$) coordinate (I) ;
			\draw (A)--(B)--(C)--(D)--cycle (E)--(F);
			\draw[dashed] (E)--(I);
			\foreach \d/\g in {A/90,B/-90,C/0,D/90,F/-90,E/-60}
			\draw[fill=black] (\d) circle (1pt) +(\g:0.2) node{$\d$};
			\draw (E) arc (-90:0:1cm);
			\fill[color=gray,opacity=0.5] (A)--(B)--(F)--(I)--cycle (E) arc (-90:0:1cm)--(D)--(I)--cycle;
		\end{tikzpicture}
	}
	\loigiai{
		\immini
		{
			Gắn hệ trục tọa độ như hình vẽ bên.\\
			Quay hình chữ nhật $GEFB$ xung quanh trục $AB$ ta được khối trụ có bán kính đáy bằng $1$ và chiều cao bằng $2$.\\
			Phương trình đường tròn tâm $I$ bán kính bằng $1$ là $x^2+(y-1)^2=1$, suy ra phương trình của cung $DE$ là $y=1+\sqrt{1-x^2}$.\\
			Vây thể tích của vật trang trí là
			\[V=\pi \displaystyle\int\limits_{0}^{1} \left(1+\sqrt{1-x^2}\right)^2 \mathrm{\,d}x+\pi \cdot 1^2\cdot 2\approx 16{,}5\ (\mathrm{cm}^3).\]
		}
		{
			\begin{tikzpicture}[scale=1.2, line join=round, line cap=round,>=stealth,font=\footnotesize]
				\draw[->] (-0.5,0)--(4,0) node[below]{$x$};
				\draw[->] (0,-0.5)--(0,3) node[left]{$y$};
				\path (0,0) coordinate (A)	(3,0) coordinate (B) (3,2) coordinate (C) (0,2) coordinate (D) ($(B)!.5!(C)$) coordinate (F) ($(F)-(2,0)$) coordinate (E) ($(A)!.5!(D)$) coordinate (I) (1,0) coordinate (G) ;
				\draw (A)--(B)--(C)--(D)--cycle (E)--(F);
				\draw[dashed] (E)--(I) (E)--(G);
				\foreach \d/\g in {A/-120,B/-90,C/0,D/120,F/0,E/-60,I/180,G/-90}
				\draw[fill=black] (\d) circle (1pt) +(\g:0.2) node{$\d$};
				\draw (E) arc (0:90:1cm);
				\fill[color=gray,opacity=0.5] (A)--(B)--(F)--(I)--cycle (E) arc (0:90:1cm)--(D)--(I)--cycle;
			\end{tikzpicture}
		}
	}
\end{ex}

\Closesolutionfile{ans}

\TNTF
\begin{ex}%[2D4H2-2]
	Tính tích phân $A=\displaystyle\int\limits_{-3}^5 {\left(|x+2|-|x-2|\right)}\mathrm{\,d}x$.

	\loigiai{
	$A=\displaystyle\int\limits_{-3}^5 {\left(|x+2|-|x-2|\right)}\mathrm{\,d}x$.\\
	Ta có bảng xét dấu để phá trị tuyệt đối:
	\begin{center}
		\begin{tikzpicture}
			\tkzTabInit[nocadre=false,lgt=3,espcl=2.5,deltacl=0.5]
			{$x$/.6,$|x+2|$/.6,$|x-2|$/.6,$|x+2|+|x-2|$/.6}
			{$-3$,$-2$,$2$,$5$}
			\tkzTabLine{,-x-2,0,x+2,|,-x-2,}
			\tkzTabLine{,-x+2,|,-x+2,0,x-2,}
			\tkzTabLine{,-4,|,2x,|,4,}
		\end{tikzpicture}
	\end{center}
	Khi đó
	\begin{eqnarray*}
		A=	\displaystyle\int\limits_{-3}^5 {\left(|x+2|-|x-2|\right)}\mathrm{\,d}x
		&=& \displaystyle\int\limits_{-3}^{-2} {(-4)}\mathrm{\,d}x +
		\displaystyle\int\limits_{-2}^2 {2x}\mathrm{\,d}x +
		\displaystyle\int\limits_2^5 {4}\mathrm{\,d}x\\
		&=& -4x \Big|_{-3}^{-2} + x^2 \Big|_{-2}^2 + \cdot 4x \Big|_2^5=8.
	\end{eqnarray*}
	}
\end{ex}


\begin{ex}%[2D4V1-4]
	Cho hàm số $ y=f(x)$ đồng biến và có đạo hàm liên tục trên $\mathbb{R}$ thỏa mãn $\left(f'(x)\right)^2=f(x)\cdot\mathrm{e}^x$, $\forall x\in\mathbb{R}$ và $f(0)=2$. Tính $ f(2)$. (Kết quả làm tròn đến hàng phần trăm).
	% \shortans{$9{,}81$}
	\loigiai{
	Vì hàm số $ y=f(x)$ đồng biến và có đạo hàm liên tục trên $\mathbb{R}$ đồng thời $ f(0)=2$ nên $f'(x)\ge 0$ và $ f(x)>0$ với mọi $ x\in\left[0;+\infty\right)$.\\
	Từ giả thiết $\left(f'(x)\right)^2=f(x)\cdot \mathrm{e}^x$, $\forall x\in\mathbb{R}$ suy ra $f'(x)=\sqrt{f(x)}\cdot\mathrm{e}^{\tfrac{x}{2}}$, $\forall x\in\left[0;+\infty\right).$\\
	Do đó $\dfrac{f'(x)}{2\sqrt{f(x)}}=\dfrac{1}{2}{\mathrm{e}^{\tfrac{x}{2}}}$, $\forall x\in\left[0;+\infty\right).$\\
	Lấy nguyên hàm hai vế, ta được $\sqrt{f(x)}=e^{\tfrac{x}{2}}+C$, $\forall x\in\left[0;+\infty\right)$ với $C$ là hằng số.\\
	Kết hợp với $ f(0)=2$, ta được $C=\sqrt{2}-1$.\\
	Suy ra $ f(2)=\left(\mathrm{e}+\sqrt{2}-1\right)^2\approx 9{,}81$.}
\end{ex}

\begin{ex}%[EX-Ôn Tập TN 2025, Võ Thanh Phong]%[2D4C3-2]
	Một cổng có đạng hình parabol với chiều cao $8$ m, chiều rộng chân đế $8$ m. Người ta căng hai sợi dây trang trí $AB$,  $CD$ nằm ngang, đồng thời chia cổng thành ba phần sao cho hai phần ở phía trên có diện tích bằng nhau. Tỉ số $\dfrac{CD}{AB}$ bằng bao nhiêu (làm tròn kết quả đến hàng phần trăm)?
	\begin{center}
		\begin{tikzpicture}[scale=0.7, font=\footnotesize, line join=round, line cap=round,>=stealth]
			\begin{scope}
				\clip (-4,-4.5) rectangle (4,0);
				\draw [smooth,domain=-5:4, samples=200] plot (\x, {-0.5*(\x)^2});
			\end{scope}
			\draw [<->] (3.3,0)--(3.3,-4.5);
			\node[right] at (3.3,-2){$8\,\,m$};
			\draw [<->] (-3,-4.5)--(3,-4.5);
			\node[below] at (0,-4.5){$8\,\,m$};
			\node[left] at (-1.8,-1.62){$A$};
			\node[right] at (1.8,-1.62){$B$};
			\node[left] at (-2.5,-3.125){$C$};
			\node[right] at (2.5,0-3.125){$D$};
			\draw [-] (-1.8,-1.62)--(1.8,-1.62);
			\draw [-] (-2.5,-3.125)--(2.5,0-3.125);
			\draw [dashed] (0,0)--(4,0);
			\node[below] at (current bounding box.south){\textit{Hình 5}};
		\end{tikzpicture}
	\end{center}
	% \shortans{ $1{,}26$}
	\loigiai{
	\immini{Gắn hệ trục tọa độ $Oxy$ vào cổng parabol như hình bên với trục $Oy$ trùng với đường đối xứng của parabol, gốc $O$ nằm ở đỉnh của parabol, đơn vị trên mỗi trục tính theo mét. Khi đó, phương trình parabol có dạng $y=ax^2$.\\
		Vì parabol đi qua điểm có toạ độ $(-4 ;-8)$ nên $a=-\dfrac{1}{2}$. Suy ra phương trình parabol là $y=-\dfrac{1}{2} x^2$.\\}{
		\begin{tikzpicture}[scale=0.7, font=\footnotesize, line join=round, line cap=round,>=stealth]
			\draw[->] (-4,0) --(4,0) node[below]{$x$};
			\draw[->] (0,-5) --(0,1) node[left]{$y$};
			\draw (0,0) node[above left=-3pt]{$O$};
			\begin{scope}
				\clip (-4,-4.5) rectangle (4,0);
				\draw [smooth,domain=-5:4, samples=200] plot (\x, {-0.5*(\x)^2});
			\end{scope}
			\draw [<->] (3.3,0)--(3.3,-4.5);
			\node[right] at (3.3,-2){$8\,\,m$};
			\draw [<->] (-3,-4.5)--(3,-4.5);
			\node[below] at (0,-4.5){$8\,\,m$};
			\node[left] at (-1.8,-1.62){$A$};
			\node[right] at (1.8,-1.62){$B$};
			\node[left=-3pt] at (-2.5,-3.125){$C$};
			\node[right] at (2.5,0-3.125){$D$};
			\draw [-] (-1.8,-1.62)--(1.8,-1.62);
			\draw [-] (-2.5,-3.125)--(2.5,0-3.125);
			\draw [dashed] (0,0)--(4,0) (1.8,-1.62)--(1.8,0) (2.5,0-3.125)--(2.5,0) (-3,-4.5)--(-3,0);
			\node[above] at (1.8,0){$x_1$};
			\node[above] at (2.5,0){$x_2$};
			\node[above] at (-3,0){$-4$};
		\end{tikzpicture}}
	Giả sử $B$ có hoành độ $x_1$,  $D$ có hoành độ $x_2$. Khi đó, phương trình đường thẳng $AB$ là $y=-\dfrac{1}{2} x_1^2$, phương trình đường thẳng $CD$ là $y=-\dfrac{1}{2} x_2^2$.\\
	Diện tích hình phẳng giới hạn bởi parabol và đường thẳng $AB$ là
	\[
		S_1=2\displaystyle\int\limits_0^{x_1}\left[-\dfrac 12x^2-\left(-\dfrac 12x_1^2\right)\right]\mathrm{\,d} x=\left.2\left(-\dfrac{x^3}6+\dfrac{x_1^2}2x\right)\right|_0^{x_1}=\dfrac 23x_1^3\,\,\left(\mathrm{m}^2\right).
	\]
	Diện tích hình phẳng giới hạn bởi parabol và đường thẳng $CD$ là
	\[
		S_2=2\displaystyle\int\limits_0^{x_2}\left[-\dfrac 12x^2-\left(-\dfrac 12x_2^2\right)\right]\mathrm{\,d} x=\left.2\left(-\dfrac{x^3}6+\dfrac{x_2^2}2x\right)\right|_0^{x_2}=\dfrac 23x_2^3\,\,\left(\mathrm{m}^2\right).
	\]
	Theo giả thiết, ta có  $S_2=2S_1\Leftrightarrow x_2^3=2x_1^3\Leftrightarrow\dfrac{x_2}{x_1}=\sqrt[3]2\approx 1{,}26$.\\
	Khi đó, $\dfrac{CD}{AB}=\dfrac{2x_2}{2x_1}\approx 1{,}26$.
	}
\end{ex}

% \Closesolutionfile{ansbook}
% \HetDe
% \label{De3}
% %
% \cleardoublepage
% \setcounter{page}{1}
% \rfoot{Trang \thepage/\pageref{DA3} - Đáp án trắc nghiệm Mã đề 3}
% \begin{center}
% 	\bfseries ĐÁP ÁN TRẮC NGHIỆM MÃ ĐỀ 3
% \end{center}

% \inputansbox{10}{ans/ansDe3-TN1}
% \inputansbox[3]{2}{ans/ansDe3-TN2}
% \inputansbox{3}{ans/ansDe3-TN3}
% \label{DA3}
%

% \begin{name}
	{\tenchude}
	{TOÁN 12}
	{LỚP TOÁN THẦY PHÁT}
	{Thời gian: 90 phút - Không kể thời gian phát đề}
\end{name}
\TN
\Opensolutionfile{ans}[ans/ansDe4-TN1]
\begin{ex}%[2D4N1-1]
	Cho hàm số $F(x)$ là một nguyên hàm của hàm số $f(x)$ trên $K$. Các mệnh đề sau, mệnh đề nào \textbf{sai}.
	\choice
	{$\displaystyle\int{f(x)\mathrm{\,d}x=}F(x)+C$}
	{$\displaystyle{\left(\displaystyle\int{f(x)\mathrm{\,d}x}\right)'}=f(x)$}
	{\True $\displaystyle{\left(\displaystyle\int{f(x)\mathrm{\,d}x}\right)'}=f'(x)$}
	{$\displaystyle{\left(\displaystyle\int{f(x)\mathrm{\,d}x}\right)'}=F'(x)$}
	\loigiai{
		Ta có $\displaystyle\int{f(x)\mathrm{\,d}x=}F(x)+C\Leftrightarrow F'(x)=f(x)$ nên phương án $\left(\displaystyle\int{f(x)\mathrm{\,d}x}\right)'=f'(x)$ sai.}
\end{ex}

\begin{ex}%[2D4N1-1]%[To 20 - Dot 17 - Chuong 4 - Bai 3 - CD - De 1 - TN]%[Nguyễn Hữu Duy]
	Nếu hàm số $f(x)$ liên tục trên đoạn $[a;b]$ và $c$ là số thực tùy ý thuộc đoạn $[a;b]$, thì tính chất nào sau đây đúng?
	\choice
	{\True $\displaystyle\int_a^b f(x) \mathrm{\,d}x = \displaystyle\int_a^c f(x) \mathrm{\,d}x + \displaystyle\int_c^b f(x)\mathrm{\,d}x$}
	{$\displaystyle\int_a^b f(x)\mathrm{\,d}x = \displaystyle\int_a^c f(x) \mathrm{\,d}x - \displaystyle\int_c^b f(x) \mathrm{\,d}x$}
	{$\displaystyle\int_a^b f(x)\mathrm{\,d}x = \displaystyle\int_a^c f(x)\mathrm{\,d}y + \displaystyle\int_c^b f(x) \mathrm{\,d}z$}
	{$\displaystyle\int_a^b f(x)\mathrm{\,d}x = \displaystyle\int_a^c f(x)\mathrm{\,d}y - \displaystyle\int_c^b f(x)\mathrm{\,d}z$}
	\loigiai{
		Theo định nghĩa tích phân ta có $\displaystyle\int_a^b f(x) \mathrm{\,d}x = \displaystyle\int_a^c f(x) \mathrm{\,d}x + \displaystyle\int_c^b f(x)\mathrm{\,d}x$.
	}
\end{ex}

\begin{ex}%[2D4N1-2]
	Cho hàm số $f(x)=x^2+4$. Mệnh đề nào sau đây đúng?

	\choice
	{$\displaystyle{\displaystyle\int f(x)\mathrm{\,d}x=2 x+C}$}
	{$\displaystyle{\displaystyle\int f(x)\mathrm{\,d}x=x^2+4 x+C}$}
	{\True $\displaystyle{\displaystyle\int f(x)\mathrm{\,d}x=\dfrac{x^3}{3}+4 x+C}$}
	{$\displaystyle{\displaystyle\int f(x)\mathrm{\,d}x=x^3+4 x+C}$}
	\loigiai{
		Ta có $f(x)=x^2+4 $ nên $ \displaystyle\int f(x)\mathrm{\,d}x=\dfrac{x^3}{3}+4 x+C$.}
\end{ex}

\begin{ex}%[2D4N1-3]
	Tìm nguyên hàm của hàm số $ f(x)=\cos 3x$.
	\choice
	{ $\displaystyle\int{\cos 3x\mathrm{d}x=3\sin 3x+C}$}
	{\True $\displaystyle\int{\cos 3x\mathrm{d}x=\dfrac{\sin 3x}{3}+C}$}
	{ $\displaystyle\int{\cos 3x\mathrm{d}x=\sin 3x+C}$}
	{$\displaystyle\int{\cos 3x\mathrm{d}x=-\dfrac{\sin 3x}{3}+C}$}
	\loigiai{
		Ta có:$\displaystyle\int{\cos 3x\mathrm{d}x=\dfrac{\sin 3x}{3}+C}$.
	}
\end{ex}

\begin{ex}%[2D4N1-4]
	Hàm số $F(x)=\mathrm{e}^{2x}$ là một nguyên hàm của hàm số nào dưới đây?
	\choice
	{$f_4(x)=\dfrac{1}{2}\mathrm{e}^{2x}$}
	{$f_1(x)=\mathrm{e}^{2x}$}
	{$f_2(x)=\mathrm{e}^{x^2}$}
	{\True $f_3(x)=2\mathrm{e}^{2x}$}
	\loigiai{
		Ta có $F'(x)=f(x)$ nên $f(x)=\left(\mathrm{e}^{2x}\right)^\prime=2\mathrm{e}^{2x}$.}
\end{ex}

\begin{ex}%[BG-12-4in1, Phạm Đức]%[2D4N1-4]
	Họ nguyên hàm của hàm số $f(x)=2024^x$ là
	\choice
	{\True $\dfrac{2024^x}{\ln 2024}+C$}
	{$2024^x\ln 2024+C$}
	{$2024^x+C$}
	{$2024^x\ln x+C$}
	\loigiai{

	}
\end{ex}

\begin{ex}%[2D4H1-2]
	Tìm nguyên hàm của hàm số $f(x)=(5x+3)^5$.
	\choice
	{$(5x+3)^6+C$}
	{$(5x+3)^4+C$}
	{\True $\dfrac{(5x+3)^6}{30}+C$}
	{$\dfrac{(5x+3)^4}{30}+C$}
	\loigiai{
		$f(x)=(5x+3)^5$ $\displaystyle \Rightarrow \displaystyle\int{f(x)\mathrm{\,d}x=}\displaystyle\int{(5x+ 3)^5\mathrm{\,d}x=}\dfrac{1}{5}\cdot \dfrac{(5x+3)^6}{6}+C=\dfrac{(5x+3)^6}{30}+C$.}
\end{ex}

\begin{ex}%[2D4H1-3]
	Họ nguyên hàm của hàm số $f(x)=\cos 2x$ là
	\choice
	{\True $\dfrac{1}{2} \sin 2x +C$}
	{$-2 \sin 2x +C$}
	{$-\dfrac{1}{2} \sin 2x +C$}
	{$2 \sin 2x +C$}
	\loigiai{
		Ta có  $\displaystyle\int\limits \cos 2x\mathrm{\,d}x=\dfrac{1}{2} \sin 2x+C$.
	}
\end{ex}

\begin{ex}%[2D4N2-1]
	Cho hàm số $f(t)$ liên tục trên $K$ và $a$, $b\in K$, $F(t)$ là một nguyên hàm của $ f(t)$ trên $K$. Chọn khẳng định \textbf{sai} trong các khẳng định sau.
	\choice
	{\True $F(a)-F(b)=\displaystyle\int\limits_a^b f(t)\mathrm{d}t$}
	{$\displaystyle\int\limits_a^bf(t)\mathrm{d}t=F(t)\big|^b_a$}
	{$\displaystyle\int\limits_a^bf(t)\mathrm{d}t=\left(\displaystyle\int f(t)\mathrm{d}t\right)\bigg|^b_a$}
	{$\displaystyle\int\limits_a^bf(x)\mathrm{d}x=\displaystyle\int\limits_a^bf(t)\mathrm{d}t$}
	\loigiai{
		Theo tính chất của tích phân.
		Ta có  $\displaystyle\int\limits_a^b f(t)\mathrm{d}t=F(b)-F(a)$.}
\end{ex}

\begin{ex}%[2D4N2-2]%[Tổ 20 - Đợt 17 - Chương 4 - - CD]%[Phạm Hà Giang]
	Tính tích phân $\displaystyle\int\limits_1^2 \dfrac{1}{x}\mathrm{\,d}x$
	\choice
	{$0$}
	{\True $\ln 2$}
	{$\dfrac{1}{2}$}
	{$\dfrac{-1}{2}$}
	\loigiai
	{
		$\displaystyle\int\limits_1^2 \dfrac{1}{x}\mathrm{\,d}x=\left. \ln \left| x\right| \right| _1^2 =\ln 2$.
	}
\end{ex}

\begin{ex}%[Nguyễn Tuấn, dự án sáng tác đề 12]%[2D4N2-3]
	Giá trị của $\displaystyle\int\limits_0^{\frac{\pi}{2}} \cos x\mathrm{\,d}x$ bằng
	\choice
	{$0$}
	{\True $1$}
	{$\dfrac{\pi}{2}$}
	{$\pi$}
	\loigiai
	{
	Ta có $\displaystyle\int\limits_0^{\frac{\pi}{2}} \cos x\mathrm{\,d}x = \sin x\Big|_0^{\frac{\pi}{2}} = \sin\dfrac{\pi}{2}-\sin 0 = 1$.
	}
\end{ex}

\begin{ex}%[Tổ 20 - Chương 4 - - CD]%[Nguyễn Văn Sang]%[2D4N2-4]
	Biết $I=\displaystyle\int\limits_0^1 3^x \cdot 4^{2 x} \cdot \mathrm{\,d} x=\dfrac{a}{\ln 48}$.  Khi đó $a+1$ bằng
	\choice
	{\True $48$}
	{$46$}
	{$47$}
	{$49$}
	\loigiai{
		Ta có	\[
			I=\displaystyle\int\limits_0^1 3^x \cdot 4^{2 x}\mathrm{\,d}x=\displaystyle\int\limits_0^1 3^x \cdot 16^x d x=\displaystyle\int\limits_0^1 48^x\mathrm{\,d} x=\dfrac{48^x}{\ln 48}\bigg|_0 ^1=\dfrac{47}{\ln 48}.
		\]
		Suy ra $a=47$ và $a+1=48$.
	}
\end{ex}
\Closesolutionfile{ans}

\TNTF
\Opensolutionfile{ans}[ans/ansDe4-TN2]
\begin{ex}%[2025-TLDH- Huỳnh Xuân Tín]%[2D4N1-4]
	Cho $I_1=\displaystyle\int\left(\mathrm e^x+\dfrac{1}{x^2}\right) \mathrm{d}x$ và $I_2=\displaystyle\int\left( \mathrm e^{2x-1}-\dfrac{1}{x^2}\right) \mathrm{d}x$.
	\choiceTF
	{\True  $I_1=\mathrm{e}^x-\dfrac{1}{x}+C$}
	{$I_2=\dfrac{\mathrm{e}^{2x-1}}{2}+\ln |x|+C$ }
	{\True $I_1+I_2=\mathrm{e}^x+\dfrac{{\mathrm{e}^{2x-1}}}{2}+C$ }
	{Gọi $F(x)$ là nguyên hàm của hàm số $f(x)$, với $f(x)=\mathrm{e}^x+\dfrac{1}{x^2}$. Nếu $F(1)=\mathrm{e}$ thì $F(\ln 2)=1-\dfrac{1}{\ln 2}$}
	\loigiai{
		\begin{itemchoice}
			\itemch \textbf{Đúng.}\\
			Vì $I_1=\displaystyle\int(\mathrm{e}^x+\dfrac{1}{x^2})\mathrm{d}x=\mathrm{e}^x-\dfrac{1}{x}+C$.
			\itemch \textbf{Sai.}\\
			Ta có $I_2=\displaystyle\int\left( \mathrm{e}^{2x-1}-\dfrac{1}{x^2}\right) \mathrm{d}x=\dfrac{\mathrm{e}^{2x-1}}{2}+\dfrac{1}{x}+C$.
			\itemch \textbf{Đúng.}\\
			Ta có
			\begin{eqnarray*}
				I_1+I_2	&= & f(x)=g(x)\displaystyle\int( \mathrm{e}^x+\dfrac{1}{x^2} )\mathrm{\,d}x+\displaystyle\int( {e^{2x-1}}-\dfrac{1}{x^2} )\mathrm{\,d}x\\
				&=& \displaystyle\int\left(\mathrm{e}^x+\mathrm{e}^{2x-1} \right) \mathrm{\,d}x\\
				&= &\mathrm{e}^x+\dfrac{\mathrm{e}^{2x-1}}{2}+C.
			\end{eqnarray*}
			\itemch \textbf{Sai.}\\
			Ta có $I_1=\displaystyle\int(\mathrm{e}^x+\dfrac{1}x^2)\mathrm{d}x=\mathrm{e}^x-\dfrac{1}{x}+C$. Vì $F(1)=\mathrm{e}\Rightarrow \mathrm{e}-1+C=\mathrm{e}\Rightarrow C=1$.\\
			$F(x)=\mathrm{e}^x-\dfrac{1}{x}+1\Rightarrow F(\ln 2)=\mathrm{e}^{\ln 2}-\dfrac{1}{\ln 2}+1=2-\dfrac{1}{\ln 2}+1=3-\dfrac{1}{\ln 2}$.
		\end{itemchoice}
	}
\end{ex}

\begin{ex}%[Dự án 2025 - đề cấu trúc mới, Nguyễn Kiều Nhã Tú]%[2D4H2-3]
	Cho hàm số $y=f(x)$. Biết $f'(x)=2\cos^2 x + 3$, $\forall x\in \mathbb{R}$.
	\choiceTF
	{$f'(x)>0$ với $\forall x\in\mathbb{R}$ nên $f(x)>0$,  $\forall x\in \mathbb{R}$}
	{$f'(x)=\displaystyle\int f(x)\mathrm{\,d}x$}
	{\True $f(x)=\dfrac{1}{2}\sin 2x+4x+C$}
	{\True Biết $f(0)=4$. Khi đó $\displaystyle\int\limits_0^{\frac{\pi}{4}}f(x) \mathrm{\,d}x$ bằng $\dfrac{\pi^2+8\pi+2}{8}$}
	\loigiai{
		\begin{itemchoice}
			\itemch \textbf{Sai}. Vì vì đạo hàm không có tính chất này.
			\itemch \textbf{Sai}. Vì $f(x)=\displaystyle\int f'(x)\mathrm{\,d}x$.
			\itemch \textbf{Đúng}. Vì
			\begin{align*}
				f(x) & =\displaystyle\int f'(x)\mathrm{\,d}x=\displaystyle\int\left(2\cos^2 x + 3\right)\mathrm{\,d}x \\
				     & =\displaystyle\int \left(2\cdot\dfrac{1+\cos 2x}{2}+3\right)  \mathrm{\,d}x
				= \displaystyle\int \left(\cos 2x+4\right)\mathrm{\,d}x                                               \\
				     & = \dfrac{1}{2} \sin{2x} + 4x + C.
			\end{align*}
			\itemch \textbf{Đúng}.
			Ta có $f(x)=\dfrac{1}{2}\sin 2x + 4x + C$.
			Do $f(0)=4 \Rightarrow C=4$ \\
			Vậy $f(x)=\dfrac{1}{2}\sin 2x + 4x + 4$ nên \\
			$\displaystyle\int\limits_0^{\frac{\pi}{4}}f(x)\mathrm{\,d}x = \displaystyle\int\limits_0^{\frac{\pi}{4}}\left(\dfrac{1}{2}\sin 2x+ 4x+4\right)\mathrm{\,d}x
				=\left(-\dfrac{1}{4}\cos 2x+2x^2+4x\right)\Big|_0^{\frac{\pi}{4}}
				= \dfrac{\pi^2+8\pi+2}{8}$.
		\end{itemchoice}
	}
\end{ex}
\Closesolutionfile{ans}

\TNSA
\Opensolutionfile{ans}[ans/ansDe4-TN3]
\begin{ex}%[2D4H1-1]%[Đào Trung Kiên]
	Giả sử $F(x)$ là một nguyên hàm của hàm số $f(x)=\mathrm{e}^x$, biết $F(0)=4$. Tìm $F(1)$ (làm tròn kết quả tới phần mười).
	\shortans[]{$5,7$}
	\loigiai{
		Do $F(x)$ là một nguyên hàm của $f(x)=\mathrm{e}^x$ nên $F(x)=\mathrm{e}^x+C$.\\
		Lại có $F(0)=4$ nên $C=3$ hay $F(x)=\mathrm{e}^x+3$ nên $F(1)=\mathrm{e}+3\approx 5{,}7$.
	}
\end{ex}

\begin{ex}%[2D4H2-2]%[Tổ 20 - Đợt 17 - Chương 4 - - CD - Đề 7]%[Lê Thị Thanh Tuyền]
	Có bao nhiêu giá trị nguyên của $a$ để $\displaystyle\int\limits_1^a(2x-3) \mathrm{\,d} x \leq 6$?
	\shortans{$6$}

	\loigiai{
		\begin{itemize}
			\item Ta có: $\displaystyle\int\limits_1^a(2x-3)\mathrm{\,d} x=\left.\left(x^2-3x\right)\right|_1 ^a=a^2-3a+2$.
			\item Khi đó: $\displaystyle\int\limits_1^a(2x-3) \mathrm{\,d} x \leq 6\Leftrightarrow a^2-3a+2\leq 6\Leftrightarrow-1\leq a \leq 4$
			\item	Mà $a$ là số nguyên nên $a \in\{-1; 0; 1; 2; 3; 4\}$.
			\item	Vậy có $6$ giá trị của $a$ thỏa đề bài.
		\end{itemize}


	}
\end{ex}

\begin{ex}%[2D4H3-1]
	Gọi $S$ là hình phẳng giới hạn bởi đồ thị hàm số $(H)\colon y=\dfrac{x-1}{x+1}$ và các trục tọa độ. Tính diện tích hình phẳng $(S)$ (làm tròn đến chữ số thứ hai sau dấu phẩy).
	\shortans{$0{,}39$}
	\loigiai{
		Điều kiện $x\ne -1$.\\
		Hoành độ giao điểm của đồ thị hàm số và trục $Ox$ là nghiệm của phương trình \[\dfrac{x-1}{x+1}=0\Leftrightarrow x=1.\]
		Vậy diện tích hình phẳng cần tìm là
		\allowdisplaybreaks
		\begin{eqnarray*}
			S&=&\displaystyle\int\limits_0^1 \left|\dfrac{x-1}{x+1}\right| \mathrm{\,d}x=\displaystyle\int\limits_0^1 \dfrac{1-x}{x+1} \mathrm{\,d}x\\
			&=&\displaystyle\int\limits_0^1 \left(-1+\dfrac{2}{x+1}\right) \mathrm{\,d}x\\
			&=&\left(-x+2\ln |x+1|\right)\Bigr\rvert_0^1=2\ln 2-1\approx 0{,}39.
		\end{eqnarray*}
	}
\end{ex}

\begin{ex}%[2D4H3-3]
	Tính thể tích của vật thể tròn xoay được tạo thành khi quay hình $(H)$ quanh $Ox$ với $(H)$ được giới hạn bởi đồ thị hàm số $y=\sqrt{4x-x^2}$ và trục hoành. (kết quả làm tròn đến hàng phần mười)
	\shortans{$33{,}5$}
	\loigiai{
	Điều kiện xác định: $4x-x^2\ge 0\Leftrightarrow 0\le x\le 4$.\\
	Phương trình hoành độ giao điểm của đồ thị hàm số $y=\sqrt{4x-x^2}$ và trục hoành là
	\[\sqrt{4x-x^2}=0\Leftrightarrow 4x-x^2=0\Leftrightarrow \hoac{
			&x=0 \\
			&x=4. \\
		}\]
	Thể tích của vật thể tròn xoay khi quay hình $(H)$ quanh $Ox$ là
	\[V=\pi \displaystyle \int\limits_0^4\left(\sqrt{4x-x^2}\right)^2\mathrm{\,d}x=\pi \displaystyle \int\limits_0^4{(4x-x^2)}\mathrm{\,d}x=\dfrac{32}{3}\pi.\]
	Vậy thể tích của vật thể tròn xoay khi quay hình $(H)$ quanh $Ox$ là $\dfrac{32}{3}\pi\approx33{,}5$.
	}
\end{ex}

\Closesolutionfile{ans}

\TL
\begin{ex}%[Mức độ 2]%[BG12, Nguyễn Kiều Nhã Tú]%[2D4H2-2]
	Cho hàm số $f(x)=\heva{&x^2\,\, \text{khi}\,\, 0\le x \le 1\\&2-x\,\, \text{khi} \,\,1< x \le 2}$. Tính $\displaystyle\int_0^2 f(x) \mathrm{\,d}x$.
	\loigiai{
		Ta có $\displaystyle\int_0^2 f(x) \mathrm{\,d}x=\displaystyle\int_0^1 x^2 \mathrm{\,d}x+\displaystyle\int_1^2 (2-x) \mathrm{\,d}x=\dfrac{x^3}{3}\bigg|_0^1+\left( 2x-\dfrac{x^2}{2} \right)\bigg|_1^2=\dfrac{5}{6}$.
	}
\end{ex}

\begin{ex}%[2D4V1-4]
	Cho hàm số $ f(x)$ nhận giá trị dương và thỏa mãn $ f(0)=1$, $\left(f'(x)\right)^3=\mathrm{\mathrm{e}}^ x{\left(f(x)\right)^2}$, $\forall x\in\mathbb{R}$. Tính $ f(3)$ (\textit{kết quả làm tròn đến hàng phần mười}).
	% \shortans{$20{,}1$}
	\loigiai{
	Ta có

	\begin{align*}
		\left(f'(x)\right)^3=\mathrm{e}^x{\left(f(x)\right)^2},\,\forall x\in\mathbb{R}
		 & \Leftrightarrow{f}'(x)=\sqrt[3]{\mathrm{e}^x}\cdot \sqrt[3]{\left(f(x)\right)^2}\Leftrightarrow\dfrac{f'(x)}{\sqrt[3]{\left(f(x)\right)^2}}=\sqrt[3]{\mathrm{e}^x}     \\
		 & \Leftrightarrow\dfrac{f'(x)}{\sqrt[3]{\left(f(x)\right)^2}}=\sqrt[3]{\mathrm{e}^x}\Leftrightarrow{f}'(x)\cdot \left(f(x)\right)^{-\tfrac{2}{3}}=\sqrt[3]{\mathrm{e}^x} \\&\Leftrightarrow 3\left[\left(f(x)\right)^{\tfrac{1}{3}}\right]'=\sqrt[3]{\mathrm{e}^x}\Leftrightarrow{\left[\left(f(x)\right)^{\tfrac{1}{3}}\right]'}=\dfrac{1}{3}\sqrt[3]{\mathrm{e}^x}\\&\Leftrightarrow{\left(f(x)\right)^{\tfrac{1}{3}}}=\dfrac{1}{3}\displaystyle\int{\sqrt[3]{\mathrm{e}^x}}\mathrm{\,d} x \Leftrightarrow{\left(f(x)\right)^{\tfrac{1}{3}}}=e^{\tfrac{x}{3}}+C.
	\end{align*}
	Vì	$f(0)=1$ nên $1=1+C\Rightarrow C=0\Rightarrow{\left(f(x)\right)^{\tfrac{1}{3}}}=e^{\tfrac{x}{3}}\Rightarrow f(x)=\mathrm{e}^x$.\\
	Vậy	$f(3)=e^3\approx 20{,}1$.
	}
\end{ex}

\begin{ex}%[2D4C3-5]
	Cho một mô hình $3-D$ mô phỏng một đường hầm như hình vẽ bên. Biết rằng đường hầm mô hình có chiều dài $5$ (cm); khi cắt hình này bởi mặt phẳng vuông góc với đáy của nó, ta được mặt cắt là một hình parabol có độ dài đáy gấp đôi chiều cao parabol. Chiều cao của mỗi mặt cắt hình parabol cho bởi công thức $ y=3-\dfrac{2}{5}x$ (cm), với $x$ (cm) là khoảng cách tính từ lối vào lớn hơn của đường hầm mô hình. Tính thể tích (theo đơn vị cm$^3$) không gian bên trong đường hầm mô hình (làm tròn kết quả đến hàng đơn vị).
	% \shortans{$29$}
	\begin{center}
	\begin{tikzpicture}[scale=1,declare function={a=0.8;b=0.6;c=0.4;d=0.2;}]
	\tikzset{
	homothety at/.style args={#1 scaled by #2}{shift={($(#1)!#2!(0,0)$)},scale=#2},
	}
	\def\mypath{(-120:2)..controls +(90:0.6) and +(-180:0.6)..(0,3)}
	\def\mydot{(0,3)..controls +(0:0.25) and +(95:0.05)..(60:2)}
	\draw \mypath;
	\draw[dashed] \mydot;
	\path (7,0) coordinate (c1);
	\begin{scope}[homothety at=c1 scaled by a]
	\draw \mypath;
	\draw[dashed] \mydot;
	\end{scope}
	\begin{scope}[homothety at=c1 scaled by b]
	\draw \mypath;
	\draw[dashed] \mydot;
	\end{scope}
	\begin{scope}[homothety at=c1 scaled by c]
	\draw \mypath;
	\draw[dashed] \mydot;
	\end{scope}
	\begin{scope}[homothety at=c1 scaled by d]
	\draw \mypath;
	\draw \mydot;
	\end{scope}
	\path
	(-120:2) coordinate (A)
	(0,3) coordinate (B)
	(60:2) coordinate (C);
	\foreach \x in {A,B,C}{\path ($(c1)!a!(\x)$) coordinate (\x_1);}
	\foreach \x in {A,B,C}{\path ($(c1)!b!(\x)$) coordinate (\x_2);}
	\foreach \x in {A,B,C}{\path ($(c1)!c!(\x)$) coordinate (\x_3);}
	\foreach \x in {A,B,C}{\path ($(c1)!d!(\x)$) coordinate (\x_4);}
	\path ($(A_4)!0.5!(B_4)$) coordinate (D);
	\draw (A)--(A_4) (B)--(B_4) (A_4)--(C_4)
	;
	\draw[dashed] (B)node[above]{$3$}--(0,0)--(D)node[below right]{$5$} (A)--(C) (A_1)--(C_1) (A_2)--(C_2) (A_3)--(C_3)  (C)--(C_4);
	\end{tikzpicture}
	\end{center}
	\loigiai{
	\begin{center}
	\begin{tikzpicture}[scale=1,font=\footnotesize]
	\path (0,0) coordinate (O)
	(2,0) coordinate (A)
	(0,2) coordinate (B)
	;
	\draw[-stealth] (-3.5,0)--(0,0)--(3,0)node[below]{$x$};
	\draw[-stealth] (0,-1.5)--(0,4)node[left]{$y$};
	\draw[smooth,samples=100] plot[domain=-2:2](\x,{(-1/2)*(\x)^2+2});
	\foreach \x in {O,A,B}{\draw[fill=blue!40] (\x) circle (1pt);}
	\foreach \x in {-3,-2,-1,1}{\draw (\x,0.05)--(\x,-0.05);}
	\foreach \x in {-1,1,3}{\draw (-0.05,\x)--(0.05,\x);}
	\node[above left] at (B) {$h$};
	\path (O)--(A)node[below]{$h$};
	\node at (0,0) [below left]{$O$};
	\end{tikzpicture}
	\end{center}
	Xét một mặt cắt hình parabol có chiều cao là $h$ và độ dài đáy $2h$ và chọn hệ trục $Oxy$ như hình vẽ trên.\\
	Parabol $(P)$ có phương trình $(P)\colon y=ax^2+h$, $(a<0)$.\\
	Có $B(h;0)\in(P)\Leftrightarrow 0=ah^2+h\Leftrightarrow a=-\dfrac{1}{h}$ (do $h>0$).\\
	Diện tích $S$ của mặt cắt là \[S=\displaystyle\int\limits_{-h}^h\left(-\dfrac{1}{h}{x^2}+h\right)\mathrm{\,d}x=\dfrac{4h^2}{3}, h=3-\dfrac{2}{5}x.\]
	$\Rightarrow S(x)=\dfrac{4}{3}{\left(3-\dfrac{2}{5}x\right)^2}.$\\
	Suy ra thể tích không gian bên trong của đường hầm mô hình
	\[ V=\displaystyle\int\limits_0^5S(x)\mathrm{\,d}x=\displaystyle\int\limits_0^5\dfrac{4}{3}\left(3-\dfrac{2}{5}x\right)^2\mathrm{\,d}x=\dfrac{260}{9}\approx 29\,\left(\text{cm}^3\right).\]
	}
	\end{ex}

% \Closesolutionfile{ansbook}
% \HetDe
% \label{De4}
% %
% \cleardoublepage
% \setcounter{page}{1}
% \rfoot{Trang \thepage/\pageref{DA4} - Đáp án trắc nghiệm Mã đề 4}
% \begin{center}
% 	\bfseries ĐÁP ÁN TRẮC NGHIỆM MÃ ĐỀ 4
% \end{center}

% \inputansbox{10}{ans/ansDe4-TN1}
% \inputansbox[3]{2}{ans/ansDe4-TN2}
% \inputansbox{3}{ans/ansDe4-TN3}
% \label{DA4}
%

% \begin{name}
	{\tenchude}
	{TOÁN 12}
	{LỚP TOÁN THẦY PHÁT}
	{Thời gian: 90 phút - Không kể thời gian phát đề}
\end{name}
\Opensolutionfile{ans}[ans/ansDe1-TN1]
\begin{ex}%[2D4N1-1]%[To 20 - Dot 17 - Chuong 4 - Bai 3 - CD - De 1 - TN]%[Nguyễn Hữu Duy]
Nếu hàm số $f(x)$ liên tục trên đoạn $[a;b]$ và $c$ là số thực tùy ý thuộc đoạn $[a;b]$, thì tính chất nào sau đây đúng?
\choice
{\True $\displaystyle\int_a^b f(x) \mathrm{\,d}x = \displaystyle\int_a^c f(x) \mathrm{\,d}x + \displaystyle\int_c^b f(x)\mathrm{\,d}x$}
{$\displaystyle\int_a^b f(x)\mathrm{\,d}x = \displaystyle\int_a^c f(x) \mathrm{\,d}x - \displaystyle\int_c^b f(x) \mathrm{\,d}x$}
{$\displaystyle\int_a^b f(x)\mathrm{\,d}x = \displaystyle\int_a^c f(x)\mathrm{\,d}y + \displaystyle\int_c^b f(x) \mathrm{\,d}z$}
{$\displaystyle\int_a^b f(x)\mathrm{\,d}x = \displaystyle\int_a^c f(x)\mathrm{\,d}y - \displaystyle\int_c^b f(x)\mathrm{\,d}z$}
\loigiai{
Theo định nghĩa tích phân ta có $\displaystyle\int_a^b f(x) \mathrm{\,d}x = \displaystyle\int_a^c f(x) \mathrm{\,d}x + \displaystyle\int_c^b f(x)\mathrm{\,d}x$.
}
\end{ex}

\begin{ex}%[2D4N1-2]
Tìm họ nguyên hàm của hàm số $f(x)=\dfrac{1}{x}+1$. \choice
{$F(x)=-\dfrac{1}{x^2}+x+C$}
{\True $F(x)=\ln |x|+x+C$}
{$F(x)=\ln x+x+C$}
{$F(x)=\ln |x|+C$}
\loigiai{
Họ nguyên hàm của hàm số $f(x)=\dfrac{1}{x}+1$ là $F(x)=\ln |x|+x+C$.}
\end{ex}

\begin{ex}%[2D4H1-3]
Hàm số $F(x)=x\sin x+\cos x+2024$ là một nguyên hàm của hàm số nào trong các hàm số sau?
\choice
{ $f(x)=x\sin x$ }
{ $f(x)=-x\cos x$ }
{ $f(x)=-x\sin x$ }
{\True $f(x)=x\cos x$ }
\loigiai{
$F'(x)=( x\sin x+\cos x+2024 )' = \sin x + x\cos x - \sin x = x\cos x$, $\forall x\in \mathbb{R}$.\\
$\Rightarrow$ Hàm số $F(x)$ là một nguyên hàm của hàm số $f(x)=x\cos x$ trên $\mathbb{R}$.
}
\end{ex}

\begin{ex}%[2D4H2-1]
Cho hàm số $f(x)$ và $F(x)$ liên tục trên $\mathbb{R}$ thoả mãn $F'(x)=f(x)$ với mọi số thực $x$. Tính   $\displaystyle \int\limits_0^1 f(x)\mathrm{\,d}x$. Biết $F(0)=2; F(1)=5$.
\choice
{\True $3$}
{$4$}
{$5$}
{$6$}
\loigiai{
$\displaystyle \int\limits_0^1 f(x)\mathrm{\,d}x = F(1)-F(0)=5-2=3$.
}
\end{ex}

\begin{ex}%[2D4N3-1]
\immini[thm]{Diện tích hình phẳng giới hạn bởi đồ thị hàm số $ y=f(x)$ và trục hoành (phần gạch chéo trong hình vẽ) là
\choice
{\True $ S=\displaystyle\int\limits_{-2}^0f(x)\mathrm{d}x-\displaystyle\int\limits_0^1f(x)\mathrm{d}x$}
{$ S=\displaystyle\int\limits_{-2}^0f(x)\mathrm{d}x+\displaystyle\int\limits_0^1f(x)\mathrm{d}x$}
{$ S=\displaystyle\int\limits_0^1f(x)\mathrm{d}x-\displaystyle\int\limits_{-2}^0f(x)\mathrm{d}x$}
{$\left|\displaystyle\int\limits_{-2}^1f(x)\mathrm{d}x\right|$}
}
{
\begin{tikzpicture}[scale=0.8,>=stealth, font=\footnotesize, line join=round, line cap=round]
\def\a{1} \def\b{-3} \def\c{0} \def\d{2} % Hệ số
\def\xmin{-3} \def\xmax{2}
\def\ymin{-1} \def\ymax{3}
%	\draw[color=gray!50,dashed] (\xmin,\ymin) grid (\xmax,\ymax);
\draw[->] (\xmin,0)--(\xmax,0) node [below]{$x$};
\draw[->] (0,\ymin)--(0,\ymax) node [left]{$y$};
\node at (0,0) [below left]{$O$};
\clip (\xmin+0.1,\ymin+0.1) rectangle (\xmax-0.5,\ymax-0.1);
\draw[smooth,samples=300] plot(\x,{\a*(\x+2)*(\x)*(\x-1)});

\fill[pattern=north east lines,opacity=0.8] plot[domain=-2:1](\x,{\a*(\x+2)*(\x)*(\x-1)})--cycle;
%	\draw[dashed](-1,0)--(-1,-2)
%	(3,0)--(3,2);
\draw[fill=black](-2,0)node[above left]{$-2$}circle(1pt)
(1,0)node[above]{$1$}circle(1pt)
(1.2,1.5)node[above,rotate=80,scale=0.8]{$y=f(x)$}
;
\end{tikzpicture}
}
\loigiai{
Diện tích hình phẳng giới hạn bởi đồ thị hàm số $ y=f(x)$ và trục hoành (phần gạch chéo trong hình vẽ) là $ S=\displaystyle\int\limits_{-2}^1\left| f(x)\right|\mathrm{d}x=\displaystyle\int\limits_{-2}^0f(x)\mathrm{d}x-\displaystyle\int\limits_0^1f(x)\mathrm{d}x$.}
\end{ex}

\begin{ex}%[2D4H3-3]
Tính thể tích vật thể tạo thành khi quay hình phẳng $(H)$ quanh trục $Ox$, biết $(H)$ được giới hạn bởi các đường $y=4x^2-1$, $y=0$.
\choice
{\True $\dfrac{8\pi}{15}$}
{$\dfrac{4\pi}{15}$}
{$\dfrac{16\pi}{15}$}
{$\dfrac{2\pi}{15}$}
\loigiai{
Phương trình hoành độ giao điểm $4x^2-1=0\Leftrightarrow x=\pm\dfrac{1}{2}$.\\
Suy ra $V=\pi\displaystyle\int\limits_{-\frac{1}{2}}^{\frac{1}{2}}(4x^2-1)^2\mathrm{\,d}x=\pi\displaystyle\int\limits_{-\frac{1}{2}}^{\frac{1}{2}}(16x^4-8x^2+1)\mathrm{\,d}x=\pi\left.\left(\dfrac{16}{5}x^5-\dfrac{8}{3}x^3+x\right)\right|_{-\frac{1}{2}}^{\frac{1}{2}}=\dfrac{8\pi}{15}.$
}
\end{ex}

\begin{ex}%[2H5N1-1]
Trong không gian $Oxyz$, phương trình của mặt phẳng $(Oxy)$ là
\choice
{\True $z=0$}
{$x=0$}
{$y=0$}
{$x+y=0$}
\loigiai{
Phương trình của mặt phẳng $(Oxy)$ là $z=0$.
}
\end{ex}

\begin{ex}%[2H5N1-2]
Trong không gian $Oxyz$, cho mặt phẳng $(P)\colon 2x+y-z+3=0$. Véc-tơ nào sau đây là véc-tơ pháp tuyến của mặt phẳng $(P)$?
\choice
{$\overrightarrow{n}_1=(1;-1;3)$}
{$\overrightarrow{n}_2=(2;-1;3)$}
{\True $\overrightarrow{n}_3=(2;1;-1)$}
{ $\overrightarrow{n}_4=(2;1;3)$}
\loigiai{
Mặt phẳng $Ax+By+Cz+D=0$ nhận véc-tơ $\overrightarrow{n}=(A;B;C)$ làm một véc-tơ pháp tuyến.
}
\end{ex}

\begin{ex}%[2H5H1-3]
Cho hai mặt phẳng $(\alpha)\colon  3 x-2 y+2 z+7=0,$ $(\beta)\colon 5 x-4 y+3 z+1=0$. Phương trình mặt phẳng đi qua gốc tọa độ $O$ đồng thời vuông góc với cả $(\alpha)$ và $(\beta)$ là
\choice
{$2 x-y-2 z=0$}
{$2 x-y+2 z=0$}
{\True $2 x+y-2 z=0$}
{$2 x+y-2 z+1=0$}
\loigiai{
Véc-tơ pháp tuyến của hai mặt phẳng lần lượt là $\overrightarrow{n}_\alpha=(3 ;-2 ; 2), \overrightarrow{n}_\beta=(5 ;-4 ; 3)$.\\
Suy ra $\left[\overrightarrow{n}_\alpha ; \overrightarrow{n}_\beta\right]=(2 ; 1 ;-2)$ là véc-tơ pháp tuyến của mặt phẳng cần tìm.\\
Phương trình mặt phẳng đi qua gốc tọa độ $O, $ có véc-tơ pháp tuyến $\vec{n}=(2 ; 1 ;-2)$ là $2 x+y-2 z=0$.
}
\end{ex}

\begin{ex}%[2H5N2-1]%[Dự án 2025 - Đề cấu trúc mới của Bộ theo [Thành Đức Trung]
Trong không gian $Oxyz$, đường thẳng $d\colon \dfrac{x-1}{2}=\dfrac{y-2}{-1}=\dfrac{z-3}{2}$ đi qua điểm nào dưới đây?
\choice
{$M(-1;-2;-3)$}
{\True $P(1;2;3)$}
{$Q(2;-1;2)$}
{$N(-2;1;-2)$}
\loigiai
{
Vì $\dfrac{-1-1}{2}=\dfrac{2-2}{-1}=\dfrac{3-3}{2}=0$ nên đường thẳng $d$ đi qua điểm $P(1;2;3)$.
}
\end{ex}

\begin{ex}%[2H5N2-7]%[Dự án EX-TF-TLN-2024 Đợt 3-  GV. Đỗ Chí Tâm]
Trong không gian $Oxyz$, góc giữa đường thẳng $d:\dfrac{x-3}{2}=\dfrac{y+1}{1}=\dfrac{z-3}{1}$ và mặt phẳng  $(P):x+2y-z+5=0$ là
\choice
{\True $30^{\circ}$}
{ $45^{\circ}$}
{$60^{\circ}$}
{$90^{\circ}$}
\loigiai{
Gọi $\varphi$ là góc giữa $d$ và $(P)$.\\
Đường thẳng $d$ có véc-tơ chỉ phương $\vec{u}=(2;1;1)$, $(P)$ có véc-tơ pháp tuyến $n=(1;2;-1)$.\\
\[\sin \varphi =\dfrac{|\vec{u}\cdot \vec{n}|}{|\vec{u}|\cdot|\vec{n}|}= \dfrac{\left| 2\cdot 1+1\cdot 2+1\cdot (-1)\right|}{\sqrt{2^2+1^2+1^2} \cdot \sqrt{1^2 +2^2+(-1)^2}}=\dfrac{1}{2} \Rightarrow \varphi = 30^{\circ}.\]
}
\end{ex}

\begin{ex}%[2H5H2-4]
Trong không gian $Oxyz$, cho hai đường thẳng $d_{1}  \colon \dfrac{x-3}{-1} =\dfrac{y-3}{-2} =\dfrac{z+2}{1} $; $d_{2}  \colon \dfrac{x-5}{-3} =\dfrac{y+1}{2} =\dfrac{z-2}{1} $ và mặt phẳng $\left(P\right) \colon x+2y+3z-5=0$. Đường thẳng vuông góc với $\left(P\right)$, cắt $d_{1} $ và $d_{2} $ có phương trình là
\choice
{$\dfrac{x-1}{3} =\dfrac{y+1}{2} =\dfrac{z}{1} $}
{$\dfrac{x-2}{1} =\dfrac{y-3}{2} =\dfrac{z-1}{3} $}
{$\dfrac{x-3}{1} =\dfrac{y-3}{2} =\dfrac{z+2}{3} $}
{\True $\dfrac{x-1}{1} =\dfrac{y+1}{2} =\dfrac{z}{3} $}
\loigiai{
Phương trình $d_{1}  \colon \heva{x&=3-t_{1}  \\ y&=3-2t_{1}  \\ z&=-2+t_{1} } $ và $d_{2}  \colon \heva{x&=5-3t_{2}  \\ y&=-1+2t_{2}  \\ z&=2+t_{2}.} $\\
Gọi đường thẳng cần tìm là $\Delta $. \\
Giả sử đường thẳng $\Delta $ cắt đường thẳng $d_{1} $ và $d_{2} $ lần lượt tại $A$, $B$. \\
Gọi $A\left(3-t_{1} ;3-2t_{1} ;-2+t_{1} \right)$, $B\left(5-3t_{2} ;-1+2t_{2} ;2+t_{2} \right)$.\\ $\overrightarrow{AB}=\left(2-3t_{2} +t_{1} ;-4+2t_{2} +2t_{1} ;4+t_{2} -t_{1} \right).$ \\
véc-tơ pháp tuyến của $\left(P\right)$ là $\vec{n}=\left(1;2;3\right)$. \\
Do $\overrightarrow{AB}$ và $\vec{n}$ cùng phương nên $\dfrac{2-3t_{2} +t_{1} }{1} =\dfrac{-4+2t_{2} +2t_{1} }{2} =\dfrac{4+t_{2} -t_{1} }{3} $\\
$\Leftrightarrow \heva{\dfrac{2-3t_{2} +t_{1} }{1} =\dfrac{-4+2t_{2} +2t_{1} }{2}  \\ \dfrac{-4+2t_{2} +2t_{1} }{2} =\dfrac{4+t_{2} -t_{1} }{3} } $$\Leftrightarrow \heva{t_{1} =2. \\ t_{2} =1} $ Do đó $A\left(1;-1;0\right)$, $B\left(2;-1;3\right)$.\\
Phương trình đường thẳng $\Delta $ đi qua $A\left(1;-1;0\right)$ và có véc-tơ chỉ phương $\vec{n}=\left(1;2;3\right)$ là $\dfrac{x-1}{1} =\dfrac{y+1}{2} =\dfrac{z}{3} .$}
\end{ex}
\Closesolutionfile{ans}

\TNTF
\Opensolutionfile{ans}[ans/ansDe1-TN2]
\begin{ex}%[MĐ2]%[2D4H2-4]
Nếu các số hữu tỉ $a,b$ thỏa mãn $\displaystyle\int \limits_0^1(a\mathrm{e}^x+b)\mathrm{d}x=3\mathrm{e}+4$ thì các phát biểu sau đúng hay sai?
\choiceTF[t]
{$a>b$}
{$a=2b$}
{\True $a=3,\, b=7$}
{\True $2a-b=-1$}
\loigiai{
Ta có $\displaystyle\int \limits_0^1(a\mathrm{e}^x+b)\mathrm{d}x=(a\mathrm{e}^x+bx)\bigg|_0^1=a\mathrm{e}+(-a+b)\Rightarrow \heva{&a=3 \\ &-a+b=4} \Rightarrow \heva{&a=3 \\ &b=7.}$\\
\begin{itemchoice}
\itemch Suy ra $a>b$ là sai.
\itemch Do $a=3, \, b=7$ nên $a=2b$ là sai.
\itemch $a=3,\, b=7$ là khẳng định đúng.
\itemch $2a-b= 2\cdot 3-7=-1$ là khẳng định đúng.
\end{itemchoice}
}
\end{ex}

\begin{ex}%[Dat Thai, Dự án Ex-TF-TLN-2024-Dot03]%[2H5H2-5]
Cho đường thẳng $d$ đi qua điểm $A(0; 0; 1)$ có véc-tơ chỉ phương $\overrightarrow{u} = (1; 1; 3)$ và mặt phẳng $(\alpha) \colon 2x + y - z + 5 = 0$.
\choiceTF
{\True $B(2;2;7)\in d$}
{Phương trình đường thẳng $d$ là $\dfrac{x}{1} = \dfrac{y}{1} = \dfrac{z+1}{3}$}
{$\overrightarrow{n} = (2;1;1)$ là một véc-tơ pháp tuyến của mặt phẳng $(\alpha)$}
{\True Đường thẳng $d$ song song với mặt phẳng $(\alpha)$}
\loigiai{
\begin{itemchoice}
\itemch \textbf{Đúng}. Vì $\overrightarrow{AB} = (2;2;6) = 2\overrightarrow{u}$.
\itemch \textbf{Sai}. Vì phương trình đường thẳng $d$ là $\dfrac{x}{1} = \dfrac{y}{1} = \dfrac{z-1}{3}$.
\itemch \textbf{Sai}. Vì họ véc-tơ pháp tuyến của mặt phẳng $(\alpha)$ có dạng $k\cdot(2;1;-1)$ ($k$ khác $0$), nên nếu giả sử $\overrightarrow{n} = (2;1;1)$ là một véc-tơ pháp tuyến của mặt phẳng $(\alpha)$ thì tồn tại số thực $k$ khác $0$ để
\[
\heva{&2 = 2k\\ & 1 = k\\ & 1=-k} \Leftrightarrow\heva{&k=1\\ &k=-1} (\text{vô lí}).
\]
\itemch \textbf{Đúng}. Vì $\vec{n} = (2;1;-1)$ là một véc-tơ pháp tuyến của mặt phẳng $(\alpha)$ thoả mãn
\[
\vec{n} \cdot \vec{u} = 2\cdot 1 + 1\cdot 1 + (-1)\cdot 3 = 0 \text{ hay } \vec{n}\perp \vec{u}
\]
nên $d\parallel (\alpha)$ hoặc $d\subset(\alpha)$.\\
Kết hợp với $A\notin (\alpha)$ ta thu được $d\parallel (\alpha)$.
\end{itemchoice}
}
\end{ex}
\Closesolutionfile{ans}

\TNSA
\Opensolutionfile{ans}[ans/ansDe1-TN3]
\begin{ex}%[2D4H1-1]%[Đào Trung Kiên]
Biết $ F(x) $ là một nguyên hàm của hàm số $ f(x) = \mathrm{e}^{2x} $ và $ F(0) = 0$. Tính giá trị của $F(\ln 3)$.
\shortans[]{$4$}
\loigiai{
Ta có $ \heva{& F(0) = 0 \\ & F(x) = \dfrac{1}{2} \cdot \mathrm{e}^{2x} + C } \Rightarrow F(x) = \dfrac{1}{2} \cdot \mathrm{e}^{2x} - \dfrac{1}{2} \Rightarrow F(\ln 3) =  \dfrac{1}{2}  \cdot \left  (  \mathrm{e}^{ 2 \cdot \ln 3 } - 1 \right ) = 4$.
}
\end{ex}

\begin{ex}%[2D4V3-1]
	\immini{
	Gọi $H$ là hình phẳng giới hạn bởi đồ thị hàm số $y=-x^2+4x$ và trục hoành. Hai đường thẳng $y=m$ và $y=n$ chia $(H)$ thành ba phần có diện tích bằng nhau (tham khảo hình vẽ). Giá trị của biểu thức $T=(4-m)^3+(4-n)^3$ bằng bao nhiêu? (Kết quả làm tròn đến hàng phần mười)
	}{
	\begin{tikzpicture}[thick,>=stealth,x=1cm,y=1cm,scale=.8]
	\draw[->] (-1,0) -- (5,0) node[below] {\small $x$};
	\draw[->] (0,-1) -- (0,5) node[right] {\small $y$};
	\draw [fill=white,draw=black] (0,0) circle (1pt)node[above left] {\footnotesize $O$};
	\clip(-1,-1) rectangle (5,5);
	\draw[thick,smooth,samples=100,domain=-1:5] plot(\x,{-(\x)^2+4*(\x)});
	\draw (-1,1.2)--(5.1,1.3)node[above left]{$y=n$};
	\draw (-1,2.3)--(5.1,2.4)node[above left]{$y=m$};
	\end{tikzpicture}}
	\shortans[]{$35{,}6$}
	\loigiai{
	\immini{
	Gọi $S$ là diện tích hình phẳng giới hạn bởi đồ thị hàm số $y=-x^2+4x$ và trục $Ox$ và hai đường thẳng $x=0$, $x=2$.\\
	Khi đó $S=\displaystyle\int\limits_{0}^{2} (-x^2+4x)\mathrm{\,d}x =\dfrac{16}{3}$.\\
	Đường thẳng $y=m$ và $y=n$ chia $S$ thành ba phần bằng nhau có diện tích theo thứ tự từ trên xuống là $S_1$; $S_2$; $S_3$.\\
	Gọi hoành độ các giao điểm của parabol với hai đường thẳng như hình bên.\\
	Ta có
	\begin{eqnarray*}
	&& S_1=2\displaystyle\int\limits_{a}^{2} (-x^2+4x-m)\mathrm{\,d}x =\dfrac{1}{3}S\\
	&\Leftrightarrow & \left(-\dfrac{x^3}{3}+2x^2-mx\right)\Big|_{a}^{2}=\dfrac{1}{3}\cdot\dfrac{16}{3}\\
	&\Leftrightarrow & \left(\dfrac{16}{3}-2m\right)-\left(-\dfrac{a^3}{3}+2a^2-ma\right)=\dfrac{16}{9}\quad(1).
	\end{eqnarray*}
	Mà $x=a$ là nghiệm của phương trình $-x^2+4x=m$ nên ta có $-a^2+4a=m\quad(2)$.\\
	Thay $(2)$ vào $(1)$ ta được $-\dfrac{2a^3}{3}+4a^2-8a+\dfrac{32}{9}=0\Leftrightarrow a\approx 0{,}613277$.\\
	Suy ra $m=-a^2+4a\approx 2{,}077$.\\
	Tương tự ta có
	\begin{eqnarray*}
	&& S_1+S_2=\dfrac{2}{3}S\\
	&\Rightarrow & 2\displaystyle\int\limits_{b}^{2} (-x^2+4x-n)\mathrm{\,d}x =\dfrac{2}{3}\cdot 2\cdot\displaystyle\int\limits_{0}^{2} (-x^2+4x)\mathrm{\,d}x\\
	&\Leftrightarrow & -\dfrac{2}{3}b^3+4b^2-8b+\dfrac{16}{9}=0\\
	&\Leftrightarrow & b\approx 0{,}252839\Rightarrow n=-b^2+4b=0{,}947428.
	\end{eqnarray*}
	Khi đó $T=(4-m)^3+(4-n)^3=\dfrac{320}{9}\approx35{,}6$.
	}{
	\begin{tikzpicture}[thick,>=stealth,x=1cm,y=1cm,scale=.8]
	\draw[->] (-1,0) -- (5,0) node[below] {\small $x$};
	\draw[->] (0,-1) -- (0,5) node[right] {\small $y$};
	\draw [fill=white,draw=black] (0,0) circle (1pt)node[above left] {\footnotesize $O$};
	\clip(-1,-1) rectangle (5,5);
	\draw[thick,smooth,samples=100,domain=-1:5] plot(\x,{-(\x)^2+4*(\x)});
	\draw (-1,1.2)--(5,1.2)node[above left]{$y=n$};
	\draw (-1,2.3)--(5,2.3)node[above left]{$y=m$};
	\draw[dashed](0.33,0)node[below]{$b$}--(0.33,1.2) (0.7,0)node[below]{$a$}--(0.7,2.3) (2,0)node[below]{$2$}--(2,4);
	\draw[dashed] (0,4) node[below left]{$4$}--(2,4);
	\end{tikzpicture}
	}
	}
	\end{ex}

\begin{ex}%[2H5H2-7]
Gọi $\varphi$ là góc giữa hai đường thẳng $d_1 \colon \dfrac{x-1}{-2}= \dfrac{y+2}{1}= \dfrac{z-3}{2}$ và $d_2 \colon \dfrac{x+3}{1}= \dfrac{y-1}{1}= \dfrac{z+2}{-4}$. Tính $\cos \varphi$ (làm tròn đến hàng phần trăm).
\shortans{$0{,}71$}
\loigiai
{
Đường thẳng $ d_1 $ có một véc-tơ chỉ phương $ \vec u_1 =(-2;1;2)$.\\
Đường thẳng $ d_2 $ có một véc-tơ chỉ phương $ \vec u_2 =(1;1;-4)$.\\
Ta có
\begin{eqnarray*}
\cos \varphi
&=&\left| \cos \left( \vec u_1, \vec u_2 \right)\right| = \dfrac{\left|  \vec u_1 \cdot  \vec u_2 \right|}{\left| { \vec u_1} \right| \cdot \left| { \vec u_2} \right|}\\
&=&\dfrac{|-2 \cdot 1+ 1\cdot 1+ 2\cdot (-4)|}{\sqrt{(-2)^2+1^2+2^2} \cdot \sqrt{1^2+1^2+(-4)^2}}\\
&=& \dfrac{\sqrt {2}}{2} \approx 0{,}71.
\end{eqnarray*}
}
\end{ex}

\begin{ex}%[2H5V1-7]
Một công trình đang xây dựng được gắn hệ trục $Oxyz$ (đơn vị trên mỗi trục tọa độ là mét). Ba bức tường $(P),(Q),(R)$ (như hình vẽ) của tòa nhà lần lượt có phương trình $(P)\colon 2x-y-z+1=0$, $(Q)\colon x+3y-z-2=0,(R)\colon 4x-2y-2z+9=0$. Tính chiều rộng bức tường $(Q)$ của tòa nhà. (Kết quả làm tròn đến hàng phần chục).
\begin{center}
\includegraphics[width=0.7\textwidth]{images/C5B1CD3-H3.png}
\end{center}
\shortans{$2{,}9$}
\loigiai{
\begin{itemize}
\item Kiểm tính song song hoặc vuông góc giữa các bức tường $(P),(Q),(R)$ của tòa nhà.\\
Ta có $(P)$ có vectơ pháp tuyến là $\vec{n}_P=(2;-1;-1)$, $(Q)$ có vectơ pháp tuyến là $\vec{n}_Q=(1; 3;-1)$, $(R)$ có vectơ pháp tuyến là $\vec{n}_R=(4;-2;-2)$.\\
Khi đó $\vec{n}_R=(4;-2;-2)=2(2;-1;-1) \Rightarrow \vec{n}_R=2\vec{n}_P$ nên hai bức tường $(P)$ và $(R)$ song song nhau.\\
Mặt khác $\vec{n}_P \cdot \vec{n}_Q=2\cdot 1+(-1) \cdot 3+(-1) \cdot(-1)=0\Rightarrow \vec{n}_P \perp \vec{n}_Q$ nên bức tường $(Q)$ vuông góc với hai bức tường $(P)$ và $(R)$.
\item Tính chiều rộng bức tường $(Q)$ của tòa nhà.\\
Do hai bức tường $(P)$ và $(R)$ song song nhau nên chiều rộng bức tường $(Q)$ là khoảng cách giữa hai bức tường $(P)$ và $(R)$.\\
Chọn điểm $N(0; 0; 1) \in(P)$. Do hai bức tường $(P)$ và $(R)$ song song nhau nên
\[\mathrm{d}((P),(R))=\mathrm{d}(N,(R))=\dfrac{|4\cdot 0-2\cdot 0-2\cdot 1+9|}{\sqrt{4+1+1}}=\dfrac{7}{\sqrt{6}} \approx 2{,}9.\]
\end{itemize}
}
\end{ex}

\TL
\begin{ex}%[2H5H2-3]%[Dự án 2025 - Đề cấu trúc mới của Bộ theo [Thành Đức Trung]
Trong không gian $Oxyz$, cho tam giác $ABC$ có $A(0;0;1)$, $B(-3;2;0)$, $C(2;-2;3)$. Viết phương trình tham số đường cao kẻ từ $B$ của tam giác $ABC$.
% \shortans{$-2$}
\loigiai
{
Gọi $\Delta$ là đường cao kẻ từ $B$ của tam giác $ABC$.\\
Ta có $\heva{& \overrightarrow{AB}=(-3;2;-1) \\ & \overrightarrow{AC}=(2;-2;2)} \Rightarrow \left[\overrightarrow{AB},\overrightarrow{AC}\right]=(2;4;2)$. \\
Suy ra một véc-tơ pháp tuyến của mặt phẳng $(ABC)$ là $\overrightarrow{n}=(1;2;1)$.\\
Ta có $\heva{ & \Delta \subset(ABC) \\ & \Delta \perp AC}$, suy ra đường thẳng $\Delta$ nhận $\left[\overrightarrow{n},\overrightarrow{AC}\right]$ làm một véc-tơ chỉ phương.\\
Có $\left[\overrightarrow{n},\overrightarrow{AC}\right]=(6;0;-6)=6\overrightarrow{u}$ với $\overrightarrow{u}=(1;0;-1)$. \\
Suy ra đường thẳng $\Delta$ nhận $\overrightarrow{u}=(1;0;-1)$ làm véc-tơ chỉ phương.\\
Do đó phương trình đường thẳng $\Delta$ là $\Delta \colon \heva{ & x=-3+t \\ & y=2 \\ & z=-t}$.
}
\end{ex}

\begin{ex}%[12-MH-2-MH2025]%[MH-2025, Nguyễn Trần Phong]%[2D4C3-2]
	\immini{Chướng ngại vật \lq\lq  tường cong\rq\rq trong một sân thi đấu X-Game là một khối bê tông có chiều cao từ mặt đất lên là $3$ m. Giao của mặt tường cong và mặt đất là đoạn thẳng $AB = 2$ m. Thiết diện của khối tường cong cắt bởi mặt phẳng vuông góc với $AB$ tại $A$ là một hình tam giác vuông cong $ACE$ với $AC = 4$ m, $CE = 3$ m và cạnh cong $AE$ nằm trên một đường Parabol có trục đối xứng vuông góc với mặt đất. Tại vị trí $M$ là trung điểm của $AC$ thì tường cong có độ cao $1$ m. Thể tích bê tông cần sử dụng để tạo nên khối tường cong đó gần nhất với số nào dưới đây?
	\shortans{$9{,}3$}
	}{\begin{tikzpicture}[>=stealth,x=0.8cm,y=0.8cm,scale=0.7]
	\coordinate[label=below:$A$] (A) at (0,0);
	\coordinate[label=left:$B$] (B) at (-2,2);
	\coordinate[label=below:$C$] (C) at (6,0);
	\coordinate[label=right:$E$] (E) at (6,6);
	\coordinate (G) at (2,4);
	\coordinate (H) at (4,1.8);
	\coordinate (D) at ($(C)+(B)-(A)$);
	\coordinate (F) at ($(E)+(D)-(C)$);
	\coordinate[label=below:$M$] (M) at ($(A)!0.5!(C)$);
	\coordinate (K) at ($(A)!0.5!(B)$);
	\coordinate (N) at ($(M)+(0,1.1)$);
	\draw (D)--(C)--(A)--(B) (C)--(E)--(F) (M)--(N);
	\draw[dashed] (B)--(D)--(F);
	\foreach \diem in {A,B,C,D,E,F,M,F}	\fill (\diem)circle(1.5pt);
	%\tkzLabelPoints[above left](D)
	%\tkzLabelSegment[right](M,N){\footnotesize$1$ m}
	%\tkzLabelSegment[left](A,B){\footnotesize$2$ m}
	%\tkzLabelSegment[right](C,E){\footnotesize$3{,}5$ m}
	\draw(-1,.8) node[left]{\footnotesize $2$ m} (3,0.8) node[right]{\footnotesize$1$ m} (6,3) node[right]{\footnotesize$3$ m};
	
	\draw plot[smooth,tension=.65] coordinates{(B) (G) (F)};
	\draw plot[smooth,tension=.65] coordinates{(A) (H) (E)};
	\fill [pattern = north east lines] plot[smooth,tension=.65] coordinates{(A) (H) (E)} (0,0) --(-2,2)--(4,8)--(6,6)--cycle;
	\fill [draw=none, pattern = north east lines, color=white] (0,0) plot[smooth,tension=.65] coordinates{(B) (G) (F)} (-2,2)--(4,2)--cycle;
	\end{tikzpicture}
	}
	\loigiai{
	\immini{Chọn hệ trục tọa độ như hình vẽ.\\
	Gọi $AE \colon y = ax^2 + bx + c$.\\
	Do $AE$ đi qua $A(-4; 0)$ nên ta có $16a - 4b + c = 0$.\\
	Do $E (0; 3)$ thuộc cạnh cong $AE$ nên $c = 3$ (2).\\
	Do $N(-2; 1)$ thuộc cạnh cong $AE$ nên $4a - 2b + c = 1$ (3).\\
	Từ (1), (2), (3) suy ra $a = \dfrac{1}{8}$, $b = \dfrac{5}{4}$, $c = 3 \Rightarrow AE \colon y = \dfrac{1}{8}x^2 + \dfrac{5}{4}x + 3$.\\
	Khi đó $S_{AEC} = \displaystyle\int_{-4}^0\left(\dfrac{1}{8} x^2 + \dfrac{5}{4}x + 3\right) dx = \dfrac{14}{3}\left(m^2\right)$.
	}{\begin{tikzpicture}[>=stealth,x=0.8cm,y=0.8cm,scale=0.7]
	\coordinate[label=below:$A$] (A) at (0,0);
	\coordinate[label=left:$B$] (B) at (-2,2);
	\coordinate[label=below:$C$] (C) at (6,0);
	\coordinate[label=right:$E$] (E) at (6,6);
	\coordinate (G) at (2,4);
	\coordinate (H) at (4,1.8);
	\coordinate[label = above left:$D$] (D) at ($(C)+(B)-(A)$);
	\coordinate[label = above:$F$] (F) at ($(E)+(D)-(C)$);
	\coordinate[label=below:$M$] (M) at ($(A)!0.5!(C)$);
	\coordinate (K) at ($(A)!0.5!(B)$);
	\coordinate[label=above left:$N$] (N) at ($(M)+(0,1.1)$);
	\draw (D)--(C)--(A)--(B) (C)--(E)--(F) (M)--(N);
	\draw[dashed] (B)--(D) (D)--(F);
	\foreach \diem in {A,B,C,D,E,F,M,F,N}	\fill (\diem)circle(1.5pt);
	\coordinate (x) at ($(M)!1.5!(C)$);
	\draw[->](C)--(x); \draw (x) node[right]{$x$};
	\coordinate (y) at ($(C)!1.4!(E)$);
	\draw[->](E)--(y); \draw (y) node[right]{$y$};
	%\tkzLabelPoints[above left](D)
	%\tkzLabelSegment[right](M,N){\footnotesize$1$ m}
	%\tkzLabelSegment[left](A,B){\footnotesize$2$ m}
	%\tkzLabelSegment[right](C,E){\footnotesize$3{,}5$ m}
	\draw plot[smooth,tension=.65] coordinates{(B) (G) (F)};
	\draw plot[smooth,tension=.65] coordinates{(A) (H) (E)};
	\fill [pattern = north east lines] plot[smooth,tension=.65] coordinates{(A) (H) (E)} (0,0) --(-2,2)--(4,8)--(6,6)--cycle;
	\fill [draw=none, pattern = north east lines, color=white] (0,0) plot[smooth,tension=.65] coordinates{(B) (G) (F)} (-2,2)--(4,2)--cycle;
	\end{tikzpicture}
	}
	\noindent Thể tích khối tường cong là $V = S_{AEC} \cdot AB = \frac{14}{3} \cdot 2 = \dfrac{28}{3} = 9{,}3\left(\mathrm{~m}^3\right)$.	}
	\end{ex}

\begin{ex}%[2H5C1-7]
Người ta thiết kế một mái che hình chữ nhật $ ABCD $ phía trên sân khấu. Gắn hệ trục tọa độ $ Oxyz $ (đơn vị trên trục là mét), người ta xác định được toạ dộ của các điểm như sau: $ A(0;0;8)$, $B(0;20;8)$, $D(15;0;14)$, $C(15;20;14) $. Một cổng chào hình chữ nhật $ EFHG $ với tọa độ điểm $ G(8;0;4) $ dựng vuông góc với mặt đất. Người ta muốn làm các đoạn dây nối thanh ngang $ GE $ với mái che để gắn hoa và đèn led. Độ dài ngắn nhất của mỗi đoạn dây này bằng bao nhiêu mét? (làm tròn đến chữ số thập phân thứ nhất)
\begin{center}
\includegraphics[scale=.3]{images/2P5-1-H5-16}
\end{center}
\shortans{$1{,}8$}
\loigiai{
Ta có $ A(0;0;8)$, $B(0;20;8)$, $D(15;0;14)$, $C(15;20;14) $.\\
Ta có $ \vec{AB}=(0;20;0) $, $\vec{AC}=(15;20;6)$ nên $ \vec{n}_1=\left[\vec{AB},\vec{AC}\right]=(80;0;300) $ là vectơ pháp tuyến của $ (ABCD) $.\\
Mà mặt phẳng mái che $ (ABCD) $ qua $ A(0;0;8)$ nên có phương trình
\[ 80(x-0)+0(y-0)+300(z-8)=0\Leftrightarrow 4x+15z-120=0 .\]
Độ dài ngắn nhất của dây nối thanh ngang $ GE $ với mái che là khoảng cách từ $ G $ đến mái che (mặt phẳng $ ABCD $) là \[ \mathrm{d}(G,(ABCD))=\dfrac{|4\cdot8+0+15\cdot4-120|}{\sqrt{4^2+0^2+15^2}}=\dfrac{28}{\sqrt{241}}=1{,}8\ (\text{m}). \]
}
\end{ex}
\Closesolutionfile{ans}


% \Closesolutionfile{ansbook}
% \HetDe
% \label{De1}
% %
% \cleardoublepage
% \setcounter{page}{1}
% \rfoot{Trang \thepage/\pageref{DA1} - Đáp án trắc nghiệm Mã đề 1}
% \begin{center}
% 	\bfseries ĐÁP ÁN TRẮC NGHIỆM MÃ ĐỀ 1
% \end{center}

% \inputansbox{10}{ans/ansDe1-TN1}
% \inputansbox[3]{2}{ans/ansDe1-TN2}
% \inputansbox{3}{ans/ansDe1-TN3}
% \label{DA1}
% %

% \begin{name}
	{\tenchude}
	{TOÁN 12}
	{LỚP TOÁN THẦY PHÁT}
	{Thời gian: 90 phút - Không kể thời gian phát đề}
\end{name}
\Opensolutionfile{ans}[ans/ansDe2-TN1]
\begin{ex}%[2D4N1-1]%[To 20 - Dot 17 - Chuong 4 - Bai 3 - CD - De 1 - TN]%[Nguyễn Hữu Duy]
Cho hàm số $f(x)$ liên tục trên đoạn $[a;c]$ và $b$ là số thực tùy ý thuộc đoạn $[a;c]$. Nếu biết $\displaystyle\int_a^b f(x) \mathrm{\,d}x = 3$ và $\displaystyle\int_b^c f(x) \mathrm{\,d}x = 8$, thì giá trị của $\displaystyle\int_a^c f(x) \mathrm{\,d}x$ là bao nhiêu?
\choice
{\True $11$}
{$-5$}
{$5$}
{$-11$}
\loigiai
{
Ta có $\displaystyle\int_a^c f(x) \mathrm{\,d}x = \displaystyle\int_a^b f(x) \mathrm{\,d}x + \displaystyle\int_b^c f(x) \mathrm{\,d}x = 3 + 8 = 11$.
}
\end{ex}

\begin{ex}%[12-MH-1-MH2025]%[MH-2025, Mã/Tên TV biên soạn]%[2D4N1-2]
Cho hàm số $y = f(x)$ là một nguyên hàm của hàm số $y = x^3$. Phát biểu nào sau đây là đúng?
\choice
{$f(x) = \dfrac{x^4}{4} + C$}
{$f(x) = 3x^2$}
{$f(x) = 4x^3$}
{\True $f(x) = \dfrac{x^4}{4}$}
\loigiai{Hàm số $f(x) = \dfrac{x^4}{4}$ là một nguyên hàm của hàm số $y = x^3$ vì $f'(x) = x^3$.
}
\end{ex}

\begin{ex}%[2D4H1-3]
Tìm nguyên hàm $\displaystyle\int \dfrac{\cos 2x}{\sin^2 x \cos^2 x} \mathrm{d}x$.
\choice
{ $F(x) = -\cos x - \sin x + C$ }
{ $F(x) = \cos x + \sin x + C$ }
{ $F(x) = \cot x - \tan x + C$ }
{\True $F(x) = -\cot x - \tan x + C$ }
\loigiai{
Ta có: $\displaystyle\int \dfrac{\cos 2x}{\sin^2 x \cos^2 x} \mathrm{d}x = \displaystyle\int \left( \dfrac{1}{\sin^2 x} - \dfrac{1}{\cos^2 x} \right) \mathrm{d}x = -\cot x - \tan x + C$.
}
\end{ex}

\begin{ex}%[2D4H2-1]
Tích phân $\displaystyle\int\limits_{1}^{3}\left[2f(x)+1\right]\mathrm{\,d}x=5$ thì $\displaystyle\int\limits_{1}^{3} f(x)\mathrm{\,d}x$ bằng
\choice
{$3$}
{$2$}
{$\dfrac{3}{4}$}
{\True$\dfrac{3}{2}$}
\loigiai{
Ta có
\begin{eqnarray*}
\displaystyle\int\limits_{1}^{3} \left[2f(x)+1\right]\mathrm{\,d}x=5
&\Leftrightarrow&2\displaystyle\int\limits_{1}^{3} f(x)\mathrm{\,d}x+\displaystyle \int_{1}^{3} \mathrm{\,d}x =5\\
&\Leftrightarrow& 2\displaystyle\int\limits_{1}^{3} f(x)\mathrm{\,d}x+\displaystyle 2 =5 \Leftrightarrow \displaystyle\int\limits_{1}^{3} f(x)\mathrm{\,d}x =\dfrac{3}{2}.
\end{eqnarray*}
}
\end{ex}

\begin{ex}%[2D4N3-1]
\immini{Diện tích phần hình phẳng gạch chéo trong hình vẽ bên được tính theo công thức nào?
\choice
{$\displaystyle\int\limits_{-5}^{-3}\left(x+5\right)\mathrm{d}x-\displaystyle\int\limits_{-3}^1\sqrt{1-x}\mathrm{d}x$}
{\True $\displaystyle\int\limits_{-5}^{-3}\left(x+5\right)\mathrm{d}x+\displaystyle\int\limits_{-3}^1\sqrt{1-x}\mathrm{d}x$}
{$\displaystyle\int\limits_{-5}^1\left[\left(x+5\right)-\sqrt{1-x}\right]\mathrm{d}x$}
{$\displaystyle\int\limits_{-5}^1\left[\sqrt{1-x}-\left(x+5\right)\right]\mathrm{d}x$}
}
{
\begin{tikzpicture}[scale=0.8,>=stealth, font=\footnotesize, line join=round, line cap=round]
\def\a{1} \def\b{-3} \def\c{0} \def\d{2} % Hệ số
\def\xmin{-6} \def\xmax{2}
\def\ymin{-1} \def\ymax{5.5}
%	\draw[color=gray!50,dashed] (\xmin,\ymin) grid (\xmax,\ymax);
\draw[->] (\xmin,0)--(\xmax,0) node [below]{$x$};
\draw[->] (0,\ymin)--(0,\ymax) node [left]{$y$};
\node at (0,0) [below left]{$O$};
\clip (\xmin+0.1,\ymin+0.1) rectangle (\xmax-0.5,\ymax-0.1);
\draw[smooth,samples=300,domain=-5.5:1] plot(\x,{sqrt(1-\x)});
\draw[smooth,samples=300] plot(\x,{1*(\x)+5});
\fill[pattern=north east lines,opacity=0.8] (-5,0)--plot[domain=-5:-3](\x,{1*(\x)+5})--plot[domain=-3:1](\x,{sqrt(1-\x)})--(1,0);
\draw[dashed](-3,0)--(-3,2)
;
\draw[fill=black](-3,0)node[below]{$-3$}circle(1pt)
(1,0)node[below]{$1$}circle(1pt)
(-5,0)node[below]{$-5$}circle(1pt)
(-2,3)node[above,rotate=45,scale=0.8]{$y=x+5$}
(-1.5,1.5)node[above,rotate=-20,scale=0.8]{$y=\sqrt{1-x}$}
;
\end{tikzpicture}
}
\loigiai{
Ta chia hình phẳng gạch chéo làm $2$ phần. Nên diện tích hình phẳng là \[ S=\displaystyle\int\limits_{-5}^{-3}\left(x+5\right)\mathrm{d}x+\displaystyle\int\limits_{-3}^1\sqrt{1-x}\mathrm{d}x.\]}
\end{ex}

\begin{ex}%[2D4H3-3]
\immini[thm]{Cho tam giác $OAB$ vuông tại $A$, có cạnh $OA=a$ nằm trên tục $Ox$ và $\widehat{AOB}=\dfrac{\pi}{3}$.
Gọi $\beta$ là khối tròn xoay sinh ra khi quay miền tam giác $OAB$ xung quanh trục $Ox$. Thể tích của khối $\beta$ bằng
\choice
{$3\pi a^3$}
{\True $\pi a^3$}
{$\dfrac{\pi a^3}{3}$}
{$\dfrac{\pi a^3}{9}$}
}{
\begin{tikzpicture}[scale=0.7, font=\footnotesize, line join=round, line cap=round, >=stealth]
\fill[yellow!30]	(0,0)--(4,2)--(4,0)--cycle;
\draw[->] (4,0) -- (5.5,0) node[right] {$x$};
\draw[->] (0,-2.5) -- (0,2.5) node[above] {$y$};
\draw[->] (0,0) -- (-1.5,-1.5) node[below left] {$z$};
\draw (0,0)--(4,2)	(0,0)--(4,-2)	(4,2)--(4,0)	(-0.5,0)--(0,0);
\draw[dashed] (0,0)--(4,0)	;
\draw (4,0) ellipse (0.5 and 2);
\fill (0,0) circle (1pt) node[above left]{$O$};
\fill (4,0) circle (1pt) node[below]{$A$};
\fill (4,2) circle (1pt) node[above]{$B$};
\clip (4,0) -- (0,0) -- (4,2);
\draw (0,0) circle (1cm);
\draw ($(0,0)+(1,0)$) node[above right]{$\alpha$};
%\node[below] at (2.5,-2.5) {Hình $4.31$};
\end{tikzpicture}
}
\loigiai{
Do $OB$ đi qua gốc tọa độ và tạo với $Ox$ một góc $\dfrac{\pi}{3}$ nên $OB\colon y=\tan \dfrac{\pi}{3}x=\sqrt{3} x$.\\
Khi đó, thể tích của khối $\beta$ là
\[
V=\pi \displaystyle\int_0^a\left(\sqrt{3} x\right)^2\mathrm{d}x=\pi \displaystyle\int_0^a3x^2\mathrm{d}x= \pi x^3\bigg|_0^a=\pi a^3.
\]
}
\end{ex}

\begin{ex}%[2H5N1-1]
Trong không gian với hệ toạ độ $Oxyz$, phương trình nào dưới đây là phương trình của mặt phẳng $(Oyz)$?
\choice
{$y=0$}
{\True $x=0$}
{$y-z=0$}
{$z=0$}
\loigiai{
Mặt phẳng $(Oyz)$ đi qua điểm $O(0 ; 0 ; 0)$ và có véc-tơ  pháp tuyến là $\vec{i}=(1 ; 0 ; 0)$ nên ta có phương trình mặt phẳng $(O y z)$ là  $1(x-0)+0(y-0)+0(z-0)=0 \Leftrightarrow x=0$.
}
\end{ex}

\begin{ex}%[2H5N1-2]
Trong không gian $Oxyz$, mặt phẳng nào sau đây nhận véc-tơ $\overrightarrow{n}=(1;2;3)$ làm véc-tơ pháp tuyến?
\choice
{\True $2x+4y+6z=1$}
{$x-2y+3z+1=0$}
{$x+2y-3z-1=0$}
{$2x-4z+6=0$}
\loigiai{
Mặt phẳng $2x+4y+6z=1$ có một véc-tơ pháp tuyến là $\overrightarrow{m}=(2;4;6)$, nên $\overrightarrow{n}=\dfrac{1}{2}\overrightarrow{m}$ cũng là véc-tơ pháp tuyến của mặt phẳng $2x+4y+6z=1$.
}
\end{ex}

\begin{ex}%[2H5H1-3]
Trong không gian với hệ tọa độ $O x y z$, cho điểm $A(2 ; 4 ; 1) ;$ $ B(-1 ; 1 ; 3)$ và mặt phẳng $(P)\colon x-3 y+2 z-5=0$. Một mặt phẳng $(Q)$ đi qua hai điểm $A, B$ và vuông góc với mặt phẳng $(P)$ có dạng $a x+b y+c z-11=0$. Khẳng định nào sau đây là đúng?
\choice
{\True $a+b+c=5$}
{$a+b+c=15$}
{$a+b+c=-5$}
{$a+b+c=-15$}
\loigiai{Vì $(Q)$ vuông góc với $(P)$ nên $(Q)$ nhận véc-tơ pháp tuyến $\vec{n}=(1 ;-3 ; 2)$ của $(P)$ làm véc-tơ chỉ phương.\\
Mặt khác $(Q)$ đi qua $A$ và $B$ nên $(Q)$ nhận $\overrightarrow{A B}=(-3 ;-3 ; 2)$ làm véc-tơ chỉ phương.\\
$(Q)$ nhận $\overrightarrow{n}_Q=[\vec{n}, \overrightarrow{A B}]=(0 ; 8 ; 12)$ làm véc-tơ pháp tuyến.\\
Vậy phương trình mặt phẳng $(Q)\colon  0(x+1)+8(y-1)+12(z-3)=0\Leftrightarrow 2 y+3 z-11=0$.\\
Vậy $a+b+c=5$.}
\end{ex}

\begin{ex}%[2H5N2-1]%[Dự án 2025 - Đề cấu trúc mới của Bộ theo [Thành Đức Trung]
Trong không gian $Oxyz$, đường thẳng $d\colon \heva{ & x=1+2t \\ & y=3-t \\ & z=1-t}$ $(t\in \mathbb{R})$ đi qua điểm nào dưới đây?
\choice
{$M(1;3;-1)$}
{\True $N(-3;5;3)$}
{$P(3;5;3)$}
{$Q(1;2;-3)$}
\loigiai
{
Thay tọa độ các điểm vào phương trình $d$, ta có $\heva{ & -3=1+2t \\ &5=3-t \\ & 3=1-t} \Leftrightarrow t=-2$. \\
Vậy $N(-3;5;3)\in d$.
}
\end{ex}

\begin{ex}%[2H5N2-7]
Trong không gian với hệ trục $Oxyz$, cho hai đường thẳng $d_1:\dfrac{x}{1}=\dfrac{y+1}{-1}=\dfrac{z-1}{2}$ và $d_2:\dfrac{x+1}{-1}=\dfrac{y}{1}=\dfrac{z-3}{1}$. Góc giữa hai đường thẳng đó bằng
\choice
{$45^\circ $}
{\True $90^\circ $}
{$60^\circ $}
{$30^\circ $}
\loigiai{
Đường thẳng $d_1$ có véctơ chỉ phương $\overrightarrow{u}_1=\left( 1;-1;2 \right)$.\\
Đường thẳng $d_2$ có véctơ chỉ phương $\overrightarrow{u}_2=\left( -1;1;1 \right)$.\\
Gọi $\alpha$ là góc giữa hai đường thẳng trên.\\
Khi đó ta có $\cos \alpha =\left| \cos \left( \overrightarrow{u}_1,\overrightarrow{u}_2 \right) \right|=\dfrac{\left| 1\cdot\left( -1 \right)+\left( -1 \right)\cdot1+2\cdot1 \right|}{\sqrt{1^2+{{\left( -1 \right)}^2}+2^2}\cdot\sqrt{{{\left( -1 \right)}^2}+1^2+1^2}}=0$.\\$\Rightarrow \left( \widehat{d_1,d_2} \right)=90^\circ $.}
\end{ex}

\begin{ex}%[2H5H2-4]
Trong không gian $Oxyz$ cho đường thẳng $\Delta  \colon \dfrac{x}{1} =\dfrac{y+1}{2} =\dfrac{z-1}{1} $ và mặt phẳng $\left(P\right) \colon x-2y-z+3=0$. Đường thẳng nằm trong $\left(P\right)$ đồng thời cắt và vuông góc với $\Delta $ có phương trình là
\choice
{$\heva{x&=1+2t \\ y&=1-t \\ z&=2} $}
{$\heva{x&=-3 \\ y&=-t \\ z&=2t} $}
{$\heva{x&=1+t \\ y&=1-2t \\ z&=2+3t} $}
{\True $\heva{x&=1 \\ y&=1-t \\ z&=2+2t} $}
\loigiai{
Ta có $\Delta  \colon \dfrac{x}{1} =\dfrac{y+1}{2} =\dfrac{z-1}{1} $$\Rightarrow \Delta  \colon \heva{x&=t \\ y&=-1+2t \\ z&=1+t.} $ \\
Gọi $M=\Delta \cap \left(P\right)$ $\Rightarrow M\in \Delta \Rightarrow M\left(t;2t-1;t+1\right)$ $M\in \left(P\right)\Rightarrow t-2\left(2t-1\right)-\left(t+1\right)+3=0 \Leftrightarrow 4-4t=0\Leftrightarrow t=1\Rightarrow M\left(1;1;2\right)$. \\
Véc-tơ pháp tuyến của mặt phẳng $\left(P\right)$ là $\overrightarrow{n}=\left(1;-2;-1\right)$. \\
Véc-tơ chỉ phương của đường thẳng $\Delta $ là $\overrightarrow{u}=\left(1;2;1\right)$.\\
Đường thẳng $d$ nằm trong mặt phẳng $\left(P\right)$ đồng thời cắt và vuông góc với $\Delta $. \\
$\Rightarrow $ đường thẳng $d$ nhận $\dfrac{1}{2} \left[\overrightarrow{n},\overrightarrow{u}\right]=\left(0;-1;2\right)$ làm véc-tơ chỉ phương và $M\left(1;1;2\right)\in d$.\\
$\Rightarrow $ Phương trình đường thẳng $d \colon \heva{x&=1 \\ y&=1-t \\ z&=2+2t.}$}
\end{ex}
\Closesolutionfile{ans}

\TNTF
\Opensolutionfile{ans}[ans/ansDe2-TN2]
\begin{ex}%[2D4H2-4]%[Tổ 20 - Đợt 17 - Chương 4 - - CD - Đề 6]%[Nắng Đông]
Cho $A=\displaystyle\int\limits_0^1 \dfrac{3}{2^x} \mathrm{\,d}x$ và  $B=\displaystyle\int\limits_0^1 4\mathrm{e}^{-2x} \mathrm{\,d}x$.
\choiceTF[t]
{\True $A=-\dfrac{3}{2^x\ln 2}\bigg|_0^1$}
{$B=\dfrac{2}{\mathrm{e}^{2x}}\bigg|_0^1$}
{$A=-\dfrac{3}{2\ln 2}$}
{\True $B=a+\dfrac{b}{\mathrm{e}^2}$, với $a$, $b$ là các số nguyên thì $a\cdot b = -4$}
\loigiai
{Ta có $A=\displaystyle\int\limits_0^1 \dfrac{3}{2^x} \mathrm{\,d}x
= \displaystyle\int\limits_0^1 {3\cdot \left(\dfrac{1}{2}\right)^x} \mathrm{\,d}x
= \dfrac{3}{2^x\ln \dfrac{1}{2}}\bigg|_0^1
=-\dfrac{3}{2^x\ln 2}\bigg|_0^1
= \dfrac{3}{2\ln 2}$.\\
Và $B=\displaystyle\int\limits_0^1 4\mathrm{e}^{-2x} \mathrm{\,d}x
=\displaystyle\int\limits_0^1 4\left(\mathrm{e}^{-2}\right)^x \mathrm{\,d}x
= 4\cdot \dfrac{\left(\mathrm{e}^{-2}\right)^x}{\ln \mathrm{e}^{-2}}\bigg|_0^1
= -\dfrac{2}{\mathrm{e}^{2x}}\bigg|_0^1
=2 - \dfrac{2}{\mathrm{e}^2}$.
\begin{itemchoice}
\itemch Đúng.
\itemch Sai.
\itemch Sai.
\itemch Đúng. Vì $a=2$, $b=-2$ suy ra $a\cdot b = -4$.
\end{itemchoice}
}
\end{ex}

\begin{ex}%[Dat Thai, Dự án Ex-TF-TLN-2024-Dot03]%[2H5H2-5]
Cho đường thẳng $d\colon \heva{& x = 1 + t\\& y = 2 - t\\& z = 1 + 2t}, t\in \mathbb{R}$ và mặt phẳng $(P)\colon x + 2y + z - 5 = 0$. Tọa độ giao điểm $A$ của đường thẳng $d$ và mặt phẳng $(P)$ là $(a;b;c)$.
\choiceTF
{\True $a+b+c = 2$}
{\True Có đúng $1$ số dương trong ba số $a$, $b$, $c$}
{$a$ là số lớn nhất trong ba số $a$, $b$, $c$}
{$a$, $b$, $c$ theo thứ tựu lập thành một cấp số cộng}
\loigiai{
Tọa độ giao điểm $A$ của đường thẳng $d$ và mặt phẳng $P$ là nghiệm của hệ phương trình sau
\begin{eqnarray*}
\heva{& x = 1 + t\\& y = 2 - t\\ & z = 1 + 2t \\& z + 2y + z -5 = 0} \Leftrightarrow \heva{& t  =-1\\& x = 0\\& y = 3 \\& z = -1} \Rightarrow A(0; 3; -1).
\end{eqnarray*}
Vậy
\begin{itemchoice}
\itemch \textbf{Đúng}.
\itemch \textbf{Đúng}. Vì chỉ có mỗi $b$ là số dương trong ba số $a$, $b$, $c$.
\itemch \textbf{Sai}. Vì $b$ là số lớn nhất trong ba số $a$, $b$, $c$.
\itemch \textbf{Sai}. Vì $2b \ne a+ c$.
\end{itemchoice}
}
\end{ex}
\Closesolutionfile{ans}

\TNSA
\Opensolutionfile{ans}[ans/ansDe2-TN3]
\begin{ex}%[2D4H1-1]%[Đào Trung Kiên]
Biết $F(x)$ là một nguyên hàm của hàm số $f(x)=\sin x$ và đồ thị hàm số $y=F(x)$ đi qua điểm $M\left(0;1\right)$. Tính $F\left(\dfrac{\pi}{2}\right)$ (làm tròn kết quả tới hàng đơn vị).
\shortans[]{$2$}
\loigiai{
Ta có $F(x)=\displaystyle\int f(x)\mathrm{\,d}x=-\cos x+C$.\\
Mà đồ thị hàm số $y=F(x)$ đi qua $M(0;1)$ nên $F(0)=1\Leftrightarrow -1+C=1\Leftrightarrow C=2$.\\
Suy ra $F(x)=-\cos x+2$ nên $F\left(\dfrac{\pi}{2}\right)=2$.}
\end{ex}

\begin{ex}%[2D4V3-2]
\immini{
Sàn của một viện bảo tàng mỹ thuật được lát bằng những viên gạch hình vuông cạnh $40$ cm  như hình bên. Biết rằng người thiết kế đã sử dụng các đường cong có phương trình $4x^2=y^4$  và  $4y^2 =x^4$ để tạo hoa văn cho viên gạch. Tính diện tích (đơn vị cm$^2$) phần được tô đậm (làm tròn kết quả đến hàng đơn vị).
}
{
\begin{tikzpicture}
\draw (2,2)--(2,-2)--(-2,-2)--(-2,2)--cycle;
\draw[fill=cyan, smooth, samples=200] plot[domain=-2:2,variable=\y]({0.5*(\y)^2},{\y})--plot[domain=2:0,variable=\x]({\x},{0.5*(\x)*(\x)})--plot[domain=0:2,variable=\x]({\x},{(-0.5)*(\x)*(\x)});
\draw[fill=cyan, smooth, samples=200] plot[domain=-2:2,variable=\y]({-0.5*(\y)^2},{\y})--plot[domain=-2:0,variable=\x]({\x},{0.5*(\x)*(\x)})--plot[domain=0:-2,variable=\x]({\x},{(-0.5)*(\x)*(\x)});
\end{tikzpicture}
}
\shortans{$533$}
\loigiai{
\immini{
Do tính đối xứng nên diện tích cần tìm bằng $4$ lần diện tích phần tô đậm của hình vẽ bên. Ta chỉ xét đồ thị trong góc phần tư thứ nhất do đó
\begin{align*}
&4x^2=y^4 \Leftrightarrow y = \sqrt{2x} \\
&4y^2 =x^4 \Leftrightarrow  y = \dfrac{1}{2} x^2.
\end{align*}
}
{
\begin{tikzpicture}[scale=1,>=stealth]
\draw[->] (-1,0)--(3,0) node[below] {$x$};
\draw[->] (0,-1)--(0,3) node[left] {$y$};
%\draw[fill] (1,0) circle (1pt) node[below] {$1$};
\draw[fill] (0,0) circle node[below left=-2pt] {$O$};
\draw[dashed] (2,0)--(2,2)--(0,2);
\draw[fill] (2,0) circle (1pt) node[below] {$2$};
\draw[fill] (0,2) circle (1pt) node[left] {$2$};
\draw[fill=cyan, smooth, samples=200] plot[domain=0:2,variable=\y]({0.5*(\y)^2},{\y})--plot[domain=2:0,variable=\x]({\x},{0.5*(\x)*(\x)});
\end{tikzpicture}
}
\noindent Một đơn vị trong hệ tọa độ $Oxy$ bằng $10$ cm, do đó diện tích của phần tô đậm ban đầu là
\[
S =4\cdot 10^2 \int \limits_0^2 \left ( \sqrt{2x}-\dfrac{1}{2}x ^2\right) \mathrm{d}x = 400 \left ( \dfrac{2\sqrt{2}}{3} \sqrt{x^3}-\dfrac{1}{6}x^3 \right ) \Bigg|_0^2= \dfrac{1600}{3}\approx 533 \ \mathrm{(cm^2)}.
\]
}
\end{ex}

\begin{ex}%[2H5H2-7]
Số các mặt phẳng $(\alpha)$ chứa đường thẳng $d\colon\dfrac{x}{1}=\dfrac{y}{-1}=\dfrac{z}{-3}$ và tạo với mặt phẳng $(P)\colon 2x-z+1=0$ góc $45^\circ $ bằng
\shortans{$2$}
\loigiai{
Đường thẳng $d$ đi qua điểm $O(0;0;0)$ có véc-tơ chỉ phương $\overrightarrow{u}=(1;-1;-3)$.\\
Ta có $(\alpha)$ qua $O$ có véc-tơ pháp tuyến $\overrightarrow{n}=(a;b;c)$ có dạng $ax+by+cz=0$.\\
Vì $\overrightarrow{n}\perp \overrightarrow{u}$ nên $\overrightarrow{n}\cdot \overrightarrow{u}=0$. Do đó $ a-b-3c=0$.\\
Mặt phẳng $(P)\colon 2x-z+1=0$ có véc-tơ pháp tuyến $\overrightarrow{k}=(2;0;-1)$.\\
Ta có
\begin{eqnarray*}
&&\cos 45^\circ =\dfrac{\left| \overrightarrow{n}\cdot \overrightarrow{k}\right|}{\left|\overrightarrow{n}\right|\cdot\left|\overrightarrow{k}\right|}\\
&\Leftrightarrow&\dfrac{\left| 2a-c\right|}{\sqrt{5(a^2+b^2+c^2)}}=\dfrac{\sqrt{2}}{2}\\
&\Leftrightarrow&10(a^2+b^2+c^2)=(4a-2c)^2\\
&\Leftrightarrow&10(b^2+6bc+9c^2+b^2+c^2)=(4b+12c-2c)^2\\
&\Leftrightarrow&10(2b^2+6bc+10c^2)=(4b+10c)^2\\
&\Leftrightarrow&4b^2-20bc=0\\
&\Leftrightarrow&\hoac{&b=0 \\&b=5c.}
\end{eqnarray*}
Xét
\begin{itemize}
\item $b=0\Rightarrow a=3c$ nên $(\alpha)\colon x+3z=0$.
\item $b=5c$, chọn $c=1\Rightarrow b=5$, $a=8$ nên $(\alpha)\colon 8x+5y+z=0$.
\end{itemize}
}
\end{ex}

\begin{ex}%[12-MH-2-MH2025]%[MH-2025, Nguyễn Trần Phong]%[2H5V1-7]
Một phần sân nhà bác An có dạng hình thang $ABCD$ vuông tại $A$ và $B$ với độ dài $AB=9$ m, $AD=5$ m và $BC=6$ m. Theo thiết kế ban đầu thì mặt sân bằng phẳng và $A$, $B$, $C$, $D$ có độ cao như nhau. Sau đó bác An thay đổi thiết kế để nước có thể thoát về phía góc sân ở vị trí $C$ bằng cách giữ nguyên độ cao ở $A$, giảm độ cao của sân ở vị trí $B$ và $D$ xuống thấp hơn độ cao ở $A$ lần lượt là $6$ cm và $3{,}6$ cm. Để mặt sân sau khi lát gạch vẫn là bề mặt phẳng thì bác An cần phải giảm độ cao ở $C$ xuống bao nhiêu cen-ti-mét so với độ cao ở $A$? \textit{Kết quả làm tròn đến hàng phần mười)}
\shortans{$10{,}3$}
\indent\indent\indent\indent\indent\indent\indent\indent\indent\begin{tikzpicture}[scale=0.5, font=\footnotesize,line join=round, line cap=round, >=stealth]
\path
(0,0) coordinate (A)
(9,0) coordinate(B)
(9,-6) coordinate(C)
(0,-5) coordinate(D)
;
\draw[thick] (A)--(B)--(C)--(D)--cycle;
\node [above] at ($(A)!0.5!(B)$) {$9$ m};
\node [right] at ($(B)!0.5!(C)$) {$6$ m};
\node [left] at ($(A)!0.5!(D)$) {$5$ m};
\foreach \i/\g in {A/90,B/90,C/-90,D/-90}{\draw[fill=black](\i) circle (0pt) ($(\i)+(\g:4mm)$) node[scale=1]{$\i$};}
\end{tikzpicture}
\loigiai{
Tại vị trí ban đầu $A$, $B$, $C$, $D$ có độ cao như nhau, chọn hệ trục tọa độ có gốc tọa độ là điểm $A$ và các trục tọa độ lần lượt là $AD$, $AB$ và $Az$, với $Az \perp (ABCD)$.\\
Khi đó $A(0 ; 0 ; 0)$, $D(5; 0 ; 0)$, $B(0; 9 ; 0)$, $C(6; 9 ; 0)$.\\
Sau đó bác An thay đổi thiết kế để nước có thể thoát về phía góc sân ở vị trí $C$ bằng cách giữ nguyên độ cao ở $A$, giảm độ cao của sân ở vị trí $B$ và $D$ xuống thấp hơn độ cao ở $A$ lần lượt là $6$ cm và $3{,}6$ cm.\\
Khi đó, $A(0 ; 0 ; 0)$, $D(5; 0 ; -3{,}6)$, $B(0; 9 ; -6)$.\\
Ta có $\overrightarrow{AB}=(0 ; 9 ; -6)$, $ \overrightarrow{AD}=(5 ; 0 ; -3{,}6)$ là cặp véc-tơ chỉ phương của mặt phẳng $(ABD)$ nên một véc-tơ pháp tuyến của $(ABD)$ là $\left[\overrightarrow{AB}, \overrightarrow{AD}\right]=(-32{,}4 ; -30 ; -45)$.\\
Vậy mặt phẳng $(ABD)$ qua $A(0 ; 0 ; 0)$ và có véc-tơ pháp tuyến $\vec{n}=(-32{,}4 ; -30 ; -45)$ nên có phương trình là
\allowdisplaybreaks
\begin{eqnarray*}
-32{,}4 (x-2)-30(y+1)-45(z-3)=0 \qquad \text{hay } -32{,}4 x -30y -45z=0.
\end{eqnarray*}
Để mặt sân sau khi lát gạch vẫn là bề mặt phẳng thì bác An cần phải giảm độ cao ở $C$ xuống $k$ centimét so với độ cao ở $A$ nên suy ra $C(6; 9 ; -k)$.\\
Ta có $A$, $B$, $C$, $D$ đồng phẳng\\
$\Leftrightarrow C \in (ABD)$\\
$\Leftrightarrow -32{,}4\cdot 6 -30 \cdot 9 -45\cdot (-k)=0$\\
$\Leftrightarrow k=10{,}32$.\\
Vậy bác An cần phải giảm độ cao ở $C$ xuống $10{,}3$ cen-ti-mét so với độ cao ở $A$.
}\end{ex}

\TL
\begin{ex} %[2H5H2-3].
Trong không gian $Oxyz$, cho ba điểm $A(3 ;-2 ;-2), B(3 ; 2 ; 0), C(0 ; 2 ; 1)$. Phương trình mặt phẳng $(ABC)$ có dạng $=ax+by+cz+d=0$. Tính $a+b+c$.
\shortans{$5$}
\loigiai{
Ta có $\overrightarrow{AB}=(0 ; 4 ; 2), \overrightarrow{AC}=(-3 ; 4 ; 3), \overrightarrow{n}=\left[ \overrightarrow{B} ; \overrightarrow{C}\right]=(4 ;-6 ; 12)$.\\
Ta có $\overrightarrow{n}=(4 ;-6 ; 12)$ cùng phương $\overrightarrow{n}_{1}=(2 ;-3 ; 6)$.\\
Mặt phẳng $(ABC)$ đi qua điểm $C(0 ; 2 ; 1)$ và có một véc-tơ pháp tuyến $\overrightarrow{n}_{1}=(2 ;-3 ; 6)$ nên $(ABC)$ có phương trình là
\[2(x-0)-3(y-2)+6(z-1)=0 \Leftrightarrow 2 x-3 y+6 z=0.\]
Vậy phương trình mặt phẳng cần tìm là $2x-3y+6z=0$.\\
Suy ra $a+b+c=5$.
}
\end{ex}

\begin{ex}%[2D4C3-2]
	\immini[thm]{Một họa tiết hình cánh bướm như hình vẽ bên. Phần tô đậm được đính đá với giá thành $500\,000$/$\,\mathrm{m^2}$. Phần còn lại được tô màu với giá thành $250\,000$/$\,\mathrm{m^2}$. Cho $AB=4$\,dm; $BC=8$\,dm. Hỏi để trang trí $1\,000$ họa tiết như vậy cần số tiền là bào nhiêu? (làm tròn đến hàng nghìn)
	% \choice
	% {$105\,660\,667$}
	% {\True $106\,666\,667$}
	% {$ 107\,665\,667$}
	% {$ 108\,665\,667$}
	}{
	\begin{tikzpicture}[line join=round, line cap=round,>=stealth,thick,scale=0.5]
	\tikzset{every node/.style={scale=0.9}}
	\begin{scope}
	\draw[fill=gray!35](-2,0)--plot[samples=200,domain=-2:2,smooth,variable=\x] (\x,{(\x)^2})--(2,0);
	\draw[fill=gray!35](-2,0)--plot[samples=200,domain=-2:2,smooth,variable=\x] (\x,{-1*(\x)^2})--(2,0);
	\draw[fill=black](-2,4) circle (1.5pt) node[left]{$A$} (2,4) circle (1.5pt) node[right]{$B$} (2,-4) circle (1.5pt) node[right]{$C$} (-2,-4) circle (1.5pt) node[left]{$D$};
	\draw (-2,4)--(2,4) (2,-4)--(-2,-4);
	\end{scope}
	\draw[->] (-3,0)--(3,0) node[below left] {$x$};
	\draw[->] (0,-5)--(0,5) node[below left] {$y$};
	\end{tikzpicture}
	}
	\loigiai{
	Vì $AB=4$\,dm; $BC=8$\,dm $\Rightarrow A(-2;4)$, $B(2;4)$, $C(2;-4)$, $D(-2;-4)$.\\
	parabol là $y=x^2$ hoặc $y=-x^2$.\\
	Diện tích phần tô đậm là $S_1=4\displaystyle\int\limits_0^2{x^2}\mathrm{\,d}x=\dfrac{32}{3}\mathrm{\,(dm^2)}$.\\
	Diện tích hình chữ nhật là $S=4\cdot 8=32\mathrm{\,(dm^2)}$.\\
	Diện tích phần trắng là $S_2=S-S_1=32-\dfrac{32}{3}=\dfrac{64}{3}\mathrm{\,(dm^2)}$.\\
	Tổng chi phí trang chí là $T=\left(\dfrac{32}{3}\cdot 5\,000+\dfrac{64}{3}\cdot 2\,500\right)\cdot 1\,000\approx 106\,667\,000$.
	}
	\end{ex}

\begin{ex}%[Mức độ ]giảng 12 New - 4in1, Đoàn Hùng]%[2H5V1-7]
	\immini
	{Trong không gian với hệ tọa độ $Oxyz$ (đơn vị trên mỗi trục toạ độ là km), một máy bay đang ở vị trí $A(3;-2{,}5; 0{,}5)$ và sẽ hạ cánh ở vị trí $B(3; 7{,}5; 0)$ trên đường băng (hình bên). Có một lớp mây được mô phỏng bởi một mặt phẳng $(\alpha)$ đi qua ba điểm $M(9;0;0)$, $N(0;-9;0)$, $P(0;0;0{,}9)$. Tính độ cao của máy bay khi máy bay xuyên qua đám mây để hạ cánh.}
	{\begin{tikzpicture}[line join = round, line cap = round,>=stealth,font=\footnotesize,scale=.5]
	\path
	(0,0) coordinate (O)
	(-9,0) coordinate (N)
	($(O)!1!40:(N)$) coordinate (M)
	(0,0.9) coordinate (P)
	($(O)!1.2!(M)$) coordinate (x)
	(-5,0.8) coordinate (A)
	(4,-1.5) coordinate (B)
	($(A)!2.1cm!(B)$) coordinate (C)
	(intersection of A--B and O--M) coordinate (B1)
	;
	\draw[line width=0.3mm,red] (A)--(C) (B1)--(B);
	\draw[line width=0.3mm,red,dashed] (C)--(B1);
	\draw[->,>=stealth,line width=0.3mm,blue] (0,0)--(x) node[right=0.2cm]{$x$};
	\draw[->,>=stealth,line width=0.3mm,blue] (-10,0)--(-9,0) (0,0) node[above right]{$O$}--(7,0) node[below]{$y$};
	\draw[line width=0.3mm,blue,dashed] (-9,0)--(0,0);
	\draw[->,>=stealth,line width=0.3mm,blue] (0,0)--(0,3) node[left]{$z$};
	\draw[line width=0.3mm,blue] (N)--(P)--(M) (N)--(M);
	\foreach \x/\gm in {N/90,P/140} \fill (\x) circle (1pt) ($(\x)+(\gm:5mm)$)node[blue]{$\x$};
	\filldraw[red] (N)node[below left,blue]{$-9$} (P)node[blue,above right]{$0{,}9$} (M)node[blue,right=0.1cm]{$9$}node[blue,left=0.1cm]{$M$} (C) circle (3pt) node[below=0.2cm,blue]{\scriptsize $C$} (A) circle (3pt)node[above left,red,blue]{$A$} (B)circle (3pt) node[below,blue]{$B$};
	\end{tikzpicture}}
	% \shortans{$0{,}45$}
	\loigiai{
	Giả sử điểm $C\left(x_C;y_C;z_C\right)$ là vị trí mà máy bay xuyên qua đám mây để hạ cánh, suy ra $C\in (\alpha)$. Áp dụng phương trình mặt phẳng theo đoạn chắn, ta thấy mặt phẳng $(\alpha)$ có phương trình là
	\[\dfrac{x}{9}-\dfrac{y}{9}+\dfrac{z}{0{,}9}=1 \Leftrightarrow x-y+10z=9 \Rightarrow x_C-y_C+10z_C=9.\]
	Mặt khác, vì $\vec{AC}$, $\vec{AB}$ là hai véc-tơ cùng hướng nên tồn tại số thực $t>0$ sao cho $\vec{AC}=t\cdot \vec{AB}$.\\
	Do $\vv{AC}=\left(x_C-3;y_C+2{,}5;z_C-0{,}5\right)$; $\vv{AB}=\left(3-3;7{,}5+2{,}5;0-0{,}5\right)=\left(0;10;-0{,}5\right)$\\
	nên $\heva{&x_C-3=0t\\&y_C+2{,}5=10t\\&z_C-0{,}5=-0{,}5t} \Leftrightarrow \heva{&x_C=3\\&y_C=10t-2{,}5\\&z_C=-0{,}5t+0{,}5.}$\\
	Vì $C\in(\alpha)$ nên $3-(10 t-2{,}5)+10(-0{,}5 t+0{,}5)=9 \Leftrightarrow t=0{,}1$. Suy ra $C(3;-1{,}5;0{,}45)$.\\
	Vậy tại vị trí $C$, độ cao của máy bay là $0{,}45$ km.
	}
	\end{ex}
\Closesolutionfile{ans}


% \Closesolutionfile{ansbook}
% \HetDe
% \label{De2}
% %
% \cleardoublepage
% \setcounter{page}{1}
% \rfoot{Trang \thepage/\pageref{DA2} - Đáp án trắc nghiệm Mã đề 2}
% \begin{center}
% 	\bfseries ĐÁP ÁN TRẮC NGHIỆM MÃ ĐỀ 2
% \end{center}

% \inputansbox{10}{ans/ansDe2-TN1}
% \inputansbox[3]{2}{ans/ansDe2-TN2}
% \inputansbox{3}{ans/ansDe2-TN3}
% \label{DA2}
% %

% \begin{name}
	{\tenchude}
	{TOÁN 12}
	{LỚP TOÁN THẦY PHÁT}
	{Thời gian: 90 phút - Không kể thời gian phát đề}
\end{name}
\Opensolutionfile{ans}[ans/ansDe3-TN1]
\begin{ex}%[2D4N1-1]%[To 20 - Dot 17 - Chuong 4 - Bai 3 - CD - De 1 - TN]%[Nguyễn Hữu Duy]
Cho hàm số $f(x)$ liên tục trên đoạn $[a;b]$. Nếu biết $\displaystyle\int_a^b f(x) \mathrm{\,d}x = 2025$, thì giá trị $\displaystyle\int_a^b 2f(x) \mathrm{\,d}x$ là bao nhiêu?
\choice
{\True $4050$}
{$4051$}
{$4052$}
{$4053$}
\loigiai
{ Ta có $\displaystyle\int_a^b 2f(x) \mathrm{\,d}x = 2\displaystyle\int_a^b f(x) \mathrm{\,d}x = 2 \cdot 2025 = 4050$.
}
\end{ex}

\begin{ex}%[2D4N1-2]%[2015_K12 Huyên Nguyễn]
Họ nguyên hàm của hàm số $f(x)=2x+\dfrac{3}{x^2}$ là
\choice
{\True $x^2-\dfrac{3}{x}+C$}
{$x^2+\dfrac{3}{x}+C$}
{$x^2+3\ln {x^2}+C$}
{$x^2+\dfrac{3}{2}\ln |x^2|+C$}
\loigiai{
$\displaystyle\int\left(2x+\dfrac{3}{x^2}\right)\mathrm{\,d}x=x^2-\dfrac{3}{x}+C$.
}
\end{ex}

\begin{ex}%[2D4H1-3]
Nguyên hàm $F(x)$ của hàm số $f(x) = \cos x$ thỏa mãn $F(0) = 1$ là
\choice
{\True $F(x) = \sin x + 1$ }
{ $F(x) = -\sin x + 1$ }
{ $F(x) = \cos x$ }
{ $F(x) = -\cos x + 2$ }
\loigiai{
Ta có $F(x) = \displaystyle\int f(x) \mathrm{d}x = \displaystyle\int \cos x \mathrm{d}x = \sin x + C$. \\
Vì $F(0) = 1$, nên $\sin 0 + C = 1 \Rightarrow C = 1$. \\
Vậy $F(x) = \sin x + 1$.
}
\end{ex}

\begin{ex}%[DỰ ÁN TEX 12 TOÁN TỪ TÂM - TRƯƠNG ĐĂNG KHOA]%[2D4H2-1]
Cho hàm số $y=f(x)$ có đạo hàm $f'(x)$ và $f'(x)$ liên tục trên đoạn $[a;b]$. Gọi $F(x)$ là một nguyên hàm của hàm số $f(x)$ trên đoạn $[a;b]$. Chọn mệnh đề đúng.
\choice
{\True $f(b)-f(a)=\displaystyle\int\limits_{a}^{b}f'(x)\mathrm{\,d}x$}
{$F(b)-F(a)=\displaystyle\int\limits_{a}^{b}f'(x)\mathrm{\,d}x$}
{$f(b)-f(a)=\displaystyle\int\limits_{a}^{b}F(x)\mathrm{\,d}x$}
{$f'(b)-f'(a)=\displaystyle\int\limits_{a}^{b}f'(x)\mathrm{\,d}x$}
\loigiai{Ta có $f(b)-f(a)=\displaystyle\int\limits_{a}^{b}f'(x)\mathrm{\,d}x$.}
\end{ex}

\begin{ex}%[Nguyễn Tuấn, dự án sáng tác đề 12]%[2D4N3-1]
\immini{Diện tích phần hình phẳng gạch chéo trong hình vẽ bên được tính theo công thức nào dưới đây?
\choice
{$\displaystyle\int\limits_{-1}^2\left(2x^2-2x-4\right)\mathrm{\,d}x$}
{$\displaystyle\int\limits_{-1}^2(-2x+2)\mathrm{\,d}x$}
{$\displaystyle\int\limits_{-1}^2(2x-2)\mathrm{\,d}x$}
{\True $\displaystyle\int\limits_{-1}^2\left(-2x^2+2x+4\right)\mathrm{\,d}x$}}
{\begin{tikzpicture}[scale=.7, font=\footnotesize, line join=round, line cap=round, >=stealth]
\draw[->] (-1.5,0) -- (3,0)node[below]{\footnotesize $x$};
\draw (-1,0) circle (.5pt)node[below]{\footnotesize $-1$};
\draw (2,0) circle (.5pt)node[above]{\footnotesize $2$};
\draw[->,color=black] (0,-2.5) -- (0,3.5)node[below left]{\footnotesize $y$};
\fill[pattern=north west lines] plot[smooth,samples=100,domain=-1:2] (\x,{(\x)^2-2*(\x)-1})--plot[smooth,samples=100,domain=2:-1] (\x,{-(\x)^2+3});
\draw[thick,smooth,samples=100,domain=-1.5:2.2] plot(\x,{-(\x)^2+3});
\draw[thick,smooth,samples=100,domain=-1.2:3] plot(\x,{(\x)^2-2*(\x)-1});
\draw[dashed] (2,0) -- (2,-1) (-1,0) -- (-1,2);
\filldraw[fill=white] (0,0) circle (1pt)node[shift={(-45:6pt)}]{\footnotesize $O$};
\draw (2,3) node{\footnotesize $y=-x^2+3$};
\draw (-1,-2.2) node{\footnotesize $y=x^2-2x-1$};
\end{tikzpicture}}
\loigiai{
$S=\displaystyle\int\limits_{-1}^2\left[\left(-x^2+3\right)-\left(x^2-2x-1\right)\right]\mathrm{\,d}x=\displaystyle\int\limits_{-1}^2\left(-2x^2+2x+4\right)\mathrm{\,d}x$.
}
\end{ex}

\begin{ex}%[2D4H3-3]
\immini[thm]{
Cho tam giác vuông $OAB$, có cạnh $OA=a$ nằm trên tục $Ox$ và $\widehat{AOB}=\alpha \left(0< \alpha \le \dfrac{\pi}{4} \right)$. Gọi $\beta$ là khối tròn xoay sinh ra khi quay miền tam giác $OAB$ xung quanh trục $Ox$. Thể tích $V$ của $\beta$ tính theo $a$ và $\alpha$ là
\choice
{\True $V=\dfrac{\pi \tan ^2\alpha\cdot a^3}{3}$}
{$V=\dfrac{3\pi \tan ^2\alpha\cdot a^3}{2}$}
{$V=\dfrac{\pi \sin ^2\alpha\cdot a^3}{3}$}
{$V=\dfrac{2\pi \cos^2\alpha\cdot a^3}{3}$}
}{
\begin{tikzpicture}[scale=0.7, font=\footnotesize, line join=round, line cap=round, >=stealth]
\fill[yellow!30]	(0,0)--(4,2)--(4,0)--cycle;
\draw[->] (4,0) -- (5.5,0) node[right] {$x$};
\draw[->] (0,-2.5) -- (0,2.5) node[above] {$y$};
\draw[->] (0,0) -- (-1.5,-1.5) node[below left] {$z$};
\draw (0,0)--(4,2)	(0,0)--(4,-2)	(4,2)--(4,0)	(-0.5,0)--(0,0);
\draw[dashed] (0,0)--(4,0)	;
\draw (4,0) ellipse (0.5 and 2);
\fill (0,0) circle (1pt) node[above left]{$O$};
\fill (4,0) circle (1pt) node[below]{$A$};
\fill (4,2) circle (1pt) node[above]{$B$};
\clip (4,0) -- (0,0) -- (4,2);
\draw (0,0) circle (1cm);
\draw ($(0,0)+(1,0)$) node[above right]{$\alpha$};
%\node[below] at (2.5,-2.5) {Hình $4.31$};
\end{tikzpicture}
}
\loigiai{
Do $OB$ đi qua gốc tọa độ và tạo với $Ox$ một góc $\alpha$ nên $OB\colon y=x\cdot \tan \alpha$.\\
Khi đó, thể tích của khối $\beta$ là
\[
V=\pi \displaystyle\int_0^a\left(x\cdot \tan \alpha \right)^2\mathrm{d}x=\pi \tan ^2\alpha \displaystyle\int_0^ax^2\mathrm{d}x=\dfrac{\pi \tan ^2\alpha\cdot x^3}{3}\bigg|_0^a=\dfrac{\pi \tan ^2\alpha\cdot a^3}{3}.
\]
}
\end{ex}

\begin{ex}%[2H5N1-1]
Trong không gian với hệ tọa độ $O x y z$, phương trình nào sau đây là phương trình của mặt phẳng $O z x$ ?
\choice
{$x=0$}
{$y-1=0$}
{\True $y=0$}
{$z=0$}
\loigiai{
Ta có mặt phẳng $(Oxz)$ đi qua điểm $O(0 ; 0 ; 0)$ và vuông góc với trục $O y$ nên có VTPT $\vec{n}=(0 ; 1 ; 0)$.\\
Do đó phương trình của mặt phẳng $(Oxz)$ là $y=0$.
}
\end{ex}

\begin{ex}%[2H5N1-2]
Trong không gian $Oxyz$, véc-tơ nào sau đây là một véc-tơ pháp tuyến của mặt phẳng $(Oxy)$?
\choice
{$\overrightarrow{n}=(1;1;0)$}
{$\overrightarrow{j}=(0;1;0)$}
{$\overrightarrow{i}=(1;0;0)$}
{\True $\overrightarrow{k}=(0;0;1)$}
\loigiai{
Mặt phẳng $(Oxy)$ vuông góc với trục $Oz$ nên véc-tơ $\overrightarrow{k}=(0;0;1)$ là một véc-tơ pháp tuyến của mặt phẳng $(Oxy)$.
}
\end{ex}

\begin{ex}%[2H5H1-3]
Trong không gian với hệ trục tọa độ $O x y z$, cho mặt phẳng $(P)\colon  a x+b y+c z-9=0$ chứa hai điểm $A(3 ; 2 ; 1),$ $ B(-3 ; 5 ; 2)$ và vuông góc với mặt phẳng $(Q)\colon  3 x+y+z+4=0$. Tính tổng $S=a+b+c$?
\choice
{$S=-12$}
{$S=2$}
{\True $S=-4$}
{$S=-2$}
\loigiai{
$\overrightarrow{A B}=(-6 ; 3 ; 1)$.\\
$\overrightarrow{n}_{(Q)}=(3 ; 1 ; 1)$ là véc-tơ pháp tuyến  của $(Q)$.\\
Mặt phẳng $(P)$ chứa hai điểm $A(3 ; 2 ; 1),$ $ B(-3 ; 5 ; 2)$ và vuông góc với mặt phẳng $(Q)$.\\
Suy ra $ \overrightarrow{n}_{(P)}=\left[\overrightarrow{A B}, \overrightarrow{n}_{(Q)}\right]=(2 ; 9 ;-15)$ là véc-tơ pháp tuyến  của $(P)$.\\
$A(3 ; 2 ; 1) \in(P)\Rightarrow(P)\colon 2 x+9 y-15 z-9=0$ hoặc $(P)\colon -2 x-9 y+15 z+9=0$.\\
Mặt khác $(P)\colon a x+b y+c z-9=0 \Rightarrow a=2 ; $ $b=9 ;$ $ c=-15$.\\
Vậy $S=a+b+c=2+9+(-15)=-4$.
}
\end{ex}

\begin{ex}%[2H5N2-1]
Đường thẳng $(\Delta)\colon \dfrac{x-1}{2}=\dfrac{y+2}{1}=\dfrac{z}{-1}$ đi qua điểm nào dưới đây?
\choice
{\True $M(1 ;-2 ; 0)$}
{$N(-1 ; 2 ; 0)$}
{$P(3 ; 1 ;-1)$}
{$Q(-1 ;-2 ; 0)$}
\loigiai{
Ta có $\dfrac{1-1}{2}=\dfrac{2-2}{1}=\dfrac{0}{-1}$ nên điểm $M(1 ;-2 ; 0)$ thuộc đường thẳng $(\Delta)$.
}
\end{ex}

\begin{ex}%[2H5N2-7]
Trong không gian $Oxyz$, cho mặt phẳng $(P)\colon -\sqrt{3}x+y+1=0$. Tính góc tạo bởi $(P)$ với trục $Ox$?
\choice
{\True $60^\circ$}
{$30^\circ$}
{$120^\circ$}
{$150^\circ$}
\loigiai{
Mặt phẳng $(P)$ có véc tơ pháp tuyến $\overrightarrow{n}=(-\sqrt{3};1;0)$\\
Trục $Ox$ có có véc tơ pháp tuyến $\overrightarrow{i}=(1;0;0)$.\\
Góc tạo bởi $(P)$ với trục $Ox$\\
$\sin((P),Ox)=\left| \cos((P), Ox) \right|=\dfrac{\left| \overrightarrow{n}\cdot\overrightarrow{i} \right|}{\left| \overrightarrow{n} \right|\cdot\left| \overrightarrow{i} \right|}=\dfrac{\left| -\sqrt{3}\cdot1+1\cdot0+0\cdot0 \right|}{\sqrt{3+1}\cdot\sqrt{1}}=\dfrac{\sqrt{3}}{2}$.\\
Vậy góc tạo bởi $(P)$ với trục $Ox$ bằng $60^\circ$.}
\end{ex}

\begin{ex}%[2H5H2-4]
Trong không gian với hệ tọa độ $Oxyz$ cho $A\left(1; -1; 3\right)$ và hai đường thẳng $d_{1}  \colon \dfrac{x-4}{1} =\dfrac{y+2}{4} =\dfrac{z-1}{-2} ,$ $d_{2}  \colon \dfrac{x-2}{1} =\dfrac{y+1}{-1} =\dfrac{z-1}{1}$. Phương trình đường thẳng qua $A$, vuông góc với $d_{1} $ và cắt $d_{2} $ là
\choice
{$\dfrac{x-1}{2} =\dfrac{y+1}{1} =\dfrac{z-3}{3} $}
{$\dfrac{x-1}{4} =\dfrac{y+1}{1} =\dfrac{z-3}{4} $}
{$\dfrac{x-1}{-1} =\dfrac{y+1}{2} =\dfrac{z-3}{3} $}
{\True $\dfrac{x-1}{2} =\dfrac{y+1}{-1} =\dfrac{z-3}{-1} $}
\loigiai{
Gọi $d$ là đường thẳng qua $A$ và $d$ cắt $d_{2} $ tại $K$. Khi đó $K\left(2+t; -1-t; 1+t\right)$. \\
Ta có $\overrightarrow{AK}=\left(1+t; -t; t-2\right)$. Đường $AK\perp d_{1} $$\Leftrightarrow \overrightarrow{AK}\cdot\overrightarrow{u_{1} }=0$, với $\vec{u}_{1} =\left(1; 4; -2\right)$ là một véc-tơ chỉ phương của $d_{1} $. \\
Do đó $1+t-4t-2t+4=0\Leftrightarrow t=1$, suy ra $\overrightarrow{AK}=\left(2; -1; -1\right)$. \\
Vậy phương trình đường thẳng $d \colon \dfrac{x-1}{2} =\dfrac{y+1}{-1} =\dfrac{z-3}{-1}.$}
\end{ex}
\Closesolutionfile{ans}

\TNTF
\Opensolutionfile{ans}[ans/ansDe3-TN2]
\begin{ex}%[2D4H2-4]
Các mệnh đề sau đây đúng hay sai.
\choiceTF
{\True $\displaystyle\int\limits_0^1\dfrac{\mathrm{e}^{2x}-4}{\mathrm{e}^x+2}\mathrm{\,d}x=\mathrm{e}-3$}
{$\displaystyle\int\limits_0^1\dfrac{\mathrm{e}^x}{2^x}\mathrm{\,d}x=\dfrac{\mathrm{e}}{2}+1$}
{\True $\displaystyle\int\limits_1^2\mathrm{e}^x\left(1-\dfrac{\mathrm{e}^{-x}}{x}\right)\mathrm{\,d}x=\mathrm{e}^2-\mathrm{e}-\ln 2$}
{$\displaystyle\int\limits_0^1\dfrac{\mathrm{e}^{2x-1}-\mathrm{e}^{-3x}+1}{\mathrm{e}^x}\mathrm{\,d}x=\mathrm{e}^4-1$}
\loigiai{\begin{itemchoice}
\itemch Đúng. \allowdisplaybreaks
\begin{eqnarray*} \displaystyle\int\limits_0^1\dfrac{\mathrm{e}^{2x}-4}{\mathrm{e}^x+2}\mathrm{\,d}x&=&\displaystyle\int\limits_0^1\dfrac{\left(\mathrm{e}^x-2\right)\left(\mathrm{e}^x+2\right)}{\mathrm{e}^x+2}\mathrm{\,d}x\\
&=&\displaystyle\int\limits_0^1\left(\mathrm{e}^x-2\right)\mathrm{\,d}x=\left(\mathrm{e}^x-2x\right)\big|_0^1=\mathrm{e}-3.
\end{eqnarray*}
\itemch Sai.  $\displaystyle\int\limits_0^1\dfrac{\mathrm{e}^x}{2^x}\mathrm{\,d}x=\displaystyle\int\limits_0^1\left(\dfrac{\mathrm{e}}{2}\right)^x\mathrm{\,d}x=\left[\left(\dfrac{\mathrm{e}}{2}\right)^x\right]\Big|_0^1=\dfrac{\mathrm{e}}{2}-1$.
\itemch Đúng. $\displaystyle\int\limits_1^2\mathrm{e}^x\left(1-\dfrac{\mathrm{e}^{-x}}{x}\right)\mathrm{\,d}x=\displaystyle\int\limits_1^2\left(\mathrm{e}^x-\dfrac{1}{x}\right)\mathrm{\,d}x=\left(\mathrm{e}^x-\ln \left| x\right|\right)\big|_1^2=\mathrm{e}^2-\mathrm{e}-\ln 2$.
\itemch Sai.\allowdisplaybreaks
\begin{eqnarray*} \displaystyle\int\limits_0^1\dfrac{\mathrm{e}^{2x-1}-\mathrm{e}^{-3x}+1}{\mathrm{e}^x}\mathrm{\,d}x&=&\displaystyle\int\limits_0^1\left(\mathrm{e}^{x-1}-\mathrm{e}^{-4x}+\mathrm{e}^{-x}\right)\mathrm{\,d}x\\
&=&\left(\mathrm{e}^{x-1}-\mathrm{e}^{-4x}+\mathrm{e}^{-x}\right)\big|_0^1=\dfrac{1-\mathrm{e}^4}{\mathrm{e}^4}=\mathrm{e}^{-4}-1.
\end{eqnarray*}
\end{itemchoice}
}
\end{ex}

\begin{ex}%[2H5H2-5]%[Dự án 2025 - Đề cấu trúc mới của Bộ theo [Thành Đức Trung]
Trong không gian $Oxyz$, cho hai điểm $A(-1;1;2)$, $B(2;-2;3)$.
\choiceTF
{\True Đường thẳng $AB$ có một véc-tơ chỉ phương là $\overrightarrow{u}=(3;-3;1)$}
{\True Phương trình tham số của đường thẳng $AB$ là $\heva{ & x=-1+3t \\ & y=1-3t \\ & z=2+t}$}
{\True Mặt phẳng $(P)$ đi qua $A$ và vuông góc với $AB$ là $(P)\colon 3x-3y+z+4=0$}
{Mặt phẳng $(Q)$ là mặt phẳng trung trực của đoạn thẳng $AB$ là $(Q) \colon 3x-3y+z-11=0$}
\loigiai
{
\begin{itemchoice}
\itemch \textbf{Đúng}. Vì đường thẳng $AB$ có một véc-tơ chỉ phương là $\overrightarrow{u}=\overrightarrow{AB}=(3;-3;1)$.
\itemch \textbf{Đúng}. Vì phương trình tham số của đường thẳng $AB$ là $\heva{ & x=-1+3t \\ & y=1-3t \\ & z=2+t.}$
\itemch \textbf{Đúng}. Vì mặt phẳng $(P)$ đi qua $A$ và vuông góc với $AB$ nên có một véc-tơ pháp tuyến là $\overrightarrow{n}=(3;-3;1)$. \\
Vậy $(P)\colon 3(x+1)-3(y-1)+1(z-2)=0 \Leftrightarrow 3x-3y+z+4=0$.
\itemch \textbf{Sai}. Vì gọi $M$ là trung điểm $AB$ là $M\left(\dfrac{1}{2};-\dfrac{1}{2};\dfrac{5}{2}\right)$.\\
Mặt phẳng $(Q)$ đi qua điểm $M\left(\dfrac{1}{2};-\dfrac{1}{2};\dfrac{5}{2}\right)$ và có một véc-tơ pháp tuyến $\overrightarrow{n}=(3;-3;1)$ là
\[(Q)\colon 3\left(x-\dfrac{1}{2}\right)-3\left(y+\dfrac{1}{2}\right)+1\left(z-\dfrac{5}{2}\right)=0 \Leftrightarrow 3x-3y+z-\dfrac{11}{2}=0.\]
\end{itemchoice}
}
\end{ex}
\Closesolutionfile{ans}

\TNSA
\Opensolutionfile{ans}[ans/ansDe3-TN3]
\begin{ex}%[2D4H1-1]%[Đào Trung Kiên]
Cho $F(x)$ là một nguyên hàm của hàm số $f(x) = ax + \dfrac{b}{x^2}$ $(x \neq 0)$. Biết $F(-1) = 1$, $F(1) = 4$, $f(1) = 0$. Tính giá trị của $M = 2a - b$ (làm tròn tới hàng phần mười).
\shortans[]{$4,5$}
\loigiai{
Ta có $\displaystyle\int f(x)\mathrm{\,d}x = \displaystyle\int \left(ax + \dfrac{b}{x^2}\right)\mathrm{\,d}x = \dfrac{ax^2}{2} - \dfrac{b}{x} + C$.\\
Theo giả thiết, ta có hệ phương trình $\heva{&F(-1) = 1 \\ &F(1) = 4 \\ &f(1) = 0} \Leftrightarrow \heva{&a + b + C = 1 \\ &a - b + C = 4 \\ &a + b = 0} \Rightarrow \heva{&a = \dfrac{3}{2} \\ &b = -\dfrac{3}{2}\cdot}$\\
Vậy $M = 2a - b = 3 + \dfrac{3}{2} = \dfrac{9}{2}=4{,}5$
}
\end{ex}

\begin{ex}%[2D4V3-4]%[Dự Án EX-TF-TLN Toán 12 - Đợt 2 - Quan Ón]
	Một vật thể nằm giữa hai mặt phẳng $x = 0$ và $x = \dfrac{\pi}{2}$; biết rằng mặt cắt của vật thể cắt bởi mặt phẳng vuông góc với trục $Ox$ tại điểm có hoành độ $x$ $\left(0 \leq x \leq \dfrac{\pi}{2}\right)$ là tam giác đều có cạnh là $2\sqrt{\cos x + \sin x}$. Thể tích của vật thể trên có dạng $a\sqrt{b}$. Hãy tính $5a + b^2$.
	\shortans{$19$}
	\loigiai{
	Diện tích của mặt cắt là $S(x) = \dfrac{\left(2\sqrt{\cos x + \sin x} \right)^2\cdot \sqrt{3}}{4} = \sqrt{3}\left(\cos x + \sin x\right)$.
	Vậy thể tích của vật thể đã cho là
	\[ V = \int_{0}^{\frac{\pi}{2}}S(x)\,\mathrm{d}x = \int_{0}^{\frac{\pi}{2}}\sqrt{3}\left(\cos x + \sin x\right)\,\mathrm{d}x = 2\sqrt{3}.\]
	Do đó $a = 2$ và $b = 3$.\\
	Vậy $5a + b^2 = 5\cdot 2 + 3^2 = 19$.
	}
	\end{ex}

\begin{ex}%[Nguyễn Tuấn, dự án sáng tác đề 12 theo chủ đề]%[2H5H2-7]
Cho mặt phẳng $(P)$ có véc-tơ pháp tuyến $\overrightarrow{n}=(1 ; 2 ; 2)$ và đường thẳng $\Delta$ có véc-tơ chỉ phương $\overrightarrow{u}=(2 ; 2 ;-1)$. Góc giữa đường thẳng $\Delta$ và mặt phẳng $(P)$ bằng bao nhiêu độ (làm tròn kết quả đến hàng đơn vị)?
\shortans{$26$}
\loigiai{
Ta có $\sin (\Delta,(P))=\dfrac{|1 \cdot 2+2 \cdot 2+2 \cdot(-1)|}{\sqrt{1^2+2^2+2^2} \cdot \sqrt{2^2+2^2+(-1)^2}}=\dfrac{4}{9}$. \\
Suy ra $(\Delta,(P)) \approx 26^{\circ}$.
}
\end{ex}

\begin{ex}%[2H5V1-7]
\immini
{
Để chuẩn bị cho chuyến đi dã ngoại, nhóm bạn Đức thiết kế lều cắm trại dạng hình chóp tứ giác đều có đáy là hình vuông cạnh $4$m. Theo bản vẽ thiết kế thì góc giữa hai mặt bên của lều bằng $60^{\circ}$. Bằng phương pháp tọa độ, hãy tính chiều cao của lều này.
}
{
\begin{tikzpicture}[>=stealth,line join=round,line cap=round,font=\footnotesize,scale=1]
\def \cao{3};
\def \x{3};
\coordinate (D) at (0,0);
\coordinate (C) at (1,.8);
\coordinate (A) at (\x,0);
\coordinate (B) at ($(A)-(D)+(C)$);
\coordinate (O) at ($(C)!.5!(A)$);
\coordinate (S) at ($(O)+(90:\cao)$);
\coordinate (X) at (intersection of S--O and  D--A);
\draw (D)--(A)--(B)--(S)--cycle (S)--(A);
\draw[dashed] (C)--(B) (S)--(C)--(D);
\draw[dashed,red] (A)--($(A)!1.2!(C)$) (D)--(B) (S)--(X);
\draw[->,red] (A)--($(O)!1.6!(A)$) node[right]{$x$};
\draw[->,red] (B)--($(O)!1.4!(B)$) node[above]{$y$};
\draw[->,red] (S)--($(O)!1.3!(S)$) node[left]{$z$};
\draw[red] (X)--++(-90:0.7) (D)--($(O)!1.3!(D)$);
\foreach \diem/\goc in {C/45,D/-90,A/-90,B/-30,S/45,O/-130} \fill[black](\diem) circle (1pt) ($(\diem)+(\goc:3mm)$) node{$\diem$};
\end{tikzpicture}
}
\shortans{$12$}
\loigiai{
Ta có $AC=BD=\sqrt{AD^2+AB^2}=\sqrt{4^2+4^2}=4\sqrt{2}$.\\
Chọn hệ trục tọa độ $Oxyz$ với gốc tọa độ $O$ là giao điểm của hai đường chéo $AC$ và $BD$ như hình vẽ.\\
Gọi $z$ là chiều cao của lều.\\
Ta có $O(0;0;0),A(2\sqrt{2};0;0),B(0;2\sqrt{2};0),C(-2\sqrt{2};0;0),D(0;-2\sqrt{2};0),S(0;0;z)$, với $z>0$.\\
Ta có $\overrightarrow{AD}=(-2\sqrt{2};-2\sqrt{2};0),\overrightarrow{AB}=(-2\sqrt{2};2\sqrt{2};0),\overrightarrow{AS}=(-2\sqrt{2};0;z)$.\\
Véc-tơ pháp tuyến của $(SAD)$ là $\overrightarrow{n_1}=[\overrightarrow{AD},\overrightarrow{AS}]=(-2z\sqrt{2};2z\sqrt{2};-8)$.\\
Véc-tơ pháp tuyến của $(SAB)$ là $\overrightarrow{n_2}=[\overrightarrow{AB},\overrightarrow{AS}]=(2z\sqrt{2};2z\sqrt{2};8)$.\\
Ta có
\begin{eqnarray*}
\cos((SAD),(SAB))&=&\dfrac{|\overrightarrow{n_1}\cdot \overrightarrow{n_2}|}{|\overrightarrow{n_1}|\cdot |\overrightarrow{n_2}|}\\
&=&\dfrac{|-2z\sqrt{2}\cdot 2z\sqrt{2}+2z\sqrt{2}\cdot 2z\sqrt{2}+8\cdot(-8)|}{\sqrt{(-2z\sqrt{2})^2+(2z\sqrt{2})^2+(-8)^2}\cdot \sqrt{(2z\sqrt{2})^2+(2z\sqrt{2})^2+8^2}}\\
&=&\dfrac{64}{16z^2+64}.
\end{eqnarray*}
Vì góc giữa hai mặt bên bằng $60^{\circ}$ nên góc giữa hai mặt phẳng $(SAD)$ và $(SAB)$ bằng $60^{\circ}$.\\
Do đó
\[\cos 60^{\circ}=\dfrac{64}{16z^2+64} \Leftrightarrow \dfrac{1}{2}=\dfrac{64}{16z^2+64}\Leftrightarrow 16z^2+64=128\Leftrightarrow \hoac{&z=2\\&z=-2}\]
Vì $z>0$ nên ta có $S(0;0;2).$\\
Vậy chiều cao của lều là $2$ m.
}

\end{ex}

\TL
\begin{ex}%[2H5H2-3]
Trong không gian $Oxyz$, gọi $M, N, P$ lần lượt là hình chiếu vuông góc của $A(2 ;-3 ; 1)$ lên các mặt phẳng tọa độ. Tính $a+b+c$ của phương trình mặt phẳng $(MNP)\colon ax+by+cz+d=0$.
% \shortans{$7$}
\loigiai{
Không mất tính tổng quát, ta giả sử $M, N, P$ lần lượt là hình chiếu vuông góc của $A(2 ;-3 ; 1)$ lên các mặt phẳng tọa độ $(Oxy),(Oxz),(Oyz)$. \\
Khi đó $M(2 ;-3 ; 0), N(2 ; 0 ; 1)$ và $P(0 ;-3 ; 1).$\\
$\overrightarrow{MN}=(0 ; 3 ; 1)$ và $\overrightarrow{MP}=(-2 ; 0 ; 1)$. \\
Ta có $\overrightarrow{MN}$ và $\overrightarrow{MP}$ là cặp véc-tơ không cùng phương và có giá nằm trong $(MNP)$.\\
Do đó $(MNP)$ có một véc-tơ pháp tuyến là $\overrightarrow{n}=\left[\overrightarrow{M N}, \overrightarrow{MP}\right]=(3 ;-2 ; 6)$.\\
Mặt khác $(MNP)$ đi qua $M(2 ;-3 ; 0)$ nên có phương trình là
\[3(x-2)-2(y+3)+6(z-0)=0 \Leftrightarrow 3x-2y+6z-12=0.\]
Suy ra $a+b+c=7$.
}
\end{ex}

\begin{ex}%[2D4C3-2]
\immini
{Ông An xây dựng một sân bóng đá mini hình chữ nhật có chiều rộng $30$m và chiều dài $50$m. Để giảm bớt chi phí cho việc trồng cỏ nhân tạo, ông An chia sân bóng ra làm hai phần (tô đen và không tô đen) như hình bên. Phần tô đen gồm hai phần diện tích bằng nhau và đường cong $AIB$ là một parabol đỉnh $I$ được trồng cỏ nhân tạo với giá $130\,000$ đồng/m$^2$ và phần còn lại được trồng với giá $90\,000$ đồng/m$^2$.
}
{\begin{tikzpicture}[scale=0.9, font=\footnotesize,line join=round, line cap=round, >=stealth]
\coordinate (I) at (0,0);
\coordinate (A) at (1.5,2.25);
\coordinate (B) at ($(A)+(0,-4.5)$);
\coordinate (C) at ($(B)+(-7.5,0)$);
\coordinate (D) at ($(A)-(B)+(C)$);
\coordinate (M) at ($(A)!1/2!(D)$);
\coordinate (N) at ($(B)!1/2!(C)$);
\coordinate (P) at ($(A)+(-1.5,0)$);
\coordinate (Q) at ($(B)+(-1.5,0)$);
\coordinate (R) at ($(D)+(1.5,0)$);
\coordinate (S) at ($(C)+(1.5,0)$);
\coordinate (td) at ($(D)+(0,0.3)$);
\coordinate (dt) at ($(D)+(-0.3,0)$);
\coordinate (ct) at ($(C)+(-0.3,0)$);
\coordinate (tr) at ($(R)+(0,0.3)$);
\coordinate (rp) at ($(R)+(0.3,0)$);
\coordinate (I') at ($(R)!1/2!(S)$);
\coordinate (g) at ($(I')+(0.3,0)$);
\fill[gray]plot[domain=0:1.5](\x,{sqrt(3.375*(\x))})--(A)--plot[domain=1.5:0](\x,{-sqrt(3.375*(\x))})--cycle;
\fill[gray](C)--(D)--plot[domain=-6:-4.5](\x,{sqrt(3.375*(-\x-4.5))})--plot[domain=-4.5:-6](\x,{-sqrt(3.375*(-\x-4.5))})--cycle;
\draw (A)--(B)--(C)--(D)--cycle (M)--(N);
\draw[dashed] (P)--(Q) (R)--(S);
\draw[<->](td)--(tr);
\node at ($(td)!1/2!(tr)$)[above]{$10$ m};
\draw[<->](dt)--(ct);
\node at ($(dt)!1/2!(ct)$)[above,rotate=90]{$30$ m};
\draw[<->](rp)--(g);
\node at ($(rp)!1/2!(g)$)[above,rotate=-90]{$15$ m};
\foreach \x/\g in {A/90,B/-90,I/180,I'/-40}\draw[fill=black] (\x) circle (.05) +(\g:.5)node{\footnotesize$\x$};
\end{tikzpicture}}
\noindent
Hỏi ông An phải trả bao nhiêu tiền (triệu đồng) để trồng cỏ nhân tạo cho sân bóng.
% \shortans{$151$}
\loigiai{
\immini{
Chọn hệ trục tọa độ như hình vẽ ($I$ là gốc tọa độ). Khi đó đường cong $IAB$ là một parabol có phương trình dạng $y=ax^2$.\\
Parabol đi qua điểm $\left(15;10 \right)$, suy ra
\[a \cdot 15^2=10 \Rightarrow a=\dfrac{2}{45}.\]
}
{
\begin{tikzpicture}[smooth,samples=300,scale=0.6,>=stealth]
\fill[gray!30] (-4,2)--(4,2)--plot[domain=4:-4](\x,{0.125*(\x)^2});
\draw[->] (-5,0)--(5,0) node[below]{$x$};
\draw[->] (0,-1)--(0,3) node[right]{$y$};
\draw (0,0) node[below left]{$I$};
\draw[domain=-4:4] plot(\x,{0.125*(\x)^2});
\draw[fill=black] (4,2) circle(1.5pt) (-4,2) circle(1.5pt);
\draw[dashed] (4,0)node[below]{$15$}--(4,2)node[right]{$B$}--(-4,2)node[left]{$A$}--(-4,0)node[below]{$-15$};

\node[right] at (0,2.4) {$10$};
\end{tikzpicture}}
\noindent
Vậy $y=\dfrac{2}{45}x^2$. Diện tích phần tô đen là $S=2 \cdot \displaystyle\int\limits_{-15}^{15} \left(10-\dfrac{2}{45}x^2 \right) \mathrm{\,d}x=400 \, (\text{m}^2)$.\\
Diện tích phần còn lại của sân bóng là $S_2=30 \cdot 50-400=1100\,(\text{m}^2).$\\
Số tiền Ông An phải trả để trồng cỏ nhân tạo cho sân bóng là
\[130000\times 400+90000\times 1100=151000000\text{ đồng}=151\text{ triệu đồng.}\]}
\end{ex}

\begin{ex}%[2H5C1-7]
Một phần sân nhà bác An có dạng hình thang $ABCD$ vuông tại $A$ và $B$ với độ dài $AB=9$ m, $AD=5$ m và $BC=6$ m như Hình 5.9. Theo thiết kế ban đầu thì mặt sân bằng phẳng và $A$, $B$, $C$, $D$ có độ cao như nhau. Sau đó bác An thay đổi thiết kế để nước có thể thoát về phía góc sân ở vị trí $C$ bằng cách giữ nguyên độ cao ở $A$, giảm độ cao của sân ở vị trí $B$ và $D$ xuống thấp hơn độ cao ở $A$ lần lượt là $6$ cm và $3{,}6$ cm. Để mặt sân sau khi lát gạch vẫn là bề mặt phẳng thì bác An cần phải giảm độ cao ở $C$ xuống bao nhiêu cm so với độ cao ở $A$?
\begin{center}
\includegraphics[scale=.4]{images/2P5-1-H5-9}
\hspace{0.5cm}
\begin{tikzpicture}[scale=0.5, font=\footnotesize,line join=round, line cap=round, >=stealth]
\path
(0,0) coordinate (A)
(9,0) coordinate(B)
(9,-6) coordinate(C)
(0,-5) coordinate(D)
;
\draw[thick] (A)--(B)--(C)--(D)--cycle;
\node [above] at ($(A)!0.5!(B)$) {$9$ m};
\node [right] at ($(B)!0.5!(C)$) {$6$ m};
\node [left] at ($(A)!0.5!(D)$) {$5$ m};
\foreach \i/\g in {A/90,B/90,C/-90,D/-90}{\draw[fill=black](\i) circle (0pt) ($(\i)+(\g:4mm)$) node[scale=1]{$\i$};}
\end{tikzpicture}
\end{center}
% \shortans{$10{,}32$}
\loigiai{
Tại vị trí ban đầu $A$, $B$, $C$, $D$ có độ cao như nhau, chọn hệ trục tọa độ có gốc tọa độ là điểm $A$ và các trục tọa độ lần lượt là $AD$, $AB$ và $Az$, với $Az \perp (ABCD)$.\\
Khi đó $A(0 ; 0 ; 0)$, $D(5; 0 ; 0)$, $B(0; 9 ; 0)$, $C(6; 9 ; 0)$.\\
Sau đó bác An thay đổi thiết kế để nước có thể thoát về phía góc sân ở vị trí $C$ bằng cách giữ nguyên độ cao ở $A$, giảm độ cao của sân ở vị trí $B$ và $D$ xuống thấp hơn độ cao ở $A$ lần lượt là $6$ cm và $3{,}6$ cm.\\
Khi đó, $A(0 ; 0 ; 0)$, $D(5; 0 ; -3{,}6)$, $B(0; 9 ; -6)$.\\
Ta có $\overrightarrow{AB}=(0 ; 9 ; -6)$, $ \overrightarrow{AD}=(5 ; 0 ; -3{,}6)$ là cặp véc-tơ chỉ phương của mặt phẳng $(ABD)$ nên một véc-tơ pháp tuyến của $(ABD)$ là $\left[\overrightarrow{AB}, \overrightarrow{AD}\right]=(-32{,}4 ; -30 ; -45)$.\\
Vậy mặt phẳng $(ABD)$ qua $A(0 ; 0 ; 0)$ và có véc-tơ pháp tuyến $\vec{n}=(-32{,}4 ; -30 ; -45)$ nên có phương trình là
\[-32{,}4 (x-2)-30(y+1)-45(z-3)=0 \qquad \text{hay } -32{,}4 x -30y -45z=0.\]
Để mặt sân sau khi lát gạch vẫn là bề mặt phẳng thì bác An cần phải giảm độ cao ở $C$ xuống $k$ centimét so với độ cao ở $A$ nên suy ra $C(6; 9 ; -k)$.\\
Ta có $A$, $B$, $C$, $D$ đồng phẳng	$\begin{aligned}[t]
&\Leftrightarrow C \in (ABD)\\
&\Leftrightarrow -32{,}4\cdot 6 -30 \cdot 9 -45\cdot (-k)=0\\
&\Leftrightarrow k=10{,}32.
\end{aligned}$\\
Vậy bác An cần phải giảm độ cao ở $C$ xuống $10{,}32$ centimét so với độ cao ở $A$.
}
\end{ex}
\Closesolutionfile{ans}


% \Closesolutionfile{ansbook}
% \HetDe
% \label{De3}
% %
% \cleardoublepage
% \setcounter{page}{1}
% \rfoot{Trang \thepage/\pageref{DA3} - Đáp án trắc nghiệm Mã đề 3}
% \begin{center}
% 	\bfseries ĐÁP ÁN TRẮC NGHIỆM MÃ ĐỀ 3
% \end{center}

% \inputansbox{10}{ans/ansDe3-TN1}
% \inputansbox[3]{2}{ans/ansDe3-TN2}
% \inputansbox{3}{ans/ansDe3-TN3}
% \label{DA3}
% %

% \begin{name}
	{\tenchude}
	{TOÁN 12}
	{LỚP TOÁN THẦY PHÁT}
	{Thời gian: 90 phút - Không kể thời gian phát đề}
\end{name}
\Opensolutionfile{ans}[ans/ansDe4-TN1]
\begin{ex}%[2D4N1-1]%[2015_K12 Huyên Nguyễn]
	Cho $F(x)$ là nguyên hàm của hàm số $f(x)=\sin \left(\dfrac{\pi}{2}-x\right)$. Khi đó $F'(x)$ bằng
	\choice
	{$\cos \left(\dfrac{\pi}{2}-x\right)$}
	{$\sin x$}
	{$-\cos \left(\dfrac{\pi}{2}-x\right)$}
	{\True $\cos x$}
	\loigiai{
		Vì $F(x)$ là một nguyên hàm của hàm số $f(x)$ nên $F'(x)=f(x)=\sin \left(\dfrac{\pi}{2}-x\right)=\cos x$.
	}
\end{ex}

\begin{ex}%[2D4N1-2]%[2015_K12 Huyên Nguyễn]
	Cho $F(x)$ là một nguyên hàm của hàm số $f(x)=\dfrac{1}{x}$ và thỏa mãn $F\left(\mathrm{e}^2\right)=3$. Khi đó $F(x)$ bằng
	\choice
	{$\ln |x|+5$}
	{\True $\ln |x|+1$}
	{$\ln |x|+3$}
	{$\ln |x|-1$}
	\loigiai{
		Ta có $F(x)=\displaystyle\int\dfrac{1}{x}\mathrm{\,d}x=\ln |x|+C$.\\
		Mà $F\left(\mathrm{e}^2\right)=3\Leftrightarrow \ln |\mathrm{e}^2|+C=3\Leftrightarrow C=1$.\\
		Vậy $F(x)=\ln |x|+1$.
	}
\end{ex}

\begin{ex}%[2D4H1-3]
	Cho hàm số $f(x)=3\cos x-\dfrac{2}{x}+\dfrac{4}{{{\sin }^2}x}$. Khẳng định nào dưới đây là đúng?
	\choice
	{\True $\displaystyle\int{f(x)\mathrm{d}x}=3\sin x-2\ln |x|-4\cot x+C$}
	{ $\displaystyle\int{f(x)\mathrm{d}x}=3\sin x-2\ln x-4\cot x+C$}
	{ $\displaystyle\int{f(x)\mathrm{d}x}=3\sin x-2\ln |x|+4\cot x+C$}
	{ $\displaystyle\int{f(x)\mathrm{d}x}=-3\sin x-2\ln |x|-4\cot x+C$}
	\loigiai{
		Ta có $\displaystyle\int{f(x)\mathrm{d}x}=\displaystyle\int{(3\cos x-\dfrac{2}{x}+\dfrac{4}{{\sin^2}x})\mathrm{d}x}$ \\
		$=3\displaystyle\int{\cos x\mathrm{d}x}-2\displaystyle\int{\dfrac{1}{x}\mathrm{d}x}+4\displaystyle\int{\dfrac{1}{{\sin^2}x}\mathrm{d}x}$ \\
		$=3\sin x-2\ln |x|-4\cot x+C$.
	}
\end{ex}

\begin{ex}%[2D4H2-1]
	Biết $F(x)=x^2$ là một nguyên hàm của hàm số $f(x)$ trên $\mathbb{R}$. Giá trị của $\displaystyle  \int\limits_1^2\left[2+f(x)\right]\mathrm{\,d}x$ bằng
	\choice
	{$3$}
	{\True $5$}
	{$\dfrac{13}{3}$}
	{$\dfrac{7}{3}$}
	\loigiai{
		Ta có $\displaystyle  \int\limits_1^2\left[2+f(x)\right]\mathrm{\,d}x$ $=\left.(2x+x^2)\right|_1^2=8-3=5$.
	}
\end{ex}

\begin{ex}%[2D4N3-1]
	Gọi $S$ là diện tích hình phẳng giới hạn bởi các đường $y=3^x$, $y=0$, $x=0$ và $x=2$. Mệnh đề nào dưới đây là đúng?
	\choice
	{$S=\pi \displaystyle\int\limits_{0}^{2}{3^x}\mathrm{\,d}x$}
	{$S=\displaystyle\int\limits_{0}^{2}{3^{2x}}\mathrm{\,d}x$}
	{\True $S=\displaystyle\int\limits_{0}^{2}{3^x}\mathrm{\,d}x$}
	{$S=\pi \displaystyle\int\limits_{0}^{2}{3^{2x}}\mathrm{\,d}x$}
	\loigiai
	{Diện tích hình phẳng giới hạn bởi các đường $y=3^x$, $y=0$, $x=0$ và $x=2$ là $S=\displaystyle\int\limits_{0}^{2}{\left|3^x\right |}\mathrm{\,d}x = \displaystyle\int\limits_{0}^{2}{3^x}\mathrm{\,d}x$.}
\end{ex}

\begin{ex}%[Dự án 2025 - đề cấu trúc mới, Hung Doan]%[2D4H3-3]
	Tính thể tích khối tròn xoay được tạo bởi hình phẳng giới hạn bởi đồ thị hàm số $y=3x-x^2$ và trục hoành khi quay quanh trục hoành.
	\choice
	{$\dfrac{85\pi}{7}$}
	{$\dfrac{8\pi}{7}$}
	{\True $\dfrac{81\pi}{10}$}
	{$\dfrac{41\pi}{7}$}
	\loigiai{
		Phương trình hoành độ giao điểm của đồ thị hàm số $y=3x-x^2$ và trục hoành là \[3x-x^2=0\Leftrightarrow \hoac{&x=0\\&x=3.}\]
		Thể tích của khối tròn xoay là $V=\pi \displaystyle\int \limits_0^3 (3x-x^2)^2\mathrm{\,d}x=\dfrac{81\pi}{10}$.}
\end{ex}

\begin{ex}%[2H5N1-1]%[Dự án Khối 12- Ex-TF-TLN-2024]%[VU Ngoc Hao]
	\immini{
		Cho hình hộp chữ nhật $ABCD.A'B'C'D'$. Bốn véc-tơ pháp tuyến  của mặt phẳng $\left(AA'B'B\right)$ là
		\choice
		{$\overrightarrow{AD}$, $\overrightarrow{A'D'}$, $ \overrightarrow{BD}$, $\overrightarrow{B'C'}$}
		{$\overrightarrow{AD}$, $\overrightarrow{A'D'}$, $ \overrightarrow{BC}$, $\overrightarrow{BC'}$}
		{$\overrightarrow{AC}$, $\overrightarrow{A'D'}$, $ \overrightarrow{BC}$, $\overrightarrow{B'C'}$}
		{\True  $\overrightarrow{AD}$, $\overrightarrow{A'D'}$, $ \overrightarrow{BC}$, $\overrightarrow{B'C'}$}
	}
	{
		\begin{tikzpicture}[scale=0.5, font=\footnotesize,line join=round, line cap=round, >=stealth]
			\coordinate (A) at (0,0);
			\coordinate (B) at (-2,-1.5);
			\coordinate (D) at (5,0);
			\coordinate (C) at ($(B)+(D)-(A)$);
			\foreach \i in {A,B,C,D}{\coordinate (\i') at ($(\i)+(0,4)$);}
			\draw (A')--(B')--(C')--(D')--cycle;
			\draw (B)--(B') (C)--(C') (D)--(D')  (B)--(C)--(D);
			\draw[dashed,thin](B)--(A)--(A') (A)--(D);
			\foreach \i/\g in {A'/90,B'/90,C'/90,D'/90,A/-90,B/-90,C/-90,D/-90}{\draw[fill=black](\i) circle (1pt) ($(\i)+(\g:5mm)$) node[scale=1]{$\i$};}
		\end{tikzpicture}
	}
	\loigiai{
		Bốn véc-tơ pháp tuyến của mặt phẳng $\left(AA'B'B\right)$ là  $\overrightarrow{AD}$, $\overrightarrow{A'D'}$, $ \overrightarrow{BC}$, $\overrightarrow{B'C'}$.
	}
\end{ex}

\begin{ex}%[2H5N1-2]
	Trong không gian với hệ tọa độ $Oxyz$, cho mặt phẳng $(P)\colon x-y+3=0$. Véc-tơ nào sau đây \textbf{không phải} là véc-tơ pháp tuyến của mặt phẳng $(P)$?
	\choice
	{$\overrightarrow{a}=(3;-3;0)$}
	{\True $\overrightarrow{a}=(1;-1;3)$}
	{$\overrightarrow{a}=(1;-1;0)$}
	{$\overrightarrow{a}=(-1;1;0)$}
	\loigiai{
		Véc-tơ pháp tuyến của mặt phẳng $(P)$ là $\overrightarrow{n}=(1;-1;0)$.\\
		Ta có $\overrightarrow{a}=(-1;1;0)=-(1;-1;0)=-\overrightarrow{n}$. Vậy $\overrightarrow{a}=(-1;1;0)$ là một véc-tơ pháp tuyến của mặt phẳng $(P)$.\\
		Tương tự $\overrightarrow{a}=(3;-3;0)=3(1;-1;0)=3\overrightarrow{n}$. Vậy $\overrightarrow{a}=(3;-3;0)$ là một véc-tơ pháp tuyến của mặt phẳng $(P)$.\\
		Do véc-tơ $\overrightarrow{a}=(1;-1;3)$ không cùng phương với véc-tơ $\overrightarrow{n}=(1;-1;0)$. Nên $\overrightarrow{a}=(1;-1;3)$ không là véc-tơ pháp tuyến của mặt phẳng $(P)$.
	}
\end{ex}

\begin{ex}%[2H5H1-3]
	Trong không gian $O x y z$, phương trình của mặt phẳng $(P)$ đi qua điểm $B(2 ; 1 ;-3)$, đồng thời vuông góc với hai mặt phẳng $(Q)\colon x+y+3 z=0,$ $(R)\colon 2 x-y+z=0$ là
	\choice
	{$4 x+5 y-3 z+22=0$}
	{$4 x-5 y-3 z-12=0$}
	{$2 x+y-3 z-14=0$}
	{\True $4 x+5 y-3 z-22=0$}
	\loigiai{
		Mặt phẳng $(Q)\colon x+y+3 z=0,$ $(R)\colon 2 x-y+z=0$ có các véc-tơ pháp tuyến lần lượt là $\overrightarrow{n}_1=(1 ; 1 ; 3)$ và $\overrightarrow{n}_2=(2 ;-1 ; 1)$.\\
		Vì $(P)$ vuông góc với hai mặt phẳng $(Q),$ $(R)$ nên $(P)$ có véc-tơ pháp tuyến là $\vec{n}=\left[\overrightarrow{n}_1, \overrightarrow{n}_2\right]=(4 ; 5 ;-3)$.\\
		Ta lại có $(P)$ đi qua điểm $B(2 ; 1 ;-3)$ nên \[(P)\colon 4(x-2)+5(y-1)-3(z+3)=0\Leftrightarrow 4 x+5 y-3 z-22=0.\]
	}
\end{ex}

\begin{ex}%[Dự án EX-Ôn Tập TN 2025, Đoàn Hùng]%[2H5N2-1]
	Trong không gian toạ độ, phương trình nào sau đây là phương trình tham số của đường thẳng?
	\choice
	{$\heva{&x=2+t^2\\&y=3-t\\&z=4+t}$}
	{$\heva{&x=2+y\\&y=3-t^2\\&z=-4+2t}$}
	{$\heva{&x=2+t\\&y=3-t\\&z =t^2}$}
	{\True $\heva{&x=2+3t\\&y=4+5t\\&z=5+6t}$}
	\loigiai{
		Phương trình tham số của đường thẳng có dạng $\heva{&x=x_0+at\\&y=y_0+bt\\&z=z_0+ct}$ với $a^2+b^2+c^2\neq0$. Do đó đáp án cần chọn là
		\[\heva{&x=2+3t\\&y=4+5t\\&z=5+6t}.\]
	}
\end{ex}

\begin{ex}%[2H5N2-7]
	Trong không gian với hệ tọa độ $Oxyz$, tính góc giữa hai đường thẳng $d_1\colon\dfrac{x}{1}=\dfrac{y+1}{-1}=\dfrac{z-1}{2}$ và $d_2\colon\dfrac{x+1}{-1}=\dfrac{y}{1}=\dfrac{z-3}{1}$.
	\choice
	{$45^\circ $}
	{$30^\circ $}
	{$60^\circ $}
	{\True $90^\circ $}
	\loigiai{
		Ta có $\overrightarrow{u}_{d_1}=\left( 1;-1;2 \right)$ và
		$\overrightarrow{u}_{d_2}=\left( -1;1;1 \right)$ lần lượt là véc tơ chỉ phương của $d_1$ và $d_2$.\\
		$\overrightarrow{u}_{d_1}\cdot\overrightarrow{u}_{d_2}=1\cdot\left( -1 \right)+\left( -1 \right)\cdot1+2\cdot1=0\Rightarrow {d_1}\bot {d_2}\Rightarrow \left( \widehat{d_1,d_2} \right)=90^\circ $.}
\end{ex}

\begin{ex}%[2H5H2-4]
	Trong không gian $Oxyz,$ cho hai đường thẳng $d_1\colon \dfrac{x-1}{2}=\dfrac{y}{1}=\dfrac{z+2}{-2}$, $d_2\colon \dfrac{x+2}{-2}=\dfrac{y-1}{-1}=\dfrac{z}{2}$. Xét vị trí tương đối của hai đường thẳng đã cho.
	\choice
	{Chéo nhau}
	{Trùng nhau}
	{\True Song song}
	{Cắt nhau}
	\loigiai{
		Ta có $\heva{& \overrightarrow{u}_{d_1}=(2;1;-2) \\ & \overrightarrow{u}_{d_2}=(-2;-1;2 )} \Rightarrow \overrightarrow{u}_1=-\overrightarrow{u}_2 $. Do đó $d_1$ song song hoặc trùng với $d_2$.\\
		Gọi điểm $M( 1;0;-2 )\in d_1 $ thay $M$ vào $d_2$ ta được $\dfrac{1+2}{-2}=\dfrac{0-1}{-1}=\dfrac{-2}{2}$ (vô lí).\\
		Vậy $d_1\parallel d_2$.
	}
\end{ex}
\Closesolutionfile{ans}

\TNTF
\Opensolutionfile{ans}[ans/ansDe4-TN2]
\begin{ex}%[DA-EX-TF-TLN2024 - Trần Xuân Hòa]%[2D4H2-4]
	Cho hàm số $f(x)=\heva{&\mathrm{e}^{2x} \text{ khi }x\ge 0\\&x^2+x+2\text{ khi }x<0}$.
	\choiceTF
	{Giá trị $\displaystyle\int\limits_0^1f(x)\mathrm{\,d}x=4$}
	{\True Giá trị $\displaystyle\int\limits_{-1}^0f(x)\mathrm{\,d}x=\dfrac{11}{6}$}
	{Giá trị của $m, m>0$ để $\displaystyle\int\limits_m^{1}f(x)\mathrm{\,d}x=4$ bằng $1$}
	{\True Biết $\displaystyle\int\limits_{-1}^1f(x)\mathrm{\,d}x=\dfrac{a}{b}+\dfrac{\mathrm{e}^2}{c}$ với $\dfrac{a}{b}$ tối giản. Giá trị $a+b+c$ bằng $9$}
	\loigiai{
		\begin{itemchoice}
			\itemch Sai. Vì $\displaystyle\int\limits_0^1f(x)\mathrm{\,d}x=\displaystyle\int\limits_0^1\mathrm{e}^{2x}\mathrm{\,d}x=\dfrac{1}{2}\mathrm{e}^{2x}\bigg|_0^1=\dfrac{1}{2}(\mathrm{e}^2-1)$.
			\itemch Đúng. Vì $\displaystyle\int\limits_{-1}^0f(x)\mathrm{\,d}x=\displaystyle\int\limits_{-1}^0(x^2+x+2)\mathrm{\,d}x=\left(\dfrac{x^3}{3}+\dfrac{x^2}{2}+2x\right)\Bigg|_0^1=\dfrac{11}{6}$
			\itemch Sai. Vì $\displaystyle\int\limits_m^{1}f(x)\mathrm{\,d}x=4\Leftrightarrow \left(\dfrac{1}{2}\mathrm{e}^{2x}\right)\Bigg|_m^1=4\Leftrightarrow \dfrac{1}{2}(\mathrm{e}^2-\mathrm{e}^{2m})=4\Leftrightarrow \mathrm{e}^{2m}=\mathrm{e}^2-8<0$ (vô nghiệm).\\
			Vậy không có $m$ thỏa mãn.
			\itemch Đúng. \begin{eqnarray*}
				\displaystyle\int\limits_{-1}^1f(x)\mathrm{\,d}x&=&\displaystyle\int\limits_{-1}^0f(x)\mathrm{\,d}x+\displaystyle\int\limits_{0}^1f(x)\mathrm{\,d}x\\
				&=&\displaystyle\int\limits_{-1}^0(x^2+x+2)\mathrm{\,d}x+\displaystyle\int\limits_{0}^1\mathrm{e}^{2x}\mathrm{\,d}x\\
				&=&\dfrac{11}{6}+\dfrac{1}{2}(\mathrm{e}^2-1)\\
				&=&\dfrac{\mathrm{e}^2}{2}+\dfrac{4}{3}.
			\end{eqnarray*}
			Do đó $a=4$, $b=3$, $c=2$. Giá trị $a+b+c=9$.
		\end{itemchoice}
	}
\end{ex}

\begin{ex}%[2H5H2-5]
	Trong không gian $O x y z$ cho đường thẳng $d$ có phương trình tham số $\heva{&x=-1+2 t \\& y=1+t \\& z=3-2 t.}$
	\choiceTF
	{\True Phương trình chính tắc của đường thẳng $d$ là $\dfrac{x+1}{2}=\dfrac{y-1}{1}=\dfrac{z-3}{-2}$}
	{\True  Đường thẳng $d$ đi qua điểm $A(-1 ; 1 ; 3)$}
	{Véc-tơ $\overrightarrow{a}=(4 ; 2 ;-3)$ là một véc-tơ chỉ phương   của đường thẳng $d$}
	{Giao điểm của đt $d$ và mặt phẳng $(P)\colon x+2 y-3 z+2=0$  là $I(0 ; 1 ; 2)$}
	\loigiai{
		\begin{itemchoice}
			\itemch \textbf{Đúng.} Vì  đường thẳng $d$ đi qua điểm $M(-1 ; 1 ; 3)$ và có một  véc-tơ chỉ phương   là $\overrightarrow{u}_d=(2 ; 1 ;-2)$ nên có phương trình chính tắc là $\dfrac{x+1}{2}=\dfrac{y-1}{1}=\dfrac{z-3}{-2}$.
			\itemch \textbf{Đúng.} Vì theo phương trình tham số của đường thẳng $d$ thì $d$ đi qua điểm $A(-1 ; 1 ; 3)$.
			\itemch \textbf{Sai.} Vì véc-tơ chỉ phương của đường thẳng $d$ là $\overrightarrow{u}_d=(2 ; 1 ;-2)$. \\Xét hai véc-tơ $\overrightarrow{a}=(4 ; 2 ;-3)$ và $\overrightarrow{u}_d=(2 ; 1 ;-2)$.\\
			Vì $\dfrac{1}{2} \neq \dfrac{-2}{-3}$ nên $\overrightarrow{a}=(4 ; 2 ;-3)$ và $\overrightarrow{u}_d$ không cùng phương. \\Do đó $\overrightarrow{a}$ không là véc-tơ chỉ phương của đường thẳng $d$.
			\itemch \textbf{Sai.} Vì  ta có $-1+2 t+2(1+t)-3(3-2 t)+2=0$ $\Leftrightarrow 10 t-6=0 \Leftrightarrow t=\dfrac{3}{5}$.\\
			Giao điểm của $d$ và $(P)$ là $B\left(\dfrac{1}{5} ; \dfrac{8}{5} ; \dfrac{9}{5}\right)$.\end{itemchoice}
	}
\end{ex}
\Closesolutionfile{ans}

\TNSA
\Opensolutionfile{ans}[ans/ansDe4-TN3]
\begin{ex}%[2D4H1-1]%[Đào Trung Kiên]
	Giả sử $F(x)$ là một nguyên hàm của hàm số $f(x)=\mathrm{e}^x$, biết $F(0)=4$. Tìm $F(1)$ (làm tròn kết quả tới phần mười).
	\shortans[]{$5,7$}
	\loigiai{
		Do $F(x)$ là một nguyên hàm của $f(x)=\mathrm{e}^x$ nên $F(x)=\mathrm{e}^x+C$.\\
		Lại có $F(0)=4$ nên $C=3$ hay $F(x)=\mathrm{e}^x+3$ nên $F(1)=\mathrm{e}+3\approx 5{,}7$.
	}
\end{ex}

\begin{ex} %[2D4V3-3]
	Cho tam giác vuông $OAB$ có cạnh $OA$ nằm trên trục $Ox$ và $\widehat{AOB}=\alpha \, (0<\alpha <\frac{\pi }{2})$ và $B(a;b)$ với $a$, $b$ là các số thực thỏa ${a^2}+{b^2}=1$. Gọi $\beta $ là khối tròn xoay sinh ra khi quay miền tam giác $OAB$ xung quanh trục $Ox$.
	\begin{center}
	\begin{tikzpicture}[scale=1,line join=round, line cap=round,>=stealth,declare function={rt=.03;xmin=-2.5;xmax=8;ymin=-2.5;ymax=3;a=2.5;b=a/3;}]
	\path (0,0) coordinate (O)
	(6,0)coordinate (A)++(90:a) coordinate (B)
	;
	\draw[->] (0,ymin)--(0,ymax) node[right] {$y$};
	\draw[->] (0,0)--(-150:2) node[above left] {$z$};
	\draw[fill=black] (0,0) circle (rt);
	\fill[red!15] (O)--(B)--(A)--cycle;
	\draw[->]  (A)--(xmax,0) node[below] {$x$};
	\draw (A) ellipse ({b} and {a})
	(-1,0)--(0,0) node[above left]{$O$} (O)--(B)--(A) (O)--($(A)+(-90:a)$)
	pic["$\alpha$",angle radius=15mm] {angle = A--O--B}
	pic[draw,angle radius=7mm] {angle = A--O--B}
	;
	\draw[dashed](O)--(A);
	\foreach \p/\g in {A/-90,B/90}
	\fill (\p) circle(1pt)+(\g:.3) node{$\p$};
	\end{tikzpicture}
	\end{center}
	Tính giá trị $\tan \alpha $ khi thể tích của khối $\beta $ đạt giá trị lớn nhất (Làm tròn kết quả đến chữ số thập phân thứ 2).
	\shortans[1]{$1{,}41$}
	\loigiai{
	Do $OB$ đi qua gốc tọa độ nên ta đặt $OB\colon y=kx$ với $ k$ là số thực dương.\\
	Do $OB$ đi qua $B(a;b) \Rightarrow OB\colon y=\frac{b}{a}x$ và $\tan \alpha =\frac{b}{a}$.\\
	Khi đó, $V=\pi \int\limits_0^a{{\left(  \dfrac{b}{a}x\right) ^2}\mathrm{\,d}x}=
	\pi \int\limits_0^a{\dfrac{{b^2}}{{a^2}}{x^2}\mathrm{\,d}x}=
	\dfrac{\pi {b^2}{x^3}}{3{a^2}} \bigg|_0^a=
	\dfrac{\pi }{3}{b^2}a$.\\
	Áp dụng bất đẳng thức Am – Gm:\\
	\[{V^2}=\dfrac{{{\pi }^2}}{9}{b^4}{a^2}=\dfrac{{{\pi }^2}}{18}{b^2}{b^2}2{a^2}\le \dfrac{{{\pi }^2}}{18}\dfrac{{{( {b^2}+{b^2}+2{a^2} )}^3}}{27}=\dfrac{4{{\pi }^2}}{243}\\
	\Rightarrow V\le \frac{2\pi \sqrt {3}}{27}.\]
	Đẳng thức xảy ra khi và chỉ khi ${b^2}=2{a^2} \Rightarrow \dfrac{b}{a}=\tan \alpha =\sqrt {2} \approx 1{,}41$.}
	\end{ex}

\begin{ex}%[2H5H2-7]
	Trong không gian với hệ tọa độ $Oxyz$, cho điểm $A(3;-1;0)$ và đường thẳng $d\colon \dfrac{x-2}{-1}=\dfrac{y+1}{2}=\dfrac{z-1}{1}$. Phương trình mặt phẳng $(\alpha)$ chứa $d$ sao cho khoảng cách từ $A$ đến $(\alpha)$ lớn nhất có dạng $ax+by+cz=0$. Khi đó $\dfrac{a}{b}$ bằng
	\shortans{$1$}
	\loigiai{
		Gọi $H$ là hình chiếu của $A$ lên $d$.\\
		Khi đó $H(2-t;-1+2t;1+t)\Rightarrow \overrightarrow{AH}=(-1-t;2t;1+t)$.\\
		Do $AH\perp d$ nên $ -(-1-t)+2\cdot 2t+1+t=0\Leftrightarrow t=-\dfrac{1}{3}$. Khi đó $\overrightarrow{AH}=\left(-\dfrac{2}{3};-\dfrac{2}{3};\dfrac{2}{3}\right)$.\\
		Mặt phẳng $(\alpha)$ chứa $d$ sao cho khoảng cách từ $A$ đến $(\alpha)$ lớn nhất khi $AH\perp (\alpha)$.\\
		Do đó $(\alpha)$ có véc-tơ pháp tuyến là $\overrightarrow{n}=(1;1;-1)$.\\
		Vậy $(\alpha)\colon 1(x-2)+1(y+1)-1(z-1)=0\Leftrightarrow x+y-z=0$.\\ Do đó $a=1$, $b=1$, $c=-1$ và $\dfrac{a}{b}=1$.}
\end{ex}

\begin{ex}%[2H5V1-7]
	\immini{
		Một ngôi nhà có nền nhà là hình vuông, cạnh là $7$ mét. Các vách tường hình vuông và vị trí cao nhất trên mái nhà cách sàn nhà $8$ mét. Biết rằng hai mái nhà là hai hình chữ nhật bằng nhau. Khi gắn hệ trục tọa độ $Oxyz$ (đơn vị trên mỗi trục tính theo mét) vào một căn nhà sao cho nền nhà thuộc mặt phẳng $\left(Oxy\right)$, người ta coi mỗi mái nhà là một phần của mặt phẳng. Góc giữa mái nhà bên phải và nền nhà bằng bao nhiêu độ (làm tròn kết quả đến hàng đơn vị)?}{\begin{tikzpicture}[scale=0.6, font=\footnotesize, line join=round, line cap=round, >=stealth]
			\def\bc{6} % cạnh BC
			\def\ba{3} % cạnh BA
			\def\gocB{35} % góc B của đáy
			\coordinate[label=below left:$O(0;0;0)$] (O) at (0,0);
			\coordinate[label=above left:$C$] (C) at (\gocB:\ba);
			\coordinate[label=below:$A$] (A) at (\bc,0);
			\coordinate[label=right:$B(7;7;0)$] (B) at ($(A)-(O)+(C)$);
			\coordinate[label=above left:$H$] (H) at ($(C)+(90:\bc)$);
			\coordinate[label=left:$E(0;0;7)$] (E) at ($(O)-(C)+(H)$);
			\coordinate[label=below right:$F$] (F) at ($(A)-(C)+(H)$);
			\coordinate[label=right:$G$] (G) at ($(B)-(C)+(H)$);
			\coordinate (P) at (3,7);
			\coordinate[label=above:$Q\left(\dfrac{7}{2}; 7; 8 \right)$] (Q) at ($(P)-(E)+(H)$);
			\draw (E)--(O)--(A)--(B)--(G) (H)--(E)--(F)--(G) (A)--(F) (E)--(P)--(F) (H)--(Q)--(G) (P)--(Q);
			\draw[dashed] (H)--(C)--(B) (C)--(O)  (Q)--(G)--(H);
			\foreach \diem in {C,O,A,B,H,E,F,G,P, Q}	\fill (\diem)circle(1.5pt);
			\draw[black,->,>=stealth] (O)--($(O)!1.25!(C)$);
			\draw[black,->,>=stealth] (O)--($(O)!1.25!(E)$);
			\draw[black,->,>=stealth] (O)--($(O)!1.25!(A)$);
			\path (O)--($(O)!1.15!(C)$)node[pos=1.05,sloped,black,above]{$y$};
			\path (O)--($(O)!1.15!(E)$)node[pos=1.05,black,right]{$z$};
			\path (O)--($(O)!1.15!(A)$)node[pos=1.05,sloped,black,above]{$x$};
		\end{tikzpicture}}
	\shortans{$16$}
	\loigiai{
		Gắn hệ trục tọa độ $Oxyz$ như hình vẽ. Theo giả thiết, ta có $F(7;0;7)$, $G(7;7;7)$ và $Q\left(\dfrac{7}{2}; 7;8\right)$
		Ta gọi $\alpha$ là góc giữa mái nhà bên phải và nền nhà.\\ Khi đó $\alpha = \left(\left(QGF\right),(Oxy)\right)$.\\
		Với $\left(QGF\right)$ nhận véc-tơ $\overrightarrow{n}=\left[\overrightarrow{FG},\overrightarrow{GQ}\right]$ với $\overrightarrow{FG}=\left(0;7;0\right)$ và $\overrightarrow{GQ}=\left(\dfrac{7}{2};0;1\right)$. Khi đó $\overrightarrow{n}=\left(7;0;-\dfrac{49}{2}\right)= \dfrac{7}{2} \left( 2;0;-7\right)$.\\
		Với $\left(Oxy\right)$  nhận $\overrightarrow{k}=\left(0;0;1\right)$ làm véc-tơ pháp tuyến.\\
		Vậy $\cos \alpha =\dfrac{7}{\sqrt{53}} $.\\ Vậy $\alpha \approx 16^{\circ}$.
	}
\end{ex}

\TL
\begin{ex}%[2H5H2-3]%[Dự án EX-TF-TLN lần 3 - Nguyen Chín Em]
	Trong không gian với hệ tọa độ $Oxyz$. Viết phương trình chính của $d$ biết đường thẳng $d$ đi qua $A(-1; 1; -3)$ và $B(-2; 2; 0)$. Đường thẳng $AB$ đi qua điểm $(x_0;0;z_0)$. Tính $x_0+z_0$.
	% \shortans{$-6$}
	\loigiai{
		Đường thẳng $d$ đi qua $A, B$ nên có véc-tơ chỉ phương $\overrightarrow{AB} = (-1; 1; 3)$.\\
		Phương trình chình tắc của $d\colon \dfrac{x+1}{-1}=\dfrac{y-1}{1}=\dfrac{z+3}{3}$.\\
		Cho $y=0$, suy ra được $x_0=0$, $z_0=-6$. Khi đó $x_0+z_0=-6$.
	}
\end{ex}

\begin{ex}%[EX-Ôn Tập TN 2025, Võ Thanh Phong]%[2D4C3-2]
	Một cổng có đạng hình parabol với chiều cao $8$ m, chiều rộng chân đế $8$ m. Người ta căng hai sợi dây trang trí $AB$,  $CD$ nằm ngang, đồng thời chia cổng thành ba phần sao cho hai phần ở phía trên có diện tích bằng nhau. Tỉ số $\dfrac{CD}{AB}$ bằng bao nhiêu (làm tròn kết quả đến hàng phần trăm)?
	\begin{center}
		\begin{tikzpicture}[scale=0.7, font=\footnotesize, line join=round, line cap=round,>=stealth]
			\begin{scope}
				\clip (-4,-4.5) rectangle (4,0);
				\draw [smooth,domain=-5:4, samples=200] plot (\x, {-0.5*(\x)^2});
			\end{scope}
			\draw [<->] (3.3,0)--(3.3,-4.5);
			\node[right] at (3.3,-2){$8\,\,m$};
			\draw [<->] (-3,-4.5)--(3,-4.5);
			\node[below] at (0,-4.5){$8\,\,m$};
			\node[left] at (-1.8,-1.62){$A$};
			\node[right] at (1.8,-1.62){$B$};
			\node[left] at (-2.5,-3.125){$C$};
			\node[right] at (2.5,0-3.125){$D$};
			\draw [-] (-1.8,-1.62)--(1.8,-1.62);
			\draw [-] (-2.5,-3.125)--(2.5,0-3.125);
			\draw [dashed] (0,0)--(4,0);
			\node[below] at (current bounding box.south){\textit{Hình 5}};
		\end{tikzpicture}
	\end{center}
	% \shortans{ $1{,}26$                 }
	\loigiai{
	\immini{Gắn hệ trục tọa độ $Oxy$ vào cổng parabol như hình bên với trục $Oy$ trùng với đường đối xứng của parabol, gốc $O$ nằm ở đỉnh của parabol, đơn vị trên mỗi trục tính theo mét. Khi đó, phương trình parabol có dạng $y=ax^2$.\\
		Vì parabol đi qua điểm có toạ độ $(-4 ;-8)$ nên $a=-\dfrac{1}{2}$. Suy ra phương trình parabol là $y=-\dfrac{1}{2} x^2$.\\}{
		\begin{tikzpicture}[scale=0.7, font=\footnotesize, line join=round, line cap=round,>=stealth]
			\draw[->] (-4,0) --(4,0) node[below]{$x$};
			\draw[->] (0,-5) --(0,1) node[left]{$y$};
			\draw (0,0) node[above left=-3pt]{$O$};
			\begin{scope}
				\clip (-4,-4.5) rectangle (4,0);
				\draw [smooth,domain=-5:4, samples=200] plot (\x, {-0.5*(\x)^2});
			\end{scope}
			\draw [<->] (3.3,0)--(3.3,-4.5);
			\node[right] at (3.3,-2){$8\,\,m$};
			\draw [<->] (-3,-4.5)--(3,-4.5);
			\node[below] at (0,-4.5){$8\,\,m$};
			\node[left] at (-1.8,-1.62){$A$};
			\node[right] at (1.8,-1.62){$B$};
			\node[left=-3pt] at (-2.5,-3.125){$C$};
			\node[right] at (2.5,0-3.125){$D$};
			\draw [-] (-1.8,-1.62)--(1.8,-1.62);
			\draw [-] (-2.5,-3.125)--(2.5,0-3.125);
			\draw [dashed] (0,0)--(4,0) (1.8,-1.62)--(1.8,0) (2.5,0-3.125)--(2.5,0) (-3,-4.5)--(-3,0);
			\node[above] at (1.8,0){$x_1$};
			\node[above] at (2.5,0){$x_2$};
			\node[above] at (-3,0){$-4$};
		\end{tikzpicture}}
	Giả sử $B$ có hoành độ $x_1$,  $D$ có hoành độ $x_2$. Khi đó, phương trình đường thẳng $AB$ là $y=-\dfrac{1}{2} x_1^2$, phương trình đường thẳng $CD$ là $y=-\dfrac{1}{2} x_2^2$.\\
	Diện tích hình phẳng giới hạn bởi parabol và đường thẳng $AB$ là
	\[
		S_1=2\displaystyle\int\limits_0^{x_1}\left[-\dfrac 12x^2-\left(-\dfrac 12x_1^2\right)\right]\mathrm{\,d} x=\left.2\left(-\dfrac{x^3}6+\dfrac{x_1^2}2x\right)\right|_0^{x_1}=\dfrac 23x_1^3\,\,\left(\mathrm{m}^2\right).
	\]
	Diện tích hình phẳng giới hạn bởi parabol và đường thẳng $CD$ là
	\[
		S_2=2\displaystyle\int\limits_0^{x_2}\left[-\dfrac 12x^2-\left(-\dfrac 12x_2^2\right)\right]\mathrm{\,d} x=\left.2\left(-\dfrac{x^3}6+\dfrac{x_2^2}2x\right)\right|_0^{x_2}=\dfrac 23x_2^3\,\,\left(\mathrm{m}^2\right).
	\]
	Theo giả thiết, ta có  $S_2=2S_1\Leftrightarrow x_2^3=2x_1^3\Leftrightarrow\dfrac{x_2}{x_1}=\sqrt[3]2\approx 1{,}26$.\\
	Khi đó, $\dfrac{CD}{AB}=\dfrac{2x_2}{2x_1}\approx 1{,}26$.
	}
\end{ex}

\begin{ex}%[2H5V1-7]
	Từ mặt nước trong một bể nước, tại ba vị trí đôi một cách nhau $2$ m, người ta lần lượt thả dây dọi để quả dọi chạm đáy bể. Phần dây dọi (thẳng) nằm trong nước tại ba vị trí đó lần lượt có độ dài $4$ m; $4{,}4$ m; $4{,}8$ m. Biết đáy bể là phẳng. Hỏi đáy bể nghiêng so với mặt phẳng nằm ngang một góc bao nhiêu độ (làm tròn đến hàng phần chục)?
	% \shortans{$21{,}8$}
	\loigiai{
	Gọi ba vị trí trên mặt nước là $A$, $B$, $C$ thì tam giác $ABC$ là tam giác đều cạnh bằng $2$ m. Gọi dây dọi lần lượt là $AA'$, $BB'$, $CC'$ có độ dài lần lượt là $4$ m; $4{,}4$ m; $4{,}8$ m.\\
	Chọn hệ trục toạ độ $Oxyz$ sao cho $O$ là trung điểm của $BC$, tia $Ox$ chứa điểm $A$, tia $Oy$ chứa điểm $B$, tia $Oz$ đi qua trung điểm của $B'C'$ và đơn vị trên các trục là mét.\\
	Ta có $OB=OC=1$, $OA=\sqrt{3}$ $\Rightarrow$ $A'\left(\sqrt{3};0;4\right)$, $B'(0;1;4{,}4)$, $C'(0;-1;4{,}8)$.\\
	Khi đó, $\overrightarrow{A'B'}=\left(-\sqrt{3};1;0{,}4\right)$, $\overrightarrow{A'C'}=\left(-\sqrt{3};-1;0{,}8\right)$.\\
	Mặt phẳng $(A'B'C')$ có một véc-tơ pháp tuyến là $\overrightarrow{n}=\left[\overrightarrow{A'B'},\overrightarrow{A'C'}\right]=0{,}4\sqrt{3}\left(\sqrt{3};1;5\right)$.\\
	Mặt phẳng $(ABC)$ có một véc-tơ pháp tuyến là $\overrightarrow{k}=(0;0;1)$.\\
	Do đó, $\cos\big((ABC),(A'B'C')\big)=\left|\cos\left(\overrightarrow{n},\overrightarrow{k}\right)\right|=\dfrac{5}{\sqrt{29}}$. Góc cần tìm gần bằng $21{,}8^\circ$.
	}
\end{ex}
\Closesolutionfile{ans}


% \Closesolutionfile{ansbook}
% \HetDe
% \label{De4}
% %
% \cleardoublepage
% \setcounter{page}{1}
% \rfoot{Trang \thepage/\pageref{DA4} - Đáp án trắc nghiệm Mã đề 4}
% \begin{center}
% 	\bfseries ĐÁP ÁN TRẮC NGHIỆM MÃ ĐỀ 4
% \end{center}

% \inputansbox{10}{ans/ansDe4-TN1}
% \inputansbox[3]{2}{ans/ansDe4-TN2}
% \inputansbox{3}{ans/ansDe4-TN3}
% \label{DA4}
% %


%%CK2
% \begin{name}
	{\tenchude}
	{TOÁN 12}
	{LỚP TOÁN THẦY PHÁT}
	{Thời gian: 90 phút - Không kể thời gian phát đề}
\end{name}
\Opensolutionfile{ansbook}[ans/ansbookDe1]
\TN
\Opensolutionfile{ans}[ans/ansDe1-TN1]
\begin{ex}%[2D4N1-1]
Khẳng định nào sau đây là \textbf{sai}?
\choice
{ Mọi hàm số $f(x)$ liên tục trên đoạn $[a;b]$ đều có nguyên hàm trên đoạn $[a;b]$}
{\True $\displaystyle\int x^\alpha \mathrm{d}x=\dfrac{x^{\alpha +1}}{\alpha +1}+C$ ($C$ là hằng số, $\alpha $ là hằng số)}
{$\displaystyle\int \mathrm{e}^x\mathrm{d}x=\mathrm{e}^x+C$ ($C$ là hằng số)}
{$\displaystyle\int{\dfrac{1}{x}\mathrm{d}x=\ln \left| x \right|+C}$ ($C$ là hằng số) với $x\ne 0$}
\loigiai{
$\displaystyle\int x^{\alpha} \mathrm{d}\,x=\dfrac{x^{\alpha +1}}{\alpha +1}+C$ ($C$ là hằng số, $\alpha $ là hằng số và $\alpha \ne -1$).}
\end{ex}

\begin{ex}%[2D4N1-2]
Tìm họ nguyên hàm $F(x)$ của hàm số $f(x)=\dfrac{1}{x}$.
\choice
{\True  $F(x)=\ln \left| x \right|+C$}
{$F(x)=\ln x+C$}
{$F(x)=\ln \left| x \right|$}
{$F(x)=-\dfrac{1}{x^2}+C$}
\loigiai{
Áp dụng công thức nguyên hàm của hàm số ta có $\displaystyle\int{\frac{1}{x}\mathrm{d}\,x}=\ln \left| x \right|+C$.}
\end{ex}

\begin{ex}%[2D4N2-1]
Nếu $\displaystyle\int\limits_0^2f(x)\mathrm{\,d}x=5$ thì $\displaystyle\int\limits_0^2\left[2f(x)-1\right]\mathrm{\,d}x$ bằng
\choice
{\True $8$}
{$9$}
{$10$}
{$12$}
\loigiai{
Ta có $\displaystyle\int _0^2\left[2f(x)-1\right]\mathrm{\,d}x=2\displaystyle\int _0^2f(x)\mathrm{\,d}x-\displaystyle\int _0^21\mathrm{\,d}x=2\cdot 5-2=8$.}
\end{ex}

\begin{ex}%[2H5N1-1]
Trong không gian $Oxyz$, phương trình của mặt phẳng $(Oxy)$ là
\choice
{\True $z=0$}
{$x=0$}
{$y=0$}
{$x+y=0$}
\loigiai{
Phương trình của mặt phẳng $(Oxy)$ là $z=0$.
}
\end{ex}

\begin{ex}%[Mức độ 1]%[BG-12-New-4in1, Hiệp Hà]%[2H5N1-2]
Cho $(\alpha)$ vuông góc với giá của $\vec{a}=(-4;2;6)$. Vectơ nào dưới đây là một vectơ pháp tuyến của $(\alpha)$?
\choice
{$\vec{n_1}=(2;1;3)$}
{\True $\vec{n_2}=(-2;1;3)$}
{$\vec{n_3}=(4;-2;6)$}
{$\vec{n_4}=(4;2;-6)$}
\loigiai{
$(\alpha)$ vuông góc với giá của $\vec{a}=(-4;2;6)$ nên $\vec{a}$ là một vectơ pháp tuyến của $(\alpha)$.\\
Do đó $\vec{n_2}=\dfrac{1}{2}\vec{a}$ cũng là một vectơ pháp tuyến của $(\alpha)$.
}
\end{ex}

\begin{ex}%[2H5N2-1]
Trong không gian $Oxyz$, điểm nào dưới đây thuộc đường thẳng $d\colon\heva{& x=1+2t\\ & y=-3+t\\ & z=4+5t}$?
\choice
{$Q(4;1;3)$}
{$P(3;-2;-1)$}
{$N(2;1;5)$}
{\True $M(1;-3;4)$}
\loigiai{
Dễ thấy đường thẳng $d$ đi qua điểm $M(1;-3;4)$.
}
\end{ex}

\begin{ex}%[2025-TLOT-2018,Trần Xuân Hòa]%[2H5N2-2]
Trong không gian $O x y z$, vectơ nào sau đây là một vectơ chỉ phương của đường thẳng $\Delta\colon \dfrac{x-5}{8}=\dfrac{y-9}{6}=\dfrac{z-12}{3}$?
\choice
{\True $\overrightarrow{u}_1=(8 ; 6 ; 3)$}
{$\overrightarrow{u}_2=(8 ; 6 ;-3)$}
{$\overrightarrow{u}_3=(-8 ; 6 ;-3)$}
{$\overrightarrow{u}_4=(5 ; 9 ; 12)$}
\loigiai{
Vectơ chỉ phương của đường thẳng $\Delta\colon \dfrac{x-5}{8}=\dfrac{y-9}{6}=\dfrac{z-12}{3}$ là $\overrightarrow{u}_1=(8 ; 6 ; 3)$.
}
\end{ex}

\begin{ex}%[2H5N3-2]
Trong không gian $Oxyz$, mặt cầu $(S) \colon x^2+y^2+z^2+4x-2y+8z-1=0$ có tọa độ tâm là
\choice
{$(4;-2;8)$}
{ $(2;-1;4)$}
{\True $(-2;1;-4)$}
{$(2;-1;-4)$}
\loigiai{

}
\end{ex}

\begin{ex}%[2D6N1-1]
Cho $P(A)$, $P(B)>0$. Chọn khẳng định đúng trong các khẳng định sau.
\choice
{$\mathrm{P}(A\mid B)=\dfrac{\mathrm{P}(A\cup B)}{\mathrm{P}(A)}$}
{$\mathrm{P}(A\mid B)=\dfrac{\mathrm{P}(A\cup B)}{\mathrm{P}(B)}$}
{\True $\mathrm{P}(A\mid B)=\dfrac{\mathrm{P}(A\cap B)}{\mathrm{P}(B)}$}
{$\mathrm{P}(A\mid B)=\dfrac{\mathrm{P}(A\cap B)}{\mathrm{P}(A)}$}
\loigiai{
Nếu $\mathrm{P}(B)>0$ thì $\mathrm{P}(A\mid B)=\dfrac{\mathrm{P}(A\cap B)}{\mathrm{P}(B)}$.
}
\end{ex}

\begin{ex}%[2D6N2-1]
Cho hai biến cố ngẫu nhiên $A$, $B$ thoả mãn $\mathrm{P}(A)>0$ và $0<\mathrm{P}(B)<1$. Chọn khẳng định đúng trong các khẳng định sau.
\choice
{$\mathrm{P}(B\mid A)=\dfrac{\mathrm{P}(A)\cdot \mathrm{P}(A\mid \overline{B})}{\mathrm{P}(B)\cdot \mathrm{P}(A\mid B)+\mathrm{P}(B)\cdot \mathrm{P}(A\mid B)}$}
{$\mathrm{P}(B\mid A)=\dfrac{\mathrm{P}(\overline{B})\cdot \mathrm{P}(A\mid \overline{B})}{\mathrm{P}(B)\cdot \mathrm{P}(A\mid B)+\mathrm{P}(B)\cdot \mathrm{P}(A\mid B)}$}
{$\mathrm{P}(B\mid A)=\dfrac{\mathrm{P}(A)\cdot \mathrm{P}(A\mid B)}{\mathrm{P}(B)\cdot \mathrm{P}(A\mid B)+\mathrm{P}(\overline{B})\cdot \mathrm{P}(A\mid \overline{B})}$}
{\True $\mathrm{P}(B\mid A)=\dfrac{\mathrm{P}(B)\cdot \mathrm{P}(A\mid B)}{\mathrm{P}(B)\cdot \mathrm{P}(A\mid B)+\mathrm{P}(\overline{B})\cdot \mathrm{P}(A\mid \overline{B})}$}
\loigiai{
$\mathrm{P}(B\mid A)=\dfrac{\mathrm{P}(B)\cdot \mathrm{P}(A\mid B)}{\mathrm{P}(B)\cdot \mathrm{P}(A\mid B)+\mathrm{P}(\overline{B})\cdot \mathrm{P}(A\mid \overline{B})}$.
}
\end{ex}

\begin{ex}%[2D6N2-3]
Cho hai biến cố $A$, $B$ thoả mãn $\mathrm{P}(A)=0{,}4$; $\mathrm{P}(B)=0{,}3$; $\mathrm{P}(A | B)=0{,}25$. Khi đó, $\mathrm{P}(B | A)$ bằng:
\choice
{\True $0{,}1875$}
{$0{,}48$}
{$0{,}333$}
{$0{,}95$}
\loigiai{
Theo công thức Bayes, ta có $\mathrm{P}(B|A)=\dfrac{\mathrm{P}(B) \cdot \mathrm{P}(A|B)}{\mathrm{P}(A)}=\dfrac{0{,}3 \cdot 0{,}25}{0{,}4}=0{,}1875$.
}
\end{ex}

\begin{ex}%[2H5N3-3]
Trong không gian $Oxyz$, mặt cầu có tâm $I(2; 1; -3)$ và bán kính $9$ và có phương trình là
\choice
{\True $(x-2)^2+(y-1)^2+(z+3)^2=81$}
{$(x+2)^2+(y+1)^2+(z-3)^2=81$}
{$(x-2)^2+(y-1)^2+(z+3)^2=9$}
{$(x+2)^2+(y+1)^2+(z-3)^2=9$}
\loigiai{
Trong không gian $Oxyz$, mặt cầu có tâm $I(2; 1; -3)$ và bán kính $9$ và có phương trình là $(x-2)^2+(y-1)^2+(z+3)^2=81$.
}
\end{ex}
\Closesolutionfile{ans}

\TNTF
\Opensolutionfile{ans}[ans/ansDe1-TN2]
\begin{ex}
Cho hàm số $f(x)=\dfrac{2x+1}{x+2}$.
\choiceTF
{\True $\displaystyle\int f'(x)\mathrm{\,d}x=\dfrac{2x+1}{x+2}+C$}
{$\displaystyle\int f(x)\mathrm{\,d}x=2\ln \left| x+2 \right|+C$}
{\True $\displaystyle\int_{-1}^2 [f'(x)-2] \mathrm{\,d}x >-3$}
{\True Diện tích hình phẳng giới hạn với đường cong $y=f(x)$, trục hoành và các đường thẳng $x=-1$ và $x=2$ bằng $4+6\ln a$ và $0<a<1$}
\loigiai{
    \begin{itemchoice}
    \itemch $\displaystyle\int f'(x)\mathrm{\,d}x= f(x)+C = \dfrac{2x+1}{x+2}+C$.
    \itemch $\displaystyle\int f(x)\mathrm{\,d}x=\displaystyle\int \left(2-\dfrac{3}{x+2}\right)\mathrm{\,d}x=2x-3\ln \left| x+2 \right|+C$.
    \itemch $\displaystyle\int_{-1}^2 [f'(x)-2] \mathrm{\,d}x=f(2)-f(-1)-2x\vert_{-1}^2=\dfrac54+\dfrac13-2\cdots 3=0{,}5>-3$.
    \itemch Diện tích hình phẳng giới hạn với đường cong $y=f(x)$, trục hoành và các đường thẳng $x=-1$ và $x=2$ là 
    $$S=\displaystyle\int_{-1}^2 |f(x)| \mathrm{\,d}x = -\int_{-1}^{-\frac12} f(x) \mathrm{\,d}x + \int_{-\frac12}^{2} f(x) \mathrm{\,d}x = - \left(2x-3\ln \left| x+2 \right| \right)\vert_{-1}^{-\frac12} + \left(2x-3\ln \left| x+2 \right| \right)\vert_{-\frac12}^{2} = 4+3\ln \dfrac{9}{16} = 4+6\ln \dfrac{3}{4}$$
    Vậy $0<a=\dfrac{3}{4}<1$.
    \end{itemchoice}
}


\end{ex}

\begin{ex}%[Mức độ 2]%[BG-12-New-4in1, Hiệp Hà]%[2H5H1-2]
Cho mặt phẳng $(\alpha)$ đi qua $A(-1;1;2)$ có cặp vectơ chỉ phương là $\vec{a} = (1; -2; 3)$ và $\vec{b}= (2; 1; -1)$.
\choiceTF
{$\vec{a}$ có giá song song với mặt phẳng $(\alpha)$}
{\True Mặt phẳng $(\alpha)$ có phương trình là $x-7y-5z+18=0$}
{Đường thẳng $d \colon \dfrac{x+1}{2}=\dfrac{y-1}{1}=\dfrac{z-2}{-1}$ song song với mặt phẳng $(\alpha)$}
{Mặt phẳng $(\alpha)$ cắt mặt cầu $(S)$ có tâm $I(2; 1; -3)$ và bán kính $1$ theo giao tuyến là một đường tròn}
\loigiai{
\begin{itemchoice}
\itemch $\vec{a}$ là một vectơ trong cặp vectơ chỉ phương nên $\vec{a}$ có giá song song hoặc nằm trong mặt phẳng $(\alpha)$.
\itemch Ta có
\[
\begin{aligned}
\vec{n}=[\vec{a}, \vec{b}] & =\left(\left|\begin{array}{cc}
-2 & 3 \\ 1 & -1
\end{array}\right| ;\left|\begin{array}{cc}
3 & 1 \\ -1 & 2
\end{array}\right| ;\left|\begin{array}{cc}
1 & -2 \\ 2 & 1
\end{array}\right|\right) \\
& =(-1; 7; 5) .
\end{aligned}
\]
Do đó $\vec{n_1}=(1;-7;-5) =-\vec{n}$ là một vectơ pháp tuyến của mặt phẳng $(\alpha)$ nên mặt phẳng $(\alpha)$ đi qua $A(-1;1;2)$ có phương trình là
$$1(x+1)-7(y-1)-5(z-2)=0 \Leftrightarrow x-7y-5z+18=0.$$
\itemch Đường thẳng $d \colon \dfrac{x+1}{2}=\dfrac{y-1}{1}=\dfrac{z-2}{-1}$ có vectơ chỉ phương là $\vec{b}=(2;1;-1)$ và đi qua $A(-1;1;2)$ nên $d$ nằm trên $(\alpha)$.
\itemch Vì $d(I,(\alpha))=\dfrac{\left| 1\cdot 2-7\cdot 1-5\cdot (-3) \right|}{\sqrt{1^2+7^2+5^2}} = \dfrac{2\sqrt{3}}{3}>1$ nên mặt phẳng $(\alpha)$ không cắt mặt cầu $(S)$.
\end{itemchoice}
}
\end{ex}
\Closesolutionfile{ans}

\TNSA
\Opensolutionfile{ans}[ans/ansDe1-TN3]
\begin{ex}%[2D4H1-4]
Cho hàm số $f(x)=2x+\mathrm{e}^x$. Một nguyên hàm $F(x)$ của hàm số $f(x)$ thỏa mãn $F(0)=2024$. Biết $F(x)=ax^2+b\mathrm{e}^x+c$, giá trị của $a+b+c$ là
\shortans{$2025$}
\loigiai{
Ta có $\displaystyle\int f(x)\mathrm{\,d}x=\displaystyle\int (2x+\mathrm{e}^x)\mathrm{\,d}x=x^2+\mathrm{e}^x+C$.\\
Có $F(x)$ là một nguyên hàm của $f(x)$ và $F(0)=2024$.\\
Tìm được $\heva{&F(x)=x^2+\mathrm{e}^x+C\\ &F(0)=2024} \Rightarrow 1+C=2024 \Leftrightarrow C=2023$.\\
Suy ra $F(x)=x^2+\mathrm{e}^x+2023$.\\
Vậy $a+b+c=2025$.
}
\end{ex}

\begin{ex}%[2D4V2-6]
Gọi $h(t)$ cm là mức nước trong bồn chứa sau khi bơm được $t$ giây. Biết rằng tốc độ tăng giảm của mực nước là $h^{\prime}(t)=\dfrac{1}{5} \sqrt[3]{t+8}$ (cm/s) và lúc đầu bồn không có nước. Tìm mức nước ở bồn (đơn vị: cm) sau khi bơm nước được $6$ giây (làm tròn đến chữ số hàng phần trăm).
\shortans{$2{,}66$}
\loigiai{Hàm $h(t)=\displaystyle\int\limits \dfrac{1}{5} \sqrt[3]{t+8} \mathrm{\,d} t=\dfrac{3}{20}(t+8) \sqrt[3]{t+8}+C$.\\
Lúc $t=0$, bồn không chứa nước. Suy ra $h(0)=0 \Rightarrow \dfrac{12}{5}+C=0 \Leftrightarrow C=-\dfrac{12}{5}$.\\
Vậy, hàm $h(t)=\dfrac{3}{20}(t+8) \sqrt[3]{t+8}-\dfrac{12}{5}$.\\
Mức nước trong bồn sau $6$ giây là $h(6) \simeq 2{,}66$ cm.
}
\end{ex}

\begin{ex}%[2H5V2-8]
    Hình vẽ dưới đây là hình ảnh Cầu Cổng Vàng (The Golden Gate Bridge) ở Mỹ. Xét hệ trục toạ độ $O x y z$ với $O$ là bệ của chân cột trụ tại mặt nước, trục $O z$ trùng với cột trụ, mặt phẳng $O x y$ là mặt nước và xem như trục $O y$ cùng phương với cầu như hình vẽ. Dây cáp $A D$ (xem như là một đoạn thẳng) đi qua đỉnh $D$ thuộc trục $O z$ và điểm $A$ thuộc mặt phẳng $O y z$, trong đó điểm $D$ là đỉnh cột trụ cách mặt nước $227$ m, điểm $A$ cách mặt nước $75$ m và cách trục $O z$ khoảng $343$ m. \begin{flushright}
    \textit{(Nguồn: https://www.goldengate.org/assets/1/6/ggb-exhibit-chapter-statistics.pdf)}
    \end{flushright}
    \begin{center}
    \includegraphics[scale=.5]{images/Cau-cong-vang}
    \end{center}
    Giả sử ta dùng một đoạn dây nối điểm $N$ trên dây cáp $A D$ và điểm $M$ trên thành cầu, biết $M$ cách mặt nước $75$ m và $M N$ song song với cột trụ. Tính độ dài $M N$ (đơn vị mét) biết điểm $M$ cách trục $O z$ một khoảng bằng $230$ m (kết quả làm tròn đến hàng phần mười).
    \shortans{$50{,}1$}
    \loigiai{
    Chọn một đơn vị trên các trục bằng $1$ m.\\
    Ta có $D(0 ; 0 ; 227),$ $ A(0 ;-343 ; 75),$ $ M(0 ;-230 ; 75)$, $\overrightarrow{A D}=(0 ; 343 ; 152)$.\\ Phương trình đường thẳng $A D\colon \heva{&x=0 \\& y=343 t \\& z=227+152 t} \Rightarrow N(0 ; 343 t ; 227+152 t)$.\\
    Ta có $\overrightarrow{M N}=(0 ; 343 t+230 ; 152+152 t)$, $M N$ song song với trục $O z$, suy ra \[ 343 t+230=0 \Rightarrow t=-\dfrac{230}{343} \Rightarrow M N=152+152 \cdot \left(-\dfrac{230}{343}\right) \approx 50,1(m).\]
    }
    \end{ex}

\begin{ex}%[2D6V1-3]
Ông An hằng ngày đi làm bằng xe máy hoặc xe buýt. Nếu hôm nay ông đi làm bằng xe buýt thì xác suất để hôm sau ông đi làm bằng xe máy là $0,4$ . Nếu hôm nay ông đi làm bằng xe máy thì xác suất để hôm sau ông đi làm bằng xe buýt là $0,7$ . Xét một tuần mà thứ Hai ông An đi làm bằng xe buýt. Tính xác suất để thứ Tư trong tuần ông An đi làm bằng xe máy.

\shortans{0,36}
\loigiai
{
Gọi $A$ là biến cố \textquotedblleft Thứ Ba, ông An đi làm bằng xe máy\textquotedblright.\\
$B$ là biến cố \textquotedblleft Thứ Tư, ông An đi làm bằng xe máy\textquotedblright.\\
Khi đó $\heva{&P(A) = 0{,}4&\Rightarrow& P(\overline{A}) = 1-0{,}4 = 0{,}6\\ &P(\overline{B}|A) = 0{,}7 &\Rightarrow& P(B|A) = 1-0{,}7 = 0{,}3\\ &P(B|\overline{A}) = 0{,}4 &\Rightarrow& P(\overline{B}|\overline{A}) = 1-0{,}4=0{,}6.}$
\begin{center}
\begin{tikzpicture}[>=stealth]
%Khung 1
\draw (1,3.0) node{\textbf{Thứ Hai}};
\draw (-0,-1) rectangle (2.2,0);
\draw (1.1,-0.5) node{Buýt};
%Mui ten 1,2
\draw [->] (2.2,-0.5)--(3.8,1.6) node[pos=0.5,sloped,above]{$0{,}4$};
\draw [->] (2.2,-0.5)--(3.8,-2.6) node[pos=0.5,sloped,below]{\color{red}$0{,}6$};
%Khung 2.1
\draw (4.5,3.0) node{\textbf{Thứ Ba}};
\draw (3.8,1.1) rectangle (5.1,2.1);
\draw (8.9/2,1.6) node{$A$} ;
%Khung 2.2
\draw (3.8,-2.1) rectangle (5.1,-3.1);
\draw (8.9/2,-2.6) node{$\overline{A}$};
%Mui ten 3,4
\draw [->] (5.1,1.6)--(6.5,2.6) node[pos=0.5,sloped,above]{\color{red}$0{,}3$};
\draw [->] (5.1,1.6)--(6.5,0.6) node[pos=0.5,sloped,below]{$0{,}7$};
%Mui ten 5,6
\draw [->] (5.1,-2.6)--(6.5,-1.6) node[pos=0.5,sloped,above]{$0{,}4$};
\draw [->] (5.1,-2.6)--(6.5,-3.6) node[pos=0.5,sloped,below]{\color{red}$0{,}6$};
%Khung 3.1
\draw (6.5,2.2) rectangle (7.7,3.2);
\draw (7.1,5.4/2) node{$B$} ;
%Khung 3.2
\draw (7.0,3.7) node{\textbf{Thứ Tư}};
\draw (6.5,1.2) rectangle (7.7,0.2);
\draw (7.1,1.4/2) node{$\overline{B}$} ;
%Khung 3.3
\draw (6.5,-1.1) rectangle (7.7,-2.1);
\draw (7.1,-3.2/2) node{$B$} ;
%Khung 3.3
\draw (6.5,-2.9) rectangle (7.7,-3.9);
\draw (7.1,-3.4) node{$\overline{B}$} ;
%Kết quả
\draw (9.5,3.7) node{\textbf{Kết quả}};
\draw (9.5,2.7) node{$AB$};
\draw (9.5,0.7) node{$A \overline{B}$};
\draw (9.5,-1.6) node{$\overline{A}B$};
\draw (9.5,-3.4) node{$\overline{A}~\overline{B}$};
%Xác suất
\draw (12.5,3.7) node{\textbf{Xác suất}};
\draw (12.5,2.7) node{$0{,}12$};
\draw (12.5,0.7) node{$0{,}28$};
\draw (12.5,-1.6) node{$0{,}24$};
\draw (12.5,-3.4) node{$0{,}36$};
\end{tikzpicture}
\end{center}
Áp dụng công thức xác suất toàn phần để tính xác suất thứ Tư ông An đi làm bằng xe máy là
\[P(B) = P(A) \cdot P(B|A) + P(\overline{A}) \cdot P(B|\overline{A}) = 0{,}4 \cdot 0{,}3 + 0{,}6 \cdot 0{,}4 = 0{,}36.\]
}
\end{ex}
\TL
\begin{ex}%[2H5H2-3]%[Dự án 2025 - Đề cấu trúc mới của Bộ theo [Thành Đức Trung]
Trong không gian $Oxyz$, cho tam giác $ABC$ có $A(0;0;1)$, $B(-3;2;0)$, $C(2;-2;3)$. Viết phương trình đường cao kẻ từ $B$ của tam giác $ABC$.% đi qua điểm $K(a;b;-2)$. Tính $ab$.
% \shortans{$-2$}
\loigiai
{
Gọi $\Delta$ là đường cao kẻ từ $B$ của tam giác $ABC$.\\
Ta có $\heva{& \overrightarrow{AB}=(-3;2;-1) \\ & \overrightarrow{AC}=(2;-2;2)} \Rightarrow \left[\overrightarrow{AB},\overrightarrow{AC}\right]=(2;4;2)$. \\
Suy ra một véc-tơ pháp tuyến của mặt phẳng $(ABC)$ là $\overrightarrow{n}=(1;2;1)$.\\
Ta có $\heva{ & \Delta \subset(ABC) \\ & \Delta \perp AC}$, suy ra đường thẳng $\Delta$ nhận $\left[\overrightarrow{n},\overrightarrow{AC}\right]$ làm một véc-tơ chỉ phương.\\
Có $\left[\overrightarrow{n},\overrightarrow{AC}\right]=(6;0;-6)=6\overrightarrow{u}$ với $\overrightarrow{u}=(1;0;-1)$. \\
Suy ra đường thẳng $\Delta$ nhận $\overrightarrow{u}=(1;0;-1)$ làm véc-tơ chỉ phương.\\
Do đó phương trình đường thẳng $\Delta$ là $\Delta \colon \heva{ & x=-3+t \\ & y=2 \\ & z=-t} $%\Rightarrow K(-1;2;-2)$.\\
% Vậy $a=-1$; $b=2$ và $ab=-2$.
}
\end{ex}

\begin{ex}%[BAI-GIANG-12-4IN1, Võ Thị Thùy Trang]%[Cánh diều]%[2D6C2-4]
Giả sử có một loại bệnh mà tỉ lệ người mắc bệnh là $0{,}1\%$. Giả sử có một loại xét nghiệm, mà ai mắc bệnh khi xét nghiệm cũng có phản ứng dương tính, nhưng tỉ lệ phản ứng dương tính giả là $5\%$ (tức là trong số những người không bị bệnh có $5\%$ số người xét nghiệm lại có phản ứng dương tính). Khi một người xét nghiệm có phản ứng dương tính thì khả năng mắc bệnh của người đó là bao nhiêu phần trăm (làm tròn kết quả đến hàng phần trăm)?
% \shortans{$1{,}96$}
\loigiai{
\begin{itemize}
\item Xét hai biến cố\\
$K$ \lq\lq Người được chọn ra không mắc bệnh\rq\rq;\\
$D$ \lq\lq Người được chọn ra có phản ứng dương tính\rq\rq.\\
Do tỉ lệ người mắc bệnh là $0{,}1 \%=0{,}001$ nên $\mathrm{P}(K)=1-0{,}001=0{,}999$.\\
Trong số những người không mắc bệnh có $5 \%$ số người có phản ứng dương tính nên \break$\mathrm{P}(D \mid K)=5 \%=0{,}05$. Vì ai mắc bệnh khi xét nghiệm cũng có phản ứng dương tính nên $\mathrm{P}(D \mid \overline{K})=1$.\\
Sơ đồ hình cây ở bên dưới biểu thị tình huống đã cho.
\begin{center}
\begin{tikzpicture}[teal,grow=right, edge from parent/.style={draw,-latex}, label distance = 0.2cm,
level 1/.style = {level distance=3.5cm, sibling distance=28mm},
level 2/.style = {level distance=4.5cm, sibling distance=22mm},
level 3/.style = {level distance=4.5cm, sibling distance=22mm},
]

\node {}
child {node {$\overline{K}$
}
child {node {$\overline {D}$}
}
child {node[opacity=.75] {$D$}
edge from parent
node[above,sloped] {$\mathrm{P}(D\mid \overline {K})=1$}
}
edge from parent
node[below,sloped] {$\mathrm{P}(\overline {K})=0{,}001$}
}
child {node[] {$K$}
child {node[opacity=.75] {$\overline {D}$}
}
child {node[] {$D$}
edge from parent
node[above,sloped] {$\mathrm{P}(D\mid K)=0{,}05$}
}
edge from parent
node[above,sloped] {$\mathrm{P}(K)=0{,}009$}
};
\end{tikzpicture}
\end{center}
\item Ta thấy, khả năng mắc bệnh của một người xét nghiệm có phản ứng dương tính chính là $\mathrm{P}(\overline{K} \mid D)$. Áp dụng công thức Bayes, ta có
\[
\mathrm{P}(\overline{K}|D)=\dfrac{\mathrm{P}(\overline{K}) \cdot \mathrm{P}(D| \overline{K})}{\mathrm{P}(\overline{K}) \cdot \mathrm{P}(D| \overline{K})+\mathrm{P}(K) \cdot \mathrm{P}(D| K)}=\dfrac{0{,}001}{0{,}001+0{,}999 \cdot 0{,}05} \approx 1{,}96 \% .
\]
Vậy xác suất mắc bệnh của một người xét nghiệm có phản ứng dương tính là $1{,}96\%$.
\end{itemize}
}
\end{ex}

\begin{ex}%[Mức độ 3]giảng 12 New - 4in1, Đoàn Hùng]%[2H5C1-7]
Trong không gian với hệ tọa độ $Oxyz$, cho ba điểm $A(1;4;5)$, $B(3;4;0)$, $C(2;-1;0)$ và mặt phẳng $(P)\colon 3x-3y-2z-12=0$. Gọi $M(a;b;c)$ thuộc $(P)$ sao cho $MA^2+MB^2+3MC^2$ đạt giá trị nhỏ nhất. Tính tổng $a+b+c$.
% \shortans{$3$}
\loigiai{
Gọi $I(x;y;z)$ là điểm thỏa mãn $\overrightarrow{IA}+\overrightarrow{IB}+3\overrightarrow{IC}=\overrightarrow{0}$.\\
Ta có $\overrightarrow{IA}=(1-x;4-y;5-z)$, $\overrightarrow{IB}=(3-x;4-y;-z)$ và $3\overrightarrow{IC}=(6-3x;-3-3y;-3z)$.\\
Từ ta có hệ phương trình: $\heva{&1-x+3-x+6-3x=0\\&4-y+4-y-3-3y=0\\&5-z-z-3z=0} \Leftrightarrow \heva{&x=2\\&y=1\\&z=1} $$\Rightarrow I(2;1;1)$.\\
Ta có
\begin{itemize}
\item $MA^2=\overrightarrow{MA}^2=(\overrightarrow{MI}+\overrightarrow{IA})^2=MI^2+2\overrightarrow{MI}\cdot \overrightarrow{IA}+IA^2$.
\item $MB^2=\overrightarrow{MB}^2=(\overrightarrow{MI}+\overrightarrow{IB})^2=MI^2+2\overrightarrow{MI}\cdot \overrightarrow{IB}+IB^2$.
\item $3MC^2=3\overrightarrow{MC}^2=3(\overrightarrow{MI}+\overrightarrow{IC})^2=3(MI^2+2\overrightarrow{MI}\cdot \overrightarrow{IC}+IC^2)$.
\end{itemize}
Do đó $S=MA^2+MB^2+3MC^2=5MI^2+IA^2+IB^2+3IC^2$.\\
Do $IA^2+IB^2+3IC^2$ không đổi nên $S$ đạt giá trị nhỏ nhất khi và chỉ khi $MI$ đạt giá trị nhỏ nhất. Tức là $M$ là hình chiếu của $I$ lên mặt phẳng $(P)\colon 3x-3y-2z-12=0$ Suy ra $\overrightarrow{IM}$ cùng phương với véc-tơ chỉ phương pháp tuyến $\overrightarrow{n}=(3;-3;-2)$ của $(P)$.\\
Suy ra $M(2+3t;1-3t;1-2t)$.\\
Vì $M\in (P)$ nên $3(2+3t)-3(1-3t)-2(1-2t)-12=0\Leftrightarrow 22t-11=0\Leftrightarrow t=\dfrac{1}{2}$.\\
Suy ra $M\left(\dfrac{7}{2};-\dfrac{1}{2};0\right)$.\\
Vậy $a+b+c=\dfrac{7}{2}-\dfrac{1}{2}=3$.
}
\end{ex}
\Closesolutionfile{ans}


\Closesolutionfile{ansbook}

% \begin{name}
	{\tenchude}
	{TOÁN 12}
	{LỚP TOÁN THẦY PHÁT}
	{Thời gian: 90 phút - Không kể thời gian phát đề}
\end{name}
\Opensolutionfile{ansbook}[ans/ansbookDe2]
\TN
\Opensolutionfile{ans}[ans/ansDe2-TN1]
\begin{ex}%[Dự án 2025 - đề cấu trúc mới, Nguyễn Kiều Nhã Tú]%[2D4N1-1]
Hàm số $F(x)$ là một nguyên hàm của hàm số $f(x)$ trên khoảng $K$ nếu
\choice
{$F'(x)=-f(x)$, $\forall x\in K$}
{$f'(x)=F(x)$, $\forall x\in K$}
{\True $F'(x)=f(x)$, $\forall x \in K$}
{$f'(x)=-F(x)$, $\forall x\in K$}
\loigiai{
Theo tính chất của nguyên hàm có $F'(x)=f(x)$, $\forall x\in K$.
}
\end{ex}

\begin{ex}%[2D4N1-2]
Tìm nguyên hàm của hàm số $f(x)=\sqrt {2x-1}$.
\choice
{ $\displaystyle\int{f(x)\mathrm{d}x=\dfrac{2}{3}( 2x-1 )\sqrt {2x-1}+C}$}
{\True $\displaystyle\int{f(x)\mathrm{d}x=\dfrac{1}{3}( 2x-1 )\sqrt {2x-1}+C}$}
{ $\displaystyle\int{f(x)\mathrm{d}x=-\dfrac{1}{3}\sqrt {2x-1}+C}$}
{$\displaystyle\int{f(x)\mathrm{d}x=\dfrac{1}{2}\sqrt {2x-1}+C}$}
\loigiai{
$\displaystyle\int{f\left( x \right)\mathrm{d}x=\displaystyle\int{\sqrt {2x-1}\mathrm{d}x=\dfrac{1}{2}\displaystyle\int{{\left( 2x-1 \right)}^{\frac{1}{2}}\mathrm{d}( 2x-1 )}}}=\dfrac{1}{3}( 2x-1 )\sqrt {2x-1}+C$.}
\end{ex}

\begin{ex}%[2D4N2-1]
Nếu $\displaystyle\int_0^2 f(x) d x=3$ thì $\displaystyle\int_0^2\left[2f(x)-1\right]\mathrm{\,d}x$ bằng
\choice
{$6$}
{\True $4$}
{$8$}
{$5$}
\loigiai{
Ta có $\displaystyle\int_0^2\left[2f(x)-1\right]\mathrm{\,d}x=2\displaystyle\int_0^2f(x)\mathrm{\,d}x-\displaystyle\int_0^2\mathrm{\,d}x=2\cdot 3-2=4$.}
\end{ex}

\begin{ex}%[2H5N1-1]
Trong không gian với hệ toạ độ $Oxyz$, phương trình nào dưới đây là phương trình của mặt phẳng $(Oyz)$?
\choice
{$y=0$}
{\True $x=0$}
{$y-z=0$}
{$z=0$}
\loigiai{
Mặt phẳng $(Oyz)$ đi qua điểm $O(0 ; 0 ; 0)$ và có véc-tơ  pháp tuyến là $\vec{i}=(1 ; 0 ; 0)$ nên ta có phương trình mặt phẳng $(O y z)$ là  $1(x-0)+0(y-0)+0(z-0)=0 \Leftrightarrow x=0$.
}
\end{ex}

\begin{ex}%[Mức độ 1]%[BG-12-New-4in1, Hiệp Hà]%[2H5N1-2]
Vectơ nào dưới đây là một vectơ pháp tuyến của $(Oxy)$?
\choice
{$\vec{n_1}=(2;0;0)$}
{$\vec{n_2}=(1;1;0)$}
{$\vec{n_3}=(0;3;0)$}
{\True $\vec{n_4}=(0;0;-1)$}
\loigiai{
Ta có $Oz \perp (Oxy)$ nên $\vec{k}=(0;0;1)$ là một vectơ pháp tuyến của $(\alpha)$.\\
Khi đó, $\vec{n_4}=-\vec{k}$ là một vectơ pháp tuyến của $(\alpha)$.
}
\end{ex}

\begin{ex}%[12-MH-2-MH2025]%[MH-2025,Chu Hà]%[2H5N2-1]
Trong không gian tọa độ $Oxyz$, phương trình nào sau đây là phương trình tham số của đường thẳng?
\def\dotEX{}
\choice
{$\heva{&2x+3y-z=0\\&x+y+z=8.}$}
{$\heva{&-3x+z=0\\&x+2y+z-7=0.}$}
{$\heva{&x=2+t\\&y=3-t\\&z =t^2.}$}
{\True  $\heva{&x=-4-3t\\&y=2-5t\\&z=-1+6t.}$}
\loigiai{
Phương trình đường thẳng có dạng $\heva{&x=-4-3t\\&y=2-5t\\&z=-1+6t.}$ với $t$ là tham số.
}
\end{ex}

\begin{ex}%[Ex-Ôn tập 2025, Nguyễn Văn Nay]%[2H5N2-2]
Trong không gian với hệ tọa độ $Oxyz$, vectơ nào sau đây là vectơ chỉ phương của đường thẳng $\Delta\colon\heva{&x=-4+2t \\ &y=7-3t \\&z=8-9t}$?
\choice
{$\vec{u}_1=(4;7;8)$}
{$\vec{u}_2=(-4;7;8)$}
{$\vec{u}_3=(2;3;9)$}
{\True $\vec{u}_4=(2;-3;-9)$}
\loigiai{Một vectơ chỉ phương của đường thẳng là $\vec{u}_4=(2;-3;-9)$.}
\end{ex}

\begin{ex}%[NB]giảng 12 New - 4in1, Nguyễn Vân Trường]%[2H5N3-2]
Trong hệ trục toạ độ $Oxyz$, tìm $m$ để phương trình $(S_m) \colon x^2+y^2+z^2+2x-2my+2mz-5m^2 = 0$ xác định một mặt cầu có bán kính nhỏ nhất.
\choice
{$m=1$}
{$m=7$}
{\True $m=0$}
{$m=\dfrac{1}{7}$}
\loigiai{
Để phương trình $(S_m)$ xác định một mặt cầu thì $1^2+m^2+m^2+5m^2 =1+7m^2>0$ với mọi $m \in \mathbb{R}$.
Khi đó bán kính của mặt cầu $(S_m)$ là
$R=\sqrt{1+7m^2} \ge 1$. \\
Dấu đẳng thức xảy ra khi $m = 0$. Vậy $m=0$.
}
\end{ex}

\begin{ex}%[2D6N1-1]
Cho hai biến cố xung khắc $A$, $B$ thoả mãn $\mathrm{P}(A)=0{,}35$ và $\mathrm{P}(B)=0{,}55$. Khi đó $\mathrm{P}(A\mid B)$ bằng
\choice
{\True $0$}
{$0{,}9$}
{$0{,}1$}
{$0{,}4$}
\loigiai{
Do $A$, $B$ xung khắc nên $\mathrm{P}(A\mid B)=0$.
}
\end{ex}

\begin{ex}%[2D6N2-1]%[Lê Công Trường]
Cho hai biến cố $C$ và $D$ với $0\leq \mathrm{P}(D)\leq 1$. Công thức xác suất toàn phần là
\choice
{\True $\mathrm{P}(C)=\mathrm{P}(D)\cdot\mathrm{P}(C\mid D)+\mathrm{P}(\overline{D})\cdot\mathrm{P}(C\mid \overline{D})$}
{$\mathrm{P}(C)=\mathrm{P}(\overline{D})\cdot\mathrm{P}(C\mid D)+\mathrm{P}(D)\cdot\mathrm{P}(C\mid \overline{D})$}
{$\mathrm{P}(D)=\mathrm{P}(\overline{D})\cdot\mathrm{P}(C\mid D)+\mathrm{P}(D)\cdot\mathrm{P}(C\mid \overline{D})$}
{$\mathrm{P}(D)=\mathrm{P}(\overline{D})\cdot\mathrm{P}(C\mid D)+\mathrm{P}(C)\cdot\mathrm{P}(C\mid \overline{D})$}
\loigiai{Cho hai biến cố $C$ và $D$ với $0\leq \mathrm{P}(D)\leq 1$. Công thức xác suất toàn phần là \[\mathrm{P}(C)=\mathrm{P}(D)\cdot\mathrm{P}(C\mid D)+\mathrm{P}(\overline{D})\cdot\mathrm{P}(C\mid \overline{D})\] }
\end{ex}

\begin{ex}%[2D6N2-3]%[Dự án EX-TF-TLN lần 4 - Quan Ón]
Cho các biến cố $A$ và $B$ thỏa mãn $\mathrm{P}(A) > 0$, $\mathrm{P}(B) > 0$. Khi đó $\mathrm{P}(A\mid B)$ bằng biểu thức nào dưới đây?
\choice
{\True $\dfrac{\mathrm{P}(A)\cdot \mathrm{P}(B\mid A)}{\mathrm{P}(B)}$}
{$\dfrac{\mathrm{P}(B)\cdot \mathrm{P}(B\mid A)}{\mathrm{P}(A)}$}
{$\dfrac{\mathrm{P}(B)}{\mathrm{P}(A)\cdot \mathrm{P}(B\mid A)}$}
{$\dfrac{\mathrm{P}(A)}{\mathrm{P}(B)\cdot \mathrm{P}(B\mid A)}$}
\loigiai{
Theo công thức xác suất có điều kiện, ta có $\mathrm{P}(A\mid B) = \dfrac{\mathrm{P}(AB)}{\mathrm{P}(B)}$.\\
Theo công thức nhân, ta có $\mathrm{P}(AB) = \mathrm{P}(A)\cdot \mathrm{P}(B\mid A)$.\\
Do đó $\mathrm{P}(A\mid B) = \dfrac{\mathrm{P}(AB)}{\mathrm{P}(B)} = \dfrac{\mathrm{P}(A)\cdot \mathrm{P}(B\mid A)}{\mathrm{P}(B)}$.
}
\end{ex}

\begin{ex}%[2H5N3-3]
Trong không gian $Oxyz$, mặt cầu có tâm $I(1;-1; 2)$ và bán kính $R=5$ có phương trình là
\choice
{\True $(x-1)^2+(y+1)^2+(z-2)^2=25$}
{$(x-1)^2+(y+1)^2+(z-2)^2=5$}
{$(x+1)^2+(y-1)^2+(z+2)^2=25$}
{$(x+1)^2+(y-1)^2+(z+2)^2=5$}
\loigiai{
Mặt cầu cần tìm có phương trình $(x-1)^2+(y+1)^2+(z-2)^2=25$.
}
\end{ex}
\Closesolutionfile{ans}

\TNTF
\Opensolutionfile{ans}[ans/ansDe2-TN2]
\begin{ex}%[2D4H2-2]
Cho hàm số $f(x) = \heva{
&2x^2 + 3 \text{ khi } x \geq 1 \\
&2 - x^3 \text{ khi } x < 1
}$.
\choiceTF
{\True Trên $[1;+\infty)$ hàm số $f(x)$ có nguyên hàm là $F_1(x) = \dfrac{2}{3}x^3 + 3x + C_1$}

{\True Trên $(-\infty;1)$ hàm số $f(x)$ có nguyên hàm là $F_2(x) = 2x - \dfrac{1}{4}x^4 + C_2$}
{$\displaystyle\int\limits_{-2025}^{2025} f(x) \mathrm{d}x = \displaystyle\int\limits_{1}^{2025} (2x^2 + 3) \mathrm{d}x + \displaystyle\int\limits_{-2025}^{1} (2 - x^3) \mathrm{d}x$}
{\True Diện tích hình phẳng tạo bởi đồ thị hàm số $f(x)$, trục hoành và hai đường thẳng $x=-1$ và $x=2$ là $S=\displaystyle\int\limits_{-1}^{1} (2 - x^3) \mathrm{d}x + \displaystyle\int\limits_{1}^{2} (2x^2 + 3) \mathrm{d}x$}

\loigiai{
Do $f(x) = \heva{
&2x^2 + 3 \text{ khi } x \geq 1 \\
&2 - x^3 \text{ khi } x < 1
}$ nên
\begin{itemchoice}
\itemch Trên $[1;+\infty)$ hàm số $f(x)$ có nguyên hàm là $\displaystyle\int (2x^2 + 3) \mathrm{d}x = \dfrac{2}{3}x^3 + 3x + C_1$.
\itemch Trên $(-\infty;1)$ hàm số $f(x)$ có nguyên hàm là $\displaystyle\int (2 - x^3) \mathrm{d}x = 2x - \dfrac{1}{4}x^4 + C_2$.
\itemch Ta có $\displaystyle\int\limits_{-2025}^{2025} f(x) \mathrm{d}x = \displaystyle\int\limits_{1}^{2025} (2x^2 + 3) \mathrm{d}x + \displaystyle\int\limits_{-2025}^{1} (2 - x^3) \mathrm{d}x$
\itemch Diện tích hình phẳng tạo bởi đồ thị hàm số $f(x)$, trục hoành và hai đường thẳng $x=-1$ và $x=2$ là $$S=\displaystyle\int\limits_{-1}^{1} |f(x)| \mathrm{d}x + \displaystyle\int\limits_{1}^{2} |f(x)| \mathrm{d}x = \displaystyle\int\limits_{-1}^{1} |2 - x^3| \mathrm{d}x + \displaystyle\int\limits_{1}^{2} |2x^2 + 3| \mathrm{d}x = \displaystyle\int\limits_{-1}^{1} (2 - x^3) \mathrm{d}x + \displaystyle\int\limits_{1}^{2} (2x^2 + 3) \mathrm{d}x$$
\end{itemchoice}
}

\end{ex}

\begin{ex}%[Mức độ 2]%[BG-12-New-4in1, Hiệp Hà]%[2H5H1-2]
Cho mặt phẳng $(\alpha)$ đi qua các điểm $M(1; -2; 3)$, $N(2; 1; -1)$, $P(0;-2;4)$. Mỗi khẳng định dưới đây đúng hay sai?
\choiceTF
{\True $\vec{MN}$, $\vec{PN}$ là cặp vectơ chỉ phương của mặt phẳng $(\alpha)$}
{$\vec{n_1}=(3;-1;2)$ là một vectơ pháp tuyến của mặt phẳng $(\alpha)$}
{\True Đường thẳng $d$ đi qua $M$, $N$ có phương trình $\heva{&x=1+t\\&y=-2+3t\\&z=3-4t}$}
{Mặt cầu $(S)$ có đường kính $NP$ có phương trình là $(x-1)^2+(y+2)^2+(z-4)^2=25$}
\loigiai{
\begin{itemchoice}
\itemch
$\vec{MN}=(1;3;-4)$, $\vec{PN}=(2;3;-5)$ là hai vectơ không cùng phương và có giá nằm trong $(\alpha)$.\\
Do đó, $\vec{MN}$, $\vec{PN}$ là cặp vectơ chỉ phương của mặt phẳng $(\alpha)$.
\itemch Ta có
\[
\begin{aligned}
\vec{n}=[\vec{MN}, \vec{PN}] & =\left(\left|\begin{array}{cc}
3 & -4 \\ 3 & -5
\end{array}\right| ;\left|\begin{array}{cc}
-4 & 1 \\ -5 & 2
\end{array}\right| ;\left|\begin{array}{cc}
1 & 3 \\ 2 & 3
\end{array}\right|\right) \\
& =(-3; -3; -3) .
\end{aligned}
\]
Do đó, $\vec{n_1}=(3;-1;2)$ không phải là vectơ pháp tuyến của mặt phẳng $(\alpha)$.
\itemch Đường thẳng $d$ đi qua $M$, $N$ nên có vectơ chỉ phương là $\vec{MN}=(1;3;-4)$ và có phương trình tham số là $\heva{&x=1+t\\&y=-2+3t\\&z=3-4t}$.
\itemch Mặt cầu $(S)$ có đường kính $NP$ có tâm là trung điểm của đoạn thẳng $NP$ là $I(1; -\dfrac{1}{2}; \dfrac{3}{2})$ và bán kính là $R=\dfrac{1}{2}NP=\dfrac{1}{2}\sqrt{38}$.\\
Vậy phương trình mặt cầu $(S)$ là $(x-1)^2+(y+\dfrac{1}{2})^2+(z-\dfrac{3}{2})^2=\dfrac{19}{2}$.
\end{itemchoice}
}
\end{ex}
\Closesolutionfile{ans}

\TNSA
\Opensolutionfile{ans}[ans/ansDe2-TN3]
\begin{ex}%[2D4H1-4]
Cho $F(x)$ là một nguyên hàm của hàm số $f(x)=x\sqrt{x}+\dfrac{1}{\sqrt{x}}$. Biết $F(1)=-2$. Tính $F(0)$.
\shortans{$-4{,}4$}
\loigiai{
Hàm số $f(x)=x\sqrt{x}+\dfrac{1}{\sqrt{x}}=x^{\tfrac{3}{2}}+x^{-\tfrac{1}{2}}$.\\
Có $F(x)=\displaystyle \int f(x)\mathrm{\,d}x=\displaystyle \int\left(x^{\tfrac{3}{2}}+x^{-\tfrac{1}{2}}\right)\mathrm{\,d}x=\dfrac{2}{5}x^{\tfrac{5}{2}}+2\sqrt{x}+C$.\\
Do $F(1)=-2\Rightarrow -2=\dfrac{2}{5}\cdot 1^{\tfrac{5}{2}}+2\sqrt{1} +C \Rightarrow C=-\dfrac{22}{5}$.\\
Suy ra $F(x)=\dfrac{2}{5}x^{\tfrac{5}{2}}+2\sqrt{x} -\dfrac{22}{5}$.\\
Vậy $F(0)=-4{,}4$.
}
\end{ex}

\begin{ex}%[2D4V2-6]
Một chất điểm $A$ xuất phát từ $O$, chuyển động thẳng với vận tốc biến thiên theo thời gian bởi quy luật $v \left(t\right) = \dfrac{1}{100}t^2 + \dfrac{13}{30}t$ (m/s), trong đó $t$ (giây) là khoảng thời gian tính từ lúc $A$ bắt đầu chuyển động. Từ trạng thái nghỉ, một chất điểm $B$ cũng xuất phát từ $O$, chuyển động thẳng cùng hướng với $A$ nhưng chậm hơn $10$ giây so với $A$ và có gia tốc bằng $a$ (m/s$^2$ ) ( $a$ là hằng số). Sau khi $B$ xuất phát được $15$ giây thì đuổi kịp $A$. Vận tốc của $B$ tại thời điểm đuổi kịp $A$ bằng bao nhiêu m/s?
\shortans{$25$}
\loigiai{
Ta có $v_{B}(t) = \displaystyle\int a \cdot \mathrm{\,d}t = at + C$, $v_{B} (0) = 0 \Rightarrow C = 0 \Rightarrow v_{B} \left(t\right) = at$.\\
Quãng đường chất điểm $A$ đi được trong $25$ giây là
\[S_{A} = \displaystyle\int\limits_0^{25} \left(\dfrac{1}{100}t^2 + \dfrac{13}{30}t \right) \mathrm{\,d}t = \left(\dfrac{1}{300}t^3 + \dfrac{13}{60}t^2 \right) \Big|_0^{25} = \dfrac{375}{2}.\]
Quãng đường chất điểm $B$ đi được trong $15$ giây là
\[S_{B} = \displaystyle\int\limits_0^{15} at \cdot \mathrm{\,d}t = \dfrac{at^2}{2} \Big|_0^{15} = \dfrac{225a}{2}.\]
Ta có $\dfrac{375}{2} = \dfrac{225a}{2} \Leftrightarrow a = \dfrac{5}{3}$.\\
Vận tốc của $B$ tại thời điểm đuổi kịp $A$ là $v_{B} \left(15\right) = \dfrac{5}{3} \cdot 15 = 25$ (m/s).
}
\end{ex}

\begin{ex}%[2H5C2-7]
	Cho biết kim tự tháp Memphis tại bang Tennessee (Mỹ) có dạng hình chóp tứ giác đều $S.ABCD$ với chiều cao $98$ m và cạnh đáy $180$ m. Số đo góc giữa hai mặt bên bằng bao nhiêu độ (làm tròn kết quả đến hàng đơn vị)?
	\shortans{$63$}
	\loigiai{
	\immini{
	Gọi $O=AC \cap BD$. Vì $S.ABCD$ là hình chóp đều nên $S O \perp(A B C D)$.\\
	Ta có $AC=BD=AB \sqrt{2}=180 \sqrt{2}$.\\
	Chọn hệ trục $Oxyz$ như hình vẽ với $O(0;0;0)$, $C(90 \sqrt{2} ; 0 ; 0)$, $D(0 ; 90 \sqrt{2} ; 0)$, $B(0;-900\sqrt{2};0)$ và $S(0 ; 0 ; 98)$.
	}
	{
	\begin{tikzpicture}[scale=0.55, font=\footnotesize, line join=round, line cap=round, >=stealth]
	\def\bc{5} % cạnh BC
	\def\ba{3.5} % cạnh BA
	\def\h{5} % đường cao
	\def\gocB{40} % góc B của đáy
	\coordinate (B) at (0,0);
	\coordinate (A) at (\gocB:\ba);
	\coordinate (C) at (\bc,0);
	\coordinate (D) at ($(C)-(B)+(A)$);
	\coordinate (O) at ($(A)!.5!(C)$);
	\coordinate (S) at ($(O)+(90:\h)$);
	\coordinate (z) at ($(O)+(90:\h+1)$);
	\coordinate (y) at ($(O)!1.25!(D)$);
	\coordinate (x) at ($(O)!1.5!(C)$);
	\draw (B)--(C)--(D)--(S)--cycle (S)--(C);
	\draw[dashed] (C)--(A)--(D)--(B) (O)--(S)--(A)--(B);
	\draw [->] (S)--(z);
	\draw [->] (D)--(y);
	\draw [->] (C)--(x);
	\foreach \p/\i in {S/180, A/180,B/-60,D/40,C/-120,O/170}
	\fill (\p) circle (1.5pt) node[shift={(\i:3mm)}]{$\p$};
	\foreach \p/\i in {z/180,y/-90,x/40}
	\fill (\p)  node[shift={(\i:3mm)}]{$\p$};
	\end{tikzpicture}}
	\noindent
	Phương trình mặt phẳng $(SBC)$ dưới dạng đoạn chắn là
	\[ \dfrac{x}{90\sqrt{2}}+\dfrac{y}{-90 \sqrt{2}}+\dfrac{z}{98}=1\qquad \text{hay } 49x-49y+45\sqrt{2} z-4410 \sqrt{2}=0.\]
	Phương trình mặt phẳng $(SCD)$ dưới dạng đoạn chắn là\\ $ \dfrac{x}{90 \sqrt{2}}+\dfrac{y}{90 \sqrt{2}}+\dfrac{z}{98}=1$ hay $49x+49y+45\sqrt{2} z-a \sqrt{2}=0$.\\
	Khi đó hai mặt phẳng $(SBC)$ và $(SCD)$ có véc-tơ pháp tuyến lần lượt là  $\overrightarrow{n}_1=(49 ; -49 ; 45\sqrt{2})$, $\overrightarrow{n}_2=(49 ; 49 ; 45\sqrt{2})$.\\
	Suy ra \[\cos ((S BC),(SCD))=\dfrac{|\overrightarrow{n}_1 \cdot \overrightarrow{n}_2|}{|\overrightarrow{n}_1| \cdot|\overrightarrow{n}_2|}=\dfrac{\left|49\cdot 49-49\cdot 49+45\sqrt{2}\cdot 45\sqrt{2} \right| }{\sqrt{49^2+(-49)^2+(45\sqrt{2})^2}\cdot \sqrt{49^2+49^2+(45\sqrt{2})^2}}=\dfrac{\sqrt{2025}}{4426}.\]
	Do đó $((SBC),(SCD))\approx 63^{\circ}$.
	}
	\end{ex}

	\begin{ex}%[2D6C1-4]
		Kết quả một cuộc khảo sát các vụ tai nạn giao thông ô tô về mối quan hệ giữa việc thắt dây an toàn của người lái xe khi xảy ra tai nạn giao thông và nguy cơ tử vong của người lái xe khi xảy ra tai nạn giao thông cho thấy:
		\begin{itemize}
		\item Tỉ lệ người lái xe tử vong khi xảy ra tai nạn giao thông là $0{,4}\%$.
		\item Tỉ lệ người lái xe không thắt dây an toàn giao thông khi xảy ra tai nạn giao thông là $28 \%$.
		\item Tỉ lệ người lái xe tử vong khi xảy ra tai nạn giao thông trong trường hợp không thắt dây an toàn là $0{,}3 \%$.
		\end{itemize}
		Hỏi theo kết quả khảo sát trên, việc thắt dây an toàn của người lái xe ô tô sẽ làm giảm khả năng tử vong là bao nhiêu lần? (làm tròn đến hàng phần mười).
		\shortans{$7{,}7$}
		\loigiai{
		Chọn ngẫu nhiên một một vụ tai nạn giao thông của cuộc khảo sát trên. Xét các biến cố:\\
		$A$: "Người lái xe đó tử vong khi xảy ra tai nạn giao thông."\\
		$B$: "Người lái xe đó không thắt dây an toàn khi xảy ra tai nạn giao thông."\\
		Ta có $P(A)= 0{,4}\%$; $P(B)= 28\%$; $P(A\cap B)=0{,}3\%$.\\
		Xác suất người lái xe đó tử vong khi xảy ra tai nạn giao thông trong trường hợp không thắt dây an toàn là
		\[P(A|B)=\dfrac{P(A\cap B)}{P(B)}=\dfrac{3}{280}.\]
		Xác suất người lái xe đó có thắt dây an toàn giao thông là $P(\overline{B})=72\%$.\\
		Xác suất người lái xe đó tử vong khi xảy ra tai nạn giao thông trong trường hợp có thắt dây an toàn là
		\[P(A|\overline{B})=\dfrac{P(A\cap \overline{B})}{P(\overline{B})}=\dfrac{P(A)-P(A\cap B)}{P(\overline{B}}=\dfrac{1}{720}.\]
		Ta có
		\[\dfrac{P(A|B)}{P(A|\overline{B})}=\dfrac{54}{7}\approx7{,}7.\]
		Vậy theo khảo sát trên, việc thắt dây an toàn của người lái xe ô tô sẽ làm giảm khả năng tử vong khoảng $7{,}7$ lần.
		}
		\end{ex}
\TL
		\begin{ex}%[2H5V2-4]
			Trong không gian với hệ trục tọa độ $O x y z$, cho điểm $M(3 ; 3 ;-2)$ và hai đường thẳng $d_1\colon \dfrac{x-1}{1}=\dfrac{y-2}{3}=\dfrac{z}{1} ;$ $ d_2\colon \dfrac{x+1}{-1}=\dfrac{y-1}{2}=\dfrac{z-2}{4}$. Viết phương trình đường thẳng $d$ đi qua $M$ vuông góc $d_1,$ $ d_2$.
			% \shortans{$3$}
			\loigiai{
			Ta có $d_1$ có véc-tơ chỉ phương là $\overrightarrow{u_1}=(1;3;1)$, $d_2$ có véc-tơ chỉ phương là $\overrightarrow{u_2}=(-1;2;4)$.\\
			Đường thẳng $d$ đi qua $M$ vuông góc với $d_1$, $d_2$ có véc-tơ chỉ phương là $\overrightarrow{u}=\dfrac15[\overrightarrow{u_1}, \overrightarrow{u_2}] = \left(2; -1; 1\right)$ nên có phương trình tham số là $\heva{&x=3+2t\\&y=3-t\\&z=-2+t}$.
			}
			\end{ex}

\begin{ex}%[2D6C2-4]
Tỉ lệ người dân đã tiêm vắc xin phòng bệnh A ở một địa phương là $65\%$. Trong số những  người đã tiêm phòng, tỉ lệ mắc bệnh A là $5\%$ còn trong số những người chưa tiêm, tỉ lệ mắc bệnh A là $17\%$. Gặp ngẫu nhiên một người ở địa phương đó. Biết rằng người đó mắc bệnh X. Khi đó xác suất người đó không tiêm vắc xin phòng bệnh X là bao nhiêu?
% \shortans{$65$}
\loigiai{
Gọi $A$ là biến cố \lq\lq người đó mắc bệnh $X$\rq\rq\,và $B$ là biến cố \lq\lq Gặp được người đã tiêm vắc xin phòng bệnh X\rq\rq.\\
Theo công thức xác suất toàn phần, ta có
\begin{eqnarray*}
\mathrm{P}(A) & = &\mathrm{P}(B) \cdot \mathrm{P}(A \mid B)+\mathrm{P}(\overline{B}) \cdot \mathrm{P}(A \mid \overline{B}) \\
& = &0,65 \cdot 0{,}05+0{,}35 \cdot 0{,}17=0{,}092.
\end{eqnarray*}
Suy ra
\begin{eqnarray*}
\mathrm{P}(\overline{B} \mid A) & =& \dfrac{\mathrm{P}(A \overline{B})}{\mathrm{P}(A)}=\dfrac{\mathrm{P}(\overline{B}) \mathrm{P}(A \mid \overline{B})}{\mathrm{P}(A)} \\
& =& \dfrac{0,35 \cdot 0{,}17}{0{,}092}=\dfrac{119}{184}.
\end{eqnarray*}
% Khi đó $a=119$ và $b=184$, suy ra $b-a=65$.
}
\end{ex}

\begin{ex}%[2H5C1-7]
Trong không gian $Oxyz$, cho hai điểm $A(2;1;0)$, $B(1;2;0)$ và điểm $M$ di động trên tia $Oz$. Gọi $H$, $K$ lần lượt là hình chiếu vuông góc của $A$ lên $OB$ và $MB$. Đường thẳng $HK$ cắt trục $Oz$ tại điểm $N$. Khi thể tích khối tứ diện $ABMN$ nhỏ nhất thì mặt phẳng $(AHK)$ có dạng $ax+by+cz-4=0$. Giá trị của $a+b+c$ bằng
\shortans{$1$}
\loigiai{
\immini
{Ta có $A(2;1;0)$, $B(1;2;0)$ $\Rightarrow A,~ B \in (Oxy)$.\\
Có $\heva{&AK\perp MB\\&AH\perp OB;~AH\perp OM \Rightarrow AH \perp (OBM) \Rightarrow AH \perp MB}$ mà $AK\perp MB$ nên
$ MB \perp (AHK)$.\\
Gọi $M(0;0;m)$, $(m>0)$ thuộc tia $Oz$ khi đó $\overrightarrow{MB}=(1;2;-m)$\\
$\Rightarrow (AHK)\colon 1(x-2)+2(y-1)-mz=0$.\\
$\Rightarrow N=HK \cap Oz =(AHK) \cap Oz \Rightarrow N\left(0;0;-\dfrac{4}{m}\right)$.\\
Ta có
\allowdisplaybreaks
$\begin{aligned}[t]
V_{ABMN}&=V_{M.OAB}+V_{N.OAB}=\dfrac{1}{3}S_{OAB}\cdot OM +\dfrac{1}{3}S_{OAB}\cdot ON\\
&=\dfrac{1}{3}S_{OAB}(OM+ON)=\dfrac{1}{3}S_{OAB}\cdot MN
\end{aligned}$\\
Vì $S_{OAB}$ không đổi nên $V_{ABMN}$ nhỏ nhất khi và chỉ khi $MN$ nhỏ nhất.\\
Ta có $MN=m+\dfrac{4}{m}\geq 2\sqrt{m\cdot \dfrac{4}{m}}=4$.\\
Dấu bằng xảy ra khi $m=\dfrac{4}{m}\Rightarrow m=2$.\\
Vậy $(AHK)\colon x+2y-2z-4=0 \Rightarrow a+b+c=1+2-2=1$.}
{\begin{tikzpicture}[scale=0.7, font=\footnotesize, line join=round, line cap=round,>=stealth]
\def\xmin{-2};\def\ymin{-3};\def\xmax{6};\def\ymax{5};
\coordinate (O) at (0,0);
\coordinate (B) at (2,-1);
\coordinate (A) at (3,0);
\coordinate (M) at (0,2);
\coordinate (K) at ($(M)+2/5*(B)-2/5*(M)$);
\coordinate (H) at ($(O)+1/4*(B)-1/4*(O)$);
\coordinate (N) at (intersection of M--O and K--H);
\draw (A)--(B)--(O)--(M)--cycle (M)--(B)--(N) (A)--(K)--(N)--(O);
\draw [dashed] (O)--(A)--(H) (A)--(N);
\draw[black] pic[draw, angle radius=2mm, angle eccentricity=1.5]{right angle=B--K--A};
\draw[black] pic[draw, angle radius=2mm, angle eccentricity=1.5]{right angle=B--H--A};
\foreach \t/\g in {O/180,B/-90,A/0,M/70,K/90,H/200,N/-90}{\draw[fill=white] (\t) circle (1pt) node[shift={(\g:7pt)},font=\scriptsize]{$\t$};}
\end{tikzpicture}}
}
\end{ex}
\Closesolutionfile{ans}


\Closesolutionfile{ansbook}

% \begin{name}
	{\tenchude}
	{TOÁN 12}
	{LỚP TOÁN THẦY PHÁT}
	{Thời gian: 90 phút - Không kể thời gian phát đề}
\end{name}
\Opensolutionfile{ansbook}[ans/ansbookDe3]
\TN
\Opensolutionfile{ans}[ans/ansDe3-TN1]
\begin{ex}%[Dự án 2025 - đề cấu trúc mới, Nguyễn Kiều Nhã Tú]%[2D4N1-1]
Mệnh đề nào dưới đây \textbf{sai}?
\choice
{$\displaystyle\int f'(x)\mathrm{\,d}x=f(x)+C$ với mọi hàm số $f(x)$ có đạo hàm trên $\mathbb{R}$}
{$\displaystyle\int[f(x)+g(x)]\mathrm{\,d}x=\displaystyle\int f(x) \mathrm{\,d}x+\displaystyle\int g(x)\mathrm{\,d}x$ với mọi hàm số $f(x)$, $g(x)$ có đạo hàm trên $\mathbb{R}$}
{\True $\displaystyle\int kf(x)\mathrm{\,d}x=k\displaystyle\int f(x) \mathrm{\,d}x$ với mọi hằng số $k$ và với mọi hàm số $f(x)$ có đạo hàm trên $\mathbb{R}$}
{$\displaystyle\int[f(x)-g(x)]\mathrm{\,d}x=\displaystyle\int f(x) \mathrm{\,d}x-\displaystyle\int g(x)\mathrm{\,d}x$ với mọi hàm số $f(x)$, $g(x)$ có đạo hàm trên $\mathbb{R}$}
\loigiai{
Theo tính chất của nguyên hàm, $\displaystyle\int kf(x)\mathrm{\,d}x=k \displaystyle\int f(x)\mathrm{\,d}x$ sai khi $k=0$.
}
\end{ex}

\begin{ex}%[2D4N1-2]
Họ nguyên hàm của hàm số $f(x)=x^3$ là
\choice
{$4x^4+C$}
{$3x^2+C$}
{$x^4+C$}
{\True $\dfrac{1}{4}x^4+C$}
\loigiai{
Ta có
$\displaystyle\int x^3\mathrm{\,d}x=\dfrac{1}{4}x^4+C$.
}
\end{ex}

\begin{ex}%[2D4N2-1]
Cho $\displaystyle\int\limits_0^1f(x)\mathrm{\,d}x=2$ và $\displaystyle\int\limits_0^1g(x)\mathrm{\,d}x=5$, khi $\displaystyle\int\limits_0^1\left[f(x)-2g(x)\right]\mathrm{\,d}x$ bằng
\choice
{\True $-8$}
{$1$}
{$-3$}
{$12$}
\loigiai{
Ta có $\displaystyle\int\limits_0^1\left[f(x)-2g(x)\right]\mathrm{\,d}x=\displaystyle\int\limits_0^1f(x)\mathrm{\,d}x-2\displaystyle\int\limits_0^1g(x)\mathrm{\,d}x=2-2\cdot 5=-8$.}
\end{ex}

\begin{ex}%[2H5N1-1]
Trong không gian với hệ tọa độ $O x y z$, phương trình nào sau đây là phương trình của mặt phẳng $O z x$ ?
\choice
{$x=0$}
{$y-1=0$}
{\True $y=0$}
{$z=0$}
\loigiai{
Ta có mặt phẳng $(Oxz)$ đi qua điểm $O(0 ; 0 ; 0)$ và vuông góc với trục $O y$ nên có VTPT $\vec{n}=(0 ; 1 ; 0)$.\\
Do đó phương trình của mặt phẳng $(Oxz)$ là $y=0$.
}
\end{ex}

\begin{ex}%[Mức độ 1]%[BG-12-New-4in1, Hiệp Hà]%[2H5N1-2]
Vectơ nào dưới đây là một vectơ pháp tuyến của $(Oyz)$?
\choice
{\True $\vec{n_1}=(2;0;0)$}
{$\vec{n_2}=(1;1;0)$}
{$\vec{n_3}=(0;3;0)$}
{$\vec{n_4}=(0;0;-1)$}
\loigiai{
Ta có $Ox \perp (Oyz)$ nên $\vec{i}=(1;0;0)$ là một vectơ pháp tuyến của $(\alpha)$.\\
Khi đó, $\vec{n_1}=2\vec{i}$ là một vectơ pháp tuyến của $(\alpha)$.
}
\end{ex}

\begin{ex}%[2H5N2-1]%[Dự án EX-TF-TLN lần 3 - Nguyễn Thắng]
Trong không gian $Oxyz$, cho đường thẳng $\Delta$ đi qua điểm $A(x_0;y_0;z_0)$ và có véc-tơ chỉ phương $\vec{u}=(a;b;c)\ne \vec{0}$. Khi đó hệ phương trình nào sau đây là phương trình tham số của đường thẳng $\Delta$?
\choice
{$\heva{&x=x_0-at\\&y=y_0+bt\\&z=z_0+ct}$}
{$\heva{&x=x_0+at\\&y=y_0-bt\\&z=z_0+ct}$}
{$\heva{&x=x_0-at\\&y=y_0+bt\\&z=z_0-ct}$}
{\True $\heva{&x=x_0+at\\&y=y_0+bt\\&z=z_0+ct}$}
\loigiai{

}
\end{ex}

\begin{ex}%[2H5N2-2]%[Dự án 2025 - Đề cấu trúc mới của Bộ theo [Thành Đức Trung]
Trong không gian $Oxyz$, cho đường thẳng $d\colon \heva{&x=1-t\\&y=-2+2t\\&z=1+t}$. Véc-tơ nào dưới đây là véc-tơ chỉ phương của $d$?
\choice
{$\overrightarrow{u}=(1;-2;1)$}
{$\overrightarrow{u}=(1;2;1)$}
{$\overrightarrow{u}=(-1;-2;1)$}
{\True $\overrightarrow{u}=(-1;2;1)$}
\loigiai
{
Ta có véc-tơ chỉ phương của $d$ là $\overrightarrow{u}=(-1;2;1)$.
}
\end{ex}

\begin{ex}%[2H5N3-2]
Khối cầu $(S)$ có bán kính $R$ có thể tích bằng
\choice
{$4\pi{R^2}$}
{\True $\dfrac{4}{3}\pi{R^3}$}
{$\dfrac{1}{3}\pi{R^3}$}
{$\pi{R^3}$}
\loigiai{
Thể tích khối cầu được tính bằng công thức $V=\dfrac{4}{3}\pi{R^3}$.}
\end{ex}

\begin{ex}%[2D6N1-1]
Cho hai biến cố độc lập $A$, $B$. Chọn khẳng định \textbf{sai} trong các khẳng định sau.
\choice
{$\mathrm{P}(A\cap B)=\mathrm{P}(A)\cdot \mathrm{P}(B)$}
{$\mathrm{P}(A\mid B)=\mathrm{P}(A)$}
{\True $\mathrm{P}(A\mid \overline{B})=\mathrm{P}(\overline{B})$}
{ $\mathrm{P}(B\mid A)=\mathrm{P}(B)$}
\loigiai{
$\mathrm{P}(A\mid \overline{B})=\mathrm{P}(A)$.
}
\end{ex}

\begin{ex}%[2D6N2-1]%[Lê Công Trường]
Biến cố $A_0$ và $A_1$ là hai biến cố ngẫu nhiên thoả mãn $\mathrm{P}(A_0)>0$ và $0<\mathrm{P}(A_1)<1$. Khi đó công thức Bayes là
\choice
{$\mathrm{P}(A_0\mid  A_1)=\dfrac{\mathrm{P}(A_1)\cdot\mathrm{P}(A_0\mid A_1)}{\mathrm{P}(A_1)\cdot\mathrm{P}(A_0\mid A_1)+\mathrm{P}(\overline{A_1})\cdot\mathrm{P}(A_0\mid \overline{A_0})}$}
{\True $\mathrm{P}(A_1\mid A_0)=\dfrac{\mathrm{P}(A_1)\cdot\mathrm{P}(A_0\mid A_1)}{\mathrm{P}(A_1)\cdot\mathrm{P}(A_0\mid A_1)+\mathrm{P}(\overline{A_1})\cdot\mathrm{P}(A_0\mid \overline{A_1})}$}
{$\mathrm{P}(A_1\mid A)=\dfrac{\mathrm{P}(A_1)\cdot\mathrm{P}(A_0\mid A_1)}{\mathrm{P}(A_1)\cdot\mathrm{P}(A_0\mid A_1)+\mathrm{P}(\overline{A_1})\cdot\mathrm{P}(A_0\mid \overline{A_0})}$}
{$\mathrm{P}(A_1\mid A_0)=\dfrac{\mathrm{P}(A_0)\cdot\mathrm{P}(A_0\mid A_1)}{\mathrm{P}(A_1)\cdot\mathrm{P}(A_0\mid A_1)+\mathrm{P}(\overline{A_0})\cdot\mathrm{P}(A_0\mid \overline{A_0})}$}
\loigiai{ Biến cố $A_0$ và $A_1$ là hai biến cố ngẫu nhiên thoả mãn $\mathrm{P}(A_0)>0$ và $0<\mathrm{P}(A_1)<1$. Khi đó công thức Bayes là \[\mathrm{P}(A_1\mid A_0)=\dfrac{\mathrm{P}(A_1)\cdot\mathrm{P}(A_0\mid A_1)}{\mathrm{P}(A_1)\cdot\mathrm{P}(A_0\mid A_1)+\mathrm{P}(\overline{A_1})\cdot\mathrm{P}(A_0\mid \overline{A_1})}.\] }
\end{ex}

\begin{ex}%[2D6N2-3]%[Dự án EX-TF-TLN lần 4 - Quan Ón]
Cho hai biến cố $A$, $B$ thoả mãn $\mathrm{P}(A) = 0{,}4$; $\mathrm{P}(B) = 0{,}3$ ; $\mathrm{P}(A\mid B) = 0{,}25$. Khi đó, $\mathrm{P}(B\mid A)$ bằng
\choice
{$0{,}6667$}
{\True $0{,}1875$}
{$0{,}3195$}
{$0{,}5920$}
\loigiai{
Áp dụng công thức Bayes, ta có $\mathrm{P}(B\mid A) = \dfrac{\mathrm{P}(B)\cdot\mathrm{P}(A\mid B)}{\mathrm{P}(A)} = \dfrac{0{,}3\cdot 0{,}25}{0{,}4} = 0{,}1875$.
}
\end{ex}

\begin{ex}%[2H5N3-3]
Trong không gian Oxyz , cho mặt cầu $\left(S\right)$ có tâm $I\left(-1;2;3\right)$ và tiếp xúc với mặt phẳng $(P):2x-y-2z+1=0$. Phương trình của $\left(S\right)$ là
\choice
{$\left(x+1\right)^{2}+\left(y-2\right)^{2}+\left(z-3\right)^{2}=3$}
{$\left(x{-}1\right)^{2}+\left(y+2\right)^{2}+\left(z+3\right)^{2}=9$}
{\True$\left(x+1\right)^{2}+\left(y-2\right)^{2}+\left(z-3\right)^{2}=9$}
{$\left(x{-}1\right)^{2}+\left(y+2\right)^{2}+\left(z+3\right)^{2}=3$}
\loigiai
{
Bán kính mặt cầu $r=\mathrm{d}\left(I,\left(P\right)\right)=\dfrac{\left|2\left(-1\right)-2-2.3+1\right|}{\sqrt{2^{2}+\left(-1\right)^{2}+\left(-2\right)^{2}}}=3.$\\
Phương trình mặt cầu là $\left(x+1\right)^{2}+\left(y-2\right)^{2}+\left(z-3\right)^{2}=9.$
}
\end{ex}
\Closesolutionfile{ans}

\TNTF
\Opensolutionfile{ans}[ans/ansDe3-TN2]
\begin{ex}
    Cho hàm số $f(x)=3x^2-\dfrac{2}{x}$.
    \choiceTF
    {\True $\displaystyle\int f(x) \mathrm{\,d}x= x^3-2\ln |x|+C$}
    {Hàm số $G(x)=x^3-2\ln|2x|$ là một nguyên hàm của $f(x)$}
    {\True $\displaystyle\int_{1}^{2} f(x) \mathrm{\,d}x=a-\ln b$ với $a$, $b \in \mathbb{Z}$ thoả $a+b=7$}
    {\True Thể tích khối tròn xoay tạo thành khi quay hình phẳng giới hạn bởi đồ thị hàm số $y=f(x)$, trục hoành và các đường thẳng $x=1$ và $x=2$ bằng $\dfrac{199\pi}{5}$}
\loigiai{ }
\end{ex}
\begin{ex}%[2H5H1-2]
Trong không gian $O x y z$, cho mặt phẳng $(P)\colon 2 x+3 y+z-5=0$. Các mệnh đề sau đây đúng hay sai?
\choiceTF
{\True Mặt phẳng $(P)$ có một vectơ pháp tuyến là $\overrightarrow{n}=(2; 3; 1)$}
{Phương trình mặt phẳng $(Q)$ đi qua $A(-3;1;2)$ và song song với mặt phẳng $(P)$ là $2x+3y+z=0$}
{Đường thẳng $d$ có phương trình tham số $\heva{&x=1+2t\\&y=3t\\&z=3+t}$ song song với mặt phẳng $(P)$}
{Mặt cầu $(S)$ có phương trình $x^2+y^2+z^2-2x-3y-5=0$ có tâm nằm trên mặt phẳng $(P)$}
\loigiai{
}
\end{ex}
\Closesolutionfile{ans}

\TNSA
\Opensolutionfile{ans}[ans/ansDe3-TN3]
\begin{ex}%[2D4H1-4]
Hàm số $f(x)$ có đạo hàm liên tục trên $\mathbb{R}$ và $f'(x)=1+\mathrm{e}^{2x}$, $\forall x$; $f(0)=2$. Tính giá trị của $f(2)$. (Làm tròn đến số thập phân thứ nhất)
\shortans{$30{,}8$}
\loigiai{
Hàm số $f(x)=\displaystyle \int (1+\mathrm{e}^{2x})\mathrm{\,d}x=x+\dfrac{1}{2}\mathrm{e}^{2x}+C$.\\
Do $f(0)=2\Leftrightarrow 2=\dfrac{1}{2}+C\Leftrightarrow C=\dfrac{3}{2}$.\\
Suy ra $f(x)=x+\dfrac{1}{2}\mathrm{e}^{2x}+\dfrac{3}{2}$.\\
Vậy $f(2)=30{,}8$.
}
\end{ex}

\begin{ex}%[2D4V2-6]
Một vật chuyển động trong $4$ giờ với vận tốc $v$ (km/h) phụ thuộc thời gian $t$ (h) có đồ thị của vận tốc như hình bên. Trong khoảng thời gian $3$ giờ kể từ khi bắt đầu chuyển động, đồ thị đó là một phần của đường parabol có đỉnh $I \left(2;9\right)$ với trục đối xứng song song với trục tung, khoảng thời gian còn lại đồ thị là một đoạn thẳng song song với trục hoành. Tính quãng đường $s$ mà vật di chuyển được trong $4$ giờ đó (đơn vị tính bằng km).
\begin{center}
\begin{tikzpicture}[>=stealth,scale=0.45]
% Vẽ 2 trục, điền các số lên trục
\draw[->] (-0.5,0)--(0,0) node[below left]{$O$}--(5,0) node[above]{$t$};
\foreach \x in {2,3,4}
\draw[shift={(\x,0)},color=black] (0pt,2pt)--(0pt,-2pt)
node[below] { $\x$};
\draw[->,color=black] (0,-0.5)--(0,10) node[right]{$v$};
\foreach \y in {9}
\draw[shift={(0,\y)},color=black] (2pt,0pt) -- (-2pt,0pt)
node[left] {$\y$};
\clip(-1,-1) rectangle (5,10); %vùng đồ thị
%\draw[gray!50,thin,opacity=.5] (-1,-1) grid (4,10); %ô vuông
%Vẽ đồ thị
\draw[smooth,samples=100,domain=0:3, font=\footnotesize, line join=round, line cap=round, thick, smooth]
plot(\x,{(-9/4)*(\x)^2+9*(\x)});
\draw[smooth,samples=100, font=\footnotesize, line join=round, line cap=round, thick, smooth,domain=3:4]
plot(\x,{27/4});
% Vẽ thêm mấy cái râu ria
\draw[dashed] (3,0)--(3,27/4) circle(1.5pt);  \draw[dashed] (2,0)--(2,9) circle(1.5pt) node[above]{$I$}--(0,9) circle(1.5pt); \draw[dashed] (4,0)--(4,27/4) circle(1.5pt);
%Vẽ dấu chấm tròn
\fill (0cm,0cm) circle (1.5pt);
\end{tikzpicture}
\end{center}
\shortans{$27$}
\loigiai{
Gọi $\left(P\right) \colon y = ax^2+bx+c$.\\
Vì $\left(P\right)$ qua $O\left(0;0\right)$ và có đỉnh $I\left(2;9\right)$ nên dễ tìm được phương trình là $y = \dfrac{-9}{4}x^2 + 9x$.\\
Ngoài ra tại $x=3$ ta có $y = \dfrac{27}{4}$.\\
Vậy quãng đường cần tìm là: $S = \displaystyle\int\limits_0^3 \left(\dfrac{-9}{4}x^2 +9x \right) \mathrm{\,d}x + \displaystyle\int\limits_3^4 \dfrac{27}{4} \mathrm{\,d}x = 27$ (km).
}
\end{ex}

\begin{ex}%[EX-Ôn Tập TN 2025,  Lê Hoàng Anh]%[2H5V2-8]
    Trong không gian $Oxyz$, đài kiểm soát không lưu sân bay có toạ độ $O(0;0;0)$, đơn vị trên mỗi trục tính theo kilômét. Một máy bay chuyển động hướng về đài kiểm soát không lưu, bay qua hai vị trí $A(-500;-250;150)$, $B(-200;-200;100)$. Khi máy bay ở gần đài kiểm soát nhất, toạ độ của vị trí máy bay là $(a;b;c)$. Biết rằng $a-b+c=\dfrac{m}{n}$, tính giá trị biểu thức $m-n$?
    \shortans{$4\,881$}
    \loigiai{
    Véc-tơ $\overrightarrow{AB}=(300;50;-50)$ nên $\overrightarrow{u}=(6;1;-1)$ là một véc-tơ chỉ phương của đường thẳng $AB$.\\
    Phương trình đường thẳng $AB$ là
    \[
    \dfrac{x+500}{6}=\dfrac{y+250}{1}=\dfrac{z-150}{-1}.
    \]
    Gọi $H$ là hình chiếu của điểm $O$ trên đường thẳng $AB$ thì $OH$ là khoảng cách ngắn nhất giữa máy bay và đài kiểm soát. Khi đó $H(6t-500;t-250;-t+150)$.\\
    Ta có $\overrightarrow{OH} \cdot \overrightarrow{u}=(6t-500) \cdot 6+t-250-(-t+150)=0 \Leftrightarrow t=\dfrac{1\,700}{19}$.\\
    Suy ra toạ độ của vị trí máy bay khi đó là $\left(\dfrac{700}{19};\dfrac{-3\,050}{19};\dfrac{1\,150}{19}\right)$.\\ Vậy $a-b+c= \dfrac{4\,900}{19}$, suy ra $m-n=4\,881$.
    }
    \end{ex}

\begin{ex}%[2D6V1-3]
Tất cả các học sinh của trường Hạnh Phúc đều  tham gia câu lạc bộ bóng chuyền hoặc bóng rổ, mỗi học sinh chỉ tham gia đúng $1$ câu lạc bộ. Có $60$\% học sinh của trường tham gia câu lạc bộ bóng chuyền và $40$\% học sinh của trường tham gia câu lạc bộ bóng rổ. Số học sinh nữ chiếm $65$\% trong câu lạc bộ bóng chuyền và $25$\% trong câu lạc bộ bóng rổ. Chọn ngẫu nhiên $1$ học sinh. Xác suất chọn được học sinh nữ là bao nhiêu?

\shortans{0,49}
\loigiai
{
Gọi $A$ là biến cố \textquotedblleft Số học sinh thuộc câu lạc bộ bóng chuyền\textquotedblright.\\
$B$ là biến cố \textquotedblleft Số học sinh nữ\textquotedblright.\\
Khi đó $\heva{&P(A) = 60\% = 0{,}6 &\Rightarrow& P(\overline{A}) = 1-0{,}6 = 0{,}4\\ &P(B|A) = 65\% = 0{,}65 &\Rightarrow& P(\overline{B}|A) = 1-0{,}65 = 0{,}35\\ &P(B|\overline{A}) = 25\% = 0{,}25 &\Rightarrow& P(\overline{B}|\overline{A}) = 1-0{,}25=0{,}75.}$
\begin{center}
\begin{tikzpicture}[>=stealth]
%Khung 1
\draw (-0,-1) rectangle (2.2,0);
\draw (1.1,-0.5) node{Gốc};
%Mui ten 1,2
\draw [->] (2.2,-0.5)--(3.8,1.6) node[pos=0.5,sloped,above]{$0{,}6$};
\draw [->] (2.2,-0.5)--(3.8,-2.6) node[pos=0.5,sloped,below]{\color{red}$0{,}4$};
%Khung 2.1
\draw (4.5,3.0) node{\textbf{Thuộc câu lạc bộ}};
\draw (3.8,1.1) rectangle (5.1,2.1);
\draw (8.9/2,1.6) node{$A$} ;
%Khung 2.2
\draw (3.8,-2.1) rectangle (5.1,-3.1);
\draw (8.9/2,-2.6) node{$\overline{A}$};
%Mui ten 3,4
\draw [->] (5.1,1.6)--(6.5,2.6) node[pos=0.5,sloped,above]{$0{,}65$};
\draw [->] (5.1,1.6)--(6.5,0.6) node[pos=0.5,sloped,below]{\color{red}$0{,}35$};
%Mui ten 5,6
\draw [->] (5.1,-2.6)--(6.5,-1.6) node[pos=0.5,sloped,above]{$0{,}25$};
\draw [->] (5.1,-2.6)--(6.5,-3.6) node[pos=0.5,sloped,below]{\color{red}$0{,}75$};
%Khung 3.1
\draw (6.5,2.2) rectangle (7.7,3.2);
\draw (7.1,5.4/2) node{$B$} ;
%Khung 3.2
\draw (7.0,3.7) node{\textbf{Nữ}};
\draw (6.5,1.2) rectangle (7.7,0.2);
\draw (7.1,1.4/2) node{$\overline{B}$} ;
%Khung 3.3
\draw (6.5,-1.1) rectangle (7.7,-2.1);
\draw (7.1,-3.2/2) node{$B$} ;
%Khung 3.3
\draw (6.5,-2.9) rectangle (7.7,-3.9);
\draw (7.1,-3.4) node{$\overline{B}$} ;
%Kết quả
\draw (9.5,3.7) node{\textbf{Kết quả}};
\draw (9.5,2.7) node{$AB$};
\draw (9.5,0.7) node{$A \overline{B}$};
\draw (9.5,-1.6) node{$\overline{A}B$};
\draw (9.5,-3.4) node{$\overline{A}~\overline{B}$};
%Xác suất
\draw (12.5,3.7) node{\textbf{Xác suất}};
\draw (12.5,2.7) node{$0{,}39$};
\draw (12.5,0.7) node{$0{,}21$};
\draw (12.5,-1.6) node{$0{,}1$};
\draw (12.5,-3.4) node{$0{,}3$};
\end{tikzpicture}
\end{center}
Áp dụng công thức xác suất toàn phần để tính xác chọn được học sinh là nữ
\[P(B) = P(A) \cdot P(B|A) + P(\overline{A}) \cdot P(B|\overline{A}) = 0{,}6 \cdot 0{,}65 + 0{,}4 \cdot 0{,}25 = 0{,}49.\]
}
\end{ex}
\TL
\begin{ex}%[2H5H2-3]%[Dự án EX-TF-TLN lần 3 - Nguyen Chín Em]
Trong không gian với hệ tọa độ $Oxyz$. Viết phương trình tham số của $d$ biết $d$ đi qua điểm $M(3; 1; 5)$ và song song với hai mặt phẳng $(P)\colon 2x+3y-2z+1=0$ và $(Q)\colon x-3y+z-2=0$. %Khi đó đường thẳng $d$ đi qua $(6;y_0;z_0)$. Tính $z_0-y_0$.
% \shortans{$9$}
\loigiai{
Ta có  $\overrightarrow{n_P} = (2; 3; -2)$, $\overrightarrow{n_Q} = (1; -3; 1)$ lần lượt là véc-tơ pháp tuyến của hai mặt phẳng $(P)$ và $(Q)$. Do $d \parallel (P)$ và $d \parallel (Q)$ nên véc-tơ chỉ phương của $d$ là $\overrightarrow{u}=\left[\overrightarrow{n_P}, \overrightarrow{n_Q}\right]= (-3; -4; -9)$.\\
Phương trình tham số của $d$ là $\heva{&x = 3 - 3t\\&y = 1 - 4t\\&z = 5 - 9t}, (t\in\mathbb{R})$.\\
Với $t=-1$, suy ra đường thẳng $d$ đi qua $(6;5;14)$. Suy ra được $y_0=5;z_0=14\Rightarrow z_0-y_0=9$.
}
\end{ex}

\begin{ex}%[2D6C2-4]
Ở một khu rừng nọ có $7$ chú lùn, trong đó có $4$ chú luôn nói thật, $3$ chú còn lại nói thật với  xác suất $0{,}5$. Bạn Tuyết gặp ngẫu nhiên một chú lùn. Gọi $A$ là biến cố \lq\lq Chú lùn đó luôn nói thật\rq\rq\,và $B$ là biến cố \lq\lq Chú lùn đó tự nhận mình luôn nói thật\rq\rq. Biết rằng chú lùn mà bạn Tuyết gặp tự nhận mình là người luôn nói thật. Tính xác suất để chú lùn đó luôn nói thật (làm tròn hai chữ số thập phân).
% \shortans{$0{,}73$}
\loigiai{
Ta có $\mathrm{P}(A)=\dfrac{4}{7}$; $\mathrm{P}(\overline{A})=\dfrac{3}{7}$; $\mathrm{P}(B \mid A)=1$; $\mathrm{P}(B \mid \overline{A})=0{,}5$.\\
Theo công thức xác suất toàn phần, ta có
\begin{eqnarray*}
\mathrm{P}(B) & =& \mathrm{P}(A) \cdot \mathrm{P}(B \mid A)+\mathrm{P}(\overline{A}) \cdot \mathrm{P}(B \mid \overline{A}) \\
& = & \dfrac{4}{7} \cdot 1+\dfrac{3}{7} \cdot 0{,}5=\dfrac{11}{14}.
\end{eqnarray*}
Khi đó
\[
\mathrm{P}(A \mid B)=\dfrac{\mathrm{P}(AB)}{\mathrm{P}(B)}=\dfrac{\mathrm{P}(A) \cdot \mathrm{P}(B \mid A)}{\mathrm{P}(B)}=\dfrac{\dfrac{4}{7} \cdot 1}{\dfrac{11}{14}}=\dfrac{8}{11} \approx 0{,}73.
\]
}
\end{ex}

\begin{ex}%[2H5C1-7]
Người ta thiết kế một mái che hình chữ nhật $ ABCD $ phía trên sân khấu. Gắn hệ trục tọa độ $ Oxyz $ (đơn vị trên trục là mét), người ta xác định được toạ dộ của các điểm như sau: $ A(0;0;8)$, $B(0;20;8)$, $D(15;0;14)$, $C(15;20;14) $. Một cổng chào hình chữ nhật $ EFHG $ với tọa độ điểm $ G(8;0;4) $ dựng vuông góc với mặt đất. Người ta muốn làm các đoạn dây nối thanh ngang $ GE $ với mái che để gắn hoa và đèn led. Độ dài ngắn nhất của mỗi đoạn dây này bằng bao nhiêu mét? (làm tròn đến chữ số thập phân thứ nhất)
\begin{center}
\includegraphics[scale=.7]{images/2P5-1-H5-16}
\end{center}
% \shortans{$1{,}8$}
\loigiai{
Ta có $ A(0;0;8)$, $B(0;20;8)$, $D(15;0;14)$, $C(15;20;14) $.\\
Ta có $ \vec{AB}=(0;20;0) $, $\vec{AC}=(15;20;6)$ nên $ \vec{n}_1=\left[\vec{AB},\vec{AC}\right]=(80;0;300) $ là vectơ pháp tuyến của $ (ABCD) $.\\
Mà mặt phẳng mái che $ (ABCD) $ qua $ A(0;0;8)$ nên có phương trình
\[ 80(x-0)+0(y-0)+300(z-8)=0\Leftrightarrow 4x+15z-120=0 .\]
Độ dài ngắn nhất của dây nối thanh ngang $ GE $ với mái che là khoảng cách từ $ G $ đến mái che (mặt phẳng $ ABCD $) là \[ \mathrm{d}(G,(ABCD))=\dfrac{|4\cdot8+0+15\cdot4-120|}{\sqrt{4^2+0^2+15^2}}=\dfrac{28}{\sqrt{241}}=1{,}8\ (\text{m}). \]
}
\end{ex}
\Closesolutionfile{ans}
\Closesolutionfile{ansbook}

% \begin{name}
	{\tenchude}
	{TOÁN 12}
	{LỚP TOÁN THẦY PHÁT}
	{Thời gian: 90 phút - Không kể thời gian phát đề}
\end{name}
\Opensolutionfile{ansbook}[ans/ansbookDe4]
\TN
\Opensolutionfile{ans}[ans/ansDe4-TN1]
\begin{ex}%[2D4N1-1]
Cho hai hàm số $f(x)$, $g(x)$ là hàm số liên tục, có $F(x)$, $G(x)$ lần lượt là nguyên hàm của $f(x)$, $g(x)$. Xét các mệnh đề sau.
\begin{itemize}
\item[(i)] $F(x)+G(x)$ là một nguyên hàm của $f(x)+g(x)$.
\item[(ii)] $k\cdot F(x)$ là một nguyên hàm của $k\cdot f(x)$ với $k\in\mathbb{R}$.
\item[(iii)] $F(x)\cdot G(x)$ là một nguyên hàm của $f(x)\cdot g(x)$.
\end{itemize}
Các mệnh đề đúng là
\choice
{(ii) và (iii)}
{(i), (ii) và (iii)}
{(i) và (iii)}
{\True (i)}
\loigiai{
Theo tính chất của nguyên hàm, khẳng định (ii) sai khi $ k=0$, khẳng định (iii) sai.}
\end{ex}

\begin{ex}%[2D4N1-2]
Họ nguyên hàm của hàm số $y=x^2-3x+\dfrac{1}{x}$ là
\choice
{$\dfrac{x^3}{3}-\dfrac{3x^2}{2}-\ln\left|x\right|+C$}
{$\dfrac{x^3}{3}-\dfrac{3x^2}{2}+\ln x+C$}
{\True $\dfrac{x^3}{3}-\dfrac{3x^2}{2}+\ln\left|x\right|+C$}
{$\dfrac{x^3}{3}-\dfrac{3x^2}{2}+\dfrac{1}{x^2}+C$}
\loigiai{
Ta có
$\displaystyle\int \left(x^2-3x+\dfrac{1}{x}\right) \mathrm{\,d}x=\dfrac{x^3}{3}-\dfrac{3x^2}{2}+\ln\left|x\right|+C$.

}
\end{ex}

\begin{ex}%[2D4N2-1]Biên soạn:Phùng Hoàng Em Phản biện:Nguyễn Đắc Kiên
Cho hàm số $f(x)$ liên tục trên $\mathbb{R}$ và $a$ là số thực dương. Khẳng định nào sau đây là khẳng định đúng?
\choice
{$\displaystyle\int\limits_a^a f(x) \mathrm{\,d}x=1$}
{$\displaystyle\int\limits_a^af(x) \mathrm{\,d}x=a^2$}
{\True $\displaystyle\int\limits_a^af(x) \mathrm{\,d}x=0$}
{$\displaystyle\int\limits_a^a f(x) \mathrm{\,d}x=2a$}
\loigiai{
Gọi $F(x)$ là một nguyên hàm của $f(x)$.
Ta có $\displaystyle\int\limits_a^af(x) \mathrm{\,d}x=F(x)\Bigg|_a^a=F(a)-F(a)=0$.
}
\end{ex}

\begin{ex}%[2H5N1-1]%[Dự án Khối 12- Ex-TF-TLN-2024]%[VU Ngoc Hao]
\immini{
Cho hình hộp chữ nhật $ABCD.A'B'C'D'$. Bốn véc-tơ pháp tuyến  của mặt phẳng $\left(AA'B'B\right)$ là
\choice
{$\overrightarrow{AD}$, $\overrightarrow{A'D'}$, $ \overrightarrow{BD}$, $\overrightarrow{B'C'}$}
{$\overrightarrow{AD}$, $\overrightarrow{A'D'}$, $ \overrightarrow{BC}$, $\overrightarrow{BC'}$}
{$\overrightarrow{AC}$, $\overrightarrow{A'D'}$, $ \overrightarrow{BC}$, $\overrightarrow{B'C'}$}
{\True  $\overrightarrow{AD}$, $\overrightarrow{A'D'}$, $ \overrightarrow{BC}$, $\overrightarrow{B'C'}$}
}
{
\begin{tikzpicture}[scale=0.5, font=\footnotesize,line join=round, line cap=round, >=stealth]
\coordinate (A) at (0,0);
\coordinate (B) at (-2,-1.5);
\coordinate (D) at (5,0);
\coordinate (C) at ($(B)+(D)-(A)$);
\foreach \i in {A,B,C,D}{\coordinate (\i') at ($(\i)+(0,4)$);}
\draw (A')--(B')--(C')--(D')--cycle;
\draw (B)--(B') (C)--(C') (D)--(D')  (B)--(C)--(D);
\draw[dashed,thin](B)--(A)--(A') (A)--(D);
\foreach \i/\g in {A'/90,B'/90,C'/90,D'/90,A/-90,B/-90,C/-90,D/-90}{\draw[fill=black](\i) circle (1pt) ($(\i)+(\g:5mm)$) node[scale=1]{$\i$};}
\end{tikzpicture}
}
\loigiai{
Bốn véc-tơ pháp tuyến của mặt phẳng $\left(AA'B'B\right)$ là  $\overrightarrow{AD}$, $\overrightarrow{A'D'}$, $ \overrightarrow{BC}$, $\overrightarrow{B'C'}$.
}
\end{ex}

\begin{ex}%[Mức độ 1]%[BG-12-New-4in1, Hiệp Hà]%[2H5N1-2]
Cho $(\alpha)$ vuông góc với giá của $\vec{a}=(2;-1;3)$. Vectơ nào dưới đây là vectơ pháp tuyến của $(\alpha)$?
\choice
{$\vec{n_1}=(-2;1;3)$}
{\True $\vec{n_2}=(-2;1;-3)$}
{$\vec{n_3}=(4;2;6)$}
{$\vec{n_4}=(4;-2;-6)$}
\loigiai{
$(\alpha)$ vuông góc với giá của $\vec{a}=(2;-1;3)$ nên $\vec{a}$ là một vectơ pháp tuyến của $(\alpha)$.\\
Do đó $\vec{n_2}=-\vec{a}$ cũng là một vectơ pháp tuyến của $(\alpha)$.
}
\end{ex}

\begin{ex}%[2H5N2-1]%[Dự án EX-TF-TLN lần 3 - Nguyễn Thắng]
Trong không gian $Oxyz$, cho đường thẳng $\Delta$ đi qua điểm $M(x_0;y_0;z_0)$ và có véc-tơ chỉ phương $\vec{u}=(a;b;c)$ và $abc\ne 0$. Khi đó hệ phương trình nào sau đây là phương trình chính tắc của đường thẳng $\Delta$?
\choice
{$\dfrac{x-x_0}{a}=\dfrac{y+y_0}{b}=\dfrac{z+z_0}{c}$}
{$\dfrac{x+x_0}{a}=\dfrac{y+y_0}{b}=\dfrac{z+z_0}{c}$}
{$\dfrac{x+x_0}{a}=\dfrac{y+y_0}{b}=\dfrac{z-z_0}{c}$}
{\True $\dfrac{x-x_0}{a}=\dfrac{y-y_0}{b}=\dfrac{z-z_0}{c}$}
\loigiai{

}
\end{ex}

\begin{ex}%[2H5N2-2]%[Dự án 2025 - Đề cấu trúc mới của Bộ theo [Thành Đức Trung]
Trong không gian $Oxyz$, cho mặt phẳng $(P)\colon x-2y-3z-2=0$. Đường thẳng $d$ vuông góc với mặt phẳng $(P)$ có một véc-tơ chỉ phương là
\choice
{$\overrightarrow{u}_{1}=(1;-2;-2)$}
{\True $\overrightarrow{u}_{2}=(1;-2;-3)$}
{$\overrightarrow{u}_{4}=(1;2;3)$}
{$\overrightarrow{u}_{3}=(1;-3;-2)$}
\loigiai
{
Ta có $(P)\colon x-2y-3z-2=0$, suy ra một véc-tơ pháp tuyến của $(P)$ là $\overrightarrow{u}_{2}=(1;-2;-3)$.
}
\end{ex}

\begin{ex}%[2H5N3-2]
Trong không gian $Oxyz$, cho mặt cầu $(S)\colon x^2+(y-4)^2+(z-1)^2=25$. Tọa độ tâm $I$ và bán kính $R$ của mặt cầu $(S)$ là
\choice
{$I(0;-4;-1)$, $R=25$}
{$I(0;-4;-1)$, $R=5$}
{$I(0;4;1)$, $R=25$}
{\True $I(0;4;1)$, $R=5$}
\loigiai{
Mặt cầu $(S)$ có tâm $I(0;4;1)$ và bán kính $R=5$.
}
\end{ex}

\begin{ex}%[12-PTMH-1-2025]%[Võ Thị Thùy Trang]%[2D6N1-1]
Cho hai biến cố $A$ và $B$ bất kì, với $\mathrm{P}(B)>0$. Công thức tính xác suất nào sau đây là đúng?
\choice
{$\mathrm{P}(A\mid B)= \dfrac{\mathrm{P}(A)}{\mathrm{P}(B)}$}
{\True $\mathrm{P}(A\mid B)= \dfrac{\mathrm{P}(AB)}{\mathrm{P}(B)}$}
{$\mathrm{P}(A\mid B)= \dfrac{\mathrm{P}(AB)}{\mathrm{P}(A)}$}
{$\mathrm{P}(A\mid B)= \dfrac{\mathrm{P}(B)}{\mathrm{P}(A)}$}
\loigiai
{
Theo tính chất, công thức đúng là $\mathrm{P}(A\mid B)= \dfrac{\mathrm{P}(AB)}{\mathrm{P}(B)}$.
}
\end{ex}

\begin{ex}%[2D6N2-1]%[Lê Công Trường]
Giả sử tỉ lệ người dân của tỉnh Khánh Hòa nghiện thuốc lá là $\mathrm{P}(A)$; tỉ lệ người bị bệnh phổi
trong số người nghiện thuốc lá là $\mathrm{P}(B)$, trong số người không nghiện thuốc lá là $\mathrm{P}(B\mid \overline{A})$. Hỏi khi ta gặp ngẫu nhiên một người dân của tỉnh Khánh Hòa thì khả năng mà đó bị bệnh phổi là
\choice
{$\mathrm{P}(A)=\mathrm{P}(B)\cdot\mathrm{P}(A\mid B)+\mathrm{P}(\overline{B})\cdot\mathrm{P}(A\mid \overline{B})$}
{$\mathrm{P}(A)=\mathrm{P}(A)\cdot\mathrm{P}(A\mid B)+\mathrm{P}(\overline{A})\cdot\mathrm{P}(A\mid \overline{B})$}
{\True $\mathrm{P}(B)=\mathrm{P}(A)\cdot\mathrm{P}(B\mid A)+\mathrm{P}(\overline{A})\cdot\mathrm{P}(B\mid \overline{A})$}
{$\mathrm{P}(A\mid B)=\dfrac{\mathrm{P}(A)\cdot\mathrm{P}(B\mid A)}{\mathrm{P}(B)}$}
\loigiai{Khi ta gặp ngẫu nhiên một người dân của tỉnh Khánh Hòa thì khả năng mà đó bị bệnh phổi là \[\mathrm{P}(B)=\mathrm{P}(A)\cdot\mathrm{P}(B\mid A)+\mathrm{P}(\overline{A})\cdot\mathrm{P}(B\mid \overline{A}).\]}
\end{ex}

\begin{ex}%[2D6N2-3]%[Dự án EX-TF-TLN lần 4 - Quan Ón]
Cho hai biến cố $A$, $B$ sao cho $\mathrm{P}(A) = 0{,}6$; $\mathrm{P}(B) = 0{,}4$ ; $\mathrm{P}(B\mid A) = 0{,}2$. Khi đó, $\mathrm{P}(A\mid B)$ bằng
\choice
{$0{,}11$}
{$0{,}57$}
{$0{,}83$}
{\True $0{,}30$}
\loigiai{
Áp dụng công thức Bayes, ta có $\mathrm{P}(A\mid B) = \dfrac{\mathrm{P}(A)\cdot\mathrm{P}(B\mid A)}{\mathrm{P}(B)} = \dfrac{0{,}6\cdot 0{,}2}{0{,}4} = 0{,}3$.
}
\end{ex}

\begin{ex}%[2H5N3-3]
Trong không gian $Oxyz$, phương trình mặt cầu tâm $I(-1;2;0)$ và bán kính bằng $2$ là
\choice
{$(x+1)^{2}+(y-2)^{2}+z^{2}=4$}
{$(x-1)^{2}+(y+2)^{2}+z^{2}=2$}
{$(x+1)^{2}+(y-2)^{2}+z^{2}=2$}
{\True $(x-1)^{2}+(y+2)^{2}+z^{2}=4$}
\loigiai{
Phương trình mặt cầu tâm $I(-1;2-0)$ và bán kính bằng $2$ là \[(x-(-1))^{2}+(y-2)^{2}+z^{2}=4\] hay \[(x+1)^{2}+(y-2)^{2}+z^{2}=4.\]
}
\end{ex}
\Closesolutionfile{ans}

\TNTF
\Opensolutionfile{ans}[ans/ansDe4-TN2]
\begin{ex}%[2025-TLOT-2018,Trần Xuân Hòa]%[2D4H2-2]
Cho hàm số $f(x)=x(x^2+3)$. Xét $I=\displaystyle\int\limits_{-1}^1|f(x)|\mathrm{\; d}x$.
\choiceTF
{\True Đặt $I_1=\displaystyle\int\limits_{0}^1|f(x)|\mathrm{\; d}x$ và $I_2=\displaystyle\int\limits_{-1}^0|f(x)|\mathrm{\; d}x$. Khi đó $I_1=I_2$}
{Giá trị $I=0$}
{\True Số thực dương $m$ để $\displaystyle\int\limits_0^m|f(x)|\mathrm{\;d}x=4$ bằng $\sqrt{2}$}
{Số thực $a$ để $\displaystyle\int\limits_0^1x(x^2+3-a\sqrt{x})\mathrm{\; d}x=0$ bằng $4$}
\loigiai{
\begin{itemchoice}
\itemch Đúng.
\begin{itemize}
\item Ta có $I_1=\displaystyle\int\limits_0^1(x^3+3x)\mathrm{\; d}x=\left(\dfrac{x^4}{4}+\dfrac{3x^2}{2}\right)\bigg |_0^1=\dfrac{7}{4}$.
\item Ta có $I_2=\displaystyle\int\limits_{-1}^0(-x^3-3x)\mathrm{\; d}x=\left(-\dfrac{x^4}{4}-\dfrac{3x^2}{2}\right)\bigg |_{-1}^0=\dfrac{7}{4}$.
\end{itemize}
Do đó $I_1=I_2$.
\itemch Sai. Ta có $I=\displaystyle\int\limits_{-1}^0|f(x)|\mathrm{\; d}x+\displaystyle\int\limits_{0}^1|f(x)|\mathrm{\; d}x=I_1+I_2=\dfrac{7}{4}+\dfrac{7}{4}=\dfrac{7}{2}$.
\itemch Đúng. Ta có
\begin{eqnarray*}
&&\displaystyle\int\limits_0^m|f(x)|\mathrm{\;d}x=4\\
&\Leftrightarrow&\displaystyle\int\limits_0^{m}(x^3+3x)\mathrm{\;d}x=4\\
&\Leftrightarrow&\left(\dfrac{x^4}{4}+\dfrac{3x^2}{2}\right)\bigg |_0^m=4\\
&\Leftrightarrow&\dfrac{m^4}{4}+\dfrac{3m^2}{2}-4=0\\
&\Leftrightarrow&\hoac{&m=\sqrt{2}\\&m=-\sqrt{2} \text{ ( loại )}.}
\end{eqnarray*}
Vậy $m=\sqrt{2}$ thỏa mãn.
\itemch Sai. Ta có
\begin{eqnarray*}
&&\displaystyle\int\limits_0^1x(x^2+3-a\sqrt{x})\mathrm{\; d}x=0\\
&\Leftrightarrow&\displaystyle\int\limits_0^1\left(x^3+3x-ax^{\tfrac{3}{2}}\right)\mathrm{\; d}x=0\\
&\Leftrightarrow&\left(\dfrac{1}{4}x^4+\dfrac{3}{2}x^2-\dfrac{2a}{5}x^{\tfrac{5}{2}}\right)\bigg|_0^1=0\\
&\Leftrightarrow&\dfrac{7}{4}-\dfrac{2a}{5}=0\\
&\Leftrightarrow&a=\dfrac{35}{8}.
\end{eqnarray*}
\end{itemchoice}
}
\end{ex}
\begin{ex}%[2H5H1-4]
	Trong không gian $Oxyz$, cho mặt phẳng $(P)$ đi qua $A(3;-2;5)$ và có vectơ pháp tuyến $\vec{n}=(4;-3;2)$ và mặt phẳng $(R)\colon x+2y-z+6=0$.
	\choiceTF
	{Phương trình mặt phẳng $(P)$ là $3x-2y+5z-28=0$}
	{\True $(P)$ vuông góc với mặt phẳng $(R)$}
	{Mặt phẳng $(P)$ cắt mặt phẳng $(R)$ theo giao tuyến là đường thẳng $d\colon \dfrac{x-3}{4}=\dfrac{y+2}{-3}=\dfrac{z-5}{2}$}
	{\True Mặt cầu tâm $A(3;-2;5)$ và bán kính $R=2$ cắt mặt phẳng $(R)$ theo giao tuyến là đường tròn có bán kính bằng $2$}
	\loigiai{
	\begin{itemchoice}
	\itemch Sai. Phương trình mặt phẳng $(P)$ đi qua $A(3;-2;5)$ có một vectơ pháp tuyến $\vec{n}_{(P)}=(4;-3;2)$ có dạng $4(x-3)-3(y+2)+2(z-5)=0$, hay $(P)\colon 4x-3y+2z-28=0$.
	\itemch Đúng. Vì với $\vec{n}_{(P)}=(4;-3;2)$ và $\vec{n}_{(R)}=(1;2;1)$ lần lượt là vectơ pháp tuyến của $(P)$ và $(R)$, ta thấy $\vec{n}\cdot \vec{n}_2=0$ nên $(P)\perp (R)$.
	
	\end{itemchoice}
	}
	\end{ex}
\Closesolutionfile{ans}

\TNSA
\Opensolutionfile{ans}[ans/ansDe4-TN3]
\begin{ex}%[2D4H1-4]
Hàm số $f(x)$ có đạo hàm liên tục trên $\mathbb{R}$ và $f'(x)=2^x+3^x$, $\forall x$; $f(0)=\dfrac{1}{\ln3}$. Tính giá trị của $f(1)$. (Làm tròn đến số thập phân thứ hai)
\shortans{$4{,}17$}
\loigiai{
Hàm số $f(x)=\displaystyle \int (2^x+3^x)\mathrm{\,d}x= \displaystyle \int 2^x \mathrm{\,d}x+\displaystyle \int 3^x \mathrm{\,d}x=\dfrac{2^x}{\ln2}+\dfrac{3^x}{\ln3}+C$.\\
$f(x)=\dfrac{2^x}{\ln2}+\dfrac{3^x}{\ln3}+C$.\\
Do $f(0)=\dfrac{1}{\ln3} \Leftrightarrow \dfrac{1}{\ln3}=\dfrac{1}{\ln2}+\dfrac{1}{\ln3}+C\Leftrightarrow C=-\dfrac{1}{\ln2}$\\
Suy ra $f(x)=\dfrac{2^x}{\ln2}+\dfrac{3^x}{\ln3}-\dfrac{1}{\ln2}$.\\
Vậy $f(1)=4{,}17$.
}
\end{ex}

\begin{ex}%[2D4V2-6]
Một vật chuyển động trong $6$ giờ với vận tốc $v$ (km/h) phụ thuộc vào thời gian $t$ (h) có đồ thị như hình bên dưới. Trong khoảng thời gian $2$ giờ từ khi bắt đầu chuyển động, đồ thị là một phần đường Parabol có đỉnh $I\left(3;9\right)$ và có trục đối xứng song song với trục tung. Khoảng thời gian còn lại, đồ thị vận tốc là một đường thẳng có hệ số góc bằng $\dfrac{1}{4}$. Tính quảng đường $s$ mà vật di chuyển được trong $6$ giờ? (đơn vị tính bằng km, làm tròn đến chữ số thập phân thứ nhất).
\begin{center}
\begin{tikzpicture}[>=stealth,scale=0.5]
% Vẽ 2 trục, điền các số lên trục
\draw[->] (-0.5,0)--(0,0) node[below left]{$O$}--(7,0) node[above]{$t$}; %định dạng trục Ox
\foreach \x in {2,3,6}
\draw[shift={(\x,0)},color=black] (0pt,2pt)--(0pt,-2pt)
node[below] { $\x$};
\draw[->,color=black] (0,-0.5)--(0,10) node[right]{$v$};  %định dạng trục Oy
\foreach \y in {8,9}
\draw[shift={(0,\y)},color=black] (2pt,0pt) -- (-2pt,0pt)
node[left] {$\y$};
\clip(-1,-1) rectangle (7,10); %vùng đồ thị
%\draw[gray!50,thin,opacity=.5] (-1,-1) grid (4,10); %ô vuông
%Vẽ đồ thị
\draw[smooth,samples=100,domain=0:2,font=\footnotesize, line join=round, line cap=round, thick]
plot(\x,{(-1)*(\x)^2+6*(\x)});
\draw[smooth,domain=2:6, line join=round, line cap=round,dashed]
plot(\x,{(-1)*(\x)^2+6*(\x)});
\draw[smooth,samples=100,domain=2:6,font=\footnotesize, line join=round, line cap=round, thick]
plot(\x,{(1/4)*(\x)+15/2});
% Vẽ thêm mấy cái râu ria
\draw[dashed] (3,0)--(3,9) circle(1.5pt) node[above]{$I$}--(0,9) circle(1.5pt);
\draw[dashed] (2,0)--(2,8) circle(1.5pt) --(0,8);
\draw[dashed] (6,0)--(6,9) circle(1.5pt) --(0,9);
%Vẽ dấu chấm tròn
\fill (0cm,0cm) circle (1.5pt);
\end{tikzpicture}
\end{center}
\shortans{$43{,}3$}
\loigiai{
Vì Parabol đi qua $O\left(0;0\right)$ và có tọa độ đỉnh $I\left(3;9\right)$ nên thiết lập được phương trình Parabol là $\left(P\right) \colon y = v\left(t\right) = -t^2+6t$; $\forall t \in \left[0;2\right]$.\\
Sau $2$ giờ đầu thì hàm vận tốc có dạng là hàm bậc nhất $y = \dfrac{1}{4}t + m$, dựa trên đồ thị ta thấy đi qua điểm có tọa độ $\left(6;9\right)$ nên thế vào hàm số và tìm được $m = \dfrac{15}{2}$.\\
Nên hàm vận tốc từ giờ thứ $2$ đến giờ thứ $6$ là: $y = \dfrac{1}{4}t + \dfrac{15}{2},\forall t \in \left[2;6\right]$.\\
Quảng đường vật đi được bằng tổng đoạn đường $2$ giờ đầu và đoạn đường $4$ giờ sau.
\[S = S_1 +S_2 = \displaystyle\int\limits_0^2 \left(-t^2+6t\right) \mathrm{\,d}t + \displaystyle\int\limits_2^6 \left(\dfrac{1}{4}t+\dfrac{15}{2} \right) \mathrm{\,d}t = \dfrac{130}{3} \approx 43{,}3 \left(\text{km}\right).\]
}
\end{ex}

\begin{ex}%[Nguyễn Tuấn, dự án TLDT-2]%[2H5V2-8]
	Trong không gian $Oxyz$, hai máy bay cùng xuất phát từ hai phi trường, trên màn hình rađa của trạm điều khiển (với đơn vị trên ba trục chính theo đơn vị km), sau khi xuất phát $ t$ giờ $(t\ge 0)$, vị trí của máy bay số một được xác định bởi công thức $\heva{&x=20+2t \\ &y=20+t \\ &z=-10-t }$, vị trí máy bay số hai có tọa độ là $(30+t';20+t';-10-t')$. Hỏi nếu hai máy bay không thay đổi đường bay thì sau bao lâu thì hai máy bay có thể va chạm nhau?
	\shortans{10}
	\loigiai{
		Giả sử đường bay của máy bay số 1 là $(\Delta_1)\colon \heva{&x=20+2t \\ &y=20+t \\ &z=-10-t }$ có $\overrightarrow{u}_1=(2;1;-1)$ và đường bay của máy bay số 2 thỏa $(30+t';20+t';-10-t')\in (\Delta_2)\colon \heva{&x=30+t' \\ &y=20+t' \\ &z=-10-t' }$ có $\overrightarrow{u}_2=(1;1;-1)$.\\
		Kể từ thời điểm xuất phát, để hai may bay gần nhau nhất thì hai máy bay phải gần tọa độ giao điểm của $\Delta_1$ và $\Delta_2$.\\
	Ta có $\heva{&20+2t=30+t' \\ &20+t=20+t' \\ &-10-t=-10-t' }\Leftrightarrow \heva{&2t-t'=10 \\ &t-t'=0 \\ &t-t'=0 }\Leftrightarrow \heva{&t=10 \\ &t'=10.}$\\
	Vậy sau $10$ giờ thì hai máy bay có thể va chạm nhau.
	}
	\end{ex}

\begin{ex}%[2D6V1-3]
Trường Hạnh Phúc có $20$\% học sinh tham gia câu lạc bộ âm nhạc, trong số học sinh đó có $85$\% học sinh biết chơi đàn guitar. Ngoài ra, có $10$\% số học sinh không tham câu lạc bộ âm nhạc cũng biết chơi đàn guitar. Chọn ngẫu nhiên $1$ học sinh của trường. Giả sử học sinh đó biết chơi đàn guitar. Xác suất chọn được học sinh thuộc câu lạc bộ âm nhạc là bao nhiêu?

\shortans{0,68}
\loigiai
{
Gọi $A$ là biến cố \textquotedblleft Số học sinh thuộc câu lạc bộ âm nhạc\textquotedblright.\\
$B$ là biến cố \textquotedblleft Số học sinh không thuộc câu lạc bộ\textquotedblright.\\
Khi đó $\heva{&P(A) = 20\% = 0{,}2 &\Rightarrow& P(\overline{A}) = 1-0{,}2 = 0{,}8\\ &P(B|A) = 85\% = 0{,}85 &\Rightarrow& P(\overline{B}|A) = 1-0{,}85 = 0{,}15\\ &P(B|\overline{A}) = 10\% = 0{,}1 &\Rightarrow& P(\overline{B}|\overline{A}) = 1-0{,}1=0{,}9.}$
\begin{center}
\begin{tikzpicture}[>=stealth]
%Khung 1
\draw (-0,-1) rectangle (2.2,0);
\draw (1.1,-0.5) node{Gốc};
%Mui ten 1,2
\draw [->] (2.2,-0.5)--(3.8,1.6) node[pos=0.5,sloped,above]{$0{,}2$};
\draw [->] (2.2,-0.5)--(3.8,-2.6) node[pos=0.5,sloped,below]{\color{red}$0{,}8$};
%Khung 2.1
\draw (4.5,3.0) node{\textbf{Thuộc câu lạc bộ}};
\draw (3.8,1.1) rectangle (5.1,2.1);
\draw (8.9/2,1.6) node{$A$} ;
%Khung 2.2
\draw (3.8,-2.1) rectangle (5.1,-3.1);
\draw (8.9/2,-2.6) node{$\overline{A}$};
%Mui ten 3,4
\draw [->] (5.1,1.6)--(6.5,2.6) node[pos=0.5,sloped,above]{$0{,}85$};
\draw [->] (5.1,1.6)--(6.5,0.6) node[pos=0.5,sloped,below]{\color{red}$0{,}15$};
%Mui ten 5,6
\draw [->] (5.1,-2.6)--(6.5,-1.6) node[pos=0.5,sloped,above]{$0{,}1$};
\draw [->] (5.1,-2.6)--(6.5,-3.6) node[pos=0.5,sloped,below]{\color{red}$0{,}9$};
%Khung 3.1
\draw (6.5,2.2) rectangle (7.7,3.2);
\draw (7.1,5.4/2) node{$B$} ;
%Khung 3.2
\draw (6.7,3.7) node{\textbf{Biết chơi guitar}};
\draw (6.5,1.2) rectangle (7.7,0.2);
\draw (7.1,1.4/2) node{$\overline{B}$} ;
%Khung 3.3
\draw (6.5,-1.1) rectangle (7.7,-2.1);
\draw (7.1,-3.2/2) node{$B$} ;
%Khung 3.3
\draw (6.5,-2.9) rectangle (7.7,-3.9);
\draw (7.1,-3.4) node{$\overline{B}$} ;
%Kết quả
\draw (9.5,3.7) node{\textbf{Kết quả}};
\draw (9.5,2.7) node{$AB$};
\draw (9.5,0.7) node{$A \overline{B}$};
\draw (9.5,-1.6) node{$\overline{A}B$};
\draw (9.5,-3.4) node{$\overline{A}~\overline{B}$};
%Xác suất
\draw (12.5,3.7) node{\textbf{Xác suất}};
\draw (12.5,2.7) node{$0{,}17$};
\draw (12.5,0.7) node{$0{,}03$};
\draw (12.5,-1.6) node{$0{,}08$};
\draw (12.5,-3.4) node{$0{,}72$};
\end{tikzpicture}
\end{center}
Áp dụng công thức xác suất Bayes để tính xác chọn được học sinh thuộc câu lạc bộ âm nhạc
\[P(A|B) = \dfrac{P(A)\cdot P(B|A)}{P(A) \cdot P(B|A) + P(\overline{A}) \cdot P(B|\overline{A})} = \dfrac{0{,}2 \cdot 0{,}85}{0{,}2 \cdot 0{,}85 + 0{,}8\cdot 0{,}1} = 0{,}68.\]
}
\end{ex}
\TL
\begin{ex}%[2H5H2-3]%[Dự án EX-TF-TLN lần 3 - Nguyen Chín Em]
Trong không gian với hệ tọa độ $Oxyz$. Viết phương trình tham số của đường thẳng $d$ biết $d$ đi qua điểm $M(2; -3; 4)$, vuông góc với $d_1$ và $d_2: \dfrac{x + 1}{2} = \dfrac{y}{5} =\dfrac{z + 3}{3}$. Khi đó đường thẳng $d$ đi qua điểm $(x_0;y_0;-13)$. Tính $x_0^{2}+y_0^{2}$.
% \shortans{$125$}
\loigiai{
Ta có véc-tơ chỉ phương của $d_1$ là $\overrightarrow{u_1} = (-3; 1; 2)$ và véc-tơ chỉ phương của $d_2$ là $\overrightarrow{u_2} = (2; 5; 3)$.\\
Do $d \perp d_1$ và $d \perp d_2$ nên véc-tơ chỉ phương của $d$ là $\overrightarrow{u} =\left[\overrightarrow{u_1}, \overrightarrow{u_2}\right] = (-7; 13; -17)$.\\
Phương trình tham số của đường thẳng $d$ là $\heva{&x = 2 - 7t \\&y = -3 + 13t \\&z = 4 - 17t},(t\in\mathbb{R})$.\\
% Với $t=1$ đường thẳng $d$ đi qua điểm $(-5;10;-13)$, suy ra $x_0^{2}+y_0^{2}=125$.
}
\end{ex}

\begin{ex}%[12-PTMH-1-2025]%[Võ Thị Thùy Trang]%[2D6C2-4]
Ở một khu rừng nọ có $7$ chú lùn, trong đó có $4$ chú luôn nói thật, $3$ chú còn lại luôn tự nhận mình nói nhật nhưng xác suất để mỗi chú này nói thật là $0{,}5$. Bạn Tuyết gặp ngẫu nhiên một chú lùn. Gọi $A$ là biến cố \lq \lq Chú lùn đó luôn nói thật\rq \rq \, và $B$ là biến cố \lq \lq Chú lùn đó tự nhận mình luôn nói thật\rq \rq.
Biết rằng chú lùn mà bạn Tuyết gặp tự nhận mình là người luôn nói thật. Biết xác suất để chú lùn đó luôn nói thật có thể được biểu diễn dưới dạng $\dfrac{a}{b}$ sao cho $\dfrac{a}{b}$ là phân số tối giản. Tính $2a+b$.
% \shortans{$27$}
\loigiai{

Gọi $A$ là biến cố \lq \lq Chú lùn bạn Tuyết gặp luôn nói thật\rq \rq
\,và $B$ là biến cố \lq \lq Chú lùn đó luôn tự nhận mình nói thật\rq \rq.\\
Vì có $4$ chú lùn luôn nói thật nên xác suất để bạn Tuyết gặp chú lùn luôn nói thật là $\mathrm{P}(A)=\dfrac{4}{7}$ và gặp chú lùn tự nhận mình luôn nói thật là $\mathrm{P}(\overline{A})=\dfrac{3}{7}$.\\
Theo đề bài, ta có $\mathrm{P}(B\mid A)=1$, $\mathrm{P}(B\mid \overline{A})=0{,}5$.\\
Theo công thức xác suất toàn phần, ta có
\begin{align*}
\mathrm{P}(B) &=\mathrm{P}(B\mid A)\cdot\mathrm{P}(A)+\mathrm{P}(B\mid\overline{A})\cdot\mathrm{P}(\overline{A})\\
&=1 \cdot \dfrac{4}{7} + 0{,}5 \cdot \dfrac{3}{7}\\
&=\dfrac{11}{14}.
\end{align*}
Theo công thức Bayes, ta có
\[\mathrm{P}(A\mid B)=\dfrac{\mathrm{P}(B\mid A) \cdot \mathrm{P}(A)}{\mathrm{P}(B)}=\dfrac{1 \cdot \dfrac{4}{7}}{\dfrac{11}{14}}=\dfrac{8}{11}.\]
Vậy nếu chú lùn mà bạn Tuyết gặp tự nhận mình là người luôn nói thật, xác suất để chú lùn đó luôn nói thật là $\dfrac{8}{11}$.\\
Do đó $a=8$, $b=11$ nên $2a+b=2 \cdot 8+11=27$.
}
\end{ex}

\begin{ex}%[Mức độ ]giảng 12 New - 4in1, Đoàn Hùng]%[2H5V1-7]
	\immini
	{Trong không gian với hệ tọa độ $Oxyz$ (đơn vị trên mỗi trục toạ độ là km), một máy bay đang ở vị trí $A(3;-2{,}5; 0{,}5)$ và sẽ hạ cánh ở vị trí $B(3; 7{,}5; 0)$ trên đường băng (hình bên). Có một lớp mây được mô phỏng bởi một mặt phẳng $(\alpha)$ đi qua ba điểm $M(9;0;0)$, $N(0;-9;0)$, $P(0;0;0{,}9)$. Tính độ cao của máy bay khi máy bay xuyên qua đám mây để hạ cánh.}
	{\begin{tikzpicture}[line join = round, line cap = round,>=stealth,font=\footnotesize,scale=.5]
	\path
	(0,0) coordinate (O)
	(-9,0) coordinate (N)
	($(O)!1!40:(N)$) coordinate (M)
	(0,0.9) coordinate (P)
	($(O)!1.2!(M)$) coordinate (x)
	(-5,0.8) coordinate (A)
	(4,-1.5) coordinate (B)
	($(A)!2.1cm!(B)$) coordinate (C)
	(intersection of A--B and O--M) coordinate (B1)
	;
	\draw[line width=0.3mm,red] (A)--(C) (B1)--(B);
	\draw[line width=0.3mm,red,dashed] (C)--(B1);
	\draw[->,>=stealth,line width=0.3mm,blue] (0,0)--(x) node[right=0.2cm]{$x$};
	\draw[->,>=stealth,line width=0.3mm,blue] (-10,0)--(-9,0) (0,0) node[above right]{$O$}--(7,0) node[below]{$y$};
	\draw[line width=0.3mm,blue,dashed] (-9,0)--(0,0);
	\draw[->,>=stealth,line width=0.3mm,blue] (0,0)--(0,3) node[left]{$z$};
	\draw[line width=0.3mm,blue] (N)--(P)--(M) (N)--(M);
	\foreach \x/\gm in {N/90,P/140} \fill (\x) circle (1pt) ($(\x)+(\gm:5mm)$)node[blue]{$\x$};
	\filldraw[red] (N)node[below left,blue]{$-9$} (P)node[blue,above right]{$0{,}9$} (M)node[blue,right=0.1cm]{$9$}node[blue,left=0.1cm]{$M$} (C) circle (3pt) node[below=0.2cm,blue]{\scriptsize $C$} (A) circle (3pt)node[above left,red,blue]{$A$} (B)circle (3pt) node[below,blue]{$B$};
	\end{tikzpicture}}
	\shortans{$0{,}45$}
	\loigiai{
	Giả sử điểm $C\left(x_C;y_C;z_C\right)$ là vị trí mà máy bay xuyên qua đám mây để hạ cánh, suy ra $C\in (\alpha)$. Áp dụng phương trình mặt phẳng theo đoạn chắn, ta thấy mặt phẳng $(\alpha)$ có phương trình là
	\[\dfrac{x}{9}-\dfrac{y}{9}+\dfrac{z}{0{,}9}=1 \Leftrightarrow x-y+10z=9 \Rightarrow x_C-y_C+10z_C=9.\]
	Mặt khác, vì $\vec{AC}$, $\vec{AB}$ là hai véc-tơ cùng hướng nên tồn tại số thực $t>0$ sao cho $\vec{AC}=t\cdot \vec{AB}$.\\
	Do $\vv{AC}=\left(x_C-3;y_C+2{,}5;z_C-0{,}5\right)$; $\vv{AB}=\left(3-3;7{,}5+2{,}5;0-0{,}5\right)=\left(0;10;-0{,}5\right)$\\
	nên $\heva{&x_C-3=0t\\&y_C+2{,}5=10t\\&z_C-0{,}5=-0{,}5t} \Leftrightarrow \heva{&x_C=3\\&y_C=10t-2{,}5\\&z_C=-0{,}5t+0{,}5.}$\\
	Vì $C\in(\alpha)$ nên $3-(10 t-2{,}5)+10(-0{,}5 t+0{,}5)=9 \Leftrightarrow t=0{,}1$. Suy ra $C(3;-1{,}5;0{,}45)$.\\
	Vậy tại vị trí $C$, độ cao của máy bay là $0{,}45$ km.
	}
	\end{ex}
\Closesolutionfile{ans}

\Closesolutionfile{ansbook}

%


%%Ôn THI THPTQG
% % \def\tendethi{ĐỀ MINH HOẠ 2025}
\begin{name}
    {\tenchude}
    {ĐỀ MINH HOẠ 2025}
    {\tentruong}
    {\thoigian}
\end{name}

\Opensolutionfile{ansbook}[ans/ansbook-DE-PNL-1-MH]
\TN
\Opensolutionfile{ans}[ans/ans-DE-PNL-01-T]


%%==========Câu 1
\begin{ex}%[Đề Minh Hoạ TNTHPT 2024-2025]%[2PhatTrien-DMH-2024-2025, GV: Hoàng Trọng Tấn]%[2D4N1-4]
    Nguyên hàm của hàm số $f(x) = \mathrm{e}^x$ là:
    \choice
    {$\dfrac{\mathrm{e}^{x+1}}{x+1} + C$}
    {\True $\mathrm{e}^x + C$}
    {$\dfrac{\mathrm{e}^x}{x} + C$}
    {$x \mathrm{e}^{x-1} + C$}
    \loigiai{
        Nguyên hàm của $f(x)=\mathrm{e}^x$ là $F(x)=\mathrm{e}^x+C$.	
    }
\end{ex}

%%==========Câu 2
\begin{ex}%[Đề Minh Hoạ TNTHPT 2024-2025]%[2PhatTrien-DMH-2024-2025, GV: Hoàng Trọng Tấn]%[2D4N3-3]
    Cho hàm số $y = f(x)$ liên tục, nhận giá trị dương trên đoạn $[a;b]$. Xét hình phẳng $(H)$ giới hạn bởi đồ thị hàm số $y = f(x)$, trục hoành và hai đường thẳng $x = a$, $x = b$. Khối tròn xoay được tạo thành khi quay hình phẳng $(H)$ quanh trục $Ox$ có thể tích là:
    \choice
    {$V = \pi \displaystyle \int_a^b [f(x)] \mathrm{\,d}x$}
    {$V = \pi^2 \displaystyle \int_a^b f(x) \mathrm{\,d}x$}
    {$V = \pi^2 \displaystyle \int_a^b [f(x)]^2 \mathrm{\,d}x$}
    {\True $V = \pi \displaystyle \int_a^b [f(x)]^2 \mathrm{\,d}x$}
    \loigiai{
        Khối tròn xoay được tạo thành khi quay hình phẳng $(H)$ quanh trục $Ox$ có thể tích là $V = \pi \displaystyle \int_a^b [f(x)]^2 \mathrm{\,d}x$.
    }
\end{ex}

%%==========Câu 3
\begin{ex}%[Đề Minh Hoạ TNTHPT 2024-2025]%[2PhatTrien-DMH-2024-2025, GV: Hoàng Trọng Tấn]%[2D3H2-1]
    Hai mẫu số liệu ghép nhóm $M_1$, $M_2$ có bảng tần số ghép nhóm như sau:
    \begin{center}
        \begin{tabular}{|c|c|c|c|c|c|}
            \hline
            Nhóm & $[8;10)$ & $[10;12)$ & $[12;14)$ & $[14;16)$ & $[16;18)$ \\
            \hline
            $M_1$ & $3$ & $4$ & $8$ & $6$ & $4$ \\
            \hline
            $M_2$ & $6$ & $8$ & $16$ & $12$ & $8$ \\
            \hline
        \end{tabular}
    \end{center}	
    Gọi $s_1$, $s_2$ lần lượt là độ lệch chuẩn của mẫu số liệu ghép nhóm $M_1$, $M_2$. Phát biểu nào sau đây là đúng?
    \choice
    {\True $s_1 = s_2$}
    {$s_1 = 2s_2$}
    {$2s_1 = s_2$}
    {$4s_1 = s_2$}
    \loigiai{
        Giá trị trung bình của mẫu $M_1$:
        $$
        \overline{x}_1 = \dfrac{9 \cdot 3 + 11 \cdot 4 + 13 \cdot 8 + 15 \cdot 6 + 17 \cdot 4}{3 + 4 + 8 + 6 + 4}
        = \dfrac{333}{25} = 13{,}32
        $$	
        Tính các sai lệch bình phương:
        $$
        (9 - 13{,}32)^2 = 18{,}6624,  (11 - 13{,}32)^2 = 5{,}3824, (13 - 13{,}32)^2 = 0{,}1024, 
        $$
        $$
        (15 - 13{,}32)^2 = 2{,}8224, (17 - 13{,}32)^2 = 13{,}5424
        $$
        Phương sai của mẫu $M_1$:
        $$
        s_1^2 = \dfrac{3\cdot 18{,}6624 + 4\cdot 5{,}3824 + 8\cdot 0{,}1024 + 6\cdot 2{,}8224 + 4\cdot 13{,}5424}{25}=5{,}9776
        $$
        Vậy, độ lệch chuẩn của mẫu $M_1$ là:
        $$
        s_1 = \sqrt{5{,}9776} \approx 2{,}445
        $$
        Tương tự 
        \\
        Giá trị trung bình của mẫu $M_2$:
        $$
        \overline{x}_2 = \dfrac{9 \cdot 6 + 11 \cdot 8 + 13 \cdot 16 + 15 \cdot 12 + 17 \cdot 8}{6 + 8 + 16 + 12 + 8}
        = \dfrac{666}{50} = 13{,}32
        $$	
        Tính các sai lệch bình phương:
        $$
        (9 - 13{,}32)^2 = 18{,}6624, \quad (11 - 13{,}32)^2 = 5{,}3824, \quad (13 - 13{,}32)^2 = 0{,}1024,
        $$
        $$
        (15 - 13{,}32)^2 = 2{,}8224, \quad (17 - 13{,}32)^2 = 13{,}5424
        $$
        Phương sai của mẫu $M_2$:
        $$
        s_2^2 = \dfrac{6\cdot 18{,}6624 + 8\cdot 5{,}3824 + 16\cdot 0{,}1024 + 12\cdot 2{,}8224 + 8\cdot 13{,}5424}{50}= 5{,}9776
        $$
        
        Vậy, độ lệch chuẩn của mẫu $M_2$ là:
        $$
        s_2 = \sqrt{5{,}9776} \approx 2{,}445
        $$
        \\
        Cả hai mẫu $M_1$ và $M_2$ đều có độ lệch chuẩn bằng nhau:
        $$
        s_1 = s_2 \approx 2{,}445
        $$
    }
\end{ex}

%%==========Câu 4
\begin{ex}%[Đề Minh Hoạ TNTHPT 2024-2025]%[2PhatTrien-DMH-2024-2025, GV: Hoàng Trọng Tấn]%[2H5N2-3]
    Trong không gian với hệ trục tọa độ $Oxyz$, phương trình của đường thẳng đi qua điểm $M(1;-3;5)$ và có một vectơ chỉ phương $\overrightarrow{u}(2;-1;1)$ là:
    \choice
    {$\dfrac{x-1}{2} = \dfrac{y+3}{-1} = \dfrac{z-5}{1}$}
    {$\dfrac{x+1}{2} = \dfrac{y+3}{-1} = \dfrac{z+5}{1}$}
    {\True $\dfrac{x-1}{2} = \dfrac{y+3}{1} = \dfrac{z-5}{-1}$}
    {$\dfrac{x+1}{2} = \dfrac{y+3}{-1} = \dfrac{z-5}{-1}$}
    \loigiai{
        Phương trình của đường thẳng đi qua điểm $M(1;-3;5)$ và có một vectơ chỉ phương $\overrightarrow{u}(2;-1;1)$ là $\dfrac{x-1}{2} = \dfrac{y+3}{1} = \dfrac{z-5}{-1}$.
    }
\end{ex}

%%==========Câu 5
\begin{ex}%[Đề Minh Hoạ TNTHPT 2024-2025]%[2PhatTrien-DMH-2024-2025, GV: Hoàng Trọng Tấn]%[2D1N4-1]
    %[id6]
    \immini{
        Cho hàm số $y = \dfrac{ax + b}{cx + d}$ ($c \neq 0$, $ad - bc \neq 0$) có đồ thị như hình vẽ bên. Tiệm cận ngang của đồ thị hàm số là:
        \choice
        {$x = -1$}
        {\True $y = \dfrac{1}{2}$}
        {$y = -1$}
        {$x = \dfrac{1}{2}$}
    }
    {
        \begin{tikzpicture}[line join = round, line cap=round,>=stealth,font=\footnotesize,scale=0.7]
            \clip (-5,-3) rectangle (3,3);
            \draw[smooth,samples=100,domain=-0.7:4] plot(\x,{
                ((\x)-1)/(2*(\x)+2)
            });
            \draw[smooth,samples=100,domain=-5:-1.1] plot(\x,{
                ((\x)-1)/(2*(\x)+2)
            });
            \draw[->] (0,-3)--(0,3) node[below right]{$y$};
            \draw[->] (-5,0)--(3,0) node[below left]{$x$};
            \draw[dashed] 
            (-1,-3)--(-1,3)(-5,0.5)--(5,0.5)
            ;
            \fill 
            (0,0) circle(1.5pt) node[below left]{$O$}
            (0,0.5) circle(1.5pt) node[above right]{$\frac{1}{2}$}
            (-1,0) circle(1.5pt) node[below left]{$-1$}
            (1,0) circle(1.5pt) node[below right]{$1$}
            (0,-0.5) circle(1.5pt) node[right,yshift=-5pt]{$-\frac{1}{2}$}
            ;
        \end{tikzpicture}
    }
    \loigiai{ 
        Dựa vào hình vẽ bên ta thấy đường tiệm cận ngang của đồ thị hàm là $y=\dfrac{1}{2}$.
    }
\end{ex}


%%==========Câu 6
\begin{ex}%[Đề Minh Hoạ TNTHPT 2024-2025]%[2PhatTrien-DMH-2024-2025, GV: Hoàng Trọng Tấn]%[1D6N4-3]
    Tập nghiệm của bất phương trình $\log_2 (x-1) < 3$ là:
    \choice
    {\True $(1;9)$}
    {$(-\infty;9)$}
    {$(9; +\infty)$}
    {$(1;7)$}
    \loigiai{
        Ta có $\log_2 (x-1) < 3\Leftrightarrow \log_2(x-1)< \log_2 8\Leftrightarrow 0<x-1<8\Leftrightarrow 1<x<9$.
    }
\end{ex}

%%==========Câu 7
\begin{ex}%[Đề Minh Hoạ TNTHPT 2024-2025]%[2PhatTrien-DMH-2024-2025, GV: Hoàng Trọng Tấn]%[2H5N1-2]
    Trong không gian với hệ trục tọa độ $Oxyz$, cho mặt phẳng $(P)$ có phương trình $x - 3y - z + 8 = 0$. Véctơ nào sau đây là một vectơ pháp tuyến của mặt phẳng $(P)$?
    \choice
    {$\overrightarrow{n}_1(1;-3;1)$}
    {\True $\overrightarrow{n}_2(1;-3;-1)$}
    {$\overrightarrow{n}_3(1;3;8)$}
    {$\overrightarrow{n}_4(1;3;8)$}
    \loigiai{
        Véctơ $\overrightarrow{n}_2(1;-3;-1)$ là một vectơ pháp tuyến của mặt phẳng $(P)$.
    }
\end{ex}

%%==========Câu 8
\begin{ex}%[Đề Minh Hoạ TNTHPT 2024-2025]%[2PhatTrien-DMH-2024-2025, GV: Hoàng Trọng Tấn]%[1H8N4-2]
    Cho hình chóp $S.ABCD$ có đáy $ABCD$ là hình chữ nhật và $SA \perp (ABCD)$. Mặt phẳng nào sau đây vuông góc với mặt phẳng $(ABCD)$?
    \choice
    {\True $(SAB)$}
    {$(SBC)$}
    {$(SCD)$}
    {$(SBD)$}
    \loigiai{
        Ta có $SA\subset (SAB)$ và $SA \perp (ABCD)\Rightarrow (SAB)\perp (ABCD)$.
    }
\end{ex}

%%==========Câu 9
\begin{ex}%[Đề Minh Hoạ TNTHPT 2024-2025]%[2PhatTrien-DMH-2024-2025, GV: Hoàng Trọng Tấn]%[1D6N4-2]
    Nghiệm của phương trình $2^x = 6$ là:
    \choice
    {$x = \log_2 2$}
    {$x = 3$}
    {$x = 4$}
    {\True $x = \log_2 6$}
    \loigiai{
        Ta có $2^x = 6\Leftrightarrow x=\log_2 6$.	
    }
\end{ex}

%%==========Câu 10
\begin{ex}%[Đề Minh Hoạ TNTHPT 2024-2025]%[2PhatTrien-DMH-2024-2025, GV: Hoàng Trọng Tấn]%[1D2N1-3]
    Cấp số cộng $(u_n)$ có $u_1 = 1$ và $u_2 = 3$. Số hạng $u_5$ của cấp số cộng là:
    \choice
    {$5$}
    {$7$}
    {\True $9$}
    {$11$}
    \loigiai{
        Ta có công sai $d=u_2-u_1=2$.
        \\
        Suy ra $u_5=u_1+4d\Leftrightarrow u_5=1+4\cdot 2=9$.	
    }
\end{ex}

%%==========Câu 11
\begin{ex}%[Đề Minh Hoạ TNTHPT 2024-2025]%[2PhatTrien-DMH-2024-2025, GV: Hoàng Trọng Tấn]%[2H2H1-2]
    
    \immini{
        Cho hình hộp $ABCDA'B'C'D'$ (minh họa như hình bên). Phát biểu nào sau đây là đúng?
        \choice
        {$\overrightarrow{AB} + \overrightarrow{BB'} + \overrightarrow{B'A'} = \overrightarrow{AC'}$}
        {$\overrightarrow{AB} + \overrightarrow{BC} + \overrightarrow{C'D'} = \overrightarrow{AC'}$}
        {$\overrightarrow{AB} + \overrightarrow{AC} + \overrightarrow{AA'} = \overrightarrow{AC'}$}
        {\True $\overrightarrow{AB} + \overrightarrow{AA'} + \overrightarrow{AD} = \overrightarrow{AC'}$}
    }
    {
        \begin{tikzpicture}[line join = round, line cap=round,>=stealth,font=\footnotesize,scale=0.7]
            \def\r{3}
            \path 
            (0,0) coordinate (A)
            (4,0) coordinate (D)
            (-2,-2) coordinate (B)
            ($(B)+(D)-(A)$) coordinate (C)
            ($(A)+(1,\r)$) coordinate (A')
            ($(B)+(1,\r)$) coordinate (B')
            ($(C)+(1,\r)$) coordinate (C')
            ($(D)+(1,\r)$) coordinate (D')
            ;
            
            \draw (B)--(C)--(D)--(D')--(A')--(B')--(B)
            (B')--(C')--(D') (C)--(C')
            ;
            \draw[dashed]
            (B)--(A)--(D) (A)--(A') (A)--(C');
            \foreach \p/\r in {A/-90,B/-90,C/-90,D/-90,A'/90,B'/90,C'/90,D'/90}
            \fill (\p) circle (1.2pt) node[shift={(\r:3mm)}]{$\p$};
        \end{tikzpicture}
    }
    \loigiai{
        Ta có $\overrightarrow{AB}+\overrightarrow{AD}=\overrightarrow{AC}$.
        \\
        Suy ra $\overrightarrow{AB} + \overrightarrow{AA'} + \overrightarrow{AD} = \overrightarrow{AC}+\overrightarrow{AA'}=\overrightarrow{AC'}$.
    }
\end{ex}

%%==========Câu 12
\begin{ex}%[Đề Minh Hoạ TNTHPT 2024-2025]%[2PhatTrien-DMH-2024-2025, GV: Hoàng Trọng Tấn]%[2D1N1-2]
    \immini{
        Cho hàm số có đồ thị như hình vẽ bên. Hàm số đã cho đồng biến trên khoảng nào sau đây?
        \choice
        {$(-\infty;1)$}
        {$(-\infty;-1)$}
        {\True $(-1;1)$}
        {$(1;+ \infty)$}	
    }
    {
        \begin{tikzpicture}[line join = round, line cap=round,>=stealth,font=\footnotesize,scale=0.7]
            \clip (-3,-2.6) rectangle (3,2.6);
            \draw[smooth,samples=100,domain=-3:3] plot(\x,{
                -(\x)^3+3*(\x)
            });
            \draw[->] (-2,0)--(2.7,0) node[above]{$x$};
            \draw[->] (0,-3)--(0,2.5) node[right]{$y$};
            \fill(0,0) circle(1.5pt) node[below left]{$O$};
            \fill 
            (1,0) circle(1.5pt) node[below]{$1$}
            (-1,0) circle(1.5pt) node[above]{$-1$}
            (0,2) circle(1.5pt) node[left]{$2$}
            (0,-2) circle(1.5pt) node[right]{$-2$}
            ;
            \draw[dashed] 
            (-1,0)--(-1,-2)--(0,-2)
            (1,0)--(1,2)--(0,2)
            ;
        \end{tikzpicture}
    }
    \loigiai{
        Quan sát hình vẽ từ trái sang phải ta thấy hình vẽ hướng lên, trên khoảng $(-1;1)$.
        \\
        Suy ra hàm số đã cho đồng biến trên $(-1;1)$.
    }
\end{ex}

\Closesolutionfile{ans}
\TNTF
\Opensolutionfile{ans}[ans/ans-DE-PNL-01-TF]




%%==========Câu 13
\begin{ex}%[Đề Minh Hoạ TNTHPT 2024-2025]%[2PhatTrien-DMH-2024-2025, GV: Hoàng Trọng Tấn]%[2D1H3-1]
    Cho hàm số $f(x) = 2\cos x + x$
    \choiceTF
    {\True $f(0) = 2$; $f\left(\dfrac{\pi}{2}\right) = \dfrac{\pi}{2}$}
    {$f'(x) = 2\sin x + 1$}
    {\True $f'(x) = 0$ có nghiệm trên đoạn $\left[0; \dfrac{\pi}{2}\right]$ là $\dfrac{\pi}{6}$}
    {\True Giá trị lớn nhất của $f(x)$ trên đoạn $\left[0; \dfrac{\pi}{2}\right]$ là $\sqrt{3} + \dfrac{\pi}{6}$}
    \loigiai{
        
        \begin{itemchoice}	
            \itemch \textbf{Đúng.}\\	Ta có $f(0)=2\cos 0 +0 =2$ và $f\left(\dfrac{\pi}{2}\right)=2\cdot 0+\dfrac{\pi}{2}$.
            \itemch \textbf{Sai.} \\	$f'(x)=-2\sin x +1$.
            \itemch  \textbf{Đúng.} \\	$f'(x)=0\Leftrightarrow -2\sin x +1 =0 \Leftrightarrow \sin x=\dfrac{1}{2}$.
            \\
            Mà $x\in \left[0;\dfrac{\pi}{2}\right]\Rightarrow x=\dfrac{\pi}{6}$.
            \itemch \textbf{Đúng.} \\Ta có $f\left(\dfrac{\pi}{6}\right)=\sqrt{3}+\dfrac{\pi}{6}>2$.
            Suy ra $\max\limits_{[0;\tfrac{\pi}{2}]}=\sqrt{3}+\dfrac{\pi}{6}$.
        \end{itemchoice}
    }
\end{ex}



%%==========Câu 14
\begin{ex}%[Đề Minh Hoạ TNTHPT 2024-2025]%[2PhatTrien-DMH-2024-2025, GV: Hoàng Trọng Tấn]%[2D4V2-6]
    Một người điều khiển ô tô đang ở đường dẫn muốn nhập làn vào đường cao tốc. Khi ô tô cách điểm nhập làn $200$ m, tốc độ của ô tô là $36$ km/h. Hai giây sau đó, ô tô bắt đầu tăng tốc với tốc độ $v(t) = at + b$ ($a$, $b \in \mathbb{R}$, $a > 0$), trong đó $t$ là thời gian tính bằng giây kể từ khi bắt đầu tăng tốc. Biết rằng ô tô nhập làn cao tốc sau $12$ giây và duy trì sự tăng tốc trong $24$ giây kể từ khi bắt đầu tăng tốc.
    \choiceTF
    {\True Quãng đường ô tô đi được từ khi bắt đầu tăng tốc đến khi nhập làn là $180$ m}
    {\True Giá trị của $b$ là $10$}
    {Quãng đường $S(t)$ (đơn vị: mét) mà ô tô đi được trong thời gian $t$ giây ($0 \leq t \leq 24$) kể từ khi tăng tốc được tính theo công thức $S(t) =\displaystyle \int_0^{24} v(t) \mathrm{\,d}t$}
    {Sau $24$ giây kể từ khi tăng tốc, tốc độ của ô tô không vượt quá tốc độ tối đa cho phép là $100$ km/h}
    \loigiai{
        \begin{center}
            \begin{tikzpicture}[line join = round, line cap=round,>=stealth,font=\footnotesize,scale=1]
                \draw 
                (0,0)node[above]{$A$} node[yshift=25pt,pos=0.5]{$v_0=36$ km/h}
                -- (3,0)node[above]{$B$} node[yshift=-10pt,pos=0.5]{$2$ giây} 
                node[xshift=20pt,yshift=25pt]{$v(t)=at+b$ (m/s)}--(8,0) node[above]{$C$}--(14,0) node[above]{$D$}
                ;
                \draw[<->] (3,-0.75)--(8,-0.75) node[below,pos=0.5]{$12$ giây}	;
                \draw[<->] (3,-1.5)--(14,-1.5) node[below,pos=0.5]{$24$ giây}	;
                %\draw[<->] (0,-2.5)--(3,-2.5) node[below,pos=0.5]{$S_1$ m}	;
                \draw[<->] (0,-2.5)--(8,-2.5) node[below,pos=0.5]{$200$ m}	;
            \end{tikzpicture}
        \end{center}
        
        \begin{itemchoice}
            \itemch \textbf{Đúng.}	\\ 
            Đổi đơn vị $36$ km/h bằng $10$ m/s.
            \\
            Quãng đường ô tô đi được $2$ giây đầu là $AB=10\cdot 2=20$ m. 
            \\
            Quãng đường ô tô đi được từ khi bắt đầu tăng tốc đến khi nhập làn là $BC=AC-AB=200-20=180$ m.
            
            \itemch \textbf{Đúng.}\\
            Khi $t=0$ thì $v_t=v_o=10\Rightarrow 10=a\cdot 0 +b\Rightarrow b=10$.
            
            \itemch\textbf{Sai.}\\
            Quãng đường đi được trong $t$ giây là $S(t)-S(0)=\displaystyle \int_0^t v(t) \mathrm{\,d}t$.
            
            \itemch \textbf{Sai.}
            \\
            Ta có $v(t)=at+10$.	Lại có 
            \begin{eqnarray*}
                BC=\displaystyle\int_0^{12} (at+10) \mathrm{\,d}t
                &\Leftrightarrow&	\left( \dfrac{at^2}{2}+10t\right)\Bigg|_0^{12}=180
                \\
                &\Leftrightarrow& 72a+120=180
                \\
                &\Leftrightarrow& a=\dfrac{5}{6}
            \end{eqnarray*}
            Vậy $v(t)=\dfrac{5}{6}t +10 \Rightarrow v(24)=30$ m/s $=108$ km/h. 
        \end{itemchoice}
    }
\end{ex}



%%==========Câu 15
\begin{ex}%[Đề Minh Hoạ TNTHPT 2024-2025]%[2PhatTrien-DMH-2024-2025, GV: Hoàng Trọng Tấn]%[2D6V2-3]
    Trước khi đưa một loại sản phẩm ra thị trường, người ta đã phỏng vấn ngẫu nhiên $200$ khách hàng về sản phẩm đó. Kết quả thống kê như sau: có $105$ người trả lời 
    \lq\lq  sẽ mua\rq\rq; có $95$ người trả lời \lq\lq  không mua\rq\rq.
    Kinh nghiệm cho thấy tỉ lệ khách hàng thực sự sẽ mua sản phẩm tương ứng với những câu trả lời \lq\lq  sẽ mua\rq\rq\, và \lq\lq  không mua\rq\rq\, lần lượt là $70\%$ và $30\%$. Gọi $A$ là biến cố \lq\lq  Người được phỏng vấn thực sự sẽ mua sản phẩm\rq\rq. Gọi $B$ là biến cố \lq\lq  Người được phỏng vấn trả lời sẽ mua sản phẩm\rq\rq.
    \choiceTF
    {\True $\mathrm{P}(B) = \dfrac{21}{40}$ và $\mathrm{P}(\overline{B}) = \dfrac{19}{40}$}
    {$\mathrm{P}(A \mid B) = 0{,}3$}
    {\True $\mathrm{P}(A) = 0{,}51$}
    {Trong số những người được phỏng vấn thực sự sẽ mua sản phẩm có $70\%$ người đã trả lời \lq\lq  sẽ mua\rq\rq\, khi được phỏng vấn (kết quả tính theo phần trăm được làm tròn đến hàng đơn vị)}
    \loigiai{			
        \begin{itemchoice}
            \itemch \textbf{Đúng}.\\
            Xác suất người được phỏng vấn thực sự sẽ mua sản phẩm $B$ và không mua sản phẩm $\overline{B} $ được cho như sau:
            $\mathrm{P}(B) = \dfrac{105}{200}=\dfrac{21}{40}$, $\mathrm{P}(\overline{B}) =1-\dfrac{21}{40}=\dfrac{19}{40}$.
            
            \itemch \textbf{Sai.}\\
            $\mathrm{P}(A\mid B)$ là xác suất có điều kiện, mô tả xác suất một người thực sự sẽ mua sản phẩm (biến cố $A$) khi biết rằng người đó đã trả lời \lq\lq  sẽ mua\rq\rq\, trong cuộc phỏng vấn (biến cố $B$).
            \\
            Vậy $\mathrm{P}(A\mid B)=70\%=0{,}7$.
            
            \itemch \textbf{Đúng.}\\
            Từ $\mathrm{P}(A\mid B)=0{,}7\Rightarrow P(A\mid \overline{B})=1-0{,}7=0{,}3$.
            \\
            Áp dụng công thức xác suất toàn phần ta có 
            $$
            \mathrm{P}(A) = \mathrm{P}(A \mid B) \cdot \mathrm{P}(B) + \mathrm{P}(A \mid \overline{B}) \cdot \mathrm{P}(\overline{B})=0{,}7 \cdot \dfrac{21}{40} + 0{,}3 \cdot \dfrac{19}{40}=0{,}51.
            $$
            
            \itemch \textbf{Sai.}\\
            Phát biểu này yêu cầu tính xác suất có điều kiện $\mathrm{P}(B\mid A)$, tức là xác suất một người đã trả lời \lq\lq  sẽ mua\rq\rq\, khi biết rằng người đó thực sự sẽ mua sản phẩm.
            \\
            Để tính $\mathrm{P}(B \mid A)$, ta áp dụng định lý Bayes:
            \begin{eqnarray*}
                && \mathrm{P}(B \mid A) = \dfrac{\mathrm{P}(A \mid B) \cdot \mathrm{P}(B)}{\mathrm{P}(A)}
                \\
                &\Leftrightarrow& \mathrm{P}(B \mid A) = \dfrac{0{,}7 \cdot \dfrac{21}{40}}{0{,}51}\approx 0{,}72=72\%.
            \end{eqnarray*}
            
        \end{itemchoice}
        
    }
\end{ex}

%%==========Câu 16
\begin{ex}%[Đề Minh Hoạ TNTHPT 2024-2025]%[2PhatTrien-DMH-2024-2025, GV: Hoàng Trọng Tấn]%[id6]
    \immini{
        Các thiên thạch có đường kính lớn hơn $140$ m và có thể lại gần Trái Đất ở khoảng cách nhỏ hơn $7\,500\,000$ km được coi là những vật thể có khả năng va chạm gây nguy hiểm cho Trái Đất. Để theo dõi những thiên thạch này, người ta đã thiết lập các trạm quan sát các vật thể bay gần Trái Đất. Giả sử có một hệ thống quan sát có khả năng theo dõi các vật thể ở độ cao không vượt quá $6\,600$ km so với mực nước biển. Coi Trái Đất là khối cầu có bán kính $6\,400$ km. Chọn hệ trục tọa độ $Oxyz$ trong không gian có gốc $O$ tại tâm Trái Đất và đơn vị độ dài trên mỗi trục tọa độ là $1\,000$ km. Một thiên thạch (coi như một hạt) chuyển động với tốc độ không đổi theo một đường thẳng từ điểm $M(6;20;0)$ đến điểm $N(-6;-12;16)$.
        
    }
    {
        \begin{tikzpicture}[line join = round, line cap=round,>=stealth,font=\footnotesize,transform shape,scale=0.8]
            \draw 
            (0,0) circle(3.5cm)
            ;
            \draw[fill=black!30] 
            (0,0) circle(1.6cm);
            \draw[<->] (0:3.5)--(0:1.6) node[pos=0.5,above]{$6\, 600$ km};
            \draw[<->] (0:0)--(0:1.6) node[pos=0.5,above]{$6\, 400$ km};
            \draw (130:3.5)coordinate (A)--(40:3.5)coordinate (B);
            \draw ($(A)!-0.4!(B)$) --(A)
            ($(A)!1.4!(B)$)--(B)
            ;
            
            \fill 
            (A)circle(1.5pt)node[above left]{$A$}
            (B)circle(1.5pt)node[above right]{$B$}
            ($(A)!-0.4!(B)$) circle(1.5pt)node[above]{$M$}
            ($(A)!1.4!(B)$)circle(1.5pt)node[above]{$N$}
            ;
            
        \end{tikzpicture}
    }
    \choiceTF
    {\True Đường thẳng $MN$ có phương trình tham số là $\heva{&x = 6 + 3t \\& y = 20 + 8t \\& z = -4t}$, $(t \in \mathbb{R})$}
    {Vị trí đầu tiên thiên thạch đi chuyển vào phạm vi theo dõi của hệ thống quan sát là điểm $A(-3; -4; 12)$}
    {\True Khoảng cách giữa vị trí đầu tiên và vị trí cuối cùng mà thiên thạch di chuyển trong phạm vi theo dõi của hệ thống quan sát  là $18\,900$ km (kết quả làm tròn đến hàng trăm đơn vị ki-lô-mét)}
    {\True Nếu thời gian di chuyển của thiên thạch trong phạm vi theo dõi của hệ thống quan sát là $3$ phút thì thời gian nó di chuyển từ $M$ đến $N$ là $6$ phút}
    \loigiai{
        \begin{itemchoice}
            \itemch \textbf{Đúng}.\\
            Ta có $\overrightarrow{MN}=(-12;-32;16)=-4\cdot (3;8;-4)$.
            \\
            Suy ra $MN$ có một véctơ chỉ phương là $\overrightarrow{u}=(3;8;-4)$.
            \\
            Phương trình tham số của đường thẳng $MN$ là $\heva{&x = 6 + 3t \\& y = 20 + 8t \\& z = -4t}$, $(t \in \mathbb{R})$.
            
            \itemch \textbf{Sai.}\\
            \begin{center}
                \begin{tikzpicture}[line join = round, line cap=round,>=stealth,font=\footnotesize,transform shape,scale=0.7]
                    \draw 
                    (0,0)coordinate (O) circle(3.5cm)
                    ;
                    \draw[fill=black!30] 
                    (0,0) circle(1.6cm);
                    \draw[<->] (0:3.5)--(0:1.6) node[pos=0.5,above]{$6\, 600$ km};
                    \draw[<->] (0:0)--(0:1.6) node[pos=0.5,above]{$6\, 400$ km};
                    \draw (130:3.5)coordinate (A)--(40:3.5)coordinate (B);
                    \path (0,0) node{\hypersetup{hidelinks}\href{HWKJsE1}{ }};
                    \draw ($(A)!-0.4!(B)$) --(A)
                    ($(A)!1.4!(B)$)--(B)
                    ;
                    
                    \fill 
                    (A)circle(1.5pt)node[above left]{$A$}
                    (B)circle(1.5pt)node[above right]{$B$}
                    ($(A)!-0.4!(B)$)coordinate (M) circle(1.5pt) circle(1.5pt)node[above]{$M$}
                    ($(A)!1.4!(B)$) node[above]{$N$}
                    ;
                    \draw 
                    (O)--($(A)!1/2!(B)$)coordinate(H) 
                    (O)--(A)
                    (O)--(M)
                    ; 
                    
                    
                    \foreach \p/\r in {O/-90,H/90}
                    \fill (\p) circle (1.2pt) node[shift={(\r:3mm)}]{$\p$};			
                    
                    \draw pic[draw,angle radius=3mm] {right angle = O--H--A}; 
                    
                \end{tikzpicture}
            \end{center}
            Do $H\in MN\Rightarrow H(6+3t;20+8t;-4t)$ suy ra $\overrightarrow{OH}=(6+3t;20+8t;-4t)$ 
            \\
            Hơn nữa
            \begin{eqnarray*}
                OH\perp MN &\Leftrightarrow& \overrightarrow{OH}\cdot \overrightarrow{u}=0
                \\
                &\Leftrightarrow& 3(6 + 3t) + 8(20 + 8t) + (-4)\cdot (-4t)=0
                \\
                &\Leftrightarrow&178 + 89t=0 \Leftrightarrow t=-2.
            \end{eqnarray*}
            Suy ra $H(0;4;8)$ và độ dài của $OH=\sqrt{4^2+8^2}=4\sqrt 5$ (nghìn km).
            \\
            Xét tam giác $OMH$
            \\ 
            $OM=\sqrt{6^2+20^2+0}=2\sqrt{109}$ (nghìn km).
            \\
            Suy ra $MH=\sqrt{OM^2-OH^2}=\sqrt{436-80}=2\sqrt{89}$ (nghìn km).
            \\
            Xét tam giác $OAH$, ta có $OA=6\,400+6\,600=13$ (nghìn km).
            \\
            $AH=\sqrt{OA^2-OH^2}=\sqrt{169-80}=\sqrt{89}$ (nghìn km).
            \\
            Suy ra $MH=2AH$ hay $A$ là trung điểm $MH$.
            \\
            Vậy toạ độ $A(3;12;4)$.
            
            
            \itemch \textbf{Đúng.}
            \\
            Độ dài đoạn $AB=2AH=2\sqrt{89}\approx 18{,}87$ nghìn km tức là $AH=18\, 870$ km $\approx 18\, 900$ km.
            
            \itemch \textbf{Đúng.}
            \\
            Ta có $3$ phút bằng $\dfrac{1}{20}$ giờ và $3$ phút bằng $\dfrac{1}{10}$ giờ.
            \\
            Vận tốc của thiên thạch $v=\dfrac{AB}{\dfrac{1}{20}}=378\,000$ (km/h) $=378$ (nghìn km/h).
            \\
            Độ dài của $MN=2MH=4\sqrt{89}$ (nghìn km).
            \\
            Thời gian đi từ $M$ đến $N$ là $\dfrac{4\sqrt{89}}{378}\approx 0{,}1$ h bằng $6$ phút.
        \end{itemchoice}
    }
\end{ex}
\Closesolutionfile{ans}
\TNSA
\Opensolutionfile{ans}[ans/ans-DE-PNL-01-SA]


%%==========Câu 17
\begin{ex}%[Đề Minh Hoạ TNTHPT 2024-2025]%[2PhatTrien-DMH-2024-2025, GV: Hoàng Trọng Tấn]%[id6]
    Cho hình lăng trụ đứng $ABC.A'B'C'$ có $AB = 5$, $BC = 6$, $CA = 7$. Khoảng cách giữa hai đường thẳng $AA'$ và $BC$ bằng bao nhiêu? (làm tròn kết quả đến hàng phần mười)
    \shortans[]{$4{,}9$}
    \loigiai{
        \begin{center}
            \begin{tikzpicture}[line join = round, line cap=round,>=stealth,font=\footnotesize,scale=1]
                \def \dai{4} 
                \def\rong{-1}
                \def\cao{3}
                \path 
                (0,0) coordinate (A) 
                (\dai,0) coordinate (B)
                (\dai/3,\rong) coordinate (C)
                ($(A)+(0,\cao)$) coordinate (A')
                ($(B)+(0,\cao)$) coordinate (B')
                ($(C)+(0,\cao)$) coordinate (C')
                ($(C)!0.3!(B)$) coordinate (H)
                ;
                \draw 
                (A')--(B')--(C')--cycle
                (A')--(A) (B')--(B) (C')--(C)
                (A)--(C)--(B)
                ;
                \draw pic[draw,angle radius=3mm] {right angle = A--H--B}; 
                \draw[dashed] (A)--(B) (A)--(H);
                \foreach \p/\r in {A/-120,B/-40,C/-90,A'/120,B'/40,C'/90,H/-90}
                \fill (\p) circle (1.2pt) node[shift={(\r:3mm)}]{$\p$};
                
                
            \end{tikzpicture}
        \end{center}	
        Kẻ đường cao $AH$ của tam giác $ABC$.
        \\
        Suy ra $AH\perp BC$ hơn nữa $AA'\perp (ABC)$ mà $AH\subset (ABC)\Rightarrow AA'\perp AH$.
        \\
        Suy ra $AH$ là đoạn vuông góc chung của $AA'$ và $BC$ hay $d[AA',BC]=AH$.
        \\
        Nửa chu vi tam giác $ABC$ là $\mathrm{P}=\dfrac{5+6+7}{2}=9$.
        \\
        Áp dụng công thức Hê Rông ta có diện tích tam giác $ABC$ là
        $$
        S=\sqrt{p(p-5)(p-6)(p-7)}=6\sqrt 6.
        $$
        Suy ra độ dài đường cao $AH$ là $\dfrac{2\cdot 6\sqrt 6}{6}=2\sqrt 6\approx 4{,}9$.
    }
\end{ex}

%%==========Câu 18
\begin{ex}%[Đề Minh Hoạ TNTHPT 2024-2025]%[2PhatTrien-DMH-2024-2025, GV: Hoàng Trọng Tấn]%[id6]
    \immini{
        Một trò chơi điện tử quy định như sau: Có $4$ trụ $A$, $B$, $C$, $D$ với số lượng các thử thách trên đường đi giữa các cặp trụ được mô tả trong hình bên. Người chơi xuất phát từ một trụ nào đó, đi qua tất cả các trụ còn lại, mỗi khi đi qua một trụ thì trụ đó sẽ bị phá hủy và không thể quay trở lại trừ khi trò chơi kết thúc. Người chơi vẫn phải trở về trụ ban đầu. Tổng số thử thách của đường đi thỏa mãn điều kiện trên nhận giá trị nhỏ nhất là bao nhiêu?
    }
    {
        \begin{tikzpicture}[line join = round, line cap=round,>=stealth,font=\footnotesize,scale=1]
            \path 
            (0,0) coordinate (B)
            (0.5,3) coordinate (A)
            (4,0) coordinate (C)
            (1,1)coordinate (D)
            ;
            \draw (A)--(B)--(C)--cycle
            (A)--(D) (B)--(D) (C)--(D)
            ;
            \path 
            (A)--(B)node[left,pos=0.5]{$10$}
            (A)--(C)node[right,pos=0.5]{$11$}
            (C)--(B)node[below,pos=0.5]{$12$}
            (A)--(D)node[right,pos=0.5]{$9$}
            (B)--(D)node[below,xshift=7pt,yshift=2pt,pos=0.5]{$11$}
            (C)--(D)node[above,pos=0.5]{$14$}
            ;
            \foreach \p/\r in {A/90,B/-140,C/0,D/40}
            \fill (\p) circle (1.2pt) node[shift={(\r:3mm)}]{$\p$};
            
            
        \end{tikzpicture}
    }
    \shortans{$43$}
    \loigiai{
        Do $AD$ có $9$ thử thách là đường đi có số thử thách ít nhất nên ta sẽ bắt đầu từ $A$ đến $D$.
        \\
        Do $DB$ có $11$ thử thách và $DC$ có $14$ thử thách nên ta sẽ đi từ $D$ đến $B$.
        \\
        Lúc này trụ $A$ và $D$ đã bị phá huỷ nên ta đi từ $B$ đến $C$ với số thử thách là $12$.
        \\
        Cuối cùng đi từ $C$ đến $A$ với số thử thách là $11$.
        \\
        Vậy tổng cộng có $9+11+12+11=43$ thử thách.
    }
\end{ex}

%%==========Câu 19
\begin{ex}%[Đề Minh Hoạ TNTHPT 2024-2025]%[2PhatTrien-DMH-2024-2025, GV: Hoàng Trọng Tấn]%[id6]
    Hệ thống định vị toàn cầu GPS là một hệ thống cho phép xác định vị trí của một vật thể trong không gian. Trong cùng một thời điểm, vị trí của một điểm $M$ trong không gian sẽ được xác định bởi bốn vệ tinh cho trước nhờ các bộ thu phát tín hiệu đặt trên các vệ tinh. Giả sử trong không gian với hệ tọa độ $Oxyz$, có bốn vệ tinh lần lượt đặt tại các điểm $A(3;1;0)$, $B(3;6;6)$, $C(4;6;2)$, $D(6;2;14)$; vị trí $M(a;b;c)$ thỏa mãn $MA = 3$, $MB = 6$, $MC = 5$, $MD = 13$. Khoảng cách từ điểm $M$ đến điểm $O$ bằng bao nhiêu?
    \shortans{3}
    \loigiai{
        Giả sử $M(a;b;c)$. Ta có
        \begin{eqnarray*}
            MA=3&\Leftrightarrow& \sqrt{(a-3)^2+(b-1)^2+c^2}=3
            \\
            MB=6&\Leftrightarrow&\sqrt{(a-3)^2+(b-6)^2+(c-6)^2}=6
            \\
            MC=5&\Leftrightarrow&\sqrt{(a-4)^3+(b-6)^2+(c-2)^2}=5
            \\
            MD=13&\Leftrightarrow&\sqrt{(a-6)^3+(b-2)^2+(c-14)^2}=13
        \end{eqnarray*}
        Ta có hệ phương trình $\heva{
            & a^2 + b^2 + c^2 - 6a - 2b + 1 = 0\\
            & a^2 + b^2 + c^2 - 6a - 12b - 12c + 45 = 0 \\
            & a^2 + b^2 + c^2 - 8a - 12b - 4c + 31 = 0 \\
            & a^2 + b^2 + c^2 - 12a - 4b - 28c + 67 = 0.
        }$
        \\
        Giữ nguyên phương trình thứ nhất, lấy phương trình thứ nhất trừ vế theo vế với các phương trình còn lại ta được hệ phương trình mới như sau 
        \\
        $\heva{
            & a^2 + b^2 + c^2 - 6a - 2b + 1 = 0 \\
            & 10b + 12c = 44 \\
            & 2a + 10b + 4c = 30 \\
            & 6a + 2b + 28c = 66
        }\Leftrightarrow \heva{
            & a^2 + b^2 + c^2 - 6a - 2b + 1 = 0 \\
            & a = 1 \\
            & b = 2 \\
            & c = 2.
        }$
        \\
        Thế $a=1$, $b=2$, $c=2$ vào phương trình thứ nhất ta thấy thoả mãn. 
        \\
        Vậy điểm $M(1;2;2)\Rightarrow OM=\sqrt{1+4+4}=3$.
    }
\end{ex}

%%==========Câu 20
\begin{ex}%[Đề Minh Hoạ TNTHPT 2024-2025]%[2PhatTrien-DMH-2024-2025, GV: Hoàng Trọng Tấn]%[id6]
    \immini{
        Kiến trúc sư thiết kế một khu sinh hoạt cộng đồng có dạng hình chữ nhật với chiều rộng và chiều dài lần lượt là $60$ m và $80$ m. Trong đó, phần được tô màu đậm là sân chơi, phần còn lại để trồng hoa. Mỗi phần trồng hoa có đường biên cong là một phần của parabola với đỉnh thuộc một trục đối xứng của hình chữ nhật và có khoảng cách từ đường biên cong đến trục đối xứng bằng $20$ m (xem hình minh họa). Diện tích của phần sân chơi là bao nhiêu mét vuông?
    }
    {
        \begin{tikzpicture}[line join = round, line cap=round,>=stealth,font=\footnotesize,scale=1]
            \draw (0,0) rectangle (3,5);
            \draw[fill=black!40] (0,0)..controls +(60:2) and +(120:2)..(3,0)--(3,5)
            ..controls +(-120:2) and +(-60:2).. (0,5)--cycle
            ;
            
            \draw[<->] (1.5,0)--(1.5,1.3)node[right,pos=0.5]{$20$ m};
            \draw[<->] (1.5,5)--(1.5,3.7)node[right,pos=0.5]{$20$ m};
            
            %\draw[->,line width=1pt] (1.5,0)--(1.5,6)node[right]{$y$};
            %\draw[->,line width=1pt] (0,0)--(4,0)node[above]{$x$};
            %\fill (1.5,0) circle(1.2pt) node[below]{$O$};
        \end{tikzpicture}
    }
    \shortans{3200}
    \loigiai{
        \immini{
            Dựng hệ trục $Oxy$ như hình vẽ, dễ thấy Parabola $(P)$ có phương trình
            $(P)\colon y=ax^2+b$.
            \\
            Đồng thời $(P)$ đi qua điểm $(30;0)$ và $(0;20)$ nên ta có hệ phương trình $\heva{&0=a\cdot 30^2+b\\&20=a\cdot 0^2+b}\Leftrightarrow \heva{&a=-\dfrac{1}{45}\\&b=20.}$
            \\
            Suy ra $(P)\colon y=-\dfrac{1}{45}x^2+20$.
            \\
            Diện tích một nửa phần trồng hoa là 
            $$
            \displaystyle \int_{-30}^{30} \left(-\dfrac{1}{45}x^2+20\right) \mathrm{\,d}x=800.
            $$
            Vậy diện tích phần sân chơi là $60\times 80 -800 \cdot 2=3\,200$ (m$^2$).
        }
        {
            \begin{tikzpicture}[line join = round, line cap=round,>=stealth,font=\footnotesize,scale=0.8]
                \draw (0,0) rectangle (3,5);
                \draw[fill=black!40] (0,0)..controls +(60:2) and +(120:2)..(3,0)--(3,5)
                ..controls +(-120:2) and +(-60:2).. (0,5)--cycle
                ;
                \draw[blue,line width=1pt] (0,0)..controls +(60:2) and +(120:2)..(3,0);
                \draw (2.5,1.5)node[]{$(P)$};
                \draw[] (1.5,0)--(1.5,1.3)node[right,pos=0.5]{$20$};
                \draw[] (1.5,5)--(1.5,3.7)node[right,pos=0.5]{$20$};
                
                \draw[->,line width=1pt] (1.5,0)--(1.5,6)node[right]{$y$};
                \draw[->,line width=1pt] (0,0)--(4,0)node[above]{$x$};
                \fill (1.5,0) circle(1.2pt) node[below]{$O$};
            \end{tikzpicture}
        }
    }
\end{ex}

%%==========Câu 21
\begin{ex}%[Đề Minh Hoạ TNTHPT 2024-2025]%[2PhatTrien-DMH-2024-2025, GV: Hoàng Trọng Tấn]%[id6]
    Một doanh nghiệp dự định sản xuất không quá $500$ sản phẩm. Nếu doanh nghiệp sản xuất $x$ sản phẩm ($1 \leq x \leq 500$) thì doanh thu nhận được khi bán hết số sản phẩm đó là $F(x) = x^3 - 1\,999x^2 + 1\,001\,000x + 250\,000$ đồng, trong khi chi phí sản xuất bình quân cho một sản phẩm là $G(x) = x + 1\,000 + \dfrac{250\,000}{x}$ (đồng). Doanh nghiệp cần sản xuất bao nhiêu sản phẩm để lợi nhuận thu được là lớn nhất?
    \shortans[]{$333$}
    \loigiai{
        Chi phí sản xuất cho $x$ sản phẩm là $xG(x)=x^2+1\,000x+250\,000$ (đồng).
        \\
        Lợi nhuận thu được $L(x)=F(x)-xG(x)= x^3-2\,000x^2+1\,000\,000x$ (đồng).
        \\
        Đạo hàm $L'(x)=3x^2-4\,000x+1\,000\,000$, giải phương trình $L'(x)=0$
        \\
        $3x^2-4\,000x+1\,000\,000=0\Leftrightarrow x=1000$ hoặc $x=\dfrac{2\,000}{6}$.
        \\
        Do $1\le x \le 500$ nên $x=\dfrac{2\,000}{6}$.
        \\
        Mà $333<x=\dfrac{2\,000}{6}<334$.
        \\
        Ta có $L(333)=148\, 148\, 037$, $L(334)=148\,147\,704$, $L(1)=998\,001$ và $L(500)=125\,000\,000$.
        \\
        Vậy doanh nghiệp cần sản xuất $333$ sản phẩm để lợi nhuận thu được là lớn nhất.
    }
\end{ex}

%%==========Câu 22
\begin{ex}%[Đề Minh Hoạ TNTHPT 2024-2025]%[2PhatTrien-DMH-2024-2025, GV: Hoàng Trọng Tấn]%[id6]
    Có hai chiếc hộp, hộp I có $6$ quả bóng màu đỏ và $4$ quả bóng màu vàng, hộp II có $7$ quả bóng màu đỏ và $3$ quả bóng màu vàng, các quả bóng có cùng kích thước và khối lượng. Lấy ngẫu nhiên một quả bóng từ hộp I bỏ vào hộp II. Sau đó, lấy ra ngẫu nhiên một quả bóng từ hộp II. Tính xác suất để quả bóng được lấy ra từ hộp II là quả bóng được chuyển từ hộp I sang, biết rằng quả bóng đó có màu đỏ (làm tròn kết quả đến hàng phần trăm).
    \shortans[]{0,21}
    \loigiai{
        Gọi $A$ là biến cố \lq\lq  quả lấy ra ở II là quả bóng được đưa từ I vào\rq\rq.
        \\
        Gọi $B$ là biến cố \lq\lq  quả bóng lấy ra ở II là đỏ\rq\rq.
        \\
        $\mathrm{P}(B)$ xảy ra theo $2$ trường hợp:
        \\
        \textbf{TH1:} Chuyển một quả đỏ từ I sang II xác suất trường hợp này là $\dfrac{6}{10}\cdot \dfrac{8}{11}$.
        \\
        \textbf{TH2:} Chuyển một quả vàng từ I sang II xác suất trường hợp này là $\dfrac{4}{10}\cdot \dfrac{7}{11}$.
        \\
        Suy ra $\mathrm{P}(B)=\dfrac{6}{10}\cdot \dfrac{8}{11}+\dfrac{4}{10}\cdot \dfrac{7}{11}=\dfrac{38}{55}$.
        \\
        $A\cap B$ là biến cố \lq\lq  quả bóng lấy ra ở II là đỏ và nó là quả bóng thuộc I\rq\rq.
        \\
        Phép thử gồm $2$ hành động: lấy $1$ quả ở I đưa vào II và từ II lấy $1$ quả.
        \\
        Không gian mẫu có $10\cdot 11=110$ kết quả.
        \\
        $A\cap B$ có số kết quả thuận lợi là $6\cdot 1=6$ kết quả.
        \\
        Suy ra $\mathrm{P}(A\cap B)=\dfrac{6}{110}$.
        \\
        Theo định lý Bayes ta có $\mathrm{P}(A\mid B)=\dfrac{\mathrm{P}(A\cap B)}{\mathrm{P}(B)}=\dfrac{\dfrac{6}{110}}{\dfrac{38}{55}}\approx 0{,}08$.
    }
\end{ex}
\Closesolutionfile{ans}
\Closesolutionfile{ansbook}
\inputansbox{6,2,3}{ans/ans-DE-PNL-01-T,ans/ans-DE-PNL-01-TF,ans/ans-DE-PNL-01-SA}


% %=================================
\begin{name}
	{\tenchude}
	{\tendethi}
	{\tentruong}
	{\thoigian}
\end{name}
\Opensolutionfile{ansbook}[ans/ansbook-DE-PNL-02]
\TN
\Opensolutionfile{ans}[ans/ans-DE-PNL-02-T]
\begin{ex}%Câu 1
	Cho hàm số $y=f(x)$ có bảng biến thiên như sau:
	\begin{center}
		\begin{tikzpicture}[>=stealth]
			\tkzTabInit[espcl=2.5,lgt=1.5,nocadre=false]
			{$x$/0.7,$f'(x)$/0.7,$f(x)$/2.1}
			{$-\infty$,$-1$,$1$,$+\infty$}
			\tkzTabLine{,+,0,-,0,+,}
			\tkzTabVar{-/$-\infty$,+/$2$,-/$-2$,+/$+\infty$}
		\end{tikzpicture}
	\end{center}
	Hàm số đã cho nghịch biến trên khoảng nào dưới đây?
	\choice
	{$\left(-2;2\right)$}
	{\True $\left(-1;1\right)$}
	{$\left(-2;1\right)$}
	{$\left(-1;+\infty\right)$}
	\loigiai{
		Từ bảng biến thiên ta suy ra hàm số nghịch biến trên khoảng $\left(-1;1\right)$.}
\end{ex}

\begin{ex}%Câu 2
	Cho cấp số nhân $\left(u_n\right)$ với $u_1=6$ và $u_2=-12.$ Công bội $q$ của cấp số nhân đã cho là
	\choice
	{$q=-\dfrac{1}{2}$}
	{\True $q=-2$}
	{$q=-18$}
	{$q=-6$}
	\loigiai{
		Ta có $u_2=u_1.q\Leftrightarrow-12=6.q\Leftrightarrow q=-2$.}
\end{ex}

\begin{ex}%Câu 3
	Trong không gian $Oxyz$, cho hai điểm $A\left(1;1;-2\right)$ và $B\left(3;-1;2\right)$. Tọa độ của vectơ $\overrightarrow{BA}$ là
	\choice
	{$\left(2;-2;4\right)$}
	{$\left(2;0;0\right)$}
	{$\left(1;-1;2\right)$}
	{\True $~\left(-2;2;-4\right)$}
	\loigiai{
		Tọa độ của vectơ $\overrightarrow{BA}$ là: $\overrightarrow{BA}=\left(x_A-x_B;y_A-y_B;z_A-z_B\right)=\left(-2;2;-4\right)$ .}
\end{ex}

\begin{ex}%Câu 4
	Tính thể tích vật thể tròn xoay khi quay hình phẳng giới hạn bởi các đường cong $y=\sqrt{e^x-x},$ $y=0$, $x=1$, $x=2$ xung quanh trục $Ox$ là
	\choice
	{\True $\pi\left(e^2-e-\dfrac{3}{2}\right)$}
	{$e^2-e-\dfrac{5}{2}$}
	{$\pi\left(e^2-e-\dfrac{5}{2}\right)$}
	{$e^2-e-\dfrac{3}{2}$}
	\loigiai{
		Hàm số $y=\sqrt{e^x-x}$ liên tục và không âm trên đoạn $\left[1;2\right]$ nên thể tích vật thể tròn xoay khi quay hình phẳng giới hạn bởi các đường $y=\sqrt{e^x-x},$ $y=0,x=1,x=2$ xung quanh trục Ox là:\\
		$V=\pi\displaystyle\int\limits_1^2\left(\sqrt{e^x-x}\right)^2dx=\pi\displaystyle\int\limits_1^2\left(e^x-x\right)dx=\left.\pi\left(e^x-\dfrac{1}{2}{x^2}\right)\right|_1^2$ $=\pi\left[\left(e^2-\dfrac{1}{2}{2^2}\right)-\left(e^1-\dfrac{1}{2}{1^2}\right)\right]=\pi\left(e^2-e-\dfrac{3}{2}\right)$ .}
\end{ex}

\begin{ex}%Câu 5
	Với mọi số thực dương a, $\log_3\left(27a\right)-\log_3a$ bằng
	\choice
	{$\log_3\left(26a\right)$}
	{$9$}
	{\True 3}
	{$3-2\log_3a$}
	\loigiai{
		Ta có: $\log_3\left(27a\right)-\log_3a=\log_327+\log_3a-\log_3a=3$ .}
\end{ex}

\begin{ex}%Câu 6
	Trong không gian $Oxyz$ , cho đường thẳng $d:\dfrac{x-1}{4}=\dfrac{-y}{2}=\dfrac{z+2}{-6}$. Vectơ nào dưới đây là một vectơ chỉ phương của $d$?
	\choice
	{$\overrightarrow{u_2}=\left(2;-1;3\right)$}
	{$\overrightarrow{u_1}=\left(4;2;-6\right)$}
	{\True $\overrightarrow{u_3}=\left(-2;1;3\right)$}
	{$\overrightarrow{u_4}=\left(1;0;2\right)$}
	\loigiai{
		Phương trình đường thẳng $d:\dfrac{x-1}{4}=\dfrac{-y}{2}=\dfrac{z+2}{-6}$ được viết lại $\dfrac{x-1}{4}=\dfrac{y}{-2}=\dfrac{z+2}{-6}$\\
		Suy ra một vectơ chỉ phương của $d$ là $\overrightarrow{u_3}=\left(-2;1;3\right)$.}
\end{ex}

\begin{ex}%Câu 7
	Tiệm cận ngang của đồ thị hàm số $y=\dfrac{2x-3}{x+1}$ là đường thẳng có phương trình:
	\choice
	{$y=-1$}
	{$x=-1$}
	{\True $y=2$}
	{$x=2$}
	\loigiai{
		Tập xác định: $D=\mathbb{R}\setminus\left\{-1\right\}$ thì ta có $\underset{x\to+\infty}{\mathop{\lim y}}=\underset{x\to+\infty}{\lim}\,\dfrac{2x-3}{x+1}=2;\underset{x\to-\infty}{\mathop{\lim y}}\,=\underset{x\to-\infty}{\lim}\,\dfrac{2x-3}{x+1}=2\,$ .\\
		Suy ra $y=2$ là tiệm cận ngang của đồ thị hàm số.}
\end{ex}

\begin{ex}%Câu 8
	Trong không gian $Oxyz,$ cho mặt cầu $(S)$ có tâm $I\left(0;-2;1\right)$ và bán kính $R=5$. Phương trình của $(S)$ là
	\choice
	{\True $x^2+\left(y+2\right)^2+\left(z-1\right)^2=25$}
	{$x^2+\left(y-2\right)^2+\left(z+1\right)^2=25$}
	{$x^2+\left(y+2\right)^2+\left(z-1\right)^2=5$}
	{$x^2+\left(y-2\right)^2+\left(z+1\right)^2=5$}
	\loigiai{
		Mặt cầu tâm $I\left(a;b;c\right)$ và bán kính $R$ có phương trình là $\left(x-a\right)^2+\left(y-b\right)^2+\left(z-c\right)^2=R^2$.}
\end{ex}

\begin{ex}%Câu 9
	Công thức tính thể tích của một khối trụ có bán kính đáy là $R$ và chiều cao $h$ là
	\choice
	{$V=2\pi{R^2}h$}
	{$V=\dfrac{4}{3}\pi{R^2}h$}
	{$V=\dfrac{1}{3}\pi{R^2}h$}
	{\True $V=\pi{R^2}h$}
	\loigiai{
		Công thức tính thể tích khối trụ là $V=\pi{R^2}h$}
\end{ex}

\begin{ex}%Câu 10
	Diện tích hình phẳng gạch sọc trong hình vẽ bên dưới bằng
	\begin{center}
		\begin{tikzpicture}[>=stealth,thick]
			\tikzset{every node/.style={scale=0.9},y=.7cm}
			\draw[->] (-3.1,0)--(5.1,0) node[below left] {$x$};
			\draw[->] (0,-1.1)--(0,9.1) node[below left] {$y$};
			\draw (0,0) node [below left] {$O$};
			\foreach \x/\nx in {1/1,3/3}
			\draw[thin] (\x,1pt)--(\x,-1pt) node [below] {$\nx$};
			\foreach \y/\ny in {2/2,8/8}
			\draw[thin] (1pt,\y)--(-1pt,\y) node [left] {$\ny$};
			\draw[dashed,thin](1,0)--(1,2)--(0,2) (1,2)--(3,2);
			\draw[dashed,thin](3,0)--(3,8)--(0,8);
			\draw[pattern=vertical lines,pattern color=red] (1,2)--(3,2)--(3,8)--plot[samples=200,domain=3:1,smooth](\x,{2^(\x)});
			\draw (2,6) node{$y=2^x$};
			\begin{scope}
				\clip (-3,-1) rectangle (5,9);
				\draw[samples=200,domain=-3:3.1,smooth] plot (\x,{2^(\x)});
			\end{scope}
		\end{tikzpicture}
	\end{center}
	\choice
	{\True $\displaystyle\int\limits_1^3\left(2^x-2\right)\,\mathrm{\,d}x$}
	{$\displaystyle\int\limits_1^3\left(2^x+2\right)\,\mathrm{\,d}x$}
	{$\displaystyle\int\limits_1^3\left(2-2^x\right)\,\mathrm{\,d}x$}
	{$\displaystyle\int\limits_1^32^x\,\mathrm{\,d}x$}
	\loigiai{
		Hình phẳng gạch sọc trong hình vẽ được giới hạn bởi các đường $y=2^x,y=2,x=1$ và $x=3$ .\\
		Do đó diện tích hình phẳng gạch sọc trong hình vẽ bằng $\displaystyle\int\limits_1^3\left|2^x-2\right|\,\mathrm{\,d}x=\displaystyle\int\limits_1^3\left(2^x-2\right)\,\mathrm{\,d}x$.}
\end{ex}

\begin{ex}%Câu 11
	Cho tứ diện $ABCD.$ Gọi $M$ và $P$ lần lượt là trung điểm của các cạnh $AB$ và $CD.$ Đặt $\overrightarrow{BA}=\overrightarrow{b}$,  $\overrightarrow{AC}=\overrightarrow{c}$,  $\overrightarrow{AD}=\overrightarrow{d}$. Khẳng định nào sau đây là đúng?
	\choice
	{\True $\overrightarrow{MP}=\dfrac{1}{2}\left(\overrightarrow{c}+\overrightarrow{d}+\overrightarrow{b}\right)$}
	{$\overrightarrow{MP}=\dfrac{1}{2}\left(\overrightarrow{d}+\overrightarrow{b}-\overrightarrow{c}\right)$}
	{$\overrightarrow{MP}=\dfrac{1}{2}\left(\overrightarrow{c}+\overrightarrow{b}-\overrightarrow{d}\right)$}
	{$\overrightarrow{MP}=\dfrac{1}{2}\left(\overrightarrow{c}+\overrightarrow{d}-\overrightarrow{b}\right)$}
	\loigiai{
		\immini{
			\begin{align*}
				\overrightarrow{MP} & = \dfrac{1}{2}\left(\overrightarrow{MC}+\overrightarrow{MD}\right)                                        \\
				                    & =\dfrac{1}{2}\left(\overrightarrow{AC}-\overrightarrow{AM}+\overrightarrow{AD}-\overrightarrow{AM}\right) \\
				                    & =\dfrac{1}{2}\left(\overrightarrow{AC}+\overrightarrow{AD}-2\overrightarrow{AM}\right)                    \\
				                    & =\dfrac{1}{2}\left(\overrightarrow{AC}+\overrightarrow{AD}-\overrightarrow{AB}\right)                     \\
				                    & =\dfrac{1}{2}\left(\overrightarrow{c}+\overrightarrow{d}+\overrightarrow{b}\right).
			\end{align*}
		}
		{\begin{tikzpicture}[line join = round, line cap = round, thick, font = \small, scale = .7]
				\path
				(0:0) coordinate (B)
				+(0:5) coordinate (C)
				+(-70:3) coordinate (D)
				+(70:4) coordinate (A)
				($(A)!.5!(B)$) coordinate (M)
				($(C)!.5!(D)$) coordinate (P);
				\draw[dashed]
				(B)--(C) (M)--(P)
				;
				\draw
				(A)--(B)--(D)--(C)--cycle
				(A)--(D)
				;
				\foreach \x/\g in {B/-135,C/-45,D/-45,A/135,M/135,P/-45}
				\fill (\x) circle (1.5pt)
				+(\g:3mm) node {$\x$};
			\end{tikzpicture}}
	}
\end{ex}

\begin{ex}%Câu 12
	Thống kê điểm trung bình môn Toán của một số học sinh lớp $12$ được mẫu số liệu sau:\\
	\centerline{\begin{tabular}{|c|c|c|c|c|c|c|c|}
			\hline
			Khoảng điểm & $\left[6,5;7\right)$ & $\left[7;7,5\right)$ & $\left[7,5;8\right)$ & $\left[8;8,5\right)$ & $\left[8,5;9\right)$ & $\left[9;9,5\right)$ & $\left[9,5;10\right)$ \\
			\hline
			Tần số      & $8$                  & $10$                 & $16$                 & $24$                 & $13$                 & $7$                  & $4$                   \\
			\hline
		\end{tabular}}\\
	Phương sai của mẫu số liệu về điểm trung bình môn Toán của các học sinh đó là
	\choice
	{$0{,}616$}
	{$0{,}785$}
	{$0{,}78$}
	{\True $0{,}609$}
	\loigiai{
		Cỡ mẫu $n=8+10+16+24+13+7+4=82$ .\\
		Số trung bình của mẫu số liệu ghép nhóm là\\
		$$\bar{x}=\dfrac{8.6,75+10.7,25+16.7,75+24.8,25+13.8,75+7.9,25+4.9,75}{82}=\dfrac{333}{41}$$
		Phương sai của mẫu số liệu ghép nhóm là:\\
		$$s^2=\dfrac{1}{82}\left(8.6,75^2+10.7,25^2+16.7,75^2+24.8,25^2+13.8,75^2+7.9,25^2+4.9,75^2\right)-\left(\dfrac{333}{41}\right)^2 \approx 0,609$$.}
\end{ex}

\Closesolutionfile{ans}
\TNTF
\Opensolutionfile{ans}[ans/ans-DE-PNL-02-TF]

\begin{ex}%Câu 13
	Vẽ bảng biến thiên hàm số $f(x)=4\sin x\cos x+2x$ trên $\left[-\pi ;\pi\right]$.
	\choiceTF
	{Đạo hàm của hàm số đã cho là $f'(x)=4\sin 2x+2$}
	{\True Hàm số $y=f(x)$ có $4$ điểm cực trị thuộc $\left[-\pi ;\pi\right]$}
	{Hàm số $y=f(x)$ nghịch biến trên khoảng $\left(-2;-1\right)$}
	{\True Giá trị lớn nhất của $f(x)$ trên đoạn $\left[0;\dfrac{\pi}{2}\right]$ là $\dfrac{2\pi}{3}+\sqrt{3}$}
	\loigiai{
		\begin{itemchoice}
			\itemch $f(x)=4\sin x\cos x+2x=2\sin 2x+2x$ $\Rightarrow{f}'(x)=4\cos 2x+2$
			\itemch $f'(x)=0\Leftrightarrow\cos 2x=-\dfrac{1}{2}
				\Leftrightarrow\hoac{ & 2x=\dfrac{2\pi}{3}+k2\pi\\ & 2x=-\dfrac{2\pi}{3}+k2\pi}
				\Leftrightarrow \hoac{ & x=\dfrac{\pi}{3}+k\pi\\ & x=-\dfrac{\pi}{3}+k\pi},k\in\mathbb{Z}.$\\
			Trên đoạn $\left[-\pi ;\pi\right],f'(x)=0\Leftrightarrow x\in\left\{\dfrac{\pi}{3};-\dfrac{\pi}{3};\dfrac{2\pi}{3};-\dfrac{2\pi}{3}\right\}$. Vẽ bảng biến thiên của hàm số $y=f(x)$ trên đoạn $\left[-\pi ;\pi\right]$.
			\begin{center}
				\begin{tikzpicture}[>=stealth]
					\tkzTabInit[espcl=2.5,lgt=1.5,nocadre=false]
					{$x$/0.7,$f'(x)$/0.7,$f(x)$/2.1}
					{$-\pi$,$-\dfrac{2\pi}{3}$,$-\dfrac{\pi}{3}$,$\dfrac{\pi}{3}$,$\dfrac{2\pi}{3}$,$\pi$}
					\tkzTabLine{,+,0,-,0,+,0,-,0,+,}
					\tkzTabVar{-/ , +/ , -/ , +/ , -/ , +/ }
				\end{tikzpicture}
			\end{center}
			Từ bảng biến thiên ta thấy trên đoạn $\left[-\pi ;\pi\right]$ hàm số có 4 điểm cực trị.
			\itemch Nhận thấy $-\dfrac{2\pi}{3}< -2 < -\dfrac{\pi}{3} < -1$ nên từ bảng biến thiên trên ta có hàm số $y=f(x)$ không nghịch biến trên khoảng $\left(-2;-1\right)$.
			\itemch Dựa vào ta có $\max \limits _ {\left[0;\frac{\pi}{2}\right]} = f\left(\dfrac{\pi}{3}\right) = \dfrac{2\pi}{3}+\sqrt{3}$.
		\end{itemchoice}
	}
\end{ex}

\begin{ex}%Câu 14
	Một người điều khiển ô tô đang ở đường dẫn muốn nhập làn vào đường cao tốc. Khi ô tô cách điểm nhập làn $200$ m thì tốc độ của ô tô là $36$ (km/h). Hai giây sau đó, ô tô bắt đầu tăng tốc với tốc độ $v(t)=at+b$ ($a,b\in\mathbb{R},a>0$), trong đó $t$ là thời gian tính bằng giây kể từ khi bắt đầu tăng tốc. Biết rằng ô tô nhập làn cao tốc sau $12$ giây và duy trì sự tăng tốc trong $24$ giây kể từ khi bắt đầu tăng tốc. Sau $24$ giây đó ô tô duy trì tốc độ cao nhất trong thời gian còn lại trên cao tốc.\\
	\centerline{\includegraphics[height=5cm]{images/2.14}}
	\choiceTF
	{\True Quãng đường ô tô đi được từ khi bắt đầu tăng tốc đến khi nhập làn là $180$ m}
	{Vận tốc của ô tô tại thời điểm nhập làn là $72$ (km/h)}
	{Quãng đường mà ô tô đi được trong thời gian $30$ giây kể từ khi ô tô cách điểm nhập làn $200$ m là $620$ m}
	{Sau 24 giây kể từ khi tăng tốc, ô tô duy trì tốc độ cao nhất trong vòng $5$ giây thì phát hiện chướng ngoại vật cách đó $300$ m. Người điều khiển lập tức đạp phanh và ô tô chuyển động chậm dần đều với $a(t)=-3$ (m/s$^2$). Khi đó ô tô dừng lại cách chứng ngại vật $10$ m}
	\loigiai{
		\begin{itemchoice}
			\itemch Ta có $36$ (km/h)$=10$ (m/s).\\
			Vận tốc ô tô lúc đầu là $10$ (m/s) nên trong $2$ giây ô tô đi được quãng đường $2 \cdot 10=20$ m\\
			Quãng đường ô tô đi được từ khi bắt đầu tăng tốc đến khi nhập làn là $200-20=180$ m.
			\itemch Ôtô nhập làn sau $12$ giây tăng tốc nên
			$$s(12)=180 \Leftrightarrow \displaystyle\int\limits_0^{12}{v(t)dt}=180 \Leftrightarrow \dfrac{1}{2}.a.12^2+12b=180 \Leftrightarrow 72a+12b=180.$$
			Mặt khác: $v(0)=10$ nên $b=10$ suy ra $a=\dfrac{5}{6}$\\
			Do đó vận tốc ô tô tại thời điểm nhập làn là $v(12)=20$ (m/s) $=72$ (km/h).
			\itemch Quãng đường ô tô đi được sau $24$ giây tăng tốc là $s(24)=\displaystyle\int\limits_0^{24}{v(t)dt}=\displaystyle\int\limits_0^{24}{\left(at+b\right)dt}=480$ m\\
			Vận tốc tại thời điểm 24 giây từ lúc tăng tốc là $v\left(24\right)=\dfrac{5}{6}.24+10=30$ (m/s)\\
			Quãng đường $4$ giây kể từ lúc vận tốc đạt lớn nhất là $30.4=120$ m\\
			Vậy tổng quãng đường đi được sau $30$ giây kể từ khi ô tô cách điểm nhập làn $200$ m là\\
			$20+480+120=620$ m.
			\itemch Vận tốc tốc tại thời điểm vượt chướng ngại vật là $30$ (m/s)\\
			Vận tốc từ khi bắt đầu phanh là $v=\displaystyle\int a dt=\displaystyle\int -3dt=-3t+C$\\
			Vì vận tốc khi phanh là $30$ (m/s) nên $C=30$ do đó vận tốc là $v=-3t+30$\\
			Khi xe dừng thì vận tốc bằng $0$ nên ta có $-3t+30=0\Leftrightarrow t=10$ (giây)\\
			Quãng đường đi được đến khi xe dừng là $S=\displaystyle\int\limits_0^{10}{\left(-3t+30\right)dt}=150$ m\\
			Vậy ôtô cách chướng ngại vật $300-150=150$ m.
		\end{itemchoice}
	}
\end{ex}

\begin{ex}%Câu 15
	Một loại sản phẩm do hai nhà máy số I, số II cùng sản xuất. Tỷ lệ phế phẩm của các nhà máy I, II lần lượt là $0,04$ ; $0,03$ . Trong một lô sản phẩm để lẫn lộn $80$ sản phẩm của nhà máy số I và $120$ sản phẩm nhà máy số II. Một khách hàng lấy ngẫu nhiên 1 sản phẩm từ lô hàng đó.
	\choiceTF
	{\True Số phần tử của không gian mẫu là $200$}
	{\True Xác suất để lấy được sản phẩm tốt là $\dfrac{483}{500}$}
	{Biết sản phẩm lấy được là phế phẩm, xác suất sản phẩm được sản xuất từ nhà máy I là $\dfrac{8}{19}$}
	{Khả năng lấy được sản phẩm không tốt của nhà máy II là thấp hơn nhà máy I}
	\loigiai{
		\begin{itemchoice}
			\itemch Số phần tử của không gian mẫu là $n\left(\Omega\right)=80+120=200$.
			\itemch Gọi $B$ là biến cố lấy được sản phẩn không tốt.\\
			Xác suất để lấy được sản phẩm tốt là $P(\bar{B})=\dfrac{80}{200}.0,96+\dfrac{120}{200}.0,97=\dfrac{483}{500}$.
			\itemch Gọi $A$ là biến cố lấy được sản phẩm của nhà máy I.\\
			Khi đó $P\left(A|B\right)=\dfrac{P\left(A\cap B\right)}{P(B)}=\dfrac{P\left(B|A\right)P(A)}{P(B)}=\dfrac{0,04 \cdot \dfrac{80}{200}}{1-\dfrac{483}{500}}=\dfrac{8}{17}$.
			\itemch Xác suất để lấy được sản phẩm không tốt ở máy I là $P(AB)=P(B|A) \cdot P(A)=0,04 \cdot \dfrac{80}{200}=\dfrac{8}{100}=0,016$\\
			Xác suất để lấy được sản phẩm không tốt ở máy II là $P\left(\overline{A}B\right)=P(B|\overline{A}) \cdot P(\overline{A})=0,03 \cdot \dfrac{120}{200}=\dfrac{3}{100}=0,018$\\
			Vậy khả năng lấy được sản phẩm không tốt ở máy II cao hơn máy I.
		\end{itemchoice}
	}
\end{ex}

\begin{ex}%Câu 16
	Các thiên thạch có đường kính lớn hơn $140m$ và có thể lại gần Trái Đất ở khoảng cách nhỏ hơn $7500000km$ được coi là những vật thể có khả năng va chạm gây nguy hiểm cho Trái Đất. Để theo dõi những thiên thạch này, các nhà nghiên cứu của trung tâm Vũ Trụ Nasa đã thiết lập các trạm quan sát các vật thể bay gần Trái Đất. Giả sử có một hệ thống quan sát có khả năng theo dõi các vật thể ở độ cao không vượt quá $4600km$ so với mực nước biển. Coi Trái Đất là khối cầu có bán kính $6400km$ . Chọn hệ trục tọa độ $Oxyz$ trong không gian có gốc $O$ tại tâm Trái Đất và đơn vị độ dài trên mỗi trục tọa độ là $1000$ km. Một thiên thạch (coi như một hạt) chuyển động với tốc độ $v_1=2\sqrt{2}.10^3$ (km/h) không đổi theo đường thẳng xuất phát từ điểm $M\left(0;5;12\right)$ đến $N\left(12;5;0\right)$\\
	\centerline{\includegraphics[width=.4\textwidth]{images/2.16.png}}
	\choiceTF
	{\True Khoảng cách thiên thạch gần với trái đất nhất có độ dài bằng $3449km$ (Kết quả làm tròn đến hàng đơn vị)}
	{\True Các nhà nghiên cứu của trung tâm vũ trụ Nasa đưa ra giả thiết nếu lúc thiên thạch đang ở vị trí $M$ bất ngờ đổi hướng và lao xuống trái đất với phương thẳng thì quãng đường dài nhất nó có thể va chạm với trái đất là $11315$ km (Kết quả làm tròn đến hàng đơn vị)}
	{Tại thời điểm thiên thạch đang ở vị trí $M$ thì có 2 vệ tinh đang ở vị trí $A\left(-6;\,-5;\,-6\right)$ và $B\left(7;\,-6;\,7\right)$ có vận tốc khác nhau di chuyển trong mặt phẳng trung trực của $MN$ và luôn cách trái đất với khoảng cố định. Khoảng cách xa nhất của 2 vệ tinh có thể đạt là $18412km$ (Kết quả làm tròn đến hàng đơn vị)}
	{\True Nếu vệ tinh $A$ đi với vận tốc $v_2=\dfrac{\pi\sqrt{97}}{3}.10^3$ (km/h) thì sẽ va chạm với thiên thạch}
	\loigiai{
		\begin{itemchoice}
			\itemch Phương trình đường thẳng $MN$ là: $\heva{& x=12t\\&y=5\\&z=12-12t}$\\
			Khoảng cách thiên thạch gần tâm Trái Đất nhất là $d(O,MN)=\dfrac{|[\vec{OM},\vec{MN}]|}{|\vec{MN}|}=\sqrt{97}$\\
			Suy ra khoảng cách thiên thạch gần Trái Đất nhất là $d(O,MN)-R=\sqrt{97}.1000-6400\approx 3449$ (km)\\
			\itemch Quãng đường dài nhất thiên thạch va chạm trái đất là $MK=\sqrt{OM^2-R^2} \approx \sqrt{13^2-6,4^2}.10^3\approx 11315$ (km)\\
			\itemch Phương trình mặt phẳng trung trực $MN$ là $x-z=0$\\
			Hai vệ tinh $A$, $B$ và tâm $O$ đều nằm trên mặt phẳng trung trực của $MN$ nên khoảng cách xa nhất của hai vệ tinh là $AB=OA+OB=\sqrt{6^2+5^2+6^2}+\sqrt{7^2+6^2+7^2}=\sqrt{97}+\sqrt{134}\approx 21425$ (km)\\
			\itemch Giả sử vệ tinh $A$ có thể va chạm với thiên thạch. Gọi vị trí va chạm là $H$. Khi đó $H$ là giao điểm của $MN$ với mặt phẳng trung trực của nó (vì $A$ nằm trên đó), do đó $H$ là trung điểm $MN$ và có toạ độ $H\left(6;5;6\right)$\\
			Ta có $AH=\sqrt{12^2+10^2+12^2}=\sqrt{388}$ (nghìn km)\\
			Khi đó: $\cos\widehat{AOH}=\dfrac{2OA^2-AH^2}{2OA \cdot OH}=\dfrac{2 \cdot 97-388}{2\sqrt{97} \cdot \sqrt{97}}=-1\Rightarrow\widehat{AOH}=180^\circ$\\
			Độ dài cung trên $AH$ là $l_{AH}=\pi\sqrt{97}.10^3$ (km)\\
			Thời gian vệ tinh di chuyển từ $A$ đến $H$ là $t_2=\dfrac{l_{AH}}{v_2}=\dfrac{\left(\pi \cdot \sqrt{97}.10^3\right)}{\left(\pi \cdot \dfrac{\sqrt{97}}{3}.10^3\right)}=3$ (giờ)\\
			Thời gian thiên thạch đi từ $M$ tới $H$ là $t_{MH}=\dfrac{MH}{v_1}=\dfrac{\sqrt{72}.10^3}{2\sqrt{2}.10^3}=3$ (giờ)\\
			Vậy vì $t_1=t_2$ nên thiên thạch và vệ tinh $A$ sẽ va chạm với nhau.
		\end{itemchoice}
	}
\end{ex}
\Closesolutionfile{ans}
\TNSA
\Opensolutionfile{ans}[ans/ans-DE-PNL-02-SA]
\begin{ex}%Câu 17
	Cho hình lăng trụ đứng $ABC.\,A'{B}'{C}'$ có đáy $ABC$ là tam giác đều cạnh bằng$\sqrt{2}$ và độ dài cạnh$B{A}'=\sqrt{6}$ . Hãy tính khoảng cách giữa hai đường thẳng $A'B$ và $B'C$ (Kết quả làm tròn đến hàng phần trăm)?\\
	\centerline{\begin{tikzpicture}[scale=1, font=\footnotesize, line join=round, line cap=round, >=stealth,thick]
			\def\ac{4} % cạnh AC
			\def\ab{2} % cạnh AB
			\def\h{4} % chiều cao
			\def\gocA{50} % góc A của đáy
			\path
			(0,0) coordinate (A)
			(\ac,0) coordinate (C)
			(-\gocA:\ab) coordinate (B)
			($(A)+(90:\h)$) coordinate (A')
			($(B)-(A)+(A')$) coordinate (B')
			($(C)-(A)+(A')$) coordinate (C');
			\draw (B)--(A')--(A)--(B)--(C)--(C')--(A')--(B')--(C') (B)--(B')--(C);
			\draw[dashed] (A)--(C);
			\foreach \x/\g in {A/180,B/-90,C/0,A'/180,C'/0,B'/-140}\fill[red] (\x) circle (1pt)+(\g:3mm) node[black]{$ \x $};
		\end{tikzpicture}}
	\shortans{0,67}
	\loigiai{
	Gọi $M,\,N$ lần lượt là trung điểm của $AC$ và $A'B$ suy ra $MN\parallel{B}'C$ nên $B'C\parallel\left(A'BM\right)$ .\\
	Khi đó: $d\left(\,B'C,\,A'B\right)=d\left(\,B'C,\,\left(A'BM\right)\right)=d\left(C,\,\left(A'BM\right)\right)=d\left(A,\,\left(A'BM\right)\right)$ .\\
	Kẻ $AH\bot{A}'M$ tại $H$ thì ta có $\left\{\begin{aligned}
			 & BM\perp AC  \\
			 & BM\perp AA' \\
		\end{aligned}\right.\Rightarrow BM\perp\left(AC{C}'{A}'\right)$ .\\
	Khi đó $\left\{\begin{aligned}
			 & AH\bot{A}'M \\
			 & AH\perp BM  \\
		\end{aligned}\right.\Rightarrow AH\perp\left(A'BM\right)$ nên $d\left(A,\,\left(A'BM\right)\right)=AH$ .\\
	Ta có $AM=\dfrac{1}{2}AC=\dfrac{\sqrt{2}}{2}$ ; $A{A}'=\sqrt{A'{B^2}-A{B^2}}=2$ .\\
	Trong $\Delta A{A}'K$ ta có: $\dfrac{1}{A{H^2}}=\dfrac{1}{A{A'^2}}+\dfrac{1}{A{M^2}}\Rightarrow d\left(\,B'C,\,A'B\right)=AH=\dfrac{A{A}'.\,AM}{\sqrt{A{A'^2}+A{M^2}}}=\dfrac{2}{3}\approx 0,67$
	}
\end{ex}

\begin{ex}%Câu 18
	Có năm địa điểm $A,B,C,D,E$. Một số địa điểm có đường đi tới nhau môt tả bằng các cạnh với độ dài quãng đường tính theo kilomet cho bởi số gắn với cạnh đó như hình vẽ. Một người đưa thư xuất phát từ bưu điện ở vị trí $C$ , cần đi qua tất cả các đường (mỗi đường đi qua ít nhất một lần) và sau đó phải trở về vị trí ban đầu $C$. Tổng số kilomet mà người đưa thư phải đi nhỏ nhất là bao nhiêu?\\
	\centerline{
		\begin{tikzpicture}[declare function={r=3;},thick]
			\path
			(0,0) coordinate (B)
			(0:r) coordinate (C)
			(-90:r) coordinate (D)
			($(C)+(D)-(B)$) coordinate (E)
			($(B)!.5!(C) +(90:2)$) coordinate (A)
			;
			\draw (B) rectangle (E) (E)--(B)node[midway,below]{$5$}--(A) node[midway,left]{$1$}--(C)node[midway,above]{$2$}
			;
			\path (B)--(D) node[left,midway]{$3$}
			(B)--(C) node[left,midway]{$3$}
			(E)--(D) node[below,midway]{$6$}
			(E)--(C) node[right,midway]{$10$};
			\foreach \x/\g in {A/90,B/180,C/0,D/-90,E/-90}\draw[fill=white] (\x) circle (1pt)+(\g:3mm) node{$\x$};
		\end{tikzpicture}
	}
	\shortans{38}
	\loigiai{
		Đi từ $C$ đến $E$ theo đường đi Euler $CABCEBDE$ dài: $2+1+3+3+5+6+10=30$ (km)\\
		Đi từ $E$ đến $C$ với quảng đường ngắn nhất dài: $5+3=8$ (km)\\
		Vậy tổng số kilomet mà người đưa thư phải đi nhỏ nhất là $30+8=38$ (km).}
\end{ex}

\begin{ex}%Câu 19
	Hệ thống định vị toàn cầu GPS (Global Positioning System) là một hệ thống cho phép xác định vị trí của một vật thể trong không gian. Trong cùng một thời điểm, vị trí của một điểm $M$ trong không gian sẽ được xác định bởi bốn vệ tinh cho trước nhờ các bộ thu phát tín hiệu đặt trên các vệ tinh. Giả sử trong không gian với hệ tọa độ $Oxyz$ , tỉ lệ độ dài trên các trục là $10$ km tính cho một đơn vị tỉ lệ trên mỗi trục có 4 vệ tinh lần lượt đặt tại các điểm $A(-1;2;-1)$ ,$B(1;4;0)$ ,$C(3;0;9)$ và $D(7;10;-1)$ . Ở một thời điểm cả bốn vệ tinh bắn tín hiệu về điểm $M$ và đo được độ dài $MA=6,\,MB=3,\,MC=10,\,MD=6$ . Ngay sau đó $10$ giây, cả bốn vệ tinh lại bắn tin hiệu về vật $M$ và đo được $MA=3,\,MB=4,\,MC=\sqrt{85},\,MD=\sqrt{137}$. Nếu coi như vật $M$ chuyển động thẳng đều thì tốc độ của vật bằng bao nhiêu (đơn vị: m/s và kết quả làm tròn đến hang đơn vị)? (Bỏ qua khoảng thời gian phát và thu tín hiệu)
	\shortans{6403}
	\loigiai{
		Gọi $M\left(a;\,b;\,c\right)$ .\\
		Tại thời điểm đầu bốn vị tinh bắn tín hiệu về điểm $M$ . Khi đó ta có hệ phương trình sau:\\
		$\Leftrightarrow\left\{\begin{matrix}
				{\left(a+1\right)^2}+\left(b-2\right)^2+\left(c+1\right)^2=36\,\,\,\,\,\,\,\,\,\,\,\,\,\,(1)               \\
				{\left(a-1\right)^2}+\left(b-4\right)^2+c^2=9\,\,\,\,\,\,\,\,\,\,\,\,\,\,\,\,\,\,\,\,\,\,\,\,\,\,\,\,\,(2) \\
				{\left(a-3\right)^2}+b^2+\left(c-9\right)^2=100\,\,\,\,\,\,\,\,\,\,\,\,\,\,\,\,\,\,\,\,\,\,\,(3)           \\
				{\left(a-7\right)^2}+\left(b-10\right)^2+\left(c+1\right)^2=36\,\,\,\,\,\,\,\,\,\,\,(4)                    \\
			\end{matrix}\right.\Leftrightarrow\left\{\begin{aligned}
				 & (2)-(1):-4a-4b-2c=-38  \\
				 & (3)-(1):-8a+4b-20c=-20 \\
				 & (4)-(1):-14a-14b=-144  \\
			\end{aligned}\right.$\\
		Giải hệ phương trình này, ta tìm được $a~=~3,b=6,c=1\Rightarrow M\left(3;\,6;\,1\right)$\\
		Sau đó 10 giây, về tinh lại bán tín hiệu về điểm $M$ . Gọi $M'\left(x;\,y;\,z\right)$ khi đó ta có:\\
		$\Leftrightarrow\left\{\begin{matrix}
				{\left(x+1\right)^2}+\left(y-2\right)^2+\left(z+1\right)^2=9\,\,\,\,\,\,\,\,\,\,\,\,\,\,(1)     \\
				{\left(x-1\right)^2}+\left(y-4\right)^2+z^2=16\,\,\,\,\,\,\,\,\,\,\,\,\,\,\,\,\,\,\,\,\,\,\,(2) \\
				{\left(x-3\right)^2}+y^2+\left(z-9\right)^2=85\,\,\,\,\,\,\,\,\,\,\,\,\,\,\,\,\,\,\,\,\,\,(3)   \\
				{\left(x-7\right)^2}+\left(y-10\right)^2+\left(z+1\right)^2=137\,\,\,\,(4)                      \\
			\end{matrix}\right.\Leftrightarrow\left\{\begin{aligned}
				 & (2)-(1):-4x-4y-2z=-4  \\
				 & (3)-(1):-8x+4y-20z=-8 \\
				 & (4)-(1):-14x-14y=-16  \\
			\end{aligned}\right.$\\
		Giải hệ phương trình này, ta tìm được $x~=~1,y=0,z=0\Rightarrow{M}'\left(1;\,0;\,0\right)$\\
		Khi đó khoảng cách $M{M}'=\sqrt{\left(1-3\right)^2+\left(0-6\right)^2+\left(0-1\right)^2}=\sqrt{41}$\\
		Do vật chuyển động thẳng đều nên vận tốc của vật là: $v_M=\dfrac{M{M}'}{t}=\dfrac{\sqrt{41}{10^4}}{10}=6403$ (m/s)
	}
\end{ex}
\begin{ex}%Câu 20
	Một ly trà sữa dạng hình nón cụt, có đường kính đáy ly 6 cm, đường kính miệng ly 9 cm, chiều cao 13,4 cm, ở miệng ly có sử dụng một nắp đậy có hình dạng nửa mặt cầu và ở đỉnh của nửa mặt cầu này có một hình tròn có đường kính 2 cm để cắm ống hút, mặt phẳng chứa hình tròn này song song với mặt phẳng chứa miệng ly (tham khảo hình vẽ sau).
	\begin{center}
		\includegraphics[width=.3\textwidth]{images/2.20a.png} \includegraphics[width=.22\textwidth]{images/2.20b.png}
		\includegraphics[width=.3\textwidth]{images/2.20c.png}
	\end{center}
	Chọn hệ trục $Oxy$ (đơn vị trên trục là centimet) với trục $Ox$ đi qua tâm của 2 đáy hình nón cụt và gốc tọa độ $O$ trùng với tâm của đáy lớn như hình vẽ trên. Tính thể tích bên trong của ly bao gồm cả thể tích của nắp (Kết quả làm tròn kết quả đến hàng đơn vị).
	\shortans{790}
	\loigiai{
	Thể tích bên trong của ly không bao gồm nắp là: $V_1=\pi{\displaystyle\int\limits_{-13,4}^0\left(\dfrac{1,5x+60,3}{13,4}\right)^2}dx\approx 600$ cm3\\
	Thể tích bên trong của ly bao gồm cả thể tích của nắp là: $V=V_1+\left(V_2-V_3\right)$ trong đó $V_2$ là nửa thể tích của khối cầu và $V_3$ là thể tích của chỏm cầu.\\
	Nửa thể tích khối cầu là: $V_2=\dfrac{1}{2}.\dfrac{4}{3}.\pi .R^3=\dfrac{243\pi}{4}$\\
	Thể tích chỏm cầu: $V_3=\pi\displaystyle\int\limits_{R-h}^R{\left(\sqrt{R^2-x^2}\right)^2dx}$ $\Leftrightarrow{V_3}=\pi\displaystyle\int\limits_{R-h}^R{\left(R^2-x^2\right)dx\Leftrightarrow{V_3}=\,}\pi\left.\left(R^2x-\dfrac{x^3}{3}\right)\right|_{R-h}^R$\\
	$\Leftrightarrow{V_3}=\pi\left[\left(R^3-\dfrac{R^3}{3}\right)-\left(R^2\left(R-h\right)-\dfrac{\left(R-h\right)^3}{3}\right)\right]$ $\Leftrightarrow{V_3}=\pi{h^2}\left(R-\dfrac{h}{3}\right)$ .\\
	Thay số ta có: $h=4,5-\dfrac{\sqrt{77}}{2}$ ; $R=4,5$ $\Rightarrow{V_3}=\dfrac{\left(729-83\sqrt{77}\right)\pi}{12}$\\
	Thể tích bên trong của ly bao gồm cả thể tích của nắp là:\\
	$V=600+\left[\dfrac{243\pi}{4}-\dfrac{\left(729-83\sqrt{77}\right)\pi}{12}\right]\approx 790$ ml.}
\end{ex}

\begin{ex}%Câu 21
	Một doanh nghiệp dự định sản xuất không quá $400$ sản phẩm. Nếu doanh nghiệp sản xuất $x$ sản phẩm $\left(1\le x\le 400\right)$ thì doanh thu nhận được khi bán hết số sản phẩm đó được biểu diễn bởi công thức là $F(x)=x^3-1999x^2+1001000x+250000$ (đồng). Trong đó chi phí vận hành máy móc cho mỗi sản phẩm là $G(x)=\dfrac{100000x}{\dfrac{3}{2}x+1}$ (đồng). Tổng chi phí mua nguyên vật liệu được biểu diễn bởi hàm $H(x)=2x^3+100000x-50000$ (đồng) nhưng do doanh nghiệp đó mua nguyên vật liệu với số lượng lớn nên được giảm $1\%$ cho 200 sản phẩm đầu tiên doanh nghiệp sản xuất và giảm $2\%$ cho sản phẩm tiếp theo. Doanh nghiệp cần sản xuất bao nhiêu sản phẩm để lợi nhuận thu được là lớn nhất?
	\shortans{184}
	\loigiai{
	Lợi nhuận $P(x)$ được tính bằng doanh thu trừ đi tổng chi phí: $P(x)=F(x)-xG(x)-H(x)$ .\\
	Khi $x\le 200$ thì chi phí mua nguyên liệu là:
	$$H_1(x)=0,99H(x)=0,99\left(2x^3+100000x-50000\right) \text{ (đồng).}$$
	Khi $x>200$ thì chi phí mua nguyên liệu là:
	$$H_2(x)=0,99H(200)+0,98[H(x)-H(200)]=0,01H(200)+0,98H(x) \text{ (đồng).}$$
	Xét đồng thời 2 trường hợp:\\
	Trường hợp 1: Với $1\le x\le 200$ thì ta có lợi nhuận thu được là:
	\begin{align*}
		P_1(x) & =F(x)-xG(x)-H_1(x)                                                 \\
		       & =-0,98x^3-1999x^2+902000x+299500-\dfrac{100000x^2}{\frac{3}{2}x+1} \\
	\end{align*}
	Ta có: $P_1'(x)=-2,94{x^2}-3998x-\dfrac{600000x^2+800000x}{\left(3x+2\right)^2}+902000$\\
	Phương trình $P_1'(x)=0$ có nghiệm $x=184,03\in\left(1;200\right)$ .\\
	Ta thấy $\max\limits_{\left[1;200\right]} P_1(x)=80037062,09$ tại $x=184,03$ .\\
	Trường hợp 2: Với $201\le x\le 400$ ta có lợi nhuận thu được là:
	\begin{align*}
		P_2(x) & =F(x)-xG(x)-H_2(x)                                                 \\
		       & =-0,96x^3-1999x^2+903000x+609500-\dfrac{100000x^2}{\frac{3}{2}x+1}
	\end{align*}
	Ta có $P_2'(x)=-2,88{x^2}-3998x+903000-\dfrac{600000x^2+800000x}{\left(3x+2\right)^2}$\\
	Phương trình $P_2'(x)=0$ không có nghiệm thuộc $\left(201;400\right)$ .\\
	Suy ra $\max\limits_{\left[201;400\right]} P_2(x)=P(201)\approx 7,959.10^8$\\
	Vậy doanh nghiệp cần sản xuất 184 sản phẩm thì lợi nhuận thu được là lớn nhất.
	}
\end{ex}

\begin{ex}%Câu 22
	Có 10 học sinh làm bài kiểm tra xác suất thống kê, trong đó có 2 học sinh giỏi (trả lời được 100\% các câu hỏi), 3 học sinh khá (trả lời được 80\% các câu hỏi), 5 học sinh trung bình (trả lời được 50\% các câu hỏi). Bài kiểm tra có 4 câu hỏi được lấy ngẫu nhiên từ 20 câu hỏi. Giáo viên chọn ngẫu nhiên một bài làm của học sinh để chấm điểm. Xác suất bài làm đó trả lời được cả 4 câu hỏi là bao nhiêu (kết quả làm tròn đến hàng phần trăm)?

	\shortans{0,33}
	\loigiai{
		Gọi $A$ là biến cố chọn được bài làm trả lời được cả 4 câu hỏi.\\
		Gọi $B_1,B_2,B_3$ lần lượt là biến cố chọn được bài làm của học sinh giỏi, khá, trung bình.\\
		Khi đó: $P\left(B_1\right)=\dfrac{2}{10},P\left(B_2\right)=\dfrac{3}{10},P\left(B_3\right)=\dfrac{5}{10}$ .\\
		Học sinh giỏi trả lời được 100\% các câu hỏi, nghĩa là trả lời được 20 câu suy ra $P\left(A|B_1\right)=\dfrac{C_{20}^4}{C_{20}^4}.$\\
		Học sinh khá trả lời được 80\% các câu hỏi, nghĩa là trả lời được 16 câu suy ra $P\left(A|B_2\right)=\dfrac{C_{16}^4}{C_{20}^4}.$\\
		Học sinh trung bình trả lời được 50\% các câu hỏi, nghĩa là trả lời được 10 câu suy ra\\
		$P\left(A|B_3\right)=\dfrac{C_{10}^4}{C_{20}^4}$ .\\
		Theo công thức xác suất toàn phần ta có:\\
		$P(A)=P\left(B_1\right).P\left(A|B_1\right)+P\left(B_2\right).P\left(A|B_2\right)+P\left(B_3\right).P\left(A|B_3\right)$\\
		$=\dfrac{2}{10}.\dfrac{C_{20}^4}{C_{20}^4}+\dfrac{3}{10}.\dfrac{C_{16}^4}{C_{20}^4}+\dfrac{5}{10}.\dfrac{C_{10}^4}{C_{20}^4}=\dfrac{108}{323}\approx 0,33$}
\end{ex}
\Closesolutionfile{ans}
\Closesolutionfile{ansbook}
\inputansbox{6,2,3}{ans/ans-DE-PNL-02-T,ans/ans-DE-PNL-02-TF,ans/ans-DE-PNL-02-SA}
% \begin{name}
	{\tenchude}
	{\tendethi}
	{\tentruong}
	{\thoigian}
\end{name}
\Opensolutionfile{ansbook}[ans/ansbook-3]
\TN
\Opensolutionfile{ans}[ans/ans-3-T]
\begin{ex}%Câu 1
	Hàm số $y=f(x)$ có đồ thị như sau. Hàm số $y=f(x)$ đồng biến trên khoảng nào dưới đây?\\
	\centerline{
		\begin{tikzpicture}[line join=round, line cap=round,>=stealth,thick]
			\tikzset{every node/.style={scale=0.9}}
			\draw[->] (-3.1,0)--(3.1,0) node[below left] {$x$};
			\draw[->] (0,-3.1)--(0,3.1) node[below left] {$y$};
			\draw (0,0) node [below left] {$O$};
			\foreach \x/\nx in {-2/-2,-1/-1,1/1}
			\draw[thin] (\x,1pt)--(\x,-1pt) node [below] {$\nx$};
			\foreach \y/\ny in {-2/-2,2/2}
			\draw[thin] (1pt,\y)--(-1pt,\y) node [below left] {$\ny$};
			\draw[dashed,thin](-2,0)--(-2,-2)--(0,-2);
			\draw[dashed,thin](1,0)--(1,-2)--(0,-2);
			\draw[dashed,thin](-1,0)--(-1,2)--(0,2);
			\begin{scope}
				\clip (-3,-3) rectangle (3,3);
				\draw[samples=200,domain=-2.1:2.1,smooth,variable=\x] plot (\x,{1*((\x)^3)+0*((\x)^2)+-3*(\x)+0});
			\end{scope}
		\end{tikzpicture}
	}
	\choice
	{\True $\left(-2;-1\right)$}
	{$\left(-1;1\right)$}
	{$\left(-2;1\right)$}
	{$\left(-1;2\right)$}
	\loigiai{
		Từ đồ thị, hàm số đã cho đồng biến trên khoảng $\left(-2;-1\right)$ .}
\end{ex}
\begin{ex}%Câu 2
	Trong không gian với hện tọa độ $Oxyz$ , cho mặt cầu $(S):{x^2}+y^2+z^2-2x+4y+6z-2=0$ . Bán kính của mặt cầu $(S)$ bằng:
	\choice
	{8}
	{12}
	{\True 4}
	{16}
	\loigiai{
		Mặt cầu $(S)$ có tâm $I\left(1\,;\,-2\,;\,-3\right)$ và có bán kính $R=\sqrt{1+4+9-\left(-2\right)}=\sqrt{16}=4$}
\end{ex}
\begin{ex}%Câu 3
	Tiệm cận ngang của đồ thị hàm số $y=\dfrac{1-2x}{x-2}$ là đường thẳng:
	\choice
	{$y=1$}
	{$x=2$}
	{\True $y=-2$}
	{$x=-2$}
	\loigiai{
		Đường tiệm cận ngang của đồ thị hàm số là: $y=\dfrac{a}{c}=-\dfrac{2}{1}=-2$ .}
\end{ex}
\begin{ex}%Câu 4
	Tìm tập nghiệm $S$ của bất phương trình $\log_{\tfrac{1}{2}}\left(x-3\right)\ge\log_{\tfrac{1}{2}}4$ .
	\choice
	{\True $S=\left(3;7\right]$}
	{$S=\left[3;7\right]$}
	{$S=\left(-\infty ;7\right]$}
			{$S=\left[7;+\infty\right)$}
	\loigiai{
		$\log_{\tfrac{1}{2}}\left(x-3\right)\ge\log_{\tfrac{1}{2}}4\Leftrightarrow 0<x-3\le 4\Leftrightarrow 3<x\le 7$ (vì $0<\dfrac12<1$).\\
		Vậy tập nghiệm của bất phương trình là: $S=\left(3;7\right]$}
\end{ex}
\begin{ex}%Câu 5
	Trong không gian $Oxyz$, một vectơ chỉ phương của đường thẳng $d\colon \heva{&x=-t+2\\&    y=2t\\&    z=3t}$ là
	\choice
	{$\left(2;2;3\right)$}
	{$\left(1;2;3\right)$}
	{\True $\left(1;-2;-3\right)$}
	{$\left(2;-2;-3\right)$}
	\loigiai{
		Một vectơ chỉ phương của đường thẳng $d$ là $\overrightarrow{u_d}=\left(-1\,;\,2\,;\,3\right)$}
\end{ex}
\begin{ex}%Câu 6
	Cho hình chóp $S.ABCD$ có đáy $ABCD$ là hình chữ nhật, cạnh $AB=a,BC=2a$ . Hai mặt bên $\left(SAB\right)$ và ($SAD)$ cùng vuông góc với mặt phẳng $\left(ABCD\right)$ và cạnh $SA=a$ . Tính theo $a$ thể tích $V$ của khối chóp $S.ABCD$
	\choice
	{$V=a^3$}
	{\True $V=\dfrac{2a^3}{3}$}
	{$V=2a^3$}
	{$V=\dfrac{a^3}{3}$}
	\loigiai{
		Ta có: $SA\perp\left(ABCD\right)$ và $SA=a$\\
		Thể tích khối chóp $S.ABCD$ là: $V_{S.ABCD}=\dfrac{1}{3} \cdot a \cdot a \cdot 2a=\dfrac{2a^3}{3}$}
\end{ex}
\begin{ex}%Câu 7
	Hàm số $y=3^{x^2+1}$ có giá trị nhỏ nhất bằng
	\choice
	{$1$}
	{$5$}
	{\True $3$}
	{$0$}
	\loigiai{
		Ta thấy: $x^2+1\ge 1$ và dấu \lq\lq =\rq\rq~ xảy ra khi $x=0$ nên $3^{x^2+1}\ge{3^1}=3$}
\end{ex}
\begin{ex}%Câu 8
	Cho hàm số $f(x)=x^2-\dfrac{4}{x}$. Giá trị của $\displaystyle\int\limits_1^2 f'(x)\mathrm{\,d}x$ bằng
	\choice
	{$\dfrac{7}{3}- \ln 2$}
	{$3$}
	{$\dfrac{7}{3}$}
	{\True $5$}
	\loigiai{
		Ta có: $\displaystyle\int\limits_1^2f'(x)\mathrm{\,d}x= f(x)\vert_1^2=f(2)-f(1)=\left(2^2-\dfrac{4}{2}\right)-\left(1^2-\dfrac{4}{1}\right)=5$
	}
\end{ex}
\begin{ex}%Câu 9
	Cho một cấp số cộng có số hạng đầu là $u_1=2$ và công sai $d=3$ . Số hạng thứ 10 bằng:
	\choice
	{\True 29}
	{32}
	{26}
	{30}
	\loigiai{
		Ta có: $u_{10}=u_1+9d=2+9.3=2+27=29$}
\end{ex}
\begin{ex}%[Nguyễn Tuấn, dự án sáng tác đề 12]%[2D4N3-1]
	\immini{Diện tích phần hình phẳng gạch chéo trong hình vẽ bên được tính theo công thức nào dưới đây?
		\choice
		{$\displaystyle\int\limits_{-1}^2\left(2x^2-2x-4\right)\mathrm{\,d}x$}
		{$\displaystyle\int\limits_{-1}^2(-2x+2)\mathrm{\,d}x$}
		{$\displaystyle\int\limits_{-1}^2(2x-2)\mathrm{\,d}x$}
		{\True $\displaystyle\int\limits_{-1}^2\left(-2x^2+2x+4\right)\mathrm{\,d}x$}}
	{\begin{tikzpicture}[scale=.7, font=\footnotesize, line join=round, line cap=round, >=stealth]
			\draw[->] (-1.5,0) -- (3,0)node[below]{\footnotesize $x$};
			\draw (-1,0) circle (.5pt)node[below]{\footnotesize $-1$};
			\draw (2,0) circle (.5pt)node[above]{\footnotesize $2$};
			\draw[->,color=black] (0,-2.5) -- (0,3.5)node[below left]{\footnotesize $y$};
			\fill[pattern=north west lines] plot[smooth,samples=100,domain=-1:2] (\x,{(\x)^2-2*(\x)-1})--plot[smooth,samples=100,domain=2:-1] (\x,{-(\x)^2+3});
			\draw[thick,smooth,samples=100,domain=-1.5:2.2] plot(\x,{-(\x)^2+3});
			\draw[thick,smooth,samples=100,domain=-1.2:3] plot(\x,{(\x)^2-2*(\x)-1});
			\draw[dashed] (2,0) -- (2,-1) (-1,0) -- (-1,2);
			\filldraw[fill=white] (0,0) circle (1pt)node[shift={(-45:6pt)}]{\footnotesize $O$};
			\draw (2,3) node{\footnotesize $y=-x^2+3$};
			\draw (-1,-2.2) node{\footnotesize $y=x^2-2x-1$};
		\end{tikzpicture}}
	\loigiai{
		$S=\displaystyle\int\limits_{-1}^2\left[\left(-x^2+3\right)-\left(x^2-2x-1\right)\right]\mathrm{\,d}x=\displaystyle\int\limits_{-1}^2\left(-2x^2+2x+4\right)\mathrm{\,d}x$.
	}
\end{ex}
\begin{ex}%Câu 11
	Trong không gian với hệ tọa độ $Oxyz$ , cho điểm $M\left(3;5;-7\right)$ . Tìm tọa độ của điểm $M'$ đối xứng với điểm $M$ qua trục $Oy$ .
	\choice
	{$M'\left(3;-5;-7\right)$}
	{$M'\left(3;5;7\right)$}
	{$M'\left(-3;5;-7\right)$}
	{\True $M'\left(-3;5;7\right)$}
	\loigiai{
		Gọi $H$ là hình chiếu của $M$ lên $Oy$ nên suy ra $H\left(0\,;\,5\,;\,0\right)$\\
		Điểm $M'$ là điểm đối xứng với $M$ qua $H$ nên $M'=2H-M=\left(-3\,;\,5\,;\,7\right)$}
\end{ex}
\begin{ex}%Câu 12
	Một công ty thống kê lương của nhân viên theo tuần (đơn vị: USD) theo bảng sau:
	\begin{center}
		\begin{tabular}{|c|c|c|c|c|c|}
			\hline
			Lương theo tuần (USD) & $[10; 20)$ & $[20; 30)$ & $[30; 40)$ & $[40; 50)$ & $[50; 60]$ \\
			\hline
			Số công nhân          & 4          & 6          & 10         & 20         & 10         \\
			\hline
		\end{tabular}
	\end{center}
	Độ lệch chuẩn của mẫu số liệu này bằng bao nhiêu? (làm tròn tới hàng phần chục)
	\choice
	{\True 11,7}
	{12}
	{11,4}
	{12,5}
	\loigiai{
		Ta có: $n=50$\\
		Khi đó: $\overline{x}=\dfrac{1}{50}\left(4.15+6.25+10.35+20.45+10.55\right)=40,2$\\
		Phương sai của mẫu số liệu: $s^2=\dfrac{1}{50}\left(4.15^2+6.25^2+10.35^2+20.45^2+10.55^2\right)-\overline{x}^2=136,96$\\
		Độ lệch chuẩn của mẫu số liệu là: $s=\sqrt{s^2}=\sqrt{136,96}\approx 11,7$}
\end{ex}
\Closesolutionfile{ans}
\TNTF
\Opensolutionfile{ans}[ans/ans-3-TF]
\begin{ex}%Câu 13
	Cho hàm số $y=f(x)=ax+b-\dfrac{1}{x+c}$ có đồ thị như hình vẽ.\\
	\centerline{
		\begin{tikzpicture}[thick,scale=.7]
			\tikzset{every node/.style={scale=0.9}}
			\draw[->] (-5.1,0)--(5.1,0) node[below left] {$x$};
			\draw[->] (0,-5.1)--(0,5.1) node[below left] {$y$};
			\draw (0,0) node [above right] {$O$};
			\draw[dashed] (-0.99,-5)--(-0.99,5)
			(-1,-2)--(0,-2);
			\path
			(-1,0) node[above right]{$-1$}
			(0,-2)node[below right]{$-2$}
			(0,-1)node[below]{$-1$}
			;
			\begin{scope}
				\clip (-5,-5) rectangle (5,5);
				\draw[samples=200,domain=-4:-1.01,smooth,variable=\x] plot (\x,{(1*((\x)^2)+0*(\x)+-2)/(1*(\x)+1)});
				\draw[samples=200,domain=-0.99:4,smooth,variable=\x] plot (\x,{(1*((\x)^2)+0*(\x)+-2)/(1*(\x)+1)});
				\draw[dashed,thin] (-4.1,-5.1)--(4.1,3.1);
			\end{scope}
		\end{tikzpicture}
	}
	\choiceTF
	{Đồ thị hàm số nhận đường $x=-1$ làm tiệm cận đứng và đường $y=-x-1$ làm tiệm cận xiên}
	{\True Tâm đối xứng của đồ thị hàm số là điểm có tọa độ $\left(-1;-2\right)$}
	{\True $a+b+c=1$}
	{Gọi $I$ là tâm đối xứng của đồ thị hàm số và $M$ là 1 điểm bất kì thuộc đồ thị. Giá trị nhỏ nhất của độ dài đoạn thẳng $IM$ bằng $2\left(\sqrt{2}-1\right)$}
	\loigiai{
		a) Sai: Tiệm cận xiên của đồ thị hàm số có dạng $y=ax+b$ đi qua $\left(0\,;\,-1\right)$ và $\left(-1\,;\,-2\right)$ nên suy ra $\left\{\begin{aligned}
				 & a=1  \\
				 & b=-1 \\
			\end{aligned}\right.$ nên đường tiệm cận xiên là $y=x-1$\\
		Tiệm cận đứng là đường thẳng $x=-1\Leftrightarrow x=-c\Rightarrow-c=-1\Rightarrow c=1$\\
		b) Đúng: Từ đồ thị ta thấy tâm đối xứng của đồ thị hàm số là điểm có tọa độ $\left(-1;-2\right)$\\
		c) Đúng: $a+b+c=1+\left(-1\right)+1=1$.\\
		d) Sai: Dễ thấy $I\left(-1\,;\,-2\right)$ và hàm số ban đầu là $f(x)=x-1-\dfrac{1}{x+1}$ nên $M\left(x;\,x-1-\dfrac{1}{x+1}\right)$ với mọi $x\ne-1$ .\\
		Xét biểu thức $I{M^2}=\left(x+1\right)^2+\left[x-1-\dfrac{1}{x+1}-\left(-2\right)\right]^2=\left(x+1\right)^2+\left(x+1-\dfrac{1}{x+1}\right)^2$\\
		$=2\left(x+1\right)^2+\dfrac{1}{\left(x+1\right)^2}-2\ge 2\sqrt{2\left(x+1\right)^2.\dfrac{1}{\left(x+1\right)^2}}-2=2\sqrt{2}-2$ (Bất đẳng thức Cauchy)\\
		Vậy $IM=\sqrt{2\left(\sqrt{2}-1\right)}$ .}
\end{ex}
\begin{ex}%Câu 14
	Một nghệ nhân muốn làm một bình gốm có dạng mô hình như hình 2 bằng cách quay hình phẳng $(H)$ ở hình 1 quanh trục hoành. Biết đường cong trong hình 1 là phần đồ thị $y=0,125x^3$ trên đoạn $\left[0;2\right]$ và mỗi đơn vị trên đồ thị ở hình 1 có độ dài bằng 10 cm.\\
	\centerline{
		\begin{tikzpicture}[line join=round, line cap=round,>=stealth,thick,declare function={hs(\x)=0.125*((\x)^3);},x=1.75cm]
			\draw[->] (-.5,0)--(2.5,0) node[below] {$x$};
			\draw[->] (0,-.7)--(0,2) node[left] {$y$};
			\draw (0,0) node [below left] {$O$};
			\foreach \x/\nx in {2/2}
			\draw[thin] (\x,1pt)--(\x,-1pt) node [below] {$\nx$};
			\draw[dashed] (2,0)--(2,1);
			\fill (2,1) circle(1.5pt);
			\fill[gray,opacity=.7](0,0)--(2,0)--(2,1) plot[samples=200,domain=2:0,smooth](\x,{hs(\x)});
			\draw[samples=200,domain=0:2,smooth] plot (\x,{hs(\x)});
		\end{tikzpicture}
		\begin{tikzpicture}[line join=round, line cap=round,>=stealth,thick,declare function={hs(\x)=0.125*((\x)^3);},x=1.75cm,y=1.5cm]
			\draw[dashed] (2,-1)--(2,1);
			\fill (2,1) circle(1.5pt) (2,-1) circle(1.5pt);
			\fill[gray,opacity=.7] plot[samples=200,domain=2:0,smooth](\x,{hs(\x)}) plot[samples=200,domain=0:2,smooth](\x,{hs(-\x)})--(2,1);
			\draw[samples=200,domain=0:2,smooth] plot (\x,{hs(\x)});
			\draw[samples=200,domain=0:2,smooth] plot (\x,{hs(-\x)});
			\draw[dashed] (2,0) ellipse (0.14 and 1);
		\end{tikzpicture}
	}
	\choiceTF
	{\True Chiều cao của bình gốm bằng 20 cm}
	{Đường kính đáy của bình gốm bằng 10 cm}
	{Khi cắt bình gốm bởi 1 mặt phẳng qua trục thì thiết diện thu được có diện tích bằng $10\text\!\!~\!\!\text{c}{\text{m}^2}$}
	{\True Thể tích bình gốm bằng 0,90 (đơn vị: lít, kết quả làm tròn tới hàng phần trăm)}
	\loigiai{

		a) Đúng: Chiều cao của bình gốm bằng $2.10=20$ cm.\\
		b) Sai: Ta có $y_A=f\left(x_A\right)=f(2)=0,125.2^3=\dfrac{1}{8}.8=1$ nên đường kính đáy của bình gốm là $20$ cm.\\
		c) Sai: Thiết diện thu được là: $S=2\displaystyle\int\limits_0^2\left(0,125x^3\right)\mathrm{\,d}x=1$ nhưng thực tế mỗi đơn vị trên đồ thị có độ dài bằng 10 cm nên diện tích thiết diện là: $S=1.10^2=100\text\!\!~\!\!\text{c}{\text{m}^2}$ \\
		d) Đúng: Đổi 1 lít $\text{=1d}{\text{m}^3}=1000\,\text{c}{\text{m}^3}$\\
		Thể tích bình gốm là: $S=\pi\displaystyle\int\limits_0^2\left(0,125x^3\right)^2\mathrm{\,d}x=\dfrac{2\pi}{7}$ (đvtt) nhưng thực thế mỗi đơn vị trên đồ thị có độ dài bằng 10 cm nên một đơn vị thể thích là $10^3\text{c}{\text{m}^3}$ .\\
		Thể tích bình gốm thực tế là: $V=\dfrac{2\pi}{7}{10^3}\left(\text{c}{\text{m}^3}\right)=\dfrac{2\pi}{7}\left(\mathrm{\,d}{\text{m}^3}\right)\approx 0,90$ lít.}
\end{ex}
\begin{ex}%Câu 15
	Một hộp có 80 viên bi, trong đó có 50 viên bi màu đỏ và 30 viên bi màu vàng. Các viên bi có kích thước và khối lượng như nhau. Sau khi kiểm tra, người ta thấy có 60\% số viên bi màu đỏ đánh số và 50\% số viên bi màu vàng có đánh số, những viên bi còn lại không đánh số. Lấy ra ngẫu nhiên 1 viên bi trong hộp.
	\choiceTF
	{\True Xác suất chọn được viên bi màu đỏ bằng $62,5\%$}
	{Xác suất chọn được viên bi màu vàng có đánh số bằng 18,57\%}
	{\True Xác suất chọn được viên bi không đánh số bằng 43,75\%}
	{Giả sử viên bi được lấy ra là viên bi chưa được đánh số, xác suất để viên bi đó là bi đỏ thấp hơn xác suất viên bi đó là bi vàng}
	\loigiai{
		Xét phép thử chọn ngẫu nhiên một viên bi.\\
		Gọi $A$ là biến cố: \lq\lq Chọn được viên bi màu đỏ\rq\rq~; $B$ là biến cố: \lq\lq Chọn được viên bi đã được đánh số\rq\rq~.\\
		Theo bài ra ta có $P\left(B\left| A\right.\right)=60\%,\,\,P\left(B\left|\overline{A}\right.\right)=50\%$ \\
		a) Đúng: Xác suất chọn được viên bi màu đỏ là $P(A)=\dfrac{50}{80}=62,5\%$ .\\
		b) Sai: Ta có $P\left(\overline{A}\right)=\dfrac{30}{80}=37,5\%$ , $P\left(B\left|\overline{A}\right.\right)=50\%$ .\\
		Xác suất chọn được viên bi màu vàng đã được đánh số là\\
		$P\left(\overline{A}\cap B\right)=P\left(\overline{A}\right).P\left(B\left|\overline{A}\right.\right)=37,5\%.50\%=18,75\%$ .\\
		c) Đúng: Xác suất chọn được viên bi đã được đánh số là\\
		$P(B)=P(A).P\left(B\left| A\right.\right)+P\left(\overline{A}\right).P\left(B\left|\overline{A}\right.\right)=62,5\%.60\%+37.5\%.50\%=56,25\%$\\
		Suy ra xác suất chọn được viên bi chưa đánh số là $P\left(\overline{B}\right)=1-P(B)=43,75\%$ .\\
		d) Sai: Giả sử viên bi được lấy ra là viên bi chưa được đánh số. Khi đó:\\
		Xác suất để viên bi đó là bi đỏ: $P\left(A\left|\overline{B}\right.\right)=\dfrac{P(A).P\left(\overline{B}\left| A\right.\right)}{P\left(\overline{B}\right)}=\dfrac{62.5\%.\left(1-60\%\right)}{43,75\%}=\dfrac{4}{7}$ .\\
		Xác suất để viên bi đó là bi vàng: $P\left(\overline{A}\left|\overline{B}\right.\right)=\dfrac{P\left(\overline{A}\right).P\left(\overline{B}\left| A\right.\right)}{P\left(\overline{B}\right)}=\dfrac{37,5\%.\left(1-50\%\right)}{43,75\%}=\dfrac{3}{7}$ .\\
		Vây xác suất viên bi đó là bi đỏ cao hơn xác suất viên bi đó là bi xanh.
	}
\end{ex}
\begin{ex}%Câu 16
	Trong không gian với hệ tọa độ $Oxyz$ , một cabin cáp treo xuất phát từ điểm $A\left(10;3;0\right)$ và chuyển động đều theo đường cáp có vectơ chỉ phương là $\vec{u}=\left(2;-2;1\right)$ với tốc độ $4,5$ m/s (đơn vị trên mỗi trục tọa độ là mét).\\
	\centerline{\includegraphics{images/3.16.jpg}}
	\choiceTF
	{\True Phương trình tham số của đường cáp là: $\left\{\begin{aligned}
				 & x=10+2t \\
				 & y=3-2t  \\
				 & z=t     \\
			\end{aligned}\right.\left(t\in\mathbb{R}\right)$}
	{\True Giả sử sau thời gian $t$ (s) kể từ lúc xuất phát $\left(t\ge 0\right)$ thì cabin đến điểm $M$ . Khi đó tọa độ điểm $M$ là $M\left(3t+10;-3t+3;\dfrac{3t}{2}\right)$}
	{Cabin dừng ở điểm $B$ có hoành độ $x_B=550$ , khi đó quãng đường $AB$ dài 800 m}
	{Đường cáp $AB$ tạo với mặt phẳng $\left(Oxy\right)$ một góc $30^\circ $}
	\loigiai{
		a) Đúng: Phương trình tham số của đường cáp là: $\left\{\begin{aligned}
				 & x=10+2t \\
				 & y=3-2t  \\
				 & z=t     \\
			\end{aligned}\right.\left(t\in\mathbb{R}\right)$\\
		b) Đúng: Ta có $AM=v.t=4,5t$ và ta gọi $M\left(10+2m\,;\,\,3-2m\,;\,\,m\right)$ thuộc đường thẳng $d$\\
		Khi đó: $\overrightarrow{AM}=\left(2m;\,-2m;\,m\right)$ và $\overrightarrow{AM}$ cùng hướng với vectơ $\overrightarrow{u}$ nên $m$ dương\\
		Suy ra $AM=\sqrt{4m^2+4m^2+m^2}=3\left| m\right|\xrightarrow{m>0}$ $m=1,5t$ nên $M\left(3t+10;-3t+3;\dfrac{3t}{2}\right)$\\
		c) Sai: Từ câu trên suy ra $M\equiv B\Leftrightarrow 10+3t=550\Leftrightarrow t=180$\\
		Khi đó: $AB=vt=4,5.t=4,5.180=810$ mét\\
		d) Sai: Ta có $\overrightarrow{u_{AB}}=\left(2;\,-2;\,1\right)$ và mặt phẳng $\left(Oxy\right)$ là $z=0$ nên ta có $\overrightarrow{n}=\left(0\,;\,0;\,1\right)$\\
		Từ đó: $\sin\alpha=\left|\dfrac{\overrightarrow{u}.\overrightarrow{n}}{\left|\overrightarrow{u}\right|.\left|\overrightarrow{n}\right|}\right|=\dfrac{1}{3}$ nên $\alpha\ne 30^\circ $ }
\end{ex}
\Closesolutionfile{ans}
\TNSA
\Opensolutionfile{ans}[ans/ans-3-SA]
\begin{ex}%Câu 17
	Cho khối lăng trụ tam giác đều $ABC.A'{B}'{C}'$ . Biết số đo góc nhị diện $\left[A',BC,A\right]$ bằng $30^\circ $ và tam giác $A'BC$ có diện tích bằng 32. Khoảng cách giữa hai đường thẳng $AB$ và $A'{C}'$ bằng bao nhiêu?
	\shortans{4}
	\loigiai{
	\immini{
	Gọi $M$ là trung điểm của $BC$ thì ta có $\left\{\begin{matrix}
			BC\perp AM    \\
			BC\perp A{A}' \\
		\end{matrix}\Rightarrow BC\perp\left(A'AM\right)\right.$\\
	Suy ra $\left[A',BC,A\right]=\widehat{A'MA}=30^\circ $ và đặt $AB=BC=CA=x$ $\Rightarrow\left\{\begin{matrix}
			AM=\dfrac{\sqrt{3}}{2}x                   \\
			{A}'M=\dfrac{2S{A}'.BC}{BC}=\dfrac{64}{x} \\
		\end{matrix}\right.$\\
	$\Rightarrow\text{cos}30^\circ=\dfrac{AM}{A'M}=\dfrac{\sqrt{3}}{2}x:\dfrac{64}{x}=\dfrac{\sqrt{3}{x^2}}{128}=\dfrac{\sqrt{3}}{2}\Rightarrow x=8$\\
	Vậy $A{A}'=AM.\text{tan}30^\circ=\dfrac{\sqrt{3}}{2}x.\dfrac{1}{\sqrt{3}}=4\Rightarrow d=4$\\
	Khoảng cách giữa hai đường thẳng $AB$ và $A'{C}'$ bằng $4$}
	{\begin{tikzpicture}[line cap = round, line join = round, scale = 1, declare function={a=4; b=3; h=4; g=30;}]
		\path
		(0,0) coordinate (A)
		+(90:h) coordinate (A')
		(-g:b) coordinate (B)
		+(90:h) coordinate (B')
		(0:a) coordinate (C)
		+(90:h) coordinate (C')
		($(B)!.5!(C)$) coordinate (M)
		;
		\draw (A)--(B)--(C)--(C')--(A')--cycle (A')--(B')--(C') (A')--(B)--(B');
		\draw [dashed] (A')--(C)--(A)--(M)--(A');
		\foreach \x/\g in {A/180,B/-45,C/0,A'/135,B'/-45,C'/45,M/-45} {\fill (\x)circle(1.5pt) node[shift=(\g:10pt)] {$\x$};}
	\end{tikzpicture}}
	}
\end{ex}
\begin{ex}%Câu 18
	\immini[thm]{
		Một người đưa thư xuất phát từ bưu điện ở vị trí $A$ , các điểm cần phát thư nằm dọc theo các con đường cần phải đi qua. Biết người này phải đi trên mỗi con đường ít nhất một lần (để phát được thư cho tất cả các điểm cần phát nằm dọc theo con đường đó) và cuối cùng quay lại điểm xuất phát. Độ dài các con đường như hình vẽ (đơn vị độ dài). Hỏi tổng quãng đường người đưa thư có thể đi ngắn nhất có thể bằng bao nhiêu?
	}{\begin{tikzpicture}[line cap = round, line join = round,scale=1]
			\tikzstyle{vertex}=[circle,draw,minimum size=12pt,inner sep=2pt]
			\foreach \d/\x/\y in {A/0/0, B/3/0, C/4/-1.5, D/3/-3, E/0/-3} {\node[vertex] (\d) at (\x,\y) {$\d$};}
			\foreach \a/\b/\l in {A/B/8, B/C/5, C/D/2, D/B/4, D/E/9, E/B/10, A/E/6} {\draw (\a)-- node [midway,sloped,above] (L){\phantom{L}} (\b); \node at (L) {$\l$};}
			\draw (A) to[out=-135, in=135] node [midway,left] {$7$} (E);
		\end{tikzpicture}}
	\shortans{39}
	\loigiai{
		Quãng đường người đưa thư đi từ $A$ đến $D$ là: $7+6+10+4+8+5+2+9=51$\\
		Quãng đường ngắn nhất đi từ $D$ về $A$ là qua $B$ : $4+8=12$\\
		Tổng quãng đường người đưa thư có thể đi ngắn nhất có thể là: $51+12=63$ .}
	\loigiai{
		Đi từ $C$ đến $E$ theo đường đi Euler dài: $2+1+4+3+5+6+10=31$ (km)\\
		Đi từ $E$ đến $C$ với quảng đường ngắn nhất dài: $5+1+2=8$ (km)\\
		Vậy tổng số kilomet mà người đưa thư phải đi nhỏ nhất là $31+8=39$ (km)}
\end{ex}
\begin{ex}%Câu 19
	Cho một tấm nhôm hình lục giác đều cạnh $90$ cm. Người ta cắt ở mỗi đỉnh của tấm nhôm hai hình tam giác vuông bằng nhau, biết cạnh góc vuông nhỏ bằng $x$ (cm) (cắt phần tô đậm của tấm nhôm) rồi gập tấm nhôm như hình vẽ để được một hình lăng trụ lục giác đều không có nắp. Tìm x để thể tích của khối lăng trụ lục giác đều trên là lớn nhất.\\
	\centerline{\includegraphics[width=.7\textwidth]{images/3.19}}
	\shortans{15}
	\loigiai{
	% {\color{red}HÌNH Ở ĐÂY}\\
	Điều kiện $0<x<45$\\
	Cạnh đáy của lăng trụ lục giác đều:$AB=HK=90-2x$\\
	Chiều cao của lăng trụ lục giác đều: $HA=MH.\tan 60^\circ=x\sqrt{3}$\\
	Diện tích đáy của lăng trụ lục giác đều: $S_{ABCDEF}=6S_{ABO}=6.\dfrac{\sqrt{3}}{4}{\left(90-2x\right)^2}=9x(45-x)^2$\\
	Thể tích của khối lăng trụ lục giác đều: $V(x)=HA.S_{ABCDEF}=9x\left(45-x\right)^2$\\
	Hay $V(x)=9x\left(45-x\right)^2$\\
	Xét hàm số $V(x)=9x\left(45-x\right)^2$ trên khoảng $\left(0;\,45\right)$.\\
	Bảng biến thiên:
	\begin{center}
		\begin{tikzpicture}
			\tkzTabInit[lgt=1.2,espcl=2]
			{$x$ /.7, $V’(x)$ /.7, $V(x)$ /2}
			{$-\infty$,$15$,$45$,$+\infty$}
			\tkzTabLine{ ,+,z,-,z,+, }
			\tkzTabVar{-/$-\infty$,+/$V(15)$,-/$V(45)$,+/$+\infty$}
		\end{tikzpicture}
	\end{center}
	Từ bảng biến thiên ta có: $\max \limits_{(0;45)}V(x)=V(15)=243000$ (cm$^3$)\\
	Vậy thể tích của khối lăng trụ lục giác đều lớn nhất khi và chỉ khi $x=15$ cm}
\end{ex}
\begin{ex}%Câu 20
	Từ một quả cầu bằng đá trắng sứ bán kính bằng 1 dm, người ta khoan rút lõi ngay \lq\lq chính giữa\rq\rq~ quả cầu (trục đối xứng của lõi và quả cầu trùng nhau) như hình sau với đường kính mũi khoan là 1 dm được một vật thể có thể tích $V$ là bao nhiêu dm3? (Bỏ qua độ dày mũi khoan và kết quả làm tròn đến chữ số thập phân thứ hai sau dấu phẩy)\\
	\centerline{
		\includegraphics[width=.2\textwidth]{images/3.20a.png}
		\includegraphics[width=.25\textwidth]{images/3.20b.jpg}
		\includegraphics[width=.25\textwidth]{images/3.20c.png}
	}
	\shortans{1,47}
	\loigiai{
	Gọi $V_1$ là thể tích của khối trụ và $V_2$ là thể tích của chỏm cầu\\
	Nửa chiều cao của khối trụ là: $l=\sqrt{1^2-\left(0,5\right)^2}=\dfrac{\sqrt{3}}{2}$ nên ta có thể suy ra chiều cao của chỏm cầu là: $h=1-\dfrac{\sqrt{3}}{2}$ . Thể tích khối trụ là: $V_1=\pi{R^2}.H=\pi .R^2.2l=\dfrac{\pi\sqrt{3}}{4}$\\
	Thể tích chỏm cầu: $V_2=\pi\displaystyle\int\limits_{R-h}^R{\left(\sqrt{R^2-x^2}\right)^2dx}$ $\Leftrightarrow{V_2}=\pi\displaystyle\int\limits_{R-h}^R{\left(R^2-x^2\right)dx\Leftrightarrow{V_2}=\,}\pi\left.\left(R^2x-\dfrac{x^3}{3}\right)\right|_{R-h}^R$\\
	$\Leftrightarrow{V_2}=\pi\left[\left(R^3-\dfrac{R^3}{3}\right)-\left(R^2\left(R-h\right)-\dfrac{\left(R-h\right)^3}{3}\right)\right]$ $\Leftrightarrow{V_2}=\pi{h^2}\left(R-\dfrac{h}{3}\right)$ .\\
	Thay số ta suy ra được thể tích của chỏm cầu là $V_2=\pi\left(\dfrac{2}{3}-\dfrac{3\sqrt{3}}{8}\right)$\\
	Khi đó thể tích của khối cần tìm là $V=V_1+2V_2=\dfrac{\pi\sqrt{3}}{4}+\pi\left(\dfrac{2}{3}-\dfrac{3\sqrt{3}}{8}\right)=\dfrac{8-3\sqrt{3}}{6}\pi\approx 1,47$ dm3}
\end{ex}
\begin{ex}%Câu 21
	Sân hiên hình chữ nhật của một ngôi nhà là khoảng đất $ABCD$ được lợp mái bằng kính màu để hạn chế ánh sáng đi qua với mái dốc. Các bề mặt bên $ADHE$ và $CGHD$ nằm ở bức tường bên ngoài ngôi nhà. Đặt vào mô hình hệ trục tọa độ như hình vẽ thì ta có $B\left(5;\dfrac{7}{2};0\right)\,;\,\,E\left(5;0;2\right)$ và $H\left(0;0;3\right)$ . Trên tường nhà có một ngọn đèn đặt tại điểm $L$ cách điểm $D$ một khoảng $6$ m theo phương thẳng đứng. Phần có mái của sân hiên in bóng lên khu vườn bằng phẳng phía trước ngôi nhà dưới ánh đèn tạo thành khoảng đất hạn chế ánh sáng. Tính diện tích khoảng đất đó (Kết quả làm tròn kết quả đến hàng phần chục).
	\shortans{45,9}
	\loigiai{
	{\color{red}HÌNH Ở ĐÂY}\\
	Ta có $H$ là trung điểm của $DL$ nên $GH$ là đường trung bình của $\Delta LD{G}'$ nên ta suy ra $G$ là trung điểm của $L{G}'$ .\\
	Mặt khác: $GC$ là đường trung bình của $\Delta LD{G}'$ nên $G'\left(0;\,7\,;0\right)$\\
	Ta có: $\dfrac{E'A}{ED}=\dfrac{EA}{LD}=\dfrac{2}{6}=\dfrac{1}{3}\Rightarrow{E}'A=\dfrac{1}{3}{E}'D\Rightarrow{E}'A=\dfrac{1}{2}AD=2,5\Rightarrow{E}'\left(7,5;0 ;0\right)$\\
	Mặt khác: $\dfrac{EF}{E'{F}'}=\dfrac{LE}{L{E}'}=\dfrac{2}{3}\Rightarrow{E}'{F}'=1,5.EF=1,5.3,5=5,25$\\
	Diện tích khoảng đất khi đó là: $S_{DG'F'E'}=\dfrac{\left(DG'+E'F'\right).DE'}{2}=\dfrac{(7+5,25).7,5}{2}=45,9$
	}
\end{ex}
\begin{ex}%Câu 22
	Có hai chiếc hộp, hộp I có $5$ viên bi màu trắng và $5$ viên bi màu đen; hộp II có $6$ viên bi màu trắng và $4$ viên bi màu đen. Các viên bi có cùng kích thước và khối lượng. Lấy ngẫu nhiên đồng thời hai viên bi từ hộp I bỏ sang hộp II. Sau đó lấy ngẫu nhiên một viên bi từ hộp II. Lấy ra ngẫu nhiên một viên bi, giả sử viên bi được lấy ra là viên bi màu trắng. Tính xác suất viên bi màu trắng đó thuộc hộp I (Kết quả làm tròn $2$ chữ số thập phân)
	\shortans{0,14}
	\loigiai{
		Xét các biến cố: $A$ : \lq\lq Viên bi lấy ra là viên màu trắng\rq\rq~\\
		$B_1$ : \lq\lq 2 viên bi lấy ra từ hộp I có màu trắng\rq\rq~; $B_2$ : \lq\lq 2 viên bi lấy ra từ hộp I có màu đen\rq\rq~.\\
		$B_3$ : \lq\lq 2 viên bi lấy ra từ hộp I có cả hai màu đen trắng\rq\rq~.\\
		Ta có: $P\left(B_1\right)=\dfrac{\text{C}_5^2}{\text{C}_{10}^2}=\dfrac{2}{9};\,\,P\left(B_2\right)=\dfrac{\text{C}_5^2}{\text{C}_{10}^2}=\dfrac{2}{9};\,P\left(B_3\right)=\dfrac{\text{C}_5^1\cdot\text{C}_5^1}{\text{C}_{10}^2}=\dfrac{5}{9}.$\\
		Áp dụng công thức xác suất toàn phần, ta có\\
		$P(A)=P\left(A|B_1\right).P\left(B_1\right)+P\left(A|B_2\right).P\left(B_2\right)+P\left(A|B_3\right).P\left(B_3\right)$ $=\dfrac{8}{12}.\dfrac{2}{9}+\dfrac{6}{12}.\dfrac{2}{9}+\dfrac{7}{12}.\dfrac{5}{9}=\dfrac{7}{12}$\\
		Gọi $C$ là biến cố: \lq\lq Viên bi lấy ra là bi trắng từ hộp I\rq\rq.\\
		Khi đó ta có:\\
		$P(C)=P(C|B_1).P(B_1)+P(C|B_2).P(B_2)+P(C|B_3).P(B_3) = \dfrac{2}{12}. \dfrac{2}{9}+0.\dfrac{2}{9}+\dfrac{1}{12}.\dfrac{5}{9}=\dfrac{1}{12}$.\\
		Vậy xác suất cần tìm là:\\ 
		$P\left(C|A\right)=\dfrac{P(CA)}{P(A)}=\dfrac{\dfrac{1}{12}}{\dfrac{7}{12}}=\dfrac{1}{7}\approx 0,14 $ (vì $C \subset A$)\\
		}
\end{ex}
\Closesolutionfile{ans}
\Closesolutionfile{ansbook}
\inputansbox{6,2,3}{ans/ans-3-T,ans/ans-3-TF,ans/ans-3-SA}
% \begin{name}
    {\tenchude}
    {\tendethi}
    {\tentruong}
    {\thoigian}
\end{name}
\Opensolutionfile{ansbook}[ans/ansbook-4]
\TN
\Opensolutionfile{ans}[ans/ans-4-T]
\begin{ex}%Câu 1
Cho cấp số cộng $\left(u_n\right)$ với $u_1=-1$ và $u_2=4.$ Giá trị của $u_3$ bằng
\choice
{\True $ 9$}
{$-16$}
{$ 7$}
{$-8$}
\loigiai{
Ta có: $ d=u_2-u_1=4-\left(-1\right)=5$ nên $u_3=u_2+d=4+5=9$.}
\end{ex}

\begin{ex}%Câu 2
Cho hàm số $ y=\dfrac{ax+b}{cx+d}$ với $ a,b,c \in\mathbb{R}$ có đồ thị như hình vẽ. Mệnh đề nào đúng?\\
\centerline{\begin{tikzpicture}[line join=round, line cap=round,>=stealth,thick]
        \tikzset{every node/.style={scale=0.9}}
        \draw[->] (-4.1,0)--(4.1,0) node[below left] {$x$};
        \draw[->] (0,-4.1)--(0,4.1) node[below left] {$y$};
        \draw (0,0) node [below left] {$O$}
        (1,0)node[below right]{$1$};
        \draw[dashed,thin] (1.01,-4)--(1.01,4);
        \begin{scope}
            \clip (-4,-4) rectangle (4,4);
            \draw[samples=200,domain=-4:0.99,smooth,variable=\x] plot (\x,{(1*(\x)+1)/(1*(\x)+-1)});
            \draw[samples=200,domain=1.01:4,smooth,variable=\x] plot (\x,{(1*(\x)+1)/(1*(\x)+-1)});
            \draw[dashed,thin] (-4,1/1)--(4,1/1);
        \end{scope}
\end{tikzpicture}}
\choice
{$y'>0,\forall x\ne 1$}
{$y'>0,\forall x\in\mathbb{R}$}
{$y'<0,\forall x\in\mathbb{R}$}
{\True $y'<0, \forall x\ne 1$}
\loigiai{
Từ đồ thị ta thấy $y'<0,\forall x\ne 1.$}
\end{ex}

\begin{ex}%Câu 3
Trong không gian $ Oxyz,$ cho điểm $ M\left(1;2;-1\right)$ và mặt phẳng $(P):x+2y+z=0.$ Mặt phẳng $(Q)$ qua $ M$ và song song với $(P)$ có phương trình là
\choice
{$ x+2y+z+4=0$}
{$ x+2y+z-1=0$}
{$ x+2y-z-6=0$}
{\True $ x+2y+z-4=0$}
\loigiai{
Mặt phẳng $(Q)$ có phương trình: $ x+2y+z+d=0$ với $ d\ne 0$\\
Điểm $ M\in(Q)$ nên suy ra $ 1+2.2+\left(-1\right)+d=0\Rightarrow d=-4$\\
Vậy phương trình mặt phẳng $(Q)$ là: $ x+2y+z-4=0.$}
\end{ex}

\begin{ex}%Câu 4
Số nghiệm nguyên của bất phương trình $\log_{0,5}\left(2x+6\right)\ge-5$ là
\choice
{\True $ 16$}
{$ 13$}
{$ 15$}
{$ 8$}
\loigiai{
Bất phương trình đã cho $\Leftrightarrow 0<2x+6\le{\left(0,5\right)^{-5}}\Leftrightarrow 0<2x+6\le 32\Leftrightarrow-3<x\le 13$\\
Vì $ x\in\mathbb{Z}$ nên có tất cả $ 16$ nguyện nguyên thỏa mãn.}
\end{ex}

\begin{ex}%Câu 5
Cho hình lập phương $ABCD.A'{B}'{C}'{D}'$ có cạnh $ 2$ (tham khảo hình vẽ bên dưới). Độ dài của vectơ $\vec{u}=\vec{A'{C}'}-\vec{A'A}$ bằng\\
\centerline{
\begin{tikzpicture}[line cap=round,line join=round, >=stealth,scale=.7]
    \def \a{-1.5} \def \b{-1}\def \c{4.5} \def \h{4}
    \path (.5,.5)coordinate(A) 
    +(\a,\b)coordinate(B)
    +(\c,0)coordinate(D)
    ($(B)+(D)-(A)$)coordinate(C)
    +(0,\h) coordinate(C')
    ($(B)+(C')-(C)$)coordinate(B')
    ($(A)+(C')-(C)$)coordinate(A')
    ($(D)+(C')-(C)$)coordinate(D');
    \draw [dashed] (A)--(B)(D)--(A)--(A');
    \draw (B')--(B)--(C)(B')--(C')--(C)--(D)--(D')--(A')--(B')(C')--(D');
    \foreach \x/\g in {A/135,B/-135,C/-45,D/0,A'/135,B'/180,C'/-20,D'/0}\fill[red] (\x) circle (1pt)+(\g:3mm) node[black]{$\x$};
\end{tikzpicture}
}
\choice
{$ 2\sqrt{2}$}
{$\sqrt{3}$}
{$ 2\sqrt{6}$}
{\True $ 2\sqrt{3}$}
\loigiai{
Ta có $\vec{u'}=\vec{A{C}'}-\vec{A'A}=\vec{A{A}'}+\vec{A'{C}'}=\vec{A{C}'}\Rightarrow\left|\vec{u'}\right|=\left|\vec{A{C}'}\right|=A{C}'$\\
Khi đó $A{C}'=\sqrt{A{A^2}+A'{C^{\prime 2}}}=\sqrt{2^2+8}=2\sqrt{3}$}
\end{ex}

\begin{ex}%Câu 6
Biết $\displaystyle\int{f(x)\mathrm{\,d}x=\cos x+C}$ thì $\displaystyle\int{f'(x)\mathrm{\,d}x}$ bằng
\choice
{$\sin x+C'$}
{$\cos x+C'$}
{\True $-\sin x+C'$}
{$-\cos x+C$}
\loigiai{
Ta có $ f(x)=\left(\displaystyle\int{f(x)\mathrm{\,d}x}\right)'=\left(\cos x+C\right)'=-\sin x$ nên $\displaystyle\int{f(x)\mathrm{\,d}x=-\sin x+C}$}
\end{ex}

\begin{ex}%Câu 7
Cho hàm số $f(x)$ xác định trên $\left(-\infty ;0\right)\setminus\left\{-2\right\}$ và có bảng biến thiên bên dưới. Đồ thị hàm số đã cho có tổng số đường tiệm cận ngang và tiệm cận đứng là\\
\centerline{
\begin{tikzpicture}[>=stealth]
    \tkzTabInit[espcl=3,lgt=1.5,nocadre=false]
    {$x$/0.7,$f'(x)$/1,$f(x)$/3}
    {$-\infty$,$-2$,$+\infty$}
    \tkzTabLine{,+,d,-,}
    \tkzTabVar{-/$-\infty$,+D+/$+\infty$/$1$,-/$0$}
\end{tikzpicture}
}
\choice
{\True $ 1$}
{$ 3$}
{$ 0$}
{$ 2$}
\loigiai{
Từ đồ thị hàm số không có tiệm cận ngang và có một tiệm cận đứng là đường thẳng $ x=-2$.}
\end{ex}

\begin{ex}%Câu 8
Khi cắt vật thể bởi mặt phẳng vuông góc với trục $ Ox$ tại điểm có hoành độ là $ x$ $\left(0\le x\le 3\right),$ ta được mặt cắt là một hình vuông có cạnh là $\sqrt{9-x^2}$ (được mô hình hóa bởi hình vẽ bên dưới). Thể tích của vật thể đó bằng\\
\centerline{\includegraphics{images/4.8}}
\choice
{$ 171\pi $}
{$ 171$}
{$ 18\pi $}
{\True $18$}
\loigiai{
Thể tích vật thể là: $V=\displaystyle\int\limits_0^3S(x)\mathrm{\,d}x=\displaystyle\int\limits_0^3\left(9-x^2\right)\mathrm{\,d}x=18$ .}
\end{ex}

\begin{ex}%Câu 9
Cho lăng trụ tam giác $ ABC.A'{B}'{C}'.$ Biết diện tích mặt bên $ AB{B}'{A}'$ bằng $ 15$ và khoảng cách từ $ C$ đến mặt phẳng $\left(AB{B}'{A}'\right)$ bằng $ 6$ (tham khảo hình vẽ bên cạnh). Thể tích của khối lăng trụ $ ABC.A'{B}'{C}'$ bằng bao nhiêu?\\
\centerline{
    \begin{tikzpicture}[scale=1, font=\footnotesize, line join=round, line cap=round, >=stealth]
        \def\ac{4} % cạnh AC
        \def\ab{2} % cạnh AB
        \def\h{4} % chiều cao
        \def\gocA{50} % góc A của đáy
        \path
        (0,0) coordinate (A)
        (\ac,0) coordinate (C)
        (-\gocA:\ab) coordinate (B)
        ($(A)+(80:\h)$) coordinate (A')
        ($(B)-(A)+(A')$) coordinate (B')
        ($(C)-(A)+(A')$) coordinate (C');
        \draw (A')--(A)--(B)--(C)--(C')--(A')--(B')--(C') (B)--(B');
        \draw[dashed] (A)--(C);
        \foreach \x/\g in {A/180,B/-90,C/0,A'/180,C'/0,B'/-40}\fill[red] (\x) circle (1pt)+(\g:3mm) node[black]{$ \x $};
    \end{tikzpicture}
}
\choice
{$60$}
{\True $45$}
{$90$}
{$30$}
\loigiai{
Ta có: $V_{C.AB{B}'{A}'}=\dfrac{1}{3}.S_{AB{B}'{A}'}.d\left(C,\left(AB{B}'{A}'\right)\right)=\dfrac{1}{3}.15.6=30$\\
Mặt khác: $V_{C.A'{B}'{C}'}=\dfrac{1}{3}.d\left(C,\left(A'{B}'{C}'\right)\right).S_{A'{B}'{C}'}=\dfrac{1}{3}$ thể tích khối lăng trụ\\
Suy ra $V_{C.AB{B}'{A}'}=\dfrac{2}{3}$ thể tích khối lăng trụ nên $V_{ABC.A'{B}'{C}'}=\dfrac{3}{2}.30=45$.}
\end{ex}

\begin{ex}%Câu 10
Trong không gian $Oxyz,$ cho hai điểm $A\left(1;3;2\right)$ và $B\left(4;5;6\right).$ Gọi $\alpha $ là góc giữa đường thẳng $AB$ và mặt phẳng $\left(Oxy\right).$ Giá trị của $\text{cos}\alpha $ bằng
\choice
{$\dfrac{4\sqrt{29}}{29}\cdot $}
{$\dfrac{16}{29}\cdot $}
{\True $\dfrac{\sqrt{377}}{29}\cdot $}
{$\dfrac{13}{29}\cdot $}
\loigiai{
Vectơ pháp tuyến của mặt phẳng $\left(Oxy\right)$ là: $\vec{n_{\left(Oxy\right)}}=\left(0;\,0;\,1\right)$\\
Vectơ chỉ phương của đường thẳng $ AB$ là: $\vec{n_{AB}}=\left(3;\,2;\,4\right)$\\
Ta có: $\sin\alpha=\dfrac{\left|\vec{n}.\vec{u}\right|}{\left|\vec{n}\right|.\left|\vec{u}\right|}=\dfrac{4}{1.\sqrt{9+4+16}}=\dfrac{4}{\sqrt{29}}$ suy ra $\cos\alpha=\sqrt{1-\sin^2\alpha}=\dfrac{\sqrt{377}}{29}$.}
\end{ex}

\begin{ex}%Câu 11
Theo thống kê điểm trung bình môn Toán của một số học sinh đã trúng tuyển vào lớp 10 năm học 2024 – 2025 của một trường được kết quả như bảng sau:
\begin{center}
\begin{tabular}{|c|c|c|c|c|c|c|c|}
    \hline
    Khoảng điểm & $[6,5; 7)$ & $[7; 7,5)$ & $[7,5; 8)$ & $[8; 8,5)$ & $[8,5; 9)$ & $[9; 9,5)$ & $[9,5; 10)$ \\
    \hline
    Tần số & 7 & 10 & 17 & 24 & 13 & 8 & 5 \\
    \hline
\end{tabular}
\end{center}

Khoảng tứ phân vị của mẫu số liệu ghép nhóm trên là (Kết quả làm tròn đến hàng phần chục)
\choice
{\True $\Delta_Q=1,1$}
{$\Delta_Q=1$}
{$\Delta_Q=1,2$}
{$\Delta_Q=0,6$}
\loigiai{
Ta có: $ n=84$ và $\Delta Q=Q_3-Q_1$.\\
Mặt khác: $\dfrac{3n}{4}=63\Rightarrow{Q_3}=\dfrac{\dfrac{3n}{4}-\left(7+10+17+24\right)}{13}.0,5+8,5=\dfrac{113}{13}$\\
$\dfrac{n}{4}=21\Rightarrow{Q_1}=\dfrac{\dfrac{n}{4}-7-10}{17}.\left(8-7,5\right)+7,5=\dfrac{259}{34}$\\
Vậy khoảng tứ phân vị là: $\Delta Q=Q_3-Q_1=\dfrac{113}{13}-\dfrac{259}{34}\approx 1,1$}
\end{ex}

\begin{ex}%Câu 12
Một người gửi tiết kiệm $ 10$ triệu đồng vào một ngân hàng với lãi suất $7\%/$ một năm. Biết rằng nếu không rút tiền ra khỏi ngân hàng thì cứ sau mỗi năm, số tiền lãi sẽ được nhập vào vốn ban đầu. Sau $5$ năm mới rút lãi thì người đó thu được số tiền lãi là
\choice
{$14,026$ triệu đồng}
{$50,7$ triệu đồng}
{\True $4,026$ triệu đồng}
{$3,5$ triệu đồng}
\loigiai{
Số tiền cả gốc và lãi sau $ 5$ năm là: $P_5=P_0\left(1+r\right)^5=10\left(1+7\%\right)^5\approx 14,026$ triệu đồng.\\
Vậy số tiền lãi mà người đó nhận là: $ 14,026-10=4,026$ triệu đồng}
\end{ex}
\Closesolutionfile{ans}
\TNTF
\Opensolutionfile{ans}[ans/ans-4-TF]
\begin{ex}%Câu 13
Cho hàm số $y=\dfrac{a{x^2}+bx+c}{mx+n}$ có đồ thị như hình vẽ sau:\\
\centerline{
    \begin{tikzpicture}[line join=round, line cap=round,>=stealth,thick]
        \tikzset{every node/.style={scale=0.9}}
        \draw[->] (-4.1,0)--(4.1,0) node[below left] {$x$};
        \draw[->] (0,-4.1)--(0,4.1) node[below left] {$y$};
        \draw (0,0) node [below left] {$O$};
        \foreach \x/\nx in {-2/-2,-1/-1}
        \draw[thin] (\x,1pt)--(\x,-1pt) node [below] {$\nx$};
        \foreach \y/\ny in {-2/-2,1/1}
        \draw[thin] (1pt,\y)--(-1pt,\y) node [right] {$\ny$};
        \draw[dashed,thin](-2,0)--(-2,-2)--(0,-2);
        \draw[dashed,thin] (-0.99,-4)--(-0.99,4);
        \begin{scope}
            \clip (-4,-4) rectangle (4,4);
            \draw[samples=200,domain=-4:-1.01,smooth,variable=\x] plot (\x,{(1*((\x)^2)+2*(\x)+2)/(1*(\x)+1)});
            \draw[samples=200,domain=-0.99:4,smooth,variable=\x] plot (\x,{(1*((\x)^2)+2*(\x)+2)/(1*(\x)+1)});
            \draw[dashed,thin] (-4.1,-3.1)--(4.1,5.1);
        \end{scope}
    \end{tikzpicture}
}
\choiceTFt
{Hàm số đã cho nghịch biến trên khoảng $(-2;0)$}
{\True Đồ thị của hàm số đã cho có tiệm cận xiên $y=x+1$}
{Gọi $A,B$ là hai điểm cực trị của hàm số đã cho, diện tích của tam giác $OAB$ bằng $8$ (với $O$ là gốc tọa độ)}
{\True Một trục đối xứng của đồ thị đã cho là $d:y=\left(x+1\right)\tan\dfrac{3\pi}{8}\cdot $}
\loigiai{
a) Sai: Đồ thị đi xuống trên các khoảng $\left(-2;-1\right),\left(-1;0\right)$ nên nghịch biến trên các khoảng này.\\
b) Đúng: Tiệm cận xiên qua hai điểm $\left(-1;0\right),\left(0;1\right)$ là $ y=x+1$.\\
c) Sai: Hai điểm cực trị của đồ thị là $ A\left(0;2\right),B\left(-2;-2\right)\Rightarrow{S_{OAB}}=\dfrac{1}{2}\left| 0.\left(-2\right)-\left(-2\right).2\right|=2$.\\
d) Đúng: Tiệm cận đứng $ x=-1$. Trục đối xứng của đồ thị hàm số là đường phân giác của góc tạo bởi tiệm cận đứng và tiệm cận xiên:\\
$\dfrac{x-y+1}{\sqrt{2}}=\pm\left(x+1\right)\Leftrightarrow y=\left(x+1\right)\pm\sqrt{2}\left(x+1\right)=\left(1\pm\sqrt{2}\right)\left(x+1\right)=\tan\left(\dfrac{3\pi}{8}\right)$.}
\end{ex}
\begin{ex}%[Nguyễn Tuấn, dự án sáng tác đề 12 theo chủ đề]%[2H5V2-7] 
    Trong không gian với hệ tọa độ $O x y z$, một cabin cáp treo xuất phát từ điểm $A(10 ; 3 ; 0)$ và chuyển động đều theo đường cáp có véc-tơ chỉ phương là $\vec{u}=(2 ;-2 ; 1)$ với tốc độ là 4,5 m/s (đơn vị trên mỗi trục tọa độ là mét) (Hình bên dưới).
    \definecolor{ecru}{rgb}{0.76, 0.7, 0.5}
    \definecolor{darkolivegreen}{rgb}{0.33, 0.42, 0.18}
    \definecolor{deepskyblue}{rgb}{0.0, 0.75, 1.0}
    \definecolor{antiquebrass}{rgb}{0.8, 0.58, 0.46}
    \definecolor{arsenic}{rgb}{0.23, 0.27, 0.29}
    \definecolor{ashgrey}{rgb}{0.7, 0.75, 0.71}\definecolor{alizarin}{rgb}{0.82, 0.1, 0.26}
    \begin{center}
        \begin{tikzpicture}[line join=round, line cap=round,scale=1,transform shape]
            \clip (-3,-3) rectangle (3,3);
            \fill[bottom color=white,top color=deepskyblue!90, middle color=white] (-3,-3) rectangle (3,3);
            \tikzset{dat/.pic={
                    \def\N{ 
                        (-3,3)
                        ..controls +(20:.7) and +(150:.7) ..(-.8,1.5)
                        ..controls +(-30:.7) and +(150:.6) ..(1.5,-.8)
                        ..controls +(-30:1) and +(90:.4) ..(3,-3)--(-3,-3)--cycle
                        ;
                    }
                    \draw[black]\N;
                    \fill[darkolivegreen!50] \N;
            }}
            \tikzset{cap_treo/.pic={
                    \def\X{ 
                        (-.85,1)--(-.8,1)
                        ..controls +(-90:.9) and +(-180:.6) ..(-.1,0.1)--(-.1,.05)
                        ..controls +(-180:.65) and +(-90:.9) ..cycle
                        (-.15,0.05)--(-.15,-.1)--(-.23,-.22)--(-.6,-.22)--(-.5,-.1)--(-.5,.08)
                        ;
                    }
                    %\fill[black] \X;
                    \draw\X;
                    \draw(-3,2.3)--(3.5,-1.2);
                    \draw(-1.2,1)--(-.5,1);
                    \draw(-.5,1)--(-.9,1.2)--(-.95,1.17)--(-.6,.97)--cycle;
                    \draw[fill=arsenic!80](-.9,-.05)--(.52,-.05)--(.28,-.3)--(-1.05,-.3)--cycle;
                    \def\M{ 
                        (-.85,1)--(-.8,1)
                        ..controls +(-90:.9) and +(-180:.6) ..(-.1,0.1)--(-.1,.05)
                        ..controls +(-180:.65) and +(-90:.9) ..cycle
                        (-.15,0.05)--(-.15,-.1)--(-.23,-.22)--(-.6,-.22)--(-.5,-.1)--(-.5,.08)
                        ;
                    }
                    \fill[arsenic] \M;
                    \draw\M;
                    \def\N{ 
                        (-1.05,-.3)
                        ..controls +(-120:.6) and +(120:.6) ..(-.98,-1.5)--(.4,-1.5)
                        ..controls +(70:.6) and +(-60:.3) ..(.28,-.3)--cycle
                        ;
                    }
                    \fill[arsenic] \N;
                    \draw\N;
                    \def\P{ 
                        (.4,-1.5)
                        ..controls +(70:.6) and +(-60:.3) ..(.28,-.3)--(.52,-.05)
                        ..controls +(-40:.4) and +(50:.4) ..(.6,-1.2)--cycle
                        ;
                    }
                    \fill[arsenic!80] \P;
                    \draw\P;
                    \draw (-1.23,-1)--(.5,-1)--(.77,-.6);
                    \draw (-1.05,-.5)--(-1,-.4)--(.2,-.4)--(.25,-.5)--cycle
                    (.25,-.5)--(.3,-.6)--(.3,-.6)--(-1.1,-.6)--(.-1.05,-.5)--cycle
                    ;
                    \draw[alizarin](.42,-1)..controls +(100:.4) and +(-65:.4) ..(.2,-.3);
                    \draw[alizarin](-1.15,-1)..controls +(100:.3) and +(-120:.3) ..(-.95,-.3);
            }}
            \path
            (0,0)pic[scale=1]{dat}(0,0)pic[scale=1]{cap_treo};
        \end{tikzpicture}
        \begin{tikzpicture}[scale=1, font=\footnotesize, line join=round,xscale=.2, line cap=round,>=stealth]
            \def\a{1/16} 
            \def\xmin{-3} \def\xmax{12}
            \def\ymin{-3} \def\ymax{3} 
            \coordinate (O) at (0,0);
            \coordinate (E) at (-10,-3);
            \coordinate (N) at ($(E)!.7!(O)$);
            \coordinate (P) at ($(E)!.2!(O)$);
            \coordinate (D) at ($(E)!.3!(O)$);
            \coordinate (A) at ($(N)+(3,0)$);
            \coordinate (B) at (-14,1);
            \coordinate (M) at ($(A)!.7!(B)$);
            \draw[->] (\xmin,0)--(\xmax,0) node [above right]{$y$};
            \draw[->] (O)--(0,\ymax) node [left]{$z$};
            \draw[->] (O)--(E) node [below right]{$x$};
            \node at (0,0)[above right]{$O$};
            \draw[dashed] (B)--(P)node [right]{$550$} (N)--(A)node [below right]{$A(10;3;0)$}--(3,0);
            \draw(B)node [above]{$B$}--(A);
            \draw[->,red](O)--(-4,.6)node [above right]{$\vec{u}$};
            \node at (N) [left]{$10$};
            \node at (M) [above]{$M$};
            \node at (P) [left]{$x_B$};
            \draw[fill=black] (3,0) circle (.5pt) node[below]{\footnotesize $3$};
            %	\path (-28,0) node[opacity=.5,scale=.5] {\includegraphics{images/h35}};
            \clip (\xmin+0.1,\ymin+0.1) rectangle (\xmax-0.1,\ymax-0.1);
        \end{tikzpicture}
    \end{center}
    \choiceTF
    {Phương trình chính tắc của đường cáp là $\dfrac{x+10}{2}=\dfrac{y+3}{-2}=\dfrac{z}{1}$}
    {\True Giả sử sau $t$ (s) kể từ lúc xuất phát $(t \geq 0)$, cabin đến điểm $M$. Khi đó điểm $M$ có tọạ độ là $\left(3 t+10 ;-3 t+3 ;\dfrac{3t}{2}\right)$}
    {Cabin dừng ở điểm $B$ có hoành độ $x_B=550$. Khi đó $AB=750$ m}
    {\True Đường cáp $AB$ tạo với mặt phẳng $(Oxy)$ góc $19^\circ$ (kết quả làm tròn đến hàng đơn vị của độ)}
    \loigiai{
        \begin{itemchoice}
            \itemch Phương trình chính tắc của đường cáp là $\dfrac{x-10}{2}=\dfrac{y-3}{-2}=\dfrac{z}{1}$.
            \itemch Do tốc độ chuyển động của cabin là $4{,}5$ m/s nên độ dài $A M$ bằng $4{,}5 t$ m. \\
            Vì vậy $\left|\vec{AM}\right|=4{,}5 t$ $(t \geq 0)$.\\
            Do hai véc-tơ $\vec{A M}$ và $\vec{u}$ là cùng phương và cùng hướng nên $\vec{A M}=k \vec{u}$ với $k$ là số thực dương nào đó. \\
            Suy ra $\left|\vec{A M}\right|=k|\vec{u}|=k \cdot \sqrt{2^2+(-2)^2+1}=3 k$. Do đó $3 k=4{,}5 t$. Suy ra $k=\dfrac{3 t}{2}$. \\
            Vì thế, ta có $\vec{A M}=\dfrac{3 t}{2} \vec{u}=\left(3 t ;-3 t ; \dfrac{3 t}{2}\right)$.\\
            Gọi tọa độ của điểm $M$ là $\left(x_M ; y_M ; z_M\right)$.\\
            Ta có $\vec{A M}=\left(x_M-x_A ; y_M-y_A ; z_M-z_A\right)=\left(3 t ;-3 t ; \dfrac{3 t}{2}\right)$.\\
            Nên $\heva{&x_M=3 t+x_A \\ &y_M=-3 t+y_A \\ &z_M=\dfrac{3 t}{2}+z_A}\Leftrightarrow\heva{&x_M=3 t+10 \\ &y_M=-3 t+3 \\ &z_M=\dfrac{3 t}{2}.}$\\
            Vậy điểm $M$ có tọạ độ là $\left(3 t+10 ;-3 t+3 ;\dfrac{3 t}{2}\right)$.
            \itemch Do $x_B=550$ nên $3 t+10=550$, tức là $t=180$ s. Do đó, ta có điểm $B(550 ;-537 ; 270)$. \\
            Vậy $A B=\sqrt{(550-10)^2+(-537-3)^2+(270-0)^2}=\sqrt{656100}=810$ m.
            \itemch Đường thẳng $AB$ có véc-tơ chỉ phương $\vec{u}=(2 ;-2 ; 1)$ và mặt phẳng $(Oxy)$ có véc-tơ pháp tuyến $\vec{k}=(0 ; 0 ; 1)$. Do đó, ta có
            $$
            \sin (\Delta,(O x y))=|\cos (\vec{u},\vec{k})|=\dfrac{\left|\vec{u} \cdot \vec{k}\right|}{|\vec{u}| \cdot|\vec{k}|}=\dfrac{1}{3 \cdot 1}=\dfrac{1}{3}.$$
            Vậy $(\Delta,(O x y)) \approx 19^{\circ}$.
        \end{itemchoice}
    }
\end{ex}

\begin{ex}%Câu 15
Một đoàn tàu đang đứng yên trong sân ga, ngay trước đầu tàu có một cái cây. Đoàn tàu khởi hành từ trạng thái đứng yên với gia tốc $ a=0,005t\text(\text{m/}{\text{s}^{\text{2}}})$ và đi qua cái cây trong thời gian $60$ giây. Sau $80$ giây đoàn tàu chuyển sang trạng thái chuyển động đều. Xét tính đúng sai của các khẳng định sau:\\
\centerline{\includegraphics[width=.6\textwidth]{images/4.15}}
\choiceTF
{Vận tốc của đoàn tàu là $v=5.10^{-3}{t^2}\text(\text{m/s})$}
{\True Chiều dài của đoàn tàu là $l=180$m}
{\True Sau $80$ giây, đoàn tàu chuyển động với tốc tốc $57,6\text(\text{km/h})$}
{Sau khi chuyển động đều một thời gian, đoàn tàu gặp một cây cầu có chiều dài $480$ m, khi đó đoàn tàu đi qua cây cầu đó trong thời gian $30$ giây}
\loigiai{

a) Sai: Vận tốc của tàu là $ v(t)=\displaystyle\int{a(t)\mathrm{\,d}t}=\displaystyle\int{0,005t\,\mathrm{\,d}t}=2,5.10^{-3}{t^2}\,(\text{m/s})$ với $ v(0)=0$\\
b) Đúng: Chiều dài của đoàn tàu bằng quãng đường tàu đi trong 60 giây đầu tiên và bằng\\
$ l=\displaystyle\int\limits_0^{60}{v(t)\mathrm{\,d}t}=\displaystyle\int\limits_0^{60}{2,5.10^{-3}{t^2}\mathrm{\,d}t}=180$ m\\
c) Đúng: Sau 80 giây, đoàn tàu chuyển động với tốc tốc:\\
$ v\left(80\right)=2,5.10^{-3}{80^2}=16\,\,\left(\text{m}/\text{s}\right)=57,6\left(\text{km}/\text{h}\right)$\\
d) Sai: Sau khi chuyển động đều một thời gian, đoàn tàu gặp một cây cầu có chiều dài 480 (m).\\
Khi đó đoàn tàu đi qua cây cầu đó trong thời gian $ t=\dfrac{480+180}{16}=41,25$ giây.}
\end{ex}

\begin{ex}%Câu 16
Có hai hộp đựng phiếu thi, mỗi phiếu ghi một câu hỏi. Hộp thứ nhất có $15$ phiếu và hộp thứ hai có $9$ phiếu. Sinh viên A đi thi chỉ thuộc $10$ câu ở hộp thứ nhất và $8$ câu ở hộp thứ hai (Kết quả làm tròn đến 2 chữ số sau dấu phẩy)
\choiceTF
{\True Thầy giáo rút ngẫu nhiên từ mỗi hộp ra một phiếu thi, sau đó cho sinh viên A rút ngẫu nhiên ra 1 phiếu từ 2 phiếu mà thầy giáo đã rút. Xác suất để sinh viên A trả lời được câu hỏi trong phiếu là $0,78$}
{Thầy giáo rút ngẫu nhiên ra 1 phiếu từ hộp thứ nhất bỏ vào hộp thứ hai, sau đó cho sinh viên A rút ngẫu nhiên ra 1 phiếu từ hộp thứ hai. Xác suất để sinh viên trả lời được câu hỏi trong phiếu là $0,73$}
{Thầy giáo rút ngẫu nhiên ra 2 phiếu từ hộp thứ nhất bỏ vào hộp thứ hai, sau đó cho sinh viên A rút ngẫu nhiên ra 2 phiếu từ hộp thứ hai, xác suất để sinh viên đó rút được hai câu thuộc là $0,62$}
{Thầy giáo rút ngẫu nhiên ra 1 phiếu từ hộp thứ nhất bỏ vào hộp thứ hai, sau đó cho sinh viên A rút ngẫu nhiên ra 2 phiếu từ hộp thứ hai, xác suất để sinh viên đó rút được hai câu thuộc là $0,83$}
\loigiai{
a) Đúng: Gọi $A_1$ là biến cố sinh viên rút ra được phiếu thuộc hộp 1.\\
$A_2$ là biến cố sinh viên rút ra được phiếu thuộc hộp 2.\\
$A$ là biến cố sinh viên rút ra $1$ câu thuộc, khi đó:\\
$A=\left(A_1\cap A\right)\cup\left(A_2\cap A\right)\Rightarrow P(A)=P\left(A_1\right).P\left(A|A_1\right)+P\left(A_2\right).P\left(A|A_2\right)$ .\\
Ta có: $P\left(A_1\right)=\dfrac{1}{2};P\left(A_2\right)=\dfrac{1}{2}\Rightarrow P\left(A|A_1\right)=\dfrac{C_{10}^1}{C_{15}^1}=\dfrac{2}{3};P\left(A|A_2\right)=\dfrac{C_8^1}{C_9^1}=\dfrac{8}{9}$
Vậy $P(A)=\dfrac{7}{9}\approx 0,78$ .\\
b) Sai: Gọi $B_1$ là biến cố thầy giáo rút 1 câu thuộc từ hộp 1 bỏ vào hộp 2, khi đó hộp 2 có 9 câu thuộc và 1 câu không thuộc.\\
Gọi $B_2$ là biến cố thầy giáo rút 1 câu không thuộc từ hộp 1 bỏ vào hộp 2, khi đó hộp 2 có 8 câu thuộc và 2 câu không thuộc.\\
Gọi $B$ là biến cố sinh viên rút ra $1$ câu thuộc, khi đó:\\
$B=\left(B_1\cap B\right)\cup\left(B_2\cap B\right)\Rightarrow P(B)=P\left(B_1\right).P\left(B|B_1\right)+P\left(B_2\right).P\left(B|B_2\right)$ .\\
Ta có: $P\left(B_1\right)=\dfrac{C_{10}^1}{C_{15}^1}=\dfrac{2}{3};P\left(B_2\right)=\dfrac{C_5^1}{C_{15}^1}=\dfrac{1}{3}\Rightarrow P\left(B|B_1\right)=\dfrac{C_9^1}{C_{10}^1}=\dfrac{9}{10};P\left(B|B_2\right)=\dfrac{C_8^1}{C_{10}^1}=\dfrac{4}{5}$
Vậy $P(B)\approx 0,94$ .\\
c) Sai: Gọi $C_1$ là biến cố thầy giáo rút 2 câu thuộc từ hộp 1 bỏ vào hộp 2, khi đó hộp 2 có 10 câu thuộc và 1 câu không thuộc.\\
Gọi $C_2$ là biến cố thầy giáo rút 1 câu thuộc và 1 câu không thuộc từ hộp 1 bỏ vào hộp 2, khi đó hộp 2 có 9 câu thuộc và 2 câu không thuộc.\\
Gọi $C_3$ là biến cố thầy giáo rút 2 câu không thuộc từ hộp 1 bỏ vào hộp 2, khi đó hộp 2 có 8 câu thuộc và 3 câu không thuộc.\\
Gọi $C$ là biến cố sinh viên rút ra $2$ câu thuộc, khi đó: $C=\left(C_1\cap C\right)\cup\left(C_2\cap C\right)\cup\left(C_3\cap C\right)$\\
$\Rightarrow P(C)=P\left(C_1\right).P\left(C|C_1\right)+P\left(C_2\right).P\left(C|C_2\right)+P\left(C_3\right).P\left(C|C_3\right)$\\
Ta có: $P\left(C_1\right)=\dfrac{C_{10}^2}{C_{15}^2}=\dfrac{3}{7};P\left(C_2\right)=\dfrac{C_5^1.C_{10}^1}{C_{15}^2}=\dfrac{10}{21};P\left(C_3\right)=\dfrac{C_5^2}{C_{15}^2}=\dfrac{2}{21}$\\
$P\left(C|C_1\right)=\dfrac{C_{10}^2}{C_{11}^2}=\dfrac{9}{11};P\left(C|C_2\right)=\dfrac{C_9^2}{C_{11}^2}=\dfrac{12}{35};P\left(C|C_3\right)=\dfrac{C_8^2}{C_{11}^2}=\dfrac{3}{55}$
Vậy $P(C)\approx 0,52$ .\\
d) Sai: Gọi $D_1$ là biến cố thầy giáo rút 1 câu thuộc từ hộp 1 bỏ vào hộp 2, khi đó hộp 2 có 9 câu thuộc và 1 câu không thuộc.\\
Gọi $D_2$ là biến cố thầy giáo rút 1 câu không thuộc từ hộp 1 bỏ vào hộp 2, khi đó hộp 2 có 8 câu thuộc và 2 câu không thuộc.\\
Gọi $D$ là biến cố sinh viên rút ra 2 câu thuộc, khi đó:\\
$D=\left(D_1\cap D\right)\cup\left(D_2\cap D\right)\Rightarrow P(D)=P\left(D_1\right).P\left(D|D_1\right)+P\left(D_2\right).P\left(D|D_2\right)$ .\\
Ta có: $P\left(D_1\right)=\dfrac{C_{10}^1}{C_{15}^1}=\dfrac{2}{3};P\left(D_2\right)=\dfrac{C_5^1}{C_{15}^1}=\dfrac{1}{3}$ $\Rightarrow P\left(D|D_1\right)=\dfrac{C_9^2}{C_{10}^2}=\dfrac{4}{5};P\left(D|D_2\right)=\dfrac{C_8^2}{C_{10}^2}=\dfrac{28}{45}$ .\\
Vậy $P(D)\approx 0,74$ }
\end{ex}
\Closesolutionfile{ans}
\TNSA
\Opensolutionfile{ans}[ans/ans-4-SA]
\begin{ex}%Câu 17
Cho hình lăng trụ đứng $ ABC.A'B'C'$ có đáy $ ABC$ là tam giác vuông tại $ B$ và có độ dài các cạnh$ AB=a\sqrt{3}$, $ BC=2a$, $ AA'=a\sqrt{2}$. Gọi $ M$ là trung điểm của $ BC$. Tính khoảng cách giữa hai đường thẳng $ AM$ và $B'C$ khi $ a=1$ (Kết quả làm tròn đến hàng phần trăm).
\shortans{0,55}
\loigiai{
{\color{red}HÌNH Ở ĐÂY}\\
Gọi $ N$ là trung điểm của $ B{B}'$ thì $ MN\text{//}{B}'C\Rightarrow{B}'C\text{//}\left(AMN\right)$.\\
Ta có: $ d\left(B'C,AM\right)=d\left(B'C,\left(AMN\right)\right)=d\left(B',\left(AMN\right)\right)=d\left(B,\left(AMN\right)\right)$.\\
Dựng $ BI\perp AM, BH\perp NI$$\Rightarrow BH\perp\left(AMN\right)$ nên do đó $ d\left(B'C,AM\right)=d\left(B,\left(AMN\right)\right)=BH$.\\
Vì $\Delta ABM$ vuông tại $ B$, ta có $BM=\dfrac{BC}{2}=a$ ; $ BI=\dfrac{BA.BM}{\sqrt{B{A^2}+B{M^2}}}=\dfrac{a\sqrt{3}.a}{\sqrt{3a^2+a^2}}=\dfrac{a\sqrt{3}}{2}$.\\
Xét $\Delta BIN$ vuông tại $ B$, ta có:\\
$ BN=\dfrac{B{B}'}{2}=\dfrac{a\sqrt{2}}{2}$; $ BH=\dfrac{BN.BI}{\sqrt{B{N^2}+B{I^2}}}=\dfrac{\dfrac{a\sqrt{2}}{2}.\dfrac{a\sqrt{3}}{2}}{\sqrt{\left(\dfrac{a\sqrt{2}}{2}\right)^2+\left(\dfrac{a\sqrt{3}}{2}\right)^2}}=\dfrac{a\sqrt{30}}{10}$.\\
Vậy $ d\left(B'C,AM\right)=BH=\dfrac{\sqrt{30}}{10}\xrightarrow{a=1}d\left(B'C,AM\right)\approx 0,55$.}
\end{ex}

\begin{ex}%Câu 18
Trong một cuộc thi về “bữa ăn dinh dưỡng”, ban tổ chức yêu cầu để đảm bảo lượng dinh dưỡng hằng ngày thì mỗi gia đình có $4$ thành viên cần ít nhất $900$ đơn vị prôtêin và $400$ đơn vị lipít trong thức ăn hằng ngày. Mỗi kg thịt bò chứa $800$ đơn vị prôtêin và $200$ đơn vị lipit, $1\text{(kg)}$ thịt heo chứa $600$ đơn vị prôtêin và $400$ đơn vị lipit. Biết rằng người nội trợ chỉ được chi tối đa $200$ ngàn đồng để mua thịt. Biết rằng $1\text{(kg)}$ thịt bò giá $200$ ngàn đồng, $1\text{(kg)}$ thịt heo giá $100$ ngàn đồng. Người nội trợ nên mua $x\text{(kg)}$ thịt bò và $y\text{(kg)}$ thịt heo để phí thấp nhất cho khẩu phần thức ăn mà vẫn đảm bảo chất dinh dưỡng. Khi đó hãy tìm $x+2y.$
\shortans{3}
\loigiai{
Chi phí mua thịt là $ F\left(x;y\right)=200x+100y$ (ngàn đồng).\\
Hệ điều kiện ràng buộc giữa $ x$ và $ y$ là $\left\{\begin{matrix}
x\ge 0\\
y\ge 0\\
800x+600y\ge 900\\
200x+400y\ge 400\\
\end{matrix}\Leftrightarrow\left\{\begin{matrix}
x\ge 0\\
y\ge 0\\
8x+6y\ge 9\\
x+2y\ge 2\\
\end{matrix}\right.\right.$.\\
Miền nghiệm của hệ bất phương trình này là miền không bị gạch sọc trong hình vẽ dưới đây\\
{\color{red}HÌNH Ở ĐÂY}\\
Các điểm cực biên là $ A\left(0;\,1,5\right),\,\,B\left(0,6;\,0,7\right),\,\,C\left(2;\,0\right)$.\\
Ta có: $ F\left(0;1,5\right)=150,\,\,\,F\left(0,6;0,7\right)=190,\,\,\,F\left(2;0\right)=400$.\\
Vậy chi phí mua thịt thấp nhất khi $ x=0,y=1,5\Rightarrow x+2y=3$.
}
\end{ex}
%
\begin{ex}%Câu 19
Đường đi của một khinh khí cầu được gắn trong hệ trục tọa độ là một đường cong bậc hai trên bậc nhất có đồ thị cắt trục hoành tại hai điểm có tọa độ là $\left(1;0\right)$ và $\left(8;0\right)$ với đơn vị trên hệ trục tọa độ là $ 1$ km. Biết rằng điểm cực đại của đồ thị hàm số là điểm $\left(6;5\right).$ Hỏi khi khí cầu đi qua điểm cực đại và cách mặt đất $3875$ M thì khí cầu cách gốc tọa độ theo phương ngang bao nhiêu? (đơn vị: km)\\
\centerline{\includegraphics[width=.3\textheight]{images/4.19}}
\shortans{7,2}
\loigiai{
Đồ thị hàm số bậc hai trên bậc nhất $ y=\dfrac{a{x^2}+bx+c}{x+m}$ cắt trục hoành tại hai điểm có hoành độ $ x=1,\,\,x=8$ nên ta có: $ y=\dfrac{a\left(x-1\right)\left(x-8\right)}{x+m}=\dfrac{a\left(x^2-9x+8\right)}{x+m}.$\\
Đường thẳng qua hai điểm cực trị của đồ thị hàm số là $(d):y=\dfrac{\left[a\left(x^2-9x+8\right)\right]'}{\left(x+m\right)'}=a\left(2x-9\right)$ qua điểm cực đại $\left(6;5\right)\Leftrightarrow 5=a\left(2.6-9\right)\Leftrightarrow a=\dfrac{5}{3}\Rightarrow y=\dfrac{5\left(x^2-9x+8\right)}{3\left(x+m\right)}$.\\
Ta có $y'(6)=0\Leftrightarrow m=-\dfrac{28}{3}\Rightarrow y=\dfrac{5\left(x^2-9x+8\right)}{3\left(x-\dfrac{28}{3}\right)}$.\\
Xét $ y=3,875\Leftrightarrow\dfrac{5\left(x^2-9x+8\right)}{3\left(x-\dfrac{28}{3}\right)}=3,875\Leftrightarrow\left[\begin{aligned}
& x\approx 4,125\\ 
& x\approx 7,2\\ 
\end{aligned}\right.$\\
Vậy khi khí cầu đi qua điểm cực đại và cách mặt đất $ 3875$(m) thì khí cầu cách gốc tọa độ theo phương ngang $ 7,2$ km.}
\end{ex}

\begin{ex}%Câu 20
Hệ thống lọc nước bể bơi vô cùng quan trọng khi tiến hành xây dựng công trình bơi lội để nguồn nước được làm sạch thường xuyên và giữ vệ sinh cho người bơi. Trong quá trình vận hành lọc nước thì lượng nước trong bể sẽ thay đổi theo thời gian. Lượng nước trong bể giảm nếu hệ thống đang xả nước bẩn ra khỏi bể và tăng nếu hệ thống đang cấp thêm nước sạch cho bể. Biết rằng $1$ gallon gần bằng $3,785$ lít, dung tích của bể là $ 1000$ gallon và thời điểm $6$ giờ sáng bể chứa $ 250$ gallon nước. Hàm số $ f(t)$ biểu thị cho tốc độ thay đổi lượng nước trong bể theo thời gian $ t$ giờ, từ thời điểm $6$ giờ sáng đến $6$ giờ chiều được cho bởi $f(t)=\left\{\begin{aligned}
& 100t, \text{ khi }0\le t\le 3\\ 
& 900-200t, \text{ khi }3\le t\le 6\\ 
& 100t-900, \text{ khi }6\le t\le 12\\ 
\end{aligned}\right.$ với mốc thời gian $ t=0$ tại thời điểm $6$ giờ sáng. Hỏi ở thời điểm $6$ giờ chiều thì trong bể chứa nhiêu gallon nước?
\shortans{700}
\loigiai{
Gọi $ F(t)$ là một nguyên hàm của hàm số $ f(t)$\\
Vì $ f(t)$ biểu thị cho tốc độ thay đổi lượng nước trong bể theo thời gian $ t$ nên $ F(t)$ chính là lượng nước có trong bể theo thời gian $ t$.\\
Lượng nước trong bể lúc 6 giờ sáng (ứng với $ t=0$) là $ F(0)=250$.\\
Lượng nước trong bể lúc 6 giờ chiều (ứng với $ t=12$) là $ F\left(12\right)=F(0)+\displaystyle\int\limits_0^{12}{f(t)\mathrm{\,d}t}$\\
$=250+\displaystyle\int\limits_0^3100t\mathrm{\,d}t+\displaystyle\int\limits_3^6\left(900-200t\right)\mathrm{\,d}t+\displaystyle\int\limits_6^{12}{\left(100t-900\right)\mathrm{\,d}t}=700$ (galon)\\
Đáp án:\\
7 0 0}
\end{ex}

\begin{ex}%Câu 21
Hệ thống định vị toàn cầu GPS (Global Positioning System) là một hệ thống cho phép xác định vị trí của một vật thể trong không gian. Trong cùng một thời điểm vị trí của một điểm $ M$ trong không gian sẽ được xác định bởi bốn vệ tinh cho trước nhờ các bộ thu phát tín hiệu đặt trên các vệ tinh đó. Giả sử trong không gian với hệ trục tọa độ $ Oxyz$, có bốn vệ tinh lần lượt đặt tại các điểm $ A\left(2;4;0\right),\,B\left(0;4;6\right),\,C\left(2;0;6\right),\,D\left(-1;-2;-3\right)$ và vị trí của điểm $ M\left(a;b;c\right)$ thỏa mãn biểu thức $ MA+MB+MC+MD$ nhỏ nhất. Tính độ dài $ MO$ (kết quả làm tròn đến hàng phần chục)
\shortans{3,7}
\loigiai{
Gọi $ G$ là trọng tâm của tứ diện $OABC$. Suy ra $G(1;2;3)$.\\
Từ đó suy ra $ GA=GB=GC=GO=\sqrt{14}$ và $O$ là trung điểm $GD$.
\begin{align*}
MA+MB+MC+MD&=\dfrac{MA \cdot GA + MB \cdot GB + MC \cdot GC + MD \cdot GD}{GA} \\
& \ge \dfrac{\vec{MA} \cdot \vec{GA}+\vec{MB} \cdot \vec{GB}+\vec{MC} \cdot \vec{GC}+\vec{MD} \cdot \vec{GO}}{GA} \\
&=\dfrac{\vec{MG}\left(\vec{GA}+\vec{GB}+\vec{GC}+\vec{GO}\right)+5GA^2}{GA}=5GA
\end{align*}
Dấu bằng xảy ra khi và chỉ khi $ M$ trùng với điểm $ G$.\\
Khi đó $MO=GO=\sqrt{14}\approx 3,7 $.
}
\end{ex}

\begin{ex}%Câu 22
Một hộp chứa $10$ viên bi xanh và $5$ viên bi đỏ. Bạn An lấy ra ngẫu nhiên $1$ viên bi từ hộp, xem màu, rồi bỏ ra ngoài. Nếu viên bi An lấy ra có màu xanh, bạn Bình sẽ lấy ra ngẫu nhiên $2$ viên bi từ hộp; còn nếu viên bi An lấy ra có màu đỏ, bạn Bình sẽ lấy ra ngẫu nhiên $3$ viên bi từ hộp. Tính xác suất để An lấy được viên bi màu xanh, biết rằng tất cả các viên bi được hai bạn chọn ra đều có đủ cả hai màu.
\shortans{0,55}
\loigiai{
Gọi $ A$ là biến cố An lấy được viên bi màu xanh thì $\overline{A}$ là biến cố An lấy được viên bi màu đỏ.
Gọi $ B$ là biến cố tất cả các viên bi được hai bạn chọn ra đều có đủ cả hai màu.\\
Ta có $ P(A)=\dfrac{10}{15},P\left(\overline{A}\right)=\dfrac{5}{15}$. Ta cần tính $ P\left(A\mid B\right)=\dfrac{P\left(AB\right)}{P(B)}$.\\
Khi $ A$ xảy ra thì trong hộp còn 9 viên bi xanh và 5 viên bi đỏ, Bình cần lấy ra 2 viên bi trong đó có ít nhất một viên bi đỏ nên $ P\left(B\mid A\right)=\dfrac{C_{14}^2-C_9^2}{C_{14}^2}$.\\
Khi $\overline{A}$ xảy ra thì trong hộp còn 10 viên bi xanh và 4 viên bi đỏ, Bình cần lấy ra 3 viên bi trong đó có ít nhất một viên bi xanh nên $ P\left(B\mid\overline{A}\right)=\dfrac{C_{14}^3-C_4^3}{C_{14}^3}$.\\
Vậy $ P\left(AB\right)=P(A).P\left(B\mid A\right)=\dfrac{10}{15}\cdot\dfrac{\left(C_{14}^2-C_9^2\right)}{C_{14}^2}=\dfrac{110}{273}$\\
Mặt khác: $ P(B)=P(A).P\left(B\mid A\right)+P\left(\overline{A}\right).P\left(B\mid\overline{A}\right)$ $=\dfrac{10}{15}\cdot\dfrac{\left(C_{14}^2-C_9^2\right)}{C_{14}^2}+\dfrac{5}{15}\cdot\dfrac{\left(C_{14}^3-C_4^3\right)}{C_{14}^3}=\dfrac{200}{273}$.\\
Vậy $ P\left(A\mid B\right)=\dfrac{\dfrac{110}{273}}{\dfrac{200}{273}}=0,55$.}
\end{ex}
\Closesolutionfile{ans}
\Closesolutionfile{ansbook}
\inputansbox{6,2,3}{ans/ans-4-T,ans/ans-4-TF,ans/ans-4-SA}
% \begin{name}
    {\tenchude}
    {\tendethi}
    {\tentruong}
    {\thoigian}
\end{name}
\Opensolutionfile{ansbook}[ans/ansbook-5]
\TN
\Opensolutionfile{ans}[ans/ans-5-T]
\begin{ex}%Câu 1
Họ nguyên hàm của hàm số $ f(x)=\dfrac{1}{2x+3}$ là
\choice
{$ 3\ln \left| 2x+3\right|+C$}
{$\dfrac{1}{3}\ln \left| 2x+3\right|+C$}
{$ 2\ln \left| 2x+3\right|+C$}
{\True $\dfrac{1}{2} \ln \left| 2x+3\right|+C$}
\loigiai{
Ta có $\displaystyle\int{f(x)}\mathrm{\,d}x=\dfrac{1}{2}\ln \left| 2x+3\right|+C$.
}
\end{ex}

\begin{ex}%Câu 2
Diện tích $ S$ của hình phẳng giới hạn bởi đồ thị hàm số $ y=f(x)$, liên tục trên $\left[a;b\right]$ trục hoành và hai đường thẳng $ x=a,\,x=b$ ($ a<b$) cho bởi công thức:
\choice
{\True $ S=\displaystyle\int\limits_a^b{\left| f(x)\right|dx}$}
{$ S=\displaystyle\int\limits_a^b{f(x)dx}$}
{$ S=\pi\displaystyle\int\limits_a^b{\left| f(x)\right|dx}$}
{$ S=\pi\displaystyle\int\limits_a^b{f^2(x)dx}$}
\loigiai{
Diện tích $ S$ của hình phẳng là: $ S=\displaystyle\int\limits_a^b{\left| f(x)\right|dx}$}
\end{ex}

\begin{ex}%Câu 3
Sau khi kiểm tra sức khỏe tổng quát, kết quả số cân nặng của học sinh lớp 12A sĩ số 40 HS được thể hiện trong bảng số liệu sau: (Đơn vị: kg)
\begin{center}
\begin{tabular}{|c|c|c|c|c|c|}
    \hline
    Cân nặng & $[40; 50)$ & $[50; 60)$ & $[60; 70)$ & $[70; 80)$ & $[80; 90)$ \\
    \hline
    Số học sinh & 7 & 12 & 12 & 7 & 2 \\
    \hline
\end{tabular}
\end{center}
Tứ phân vị thứ nhất của mẫu số liệu trên gần nhất với giá trị nào trong các giá trị sau?
\choice
{$ 50$}
{$ 50,5$}
{\True $ 52,5$}
{$ 55,5$}
\loigiai{
Tứ phân vị thứ nhất của dãy số liệu thuộc nhóm $\left[50;60\right)$ nên tứ phân vị thứ nhất của mẫu số liệu là $Q_1=50+\dfrac{\dfrac{40}{4}-7}{12}\left(60-50\right)=52,5$}
\end{ex}

\begin{ex}%Câu 4
Trong không gian $Oxyz$ , phương trình của đường thẳng đi qua điểm $ A\left(1;2;\,-1\right)$ và có vectơ chỉ phương $\vec{u}=\left(1;\,3;\,2\right)\,$ là
\choice
{$\dfrac{x+1}{1}\,=\,\dfrac{y\,+\,3}{2}\,=\,\dfrac{z\,+\,2}{-1}$}
{$\dfrac{x-1}{1}\,=\,\dfrac{y\,-\,3}{2}\,=\,\dfrac{z\,-\,2}{-1}$}
{$\dfrac{x+1}{1}\,=\,\dfrac{y\,+\,2}{3}\,=\,\dfrac{z\,-\,1}{2}$}
{\True $\dfrac{x-1}{1}\,=\,\dfrac{y\,-\,2}{3}\,=\,\dfrac{z\,+\,1}{2}$}
\loigiai{
Đường thẳng đi qua điểm $ A\left(1;2;\,-1\right)$ và có vectơ chỉ phương $\vec{u}=\left(1;\,3;\,2\right)\,$ có phương trình là: $\dfrac{x-1}{1}\,=\,\dfrac{y\,-\,2}{3}\,=\,\dfrac{z\,+\,1}{2}$.}
\end{ex}

\begin{ex}%Câu 5
Hàm số số $ y=\dfrac{ax+b}{cx+d}$ có đồ thị như hình bên dưới:\\
\centerline{
    \begin{tikzpicture}[line join=round, line cap=round,>=stealth,thick]
\tikzset{every node/.style={scale=0.9}}
\draw[->] (-3.1,0)--(3.1,0) node[below left] {$x$};
\draw[->] (0,-2.1)--(0,4.1) node[below left] {$y$};
\draw (0,0) node [below left] {$O$}
(-1,0)node[below left]{$-1$}
(0,2)node[above right]{$2$};
\draw[dashed,thin] (-0.99,-2)--(-0.99,4);
\begin{scope}
\clip (-3,-2) rectangle (3,4);
\draw[samples=200,domain=-4:-1.01,smooth,variable=\x] plot (\x,{(2*(\x)+1)/(1*(\x)+1)});
\draw[samples=200,domain=-0.99:4,smooth,variable=\x] plot (\x,{(2*(\x)+1)/(1*(\x)+1)});
\draw[dashed,thin] (-4,2/1)--(4,2/1);
\end{scope}
\end{tikzpicture}
}
Đường tiệm cận đứng của đồ thị là đường thẳng có phương trình
\choice
{$ x=1$}
{$ x=2$}
{$ x=-2$}
{\True $ x=-1$}
\loigiai{
Dựa vào đồ thị ta có đường tiệm cận đứng của đồ thị là đường thẳng $ x=-1$.}
\end{ex}

\begin{ex}%Câu 6
Tập nghiệm của bất phương trình $\log_{\dfrac{1}{2}}\left(x-2\right)\le 1$ là:
\choice
{\True $\left[\dfrac{5}{2};\,\,+\infty\right)$}
{$\left(\dfrac{5}{2};\,\,+\infty\right)$}
{$\left(-\infty ;{\log_2}5\right)$}
{$\left(-\infty ;\dfrac{5}{2}\right)$}
\loigiai{
Điều kiện: $ x-2>0\Leftrightarrow x>2$.\\
Bất phương trình: $\log_{\dfrac{1}{2}}\left(x-2\right)\le 1\Leftrightarrow x-2\ge\dfrac{1}{2}\Leftrightarrow x\ge\dfrac{5}{2}$.\\
Kết hợp với điều kiện ta có tập nghiệm $ T=\left[\dfrac{5}{2};\,\,+\infty\right)$.}
\end{ex}

\begin{ex}%Câu 7
Trong không gian $Oxyz$ , mặt phẳng nào sau đây nhận $\overrightarrow{n}=\left(1;2;3\right)$ làm vectơ pháp tuyến?
\choice
{$x+2y+3=0$}
{\True $x+2y+3z=0$}
{$y+2z+3=0$}
{$x+2z+3=0$}
\loigiai{
Mặt phẳng $x+2y+3z=0$ có vectơ pháp tuyến là $\overrightarrow{n}=\left(1;2;3\right)$ .}
\end{ex}

\begin{ex}%Câu 8
Cho hình chóp $ S.ABCD$ có đáy $ ABCD$ là hình chữ nhật tâm $ I$ và cạnh bên $ SA$ vuông góc với đáy. Khẳng định nào sau đây đúng?
\choice
{\True $\left(SCD\right)\perp\left(SAD\right)$}
{$\left(SBC\right)\perp\left(SIA\right)$}
{$\left(SDC\right)\perp\left(SAI\right)$}
{$\left(SBD\right)\perp\left(SAC\right)$}
\loigiai{
% {\color{red}HÌNH Ở ĐÂY}\\
Ta có: $CD\perp AD$ (Vì $ ABCD$ là hình chữ nhật) và $ SA\perp\left(ABCD\right)\Rightarrow SA\perp CD$\\
Mặt khác: $ SA\cap AD=A$ và $ SA,AD\subset\left(SAD\right)$$\Rightarrow CD\perp\left(SAD\right)$\\
Mà $CD\subset\left(SCD\right)$ nên $\left(SCD\right)\perp\left(SAD\right)$ .}
\end{ex}

\begin{ex}%Câu 9
Nghiệm của phương trình $\left(\dfrac{1}{5}\right)^{x^2-2x-3}=5^{x+1}$ là
\choice
{\True $ x=-1;\,x=2$}
{Vô nghiệm}
{$ x=1;\,x=2$}
{$ x=1;\,x=-2$}
\loigiai{
Phương trình đã cho tương đương $5^{-x^2+2x+3}=5^{x+1}\Leftrightarrow-x^2+x+2=0\Leftrightarrow\left[\begin{aligned}
& x=-1\\ 
& x=2.\\ 
\end{aligned}\right.$\\
Vậy phương trình có nghiệm $ x=-1;\,x=2$.}
\end{ex}

\begin{ex}%Câu 10
Cho cấp số cộng $\left(u_n\right)$ có $u_1=2$, $u_2=6$. Công sai của cấp số cộng bằng
\choice
{$8$}
{$-4$}
{3}
{\True 4}
\loigiai{
Vì $u_2=6\Leftrightarrow{u_1}+d=6\Leftrightarrow 2+d=6\Leftrightarrow d=4$ .}
\end{ex}

\begin{ex}%Câu 11
Cho hình lăng trụ tam giác $ ABC.A'B'C'$. Đặt $\overrightarrow{AA'}=\overrightarrow{a},\overrightarrow{AB}=\overrightarrow{b},\overrightarrow{AC}=\overrightarrow{c}$. Khi đó biểu diễn $\overrightarrow{BC'}$ theo các vectơ $\overrightarrow{a},\overrightarrow{b},\overrightarrow{c}$\\
\centerline{
\begin{tikzpicture}[scale=1, font=\footnotesize, line join=round, line cap=round, >=stealth]
    \def\ac{4} % cạnh AC
    \def\ab{2} % cạnh AB
    \def\h{4} % chiều cao
    \def\gocA{50} % góc A của đáy
    \path
    (0,0) coordinate (A)
    (\ac,0) coordinate (C)
    (-\gocA:\ab) coordinate (B)
    ($(A)+(80:\h)$) coordinate (A')
    ($(B)-(A)+(A')$) coordinate (B')
    ($(C)-(A)+(A')$) coordinate (C');
    \draw (A')--(A)--(B)--(C)--(C')--(A')--(B')--(C') (B)--(B');
    \draw[dashed] (A)--(C);
    \foreach \x/\g in {A/180,B/-90,C/0,A'/180,C'/0,B'/-40}\fill[red] (\x) circle (1pt)+(\g:3mm) node[black]{$ \x $};
\end{tikzpicture}
}
\choice
{$\overrightarrow{BC'}=-\overrightarrow{a}+\overrightarrow{b}+\overrightarrow{c}$}
{\True $\overrightarrow{BC'}=\overrightarrow{a}-\overrightarrow{b}+\overrightarrow{c}$}
{$\overrightarrow{BC'}=\overrightarrow{a}+\overrightarrow{b}+\overrightarrow{c}$}
{$\overrightarrow{BC'}=\overrightarrow{a}+\overrightarrow{b}-\overrightarrow{c}$}
\loigiai{
Do $ ABC.A'B'C'$ là hình lăng trụ nên $\overrightarrow{A'C'}=\overrightarrow{AC}$ nên ta có:\\
$\overrightarrow{BC'}=\overrightarrow{AC'}-\overrightarrow{AB}=\overrightarrow{AA'}+\overrightarrow{A'C'}-\overrightarrow{AB}=\overrightarrow{a}-\overrightarrow{b}+\overrightarrow{c}$}
\end{ex}

\begin{ex}%Câu 12
Cho hàm số $ y=f(x)$ có bảng biến thiên như sau :\\
\centerline{
    \begin{tikzpicture}
\tkzTabInit[espcl=2.5,lgt=1.5,nocadre=false]
{$x$/0.7,$f'(x)$/0.7,$f(x)$/2.1}
{$-\infty$,$0$,$2$,$+\infty$}
\tkzTabLine{,+,0,-,0,+,}
\tkzTabVar{-/$-\infty$,+/$4$,-/$0$,+/$+\infty$}
\end{tikzpicture}
}
Hàm số đã cho đồng biến trên khoảng nào dưới đây?
\choice
{$\left(-4\,;\,1\right)$}
{$\left(0\,;+\infty\right)$}
{\True $\left(-\infty ;\,0\,\right)$}
{$\left(0\,;2\right)$}
\loigiai{
Hàm số đã cho đồng biến trên khoảng $\left(-\infty ;\,0\,\right)$.}
\end{ex}
\Closesolutionfile{ans}
\TNTF
\Opensolutionfile{ans}[ans/ans-5-TF]

\begin{ex}%Câu 13
Cho hàm số $f(x)=\cos 2x+2x+1$ . 
\choiceTF
{\True $ f\left(\dfrac{\pi}{2}\right)=\pi $}
{Đạo hàm của hàm số đã cho là $f'(x)=2\sin 2x+2$}
{\True Nghiệm của phương trình $f'(x)=0$ trên đoạn $\left[-\dfrac{\pi}{2};\pi\right]$ là $ x=\dfrac{\pi}{4}$}
{Tổng giá trị lớn nhất và giá trị nhỏ nhất của hàm số trên đoạn $\left[-\dfrac{\pi}{2};\pi\right]$ bằng $ 2\pi $}
\loigiai{

a) Đúng: $ f\left(\dfrac{\pi}{2}\right)=\cos\left(2.\dfrac{\pi}{2}\right)+2.\dfrac{\pi}{2}+1=\pi $\\
b) Sai: Đạo hàm $f'(x)=-2\sin 2x+2$\\
c) Đúng: Phương trình $f'(x)=0\Leftrightarrow-2\sin 2x+2=0\Leftrightarrow\sin 2x=1\Leftrightarrow x=\dfrac{\pi}{4}+k\pi ,\,k\in\mathbb{Z}$\\
Vì $x\in\left[-\dfrac{\pi}{2};\pi\right]$ nên $x=\dfrac{\pi}{4}$\\
d) Sai: Ta có $f\left(-\dfrac{\pi}{2}\right)=-\pi $ ; $f\left(\dfrac{\pi}{4}\right)=\dfrac{\pi}{2}+1$ ; $f\left(\pi\right)=2\pi+2$ .\\
Vậy $\underset{\left[-\dfrac{\pi}{2};\,\pi\right]}{\max}\,f(x)=f\left(-\dfrac{\pi}{2}\right)=-\pi $ và $\underset{\left[-\dfrac{\pi}{2};\pi\right]}{\min}\,f(x)=f\left(\pi\right)=2\pi+2$\\
Khi đó tổng giá trị lớn nhất và nhỏ nhất của hàm số trên đoạn $\left[-\dfrac{\pi}{2};\pi\right]$ bằng $\pi+2$.}
\end{ex}

\begin{ex}%Câu 14
Các nhà kinh tế sử dụng đường cong Lorenz để minh họa sự phân phối thu nhập trong một quốc gia. Gọi $x$ là đại diện cho phần trăm số gia đình trong một quốc gia và $y$ là phần trăm tổng thu nhập, mô hình $y=x$ sẽ đại diện cho một quốc gia mà các gia đình có thu nhập như nhau. Đường cong Lorenz $y=f(x)$ , biểu thị sự phân phối thu nhập thực tế. Diện tích giữa hai mô hình này, với $0\le x\le 100$ , biểu thị “sự bất bình đẳng về thu nhập” của một quốc gia. Năm $2005$ , đường cong Lorenz của Hoa Kỳ có thể được mô hình hóa bởi hàm số:
$$y=\left(0,00061x^2+0,0218x+1,723\right)^2,0\le x\le 100$$
Trong đó $x$ được tính từ các gia đình nghèo nhất đến giàu có nhất\\
\textit{(Theo R.Larson, Brief Calculus: An Applied Approach, 8th edition, Cengage Learning, 2009)}
\choiceTF
{\True Tính theo thứ tự từ các gia đình nghèo nhất đến giàu nhất, tổng thu nhập thực tế của $60\%$ các gia đình đầu tiên chiếm chưa đến $30\%$ so với tổng thu nhập của toàn bộ các gia đình}
{Nếu sắp xếp các gia đình theo thứ tự từ nghèo nhất đến giàu nhất, rồi chia thành $10$ nhóm bằng nhau từ $1$ đến $10$ , tổng thu nhập của các gia đình trong nhóm $3$ chiếm khoảng $8,56\%$ tổng thu nhập của toàn bộ các gia đình}
{Sự bất bình đẳng về thu nhập của Hoa Kì năm $2005$ được xác định bởi công thức:\\
$\displaystyle\int\limits_0^{100}{\left[x-\left(0,00061x^2+0,0218x+1,723\right)^2\right]\mathrm{\,d}x}$}
{\True Sự bất bình đẳng về thu nhập của Hoa Kỳ năm $2005$ đã vượt quá $2000$}
\loigiai{

a) Đúng: Tính theo thứ tự từ các gia đình nghèo nhất đến giàu nhất, tổng thu nhập của $60\%$ các gia đình của đầu tiên chiếm tỷ lệ trong tổng thu nhập là: $f\left(60\right)=27,321529(\%)$ .\\
b) Sai: Nếu sắp xếp các gia đình theo thứ tự từ nghèo nhất đến giàu nhất, rồi chia thành $10$ nhóm bằng nhau từ $1$ đến $10$ , tổng thu nhập của các gia đình trong nhóm $3$ chiếm khoảng $8,56\%$ tổng thu nhập của toàn bộ các gia đình.\\
Nếu sắp xếp các gia đình theo thứ tự từ nghèo đến giàu, rồi chia thành $10$ nhóm bằng nhau, mỗi nhóm chiếm $10\%$ số gia đình của Hoa Kỳ.\\
Tổng thu nhập của $30\%$ số gia đình (là các gia đình thuộc nhóm $1,2,3$) chiếm tỷ lệ trong tổng thu nhập của tất cả các gia đình là: $f\left(30\right)=8,561476\,\,\left(\%\right)$ .\\
Tổng thu nhập của $20\%$ số gia đình (là các gia đình thuộc nhóm $1,2$) chiếm tỷ lệ trong tổng thu nhập của tất cả các gia đình là: $f\left(20\right)=5,774409\,\,\,\left(\%\right)$ .\\
Tỷ lệ của tổng thu nhập các gia đình nhóm thứ $3$ so với toàn bộ các gia đình là:\\
$f\left(30\right)-f\left(20\right)=2,787067(\%)$ .\\
c) Sai: Sự bất bình đẳng về thu nhập của Hoa Kì năm $2005$ được xác định bởi công thức:\\
$\displaystyle\int\limits_0^{100}{\left[x-\left(0,00061x^2+0,0218x+1,723\right)^2\right]\mathrm{\,d}x}$ .\\
Sự bất bình đẳng về thu nhập của Hoa Kì vào năm $2005$ là diện tích hình phẳng $S$ giới hạn bởi hai đồ thị:\\
$\left\{\begin{aligned}
& y=x\\ 
& y=\left(0,00061x^2+0,0218+1,723\right)^2\\ 
& x=0;x=100\\ 
\end{aligned}\right.$ $\Rightarrow S=\displaystyle\int\limits_0^{100}{\left|\left(0,00061x^2+0,0218x+1,723\right)^2-x\right|\mathrm{\,d}x}$ .\\
Cách 1:\\
Sử dụng máy tính cầm tay, ta thấy phương trình $\left(0,00061x^2+0,0218x+1,723\right)^2-x=0$ có hai ngiệm $x=a\,;x=b\,\,\left(a<b\right)$ thuộc $\left[0\,;100\right]$ .\\
Xét dấu biểu thức $g(x)=\left(0,00061x^2+0,0218x+1,723\right)^2-x$ ta suy ra:\\ $S=\displaystyle\int\limits_0^{100}{\left| g(x)\right|\mathrm{\,d}x}=\displaystyle\int\limits_0^a{\left| g(x)\right|\mathrm{\,d}x}+\displaystyle\int\limits_a^b{\left| g(x)\right|\mathrm{\,d}x}+\displaystyle\int\limits_b^{100}{\left| g(x)\right|\mathrm{\,d}x}$ .\\
$=\left|\displaystyle\int\limits_0^a{g(x)\mathrm{\,d}x}\right|+\left|\displaystyle\int\limits_a^b{g(x)\mathrm{\,d}x}\right|+\left|\displaystyle\int\limits_b^{100}{g(x)\mathrm{\,d}x}\right|$ $=\displaystyle\int\limits_0^a{g(x)\mathrm{\,d}x}-\displaystyle\int\limits_a^b{g(x)\mathrm{\,d}x}+\displaystyle\int\limits_b^{100}{g(x)\mathrm{\,d}x}$ .\\
Cách 2:\\
Sử dụng máy tính cầm tay ta được: $S=\displaystyle\int\limits_0^{100}{\left|\left(0,00061x^2+0,0218x+1,723\right)^2-x\right|\mathrm{\,d}x}\approx 2068,9$ .\\
Kiểm tra phép tính của đề bài, ta có: $\displaystyle\int\limits_0^{100}{\left[x-\left(0,00061x^2+0,0218x+1,723\right)^2\right]\mathrm{\,d}x}=2059,3131$ .\\
d) Đúng: Sự bất bình đẳng về thu nhập của Hoa Kỳ năm $2005$ đã vượt quá $2000$ .\\
Sự bất bình đẳng thu nhập của Hoa Kỳ năm $2005$ là:\\
$S=\displaystyle\int\limits_0^{100}{\left|\left(0,00061x^2+0,0218x+1,723\right)^2-x\right|\mathrm{\,d}x}\approx 2068,9$ .}
\end{ex}

\begin{ex}%Câu 15
Một công ty tham gia đấu thầu 2 dự án. Khả năng thắng thầu dự án 1 là 60\% và dự án 2 là 50\%. Khả năng thắng thầu cả hai dự án là 40\%. Gọi $ A$ và $ B$ lần lượt là biến cố công ty thắng thầu dự án 1 và dự án 2.
\choiceTF
{$ A$ và $ B$ là hai biến cố độc lập}
{\True Khả năng công ty thắng thầu đúng 1 dự án là 30\%}
{Xác suất công ty thắng thầu dự án 2 biết công ty đã thắng thầu dự án 1 là $\dfrac{1}{2}$}
{Xác suất công ty không thắng thầu dự án 2, biết công ty đã không thắng thầu dự án 1 là $\dfrac{1}{4}$}
\loigiai{
Ta có $P(A)=0,6\Rightarrow P\left(\bar{A}\right)=0,4;\quad P(B)=0,5\Rightarrow P\left(\bar{B}\right)=0,5$ và $P\left(AB\right)=0,4$ 

a) Sai: Vì $P\left(AB\right)\ne P(A).P(B)$ nên $A,B$ không độc lập .\\
b) Đúng: Do $ A\overline{B}$ và $\overline{A}B$ là hai biến cố xung khắc nên xác suất công ty thắng thầu đúng 1 dự án là: $P\left(A\bar{B}\right)+P\left(\bar{A}B\right)=P(A)-P\left(AB\right)+P(B)-P\left(AB\right)=0,6-0,4+0,5-0,4=0,3$\\
c) Sai: Xác suất công ty thắng thầu dự án 2 biết công ty đã thắng thầu dự án 1 là:\\
$P\left(B|A\right)=\dfrac{P\left(BA\right)}{P(A)}=\dfrac{0,4}{0,6}=\dfrac{2}{3}$ .\\
d) Sai: Xác suất công ty thắng thầu dự án 2, biết công ty đã không thắng thầu dự án 1 là:\\
$P\left(B|\bar{A}\right)=\dfrac{P\left(B\bar{A}\right)}{P\left(\bar{A}\right)}=\dfrac{P(B)-P\left(AB\right)}{P\left(\bar{A}\right)}=\dfrac{0,5-0,4}{0,4}=\dfrac{1}{4}$ .\\
Vậy xác suất công ty không thắng thầu dự án 2, biết công ty đã không thắng thầu dự án 1 là\\
$ P\left(\overline{B}\left|\overline{A}\right.\right)=1-P\left(B\left|\overline{A}\right.\right)=1-\dfrac{1}{4}=\dfrac{3}{4}$.}
\end{ex}

\begin{ex}%Câu 16
Trong không gian $Oxyz$ (đơn vị trên mỗi trục tính theo kilômét), một trạm thu phát sóng điện thoại di động được đặt ở vị trí $I\left(1;\,3;\,7\right)$ . Trạm thu phát sóng đó được thiết kế với bán kính phủ sóng là 3 km.\\
\centerline{\includegraphics[width=.4\textwidth]{images/5.16}}
\choiceTF
{Phương trình mặt cầu $(S)$ để mô tả ranh giới bên ngoài của vùng phù sóng trong không gian là $\left(x+1\right)^2+\left(y+3\right)^2+\left(z+7\right)^2=9$}
{\True Nếu người dùng điện thoại ở vị trí điểm $A\left(2;\,2;\,7\right)$ thì có thể sử dụng dịch vụ của trạm thu phát sóng đó}
{\True Nếu người dùng điện thoại ở vị trí có toạ độ $B\left(5;\,6;\,7\right)$ thì không thể sử dụng dịch vụ của trạm thu phát sóng đó}
{\True Tính theo đường chim bay, khoảng cách lớn nhất để một người ở vị trí có toạ độ $B\left(5;\,6;\,7\right)$ di chuyển được tới vùng phủ sóng theo đơn vị km là $8$ km}
\loigiai{
a) Sai: Phương trình mặt cầu $(S)$ tâm $I\left(1;\,3;\,7\right)$ bán kính 3km mô tả ranh giới bên ngoài của vùng phủ sóng trong không gian là $\left(x-1\right)^2+\left(y-3\right)^2+\left(z-7\right)^2=9$ .\\
b) Đúng: Ta có: $IA=\sqrt{\left(2-1\right)^2+\left(2-3\right)^2+\left(7-7\right)^2}=\sqrt{2}<3$ nên điểm $A$ nằm trong mặt cầu. Vì điểm $A$ nằm trong mặt cầu nên người dùng điện thoại ở vị trí có toạ độ $A\left(2;\,2;\,7\right)$ có thể sử dụng dịch vụ của trạm thu phát sóng đó.\\
c) Đúng: Ta có: $IB=\sqrt{\left(5-1\right)^2+\left(6-3\right)^2+\left(7-7\right)^2}=5>3$ nên điểm $B$ nằm ngoài mặt cầu. Vậy người dùng điện thoại ở vị trí có toạ độ $B\left(5;\,6;\,7\right)$ không thể sử dụng dịch vụ của trạm thu phát sóng đó.\\
d) Đúng: Ta có: $\overrightarrow{IB}\left(4;\,3;\,0\right);$ $IB=\sqrt{\left(5-1\right)^2+\left(6-3\right)^2+\left(7-7\right)^2}=5>3$ nên điểm $B$ nằm ngoài mặt cầu.\\
Phương trình đường thẳng $BI$ dạng: $\left\{\begin{aligned}
& x=1+4t\\ 
& y=3+3t\\ 
& z=7\\ 
\end{aligned}\right.$ .\\
Gọi mặt cầu $(S)\cap BI\equiv E$ suy ra tọa độ $E$ là nghiệm của hệ\\
$\left\{\begin{aligned}
& x=1+4t\\ 
& y=3+3t\\ 
& z=7\\ 
&{\left(x-1\right)^2}+\left(y-3\right)^2+\left(z-7\right)^2=9\\ 
\end{aligned}\right.\Leftrightarrow\left[\begin{aligned}
&\left\{\begin{aligned}
& t=\dfrac{3}{5}\\ 
& x=\dfrac{17}{5}\\ 
& y=\dfrac{24}{5}\\ 
& z=7\\ 
\end{aligned}\right.\Rightarrow E\left(\dfrac{17}{5};\,\dfrac{24}{5};7\right)\Rightarrow EB\approx 1,7\\ 
&\left\{\begin{aligned}
& t=-\dfrac{3}{5}\\ 
& x=-\dfrac{7}{5}\\ 
& y=\dfrac{6}{5}\\ 
& z=7\\ 
\end{aligned}\right.\Rightarrow E\left(-\dfrac{7}{5};\,\dfrac{6}{5};7\right)\Rightarrow EB=8\\ 
\end{aligned}\right.$\\
Vậy khoảng cách lớn nhất để một người ở vị trí có toạ độ $B\left(5;\,6;\,7\right)$ di chuyển được tới vùng phủ sóng theo đơn vị kilomet là $8$ km.}
\end{ex}
\Closesolutionfile{ans}
\TNSA
\Opensolutionfile{ans}[ans/ans-5-SA]
\begin{ex}%Câu 17
Cho hình hộp chữ nhật $ ABCD.A'B'C'D'$ có $ ABCD$ là hình vuông cạnh $ 2$ và góc giữa hai mặt phẳng $\left(AC'D'\right)$ và $\left(ABCD\right)$ bằng $30^{\text{o}}$. Tính khoảng cách giữa hai đường thẳng $ AD'$ và $A'B$. (Kết quả làm tròn đến chữ số thập phân thứ hai sau dấu phẩy)
\shortans{0,89}
\loigiai{
% {\color{red}HÌNH Ở ĐÂY}\\
Vì $\left(ABCD\right)\parallel\left(A'B'C'D'\right)$ nên góc giữa $\left(AC'D'\right)$ và mặt phẳng $\left(ABCD\right)$ bằng góc giữa $\left(AC'D'\right)$ và mặt phẳng $\left(A'B'C'D'\right)$.\\
Ta có $\heva{&\left(AC'D'\right)\cap\left(A'B'C'D'\right)=C'D'\\ &AD'\bot C'D' \\& A'D'\bot C'D'}$$\Rightarrow\left(\left(AC'D'\right);\left(A'B'C'D'\right)\right)=\left(AD';\,A'D'\right)=\widehat{AD'A'}=30^\circ $.\\
Trong tam giác vuông $ AA'D'$ có $ AA'=A'D'.\tan 30^\circ=\dfrac{2\sqrt{3}}{3}$.\\
Vì $A'B\parallel CD'$ nên $A'B\parallel\left(ACD'\right)$ suy ra $ d\left(A'B;\,AD'\right)=d\left(A'B;\,\left(ACD'\right)\right)=d\left(A';\,\left(ACD'\right)\right)$\\
Mặt khác $A'B\cap\left(ACD'\right)=I$ là trung điểm của $A'D$ nên $ d\left(A';\,\left(ACD'\right)\right)=d\left(D;\,\left(ACD'\right)\right)$.\\
Tại $ D$ ta có ba tia $ DA,\,DC,\,DD'$ đôi một vuông góc nên:\\
$\dfrac{1}{d^2\left(D;\,\left(ACD'\right)\right)}=\dfrac{1}{D{A^2}}+\dfrac{1}{D{C^2}}+\dfrac{1}{D{D'^2}}$$=\dfrac{1}{4}+\dfrac{1}{4}+\dfrac{3}{4}=\dfrac{5}{4}$\\
Suy ra $ d\left(D;\,\left(ACD'\right)\right)=\dfrac{2\sqrt{5}}{5}$ hay $ d\left(A'B;\,AD'\right)=\dfrac{2\sqrt{5}}{5}\approx 0,89$.}
\end{ex}

\begin{ex}%Câu 18
Công ty sản xuất "SmartTech" đang lên kế hoạch sản xuất hai loại sản phẩm A và B. Mỗi sản phẩm A đem lại lợi nhuận 800.000 đồng và mỗi sản phẩm B đem lại lợi nhuận 500.000 đồng. Tuy nhiên, việc sản xuất mỗi sản phẩm đòi hỏi nguyên vật liệu và công nhân khác nhau:\\
• Để sản xuất một sản phẩm A công ty cần sử dụng 2 kg nguyên vật liệu và 3 giờ lao động.\\
• Để sản xuất một sản phẩm B công ty cần sử dụng 1 kg nguyên vật liệu và 4 giờ lao động.\\
Hiện tại, công ty có sẵn 100 kg nguyên vật liệu và có thể sử dụng tối đa 180 giờ lao động. Công ty cần sản xuất $m$ sản phẩm A và $ n$ sản phẩm B để tối đa hóa lợi nhuận, đồng thời thỏa mãn các điều kiện về nguyên vật liệu và giờ lao động. Khi đó tổng $m+n$ bẳng bao nhiêu?
\shortans{56}
\loigiai{
Gọi $x$ là số lượng sản phẩm A và $y$ là số lượng sản phẩm B (Điều kiện: $x\ge 0;y\ge 0$)\\
Điều kiện về nguyên vật liệu: $2x+y\le 100$ .\\
Điều kiện về giờ lao động: $3x+4y\le 180$ .\\
Ta có hệ bất phương trình: $\left\{\begin{aligned}
& x\ge 0\\ 
& y\ge 0\\ 
& 2x+y\le 100\\ 
& 3x+4y\le 180\\ 
\end{aligned}\right.$\\
Miền nghiệm của hệ bất phương trình là miền không bị gạch sọc như hình vẽ dưới đây:\\
{\color{red}HÌNH Ở ĐÂY}\\
Ta có miền nghiệm của hệ bất phương trình là miền trong của tứ giác $OEFG$\\
Trong đó, $E\left(0;45\right),F(44;12),G\left(50;0\right)$ .\\
Tối ưu hóa lợi nhuận: Ta sẽ tính toán lợi nhuận $P=800.000x+500.000y$ tại các đỉnh của miền nghiệm ta được:\\
Tại $E\left(0;45\right)$ : $P=22.500.000$ ; $F\left(44;12\right)$ : $P=41.200.000$ ; $G\left(50;0\right)$ : $P=40.000.000$\\
Vậy để tối đa hóa lợi nhuận thì nhà máy cần sản xuất 44 sản phẩm A và 12 sản phẩm B.\\
Khi đó tổng số là $44+12=56$ sản phẩm hay $ m+n=56$.}
\end{ex}

\begin{ex}%Câu 19
Hệ thống định vị toàn cầu GPS (Global Positioning System) là một hệ thống cho phép xác định vị trí của một vật thể trong không gian. Trong cùng một thời điểm vị trí của một điểm $ M$ trong không gian sẽ được xác định bởi bốn vệ tinh cho trước nhờ các bộ thu phát tín hiệu đặt trên các vệ tinh. Giả sử trong không gian $ Oxyz$, tỉ lệ dài trên các trục là $10$ km tính cho một đơn vị tỉ lệ trên mỗi trục, cho bốn vệ tinh có tọa độ lần lượt là $ A\left(-1;2;-1\right),B\left(1;4;0\right),C\left(3;0;9\right),D\left(7;10;-1\right)$ đang tiến hành theo dõi vật thể $ M$ (coi là một chất điểm). Các vệ tinh dùng sóng điện từ có tần số 1MHz . Ở một thời điểm cả 4 vệ tinh bắn tín hiệu về $ M$ thì sau nhận được tín hiệu trả về ngay sau đó trong những khoảng thời gian là $t_A=0,4$ m/s, $t_B=0,2 $m/s, $ t_c=\dfrac{2}{3}$ m/s, $t_D=0,4$ m/s. Tính khoảng cách từ $ M$ đến $ O$ (biết rằng vận tốc của sóng điện từ bằng vận tốc ánh sáng $ c=3.10^8$ m/s và kết quả tính được làm tròn đến hàng phần trăm).
\shortans{6,78}
\loigiai{
Từ vệ tinh $ A$ sẽ phát tín hiệu đến điểm $ M$ và sẽ thu tín hiệu quay về $ A$\\
Từ đó ta có thể suy ra được: $ 2AM=c.t_A\Rightarrow AM=\dfrac{c.t_A}{2}$\\
Do đó $ AM=\dfrac{c.t_A}{2}=\dfrac{3.10^8.0,4.10^{-3}}{2}=60\,000$m$=60$km và ta quy đổi trong hệ trục $ Oxyz$ thì ta có $ AM=6$.\\
Tương tự: $ BM=3$, $ CM=10$, $ DM=6$.\\
Tại thời điểm bốn vị tinh bắn tín hiệu về điểm $ M$. Khi đó ta có hệ phương trình sau:\\
$\Leftrightarrow\left\{\begin{matrix}
{\left(a+1\right)^2}+\left(b-2\right)^2+\left(c+1\right)^2=36\,\,\,\,\,\,\,\,\,\,\,\,\,\,(1)\\
{\left(a-1\right)^2}+\left(b-4\right)^2+c^2=9\,\,\,\,\,\,\,\,\,\,\,\,\,\,\,\,\,\,\,\,\,\,\,\,\,\,\,\,\,(2)\\
{\left(a-3\right)^2}+b^2+\left(c-9\right)^2=100\,\,\,\,\,\,\,\,\,\,\,\,\,\,\,\,\,\,\,\,\,\,\,(3)\\
{\left(a-7\right)^2}+\left(b-10\right)^2+\left(c+1\right)^2=36\,\,\,\,\,\,\,\,\,\,\,(4)\\
\end{matrix}\right.\Leftrightarrow\left\{\begin{aligned}
&(2)-(1):-4a-4b-2c=-38\\ 
&(3)-(1):-8a+4b-20c=-20\\ 
&(4)-(1):-14a-14b=-144\\ 
\end{aligned}\right.$\\
Giải hệ phương trình này, ta tìm được $ a~=~3,b=6,c=1\Rightarrow M\left(3;\,6;\,1\right)$\\
Vậy khoảng cách từ $ M$ đến $ O$ là: $\sqrt{\left(3-0\right)^2+\left(6-0\right)^2+\left(1-0\right)^2}=\sqrt{46}\approx 6,78$.}
\end{ex}

\begin{ex}%Câu 20
Một biển quảng cáo có dạng hình vuông $ ABCD$ cạnh $ AB=4\text{m}$. Trên tấm biển đó có các đường tròn tâm $ A$ và đường tròn tâm $ B$ cùng bán kính $ R=4\text{m}$, hai đường tròn cắt nhau như hình vẽ. Chi phí để sơn phần gạch chéo là $ 150\, 000$ đồng/$\text{m}^2$, chi phí sơn phần màu đen là $ 100\,000$ đồng/ $\text{m}^2$ và chi phí để sơn phần còn lại là $ 250\,000$ đồng/$\text{m}^2$\\
\centerline{\includegraphics[width=.25\textwidth]{images/5.20}}\\
Hỏi số tiền để sơn biển quảng cáo theo cách trên (Đơn vị: triệu đồng và kết quả làm tròn đến chữ số thập phân thứ hai sau dấu phẩy)?
\shortans{2,2}
\loigiai{
Gọi $ I$ là giao điểm của 2 cung tròn $\overset\frown{AC};\overset\frown{BD}$ . Chọn gốc toạ độ $ A\left(0;0\right)$$\Rightarrow B\left(4,0\right)$\\
Xét cung tròn có phương trình $ y=\sqrt{16-x^2}$\\
Phần diện tích gạch chéo $ S=2.\displaystyle\int\limits_2^4\sqrt{16-x^2}\mathrm{\,d}x=16\left.\left(x+\dfrac{1}{2}\sin 2x\right)\right|_{\dfrac{\pi}{6}}^{\dfrac{\pi}{2}}=\dfrac{16\pi}{3}-4\sqrt{3}$\\
Phần diện tích màu đen: $ 2.\left(\dfrac{1}{4}\pi{4^2}-\dfrac{16\pi}{3}+4\sqrt{3}\right)=\dfrac{-8\pi}{3}+8\sqrt{3}$\\
Phần diện tích còn lại: $ 16-\left(\dfrac{16\pi}{3}-4\sqrt{3}+\dfrac{-8\pi}{3}+8\sqrt{3}\right)=16-\dfrac{8\pi}{3}-4\sqrt{3}$\\
Số tiền để sơn biển quảng cáo:\\
$\left(\dfrac{16\pi}{3}-4\sqrt{3}\right).150\text{000}\,\text{+}\left(\dfrac{-8\pi}{3}+8\sqrt{3}\right).100\text{000}+\left(16-\dfrac{8\pi}{3}-4\sqrt{3}\right).250\text{000}\,\approx 2,2$ triệu đồng.}
\end{ex}

\begin{ex}%Câu 21
Anh An thành lập một công ty sản xuất in ấn Sách Giáo Khoa chương trình "Chân trời sáng tạo". Nhằm tạo điều kiện cho các nhà sách tiêu thụ giá hợp lí, đơn giá mỗi bộ sách ban đầu được biểu diễn theo hàm $ p(x)=200-3x$ (nghìn đồng) với $ x$ là số lượng từng bộ sách bán ra và tổng chi phí sản xuất được biểu diễn theo hàm $ C(x)=75+\left(80+T\right)x-x^2$ (nghìn đồng) với mọi $ x$ thỏa $ 0\le x\le 40$, trong đó $ T$ (nghìn đồng) là mức thuế giá trị gia tăng VAT phải đóng trên từng số lượng bộ sách sản xuất ra mà công ty anh An phải chi trả. Xem như công ty anh An sản xuất đều đặn trong điều kiện lí tưởng, khi lợi nhuận của công ty đạt giá trị cao nhất thì tổng mức thuế phải chi trả cũng đồng thời cao nhất. Khi đó mức thuế của mỗi bộ sách mà công ty phải trả là bao nhiêu (đơn vị: nghìn đồng)?
\shortans{60}
\loigiai{
Trước hết ta có hàm chi phí sản xuất là:$ C(x)=75+\left(80+T\right)x-x^2$\\
Doanh thu của công ty anh An biểu diễn theo hàm $ R(x)=x.p(x)=x\left(200-3x\right)$\\
Lợi nhuận mà công ty anh An có được là:$ P(x)=R(x)-C(x)=-2x^2+\left(120-T\right)x-75$\\
Do cần xác định số lượng bộ sách bán ra đề lợi nhuận là cao nhất nên ta có:\\
$P'(x)=0\Leftrightarrow-4x^2+\left(120-T\right)=0\Leftrightarrow x=30-\dfrac{T}{4}$\\
Khi đó với thuế mỗi bộ sách là $ T$ thì tồng mức thuế công ty phải trả là $ G(T)=T\left(30-\dfrac{T}{4}\right)$\\
Khi lợi nhuận của công ty đạt giá trị cao nhất thì tổng mức thuế phải chi trả cũng đồng thời cao nhất nên ta suy ra $G'(T)=0\Leftrightarrow\dfrac{T}{2}-30=0\Leftrightarrow T=60$ (nghìn đồng).}
\end{ex}

\begin{ex}%Câu 22
Bạn Tuấn hằng ngày ăn sáng bằng xôi hoặc bún. Nếu hôm nay bạn ăn sáng bằng xôi thì xác suất để hôm sau bạn ăn sáng bằng bún là $ 0,7$. Xét một tuần mà thứ ba bạn ăn sáng bằng xôi. Biết xác suất để thứ năm tuần đó, bạn Tuấn ăn sáng bằng bún là $ 0,63$. Hỏi nếu hôm nay bạn ăn sáng bằng bún thì xác suất để hôm sau bạn ăn sáng bằng xôi là bao nhiêu?
\shortans{0,4}
\loigiai{
Giả sử nếu hôm nay bạn ăn sáng bằng bún thì xác suất để hôm sau bạn ăn sáng bằng xôi là $ x$ $\left(x<1\right)$.\\
Gọi $ A$ là biến cố “Thứ tư, bạn Tuấn ăn sáng bằng bún”,\\
$ B$ là biến cố “Thứ năm, bạn Tuấn ăn sáng bằng bún”, khi đó $ P(B)=0,63$\\
Ta có thứ ba bạn Tuấn ăn sáng bằng xôi nên $ P(A)=0,7$, $ P\left(\overline{A}\right)=1-0,7=0,3$\\
Vì nếu hôm nay bạn ăn sáng bằng bún thì xác suất để hôm sau bạn ăn sáng bằng xôi là $ x$ và ăn sáng bằng bún là $ 1-x$ hay $ P\left(B|A\right)=1-x$.\\
Ta có $ P\left(B|\overline{A}\right)=0,7$\\
Theo công thức xác suất toàn phần: $ P(B)=P(A).P\left(B|A\right)+P\left(\overline{A}\right).P\left(B|\overline{A}\right)$\\
$\Rightarrow 0,63=0,7.\left(1-x\right)+0,3.0,7$$\Rightarrow x=0,4$\\
Vậy nếu hôm nay bạn ăn sáng bằng bún thì xác suất để hôm sau bạn ăn sáng bằng xôi là $ 0,4$.
}
\end{ex}
\Closesolutionfile{ans}
\Closesolutionfile{ansbook}
\inputansbox{6,2,3}{ans/ans-4-T,ans/ans-4-TF,ans/ans-4-SA}
% \setcounter{deso}{5}
% \begin{name}
	{\tenchude}
	{\tendethi}
	{\tentruong}
	{\thoigian}
	\end{name}
\TN
\Opensolutionfile{ans}[ans/de1-phanI]
\begin{ex}%Câu 1
	Cho cấp số nhân $(u_n)$ với $u_1=2$ và $u_4=16$. Công bội của cấp số nhân đã cho bằng
	\choice
	{$q=4$}
	{\True $q=2$}
	{$q=-2$}
	{$q=-4$}
	\loigiai{
		Ta có $\dfrac{u_4}{u_1}=\dfrac{16}{2}=8=q^3$ nên $q=2$.}
\end{ex}
\begin{ex}%Câu 2
	Cho khối lăng trụ đứng có cạnh bên bằng $5$, đáy là hình vuông có cạnh bằng $4$. Hỏi thể tích của khối lăng trụ bằng bao nhiêu?
	\choice
	{$100$}
	{$20$}
	{$64$}
	{\True $80$}
	\loigiai{
		Ta có thể tích khối lăng trụ bằng $4^2\cdot 5=80$.}
\end{ex}
\textbf{\textit{Sử dụng thông tin dưới đây để trả lời câu \ref{câu 3-đề 1} và câu \ref{câu 4-đề 1}}}\\[0.5em]
Cho hàm số đa thức bậc ba $y=f(x)$ có bảng biến thiên như hình vẽ dưới đây
\begin{center}
	\begin{tikzpicture}
		\tkzTabInit[nocadre=false,lgt=1.2,espcl=2.5,deltacl=0.6]
		{$x$ /0.6, $y'$ /0.6, $y$ /2}
		{$-\infty$,$-1$,$1$,$+\infty$}
		\tkzTabLine{,+,$0$,-,$0$,+,}
		\tkzTabVar{-/$-\infty$,+/$2$,-/$-2$,+/$+\infty$}
	\end{tikzpicture}
\end{center}
\begin{ex}%Câu 3
	\label{câu 3-đề 1}
	Giá trị cực đại của hàm số $y=f(x)$ bằng
	\choice
	{$y_{\text{CĐ}}=-2$}
	{\True $y_{\text{CĐ}}=2$}
	{$y_{\text{CĐ}}=-1$}
	{$y_{\text{CĐ}}=1$}
	\loigiai{
		Dựa vào bảng biến thiên, ta có $y_{\text{CĐ}}=2$.}
\end{ex}
\begin{ex}%Câu 4
	\label{câu 4-đề 1}
	Hàm số nào dưới đây có bảng biến thiên như hình vẽ trên?\vspace{3pt}
	\choice
	{$y=\dfrac{x+1}{x-1}$}
	{$y=-x^3+3x$}
	{$y=x^3+3x$}
	{\True $y=x^3-3x$}
	\loigiai{
		Hàm số có bảng biến thiên như trên có tập xác định là $\mathbb{R}$, đạo hàm có hai nghiệm là $x=1$ và $x=-1$ nên ta loại phương án $\circled{A}$ và $\circled{C}$.\\
		Lại có, $f(x)$ tiến về $+\infty$ khi $x\to+\infty$ nên ta chọn đáp án $\circled{D}$.}
\end{ex}
\begin{ex}%Câu 5
	Tập nghiệm của bất phương trình $3^{-x}\ge\dfrac{1}{27}$ là
	\choice
	{$(-\infty;-3)$}
	{$[3;+\infty)$}
	{$[-3;+\infty)$}
	{\True $(-\infty;3]$}
	\loigiai{
		Ta có $3^{-x}\geq\dfrac{1}{27}\Leftrightarrow -x\geq\log_3\left(\dfrac{1}{27}\right)=-3\Leftrightarrow x\leq 3$.}
\end{ex}
\begin{ex}%Câu 6
	Trong không gian $Oxyz$, cho mặt cầu $(S)\colon x^2+y^2+z^2-4x+2y-2z-3=0$. Tìm tọa độ tâm $I$ và bán kính $R$ của $(S)$.
	\choice
	{\True $I(2;-1;1)$ và $R=3$}
	{$I(-2;1;-1)$ và $R=3$}
	{$I(2;-1;1)$ và $R=9$}
	{$I(-2;1;-1)$ và $R=9$}
	\loigiai{
		Ta có $x^2+y^2+z^2-4x+2y-2z-3=0\Leftrightarrow (x-2)^2+(y+1)^2+(z-1)^2=9$.\\
		Do đó, tâm $I(2;-1;1)$ và $R=3$.}
\end{ex}
\begin{ex}%Câu 7
	Mệnh đề nào \textbf{sai} trong các mệnh đề sau?\vspace{3pt}
	\choice[0.5em]
	{$\displaystyle\int \dfrac{1}{\sin^2 x} \mathrm{\,d}x=-\cot x+C$}
	{$\displaystyle\int \cos x \mathrm{\,d}x=\sin x+C$}
	{$\displaystyle\int \dfrac{1}{\cos^2 x} \mathrm{\,d}x=\tan x+C$}
	{\True $\displaystyle\int \sin x \mathrm{\,d}x=\cos x+C$}
	\loigiai{
		Ta có $\displaystyle\int \sin x \mathrm{\,d}x=-\cos x+C$.}
\end{ex}
\begin{ex}%Câu 8
	Với $a$ là số thực dương tùy ý, $\log_{\sqrt{3}}\left(9a^3\right)$ bằng\vspace{3pt}
	\choice
	{\True $4+6\log_3 a$}
	{$1+\dfrac{3}{2}\log_3 a$}
	{$4-6\log_3 a$}
	{$1-\dfrac{3}{2}\log_3 a$}
	\loigiai{
		Ta có $\log_{\sqrt{3}}\left(9a^3\right)=2\cdot\log_3\left(9a^3\right)=2\cdot\left(2+3\log_3 a\right)=4+6\log_3 a$.}
\end{ex}
\renewcommand{\baselinestretch}{1.55}
\begin{ex}%Câu 9
	\immini[thm]
	{
		Cho hình chóp $S.ABCD$ có đáy là hình bình hành (như hình vẽ minh họa). Hãy chọn khẳng định đúng trong các khẳng định sau.\vspace{3pt}
		\choice[0.3em]
		{\True $\overrightarrow{SA}+\overrightarrow{SC}=\overrightarrow{SB}+\overrightarrow{SD}$}
		{$\overrightarrow{SA}+\overrightarrow{AB}=\overrightarrow{SD}+\overrightarrow{DC}$}
		{$\overrightarrow{SA}+\overrightarrow{AD}=\overrightarrow{SB}+\overrightarrow{BC}$}
		{$\overrightarrow{SA}+\overrightarrow{SB}=\overrightarrow{SC}+\overrightarrow{SD}$}
	}
	{
		\begin{tikzpicture}[scale=0.6,>=stealth, font=\footnotesize, line join=round, line cap=round]
			\def\a{4}
			\path 	(0:0) coordinate (A)
			++(0:\a) coordinate (D)
			++(-130:\a/2) coordinate (C)
			($(A)+(C)-(D)$) coordinate (B)
			($(A)+(80:\a)$) coordinate (S)
			(intersection of A--C and B--D) coordinate (O);%giao điểm O
			\draw[dashed,thick] 	(B)--(A)--(D)	(A)--(S);
			\draw[thick] 			(B)-- (C)--(D)
			(B)--(S)	(C)--(S)	(D)--(S);
			\foreach \x/\g in {A/135,B/-135,C/-45,D/45,S/90}
			\fill[black] 	(\x) circle (1pt)
			($(\g:3mm)+(\x)$) node {$\x$};	
		\end{tikzpicture}
	}
	\loigiai{
		Ta có $\overrightarrow{SA}+\overrightarrow{SC}=\overrightarrow{SB}+\overrightarrow{SD}\Leftrightarrow \overrightarrow{SA}-\overrightarrow{SB}=\overrightarrow{SD}-\overrightarrow{SC}\Leftrightarrow \overrightarrow{BA}=\overrightarrow{CD}$.\\
		Do $ABCD$ là hình bình hành nên $\overrightarrow{BA}=\overrightarrow{CD}$ là đẳng thức đúng.}
\end{ex}
\vspace{15pt}
\begin{ex}%Câu 10
	\immini[thm]
	{
		Hình thang cong $ABCD$ ở hình vẽ bên có diện tích bằng\vspace{3pt}
		\choice[0.3em]
		{$\displaystyle\int\limits_{1}^{3} \left(\dfrac{3}{x}-x+2\right) \mathrm{\,d}x$}
		{$\displaystyle\int\limits_{1}^{3} \left(\dfrac{3}{x}-x-2\right) \mathrm{\,d}x$}
		{$\displaystyle\int\limits_{1}^{3} \left(\dfrac{3}{x}+x+2\right) \mathrm{\,d}x$}
		{\True $\displaystyle\int\limits_{1}^{3} \left(\dfrac{3}{x}+x-2\right) \mathrm{\,d}x$}
	}
	{
		\begin{tikzpicture}[scale=0.6,>=stealth, font=\footnotesize, line join=round, line cap=round]
			\def\xmin{-1} \def\xmax{4}
			\def\ymin{-2} \def\ymax{4} 
			\draw[->] (\xmin,0)--(\xmax,0) node [below]{$x$};
			\draw[->] (0,\ymin)--(0,\ymax) node [left]{$y$};
			\node at (0,0) [below left]{$O$};
			\draw (3,-1.8)node[]{$y=-x+2$} (1,4.2)node[]{$y=\dfrac{3}{x}$};
			\clip (\xmin+0.1,\ymin+0.1) rectangle (\xmax-0.5,\ymax-0.1);
			\draw[smooth,samples=300,domain=\xmin:\xmax] plot(\x,{-(\x)+2});
			\draw[smooth,samples=300,domain=0.1:\xmax] plot(\x,{3/(\x)});
			\draw[dashed] (1,0)node[below]{$1$}--(1,3)--(0,3)node[left]{$3$} (0,-1)node[left]{$-1$}--(3,-1)--(3,0)node[below right]{$3$}--(3,1)--(0,1)node[left]{$1$};
			\fill (1,1)node[below left]{$A$}circle(1pt);
			\fill (1,3)node[above right]{$B$}circle(1pt);
			\fill (3,1)node[above right]{$C$}circle(1pt);
			\fill (3,-1)node[below]{$D$}circle(1pt);
			\fill[color=gray,opacity=0.4] plot[domain=1:3](\x,{3/(\x)})--plot[domain=3:1](\x,{-(\x)+2})--cycle;
		\end{tikzpicture}
	}
	\loigiai{
		Hình thang cong bị giới hạn bởi các đường  $y=\dfrac{3}{x}$, $y=-x+2$, $x=1$ và $x=3$.\\
		Trong khoảng $(1;3)$ đồ thị hàm số $y=\dfrac{3}{x}$ nằm trên đồ thị hàm số $y=-x+2$ nên diện tích hình thang cong $ABCD$ bằng $\displaystyle\int\limits_{1}^{3} \left(\dfrac{3}{x}+x-2\right) \mathrm{\,d}x$.}
\end{ex}
\begin{ex}%Câu 11
	Trong không gian $Oxyz$, cho mặt phẳng $(P)\colon 2x-y-2z+3=0$. Đường thẳng $\Delta$ đi qua điểm $M(4;1;-3)$ và vuông góc với $(P)$ có phương trình chính tắc là\vspace{4pt}
	\choice[0.3em]
	{$\dfrac{x+4}{2}=\dfrac{y+1}{-1}=\dfrac{z-3}{-2}$}
	{$\dfrac{x-2}{4}=\dfrac{y+1}{1}=\dfrac{z+2}{-3}$}
	{$\dfrac{x+2}{2}=\dfrac{y+2}{1}=\dfrac{z-3}{-2}$}
	{\True $\dfrac{x-4}{2}=\dfrac{y-1}{-1}=\dfrac{z+3}{-2}$}
	\loigiai{
		Đường thẳng $\Delta$ vuông góc với $(P)$ nên có một vectơ chỉ phương là $(2;-1;-2)$.\\
		Phương trình chính tắc của $\Delta$ là $\dfrac{x-4}{2}=\dfrac{y-1}{-1}=\dfrac{z+3}{-2}$.}
\end{ex}
\begin{ex}%Câu 12
	Kết quả khảo sát năng suất (đơn vị: tấn/ha) của một số thửa ruộng được minh họa ở biểu đồ sau
	\begin{center}
		\begin{tikzpicture}[line join=round, line cap=round,>=stealth,xscale=1.2,yscale=0.7,font=\footnotesize,scale=0.8]
			\def\a{1}
			\def\xmax{9}
			\def\ymax{7}
			\tikzset{label style/.style={font=\footnotesize}}
			\draw[->] (0,0)--(\xmax,0) node[below] {\text{Năng suất (tấn/ha)}};
			\draw[->] (0,0)--(0,\ymax) node[left] {\text{Số thửa ruộng}};
			\draw (0,0) node [below left] {$O$};
			\draw[thin]
			(0,1)--(\xmax,1)
			(0,2)--(\xmax,2)
			(0,3)--(\xmax,3)
			(0,4)--(\xmax,4)
			(0,5)--(\xmax,5)
			(0,6)--(\xmax,6)
			;
			\foreach \i in {1,2,3,4,5,6}{
				\draw (0,\i) node[left]{$\i$};
			}
			%		\foreach \i/\j in {0.5*\a/3,1.5*\a/4,2.5*\a/6,3.5*\a/5,4.5*\a/5,5.5*\a/2}{
				%			\draw (\i,\j) node[above]{$\j$};
				%		}
			\foreach \i/\j in {1.5*\a/{$[5{,}5;5{,}7)$},2.5*\a/{$[5{,}7;5{,}9)$},3.5*\a/{$[5{,}9;6{,}1)$},4.5*\a/{$[6{,}1;6{,}3)$},5.5*\a/{$[6{,}3;6{,}5)$},6.5*\a/{$[6{,}5;6{,}7)$}}{
				\draw (\i-0.1,-0.6) node[below,rotate=45]{$\j$};
			}
			\fill[blue!20]
			(\a,0) rectangle (2*\a,3)
			(2*\a,0) rectangle (3*\a,4)
			(3*\a,0) rectangle (4*\a,6)
			(4*\a,0) rectangle (5*\a,5)
			(5*\a,0) rectangle (6*\a,5)
			(6*\a,0) rectangle (7*\a,2)
			;
			\begin{scope}
				\draw
				(\a,3)--(2*\a,3) 
				(2*\a,4)--(3*\a,4)
				(3*\a,6)--(4*\a,6)
				(4*\a,5)--(5*\a,5)
				(5*\a,5)--(6*\a,5)
				(6*\a,2)--(7*\a,2)
				(\a,0)--(\a,3)
				(2*\a,0)--(2*\a,4)
				(3*\a,0)--(3*\a,6)
				(4*\a,0)--(4*\a,6)
				(5*\a,0)--(5*\a,5)
				(6*\a,0)--(6*\a,5)
				(7*\a,0)--(7*\a,2)
				;
				\draw (3.5,\ymax) node[above]{\textbf{Năng suất lúa của một số thửa ruộng}};
			\end{scope}
		\end{tikzpicture}
	\end{center}
	Lập bảng tần số ghép nhóm ta tính được khoảng tứ phân vị của mẫu số liệu trên \textbf{gần bằng} giá trị nào dưới đây?
	\choice
	{$0{,}3$}
	{$0{,}4$}
	{\True $0{,}5$}
	{$0{,}6$}
	\loigiai{
		Bảng tần số ghép nhóm của mẫu số liệu trên.
		\begin{center}
			\begin{tabular}{|c|c|c|c|c|c|c|}
				\hline
				Nhóm &$[5{,}5;5{,}7]$ &$[5{,}7;5{,}9]$  &$[5{,}9;6{,}1]$  &$[6{,}1;6{,}3]$  &$[6{,}3;6{,}5]$  &$[6{,}5;6{,}7]$  \\
				\hline
				Tần số &$3$ &$4$  &$6$  &$5$  &$5$  &$2$  \\
				\hline
			\end{tabular}
		\end{center}
		Mẫu số liệu có $25$ giá trị nên trung vị là giá trị thứ $13$.\\ 
		Do đó, tứ phân vị thứ nhất là trung bình cộng của giá trị thứ $6$ và thứ $7$ nên $Q_1\in[5{,}7;5{,}9]$, tứ phân vị thứ ba là trung bình cộng của giá trị thứ $19$ và $20$ nên $Q_3\in[6{,}3;6{,}5]$.\\
		Ta có $Q_1=5{,}7+\dfrac{\dfrac{25}{4}-3}{4}\cdot0{,}2=5{,}8625$; $Q_3=6{,}3+\dfrac{\dfrac{25\cdot 3}{4}-18}{5}\cdot0{,}2=6{,}33$.\\
		Vậy khoảng tứ phân vị $\Delta_Q=6{,}33-5{,}8625=0{,}4675\approx 0{,}5$.}
\end{ex}
\Closesolutionfile{ans}
%\renewcommand{\baselinestretch}{1.35}
%{\fontfamily{qtm}\fontsize{13pt}{2pt}\selectfont\textbf{PHẦN II. Câu trắc nghiệm đúng sai}. Thí sinh trả lời từ câu 1 đến câu 4. Trong mỗi ý \textbf{a)}, \textbf{b)}, \textbf{c)}, \textbf{d)} ở mỗi câu, thí sinh chọn đúng hoặc sai.}
%\setcounter{ex}{0}% Reset lại số đếm câu hỏi
\TNTF
\Opensolutionfile{ans}[ans/de1-phanII]
\begin{ex}%Câu 1
	Cho hàm số $f(x)=\left(x^2-3x-3\right)\mathrm{e}^x$. Xét tính đúng sai của các mệnh đề sau
	\choiceTF
	{\True Hàm số đã cho xác định với mọi $x\in\mathbb{R}$}
	{Đạo hàm của hàm số đã cho là $f'(x)=\left(x^2+x-6\right)\mathrm{e}^x$}
	{\True Phương trình $f'(x)=0$ có hai nghiệm thực phân biệt}
	{\True Hàm số $f(x)$ nghịch biến trên khoảng $(-2;3)$}
	\loigiai{
		\begin{itemchoice}
			\itemch \textbf{Đúng.}\\
			Hàm số đã cho là tích của hàm đa thức và hàm mũ nên tập xác định là $\mathbb{R}$.
			\itemch \textbf{Sai.}\\
			Ta có $f'(x)=(2x-3)\cdot\mathrm{e}^x+(x^2-3x-3)\cdot\mathrm{e}^x=(x^2-x-6)\mathrm{e}^x$.
			\itemch \textbf{Đúng.}\\
			Ta có $f'(x)=0\Leftrightarrow (x^2-x-6)\mathrm{e}^x=0\Leftrightarrow x^2-x-6=0\Leftrightarrow\hoac{&x=-2 \\&x=3.}$
			\itemch \textbf{Đúng.}\\
			Ta có $f'(x)\leq 0\Leftrightarrow (x^2-x-6)\mathrm{e}^x\leq 0\Leftrightarrow x^2-x-6\leq 0\Leftrightarrow -2\leq x\leq 3$.\\
			Vậy hàm số nghịch biến trên $[-2;3]$ nên cũng nghịch biến trên $(-2;3)$.
	\end{itemchoice}}
\end{ex}
\begin{ex}%Câu 2
	Một nắp bể nước hình chữ nhật $ABCD$ nằm cạnh bờ tường có kích thước $9\text{ dm}\times 12\text{ dm}$ được kéo ra từ mặt sàn, do tác dụng của trọng lực nên nắp bể không thể mở ra được nếu không có người giữ. Người ta dùng một sợi dây dài $15$ dm và kéo căng nối đỉnh $C$ của hình chữ nhật với điểm $M$ nằm phía trên bờ tường sao cho $AM=9$ dm và $AM$ vuông góc với mặt sàn. Chọn hệ trục $Oxyz$ như hình vẽ, khi đó nắp bể mở ra và tạo với mặt sàn một góc $\alpha$ (đơn vị trên mỗi trục tọa độ tính bằng dm). Bỏ qua độ dày của nắp bể.
	\immini{
	Xét tính đúng sai của các mệnh đề sau
	\choiceTF
	{Điểm $M$ thuộc mặt phẳng có phương trình $z=0$}
	{Tọa độ điểm $C$ là $C\left(9\sin\alpha;12;9\cos\alpha\right)$}
	{Góc giữa nắp bể và mặt sàn sau khi kéo lên là $\alpha=60^\circ$}
	{\True Phương trình mặt phẳng chứa nắp bể sau khi kéo lên là $x-\sqrt{3}z=0$}}
	{\begin{tikzpicture}[scale=0.5,>=stealth, font=\footnotesize, line join=round, line cap=round]
			\draw[fill=gray!90] (-6.4,1.5)--(-6,1.3)--(-5.82,-3.49)--(-2.02,-6.02)--(5.8,-1.33)--(6.2,-1.6)--(-2,-6.52)--(-6.2,-3.72)--cycle;
			\draw[fill=gray!40] (-6.4,1.5)--(1.1,5.7)--(1.4,5.44)--(-6,1.3)--cycle;
			\draw[fill=gray!60] (-6,1.3)--(1.4,5.44)--(2,1.2)--(-5.82,-3.49)--cycle;
			\draw (1.4,5.44)--(2,1.2)--(6.2,-1.6);
			\coordinate (A) at (0,0);
			\coordinate (C) at (-1.42,-2.71);
			\coordinate (D) at (-4.3,-2.58);
			\coordinate (B) at ($(A)+(C)-(D)$);
			\draw[fill=black!60,opacity=0.9] (A)--(B)--(C)--(D)--cycle;
			\draw[fill=black!80] (B)--(3.05,-0.28)--(-1.42,-2.97)--(-3.99,-2.85)--(-4.3,-2.58)--(-1.42,-2.71)--cycle;
			\coordinate (E) at (-1.73,-4.82);
			\coordinate (F) at ($(A)+(E)-(D)$);
			\coordinate (M) at (0,3.72);
			\draw (D)--(E) (A)--(F) (E)--(F) (M)node[above right]{$M$}--(C);
			\fill[gray!40] (-5.82,-3.49)--(-2.02,-6.02)--(5.8,-1.33)--(2,1.2)--(A)--(B)--(3.05,-0.28)--(1.44,-1.25)--(F)--(E)--(D)--cycle;
			\draw[->] (0,0)node[above left,xshift=0.1cm,yshift=-0.1cm]{$O$}--(0,6)node[right]{$z$};
			\draw[->] (2,1.2)--(-7,-4.2)node[above]{$y$};
			\draw[->] (0,0)--(5.9,-3.5)node[above]{$x$};
			\draw (A)node[above right,yshift=0.2cm,xshift=-0.1cm]{$A$} (B)node[above]{$B$} ($(A)!0.5!(B)$)node[above,xshift=0.1cm]{$9$ dm} ($(A)!0.5!(D)$)node[above,xshift=-0.25cm]{$12$ dm} (D)node[above]{$D$} (C)node[above left]{$C$} (D)node[xshift=0.7cm,yshift=-0.4cm]{$\alpha$};
			\draw[dashed] (C) to[bend left] (E);
			\draw[dashed] (B) to[bend left] (F);
			\draw[decorate,decoration={markings,mark=between positions 2mm and \pgfdecoratedpathlength-2mm step 2mm with{\draw[black] (-3.5mm,-1.25mm) to[out=0,in=160] (-2mm,-1.25mm) to[out=-20,in=160] (2mm,1.25mm) to[out=-20,in=180] (3.5mm,1.25mm);}}] (M) -- (C);
			\fill (A)circle(3pt);
			\fill (B)circle(3pt);
			\fill (C)circle(3pt);
			\fill (D)circle(3pt);
			\fill (M)circle(3pt);
	\end{tikzpicture}}
	\loigiai{
		\begin{itemchoice}
			\itemch \textbf{Sai.}\\
			Điểm $M\in (Oxz)$ có phương trình $x=0$.
			\itemch \textbf{Sai.}\\
			Tọa độ điểm $C$ là $C(9\cos\alpha;12;9\sin\alpha)$.
			\itemch \textbf{Sai.}\\
			Ta có 
			\begin{align*}
				CM=15\Leftrightarrow CM^2=225&\Leftrightarrow (9\cos\alpha)^2+12^2+(9\sin\alpha-9)^2=225\\
				&\Leftrightarrow 81+12^2+9^2-162\sin\alpha=225\\
				&\Leftrightarrow162\sin\alpha=81\\
				&\Leftrightarrow\sin\alpha=\dfrac{1}{2}\Leftrightarrow\alpha=30^\circ.
			\end{align*}
			\itemch \textbf{Đúng.}\\
			Mặt phẳng chứa nắp bể sau khi kéo lên đi qua $3$ điểm $A(0;0;0)$, $C\left(\dfrac{9\sqrt{3}}{2};12;\dfrac{9}{2}\right)$ và $D(0;12;0)$.\\
			Do đó, vectơ pháp tuyến của mặt phẳng cùng phương với $\left[\overrightarrow{AC},\overrightarrow{AD}\right]$.\\
			Ta có $\overrightarrow{AC}=\left(\dfrac{9\sqrt{3}}{2};12;\dfrac{9}{2}\right)$, $\overrightarrow{AD}=(0;12;0)$ nên $\left[\overrightarrow{AC},\overrightarrow{AD}\right]=(-54;0;54\sqrt{3})$.\\
			Suy ra một vectơ pháp tuyến của mặt phẳng là $\overrightarrow{n}=(1;0;-\sqrt{3})$.\\
			Vậy phương trình mặt phẳng là $x-\sqrt{3}z=0$.
	\end{itemchoice}}
\end{ex}
\begin{ex}%Câu 3
	Trong một ngôi làng có $500$ người thì $240$ người là nam. Thống kê cho thấy rằng, khả năng mắc bệnh hô hấp ở người nam trong làng là $0{,}6\%$ và ở người nữ trong làng là $0{,}35\%$. Giả sử gặp một người trong làng.
	\begin{itemize}
		\item Gọi $A$ là biến cố \lq\lq gặp người mắc bệnh trong làng\rq\rq.
		\item Gọi $B$ là biến cố \lq\lq gặp được nam trong làng\rq\rq.
	\end{itemize}
	Xét tính đúng sai của các mệnh đề sau\vspace{3pt}
	\choiceTF
	{\True $\mathrm{P}(\overline{B})=\dfrac{13}{25}$}
	{Xác suất có điều kiện $\mathrm{P}(A\mid\overline{B})=0{,}006$}
	{Tỉ lệ mắc bệnh hô hấp chung của cả làng là $0{,}42\%$}
	{Giả sử có một người trong làng không mắc bệnh. Xác suất để người đó là nữ bằng $47{,}94\%$}
	\loigiai{
		\begin{itemchoice}
			\itemch \textbf{Đúng.}\\
			Ta có $\mathrm{P}(B)=\dfrac{240}{500}=\dfrac{12}{25}\Rightarrow\mathrm{P}(\overline{B})=1-\dfrac{12}{25}=\dfrac{13}{25}$.
			\itemch \textbf{Sai.}\\
			Ta có $\mathrm{P}(A\mid\overline{B})=0{,}35\%=0{,}0035$.
			\itemch \textbf{Sai.}\\
			Ta có $\mathrm{P}(A)=\mathrm{P}(B)\cdot\mathrm{P}(A\mid B)+\mathrm{P}(\overline{B})\cdot\mathrm{P}(A\mid\overline{B})=\dfrac{12}{25}\cdot 0{,}6\%+\dfrac{13}{25}\cdot 0{,}35\%=0{,}47\%$.
			\itemch \textbf{Sai.}\\
			Ta cần tính $\mathrm{P}(\overline{B}\mid \overline{A})$. Ta có $\mathrm{P}(\overline{B}\mid \overline{A})=\dfrac{\mathrm{P}(\overline{B})\cdot\mathrm{P}(\overline{A}\mid\overline{B})}{\mathrm{P}(\overline{B})\cdot\mathrm{P}(\overline{A}\mid\overline{B})+\mathrm{P}(B)\cdot\mathrm{P}(\overline{A}\mid B)}$.\\
			Lại có $\mathrm{P}(\overline{A}\mid\overline{B})=99{,}65\%$, $\mathrm{P}(\overline{A}\mid B)=99{,}4\%$. Do đó, $\mathrm{P}(\overline{B}\mid\overline{A})=\dfrac{\dfrac{13}{25}\cdot99{,}65\%}{\dfrac{13}{25}\cdot99{,}65\%+\dfrac{12}{25}\cdot 99{,}4\%}\approx 52{,}06\%$.
	\end{itemchoice}}
\end{ex}
\renewcommand{\baselinestretch}{1.55}
\begin{ex}%Câu 4
	Một quần thể vi khuẩn $(A)$ có số lượng cá thể là $P(t)$ sau $t$ phút quan sát được phát hiện thay đổi với tốc độ là $P'(t)=a\mathrm{e}^{0{,}1t}+150\mathrm{e}^{-0{,}03t}$ (vi khuẩn/phút) $(a\in\mathbb{R})$. Biết rằng lúc bắt đầu quan sát, quần thể có $200\,000$ vi khuẩn và đạt tốc độ tăng trưởng là $350$ vi khuẩn/phút. Xét tính đúng sai của các mệnh đề sau
	\choiceTF
	{\True Giá trị của $a=200$}
	{$P(t)=2\,000\mathrm{e}^{0{,}1t}-5\,000\mathrm{e}^{-0{,}03t}+200\,000$}
	{\True Sau $12$ phút số lượng vi khuẩn trong quần thể là $206\,152$ con (làm tròn kết quả đến hàng đơn vị)}
	{Sau $12$ phút, một quần thể vi khuẩn $(B)$ có tốc độ tăng trưởng là $G'(t)=500\mathrm{e}^{0{,}2t}$ (vi khuẩn/phút) bắt đầu cạnh tranh nguồn thức ăn trực tiếp với quần thể $(A)$, một cá thể tại quần thể $(B)$ triệt tiêu một cá thể tại quần thể $(A)$. Sau $5$ phút cạnh tranh quần thể $(A)$ bị triệt tiêu hoàn toàn. Số lượng vi khuẩn của quần thể $(B)$ ở thời điểm bắt đầu cạnh tranh là $191\,967$ con (làm tròn kết quả đến hàng đơn vị)}
	\loigiai{
		\begin{itemchoice}
			\itemch \textbf{Đúng.}\\
			Tại $t=0$, ta có $P'(0)=350\Leftrightarrow a+150 =350\Leftrightarrow a=200$.
			\itemch \textbf{Sai.}\\
			Ta có 
			\begin{align*}
				P(t)=\displaystyle\int P'(t)\mathrm{\,d}t&=\int\left(200\mathrm{e}^{0{,}1t}+150\mathrm{e}^{-0{,}03t}\right)\mathrm{\,d}t\\
				&=\dfrac{200}{0{,}1}\mathrm{e}^{0{,}1t} - \dfrac{150}{0{,}03}\mathrm{e}^{-0{,}03t}+C\\
				&=2\,000\mathrm{e}^{0{,}1t} - 5\,000\mathrm{e}^{-0{,}03t}+C.
			\end{align*}
			Tại $t=0$, ta có $P(0)=200\,000\Leftrightarrow 2\,000-5\,000+ C= 200\,000\Leftrightarrow C=203\,000$.\\
			Vậy $P(t)=2\,000\mathrm{e}^{0{,}1t}-5\,000\mathrm{e}^{-0{,}03t}+203\,000$.
			\itemch \textbf{Đúng.}\\
			Ta có $P(12)=2\,000\mathrm{e}^{0{,}1\cdot12}-5\,000\mathrm{e}^{-0{,}03\cdot 12}+203\,000\approx 206\,152$.
			\itemch \textbf{Sai.}\\
			Sau $5$ phút từ khi quần thể $(B)$ xuất hiện, số lượng vi khuẩn của quần thể $(A)$ là $P(17)\approx 210\, 945$ con.\\
			Ta có số lượng vi khuẩn của quần thể $(B)$ là $G(t)=\displaystyle\int G'(t)\mathrm{\,d}t=2\,500\mathrm{e}^{0{,}2t}+C$.\\
			Vì một cá thể tại quần thể $(B)$ triệt tiêu một cá thể tại quần thể $(A)$ và sau $5$ phút cạnh tranh quần thể $(A)$ bị triệt tiêu hoàn toàn nên $G(5)=P(17)$, tức là
			\[2\,500\mathrm{e}^{0{,}2\cdot5}+C=210\, 945\Leftrightarrow C\approx 204\, 149.\]
			Vậy số lượng vi khuẩn của quần thể $(B)$ ban đầu là $G(0)\approx 206\,649$ vi khuẩn.
	\end{itemchoice}}
\end{ex}
\Closesolutionfile{ans}
%{\fontfamily{qtm}\fontsize{13pt}{2pt}\selectfont\textbf{PHẦN III. Câu trắc nghiệm trả lời ngắn}. Thí sinh trả lời từ câu 1 đến câu 6 và điền đáp án vào ô trống.}
%\setcounter{ex}{0}% Reset lại số đếm câu hỏi
\TNSA
\Opensolutionfile{ans}[ans/de1-phanIII]
\begin{ex}%Câu 1
	% \immini[thm]
	% {
		Cho hình lập phương $ABCD.A'B'C'D'$ có cạnh $a$. Gọi $I$ là trung điểm của cạnh $BD$. Góc giữa hai đường thẳng $A'D$ và $B'I$ bằng bao nhiêu độ?
	% }
	% {
	% 	\begin{tikzpicture}[scale=0.75,>=stealth, font=\footnotesize, line join=round, line cap=round]
	% 		\def\a{3.5}
	% 		\path 	(0:0) coordinate (A)
	% 		++(0:\a) coordinate (D)
	% 		++(-130:\a/2) coordinate (C)
	% 		($(A)+(C)-(D)$) coordinate (B)
	% 		($(A)+(90:\a)$) coordinate (A')
	% 		($(B)+(90:\a)$) coordinate (B')
	% 		($(C)+(90:\a)$) coordinate (C')
	% 		($(D)+(90:\a)$) coordinate (D')
	% 		($(B)!0.5!(D)$) coordinate (I)
	% 		;
	% 		\draw[dashed,thick] 	(B)--(A)--(D)	(A)--(A');
	% 		\draw[thick] 	(C)--(C') 	(D)--(D') 	(B)--(B') 	(B)--(C)--(D) (A')--(B')--(C')--(D')--cycle;
	% 		\foreach \x/\g in {A/180,B/180,C/0,D/0,A'/180,B'/180,C'/0,D'/0}
	% 		\fill[black] 	(\x) circle (1pt)
	% 		($(\g:4mm)+(\x)$) node {$\x$};	
	% 	\end{tikzpicture}
	% }
	\shortans[0]{$30$}
	\loigiai{
		\begin{center}
			\begin{tikzpicture}[scale=0.75,>=stealth, font=\footnotesize, line join=round, line cap=round]
				\def\a{3.5}
				\path 	(0:0) coordinate (A)
				++(0:\a) coordinate (D)
				++(-130:\a/2) coordinate (C)
				($(A)+(C)-(D)$) coordinate (B)
				($(A)+(90:\a)$) coordinate (A')
				($(B)+(90:\a)$) coordinate (B')
				($(C)+(90:\a)$) coordinate (C')
				($(D)+(90:\a)$) coordinate (D')
				($(B)!0.5!(D)$) coordinate (I)
				;
				\draw[dashed,thick] 	(B)--(A)--(D)	(A)--(A') (B)--(D)--(A') (B')--(I);
				\draw[thick] 	(C)--(C') 	(D)--(D') 	(B)--(B')--(C) 	(B)--(C)--(D) (A')--(B')--(C')--(D')--cycle;
				\foreach \x/\g in {A/180,B/180,C/0,D/0,A'/180,B'/180,C'/0,D'/0,I/-90}
				\fill[black] 	(\x) circle (1pt)
				($(\g:4mm)+(\x)$) node {$\x$};	
			\end{tikzpicture}
		\end{center}
		Ta có $A'D\parallel B'C$ nên $(A'D,B'I)=(B'C,B'I)=\widehat{IB'C}$.\\
		Lại có $B'C=a\sqrt{2}$, $IC=\dfrac{a\sqrt{2}}{2}$, $B'I=\sqrt{BI^2+BB'^2}=\sqrt{\dfrac{a^2}{2}+a^2}=\dfrac{a\sqrt{6}}{2}$.\\
		Do đó, $\cos\widehat{IB'C}=\dfrac{B'C^2+B'I^2-IC^2}{2B'I\cdot B'C}=\dfrac{\sqrt{3}}{2}\Rightarrow\widehat{IB'C}=30^\circ$.}
\end{ex}
\vspace{15pt}
\begin{ex}%Câu 2
	\immini[thm]
	{
		Từ kho $D$ xe bưu chính đến lấy thư từ các hộp thư tại $E$, $F$, $G$ và $H$ rồi quay lại kho. Sơ đồ bên hiển thị thời gian xe bưu chính di chuyển giữa các hộp thư (đơn vị: phút). Thời gian ngắn nhất để xe bưu chính thực hiện điều đó là bao nhiêu phút?
	}
	{
		\begin{tikzpicture}[scale=0.81,>=stealth, font=\footnotesize, line join=round, line cap=round]
			\coordinate (H) at (0,0);
			\coordinate (G) at (4,0);
			\coordinate (E) at (2.5,4);
			\coordinate (D) at (-0.5,2);
			\coordinate (F) at (4.6,2.4);
			\draw (E)--(D)--(H)--(G)--(F)--cycle (E)--(H) (E)--(G) (D)--(F) (D)--(G) (H)--(F);
			\foreach \x/\g in {D/180,E/90,F/0,G/-45,H/-150} 
			\fill[black] (\x) circle(2pt) +(\g:4mm) node {$\x$};
			\draw ($(E)!0.5!(D)$)node[above left]{$11$} ($(E)!0.5!(F)$)node[above right]{$7$} ($(D)!0.5!(H)$)node[below left]{$3$} ($(G)!0.5!(F)$)node[below right]{$13$} ($(H)!0.6!(F)$)node[above left,yshift=-0.15cm]{$10$} ($(D)!0.6!(F)$)node[above]{$9$} ($(E)!0.7!(G)$)node[right]{$10$} ($(D)!0.4!(G)$)node[above]{$7$} ($(H)!0.7!(E)$)node[left]{$9$} ($(H)!0.5!(G)$)node[below]{$6$};
		\end{tikzpicture}
	}
	\shortans[0]{$35$}
	\loigiai{
		Xét một lộ trình di chuyển thỏa đề bài, chẳng hạn $D\to E\to F\to G\to H\to D$. Ta nhận thấy trong lộ trình di chuyển này, tại mỗi điểm, xe bưu chính bao giờ cũng sẽ có một tuyến đường vào và có một tuyến đường ra. Chẳng hạn, tại $D$ tuyến đường vào là $H\to D$, tuyến đường ra là $D\to E$.\\
		Điều này cho ta thấy, để chọn một lộ trình thỏa mãn đề bài, ta phải bỏ đi $2$ trong $4$ tuyến đường xuất phát từ mỗi đỉnh của sơ đồ.\\
		Mặt khác, sơ đồ có tất cả $10$ tuyến đường và một lộ trình thỏa mãn chỉ cần $5$ tuyến nên số đường đi phải bỏ đi là $5$.
		Để thời gian di chuyển của xe bưu chính là ngắn nhất thì tổng thời gian trên $5$ tuyến bị bỏ đi phải là lâu nhất. Tuyến đường nhiều thời gian nhất thỏa mãn lập luận trên là $D\to E\to H\to F\to G\to D$, thời gian thực hiện là $50$.\\
		Mà tổng thời gian trên các tuyến đường là $85$ nên thời gian ngắn nhất để xe bưu chính thực hiện là $85-50=35$ phút.}
\end{ex}
\vspace{15pt}
\begin{ex}%Câu 3
	\immini[thm]
	{
		Kiến trúc sư thiết kế một khu sinh hoạt cộng đồng có dạng hình vuông $ABCD$ có độ dài đường chéo $AC=120$ m. Trong đó, phần được tô màu đậm là sân chơi, phần còn lại để trồng hoa. Mỗi phần trồng hoa có đường biên cong là một phần của đường parabol với đỉnh thuộc một trục đối xứng của hình vuông, khoảng cách từ đỉnh đó đến đỉnh tương ứng của hình vuông bằng $40$ m và\break $AM=MN=NB$ (xem hình minh họa). Diện tích của phần sân chơi là bao nhiêu mét vuông? (làm tròn kết quả đến hàng đơn vị).
	}
	{
		\begin{tikzpicture}[scale=0.7,>=stealth, font=\footnotesize, line join=round, line cap=round]
			\coordinate (A) at (0,3);
			\coordinate (B) at (3,0);
			\coordinate (C) at (0,-3);
			\coordinate (D) at (-3,0);
			\coordinate (M) at (1,2);
			\coordinate (N) at (2,1);
			\draw (A)--(B)--(C)--(D)--cycle;
			\draw[smooth,samples=300,domain=-1:1] plot(\x,{(\x)^2+1});
			\draw[smooth,samples=300,domain=-1:1] plot(\x,{-(\x)^2-1});
			\draw[smooth,samples=300,domain=-1:1] plot({(\x)^2+1},{\x});
			\draw[smooth,samples=300,domain=-1:1] plot({-(\x)^2-1},{\x});
			\foreach \x/\g in {A/90,B/0,C/-90,D/180,M/60,N/60} 
			\fill[black] (\x) circle(2pt) +(\g:4mm) node {$\x$};
			\draw[latex-latex] (A)--(0,1);
			\draw[latex-latex] (B)--(1,0);
			\draw[latex-latex] (C)--(0,-1);
			\draw[latex-latex] (D)--(-1,0);
			\draw (0,2)node[left]{$40$} (0,-2)node[left]{$40$} (2,0)node[above]{$40$} (-2,0)node[above]{$40$};
			\fill[color=gray!80] plot[domain=-1:1](\x,{(\x)^2+1})--plot[domain=1:-1]({(\x)^2+1},{\x})--plot[domain=1:-1](\x,{-(\x)^2-1})--plot[domain=-1:1]({-(\x)^2-1},{\x})--cycle;
		\end{tikzpicture}
	}
	\shortans[0]{$3467$}
	\loigiai{
		\begin{center}
			\begin{tikzpicture}[scale=0.7,>=stealth, font=\footnotesize, line join=round, line cap=round]
				\coordinate (A) at (0,3);
				\coordinate (B) at (3,0);
				\coordinate (C) at (0,-3);
				\coordinate (D) at (-3,0);
				\coordinate (M) at (1,2);
				\coordinate (N) at (2,1);
				\draw (A)--(B)--(C)--(D)--cycle;
				\draw[smooth,samples=300,domain=-1:1] plot(\x,{(\x)^2+1});
				\draw[smooth,samples=300,domain=-1:1] plot(\x,{-(\x)^2-1});
				\draw[smooth,samples=300,domain=-1:1] plot({(\x)^2+1},{\x});
				\draw[smooth,samples=300,domain=-1:1] plot({-(\x)^2-1},{\x});
				\foreach \x/\g in {A/120,B/-40,C/-120,D/130,M/60,N/60} 
				\fill[black] (\x) circle(2pt) +(\g:4mm) node {$\x$};
				\draw[latex-latex] (A)--(0,1);
				\draw[latex-latex] (B)--(1,0);
				\draw[latex-latex] (C)--(0,-1);
				\draw[latex-latex] (D)--(-1,0);
				\draw (0,2)node[left]{$40$} (0,-2)node[left]{$40$} (2,0)node[below]{$40$} (-2,0)node[above]{$40$};
				\fill[color=gray!80] plot[domain=-1:1](\x,{(\x)^2+1})--plot[domain=1:-1]({(\x)^2+1},{\x})--plot[domain=1:-1](\x,{-(\x)^2-1})--plot[domain=-1:1]({-(\x)^2-1},{\x})--cycle;
				\path 	(0,0) coordinate (O);	
				\draw[-stealth] (-5,0)--(5,0)node[above]{$x$};
				\draw[-stealth] (0,-5)--(0,5)node[left]{$y$};
				\node [below left] at (0,0) {$O$};
				\draw[dashed, thick] (2,1)--(2,0) (1,2)--(0,2); 
			\end{tikzpicture}
		\end{center}
		Ta chọn hệ trục tọa độ $Oxy$ như hình vẽ, trong đó $O$ là giao điểm hai đường chéo $AC$ và $BD$, trục $Ox$ trùng với đường thẳng $BD$, trục $Oy$ trùng với đường thẳng $AC$.\\
		Khi đó, tọa độ các điểm là $A(0;60)$, $B(60;0)$, $C(0;-60)$, $D(-60;0)$.\\
		Vì $AM=MN=NB$ nên $\overrightarrow{AM}=\dfrac{1}{3}\overrightarrow{AB}$. Giả sử $M(x;y)$, khi đó $\heva{&x=\dfrac{1}{3}\cdot 60=20 \\&y-60=\dfrac{1}{3}\cdot(-60)=-20}\Leftrightarrow\heva{&x=20 \\&y=40.}$\\
		Tương tự, ta có $\overrightarrow{AN}=\dfrac{2}{3}\overrightarrow{AB}$ nên $N(40;20)$.\\
		Đường parabol đi qua điểm $M$ có đỉnh nằm trên trục $Oy$, cắt trục $Oy$ tại điểm có tung độ $20$ có phương trình là $y=\dfrac{1}{20}x^2+20$.\\
		Đường parabol đi qua điểm $N$ có đỉnh nằm trên trục $Ox$, cắt trục $Ox$ tại điểm có hoành độ $20$ có phương trình là $x=\dfrac{1}{20}y^2+20$.\\
		Ta gọi $(H_1)$ là hình giới hạn bởi parabol $y=\dfrac{1}{20}x^2+20$, trục $Oy$, đường thẳng $AM$ có phương trình $y=-x+60$.\\
		Ta gọi $(H_2)$ là hình giới hạn bởi parabol $x=\dfrac{1}{20}y^2+20$, trục $Ox$, đường thẳng $BN$ có phương trình $y=-x+60$.\\
		Khi đó, \[S_{H_1}=\displaystyle\int\limits_{20}^{40}\sqrt{20\cdot(y-20)}\mathrm{\,d}y+\int\limits_{40}^{60}(60-y)\mathrm{\,d}y=\dfrac{1\,400}{3};\quad S_{H_2}=\displaystyle\int\limits_{20}^{40}\sqrt{20\cdot(x-20)}\mathrm{\,d}x+\int\limits_{40}^{60}(60-x)\mathrm{\,d}x=\dfrac{1\,400}{3}.\]
		Vậy diện tích một phần tư sân chơi bẳng $S_{AOB}-S_{H_1}-S_{H_2}=\dfrac{1}{2}\cdot 60\cdot 60 -2\cdot\dfrac{1\,400}{3}=\dfrac{2\,600}{3}$ nên diện tích sân chơi là $4\cdot\dfrac{2\,600}{3}\approx 3\,467$ m$^2$.}
\end{ex}
\renewcommand{\baselinestretch}{1.5}
\begin{ex}%Câu 4
	Sách Toán của một đơn vị xuất bản được in tại hai phân xưởng $A$ và $B$ và được vận chuyển về kho sau khi in xong. Xưởng $A$ có nhiệm vụ in $60\%$ tổng số lượng sách, xưởng $B$ sẽ in số lượng sách còn lại. Biết rằng số lượng sách Toán xưởng $A$ và $B$ in đạt yêu cầu về chất lượng và chuyển về kho lần lượt là $95\%$ và $90\%$. Nhân viên kiểm kho chọn ra ngẫu nhiên một cuốn sách Toán để kiểm tra thì thấy cuốn sách này không đạt yêu cầu về chất lượng. Xác suất để cuốn sách Toán đó được in ở xưởng $A$ là bao nhiêu phần trăm? (Làm tròn kết quả đến hàng phần chục).
	
	\shortans[0]{$42{,}9$}
	\loigiai{
		Gọi $M$ là biến cố quyển sách được in ở xưởng $A$, $N$ là biến cố quyển sách in không đạt yêu cầu về chất lượng.\\
		Theo đề, ta cần tính $\mathrm{P}(M\mid N)=\dfrac{\mathrm{P}(M)\cdot\mathrm{P}(N\mid M)}{\mathrm{P}(M)\cdot\mathrm{P}(N\mid M)+\mathrm{P}(\overline{M})\cdot\mathrm{P}(N\mid \overline{M})}$.\\
		Lại có $\mathrm{P}(M)=0{,}6$, $\mathrm{P}(\overline{M})=0{,}4$, $\mathrm{P}(N\mid M)=0{,}05$, $\mathrm{P}(N\mid \overline{M})=0{,}1$.\\
		Suy ra $\mathrm{P}(M\mid N)\approx 42{,}9\%$.}
\end{ex}
\begin{ex}%Câu 5
	\immini[thm]{Một phần của đường chạy của tàu lượn siêu tốc khi gắn hệ trục tọa độ $Oxy$ được mô phỏng ở hình dưới. Biết đường chạy của nó có dạng đồ thị hàm số bậc ba $y=ax^3+bx^2+cx+d$ $(0\le x\le 90)$; tàu lượn xuất phát từ điểm $A$ đồng thời đi qua các điểm $B$, $C$, $D$ (như hình vẽ). Đơn vị mỗi trục là mét, dựa vào đồ thị hình dưới, em hãy tính độ cao lớn nhất (theo đơn vị mét) mà tàu lượn siêu tốc đạt được so với mặt đất (xem $Ox$ là mặt đất). Kết quả làm tròn đến hàng phần mười.}
	{\begin{tikzpicture}[scale=0.7,>=stealth, font=\footnotesize, line join=round, line cap=round]
			\def\a{-1/15} \def\b{13/15} \def\c{-8/3} \def\d{3} % Hệ số
			\def\xmin{-1} \def\xmax{10}
			\def\ymin{-1} \def\ymax{5} 
			\draw[->] (\xmin,0)--(\xmax,0) node [below]{$x$};
			\draw[->] (0,\ymin)--(0,\ymax) node [left]{$y$};
			\node at (0,0) [below left]{$O$};
			\clip (\xmin+0.1,\ymin+0.1) rectangle (\xmax-0.5,\ymax-0.1);
			\draw[smooth,samples=300,domain=0:9] plot(\x,{\a*(\x)^3+\b*(\x)^2+\c*(\x)+\d});
			\draw[dashed] (3,0)node[below]{$30$}--(3,1)node[above]{$B$}--(0,1)node[left]{$10$} (5,0)node[below]{$50$}--(5,3)node[above]{$C$} (8,0)node[below]{$80$}--(8,3)node[above right]{$D$} (0,3)node[left]{$30$} (0,3)node[above right]{$A$}--(8,3);
	\end{tikzpicture}}	
	\shortans[0]{$39{,}9$}
	\loigiai{
		Đồ thị hàm số bậc ba đi qua điểm $A(0;30)$ nên $d=30$. Lại có, đồ thị đi qua điểm $B(30;10)$, $C(50;30)$, $D(80;30)$ nên ta có hệ phương trình sau
		\[\heva{&a\cdot 30^3+b\cdot 30^2+c\cdot30 +30=10 \\&a\cdot 50^3+b\cdot 50^2+c\cdot50 +30=30 \\&a\cdot 80^3+b\cdot 80^2+c\cdot80 +30=30}\Leftrightarrow\heva{&a=-\dfrac{1}{1\,500} \\&b=\dfrac{13}{150} \\&c=-\dfrac{8}{3}.}\]
		Ta có $f(x)=-\dfrac{1}{1\,500}x^3+\dfrac{13}{150}x^2-\dfrac{8}{3}x+30$. Khi đó, $f'(x)=-\dfrac{1}{500}x^2+\dfrac{13}{75}x-\dfrac{8}{3}$.\\
		Phương trình $f'(x)=0\Leftrightarrow -\dfrac{1}{500}x^2+\dfrac{13}{75}x-\dfrac{8}{3}=0\Leftrightarrow\hoac{&x=20 \\&x=\dfrac{200}{3}.}$\\
		Dựa vào đồ thị, ta thấy tàu đạt độ cao lớn nhất so với mặt đất khi đi qua điểm cực đại của đồ thị hàm số $f(x)$. Độ cao lớn nhất khi đó bằng $f\left(\dfrac{200}{3}\right)\approx 39{,}9$ m.}
\end{ex}
\begin{ex}%Câu 6
	\immini[thm]{Một quả bóng hình cầu có bán kính $r$ đang được treo trong một góc tường nhà (hai bờ tường vuông góc), một điểm $B$ cố định nằm trên mép của hai bờ tường và cách mặt đất $80$ cm, sợi dây treo bóng có độ dài $AB=30$ cm và đây cũng là độ dài ngắn nhất nối điểm $B$ với mặt xung quanh của quả bóng.\\
	Biết rằng quả bóng tiếp xúc với hai bên bờ tường và điểm thấp nhất của quả bóng cách mặt đáy $20$ cm. Hỏi đường kính của quả bóng là bao nhiêu centimet (làm tròn kết quả đến hàng đơn vị).
	}
	{\begin{tikzpicture}[scale=1,>=stealth, font=\footnotesize, line join=round, line cap=round]
			\fill[gray!60] (0,0)--(-1.8,-0.52)--(-1.8,3.98)--(0,4.5)--cycle;
			\fill[gray!35] (-1.8,-0.52)--(-2.69,0)--(-2.69,4.5)--(-1.8,3.98)--cycle;
			\fill[gray!35] (0,0)--(0,4.5)--(1.45,3.65)--(1.45,-0.85)--cycle;
			\fill[gray!60] (1.45,-0.85)--(2.36,-0.58)--(2.36,3.92)--(1.45,3.65)--cycle;
			\coordinate (I) at (-0.16,1.58);
			\coordinate (B) at (0,4);
			\coordinate (r) at (-0.69,1.03);
			\coordinate (x) at (-0.16,-0.52);
			\coordinate (L) at (0.29,-0.52);
			%			\tkzInterLC[R](I,B)(I,1.06)\tkzGetPoints{D}{A}
			%			\tkzInterLC[R](I,x)(I,1.06)\tkzGetPoints{E}{F}
			\draw[shading=ball,ball color=gray,opacity=0.8,name path=C1] (I) circle(1.06cm);
			\path[name path=IB] (I)--(B);
			\path[name path=Ix] (I)--(x);
			\path  	[name intersections={of=C1 and IB,by={A}}];	
			\path  	[name intersections={of=C1 and Ix,by={D}}];	
			\fill (I)circle(1.5pt);
			\fill (B)circle(1.5pt);
			\draw[->] (0,0)--(-2.76,-0.79)node[below]{$x$};
			\draw[->] (0,0)--(2.25,-1.32)node[below]{$y$};
			\draw[->] (0,0)--(0,5)node[right]{$z$};
			\draw (A)node[above left,xshift=0.1cm,yshift=-0.1cm]{$A$}--(B)node[left,xshift=0.1cm]{$B$} (0,0)node[right,yshift=0.1cm,xshift=-0.1cm]{$O$} ($(A)!0.5!(B)$)node[left]{$30$ cm} ($(I)!0.5!(r)$)node[above left,xshift=0.2cm]{$r$} (r)node[rotate=30]{$\times$} 
			(x)--(D)
			(L)--($(L)!2!(x)$) ($(I)!0.6!(x)$)node[left]{$20$ cm};
			\draw[->] (I)--(r);
			\draw[dashed] (I)--(A) (D)--(I) (I)--(B);
			\draw pic[draw,angle radius=1mm]{right angle=L--x--I};
			
			%	\foreach \x/\g in {A/90,B/180,I/-45,F/0,x/0,L/0,r/0}
			%			\fill[black] 	(\x) circle (1pt)
			%			($(\g:3mm)+(\x)$) node {$\x$};
	\end{tikzpicture}}	
	
	\shortans[0]{$30$}
	\loigiai{
		Vì $AB$ là độ dài ngắn nhất nối điểm $B(0;0;80)$ với mặt xung quanh của quả bóng nên $AB$ đi qua tâm $I$ của quả bóng.\\
		Thật vậy, với $M$ bất kì nằm trên mặt xung quanh của quả bóng, ta luôn có $BM+MI\geq BI$ hay $BM\geq BI-r$. Do đó, độ dài nhỏ nhất của $BM$ bằng $BI-r$. Điều này xảy ra khi $M\equiv A$, tức là $BA=BI-r$.\\
		Giả sử $H$ là hình chiếu của $I$ trên $(Oyz)$, khi đó, ta có $H(0;r;20+r)$. Do đó, $BH=\sqrt{r^2+(r-60)^2}$.\\
		Tam giác $BIH$ vuông tại $H$ nên theo định lý Py-ta-go, ta có
		\allowdisplaybreaks
		\begin{eqnarray*}
		&&BI^2=BH^2+IH^2\\
		&\Leftrightarrow& (30+r)^2=r^2+(r-60)^2+r^2\\
		&\Leftrightarrow& 2r^2-180r+2700=0\\
		&\Leftrightarrow&\hoac{&r=45-15\sqrt{3}\\& r=45+\sqrt{3} \,(\text{loại}).}
		\end{eqnarray*}
		Vậy đường kính quả bóng là $d=2r=2\cdot (45-\sqrt{3})=90-30\sqrt{3}\approx 38$ cm.
		}
\end{ex}
\Closesolutionfile{ans}
% \begin{name}
	{\tenchude}
	{\tendethi}
	{\tentruong}
	{\thoigian}
	\end{name}
\TN
\Opensolutionfile{ans}[ans/de3-phanI]
\begin{ex}%Câu 1
	Cho hàm số $f(x)=3\cos x$. Khẳng định nào dưới đây đúng?
	\choice
	{$\displaystyle\int f(x) \mathrm{\,d}x=3x\cdot\sin x+C$}
	{$\displaystyle\int f(x) \mathrm{\,d}x=-3\sin x+C$}
	{$\displaystyle\int f(x) \mathrm{\,d}x=3x+\sin x+C$}
	{\True $\displaystyle\int f(x) \mathrm{\,d}x=3\sin x+C$}
	\loigiai{
	Ta có $\displaystyle\int f(x) \mathrm{\,d}x=3\sin x+C$.}
\end{ex}
\begin{ex}%Câu 2
	Cho cấp số cộng $(u_n)$ với $u_1=2$ và công sai $d=3$. Giá trị của $u_5$ bằng
	\choice
	{$162$}
	{\True $14$}
	{$30$}
	{$10$}
	\loigiai{
	Ta có $u_5=u_1+(5-1)\cdot 3=14$.}
\end{ex}
\begin{ex}%Câu 3
	Nghiệm của phương trình $2^{2x+1}=\dfrac{1}{8}$ là
	\choice
	{$x=-1$}
	{$x=2$}
	{$x=1$}
	{\True $x=-2$}
	\loigiai{
	Ta có $2^{2x+1}=\dfrac{1}{8}\Leftrightarrow 2^{2x+1}=2^{-3}\Leftrightarrow 2x+1=-3\Leftrightarrow x=-2$.}
\end{ex}
\begin{ex}%Câu 4
	Cho hình phẳng $(H)$ giới hạn bởi đường cong $y=\sqrt{x+1}$, trục hoành và các đường thẳng $x=-1$, $x=2$. Thể tích $V$ của khối tròn xoay tạo thành khi quay $(H)$ quanh trục hoành được tính bởi công thức nào sau đây?
	\choice[0.3em]
	{$V=\pi\displaystyle\int\limits_{-1}^{2} \sqrt{x^2+1} \mathrm{\,d}x$}
	{$V=\pi^2\displaystyle\int\limits_{-1}^{2} (x+1) \mathrm{\,d}x$}
	{\True $V=\pi\displaystyle\int\limits_{-1}^{2} (x+1) \mathrm{\,d}x$}
	{$V=\pi\displaystyle\int\limits_{-1}^{2} \sqrt{x+1} \mathrm{\,d}x$}
	\loigiai{
	Thể tích khối tròn xoay tính bởi công thức $V=\pi\displaystyle\int\limits_{-1}^{2} (x+1) \mathrm{\,d}x$.}
\end{ex}
\begin{ex}%Câu 5
	Tập xác định của hàm số $y=\ln(\ln x)$ là
	\choice
	{\True $(1;+\infty)$}
	{$(0;+\infty)$}
	{$(\mathrm{e};+\infty)$}
	{$\mathbb{R}$}
	\loigiai{
	Hàm số xác định khi và chỉ khi $\heva{&x>0 \\&\ln x>0}\Leftrightarrow\heva{&x>0 \\& x>1}\Leftrightarrow x>1$.\\
	Vậy tập xác định là $\mathscr{D}=(1;+\infty)$.}
\end{ex}
\textbf{\textit{Sử dụng thông tin dưới đây để trả lời câu \ref{câu 6-đề 3} và câu \ref{câu 7-đề 3}}}\\[0.5em]
Trong không gian với hệ tọa độ $Oxyz$, cho ba điểm $A(1;-2;-1)$, $B(1;0;2)$ và $C(0;2;1)$.
\begin{ex}%Câu 6
	\label{câu 6-đề 3}
	Độ dài vectơ $\overrightarrow{AB}$ bằng
	\choice
	{$5$}
	{$\sqrt{5}$}
	{$\sqrt{14}$}
	{\True $\sqrt{13}$}
	\loigiai{
	Độ dài vectơ $\overrightarrow{AB}$ bằng $\sqrt{(1-1)^2+(0+2)^2+(2+1)^2}=\sqrt{13}$.}
\end{ex}
\begin{ex}%Câu 7
	\label{câu 7-đề 3}
	Mặt phẳng đi qua $A$ và vuông góc với đường thẳng $BC$ có phương trình là
	\choice
	{\True $x-2y+z-4=0$}
	{$x-2y+z+4=0$}
	{$x-2y-z-6=0$}
	{$z-2y-z+4=0$}
	\loigiai{
	Mặt phẳng đi qua $A$ và vuông góc với $BC$ có vectơ pháp tuyến cùng phương với $\overrightarrow{BC}$.\\
	Ta có $\overrightarrow{BC}=(-1;2;-1)$ nên một vectơ pháp tuyến của mặt phẳng là $\overrightarrow{n}=(1;-2;1)$.\\
	Phương trình mặt phẳng là $x-2y+z-4=0$.}
\end{ex}
\begin{ex}%Câu 8
	Thống kê thu nhập theo tháng (đơn vị: triệu đồng) của một nhóm người chạy xe máy Xanh SM được cho trong bảng sau
	\begin{center}
		\begin{tabular}{|c|c|c|c|c|}
			\hline
			Thu nhập (triệu đồng) & $[3;5)$ & $[5;7)$ & $[7;9)$ & $[9;11)$\\
			\hline
			Số người & $5$ & $10$ & $5$ & $2$ \\
			\hline
		\end{tabular}
	\end{center}
	Tứ phân vị thứ ba của mẫu số liệu ghép nhóm trên là 
	\choice
	{\True $7{,}6$}
	{$8{,}1$}
	{$7{,}5$}
	{$8{,}2$}
	\loigiai{
	Mẫu số liệu có $22$ giá trị nên trung vị là trung bình cộng của số đứng thứ $11$ và $12$. Do đó, tứ phân vị thứ ba là số đứng thứ $17$.\\
	Dựa vào bảng số liệu, ta thấy $Q_3\in[7;9)$, do đó $Q_3=7+\dfrac{\dfrac{22\cdot3}{4}-15}{5}\cdot 2=7{,}6$.}
\end{ex}
\renewcommand{\baselinestretch}{1.52}
\begin{ex}%Câu 9
	Hàm số $f(x)=\sqrt{x^2-4}$ đồng biến trên khoảng nào dưới đây?
	\choice
	{$(-\infty;-2)$}
	{\True $(2;+\infty)$}
	{$(0;+\infty)$}
	{$(-2;2)$}
	\loigiai{
	Tập xác định của hàm số là $\mathscr{D}=(-\infty;-2)\cup(2;+\infty)$.\\
	Ta có $f'(x)=\dfrac{x}{\sqrt{x^2-4}}$, $\forall x\in \mathscr{D}$.\\
	Khi đó, $f'(x)\geq 0\Leftrightarrow x\geq 0$.\\
	Kết hợp với tập xác định của hàm số, ta kết luận hàm số đồng biến trên $(2;+\infty)$.}
\end{ex}
\begin{ex}%Câu 10
	Cho hình chóp tứ giác $S.ABCD$, gọi $M$ và $N$ lần lượt là trung điểm của $SA$ và $SC$. Mặt phẳng nào sau đây song song với đường thẳng $MN$?
	\choice
	{$(SAB)$}
	{$(SCD)$}
	{$(SBC)$}
	{\True $(ABCD)$}
	\loigiai{
	\begin{center}
		\begin{tikzpicture}[scale=0.8,>=stealth, font=\footnotesize, line join=round, line cap=round]
		\def\a{4}
		\path 	(0:0) coordinate (A)
		++(0:\a) coordinate (D)
		++(-130:\a/2) coordinate (C)
		++(-165:2*\a/3) coordinate (B)
		($(A)+(70:\a)$) coordinate (S)
		(intersection of A--C and B--D) coordinate (O)
		($(S)!0.5!(A)$) coordinate (M)
		($(S)!0.5!(C)$) coordinate (N);%giao điểm O
		\draw[dashed,thick] 	(C)--(A)--(D) (M)--(N);
		\draw[thick] 			(A)--(B)--(C)--(D)
		(A)--(S)	(B)--(S)	(C)--(S)	(D)--(S);
		\foreach \x/\g in {A/180,B/-135,C/-45,D/0,S/90,M/180,N/0}
		\fill[black] 	(\x) circle (1pt)
		($(\g:3mm)+(\x)$) node {$\x$};	
		\end{tikzpicture}
		\end{center}
	Ta có $MN$ là đường trung bình của tam giác $SAC$ nên $MN\parallel AC$, do đó $MN\parallel (ABCD)$.}
\end{ex}
\begin{ex}%Câu 11
	\immini[thm]
	{
		Cho hình lăng trụ tam giác $ABC.A'B'C'$ (minh họa như hình bên). Đặt $\overrightarrow{AA'}=\overrightarrow{a}$, $\overrightarrow{AB}=\overrightarrow{b}$, $\overrightarrow{AC}=\overrightarrow{c}$. Phát biểu nào sau đây đúng?
	\choice
	{$\overrightarrow{BC'}=-\overrightarrow{a}+\overrightarrow{b}+\overrightarrow{c}$}
	{\True $\overrightarrow{BC'}=\overrightarrow{a}-\overrightarrow{b}+\overrightarrow{c}$}
	{$\overrightarrow{BC'}=\overrightarrow{a}+\overrightarrow{b}+\overrightarrow{c}$}
	{$\overrightarrow{BC'}=\overrightarrow{a}+\overrightarrow{b}-\overrightarrow{c}$}
	}
	{
		\begin{tikzpicture}[scale=0.9,>=stealth, font=\footnotesize, line join=round, line cap=round]
		\def\a{3}
	\def\h{3}
	\path 	(0:0) coordinate (A)
			++(0:\a) coordinate (C)
			(-30:\a/2) coordinate (B)
			($(B)!0.5!(C)$) coordinate (M)
			($(A)!2/3!(M)$) coordinate (G)			
			($(G)+(90:\h)$) coordinate (A')
			($(A')+(C)-(A)$) coordinate (C')
			($(C')+(B)-(C)$) coordinate (B'); 
	\draw[dashed,thick] (A)--(C);
	\draw[thick] (C)--(C') 	(B)--(B') 	(A)--(A') 
				(A)--(B)--(C) (A')--(B')--(C')--cycle;
	\foreach \x/\g in {A/180,B/-45,C/0,A'/180,B'/-45,C'/0}
				\fill[black] 	(\x) circle (1pt)
				($(\g:4mm)+(\x)$) node {$\x$};	
%	\draw pic[draw,angle radius=2mm]{right angle=A'--G--C};%Theo chiều dương	
\end{tikzpicture}
	}
	\loigiai{
	Ta có $\overrightarrow{BC'}=\overrightarrow{AC'}-\overrightarrow{AB}=\overrightarrow{AA'}+\overrightarrow{AC}-\overrightarrow{AB}=\overrightarrow{a}+\overrightarrow{c}-\overrightarrow{b}$.}
\end{ex}
\begin{ex}%Câu 12
	Cho hàm số $f(x)$ liên tục trên $\mathbb{R}$, có bảng xét dấu $f'(x)$ như sau
	\begin{center}
		\begin{tikzpicture}
			\tikzset{double style/.append style={double distance=1.5pt}}
			\tkzTabInit[nocadre=false,lgt=1.2,espcl=2.4,deltacl=0.6]
			{$x$ /0.6,$f'(x)$ /0.6}
			{$-\infty$,$-2$,$0$,$1$,$2$,$+\infty$}
			\tkzTabLine{,+,$0$,-,$0$,-,$0$,+,d,-,}
		\end{tikzpicture}
	\end{center}
	Số điểm cực đại của hàm số $y=f(x)+1$ là
	\choice
	{$4$}
	{\True $2$}
	{$1$}
	{$3$}
	\loigiai{
	Đạo hàm đổi dấu khi đi qua $x=-2$ và $x=2$ và hàm số liên tục trên $\mathbb{R}$ nên $f(x)$ có $2$ điểm cực đại.}
\end{ex}
\Closesolutionfile{ans}
\TNTF
% \setcounter{ex}{0}% Reset lại số đếm câu hỏi
\Opensolutionfile{ans}[ans/de3-phanII]
\begin{ex}%Câu 1
	Cho hàm số $f(x)=\dfrac{x^2-x+2}{x+1}$. Xét tính đúng sai của các mệnh đề sau
	\choiceTF
	{\True Tiệm cận đứng của đồ thị hàm số đã cho là $x=-1$}
	{\True Tiệm cận xiên của đồ thị hàm số đã cho có hệ số góc bằng $1$}
	{\True Tiệm cận xiên của đồ thị hàm số đã cho đi qua điểm có tọa độ là $(0;-2)$}
	{Hai đường tiệm cận của đồ thị hàm số đã cho cùng với hai trục tọa độ $Ox$, $Oy$ tạo thành một đa giác có diện tích bằng $3$}
	\loigiai{
	\begin{itemchoice}
		\itemch \textbf{Đúng.}\\
		Ta có $x+1=0\Leftrightarrow x=-1$ và $x=-1$ không là nghiệm của tử thức nên hàm số có một tiệm cận đứng là $x=-1$.
		\itemch \textbf{Đúng.}\\
		Giả sử tiệm cận xiên của đồ thị hàm số là $y=ax+b$. Khi đó, ta có
		\[a=\lim\limits_{x\to+\infty}\dfrac{x^2-x+2}{x(x+1)}=1.\]
		\itemch \textbf{Đúng.}\\
		Ta có $b=\displaystyle\lim_{x\to+\infty}\left(\dfrac{x^2-x+2}{x+1}-x\right)=\lim_{x\to+\infty}\dfrac{-2x+2}{x+1}=-2$.\\
		Vậy tiệm cận xiên của đồ thị hàm số là $y=x-2$. Đường thẳng này đi qua điểm có tọa độ $(0;-2)$.
		\itemch \textbf{Sai.}
		\begin{center}
			\begin{tikzpicture}[line cap=butt,line join=miter,>=stealth,xscale=0.62,yscale=0.62]
				\tikzset{declare function={xmin=-5;xmax=3;Xkxd=-1;
						ymin=-7;ymax=1;
						a=1; b=-1; c=2; d=1;et=1;
						akxd=1;bkxd=-2;
						f(\x)=(a*(\x)^2+b*\x+c)/(d*(\x)+et);
						h(\x)=akxd*\x+bkxd;
					},
					smooth,samples=50
				}
				\draw[->] (xmin-0.25,0)--(xmax+0.5,0)
				node[shift={(-95:7pt)},font=\normalsize]{$ x $};
				\draw[->] (0,ymin-0.25)--(0,ymax+0.5)
				node[shift={(5:7pt)},font=\normalsize]{$ y $};
				\fill (0,0) node[shift={(45:9pt)},font=\normalsize]{$ O $};
				\foreach \x in {-5, -4, -3, -2, -1, 1, 2, 3}{
					\draw (\x,2pt)--(\x,-2pt) +(0,-9pt) node[font=\scriptsize,fill=white,inner sep=1pt]{$\x$};
				}
				\foreach \y in {-7, -6, -5, -4, -3, -2, -1, 1}{
					\draw (2pt,\y)--(-2pt,\y) +(-3pt,0) node[font=\scriptsize,anchor=east,fill=white,inner sep=1pt]{$\y$};
				}
				\begin{scope}[thick]
					\clip (xmin,ymin) rectangle (xmax,ymax);
					\draw[blue!50!black] (Xkxd,ymin)--(Xkxd,ymax);
					\draw[blue!50!black] plot[domain=xmin:xmax] (\x, {h(\x)});
					\draw[cyan!85!blue] plot[domain=xmin:{Xkxd-0.01}] (\x, {f(\x)});
					\draw[cyan!85!blue] plot[domain={Xkxd+0.01}:xmax] (\x, {f(\x)});
				\end{scope}
				\node [left] at (-1,-3) {$B$}; 
				\node [ above left] at (-1,0) {$A$};
				\node [right] at (0,-2) {$C$};  
			\end{tikzpicture}
		\end{center}
		Tứ giác cần tìm là hình thang vuông $OABC$, ta có $S_{OABC}=\dfrac{(OC+AB)\cdot OA}{2}=\dfrac{5\cdot 2}{2}=5$.
	\end{itemchoice}}
\end{ex}
\begin{ex}%Câu 2
	Một con sư tử đang đuổi theo một con ngựa vằn. Con ngựa vằn nhận ra con sư tử khi con sư tử cách xa nó $40$ m. Từ thời điểm này, con sư tử đuổi con ngựa vằn với tốc độ $v_1(t)=15\mathrm{e}^{-0{,}1t}$ m/s và con ngựa vằn chạy trốn với tốc độ $v_2(t)=20-20\mathrm{e}^{-0{,}1t}$ m/s trên cùng một đường thẳng (với $t$ tính theo giây và $0\le t\le 60$). Xét tính đúng sai của các mệnh đề sau
	\choiceTF
	{Tại thời điểm $t=0$ vận tốc của con ngựa vằn là $20$ m/s}
	{\True Tốc độ của sư tử giảm dần theo thời gian trong khi tốc độ của ngựa vằn tăng dần theo thời gian}
	{Sư tử sẽ ở gần với ngựa vằn nhất khi $v'_1(t)=v'_2(t)$}
	{Sư tử sẽ không bắt được con ngựa vằn và khoảng cách ngắn nhất giữa chúng là $1{,}42$ mét (kết quả làm tròn đến hàng phần trăm)}
	\loigiai{
	\begin{itemchoice}
		\itemch \textbf{Sai.}\\
		Tại $t=0$, ta có $v_2(0)=20-20=0$ m/s.
		\itemch \textbf{Đúng.}\\
		Ta có $v'_1(t)=-0{,}1\cdot 15\mathrm{e}^{-0{,}1t}<0$, $\forall 0\le t\le 60$. Do đó, tốc độ của sử tử giảm dần theo thời gian.\\
		Lại có $v'_2(t)=0{,}1\cdot 20\mathrm{e}^{-0{,}1t}>0$, $\forall 0\le t\le 60$. Do đó, tốc độ của ngựa vằn tăng dần theo thời gian.
		\itemch \textbf{Sai.}\\
		Ta có $v'_1(t)=v'_2(t)\Leftrightarrow 15\mathrm{e}^{-0{,}1t}=-20\mathrm{e}^{-0{,}1t}$. Phương trình này vô nghiệm.
		\itemch \textbf{Sai.}\\
		Chọn hệ quy chiếu với gốc của chuyển động ở vị trí con sư tử bắt đầu đuổi con ngựa vằn.\\
		Gọi $s_1(t)$ là quãng đường con sư tử di chuyển được trong $t$ giây, $s_2(t)$ là quãng đường con ngựa vằn di chuyển được trong $t$ giây.\\
		Ta có $s_1(t)=\displaystyle\int v_1(t)\mathrm{\,d}t=-150\mathrm{e}^{-0{,}1t}+C$.\\
		Tại $t=0$, ta có $s_1(0)=0\Leftrightarrow -150+C=0\Leftrightarrow C=150$. Do đó, $s_2(t)=-150\mathrm{e}^{-0{,}1t}+150$.\\
		Ta có $s_2(t)=\displaystyle\int v_2(t)\mathrm{\,d}t=20t+200\mathrm{e}^{-0{,}1t}+C$.\\
		Tại $t=0$, ta có $s_2(0)=40\Leftrightarrow 200+C=40\Leftrightarrow C=-160$. Do đó, $s_1(t)=20t+200\mathrm{e}^{-0{,}1t}-160$.\\
		Sư tử sẽ ở gần ngựa vằn nhất khi $s_2(t)-s_1(t)$ đạt $\min$. Ta có
		\[s_2(t)-s_1(t)=f(t)=20t+350\mathrm{e}^{-0{,}1t}-310,\forall 0\le t\le 60.\]
		Lại có $f'(t)=20-35\mathrm{e}^{-0{,}1t}$, $\forall 0\le t\le 60$. Khi đó, $f'(t)=0\Leftrightarrow \mathrm{e}^{-0{,}1t}=\dfrac{4}{7}\Leftrightarrow t\approx 5{,}6$ s.\\
		Ta có bảng biến thiên
		\begin{center}
			\begin{tikzpicture}
			\tkzTabInit[nocadre=false,lgt=1.2,espcl=2.5,deltacl=0.6]
			{$t$/0.6,$f'(t) $/0.6,$f(t)$/2}
			{$0$,$5{,}6$,$60$}
			\tkzTabLine{,-,0,+,} %
			\tkzTabVar{+/$40$,-/$1{,}92$, +/$890{,}1$} %dấu mũi tên, + trên, -dưới
		\end{tikzpicture}
		\end{center}
		Vậy khoảng cách ngắn nhất là $1{,}92$ m.
	\end{itemchoice}}
\end{ex}
\renewcommand{\baselinestretch}{1.5}
\begin{ex}%Câu 3
	Một vật dụng bằng sắt đang nằm trên mặt sàn có tay cầm dài $58$ cm nối với một ống trụ dày $4$ cm và có đường kính đáy bằng $30$ cm. Nếu không giữ thì sẽ luôn có một lực làm vật rung động, để vật đứng yên thì người ta đã nối một đoạn dây từ điểm $B$ (là một điểm nằm trên đường tròn chính giữa của ống trụ to) đến điểm $C$ nằm trên bờ tường. Trên hệ trục $Oxyz$, xét gốc tọa độ là điểm gắn ống trụ với bờ tường, bờ tường là mặt phẳng $(Oxz)$, trục $Oy$ là trục của hình trụ, điểm $A$ nằm chính giữa ống trụ to, điểm $B$ có hoành độ âm, cao độ dương và $AB$ tạo với trục $Oz$ một góc $30^\circ$, các số liệu được cho như hình vẽ, đơn vị trên các hệ trục tính theo cm. Biết rằng lực căng $\overrightarrow{T}$ trên đoạn dây $BC$ có độ lớn bằng $500$ N. Xét tính đúng sai của các mệnh đề sau
	\begin{center}
		\begin{tikzpicture}[scale=1,>=stealth, font=\footnotesize, line join=round, line cap=round]
			\fill[gray!20] (-0.58,-0.85)--(-0.58,3.38)--(4.01,5.22)--(4.01,0.98)--cycle;
			\coordinate (O) at (0,0);
			\coordinate (y) at (5.27,-2.17);
			\coordinate (x) at (4.48,1.79);
			\coordinate (A) at ($(O)!0.7!(y)$);
			\coordinate (E) at (2.75,-0.9);
			\path 
			
			;
			\tkzDefLine[parallel=through E](O,y) \tkzGetPoint{d}
			\tkzInterLL(O,x)(E,d)\tkzGetPoint{E'}
			\draw (E)--(E');
			\coordinate (F) at (2.6,-1.3);
			\tkzDefLine[parallel=through F](O,y) \tkzGetPoint{d'}
			\tkzInterLL(O,x)(F,d')\tkzGetPoint{F'}
			\draw (F)--(F');
			\fill[gray!40] (E)--(E')--(E')..controls+(150:0.5) and +(50:0.1)..(F')--(F)--cycle;
			\begin{scope}[rotate=-20]
				\draw[fill=gray!80] (A) ellipse (0.8 cm and 1.38 cm);
				\coordinate (M) at ($(A)+(90:0.8 cm and 1.38 cm)$);
				\coordinate (M') at ($(M)+(-0.3,0)$);
				\coordinate (N) at ($(A)+(-90:0.8 cm and 1.38 cm)$);
				\coordinate (N') at ($(N)+(-0.3,0)$);
				\draw (M)--(M');
				\draw (N)--(N');
				\draw (M') arc(90:270:0.8 cm and 1.38 cm);
				\fill[gray!40] (M)--(M') arc(90:270:0.8 cm and 1.38 cm)--(N)--(N) arc(270:90:0.8 cm and 1.38 cm)--cycle;
			\end{scope}
			\coordinate (B) at (3.05,-0.7);
			\coordinate (C) at (2.58,3.7);
			\coordinate (I) at ($(A)+(0,2)$);
			\fill (B)node[above left]{$B$}circle(2pt);
			\draw (1.64,-2.21)--(A)--(B) (0.26,-1.33)node[below left]{$60$ cm} (5,-1.29)node[below right]{$15$ cm} (4.47,-0.31)--(5.56,-0.76) ($(B)!0.7!(C)$)--(C)--($(C)+(0,0.5)$) (1.27,3.64)node[above]{$35$ cm} (C)--(3.84,4.21) (3.39,2.69)node[right]{$30$ cm} ($(B)!0.5!(C)$)node[left]{$\overrightarrow{T}$} (A)--(I);
			\draw[->,dashed] (-1.88,-0.75)--(4.48,1.79)node[below]{$x$};
			\draw[->,dashed] (O)node[below left,xshift=0.2cm,yshift=-0.1cm]{$O$}--(0,4)node[right]{$z$};
			\draw[->,dashed] (O)--(y)node[below]{$y$};
			\fill (O)circle(2pt);
			\fill (A)node[above right]{$A$}circle(2pt);
			\fill (C)node[above right,yshift=0.1cm]{$C$}circle(2pt);
			\draw[<->] (-1.51,-0.6)--(2.02,-2.06);
			\draw[<->] (4.7,-1.94)--(5.3,-0.65);
			\draw[<->] (0,3.12)--(2.58,4.16);
			\draw[<->] (3.39,4.03)--(3.39,1.36);
			\draw[->] (B)--($(B)!0.7!(C)$);
			\draw pic[draw,,angle radius=4mm]{angle=I--A--B};
			\draw pic["$30^\circ$",-stealth,angle radius=11mm]{angle=I--A--B};
		\end{tikzpicture}
	\end{center}
	\choiceTF
	{\True Hình chiếu của $C$ lên mặt phẳng $(Oxy)$ có tọa độ $(35;0;0)$}
	{Góc giữa đường thẳng $AB$ và mặt phẳng $(Oxy)$ bằng $30^\circ$}
	{Vectơ $\overrightarrow{BC}$ có tọa độ $(a;b;c)$. Khi đó $2a-b=40$}
	{Vectơ lực tác dụng lên đoạn dây $BC$ có hoành độ là $120$ (làm tròn kết quả đến hàng đơn vị theo Newton)}
	\loigiai{
	\begin{itemchoice}
		\itemch \textbf{Đúng.}\\
		Điểm $C$ có tọa độ $(35;0;30)$ nên hình chiếu của $C$ lên mặt phẳng $(Oxy)$ có tọa độ $(35;0;0)$.
		\itemch \textbf{Sai.}\\
		Ta có $(AB;Oz)=30^\circ$ mà $Oz\perp (Oxy)$ nên $(AB;(Oxy))=60^\circ$.
		\itemch \textbf{Sai.}\\
		Vì đường kính đáy bằng $30$ nên bán kính đáy bằng $15$.\\
		Ta có $|x_B|=AB\cdot\cos 60^\circ=7{,}5$. Mà $x_B<0$ nên $x_B=-7{,}5$.\\
		Lại có $|z_B|=AB\cdot \cos 30^\circ=\dfrac{15\sqrt{3}}{2}$ mà $z_B>0$ nên $z_B=\dfrac{15\sqrt{3}}{2}$. Do đó, $B=\left(-7{,}5;60;\dfrac{15\sqrt{3}}{2}\right)$.
		Khi đó, $\overrightarrow{BC}=\left(42{,}5;-60;30-\dfrac{15\sqrt{3}}{2}\right)$. Suy ra $2a-b=2\cdot42{,}5+60=145$.
		\itemch \textbf{Sai.}\\
		Ta có $\left|\overrightarrow{BC}\right|=\sqrt{(42{,}5)^2+60^2+\left(30-\dfrac{15\sqrt{3}}{2}\right)}\approx 75$.\\
		Suy ra $\overrightarrow{T}=\dfrac{500}{75}\overrightarrow{BC}$ nên hoành độ của $\overrightarrow{T}$ bằng $\dfrac{500}{75}\cdot 42{,}5\approx 283$.
	\end{itemchoice}}
\end{ex}
\begin{ex}%Câu 4
	Trong một khu dân cư, tỉ lệ người nghiện thuốc lá và và bị ung thư vòm họng là $15\%$. Có $25\%$ người nghiện thuốc lá nhưng không bị ung thư vòm họng, $50\%$ người không nghiện thuốc lá và không bị ung thư họng và có $10\%$ số người không nghiện thuốc lá nhưng mắc ung thư vòm họng.
	\begin{itemize}
		\item Gọi $A$ là biến cố \lq\lq người đó nghiện thuốc lá\rq\rq.
		\item Gọi $B$ là biến cố \lq\lq người đó bị ung thư vòm họng\rq\rq.
	\end{itemize}
	Xét tính đúng sai của các mệnh đề sau
	\choiceTF
	{$\mathrm{P}(AB)=0{,}25$ và $\mathrm{P}(\overline{A}B)=0{,}15$}
	{$\mathrm{P}(A)=0{,}6$}
	{\True $\mathrm{P}(B\mid A)=0{,}375$}
	{\True Với những dữ liệu thống kê như trên có thể thấy nguy cơ của người nghiện thuốc là mắc ung thư vòm họng gấp $2{,}25$ lần người không nghiện thuốc lá}
	\loigiai{
	\begin{itemchoice}
		\itemch \textbf{Sai.}\\
		Ta có $\mathrm{P}(AB)=0{,}15$, $\mathrm{P}(\overline{A}B)=0{,}1$.
		\itemch \textbf{Sai.}\\
		Ta có $\mathrm{P}(A)=\mathrm{P}(AB)+\mathrm{P}(A\overline{B})$.\\
		Theo đề, ta có $\mathrm{P}(A\overline{B})=0{,}25$, $\mathrm{P}(\overline{A}\cap\overline{B})=0{,}5$, $\mathrm{P}(\overline{A}B)=0{,}1$.\\
		Do đó, $\mathrm{P}(A)=0{,}15+0{,}25=0{,}4$.
		\itemch \textbf{Đúng.}\\
		Ta có $\mathrm{P}(B\mid A)=\dfrac{\mathrm{P}(AB)}{\mathrm{P}(A)}=\dfrac{0{,}15}{0{,}4}=0{,}375$.
		\itemch \textbf{Đúng.}\\
		Ta có $\mathrm{P}(B\mid \overline{A})=\dfrac{\mathrm{P}(B\overline{A})}{\mathrm{P}(\overline{A})}=\dfrac{0{,}1}{0{,}6}=\dfrac{1}{6}$.\\
		Do đó $\mathrm{P}(B\mid A)=2{,}25\cdot\mathrm{P}(B\mid\overline{A})$.
	\end{itemchoice}}
\end{ex}
\Closesolutionfile{ans}
\TNSA
% \setcounter{ex}{0}% Reset lại số đếm câu hỏi
\Opensolutionfile{ans}[ans/de3-phanIII]
\begin{ex}%Câu 1
	Cho hình chóp $S.ABC$ có cạnh bên $SA$ vuông góc với mặt phẳng $(ABC)$ và $AB=1$, $AC=2$, $\widehat{BAC}=60^\circ$. Khoảng cách giữa hai đường thẳng $SA$ và $BC$ bằng bao nhiêu?
	
	\shortans[]{$1$}
	\loigiai{
	\begin{center}
		\begin{tikzpicture}
		\def\a{4} %Khai báo cạnh
		\def\h{4}
		\path 	(0:0) coordinate (A)
		++(0:\a) coordinate (C)
		++(-150:4*\a/5) coordinate (B)
		($(A)+(90:\h)$) coordinate (S)
		($(B)!0.3!(C)$) coordinate (H);
		\draw[thick] 	(A)--(B)--(C)
		(A)--(S)--(H)	(C)--(S)	(B)--(S);
		\draw[dashed,thick] 	(H)--(A)--(C);
		\foreach \x /\goc in {A/180,C/0,B/-135,S/90,H/-45}
		\fill[black] (\x) circle (1.5pt)
		($(\x)+(\goc:3mm)$) node {$\x$};
		\draw pic[draw,angle radius=2mm]{right angle=C--A--S};%Theo chiều dương
		\draw pic[draw,angle radius=2mm]{right angle=A--H--C};
		\end{tikzpicture}
		\end{center}
	Kẻ $AH\perp BC$ ($H\in BC$).\\
	Ta có $SA\perp(ABC)$ nên $SA\perp AH$. Do đó, $AH$ là đoạn vuông góc chung của $SA$ và $BC$.\\
	Suy ra $\mathrm{d}(SA,BC)=AH$.\\
	Lại có $BC=\sqrt{AB^2+AC^2-2AB\cdot AC\cdot \cos \widehat{BAC}}=\sqrt{1+4-2\cdot1\cdot2\cdot\cos 60^\circ}=\sqrt{3}$.\\
	Mà $S_{ABC}=\dfrac{1}{2}AH\cdot BC=\dfrac{1}{2}AB\cdot AC\sin\widehat{BAC}$ nên $AH=\dfrac{AB\cdot AC\sin\widehat{BAC}}{BC}=1$.}
\end{ex}
\begin{ex}%Câu 2
	Trong một vườn cây ăn trái, có ba loại cây: cây cam, cây chanh và cây bưởi. Sau $3$ năm, số cây cam tăng gấp ba lần, số cây chanh tăng gấp hai lần và cây bưởi tăng gấp bốn lần số lượng cây ban đầu. Tổng số cây sau $3$ năm là $330$ cây. Biết rằng ban đầu số lượng cây bưởi bằng trung bình cộng của số lượng cây cam và cây chanh. Sau $3$ năm thu hoạch, tổng số cây cam và cây chanh tăng thêm nhiều hơn $15$ cây so với số cây bưởi tăng thêm. Vậy tổng số cây cam và cây bưởi ban đầu là bao nhiêu?
	
	\shortans[]{$85$}
	\loigiai{
	Gọi $x$ (cây) là số cây cam ban đầu, $y$ (cây) là số cây bưởi ban đầu, $z$ (cây) là số cây chanh ban đầu.\\
	Theo đề, ta có $\heva{&3x+4y+2z=330 \\&y=\dfrac{x+z}{2} \\&2x+z-3y=15}\Leftrightarrow\heva{&x=50 \\&y=35 \\&z=20.}$\\
	Vậy tổng số cây cam và cây bưởi ban đầu là $50+35=85$ cây.}
\end{ex}
\begin{ex}%Câu 3
	Có hai bình như sau: Bình $A$ chứa $5$ bi đỏ, $3$ bi trắng và $8$ bi xanh; bình $B$ chứa $3$ bi đỏ và $5$ bi trắng. Gieo một con xúc xắc ngẫu nhiên: Nếu mặt $3$ hoặc mặt $5$ xuất hiện thì chọn ngẫu nhiên một bi từ bình $B$; các trường hợp khác thì chọn ngẫu nhiên một bi từ bình $A$. Nếu viên bi trắng được chọn ra, hãy tính xác suất để mặt $5$ của con xúc xắc xuất hiện nhiều nhất.
	
	\shortans[]{$0{,}31$}
	\loigiai{
	Gọi $A$ là biến cố \lq\lq Bi được chọn là bi trắng\rq\rq, $B$ là biến cố \lq\lq Xúc xắc xuất hiện mặt 5\rq\rq.\\
	Ta cần tính $\mathrm{P}(B\mid A)$. Theo công thức xác suất Bayes, ta có $\mathrm{P}(B\mid A)=\dfrac{\mathrm{P}(B)\cdot\mathrm{P}(A\mid B)}{\mathrm{P}(A)}$.\\
	Theo đề, ta có $\mathrm{P}(B)=\dfrac{1}{6}$, $\mathrm{P}(A\mid B)=\dfrac{5}{8}$.\\
	Lại có $\mathrm{P}(A)=\mathrm{P}(AB)+\mathrm{P}(A\overline{B})$.\\
	Nếu xúc xắc đổ ra số $3$, khi đó xác suất bốc trúng bi trắng là $\dfrac{5}{8}$.\\
	Nếu xúc xắc đổ ra các số còn lại (ngoại trừ số $5$) thì xác suất bốc trúng bi trắng là $\dfrac{3}{16}$.\\
	Do đó, $\mathrm{P}(A\overline{B})=\dfrac{1}{6}\cdot\dfrac{5}{8}+\dfrac{2}{3}\cdot\dfrac{3}{16}=\dfrac{11}{48}$. Suy ra $\mathrm{P}(A)=\dfrac{1}{6}\cdot\dfrac{5}{8}+\dfrac{11}{48}=\dfrac{1}{3}$.\\
	Vậy $\mathrm{P}(B\mid A)=\dfrac{\dfrac{1}{6}\cdot\dfrac{5}{8}}{\dfrac{1}{3}}=\dfrac{5}{16}=0{,}3125$.
	}
\end{ex}
\begin{ex}%Câu 4
	Cần trục chân đế là kiểu cột quay được sử dụng để phục vụ công việc xếp dỡ hàng hóa chủ yếu ngoài các cảng, bến, bãi (như hình minh họa).
	
	{\centering \begin{tikzpicture}[scale=0.7,>=stealth, font=\footnotesize, line join=round, line cap=round]
			\draw (0,0)node[opacity=0.65]{\includegraphics[scale=0.21]{images/de3-1}};
			\coordinate (K) at (1.85,1.3);
			\coordinate (M) at (-3,2.9);
			\coordinate (O) at (2.75,-3.15);
			\fill (K)node[above right,font=\bfseries]{$K$}circle(3pt);
			\fill (M)node[above,font=\bfseries]{$M$}circle(3pt);
			\draw[->,thick] (O)--($(O)!1.4!(K)$)node[right,font=\bfseries]{$z$};
			\draw[->,thick] (O)--(3.7,-2.85)node[above,font=\bfseries]{$x$};
			\draw[->,thick] (O)--($(O)+(-6.5,0)$)node[above,font=\bfseries]{$y$};
	\end{tikzpicture}\par}\noindent
	Ta chọn hệ trục $Oxyz$ thỏa mãn $(Oxy)$ song song với mặt đất, trục $Ox$ trùng với trục chân đế, trục $Oz$ trùng với trục cần cẩu và trục $Oy$ như hình vẽ. Gọi $M$ là vị trí tại đỉnh cần cẩu, $H$ là hình chiếu của $M$ lên $(Oxy)$. Biết tay cần $KM$ của cần trục dài $50$ m, trục cần $OK$ dài $50$ m, $\left(\overrightarrow{k},\overrightarrow{KM}\right)=60^\circ$; $\left(\overrightarrow{i},\overrightarrow{OH}\right)=45^\circ$. Biết điểm $M$ có tọa độ $M(a;b;c)$ trong hệ tọa độ $Oxyz$ trên, giá trị của $a+b+c$ bằng bao nhiêu? (làm tròn kết quả đến hàng đơn vị).
	
	\shortans[]{$136$}
	\loigiai{
	Theo đề, ta có $c=OK+KM\cdot\cos 60^\circ=\dfrac{50}{2}=75$, $a=x_H$, $b=y_H$.\\
	Vì $\left(\overrightarrow{i},\overrightarrow{OH}\right)=45^\circ$ và $OH=MK\cdot\sin 60^\circ=25\sqrt{3}$ nên $x_H=y_H=\dfrac{OH}{\sqrt{2}}=\dfrac{25\sqrt{3}}{\sqrt{2}}$.\\
	Vậy $M\left(\dfrac{25\sqrt{3}}{\sqrt{2}};\dfrac{25\sqrt{3}}{\sqrt{2}};75\right)$ nên $a+b+c=25\sqrt{6}+75\approx 136$.}
\end{ex}
\begin{ex}%Câu 5
	\immini[thm]
	{
		Hai hình chữ nhật bằng nhau, nội tiếp trong đường tròn tâm $O$, bán kính $r=1$ cm tạo thành một hình chữ thập đối xứng (như hình vẽ bên). Diện tích lớn nhất của hình chữ thập là bao nhiêu cm$^2$? (Kết quả làm tròn đến hàng phần trăm).
		
	\shortans[]{$2{,}47$}
	}
	{
		\begin{tikzpicture}[scale=0.8,>=stealth, font=\footnotesize, line join=round, line cap=round]
			\coordinate (O) at (0,0);
			\coordinate (A) at (-1,2);
			\coordinate (B) at (1,2);
			\coordinate (C) at (1,1);
			\coordinate (D) at (2,1);
			\coordinate (E) at (2,-1);
			\coordinate (F) at (1,-1);
			\coordinate (G) at (1,-2);
			\coordinate (H) at (-1,-2);
			\coordinate (I) at (-1,-1);
			\coordinate (J) at (-2,-1);
			\coordinate (K) at (-2,1);
			\coordinate (L) at (-1,1);
			\draw (O) circle(2.236 cm);
			\draw (A)--(B)--(C)--(D)--(E)--(F)--(G)--(H)--(I)--(J)--(K)--(L)--cycle;
			\fill[color=gray!80,opacity=0.5pt] (A)--(B)--(C)--(D)--(E)--(F)--(G)--(H)--(I)--(J)--(K)--(L)--cycle;
			\draw (O)--(D) ($(O)!0.5!(D)$)node[below]{$r$};
			\fill (O)node[below]{$O$}circle(2pt);
			
		\end{tikzpicture}
	}
	\loigiai{
	\begin{center}
		\begin{tikzpicture}[scale=0.8,>=stealth, font=\footnotesize, line join=round, line cap=round]
		\coordinate (O) at (0,0);
		\coordinate (A) at (-1,2);
		\coordinate (B) at (1,2);
		\coordinate (C) at (1,1);
		\coordinate (D) at (2,1);
		\coordinate (E) at (2,-1);
		\coordinate (F) at (1,-1);
		\coordinate (G) at (1,-2);
		\coordinate (H) at (-1,-2);
		\coordinate (I) at (-1,-1);
		\coordinate (J) at (-2,-1);
		\coordinate (K) at (-2,1);
		\coordinate (L) at (-1,1);
		\draw (O) circle(2.236 cm);
		\draw (A)--(B)--(C)--(D)--(E)--(F)--(G)--(H)--(I)--(J)--(K)--(L)--cycle;
		\fill[color=gray!80,opacity=0.5pt] (A)--(B)--(C)--(D)--(E)--(F)--(G)--(H)--(I)--(J)--(K)--(L)--cycle;
		\draw (O)--(D) ($(O)!0.5!(D)$)node[left]{$r$};
		\fill (O)node[below]{$O$}circle(2pt);
		\foreach \x /\goc in {A/160,C/45,B/45,D/0,E/-45,F/-45,I/-135,G/-30,H/-120,J/180,K/180,L/135}
		\fill[black] (\x) circle (1.5pt)
		($(\x)+(\goc:3mm)$) node {$\x$};
		\draw[dashed] (O)--(2,0) (L)--(C)--(F)--(I)--cycle;
	\end{tikzpicture}
	\end{center}
	Đặt $DE=AB=KJ=HG=x$ (cm). Ta có $\dfrac{JE}{2}=\sqrt{1-\dfrac{x^2}{4}}\Rightarrow JE=\sqrt{4-x^2}$.\\
	Khi đó, diện tích hình chữ thập bằng \[S_{KDEJ}+2S_{ABCL}=x\sqrt{4-x^2}+x\left(\sqrt{4-x^2}-x\right)=-x^2+2x\sqrt{4-x^2}.\]
	Xét $f(x)=-x^2+2x\sqrt{4-x^2}$, với $0<x<2$.\\
	Ta có $f'(x)=-2x+2\sqrt{4-x^2}-\dfrac{2x^2}{\sqrt{4-x^2}}$, với $0<x<2$.\\
	Khi đó $f'(x)=0\Leftrightarrow x+\dfrac{x^2}{\sqrt{4-x^2}}=\sqrt{4-x^2}\Leftrightarrow x\sqrt{4-x^2}=4-2x^2\Leftrightarrow x\approx1{,}05$.\\
	Do đó, $\displaystyle\max_{(0;2)}f(x)=f(1{,}05)\approx 2{,}47$.
	}
\end{ex}
\begin{ex}%[1H8V7-9]%[TEX ĐỀ MOON 2025]%[Huỳnh Thanh Chí]
	Người ta cần trang trí một kim tự tháp hình chóp tứ giác đều $S.ABCD$ có cạnh bên bằng $200$ m, góc $\widehat{ASB}=15^\circ$ bằng đường gấp khúc dây đèn led vong quanh kim tự tháp $AEFGHIJKLS$. Trong đó điểm $L$ cố định và $LS=40$ m.
	\begin{center}
		\begin{tikzpicture}[scale=1,>=stealth, font=\footnotesize, line join=round, line cap=round]
			\coordinate (A) at (-1.9,-1.6);
			\coordinate (B) at (0,0);
			\coordinate (D) at (1.6,-1.6);
			\coordinate (C) at ($(B)+(D)-(A)$);
			\coordinate (O) at ($(A)!1/2!(C)$);
			\coordinate (S) at ($(O)+(0,4)$);
			\coordinate (L) at ($(S)!0.2!(A)$);
			\coordinate (K) at ($(S)!0.28!(D)$);
			\coordinate (J) at ($(S)!0.4!(C)$);
			\coordinate (I) at ($(S)!0.65!(B)$);
			\coordinate (H) at ($(S)!0.45!(A)$);
			\coordinate (G) at ($(S)!0.55!(D)$);
			\coordinate (F) at ($(S)!0.7!(C)$);
			\coordinate (E) at ($(S)!0.8!(B)$);
			\draw (S)--(A)--(D)--(C)--cycle (S)--(D) (F)--(G)--(H) (J)--(K)--(L);
			\draw[dashed] (A)--(B)--(C) (S)--(B) (A)--(E)--(F) (H)--(I)--(J);
			\foreach \x/\g in {S/90,A/-150,B/-60,C/0,D/-45,E/170,F/45,G/-30,H/170,I/140,J/30,K/60,L/170}
			\fill[black] (\x) circle (1pt) ($(\g:3mm)+(\x)$) node {$\x$};
		\end{tikzpicture}
	\end{center}
	Hỏi khi đó cần dùng ít nhất bao nhiêu mét dây đèn led để trăng trí? (làm tròn đến hàng đơn vị).
	
	\shortans[]{$263$}
	\loigiai{
		Ta trải hình chóp tứ giác đều thành vẽ như sau
		\begin{center}
			\begin{tikzpicture}[scale=1.5,font=\footnotesize,line join=round,line cap=round,>=stealth]
				\def\a{4}
				\def\r{15}
				\path 
				(0,0) coordinate (S)
				(-135:\a) coordinate (A)
				(-135+\r:\a) coordinate (D)
				($(A)!1!-90:(D)$) coordinate (B_2)
				($(D)!1!90:(A)$) coordinate (C_2)
				;
				\foreach \x/\i in {C/2,B/3,A_1/4,D_1/5,C_1/6,B_1/7,A_2/8}{
					\path 
					(-135+\i*\r:\a) coordinate (\x)
					;
					\draw (S)--(\x);	
				}
				\foreach \x/\y/\i in {A/L/1,D/K/2,C/J/3,B/I/4,A_1/H/5,D_1/G/6,C_1/F/7,B_1/E/8,
					A_1/H_1/7,B/I_1/6,C/J_1/5,D/K_1/4
				}{
					\path 
					($(S)!1/9*\i!(\x)$) coordinate (\y)
					;}
				\draw (S)--(D)--(C_2)--(B_2)--(A)--(S) (A)--(D)--(C)--(B)--(A_1)--(D_1)--(C_1)--(B_1)--(A_2)
				(L)--(K)--(J)--(I)--(H)--(G)--(F)--(E)--(A_2)--(L)
				%	(K1)--(J1)--(I1)--(H1)
				;
				
				\foreach \x/\g in {A/135,D/-65,S/180,C/-90,B/-90,A_1/-90,D_1/-90,C_1/-60,B_1/-45,A_2/0,C_2/-135,B_2/-90}
				\fill 	(\x) circle (1pt)
				($(\g:3mm)+(\x)$) node {$\x$};
				\foreach \x/\g in {L/135,K/-90,J/-90,I/-90,H/-90,G/-90,F/-90,E/-90}
				\fill 	(\x) circle (1pt)
				($(\g:3mm)+(\x)$) node {$\x$};
			\end{tikzpicture}
		\end{center}
		Ta có $T=SL+LK+KJ+\ldots+EA_2\ge SL+LA_2$ (vì $SL$ không đổi).\\
		Để sợi dây trang trí ngắn nhất thì $T=SL+LA_2$.\\
		Ta có $\widehat{LSA_2}=15^\circ \cdot 8=120^\circ$.\\
		Áp dụng định lí cosin vào $\triangle SLA_2$ có
		\allowdisplaybreaks
		\begin{eqnarray*}
			LA_2=\sqrt{SL^2+SA_2^2-2\cdot SL\cdot SA_2\cdot\cos \widehat{SLA_2}}=40\sqrt{31}.
		\end{eqnarray*}
		Vậy $T=40+40\sqrt{31}\approx 263$.
	}
\end{ex}
\Closesolutionfile{ans}
% \begin{name}
	{\tenchude}
	{\tendethi}
	{\tentruong}
	{\thoigian}
	\end{name}
\TN
\Opensolutionfile{ans}[ans/de5-phanI]
\begin{ex}%Câu 1
	Cho hàm số $y=f(x)$ có bảng biến thiên như sau
	\begin{center}
		\begin{tikzpicture}
			\tkzTabInit[nocadre=false,lgt=1.2,espcl=2.5,deltacl=0.6]
			{$x$ /0.6,$y'$ /0.6,$y$ /2}
			{$-\infty$,$-1$,$0$,$1$,$+\infty$}
			\tkzTabLine{,-,$0$,+,$0$,-,$0$,+,}
			\tkzTabVar{+/$+\infty$, -/$1$,+/$2$,-/$1$,+/$+\infty$}
		\end{tikzpicture}
	\end{center}
	Hàm số đã cho nghịch biến trên khoảng nào dưới đây?
	\choice
	{$(0;+\infty)$}
	{$(-\infty;1)$}
	{\True $(0;1)$}
	{$(-1;0)$}
	\loigiai{
		Quan sát bảng biến thiên ta thấy $y'<0$ trên các khoảng $(-\infty;-1)$ và $(0;1)$ nên hàm số đã cho nghịch biến trên các khoảng $(-\infty;-1)$ và $(0;1)$.
	}
\end{ex}
\begin{ex}%Câu 2
	Họ tất cả các nguyên hàm của hàm số $f(x)=4x+\sin x$ là
	\choice
	{\True $2x^2-\cos x+C$}
	{$2x^2+\cos x+C$}
	{$2x^2-\sin x+C$}
	{$2x^2+\sin x+C$}
	\loigiai{
		Ta có 
		\begin{align*}
			\displaystyle \int f(x)\mathrm{\,d}x &= \int \left(4x+\sin x\right) \mathrm{\,d}x \\
			&= \int 4x\mathrm{\,d}x+\int \sin x \mathrm{\,d}x\\
			&=4\cdot \dfrac{x^2}{2}-\cos x+C\\
			&=2x^2-\cos x+C.
		\end{align*}
	}
\end{ex}
\begin{ex}%Câu 3
	Kết quả khảo sát cân nặng số táo ở lô hàng $B$ được cho ở bảng sau
	\begin{center}
		\begin{tabular}{|c|c|c|c|c|c|}
			\hline
			Cân nặng (g) & $[150;155)$ & $[155;160)$ & $[160;165)$ & $[165;170)$ & $[170;175)$ \\
			\hline
			Số quả táo ở lô hàng $B$ & $1$ & $3$ & $7$ & $10$ & $4$ \\
			\hline
		\end{tabular}
	\end{center}
	Số tạo được khảo sát trong bảng số liệu là
	\choice
	{$6$}
	{\True $25$}
	{$7$}
	{$5$}
	\loigiai{
		Số táo 	được khảo sát trong bảng số liệu là $n=1+3+7+10+4=25$.
	}
\end{ex}
\textbf{\textit{Sử dụng thông tin dưới đây để trả lời câu \ref{câu 4-đề 5} và câu \ref{câu 5-đề 5}}}\\[0.5em]
Trong không gian $Oxyz$, cho mặt phẳng $(P)\colon 2x-3y+6z-5=0$ và điểm $A(2;-3;1)$.
\begin{ex}%Câu 4
	\label{câu 4-đề 5}
	Một vectơ pháp tuyến của mặt phẳng $(P)$ là
	\choice
	{$\overrightarrow{n}_1=(2;-3;-5)$}
	{\True $\overrightarrow{n}_2=(2;-3;6)$}
	{$\overrightarrow{n}_3=(2;3;-5)$}
	{$\overrightarrow{n}_4=(2;-3;5)$}
	\loigiai{
		Một vectơ pháp tuyến của mặt phẳng $(P)$ là $\overrightarrow{n}=(2;-3;6)$.
	}
\end{ex}
\begin{ex}%Câu 5
	\label{câu 5-đề 5}
	Đường thẳng $d$ đi qua điểm $A$ và vuông góc với mặt phẳng $(P)$ có phương trình tham số là
	\choice
	{$d\colon\heva{& x=2+2t \\ & y=-3-3t\\ & z=1-5t}$}
	{$d\colon\heva{& x=2+2t \\ & y=-3-3t\\ & z=6+t}$}
	{\True $d\colon\heva{& x=2+2t \\ & y=-3-3t\\ & z=1+6t}$}
	{$d\colon\heva{& x=-2+2t \\ & y=3-3t\\ & z=-1+6t}$}
	\loigiai{
		Ta có đường thẳng $d\perp (P)$  nên có vectơ chỉ phương là $\overrightarrow{u}_d=\overrightarrow{n}_{(P)}=(2;-3;6)$.\\
		Đường thẳng $d$ đi qua $A$ và có vectơ chỉ phương là $\overrightarrow{u}_d==(2;-3;6)$ có phương trình tham số là
		\[\heva{&x=2+2t\\&y=-3-3t\\&z=1+6t.}\]
	}
\end{ex}
\begin{ex}%Câu 6
	$\lim\left(-3n^3+2n^2-5\right)$ bằng
	\choice
	{$-3$}
	{$-6$}
	{\True $-\infty$}
	{$+\infty$}
	\loigiai{
		Ta có $\lim \left(-3n^3+2n^2-5\right)= \lim n^3 \left(-3+\dfrac{2}{n}-\dfrac{5}{n^3}\right)$.\\
		Vì $
		\heva{&\lim n^3=+\infty\\&\lim\left(-3+\dfrac{2}{n}- \dfrac{5}{n^3}\right)=-3<0}$ nên  $\lim n^3 \left(-3+\dfrac{2}{n}-\dfrac{5}{n^3}\right)=-\infty$.\\ Suy ra $\lim \left(-3n^3+2n^2-5\right)=-\infty$ .
	}
\end{ex}
\begin{ex}%Câu 7
	\immini[thm]
	{
		Diện tích hình thang cong ở hình vẽ bên là $S=10$. Tích phân $\displaystyle\int\limits_{0}^{4} \left[4x+f(x)\right] \mathrm{\,d}x$ bằng
	\choice[2]
	{$14$}
	{\True $42$}
	{$32$}
	{$26$}
	}
	{
		\begin{tikzpicture}[scale=0.6,>=stealth, font=\footnotesize, line join=round, line cap=round]
			\def\a{0.25} \def\b{-1.5} \def\c{2.25} \def\d{2} % Hệ số
			\def\xmin{-1} \def\xmax{5}
			\def\ymin{-1} \def\ymax{4} 
			\draw[->] (\xmin,0)--(\xmax,0) node [below]{$x$};
			\draw[->] (0,\ymin)--(0,\ymax) node [left]{$y$};
			\node at (0,0) [below left]{$O$};
			\clip (\xmin+0.1,\ymin+0.1) rectangle (\xmax-0.5,\ymax-0.1);
			\draw[smooth,samples=300] plot(\x,{\a*(\x)^3+\b*(\x)^2+\c*(\x)+\d});
			\fill[pattern=north east lines,opacity=0.8] (0,0)--plot[domain=0:4](\x,{\a*(\x)^3+\b*(\x)^2+\c*(\x)+\d})--(4,0)--cycle;
			\draw[dashed] (4,0)node[below]{$4$}--(4,3) (0,2)node[left]{$2$};
		\end{tikzpicture}
	}
	\loigiai{
		Ta có $S=\displaystyle\int\limits_0^4 \left|f(x)\right|\mathrm{\,d}x=\displaystyle\int\limits_0^4 f(x)	\mathrm{\,d}x=10$.\\
		Do đó,
		\[
		\displaystyle\int\limits_0^4 \left[4x+f(x)\right]\mathrm{\,d}x=\displaystyle\int\limits_0^4 4x \mathrm{\,d}x+\displaystyle\int\limits_0^4 f(x)\mathrm{\,d}x
		=\left( 2x^2\right)\bigg|_0^4+10=32+10=42.
		\]
	}
\end{ex}
%\renewcommand{\baselinestretch}{1.4}
\begin{ex}%Câu 8
	Cho các số thực dương $a$, $b$ thỏa mãn $3\log a+2\log b=1$. Mệnh đề nào sau đây đúng?
	\choice
	{$a^3+b^2=1$}
	{$3a+2b=10$}
	{\True $a^3b^2=10$}
	{$a^3+b^2=10$}
	\loigiai{
		Ta có
		\[
		3\log a+2\log b=1\Leftrightarrow
		\log a^3+\log b^2 =1 \Leftrightarrow \log \left(a^3b^2\right) =1 \Leftrightarrow a^3b^2=10.
		\]
	}
\end{ex}
\begin{ex}%Câu 9
	Cho hình hộp $ABCD.A'B'C'D'$. Tính tổng $\overrightarrow{AB}+\overrightarrow{AD}+\overrightarrow{A'C'}$.
	\choice
	{$2\overrightarrow{AA'}$}
	{$\overrightarrow{0}$}
	{\True $2\overrightarrow{AC}$}
	{$2\overrightarrow{C'A'}$}
	\loigiai{
		Ta có $\overrightarrow{A'C'}=\overrightarrow{AC}$ và $\overrightarrow{AB}+\overrightarrow{AD}=\overrightarrow{AC}$ (quy tắc hình bình hành).\\
		Suy ra $\overrightarrow{AB}+\overrightarrow{AD}+\overrightarrow{A' C'}=\left(\overrightarrow{AB}+\overrightarrow{AD}\right)+\overrightarrow{A' C'}=\overrightarrow{AC}+\overrightarrow{AC}=2\overrightarrow{AC}$.
	}
\end{ex}
\begin{ex}%Câu 10
	Đồ thị hàm số $y=x^3-3x^2+2$ là đường cong nào trong hình sau đây?
	\choice
	{\begin{tikzpicture}[scale=0.53,>=stealth, font=\footnotesize, line join=round, line cap=round]
			\def\a{1} \def\b{-2} \def\c{2} % Hệ số
			\def\xmin{-2.5} \def\xmax{2.5}
			\def\ymin{-1} \def\ymax{4} 
			\draw[->] (\xmin,0)--(\xmax,0) node [below]{$x$};
			\draw[->] (0,\ymin)--(0,\ymax) node [left]{$y$};
			\node at (0,0) [below left]{$O$};
			\clip (\xmin+0.1,\ymin+0.1) rectangle (\xmax-0.5,\ymax-0.1);
			\draw[smooth,samples=300,domain=\xmin:\xmax] plot(\x,{\a*(\x)^4+\b*(\x)^2+\c});
	\end{tikzpicture}}
	{\begin{tikzpicture}[scale=0.44,>=stealth, font=\footnotesize, line join=round, line cap=round]
			\def\a{1} \def\b{0} \def\c{-3} \def\d{2} % Hệ số
			\def\xmin{-3} \def\xmax{3}
			\def\ymin{-1} \def\ymax{5} 
			\draw[->] (\xmin,0)--(\xmax,0) node [below]{$x$};
			\draw[->] (0,\ymin)--(0,\ymax) node [left]{$y$};
			\node at (0,0) [below left]{$O$};
			\clip (\xmin+0.1,\ymin+0.1) rectangle (\xmax-0.5,\ymax-0.1);
			\draw[smooth,samples=300] plot(\x,{\a*(\x)^3+\b*(\x)^2+\c*(\x)+\d});
	\end{tikzpicture}}
	{\True \begin{tikzpicture}[scale=0.44,>=stealth, font=\footnotesize, line join=round, line cap=round]
			\def\a{1} \def\b{-3} \def\c{0} \def\d{2} % Hệ số
			\def\xmin{-2} \def\xmax{4}
			\def\ymin{-3} \def\ymax{3} 
			\draw[->] (\xmin,0)--(\xmax,0) node [below]{$x$};
			\draw[->] (0,\ymin)--(0,\ymax) node [left]{$y$};
			\node at (0,0) [below left,xshift=0.1cm]{$O$};
			\clip (\xmin+0.1,\ymin+0.1) rectangle (\xmax-0.5,\ymax-0.1);
			\draw[smooth,samples=300] plot(\x,{\a*(\x)^3+\b*(\x)^2+\c*(\x)+\d});
	\end{tikzpicture}}
	{\begin{tikzpicture}[scale=0.44,>=stealth, font=\footnotesize, line join=round, line cap=round]
			\def\a{-1} \def\b{3} \def\c{0} \def\d{0} % Hệ số
			\def\xmin{-2} \def\xmax{4}
			\def\ymin{-1} \def\ymax{5} 
			\draw[->] (\xmin,0)--(\xmax,0) node [below]{$x$};
			\draw[->] (0,\ymin)--(0,\ymax) node [left]{$y$};
			\node at (0,0) [below left]{$O$};
			\clip (\xmin+0.1,\ymin+0.1) rectangle (\xmax-0.5,\ymax-0.1);
			\draw[smooth,samples=300] plot(\x,{\a*(\x)^3+\b*(\x)^2+\c*(\x)+\d});
	\end{tikzpicture}}
	\loigiai{
		Đồ thị hàm bậc ba $y=x^3-3x^2+2$ với $a>0$ nên loại 
		\begin{center}
			\begin{tabular}{cc}
				\begin{tikzpicture}[thick,>=stealth,scale=0.6] 
					\clip(-2.5,-1) rectangle (2.5,3.5);
					\draw[->,very thick,blue] (-2.5,0) -- (2.5,0) node[below left] {\small $x$};
					\draw[->,very thick,blue] (0,-1) -- (0,3.5) node[below left] {\small $y$};
					\draw [fill=white,draw=blue] (0,0) circle (1pt)node[below right] {\footnotesize $O$};
					\draw[very thick,black,smooth,samples=100,domain=-2.5:2.5] plot(\x,{(\x)^4-2*(\x)^2+2});
					%	\draw[dashed, thick,blue] 
					%	(1,0) node[below]{1}|-(0,1) node[above left]{1}
					%	(-1,0) node[below]{-1}|-(0,1)node[left]{};
				\end{tikzpicture}
				&
				\begin{tikzpicture}[thick,>=stealth,scale=0.6] 
					\clip(-2,-1.5) rectangle (4.5,4.5);
					\draw[->,very thick,blue] (-2,0) -- (4.5,0) node[below left] {\small $x$};
					\draw[->,very thick,blue] (0,-1.5) -- (0,4.5) node[below left] {\small $y$};
					\draw [fill=white,draw=blue] (0,0) circle (1pt)node[below right] {\footnotesize $O$};
					\draw[very thick,black,smooth,samples=100,domain=-2:4.5] plot(\x,{-(\x)^3+3*(\x)^2});
					%	\draw[dashed, thick,blue] 
					%	(1,0) node[below]{1}|-(0,1) node[above left]{1}
					%	(-1,0) node[below]{-1}|-(0,1)node[left]{};
				\end{tikzpicture}
			\end{tabular}
		\end{center}
		Hàm số $y=x^3-3x^2+2$ có $y'=3x^2-6x$. Cho $y'=0 \Rightarrow \hoac{&x=0\\&x=2.}$\\
		Suy ra $x=0$ và $x=2$ là hai điểm cực trị nên chọn
		\begin{center}
			\begin{tikzpicture}[thick,>=stealth,scale=0.6] 
				\clip(-2,-2.5) rectangle (4.5,3.5);
				\draw[->,very thick,blue] (-2,0) -- (4.5,0) node[below left] {\small $x$};
				\draw[->,very thick,blue] (0,-2.5) -- (0,3.5) node[below left] {\small $y$};
				\draw [fill=white,draw=blue] (0,0) circle (1pt)node[below right] {\footnotesize $O$};
				\draw[very thick,black,smooth,samples=100,domain=-2:4.5] plot(\x,{(\x)^3-3*(\x)^2+2});
			\end{tikzpicture}
		\end{center}
	}
\end{ex}
\begin{ex}%Câu 11
	Cho hình chóp $S.ABCD$ có đáy $ABCD$ là hình thoi tâm $O$. Biết rằng $SA=SC$ và $SB=SD$. Khẳng định nào sau đây là \textbf{sai}?
	\choice
	{$SO\perp(ABCD)$}
	{$AC\perp BD$}
	{$(SBD)\perp(SAC)$}
	{\True $BD\perp SD$}
	\loigiai{
		\immini{
			Ta có $\heva{& SO\perp AC\\& SO\perp BD}\Rightarrow SO\perp (ABCD)$ đúng.\\
			Vì $ABCD$ là hình thoi nên $AC\perp BD$ đúng.\\
			Ta có $\heva{& BD\perp AC\\& BD\perp SO} \Rightarrow BD\perp (SAC)$
			Suy ra $\heva{& BD\subset (SBD)\\& BD\perp (SAC)}\Rightarrow (SBD)\perp (SAC)
			$ đúng.\\
			Ta có $BD\subset (SBD)$ mà tam giác $SBD$ cân tại $S$ nên $\widehat{SDB}=\widehat{SBD}<90^\circ$ do đó $BD\perp SD$ sai.
		}{
			\begin{tikzpicture}[declare function={a=2;b=4;h=4;},line join=round]
				\path (0,0) coordinate (A)
				(-145:a) coordinate (B)
				(b,0) coordinate (D)
				($ (B)!0.5!(D) $) coordinate (O)
				($(O)+(0,h) $) coordinate (S);
				\path ($(D)-(A)+(B)$) coordinate (C);
				\draw[dashed] (S)--(A)--(B) (A)--(D)--(B) (A)--(C) (S)--(O);
				\draw (B)--(C)--(D) (B)--(S)  (D)--(S)--(C);
				\foreach \x/\y/\z in {S/O/B,S/O/C,A/O/B}{
					\path pic[draw,angle radius=5pt]{right angle= \x--\y--\z};
				}
				\foreach \t/\g in {A/150,B/-90,C/-90,D/0,S/90,O/-90}{
					\draw[fill=black] (\t) circle (1pt) node[shift={(\g:7pt)},font=\scriptsize]{$ \t $};
				}
			\end{tikzpicture}
		}
	}
\end{ex}
\begin{ex}%Câu 12
	Tìm tập nghiệm $S$ của bất phương trình $\log_{\tfrac{1}{5}}\left(x^2-1\right)<\log_{\tfrac{1}{5}}\left(3x-3\right)$.
	\choice
	{\True $S=(2;+\infty)$}
	{$S=(-\infty;1)\cup(2;+\infty)$}
	{$S=(-\infty;-1)\cup(2;+\infty)$}
	{$S=(1;2)$}
	\loigiai{
		Phương trình đã cho tương đương
		\[
		\heva{&x^2-1>3x-3\\&3x-3>0}\Leftrightarrow \heva{&x^2-3x+2>0\\&x>1} \Leftrightarrow \heva{&\hoac{&x<1\\&x>2}\\&x>1} \Leftrightarrow x>2.
		\]
		Vậy tập nghiệm của bất phương trình là $S=(2;+\infty)$.
	}
\end{ex}
\Closesolutionfile{ans}
%{\fontfamily{qtm}\fontsize{13pt}{2pt}\selectfont\textbf{PHẦN II. Câu trắc nghiệm đúng sai}. Thí sinh trả lời từ câu 1 đến câu 4. Trong mỗi ý \textbf{a)}, \textbf{b)}, \textbf{c)}, \textbf{d)} ở mỗi câu, thí sinh chọn đúng hoặc sai.}
%\setcounter{ex}{0}% Reset lại số đếm câu hỏi
\TNTF
\Opensolutionfile{ans}[ans/de5-phanII]
\begin{ex}%Câu 1
	Trong không gian $Oxyz$, cho các điểm $A(0;-1;1)$, $B(-2;1;-1)$ và $C$ thỏa mãn điều kiện $\overrightarrow{OC}=-\overrightarrow{i}+3\overrightarrow{j}+2\overrightarrow{k}$. Xét tính đúng sai của các mệnh đề sau
	\choiceTF
	{Tọa độ điểm $C$ là $(-1;2;3)$}
	{\True Tọa độ các vectơ $\overrightarrow{AB}=(-2;2;-2)$ và $\overrightarrow{AC}=(-1;4;1)$}
	{Một vectơ pháp tuyến của mặt phẳng $(ABC)$ là $(5;-2;-3)$}
	{Khoảng cách từ gốc tọa độ $O$ đến mặt phẳng $(ABC)$ bằng $\dfrac{5}{\sqrt{33}}$}
		\loigiai{
		\begin{itemchoice}
			\itemch 	Ta có $\overrightarrow{OC}=-\overrightarrow{i}+3\overrightarrow{j}+2\overrightarrow{k}$ nên tọa độ điểm $C(-1;3;2)$.
			\itemch Tọa độ $\overrightarrow{AB}=(-2;2;-2)$ và $\overrightarrow{AC}=(-1;4;1)$.
			\itemch 	Một vectơ pháp tuyến của mặt phẳng $(ABC)$ là \[\overrightarrow{n}=[\overrightarrow{AB},\overrightarrow{AC}] = (10;4;-6)=2(5;2;-3).\]
			\itemch Phương trình mặt phẳng $(ABC)$ là
			\[
			5(x-0)+2(y+1)-3(z-1)=0 \Leftrightarrow 5x+2y-3z+5=0.
			\]
			Khoảng cách từ $O$ đến $(ABC)$ là
			\[
			\mathrm{d}(O,(ABC))=\dfrac{|5\cdot 0+2\cdot 0-3\cdot 0+5|}{\sqrt{5^2+2^2+(-3)^2}}=\dfrac{5}{\sqrt{38}}=\dfrac{5\sqrt{38}}{38}.
			\]
		\end{itemchoice}	
	}
\end{ex}
\begin{ex}%Câu 2
	Tốc độ giao đổi chất cơ bản của sinh vật có thể tăng hoặc giảm tùy thuộc vòa hoạt động của sinh vật. Cụ thể sau khi hấp thụ chất dinh dưỡng, sinh vật thường trải qua một sự tăng đột biến trong tốc độ trao đổi chất của nó, sau đó dần dần trở lại mức cơ bản. Linh vừa kết thúc bữa ăn tối của mình với năng lượng nạp vào là $5120$ J và tốc độ trao đổi chất của cô đã tăng đột biến từ mức cơ bản $M_0$. Sau đó cô đã tiêu hao hết năng lượng đó trong $12$ giờ tiếp theo. Giả sử $t$ giờ sau bữa ăn Linh tiêu hao được $M(t)$ kJ, tốc độ trao đổi chất của cô được cho bởi hàm số $M'(t)=M_0+t\mathrm{e}^{-0{,}1t^2}$ (kJ/h), $t\in[0;12]$. Xét tính đúng sai của các mệnh đề sau
	\choiceTF
	{\True $M(t)=M_0t-5\mathrm{e}^{-0{,}1t^2}+C$ là họ nguyên hàm của hàm số $M'(t)$}
	{\True $M_0=0{,}01$ (làm tròn đến hàng phần trăm)}
	{\True Năng lượng còn lại sau $6$ giờ đâu là $197$ J (làm tròn đến hàng đơn vị)}
	{Tốc độ tiêu hao năng lượng trung bình trong khoảng thời gian từ $a$ giờ tới $b$ giờ được tính bởi công thức $v_{\text{tb}}=\dfrac{M(b)-M(a)}{b-a}$. Tốc độ tiêu hao năng lượng trung bình từ $6$ giờ đến $12$ giờ của Linh là $32{,}76$ J/h (làm tròn kết quả đến hàng đơn vị)}
	\loigiai{
		\begin{itemchoice}
			\itemch Ta có $M'(t)=\left(M_0 t-5\mathrm{e}^{-0{,}1t^2}+C\right)' = M_0+t\mathrm{e}^{-0{,}1t^2}$.\\
			Suy ra $\displaystyle \int M'(t)\mathrm{\,d}t=M(t)=M_0 t-5\mathrm{e}^{-0{,}1t^2}+C$.
			\itemch Tổng năng lượng tiêu hao trong khoảng thời gian $12$ giờ là $5\,120$ J $ =5{,}12$ kJ tức là
			\[
			M(12)-M(0)=5{,}12\,\,\text{kJ}.
			\]
			Khi $t=12$ ta có $M(12)=M_0\cdot 12-5\mathrm{e}^{-0{,}1\cdot 12^2}+C$.\\
			Khi $t=0$ ta có $M(0)=M_0\cdot 0-5\mathrm{e}^{-0{,}1\cdot 0^2}+C=-5+C$.\\
			Suy ra $M(12)-M(0)=12M_0-5\left( \mathrm{e}^{-0{,}1\cdot 12^2}-1\right)$.\\
			Thay $M(12)-M(0)=5\,120$ ta được
			\[
			12M_0-5\left( \mathrm{e}^{-0{,}1\cdot 12^2}-1\right) = 5{,}12
			\Leftrightarrow M_0=0{,}01\,\, \rm{kJ/h}.
			\]
			\itemch Năng lượng tiêu hao trong $6$ giờ
			\[
			M(6)-M(0)=6M_0-5\left(\mathrm{e}^{-0{,}1\cdot 6^2}-1\right)\approx 4{,}92\,\,\text{kJ}.
			\]
			Năng lượng còn lại sau $6$ giờ
			\[
			5120\,\,\text{J}-4920\,\,\text{J}=200\,\,{J}.
			\]
			\itemch 	Tốc độ tiêu hao trung bình từ $6$ giờ đến $12$ giờ là
			\[
			v_{\text{tb}}=\dfrac{M(12)-M(6)}{12-6}=\dfrac{6M_0-5\left(\mathrm{e}^{-0{,}1\cdot 12^2}-\mathrm{e}^{-0{,}1\cdot 6^2}\right)}{6}\approx 0{,}0327695\,\,\rm{kJ/h}\approx 32{,}77 \,\,\rm{J/h}.
			\]
		\end{itemchoice}	
	}
\end{ex}
%\renewcommand{\baselinestretch}{1.4}
\begin{ex}%Câu 3
	\immini[thm]
	{
		Ta có Trái Đất là hình cầu hoàn hảo với bán kính $R=6370$ km và diện tích toàn phần là $S=4\pi R^2$. Các phi hành gia từ tàu vũ trụ chỉ có thể nhìn thấy một phần bề mặt Trái Đất. Ở độ cao $h$, phần diện tích Trái Đất các phi hành gia có thể nhìn thấy sẽ được tính theo công thức $S_{T}=2\pi R^2\left(1-\dfrac{R}{R+h}\right)$, trong đó $R$ là bán kính Trái Đất. Gọi $K$ là tỷ số diện tích bề mặt Trái Đất nhìn thấy được ở độ cao $h$ với diện tích toàn phần của Trái Đất. Xét tính đúng sai của các mệnh đề sau
	}
	{
		\begin{tikzpicture}[scale=0.7,>=stealth, font=\footnotesize, line join=round, line cap=round]
			\coordinate (O) at (0,0);
			\draw (O)node[opacity=0.85]{\includegraphics[width=3.45cm]{images/de5-2}};
			\coordinate (M) at ($(O)+(-130:1.5cm)$);
			\coordinate (A) at (120:3 cm and 3 cm);
			\draw[dashed] (A) arc (120:-120:3 cm and 3 cm);
			\coordinate (A') at (0:3 cm and 3 cm);
			\draw (A')node[rotate=90,left,xshift=0.1cm]{\includegraphics[width=0.6cm]{images/de5-1}};
			\draw[dashed] (O)--(A') ($(O)!0.7!(A')$)node[above]{$h$};
			\coordinate (B) at (45:4.5 cm and 4.5 cm);
			\draw[dashed] (B) arc (45:-45:4.5 cm and 4.5 cm);
			\coordinate (B') at (0:4.5 cm and 4.5 cm);
			\draw (B')node[rotate=90,left,xshift=0.1cm]{\includegraphics[width=0.6cm]{images/de5-1}};
			\draw[thick] (O)--(M);
			\tkzDefLine[tangent from = A'](O,M)% with 4.25
			\tkzGetPoints{X}{Y} 
			\draw (A')--($(A')!1.5!(X)$) (A')--($(A')!1.5!(Y)$);
			\tkzDefLine[tangent from = B'](O,M)% with 4.25
			\tkzGetPoints{X'}{Y'}
			\draw[dashed] (B')--(X') (B')--(Y');
		\end{tikzpicture}
	}\vspace{3pt}
	\choiceTF
	{Công thức tính $K$ là $K=\dfrac{1}{2}\left(1-\dfrac{h}{R+h}\right)$}
	{Trong một chuyến bay của tàu con thoi, các phi hành gia đã thực hiện một hoạt động ngoài tàu ở độ cao $280$ km. Có $2{,}5\%$ (làm tròn đến hàng phần mười) diện tích bề mặt Trái Đất có thể nhìn thấy ở độ cao đó}
	{Muốn nhìn thấy $\dfrac{1}{4}$ diện tích bề mặt Trái Đất, các phi hành gia cần đưa tàu con thoi đạt đến độ cao $6470$ km}
	{\True Khi độ cao $h$ càng tăng lên thì $K$ càng tăng nhưng không vượt quá $50\%$}
	\loigiai{
		\begin{itemchoice}
			\itemch 	Ta có tỷ số $K$ là diện tích nhìn thấy được chia cho diện tích toàn phần
			\[
			K =\dfrac{S_T}{S}=\dfrac{2\pi R^2\left(1-\dfrac{R}{R+h}\right)}{4\pi R^2}
			= \dfrac{1}{2}\left(1-\dfrac{R}{R+h}\right)=\dfrac{1}{2}\cdot \dfrac{h}{R+h}.
			\]
			\itemch Với bán kính $R=6\,370$ km và chiều cao $h=280$ km thì tỷ số $K$ tại độ cao $h$ là
			\[
			K=\dfrac{1}{2}\cdot \dfrac{h}{R+h}=\dfrac{1}{2}\cdot \dfrac{280}{6\,370+280}=\dfrac{2}{95}\approx0{,}021\approx 2{,}1\%.
			\]		
			\itemch Với $K=\dfrac{1}{4}$ thì
			\[
			K=\dfrac{1}{2}\cdot \dfrac{h}{R+h}\Leftrightarrow \dfrac{1}{4}=\dfrac{1}{2}\cdot \dfrac{h}{6\,370+h}\Leftrightarrow h=6\,370\,\,\rm{(km)}.\]
			\itemch 	Khi $h$ càng lớn ta có
			\[
			\lim\limits_{h\rightarrow +\infty} K =\lim\limits_{h\rightarrow +\infty} \dfrac{1}{2}\dfrac{h}{R+h}=\dfrac{1}{2}\lim\limits_{h\rightarrow +\infty} \dfrac{1}{\dfrac{R}{h}+1}=\dfrac{1}{2}=0{,}5=50\%.
			\]
		\end{itemchoice}
	}
\end{ex}
\begin{ex}%Câu 4
	Ở một khu rừng nọ có $7$ chú lùn, trong đó có $5$ chú luôn nói thật, $2$ chú còn lại nói thật với xác suất $0{,}5$. Nàng Bạch Tuyết lạc vào trong rừng và gặp một chú lùn.
	\begin{itemize}
		\item Gọi $A$ là biến cố \lq\lq Chú lùn gặp được luôn nói thật\rq\rq.
		\item Gọi $B$ là biến cố \lq\lq Chú lùn đó tự nhận mình luôn nói thật\rq\rq.
	\end{itemize}
	Xét tính đúng sai của các mệnh đề sau\vspace{5pt}
	\choiceTF
	{\True $\mathrm{P}(A)=\dfrac{5}{7}$ và $\mathrm{P}(\overline{A})=\dfrac{2}{7}$}
	{Xác suất có điều kiện $\mathrm{P}(B\mid A)=0{,}5$}
	{\True $\mathrm{P}(B)=\dfrac{6}{7}$}
	{\True Nàng Bạch Tuyết gặp ngẫu nhiên một chú lùn. Biết rằng chú lùn mà Bạch Tuyết gặp tự nhận mình là luôn nói thật. Xác suất để chú lùn đó luôn nói thật là $\dfrac{5}{6}$}
	\loigiai{
		\begin{itemchoice}
			\itemch Ta có $\mathrm{P}(A)=\dfrac{5}{7}\Rightarrow \mathrm{P}(\overline{A}) = \dfrac{2}{7}$ 
			\itemch Xác xuất có điều kiện $P(B\mid A)=1$.
			\itemch Theo công thức Bayes, ta có
			\[\mathrm{P}(B)=\mathrm{P}(A) \cdot \mathrm{P}(B|A)+\mathrm{P}(\overline{A})\cdot\mathrm{P}(B|\overline{A}) \Leftrightarrow \mathrm{P}(B)=1\cdot \dfrac{5}{7}+ 0{,}5\cdot \dfrac{2}{7}=\dfrac{6}{7}.\]
			\itemch Ta có
			\[\mathrm{P}(A|B)=\dfrac{\mathrm{P}(AB)}{\mathrm{P(B)}}=\dfrac{\mathrm{P}(A)\cdot \mathrm{P}(B|A)}{\mathrm{P}(B)}= \dfrac{\dfrac{5}{7}\cdot 1}{\dfrac{6}{7}}=\dfrac{5}{6}.\]
		\end{itemchoice}	
	}
\end{ex}
\Closesolutionfile{ans}
%{\fontfamily{qtm}\fontsize{13pt}{2pt}\selectfont\textbf{PHẦN III. Câu trắc nghiệm trả lời ngắn}. Thí sinh trả lời từ câu 1 đến câu 6 và điền đáp án vào ô trống.}
%\setcounter{ex}{0}% Reset lại số đếm câu hỏi
\TNSA
\Opensolutionfile{ans}[ans/de5-phanIII]
\begin{ex}%Câu 1
	Cho hình chóp đều $S.ABCD$ có cạnh đáy bằng $4$, khoảng cách giữa hai đường thẳng $SA$ và $CD$ bằng $2$. Thể tích của khối chóp $S.ABCD$ bằng bao nhiêu? (làm tròn kết quả đến hàng phần mười).
	
	\shortans[0]{$6{,}2$}
	\loigiai{
		\immini{
			Diện tích mặt đáy $S_{ABCD}=4^2=16$.\\
			Chọn $(SAB)$ chứa $SA$, ta có $\heva{
				&CD\parallel AB\\
				&AB\subset (SAB).
			}$\\
			Suy ra $CD\parallel (SAB)$ nên 
			\begin{align*}
				\mathrm{d}(CD,SA)&=\mathrm{d}(CD,(SAB))=\mathrm{d}(D,(SAB))\\
				&=2\mathrm{d}(O,(SAB))=2\\
				\Rightarrow &\mathrm{d}(O,(SAB))=1.
			\end{align*}
			Gọi $H$ là hình chiếu vuông góc của $O$ trên $AB$ và 
		}{
			\begin{tikzpicture}[declare function={a=2;b=4;h=4;},line join=round]
				\path (0,0) coordinate (A)
				(-145:a) coordinate (B)
				(b,0) coordinate (D)
				($ (B)!0.5!(D) $) coordinate (O)
				($(O)+(0,h) $) coordinate (S);
				\path ($(D)-(A)+(B)$) coordinate (C);
				\path ($(A)!0.5!(B)$) coordinate (H);
				\path ($(S)!0.7! (H)$) coordinate (K);
				\draw[dashed] (S)--(A)--(B) (A)--(D)--(B) (A)--(C) (S)--(O)--(H)--(S) (K)--(O)--(H);
				\draw (B)--(C)--(D) (B)--(S)  (D)--(S)--(C);		
				\foreach \x/\y/\z in {S/O/B,S/O/C,A/O/B,O/H/A,O/K/S}{
					\path pic[draw,angle radius=5pt]{right angle= \x--\y--\z};
				}
				\foreach \t/\g in {A/160,B/-90,C/-90,D/0,S/90,O/-90,H/160,K/-140}{
					\draw[fill=black] (\t) circle (1pt) node[shift={(\g:7pt)},font=\scriptsize]{$ \t $};
				}
			\end{tikzpicture}
		}
		$K$ là hình chiếu vuông góc của $O$ trên $SH$, ta có
		\[
		\heva{
			&AB\perp OH\\
			&AB\perp SO
		}\Rightarrow AB\perp (SOH)\Rightarrow AB\perp OK.\quad (1)
		\]
		Mà $OK\perp SH$. \quad (2)\\
		Suy ra $OK\perp (SAB)\Rightarrow \mathrm{d}(O,(SAB))=OK=1$.\\
		Với $OH=\dfrac{1}{2}AD=2$ ta có
		\[
		\dfrac{1}{OK^2}=\dfrac{1}{OH^2}+\dfrac{1}{SO^2}\Leftrightarrow
		\dfrac{1}{SO^2}=\dfrac{1}{1^2}-\dfrac{1}{2^2}=\dfrac{3}{4}\Rightarrow
		SO^2=\dfrac{4}{3}\Rightarrow SO=\dfrac{2\sqrt{3}}{3}.
		\]
		Thể tích khối chóp $S.ABCD$ là
		\[
		V=\dfrac{1}{3}\cdot S_{ABCD}\cdot SO=\dfrac{32\sqrt{3}}{9}\approx 6{,}2.
		\]
	}
\end{ex}
\begin{ex}%Câu 2
	Một xạ thủ bắn hai viên đạn vào một bia. Xác suất bắn trúng viên thứ nhất là $0{,}7$. Nếu bắn trúng viên thứ nhất thì khả năng bắn trung viên thứ hai là $0{,}8$, nhưng nếu bắn trượt viên thứ nhất thì sẽ bị tâm lí dẫn đến khả năng bắn trúng viên thứ hai chỉ còn $0{,}3$. Biết rằng viên thứ hai xạ thủ bắn trúng, xác suất xạ thủ bắn trung viên thứ nhất là bao nhiêu $\%$ (làm tròn kết quả đến hàng phần chục).
	
	\shortans[0]{$86{,}2$}
	\loigiai{
		Gọi $A$ là biến cố: \lq\lq xạ thủ bắn trúng viên thứ nhất\rq\rq.\\
		Gọi $B$ là biến cố: \lq\lq xạ thủ bắn trúng viên thứ hai\rq\rq.\\
		Ta có $\mathrm{P}(A)=0{,}7$; $\mathrm{P}(\overline{A})=0{,}3$; 
		$\mathrm{P}(B\mid A)=0{,}8$; $\mathrm{P}(B\mid \overline{A})=0{,}3$; \\
		Theo công thức Bayes
		\[
		\mathrm{P}(A\mid B)=\dfrac{P(AB)}{P(B)}.
		\]
		Với $\mathrm{P}(AB)=\mathrm{P}(A)\cdot \mathrm{P}(B\mid A)=0{,}7\cdot 0{,}8=0{,}56$.\\
		Đồng thời $\mathrm{P}(B)=\mathrm{P}(AB)+\mathrm{P}(\overline{A}B)$\\
		Với $\mathrm{P}(AB)=0{,}56$ và $\mathrm{P}(\overline{A}B)=\mathrm{P}(\overline{A})\cdot \mathrm{P}(B\mid \overline{A})=0{,}3\cdot 0{,}3=0{,}09$.\\
		Suy ra $\mathrm{P}(B)=0{,}56+0{,}09=0{,}65$.\\
		Vậy
		\[
		\mathrm{P}(A\mid B)=\dfrac{0{,}56}{0{,}65}\approx 0{,}8615 = 86{,} 2\%.
		\]
	}
\end{ex}
\renewcommand{\baselinestretch}{1.4}
\begin{ex}%Câu 3
	Một con bọ di chuyển từ điểm $A$ đến điểm $B$ dọc theo các đoạn thẳng trong mạng lưới lục giác như hình bên dưới.
%	\begin{center}
%		\includegraphics[scale=0.85]{images/de5-3}
%	\end{center}
	\begin{center}
		\begin{tikzpicture}[>=stealth,line join=round,line cap=round,font=\small,scale=.7]
			\def\l{1.2}
			\newcommand{\drawhexagon}[3]{
				\begin{scope}[shift={(#1,#2)}]
					% Vẽ lục giác
					\draw (0:\l) -- (60:\l) -- (120:\l) -- (180:\l) -- (240:\l) -- (300:\l) -- cycle;
				\end{scope}
			}
			% Hàng 1
			\drawhexagon{0}{0}{}
			\draw [->] (0,1.04)--(0.1,1.04) node [above] {$1$} ;
			\draw [->] (0,-1.04)--(0.1,-1.04)node [below] {$2$};
			\draw (0,1.3) node[above] {$(C_1)$};
			% Hàng 2
			\drawhexagon{1.8}{1.04}{}
			\drawhexagon{1.8}{-1.04}{}
			\draw [->] (1.8,2.07)--(1.9,2.07) node [above] {$3$};
			\draw [<-] (1.8,0)--(1.9,0) ;
			\draw [->] (1.8,-2.08)--(1.9,-2.08)node [below] {$4$};
			\draw (1.8,2.4) node[above] {$(C_2)$};
			% Hàng 3
			\drawhexagon{3.6}{2.08}{}
			\drawhexagon{3.6}{0}{}
			\drawhexagon{3.6}{-2.08}{}
			\draw [->] (3.5,3.12)--(3.6,3.12) node [above] {$5$};
			\draw [->] (3.5,1.04)--(3.6,1.04) node [above] {$6$} ;
			\draw [->] (3.5,-1.04)--(3.6,-1.04) node [above] {$7$} ;
			\draw [->] (3.5,-3.12)--(3.6,-3.12) node [above] {$8$} ;
			\draw (3.5,3.4) node[above] {$(C_3)$};
			% Hàng 4
			\drawhexagon{5.4}{1.04}{}
			\drawhexagon{5.4}{-1.04}{}
			\draw [->] (5.2,2.07)--(5.3,2.07) node [above] {$9$};
			\draw [<-] (5.2,0)--(5.3,0) ;
			\draw [->] (5.2,-2.08)--(5.3,-2.08)node [below] {$10$};
			\draw (5.2,2.4) node[above] {$(C_4)$};
			% Hàng 5
			\drawhexagon{7.2}{0}{}
			\draw [->] (7,1.04)--(7.1,1.04) node [above] {$11$} ;
			\draw [->] (7,-1.04)--(7.1,-1.04)node [below] {$12$};
			\draw (7,1.3) node[above] {$(C_5)$};
			% Điểm A và B
			\node[fill=black,circle,inner sep=1pt,label=left:A] at (-1.2,0) {};
			\node[fill=black,circle,inner sep=1pt,label=right:B] at (8.4,0) {};
		\end{tikzpicture}
	\end{center}
	Các đoạn thẳng này có dấu mũi tên chỉ được di chuyển theo hướng của mũi tên và con bọ không bao giờ di chuyển trên cùng một đoạn thẳng quá một lần. Vậy con bọ có bao nhiêu con đường khác nhau từ $A$ đến $B$?
	
	\shortans[0]{$100$}
	\loigiai{
		\begin{itemize}
			\item Từ $A$ có $2$ cách đến $C1$
			\begin{itemize}
				\item Từ $A$ có $1$ cách đến mũi tên số $1$;
				\item Từ $A$ có $1$ cách đến mũi tên số $2$.
			\end{itemize}
			\item Từ $C1$ có $5$ cách đến $C3$\\
			Không mất tính tổng quát giả sử đi từ $C1$ mũi tên số $1$ đến $C3$ mũi tên số $6$ hoặc mũi tên số $7$.
			\begin{itemize}
				\item Từ mũi tên số $1$ có $2$ cách để đi đến mũi tên số $6$.
				\item Từ mũi tên số $1$ có $3$ cách để đi đến mũi tên số $7$.
			\end{itemize}
			Suy ra từ $C1$ có $5$ cách đến $C3$.
			\item Từ $C3$ có $5$ cách đến $C5$\\
			Không mất tính tổng quát giả sử đi từ $C3$ mũi tên số 5 đến $C5$ mũi tên số $11$ hoặc mũi tên số $12$.
			\begin{itemize}
				\item Từ mũi tên số $5$ có $2$ cách để đi đến mũi tên số $11$.
				\item Từ mũi tên số $5$ có $3$ cách để đi đến mũi tên số $11$.
			\end{itemize}
			Suy ra từ $C3$ có $5$ cách đến $C5$.
			\item Từ $C5$ có $2$ cách đến $B$
			\begin{itemize}
				\item Từ mũi tên số $11$ có $1$ cách đến $B$.
				\item Từ mũi tên số $12$ có $1$ cách đến $B$.
			\end{itemize}
		\end{itemize}
		Vậy có $2\cdot 5\cdot 5\cdot 2 = 100$ cách đi từ $A$ đến $B$.
	}
\end{ex}
\begin{ex}%Câu 4
	Một tấm ván gỗ chỉ được hỗ trợ ở hai đầu $O$ và $P$, cách nhau $4$ m. Tấm ván võng xuống dưới do trọng lượng của nó tạo thành một đường cong. Xét trên hệ trục $Oxy$ như hình vẽ dưới, đơn vị mỗi trục là mét, đường cong trong hình vẽ có phương trình $y=f(x)$.
	\begin{center}
		\begin{tikzpicture}[scale=1,>=stealth, font=\footnotesize, line join=round, line cap=round]
			\def\a{0.04} \def\b{-0.4} \def\c{0} % Hệ số
			\def\xmin{-1.5} \def\xmax{12}
			\def\ymin{-1.8} \def\ymax{1.5}
			\draw[->] (\xmin,0)--(\xmax,0) node [below]{$x$};
			\draw[->] (0,\ymin)--(0,\ymax) node [left]{$y$};
			\node at (0,0) [above left,xshift=0.1cm]{$O$};
			\clip (\xmin+0.1,\ymin+0.1) rectangle (\xmax-0.5,\ymax-0.1);
			\draw[smooth,samples=300,domain=-1:11] plot(\x,{\a*(\x)^2+\b*(\x)+\c});
			\draw (5,-1)node[below]{$y=f(x)$} (10,0)node[above]{$P$};
			\fill (0,0)circle(3pt);
			\fill (10,0)circle(3pt);
			\coordinate (A) at ($(0,0)+(-60:1cm)$);
			\coordinate (B) at ($(0,0)+(-120:1cm)$);
			\draw[fill=black] (0,0)--(A)--(B)--cycle;
			\coordinate (C) at ($(10,0)+(-60:1cm)$);
			\coordinate (D) at ($(10,0)+(-120:1cm)$);
			\draw[fill=black] (10,0)--(C)--(D)--cycle;
		\end{tikzpicture}
	\end{center}
	Người ta chứng minh được $f''(x)=\dfrac{1}{100}\left(2x-\dfrac{x^2}{2}\right)$ với $0\le x\le 4$. Tại điểm cách $P$ một khoảng $1$ mét, tấm ván bị võng xuống bao nhiêu cm? (làm tròn kết quả đến hàng phần trăm).
	
	\shortans[0]{$2{,}38$}
	\loigiai{
		Ta có $\displaystyle  f'(x)=\int f''(x)\mathrm{\,d}x=\int \dfrac{1}{100}\left(2x-\dfrac{x^2}{2}\right)\mathrm{\,d}x=\dfrac{1}{100}\left(x^2-\dfrac{x^3}{6}\right)+C_1$.\\
		Suy ra $\displaystyle f(x)=\int f'(x)\mathrm{\,d}x=\dfrac{1}{100}\left(\dfrac{x^3}{3}-\dfrac{x^4}{24}\right)+C_1x+C_2$.\\
		Vì $f(0)=0$ và $f(4)=0$ nên ta có
		\[
		\heva{
			&\dfrac{1}{100}\left(\dfrac{0^3}{3}-\dfrac{0^4}{24}\right)+C_1\cdot 0+C_2=0\\
			&\dfrac{1}{100}\left(\dfrac{4^3}{3}-\dfrac{4^4}{24}\right)+C_1\cdot 4+C_2=0
		}\Leftrightarrow \heva{
			&C_2=0\\
			&C_1=-\dfrac{2}{75}.
		}
		\]
		Vậy $f(x)=\dfrac{1}{100}\left(\dfrac{x^3}{3}-\dfrac{x^4}{24}\right)-\dfrac{2}{75}x$.\\
		Suy ra $f(3)=-\dfrac{19}{800}$ (m).\\
		Tại điểm cách $P$ $1$ (m), tấm ván bị võng xuống khoảng $2{,}38$ (cm).
	}
\end{ex}
\begin{ex}%Câu 5
	Một tấm bìa cứng có kích thước $60\text{ cm}\times 90\text{ cm}$ được gấp đôi thành một hình chữ nhật $60\text{ cm}\times 45\text{ cm}$ như hình vẽ. Sau đó, cắt ra từ các góc của hình chữ nhật vừa gấp bốn hình vuông bằng nhau có cạnh $x$ (cm). Tấm bìa được mở ra và sáu mép được gấp lên để tạo thành một hộp chữ nhật $(H)$ có nắp và đáy (như hình vẽ). Thể tích lớn nhất của khối $(H)$ bằng bao nhiêu lít? Làm tròn đến hàng phần mười.
	\begin{center}
		\begin{tikzpicture}[scale=1,>=stealth, font=\footnotesize, line join=round, line cap=round]
			\draw (0,0)--(4.5,0)--(4.5,-3)--(0,-3)--cycle;
			\draw[dashed] (2.25,0)--(2.25,-3);
			\draw (5,-0.5)--(5.5,-0.5)--(5.5,0)--(6.75,0)--(6.75,-0.5)--(7.25,-0.5)--(7.25,-2.5)--(6.75,-2.5)--(6.75,-3)--(5.5,-3)--(5.5,-2.5)--(5,-2.5)--cycle (5.25,-0.5)node[below]{$x$} (5.25,-2.5)node[above]{$x$} (7,-0.5)node[below]{$x$} (7,-2.5)node[above]{$x$} (5.5,-0.25)node[right]{$x$} (5.5,-2.75)node[right]{$x$} (6.75,-0.25)node[left]{$x$} (6.75,-2.75)node[left]{$x$};
			\draw[xshift=0.5cm] (9.5,-0.5)--(10,-0.5)--(10,0)--(11.25,0)--(11.25,-0.5)--(12.25,-0.5)--(12.25,0)--(13.5,0)--(13.5,-0.5)--(14,-0.5)--(14,-2.5)--(13.5,-2.5)--(13.5,-3)--(12.25,-3)--(12.25,-2.5)--(11.25,-2.5)--(11.25,-3)--(10,-3)--(10,-2.5)--(9.5,-2.5)--cycle;
			\draw[dashed,xshift=0.5cm] (10,-0.5)--(11.25,-0.5)--(11.25,-2.5)--(10,-2.5)--cycle (12.25,-0.5)--(13.5,-0.5)--(13.5,-2.5)--(12.25,-2.5)--cycle;
			\draw[|<->|] (-0.75,0)--(-0.75,-3) node[pos=.5,fill=white]{$60$ cm};
			\draw[|<->|] (8,0)--(8,-3) node[pos=.5,fill=white]{$60$ cm};
			\draw[|<->|] (9.25,0)--(9.25,-3) node[pos=.5,fill=white]{$60$ cm};
			\draw[|<->|] (0,-3.5)--(4.5,-3.5) node[pos=.5,fill=white]{$90$ cm};
			\draw[|<->|] (5,-3.5)--(7.25,-3.5) node[pos=.5,fill=white]{$45$ cm};
			\draw[|<->|] (10,-3.5)--(14.5,-3.5) node[pos=.5,fill=white]{$90$ cm};
		\end{tikzpicture}
	\end{center}
	\shortans[0]{$20{,}5$}
	\loigiai{
		Sau khi cắt bốn hình vuông cạnh $x, (0<x<\dfrac{45}{2})$ cm và gấp tấm bìa, kích thước của hình hộp là
		\begin{itemize}
			\item Chiều dài $60-2x$;
			\item Chiều rộng $45-2x$;
			\item Chiều cao $2x$.
		\end{itemize}    
		Thể tích khối hộp là
		\[
		V(x)=2x(60-2x)(45-2x)=8x^3-420x^2+5\,400x.
		\]
		Ta có $V'(x)=24x^2-840x+5400; V'(x)=0 \Leftrightarrow \hoac{
			&x=\dfrac{35+5\sqrt{13}}{2}\approx 36{,}18\,\,(\text{loại})\\
			&x=\dfrac{35-5\sqrt{13}}{2}\approx 13{,}82\,\,(\text{nhận}).
		}$\\
		Bảng biến thiên
		\begin{center}
			\begin{tikzpicture}
				\tkzTabInit[nocadre,lgt=2.2,espcl=2.5,deltacl=0.5]
				{$x$/1.6,$V'(x)$/0.6,$V(x)$/2}
				{$0$,$\dfrac{35-5\sqrt{13}}{2}$,$\dfrac{45}{2}$}
				\tkzTabLine{,-,0,+,}
				\tkzTabVar{-/$ $,+/$V\left(\dfrac{35-5\sqrt{13}}{2}\right)$,-/$ $}
			\end{tikzpicture}
		\end{center}
		Vậy thể tích lớn nhất của khối hộp là $V\left(\dfrac{35-5\sqrt{13}}{2}\right)\approx 20\,468{,}04 \,\,\rm{cm^3}\approx 20{,}5$ lít.
	}
\end{ex}
\begin{ex}%Câu 6
	Trong không gian $Oxyz$, cho hai điểm $A(5;0;6)$ và $B(3;5;0)$. Điểm $M$ di động trên trục $Oz$, điểm $N$ di động trên trục $Oy$. Độ dài đường gấp khúc $AMNB$ có độ dài nhỏ nhất bằng bao nhiêu? (Kết quả làm tròn đến hàng phần chục).
	
	\shortans[0]{$13{,}5$}
	\loigiai{
		Để tìm độ dài ngắn nhất của đường gấp khúc $AMNB$ ta sẽ \lq\lq trải\rq\rq \, các điểm $A$, $B$ về cùng $1$ mặt phẳng $(Oyz)$ với các điểm $M$, $N$ và thoả mãn đoạn thẳng mới bằng với đoạn thẳng ban đầu (tức
		$AM=A'M; BM=BM'$) và đoạn gấp khúc ngắn nhất khi $4$ điểm trên thẳng hàng.\\
		Ta quay vuông góc mặt phẳng chứa điểm $A(5; 0; 6)$; (tức mặt phẳng màu xanh) xuống mặt phẳng $(Oyz$) ta được điểm $A'(0;-5; 6)$.\\
		Giả sử điểm $M(0; 0; z) \in Oz$. \\
		Suy ra $\heva{&AM=\sqrt{5^2+(z-6)^2}\\& A'M=\sqrt{(-5)^2+(z-6)^2}} \Rightarrow AM=A'M$.\\
		Tương tự, ta quay vuông góc mặt phẳng chứa điểm $B(3; 5; 0)$ (tức mặt phẳng màu hồng) xuống mặt phẳng $(Oyz)$ ta được điểm $B'(0;5;-3)$.\\
		Giả sử điểm $N(0; y; 0) \in Oy$. Suy ra $BN=BN'$.\\
		Ta có độ dài đường gấp khúc $AMNB=A'M+MN+NB'$.\\
		Suy ra $\left(AM'+MN+N'B\right)_{min}$ xảy ra khi $A'$, $M$, $N$, $B'$ thẳng hàng và bằng $A'B'$.
		\[A'B'=\sqrt{(5+5)^2+(-3-6)^2}=\sqrt{181}\approx 13{,}5.\]
	}
\end{ex}
\Closesolutionfile{ans}
% \begin{name}
	{\tenchude}
	{\tendethi}
	{\tentruong}
	{\thoigian}
	\end{name}
\TN
\Opensolutionfile{ans}[ans/de7-ABCD]
\begin{ex}%[1D2H1-2]%Câu 1
	Cho dãy số $(u_n)$ với $u_n=\dfrac{2n}{3n+2}$, $n\in\mathbb{N}^*$. Khẳng định nào sau đây đúng?\vspace{3pt}
	\choice
	{$u_2=1$}
	{\True $u_2=\dfrac{1}{2}$}
	{$u_2=-\dfrac{1}{2}$}
	{$u_2=\dfrac{1}{3}$}
	\loigiai{
		Ta có $(u_n)$ với 
		$\begin{aligned}[t]
			u_n&=\dfrac{2n}{3n+2}, n\in\mathbb{N}^*.
		\end{aligned}$\\
	Suy ra $u_2=\dfrac{2\cdot2}{3\cdot2+2}=\dfrac{4}{8}=\dfrac{1}{2}$.
	}
\end{ex}
\begin{ex}%[2H5H2-3]%Câu 2
	Trong không gian $Oxyz$, cho đường thẳng $d\colon\dfrac{x-2}{2}=\dfrac{y+1}{-1}=\dfrac{z+2}{2}$. Đường thẳng $d$ đi qua điểm nào dưới đây?
	\choice
	{$A(2;-1;2)$}
	{$B(-2;1;2)$}
	{$C(2;1;2)$}
	{\True $D(2;-1;-2)$}
	\loigiai{
		Đường thẳng $d\colon\dfrac{x-2}{2}=\dfrac{y+1}{-1}=\dfrac{z+2}{2}=t$.\\
		Phương trình tham số $d\colon\heva{&x=2+3t\\ &y=-1-t\\ &z=-2+2t.}$\\
		Chọn $t=0$, đường thẳng $d$ đi qua điểm $D(2;-1;-2)$.
	}
\end{ex}
\begin{ex}%[1D6H2-2]%Câu 3
	Với $a$, $b$ là các số thực dương khác $1$ thỏa mãn $\log_a b=\dfrac{3}{2}$. Giá trị của biểu thức $\log_a b^4$ bằng\vspace{3pt}
	\choice
	{$\dfrac{8}{3}$}
	{$3$}
	{$\dfrac{4}{3}$}
	{\True $6$}
	\loigiai{
		Ta có $\log_ab^4=4\log_ab$.\\
		Mặt khác bài cho $\log_ab=\dfrac{3}{2}$.\\
		Suy ra $\log_ab^4=4\cdot\dfrac{3}{2}=6$.
	}
\end{ex}
\begin{ex}%[1H8H7-2]%Câu 4
	Cho khối chóp $S.ABCD$ có chiều cao bằng $5$, đáy $ABCD$ là hình bình hành có diện tích bằng $6$. Thể tích khối chóp $S.ABC$ bằng
	\choice
	{$10$}
	{$15$}
	{\True $5$}
	{$30$}
	\loigiai{
		Ta có công thức thể tích $V_{\text{chóp}}=\dfrac{1}{3}\cdot S_d\cdot h$.\\
		Suy ra $V_{S.ABC}=\dfrac{1}{3}\cdot S_{\triangle ABC}\cdot h$.\\
		Mặt khác ta có $S_{\triangle ABC}=\dfrac{1}{2}S_{ABCD}=\dfrac{1}{2}\cdot6=3$.\\
		Suy ra $V_{S.ABC}=\dfrac{1}{3}\cdot S_{\triangle ABC}\cdot h=\dfrac{1}{3}\cdot3\cdot5=5$.
	}
\end{ex}
\begin{ex}%[2H5H1-3]%Câu 5
	Trong không gian $Oxyz$, mặt phẳng $(\alpha)\colon x+2y+3z-6=0$ cắt trục tung tại điểm có tung độ bằng
	\choice
	{$2$}
	{$6$}
	{\True $3$}
	{$1$}
	\loigiai{
		Trong không gian $Oxyz$, mặt phẳng $(\alpha)\colon x+2y+3z-6=0$ cắt trục tung, suy ra mặt phẳng $(\alpha)$ sẽ cắt trục tung tại điểm $(0;a;0)$
		\begin{eqnarray*}
			 \Rightarrow 1\cdot0+2\cdot a+3\cdot0-6=0 \Leftrightarrow a=3.
		\end{eqnarray*}
	Vậy mặt phẳng $(\alpha)\colon x+2y+3z-6=0$ cắt trục tung tại điểm có tung độ bằng $3$.
	}
\end{ex}
\textbf{\textit{Sử dụng thông tin dưới đây để trả lời câu \ref{câu 6-đề 7} và câu \ref{câu 7-đề 7}}}\\[0.5em]
Cho hàm số $y=f(x)$ có bảng biến thiên như sau
\begin{center}
	\begin{tikzpicture}
		\tikzset{double style/.append style={double distance=1.5pt}}
		\tkzTabInit[nocadre=false,lgt=1.2,espcl=2.5,deltacl=0.6]
		{$x$ /0.6,$y'$ /0.6,$y$ /2}
		{$-\infty$,$-2$,$3$,$+\infty$}
		\tkzTabLine{,+,d,-,$0$,+,}
		\tkzTabVar{-/$5$,+D+/$+\infty$/$4$,-/$0$,+/$+\infty$}
	\end{tikzpicture}
\end{center}
\begin{ex}%[2D1V3-1]%Câu 6
	\label{câu 6-đề 7}
	Giá trị nhỏ nhất của hàm số $y=f(x)$ trên đoạn $[0;6]$ bằng
	\choice
	{\True $0$}
	{$4$}
	{$f(4)$}
	{$f(6)$}
	\loigiai{
		Từ bảng biến thiên, ta thấy giá trị nhỏ nhất của hàm số $y=f(x)$ trên đoạn $\left[0;6\right]$ bằng $0$.
	}
\end{ex}
\begin{ex}%[2D1H4-1]%Câu 7
	\label{câu 7-đề 7}
	Tổng số đường tiệm cận đứng và tiệm cận ngang của đồ thị hàm số $y=f(x)$ là
	\choice
	{\True $2$}
	{$1$}
	{$3$}
	{$0$}
	\loigiai{
		Từ bảng biến thiên, ta có
		$
			\lim\limits_{x\to-\infty}y=5; \lim\limits_{x \to +\infty} y=+\infty
		$
		nên đồ thị hàm số $y=f(x)$ có một tiệm cận ngang.\\
		Ta lại có $\lim\limits_{x \to -2^-} y=+\infty$ nên đồ thị hàm số $y=f(x)$ có một tiệm cận đứng.\\
		Vậy tổng số đường tiệm cận đứng và tiệm cận ngang của đồ thị hàm số $y=f(x)$ là $2$.
	}
\end{ex}
\begin{ex}%[1D6H4-2]%Câu 8
	Tập nghiệm của bất phương trình $3^{2x-3}\le\dfrac{1}{3}$ là
	\choice
	{\True $(-\infty;1]$}
	{$[1;+\infty)$}
	{$(-\infty;2]$}
	{$[2;+\infty)$}
	\loigiai{
		Ta có 
		$\begin{aligned}[t]
			3^{2x-3}\leq\dfrac{1}{3}
			& \Leftrightarrow 3^{2x-3}\leq3^{-1}\\
			& \Leftrightarrow 2x-3\leq-1\\
			& \Leftrightarrow 2x\leq2\\
			& \Leftrightarrow x\leq1.
		\end{aligned}$\\
	Vậy tập nghiệm của bất phương trình là $\left(-\infty;1\right]$.
	}
\end{ex}

\begin{ex}%[2H2H1-2]%Câu 9
	% \immini[thm]{
		Cho hình lập phương $ABCD.A'B'C'D'$ có cạnh bằng $a$. Độ dài của vectơ $\overrightarrow{u}=\overrightarrow{A'C'}-\overrightarrow{A'A}$ bằng
	\choice
	{$\sqrt{2}a$}
	{$\dfrac{\sqrt{3}a}{2}$}
	{$\sqrt{6}a$}
	{\True $\sqrt{3}a$}
	% }{
	% 	\begin{tikzpicture}[line cap=round, line join=round, font=\footnotesize, thick]
	% 		\def\a{2.75}
	% 		\path 	(0:0) coordinate (A)
	% 				++(0:\a) coordinate (D)
	% 				++(-130:\a/2) coordinate (C)
	% 				($(A)+(C)-(D)$) coordinate (B)
	% 				($(A)+(90:\a)$) coordinate (A')
	% 				($(B)+(90:\a)$) coordinate (B')
	% 				($(C)+(90:\a)$) coordinate (C')
	% 				($(D)+(90:\a)$) coordinate (D');
	% 		\draw[dashed,thin] (B)--(A)--(D) (A)--(A');
	% 		\draw[thick] (C)--(C') (D)--(D') (B)--(B') (B)--(C)--(D) (A')--(B')--(C')--(D')--cycle;
	% 		\foreach \x/\y in {A/180,B/180,C/0,D/0,A'/180,B'/180,C'/0,D'/0}
	% 			\fill[black] (\x) circle (1pt) ($(\y:3mm)+(\x)$) node {$\x$};	
	% 	\end{tikzpicture}
	% }
	\loigiai{
		Theo qui tắc ba điểm, ta có $\overrightarrow{u} =\overrightarrow{A'C'}-\overrightarrow{A'A} =\overrightarrow{AC'}$.\\
		Suy ra độ dài $|\overrightarrow{u}| =\left|\overrightarrow{A'C'}-\overrightarrow{A'A}\right| =\left|\overrightarrow{AC'}\right| =AC'$.
		\immini{
			Có $ABCD.A'B'C'D'$ là hình lập phương cạnh $a$ và có $AC'$ là đường chéo của hình lập phương nên ta suy ra
			\begin{align*}
				AC'&=\sqrt{AC^2+CC'^2}=\sqrt{AB^2+AD^2+CC'^2} \\
				&=\sqrt{AB^2+AD^2+AA'^2}=\sqrt{a^2+a^2+a^2}=a\sqrt{3}.
			\end{align*}
		}{
			\begin{tikzpicture}[line cap=round, line join=round, font=\footnotesize, thick, blue, >=stealth]
				\def\a{2.75}
				\path 	(0:0) coordinate (A)
				++(0:\a) coordinate (D)
				++(-130:\a/2) coordinate (C)
				($(A)+(C)-(D)$) coordinate (B)
				($(A)+(90:\a)$) coordinate (A')
				($(B)+(90:\a)$) coordinate (B')
				($(C)+(90:\a)$) coordinate (C')
				($(D)+(90:\a)$) coordinate (D');
				\draw[dashed,thin,->,magenta] (A)--(A');
				\draw[dashed,thin,->,magenta] (A)--(B);
				\draw[dashed,thin,->,magenta] (A)--(D);
				\draw[thick] (C)--(C') (D)--(D') (B)--(B') (B)--(C)--(D) (A')--(B')--(C')--(D')--cycle;
				\foreach \x/\y in {A/180,B/180,C/0,D/0,A'/180,B'/180,C'/0,D'/0}
				\fill[black] (\x) circle (1pt) ($(\y:3mm)+(\x)$) node {$\x$};
				\draw[->,very thick,magenta] (A)--(C');
			\end{tikzpicture}
		}
		
	}
\end{ex}
\begin{ex}%[1D5H2-3]%Câu 10
	Một vườn thú ghi lại tuổi thọ (đơn vị: năm) của $20$ con hổ và thu được kết quả như sau
	\begin{center}
		\begin{tabular}{|c|c|c|c|c|c|}\hline
			Tuổi thọ & $[14;15)$ & $[15;16)$ & $[16;17)$ & $[17;18)$ & $[18;19)$ \\ \hline
			Số con hổ & $1$ & $3$ & $8$ & $6$ & $2$ \\ \hline
		\end{tabular}
	\end{center}
	Nhóm chứa tứ phân vị thứ nhất là
	\choice
	{$[14;15)$}
	{$[15;16)$}
	{\True $[16;17)$}
	{$[17;18)$}
	\loigiai{
		Số con hổ được khảo sát là $n=20$.\\
		Gọi $x_1,x_2,\ldots,x_{20}$ là tuổi thọ của $20$ con hổ được sắp xếp theo thứ tự không giảm.\\
		Ta có 
		$\begin{aligned}[t]
			x_1 & \in \left[14;15\right); \\
			x_2,x_3,x_4 & \in \left[15;16\right); \\
			x_5,x_6,\ldots,x_{12} & \in \left[16;17\right); \\
			x_{13},x_{14},\ldots,x_{18} & \in \left[17;18\right); \\
			x_{19},x_{20} & \in \left[18;19\right).
		\end{aligned}$\\
	Do đó đối với dãy số liệu $x_1, x_2, \ldots, x_{20}$ thì tứ phân vị thứ nhất của dãy số $x_1, x_2, \ldots, x_{20}$ là $\dfrac{1}{2}\left(x_5+x_6\right)$.\\
	Do đó $x_5, x_6$ thuộc nhóm $\left[16;17\right)$ nên tứ phân vị thứ nhất thuộc nhóm $\left[16;17\right)$.
	}
\end{ex}
\begin{ex}%[2D4V3-4]%Câu 11

		Cho hình phẳng $(H)$ giới hạn bởi đồ thị hàm số $y=4-x^2$ và trục hoành. Thể tích của khối tròn xoay được tạo thành khi quay $(H)$ xung quanh trục $Ox$ bằng\vspace{3pt}
	\choice
	{$\dfrac{32\pi}{3}$}
	{$\dfrac{512}{15}$}
	{\True $\dfrac{512\pi}{15}$}
	{$\dfrac{32}{3}$}
	
	\loigiai{
		\immini{
		Ta có phương trình hoành độ giao điểm của $y=4-x^2$ và trục hoành $y=0$
		$$4-x^2=0 \Leftrightarrow x=\pm2.$$
		Thể tích khối tròn xoay được tạo thành khi quay $(H)$ xung quanh trục $Ox$ là
		\begin{align*}
			V=\pi\displaystyle\int\limits_{-2}^2 \left(4-x^2\right)^2 \mathrm{\,d}x =\dfrac{512}{15}\pi.
		\end{align*}
		}{
		\begin{tikzpicture}[scale=0.8,>=stealth, font=\footnotesize, line join=round, line cap=round, thick]
			\def\a{-1} \def\b{0} \def\c{4} % Hệ số
			\def\xmin{-3} \def\xmax{3}
			\def\ymin{-1} \def\ymax{5}
			\fill[gray!60] plot[domain=-2:2](\x,{\a*(\x)^2+\b*(\x)+\c})--cycle;
			\draw[->] (\xmin,0)--(\xmax,0) node [below]{$x$};
			\draw[->] (0,\ymin)--(0,\ymax) node [left]{$y$};
			\node at (0,0) [below left]{$O$};
			\clip (\xmin+0.1,\ymin+0.1) rectangle (\xmax-0.1,\ymax-0.1);
			\draw[smooth,samples=300] plot(\x,{\a*(\x)^2+\b*(\x)+\c});
			\draw (0,4)node[above right]{$y=4-x^2$} (-0.75,1)node[]{$(H)$};
		\end{tikzpicture}
	}
	}
\end{ex}
\begin{ex}%[2D4H1-2]%Câu 12
	Cho hàm số $f(x)$ liên tục trên $\mathbb{R}$. Biết $F(x)$ là nguyên hàm của hàm số $f(x)$ thỏa mãn $F(2)=2$ và $F(x)=\displaystyle\int \left[x-f(x)\right] \mathrm{\,d}x$, $\forall x\in\mathbb{R}$, Giá trị của $F(4)$ bằng
	\choice
	{\True $5$}
	{$6$}
	{$8$}
	{$9$}
	\loigiai{
		Ta có 
		$\begin{aligned}[t]
			F(x)=\displaystyle\int x \mathrm{\,d}x -\displaystyle\int f(x) \mathrm{\,d}x
			=\dfrac{x^2}{2}+C-F(x) \Leftrightarrow 2F(x) =\dfrac{x^2}{2}+C.
		\end{aligned}$\\
	Tại $x=2$ ta có $2F(2)=2+C \Leftrightarrow C=2$.\\
	Suy ra $x=4$ ta được $2F(4)=\dfrac{4^2}{2}+C \Leftrightarrow 2F(4)=\dfrac{4^2}{2}+2 \Leftrightarrow 2F(4)=10 \Leftrightarrow F(4)=5$.
	}
\end{ex}

\Closesolutionfile{ans}

\inputansbox{6}{ans/de7-ABCD}

\TNTF

\Opensolutionfile{ans}[ans/de7-DS]

\begin{ex}%[1D7H2-1]%Câu 1
	Cho hàm số $f(x)=\ln\left(x^2-2x+1\right)-x$. Xét tính đúng sai của các mệnh đề sau
	\choiceTF
	{Tập xác định của hàm số là $\mathscr{D}=\mathbb{R}$}
	{\True Đạo hàm của hàm số $f(x)$ trên tập xác định của nó là $f'(x)=\dfrac{2}{x-1}-1$}
	{\True Số nghiệm của phương trình $f'(x)=0$ trên khoảng $(1;+\infty)$ là $1$}
	{Giá trị cực đại của $f(x)$ trên khoảng $(1;+\infty)$ là $a\ln 2+b$ với $a$, $b\in\mathbb{Z}$ thì $a+b=1$}
	\loigiai{
		\begin{itemchoice}
			\itemch Điều kiện $x^2-2x+1>0 \Leftrightarrow (x-1)^2>0 \Leftrightarrow x\ne1$.\\
			Suy ra tập xác định của hàm số là $\mathscr{D}=\mathbb{R}\setminus\{1\}$.
			\itemch Áp dụng công thức $\left(\ln u\right)'=\dfrac{u'}{u}$.\\
			Ta có $f(x)=\ln\left(x^2-2x+1\right)-x $.\\
			$\Rightarrow f'(x)=\dfrac{(x^2-2x+1)'}{x^2-2x+1}-1 =\dfrac{2x-2}{x^2-2x+1}-1 =\dfrac{2(x-1)}{(x-1)^2}-1 =\dfrac{2}{x-1}-1$.
			\itemch Ta có $f'(x)=0 \Leftrightarrow \dfrac{2}{x-1}-1=0 \Leftrightarrow x-1=2 \Leftrightarrow x=3$.\\
			Suy ra $f'(x)=0$ có nghiệm duy nhất.
			\itemch Lập bảng biến thiên $ \Rightarrow x=3$ là điểm cực đại của hàm số.\\
			Do đó $y_{CD}=f(3)=\ln4-3=2\ln2-3=a\ln2+b$.\\
			Suy ra $a=2, b=-3$ nên $a+b=2+(-3)=-1$.
		\end{itemchoice}
	}
\end{ex}

\begin{ex}%[2D5V1-2]%[2D5V1-3]%Câu 2
	Một nhà máy sản xuất bóng đèn có tỷ lệ bóng đèn đạt tiêu chuẩn là $82\%$. Trước khi xuất ra thị trường, mỗi bóng đèn được sản xuất ra đều phải qua một khâu kiểm tra chất lượng tự động. Vì sự kiểm tra này không chính xác tuyệt đối nên một bóng đèn tốt chỉ có xác suất $92\%$ được công nhận và một bóng đèn hỏng có xác suất $96\%$ được loại bỏ.
	\begin{itemize}
		\item Gọi $A$ là biến cố \lq\lq Bóng được công nhận đạt tiêu chuẩn sau khi qua kiểm tra chất lượng\rq\rq.
		\item Gọi $B$ là biến cố \lq\lq Sản phẩm đạt tiêu chuẩn\rq\rq.
	\end{itemize}
	Xét tính đúng sai của các mệnh đề sau
	\choiceTF[t]
	{$\mathrm{P}(B)=0{,}18$ ; $\mathrm{P}(\overline{B})=0{,}82$}
	{Xác suất có điều kiện $\mathrm{P}(A|\overline{B})=0{,}92$}
	{\True Tỉ lệ bóng được công nhận đạt tiêu chuẩn sau khi qua kiểm tra chất lượng là $76{,}16\%$}
	{Tỷ lệ bóng đèn tốt trong số những bóng đèn được công nhận là $98{,}01\%$ (kết quả làm tròn đến hàng phần trăm)}
	\loigiai{
		\textbf{Chú ý:} Bóng đèn đạt tiêu chuẩn là bóng đèn tốt và bóng đèn không đạt tiêu chuẩn là bóng đèn hỏng.
		\begin{itemchoice}
			\itemch Theo bài ra ta có $\mathrm{P}(B)=0{,}82$; $\mathrm{P}(\overline{B})=1-0{,}82=0{,}18$.
			\itemch Do tỉ lệ công nhận bóng đèn đạt tiêu chuẩn là $0{,}92$ nên $\mathrm{P}(A|B)=0{,92}$.\\
			Tie lệ loại bỏ một bóng đèn hỏng là $0{,}96$ nên $\mathrm{P}(A|\overline{B})=1-0{,}92=0{,}04$.\\
			Theo sơ đồ hình cây
			\itemch Theo công thức xác suất toàn phần ta có $$\mathrm{P}(A)=\mathrm{P}(B)\cdot\mathrm{P}(A|B)+\mathrm{P}(\overline{B})\cdot\mathrm{P}(A|\overline{B}) =0{82\cdot0{,}92}+0{,}18\cdot0{,}04 =0{,}7616.$$
			\itemch Tỉ lệ bóng đèn tốt trong số những bóng đèn được công nhận là $$\dfrac{0{,}82\cdot0{,}92}{0{,}82\cdot0{,}92+0{,}18\cdot0{,}04}=99{,}05.$$
		\end{itemchoice}
	}
\end{ex}

\begin{ex}%[2H2C2-6]%Câu 3
	\immini[thm]{
		Xét hai chiếc khinh khí cầu bay lên từ cùng một điểm trong cùng một ngày. Lúc $9$ giờ sáng, chiếc thứ nhất đang ở vị trí $A$ cách điểm xuất phát $2$ km về phía nam và $1$ km về phía đông, đồng thời cách mặt đất $0{,}5$ km. Chiếc thứ hai đang ở vị trí $B$ nằm cách điểm xuất phát $1$ km về phía bắc và $1{,}5$ km về phía tây đồng thời cách mặt đất $0{,}8$ km. Chọn hệ trục tọa độ $Oxyz$ với gốc $O$ đặt tại điểm xuất phát của hai khinh khí cầu, mặt phẳng $(Oxy)$ trùn với mặt đất, trục $Ox$ hướng về phía nam, trục $Oy$ hướng về phía đông và trục $Oz$ hướng thẳng đứng lên trời (như hình vẽ). Lấy đơn vị đo trên mỗi trục là km.
	}{\begin{tikzpicture}[scale=1,font=\footnotesize,>=stealth, line join=round, line cap=round]
		\def\xmin{-2} \def\xmax{4}
		\def\ymin{-2} \def\ymax{2}
		\def\zmax{3}
		\draw[->] (\xmin,0)--(\xmax,0) node [below]{$x$};
		\draw[->] (\ymax,\ymax)--(\ymin,\ymin)node [above]{$y$};
		\draw[->] (0,0)--(0,\zmax) node [left]{$z$};
		\draw (\xmax,0)node[above]{Nam} (\xmin,0)node[above]{Bắc} (\ymax,\ymax)node[above]{Tây} (\ymin,\ymin)node[below]{Đông};
		\node at (0,0) [below]{$O$};
		\coordinate (O) at (0,0);
		\coordinate (E) at (3,0);
		\coordinate (F) at (-1,-1);
		\coordinate (G) at ($(E)+(F)-(O)$);
		\coordinate (A) at ($(G)+(0,1.7)$);
		\draw[dashed] (F)--(G)--(E) (O)--(G) (O)--(A)node[above,blue,font=\fontsize{25pt}{2pt}\selectfont]{\faFly}--(G);
		\fill (A)circle(2pt);
		\coordinate (M) at (-1.5,0);
		\coordinate (N) at (0.5,0.5);
		\coordinate (P) at ($(M)+(N)-(O)$);
		\coordinate (B) at ($(P)+(0,1.5)$);
		\draw[dashed] (M)--(P)--(N) (O)--(P) (O)--(B)node[above,magenta,font=\fontsize{25pt}{2pt}\selectfont]{\faFly}--(P);
		\fill (B)circle(2pt);
	\end{tikzpicture}}
	\choiceTF
	{\True Tọa độ của khinh khí cầu thứ nhất lúc $9$ giờ sáng là $A(2;1;0{,}5)$}
	{Phương trình chính tắc của đường thẳng $AB$ là $\dfrac{x-2}{30}=\dfrac{y-1}{25}=\dfrac{z-0{,}5}{3}$}
	{\True Lúc $9$ giờ sáng, khoảng cách giữa hai chiếc khinh khí cầu là $3{,}92$ km (làm tròn đến hàng phần trăm)}
	{Từ $9$ giờ sáng đến $9$ giờ $10$ phút sáng, khinh khí cầu thứ nhất đi thẳng về hướng Nam với vận tốc $50$ km/h và độ cao không đổi để đến điểm $M$, khinh khí cầu thứ hai chuyển động thẳng đến điểm $N$ với vận tốc $60$ km/h, biết vectơ $\overrightarrow{BN}$ cùng hướng với vectơ $\overrightarrow{u}=(2;2;1)$. Bỏ qua lực cản của gió, khoảng cách $MN$ là $4{,}66$ km (làm tròn đến hàng phần trăm)}
	\loigiai{
		\begin{itemchoice}
			\itemch Vị trí của khinh khí cầu thứ nhất lúc $9$ giờ sáng là $A(x_A;y_A;z_A)$ biết
			\begin{itemize}
				\item $A$ cách điểm xuất phát $2$ km về phía Nam $ \Rightarrow x_A=2$.
				\item $1$ km về phía Đông $ \Rightarrow y_A=1$.
				\item Cách mặt đất $0{,}5$ km $ \Rightarrow z_A=0{,}5$.
			\end{itemize}
		Vậy $A(2;1;0{,}5)$.
			\itemch Dựa vào giả thiết, ta có $B(-1;-1{,}5;0{,}8)$.\\
			Suy ra $\overrightarrow{AB}=(-3;-2{,}5;0{,3})$.\\
			Đường thẳng $AB$ đi qua điểm $A(2;1;0{,}5)$ và có vec-tơ chỉ phương $\overrightarrow{u}=(30;25;-3)$ có phương trình chính tắc là $\dfrac{x-2}{30}=\dfrac{y-1}{25}=\dfrac{x-0{,}5}{-3}$.
			\itemch Lúc $9$ giờ sáng, khoảng cách giữa hai kinh khí cầu bằng $$AB=\sqrt{(-3)^2+(-2{,}5)^2+(0{,}3)^2}\approx3{,}92\, (\rm km).$$
			\itemch Để tính được khoảng cách $MN$ ta cần tìm được toạ độ của điểm $M$ và $N$ lúc $9$ giờ $10$ phút sáng
			\begin{itemize}
				\item Tìm điểm $M(x_M;y_M;z_M)$.\\
				Ta có khinh khí cầu thứ nhất di chuyển thẳng đều về phía Nam (tức hoành độ của khinh khí cầu không đổi $ \Leftrightarrow y_M=1$) với vận tốc $50\,\rm km/h$ suy ra từ $9$ giờ sáng đến $9$ giờ $10$ phút sáng, khinh khí cầu đi được $50\cdot\dfrac{1}{6}=\dfrac{25}{3}$ (km).\\
				$ \Rightarrow x_M=2+\dfrac{25}{3}=\dfrac{31}{3}$.\\
				Và cao độ không đổi $ \Rightarrow z_M=0{,}5$.\\
				Vậy điểm $M(\dfrac{31}{3};1;0{,}5)$.
				\item Tìm điểm $N(x_N;y_N;z_N)$\\
				Vì vectơ $\overrightarrow{BN}$ cùng hướng với vectơ $\overrightarrow{u}$ nên suy ra $\overrightarrow{BN}=k(2;2;1)\, (k>0)$.\\
				Khinh khí cầu thứ hai chuyển động thẳng đều đến điểm $N$ với vận tốc $60\, \rm km/h$, nên trong khoảng thời gian từ $9$ giờ sáng đến $9$ giờ $10$ phút sáng khinh khí cầu đi được $1$ đoạn $BN=60\cdot\dfrac{1}{6}=10 \Leftrightarrow k|(2;2;1)|=10 \Leftrightarrow k\sqrt{2^2+2^1+1^1}=10 \Leftrightarrow k=\dfrac{10}{3}$.\\
				$\overrightarrow{BN}=\left(x_N+1;y_N+1;z_N-0{,}8\right) =\dfrac{10}{3}(2;2;1) =\left(\dfrac{20}{3};\dfrac{20}{3};\dfrac{10}{3}\right) \Rightarrow \heva{&x_N=\dfrac{17}{3}\\ &y_N=\dfrac{31}{6}\\ &z_N=\dfrac{62}{15}.}$\\
				$ \Rightarrow N\left(\dfrac{17}{3};\dfrac{31}{3};\dfrac{62}{15}\right)$.\\
				Suy ra $MN=\sqrt{\left(\dfrac{17}{3}-\dfrac{31}{3}\right)^2+\left(\dfrac{31}{3}-1\right)^2+\left(\dfrac{62}{15}-0{,}5\right)^2} \approx 7{,}23$.
			\end{itemize}
		\end{itemchoice}
	}
\end{ex}

\begin{ex}%[2D4C3-2]%Câu 4
	\immini[thm]{
		Hình vẽ bên mô tả hiệu suất làm việc của hai công nhân trong một nhà máy trong thời gian $6$ giờ. Công nhân $A$ đang sản xuất với hiệu suất $Q'_1(t)=-2t^2+4t+58$ sản phẩm mỗi giờ, trong khi công nhân $B$ đang sản xuất với hiệu suất $Q'_2(t)=53+at$ sản phẩm mỗi giờ $(a\in\mathbb{R})$. Biết rằng hàm $Q_1(t)$ và $Q_2(t)$ mô phỏng số lượng sản phẩm mới làm được của công nhân $A$ và công nhân $B$ sau $t$ giờ. Xét tính đúng sai của các mệnh đề sau
	}{
		\begin{tikzpicture}[>=stealth, font=\footnotesize, line join=round, line cap=round,xscale=1,yscale=0.1,scale=0.48]
			\def\xmin{-1} \def\xmax{8}
			\def\ymin{-10} \def\ymax{70}
			\draw[->] (\xmin,0)--(\xmax,0) node [below]{$t$ (h)};
			\draw[->] (0,\ymin)--(0,\ymax) node [left]{$Q'(t)$};
			\node at (0,0) [below left]{$O$};
			\draw[dashed] (5,0)node[below]{$5$}--(5,28) (6,0)node[below]{$6$}--(6,10)node[right]{$Q'_1(t)$}--(6,23)node[right]{$Q'_2(t)$};
			\clip (\xmin+0.1,\ymin+0.1) rectangle (\xmax-0.5,\ymax-0.1);
			\draw[smooth,samples=300,domain=0:6] plot(\x,{-2*(\x)^2+4*(\x)+58});
			\draw[smooth,samples=300,domain=0:6] plot(\x,{-5*(\x)+53});
			\fill[gray] plot[domain=0:6](\x,{-2*(\x)^2+4*(\x)+58})--plot[domain=6:0](\x,{-5*(\x)+53})--cycle;
		\end{tikzpicture}
	}
	\choiceTF
	{\True Hiệu suất cực đại của công nhân $A$ là $60$ sản phẩm mỗi giờ}
	{Phần diện tích tô đậm biểu diễn cho tổng số lượng sản phẩm mới mà $2$ công nhân làm được trong $6$ giờ}
	{\True Sau $5$ giờ số lượng sản phẩm mới mà công nhân $A$ hoàn thành nhiều hơn công nhân $B$ là $54$ sản phẩm (kết quả làm tròn đến hàng đơn vị)}
	{Sau $6$ giờ làm việc tổng số lượng sản phẩm mới mà $2$ công nhân hoàn thành là $502$ sản phẩm}
	\loigiai{
		\begin{itemchoice}
			\itemch Ta có hiệu suất của công nhân $A$ là\\ $Q_1'(t)=-2t^2+4t+58 =-2(t^2-2t+1)+2+58 =-2(t-1)^2+60 \leq60$, $\forall t\in\left[0;6\right]$.\\
			Dấu \lq\lq$=$\rq\rq\, xảy ra khi $t=1$.\\
			Vậy hiệu suất cực đại của công nhân $A$ là $60$ sản phẩm mỗi giờ.
			\itemch Diện tích tô đậm bằng $\displaystyle\int\limits_0^6 \big|Q_1'-Q_2'\big| \mathrm{\,d}t$ nên nó không biểu diễn cho tổng số lượng sản phẩm mới mà 2 công nhân làm được trong $6$ giờ.
			\itemch Dựa vào biểu đồ, ta có 
			$$\begin{aligned}[t]
				Q_1'(5)=Q_2'(5) & \Leftrightarrow -2\cdot5^2+4\cdot5+58=53+5a\\
				& \Leftrightarrow a=-5.
			\end{aligned}$$
		Suy ra $Q_2'(5)=53-5t$.\\
		Sau 5 giờ, số lượng sản phẩm mới của công nhân $A$ hoàn thành nhiều hơn số lượng sản phẩm mới của công nhân $B$ bằng 
		\begin{align*}
			\displaystyle\int\limits_0^5 Q_1'(t) \mathrm{\,d}t -\displaystyle\int\limits_0^5 Q_2'(t) \mathrm{\,d}t =\displaystyle\int\limits_0^5 (-2t^2+4t+58) \mathrm{\,d}t -\displaystyle\int\limits_0^5 (53-5t) \mathrm{\,d}t =\dfrac{325}{6} \approx 54.
		\end{align*}
			\itemch Sau $6$ giờ làm việc tổng số lượng sản phẩm mới 2 công nhân hoàn thành bằng 
		\begin{align*}
			\displaystyle\int\limits_0^5 Q_1'(t) \mathrm{\,d}t +\displaystyle\int\limits_0^5 Q_2'(t) \mathrm{\,d}t =\displaystyle\int\limits_0^5 (-2t^2+4t+58) \mathrm{\,d}t +\displaystyle\int\limits_0^5 (53-5t) \mathrm{\,d}t =\dfrac{770}{3}+\dfrac{405}{2}= \dfrac{2755}{6} \approx 459.
		\end{align*}
		\end{itemchoice}
	}
\end{ex}

\Closesolutionfile{ans}

\inputansbox{3}{ans/de7-DS}

\TNSA

\Opensolutionfile{ans}[ans/de7-KQ]

\begin{ex}%[1H8V6-2]%Câu 1
	Cho hình chóp tam giác đều $S.ABC$ có $AB=2$, $SA=3$. Gọi $\alpha$ là số đo của góc nhị diện $[S,BC,A]$. Giá trị $\tan\alpha$ bằng bao nhiêu? (làm tròn kết quả đến hàng phần mười).
	\shortans[oly]{$4{,}8$}
	\loigiai{
		\begin{center}		
			\begin{tikzpicture}[line cap=round, line join=round, font=\footnotesize, thick, blue]
				\def\a{5} \def\h{3.5}
				\path 	(0:0) coordinate (A)
						(-45:2) coordinate (B)
						(0:\a) coordinate (C)
						($(C)!0.5!(B)$) coordinate (H)
						($(A)!2/3!(H)$) coordinate (O)
						($(O)+(90:\h)$) coordinate (S);
				\draw[] (S)--(A)--(B)--(C)--cycle (B)--(S)--(H);
				\draw[dashed,thin] (C)--(A)--(H) (S)--(O);
				\foreach \x/\y in {A/180,B/-90,C/0,H/-60,O/-90,S/90}
				\fill (\x) circle (1pt) ($(\x)+(\y:3mm)$) node {$\x$};
				\pic[draw,angle radius=2.5mm] {right angle=A--H--B};
				\pic[draw,angle radius=2mm] {right angle=S--O--H};
				\path (A)--(B) node[below,midway]{$2$}
				(A)--(S) node[left,midway]{$3$};
			\end{tikzpicture}
		\end{center}
	Kẻ $AH\perp BC$ (1)\\
	Gọi $O$ là tâm của $\triangle ABC$.\\
	Suy ra $SO\perp(ABC)$ (theo tính chất của hình chóp đều).\\
	$ \Rightarrow SO\perp BC$ (2)\\
	Từ (1) và (2) suy ra $BC\perp(SHA)$ $ \Rightarrow \left[S,BC,A\right] =\widehat{SHO}$.\\
	Vì $\triangle ABC$ đều nên ta có $AH=\sqrt{3} \Rightarrow \heva{&OH=\dfrac{\sqrt{3}}{3}\\ &AO=\dfrac{2\sqrt{3}}{3}.}$\\
	Suy ra $SO=\sqrt{3^2-\left(\dfrac{2\sqrt{3}}{3}\right)^2} =\dfrac{\sqrt{69}}{3}$.\\
	$ \Rightarrow \tan\alpha=\tan\widehat{SHO}=\dfrac{SO}{OH}=\dfrac{\sqrt{69}}{3}\div\dfrac{\sqrt{3}}{3} =\sqrt{23} \approx 4{,8}$.
	}
\end{ex}

\begin{ex}%[2D5V1-4]%Câu 2
	Một hộp đựng $12$ bóng đèn, các bóng đèn trong cùng hộp thì cùng màu. Số hộp đựng bóng đèn màu xanh nhiều gấp $9$ lần số hộp đựng bóng đèn màu vàng. Trong mỗi hộp đựng bóng đèn màu xanh có $3$ bóng bị hỏng, mỗi hộp đựng bóng đèn màu vàng có $2$ bóng bị hỏng. Tính xác suất để lấy ra hai bóng đèn màu xanh ở cùng một hộp, biết cả hai bóng đều bị hỏng. Viết kết quả làm tròn đến hàng phần trăm.
	\shortans[oly]{$0{,}96$}
	\loigiai{
		Gọi $A_1$ là biến cố lấy được một hộp đựng bóng đèn màu vàng.\\
		Suy ra $\mathrm{P}(A_1)=\dfrac{1}{1+9}=\dfrac{1}{10}$.\\
		Gọi $A_2$ là biến cố lấy được một hộp đựng bóng đèn màu xanh.\\
		Suy ra $\mathrm{P}(A_2)=\dfrac{9}{1+9}=\dfrac{9}{10}$.\\
		Gọi $B$ là biến cố lấy được hai bóng đèn hỏng ở cùng 1 hộp.\\
		Ta có xác suất lấy được 2 bóng đèn hỏng từ một hộp đựng bóng đèn vàng là $\mathrm{P}(B|A_1)=\dfrac{\mathrm{C}_2^2}{\mathrm{C}_{12}^2}$ (vì trong mỗi hộp đựng bóng đèn vàng có
		2 bóng bị hỏng).\\
		Tương tự, vì trong mỗi hộp đựng bóng đèn màu xanh có 3 bóng bị hỏng nên xác suất lấy được 2 bóng đèn hỏng từ một hộp đựng bóng đèn xanh là $\mathrm{P}(B|A_2)=\dfrac{\mathrm{C}_3^2}{\mathrm{C}_{12}^2}$.\\
		Ta có sơ đồ hình cây sau\\
		Ta có 
		$\mathrm{P}(B)=\mathrm{P}(A_1)\cdot\mathrm{P}(B|A_1)+\mathrm{P}(A_2)\cdot\mathrm{P}(B|A_2) =\dfrac{1}{10}\cdot\dfrac{\mathrm{C}_2^2}{\mathrm{C}_{12}^2} +\dfrac{9}{10}\cdot\dfrac{\mathrm{C}_3^2}{\mathrm{C}_{12}^2} =\dfrac{7}{165}$.\\
		Suy ra $\mathrm{P}(A_2|B)=\dfrac{\mathrm{P}(A_2)\cdot\mathrm{P}(B|A_2)}{\mathrm{P}(B)} =\dfrac{\dfrac{9}{10}\cdot\dfrac{\mathrm{C}_3^2}{\mathrm{C}_{12}^2}}{\dfrac{7}{167}} =\dfrac{27}{28} \approx 0{,}96$.
	}
\end{ex}

\begin{ex}%[1C2C3-1]%Câu 3
	Trên đường Mạnh đi từ nhà $(M)$ đến công ty $(C)$ có điểm $A$ người ta đang thi công sửa chữa đường nên không thể đi qua $A$.
	\begin{center}
		\begin{tikzpicture}[scale=0.9,>=stealth, font=\footnotesize, line join=round, line cap=round]
			\coordinate (M) at (0,0);
			\coordinate (A) at (3,1);
			\coordinate (C) at (6,4);
			\draw (M)node[above right]{$M$}--(6,0)--(C)node[above right]{$C$}--(0,4)--cycle (0,1)--(6,1) (0,2)--(3,2) (3,3)--(4,3) (1,0)--(1,2) (2,0)--(2,2) (A)node[below,red]{$A$}--(3,4) (4,1)--(4,4) (5,0)--(5,1);
			\draw[->] (-0.3,0)--(-0.3,1);
			\draw[->] (0,-0.3)--(1,-0.3);
			\fill[red] (A)circle(2pt);
			\fill (C)circle(2pt);
			\fill (M)circle(2pt);
		\end{tikzpicture}
	\end{center}
	Biết rằng toàn bộ cung đường theo bản đồ từ dưới lên trên và từ trái qua phải là đường một chiều vì vậy Mạnh chỉ được phép đi lên hoặc đi sang phải. Vậy Mạnh có bao nhiêu cách đến công ty?
	\shortans[oly]{$15$}
	\loigiai{
		Số cách Mạnh đến công ty là $15$ cách.\\
		Minh hoạ
		\begin{center}
			\begin{tikzpicture}[scale=0.9,>=stealth, font=\footnotesize, line join=round, line cap=round, thick]
				\coordinate (M) at (0,0);
				\coordinate (A) at (3,1);
				\coordinate (C) at (6,4);
				\draw (M)node[above right]{$M$}--(6,0)--(C)node[ right]{$C$}--(0,4)--cycle (0,1)--(6,1) (0,2)--(3,2) (3,3)--(4,3) (1,0)--(1,2) (2,0)--(2,2) (A)node[below,red]{$A$}--(3,4) (4,1)--(4,4) (5,0)--(5,1);
				\draw[->] (-0.185,0)--(-0.185,1);
				\draw[->] (0,-0.185)--(1,-0.185);
				\fill[red] (A)circle(2pt);
				\fill (C)circle(2pt);
				\fill (M)circle(2pt);
				\path 
				(1,0) node[below]{$1$} 			(2,0) node[below]{$1$} 
				(5,0) node[below]{$1$} 			(6,0) node[below]{$1$}
				(0,1) node[left]{$1$} 			(0,2) node[left]{$1$} 
				(0,4) node[left]{$1$}			(5,1) node[above]{$1$}
				(1,1) node[above right]{$2$} 	(6,1) node[above right]{$2$}
				(1,2) node[above]{$3$}			(2,1) node[above right]{$3$}
				(2,2) node[above]{$6$}			(3,2) node[above left]{$6$}
				(3,3) node[left]{$6$}			(4,3) node[right]{$6$}
				(3,4) node[above]{$7$}
				(4,4) node[above]{$13$}
				(6,4) node[above]{$15$}
				;
			\end{tikzpicture}
		\end{center}
	}
\end{ex}

\begin{ex}%[2D4V3-5]%Câu 4
	\immini[thm]{
		Bên trong hình vuông cạnh $4$, dựng hình sao bốn cạnh đều như hình vẽ bên (các kích thước cần thiết cho như ở trong hình vẽ).
		Tính thể tích $V$ của khối tròn xoay sinh ra khi quay hình sao đó quanh trục $Ox$ (làm tròn kết quả đến hàng phần mười).
	}{
		\begin{tikzpicture}[scale=0.8,>=stealth, font=\footnotesize, line join=round, line cap=round]
			\def\xmin{-3} \def\xmax{3}
			\def\ymin{-3} \def\ymax{3}
			\fill[gray!70] (-2,2)--(0,1)--(2,2)--(1,0)--(2,-2)--(0,-1)--(-2,-2)--(-1,0)--cycle;
			\draw[->] (\xmin,0)--(\xmax,0) node [below]{$x$};
			\draw[->] (0,\ymin)--(0,\ymax) node [left]{$y$};
			\node at (0,0) [below left]{$O$};
			\draw[dashed] (-2,2)--(2,2)--(2,-2)--(-2,-2)--cycle (-2,0)node[below left]{$-2$} (2,0)node[below right]{$2$} (0,2)node[above left]{$2$} (0,-2)node[below left]{$-2$};
			\draw (-2,2)--(0,1)node[above left,yshift=0.1cm]{$1$}--(2,2)--(1,0)node[below right,xshift=0.1cm]{$1$}--(2,-2)--(0,-1)node[below left,yshift=-0.1cm]{$-1$}--(-2,-2)--(-1,0)node[below left,xshift=-0.1cm]{$-1$}--cycle;
		\end{tikzpicture}
	}
	\shortans[0ly]{$20{,}9$}
	\loigiai{
		\begin{center}
			\begin{tikzpicture}[scale=1,>=stealth, font=\footnotesize, line join=round, line cap=round, thick, blue]
				\def\xmin{-3} \def\xmax{3} \def\ymin{-3} \def\ymax{3}
				\path 
					(-2,2) coordinate (A)
					(2,2) coordinate (B)
					(2,-2) coordinate (C)
					(-2,-2) coordinate (D)
					(0,1) coordinate (M)
					(1,0) coordinate (N)
					(0,-1) coordinate (P)
					(-1,0) coordinate (Q);
				\fill[gray!50] (-2,2)--(0,1)--(2,2)--(1,0)--(2,-2)--(0,-1)--(-2,-2)--(-1,0)--cycle;
				\draw[->] (\xmin,0)--(\xmax,0) node [below]{$x$};
				\draw[->] (0,\ymin)--(0,\ymax) node [left]{$y$};
				\node at (0,0) [below left]{$O$};
				\draw[dashed,thin] (-2,2)--(2,2)--(2,-2)--(-2,-2)--cycle (-2,0)node[below left]{$-2$} (2,0)node[below right]{$2$} (0,2)node[above left]{$2$} (0,-2)node[below left]{$-2$};
				\draw (-2,2)--(0,1)node[above left,yshift=0.1cm]{$1$}--(2,2)--(1,0)node[below right,xshift=0.1cm]{$1$}--(2,-2)--(0,-1)node[below left,yshift=-0.1cm]{$-1$}--(-2,-2)--(-1,0)node[below left,xshift=-0.1cm]{$-1$}--cycle;
				\foreach \x/\y in {A/135,B/45,C/-45,D/225,M/-45,N/225,P/45,Q/-45}
				\fill[black] (\x) circle (1pt) ($(\y:3mm)+(\x)$) node {$\x$};
			\end{tikzpicture}
		\end{center}
		Ta kí hiệu các điểm như hình vẽ.\\
		Ta có khối tròn xoay đó được tạo thành khi quay hình phẳng $QAMBN$ quanh trục $Ox$.\\
		Mà $S_{OQAM}=S_{ONBM}$ nên thể tích của khối tròn xoay đó sẽ bằng 2 lần thể tích của khối tròn xoay khi quay hình phẳng $ONBM$ quanh trục $Ox$.\\
		Suy ra ta có thể tích $V=2\left(\pi\displaystyle\int\limits_0^2 MB^2 \mathrm{\,d}x -\pi\displaystyle\int\limits_1^2 NB^2 \mathrm{\,d}x\right)$.
		\begin{itemize}
			\item[+)] Viết phương trình đường thẳng $MB$, với $M(0;1)$, $B(2;2)$.\\
			Có vectơ chỉ phương $\overrightarrow{MB}=(2;1)$ suy ra một vectơ pháp tuyến của đường thẳng là $\overrightarrow{n}_1=(-1;2)$.\\
			Suy ra $MB\colon -1\cdot(x-0)+2\cdot(y-1)=0 \Rightarrow -x+2y-2=0 \Rightarrow y=\dfrac{1}{2}x+1$.
			\item[+)] Tương tự, ta viết được phương trình đường thẳng $NB$ là\\
			$NB\colon -2\cdot(x-1)+1\cdot(y-0)=0 \Leftrightarrow -2x+y+2=0 \Rightarrow y=2x-2$
		\end{itemize}
		Thể tích là $V=2\left(\pi\displaystyle\int\limits_0^2 \left(\dfrac{1}{2}x+1\right)^2 \mathrm{\,d}x -\pi\displaystyle\int\limits_1^2 \left(2x-2\right)^2 \mathrm{\,d}x\right) =\dfrac{20}{3}\pi \approx 20{,}9$.
		}
\end{ex}

\begin{ex}%[2H2C2-6]%Câu 5
	Một ống phun nước có hình dạng như hình vẽ dưới. Để giữ cho ống nước được cân bằng không bị nghiêng kỹ sư sử dụng ba đoạn thép để nối các điểm $C$, $A$, $G$ với mặt đất, các đoạn thép $CD$, $GF$, $AE$ có độ lớn lực căng lần lượt bằng $1200$ N, $800$ N và $600$ N. Trong hệ tọa độ $Oxyz$, coi gốc tọa độ là chân ống nước, trục $Oz$ hướng lên trời, mặt đất là mặt phẳng $(Oxy)$ các thông số được cho như hình vẽ, đơn vị trên các hệ trục tọa độ tính bằng mét. Coi đường kính ống không đáng kể, độ lớn vectơ hợp lực của ba sợi thép tác động lên ông nước là bao nhiêu Newton (làm tròn kết quả đến hàng đơn vị).
	\begin{center}
		\includegraphics[scale=0.45]{images/de7-1}
	\end{center}
	\shortans[oly]{$2311$}
	\loigiai{
		Giả sử lực tác dụng lên 3 đoạn dây $CD$, $GF$, $AE$ lần lượt là $\overrightarrow{T}_1$, $\overrightarrow{T}_2$, $\overrightarrow{T}_3$.\\
		Suy ra hợp lực $\overrightarrow{T} =\overrightarrow{T}_1 +\overrightarrow{T}_2 +\overrightarrow{T}_3$ (*).\\
		(Áp dụng công thức: Cho 2 vec-tơ $\overrightarrow{u}$, $\overrightarrow{v}$ cùng hướng, ta có $\overrightarrow{u} =\dfrac{|\overrightarrow{u}|}{|\overrightarrow{v}|}\cdot\overrightarrow{v}$)
		\begin{enumerate}[+)]
			\item Vì $\overrightarrow{T_1}$ và $\overrightarrow{CD}$ cùng hướng nên ta suy ra $\overrightarrow{T_1}=\dfrac{T_1}{CD}\cdot\overrightarrow{CD}$.
			\item Vì $\overrightarrow{T_2}$ và $\overrightarrow{GF}$ cùng hướng nên ta suy ra $\overrightarrow{T_2}=\dfrac{T_2}{GF}\cdot\overrightarrow{GF}$.
			\item Vì $\overrightarrow{T_3}$ và $\overrightarrow{AE}$ cùng hướng nên ta suy ra $\overrightarrow{T_3}=\dfrac{T_3}{AE}\cdot\overrightarrow{AE}$.
		\end{enumerate}
		Suy ra $\overrightarrow{T} =\dfrac{T_1}{CD}\cdot\overrightarrow{CD} +\dfrac{T_2}{GF}\cdot\overrightarrow{GF} +\dfrac{T_3}{AE}\cdot\overrightarrow{AE}$.
		\begin{enumerate}[+)]
			\item Ta có $C(-1{,}5;0;4{,}5)$, $D(0;3;0)$ $\Rightarrow \overrightarrow{CD}=(1{,}5;3;-4{,}5) \Rightarrow CD=1{,}5\sqrt{14}$.
			\item Ta có $G(0;-1;3)$, $F(2;-1;0)$ $\Rightarrow \overrightarrow{GF}=(2;0;-3) \Rightarrow GF=\sqrt{13}$.
			\item Ta có $A(0;0;3)$, $E(-1{,}5;0;0)$ $\Rightarrow \overrightarrow{AE}=(-1{,}5;0;-3) \Rightarrow AE=1{,}5\sqrt{5}$.
		\end{enumerate}
		Suy ra 
		$\begin{aligned}[t]
			\overrightarrow{T} &=\dfrac{1\,200}{1{,}5\sqrt{14}}\cdot(1{,}5;3;-4{,}5) +\dfrac{800}{\sqrt{13}}\cdot(2;0;-3) +\dfrac{600}{1{,}5\sqrt{5}}\cdot(-1{,}5;0;-3) \\
			&=(496{,}145;641{,}427;-2164{,}437).
		\end{aligned}$\\
		Độ lớn của vec-tơ hợp lực là $|\overrightarrow{T}| =\sqrt{(496{,}145)^2+(641{,}427)^2+(-2164{,}437)^2} \approx 2311$.
	}
\end{ex}

\begin{ex}%[0H9C3-5]%Câu 6
	Hình vẽ sau mô tả một con thuyền đang kéo một người đàn ông trượt ván bằng một đoạn dây dài $9$ mét. Xét trên hệ trục $Oxy$ (đơn vị trên các hệ trục bằng mét), ban đầu con thuyền đang ở gốc tọa độ và di chuyển trên tia $Oy$, người đàn ông xuất phát từ điểm có tọa độ $(9;0)$ bị kéo theo và quãng đường di chuyển tạo thành một đường cong $y=f(x)$ (tham khảo hình vẽ dưới), bờ biển là đường thẳng $x+2y+1=0$. Khi người đàn ông đến gần bờ biển nhất thì khoảng cách giữa người đàn ông và trục $Oy$ bằng bao nhiêu mét (làm tròn kết quả đến hàng phần trăm)? Biết rằng trong quá trình di chuyển, người đàn ông luôn hướng về phía thuyền, đoạn dây luôn căng và nằm trên tiếp tuyến của đường cong $y=f(x)$.
	\begin{center}
		\includegraphics[scale=0.4]{images/de7-2}
	\end{center}
	\shortans[oly]{$8{,}05$}
	\loigiai{
%		\begin{center}
%			\includegraphics[scale=0.4]{images/de7-3}
%		\end{center}
	Giả sử người đó ở vị trí $M$, $M_0$ là điểm mà tiếp tuyến tại $M_0$ song song
	với đường thẳng $d\colon x+2y+1=0$.\\
	Dựng $M_0N\perp d$, $MK\perp d$.\\
	Dựa vào hình vẽ, ta có $MK\geq M_0N$ nên khoảng cách từ người đến bờ biển ngắn nhất khi tiếp tuyến tại $M$ song song với đường thẳng $x+2y+1=0$.\\
	Dựa vào hình, ta có
	\begin{enumerate}[+)]
		\item $\widehat{PM_0Q}=\widehat{PFO}$ (2 góc đồng vị)
		\item $\widehat{PFO}=\widehat{FED}$ (2 góc so le trong)
	\end{enumerate}
	Suy ra $\widehat{PM_0Q}=\widehat{FED}$.\\
	Thay lần lượt $x=0, y=0$ vào đường thẳng $x+2y+1=0$, ta có $E(-1;0)$, $D\left(0;\frac{1}{2}\right)$ $\Rightarrow \heva{&OE=1\\ &OD=\dfrac{1}{2}.}$\\
	Xét tam giác vuông $OED$, ta có $\tan\widehat{OED}=\dfrac{OD}{OE}=\dfrac{\frac{1}{2}}{1} =\dfrac{1}{2} \Rightarrow \widehat{OED}=\widehat{PM_0Q}\approx26{,}565^\circ$.\\
	Xét tam giác vuông $PM_0Q$, ta có $$\cos\widehat{PM_0Q}=\dfrac{QM_0}{PM_0} \Rightarrow QM_0=\cos\widehat{PM_0Q}\cdot PM_0 =9\cdot\cos(26{,}565^\circ)\approx8{,}05\, (\rm m).$$
	}
\end{ex}
\Closesolutionfile{ans}

\inputansbox{3}{ans/de7-KQ}
% \begin{name}
	{\tenchude}
	{\tendethi}
	{\tentruong}
	{\thoigian}
	\end{name}
\TN
\Opensolutionfile{ans}[ans/de11-phanI]
\begin{ex}%[2D1N2-2]
	Cho hàm số $y=f(x)$ có bảng biến thiên như hình sau:
	\begin{center}
		\begin{tikzpicture}[scale=1, font=\footnotesize, line width=1pt]
			\tkzTabInit[nocadre=true, lgt=1.2, espcl=2, deltacl=0.6]
			{$x$/0.8,$f'(x)$/0.6,$f(x)$/2}
			{$-\infty$,$0$,$2$,$+\infty$};
			\tkzTabLine{,-,$0$,+,$0$,-,};
			\tkzTabVar{+/$+\infty$,-/$1$,+/$5$,-/$-\infty$};
		\end{tikzpicture}
	\end{center}
	Giá trị cực đại của hàm số đã cho bằng
	\choice
	{$0$}
	{$2$}
	{\True $5$}
	{$1$}
	\loigiai
	{
		Từ bảng biến thiên ta có giá trị cực đại của hàm số đã cho bằng $5$ tại $x=2$.
	}
\end{ex}

\begin{ex}%[2H5N3-2]
	Trong không gian $Oxyz$, cho mặt cầu $(S)\colon(x-2)^2+(y+1)^2+(z-3)^2=4$. Tâm của $(S)$ có tọa độ là
	\choice
	{$(-4;2;-6)$}
	{$(-2;1;-3)$}
	{\True $(2;-1;3)$}
	{$(4;-2;6)$}
	\loigiai
	{
		Tâm của $(S)$ có tọa độ là $(2,-1,3)$.
	}
\end{ex}

\begin{ex}%[1D6H4-2]
	Nếu $\log_8p=m$ thì $\log_2p$ bằng
	\choice
	{$\dfrac{m}{3}$}
	{$\dfrac{3}{m}$}
	{$m^3$}
	{\True $3m$}
	\loigiai
	{
		$\log_8p=m \Rightarrow p=8^m \Rightarrow \log_2p=\log_28^m=m\log_28=3m$.
	}
\end{ex}

\begin{ex}%[1H8N1-2]
	Cho hình hộp chữ nhật $ABCD.A'B'C'D'$. Hai đường thẳng nào sau đây vuông góc với nhau?
	\choice
	{$BD$ và $C'D'$}
	{\True $AA'$ và $BD$}
	{$A'B$ và $CD$}
	{$BB'$ và $DD'$}
	\loigiai
	{
		\begin{center}
			\begin{tikzpicture}[scale=1, font=\footnotesize,>=stealth, line width=1pt]%<DTools>
				%Gán số liệu.
				\def\canhAD{3};\def\canhBA{2};\def\gocBAD{-130};\def\h{4};\def\xdinhA'{0};
				%Gán tọa độ.
				\coordinate (A) at (0,0);
				\coordinate (B) at ($(A)+(\gocBAD:\canhBA)$);
				\coordinate (C) at ($(B)+(0:\canhAD)$);
				\coordinate (D) at ($(A)+(0:\canhAD)$);
				\coordinate (A') at ($(A)+(\xdinhA',\h)$);
				\coordinate (B') at ($(B)+(\xdinhA',\h)$);
				\coordinate (C') at ($(C)+(\xdinhA',\h)$);
				\coordinate (D') at ($(D)+(\xdinhA',\h)$);
				%Vẽ khối lẳng trụ ABCD.A'B'C'D'.
				\draw (A')--(B')--(B)--(C)--(C')--(D')--cycle (B')--(C') (D')--(D)--(C);
				\draw[dashed] (A)--(D) (A')--(A)--(B);
				%Gán nhãn.
				\foreach \x/\y in {A/180, B/180, C/0, D/0, A'/180, B'/180, C'/0, D'/0}{\fill (\x) circle(1pt) ($(\x)+(\y:0.3cm)$) node{$\x$};}
			\end{tikzpicture}
		\end{center}
		Ta có $AA'\perp (ABCD) \Rightarrow AA'\perp BD$.
	}
\end{ex}

\begin{ex}%[2D4H1-3]
	Hàm số $F(x)=\sin 2x$ là một nguyên hàm của hàm số nào sau đây?
	\choice
	{$f_3(x)=\cos 2x$}
	{\True $f_2(x)=2\cos 2x$}
	{$f_1(x)=\dfrac{1}{2}\cos 2x$}
	{$f_4(x)=-\dfrac{1}{2}\cos 2x$}
	\loigiai
	{
		Ta có $f(x)=F'(x) =(\sin 2x)'=2\cos 2x$.
	}
\end{ex}

\begin{ex}%[2D1H1-1]
	Hàm số $y=x^4-2x^2+5$ đồng biến trên khoảng nào sau đây?
	\choice
	{$(-\infty;-1)$}
	{$(0;1)$}
	{\True $(-1;0)$}
	{$(0;+\infty)$}
	\loigiai
	{
		Hàm số trên có bảng biến thiên
		\begin{center}
			\begin{tikzpicture}[scale=1, font=\footnotesize, line width=1pt]%<DTools>
				\tkzTabInit[nocadre=true, lgt=1.2, espcl=2, deltacl=0.6]
				{$x$/0.8,$f'(x)$/0.6,$f(x)$/2}
				{$-\infty$,$-1$,$0$,$1$,$+\infty$};
				\tkzTabLine{,-,$0$,+,$0$,-,$0$,+,};
				\tkzTabVar{+/$+\infty$,-/$4$,+/$5$,-/$4$,+/$+\infty$};
			\end{tikzpicture}
		\end{center}
		Từ bảng biến thiên ta thấy hàm số đồng biến trên $(-1;0)$ và $(1;+\infty)$.
	}
\end{ex}

\begin{ex}%[0H5V4-1]
	Cho hình chóp $S.ABC$ có đáy $ABC$ là tam giác đều, $AB=1$, cạnh bên $SB$ vuông góc với mặt phẳng đáy và $SB=1$. Tích vô hướng của hai vectơ $\overrightarrow{SA}$ và $\overrightarrow{SB}$ bằng
	\choice
	{\True $1$}
	{$\sqrt{2}$}
	{$2$}
	{$\dfrac{\sqrt{2}}{2}$}
	\loigiai
	{
		\begin{center}
			\begin{tikzpicture}[scale=1, font=\footnotesize,>=stealth, line width=1pt]%<DTools>
				%Gán số liệu.
				\def\canhBC{4};\def\canhAB{2};\def\gocABC{-50};\def\h{3};\def\xdinhS{0};
				%Gán tọa độ.
				\coordinate (B) at (0,0);
				\coordinate (A) at ($(B)+(\gocABC:\canhAB)$);
				\coordinate (C) at ($(B)+(0:\canhBC)$);
				\coordinate (S) at ($(B)+(\xdinhS,\h)$);
				%Vẽ khối chóp S.BAC.
				\draw (S)--(A) (S)--(B)--(A) (S)--(C)--(A);
				\draw[dashed] (B)--(C);
				%Gán nhãn.
				\foreach \x/\y in {S/90,B/180,A/-90,C/0}{\fill (\x) circle (1pt) ($(\x)+(\y:0.3cm)$) node{$\x$};}
			\end{tikzpicture}
		\end{center}
		$SB\perp (ABC) \Rightarrow SB \perp AB$ mà $SB=AB=1$ nên tam giác $ABC$ là tam giác vuông cân.\\
		Suy ra $\widehat{ASB}=45^\circ$.\\
		Ta có $SA=\sqrt{SB^2+AB^2}=\sqrt{1^2+1^2}=\sqrt{2}$.\\
		Vậy $\overrightarrow{SA}\cdot \overrightarrow{SB}=\left|\overrightarrow{SA}\right|\cdot \left|\overrightarrow{SB}\right|\cdot \cos \widehat{ASB}=\sqrt{2} \cdot 1 \cdot \dfrac{\sqrt{2}}{2}=1$.
	}
\end{ex}

\begin{ex}%[1D2H2-4]
	Cho cấp số cộng $6$, $17$, $28$, $\ldots$, số hạng thứ $10$ của cấp số cộng đã cho bằng
	\choice
	{$108$}
	{$106$}
	{$107$}
	{\True $105$}
	\loigiai
	{
		Cấp số cộng có $u_1=6$ và công sai $d=11$.\\
		Số hạng thứ $10$ là $u_{10}=u_1+(10-1)d=6+9\cdot 11=105$.
	}
\end{ex}

\begin{ex}%[2D4H2-1]
	Nếu $\displaystyle\int\limits_0^2f(x)\mathrm{\, d}x=4$ thì $\displaystyle\int\limits_0^2\left[f(x)-3\right]\mathrm{\, d}x$ bằng
	\choice
	{$-4$}
	{$1$}
	{\True $-2$}
	{$3$}
	\loigiai
	{
		$\displaystyle\int\limits_0^2\left[f(x)-3\right]\mathrm{\, d}x=\displaystyle\int\limits_0^2f(x)\mathrm{\, d}x-\displaystyle\int\limits_0^2 3\mathrm{\, d}x=4-6=-2$.
	}
\end{ex}

\begin{ex}%[2D3N1-4]
	Cân nặng của một số quả mít trong khu vườn được thống kê ở bảng sau
	\begin{center}
		\begin{tabular}{|l|c|c|c|c|c|}
			\hline Cân nặng (kg)&{$[4;6)$}&{$[6;8)$}&{$[8;10)$}&{$[10;12)$}&{$[12;14)$}\\
			\hline Số quả mít&$6$&$12$&$19$&$9$&$4$\\
			\hline
		\end{tabular}
	\end{center}
	Số quả mít có cân nặng ít hơn $10$ kg trong bảng trên là
	\choice
	{$19$}
	{$46$}
	{$40$}
	{\True $37$}
	\loigiai
	{
		Từ bảng số liệu ta có số quả mít có cân nặng ít hơn $10$ kg là $$6+12+19=37.$$
	}
\end{ex}

\begin{ex}%[2H2H2-3]
	Trong không gian $Oxyz$, cho điểm $M(2;0;1)$. Gọi $A$, $B$ lần lượt là hình chiếu vuông góc của $M$ trên trục $O x$ và trên mặt phẳng $(Oyz)$. Đường thẳng $AB$ có một vectơ chỉ phương là vectơ nào sau đây?
	\choice
	{$\vec{u}_1=(2;0;1)$}
	{\True $\vec{u}_2=(-2;0;1)$}
	{$\vec{u}_3=(1;0;-2)$}
	{$\vec{u}_4=(1;0;2)$}
	\loigiai
	{
		$A$ là hình chiếu vuông góc của $M$ trên trục $Ox$ nên $A(2;0;1)$.\\
		$B$ là hình chiếu vuông góc của $M$ trên mặt phẳng $Oyz$ nên $B(0;0;1)$.\\
		Véctơ chỉ phương của đường thẳng $AB$ là véctơ $\overrightarrow{AB}=(-2;0;1)$.
	}
\end{ex}

\begin{ex}%[1D6H4-5]
	Tập nghiệm của bất phương trình $5^{x^2}\le 25^x$ chứa bao nhiêu số nguyên?
	\choice
	{\True $3$}
	{$1$}
	{$2$}
	{$4$}
	\loigiai
	{
		$5^{x^2}\le 25^x \Leftrightarrow x^2 \leq 2x \Leftrightarrow x^2-2x\leq 0 \Leftrightarrow 0\leq x \leq 2$.\\
		Suy ra tập nghiệm của bất phương trình chứa các số nguyên là $0$; $1$; $2$.\\
		Vậy tập nghiệm của bất phương trình chứa ba số nguyên.
	}
\end{ex}
\Closesolutionfile{ans}
%{\fontfamily{qtm}\fontsize{13pt}{2pt}\selectfont\textbf{PHẦN II. Câu trắc nghiệm đúng sai}. Thí sinh trả lời từ câu 1 đến câu 4. Trong mỗi ý \textbf{a)}, \textbf{b)}, \textbf{c)}, \textbf{d)} ở mỗi câu, thí sinh chọn đúng hoặc sai.}
%\setcounter{ex}{0}% Reset lại số đếm câu hỏi
\TNTF
\Opensolutionfile{ans}[ans/de11-phanII]
\begin{ex}%[2D4H1-3]
	Cho hàm số $f(x)=2x-3\cos x$. Gọi $F(x)$ là một nguyên hàm của hàm số $f(x)$ thoả mãn điều kiện $F\left(\dfrac{\pi}{2}\right)=3$.
	\choiceTF
	{\True $F'(x)=2x-3\cos x$}
	{$\displaystyle\int f(x)\mathrm{\, d}x=x^2+3\sin x+C$}
	{\True $F(x)=x^2-3\sin x+6-\dfrac{\pi^2}{4}$}
	{$F(0)=3-\dfrac{\pi^2}{4}$}
	\loigiai{
		\begin{itemchoice}
			\itemch {\bf Đúng}.\\
			$F(x)$ là nguyên hàm của $f(x)$ nên $F'(x)=f(x)=2x-3\cos x$.
			\itemch {\bf Sai}.\\
			$F(x)=\displaystyle\int f(x)\mathrm{\, d}x=\displaystyle\int 2x-3\cos x\mathrm{\, d}x=x^2-3\sin x +C$.
			\itemch {\bf Đúng}.\\
			$F\left(\dfrac{\pi}{2}\right)=3 \Rightarrow 3=\left(\dfrac{\pi}{2}\right)^2-3\sin \left(\dfrac{\pi}{2}\right)+C \Rightarrow C=6-\left(\dfrac{\pi^2}{4}\right)$.\\
			Vậy $F(x)=x^2-3\sin x+6-\dfrac{\pi^2}{4}$
			\itemch {\bf Sai}.\\
			$F(0)=6-\dfrac{\pi^2}{4}$.
		\end{itemchoice}
		
		
		
	}
\end{ex}

%%%%==============HetCau_EX1==============%%%
%%%%==============Cau_EX2==============%%%
\begin{ex} Vào lúc $12$ giờ trưa, tàu $B$ đang nằm ở vị trí $O$, tàu $A$ cách tàu $B$ $12$ km. Tàu $A$ đang di chuyển về phía $O$ với vận tốc $12$ km/h và tiếp tục di chuyển như vậy cả ngày. Tàu $B$ có vận tốc $8$ km/h đang di chuyển theo hướng vuông góc với hướng đi của tàu $A$ và tiếp tục di chuyển như vậy cả ngày. Quãng đường tàu $A$ và tàu $B$ di chuyển được sau $t$ (giờ) (tính từ lúc $12$ giờ trưa lần lượt là $S_A$ và $S_B$.%[1D7V1-4]
	\choiceTF
	{\True $S_A=12t$ (km) và $S_B=8t$ (km)}
	{Khoảng cách giữa $2$ tàu được xác định bởi công thức $S=\sqrt{S_A^2+S_B^2}$ (km)}
	{Lúc $13$ giờ, khoảng cách giữa $2$ tàu bằng $8\sqrt{10}$ (km)}
	{Lúc $13$ giờ, tốc độ thay đổi khoảng cách giữa $2$ tàu bằng $\dfrac{22\sqrt{10}}{5}$ km/h}
	\loigiai{
		\begin{itemchoice}
			\itemch {\bf Đúng}.\\
			Quãng đường tàu $A$ đi được sau $t$ giờ là $S_A=12t$ (km).\\
			Quãng đường tàu $B$ đi được sau $t$ giờ là $S_B=8t$ (km).
			\itemch {\bf Sai}.\\
			Gọi $M$ là vị trí ban đầu của tàu $A$, sau $t$ giờ tàu $A$ đến được vị trí $A$ mới và đi được quãng đường $MA=S_A$, và tàu $B$ đến vị trí $B$ như hình vẽ.
			\begin{center}
				\begin{tikzpicture}[scale=1, font=\footnotesize,line join=round, line cap=round, >=stealth]
					\path
					(0,0) coordinate (O)
					+(-90:3) coordinate (x)
					+(-90:2) coordinate (B)
					+(180:5) coordinate (M)
					+(180:4) coordinate (A)
					;
					\draw[-stealth] (M)--(A);
					\draw[-stealth] (O)--(B);
					\draw (A)--(O)--(x)
					(B)--(A)
					;
					\draw[stealth-stealth] ($(M)+(-90:.3)$)--($(A)+(-90:.3)$) node[midway,below] {$S_A$};
					\draw[stealth-stealth] ($(O)+(0:.5)$)--($(B)+(0:.5)$) node[midway,right] {$S_B$};
					\foreach \x/\g in {M/90,A/90,O/45,B/0}\fill (\x) circle (1pt)+(\g:3mm) node{$\x$};
				\end{tikzpicture}
			\end{center}
			Khoảng cách giữa hai tàu tại thời điểm $t$ là $$S(t)=\sqrt{(12-S_A)^2+S_B^2}=\sqrt{(12-12t)^2+(8t)^2}.$$
			\itemch {\bf Sai}.\\
			Lúc $13$ giờ tàu $B$ cách $O$ một khoảng là $OB=8\cdot 1=8$ (km).\\
			Tàu $A$ cách $O$ một đoạn $OA=12-12=0$ (km).\\
			Vậy khoảng cách giữa hai tàu lúc này chính bằng đoạn $OB=8$ (km).
			\itemch {\bf Sai}.\\
			Ta có
			$S(t)=\sqrt{(12-12t)^2+(8t)^2}=\sqrt{208t^2-288t+144}$.\\
			Vậy $S'(t)=\dfrac{416t-288}{2\sqrt{208t^2-288t+144}}$.\\
			Lúc $13$ giờ ứng với $t=1$ nên ta có
			$S'(1)=\dfrac{32}{2\sqrt{16}}=8$ km/h.
		\end{itemchoice}
		
	}
\end{ex}
%%%==============HetCau_EX2==============%%%
%%%==============Cau_EX3==============%%%
\begin{ex} Một tờ tiền giả lần lượt bị hai người $A$ và $B$ kiểm tra. Xác suất để người $A$ phát hiện ra tờ này giả là $0{,}7$. Nếu người $A$ cho rằng tờ này tiền giả, thì xác suất để người $B$ cũng nhận định như thế là $0{,}8$. Ngược lại, nếu người $A$ cho rằng tờ này là tiền thật thì xác suất để người $B$ cũng nhận định như thế là $0{,}4$.%[2D6V2-3]
	\choiceTF
	{Xác suất để $A$ không phát hiện ra tờ tiền đó giả là $0{,}2$}
	{\True Xác suất để hai người này đều không phát hiện đây là tờ tiền giả là $0{,}12$}
	{\True Xác suất để ít nhất một trong hai người này phát hiện ra tờ tiền đó là giả là $0{,}88$}
	{\True Biết tờ tiền đó đã bị ít nhất một trong hai người này phát hiện là giả, xác suất để $A$ phát hiện ra nó giả là $79{,}5\%$ (làm tròn đến hàng phần chục)}
	\loigiai{
		\begin{itemchoice}
			\itemch {\bf Sai}.\\
			Gọi $A$ là biến cố \lq\lq người $A$ phát hiện tờ tiền này là giả\rq\rq.\\
			Vậy $\overline{A}$ là biến cố \lq\lq người $A$ không phát hiện tờ tiền này là giả\rq\rq.\\
			Gọi $B$ là biến cố \lq\lq người $B$ phát hiện tờ tiền này là giả\rq\rq.\\
			Vậy $\overline{B}$ là biến cố \lq\lq người $B$ không phát hiện tờ tiền này là giả\rq\rq.\\
			Theo đề bài $\mathrm{P}(A)=0{,}7 \Rightarrow \mathrm{P}\left(\overline{A}\right)=1-\mathrm{P}(A)=1-0{,}7=0{,}3$.
			\itemch {\bf Đúng}.\\
			Nếu người $A$ cho rằng tờ tiền này là thật, xác suất để người $B$ cũng nhận định như thế là $\mathrm{P}\left(B\mid\overline{A}\right)=0{,}4$.\\
			Xác suất để cả hai người đều không phát hiện ra tờ tiền giả:
			$$\mathrm{P}\left(\overline{A} \cap\overline{B}\right)=\mathrm{P}(\overline{A}) \cdot \mathrm{P}(B \mid\overline{A})=0{,}3 \cdot 0{,}4=0{,}12.$$
			\itemch {\bf Đúng}.\\
			Gọi $C$ là biến cố \lq\lq Ít nhất một trong hai người này phát hiện ra tờ tiền đó là giả\rq\rq.\\
			Xác suất để ít nhất một trong hai người này phát hiện ra tờ tiền đó là giả là
			$$\mathrm{P}(C)=1-\mathrm{P}\left(\overline{A} \cap\overline{B}\right)=1-0{,}12=0{,}88.$$
			\itemch {\bf Đúng}.\\
			Ta có
			$\mathrm{P}(A \mid C)=\dfrac{\mathrm{P}(C\mid A)\cdot \mathrm{P}(A)}{\mathrm{P}(C)}=\dfrac{1\cdot 0{,}7}{0{,}88}=79{,}5\%$.
		\end{itemchoice}
		
	}
\end{ex}
%%%==============HetCau_EX3==============%%%
%%%==============Cau_EX4==============%%%
\begin{ex}%[2H5V2-8]
	\immini[thm]{
		Một tháp kiểm soát không lưu ở sân bay cao $109$ m đặt một đài kiểm soát không lưu ở độ cao $105$ m. Máy bay trong phạm vi cách đài kiểm soát $450$ km sẽ hiển thị trên màn hình ra đa. Chọn hệ trục tọa độ $Oxyz$ có gốc $O$ trùng với vị trí chân tháp, mặt phẳng ($Oxy$) trùng với mặt đất sao cho trục $Ox$ là hướng Tây, trục $Oy$ là hướng Nam và trục $Oz$ là trục thẳng đứng (Hình vẽ), đơn vị trên mỗi trục là kilômét.
	}
	{
		\tikzset{khongluu/.pic={
				\definecolor{cdce1eb}{RGB}{220,225,235}
				\definecolor{cafb4c8}{RGB}{175,180,200}
				\definecolor{c82d1f5}{RGB}{130,209,245}
				\definecolor{c8ab0e0}{RGB}{138,176,224}
				\definecolor{ce9edf5}{RGB}{233,237,245}
				\definecolor{caaccfa}{RGB}{170,204,250}
				\definecolor{cbec3d2}{RGB}{190,195,210}
				\begin{scope}
					\path[fill=cdce1eb,nonzero rule] (5.13, 5.58) -- (13.02, 5.58) -- (13.02, 0.21) -- (5.13, 0.21) --cycle
					(5.13, 5.58);
					\path[fill=cafb4c8,nonzero rule] (5.13, 5.58) -- (13.02, 5.58) -- (13.02, 4.44) -- (5.13, 4.44) --cycle
					(5.13, 5.58);
					\path[fill=c82d1f5,nonzero rule] (5.14, 4.44) -- (13.03, 4.44) -- (13.03, 3.3) -- (5.14, 3.3) --cycle
					(5.14, 4.44);
					\path[fill=cdce1eb,nonzero rule] (5.13, 5.58) -- (13.02, 5.58) -- (13.02, 0.21) -- (5.13, 0.21) --cycle
					(5.13, 5.58);
					\path[fill=cdce1eb,nonzero rule] (5.13, 5.58) -- (13.02, 5.58) -- (13.02, 4.44) -- (5.13, 4.44) --cycle
					(5.13, 5.58);
					\path[fill=c8ab0e0,nonzero rule] (5.14, 4.44) -- (13.03, 4.44) -- (13.03, 3.3) -- (5.14, 3.3) --cycle
					(5.14, 4.44);
					\path[fill=ce9edf5,nonzero rule] (5.5, 3.3) -- (13.02, 3.3) -- (13.02, 0.7) -- (6.19, 0.7) .. controls (6.17, 0.7) and (6.14, 0.7) ..
					(6.12, 0.7) .. controls (6.1, 0.7) and (6.08, 0.7) ..
					(6.05, 0.71) .. controls (6.03, 0.71) and (6.01, 0.72) ..
					(5.99, 0.73) .. controls (5.97, 0.73) and (5.94, 0.74) ..
					(5.92, 0.75) .. controls (5.9, 0.76) and (5.88, 0.77) ..
					(5.86, 0.78) .. controls (5.84, 0.79) and (5.82, 0.8) ..
					(5.8, 0.81) .. controls (5.79, 0.82) and (5.77, 0.84) ..
					(5.75, 0.85) .. controls (5.73, 0.87) and (5.72, 0.88) ..
					(5.7, 0.9) .. controls (5.68, 0.91) and (5.67, 0.93) ..
					(5.65, 0.95) .. controls (5.64, 0.97) and (5.63, 0.98) ..
					(5.61, 1) .. controls (5.6, 1.02) and (5.59, 1.04) ..
					(5.58, 1.06) .. controls (5.57, 1.08) and (5.56, 1.1) ..
					(5.55, 1.12) .. controls (5.54, 1.14) and (5.53, 1.16) ..
					(5.53, 1.19) .. controls (5.52, 1.21) and (5.52, 1.23) ..
					(5.51, 1.25) .. controls (5.51, 1.27) and (5.5, 1.3) ..
					(5.5, 1.32) .. controls (5.5, 1.34) and (5.5, 1.36) ..
					(5.5, 1.39) --cycle
					(5.5, 3.3);
					\path[fill=ce9edf5,nonzero rule] (13.02, 5.58) -- (13.02, 4.8) -- (6.07, 4.8) .. controls (6.06, 4.8) and (6.04, 4.8) ..
					(6.02, 4.8) .. controls (6, 4.8) and (5.98, 4.8) ..
					(5.96, 4.81) .. controls (5.94, 4.81) and (5.93, 4.82) ..
					(5.91, 4.82) .. controls (5.89, 4.83) and (5.87, 4.83) ..
					(5.85, 4.84) .. controls (5.84, 4.85) and (5.82, 4.86) ..
					(5.8, 4.86) .. controls (5.79, 4.87) and (5.77, 4.88) ..
					(5.75, 4.89) .. controls (5.74, 4.9) and (5.72, 4.92) ..
					(5.71, 4.93) .. controls (5.69, 4.94) and (5.68, 4.95) ..
					(5.67, 4.97) .. controls (5.65, 4.98) and (5.64, 4.99) ..
					(5.63, 5.01) .. controls (5.62, 5.02) and (5.61, 5.04) ..
					(5.6, 5.05) .. controls (5.58, 5.07) and (5.58, 5.08) ..
					(5.57, 5.1) .. controls (5.56, 5.12) and (5.55, 5.13) ..
					(5.54, 5.15) .. controls (5.53, 5.17) and (5.53, 5.19) ..
					(5.52, 5.21) .. controls (5.52, 5.22) and (5.51, 5.24) ..
					(5.51, 5.26) .. controls (5.51, 5.28) and (5.5, 5.3) ..
					(5.5, 5.32) .. controls (5.5, 5.34) and (5.5, 5.35) ..
					(5.5, 5.37) -- (5.5, 5.58) --cycle
					(13.02, 5.58);
					\path[fill=caaccfa,nonzero rule] (13.03, 4.44) -- (13.03, 3.65) -- (6.08, 3.65) .. controls (6.07, 3.65) and (6.05, 3.66) ..
					(6.03, 3.66) .. controls (6.01, 3.66) and (5.99, 3.66) ..
					(5.97, 3.67) .. controls (5.95, 3.67) and (5.94, 3.67) ..
					(5.92, 3.68) .. controls (5.9, 3.69) and (5.88, 3.69) ..
					(5.86, 3.7) .. controls (5.85, 3.71) and (5.83, 3.71) ..
					(5.81, 3.72) .. controls (5.8, 3.73) and (5.78, 3.74) ..
					(5.76, 3.75) .. controls (5.75, 3.76) and (5.73, 3.77) ..
					(5.72, 3.79) .. controls (5.7, 3.8) and (5.69, 3.81) ..
					(5.68, 3.82) .. controls (5.66, 3.84) and (5.65, 3.85) ..
					(5.64, 3.87) .. controls (5.63, 3.88) and (5.62, 3.9) ..
					(5.6, 3.91) .. controls (5.59, 3.93) and (5.58, 3.94) ..
					(5.58, 3.96) .. controls (5.57, 3.98) and (5.56, 3.99) ..
					(5.55, 4.01) .. controls (5.54, 4.03) and (5.54, 4.05) ..
					(5.53, 4.06) .. controls (5.53, 4.08) and (5.52, 4.1) ..
					(5.52, 4.12) .. controls (5.52, 4.14) and (5.51, 4.16) ..
					(5.51, 4.18) .. controls (5.51, 4.19) and (5.51, 4.21) ..
					(5.51, 4.23) -- (5.51, 4.44) --cycle
					(13.03, 4.44);
					\path[fill=cafb4c8,nonzero rule] (11.51, 9.57) -- (10.8, 8.77) .. controls (10.73, 8.77) and (10.21, 8.8) ..
					(9.54, 8.85) -- (10.47, 10.1) -- (10, 10.12) .. controls (9.94, 10.12) and (9.87, 10.11) ..
					(9.81, 10.09) .. controls (9.75, 10.07) and (9.7, 10.04) ..
					(9.65, 9.99) -- (8.48, 8.92) .. controls (8.17, 8.94) and (7.83, 8.96) ..
					(7.72, 8.97) .. controls (7.25, 8.99) and (6.92, 8.5) ..
					(6.92, 8.5) .. controls (6.92, 8.5) and (7.09, 8.09) ..
					(7.54, 8.07) .. controls (7.91, 8.06) and (10, 8.06) ..
					(10.79, 8.06) .. controls (10.84, 8.06) and (10.89, 8.06) ..
					(10.93, 8.07) .. controls (10.98, 8.08) and (11.02, 8.1) ..
					(11.06, 8.12) .. controls (11.1, 8.14) and (11.14, 8.17) ..
					(11.17, 8.2) .. controls (11.21, 8.23) and (11.24, 8.27) ..
					(11.26, 8.31) -- (12.09, 9.54) --cycle
					(11.51, 9.57);
					\path[fill=cbec3d2,nonzero rule] (11, 8.26) .. controls (11.1, 8.26) and (11.2, 8.29) ..
					(11.28, 8.34) -- (12.09, 9.54) -- (11.51, 9.57) -- (10.8, 8.77) .. controls (10.73, 8.77) and (10.21, 8.8) ..
					(9.54, 8.85) -- (10.47, 10.1) -- (10, 10.12) .. controls (9.94, 10.12) and (9.87, 10.11) ..
					(9.81, 10.09) .. controls (9.75, 10.07) and (9.7, 10.04) ..
					(9.65, 9.99) -- (8.48, 8.92) .. controls (8.17, 8.94) and (7.83, 8.96) ..
					(7.71, 8.97) .. controls (7.49, 8.98) and (7.31, 8.87) ..
					(7.17, 8.76) .. controls (7.14, 8.73) and (7.12, 8.71) ..
					(7.12, 8.71) .. controls (7.12, 8.71) and (7.29, 8.29) ..
					(7.75, 8.27) .. controls (8.12, 8.26) and (10.21, 8.26) ..
					(11, 8.26) --cycle
					(11, 8.26);
					\path[fill=cdce1eb,nonzero rule] (11.61, 4.44) -- (11.82, 4.44) -- (11.82, 3.3) -- (11.61, 3.3) --cycle
					(11.61, 4.44);
					\path[fill=cdce1eb,nonzero rule] (10.3, 4.44) -- (10.5, 4.44) -- (10.5, 3.3) -- (10.3, 3.3) --cycle
					(10.3, 4.44);
					\path[fill=cdce1eb,nonzero rule] (8.98, 4.44) -- (9.19, 4.44) -- (9.19, 3.3) -- (8.98, 3.3) --cycle
					(8.98, 4.44);
					\path[fill=cdce1eb,nonzero rule] (7.67, 4.44) -- (7.87, 4.44) -- (7.87, 3.3) -- (7.67, 3.3) --cycle
					(7.67, 4.44);
					\path[fill=cdce1eb,nonzero rule] (6.35, 4.44) -- (6.56, 4.44) -- (6.56, 3.3) -- (6.35, 3.3) --cycle
					(6.35, 4.44);
					\path[fill=cafb4c8,nonzero rule] (2.28, 13.23) .. controls (2.27, 13.23) and (2.25, 13.23) ..
					(2.24, 13.23) .. controls (2.23, 13.22) and (2.21, 13.22) ..
					(2.2, 13.21) .. controls (2.19, 13.21) and (2.18, 13.2) ..
					(2.16, 13.19) .. controls (2.15, 13.19) and (2.14, 13.18) ..
					(2.13, 13.17) .. controls (2.12, 13.16) and (2.11, 13.15) ..
					(2.11, 13.14) .. controls (2.1, 13.13) and (2.09, 13.11) ..
					(2.09, 13.1) .. controls (2.08, 13.09) and (2.08, 13.08) ..
					(2.08, 13.06) .. controls (2.07, 13.05) and (2.07, 13.04) ..
					(2.07, 13.02) -- (2.07, 11.74) -- (2.49, 11.74) -- (2.49, 13.02) .. controls (2.49, 13.04) and (2.48, 13.05) ..
					(2.48, 13.06) .. controls (2.48, 13.08) and (2.48, 13.09) ..
					(2.47, 13.1) .. controls (2.46, 13.11) and (2.46, 13.13) ..
					(2.45, 13.14) .. controls (2.44, 13.15) and (2.43, 13.16) ..
					(2.43, 13.17) .. controls (2.42, 13.18) and (2.41, 13.19) ..
					(2.39, 13.19) .. controls (2.38, 13.2) and (2.37, 13.21) ..
					(2.36, 13.21) .. controls (2.35, 13.22) and (2.33, 13.22) ..
					(2.32, 13.23) .. controls (2.31, 13.23) and (2.29, 13.23) ..
					(2.28, 13.23) --cycle
					(2.28, 13.23);
					\path[fill=cdce1eb,nonzero rule] (3.93, 10.95) -- (3.93, 10.9) -- (0.63, 10.9) -- (0.63, 10.96) .. controls (0.63, 10.98) and (0.63, 11.01) ..
					(0.63, 11.04) .. controls (0.63, 11.06) and (0.64, 11.09) ..
					(0.64, 11.12) .. controls (0.65, 11.14) and (0.65, 11.17) ..
					(0.66, 11.2) .. controls (0.67, 11.22) and (0.68, 11.25) ..
					(0.69, 11.27) .. controls (0.7, 11.3) and (0.71, 11.32) ..
					(0.72, 11.34) .. controls (0.74, 11.37) and (0.75, 11.39) ..
					(0.76, 11.41) .. controls (0.78, 11.44) and (0.79, 11.46) ..
					(0.81, 11.48) .. controls (0.83, 11.5) and (0.85, 11.52) ..
					(0.87, 11.54) .. controls (0.89, 11.56) and (0.91, 11.58) ..
					(0.93, 11.59) .. controls (0.95, 11.61) and (0.97, 11.63) ..
					(0.99, 11.64) .. controls (1.01, 11.66) and (1.04, 11.67) ..
					(1.06, 11.68) .. controls (1.08, 11.69) and (1.11, 11.71) ..
					(1.13, 11.72) .. controls (1.16, 11.73) and (1.18, 11.74) ..
					(1.21, 11.74) .. controls (1.23, 11.75) and (1.26, 11.76) ..
					(1.29, 11.76) .. controls (1.31, 11.77) and (1.34, 11.77) ..
					(1.37, 11.78) .. controls (1.39, 11.78) and (1.42, 11.78) ..
					(1.45, 11.78) -- (3.11, 11.78) .. controls (3.14, 11.78) and (3.16, 11.78) ..
					(3.19, 11.78) .. controls (3.22, 11.77) and (3.24, 11.77) ..
					(3.27, 11.76) .. controls (3.3, 11.76) and (3.32, 11.75) ..
					(3.35, 11.74) .. controls (3.37, 11.74) and (3.4, 11.73) ..
					(3.42, 11.72) .. controls (3.45, 11.71) and (3.47, 11.69) ..
					(3.5, 11.68) .. controls (3.52, 11.67) and (3.54, 11.66) ..
					(3.57, 11.64) .. controls (3.59, 11.63) and (3.61, 11.61) ..
					(3.63, 11.59) .. controls (3.65, 11.58) and (3.67, 11.56) ..
					(3.69, 11.54) .. controls (3.71, 11.52) and (3.73, 11.5) ..
					(3.75, 11.48) .. controls (3.76, 11.46) and (3.78, 11.44) ..
					(3.79, 11.41) .. controls (3.81, 11.39) and (3.82, 11.37) ..
					(3.84, 11.34) .. controls (3.85, 11.32) and (3.86, 11.3) ..
					(3.87, 11.27) .. controls (3.88, 11.25) and (3.89, 11.22) ..
					(3.9, 11.2) .. controls (3.9, 11.17) and (3.91, 11.14) ..
					(3.92, 11.12) .. controls (3.92, 11.09) and (3.93, 11.06) ..
					(3.93, 11.04) .. controls (3.93, 11.01) and (3.93, 10.98) ..
					(3.93, 10.96) --cycle
					(3.93, 10.95);
					\path[fill=ce9edf5,nonzero rule] (3.11, 11.78) -- (1.45, 11.78) .. controls (1.39, 11.78) and (1.34, 11.77) ..
					(1.29, 11.76) .. controls (1.24, 11.7) and (1.21, 11.64) ..
					(1.19, 11.56) .. controls (1.19, 11.55) and (1.18, 11.54) ..
					(1.18, 11.53) .. controls (1.18, 11.52) and (1.18, 11.51) ..
					(1.18, 11.5) .. controls (1.18, 11.48) and (1.18, 11.47) ..
					(1.18, 11.46) .. controls (1.18, 11.45) and (1.18, 11.44) ..
					(1.18, 11.43) .. controls (1.18, 11.42) and (1.18, 11.4) ..
					(1.19, 11.39) .. controls (1.19, 11.38) and (1.19, 11.37) ..
					(1.2, 11.36) .. controls (1.2, 11.35) and (1.21, 11.34) ..
					(1.21, 11.33) .. controls (1.22, 11.32) and (1.22, 11.31) ..
					(1.23, 11.3) .. controls (1.24, 11.29) and (1.25, 11.28) ..
					(1.25, 11.27) .. controls (1.26, 11.27) and (1.27, 11.26) ..
					(1.28, 11.25) .. controls (1.29, 11.24) and (1.3, 11.24) ..
					(1.31, 11.23) .. controls (1.31, 11.22) and (1.32, 11.22) ..
					(1.33, 11.21) .. controls (1.35, 11.21) and (1.36, 11.2) ..
					(1.37, 11.2) .. controls (1.38, 11.19) and (1.39, 11.19) ..
					(1.4, 11.19) .. controls (1.41, 11.19) and (1.42, 11.18) ..
					(1.43, 11.18) .. controls (1.44, 11.18) and (1.46, 11.18) ..
					(1.47, 11.18) -- (3.9, 11.18) .. controls (3.89, 11.22) and (3.87, 11.26) ..
					(3.86, 11.31) .. controls (3.84, 11.35) and (3.81, 11.38) ..
					(3.79, 11.42) .. controls (3.76, 11.46) and (3.74, 11.49) ..
					(3.71, 11.52) .. controls (3.67, 11.56) and (3.64, 11.59) ..
					(3.61, 11.61) .. controls (3.57, 11.64) and (3.53, 11.66) ..
					(3.49, 11.68) .. controls (3.45, 11.7) and (3.41, 11.72) ..
					(3.37, 11.74) .. controls (3.33, 11.75) and (3.29, 11.76) ..
					(3.24, 11.77) .. controls (3.2, 11.78) and (3.15, 11.78) ..
					(3.11, 11.78) --cycle
					(3.11, 11.78);
					\path[fill=cdce1eb,nonzero rule] (0.83, 7.63) .. controls (0.83, 7.63) and (1.41, 6.21) ..
					(1.41, 4.53) -- (1.41, 0.21) -- (3.15, 0.21) -- (3.15, 4.53) .. controls (3.15, 6.21) and (3.73, 7.63) ..
					(3.73, 7.63) -- (3.73, 7.69) -- (0.83, 7.69) --cycle
					(0.83, 7.63);
					\path[fill=ce9edf5,nonzero rule] (1.93, 5.02) -- (1.93, 1.54) .. controls (1.93, 1.51) and (1.93, 1.48) ..
					(1.93, 1.46) .. controls (1.93, 1.43) and (1.94, 1.4) ..
					(1.94, 1.38) .. controls (1.95, 1.35) and (1.95, 1.32) ..
					(1.96, 1.29) .. controls (1.97, 1.27) and (1.98, 1.24) ..
					(1.99, 1.22) .. controls (2, 1.19) and (2.01, 1.17) ..
					(2.03, 1.14) .. controls (2.04, 1.12) and (2.05, 1.09) ..
					(2.07, 1.07) .. controls (2.08, 1.05) and (2.1, 1.03) ..
					(2.12, 1) .. controls (2.14, 0.98) and (2.15, 0.96) ..
					(2.17, 0.94) .. controls (2.19, 0.92) and (2.21, 0.9) ..
					(2.23, 0.89) .. controls (2.26, 0.87) and (2.28, 0.85) ..
					(2.3, 0.84) .. controls (2.32, 0.82) and (2.35, 0.81) ..
					(2.37, 0.8) .. controls (2.4, 0.78) and (2.42, 0.77) ..
					(2.45, 0.76) .. controls (2.47, 0.75) and (2.5, 0.74) ..
					(2.53, 0.73) .. controls (2.55, 0.72) and (2.58, 0.72) ..
					(2.61, 0.71) .. controls (2.63, 0.71) and (2.66, 0.7) ..
					(2.69, 0.7) .. controls (2.71, 0.7) and (2.74, 0.7) ..
					(2.77, 0.7) -- (3.15, 0.7) -- (3.15, 4.53) .. controls (3.15, 6.21) and (3.73, 7.63) ..
					(3.73, 7.63) -- (3.73, 7.69) -- (1.5, 7.69) .. controls (1.67, 7.15) and (1.93, 6.14) ..
					(1.93, 5.02) --cycle
					(1.93, 5.02);
					
					\path[fill=cafb4c8,nonzero rule] (13.02, -0) -- (0.21, -0) .. controls (0.2, -0) and (0.18, 0) ..
					(0.17, 0) .. controls (0.16, 0.01) and (0.15, 0.01) ..
					(0.13, 0.02) .. controls (0.12, 0.02) and (0.11, 0.03) ..
					(0.1, 0.03) .. controls (0.09, 0.04) and (0.08, 0.05) ..
					(0.07, 0.06) .. controls (0.06, 0.07) and (0.05, 0.08) ..
					(0.04, 0.09) .. controls (0.03, 0.1) and (0.03, 0.12) ..
					(0.02, 0.13) .. controls (0.02, 0.14) and (0.01, 0.15) ..
					(0.01, 0.17) .. controls (0.01, 0.18) and (0.01, 0.19) ..
					(0.01, 0.21) .. controls (0.01, 0.22) and (0.01, 0.23) ..
					(0.01, 0.25) .. controls (0.01, 0.26) and (0.02, 0.27) ..
					(0.02, 0.29) .. controls (0.03, 0.3) and (0.03, 0.31) ..
					(0.04, 0.32) .. controls (0.05, 0.33) and (0.06, 0.34) ..
					(0.07, 0.35) .. controls (0.08, 0.36) and (0.09, 0.37) ..
					(0.1, 0.38) .. controls (0.11, 0.39) and (0.12, 0.39) ..
					(0.13, 0.4) .. controls (0.15, 0.4) and (0.16, 0.41) ..
					(0.17, 0.41) .. controls (0.18, 0.41) and (0.2, 0.41) ..
					(0.21, 0.41) -- (13.02, 0.41) .. controls (13.03, 0.41) and (13.04, 0.41) ..
					(13.06, 0.41) .. controls (13.07, 0.41) and (13.08, 0.4) ..
					(13.1, 0.4) .. controls (13.11, 0.39) and (13.12, 0.39) ..
					(13.13, 0.38) .. controls (13.14, 0.37) and (13.15, 0.36) ..
					(13.16, 0.35) .. controls (13.17, 0.34) and (13.18, 0.33) ..
					(13.19, 0.32) .. controls (13.2, 0.31) and (13.2, 0.3) ..
					(13.21, 0.29) .. controls (13.21, 0.27) and (13.22, 0.26) ..
					(13.22, 0.25) .. controls (13.22, 0.23) and (13.22, 0.22) ..
					(13.22, 0.21) .. controls (13.22, 0.19) and (13.22, 0.18) ..
					(13.22, 0.17) .. controls (13.22, 0.15) and (13.21, 0.14) ..
					(13.21, 0.13) .. controls (13.2, 0.12) and (13.2, 0.1) ..
					(13.19, 0.09) .. controls (13.18, 0.08) and (13.17, 0.07) ..
					(13.16, 0.06) .. controls (13.15, 0.05) and (13.14, 0.04) ..
					(13.13, 0.03) .. controls (13.12, 0.03) and (13.11, 0.02) ..
					(13.1, 0.02) .. controls (13.08, 0.01) and (13.07, 0.01) ..
					(13.06, 0) .. controls (13.04, 0) and (13.03, -0) ..
					(13.02, -0) --cycle
					(13.02, -0);
					
					\path[fill=c8ab0e0,nonzero rule] (0.63, 10.95) -- (3.93, 10.95) -- (3.93, 10.09) -- (0.63, 10.09) --cycle
					(0.63, 10.95);
					
					\path[fill=caaccfa,nonzero rule] (1.26, 10.38) -- (3.93, 10.38) -- (3.93, 10.95) -- (1.01, 10.95) -- (1.01, 10.64) .. controls (1.01, 10.62) and (1.01, 10.6) ..
					(1.02, 10.59) .. controls (1.02, 10.57) and (1.02, 10.56) ..
					(1.03, 10.54) .. controls (1.04, 10.53) and (1.04, 10.51) ..
					(1.05, 10.5) .. controls (1.06, 10.48) and (1.07, 10.47) ..
					(1.08, 10.46) .. controls (1.1, 10.45) and (1.11, 10.44) ..
					(1.12, 10.43) .. controls (1.14, 10.42) and (1.15, 10.41) ..
					(1.17, 10.4) .. controls (1.18, 10.4) and (1.2, 10.39) ..
					(1.21, 10.39) .. controls (1.23, 10.39) and (1.25, 10.38) ..
					(1.26, 10.38) --cycle
					(1.26, 10.38);
					
					\path[fill=cdce1eb,nonzero rule] (2.18, 10.96) -- (2.38, 10.96) -- (2.38, 10.08) -- (2.18, 10.08) --cycle
					(2.18, 10.96);
					
					\path[fill=cdce1eb,nonzero rule] (1.35, 10.96) -- (1.56, 10.96) -- (1.56, 10.08) -- (1.35, 10.08) --cycle
					(1.35, 10.96);
					
					\path[fill=cdce1eb,nonzero rule] (3, 10.96) -- (3.21, 10.96) -- (3.21, 10.08) -- (3, 10.08) --cycle
					(3, 10.96);
					
					\path[fill=c8ab0e0,nonzero rule] (0.83, 9.34) -- (3.73, 9.34) -- (3.73, 7.63) -- (0.83, 7.63) --cycle
					(0.83, 9.34);
					
					\path[fill=caaccfa,nonzero rule] (1.61, 7.91) -- (3.73, 7.91) -- (3.73, 9.34) -- (1.25, 9.34) -- (1.25, 8.27) .. controls (1.25, 8.24) and (1.25, 8.22) ..
					(1.26, 8.2) .. controls (1.26, 8.17) and (1.27, 8.15) ..
					(1.28, 8.13) .. controls (1.29, 8.11) and (1.3, 8.09) ..
					(1.31, 8.07) .. controls (1.33, 8.05) and (1.34, 8.03) ..
					(1.36, 8.01) .. controls (1.37, 8) and (1.39, 7.98) ..
					(1.41, 7.97) .. controls (1.43, 7.95) and (1.45, 7.94) ..
					(1.48, 7.93) .. controls (1.5, 7.92) and (1.52, 7.92) ..
					(1.54, 7.91) .. controls (1.57, 7.91) and (1.59, 7.91) ..
					(1.61, 7.91) --cycle
					(1.61, 7.91);
					
					\path[fill=cdce1eb,nonzero rule] (3.73, 8.57) -- (2.38, 8.57) -- (2.38, 9.34) -- (2.18, 9.34) -- (2.18, 8.57) -- (0.83, 8.57) -- (0.83, 8.36) -- (2.18, 8.36) -- (2.18, 7.63) -- (2.38, 7.63) -- (2.38, 8.36) -- (3.73, 8.36) --cycle
					(3.73, 8.57);
					
					\path[fill=cdce1eb,nonzero rule] (0.63, 10.12) -- (3.93, 10.12) .. controls (3.96, 10.12) and (3.99, 10.12) ..
					(4.01, 10.11) .. controls (4.04, 10.11) and (4.07, 10.1) ..
					(4.09, 10.09) .. controls (4.12, 10.08) and (4.14, 10.06) ..
					(4.16, 10.05) .. controls (4.18, 10.03) and (4.21, 10.02) ..
					(4.22, 10) .. controls (4.24, 9.98) and (4.26, 9.96) ..
					(4.28, 9.94) .. controls (4.29, 9.91) and (4.3, 9.89) ..
					(4.31, 9.86) .. controls (4.32, 9.84) and (4.33, 9.81) ..
					(4.34, 9.79) .. controls (4.34, 9.76) and (4.35, 9.73) ..
					(4.35, 9.71) .. controls (4.35, 9.68) and (4.34, 9.65) ..
					(4.34, 9.63) .. controls (4.33, 9.6) and (4.32, 9.57) ..
					(4.31, 9.55) .. controls (4.3, 9.52) and (4.29, 9.5) ..
					(4.28, 9.48) .. controls (4.26, 9.45) and (4.24, 9.43) ..
					(4.22, 9.41) .. controls (4.21, 9.39) and (4.18, 9.38) ..
					(4.16, 9.36) .. controls (4.14, 9.35) and (4.12, 9.33) ..
					(4.09, 9.32) .. controls (4.07, 9.31) and (4.04, 9.31) ..
					(4.01, 9.3) .. controls (3.99, 9.3) and (3.96, 9.29) ..
					(3.93, 9.29) -- (0.63, 9.29) .. controls (0.6, 9.29) and (0.57, 9.3) ..
					(0.54, 9.3) .. controls (0.52, 9.31) and (0.49, 9.31) ..
					(0.47, 9.32) .. controls (0.44, 9.33) and (0.42, 9.35) ..
					(0.4, 9.36) .. controls (0.37, 9.38) and (0.35, 9.39) ..
					(0.33, 9.41) .. controls (0.31, 9.43) and (0.3, 9.45) ..
					(0.28, 9.48) .. controls (0.27, 9.5) and (0.25, 9.52) ..
					(0.24, 9.55) .. controls (0.23, 9.57) and (0.23, 9.6) ..
					(0.22, 9.63) .. controls (0.21, 9.65) and (0.21, 9.68) ..
					(0.21, 9.71) .. controls (0.21, 9.73) and (0.21, 9.76) ..
					(0.22, 9.79) .. controls (0.23, 9.81) and (0.23, 9.84) ..
					(0.24, 9.86) .. controls (0.25, 9.89) and (0.27, 9.91) ..
					(0.28, 9.94) .. controls (0.3, 9.96) and (0.31, 9.98) ..
					(0.33, 10) .. controls (0.35, 10.02) and (0.37, 10.03) ..
					(0.4, 10.05) .. controls (0.42, 10.06) and (0.44, 10.08) ..
					(0.47, 10.09) .. controls (0.49, 10.1) and (0.52, 10.11) ..
					(0.54, 10.11) .. controls (0.57, 10.12) and (0.6, 10.12) ..
					(0.63, 10.12) --cycle
					(0.63, 10.12);
					
					\path[fill=ce9edf5,nonzero rule] (3.93, 10.12) -- (0.63, 10.12) .. controls (0.56, 10.12) and (0.49, 10.1) ..
					(0.43, 10.07) .. controls (0.43, 10.06) and (0.43, 10.04) ..
					(0.42, 10.03) .. controls (0.42, 10.01) and (0.42, 9.99) ..
					(0.42, 9.98) .. controls (0.42, 9.96) and (0.42, 9.95) ..
					(0.42, 9.93) .. controls (0.42, 9.91) and (0.42, 9.9) ..
					(0.43, 9.88) .. controls (0.43, 9.87) and (0.43, 9.85) ..
					(0.44, 9.84) .. controls (0.44, 9.82) and (0.45, 9.81) ..
					(0.45, 9.79) .. controls (0.46, 9.78) and (0.47, 9.76) ..
					(0.48, 9.75) .. controls (0.48, 9.74) and (0.49, 9.72) ..
					(0.5, 9.71) .. controls (0.51, 9.7) and (0.52, 9.68) ..
					(0.53, 9.67) .. controls (0.54, 9.66) and (0.56, 9.65) ..
					(0.57, 9.64) .. controls (0.58, 9.63) and (0.59, 9.62) ..
					(0.61, 9.61) .. controls (0.62, 9.6) and (0.63, 9.6) ..
					(0.65, 9.59) .. controls (0.66, 9.58) and (0.68, 9.58) ..
					(0.69, 9.57) .. controls (0.71, 9.56) and (0.72, 9.56) ..
					(0.74, 9.56) .. controls (0.75, 9.55) and (0.77, 9.55) ..
					(0.78, 9.55) .. controls (0.8, 9.55) and (0.82, 9.55) ..
					(0.83, 9.55) -- (4.14, 9.55) .. controls (4.21, 9.55) and (4.27, 9.56) ..
					(4.33, 9.59) .. controls (4.34, 9.63) and (4.35, 9.67) ..
					(4.35, 9.71) .. controls (4.35, 9.73) and (4.34, 9.76) ..
					(4.34, 9.79) .. controls (4.33, 9.81) and (4.32, 9.84) ..
					(4.31, 9.86) .. controls (4.3, 9.89) and (4.29, 9.91) ..
					(4.28, 9.94) .. controls (4.26, 9.96) and (4.24, 9.98) ..
					(4.22, 10) .. controls (4.21, 10.02) and (4.18, 10.03) ..
					(4.16, 10.05) .. controls (4.14, 10.06) and (4.12, 10.08) ..
					(4.09, 10.09) .. controls (4.07, 10.1) and (4.04, 10.11) ..
					(4.01, 10.11) .. controls (3.99, 10.12) and (3.96, 10.12) ..
					(3.93, 10.12) --cycle
					(3.93, 10.12);
				\end{scope}
		}}
		\begin{tikzpicture}[scale=.7,transform shape]
			\path (0,0) pic[scale=.3]{khongluu};
			\path (.685,0) coordinate (O)
			+(0:4) coordinate (y) node[above] {$y$}
			+(90:5) coordinate (z) node[above] {$z$}
			+(-145:3) coordinate (x) node[above] {$x$}
			;
			\draw[-stealth] (O)--(z);
			\draw[-stealth] (O)--(y);
			\draw[-stealth] (O)--(x);
		\end{tikzpicture}
	}
	Một máy bay đang ở vị trí $A$ cách mặt đất $8$ km, cách $268$ km về phía Đông, $185$ km về phía Nam so với tháp kiểm soát không lưu và đang chuyển động theo đường thẳng $d$ có vectơ chỉ phương là $\overrightarrow{u}=(82;76;0)$ hướng về đài kiểm soát không lưu.
	\choiceTF
	{Vị trí $A$ có tọa độ là $(268;-185;-8)$}
	{\True Đài kiểm soát không lưu có phát hiện được máy bay tại vị trí $A$}
	{\True Phương trình tham số của đường thẳng $d$ là $\heva{&x=-268+82t\\&y=185+76t\\&z=8}$ ($t$ là tham số)}
	{Khoảng cách gần nhất giữa máy bay và đài kiểm soát không lưu là $217{,}98$ km (làm tròn kết quả đến hàng phần trăm)}
	\loigiai{
		\begin{center}
			\tikzset{khongluu/.pic={
					\definecolor{cdce1eb}{RGB}{220,225,235}
					\definecolor{cafb4c8}{RGB}{175,180,200}
					\definecolor{c82d1f5}{RGB}{130,209,245}
					\definecolor{c8ab0e0}{RGB}{138,176,224}
					\definecolor{ce9edf5}{RGB}{233,237,245}
					\definecolor{caaccfa}{RGB}{170,204,250}
					\definecolor{cbec3d2}{RGB}{190,195,210}
					\begin{scope}
						\path[fill=cdce1eb,nonzero rule] (5.13, 5.58) -- (13.02, 5.58) -- (13.02, 0.21) -- (5.13, 0.21) --cycle
						(5.13, 5.58);
						\path[fill=cafb4c8,nonzero rule] (5.13, 5.58) -- (13.02, 5.58) -- (13.02, 4.44) -- (5.13, 4.44) --cycle
						(5.13, 5.58);
						\path[fill=c82d1f5,nonzero rule] (5.14, 4.44) -- (13.03, 4.44) -- (13.03, 3.3) -- (5.14, 3.3) --cycle
						(5.14, 4.44);
						\path[fill=cdce1eb,nonzero rule] (5.13, 5.58) -- (13.02, 5.58) -- (13.02, 0.21) -- (5.13, 0.21) --cycle
						(5.13, 5.58);
						\path[fill=cdce1eb,nonzero rule] (5.13, 5.58) -- (13.02, 5.58) -- (13.02, 4.44) -- (5.13, 4.44) --cycle
						(5.13, 5.58);
						\path[fill=c8ab0e0,nonzero rule] (5.14, 4.44) -- (13.03, 4.44) -- (13.03, 3.3) -- (5.14, 3.3) --cycle
						(5.14, 4.44);
						\path[fill=ce9edf5,nonzero rule] (5.5, 3.3) -- (13.02, 3.3) -- (13.02, 0.7) -- (6.19, 0.7) .. controls (6.17, 0.7) and (6.14, 0.7) ..
						(6.12, 0.7) .. controls (6.1, 0.7) and (6.08, 0.7) ..
						(6.05, 0.71) .. controls (6.03, 0.71) and (6.01, 0.72) ..
						(5.99, 0.73) .. controls (5.97, 0.73) and (5.94, 0.74) ..
						(5.92, 0.75) .. controls (5.9, 0.76) and (5.88, 0.77) ..
						(5.86, 0.78) .. controls (5.84, 0.79) and (5.82, 0.8) ..
						(5.8, 0.81) .. controls (5.79, 0.82) and (5.77, 0.84) ..
						(5.75, 0.85) .. controls (5.73, 0.87) and (5.72, 0.88) ..
						(5.7, 0.9) .. controls (5.68, 0.91) and (5.67, 0.93) ..
						(5.65, 0.95) .. controls (5.64, 0.97) and (5.63, 0.98) ..
						(5.61, 1) .. controls (5.6, 1.02) and (5.59, 1.04) ..
						(5.58, 1.06) .. controls (5.57, 1.08) and (5.56, 1.1) ..
						(5.55, 1.12) .. controls (5.54, 1.14) and (5.53, 1.16) ..
						(5.53, 1.19) .. controls (5.52, 1.21) and (5.52, 1.23) ..
						(5.51, 1.25) .. controls (5.51, 1.27) and (5.5, 1.3) ..
						(5.5, 1.32) .. controls (5.5, 1.34) and (5.5, 1.36) ..
						(5.5, 1.39) --cycle
						(5.5, 3.3);
						\path[fill=ce9edf5,nonzero rule] (13.02, 5.58) -- (13.02, 4.8) -- (6.07, 4.8) .. controls (6.06, 4.8) and (6.04, 4.8) ..
						(6.02, 4.8) .. controls (6, 4.8) and (5.98, 4.8) ..
						(5.96, 4.81) .. controls (5.94, 4.81) and (5.93, 4.82) ..
						(5.91, 4.82) .. controls (5.89, 4.83) and (5.87, 4.83) ..
						(5.85, 4.84) .. controls (5.84, 4.85) and (5.82, 4.86) ..
						(5.8, 4.86) .. controls (5.79, 4.87) and (5.77, 4.88) ..
						(5.75, 4.89) .. controls (5.74, 4.9) and (5.72, 4.92) ..
						(5.71, 4.93) .. controls (5.69, 4.94) and (5.68, 4.95) ..
						(5.67, 4.97) .. controls (5.65, 4.98) and (5.64, 4.99) ..
						(5.63, 5.01) .. controls (5.62, 5.02) and (5.61, 5.04) ..
						(5.6, 5.05) .. controls (5.58, 5.07) and (5.58, 5.08) ..
						(5.57, 5.1) .. controls (5.56, 5.12) and (5.55, 5.13) ..
						(5.54, 5.15) .. controls (5.53, 5.17) and (5.53, 5.19) ..
						(5.52, 5.21) .. controls (5.52, 5.22) and (5.51, 5.24) ..
						(5.51, 5.26) .. controls (5.51, 5.28) and (5.5, 5.3) ..
						(5.5, 5.32) .. controls (5.5, 5.34) and (5.5, 5.35) ..
						(5.5, 5.37) -- (5.5, 5.58) --cycle
						(13.02, 5.58);
						\path[fill=caaccfa,nonzero rule] (13.03, 4.44) -- (13.03, 3.65) -- (6.08, 3.65) .. controls (6.07, 3.65) and (6.05, 3.66) ..
						(6.03, 3.66) .. controls (6.01, 3.66) and (5.99, 3.66) ..
						(5.97, 3.67) .. controls (5.95, 3.67) and (5.94, 3.67) ..
						(5.92, 3.68) .. controls (5.9, 3.69) and (5.88, 3.69) ..
						(5.86, 3.7) .. controls (5.85, 3.71) and (5.83, 3.71) ..
						(5.81, 3.72) .. controls (5.8, 3.73) and (5.78, 3.74) ..
						(5.76, 3.75) .. controls (5.75, 3.76) and (5.73, 3.77) ..
						(5.72, 3.79) .. controls (5.7, 3.8) and (5.69, 3.81) ..
						(5.68, 3.82) .. controls (5.66, 3.84) and (5.65, 3.85) ..
						(5.64, 3.87) .. controls (5.63, 3.88) and (5.62, 3.9) ..
						(5.6, 3.91) .. controls (5.59, 3.93) and (5.58, 3.94) ..
						(5.58, 3.96) .. controls (5.57, 3.98) and (5.56, 3.99) ..
						(5.55, 4.01) .. controls (5.54, 4.03) and (5.54, 4.05) ..
						(5.53, 4.06) .. controls (5.53, 4.08) and (5.52, 4.1) ..
						(5.52, 4.12) .. controls (5.52, 4.14) and (5.51, 4.16) ..
						(5.51, 4.18) .. controls (5.51, 4.19) and (5.51, 4.21) ..
						(5.51, 4.23) -- (5.51, 4.44) --cycle
						(13.03, 4.44);
						\path[fill=cafb4c8,nonzero rule] (11.51, 9.57) -- (10.8, 8.77) .. controls (10.73, 8.77) and (10.21, 8.8) ..
						(9.54, 8.85) -- (10.47, 10.1) -- (10, 10.12) .. controls (9.94, 10.12) and (9.87, 10.11) ..
						(9.81, 10.09) .. controls (9.75, 10.07) and (9.7, 10.04) ..
						(9.65, 9.99) -- (8.48, 8.92) .. controls (8.17, 8.94) and (7.83, 8.96) ..
						(7.72, 8.97) .. controls (7.25, 8.99) and (6.92, 8.5) ..
						(6.92, 8.5) .. controls (6.92, 8.5) and (7.09, 8.09) ..
						(7.54, 8.07) .. controls (7.91, 8.06) and (10, 8.06) ..
						(10.79, 8.06) .. controls (10.84, 8.06) and (10.89, 8.06) ..
						(10.93, 8.07) .. controls (10.98, 8.08) and (11.02, 8.1) ..
						(11.06, 8.12) .. controls (11.1, 8.14) and (11.14, 8.17) ..
						(11.17, 8.2) .. controls (11.21, 8.23) and (11.24, 8.27) ..
						(11.26, 8.31) -- (12.09, 9.54) --cycle
						(11.51, 9.57);
						\path[fill=cbec3d2,nonzero rule] (11, 8.26) .. controls (11.1, 8.26) and (11.2, 8.29) ..
						(11.28, 8.34) -- (12.09, 9.54) -- (11.51, 9.57) -- (10.8, 8.77) .. controls (10.73, 8.77) and (10.21, 8.8) ..
						(9.54, 8.85) -- (10.47, 10.1) -- (10, 10.12) .. controls (9.94, 10.12) and (9.87, 10.11) ..
						(9.81, 10.09) .. controls (9.75, 10.07) and (9.7, 10.04) ..
						(9.65, 9.99) -- (8.48, 8.92) .. controls (8.17, 8.94) and (7.83, 8.96) ..
						(7.71, 8.97) .. controls (7.49, 8.98) and (7.31, 8.87) ..
						(7.17, 8.76) .. controls (7.14, 8.73) and (7.12, 8.71) ..
						(7.12, 8.71) .. controls (7.12, 8.71) and (7.29, 8.29) ..
						(7.75, 8.27) .. controls (8.12, 8.26) and (10.21, 8.26) ..
						(11, 8.26) --cycle
						(11, 8.26);
						\path[fill=cdce1eb,nonzero rule] (11.61, 4.44) -- (11.82, 4.44) -- (11.82, 3.3) -- (11.61, 3.3) --cycle
						(11.61, 4.44);
						\path[fill=cdce1eb,nonzero rule] (10.3, 4.44) -- (10.5, 4.44) -- (10.5, 3.3) -- (10.3, 3.3) --cycle
						(10.3, 4.44);
						\path[fill=cdce1eb,nonzero rule] (8.98, 4.44) -- (9.19, 4.44) -- (9.19, 3.3) -- (8.98, 3.3) --cycle
						(8.98, 4.44);
						\path[fill=cdce1eb,nonzero rule] (7.67, 4.44) -- (7.87, 4.44) -- (7.87, 3.3) -- (7.67, 3.3) --cycle
						(7.67, 4.44);
						\path[fill=cdce1eb,nonzero rule] (6.35, 4.44) -- (6.56, 4.44) -- (6.56, 3.3) -- (6.35, 3.3) --cycle
						(6.35, 4.44);
						\path[fill=cafb4c8,nonzero rule] (2.28, 13.23) .. controls (2.27, 13.23) and (2.25, 13.23) ..
						(2.24, 13.23) .. controls (2.23, 13.22) and (2.21, 13.22) ..
						(2.2, 13.21) .. controls (2.19, 13.21) and (2.18, 13.2) ..
						(2.16, 13.19) .. controls (2.15, 13.19) and (2.14, 13.18) ..
						(2.13, 13.17) .. controls (2.12, 13.16) and (2.11, 13.15) ..
						(2.11, 13.14) .. controls (2.1, 13.13) and (2.09, 13.11) ..
						(2.09, 13.1) .. controls (2.08, 13.09) and (2.08, 13.08) ..
						(2.08, 13.06) .. controls (2.07, 13.05) and (2.07, 13.04) ..
						(2.07, 13.02) -- (2.07, 11.74) -- (2.49, 11.74) -- (2.49, 13.02) .. controls (2.49, 13.04) and (2.48, 13.05) ..
						(2.48, 13.06) .. controls (2.48, 13.08) and (2.48, 13.09) ..
						(2.47, 13.1) .. controls (2.46, 13.11) and (2.46, 13.13) ..
						(2.45, 13.14) .. controls (2.44, 13.15) and (2.43, 13.16) ..
						(2.43, 13.17) .. controls (2.42, 13.18) and (2.41, 13.19) ..
						(2.39, 13.19) .. controls (2.38, 13.2) and (2.37, 13.21) ..
						(2.36, 13.21) .. controls (2.35, 13.22) and (2.33, 13.22) ..
						(2.32, 13.23) .. controls (2.31, 13.23) and (2.29, 13.23) ..
						(2.28, 13.23) --cycle
						(2.28, 13.23);
						\path[fill=cdce1eb,nonzero rule] (3.93, 10.95) -- (3.93, 10.9) -- (0.63, 10.9) -- (0.63, 10.96) .. controls (0.63, 10.98) and (0.63, 11.01) ..
						(0.63, 11.04) .. controls (0.63, 11.06) and (0.64, 11.09) ..
						(0.64, 11.12) .. controls (0.65, 11.14) and (0.65, 11.17) ..
						(0.66, 11.2) .. controls (0.67, 11.22) and (0.68, 11.25) ..
						(0.69, 11.27) .. controls (0.7, 11.3) and (0.71, 11.32) ..
						(0.72, 11.34) .. controls (0.74, 11.37) and (0.75, 11.39) ..
						(0.76, 11.41) .. controls (0.78, 11.44) and (0.79, 11.46) ..
						(0.81, 11.48) .. controls (0.83, 11.5) and (0.85, 11.52) ..
						(0.87, 11.54) .. controls (0.89, 11.56) and (0.91, 11.58) ..
						(0.93, 11.59) .. controls (0.95, 11.61) and (0.97, 11.63) ..
						(0.99, 11.64) .. controls (1.01, 11.66) and (1.04, 11.67) ..
						(1.06, 11.68) .. controls (1.08, 11.69) and (1.11, 11.71) ..
						(1.13, 11.72) .. controls (1.16, 11.73) and (1.18, 11.74) ..
						(1.21, 11.74) .. controls (1.23, 11.75) and (1.26, 11.76) ..
						(1.29, 11.76) .. controls (1.31, 11.77) and (1.34, 11.77) ..
						(1.37, 11.78) .. controls (1.39, 11.78) and (1.42, 11.78) ..
						(1.45, 11.78) -- (3.11, 11.78) .. controls (3.14, 11.78) and (3.16, 11.78) ..
						(3.19, 11.78) .. controls (3.22, 11.77) and (3.24, 11.77) ..
						(3.27, 11.76) .. controls (3.3, 11.76) and (3.32, 11.75) ..
						(3.35, 11.74) .. controls (3.37, 11.74) and (3.4, 11.73) ..
						(3.42, 11.72) .. controls (3.45, 11.71) and (3.47, 11.69) ..
						(3.5, 11.68) .. controls (3.52, 11.67) and (3.54, 11.66) ..
						(3.57, 11.64) .. controls (3.59, 11.63) and (3.61, 11.61) ..
						(3.63, 11.59) .. controls (3.65, 11.58) and (3.67, 11.56) ..
						(3.69, 11.54) .. controls (3.71, 11.52) and (3.73, 11.5) ..
						(3.75, 11.48) .. controls (3.76, 11.46) and (3.78, 11.44) ..
						(3.79, 11.41) .. controls (3.81, 11.39) and (3.82, 11.37) ..
						(3.84, 11.34) .. controls (3.85, 11.32) and (3.86, 11.3) ..
						(3.87, 11.27) .. controls (3.88, 11.25) and (3.89, 11.22) ..
						(3.9, 11.2) .. controls (3.9, 11.17) and (3.91, 11.14) ..
						(3.92, 11.12) .. controls (3.92, 11.09) and (3.93, 11.06) ..
						(3.93, 11.04) .. controls (3.93, 11.01) and (3.93, 10.98) ..
						(3.93, 10.96) --cycle
						(3.93, 10.95);
						\path[fill=ce9edf5,nonzero rule] (3.11, 11.78) -- (1.45, 11.78) .. controls (1.39, 11.78) and (1.34, 11.77) ..
						(1.29, 11.76) .. controls (1.24, 11.7) and (1.21, 11.64) ..
						(1.19, 11.56) .. controls (1.19, 11.55) and (1.18, 11.54) ..
						(1.18, 11.53) .. controls (1.18, 11.52) and (1.18, 11.51) ..
						(1.18, 11.5) .. controls (1.18, 11.48) and (1.18, 11.47) ..
						(1.18, 11.46) .. controls (1.18, 11.45) and (1.18, 11.44) ..
						(1.18, 11.43) .. controls (1.18, 11.42) and (1.18, 11.4) ..
						(1.19, 11.39) .. controls (1.19, 11.38) and (1.19, 11.37) ..
						(1.2, 11.36) .. controls (1.2, 11.35) and (1.21, 11.34) ..
						(1.21, 11.33) .. controls (1.22, 11.32) and (1.22, 11.31) ..
						(1.23, 11.3) .. controls (1.24, 11.29) and (1.25, 11.28) ..
						(1.25, 11.27) .. controls (1.26, 11.27) and (1.27, 11.26) ..
						(1.28, 11.25) .. controls (1.29, 11.24) and (1.3, 11.24) ..
						(1.31, 11.23) .. controls (1.31, 11.22) and (1.32, 11.22) ..
						(1.33, 11.21) .. controls (1.35, 11.21) and (1.36, 11.2) ..
						(1.37, 11.2) .. controls (1.38, 11.19) and (1.39, 11.19) ..
						(1.4, 11.19) .. controls (1.41, 11.19) and (1.42, 11.18) ..
						(1.43, 11.18) .. controls (1.44, 11.18) and (1.46, 11.18) ..
						(1.47, 11.18) -- (3.9, 11.18) .. controls (3.89, 11.22) and (3.87, 11.26) ..
						(3.86, 11.31) .. controls (3.84, 11.35) and (3.81, 11.38) ..
						(3.79, 11.42) .. controls (3.76, 11.46) and (3.74, 11.49) ..
						(3.71, 11.52) .. controls (3.67, 11.56) and (3.64, 11.59) ..
						(3.61, 11.61) .. controls (3.57, 11.64) and (3.53, 11.66) ..
						(3.49, 11.68) .. controls (3.45, 11.7) and (3.41, 11.72) ..
						(3.37, 11.74) .. controls (3.33, 11.75) and (3.29, 11.76) ..
						(3.24, 11.77) .. controls (3.2, 11.78) and (3.15, 11.78) ..
						(3.11, 11.78) --cycle
						(3.11, 11.78);
						\path[fill=cdce1eb,nonzero rule] (0.83, 7.63) .. controls (0.83, 7.63) and (1.41, 6.21) ..
						(1.41, 4.53) -- (1.41, 0.21) -- (3.15, 0.21) -- (3.15, 4.53) .. controls (3.15, 6.21) and (3.73, 7.63) ..
						(3.73, 7.63) -- (3.73, 7.69) -- (0.83, 7.69) --cycle
						(0.83, 7.63);
						\path[fill=ce9edf5,nonzero rule] (1.93, 5.02) -- (1.93, 1.54) .. controls (1.93, 1.51) and (1.93, 1.48) ..
						(1.93, 1.46) .. controls (1.93, 1.43) and (1.94, 1.4) ..
						(1.94, 1.38) .. controls (1.95, 1.35) and (1.95, 1.32) ..
						(1.96, 1.29) .. controls (1.97, 1.27) and (1.98, 1.24) ..
						(1.99, 1.22) .. controls (2, 1.19) and (2.01, 1.17) ..
						(2.03, 1.14) .. controls (2.04, 1.12) and (2.05, 1.09) ..
						(2.07, 1.07) .. controls (2.08, 1.05) and (2.1, 1.03) ..
						(2.12, 1) .. controls (2.14, 0.98) and (2.15, 0.96) ..
						(2.17, 0.94) .. controls (2.19, 0.92) and (2.21, 0.9) ..
						(2.23, 0.89) .. controls (2.26, 0.87) and (2.28, 0.85) ..
						(2.3, 0.84) .. controls (2.32, 0.82) and (2.35, 0.81) ..
						(2.37, 0.8) .. controls (2.4, 0.78) and (2.42, 0.77) ..
						(2.45, 0.76) .. controls (2.47, 0.75) and (2.5, 0.74) ..
						(2.53, 0.73) .. controls (2.55, 0.72) and (2.58, 0.72) ..
						(2.61, 0.71) .. controls (2.63, 0.71) and (2.66, 0.7) ..
						(2.69, 0.7) .. controls (2.71, 0.7) and (2.74, 0.7) ..
						(2.77, 0.7) -- (3.15, 0.7) -- (3.15, 4.53) .. controls (3.15, 6.21) and (3.73, 7.63) ..
						(3.73, 7.63) -- (3.73, 7.69) -- (1.5, 7.69) .. controls (1.67, 7.15) and (1.93, 6.14) ..
						(1.93, 5.02) --cycle
						(1.93, 5.02);
						
						\path[fill=cafb4c8,nonzero rule] (13.02, -0) -- (0.21, -0) .. controls (0.2, -0) and (0.18, 0) ..
						(0.17, 0) .. controls (0.16, 0.01) and (0.15, 0.01) ..
						(0.13, 0.02) .. controls (0.12, 0.02) and (0.11, 0.03) ..
						(0.1, 0.03) .. controls (0.09, 0.04) and (0.08, 0.05) ..
						(0.07, 0.06) .. controls (0.06, 0.07) and (0.05, 0.08) ..
						(0.04, 0.09) .. controls (0.03, 0.1) and (0.03, 0.12) ..
						(0.02, 0.13) .. controls (0.02, 0.14) and (0.01, 0.15) ..
						(0.01, 0.17) .. controls (0.01, 0.18) and (0.01, 0.19) ..
						(0.01, 0.21) .. controls (0.01, 0.22) and (0.01, 0.23) ..
						(0.01, 0.25) .. controls (0.01, 0.26) and (0.02, 0.27) ..
						(0.02, 0.29) .. controls (0.03, 0.3) and (0.03, 0.31) ..
						(0.04, 0.32) .. controls (0.05, 0.33) and (0.06, 0.34) ..
						(0.07, 0.35) .. controls (0.08, 0.36) and (0.09, 0.37) ..
						(0.1, 0.38) .. controls (0.11, 0.39) and (0.12, 0.39) ..
						(0.13, 0.4) .. controls (0.15, 0.4) and (0.16, 0.41) ..
						(0.17, 0.41) .. controls (0.18, 0.41) and (0.2, 0.41) ..
						(0.21, 0.41) -- (13.02, 0.41) .. controls (13.03, 0.41) and (13.04, 0.41) ..
						(13.06, 0.41) .. controls (13.07, 0.41) and (13.08, 0.4) ..
						(13.1, 0.4) .. controls (13.11, 0.39) and (13.12, 0.39) ..
						(13.13, 0.38) .. controls (13.14, 0.37) and (13.15, 0.36) ..
						(13.16, 0.35) .. controls (13.17, 0.34) and (13.18, 0.33) ..
						(13.19, 0.32) .. controls (13.2, 0.31) and (13.2, 0.3) ..
						(13.21, 0.29) .. controls (13.21, 0.27) and (13.22, 0.26) ..
						(13.22, 0.25) .. controls (13.22, 0.23) and (13.22, 0.22) ..
						(13.22, 0.21) .. controls (13.22, 0.19) and (13.22, 0.18) ..
						(13.22, 0.17) .. controls (13.22, 0.15) and (13.21, 0.14) ..
						(13.21, 0.13) .. controls (13.2, 0.12) and (13.2, 0.1) ..
						(13.19, 0.09) .. controls (13.18, 0.08) and (13.17, 0.07) ..
						(13.16, 0.06) .. controls (13.15, 0.05) and (13.14, 0.04) ..
						(13.13, 0.03) .. controls (13.12, 0.03) and (13.11, 0.02) ..
						(13.1, 0.02) .. controls (13.08, 0.01) and (13.07, 0.01) ..
						(13.06, 0) .. controls (13.04, 0) and (13.03, -0) ..
						(13.02, -0) --cycle
						(13.02, -0);
						
						\path[fill=c8ab0e0,nonzero rule] (0.63, 10.95) -- (3.93, 10.95) -- (3.93, 10.09) -- (0.63, 10.09) --cycle
						(0.63, 10.95);
						
						\path[fill=caaccfa,nonzero rule] (1.26, 10.38) -- (3.93, 10.38) -- (3.93, 10.95) -- (1.01, 10.95) -- (1.01, 10.64) .. controls (1.01, 10.62) and (1.01, 10.6) ..
						(1.02, 10.59) .. controls (1.02, 10.57) and (1.02, 10.56) ..
						(1.03, 10.54) .. controls (1.04, 10.53) and (1.04, 10.51) ..
						(1.05, 10.5) .. controls (1.06, 10.48) and (1.07, 10.47) ..
						(1.08, 10.46) .. controls (1.1, 10.45) and (1.11, 10.44) ..
						(1.12, 10.43) .. controls (1.14, 10.42) and (1.15, 10.41) ..
						(1.17, 10.4) .. controls (1.18, 10.4) and (1.2, 10.39) ..
						(1.21, 10.39) .. controls (1.23, 10.39) and (1.25, 10.38) ..
						(1.26, 10.38) --cycle
						(1.26, 10.38);
						
						\path[fill=cdce1eb,nonzero rule] (2.18, 10.96) -- (2.38, 10.96) -- (2.38, 10.08) -- (2.18, 10.08) --cycle
						(2.18, 10.96);
						
						\path[fill=cdce1eb,nonzero rule] (1.35, 10.96) -- (1.56, 10.96) -- (1.56, 10.08) -- (1.35, 10.08) --cycle
						(1.35, 10.96);
						
						\path[fill=cdce1eb,nonzero rule] (3, 10.96) -- (3.21, 10.96) -- (3.21, 10.08) -- (3, 10.08) --cycle
						(3, 10.96);
						
						\path[fill=c8ab0e0,nonzero rule] (0.83, 9.34) -- (3.73, 9.34) -- (3.73, 7.63) -- (0.83, 7.63) --cycle
						(0.83, 9.34);
						
						\path[fill=caaccfa,nonzero rule] (1.61, 7.91) -- (3.73, 7.91) -- (3.73, 9.34) -- (1.25, 9.34) -- (1.25, 8.27) .. controls (1.25, 8.24) and (1.25, 8.22) ..
						(1.26, 8.2) .. controls (1.26, 8.17) and (1.27, 8.15) ..
						(1.28, 8.13) .. controls (1.29, 8.11) and (1.3, 8.09) ..
						(1.31, 8.07) .. controls (1.33, 8.05) and (1.34, 8.03) ..
						(1.36, 8.01) .. controls (1.37, 8) and (1.39, 7.98) ..
						(1.41, 7.97) .. controls (1.43, 7.95) and (1.45, 7.94) ..
						(1.48, 7.93) .. controls (1.5, 7.92) and (1.52, 7.92) ..
						(1.54, 7.91) .. controls (1.57, 7.91) and (1.59, 7.91) ..
						(1.61, 7.91) --cycle
						(1.61, 7.91);
						
						\path[fill=cdce1eb,nonzero rule] (3.73, 8.57) -- (2.38, 8.57) -- (2.38, 9.34) -- (2.18, 9.34) -- (2.18, 8.57) -- (0.83, 8.57) -- (0.83, 8.36) -- (2.18, 8.36) -- (2.18, 7.63) -- (2.38, 7.63) -- (2.38, 8.36) -- (3.73, 8.36) --cycle
						(3.73, 8.57);
						
						\path[fill=cdce1eb,nonzero rule] (0.63, 10.12) -- (3.93, 10.12) .. controls (3.96, 10.12) and (3.99, 10.12) ..
						(4.01, 10.11) .. controls (4.04, 10.11) and (4.07, 10.1) ..
						(4.09, 10.09) .. controls (4.12, 10.08) and (4.14, 10.06) ..
						(4.16, 10.05) .. controls (4.18, 10.03) and (4.21, 10.02) ..
						(4.22, 10) .. controls (4.24, 9.98) and (4.26, 9.96) ..
						(4.28, 9.94) .. controls (4.29, 9.91) and (4.3, 9.89) ..
						(4.31, 9.86) .. controls (4.32, 9.84) and (4.33, 9.81) ..
						(4.34, 9.79) .. controls (4.34, 9.76) and (4.35, 9.73) ..
						(4.35, 9.71) .. controls (4.35, 9.68) and (4.34, 9.65) ..
						(4.34, 9.63) .. controls (4.33, 9.6) and (4.32, 9.57) ..
						(4.31, 9.55) .. controls (4.3, 9.52) and (4.29, 9.5) ..
						(4.28, 9.48) .. controls (4.26, 9.45) and (4.24, 9.43) ..
						(4.22, 9.41) .. controls (4.21, 9.39) and (4.18, 9.38) ..
						(4.16, 9.36) .. controls (4.14, 9.35) and (4.12, 9.33) ..
						(4.09, 9.32) .. controls (4.07, 9.31) and (4.04, 9.31) ..
						(4.01, 9.3) .. controls (3.99, 9.3) and (3.96, 9.29) ..
						(3.93, 9.29) -- (0.63, 9.29) .. controls (0.6, 9.29) and (0.57, 9.3) ..
						(0.54, 9.3) .. controls (0.52, 9.31) and (0.49, 9.31) ..
						(0.47, 9.32) .. controls (0.44, 9.33) and (0.42, 9.35) ..
						(0.4, 9.36) .. controls (0.37, 9.38) and (0.35, 9.39) ..
						(0.33, 9.41) .. controls (0.31, 9.43) and (0.3, 9.45) ..
						(0.28, 9.48) .. controls (0.27, 9.5) and (0.25, 9.52) ..
						(0.24, 9.55) .. controls (0.23, 9.57) and (0.23, 9.6) ..
						(0.22, 9.63) .. controls (0.21, 9.65) and (0.21, 9.68) ..
						(0.21, 9.71) .. controls (0.21, 9.73) and (0.21, 9.76) ..
						(0.22, 9.79) .. controls (0.23, 9.81) and (0.23, 9.84) ..
						(0.24, 9.86) .. controls (0.25, 9.89) and (0.27, 9.91) ..
						(0.28, 9.94) .. controls (0.3, 9.96) and (0.31, 9.98) ..
						(0.33, 10) .. controls (0.35, 10.02) and (0.37, 10.03) ..
						(0.4, 10.05) .. controls (0.42, 10.06) and (0.44, 10.08) ..
						(0.47, 10.09) .. controls (0.49, 10.1) and (0.52, 10.11) ..
						(0.54, 10.11) .. controls (0.57, 10.12) and (0.6, 10.12) ..
						(0.63, 10.12) --cycle
						(0.63, 10.12);
						
						\path[fill=ce9edf5,nonzero rule] (3.93, 10.12) -- (0.63, 10.12) .. controls (0.56, 10.12) and (0.49, 10.1) ..
						(0.43, 10.07) .. controls (0.43, 10.06) and (0.43, 10.04) ..
						(0.42, 10.03) .. controls (0.42, 10.01) and (0.42, 9.99) ..
						(0.42, 9.98) .. controls (0.42, 9.96) and (0.42, 9.95) ..
						(0.42, 9.93) .. controls (0.42, 9.91) and (0.42, 9.9) ..
						(0.43, 9.88) .. controls (0.43, 9.87) and (0.43, 9.85) ..
						(0.44, 9.84) .. controls (0.44, 9.82) and (0.45, 9.81) ..
						(0.45, 9.79) .. controls (0.46, 9.78) and (0.47, 9.76) ..
						(0.48, 9.75) .. controls (0.48, 9.74) and (0.49, 9.72) ..
						(0.5, 9.71) .. controls (0.51, 9.7) and (0.52, 9.68) ..
						(0.53, 9.67) .. controls (0.54, 9.66) and (0.56, 9.65) ..
						(0.57, 9.64) .. controls (0.58, 9.63) and (0.59, 9.62) ..
						(0.61, 9.61) .. controls (0.62, 9.6) and (0.63, 9.6) ..
						(0.65, 9.59) .. controls (0.66, 9.58) and (0.68, 9.58) ..
						(0.69, 9.57) .. controls (0.71, 9.56) and (0.72, 9.56) ..
						(0.74, 9.56) .. controls (0.75, 9.55) and (0.77, 9.55) ..
						(0.78, 9.55) .. controls (0.8, 9.55) and (0.82, 9.55) ..
						(0.83, 9.55) -- (4.14, 9.55) .. controls (4.21, 9.55) and (4.27, 9.56) ..
						(4.33, 9.59) .. controls (4.34, 9.63) and (4.35, 9.67) ..
						(4.35, 9.71) .. controls (4.35, 9.73) and (4.34, 9.76) ..
						(4.34, 9.79) .. controls (4.33, 9.81) and (4.32, 9.84) ..
						(4.31, 9.86) .. controls (4.3, 9.89) and (4.29, 9.91) ..
						(4.28, 9.94) .. controls (4.26, 9.96) and (4.24, 9.98) ..
						(4.22, 10) .. controls (4.21, 10.02) and (4.18, 10.03) ..
						(4.16, 10.05) .. controls (4.14, 10.06) and (4.12, 10.08) ..
						(4.09, 10.09) .. controls (4.07, 10.1) and (4.04, 10.11) ..
						(4.01, 10.11) .. controls (3.99, 10.12) and (3.96, 10.12) ..
						(3.93, 10.12) --cycle
						(3.93, 10.12);
					\end{scope}
			}}
			\begin{tikzpicture}[scale=.7,transform shape]
				\path (0,0) pic[scale=.3]{khongluu};
				\path (.685,0) coordinate (O) node[below right] {$O$}
				+(0:4) coordinate (y) node[below] {$y$ (Hướng Nam)}
				+(90:5) coordinate (z) node[above] {$z$}
				+(-145:3) coordinate (x) node[below] {$x$ (Hướng Tây)}
				;
				\draw[-stealth] (O)--(z);
				\draw[-stealth] (O)--(y);
				\draw[-stealth] (O)--(x);
			\end{tikzpicture}
		\end{center}
		\begin{itemchoice}
			\itemch {\bf Sai}.\\
			Vị trí $A$ cách mặt đất $8$ km, cách $268$ km về phía Đông, $185$ km về phía Nam nên ta có $A\left(-268;185;8\right)$.
			\itemch {\bf Đúng}.\\
			Tọa độ của đài kiểm soát là $M(0;0;0{,}105)$.\\
			Khoảng cách từ đài kiểm soát đến máy bay là
			$$
			MA=\sqrt{(-268)^2+185^2+(8-0{,}105)^2}\approx 325{,}7.
			$$
			Vậy $MA<450$ nên đài kiểm soát có phát hiện được máy bay tại vị trí $A$.
			\itemch {\bf Đúng}.\\
			Đường thẳng $d$ đi qua điểm $A(-268;185;8)$ và có véctơ chỉ phương là $\overrightarrow{u}=(86;76;0)$ nên phương trình tham số của đường thẳng $d$ là $\heva{&x=-268+82t\\&y=185+76t\\&z=8}$ ($t$ là tham số)
			\itemch {\bf Sai}.\\
			Khoảng cách gần nhất giữa máy bay và đài kiểm soát không lưu chính là khoảng cách từ đài kiểm soát không lưu đến quỹ đạo chuyển động $d$ của máy bay.\\
			Ta có $\overrightarrow{MA}=(-268;185;7{,}895)$.\\
			Vậy $h=\dfrac{\left|\left[\overrightarrow{MA},\overrightarrow{u}\right]\right|}{|\overrightarrow{u}|}=317{,}96$ (km).
		\end{itemchoice}
	}
\end{ex}
\Closesolutionfile{ans}
%{\fontfamily{qtm}\fontsize{13pt}{2pt}\selectfont\textbf{PHẦN III. Câu trắc nghiệm trả lời ngắn}. Thí sinh trả lời từ câu 1 đến câu 6 và điền đáp án vào ô trống.}
%\setcounter{ex}{0}% Reset lại số đếm câu hỏi
\TNSA
\Opensolutionfile{ans}[ans/de11-phanIII]
\begin{ex}%[1H8V6-2]
	Cho hình chóp $S.ABC$ có $ABC$, $SAB$ là các tam giác đều và mặt bên $(SAB)$ vuông góc với mặt đáy. Gọi $\alpha$ là góc phẳng nhị diện $[S,BC,A]$. Tính $\cos^2\alpha$.
	\shortans{$0{,}2$}
	\loigiai{
		\begin{center}
			\begin{tikzpicture}[scale=1, font=\footnotesize,>=stealth, line width=1pt]%<DTools>
				%Gán số liệu.
				\def\canhAC{4};\def\canhBA{2};\def\gocBAC{-50};\def\h{3};\def\xdinhS{0};
				%Gán tọa độ.
				\coordinate (A) at (0,0);
				\coordinate (B) at ($(A)+(\gocBAC:\canhBA)$);
				\coordinate (C) at ($(A)+(0:\canhAC)$);
				\path
				($(A)!.5!(B)$) coordinate (M)
				+(90:\h) coordinate (S)
				($(B)!.5!(C)$) coordinate (N)
				($(B)!.5!(N)$) coordinate (P)
				;
				
				
				
				%Vẽ khối chóp S.ABC.
				\draw (S)--(B) (S)--(A)--(B) (S)--(C)--(B)
				(S)--(M)
				(S)--(P)
				\foreach \x/\y/\z in {B/M/S,P/M/S,N/P/M,C/N/A}{
					pic[draw, thin, angle radius = 6pt]{right angle = \x--\y--\z}
				}
				pic[draw, thin, angle radius = 12pt, "$\alpha$", angle eccentricity = 1.5]{ angle = S--P--M}
				
				
				;
				\draw[dashed] (A)--(C)
				(M)--(P)
				(A)--(N)
				
				;
				%Gán nhãn.
				\foreach \x/\y in {S/90,A/180,B/-90,C/0,M/180,P/-30,N/-30}{\fill (\x) circle (1pt) ($(\x)+(\y:0.3cm)$) node{$\x$};}
			\end{tikzpicture}
		\end{center}
		Gọi $M$ là trung điểm của $AB$, $N$ là trung điểm của $BC$, $P$ là trung điểm của $BN$.\\
		Ta có $MPallel AN$ mà $AN\perp BC$ (do tam giác $ABC$ đều) nên $MP\perp BC$.\\
		Mặt bên $(SAB)$ vuông góc với đáy mà $SM\perp AB$ ($SAB$ là tam giác đều), suy ra $SM \perp (ABC) \Rightarrow SM \perp BC$.\\
		Vậy góc $\alpha =\widehat{SPM} \Rightarrow \cos \alpha = \dfrac{MP}{SP}$.\\
		Gọi $a$ là độ dài cạnh của tam giác đều $ABC$ và $SAB$.\\
		Ta có $MP=\dfrac{AN}{2}=\dfrac{a\sqrt{3}}{4}$; $AM=\dfrac{a\sqrt{3}}{2}$.\\
		Suy ra $SP=\sqrt{SM^2+MP^2}=\sqrt{\dfrac{3a^2}{4}+\dfrac{3a^2}{16}}=\dfrac{a\sqrt{15}}{4}$.\\
		Vậy $\cos \alpha = \dfrac{MP}{SP}=\dfrac{\dfrac{a\sqrt{3}}{4}}{\dfrac{a\sqrt{15}}{4}}=\dfrac{1}{\sqrt{5}}$.
		Suy ra $\cos^2 \alpha=\dfrac{1}{5}=0{,}2$.
	}
\end{ex}

\begin{ex}%[2D4V1-6]
	Một người bình thường với chiều cao $h$ cm, nặng $w$ kilogram có diện tích bề mặt cơ thể $S$ được mô hình hoá bởi công thức $S=\dfrac{1}{60}\cdot w^{0{,}5}\cdot h^{0{,}5}$ (m$^2$) (công thức Mosteller).\\
	Một đối tượng có chiều cao bằng $168$ cm, nặng $62$ kg tham gia một cuộc nghiên cứu về sức khỏe trong $5$ năm. Người ta nhận thấy cân nặng của đối tượng quan sát thay đổi với tốc độ $w'(t)=0{,}02t^2+0{,}2t$ kg/năm ($0\le t\le 5$) và chiều cao tăng đều mỗi năm $0{,}5$ cm. Sau $5$ năm quan sát, diện tích bề mặt cơ thể của đối tượng trên tăng thêm bao nhiêu centimet vuông so với ban đầu? (làm tròn kết quả đến hàng đơn vị).
	\shortans{$581$}
	\loigiai{
		Chiều cao của người này sau $5$ năm là
		\begin{center}
			$h_5=h_0+0{,}5t=168+5\times 0{,}5=170{,}5$ cm $=1{,}705$ m.
		\end{center}
		Cân nặng của người ngày sau $5$ năm là
		$$w_5=w_0++\displaystyle\int\limits_0^5\left(0{,}02t^2+0{,}2t\right)\mathrm{\, d}t=62+\dfrac{10}{3}=\dfrac{196}{3}~\mathrm{kg}.$$
		Diện tích bề mặt cơ thể sau $5$ năm tăng thêm là
		$$\begin{aligned}
			\Delta S &= S_5-S_0=\dfrac{1}{60}w_5^{0{,}5}\cdot h_5^{0{,}5}-\dfrac{1}{60}w_0^{0{,}5}\cdot h_0^{0{,}5}\\&=\dfrac{1}{60}\cdot\left(\dfrac{196}{3}\right)^{0{,}5}\cdot 1{,}705^{0{,}5}-\dfrac{1}{60}\cdot62^{0{,}5}\cdot 1{,}68^{0{,}5}=58{,}1\cdot 10^{-3}~\mathrm{m^2}\approx 581~\mathrm{cm^2}
		\end{aligned}$$
	}
\end{ex}

\begin{ex}%[0D8V2-4]
	Một lớp học hè có $15$ học sinh. Biết rằng mỗi ngày $3$ học sinh trong lớp có nhiệm vụ trực nhật sau giờ học. Sau khi kết thúc khóa học hè, người ta thấy rằng hai học sinh bất kỳ trực nhật cùng nhau đúng một ngày. Hỏi lớp học hè kéo dài trong bao nhiêu ngày?
	\shortans{$35$}
	\loigiai{
		Tổng số cặp học sinh có thể tạo thành từ $15$ học sinh là một tổ hợp chập $2$ của $15$ phần tử
		$$\mathrm{C}_{15}^2=105~\text{cặp}.$$
		Mỗi nhóm trực có $3$ học sinh nên số cặp trong mỗi nhóm là $3$ cặp.\\
		Để mỗi học sinh trực nhật cùng nhau đúng một ngày thì tổng số ngày tối đa là $105:3=35$ ngày.
		
	}
\end{ex}

\begin{ex}%[2D1V5-8]
	Giả sử cường độ ánh sáng của một nguồn điểm tỉ lệ thuận với cường độ của nguồn sáng đó và tỉ lệ nghịch với bình phương khoảng cách từ điểm đó đến nguồn sáng. Hai nguồn điểm có cường độ lần lượt là $S$ và $8S$, cách nhau $90$ cm. Xét một điểm $M$ nằm trên đoạn thẳng nối hai nguồn, cường độ ánh sáng tại điểm đó nhỏ nhất thì điểm đó cách nguồn có cường độ $S$ bằng bao nhiêu centimet? (cho biết cường độ sáng tại điểm $M$ bằng tổng cường độ sáng mỗi nguồn tại điểm đó).
	\begin{center}
		\begin{tikzpicture}[scale=1, font=\footnotesize,line join=round, line cap=round, >=stealth]
			%			\draw[gray,xstep = 1, ystep = 1] (0,0) grid (5,5);
			\path
			(3,0) coordinate (M) node[above] {$M$}
			;
			\node[circle, line width = .2 mm, draw = black, anchor = center, minimum size = 1cm] (D1) at (0,0) {};
			\node[circle, line width = .2 mm, draw = black, anchor = center, minimum size = 1cm] (D2) at (9,0) {};
			
			\draw (D1.0)--(D2.180)
			(D1.45)--(D1.-135)
			(D1.135)--(D1.-45)
			(D2.45)--(D2.-135)
			(D2.135)--(D2.-45)
			;
			\fill
			(D1.center) circle (2pt)
			(D2.center) circle (2pt)
			(M) circle (2pt)
			;
			\draw[stealth-stealth] (0,1) -- (9,1) node[midway,above] {$90$ cm};
		\end{tikzpicture}
	\end{center}
	\shortans{$30$}
	\loigiai{
		Gọi $I$ là cường độ ánh sáng. Vì $I$ tỉ lệ thuận với cường độ của nguồn sáng và tỉ lệ nghịch với bình phương khoảng cách từ điểm đó đến nguồn sáng nên $I=k\dfrac{S}{r^2}$.\\
		Giả sử điểm $M$ nằm cách nguồn sáng $S$ (bên trái) một khoảng là $x$ thì $M$ cách nguồn sáng $8S$ một khoảng là $90-x$.\\
		Ta có cường độ ánh sáng tại điểm $M$ do nguồn sáng $1$ gây ra là $I_1=k\dfrac{S}{x^2}$.\\
		Tương tự cường độ ánh sáng tại điểm $M$ do nguồn sáng $1$ gây ra là $I_1=k\dfrac{8S}{(90-x)^2}$.\\
		Vậy cường độ sáng tổng hợp lại $M$ là
		$$I(x)=I_1+I_2=k\dfrac{S}{x^2}+k\dfrac{8S}{(90-x)^2}$$
		Ta có $I'(x)=-k\dfrac{2Sx}{x^4}-k\dfrac{-16S(90-x)}{(90-x)^4}=kS\left(\dfrac{16}{(90-x)^3}-\dfrac{2}{x^3}\right)$.\\
		$I'(x)=0 \Rightarrow (90-x)^3=8x^3 \Rightarrow 90-x=2x \Rightarrow x=30$ cm.\\
		
		
	}
\end{ex}

\begin{ex}%[2D6V2-3]
	Ở vùng $A$ có hai nhóm, nhóm $1$ là nhóm người có thu nhập tốt (trên $15$ triệu đồng/tháng) và nhóm $2$ là nhóm có thu nhập không tốt. Ở vùng $A$ có $40\%$ người có thu nhập tốt và $58\%$ người không gửi tiết kiệm. Khảo sát độc lập những người thuộc nhóm $1$ và nhóm $2$ và tính tỉ lệ phần trăm số người gửi tiết kiệm của từng nhóm thì thấy rằng: Tỉ lệ người gửi tiết kiệm của nhóm $1$ gấp đôi tỉ lệ người tiết kiệm của nhóm $2$. Giả sử một người ở vùng $A$ không gửi tiết kiệm. Xác suất để người ấy có thu nhập tốt là bao nhiêu $\%$? (kết quả làm tròn đến hàng phần chục).
	
	\shortans{$28$}
	\loigiai{
		Gọi $A$ là biến cố \lq\lq Thu nhập tốt\rq\rq.\\
		Gọi $B$ là biến cố \lq\lq Không gửi tiết kiệm\rq\rq.\\
		Gọi $x$ là tỉ lệ người gửi tiết kiệm ở nhóm $1$, $y$ là tỷ lệ người gửi tiết kiệm ở nhóm $2$.\\
		Theo đề bài ta có $x=2y\quad (1)$.\\
		Gọi $N$ là tổng số người ở vùng $A$, thì số người không gửi tiết kiệm ở nhóm $1$ là $(1-x)\cdot 0{,}4 N$.\\
		Số người không gửi tiết kiệm ở nhóm $2$ là $(1-y)\cdot 0{,}6 N$.\\
		Vì tổng số người không gửi tiết kiệm ở vùng $A$ là $0{,}58N$ nên ta có
		$$(1-x)\cdot 0{,}4 N + (1-y)\cdot 0{,}6 N = 0{,}58N \Leftrightarrow (1-x)\cdot 0{,}4  + (1-y)\cdot 0{,}6  = 0{,}58.\quad(2)$$
		Giải $(1)$ và $(2)$ ta có $x=0{,}6$; $y=0{,}3$.\\
		Theo đề bài ta có
		$P(A)=0{,}4$ và $P(B)=0{,}58$.\\
		Xác suất người không gửi tiết kiệm, biết người đó thu nhập tốt là
		$\mathrm{P}(B\mid A)= 1-0{,}6=0{,}4$.\\
		Vậy xác suất người có thu nhập tốt khi biết người đó không gửi tiết kiệm là
		$$P(A\mid B)=\dfrac{\mathrm{P}(A)\cdot \mathrm{P}(B\mid A)}{\mathrm{P}(B)}=\dfrac{0{,}4\cdot 0{,}4}{0{,}58}=28\%.$$
	}
\end{ex}

\begin{ex}%[2H5V2-7]
	Một radar có thể quay $180^\circ$ để quan sát máy bay quanh vùng phủ sóng của nó. Một máy bay cất cánh từ điểm $A$ nằm trên mặt đất theo chiều cùng chiều với vectơ $\vec{AB}$. Trong hệ tọa độ $Oxyz$ mặt đất là mặt phẳng $(Oxy)$, trục $O z$ hướng lên trời, điểm $A$ nằm trên trục $O y$ cách gốc tọa độ $0{,}6$ km; điểm $B$ nằm trên trục $O z$ có cao độ bằng $0{,}3$ km; radar đang nằm trên trục $Ox$ có hoành độ bằng $0{,}4$ km. Máy bay đang ở điểm $B$ bay theo hướng bay như cũ đến điểm $C(a;b;c)$ thì radar quay một góc bằng $60^\circ$. Tính $a+b+c$ theo đơn vị mét (làm tròn đến hàng đơn vị).
	\begin{center}
		\tikzset{rada/.pic={
				\definecolor{c191716}{RGB}{25,23,22}
				
				\begin{scope}[line cap=round,line join=round]
					\path[fill=white,nonzero rule] (0, 13.21) -- (10.42, 13.21) -- (10.42, 0.02) -- (0, 0.02) --cycle
					(0, 13.21);
					
					\path[fill=white,nonzero rule] (0, 13.21) -- (10.42, 13.21) -- (10.42, 0.02) -- (0, 0.02) --cycle
					(0, 13.21);
					
					\path[fill=c191716,nonzero rule] (0.03, 1.5) .. controls (0, 1.74) and (0.05, 1.99) ..
					(0.18, 2.21) .. controls (0.18, 2.21) and (0.18, 2.21) ..
					(0.18, 2.22) .. controls (0.2, 2.23) and (0.22, 2.23) ..
					(0.24, 2.22) -- (1.73, 0.74) -- (1.73, 0.74) .. controls (1.73, 0.74) and (1.73, 0.73) ..
					(1.73, 0.73) .. controls (1.75, 0.71) and (1.74, 0.69) ..
					(1.72, 0.67) .. controls (1.58, 0.59) and (1.42, 0.54) ..
					(1.25, 0.53) .. controls (1.09, 0.51) and (0.92, 0.53) ..
					(0.77, 0.59) .. controls (0.69, 0.61) and (0.62, 0.65) ..
					(0.55, 0.69) .. controls (0.48, 0.74) and (0.41, 0.79) ..
					(0.35, 0.85) .. controls (0.17, 1.03) and (0.06, 1.26) ..
					(0.03, 1.5);
					
					\path[fill=c191716,even odd rule] (0.39, 0.71) -- (0.06, 0.03) -- (1.2, 0.03) -- (1.2, 0.03) -- (1.22, 0.03) -- (1.04, 0.44) .. controls (0.94, 0.45) and (0.84, 0.47) ..
					(0.74, 0.51) .. controls (0.66, 0.54) and (0.58, 0.58) ..
					(0.5, 0.62) .. controls (0.46, 0.65) and (0.42, 0.68) ..
					(0.39, 0.71);
					
					\path[fill=c191716,even odd rule] (1.21, 1.76) .. controls (1.24, 1.79) and (1.28, 1.79) ..
					(1.31, 1.76) .. controls (1.34, 1.73) and (1.34, 1.69) ..
					(1.31, 1.66) -- (1.03, 1.38) .. controls (1, 1.35) and (0.95, 1.35) ..
					(0.93, 1.38) .. controls (0.9, 1.41) and (0.9, 1.45) ..
					(0.93, 1.48) --
					(1.21, 1.76);
					
					\path[fill=c191716,nonzero rule] (1.33, 1.94) .. controls (1.37, 1.94) and (1.41, 1.92) ..
					(1.45, 1.89) .. controls (1.48, 1.86) and (1.49, 1.82) ..
					(1.49, 1.78) .. controls (1.49, 1.74) and (1.48, 1.7) ..
					(1.45, 1.67) .. controls (1.41, 1.64) and (1.37, 1.62) ..
					(1.33, 1.62) .. controls (1.29, 1.62) and (1.25, 1.64) ..
					(1.22, 1.67) .. controls (1.19, 1.7) and (1.17, 1.74) ..
					(1.17, 1.78) .. controls (1.17, 1.82) and (1.19, 1.86) ..
					(1.22, 1.89) .. controls (1.25, 1.92) and (1.29, 1.94) ..
					(1.33, 1.94);
					
					\path[fill=c191716,even odd rule] (1.49, 2.47) .. controls (1.56, 2.49) and (1.65, 2.49) ..
					(1.72, 2.47) .. controls (1.8, 2.45) and (1.87, 2.42) ..
					(1.93, 2.36) .. controls (1.99, 2.3) and (2.03, 2.23) ..
					(2.05, 2.15) .. controls (2.06, 2.08) and (2.06, 1.99) ..
					(2.04, 1.92) .. controls (2.03, 1.88) and (2.06, 1.84) ..
					(2.1, 1.83) .. controls (2.13, 1.82) and (2.17, 1.85) ..
					(2.18, 1.88) .. controls (2.2, 1.98) and (2.21, 2.09) ..
					(2.18, 2.19) .. controls (2.16, 2.29) and (2.11, 2.38) ..
					(2.03, 2.46) .. controls (1.96, 2.53) and (1.86, 2.58) ..
					(1.76, 2.61) .. controls (1.66, 2.63) and (1.55, 2.63) ..
					(1.45, 2.61) .. controls (1.42, 2.6) and (1.39, 2.56) ..
					(1.4, 2.52) .. controls (1.41, 2.48) and (1.45, 2.46) ..
					(1.49, 2.47) --cycle
					(1.43, 2.26) .. controls (1.49, 2.28) and (1.55, 2.28) ..
					(1.6, 2.26) .. controls (1.66, 2.25) and (1.71, 2.22) ..
					(1.75, 2.18) .. controls (1.8, 2.14) and (1.82, 2.09) ..
					(1.84, 2.03) .. controls (1.85, 1.98) and (1.85, 1.92) ..
					(1.83, 1.86) .. controls (1.83, 1.82) and (1.85, 1.78) ..
					(1.89, 1.78) .. controls (1.93, 1.77) and (1.96, 1.79) ..
					(1.97, 1.83) .. controls (1.99, 1.91) and (1.99, 1.99) ..
					(1.97, 2.07) .. controls (1.96, 2.15) and (1.91, 2.22) ..
					(1.85, 2.28) .. controls (1.79, 2.34) and (1.72, 2.38) ..
					(1.64, 2.4) .. controls (1.56, 2.42) and (1.48, 2.42) ..
					(1.4, 2.4) .. controls (1.36, 2.39) and (1.34, 2.35) ..
					(1.34, 2.32) .. controls (1.35, 2.28) and (1.39, 2.25) ..
					(1.43, 2.26) --cycle
					(1.47, 2) .. controls (1.43, 2) and (1.39, 2.02) ..
					(1.38, 2.06) .. controls (1.38, 2.09) and (1.4, 2.13) ..
					(1.44, 2.14) .. controls (1.47, 2.15) and (1.51, 2.15) ..
					(1.55, 2.14) .. controls (1.59, 2.13) and (1.63, 2.11) ..
					(1.66, 2.08) .. controls (1.68, 2.06) and (1.7, 2.02) ..
					(1.71, 1.98) .. controls (1.72, 1.94) and (1.72, 1.9) ..
					(1.71, 1.87) .. controls (1.7, 1.83) and (1.67, 1.81) ..
					(1.63, 1.82) .. controls (1.59, 1.82) and (1.57, 1.86) ..
					(1.57, 1.9) .. controls (1.58, 1.92) and (1.58, 1.93) ..
					(1.58, 1.95) .. controls (1.57, 1.96) and (1.57, 1.97) ..
					(1.56, 1.98) .. controls (1.54, 1.99) and (1.53, 2) ..
					(1.52, 2.01) .. controls (1.5, 2.01) and (1.49, 2.01) ..
					(1.47, 2);
					
					\path[fill=c191716,even odd rule] (10.37, 13.11) .. controls (10.4, 13.05) and (10.41, 12.96) ..
					(10, 12.72) .. controls (9.98, 12.72) and (9.97, 12.71) ..
					(9.96, 12.7) -- (9.77, 11.86) -- (9.63, 11.78) -- (9.62, 12.52) .. controls (9.39, 12.41) and (9.29, 12.38) ..
					(9.15, 12.33) -- (9.09, 11.97) -- (8.96, 11.9) -- (8.93, 12.27) -- (8.63, 12.48) -- (8.75, 12.56) -- (9.08, 12.43) .. controls (9.2, 12.53) and (9.27, 12.61) ..
					(9.48, 12.74) -- (8.86, 13.12) -- (9, 13.2) -- (9.81, 12.95) .. controls (9.83, 12.96) and (9.84, 12.96) ..
					(9.85, 12.97) .. controls (10.27, 13.21) and (10.34, 13.16) ..
					(10.37, 13.11) --cycle
					(10.37, 13.11);
					
				\end{scope}
				
				
		}}
		\begin{tikzpicture}[scale=.7,transform shape]
			\path (0,0) pic[scale=.3]{rada};
			\path (2.9,2) coordinate (O)
			+(-30:3) coordinate (y) node[above] {$y$}
			+(90:3) coordinate (z) node[above] {$z$}
			+(90:1.8) coordinate (B)
			(.2,0) coordinate (M)
			($(O)!1.3!(M)$) coordinate (x) node[above] {$x$}
			($(O)!-1.5!(y)$) coordinate (y')
			($(O)!-1!(y)$) coordinate (A)
			($(A)!1.5!(B)$) coordinate (v) node[above] {$\overrightarrow{v}$}
			($(A)!1.9!(B)$) coordinate (v')
			;
			%			\draw[gray,xstep = 1, ystep = 1] (0,0) grid (5,5);
			
			\draw[-stealth,dashed] (O)--(z);
			\draw[-stealth,dashed] (O)--(y);
			\draw[-stealth,dashed] (O)--(x);
			\draw[dashed] (O)--(y')
			(A)--(v')
			;
			\draw (0.4,.55)--(B) node[midway,left] {$R$};
			\draw[-stealth] (B)--(v);
			\draw[-stealth] (.5,1) to[bend left = 30] (1.5,0.1) node[right] {$\theta$};
			
			\foreach \x/\g in {B/-45,M/-90,O/-90,A/-90}\fill (\x) circle (1.5pt)+(\g:3mm) node{$\x$};
			
		\end{tikzpicture}
		
	\end{center}
	\shortans{2630}
	\loigiai{
		\begin{center}
			\tikzset{rada/.pic={
					\definecolor{c191716}{RGB}{25,23,22}
					
					\begin{scope}[line cap=round,line join=round]
						\path[fill=white,nonzero rule] (0, 13.21) -- (10.42, 13.21) -- (10.42, 0.02) -- (0, 0.02) --cycle
						(0, 13.21);
						
						\path[fill=white,nonzero rule] (0, 13.21) -- (10.42, 13.21) -- (10.42, 0.02) -- (0, 0.02) --cycle
						(0, 13.21);
						
						\path[fill=c191716,nonzero rule] (0.03, 1.5) .. controls (0, 1.74) and (0.05, 1.99) ..
						(0.18, 2.21) .. controls (0.18, 2.21) and (0.18, 2.21) ..
						(0.18, 2.22) .. controls (0.2, 2.23) and (0.22, 2.23) ..
						(0.24, 2.22) -- (1.73, 0.74) -- (1.73, 0.74) .. controls (1.73, 0.74) and (1.73, 0.73) ..
						(1.73, 0.73) .. controls (1.75, 0.71) and (1.74, 0.69) ..
						(1.72, 0.67) .. controls (1.58, 0.59) and (1.42, 0.54) ..
						(1.25, 0.53) .. controls (1.09, 0.51) and (0.92, 0.53) ..
						(0.77, 0.59) .. controls (0.69, 0.61) and (0.62, 0.65) ..
						(0.55, 0.69) .. controls (0.48, 0.74) and (0.41, 0.79) ..
						(0.35, 0.85) .. controls (0.17, 1.03) and (0.06, 1.26) ..
						(0.03, 1.5);
						
						\path[fill=c191716,even odd rule] (0.39, 0.71) -- (0.06, 0.03) -- (1.2, 0.03) -- (1.2, 0.03) -- (1.22, 0.03) -- (1.04, 0.44) .. controls (0.94, 0.45) and (0.84, 0.47) ..
						(0.74, 0.51) .. controls (0.66, 0.54) and (0.58, 0.58) ..
						(0.5, 0.62) .. controls (0.46, 0.65) and (0.42, 0.68) ..
						(0.39, 0.71);
						
						\path[fill=c191716,even odd rule] (1.21, 1.76) .. controls (1.24, 1.79) and (1.28, 1.79) ..
						(1.31, 1.76) .. controls (1.34, 1.73) and (1.34, 1.69) ..
						(1.31, 1.66) -- (1.03, 1.38) .. controls (1, 1.35) and (0.95, 1.35) ..
						(0.93, 1.38) .. controls (0.9, 1.41) and (0.9, 1.45) ..
						(0.93, 1.48) --
						(1.21, 1.76);
						
						\path[fill=c191716,nonzero rule] (1.33, 1.94) .. controls (1.37, 1.94) and (1.41, 1.92) ..
						(1.45, 1.89) .. controls (1.48, 1.86) and (1.49, 1.82) ..
						(1.49, 1.78) .. controls (1.49, 1.74) and (1.48, 1.7) ..
						(1.45, 1.67) .. controls (1.41, 1.64) and (1.37, 1.62) ..
						(1.33, 1.62) .. controls (1.29, 1.62) and (1.25, 1.64) ..
						(1.22, 1.67) .. controls (1.19, 1.7) and (1.17, 1.74) ..
						(1.17, 1.78) .. controls (1.17, 1.82) and (1.19, 1.86) ..
						(1.22, 1.89) .. controls (1.25, 1.92) and (1.29, 1.94) ..
						(1.33, 1.94);
						
						\path[fill=c191716,even odd rule] (1.49, 2.47) .. controls (1.56, 2.49) and (1.65, 2.49) ..
						(1.72, 2.47) .. controls (1.8, 2.45) and (1.87, 2.42) ..
						(1.93, 2.36) .. controls (1.99, 2.3) and (2.03, 2.23) ..
						(2.05, 2.15) .. controls (2.06, 2.08) and (2.06, 1.99) ..
						(2.04, 1.92) .. controls (2.03, 1.88) and (2.06, 1.84) ..
						(2.1, 1.83) .. controls (2.13, 1.82) and (2.17, 1.85) ..
						(2.18, 1.88) .. controls (2.2, 1.98) and (2.21, 2.09) ..
						(2.18, 2.19) .. controls (2.16, 2.29) and (2.11, 2.38) ..
						(2.03, 2.46) .. controls (1.96, 2.53) and (1.86, 2.58) ..
						(1.76, 2.61) .. controls (1.66, 2.63) and (1.55, 2.63) ..
						(1.45, 2.61) .. controls (1.42, 2.6) and (1.39, 2.56) ..
						(1.4, 2.52) .. controls (1.41, 2.48) and (1.45, 2.46) ..
						(1.49, 2.47) --cycle
						(1.43, 2.26) .. controls (1.49, 2.28) and (1.55, 2.28) ..
						(1.6, 2.26) .. controls (1.66, 2.25) and (1.71, 2.22) ..
						(1.75, 2.18) .. controls (1.8, 2.14) and (1.82, 2.09) ..
						(1.84, 2.03) .. controls (1.85, 1.98) and (1.85, 1.92) ..
						(1.83, 1.86) .. controls (1.83, 1.82) and (1.85, 1.78) ..
						(1.89, 1.78) .. controls (1.93, 1.77) and (1.96, 1.79) ..
						(1.97, 1.83) .. controls (1.99, 1.91) and (1.99, 1.99) ..
						(1.97, 2.07) .. controls (1.96, 2.15) and (1.91, 2.22) ..
						(1.85, 2.28) .. controls (1.79, 2.34) and (1.72, 2.38) ..
						(1.64, 2.4) .. controls (1.56, 2.42) and (1.48, 2.42) ..
						(1.4, 2.4) .. controls (1.36, 2.39) and (1.34, 2.35) ..
						(1.34, 2.32) .. controls (1.35, 2.28) and (1.39, 2.25) ..
						(1.43, 2.26) --cycle
						(1.47, 2) .. controls (1.43, 2) and (1.39, 2.02) ..
						(1.38, 2.06) .. controls (1.38, 2.09) and (1.4, 2.13) ..
						(1.44, 2.14) .. controls (1.47, 2.15) and (1.51, 2.15) ..
						(1.55, 2.14) .. controls (1.59, 2.13) and (1.63, 2.11) ..
						(1.66, 2.08) .. controls (1.68, 2.06) and (1.7, 2.02) ..
						(1.71, 1.98) .. controls (1.72, 1.94) and (1.72, 1.9) ..
						(1.71, 1.87) .. controls (1.7, 1.83) and (1.67, 1.81) ..
						(1.63, 1.82) .. controls (1.59, 1.82) and (1.57, 1.86) ..
						(1.57, 1.9) .. controls (1.58, 1.92) and (1.58, 1.93) ..
						(1.58, 1.95) .. controls (1.57, 1.96) and (1.57, 1.97) ..
						(1.56, 1.98) .. controls (1.54, 1.99) and (1.53, 2) ..
						(1.52, 2.01) .. controls (1.5, 2.01) and (1.49, 2.01) ..
						(1.47, 2);
						
						\path[fill=c191716,even odd rule] (10.37, 13.11) .. controls (10.4, 13.05) and (10.41, 12.96) ..
						(10, 12.72) .. controls (9.98, 12.72) and (9.97, 12.71) ..
						(9.96, 12.7) -- (9.77, 11.86) -- (9.63, 11.78) -- (9.62, 12.52) .. controls (9.39, 12.41) and (9.29, 12.38) ..
						(9.15, 12.33) -- (9.09, 11.97) -- (8.96, 11.9) -- (8.93, 12.27) -- (8.63, 12.48) -- (8.75, 12.56) -- (9.08, 12.43) .. controls (9.2, 12.53) and (9.27, 12.61) ..
						(9.48, 12.74) -- (8.86, 13.12) -- (9, 13.2) -- (9.81, 12.95) .. controls (9.83, 12.96) and (9.84, 12.96) ..
						(9.85, 12.97) .. controls (10.27, 13.21) and (10.34, 13.16) ..
						(10.37, 13.11) --cycle
						(10.37, 13.11);
						
					\end{scope}
					
					
			}}
			\begin{tikzpicture}[scale=.7,transform shape]
				\path (0,0) pic[scale=.3]{rada};
				\path (2.9,2) coordinate (O)
				+(-30:3) coordinate (y) node[above] {$y$}
				+(90:3) coordinate (z) node[above] {$z$}
				+(90:1.8) coordinate (B)
				(.2,0) coordinate (M)
				($(O)!1.3!(M)$) coordinate (x) node[above] {$x$}
				($(O)!-1.5!(y)$) coordinate (y')
				($(O)!-1!(y)$) coordinate (A)
				($(A)!1.5!(B)$) coordinate (v) node[above] {$\overrightarrow{v}$}
				($(A)!1.9!(B)$) coordinate (v') node[above] {$C$}
				;
				%			\draw[gray,xstep = 1, ystep = 1] (0,0) grid (5,5);
				
				\draw[-stealth,dashed] (O)--(z);
				\draw[-stealth,dashed] (O)--(y);
				\draw[-stealth,dashed] (O)--(x);
				\draw[dashed] (O)--(y')
				(A)--(v')
				(M)--(v')
				;
				\draw (0.4,.55)--(B) node[midway,left] {$R$};
				\draw[-stealth] (B)--(v);
				%			\draw[-stealth] (.5,1) to[bend left = 30] (1.5,0.1) node[right] {$\theta$};
				
				\foreach \x/\g in {B/-45,M/-90,O/-90,A/-90}\fill (\x) circle (1.5pt)+(\g:3mm) node{$\x$}
				;
				\fill (v') circle(1.5pt);
				\draw[thin,fill=gray] pic[angle radius = 40pt, "$60^\circ$", angle eccentricity = 1.3]{ angle = v'--M--B};
			\end{tikzpicture}
		\end{center}
		Ta có điểm $A(0;-0{,}6;0)$; $B(0;0;0{,}3)$, $M(0{,}4;0;0)$.\\
		Vậy $\overrightarrow{AB}=(0;0{,}6;0{,}3)$ nên véctơ $\overrightarrow{u}=(0;2;1)$.\\
		Do đó phương trình đường thẳng $AB$ là $\heva{&x=0\\&y=2t\\&z=0{,}3t.}$\\
		Điểm $C$ thuộc đường thẳng $AB$ nên có tọa độ là $C(0;2t;0{,}3t)$.
		Vì $C$ ứng với góc quay $60^\circ$ của radar nên $\widehat{BMC}=60^\circ$.\\
		$\overrightarrow{MB}=(-0{,}4;0;0{,}3)$; $\overrightarrow{MC}=(-0{,}4;2t;0{,}3+t)$.\\
		Ta có $$\cos \widehat{BMC}=\dfrac{\overrightarrow{MB}\cdot \overrightarrow{MC}}{\left|\overrightarrow{MB}\right|\cdot \left|\overrightarrow{MC}\right|}=\dfrac{0{,}25+0{,}3t}{0{,}5\cdot \sqrt{5t^2+0{,}6t+0{,}25}}=\dfrac{1}{2}.\quad (*)$$
		Rút gọn $(*)$ ta thu được phương trình bậc hai
		$$
		3{,}56t^2-1{,}8t-0{,}75=0.
		$$
		Phương trình trên có hai nghiệm là $t\approx -0{,}27120$ (loại) và $t\approx 0{,}776819$.\\
		Vậy $a+b+c=0+2t+0{,}3+t=0{,}3+3t=2{,630457}$ (km) $\approx 2630$ (m).
	}
\end{ex}
\Closesolutionfile{ans}
% % \setcounter{section}{0}
\section*{TỔNG HỢP BÀI TẬP}
% \subsection{Kiến thức trọng tâm}


% \subsection{Ví dụ minh họa}
% %\hideansEX{vd}	% ví dụ luôn ẩn lời giải
% %\dotlinefull{vd}	%ví dụ luôn ẩn lời giải và hiện dòng kẻ



% \subsection{Bài tập tự luyện}
% \hideansEX{ex}	%bài tập luôn ẩn lời giải
%\dotlinefull{ex}	%bài tập luôn ẩn lời giải và hiện dòng kẻ

% \Opensolutionfile{ans}[ans/ansBTchoice]

% \subsubsection{Trả lời các câu hỏi sau, mỗi câu hỏi chỉ chọn một phương án}
% \paragraph{Mức độ N}
% \paragraph{Mức độ H}
% \paragraph{Mức độ V}
% \paragraph{Mức độ C}

% \Closesolutionfile{ans}
\setcounter{ex}{0}
\Opensolutionfile{ans}[ans/ansBTchoiceTF]

\subsection{Trả lời các câu hỏi sau, trong mỗi ý a), b), c), d), \ldots ở mỗi câu, thí sinh chọn đúng hoặc sai}
% \paragraph{Mức độ N}
\begin{ex}%[50 Đề minh họa tốt nghiệp 2025 - Đề 13]%[Lê Hữu Kiệt - Lê Quân]%[2D4N1-2]
Biết $F(x)$ là một nguyên hàm của hàm số $f(x)=\dfrac{x^2+1}{x}$ trên khoảng $(0;+\infty)$.
\choiceTF
{\True $f(x)=x+\dfrac{1}{x}$}
{$F(x)=f'(x), \forall x \in (0;+\infty)$}
{$F(x)=\dfrac{1}{2}x^2-\dfrac{1}{x^2}+C$, với $C$ là hằng số}
{Biết rằng đồ thị của hàm số $F(x)$ đi qua $M\left(\mathrm{e};\dfrac{\mathrm{e}^2}{2}\right)$. Khi đó $F(1)=\dfrac{1}{2}$}
\loigiai{
\begin{itemchoice}
\itemch Trên khoảng $(0;+\infty)$, ta có $f(x)=\dfrac{x^2+1}{x}=x+\dfrac{1}{x}$.
\itemch Ta có $F(x)=\displaystyle\int f(x)\mathrm{\,d}x$, $\forall x\in(0;+\infty)$.
\itemch Với $x\in(0;+\infty)$, ta có $\left(\dfrac{1}{2}x^2-\dfrac{1}{x^2}+C\right)'=x+\dfrac{2}{x^3}=\dfrac{x^4+2}{x^3}\ne f(x)$.
\itemch Trên khoảng $(0;+\infty)$, ta có $\displaystyle\int f(x)\mathrm{\,d}x = \displaystyle\int \left(x+\dfrac{1}{x}\right) \mathrm{\,d}x = \dfrac{1}{2}x^2+\ln x + C$, với $C$ là hằng số.\\
Do $M\left(\mathrm{e};\dfrac{\mathrm{e}^2}{2}\right)$ thuộc đồ thị hàm số $F(x)$, suy ra $\dfrac{\mathrm{e}^2}{2} = \dfrac{1}{2}\cdot \mathrm{e}^2 + \ln \mathrm{e} + C \Leftrightarrow C =-1$.\\
Suy ra $F(x)=\dfrac{1}{2}x^2+\ln x - 1$.\\
Khi đó $F(1)=-\dfrac{1}{2}.$
\end{itemchoice}
}
\end{ex}

% \paragraph{Mức độ H}
\begin{ex}%[1D6H3-3]
Cho hàm số $f(x)=\log_2\left(x^2-4x+5\right)$ có đồ thị là $(C)$ và điểm cực trị của đồ thị là $M$.
\choiceTF
{\True Tập xác định của hàm số đã cho là $\mathscr{D}=\mathbb{R}$}
{Đạo hàm của hàm số đã cho là $f'(x)=\dfrac{2x-4}{x^2-4x+5}$}
{\True Tọa độ của điểm $M$ là $(2;0)$}
{\True Đường thẳng $y=1$ cắt đồ thị $(C)$ tại hai điểm phân biệt $A$, $B$ thì tam giác $MAB$ có diện tích bằng $1$}
\loigiai{
\begin{itemchoice}
\itemch Điều kiện xác định: $x^2-4x+5>0 \Leftrightarrow (x-2)^2+1>0$. Luôn đúng.\\
Vậy tập xác định của hàm số đã cho là $\mathscr{D}=\mathbb{R}$.
\itemch
Ta có \[f(x)=\log _2\left(x^2-4 x+5\right)  \Rightarrow f(x)=\dfrac{2 x-4}{\left(x^2-4 x+5\right) \ln 2} .\]
\itemch Ta có \begin{eqnarray*}
f(x)=0
& \Leftrightarrow& \dfrac{2 x-4}{\left(x^2-4 x+5\right) \ln 2}=0 \\
& \Leftrightarrow& 2 x-4=0 \Leftrightarrow x=2.
\end{eqnarray*}
Khi đó $f(2)=\log \left(2^2-4.2+5\right)=\log 1=0$. Vậy $M(2;0)$.
\itemch Phương trình hoành độ giao điểm là
\begin{eqnarray*}
&&\log _2\left(x^2-4 x+5\right)=1 \\
&\Leftrightarrow& x^2-4 x+5=2 \\
&\Leftrightarrow& x^2-4 x+3=0 \\
&\Leftrightarrow&\hoac{
&x=1 \Rightarrow A(1 ; 1) \\
&x=3 \Rightarrow B(3 ; 1) .
}
\end{eqnarray*}
Diện tích tam giác $MAB$ là
\begin{eqnarray*}
S_{M A B}&=&\dfrac{1}{2}\left|\left(x_M-x_A\right)\left(y_B-y_A\right)-\left(x_B-x_A\right)\left(y_M-y_A\right)\right| \\
&=&\dfrac{1}{2}|(2-1)(1-1)-(3-1)(0-1)|=1 .
\end{eqnarray*}
\end{itemchoice}
}
\end{ex}

\begin{ex}%[2H5H3-4]%[TEX ĐỀ MOON 2025]%[Nguyễn Cường]
Một máy bay di chuyển từ sân bay $A$ với tọa độ $A(0;0;0)$ đến sân bay $B$ tại tọa độ $B(760;120;10)$ (đơn vị tính là km). Trên hành trình, máy bay sẽ đi qua vùng kiểm soát không lưu trung gian có bán kính $100$ km, với tâm trạm kiểm soát đặt tại tọa độ $O(380;60;0)$. Máy bay bay với vận tốc không đổi, hoàn thành quãng đường trong $1$ giờ $25$ phút.
\choiceTF
{\True Phương trình tham số của đường bay từ $A$ đến $B$ được cho bởi $\heva{& x=760t \\ & y=120t\\ & z=10t}$, $t\in[0;1{,}42]$ ($t$ được tính bằng giờ)}
{\True Máy bay đi vào phạm vi kiểm soát không lưu (bán kính $100$ km, tâm tại $O(380;60;0)$) tại thời điểm $t=0{,}5$}
{Quãng đường từ $A$ đến $B$ theo đường bay là $766$ km (làm tròn đến hàng đơn vị)}
{Nếu máy bay bay trong vùng kiểm soát trong $15$ phút, nó sẽ bay đúng $\dfrac{1}{6}$ quãng đường từ lúc vào đến khi ra khỏi vùng này}
\loigiai{
\begin{itemchoice}
\itemch Đổi đơn vị $1$ giờ $25$ phút tương ứng với $1{,}41\overline{6}$ giờ.\\
Ta có $\overrightarrow{AB}=(760;120;10)$ là véc-tơ chỉ phương của phương trình đường bay từ $A$ đến $B$.\\
Phương trình tham số của đường bay này là $\heva{& x=760t \\ & y=120t\\ & z=10t}$, $t\in[0;1{,}42]$ ($t$ được tính bằng giờ).
\itemch Phương trình mặt cầu mô tả vùng kiểm soát không lưu trung gian là \[(S)\colon (x-380)^2+(y-60)^2+z^2=100^2.\]
Khi $t=0{,}5$ thì máy bay đang ở tọa độ $C(380;60;5)$.\\
Ta có $\overrightarrow{OC}=(0;0;5)$, suy ra $OC=5<R$ nên máy bay đã đi vào phạm vi kiểm soát không lưu.
\itemch Ta có $\overrightarrow{AB}=(760;120;10)$ nên $AB=\sqrt{760^2+120^2+10^2}\approx 770$\,(km).
\itemch Máy bay bay $15$ phút thì chỉ hoàn thành $\dfrac{25}{85}=\dfrac{5}{17}$ quãng đường từ lúc cất cánh đến khi hoàn thành chuyến bay.
\end{itemchoice}
}
\end{ex}

\begin{ex}%[2H5H3-3]%[TEX ĐỀ MOON 2025]%[Lê Hữu Kiệt]
Trong không gian với hệ tọa độ $Oxyz$, cho mặt phẳng $(P)\colon x-2y-2z-1=0$ và hai điểm $A(1;1;2)$, $B(3;2;-3)$.
\def\dotEX{}
\choiceTF
{\True Điểm $A$ không thuộc mặt phẳng $(P)$.}
{Khoảng cách từ điểm $B$ đến mặt phẳng $(P)$ bằng $3$.}
{Phương trình tham số của đường thẳng $AB$ là $\heva{& x=1+3t \\ & y=1+2t\\ & z=2-3t.}$}
{Mặt cầu $(S)$ có tâm $I$ thuộc trục $Oz$ và đi qua hai điểm $A$, $B$ có phương trình là \begin{center}
$x^2+y^2+z^2-8z+2=0$.
\end{center}}
\loigiai{
\begin{itemchoice}
\itemch Thay tọa độ điểm $A$ vào phương trình mặt phẳng $(P)$ ta được $1-2\cdot1-2\cdot2-1=-6\ne0$ nên $A\not\in(P)$.
\itemch Khoảng cách từ điểm $B$ đến mặt phẳng $(P)$ là
\[\mathrm{d}\left(B,(P)\right)=\dfrac{|3-2\cdot2-2\cdot(-3)-1|}{\sqrt{1^2+(-2)^2+(-2)^2}}=\dfrac{4}{3}.\]
\itemch Ta có $\overrightarrow{AB}=(2;1;-5)$.\\
Phương trình tham số của đường thẳng $AB$ đi qua điểm $A$, nhận $\overrightarrow{AB}$ là vectơ chỉ phương là
\[\heva{&x=1+2t\\&y=1+t\\&z=2-5t.}\]
\itemch Gọi $I(0;0;x_I)$ là tâm của mặt cầu $(S)$.\\
Ta có $IA=\sqrt{1^2+1^2+(2-z_I)^2}=\sqrt{6-4z_I+z_I^2}$,\\$IB=\sqrt{3^2+2^2+(-3-z_I)^2}=\sqrt{22+6z_I+z_I^2}$.\\
Do $A$, $B$ thuộc mặt cầu $(S)$ nên
\allowdisplaybreaks
\begin{eqnarray*}
&&IA=IB\\
&\Leftrightarrow&IA^2=IB^2 \\
&\Leftrightarrow&6-4z_I+z_I^2=22+6z_I+z_I^2\\
&\Leftrightarrow&z_I=-\dfrac{8}{5}
\end{eqnarray*}
Suy ra $I\left(0;0;-\dfrac{8}{5}\right)$, $IA=\dfrac{\sqrt{374}}{5}$.\\
Khi đó phương trình mặt cầu $(S)$ là
\begin{eqnarray*}
&& x^2+y^2+\left(z+\dfrac{8}{5}\right)^2=\dfrac{374}{25} \\
&\Leftrightarrow& x^2+y^2+z^2+\dfrac{16}{5}z-\dfrac{62}{5}=0.
\end{eqnarray*}
\end{itemchoice}
}
\end{ex}

\begin{ex}%[2H5H2-3]
Trong không gian $Oxyz$, gọi $d$ là giao tuyến của mặt phẳng $(P)\colon 2x-y-2z-3=0$ và mặt phẳng $(Q)\colon x-2y-z-6=0$. Xét tính đúng sai của các mệnh đề sau
\choiceTF
{Một véc-tơ pháp tuyến của mặt phẳng $(P)$ là $(2;-1;-3)$}
{\True Đường thẳng $d$ có vectơ chỉ phương là $\overrightarrow{u}_d=(2;0;2)$}
{\True Điểm $A(0;-3;0)$ thuộc đường thẳng $d$}
{\True Phương trình tham số của đường thẳng $d$ là $\heva{& x=1+t \\ & y=-3\\ & z=1+t}$}
\loigiai{
\begin{itemchoice}
\itemch
Một véc-tơ pháp tuyến của mặt phẳng $(P)$ là $(2;-1;-2)$.
\itemch
Vì $d$ là giao tuyến của mặt phẳng $(P)$ và $(Q)$ nên $d$ có hai véc-tơ pháp tuyến là\\ $\overrightarrow{n}_{(P)}=(2;-1;-2)$; $\overrightarrow{n}_{(Q)}=(1;-2;-1)$.\\
Suy ra $\overrightarrow{u}_{d}=\left[\overrightarrow{n}_{(P)};\overrightarrow{n}_{(Q)}\right]=(-3;0;-3)=-3(1;0;1)$.\\
Vậy đường thẳng $d$ có một véc-tơ chỉ phương là $(2;0;2)$.
\itemch Nếu $A\in d \Rightarrow \heva{&A\in(P)\\&A\in (Q).}$\\
Cho $z=0$ ta xét hệ phương trình $\heva{&2x-y=3\\&x-2y=6}\Leftrightarrow \heva{&x=0\\&y=-3.}$\\
Vậy $A(0;-3;0)\in d$.
\itemch Đường thẳng  $d$ đi qua điểm $A(0;-3;0)$ và có véc-tơ chỉ phương $\overrightarrow{u}_{d}=(1;0;1)$ có phương trình tham số là $\heva{&x=t\\&y=-3\\&z=t.}$\\
Khi đó đường thẳng $d$ đi qua điểm $(1;-3;1)$ và có véc-tơ chỉ phương $\overrightarrow{u}_{d}=(1;0;1)$ có phương trình tham số là $\heva{&x=1+t\\&y=-3\\&z=1+t.}$
\end{itemchoice}
}
\end{ex}

\begin{ex}%[2H5H1-3]
Trong không gian với hệ tọa độ $Oxyz$ (đơn vị trên mỗi trục tọa độ là km), một máy bay đang ở vị trí $A(3;-2{,}5;0{,}5)$ và sẽ hạ cánh ở vị trí $B(3;8{,}5;0)$ trên đường băng (hình minh họa bên dưới). Có một lớp mây được mô phỏng bởi một mặt phẳng $(\alpha)$ đi qua ba điểm $M(9;0;0)$, $N(0;-9;0)$ và $P(0;0;0{,}9)$.

{\centering \begin{tikzpicture}[scale=0.9,font=\footnotesize,>=stealth, line join=round, line cap=round]
\def\xmin{-2.5}
\def\ymin{-3.5} \def\ymax{1.5}
\def\zmax{2}
\coordinate (O) at (0,0);
\coordinate (M) at (-1.5,-1.5);
\coordinate (N) at (-2.5,0);
\coordinate (P) at (0,1);
\coordinate (A) at (-1.7,1.4);
\coordinate (B) at (1,-0.5);
\coordinate (C) at ($(A)!0.44!(B)$);
\coordinate (X) at (intersection of A--B and O--P);
\draw[->] (0,0)--(\xmin+0.3,\xmin+0.3)node [below]{$x$};
\draw[->] (0,0)--(\ymax,0) node [above]{$y$};
\draw[dashed] (0,0)--(\ymin,0);
\draw[->] (0,0)--(0,\zmax) node [left]{$z$};
\node at (O) [below right,xshift=-0.1cm]{$O$};
\draw (M)node[below right]{$M$}--(N)node[above]{$N$}--(P)node[right]{$P$}--cycle;
\draw (A)--(C) (X)--(B);
\draw[dashed] (C)--(X);
\fill (A)node[above]{$A$} circle(2pt);
\fill (B)node[below right]{$B$} circle(2pt);
\fill (C)node[below left]{$C$} circle(2pt);
\end{tikzpicture}\par}\noindent

\choiceTF
{\True Khoảng cách hai điểm $A$ và $B$ bằng $11$ km (làm tròn kết quả đến hàng đơn vị)}
{\True Biết tốc độ của máy bay là $250$ km/h trên quãng đường $AB$ thì sau $2{,}64$ phút (làm tròn đến hàng phần trăm) máy bay từ vị trí $A$ hạ cánh tại vị trí $B$}
{Phương trình mặt phẳng $(\alpha)$ là $x-y-10z-9=0$}
{Độ cao của máy bay khi xuyên qua đám mây để hạ cánh là $0{,}35$ km}
\loigiai{
\begin{itemchoice}
\itemch Ta có $A B=\sqrt{0^2+11^2+(-0{,}5)^2}=\sqrt{121{,}25} \approx 11 \mathrm{~km}$.
\itemch Thời gian để máy bay từ vị trí $A$ hạ cánh tại vị trí $B$ là\\
$\dfrac{\sqrt{121{,}25}}{250}(\text{h})=\dfrac{\sqrt{121{,}25}}{250} \cdot 60~ (\text{phút}) \approx 2{,}64$ (phút).
\itemch Giả sử điểm $C\left(x_C ; y_C ; z_C\right)$ là vị trí mà máy  bay xuyên qua đám mây để hạ cánh, suy ra $C \in(\alpha)$. Mặt phẳng $(\alpha)$ có phương trình
\begin{eqnarray*}
&&\dfrac{x}{9}-\dfrac{y}{9}+\dfrac{z}{0{,}9}=1 \\
&\Leftrightarrow&x-y+10 z-9=0.
\end{eqnarray*}
\itemch Do hai vectơ $\overrightarrow{AC}$, $\overrightarrow{AB}$ cùng hướng nên tồn tại số thực $t>0$ sao cho
\begin{eqnarray*}
\overrightarrow{AC}=t \overrightarrow{AB}&\Leftrightarrow&\heva{
&x_C-3=0 \\
&y_C+2{,}5=11 t \\
&z_C-0{,}5=-0{,}5 t
} \\
&\Leftrightarrow&\heva{
&x_C=3 \\
&y_C=-2{,}5+11 t \\
&z_C=0{,}5-0{,}5 t.
}
\end{eqnarray*}
Vì $C \in(\alpha)$ nên $3-(-2{,}5+10 t)+10(0{,}5-0{,}5 t)-9=0 \Leftrightarrow t=0{,}1$.\\
Suy ra $C(3 ;-1{,}4 ; 0{,}45)$.\\
Vậy độ cao của máy bay khi xuyên qua đám mây là $0{,}45$ km
\end{itemchoice}
}
\end{ex}

\begin{ex}%[50 Đề minh họa tốt nghiệp 2025 - Đề 13]%[Lê Hữu Kiệt - Lê Quân]%[2D6H2-4]
Một thùng chứa $100$ quả táo trong đó có $80\%$ số quả táo được dán nhãn, số còn lại không được dán nhãn. Bạn Hoàng lấy ra một quả trong thùng, sau đó bạn Hà lấy ra một quả thứ hai.
\begin{itemize}
\item Gọi $A$ là biến cố: \lq\lq Quả táo bạn Hoàng lấy ra có dán nhãn\rq\rq.
\item Gọi $B$ là biến cố: \lq\lq Quả táo bạn Hà lấy ra có dán nhãn\rq\rq.
\end{itemize}
\choiceTF
{\True $\mathrm{P}(A)=\dfrac{4}{5}$}
{Xác suất có điều kiện $\mathrm{P}(B\mid A)=\dfrac{79}{100}$}
{\True Xác xuất bạn Hà lấy ra quả táo có dán nhãn bằng $0{,}8$}
{Biết rằng bạn Hà lấy ra quả táo có dán nhãn. Xác suất để Hoàng cũng lấy ra quả táo có dán nhãn là $20{,}2\%$ (làm tròn kết quả đến hàng phần mười theo đơn vị phần trăm)}
\loigiai{
Số quả táo được dán nhãn là $80\%\cdot 100=80$ quả; số táo không được dán nhãn là $20$ quả.\\
Khi đó $\mathrm{P}(A)=\dfrac{80}{100}=\dfrac{4}{5}$; $\mathrm{P}(\overline{A})=1-\dfrac{4}{5}=\dfrac{1}{5}$; $\mathrm{P}(B\mid A)=\dfrac{79}{99}$; $\mathrm{P}(B\mid\overline{A})=\dfrac{80}{99}$.
\begin{itemchoice}
\itemch Ta có $n(A)=80$, suy ra $\mathrm{P}(A)=\dfrac{80}{100}=\dfrac{4}{5}$.
\itemch Ta có $\mathrm{P}(B\mid A)=\dfrac{79}{99}$.
\itemch Áp dụng công thức xác suất toàn phần, ta có
\[ \mathrm{P}(B)=\mathrm{P}(A)\cdot \mathrm{P}(B\mid A) + \mathrm{P}(\overline{A})\cdot \mathrm{P}(B\mid\overline{A})=\dfrac{4}{5}\cdot\dfrac{79}{99}+\dfrac{1}{5}\cdot\dfrac{80}{99}=0{,}8.\]
\itemch Xác suất để Hoàng cũng lấy ra quả táo có dán nhãn biết bạn Hà lấy ra quả táo có dán nhãn là $\mathrm{P}(A\mid B)$. Áp dụng công thức Bayes ta có
\[ \mathrm{P}(A\mid B) =\dfrac{\mathrm{P}(A)\cdot \mathrm{P}(B\mid A)}{\mathrm{P}(B)} = \dfrac{\dfrac{4}{5}\cdot\dfrac{79}{99}}{0{,}8} \approx 79,{80}\%.\]
\end{itemchoice}
}
\end{ex}

\begin{ex}%[2D6H2-4]%[TEX Đề Moon 2025]%[Võ Nguyên Thạch]
Một xưởng máy sử dụng một loại linh kiện được sản xuất từ hai cơ sở I và II. Số linh kiện do cơ sở I sản xuất chiếm $61\%$, số linh kiện do cơ sở II sản xuất chiếm $39\%$. Tỉ lệ linh kiện đạt tiêu chuẩn của cơ sở I, cơ sở II lần lượt là $93\%$, $82\%$. Kiểm tra ngẫu nhiên một linh kiện ở xưởng máy. Xét các biến cố
\begin{itemize}
\item $A_1\colon$\lq\lq Linh kiện được kiểm tra do cơ sở I sản xuất\rq\rq.
\item $A_2\colon$\lq\lq Linh kiện được kiểm tra do cơ sở II sản xuất\rq\rq.
\item $B\colon$\lq\lq Linh kiện được kiểm tra đạt tiêu chuẩn\rq\rq.
\end{itemize}
Xét tính đúng sai của các mệnh đề sau
\choiceTF
{\True Xác suất $P(A_1)=0{,}61$}
{\True Xác suất có điều kiện $P(B\mid A_2)=0{,}82$}
{\True Xác suất $P(B)=0{,}8871$}
{Xác suất có điều kiện $P(A_1\mid B)=0{,}55$}
\loigiai{
\begin{itemchoice}
\itemch Do số linh kiện do cơ sở I sản xuất chiếm $61\%$ nên $P(A_1)=0{,}61$.
\itemch Do tỉ lệ linh kiện đạt tiêu chuẩn của cơ sở II là $82\%$ nên $P(B\mid A_2)=0{,}82$.
\itemch Áp dụng định lý xác suất toàn phần ta có
\allowdisplaybreaks
\begin{eqnarray*}
\mathrm{P}(B)&=&\mathrm{P}(A_1)\cdot \mathrm{P}(B|A_1)+\mathrm{P}(A_2)\cdot \mathrm{P}(B|A_2)\mathrm{P}(B)\\
&=&0{,}61\cdot 0{,}93+0{,}39\cdot 0{,}82\\
&=&0{,}5673+0{,}3198\\
&=&0,8871.
\end{eqnarray*}
\itemch Áp dụng công thức Bayes ta có
\allowdisplaybreaks
\begin{eqnarray*}
\mathrm{P}(A_1|B)&=&\dfrac{\mathrm{P}(A_1)\cdot \mathrm{P}(B|A_1)}{\mathrm{P}(B)}\\
&=&\dfrac{0{,}61\cdot 0{,}93}{0{,}8871}\\
&=&\dfrac{0{,}5673}{0{,}8871}\approx 0{,}6395.
\end{eqnarray*}
\end{itemchoice}
}
\end{ex}

\begin{ex}%[2D6H1-2]
Cho hai biến cố $A$ và $B$, biết $\mathrm{P}(A)=0{,}6$, $\mathrm{P}\left(\overline{B}\right)=0{,}2$, $\mathrm{P}(AB)=0{,}42$.
\choiceTF
{\True $\mathrm{P}(B)=0{,}8$}
{$A$ và $B$ là hai biến cố độc lập}
{$\mathrm{P}\left(\overline{A}B\right)=0{,}48$}
{\True $\mathrm{P}\left(B\mid \overline{A}\right)=0{,}95$}
\loigiai{
\begin{itemchoice}
\itemch {\bf Đúng.}\\ Ta có $\mathrm{P}(B)=1-\mathrm{P}\left(\overline{B}\right)=0{,}8$.
\itemch  {\bf Sai.}\\ Ta có $\mathrm{P}(A)\cdot \mathrm{P}(B)=0{,}6\cdot 0{,}8 =0{,}48$.\\
Suy ra $\mathrm{P}(AB)\neq \mathrm{P}(A)\cdot \mathrm{P}(B)$ nên hai biến cố $A$ và $B$ không phải là hai biến cố độc lập.
\itemch  {\bf Sai.}\\ Ta có sơ đồ cây như sau
\begin{center}
\begin{tikzpicture}[>=stealth,line cap=round,line join=round]
\path(0,0)node(a){Xác suất }
++(3,2)node(b){$ A $} ++(3,1.5)node(d){$ B $} (b)++(3,-1.5)node(e){$ \overline{B} $}
(a)++(3,-2)node(c){$\overline{A} $}++(3,1.5)node(f){$ B $}(c)++(3,-1.5)node(g){$ \overline{B} $}
;
\draw(a)--(b) (b)--(e) (b)--(d) (a)--(c) --(f) (c)--(g);
\end{tikzpicture}
\end{center}
Do đó
$\mathrm{P}(B)=\mathrm{P}\left(\overline{A}B\right)+ \mathrm{P}\left(AB\right)$.\\
Suy ra $\mathrm{P}\left(\overline{A}B\right)=\mathrm{P}(B)-\mathrm{P}\left(AB\right)=0{,}8-0{,}42 =0{,}38$.
\itemch {\bf Đúng.}\\ Ta có $\mathrm{P}\left(B\mid \overline{A}\right) =\dfrac{\mathrm{P}\left(\overline{A}B\right)}{\mathrm{P}\left(\overline{A}\right)}=\dfrac{0{,}38}{1-0{,}6}=0{,}95$.
\end{itemchoice}

}
\end{ex}

\begin{ex}%[2D6H1-2]%[TEX ĐỀ MOON 2025]%[Nguyễn Văn Hiệp]
Một chiếc hộp có $80$ viên bi, trong đó có $50$ viên bi màu đỏ và $30$ viên bi màu vàng; các viên bi có kích thước và khối lượng như nhau. Sau khi kiểm tra, người ta thấy có $60\%$ số viên bi màu đỏ đánh số và $50\%$ số viên bi màu vàng có đánh số, những viên bi còn lại không đánh số. Xét tính đúng sai của các mệnh đề sau
\choiceTF
{\True Số viên bi màu đỏ có đánh số là $30$}
{\True Số viên bi màu vàng không đánh số là $15$}
{Lấy ra ngẫu nhiên một viên bi trong hộp. Xác suất để viên bi được lấy ra có đánh số là $\dfrac{3}{5}$}
{\True Lấy ra ngẫu nhiên một viên bi trong hộp. Xác suất để viên bi được lấy ra không có đánh số $\dfrac{7}{16}$}
\loigiai
{
\begin{itemchoice}
\itemch $50 \times 60\% = 30$ bi đỏ có đánh số.
\itemch $30 \times 50\% = 15$ bi không đánh số.
\itemch
Đặt $C$ là biến cố \lq\lq Viên bi được lấy ra có đánh số\rq\rq.\\
$n\left(C\right)=30+15= 45$; $n\left(\Omega\right)=80$; $\mathrm{P}(C) = \dfrac{45}{80} = \dfrac{9}{16}$.
\itemch Đúng. Đặt $\overline{C}$ là biến cố \lq\lq Viên bi được lấy ra không có đánh số\rq\rq.\\ $\mathrm{P}\left(\overline{C}\right) =1-\mathrm{P}\left(C\right) =1 - \dfrac{9}{16} = \dfrac{7}{16}$.
\end{itemchoice}
}
\end{ex}

\begin{ex}%[2D4H3-1]
Cho hình phẳng $(H)$ giới hạn bởi các đồ thị hàm số $f(x)=x^2-2x-1$ và $g(x)=2x-4$. Xét hàm số $Q(x)=\dfrac{x^3}{3}-2x^2+3x$.
\choiceTF
{\True Phương trình $f(x)-g(x)=0$ có hai nghiệm phân biệt}
{Hiệu $f(x)-g(x) > 0$ với mọi $x \in(1; 3)$}
{\True Hàm số $Q(x)$ là một nguyên hàm của hàm số $f(x)-g(x)$}
{\True Diện tích hình phẳng $(H)$ bằng $Q(1)-Q(3)$}
\loigiai{
\begin{itemchoice}
\itemch Phương trình $f(x)=g(x)\Leftrightarrow x^2-2x-1=2x-4\Leftrightarrow x^2-4x+3=0\Leftrightarrow \hoac{&x=1\\&x=3}$.
\itemch Ta có bảng xét dấu sau
\begin{center}
\begin{tikzpicture}
\tkzTabInit[nocadre=false,lgt=2.5,espcl=2,deltacl=0.6]
{$x$ /0.6,$f(x)-g(x)$ /1.1}
{$-\infty$,$1$,$3$,$+\infty$}
\tkzTabLine{,+,$0$,-,$0$,+,}
\end{tikzpicture}
\end{center}
Từ bảng xét dấu suy ra $f(x)-(g(x)<0$ với $x\in(1;3)$.
\itemch
Ta có $\displaystyle\int (f(x)-g(x)) \mathrm{\, d}x=\displaystyle\int (x^2-4x+3) \mathrm{\, d}x=\dfrac{x^3}{3}-2x^2+3x+C$.\\
Vậy $Q(x)=\dfrac{x^3}{3}-2x^2+3x$ là một nguyên hàm của $f(x)-g(x)$.
\itemch
Diện tích hình phẳng $(H)$ là $S=\displaystyle\int\limits_1^3 |x^2-4x+3| \mathrm{\, d}x=\displaystyle\int\limits_1^3 -(x^2-4x+3) \mathrm{\, d}x=Q(1)-Q(3)$.
\end{itemchoice}

}
\end{ex}

\begin{ex}%[2D4H2-6]%[TEX ĐỀ MOON 2025]%[Nguyễn Cường]
Để đảm bảo an toàn khi lưu thông trên đường, các xe ô tô khi dừng đèn đỏ phải cách nhau tối thiểu $5$ m. Một ô tô $A$ đang chạy với vận tốc $16$ m/s thì gặp ô tô $B$ đang dừng đèn đỏ nên ô tô $A$ hãm phanh và chuyển động chậm dần đều với vận tốc được biểu thị bởi công thức $v_A(t)=16-4t$ (đơn vị tính bằng m/s, thời gian $t$ tính bằng giây).
\choiceTF
{\True Thời điểm xe ô tô $A$ dừng lại là $4$ s}
{Quãng đường $S(t)$ (đơn vị: mét) mà ô tô $A$ đi được trong thời gian $t$ giây $(0\le t\le 4)$ kể từ khi hãm phanh được tính theo công thức $S(t)=\displaystyle\int\limits_{0}^{4} v(t)\mathrm{\,d}t$}
{\True Từ khi bắt đầu hãm phanh đến khi dừng lại xe ô tô $A$ đi được quãng đường $32$ m}
{\True Khoảng cách an toàn tối thiểu giữa xe ô tô $A$ và ô tô $B$ là $37$ m}
\loigiai{
\begin{itemchoice}
\itemch Khi xe ô tô $A$ dừng lại, tức là $v_A(t)=16-4t=0\Leftrightarrow t=4$ giây.
\itemch Quãng đường $S(t)$ (đơn vị: mét) mà ô tô $A$ đi được trong thời gian $t$ giây $(0\le t\le 4)$ kể từ khi hãm phanh được tính theo công thức $S(t)=\displaystyle\int\limits_{0}^{4} v_A(t)\mathrm{\,d}t$
\itemch Từ khi bắt đầu hãm phanh đến khi dừng lại xe ô tô $A$ đi được quãng đường
\[S(t)=\displaystyle\int\limits_{0}^{4} (16-4t)\mathrm{\,d}t=32~(\mathrm{m}).\]
\itemch Khoảng cách tối thiểu giữa hai xe ô tô $A$ và $B$ là $32+5=37$\,(m).
\end{itemchoice}
}
\end{ex}

\begin{ex}%[2D4H2-6]
Nam đang tham gia một bài học từ mới tiếng Anh trong vòng $60$ phút. Biết rằng $M(t)$ là số từ mới mà Nam có thể ghi nhớ trong $t$ phút. Tốc độ ghi nhớ từ mới của Nam được xác định bởi hàm số $M'(t) = at - bt^2$ (với $a, b \in \mathbb{R}$) (từ/phút) và đạt cao nhất tại thời điểm $40$ phút. Biết rằng Nam ghi nhớ được $18$ từ mới trong $10$ phút đầu tiên của bài học.
\choiceTF
{$a = 0{,}4$}
{Khả năng ghi nhớ của Nam tại thời điểm $20$ phút là $6$ từ/phút}
{Trong cả bài học Nam ghi nhớ được tổng cộng $427$ từ mới}
{Biết rằng tốc độ học trung bình (từ/phút) tại thời điểm $n$ đến thời điểm $m$ được tính bởi công thức
$\dfrac{1}{m-n}\displaystyle\int_n^m M'(t)\mathrm{\,d}t$.
Tốc độ học trung bình của Nam trong cả bài học là $6$ từ/phút}
\loigiai{
\begin{itemchoice}
\itemch Tốc độ ghi nhớ từ mới của Nam đạt cao nhất tại thời điểm $40$ phút nên $\dfrac{a}{2b} = 40$ hay $a = 80b$.\\
Số từ mới mà Nam có thể ghi nhớ trong $t$ phút là
\[M(t) = \displaystyle\int M'(t) \mathrm{\,d}t = \displaystyle\int (at - bt^2) \mathrm{\,d}t = \dfrac{a}{2}t^2 - \dfrac{b}{3}t^3 + C.\]
Mặt khác, $M(0) = 0 \Rightarrow C = 0 \Rightarrow M(t) = \dfrac{a}{2}t^2 - \dfrac{b}{3}t^3$.

Nam có thể ghi nhớ được $18$ từ mới trong $10$ phút đầu tiên của bài học nên $M(10) = 18$ hay
\[\dfrac{a}{2} \cdot 10^2 - \dfrac{b}{3} \cdot 10^3 = 18 \Rightarrow \dfrac{80b}{2} \cdot 10^2 - \dfrac{b}{3} \cdot 10^3 = 18 \Rightarrow \dfrac{11000}{3}b = 18 \Rightarrow b = \dfrac{27}{5500}.\]

Suy ra $a = \dfrac{108}{275} \approx 0,39$.\\
Vậy $M(t) = \dfrac{54}{275}t^2 - \dfrac{9}{5500}t^3$, $M'(t) = \dfrac{108}{275}t - \dfrac{27}{5500}t^2$.
\itemch Khả năng ghi nhớ của Nam tại thời điểm 20 phút là
\[M'(20) = \dfrac{108}{275} \cdot 20 - \dfrac{27}{5500} \cdot 20^2 \approx 5{,}9 \text{ (từ/phút)}.\]
\itemch Trong cả tiết học Nam ghi nhớ được tổng cộng số từ mới là
\[\displaystyle\int_0^{60} \left(\dfrac{108}{275}t - \dfrac{9}{5500}t^2\right) \mathrm{\,d}t \approx 353 \text{ (từ)}.\]
\itemch Tốc độ học trung bình của Nam trong cả tiết học là
\[\dfrac{1}{60-0} \displaystyle\int_0^{60} M'(t) \mathrm{\,d}t = \dfrac{1}{60} \displaystyle\int_0^{60} \left(\dfrac{54}{275}t - \dfrac{9}{5500}t^2\right) \mathrm{\,d}t = \dfrac{324}{55} \approx 5{,}9 \text{ (từ/phút)}.\]

\end{itemchoice}
}
\end{ex}

\begin{ex}%[2D4H2-6]%[TEX ĐỀ MOON 2025]%[Nguyễn Thế Duy]
Một ô tô đang chạy đều với vận tốc $x$ (m/s) thì người lái xe đạp phanh. Từ thời điểm đó, ô tô chuyển động chậm dần đều với vận tốc thay đổi theo hàm số $v=-5t+20$ (m/s), trong đó $t$ là thời gian tính bằng giây kể từ lúc đạp phanh. Xét tính đúng sai của các mệnh đề sau
\choiceTF
{\True Khi xe dừng hẳn thì vận tốc bằng $0$ (m/s)}
{Thời gian từ lúc người lái xe đạp phanh cho đến khi xe dừng hẳn là $5$ giây}
{\True $\displaystyle\int \left(-5t+20\right) \mathrm{\,d}t=-\dfrac{5t^2}{2}+20t+C$}
{Quãng đường từ lúc đạp phanh cho đến khi xe dừng hẳn là $400$ m}
\loigiai{
\begin{itemchoice}
\itemch \textbf{Đúng}.\\
Khi xe dừng hẳn thì vận tốc bằng $0$ (m/s).
\itemch \textbf{Sai}.\\
Xét $v(t) = 0 \Leftrightarrow -5t + 20 = 0 \Leftrightarrow t = 4$.\\
Vậy xe dừng hẳng sau $4$ giây đạp phanh.
\itemch \textbf{Đúng}.\\
Ta có $\displaystyle\int \left(-5t+20\right) \mathrm{\,d}t=-\dfrac{5t^2}{2}+20t+C$.
\itemch \textbf{Sai}.\\
Quãng đường từ lúc đạp phanh đến khi xe dừng hẳng là $s = \displaystyle\int_0^4 \left(-5t+20\right) \mathrm{\,d}t = 40$ m.
\end{itemchoice}
}
\end{ex}

\begin{ex}%[2D4H1-6]
Ở nhiệt độ $37^{\circ}$ C, một phản ứng hóa học từ chất đầu A, chuyển hóa thành chất B theo phương trình A $\longrightarrow$ B. Giả sử $y(x)$ là nồng độ chất $A$ (đơn vị mol $L^{-1}$) tại thời điểm $x$ (giây), $y(x)>0$ với $x \geq 0$, thỏa mãn hệ thức $y'(x)=-7 \cdot 10^{-4} y(x)$ với $x \geq 0$. Biết rằng tại $x=0$, nồng độ (đầu) của A là $0{,}05$ mol $L^{-1}$. Xét hàm số $f(x)=\ln y(x)$ với $x \geq 0$. Các phát biểu sau đây đúng hay sai?
\choiceTF
{\True $f'(x)=-7 \cdot 10^{-4}$}
{\True $f(x)=-7 \cdot 10^{-4} x+\ln (0{,}05)$}
{$y(30)-y(15)=-6\cdot 10^{-4}$}
{\True Nồng độ trung bình của chất A từ thời điểm $15$ giây đến thời điểm $30$ giây gần bằng $0{,}05$}
\loigiai{
\begin{itemchoice}
\itemch {\bf Đúng.}\\
Ta có $f'(x)=\left[\ln y(x)\right]'=\dfrac{y'(x)}{y(x)}=-7\cdot 10^{-4}$.
\itemch {\bf Đúng.}\\
Ta có $f(x)=\displaystyle\int f'(x)\mathrm{\,d}x=\displaystyle\int -7\cdot 10^{-4}\mathrm{\,d}x =-7\cdot 10^{-4}x+C$.\\
Theo giả thiết, $y(0)=0{,}05$ nên $f(0)=\ln y(0)=\ln 0{,}05$.\\
Vì vậy $C=\ln 0{,}05$, suy ra $f(x)=-7\cdot 10^{-4}x+\ln0{,}05$.
\itemch {\bf Sai.}\\
Từ $f(x)=\ln y(x)$, suy ra $y(x)=\mathrm{e}^{f(x)}=\mathrm{e}^{-7\cdot 10^{-4}x+\ln0{,}05}=\dfrac{1}{20}\cdot \mathrm{e}^{-7\cdot 10^{-4}x}$.\\
Do đó $y(30)-y(15)=\dfrac{1}{20}\left(\mathrm{e}^{-7\cdot 10^{-4}\cdot 30}-\mathrm{e}^{-7\cdot 10^{-4}\cdot15}\right)\approx-5{,}2\cdot 10^{-4}$.
\itemch {\bf Đúng.}\\
Nồng độ trung bình của chất A từ thời điểm $15$ giây đến thời điểm $30$ giây là
\[\dfrac{1}{30-15}\displaystyle\int\limits_{15}^{30}y(x)\mathrm{\, d}x=\dfrac{1}{15}\displaystyle\int\limits_{15}^{30}\left(-\dfrac{1}{7\cdot 10^{-4}}\right)y'(x)\mathrm{\, d}x=-\dfrac{10^4}{105}\cdot y(x)\Bigg|_{15}^{30}\approx 0{,}05.\]
\end{itemchoice}
}
\end{ex}

\begin{ex}%[2D3H2-3]%[TEX ĐỀ MOON 2025]%[Lê Hữu Kiệt]
Kết quả khảo sát năng suất (đơn vị: tấn/ha) của một thửa ruộng được minh họa ở biểu đồ sau
\begin{center}
\begin{tikzpicture}[line join=round, line cap=round,>=stealth,font=\footnotesize,scale=1,declare function={xmax=9;}, y=0.8cm]
\draw[<->] (0,7)node[left]{Số thửa ruộng}--(0,0)node[below left]{$O$}--(xmax,0)node[below]{Năng suất (tấn/ha)};
\foreach \y in {1,...,6}{
\draw (0,\y) node[left]{$\y$};
\draw[gray, thin] (0,\y)--(xmax,\y);
}
\foreach \sothuaruong [count=\i from 1] in {3,4,6,5,5,2}{
\draw[fill=blue!30] (\i,0) rectangle (\i+1,\sothuaruong);
}
\foreach \nhom [count=\i from 1] in {{$[5{,}5;5{,}7)$}, {$[5{,}7;5{,}9)$}, {$[5{,}9;6{,}1)$}, {$[6{,}1;6{,}3)$}, {$[6{,}3;6{,}5)$}, {$[6{,}5;6{,}7)$}}{
\draw (\i+0.1,0)node[below=0.4cm, rotate=45]{\nhom};
}
\node at (current bounding box.north) {\textbf{Năng suất lúa của một số thửa ruộng}};
\end{tikzpicture}
\end{center}
\choiceTF
{\True Có $25$ thửa ruộng đã được khảo sát}
{\True Khoảng biến thiên của mẫu số liệu trên là $1{,}2$ (tấn/ha)}
{Khoảng tứ phân vị của mẫu số liệu ghép nhóm trên là $0{,}4675$}
{\True Phương sai của mẫu số liệu ghép nhóm trên xấp xỉ bằng $0{,}086656$}
\loigiai{
Bảng phân bố tần số ghép nhóm
\begin{center}
\begin{tabular}{|c|c|c|c|c|c|c|}
\hline
Khoảng năng suất (tấn/ha) & $[5{,}5;5{,}7)$ & $[5{,}7;5{,}9)$ & $[5{,}9;6{,}1)$ & $[6{,}1;6{,}3)$ & $[6{,}3;6{,}5)$ &$ [6{,}5;6{,}7)$ \\
\hline
Giá trị đại diện & $5{,}6$ & $5{,}8$ & $6{,}0$ & $6{,}2$ & $6{,}4$ & $6{,}6$ \\ \hline
Số thửa ruộng & $3$ & $4$ & $6$ & $5$ & $5$ & $2$ \\
\hline
Tần số tích lũy & $3$ & $7$ & $13$ & $18$ & $23$ & $25$ \\ \hline
\end{tabular}
\end{center}
\begin{itemchoice}
\itemch Mẫu số liệu có $n=25$. Vậy có $25$ thửa ruộng đã được khảo sát.
\itemch Ta có $x_{\min}=5{,}5$ và $x_{\max}=6{,}7$. Khoảng biến thiên là $R=x_{\max}-x_{\min}=1{,}2$.
\itemch Ta có $\dfrac{n}{4}=6{,}25$, suy ra nhóm chứa $Q_1$ là $[5{,}7;5{,}9)$. Khi đó $Q_1=5{,}7+\dfrac{6{,}25-3}{4}\cdot0{,}2=5{,}8625$.\\
Ta có $\dfrac{3n}{4}=18{,}75$, suy ra nhóm chứa $Q_3$ là $[6{,}3;6{,}5)$. Khi đó $Q_3=6{,}3+\dfrac{18{,}75-18}{5}\cdot0{,}2=6{,}31$.\\
Khoảng tứ phân vị $\Delta_Q=Q_3-Q_1=0{,}4475$.
\itemch Giá trị trung bình cộng
\[\overline{x}=\dfrac{5{,}6\cdot 3+5{,}8\cdot 4+6{,}0\cdot 6+6{,}2\cdot 5+6{,}4\cdot 5+6{,}6\cdot 2}{25}=6{,}088.\]
Phương sai của mẫu số liệu
\[s^2=\dfrac{(5{,}6-\overline{x})^2+(5{,}8-\overline{x})^2+(6{,}0-\overline{x})^2+(6{,}2-\overline{x})^2+(6{,}4-\overline{x})^2+(6{,}6-\overline{x})^2}{25}=0{,}086656.\]
\end{itemchoice}
}
\end{ex}

\begin{ex}%[2D1H5-4]
Cho hàm số $y = x^3 - 3x + 1$. Xét tính đúng sai của các mệnh đề sau:
\choiceTF
{\True Hàm số đã cho đồng biến trên $(1; +\infty)$}
{Hàm số đã cho có giá trị cực tiểu bằng $3$}
{\True Đồ thị hàm số đã cho cắt trục tung tại điểm có tung độ bằng $1$}
{\True Giá trị lớn nhất của hàm số đã cho trên $[-2; 1]$ bằng $3$}
\loigiai{
Tập xác định $\mathscr{D} = \mathbb{R}$.\\
Ta có $y' = 3x^2 - 3$.\\
Cho $y' = 0 \Leftrightarrow 3x^2 - 3 = 0 \Leftrightarrow x^2 = 1 \Leftrightarrow \hoac{& x = 1 \\ & x = -1.}$\\
Bảng biến thiên:
\begin{center}
\begin{tikzpicture}
\tkzTabInit[nocadre=false, lgt=1.2, espcl=3, deltacl=0.5]
{$x$/0.7, $y'$/0.7, $y$/2}
{$-\infty$, $-1$, $1$, $+\infty$}
\tkzTabLine{+,0,-,0,+}
\tkzTabVar{-/$-\infty$, +/$3$, -/$-1$, +/$+\infty$}
\end{tikzpicture}
\end{center}
\begin{itemchoice}
\itemch
Dựa vào bảng biến thiên, hàm số đồng biến trên các khoảng $(-\infty; -1)$ và $(1; +\infty)$.\\
Vậy hàm số đồng biến trên $(1; +\infty)$.

\itemch
Dựa vào bảng biến thiên, hàm số đạt cực tiểu tại $x_{\text{CT}} = 1$ và giá trị cực tiểu $y_{\text{CT}} = y(1) = -1$.

\itemch
Đồ thị hàm số cắt trục tung tại điểm có hoành độ $x = 0$.\\
Khi $x = 0$, ta có $y = 0^3 - 3 \cdot 0 + 1 = 1$.\\
Vậy đồ thị cắt trục tung tại điểm $(0; 1)$, có tung độ bằng $1$.

\itemch
Xét hàm số $y = x^3 - 3x + 1$ trên đoạn $[-2; 1]$.\\
Ta có $y(-2) = (-2)^3 - 3(-2) + 1 = -8 + 6 + 1 = -1$.\\
$y(-1) = (-1)^3 - 3(-1) + 1 = -1 + 3 + 1 = 3$.\\
$y(1) = 1^3 - 3(1) + 1 = 1 - 3 + 1 = -1$.\\
Vậy $\max\limits_{x \in [-2; 1]} y = 3$, đạt được tại $x = -1$.
\end{itemchoice}
}
\end{ex}

\begin{ex}%[2D1H3-6]%[TEX ĐỀ MOON 2025]%[Nguyễn Thế Duy]
Một người muốn xây một cái bể chứa nước, dạng một khối hộp chữ nhật không nắp. Xét tính đúng sai của các mệnh đề sau
\choiceTF
{Nếu đáy bể là hình vuông cạnh bằng $50$ m, lượng nước trong bể cao $1{,}5$ m thì thể tích nước trong bể là $1250$ m$^3$}
{\True Nếu thể tích bể bằng $\dfrac{256}{3}$ m$^3$, đáy bể là hình chữ nhật có chiều dài gấp đôi chiều rộng. Gọi chiều rộng của bể là $x$ (m) thì biểu thức xác định chiều cao bể theo $x$ là $h=\dfrac{128}{3x^2}$ (m)}
{\True Nếu thể tích bể bằng $\dfrac{256}{3}$ m$^3$, đáy bể là hình chữ nhật có chiều dài gấp đôi chiều rộng. Gọi chiều rộng bể là $x$ (m) thì công thức xác định diện tích xung quanh của bể là $S=\dfrac{256}{x}$ (m$^2$)}
{Nếu thể tích bể bằng $\dfrac{256}{3}$ m$^3$, đáy bể là hình chữ nhật có chiều dài gấp đôi chiều rộng. Giá thuê nhân công để xây thành bể là $500.000$ đồng/m$^2$, đổ bê tông đáy bể là $250.000$ đồng/m$^2$. Chi phí thấp nhất để thuê nhân công xây dựng bể đó là $24.100.000$ (kết quả làm tròn đến hàng trăm nghìn)}
\loigiai{
\begin{itemchoice}
\itemch \textbf{Sai}.\\
Thể tích của bể nước có đáy là hình vuông cạnh bằng $50$ m, lượng nước trong bể cao $1{,}5$ m là $V_1 = 50^2 \cdot 1{,}5 = 3750$ m$^3$.
\itemch \textbf{Đúng}.\\
Ta có $\dfrac{256}{3} = 2x \cdot x \cdot h \Rightarrow h = \dfrac{128}{3x^2}$ m.
\itemch \textbf{Đúng}.\\
Diện tích xung quanh của bể là $S = 2\left(2x+x \right) \cdot \dfrac{128}{3x^2} = \dfrac{256}{x}$ m$^2$.
\itemch \textbf{Sai}.\\
Số tiền thuê nhân công là
$T(x) = 2x^2 \cdot 2{,}5 + \dfrac{256}{x} \cdot 5 = 5x^2 + \dfrac{1280}{x}$ (trăm nghìn đồng).\\
Ta có $T'(x) = 10x - \dfrac{1280}{x^2}$.\\
Xét $T'(x) = 0 \Leftrightarrow 10x - \dfrac{1280}{x^2} = 0 \Leftrightarrow x = \sqrt[3]{1280}$.\\
Ta có bảng biến thiên
\begin{center}
\begin{tikzpicture}
\tkzTabInit[nocadre=false,lgt=1.2,espcl=2.7,deltacl=0.7]
{$x$ /0.6, $T'(x)$ /0.6, $T(x)$ /2}
{$0$,$\sqrt[3]{1280}$,$+\infty$}
\tkzTabLine{,-,$0$,+,}
\tkzTabVar{+/$0$,-/$T\left(\sqrt[3]{1280} \right)$,+/$+\infty$}
\end{tikzpicture}
\end{center}
Vậy số tiền tối thiểu cần để thuê nhân công là \\
$T = T\left(\sqrt[3]{1280} \right) \approx 707$ (trăm nghìn đồng) $\approx 70\,700\,000$ (đồng).
\end{itemchoice}
}
\end{ex}

\begin{ex}%[2D1H3-1]%[TEX ĐỀ MOON 2025]%[Nguyễn Văn Hiệp]
Cho hàm số $f(x)=-2\sin x-x$. Xét tính đúng sai của các mệnh đề sau
\choiceTF[0.3em]
{\True $f(0)=0$ và $f(\pi)=-\pi$}
{Đạo hàm của hàm số đã cho là $f'(x)=2\cos x-1$}
{\True Nghiệm của phương trình $f'(x)=0$ trên đoạn $[0;\pi]$ là $\dfrac{2\pi}{3}$}
{\True Giá trị nhỏ nhất của hàm số đã cho trên đoạn $[0;\pi]$ là $-\dfrac{2\pi}{3}-\sqrt{3}$}
\loigiai
{
\begin{itemchoice}
\itemch  $f(0) = -2\sin 0 - 0 = 0$ và $f(\pi) = -2\sin \pi - \pi = -\pi$.
\itemch  $f'(x) = -2\cos x - 1$.
\itemch Giải $-2\cos x - 1 = 0 \Rightarrow x = \dfrac{2\pi}{3} \in [0; \pi]$.
\itemch
$f(0) = -2\sin 0 - 0 = 0$ và $f(\pi) = -2\sin \pi - \pi = -\pi$, $f\left(\dfrac{2\pi}{3}\right) = -2\sin\left(\dfrac{2\pi}{3}\right) - \dfrac{2\pi}{3} = -\sqrt{3} - \dfrac{2\pi}{3}$.\\
Suy ra $\min\limits_{[0;\pi]}=f\left(\dfrac{2\pi}{3}\right)= -\sqrt{3} - \dfrac{2\pi}{3}$.
\end{itemchoice}
}
\end{ex}

\begin{ex}%[2D1H3-1]%[TEX ĐỀ MOON 2025]%[Nguyễn Cường]
Cho hàm số $f(x)=2\sin x-x$.
\choiceTF
{\True $f(0)=0$ và $f\left(\dfrac{\pi}{2}\right)=2-\dfrac{\pi}{2}$}
{Đạo hàm của hàm số đã cho là $f'(x)=-2\cos x-1$}
{\True Nghiệm của phương trình $f'(x)=0$ trên đoạn $\left[0;\dfrac{\pi}{2}\right]$ là $\dfrac{\pi}{3}$}
{\True Giá trị lớn nhất của $f(x)$ trên đoạn $\left[0;\dfrac{\pi}{2}\right]$ là $\sqrt{3}-\dfrac{\pi}{3}$}
\loigiai{
\begin{itemchoice}
\itemch Ta có $f(0)=2\sin 0-0=0$ và $f\left(\dfrac{\pi}{2}\right)=2\sin\dfrac{\pi}{2}-\dfrac{\pi}{2}=2-\dfrac{\pi}{2}$.
\itemch Đạo hàm $f'(x)=2\cos x-1$.
\itemch Ta có
\allowdisplaybreaks
\begin{eqnarray*}
f'(x)=0&\Leftrightarrow& 2\cos x-1=0\\
&\Leftrightarrow& \cos x=\dfrac{1}{2}\\
&\Leftrightarrow& x=\dfrac{\pi}{3}\in \left[0;\dfrac{\pi}{2}\right]
\end{eqnarray*}
\itemch Hàm số liên tục trên đoạn $\left[0;\dfrac{\pi}{2}\right]$.\\
Ta có $f(0)=0$, $f\left(\dfrac{\pi}{2}\right)=2-\dfrac{\pi}{2}$ và $f\left(\dfrac{\pi}{3}\right)=\sqrt{3}-\dfrac{\pi}{3}$.\\
Vậy $\max\limits_{\left[0;\dfrac{\pi}{2}\right]}f(x)=\sqrt{3}-\dfrac{\pi}{3}$.
\end{itemchoice}
}
\end{ex}

\begin{ex}%[2D1H3-1]%[TEX ĐỀ MOON 2025]%[Huỳnh Thanh Chí]
Cho hàm số $f(x)=\sin 2x-x$. Xét tính đúng sai của các mệnh đề sau
\choiceTF[0.4em]
{\True $f\left(-\dfrac{\pi}{2}\right)=\dfrac{\pi}{2}$ và $f\left(\dfrac{\pi}{2}\right)=-\dfrac{\pi}{2}$}
{Đạo hàm của hàm số đã cho là $f'(x)=\cos 2x-1$}
{\True Nghiệm của phương trình $f'(x)=0$ trên đoạn $\left[-\dfrac{\pi}{2};\dfrac{\pi}{2}\right]$ là $x=\pm\dfrac{\pi}{6}$}
{\True Giá trị nhỏ nhất của hàm số đã cho trên đoạn $\left[-\dfrac{\pi}{2};\dfrac{\pi}{2}\right]$ là $-\dfrac{\pi}{2}$}
\loigiai{
\begin{itemchoice}
\itemch Ta có $f\left(-\dfrac{\pi}{2}\right)=\dfrac{\pi}{2}$ và $f\left(\dfrac{\pi}{2}\right)=-\dfrac{\pi}{2}$.
\itemch Ta có $f'(x)=2\cos 2x-1$.
\itemch Ta có \allowdisplaybreaks
\begin{eqnarray*}
f'(x)=0\Leftrightarrow 2\cos 2x-1\Leftrightarrow \cos 2x=\dfrac{1}{2}\Leftrightarrow \hoac{& 2x=\dfrac{\pi}{3}+k2\pi\\ & 2x=-\dfrac{\pi}{3}+k2\pi}
\Leftrightarrow \hoac{& x=\dfrac{\pi}{6}+k\pi\\ & x=-\dfrac{\pi}{6}+k\pi}, k\in\mathbb{Z}.
\end{eqnarray*}
Vì $x\in \left[-\dfrac{\pi}{2};\dfrac{\pi}{2}\right]$ nên tập nghiệm của phương trình là $S=\left\{-\dfrac{\pi}{6};\dfrac{\pi}{6}\right\}$.
\itemch Ta có $f\left(-\dfrac{\pi}{2}\right)=\dfrac{\pi}{2}$, $f\left(\dfrac{\pi}{2}\right)=-\dfrac{\pi}{2}$, $f\left(\dfrac{\pi}{6}\right)=\dfrac{\sqrt{3}}{2}-\dfrac{\pi}{6}$ và $f\left(-\dfrac{\pi}{6}\right)=-\dfrac{\sqrt{3}}{2}+\dfrac{\pi}{6}$.\\
Vậy $\min\limits_{\mathscr{D}} f(x)=f\left(\dfrac{\pi}{2}\right)=-\dfrac{\pi}{2}$, với $\mathscr{D}=\left[-\dfrac{\pi}{2};\dfrac{\pi}{2}\right]$.
\end{itemchoice}
}
\end{ex}

\begin{ex}%[2D1H2-1]
Cho hàm số $f(x)=\dfrac{2x-3}{x^2+4}$.
\choiceTF
{\True Hàm số đã cho xác định với mọi $x\in\mathbb{R}$}
{\True Nghiệm của phương trình $f'(x)=0$ là $x=-1$; $x=4$}
{\True Giá trị cực đại của hàm số $f(x)$ là $\dfrac{1}{4}$}
{Tập giá trị của hàm số $f(x)$ là đoạn $[a;b]$ thì $3a+2b=-2$}
\loigiai{
\begin{itemchoice}
\itemch Ta có $x^2+4>0$ với mọi giá trị $x$.\\
Hàm số đã cho xác định với mọi $x\in\mathbb{R}$.
\itemch Ta có $f'(x)=\dfrac{-2x^2+6x+8}{(x^2+4)^2}$.\\
Xét $f'(x)=0\Leftrightarrow -2x^2+6x+8=0\Leftrightarrow\hoac{&x=-1\\&x=4.}$
\itemch Bảng biến thiên của $f(x)$
\begin{center}
\begin{tikzpicture}
\tkzTabInit[nocadre=false,lgt=1.5,espcl=2.5,deltacl=0.6]
{$x$/0.7,$f'(x)$/0.7,$f(x)$/1.5}{$-\infty$,$-1$,$4$,$+\infty$}
\tkzTabLine{,-,0,+,0,-,}
\tkzTabVar{+/$0$,-/$-1$,+/$\dfrac{1}{4}$,-/$0$}
\end{tikzpicture}
\end{center}
\itemch Dựa vào bảng biến thiên ta có tập giá trị của hàm số là $\left[-1;\dfrac{1}{4} \right]$.\\
Vậy $3a+2b=3\cdot(-1)+2\cdot\dfrac{1}{4}=\dfrac{5}{2}$.

\end{itemchoice}
}
\end{ex}

\begin{ex}%[2D1H1-2]%[TEX ĐỀ MOON 2025]%[Nguyễn Thế Duy]
Cho hàm số $y=f(x)$ có bảng biến thiên như sau
\begin{center}
\begin{tikzpicture}
\tkzTabInit[nocadre=false,lgt=1.2,espcl=2.5,deltacl=0.6]
{$x$ /0.6, $y'$ /0.6, $y$ /2}
{$-\infty$,$-1$,$1$,$+\infty$}
\tkzTabLine{,+,$0$,-,$0$,+,}
\tkzTabVar{-/$-\infty$,+/$3$,-/$-1$,+/$+\infty$}
\end{tikzpicture}
\end{center}
Xét tính đúng sai của các mệnh đề sau
\choiceTF
{\True $y'=0\Leftrightarrow\hoac{& x=-1 \\ & x=1}$}
{Điểm cực tiểu của hàm số là $x=-1$}
{\True Đồ thị hàm số $y=f(x)$ cắt trục $Ox$ tại ba điểm phân biệt}
{\True Hàm số $y=f(2-x)$ đồng biến trên khoảng $(1;3)$}
\loigiai{
\begin{itemchoice}
\itemch \textbf{Đúng}.\\
Từ bảng biến thiên ta có $y'=0\Leftrightarrow\hoac{&x=-1 \\ &x=1.}$
\itemch \textbf{Sai}.\\
Điểm cực tiểu của hàm số là $\left(1; -1 \right)$.
\itemch \textbf{Đúng}.\\
Từ bảng biến thiên ta có $f(x) = 0$ có ba nghiệm phân biệt.
\itemch \textbf{Đúng}.\\
Xét hàm số $y = f(2-x)$ ta có $y' = -f(2-x)$.\\
Xét $y' = 0 \Leftrightarrow \hoac{&2-x = -1\\&2-x=1} \Leftrightarrow \hoac{&x = 3\\&x = 1.}$\\
Ta có bảng xét dấu
\begin{center}
\begin{tikzpicture}
\tkzTabInit[nocadre=false,lgt=1.2,espcl=2.5,deltacl=0.6]
{$x$ /0.6, $y'$ /0.6}
{$-\infty$,$1$,$3$,$+\infty$}
\tkzTabLine{,-,$0$,+,$0$,-,}
\end{tikzpicture}
\end{center}
Vậy hàm số $y = f(2-x)$ đồng biến trên khoảng $\left(1; 3\right)$.
\end{itemchoice}
}
\end{ex}

\begin{ex}%[2D1H3-1]%[TEX Đề Moon 2025]%[Võ Nguyên Thạch]
Cho hàm số $f(x)=x-\sin 2x$.
\choiceTF
{\True $f(0)=0$ và $f(\pi)=\pi$}
{Đạo hàm của hàm số đã cho là $f'(x)=1+2\cos 2x$}
{\True Nghiệm của phương trình $f'(x)=0$ trên đoạn $[0;\pi]$ là $\dfrac{\pi}{6}$ và $\dfrac{5\pi}{6}$}
{\True Giá trị nhỏ nhất của hàm số đã cho trên đoạn $[0;\pi]$ là $\dfrac{\pi}{6}-\dfrac{\sqrt{3}}{2}$}
\loigiai{
\begin{itemchoice}
\itemch Ta có $f(0)=0-\sin(2\cdot 0)=0-0=0$; $f(\pi)=\pi-\sin(2\cdot \pi)=\pi-0=\pi$.
\itemch Ta có $f'(x)=(x-\sin 2x)'=1-(2x)'\cos 2x=1-2\cos 2x$.
\itemch Ta có
\allowdisplaybreaks
\begin{eqnarray*}
&&f'(x)=0\\
&\Leftrightarrow&1-2\cos 2x=0\\
&\Leftrightarrow&\cos 2x=\dfrac{1}{2}\\
&\Leftrightarrow&\hoac{&2x=\dfrac{\pi}{3}+k2\pi\\&2x=-\dfrac{\pi}{3}+k2\pi},\, k\in \mathbb{Z}\\
&\Leftrightarrow&\hoac{&x=\dfrac{\pi}{6}+k\pi\\&x=-\dfrac{\pi}{6}+k\pi},\, k\in \mathbb{Z}.
\end{eqnarray*}
Do $x\in [0;\pi]$ nên $x=\dfrac{\pi}{6}$ và $x=\dfrac{5\pi}{6}$.
\itemch Ta có
\begin{itemize}
\item $f(0)=0-2\sin (2\cdot 0)=0$.
\item $f\left(\dfrac{\pi}{6}\right)=\dfrac{\pi}{6}+\sin \left(2\cdot \dfrac{\pi}{6}\right)=\dfrac{\pi}{6}+\dfrac{\sqrt 3}{2}$.
\item $f\left(\dfrac{5\pi}{6}\right)=\dfrac{5\pi}{6}+\sin \left(2\cdot \dfrac{5\pi}{6}\right)=\dfrac{5\pi}{6}-\dfrac{\sqrt 3}{2}$.
\item $f(\pi)=\pi+\sin(2\pi)=\pi$.
\end{itemize}
Vậy giá trị nhỏ nhất của hàm số đã cho trên đoạn $[0;\pi]$ là $\dfrac{\pi}{6}-\dfrac{\sqrt{3}}{2}$.
\end{itemchoice}
}
\end{ex}

% \paragraph{Mức độ V}
\begin{ex}%[2H5V3-4]%[TEX Đề Moon 2025]%[Vũ Hồng Toàn]
Trong không gian $Oxyz$, cho đường thẳng $\Delta\colon\heva{& x=3+t \\ & y=-1-t\\ & z=-2+t}$, điểm $M(1;2;-1)$ và mặt cầu $(S)\colon x^2+y^2+z^2-4x+10y+14z+64=0$. Xét tính đúng sai của các mệnh đề sau
\choiceTF
{\True Đường thẳng $\Delta $ có một vectơ chỉ phương là $\overrightarrow{u}=(1;-1;1)$}
{\True Mặt cầu $(S)$ có tâm $I(2;-5;-7)$ và bán kính $R=\sqrt{14}$}
{Mặt phẳng đi qua điểm $M$ và vuông góc với đường thẳng $\Delta$ là $x-y+z-2=0$}
{\True Gọi $\Delta'$ là đường thẳng đi qua $M$ cắt đường thẳng $\Delta$ tại $A$, cắt mặt cầu tại $B$ sao cho $\dfrac{AM}{AB}=\dfrac{1}{3}$ và điểm $B$ có hoành độ là số nguyên. Mặt phẳng trung trực của đoạn $AB$ có phương trình là $2x-4y-4z-43=0$}
\loigiai{
\begin{itemchoice}
\itemch Đường thẳng $\Delta $ có một vectơ chỉ phương là $\overrightarrow{u}=(1;-1;1)$.
\itemch Ta có $x^2+y^2+z^2-4x+10y+14z+64=0\Rightarrow (x-2)^2+(y+5)^2+(z+7)^2=14$.\\
Do đó $(S)$ có tâm $I(2;-5;-7)$ và bán kính $R=\sqrt{14}$.
\itemch Mặt phẳng đi qua điểm $M$ và vuông góc với đường thẳng $\Delta$ nên có một vectơ pháp tuyến là $\overrightarrow{n}=\overrightarrow{u}=(1;-1;1)$ nên có phương trình
\[(x-1)-(y-2)+(z+1)=0\Rightarrow x-y+z+2=0.\]
\itemch Do $A\in \Delta\Rightarrow A(3+t;-1-t;-2+t)$, $B(x;y;z)\in (S)$;\\ $\overrightarrow{MB}=(x-1;y-2;z+1)$, $\overrightarrow{MA}=(2+t;-3-t;-1+t)$.\\
Vì $\dfrac{AM}{AB}=\dfrac{1}{3}$. Xét hai trường hợp
\begin{itemize}
\item Trường hợp $M$ nằm giữa $A$ và $B$. Suy ra $\overrightarrow{MB}=-2\overrightarrow{AM}$.\\ Từ đó ta có hệ phương trình
\[\heva{&x-1=-2(2+t)\\&y-2=-2(-3-t)\\&z+1=-2(-1+t)}\Rightarrow \heva{&x=-3-2t\\&y=8+2t\\&z=1-2t.}\]
Mà $B\in(S)$ nên ta có
\allowdisplaybreaks
\begin{eqnarray*}
&&(-3 - 2t - 2)^2 + (8 + 2t + 5)^2 + (1 - 2t + 7)^2 = 14\\
&\Leftrightarrow& 3t^2+10t+61=0\\
&\Rightarrow& \text{phương trình vô nghiệm.}
\end{eqnarray*}
Do đó không thoả mãn.
\item  Trường hợp $M$ nằm ngài $A$ và $B$. Suy ra $\overrightarrow{MB}=4\overrightarrow{AM}$.\\
Từ đó ta có hệ phương trình
\[\heva{&x-1=4(2+t)\\&y-2=4(-3-t)\\&z+1=4(-1+t)}\Rightarrow \heva{&x=9+4t\\&y=-10-4t\\&z=-5+4t.}\]
Mà $B\in(S)$ nên ta có
\allowdisplaybreaks
\begin{eqnarray*}
&&(9 + 4t - 2)^2 + (-10 - 4t + 5)^2 + (-5 + 4t + 7)^2  = 14\\
&\Leftrightarrow& 3t^2+7t+4=0
\Leftrightarrow\hoac{&t=-1&\text{ (nhận)}\\&t=-\dfrac{4}{3}&\text{ (loại)}.}
\end{eqnarray*}
\end{itemize}
Khi đó $A(2;0;-3)$ và $B(5;-6;-9)$.\\
Gọi $C$ là trung điểm của đoạn $AB$ suy ra $C\left(\dfrac{7}{2};-3;-6\right)$. Mặt phẳng trung trực của đoạn $AB$ đi qua $C$ và có một vectơ pháp tuyến là $\dfrac{1}{3}\cdot\overrightarrow{AB}=(1;-2;-2)$ nên có phương trình
\[\left(x-\dfrac{7}{2}\right)-2(y+3)-2(z+6)=0\Rightarrow 2x-4y-4z-43=0.\]
\end{itemchoice}
}
\end{ex}

\begin{ex}%[2H5V3-4]
\immini{Vệ tinh hoạt động dựa trên nguyên lý của vật lý Newton - một vật thể bị kéo bởi một lực hấp dẫn từ một vật thể khác sẽ chuyển động theo một quỹ đạo elip xung quanh vật thể đó. Để đưa vệ tinh lên quỹ đạo, người ta sử dụng các loại tên lửa đẩy khác nhau để cung cấp cho vệ tinh động lượng cần thiết để thoát khỏi trọng lực của Trái Đất và duy trì quỹ đạo ổn định. Để thuận tiện ta quy ước một quỹ đạo gần tròn thành một đường tròn. }{\begin{tikzpicture}[line join = round, line cap=round,>=stealth,font=\footnotesize,transform shape,scale=0.8]
\draw
(0,0) circle(3.5cm)
;
\draw[fill=black!30]
(0,0) circle(1.6cm);
\draw[<->] (0:3.5)--(0:1.6) node[pos=0.5,above]{$h$ km};
\draw[<->] (0:0)--(0:1.6) node[pos=0.5,above]{$6\, 400$ km};
\fill
(3.5,0)circle(1.5pt)node[above right]{$B$};
\end{tikzpicture}}
Trong hệ tọa độ $Oxyz$, gốc tọa độ là tâm trái đất, một vệ tinh nhân tạo tạo quỹ đạo được coi như một đường tròn có bán kính $13440$ km có điểm xuất phát là điểm $B (4032;0;-5376)$ và đây cũng là điểm gần Trái Đất nhất của vệ tinh. Quỹ đạo của vệ tinh này nằm trên mặt phẳng vuông góc với trục tung và có tâm nằm trên đường thẳng $OB$. Coi Trái Đất là hình cầu hoàn hảo có bán kính bằng $6400$ km.
\choiceTF
{Phương trình mặt phẳng chứa quỹ đạo của vệ tinh là $x+z=0$}
{\True Khi xuất phát tại điểm $B$ vệ tinh đang ở độ cao $320$ km so với mặt đất}
{Quỹ đạo của tên lửa là đường tròn có tâm $I(-4032; 0;5120)$}
{Khi Trái Đất quay, điểm cực Nam và cực Bắc của Trái Đất không thay đổi vị trí. Biết rằng điểm cực Nam của Trái Đất có tọa độ là $(0; 3840;5120)$. Khoảng cách gần nhất giữa vệ tinh và điểm cực Nam bằng $10112$ km (làm tròn kết quả đến hàng đơn vị)}
\loigiai{
\begin{itemchoice}
\itemch Quỹ đạo của vệ tinh này nằm trên mặt phẳng vuông góc với trục tung nên một vectơ chỉ phương của
mặt phẳng chứa quỹ đạo của vệ tinh là $(0; 1; 0)$.\\
Khi đó, phương trình mặt phẳng chứa quỹ đạo của vệ tinh có dạng $y+a=0$.\\
Quỹ đạo đi qua $B(4032;0;-5376)$ nên $0+a=0$ hay $a=0$.\\
Vậy phương trình mặt phẳng chứa quỹ đạo của vệ tinh là $y=0$.
\itemch Khoảng cách ngắn nhất từ Trái Đất đến vệ tinh bằng
\[OB-R=\sqrt{4032^{2}+0^{2}+(-5376)^{2}}-6400=320(km).\]
Vậy khi xuất phát tại điểm $B$ vệ tình đang ở độ cao $320$ km so với mặt đất.
\itemch
Quỹ đạo của vệ tinh có tâm nằm trên đường thẳng $OB$ nên $I$ nằm trên đường thẳng $OB$.\\
Mặt khác $IB=R_{qd}=13440=2\cdot OB$ nên $O$ là trung điểm của $IB$.\\
Khi đó
\[\begin{cases}
x_I=2x_O-x_B \\
y_I=2y_O-y_B \\
z_I=2z_O-z_B
\end{cases}
\Leftrightarrow
\begin{cases}
x_I=-4032\\
y_I=0\\
z_I=5376\end{cases}
\Rightarrow I(-4032; 0; 5376).\]
\itemch
Gọi $H$ là hình chiếu của $K$ trên mặt phẳng chứa quỹ đạo $(\alpha)\colon y=0\Rightarrow H(0; 0; 5120)$.\\
Ta có
\begin{itemize}
\item $KH=\mathrm{d}(K, (\alpha))=3840$.
\item $IH=\sqrt{4032^2+(5376-5120)^2}=64\sqrt{3985}$.
\item $NH=IN-IH=13440-64\sqrt{3985}$.
\end{itemize}
Nối $I$ và $H$ cắt vệ tinh tại $N$. Khi đó:
\[KN_{\text{min}}=\sqrt{KH^2+NH^2}=\sqrt{3840^2+(13440-64\sqrt{3985})^2} \approx 10154 \text{ (km)}.\]

\end{itemchoice}
}
\end{ex}

\begin{ex}%[2H5V3-4]%[TEX Đề Moon 2025]%[Võ Nguyên Thạch]
Trong không gian $Oxyz$ (đơn vị trên mỗi trục tính theo mét), một ngọn hải đăng được đặt ở vị trí $I(17;20;45)$. Biết rằng ngọn hải đăng đó được thiết kế với bán kính phủ sáng là $4$ km.
\choiceTF
{\True Phương trình mặt cầu để mô tả ranh giới bên ngoài của vùng phủ sáng trên biển của hải đăng là $(x-17)^2+(y-20)^2+(z-45)^2=16\,000\,000$}
{Nếu người đi biển ở vị trí $M(18;21;50)$ thì không thể nhìn thấy được ánh sáng từ ngọn hải đăng}
{Nếu người đi biển ở vị trí $N(4\,019;21;44)$ thì có thể nhìn thấy được ánh sáng từ ngọn hải đăng}
{\True Nếu hai người đi biển ở vị trí có thể nhìn thấy được ánh sáng từ ngọn hải đăng thì khoảng cách giữa hai người đó không quá $8$ km}
\loigiai{
\begin{itemchoice}
\itemch Ta có $4$ km $=4\,000$ m.\\
Phương trình mặt cầu mô tả ranh giới bên ngoài vùng phủ sáng trên biển của hải đăng là phương trình mặt cầu tâm $I(17;20;45)$, bán kính $4\,000$ m là
\[(x-17)^2+(y-20)^2+(z-45)^2=16\,000\,000.\]
\itemch Ta có $IM=\sqrt{(18-17)^2+(21-20)^2+(50-45)^2}=\sqrt 27<4\,000$.\\
Khi đó, người ở vị trí điểm $M$ có thể nhìn thấy ánh sáng từ ngọn hải đăng.
\itemch Ta có $IN=\sqrt{(4\,019-17)^2+(21-20)^2+(44-45)^2}=\sqrt{16\,016\,006}>4\,000$.\\
Khi đó, người ở vị trí điểm $N$ không thể nhìn thấy ánh sáng từ ngọn hải đăng.
\itemch Vì đường kính của mặt cầu trên bằng $8\,000$ m hay $8$ km nên khoảng cách giữa hai người đi biển ở vị trí có thể nhìn thấy ánh sáng từ ngọn hải đăng không quá $8$ km.
\end{itemchoice}
}
\end{ex}

\begin{ex}%[2H5V3-3]
Trong không gian với tọa độ $Oxyz$, cho mặt cầu $(S)\colon  x^2+y^2+z^2-2x+4y+1=0$ và mặt phẳng $(P)\colon x+y-z-2=0$.
\choiceTF
{Mặt cầu $(S)$ có tâm $I(-1;2;0)$}
{Bán kính của mặt cầu $(S)$ là $R=4$}
{\True Khoảng cách từ tâm $I$ của mặt cầu $(S)$ đến mặt phẳng $(P)$ bằng $\sqrt{3}$}
{\True Mặt phẳng $(P)$ cắt mặt cầu $(S)$ theo giao tuyến là một đường tròn có bán kính $r=1$}
\loigiai{
\begin{itemchoice}
\itemch Mặt cầu $(S)\colon x^2+y^2+z^2-2x+4y+1=0$ có tâm $I(1;-2;0)$.
\itemch Bán kính mặt cầu $(S)\colon  x^2+y^2+z^2-2x+4y+1=0$ là $R=\sqrt{1^2+(-2)^2+0^2-1}=2$.
\itemch Khoảng cách từ tâm $I(1;-2;0)$ của mặt cầu $(S)$ đến mặt phẳng $(P)\colon x+y-z-2=0$ là \[\mathrm{d}(I,(P))=\dfrac{|1+(-2)-0-2|}{\sqrt{1^2+1^2+(-1)^2}}=\sqrt{3}.\]
\itemch Ta thấy $\mathrm{d}(I,(\mathrm{P}))<R$ nên $(P)$ cắt mặt cầu $(S)$ theo giao tuyến là một đường tròn với bán kính
\[R=\sqrt{R^2-\mathrm{d}(I, (P))^2}=\sqrt{2^2-\left(\sqrt{3}\right)^2}=1.\]
\end{itemchoice}
}
\end{ex}

\begin{ex}%[2H5V2-8]%[TEX ĐỀ MOON 2025]%[Huỳnh Thanh Chí]
Trong không gian với hệ tọa độ $Oxyz$, một cabin cáp treo xuất phát từ điểm $A(10;3;0)$ và chuyển động đều theo đường cáp có vectơ chỉ phương $\overrightarrow{u}=(2;-2;1)$ (hướng chuyển động cùng chiều với hướng vectơ $\overrightarrow{u}$ với tốc độ là $4{,}5$ (m/s)) (đơn vị trên mỗi trục là mét). Xét tính đúng sai của các mệnh đề sau
\choiceTF
{\True Phương trình tham số của đường cáp là $\heva{& x=10+2t' \\ & y=3-2t'\\ & z=t'}$ $(t\in\mathbb{R})$}
{\True Giả sử sau thời gian $t$ (s) kể từ khi xuất phát $(t\ge 0)$, cabin đến điểm $M$. Khi đó tọa độ điểm $M$ là $M\left(3t+10;-3t+3;\dfrac{3t}{2}\right)$}
{Cabin dừng ở điểm $B$ có hoành độ $x_B=550$, khi đó quãng đường $AB$ dài $800$ m}
{Đường cáp $AB$ tạo với mặt phẳng $(Oxy)$ một góc $30^\circ$}
\definecolor{ecru}{rgb}{0.76, 0.7, 0.5}
\definecolor{darkolivegreen}{rgb}{0.33, 0.42, 0.18}
\definecolor{deepskyblue}{rgb}{0.0, 0.75, 1.0}
\definecolor{antiquebrass}{rgb}{0.8, 0.58, 0.46}
\definecolor{arsenic}{rgb}{0.23, 0.27, 0.29}
\definecolor{ashgrey}{rgb}{0.7, 0.75, 0.71}\definecolor{alizarin}{rgb}{0.82, 0.1, 0.26}
\begin{center}
\begin{tikzpicture}[line join=round, line cap=round,scale=1,transform shape]
\clip (-3,-3) rectangle (3,3);
\fill[bottom color=white,top color=deepskyblue!90, middle color=white] (-3,-3) rectangle (3,3);
\tikzset{dat/.pic={
\def\N{
(-3,3)
..controls +(20:.7) and +(150:.7) ..(-.8,1.5)
..controls +(-30:.7) and +(150:.6) ..(1.5,-.8)
..controls +(-30:1) and +(90:.4) ..(3,-3)--(-3,-3)--cycle
;
}
\draw[black]\N;
\fill[darkolivegreen!50] \N;
}}
\tikzset{cap_treo/.pic={
\def\X{
(-.85,1)--(-.8,1)
..controls +(-90:.9) and +(-180:.6) ..(-.1,0.1)--(-.1,.05)
..controls +(-180:.65) and +(-90:.9) ..cycle
(-.15,0.05)--(-.15,-.1)--(-.23,-.22)--(-.6,-.22)--(-.5,-.1)--(-.5,.08)
;
}
%\fill[black] \X;
\draw\X;
\draw(-3,2.3)--(3.5,-1.2);
\draw(-1.2,1)--(-.5,1);
\draw(-.5,1)--(-.9,1.2)--(-.95,1.17)--(-.6,.97)--cycle;
\draw[fill=arsenic!80](-.9,-.05)--(.52,-.05)--(.28,-.3)--(-1.05,-.3)--cycle;
\def\M{
(-.85,1)--(-.8,1)
..controls +(-90:.9) and +(-180:.6) ..(-.1,0.1)--(-.1,.05)
..controls +(-180:.65) and +(-90:.9) ..cycle
(-.15,0.05)--(-.15,-.1)--(-.23,-.22)--(-.6,-.22)--(-.5,-.1)--(-.5,.08)
;
}
\fill[arsenic] \M;
\draw\M;
\def\N{
(-1.05,-.3)
..controls +(-120:.6) and +(120:.6) ..(-.98,-1.5)--(.4,-1.5)
..controls +(70:.6) and +(-60:.3) ..(.28,-.3)--cycle
;
}
\fill[arsenic] \N;
\draw\N;
\def\P{
(.4,-1.5)
..controls +(70:.6) and +(-60:.3) ..(.28,-.3)--(.52,-.05)
..controls +(-40:.4) and +(50:.4) ..(.6,-1.2)--cycle
;
}
\fill[arsenic!80] \P;
\draw\P;
\draw (-1.23,-1)--(.5,-1)--(.77,-.6);
\draw (-1.05,-.5)--(-1,-.4)--(.2,-.4)--(.25,-.5)--cycle
(.25,-.5)--(.3,-.6)--(.3,-.6)--(-1.1,-.6)--(.-1.05,-.5)--cycle
;
\draw[alizarin](.42,-1)..controls +(100:.4) and +(-65:.4) ..(.2,-.3);
\draw[alizarin](-1.15,-1)..controls +(100:.3) and +(-120:.3) ..(-.95,-.3);
}}
\path
(0,0)pic[scale=1]{dat}(0,0)pic[scale=1]{cap_treo};
\end{tikzpicture}
\begin{tikzpicture}[scale=1, font=\footnotesize, line join=round,xscale=.2, line cap=round,>=stealth]
\def\a{1/16}
\def\xmin{-3} \def\xmax{12}
\def\ymin{-3} \def\ymax{3}
\coordinate (O) at (0,0);
\coordinate (E) at (-10,-3);
\coordinate (N) at ($(E)!.7!(O)$);
\coordinate (P) at ($(E)!.2!(O)$);
\coordinate (D) at ($(E)!.3!(O)$);
\coordinate (A) at ($(N)+(3,0)$);
\coordinate (B) at (-14,1);
\coordinate (M) at ($(A)!.7!(B)$);
\draw[->] (\xmin,0)--(\xmax,0) node [above right]{$y$};
\draw[->] (O)--(0,\ymax) node [left]{$z$};
\draw[->] (O)--(E) node [below right]{$x$};
\node at (0,0)[above right]{$O$};
\draw[dashed] (B)--(P)node [right]{$550$} (N)--(A)node [below right]{$A(10;3;0)$}--(3,0);
\draw(B)node [above]{$B$}--(A);
\draw[->,red](O)--(-4,.6)node [above right]{$\overrightarrow{u}$};
\node at (N) [left]{$10$};
\node at (M) [above]{$M$};
\node at (P) [left]{$x_B$};
\draw[fill=black] (3,0) circle (.5pt) node[below]{\footnotesize $3$};
%	\path (-28,0) node[opacity=.5,scale=.5] {\includegraphics{image/h35}};
\clip (\xmin+0.1,\ymin+0.1) rectangle (\xmax-0.1,\ymax-0.1);
\end{tikzpicture}
\end{center}
\loigiai{
\begin{itemchoice}
\itemch Phương trình chính tắc của đường cáp là $\dfrac{x-10}{2}=\dfrac{y-3}{-2}=\dfrac{z}{1}$.\\
Phương trình tham số của đường cáp là $\heva{& x=10+2t' \\ & y=3-2t'\\ & z=t'}$ $(t'\in\mathbb{R})$.
\itemch Do tốc độ chuyển động của cabin là $4{,}5$ m/s nên độ dài $A M$ bằng $4{,}5 t$ m. \\
Vì vậy $\left|\overrightarrow{AM}\right|=4{,}5 t$ $(t \geq 0)$.\\
Do hai vectơ $\overrightarrow{A M}$ và $\overrightarrow{u}$ là cùng phương và cùng hướng nên $\overrightarrow{A M}=k \overrightarrow{u}$ với $k$ là số thực dương nào đó. \\
Suy ra $\left|\overrightarrow{A M}\right|=k|\overrightarrow{u}|=k \cdot \sqrt{2^2+(-2)^2+1}=3 k$. Do đó $3 k=4{,}5 t$. Suy ra $k=\dfrac{3 t}{2}$. \\
Vì thế, ta có $\overrightarrow{A M}=\dfrac{3 t}{2} \overrightarrow{u}=\left(3 t ;-3 t ; \dfrac{3 t}{2}\right)$.\\
Gọi tọa độ của điểm $M$ là $\left(x_M ; y_M ; z_M\right)$.\\
Ta có $\overrightarrow{A M}=\left(x_M-x_A ; y_M-y_A ; z_M-z_A\right)=\left(3 t ;-3 t ; \dfrac{3 t}{2}\right)$.\\
Nên $\heva{&x_M=3 t+x_A \\ &y_M=-3 t+y_A \\ &z_M=\dfrac{3 t}{2}+z_A}\Leftrightarrow\heva{&x_M=3 t+10 \\ &y_M=-3 t+3 \\ &z_M=\dfrac{3 t}{2}.}$
Vậy điểm $M$ có tọạ độ là $\left(3 t+10 ;-3 t+3 ;\dfrac{3 t}{2}\right)$.
\itemch Do $x_B=550$ nên $3 t+10=550$, tức là $t=180$ s. Do đó, ta có điểm $B(550 ;-537 ; 270)$. \\
Vậy $A B=\sqrt{(550-10)^2+(-537-3)^2+(270-0)^2}=\sqrt{656100}=810$ m.
\itemch Đường thẳng $AB$ có vectơ chỉ phương $\overrightarrow{u}=(2 ;-2 ; 1)$ và mặt phẳng $(Oxy)$ có vectơ pháp tuyến $\overrightarrow{k}=(0 ; 0 ; 1)$. Do đó, ta có
\[
\sin (\Delta,(O x y))=|\cos (\overrightarrow{u},\overrightarrow{k})|=\dfrac{\left|\overrightarrow{u} \cdot \overrightarrow{k}\right|}{|\overrightarrow{u}| \cdot|\overrightarrow{k}|}=\dfrac{1}{3 \cdot 1}=\dfrac{1}{3}.\]
Vậy $(\Delta,(O x y)) \approx 19^{\circ}$.
\end{itemchoice}
}
\end{ex}

\begin{ex}%[2H5V2-3]%[TEX ĐỀ MOON 2025]%[Nguyễn Thế Duy]
Trong không gian $Oxyz$, cho mặt phẳng $(P)\colon x-z=0$, đường thẳng $d\colon\heva{& x=1+2t \\ & y=t\\ & z=t}$ và hai điểm $A(1;2;1)$, $B(2;1;4)$. Xét tính đúng sai của các mệnh đề sau
\choiceTF
{\True Điểm $A$ thuộc mặt phẳng $(P)$}
{Hoành độ giao điểm của đường thẳng $d$ và mặt phẳng $(P)$ bằng $1$}
{Biết điểm $I(a;b;c)\in d$, $a>0$ sao cho mặt cầu $(S)$ có tâm $I$ bán kính $R=2\sqrt{2}$ tiếp xúc với $(P)$. Khi đó $a+b+c=9$}
{\True Gọi $\Delta$ là đường thẳng vuông góc với mặt phẳng $(P)$ sao cho khoảng cách từ điểm $A$ đến $\Delta$ bằng $1$. Khi khoảng cách từ $B$ đến $\Delta$ đạt giá trị nhỏ nhất thì $\Delta$ đi qua điểm $M\left(\dfrac{5}{3};\dfrac{5}{3};\dfrac{5}{3}\right)$}
\loigiai{
\begin{itemchoice}
\itemch \textbf{Đúng}.\\
Thay tọa độ điểm $A\left(1; 2; 1\right)$ vào $(P)$ ta được $1 - 1 = 0$.\\
Suy ra $A$ thuộc mặt phẳng $(P)$.
\itemch \textbf{Sai}.\\
Gọi $M$ là giao điểm của đường thẳng $d$ và mặt phẳng $(P)$ suy ra $M\left(1+2t; t; t \right)$ (do $M \in d$).\\
Mà $M \in (P)$ suy ra $1 + 2t - t = 0 \Leftrightarrow t = -1$.\\
Suy ra $M\left(-1; -1; -1 \right)$.\\
Vậy hoành độ giao điểm của đường thẳng $d$ và mặt phẳng $(P)$ bằng $-1$.
\itemch \textbf{Sai}.\\
Do mặt cầu $(S)$ tiếp xúc với $(P)$ nên ta có $\mathrm{d} \left(I, (P) \right) = R = 2\sqrt{2}$.\\
Mà $I \in d$ nên $I\left(1+2t; t; t \right)$.\\
Từ đó ta được $\dfrac{\left|1+2t - t \right|}{\sqrt{1^2 + (-1)^2}} = 2\sqrt{2} \Leftrightarrow \hoac{&t=3\\&t = -5.}$\\
Với $t = -5$ ta được $I\left(-9; -5; -5 \right)$ không thỏa mãn.\\
Với $t = 3$ ta được $I\left(7; 3; 3 \right)$ thỏa mãn.\\
Suy ra $a + b + c = 7 + 3 + 3 = 13$.
\itemch \textbf{Đúng}.\\
Gọi $d_1$ là đường thẳng đi qua $A$ và vuông góc với $P$, ta được $d_1 \colon \heva{&x = 1+t\\&y=2\\&z=1-t.}$\\
Gọi $N$ là hình chiếu vuông góc của $B$ lên $d_1$.\\
Ta có $N \left(1 + t; 2; 1-t \right)$ và $\overrightarrow{BN} = \left(t-1; 1; -t-3 \right)$.\\
Mặt khác $BN \perp d_1$ suy ra $\overrightarrow{BN} \cdot \overrightarrow{AN} = 0 \Leftrightarrow\left(t - 1 \right) \cdot 1 + 1 \cdot 0 + \left(-t - 3 \right) \cdot 1 = 0 \Leftrightarrow t = -1$.\\
Khi đó $N \left(0;2;2 \right)$ và $\overrightarrow{BN} = \left(-2;1;-2 \right)$.\\
Gọi $K$ là hình chiếu vuông góc của $B$ lên $\Delta$. \\
Ta có $BK + KN \geq NB$ suy ra $BK \geq NB - NK$.\\
Dấu \lq\lq $=$\rq\rq \, xảy ra khi $K$, $N$, $B$ thẳng hàng ($K$ nằm giữa $B$, $N$), suy ra $K = \Delta \cap BN$.\\
Ta có $BN \colon \heva{&x = 2 + 2t\\&y=1-t\\&z=4+2t}$ suy ra $K\left(2+2t; 1-t; 4+2t \right)$.\\
Suy ra $\overrightarrow{BK} = \left(2t; -t; 2t\right)$ và $\overrightarrow{KN} = \left(-2-2t;1+t;-2-2t \right)$.\\
Mặt khác $NK = 1$ và $BN = \sqrt{(-2)^2 + 1^2 + (-2)^2} = 3$.\\
Ta có $\overrightarrow{BN} = k\overrightarrow{KN} \Rightarrow \dfrac{2t}{-2-2t} = \dfrac{-t}{1+t} = \dfrac{2t}{-2-2t} = \dfrac{2}{1}$.\\
Từ đó ta được $t = \dfrac{-2}{3}$ suy ra $K\left(\dfrac{2}{3}; \dfrac{5}{3}; \dfrac{8}{3} \right)$.\\
Suy ra $\Delta \colon \heva{&x = \dfrac{2}{3} + t\\&y = \dfrac{5}{3}\\&z = \dfrac{8}{3} - t.}$\\
Dễ thấy với $t = 1$ ta được điểm $\left(\dfrac{5}{3}; \dfrac{5}{3}; \dfrac{5}{3} \right)$ thuộc đường thẳng $\Delta$.
\end{itemchoice}
}
\end{ex}

\begin{ex}%[2H5V1-7]
\immini{Một mái nhà hình tròn được đặt trên 3 cây cột trụ. Các cây cột trụ vuông góc với mặt sàn nhà phẳng và có độ cao lần lượt là $8$m, $9$m, $10$m. Ba chân cột là ba đỉnh của một tam giác đều trên mặt sàn nhà với cạnh dài $8$m. Chọn hệ trục tọa độ như hình vẽ, với $B$ thuộc tia $Ox$, $C$ thuộc tia $Oy$, tia $Oz$ cùng hướng với véc-tơ $\vv{AA'}$; gốc tọa độ $O$ trùng với trung điểm của $AC$ và mỗi đơn vị trên trục có độ dài 1 mét (xem hình vẽ).}{\begin{tikzpicture}[>=stealth,line join=round,line cap=round,font=\footnotesize,scale=1]
\coordinate[label=center:$I$] (I)at(0,2.3);
\begin{scope}[rotate=-18]
\draw[red,fill=blue!30] (I) ellipse (2cm and 1.2cm);
\end{scope}
\coordinate[label=below right:$O$] (O)at(0,0);
\draw[dashed,black]  (-1.4,3.2)coordinate(A')--(1.5,1.4)coordinate(C')--(-0.8,1.8)coordinate(B')--(A') (-1.4,0)coordinate(A)--(1.5,0)coordinate(C) (0,2.33)coordinate(T)--(O)--(-0.8,-0.8)coordinate(B) ($(B')!2/3!(T)$)coordinate(I)--($(B)!2/3!(O)$)coordinate(J);
\draw (-1.8,0)--(A)--(B)--(C);
\draw[->] (1.5,0)--(2,0)node[below]{$y$};
\draw[->] (0,2.33)--(0,4)node[right]{$z$};
\draw[->] (-0.8,-0.8)--(-1.2,-1.2)node[right]{$x$};
\draw[line width=3pt,orange!70!black] (-1.4,0.05)--(-1.4,3.15) (1.5,0.05)--(1.5,1.35) (-0.8,-0.75)--(-0.8,1.75);
\foreach \diem in {T,I,J,O} \fill (\diem)circle(1pt);
\foreach \diem/\vitri in {A/below left,B/left,C/below,A'/above,B'/left,C'/right} \fill (\diem)circle(1pt)node[\vitri]{$\diem$};
\end{tikzpicture}}
\choiceTF
{\True Tọa độ các điểm $A'(0;-4; 10)$, $B'(4\sqrt{3}; 0; 9)$, $C'(0; 4; 8)$}
{\True Phương trình mặt phẳng $\left(A' B' C'\right)$ là $y+4z-36=0$}
{Tọa độ các điểm $A'(0;-4; 10)$, $B'(4\sqrt{3}; 0; 9)$, $C'(0; 4; 8)$}
{\True Phương trình mặt phẳng $\left(A' B' C'\right)$ là $y+4z-36=0$}
\loigiai{
\begin{itemchoice}
\itemch Tọa độ các điểm $A(0;-4; 0)$, $B(4\sqrt{3};0; 0)$, $C(0;4; 0)$, $A'(0;-4; 10)$, $B'(4\sqrt{3}; 0; 9)$, $C'(0; 4; 8)$
\itemch Ta có $\vv{A'B'}=\left(4\sqrt{3};4;-1\right)$; $\vv{A'C'}=\left(-4\sqrt{3};4;-1\right)$.\\
Véc-tơ pháp tuyến của mặt phẳng $(A'B'C')$ là $\vv{n}=\left[\vv{A'B'},\vv{A'C'}\right]=\left(0;8\sqrt{3};32\sqrt{3}\right)=8\sqrt{3}\left(0;1;4\right)$.\\
Phương trình mặt phẳng $(A'B'C')$ là $y+z-36=0$.
\itemch Vec-tơ pháp tuyến của mặt phẳng $(ABC)$ là $\vv{k}=(0;0;1)$.\\
Khi đó $\cos\left((ABC),(A'B'C')\right)=\dfrac{|5|}{\sqrt{4^2+1^2}}=\dfrac{4}{\sqrt{17}}\Rightarrow \left((ABC),(A'B'C')\right)\approx 14^\circ$.\\
Vậy độ dốc của mái khoảng $14^\circ$, mái nhà trên không đạt tiêu chuẩn.
\itemch Gọi $I(a;b;c)$. Suy ra
$\heva{&b+c=36\\&a^2+(b+4)^2+(c-10)^2=\left(a-4\sqrt{3}\right)^2+b^2+(c-9)^2\\&a^2+(b+4)^2+(c-10)^2=a^2+(b-4)^2+(c-8)^2}\Leftrightarrow \heva{&a=\sqrt{5}\\&b=0\\&c=9.}$\\
Vậy $I(\sqrt{5};0;9)$, điểm $I$ cách mặt sàn khoảng $9$ mét.
\end{itemchoice}

}
\end{ex}

\begin{ex}%[2H5V1-7]%[TEX ĐỀ MOON 2025]%[Nguyễn Văn Hiệp]
\immini[thm]
{
Hình minh hoạ sơ đồ một ngôi nhà trong hệ trục tọa độ $Oxyz$, trong đó nền nhà, bốn bức tường và hai mái nhà đều là hình chữ nhật. Xét tính đúng sai của các mệnh đề sau
\choiceTF
{\True Tọa độ của điểm $A$ là $(4;0;0)$}
{Tọa độ của vectơ $\overrightarrow{AH}$ là $(4;5;3)$}
{Tích vô hướng của $\overrightarrow{AH}$ và $\overrightarrow{AF}$ bằng $3$}
{\True Góc dốc của mái nhà, tức là số đo của góc nhị diện có cạnh là đường thẳng $FG$, hai mặt lần lượt là $(FGQP)$ và $(FGHE)$ bằng $26{,}6^{\circ}$ (làm tròn kết quả đến hàng phần mười của độ)}
}
{
\begin{tikzpicture}[font=\footnotesize, line join=round, line cap=round, >=stealth, scale=0.7]
\def\a{3}
\def\b{5}
\def\h{3}
\path (0:0) coordinate (C)
++(0:\a) coordinate (B)
++(-160:\b) coordinate (O)
($(O)+(B)-(C)$) coordinate (A)
($(O)+(90:\h)$) coordinate (E)
($(B)+(90:\h)$) coordinate (G)
($(C)+(90:\h)$) coordinate (H)
($(A)+(90:\h)$) coordinate (F)
($(A)+(0:1)$) coordinate (x)
($(H)+(35:2)$) coordinate (Q)
($(E)+(35:2)$) coordinate (P)
($(E)+(90:1)$) coordinate (z)
($(O)!1.3!(C)$) coordinate (y);
\draw[dashed] (G)--(H)--(C)--(B) (C)--(O);
\draw[] (G)--(Q)--(H)--(E)--(F)--(G)--(B)--(A)--(O)--(E) (F)--(A) (F)--(P)--(E) (P)--(Q);
\draw [->] (A)--(x);
\draw [->] (E)--(z);
\draw [->,dashed] (C)--(y);
\draw [] (Q)node[above]{$Q(2; 5; 4)$} (G)node[right]{$G(4; 5; 3)$} (B)node[right]{$B(4; 5; 0)$} (P)node[right]{$P(2; 0; 4)$} (O)node[below]{$O(0; 0; 0)$} (E)node[left]{$E(0; 0; 3)$} (x)node[below]{$x$} (y)node[above]{$y$} (z)node[left]{$z$};
\foreach \x/\g in {A/-90,C/180,F/0,H/90}
\fill[black] (\x) circle (1pt)
($(\g:4mm)+(\x)$) node {$\x$};
\end{tikzpicture}
}
\loigiai
{
\begin{itemchoice}
\itemch Theo hình vẽ tọa độ điểm $A(4;0;0)$.
\itemch Tọa độ $H(0;5;3)$ suy ra $\overrightarrow{AH} = (-4;5;3)$.
\itemch
Tọa độ $F(4;0;3)$ suy ra $\overrightarrow{AF} = (4;0;3)$ và $\overrightarrow{AH}\cdot \overrightarrow{AF}=-4\cdot 4 + 5\cdot 0 + 3\cdot 3 = -7 \neq 3$.
\itemch
Một vectơ chỉ phương của mặt phẳng $(FGQP)$ là\\ $\overrightarrow{n_1}=\left[\overrightarrow{FP},\overrightarrow{FG}\right]=\left[(-2;0;1),(0;5;0)\right]=(-5;0;10)$.\\
Một vectơ chỉ phương của mặt phẳng $(FGHE)$ là\\ $\overrightarrow{n_2}=\left[\overrightarrow{FG},\overrightarrow{FE}\right]=\left[(0;5;0),(-4;0;0)\right]=(0;0;20)$.\\
Đặt $\theta$ góc nhị diện cạnh $FG$,  hai mặt lần lượt là $(FGQP)$ và $(FGHE)$.\\
Ta có $\cos \theta = \dfrac{\overrightarrow{n}_1 \cdot \overrightarrow{n}_2}{\left|\overrightarrow{n}_1\right|\left|\overrightarrow{n}_2\right|}=\dfrac{2\sqrt{5}}{5}\Rightarrow \theta \approx 26{,}6^\circ$.
\end{itemchoice}
}
\end{ex}

\begin{ex}%[2H5V1-5]%[Tex đề Moon 2025]%[Nguyễn Hồng Thạch]
Trong không gian với hệ tọa độ $Oxyz$, cho hai điểm $A(1;2;-1)$, $B(2;1;0)$ và mặt phẳng $(P)\colon 2x+y-3z+1=0$. Gọi $(Q)$ là mặt phẳng chứa $A$, $B$ và vuông góc với mặt phẳng $(P)$.
\choiceTF
{\True Một vec-tơ pháp tuyến của mặt phẳng $(P)$ là $(2;1;-3)$}
{Một vec-tơ pháp tuyến của mặt phẳng $(Q)$ là $(2;1;-3)$}
{\True Phương trình mặt phẳng $(Q)$ có dạng $ax+by+cz+9=0$. Khi đó $a+b+c=-10$}
{Khoảng cách từ điểm $O$ đến mặt phẳng $(Q)$ bằng $\dfrac{15\sqrt{38}}{38}$}
\loigiai{\begin{itemchoice}
\itemch Dựa vào phương trình mặt phẳng $(P)$ ta thấy $(P)$ có một vec-tơ pháp tuyến là\\ $\overrightarrow{n}_P=(2;1;-3)$.
\itemch Ta có $\overrightarrow{AB}= (1;-1;1)$.\\
Gọi $\overrightarrow{n}_Q$ là vec-tơ pháp tuyến của mặt phẳng $(Q)$.\\
Ta có $\heva{&(P)\perp (Q)\\&(P)\ \text{chứa}\ A,\ B}\Rightarrow \heva{&\overrightarrow{n}_P\cdot \overrightarrow{n}_Q=0\\&\overrightarrow{n}_P\cdot\overrightarrow{AB}=0}\Rightarrow \overrightarrow{n}_Q=\left[\overrightarrow{n}_P,\overrightarrow{AB}\right]=(-2;-5;-3)$.\\
Ta thấy $\overrightarrow{n}_Q=(-2;-5;3)$ khồng cùng phương với $\overrightarrow{n}=(2;1;-3)$ nên $\overrightarrow{n}=(2;1;-3)$ không phải là vec-tơ pháp tuyến của mặt phẳng $(Q)$.
\itemch Mặt phẳng $(Q)$ có vec-tơ pháp tuyến là $\overrightarrow{u}_Q=(-2;-5;-3)$ và đi qua điểm $B(2;1;0)$ nên có phương trình là \[-2\cdot(x-2)-5\cdot(y-1)-3\cdot(z-0)=0\Leftrightarrow -2x-5y-3z+9=0.\]
Suy ra $a=-2$, $b=-5$, $c=-3$.\\
Vậy $a+b+c=-10$.
\itemch Khoảng cách từ  điểm $O$ đến mặt phẳng $(Q)$ là
\[\mathrm{d}(O,(Q))=\dfrac{\left|-2\cdot0-5\cdot0-2\cdot0+9\right|}{\sqrt{(-2)^2+(-5)^2+(-3)^2}}=\dfrac{9\sqrt{38}}{38}.\]
\end{itemchoice}}
\end{ex}

\begin{ex}%[2H2V2-6]
Hình vẽ sau mô tả vị trí của máy bay vào thời điểm $9$h$30$ phút. Biết các đơn vị trên hình tính theo đơn vị km.
\begin{center}
\begin{tikzpicture}[line join = round, line cap=round,>=stealth,font=\footnotesize,scale=1]
\path
(0,0) coordinate (O)
(5,0) coordinate (B)
(-3,-2) coordinate (A)
(0,4) coordinate (C)
($(A)+(B)-(O)$) coordinate (N)
($(N)+(0,4)$) coordinate (M)
;
\draw 	(O)--(B)(O)--(A) (O)--(C);

\foreach \x/\r/\p in{A/180/x,B/90/y,C/90/z}
\draw[->,line width=2pt] (O)--($(O)!1.2!(\x)$)node[scale=1.5,shift={(\r:3mm)}]{$\p$};
;

%\draw pic[draw,angle radius=7mm] {angle = M--O--C};
\draw (A) node[shift={(150:4mm)}]{$150$};
\draw (B) node[shift={(90:4mm)}]{$300$};
\draw (C) node[shift={(180:4mm)}]{$9$};
\draw (B) node[shift={(-70:4mm)}]{(Đông)};
\draw (A) node[shift={(-30:4mm)}]{(Nam)};


\draw[dashed] (O)--(N)--(B) (A)--(N)--(M)--(C);
\draw[->,line width=2pt] (O)--(M);


\draw (M) node[yshift=.4cm]{	\begin{tikzpicture}[line join = round, line cap=round,>=stealth,font=\footnotesize,scale=0.15]

\draw[cyan,line width=3pt]
(0,0)--(0.2,-0.5)coordinate (A)--(5,-1.2)coordinate (O)
(10,0)--(9.8,-0.5)coordinate (A')--(5,-1.2)
(5,-0.8) circle(0.7cm)
($(A)!0.6!(O)+(0,-0.3)$) circle(0.3cm)
($(A)!0.6!(O)+(0,-0.3)$) circle(0.2cm)
($(A')!0.6!(O)+(0,-0.3)$) circle(0.3cm)
($(A')!0.6!(O)+(0,-0.3)$) circle(0.2cm)
(7,0)--(5,-0.3) -- (3,0)
(5,-0.3)--(5,1.2)
;
\fill[cyan] (5,-0.8) circle(0.7cm);

\fill[black,xshift=-0.05cm] (4.5,-0.7) rectangle (4.6,-0.5)
(4.7,-0.7) rectangle (4.8,-0.5)
(4.9,-0.7) rectangle (5,-0.5)
(5.1,-0.7) rectangle (5.2,-0.5)
(5.3,-0.7) rectangle (5.4,-0.5)
(5.5,-0.7) rectangle (5.6,-0.5)
;


\draw[line width=2pt]
($(A')!0.75!(O)$)--($(A')!0.75!(O)+(0,-0.65)$)
($(A')!0.75!(O)+(0,-0.65)$)--++(0:0.2)
($(A')!0.75!(O)+(0,-0.65)$)--++(180:0.2)--++(-90:0.1)
($(A')!0.75!(O)+(0,-0.65)$)--++(180:0.2)--++(90:0.1)
($(A')!0.75!(O)+(0,-0.65)$)--++(0:0.2)--++(-90:0.1)
($(A')!0.75!(O)+(0,-0.65)$)--++(0:0.2)--++(90:0.1)
($(A)!0.75!(O)$)--($(A)!0.75!(O)+(0,-0.65)$)
($(A)!0.75!(O)+(0,-0.65)$)--++(0:0.2)
($(A)!0.75!(O)+(0,-0.65)$)--++(180:0.2)--++(-90:0.1)
($(A)!0.75!(O)+(0,-0.65)$)--++(180:0.2)--++(90:0.1)
($(A)!0.75!(O)+(0,-0.65)$)--++(0:0.2)--++(-90:0.1)
($(A)!0.75!(O)+(0,-0.65)$)--++(0:0.2)--++(90:0.1)
;
\end{tikzpicture}	}
;
\end{tikzpicture}
\end{center}
\choiceTF
{\True Máy bay đang ở độ cao $9$ km}
{ Tọa độ của máy bay lúc này là $(300; 150; 9)$}
{\True Phi công để máy bay ở chế độ tự động với vận tốc theo hướng đông là $750$ km/h, độ cao không đổi. Biết rằng gió thổi theo hướng đông với vận tốc $10$ m/s. Giả sử vận tốc và hướng gió không đổi thì lúc $10$h$30$ phút máy bay ở tọa độ là $(150; 1086; 9)$}
{Sau khi bay đến vị trí lúc $10$h$30$ thì máy bay bay ngược lại (hướng tây) với vận tốc $800$ km/h với độ cao không đổi, biết lúc đó trời lặng gió thì lúc $11$h máy bay ở tọa độ $(686; 150; 9)$}
\loigiai{
\begin{itemchoice}
\itemch  {\bf Đúng.}\\
Dựa vào hình vẽ ta thấy máy bay đang ở độ cao $9$ km.
\itemch  {\bf Sai.}\\
Máy bay đang ở tọa độ $(150; 300; 9)$.
\itemch  {\bf Đúng.}\\
Vận tốc gió $10$ m/s $= 36$ km/h.\\
Máy bay bay tự động trong khoảng thời gian từ $9$h$30$ đến $10$h$30$ với quãng đường $750$ km.\\
Quãng đường thực tế máy bay bay được là $750+36=786$ km.\\
Do đó tọa độ máy bay là $(150; 1\,086; 9)$.
\itemch  {\bf Sai.}\\
Quãng đường máy bay bay được trong khoảng thời gian từ $10$h$30$ đến $11$h là \[800\cdot \dfrac{1}{2}=400\,\, \text{km}.\]
Do đó tọa độ máy bay là $(150; 868; 9)$.
\end{itemchoice}
}
\end{ex}

\begin{ex}%[50 Đề minh họa tốt nghiệp 2025 - Đề 13]%[Lê Hữu Kiệt - Lê Quân]%[2H2V2-6]
Một chiếc điện thoại được đặt trên một giá đỡ có ba chân với điểm đặt $S(0;0;20)$ và các điểm chạm mặt đất của ba chân lần lượt là $A(0;-6;0)$, $B(3\sqrt{3};3;0)$, $C(-3\sqrt{3};3;0)$ (đơn vị cm). Cho biết điện thoại có trọng lượng là $2$ N và ba lực tác dụng lên giá đỡ được phân bố như hình vẽ là ba lực $\overrightarrow{F}_1$, $\overrightarrow{F}_2$, $\overrightarrow{F}_3$ có độ lớn bằng nhau và đo bằng đơn vị N.
\begin{center}
\tdplotsetmaincoords{75}{115}
\begin{tikzpicture}[font=\footnotesize, line join=round, line cap=round, >=stealth, scale=0.4, tdplot_main_coords]
\pgfmathsetmacro\bancanba{3*sqrt(3)}
\draw[->] (-7,0,0) -- (9,0,0) node[anchor=north east] {$x$};
\draw[->] (0,-7,0) -- (0,7,0) node[anchor=north west] {$y$};
\draw[->] (0,0,10.8) -- (0,0,12) node[anchor=south] {$z$};
\path
(0,0,0) coordinate (O)
(0,-6,0) coordinate (A)
(\bancanba,3,0) coordinate (B)
(-\bancanba,3,0) coordinate (C)
(0,0,9) coordinate (S);
\draw[dashed] (0,0,0) circle [radius=6] (A)--(B)--(C)--cycle (S)--(O);
\draw (S)--(A) (S)--(B) (S)--(C);
\foreach \x [count=\i from 1] in {A,B,C}{
\path ($(S)!1/3!(\x)$) coordinate (f\i);
\draw[->] (S)--(f\i)node[right=-1mm]{$\overrightarrow{F}_\i$};
}
\draw[rounded corners=1] (0,-2,8.9) rectangle (0,2,11);
\fill (0,-2,9.2) rectangle (0,-1.8,10.2);
\foreach \x/\g in {C/above right, A/above left, B/below, S/above right, O/below}{
\fill (\x) circle (3.3pt)node[\g]{$\x$};
}
\end{tikzpicture}
\end{center}
\choiceTF
{\True $\overrightarrow{SA}=(0;-6;-20)$}
{$\overrightarrow{F}_1+\overrightarrow{F}_2+\overrightarrow{F}_3=\overrightarrow{F}(0;0;2)$}
{$\left|\overrightarrow{F}_1\right|=\dfrac{1}{20}\left|\overrightarrow{SA}\right|$}
{\True Biết tọa độ của lực $\overrightarrow{F}_1=(a;b;c)$, khi đó $T=2a+5b+6c=-5$}
\loigiai{
\begin{itemchoice}
\itemch Ta có $\overrightarrow{SA}=(0;-6;-20)$.
\itemch Ta có $\overrightarrow{F}_1+\overrightarrow{F}_2+\overrightarrow{F}_3=\overrightarrow{F}$, do điện thoại có trọng lượng là $2$ N nên $\left|\overrightarrow{F}\right|=2$.\\
Lại có $\left|\overrightarrow{SO}\right|=20$, $\overrightarrow{F}$ và $\overrightarrow{SO}=(0;0;-20)$ cùng hướng nên $\overrightarrow{F}=\dfrac{1}{10}\overrightarrow{SO}$.\\
Suy ra $\overrightarrow{F}=(0;0;-2)$.
\itemch Do ba lực $\overrightarrow{F}_1$, $\overrightarrow{F}_2$, $\overrightarrow{F}_3$ có độ lớn bằng nhau nên với cùng số $k$, ta có $\overrightarrow{F}_1=k\overrightarrow{SA}$, $\overrightarrow{F}_2=k\overrightarrow{SB}$ và $\overrightarrow{F}_3=k\overrightarrow{SC}$.\\
Ta có $\overrightarrow{SB}=(3\sqrt3;3;-20)$, $\overrightarrow{SC}=(-3\sqrt3;3;-20)$. Khi đó
\[ \overrightarrow{F}_1+\overrightarrow{F}_2+\overrightarrow{F}_3=\overrightarrow{F} \Leftrightarrow \heva{& k\cdot0+k\cdot3\sqrt3+k\cdot(-3\sqrt3) =0 \\& k\cdot(-6)+k\cdot3+k\cdot3=0 \\& k\cdot(-20)+k\cdot(-20)+k\cdot(-20)=-2} \Leftrightarrow k=\dfrac{1}{30}. \]
Suy ra $\overrightarrow{F}_1=\dfrac{1}{30}\overrightarrow{SA}$.\\
Vậy $\left|\overrightarrow{F}_1\right|=\dfrac{1}{30}\left|\overrightarrow{SA}\right|$.
\itemch Ta có $\overrightarrow{F}_1=\dfrac{1}{30}\overrightarrow{SA}$ suy ra $\overrightarrow{F}_1=\left(0;-\dfrac{1}{5};-\dfrac{2}{3}\right)$. Do đó $a=0$, $b=-\dfrac{1}{5}$, $c=-\dfrac{2}{3}$.\\
Vậy $T=2a+5b+6c=-5$.
\end{itemchoice}
}
\end{ex}

\begin{ex}%[2H2V2-6]
Một máy bay đang di chuyển về phía sân bay. Tại thời điểm hiện tại, vị trí của máy bay là $B(150;150;5\,000)$ (trong đó $5\,000$ m là độ cao của máy bay so với mặt đất). Máy bay đang di chuyển thẳng tới sân bay $C(0;0;0)$ với vận tốc $700$ km/h. Xét tính đúng sai của các mệnh đề sau:
\choiceTF
{\True Phương trình tham số của đường thẳng mà máy bay di chuyển theo là $\heva{& x=150-150t \\ & y=150-150t\\ & z=5\,000-5\,000t}$}
{Khoảng cách từ vị trí hiện tại của máy bay $B$ đến sân bay $C$ xấp xỉ bằng $3\,905{,}6$ km}
{Với vận tốc trung bình của máy bay là $700$ km/h, thời gian để máy bay hạ cánh là khoảng $5{,}5$ giờ}
{Nếu hệ thống kiểm soát không lưu yêu cầu liên lạc với máy bay khi nó còn cách sân bay $40$ km thì khi máy bay ở vị trí $(6;6;200)$, nó còn cách sân bay là $40$ km}
\loigiai{
\begin{itemchoice}
\itemch
Véc-tơ chỉ phương của đường thẳng $BC$ là $\overrightarrow{BC}=(-150;-150;-5\,000)$.\\
Phương trình tham số của đường thẳng đi qua $B(150;150;5\,000)$ và nhận $\overrightarrow{BC}$ làm véc-tơ chỉ phương  là
$\heva{& x=150-150t \\ & y=150-150t \\ & z=5\,000-5\,000t}$.
\itemch
Khoảng cách từ $B$ đến $C$ là độ dài đoạn thẳng \allowdisplaybreaks
\begin{eqnarray*}
BC=\left|\overrightarrow{BC}\right|=\sqrt{(-150)^2+(-150)^2+(-5\,000)^2}\approx 5\,004{,}5\,\, (\text{m})=5{,}0\,045\,\, (\text{km}).
\end{eqnarray*}
\itemch
Vận tốc $v=700$ km/h. Quãng đường $d=BC \approx 5{,}0\,045$ km.\\
Thời gian để máy bay hạ cánh (đi hết quãng đường $BC$) là $\dfrac{5{,}0\,045}{700} \approx 0{,}00\,715=$ (giờ).\\
\itemch
Xét vị trí $\mathrm{P}\left(6;6;200\right)$. \\	Khoảng cách từ $\mathrm{P}\left(6;6;200\right)$ đến sân bay $C(0;0;0)$ là \\
$PC=\sqrt{(6-0)^2+(6-0)^2+(200-0)^2}\approx200{,}18\,\, (\text{m})= 0{,}200$ km.
\end{itemchoice}
}
\end{ex}

\begin{ex}%[2H2V2-2]%[TEX Đề Moon 2025]%[Vũ Hồng Toàn]
Trong không gian $Oxyz$, cho tam giác $ABC$ có $A(-1;2;4)$, $B(3;0;-2)$, $C(1;3;7)$. Gọi $D$ là chân đường phân giác trong của góc $A$. Xét tính đúng sai của các mệnh đề sau
\choiceTF
{\True Độ dài cạnh $AB$ là $2\sqrt{14}$}
{Trọng tâm của tam giác $ABC$ là điểm $G(1;2;3)$}
{\True Tích vô hướng $\overrightarrow{AB}\cdot \overrightarrow{AC}$ bằng $-12$}
{\True Độ dài vectơ $\overrightarrow{OD}$ bằng $\dfrac{\sqrt{205}}{3}$}
\loigiai{
\begin{itemchoice}
\itemch Ta có $AB = \sqrt{4^2 + (-2)^2 + (-6)^2} = \sqrt{16 + 4 + 36} = 2\sqrt{14}$.
\itemch Ta có $G\left(\dfrac{-1+3+1}{3};\dfrac{2+0+3}{3};\dfrac{4-2+7}{3}\right)\Rightarrow G\left(1;\dfrac{5}{3};3\right)$.
\itemch $\overrightarrow{AB}=(4;-2;-6)$ và $\overrightarrow{AC}=(2;1;3)$.\\
Vậy $\overrightarrow{AB}\cdot \overrightarrow{AC}=4\cdot 2-2\cdot 1-6\cdot 3=-12$.
\itemch
\immini{
Ta có $AB = 2\sqrt{14}$ và $AC = \sqrt{14}$. Gọi $D(x;y;z)$\\
Khi đó $\dfrac{BD}{DC}=\dfrac{AB}{AC}=\dfrac{2\sqrt{14}}{\sqrt{14}}=2$.\\
Suy ra $\overrightarrow{DB}=-2\overrightarrow{DC}$
}
{
\begin{tikzpicture}[declare function={gocc=35;a=4;b=3;}]
\path (0,0) coordinate (B)++(a,0) coordinate (C) ++(180-gocc:b) coordinate (A)
($(A)!1cm!(B)$) coordinate (AB)
($(A)!1cm!(C)$) coordinate (AC)
($(AB)!0.5!(AC)$) coordinate (At)
(intersection of A--At and B--C) coordinate (D);
\foreach\x/\y/\z in {B/A/D}{
\path pic[draw,angle radius=7pt]{ angle=\x--\y--\z};}
\foreach\x/\y/\z in {D/A/C}{
\path pic[draw,angle radius=9pt]{ angle=\x--\y--\z};}
\draw (A)--(B)--(C)--cycle --(D);
\foreach \x/\goc in {A/90,B/180,C/-90,D/-90}{
\draw[fill] (\x) circle (1pt) node[shift={(\goc:7pt)},font=\small]{$\x$};
}
\end{tikzpicture}
}
$\Rightarrow\heva{&3-x=-2(1-x)\\&-y=-2(3-y)\\&-2-x=-2(7-x)}\Rightarrow\heva{&x=\dfrac{5}{3}\\&y=2\\&z=4.}$\\
Vậy $OD=\sqrt{\left(\dfrac{5}{3}\right)^2+2^2+4^2}=\dfrac{\sqrt{205}}{3}$.
\end{itemchoice}
}
\end{ex}

\begin{ex}%[2D6V2-4]%[TEX ĐỀ MOON 2025]%[Nguyễn Cường]
Lớp 12A có $30$ học sinh, trong đó có $17$ bạn nữ còn lại là nam. Có $3$ bạn tên Hiền, trong đó có $1$ bạn nữ và $2$ bạn nam. Thầy giáo gọi ngẫu nhiên $1$ bạn lên bảng. Các mệnh đề sau đúng hay sai?
\choiceTF
{\True Xác suất để có tên Hiền là $\dfrac{1}{10}$}
{Xác suất để có tên Hiền, nhưng với điều kiện bạn đó nữ là $\dfrac{3}{17}$}
{\True Xác suất để có tên Hiền, nhưng với điều kiện bạn đó nam là $\dfrac{2}{13}$}
{Nếu thầy giáo gọi $1$ bạn có tên là Hiền lên bảng thì xác xuất để bạn đó là bạn nữ là $\dfrac{3}{17}$}
\loigiai{
\begin{itemchoice}
\itemch Gọi $A$ là biến cố \lq\lq bạn Hiền lên bảng\rq\rq.\\
Suy ra $n(A)=3$.\\
Xác suất $\mathrm{P}(A)=\dfrac{n(A)}{n(\Omega)}=\dfrac{3}{30}=\dfrac{1}{10}$.
\itemch Gọi $B$ là biến cố \lq\lq bạn nữ lên bảng\rq\rq, suy ra $\mathrm{P}(B)=\dfrac{17}{30}$.\\
$AB$ là biến cố \lq\lq bạn nữ tên Hiền lên bảng\rq\rq, suy ra $\mathrm{P}(AB)=\dfrac{1}{30}$.\\
Khi đó, xác suất để có tên Hiền, nhưng với điều kiện bạn đó nữ là $\mathrm{P}(A\mid B)=\dfrac{\mathrm{P}(AB)}{\mathrm{P}(B)}=\dfrac{1}{17}$.
\itemch Xác suất để có tên Hiền, nhưng với điều kiện bạn đó nam là $\mathrm{P}(A\mid \overline{B})=\dfrac{2}{13}$.
\itemch Áp dụng công thức Bayes ta có
\allowdisplaybreaks
\begin{eqnarray*}
\mathrm{P}(B\mid A)&=&\dfrac{\mathrm{P}(A\mid B)\cdot \mathrm{P}(B)}{\mathrm{P}(A)}\\
&=&\dfrac{\dfrac{1}{17}\cdot\dfrac{1}{30}}{\dfrac{1}{10}}\\
&=&\dfrac{1}{51}.
\end{eqnarray*}
\end{itemchoice}
}
\end{ex}

\begin{ex} %[2D6V2-3]
Nobita và Shizuka chuẩn bị đi tham quan hòn đảo Honshu trong hai ngày thứ Bảy và Chủ nhật tuần này. Ở hòn đảo Honshu này, mỗi ngày chi có nắng hoặc mưa, nếu một ngày là nắng thì khả năng xảy ra mưa ở ngày ngày tiếp theo là $20\%$, còn nếu một ngày là mưa thì khả năng ngày hôm sau vẫn mưa là $30\%$. Theo dự báo thời tiết, xác suất trời sẽ nắng vào thứ Bảy tuần này là $0{,}7$. Gọi $A$ là biến cố \lq\lq Ngày thứ Bảy tuần này trời nắng\rq\rq\, và $B$ là biến cố \lq\lq Ngày Chủ nhật tuần này trời mưa\rq\rq.
\choiceTF
{\True $P(A)=0{,}7$}
{Xác suất có điều kiện $P(\overline{B} \mid A)=0{,}77$}
{Xác suất ngày chủ nhật tuần này trời nắng là $80\%$}
{Bạn mèo máy Doraemon có thể đến được tương lai nhưng lại chỉ đến hòn đảo vào ngày Chủ nhật và báo cho Nobita biết rằng Chủ nhật tuần này trời mưa, khi đó xác suất ngày thứ 7 trời nắng là $62\%$ (làm tròn đến hàng đơn vị theo đơn vị phần trăm)}
\loigiai{
\begin{center}
\begin{tikzpicture}
\def\gocm{20}
\def\gocn{10}
\def\r{4}
\tikzset{s/.style={outer sep=0.5 mm,draw=magenta,rectangle,minimum width=2.75cm,rounded corners=1mm}}
\path(0,0)node(O){}++(\gocm:\r)node[s](A1){Nắng (A)}++(\gocn:\r)node[s](A2){Mưa (B)};
\path(A1)++({-\gocn}:\r)node[s](a2){Nắng $(\overline{B})$};
\path(O)++(-\gocm:\r)node[s](B1){Mưa $(\overline{A})$}++(\gocn:\r)node[s](B2){Mưa (B)};
\path(B1)++({-\gocn}:\r)node[s](b2){Nắng $(\overline{B})$};
\foreach \x/\y in {
O/A1,A1/A2,
O/B1,B1/B2,
A1/a2,
B1/b2}
\draw[-stealth](\x.east)--(\y.west);
\path(O)--(A1.west)node[pos=0.5,above,sloped]{$\mbox{0{,}7}$}(O)--(B1.west)node[pos=0.5,below]{$\mbox{0{,}3}$}(B1.east)--(B2.west)node[pos=0.5,above]{$\mbox{0{,}3}$}(A1.east)--(A2.west)node[pos=0.5,above]{$\mbox{0{,}2}$}
(A1.east)--(a2.west)node[pos=0.5,below,sloped]{$\mbox{0{,}8}$}
(B1.east)--(b2.west)node[pos=0.5,below,sloped]{$\mbox{0{,}7}$};
%%Node dòng trên
\path(A2)++(0,1)node{\textbf{Chủ nhật}}++(180:4)node{\textbf{Thứ bảy}};
\end{tikzpicture}
\end{center}
Ta có sơ đồ hình cây như hình vẽ.\\
Ta có $A$ là biến cố \lq\lq Ngày thứ Bảy tuần này trời nắng\rq\rq\, và $B$ là biến cố \lq\lq Ngày Chủ nhật tuần này trời mưa\lq\lq.
\begin{itemchoice}
\itemch Theo giả thiết ta có $P(A)=0{,}7$.
\itemch Ta có $ \mathrm{P}(\overline{B} \mid A)=\dfrac{ \mathrm{P}(\overline{B} \cap A)}{ \mathrm{P}(A)}=\dfrac{ \mathrm{P}(\overline{B})\cdot  \mathrm{P}(A \mid \overline{B})}{ \mathrm{P}(A)}=0{,}8$.
\itemch Theo công thức xác suất toàn phần ta có
\begin{eqnarray*}
\mathrm{P}(\overline{B}) &=& \mathrm{P}(A) \cdot \mathrm{P}(\overline{B}|A) +\mathrm{P}\left(\overline{A}\right) \cdot  \mathrm{P}\left(\overline{B}|\overline{A}\right)\\
&=& 0{,}7 \cdot 0{,}8+ 0{,}3\cdot 0{,}7= 0{,}77.
\end{eqnarray*}
\itemch Ta có $ \mathrm{P}(A\mid B)=\dfrac{ \mathrm{P}(A \cap B)}{ \mathrm{P}(B)}=\dfrac{ \mathrm{P}(A)\cdot  \mathrm{P}(B \mid A)}{ \mathrm{P}(B)}$.\\
$ \mathrm{P}(B)=\mathrm{P}(A \cap B)+\mathrm{P}(\overline{A} \cap B)=\mathrm{P}(A)\cdot\mathrm{P}(B\mid A)+\mathrm{P}(\overline{A})\cdot\mathrm{P}(B\mid \overline{A})=0{,}23$.\\
Do đó $ \mathrm{P}(A\mid B)=\dfrac{ \mathrm{P}(A \cap B)}{ \mathrm{P}(B)}=\dfrac{ \mathrm{P}(A)\cdot  \mathrm{P}(B \mid A)}{ \mathrm{P}(B)}=\dfrac{0{,}7 \cdot 0{,}2}{0{,}23}=0{,}6087\approx 61\%$.
\end{itemchoice}
}
\end{ex}

\begin{ex}%[2D6V2-2]
Chuồng I có $3$ con gà trống và $7$ con gà mái, chuồng II có $4$ con gà trống và $5$ con gà mái. Có $1$ con gà từ chuồng I sang chuồng II. Sau đó, có $1$ con gà từ chuồng II chạy ra ngoài.\\
Gọi $A$ là biến cố có $1$ con gà mái từ chuồng I sang chuồng II.\\
Gọi $B$ là biến cố một con gà từ chuồng II chạy ra ngoài là gà trống.
\choiceTF
{\True $\mathrm{P}(A)=0{,}7$}
{$\mathrm{P}(B\mid A)=0{,}5$}
{\True Xác suất để con gà từ chuồng II chạy ra ngoài là gà trống là $43\%$}
{\True Biết con gà từ chuồng II chạy ra ngoài là gà mái, xác suất để con gà từ chuồng I sang chuồng II là gà trống là $\dfrac{5}{19}$}
\loigiai{
\begin{itemchoice}
\itemch Ta có $\mathrm{P}(A)=\dfrac{\mathrm{C}_7^1}{\mathrm{C}_{10}^1}=0{,}7$.\\
Gọi $A$ là biến cố có $1$ con gà mái từ chuồng I sang chuồng II.\\
Suy ra $\mathrm{P}(A)=\dfrac{\mathrm{C}_3^1}{\mathrm{C}_{10}^1}=0{,}3$.
\itemch Ta có $\mathrm{P}(B\mid A)=\dfrac{\mathrm{C}_4^1}{\mathrm{C}_{10}^1}=0{,}4$.
\itemch Ta có $P\left(B \mid \overline{A}\right)=\dfrac{\mathrm{C}_5^1}{\mathrm{C}_{10}^1}=0{,}5$.\\
Ta có sơ đồ cây sau
\begin{center}
\begin{tikzpicture}[declare function={dai=2.5;cao=0.65;},>=stealth,font=\scriptsize]
\tikzset{nhan/.style={minimum size=19pt,font=\small,inner sep=0pt}}
\path (0,0) node[nhan] (G){\text{Gốc}}
(dai,{1.5*cao}) node[nhan] (B) {$A$}
(dai,{-1.5*cao}) node[nhan] (nB) {$\overline{A}$}
({2*dai},{3*cao}) node[nhan] (BA) {$B$}
({2*dai},{cao}) node[nhan] (BnA) {$\overline{B}$}
({2*dai},{-cao}) node[nhan] (nBA) {$B$}
({2*dai},{-3*cao}) node[nhan] (nBnA) {$\overline{B}$};

%Phần mũi tên
\draw[->] (G.0)--(B.200) node[sloped,pos=0.5,above]{$0{,}7$};
\draw[->] (G.0)--(nB.160) node[sloped,pos=0.5,below]{$0{,}3$};
\draw[->] (B.10)--(BA.190) node[sloped,pos=0.5,above]{$0{,}4$};
\draw[->] (B.10)--(BnA.170) ;
\draw[->] (nB.-10)--(nBA.190) node[sloped,pos=0.5,above]{$0{,}5$};
\draw[->] (nB.-10)--(nBnA.170) ;
\end{tikzpicture}
\end{center}
Áp dụng công thức xác suất toàn phần, ta có \[\mathrm{P}(B)=\mathrm{P}(A)\cdot \mathrm{P}(B\mid A)+\mathrm{P}\left(\overline{A}\right)\cdot \mathrm{P}\left(B\mid\overline{A}\right)=0{,}7 \cdot 0{,}4+0{,}3 \cdot 0{,}5=0{,}43=43\%.\]
\itemch
Ta có $\mathrm{P}\left(\overline{B}\right)=1-\mathrm{P}(B)=0{,}57$.\\
Suy ra \[\mathrm{P}\left(\overline{A} \mid\overline{B}\right)=\dfrac{\mathrm{P}\left(\overline{A B}\right)}{\mathrm{P}\left(\overline{B}\right)}=\dfrac{\dfrac{3}{10} \cdot\dfrac{ \mathrm{C}_5^1}{ \mathrm{C}_{10}^1}}{0,57}=\dfrac{5}{19}.\]
\end{itemchoice}
}
\end{ex}

\begin{ex}%[2D6V1-4]
Một công ty đấu thầu hai dự án. Khả năng thắng thầu của các dự án lần lượt là $0{,}4$ và $0{,}5$. Khả năng thắng thầu cả hai dự án là $0{,}3$. Gọi $A$, $B$ lần lượt là biến cố thắng thầu dự án $1$ và dự án $2$. Xét tính đúng sai của các mệnh đề sau
\choiceTF
{Hai biến cố $A$ và $B$ độc lập}
{\True Biết công ty thắng thầu dự án $1$, thì xác suất công ty thắng thầu dự án $2$ là $0{,}75$}
{Biết công ty không thắng thầu dự án $1$, thì xác suất công ty thắng thầu dự án $2$ là $\dfrac{2}{3}$}
{\True Xác suất công ty thắng thầu đúng $1$ dự án là $0{,}3$}
\loigiai{
\begin{itemchoice}
\itemch
Ta có $P(A)\cdot B(B)=0{,}4\cdot 0{,}5=0{,}2\ne 0{,}3=P(AB)$.
\itemch
Xác suất để công ty thắng thầu dự án $2$ khi đã biết thắng thầu dự án $1$ là $P(B|A)$.\\
Ta có $P(B| A)=\dfrac{P(AB)}{P(A)}=\dfrac{0{,}3}{0{,}4}=0{,}75$.
\itemch
Xác suất để công ty thắng thầu dự án $2$ khi đã biết điều kiện không thắng thầu dự án $1$ là $P(B\setminus \overline{A})=\dfrac{P(\overline{A}B)}{P(\overline{A})}$.\\
Vì hai biến cố $\overline{A}B$ và $AB$ xung khắc và $\overline{A}B\cap AB=B$ nên theo tính chất của xác suất ta có
$P(\overline{A}B)=P(B)-P(AB)$. Suy ra
\begin{eqnarray*}
P(B|\overline{A})&=&\dfrac{P(\overline{A}B)}{P(\overline{A})}=\dfrac{P(B)-P(AB)}{1-P(A)}\\
&=&\dfrac{0{,}5-0{,}3}{1-0{,}4}=\dfrac{1}{3}.
\end{eqnarray*}
\itemch
Xác suất để công ty thắng thầu đúng $1$ dự án là $P(A\overline{B})+P(\overline{A}B)$.\\
Vì hai biến cố $\overline{A}B$ và $AB$ xung khắc và $\overline{A}B\cap AB=B$ nên theo tính chất của xác suất ta có
\[P(\overline{A}B)=P(B)-P(AB)\quad(1).\]
Vì hai biến cố $A\overline{B}$ và $AB$ xung khắc và $A\overline{B}\cap AB=A$ nên theo tính chất của xác suất ta có \[P(A\overline{B})=P(A)-P(AB)\quad(2).\]
Từ $(1)$ và $(2)$ ta có
\begin{eqnarray*}
P(A\overline{B})+P(\overline{A}B)&=&P(A)-P(AB)+P(B)-P(AB)\\
&=& P(A)+P(B)-2P(AB)\\
&=& 0{,}4+0{,}5-2\cdot 0{,}3=0{,}3.
\end{eqnarray*}
\end{itemchoice}
}
\end{ex}

\begin{ex}%[2D6V1-2]
Một công ty truyền thông đấu thầu $2$ dự án.Khả năng thắng thầu của dự án $1$ là $0{,}5$ và dự án $2$ là $0{,}6$.Khả năng thắng thầu của cả $2$ dự án là $0{,}4$.Gọi $A$,$B$ lần lượt là biến cố thắng thầu dự án $1$ và dự án $2$.Xét tính đúng sai của các mệnh đề sau
\choiceTF
{\True Xác suất $\mathrm{P}\left(\overline{A}\right)=0{,}5$ và $\mathrm{P}\left(\overline{B}\right)=0{,}4$}
{\True Xác suất công ty thắng thầu đúng $1$ dự án là $0{,}3$}
{Biết công ty thắng thầu dự án $1$,xác suất công ty thắng thầu dự án $2$ là $0{,}4$}
{Biết công ty không thắng thầu dự án $1$,xác suất công ty thắng thầu dự án $2$ là $0{,}8$}
\loigiai{
Ta có $\mathrm{P}(A)=0{,}5$,$\mathrm{P}(B)=0{,}6$,$\mathrm{P}\left(A \cap B\right)=0{,}4$.
\begin{itemchoice}
\itemch
Xác suất không thắng thầu dự án $1$ là $\mathrm{P}\left(\overline{A}\right)=1-\mathrm{P}(A)=1-0{,}5=0{,}5$.\\
Xác suất không thắng thầu dự án $2$ là $\mathrm{P}\left(\overline{B}\right)=1-\mathrm{P}(B)=1-0{,}6=0{,}4$.
\itemch
Biến cố công ty thắng thầu đúng $1$ dự án là $\left(A \cap \overline{B}\right) \cup \left(\overline{A} \cap B\right)$.\\
Xác suất thắng dự án $1$ mà không thắng dự án $2$ là \allowdisplaybreaks
\begin{eqnarray*}
\mathrm{P}\left(A \cap \overline{B}\right)=\mathrm{P}(A)-\mathrm{P}\left(A \cap B\right)=0{,}5-0{,}4=0{,}1.
\end{eqnarray*}
Xác suất không thắng dự án 1 mà thắng dự án $2$ là \allowdisplaybreaks
\begin{eqnarray*}
\mathrm{P}\left(\overline{A} \cap B\right)=\mathrm{P}(B)-\mathrm{P}\left(A \cap B\right)=0{,}6-0{,}4=0{,}2.
\end{eqnarray*}
Vì hai biến cố $(A \cap \overline{B})$ và $(\overline{A} \cap B)$ xung khắc nên xác suất thắng đúng $1$ dự án là  \allowdisplaybreaks
\begin{eqnarray*}
\mathrm{P}\left((A \cap \overline{B}\right) \cup (\overline{A} \cap B))=\mathrm{P}\left(A \cap \overline{B}\right)+\mathrm{P}\left(\overline{A} \cap B\right)=0{,}1+0{,}2=0{,}3.
\end{eqnarray*}

\itemch
Xác suất công ty thắng thầu dự án $2$ biết đã thắng thầu dự án $1$ là xác suất có điều kiện
\allowdisplaybreaks
\begin{eqnarray*}
\mathrm{P}\left(B \mid A\right)=\dfrac{\mathrm{P}\left(A \cap B\right)}{\mathrm{P}(A)}=\dfrac{0{,}4}{0{,}5}=\dfrac{4}{5}=0{,}8.
\end{eqnarray*}
\itemch
Xác suất công ty thắng thầu dự án $2$ biết đã không thắng thầu dự án $1$ là xác suất có điều kiện \allowdisplaybreaks
\begin{eqnarray*}
\mathrm{P}\left(B \mid \overline{A}\right)=\dfrac{\mathrm{P}\left(\overline{A} \cap B\right)}{\mathrm{P}\left(\overline{A}\right)}=\dfrac{0{,}2}{0{,}5}=\dfrac{2}{5}=0{,}4.
\end{eqnarray*}
\end{itemchoice}
}
\end{ex}

\begin{ex}%[2D4V3-5]%[Tex đề Moon 2025]%[Nguyễn Hồng Thạch]
Cho hai hình trụ có cùng bán kính bằng $3$ được đặt lồng vào nhau sao cho trục của hai hình trụ vuông góc với nhau và cắt nhau tại $O$ (hình 1). Gọi $(H)$ là phần giao của hai hình trụ (hình 2). Chọn trục $Ox$ vuông góc với hai trục của hình trụ như hình vẽ. Cắt khối $(H)$ bởi mặt phẳng vuông góc với trục $Ox$ tại điểm có hoành độ $x$ $(-3\le x\le 3)$, ta được thiết diện có diện tích là $S(x)$.
\begin{center}
\begin{tikzpicture}[scale=1,>=stealth, font=\footnotesize, line join=round, line cap=round]
\fill (0,0)node[above left]{$O$}circle(2pt);
\begin{scope}[rotate=-10]
\fill (4,0)circle(2pt);
\fill (-4,0)circle(2pt);
\draw[dashed] (-4,0)--(4,0);
\draw (-4,1) arc(90:270:0.4 cm and 1 cm);
\draw[dashed] (-4,1) arc(90:-90:0.4 cm and 1 cm);
\draw[dashed] (4,1) arc(90:270:0.4 cm and 1 cm);
\draw (4,1) arc(90:-90:0.4 cm and 1 cm);
\draw (-4,1)--(4,1) (-4,-1)--(4,-1);
\end{scope}
\begin{scope}[rotate=-30]
\fill (0,-3)circle(2pt);
\fill (0,3)circle(2pt);
\draw[dashed] (0,3)--(0,-3);
\draw (1,3) arc(0:180:1 cm and 0.4 cm);
\draw[dashed] (1,3) arc(360:180:1 cm and 0.4 cm);
\draw[dashed] (1,-3) arc(0:180:1 cm and 0.4 cm);
\draw (1,-3) arc(360:180:1 cm and 0.4 cm);
\draw (1,-3)--(1,3) (-1,-3)--(-1,3);
\end{scope}
\begin{scope}[rotate=25]
\draw[dashed] (0,0) ellipse (1.4 cm and 0.72 cm);
\draw[dashed] (0,0) ellipse (0.2 cm and 1.21 cm);
\end{scope}
\draw (0,-3.5)node[below]{Hình 1};
% \begin{scope}[xshift=8cm,scale=2]
% \draw[->] (0,-1)--(0,1.5)node[left]{$x$};
% \fill (0,0)node[left]{$O$}circle(1pt);
% \draw[rotate=25] (-15:1.4 cm and 0.72 cm) arc(-15:195:1.4 cm and 0.72 cm);
% \draw[rotate=25,dashed] (-15:1.4 cm and 0.72 cm) arc(-15:-165:1.4 cm and 0.72 cm);
% \draw[rotate=25] (0,-1.21) arc(-90:90:0.2 cm and 1.21 cm);
% \draw[dashed,rotate=25] (0,1.21) arc(90:270:0.2 cm and 1.21 cm);
% \draw[rotate=25] (0,1.21)--(32:1.4 cm and 0.72 cm) (0,1.21)--(148:1.4 cm and 0.72 cm);
% \draw[rotate=25] (-15:0.2 cm and 1.21 cm)--(22:1.4 cm and 0.72 cm) (-15:0.2 cm and 1.21 cm)--(158:1.4 cm and 0.72 cm);
% \draw[rotate=25] (-165:1.4 cm and 0.72 cm) parabola bend (0,-1.21) (-15:1.4 cm and 0.72 cm);
% \draw (0,-1.75)node[below]{Hình 2};
% \end{scope}
\end{tikzpicture}
\end{center}
Xét tính đúng sai của các mệnh đề sau
\choiceTF
{Hình khối $(H)$ là một khối tròn xoay}
{Công thức tính thể tích khối $(H)$ là $V=\pi\displaystyle\int\limits_{-3}^{3} S^2(x) \mathrm{\,d}x$}
{Diện tích $S(x)$ được xác định bởi công thức $S(x)=2\left(9-x^2\right)$}
{\True Thể tích của khối $(H)$ bằng $144$ (đvtt)}
\loigiai{
\begin{itemchoice}
\itemch  Khối $(H)$ là phần giao của hai hình trụ, không phải khối tròn xoay.
\itemch  Thể tích là $V = \displaystyle \int_{-3}^{3} S(x)\mathrm{\,d}x$.
\itemch  Thiết diện vuông góc trục $Ox$ là hình vuông cạnh $2\sqrt{9 - x^2}$ nên diện tích là
\[
S(x) = \left(2\sqrt{9 - x^2}\right)^2 = 4(9 - x^2).
\]
\itemch  Ta có
\[
S(x) = 4(9 - x^2) \Rightarrow V = \displaystyle\int_{-3}^{3} 4(9 - x^2)\mathrm{\,d}x = 4\displaystyle\int_{-3}^{3} (9 - x^2)\mathrm{\,d}x.
\]
Do hàm chẵn nên
\[
V = 4 \cdot 2\displaystyle\int_{0}^{3} (9 - x^2)\mathrm{\,d}x = 8 \left[ 9x - \dfrac{x^3}{3} \right]\Big|_0^3 = 8(27 - 9) = 144.
\]
\end{itemchoice}
}
\end{ex}

\begin{ex}%[2D4V3-5]
Một bình nhiên liệu trên cánh máy bay phản lực được mô hình hóa bằng cách quay hình phẳng giới hạn bởi đồ thị $y=f(x)=\dfrac{3}{5}x^2\sqrt{2-ax}$ $(a\in\mathbb{R})$ và trục $Ox$ quanh trục hoành, trong đó $x$ và $y$ được đo bằng mét (như hình vẽ). Biết rằng chiếc máy bay đó có $4$ bình chứa nhiên liệu như nhau và được đổ đầy trước khi bay. Giả sử tốc độ tiêu hao nhiên liệu trên máy bay được mô phỏng bằng hàm số $h'(t)=-3t^2+120t+2000$ lít/giờ ($t$ tính theo giờ, $0\le t\le 6$).
\begin{center}
\begin{tikzpicture}[scale=1,>=stealth, font=\footnotesize, line join=round, line cap=round]
\def\xmin{-1} \def\xmax{3}
\def\ymin{-1} \def\ymax{2}
\draw[->] (\xmin,0)--(\xmax,0) node [below]{$x$};
\draw[->] (0,\ymin)--(0,\ymax) node [left]{$y$};
\node at (0,0) [below left]{$O$};
\draw (1.5,1.3)node[]{$y=\dfrac{3}{5}x^2\sqrt{2-ax}$};
\clip (\xmin+0.1,\ymin+0.1) rectangle (\xmax-0.5,\ymax-0.1);
\draw[smooth,samples=300,domain=0:2] plot(\x,{3/5*(\x)^2*(2-\x)});
\end{tikzpicture}
\end{center}

\choiceTF
{Giá trị $a=2$}
{Thể tích của nhiên liệu (lít) trên mỗi cánh máy bay được xác định bởi công thức $V=\pi\displaystyle\int\limits_{0}^{2} f^2(x) \mathrm{\,d}x$}
{\True Máy bay đó có thể chứa tối đa $9650$ lít nhiên liệu (làm tròn đến hàng đơn vị)}
{\True Máy bay đó tiêu hao hết $90\%$ năng lượng sau $3{,}91$ giờ (làm tròn đến hàng phần trăm)}
\loigiai{
\begin{itemchoice}
\itemch Ta có \begin{eqnarray*}
y(2)=0 & \Leftrightarrow& \dfrac{3}{5} \cdot 2^2 \sqrt{2-2a}=0 \\
& \Leftrightarrow& \sqrt{2-2a}=0 \Leftrightarrow 2-2 a=0 \\
& \Leftrightarrow& a=1 .
\end{eqnarray*}
\itemch Thể tích của nhiên liệu trên mỗi bình nhiên liệu được xác định bởi công thức $V=\pi \displaystyle\int\limits_0^2 f^2(x) \mathrm{d} x ~\left(\mathrm{m}^3\right)$.
\itemch Thể tích của nhiên liệu trên mỗi bình nhiêu liệu bằng
\begin{eqnarray*}
V_{1 b}&=&\pi \displaystyle\int\limits_0^2 f^2(x) \mathrm{d} x \\
& =&\pi \displaystyle\int\limits_0^2\left(\dfrac{3}{5} x^2 \sqrt{2-x}\right)^2 \mathrm{d} x \\
& =&\dfrac{96 \pi}{125}\left(\mathrm{m}^3\right) .
\end{eqnarray*}
Suy ra thể tích của 4 bình nhiên liệu bằng
\[V_{4 b}=4 \cdot \dfrac{96 \pi}{125}=\dfrac{384 \pi}{125}\left(\mathrm{m}^3\right)=3072 \pi~(\mathrm{l}) \approx 9651~(\mathrm{l}) .\]
\itemch Tốc độ tiêu hao nhiên liệu trên máy bay được mô phỏng bởi hàm số\\
$h'(t)=-3 t^2+120 t+2000$ (lít/giờ).\\
$V_{t t}=0{,}9 \cdot V_{4 b}=0{,}9 \cdot 3072 \pi=\dfrac{13824}{5} \pi$ (l) .\\
Gọi $m$ là thời gian máy bay đó sẽ tiêu hao hết $90$ năng lượng, khi đó
\begin{eqnarray*}
\displaystyle\int\limits_0^m h'(t) \mathrm{d}t=\dfrac{13824 \pi}{5}
& \Leftrightarrow& \displaystyle\int\limits_0^m\left(-3 t^2+120 t+2000\right) \mathrm{d}t=\dfrac{13824 \pi}{5} \\
& \Leftrightarrow&-t^3+60 t^2+\left.2000 t\right|_0 ^m=\dfrac{13824 \pi}{5} \\
& \Leftrightarrow&-m^3+60 m^2+2000 m=\dfrac{13824 \pi}{5} \\
& \Leftrightarrow&\hoac{
&m \approx 82{,}87 \\
&m \approx 3{,}91 \\
&m \approx-26{,}78
} \\
& \Leftrightarrow& m \approx 3{,}91 .
\end{eqnarray*}
Vậy máy bay đó sẽ tiêu hao hết $90$ năng lượng sau $3{,}91$ giờ bay.
\end{itemchoice}
}
\end{ex}

\begin{ex}%[2D4V3-3]%[TEX ĐỀ MOON 2025]%[Lê Hữu Kiệt]
Trong mặt phẳng tọa độ $Oxy$, cho hàm số $f(x)=x^2-x-6$ có đồ thị $(C)$.
\choiceTF
{\True Thể tích của vật thể tròn xoay được sinh ra khi hình phẳng giới hạn bởi đồ thị $(C)$ và trục hoành quay quanh $Ox$ là $V=\pi\displaystyle\int\limits_{-2}^{3} \left(x^2-x-6\right)^2 \mathrm{d}x$}
{Diện tích hình phẳng giới hạn bởi đồ thị $(C)$ và trục hoành là $S=\displaystyle\int\limits_{-2}^{3} \left(x^2-x-6\right) \mathrm{d}x$}
{Giả sử một vật $M$ chuyển động dọc theo một đường thẳng sao cho vận tốc của nó tại thời điểm $x$ (giây) là $f(x)=x^2-x-6$ (m/s). Khi đó độ dịch chuyển của vật $M$ trong khoảng thời gian $x\in[1;4]$ là $\dfrac{9}{2}$}
{\True Tổng quãng đường của vật $M$ ở trên đi được trong khoảng thời gian $x\in[1;4]$ là $\dfrac{61}{6}$ (m)}
\loigiai{
\begin{itemchoice}
\itemch Ta có $f(x)=0\Leftrightarrow x^2-x-6=0 \Leftrightarrow \hoac{&x=3\\&x=-2.}$\\
Khi đó thể tích của vật thể tròn xoay được sinh ra khi hình phẳng giới hạn bởi đồ thị $(C)$ và trục hoành quay quanh $Ox$ là $V=\pi\displaystyle\int\limits_{-2}^{3} \left(x^2-x-6\right)^2 \mathrm{d}x$.
\itemch Diện tích hình phẳng giới hạn bởi đồ thị $(C)$ và trục hoành là
\[S=\displaystyle\int\limits_{-2}^{3} \left|x^2-x-6\right|\mathrm{d}x=\displaystyle\int\limits_{-2}^{3}\left(-x^2+x+6\right)\mathrm{d}x.\]
\itemch Với $x=1$ ta có $f(1)=-6$, ta được điểm $A(1;-6)$.\\
Với $x=4$ ta có $f(4)=6$, ta được điểm $B(4;6)$.\\
Khi đó độ dịch chuyển của chất điểm $M$ trong khoảng thời gian $x\in[1;4]$ là
\[AB=\sqrt{(4-1)^2+(6+6)^2}=3\sqrt{17}.\]
\itemch Tổng quãng đường của một vật $M$ trong khoảng thời gian $x\in[1;4]$ là
\[\int\limits_1^4 \left|x^2-x-6\right|\mathrm{d}x=-\int\limits_1^3\left(x^2-x-6\right)\mathrm{d}x+\int\limits_3^4\left(x^2-x-6\right)\mathrm{d}x=\dfrac{61}{6}.\]
\end{itemchoice}
}
\end{ex}

\begin{ex}%[2D4V3-2]%[TEX Đề Moon 2025]%[Vũ Hồng Toàn]
Cho hàm số $y=f(x)$ có $f'(x)=2\cos^2\dfrac{x}{2}+3$, $\forall x\in \mathbb{R}$. Các mệnh đề sau đúng hay \textbf{sai}?
\choiceTF
{\True Hàm số $y=f(x)$ có dạng $f(x)=\sin x+4x+C$ với $C$ là hằng số}
{\True $\displaystyle\int\limits_{0}^{\tfrac{\pi}{2}} f(x)\mathrm{\,d}x=F\left(\dfrac{\pi}{2}\right)-F(0)$ với $F(x)$ là một nguyên hàm của $f(x)$}
{\True Nếu $f(0)=4$ thì $f\left(\dfrac{\pi}{2}\right)=2\pi+5$}
{Diện tích hình phẳng giới hạn bởi đồ thị của hai hàm số $y=f'(x)$; $y=6$ và hai đường thẳng $x=0$, $x=\dfrac{\pi}{2}$ có dạng $S=a+b\pi$ thì $a+2b=-1$}
\loigiai{
\begin{itemchoice}
\itemch $\forall x\in \mathbb{R}$ ta có $f'(x)=2\cos^2\dfrac{x}{2}+3=\left(2\cos^2\dfrac{x}{2}-1\right)+4=\cos x+4$.\\
Do đó $f(x)=\sin x+4x+C$.
\itemch Với $F(x)$ là một nguyên hàm của $f(x)$ khi đó $\displaystyle\int\limits_{0}^{\tfrac{\pi}{2}} f(x)\mathrm{\,d}x=F(x)\bigg |_0^{\tfrac{\pi}{2}}=F\left(\dfrac{\pi}{2}\right)-F(0)$.
\itemch Do $f(x)=\sin x+4x+C$. Nếu $f(0)=4\Rightarrow C=4$ thì $f(x)=\sin x+4x+4$.\\
Vậy $f\left(\dfrac{\pi}{2}\right)=\sin\dfrac{\pi}{2}+4\cdot \dfrac{\pi}{2}+4=2\pi+5$.
\itemch Ta có\\ $S=\displaystyle\int\limits_{0}^{\tfrac{\pi}{2}} \big|\cos x+4-6\big|\mathrm{\,d}x=\displaystyle\int\limits_{0}^{\tfrac{\pi}{2}} \big(2-\cos x\big)\mathrm{\,d}x=\left(2x-\sin x\right)\bigg|_0^{\tfrac{\pi}{2}}=\pi -1$.\\
Suy ra $a=-1$ và $b=1$. Vậy $a+2b=-1+2\cdot 1=1$.
\end{itemchoice}
}
\end{ex}

\begin{ex}%[2D4V2-6]
Hệ thống lọc nước bể bơi vô cùng quan trọng để nguồn nước được làm sạch thường xuyên và giữ vệ sinh cho người bơi. Trong quá trình vận hành lọc nước thì lượng nước trong bể sẽ thay đổi theo thời gian. Lượng nước trong bể giảm nếu hệ thống đang xả nước bẩn ra khỏi bể và tăng nếu hệ thống đang cấp thêm nước sạch cho bể. Biết rằng $1$ gallon gần bằng $3{,}785$ lít, dung tích của bể là $1000$ gallon và thời điểm $6$ giờ sáng bể chứa $250$ gallon nước. Hàm số $f(t)$ liên tục trên đoạn [$0;12$] biểu thị cho tốc độ thay đổi lượng nước trong bể theo thời gian $t$ giờ, từ thời điểm $6$ giờ sáng đến $6$ giờ chiều được cho bởi hàm số $
f (t)=\heva{&100 t,&&0 \le t \le 3\\
&-200 t+a,&&3 \le t \le 6\\
&100 t-900, && \le t \le 12
},(a \in\mathbb{R})$.\\
Với mốc thời gian $t=0$ tại thời điểm $6$ giờ sáng.
\choiceTF
{ Tại thời điểm $9$ giờ sáng, nước trong bể đang tăng với tốc độ $600$ gallon/giờ}
{\True Giá trị của $a=900$}
{Tốc độ thay đổi Iượng nước trong bể bằng $0$ vào lúc $11$ giờ trưa và $15$ giờ chiều}
{\True Lượng nước trong bể lớn nhất trong khoảng thời gian từ $9$ giờ sáng đến $18$ giờ chiều là $700$ gallon nước}
\loigiai{
\begin{itemchoice}
\itemch Do $t=0$ tại thời điểm $6$ giờ sáng nên tại thời điểm $9$ giờ sáng thì $t=3$.\\
Ta có $f(3)=100\cdot 3=300$ gallon/giờ.
\itemch Với $t=3$, $100t=-200t+a\Leftrightarrow-600+a=300\Leftrightarrow a=900$.
\itemch Tại $11$ giờ trưa (tương đương $t=5$), tốc độ thay đổi lượng nước trong bể là $f(5)=-200\cdot 5+900=-100$ gallon/giờ.\\
Tại $15$ giờ chiều (tương đương $t=9$), tốc độ thay đổi lượng nước trong bể là $f(9)=100\cdot 9-900=0$ gallon/giờ.
\itemch Từ $9$ giờ sáng đến $18$ giờ chiều tức là $3\le t\le 12$.\\
Từ $3\le t\le 9$, lượng nước đang giảm về $0$.\\
Lượng nước trong bể bắt đầu tăng trở lại từ $9\le t\le 12$.\\
Lượng nước trong bể từ $15$ giờ chiều đến $18$ giờ chiều là
\[250+\displaystyle\int\limits_9^{12}(100 t-900)\mathrm{\, d}t=700 .\]
\end{itemchoice}
}
\end{ex}

\begin{ex}%[2D4V2-6]%[TEX ĐỀ MOON 2025]%[Nguyễn Văn Hiệp]
Một vật được ném lên từ độ cao $300$ m với vận tốc cho bởi công thức $v(t)=-9{,}81t+29{,}43$ (m/s). Gọi $h(t)$ (m) là độ cao của vât so với mặt đất tại thời điểm $t$ (s) tính từ lúc bắt đầu ném vật. Xét tính đúng sai của các mệnh đề sau
\choiceTF
{\True Vận tốc của vật triệt tiêu tại thời điểm $t=3$ giây}
{Hàm số $h(t)=-4{,}985t^2+29{,}43t$}
{\True Vật đạt độ cao lớn nhất là $344$ (m) (làm tròn đến hàng đơn vị)}
{\True Sau $11$ giây tính từ lúc ném thì vật đó chạm đất (làm tròn đến hàng đơn vị)}
\loigiai
{
\begin{itemchoice}
\itemch  $v(t)=-9{,}81t+29{,}43=0\Rightarrow t=3$ m/s
\itemch  $h(t) = 300+\displaystyle \int\limits_{0}^t\left(-9{,}81z+29{,}43\right)\mathrm{\,d}z=-4,905t^2 + 29,43t + 300$.
\itemch Vật đạt độ cao lớn nhất khi $t =-\dfrac{b}{2a}=-\dfrac{29{,}43}{2\cdot \left(-4{,}905\right)} = 3$ s; $h_{\text{max}}=h(3) \approx 344$ m.
\itemch Vật chạm đất khi $h(t)=0\Leftrightarrow -4,905t^2 + 29,43t + 300 = 0$ giải được $t \approx 11$ s.
\end{itemchoice}
}
\end{ex}

\begin{ex}%[2D4V2-6]
Một ô tô đang chạy với tốc độ $108$ km/h thì người lái xe bất ngờ phát hiện chướng ngại vật trên đường. Người lái xe phản ứng một giây sau đó bằng cách đạp phanh khẩn cấp. Kể từ thời điểm này, ô tô chuyển động chậm dần đều với tốc độ $v(t)=-10t+30$ (m/s), trong đó $t$ là thời gian tính bằng giây kể từ lúc đạp phanh. Gọi $s(t)$ là quãng đường xe ô tô đi được trong $t$ (s) kể từ lúc đạp phanh. Xét tính đúng sai của các mệnh đề sau:
\choiceTF
{\True Công thức biểu diễn hàm số $s(t)=-5t^2+30t$ (m)}
{Thời gian kể từ lúc đạp phanh đến khi xe ô tô dừng hẳn là $6$ giây}
{\True Sau $3$ giây kể từ lúc đạp phanh,quãng đường xe ô tô di chuyển được là $45$ (m)}
{Quãng đường xe ô tô đã di chuyển kể từ lúc người lái xe phát hiện chướng ngại vật trên đường đến khi xe ô tô dừng hẳn là $120$ (m)}
\loigiai{
Đổi đơn vị: $108$ km/h $= \dfrac{108 \cdot 1000}{3600}$ m/s $= 30$ m/s.\\
Vận tốc của xe tại thời điểm bắt đầu đạp phanh ($t=0$) là $v(0) = -10(0) + 30 = 30$ (m/s).
\begin{itemchoice}
\itemch
Quãng đường $s(t)$ xe đi được kể từ lúc đạp phanh ($t=0$) là nguyên hàm của vận tốc $v(t)$.\\
$s(t) = \displaystyle\int v(t) \, \mathrm{d}t = \displaystyle\int (-10t+30) \, \mathrm{d}t = -10 \dfrac{t^2}{2} + 30t + C = -5t^2 + 30t + C$.\\
Tại $t=0$ (lúc bắt đầu đạp phanh), quãng đường đi được kể từ lúc đó là $s(0)=0$. \\
Thay $t=0$ vào biểu thức $s(t)$, ta có $s(0) = -5(0)^2 + 30(0) + C = 0 \Rightarrow C = 0$. \\
Vậy $s(t) = -5t^2 + 30t$ (m).
\itemch
Xe dừng hẳn khi vận tốc $v(t) = 0$. \\
$-10t + 30 = 0 \Leftrightarrow 10t = 30 \Leftrightarrow t = 3$ (s). \\
Vậy thời gian kể từ lúc đạp phanh đến khi xe dừng hẳn là $3$ giây.
\itemch
Quãng đường xe ô tô di chuyển được sau $3$ giây kể từ lúc đạp phanh là $s(3)$.\\
$s(3) = -5(3)^2 + 30(3) = -5 \cdot 9 + 90 = -45 + 90 = 45$ (m).
\itemch
Quá trình di chuyển gồm $2$ giai đoạn:
\begin{itemize}
\item Giai đoạn 1: Phản ứng (1 giây). Xe chạy với tốc độ không đổi $30$ m/s.
Quãng đường $s_1 = 30 \cdot 1 = 30$ (m).
\item Giai đoạn 2: Đạp phanh đến khi dừng hẳn (từ $t=0$ đến $t=3$ giây).
Quãng đường $s_2 = s(3) = 45$ (m) (tính ở câu trên).
\end{itemize}
Tổng quãng đường từ lúc phát hiện chướng ngại vật đến khi dừng hẳn là: \\
$S = s_1 + s_2 = 30 + 45 = 75$ (m).
\end{itemchoice}
}
\end{ex}

\begin{ex}%[2D4V1-6]%[TEX ĐỀ MOON 2025]%[Huỳnh Thanh Chí]
Một ô tô bắt đầu chuyển động thẳng nhanh dần đều với tốc độ $v(t)=5t$ (m/s); trong đó $t$ là thời gian tính bằng giây kể từ khi ô tô bắt đầu chuyển động. Đi được $6$ (s) người lái xe phát hiện chướng ngại vật và phanh gấp, ô tô tiếp tục chuyển động chậm dần đều với gia tốc $a=-5$ (m/s$^2$). Xét tính đúng sai của các mệnh đề sau
\choiceTF
{\True Tốc độ của ô tô tại thời điểm $10$ (s) tính từ lúc xuất phát là $10$ (m/s)}
{Quãng đường ô tô chuyển động được trong $6$ giây đầu tiên là $80$ m}
{\True Quãng đường $s$ (đơn vị: mét) mà ô tô chuyển động được kể từ lúc bắt đầu đạp phanh đến khi dừng lại được tính theo công thức $s=\displaystyle\int\limits_{0}^{6} \left(30-5t\right) \mathrm{\,d}t$}
{Quãng đường ô tô chuyển động được kể từ lúc bắt đầu chuyển động cho đến khi dừng lại là $170$ m}
\loigiai{
Gọi $v_1(t)$ là tốc độ chuyển động chậm dần đều của ô tô khi đạp phanh.\\
Ta có $v_1(t)=\int a\mathrm{\,d}t=\int -5\mathrm{\,d}t=-5t+C$.\\
Ta có vận tốc lúc ô tô đạp phanh là $v_1(0)=v(6)=5\cdot 6=30$ m/s.\\
Suy ra tốc độ chuyển động chậm dần đều khi đạp phanh là $v_1(t)=30-5t$ (m/s).
\begin{itemchoice}
\itemch Khi ô tô di chuyển được $6$ (s) thì bắt đầu đạp phanh và tiếp tục chuyển động với tốc độ $v_1(t)$ (m/s).\\
Do đó, tốc độ của ô tô tại thời điểm $10$ (s) tính từ lúc xuất phát bằng tốc độ ô tô chuyển động tại thời điểm $4$ (s) với tốc độ $v_1(t)$ (m/s).\\
Vậy $v_1(4)=30-5\cdot 4=10$ m/s.
\itemch Quãng đường ô tô chuyển động được trong $6$ giây đầu tiên là $S_1=\displaystyle\int\limits_{0}^{6} v(t)\mathrm{\,d}t=90$ m.
\itemch Quãng đường $s$ (đơn vị: mét) mà ô tô chuyển động được kể từ lúc bắt đầu đạp phanh đến khi dừng lại được tính theo công thức $s=\displaystyle\int\limits_{0}^{6} \left(30-5t\right) \mathrm{\,d}t$.
\itemch Khi ô tô dừng lại thì ta có phương trình $v_1(t)=0\Leftrightarrow t=6$ (s).\\
Quãng đường ô tô chuyển động được kể từ lúc bắt đầu chuyển động cho đến khi dừng lại là $S=S_1+S_2=30+\int\limits_{0}^{6} v_1(t)\mathrm{\,d}t=90+90=180$ m.
\end{itemchoice}
}
\end{ex}

\begin{ex}%[2D4V1-6]
Một viên muối hình cầu có đường kính $8$ cm đang tan trong nước với tốc độ giảm thể tích tại bất kỳ thời điểm nào tỷ lệ thuận với diện tích bề mặt quả cầu tại thời điểm đó. Sau $30$ giây thì viên muối tan được một nửa. Gọi $V(t)$ và $r(t)$ lần lượt là thể tích và bán kính của viên muối sau $t$ phút. Xét tính đúng sai của các mệnh đề sau
\choiceTF
{\True Thể tích của viên muối sau $t$ phút được xác định bởi công thức $V(t)=\dfrac{4}{3}\pi r^3(t)$}
{\True Tốc độ giảm thể tích của viên muối là $V'(t)=k\pi r^2(t)$ với $k$ là hằng số}
{Giá trị của $k=-4$}
{Sau $45$ giây thì thể tích của viên muối còn lại là $4{,}2$ cm$^3$ (làm tròn đến hàng phần mười)}
\loigiai{
\begin{itemchoice}
\itemch
Thể tích của viên muối sau thời gian $t$ phút là $V(t)=\dfrac{4}{3}\pi r^3(t)$.
\itemch
Ta có $r(0)=4$ cm.\\
$V(0)=\dfrac{4}{3}\pi r^3(0)=\dfrac{4}{3}\pi\cdot4^3=\dfrac{256\pi}{3}$.
\itemch
Ta có $V(t)=\dfrac{4}{3} \pi r^3(t) \Rightarrow V^{\prime}(t)=\dfrac{4}{3} \pi \cdot 3 r^2(t) \cdot r^{\prime}(t)=4 \pi \cdot r^2(t) \cdot r^{\prime}(t)$.\\
Ta lại có $V'(t)=k\pi\cdot r^2(t)$. Suy ra $r'(t)=\dfrac{k}{4}\Rightarrow r(t)=\dfrac{kt}{4}+C$.\\
Mặt khác $r(0)=4\Rightarrow C=4\Rightarrow \dfrac{kt}{4}+4$.
\allowdisplaybreaks
\begin{eqnarray*}
&&V\left(\dfrac{1}{2}\right)=\dfrac{228\pi}{3}\\
&\Leftrightarrow&
\dfrac{4}{3} \pi r^3\left(\dfrac{1}{2}\right)=\dfrac{128 \pi}{3} \\
& \Leftrightarrow& r^3\left(\dfrac{1}{2}\right)=32 \\
& \Leftrightarrow& r\left(\dfrac{1}{2}\right)=\sqrt[3]{32}\\
&\Leftrightarrow& r\left(\dfrac{1}{2}\right)=\dfrac{k \cdot \frac{1}{2}}{4}+4=\sqrt[3]{32}.
\end{eqnarray*}
Suy ra $\dfrac{k}{8}=\sqrt[3]{32}-4
\Rightarrow k=8 \sqrt[3]{32}-32 \approx-6{,}6$.
\itemch
Ta có $r\left(\dfrac{3}{4}\right)=\dfrac{\left(8\sqrt[3]{32}-32\right)\cdot\frac{3}{4}}{4}+4=\dfrac{\left(24\sqrt[3]{32}-96\right)+64}{16}$.\\
$V\left(\dfrac{3}{4}\right)=\dfrac{4}{3}\pi r^3 \left(\dfrac{3}{4}\right)=\dfrac{4}{3}\left(\dfrac{\left(24\sqrt[3]{32}-96\right)+64}{16}\right)^3\approx88{,}28$ (cm$^3$).
\end{itemchoice}
}
\end{ex}

\begin{ex}%[2D4V1-6]
Một miếng thịt sống được lấy ra khỏi ngăn đá của tủ lạnh và để trên bàn để rã đông. Nhiệt độ của miếng thị khi nó được lấy ra khỏi ngăn đá là $-4^\circ$C và sau $t$ giờ thì nhiệt độ của miếng thịt tăng với tốc độ $T'(t)=7{\mathrm{e}^{-0{,}35t}}^\circ$C/giờ. Miếng thịt này được rã đông khi nhiệt độ của nó đạt đến $10^\circ$C. Xét tính đúng sai của các mệnh đề sau
\choiceTF
{\True Sau $2$ giờ tốc độ thay đổi nhiệt độ của miếng thịt bằng $3{,}48^\circ$C/giờ (làm tròn kết quả đến hàng phần trăm)}
{Nhiệt độ của miếng thị bằng $0^\circ$C sau $43$ phút (làm tròn kết quả đến hàng đơn vị của phút)}
{Cần mất $2{,}44$ giờ để miếng thịt được rã đông (làm tròn kết quả đến hàng phần trăm của giờ)}
{Sau khi rã đông được $2$ tiếng, miếng thịt được đem đi nướng trong lò nướng. Tốc độ thay đổi nhiệt độ của miếng thịt trong lò nướng sau $t$ giờ được xác định bởi hàm số\break $L'(t)=80{\mathrm{e}^{0{,}2t}}^\circ$C/giờ. Miếng thịt được coi là chín đều nếu nhiệt độ của nó là $77^\circ$C. Thời gian để nướng chín đều miếng thị là $48$ phút (làm tròn kết quả đến hàng đơn vị của phút)}
\loigiai{
\begin{itemchoice}
\itemch
Tốc độ thay đổi nhiệt sau $2$ giờ là $T'(2)=7{\mathrm{e}^{-0{,}35\cdot2}}\approx 3{,}48^\circ$ C/h.
\itemch Nhiệt độ của miếng thịt sau $t$ giờ là
\allowdisplaybreaks
\begin{eqnarray*}
T(t)&=&\displaystyle\int T^{\prime}(t) \mathrm{\,d}t\\
&=&\displaystyle\int 7{\mathrm{e}^{-0{,}35t}} \mathrm{\,d}t\\
&=&\dfrac{-7}{0{,}35} e^{-0{,}35 t}+C \\
&=&-20\mathrm{e}^{-0{,}35 t}+C.
\end{eqnarray*}
Nhiệt độ khi lấy miếng thịt ra khỏi ngăn đá	\\
$T(0)=-4\Leftrightarrow-20 \mathrm{e}^{-0{,}35.0}+C=-4\Leftrightarrow C=16\Rightarrow T(t)=-20 \mathrm{e}^{-0{,}35 t}+16.$\\
Ta có
\allowdisplaybreaks
\begin{eqnarray*}
T=0&\Leftrightarrow&-20 \mathrm{e}^{-0{,}35 t}+16=0\\
&\Leftrightarrow&\mathrm{e}^{-0{,}35 t}=\dfrac{16}{20}=\dfrac{4}{5}\\
&\Leftrightarrow&-0{,}35 t=\ln \dfrac{4}{5} \\
&\Leftrightarrow&t=-\dfrac{1}{0{,}35} \ln \dfrac{4}{5} \approx 0,64 \text{h} \approx 38\text{p}.
\end{eqnarray*}
Vậy nhiệt độ miếng thịt bằng $0^\circ$C sau $38$ phút.
\itemch Ta có
\allowdisplaybreaks
\begin{eqnarray*}
T=10&\Leftrightarrow&-20 \mathrm{e}^{-0{,}35 t}+16=10\\
&\Leftrightarrow&\mathrm{e}^{-0{,}35 t}=\dfrac{6}{20}=\dfrac{3}{10}\\
&\Leftrightarrow&-0{,}35 t=\ln \dfrac{3}{10} \\
&\Leftrightarrow&t=-\dfrac{1}{0{,}35} \ln \dfrac{3}{10} \approx 3{,}44 \text{h}.
\end{eqnarray*}
Vậy cần $3{,}44$ giờ thì miếng thịt được rã đông.
\itemch Nhiệt độ miếng thịt sau $t$ giờ đưa vào lò vi sóng.\\
$L(t)=\displaystyle\int L^{\prime}(t) \mathrm{\,d}t=\displaystyle\int 80\mathrm{e}^{0{,}2t} \mathrm{\,d}t =400 \mathrm{e}^{0{,}2t}+C$.\\
Thời điểm miếng thịt được đưa vào lò vi sóng là
\[t_1=t+2=3{,}44+2=5{,}44\,(h).\]
Nhiệt độ miếng thịt lúc đưa vào lò vi sóng là
\[L(0)-T(5{,}44)=-20\mathrm{e}^{-0{,}35\cdot5{,}44}+16\approx13{,}0205.\]
Ta có $L(0)=13{,}0205\Leftrightarrow 400\mathrm{e}^{0{,}2\cdot0}+C=13{,}0205\Leftrightarrow C=-386{,}9795\Rightarrow L(t)=400 \mathrm{e}^{0{,}2t}-386{,}9795$.
\allowdisplaybreaks
\begin{eqnarray*}
L(t)=77&\Leftrightarrow&400 \mathrm{e}^{0{,}2t}-386{,}9795=77\\
&\Leftrightarrow&400 \mathrm{e}^{0{,}2t}=463{,}9795\\
&\Leftrightarrow& \mathrm{e}^{0{,}2t}=\dfrac{463{,}9795}{400}\\
&\Leftrightarrow& t=\dfrac{1}{0{,}2}\ln\dfrac{463{,}9795}{400}\approx0{,}74h\approx45(p).
\end{eqnarray*}
\end{itemchoice}
}
\end{ex}

\begin{ex}%[2D1V5-8]
Hai nhà máy được đặt tại các vị trí $A$ và $B$ cách nhau $8$ km. Nhà máy xử lí nước thải được đặt ở vị trí $C$ trên đường trung trực của đoạn thẳng $AB$, cách trung điểm $M$ của đoạn thẳng $AB$ một khoảng là $3$ km. Người ta muốn làm đường ống dẫn nước thải từ hai nhà máy $A$, $B$ đến nhà máy xử lí nước thải $C$ gồm các đoạn thẳng $AI$, $BI$ và $IC$, với $I$ là vị trí nằm giữa $M$ và $C$. Đặt $I M=x$ km (với $0<x<3$).
\begin{center}
\begin{tikzpicture}[>=stealth,line cap=round,line join=round]
\path(-4,0)node[left](A){$A$}
++(4,0)node[below](M){$ M$} ++(4,0)node[right](B){$ B $}
(M)++(0,2.2)node[above left](I){$I$}++(0,2.2)node[above](C){$ C $}
;
\draw(-4,0)--(4,0) (0,0)--(0,4) (-4,0)--(0,2)(4,0)--(0,2);
\end{tikzpicture}
\end{center}
\choiceTF
{$I A=I B=\sqrt{x^2+9}$ (km)}
{ Tổng độ dài đường ống được biểu diễn qua hàm số $f(x)=2 \sqrt{x^2+9}+3-x$ (km)}
{\True Tổng độ dài đường ống nhỏ nhất bằng $9,9$ (km) (làm tròn kết quả đến hàng phần chục)}
{\True Khi tổng độ dài đường ống nhỏ nhất thì góc $\widehat{AIB}=120^{\circ}$}
\loigiai{
\begin{itemchoice}
\itemch {\bf Sai.}\\
Ta có $AM=MB=\dfrac{AB}{2}=4$ (km).\\
Suy ra $IA=IB=\sqrt{x^2+16}$ (km).
\itemch {\bf Sai.}\\
Độ dài đường ống nước là $IA+IB+IC=2\sqrt{x^2+16}+3-x$.\\
Vì vậy $f(x)=2\sqrt{x^2+16}+3-x$ (km).
\itemch {\bf Đúng.}\\
Xét hàm số $f(x)=2\sqrt{x^2+16}+3-x$ với $0<x<3$.\\
Ta có $f'(x)=\dfrac{2x}{\sqrt{x^2+16}}-1=0\Leftrightarrow \sqrt{x^2+16}=2x\Rightarrow x=\dfrac{4\sqrt{3}}{3}$.\\
Bảng biến thiên
\begin{center}
\begin{tikzpicture}
\tkzTabInit[lgt=1.2,espcl=4]
{$x$/1,$f’(x)$/1,$f(x)$/2.5}
{$0$,$\dfrac{4\sqrt{3}}{3}$,$3$}
\tkzTabLine{ ,-,z,+, }
\tkzTabVar{+/$11$,-/$3+4\sqrt{3}$,+/$10$}
\end{tikzpicture}
\end{center}
Suy ra $\min\limits_{(0; 3)}f(x) =f\left(\dfrac{4\sqrt{3}}{3}\right)=3+4\sqrt{3}\approx 9{,}9$ (km).
\itemch {\bf Đúng.}\\
Ta có $\tan \widehat{AIM}=\dfrac{AM}{IM}=\dfrac{4}{\tfrac{4\sqrt{3}}{3}}=\sqrt{3}\Rightarrow \widehat{AIM}=60^\circ\Rightarrow \widehat{AIB}=120^\circ$.
\end{itemchoice}
}
\end{ex}

\begin{ex}%[2D1V5-4]%[TEX Đề Moon 2025]%[Vũ Hồng Toàn]
\immini[thm]
{
Cho hàm số $f(x)=\dfrac{ax^2+bx+c}{x+n}$ có đồ thị $(C)$ như hình vẽ bên. Xét tính đúng sai của các mệnh đề sau
\choiceTF
{\True Hàm số đã cho đồng biến trên khoảng $(0;1)$}
{\True Hàm số đã cho có hai điểm cực trị}
{Đồ thị $(C)$ có tiệm cận xiên đi qua điểm $A(-1;2)$}
{\True Phương trình $x\cdot\left|f(x)\right|=x-4$ có đúng $3$ nghiệm thực phân biệt}
}
{
\begin{tikzpicture}[scale=0.55,>=stealth, font=\footnotesize, line join=round, line cap=round]
\def\a{1} \def\b{2} \def\c{2} \def\d{1}\def\e{1} % Hệ số
\def\xmin{-6} \def\xmax{4}
\def\ymin{-6} \def\ymax{6}
\draw[->] (\xmin,0)--(\xmax,0) node [below]{$x$};
\draw[->] (0,\ymin)--(0,\ymax) node [right]{$y$};
\node at (0,0) [below right]{$O$};
\clip (\xmin+0.1,\ymin+0.1) rectangle (\xmax-0.1,\ymax-0.1);
\draw[smooth,samples=300,domain=\xmin:(-\e/\d-0.1)] plot(\x,{(\a*(\x)^2+\b*(\x)+\c)/(\d*(\x)+\e)});
\draw[smooth,samples=300,domain=(-\e/\d+0.1:\xmax)] plot(\x,{(\a*(\x)^2+\b*(\x)+\c)/(\d*(\x)+\e)});
\draw (-\e/\d,\ymin)--(-\e/\d,\ymax);
\draw[smooth,samples=300,domain=\xmin:\xmax] plot(\x,{(\a/\d)*(\x)+((\b)*(\d)-(\a)*(\e))/((\d)^2)});
\draw[dashed] (0,2)node[below left]{$2$} (-2,0)node[above,xshift=-0.15cm]{$-2$}--(-2,-2)--(0,-2)node[right]{$-2$} (-1,0)node[below right,xshift=-0.15cm]{$-1$};
\end{tikzpicture}
}
\loigiai{
Quan sát đồ thị hàm số $f(x)$ ta thấy
\begin{itemchoice}
\itemch $f(x)$ đồng biến trên các khoảng $(-\infty;-2)$ và $(0;+\infty)$. Do đó $f(x)$ đồng biến trên khoảng $(0;1)$.
\itemch Hàm số đã cho có hai điểm cực trị là $x=-2$ và $x=0$.
\itemch Đồ thị $(C)$ có tiệm cận xiên đi qua điểm $I(-1;0)$.
\itemch Dễ thấy $x=0$ không là nghiệm của phương trình đã cho.\\ Chia hai vế của phương trình cho $x\ne0$, ta được
\[\left|f(x)\right|=\dfrac{x-4}{x}.\]
Vẽ đồ thị hàm số $y=|f(x)|$ và đồ thị hàm số $y=\dfrac{x-4}{x}$ trên cùng một hệ trục tọa độ
\begin{center}
\begin{tikzpicture}[scale=0.55,>=stealth, font=\footnotesize, line join=round, line cap=round,declare function={a=1;b=2;c=2;d1=1;e1=1;xmin=-7;xmax=6;ymin=-5;ymax=7.5;f(\x)=(a*(\x)^2+b*(\x)+c)/(d1*(\x)+e1);g(\x)=((\x)-4)/(\x);} ]
\draw[->] (xmin,0)--(xmax,0) node [below]{$x$};
\draw[->] (0,ymin)--(0,ymax) node [right]{$y$};
\node at (0,0) [below right]{$O$};
\clip (xmin+0.1,ymin+0.1) rectangle (xmax-0.1,ymax-0.1);
\draw[dashed,smooth,samples=300,domain=xmin:(-e1/d1-0.1)] plot(\x,{f(\x)});
\draw[blue,smooth,samples=300,domain=xmin:-0.1] plot(\x,{g(\x)});
\draw[blue,smooth,samples=300,domain=0.1:xmax] plot(\x,{g(\x)});
\draw[red,smooth,samples=300,domain=(-e1/d1+0.1:xmax)] plot(\x,{f(\x)});
\draw (-e1/d1,ymin)--(-e1/d1,ymax) (xmin,1)--(xmax,1);
\draw[red,smooth,samples=300,domain=xmin:(-e1/d1-0.1)] plot(\x,{-f(\x)});
\draw[dashed] (0,2)node[below left]{$2$} (-2,0)node[above,xshift=-0.15cm]{$-2$}--(-2,-2)--(0,-2)node[right]{$-2$} (-1,0)node[below right,xshift=-0.15cm]{$-1$};
\end{tikzpicture}
\end{center}
Quan sát đồ thị ta thấy đồ thị hàm số $y=|f(x)|$ và đồ thị hàm số $y=\dfrac{x-4}{x}$ cắt nhau tại ba điểm phân biệt.\\ Do đó phương trình $x\cdot\left|f(x)\right|=x-4$ có đúng $3$ nghiệm thực phân biệt
\end{itemchoice}
}
\end{ex}

\begin{ex} %[2D1V3-6]
Khi thả một quả bóng từ đỉnh một toà tháp xuống, nó chạm đất sau 3 giây. Sau đó, quả bóng nảy lên trước khi chạm đất lần nữa 4 giây sau đó. Chiều cao tinh bằng mét của quả bóng so với mặt đất sau $t$ giây tuân theo một hàm số liên tục trên $[0; 7]$ như sau:
\[
H(t)=\left\{\begin{array}{lll}-5 t^2+c & \text{khi} & 0 \leq t < 3 \\-5 t^2+d t+e & \text{khi} & 3 \leq t \leq 7
\end{array}(c, d, e \in \mathbb{R}).\right.
\]
\begin{center}
\begin{tikzpicture}[scale=0.8,>=stealth, font=\footnotesize, line join=round, line cap=round]
\def\a{-0.5} \def\b{0} \def\c{4.5} % Hệ số
\def\xmin{-1} \def\xmax{9}
\def\ymin{-1} \def\ymax{6}
\draw[color=gray!50,dashed] (\xmin,\ymin) grid (\xmax,\ymax);
\draw[->] (\xmin,0)--(\xmax,0) node [below]{$t(s)$};
\draw[->] (0,\ymin)--(0,\ymax) node [left]{$H(m)$};
\node at (0,0) [below left]{$O$};
\clip (\xmin+0.1,\ymin+0.1) rectangle (\xmax-0.5,\ymax-0.1);
\draw[smooth,samples=300,domain=0:3] plot(\x,{\a*(\x)^2+\b*(\x)+\c});
\draw[smooth,samples=300,domain=3:7] plot(\x,{-0.5*(\x)^2+5*(\x)-10.5});
\fill[blue] (1,4) circle (5pt);
\fill[blue] (5,2) circle (5pt);
\draw (3,0) circle (1pt);
\draw (6,1.5)  node [above]{$H(t)$};
\draw (3,0)  node [below]{$3$};
\draw (7,0)  node [below]{$7$};
\end{tikzpicture}
\end{center}
\choiceTF
{\True $H(3)=H(7)=0$}
{Quả bóng được thả từ độ cao $40$ m}
{Giá trị của $d$ là $d=100$}
{\True Độ cao lớn nhắt mà quả bóng đạt được sau lần nảy đầu tiên là $20$ m}
\loigiai{
\begin{itemchoice}
\itemch Dựa vào đôg thị ta có $H(3)=H(7)=0$.
\itemch Quả bóng được thả tại thời điểm $t=0$, nên để tìm độ cao của quả bóng ta tìm $H(0)$.\\
Vì hàm số liên tục tại $x=3$ nên $\lim\limits_{x\to3}H(t)=H(3)\Leftrightarrow -5\cdot3^2+c=0\Leftrightarrow c=45
$.\\
Vậy $H(0)=45$. Do đó quả bóng được thả từ độ cao $45$ m.
\itemch
Ta có $\heva{&H(3)=0\\&H(7)=0}\Leftrightarrow \heva{&-5\cdot3^2+3d+e=0\\&-5\cdot7^2+7d+e=0}\Leftrightarrow \heva{&d=50\\&e=-105.}$\\
Vậy $H(t)=-5t^2+50t-105$.
\itemch
Độ cao của quả bóng sau lần nảy đầu tiên trong khoảng thời gian $t\in[3;7]$ nên được mô tả bởi $H(t)=-5t^2+50t-105$.\\
$H'(t)=-10t+50=0\Leftrightarrow t=5$ (tm).\\
Ta có $H(3)=H(7)=0$; $H(5)=20$.\\
Vậy độ cao lớn nhất của quả bóng sau lần nảy đầu tiên là $20$ m.
\end{itemchoice}

}
\end{ex}

\begin{ex}%[2D1V3-6]%[Tex đề Moon 2025]%[Nguyễn Hồng Thạch]
Trong một số trường hợp, tin đồn lan truyền và được mô hình hóa bằng hàm số\break $p(t)=\dfrac{1}{1+a\cdot\mathrm{e}^{-kt}}$, trong đó $p(t)$ là tỉ lệ dân số biết tin đồn tại thời điểm $t$ (giờ) và $a$, $k$ là hằng số dương. Giả sử $a=10$ và $k=0{,}5$. Khi đó
\choiceTF
{\True $\lim\limits_{t\to+\infty}p(t)=1$}
{Tốc độ lan truyền tin đồn là $p'(t)=\dfrac{10\mathrm{e}^{-0{,}5t}}{\left(1+10\mathrm{e}^{-0{,}5t}\right)^2}$}
{Tốc độ lan truyền tin đồn lớn nhất sau $9{,}2$ giờ (kết quả làm tròn đến hàng phần mười)}
{\True Tại thời điểm tin đồn lan truyền với tốc độ lớn nhất thì có $50\%$ dân số biết tin đồn (làm tròn kết quả đến hàng đơn vị)}
\loigiai{
\begin{itemchoice}
\itemch Vì $\lim\limits_{t \to +\infty} \mathrm{e}^{-0{,}5t} = 0 \Rightarrow \lim\limits_{t \to +\infty} p(t) = \dfrac{1}{1 + 10 \cdot 0} = 1$.
\itemch Ta có
\[
p(t) = \dfrac{1}{1 + 10\mathrm{e}^{-0{,}5t}} \Rightarrow p'(t) = \dfrac{10 \cdot 0{,}5 \mathrm{e}^{-0{,}5t}}{(1 + 10\mathrm{e}^{-0{,}5t})^2} = \dfrac{5\mathrm{e}^{-0{,}5t}}{(1 + 10\mathrm{e}^{-0{,}5t})^2}.
\]
\itemch Ta có \begin{eqnarray*}
p''(t)&=&\dfrac{-2{,}5\mathrm{e}^{-0{,}5t}\left(1+10\mathrm{e}^{-0{,}5t}\right)^2-2\left(1+10\mathrm{e}^{-0{,}5t}\right)\cdot\left(-5\mathrm{e}^{-0{,}5t}\right)\cdot 5\mathrm{e}^{-0{,}5t}}{\left(1+10\mathrm{e}^{-0{,}5t}\right)^4}\\
&=&\dfrac{-2{,5}\mathrm{e}^{-0{,}5t}+25\mathrm{e}^{-t}}{\left(1+10\mathrm{e}^{-0{,}5t}\right)^3}
\end{eqnarray*}
Suy ra \[p''(t)=0\Leftrightarrow-2{,5}\mathrm{e}^{-0{,}5t}+25\mathrm{e}^{-t}=0\Leftrightarrow \mathrm{e}^{-0{,}5t}=\dfrac{1}{10}\Leftrightarrow t=\dfrac{\ln 10}{0{,}5}\approx4{,}6.\]
Ta có 	\begin{center}
\begin{tikzpicture}[scale=0.8]
\tkzTabInit
[lgt=1.2,espcl=2.5] % tùy chọn
{$t$/1,$p''(t) $/1,$p'(t)$/2.5}
{$0$,$\tfrac{\ln10}{0{,}5}$,$+\infty$}
\tkzTabLine{,-,0,+,} %
\tkzTabVar{-/, +/,-/} %dấu mũi tên, + trên, -dưới
\end{tikzpicture}
\end{center}
Dựa vào bảng biến thiên ta thấy tốc độ lan truyền lớn nhất tại thời điểm $t\approx 4{,}6$ giờ.
\itemch Thay $t = \dfrac{\ln 10}{0{,}5}$ vào $p(t)$:
\[
p\left(\dfrac{\ln 10}{0{,}5}\right) = \dfrac{1}{1 + 10e^{-0{,}5 \cdot \tfrac{\ln 10}{0{,}5}}} = \dfrac{1}{1 + 10 \cdot \dfrac{1}{10}} = \dfrac{1}{2} = 50\%.
\]
\end{itemchoice}
}
\end{ex}

\begin{ex}%[2D1V3-6]%[TEX Đề Moon 2025]%[Võ Nguyên Thạch]
Một nhà sản xuất trung bình bán được $1\,000$ ti vi màn hình phẳng mỗi tuần với giá $14$ triệu đồng một chiếc. Một cuộc khảo sát thị trường chỉ ra rằng nếu cứ giảm giá bán $500$ nghìn đồng, số lượng ti vi bán ra sẽ tăng thêm khoảng $100$ ti vi mỗi tuần. Gọi $x$ là số ti vi bán được mỗi tuần, $p$ (triệu đồng) là giá bán của mỗi ti vi. Khi đó $p=p(x)$ được gọi là hàm cầu.
\choiceTF
{\True Hàm cầu là $p=-\dfrac{1}{200}x+19$ (triệu đồng)}
{Tổng doanh thu từ tiền bán ti vi là $200p^2+3\,800p$ (triệu đồng)}
{\True Công ty giảm giá $4{,}5$ triệu đồng cho người mua thì doanh thu của công ty sẽ lớn nhất}
{\True Nếu hàm chi phí hằng tuần là $C(x)=12\,000-3x$ (triệu đồng), trong đó $x$ là số ti vi bán ra trong tuần, nhà sản xuất nên đặt giá bán $8$ triệu đồng thì lợi nhuận là lớn nhất}
\loigiai{
\begin{itemchoice}
\itemch Theo giả thiết, tốc độ thay đổi của $x$ tỉ lệ với tốc độ thay đổi của $p$ nên hàm số $p=p(x)$ là hàm số bậc nhất.\\
Khi đó, $p(x)=ax+b$, ($a$ khác $0$).\\
Giá tiền $p_1=14$ ứng với $x_1=1\,000$; giá tiền $p_2=13{,}5$ ứng với $x_2=1\,000+100=1\,100$.\\
Do đó phương trình đường thẳng $p(x)=ax+b$ đi qua hai điểm $(1\,000;14)$ và $(1\,100;13{,}5)$.\\
Ta có hệ phương trình $\heva{&14=1\,000a+b\\&13{,}5=1\,100a+b}\Leftrightarrow \heva{&a=-\dfrac{1}{200}\\&b=19}$ (thỏa mãn).\\
Vậy hàm cầu là
\[p(x)=-\dfrac{1}{200}x+19.\]
\itemch Ta có $p=-\dfrac{1}{200}x+19\Rightarrow x=-200p+3\,00$.\\
Suy ra Tổng doanh thu từ tiền bán ti vi là
\[R(x)=px=p(-200p+3\,800)=-200p^2+3\,800p \text{ (triệu đồng).}\]
\itemch Để doanh thu là lớn nhất thì ta cần tìm $p$ sao cho $R$ đạt giá trị lớn nhất.\\
Ta có $R'=-400p+3\,800=0\Rightarrow p=\dfrac{19}{2}$.\\
Bảng biến thiên
\begin{center}
\begin{tikzpicture}
\tkzTabInit[espcl=2.5,lgt=1.5,nocadre]
{$p$/0.9,$R'(p)$/0.7,$R(x)$/2.1}
{$0$,$\dfrac{19}{2}$,$+\infty$}
\tkzTabLine{,+,0,-,}
\tkzTabVar{-/$0$,+/$18\,050$,-/$-\infty$}
\end{tikzpicture}
\end{center}
Vậy công ty nên giảm giá số tiền một chiếc ti vi là $14-\dfrac{19}{2}=4{,}5$ (triệu đồng) thì doanh thu là lớn nhất.
\itemch Doanh thu bán hàng của x sản phẩm là
\[R(x)=x\cdot p(x)=x\cdot \left(-\dfrac{1}{200}x+19\right)=-\dfrac{x^2}{200}+19x \text{ (triệu đồng).}\]
Do đó hàm số thể hiện lợi nhuận thu được khi bán $x$ sản phẩm là
\[P(x)=R(x)-C(x)=-\dfrac{x^2}{200}+19x-12\,000+3x=-\dfrac{x^2}{200}+22x-12\,000 \text{ (triệu đồng).}\]
Để lợi nhuận là lớn nhất thì $P(x)$ là lớn nhất.\\
Ta có $P'(x)=-\dfrac{x}{100}+22=0\Leftrightarrow x=2\,200$.\\
Bảng biến thiên:
\begin{center}
\begin{tikzpicture}
\tkzTabInit[espcl=2.5,lgt=1.5,nocadre]
{$x$/0.7,$P'(x)$/0.7,$P(x)$/2.1}
{$0$,$22\,$,$+\infty$}
\tkzTabLine{,+,0,-,}
\tkzTabVar{-/$0$,+/$12\,000$,-/$-\infty$}
\end{tikzpicture}
\end{center}
Vậy có $2\,200$ ti vi được bán ra thì lợi nhuận là cao nhất. Số ti vi mua tăng lên là $2\,200-1\,000=1\,200$ (chiếc).\\
Vậy cửa hàng nên đặt giá bán là $14-0{,}5\cdot \dfrac{1\,200}{100}=8 \text{ (triệu đồng)}$.
\end{itemchoice}
}
\end{ex}

\begin{ex}%[2D1V2-2]%[TEX ĐỀ MOON 2025]%[Lê Hữu Kiệt]
\immini[thm]
{Cho hàm số $y=f(x)$ có đạo hàm trên $\mathbb{R}$ và hàm số $y=f'(x)$ là hàm số bậc ba có đồ thị là đường cong trong hình vẽ.
\choiceTF
{Hàm số $y=f(x)$ đồng biến trên khoảng $(-\infty;-2)$}
{Hàm số $y=f(x)$ có hai điểm cực trị}
{$f'(2)=4$}
{\True Hàm số $g(x)=f(x)-\dfrac{1}{2}x^2+x+2024$ đồng biến trên khoảng $\left(-\dfrac{5}{2};-\dfrac{3}{2}\right)$}
}
{\begin{tikzpicture}[scale=0.8,>=stealth, font=\footnotesize, line join=round, line cap=round]
\def\a{1} \def\b{3} \def\c{0} \def\d{-4} % Hệ số
\def\xmin{-4} \def\xmax{2}
\def\ymin{-5} \def\ymax{1}
\draw[->] (\xmin,0)--(\xmax,0) node [below]{$x$};
\draw[->] (0,\ymin)--(0,\ymax) node [left]{$y$};
\node at (0,0) [above right]{$O$};
\draw (1.3,1.2)node[]{$y=f'(x)$};
\clip (\xmin+0.1,\ymin+0.1) rectangle (\xmax-0.5,\ymax-0.1);
\draw[smooth,samples=300] plot(\x,{\a*(\x)^3+\b*(\x)^2+\c*(\x)+\d});
\draw[dashed] (-3,0)node[above,xshift=-0.15cm]{$-3$}--(-3,-4)--(0,-4)node[below right,xshift=-0.1cm]{$-4$} (-2,0)node[above,xshift=-0.15cm]{$-2$} (-1,0)node[above,xshift=-0.15cm]{$-1$}--(-1,-2)--(0,-2)node[right,xshift=-0.1cm]{$-2$} (1,0)node[above right]{$1$};
\foreach \x/\y in {-3/-4, -2/0, -1/-2, 0/-4, 1/0}{\fill (\x,\y) circle (1.25pt);}
\end{tikzpicture}}
\loigiai{
\begin{itemchoice}
\itemch Ta có $f'(x)<0$, $\forall x\in (-\infty;-2)$ nên $y=f(x)$ nghịch biến trên khoảng $(-\infty;-2)$.
\itemch Ta có $y=f'(x)$ chỉ đổi dấu khi qua điểm $x=1$ nên hàm số có $1$ điểm cực trị.
\itemch Gọi $y=f'(x)=ax^3+bx^2+cx+d$ ($a\ne0$). Ta có đồ thị hàm số $y=f'(x)$ đi qua các điểm có tọa độ $(-3;-2)$, $(-2;0)$, $(-1;-2)$ và $(0;-4)$ nên ta có hệ phương trình
\[\heva{&-81a+27b-9c+d=-4\\&-8a+4b-2c+d=0\\&-a+b-c+d=-2\\&d=-4} \Leftrightarrow \heva{&a=1\\&b=3\\&c=0\\&d=-4.}\]
Suy ra $y=f'(x)=x^3+3x^2-4$.\\
Khi đó $f'(2)=16$.
\itemch Ta có $g'(x)=f'(x)-x+1=x^3+3x^2-x-3$.\\
Khi đó $g'(x)=0\Leftrightarrow x^3+3x^2-x-3=0 \Leftrightarrow \hoac{&x=-3\\&x=-1\\&x=1.}$\\
Bảng biến thiên
\begin{center}
\begin{tikzpicture}[font=\footnotesize, line join=round, line cap=round, >=stealth, scale=1]
\tkzTabInit[espcl=2.5,lgt=1.5]
{$x$/0.7,$g'(x)$/0.7,$g(x)$/2}
{$-\infty$, $-3$, $-1$, $1$, $+\infty$}
\tkzTabLine{,-,$0$,+,$0$,-,$0$,+,}
\tkzTabVar{+/, -/, +/, -/, +/}
\end{tikzpicture}
\end{center}
Suy ra hàm số $y=g(x)$ đồng biến trên các khoảng $(-3;-1)$, $(1;+\infty)$.\\
Mà $\left(-\dfrac{5}{2};-\dfrac{3}{2}\right)\subset(-3;-1)$ nên hàm số $y=g(x)$ đồng biến trên khoảng $\left(-\dfrac{5}{2};-\dfrac{3}{2}\right)$.
\end{itemchoice}
}
\end{ex}

\begin{ex}%[2D1V3-6]
\immini{Người ta muốn thiết kế một lồng nuôi cá có bề mặt hình chữ nhật bao gồm phần mặt nước có diện tích bằng $54$ m$^2$ và phần đường đi xung quanh với kích thước (đơn vị: m) như hình vẽ
\choiceTF
{\True Kích thước hình chữ nhật phần mặt nước là $(a-3)$ (m) và $(b-2)$ (m), với $a>3$, $b>2$}
{ Biểu diễn $b$ theo $a$ là $b=\dfrac{54}{a-2}+3$}
{Diện tích phần đường đi theo $a$ là $S(a)=\dfrac{54a}{a-3}+3a-5$, $(a>3)$}
{\True Diện tích phần đường đi là bé nhất bằng $42$ (m$^2$)}}{
\begin{tikzpicture}
\coordinate (A) at (0,0);
\coordinate (B) at (4,0);
\coordinate (C) at (4,3);
\coordinate (D) at (0,3);
\fill[draw=black, fill=gray!50] (A) rectangle (C) ;
\fill[draw=black, fill=gray!20] (1,0.5)--(3.5,0.5)--(3.5,2.5)--(1,2.5)--cycle;
\draw[<->] (0,3.5)--node[above]{$a$}(4,3.5);
\draw[<->] (-0.5,0)--node[left]{$b$}(-0.5,3);
\draw[<->] (0,1)--node[below]{$2$}(1,1);
\draw[<->] (3.5,1)--node[below]{$1$}(4,1);
\draw[<->] (2,0)--node[right]{$1$}(2,0.5);
\draw[<->] (2,2.5)--node[right]{$1$}(2,3);
\end{tikzpicture}
}
\loigiai{
\begin{itemchoice}
\itemch Kích thước hình chữ nhật phần mặt nước là $(a-3)$ (m) và $(b-2)$ (m), với $a>3$, $b>2$.
\itemch Diện tích phần mặt nước là $54$ (m$^2$) nên \[(a-3)(b-2)=54\Leftrightarrow b-2=\dfrac{54}{a-3}\Leftrightarrow b=\dfrac{54}{a-3}+2.\]
\itemch Diện tích cá lồng nuôi cá là \[a\cdot b=a\cdot\left(\dfrac{54}{a-3}+2\right)=\dfrac{54a}{a-3}+2a.\]
Diện tích phần đường đi là $\dfrac{54a}{a-3}+2a-54$.
\itemch Xét hàm số $f(a)=\dfrac{54a}{a-3}+2a-54\Rightarrow f'(a)=\dfrac{-162}{(a-3)^2}+2$.\\
Ta có $f'(a)=0\Leftrightarrow a=-6$ (loại) hoặc $a=12$ (thoả mãn).\\
Tính giá trị của $f(a)$ tại điểm cực trị, ta có diện tích phần đường đi nhó nhất khi $a=12$ (m) và có diện tích $f(12)=42$ (m$^2$).
\end{itemchoice}
}
\end{ex}

% \paragraph{Mức độ C}
\begin{ex}%[50 Đề minh họa tốt nghiệp 2025 - Đề 13]%[Lê Hữu Kiệt - Lê Quân]%[2D1C2-7]
Trên trục $Os$, cho hai chất điểm chuyển động có toạ độ theo thời gian $t$ (giây) lần lượt là $s_1=\sin t$ và $s_2=\sin\left(t+\dfrac{\pi}{3}\right)$ (đơn vị: mét).
\begin{center}
\begin{tikzpicture}[line join=round, line cap=round, >=stealth, scale=1]
\draw[->] (-3.2,0)--(4.5,0)node[below]{$s$};
\draw (0,0)node[above]{$O$};
\fill (-0.8,0) circle (2pt) node[above]{$s_1$} (1.5,0) circle (2pt) node[above]{$s_2$};
\foreach \x in {-3,...,4}{
\draw (\x,0.1)--(\x,-0.1)node[below]{$\x$};
}
\end{tikzpicture}
\end{center}
\choiceTF
{Tại thời điểm ban đầu hai chất điểm cách nhau một khoảng bằng $50$ cm}
{Khoảng cách giữa hai chất điểm được xác định bởi hàm số $d=s_1-s_2$ (mét)}
{\True Trong $6$ giây đầu tiên, có hai thời điểm mà vận tốc của hai chất điểm bằng nhau}
{\True Trong $6$ giây đầu tiên, khoảng cách xa nhất của hai chất điểm là $100$ cm}
\loigiai{
\begin{itemchoice}
\itemch Tại thời điểm bắt đầu thì $t=0$. Khi đó $s_1=\sin 0 = 0$; $s_2=\sin\left(0+\dfrac{\pi}{3}\right)=\dfrac{\sqrt3}{2}$.\\
Khoảng cách giữa hai chất điểm là $s_2-s_1=\dfrac{\sqrt3}{2}$ m.
\itemch Khoảng cách giữa hai chất điểm được xác định bởi hàm số $d=|s_1-s_2|$ (mét).
\itemch Vận tốc của chất điểm thứ nhất và thứ hai lần lượt là $v_1=s_1'=\cos t$ và $v_2=s_2'=\cos\left(t+\dfrac{\pi}{3}\right)$.\\
Khi hai chất điểm có vận tốc bằng nhau
\begin{eqnarray*}
&& v_1=v_2 \\
&\Leftrightarrow& \cos t = \cos\left(t+\dfrac{\pi}{3}\right) \\
&\Leftrightarrow& \hoac{&t=t+\dfrac{\pi}{3}+k2\pi \\& t=-t-\dfrac{\pi}{3}+k2\pi} \\
&\Leftrightarrow& t=-\dfrac{\pi}{6}+k\pi,\, (k\in\mathbb{Z}).
\end{eqnarray*}
Trong $6$ giây đầu tiên, tức $0\leq t \leq 6 \Leftrightarrow 0\leq -\dfrac{\pi}{6}+k\pi \leq 6 \Leftrightarrow \dfrac{1}{6} \leq k < 2{,}08$.\\
Do $k\in\mathbb{Z}$ nên $k\in\{1;2\}$.\\
Vậy trong $6$ giây đầu tiên, có hai thời điểm mà vận tốc của hai chất điểm bằng nhau.
\itemch Xét $y=s_1-s_2=\sin t - \sin\left(t+\dfrac{\pi}{3}\right)$.\\
Tập xác định $\mathscr{D}=\mathbb{R}$.\\
Ta có $y'=\cos t - \cos\left(t+\dfrac{\pi}{3}\right)$.\\
Cho $y'=0 \Leftrightarrow \cos t- \cos\left(t+\dfrac{\pi}{3}\right)=0 \Leftrightarrow t=-\dfrac{\pi}{6}+k\pi$, $k\in\mathbb{Z}$.\\
Bảng biến thiên của $y$ trên đoạn $[0;6]$
\begin{center}
\begin{tikzpicture}[font=\footnotesize, line join=round, line cap=round, >=stealth, scale=1]
\tkzTabInit[lgt=1.2,espcl=2.5,deltacl=0.6]
{$x$/1, $y'$/0.7, $y$/2}
{$0$, $\dfrac{5\pi}{6}$, $\dfrac{11\pi}{6}$, $6$}
\tkzTabLine
{, + , $0$ , - , $0$ , + , }
\tkzTabVar
{-/$-\dfrac{\sqrt3}{2}$ , +/$1$, -/$-1$, +/$-0{,}97$}
\end{tikzpicture}
\end{center}
Suy ra, khoảng cách giữa hai chất điểm là $d=|y|$ có giá trị lớn nhất là $1$ m, hay $100$ cm khi $t=\dfrac{5\pi}{6}$ và $t=\dfrac{11\pi}{6}$.\\
\end{itemchoice}
}
\end{ex}


\Closesolutionfile{ans}

\Opensolutionfile{ans}[ans/ansBTshortans]

\subsection{Câu trắc nghiệm trả lời ngắn}
% \paragraph{Mức độ N}
% \paragraph{Mức độ H}
\setcounter{ex}{0}
\begin{ex}%[1H8H5-4]%[TEX ĐỀ MOON 2025]%[Nguyễn Văn Hiệp]
Cho hình lăng trụ đứng $ABC.A'B'C'$ có đáy là tam giác đều độ dài cạnh bằng $6\sqrt{3}$. Khoảng cách giữa hai đường thẳng $AA'$ và $BC$ bằng bao nhiêu?
\shortans{$9$}
\loigiai{\immini{ \textbf{Bước 1: Xác định hình chiếu vuông góc} \\
Vì lăng trụ đứng nên $AA' \perp (ABC)$. Gọi $M$ là trung điểm $BC$, ta có $AM \perp BC$.\\
\textbf{Bước 2: Tính độ dài đường cao} \\
Tam giác $ABC$ đều cạnh $6\sqrt{3}$ có
\[
AM = \dfrac{6\sqrt{3} \cdot \sqrt{3}}{2} = 9.
\]
\textbf{Bước 3: Kết luận khoảng cách} \\
Vì $AA' \perp (ABC)$ và $AM \perp BC$ nên
\[
\mathrm{d}(AA', BC) = AM = 9.
\]}{\begin{tikzpicture}[line join=round, line cap=round,>=stealth,thick,scale=0.8,font=\scriptsize]
\def\a{4}
\def\h{4.5}
\path 	(0:0) coordinate (A)
++(0:\a) coordinate (C)
++(-150:3*\a/4) coordinate (B)
($(A)+(90:\h)$) coordinate (A')
($(B)+(90:\h)$) coordinate (B')
($(C)+(90:\h)$) coordinate (C')
($(B)!0.5!(C)$) coordinate (M)
;
\draw[dashed,thick] 	(A)--(C) (A)--(M);
\draw[thick]	(C)--(C') 	(B)--(B')	(A)--(A') (A)--(B)--(C) (A')--(B')--(C')--cycle;
\foreach \x/\g in {A/180,B/-45,C/0,A'/180,B'/-45,C'/0,M/-45}
\fill[black] 	(\x) circle (1pt)
($(\g:4mm)+(\x)$) node {$\x$};
\end{tikzpicture}
}

}
\end{ex}

\begin{ex}%[1H8H5-4]%[TEX ĐỀ MOON 2025]%[Nguyễn Cường]
Cho hình lăng trụ đứng $ABC.A'B'C'$ có $AB=5$, $AC=6$, $\widehat{A}=60^\circ$. Khoảng cách giữa hai đường thẳng $AA'$ và $BC$ (làm tròn kết quả đến hàng phần mười) bằng bao nhiêu?
\shortans{$4{,}7$}
\loigiai
{
\begin{center}
\begin{tikzpicture}[scale=1, font=\footnotesize, line join=round, line cap=round, >=stealth]
\path
(0,0)coordinate(A)++(0:4)coordinate(C)++(210:3)coordinate(B)
(A)++(90:3)coordinate(A')++(0:4)coordinate(C')++(210:3)coordinate(B')
($(B)!.4!(C)$)coordinate(H)
;
\draw (A')--(A)--(B)--(C)--(C')--(B')--(A')--(C')
(B)--(B')
;
\draw[dashed] (H)--(A)--(C);
\foreach \i/\g in {A/180,B/-90,C/0,A'/90,B'/90,C'/90,H/-90}{\draw[fill=black](\i) circle (1pt) ($(\i)+(\g:3mm)$) node[scale=1]{$\i$};}
\end{tikzpicture}
\end{center}
Gọi $AH$ là đường cao của $\triangle ABC$.\\
Ta có $\heva{&AA'\perp AH\\&BC\perp AH}\Rightarrow \mathrm{d}\big(AA',BC\big)=AH$.\\
Xét $\triangle ABC$ có
\begin{itemize}
\item $BC=\sqrt{AB^2+AC^2-2AB\cdot AC\cos\widehat{A}}=\sqrt{25+36-2\cdot 5\cdot 6\cdot \cos 60^\circ}=\sqrt{31}$.
\item $AH=\dfrac{AB\cdot AC\cdot\sin 60^\circ}{BC}=\dfrac{5\cdot 6\cdot \sin 60^\circ}{\sqrt{31}}=\dfrac{15\sqrt{93}}{31}\approx 4{,}67$.
\end{itemize}
Vậy $\mathrm{d}\big(AA',BC\big)=AH\approx 4{,}67$.
}
\end{ex}

\begin{ex}%[1H8H5-3]
Cho hình chóp $S.ABCD$ có đáy là hình vuông cạnh bằng $1$, $SA$ vuông góc với mặt phẳng $(ABCD)$ và $SA=\dfrac{\sqrt{3}}{3}$. Khoảng cách từ điểm $A$ đến mặt phẳng $(SCD)$ bằng bao nhiêu? (làm tròn kết quả đến hàng phần mười).
\shortans{0{,}5}
\loigiai{
\immini{
Ta có $\heva{& CD \perp AD \\ & CD \perp SA \\ & AD, SA \subset (SAD) \\ & AD \cap SA = \{A\}} \Rightarrow CD \perp (SAD)$. \\
Trong mặt phẳng $(SAD)$, kẻ $AH \perp SD$ tại $H$. \\
Vì $CD \perp (SAD)$ và $AH \subset (SAD)$ nên $CD \perp AH$. \\
Ta có $\heva{& AH \perp SD \\ & AH \perp CD \\ & SD, CD \subset (SCD) \\ & SD \cap CD = \{D\}} \Rightarrow AH \perp (SCD)$. \\
Do đó, khoảng cách từ điểm $A$ đến mặt phẳng $(SCD)$ là $\mathrm{d}(A, (SCD)) = AH$. \\
Xét tam giác $\triangle SAD$ vuông tại $A$, có $AD=1$ và $SA=\dfrac{\sqrt{3}}{3}$. \\
Áp dụng hệ thức lượng trong tam giác vuông $SAD$, ta có \\
$\dfrac{1}{AH^2} = \dfrac{1}{SA^2} + \dfrac{1}{AD^2} = \dfrac{1}{\left(\dfrac{\sqrt{3}}{3}\right)^2} + \dfrac{1}{1^2} = \dfrac{1}{\frac{3}{9}} + 1 = \dfrac{1}{\frac{1}{3}} + 1 = 3 + 1 = 4$. \\
$\Rightarrow AH^2 = \dfrac{1}{4} \Rightarrow AH = \dfrac{1}{2} = 0{,}5$. \\
Vậy khoảng cách từ điểm $A$ đến mặt phẳng $(SCD)$ bằng $0{,}5$.
}{
\begin{tikzpicture}[>=stealth,line join=round,line cap=round,font=\footnotesize,scale=1]
\tikzset{
pics/hinhChopTuGiac/.style  n args={5}{
code={
\tikzset{
declare function={a=4;b=2;h=3;goc=-120;}
}
\path
(0,0)coordinate (#1)+(0:a)coordinate (#2)+(goc:b)coordinate (#4)+(90:h)coordinate (#5)
($(#2)+(#4)-(#1)$)coordinate (#3)
;
}
}}
\path
(0,0)pic {hinhChopTuGiac={A}{D}{C}{B}{S}}
($(S)!.4!(D)$)coordinate (H)
pic[draw,angle radius=2mm,angle eccentricity=1.5]{right angle=A--H--D}
;
\foreach \pointo/\pointt in {S/B,S/C,S/D,B/C,C/D}{
\draw[fill=black](\pointo)--(\pointt);
}
\foreach \pointo/\pointt in {S/A,A/B,A/D,A/H}{
\draw[fill=black,dashed](\pointo)--(\pointt);
}
\foreach \point/\goc in {A/160,S/90,B/190,D/10,C/-45,H/45}{
\draw[fill=black](\point)circle(.8pt)+(\goc:2mm)node[scale=.8]{$\point$};
}
\end{tikzpicture}
}

}
\end{ex}

\begin{ex}%[1D6H4-6]%[TEX ĐỀ MOON 2025]%[Lê Hữu Kiệt]
Các khí thải gây hiệu ứng nhà kính là nguyên nhân chủ yếu làm Trái Đất nóng lên. Theo OECD (Tổ chức Hợp tác và Phát triển kinh tế Thế giới), khi nhiệt độ Trái Đất tăng lên thì tổng giá trị kinh tế toàn cầu giảm. Người ta ước tính rằng, khi nhiệt độ Trái Đất tăng thêm $2^\circ$C thì tổng giá trị kinh tế toàn cầu giảm $3\%$; còn khi nhiệt độ Trái Đất tăng thêm $5^{\circ}$C thì tổng giá trị kinh tế toàn cầu giảm $10\%$. Biết rằng, nếu nhiệt độ Trái Đất tăng thêm $t^\circ$C, tổng giá trị kinh tế toàn cầu giảm $f(t)\%$ thì $f(t)=k\cdot a^t$, trong đó $k$, $a$ là các hằng số dương. Khi nhiệt độ Trái Đất tăng thêm bao nhiêu độ C thì tổng giá trị kinh tế toàn cầu giảm đến $20\%$ (Làm tròn đến hàng phần chục)?
\shortans{$6{,}7$}
\loigiai{
Với $t=2$ thì $f(t)=3$, suy ra $3=ka^2\Leftrightarrow k=\dfrac{3}{a^2}$.\\
Với $t=5$ thì $f(t)=10$, suy ra
\[10=ka^5\Leftrightarrow 10=3a^3 \Leftrightarrow a=\left(\dfrac{10}{3}\right)^{\tfrac{1}{3}}.\]
Suy ra $k=\dfrac{3}{\left(\dfrac{10}{3}\right)^{\tfrac{2}{3}}}.$ Do đó $f(t)=\dfrac{3}{\left(\dfrac{10}{3}\right)^{\tfrac{2}{3}}}\cdot\left(\dfrac{10}{3}\right)^{\tfrac{t}{3}}=3\cdot\left(\dfrac{10}{3}\right)^{\tfrac{t-2}{3}}$.\\
Khi kinh tế toàn cầu giảm đến $20\%$, tức $f(t)=20$, ta có
\begin{eqnarray*}
&&3\cdot\left(\dfrac{10}{3}\right)^{\tfrac{t-2}{3}}=20 \\
&\Leftrightarrow& \left(\dfrac{10}{3}\right)^{\tfrac{t-2}{3}}=\dfrac{20}{3} \\
&\Leftrightarrow& \dfrac{t-2}{3}=\log_{\tfrac{10}{3}}\dfrac{20}{3} \\
&\Leftrightarrow&t=3\log_{\tfrac{10}{3}}\dfrac{20}{3}+2 \\
&\Leftrightarrow& t\approx 6{,}7.
\end{eqnarray*}
Vậy khi nhiệt độ Trái Đất tăng thêm khoảng $6{,}7$ độ C thì tổng giá trị kinh tế toàn cầu giảm đến $20\%$.
}
\end{ex}

\begin{ex}%[1C2H3-2]%[TEX ĐỀ MOON 2025]%[Huỳnh Thanh Chí]
Một người đưa thư xuất phát từ bưu điện ở vị trí $A$, các điểm cần phát thư nằm dọc các con đường cần đi qua. Biết rằng người này phải đi trên mỗi con đường ít nhất một lần (để phát được thư cho tất cả các điểm cần phát nằm dọc theo con đường đó) và cuối cùng quay lại điểm xuất phát. Độ dài các con đường như hình vẽ (đơn vị độ dài).
\begin{center}
\begin{tikzpicture}[scale=0.75,>=stealth, font=\footnotesize, line join=round, line cap=round]
\coordinate (A) at (0,0);
\coordinate (B) at (4,0);
\coordinate (C) at (6,-1.6);
\coordinate (D) at (4,-2.6);
\coordinate (E) at (0,-3);
\draw (A)node[above]{$A$}--(B)node[above]{$B$}--(C)node[right]{$C$}--(D)node[below]{$D$}--(E)node[below]{$E$}--(A) (E)--(B)--(D) ($(A)!0.5!(B)$)node[above]{$8$} ($(B)!0.5!(C)$)node[above right]{$5$} ($(D)!0.5!(C)$)node[below right]{$2$} ($(B)!0.5!(D)$)node[right]{$4$} ($(B)!0.5!(E)$)node[above left]{$10$} ($(A)!0.5!(E)$)node[right]{$6$} ($(E)!0.5!(D)$)node[below]{$9$};
\draw (A)..controls (-0.5,-1.5) and (-0.5,-1.5)..(E) (-0.5,-1.5)node[left]{$7$};
\end{tikzpicture}
\end{center}
Hỏi tổng quãng đường người đưa thư có thể đi ngắn nhất có thể là bao nhiêu?

\shortans[]{$63$}
\loigiai{
Bài toán yêu cầu tìm chu trình Euler có độ dài nhỏ nhất.\\
Tổng độ dài các cạnh của đồ thị là: $8+5+2+4+10+6+9+7 = 51$.\\
Các đỉnh có bậc lẻ là $A$ (bậc 3), $C$ (bậc 3), $D$ (bậc 3).\\
Cần tìm đường đi ngắn nhất để nối các đỉnh bậc lẻ này.\\
Ta có các cặp đường đi:
\begin{itemize}
\item $AC = AB+BC = 8+5 = 13$.
\item $AD = AB+BD = 8+4 = 12$.
\item $CD = 2$.
\end{itemize}
Chọn $CD=2$, khi đó:
\begin{itemize}
\item $AC=13$.
\item $AD=12$.
\end{itemize}
Ta nối $CD=2$ và chọn $AC,AD$. Ta thấy chỉ cần chọn cặp $CD$ thôi, như vậy đường đi là $AC=13$ hoặc $AD=12$, nhưng tổng $CD=2$. Ta ghép $A,C$ thì tạo đường đi $AC = 13$, và chọn ghép $AD$, có $AD=12$, hoặc $AD=12$, $CD = 2$ chọn $CD$ ngắn hơn, với các cạnh là: CD nối 2 cạnh lại: $CD=2$. Và $A,C,D$ bậc lẻ cần thêm cạnh $AC$ hoăc $AD$ cho chẵn cạnh. Ta có:
\begin{itemize}
\item $AC=13$.
\item $AD=12$.
\end{itemize}
Ta ưu tiên ghép $CD$ nên là $CD=2$. Ghép $A$ vào đỉnh còn lại: chọn đỉnh cách xa 1 đoạn: chọn $AD$ hay $AC$. Theo đó ta chọn cạnh $CD$ với $AD$ hoặc $AC$. Vì $AC,AD >CD$ ta loại $AD$, chọn CD. Sau đó chọn từ $A$ đi $C$ hoặc $D$. Chọn $A,D$.

Vậy, cặp cạnh cần lặp lại có tổng độ dài nhỏ nhất là: $CD+AD= 2+12 = 14$.
Tổng quãng đường người đưa thư phải đi là: $51 + (CD+AD) =51+14 = 63$.

Vậy tổng độ dài quãng đường ngắn nhất là $63$.
}
\end{ex}

\begin{ex}%[1C2H3-1]%[TEX ĐỀ MOON 2025]%[Nguyễn Văn Hiệp]
Giả sử $4$ thành phố $A$, $B$, $C$, $D$ với khoảng cách (đơn vị: km) giữa các thành phố được cho bởi bảng sau
\begin{center}
\begin{tblr}{hlines={0.6pt},vlines={0.6pt},width=0.7\linewidth,rows={abovesep=1pt,belowsep=1pt},colspec={X[1,c]X[1,c]X[1,c]X[1,c]X[1,c]}}
& $A$ & $B$ & $C$ & $D$ \\
$A$ & $0$ & $10$ & $15$ & $20$ \\
$B$ & $10$ & $0$ & $25$ & $35$ \\
$C$ & $15$ & $25$ & $0$ & $30$ \\
$D$ & $20$ & $35$ & $30$ & $0$ \\
\end{tblr}
\end{center}
Hãy tính quãng đường ngắn nhất để đi qua tất cả các thành phố đúng một lần rồi quay lại thành phố xuất phát?
\shortans{$85$}
\loigiai{
\textbf{Bước 1: Liệt kê các chu trình Hamilton} \\
Xét tất cả các hoán vị
\begin{itemize}
\item $A \rightarrow B \rightarrow C \rightarrow D \rightarrow A$: $10 + 25 + 30 + 20 = 85$ km.
\item $A \rightarrow B \rightarrow D \rightarrow C \rightarrow A$: $10 + 35 + 30 + 15 = 90$ km.
\item $A \rightarrow C \rightarrow B \rightarrow D \rightarrow A$: $15 + 25 + 35 + 20 = 95$ km.
\item $A \rightarrow C \rightarrow D \rightarrow B \rightarrow A$: $15 + 30 + 35 + 10 = 90$ km.
\item $A \rightarrow D \rightarrow B \rightarrow C \rightarrow A$: $20 + 35 + 25 + 15 = 95$ km.
\item $A \rightarrow D \rightarrow C \rightarrow B \rightarrow A$: $20 + 30 + 25 + 10 = 85$ km.
\end{itemize}
\textbf{Bước 2: Chọn chu trình tối ưu} \\
Quãng đường ngắn nhất là $85$ km.
}
\end{ex}

\begin{ex}%[50 Đề minh họa tốt nghiệp 2025 - Đề 13]%[Lê Hữu Kiệt - Lê Quân]%[1C2H3-1]
Biểu đồ thể hiện các con đường nối giữa các thị trấn (đơn vị: km). Cán bộ thanh tra xuất phát từ thị trấn $L$ đi kiểm tra tất cả các tuyến đường nối giữa các thị trấn $M$, $N$, $O$ và quay lại $L$. Chiều dài quãng đường tối thiểu thanh tra cần phải đi là bao nhiêu km?
\begin{center}
\begin{tikzpicture}[font=\footnotesize, line join=round, line cap=round, >=stealth, scale=1]
\def\bankinh{2}
\path (0,0) coordinate (O) (-\bankinh,0) coordinate (L) (\bankinh,0) coordinate (N) (60:\bankinh) coordinate (M);
\draw
(L) arc(180:60:\bankinh) node[pos=0.5, above]{$16$}
(M) arc(60:0:\bankinh) node[pos=0.5, above]{$8$}
(L) to[bend right=60] node[pos=0.5, below]{$15$} (N)
(L) to[bend right=40] node[pos=0.5, below]{$9$} (O)
(L) to[bend left=40] node[pos=0.5, above]{$7$} (O)
(O) to[bend right=40] node[pos=0.5, below]{$8$} (N)
(O) to[bend left=40] node[pos=0.5, above]{$6$} (N)
;
\foreach \x/\g in {O/90, L/180, N/0, M/60}{
\fill (\x) circle (2pt)+(\g:0.3)node{$\x$};
}
\end{tikzpicture}
\end{center}
\par\shortans{$37$}
\loigiai{
Để chiều dài quãng đường là tối thiểu, cán bộ thanh tra cần xuất phát từ $L$, đi qua mỗi thị trấn $M$, $N$, $O$ đúng một lần trước khi qua lại $L$ và ưu tiên chọn con đường ngắn hơn trong hai con đường nối hai thị trấn.\\
Ta có quãng đường tối ưu là $L\rightarrow M \rightarrow N \rightarrow O \rightarrow L$ với độ dài là $16+8+6+7=37$ km.
}
\end{ex}

\begin{ex}%[1C2H3-1]%[TEX ĐỀ MOON 2025]%[Nguyễn Cường]
\immini[thm]
{
Công ty giao hàng nhanh có $4$ kho hàng $A$, $B$, $C$ và $D$. Quản lý muốn lên kế hoạch cho xe giao hàng đi qua tất cả các kho hàng để lấy hàng và quay lại kho hàng ban đầu, với điều kiện là mỗi kho hàng chỉ ghé qua một lần. Khoảng cách giữa các kho hàng (km) được mô tả trong hình bên. Quãng đường ngắn nhất để xe giao hàng hoàn thành việc lấy hàng ở các kho và quay trở lại kho hàng ban đầu là bao nhiêu?
}
{
\begin{tikzpicture}[scale=0.8,>=stealth, font=\footnotesize, line join=round, line cap=round]
\coordinate (A) at (0,0);
\coordinate (B) at (1,-2);
\coordinate (C) at (0,-4);
\coordinate (D) at (5,-3);
\draw (A)--(C)--(D)--cycle (A)--(B)--(C) (B)--(D) ($(A)!0.5!(C)$)node[left]{$3$} ($(A)!0.5!(B)$)node[above right,yshift=-0.2cm]{$3$} ($(B)!0.5!(D)$)node[below]{$4$} ($(B)!0.5!(C)$)node[below right]{$2$} ($(C)!0.5!(D)$)node[below]{$5$} ($(A)!0.5!(D)$)node[above right]{$7$};
\foreach \x/\g in {A/90,B/180,C/-135,D/0}
\fill[black] (\x) circle(1pt) +(\g:4mm) node {$\x$};
\end{tikzpicture}
}
\shortans{$15$}
\loigiai{
\begin{itemize}
\item Lộ trình $A \to B \to C \to D \to A$.\\
Tổng quãng đường=$AB+BC+CD+DA=3+2+5+7=17$ km.
\item Lộ trình $A \to B \to D \to C \to A$.\\
Tổng quãng đường=$AB+BD+DC+CA=3+4+5+3=15$ km.
\item Lộ trình $A \to C \to B \to D \to A$.\\
Tổng quãng đường=$AC+CB+BD+DA=3+2+4+7=16$ km.
\item Lộ trình $A \to C \to D \to B \to A$.\\
Tổng quãng đường=$AC+CD+DB+BA=3+5+4+3=15$ km.
\item Lộ trình $A \to D \to B \to C \to A$.\\
Tổng quãng đường=$AD+DB+BC+CA=7+4+2+3=16$ km.
\item Lộ trình $A \to D \to C \to B \to A$.\\
Tổng quãng đường=$AD+DC+CB+BA=7+5+2+3=17$ km.
\end{itemize}
\textit{Lưu ý: Các lộ trình theo chiều ngược lại sẽ có cùng tổng độ dài.}\\
Vậy quãng đường ngắn nhất để xe giao hàng hoàn thành việc lấy hàng ở các kho và quay trở lại kho hàng ban đầu là $15$ km.
}
\end{ex}

\begin{ex}%[1H8H5-4]%[TEX ĐỀ MOON 2025]%[Huỳnh Thanh Chí]
Cho tứ diện đều $ABCD$ có cạnh $2$. Khoảng cách giữa hai đường thẳng $AB$ và $CD$ bằng bao nhiêu? (làm tròn kết quả đến hàng phần trăm).

\shortans[]{$1{,}41$}
\loigiai{
\immini{Gọi $M$, $N$ lần lượt là trung điểm $AB$ và $CD$.\\
Ta có $\triangle ACD=\triangle BCD$ nên $AN=BN$ ($2$ đường trung tuyến tương ứng).\\
Suy ra $\triangle AMB$ cân tại $M$, suy ra $MN\perp AB$.\\
Chứng minh tương tự ta có $MN\perp CD$.\\
Vậy $MN$ là đoạn vuông góc chung của $AB$ và $CD$.\\
Khi đó $MN=\mathrm{d}\left(AB,CD\right)$.
}{\begin{tikzpicture}[scale=0.9,font=\footnotesize,line join=round,line cap=round,>=stealth]
\def\a{4}
\path 	(0:0) coordinate (B)
++(0:\a) coordinate (D)
++(-120:\a/2) coordinate (C)
($(B)+(70:\a)$) coordinate (A)
($(A)!1/2!(B)$) coordinate (M)
($(C)!1/2!(D)$) coordinate (N)
;
\draw[dashed] 	(B)--(D) (M)--(N)--(B);
\draw			(B)--(A)--(D) (A)--(C) (A)--(N)
(B)--(C)--(D);
\foreach \x/\g in {A/90,B/180,C/-45,D/0,M/135,N/-45}
\fill[black] 	(\x) circle (1pt)
($(\g:3mm)+(\x)$) node {$\x$};
%Hình chóp S.ABC có SA vuông góc đáy
\end{tikzpicture}}
Xét $\triangle MBN$ vuông tại $M$ có \allowdisplaybreaks
\begin{eqnarray*}
MN&=&\sqrt{BN^2-MB^2}=\sqrt{\left(\dfrac{2\sqrt{3}}{2}\right)^2-1^2}\\
&=&\sqrt{2}\approx 1{,}41.
\end{eqnarray*}
}
\end{ex}

\begin{ex}%[1C2H2-2]%[TexDeMoon2025]%[NguyenKieuNhaTu]
Cho tứ diện $ABCD$, một con bọ đang đậu ở đỉnh $A$ của tứ diện. Mỗi lần nghe một tiếng trống thì nó nhảy sang một đỉnh bất kì của tứ diện $ABCD$ mà kề với đỉnh nó đang đậu. Hỏi sau $4$ tiếng trống nó có bao nhiêu cách trở về đỉnh $A$?
\shortans[]{$21$}
\loigiai{
Sử dụng sơ đồ cây biểu diễn các cách đi ta được $21$ cách.
\begin{center}
\begin{tikzpicture}[scale=.85, font=\footnotesize, line join=round, line cap=round,>=stealth]
\node (a) at (0,0) {A};
\node (d1) at (6,-1) {D};
\node (c1) at (0,-1) {C};
\node (b1) at (-6,-1) {B};
\node (c21) at (-8,-2) {C};
\node (a21) at (-6,-2) {A};
\node (d21) at (-4,-2) {D};
\node (a22) at (-2,-2) {A};
\node (b22) at (0,-2) {B};
\node (d22) at (2,-2) {D};
\node (c23) at (4,-2) {C};
\node (a23) at (6,-2) {A};
\node (b23) at (8,-2) {B};
\node (b31) at (-10,-3) {B};
\node (d31) at (-9,-3) {D};
\node (b32) at (-8,-3) {B};
\node (c32) at (-7,-3) {C};
\node (d32) at (-6,-3) {D};
\node (b33) at (-5,-3) {B};
\node (c33) at (-4,-3) {C};
\node (b43) at (-3,-3) {B};
\node (c43) at (-2,-3) {C};
\node (d43) at (-1,-3) {D};
\node (c53) at (0,-3) {C};
\node (d53) at (1,-3) {D};
\node (c63) at (2,-3) {C};
\node (b63) at (3,-3) {B};
\node (b73) at (4,-3) {B};
\node (d73) at (5,-3) {D};
\node (b83) at (6,-3) {B};
\node (c83) at (7,-3) {C};
\node (d83) at (8,-3) {D};
\node (c93) at (9,-3) {C};
\node (d93) at (10,-3) {D};

\foreach \from/\to in {a/b1,a/c1,a/d1,b1/c21,b1/a21,b1/d21,c1/a22,c1/b22,c1/d22,d1/c23,d1/a23,d1/b23,c21/b31,c21/d31,a21/b32,a21/c32,a21/d32,d21/b33,d21/c33,a22/b43,a22/c43,a22/d43,b22/c53,b22/d53,d22/c63,d22/b63,c23/b73,c23/d73,a23/b83,a23/c83,a23/d83,b23/c93,b23/d93}
\draw[->]
(\from)--(\to)
;
\end{tikzpicture}
\end{center}
}
\end{ex}

\begin{ex}%[2H5H3-1]%[TEX ĐỀ MOON 2025]%[Lê Hữu Kiệt]
Trong không gian với hệ trục tọa độ $Oxyz$, có tất cả bao nhiêu giá nguyên của $m$ để phương trình $x^2+y^2+z^2+2(m+2)x-2(m-1)z+3m^2-5=0$ là phương trình một mặt cầu?
\shortans{$7$}
\loigiai{
Phương trình có dạng $x^2+y^2+z^2-2ax-2by-2cz+d=0$ là phương trình mặt cầu khi và chỉ khi $a^2+b^2+c^2-d>0$.\\
Từ phương trình đa cho ta có $a=-(m+2)$, $b=0$, $c=m-1$ và $d=3m^2-5$.\\
Khi đó phương trình đã cho là phương trình mặt cầu khi và chỉ khi
\begin{eqnarray*}
&& [-(m+2)]^2+(m-1)^2-\left(3m^2-5\right)>0 \\
&\Leftrightarrow& -m^2+2m+10>0 \\
&\Leftrightarrow& 1-\sqrt{11}<m<1+\sqrt{11}.
\end{eqnarray*}
Do $m$ nguyên nên $m\in\{-2;-1;0;1;2;3;4\}$.\\
Vậy có $7$ giá trị nguyên của $m$ để phương trình đã cho là phương trình mặt cầu.
}
\end{ex}

\begin{ex}%[2H5H2-8]%[TEX ĐỀ MOON 2025]%[Huỳnh Thanh Chí]
Khi gắn hệ tọa độ $Oxy$ (đơn vị trên mỗi trục tính theo kilomet) vào một sân bay, mặt phẳng $(Oxy)$ trùng với mặt sân bây. Một máy bay, bay theo đường thẳng từ vị trí $A(5;0;5)$ đến vị trí $B(10;10;3)$, sau đó tiếp tục bay thẳng và hạ cánh tại vị trí $M(a;b;0)$. Giá trị của $a+b$ bằng bao nhiêu (viết kết quả dưới dạng số thập phân)?

\shortans[]{$42{,}5$}
\loigiai{
Vectơ $\overrightarrow{AB} = (10-5; 10-0; 3-5) = (5; 10; -2)$.\\
Phương trình tham số của đường thẳng $AB$ là $\heva{& x = 5 + 5t \\& y = 10t \\& z = 5 - 2t.}$\\
Vì điểm $M(a; b; 0)$ thuộc đường thẳng $AB$ và nằm trên mặt phẳng $(Oxy)$ nên $z = 0$.\\
Ta có: $5 - 2t = 0 \Rightarrow t = \dfrac{5}{2} = 2{,}5$.\\
Thay $t = 2{,}5$ vào phương trình tham số, ta được:
\[\heva{& a = 5 + 5 \cdot 2{,}5 = 17{,}5 \\
& b = 10 \cdot 2{,}5 = 25.}\]
Vậy $M(17{,}5; 25; 0)$.\\
Do đó, $a + b = 17{,}5 + 25 = 42{,}5$.
}
\end{ex}

\begin{ex}%[50 Đề minh họa tốt nghiệp 2025 - Đề 13]%[Lê Hữu Kiệt - Lê Quân]%[2H5H2-7]
Cho hình lăng trụ đứng $ABC.A'B'C'$ có đáy $ABC$ là tam giác vuông tại $C$, $AC=3a$, $BC=4a$ và góc giữa đường thẳng $B'C$ và mặt phẳng $(ABC)$ bằng $45^\circ$. Tính sin của góc giữa đường thẳng $B'C$ và mặt phẳng $(ABC')$ (làm tròn kết quả đến hàng phần trăm).
\par\shortans{$0{,}73$}
\loigiai{
Ta có $CC'\perp (ABC)$ nên $C$ là hình chiếu của $C'$ trên $(ABC)$.\\
Khi đó $\left(BC',(ABC)\right)=\left(BC',BC\right)=\widehat{C'BC}=45^\circ$.\\
Xét $\triangle C'BC$ vuông tại $C$, ta có $CC'=BC\tan\widehat{C'BC}=4a\cdot\tan 45^\circ=4a$.\\
Chọn hệ trục tọa độ $Oxyz$, với $C \equiv O$, $A\in Ox$, $B\in Oy$ và $C'\in Oz$, đơn vị trên các trục là $a$.
\begin{center}
\tdplotsetmaincoords{75}{110}
\begin{tikzpicture}[font=\footnotesize, >=stealth, tdplot_main_coords]
\path
(0,0,0) coordinate (C)+(0,0,4) coordinate (C')
(3,0,0) coordinate (A)+(0,0,4) coordinate (A')
(0,4,0) coordinate (B)+(0,0,4) coordinate (B')
;
\draw (A')--(C')--(B')--cycle (A)--(A') (B)--(B') (A)--(B);
\draw[dashed] (A)--(C)--(B) (C)--(C') (B')--(C) (A)--(C')--(B);
\draw[->] (A)--++(3,0,0)node[below]{$x$};
\draw[->] (B)--++(0,1,0)node[right]{$y$};
\draw[->] (C')--++(0,0,1)node[left]{$z$};
\foreach \x/\g in {C/left, A/below, B/below, C'/left, A'/left, B'/right}{
\fill (\x) circle (1pt)node[\g]{$\x$};
}
\end{tikzpicture}
\end{center}
Khi đó tọa độ các điểm là $C(0;0;0)$, $A(3;0;0)$, $B(0;4;0)$, $C'(0;0;4)$, $A'(3;0;4)$, $B'(4;0;4)$.\\
Ta có $\overrightarrow{B'C}=(-4;0;-4)$.\\
Mặt phẳng $(ABC')$ có $\overrightarrow{AB}=(-3;4;0)$, $\overrightarrow{AC'}=(-3;0;4)$ là cặp vectơ chỉ phương.\\
Do đó $\overrightarrow{n}=\left[\overrightarrow{AB},\overrightarrow{AC'}\right]=(16;12;12)$ là một vectơ pháp tuyến của $(ABC')$.\\
Chọn $\overrightarrow{n}'=\dfrac{1}{4}\overrightarrow{n}=(4;3;3)$ là vectơ pháp tuyến của $(ABC')$.\\
Khi đó
\[\sin\left(BC',(ABC')\right)
=\left|\cos\left(\overrightarrow{BC'},\overrightarrow{n}'\right)\right|
=\dfrac{|0\cdot4+(-4)\cdot3+(-4)\cdot3|}{\sqrt{0^2+(-4)^2+(-4)^2}\cdot\sqrt{4^2+3^2+3^2}}
=\dfrac{3}{\sqrt{17}} \approx 0{,}73.\]
}
\end{ex}

\begin{ex}%[2H5H2-6]%[TexDeMoon2025]%[NguyenKieuNhaTu]
Cho hình chóp tam giác $S.ABC$ có $SA$, $AB$, $AC$ đôi một vuông góc. Biết rằng $SA=5$, $AB=3$, $AC=4$. Khoảng cách giữa $SA$ và $BC$ bằng bao nhiêu?
\shortans[]{$2{,}4$}
\loigiai{
\begin{center}
\begin{tikzpicture}[scale=.8, font=\footnotesize, line join=round, line cap=round, >=stealth]
\def\bc{4} % cạnh BC
\def\ba{2} % cạnh BA
\def\h{3.5} % đường cao
\def\gocB{30} % góc B của đáy
\path
(0,0) coordinate (B)
(\gocB:\ba) coordinate (A)
(\gocB:\ba)+(\bc,0) coordinate (C)
($(A)+(90:\h)$) coordinate (S)
(A)--(C)--([turn]0:1)coordinate (y) node[below]{$y$}
(A)--(B)--([turn]0:1)coordinate (x) node[below right]{$x$}
(A)--(S)--([turn]0:1)coordinate (z) node[right]{$z$};
\draw[->] (C)--(y);
\draw[->] (B)--(x);
\draw[->] (S)--(z);
\draw
(B)--(C)--(S)--cycle
(S)--(C);
\draw[dashed] (A)--(C)
(S)--(A)--(B)
;
\fill (A) circle (1pt)+(-40:3mm)node{$A\equiv O$};
\foreach \x/\g in {B/-105,C/-45,S/170}\fill (\x) circle (1pt)+(\g:3mm) node{$ \x $};
\end{tikzpicture}
\end{center}
Chọn hệ trục $Oxy$ như hình vẽ.\\
Ta được $A(0;0;0)$, $B(3;0;0)$, $C(0;4;0)$, $S(0;0;5)$.\\
Đường thẳng $SA$: đi qua $S(0;0;5)$ và $A(0;0;0)$ có 1 VTCP $\overrightarrow{SA}=(0;0;-5)$.\\
Đường thẳng $BC$: đi qua $B(3;0;0)$ và $C(0;4;0)$ có 1 VTCP $\overrightarrow{BC}=(-3;4;0)$.\\
Lại có $\overrightarrow{AC}=(0;4;0)$.
\[\mathrm{d}(SA,BC)=\dfrac{\left|\overrightarrow{AC}\cdot\left[\overrightarrow{SA},\overrightarrow{BC}\right] \right| }{\left|\left[\overrightarrow{SA},\overrightarrow{BC}\right]\right|}=2{,}4.\]
}
\end{ex}

\begin{ex}%[2D6H2-4]%[TEX ĐỀ MOON 2025]%[Lê Hữu Kiệt]
Căn bệnh cúm $A$ đang diễn ra ở một quốc gia Châu Phi có $1\%$ dân số mắc phải. Một phương pháp chuẩn đoán được phát triển có tỷ lệ chính xác là $99\%$. Với những người bị bệnh, phương pháp này sẽ đưa ra kết quả dương tính $99\%$ số trường hợp. Với người không mắc bệnh, phương pháp này cũng chuẩn đoán đúng $99$ trong $100$ trường hợp. Nếu một người kiểm tra và kết quả là dương tính (bị bệnh), xác suất để người đó thực sự bị bệnh là bao nhiêu?
\shortans{$0{,}5$}
\loigiai{
Gọi $B$ là biến cố \lq\lq người được kiểm tra bị bệnh\rq\rq\,và $D$ là biến cố \lq\lq người được kiểm tra có kết quả dương tính\rq\rq.\\
Từ dữ kiện đề bài ta có $P(B)=1\%$, $P\left(\overline{B}\right)=99\%$, $P(D\mid B)=1\%$.\\
Với người không mắc bệnh, phương pháp này cũng chuẩn đoán đúng $99$ trong $100$ trường hợp, tức $P\left(\overline{D}\mid\overline{B}\right)=99\%$, suy ra $P\left(D\mid\overline{B}\right)=1\%$.\\
Áp dụng công thức xác suất toàn phần, ta có xác suất người kiểm tra là dương tính là
\[P(D)=P(B)P(D\mid B)+P\left(\overline{B}\right)P\left(D\mid\overline{B}\right)=1\%\cdot99\%+99\%\cdot1\%=0{,}0198.\]
Nếu một người kiểm tra và kết quả là dương tính (bị bệnh), xác suất để người đó thực sự bị bệnh là $P(B\mid D)$. Áp dụng công thức Bayes ta có
\[P(B\mid D)=\dfrac{P(B)P(D\mid B)}{P(D)}=\dfrac{1\%\cdot99\%}{0{,}0198}=0{,}5.\]
Vậy nếu một người kiểm tra và kết quả là dương tính (bị bệnh), xác suất để người đó thực sự bị bệnh là $0{,}5$.
}
\end{ex}

\begin{ex}%[2D4H3-1]%[TEX ĐỀ MOON 2025]%[Lê Hữu Kiệt]
Trường THPT Bến Tre muốn làm một cái cửa nhà hình parabol cho nhà rèn luyện thể chất của nhà trường có chiều cao từ mặt nền nhà đến đỉnh là $2{,}25$ mét, chiều rộng tiếp giáp với mặt đất là $3$ mét. Giá thuê mỗi mét vuông là $1{,}5$ triệu đồng. Vậy số tiền nhà trường phải trả là bao nhiêu triệu đồng?
\shortans{$6{,}75$}
\loigiai{
\immini
{Chọn hệ trục tọa độ $Oxy$ sao cho hai chân của nằm trên $Ox$, đỉnh của của thuộc $Oy$.\\
Gọi $(P)\colon y=ax^2+bx+c$ ($a\ne0$) là đồ thị parabol của cái cửa có điểm đặt của hai chân cửa là $A(-1{,}5;0)$, $B(1{,}5;0)$ và đỉnh $C(0;2{,}25)$ thuộc đồ thị $(P)$.}
{\begin{tikzpicture}[font=\footnotesize, line join=round, line cap=round, >=stealth, scale=1]
\path (-1.5,0) coordinate (A) (1.5,0) coordinate (B) (0,2.25) coordinate (C);
\draw[->] (-1.7,0)--(0,0)node[below left]{$O$}--(2,0)node[below]{$x$};
\draw[->] (0,-0.2)--(0,2.8)node[left]{$y$};
\draw[smooth] plot [domain=-1.5:1.5] (\x,{-(\x)^2+2.25});
\foreach \x/\n/\g in {A/A/135, A/{$-1{,}5$}/-90, B/B/34, B/{$1{,}5$}/-90, C/C/45, C/{$2{,}25$}/160}{
\fill (\x) circle (1pt)+(\g:0.35)node{$\n$};
}
\end{tikzpicture}}
\noindent
Khi đó tạo độ các điểm $A$, $B$, $C$ thỏa phương trình của $(P)$, ta có hệ phương trình
\[\heva{&a(-1{,})^2+b(-1{,}5)+c=0\\&a\cdot1^2+b\cdot1+c=0\\&a\cdot0+b\cdot0+c=2{,}25} \Leftrightarrow \heva{&a=-1\\&b=0\\&c=2{,}25.}\]
Suy ra $(P)\colon y=-x^2+2{,}25$.\\
Diện tích của của là $\displaystyle\int\limits_{-1{,}5}^{1{,}5}\left(-x^2+2{,}25\right)\mathrm{d}x=4{,}5$ (m$^2$).\\
Số tiền nhà trường phải trả là $4{,}5\cdot1{,}5=6{,}75$ (triệu đồng).
}
\end{ex}

\begin{ex}%[2D4H3-1]%[TEX Đề Moon 2025]%[Vũ Hồng Toàn]
\immini[thm]
{
Cho đồ thị $(C)$ của hàm đa thức bậc ba và parabol $(P)$ có trục đối xứng vuông góc với trục hoành như hình vẽ bên. Biết phần hình phẳng giới hạn bởi $(C)$ và $(P)$ (phần tô đậm của hình vẽ) có diện tích bằng $\dfrac{m}{n}$ ($m$, $n\in \mathbb{N}$; $\dfrac{m}{n}$ là phân số tối giản). Tính $m+n$.
}
{
\begin{tikzpicture}[scale=0.78,>=stealth, font=\footnotesize, line join=round, line cap=round]
\def\xmin{-2} \def\xmax{4}
\def\ymin{-3} \def\ymax{3}
\draw[->] (\xmin,0)--(\xmax,0) node [below]{$x$};
\draw[->] (0,\ymin)--(0,\ymax) node [left]{$y$};
\node at (0,0) [below right]{$O$};
\draw[dashed] (-1,0)node[above,xshift=-0.15cm]{$-1$}--(-1,-2)--(0,-2)node[above right]{$-2$}--(2,-2)--(2,0)node[above]{$2$} (1,0)node[above,xshift=0.1cm]{$1$} (0,2)node[above right]{$2$} (2.6,-2.7)node[]{$(P)$} (3.5,2.8)node[]{$(C)$};
\clip (\xmin+0.1,\ymin+0.1) rectangle (\xmax-0.5,\ymax-0.1);
\draw[smooth,samples=300] plot(\x,{-(\x)^2+(\x)});
\draw[smooth,samples=300] plot(\x,{(\x)^3-3*(\x)^2+2});
\fill[gray,opacity=0.6] plot[domain=-1:2](\x,{-(\x)^2+(\x)})--plot[domain=2:-1](\x,{(\x)^3-3*(\x)^2+2})--cycle;
\end{tikzpicture}
}
\shortans{$49$}
\loigiai{
Gọi $A(-1;-2)$, $B(1;0)$, $C(2;-2)$,$D(0;2)$.\\
Giả sử $(C)$ có phương trình $f(x)=a x^3+bx^2+cx+d, a\ne0$ và $(P)$ có phương trình $g(x)= a_1x^2+b_1x+c_1, a_1\ne 0$.\\
Vì $A,B,C,D\in f(x)$ nên ta có hệ phương trình
\[\heva{&-a+b-c+d=-2\\&a+b+c+d=0\\&8a+4b+2c+d=-2\\&d=2}\Leftrightarrow\heva{&a=1\\&b=-3\\&c=0\\&d=2.}\]
Do đó $f(x)=x^3-3x^2+2$.\\
Tương tự ta cũng có $A,B,C\in g(x)$ nên ta có hệ phương trình
\[\heva{&a_1-b_1+c_1=-2\\&a_1+b_1+c_1=0\\&4a_1+2b_1+c_1=-2}\Leftrightarrow\heva{&a_1=-1\\&b_1=1\\&c_1=0.}\]
Do đó $g(x)=-x^2+x$.
Khi đó diện tích phần tô đậm trong hình vẽ là
\allowdisplaybreaks
\begin{eqnarray*}
S=\int\limits_{-1}^2\big|f(x)-g(x)\big|\mathrm{d}x&=&\int\limits_{-1}^2\big|x^3-2x^2-x+2\big|\mathrm{d}x\\
&=&	\int\limits_{-1}^1\big|x^3-2x^2-x+2\big|\mathrm{d}x+\int\limits_{1}^2\big|x^3-2x^2-x+2\big|\mathrm{d}x\\
&=&\left|\int\limits_{-1}^1\big(x^3-2x^2-x+2\big)\mathrm{d}x\right|+\left|\int\limits_{1}^2\big(x^3-2x^2-x+2\big)\mathrm{d}x\right|\\
&=&\left|\dfrac{8}{3}\right|+\left|\dfrac{5}{12}\right|=\dfrac{37}{12}.
\end{eqnarray*}
Suy ra $m=37$, $n=12$. Vậy $m+n=49$.
}
\end{ex}

\begin{ex}%[2D1H5-8]%[TEX ĐỀ MOON 2025]%[Nguyễn Văn Hiệp]
Một công ty sản xuất dụng cụ thể thao nhận được một đơn đặt hàng sản xuất $8\,000$ quả bóng tennis. Công ty này sở hữu một số máy móc, mỗi máy có thể sản xuất $30$ quả bóng trong một giờ. Chi phí thiết lập các máy này là $200$ nghìn đồng cho mỗi máy. Khi được thiết lập, hoạt động sản xuất sẽ hoàn toàn diễn ra tự động dưới sự giám sát. Số tiền phải trả cho người giám sát là $192$ nghìn đồng một giờ. Số máy móc công ty nên sử dụng là bao nhiêu để chi phí hoạt động là thấp nhất?
\shortans{$16$}
\loigiai{
\textbf{Bước 1: Thiết lập hàm chi phí} \\
Gọi $x$ là số máy ($x>0$, $x\in \mathbb{N}^*$), thời gian sản xuất
$t = \dfrac{8\,000}{30x}$ (giờ).\\
Chi phí
\[
C(x) = 200x + 192 \times \dfrac{8\,000}{30x}.
\]
\textbf{Bước 2: Tìm cực tiểu} \\
Ta có
\[
C'(x) = 200 - \dfrac{51200}{x^2} = 0 \Rightarrow x = 16.
\]
Bảng biến thiên
\begin{center}
\begin{tikzpicture}
\tkzTabInit[espcl=3.5,lgt=2.5,deltacl=1]
{$x$/0.7,$C'(x)$/1,$C(x)$/3}
{$0$,$16$,$+\infty$}
\tkzTabLine{,-,0,+,}
\tkzTabVar{+/$+\infty$,-/$C\left(16\right)$,+/$+\infty$}
\end{tikzpicture}
\end{center}
Vậy để chi phí hoạt động là thấp nhất, số máy móc công ty nên sử dụng là $16$ máy.
}
\end{ex}

\begin{ex}%[2D1H3-6]%[TEX ĐỀ MOON 2025]%[Lê Hữu Kiệt]
Trận bóng đá giao hữu giữa đội tuyển Việt Nam và Thái Lan ở sân vận động Mỹ Đình có sức chứa $55\,000$ khán giả. Ban tổ chức bán vé với giá mỗi vé là $100$ nghìn đồng, số khán giả trung bình đến sân xem bóng đá là $27\,000$ người. Qua thăm dò dư luận, người ta thấy rằng mỗi khi giá vé giảm thêm $10$ nghìn đồng, sẽ có thêm khoảng $3\,000$ khán giả. Hỏi ban tổ chức nên đặt giá vé là bao nhiêu để doanh thu từ tiền bán vé là lớn nhất với đơn vị tính giá vé là nghìn đồng?
\shortans{$95$}
\loigiai{Goi $x$ là số lần giảm $10$ nghìn đồng ($0\leq x\leq10$).\\
Khi đó, số tiền mỗi vé là  $100-10x$ (nghìn đồng), số khán giả là $27\,000+3\,000x$.\\
Doanh thu từ việc bán vé là $T(x)=(100-10x)(27\,000+3\,000x)=-30\,000x^2+30\,000x+2\,700\,000$.\\
Ta có $T'(x)=-60\,000x+30\,000$. Khi đó $T'(x)=0\Leftrightarrow x=0{,}5$.\\
Bảng biến thiên của $T(x)$ trên đoạn $[0;10]$ là
\begin{center}
\begin{tikzpicture}[font=\footnotesize, line join=round, line cap=round, >=stealth, scale=1]
\tkzTabInit[espcl=4,lgt=1.5,deltacl=1]
{$x$/0.7,$T'(x)$/0.7,$T(x)$/2}
{$0$, $0{,}5$, $10$}
\tkzTabLine{,+,$0$,-,}
\tkzTabVar{-/$2\,700\,000$, +/$2\,707\,500$, -/$0$}
\end{tikzpicture}
\end{center}
Từ bảng biến thiên suy ra hàm số $T(x)$ đạt giá trị lớn nhất khi $x=0{,}5$.\\
Vậy giá vé ban tổ chức nên đặt là $100-10\cdot0{,}5=95$ (nghìn đồng).
}
\end{ex}

\begin{ex}%[2D1H2-7]%[TEX ĐỀ MOON 2025]%[Nguyễn Cường]
Độ giảm huyết áp của một bệnh nhân được xác định bởi công thức $G(x)=0{,}024x^2(30-x)$, trong đó $x$ là liều lượng thuốc tiêm cho bệnh nhân cao huyết áp ($x$ được tính bằng mg). Tìm lượng thuốc để tiêm cho bệnh nhân cao huyết áp để huyết áp giảm nhiều nhất.
\shortans{$20$}
\loigiai
{
Ta có $G(x)=-0{,}024x^3+0{,}72x^2$ với $0\le x\le 30$.\\
Đạo hàm $G'(x)=-0{,}072x^2+1{,}44x$.\\
Xét $G'(x)=0\Leftrightarrow -0{,}072x^2+1{,}44x=0\Leftrightarrow\hoac{&x=0\\&x=20.}$\\
Lúc này $G(0)=G(30)=0$ và $G(20)=96$.\\
Vậy lượng thuốc tiêm cho bệnh nhân cao huyết áp để giảm huyết áp nhiều nhất là $20$\,(mg).
}
\end{ex}

\begin{ex}%[2H2H2-6]
Một chiếc máy bay không người lái bay lên tại một điểm. Sau một thời gian bay, chiếc máy bay cách điểm xuất phát về phía Bắc $50$ (km) và về phía Tây $20$ (km), đồng thời cách mặt đất $1$ (km). Xác định khoảng cách của chiếc máy bay với vị trí tại điểm xuất phát của nó (làm tròn kết quả đến hàng phần mười).
\shortans{53{,}9}
\loigiai{
Chọn hệ trục tọa độ $Oxyz$ với gốc $O$ là điểm xuất phát, trục $Ox$ hướng về phía Bắc, trục $Oy$ hướng về phía Tây, trục $Oz$ hướng lên trên.\\
Tọa độ của máy bay là $\mathrm{P}\left(-20;50;1\right)$.\\
Khoảng cách từ máy bay đến điểm xuất phát $O(0;0;0)$ là
$OP=\sqrt{(-20)^2+50^2+1^2} \approx 53{,}9$ (km).
}
\end{ex}

% \paragraph{Mức độ V}
\begin{ex}%[1H8V7-9]%[TEX ĐỀ MOON 2025]%[Huỳnh Thanh Chí]
Người ta cần trang trí một kim tự tháp hình chóp tứ giác đều $S.ABCD$ có cạnh bên bằng $200$ m, góc $\widehat{ASB}=15^\circ$ bằng đường gấp khúc dây đèn led vong quanh kim tự tháp $AEFGHIJKLS$. Trong đó điểm $L$ cố định và $LS=40$ m.
\begin{center}
\begin{tikzpicture}[scale=1,>=stealth, font=\footnotesize, line join=round, line cap=round]
\coordinate (A) at (-1.9,-1.6);
\coordinate (B) at (0,0);
\coordinate (D) at (1.6,-1.6);
\coordinate (C) at ($(B)+(D)-(A)$);
\coordinate (O) at ($(A)!1/2!(C)$);
\coordinate (S) at ($(O)+(0,4)$);
\coordinate (L) at ($(S)!0.2!(A)$);
\coordinate (K) at ($(S)!0.28!(D)$);
\coordinate (J) at ($(S)!0.4!(C)$);
\coordinate (I) at ($(S)!0.65!(B)$);
\coordinate (H) at ($(S)!0.45!(A)$);
\coordinate (G) at ($(S)!0.55!(D)$);
\coordinate (F) at ($(S)!0.7!(C)$);
\coordinate (E) at ($(S)!0.8!(B)$);
\draw (S)--(A)--(D)--(C)--cycle (S)--(D) (F)--(G)--(H) (J)--(K)--(L);
\draw[dashed] (A)--(B)--(C) (S)--(B) (A)--(E)--(F) (H)--(I)--(J);
\foreach \x/\g in {S/90,A/-150,B/-60,C/0,D/-45,E/170,F/45,G/-30,H/170,I/140,J/30,K/60,L/170}
\fill[black] (\x) circle (1pt) ($(\g:3mm)+(\x)$) node {$\x$};
\end{tikzpicture}
\end{center}
Hỏi khi đó cần dùng ít nhất bao nhiêu mét dây đèn led để trăng trí? (làm tròn đến hàng đơn vị).

\shortans[]{$263$}
\loigiai{
Ta trải hình chóp tứ giác đều thành vẽ như sau
\begin{center}
\begin{tikzpicture}[scale=1.5,font=\footnotesize,line join=round,line cap=round,>=stealth]
\def\a{4}
\def\r{15}
\path
(0,0) coordinate (S)
(-135:\a) coordinate (A)
(-135+\r:\a) coordinate (D)
($(A)!1!-90:(D)$) coordinate (B_2)
($(D)!1!90:(A)$) coordinate (C_2)
;
\foreach \x/\i in {C/2,B/3,A_1/4,D_1/5,C_1/6,B_1/7,A_2/8}{
\path
(-135+\i*\r:\a) coordinate (\x)
;
\draw (S)--(\x);
}
\foreach \x/\y/\i in {A/L/1,D/K/2,C/J/3,B/I/4,A_1/H/5,D_1/G/6,C_1/F/7,B_1/E/8,
A_1/H_1/7,B/I_1/6,C/J_1/5,D/K_1/4
}{
\path
($(S)!1/9*\i!(\x)$) coordinate (\y)
;}
\draw (S)--(D)--(C_2)--(B_2)--(A)--(S) (A)--(D)--(C)--(B)--(A_1)--(D_1)--(C_1)--(B_1)--(A_2)
(L)--(K)--(J)--(I)--(H)--(G)--(F)--(E)--(A_2)--(L)
%	(K1)--(J1)--(I1)--(H1)
;

\foreach \x/\g in {A/135,D/-65,S/180,C/-90,B/-90,A_1/-90,D_1/-90,C_1/-60,B_1/-45,A_2/0,C_2/-135,B_2/-90}
\fill 	(\x) circle (1pt)
($(\g:3mm)+(\x)$) node {$\x$};
\foreach \x/\g in {L/135,K/-90,J/-90,I/-90,H/-90,G/-90,F/-90,E/-90}
\fill 	(\x) circle (1pt)
($(\g:3mm)+(\x)$) node {$\x$};
\end{tikzpicture}
\end{center}
Ta có $T=SL+LK+KJ+\ldots+EA_2\ge SL+LA_2$ (vì $SL$ không đổi).\\
Để sợi dây trang trí ngắn nhất thì $T=SL+LA_2$.\\
Ta có $\widehat{LSA_2}=15^\circ \cdot 8=120^\circ$.\\
Áp dụng định lí cosin vào $\triangle SLA_2$ có
\allowdisplaybreaks
\begin{eqnarray*}
LA_2=\sqrt{SL^2+SA_2^2-2\cdot SL\cdot SA_2\cdot\cos \widehat{SLA_2}}=40\sqrt{31}.
\end{eqnarray*}
Vậy $T=40+40\sqrt{31}\approx 263$.
}
\end{ex}

\begin{ex}%[1H8V7-4]%[TEX Đề Moon 2025]%[Vũ Hồng Toàn]
Cho hình lăng trụ $ABC.A'B'C'$ có đáy $ABC$ là tam giác đều cạnh bằng $\sqrt{3}$. Hình chiếu vuông góc của $A'$ lên mặt phẳng $(ABC)$ trùng với trọng tâm tam giác $ABC$. Biết khoảng cách giữa hai đường thẳng $AA'$ và $BC$ bằng $\dfrac{3}{4}$. Tính thể tích $V$ của khối lăng trụ $ABC.A'B'C'$ (kết quả làm tròn đến hàng phần trăm).
\shortans{$0{,}75$}
\loigiai{
\begin{center}
\begin{tikzpicture}[scale=1, line join = round, line cap=round,>=stealth,font=\footnotesize,declare function={a=4; b=0.54*a; h=3.5;goc=-50;}]
\path
(0,0) coordinate (A)
(a,0) coordinate (C)
(goc:b) coordinate (B)
($(B)!.5!(C)$)coordinate (M)
($(A)!2/3!(M)$)coordinate (G)++(0,h)coordinate (A')
($(A')+(a,0)$)coordinate (C')
($(A')+(goc:b)$)coordinate (B')
($(A)!(G)!(A')$)coordinate (H)
($(A)!(M)!(A')$)coordinate (I)
;
\draw[dashed] (M)--(A)--(C) (M)--(A')--(G) (G)--(H) (M)--(I);
\draw (A')--(B')--(C')--cycle (A')--(A)--(B)--(B') (M)--(B)--(C)--(C');
\foreach \x/\goc in {A/220,B/-40,C/0,G/-120,A'/180,B'/70,C'/0,M/-30,H/180,I/130}{
\draw[fill] (\x) circle (1pt) node[shift={(\goc:7pt)},font=\small]{$\x$};
}
\end{tikzpicture}
\end{center}
Gọi $G$ là trọng tâm tam giác $ABC$ và $M$ là trung điểm của $BC$.\\
Ta có $A'G \perp(ABC)$ nên $A'G \perp BC$; $BC \perp AM \Rightarrow BC \perp\left(MAA'\right)$.\\
Kẻ $MI \perp AA'$; $BC \perp IM$ nên $\mathrm{d}\left(AA', BC\right)=I M=\dfrac{3}{4}$.\\

Kẻ $GH \perp AA'$, ta có
\begin{itemize}
\item $\dfrac{A G}{A M}=\dfrac{G H}{I M}=\frac{2}{3} \Leftrightarrow G H=\dfrac{AG}{AM}\cdot IM=\dfrac{2}{3} \cdot \dfrac{3}{4}=\dfrac{1}{2}$.
\item $AM=\dfrac{AB\sqrt{3}}{2}=\dfrac{3}{2}$; $AG=\dfrac{2}{3}AM=\dfrac{2}{3}\cdot\dfrac{3}{2}=1$;
\item $\dfrac{1}{H G^2}=\dfrac{1}{A'G^2}+\dfrac{1}{A G^2} \Leftrightarrow A'G=\dfrac{AG \cdot HG}{\sqrt{AG^2-H G^2}}=\dfrac{1 \cdot \dfrac{1}{2}}{\sqrt{1-\left(\dfrac{1}{2}\right)^2}}=\dfrac{\sqrt{3}}{3}$.
\end{itemize}
Vậy $V_{ABC.A'B'C'}=A'G \cdot S_{A B C}=\dfrac{\sqrt{3}}{3} \cdot \dfrac{3 \sqrt{3}}{4}=\dfrac{3}{4}=0{,}75$.
}
\end{ex}

\begin{ex}%[1H8V7-3]
Cho hình chóp $S.ABCD$ có đáy $ABCD$ là hình vuông cạnh $2$, tam giác $SAB$ vuông cân tại và nằm trong mặt phẳng vuông góc với đáy. Gọi $(P)$ là mặt phẳng chứa $CD$ và vuông góc với $(ABCD)$. Trên $(P)$ lấy điểm $M$ bất kỳ, thể tích khối tứ diện $SAMB$ bằng bao nhiêu? \textit{(làm tròn kết quả đến hàng phần trăm)}.
\shortans{$0{,}67$}
\loigiai{
\begin{center}
\begin{tikzpicture}[scale=1, font=\footnotesize,>=stealth]
%Gán số liệu.
\def\canhAD{4};\def\canhBA{2};\def\gocBAD{-130};\def\h{3};\def\xdinhS{-1};
%Gán tọa độ.
\coordinate (A) at (0,0);
\coordinate (B) at ($(A)+(\gocBAD:\canhBA)$);
\coordinate (C) at ($(B)+(0:\canhAD)$);
\coordinate (D) at ($(A)+(0:\canhAD)$);
\coordinate (S) at ($(A)+(\xdinhS,\h)$);
\coordinate (O) at (intersection of A--C and B--D);
\coordinate (H) at ($(A)!0.5!(B)$);
\coordinate (M) at ($(C)!0.5!(D)$);
\coordinate (a) at ($(S)+(C)-(B)$);
\coordinate (b) at ($(a)+(D)-(C)$);
%Vẽ khối chóp S.ABCD.
\draw (B)--(S)--(C)--cycle (D)--(C) (C)--(a)--(b)--(D);
\draw[dashed] (A)--(D) (A)--(C) (B)--(D) (S)--(A)--(B) (S)--(H) (S)--(D);
\draw pic[draw, angle radius=3mm, angle eccentricity=1.5]{right angle = S--H--A};
\draw pic[draw, angle radius=10mm, angle eccentricity=1.5]{ angle = a--b--D};
\draw($(b)-(0.3,0.7)$) node {P};
%Gán nhãn.
\foreach \x/\y in {A/180,B/-90,C/-90,D/0,S/90,O/225, H/-90, M/0}{\fill (\x) circle(1pt) ($(\x)+(\y:0.3cm)$) node{$\x$};}
\end{tikzpicture}
\end{center}
Gọi $H$ là trung điểm $AB$, vì tam giác $SAB$ vuông cân tại $S$ nên $SH\perp AB$ và $SH=\dfrac{AB}{2}=a$.\\
Suy ra diện tích tam giác $SAB$ là $S_{ABS}=\dfrac{AB\cdot SH}{2}=\dfrac{2}{2}=1$.\\
Vì $(SAB)\perp(ABCD)$, lại có $SH\perp AB$ nên $SH\perp(ABCD)$.\\
Mặt khác $(SAB)\perp(ABCD)$ và $(P)\perp(ABCD)$ nên $(P)\parallel(SAB)$.\\
Lấy điểm $M$ trên $CD$, vì $CD\parallel(SAB)$ nên $\mathrm{d}(M;(SAB))=\mathrm{d}(C;(SAB))$.\\
Ta có $CB\perp AB$, $CB\perp SH$ nên $CB\perp(SAB)$ hay $\mathrm{d}(C,(SAB))=BC=2\Rightarrow \mathrm{d}(M,(SAB))=2$.\\
Vậy thể tích khối tứ diện $SAMB$ là $V_{SAMB}=\dfrac{1}{3} \mathrm{d}(M,(SAB))\cdot S_{ABS}=\dfrac{1}{3}\cdot 2\cdot 1=\dfrac{2}{3}$.
}
\end{ex}

\begin{ex}%[1H8V7-3]
Cho tứ diện $ABCD$, tam giác $ABC$ vuông cân tại $B$, $DA$ vuông góc với mặt phẳng $(ABC)$, $M$ là trung điểm $AC$, $AB=2$, góc giữa đường thẳng $CD$ với mặt phẳng $(BDM)$ bằng $\alpha$ biết $\sin\alpha=\dfrac{1}{3}$. Thể tích của khối tứ diện $ABCD$ bằng bao nhiêu? (làm tròn kết quả đến hàng phần mười).
\shortans{$1{,}3$}
\loigiai{
\begin{center}


\begin{tikzpicture}[scale=1]
\def\a{4}
\def\h{4}
\path 	(0:0) coordinate (A)
++(0:\a) coordinate (B)
++(-130:4*\a/5) coordinate (C)
($(A)+(90:\h)$) coordinate (D)
($(A)!0.5!(C)$) coordinate (M)
($(D)!(A)!(M)$) coordinate (H)
;
\draw[thick] 	(A)--(C)--(B)
(A)--(D)	(B)--(D)	(C)--(D)--(M) (A)--(H);
\draw[dashed,thick] 	(A)--(B);
\foreach \x /\goc in {A/180,B/0,C/-135,D/90,M/-110,H/150}
\fill[black] (\x) circle (1.5pt)
($(\x)+(\goc:3mm)$) node {$\x$};
\draw pic[draw,angle radius=2mm]{right angle=B--A--D};
\draw pic[draw,angle radius=2mm]{right angle=A--H--M};
\end{tikzpicture}

\end{center}
Tam giác $ABC$ vuông cân tại $B$ có $AB=2 \Rightarrow S_{A B C}=\dfrac{1}{2} \cdot A B \cdot B C=\dfrac{1}{2} \cdot 2 \cdot 2=2$.\\
Kẻ $A H \perp M D$, ta có
$B M \perp A H( \text{ do } B M \perp(D A C)) \Rightarrow AH \perp (BDM)$.\\
Ta có
\begin{align*}
\sin \alpha=\dfrac{\mathrm{d}(C,(B D M))}{C D}
=\dfrac{\mathrm{d}(A,(B D M))}{C D}  =\dfrac{A H}{C D}=\dfrac{1}{3} \tag{1}
\end{align*}
Lại có
\begin{align*}
A H=\dfrac{A D \cdot A M}{\sqrt{A D^2+A M^2}} =\dfrac{A D \cdot \sqrt{2}}{\sqrt{A D^2+(\sqrt{2})^2}};
C D=\sqrt{A D^2+A C^2} =\sqrt{A D^2+8}  \tag{2}
\end{align*}
Từ (1) và (2) suy ra $AD=2$. Khi đó thể tích của khối tứ diện $ABCD$ là
\[ V_{A B C D}=\dfrac{1}{3} \cdot A D \cdot S_{A B C}=\dfrac{1}{3} \cdot 2 \cdot 2  =\dfrac{4}{3} \approx 1{,}3 .\]
}
\end{ex}

\begin{ex}%[1H8V7-3]
Cho khối trụ có trục $OO'=6$. Một khối chóp đều $O.ABCD$ có thể tích bằng $16$ và đáy $ABCD$ nội tiếp đường tròn $(O')$ là đường tròn đáy của khối trụ. Thể tích của khối trụ đã cho là $k\pi$, giá trị của $k$ bằng bao nhiêu?
\shortans[]{$24$}
\loigiai{
\begin{center}
\begin{tikzpicture}[font=\footnotesize,line join=round, line cap=round, >=stealth,scale=0.3]
\path
(0,0) coordinate (A)
(-7.5,-4) coordinate (D)
(7.5,-4) coordinate (C)
(15,0) coordinate (B)
(3.5,10.5) coordinate (O)
($(A)!0.5!(C)$) coordinate (O')
;
\draw (O)--(C) (O)--(D) (B)--(O)--(D)--(C)--(B);
\draw[dashed] (O')--(O)--(A)--(B)--(D)--(A)--(C);
\foreach \x/\pos in{A/180,D/-150,C/-30,B/60,O/50, O'/-100}
\fill (\x) circle(1pt) node[{shift=(\pos:0.25)}]{$\x$};
\end{tikzpicture}
\end{center}
Ta có $V_{O.ABCD}=\dfrac{1}{3}OO'\cdot S_{ABCD}=16\Rightarrow S_{ABCD}=8$. Vì $ABCD$ là hình vuông nên $AB=\sqrt{8}=2\sqrt{2}$.
Ta có khối trụ có bán kính đáy bằng bán kính đường tròn $(O')$ và bằng $\dfrac{1}{2}$ độ dài đường chéo của hình vuông $ABCD$ và bằng $2$.\\ Suy ra thể tích của khối trụ bằng
\[OO'\cdot\pi r^2=6\cdot 2^2\pi=24\pi.\]
Vậy $k=24$.
}
\end{ex}

\begin{ex}%[1H8V6-2]
Cho hình lập phương $ABCD.A'B'C'D'$. Số đo của góc nhị diện $\left[B',A'C,D' \right]$ bằng bao nhiêu độ?
\shortans{120}
\loigiai{
\begin{center}
\begin{tikzpicture}[scale=0.6, font=\footnotesize,line join=round, line cap=round, >=stealth]
\path
(0,0) coordinate (A)
++(-130:3) coordinate (B)
++(0:4) coordinate (C)
($(A)+(C)-(B)$) coordinate (D)
($(A')!0.3!(C)$) coordinate (H)
;
\foreach \i in {A,B,C,D}{
\coordinate (\i') at ($(\i)+(0,4)$);
}
\draw (A')--(B')--(C')--(D')--cycle;
\draw (B)--(B') (C)--(C') (D)--(D')  (B)--(C)--(D) (A')--(C') (B')--(D');
\draw[dashed,thin](B)--(A)--(A')--(C) (A)--(D) (B')--(C)--(D') (D')--(H)--(B');
\draw (A')--(C') node[midway,above right]{$O$};
\foreach \i/\g in {A'/90,B'/90,C'/90,D'/90,A/-90,B/-90,C/-90,D/-90,H/-20}
\fill[black] (\i) circle(1pt)+(\g:5mm)node[scale=1]{$\i$};
\end{tikzpicture}
\end{center}
Ta có $\heva{&A'C'\perp B'O\\&AA'\perp B'O\\&AA',A'C'\in \left( AA'CC'\right) }$ suy ra $B'O\perp\left(AA'CC'\right)\Rightarrow B'O\perp A'C$.\\
Lại có $OH\perp A'C\Rightarrow A'C\perp \left(B'OH \right)\Rightarrow A'C\perp B'H$.\\
Tương tự ta chứng minh $A'C\perp D'H$.\\
Suy ra góc nhị diện $\left[B',A'C,D' \right]$ là góc $\widehat{B'HD'}$.\\
Cho các cạnh hình lập phương là $a$.\\
Ta có $B'D'=a\sqrt{2}$.\\
Xét tam giác $A'B'C$ vuông tại $B'$ ta có
\[\dfrac{1}{B'H^2}=\dfrac{1}{A'B'^2}+\dfrac{1}{B'C^2}=\dfrac{1}{a^2}+\dfrac{1}{2a^2}=\dfrac{3}{2a^2}.\]
Suy ra $B'H=a\sqrt{\dfrac{2}{3}}$.\\
Tương tự $D'H=a\sqrt{\dfrac{2}{3}}$.\\
Ta có $\cos \widehat{B'HD'}=\dfrac{B'H^2+HD'^2-B'D'^2}{2\cdot B'H\cdot HD'}\Rightarrow \widehat{B'HD'}=120^\circ$.\\
Vậy số đo của góc nhị diện $\left[B',A'C,D' \right]$ bằng $120^\circ$.
}
\end{ex}

\begin{ex}%[1H8V5-4]%[Tex đề Moon 2025]%[Nguyễn Hồng Thạch]
Cho hình chóp $S.ABC$ có đáy $ABC$ là tam giác vuông cân tại $A$, tam giác $SBC$ là tam giác đều cạnh $1$ và thuộc mặt phẳng vuông góc với đáy. Tính khoảng cách giữa hai đường thẳng $SA$ và $BC$ (làm tròn kết quả đến hàng phần trăm).
\shortans{$0{,}43$}
\loigiai{
\begin{center}
\begin{tikzpicture}
\def\a{4}
\def\h{4}
\path 	(0:0) coordinate (A)
++(0:\a) coordinate (B)
++(-150:4*\a/5) coordinate (C)
($(C)!1/2!(B)$) coordinate (I)
($(I)+(90:\h)$) coordinate (S)
($(S)!2/3!(A)$) coordinate (H);
\draw (C)--(A) (C)--(B)
(A)--(S)	(B)--(S)	(C)--(S) (S)--(I);
\draw[dashed] (A)--(B) (A)--(I) (I)--(H);
\foreach \x / \goc in 		{A/180,B/0,C/-135,I/-45,S/90,H/135}
\fill (\x) circle (1.5pt)
($(\x)+(\goc:3mm)$) node {$\x$};

\draw pic[draw,angle radius=2mm]{right angle=A--I--S};
\draw pic[draw,angle radius=2mm]{right angle=A--H--I};%Theo chiều dương
\end{tikzpicture}
\end{center}
Gọi $I$ là trung điểm $BC$.\\
Ta có $\triangle SBC$ đều nên $SI\perp BC$.\\
Vì $\heva{&BC=(SBC)\cap (ABC)\\
&SI\subset (SBC)\\&SI\perp BC}$ nên $SI\perp (ABC)$.\\
Suy ra $SI\perp AI$ hay tam giác $SAI$ vuông tại $I$.\\
Từ $I$ kẻ $IH$ vuông góc với $SA$.\\
Ta có $\heva{&BC\perp SI\\&BC\perp AI}\Rightarrow BC\perp (SAI)\Rightarrow BC\perp IH$.\\
Khi đó ta có $IH$ là đoạn vuông góc chung của $SA$ và $BC$.\\
Trong tam giác $SAI$ có $AI=\dfrac{BC}{2}=\dfrac{1}{2}$,\quad $SI=\dfrac{\sqrt{3}}{2}$.\\
$\Rightarrow \dfrac{1}{IH^2}=\dfrac{1}{SI^2}+\dfrac{1}{AI^2}=\dfrac{1}{\left(\dfrac{\sqrt{3}}{2}\right)^2}+\dfrac{1}{\left(\dfrac{1}{2}\right)^2}=\dfrac{16}{3}$.\\
Vậy $\mathrm{d}(SA,BC)=IH=\dfrac{\sqrt{3}}{4}\approx 0{,}43$.
}
\end{ex}

\begin{ex}%[1H8V5-4]%[TEX Đề Moon 2025]%[Võ Nguyên Thạch]
Cho hình chóp $S.ABCD$ có $SA\perp(ABCD)$, đáy $ABCD$ là hình chữ nhật và $AD=6$. Góc giữa cạnh bên $SD$ và mặt đáy bằng $30^\circ$. Khoảng cách giữa hai đường thẳng $AB$ và $SD$ bằng bao nhiêu?
\shortans{3}
\loigiai{
\begin{center}
\begin{tikzpicture}
\def\a{4}
\def\h{4}
\path 	(0:0) coordinate (A)
++(0:\a) coordinate (D)
++(-130:\a/2) coordinate (C)
($(A)+(C)-(D)$) coordinate (B)
($(A)+(90:\h)$) coordinate (S)
(intersection of A--C and B--D) coordinate (O)%giao điểm O
($(S)!(A)!(D)$) coordinate (H);
\draw[dashed,thick] 	(B)--(A)--(D)	(A)--(S) (A)--(H);
\draw[thick] 			(B)-- (C)--(D)
(B)--(S)	(C)--(S)	(D)--(S);
\foreach \x/\g in {A/135,B/-135,C/-45,D/45,S/90,H/45}
\fill[black] 	(\x) circle (1.5pt)
($(\g:3mm)+(\x)$) node {$\x$};
\draw pic["$30^\circ$",draw,angle eccentricity=1.8,angle radius=0.5cm]{angle=S--D--A};
\draw pic[draw,angle radius=3mm]{right angle=D--A--S};%Theo chiều dương
\end{tikzpicture}
\end{center}
Trong mặt phẳng $(SAD)$, kẻ $AH\perp SD=H$.\\
Ta có $\heva{&SA\perp (ABCD)\Rightarrow AH\perp AB\\&AD\perp AB.}$\\
Suy ra $AB\perp (SAD)$, mà $AH\subset (SAD)$ nên $AB\perp AH$.\\
Khi đó, $AH$ là đoạn vuông góc chung của $AB$ và $SD$.\\
Xét tam giác $SAD$ vuông tại $A$, có
\[\tan \widehat{SDA}=\dfrac{SA}{AD}\Leftrightarrow SA=AD\tan \widehat{SDA}=6\tan 30^\circ = 2\sqrt 3.\]
Xét tam giác $SAD$ vuông tại $A$ có đường cao $AH$
\[\dfrac{1}{AH^2}+\dfrac{1}{SA^2}+\dfrac{1}{AD^2}\Leftrightarrow AH=\dfrac{SA\cdot AD}{\sqrt{SA^2+AD^2}}=\dfrac{2\sqrt 3\cdot 6}{\sqrt{(2\sqrt 3)^2+6^2}}=3.\]
}
\end{ex}

\begin{ex}%[1D6V4-6]
Chị Lan vừa mua một chiếc máy tính xách tay mới với giá $25$ triệu đồng vào ngày $20/9/2022$ bằng thẻ tín dụng của ngân hàng Y. Thẻ tín dụng này được phát hành vào ngày $10/9/2022$. Ngân hàng Y có chế độ không tính lãi trong $45$ ngày đầu và cộng thêm khuyến mãi $15$ ngày tiếp theo không tính lãi. Sau thời gian này, ngân hàng sẽ tính lãi với lāi suất $18\%$ /năm (tính lāi kép theo ngày). Chị Lan dự định sē hoàn tiền cho ngân hàng vào ngày $10/12/2022$. Chị Lan phải trả thêm bao nhiêu nghìn đồng tiền lāi so với giá gốc cho ngân hàng vào ngày $10/12/2022$? \textit{(kết quả làm tròn đến hàng đơn vị)}.
\shortans{$260$}
\loigiai{
Ngân hàng miễn lãi trong $45$ ngày đầu và $15$ ngày khuyến mãi $60$ ngày miễn lãi.\\
Thời gian tính lãi là từ ngày $20/11/2022$ đến ngày $10/12/2022$ nên số ngày có lãi là $21$ ngày.\\
Lãi suất hàng năm là $18\%/$ năm.\\
Suy ra lãi suất hàng ngày là $\dfrac{18}{365}$.\\
Số tiền chị Lan phải trả cho ngân hàng vào ngày $10/12/2022$ là
\[T=25\cdot \left(1+\dfrac{18}{365} \%\right)^{21} \approx 25{,}260 \text{ (triệu đồng)}.\]
Số tiền lãi chị Lan phải trả cho ngân hàng là $25 \,260\,000-25\,000\,000=260\,000$ (đồng).
}
\end{ex}

\begin{ex}%[1D6V3-5]%[TEX Đề Moon 2025]%[Vũ Hồng Toàn]
Một người vay ngân hàng số tiền $350$ triệu đồng, hàng tháng (tính từ ngày gửi) người đó trả góp $8$ triệu đồng. Lãi suất cho số tiền chưa trả là $0{,}79\%$ một tháng và kỳ trả đầu tiên là cuối tháng thứ nhất. Biết số tiền phải trả ở kỳ cuối là $m$ triệu đồng thì người đó trả hết nợ ngân hàng. Tính giá trị $m$ ($m$ làm tròn đến hàng phần trăm).
\shortans{$7{,}14$}
\loigiai{
Gọi $A$ là số tiền vay ngân hàng, $B$ là số tiền trả trong mỗi chu kì, $r=79 \%=0{,}0079$ là lãi suất cho số tiền chưa trả trên một chu kì, $n$ là số kì trả nợ.\\
Số tiền còn nợ ngàn hàng (tính cả lãi) trong từng chu kì như sau
\begin{itemize}
\item Đầu kì thứ nhất là $A$.
\item Cuối kì thứ nhất là $A(1+r)- B$.
\item Cuối kì thứ hai là
$[A(1+r)-n](1+r)-A=A(1+r)-A[(1+r)+1]$.
\item Cuối kì thứ ba là
\[\left[A(1+r)^2-B[(1+r)+1]\right]\left(1+r\right)-B=A(1+r)^3-B\left[(1+r)^2+(1+r)+1\right].\]
$\ldots$
\item Theo giả thiết quy nạp, cuối kì thứ $n$ là
\[A(1+r)^n-B\left[(1+r)^{n-1}+\cdots+(1+r)+1\right]=A(1+r)^n-B\cdot  \dfrac{(1+r)^n-1}{r}.\]
\end{itemize}
Vậy số tiền còn nợ (tính cả lãi) sau $n$ chu kì là
$A(1+r)^n-B\cdot \dfrac{(1+r)^n-1}{r}$.\\
Người đó trả hết nợ ngân hàng khi
\allowdisplaybreaks
\begin{eqnarray*}
&&A(1+r)^n-B\cdot \dfrac{(1+r)^n-1}{r}=0\\
&\Leftrightarrow&(1+r)^n=\dfrac{B}{B-Ar}\\
&\Leftrightarrow& n =\log_{(1+r)}\dfrac{B}{Ar-B}=\log_{1{,}0079}\dfrac{8}{350\cdot 0{,}0079-8}\approx 53{,}9.
\end{eqnarray*}
Tức là phải mất $54$ tháng người này mới trả hết nợ.\\
Cuối tháng thứ $53$, số tiền còn nợ (tính cả lãi) là
\[S_{53}=350\cdot 1{,}0079^{53}-8 \cdot \dfrac{1{,}0079^{53}-1}{0{,}0079} \text{(triệu đồng)}.\]
Kì trả nợ tiếp theo là cuối tháng thứ $54$, khi đó người vay phải trả số tiền
$S_{53}$ và lãi của số tiền này nữa là $S_{53}+0{,}0079 \cdot S_{53}=S_{53} \cdot 1{,}0079 \approx 7{,}14$ (triệu đồng).
}
\end{ex}

\begin{ex}%[1D2V2-7]
\immini[thm]
{
An đã tạo ra một cầu thang $3$ bậc bằng $18$ que tăm như hình minh họa. Vậy An cần thêm bao nhiêu que tăm để hoàn thành một cầu thang $5$ bậc?
}
{
\includegraphics[scale=0.16]{images/de15-1}
}
\shortans[]{$22$}
\loigiai{
\begin{itemize}
\item \textbf{Cách 1}\\
Ta thấy rằng cầu thang $1$ bậc cần $4$ tăm và cầu thang $2$ bậc cần $10$ tăm. Do đó, để đi từ cầu thang $1$ bậc đến $2$ bậc cần thêm $6$ tăm và để đi từ cầu thang $2$ bậc đến $3$ bậc cần thêm $8$ tăm.\\
Áp dụng mô hình này, để đi từ cầu thang $3$ bậc đến $4$ bậc cần thêm $10$ tăm và để đi từ cầu thang $4$ bậc đến $5$ bậc cần thêm $12$ tăm.\\
Vậy bạn An cần thêm $10+12=22$ tăm.
\item \textbf{Cách 2}\\
Với cầu thang $3$ bậc có $2\left[2\cdot 3+2+1\right]=18$ tăm.\\
Tổng quát, ta thấy rằng cầu thang có $x$ bậc có
\[2\left[2x+(x-1)+(x-2)+\ldots+1\right]\,\text{tăm}.\]
Vì vậy đối với $x=5$ bậc ta có
\[2\left[2\cdot 5+4+3+2+1\right]=40\,\text{tăm}.\]
Vậy bạn An cần thêm $40-18=22$ tăm.
\item \textbf{Cách 3}\\
Ta thấy rằng để đến được $4$ các bước, ta thêm hai khối có ba que (phía trên và bên phải) va fhai khối nữa có hai khối toạ thành các bước. Điều này sẽ có thêm $2\cdot 3+2\cdot 2=10$ quê. \\
Sau đó,để đếm được $5$ các bước, ta thêm hai khối cạnh nữa có $3$ que và $3$ nữa có hai que. Ta thêm $2\cdot 3+3\cdot 2=12$ nhiều hơn nữa để tăng tổng cộng $10+12=22$ que.
\end{itemize}
}
\end{ex}

\begin{ex}%[1D2V1-6]
\immini{Một chai soda có giá $1$ đô. Sau khi uống, hai chai rỗng sẽ được đổi lấy một chai soda. Bạn có thể uống nhiều nhất bao nhiêu chai soda nếu bạn có $100$ đô?}{\begin{tikzpicture}[very thick,>=stealth',scale=0.7]
\filldraw[black] (-0.25,3.8)--(-0.25,3.5) arc (180:360:0.25 and 0.1)--(0.25,3.8) arc (0:-180:0.25 and 0.1);
\filldraw[blue!70!green!30] (-0.6,0.5)--(-0.6,-2) arc (180:0:0.6 and 0.3)--(0.6,0.5);
\filldraw[blue!50!green!50,,] (0,0.5) ellipse (0.6 and 0.3);
\filldraw[blue!50!green!50] (0,-2) ellipse (0.6 and 0.3);
\draw (-0.6,1) -- (-0.6,-2) (0.6,1) -- (0.6,-2);
\draw (0,-2) ellipse (0.6 and 0.3);
\draw (-0.6,1)--(-0.6,1.5)--(-0.25,2.8)--(-0.25,3.8);
\draw (0.6,1)--(0.6,1.5)--(0.25,2.8)--(0.25,3.8);
\draw (0,3.8) ellipse (0.25 and 0.1);
\end{tikzpicture}}
\shortans{199}
\loigiai{
\textbf{Mua số lượng chai soda ban đầu:}\\
Với $100$ đô ta sẽ mua được $100$ chai soda.\\
Đổi chai rỗng lấy soda: Sau khi uống $100$ chai, ta sẽ có $100$ chai rỗng, $100$ chai rỗng ta sẽ đổi được 50
chai soda mới.\\
\textbf{	Tiếp tục quy trình:}\\
Sau khi uống $50$ chai soda mới, ta được $50$ chai rỗng.
Đổi $50$ chai rỗng lấy $25$ chai soda mới.\\
\textbf{Lặp lại quy trình:}\\
Uống $25$ chai so da mới, ta được $25$ chai rỗng.\\
Đổi $25$ chai rỗng ta được $12$ chai soda mới và dư $1$ chai rỗng.\\
\textbf{	Tiếp tục đổi:}\\
Uống $12$ chai soda mới ta sẽ có $12$ chai rỗng.\\
Đổi $12$ chai rỗng ta được $6$ chai so đa mới.\\
\textbf{Lặp lại quy trình:}\\
Uống $6$ chai so da mới, ta được $6$ chai rỗng.\\
Đổi $6$ chai rỗng ta được $3$ chai soda mới.\\
\textbf{Tiếp tục đổi:}\\
Uống chai $3$ soda mới vừa đổi ta được $3$ chai rỗng.\\
Cộng với $1$ chai rỗng còn dư ở phía trên ta được $4$ chai rỗng và đổi thêm được $2$ chai soda mới.\\
\textbf{Kết thúc:}
Uống $2$ chai so da mới, ta được $2$ chai rỗng.\\
Đổi $2$ chai rỗng ta được 1 chai soda mới.\\
Uống chai $1$ soda mới vừa đổi ta được 1 chai rỗng.\\
Vậy ta có thể uống được nhiều nhất là $100 + 50 + 25 + 12 + 6 + 3 + 2 + 1 = 199$ chai.
}
\end{ex}

\begin{ex}%[1C2V3-1]
Một nhân viên của bảo tàng nghệ thuật đang có kế hoạch giới thiệu nội dung cuộc triển lãm của bảo tàng đến ba trường học trong khu vực. Người đó muốn đến từng trường và quay trở lại bảo tàng sau khi thăm cả ba trường. Thời gian di chuyển (đơn vị: phút) giữa các trường học và giữa bảo tàng với mỗi trường học được mô tả trong hình vẽ. Tìm thời gian đi ít nhất để thực hiện chu trình trên.
\begin{center}
\begin{tikzpicture}[scale=1,>=stealth, font=\footnotesize, line join=round, line cap=round]
\coordinate (A) at (0,0);
\coordinate (B) at (2,3);
\coordinate (C) at (2,-1);
\coordinate (D) at (3.5,2.5);
\draw (A)--(B)--(D)--(C)--(A) (A)--(D) (B)--(C) ($(A)!0.5!(B)$)node[above left]{$38$} ($(B)!0.6!(D)$)node[above]{$19$} ($(D)!0.5!(C)$)node[right]{$51$} ($(A)!0.5!(C)$)node[below left]{$32$} ($(A)!0.5!(D)$)node[above left,yshift=-0.1cm]{$46$} ($(B)!0.7!(C)$)node[left]{$50$};
\fill (A)node[left]{Trường $A$}circle(2pt);
\fill (B)node[above]{Trường $B$}circle(2pt);
\fill (C)node[below]{Trường $C$}circle(2pt);
\fill (D)node[right]{Bảo tàng}circle(2pt);
\end{tikzpicture}
\end{center}

\shortans{140}
\loigiai{
Gọi $A, B, C$ là ba trường học và $D$ là bảo tàng.
Nhân viên cần thực hiện một chu trình xuất phát từ $D$, đi qua $A, B, C$ (mỗi nơi đúng một lần) và quay trở về $D$.
Các chu trình có thể thực hiện và tổng thời gian tương ứng (tính bằng phút) là
\begin{itemize}
\item $D \to A \to B \to C \to D$: $46+38+50+51=185$.
\item $D \to A \to C \to B \to D$: $46+32+50+19=147$.
\item $D \to B \to A \to C \to D$: $19+38+32+51=140$.
\item $D \to B \to C \to A \to D$: $19+50+32+46=147$.
\item $D \to C \to A \to B \to D$: $51+32+38+19=140$.
\item $D \to C \to B \to A \to D$: $51+50+38+46=185$.
\end{itemize}
So sánh tổng thời gian của các chu trình, ta thấy thời gian đi ít nhất là $140$ phút.
}
\end{ex}

\begin{ex}%[1C2V3-1]%[TEX Đề Moon 2025]%[Võ Nguyên Thạch]
Một người đưa thư xuất phát từ bưu điện (vị trí $A$) và phải đi qua các con đường để phát thư rồi quay lại bưu điện. Sơ đồ các con đường cần đi qua và độ dài của chúng (tính theo mét) được biểu diễn ở hình vẽ dưới.
\begin{center}
\begin{tikzpicture}[scale=1,>=stealth, font=\footnotesize, line join=round, line cap=round]
\coordinate (F) at (0,0);
\coordinate (A) at (2,2);
\coordinate (B) at (5,2);
\coordinate (E) at (2,-2);
\coordinate (D) at (5,-2);
\coordinate (C) at (7,0);
\draw (F)--(A)--(B)--(C)--(D)--(E)--cycle (A)--(E) (E)--(B) (B)--(D) ($(F)!0.5!(A)$)node[above left]{$1\,000$} ($(A)!0.5!(B)$)node[above]{$200$} ($(B)!0.5!(C)$)node[above right]{$300$} ($(C)!0.5!(D)$)node[below right]{$400$} ($(B)!0.5!(D)$)node[right]{$1\,500$} ($(E)!0.5!(D)$)node[below]{$1\,600$} ($(E)!0.5!(B)$)node[above left]{$800$} ($(A)!0.5!(E)$)node[left]{$700$} ($(F)!0.5!(E)$)node[below left]{$900$};
\foreach \x/\g in {F/180,A/90,B/90,C/0,D/-90,E/-90}
\fill[black] (\x) circle(2pt) +(\g:4mm) node {$\x$};
\end{tikzpicture}
\end{center}
Hỏi người đó phải đi như thế nào để đường đi là ngắn nhất?
\shortans{8\,300}
\loigiai{
Đồ thị trên hình chỉ có hai đỉnh bậc lẻ là $A$ và $D$ nên ta có thể tìm được một đường đi Euler từ $A$ đến $D$ (đường đi này đi qua mỗi cạnh đúng một lần).\\
Một đường đi Euler từ $A$ đến $D$ là $AFEABEDBCD$ và tổng độ dài của nó là
\[1\,000+900+700+200+800+1\,600+1\,500+300+400=7\,400.\]
Để quay trở lại điểm xuất phát và có đường đi ngắn nhất, ta cần tìm một đường đi ngắn nhất từ $D$ đến $A$ theo thuật toán gắn nhãn vĩnh viễn.\\
Đường đi ngắn nhất từ $D$ đến $A$ là $DCBA$ và có độ dài là $400+300+200=900$.\\
Vậy một chu trình cần tìm là $AFEABEDBCDCBA$ và có độ dài là $7\,400+900=8\,300$.
}
\end{ex}

\begin{ex}%[1H8V5-4]%[TEX ĐỀ MOON 2025]%[Lê Hữu Kiệt]
Cho hình chóp $S.ABCD$ có đáy $ABCD$ là hình vuông và tam giác $SAB$ đều nằm trong mặt phẳng vuông góc với đáy. Biết khoảng cách giữa hai đường thẳng $SA$ và $BD$ bằng $\sqrt{21}$. Hỏi cạnh đáy của hình chóp đã cho bằng bao nhiêu?
\shortans{$7$}
\loigiai{
\immini
{Đặt $a$ là độ dài cạnh đáy của hình chóp ($a>0$).\\
Gọi $H$ là chân đường cao kẻ từ $S$ trong $\triangle SAB$ đều, khi đó $H$ là trung điểm của $AB$ và $SH=\dfrac{a\sqrt3}{2}$.\\
Gọi $E\in(ABCD)$ sao cho $ADBE$ là hình bình hành. Khi đó $BDallel AE$.\\
Mà $AE\subset (SAE)$ nên $BDallel (SAE)$.\\
Suy ra $\mathrm{d}(BD,SA)=\mathrm{d}\left(BD,(SAE)\right)=\mathrm{d}\left(B,(SAE)\right)$.\\
Ta có $\dfrac{AH}{AB}=\dfrac{1}{2}$ nên $\mathrm{d}\left(B,(SAE)\right)=2\mathrm{d}\left(H,(SAE)\right)$.}
{\begin{tikzpicture}[font=\footnotesize, line join=round, line cap=round, >=stealth, scale=1, declare function={a=3;cao=a*sqrt(3)/2;}]
\path
(0,0) coordinate (B) (-135:a/2) coordinate (A) (0:a) coordinate (C) ($(C)-(B)+(A)$) coordinate (D)
(barycentric cs:A=1,B=1) coordinate (H)+(0,cao) coordinate (S)
($(A)-(D)+(B)$) coordinate (E)
($(E)!3/4!(A)$) coordinate (F)
($(S)!(H)!(F)$) coordinate (K);
\draw (S)--(A)--(D)--(C)--cycle (S)--(D) (S)--(E)--(A) (S)--(F);
\draw[dashed] (A)--(B)--(C) (S)--(B)--(D) (S)--(H)--(F) (B)--(E) (H)--(K);
\foreach \x/\g in {S/90, A/-90, B/45, C/0, D/0, H/-45, E/180, F/-135, K/180}{\fill (\x) circle (1pt)+(\g:0.3)node{$\x$};}
\end{tikzpicture}}
\noindent
Kẻ $HF\perp AE$, khi đó $\heva{&HF\perp AE\\& SH\perp AE\\&HF\cap SH=H}$ suy ra $AE\perp(SHF)$.\\
Suy ra $(SAE)\perp(SHF)$ và $(SAE)\cap(SHF)=SF$.\\
Khi đó $\mathrm{d}\left(H,(SAE\right)=\mathrm{d}(H,SF)$.\\
Kẻ $HK\perp SF$, suy ra $\mathrm{d}(H,SF)=HK$.\\
Xét $\triangle ABE$ ta có $AB=AE=a$, $\widehat{ABE}=90^\circ$ nên $\triangle ADE$ vuông cân tại $B$. Suy ra $\widehat{HAF}=45^\circ$.\\
Xét $\triangle HAF$ vuông tại $F$ ta có $HF=HA\sin\widehat{HAF}=\dfrac{a}{2}\sin45^\circ=\dfrac{a\sqrt2}{4}$.\\
Xét $\triangle SHF$ vuông tại $H$ ta có
\begin{eqnarray*}
&&\dfrac{1}{HK^2}=\dfrac{1}{SH^2}+\dfrac{1}{HF^2} \\
&\Leftrightarrow&\dfrac{1}{HK^2}=\dfrac{1}{\left(\dfrac{a\sqrt3}{2}\right)^2}+\dfrac{1}{\left(\dfrac{a\sqrt2}{4}\right)^2}\\
&\Leftrightarrow&\dfrac{1}{HK^2}=\dfrac{28}{3a^2} \\
&\Leftrightarrow&HK=\dfrac{a\sqrt{21}}{14}.
\end{eqnarray*}
Suy ra $\mathrm{d}(BD,SA)=2HK=\dfrac{a\sqrt{21}}{7}$.\\
Theo giả thiết ta có $\mathrm{d}(BD,SA)=\sqrt{21}$, suy ra $a=7$.
}
\end{ex}

\begin{ex}%[50 Đề minh họa tốt nghiệp 2025 - Đề 13]%[Lê Hữu Kiệt - Lê Quân]%[2H5V3-4]
Trong hệ toạ độ $Oxyz$, có một mặt cầu $(S)\colon (x-1)^2+(y-2)^2+(z+1)^2=3$ và đường thẳng $\Delta\colon \dfrac{x+4}{6}=\dfrac{y-6}{-2}=\dfrac{z-2}{-1}$. Từ điểm $M\in \Delta$ kẻ các tiếp tuyến đến mặt cầu $(S)$ và gọi $(C)$ là tập hợp các tiếp điểm. Biết diện tích hình phẳng giới hạn bởi $(C)$ đạt giá trị nhỏ nhất thì $(C)$ nằm trên mặt phẳng $x+by+cz+d=0$. Tìm $b+c+d$.
\par\shortans{$-2$}
\loigiai{
Ta có mặt cầu $(S)$ có tâm $I(1;2;-1)$, bán kính $R=\sqrt3$; đường thẳng $\Delta$ có $\overrightarrow{u}=(6;-2;-1)$ là một vectơ chỉ phương.\\
Hình phẳng được giới hạn bởi $(C)$ là một hình tròn. Gọi $AB$ là đường kính đường tròn $(C)$.\\
Với $M\in\Delta$, gọi $H=AB\cap IM$, khi đó $H$ là tâm đường tròn $(C)$.
\begin{center}
\begin{tikzpicture}[font=\footnotesize, line join=round, line cap=round, >=stealth, scale=1]
\pgfmathsetmacro\bankinh{sqrt(3)}
\pgfmathsetmacro\goc{acos(\bankinh/3)}
\path (0,0) coordinate (I) (\goc:\bankinh) coordinate (A) (-\goc:\bankinh) coordinate (B) (3,0) coordinate (M)
(intersection of A--B and I--M) coordinate (H)
pic[draw, angle radius=2mm]{right angle=A--H--I}
pic[draw, angle radius=2mm]{right angle=I--A--M}
;
\draw (I) circle (\bankinh) (I)--(A)--(M)--(B)--cycle (A)--(H) (I)--(M) (A)--(B);
\foreach \x/\g in {I/180, M/0, A/60, B/-60, H/45}{
\fill (\x) circle (1pt)+(\g:0.3)node{$\x$};
}
\end{tikzpicture}
\end{center}
Diện tích hình tròn $(C)$ là $S_{(C)}=\pi\cdot AH^2$. Do đó $S_{(C)}$ đạt giá trị nhỏ nhất khi và chỉ khi $AH$ đạt giá trị nhỏ nhất.\\
Ta có $AH$ đạt giá trị nhỏ nhất $\Leftrightarrow IM$ đạt giá trị nhỏ nhất.\\
Suy ra $M$ là hình chiếu của $I$ trên $\Delta$.\\
Ta có $M\in\Delta$ nên $M(-4+6t;6-2t;2-t)$. Khi đó $\overrightarrow{IM}=(-5+6t;4-2t;3-t)$.\\
Do $\overrightarrow{IM}\perp\overrightarrow{u} \Leftrightarrow 6(-5+6t)-2(4-2t)-1(3-t)=0 \Leftrightarrow t=1$.\\
Do đó $M(2;4;1)$ và $\overrightarrow{IM}=(1;2;2)$, suy ra $IM=3$.\\
Xét $\triangle IAM$ vuông tại $A$, ta có $IH=\dfrac{IA^2}{IM}=1$.\\
Do $\overrightarrow{IH}$ và $\overrightarrow{IM}$ cùng phương, $IH=\dfrac{1}{3}IM$, suy ra $\overrightarrow{IH}=\dfrac{1}{3}\overrightarrow{IM}$.\\
Suy ra $H\left(\dfrac{4}{3};\dfrac{8}{3};-\dfrac{1}{3}\right)$.\\
Mặt phẳng chứa $(C)$ đi qua điểm $H$ và nhận $\overrightarrow{IM}$ là vectơ pháp tuyến có dạng
\begin{eqnarray*}
&& 1\left(x-\dfrac{4}{3}\right)+2\left(y-\dfrac{8}{3}\right)+2\left(z+\dfrac{1}{3}\right) = 0 \\
&\Leftrightarrow& x+2y+2z-6=0.
\end{eqnarray*}
Suy ra $b=2$, $c=2$, $d=-6$. Vậy $T=b+c+d=-2$.
}
\end{ex}

\begin{ex}%[2H5V3-4]%[TEX Đề Moon 2025]%[Võ Nguyên Thạch]
Trong không gian với hệ tọa độ $Oxyz$, đài kiểm soát không lưu sân bay có tọa độ $O(0;0;0)$, mỗi đơn vị trên một trục ứng với $1$ km. Máy bay bay trong phạm vi cách đài kiểm soát $417$ km sẽ hiển thị trên màn hình ra đa. Một máy bay đang ở vị trí $A(-688;-185;8)$, chuyển động theo đường thẳng $d$ có véc-tơ chỉ phương là $\overrightarrow{u}=(91;75;0)$ và theo hướng về đài không lưu. Biết $E(a;b;c)$ là vị trí sớm nhất mà máy bay xuất hiện trên màn hình. Tính $T=a+b+c$.

\shortans{-367}
\loigiai{
Đường thẳng $d$ đi qua điểm $A(-688;-185;8)$, có một véc-tơ chỉ phương $\overrightarrow{u}=(91;75;0)$ có phương tình tham số là
\[\heva{&x=-688+91t\\&y=-185+75t\\&z=8}\quad (t \text{ là tham số.})\]
Gọi $B$ là vị trí sớm nhất mà máy bay xuất hiện trên màn hình ra đa.\\
Vì $B$ thuộc $d$ nên $B(-688+91t;-185+75t;8)$.\\
Để $B$ là vị trí sớm nhất mà máy bay xuất hiện trên màn hình ra đa thì $OB=417$.
\allowdisplaybreaks
\begin{eqnarray*}
\text{Do đó }\sqrt{(-688+91t)^2+(-185+75t)^2+8^2}=417&\Leftrightarrow& 13\,906t^2-152\,966t+333\,744=0\\
&\Leftrightarrow&t=3 \text{ hoặc }t=8.
\end{eqnarray*}
Với $t=3$, ta có $B(-415;40;8)$ và $AB=\sqrt{(-415+688)^2+(40+185)^2}=\sqrt{125\,154}$.\\
Với $t=8$ ta có $B(40;415;8)$ và $AB=\sqrt{(40+688)^2+(415+185)^2}=\sqrt{889\,984}$.\\
Vì $\sqrt{125\,154}<\sqrt{889\,984}$ nên tọa độ vị trí sớm nhât mà máy bay xuất hiện trên màn hình ra đa là $(-415;40;8)$.\\
Khi đó $a=-415$; $b=40$; $c=8$.\\
Suy ra $T=a+b+c=-415+40+8=-367$.
}
\end{ex}

\begin{ex}%[2H5V2-8]%[TEX Đề Moon 2025]%[Vũ Hồng Toàn]
Trong không gian với một hệ trục tọa độ $Oxyz$ cho trước (đơn vị trên các trục tính bằng kilomet), ra đa phát hiện một chiếc máy bay di chuyển với vận tốc và hướng không đổi từ điểm $A(800;500;7)$ bay thẳng đến điểm $B(940;550;8)$ trong $10$ phút. Nếu máy bay tiếp tục giữ nguyên vận tốc và hướng bay thì sau $5$ phút tiếp theo, khoảng cách từ máy bay đến gốc tọa độ $O$ bằng bao nhiêu kilomet? (kết quả làm tròn đến hàng đơn vị).
\shortans{$1162$}
\loigiai{
Vị trí của máy bay sau $5$ phút tiếp theo là $C(x ; y ; z)$.\\
Vì hướng của máy bay không đổi nên $\overrightarrow{AB}$ và $\overrightarrow{BC}$ cùng hướng.\\
Do vận tốc của máy bay không đổi và thời gian bay từ $A$ đến $B$ gấp đôi thời gian bay từ $B$ đến $C$ nên $AB=2 BC$.\\
Do đó $\overrightarrow{BC}=\frac{1}{2} \overrightarrow{A B}=\left(\dfrac{940-800}{2} ; \dfrac{550-500}{2} ; \frac{9-7}{2}\right)=(70 ; 25 ; 1)$.\\
Mặt khác, $\overrightarrow{BC}=(x-940 ; y-550 ; z-9)$ nên $\heva{&x-940=70 \\& y-550=25 \\& z-9=1}\Rightarrow\heva{&x=1\,010 \\& y=575 \\& z=10.}$\\
Vậy $OC=\sqrt{1\,101^2+575^2+10^2}\approx 1162$\,(km).
}
\end{ex}

\begin{ex}%[2H5V2-7]
Một lều trại có mặt trước và mặt sau rộng $4$ m, hai mặt bên rộng $3$ m gồm sáu thanh cọc tre, vải bạt chống thấm nước, dây dù hoặc dây thừng để cố định lều tại sáu cọc sắt cắm sát đất như hình vẽ. Biết rằng, hai thanh $AF$, $OC$ có chiều dài $2{,}2$ m, bốn thanh còn lại có chiều dài $1{,}7$ m và đoạn dây thừng $IF=2{,}75$ m.  Chọn hệ trục tọa độ $Oxyz$ như hình vẽ và cho biết góc giữa đường thẳng chứa dây thừng $IF$ và mặt phẳng chứa tấm bạt $(CDEF)$ là $\alpha$.
\begin{center}
\begin{tikzpicture}[line join = round, line cap=round,>=stealth,font=\footnotesize,scale=1]
\path
(0,0) coordinate (O)
(-5,-1) coordinate (x)
(2,-3) coordinate (y)
(0,5) coordinate (z)
%	($(A)+(B)-(O)$) coordinate (N)
%	($(N)+(0,4)$) coordinate (M)
;
\coordinate (A) at ($(O)!5/7!(x)$);
\coordinate (I) at ($(O)!6.3/7!(x)$);
\coordinate (B) at ($(O)!3/5!(y)$);
\coordinate (a1) at	($(A)+(B)-(O)$);
\coordinate (F) at ($(A)+(0,3)$);
\coordinate (C) at ($(O)+(0,3)$);
\coordinate (D) at ($(B)+(0,2.5)$);
\coordinate (E) at ($(a1)+(0,2.5)$);
\coordinate (b1) at ($(A)!-1!(a1)$);
\coordinate (b2) at ($(b1)+(0,2)$);
\coordinate (c1) at	($(b1)+(O)-(A)$);
\coordinate (c2) at ($(c1)+(0,2)$);
\draw[->] (O)--(x)node[below left]{$x$};
\draw[->] (O)--(y)node[below left]{$y$};
\draw[->] (O)--(z)node[above left]{$z$};
\draw(O)--(c1)-- (b1)--(A)--(a1)--(B) (c2)--(b2)--(F)--(E)--(D)--(C)--(c2) (F)--(C) (F)--(I);
\draw[line width=2pt] (A)--(F) (a1)--(E) (O)--(C) (B)--(D) (b1)--(b2) (c1)--(c2);
\draw(E)--(-3.5,-3)  (b2)--(-5.5,2) (D)--(1.8,-1.7) (C)--(1.5,1.2) (c2)--(-.5,2);
\path (A)--(b1) node [below left ,sloped,pos=0.1] {$4$m};
\path (a1)--(B) node [below  ,sloped,pos=0.5] {$3$m};
\foreach \i/\g in {A/60,B/-90,C/30,D/30,E/70,F/90,I/-90}{\draw[fill=black](\i) circle (1.5pt) ($(\i)+(\g:3mm)$) node[scale=1]{$\i$};}
\end{tikzpicture}
\end{center}
Tính giá trị của $\alpha$ (tính theo đơn vị độ và làm tròn kết quả đến hàng đơn vị của độ). \shortans{$51^\circ$}
\loigiai{
Ta có $AI=\sqrt{IF^2-AF^2}=\sqrt{2{,}75^2-2{,}2^2}=1{,}65$ (m).\\
Do đó $A(3; 0; 0)$, $I(4{,}65; 0; 0)$, $B(0; 2; 0)$, $E(3; 2; 1{,}7)$, $F(3; 0; 2{,}2)$ và $C(0; 0; 2{,}2)$.\\
Suy ra $\overrightarrow{IF}=(-1{,}65; 0; 2{,}2)$, $\overrightarrow{EF}=(0; -2; 0{,}5)$ và $\overrightarrow{EC}=(-3; -2; 0{,}5)$.\\
Lại có $\left[\overrightarrow{EF}, \overrightarrow{EC}\right]=(0; -1{,}5; -6)$ nên $\overrightarrow{n}=(0; 1; 4)$ là một vectơ pháp tuyến của mặt phẳng $(CDEF)$.\\
Suy ra
\[\sin\left(IF, (CDEF)\right)=\dfrac{\left|\overrightarrow{IF}\cdot \overrightarrow{n}\right|}{\left|\overrightarrow{IF}\right|\cdot \left|\overrightarrow{n}\right|}=\dfrac{\left|-1{,}65\cdot 0+0\cdot 1+2{,}2\cdot4\right|}{\sqrt{(-1{,}65)^2+0^2+2{,}2^2}\cdot\sqrt{0^2+1^2+4^2}}=\dfrac{16\sqrt{17}}{85}.\]
Do đó $\left(IF, (CDEF)\right)\approx 51^\circ\Rightarrow \alpha\approx 51^\circ$.
}
\end{ex}

\begin{ex}%[2H5V2-3]
Trong không gian $O x y z$, cho tam giác $ABC$ vuông tại $A$, $ABC=30^\circ,BC=3\sqrt{2}$, đường thẳng $BC$ có phương trình $\dfrac{x-4}{1}=\dfrac{y-5}{1}=\dfrac{z+7}{-4}$, đường thẳng $AB$ nằm trong mặt phẳng $(\alpha)\colon  x+z-3=0$. Biết đỉnh $C$ có cao độ âm. Tính hoành độ đỉnh $A$.
\shortans{$4{,}5$}
\loigiai{
Tọa độ điểm $B$ là nghiệm của hệ phương trình $\heva{&\dfrac{x-4}{1}=\dfrac{y-5}{1}=\dfrac{z+7}{-4}\\&x+z-3=0}\Rightarrow B(2;3;1)$.\\
Do $ABC=30^\circ$ nên
\begin{eqnarray*}
\heva{&AB=\dfrac{3\sqrt{6}}{2}\\&AC=\dfrac{3\sqrt{2}}{2}}&\Leftrightarrow&\heva{&(x-2)^2+(y-3)^2+(2-x)^2=\dfrac{27}{2}\\&(x-3)^2+(y-4)^2+(6-x)^2=\dfrac{9}{2}}\\
&\Leftrightarrow&\heva{&2x^2-8y+y^2-6y+\dfrac{7}{2}=0\\&2x^2-18x+y^2-8y+\dfrac{113}{2}=0}\Leftrightarrow\heva{&10x+2y-53=0\quad (1)\\&2x^2-8y+y^2-6y+\dfrac{7}{2}=0.\quad (2)}
\end{eqnarray*}
Từ $(1)$ ta có $y=\dfrac{53-10x}{2}$, thay vào $(2)$
ta có
\begin{eqnarray*}
&&2x^2-8x+\left(\dfrac{53-10x}{2}\right)^2-6\cdot\dfrac{53-10x}{x}+\dfrac{7}{2}=0\\
&\Leftrightarrow& 108 x^2-972 x+2187=0\\
&\Leftrightarrow&(2 x-9)^2=0\\
&\Leftrightarrow& x=\dfrac{9}{2}.
\end{eqnarray*}
Do đó $A\left(\dfrac{9}{2};4;-\dfrac{3}{2}\right)$.\\
Vậy hoành độ đỉnh $A$ là $\dfrac{9}{2}=4{,}5$.
}
\end{ex}

\begin{ex}%[2H5V1-2]
Trong không gian $Oxyz$, cho hình lăng trụ tam giác đều $A_1B_1C_1$ có $A_1(\sqrt{3};-1;1)$, hai đỉnh $B$, $C$ thuộc trục $Oz$ và $AA_1=1$ ($C$ không trùng với $O$). Biết $\vec{u}=(a;b;10)$ là một vectơ chỉ phương của đường thẳng $A_1C$. Giá trị của $a^2 + b^2$ bằng bao nhiêu?

\shortans{400}
\loigiai{
\begin{center}
\begin{tikzpicture}[scale=0.8, font=\footnotesize,line join=round, line cap=round, >=stealth]
\path
(0,0) coordinate (A)
++(-120:2) coordinate (B)
(3,0) coordinate (C)
($(B)!.5!(C)$)coordinate (M)
;
\foreach \i in{A,B,C}{
\coordinate (\i_1) at ($(\i)+(0,3)$);
};
\draw (B)--(B_1) (A)--(B)--(C)--(C_1) (A_1)--(B_1)--(C_1)--(A_1);
\draw[dashed] (A)--(C) (A)--(M)--(A_1)--(A);
\foreach \i/\g in {A/-180,B/-90,C/-90,A_1/90,B_1/120,C_1/90,M/-90}
\fill[black] (\i) circle(1pt)+(\g:4mm)node[scale=1]{$\i$};
\end{tikzpicture}
\end{center}
Gọi $M$ là trung điểm $BC$ khi đó $AM$ vuông góc với $BC$.\\
Ta có $\heva{&AA_1\perp BC\\&AM\perp BC}\Rightarrow BC\perp (AA_1M)$.\\
Mặt phẳng $(A_1AM)$ qua $A_1$ và nhận $\overrightarrow{k}=(0;0;1)$ làm VTPT nên $(A_1AM)\colon z-1=0$.\\
Mà $M=(A_1AM)\cap Oz$ nên $M(0;0;1)$ suy ra $A_1M=2$.\\
Trong tam giác $A_1AM$ có $AM=\sqrt{A_1M^2-AA_1^2}=\sqrt{3}$.\\
Ta có tam giác $ABC$ đều nên $AM=\dfrac{BC\sqrt{3}}{2}\Rightarrow BC=\dfrac{2AM}{\sqrt{3}}=2$.\\
Gọi $B(0;0;m)$ mà $M$ là trung điểm $BC$ nên $C(0;0;2-m)$.\\
Ta có $BC=|2-2m|=2\Leftrightarrow\hoac{&m=0\\&m=2}\Rightarrow B(0;0;0)$, $C(0;0;2)$ vì $C$ không trùng với $O$.\\
Do đó $\overrightarrow{A_1C}=\left(-\sqrt{3};1;1 \right)=\dfrac{1}{10}\left( -10\sqrt{3};10;10\right)\Rightarrow\heva{&a=-10\sqrt{3}\\&b=10}$.\\
Vậy $a^+b^2=400$.
}
\end{ex}

\begin{ex}%[2H2V2-6]%[TEX ĐỀ MOON 2025]%[Nguyễn Văn Hiệp]
Trong không gian với một hệ trục tọa độ cho trước (đơn vị tính bằng mét), một con chim đang bay với tốc độ và hướng không đổi từ điểm $A(20;40;30)$ đến điểm $B(40;50;50)$ trong vòng $4$ phút. Nếu con chim bay tiếp tục giữ nguyên vận tốc và hướng bay thì sau $2$ phút con chim ở vị trí $C(a;b;c)$. Tổng $a+b+c$ bằng bao nhiêu?
\shortans{$165$}
\loigiai{
\textbf{Bước 1: Tính vectơ vận tốc} \\
$\overrightarrow{AB} = (20; 10; 20)$ \\
Vận tốc trung bình:
\[
\overrightarrow{v} = \dfrac{\overrightarrow{AB}}{4} = (5; 2{,}5; 5) \text{ (m/phút)}
\]
\textbf{Bước 2: Tính vị trí sau 6 phút} \\
Tọa độ điểm $C$
\[
\heva{
&a = 20 + 6 \times 5 = 50 \\
&b = 40 + 6 \times 2{,}5 = 55 \\
&c = 30 + 6 \times 5 = 60.
}
\]
Vậy $C(50;55;60)$. Suy ra $a + b + c = 165$.
}
\end{ex}

\begin{ex}%[2H2V2-6]%[TEX ĐỀ MOON 2025]%[Nguyễn Cường]
Hệ thống định vị toàn cầu GPS là một hệ thống cho phép xác định vị trí của một vật thể trong không gian. Trong cùng một thời điểm, vị trí của một điểm $M$ trong không gian sẽ được xác định bởi bốn vệ tinh cho trước nhờ các bộ thu phát tín hiệu đặt trên các vệ tinh. Giả sử trong không gian với hệ tọa độ $Oxyz$, có bốn vệ tinh lần lượt đặt tại các điểm $A(3;1;0)$, $B(3;6;6)$, $C(4;6;2)$, $D(6;2;14)$; vị trí $M(a;b;c)$ thỏa mãn $MA=3$, $MB=6$, $MC=5$, $MD=13$. Khoảng cách từ điểm $M$ đến điểm $O$ bằng bao nhiêu?
\shortans{$3$}
\loigiai{
Giả sử $M(a;b;c)$. Ta có hệ phương trình
\allowdisplaybreaks
\begin{eqnarray*}
\heva{&MA=3\\&MB=6\\&MB=5\\&MD=13}&\Leftrightarrow&\heva{& \sqrt{(a-3)^2+(b-1)^2+c^2}=3\\&\sqrt{(a-3)^2+(b-6)^2+(c-6)^2}=6\\&\sqrt{(a-4)^3+(b-6)^2+(c-2)^2}=5\\&\sqrt{(a-6)^3+(b-2)^2+(c-14)^2}=13}\\
&\Leftrightarrow&\heva{&a^2+b^2+c^2-6a-2b+1=0\\&a^2+b^2+c^2-6a-12b-12c+45=0\\&a^2+b^2+c^2-8a-12b-4c+31=0\\&a^2+b^2+c^2-12a-4b-28c+67=0}
\end{eqnarray*}
Giữ nguyên phương trình thứ nhất, lấy phương trình thứ nhất trừ vế theo vế với các phương trình còn lại ta được hệ phương trình mới như sau
\[\heva{
& a^2+b^2+c^2-6a-2b+1=0 \\
& 10b+12c=44 \\
& 2a+10b+4c=30 \\
& 6a+2b+28c=66
}\Leftrightarrow \heva{
& a^2+b^2+c^2-6a-2b+1=0 \\
& a=1 \\
& b=2 \\
& c=2.
}\]
Thế $a=1$, $b=2$, $c=2$ vào phương trình thứ nhất ta thấy thoả mãn.
\\
Vậy điểm $M(1;2;2)\Rightarrow OM=\sqrt{1+4+4}=3$.
}
\end{ex}

\begin{ex}%[2H2V2-6]%[TEX ĐỀ MOON 2025]%[Nguyễn Thế Duy]
Hai chiếc máy bay không người lái cùng bay lên tại một địa điểm. Sau một thời gian bay, chiếc máy bay thứ nhất cách điểm xuất phát về phía Bắc $20$ km và về phía Tây $10$ km, đồng thời cách mặt đất $0{,}7$ km. Chiếc máy bay thứ hai cách điểm xuất phát về phía Đông $30$ km và về phía Nam $25$ km, đồng thời cách mặt đất $1$ km. Hỏi hai chiếc máy bay cách nhau bao nhiêu km? (Làm tròn kết quả đến hàng đơn vị).
\shortans{$60$}
\loigiai{
\immini{Xét hệ trục toạ độ $Oxyz$ có $O$ trùng với điểm hai máy bay xuất phát; chiều dương của $Ox$ chỉ hướng Nam; chiều dương của $Oy$ chỉ hướng Đông; chiều dương $Oz$ chỉ cao độ.\\
Khi đó máy bay thứ nhất tại thời điểm đang xét có toạ độ là điểm $A\left(-20; -10; 0{,}7 \right)$. toạ độ của máy bay thứ hai là $B\left(25; 30; 1 \right)$.\\
Khoảng cách của hai chiếc máy bay là\\
$AB =\sqrt{45^2 + 40^2 + 0{,}3^2} \approx 60$ km.}
{\begin{tikzpicture}[scale=0.9,>=stealth, font=\footnotesize, line join=round, line cap=round]
\draw[->]
(0,0) -- (4,0) node[below]{$y$};
\draw[->]
(0,0) -- (-2.5,-2.5) node[below]{$x$};
\draw[->] (0,0) -- (0,4) node[above]{$z$};
\path
(-2,2.5) coordinate (A)
(3,2) coordinate (B)
;
\draw[dashed]
(-3,0) -- (0,0) -- (2,2)
;
\fill
(A) circle(1pt) node[above]{$A$}
(B) circle(1pt) node[above]{$B$}
;
\end{tikzpicture}}
}
\end{ex}

\begin{ex}%[2H2V2-2]
Chiếc nón lá có dạng hình nón $(N)$ được đặt trong không gian với hệ trục tọa độ $Oxyz$, biết đỉnh của chiếc nón là điểm $S(1;2;3)$, $A(2;2;3)$ và $B(1;4;3)$ là các điểm nằm trên mặt xung quanh của chiếc nón, điểm $C(1;2;6)$ nằm trên đường tròn đáy. Diện tích xung quanh của chiếc nón bằng bao nhiêu? (Làm tròn kết quả đến hàng phần mười).
\shortans[]{$23{,}1$}
\loigiai{
\begin{center}
\begin{tikzpicture}[scale=0.8,>=stealth, font=\footnotesize, line join=round, line cap=round]
\def\a{3}
\def\b{1}
\def\h{5}
\pgfmathsetmacro\gtt{asin(\b/\h)};
\pgfmathsetmacro\xtt{\a*cos(\gtt)};
\pgfmathsetmacro\ytt{\b*sin(\gtt)};
\path (0,0) coordinate (O)
(0,\h)coordinate (S)
(-60:{\a} and {\b}) coordinate (B')
(240:{\a} and {\b})coordinate (C)
(-\a,0) coordinate (A')
(\a,0) coordinate (N)
($(S)!0.3!(A')$) coordinate (A)
($(S)!0.5!(B')$) coordinate (B)
;
\clip (-\a-0.5,-\b-0.5) rectangle (\a+0.5,\h+0.6);
\draw (-\xtt,\ytt) arc (-180-\gtt:\gtt:\a cm and \b cm);
\draw[dashed] (\xtt,\ytt) arc (\gtt:180-\gtt:\a cm and \b cm);
\draw (\xtt,\ytt)--(S)--(-\xtt,\ytt) (S)--(B') (S)--(C);
\draw[dashed] (O)--(S);
\foreach \x/\g in {S/90,A/180,B'/-90,O/-90,A'/180,C/-90,B/0} \fill[black] (\x) circle (1pt) ($(\x)+(\g:3mm)$)node{$\x$};
\end{tikzpicture}
\end{center}
Ta có
\begin{itemize}
\item $\vec{SC}=(0;0;3)\Rightarrow \ell=SC=3$.
\item $\vec{SA}=(1;0;0)\Rightarrow SA=1$.
\item $\vec{SB}=(0;2;0)\Rightarrow SB=2$.
\end{itemize}
Dễ thấy $SA$, $SB$, $SC$ đôi một vuông góc tại $S$. Lấy $A'$, $B'$ thoả mãn $\vec{SA'}=3\vec{SA}$, $\vec{SB'}=\dfrac{3}{2}\vec{SB}$ suy ra $A'$, $B'$ nằm trên đường tròn đáy hình nón. \\
Vậy đáy hình nón là đường tròn ngoại tiếp tam giác $CA'B'$. Các tam giác $CSA'$, $CSB'$, $A'SB'$ là các tam giác bằng nhau và đều vuông cân tại đỉnh $S$ nên tam giác $CA'B'$ là tam giác đều cạnh bằng $3\sqrt{2}$.\\
Từ đó ta tính được bán kính đường tròn ngoại tiếp tam giác $CA'B'$ bằng $r=\dfrac{2}{3}\cdot \dfrac{3\sqrt{2}\cdot \sqrt{3}}{2}=\sqrt{6}$.\\
Diện tích xung quanh hình nón $(N)$ là $S=\pi r\ell=3\pi\sqrt{6}\approx 23{,}1$.
}
\end{ex}

\begin{ex}%[2H2V1-4]%[TexDeMoon2025]%[NguyenKieuNhaTu]
Người ta cần lắp một camera phía trên sân bóng để phát sóng truyền hình một trận bóng đá, camera có thể di động để luôn thu được hình ảnh rõ nét về diễn biến trên sân. Các kĩ sư dự định trồng bốn chiếc cột cao $30$ m và sử dụng hệ thống cáp gắn vào bốn đầu cột để giữ camera ở vị trí mong muốn. Mô hình thiết kế được xây dựng như sau: Trong hệ trục toạ độ $Oxyz$ (đơn vị độ dài trên mỗi trục là $1$ m), các đỉnh của bốn chiếc cột lần lượt là các điểm $M(90;0;30)$, $N(90;120;30)$, $P(0;120;30)$, $Q(0;0;30)$ (như hình vẽ). Giả sử $K_0$ là vị trí ban đầu của camera có cao độ bằng $25$ và $K_0M=K_0N=K_0P=K_0Q$. Để theo dõi quả bóng đến vị trí $A$, camera được hạ thấp theo phương thẳng đứng xuống điểm $K_1$ có cao độ bằng $19$.

{\centering \begin{tikzpicture}[scale=0.6,>=stealth, font=\footnotesize, line join=round, line cap=round]
\def\h{3};
\draw[->] (0,0)node[above right]{$O$}--(-3,-3) node [left]{$x$};
\draw[->] (0,0)--(9,0) node [above]{$y$};
\draw[->] (0,0)--(0,4) node [left]{$z$};
\fill[green!70,opacity=0.6] (0.25,-0.5)--(6.25,-0.5)--(4.75,-2.5)--(-1.75,-2.5)--cycle;
\coordinate (O) at (0,0);
\coordinate (M') at (-2.5,-2.5);
\coordinate (P') at (7.5,0);
\coordinate (N') at ($(M')+(P')-(O)$);
\coordinate (M) at ($(M')+(0,\h)$);
\coordinate (N) at ($(N')+(0,\h)$);
\coordinate (P) at ($(P')+(0,\h)$);
\coordinate (Q) at ($(O)+(0,\h)$);
\coordinate (I) at ($(Q)!0.5!(N)$);
\coordinate (K0) at ($(I)+(0,-1)$);
\coordinate (K1) at ($(I)+(0,-2.5)$);
\foreach \x in {M,N,P}
\draw[line width=3pt,red] (\x)--(\x');
\draw[line width=3pt,red] (O)--(Q);
\foreach \x/\g in {M/90,N/90,P/90,Q/180}
\fill[black] (\x) circle(3pt) +(\g:4mm) node {$\x$};
\draw (M)--(K0) (P)--(K0) (Q)--(K0) (N)--(K0);
\draw[dashed] (M)--(K1) (P)--(K1) (Q)--(K1) (N)--(K1);
\fill[black] (K0) circle(3pt) +(90:4mm) node {$K_0$};
\fill[black] (K1) circle(3pt) +(-90:4mm) node {$K_1$};
\fill[black] ($(K1)+(3,0)$) circle(3pt) +(-90:4mm) node {$A$};
\end{tikzpicture}\par}\vspace{-5pt}\noindent
Biết trung điểm đoạn $K_0K_1$ có tọa độ là $(a;b;c)$. Khi đó hãy tính giá trị biểu thức $T=5a+7b+9c$.
\shortans[]{$843$}
\loigiai{
\begin{center}
\begin{tikzpicture}[scale=0.7, font=\footnotesize,line join=round, line cap=round, >=stealth]
\path
(0,0) coordinate (Q)
++(-130:4) coordinate (M)
++(0:6) coordinate (N)
($(Q)+(N)-(M)$) coordinate (P)
($(Q)!1/2!(N)$) coordinate (K')
(0,-6) coordinate (O)
;
\foreach \i in {M,N,P}{
\coordinate (\i_1) at ($(\i)-(0,6)$);
}
\path
($(O)!.5!(M_1)$) coordinate (A)
($(O)!.5!(P_1)$) coordinate (B)
($(O)!.5!(N_1)$) coordinate (K)
($(K)!.3!(K')$) coordinate (K_1)
($(K)!2/3!(K')$) coordinate (K_0)
;
\draw (Q)--(M)--(N)--(P)--cycle;
\draw (Q)--(N) (P)--(M) (M)--(M_1) (N)--(N_1)
(P)--(P_1)
(M_1)--(N_1)--(P_1);
\foreach \i in{M,N,P,Q}{\draw[dashed,thin] (K_0)--(\i) (K_1)--(\i);};
\draw[dashed,thin]
(M_1)--(O)--(Q) (O)--(P_1)
(A)--(K)--(B)
(K)--(K');
\pic[draw,angle eccentricity=1.8,angle radius=2mm]{right angle=Q--O--M_1};
\foreach \i/\g in {Q/90,M/90,N/90,P/90,K'/90,K_0/-45,K_1/-180,K/-60,O/-60,M_1/-90,N_1/-90,P_1/-90}
\fill[red] (\i) circle(2pt)+(\g:5mm)node[black,scale=1]{$\i$};
\end{tikzpicture}
\end{center}
Gọi $M_1$, $N_1$, $P_1$, $K$ lần lượt là hình chiếu của $M$, $N$, $P$, $K_0$ lên mặt phẳng $(Oxy)$.\\
Ta thấy $MNPQ.M_1N_1P_1O$ là hình hộp chữ nhật. Gọi $K$ là giao hai đường chéo $MP$ và $NQ$. Khi đó $K'Q = K'P = K'N = K'M$. Vì $K_0M = K_0N = K_0P = K_0Q$. và camera được hạ thấp theo phương thẳng đứng từ điểm $K_0$ xuống điểm $K_1$ nên các điểm $K'$, $K_0$, $K_1$, $K$ thẳng hàng.\\
Do đó các điểm $K'$, $K_0$, $K_1$, $K$ có hoành độ và tung độ bằng nhau.\\
Theo bài ra, cao độ của $K_0$ và $K_1$ lần lượt là $25$ và $19$. Giả sử $K_0(x;y;25)$ và $K_1(x;y;19)$.\\
Ta có $MNPQ.M_1N_1P_1O$ là hình hộp chữ nhật nên $K'K = OQ$, suy ra cao độ của $K'$ bằng $30$. Do đó $K'(x;y;30)$.\\
Ta có $\overrightarrow{K'Q} = \overrightarrow{OQ} - \overrightarrow{OK'} = (-x;-y;0)$, $\overrightarrow{NK'} - \overrightarrow{OK'} - \overrightarrow{ON} = (x - 90; y - 120; 0)$.\\
Vì $K'$ là giao hai đường chéo của hình chữ nhật $MNPQ$ nên $K'$ là trung điểm của $NQ$.\\
Suy ra $\overrightarrow{K'Q} - \overrightarrow{NK'} \Leftrightarrow \heva{&-x = x - 90 \\&-y = y - 120 \\ &0 = 0} \Leftrightarrow \heva{&x = 45 \\ &y = 60.}$\\
Do vậy $K_0(45;60;25)$, $K_1(45;60;19)$, nên ta có tọa độ trung điểm của $K_0 K_1$ là $K_2=(45;60;22)$.\\
Vậy $5a + 7b + 9c = 843$.
}
\end{ex}

\begin{ex}%[2D6V2-4]%[TEX ĐỀ MOON 2025]%[Nguyễn Văn Hiệp]
Một công ty dược phẩm giới thiệu một dụng cụ để kiểm tra sớm bệnh sốt xuất huyết. Về báo cáo kiểm định chất lượng của sản phẩm, họ cho biết như sau: Số người được thử là $8\,000$, trong số đó có $1\,200$ người đã bị nhiễm bệnh sốt xuất huyết và có $6\,800$ người không bị nhiễm bệnh sốt xuất huyết. Nhưng khi kiểm tra lại bằng dụng cụ của công ty, trong $1\,200$ người đã bị nhiễm bệnh sốt xuất huyết, có $70\%$ số người đó cho kết quả dương tính, còn lại cho kết quả âm tính. Trong $6\,800$ người không bị nhiễm bệnh sốt xuất huyết, có $5\%$ số người đó cho kết quả dương tính, còn lại cho kết quả âm tính. Xác suất mà một bệnh nhân với kết quả kiểm tra dương tính là bị nhiễm bệnh sốt xuất huyết bằng bao nhiêu? (viết kết quả dưới dạng số thập phân và làm tròn đến hàng phần trăm).
\shortans{$0{,}71$}
\loigiai{
\textbf{Bước 1: Tính các xác suất}
\begin{itemize}
\item $A$: \lq\lq Người đã bị nhiễm sốt xuất huyết\rq\rq. $\mathrm{P}(A)=\dfrac{1\,200}{8\,000}=0{,}15$.
\item $\overline{A}$: \lq\lq Người không bị nhiễm sốt xuất huyết\rq\rq. $\mathrm{P}\left(\overline{A}\right)=\dfrac{6\,800}{8\,000}=0{,}85$.
\item $D$: \lq\lq Người được kiểm tra cho kết quả dương tính\rq\rq. Ta có $\mathrm{P}\left(D\mid A\right)=0{,}7$; $\mathrm{P}\left(D\mid \overline{A}\right)=0{,}05$.
\item Theo công thức xác suất đầy đủ \[\mathrm{P}(D)=\mathrm{P}(A)\cdot \mathrm{P}\left(D\mid A\right)+ \mathrm{P}\left(\overline{A}\right)\cdot \mathrm{P}\left(D\mid \overline{A}\right)=0{,}15\cdot 0{,}7+0{,}85\cdot 0{,}05=0{,}1475.\]
\end{itemize}
\textbf{Bước 2: Áp dụng công thức Bayes}
\[
\mathrm{P}\left(A\mid D\right) =\dfrac{\mathrm{P}\left(D\mid A\right)\cdot \mathrm{P}(A)} {\mathrm{P}(D)}= \dfrac{0{,}7\cdot 0{,}15}{0{,}1475} \approx 0{,}71.
\]
}
\end{ex}

\begin{ex}%[2D6V2-4]%[TEX ĐỀ MOON 2025]%[Nguyễn Cường]
Có hai chiếc hộp, hộp I có $6$ quả bóng màu đỏ và $4$ quả bóng màu vàng, hộp II có $7$ quả bóng màu đỏ và $3$ quả bóng màu vàng, các quả bóng có cùng kích thước và khối lượng. Lấy ngẫu nhiên một quả bóng từ hộp I bỏ vào hộp II. Sau đó, lấy ra ngẫu nhiên một quả bóng từ hộp II. Tính xác suất để quả bóng được lấy ra từ hộp II là quả bóng được chuyển từ hộp I sang, biết rằng quả bóng đó có màu đỏ (làm tròn kết quả đến hàng phần trăm).
\shortans{$0{,}08$}
\loigiai{
Gọi $A$ là biến cố \lq\lq quả lấy ra ở II là quả bóng được đưa từ I vào\rq\rq.
\\
Gọi $B$ là biến cố \lq\lq quả bóng lấy ra ở II là đỏ\rq\rq.
\\
$\mathrm{P}(B)$ xảy ra theo $2$ trường hợp:
\\
\textbf{TH1:} Chuyển một quả đỏ từ I sang II xác suất trường hợp này là $\dfrac{6}{10}\cdot \dfrac{8}{11}$.
\\
\textbf{TH2:} Chuyển một quả vàng từ I sang II xác suất trường hợp này là $\dfrac{4}{10}\cdot \dfrac{7}{11}$.
\\
Suy ra $\mathrm{P}(B)=\dfrac{6}{10}\cdot \dfrac{8}{11}+\dfrac{4}{10}\cdot \dfrac{7}{11}=\dfrac{38}{55}$.
\\
$A\cap B$ là biến cố \lq\lq quả bóng lấy ra ở II là đỏ và nó là quả bóng thuộc I\rq\rq.
\\
Phép thử gồm $2$ hành động: lấy $1$ quả ở I đưa vào II và từ II lấy $1$ quả.
\\
Không gian mẫu có $10\cdot 11=110$ kết quả.
\\
$A\cap B$ có số kết quả thuận lợi là $6\cdot 1=6$ kết quả.
\\
Suy ra $\mathrm{P}(A\cap B)=\dfrac{6}{110}$.
\\
Theo định lý Bayes ta có $\mathrm{P}(A\mid B)=\dfrac{\mathrm{P}(A\cap B)}{\mathrm{P}(B)}=\dfrac{\dfrac{6}{110}}{\dfrac{38}{55}}\approx 0{,}08$.
}
\end{ex}

\begin{ex}%[2D6V2-3]
Trong quân sự, một máy bay chiến đấu của đối phương có thể xuất hiện ở vị trí X với xác suất $0{,}55$. Nếu máy bay đó không xuất hiện ở vị trí X thì nó xuất hiện ở vị trí Y. Để phòng thủ, các bệ
phóng tên lửa được bố trí tại các vị trí X và Y. Khi máy bay đối phương xuất hiện ở vị trí X hoặc Y thì tên lửa sẽ được phóng để hạ máy bay đó. Xét phương án tác chiến sau: Nếu máy bay xuất hiện tại X thì bắn $2$ quả tên lửa và nếu máy bay xuất hiện tại Y thì bắn một quả tên lửa. Biết rằng, xác suất bắn trúng máy bay của mỗi quả tên lửa là $0{,}8$ và các bệ phóng tên lửa hoạt động độc lập. Máy bay bị bắn hạ nếu nó trúng ít nhất $1$ quả tên lửa. Biết máy bay bị bắn hạ trong phương án tác chiến trên. Tính xác suất máy bay bị bắn hạ ở vị trí X. \textit{(kết quả làm tròn đến hàng phần trăm)}.
\shortans{$0{,}59$}
\loigiai{
Gọi $A$ là biến cố \lq\lq Máy bay xuất hiện tại vị trí X\rq\rq.\\
$\overline{A}$ là biến cố \lq\lq Máy bay xuất hiện tại vị trí Y\rq\rq.\\
$B$ là biến cố \lq\lq Máy bay bị bắn hạ\rq\rq.\\
$\overline{B}$ là biến cố \lq\lq Máy bay không bị bắn hạ\rq\rq.\\
Tính $\mathrm{P}(A\mid B)$.\\
Từ giả thiết, ta có $\mathrm{P}(A)=0{,}55\Rightarrow \mathrm{P}(A)=1-\mathrm{P}(A)=0{,}45$.
\begin{itemize}
\item Xác suất máy bay bị bắn hạ tại vị trí Y là $\mathrm{P}\left(B\mid\overline{A}\right)=0{,}8$.
\item Xác suất máy bay không bị bắn hạ tại vị trí X là $\mathrm{P}(B\mid A)$.\\
Vì máy bay bị bắn hạ nếu bị trúng ít nhất $1$ quả tên lửa do
đó bay không bị bắn hạ khi và chỉ khi cả $2$ quả tên lửa đều không bắn trúng (và xác suất không bắn trúng là $1-0{,}8=0{,}2)$.\\
Nên $\mathrm{P}\left(\overline{B}\mid A\right)=0{,}2\cdot 0{,}2=0{,}04$.
\end{itemize}
Suy ra xác suất máy bay bị bắn hạ tại vị trí A là $\mathrm{P}(B\mid A)=1-\mathrm{P}\left(\overline{B}\mid A\right)=0{,}96$.\\
Từ đó, ta có sơ đồ cây sau
\begin{center}
\begin{tikzpicture}[declare function={dai=2.5;cao=0.65;},>=stealth,font=\scriptsize]
\tikzset{nhan/.style={minimum size=19pt,font=\small,inner sep=0pt}}
\path (0,0) node[nhan] (G){\text{Gốc}}
(dai,{1.5*cao}) node[nhan] (B) {$A$}
(dai,{-1.5*cao}) node[nhan] (nB) {$\overline{A}$}
({2*dai},{3*cao}) node[nhan] (BA) {$B$}
({2*dai},{cao}) node[nhan] (BnA) {$\overline{B}$}
({2*dai},{-cao}) node[nhan] (nBA) {$B$}
({2*dai},{-3*cao}) node[nhan] (nBnA) {$\overline{B}$};

%Phần mũi tên
\draw[->] (G.0)--(B.200) node[sloped,pos=0.5,above]{$0{,}55$};
\draw[->] (G.0)--(nB.160) node[sloped,pos=0.5,below]{$0{,}45$};
\draw[->] (B.10)--(BA.190) node[sloped,pos=0.5,above]{$0{,}96$};
\draw[->] (B.10)--(BnA.170) ;
\draw[->] (nB.-10)--(nBA.190) node[sloped,pos=0.5,above]{$0{,}8$};
\draw[->] (nB.-10)--(nBnA.170) ;
\end{tikzpicture}
\end{center}
Áp dụng công thức xác suất toàn phần, ta có \[\mathrm{P}(B)=\mathrm{P}(A)\cdot \mathrm{P}(B\mid A)+\mathrm{P}\left(\overline{A}\right)\cdot \mathrm{P}\left(B\mid\overline{A}\right)=0{,}55\cdot 0{,}96+0{,}45\cdot 0{,}8=0{,}888.\]
Áp dụng công thức Bayes, ta có $\mathrm{P}(A\mid B)=\dfrac{\mathrm{P}(A)\cdot \mathrm{P}(B\mid A)}{\mathrm{P}(B)}=\dfrac{0{,}55\cdot 0{,}96}{0{,}888}\approx 0{,}59$.
}
\end{ex}

\begin{ex}%[50 Đề minh họa tốt nghiệp 2025 - Đề 13]%[Lê Hữu Kiệt - Lê Quân]%[2D6V2-3]
Vắc xin AstraZeneca (AZD1222) được Tổ chức Y tế Thế giới (WHO) cấp phép sử dụng khẩn cấp giúp ngăn ngừa các triệu chứng nghiêm trọng và giảm tử vong do COVID-19. Vắc xin này được tiêm ở tỉnh X, thống kê cho thấy rằng: Với người có bệnh nền thì xác suất xảy ra phản ứng phụ sau tiêm là $28\%$, với người không có bệnh nền thì xác suất xảy ra phản ứng phụ sau tiêm là $17\%$. Chọn ngẫu nhiên một người được tiêm và thấy người này có phản ứng phụ. Tính xác suất để người này bị bệnh nền. Biết tỷ lệ người có bệnh nền ở tỉnh X là $12\%$. (làm tròn kết quả đến hàng phần trăm).
\par\shortans{$0{,}18$}
\loigiai{
Gọi
\begin{itemize}
\item $B$ là biến cố \lq\lq Người được chọn có bệnh nền\rq\rq.
\item $A$ là biến cố \lq\lq Người được chọn có phản ứng phụ\rq\rq.
\end{itemize}
Ta có sơ đồ cây
\begin{center}
\begin{tikzpicture}[font=\footnotesize, line join=round, line cap=round, >=stealth, scale=1]
\draw[->] (0,0)--++(10:3) coordinate (B) node[sloped, above, pos=0.5]{$0{,}12$}node[sloped, right]{$B$};
\draw[->] (B)+(0:0.5)--+(0:3) coordinate (A1) node[sloped, above=-1mm, pos=0.5]{$0{,}72$} node[sloped, right]{$\overline{A}$};
\draw[->] (B)+(0:0.5)--+(20:3) coordinate (A2) node[sloped, above, pos=0.5]{$0{,}28$} node[sloped, right]{$A$};
\draw[->] (0,0)--++(-10:3) coordinate (B') node[sloped, below, pos=0.5]{$0{,}88$}node[sloped, right]{$\overline{B}$};
\draw[->] (B')+(0:0.5)--+(0:3) coordinate (A3) node[sloped, above=-1mm, pos=0.5]{$0{,}17$} node[sloped, right]{$A$};
\draw[->] (B')+(0:0.5)--+(-20:3) coordinate (A4) node[sloped, below, pos=0.5]{$0{,}83$} node[sloped, right]{$\overline{A}$};
\end{tikzpicture}
\end{center}
Biến cố \lq\lq Chọn một người bị bệnh nền biết người này có phản ưng phụ\rq\rq\, là $B\mid A$. Áp dụng công thức Bayes ta có
\[ \mathrm{P}(B\mid A)
=\dfrac{\mathrm{P}(B) \cdot \mathrm{P}(A\mid B)}{\mathrm{P}(B) \cdot \mathrm{P}(A\mid B) + \mathrm{P}(\overline{B}) \cdot \mathrm{P}(A\mid \overline{B})}
=\dfrac{0{,}12\cdot0{,}28}{0{,}12\cdot0{,}28 + 0{,}88\cdot0{,}17}
=\dfrac{42}{229}
\approx 0{,}18.
\]
}
\end{ex}

\begin{ex}%[2D6V2-2]
Có hai hộp bóng bàn, các quả bóng bàn có kích thước và hình dạng như nhau. Hộp I chứa $3$ bóng bàn màu trắng và $2$ bóng bàn màu vàng, tổng cộng $5$ quả. Hộp II ban đầu chứa $6$ bóng bàn màu trắng và $4$ bóng bàn màu vàng, tổng cộng $10$ quả. Lấy ngẫu nhiên $4$ quả bóng bàn ở hộp I bỏ vào hộp II rồi lấy ngẫu nhiên $1$ quả bóng bàn từ hộp II ra. Tính xác suất để quả bóng bàn lấy từ hộp II có màu vàng.
\shortans{0{,}4}
\loigiai{
Gọi $A$ là biến cố \lq\lq Có đúng $1$ bóng vàng được chuyển từ hộp I sang hộp II\rq\rq.\\
Gọi $B$ là biến cố \lq\lq Lấy được bóng vàng từ hộp II\rq\rq.
Khi đó
\begin{align*}
\mathrm{P}(A) &=\dfrac{\mathrm{C}2^1 \cdot \mathrm{C}3^3}{\mathrm{C}4^5}=\dfrac{2 \cdot 1}{5}=\dfrac{2}{5}, \\
\mathrm{P}(\overline{A}) &=\dfrac{\mathrm{C}2^2 \cdot \mathrm{C}3^2}{\mathrm{C}4^5}=\dfrac{1 \cdot 3}{5}=\dfrac{3}{5}.
\end{align*}
Sau khi chuyển, tổng số bóng ở hộp II là $14$ quả.
\begin{itemize}
\item Nếu xảy ra $A$, số bóng vàng trong hộp II là $4+1=5 \Rightarrow \mathrm{P}(B\mid A)=\dfrac{5}{14}$.
\item Nếu xảy ra $\overline{A}$, số bóng vàng trong hộp II là $4+2=6 \Rightarrow \mathrm{P}(B\mid\overline{A})=\dfrac{6}{14}$.
\end{itemize}
Áp dụng công thức xác suất toàn phần, ta được
\begin{align*}
\mathrm{P}(B) &=\mathrm{P}(A) \cdot \mathrm{P}(B\mid A)+\mathrm{P}(\overline{A}) \cdot \mathrm{P}(B\mid\overline{A}) \\
&=\dfrac{2}{5} \cdot \dfrac{5}{14}+\dfrac{3}{5} \cdot \dfrac{6}{14}=\dfrac{10}{70}+\dfrac{18}{70}=\dfrac{28}{70}=\dfrac{2}{5}=0{,}4.
\end{align*}
}
\end{ex}

\begin{ex}%[2D6V2-2]%[TEX Đề Moon 2025]%[Võ Nguyên Thạch]
Có hai thùng I và II chứa các sản phẩm có khối lượng và hình dạng như nhau. Thùng I có $5$ chính phẩm và $4$ phế phẩm, thùng $2$ có $6$ chính phẩm và $8$ phế phẩm. Lấy ngẫu nhiên $1$ sản phẩm từ thùng I sang thùng II. Sau đó, lấy ngẫu nhiên $1$ sản phẩm từ thùng II để sử dụng. Xác suất lấy được chính phẩm từ thùng II là bao nhiêu (làm tròn kết quả đến hàng phần trăm)?
\shortans{0{,}44}
\loigiai{
Xét các biến cố $A$: \lq\lq Lấy được $1$ chính phẩm từ thùng I sang thùng II\rq\rq.\\
$B$: \lq\lq Lấy được $1$ chính phẩm từ thùng II\rq\rq.\\
Khi đó $\mathrm{P}(A)=\dfrac{5}{9}$; $\mathrm{P}(\overline{A})=\dfrac{4}{9}$; $\mathrm{P}(B|A)=\dfrac{7}{15}$; $\mathrm{P}(B|\overline{A})=\dfrac{6}{15}=\dfrac{2}{5}$.\\
Theo công thức xác suất toàn phần, xác suất của biến cố $B$ là
\[\mathrm{P}(B)=\mathrm{P}(A)\cdot \mathrm{P}(B|A)+\mathrm{P}(\overline{A})\cdot \mathrm{P}(B|\overline{A})=\dfrac{5}{9}\cdot \dfrac{7}{15}+\dfrac{4}{9}\cdot \dfrac{2}{5}\approx0{,}44.\]
}
\end{ex}

\begin{ex}%[2D6V2-2]%[2D6V2-2]%[TEX ĐỀ MOON 2025]%[Nguyễn Thế Duy]
Tất cả các học sinh của trường Hạnh Phúc đều tham gia câu lạc bộ bóng chuyền hoặc bóng rổ, mỗi học sinh chỉ tham gia đúng một câu lạc bộ. Có $60\%$ học sinh của trường tham gia câu lạc bộ bóng chuyền và $40\%$ học sinh của trường tham gia câu lạc bộ bóng rổ. Số học sinh nữ chiếm $65\%$ trong câu lạc bộ bóng chuyền và $25\%$ trong câu lạc bộ bóng rổ. Chọn ngẫu nhiên một học sinh. Xác suất chọn được học sinh nữ là bao nhiêu?
\shortans{$0{,}49$}
\loigiai{
Gọi $A$ là sự kiện \lq\lq Tham gia câu lạc bộ bóng rổ\rq\rq.\\
Suy ra $\overline{A}$ là sự kiện \lq\lq Tham gia câu lạc bộ bóng chuyền\rq\rq.\\
Theo đề bài ta có $P(A) - 0{,}4$ và $P \left(\overline{A} \right) = 0{,}6$.\\
Gọi $B$ là sự kiện \lq\lq chọn được học sinh nữ\rq\rq.\\
Theo đề bài ta có $P\left(B \mid A \right) = 0{,}25$ và $P \left(B \mid \overline{A} \right) = 0{,}65$.\\
Xác suất chọn được học sinh nữ là
\begin{align*}
P(B) &= P(A) \cdot P\left(B \mid \overline{A} \right) +  P\left(\overline{A} \right) \cdot P \left(B \mid \overline{A} \right)\\
&= 0{,}4 \cdot 0{,}25 + 0{,}6 \cdot 0{,}65 = 0{,}49.
\end{align*}
}
\end{ex}

\begin{ex}%[2D6V1-4]%[TEX Đề Moon 2025]%[Vũ Hồng Toàn]
Một xí nghiệp mỗi ngày sản xuất ra $2000$ sản phẩm trong đó có $39$ sản phẩm lỗi. Lần lượt lấy ra ngẫu nhiên hai sản phẩm không hoàn lại để kiểm tra. Tính xác suất của biến cố \lq\lq Sản phẩm lấy ra lần thứ hai bị lỗi\rq\rq\ (kết quả làm tròn đến hàng phần trăm).
\shortans{$0{,}02$}
\loigiai{
Xét các biến cố:\\
$A$ : Sản phẩm lấy ra lần thứ nhất bị lỗi.\\ Khi đó, ta có $P\left(A\right)=\frac{39}{2000}$ ; $P\left(\overline{A}\right)=\dfrac{1961}{2000}$.\\
$B$ : Sản phẩm lấy ra lần thứ hai bị lỗi.\\
Khi sản phẩm lấy ra lần thứ nhất bị lỗi thì còn $1999$ sản phẩm và trong đó có $38$ sản phẩm lỗi nên ta có $P\left(B \mid A\right)=\frac{38}{1999}$, suy ra $P\left(\overline{B} \mid A\right)=\dfrac{1961}{1999}$.\\
Khi sản phẩm lấy ra lần thứ nhất không bị lỗi thì còn $1999$ sản phẩm trong đó có $39$ sản phẩm lỗi nên ta có $P\left(B \mid \overline{A}\right)=\dfrac{39}{1999}$, suy ra $P\left(\overline{B} \mid \overline{A}\right)=\dfrac{1960}{1999}$.\\
Khi đó, xác suất để sản phẩm lấy ra lần thứ hai bị lỗi là
\[
P\left(B\right)=P\left(B \mid A\right) \cdot P\left(A\right)+P\left(B \mid \overline{A}\right) \cdot P\left(\overline{A}\right)=\dfrac{38}{1999} \cdot \dfrac{39}{2000}+\dfrac{39}{1999} \cdot \dfrac{1961}{2000} \approx 0{,}02.\]
}
\end{ex}

\begin{ex}%[2D6V1-2]%[TexDeMoon2025]%[NguyenKieuNhaTu]
Có hai chiếc hộp, hộp I có $6$ bi đỏ và $4$ bi trắng, hộp II có $7$ bi đỏ và $3$ bi trắng, các bi có cùng kích thước và khối lượng. Lấy ngẫu nhiên từ mỗi hộp ra hai bi. Tính xác suất để lấy được ít nhất một bi đỏ từ hộp I, biết rằng trong bốn bi lấy ra số bi đỏ bằng số bi trắng. (Làm tròn kết quả đến hàng phần trăm).
\shortans[]{$0{,}81$}
\loigiai{
Gọi
\begin{itemize}
\item $A\colon$ \lq\lq Lấy được ít nhất một bi đỏ từ hộp I\rq\rq. Suy ra $\overline{A}\colon$ \lq\lq Không lấy được bi đỏ từ hộp I\rq\rq.
\item $B\colon$ \lq\lq Trong bốn bi số bi đỏ bằng số bi trắng\rq\rq.
\item $\overline{A}B\colon$ \lq\lq Lấy $2$ bi trắng từ hộp I và lấy $2$ bi đỏ từ hộp II\rq\rq.
\end{itemize}
Xét biến cố $B$.
\begin{enumerate}[TH 1: ]
\item Hộp I 2 bi đỏ, Hộp II 2 bi trắng: $\mathrm{C}_6^2\cdot\mathrm{C}_3^2=45$.
\item Hộp I 2 bi trắng, Hộp II 2 bi đỏ: $\mathrm{C}_4^2\cdot\mathrm{C}_7^2=126$.
\item Hộp I 1 bi đỏ và 1 bi trắng, Hộp II 1 bi đỏ và 1 bi trắng: $6\cdot4\cdot7\cdot3=504$.
\end{enumerate}
$\Rightarrow n(B)=45+126+504=675$.\\
Lại có $n(\overline{A}B)=126$.\\
$\Rightarrow\mathrm{P}(\overline{A}\mid B)=\dfrac{n(\overline{A}B)}{n(B)}=\dfrac{126}{675}$.\\
Vậy $\mathrm{P}(A\mid B)=1-\mathrm{P}(\overline{A}\mid B)=1-\dfrac{126}{675}=\dfrac{61}{75}\approx0{,}81$.
}
\end{ex}

\begin{ex}%[2D4V3-5]
\immini{Ông Duy có một mảnh vườn hình vuông cạnh bằng $8$ m. Ông dự định xây một cái bể bơi đặc biệt (phần kẻ sọc trong hình vẽ bên). Biết $AM=\dfrac{AB}{4}$, phần đường cong đi qua các điểm $C$, $M$, $N$ là một phần của đường Parabol có trục đối xứng là $MP(MP\parallel AD)$ và chi phí để làm bể bơi là $5$ triệu đồng $/ $1$\mathrm{~m}^2$. Số tiền ông Duy phải trả để xây cái bể bơi đó là bao nhiêu triệu đồng? (làm tròn kết quả đến hàng đơn vị).}{\begin{tikzpicture}[line join=round, line cap=round,>=stealth,thick,scale=0.6]
\path
(0,8)coordinate (A)
(8,8)coordinate (B)
(8,0)coordinate (C)
(0,0)coordinate (D)
(2,0)coordinate (P)
(2,8)coordinate (M)
(0,64/9)coordinate (N)
;
\begin{scope}
\clip (-2,-2) rectangle (8,8);
\fill[pattern=north west lines]plot[samples=200,domain=0:8,smooth,variable=\x] (\x,{-2/9*(\x)^2+8/9*(\x)+64/9})--plot[samples=200,domain=8:0,smooth,variable=\x] (\x,{-8/9*(\x)+64/9})--cycle;
\draw[dashed] plot[samples=200,domain=0:8,smooth,variable=\x] (\x,{-2/9*(\x)^2+8/9*(\x)+64/9});
\draw plot[samples=200,domain=0:8.1,smooth,variable=\x] (\x,{-2/9*(\x)^2+8/9*(\x)+64/9});
\draw plot[samples=200,domain=0:8.1,smooth,variable=\x] (\x,{-8/9*(\x)+64/9});
\draw[samples=200,domain=0:8,smooth,variable=\x] plot (\x,{-2/9*(\x)^2+8/9*(\x)+64/9});
\draw[samples=200,domain=0:8,smooth,variable=\x] plot (\x,{-8/9*(\x)+64/9});
\end{scope}
\draw (A)--(B)--(C)--(D)--cycle;
\draw [dashed](M)--(P);
\foreach \x/\g in {A/180,B/0,C/0,P/-90,M/90,N/180,D/180} \fill[black] (\x) circle (1pt) ($(\x)+(\g:3mm)$)node{$\x$};
\end{tikzpicture}}
\shortans[]{$95$}
\loigiai{
\begin{center}
\begin{tikzpicture}[line join=round, line cap=round,>=stealth,thick,scale=0.6]
\draw[->] (-2.1,0)--(9.1,0) node[below left] {$x$};
\draw[->] (0,-2.1)--(0,9.1) node[below left] {$y$};
\draw (0,0) node [below left] {$O$};
\foreach \x/\nx in {8/8}
\draw[thin] (\x,1pt)--(\x,-1pt) node [below] {$\nx$};
\foreach \y/\ny in {8/8}
\draw[thin] (1pt,\y)--(-1pt,\y) node [left] {$\ny$};
\draw[dashed,thin](2,0)--(2,8)--(0,8);
\path
(0,8)coordinate (A)
(8,8)coordinate (B)
(8,0)coordinate (C)
(0,0)coordinate (D)
(2,0)coordinate (P)
(2,8)coordinate (M)
(0,64/9)coordinate (N)
;
\begin{scope}
\clip (-2,-2) rectangle (8,8);
\fill[pattern=north west lines]plot[samples=200,domain=0:8,smooth,variable=\x] (\x,{-2/9*(\x)^2+8/9*(\x)+64/9})--plot[samples=200,domain=8:0,smooth,variable=\x] (\x,{-8/9*(\x)+64/9})--cycle;
\draw[dashed] plot[samples=200,domain=0:8,smooth,variable=\x] (\x,{-2/9*(\x)^2+8/9*(\x)+64/9});
\draw plot[samples=200,domain=0:8.1,smooth,variable=\x] (\x,{-2/9*(\x)^2+8/9*(\x)+64/9});
\draw plot[samples=200,domain=0:8.1,smooth,variable=\x] (\x,{-8/9*(\x)+64/9});
\draw[samples=200,domain=0:8,smooth,variable=\x] plot (\x,{-2/9*(\x)^2+8/9*(\x)+64/9});
\draw[samples=200,domain=0:8,smooth,variable=\x] plot (\x,{-8/9*(\x)+64/9});
\end{scope}
\draw (A)--(B)--(C);
\foreach \x/\g in {A/40,B/0,C/40,P/-90,M/90,N/180,D/45} \fill[black] (\x) circle (1pt) ($(\x)+(\g:3mm)$)node{$\x$};
\end{tikzpicture}
\end{center}
Gắn trục tọa độ như hình vẽ.
Gọi phương trinh trình của Parabol là $(P)\colon y=ax^2+bx+c$.\\
Ta có $(P)$ đi qua điểm $C(8;0)$, $M(2;8)$ và có hoành độ đỉnh $x=2$ nên ta có hệ phương trình sau
\[\heva{&64a+8b+c=0\\&4a+2b+c=8\\&\dfrac{-b}{2a}=2}\Leftrightarrow \heva{&a=-\dfrac{2}{9}\\&b=\dfrac{8}{9}\\&c=\dfrac{64}{9}}\Rightarrow (P)\colon y=-\dfrac{2}{9}x^2+\dfrac{8}{9}x+\dfrac{64}{9}.\]
Giao điểm của $(P)$ với trục $Oy$ là điểm $N\left(0;\dfrac{64}{9}\right)$.\\
Gọi $d\colon y=ax+b$ là đường thẳng đi qua $N$ và $C$. Khi đó phương trình của $d$ là $y=-\dfrac{8}{9}x+\dfrac{64}{9}$.\\
Diện tích hình phẳng giới hạn bởi đồ thị $(P)$ và đường thẳng $d$ là
\[S=\displaystyle\int\limits_0^8 \left(-\dfrac{2}{9}x^2+\dfrac{8}{9}x+\dfrac{64}{9}+\dfrac{8}{9}x-\dfrac{64}{9}\right) \mathrm{\, d}x=\displaystyle\int\limits_0^8 \left(-\dfrac{2}{9}x^2+\dfrac{16}{9}x\right) \mathrm{\, d}x=\dfrac{512}{27}.\]
Vậy số tiền ông Duy phải trả để xây bể bơi là $\dfrac{512}{27}\cdot 5\approx 95$ triệu đồng.
}
\end{ex}

\begin{ex}%[2D4V3-5]
\immini{Một vật trang trí có dạng khối tròn xoay tạo thành khi quay miền $(R)$ (phần tô đậm trong hình vẽ) quay xung quanh trục $AB$. Miền $(R)$ được giới hạn bởi các cạnh $AD$, $A B$, $B C$, $EF$ và các cung phần tư của các đường tròn bán kính bằng $2$ cm với tâm lần lượt là trung điểm của $AD$ và $BC$. Biết $ABCD$ là hình chữ nhật có cạnh $AB=8$ cm, $AD=4$ cm; điểm $E$ cách $AD$ một đoạn bằng $2$ cm; điểm $F$ cách $BC$ một đoạn bằng $2$ cm. Thể tích của vật thể trang trí trên là bao nhiêu centimetkhối? \textit{(quy tròn đến hàng phần mười)}.}{
\begin{tikzpicture}
% Define coordinates for the points
\coordinate (A) at (0, 4);
\coordinate (B) at (0, 0);
\coordinate (C) at (2, 0);
\coordinate (D) at (2, 4);
\coordinate (E) at (1, 3);
\coordinate (F) at (1, 1);

\fill[gray, draw=black] (A) rectangle (C);
\fill[white, draw=black] (E)--(F) to [out=0, in=90] (C)--(D) to [out=-90, in=0] (E) ;
% Label points using \foreach \x/\y
\foreach \x/\y in {A/180, B/180, C/0, D/0, E/180, F/180} {\fill (\x) circle(1pt) ($(\x)+(\y:0.3cm)$) node{$\x$};}
\end{tikzpicture}
}
\shortans{$213$}
\loigiai{
Gắn hệ trục tọa độ $A(0;0)$, $D(0;4)$, $B(8;0)$, $C(8;4)$, $E(2;2)$, $F(6;2)$.
\begin{center}
\begin{tikzpicture}[rotate=90]
% Define coordinates for the points
\coordinate (A) at (0, 4);
\coordinate (B) at (0, 0);
\coordinate (C) at (2, 0);
\coordinate (D) at (2, 4);
\coordinate (E) at (1, 3);
\coordinate (F) at (1, 1);

\fill[gray, draw=black] (A) rectangle (C);
\fill[white, draw=black] (E)--(F) to [out=0, in=90] (C)--(D) to [out=-90, in=0] (E) ;
% Label points using \foreach \x/\y
\foreach \x/\y in {A/45, B/180, C/0, D/90, E/180, F/180} {\fill (\x) circle(1pt) ($(\x)+(\y:0.3cm)$) node{$\x$};}
\draw[->] (0,5)--(0,-1)node [above]{$x$};
\draw[->] (-1,4)--(3,4) node[above]{$y$};
\end{tikzpicture}
\end{center}
Phương trình đường tròn đi qua hai điểm $D$ và $E$ là
\begin{eqnarray*}
&&x^2+(y-2)^2=2^2\\
&\Rightarrow&(y-2)^2=4-x^2\\
&\Rightarrow& y-2=\sqrt{4-x^2}\\
&\Rightarrow& y=\sqrt{4-x^2}+2.
\end{eqnarray*}
Thể tích của vật thể trang trí là
\[V=2 \pi\displaystyle\int\limits_0^2\left[\left(\sqrt{4-x^2}+2\right)^2-2^2 \right]\mathrm{\, d} x+\pi\displaystyle\int\limits_0^8 2^2 \mathrm{\, d} x \approx 213{,}0.\]
}
\end{ex}

\begin{ex}%[2D4V3-4]%[TEX ĐỀ MOON 2025]%[Nguyễn Thế Duy]
Trong chương trình nông thôn mới, tại một xã $Y$ có xây một cây cầu bằng bê tông như hình vẽ. Đường cong trong hình vẽ là các đường Parabol, chọn hệ trục $Oxy$ như hình vẽ.
\begin{center}
\begin{tikzpicture}[scale=0.9,>=stealth, font=\footnotesize, line join=round, line cap=round]
\draw[smooth,samples=300,domain=-2:2] plot(\x,{-1/2*(\x)^2+2});
\draw[smooth,samples=300,domain=-1:1] plot(\x,{-(\x)^2+1});
\draw (-2,0)--(2,0) (-1.5,0)node[below]{$0{,}5$ m} (1.5,0)node[below]{$0{,}5$ m} (0,0)node[below]{$19$ m} (0,2)--(8,5) (2,0)--(10,3);
\draw[dashed] (0,0)--(0,2) (-2,0)--(6,3) (-1,0)--(7,3) (1,0)--(9,3) (2,0)--(10,3) (0,1)--(8,4) (8,4.5)node[right,xshift=-0.1cm]{$0{,}5$ m} (8,3.5)node[right,xshift=-0.1cm]{$2$ m} (6,1.5)node[below right]{$5$ m};
\draw[smooth,samples=300,domain=6:10,dashed] plot(\x,{-1/2*(\x-8)^2+5});
\draw[smooth,samples=300,domain=7:9,dashed] plot(\x,{-(\x-8)^2+4});
\draw[->,dashed] (6,3)--(10.5,3)node[below]{$x$};
\draw[->,dashed] (8,3)--(8,5.5)node[left]{$y$};
\fill[gray,opacity=0.3] plot[domain=-2:2](\x,{-1/2*(\x)^2+2})--plot[domain=1:-1](\x,{-(\x)^2+1})--cycle;
\fill[gray,opacity=0.5] plot[domain=0:2](\x,{-1/2*(\x)^2+2})--(10,3)--plot[domain=10:8](\x,{-1/2*(\x-8)^2+5})--(8,5)--cycle;
\end{tikzpicture}
\end{center}
Tính lượng bê tông để đổ cây cầu.
\shortans{$40$}
\loigiai{
Gọi $\left(P_1 \right) \colon y = a_1x^2 + b_1x + c_1$ là Parabol phía phần phía trong của cây cầu.\\
$\left(P_2 \right) \colon y = a_2x^2 + b_2x + c_2$ là Parabol phía phần phía ngoài của cây cầu.\\
Theo đề bài ta có $\left(P_1 \right)$ đi qua các điểm  $\left(9{,}5; 0 \right)$, $\left(-9{,}5; 0 \right)$ và $\left(0; 2 \right)$.\\
Từ đó ta được $\left(P_1 \right) \colon y = -\dfrac{8}{361}x^2 + 2$.\\
Lại có $\left(P_2 \right)$ đi qua các điểm $\left(-10;0 \right)$, $\left(0; 2{,}5\right)$ và $\left(10;0 \right)$.\\
Suy ra $\left(P_2 \right) \colon y = \dfrac{-1}{40}x^2 + 2{,}5$.\\
Diện tích phần giới hạn bởi hai Parabol là
\begin{align*}
S = \displaystyle\int_{-10}^{10} \left(\dfrac{-1}{40}x^2+2{,}5\right) \mathrm{\,d}x - \displaystyle\int_{-9{,}5}^{9{,}5} \left(\dfrac{-8}{361}x^2+2\right) \mathrm{\,d}x= 8 \, \text{m}^2.
\end{align*}
Thể tích lượng bê tông cần dùng là $V = 8 \cdot 5 = 40$ m$^3$.
}
\end{ex}

\begin{ex}%[2D4V3-3]%[Tex đề Moon 2025]%[Nguyễn Hồng Thạch]
\immini[thm]
{
Một vật trang trí có dạng một khối tròn xoay được tạo thành khi quay miền $(R)$ (phần được tô màu trong hình vẽ bên) quanh trục $AB$. Miền $(R)$ được giới hạn bởi các cạnh $AB$, $AD$ của hình vuông $ABCD$ và các cung phần tư của các đường tròn bán kính bằng $1$ cm với tâm lần lượt là trung điểm của các cạnh $AD$, $AB$. Tính thể tích của vật trang trí đó, làm tròn kết quả đến hàng phần mười của centimet khối.
}
{
\begin{tikzpicture}[scale=1.5,>=stealth, font=\footnotesize, line join=round, line cap=round]
\fill[gray,opacity=0.6] (0,2) arc(90:0:1 cm and 1 cm)--(1,1) arc(90:0:1 cm and 1 cm)--(0,0)--cycle;
\draw (0,0)node[below left]{$A$}--(2,0)node[below right]{$B$}--(2,2)node[above right]{$C$}--(0,2)node[above left]{$D$}--cycle (0,2) arc(90:0:1 cm and 1 cm) (1,1) arc(90:0:1 cm and 1 cm);
\end{tikzpicture}
}
\shortans{$12{,}3$}
\loigiai{
\begin{center}
\begin{tikzpicture}[scale=1.5,>=stealth, font=\footnotesize, line join=round, line cap=round]
\fill[gray,opacity=0.6] (0,2) arc(90:0:1 cm and 1 cm)--(1,1) arc(90:0:1 cm and 1 cm)--(0,0)--cycle;
\draw (0,0)node[below left]{$A\equiv O$}--(2,0)node[above right]{$B$}--(2,2)node[above right]{$C$}--(0,2)node[above right]{$D$}--cycle (0,2) arc(90:0:1 cm and 1 cm) (1,1) arc(90:0:1 cm and 1 cm);
\draw[->] (-1,0)--(3,0);
\draw[->] (0,-1)--(0,3);
\fill (3,0)node[above]{$x$}
(2,0)node[below]{$2$}circle(0.8pt)
(0,3)node[left]{$y$}
(0,2)node[left]{$2$}circle(0.8pt)
(1,0)node[below]{$1$}circle(0.8pt)
(0,1)node[left]{$1$}circle(0.8pt)
(1,1)node[above right]{$E$}circle(0.8pt);
\end{tikzpicture}
\end{center}
Chọn hệ trục như hình vẽ.\\
Ta có phương trình đường tròn đường kính $AD$ là $x^2+(y-1)^2=1$.\\
$\Rightarrow$ phương trình cung tròn $DE$ là $y=1+\sqrt{1-x^2}$.\\
Ta có phương trình đường tròn đường kính $AB$ là $(x-1)^2+y^2=1$.\\
$\Rightarrow$ phương trình cung tròn $BE$ là $y=\sqrt{1-(x-1)^2}$.\\
Thể tích khối tròn xoay là
\[V=\pi\cdot\displaystyle\int_{0}^{1}(1+\sqrt{1-x^2})^2\mathrm{\,d}x+\pi\cdot\displaystyle\int_{1}^{2} \left(\sqrt{1-(x-1)^2}\right)^2\mathrm{\,d}x\approx 12{,}3.\]
}
\end{ex}

\begin{ex}%[2D4V3-2]%[TEX ĐỀ MOON 2025]%[Nguyễn Văn Hiệp]
Bác Năm làm một cái cửa nhà hình parabol có chiều cao từ mặt đất đến đỉnh là $2{,}25$ mét, chiều rộng tiếp giáp với mặt đất là $3$ mét. Giá thuê mỗi mét vuông là $150\,000$ đồng. Vậy số tiền bác Năm phải trả là bao nhiêu triệu đồng?
\shortans{$6{,}75$}
\loigiai{
\textbf{Bước 1: Thiết lập phương trình parabol}
\immini{Chọn hệ trục tọa độ $Oxy$ như hình vẽ. \\
Phương trình parabol $(P)\colon y=ax^2+bx+c$, ($a$, $b$, $c\in \mathbb{R}$).\\
Các điểm $(0;2{,}25)$, $(-1{,}5;0)$, $(1{,}5;0)$ thuộc đồ thị hàm số nên
\[\heva{&a\cdot 0^2+b\cdot 0+c=2{,}25\\
&a\cdot (-1{,}5)^2+b\cdot (-1{,}5)+c=0\\
&a\cdot (1{,}5)^2+b\cdot (1{,}5)+c=0
}\Leftrightarrow \heva{&a=-1\\&b=0\\&c=2{,}25.}\]
Vậy $(P)\colon y=-x^2 +2{,}25$.
}{\begin{tikzpicture}[line join=round, line cap=round,>=stealth,thick]
\tikzset{every node/.style={scale=0.9}}
\draw[->] (-2.1,0)--(2.1,0) node[below left] {$x$};
\draw[->] (0,-0.6)--(0,3.2) node[below left] {$y$};
\draw (0,0) node [below left] {$O$};
\foreach \x/\nx in {-1.5/-1.5,1.5/1.5}
\draw[thin] (\x,1pt)--(\x,-1pt) node [below] {$\nx$};
\foreach \y/\ny in {2.25/2.25}
\draw[thin] (1pt,\y)--(-1pt,\y) node [above left] {$\ny$};
\begin{scope}
\clip (-2,-0.5) rectangle (2,2.8);
\draw[samples=200,domain=-1.5:1.5,smooth,variable=\x] plot (\x,{-1*(\x)^2+0*(\x)+2.25});
\end{scope}
\end{tikzpicture}}
\noindent \textbf{Bước 2: Tính diện tích} \\
Do $(P)$ đối xứng qua trục $Oy$ nên ta có
\[
S = 2\displaystyle\int\limits_0^{1{,}5} \left(-x^2 + 2{,}25\right) \mathrm{\,d}x = 4{,}5 \,\text{ m}^2.
\]
\textbf{Bước 3: Tính chi phí}
\[
4{,}5 \times 150\,000 = 675\,000 \,\text{ đồng} = 6{,}75 \,\text{ triệu đồng}.
\]
}
\end{ex}

\begin{ex}%[2D4V3-2]%[TEX ĐỀ MOON 2025]%[Nguyễn Cường]
\immini{
Một kiến trúc sư thiết kế một khu sinh hoạt cộng đồng có dạng hình vuông với mỗi cạnh dài $120$ m. Phần sân chơi nằm ở giữa, và phần còn lại để trồng cây xanh. Các đường biên của khu vực trồng cây xanh là các đoạn parabol, với đỉnh của parabol nằm cách trung điểm của mỗi cạnh hình vuông $25$ m. Tính diện tích phần trồng cây xanh.
}
{
\begin{tikzpicture}[scale=0.7,>=stealth, font=\footnotesize, line join=round, line cap=round]
\fill[gray,opacity=0.5] plot[domain=-3:3](\x,{1/6*(\x)^2+1.5})--plot[domain=3:-3](\x,{-1/6*(\x)^2-1.5})--cycle;
\draw (-3,3)--(-3,-3)--(3,-3)--(3,3)--cycle;
\draw[smooth,samples=300,domain=-3:3] plot(\x,{1/6*(\x)^2+1.5});
\draw[smooth,samples=300,domain=-3:3] plot(\x,{-1/6*(\x)^2-1.5});
\draw[dashed] (0,1.5)--(0,-1.5);
\draw[<->] (0,1.5)--(0,3);
\draw[<->] (0,-1.5)--(0,-3);
\draw (0,2.25)node[right]{$25$ m} (0,-2.25)node[right]{$25$ m} (0,-3)node[below]{$120$ m} (3,0)node[right]{$120$ m};
\end{tikzpicture}
}
\shortans{$4000$}
\loigiai{
\immini{
Dựng hệ trục $Oxy$ như hình vẽ, dễ thấy Parabola $(P)$ có phương trình
$(P)\colon y=ax^2+b$.
\\
Đồng thời $(P)$ đi qua điểm $(60;0)$ và $(0;25)$ nên ta có hệ phương trình $\heva{&3600a+b=0\\&b=25}\Leftrightarrow \heva{&a=-\dfrac{1}{144}\\&b=25.}$
\\
Suy ra $(P)\colon y=-\dfrac{1}{144}x^2+25$.
\\
Diện tích một nửa phần trồng hoa là
\[
\displaystyle \int\limits_{-60}^{60} \left(-\dfrac{1}{144}x^2+25\right) \mathrm{\,d}x=2000.
\]
Vậy diện tích phần trồng cây xanh là $4000$ (m$^2$).
}
{
\begin{tikzpicture}[scale=0.7,>=stealth, font=\footnotesize, line join=round, line cap=round]
\fill[gray,opacity=0.5] plot[domain=-3:3](\x,{1/6*(\x)^2+1.5})--plot[domain=3:-3](\x,{-1/6*(\x)^2-1.5})--cycle;
\draw (-3,3)--(-3,-3)--(3,-3)--(3,3)--cycle;
\draw[smooth,samples=300,domain=-3:3] plot(\x,{1/6*(\x)^2+1.5});
\draw[smooth,samples=300,domain=-3:3] plot(\x,{-1/6*(\x)^2-1.5});
\draw[dashed] (0,1.5)--(0,-1.5);
\draw[<->] (0,1.5)--(0,3);
\draw[<->] (0,-1.5)--(0,-3);
\draw (0,2.25)node[right]{$25$ m} (0,-2.25)node[right]{$25$ m} (3,0)node[right]{$120$ m};
\draw[->](-4,-3)--(4,-3)node[below]{$x$} ;
\draw[->](0,-4)--(0,-3)node[below right]{$O$}--(0,4)node[right]{$y$} ;
\end{tikzpicture}
}
}
\end{ex}

\begin{ex}%[2D4V3-2]%[TEX ĐỀ MOON 2025]%[Huỳnh Thanh Chí]
Một bể chứa nhiên liệu hình trụ đặt nằm ngang, có chiều dài $5$ m, có bán kính đáy $1$ m. Chiều cao của mực nhiên liệu là $1{,}5$ m.
\begin{center}
\begin{tikzpicture}[declare function={r=2;h=6;gM=70;gN=40;d=1.5;a=0.5;},scale=0.8]
\path (0:0) coordinate (O)
(90:r) coordinate (A)
(-90:r) coordinate (B)
(A) arc (90:90+gM:{r/2} and {r}) coordinate (M)
(A) arc (90:90-gN:{r/2} and {r}) coordinate (N)
\foreach \x in {O,A,B,M,N}{(\x)++(0:h) coordinate (\x_1)}
(intersection of A_1--B_1 and M_1--N_1) coordinate (I)
($(A_1)+(0:d)$) coordinate (H)
($(B_1)+(0:d)$) coordinate (K)
($(B)+(-90:a)$) coordinate (E)
($(B_1)+(-90:a)$) coordinate (F)
;
\fill[gray!60] (M) arc (90+gM:270:{r/2} and {r})--(B) arc (-90:90-gN:{r/2} and {r})--cycle
(M_1) arc (90+gM:270:{r/2} and {r})--(B_1) arc (-90:90-gN:{r/2} and {r})--cycle
(M)--(B)--(B_1)--(M_1)--(N_1)--(N)--cycle;
\draw (A) arc (90:270:{r/2} and {r})
(A_1) arc (90:270:{r/2} and {r})
(A_1) arc (90:-90:{r/2} and {r})
(A)--(A_1) (B)--(B_1) (M)--(M_1)
(M_1)--(N_1)
;
\draw[dashed] (A) arc (90:-90:{r/2} and {r})
(M)--(N) (N)--(N_1)
;
\draw[>=stealth,<->] (A_1)--(I);
\node at ($(I)+(-90:0.6)$) {$0{,}5$ m};
\draw[>=stealth,<->] (H)--(K)node[midway,right]{$2$ m};
\draw[>=stealth,<->] (E)--(F)node[midway,below]{$5$ m};
\end{tikzpicture}
\end{center}
Tính thể tích phần nhiên liệu trong bể (theo đơn vị m$^3$, làm tròn đến hàng phần chục).

\shortans[]{$12{,}6$}
\loigiai{
\immini{Thể tích phần nhiên liệu sẽ bằng diện tích hình phẳng gạch sọc trong hình nhân với chiều dài của bồn (chiều cao của trụ).\\
Đường tròn có tâm $O(0;0)$, $R=1$ có phương trình là $x^2+y^2=1\Leftrightarrow y=\pm\sqrt{1-x^2}$. \\
Diện tích hình gạch sọc chính là diện tích hình phẳng giới hạn bởi các đường $y=\sqrt{1-x^2}$; $y=-\sqrt{1-x^2}$; $x=-1$; $x=0{,}5$.}
{\begin{tikzpicture}[scale=1,font=\footnotesize,line join=round,line cap=round,>=stealth,declare function={r=2;}]
\path (0,0) coordinate (O)
(60:r) coordinate (A)
(-60:r) coordinate (B)
(90:r) coordinate (C)
(-90:r) coordinate (D)
(180:r) coordinate (E)
(0:r) coordinate (F)
;
\fill[pattern=north east lines,pattern color=green] (A)--(A) arc (60:300:r)--(B)--cycle;
\draw[-stealth] (-r-0.5,0)--(0,0)node[below left]{$O$}--(r+0.5,0)node[below]{$x$};
\draw[-stealth] (0,-r-0.5)--(0,r+0.5)node[left]{$y$};
\draw (O) circle (r) (A)--(B);
\foreach \x in {O,A,B,C,D,E,F}{\draw[fill=green] (\x) circle (1pt);}
\node[above right] at (C) {$1$};
\node[below right] at (D) {$-1$};
\node[below left] at (E) {$-1$};
\node[below left] at (F) {$1$};
\node[above right] at (r/2,0) {$0{,}5$};
\end{tikzpicture}}
Do đó $V=S\cdot h=5\displaystyle\int\limits_{-1}^{0{,}5}\left|\sqrt{1-x^2}-\left(-\sqrt{1-x^2}\right)\right| \mathrm{d}x\approx 12{,}6$ m$^3$.
}
\end{ex}

\begin{ex}%[2D4V3-2]
\immini{Từ một tấm tôn phẳng hình chữ nhật có chiều dài $8$ cm, chiều rộng $5$ cm có gắn hệ toạ độ $Oxy$ như hình vẽ bên. Thầy Tuấn cắt miếng tôn theo ba đường: Đường cong $AIB$ là một phần của Parabol, các đường cong $AE$, $EB$ là một phần đồ thị hàm số bậc ba. Trang trí phần còn lại để tạo thành một chiếc mặt nạ đồ chơi có trục đối xứng là trục $Oy$. Biết đường cong $EB$ đi qua các điểm $(1;-2)$ và $(3;-3)$.}{\begin{tikzpicture}[very thick,>=stealth',scale=0.9]
\tikzset{declare function={xmin=-4;xmax=4;
ymin=-4;ymax=1;
f(\x)=1 - (\x)^2/16;
g(\x)=1/2*(\x)^3-17*(\x)^2/6+13*(\x)/3-4;
},
smooth,samples=50
}
\draw[black,thick] (-4,-4) rectangle (4,1);
\draw (-2,-1) circle (0.5 cm);
\draw (2,-1) circle (0.5 cm);
\draw[->] (xmin-0.25,0)--(xmax+0.5,0)
node[shift={(-100:7pt)},font=\normalsize]{$x$};
\draw[->] (0,ymin-0.25)--(0,ymax+0.5)
node[shift={(170:7pt)},font=\normalsize]{$y$};
\fill (0,0) node[shift={(135:9pt)},font=\normalsize]{$O$};
\foreach \x in {-4, 4}{
\draw (\x,2pt)--(\x,-2pt) +(0,-9pt) node[shift={(-10:5pt)},font=\footnotesize,fill=white,inner sep=1pt]{$\x$};
}
\foreach \y in {-4, 1}{
\draw (2pt,\y)--(-2pt,\y) +(-3pt,0) node[shift={(135:9pt)},font=\footnotesize,fill=white,inner sep=1pt]{$\y$};
}
\begin{scope}
\clip (xmin,ymin) rectangle (xmax,ymax);
\draw[black,thick] plot[domain=xmin:xmax] (\x, {f(\x)});
\draw[black,thick] plot[domain=xmin:xmax] (\x, {g(\x)});
\draw[black,thick] plot[domain=xmin:xmax] (\x, {g(-\x)});
\end{scope}
\fill
(4,0)circle(1.5pt)node[above right]{$B$}
(-4,0)circle(1.5pt)node[above left]{$A$}
(0,1)circle(1.5pt)node[above right]{$I$}
(0,-4)circle(1.5pt)node[above right]{$E$}
;
\end{tikzpicture}}
\noindent Tính diện tích chiếc mặt nạ đồ chơi của thầy Tuấn (làm tròn đến hàng phần mười theo đơn vị cm$^2$).

\shortans{24{,}9}
\loigiai{
Giả sử đường cong $AIB$ có phương trình là $y = f(x) = mx^2 + nx + p$.\\
Đường cong $EB$ có phương trình là $y = g(x) = ax^3 + bx^2 + cx + d$. \\
Vì mặt nạ đối xứng qua trục $Oy$ nên diện tích của mặt nạ bằng $ 2\displaystyle\int_{0}^{4} |f(x) - g(x)|\mathrm{\,d}x$. \\
\textbf{Viết phương trình của $f(x)$: }\\
Ta có đường cong $AIB$ đi qua các điểm $A(-4; 0)$; $I(0; 1)$ và $B(4; 0)$. Từ đó ta có hệ phương trình
\[\heva{&m\cdot(-4)^2-4n+p=0\\&p=1\\&m\cdot4^2+4n+p=0}\Leftrightarrow\heva{&m=-\dfrac{1}{16}&\\&n=0\\&p=1}\Rightarrow f(x)=-\dfrac{1}{16}x^2+1.\]
\textbf{Viết phương trình của $g(x)$: }\\
Vì đồ thị $g(x)$ đi qua $E(0;-4)$ nên $d=-4$. Suy ra $g(x)=ax^3+bx^2+cx-4$.\\
Lại có đường cong $EB$ qua $B(4;0)$ và các điểm $(1;-2)$ và $(3;-3)$. Từ đó ta có hệ phương trình
\[\heva{&a\cdot4^3+b\cdot4^2+4c-4=0\\&a+b+c-4=-2\\&a\cdot3^3+b\cdot3^2+3c-4=-3}\Leftrightarrow\heva{&64a+16b+4c=4\\&a+b+c=2\\&27a+9b+3c=1}\Leftrightarrow \heva{&a=\dfrac{1}{2}\\&b=-\dfrac{17}{6}\\&c=\dfrac{13}{3}.}\]
Suy ra $g(x)=\dfrac{1}{2}x^3-\dfrac{17}{6}x^2+\dfrac{13}{3}x-4$.\\
Suy ra diện tích mặt nạ là
\[2\displaystyle\int_{0}^{4} \left| -\dfrac{1}{16}x^2+1 - \dfrac{1}{2}x^3+\dfrac{17}{6}x^2-\dfrac{13}{3}x+4\right| \mathrm{\,d}x\approx24{,}9.\]
}
\end{ex}

\begin{ex}%[2D4V3-2]%[TexDeMoon2025]%[NguyenKieuNhaTu]
Một biển quảng cáo có dạng hình elip với bốn đỉnh $A_1$, $A_2$, $B_1$, $B_2$ như hình vẽ bên dưới. Biết chi phí để sơn phần tô đậm là $200.000$ (đồng) và phần còn lại $100.000$ (đồng). Biết $A_1A_2=8$ m, $B_1B_2=6$ m và tứ giác $MNPQ$ là hình chữ nhật có $MQ=3$ m.

{\centering\begin{tikzpicture}[scale=0.5,>=stealth, font=\footnotesize, line join=round, line cap=round]
\draw (0,0) ellipse (4 cm and 3 cm);
\draw (-3,2)node[above left]{$M$}--(3,2)node[above right]{$N$}--(3,-2)node[below right]{$P$}--(-3,-2)node[below left]{$Q$}--cycle;
\fill (4,0)node[right]{$A_2$} circle(3pt);
\fill (-4,0)node[left]{$A_1$} circle(3pt);
\fill (0,3)node[above]{$B_2$} circle(3pt);
\fill (0,-3)node[below]{$B_1$} circle(3pt);
\fill[gray,opacity=0.4] plot[domain=-3:3](\x,{3*sqrt(1-(\x)^2/16)})--plot[domain=3:-3](\x,{-3*sqrt(1-(\x)^2/16)})--cycle;
\end{tikzpicture}\par}\vspace{-5pt}\noindent
Hỏi số tiền để sơn theo cách trên (làm tròn đến hàng phần chục, đơn vị triệu đồng) bằng bao nhiêu?
\shortans[]{$7{,}3$}
\loigiai{
\begin{center}
\begin{tikzpicture}[scale=0.5,>=stealth, font=\footnotesize, line join=round, line cap=round]
\draw[->] (-6,0)--(0,0)node[above right]{$O$}--(6,0) node [above]{$x$};
\draw[->] (0,-5)--(0,-1.5)node[right]{$-1{,}5$}--(0,1.5)node[right]{$1{,}5$}--(0,5) node [left]{$y$};
\draw (0,0) ellipse (4 cm and 3 cm);
\draw (-3,2)node[above left]{$M$}--(3,2)node[above right]{$N$}--(3,-2)node[below right]{$P$}--(-3,-2)node[below left]{$Q$}--cycle;
\fill (4,0)node[below right]{$4$} circle(3pt);
\fill (-4,0)node[below left]{$-4$} circle(3pt);
\fill (0,3)node[above right]{$3$} circle(3pt);
\fill (0,-3)node[below right]{$-3$} circle(3pt);
\fill[gray,opacity=0.4] plot[domain=-3:3](\x,{3*sqrt(1-(\x)^2/16)})--plot[domain=3:-3](\x,{-3*sqrt(1-(\x)^2/16)})--cycle;
\end{tikzpicture}
\end{center}
Phương trình của elip có trục lớn $A_1A_2=8$ m và trục nhỏ $B_1B_2=6$ m là
\[\dfrac{x^2}{4^2}+\dfrac{y^2}{3^2}=1\Leftrightarrow\dfrac{x^2}{16}+\dfrac{y^2}{9}=1\Leftrightarrow y=\pm \dfrac{3}{4}\sqrt{16-x^2}.\]
Ta có $MQ=3$ nên $y_N=1{,}5$. Suy ra $x_N=2\sqrt{3}$.\\
Suy ra hình chữ nhật $MNPQ$ có $MN=4\sqrt{3}$.\\
Diện tích phần tô đậm là
\[S_1=2\displaystyle\int_{-2\sqrt{3}}^{2\sqrt{3}}\dfrac{3}{4} \sqrt{16-x^2} \mathrm{\,d}x.\]
Diện tích phần không tô là \[S_2=4\displaystyle\int_{2\sqrt{3}}^{4}\dfrac{3}{4} \sqrt{16-x^2} \mathrm{\,d}x.\]
Vậy tổng số tiền là $T=200\,000\cdot10^{-6}\cdot S_1+100\,000\cdot10^{-6}\cdot S_2\approx 7{,}3$ (triệu đồng).
}
\end{ex}

\begin{ex}%[2D4V2-6]%[TEX Đề Moon 2025]%[Võ Nguyên Thạch]
\immini[thm]
{
Một người có miếng tôn hình tròn có bán kính bằng $5$ (m). Người này tính trang trí sơn vẽ trên tấm tôn đó, biết mỗi mét vuông sơn hết 100 nghìn đồng. Tuy nhiên cần có một khoảng trống để treo tấm tôn nên người này bớt lại một phần tấm tôn nhỏ không trang trí (phần màu trắng như hình vẽ), trong đó $AB=6$(m). Hỏi khi trang trí xong người này hết bao nhiêu tiền chi phí (đơn vị nghìn đồng)?
}
{
\begin{tikzpicture}[scale=0.96,>=stealth, font=\footnotesize, line join=round, line cap=round]
\draw (0,0) circle(2cm);
\fill[blue,opacity=0.4] (1.41,1.41)--plot[domain=1.41:-2,smooth,samples=300](\x,{sqrt(4-(\x)^2)})--plot[domain=-2:1.41,smooth,samples=300](\x,{-sqrt(4-(\x)^2)})--cycle;
\draw (1.41,1.41)circle (1pt)node[above right]{$A$} (1.41,-1.41)circle (1pt)node[below right]{$B$};
\end{tikzpicture}
}
\shortans{7\,445}
\loigiai{
\begin{center}
\begin{tikzpicture}[scale=0.96,>=stealth, font=\footnotesize, line join=round, line cap=round]
\draw [->] (-2.5,0)--(2.5,0) node[below]{$x$};
\draw [->] (0,-2.5)--(0,2.5) node[left]{$y$};
\draw (0,0) node[below left]{$O$};
\draw (0,0) circle(2cm);
\fill[blue,opacity=0.4] (1.41,1.41)--plot[domain=1.41:-2,smooth,samples=300](\x,{sqrt(4-(\x)^2)})--plot[domain=-2:1.41,smooth,samples=300](\x,{-sqrt(4-(\x)^2)})--cycle;
\draw[fill=black] (1.41,1.41)circle (1pt)node[above right]{$A$} (1.41,-1.41)circle (1pt)node[below right]{$B$} (1.41,0)circle (1pt)node[above right]{$H$};
\end{tikzpicture}
\end{center}
Diện tích miếng tôn là $S_1=\pi R^2=25\pi$ (m$^2$).\\
Chọn hệ trục tọa độ $Oxy$ như hình vẽ.\\
Ta có phương trình của đường tròn biên là $x^2+y^2=25$ nên $R=5$; $AH=3\Rightarrow OH=4$.\\
Phương trình của cung tròn nhỏ là $y=\sqrt{25-x^2}$, với $0\le x\le 5$.\\
Diện tích phần không tô màu là $S_2=2\displaystyle\int\limits_4^5{\sqrt{25-x^2}\mathrm{\,d}x}$ (m$^2$).\\
Diện tích phần tô màu là
\[S=S_1-S_2=25\pi-2\displaystyle\int\limits_4^5{\sqrt{25-x^2}\mathrm{\,d}x}~(\text{m}^2).\]
Số tiền thu được là
\[T=100S=10\left(25\pi-2\displaystyle\int\limits_4^5{\sqrt{25-x^2}\mathrm{\,d}x}\right)\approx 7\,445~\text{(nghìn đồng)}.\]
}
\end{ex}

\begin{ex}%[2D1V3-6]%[9D0G1-Ứng dụng cực trị trong thực tế]
Trong một bài thực hành huấn luyện quân sự có một tình huống chiến sĩ phải bơi qua sông để tấn công mục tiêu ở ngay phía bờ bên kia sông. Biết rằng lòng sông rộng $100$ m và vận tốc bơi của chiến sĩ ($v_b$) bằng một phần ba vận tốc chạy trên bộ ($v_c$), tức là $v_c=3v_b$. Biết dòng sông là thẳng, mục tiêu cách chiến sĩ $1$ km theo đường chim bay và chiến sĩ đang ở bờ bên này. Hỏi chiến sĩ phải bơi bao nhiêu mét để đến được mục tiêu nhanh nhất (làm tròn kết quả đến hàng đơn vị)?
\shortans{106}
\loigiai{
\begin{center}
\begin{tikzpicture}[>=stealth,line join=round,line cap=round,font=\footnotesize,scale=1,ultra thick]
\path
(0,0)coordinate (x)+(-90:5)coordinate (x')+(0:3)coordinate(y)
($(x')+(y)-(x)$)coordinate (y')
($(x)!.8!(x')$)coordinate (A)
($(y)!.1!(y')$)coordinate (C)
($(y)!(A)!(y')$)coordinate (B)
($(B)!.3!(C)$)coordinate (D)
;
\path
(A)--(B)node[below,pos=.5]{$100$ m}
(B)--(D)node[right,pos=.5]{$x$}
(A)--(C)node[pos=.5,sloped,above]{$1$ km}
;
\foreach \pointo/\pointt in{x/x',y/y',A/B,A/D,A/C}{
\draw[fill=black](\pointo)--(\pointt);
}
\foreach \point/\goc in{A/180,B/0,D/0,C/0}{
\draw[fill=black](\point)circle(.8pt)+(\goc:2mm)node[scale=.8]{$\point$};
}
\end{tikzpicture}
\end{center}

Gọi vận tốc của chiến sĩ khi bơi là $a$ (m/s), với $a > 0$.\\
$\Rightarrow$ Vận tốc của chiến sĩ khi chạy bộ là $3a$ (m/s).\\
Ta có hình vẽ, khi đó chiến sĩ ở vị trí $A$, mục tiêu ở vị trí $C$.\\
Quãng đường chiến sĩ phải bơi là $AD$, quãng đường chiến sĩ phải chạy bộ là $DC$.\\
Ta có
\[
BC=\sqrt{A\mathrm{C}2-AB^2}=\sqrt{1000^2-100^2}=300\sqrt{11} \text{(m)}.
\]
Đặt $BD=x$ (m), với $0 < x < 300\sqrt{11}$.\\
$\Rightarrow$ Quãng đường chiến sĩ phải bơi là
\[
AD=\sqrt{AB^2+BD^2}=\sqrt{x^2+100^2} \text{(m)}.
\]
Quãng đường chiến sĩ phải chạy bộ là
\[
CD=BC-BD=300\sqrt{11}-x \text{(m)}.
\]
$\Rightarrow$ Thời gian chiến sĩ đến được mục tiêu là
\[
t=\dfrac{AD}{a}+\dfrac{DC}{3a}=\dfrac{\sqrt{x^2+100^2}}{a}+\dfrac{300\sqrt{11}-x}{3a}
=\dfrac{1}{3a} \left( 3\sqrt{x^2+100^2}+300\sqrt{11}-x \right).
\]
Xét hàm số:
\[
f(x)=3\sqrt{x^2+100^2}-x+300\sqrt{11} \text{trên} \left(0;\, 300\sqrt{11}\right),
\]
ta có
\[
f'(x)=\dfrac{3x}{\sqrt{x^2+100^2}}-1.
\]
Khi đó \allowdisplaybreaks
\begin{eqnarray*}
f'(x)=0 &\Rightarrow& \dfrac{3x}{\sqrt{x^2+100^2}}=1\\
&\Leftrightarrow& 3x=\sqrt{x^2+100^2}\\
&\Leftrightarrow&  9x^2=x^2+100^2\\
&\Leftrightarrow&  8x^2=10\,000\\
&\Leftrightarrow&  x^2=1\,250\\
&\Leftrightarrow&  x=\sqrt{1250}.
\end{eqnarray*}
$\Rightarrow$ Quãng đường bơi ngắn nhất là
\[
AD=\sqrt{x^2+100^2}=\sqrt{1250+10\,000}=\sqrt{11\,250} \approx 106 \text{(m)}.
\]
}
\end{ex}

\begin{ex}%[2D1V3-6]
Một gia đình thiết kế chiếc cổng có dạng là một parabol $(P)$ có kích thước như hình vẽ, biết chiều cao cổng bằng chiều rộng của cổng và bằng $4$ m. Người ta thiết kế cửa đi là một hình chữ nhật $CDEF$ sao cho chiều cao cửa đi là $CD=2$ m, phần còn lại dùng để trang trí. Biết chi phí phần tô đậm là $1{,}5$ triệu đồng/m$^2$. Tính số tiền (triệu đồng) gia đình đó phải trả để trang trí phần tô đậm (làm tròn kết quả đến hàng phần mười).
\begin{center}
\begin{tikzpicture}[scale=0.8,>=stealth, font=\footnotesize, line join=round, line cap=round]
\draw[smooth,samples=300,domain=-2:2] plot(\x,{-(\x)^2+4});
\draw (-2,0)--(2,0) (-1.4,0)node[below]{$F$}--(-1.4,2.04)node[above left]{$E$}--(1.4,2.04)node[above right]{$D$}--(1.4,0)node[below]{$C$} (1,3.7)node[]{$(P)$} (2,2)node[right]{$4$ m} (0,-0.8)node[below]{$4$ m};
\fill[gray,opacity=0.7] plot[domain=-2:-1.4](\x,{-(\x)^2+4})--(-1.4,0)--cycle plot[domain=1.4:2](\x,{-(\x)^2+4})--(1.4,0)--cycle plot[domain=-1.4:1.4](\x,{-(\x)^2+4})--(1.4,2.04)--(-1.4,2.04)--cycle;
\draw[<->] (2,0)--(2,4);
\draw[<->] (-2,-0.8)--(2,-0.8);
\end{tikzpicture}
\end{center}
\shortans{$7{,}5$}
\loigiai{
\begin{center}
\begin{tikzpicture}[scale=0.8,>=stealth, font=\footnotesize, line join=round, line cap=round]
\draw[smooth,samples=300,domain=-2:2] plot(\x,{-(\x)^2+4});
\draw (-2,0)--(2,0) (-1.4,0)node[below]{$F$}--(-1.4,2.04)node[above left]{$E$}--(1.4,2.04)node[above right]{$D$}--(1.4,0)node[below]{$C$} (1,3.7)node[]{$(P)$} (2,2)node[right]{$4$ m} (0,-0.8)node[below]{$4$ m};
\fill[gray,opacity=0.7] plot[domain=-2:-1.4](\x,{-(\x)^2+4})--(-1.4,0)--cycle plot[domain=1.4:2](\x,{-(\x)^2+4})--(1.4,0)--cycle plot[domain=-1.4:1.4](\x,{-(\x)^2+4})--(1.4,2.04)--(-1.4,2.04)--cycle;
\draw[->] (-3,0)--(3,0)node[above]{$x$} ;
\draw[->] (0,-1)--(0,5)node[left]{$y$} ;
\draw[<->] (2,0)--(2,4);
\draw[<->] (-2,-0.8)--(2,-0.8);
\foreach \x/\goc in {-2/-110,2/-110}{
\draw[fill=black] (\x,0)circle(1.2pt) node[shift={(\goc:2.8mm)},scale=.8]{$\x$};
}
\foreach \y/\goc in {4/135}{
\draw[fill=black] (0,\y)circle(1.2pt)node[shift={(\goc:2.8mm)},scale=.8]{$\y$};
}
\end{tikzpicture}
\end{center}
Giả sử parabol $(P)$ có phương trình là $y=ax^2+bx+c$ ($a \neq 0$). $(P)$ đi qua ba điểm $(0;4)$, $(-2;0)$ và $(2;0)$. \\
Khi đó, ta có
$\heva{
& 4=a \cdot 0^2+b \cdot 0+c \\
& 0=a \cdot 2^2+b \cdot 2+c \\
& 0=a \cdot (-2)^2+b \cdot (-2)+c
}
\Leftrightarrow
\heva{
& a=-1 \\
& b=0 \\
& c=4.
}$\\
Vậy $(P)\colon y=-x^2+4$.\\
Điểm $D$ và $E$ thuộc đồ thị của tiếp tuyến đường thẳng có phương trình $-x^2+4=2 \Leftrightarrow x=\pm \sqrt{2}$.\\ Theo đó thì, $D(\sqrt{2};2)$ và $E(-\sqrt{2};2)$.\\
Chiều dài cạnh của $DE$ là $2\sqrt{2}$ (m).\\
Diện tích của $S_{CDEF}$ là $2 \cdot 2\sqrt{2}=4\sqrt{2}$ (m$^2$).\\
Diện tích phần đồ thị $(P)$ tạo với trục hoành là $S=\displaystyle\int\limits_{-2}^2 (-x^2+4) \, \mathrm{\,d}x=\dfrac{32}{3}$ (m$^2$).\\
Diện tích cần trang trí là $S_1=S-S_{CDEF}=\dfrac{32}{3}-4\sqrt{2}=\dfrac{32-12\sqrt{2}}{3}$ (m$^2$).\\
Chi phí để trang trí là $\dfrac{32-12\sqrt{2}}{3} \cdot 1{,}5 \approx 7{,}5$ (triệu đồng).
}
\end{ex}

\begin{ex}%[2D1V3-6]%[TEX ĐỀ MOON 2025]%[Huỳnh Thanh Chí]
Có hai xã $A$, $B$ cùng ở một bên phía bờ sông. Khoảng cách từ hai xã đó đến bờ sông lần lượt là $AA'=500$ m, $BB'=600$. Người ta đo được $A'B'=2200$ m như hình vẽ bên. Các kỹ sư muốn xây dựng một trạm cung cấp nước sạch bên bờ sông cho người dân của hai xã đã sử dụng. Để tiết kiệm chi phí, các kỹ sư phải chọn một vị trí $M$ của trạm cung cấp nước sạch đó trên đoạn $A'B'$ sao cho tổng khoảng cách từ hai xã đến vị trị $M$ là nhỏ nhất.
\begin{center}
\begin{tikzpicture}[scale=0.48,>=stealth, font=\footnotesize, line join=round, line cap=round]
\coordinate (A') at (0,0);
\coordinate (B') at (8,0);
\coordinate (M) at (3.5,0);
\coordinate (A) at (0,3);
\coordinate (B) at (8,6);
\draw (A)--(A')--(B')--(B);
\draw ($(A)!0.5!(A')$)node[left]{$500$ m} ($(B)!0.5!(B')$)node[right]{$600$ m} (A)--(M)node[below]{$M$}--(B) (4,-1.5)node[below]{$2\,200$ m};
\foreach \x/\g in {A/135,B/90,A'/-135,B'/-45}
\fill[black] (\x) circle (3pt) ($(\g:8mm)+(\x)$) node {$\x$};
\draw[<->, dashed] (0,-1.5)--(8,-1.5);
\end{tikzpicture}
\end{center}
Giá trị nhỏ nhất của tổng khoảng cách đó bằng bao nhiêu mét? (Làm tròn kết quả đến hàng đơn vị).

\shortans[]{$2460$}
\loigiai{
Đặt $AA'=5$, $BB'=6$, $A'B'=22$.\\
Đặt $A'M=x$, khi đó $MB'=22-x$ với $0<x<22$.\\
Ta có $AM=\sqrt{AA'^2+A'M^2}=\sqrt{25+x^2}$ và $MB=\sqrt{BB'^2+MB'^2}=\sqrt{x^2-44x+520}$.\\
Khi đó, yêu cầu bài toán là tìm $T=AM+MB$ đạt giá trị nhỏ nhất.\\
Đặt $f(x)=\sqrt{25+x^2}+\sqrt{x^2-44x+520}$ với $0<x<22$.\\
Ta có $f'(x) = \dfrac{x}{\sqrt{25+x^2}} + \dfrac{x-22}{\sqrt{x^2-44x+520}}$. \\
Xét phương trình
\allowdisplaybreaks
\begin{eqnarray*}
&&f'(x) = 0 \\
&\Leftrightarrow& \dfrac{x}{\sqrt{25+x^2}} = \dfrac{22-x}{\sqrt{x^2-44x+520}} \\
&\Leftrightarrow& \dfrac{x^2}{25+x^2} = \dfrac{(22-x)^2}{x^2-44x+520} \\
&\Leftrightarrow& x^2(x^2-44x+520) = (25+x^2)(x^2-44x+484) \\
&\Leftrightarrow& x^4 - 44x^3 + 520x^2 = 25x^2 - 1100x + 12100 + x^4 - 44x^3 + 484x^2\\
&\Leftrightarrow& -11x^2 - 1100x + 12100 = 0\\
&\Leftrightarrow& \hoac{& x=10\\ & x= -110 \,(\text{loại}).}
\end{eqnarray*}
Bảng biến thiên
\begin{center}
\begin{tikzpicture}
\tkzTabInit[nocadre]
{$x$/1,$f'(x) $/1,$f(x)$/3}
{$0$,$10$,$22$}
\tkzTabLine{,-,0,+,} %
\tkzTabVar{+/,-/$11\sqrt{5}$, +/} %dấu mũi tên, + trên, -dưới
\end{tikzpicture}
\end{center}
Dựa vào bảng biến thiên thì $\min f(x)=f(10)=11\sqrt{5}$.\\
Vậy giá trị nhỏ nhất của tổng khoảng cách đó bằng $100\cdot 11\sqrt{5}\approx 2460$ m.
}
\end{ex}

\begin{ex}%[2D1V3-6]
Một chiếc phà chạy giữa đất liền và đảo Dedlos. Phà có công suất tối đa là $1\,000$ xe hơi mỗi chuyến, nhưng việc tải gần hết công suất rất tốn thời gian. Biết rằng số lượng xe hơi đưa lên phà mỗi chuyến là $f(t)=\dfrac{2000t}{2t+1}$ và mất một khoảng thời gian là $1$ giờ. Mỗi xe cần trung bình $3{,}6$ giây để dỡ xuống khi đến điểm đích. Thời gian di chuyển đến đảo và thời gian vòng về đều mất $1{,}28$ giờ. Nên tải bao nhiêu xe lên phà cho mỗi chuyến đi để lượng xe trung bình di chuyển qua lại đảo mỗi giờ đạt lớn nhất? (làm tròn kết quả đến hàng đơn vị).
\shortans{615}
\loigiai{
Để đưa được $f(t)$ xe lên phà cần $t$ giờ.\\
Tổng thời gian đưa xe qua đảo hoặc từ đảo về là $t+\dfrac{3{,}6}{3\,600}f(t)+1{,}28$ giờ.\\
Số xe di chuyển trung bình mỗi giờ là $g(t)$, với
\[g(t)=\dfrac{f(t)}{t+\dfrac{3{,}6}{3\,600}f(t)+1{,}28}=\dfrac{\dfrac{2000t}{2t+1}}{t+\dfrac{3{,}6}{3\,600}\cdot\dfrac{2000t}{2t+1}+1{,}28}=\dfrac{2\,000t}{2t^2+5{,}56t+1{,}28}=\dfrac{2\,000}{2t+\dfrac{1{,}28}{t}+5{,}56}.\]
Áp dụng bất đẳng thức Côsi cho $2$ số dương $2t$ và $\dfrac{1{,}28}{t}$, ta có $2t + \dfrac{1{,}28}{t} \ge 2\sqrt{2t\cdot\dfrac{1{,}28}{t}}$.\\
Suy ra $\dfrac{2\,000}{2t + \dfrac{1{,}28}{t} + 5{,}56} \le \dfrac{2\,000}{2\sqrt{2t\cdot\dfrac{1{,}28}{t}} + 5{,}56} = \dfrac{50\,000}{219}$.\\
Dấu bằng xảy ra khi $2t = \dfrac{1{,}28}{t} \Leftrightarrow t^2 = 0{,}64 \Leftrightarrow t = 0{,}8$.\\
Vậy để lượng xe trung bình di chuyển qua lại đảo mỗi giờ đạt lớn nhất cần tải lên phà mỗi chuyến $ f(0,8) \approx 615$ xe.
}
\end{ex}

\begin{ex}%[2D1V3-6]%[TEX ĐỀ MOON 2025]%[Nguyễn Thế Duy]
Một khách sạn có $50$ phòng. Hiện tại mỗi phòng cho thuê với giá $400$ nghìn đồng một ngày thì toàn bộ phòng được thuê hết. Biết rằng cứ mỗi lần tăng giá thêm $20$ nghìn đồng một phòng thì có thêm $2$ phòng trống. Giám đốc phải chọn giá phòng mới là bao nhiêu để thu nhập của khách sạn trong ngày là lớn nhất?
\shortans{$450$}
\loigiai{
Gọi $x$ là số phòng trống của khách sạn $\left(x \in \mathbb{N} \right)$.\\
Khách sạn có $x$ phòng trống khi tăng giá thuê $1$ phòng thêm $\dfrac{20}{2} \cdot x = 10x$ nghìn đồng.\\
Thu nhập của khách sạn là $T(x) = \left(50 - x \right) \left(400 + 10x \right) = -10x^2 + 100x + 200\,000$.\\
Dễ thấy $T(x)$ đạt giá trị lớn nhất khi $x = \dfrac{-100}{2 \cdot (-10)} = 5$.\\
Khi đó giá cho thuê mỗi phòng là $450$ nghìn đồng.
}
\end{ex}

\begin{ex}%[2D1V3-6]%[TEX ĐỀ MOON 2025]%[Nguyễn Thế Duy]
Một nhà sản xuất muốn thiết kế một chiếc hộp có dạng hình hộp chữ nhật không có nắp, có đáy là hình vuông và diện tích bề mặt bằng $108$ cm$^2$. Tìm tích của các kích thước của chiếc hộp sao cho thể tích của hộp là lớn nhất?
\shortans{$108$}
\loigiai{
\immini{
Gọi $x$ là chiều dài cạnh hình vuông đáy $\left(x > 0 \right)$.\\
Gọi $y$ là chiều cao của chiếc hộp.\\
Theo bài ta có, diện tích bề mặt của chiếc hộp \\
$108 = x^2 + 4xy \Rightarrow y = \dfrac{108 - x^2}{4x}$.\\
Khi đó, thể tích của chiếc hộp là $V(x) = x^2 \cdot y = 27x - \dfrac{x^3}{4}$.\\
Ta cần tìm $\underset{\left(0;+108 \right)}{\max}V(x)$.\\
Ta có $V'(x) = 27 - \dfrac{3x^2}{4}$. \\
Xét $V'(x) = 0 \Leftrightarrow \hoac{&x = 6 \quad (\text{thoả mãn})\\&x = -6 \quad (\text{không thoả mãn}).}$
}
{\begin{tikzpicture}[scale=0.9,>=stealth, font=\footnotesize, line join=round, line cap=round]
\path
(0,0) coordinate (A)
(1.5,1.5) coordinate (B)
(5.5,1.5) coordinate (C)
(4,0) coordinate (D)
;
\foreach \x in {A,B,C,D}{
\path
($(\x)+(0,3)$) coordinate (\x');
}
\draw
(A) -- (D) -- (C) -- (C') -- (B') -- (A') -- (D') -- (C')
(A) -- (A') node[pos=0.5,left]{$y$}
(D) -- (D')
;
\draw[->] (3.5,5) node[above]{không có nắp} arc(160:200:2.2)
;
\path
(A) -- (D) node[pos=0.5,below]{$x$}
(C) -- (D) node[pos=0.5,right]{$x$}
;
\draw[dashed]
(A) -- (B) -- (C)
(B) -- (B')
;

\end{tikzpicture}
}
\noindent Ta có bảng biến thiên
\begin{center}
\begin{tikzpicture}
\tkzTabInit[nocadre=false,lgt=1.2,espcl=2.7,deltacl=0.6]
{$x$ /0.6, $y'$ /0.6, $y$ /2}
{$0$,$6$,$108$}
\tkzTabLine{,+,$0$,-,}
\tkzTabVar{-/$0$,+/$108$,-/$0$}
\end{tikzpicture}
\end{center}
Suy ra thể tích của chiếc hộp lớn nhất khi $x = 6$ khi đó $y = 3$.\\
Tích các kích thước của hộp là $3 \cdot 3 \cdot 4 = 108$.
}
\end{ex}

\begin{ex}%[2D1V3-6]%[TexDeMoon2025]%[NguyenKieuNhaTu]
Một cơ sở sản xuất khăn mặt đang bán mỗi chiếc khăn với giá $30000$ đồng một chiếc và mỗi tháng cơ sở bán được trung bình $3000$ chiếc khăn. Cơ sở sản xuất đang có kế hoạch tăng giá bán để có lợi nhận tốt hơn. Sau khi tham khảo thị trường, người quản lý thấy rằng nếu từ mức giá $30000$ đồng mà cứ tăng giá thêm $1000$ đồng thì mỗi tháng sẽ bán ít hơn $100$ chiếc. Biết vốn sản xuất một chiếc khăn không thay đổi là $18000$. Để đạt lợi nhuận lớn nhất thì mỗi chiếc khăn cần bán với giá bao nhiêu nghìn đồng?
\shortans[]{$39$}
\loigiai{
Gọi $x$ là số lần tăng giá 1.000 đồng, khi đó:

\begin{itemize}
\item Giá bán mỗi khăn sẽ là: $30+x$ (nghìn đồng).
\item Số khăn bán ra mỗi tháng là: $3000-100x$ (chiếc).
\item Lợi nhuận trên mỗi khăn là: $(30+x)-18=12+x$ (nghìn đồng).
\end{itemize}
Tổng lợi nhuận mỗi tháng là: $L(x)=(3000-100x)(12+x)$.\\
Khai triển biểu thức:
\begin{eqnarray*}
L(x) &=&3000\cdot (12+x)-100x(12+x) \\
&=&36000+3000x-1200x-100x^2\\
&=&-100x^2+1800x+36000
\end{eqnarray*}
Đây là hàm bậc hai có hệ số $a=-100< 0$, nên đạt cực đại tại: $x=\dfrac{-b}{2a}=\dfrac{-1800}{2\cdot (-100)}=9$.\\
Vậy để đạt lợi nhuận lớn nhất, mỗi chiếc khăn cần bán với giá $30+9=39$ (nghìn đồng).
}
\end{ex}

\begin{ex}%[2D1V3-2]%[TEX Đề Moon 2025]%[Vũ Hồng Toàn]
Anh Vinh đang cắm trại dưới tán cây thông ở điểm $X$ cách điểm $A$ một khoảng $3$ km. Điểm $A$ nằm trên đường bờ biển (đường bờ biển là đường thẳng). Ô tô của anh Vinh đỗ ở vị trí $Y$ cách điểm $B$ một khoảng $3$ km. Điểm $B$ cũng thuộc đường bờ biển. Biết rằng $AB=18$ km, $AM=NB=x$ km và $AX=BY=3$ km (minh hoạ như hình vẽ).
\begin{center}
\begin{tikzpicture}[scale=1,>=stealth, font=\footnotesize, line join=round, line cap=round]
\coordinate (A) at (0,0);
\coordinate (M) at (2,0);
\coordinate (N) at (5,0);
\coordinate (B) at (7,0);
\coordinate (X) at (0,3);
\coordinate (Y) at (7,3);
\draw (X)--(A)node[below]{$A$} (Y)--(B)node[below]{$B$} ($(A)!1.3!(B)$)--($(B)!1.3!(A)$);
\draw[dashed] (X)node[above]{$X$}--(M)node[below]{$M$} (Y)node[above]{$Y$}--(N)node[below]{$N$} (0,1.5)node[left]{$3$ km} (7,1.5)node[right]{$3$ km} ($(M)!0.5!(N)$)node[below]{Bờ biển} (3.5,-0.8)node[below]{$18$ km};
\draw[<->] (0,-0.8)--(7,-0.8);
\fill (A)circle(2pt);
\fill (B)circle(2pt);
\fill (M)circle(2pt);
\fill (N)circle(2pt);
\fill (X)circle(2pt);
\fill (Y)circle(2pt);
\draw pic[draw,angle radius=4mm]{right angle=M--A--X};
\draw pic[draw,angle radius=4mm]{right angle=Y--B--N};
\end{tikzpicture}
\end{center}
Khi đang dựng trại tại vị trí $X$, anh Vinh không may bị rắn cắn, chất độc lan vào máu. Sau khi bị rắn cắn, nồng độ chất độc trong máu tăng theo thời gian được tính theo phương trình $y=50\log(t+2)$. Trong đó, $y$ là nồng độ, $t$ là thời gian tính bằng giờ sau khi bị rắn cắn. Anh Vinh cần quay trở lại ô tô ở vị trí $Y$ để lấy thuốc giải độc. Anh chạy từ chỗ cây thông ở điểm $X$ ra thẳng vị trí $M$  với vận tốc là $5$\,km/h và chạy trên bãi biển từ $M$ tới điểm $N$ với vận tốc là $13$\,km/h sau đó chạy thẳng đến chỗ ô tô với vận tốc $5$\,km/h. Tính nồng độ chất độc trong máu thấp nhất khi anh Vinh về đến ô tô (kết quả làm tròn đến hàng phần chục).
\shortans{$32{,}6$}
\loigiai{
Khoảng cách $XM$ là $XM=\sqrt{A X^2+A M^2}=\sqrt{3^2+x^2}=\sqrt{9+x^2}$.\\
Khoảng cách $MN$ là $MN=18-2x$.\\
Khoảng cách $NY$ là $NY=\sqrt{N B^2+B Y^2}=\sqrt{x^2+3^2}=\sqrt{x^2+9}$.\\
Tổng thời gian $f(x)=\dfrac{\sqrt{9+x^2}}{5}+\dfrac{18-2 x}{13}+\dfrac{\sqrt{x^2+9}}{5}=\dfrac{2\sqrt{x^2+9}}{5}+\dfrac{18-2 x}{13}$.\\
Ta có $f'(x)=\dfrac{2x}{5\sqrt{x^2+9}}-\dfrac{2}{13}=\dfrac{26x-10\sqrt{x^2+9}}{65\sqrt{x^2+9}}$.
\allowdisplaybreaks
\begin{eqnarray*}
&&f'(x)=0\\
&\Rightarrow& 26x-10\sqrt{x^2+9}=0\\
&\Rightarrow& 5\sqrt{x^2+9}=13x\\
&\Rightarrow& 25(x^2+9)=169x^2\\
&\Rightarrow& 144x^2-225=0\Rightarrow\hoac{&x=\dfrac{5}{4}&\text{(nhận)}\\&x=-\dfrac{5}{4}&\text{(loại)}.}
\end{eqnarray*}
Bảng biến thiên\\
\centerline{
\begin{tikzpicture}
\tkzTabInit
{$x$/0.7,$f'(x)$/0.7,$f(x)$/2.1}
{$0$,$\tfrac{5}{4}$,$+\infty$}
\tkzTabLine{,-,$0$,+}
\tkzTabVar{+/,-/$\dfrac{162}{65}$,+/}
\end{tikzpicture}
}
Nồng độ chất độc trong máu thấp nhất khi thời gian về đến ô tô là nhỏ nhất là $t=\dfrac{162}{65}$.\\
Nồng độ chất độc
\[
y=50 \log (t+2)=50 \log\left (\dfrac{162}{65}+2\right)\approx 32{,}6.
\]
}
\end{ex}

\begin{ex}%[2D1V3-6]%[TEX Đề Moon 2025]%[Võ Nguyên Thạch]
Nhà máy $A$ chuyên sản xuất một loại sản phẩm cung cấp cho nhà máy $B$. Hai nhà máy thoả thuận rằng, hàng tháng nhà máy $A$ cung cấp cho nhà máy $B$ số lượng sản phẩm theo đơn đặt hàng của $B$ (tối đa $100$ tấn sản phẩm). Nếu số lượng đặt hàng là $x$ tấn sản phẩm thì giá bán cho mỗi tấn sản phẩm là $P(x)=45-0{,}001x^2$ (triệu đồng). Chi phí để $A$ sản xuất $x$ tấn sản phẩm trong một tháng gồm $100$ triệu đồng chi phí cố định và $30$ triệu đồng cho mỗi tấn sản phẩm. Nhà máy $A$ cần bán cho nhà máy $B$ bao nhiêu tấn sản phẩm mỗi tháng để lợi nhuận thu được lớn nhất? (Làm tròn kết quả đến hàng phần mười).
\shortans{70{,}7}
\loigiai{
Doanh thu của nhà máy khi sản xuất $1$ tấn sản phẩm là $P(x)$ triệu đồng.\\
Doanh thu của nhà máy khi sản xuất $x$ tấn sản phẩm là $xP(x)$ triệu đồng.\\
Chi phí của nhà máy khi sản xuất $x$ tấn sản phẩm là $C(x)$ triệu đồng.\\
Vì Lợi nhuận = Doanh thu – Chi phí nên ta có lợi nhuận của nhà máy $A$ khi sản xuất $x$ tấn sản phẩm là
\[H(x)=xP(x)-C(x)=x(45-0{,}001x^2)-(100+30x)=-0{,}001x^2+15x-100, \text{ với } 0\le x\le 100.\]
Ta có $H'(x)=-0{,}003x^2+15=0\Leftrightarrow \hoac{&x=50\sqrt 2\\&x=-50\sqrt 2.}$\\
Chỉ có $x=50\sqrt 2$ thỏa điều kiện.\\
Ta có $H(0)=-100$; $H(50\sqrt 2)=500\sqrt 2$; $H(100)=400$.\\
Vậy lợi nhuận lớn nhất khi $A$ sản xuất $50\sqrt 2\approx 70{,}7$ tấn sản phẩm.
}
\end{ex}

% \paragraph{Mức độ C}
\begin{ex}%[1D6C4-6]%[TEX ĐỀ MOON 2025]%[Nguyễn Thế Duy]
Bác An vay ngân hàng $900$ triệu đồng theo hình thức lãi kép và trả góp hàng tháng. Cuối mỗi tháng bắt đầu từ tháng thứ nhất Bác An trả $12$ triệu đồng và chịu lãi suất $0{,}95\%$ trên tháng cho số tiền chưa trả. Với hình thức hoàn nợ như vậy thì sau bao nhiêu tháng Bác An sẽ trả hết số nợ ngân hàng, biết rằng lãi suất không đổi trong suốt quá trình vay.
\shortans{$132$}
\loigiai{
Số tiền bác An phải trả sau $n$ tháng là $S_n = 900 \left(1 + \dfrac{0{,}95}{100} \right)^n$.\\
Số tiền bác An đã trả tính đến tháng thứ $n$ là $P_n = 12 \cdot \dfrac{\left(1+0{,}0095 \right)^n - 1}{0{,}0095}$.\\
Bác An trả xong nợ khi $S_n = P_n$ hay
\begin{align*}
900 \left(1 + 0{,}0095 \right)^n = 12 \cdot \dfrac{\left(1+0{,}0095 \right)^n - 1}{0{,}0095} &\Leftrightarrow 3{,}45 \cdot 1{,}0095^n = 12\\
&\Leftrightarrow n = \log_{1{,}0095}{\dfrac{12}{3{,45}}} \approx 131{,}83.
\end{align*}
Vậy với hình thức trên bác An sẽ trả hết nợ sau $132$ tháng.
}
\end{ex}

\begin{ex}%[2H5C2-6]%[Tex đề Moon 2025]%[Nguyễn Hồng Thạch]
Trong một phần mềm 3D mô phỏng một trò chơi điện tử, có hai chất điểm $A$, $B$ luôn chuyển động trên một mặt cầu $(S)$ và cách nhau một khoảng không đổi bằng $1$. Nếu đặt trong không gian tọa độ $Oxyz$, mặt cầu $(S)$ có phương trình là $(x-3)^2+(y+4)^2+z^2=4$. Tìm giá trị nhỏ nhất của biểu thức $OA^2-OB^2$?
\shortans{$-10$}
\loigiai{
Mặt cầu $(S)$ có tâm $I(3; -4; 0)$ và bán kính $R = 2$.\\
Ta có $\heva{&OA^2 = x_A^2 + y_A^2 + z_A^2\\
&OB^2 = x_B^2 + y_B^2 + z_B^2.}$\\
Mặt khác
$\heva{&IA^2 = (x_A - 3)^2 + (y_A + 4)^2 + (z_A - 0)^2 = x_A^2 - 6x_A + 9 + y_A^2 + 8y_A + 16 + z_A^2= 4\\
&IB^2 = (x_B - 3)^2 + (y_B + 4)^2 + (z_B - 0)^2 = x_B^2 - 6x_B + 9 + y_B^2 + 8y_B + 16 + z_B^2 = 4.}$\\
Ta có
\[OA^2 = IA^2 - 9 - 16 + 6x_A - 8y_A = 4 - 25 + 6x_A - 8y_A = 6x_A - 8y_A - 21.\]
\[OB^2 = IB^2 - 9 - 16 + 6x_B - 8y_B = 4 - 25 + 6x_B - 8y_B = 6x_B - 8y_B - 21.\]
Suy ra \[OA^2 - OB^2 = (6x_A - 8y_A - 21) - (6x_B - 8y_B - 21) = 6(x_A - x_B) - 8(y_A - y_B).\]
Ta có $\overrightarrow{AB} = (x_B - x_A; y_B - y_A; z_B - z_A)$, $\overrightarrow{OI} = (3; -4; 0)$.\\
Suy ra $\overrightarrow{OI} \cdot \overrightarrow{AB} = 3\cdot(x_B - x_A) - 4\cdot(y_B - y_A) + 0\cdot(z_B - z_A) = 3\cdot(x_B - x_A) - 4\cdot(y_B - y_A)$.\\
Khi đó $OA^2 - OB^2 = 2\left[3\cdot(x_A - x_B) - 4\cdot(y_A - y_B)\right] =-2(\overrightarrow{OI} \cdot \overrightarrow{AB})$.\\
Mặt khác $\left|\overrightarrow{OI} \cdot \overrightarrow{AB}\right| \le \left|\overrightarrow{OI}\right| \cdot \left|\overrightarrow{AB}\right|$.\\
Với $\left|\overrightarrow{OI}\right| = \sqrt{3^2 + (-4)^2 + 0^2} = 5$ và $\left|\overrightarrow{AB}\right| = 1$.\\
Suy ra $|\overrightarrow{OI} \cdot \overrightarrow{AB}| \le 5 \cdot 1 = 5$\\
Hay $-5 \le \overrightarrow{OI} \cdot \overrightarrow{AB} \le 5$.\\
Khi đó $-2\cdot(5) \le OA^2 - OB^2 \le -2\cdot(-5)$ hay $-10 \le OA^2 - OB^2 \le 10$.\\
Vậy giá trị nhỏ nhất của $OA^2 - OB^2$ là $-10$.
}
\end{ex}

\begin{ex}%[2H5C2-3]
Trong không gian $Oxyz$, cho tam giác $ABC$, đường phân giác $AM$ với $M\in BC$, $M(2;0;4)$. Biết điểm $B$ thuộc đường thẳng $\dfrac{x}{1}=\dfrac{y}{1}=\dfrac{z}{1}$, điểm $C$ thuộc mặt phẳng $2x+y-z-2=0$ và $AB=2AC$. Đường thẳng $BC$ có một vectơ chỉ phương là $(a;1;b)$. Tính $a+b$.
\shortans{$5$}
\loigiai{ Gọi  $B(2 t ; 2 t ; 2 t) \in d\colon \dfrac{x}{1}=\dfrac{y}{1}=\dfrac{z}{1}$, $
C\left(x_C ; y_C ; z_C\right) \in(P)\colon 2 x+y-z-2=0$.\\ Ta có $\overrightarrow{B M}=(2-2 t ;-2 t ; 4-2 t)$, $\overrightarrow{M C}=\left(x_C-2 ; y_C ; z_C-4\right)$.\\
Tam giác $ABC$ có đường phân giác trong $AM$ với
\begin{eqnarray*}
M \in B C &\Rightarrow& \dfrac{M C}{M B}=\dfrac{A C}{A B}=\dfrac{1}{2} \\
&\Rightarrow& \overrightarrow{M C}=\dfrac{1}{2} \overrightarrow{B M}\\
&\Rightarrow& \heva{&x_C-2=\dfrac{1}{2}(2-2 t)=1-t \\
&y_C=\dfrac{1}{2}(-2 t)=-t \\
&z_c-4=\dfrac{1}{2}(4-2 t)=2-t
} \\
&\Rightarrow&\heva{
&x_C=3-t \\
&y_C=-t \\
&z_c=6-t
}\\
&\Rightarrow& C(3-t ;-t ; 6-t) .
\end{eqnarray*}
Vì $C\in (P) \Rightarrow 2(3-t)-t-(6-t)-2=0 \Rightarrow t=-1$. \\Khi đó $B(-2 ;-2 ;-2)$, $C(4 ; 1 ; 7) \Rightarrow \overrightarrow{B C}(6 ; 3 ; 9)$.\\
Suy ra $\overrightarrow{n_{B C}}=(2 ; 1 ; 3) \Rightarrow a=2,b=3\Rightarrow a+b=5$.
}
\end{ex}

\begin{ex}%[2D6C2-3]
Một công nhân đi làm ở thành phố khi trở về nhà chỉ có $2$ cách hoặc đi theo đường ngầm hoặc đi qua cầu. Nếu đi lối đường ngầm $75\%$ trường hợp ông ta về đến nhà trước $6$ giờ tối; còn nếu đi lối cầu chỉ có $70\%$ trường hợp (nhưng đi lối cầu thích hơn). Vợ ông ta nhận thấy rằng: Bình quân cứ $100$ lần về nhà thì $71$ lần ông ta về nhà trước $6$ giờ tối. Tìm xác suất để công nhân đó đã đi lối cầu biết rằng ông ta về đến nhà sau $6$ giờ tối (kết quả làm tròn đến hàng phần trăm).
\shortans[]{$0{,}83$}
\loigiai{
\begin{center}
\begin{tikzpicture}[
font=\small,
node distance=4cm and 3.5cm,
>=Stealth,
every node/.style={align=center},
state/.style={draw, minimum height=1.2cm, minimum width=2.5cm, rounded corners},
box/.style={draw, minimum height=1cm, minimum width=2.5cm, rounded corners},
good/.style={box, fill=red!10, draw=red},
mid/.style={box, fill=blue!15},
mid2/.style={box, fill=green!15}
]

% Nodes
\node[state, fill=purple!10] (start) {Gốc};
\node[mid, right=of start, yshift=1.5cm] (underground) {Ông ta đi đường ngầm};
\node[mid2, right=of start, yshift=-1.5cm] (bridge) {Ông ta đi qua cầu};

\node[good, right=of underground, yshift=0.8cm] (early1) {Về nhà trước 6 giờ tối};
\node[box, right=of underground, yshift=-0.8cm] (late1) {Về nhà sau 6 giờ tối};

\node[good, right=of bridge, yshift=0.8cm] (early2) {Về nhà trước 6 giờ tối};
\node[box, right=of bridge, yshift=-0.8cm] (late2) {Về nhà sau 6 giờ tối};

% Edges from Gốc
\draw[->] (start) -- node[above left] {$x$} node[below left] {$B$} (underground);
\draw[->] (start) -- node[below left] {$\overline{B}$} node[above left] {$1-x$} (bridge);

% Edges from underground
\draw[->] (underground) -- node[above] {$A|B$} node[below] {0.75} (early1);
\draw[->] (underground) -- node[below] {$\overline{A}|B$} (late1);

% Edges from bridge
\draw[->] (bridge) -- node[above] {$A|\overline{B}$} node[below] {0.7} (early2);
\draw[->] (bridge) -- node[below] {$\overline{A}|\overline{B}$} (late2);

\end{tikzpicture}
\end{center}
Gọi $B$ là biến cố ông ta đi đường ngầm.\\
Gọi $A$ là biến cố về nhà trước $6$ giờ tối.\\
Ta có dữ kiện: Bình quân cứ $100$ lần về nhà thì $71$ lần ông ta về nhà trước $6$ giờ tối nên\\
$P(A)=0{,}71$.
$\Leftrightarrow 0{,}75x +0{,}7(1-x)=0{,}71\Leftrightarrow x=0{,}2$.\\
Suy ra $P(B)=0{,}2$; $P(\overline{B})=0{,}8$.\\
Do đó $P(\overline{B}|\overline{A})=\dfrac{P(\overline{B})\cdot P(\overline{A}|\overline{B})}{P(\overline{A})}=\dfrac{1-0{,7}\cdot0{,8}}{1-0{,}71}\approx 0{,}83$.
}

\end{ex}

\begin{ex}%[2D4C3-5]
\immini[thm]
{
Một người thợ gốm sứ muốn thiết kế một cái bình hoa bằng cách quay hình $(H)$ (phần gạch chéo trong hình vẽ bên) quanh trục $AB$. Hình phẳng $(H)$ nằm trong hình chữ nhật $ABCD$, giới hạn bởi các đoạn thẳng $AM$, $BP$ ($M$, $P$ là hai điểm lần lượt thuộc các cạnh $AD$, $BC$, $MP\parallel CD$), cung tròn $MN$ (có tâm là trung điểm của đoạn thẳng $AE$) và cung parabol $NP$. Biết $AB=5$ dm, $AM=BE=1$ dm. Tiếp tuyến của cung tròn và cung parabol tại điểm $N$ là trùng nhau. Bình hoa đó có thể tích bằng bao nhiêu lít? Kết quả làm tròn đến hàng phần mười.
}
{
\begin{tikzpicture}[scale=1,>=stealth, font=\footnotesize, line join=round, line cap=round,rotate=90]
\fill[pattern=north east lines,opacity=0.6] plot[domain=0:4](\x,{sqrt(5-(\x-2)^2)})--plot[domain=4:5](\x,{2*(\x)^2-18*(\x)+41})--(5,0)--(0,0)--cycle;
\draw (0,0)node[below right]{$A$}--(0,2.5)node[below left]{$D$}--(5,2.5)node[above left]{$C$}--(5,0)node[above right]{$B$}--cycle;
\draw[dashed] (0,1)node[below]{$M$}--(5,1)node[above]{$P$} (4,1)node[above left]{$N$}--(4,0)node[right]{$E$};
\draw[smooth,samples=300,domain=0:4] plot(\x,{sqrt(5-(\x-2)^2)});
\draw[smooth,samples=300,domain=4:5] plot(\x,{2*(\x)^2-18*(\x)+41});
\end{tikzpicture}
}
\shortans{$47{,}5$}
\loigiai{
\begin{center}
\begin{tikzpicture}[scale=1.2, >=stealth, font=\footnotesize, line join=round, line cap=round]
\coordinate (A) at (0,0);
\coordinate (B) at (5,0);
\coordinate (C) at (5,2.5);
\coordinate (D) at (0,2.5);
\coordinate (M) at (0,1);
\coordinate (P) at (5,1);
\coordinate (N) at (4,1);
\coordinate (E) at (4,0);
\fill[pattern=north east lines,opacity=0.6]
plot[domain=0:4] ({\x}, {sqrt(5-(\x-2)^2)})
-- plot[domain=4:5] ({\x}, {2*(\x)^2 - 18*(\x) + 41})
-- (5,0) -- (0,0) -- cycle;
\draw (A) node[below left]{$O$}
-- (D) node[above left]{$D$}
-- (C) node[above right]{$C$}
-- (B) node[below right]{$B$}
-- cycle;
\draw[dashed] (M) node[left]{$M$} -- (P) node[right]{$P$};
\draw[dashed] (N) node[above right]{$N$} -- (E) node[below]{$E$};
\draw[smooth,samples=200,domain=0:4] plot(\x,{sqrt(5-(\x-2)^2)});
\draw[smooth,samples=200,domain=4:5] plot(\x,{2*(\x)^2 - 18*(\x) + 41});
\draw[->] (A) -- ($(A)+(6,0)$) node[right]{$x$};
\draw[->] (A) -- ($(A)+(0,3)$) node[above]{$y$};
\node at (2,1.9) {$y = f(x)$};
\node at (4.6,0.4) {$y = g(x)$};

\end{tikzpicture}
\end{center}
Ta có $I M=\sqrt{O M^2+O I^2}=\sqrt{2^2+1^2}=\sqrt{5}\Rightarrow R=\sqrt{5}$.\\
Phương trình đường tròn $(C)$ tâm $I(2;0)$ bán kính $R=\sqrt{5}$ là
\[(x-2)^2+y^2=5\Leftrightarrow y^2=5-(x-2)^2.\]
Gọi $y=f(x)$ là hàm số của cung $MN$, khi đó \[f^2(x)=5-(x-2)^2 \Rightarrow f(x)=\sqrt{5-(x-2)^2}.\]
Thể tích khi quay hình phẳng giới hạn bởi $y=f(x)$, $y=0$, $x=0$, $x=4$ bằng
\[V_1=\pi \displaystyle\int\limits_0^4 f^2(x) \mathrm{d}x=\pi \displaystyle\int\limits_0^4\left(5-(x-2)^2\right) \mathrm{d}x=\dfrac{44 \pi}{3}\left(\mathrm{dm}^3\right).\]
Gọi $y=g(x)$ là hàm số cung parabol đi qua $N(4;1)$ và $P(5 ; 1)$. Khi đó ta có
\[g(x)-1=a(x-4)(x-5) \Leftrightarrow  g(x)=a(x-4)(x-5)+1\]
Tiếp tuyến của cung tròn và cung parabol tại điểm $N(4 ; 1)$ là trùng nhau nên
\[f'(4)=g'(4) \Leftrightarrow-a=-2\Leftrightarrow a=2.\]
Do đó $g(x)=2(x-4)(x-5)+1=2 x^2-18 x+41.$\\
Thể tích khi quay hình phẳng giới hạn bởi $y=g(x)$, $y=0$,
$x=4$, $x=5$ bằng
\[V_2=\pi \displaystyle\int\limits_4^5 g^2(x) \mathrm{d}x=\pi \displaystyle\int\limits_4^5\left(2 x^2-18 x+41\right)^2 \mathrm{d}x=\dfrac{7 \pi}{15}\left(\mathrm{dm}^3\right).\]
Thể tích của bình hoa bằng \[V=V_1+V_2=\dfrac{44 \pi}{3}+\dfrac{7 \pi}{15} =\dfrac{227 \pi}{15} \approx 47{,}5\left(\mathrm{d m}^3\right) =47{,}5(\mathrm{l}) .
\]
}
\end{ex}

\begin{ex}%[2D4C3-4]
Một chiếc dàn Ghita có chiều cao $5$ cm, khi cắt một mặt cắt ngang của cây đàn ta thu được một mặt phẳng như hình vẽ.
\begin{center}
\begin{tikzpicture}[line join=round, line cap=round,>=stealth,thick, xscale=.4, yscale=.2]
\tikzset{every node/.style={scale=0.9}}
\draw (0,0) node [below left] {$O$};
\begin{scope}[rotate=-10, yscale=1.4]
\clip (0,-15) rectangle (11,15);
\draw[fill =gray, line width = 1.2pt,draw=none] (0,0) plot[domain=0:10.08](\x,{1/24*((\x)^4)-7/9*((\x)^3)+14/3*((\x)^2)-32/3*(\x)+0})--(10.08,0)--cycle;
\draw[samples=200,domain=0:10.08,smooth,variable=\x] plot (\x,{1/24*((\x)^4)-7/9*((\x)^3)+14/3*((\x)^2)-32/3*(\x)+0});
\end{scope}
\begin{scope}[rotate=-10]
\draw[->] (-1.1,0)--(12,0) ;
\draw[->] (0,-8.1)--(0,15.1);
\clip (0,-15) rectangle (11,15);
\draw[fill =white, line width = 1.2pt,draw=none] (0,0) plot[domain=0:10.08](\x,{1/24*((\x)^4)-7/9*((\x)^3)+14/3*((\x)^2)-32/3*(\x)+0})--(10.08,0)--cycle;
\draw[samples=200,domain=0:10.08,smooth,variable=\x] plot (\x,{-1/24*((\x)^4)+7/9*((\x)^3)+-14/3*((\x)^2)+32/3*(\x)+0});
\draw[samples=200,domain=0:10.08,smooth,variable=\x] plot (\x,{1/24*((\x)^4)-7/9*((\x)^3)+14/3*((\x)^2)-32/3*(\x)+0});
\draw[samples=200,domain=0:10.08,smooth,variable=\x] plot (\x,{1/24*((\x)^4)-7/9*((\x)^3)+14/3*((\x)^2)-32/3*(\x)+0});

\end{scope}

\end{tikzpicture}
\end{center}
Đối với mỗi vị trí, người ta đo được chiều rộng $h$ của cái đàn và ghi lại qua bảng sau ($x$, $h$ có đơn vị cm)
\begin{center}
\begin{tabular}{|c|c|c|c|c|c|c|c|c|c|c|}
\hline$x$ & $0$ & $1$ & $2$ & $3$ & $4$ & $5$ & $6$ & $7$ & $8$ & $9$ \\
\hline$h$ & $0$ & $13{,}4$ & $16{,}4$ & $15{,}3$ & $14$ & $15{,}6$ & $20$ & $25{,}4$ & $28{,}4$ & $23{,}2$ \\
\hline
\end{tabular}
\end{center}
Bạn Nhâm nhận thấy rằng số liệu được để cập trên bảng gần giống với một hàm bậc bốn. Bằng cách mô phỏng hàm bậc bốn $y=f(x)=ax^4+bx^3+cx^2+dx+e$ trên hệ trục $Oxy$, đồ thị hàm số đi qua các điểm $O, A(6; 10)$ và có $3$ điểm cực trị có hoành độ là $2 ; 4 ; 8$. Dựa vào hàm số $f(x)$ tìm được, tính thể tích của cái đàn ghita (đơn vị $\text{cm}^3$, làm tròn đến hàng đơn vị). \shortans{$889$}
%plot(\x,{1.0/3.0*((-(\x)^(4.0))/24.0+14.0*(\x)^(3.0)/18.0-14.0*(\x)^(2.0)/3.0+32.0*(\x)/3.0)});
\loigiai{
Đồ thị hàm số $y=f(x)$ có $3$ điểm cực trị có hoành độ là $2; 4; 8$ nên
\[f'(x)=k(x-2)(x-4)(x-8),\,\, \text{với}\, k\neq 0,\]
hay
\[f'(x)=k\left(x^3-14x^2+56x-64\right).\]
Suy ra
\[f(x)=k\left(\dfrac{1}{4}x^4-\dfrac{14}{3}x^3+28x^2-64x\right)+e.\]
Do $f(0)=0$ nên $e=0$.\\
Vì vậy $f(x)=k\left(\dfrac{1}{4}x^4-\dfrac{14}{3}x^3+28x^2-64x\right)$.\\
Lại có $f(6)=10\Rightarrow -60k=10\Leftrightarrow k=-\dfrac{1}{6}$.\\
Suy ra $f(x)=-\dfrac{1}{6}\left(\dfrac{1}{4}x^4-\dfrac{14}{3}x^3+28x^2-64x\right)$.\\
Ta có
\begin{eqnarray*}
f(x)=0&\Leftrightarrow &-\dfrac{1}{6}\left(\dfrac{1}{4}x^4-\dfrac{14}{3}x^3+28x^2-64x\right)\\
&\Leftrightarrow& \hoac{&x=0	\\&\dfrac{1}{4}x^3-\dfrac{14}{3}x^2+28x-64=0. }
\end{eqnarray*}
Phương trình $\dfrac{1}{4}x^3-\dfrac{14}{3}x^2+28x-64=0$ có nghiệm duy nhất $x\approx 10{,}07$.\\
Diện tích mặt cắt ngang của cây đàn là
\[S=2\displaystyle\int\limits_{0}^{10{,}07}f(x)\mathrm{\, d}x =2\displaystyle\int\limits_{0}^{10{,}07}-\dfrac{1}{6}\left(\dfrac{1}{4}x^4-\dfrac{14}{3}x^3+28x^2-64x\right)\mathrm{\, d}x\approx 177{,}85\,\, \text{cm}^2.\]
Vậy thể tích của cây đàn là
\[V=\displaystyle\int\limits_{0}^{5}S\mathrm{\, d}x\approx 5\cdot 177{,}85\approx 889\,\, \text{cm}^3.\]
}
\end{ex}

\begin{ex}%[50 Đề minh họa tốt nghiệp 2025 - Đề 13]%[Lê Hữu Kiệt - Lê Quân]%[2D4C2-6]
Một chiếc xe đua Bugatti đang chuyển động trên đường đua. Đồ thị trên hình vẽ bên dưới biểu thị vận tốc $v$ (m/s) của chiếc xe đó trong $5$ giây đầu tiên.
\begin{center}
\begin{tikzpicture}[font=\footnotesize, line join=round, line cap=round, >=stealth, scale=1, y=0.166666666cm]
\draw[smooth] plot[domain=0:2] (\x,{(\x)^3/2+(\x)});
\draw (2,6)--(3,30)--(5,30);
\draw[dashed] (2,0)|- (0,6) (3,0)|-(0,30) (5,0)--(5,30);
\draw[->] (-0.5,0)--(6,0)node[below]{$t$ (s)};
\draw (0,0)node[below left]{$O$};
\draw[->] (0,-1)--(0,35)node[left]{$v$ (m/s)};
\foreach \x/\g/\n in {(2,0)/below/2, (3,0)/below/3, (5,0)/below/5, (0,6)/left/6, (0,30)/left/30, (2,6)/above/\, , (3,30)/above/\, , (5,30)/above/\,}{
\fill \x circle (1pt)node[\g]{$\n$};
}
\end{tikzpicture}
\end{center}
Đồ thị trong $2$ giây đầu tiên là một nhánh của hàm bậc ba nhận $O$ làm tâm đối xứng, trong $1$ giây tiếp theo xe tăng tốc với gia tốc $a$ (m/s$^2$) và đạt vận tốc $30$ m/s tại giây thứ $3$, sau đó duy trì vận tốc này đến giây thứ $5$. Biết quãng đường xe đi được trong $5$ giây đầu bằng $82$ m. Vận tốc của xe tại giây đầu tiên bằng bao nhiêu? (tính theo đơn vị km/h).
\par\shortans{$5{,}4$}
\loigiai{
Gọi $v_{1}(t)=at^3+bt^2+ct+d$ ($a\ne 0$) là hàm số bậc ba biểu diễn vận tốc trong $2$ giây đầu.\\
Ta có $v_{1}'(t)=3at^2+2bt+c$, $v_{1}''(t)=6at+2b$.\\
Do $O$ thuộc đồ thị hàm số $v_{1}(t)$ nên ta có $d=0$.\\
Điểm $O$ là điểm uốn nên $v''(0)=0 \Leftrightarrow b=0$.\\
Do $(2;6)$ thuộc đồ thị hàm số $v_{1}(t)$ nên ta có $8a+2c=6$. \quad$(1)$.\\
Trong thời gian từ giây thứ $2$ đến thứ $3$ vật tăng tốc với gia tốc $a$ m/s$^2$, suy ra $v_{2}(t)=\displaystyle\int a\mathrm{\,d}t=at+C$, với $C$ là hằng số.\\
Để tránh nhầm lẫn gia tốc $a$ m/s$^2$ và hệ số $a$ của $v_{1}(t)$.\\
Ta gọi $v_{2}(t)=et+g$ ($e\ne 0$) là hàm số biểu diễn vận tốc từ giây thứ $2$ đến thứ $3$.\\
Do $(2;6)$ và $(3;30)$ thuộc đồ thị hàm số $v_{2}(t)$ nên ta có hệ phương trình
\[ \heva{& 2e+g=6 \\& 3e+g=30} \Leftrightarrow \heva{& e=24 \\& g=-42.} \]
Suy ra $v_{2}(t)=24t-42$.\\
Từ giây thứ $3$ đến giây thứ $5$ vận tốc giữ ở mức $30$ m/s, suy ra hàm số biểu diễn vận tốc của xe trong thời gian này là $v_{3}(t)=30$.\\
Gọi $v(t)$ là hàm số biểu diễn vận tốc của xe trong $5$ giây đầu tiên, khi đó
\[ v(t)=\heva{
& at^3+ct & \text{\,khi\,} & 0 \leq t < 2 \\
& 24t-42 & \text{\,khi\,} & 2 \leq t < 3 \\
& 30 & \text{\,khi\,} & 3 \leq t \leq 5.
} \]
Quãng đường xe đi được trong $5$ giây đầu bằng $82$ m, ta có
\begin{eqnarray*}
&& \displaystyle\int\limits_0^5 v(t) \mathrm{\,d}x= 82 \\
&\Leftrightarrow& \displaystyle\int\limits_0^2 (at^3+ct) \mathrm{\,d}x + \displaystyle\int\limits_2^3 (24t-42) \mathrm{\,d}x + \displaystyle\int\limits_3^5 30 \mathrm{\,d}x = 82 \\
&\Leftrightarrow& \left(\dfrac{at^4}{4}+\dfrac{ct^2}{2}\right)\Bigg|_0^2 + 18 + 60 = 82 \\
&\Leftrightarrow& 4a+2c=4. \quad(2)
\end{eqnarray*}
Từ $(1)$ và $(2)$ ta có hệ phương trình $\heva{& 8a+2c=6 \\& 4a+2c=4} \Leftrightarrow \heva{&a=\dfrac{1}{2} \\& c=1.}$\\
Suy ra $v_{1}(t)=\dfrac{1}{2}t^3+t$.\\
Vậy trong giây đầu tiên, vận tốc của xe là $v_{1}(1)=1{,}5$ m/s $=5{,}4$ km/h.
}
\end{ex}

\begin{ex}%[2D1C5-8]
\immini{Cấu trúc tổ ong là một cấu trúc đặc biệt, mỗi lỗ ong là một lăng kính hình lục giác, một đầu hở còn một đầu tạo thành một góc tam diện. Ong đã xây các lỗ này với một cách làm tối ưu về diện tích bề mặt (đã sử dụng lượng sáp ong ít nhất để xây tổ). Người ta đã quan sát, nghiên cứu thì thấy rằng góc $\theta(\mathrm{rad})$ ở đỉnh nhất quán một cách đáng kinh ngạc, dựa trên cấu trúc hình học của lỗ ong người ta đã chứng minh được diện tích bề mặt $S$ của lỗ ong là $S=6 s \cdot h-\dfrac{3}{2} s^2 \cdot \cot \theta+\dfrac{3 \sqrt{3}}{2} s^2 \cdot \dfrac{1}{\sin \theta}$ ( $s$ là chiều dài các cạnh của lỗ ong, $h$ là chiều cao, $s$ và $h$ đều là hằng số). Vậy để tối thiểu hoá diện tích bề mặt con ong đã xây một góc $\theta$ bằng bao nhiêu? (làm tròn đến hàng phần trăm).}{
\begin{tikzpicture}[scale=1, line cap=round, line join=round, >=stealth,font=\footnotesize]
\def\a{2} % Độ dài cạnh lục giác
\def\theta{-10} % Góc nghiêng lục giác
\def\scaleY{0.4} % Tỷ lệ nén theo trục y
\def\h{5} % chiều cao
% Định nghĩa các đỉnh của hình lục giác đã nghiêng và nén dọc theo trục y
\foreach \i in {0,60,120,180,240,300} {
\path ({\a*cos(\i + \theta)}, {\scaleY*\a*sin(\i + \theta)}) coordinate (P\i);
}
% Vẽ lục giác
\draw  (P180)--(P240) -- (P300)--(P0) ;
% Vẽ các đường chéo để nhấn mạnh hình dạng
\draw[dashed] (P0) -- (P60) -- (P120) -- (P180) (P0) -- (P180) (P60) -- (P240)(P120) -- (P300);
\coordinate (P0') at ($(P0) + (0,\h-0.2)$) ;
\coordinate (P60') at ($(P60) + (0,\h+0.15)$);
\coordinate (P120') at ($(P120) + (0,\h-1)$);
\coordinate (P180') at ($(P180) + (0,\h)$);
\coordinate (P240') at ($(P240) + (0,\h-0.2)$);
\coordinate (P300') at ($(P300) + (0,\h+0.5)$);

\coordinate (O) at ($(P0)!0.5!(P180)$);
\coordinate (O') at ($(O) + (0,\h+1)$);
\draw[dashed] (P60) -- (P60')  (P120) -- (P120') (O) -- (O') ;
\draw  (P0)--(P0')  (P180)--(P180') (P240)--(P240') (P300)--(P300') (P0')--(P300')--(P240')--(P180')--(O') (O')--(P180') (O')--(P300') (P0')--(P60')--(O') ;
\draw[dashed]  (P60')--(P120')--(P180') ;

\draw ($(O)!0.4!(O')$) node[right]{$h$};
\draw ($(P300)!0.5!(P240)$) node[below]{$s$};
\end{tikzpicture}
}
\shortans[]{$0{,}96$}
\loigiai{
Ta có
\begin{eqnarray*}
S^{\prime}(\theta)&=&\dfrac{3}{2} s^2 \cdot \dfrac{1}{\sin ^2 \theta}-3 s^2 \dfrac{\sqrt{3}}{2} \cdot \dfrac{\cos \theta}{\sin ^2 \theta} \\
& =&\dfrac{3}{2} s^2 \cdot \dfrac{1}{\sin ^2 \theta}(1-\sqrt{3} \cos \theta)=0 \\
&\Leftrightarrow& \cos \theta=\dfrac{1}{\sqrt{3}} \Rightarrow \theta \approx 0{,}96 .
\end{eqnarray*}
Bảng biến thiên
\begin{center}
\begin{tikzpicture}[scale=1, font=\footnotesize, line join=round, line cap=round, >=stealth]
\tkzTabInit[lgt=1]	{$x$/1,$y'$/1,$y$/2}
{$0$,$0{,}96$,$2\pi$}
\tkzTabLine{,-,0,+,}
\tkzTabVar{+/ $+\infty$,-/ $6 s\left(h+\dfrac{1}{2 \sqrt{2}} s\right)$, +/ $+\infty$}
\end{tikzpicture}
\end{center}
Vậy $S_{\min }=6 s\left(h+\dfrac{1}{2 \sqrt{2}} s\right)$ khi
$\theta=0{,}96$
}
\end{ex}

\begin{ex}%[2D1C5-8]
Trong quang học, chúng ta đã biết đến định luật quang học về độ chiếu sáng. Nó được phát biểu như sau: Độ chiếu sáng từ một nguồn sáng $A$ đến một điểm $B$ cho bởi công thức $T=\dfrac{i\cos\alpha}{AB^2}$, trong đó $i$ là độ phát sáng của nguồn $A$; $\alpha$ là góc phản xạ của ánh sáng lên người quan sát (coi rằng người quan sát nhìn thẳng xuống mặt bàn, xem hình vẽ).
\begin{center}
\includegraphics[scale=0.18]{images/de15-2}
\end{center}
Một đồng xu được đặt cách ngọn nến một khoảng $BC=20$ cm. Hỏi ngọn lửa của cây nến nên đặt ở độ cao $h$ bằng bao nhiêu cm để chiếu sáng rõ nhất đồng tiền xu nằm trên bàn (kết quả làm tròn đến hàng phần chục).
\shortans[]{$14{,}1$}
\loigiai{
Trong trường hợp này, chúng ta có thể thay đổi chiều cao của cây nến bằng cách tăng chiều cao của đế nên đặt $AC=h=$ khoảng cách từ chỗ nến cháy xuống mặt bàn.\\
Suy ra $AB=\sqrt{h^2+20^2}=\sqrt{h^2+400}$.\\
Lại có $\cos \alpha=\dfrac{h}{AB}=\dfrac{h}{\sqrt{h^2+400}}$. Khi đó độ chiếu sáng $T=\dfrac{i\dfrac{h}{\sqrt{h^2+400}}}{h^2+400}=i\cdot \dfrac{h}{\sqrt{(h^2+400)^3}}=f(h)$.
\begin{itemize}
\item \textbf{Cách 1}\\
Khảo sát hàm số $f(h)$ trong khoảng $(0;+\infty)$. Ta có
\begin{eqnarray*}
f'(h)&=& i\cdot \dfrac{\sqrt{(h^2+400)^3}-h\cdot \dfrac{3\cdot 2h\cdot (h^2+400)^2}{2\sqrt{(h^2+400)^2}}}{(h^2+400)^3}\\
&=& i \cdot \dfrac{\sqrt{\left(h^2+400\right)^3}-\dfrac{3 h^2 \cdot\left(h^2+400\right)^2}{\sqrt{\left(h^2+400\right)^3}}}{\left(h^2+400\right)^3}\\
&=&  i \cdot \dfrac{\left(h^2+400\right)^3-3 h^2 \cdot\left(h^2+400\right)^2}{\left(h^2+400\right)^3 \sqrt{\left(h^2+400\right)^3}}\\
&=& i\cdot \dfrac{400-2 h^2}{\sqrt{\left(h^2+400\right)^5}}.
\end{eqnarray*}
Khi đó $f'(h)=0\Leftrightarrow h=10\sqrt{2}$. Ta có bảng biến thiên
\begin{center}
\begin{tikzpicture}
\tkzTabInit[nocadre=false,lgt=1.2,espcl=2.5,deltacl=0.7]
{$h$ /1,$f'(h)$ /0.6,$f(h)$ /2}
{$0$,$10\sqrt{2}$,$+\infty$}
\tkzTabLine{,+,$0$,-,}
\tkzTabVar{-/, +/,-/}
\end{tikzpicture}
\end{center}
Từ bảng biến thiên suy ra $\max f(h)$ khi $h=10\sqrt{2}\approx 14{,}1$.
\item \textbf{Cách 2}\\
Dùng Côsi. Ta có
$f(h)=i\cdot \dfrac{h}{\sqrt{\left(h^2+400\right)^3}}=i\sqrt{\dfrac{h^2}{\left(h^2+200+200\right)^3}}$.\\
Áp dụng bất đẳng thức Côsi cho $3$ số dương $h^2$, $200$, $200$ ta có
\begin{eqnarray*}
& &	h^2+200+200\geq 3\sqrt[3]{h^2\cdot 200\cdot 200}\\
&\Rightarrow& \left(h^2+200+200\right)^3\geq \left(3\sqrt[3]{h^2\cdot 200\cdot 200}\right)^3\\
&\Leftrightarrow & \left(h^2+200+200\right)^3\geq 27h^2\cdot 40\, 000.
\end{eqnarray*}
Suy ra $f(h)\leq i\sqrt{\dfrac{h^2}{27h^2\cdot 40\, 000}}=i\dfrac{1}{200\cdot 3\sqrt{3}}=\dfrac{i}{600\sqrt{3}}$.\\
Dấu \lq \lq =\rq \rq xảy ra khi và chỉ khi $h^2=200\Leftrightarrow h=\sqrt{200}\approx 14{,}1$.
\end{itemize}
}
\end{ex}

\begin{ex}%[2D1C5-6]
Hình vẽ sau mô tả một đường cong Agnesi và được xây dựng trong hệ tọa độ $Oxy$ như sau: vẽ một đường tròn có tâm $I(0;1)$ và bán kính bằng $1$, từ điểm $O$ kẻ một đường thẳng cắt đường tròn tại điểm thứ $2$ là điểm $B$ và cắt đường thẳng $y=2$ tại điểm $A$. Gọi $P$ là giao điểm của đường thẳng qua $A$ và vuông góc với $Ox$ và đường thẳng qua $B$ vuông góc với $Oy$. Tập hợp các điểm $P$ tạo thành một đường cong $y=f(x)$ gọi là đường cong Agnesi. Tiếp tuyến của đồ thị hàm số $y=f(x)$ có hệ số góc lớn nhất bằng bao nhiêu? (làm tròn kết quả đến hàng phần mười).
\begin{center}
\begin{tikzpicture}[scale=1,>=stealth, font=\footnotesize, line join=round, line cap=round]
\def\xmin{-4} \def\xmax{4}
\def\ymin{-0.7} \def\ymax{3}
\draw[->] (\xmin,0)--(\xmax,0) node [below]{$x$};
\draw[->] (0,\ymin)--(0,\ymax) node [left]{$y$};
\node at (0,0) [below left]{$O$};
\clip (\xmin+0.1,\ymin+0.1) rectangle (\xmax-0.5,\ymax-0.1);
\draw[smooth,samples=300] plot(\x,{8/(4+(\x)^2)});
\draw (0,1)node[left]{$(0;1)$} circle(1cm) (\xmin,2)node[above,xshift=0.9cm]{$y=2$}--(\xmax,2) (0,0)--(2,2) (0.4,0.1)node[above right]{$t$};
\fill (0,1)circle(2pt) (0,2)node[above left]{$Q$};
\draw[dashed] (2,2)node[above]{$A$}--(2,1)node[above right]{$P(x;y)$}--(1,1)node[left]{$B$};
\fill (2,1)circle(2pt);
\draw[->] (0.5,0) arc(0:45:0.5 cm and 0.5 cm);
\end{tikzpicture}
\end{center}
\shortans{$0{,}6$}
\loigiai{
\begin{center}
\begin{tikzpicture}[scale=1,>=stealth, font=\footnotesize, line join=round, line cap=round]
\def\xmin{-4} \def\xmax{4}
\def\ymin{-0.7} \def\ymax{3}
\draw[->] (\xmin,0)--(\xmax,0) node [below]{$x$};
\draw[->] (0,\ymin)--(0,\ymax) node [left]{$y$};
\node at (0,0) [below left]{$O$};
\clip (\xmin+0.1,\ymin+0.1) rectangle (\xmax-0.5,\ymax-0.1);
\draw[smooth,samples=300] plot(\x,{8/(4+(\x)^2)});
\draw (0,1)node[left]{$(0;1)$} circle(1cm) (\xmin,2)node[above,xshift=0.9cm]{$y=2$}--(\xmax,2) (0,0)--(2,2) (0.4,0.1)node[above right]{$t$};
\fill (0,1)circle(2pt) (0,2)node[above left]{$Q$};
\draw[dashed] (2,2)node[above]{$A$}--(2,1)node[above right]{$P(x;y)$}--(1,1)node[left]{$B$};
\fill (2,1)circle(2pt);
\draw[->] (0.5,0) arc(0:45:0.5 cm and 0.5 cm);
\draw[dashed] (2,2) -- (2,0) node[below]{$K$};
\draw[dashed] (1,1) -- (1,0) node[below]{$H$};
\end{tikzpicture}
\end{center}
Gọi $K$, $H$ lần lượt là hình chiếu của $A$ và $B$ lên $Ox$.\\ Tam giác $AOQ$ vuông tại $Q$ có $QB$ là đường cao, $O A=\sqrt{O K^2+A K^2}=\sqrt{x^2+4}$\\
Suy ra $ O B \cdot O A=O Q^2  \Rightarrow O B=\dfrac{O Q^2}{O A}=\dfrac{4}{\sqrt{x^2+4}}$.\\
Ta có  $\dfrac{B H}{A K}=\dfrac{O B}{O A}  \Rightarrow B H=\dfrac{A K \cdot O B}{O A}$.\\
Xét hàm số $y=\dfrac{2 \cdot \dfrac{4}{\sqrt{x^2+4}}}{\sqrt{x^2+4}}=\dfrac{8}{x^2+4}\Rightarrow y'=\dfrac{-8 \cdot 2 x}{\left(x^2+4\right)^2}=-\dfrac{16 x}{\left(x^2+4\right)^2} . $
\begin{eqnarray*}
y''&=&-16 \cdot \dfrac{\left(x^2+4\right)^2-2\left(x^2+4\right) \cdot 2 x \cdot x}{\left(x^2+4\right)^4}\\
& =&-16 \cdot \dfrac{\left(x^2+4\right)\left(x^2+4-4 x^2\right)}{\left(x^2+4\right)^4} \\
& =&-16 \cdot \dfrac{\left(x^2+4\right)\left(-3 x^2+4\right)}{\left(x^2+4\right)^4} \\
& =&-16 \cdot \dfrac{\left(-3 x^2+4\right)}{\left(x^2+4\right)^3} \cdot \\
\end{eqnarray*}
Ta có \begin{eqnarray*}
y''=0 &\Leftrightarrow&-16 \cdot \dfrac{-3 x^2+4}{\left(x^2+4\right)^3}=0 \\
& \Leftrightarrow&-3 x^2+4=0 \Leftrightarrow x^2=\dfrac{4}{3} \\
& \Leftrightarrow& x= \pm \dfrac{2 \sqrt{3}}{3} .
\end{eqnarray*}
Bảng biến thiên
\begin{center}
\begin{tikzpicture}
% Điều chỉnh tham số để bảng vừa vặn
\tkzTabInit[lgt=1.2,espcl=2.5,deltacl=0.5]%
{$x$/0.8, $y'$/0.8, $y$/3}  % Giảm độ rộng các cột
{$-\infty$, $\frac{-2\sqrt{3}}{3}$, $\frac{2\sqrt{3}}{3}$, $+\infty$}

\tkzTabLine{, +, z, -, z, +}  % z thay vì 0 để căn chỉnh đẹp hơn

% Định vị lại các nút với vị trí chính xác
\path
(N13) node[above=25pt] (A) {$0$}
(N22) node[below=2pt] (B) {\scriptsize $\dfrac{3\sqrt{3}}{8}$}
(N33) node[above=1pt] (C) {\scriptsize $\dfrac{-3\sqrt{3}}{8}$}
(N43) node[above=25pt] (D) {$0$};

% Vẽ mũi tên thẳng với độ dày vừa phải
\draw[-stealth,line width=0.6pt] (A) -- (B);
\draw[-stealth,line width=0.6pt] (B) -- (C);
\draw[-stealth,line width=0.6pt] (C) -- (D);
\end{tikzpicture}
\end{center}
Dựa vào bảng biến thiên suy ra $y_{\max }'=\dfrac{3 \sqrt{3}}{8} \approx 0{,}6$.
}
\end{ex}

\begin{ex}%[50 Đề minh họa tốt nghiệp 2025 - Đề 13]%[Lê Hữu Kiệt - Lê Quân]%[2D1C3-6]
Hệ thống mạch máu chứa các mạch máu gồm động mạch chính, động mạch con, mao mạch và tĩnh mạch để giúp đưa máu từ tim đến các cơ quan và ngược lại. Hệ thống hoạt động để tối ưu hoá (tối thiểu) năng lượng mà tim sử dụng trong quá trình bơm máu. Đặc biệt năng lượng này giảm khi sức cản của máu giảm. Hình vẽ dưới đây minh hoạ một mạch máu chính có bán kính $r_1$ phân nhánh với một góc $\alpha^\circ$ tạo thành một mạch máu nhỏ hơn với bán kính $r_2$.
\begin{center}
\begin{tikzpicture}[font=\footnotesize, line join=round, line cap=round, >=stealth, scale=1]
\draw
(1,1) ellipse ({0.3} and {0.5})
(8,1) ellipse ({0.3} and {0.5})
(8,4) ellipse ({0.15} and {0.4})
;
\draw (1,0.5)--(8,0.5) (1,1.5)--(3,1.5) (4.5,1.5)--(8,1.5);
\draw (7.95,4.38)--(3,1.5) (8,3.6)--(4.5,1.5);
\draw[dashed] (1,1)--(8,1) (8,4)--(3,1)node[below]{$B$};
\fill
(1,1) circle (1pt)node[left=-1mm]{$A$}
(3,1) circle (1pt)node[below]{$B$}
(8,4) circle (1pt)node[right=1mm]{$C$}
;
\draw[|<->|] (8.75,1)--(8.75,4)node[pos=0.5, right]{$b$};
\draw[|<->|] (1,0.25)--(8,0.25)node[pos=0.5, below]{$a$};
\draw[<->] (2,1)--(2,1.5)node[pos=0.5, right]{$r_1$};
\draw[<->] (6.75,3.25)--(6.5,3.53)node[pos=0.5, right]{$r_2$};
\path (8,1) coordinate (D) (3,1) coordinate (B) (8,4) coordinate (C)
pic[angle eccentricity=1.2,"$\alpha$"]{angle=D--B--C};
\draw (2,2)node[align=left]{Phân nhánh\\mạch máu};
\end{tikzpicture}
\end{center}
Sử dụng mô tả Định luật Poiseuille, người ta đã chứng minh được sức cản của máu theo con đường $ABC$ là
\[R(\alpha)=C\left(\dfrac{a-b\cot\alpha}{r_1^4}+\dfrac{b}{r_2^4\sin\alpha}\right)\]
với $C, a, b$ là các hằng số. Khi bán kính mạch máu nhỏ bằng $\dfrac{2}{3}$ bán kính mạch máu chính. Xác định $\alpha$ để sức cản này là nhỏ nhất. (Làm tròn kết quả đến hàng đơn vị).
\par\shortans{$79$}
\loigiai{
Các hằng số $C$, $a$, $b$ là các hằng số dương.\\
Ta có $r_2=\dfrac{2}{3}r_1 \Leftrightarrow r_2^4=\dfrac{16}{81}r_1^4$.\\
Khi đó
\[ R(\alpha)=C\left(\dfrac{a-b\cot\alpha}{r_1^4}+\dfrac{81b}{16r_1^4\sin\alpha}\right) =\dfrac{C}{r_1^4}\left(a-b\cot\alpha + \dfrac{81b}{16\sin\alpha}\right).\]
Điều kiện xác định $0^\circ<\alpha<180^\circ$.\\
Ta có $R'(\alpha)=\dfrac{Cb}{r_1^4}\cdot\dfrac{16-81\cos\alpha}{16\sin^2\alpha}$.\\
Ta có
\begin{eqnarray*}
&& R'(\alpha)=0 \\
&\Leftrightarrow& 16-81\cos\alpha=0 \\
&\Leftrightarrow& \cos\alpha=\dfrac{16}{81} \\
&\Leftrightarrow& \hoac{&\alpha\approx 79^\circ + k360^\circ \\& \alpha\approx -79^\circ + k360^\circ.}
\end{eqnarray*}
Do $0^\circ < \alpha < 180^\circ$ suy ra $\alpha\approx 79^\circ$.\\
Bảng biến thiên
\begin{center}
\begin{tikzpicture}[font=\footnotesize, line join=round, line cap=round, >=stealth, scale=1]
\tkzTabInit[lgt=1.2,espcl=2.5,deltacl=0.6]
{$\alpha$/1, $R'(\alpha)$/0.7, $R(\alpha)$/2}
{$0^\circ$, $79^\circ$ , $180^\circ$}
\tkzTabLine
{, - , $0$ , + ,}
\tkzTabVar
{+/, -/, +/}
\end{tikzpicture}
\end{center}
Từ bảng biến thiên, sức cản của mạch máu là nhỏ nhất đạt được khi $\alpha\approx 79^\circ$.
}
\end{ex}

\begin{ex}%[2D1C3-6]%[Tex đề Moon 2025]%[Nguyễn Hồng Thạch]
\immini[thm]
{
Đường đi của một khinh khí cầu được gắn trong hệ trục tọa độ là một phần của đường cong bậc hai trên bậc nhất có đồ thị cắt trục hoành tại điểm có tọa độ là $(1;0)$ và $(8;0)$ với đơn vị trên hệ trục tọa độ là $1$ km. Biết rằng điểm cực đại của đồ thị hàm số là điểm $(6;5)$. Hỏi khi khinh khí cầu đi qua điểm cực đại và cách mặt đất $3875$ m thì khinh khí cầu cách gốc tọa độ theo phương ngang bao nhiêu km?
}
{
\begin{tikzpicture}[scale=0.4,>=stealth, font=\footnotesize, line join=round, line cap=round]
\def\a{5} \def\b{-45} \def\c{40} \def\d{3}\def\e{-28} % Hệ số
\def\xmin{-1} \def\xmax{10}
\def\ymin{-1} \def\ymax{6}
\draw[->] (\xmin,0)--(\xmax,0) node [below]{$x$};
\draw[->] (0,\ymin)--(0,\ymax) node [left]{$y$};
\node at (0,0) [below left]{$O$};
\draw[smooth,samples=300,domain=1:8] plot(\x,{(\a*(\x)^2+\b*(\x)+\c)/(\d*(\x)+\e)});
%\draw (3,50/19)node[above,font=\fontsize{15pt}{2}\selectfont,yshift=-0.15cm]{\faFly};
\end{tikzpicture}
}
\shortans{$7{,}2$}
\loigiai{
Giả sử đường đi có phương trình là $y=\dfrac{a(x-1)(x-8)}{x+b}$ vì đồ thị hàm số cắt trục hoành tại hai điểm $A(1;0)$ và $B(8;0)$.\\
Ta có $y=\dfrac{ax^2-9ax+8a}{x+b}\ (a\neq 0)\Rightarrow y'=\dfrac{(2ax-9a)(x+b)-(ax^2-9ax+8a)}{(x+b)^2}=\dfrac{ax^2+2abx-9ab-8a}{(x+b)^2}$.\\
Vì hàm số có điểm cực đại $x=6$ nên $y'(6)=0\Rightarrow 28a+3ab=0\Leftrightarrow b=-\dfrac{28}{3}$.\\
Ta có đồ thị đi qua điểm $(6;5)$ nên $5=\dfrac{a(6-1)(6-8)}{6-\dfrac{28}{3}}\Leftrightarrow a=\dfrac{5}{3}$.\\
Suy ra hàm số là $y=\dfrac{\dfrac{5}{3}(x-1)(x-8)}{x-\dfrac{28}{3}}$ hay $y=\dfrac{5x^2-45x+40}{3x-28}$.\\
Vì khinh khí cầu đi qua điểm cực đại nên $x>6$ và cách mặt đất $3\,875$ m nên $y=3{,}875$.\\
Suy ra \begin{eqnarray*}
&&\dfrac{5x^2-45x+40}{3x-28}=3{,}875\\
&\Leftrightarrow& 5x^2-45x+40=11{,}625x-108{,}5\\
&\Leftrightarrow& 5x^2-56{,}625x+148{,}5=0\\
&\Leftrightarrow& \hoac{&x= 4{,}125\\&x=7{,}2.}
\end{eqnarray*}
Kết hợp điều kiện ta có $x=7{,}2$.\\
Vậy khoảng cách từ khinh khí cầu sau khi bay qua điểm cực đại và cách mặt đất $3875$ m đến gốc tọa độ tính theo phương ngang là $7{,}2$ km.
}
\end{ex}

\begin{ex}%[2D4C3-2]
\immini[thm]
{
Một nhà sản xuất dự kiến xây dựng sân khấu cho một concept âm nhạc trên một mảnh đất hình chữ nhật có kích thước $20\text{ m}\times10\text{ m}$. Nhà sản xuất mô phỏng sân khấu thông qua bản vẽ trên hệ trục $Oxy$ như sau: vẽ hai parabol có đỉnh $I_1$, $I_2$ có cùng hoành độ, trong đó parabol đỉnh $I_1$ tiếp xúc với cạnh ngắn của hình chữ nhật. Vị trí giao nhau của hai parabol là $A$ và $B$ cùng với hai đỉnh $I_1$, $I_2$ tạo thành hình thoi có độ dài hai đường chéo là $I_1I_2=16$ (m) và $AB=8$ (m) (tham khảo hình vẽ). Trên thực tế, khu vực màu đen là khu vực thiết kế dành cho khán giả, màu xám là khu vực sân khấu và màu trắng là khu vực hậu trường. Chi phí để xây dựng khu vực sân khấu, hậu trường, khán đài lần lượt là $2$ triệu đồng, $200$ nghìn đồng và $400$ nghìn đồng mỗi mét vuông. Tổng chi phí xây dựng bằng bao nhiêu triệu đồng? Làm tròn đến hàng đơn vị.
}
{
\begin{tikzpicture}[scale=0.5,>=stealth, font=\footnotesize, line join=round, line cap=round]
\fill[gray,opacity=0.6] plot[domain=-4:4](\x,{1/2*(\x)^2-8})--plot[domain=4:-4](\x,{-1/2*(\x)^2+8})--cycle;
\fill[black,opacity=0.8] plot[domain=-5:5](\x,{1/2*(\x)^2-8})--(5,-12)--(-5,-12)--cycle;
\draw (-5,8)--(5,8)--(5,-12)--(-5,-12)--cycle;
\draw[smooth,samples=300,domain=-5:5] plot(\x,{1/2*(\x)^2-8});
\draw[smooth,samples=300,domain=-4:4] plot(\x,{-1/2*(\x)^2+8});
\draw[<->] (-4,0)node[below right]{$A$}--(4,0)node[below left]{$B$};
\draw[<->] (0,8)node[above]{$I_1$}--(0,-8)node[above left]{$I_2$};
\end{tikzpicture}
}
\shortans[]{$210$}
\loigiai{
\begin{center}
\begin{tikzpicture}[scale=0.5,>=stealth, font=\footnotesize, line join=round, line cap=round]
\fill[gray,opacity=0.6] plot[domain=-4:4](\x,{1/2*(\x)^2-8})--plot[domain=4:-4](\x,{-1/2*(\x)^2+8})--cycle;
\fill[black,opacity=0.8] plot[domain=-5:5](\x,{1/2*(\x)^2-8})--(5,-12)--(-5,-12)--cycle;
\draw (-5,8)--(5,8)--(5,-12)--(-5,-12)--cycle;
\draw[smooth,samples=300,domain=-5:5] plot(\x,{1/2*(\x)^2-8});
\draw[smooth,samples=300,domain=-4:4] plot(\x,{-1/2*(\x)^2+8});
\draw[<->] (-4,0)node[below right]{$A$}--(4,0)node[below left]{$B$};
\draw[<->] (0,8)node[above right]{$I_1$}--(0,-8)node[above left]{$I_2$};
\draw[->] (-7.5,0)--(7.5,0) node [below]{$x$};
\draw[->] (0,-8)--(0,9) node [left]{$y$};
\node at (0,0) [below left]{$O$};
\end{tikzpicture}
\end{center}
Gắn hệ trục toạ độ như hình vẽ. Gọi $f(x)$ là hàm số của parabol phía trên khi đó $y=f(x)=a(x-4)(x+4)a(x^2-16)$.\\
Parabol phía trên đi qua điểm $(0;8)$ nên $8=a(0^2-16)\Leftrightarrow a=-\dfrac{1}{2}$. Vậy $f(x)=-\dfrac{1}{2}(x^2-16)$.\\
Gọi $g(x)$ là hàm số của parabol phía dưới khi đó $y=g(x)=b(x-4)(x+4)=b(x^2-16)$.\\
Parabol phía dưới đi qua điểm $(0;-8)$ nên $-8=b(0^2-16)\Leftrightarrow b=\dfrac{1}{2}$. Vậy $g(x)=\dfrac{1}{2}(x^2-16)$.\\
Diện tích sân khấu là
\begin{eqnarray*}
S_{sk}&=&\displaystyle \int_{-4}^{4}\left(f(x)-g(x)\right)\mathrm{\,d}x\\
&=& \displaystyle \int_{-4}^{4}\left(-\dfrac{1}{2}(x^2-16)-\dfrac{1}{2}(x^2-16)\right)\mathrm{\,d}x\\
&=& \dfrac{256}{3}\, (\text{m}^2).
\end{eqnarray*}
Diện tích khán đài là
\begin{eqnarray*}
S{kd}&=&\displaystyle\int_{-5}^{5}\left(g(x)+12\right)\mathrm{\,d}x\\
&=& \displaystyle\int_{-5}^{5}\left(\dfrac{1}{2}(x^2-16)+12\right)\mathrm{\,d}x\\
&=& \dfrac{245}{3}\,(\text{m}^2).
\end{eqnarray*}
Diện tích hậu trường là
\[S_{ht}=S-S_{sk}-S_{kd}=20\cdot 10-\dfrac{256}{3}-\dfrac{245}{3}=33\, (\text{m}^2).\]
Tổng chi phí xây dựng là $=S_{sk}\cdot 2+S_{ht}\cdot 0{,}2+S_{kd}\cdot 0{,}4\approx 210$ (triệu đồng).
}
\end{ex}


\Closesolutionfile{ans}
\newpage

\begin{center}
    % \color{mycolor1}
    \bfseries\faGg~\faGg~\faGg~BẢNG ĐÁP ÁN TRẮC NGHIỆM~\faGg~\faGg~\faGg
\end{center}
% \inputansbox{10}{ans/ansBTchoice}
\inputansbox{3}{ans/ansBTchoiceTF}
\inputansbox{3}{ans/ansBTshortans}
\newpage

% \setcounter{deso}{10}
% \def\sode{1}
\begin{name}
	{\tenchude}
	{\tendethi}
	{\tentruong}
	{\thoigian}
\end{name}
\caulc
\Opensolutionfile{ans}[ans/ans-HXN-\sode-T]
%Câu hỏi
\begin{ex}%Câu 1
\immini
{
        Cho hàm số có đồ thị là đường cong trong hình bên. Hàm số đã cho đồng biến trên khoảng nào dưới đây?
    \choice
    {\True $\left(0;1\right)$}
    {$\left(-\infty;0\right)$}
    {$\left(1;+\infty\right)$}
    {$\left(-1;0\right)$}
}
{
     \begin{tikzpicture}[font=\footnotesize, line join=round, line cap=round, >=stealth]
        \draw[->] (-2,0)--(2,0) node[below]{$x$};
        \draw[->] (0,-1.25)--(0,3) node[left]{$y$};
        \draw[dashed] (-1,0)node[below]{$-1$}|-(0,2)node[above left]{$2$}-|(1,0)node[below]{$1$}
        (0,0)node[below left]{$O$};
        \draw[domain=-1.65:1.65,samples=150] plot(\x,{-(\x)^4+2*(\x)^2+1});
    \end{tikzpicture}
}
\end{ex}

\begin{ex}%Câu 2
    Thống kê điểm kiểm tra giữa kỳ 1 môn Toán của 30 học sinh lớp 12C1 của một trường THPT được ghi lại ở bảng sau:\\
    \centerline{\begin{tabular}{|c|c|c|c|c|}
            \hline
            Điểm & $\left[2;4\right)$ & $\left[4;6\right)$ & $\left[6;8\right)$ & $\left[8;10\right)$\\
            \hline
            Số học sinh & $ 4$ & $ 8$ & $ 11$ & $ 7$\\
            \hline
    \end{tabular}}\\
    Trung vị của mẫu số liệu gốc thuộc khoảng nào trong các khoảng dưới đây?
    \choice
    {$\left[2;4\right)$}
    {$\left[4;6\right)$}
    {\True $\left[6;8\right)$}
    {$\left[8;10\right)$}
\end{ex}

\begin{ex}%Câu 3
    Trong không gian $Oxyz$ , một vectơ pháp tuyến của mặt phẳng $\dfrac{x}{-2}+\dfrac{y}{-1}+\dfrac{z}{3}=1$ là
    \choice
    {\True $\overrightarrow{n}=(3;6;-2)$}
    {$\overrightarrow{n}=(2;-1;3)$}
    {$\overrightarrow{n}=(-3;-6;-2)$}
    {$\overrightarrow{n}=(-2;-1;3)$}
\end{ex}

\begin{ex}%Câu 4
    Cho cấp số cộng $\left(u_n\right)$ với số hạng đầu $u_1=-6$ và công sai $d=4$. Tính tổng $S$ của $14$ số hạng đầu tiên của cấp số cộng đó.
    \choice
    {$S=46$}
    {$S=308$}
    {$S=644$}
    {\True $S=280$}
\end{ex}

\begin{ex}%Câu 5
    Cho tứ diện đều $ABCD$ có cạnh bằng $a$. Tích vô hướng $\overrightarrow{AB}\cdot\overrightarrow{AC}$ bằng
    \choice
    {$a^2$}
    {$-a^2$}
    {\True $\dfrac{1}{2}{a^2}$}
    {$\dfrac{\sqrt{3}}{2}{a^2}$}
\end{ex}

\begin{ex}%Câu 6
    Giá trị lớn nhất của hàm số $f(x)=x^3-8x^2+16x-9$ trên đoạn $\left[1;3\right]$ là 
    \choice
    {$\max\limits_{\left[1;3\right]}f(x)=0$}
    {\True $\max\limits_{\left[1;3\right]}f(x)=\dfrac{13}{27}$}
    {$\max\limits_{\left[1;3\right]}f(x)=-6$}
    {$\max\limits_{\left[1;3\right]}f(x)=5$}
\end{ex}

\begin{ex}%Câu 7
    Trong một phép thử với $ A$, $B$ là hai biến cố bất kì, biết rằng $ P(A)=0{,}5$; $ P\left(AB\right)=0{,}3$. Khi đó $ P(B|A)$ bằng
    \choice
    {\True $ 0{,}6$}
    {$ 0{,}15$}
    {$ 0{,}7$}
    {$ 0{,}35$}
\end{ex}

\begin{ex}%Câu 8
    Cho biết $\int\limits_1^3f(x)\mathrm{\,d}x=3$, giá trị của $\int\limits_1^3\dfrac{1}{3}f(x)\mathrm{\,d}x$ bằng
    \choice
    {$ 2$}
    {\True $ 1$}
    {$\dfrac{1}{3}$}
    {$ 3$}
\end{ex}

\begin{ex}%Câu 9
    Tập nghiệm của bất phương trình $2^x\le 4$ là
    \choice
    {\True $\left(-\infty;2\right]$}
    {$\left[0;2\right]$}
    {$\left(-\infty;2\right)$}
    {$\left(0;2\right)$}
\end{ex}

\begin{ex}%Câu 10
    Phát biểu nào sau đây là đúng?
    \choice
    {$\int\dfrac{1}{x}\mathrm{\,d}x=\left| x\right|+C$}
    {\True $\int\dfrac{1}{x}\mathrm{\,d}x=\ln \left| x\right|+C$}
    {$\int\ln x\mathrm{\,d}x=x+C$}
    {$\int\ln\left| x\right|\mathrm{\,d}x=\ln x+C$}
\end{ex}

\begin{ex}%Câu 11
    Bạn An rất thích nhảy hiện đại. Thời gian tập nhảy mỗi ngày của bạn An được thống kê lại ở bảng sau:\\
    \centerline{\begin{tblr}{
                colspec={|c|c|c|c|c|c|},
                hlines,
                vlines,
            }
            Thời gian (phút) & [20;25) & [25;30) & [30;35) & [35;40) & [40;45) \\
            Số ngày & 6 & 6 & 4 & 1 & 1 \\
    \end{tblr}}
    Độ lệch chuẩn của mẫu số liệu ghép nhóm có giá trị gần nhất với giá trị nào dưới đây?
    \choice
    {$ 31{,}25$}
    {$ 31{,}26$}
    {$ 5{,}4$}
    {\True $ 5{,}6$}
\end{ex}

\begin{ex}%Câu 12
    Trong không gian với hệ tọa độ $Oxyz$ , cho đường thẳng $\Delta :\dfrac{x-2}{-3}=\dfrac{y}{1}=\dfrac{z+1}{2}$. Gọi $M$ là giao điểm của $\Delta $ với mặt phẳng $(P):x+2y-3z+2=0$ . Tọa độ điểm $M$ là
    \choice
    {$ M\left(2;0;-1\right)$}
    {$ M\left(5;-1;-3\right)$}
    {$ M\left(1;0;1\right)$}
    {\True $ M\left(-1;1;1\right)$}
        \end{ex}
\Closesolutionfile{ans}
\cauds
\Opensolutionfile{ans}[ans/ans-HXN-\sode-TF]
%Câu hỏi

\begin{ex}%Câu 13
% \immini
% {
    Năm $2025$, báo Giáo dục đã có cuộc khảo sát tại một trường đại học và thấy rằng có $40\%$ sinh viên quan tâm đến chương trình học bổng A; có $17\%$ trong số những sinh viên quan tâm đến học bổng A cũng đã quan tâm đến học bổng B. Qua khảo sát họ cũng thấy rằng có $20\%$ sinh viên quan tâm đến chương trình học bổng B. Người ta chọn ngẫu nhiên một sinh viên từ trường đại học này để thăm dò ý kiến.
% }
% {
%     \includegraphics[width=6cm]{img/HXN-1-13}
% }
    \choiceTF
    {Xác suất để sinh viên được được chọn quan tâm cả hai chương trình học bổng bằng $0{,}062$}
    {Xác suất để sinh viên quan tâm học bổng A nếu biết rằng họ đã quan tâm học bổng B bằng $0{,}4$}
    {Xác suất để sinh viên không quan tâm đến cả chương trình A lẫn học chương trình B bằng $0{,}41$}
    {\True Sinh viên được chọn cho rằng mình có quan tâm đến học bổng $B$; hai hôm sau một nhà báo khác quay lại trường và tiếp tục chọn ngẫu nhiên một sinh viên để thăm dò ý kiến thì gặp được một sinh viên quan tâm đến học bổng $B$, xác suất để người này không quan tâm đến học bổng A bằng $0{,}66$}
   \loigiai{
       Gọi $A$ là biến cố: \lq\lq Sinh viên quan tâm đến học bổng $A$\rq\rq và $B$ là biến cố: \lq\lq Sinh viên quan tâm đến học bổng $B$\rq\rq.\\
       Theo giả thiết ta có $P(A)=0{,}4$; $P\left(B|A\right)=0{,}17$; $P(B)=0{,}2$.\\
       Từ đây ta có sơ đồ hình cây như sau:\\
       \centerline{\includegraphics[width=6cm]{img/HXN-1-13-LG}}
       \begin{itemchoice}
           \itemch 
           Ta có $P(AB)=P(A)\cdot P\left(B|A\right)=0{,}4\cdot 0{,}17=0{,}068$.
           \itemch Ta có $P\left(A|B\right)=\dfrac{P(AB)}{P(B)}=\dfrac{0{,}068}{0{,}2}=0{,}34$.
           \itemch Ta có $P\left(A\cup B\right)=P(A)+P(B)-P(AB)=0{,}4+0{,}2-0{,}068=0{,}532$.\\
           Do đó $P\left(\bar{A}\bar{B}\right)=1-P\left(A\cup B\right)=1-0{,}532=0{,}468$.
           \itemch Ta có $P\left(\bar{A}|B\right)=\dfrac{P\left(\bar{A}B\right)}{P(B)}=\dfrac{P(B)-P(AB)}{P(B)}=\dfrac{0{,}2-0{,}068}{0{,}2}=0{,}66$.\\
           Vì hai cuộc khảo sát là độc lập nên lần chọn đầu không ảnh hưởng đến lần chọn sau, xác suất cần tính là $P\left(\bar{A}|B\right)=0{,}66$.
       \end{itemchoice}
   }
\end{ex}

\begin{ex}%Câu 14
\immini
{
    Cho hàm số $y=\mathrm{e}^x$ có đồ thị $(C)$. Hình phẳng $(\mathscr{D})$ giới hạn bởi các đồ thị $(C)$, tiếp tuyến của $(C)$ tại điểm $ M\left(1;e\right)$ và đường thẳng $ y=-\dfrac{1}{e}x$ được tô đậm như hình vẽ.
    \choiceTF
    {Phương trình tiếp tuyến của $(C)$ tại điểm $ M\left(1;e\right)$ là $y=e\cdot x+e$}
    {\True Đường thẳng $ y=-\dfrac{1}{e}x$ cắt đồ thị $(\mathscr{D})$ tại điểm $\left(-1;\dfrac{1}{e}\right)$}
    {\True Diện tích hình phẳng $(H)$ bằng $0{,}81$ (làm tròn đến hàng phần trăm)}
    {Khi quay hình $(\mathscr{H})$ quanh trục hoành thì được khối tròn xoay có thể tích bằng $3{,}03$ (làm tròn đến hàng phân trăm)}
}
{
   \includegraphics[width=6cm]{img/HXN-1-14}
}
    
    \loigiai{
        \begin{itemchoice}
            \itemch Phương trình tiếp tuyến của $(C)$ tại điểm $M(1;e)$ là $y=y'(1)(x-1)+e$ hay $y=ex$.
            \itemch Toạ độ giao điểm của $(C)$ và đường thẳng $y=-\dfrac{1}{e}x$ thỏa mãn $\heva{& e^x=-\dfrac{1}{e}x \\& y=e^x } \Leftrightarrow \heva{& x=-1 \\& y={e^{-1}}=\dfrac{1}{e}.} $
            \itemch Dễ thấy đường thẳng $y=ex$ cắt đường thẳng $y=-\dfrac{1}{e}x$ tại điểm có hoành độ $x=0$ và cắt đồ thị $(C)$ tại điểm có hoành độ $x=1$.\\
            Do đó diện tích hình phẳng (H) là $S=\int\limits_{-1}^0{\left(\mathrm{e}^x+\dfrac{1}{e}x\right)\mathrm{\,d}x}+\int\limits_0^1{\left(\mathrm{e}^x-ex\right)\mathrm{\,d}x}\approx0{,}81$.
            \itemch Thể tích khối tròn xoay là $V=\pi \int\limits_{-1}^1{\left(e^x\right)^2\mathrm{\,d}x}-\pi \int\limits_{-1}^0{\left(-\dfrac{1}{e}x\right)^2\mathrm{\,d}x}-\pi\int\limits_0^1{(ex)^2\mathrm{\,d}x}\approx 3{,}51$.
        \end{itemchoice}
    }
\end{ex}

\begin{ex}%Câu 15
\immini
{
Hai thành phố cách nhau một con sông. Lấy $A$ và $B$ lần lượt là hai điểm mốc của hai thành phố trong việc đo đạc, đơn vị là km. Người ta xây dựng một cây cầu $EF$ bắc qua sông biết rằng vị trí $A$ cách con sông một khoảng $AH=5$km và vị trí B cách con sông một khoảng là $BK=7$km (xem hình vẽ), biết $HE+KF=24km$ và độ dài $EF$ không đổi.
Đặt $HE=x$ (km), với $x\in\left(0;24\right)$.
}
{
    \includegraphics[width=6cm]{img/HXN-1-15}
}
    \choiceTF
    {\True $AE=\sqrt{25+x^2}$(km), $BF=\sqrt{49+\left(24-x\right)^2}$(km)}
    {Tổng quãng đường đi từ $A$ đến $B$ bằng $\sqrt{25+x^2}+\sqrt{49+x^2}+EF$ (km)}
    {\True Nếu đặt $f(x)=AE+BF$ (km) thì $f'(x)=\dfrac{x}{\sqrt{x^2+25}}+\dfrac{x-24}{\sqrt{x^2-48x+625}}$, $\forall x\in\left(0;24\right)$}
    {Người ta muốn đi từ $A$ đến $B$ theo quãng đường ngắn nhất thì họ phải xây cầu sao cho khoảng cách hai điểm $E$, $H$ bằng $9$ km}
    \loigiai{
        \begin{itemchoice}
            \itemch Với $HE=x$ thì $FK=24-x$ ($0<x<24$).\\
            Ta có $\heva{& AE=\sqrt{25+x^2} \\& BF=\sqrt{49+(24-x)^2}.} $
            \itemch Tổng quãng đường đi từ $A$ đến $B$ là $AE+EF+BF=\sqrt{25+x^2}+\sqrt{49+(24-x)^2}+EF$ (km).
            \itemch Xét hàm số $f(x)=\sqrt{x^2+25}+\sqrt{x^2-48x+625}$;
            $f'(x)=\dfrac{x}{\sqrt{x^2+25}}+\dfrac{x-24}{\sqrt{x^2-48x+625}}$, $\forall x\in (0;24)$.
            \itemch Ta cần tổng quãng đường $AE+EF+FB$ ngắn nhất, mà $EF$ không đổi nên $AE+FB$ bé nhất. \\
            Ta có $f'(x)=\dfrac{x}{\sqrt{x^2+25}}+\dfrac{x-24}{\sqrt{x^2-48x+625}}$, $\forall x\in (0;24)$; $f'(x)=0\Rightarrow x=10$.\\
            Bảng biến thiên \\
            \centerline{\begin{tikzpicture}[>=stealth]
                    \tkzTabInit[nocadre=false,lgt=1.2,espcl=2.5,deltacl=0.5]{$x$/.7 ,$f'(x)$/.7,$f(x)$/2}
                    {$0$ , $10$ , $24$}
                    \tkzTabLine{ ,-,$0$,+,}
                    \tkzTabVar{+/,-/$12\sqrt{5}$,+/}
                \end{tikzpicture}}
            Ta có $\min\limits_{(0;24)} f(x)=12\sqrt{5}$; khi đó $x=10$km  và $BF=7\sqrt{5}$km $\approx 15{,}65$km.
        \end{itemchoice}
    }
\end{ex}

\begin{ex}%Câu 16
\immini
{
    Trong Dragon Ball, quả cầu Genki là chiêu thức lợi hại mà Son Goku thường sử dụng khi gặp những đối thủ lớn. Được biết trong trận đánh với Frieza đại đế, cuộc chiến có liên quan đến vận mệnh vũ trụ, Goku đã dùng quả cầu này để tung đòn tuyệt sát với Frieza.\\
Chọn hệ trục tọa độ $Oxyz$ thích hợp, đơn vị trên mỗi trục là mét, mặt phẳng $Oxy$ là mặt đất và tia $Oz$ hướng lên trời, Son Goku đứng ở vị trí $ A\left(5;0;40\right)$, Frieza đại đế đứng ở vị trí $ B\left(85;60;40\right)$. Trước khi Goku tạo ra quả cầu Genki thì Frieza đã tấn công phủ đầu, hắn lao về phía Goku với vận tốc $50$ m/s.
}
{
    \includegraphics[width=5cm]{img/HXN-1-16}
}
\choiceTF
    {\True Frieza sẽ mất $2$ giây để đến được vị trí Goku đang đứng}
    {Vectơ vận tốc của Frieza là $\vec{v}=\left(400;300;0\right)$, đơn vị: m/s}
    {\True Sau khi tránh được đòn hiểm từ Frieza, Goku đứng ở vị trí $ C\left(8;-1;46\right)$ đã tạo ra quả cầu Genki được mô hình hóa với phương trình $\left(x-8\right)^2+\left(y+1\right)^2+\left(z-58\right)^2=100$. Khoảng cách bé nhất từ vị trí $ D\left(-182;159;45\right)$ mà Frieza đang đứng đến quả cầu bằng $ 238{,}7$m (kết quả làm tròn đến hàng phần chục)}
    {\True Quả cầu được Goku ném về phía Fide với vận tốc lên đến $ 64$ m/s. Cứ sau mỗi giây thì bán kính nó tăng lên $1$ mét. Nếu Frieza không di chuyển thì sau 3,67 giây (làm tròn đến hàng phần trăm của giây) quả cầu Genki đến được vị trí của Frieza}
    \loigiai{
        \begin{itemchoice}
            \itemch Ta có $\vec{BA}=(-80;-60;0)$ và $AB=\sqrt{(-80)^2+(-60)^2}=100$ m.\\
            Thời gian để Frieza bay từ $B$ đến $A$ để tấn công Goku là $\dfrac{100}{50}=2$s.
            \itemch Vectơ vận tốc của Frieza có dạng $\vec{v}=k\vec{BA}=(-80k;-60k;0)$, với tham số $k>0$.\\
            Ta có $\left| {\vec{v}} \right|=50\Rightarrow \sqrt{(-80k)^2+(-60k)^2}=50\Rightarrow 100|k|=50\Rightarrow k=\dfrac{1}{2}>0$.\\
            Do đó Frieza bay đến chỗ Goku với vectơ vận tốc $\vec{v}=(-40;-30;0)$.
            \itemch Quả cầu Genki có tâm $I(8;-1;58)$, bán kính $R=10\,m$.\\
            $ID=\sqrt{\left(-182-8\right)^2+(159+1)^2+(45-58)^2}=\sqrt{61\,869}\,m\approx 248{,}7\,m$.\\
            Khoảng cách ngắn cần tính là $ID-R=\sqrt{61\,869}-10\approx 238{,}7\,m$.
            \itemch Sau $t$ giây, điểm $M$ (thuộc mặt cầu gần Frieza nhất) di chuyển đoạn đường: $64t+t=65t$ (m).\\
            Khi $M$ chạm vào Frieza (nếu hắn đứng yên) thì $ID-R=65t\Rightarrow t=\dfrac{ID-R}{65}\approx 3{,}67$ (giây).
        \end{itemchoice}
    }
\end{ex}
\Closesolutionfile{ans}
\caukq
\Opensolutionfile{ans}[ans/ans-HXN-\sode-SA]
%Câu hỏi

\begin{ex}%Câu 17
\immini
{
        Một cái ly nước hình hình trụ có chiều cao 9 cm. Lượng nước trong ly chiếm $\dfrac{2}{3}$ thể tích ly nước. Hoa đặt một viên kim cương hình lập phương vào miệng ly nước thì thấy một đỉnh của viên kim cương chạm vào mặt nước, đồng thời mô hình ly nước và kim cương cùng lấy trục ly nước làm trục đối xứng. Nếu ban đầu Hoa đổ nước đầy ly thì sau khi đặt khối lập phương như trên, lượng nước tràn ra là bao nhiêu cm khối (làm tròn đến hàng phần chục và bỏ qua độ dày của ly)?
    \shortans{23,4}
}
{
    \includegraphics[width=3cm]{img/HXN-1-17}
}
\loigiai{
\immini
{
    Xét hình chóp tam giác đều $SABC$ trong đó $S$ là đỉnh của hình lập phương nằm bên trong ly nước và $A$, $B$, $C$ là các điểm chung của kim cương với miệng ly; $O$ là trọng tâm tam giác $ABC$ và $H$ là trung điểm $BC$. \\
Đặt $x$ (cm) là cạnh đáy hình chóp thì\\
\centerline{
    $AO=\dfrac{2}{3}AH=\dfrac{2}{3}\cdot \dfrac{x\sqrt{3}}{2}=\dfrac{x\sqrt{3}}{3}$.
}
Vì hình chóp $S.ABC$ có $SA$, $SB$, $SC$ bằng nhau và đôi một vuông góc (tại $S$) nên $SA=SB=SC=\dfrac{x}{\sqrt{2}}$.\\
Từ đó suy ra $SO=\sqrt{SA^2-OA^2}=\sqrt{\dfrac{x^2}{2}-\dfrac{x^2}{3}}=\dfrac{x\sqrt{6}}{6}$.
}
{
    \includegraphics[width=6cm]{img/HXN-1-17-LG}
}
Theo giả thiết thì chiều cao hình chóp $S.ABC$ bằng $\dfrac{1}{3}$ chiều cao ly nước, tức là $SO=\dfrac{1}{3}\cdot 9=3$.\\
Khi đó $ \dfrac{x\sqrt{6}}{6}=3\Rightarrow x=3\sqrt{6}$ cm.\\
Ta biết rằng thể tích nước tràn ra bằng với thể tích khối chóp $S.ABC$.\\
Thể tích đó là 
$V=\dfrac{1}{3}SO\cdot S_{ABC}=\dfrac{1}{3}\cdot 3\cdot \dfrac{\left(3\sqrt{6}\right)^2\sqrt{3}}{4}=\dfrac{27\sqrt{3}}{2}\approx 23{,}4$ cm$^3$.
}
    \end{ex}
    
    \begin{ex}%Câu 18
% \immini
% {
    Một người công nhân có thể sản xuất với tốc độ là\\
    $q(t)=100+\mathrm{e}^{-0{,}5t}$ đơn vị sản phẩm trong $1$ giờ, với $t$ (giờ) là thời gian tính từ khi bắt đầu làm việc. Biết rằng người công nhân bắt đầu làm việc từ lúc $8$ giờ sáng, hỏi người đó sẽ sản xuất được bao nhiêu đơn vị sản phẩm trong khoảng thời gian từ $9$ giờ sáng đến $11$ giờ trưa (làm tròn đến hàng đơn vị)?
\shortans{201}
% }
% {
%     \includegraphics[width=6cm]{img/HXN-1-18}
% }
\loigiai{
Gọi $Q(t)$ là số đơn vị sản phẩm mà công nhân sản xuất được sau $t$ giờ tính từ lúc $8$ giờ sáng.\\
Ta có $Q'(t)=q(t)=100+{\mathrm{e}^{-0{,}5t}}$.
Số đơn vị sản phẩm người đó sản xuất được từ $9$ giờ sáng ($t=1$) đến $11$ giờ trưa ($t=3$) là\\
$Q(3)-Q(1)=\int\limits_1^3{q(t)\mathrm{\,d}t}=\int\limits_1^3\left(100+{e^{-0.5t}}\right)\mathrm{\,d}t \approx 201$ (đơn vị sản phẩm).
}
\end{ex}

\begin{ex}%Câu 19
\immini
{
    Mảnh đất vườn của nhà anh Điệp có một phần ranh giới cũng là một phần đường cong $(C)\colon y=\dfrac{x+a}{x+b}$, bao quanh nó là sông nước. Với hệ trục tọa độ $Oxy$ thích hợp, đơn vị trên mỗi trục là $10$ mét thì đường cong $(C)$ đi qua điểm $\left(2;3\right)$ và có đường tiệm cận đứng $x=1$. Hàng ngày anh Điệp phải dùng thuyền máy để vận chuyển trái cây từ khu vườn của mình đến hai tuyến đường $\Delta_1\colon 2x+y-4=0$ và $\Delta_2\colon x+2y-2=0$ cho những người lái buôn từ nơi khác đến.
}
{
    \includegraphics[width=6cm]{img/HXN-1-19}
}
Anh Điệp cần xác định một vị trí $ M\left(x_0;y_0\right)$ thuộc khu vườn của mình để tổng các khoảng cách từ vị trí M đó đến hai tuyến đường $\Delta_1,\Delta_2$ là bé nhất. Hỏi khoảng cách từ vị trí được chọn làm gốc tọa độ đến điểm M là bao nhiêu mét (làm tròn đến hàng phần chục)?
\shortans{34,1}
\loigiai{
Đồ thị hàm số có tiệm cận đứng $x=-b=1\Rightarrow b=-1$.\\
Khi đó đồ thị hàm số $y=\dfrac{x+a}{x-1}$ qua $(2;3)\Rightarrow 3=\dfrac{2+a}{2-1}\Rightarrow a=1$; hàm số là $y=\dfrac{x+1}{x-1}$ $(C)$.\\
Gọi $M\left(x_0;\dfrac{x_0+1}{x_0-1}\right)\in (C)$, $x_0>1$. Tổng khoảng cách từ $M$ đến hai đường thẳng $\triangle _1,\triangle _2$ là
\begin{eqnarray*}
    &&d=d\left(M,\triangle_1\right)+d\left(M,\triangle _2\right)=\dfrac{\left| 2x_0+\dfrac{x_0+1}{x_0-1}-4 \right|}{\sqrt{5}}+\dfrac{\left| x_0+2\cdot \dfrac{x_0+1}{x_0-1}-2 \right|}{\sqrt{5}}\\
    &\Leftrightarrow&\sqrt{5}d=\left| \dfrac{2x_0^2-5x_0+5}{x_0-1} \right|+\left| \dfrac{x_0^2-x_0+4}{x_0-1} \right|=\dfrac{2x_0^2-5x_0+5}{x_0-1}+\dfrac{x_0^2-x_0+4}{x_0-1}\\
    &&\left(\text{vì } \heva{& 2x_0^2-5x_0+5>0 \\& x_0-1>0 \\& x_0^2-x_0+4>0 },\,\forall x_0>1\right)
\end{eqnarray*}
Đặt $\sqrt{5}d=\dfrac{3x_0^2-6x_0+9}{x_0-1}=g(x)$ với $x>1$.\\
Ta có: $g'(x)=\dfrac{3x_0^2-6x_0-3}{\left(x_0-1\right)^2}$; $g'(x)=0\Rightarrow 3x_0^2-6x_0-3=0\Rightarrow x_0=1+\sqrt{2}>1$.\\
Ta có $\min \limits_{\left(1;+\infty \right)}\,g(x)=g\left(1+\sqrt{2}\right)=6\sqrt{2}\Rightarrow \sqrt{5}d\ge 6\sqrt{2}\Rightarrow d\ge \dfrac{6\sqrt{10}}{5}$.\\
Dấu đẳng thức xảy ra khi $x_0=1+\sqrt{2} \Rightarrow M\left(1+\sqrt{2};1+\sqrt{2}\right)$.\\
Khoảng cách OM trên thực tế là $10\times \sqrt{\left(1+\sqrt{2}\right)^2+\left(1+\sqrt{2}\right)^2}=10\times \left(1+\sqrt{2}\right)\sqrt{2}\approx 34{,}1$ mét.
}
\end{ex}

\begin{ex}%Câu 20
\immini
{
    Một tên trộm đang cố gắng kéo thùng nữ trang qua một bức tường có độ dày $ BC=1m$; biết rằng tường cao $4$ m và sợi dây được kéo theo đường gấp khúc $ABCD$ có độ dài không đổi bằng $20$ m, đoạn $ BF=0{,}5m$. Trong khi kéo thì tên trộm luôn ghì đầu dây theo một thanh vịn của cầu thang (đầu dây dịch chuyển theo phương $AF$). Biết rằng thanh vịn cầu thang hợp với phương ngang một góc bằng $30^{\circ}$.
}
{
    \includegraphics[width=10cm]{img/HXN-1-20}
}
Khi hai chú cảnh sát xuất hiện thì vị trí $A$ cách F khoảng $6$ m và thùng $D$ tiến về phía $E$ với tốc độ $1$ m/s. Hỏi đầu dây $A$ rời xa điểm $F$ với tốc độ bao nhiêu m/s? (Làm tròn kết quả đến hàng phần trăm).
\shortans{0,95}
\loigiai{
Đặt $DE=x$(m), $AF=y$(m).\\
Ta có $CD=\sqrt{x^2+16}$ và $\widehat{AFB}=180^{\circ }-60^{\circ }=120^{\circ }$.\\
Suy ra $AB=\sqrt{y^2+0{,}5^2-2.0{,}5\cdot y\cos 120^{\circ }}=\sqrt{y^2+0{,}5y+0{,}25}$.\\
Ta có $AB+CD+1=20\Leftrightarrow \sqrt{x^2+16}+\sqrt{y^2+0{,}5y+0{,}25}=19$ \tagEX{*}
Thay $y=6$ vào $(*)$ ta được $\sqrt{x^2+16}+\sqrt{6^2+0{,}5.6+0{,}25}=19\Rightarrow x\approx 12{,}1$  (Lưu vào A).\\
Đạo hàm hai vế của $(*)$ theo biến $t$ ta được:\\
$\dfrac{x}{\sqrt{x^2+16}}\cdot \dfrac{\mathrm{\,d}x}{\mathrm{\,d}t}+\dfrac{2y+0{,}5}{2\sqrt{y^2+0{,}5y+0{,}25}}\cdot \dfrac{dy}{\mathrm{\,d}t}=0$ \tagEX{**}
Thay $y=6$ m; $x=A\approx 12{,}1$ m; $\dfrac{\mathrm{\,d}x}{\mathrm{\,d}t}=-1$ m/s (do $x$ ngày càng giảm theo thời gian $t$) vào $(**)$ ta tính được $\dfrac{dy}{\mathrm{\,d}t}\approx 0{,}95$ m/s hay đầu dây $A$ rời xa điểm $F$ với tốc độ khoảng $0{,}95$ m/s.
}
\end{ex}

\begin{ex}%Câu 21
\immini
{
Trong công trường xây dựng, có một bộ khung sắt hình lập phương như hình vẽ (ta xem nó là hình lập phương dạng $ 2\times 2\times 2$). Người ta nhìn thấy một con kiến và một con gián xuất phát cùng lúc trên hai đỉnh thuộc đường chéo lớn của khung sắt hình lập phương và di chuyển trên các cạnh của mỗi hình vuông nhỏ. Con kiến cần đến vị trí mà con gián xuất phát và ngược lại, mỗi con ngày càng di chuyển xa vị trí mà nó xuất phát. Tính xác suất để hai con côn trùng này gặp nhau biết rằng vận tốc của gián bằng 4 cm/s, vận tốc của kiến là 2 cm/s. Kết quả được làm tròn đến hàng phần trăm.
\shortans{0,27}
}
{
    \includegraphics[width=6cm]{img/HXN-1-21}
}
\loigiai{
\immini
{
    Ta xem mỗi bước di chuyển của mỗi con là 1 đơn vị (ứng với cạnh hình vuông nhỏ).\\
Để đi hết hành trình của mình thì gián cần đi xuống $2$ đơn vị, sang trái $2$ đơn vị và đi dọc $2$ đơn vị (có tất cả là 6 bước di chuyển) nên số cách đi của gián là $\mathrm{C}_6^2\mathrm{C}_4^2$; hoàn toàn tương tự kiến cũng có số cách đi là $\mathrm{C}_6^2\mathrm{C}_4^2$.\\
Gọi $\Omega $ là không gian mẫu thì $n\left(\Omega \right)=\left(C_6^2C_4^2\right)^2$.\\
Vận tốc của gián gấp đôi vận tốc của kiến nên nếu hai con gặp nhau thì tại vị trí chúng gặp gián đã di chuyển $4$ bước, kiến di chuyển 2 bước. Vị trí hai con gặp nhau (nếu có) được đánh dấu ở $6$ vị trí trên hình vẽ.
}
{
    \includegraphics[width=6cm]{img/HXN-1-21-LG}
}
\begin{itemize}
    \item Tại vị trí $A$: Gián có $2$ lần di chuyển sang trái, $2$ lần di chuyển dọc; sau đó đi từ $A$ đến đích thì nó cần $2$ lần đi xuống. Số cách đi của gián là $\mathrm{C}_4^2\mathrm{C}_2^2\mathrm{C}_2^2$.  Hành trình của kiến cũng tương tự mà theo chiều ngược lại nên kiến có $\mathrm{C}_4^2\mathrm{C}_2^2\mathrm{C}_2^2$ cách đi. Số cách đi hai con là $\left(C_4^2C_2^2C_2^2\right)^2$.\\
    Tại các vị trí $A$, $C$, $E$ thì số cách đi mỗi con là như nhau. 
    \item Tại vị trí $B$: Số cách đi của hai con là  $\left(\mathrm{C}_4^1\mathrm{C}_3^1\mathrm{C}_2^2\mathrm{C}_2^1\right)^2$.\\
    Tại các vị trí $B$, $D$, $F$ thì số cách đi mỗi con là như nhau. 
\end{itemize}
Gọi $X$ là biến cố hai con côn trùng gặp nhau trên đường đi, ta có\\ $P(X)=\dfrac{3\left(\mathrm{C}_4^2\mathrm{C}_2^2\mathrm{C}_2^2\right)^2+3\left(\mathrm{C}_4^1\mathrm{C}_3^1\mathrm{C}_2^2\mathrm{C}_2^1\right)^2}{\left(\mathrm{C}_6^2\mathrm{C}_4^2\right)^2}=\dfrac{17}{75}\approx0{,}27$.
}
\end{ex}

\begin{ex}%Câu 22
Trong không gian $Oxyz$, cho mặt cầu $\left(S_1\right)$ có tâm $ I\left(2;1;1\right)$, bán kính bằng $ 4$ và mặt cầu $\left(S_2\right)$ có tâm $J\left(2;1;5\right)$, bán kính bằng $ 2$. Gọi $(P)$ là mặt phẳng thay đổi tiếp xúc với hai mặt cầu $\left(S_1\right),\left(S_2\right)$ và đặt $T_1,T_2$ lần lượt là giá trị nhỏ nhất, giá trị lớn nhất của khoảng cách từ điểm $ O$ đến $(P)$. Tìm giá trị $ T_1^2+T_2^2$.
\shortans{48}
\loigiai{
\immini
{
    Ta có $IJ=4<R_1+R_2$ (với $R_1=4$, $R_2=2$)  nên hai mặt cầu $\left(S_1\right)$ và $\left(S_2\right)$ cắt nhau.\\
Gọi $M$ là giao điểm của $IJ$ và $(P)$.\\
Ta có $\dfrac{MJ}{MI}=\dfrac{R_2}{R_1}=\dfrac{1}{2}\Rightarrow J$ là trung điểm của $MI$.\\
Suy ra $M(2;1;9)$.
}
{
\includegraphics[width=6cm]{img/HXN-1-22-LG}
}
Gọi $\vec{n}=(a;b;c)$ là vectơ pháp tuyến của $(P)$ với $a^2+b^2+c^2>0$.\\
Phương trình $(P)\colon a(x-2)+b(y-1)+c(z-9)=0$ hay $ax+by+cz-2a-b-9c=0$.\\
Ta có $(P)$ tiếp xúc $\left(S_1\right) \Leftrightarrow d\left(I,(P)\right)=4\Leftrightarrow \dfrac{|8c|}{\sqrt{a^2+b^2+c^2}}=4\Leftrightarrow \dfrac{|2c|}{\sqrt{a^2+b^2+c^2}}=1$.\\
Dễ thấy $c\ne 0$ nên ta có thể chọn $c=1\Rightarrow a^2+b^2=3$.\\
Khi đó $d\left(O,(P)\right)=\dfrac{\left| -2a-b-9c \right|}{\sqrt{a^2+b^2+c^2}}=\dfrac{|2a+b+9|}{2}$\tagEX{1}
Theo bất đẳng thức Cauchy Schwarz thì $|2a+b|\le \sqrt{\left(2^2+1^2\right)\left(a^2+b^2\right)}=\sqrt{5.3}=\sqrt{15}$.\\
Dấu đẳng thức xảy ra khi và chỉ khi $\dfrac{a}{2}=\dfrac{b}{1}$.\\
Do đó 
\begin{align}
    &-\sqrt{15}\le 2a+b\le \sqrt{15} \notag\\
    \Rightarrow& \underbrace{9-\sqrt{15}}_{+}\le 2a+b+9\le 9+\sqrt{15}\notag\\
    \Rightarrow& \underbrace{9-\sqrt{15}}_{+}\le |2a+b+9|\le 9+\sqrt{15} \tag{2}
\end{align}
Từ $(1)$ và $(2)$ suy ra $\dfrac{9-\sqrt{15}}{2}\le d\left(O,(P)\right)=\dfrac{|2a+b+9|}{2}\le \dfrac{9+\sqrt{15}}{2}$.\\
Do đó $T_1=\dfrac{9-\sqrt{15}}{2}$; $T_2=\dfrac{9+\sqrt{15}}{2}$ và $T_1^2+T_2^2=48$.
}
\end{ex}
\Closesolutionfile{ans}
\inputansbox{6,4,3}{ans/ans-HXN-\sode-T,ans/ans-HXN-\sode-TF,ans/ans-HXN-\sode-SA}
% %%%%%%%%%%%%%%%%%%%- HXN
\def\sode{2}
\begin{name}
	{\tenchude}
	{\tendethi}
	{\tentruong}
	{\thoigian}
\end{name}
\caulc
\Opensolutionfile{ans}[ans/ans-HXN-\sode-T]
%Câu hỏi
\begin{ex}%Câu 1
    Cho hàm số $y=f(x)$ có bảng biến thiên như sau:\\
   \centerline{\begin{tikzpicture}[>=stealth]
           \tkzTabInit[nocadre=false,lgt=1.2,espcl=2.5,deltacl=0.5]{$x$/.7 ,$f'(x)$/.7,$f(x)$/2}
           {$-\infty$ , $-3$ , $0$ , $3$ , $+\infty$}
           \tkzTabLine{ , - , $0$ , + , $0$ , - , $0$ , + , }
           \tkzTabVar{+/$+\infty$ , -/$-1$, +/$1$ , -/$-1$ , +/$+\infty$}
   \end{tikzpicture}}
    Hàm số đã cho đồng biến trên khoảng nào dưới đây?
    \choice
    {$\left(-3;3\right)$}
    {\True $\left(-3;0\right)$}
    {$\left(0;3\right)$}
    {$\left(-\infty;-3\right)$}
\end{ex}

\begin{ex}%Câu 2
    Trong không gian $Oxyz$ , cho hai điểm $A\left(1;1;-2\right)$ và $B\left(2;2;1\right)$. Vectơ $\overrightarrow{AB}$ có tọa độ là
    \choice
    {$\left(-1;-1;-3\right)$}
    {$\left(3;1;1\right)$}
    {\True $\left(1;1;3\right)$}
    {$\left(3;3;-1\right)$}
\end{ex}

\begin{ex}%Câu 3
    Tìm tập xác định của hàm số $y=\log_2\left(x-3\right)$.
    \choice
    {$\mathscr{D}=\left(-\infty;3\right)$}
    {$\mathscr{D}=\mathbb{R}$}
    {\True $\mathscr{D}=\left(3;+\infty\right)$}
    {$\mathscr{D}=\left[3;+\infty\right)$}
\end{ex}

\begin{ex}%Câu 4
    Xác định số hạng đầu $u_1$ và công sai $d$ của cấp số cộng $\left(u_n\right)$ có $u_9=5u_2$ và $u_{13}=2u_6+5$.
    \choice
    {\True $u_1=3$ và $d=4$}
    {$u_1=3$ và $d=5$}
    {$u_1=4$ và $d=5$}
    {$u_1=4$ và $d=3$}
\end{ex}

\begin{ex}%Câu 5
    Họ nguyên hàm của hàm số $f(x)=3x^2+\sin x$ là 
    \choice
    {$x^3+\cos x+C$}
    {$x^3+\sin x+C$}
    {\True $x^3-\cos x+C$}
    {$3x^3-\sin x+C$}
\end{ex}

\begin{ex}%Câu 6
    Cho hình hộp $ABCD.A'B'C'D'$. Vectơ $\vec{v}=\overrightarrow{B'A'}+\overrightarrow{B'C'}+\overrightarrow{B'B}$ bằng vectơ nào dưới đây?
    \choice
    {$\overrightarrow{DB'}$}
    {$\overrightarrow{B'D'}$}
    {$\overrightarrow{BD'}$}
    {\True $\overrightarrow{B'D}$}
\end{ex}

\begin{ex}%Câu 7
    Người ta thống kê khối lượng của $80$ quả măng cụt (đơn vị: gam) và thu được mẫu số liệu sau:\\
    \centerline{\begin{tabular}{|c|c|c|c|c|c|}
            \hline
            Khối lượng (gam) & $\left[80;82\right)$ & $\left[82;84\right)$ & $\left[84;86\right)$ & $\left[86;88\right)$ & $\left[88;90\right)$\\
            \hline
            Số quả & $17$ & $20$ & $25$ & $16$ & $12$\\
            \hline
    \end{tabular}}\\
    Khoảng biến thiên của mẫu số liệu ghép nhóm trên là
    \choice
    {$11$ gam}
    {$12$ gam}
    {\True $10$ gam}
    {$20$ gam}
\end{ex}

\begin{ex}%[2D4N1-2]%[Tổ 19 - Đợt 17 - Chương 4 - Bài 1 - CD - Đề 2]%[Bình]
    Tìm nguyên hàm $F(x)$ của hàm số $ f(x)=3{x^2}+1$, biết $F(1)=3$ là
    \choice
    { ${x^3}+3$}
    { $\dfrac{{x^3}}{3}+x+3$}
    { ${x^3}+x$}
    {\True ${x^3}+x+1$}
    \loigiai
    {
        $\int (3{x^2}+1)\mathrm{\,d}x={x^3}+x+C=F(x)$.\\
        Ta có $F(1)=3\Rightarrow 1+1+C=3\Leftrightarrow C=1$.\\
        Vậy $F(x)={x^3}+x+1$.
    }
\end{ex}

\begin{ex}%Câu 9
    Mỗi ngày bác Mạnh đều đi bộ để rèn luyện sức khỏe. Quãng đường đi bộ mỗi ngày của bác trong 20 ngày được thống kê lại ở bảng sau:\\
    \centerline{\begin{tabular}{|c|c|c|c|c|c|}
            \hline
            Quãng đường & $\left[2,7;3,0\right)$ & $\left[3,0;3,3\right)$ & $\left[3,3;3,6\right)$ & $\left[3,6;3,9\right)$ & $\left[3,9;4,2\right)$\\
            \hline
            Số ngày & 3 & 6 & 5 & 4 & 2\\
            \hline
    \end{tabular}}\\
    Phương sai của mẫu số liệu ghép nhóm gần nhất với giá trị nào sau đây?
    \choice
    {$0{,}19$}
    {$1{,}26$}
    {\True $0{,}13$}
    {$0{,}26$}
\end{ex}

\begin{ex}%Câu 10
    Cho hàm số $y=f(x)$ có đạo hàm $f'(x)=(x-2)(x+1)$, $\forall x\in\mathbb{R}$. Mệnh đề nào dưới đây đúng?
    \choice
    {Hàm số đã cho đồng biến trên $(-1;+\infty)$}
    {Hàm số đã cho nghịch biến trên $(2;+\infty)$}
    {\True Hàm số đã cho nghịch biến trên $(-1;2)$}
    {Hàm số đã cho đồng biến trên $(-1;2)$}
\end{ex}

\begin{ex}%Câu 11
    Trong không gian $Oxyz$ , mặt phẳng đi qua tâm của mặt cầu $\left(x-1\right)^2+\left(y+2\right)^2+z^2=12$ và song song với mặt phẳng $\left(Ox\text{z}\right)$ có phương trình là:
    \choice
    {$y+1=0$}
    {$y-2=0$}
    {\True $y+2=0$}
    {$x+z-1=0$}
\end{ex}

\begin{ex}%Câu 12
    Biết đồ thị $(C)$ của hàm số $y=\dfrac{x^2-4x+5}{x-1}$ có hai điểm cực trị. Đường thẳng đi qua hai điểm cực trị của đồ thị $(C)$ cắt trục hoành tại điểm $M$ có hoành độ $x_M$ bằng
    \choice
    {\True $x_M=2$}
    {$x_M=1-\sqrt{2}$}
    {$x_M=1$}
    {$x_M=1+\sqrt{2}$}
        \end{ex}
\Closesolutionfile{ans}
\cauds
\Opensolutionfile{ans}[ans/ans-HXN-\sode-TF]
%Câu hỏi

\begin{ex}%Câu 13
\immini
{
    Một cái bể nước có dạng khối chóp tứ giác đều ngược với cạnh đáy bằng $3\sqrt{2}$dm và chiều cao bằng $6$dm (tham khảo hình vẽ bên – các kích thước được nêu ra là phần bên trong hình). Nước được bơm vào bể với tốc độ không đổi là $2$ lít/phút và ban đầu bể không chứa nước (các kết quả bên dưới được làm tròn đến hai chữ số thập phân sau dấu phẩy).
    \choiceTF
    {\True Bể nước được bơm đầy sau $18$ phút}
    {\True Tốc độ dâng lên của nước là $0{,}23$ dm/phút khi thể tích nước trong bể bằng $\dfrac{1}{3}$ thể tích của bể}
    {\True Khi mực nước cách miệng bể $0{,}5$ dm, người ta ngừng bơm và bắt đầu xả ra với ước lượng tốc độ giảm chiều cao của mực nước trong bể theo thời gian $t$ (phút) được mô hình hóa bởi hàm số: $h'(t)=\dfrac{1}{350}t-\dfrac{193}{700}$ (dm/phút). Sau $5$ phút, thể tích nước trong bể là $11,97$ dm$^3$}
    {Cùng với dữ kiện của  thì sau $23{,}59$ phút nước trong bể vừa được xả hết}
}
{
    \includegraphics[width=7cm]{img/HXN-2-13}
}
\loigiai{
    \begin{itemchoice}
        \itemch 
        Thể tích chậu nước hình chóp tứ giác đều là $V_{\text{chậu}}=\dfrac{1}{3}\cdot \left(3\sqrt{2}\right)^2\cdot 6=36\,dm^3=36$ lít.\\
        Thời gian bơm nước đầy bể là $\dfrac{36}{2}=18$ phút.
        \itemch Gọi $V$(t), $h$(t) lần lượt là thể tích và chiều cao của nước sau $t$ phút.\\
        Ta có $\dfrac{V(t)}{V_{\text{chậu}}}=\left(\dfrac{h(t)}{6}\right)^3\Rightarrow V(t)=\left(\dfrac{h(t)}{6}\right)^3\cdot 36$ hay $V(t)=\dfrac{(h(t))^3}{6}$\tagEX{*}
        Đạo hàm hai vế của $(*)$ theo biến $t$ ta được: $\dfrac{dV(t)}{\mathrm{\,d}t}=\dfrac{(h(t))^2}{2}\cdot \dfrac{dh(t)}{\mathrm{\,d}t}$\tagEX{**}
        Thời điểm thể tích nước bằng $\dfrac{1}{3}$ thể tích chậu thì \\
        \centerline{
        $\left(\dfrac{h(t)}{6}\right)^3=\dfrac{1}{3}\Rightarrow \dfrac{h(t)}{6}=\sqrt[3]{\dfrac{1}{3}}\Rightarrow h(t)=6\cdot \sqrt[3]{\dfrac{1}{3}}\approx 4{,}16\,dm$.
        }
        Thay $\dfrac{dV(t)}{\mathrm{\,d}t}=2$ lít/phút $=2$ dm/phút; $h(t)\approx 4{,}16$dm vào (**), thì:$\dfrac{dh(t)}{\mathrm{\,d}t}\approx 0{,}23$ dm/phút.
        \itemch Mực nước cách miệng bể $0{,}5$ dm nên chiều cao ban đầu bằng $5{,}5\,dm$.\\
        Chiều cao của nước trong chậu sau $5$ phút là\\
        \centerline{$h(5)=5{,}5+\int\limits_0^5{\left(\dfrac{1}{350}t-\dfrac{193}{700}\right)\mathrm{\,d}t}=\dfrac{291}{70}\approx 4{,}16$dm.}
        Thể tích nước còn lại trong bể là $V(5)=\left(\dfrac{h(5)}{6}\right)^3.36\approx 11{,}97\,dm^3$.
        \itemch Thời gian để mực nước trong chậu đang là $5{,}5$ dm trở về $0$ dm thỏa mãn phương trình\\
        \centerline{$5{,}5+\displaystyle\int\limits_0^t{\left(\dfrac{1}{350}t-\dfrac{193}{700}\right)\mathrm{\,d}t}=0\Rightarrow t\approx 22{,}59$ phút.}
        Với khoảng thời gian $22{,}59$ phút thì nước trong bể vừa được xả hết.
    \end{itemchoice}
}
\end{ex}

\begin{ex}%Câu 14
Có hai tên cướp vừa lấy được một chiếc ca nô ở vị trí $A$ thuộc bờ sông, chúng liền cho ca nô chạy theo phương hợp với bờ sông một góc $60^\circ$ với vận tốc $v=2t$ (mét/giây), trong đó $t$ (giây) là thời gian kể từ khi xuất phát. Sau $21$ giây, ca nô đến vị trí $B$ và chúng quyết định chuyển hướng cho ca nô chuyển động thẳng đều theo phương song song với bờ sông, tầm nửa phút sau thì ca nô đến $C$ (tham khảo hình vẽ).\\
\centerline{
    \includegraphics[width=8cm]{img/HXN-9-14}
}
    \choiceTF
    {\True Vị trí $B$ mà ca nô bọn cướp chuyển hướng cách bờ sông khoảng $382$ m (làm tròn đến hàng đơn vị)}
    {Khoảng cách $A$, $C$ tính theo đường chim bay bằng $1522$ m (làm tròn đến hàng đơn vị)}
    {\True Nếu các chiến sĩ công an khởi động ca nô và đi thẳng từ $D$ đến $C$ với vận tốc được tăng thêm $3$ m sau mỗi giây thì sau $21$ giây sẽ đến vị trí $D$ (làm tròn đến hàng đơn vị của giây)}
    {Trên thực tế các chiến sĩ đã chọn giải pháp cho ca nô khởi động và di chuyển vuông góc với bờ với gia tốc $a$ dương, cùng lúc đó bọn cướp từ vị trí $D$ tiến thẳng về phía trước (giữ nguyên hướng đi và tốc độ), hai bên giáp mặt nhau khi $a=4{,}78$ m/s$^2$ (làm tròn đến hàng phần trăm)}
    \loigiai{
        \begin{center}
            \begin{tikzpicture}[declare function={d=6;b=2;c=3;},thick]
                \path
                (0,0) coordinate (A)
                (-60:b) coordinate (B)
                ($(B)+(0:c)$) coordinate (C)
                (0:d) coordinate (D)
                ($(A)!(B)!(D)$) coordinate (H)
                ($(A)!(C)!(D)$) coordinate (K)
                ($(B)!(D)!(C)$) coordinate (E)
                pic[draw,angle radius=5mm,"$60^\circ$",angle eccentricity=1.7]{angle=B--A--H}
                ;
                \foreach \x/\y/\z in {B/H/K,C/K/D,C/E/D}\draw pic[draw,angle radius =2mm] {right angle = \x--\y--\z};
                \draw[dashed] (B)--(H) (C)--(K) (C)--(E)--(D);
                \draw (A)--(B)--(C) (D)--(A);
                \draw[red] (A)--(C)--(D);
                \foreach \x/\g in {A/180,B/-90,C/-90,D/0,E/-90,H/90,K/90}\draw[fill=white] (\x) circle (1pt)+(\g:3mm) node{$\x$};
            \end{tikzpicture}
        \end{center}
    \begin{itemchoice}
        \itemch Sau $21$ giây, ca nô bọn cướp đi được $AB=\int\limits_0^{21}{2t\mathrm{d}t}=441m$.\\
        Gọi $H$ là hình chiếu vuông góc của $B$ trên bờ sông, khi đó $BH=AB\sin 60^\circ=\dfrac{441\sqrt{3}}{2}\approx 382m$.
        \itemch Vận tốc của ca nô bọn cướp tại $B$ là $v_B=2\times 21=42$ m/s.\\
        Khoảng cách hai vị trí $B$, $C$ là $BC=42\times 30=1260m$.\\
        Khoảng cách hai vị trí $A$, $H$ là $AH=AB\cos 60^\circ=220{,}5m$.\\
        Gọi $K$ là hình chiếu vuông góc của $C$ trên bờ sông thì $HK=BC=1260m$.\\
        Do đó $AK=AH+HK=1480{,}5m$; $CK=BH=\dfrac{441\sqrt{3}}{2}m$ và $AC=\sqrt{AK^2+CK^2}\approx 1529 m$.
        \itemch Ta có $DK=2000-\left(220{,}5+1260\right)=519{,}5m$; suy ra $CD=\sqrt{CK^2+DK^2}\approx 644{,}8m$.\\
        Thời gian để các chiến sĩ đi được từ $D$ đến $C$ thỏa $\displaystyle\int\limits_0^t{3t\mathrm{d}t}=\sqrt{415741}\Rightarrow t\approx 21s$.
        \itemch Gọi $E$ là vị trí hai bên giáp mặt nhau (nếu có) thì tam giác $CDE$ vuông tại $E$.\\
        Khi đó $CE=DK=519{,}5m$ và $DE=\dfrac{441\sqrt{3}}{2}m$.\\
        Thời gian để ca nô bọn cướp đi từ $C$ đến $E$ là $\dfrac{CE}{42}=\dfrac{1039}{84}\approx 12{,}37s$.\\
        Gia tốc $a$ thỏa mãn $\displaystyle\int\limits_0^{\tfrac{1039}{84}}{at \mathrm{d}t}=\dfrac{441\sqrt{3}}{2}\Rightarrow 4{,}99m/s^2$.
    \end{itemchoice}
    }
\end{ex}

\begin{ex}%Câu 15
\immini
{
    Sao Thủy gần như không có khí quyển thật sự như Trái Đất hay sao Kim. Tuy nhiên, nó có một lớp khí rất mỏng gọi là exosphere – tức là thượng quyển loãng, gồm các hạt khí cực kỳ thưa thớt như hydro, heli, oxy, natri...Trong không gian Oxyz, đơn vị trên mỗi trục là nghìn km, vùng thượng quyển loãng của sao Thủy được mô hình hóa bởi phương trình mặt cầu $x^2+y^2+z^2-2x-4y-4=0$. Các nhà khoa học không gian đang quan sát các tiểu hành tinh ở các vị trí có tọa độ $A\left(4;2;4\right)$, $B\left(1;4;2\right)$ và xem xét sự di chuyển của chúng. Nếu tiểu hành tinh nằm trong vùng thượng quyển loãng thì nó sẽ bị hút xuống bề mặt sao Thủy.
}
{
    \includegraphics[width=5cm]{img/HXN-2-15}
}
\choiceTF
    {\True Vùng thượng quyển loãng sao Thủy có tâm $\left(1;2;0\right)$, bán kính bằng $3$}
    {Hai tiểu hành tinh ở các vị trí $A$, $B$ sẽ bị hút xuống bề mặt sao Thủy}
    {Các nhà quan sát cho rằng có một sao chổi mang tên Haxen di chuyển theo quỹ đạo đường thẳng với vận tốc $51{,}5$ km/s; khoảng cách ngắn nhất từ tâm sao Thủy đến sao chổi bằng $\dfrac{\sqrt{871}}{10}$ nghìn km. Thời gian sao chổi đi trong vùng thượng quyển loãng của sao Thủy bằng $20$ giây (làm tròn đến hàng đơn vị của giây)}
    {\True Sao chổi Haxen di chuyển theo phương vectơ $\vec{u}=\left(0;5;2\right)$ . Giả sử $M$, $N$ là điểm đầu và điểm cuối mà sao chổi này đi qua thuộc vùng thượng quyển loãng của sao Thủy. Giá trị nhỏ nhất của tổng $AM+BN$ bằng $3970$ km (làm tròn đến hàng đơn vị của km)}
    \loigiai{
        \begin{itemchoice}
            \itemch Vùng thượng quyển loãng sao Thủy có tâm $I(1;2;0)$, bán kính $R=\sqrt{1^2+2^2+0^2+4}=3$.
            \itemch Ta có $IA=\sqrt{(4-1)^2+(2-2)^2+(4-0)^2}=5$; $IB=\sqrt{(1-1)^2+(4-2)^2+(2-0)^2}=2\sqrt{2}$.\\
            Vì $IA>R$; $IB<R$ nên tiểu hành tinh $A$ nằm ngoài vùng thượng quyển loãng, còn tiểu hành tinh $B$ thì nằm tròn vùng thượng quyển loãng của sao Thủy và nó sẽ bị hút xuống bề mặt sao Thủy.
            \itemch Gọi $d$ là quỹ đạo đường thẳng của sao chổi và $H$ là hình chiếu vuông góc của tâm $I$ trên $d$.\\
            Gọi $M$, $N$ là điểm đầu và điểm cuối của sao chổi trong vùng thượng quyển loãng của sao Thủy.\\
            Ta có $IH=\dfrac{\sqrt{871}}{10}$; suy ra $MN=2MH=2\sqrt{R^2-IH^2}=2\sqrt{9-\dfrac{871}{100}}=\dfrac{\sqrt{29}}{5}$ (nghìn km).
            Thời gian sao chổi di chuyển trong vùng thượng quyển loãng: $\dfrac{\sqrt{29}}{5}\times 1000\colon 51{,}5\approx 21$ (giây).
            \immini
            {
                \itemch Đặt $\vec{MN}=(0;5k;2k)$;\\
                ta có $MN^2=25k^2+4k^2=\dfrac{29}{25}\Rightarrow k=\dfrac{1}{5}$;\\
                suy ra $\vec{MN}=\left(0;1;\dfrac{2}{5}\right)$.\\
            Chọn $C$ sao cho $AMNC$ là hình bình hành;\\
            suy ra $C\left(4;3;\dfrac{22}{5}\right)$.\\
            Khi đó $AM=CN$ và $AM+BN=CN+BN\ge BC=\dfrac{\sqrt{394}}{5}\approx 3{,}97$.\\
            Vậy giá trị nhỏ nhất của biểu thức $AM+BN$ là khoảng $3970$ km.
            }
            {
                \includegraphics[width=4cm]{img/HXN-2-15-LG}
            }
        \end{itemchoice}
    }
\end{ex}

\begin{ex}%Câu 16
\immini
{
    Có hai hộp bóng $A$ và $B$ chỉ đựng các quả bóng đỏ và trắng, trong đó hộp $B$ đựng $4$ quả bóng đỏ và $5$ quả bóng trắng; tổng số bóng hai hộp không qua $20$. Xét hai phép thử ngẫu nhiên sau:\\
Phép thử thứ nhất: Lấy ngẫu nhiên $1$ quả bóng từ hộp $A$ bỏ vào hộp $B$ rồi lấy ngẫu nhiên $1$ quả bóng từ hộp $B$. Bằng thực nghiệm người ta biết được rằng khả năng lấy được quả bóng đỏ từ hộp thứ hai bằng $\dfrac{33}{70}$.\\
Phép thử thứ hai: Lấy ngẫu nhiên $2$ quả bóng từ hộp $A$ bỏ vào hộp B. Sau đó tiếp tục lấy ngẫu nhiên $2$ quả bóng từ hộp $B$.
}
{
\includegraphics[width=5cm]{img/HXN-2-16}
}
\choiceTF
    {Trong phép thử thứ nhất, nếu lấy được quả bóng đỏ từ hộp $A$ bỏ sang hộp $B$ thì xác suất lấy được quả bóng trắng từ hộp $B$ bằng $0{,}4$}
    {Hộp thứ nhất đựng $4$ quả bóng đỏ và $3$ quả bóng trắng
    }
    {Xác suất để lấy được $2$ quả bóng đỏ từ hộp $B$ bằng $\dfrac{166}{1155}$}
    {Nếu biết rằng hai quả bóng lấy được từ hộp $B$ cùng có màu đỏ, xác suất để có ít nhất $1$ quả là từ hộp $A$ chuyển sang bằng $\dfrac{67}{128}$}
    \loigiai{
        \begin{itemchoice}
            \itemch Trong phép thử thứ nhất, nếu lấy được quả bóng đỏ từ hộp $A$ bỏ sang hộp $B$ thì khi đó hộp $B$ có $5$ quả bóng đỏ, $5$ quả bóng trắng. Do đó xác suất lấy được quả bóng trắng từ hộp $B$ bằng $0{,}5$.
            \itemch Trong phép thử thứ nhất, gọi các biến cố $X$: \lq\lq Lấy được quả bóng đỏ từ hộp $A$\rq\rq và $Y$: \lq\lq Lấy được quả bóng đỏ từ hộp thứ hai\rq\rq. Đặt $P(X)=x\in (0;1)$; suy ra $P\left({\bar{X}}\right)=1-x$.
            Ta có sơ đồ hình cây sau đây:\\
            \centerline{
                \includegraphics[width=5cm]{img/HXN-2-16-LG-a}
            }
            Theo giả thiết ta có $P(Y)=0{,}5x+0{,}6(1-x)=\dfrac{33}{70}\Rightarrow x=\dfrac{5}{7}$.\\
            Vì tổng số bóng hai hộp bóng không quá $20$ mà hộp $B$ có $9$ quả bóng nên hộp $A$ có không quá $11$ quả bóng, mà xác suất để lấy được quả bóng đỏ bằng $\dfrac{5}{7}$ nên hộp A có 5 quả bóng đỏ và 2 quả bóng trắng.
            \itemch Trong phép thử thứ hai, ta gọi 2Đ là biến cố \lq\lq Lấy đúng $2$ quả bóng đỏ từ hộp $B$\rq\rq.\\
            Với thông tin có được, ta có sơ đồ hình cây cho phép thử ngẫu nhiên thứ hai:\\
            \centerline{\includegraphics[width=5cm]{img/HXN-2-16-LG-b}}
            Từ đó suy ra $P(2\text{Đ})=\dfrac{\mathrm{C}_5^2}{\mathrm{C}_7^2}\cdot \dfrac{\mathrm{C}_6^2}{\mathrm{C}_{11}^2}+\dfrac{\mathrm{C}_5^1\mathrm{C}_2^1}{\mathrm{C}_7^2}\cdot \dfrac{\mathrm{C}_5^2}{\mathrm{C}_{11}^2}+\dfrac{\mathrm{C}_2^2}{\mathrm{C}_7^2}\cdot \dfrac{\mathrm{C}_4^2}{\mathrm{C}_{11}^2}=\dfrac{256}{1\,155}$.
            \itemch Gọi $X$ là biến cố lấy được ít nhất 1 quả là từ hộp $A$ chuyển sang, ta có\\
            $P\left(X|2\text{Đ}\right)=\dfrac{\dfrac{\mathrm{C}_5^2}{\mathrm{C}_7^2}\cdot \left(\dfrac{2{,}4}{\mathrm{C}_{11}^2}+\dfrac{1}{\mathrm{C}_{11}^2}\right)+\dfrac{5{,}2}{\mathrm{C}_7^2}\cdot \dfrac{1{,}4}{\mathrm{C}_{11}^2}}{P(2\text{Đ})}=\dfrac{65}{128}$.
        \end{itemchoice}
    }
\end{ex}
\Closesolutionfile{ans}
\caukq
\Opensolutionfile{ans}[ans/ans-HXN-\sode-SA]
%Câu hỏi

\begin{ex}%Câu 17
\immini
{
    Từ một khối gỗ hình lập phương có cạnh bằng $5$ dm, người thợ mộc chỉ cần đến hai lát cắt là có thể tạo ra một khối gỗ có dạng hình chóp $S.ABCD$ với đáy $ABCD$ là hình vuông và $SA=AB=5$ dm. Người thợ cần tạo ra một vật để trang trí theo yêu cầu của khách hàng, anh đã chọn $M$ là trung điểm $SB$, $N$ thuộc cạnh $SD$ sao cho $SN=2ND$; sau đó anh ta tiếp tục thực hiện các lát cắt để có được vật thể hình tứ diện $ACMN$, thể tích vật thể sau cùng mà người thợ mộc làm ra là bao nhiêu dm$^3$ (làm tròn đến hàng phần chục)?
\shortans{10,4}
}
{
    \includegraphics[width=5cm]{img/HXN-2-17}
}
\loigiai{
    Gọi $O$ là giao điểm của $AC$ và $BD$ trong mặt phẳng đáy.\\
    \centerline{
        \includegraphics[width=5cm]{img/HXN-2-17-LG}
    }
    Ta có $V_{S.ABCD}=\dfrac{1}{3}SA\cdot S_{ABCD}=\dfrac{5^3}{3}=\dfrac{125}{3}$.\\
    Vì $OM$ là đường trung bình tam giác $SBD$ nên $OM\parallel SD \Rightarrow SD\parallel(AMC)$.\\
    Do đó $d\left(N,(AMC)\right)=d\left(D,(AMC)\right)=d\left(B,(AMC)\right)$.\\
    Suy ra $V_{ACMN}=V_{N.MAC}=V_{D.MAC}=V_{B.MAC}=V_{M.ABC}$.\\
    Ta lại có $\dfrac{V_{M.ABC}}{V_{S.ABCD}}=\dfrac{d\left(M,(ABCD)\right)\cdot S_{ABC}}{d\left(S,(ABCD)\right)\cdot S_{ABCD}}=\dfrac{\dfrac{1}{2}d\left(S,(ABCD)\right)\cdot \dfrac{1}{2}\cdot S_{ABCD}}{d\left(S,(ABCD)\right)\cdot S_{ABCD}}=\dfrac{1}{4}$.\\
    $\Rightarrow V_{M.ABC}=\dfrac{1}{4}V_{S.ABCD}=\dfrac{1}{4}\cdot \dfrac{125}{3}=\dfrac{125}{12}\approx10{,}4$ dm$^3$.
}
    \end{ex}
    
    \begin{ex}%Câu 18
\immini
{
    Jack có một chiếc điện thoại thông minh đã được sạc đầy pin. Nếu Jack không sử dụng điện thoại một phút nào thì máy sẽ hết pin sau $96$ tiếng; còn nếu anh ấy sử dụng điện thoại liên tục thì máy sẽ hết pin sau $8$ tiếng. Biết Jack đã không sử dụng chiếc smartphone trong suốt $36$ tiếng, sau đó lại dùng nó $90$ phút liên tục. Hỏi Jack còn dùng điện thoại được bao nhiêu phút nữa trước khi máy hết pin?
\shortans{210}
}
{
    \includegraphics[width=5cm]{img/HXN-2-18}
}
\loigiai{
    Ta chuẩn hóa tổng thời lượng pin điện thoại ban đầu là $1$.
    \begin{itemize}
        \item Nếu Jack không sử dụng điện thoại thì sau $96$ tiếng smartphone mới hết pin.\\
        Suy ra sau mỗi tiếng không sử dụng, thời lượng pin sẽ giảm đi $\dfrac{1}{96}$.
        \item Nếu Jack dùng điện thoại liên tục thì sau $8$ tiếng, smartphone sẽ hết pin.\\
        Suy ra sau mỗi tiếng sử dụng, thời lượng pin sẽ giảm đi $\dfrac{1}{8}$.
        \item Sau khi Jack không sử dụng điện thoại trong suốt 36 tiếng, thời lượng pin giảm $36\times \dfrac{1}{96}=\dfrac{3}{8}$.\\
        Thời lượng pin còn lại là: $1-\dfrac{3}{8}=\dfrac{5}{8}$.
        \item Sau khi Jack sử dụng điện thoại liên tục trong $90$ phút, tức $\dfrac{3}{2}$ tiếng, thời lượng pin tiếp tục giảm đi: $\dfrac{3}{2}\times \dfrac{1}{8}=\dfrac{3}{16}$.\\
        Thời lượng pin còn lại là $\dfrac{5}{8}-\dfrac{3}{16}=\dfrac{7}{16}$.
    \end{itemize}
    Vậy trước khi điện thoại hết pin, Jack còn có thể sử dụng $\dfrac{7}{16}: \dfrac{1}{8}=3{,}5$ giờ = $210$ phút. 
}
\end{ex}

\begin{ex}%Câu 19
\immini
{
    Một quả trứng khủng long đồ chơi bằng nhựa có thiết diện qua trục lớn là một đường elip. Biết độ dài mỗi trục là $12$ cm và $8$ cm.\\
Bên trong quả trứng người ta cần thiết kế một chiếc hộp hình trụ để đựng các đồ chơi trẻ con như bóng đèn xanh đỏ, kẹo v.v...\\
Hỏi khối trụ như thế có thể tích tối đa bao nhiêu cm$^3$ (làm tròn đến hàng đơn vị).
\shortans{232}
}
{
    \includegraphics[height=5cm]{img/HXN-2-19-a}\includegraphics[width=5cm]{img/HXN-2-19-b}
}
\loigiai{
    Gắn elip lên hệ trục tọa độ $Oxy$ như hình vẽ, elip có $2a=12\Rightarrow a=6$; $2b=8\Rightarrow b=4$.\\
    \centerline{
    \includegraphics[width=5cm]{img/HXN-2-19-LG}
    }
    Phương trình chính tắc elip $(E)$: $\dfrac{x^2}{36}+\dfrac{y^2}{16}=1$.\\
    Đặt chiều cao và bán kính đáy hình trụ nội tiếp elip là $h$, $r$ thì điểm tiếp xúc  $M\left(\dfrac{h}{2};r\right)$ với $0<h<12;0<r<4$.\\
    Điểm $M$ thuộc $(E)$: $\dfrac{x^2}{36}+\dfrac{y^2}{16}=1\Rightarrow \dfrac{\left(\dfrac{h}{2}\right)^2}{36}+\dfrac{r^2}{16}=1 \Rightarrow \dfrac{h^2}{144}+\dfrac{r^2}{16}=1\Rightarrow r^2=16\left(1-\dfrac{h^2}{144}\right)$.\\
    Thể tích khối trụ là $V=\pi r^2h=\pi 16\left(1-\dfrac{h^2}{144}\right)\cdot h=\pi \left(16h-\dfrac{h^3}{9}\right)$ hay $V=\pi \left(16h-\dfrac{h^3}{9}\right)$.\\
    Ta có $V'=\pi \left(16-\dfrac{h^2}{3}\right)$; $V'=0\Rightarrow 16-\dfrac{h^2}{3}=0\Rightarrow h^2=48\Rightarrow h=4\sqrt{3}\in (0;12)$.\\
    Giá trị lớn nhất của thể tích khối trụ là $V_{\max }=V\left(4\sqrt{3}\right)=\pi \left(16.4\sqrt{3}-\dfrac{\left(4\sqrt{3}\right)^3}{9}\right)\approx 232$ cm$^3$.
}
\end{ex}

\begin{ex}%Câu 20
Một nhóm học sinh lớp $12$ đã lên bản thiết kế mẫu hoa văn cho một loại gạch men lát nền nhà. Các em đã vẽ $4$ đường cong như hình, từ đó tạo thành một hình $(\mathscr{H})$ khép kín ở giữa viên gạch để tạo điểm nhấn. Cụ thể cách dựng hình được thực hiện như sau:
\begin{itemize}
    \item Dựng hệ trục $Oxy$ với điểm $O$ là tâm của viên gạch, tia $Ox$ hướng sang phải và tia $Oy$ hướng lên trên, đơn vị trên mỗi trục là $5$ cm
    \item Các em lấy $O$ là tâm viên gạch và $A$ là trung điểm một cạnh viên gạch, xác định được điểm $B$ thỏa mãn $\overrightarrow{OB}=\dfrac{5}{6}\overrightarrow{OA}$.
    \item Dựng đường thẳng $\Delta \colon 5x-9=0$. Đường cong $\left(L_1\right)$ là tập hợp các điểm $M$ thỏa mãn $3MB=5d\left(M,\Delta\right)$.
    \item Lấy đối xứng đường cong $\left(L_1\right)$ qua tâm $O$ và qua các đường chéo của viên gạch thì được các đường cong còn lại.
\end{itemize}
\begin{center}
    \includegraphics[width=5cm]{img/HXN-2-20-a}\qquad  \includegraphics[width=5cm]{img/HXN-2-20-b}
\end{center}
Biết viên gạch là hình vuông có kích thước $60$cm; hỏi diện tích hình $(\mathscr{H})$ là bao nhiêu centimét vuông (làm tròn đến hàng đơn vị)?
\shortans{1168}
\loigiai{
    Ta chọn hệ trục tọa độ như hình vẽ với mỗi đơn vị trên trục bằng $5$ cm.\\
    \centerline{
        \includegraphics[width=5cm]{img/HXN-2-20-LG}
    }
    Đường cong $\left(L_1\right)$ là tập hợp điểm $M$ thỏa $3MB=5d\left(M,\triangle \right)$ hay $\dfrac{MB}{d\left(M,\triangle \right)}=\dfrac{5}{3}$\tagEX{1}
    Phương trình $\triangle \colon x=\dfrac{9}{5}$ \tagEX{2}
    Từ $(1)$ và $(2)$ ta thấy $M$ thuộc một nhánh của hyperbol $\left(L_1\right)$: $\dfrac{x^2}{a^2}-\dfrac{y^2}{b^2}=1$; trong đó điểm $B(5;0)$ là một trong hai tiểu điểm của $\left(L_1\right)$ nên $c=5$.\\
    Đường chuẩn $x=\dfrac{a}{e}=\dfrac{a^2}{c}=\dfrac{9}{5}\Rightarrow a=3$ (thử lại ta thấy $\dfrac{MB}{d\left(M,\triangle \right)}=\dfrac{5}{3}=e$ (hợp lí)).\\
    Do $c^2=a^2+b^2\Rightarrow b=\sqrt{c^2-a^2}=4$.\\
    Phương trình $\left(L_1\right)\colon \dfrac{x^2}{9}-\dfrac{y^2}{16}=1$ $(x>3)$.\\
    Xét giao điểm của $\left(L_1\right)$ với đương thẳng $y=x$, ta có $\dfrac{x^2}{9}-\dfrac{x^2}{16}=1\,\,\left(x>3\right)\Rightarrow x=\dfrac{12\sqrt{7}}{7}\approx 4{,}54$.\\
    Từ phương trình $\dfrac{x^2}{9}-\dfrac{y^2}{16}=1\Rightarrow \dfrac{y^2}{16}=\dfrac{x^2}{9}-1\Rightarrow y=4\sqrt{\dfrac{x^2}{9}-1}$ $(x>3)$.\\
    Diện tích cần tính là $S=8\left(\int\limits_0^3{x\mathrm{\,d}x}+\int\limits_3^{\tfrac{12\sqrt{7}}{7}}{\left(x-4\sqrt{\dfrac{x^2}{9}-1}\right)\mathrm{\,d}x}\right)\times 25\approx 1168$ cm$^2$.
}
\end{ex}

\begin{ex}%Câu 21
Trong không gian $Oxyz$, cho ba điểm $A\left(-8;-1;6\right)$, $B\left(1;2;3\right)$, $C\left(-4;14;\sqrt{11}\right)$. Điểm $M$ di động trên mặt cầu $\left(S_1\right)\colon {(x-4)^2}+(y-3)^2+(z+3)^2=49$ sao cho tam giác $MAB$ có $2\sin\widehat{MAB}=\sin\widehat{MBA}$. Tính giá trị nhỏ nhất của đoạn thẳng $CM^2$ (làm tròn đến hàng đơn vị).
\shortans{64}
\loigiai{
    Xét $\triangle MAB$, ta có $\dfrac{BM}{\sin \widehat{MAB}}=\dfrac{AM}{\sin \widehat{MBA}}=2R\Rightarrow \sin \widehat{MAB}=\dfrac{BM}{2R};\sin \widehat{MBA}=\dfrac{AM}{2R}$.\\
    Theo giả thiết $2\sin\widehat{MAB}=\sin\widehat{MBA} \Rightarrow 2\cdot \dfrac{BM}{2R}=\dfrac{AM}{2R}\Rightarrow AM=2BM$.\\
    Gọi $M(x;y;z)$ thì ta có 
    \allowdisplaybreaks
    \begin{eqnarray*}
        &&AM^2=4BM^2\\
        &\Leftrightarrow& (x+8)^2+(y+1)^2+(z-6)^2=4\left[(x-1)^2+(y-2)^2+(z-3)^2\right]\\ &\Leftrightarrow& 3x^2+3y^2+3z^2-24x-18y-12z-45=0\\
        &\Leftrightarrow& x^2+y^2+z^2-8x-6y-4z-15=0
    \end{eqnarray*}
    Do đó $M$ di động trên mặt cầu $(S_2)$ có tâm $I_2(4;3;2)$, bán kính $R_2=2\sqrt{11}$.\\
    Mặt khác ta cũng có $M\in (S_1)$ có tâm $I_1(4;3;-3)$, bán kính $R_1=7$.\\
    Ta có $I_1I_2=5<R_1+R_2$ nên $M$ thuộc đường tròn $(C)=(S_1)\cap (S_2)$.
    \immini
    {
        Tập hợp điểm $M$ thuộc $(C)$ thỏa hệ phương trình\\
    $\heva{& (x-4)^2+(y-3)^2+(z+3)^2=49 \\& x^2+y^2+z^2-8x-6y-4z-15=0} $\\
    $\Leftrightarrow \heva{& x^2+y^2+z^2-8x-6y+6z-15=0 \\& x^2+y^2+z^2-8x-6y-4z-15=0 }\\ \Leftrightarrow \heva{& z=0\,\,(Oxy) \\& x^2+y^2+z^2-8x-6y-4z-15=0 } $.
    }
    {
        \includegraphics[width=5cm]{img/HXN-2-21-LG}
    }
    Vậy $(C)$ thuộc mặt phẳng $(Oxy)$; hình chiếu của $I_1(4;3;-3)$ trên $(Oxy)$ là điểm $E(4;3;0)$ cũng là tâm của đường tròn $(C)$; bán kính $(C)$ là $r=\sqrt{R_1^2-I_1E^2}=\sqrt{7^2-9}=2\sqrt{10}$.\\
    Gọi $C'(-4;14;0)$ là hình chiếu của $C$ lên mặt phẳng $(Oxy)$, ta có $EC'=\sqrt{185}>r$ nên $C'$ nằm ngoài đường tròn $(E; r)$.\\
    Ta có $CM^2\ge CM_0^2=CC'^2+C'M_0^2=CC'^2+\left(EC'-r\right)^2 =11+\left(\sqrt{185}-2\sqrt{10}\right)^2\approx 64$.\\
    Vậy giá trị nhỏ nhất của $CM^2$ xấp xỉ $64$.
}
\end{ex}

\begin{ex}%Câu 22
\immini
{
    Vào một hội thi thiết kế đèn lồng Trung thu, ban tổ chức nhận được một chiếc đèn lồng đặc biệt có hình một tứ diện đều. Trên mỗi cạnh tứ diện thí sinh thiết kế $3$ bóng đèn nằm ở $3$ vị trí chia cạnh tứ diện thành $4$ đoạn bằng nhau. Cứ mỗi phút trôi qua, sẽ có ngẫu nhiên $3$ bóng đèn phát sáng, các bóng còn lại thì tắt. Tính xác suất để ngay phút đầu tiên được ban giám khảo chấm điểm, có $3$ bóng đèn phát sáng ứng với $3$ điểm tạo nên mặt phẳng song song với đúng một cạnh của tứ diện, biết rằng $3$ bóng đèn không hoàn toàn thuộc về một cạnh tứ diện. (Kết quả được làm tròn đến hàng phần trăm).
\shortans{0,27}
}
{
     \includegraphics[width=5cm]{img/HXN-2-22}
}
\loigiai{
    Tổng số cách chọn ra $3$ trong $18$ điểm là $\mathrm{C}_{18}^3=816$.\\
    Tuy nhiên, sẽ có các trường hợp ba điểm thẳng hàng, đó là khi ta lấy ba điểm thuộc cùng một cạnh, tổng số cách là $6\times \mathrm{C}_3^3=6$.\\
    Do đó $n\left(\Omega \right)=816-6=810$.\\
    Xét các mặt phẳng qua $3$ điểm ($3$ bóng đèn) và song song với đoạn $AB$.
    \immini
    {
        Trường hợp $1$: Chọn $1$ cặp điểm thuộc mặt phẳng $(ABC)$ và $1$ điểm không thuộc $(ABC)$.
    \begin{itemize}
        \item Bước $1$: Có $3$ cách chọn $1$ cặp điểm thuộc $(ABC)$.
        \item Bước $2$: Với mỗi cách chọn trong bước $1$ thì lẽ ra sẽ có $9$ cách chọn $1$ điểm không thuộc mặt phẳng $(ABC)$; tuy nhiên vì điều kiện mặt phẳng qua $3$ điểm chỉ song song đúng $1$ cạnh tứ diện (ở đây là $AB$) nên ta loại $3$ điểm trong số $9$ điểm không thuộc $(ABC)$ (ví dụ khi chọn cặp điểm $M$, $N$ thuộc $(ABC)$ thì ta loại $P$, $Q$, $R$  thuộc $AD$, $BD$, $CD$).\\
        Do vậy ta có $3\times 6=18$ cách chọn bộ ba điểm trong trường hợp này.
    \end{itemize}
    Trường hợp $2$: Chọn $1$ cặp điểm thuộc mặt phẳng $(ABD)$ và $1$ điểm không thuộc $(ABD)$.
    }
    {
        \includegraphics[width=5cm]{img/HXN-2-22-LG}
    }
    Ta cũng có $3\times 6=18$ cách chọn bộ ba điểm thỏa mãn.\\
    Vì tính chất bình đẳng của $6$ cạnh trong tứ diện đều, ta có tất cả $6\cdot (18+18)=216$ bộ ba điểm thỏa mãn.\\
    Vì vậy xác suất cần tính là $P=\dfrac{216}{810}=\dfrac{4}{15}\approx 0{,}27$.
}
\end{ex}
\Closesolutionfile{ans}
\inputansbox{6,4,3}{ans/ans-HXN-\sode-T,ans/ans-HXN-\sode-TF,ans/ans-HXN-\sode-SA}
% \def\sode{3}
\begin{name}
	{\tenchude}
	{\tendethi}
	{\tentruong}
	{\thoigian}
\end{name}
\caulc
\Opensolutionfile{ans}[ans/ans-HXN-\sode-T]
%Câu hỏi
\begin{ex}%Câu 1
 Nguyên hàm của hàm số $f(x)=2^x$ là
 \choice
 {$\dfrac{2^{x+1}}{x+1}+C$}
 {\True $\dfrac{2^x}{\ln 2}+C$}
 {$\dfrac{2^x}{x}+C$}
 {$x{2^{x-1}}+C$}
 \loigiai{
 Chọn B.\\
 Ta có $\displaystyle\int\limits_{}^{}{2^x\text{d}x}=\dfrac{2^x}{\ln 2}+C$ .}
\end{ex}
\begin{ex}%Câu 2
 Cho hàm số $y=f(x)$ có bảng biến thiên như sau.
 \begin{center}
 \begin{tikzpicture}[>=stealth]
 \tkzTabInit[nocadre=false,lgt=1,espcl=2.5,deltacl=0.5]{$x$/.7 ,$y'$/.7,$y$/2}
 {$-\infty$ , $-2$ , $3$ , $+\infty$}
 \tkzTabLine{ , - , $0$ , + , $0$ , - , }
 \tkzTabVar{+/$+\infty$ , -/$-3$ , +/$2$ , -/$-\infty$}
 \end{tikzpicture}
 \end{center}
 Giá trị cực đại của hàm số đã cho bằng
 \choice
 {$3$}
 {\True $2$}
 {$-2$}
 {$-3$}
 \loigiai{
 Chọn B.\\
 Dựa vào bảng biến thiên, giá trị cực đại của hàm số là $y_{CT}=2$ .}
\end{ex}
\begin{ex}%Câu 3
 Cho cấp số cộng $\left(u_n\right)$ có $u_2=2;\,\,u_5=11$ . Công sai $d$ của cấp số cộng là
 \choice
 {1}
 {2}
 {4}
 {\True 3}
 \loigiai{
 Chọn D.\\
 Ta có: $\left\{\begin{aligned}
 &{u_2}=2\\ 
 &{u_5}=11\\ 
 \end{aligned}\right.\Leftrightarrow\left\{\begin{aligned}
 &{u_1}+d=2\\ 
 &{u_1}+4d=11\\ 
 \end{aligned}\right.\Leftrightarrow\left\{\begin{aligned}
 &{u_1}=-1\\ 
 & d=3\\ 
 \end{aligned}\right.$ . Vậy công sai của cấp số cộng là $d=3$ .}
\end{ex}
\begin{ex}%Câu 4
 Cho hàm số $y=f(x)$ có đạo hàm $f'(x)=x+1$ với mọi $x\in\mathbb{R}$ . Hàm số đã cho nghịch biến trên khoảng nào dưới đây?
 \choice
 {$\left(-1\,;\,\,+\infty\right)$}
 {$\left(1\,;\,\,+\infty\right)$}
 {\True $\left(-\infty\,;\,\,-1\right)$}
 {$\left(-\infty\,;\,\,1\right)$}
 \loigiai{
 Chọn C.\\
 }
\end{ex}
\begin{ex}%Câu 5
 Trong không gian $Oxyz$ , đường thẳng $d$ đi qua điểm $M\left(1\,;\,\,-1\,;\,\,3\right)$ và song song với đường thẳng$\Delta :\dfrac{x-2}{2}=\dfrac{y+1}{1}=\dfrac{z+3}{-1}$ có phương trình là
 \choice
 {\True $\left\{\begin{aligned}
 & x=1+2t\\ 
 & y=-1+t\\ 
 & z=3-t\\ 
 \end{aligned}\right.$}
 {$\left\{\begin{aligned}
 & x=1+2t\\ 
 & y=-1+t\\ 
 & z=3+t\\ 
 \end{aligned}\right.$}
 {$\left\{\begin{aligned}
 & x=2+t\\ 
 & y=1-t\\ 
 & z=-1+3t\\ 
 \end{aligned}\right.$}
 {$\left\{\begin{aligned}
 & x=1+2t\\ 
 & y=1+t\\ 
 & z=3-t\\ 
 \end{aligned}\right.$}
 \loigiai{
 Chọn A.\\
 Đường thẳng d song song $\Delta :\dfrac{x-2}{2}=\dfrac{y+1}{1}=\dfrac{z+3}{-1}$ nên có vectơ chỉ phương $\vec{u}=\left(2\,;\,\,1\,;\,\,-1\right)$ ; mà d qua $M\left(1\,;\,\,-1\,;\,\,3\right)$ nên có phương trình tham số $\left\{\begin{aligned}
 & x=1+2t\\ 
 & y=-1+t\\ 
 & z=3-t\\ 
 \end{aligned}\right.$ .}
\end{ex}
\begin{ex}%Câu 6
 Tập nghiệm của bất phương trình $\log_{\dfrac{1}{2}}\left(9-x^2\right)<0$ chứa bao nhiêu số nguyên?
 \choice
 {1}
 {\True $5$}
 {$3$}
 {4}
 \loigiai{
 Chọn B.\\
 Ta có: $\log_{\dfrac{1}{2}}\left(9-x^2\right)<0\Leftrightarrow 9-x^2>\left(\dfrac{1}{2}\right)^0\Leftrightarrow 9-x^2>1\Leftrightarrow{x^2}<8\Leftrightarrow-2\sqrt{2}<x<2\sqrt{2}$ .\\
 Tập nghiệm bất phương trình chứa 5 số nguyên là: $-2\,;\,\,1\,;\,\,0\,;\,\,1\,;\,\,2$ .}
\end{ex}
\begin{ex}%Câu 7
 Trong không gian $Oxyz$ , cho hai vectơ $\vec{u}=\left(1\,;\,\,-4\,;\,\,0\right)$ và $\vec{v}=\left(-1\,;\,\,-2\,;\,\,1\right)$ . Vectơ $\vec{u}+3\vec{v}$ có tọa độ là
 \choice
 {\True $\left(-2\,;\,\,-10\,;\,\,3\right)$}
 {$\left(-2\,;\,\,-6\,;\,\,3\right)$}
 {$\left(-4\,;\,\,-8\,;\,\,4\right)$}
 {$\left(-2\,;\,\,-10\,;\,\,-3\right)$}
 \loigiai{
 Chọn A.\\
 Ta có: $\vec{u}+3\vec{v}=\left(1\,;\,\,-4\,;\,\,0\right)+3\left(-1\,;\,\,-2\,;\,\,1\right)=\left(-2\,;\,\,-10\,;\,\,3\right)$ .}
\end{ex}
\begin{ex}%Câu 8
 Tâm đối xứng của đồ thị hàm số $y=\dfrac{3x+1}{x-2}$ có tọa độ là
 \choice
 {$\left(3\,;\,\,-2\right)$}
 {$\left(3\,;\,\,2\right)$}
 {$\left(-2\,;\,\,3\right)$}
 {\True $\left(2\,;\,\,3\right)$}
 \loigiai{
 Chọn D.\\
 Đồ thị hàm số có tiệm cận đứng $x\,=2$ và tiệm cận ngang $y\,=\,3$ .\\
 Do đó tâm đối xứng của đồ thị hàm số là $\left(2\,;\,\,3\right)$ .}
\end{ex}
\begin{ex}%Câu 9
 Cho mẫu số liệu ghép nhóm ở bảng sau. Khoảng tứ phân vị của mẫu số liệu ghép nhóm (làm tròn đến hàng phần trăm) là
 \choice
 {$19,15$}
 {$21,32$}
 {$20,07$}
 {\True $22,23$}
 \loigiai{
 Chọn D.\\
 Tứ phân vị thứ nhất của mẫu số liệu gốc là $\dfrac{x_6+x_7}{2}\in\left[30\,;\,\,40\right)$ nên tứ phân vị thứ nhất của mẫu số liệu ghép nhóm là $Q_1=30+\dfrac{\dfrac{25}{4}-3}{7}\cdot 10=\dfrac{485}{14}$ .\\
 Tứ phân vị thứ ba của mẫu số liệu gốc là $\dfrac{x_{19}+x_{20}}{2}\in\left[50\,;\,\,60\right)$ nên tứ phân vị thứ ba của mẫu số liệu ghép nhóm là $Q_3=50+\dfrac{3\cdot\dfrac{25}{4}-16}{4}\cdot 10=\dfrac{455}{8}$ .\\
 Khoảng tứ phân vị của mẫu số liệu ghép nhóm trên là $Q_3\,-\,Q_1\,=\,\dfrac{1245}{56}\approx\,22,23$ .}
\end{ex}
\begin{ex}%Câu 10
 Cho hàm số $f(x)=x^2+\sin x+1$ . Biết rằng $F(x)$ là một nguyên hàm của hàm số $f(x)$ và thỏa mãn $F(0)=1$ . Khi đó $F(x)$ bằng
 \choice
 {$F(x)=x^3-\cos x+x+2$}
 {\True $F(x)=\dfrac{x^3}{3}-\cos x+x+2$}
 {$F(x)=\dfrac{x^3}{3}+\cos x+x$}
 {$F(x)=\dfrac{x^3}{3}+\cos x+2$}
 \loigiai{
 Chọn B.\\
 Ta có $F(x)=\displaystyle\int{\left(x^2+\sin x+1\right)}\text{d}x=\dfrac{x^3}{3}-\cos x+x+C$ .\\
 Theo giả thiết: $F(0)=1\Rightarrow\dfrac{0^3}{3}-\cos 0+0+C=1\Rightarrow C=2$ .\\
 Do đó $F(x)=\dfrac{x^3}{3}-\cos x+x+2$ .}
\end{ex}
\begin{ex}%Câu 11
 Người ta thống kê lại đường kính thân gỗ của một số cây xoan đào 6 năm tuổi được trồng ở một lâm trường ở bảng sau:\\
 \centerline{\begin{tabular}{|c|c|c|c|c|c|}
 \hline
 Đường kính $(cm)$ & $[40\,;\,\,45)$ & $[45\,;\,\,50)$ & $[50\,;\,\,55)$ & $[55\,;\,\,60)$ & $[60\,;\,\,65)$\\
 \hline
 Tần số & 5 & 20 & 18 & 7 & 3\\
 \hline
 \end{tabular}}\\
 Khoảng biến thiên của mẫu số liệu ghép nhóm trên là
 \choice
 {\True 25}
 {30}
 {6}
 {69,8}
 \loigiai{
 Chọn A.\\
 Khoảng biến thiên của mẫu số liệu ghép nhóm trên là $65-40=25\,\,\,cm.$}
\end{ex}
\begin{ex}%Câu 12
 Cho hình hộp $ABCD.A'{B}'{C}'{D}'$ (xem hình vẽ). Phát biểu nào sau đây là đúng? 
 \choice
 {$\overrightarrow{A{C}'}=\overrightarrow{AB}+\overrightarrow{A{B}'}+\overrightarrow{AD}$}
 {\True $\overrightarrow{D{B}'}=\overrightarrow{DA}+\overrightarrow{D{D}'}+\overrightarrow{DC}$}
 {$\overrightarrow{A{C}'}=\overrightarrow{AC}+\overrightarrow{AB}+\overrightarrow{AD}$}
 {$\overrightarrow{DB}=\overrightarrow{DA}+\overrightarrow{D{D}'}+\overrightarrow{DC}$}
 \loigiai{
 Chọn B.\\
 Theo quy tắc hình hộp ta có$\overrightarrow{D{B}'}=\overrightarrow{DA}+\overrightarrow{D{D}'}+\overrightarrow{DC}$ .
 }
 \end{ex}
\Closesolutionfile{ans}
\cauds
\Opensolutionfile{ans}[ans/ans-HXN-\sode-TF]
%Câu hỏi
\begin{ex}
    \immini[thm]{Xét tam giác $ABC$ có $AC = 2AB$ và $BC = 10\ cm$. Trên cạnh $AC$ lấy điểm $D$ sao cho $AD = \dfrac{1}{4}AC$, trên cạnh $AB$ lấy điểm $E$ sao cho $AE = \dfrac{1}{4}AB$, trên cạnh $AD$ lấy điểm $F$ sao cho $AF = \dfrac{1}{4}AD$ và tiếp tục lấy các điểm $G, H, I, J, \dots$ (vô hạn lần) theo quy luật đó.
        
        \choiceTF
        {\True $\dfrac{AB}{AC} = \dfrac{AD}{AB}$}
        {Tam giác $ABD$ đồng dạng với tam giác $ABC$}
        {$BD = 5\ cm$; $DE = 3\ cm$}
        {\True Độ dài đường gấp khúc $CBDEFGH\dots$ bằng $20\ cm$}}{\includegraphics[scale=1.5]{img/HXN-3.13}}
    \loigiai{
    \begin{itemchoice}
        \itemch Ta có: $AD = \dfrac{1}{4}AC = \dfrac{1}{4} \cdot 2AB = \dfrac{1}{2}AB$; suy ra $\dfrac{AD}{AB} = \dfrac{1}{2}$. $\dfrac{AB}{AC} = \dfrac{AB}{2AB} = \dfrac{1}{2}$. Suy ra $\dfrac{AD}{AB} = \dfrac{AB}{AC}$.
        \itemch Hai tam giác $ABD$ và $ACB$ đồng dạng vì có góc $\hat{A}$ chung và $\dfrac{AB}{AC} = \dfrac{AD}{AB}$.
        \itemch Từ câu b) ta suy ra $\dfrac{AB}{AC} = \dfrac{AD}{AB} = \dfrac{BD}{CB} = \dfrac{1}{2} \Rightarrow BD = \dfrac{1}{2}BC = \dfrac{1}{2} \cdot 10 = 5\ cm$. Hoàn toàn tương tự, ta chứng minh được hai tam giác $ADB$ và $AED$ đồng dạng, suy ra $DE = \dfrac{1}{2}BD = \dfrac{1}{2} \cdot 5 = 2,5\ cm$.
        \itemch Độ dài đường gấp khúc $CBDEFGH\dots$ bằng $l = CB + BD + DE + EF + FG + \dots = 10 + 5 + 2,5 + \dots$ Đây là tổng của một cấp số nhân lùi vô hạn có số hạng đầu $u_1 = 10$, công bội $q = \dfrac{1}{2}$. Do đó $l = \dfrac{u_1}{1-q} = \dfrac{10}{1-\dfrac{1}{2}} = \dfrac{10}{\dfrac{1}{2}} = 20\ cm$.
        \end{itemchoice}}
\end{ex}

\begin{ex}
    \immini
    {
        Xét một hệ trục tọa độ $Oxyz$ được cho sẵn, đơn vị trên mỗi trục là dm, mặt ngoài của một quả bóng được mô hình hóa bởi phương trình mặt cầu $(x-2)^2+(y+1)^2+(z+1)^2=6$, quả bóng nằm yên trên sàn nhà. Người ta nhìn thấy một tấm ván ngã xuống đè lên quả bóng, phần giao của tấm ván và sàn nhà là đường thẳng $d$ có phương trình $\dfrac{x+2}{2}=\dfrac{y+1}{-3}=\dfrac{z}{1}$.  Gọi $A$, $B$ lần lượt là hai tiếp điểm của tấm ván, sàn nhà với quả bóng và $I$ là tâm quả bóng.
        \choiceTF
    {\True Quả bóng có tâm $I(2;-1;-1)$ và bán kính $R=\sqrt{6}$}
    {Khoảng cách từ tâm quả bóng đến đường thẳng d bằng $2\sqrt{6}$}
    {\True Nếu $\cos\widehat{AIB}$ bằng $\dfrac{a}{b}$ (phân số tối giản) thì giá trị $a^2+b^2=82$}
    {\True Một con kiến bò từ vị trí $A$ đến vị trí $B$ trên quả bóng với tốc độ $2$ cm/s; thời gian ngắn nhất cho chuyến đi này là $21$ giây (làm tròn đến hàng đơn vị)}
    }
    {
        \includegraphics[width=5cm]{img/HXN-3-14}
    }
    \loigiai{
        \begin{itemchoice}
            \itemch Mặt ngoài quả bóng là mặt cầu $(S)$ có tâm $I(2;-1;-1)$ và bán kính $R=\sqrt{6}$.
            \itemch Đường thẳng  $d\colon \dfrac{x+2}{2}=\dfrac{y+1}{-3}=\dfrac{z}{1}$ qua $A(-2;-1;0)$ và có VTCP $\vec{u}_d=(2;-3;1)$.\\
            Ta có $\vec{AI}=(4;0;-1);\left[{\vec{u}}_d,\vec{AI}\right]=(3;6;12)$.\\
            Do đó $d\left(I,d\right)=\dfrac{\left| \left[{\vec{u}}_d,\vec{AI}\right] \right|}{\left| {\vec{u}}_d \right|}=\dfrac{\sqrt{3^2+6^2+12^2}}{\sqrt{2^2+(-3)^2+1^2}}=\dfrac{3\sqrt{6}}{2}$.
            \immini
            {
            \itemch Gọi $K$ là hình chiếu của $I$ trên $d$ thì $KI=\dfrac{3\sqrt{6}}{2}$ và $KA\perp IA$;\\
            suy ra $\cos\widehat{AIK}=\dfrac{IA}{IK}=\dfrac{2}{3}$\\
            Do vậy $\cos\widehat{AIB}=2\cos^2\widehat{AIK}-1 =-\dfrac{1}{9}=\dfrac{a}{b}$\\
            $\Rightarrow a^2+b^2=82$.
            \itemch Độ dài cung tròn bé nhất mà con kiến có thể đi: $$l_{\wideparen{AB}}=R\times \widehat{AIB}=\sqrt{6}\times \arccos \left(-\dfrac{1}{9}\right)\approx 4{,}12\,dm$$
            Thời gian tối thiểu để kiến đến nơi là $\dfrac{l_{\wideparen{AB}}\times 10}{2}\approx 21$ giây.
            }
            {
                \includegraphics[width=5cm]{img/HXN-3-14-LG}
            }
        \end{itemchoice}
    }
\end{ex}
\begin{ex}
Thám tử lừng danh Sherlock Holmes đang điều tra một vụ án được thực hiện độc lập bởi một trong hai nghi phạm là McFarlane và Oldacre. Ban đầu thám tử đã có bằng chứng ngang nhau chống lại cả hai người.\\
Trong quá trình điều tra thêm tại hiện trường vụ án, Sherlock Holmes phát hiện rằng thủ phạm có nhóm máu mà chỉ $10\%$ dân số có; và Oldacre có nhóm máu này, còn nhóm máu của McFarlane thì chưa biết.\\
Gọi $A$ là biến cố: \lq\lq McFarlane là thủ phạm\rq\rq; $B$ là biến cố: \rq\rq Oldacre là thủ phạm\rq\rq; $C$ là biến cố: \lq\lq Nhóm máu của nghi phạm trùng với nhóm máu thủ phạm thực sự\rq\rq.
\choiceTF
{Trong quá trình điều tra, nếu Sherlock Holmes biết chắc chắn McFarlane không là  thủ phạm, khi đó xác suất để Oldacre là thủ phạm bằng $0{,}98$}
{$P\left(C|A\right)=0{,}1;P\left(C|B\right)=0{,}5$}
{Dựa trên thông tin về nhóm máu, xác suất để Oldacre là thủ phạm bằng $\dfrac{9}{11}$}
{\True Dựa trên thông tin về nhóm máu, xác suất để McFarlane cũng có nhóm máu trùng với nhóm máu thủ phạm bằng $\dfrac{2}{11}$}
\loigiai{
    \begin{itemchoice}
        \itemch Ta có: $P\left(B|\bar{A}\right)=\dfrac{P\left(B\bar{A}\right)}{P\left({\bar{A}}\right)}=\dfrac{P(B)}{P\left({\bar{A}}\right)}=\dfrac{\dfrac{1}{2}}{\dfrac{1}{2}}=1$. \\
        Như vậy nếu McFarlane không là  thủ phạm thì chắc chắn Oldacre là thủ phạm.
        \itemch Ta có: $P\left(C|A\right)=0{,}1;P\left(C|B\right)=1$.
        \itemch Áp dụng công thức Bayes:$P\left(B\mid C\right)=\dfrac{P(B)\cdot P\left(C\mid B\right)}{P(B)\cdot P\left(C\mid B\right)+P(A)\cdot P\left(C\mid A\right)}=\dfrac{\dfrac{1}{2}\cdot 1}{\dfrac{1}{2}\cdot 1+\dfrac{1}{2}\cdot \dfrac{1}{10}}=\dfrac{10}{11}$.
        \itemch Từ câu c) ta có $P\left(\bar{B}|C\right)=1-P\left(B\mid C\right)=\dfrac{1}{11}$.
        Gọi $D$ là biến cố \lq\lq McFarlane có cùng nhóm máu với thủ phạm biết rằng Oldacre có cùng nhóm máu với thủ phạ\rq\rq.
        Ta có $P(D)=\dfrac{10}{11}\cdot \dfrac{1}{10}+\dfrac{1}{11}\cdot 1=\dfrac{2}{11}$.
        (Nếu Oldacre là thủ phạm, xác suất để McFarlane có cùng nhóm máu với thủ phạm bằng $\dfrac{1}{10}$; nếu Oldacre không là thủ phạm thì xác suất để McFarlane có cùng nhóm máu với thủ phạm bằng $1$).
    \end{itemchoice}
}
\end{ex}
\begin{ex}
\immini
{
    Cho $y=f(x)$, $y=g(x)$ lần lượt là các hàm đa thức bậc ba và bậc nhất có đồ thị như hình vẽ.
    Biết diện tích hình $S$ (được tô màu) bằng $\dfrac{250}{81}$. 
    \choiceTF
    {Hàm số $g(x)=\dfrac{3}{5}x-\dfrac{1}{5}$}
    {\True Hàm số $\displaystyle\int\limits_{-2}^{\tfrac{4}{3}}{\left[f(x)-g(x)\right]\mathrm{\,d}x}=\dfrac{250}{81}$}
    {Hàm số $f(x)=\dfrac{3}{10}(x+2)\left(x-\dfrac{4}{3}\right)(x-3)+\dfrac{3}{5}x+\dfrac{1}{5}$}
    {$\int\limits_0^2{f(x)\mathrm{\,d}x}=\dfrac{37}{15}$}
}
{
    \includegraphics[width=5cm]{img/HXN-3-16}
}
    \loigiai{
        \begin{itemchoice}
            \itemch Ta có $g(x)$ là hàm số bậc nhất đi qua các điểm $A\left(\dfrac{4}{3};1\right)$, $B(3;2)$ nên $g(x)=\dfrac{3}{5}x+\dfrac{1}{5}$.
            \itemch Ta thấy hai đồ thị hàm số $y=f(x),y=g(x)$ cắt nhau tại điểm có $y=-1$; thay vào đường thẳng $y=\dfrac{3}{5}x+\dfrac{1}{5}$  thì $x=-2$.
            Do đó $S=\displaystyle\int\limits_{-2}^{\tfrac{4}{3}}{\left[f(x)-g(x)\right]\mathrm{\,d}x}=\dfrac{250}{81}$.
            \itemch Đặt $f(x)-g(x)=a(x+2)\left(x-\dfrac{4}{3}\right)(x-3)$ với $a>0$.\\
            Ta có: $S=\int\limits_{-2}^{\tfrac{4}{3}}{\left[f(x)-g(x)\right]\mathrm{\,d}x}\Leftrightarrow \displaystyle\int\limits_{-2}^{\tfrac{4}{3}}{\left[a(x+2)\left(x-\dfrac{4}{3}\right)(x-3)\right]\mathrm{\,d}x}=\dfrac{250}{81}\Leftrightarrow a=\dfrac{3}{20}$.\\
            Khi đó $f(x)-g(x)=\dfrac{3}{20}(x+2)\left(x-\dfrac{4}{3}\right)(x-3)\Leftrightarrow f(x)=\dfrac{3}{20}(x+2)\left(x-\dfrac{4}{3}\right)(x-3)+\dfrac{3}{5}x+\dfrac{1}{5}$.
            \itemch $\displaystyle\int\limits_0^2{f(x)\mathrm{\,d}x}=\displaystyle\int\limits_0^2{\left[\dfrac{3}{20}(x+2)\left(x-\dfrac{4}{3}\right)(x-3)+\dfrac{3}{5}x+\dfrac{1}{5}\right]\mathrm{\,d}x}=\dfrac{34}{15}$.
        \end{itemchoice}
    }
\end{ex}
\Closesolutionfile{ans}
\caukq
\Opensolutionfile{ans}[ans/ans-HXN-\sode-SA]
%Câu hỏi
\begin{ex}%Câu 17
 Cho tứ diện đều ABCD có tất cả cạnh bằng 2. Tính khoảng cách của hai đường thẳng chéo nhau AB và CD (làm tròn đến hàng phần trăm).
 \shortans{1,41}
\loigiai{
 \begin{center}
 \includegraphics[scale=1]{img/HXN-3.17a}
 \end{center}
 Gọi I, J theo thứ tự là trung điểm của AB, CD.\\
 Các tam giác ABC, ABD đều có I là trung điểm AB nên\\ $\left\{\begin{aligned}
 & AB\perp CI\\ 
 & AB\perp DI\\ 
 \end{aligned}\right.\Rightarrow AB\perp\left(ICD\right)$ mà $IJ\subset\left(ICD\right)\Rightarrow AB\perp IJ$ (1).\\
 Tương tự, các tam giác ACD, BCD đều có J là trung điểm CD nên\\ $\left\{\begin{aligned}
 & CD\perp AJ\\ 
 & CD\perp BJ\\ 
 \end{aligned}\right.\Rightarrow CD\perp\left(ABJ\right)$ , mà $IJ\subset\left(JAB\right)\Rightarrow CD\perp IJ$ (2).\\
 Từ (1) và (2) suy ra IJ là đoạn vuông góc chung của hai đường thẳng AB, CD.\\
 Ta có: $CI=\dfrac{2\sqrt{3}}{2}=\sqrt{3}$ ; $IJ=\sqrt{C{I^2}-C{J^2}}=\sqrt{3-1}=\sqrt{2}\approx 1,41$ .\\
 Vậy khoảng cách hai đường thẳng AB, CD xấp xỉ 1,41.}
 \end{ex}
 \begin{ex}%Câu 18
Buổi họp mặt cuối năm của VFF diễn ra trong không khí hân hoan phấn khởi sau khi ĐTQG Việt Nam vô địch AFF cup 2024. Vào cuối buổi họp thì HLV Kim Sang-sik chỉ bắt tay với một số người, còn lại tất cả thành viên đều bắt tay với nhau, hai người bất kì thì bắt tay không quá một lần. Hỏi cuộc họp này có bao nhiêu người tham dự biết rằng đã có tổng cộng 2014 cái bắt tay được thực hiện?
\shortans{64}
\loigiai{
Gọi n là người có mặt trong cuộc họp $\left(n\in\mathbb{N}\right)$ .\\
Số cái bắt tay tối đa trong cuộc họp là $C_n^2$ .\\
Trong thực tế thì tổng cộng số cái bắt tay là 2014; vì vậy $C_n^2>2014$\\
$\Rightarrow\dfrac{n!}{2\left(n-2\right)!}>2014\Rightarrow n\left(n-1\right)>4028\Rightarrow{n^2}-n-4028>0\Rightarrow n>63,97$ .
\begin{itemize}
\item Với $n=64$ thì số cái bắt tay tối đa là $C_{64}^2=2\,016$ ; số người mà ông Kim không bắt tay là $2016-2014=2$ (thỏa mãn).\\
\item Với $n=65$ thì số cái bắt tay tối đa là $C_{65}^2=2\,080$ ; số người mà ông Kim không bắt tay là $2080-2014=66$ (vô lí vì trong phòng họp đang có 65 người).\\
\item Ta không cần thử lại với $n>65$ vì luôn xảy ra điều vô lí như trên.
\end{itemize}
Vậy số người tham dự cuộc họp là $64$.}
\end{ex}
\begin{ex}%Câu 19
 \immini[thm]{Hình dáng phần đất liền của hai xã thuộc tỉnh Đồng Tháp được mô hình hóa bởi đồ thị hàm số $y=\dfrac{x^2+ax+b}{x-2}$ ; biết đồ thị có một điểm cực trị là $\left(1\,;\,\,1\right)$ , với hệ trục tọa độ Oxy như hình vẽ, đơn vị trên mỗi trục là 10 mét. Để thuận tiện cho giao thông hai xã, lãnh đạo tỉnh đã phê duyệt dự án xây một chiếc cầu nối phần đất liền của hai xã này. Nhằm tiết kiệm chi phí cho công trình, người kỹ sư trưởng thiết kế có nhiệm vụ nghiên cứu để chọn được hai vị trí A, B trên phần đất liền hai xã sao cho độ dài chiếc cầu (đoạn AB) là ngắn nhất có thể. Hỏi độ dài ngắn nhất của chiếc cầu đó (tính theo đường chim bay) là bao nhiêu mét (làm tròn đến hàng phần chục)?
 \shortans{43,9}}{\includegraphics[scale=1]{img/HXN-3.19}}
\loigiai{
Ta có $y'=\dfrac{x^2-4x-2a-b}{\left(x-2\right)^2}$ .\\
 Vì $\left(1\,;\,\,1\right)$ là điểm cực trị của đồ thị hàm số nên $\left\{\begin{aligned}
& y(1)=1\\ 
&{y}'(1)=0\\ 
\end{aligned}\right.\Rightarrow\left\{\begin{aligned}
& a+b=-2\\ 
& 2a+b=-3\\ 
\end{aligned}\right.\Rightarrow\left\{\begin{aligned}
& a=-1\\ 
& b=-1\\ 
\end{aligned}\right.$ .\\
Hàm số trở thành $y=\dfrac{x^2-x-1}{x-2}=x+1+\dfrac{1}{x-2},\,\,\,x\ne 2$ .\\
Gọi $A\left(2+a\,;\,\,3+a+\dfrac{1}{a}\right)\,,\,\,B\left(2-b\,;\,\,3-b-\dfrac{1}{b}\right)$ là hai điểm thuộc hai nhánh đồ thị với $a\,,\,\,b>0$ .\\
 Ta có: $A{B^2}=\left(a+b\right)^2+\left(a+b+\dfrac{1}{a}+\dfrac{1}{b}\right)^2=\left(a+b\right)^2+\left(a+b+\dfrac{a+b}{ab}\right)^2=\left(a+b\right)^2\left[1+\left(1+\dfrac{1}{ab}\right)^2\right]$\\
$=\left(a+b\right)^2\left(2+\dfrac{2}{ab}+\dfrac{1}{a^2b^2}\right)\overset{AM-GM}{\mathop{\ge}}\,4ab\left(2+\dfrac{2}{ab}+\dfrac{1}{a^2b^2}\right)$ $=8+8ab+\dfrac{4}{ab}\overset{AM-GM}{\mathop{\ge}}\,8+8\sqrt{2}$ .\\
Độ dài ngắn nhất của cây cầu (theo đường chim bay) là $AB\times 10=\sqrt{8+8\sqrt{2}}\times 10\approx\, 43,9 \,m$ .\\
Dấu “=” xảy ra khi và chỉ khi $a=b$ và $8ab=\dfrac{4}{ab}\Leftrightarrow a=b=\dfrac{1}{\sqrt[4]{2}}$ .}
\end{ex}
\begin{ex}%Câu 20
 \immini[thm]{Một cái chậu đựng nước có dạng hình chóp cụt đều đáy là các tam giác cạnh bằng 1 dm và 3 dm. Chiều cao chậu nước bằng 4 dm.
 Người ta bơm nước vào chậu với lưu lượng không đổi 0,5 lít/phút. Đến phút thứ 10 thì tốc độ dâng lên của nước trong chậu là bao nhiêu dm/phút? (Kết quả được làm tròn đến hàng phần trăm).
 \shortans{0,17}}{\includegraphics[scale=.8]{img/HXN-3.20}}
\loigiai{
\begin{center}
 \includegraphics[scale=1]{img/HXN-3.20a}
\end{center}
Gọi MN là độ dài cạnh tam giác đều theo mực nước tức thời (MN thay đổi).\\
Đặt $MN=a\times h+b$ (hàm số bậc nhất theo h).\\
Khi $h=0$ thì $MN=1$ ; khi $h=4$ thì $MN=3$ .\\
Do đó $\left\{\begin{aligned}
& b=1\\ 
& 4a+b=3\\ 
\end{aligned}\right.\Rightarrow\left\{\begin{aligned}
& b=1\\ 
& a=\dfrac{1}{2}\\ 
\end{aligned}\right.$ ; suy ra $MN=\dfrac{1}{2}h+1$ .\\
Diện tích mặt nước tức thời là $S=\dfrac{M{N^2}\sqrt{3}}{4}=\dfrac{\left(0,5h+1\right)^2\sqrt{3}}{4}$ .\\
Thể tích nước tức thời là $V=\dfrac{1}{3}h\left(S_0+\sqrt{S_0S}+S\right)$ ; $S_0$ là diện tích mặt nước ban đầu (đáy nhỏ).\\
$V=\dfrac{1}{3}h\left(\dfrac{\sqrt{3}}{4}+\sqrt{\dfrac{\sqrt{3}}{4}\cdot\dfrac{\left(0,5h+1\right)^2\sqrt{3}}{4}}+\dfrac{\left(0,5h+1\right)^2\sqrt{3}}{4}\right)$ $=\dfrac{h\sqrt{3}}{12}\left[1+\left(1+0,5h\right)+\left(1+0,5h\right)^2\right]$ .\\
$V=\dfrac{h\sqrt{3}}{12}\left(0,25h^2+1,5h+3\right)=\dfrac{\sqrt{3}}{12}\left(0,25h^3+1,5h^2+3h\right)$ (*).\\
Sau 10 phút thì lượng nước trong chậu là $V=0,5\times 10=5\,\,d{m^3}$ .\\
Do đó $\dfrac{\sqrt{3}}{12}\left(0,25h^3+1,5h^2+3h\right)=5\Rightarrow h\approx 3,27\,\,dm$ (Lưu vào A).\\
Từ (*) đạo hàm hai vế theo $t$ ta được:
$$ \dfrac{dV}{dt} = \dfrac{\sqrt{3}}{12} (0,75h^2 + 3h + 3) \cdot \dfrac{dh}{dt} \quad (**) $$
Thay các giá trị $h = A; V=5; \dfrac{dV}{dt}=0,5$ vào (**) ta được:
$$ \dfrac{dh}{dt} \approx 0,17 \ m/\text{phút} $$
Vậy tốc độ dâng lên của nước trong chậu xấp xỉ $0,17 \ m/\text{phút}$.
}
\end{ex}
\begin{ex}%Câu 21
Trong một trận đấu cờ vua của hai kỳ thủ là Lê Quang Liêm và vua cờ Carlsen; mỗi ván cờ luôn có kẻ thắng người thua (vì nếu hai kỳ thủ hòa thì sẽ bốc thăm để chọn người thắng ván đó). Biết rằng trong mỗi ván đấu, xác suất để anh Liêm dành chiến thắng bằng 0,4; xác suất để Carlsen dành chiến thắng bằng 0,6. Mỗi trận thắng được tính 1 điểm cho kỳ thủ, người thua không được điểm nào. Nếu người nào tạo được cách biệt 2 điểm thì sẽ dành chiến thắng chung cuộc. Tính xác suất để Lê Quang Liêm là người chiến thắng sau cùng (làm tròn đến hàng phần trăm).\shortans{0,31}
\loigiai{
Gọi $P(n)$ là xác suất để Lê Quang Liêm dành chiến thắng khi hiệu số điểm của anh so với Carlsen là $n$ điểm. Ta có $n \in \{-2; -1; 0; 1; 2\}$.
Ta cần tính xác suất chiến thắng của anh Liêm từ trạng thái $n=0$.
\\Theo công thức xác suất toàn phần: $P(0) = 0,4 \times P(1) + 0,6 \times P(-1)$ (1).
\\Ta có $P(2)=1$ và $P(-2)=0$; $P(1)=0,4 \times P(2) + 0,6 \times P(0)$ hay $P(1)=0,4+0,6 \times P(0)$ (2);
$P(-1)=0,4 \times P(0) + 0,6 \times P(-2)$ hay $P(-1)=0,4 \times P(0)$ (3).
\\Thay (2) và (3) vào (1): $P(0) = 0,4 [0,4 + 0,6 \times P(0)] + 0,6 \times 0,4 \times P(0) \Rightarrow P(0) = \dfrac{4}{13} \approx 0,31$.
\\Vậy xác suất để Lê Quang Liêm chiến thắng Carlsen là xấp xỉ $0,31$}
\end{ex}
\begin{ex}%Câu 22
Trong không gian Oxyz, cho ba điểm $A\left(0\,;\,3\,;\,-5\right),\,\,B\left(1\,;\,1\,;\,-5\right),\,\,C\left(4\,;\,3\,;\,-1\right)$ và mặt cầu $\left(S_m\right):{x^2}+y^2+z^2+\left(m-2\right)x+4y+\left(m-2\right)z-3=0$ (m là tham số thực). Gọi (T) là tập hợp tất cả điểm cố định mà mặt cầu $\left(S_m\right)$ luôn đi qua với mọi số thực m và M là một điểm di động trên (T) sao cho thể tích tứ diện MABC đạt giá trị lớn nhất $V_{\max}$ . Tính giá trị lớn nhất $V_{\max}$ đó (làm tròn đến hàng phần chục).
\shortans{15,3}
\loigiai{
 \begin{center}
 \includegraphics[scale=1]{img/HXN-3.22}
 \end{center}
Xét $M(x;y;z)$ là điểm mà $(S_m)$ luôn đi qua với mọi $m$.
\\Ta có: $x^2 + y^2 + z^2 + (m-2)x + 4y + (m-2)z - 3 = 0, \forall m \in \mathbb{R}$\\
$\Leftrightarrow m(x+z) + x^2 + y^2 + z^2 - 2x + 4y - 2z - 3 = 0, \forall m \in \mathbb{R}$\\
$\Leftrightarrow \begin{cases} x+z = 0 \\ x^2 + y^2 + z^2 - 2x + 4y - 2z - 3 = 0 \end{cases}$
\\Tập hợp điểm $M$ là đường tròn $(C)$ là giao tuyến của mặt phẳng $(P): x+z = 0$ và mặt cầu có tâm $I(1;-2;1)$, bán kính $R=3$;
\\Ta có: $d(I,(P)) = \dfrac{|1 \cdot 1 + 0 \cdot (-2) + 1 \cdot 1|}{\sqrt{1^2+0^2+1^2}} = \dfrac{|1+1|}{\sqrt{2}} = \dfrac{2}{\sqrt{2}} = \sqrt{2}$.
\\Vì vậy $(C)$ có bán kính là $r = \sqrt{R^2 - d^2(I,(P))} = \sqrt{7}$ và tâm $J(0;-2;0)$.
\\Mặt phẳng $(ABC)$ có phương trình $2x+y-2z-13=0$; ta thấy $(ABC)$ vuông góc với $(P)$.\\
$ \Rightarrow [\vec{AB},\vec{AC}] = (-8;-4;8) \Rightarrow S_{ABC} = \dfrac{1}{2}\sqrt{(-8)^2+(-4)^2+8^2} = 6$.
\\Thể tích tứ diện $MABC$ là $V_{MABC} = \dfrac{1}{3}d(M,(ABC)) \cdot S_{ABC} = \dfrac{1}{3}d(M,(ABC)) \cdot 6 = 2d(M,(ABC))$.
\\Thể tích này lớn nhất khi và chỉ khi $d(M,(ABC))$ đạt giá trị lớn nhất.
\\Ta có: $d(M,(ABC))_{\max} = r + d(J,(ABC)) = \sqrt{7} + \dfrac{|2(0) + 1(-2) - 2(0) - 13|}{\sqrt{2^2+1^2+(-2)^2}} = \sqrt{7} + 5$.
\\Vì vậy $V_{\max} = 2\sqrt{7} + 10 \approx 15,3$.}
\end{ex}
\Closesolutionfile{ans}
\inputansbox{6,4,3}{ans/ans-HXN-\sode-T,ans/ans-HXN-\sode-TF,ans/ans-HXN-\sode-SA}
% \def\sode{4}
\begin{name}
	{\tenchude}
	{\tendethi}
	{\tentruong}
	{\thoigian}
\end{name}
\Opensolutionfile{ans}[ans/ans-HXN-\sode-T]
\caulc
\begin{ex}%Câu 1
	Cho hàm số $y=f(x)$ xác định trên $\mathbb{R}$ và có bảng biến thiên như hình vẽ sau:
	\begin{center}
		\begin{tikzpicture}[>=stealth]
			\tkzTabInit[nocadre=false,lgt=1,espcl=2.5,deltacl=0.5]{$x$/.7 ,$y'$/.7,$y$/2}
			{$-\infty$ , $-1$ , $2$ , $+\infty$}
			\tkzTabLine{ , - , $0$ , + , $0$ , - , }
			\tkzTabVar{+/$+\infty$ , -/$-2$ , +/$4$ , -/$-\infty$}
		\end{tikzpicture}
	\end{center}
	Hàm số $y=f(x)$ đồng biến trên khoảng nào?
	\choice
	{$\left(-\infty\,;\,\,-1\right)$}
	{$\left(-2\,;\,\,4\right)$}
	{$\left(2\,;\,\,+\infty\right)$}
	{\True $\left(-1\,;\,\,2\right)$}
	\loigiai{
		Chọn D.}
\end{ex}
\begin{ex}%Câu 2
	Trong không gian $Oxyz$ , cho điểm $ M\left(2\,;\,\,-1\,;\,\,4\right)$ và mặt phẳng $(P):\,3x-2y+z+1=0$. Phương trình của mặt phẳng đi qua $ M$ và song song với mặt phẳng $(P)$ là
	\choice
	{$ 2x-2y+4z-21=0$}
	{$ x-2z+1=0$}
	{$10x+9y+5z-74=0$}
	{\True $ 3x-2y+z-12=0$}
	\loigiai{
		Chọn D.\\
		Phương trình của mặt phẳng đi qua $ M$ và song song với mặt phẳng $(P)$ là\\
		$ 3\left(x-2\right)-2\left(y+1\right)+\left(z-4\right)=0$$\Leftrightarrow 3x-2y+z-12=0$.}
\end{ex}
\begin{ex}%Câu 3
	Điều tra về mức lương khởi điểm (đơn vị: triệu đồng) của $ 20$ công nhân, ta có bảng số liệu sau\\
	\centerline{\begin{tabular}{|c|c|c|c|c|c|}
			\hline
			Mức lương & $\left[5\,;\,\,6\right)$ & $\left[6\,;\,\,7\right)$ & $\left[7\,;\,\,8\right)$ & $\left[8\,;\,\,9\right)$ & $\left[9\,;\,\,10\right)$ \\
			\hline
			Tần số    & $ 4$                     & $ 5$                     & $ 5$                     & $ 4$                     & $ 2$                      \\
			\hline
		\end{tabular}}\\
	Phương sai của mẫu số liệu ghép nhóm là (làm tròn đến hàng phần trăm):
	\choice
	{$s^2=0,63$}
	{$s^2=2,52$}
	{$s^2=1,26$}
	{\True $s^2=1,59$}
	\loigiai{
		Chọn D.\\
		Ta có bảng thống kê về mức lương theo giá trị đại diện như sau:\\
		\centerline{\begin{tabular}{|c|c|c|c|c|c|}
				\hline
				Mức lương        & $\left[5\,;\,\,6\right)$ & $\left[6\,;\,\,7\right)$ & $\left[7\,;\,\,8\right)$ & $\left[8\,;\,\,9\right)$ & $\left[9\,;\,\,10\right)$ \\
				\hline
				Giá trị đại diện & $ 5,5$                   & $ 6,5$                   & $ 7,5$                   & $ 8,5$                   & $ 9,5$                    \\
				\hline
				Tần số           & $ 4$                     & $ 5$                     & $ 5$                     & $ 4$                     & $ 2$                      \\
				\hline
			\end{tabular}}\\
		Số trung bình của mẫu số liệu ghép nhóm là\\
		$\bar{x}=\dfrac{4\cdot 5,5+5\cdot 6,5+5\cdot 7,5+4\cdot 8,5+2\cdot 9,5}{20}=7,25$.\\
		Phương sai của mẫu số liệu ghép nhóm là $$s^2=\dfrac{4\cdot{\left(5,5-7,25\right)^2}+5\cdot{\left(6,5-7,25\right)^2}+5\cdot{\left(7,5-7,25\right)^2}+4\cdot{\left(8,5-7,25\right)^2}+2\cdot{\left(9,5-7,25\right)^2}}{20}\approx 1,59$$.}
\end{ex}
\begin{ex}%Câu 4
	Hàm số nào sau đây không là một nguyên hàm của hàm số $f(x)=3x^2+2x-1$ .
	\choice
	{\True $F(x)=x^3+x^2-1$}
	{$F(x)=x^3+x^2-x$}
	{$F(x)=x^3+x^2-x+2025$}
	{$F(x)=x^3+x^2-x-1$}
	\loigiai{
		Chọn A.\\
		$F(x)=\displaystyle\int{f(x)\text{d}x}=\displaystyle\int{\left(3x^2+2x-1\right)\text{d}x}=x^3+x^2-x+C$ (Với $C$ là một hằng số).}
\end{ex}
\begin{ex}%Câu 5
	Trong không gian $ Oxyz$cho ba điểm $ M\left(1\,;\,\,1\,;\,\,1\right),\,\,N\left(2\,;\,\,3\,;\,\,4\right),\,\,P\left(7\,;\,\,7\,;\,\,5\right)$. Tìm tọa độ điểm $ Q$ để tứ giác $ MNPQ$ là hình bình hành
	\choice
	{\True $ Q\left(6\,;\,\,5\,;\,\,2\right)$}
	{$ Q\left(-6\,;\,\,-5\,;\,\,-2\right)$}
	{$ Q\left(-2\,;\,\,-3\,;\,\,-4\right)$}
	{$Q\left(-4\,;\,\,-3\,;\,\,0\right)$}
	\loigiai{
		Chọn A.\\
		Ta có $\overrightarrow{MN}=\left(1\,;\,\,2\,;\,\,3\right)\,,\,\,\overrightarrow{QP}=\left(7-x_Q\,;\,\,7-y_Q\,;\,\,5-z_Q\right)$.\\
		$ MNPQ$ là hình bình hành $\Leftrightarrow\overrightarrow{MN}=\overrightarrow{QP}$$\Leftrightarrow\left\{\begin{aligned}
			 & 1=7-x_Q \\
			 & 2=7-y_Q \\
			 & 3=5-z_Q \\
		\end{aligned}\right.\Leftrightarrow\left\{\begin{aligned}
			 & {x_Q}=6 \\
			 & {y_Q}=5 \\
			 & {z_Q}=2 \\
		\end{aligned}\right.$. Vậy $ Q(6;5;2)$.}
\end{ex}
\begin{ex}%Câu 6
	Cho cấp số cộng $\left(u_n\right)$ có số hạng đầu $u_1=\dfrac{1}{4}$ và công sai $d=-\dfrac{1}{4}$ . Tổng 5 số hạng đầu tiên của cấp số cộng là
	\choice
	{$S_5=\dfrac{5}{4}$}
	{$S_5=\dfrac{4}{5}$}
	{\True $S_5=-\dfrac{5}{4}$}
	{$S_5=-\dfrac{4}{5}$}
	\loigiai{
		Chọn C.\\
		Sử dụng công thức $S_n=\dfrac{n\left[2u_1+\left(n-1\right)d\right]}{2}$ , ta có: $S_5=\dfrac{5\left[2\cdot\dfrac{1}{4}+\left(5-1\right)\cdot\left(-\dfrac{1}{4}\right)\right]}{2}=-\dfrac{5}{4}$ .}
\end{ex}
\begin{ex}%Câu 7
	Nghiệm của phương trình $3^x=10$ là
	\choice
	{$ x=\dfrac{10}{3}$}
	{\True $ x=\log_310$}
	{$ x=\log_{10}3$}
	{$ x=\dfrac{10}{3}$}
	\loigiai{
		Chọn B.\\
		Ta có $3^x=10\Leftrightarrow x=\log_310$.}
\end{ex}
\begin{ex}%Câu 8
	Cho hình chóp $ S.ABCD$ có đáy $ ABCD$ là hình vuông và $ SA\perp\left(ABCD\right)$. Mặt phẳng $\left(SBC\right)$ vuông góc với mặt phẳng nào sau đây?
	\choice
	{$\left(SCD\right)$}
	{$\left(ABCD\right)$}
	{\True $\left(SAB\right)$}
	{$\left(SBD\right)$}
	\loigiai{
		Chọn C.\\
		Ta có $\left\{\begin{aligned}
				 & BC\perp AB \\
				 & BC\perp SA \\
			\end{aligned}\right.$nên $ BC\perp\left(SAB\right)$;\\
		mà $ BC\subset\left(SBC\right)\Rightarrow (SAB) \perp (SBC) $.}
\end{ex}
\begin{ex}%Câu 9
	Đường tiệm cận xiên của đồ thị hàm số $ y=\dfrac{x^2-3x+4}{x+2}$ là đường thẳng
	\choice
	{$ y=-x+1$}
	{$ y=x-1$}
	{\True $ y=x-5$}
	{$ y=-x-5$}
	\loigiai{
		Chọn C.\\
		Ta có $ y=\dfrac{x^2-3x+4}{x+2}$$=x-5+\dfrac{14}{x+2}$; mặt khác $\underset{x\to+\infty}{\lim}\,\left[y-\left(x-5\right)\right]=\underset{x\to+\infty}{\lim}\,\dfrac{14}{x+2}=0$.\\
			Do đó tiệm cận xiên của đồ thị hàm số là đường thẳng $y=x-5 $.}
\end{ex}
\begin{ex}%Câu 10
	Cho tứ diện $ABCD$ . Gọi $M\,,\,\,N$ lần lượt là trung điểm của $AD$ và $BC$ . Tổng $\overrightarrow{AB}+\overrightarrow{DC}$ bằng
	\choice
	{$\vec{0}$}
	{$ 2\overrightarrow{AD}$}
	{$ 2\overrightarrow{NM}$}
	{\True $ 2\overrightarrow{MN}$}
	\loigiai{
		Chọn D.\\
		Ta có: $\overrightarrow{AB}+\overrightarrow{DC}=\overrightarrow{AM}+\overrightarrow{MN}+\overrightarrow{NB}+\overrightarrow{DM}+\overrightarrow{MN}+\overrightarrow{NC}$ $=\left(\overrightarrow{AM}+\overrightarrow{DM}\right)+2\overrightarrow{MN}+\left(\overrightarrow{NB}+\overrightarrow{NC}\right)=2\vec{MN}.$\\
		(Vì $M,\,N$ lần lượt là trung điểm của $AD$ và $BC$ nên $\overrightarrow{AM}+\overrightarrow{DM}=\overrightarrow{0}\,,\,\,\overrightarrow{NB}+\overrightarrow{NC}=\overrightarrow{0}$).}
\end{ex}
\begin{ex}%Câu 11
	Tiếp tuyến của đồ thị hàm số$ y=x^3-3x^2-2$ có hệ số góc $ k=-3$ có phương trình là
	\choice
	{$ y=-3x-7$}
	{$ y=-3x+7$}
	{$ y=-3x+1$}
	{\True $ y=-3x-1$}
	\loigiai{
		Chọn D\\
		Đạo hàm $y'=3x^2-6x$. Gọi $\left(x_0\,;\,\,y_0\right)$ là tiếp điểm của tiếp tuyến với đồ thị hàm số.\\
		Hệ số góc tiếp tuyến $ k=-3$ nên $ 3x_0^2-6x_0=-3\Rightarrow x_0^2-2x_0+1=0\Rightarrow{x_0}=1\Rightarrow{y_0}=-4$.\\
		Phương trình tiếp tuyến của đồ thị hàm số là: $ y=-3\left(x-1\right)-4\,\,\,\text{hay} y=-3x-1$$ $.}
\end{ex}
\begin{ex}%Câu 12
	\immini[thm]{ Đường gấp khúc trong hình vẽ là đồ thì hàm số $ y=f(x)$ trên đoạn $\left[-2\,;\,\,3\right]$. Tích phân $\displaystyle\int\limits_{-2}^3f(x)\text{d}x$ bằng
		\choice
		{\True $\dfrac{13}{2}$}
		{$\dfrac{17}{2}$}
		{$\dfrac{15}{2}$}
		{$\dfrac{5}{2}$}}{\includegraphics[scale=.8]{img/HXN-4.12}}
	\loigiai{
	Chọn A.\\
	Giá trị tích phân $\displaystyle\int\limits_{-2}^3f(x)\text{d}x$ chính là tổng diện tích hai tam giác ABC và CDE như hình vẽ.\\
	Ta có: $\displaystyle\int\limits_{-2}^3f(x)\text{d}x=S_{ABK}+S_{BCL}=\dfrac{1}{2}\cdot 3\cdot 3+\dfrac{1}{2}\cdot 2\cdot 2=\dfrac{13}{2}$.}
\end{ex}

\Closesolutionfile{ans}
\cauds
\Opensolutionfile{ans}[ans/ans-HXN-\sode-TF]


\begin{ex}
	\immini[thm]{Một mô hình trò chơi vòng quay ở công viên có chiều cao tối đa 23 $m$ so với mặt đất, bán kính vòng quay là 20 $m$. Hai bạn Hoa và Mai cùng chơi chung lượt quay và ngồi trong hai cabin B, C mà góc $BOC = 90^\circ$ (hình vẽ); $\alpha$ là góc lượng giác hợp bởi tia đầu OA, tia cuối OB.
		Xét tính đúng sai các mệnh đề sau:}{\includegraphics[scale=.5]{img/HXN-4.13}}
	\choiceTF
	{\True Chiều cao của B so với mặt đất là $h_B = 23 + 20\sin \alpha$ ($mét$)}
	{\True Khi $\alpha = 45^\circ$ thì chiều cao của B so với mặt đất là $37,14$ $m$ (làm tròn kết quả đến hàng phần trăm)}
	{\True Chiều cao của C so với mặt đất là $h_C = 23 - 20\cos \alpha$ ($mét$)}
	{ Khi B ở vị trí có độ cao 33 $m$ thì C ở độ cao 13 $m$ so với mặt đất?}
	\loigiai{
	\begin{itemchoice}
		\itemch Chiều cao của B là $h_B = 23 + 20\sin \alpha$ ($mét$).
		\itemch Với $\alpha = 45^\circ$ thì $h_B = 23 + 20\sin 45^\circ \approx 37,14$ $m$.
		\itemch Chiều cao của C là $h_C = 23 + 20\sin (OA, OC) = 23 + 20\sin (\alpha - 90^\circ) = 23 - 20\cos \alpha$ ($mét$).
		\itemch Khi B ở vị trí có độ cao 33 $m$ thì $h_B = 23 + 20\sin \alpha = 33 \Rightarrow \sin \alpha = \dfrac{1}{2}$.\\
		Khi đó $\cos \alpha = \pm \sqrt{1-\sin^2 \alpha} = \pm \dfrac{\sqrt{3}}{2}$.\\
		Do đó $h_C = 23 - 20\cos \alpha =23\pm 10\sqrt{3}$.
	\end{itemchoice}
	}
\end{ex}
\begin{ex}
	Song sinh có thể là cùng trứng (identical) hoặc khác trứng (fraternal). Biết rằng $\dfrac{1}{3}$ số cặp song sinh là cùng trứng. Hiển nhiên, song sinh cùng trứng phải cùng giới tính; song sinh khác trứng có thể cùng hoặc khác giới tính. Giả sử song sinh cùng trứng có xác suất là hai bé trai hoặc hai bé gái như nhau, trong khi với song sinh khác trứng thì tất cả bốn khả năng đều có xác suất như nhau. Một nhà khảo sát tìm gặp ngẫu nhiên một người phụ nữ đang mang thai đôi.
	Xét tính đúng sai các mệnh đề sau:
	\choiceTF
	{\True Xác suất để người phụ nữ mang thai đôi là bé gái bằng $0,5$ biết rằng đây là cặp song sinh cùng trứng}
	{ Xác suất để thai đôi của người phụ nữ là một cặp trai gái bằng $0,3$}
	{Xác suất để thai đôi không cùng trứng và cũng không phải con trai bằng $\dfrac{1}{3}$}
	{\True Xác suất để người phụ nữ mang thai đôi là cùng trứng bằng $0,5$ biết rằng cô ấy hạ sinh được hai bé gái}
	\loigiai{
		Gọi A là biến cố: “Người phụ nữ mang thai đôi cùng trứng”, ký hiệu BB, GG, BG, GB lần lượt chỉ các biến cố thai đôi là trai-trai, gái-gái, trai-gái, gái trai. Ta có sơ đồ hình cây sau:
        \begin{center}
			\includegraphics[scale=.8]{img/HXN-4.14a}
		\end{center}
        \begin{itemchoice}
            \itemch Mệnh đề đúng. Ta có $P(GG|A) = \dfrac{1}{2}$.
			\itemch Mệnh đề sai. Ta có $P(BG \cup GB | \bar{A}) = \dfrac{2}{3} (\dfrac{1}{4} + \dfrac{1}{4}) = \dfrac{1}{3}$.
			\itemch Mệnh đề sai. Ta có $P(\bar{A} \cap GG) = \dfrac{2}{3} \cdot \dfrac{1}{4} = \dfrac{1}{6}$.
			\itemch Mệnh đề đúng. Ta có $P(GG) = \dfrac{1}{3} \cdot \dfrac{1}{2} + \dfrac{2}{3} \cdot \dfrac{1}{4}= \dfrac{1}{3}$.\\
			Do đó $P(A|GG) = \dfrac{P(A) \cdot P(GG|A)}{P(GG)} = \dfrac{\dfrac{1}{3} \cdot \dfrac{1}{2}}{\dfrac{1}{3}}= 0,5$.
        \end{itemchoice}
	}
\end{ex}
\begin{ex}
	\immini[thm]{ Một bể bơi hình trụ có đường kính $5$m và chiều cao $1$ m; bể được bơm nước vào với tốc độ không đổi $v_0$. Sau khi nước được bơm đầy, bể bị thủng một lỗ ở đáy và nước chảy ra ngoài; bể chảy hết nước trong $8$ giờ. Biết tốc độ giảm chiều cao của bể khi nước chảy ra ngoài vào thời điểm $t$ giờ (tính từ lúc nước đầy bể và ngừng bơm) được cho bởi hàm số $h'(t) = at + b$, với $a, b \in \mathbb{R}$. Lúc nước chảy hết ra ngoài thì tốc độ giảm chiều cao bằng $0$.
		Xét tính đúng sai các mệnh đề sau:}{\includegraphics[scale=.8]{img/HXN-4.15}}
	\choiceTF
	{\True Thể tích của bể bơi sau khi nước được làm đầy là $6,25\pi$ m$^3$}
	{$32a + 1 = 0$ và $4b - 1 = 0$}
	{\True Sau $4$ giờ kể từ lúc bể bị rò, lượng nước bị mất đi bằng $\dfrac{75\pi}{16}$ m$^3$}
	{\True Lượng nước bị rò rỉ ra ngoài một nửa sau $8 - 4\sqrt{2}$ giờ}
	\loigiai{
		Bể nước hình trụ có bán kính đáy $r = 2,5\ m$, chiều cao $h=1\ m$.
		\begin{itemchoice}
			\itemch Thể tích khi đầy là $V = \pi r^2 h = \pi \cdot (2,5)^2 \cdot 1 = 6,25\pi\ m^3$.
			\itemch Ta có $h'(8)=0 \Rightarrow 8a+b=0\ (1)$.\\ Chiều cao của nước thời điểm $t$ là $h(t) = \int\limits_{ }^{ }(at+b)\mathrm{d} t = \dfrac{at^2}{2} + bt + c$. \\Vì $h(0)=1 \Rightarrow c=1$. $h(8)=0 \Rightarrow 32a+8b+1=0\ (2)$. \\Từ (1) và (2) suy ra $a = \dfrac{1}{32}$, $b = -\dfrac{1}{4}$. Khi đó $32a-1=0$; $4b+1=0$ và $h(t) = \dfrac{1}{64}t^2 - \dfrac{1}{4}t + 1$.
			\itemch Chiều cao mực nước trong bể sau 4 giờ là: $h(4) = \dfrac{1}{64} \cdot 4^2 - \dfrac{1}{4} \cdot 4 + 1 = \dfrac{16}{64} - 1 + 1 = \dfrac{1}{4} = 0,25\ m$. Lượng nước còn lại trong bể sau 4 giờ là $\pi r^2 h(4) = \pi \cdot (2,5)^2 \cdot 0,25 = 6,25\pi \cdot 0,25 = 6,25\pi \cdot \dfrac{1}{4} = \dfrac{25\pi}{16}\ m^3$. Lượng nước đã thoát ra sau 4 giờ là $6,25\pi - \dfrac{25\pi}{16} = \dfrac{100\pi}{16} - \dfrac{25\pi}{16} = \dfrac{75\pi}{16}\ m^3$.
			\itemch Lượng nước còn lại khi bể mất một nửa nước là $\dfrac{6,25\pi}{2} = \dfrac{25\pi}{8}\ m^3$. \\Chiều cao tương ứng $h(t_1)$ của bể thỏa mãn $\pi r^2 h(t_1) = \dfrac{25\pi}{8} \Rightarrow h(t_1) = 0,5\ m$.\\ Ta có $h(t_1) = \dfrac{1}{64}t_1^2 - \dfrac{1}{4}t_1 + 1 = 0,5 \Leftrightarrow \dfrac{1}{64}t_1^2 - \dfrac{1}{4}t_1 + 0,5 = 0$. $\left[ \begin{array}{l} t_1 = 8 + 4\sqrt{2} \approx 13,7 > 8 \\ t_1 = 8 - 4\sqrt{2} \approx 2,3 \in (0; 8) \end{array} \right.$. \\Ta thấy $t_1 = 8 - 4\sqrt{2}$ (giờ) thỏa mãn đề bài.
		\end{itemchoice}
	}
\end{ex}
\begin{ex}
	Trong một mô hình game 3D, với hệ trục tọa độ thích hợp, người chơi đứng cùng với khẩu súng của anh ta được mô phỏng như một chất điểm di chuyển trên mặt phẳng $(P): x - 2y + 2z - 3 = 0$ và nhắm bắn các mục tiêu di động trên mặt cầu $(S)$ có phương trình $x^2 + y^2 + z^2 +2x - 4y - 2z + 5 = 0$. Người chơi vẫn có thể bắn trúng mục tiêu nếu nó di chuyển trên bán cầu khuất phía sau tầm nhìn. Sau khi trò chơi bắt đầu, anh ta quyết định nhắm bắn theo phương vector $\vec{u} = (1; 0; 1)$.
	Xét tính đúng sai các mệnh đề sau:
	\choiceTF
	{\True Mặt cầu $(S)$ có tâm $I(-1; 2; 1)$ và bán kính $R = 1$}
	{\True Mặt phẳng $(P)$ và mặt cầu $(S)$ không có điểm chung}
	{\True Người chơi đứng ở vị trí giao điểm của $(P)$ và $Ox$, khoảng cách từ tâm quả cầu đến đường bay viên đạn bằng $\dfrac{\sqrt{66}}{2}$}
	{\True Khoảng cách lớn nhất từ vị trí người bắn đến mục tiêu bằng $3\sqrt{2}$}
	\loigiai{
		\begin{enumerate}
			\itemch Mệnh đề đúng. Mặt cầu $(S)$ có tâm $I(-1; 2; 1)$ và bán kính $R = 1$.
			\itemch Mệnh đề đúng. Mặt phẳng $(P)$ và mặt cầu $(S)$ không có điểm chung.\\
			      Ta có $d(I, (P)) = \dfrac{|-1-4+2-3|}{\sqrt{1+4+4}} = 2$. $2 \ge R$. Do đó $(P)$ và mặt cầu $(S)$ không có điểm chung.
			\itemch Mệnh đề đúng. \\
			      Giao điểm của $(P)$ và $Ox$ là điểm $M(0; 3; 0) \Rightarrow \vec{IM} = (4; -2; -1)$; $[\vec{IM}, \vec{u}] = (-2; -5; 2)$.\\
			      Khoảng cách từ $I$ đến đường bay viên đạn là $d = \dfrac{|[\vec{IM}, \vec{u}]|}{|\vec{u}|} = \dfrac{\sqrt{(-2)^2 + (-5)^2 + 2^2}}{\sqrt{1^2 + 0^2 + 1^2}} =\dfrac{\sqrt{66}}{2}$.
			\itemch Mệnh đề đúng. Gọi $M$ thuộc $(P)$, $N$ thuộc $(S)$ theo thứ tự là vị trí người chơi và vị trí mục tiêu đang bắn; $H$ là hình chiếu của điểm $N$ trên $(P)$.\\
			      \begin{center}
				      \includegraphics[scale=1]{img/HXN-4.16a}
			      \end{center}
			      $MN$ hợp với $(P)$ một góc $\phi$ thỏa mãn $\sin \phi = \dfrac{|\vec{u}\cdot\vec{n_P}|}{|\vec{u}|\cdot|\vec{n_P}|} = \dfrac{|1+0+2|}{\sqrt{1+0+1}\cdot\sqrt{1+4+4}} = \dfrac{\sqrt{2}}{2}$.\\
			      Xét tam giác $MNH$ vuông tại $H$, ta có
			      $\sin \varphi = \dfrac{NH}{MN} \Rightarrow MN = \dfrac{NH}{\sin \varphi}$ hay $\boxed{MN = \sqrt{2}NH}$.
			      Dễ thấy $MN$ lớn nhất khi và chỉ khi $NH$ lớn nhất; mà
			      $NH \le d(I, (P)) + R = 2+1=3$.
			      Do đó $MN$ lớn nhất bằng $\boxed{3\sqrt{2}}$; khi đó $N, I, H$ nằm trên đường thẳng vuông góc với $(P)$.
		\end{enumerate}
	}
\end{ex}

\Closesolutionfile{ans}
\caukq
\Opensolutionfile{ans}[ans/ans-HXN-\sode-SA]


\begin{ex}
	\immini[thm]{ Một tấm cầu dốc kê bậc thềm được làm bằng kim loại như hình vẽ. Biết chiều cao tối đa của cầu dốc là $0,3 \, m$ và bề mặt cầu là hình vuông có cạnh bằng $1 \, m$. Hãy tính góc tạo bởi đường chéo bề mặt cầu dốc với mặt phẳng sàn nhà theo đơn vị độ (làm tròn kết quả đến hàng phần chục).
		\shortans{12,2}}{\includegraphics[scale=.8]{img/HXN-4.17}}
	\loigiai{
		Xét mô hình cầu dốc với các kí hiệu như hình vẽ.
		Vì $AF$ là hình chiếu của $BF$ trên mặt phẳng $(ACFD)$ nên góc giữa $(BF, (ACFD))$ là góc $\angle BFA$.
		Hình vuông $BCFE$ có cạnh bằng $1 \, m$ nên đường chéo $BF = \sqrt{1^2 + 1^2} = \sqrt{2} \, m$; $AB = 0,3 \, m$.
		Tam giác $ABF$ vuông tại $A$ có:
		$$ \sin \angle BFA = \frac{AB}{BF} = \frac{0,3}{\sqrt{2}} = \frac{0,3 \sqrt{2}}{2} = \frac{3 \sqrt{2}}{20} $$
		$$ \Rightarrow \angle BFA \approx 12,2^\circ $$
		Vậy góc giữa $(BF, (ACFD)) = \angle BFA \approx 12,2^\circ$.
	}
\end{ex}
\begin{ex}%Câu 1
	Có hai vợ chồng đã nghĩ ra một trò chơi đầy trí tuệ như sau: Họ sử dụng hai ly nước giống hệt nhau, mỗi ly chứa tối đa 240 ml nước. Ban đầu, người vợ có một ly nước đầy và người chồng có một cái ly rỗng. Bước thứ nhất người vợ rót 1/2 lượng nước trong ly của mình sang ly của người chồng; bước tiếp theo người chồng lại rót 1/3 lượng nước trong ly của mình sang cho ly người vợ. Quá trình này cứ tiếp tục mà mỗi lần rót thì mẫu số được cộng thêm 1; trò chơi này hấp dẫn đến mức cả hai người thực hiện đến bước thứ 100 thì dừng lại, hỏi lượng nước trong ly người chồng khi đó là bao nhiêu ml? (Làm tròn kết quả đến hàng đơn vị, giả sử trong quá trình rót nước không có giọt nước nào tràn ra ngoài).
	\shortans{119}
	\loigiai{
		Ta có thể thực hiện việc rót theo sơ đồ sau:
		\begin{center}
			\includegraphics[scale=.5]{img/HXN-4.18a}
		\end{center}
		Quá trình này được lặp đi lặp lại và ta thấy rằng trong các bước lẻ (người vợ rót nước cho người chồng) thì lượng nước hai ly bằng nhau.\\
		• Bước thứ 99 thì lượng nước hai ly bằng nhau.\\
		• Bước thứ 100 (người chồng rót 1/101 nước trong ly cho vợ), lượng nước trong ly người chồng là $\dfrac{1}{2}V-\dfrac{1}{101}\cdot\dfrac{1}{2}V=\dfrac{50V}{101}=\dfrac{50\cdot 240}{101}\approx\,119\,ml$ .}
\end{ex}
\begin{ex}%Câu 2
	Cho tập hợp $X=\left\{ 3\,;\,\,4\,;\,\,5\,;\,\,6\right\}$ và Y là tập hợp tất cả số tự nhiên có 2025 chữ số lấy từ X. Chọn ngẫu nhiên một số trong tập Y, biết rằng xác suất để số đó chia hết cho 3 bằng $\dfrac{1}{3}\left(\dfrac{1}{2^a}+b\right)$ , trong đó $a\,,\,\,b$ là các số nguyên dương. Tính giá trị $a-18b$ .
	\shortans{4031}
	\loigiai{
	Gọi $A_n, B_n$ là tập con của $Y$ gồm các số có $n$ chữ số với $A_n$ là tập các số chia hết cho 3 và $B_n$ là tập các số không chia hết cho 3.
	\begin{itemize}
		\item Với mỗi số thuộc $A_n$, có hai cách thêm vào cuối một chữ số 3 hoặc 6 để được $A_{n+1}$ và hai cách thêm vào cuối chữ số 4 hoặc chữ số 5 để được $B_{n+1}$.
		      \begin{itemize}
			      \item Ví dụ: $36 \in A_2$, nếu ta thêm 3 hoặc 6 vào sau nó thì được 363 hoặc 366 đều thuộc $A_3$.
			      \item Ví dụ: $36 \in A_2$, nếu ta thêm 4 hoặc 5 vào sau nó thì được 364 hoặc 365 đều thuộc $B_3$.
		      \end{itemize}
		\item Với mỗi số thuộc $B_n$, có một cách thêm vào cuối một chữ số 4 (hoặc chữ số 5) để được $A_{n+1}$ và có ba cách thêm một số để được $B_{n+1}$.
		      \begin{itemize}
			      \item Ví dụ: $34 \in B_2$, nếu ta thêm 5 vào sau nó thì được 345 thuộc $A_3$.
			      \item Ví dụ: $34 \in B_2$, nếu ta thêm 3 hoặc 4 hoặc 6 vào sau nó thì được 343 hoặc 344 hoặc 346 đều thuộc $B_3$.
		      \end{itemize}
	\end{itemize}
	Do đó ta có:
	\begin{align} |A_{n+1}| &= 2|A_n| + |B_n| & \\ |B_{n+1}| &= 2|A_n| + 3|B_n| & \end{align}
	Thay (1) vào (2), ta được: $|B_{n+1}| = 2|A_n| + 3(|A_{n+1}| - 2|A_n|) = 3|A_{n+1}| - 4|A_n| \quad (3)$.
	\\ Từ (3) suy ra $|B_n| = 3|A_n| - 4|A_{n-1}| \quad (4)$.
	\\Thay (4) vào (1) suy ra $|A_{n+1}| = 2|A_n| + 3|A_n| - 4|A_{n-1}| = 5|A_n| - 4|A_{n-1}|$.
	\\ Do đó $|A_{n+1}| = 5|A_n| - 4|A_{n-1}|$. Xét phương trình đặc trưng $t^2 = 5t - 4 \Rightarrow t=1 \lor t=4$.
	\\Phương trình $(*)$ có nghiệm dạng $|A_n| = a \cdot 1^n + b \cdot 4^n$. $|A_n| = a + b \cdot 4^n$ (a, b là các tham số).
	$A_1 = \{3; 6\} \Rightarrow |A_1| = 2 \Rightarrow a+4b = 2 \quad (5)$.
	\\$A_2 = \{36; 63; 33; 66; 45; 54\} \Rightarrow |A_2| = 6 \Rightarrow a+16b = 6 \quad (6)$.
		\\Từ (5) và (6) ta giải ra được $a = \dfrac{2}{3}$, $b = \dfrac{1}{3}$.
		\\Do đó $|A_{2025}| = \dfrac{2+4^{2025}}{3}$.
		\\Xác suất số đó chia hết cho 3 từ tập $Y$ là:
	$\dfrac{2+4^{2025}}{3} \times \dfrac{1}{4^{2025}} = \dfrac{1}{3} (\dfrac{2}{3 \cdot 4^{2050}} + \dfrac{1}{3 \cdot 4^{2051}})$.
		\\ Do đó $a = 4049, b = 1 \Rightarrow |A_{n}| = a - 18b = 4031$.}
\end{ex}
\begin{ex}
	\immini[thm]{ Cho hàm số $y = -\dfrac{1}{3}x^3 + \dfrac{3}{4}x^2 + 3x$ có đồ thị $(C)$ và đường thẳng $d$ đi qua gốc tọa độ tạo thành hai miền phẳng có diện tích $S_1$ và $S_2$ như hình vẽ. Biết $S_1 = \dfrac{27}{4}$ và $S_2 = \dfrac{m}{n}$ (hai số $m, n$ là nguyên tố cùng nhau), tính giá trị $2m-n$.\shortans{142}}{\includegraphics[scale=.8]{img/HXN-4.20}}
	\loigiai{
		\begin{center}
			\includegraphics[scale=.7]{img/HXN-4.20a}
		\end{center}
		Gọi $a>0$ là hoành độ giao điểm của $(C)$ và $d$.\\
		Đường thẳng $d$ có hệ số góc $k =\dfrac{-\dfrac{1}{3}a^3 + \dfrac{3}{4}a + 3a}{a} = -\dfrac{1}{2}a^2+\dfrac{3}{4}a + 3$.\\
		Mặt khác $d$ đi qua gốc tọa độ nên phương trình là $y = \left(-\dfrac{1}{2}a^2+\dfrac{3}{4}a + 3\right)x$.\\
		Ta có: $S_1 = \int\limits_{0}^{a} \left[-\dfrac{1}{2}x^3 +\dfrac{3}{4}x^2 + 3x - \left(-\dfrac{1}{2}a^2 + \dfrac{3}{4}a + 3\right)x\right] \mathrm{d} x$
		$\Leftrightarrow \dfrac{27}{4} = \dfrac{1}{8}a^4-\dfrac{1}{8}a^3 \Rightarrow a=3 > 0$.
		Ta có phương trình $d: y = \dfrac{3}{4}x$.
		Khi đó phương trình hoành độ giao điểm của $(C)$ và $d$ là:
		$\dfrac{3}{4}x - \left(-\dfrac{1}{2}x^3 + \dfrac{3}{4}x^2 + 3x\right) = 0 \Leftrightarrow x= 3 ; x = 0 ; x = -\dfrac{3}{2}$.\\
		Do đó $S_2 = \int\limits_{-\frac{3}{2}}^{0} \left|\dfrac{1}{2}x^3 - \dfrac{3}{4}x^2 - \dfrac{9}{4}x\right| \mathrm{d} x = \dfrac{135}{128} = \dfrac{m}{n}$. Từ đó suy ra $2m-n=142$.
	}
\end{ex}
\begin{ex}%Câu 4
	\immini[thm]{Một mảnh đất hình chữ nhật có kích thước $40\,m\times 50\,m$ đang được người chủ trồng cỏ tự nhiên. Vào buổi sáng, khi mặt trời vừa lên, mảnh đất này bị một mái nhà xưởng gần đó chắn ánh sáng. Khi mặt trời lên cao hơn, ánh sáng đã chiếu từ từ lên mảnh đất. Ta xem ranh giới giữa phần được chiếu sáng và phần tối là các đường thẳng song song thay đổi.
		Có thời điểm đường ranh giới này đi qua hai điểm A, M như hình vẽ (M là trung điểm một cạnh hình chữ nhật).
		Khi diện tích phần tối của mảnh đất bằng $75\,\,m^2$ , người ta đo được tốc độ giảm cạnh theo phương AD bằng 2 cm/s; hỏi tốc độ giảm diện tích phần tối của mảnh đất là bao nhiêu $c{m^2}$ /s? Kết quả được làm tròn đến hàng phần chục.\shortans{3873}}{\includegraphics[scale=1.1]{img/HXN-4.21}}

	\loigiai{
	\begin{center}
		\includegraphics[scale=1]{img/HXN-4.21a}
	\end{center}
	Xét tam giác ABM vuông tại B có $\tan\widehat{BAM}=\dfrac{BM}{AB}=\dfrac{2}{5}$ .\\
	Đặt $x=DF\in\left[0\,;\,\,40\right]\,,\,\,y=DE\in\left[0\,;\,\,50\right]$ (x, y thay đổi (giảm) vì ánh sáng ngày càng lan rộng).\\
	Vì $EF\text{//}AM$ nên $\tan\widehat{FED}=\tan\widehat{BAM}\Leftrightarrow\dfrac{x}{y}=\dfrac{2}{5}\Leftrightarrow y=\dfrac{5x}{2}$ .\\
	Diện tích phần tối tức thời là $S=\dfrac{1}{2}xy=\dfrac{1}{2}x\cdot\dfrac{5x}{2}=\dfrac{5}{4}x^2$ (1).\\
	Khi diện tích phần tối bằng $75\,\,m^2$ thì $\dfrac{5}{4}{x^2}=75\Rightarrow{x^2}=60\Rightarrow x=2\sqrt{15}$ m.\\
	Đạo hàm hai vế của (1) theo biến t ta được: $ \dfrac{dS}{dt}=\dfrac{5}{2}x .\dfrac{dx}{dt}\quad (2)$ .\\
	Thay $x=2\sqrt{15}\,\,m=200\sqrt{15}\,\,cm$ và $\dfrac{dx}{dt}=2$ cm/s vào (2), ta được $\dfrac{dS}{dt}\approx 3873\,\,c{m^2}$ /s.}
\end{ex}
\begin{ex}%Câu 5
	Trong không gian $Oxyz$ , cho điểm $A\left(-2\,;\,\,6\,;\,\,0\right)$ và mặt phẳng $\left(\alpha\right):3x+4y+89=0$ .
	Đường thẳng $d$ thay đổi nằm trên mặt phẳng $\left(Oxy\right)$ và luôn đi qua điểm $A$ .
	Gọi $H$ là hình chiếu vuông góc của $M\left(4\,;\,\,-2\,;\,\,3\right)$ trên đường thẳng d.
	Khoảng cách nhỏ nhất từ $H$ đến mặt phẳng $\left(\alpha\right)$ bằng bao nhiêu?
	\shortans{15}
	\loigiai{
	\begin{center}
		\includegraphics[scale=1]{img/HXN-4.22}
	\end{center}
	Gọi $K$ là hình chiếu vuông góc của $M$ lên $\left(Oxy\right)$ , suy ra $K\left(4\,;\,\,-2\,;\,\,0\right)$ .\\
	Vì $\left\{\begin{aligned}
			 & AH\perp MK\,\,\,\left(\text{do}\,\,\,MK\perp\left(Oxy\right)\right) \\
			 & AH\perp MH                                                          \\
		\end{aligned}\right.\Rightarrow AH\perp\left(MKH\right)\Rightarrow AH \perp KH $ .\\
	Khi đó $H$ luôn thuộc đường tròn $(C)$ có tâm là trung điểm $I\left(1\,;\,\,2\,;\,\,0\right)$ của đoạn $AK$ , bán kính $R=\dfrac{AK}{2}=5$ .\\
	Gọi $\Delta=\left(\alpha\right)\cap\left(Oxy\right)$ , ta thấy $\left(\alpha\right)\perp\left(Oxy\right)$ (vì $\vec{n}_{\left(\alpha\right)}\cdot\vec{k}=0$) . Khi đó: $d\left(I\,,\,\,\Delta\right)=d\left(I\,,\,\,\left(\alpha\right)\right)=20$ .\\
	Khoảng cách nhỏ nhất từ $H$ đến mặt phẳng $\left(\alpha\right)$ là $d{\left(H\,,\,\,\left(\alpha\right)\right)_{\text{min}}}=d{\left(H\,,\,\,\Delta\right)_{\text{min}}}=d\left(H_0\,,\,\,\Delta\right)$ $=d\left(I\,,\,\,\Delta\right)-R=20-5=15$ .}
\end{ex}
\Closesolutionfile{ans}
\inputansbox{6,4,3}{ans/ans-HXN-\sode-T,ans/ans-HXN-\sode-TF,ans/ans-HXN-\sode-SA}
% \def\sode{5}
\begin{name}
	{\tenchude}
	{\tendethi}
	{\tentruong}
	{\thoigian}
\end{name}
\Opensolutionfile{ans}[ans/ans-HXN-\sode-T]
\caulc
\begin{ex}%Câu 1
	Cho cấp số cộng $\left(u_n\right)$ có $u_2=3,\,\,u_3=5$. Công sai $d$ của cấp số cộng là:
	\choice
	{1}
	{\True 2}
	{8}
	{4}
	\loigiai{
		Chọn B.\\
		Ta có: $\,u_3=u_2+d\Leftrightarrow 5=3+d\Leftrightarrow d=2$ .}
\end{ex}
\begin{ex}%Câu 2
	\immini[thm]{ Cho hàm số có đồ thị như hình vẽ bên. Hàm số đã cho đồng biến trên khoảng nào sau đây?
		\choice
		{$\left(-\infty\,;\,\,-1\right)$}
		{$\left(-1\,;\,\,1\right)$}
		{$\left(-2\,;\,\,1\right)$}
		{$\left(1\,;\,\,+\infty\right)$}}{\includegraphics[scale=.8]{img/HXN-5.2}}
	\loigiai{
		Chọn B.\\
		Từ đồ thị hàm số, ta thấy hàm số đồng biến trên khoảng $\left(-1\,;\,\,1\right)$.}
\end{ex}
\begin{ex}%Câu 3
	Cho hình lăng trụ đứng $ ABC.A'{B}'{C}'$ có đáy là tam giác vuông cân tại $ B$ với $ AB=a$ và $A'B=a\sqrt{3}$. Thể tích khối lăng trụ $ ABC.A'{B}'{C}'$ là
	\choice
	{$\dfrac{a^3\sqrt{3}}{2}$}
	{$\dfrac{a^3}{6}$}
	{$\dfrac{a^3}{2}$}
	{\True $\dfrac{a^3\sqrt{2}}{2}$}
	\loigiai{
	Chọn D.\\
	Ta có $ A{A}'=\sqrt{A'{B^2}-A{B^2}}=a\sqrt{2}$, $S_{ABC}=\dfrac{1}{2}A{B^2}=\dfrac{a^2}{2}$.\\
	Thể tích khối lăng trụ là $ V=A{A}'\cdot{S_{ABC}}=\dfrac{a^3\sqrt{2}}{2}$.}
\end{ex}
\begin{ex}%Câu 4
	Gọi $ S$ là diện tích hình phẳng giới hạn bởi các đường $ y=\text{e}^x$, $ y=0$, $ x=0$, $ x=2$. Mệnh đề nào dưới đây đúng?
	\choice
	{$ S=\pi\displaystyle\int\limits_0^2\text{e}^{2x}\text{d}x$}
	{\True $ S=\displaystyle\int\limits_0^2\text{e}^x\text{d}x$}
	{$ S=\pi\displaystyle\int\limits_0^2\text{e}^x\text{d}x$}
	{$ S=\pi\displaystyle\int\limits_0^2\text{e}^x\text{d}x$}
	\loigiai{
		Chọn B.\\
		Diện tích hình phẳng giới cần tính là $ S=\displaystyle\int\limits_0^2e^x\text{d}x$.}
\end{ex}
\begin{ex}%Câu 5
	Mặt phẳng đi qua ba điểm $ A\left(0\,;\,\,0\,;\,\,2\right)$, $ B\left(1\,;\,\,0\,;\,\,0\right)$ và $C\left(0\,;\,\,3\,;\,\,0\right)$ có phương trình là
	\choice
	{\True $\dfrac{x}{1}+\dfrac{y}{3}+\dfrac{z}{2}=1$}
	{$\dfrac{x}{1}+\dfrac{y}{3}+\dfrac{z}{2}=-1$}
	{$\dfrac{x}{2}+\dfrac{y}{1}+\dfrac{z}{3}=1$}
	{$\dfrac{x}{2}+\dfrac{y}{1}+\dfrac{z}{3}=-1$}
	\loigiai{
		Chọn A.\\
		Mặt phẳng (ABC) chắn các trục tọa độ Ox, Oy, Oz lần lượt tại $ A\left(0\,;\,\,0\,;\,\,2\right)$, $ B\left(1\,;\,\,0\,;\,\,0\right)$ và $C\left(0\,;\,\,3\,;\,\,0\right)$ nên có phương trình $\dfrac{x}{1}+\dfrac{y}{3}+\dfrac{z}{2}=1$.}
\end{ex}
\begin{ex}%Câu 6
	Nếu $\displaystyle\int\limits_{-1}^2f(x)\text{d}x=5$ thì $\displaystyle\int\limits_{-1}^24f(x)\text{d}x$ bằng:
	\choice
	{\True $ 20$}
	{$ 10$}
	{$\dfrac{5}{2}$}
	{$\dfrac{5}{4}$}
	\loigiai{
		Chọn A.\\
		Ta có: $\displaystyle\int\limits_{-1}^24f(x)\text{d}x=4\displaystyle\int\limits_{-1}^2f(x)\text{d}x=4.5=20$.}
\end{ex}
\begin{ex}%Câu 7
	Trong không gian tọa độ $ Oxyz$, mặt cầu $(S)$ có tâm $ I\left(2\,;\,1;\,-1\right)$ và đường kính 6 có phương trình là
	\choice
	{$(x-2)^2+(y-1)^2+(z+1)^2=36$}
	{\True $(x-2)^2+(y-1)^2+(z+1)^2=9$}
	{$(x+2)^2+(y+1)^2+(z-1)^2=9$}
	{$(x+2)^2+(y+1)^2+(z-1)^2=36$}
	\loigiai{
		Chọn B.\\
		Mặt cầu $ (S)$ có tâm $ I(2;1;-1)$, bán kính $ R=3$ nên có phương trình là\\
		$\left(x-2\right)^2+\left(y-1\right)^2+\left(z+1\right)^2=9$.}
\end{ex}
\begin{ex}%Câu 8
	Một mẫu số liệu ghép nhóm về chiều cao của một lớp (đơn vị là centimét) có phương sai là $ 6,25$. Độ lệch chuẩn của mẫu số liệu đó bằng bao nhiêu cm:
	\choice
	{\True $ 2,5$ }
	{$ 12,5$}
	{$ 3,125$}
	{$ 42,25$}
	\loigiai{
		Chọn A.\\
		Độ lệch chuẩn của mẫu số liệu là: $\sqrt{6,25}=2,5$.}
\end{ex}
\begin{ex}%Câu 9
	Tìm giá trị lớn nhất $ M$ của hàm số $ y=\dfrac{3x-1}{x-3}$ trên đoạn $\left[0\,;\,2\right]$.
	\choice
	{$ M=5$}
	{$ M=-5$}
	{\True $ M=\dfrac{1}{3}$}
	{$ M=-\dfrac{1}{3}$}
	\loigiai{
		Chọn C.\\
		Ta có: $y'=\dfrac{-8}{\left(x-3\right)^2}<\,0\,,\,\,\forall x\in\left[0\,;\,\,2\right]$. Hàm số luôn nghịch biến trên $\left[0\,;\,2\right]$.\\
		Ta tính được: $ y(0)=\dfrac{1}{3}$, $ y(2)=\,-5$.\\
		Do đó giá trị lớn nhất của hàm số trên $\left[0\,;\,2\right]$ là $ M=y(0)=\dfrac{1}{3}$.}
\end{ex}
\begin{ex}%Câu 10
	Cho hai biến cố $ A\,,\,\,B$ với $ 0<P(B)<1.$ Phát biểu nào sau đây là đúng?
	\choice
	{$ P(A)=P\left(\overline{B}\right).P\left(A|B\right)+P(B).P\left(A|\overline{B}\right)$}
	{$ P(A)=P(B).P\left(A|B\right)-P\left(\overline{B}\right).P\left(A|\overline{B}\right)$}
	{$ P(A)=P\left(\overline{B}\right).P\left(A|\overline{B}\right)-P(B).P\left(A|B\right)$}
	{\True $ P(A)=P(B).P\left(A|B\right)+P\left(\overline{B}\right).P\left(A|\overline{B}\right)$}
	\loigiai{
		Chọn D.\\
		Theo công thức xác suất toàn phần ta có: $ P(A)=P(B).P\left(A|B\right)+P\left(\overline{B}\right).P\left(A|\overline{B}\right)$.}
\end{ex}
\begin{ex}%Câu 11
	Một thư viện ghi lại số giờ đọc sách của 50 sinh viên trong một ngày và thu được mẫu số liệu ghép nhóm sau:\\
	\centerline{\begin{tabular}{|c|c|c|c|c|c|}
			\hline
			Nhóm giờ     & $\left[0\,;\,\,1\right)$ & $\left[1\,;\,\,2\right)$ & $\left[2\,;\,\,3\right)$ & $\left[3\,;\,\,4\right)$ & $\left[4\,;\,\,5\right)$ \\
			\hline
			Số sinh viên & 8                        & 11                       & 15                       & 9                        & 7                        \\
			\hline
		\end{tabular}}\\
	Khoảng tứ phân vị của mẫu số liệu ghép nhóm gần nhất với giá trị nào sau đây?
	\choice
	{$1,69$}
	{$1,85$}
	{$2,02$}
	{\True $1,98$}
	\loigiai{
	Chọn D.\\
	Giả sử mẫu số liệu gốc là $x_1;\,\,x_2;\,\,...;\,\,x_{50}$ được xếp theo thứ tự không giảm.\\
	Xét nửa bên trái mẫu số liệu gốc là $x_1;\,\,x_2;\,\,...;\,\,x_{25}$. Tứ phân vị thứ nhất của mẫu số liệu gốc là $x_{13}\in\left[1\,;\,\,2\right)$ nên tứ phân vị thứ nhất của mẫu số liệu ghép nhóm là $Q_1=1+\dfrac{\dfrac{50}{4}-8}{11}.1=\dfrac{31}{22}\approx 1,41$ (giờ).\\
	Xét nửa bên phải mẫu số liệu gốc là $x_{26};\,\,x_2;\,\,...;\,\,x_{50}$.\\
	Tứ phân vị thứ ba của mẫu số liệu gốc là $x_{38}\in\left[3\,;\,\,4\right)$ nên tứ phân vị thứ ba của mẫu số liệu ghép nhóm là $Q_3=3+\dfrac{3.\dfrac{50}{4}-34}{9}.1=\dfrac{61}{18}\approx 3,39$ (giờ).\\
	Khoảng tứ phân vị của mẫu số liệu ghép nhóm: $\Delta Q=Q_3-Q_1\approx 1,98$ (giờ).}
\end{ex}
\begin{ex}%Câu 12
	Tập nghiệm của bất phương trình $\log_5\left(2x-1\right)<\log_5\left(x+2\right)$ là
	\choice
	{$S=\left(3\,;\,\,+\infty\right)$}
	{$S=\left(-\infty\,;\,\,3\right)$}
	{\True $S=\left(\dfrac{1}{2}\,;\,\,3\right)$}
	{$S=\left(-2\,;\,\,3\right)$}
	\loigiai{
		Chọn C.\\
		Ta có: $\log_5\left(2x-1\right)<\log_5\left(x+2\right)\Leftrightarrow\left\{\begin{aligned}
				 & 2x-1>0   \\
				 & 2x-1<x+2 \\
			\end{aligned}\right.\Leftrightarrow\left\{\begin{aligned}
				 & x>\dfrac{1}{2} \\
				 & x<3            \\
			\end{aligned}\right.$ .\\
		Vậy tập nghiệm phương trình $S=\left(\dfrac{1}{2}\,;\,\,3\right)$ .}
\end{ex}
\Closesolutionfile{ans}
\cauds
\Opensolutionfile{ans}[ans/ans-HXN-\sode-TF]
\begin{ex}
	Một người đang bơm khí vào một quả bóng bay với tốc độ $100 $cm$^3/s$. Quả bóng ngày càng to dần nhưng luôn có dạng hình cầu. Đây là loại bóng bóng mà nếu người bơm để bán kính vượt quá $30$cm thì bóng sẽ bể.
	Xét tính đúng sai các mệnh đề sau:
	\choiceTF
	{Sau 10 giây, bán kính quả bóng bóng bằng $6,4 cm$ (làm tròn đến hàng phần chục của $cm$)}
	{\True Người bơm không thể để cho thể tích quả bóng bóng vượt quá $113$ lít (làm tròn đến hàng phần chục của lít)}
	{\True Khi đường kính của quả bóng bóng là $50 cm$ thì bán kính của quả bóng đang tăng với tốc độ $0,01 cm/s$ (làm tròn đến hàng phần trăm của $cm/s$)}
	{Nếu sau khi bơm được 4 giây, người bơm tăng tốc độ bơm thêm $5 cm^3$ trên một giây thì sau 189 giây (làm tròn đến hàng đơn vị của giây), bóng bóng sẽ bể}
	\loigiai{
		\begin{itemchoice}
			\itemch Gọi $V(t), R(t)$, là thể tích và bán kính quả bóng bóng sau $t$ giây, ta có $V(t) = \dfrac{4}{3}\pi R^3(t)$.
			Sau 10 giây, thể tích quả bóng là $V(10) = 100 \times 10 = 1000 cm^3$.
			Ta có $V(10) = \dfrac{4}{3}\pi R_{10}^3 = 1000 \Rightarrow R_{10} \approx 6.2 cm$.
			\itemch Bán kính tối đa của quả bóng bóng là $30 cm$; thể tích tối đa của quả bóng bóng là $\dfrac{4}{3}\pi \cdot 30^3 \approx 113097 cm^3 \approx 113$ lít.
			\itemch Khi bán kính bong bóng bóng bằng $\dfrac{50}{2} = 25 cm$ thì thể tích bong bóng là $\dfrac{4\pi \cdot 25^3}{3} = \dfrac{62500\pi}{3} cm^3$.
			Đạo hàm hai vế của $V(t) = \dfrac{4}{3}\pi R^3(t)$ theo t, ta được: $\dfrac{dV(t)}{dt} = 4\pi R^2\cdot \dfrac{dR}{dt}$.
			Thay $R_t = 25 cm$; $\dfrac{dV(t)}{dt} = 100 cm^3/s$, ta có: $\dfrac{dR}{dt} \approx 0.01 cm/s$.
			\itemch Thể tích bong bóng sau $t+4$ giây ($t \ge 0$) là $V(t) = 100\cdot 4 + \int\limits_{0}^{t}(5t+100)dt$.
			Thể tích tối đa của quả bóng bóng là $\dfrac{4}{3}\pi \cdot 30^3 cm^3$.
			Xét $V(t) = 100\cdot 4 + \int\limits_{0}^{t}(5t+100)dt = \dfrac{4}{3}\pi \cdot 30^3 \Rightarrow t \approx 193$ giây.
		\end{itemchoice}
	}
\end{ex}
\begin{ex}
	\immini[thm]{ Vịnh Hạ Long là một địa danh du lịch được nhiều người biết đến trên thế giới, nơi đây vẫn còn nhiều quần thể đảo lớn nhỏ chưa được khám phá. Một công ty du lịch quyết định khai thác khu vực có một số đảo nhỏ với hình dáng đặc biệt nếu nhìn từ trên xuống; trong số đó có hai hòn đảo mà phần giới hạn lát cắt của nó được mô phỏng như hai đồ thị hàm số trên hình. Với hệ trục tọa độ $Oxy$ thích hợp, đơn vị trên mỗi trục là $100$ mét, đường cong mô tả cho hòn đảo thứ nhất có dạng $y = \log_{a}{x}$ đi qua điểm có tọa độ $(3; 1)$.
	}{\includegraphics[scale=1]{img/HXN-5.14}}
	\choiceTF
	{ Điểm có tọa độ $(9; 3)$ thuộc đường cong $y = \log_{a}{x}$}
	{ Chủ dự án muốn xây dựng một nơi trực tiếp nhìn ra biển để du khách tham quan, ăn uống... Họ đã lựa chọn khu vực tam giác cong $ABC$ như trong hình (đường cong $AC$ tiếp giáp biển); diện tích khu vực này là $536$ $m^2$ (làm tròn đến hàng đơn vị)}
	{\True Chủ dự án đã thuê một số kỹ sư rất giỏi toán (đặc biệt giỏi về hàm số mũ-lôgarit) đi khảo sát khu vực này và họ nhận thấy có thể bồi đắp thêm cho hòn đảo thứ hai để đường cong giáp biển $y = g(x)$ của nó đối xứng với đường cong $y = \log_{a}{x}$ qua đường thẳng $y = x+1$. Khi đó đường cong $g(x) = 1+3 \cdot 3^x$}
	{ Chủ dự án định xây một cây cầu nối liền hai hòn đảo, khoảng cách ngắn nhất theo đường chim bay của cây cầu bằng $285$ $m$ (làm tròn đến hàng đơn vị mét)}
	\loigiai{
		\begin{itemchoice}
			\itemch Đường cong $y = \log_{a}{x}$ đi qua điểm $(3; 1)$ nên $1 = \log_{a}{3} \Rightarrow a = 3$.\\
			Khi đó hàm số trở thành $y = \log_{3}{x}$; đường cong này không đi qua điểm $(9; 3)$.
			\itemch Điểm $A(x_A; -1)$ thuộc đồ thị hàm số $y = \log_3 x \Rightarrow \log_3 x_A = -1 \Rightarrow x_A = 3^{-1} = \dfrac{1}{3}$.\\
			Diện tích tam giác cong $ABC$ là phần hình phẳng được giới hạn bởi hai đồ thị $y=\log_3 x$; $y = -1$ cùng các đường thẳng $x = \dfrac{1}{3}$; $x = 4$.\\
			Do đó diện tích cần tính là $S = \int\limits_{\frac{1}{3}}^{4} |\log_3 x - (-1)| \mathrm{d} x \approx 571$ $m^2$.
			\itemch Gọi $M(x_M; y_M) \in (C_1): y = \log_3 x$ và $N(x; y) \in (C_2): y = g(x)$.
			\begin{center}
				\includegraphics[scale=.8]{img/HXN-5.14a}
			\end{center}
			$M, N$ đối xứng qua $x - y + 1 = 0$ nên ta có:\\
			$\begin{cases} \dfrac{x + x_M}{2} - \dfrac{y + y_M}{2} + 1 = 0 \\ 1 \cdot (x - x_M) + (-1) \cdot (y - y_M) = 0 \end{cases} \Leftrightarrow \begin{cases} x + x_M - y - y_M + 2 = 0 \\ x - x_M - y + y_M = 0 \end{cases} \Rightarrow \begin{cases} y_M = x+1 \\ x_M = y-1 \end{cases}$ hay $M(y-1; x+1)$.\\
			Vì $M \in (C_1)$ nên $x+1 = \log_3 (y-1) \Rightarrow y-1 = 3^{x+1} \Rightarrow y = 3^{x+1} + 1$ hay $y = g(x) = 1 + 3 \cdot 3^x$.\\
			Cách giải khác (nhấn vào link) \hyperlink{Cách giải khác(nhấn vào link)}{https://www.tiktok.com/@tp1.phatvn.68/photo/7517274797793955079}
			\itemch Xét tiếp tuyến của đường cong $y = \log_3 x$ biết tiếp tuyến song song với đường thẳng $y = x+1$.\\
			Hệ số góc tiếp tuyến là $k = 1$; gọi $M(x_0; y_0)$ là tiếp điểm.\\
			Thì $f'(x_0) = \dfrac{1}{x_0 \ln 3} = 1 \Rightarrow x_0 = \dfrac{1}{\ln 3}$; $y_0 = \log_3 \dfrac{1}{\ln 3}$.\\
			Độ dài ngắn nhất cây cầu (theo đường chim bay) bằng hai lần khoảng cách từ $M\left(\dfrac{1}{\ln 3}; \log_3 \dfrac{1}{\ln 3}\right)$ đến đường thẳng $y = x+1$.\\
			Ta có: $d_{\min} = 2 \cdot \dfrac{\left| \dfrac{1}{\ln 3} - \log_3 \dfrac{1}{\ln 3} + 1 \right|}{\sqrt{1^2 + (-1)^2}} \times 100 \approx 282$ $m$.
		\end{itemchoice}
	}
\end{ex}
\begin{ex}
	Hộp A đựng 4 bi xanh và 4 bi trắng, hộp B đựng 6 bi xanh và 3 bi trắng, hộp C không có viên bi nào. Người ta thực hiện liên tiếp ba hành động sau đây hoàn toàn ngẫu nhiên:
	\begin{itemize}
		\item Lấy 1 viên bi từ hộp A bỏ sang hộp B.
		\item Lấy 1 viên bi từ hộp B bỏ sang hộp C.
		\item Lấy 1 viên bi từ hộp A bỏ sang hộp C.
	\end{itemize}
	Xét tính đúng sai các mệnh đề sau:
	\choiceTF
	{Nếu từ hộp A đã lấy 1 bi trắng bỏ sang hộp B thì xác suất để lấy bi trắng từ hộp B bỏ sang hộp C bằng $\dfrac{2}{5}$}
	{\True Xác suất để lấy được bi trắng từ hộp B bỏ sang hộp C bằng $\dfrac{7}{20}$}
	{\True Xác suất để lấy từ C được 2 bi xanh bằng $\dfrac{9}{28}$}
	{\True Xác suất để 2 bi lấy từ hộp C đều là các bi từ hộp A chuyển sang bằng $\dfrac{1}{15}$ biết rằng đó là 2 bi xanh}
	\loigiai{
		\begin{itemchoice}
			\itemch Nếu từ hộp A đã lấy 1 bi trắng bỏ sang hộp B thì khi đó hộp B có 6 bi xanh và 4 bi trắng; xác suất để lấy 1 bi trắng từ hộp B là $\dfrac{4}{10} = \dfrac{2}{5}$.
			\itemch Ta mô phỏng bài toán bởi sơ đồ sau:
			\begin{center}
				\includegraphics[scale=1]{img/HXN-5.15}
			\end{center}
			Ta có: $P(\text{Trắng}_{[B]\to[C]}) = \dfrac{1}{2} \cdot \dfrac{3}{10} + \dfrac{1}{2} \cdot \dfrac{4}{10} = \dfrac{7}{20}$.
			\itemch Ta có: $P(2\text{Xanh}_{[C]}) = \dfrac{1}{2} \cdot \dfrac{7}{10} \cdot \dfrac{3}{7} + \dfrac{1}{2} \cdot \dfrac{6}{10} \cdot \dfrac{4}{7} = \dfrac{9}{28}$.\\
			(Trong đó ta xem kí hiệu $2\text{Xanh}_{[C]}$ là lấy được 2 viên bi xanh từ hộp C).
			\itemch Ta có: $P(2\text{ bi}_{[A]\to[C]} | 2\text{Xanh}_{[C]}) = \dfrac{\dfrac{1}{2} \cdot \dfrac{1}{10} \cdot \dfrac{3}{7}}{\dfrac{9}{28}} = \dfrac{1}{15}$.\\
			(Trong đó ta xem kí hiệu $2\text{ bi}_{[A]\to[C]}$ là lấy từ hộp C đúng 2 viên bi từ hộp A chuyển qua).
		\end{itemchoice}
	}
\end{ex}
\begin{ex}
	Trong không gian $Oxyz$ cho trước, đơn vị trên mỗi trục là mét, có hai chiếc chiến đấu cơ từ hai vị trí $A(40;-15;15)$ và $B(55;-10;65)$ cần đáp xuống hai vị trí thuộc tàu sân bay hải quân để nạp nhiên liệu. Bề mặt chứa các đường băng trên tàu là mặt phẳng $(P)$ có phương trình $3x - y + 2z - 25 = 0$. Xét tính đúng sai các mệnh đề sau:
	\choiceTF
	{\True Đường thẳng qua $A$ và vuông góc với mặt phẳng $(P)$ có phương trình chính tắc là $\dfrac{x - 40}{3} = \dfrac{y + 15}{-1} = \dfrac{z - 15}{2}$}
	{ Tổng khoảng cách từ hai vị trí chiến đấu cơ đến mặt phẳng chứa đường băng là $110$ $m$ (làm tròn đến hàng đơn vị của mét)}
	{\True Tọa độ $A'$ đối xứng với $A$ qua $(P)$ là $A'(-20; 5; -25)$}
	{ Người chỉ huy ở tàu sân bay phát tín hiệu để hai chiến đấu cơ đáp xuống các vị trí $M, N$ cách nhau $5\sqrt{6}$ $m$. Tổng đường bay ngắn nhất $AM + BN$ bằng $115$ $m$ (làm tròn đến hàng đơn vị)}
	\loigiai{
		\begin{itemchoice}
			\itemch Đường thẳng qua $A$ và vuông góc với mặt phẳng $(P)$ có phương trình chính tắc là $\dfrac{x - 40}{3} = \dfrac{y + 15}{-1} = \dfrac{z - 15}{2}$.
			\itemch Ta có: $d(A, (P)) + d(B, (P)) = \dfrac{|3 \cdot 40 - (-15) + 2 \cdot 15 - 0|}{\sqrt{3^2 + (-1)^2 + 2^2}} + \dfrac{|3 \cdot 55 - (-10) + 2 \cdot 65 - 0|}{\sqrt{3^2 + (-1)^2 + 2^2}} = 30\sqrt{14} \approx 112$ $m$.
			\begin{center}
				\includegraphics[scale=.8]{img/HXN-5.16}
			\end{center}
			\itemch Gọi $H$ là hình chiếu vuông góc của $A$ trên $(P)$ thì tọa độ $H$ thỏa hệ phương trình\\ $\begin{cases} \dfrac{x - 40}{3} = \dfrac{y + 15}{-1} = \dfrac{z - 15}{2} \\ 3x - y + 2z - 0 = 0 \end{cases} \Leftrightarrow \begin{cases} x+3y+5=0 \\ 2y+z+15=0 \\ 3x-y+2z=25 \end{cases}$. Giải hệ này ta được $\begin{cases} x=10 \\ y=-5 \\ z=-5 \end{cases}$ hay $H(10; -5; -5)$.\\
			$A'$ đối xứng với $A$ qua $(P)$ nên $H$ là trung điểm của $AA'$. Suy ra $A'(-20; 5; -25)$.
			\itemch Lấy điểm $E$ thỏa mãn $\overrightarrow{AE} = \overrightarrow{MN}$, suy ra $A'M = EN$.\\
			Vì $A'$ cố định mà $A'E = 5\sqrt{6}$ nên $E$ thuộc đường tròn tâm $A'$, bán kính $r = 5\sqrt{6}$; đường tròn này thuộc mặt phẳng $(Q)$ qua $A'$ và song song với $(P)$.\\
			Gọi $K, F$ theo thứ tự là hình chiếu vuông góc của $B$ trên $(P)$, $(Q)$ suy ra $K(-5; 10; 25)$. $HK = 15\sqrt{6}$; $KF = HA' = AH = 10\sqrt{33}$.\\
			Ta có $AM + BN = A'M + BN = EN + BN \ge BE$.\\
			Đẳng thức xảy ra khi $E, N, B$ thẳng hàng theo thứ tự đó ($H, M, N, K$ thẳng hàng).\\
			Ta có: $BE = \sqrt{(\sqrt{20\sqrt{14}+10\sqrt{14}})^2 + (15\sqrt{6} - 5\sqrt{6})^2} = \sqrt{(30\sqrt{14})^2 + (10\sqrt{6})^2} = \sqrt{12600 + 600} = \sqrt{13200} = 20\sqrt{33}$.\\
			Vậy tổng độ dài bé nhất $AM + BN$ là $20\sqrt{33} \approx 115$ $m$.
		\end{itemchoice}
	}
\end{ex}

\Closesolutionfile{ans}
\caukq
\Opensolutionfile{ans}[ans/ans-HXN-\sode-SA]
\begin{ex}%Câu 13
	\immini[thm]{ Một trò chơi điện tử có luật chơi như sau:
		\begin{itemize}
			\item Người chơi xuất phát từ A và đi qua tất cả vị trí B, C, D, E trước khi về lại A để kết thúc lượt chơi của mình. Mỗi vị trí người chơi đi qua đúng 1 lần (trừ điểm A).
			\item Thông số trên mỗi đoạn đường đi gồm: x (huy chương) liên quan đến phần thưởng và y (quái vật) liên quan đến chướng ngại vật; điểm số người chơi đạt được trên mỗi đoạn đường có dạng $3x-2y$ .
		\end{itemize}
		Hỏi tổng số điểm tối đa mà người chơi đạt được là bao nhiêu?
		\shortans{ 44}}{\includegraphics[scale=1]{img/HXN-5.17}}
	\loigiai{
		Người chơi đi qua các con đường hợp lệ cùng với số điểm tương ứng như sau:
		\begin{itemize}
			\item $A \to B \to C \to E \to D \to A$; số điểm là $3(5+6+6+3+4)-2(3+4+6+0+1)=44$.
			\item $A \to B \to C \to D \to E \to A$; số điểm là $3(5+6+4+3+8)-2(3+4+5+0+6)=42$.
			\item $A \to E \to B \to C \to D \to A$; số điểm là $3(8+0+6+4+4)-2(6+1+4+5+1)=32$.
			\item $A \to E \to D \to C \to B \to A$; số điểm là $3(8+3+4+6+5)-2(6+0+5+4+3)=42$.
			\item $A \to D \to C \to B \to E \to A$; số điểm là $3(4+4+6+0+8)-2(1+5+4+1+6)=32$.
			\item $A \to D \to E \to C \to B \to A$; số điểm là $3(4+3+6+6+5)-2(1+0+6+4+3)=44$.
		\end{itemize}
		Số điểm tối đa mà người chơi đạt được là 44.}
\end{ex}
\begin{ex}%Câu 14
	Một hộp phấn không bụi có dạng hình hộp chữ nhật, chiều cao hộp phấn bằng 8,2 cm và đáy của nó có hai kích thước là 8,5 cm; 10,5 cm (xem hình vẽ). Tìm số đo góc phẳng nhị diện $\left[A,\,\,B'{D}',\,\,A'\right]$ (tính theo độ, làm tròn kết quả đến hàng phần chục).
	\begin{center}
		\includegraphics[scale=.5]{img/HXN-5.18}
	\end{center}
	\shortans{ 51,1}
	\loigiai{
	\begin{center}
		\includegraphics[scale=1]{img/HXN-5.18a}
	\end{center}
	Trong mặt phẳng $\left(A'{B}'{C}'{D}'\right)$, kẻ $A'H\bot{B}'{D}'$ tại H.\\
	Ta có: $\left\{\begin{aligned}
		 & {B}'{D}'\bot{A}'H                                                                      \\
		 & {B}'{D}'\perp A{A}'\,\,\left(\text{do}\,\,A{A}'\perp\left(A'{B}'{C}'{D}'\right)\right) \\
	\end{aligned}\right.$$\Rightarrow{B}'{D}'\perp\left(A{A}'H\right)\Rightarrow{B}'{D}'\perp AH$.\\
		Do đó $\widehat{AH{A}'}$ là góc phẳng nhị diện $\left[A,\,\,B'{D}',\,\,A'\right]$.\\
		Tam giác $A'{B}'{C}'$ vuông tại $A'$ có đường cao $A'H$ nên $\dfrac{1}{A'{H^2}}=\dfrac{1}{A'{B'^2}}+\dfrac{1}{A'{D'^2}}\Rightarrow{A}'H=\dfrac{A'{B}'.A'{D}'}{\sqrt{A'{B'^2}+A'{D'^2}}}=\dfrac{357}{2\sqrt{730}}$.\\
		Tam giác $ AH{A}'$ vuông tại $A'$ có $\tan\widehat{AH{A}'}=\dfrac{A{A}'}{A'H}=\dfrac{8,2}{\dfrac{357}{2\sqrt{730}}}\Rightarrow\widehat{AH{A}'}\approx 51,1^\circ$.}
\end{ex}
\begin{ex}%Câu 15
	Lan đang dự tính ghi danh học các lớp kỹ năng Anh ngữ, kỹ năng giao tiếp, kỹ năng quản lí v.v... tại một Hệ thống giáo dục trong thành phố, nơi mỗi lớp học chỉ học một lần mỗi tuần. Cô ấy đang chọn giữa 30 lớp học không trùng nhau. Có 6 lớp để lựa chọn cho mỗi ngày trong tuần, từ thứ Hai đến thứ Sáu. Sau nhiều ngày cân nhắc và tìm kiếm lời khuyên, Lan vẫn chưa thể đưa ra lựa chọn phù hợp. Sau cùng cô quyết định đăng ký 7 lớp được chọn ngẫu nhiên trong số 30 lớp đó, với mọi lựa chọn là đồng xác suất. Xác suất để Lan có lớp học vào tất cả các ngày từ thứ Hai đến thứ Sáu bằng $\dfrac{m}{n}$ (trong đó hai số m, n là nguyên tố cùng nhau). Tính $ m+n$.\\
	\shortans{ 491}
	\loigiai{
		Có hai khả năng chính để Lan có lớp học mỗi ngày trong tuần:\\
		• Trường hợp 1: Có 2 ngày có 2 lớp học, và 3 ngày còn lại có 1 lớp học.\\
		Số khả năng cho trường hợp 1 là $ C_5^2\cdot{\left(C_6^2\right)^2}\cdot{\left(C_6^1\right)^3}$.\\
		(Chọn 2 ngày trong 5 ngày có 2 lớp học, mỗi ngày đó chọn 2 lớp trong số 6 lớp; 3 ngày còn lại mỗi ngày chọn 1 lớp trong 6 lớp → có $\left(C_6^1\right)^3$ cách).\\
		• Trường hợp 2: Có 1 ngày có 3 lớp học, và 4 ngày còn lại mỗi ngày có 1 lớp học.\\
		Số khả năng cho trường hợp 2 là $ C_5^1C_6^3\cdot{\left(C_6^1\right)^4}$.\\
		(Chọn 1 ngày có 3 lớp học trong 5 ngày, chọn 3 lớp trong 6 lớp cho ngày đó; 4 ngày còn lại mỗi ngày chọn 1 lớp → $\left(C_6^1\right)^4$ cách).\\
		Vậy xác suất cần tính là $\dfrac{C_5^2\cdot{\left(C_6^2\right)^2}\cdot{\left(C_6^1\right)^3}+C_5^1C_6^3\cdot{\left(C_6^1\right)^4}}{C_{30}^7}=\dfrac{114}{377}=\dfrac{m}{n}$. Suy ra $m+n=491 $.}
\end{ex}
\begin{ex}%Câu 16
	Một chiến sĩ đặc công đang nấp ở bờ sông, cần phải bơi qua bờ bên kia để tấn công mục tiêu. Có thể xem con sông này là thẳng và có độ rộng 100 m; vận tốc bơi của chiến sĩ bằng một phần ba vận tốc chạy bộ. Biết rằng mục tiêu tấn công cách chiến sĩ 1 km theo đường chim bay; hỏi chiến sĩ phải bơi bao nhiêu mét để đến được mục tiêu nhanh nhất (làm tròn kết quả đến hàng đơn vị)?
	\shortans{106 }
	\begin{center}
		\includegraphics[scale=1.2]{img/HXN-5.20}
	\end{center}
	\loigiai{
		\begin{center}
			\includegraphics[scale=1.2]{img/HXN-5.20a}
		\end{center}
		Gọi C là hình chiếu vuông góc của A (vị trí chiến sĩ xuất phát) đối với bờ bên kia và D thuộc đoạn BC là vị trí mà chiến sĩ sẽ bơi đến trước khi chạy bộ tấn công mục tiêu tại A.
		Ta chuẩn hóa bài toán như sau:
		\begin{itemize}
			\item 1 đơn vị độ dài = $100 m$; khi đó $AC = 1$, $AB = 10$.
			\item Vận tốc bơi trên sông của chiến sĩ là 1 (đơn vị vận tốc); vận tốc chạy của chiến sĩ là 3 (đơn vị vận tốc).
		\end{itemize}
		Đặt $AD = x \in (1; 10) \Rightarrow CD = \sqrt{x^2-1}$; $BC = \sqrt{AB^2 - AC^2} = 3\sqrt{11}$.\\
		$BD = BC - CD = 3\sqrt{11} - \sqrt{x^2-1}$.\\
		Tổng thời gian từ khi chiến sĩ xuất phát đến khi tiếp cận mục tiêu là:
		$$t = \dfrac{AD}{1} + \dfrac{BD}{3} = \dfrac{x}{1} + \dfrac{3\sqrt{11} - \sqrt{x^2-1}}{3} = x + \sqrt{11} - \dfrac{\sqrt{x^2-1}}{3}$$
		Xét hàm $f(x) = x + \sqrt{11} - \dfrac{\sqrt{x^2-1}}{3}$; $x \in (1; 10)$; $f'(x) = 1 - \dfrac{1}{3}\dfrac{2x}{2\sqrt{x^2-1}} = 1 - \dfrac{x}{3\sqrt{x^2-1}}$.\\
		$f'(x) = 0 \Rightarrow 1 - \dfrac{x}{3\sqrt{x^2-1}} = 0 \Rightarrow \dfrac{x}{3\sqrt{x^2-1}} = 1 \Rightarrow x = 3\sqrt{x^2-1} \Rightarrow x^2 = 9(x^2-1) \Rightarrow x^2 = 9x^2 - 9 \Rightarrow 8x^2 = 9 \Rightarrow x^2 = \dfrac{9}{8} \Rightarrow x = \dfrac{3\sqrt{2}}{4}$.\\
		Bảng biến thiên:
		\begin{tabular}{|c|ccccc|}
			\hline
			$x$                           & 1            &            & $\dfrac{3\sqrt{2}}{4}$ &            & 10           \\
			\hline
			$f'(x)$                       & $\vert\vert$ & $-$        & 0                      & $+$        & $\vert\vert$ \\
			\hline
			\rule[-.1in]{0in}{.3in}$f(x)$ &              & $\searrow$ & Min                    & $\nearrow$ &              \\
			\hline
		\end{tabular}
		\\Chiến sĩ tiếp cận mục tiêu nhanh nhất khi $BD = x = \dfrac{3\sqrt{2}}{4}$.\\
		Do đó chiến sĩ sẽ bơi một đoạn $AD = 100 \times \dfrac{3\sqrt{2}}{4} \approx 106 m$.
	}
\end{ex}
\begin{ex}%Câu 17
	\immini[thm]{Một người nghệ sĩ đã vẽ hình chiếc nơ theo một cách khác lạ so với các nhà thiết kế. Anh ta vẽ hình chữ nhật ABCD tâm O có chiều dài bằng 4 dm, chiều rộng bằng 2 dm. Chiếc nơ chính là hình (H) nằm bên trong hình chữ nhật sao cho khi kẻ tia Ot bất kì cắt (H) và cạnh hình chữ nhật lần lượt tại M và N thì $ MN=1$ dm. Tính diện tích chiếc nơ hình (H) đó theo $ d{m^2}$ (làm tròn đến hàng phần chục).
		\shortans{1,52 }}{\includegraphics[scale=1]{img/HXN-5.21}}
	\loigiai{
	\begin{center}
		\includegraphics[scale=1]{img/HXN-5.21a}
	\end{center}
	Xét hình vẽ và các kí hiệu như sau.\\
	Gọi $\varphi=\left(Ox\,,\,\,Ot\right)$ thì $\cos\varphi=\dfrac{OH}{ON}=\dfrac{2}{r_{\varphi}+1}\Rightarrow r_{\varphi} =\dfrac{2}{\cos \varphi} -1$ ; với $r_{\varphi}=OM$ quay quanh gốc O khi $0<\varphi <\widehat{HOA}$ .\\
	Gọi $\theta=\left(Oy\,,\,\,O{t}'\right)$ thì $\cos\theta=\dfrac{OK}{O{N}'}=\dfrac{1}{r_{\theta}+1}$ $\Rightarrow r_{\theta}=\dfrac{1}{\cos \theta}-1 $ ; với $r_{\theta}=O{M}'$ quay quanh gốc O khi $0<\theta <\widehat{AOK}$ .\\
	Ta có: $\tan\widehat{AOK}=\dfrac{AK}{OK}=2\Rightarrow\widehat{AOK}=\arctan 2$ .\\
	Do đó diện tích cần tính là $S=4\cdot\dfrac{1}{2}\cdot\left[\displaystyle\int\limits_0^{\arctan 0,5}{\left(\dfrac{2}{\cos\varphi}-1\right)^2\text{d}\varphi}+\displaystyle\int\limits_0^{act\tan 2}{\left(\dfrac{1}{\cos\theta}-1\right)^2\text{d}\theta}\right]\approx 1,52\,\,d{m^2}$ .}
\end{ex}
\begin{ex}%Câu 18
	Trong không gian với hệ trục tọa độ $Oxyz$ cho ba mặt phẳng: $(P):x-2y+z-1=0$, $(Q):x-2y+z+8=0$, $(R):x-2y+z-4=0$. Một đường thẳng $d$ thay đổi cắt ba mặt phẳng $(P)$, $(Q)$, $(R)$ lần lượt tại $ A$, $ B$, $ C$. Tìm giá trị nhỏ nhất của $ T=A{B^2}+\dfrac{144}{AC}$.\\
	\shortans{108 }
	\loigiai{
	\begin{center}
		\includegraphics[scale=.7]{img/HXN-5.22a}
	\end{center}
	Dựa vào phương trình ba mặt phẳng $(P),\,\,(Q),\,\,(R)$ đã cho, ta thấy chúng song song nhau; so sánh hệ số tự do trong phương trình ba mặt phẳng thì: $-4<-1<8$, do vậy mặt phẳng $(P)$ nằm giữa hai mặt phẳng $(Q),\,\,(R)$.\\
	Ta tính khoảng cách giữa $(P)$ với hai mặt phẳng còn lại:\\ $d\left((P),(Q)\right)=\dfrac{\left| 8-\left(-1\right)\right|}{\sqrt{1^2+\left(-2\right)^2+1^2}}=\dfrac{9}{\sqrt{6}}$ ; $ d\left((P),(R)\right)=\dfrac{\left|-4-\left(-1\right)\right|}{\sqrt{1^2+\left(-2\right)^2+1^2}}=\dfrac{3}{\sqrt{6}}.$\\
	Do vậy $d\left((P),(Q)\right)=3 d\left((P),(R)\right)$.\\
	Gọi $A',\,\,B'$ lần lượt là hình chiếu của $ C$ trên các mặt phẳng $(P),\,\,(Q)$$\Rightarrow C{A}'=\dfrac{3}{\sqrt{6}},\,\,A'{B}'=\dfrac{9}{\sqrt{6}}$. Vì $ A{A}'\text{//}B{B}'$ nên $\dfrac{AC}{AB}=\dfrac{C{A}'}{A'{B}'}=\dfrac{\dfrac{3}{\sqrt{6}}}{\dfrac{9}{\sqrt{6}}}=\dfrac{1}{3}$ hay $AC=\dfrac{1}{3}AC $.\\
		Ta có: $ T=A{B^2}+\dfrac{144}{AC}=A{B^2}+\dfrac{144}{\dfrac{1}{3}AB}=A{B^2}+\dfrac{432}{AB}=A{B^2}+\dfrac{216}{AB}+\dfrac{216}{AB}\overset{AM-GM}{\mathop{\ge}}\,3\sqrt[3]{A{B^2}.\dfrac{216}{AB}.\dfrac{216}{AB}}=108$.\\
		Dấu $ ''=''$ xảy ra khi và chỉ khi $ A{B^2}=\dfrac{216}{AB}\Leftrightarrow A{B^3}=216\Leftrightarrow AB=6$, suy ra $ AC=2$.\\
		Vậy $T_{\min}=108$.}
\end{ex}
\Closesolutionfile{ans}
\inputansbox{6,4,3}{ans/ans-HXN-\sode-T,ans/ans-HXN-\sode-TF,ans/ans-HXN-\sode-SA}
% \setcounter{deso}{15}
% \def\sode{6}
\begin{name}
	{\tenchude}
	{\tendethi}
	{\tentruong}
	{\thoigian}
\end{name}
\caulc
\Opensolutionfile{ans}[ans/ans-HXN-\sode-T]
\begin{ex}%Câu 1
 Trong không gian $ Oxyz$, mặt phẳng đi qua điểm $ K(1\,;\,\,1\,;\,\,1)$ nhận $\vec{u}=(1\,;\,\,0\,;\,\,1)$, $\vec{v}=(1\,;\,\,1\,;\,\,0)$ làm căp vectơ chỉ phương có phương trình tổng quát là
 \choice
 {$ x+y+z-3=0$}
 {$ x-y+z-1=0$}
 {$ x+y-z-1=0$}
 {\True $-x+y+z-1=0$}
 \loigiai{
 Chọn D.\\
 Mặt phẳng có vectơ pháp tuyến $\vec{n}=\left[\vec{u},\vec{v}\right]=\left(-1\,;\\,1\ ;\\,1\right)$.\\
 Phương trình mặt phẳng là $-\left(x-1\right)+1\left(y-1\right)+1\left(z-1\right)=0\Leftrightarrow-x+y+z-1=0$.}
\end{ex}
\begin{ex}%Câu 2
 Cho bảng phân bố tần số ghép nhóm về độ dài của 60 lá dương xỉ trưởng thành như sau:\\
 \centerline{\begin{tabular}{|c|c|c|c|c|}
 \hline
 Độ dài (cm) & $\left[10\,;20\right)$ & $\left[20\,;30\right)$ & $\left[30\,;40\right)$ & $\left[40\,;50\right]$\\
 \hline
 Tần số & $ 8$ & $ 18$ & $ 24$ & $ 10$\\
 \hline
 \end{tabular}}\\
 Tính phương sai của mẫu số liệu ghép nhóm đã cho.
 \choice
 {$s_{}^2=83$}
 {\True $s_{}^2=84$}
 {$s_{}^2=85$}
 {$s_{}^2=86$}
 \loigiai{
 Chọn B.\\
 Ta viết lại bảng trên có bổ sung giá trị đại diện:\\
 \centerline{\begin{tabular}{|c|c|c|c|c|}
 \hline
 Độ dài (cm) & $\left[10\,;20\right)$ & $\left[20\,;30\right)$ & $\left[30\,;40\right)$ & $\left[40\,;50\right]$\\
 \hline
 Giá trị đại diện & $15$ & $25$ & $35$ & $45$\\
 \hline
 Tần số & $ 8$ & $ 18$ & $ 24$ & $ 10$\\
 \hline
 \end{tabular}}\\
 Giá trị trung bình của mẫu số liệu ghép nhóm: $\bar{x}=\dfrac{15\times 8+25\times 18+35\times 24+45\times 10}{60}=31$.\\
 Phương sai mẫu số liệu ghép nhóm là:\\
 $ s_{}^2=\dfrac{8\times{(15-31)^2}+18\times{(25-31)^2}+24\times{(35-31)^2}+10\times{(45-31)^2}}{60}=84$.}
\end{ex}
\begin{ex}%Câu 3
 Cho lăng trụ đều $ABC.A'{B}'{C}'$ . Góc giữa hai vectơ $\overrightarrow{BA}$ và $\overrightarrow{C'{B}'}$ bằng bao nhiêu?
 \begin{center}
\begin{tikzpicture}[scale=.7, font=\footnotesize, line join=round, line cap=round, >=stealth]
    \def\ac{4} % cạnh AC
    \def\ab{2} % cạnh AB
    \def\ben{4} % cạnh bên
    \def\gocnghieng{90} % góc nghiêng cạnh bên
    \def\gocA{50} % góc A của đáy
    \coordinate[label=left:$A$] (A) at (0,0);
    \coordinate[label=right:$C$] (C) at (\ac,0);
    \coordinate[label=below left:$B$] (B) at (-\gocA:\ab);
    \coordinate[label=left:$A'$] (A') at ($(A)+(\gocnghieng:\ben)$);
    \coordinate[label=below left:$B'$] (B') at ($(B)-(A)+(A')$);
    \coordinate[label=right:$C'$] (C') at ($(C)-(A)+(A')$);
    \draw (A')--(A)--(B)--(C)--(C')--(A')--(B')--(C') (B)--(B');
    \draw[dashed] (A)--(C);
    \foreach \diem in {A,B,C,A',B',C'} \fill (\diem)circle(1.5pt);
\end{tikzpicture}

 \end{center}
 \choice
 {$30^\circ $}
 {$60^\circ $}
 {\True $120^\circ $}
 {$90^\circ $}
 \loigiai{
 Chọn C.\\
 Ta có $\left(\overrightarrow{BA}\,,\,\,\,\overrightarrow{C'{B}'}\right)=\left(\overrightarrow{BA}\,,\,\,\overrightarrow{CB}\right)=180^\circ-\widehat{ABC}=180^\circ-60^\circ=120^\circ $ .}
\end{ex}
\begin{ex}%Câu 4
 Thống kê điểm thi đánh giá năng lực của một trường THPT qua thang điểm 100 được cho ở bảng sau:
 \begin{center}
     \begin{tabular}{|l|c|c|c|c|c|}
         \hline
         Điểm & $[0; 20)$ & $[20; 40)$ & $[40; 60)$ & $[60; 80)$ & $[80; 100]$ \\
         \hline
         Số học sinh & 25 & 35 & 37 & 15 & 8 \\
         \hline
     \end{tabular}
 \end{center}
 Trung vị của mẫu số liệu ghép nhóm là giá trị nào sau đây?
 \choice
 {$38,2$}
 {\True $40$}
 {$39,6$}
 {$42$}
 \loigiai{
 Chọn B.\\
 Kích thước mẫu số liệu $ n=25+35+37+15+8=120$.\\
 Trung vị của mẫu số liệu gốc là $\dfrac{x_{60}+x_{61}}{2}$; mà $x_{60}\in\left[20\,;\,\,40\right)\,,\,\,x_{61}\in\left[40\,;\,\,60\right)$ nên trung vị của mẫu số liệu ghép nhóm bằng $M_e$=40.}
\end{ex}
\begin{ex}%Câu 5
 Cho cấp số nhân $\left(u_n\right)$ có tổng $n$ số hạng đầu tiên là $S_n=5^n-1$ với $n\in{\mathbb{N}^*}$ . Tìm số hạng đầu $u_1$ và công bội $q$ của cấp số nhân đó.
 \choice
 {$u_1=5$, $q=4$}
 {$u_1=4$, $q=6$}
 {\True $u_1=4$, $q=5$}
 {$u_1=6$, $q=5$}
 \loigiai{
 Chọn C.\\
 Ta có $u_1=S_1=5^1-1=4$; $u_2=S_2-S_1=(5^2-1)-(5^1-1)=20$.\\
 Công bội cấp số nhân là $ q=\dfrac{u_2}{u_1}=\dfrac{20}{4}=5$.}
\end{ex}
\begin{ex}%Câu 6
 Trong không gian với hệ trục tọa độ $Oxyz$ , cho hai điểm $ A(2;-2;1)$, $B(0;1;2)$. Tọa độ điểm $ M$ thuộc mặt phẳng $\left(Oxy\right)$ sao cho ba điểm $ A,B, M$ thẳng hàng là
 \choice
 {\True $M\left(4\,;\,\,-5\,;\,\,0\right)$}
 {$M\left(2\,;\,\,-3\,;\,\,0\right)$}
 {$M\left(0\,;\,\,0\,;\,\,1\right)$}
 {$M\left(4\,;\,\,5\,;\,\,0\right)$}
 \loigiai{
 Chọn A.\\
 Gọi $ M\left(x\,;\,\,y\,;\,\,0\right)\in\left(Oxy\right)$; $\overrightarrow{AB}=\left(-2\,;\,\,3\,;\,\,1\right);\overrightarrow{AM}=\left(x-2\,;\,\,y+2\,;\,\,-1\right)$.\\
 Ba điểm $ A,B, M$ thẳng hàng $\Leftrightarrow $$\overrightarrow{AB}$ và $\overrightarrow{AM}$ cùng phương $\Leftrightarrow\dfrac{x-2}{-2}=\dfrac{y+2}{3}=\dfrac{-1}{1}$$\Leftrightarrow\left\{\begin{aligned}
 &\dfrac{x-2}{-2}=-1\\ 
 &\dfrac{y+2}{3}=-1\\ 
 \end{aligned}\right.\Leftrightarrow\left\{\begin{aligned}
 & x=4\\ 
 & y=-5\\ 
 \end{aligned}\right.$. Vậy $M\left(4\,;\,\,-5\,;\,\,0\right)$ .}
\end{ex}

\begin{ex}%Câu 7
 Giá trị nhỏ nhất của hàm số $ y=\dfrac{x^2+3}{x-1}$ trên đoạn $\left[2\,;\,\,4\right]$là
 \choice
 {\True $\underset{\left[2\,;\,\,4\right]}{\min}\,y=6$}
 {$\underset{\left[2\,;\,\,4\right]}{\min}\,y=-2$}
 {$\underset{\left[2\,;\,\,4\right]}{\min}\,y=-3$}
 {$\underset{\left[2\,;\,\,4\right]}{\min}\,y=\dfrac{19}{3}$}
 \loigiai{
 Chọn A.\\
 Ta có $y'=\dfrac{2x\left(x-1\right)-\left(x^2+3\right)}{x-1}=\dfrac{x^2-2x-3}{\left(x-1\right)^2}$; $y'=0\Rightarrow{x^2}-2x-3=0\Rightarrow\left[\begin{aligned}
 & x=-1\notin\left(2\,;\,\,4\right)\\ 
 & x=3\in\left(2\,;\,\,4\right)\\ 
 \end{aligned}\right.$.\\
 Ta có:$ y(2)=7$, $ y(3)=6$, $ y(4)=\dfrac{19}{3}$. Vậy $\underset{\left[2\,;\,\,4\right]}{\min}\,y=y(3)=6$.}
\end{ex}

\begin{ex}%Câu 8
 Tập nghiệm của bất phương trình $5^{x-1}\ge{5^{x^2-x-9}}$ là
 \choice
 {\True $\left[-2\,;\,\,4\right]$}
 {$\left[-4\,;\,\,2\right]$}
 {$\left(-\infty\,;\,\,-2\right]\cup\left[4\,;\,\,+\infty\right)$}
 {$\left(-\infty\,;\,\,-4\right]\cup\left[2\,;\,\,+\infty\right)$}
 \loigiai{
 Chọn A.\\
 Ta có: $5^{x-1}\ge{5^{x^2-x-9}}\Leftrightarrow x-1\ge{x^2}-x-9\Leftrightarrow{x^2}-2x-8\le 0\Leftrightarrow-2\le x\le 4$.\\
 Tập nghiệm của bất phương trình là $ S=\left[-2\,;\,\,4\right]$.}
\end{ex}

\begin{ex}%Câu 9
    Diện tích hình phẳng giới hạn bởi hai đường $y = x^2 - 1$ và $y = x - 1$ bằng
    \choice
    {$\dfrac{\pi}{6}$}
    {$\dfrac{13}{6}$}
    {$\dfrac{13\pi}{6}$}
    {\True $\dfrac{1}{6}$}
  \loigiai{
 Chọn D.}
\end{ex}
%
\begin{ex}%Câu 10
 Cho hàm số $y=\dfrac{x-2}{x+3}$ . Mệnh đề nào sau đây đúng?
 \choice
 {Hàm số nghịch biến trên khoảng $\left(-\infty\,;\,\,+\infty\right)$}
 {Hàm số nghịch biến trên từng khoảng $\left(-\infty\,;\,\,-3\right)$ và $\left(-3\,;\,\,+\infty\right)$}
 {\True Hàm số đồng biến trên từng khoảng $\left(-\infty\,;\,\,-3\right)$ và $\left(-3\,;\,\,+\infty\right)$}
 {Hàm số đồng biến trên khoảng $\left(-\infty\,;\,\,+\infty\right)$}
 \loigiai{
 Chọn C.\\
 Tập xác định hàm số $D=\mathbb{R}\setminus\left\{-3\right\}$ . Ta có $y'=\dfrac{5}{\left(x+3\right)^2}>0$ , $\forall x\in D$ .\\
 Vậy hàm số đồng biến trên các khoảng $\left(-\infty\,;\,\,-3\right)$ và $\left(-3\,;\,\,+\infty\right)$ .}
\end{ex}

\begin{ex}%Câu 11
 Cho hai biến cố A và B với $ P(A)=0,3;\,\,P(B)=0,4$ và $ P\left(AB\right)=0,2.$ Xác suất để A hoặc B xảy ra bằng
 \choice
 {$ 0,3$}
 {$ 0,4$}
 {$ 0,6$}
 {\True $ 0,5$}
 \loigiai{
 Chọn D.\\
 Ta có: $ P\left(A\cup B\right)=P(A)+P(B)-P\left(AB\right)=0,3+0,4-0,2=0,5$.}
\end{ex}

\begin{ex}%Câu 12
 Biết $ F(x)=x^2$ là một nguyên hàm của hàm số $ f(x)$ trên $\mathbb{R}$. Giá trị của $\displaystyle\int\limits_1^3\left[1+f(x)\right]\text{d}x$ bằng
 \choice
 {\True $ 10$}
 {$ 8$}
 {$\dfrac{26}{3}$}
 {$\dfrac{32}{3}$}
 \loigiai{
 Chọn A.\\
 Ta có $\displaystyle\int\limits_1^3\left[1+f(x)\right]\text{d}x=\left.\left[x+F(x)\right]\right|_1^3=\left.\left(x+x^2\right)\right|_1^3=12-2=10.$}
 \end{ex}
 
\Closesolutionfile{ans}
\cauds
\Opensolutionfile{ans}[ans/ans-HXN-\sode-TF]
% 
 \begin{ex}%Câu 13
 Khi Mặt Trăng quay quanh Trái Đất, mặt đối diện với Trái Đất thường chỉ được Mặt Trời chiếu sáng một phần. Các pha của Mặt Trăng mô tả mức độ phần bề mặt của nó được Mặt Trời chiếu sáng. Khi góc giữa Mặt Trời, Trái Đất và Mặt Trăng là $\alpha$ ($0^\circ \le \alpha \le 360^\circ$) thì tỉ lệ $F$ của phần Mặt Trăng được chiếu sáng cho bởi công thức $F = \dfrac{1}{2}(1-\cos\alpha)$.
 Biết rằng $F = 0$ khi có trăng mới; $F = 0,25$ khi có trăng lưỡi liềm; $F = 0,5$ khi có trăng bán nguyệt đầu tháng và cuối tháng; $F = 1$ khi trăng tròn. 
 \choiceTF
 {Khi có trăng mới thì $\alpha = 90^\circ$}
 {\True Khi có trăng lưỡi liềm thì $\alpha = 60^\circ$ hoặc $\alpha = 300^\circ$}
 {\True Khi có trăng bán nguyệt đầu tháng hoặc cuối tháng thì $\alpha = 90^\circ$ hoặc $\alpha = 270^\circ$}
 {\True Khi có trăng tròn thì $\alpha = 180^\circ$}
 \loigiai{
 \begin{listEX}
     \item Khi có trăng mới thì $F = \dfrac{1}{2}(1-\cos\alpha) = 0 \Rightarrow \cos\alpha = 1 \Rightarrow \alpha = k360^\circ$, $k \in \mathbb{Z}$; mà $0^\circ \le \alpha \le 360^\circ$ nên $\alpha \in \{0^\circ; 360^\circ\}$.
     \item Khi có trăng lưỡi liềm thì $F = \dfrac{1}{2}(1-\cos\alpha) = 0,25 \Rightarrow \cos\alpha = \dfrac{1}{2}$
     $\Rightarrow \begin{cases} \alpha = 60^\circ + k360^\circ \\ \alpha = -60^\circ + k360^\circ \end{cases}$ ($k \in \mathbb{Z}$); mà $0^\circ \le \alpha \le 360^\circ$ nên $\alpha \in \{60^\circ; 300^\circ\}$.
     \item Khi có trăng bán nguyệt đầu tháng hoặc cuối tháng thì
     $F = \dfrac{1}{2}(1-\cos\alpha) = \dfrac{1}{2} \Rightarrow \cos\alpha = 0 \Rightarrow \alpha = 90^\circ + k180^\circ$, $k \in \mathbb{Z}$;
     mà $0^\circ \le \alpha \le 360^\circ$ nên $\alpha \in \{90^\circ; 270^\circ\}$.
     \item Khi có trăng tròn thì $F = \dfrac{1}{2}(1-\cos\alpha) = 1 \Rightarrow \cos\alpha = -1 \Rightarrow \alpha = 180^\circ + k360^\circ$, $k \in \mathbb{Z}$; mà $0^\circ \le \alpha \le 360^\circ$ nên $\alpha = 180^\circ$.
 \end{listEX}}
 \end{ex}
% 
 \begin{ex}%Câu 14
     Hai nguồn sáng giống hệt nhau được đặt cách nhau 10 mét. Một vật sẽ được đặt tại một điểm $P$ nằm trên một đường thẳng $l$, song song với đường nối hai nguồn sáng và cách đường đó một khoảng $d$ mét (xem hình vẽ).
     Ta muốn đặt điểm $P$ trên đường $l$ sao cho cường độ chiếu sáng tại $P$ là nhỏ nhất. Ta cần biết rằng cường độ chiếu sáng tại điểm đơn lẻ tỉ lệ thuận với cường độ của nguồn và tỉ lệ nghịch với bình phương khoảng cách đến nguồn đó.
     Đặt hệ trục tọa độ $Oxy$ sao cho tâm $O$ trùng với nguồn sáng bên trái, tia $Ox$ chứa đoạn nối hai nguồn sáng, tia $Oy$ hướng lên trên, đơn vị trên mỗi trục là mét.
     \begin{center}
         \includegraphics[scale=1]{img/HXN-6.14}
     \end{center}
     Xét tính đúng sai các mệnh đề sau:
     \choiceTF
     {Khoảng cách từ $P$ đến các nguồn sáng là $r_1 = \sqrt{x^2 + d^2}$; $r_2 = \sqrt{(x+10)^2 + d^2}$}
     {\True Tổng cường độ chiếu sáng tại $P$ là $I(x) = k\left( \dfrac{1}{x^2+d^2} + \dfrac{1}{(x-10)^2+d^2} \right)$; với $k > 0$ là hằng số tỉ lệ}
     {\True Khi $d=5$ mét, cường độ ánh sáng tại $P$ đạt cực tiểu khi $x=5$ mét}
     {\True Khi $d=10$ mét, cường độ ánh sáng không đạt cực tiểu khi $P$ ở vị trí chính giữa của thanh $l$}
 \loigiai{
 \begin{listEX}
     \item Mệnh đề sai.
     Khoảng cách từ $P$ đến các nguồn sáng là $r_1 = \sqrt{x^2+d^2}$; $r_2 = \sqrt{(x-10)^2+d^2}$.
     \item Mệnh đề đúng.
     Cường độ chiếu sáng tại $P$ từ mỗi nguồn là:
     \begin{itemize}
         \item Từ nguồn $O$: $I_1 = \dfrac{k}{r_1^2} = \dfrac{k}{x^2+d^2}$.
         \item Từ nguồn bên phải: $I_2 = \dfrac{k}{r_2^2} = \dfrac{k}{(x-10)^2+d^2}$.
     \end{itemize}
     Tổng cường độ sáng tại $P$ là $I(x) = I_1+I_2 = k\left[\dfrac{1}{x^2+d^2} + \dfrac{1}{(x-10)^2+d^2}\right]$.
     \item Mệnh đề đúng.
     Khi $d=5$ thì $I(x) = k\left[\dfrac{1}{x^2+25} + \dfrac{1}{(x-10)^2+25}\right]$; $I'(x) = k\left[\dfrac{-2x}{(x^2+25)^2} + \dfrac{-2(x-10)}{((x-10)^2+25)^2}\right]$.
     Ta có $I'(x)=0 \Leftrightarrow k\left[\dfrac{-2x}{(x^2+25)^2} + \dfrac{-2(x-10)}{((x-10)^2+25)^2}\right] = 0 \Leftrightarrow x=5$.
     
     Bảng biến thiên:
     \begin{center}
         \begin{tabular}{|c|ccccc|}
             \hline
             $x$ & $0$ & & $5$ & & $10$ \\
             \hline
             $I'(x)$ & & $-$ & $0$ & $+$ & \\
             \hline
         \end{tabular}
     \end{center}
     Ta thấy cường độ ánh sáng tại $P$ đạt cực tiểu khi $x=5$ mét.
     \item Mệnh đề đúng.
     Khi $d=10$ thì $I(x) = k\left[\dfrac{1}{x^2+100} + \dfrac{1}{(x-10)^2+100}\right]$.
     \begin{itemize}
         \item Cách giải 1: Học sinh có thể làm như câu c): lập bảng xét dấu đạo hàm và kết luận.
         \item Cách giải 2: So sánh cường độ sáng tại các điểm đặc biệt.
         \begin{itemize}
             \item Tại $x=0$ thì $I(0) = k\left(\dfrac{1}{100} + \dfrac{1}{200}\right) = \dfrac{3k}{200}$.
             \item Tại $x=5$ thì $I(5) = k\left(\dfrac{1}{200} + \dfrac{1}{100}\right) = \dfrac{2k}{125}$. (Chỗ này trong ảnh là $I(5) = k(\dfrac{1}{200} + \dfrac{1}{100}) = \dfrac{2k}{125}$, nhưng $k(\dfrac{1}{200} + \dfrac{1}{100}) = k(\dfrac{1+2}{200}) = \dfrac{3k}{200}$. Nếu $I(5) = k(\dfrac{1}{125} + \dfrac{1}{25+100-100+25}) = k(\dfrac{1}{125}+\dfrac{1}{50})$? Kiểm tra lại đề gốc của $I(5)$ khi $d=10$. $I(5) = k\left[\dfrac{1}{5^2+100} + \dfrac{1}{(5-10)^2+100}\right] = k\left[\dfrac{1}{25+100} + \dfrac{1}{(-5)^2+100}\right] = k\left[\dfrac{1}{125} + \dfrac{1}{25+100}\right] = k\left[\dfrac{1}{125} + \dfrac{1}{125}\right] = \dfrac{2k}{125}$. Phần tính toán trong ảnh là đúng.)
         \end{itemize}
     \end{itemize}
     Vì $\dfrac{3k}{200} < \dfrac{2k}{125}$ nên cường độ sáng không đạt cực tiểu tại $x=5$ ($P$ không ở chính giữa $l$).
 \end{listEX}}
 \end{ex}
% 
 \begin{ex}%Câu 15
  Hộp A có 5 bi đỏ và 3 bi vàng, hộp B có 2 bi đỏ và 2 bi vàng, hộp C có 2 bi đỏ và 2 bi vàng. Lấy ngẫu nhiên 1 bi từ hộp A bỏ sang hộp B, rồi lấy ngẫu nhiên 2 bi từ hộp B bỏ sang hộp C, sau cùng lấy ngẫu nhiên 3 bi từ hộp C.
 Xét tính đúng sai các mệnh đề sau: 

 \choiceTF
 {Xác suất để lấy được 1 bi vàng từ hộp A bằng $\dfrac{5}{8}$}
 {Xác suất để lấy được 2 bi khác màu từ hộp B bằng $\dfrac{2}{5}$}
 {Xác suất để lấy được cả 3 bi màu đỏ từ hộp C bằng $\dfrac{1}{20}$}
 {\True Biết rằng 3 bi được lấy từ hộp C màu đỏ, xác suất để 3 bi đó vốn thuộc về 3 hộp khác nhau bằng $\dfrac{1}{6}$}
 \loigiai{
 a) Mệnh đề sai.
 Xác suất để lấy được 1 bi vàng từ hộp A bằng $\dfrac{3}{8}$.\\
 b) Mệnh đề sai.
 Ta có sơ đồ bài toán như hình bên. Gọi X là biến cố lấy được hai bi khác màu từ hộp B: $ P(X)=\dfrac{5}{8}\times\dfrac{3}{5}+\dfrac{3}{8}\times\dfrac{3}{5}=\dfrac{3}{5}$.\\
 c) Mệnh đề sai.
 Gọi Y là biến cố lấy được cả 3 bi đỏ từ hộp C, ta có: $ P(Y)=\dfrac{5}{8}\times\left(\dfrac{3}{10}\times\dfrac{C_4^3}{C_6^3}+\dfrac{3}{5}\times\dfrac{C_3^3}{C_6^3}\right)+\dfrac{3}{8}\times\left(\dfrac{1}{10}\times\dfrac{C_4^3}{C_6^3}+\dfrac{3}{5}\times\dfrac{C_3^3}{C_6^3}\right)=\dfrac{3}{40}$.\\
 d) Mệnh đề đúng.
 Gọi Z là biến cố: “Lấy từ hộp C 3 viên bi mà mỗi bi thuộc về mỗi hộp A, B, C trước đó”.\\
 Ta có: $ P\left(Z|Y\right)=\dfrac{P\left(ZY\right)}{P(Y)}=\dfrac{\dfrac{5}{8}\times\dfrac{1\times 2}{C_5^2}\times\dfrac{1\times 1\times 2}{C_6^3}}{\dfrac{3}{40}}=\dfrac{1}{6}$.}
 \end{ex}
%
 \begin{ex}
     \immini[thm]{ Trong một mô hình game 3D, với hệ trục tọa độ $Oxyz$ cho trước, đơn vị trên mỗi trục là mét, mặt đất là mặt phẳng $(Oxy)$. Người chơi cùng với khẩu súng của anh ta được xem như một chất điểm, xuất phát từ vị trí gốc tọa độ $O$ và di chuyển trên mặt đất. Hai mục tiêu nhắm bắn là các vị trí cố định $(5;5;5)$, $(5;10;10)$.
         Để tăng thêm độ hấp dẫn trò chơi, người ta dùng công nghệ hologram tạo ra một quả cầu giả lập bán kính thay đổi, có ánh sáng lấp lánh luôn đi qua hai điểm mục tiêu và lăn trên mặt đất, làm người chơi có phần hoa mắt và khó nhắm bắn trúng các mục tiêu này.
         Xét tính đúng sai các mệnh đề sau:
         }{\includegraphics[scale=1]{img/HXN-6.16}}
   \choiceTF
   {Mục tiêu gần nhất cách người chơi khoảng $8,5~m$ (làm tròn đến hàng phần chục)}
   {\True Biết viên đạn có thể bắn xuyên quả cầu, người chơi cần đứng vị trí $(5;0;0)$ để bắn trúng cùng lúc hai mục tiêu}
   {Gọi $M$ là điểm tiếp xúc của quả cầu với mặt đất thì $M$ luôn thuộc một đường tròn cố định có bán kính bằng $8~m$}
   {\True Khoảng cách ngắn nhất từ vị trí người chơi (gốc tọa độ $O$) đến điểm $M$ bằng $5~m$}
    \loigiai{
    \begin{listEX}
        \item Mệnh đề sai.
        Gọi $A(5;5;5)$, $B(5;10;10) \Rightarrow \overrightarrow{AB}=(0;5;5)=5(0;1;1)$.
        $OA = \sqrt{5^2+5^2+5^2} = 5\sqrt{3}$; $OB = \sqrt{5^2+10^2+10^2} = 15 > OA$.
        Do vậy mục tiêu $A$ gần người chơi nhất, có khoảng cách $OA = 5\sqrt{3} \approx 8,7~m$.
        
        \item Mệnh đề đúng.
        Người chơi cần đứng vị trí $I$ sao cho $I, A, B$ thẳng hàng để bắn trúng cùng lúc hai mục tiêu.
        $AB$ qua $A(5;5;5)$, vectơ chỉ phương $\vec{u}=(0;1;1)$ nên có phương trình $AB: \begin{cases} x=5 \\ y=5+t \\ z=5+t \end{cases}$.
        Gọi $I(5;5+t;5+t) \in AB$; $I \in (Oxy): z=0 \Rightarrow 5+t=0 \Rightarrow t=-5 \Rightarrow I(5;0;0)$.
        
        \item Mệnh đề sai.
        Vì $IM$ là tiếp tuyến của mặt cầu nên $IM^2 = IA \cdot IB$
        $= \sqrt{0^2+5^2+5^2} \cdot \sqrt{0^2+10^2+10^2} = 100$
        $\Rightarrow IM=10$.
        Do vậy $M$ luôn thuộc một đường tròn tâm $I$, bán kính $r=10~m$.
        (Hình vẽ minh họa một mặt cầu tiếp xúc với mặt phẳng $(Oxy)$ tại $M$, với $A, B$ là hai điểm trên mặt cầu, $I$ là tâm đường tròn ngoại tiếp tam giác $ABM'$ với $M'$ là hình chiếu của $M$ trên $AB$, hoặc $I$ là điểm trên $(Oxy)$ sao cho $IA, IB$ là tiếp tuyến tới đường tròn $(M, R)$.)
        
        \item Mệnh đề đúng.
        Ta có $OI=5 < 10 = R$. Vì vậy khoảng cách từ $O$ đến tâm quả cầu ngắn nhất khi $I, O, M$ thẳng hàng theo thứ tự đó.
        Ta có $OM_{\min} = R - OI = 10-5=5$.
        Vậy khoảng cách ngắn nhất từ vị trí người chơi đến điểm tiếp xúc quả cầu, mặt đất bằng $5~m$.
        (Hình vẽ minh họa một đường tròn trên mặt phẳng $(Oxy)$ tâm $I$ bán kính $R$. Điểm $O$ nằm trong đường tròn. $M$ là điểm trên đường tròn sao cho $O, M, I$ thẳng hàng và $OM$ ngắn nhất.)
    \end{listEX}
    \begin{center}
        \includegraphics[scale=.7]{img/HXN-6.16a}
    \end{center}
    }
\end{ex}
\Closesolutionfile{ans}
\caukq
\Opensolutionfile{ans}[ans/ans-HXN-\sode-SA]
% 
 \begin{ex}%Câu 17
 Mai là một cô gái dễ mến, tuy nhiên trong chuyện tình cảm thì cô không phải là người đơn giản. Hôm ấy có người bạn trai hẹn đi ăn trưa, Mai cho biết cô sẽ gặp người đó vào thời điểm kim giờ và kim phút gặp nhau lần đầu tiên kể từ sau 12h trưa. Nếu người bạn trai ấy đến địa điểm hẹn vào lúc 12h30 trưa thì anh ấy sẽ phải chờ Mai bao nhiêu phút (làm tròn đến hàng phần chục)?
 \shortans{35,5}
 \loigiai{
 Mỗi phút kim giờ quét được một góc $\dfrac{2\pi}{60.12}=\dfrac{\pi}{360}$ (rad).\\
 Mỗi phút kim phút quét được một góc $\dfrac{2\pi}{60}=\dfrac{\pi}{30}$ (rad).\\
 Gọi t là thời gian (phút) mà kim giờ và kim phút gặp nhau kể từ 12h trưa.\\
 Ta có $\dfrac{\pi}{360}t+2\pi=\dfrac{\pi}{30}t\Rightarrow t\approx 65,5$ (phút).\\
 Người bạn trai đến điểm hẹn lúc 12h30 nên phải chờ Mai khoảng 35,5 phút.}
 \end{ex}
 
 \begin{ex}%Câu 18
     \immini[thm] {Một vật đựng đồ chơi lắp ghép của trẻ con có dạng hình trụ với chiều cao bằng 15 cm. Người ta nhìn vào bên trong hình trụ này thì thấy có các mảnh ghép hình vuông được đặt vừa khít như hình vẽ. Biết mỗi mảnh ghép hình vuông có cạnh 2 cm. Thể tích vật đựng hình trụ này là bao nhiêu $c{m^3}$ (làm tròn đến hàng đơn vị, bỏ qua độ dày của vỏ hộp).  \shortans{589}}{  \includegraphics[scale=.8]{img/HXN-6.18}}

 \loigiai{
 Dựng hệ trục Oxy như hình vẽ, mỗi hình vuông cạnh 1 đơn vị tương ứng với 2 cm). Gọi phương trình đường tròn đáy hình trụ là $x^2+y^2-2ax-2by+c=0$ với $a^2+b^2-c>0$ .\\
 Các điểm $\left(-4\,;\,\,0\right)\,,\,\,\left(-4\,;\,\,1\right)\,,\,\,\left(2\,;\,\,-2\right)$ thuộc đường tròn đáy nên $\left\{\begin{aligned}
 & 16+0+8a-0+c=0\\ 
 & 16+1+8a-2b+c=0\\ 
 & 4+4-4a+4b+c=0\\ 
 \end{aligned}\right.\Leftrightarrow\left\{\begin{aligned}
 & a=-\dfrac{1}{2}\\ 
 & b=\dfrac{1}{2}\\ 
 & c=-12\\ 
 \end{aligned}\right.$ .\\
 Bán kính đáy hình trụ là: $\sqrt{a^2+b^2-c}=\dfrac{5\sqrt{2}}{2}$ ; bán kính thực tế: $R=\dfrac{5\sqrt{2}}{2}\times 2=5\sqrt{2}$ cm.\\
 Thể tích vật đựng hình trụ là $V=\pi{R^2}h=\pi{\left(5\sqrt{2}\right)^2}\cdot 15=750\pi\approx\,\,c{m^3}$ .}
 \end{ex}
 
 \begin{ex}%Câu 19
     \immini[thm]{ Khi ca sĩ ST bước ra sân khấu, có một đèn pha luôn chiếu sáng vào anh. Đèn pha được đặt ở vị trí B, cách đoạn đường mà ca sĩ đang đi một khoảng BH bằng 10 m. Biết rằng ST Cát đi ra với tốc độ 1 m/s, khi ca sĩ cách điểm H trên con đường khoảng 6 m thì đèn pha đang quay với tốc độ bao nhiêu rad/s (làm tròn đến hàng phần trăm)?
         \shortans{0,07}}{\includegraphics[scale=1]{img/HXN-6.19}}

 \loigiai{
 Gọi $\varphi=\widehat{ABH}$ với $0\le\varphi <\dfrac{\pi}{2}\,\,\,(rad)$ ; đăt $x=HA$ (thay đổi).\\
 Tam giác ABH vuông tại H có $\tan\varphi=\dfrac{AH}{BH}=\dfrac{x}{10}\Rightarrow\,\,\,(1)$ .\\
 Đạo hàm hai vế của (1) theo t, ta được: $\,\,\,(2)$ .\\
 Khi $x=6\,\,m$ thì (1) trở thành $6=10\tan\varphi\Rightarrow\varphi\approx 0,54\,\,rad$ (lưu vào A).\\
 Thay lần lượt $\varphi\approx 0,54\,\,rad$ và $\dfrac{\text{d}x}{\text{d}t}=1$ m/s vào (2) ta được $\dfrac{\text{d}\varphi}{\text{d}t}=\dfrac{5}{68}\approx $ rad/s.}
\end{ex}

\begin{ex}%Câu 20
    \immini[thm]
{ Một con châu chấu nhảy lên cầu thang có 18 bậc. Mỗi lần nhảy con châu chấu có thể nhảy 1 bậc hoặc 2 bậc. Tính xác suất để con châu chấu hoàn thành 18 bậc thang với số lần nhảy 2 bậc không bé hơn số lần nhảy 1 bậc (làm tròn kết quả đến hàng phần trăm).
 \shortans{0,31}}{\includegraphics[scale=1]{img/HXN-6.20}}
\loigiai{
 Gọi x là số bước nhảy 1 bậc, y là số bước nhảy 2 bậc; suy ra $\left\{\begin{aligned}
 & x+2y=18\\ 
 & x\,,\,\,y\in\mathbb{N}\\ 
 \end{aligned}\right.$ .\\
 Các cặp (x ; y) thỏa mãn là $\left(18\,;\,\,0\right)\,,\,\,\left(16\,;\,\,1\right)\,,\,\,\left(14\,;\,\,2\right)\,,\,\,\left(12\,;\,\,3\right)\,,\,\,\left(10\,;\,\,4\right)$ , $\left(8\,;\,\,5\right)\,,\,\,\left(6\,;\,\,6\right)\,,\,\,\left(4\,;\,\,7\right)\,,\,\,\left(2\,;\,\,8\right)\,,\,\,\left(0\,;\,\,9\right)$ .\\
 • Nếu $\left(x\,;\,\,y\right)\in\left\{\left(18\,;\,\,0\right)\,,\,\,\left(0\,;\,\,9\right)\right\}$ thì con châu chấu có 2 cách đi.\\
 • Nếu $\left(x\,;\,\,y\right)=\left(16\,;\,\,1\right)$ thì số cách đi của châu chấu là $C_{17}^1$ .\\
 • Tương tự như vậy các trường hợp còn lại sẽ có số cách là $C_{16}^2+C_{15}^3+C_{14}^4+C_{13}^5+C_{12}^6+C_{11}^7+C_{10}^8=4\,162$ (cách).\\
 Vậy tổng số cách nhảy của châu chấu để thoàn thành 18 bậc cầu thang là $2+C_{17}^1+4162=$ (cách). Số phần tử không gian mẫu là $n\left(\Omega\right)=$ .\\
 Số lần nhảy 2 bậc không bé hơn số lần nhảy 1 bậc nên $\left(x\,;\,\,y\right)\in\left\{\left(6\,;\,\,6\right)\,,\,\,\left(4\,;\,\,7\right)\,,\,\,\left(2\,;\,\,8\right)\,,\,\,\left(0\,;\,\,9\right)\right\}$ .\\
 Gọi A là biến cố thỏa đề bài thì $n(A)=C_{12}^6+C_{11}^7+C_{10}^8+C_9^9=$ .\\
 Do vậy $P(A)=\dfrac{n(A)}{n\left(\Omega\right)}=\dfrac{1\,300}{4\,180}=\dfrac{65}{209}\approx $ .}
 \end{ex}
 
 \begin{ex}%Câu 21
Ông Vượng mới khai phá được một mảnh đất hình chữ nhật, nhà nước chưa cấp sổ nên ông cũng chưa biết rõ diện tích mảnh đất là bao nhiêu, chỉ nhớ rằng bản thân là học sinh giỏi toán 12 năm liền thời phổ thông mà thôi. Mảnh đất của ông Vượng nằm ở một vị trí thuận lợi để trồng trọt vì có một dòng suối nhỏ chảy qua với hình dáng một parabol, dòng suối nhỏ này chia mảnh đất ra làm hai phần có diện tích $S_1\,,\,\,S_2\,\,\,\left(S_1>S_2\right)$ . Riêng mảnh đất có diện tích $S_2$ được xem như hình phẳng giới hạn bởi parabol cùng hai tiếp tuyến vuông góc của parabol đó.
Vào vụ Hè thu, ông Vượng quyết định trồng lúa trên phần đất có diện tích $S_1$ và trồng ớt trên phần đất có diện tích $S_2$ . Dự kiến lợi nhuận mang lại từ việc trồng lúa là $30$ nghìn/ $m^2$ và lợi nhuận từ việc trồng ớt là $40$ nghìn/ $m^2$ (trong một vụ mùa).
Ông quyết định dựng hệ trục Oxy như hình vẽ với gốc O trùng với điểm cực trị của dòng suối dạng parabol, đơn vị trên mỗi trục là 100 mét.
\begin{center}
    \includegraphics[scale=.5]{img/HXN-6.21}
\end{center}
 Tính tổng lợi nhuận theo dự kiến của ông Vượng sau vụ Hè thu này (làm tròn đến hàng đơn vị của triệu đồng), biết rằng diện tích con suối không đáng kể.
\shortans{309}

\loigiai{
Đầu tiên ta đặt $ A\left(a\,;\,\,a^2\right)\,,\,\,B\left(b\,;\,\,b^2\right)\,,\,\,a>0\,,\,\,b<0$ là hai tiếp điểm ứng với hai tiếp tuyến vuông góc của parabol (P).\\
Gọi $d_1\,,\,\,d_2$ lần lượt là các tiếp tuyến của đồ thị $(C)$ tại $ A\,,\,\,B$, khi đó: $\left\{\begin{matrix}
 {d_1}:y=2ax-a^2\\
 {d_2}:y=2bx-b^2\\
\end{matrix}\right.$.\\
Do $d_1\bot{d_2}$ nên $ 2a\cdot 2b=-1\Rightarrow b=-\dfrac{1}{4a}\Rightarrow B\left(-\dfrac{1}{4a}\,;\,\,\dfrac{1}{16a^2}\right)$, khi đó $d_2:y=-\dfrac{x}{2a}-\dfrac{1}{16a^2}$.\\
Gọi $ E=d_2\cap{d_1}$, suy ra $ E\left(\dfrac{4a^2-1}{8a}\,;\,\,-\dfrac{1}{4}\right)$; $ EA=\dfrac{\sqrt{\left(4a+1\right)^3}}{8a}\,;\,\,EB=\dfrac{\sqrt{\left(4a+1\right)^3}}{16a^2}$.\\
Ta có $ EA=2EB$; suy ra $ a=1$. Do đó diện tích mảnh đất $ S=EA\cdot EB=\dfrac{\left(4a+1\right)^3}{128a^3}\left|\begin{aligned}
 &\\ 
 & a=1\\ 
\end{aligned}\right.=$.\\
Khi đó phương trình $d_1:y=2x-1\,;\,\,d_2:y=-\dfrac{x}{2}-\dfrac{1}{16}$ và $ A\left(1\,;\,\,1\right)\,,\,\,B\left(-\dfrac{1}{4}\,;\,\,\dfrac{1}{16}\right)\,,\,\,E\left(\dfrac{3}{8}\,;\,\,-\dfrac{1}{4}\right)$.\\
Diện tích $S_2=\displaystyle\int\limits_{-1/4}^{3/8}{\left[x^2-\left(-\dfrac{x}{2}-\dfrac{1}{16}\right)\right]\text{d}x}+\displaystyle\int\limits_{3/8}^1\left[x^2-\left(2x-1\right)\right]\text{d}x=$ và $S_1=S-S_2=$.\\
Tổng số tiền thu được của ông Vượng sau vụ hè thu bằng $\dfrac{625}{768}\times{10^2}\times 30+\dfrac{125}{768}\times{10^2}\times 40\approx 309\,244$ nghìn đồng $\approx $ triệu đồng.}
\end{ex}

\begin{ex}%Câu 22
    \immini[thm]{Trong không gian $ Oxyz$, cho mặt cầu $(S)$ có tâm $ I\left(-1\,;\,\,0\,;\,\,2\right)$ và đi qua điểm $ A\left(0\,;\,\,1\,;\,\,1\right)$. Xét các điểm $ B\,,\,\,C\,,\,\,D$ thuộc $(S)$ sao cho $ AB,\,\,AC,\,\,AD$ đôi một vuông góc với nhau. Thể tích của khối tứ diện $ ABCD$ có giá trị lớn nhất bằng bao nhiêu (làm tròn đền hàng phần trăm)?
        \shortans{1,33}}{\includegraphics[scale=.7]{img/HXN-6.22}}
\loigiai{
Ta nhận diện được đây là bài toán mặt cầu ngoại tiếp tứ diện có ba cạnh đôi một vuông góc nhau. Bán kính mặt cầu là $ R=IA=\sqrt{3}$.\\
Do $ AB,AC,AD$ đôi một vuông góc với nhau nên $ R=\dfrac{\sqrt{A{B^2}+A{C^2}+A{D^2}}}{2}$.\\
Suy ra $ A{B^2}+A{C^2}+A{D^2}=4R^2=12$.\\
Thể tích tứ diện: $$ V_{ABCD}=\dfrac{1}{6}AB.AC.AD=\dfrac{1}{6}\sqrt{A{B^2}.A{C^2}.A{D^2}}\overset{AM-GM}{\mathop{\le}}\,\dfrac{1}{6}\sqrt{\left(\dfrac{A{B^2}+A{C^2}+A{D^2}}{3}\right)^3}=\dfrac{1}{6}\sqrt{\left(\dfrac{12}{3}\right)^3}=\dfrac{4}{3}$$
Do đó $\left(V_{ABCD}\right)_{Max}=\dfrac{4}{3} $. Dấu đẳng thức xảy ra khi và chỉ khi $ AB=AC=AD=2$.}
\end{ex}
\Closesolutionfile{ans}
\inputansbox{6,4,3}{ans/ans-HXN-\sode-T,ans/ans-HXN-\sode-TF,ans/ans-HXN-\sode-SA}
% \def\sode{7}
\begin{name}
	{\tenchude}
	{\tendethi}
	{\tentruong}
	{\thoigian}
\end{name}
\caulc
\Opensolutionfile{ans}[ans/ans-HXN-\sode-T]
\begin{ex}%Câu 1
 \immini[thm]{ Cho hàm số $y=f(x)$ liên tục trên $\left[-1\,;+\infty\right)$ và có đồ thị như hình vẽ. Tìm giá trị lớn nhất của hàm số $y=f(x)$ trên $\left[1\,;\,\,4\right]$ .
 \choice
 {0}
 {1}
 {4}
 {\True 3}}{\includegraphics[scale=.8]{img/HXN-7.1}}
 \loigiai{
 Chọn D.}
\end{ex}
\begin{ex}%Câu 2
 Cho $\log_ab=2$ (với $a>0\,,\,\,b>0\,,\,\,a\ne 1$). Tính $\log_a\left(a\cdot b\right)$ .
 \choice
 {$2$}
 {$4$}
 {$5$}
 {\True $3$}
 \loigiai{
 Chọn D.\\
 Ta có: $\log_a\left(a\cdot b\right)=\log_aa+\log_ab=1+2=3$ .}
\end{ex}
\begin{ex}%Câu 3
 Trong không gian $Oxyz$ , cho mặt cầu $(S):{(x-1)^2}+(y+2)^2+(z-3)^2=16$ . Tâm của $(S)$ có tọa độ là
 \choice
 {$\left(-1\,;\,\,-2\,;\,\,-3\right)$}
 {$\left(1\,;\,\,2\,;\,\,3\right)$}
 {$\left(-1\,;\,\,2\,;\,\,-3\right)$}
 {\True $\left(1\,;\,\,-2\,;\,\,3\right)$}
 \loigiai{
 Chọn D.\\
 Mặt cầu (S) có tâm $I\left(1\,;\,\,-2\,;\,\,3\right)$ , bán kính $R=4$ .}
\end{ex}
\begin{ex}%Câu 4
 Hai đường tiệm cận của đồ thị hàm số $y=\dfrac{2x+1}{x-1}$ tạo với hai trục tọa độ một hình chữ nhật có diện tích bằng bao nhiêu?
 \choice
 {\True $2$}
 {$1$}
 {$3$}
 {$4$}
 \loigiai{
 Chọn A.\\
 Đồ thị hàm số $y=\dfrac{2x+1}{x-1}$ có đường tiệm cận đứng $x=1$ và đường tiệm cận ngang $y=2$ .\\
 Gọi A là giao điểm của tiệm cận đứng với Ox, suy ra $OA=1$ .\\
 Gọi B là giao điểm của tiệm cận ngang với Oy, suy ra $OB=2$ .\\
 Hình chữ nhật cần tính diện tích là hình chữ nhật OAIB với $I\left(1\,;\,\,2\right)$ là tâm đối xứng đồ thị.\\
 Diện tích hình chữ nhật ABCD là $S=OA\cdot OB=2$ .}
\end{ex}
\begin{ex}
 Doanh thu bán hàng trong 20 ngày được lựa chọn ngẫu nhiên của một cửa hàng được ghi lại ở bảng sau (đơn vị: triệu đồng):
 \begin{center}
\begin{tabular}{|c|c|c|c|c|c|}
    \hline
    Doanh thu & $[5;7)$ & $[7;9)$ & $[9;11)$ & $[11;13)$ & $[13;15)$ \\
    \hline
    Số ngày & 2 & 7 & 7 & 3 & 1 \\
    \hline
\end{tabular}
 \end{center}
 Giá trị trung bình của mẫu số liệu ghép nhóm trên thuộc khoảng nào sau đây?
 \choice
 {$[7; 9)$}
 {\True $[9; 11)$}
 {$[11; 13)$}
 {$[13; 15)$}
 \loigiai{
 Chọn B.
 Ta viết lại mẫu số liệu ghép nhóm có thêm giá trị đại diện như sau:
 \begin{center}
 \begin{tabular}{|c|c|c|c|c|c|}
 \hline
 Doanh thu & $[5;7)$ & $[7;9)$ & $[9;11)$ & $[11;13)$ & $[13;15)$ \\
 \hline
 Giá trị đại diện & 6 & 8 & 10 & 12 & 14 \\
 \hline
 Số ngày & 2 & 7 & 7 & 3 & 1 \\
 \hline
 \end{tabular}
 \end{center}
 Giá trị trung bình của mẫu số liệu ghép nhóm là
 $ \bar{x} = \dfrac{2 \cdot 6 + 7 \cdot 8 + 7 \cdot 10 + 3 \cdot 12 + 1 \cdot 14}{20} = \dfrac{12 + 56 + 70 + 36 + 14}{20} = \dfrac{188}{20} = 9,4 $.
 Vì $9,4 \in [9;11)$ nên đáp án đúng là B.
 }
\end{ex}
\begin{ex}%Câu 6
 Cho cấp số nhân $\left(u_n\right)$ với $u_1=2$ và công bội $q=3$ . Tìm số hạng thứ $4$ của cấp số nhân?
 \choice
 {$24$}
 {\True $54$}
 {$162$}
 {$48$}
 \loigiai{
 Chọn B.\\
 Ta có: $u_4=u_1\cdot{q^3}=2\cdot{3^3}=54.$}
\end{ex}
\begin{ex}%Câu 7
 Cho hai hàm số $y=f(x)$ và $y=g(x)$ liên tục trên $\left[a\,;\,\,b\right]$ . Diện tích hình phẳng giới hạn bởi đồ thị của các hàm số $y=f(x)$ , $y=g(x)$ và các đường thẳng $x=a$ , $x=b$ bằng
 \choice
 {$\left|\displaystyle\int\limits_a^b{\left[f(x)-g(x)\right]\text{d}x}\right|$}
 {$\displaystyle\int\limits_a^b{\left| f(x)+g(x)\right|\text{d}x}$}
 {\True $\displaystyle\int\limits_a^b{\left| f(x)-g(x)\right|\text{d}x}$}
 {$\displaystyle\int\limits_a^b{\left[f(x)-g(x)\right]\text{d}x}$}
 \loigiai{
 Chọn C.}
\end{ex}
\begin{ex}%Câu 8
 Điểm kiểm tra 15 phút của lớp 12A được cho bởi bảng sau:\\
 \centerline{\begin{tabular}{|c|c|c|c|c|c|c|c|}
 \hline
 Điểm &[3; 4) &[4; 5) &[5; 6) &[6; 7) &[7; 8) &[8; 9) &[9; 10)\\
 \hline
 Số học sinh & 3 & 8 & 7 & 12 & 7 & 1 & 1\\
 \hline
 \end{tabular}}\\
 Tứ phân vị thứ nhất của mẫu số liệu ghép nhóm trên (làm tròn đến hàng phần trăm) là
 \choice
 {\True $4,84$}
 {$2,10$}
 {$2,09$}
 {$6,94$}
 \loigiai{
 Chọn A.\\
 Mẫu số liệu ghép nhóm có cỡ mẫu là $n=3+8+7+12+7+1+1=39$ .\\
 Tứ phân vị thứ nhất của mẫu số liệu gốc là $x_{10}\in\left[4\,;\,\,5\right)$ ; do đó tứ phân vị thứ nhất của mẫu số liệu ghép nhóm là $Q_1=4+\dfrac{\dfrac{39}{4}-3}{8}.1=\dfrac{155}{32}\approx 4,84$ .}
\end{ex}
\begin{ex}%Câu 9
 Trong không gian $Oxyz$ , cho đường thẳng $\Delta :\left\{\begin{aligned}
 & x=1-2t\\ 
 & y=-1\\ 
 & z=3+t\\ 
 \end{aligned}\right.$ . Vectơ nào sau đây là một vectơ chỉ phương của đường thẳng $\Delta $ ?
 \choice
 {$(-2\,;\,\,-1\,;\,\,1)$}
 {$(1\,;\,\,-1\,;\,\,3)$}
 {\True $(-2\,;\,\,0\,;\,\,1)$}
 {$(2\,;\,\,0\,;\,\,1)$}
 \loigiai{
 Chọn C.\\
 $\Delta $ có vectơ chỉ phương $\vec{u}=\left(-2\,;\,\,0\,;\,\,1\right)$ .}
\end{ex}
\begin{ex}%Câu 10
 Cho tích phân $\displaystyle\int\limits_0^1\left[f(x)+2x\right]\text{d}x=2$ . Khi đó tích phân $\displaystyle\int\limits_0^1f(x)\text{d}x$ bằng ?
 \choice
 {\True $1$}
 {$4$}
 {$2$}
 {$0$}
 \loigiai{
 Chọn A.\\
 Ta có: $\displaystyle\int\limits_0^1\left[f(x)+2x\right]\text{d}x=2\Leftrightarrow\displaystyle\int\limits_0^1f(x)\text{d}x+\left.x^2\right|_0^1=2\Leftrightarrow\displaystyle\int\limits_0^1f(x)\text{d}x+1=2\Leftrightarrow\displaystyle\int\limits_0^1f(x)\text{d}x=1$ .}
\end{ex}
\begin{ex}%Câu 11
 Trong không gian $Oxyz$ , cho ba điểm $A\left(1\,;\,\,1\,;\,\,1\right)$ , $B\left(0\,;\,\,2\,;\,\,1\right)$ và $C\left(1\,;\,\,-1\,;\,\,2\right)$ . Mặt phẳng đi qua $A$ và vuông góc với $BC$ có phương trình là
 \choice
 {$\dfrac{x+1}{1}=\dfrac{y+1}{-3}=\dfrac{z+1}{1}$}
 {$x-3y+z-1=0$}
 {\True $x-3y+z+1=0$}
 {$\dfrac{x-1}{1}=\dfrac{y-1}{-3}=\dfrac{z-1}{1}$}
 \loigiai{
 Chọn C.\\
 Mặt phẳng qua $A\left(1\,;\,\,1\,;\,\,1\right)$ , có vectơ pháp tuyến $\overrightarrow{BC}=\left(1\,;\,\,-3\,;\,\,1\right)$ nên có phương trình\\
 $1(x-1)-3(y-1)+1(z-1)=0\Leftrightarrow x-3y+z+1=0$ .}
\end{ex}
\begin{ex}%Câu 12
 Họ nguyên hàm của hàm số $f(x)=e^{2x}+\dfrac{3}{x}$ là
 \choice
 {$\mathop{\displaystyle\int}f(x)\text{d}x=e^{2x}+3\text{ln}x+C$}
 {\True $\mathop{\displaystyle\int}f(x)\text{d}x=\dfrac{e^{2x}}{2}+3\text{ln}\left| x\right|+C$}
 {$\mathop{\displaystyle\int}f(x)\text{d}x=\dfrac{e^{2x}}{2}+3\text{ln}x+C$}
 {$\mathop{\displaystyle\int}f(x)\text{d}x=e^{2x}+3\text{ln}\left| x\right|+C$}
 \loigiai{
 Chọn B.\\
 Ta có: $\mathop{\displaystyle\int}f(x)\text{d}x=\mathop{\displaystyle\int}\left(e^{2x}+\dfrac{3}{x}\right)\text{d}x=\dfrac{e^{2x}}{2}+3\text{ln}\left| x\right|+C$ .}
 \end{ex}
 \Closesolutionfile{ans}
 \cauds
 \Opensolutionfile{ans}[ans/ans-HXN-\sode-TF]
\begin{ex}
 Cho hàm số $f(x) = \begin{cases} 3 & \text{khi } x \le 1 \\ ax+b & \text{khi } 1 < x < 2 \\ 5 & \text{khi } x \ge 2 \end{cases}$.
 Xét tính đúng sai các mệnh đề sau:
 \choiceTF
 {Hàm số liên tục trên khoảng $(-\infty; 1)$}
 {Hàm số không liên tục trên khoảng $(1; 2)$}
 {Hàm số liên tục tại $x=1$ khi $a+b=5$}
 {\True Hàm số liên tục trên $\mathbb{R}$ khi và chỉ khi $a=2, b=1$}
 \loigiai{
 \begin{listEX}
 \item Mệnh đề đúng.
 Khi $x<1$ thì $f(x)=3$ là hàm hằng số nên $f(x)$ liên tục trên $(-\infty; 1)$.
 \item Mệnh đề sai.
 Khi $x \in (1;2)$ thì $f(x)=ax+b$ là hàm số bậc nhất (nếu $a$ khác $0$) hoặc là hàm số không đổi (nếu $a=0$), do đó $f(x)$ liên tục trên $(1;2)$.
 \item Mệnh đề sai.
 Ta có: $\lim_{x \to 1^+} f(x) = 3$; $f(1)=3$; $\lim_{x \to 1^-} f(x) = \lim_{x \to 1^-} (ax+b) = a+b$.
 Hàm số liên tục tại $x=1$ suy ra $\lim_{x \to 1^+} f(x) = \lim_{x \to 1^-} f(x) = f(1) \Rightarrow a+b=3$.
 \item Mệnh đề đúng.
 Dễ thấy hàm số $f(x)$ liên tục trên các khoảng $(-\infty;1)$, $(1;2)$ và $(2;+\infty)$. Vì vậy hàm số liên tục trên $\mathbb{R}$ khi và chỉ khi hàm số liên tục tại các điểm $x=1; x=2$.
 Ta có: $\lim_{x \to 2^-} f(x) = \lim_{x \to 2^-} (ax+b) = 2a+b$; $\lim_{x \to 2^+} f(x)=5$; $f(2)=5$.
 Hàm số liên tục tại $x=2$ suy ra $\lim_{x \to 2^-} f(x) = \lim_{x \to 2^+} f(x) = f(2) \Rightarrow 2a+b=5$.
 Kết hợp với câu c) ta có hệ phương trình $\begin{cases} a+b=3 \\ 2a+b=5 \end{cases} \Leftrightarrow \begin{cases} a=2 \\ b=1 \end{cases}$.
 \end{listEX}
 }
\end{ex}
\begin{ex}
 Trên một vùng cao nguyên rộng lớn, với hệ tọa độ $Oxyz$ thích hợp, đơn vị trên mỗi trục tọa độ là 5 mét, một con đại bàng đang đậu trên vách đá phẳng được mô hình hóa bởi phương trình $(P): 2x+2y-z+9=0$. Con đại bàng này đang ngắm các mục tiêu là hai con dê núi đang ở các vị trí $A(1;2;-3)$ và $B(-2;-2;1)$.
 \choiceTF
 {\True Con dê ở vị trí $B$ thuộc vách núi đá nơi đại bàng đang đậu}
 {Khoảng cách giữa hai con dê núi là $\sqrt{41}$ mét}
 {Khoảng cách ngắn nhất từ đại bàng đến con dê ở vị trí $A$ bằng 32 mét}
 {\True Đại bàng luôn quan sát hai con dê với một góc $90^\circ$ và con dê ở vị trí $B$ cũng đã biết được sự nguy hiểm sau lưng nó; khoảng cách xa nhất giữa nó với đại bàng bằng 11,2 mét (làm tròn đến hàng phần chục)}
 \loigiai{
 \begin{listEX}
 \item Mệnh đề đúng.
 Thay tọa độ $B$ vào phương trình $(P): 2x+2y-z+9=0$ thì $2(-2)+2(-2)-(1)+9 = -4-4-1+9=0$ (thỏa mãn).
 Do đó con dê ở vị trí $B$ thuộc vách núi đá nơi đại bàng đang đậu.
 \item Mệnh đề sai.
 Ta có: $\overrightarrow{AB}=(-3;-4;4) \Rightarrow AB = \sqrt{(-3)^2+(-4)^2+4^2} = \sqrt{9+16+16} = \sqrt{41}$.
 Khoảng cách thực tế hai con dê là $5 \cdot AB = 5\sqrt{41}$ mét.
 \item Mệnh đề sai.
 Gọi $H$ là hình chiếu của $A$ trên $(P)$. Đường thẳng $AH$ đi qua $A(1;2;-3)$ và nhận $\vec{n_P}=(2;2;-1)$ làm vectơ chỉ phương.
 Phương trình $AH: \begin{cases} x=1+2t \\ y=2+2t \\ z=-3-t \end{cases}$.\\
 Vì $H \in AH$, tọa độ $H$ có dạng $(1+2t; 2+2t; -3-t)$.\\
 Mà $H \in (P)$ nên $2(1+2t)+2(2+2t)-(-3-t)+9=0 \Rightarrow 2+4t+4+4t+3+t+9=0 \Rightarrow 9t+18=0 \Rightarrow t=-2$.
$ \Rightarrow H(-3;-2;-1)$.
 Khoảng cách ngắn nhất từ đại bàng (trên vách đá $P$) đến con dê $A$ chính là khoảng cách từ $A$ đến mặt phẳng $(P)$, tức là $5 \cdot AH = 5 \cdot 6 = 30$ mét.
 \item Mệnh đề đúng.
 Gọi $M$ là vị trí đại bàng trên vách đá (mặt phẳng $(P)$).
 $\begin{cases} BM \perp AH \\ BM \perp AM \end{cases} \Rightarrow BM \perp (AMH) \Rightarrow BM \perp MH$. Do đó $BM \le BH = \sqrt{5}$\\
Vậy khoảng cách lớn nhất là $5 \cdot BH = 5\sqrt{5} \approx 11,2$ mét.
  \end{listEX}
 }
\end{ex}
\begin{ex}
 Cho ba biến cố $A, B, C$, trong đó các cặp biến cố $A$ và $C$ là độc lập, $B$ và $C$ là độc lập, $A$ và $B$ là xung khắc.
 Biết rằng $P(A \cup C) = \dfrac{2}{3}$, $P(B \cup C) = \dfrac{3}{4}$, $P(A \cup B \cup C) = \dfrac{11}{12}$; đặt $a=P(A), b=P(B), c=P(C)$.
 \choiceTF
 {\True $P(A \cap C) = P(A) \cdot P(C)$; $P(A \cap B) = P(A)+P(B)$}
 {$a+c = \dfrac{2}{3}$; $b+c = \dfrac{3}{4}$}
 {$a+b+c-ac = \dfrac{11}{12}$}
 {Xác suất để $A$ xảy ra nếu $B$ hay $C$ xảy ra bằng $\dfrac{1}{9}$}
 \loigiai{
 \includegraphics[scale=.8]{img/HXN-7.15}
 \begin{listEX}
     \item Mệnh đề đúng.\\
     Vì $A, C$ độc lập nên $P(A \cap C) = P(A) \cdot P(C)$; tương tự $P(B \cap C) = P(B) \cdot P(C)$.\\
     Vì $A$ và $B$ xung khắc nên $P(A \cup B) = P(A)+P(B)$ và $P(A \cap B)=0$.
     \item Mệnh đề sai.
     Ta có: $P(A \cup C) = P(A)+P(C)-P(A \cap C)$
     $= P(A)+P(C)-P(A)P(C) = a+c-ac = \dfrac{2}{3}$ (1);\\
     $P(B \cup C) = P(B)+P(C)-P(B \cap C)$
     $= P(B)+P(C)-P(B)P(C) = b+c-bc = \dfrac{3}{4}$ (2).
     \item Mệnh đề sai.\\
     Ta có: $P(A \cup B \cup C) = P(A)+P(B)+P(C)-P(A \cap B)-P(A \cap C)-P(B \cap C)+P(A \cap B \cap C)$
     $= P(A)+P(B)+P(C)-P(A)P(C)-P(B)P(C) = a+b+c-ac-bc = \dfrac{11}{12}$ (3).\\
     (Dễ thấy vì $A$ và $B$ xung khắc nên $P(A \cap B)=0$ và $P(A \cap B \cap C)=0$).
     \item Mệnh đề sai.
     Lấy (3) trừ (1) và (2) ta được $-c = -\dfrac{1}{2} \Rightarrow c=\dfrac{1}{2}$; (2) suy ra $b=\dfrac{1}{2}$; (1) suy ra $a=\dfrac{1}{3}$.\\
     Do đó: $P(A|B \cup C) = \dfrac{P(A \cap (B \cup C))}{P(B \cup C)}$
     $= \dfrac{P((A \cap B) \cup (A \cap C))}{P(B \cup C)}$
     $= \dfrac{P(A \cap C)}{P(B \cup C)} = \dfrac{P(A)P(C)}{P(B \cup C)} = \dfrac{\dfrac{1}{3}\cdot\dfrac{1}{2}}{\dfrac{3}{4}} = \dfrac{\dfrac{1}{6}}{\dfrac{3}{4}} = \dfrac{2}{9}$.
 \end{listEX}
 }
\end{ex}
\begin{ex}
 \immini[thm]{ Tháp giải nhiệt tại nhà máy Nhiệt điện Phả Lại (Tỉnh Hải Dương, Việt Nam) có mặt cắt qua trục theo phương thẳng đứng là một hình hyperbol (H). Tháp có chiều cao là 120 mét, bán kính đáy dưới bằng 40 mét. Một nhóm kỹ sư đã thiết lập hệ trục tọa độ $Oxy$ như hình vẽ sao cho mặt cắt dạng hypebol của tháp nhận $Ox, Oy$ làm các trục đối xứng; lấy đơn vị trên mỗi trục là mét. Biết rằng đoạn giao nhau giữa trục $Ox$ với tháp bằng 30 mét và gốc $O$ ở vị trí có độ cao 80 mét so với mặt đất.
 }{\includegraphics[scale=1]{img/HXN-7.16}}
 
 \choiceTF
 {\True Diện tích đáy dưới của tháp bằng $5027~m^2$ (làm tròn đến hàng đơn vị)}
 {Các điểm $(-20;0), (20;0)$ thuộc hyperbol $(H)$}
 {Phương trình $(H)$ là $\dfrac{x^2}{15^2} - \dfrac{y^2}{11520} = 1$}
 {\True Thể tích của tháp giải nhiệt này bằng $214414~m^3$ (làm tròn đến hàng đơn vị)}
 \loigiai{
 \includegraphics[scale=1]{img/HXN-7.16a}
 \begin{listEX}
     \item Mệnh đề đúng.
     Bán kính đáy dưới của tháp là $R=40~m$.
     Diện tích đáy dưới tháp $S=\pi R^2 = 1600\pi \approx 5027~m^2$.
     \item Mệnh đề sai.
     Hypebol $(H)$ cắt $Ox$ tại các điểm $(-15;0), (15;0)$.
     \item Mệnh đề sai.
     Gọi phương trình chính tắc của $(H)$ là $\dfrac{x^2}{a^2} - \dfrac{y^2}{b^2} = 1$ ($a>0, b>0$).\\
     Ta có $2a=30 \Rightarrow a=15$; do đó $(H): \dfrac{x^2}{15^2} - \dfrac{y^2}{b^2} = 1$.
     $(H)$ qua điểm $A(40;-80)$ nên $\dfrac{40^2}{15^2} - \dfrac{(-80)^2}{b^2} = 1 \Rightarrow b^2 = \dfrac{11520}{11}$.\\
     Phương trình $(H)$ là $\dfrac{x^2}{15^2} - \dfrac{y^2}{\dfrac{11520}{11}} = 1$.
     \item Mệnh đề đúng.
     Khoảng cách từ $O$ đến nóc bằng $120-80=40$ mét.\\
     Từ câu c) ta có $x^2 = 15^2 \left(1+\dfrac{11y^2}{11520}\right)$; với $x=f(y)$.\\
     Thể tích tháp là $V = \pi \int\limits_{-80}^{40} (f(y))^2 \mathrm{d}y = \pi \int\limits_{-80}^{40} 15^2 \left(1+\dfrac{11y^2}{11520}\right) \mathrm{d}y \approx 214414~m^3$.
 \end{listEX}
 }
\end{ex}
\Closesolutionfile{ans}
\caukq
\Opensolutionfile{ans}[ans/ans-HXN-\sode-SA]
% 
 \begin{ex}%Câu 17
     Trong một lễ hội mùa hè, ba người bạn An, Bình và Cường tham gia cuộc thi xếp tháp ly. Luật chơi như sau: người chơi lần lượt xếp ly vào các tầng của một kim tự tháp chung. An bắt đầu, xếp 1 chiếc ly. Đến lượt Bình, cậu xếp 2 chiếc ly. Cường xếp tiếp 3 chiếc ly. Trở lại lượt An, cậu xếp 4 chiếc ly, rồi Bình xếp 5 chiếc, Cường xếp 6 chiếc... Cuộc thi diễn ra sôi nổi cho đến khi số ly không còn đủ để xếp theo quy luật tăng dần, người đến lượt ở vòng cuối sẽ dùng hết số ly còn lại để hoàn thành tầng của mình (hoặc bắt đầu tầng mới nếu có thể). Sau khi cuộc thi kết thúc, An tự hào khoe rằng mình đã góp tay xếp được khoảng 317 chiếc ly vào ngọn tháp. Hỏi tổng cộng cả ba người bạn đã sử dụng bao nhiêu chiếc ly để xây ngọn tháp đó?
 \shortans{ 933}
 \loigiai{
     Số ly mà An đã xếp là $1; 4;7;\ldots$ tạo thành một cấp số cộng với $u_1=1; d=3$\\
     Sau $n$ lượt, tổng số ly mà An đã xếp là: 
     $$S_n=\dfrac{(u_1+u_n)n}{2}=\dfrac{(3n-1)n}{2}$$ 
     Xét $S_n=317\Rightarrow\dfrac{n(3n-1)}{2}=317\Rightarrow 3n^2-n-634=0\Rightarrow n\approx 14,7$ (không thỏa mãn).\\
     Do đó An là người xếp ly cuối cùng. Sau 14 lượt thì An xếp được $S_{14}=\dfrac{14(3\cdot 14-1)}{2}=287$ ; lượt cuối An xếp thêm $317-287=30$ (ly).\\
     Số ly mà Bình đã xếp được là tổng cấp số cộng có số hạng đầu bằng 2, công sai bằng 3.\\
      Số ly mà Cường đã xếp được là tổng cấp số cộng có số hạng đầu bằng 3, công sai bằng 3.\\
     Tổng số ly cả 3 bạn xếp được là $$317+\dfrac{14\left(2\cdot 2+13\cdot 3\right)}{2}+\dfrac{14\left(2\cdot 3+13\cdot 3\right)}{2}=933$$ .
 }
 \end{ex}
 
 \begin{ex}%Câu 18
 Một khối gỗ có hình dạng của một lăng trụ đứng $ABC.A'{B}'{C}'$ , trong đó $AC=1\,\,m,\,\,BC=2\,\,m,$ $\widehat{ACB}=120^\circ $ . Người thợ mộc đánh dấu điểm $M$ nằm chính giữa đoạn $B{B}'$ . Tính khoảng cách giữa hai đường $AM$ và $C{C}'$ và làm tròn đến hàng phần trăm theo đơn vị mét.
  \shortans{0,65 }
 \loigiai{
     \begin{center}
         \includegraphics[scale=.8]{img/HXN-7.18}
     \end{center}
 Ta có: $C{C}'\text{//}B{B}'\Rightarrow C{C}'\text{//}\left(AB{B}'{A}'\right)$ nên $d\left(C{C}',\,\,\left(AB{B}'{A}'\right)\right)=d\left(C,\left(AB{B}'{A}'\right)\right)$ .\\
 Trong mặt phẳng (ABC), kẻ $CH\perp AB$ tại H (1).\\
 $ABC.A'{B}'{C}'$ là hình lăng trụ đứng nên $A{A}'\perp\left(ABC\right)\Rightarrow CH\perp A{A}'$ (2).\\
 Từ (1) và (2) suy ra $CH\perp\left(AB{B}'{A}'\right)$ $\Rightarrow d\left(C,\,\,\left(AB{B}'{A}'\right)\right)=CH$ .\\
 Xét tam giác $ABC$ có $A{B^2}=C{A^2}+C{B^2}-2.CA.CB.\cos 120^\circ=7$ $\Rightarrow AB=\sqrt{7}$ m.\\
 Diện tích tam giác ABC: $S_{\Delta ABC}=\dfrac{1}{2}CA.CB.\sin C=\dfrac{1}{2}AB.CH$\\ $\Rightarrow CH=\dfrac{CA.CB.\sin{120^0}}{AB}=\dfrac{2.\dfrac{\sqrt{3}}{2}}{\sqrt{7}}=\dfrac{\sqrt{21}}{7}$ m.
 Vậy $d\left(C{C}',\,\,\left(AB{B}'{A}'\right)\right)=CH=\dfrac{\sqrt{21}}{7}\,\,m$ .\\
 Ta có AM và $C{C}'$ là hai đường thẳng chéo nhau mà $\left\{\begin{aligned}
 & C{C}'\text{//}\left(AB{B}'{A}'\right)\\ 
 & AM\subset\left(AB{B}'{A}'\right)\\ 
 \end{aligned}\right.$ nên $d\left(C{C}',\,\,AM\right)=d\left(C{C}',\,\,\left(AB{B}'{A}'\right)\right)=\dfrac{\sqrt{21}}{7}\approx\,0,65\,m$ .}
 \end{ex}
 
 \begin{ex}%Câu 19
 Nhà máy $A$ chuyên sản xuất một loại sản phẩm cung cấp cho nhà máy $B$ . Hai nhà máy thoả thuận rằng: Hàng tháng nhà máy $A$ cung cấp cho nhà máy $B$ số lượng sản phẩm theo đơn đặt hàng của $B$ (tối đa $100$ tấn sản phẩm). Nếu số lượng đặt hàng là $x$ tấn sản phẩm thì giá bán cho mỗi tấn sản phẩm là $P(x)=45-0,001x^2$ (triệu đồng).\\
 Chi phí để $A$ sản xuất $x$ tấn sản phẩm trong một tháng bao gồm:
 \begin{itemize}
    \item Chi phí cố định: $100$ triệu đồng.
   \item  Cho phí cho mỗi tấn sản phẩm làm ra: $30$ triệu đồng.
 \end{itemize}
 Hỏi nhà máy $A$ cần bán cho nhà máy $B$ bao nhiêu tấn sản phẩm mỗi tháng để lợi nhuận thu được là lớn nhất? (Làm tròn kết quả đến hàng phần chục).
  \shortans{ 70,7}
 \loigiai{
 Số tiền mà nhà máy $A$ thu được từ việc bán $x$ tấn sản phẩm $\left(0\le x\le 100\right)$ cho nhà máy $B$ là: $R(x)=x.P(x)=x\left(45-0,001x^2\right)=45x-0,001x^3$ (triệu đồng).\\
 Chi phí để $A$ sản xuất $x$ tấn sản phẩm trong một tháng là $C(x)=100+30x$ (triệu đồng).\\
 Lợi nhuận (triệu đồng) mà nhà máy$A$ thu được là:\\
 $P(x)=R(x)-C(x)=45x-0,001x^3-\left(100+30x\right)=-0,001x^3+15x-100$\\
 Xét hàm số $P(x)=-0,001x^3+15x-100$ với $\left(0\le x\le 100\right)$ ta có:\\
 $P'(x)=-0,003x^2+15\,;\,\,P'(x)=0\Rightarrow{x^2}=5000\Rightarrow x=50\sqrt{2}$ .\\
 Ta có $P(0)=-100;\,\,P\left(50\sqrt{2}\right)=500\sqrt{2}-100\approx 607;\,\,P\left(100\right)=400$ .\\
 Vậy nhà máy$A$ thu được lợi nhuận lớn nhất khi bán khoảng $50\sqrt{2}\approx 70,7$ tấn sản phẩm cho nhà máy $B$ mỗi tháng.}
 \end{ex}
 
 \begin{ex}%Câu 20
     \immini[thm]{ Một cái chậu đựng nước hình bán cầu có bán kính bằng 2 dm. Người ta đặt một ống nhựa và cho nước vào chậu với lưu lượng nước không đổi bằng $0,3$ lít/phút. Đến phút thứ 6, tốc độ dâng lên của nước trong chậu bằng bao nhiêu dm/phút (làm tròn đến hàng phần trăm)?
         \shortans{0,05 }}{\includegraphics[scale=.8]{img/HXN-7.20}}

 \loigiai{
     \begin{center}
         \includegraphics[scale=1.5]{img/HXN-7.20a}
     \end{center}
 Sau 6 phút bơm nước thì thể tích trong bát bằng $6 \times 0,3 = 1,8$ lít.
 
 Gọi $h$ là chiều cao tức thời của mực nước trong chậu, thể tích nước tương ứng chiều cao $h$ được tính theo công thức thể tích chỏm cầu $V=\dfrac{1}{3}\pi h^2 (3R-h)$; trong đó $R=2~dm$ nên
 $$ \boxed{V=\dfrac{1}{3}\pi h^2 (6-h) = 2\pi h^2 - \dfrac{1}{3}\pi h^3 \quad (1)} $$
 Xét $V=1,8 \Rightarrow \dfrac{1}{3}\pi h^2 (3 \cdot 2 - h) = 1,8 \Leftrightarrow h \approx 0,56~dm$ (lưu vào A).
 
 Đạo hàm hai vế của (1) theo $t$, ta được:
 $$ \dfrac{dV}{dt} = (4\pi h - \pi h^2)\dfrac{dh}{dt} \quad (2) $$
 Thay $\dfrac{dV}{dt} = 0,3~dm^3/\text{phút}$; $h=A \approx 0,56~dm$ vào (2), ta được: $\dfrac{dh}{dt} \approx 0,05~dm/\text{phút}$.
 
 Vậy tốc độ dâng lên của nước là khoảng $\boxed{0,05}~dm/\text{phút}$.}
\end{ex}

\begin{ex}%Câu 21
 Vào ngày lễ Tổng kết năm học 2024-2025, tại một trường Tiểu học nghèo ở miền núi, có 10 em học sinh hiếu học được vinh dự nhận 20 phần quà từ các anh chị cựu học sinh của trường nay đã thành đạt. Các phần quà này là đồng giá, gồm có: 9 đôi giày, 7 cái áo và 4 cái cặp; những món quà cùng loại thì giống hệt nhau.
 Trong số 10 em học sinh được nhận quà thì có Bình và Minh là đôi bạn rất thân thiết, tính xác suất để đôi bạn này cùng nhận các món quà như nhau.
  \shortans{0,4 }
\loigiai{
 Gọi x là số cặp quà (giày, áo); gọi y là số cặp quà (giày, cặp); gọi z là số cặp quà (áo, cặp).
 
 Ta có: $\begin{cases} x+y=9 \\ x+z=7 \\ y+z=4 \end{cases} \Leftrightarrow \begin{cases} x=6 \\ y=3 \\ z=1 \end{cases}$.
 
 Số cách tặng quà cho 10 học sinh, mỗi người hai phần khác nhau là: $n(\Omega) = C_{10}^6 \times C_4^3 \times C_1^1$.
 (Tức là chọn 6 học sinh trong 10 học sinh để trao (giày, áo); chọn 3 trong 4 học sinh tiếp theo để trao (giày, cặp); 1 học sinh cuối cùng buộc phải nhận món quà còn lại).
 
 Có hai trường hợp để trao quà cho 10 học sinh mà Bình và Minh được nhận quà như nhau:
 \begin{itemize}
     \item \textbf{Trường hợp 1:} Bình và Minh nhận quà (giày, áo).
     Số cách trao quà là $1 \times 1 \times C_8^4 \times C_4^3 \times C_1^1 = \boxed{280}$ (cách).\\
     (Tức là có 1 cách để Bình và Minh nhận quà; chọn 4 học sinh trong 8 học sinh còn lại tiếp theo nhận (giày, áo) $\rightarrow C_8^4$ (cách); chọn 3 học sinh trong 4 học sinh còn lại nhận (giày, cặp) $\rightarrow C_4^3$ (cách)).
     \item \textbf{Trường hợp 2:} Bình và Minh nhận quà (giày, cặp).
     Số cách trao quà là $1 \times 1 \times C_8^1 \times C_7^6 \times C_1^1 = \boxed{56}$ (cách).\\
     (Tức là có 1 cách để Bình và Minh nhận quà; chọn 1 học sinh trong 8 học sinh tiếp theo nhận (giày, cặp) còn lại $\rightarrow C_8^1$ (cách); chọn 6 học sinh trong 7 học sinh còn lại nhận (giày, áo) $\rightarrow C_7^6$ (cách)).
 \end{itemize}
 Số cách trao quà mà Bình và Minh được nhận quà như nhau là $n(A) = 280+56 = \boxed{336}$.\\
 Vậy xác suất cần tính là $n(A) = 280+56 = \boxed{336}$. $P(A) = \dfrac{n(A)}{n(\Omega)} = \dfrac{336}{C_{10}^6 \times C_4^3 \times C_1^1} = \boxed{0.4}$.}
 \end{ex}
 
 \begin{ex}%Câu 22
Trong không gian với trục tọa độ $Oxyz$ , cho ba điểm $A\left(-1\,;\,-4\,;\,\,4\right)$ , $B\left(1\,;\,\,7\,;\,\,-2\right)$ ; $C\left(1\,;\,\,4\,;\,\,-2\right)$ . Mặt phẳng $(P)$ : $2x+by+cz+d=0$ đi qua điểm $A$ sao cho B và C cùng phía so với (P). Đặt $h_1=d\left(B\,,\,\,(P)\right)$ và $h_2=2d\left(C\,,\,\,(P)\right)$ . Khi đó $h_1+h_2$ đạt giá trị lớn nhất. Tính $T=b+c+d$ .
 \shortans{65}
\loigiai{
    \begin{center}
       \includegraphics[scale=1]{img/HXN-7.22}
    \end{center}
Gọi $D$ là điểm đối xứng với $A$ qua $C$ và $I$ là trung điểm $BD$.\\
Suy ra $D(3;12;-8)$, $I\left(2;\dfrac{19}{2};-5\right)$.
Khi đó $h_1+h_2 = d(B,(P)) + d(D,(P)) = 2d(I,(P)) \le 2IA$.\\
Do vậy $h_1+h_2$ đạt giá trị lớn nhất khi $(P)$ qua $A$ và vuông góc với $IA$.\\
$\overrightarrow{IA}=\left(-3;-\dfrac{27}{2};9\right)=-\dfrac{3}{2}(2;9;-6) \Rightarrow (P)$ nhận $\vec{n}=(2;9;-6)$ làm vec tơ pháp tuyến.\\
Phương trình mặt phẳng $(P): 2x+9y-6z+62=0$.\\
Vậy $b=9; c=-6; d=62 \Rightarrow b+c+d=65$. }
\end{ex}
\Closesolutionfile{ans}
\inputansbox{6,4,3}{ans/ans-HXN-\sode-T,ans/ans-HXN-\sode-TF,ans/ans-HXN-\sode-SA}
% \def\sode{8}
\begin{name}
	{\tenchude}
	{\tendethi}
	{\tentruong}
	{\thoigian}
\end{name}
\caulc
\Opensolutionfile{ans}[ans/ans-HXN-\sode-T]
\begin{ex}%Câu 1
\immini
{
     Hình vẽ bên là đồ thị của hàm số $y=\dfrac{ax+b}{cx+d}$ . Đường tiệm cận đứng của đồ thị hàm số có phương trình là 
 \choice
 {\True $x=1$}
 {$x=2$}
 {$y=1$}
 {$y=2$}
}
{
    \begin{tikzpicture}[>=stealth, line join=round, line cap=round, font=\footnotesize, scale=1, declare function={a=2; b=-1; c=1; d=-1; hsf(\x)=(a*\x+b)/(c*\x+d);},x=.4cm,y=.3cm,thick]
        \draw[->] (-5,0) -- (5,0)node[below]{$x$};
        \draw[->] (0,-5) -- (0,5)node[left]{$y$};
        \draw (0,0) node[below left]{$O$}
        (1,0)node[below right]{$1$}
        (0,2)node[above left]{$2$};
        \draw[dashed] ({-d/c},-5)--({-d/c},5) (-5,{a/c})--(5,{a/c});
        \clip (-5,-5) rectangle (5,5);
        \pgfmathsetmacro{\can}{-d/c}
        \draw[,samples=150,smooth,domain=-5:{\can-.1}] plot(\x,{hsf(\x)});
        \draw[,samples=150,smooth,domain={\can+.1}:5] plot(\x,{hsf(\x)});
    \end{tikzpicture}
}
\end{ex}
\begin{ex}%Câu 2
 Gọi $S$ là diện tích của hình phẳng giới hạn bởi các đường $y=5^x$, $y=0$, $x=0$, $x=2$. Mệnh đề nào dưới đây đúng?
 \choice
 {\True $S=\int\limits_0^25^x\text{d}x$}
 {$S=\pi\int\limits_0^25^{2x}\text{d}x$}
 {$S=\dfrac{1}{\ln 5}\int\limits_0^25^x\text{d}x$}
 {$S=\ln 5\int\limits_0^25^x\text{d}x$}
\end{ex}
\begin{ex}%Câu 3
 Giá trị lớn nhất của hàm số $f(x)=x^3-3x^2-9x+10$ trên đoạn $\left[-2;2\right]$ bằng
 \choice
 {$-12$}
 {$10$}
 {\True $15$}
 {$-2$}
\end{ex}
\begin{ex}%Câu 4
 Cho hàm số $y=f(x)$ có bảng xét dấu đạo hàm như sau:\\
 \centerline{
 \begin{tikzpicture}
     \tkzTabInit[nocadre=false,lgt=1.4,espcl=3,deltacl=0.5]{$x$/.7,$f'(x)$/.7}
     {$-\infty$ , $-1$ , $0$ , $1$ , $+\infty$}
     \tkzTabLine{, - , $0$ , + , $0$ , - , $0$ , + }
 \end{tikzpicture}
 }
 Hàm số$f(x)$ đồng biến trên khoảng nào sau đây?
 \choice
 {$\left(0;1\right)$}
 {$\left(-1;0\right)$}
 {\True $\left(-\infty;-1\right)$}
 {$\left(-1;+\infty\right)$}
\end{ex}
\begin{ex}%Câu 5
 Tập nghiệm của bất phương trình $3^{2x-1}>27$ là
 \choice
 {$\left(\dfrac{1}{2};+\infty\right)$}
 {$\left(3;+\infty\right)$}
 {\True $\left(2;+\infty\right)$}
 {$\left(\dfrac{1}{3};+\infty\right)$}
\end{ex}
\begin{ex}%Câu 6
 Tìm nguyên hàm $F(x)$ của hàm số $f(x)=\sin x+\cos x$ thoả mãn $F\left(\dfrac{\pi}{2}\right)=2$.
 \choice
 {$F(x)=-\cos x+\sin x+3$}
 {$F(x)=-\cos x+\sin x-1$}
 {\True $F(x)=-\cos x+\sin x+1$}
 {$F(x)=\cos x-\sin x+3$}
\end{ex}
\begin{ex}%Câu 7
 Cho cấp số cộng $\left(u_n\right)$ với năm số hạng đầu là $2$; $7$; $12$; $17$; $22$. Số hạng tổng quát của cấp số cộng là
 \choice
 {$u_n=3n+5$}
 {$u_n=3n-5$}
 {$u_n=5n+3$}
 {\True $u_n=5n-3$}
\end{ex}
\begin{ex}%Câu 8
 Bảng dưới đây thống kê cự li ném tạ trong quá trình luyện tập của một vận động viên trong một tuần (đơn vị: mét).\\
 \centerline{\begin{tabular}{|c|c|c|c|c|c|}
 \hline
 Cự li (m) & $\left[19;19,5\right)$ & $\left[19,5;20\right)$ & $\left[20;20,5\right)$ & $\left[20,5;21\right)$ & $\left[21;21,5\right)$\\
 \hline
 Tần số & 13 & 45 & 24 & 12 & 6\\
 \hline
 \end{tabular}}\\
 Hãy tính phương sai của mẫu số liệu ghép nhóm trên (làm tròn đến hàng phần nghìn).
 \choice
 {\True $0{,}277$}
 {$0{,}526$}
 {$0{,}326$}
 {$0{,}211$}
 \end{ex}
\begin{ex}%Câu 9
 Cho hai mặt phẳng $(P)\colon 2x-y-z-3=0$ và $(Q)\colon x-z-2=0$. Góc giữa hai mặt phẳng $(P)$ và $(Q)$ bằng
 \choice
 {\True $30^{\circ}$}
 {$45^{\circ}$}
 {$60^{\circ}$}
 {$90^{\circ}$}
\end{ex}
\begin{ex}%Câu 10
 Khảo sát thời gian tập thể dục của một số học sinh khối 11 thu được mẫu số liệu ghép nhóm sau:\\
\centerline{
\begin{tblr}{colspec={|c|c|c|c|c|c|}, hlines, vlines}
    Thời gian (phút) & [0;20) & [20;40) & [40;60) & [60;80) & [80;100) \\
    Số học sinh & 5 & 9 & 12 & 10 & 6 \\
\end{tblr}
}
 Mốt của mẫu số liệu ghép nhóm trên là
 \choice
 {\True $52$}
 {$42$}
 {$53$}
 {$54$}
\end{ex}
\begin{ex}%Câu 11
 Một vật chuyển động có phương trình $s(t)=3\cos t$ . Khi đó, vận tốc tức thời tại thời điểm $t$ của vật là:
 \choice
 {\True $v(t)=-3\sin t$}
 {$v(t)=-3\cos t$}
 {$v(t)=3\cos t$}
 {$v(t)=3\sin t$}
\end{ex}
\begin{ex}%Câu 12
 Trong không gian $Oxyz$ , một vectơ pháp tuyến của mặt phẳng $\dfrac{x}{-2}+\dfrac{y}{-1}+\dfrac{z}{3}=1$ là
 \choice
 {\True $\vec{n}=(3;6;-2)$}
 {$\vec{n}=(2;-1;3)$}
 {$\vec{n}=(-3;-6;-2)$}
 {$\overrightarrow{n}=(-2;-1;3)$}
 \end{ex}
\Closesolutionfile{ans}
\cauds
\Opensolutionfile{ans}[ans/ans-HXN-\sode-TF]
\begin{ex}%Câu 13
 Cho hàm số $y=f(x)=\dfrac{x-m^2+m}{x+1}$ , $m$ là tham số.
 \choiceTF
 {Với $m=1$ thì hàm số $y=f(x)$ luôn nghịch biến trên các khoảng $\left(-\infty;-1\right)$ và $\left(-1;+\infty\right)$}
 {\True Với mọi số thực $m$ thì hàm số $y=f(x)$ luôn đồng biến trên các khoảng $\left(-\infty;-1\right)$ và $\left(-1;+\infty\right)$}
 {\True $\underset{\left[1;2\right]}{\max}f(x)=f(2)$}
 {\True Có hai giá trị nguyên $m$ để $\underset{\left[0;1\right]}{\min}f(x)=-2$}
\end{ex}
\begin{ex}%Câu 14
\immini
{
    Một vật chuyển động trong $3$ giờ với vận tốc $v$ (km/h) phụ thuộc vào thời gian $t$ (h), đồ thị của hàm vận tốc được cho như hình vẽ. Trong thời gian $1$ giờ kể từ khi vật bắt đầu chuyển động, đồ thị hàm vận tốc của nó là một phần của parabol có đỉnh $S(2;9)$ , khoảng thời gian còn lại đồ thị là một đoạn thẳng song song với trục hoành.
 \choiceTF
 {\True Tại thời điểm bắt đầu chuyển động, vật có vận tốc bằng $4$ km/h}
 {Trong thời gian $1$ giờ kể từ khi bắt đầu chuyển động, phương trình vận tốc của vật là $v(t)=-\dfrac{5}{4}{t^2}-5t+4$}
 {\True Sau $30$ phút kể từ khi bắt đầu chuyển động, gia tốc của vật bằng $3,75$ km/h$^2$}
 {\True Quãng đường $S$ mà vật đi được được trong $3$ giờ (làm tròn đến hàng phần trăm) là $21{,}58$ km}
}
{
    \begin{tikzpicture}[>=stealth, thick, scale=1, declare function={a=-5/4; b=5; c=4; hsf(\x)=a*(\x)^2+b*\x+c;},y=.5cm]
        \draw[->] (-.5,0) -- (3.5,0)node[below]{$x$};
        \draw[->] (0,-.5) -- (0,10)node[left]{$y$};
        \draw (0,0) node[below left]{$O$};
        \draw[dashed]  (2,0) node[below]{$2$} |-(0,9) node[left]{$9$}
        (1,0)node[below]{$1$}--(1,{hsf(1)})
        (3,0)node[below]{$3$}--(3,{hsf(3)});
        \fill (2,{hsf(2)}) circle (1.5pt);
        \draw[samples=150, smooth,line width=1.5pt,blue , domain=0:1] plot(\x,{hsf(\x)})--(3,{hsf(3)});
        \draw[samples=150, smooth, dashed, domain=3:1] plot(\x,{hsf(\x)});
    \end{tikzpicture}
}
\end{ex}
\begin{ex}%Câu 15
Trong một căn phòng có chiều ngang $4$ m, chiều rộng $8$ m và chiều cao $4$ m, người chủ đã thiết kế $4$ dãy đèn led chạy dọc theo các đường chéo của hình chữ nhật tương ứng với các bức tường căn phòng sao cho chúng có tính liên tục. Thiết lập hệ trục tọa độ Oxyz như hình vẽ với căn phòng là hình hộp chữ nhật $ABCD.A'B'C'D'$ , trong đó điểm $A$ là gốc tọa độ, đơn vị trên mỗi trục là mét.
\begin{center}
    \includegraphics[width=5cm]{img/HXN-8-15a}\qquad \includegraphics[width=5cm]{img/HXN-8-15b}
\end{center}
 Chủ căn phòng quyết định sử dụng loại đèn LED neon Flex (không chói mắt) với giá thị trường khoảng 85 nghìn đồng/mét.\\
 \choiceTF
 {Phương trình cạnh BD là $\heva{& x=0\\& y=4+t\\& z=t}$ ($t$ là tham số)}
 {\True Số tiền để mua đèn led trang trí trong căn phòng là $2482$ (nghìn đồng), làm tròn đến hàng đơn vị của nghìn đồng}
 {Khoảng cách từ $D$ đến mặt phẳng $\left(A'B'C'\right)$ bằng $5{,}2$m (làm tròn đến hàng phần chục)}
 {Biết đèn LED có điểm sáng $M$ chạy từ $B$ đến $D$ với tốc độ $0{,}2$ m/s , đèn LED có điểm sáng $N$ chạy từ $A'$ đến $C'$ với tốc độ $0{,}3$ m/s. Sau $9{,}1$ giây (làm tròn đến hàng phần chục) kể từ khi mở nguồn thì hai điểm sáng $M$, $N$ có khoảng cách ngắn nhất trước khi có ít nhất một điểm sáng về đích}
 \loigiai{
 \begin{itemchoice}
     \itemch 
     \itemch 
     \itemch 
     \itemch 
     Ta có $\vec{BD}=(0;-4;4)\Rightarrow \vec{u}=0{,}2\times \dfrac{1}{BD}\vec{BD}=\left(0;-\dfrac{\sqrt{2}}{10};\dfrac{\sqrt{2}}{10}\right)$ là vectơ chỉ phương của $BD$.\\
     Điểm $M\in BD$ nên tại thời điểm $t$, điểm $M$ ở vị trí $M\left( 0;4-\dfrac{\sqrt{2}}{10}t;\dfrac{\sqrt{2}}{10}t \right)$\\
     Ta có $\vec{A'C'}=(0;4;4)\Rightarrow \vec{v}=0{,}3\times \dfrac{1}{A'C'}\vec{A'C'}=\left(0;\dfrac{3\sqrt{2}}{20};\dfrac{3\sqrt{2}}{20}\right)$ là vectơ chỉ phương của $A'C'$.\\
     Điểm $N\in A'C'$ nên tại thời điểm $t$, điểm $N$ ở vị trí $N\left(8;\dfrac{3\sqrt{2}}{20}t;\dfrac{3\sqrt{2}}{20}t\right)$.\\
     Do đó $\vec{MN}=\left(8;\dfrac{\sqrt{2}}{4}t-4;\dfrac{\sqrt{2}}{20}t\right)$\\
     $\Rightarrow MN^2=8^2+\left(\dfrac{\sqrt{2}}{4}t-4\right)^2+\left(\dfrac{\sqrt{2}}{20}t\right)^2=\dfrac{13}{100}t^2-2\sqrt{2}t+80$.\\
     Dễ thấy $MN=\sqrt{\dfrac{13}{100}t^2-2\sqrt{2}t+80}$ đạt giá trị nhỏ nhất $MN_{\min }\approx 8{,}04$; khi đó $t\approx 10{,}9$  giây.
     \end{itemchoice}
 }
\end{ex}
\begin{ex}%Câu 16
Hộp I đựng $3$ bi xanh và $2$ bi vàng; hộp II có $3$ bi xanh, $1$ bi đen và $1$ bi vàng; hộp III có $1$ bi xanh, $1$ bi đen và $3$ bi vàng. Lấy ngẫu nhiên $1$ viên bi từ hộp I bỏ sang hộp II; đồng thời lấy ngẫu nhiên 1 viên bi từ hộp III bỏ sang hộp II; sau đó lấy ngẫu nhiên $2$ viên bi từ hộp II.\\
\centerline{
\includegraphics[width=6cm]{img/HXN-8-16}
}
 \choiceTF
 {Xác suất để hộp thứ II nhận được 2 bi cùng màu bằng $\dfrac{8}{25}$}
 {\True Xác suất để lấy được 2 bi đen từ hộp II bằng $\dfrac{1}{105}$}
 {\True Xác suất để lấy được 2 bi vàng từ hộp II bằng $\dfrac{31}{525}$}
 {Xác suất để lấy được 2 bi từ hộp II cũng là 2 bi được chuyển sang từ hai hộp I, III bằng $\dfrac{5}{31}$ , biết rằng đó là 2 viên bi vàng}
\end{ex}
\Closesolutionfile{ans}
\caukq
\Opensolutionfile{ans}[ans/ans-HXN-\sode-SA]
\begin{ex}%Câu 17
 \immini
 {
     Một người đưa thư xuất phát từ bưu điện (vị trí A) và phải đi qua các con đường để phát thư trước khi quay trở lại bưu điện. Sơ đồ các con đường cần đi qua và độ dài của chúng (tính theo mét) được biểu diễn ở hình vẽ dưới. Hỏi người đó phải đi như thế nào để đường đi là ngắn nhất?
 \shortans{8300}
 }
 {
     \includegraphics[width=6cm]{img/HXN-8-17}
 }
 \end{ex}
 \begin{ex}%Câu 18
Cho đồ thị hàm số $y=\sqrt{x^2-4x+3}$ có các đường tiệm cận xiên $d_1$ và $d_2$ . Tìm tổng khoảng cách từ gốc tọa độ đến hai đường tiệm cận xiên $d_1,d_2$ (kết quả được làm tròn đến hàng phần trăm).
\shortans{2,83}
\end{ex}
\begin{ex}%Câu 19
\immini
{
    Một chiếc bánh kem mừng sinh nhật có dạng hình chóp cụt đều $ABC.A'{B}'{C}'$ với cạnh đáy lớn bằng $4$dm, cạnh đáy nhỏ bằng $2$dm và chiều cao của nó bằng $1{,}5$dm . Tìm thể tích của chiếc bánh kem đó theo đơn vị $d$ m$^3$ (làm tròn đến hàng phần trăm, bỏ qua những thứ trang trí quanh chiếc bánh).
\shortans{6,06}
}
{
    \includegraphics[width=6cm]{img/HXN-8-19-LG}
}
\loigiai{
\immini
{
    Xét hình chóp cụt đều $ABC.A'B'C'$ như hình vẽ; trong đó chiều cao $h=IO=1{,}5$dm.\\
Diện tích hai đáy hình chóp cụt đều $S_1=S_{\triangle ABC}=\dfrac{4^2\sqrt{3}}{4}=4\sqrt{3}\,dm^2$; $S_2=S_{\triangle A'B'C'}=\dfrac{2^2\sqrt{3}}{4}=\sqrt{3}\,dm^2$.\\
Thể tích khối chóp cụt đều: $V=\dfrac{1}{3}h\left(S_1+\sqrt{S_1S_2}+S_2\right)=\dfrac{1}{3}\times 1{,}5\left(4\sqrt{3}+\sqrt{4\sqrt{3}\times \sqrt{3}}+\sqrt{3}\right)=\dfrac{7\sqrt{3}}{2}\approx 6{,}06dm^3$
}
{
    \includegraphics[width=6cm]{img/HXN-8-19-lg}
}
}
\end{ex}
\begin{ex}%Câu 20
\immini
{
    Cho hai nửa đường tròn như hình vẽ bên, trong đó đường kính của nửa đường tròn lớn gấp đôi đường kính của nửa đường tròn nhỏ. Biết rằng nửa hình tròn đường kính AB có diện tích $8\pi $ và $\widehat{ABC}=60^{\circ}$ . 
Tính thể tích vật thể tròn xoay tạo thành khi quay hình phẳng $\mathscr{H}$ (phần được tô đậm) quanh đường thẳng $AB$? Kết quả được làm tròn đến hàng phần trăm.
\shortans{85,9}
}
{
\begin{tikzpicture}[>=stealth, line join=round, line cap=round, font=\footnotesize, scale=1,declare function={R=4;r=2;goc=60;},thick]
    \path
    (0,0) coordinate (A)
    (0:R) coordinate (I)
    ($(I)+(0:R)$) coordinate (B)
    (0:r) coordinate (K)
    ($(I)+(goc:R)$) coordinate (C)
    ;
    \fill[fill=gray] (C) --(B)--(I) arc(0:goc:r)--(C);
    \draw[->] (-1,0)--(9,0)node[below]{$x$};
    \draw[->] (0,-1)--(0,5)node[left]{$y$};
    \draw[red] (A) arc (180:0:r) ;
    \draw[blue] (A) arc(180:0:R);
    \draw (A)--(C)--(B);
    \foreach \x/\g in {A/-135,B/-90,C/70}\draw[fill=white] (\x) circle (1pt)+(\g:3mm) node{$\x$};
    \draw[fill=white] (K) circle (1pt)node[below]{$2$} (I)circle (1pt)node[below]{$4$};
\end{tikzpicture}
}
\loigiai{
    Diện tích nửa đường tròn đường kính AB là $\dfrac{1}{2}\cdot \pi \left(\dfrac{AB}{2}\right)^2=8\pi \Rightarrow AB=8$.\\
    Xét hệ trục tọa độ $Oxy$ như hình vẽ với $O\equiv A$ và tia AB trùng với tia $Ox$.\\
    \centerline{
     \begin{tikzpicture}[>=stealth, line join=round, line cap=round, font=\footnotesize, scale=1,declare function={R=4;r=2;goc=60;},thick]
         \path
         (0,0) coordinate (A)
         (0:R) coordinate (I)
         ($(I)+(0:R)$) coordinate (B)
         (0:r) coordinate (K)
         ($(I)+(goc:R)$) coordinate (C)
         ($(I)+(90:{4/3*sqrt(3)})$)coordinate (M)
         ;
         \fill[fill=none] (C) --(B)--(I) arc(0:goc:r) coordinate (D)--(C);
         \fill[fill=gray] (M)--(I) arc(0:goc:r)--(M);
         \fill[fill=violet] (M)--(C)--(6,0)--(I)--(M);
         \fill[fill=blue!30] (C)--(6,0)--(B)--(C);
         \draw[->] (-1,0)--(9,0)node[below]{$x$};
         \draw[->] (0,-1)--(0,5)node[left]{$y$};
         \draw[red] (A) arc (180:0:r) ;
         \draw[blue] (A) arc(180:0:R);
         \draw (A)--(C)--(B)
         (D)--(C)node[midway,sloped,above]{$y=\dfrac{\sqrt{3}}{3}x$}
         ;
         \draw[dashed] (3,0)node[below]{$3$}--(D)
         (6,0)node[below]{$6$}--(C)
         (4,0)--(M);
         
         \foreach \x/\g in {A/-135,B/-90,C/70}\draw[fill=white] (\x) circle (1pt)+(\g:3mm) node{$\x$};
         \draw[fill=white] (K) circle (1pt)node[below]{$2$} (I)circle (1pt)node[below]{$4$};
     \end{tikzpicture}
    }
    Đường thẳng $AC$ có phương trình là $d\colon y=\dfrac{\sqrt{3}}{3}x$ (vì $AC$ đi qua $A(0;0)$ và có hệ số góc $k=\tan \widehat{BAC}=\tan 30^{\circ }=\dfrac{\sqrt{3}}{3}$).\\
    Gọi $(C)$ là đường tròn tâm $(2;0)$, bán kính $R=2$; khi đó phương trình $(C)\colon (x-2)^2+y^2=4$.\\
    Hoành độ giao điểm giữa $d$ và đường tròn $(C)$ thỏa mãn phương trình $$(x-2)^2+\left(\dfrac{\sqrt{3}}{3}x\right)^2=4\Leftrightarrow \dfrac{4}{3}x^2-4x=0\Leftrightarrow \hoac{& x=0 \\& x=3.} $$
    Đường thẳng $BC$ qua $B(8;0)$, vuông góc $d\colon y=\dfrac{\sqrt{3}}{3}x$ nên có phương trình $y=-\sqrt{3}(x-8)$.\\
    Hai đường thẳng $AC$, $BC$ cắt nhau tại $C$ thỏa hệ $\heva{& y=\dfrac{\sqrt{3}}{3}x \\& y=-\sqrt{3}(x-8) } \Rightarrow C\left(6;2\sqrt{3}\right)$.\\
    Thể tích khối tròn xoay khi quay hình $\mathscr{H}$ quanh $AB$ là 
    $$V=\pi \int\limits_3^4{\left[\dfrac{1}{3}x^2-\left(4-(x-2)^2\right)\right]\mathrm{\,d}x}+\pi \int\limits_4^6{\dfrac{1}{3}x^2\mathrm{\,d}x}+\pi \displaystyle\int\limits_6^8{3(x-8)^2\mathrm{\,d}x}=\dfrac{82}{3}\pi \approx 85{,}9$$
}
\end{ex}
\begin{ex}%Câu 21
\immini
{
    Trên một banner quảng cáo, người ta gắn $17$ chiếc bóng đèn vào một khung hình vuông cũng như hai đường chéo của hình vuông đó. Biết rằng các bóng đèn trên một cạnh hoặc đường chéo thì chia cạnh hoặc đường chéo đó làm các đoạn bằng nhau (xem hình vẽ). Các bóng đèn sẽ sáng lên theo quy luật sau:
\begin{itemize}
    \item Vào phút thứ nhất sẽ có ngẫu nhiên 1 bóng đèn sáng lên, đến cuối phút thứ nhất nó sẽ tắt.
    \item Vào phút thứ 2 sẽ có ngẫu nhiên 2 bóng đèn sáng lên, đến cuối phút thứ hai chúng sẽ tắt.
    \item Vào phút thứ 3 sẽ có ngẫu nhiên 3 bóng đèn sáng lên, đến cuối phút thứ ba chúng sẽ tắt.
\end{itemize}
}
{
    \includegraphics[width=6cm]{img/HXN-8-21}
}
Quy luật này cứ tiếp diễn cho đến phút thứ $17$ và một chu trình mới sẽ được lặp lại. Tính xác suất để từ phút thứ $3$ cho đến phút thứ $17$, luôn có ít nhất $3$ bóng đèn sáng lên ở $3$ đỉnh của một tam giác (làm tròn đến hàng phần trăm).
\shortans{0,84}
\loigiai
{
    \centerline{\includegraphics[width=6cm]{img/HXN-8-21-LG}}
Gọi $A_i$ là biến cố: \lq\lq Tại phút thứ $i$ thì có ít nhất $3$ bóng đèn sáng lên ở $3$ đỉnh của tam giác\rq\rq; khi đó $i\in \left\{ 3;4;\ldots;17 \right\}$.
\begin{itemize}
    \item Phút thứ 3:\\
    Số phần tử không gian mẫu là $n\left(\Omega _3\right)=\mathrm{C}_{17}^3$.\\
    Số khả năng để $3$ bóng đèn sáng lên là $3$ đỉnh tam giác: $n\left(\mathrm{A}_3\right)=\mathrm{C}_{17}^3-2C_7^3-4-10=596$.\\
    (Ta loại trừ các trường hợp $3$ điểm thẳng hàng gồm: $2$ trường hợp $3$ điểm thuộc các đường chéo, $4$ trường hợp $3$ điểm thuộc các cạnh, $10$ trường hợp $3$ điểm thẳng hàng khi vẽ thêm hình).\\
    Xác suất tương ứng là $P\left(\mathrm{A}_3\right)=\dfrac{n\left(\mathrm{A}_3\right)}{n\left(\Omega _3\right)}=\dfrac{598}{\mathrm{C}_{17}^3}=\dfrac{299}{340}$.
    \item Phút thứ 4: Xác suất tương ứng là $P\left(\mathrm{A}_4\right)=\dfrac{\mathrm{C}_{17}^4-2\mathrm{C}_7^4}{\mathrm{C}_{17}^4}=\dfrac{33}{34}$.
    \item Phút thứ 5: Xác suất tương ứng là $P\left(\mathrm{A}_5\right)=\dfrac{\mathrm{C}_{17}^5-2\mathrm{C}_7^5}{\mathrm{C}_{17}^5}=\dfrac{439}{442}$.
    \item Phút thứ 6: Xác suất tương ứng là $P\left(\mathrm{A}_6\right)=\dfrac{\mathrm{C}_{17}^6-2\mathrm{C}_7^6}{\mathrm{C}_{17}^6}=\dfrac{883}{884}$.
    \item Phút thứ 7: Xác suất tương ứng là $P\left(\mathrm{A}_7\right)=\dfrac{\mathrm{C}_{17}^7-2}{\mathrm{C}_{17}^7}=\dfrac{9\,723}{9\,724}$.
    \item Từ phút thứ 8 trở đi thì chắc chắn luôn có ít nhất $3$ bóng sáng lên ở $3$ đỉnh của tam giác.\\
    Xác suất cần tìm là $P(A)=P\left(\mathrm{A}_1\right)\times P\left(\mathrm{A}_2\right)\times \cdot \cdot \cdot \times P\left(\mathrm{A}_{17}\right)\approx0{,}84$.
\end{itemize}
}
\end{ex}
\begin{ex}%Câu 22
Trong không gian với hệ trục tọa độ $Oxyz$, cho ba điểm $A(3;0;0)$, $B(-3;0;0)$ và $C(0;5;1)$. Gọi $M$ là một điểm nằm trên mặt phẳng tọa độ $(Oxy)$ sao cho $MA+MB=10$, giá trị nhỏ nhất của $MC$ bằng bao nhiêu (làm tròn đến hàng phần trăm).
\shortans{1,41}
\loigiai{
    \centerline{\includegraphics[width=6cm]{img/HXN-8-22-LG}}
Hai điểm $A$, $B$ cùng thuộc mặt phẳng $(Oxy)$ và $MA+MB=10>6=AB$.\\
Do vậy, tập hợp điểm $M$ là một elip thuộc mặt phẳng $(Oxy)$ với hai tiểu điểm là $A$ và $B$.\\
Đặt $MA+MB=2a=10\Rightarrow a=5$, $AB=2c=6\Rightarrow c=3$, $b=\sqrt{a^2-c^2}=\sqrt{5^2-3^2}=4$.\\
Do vậy $M\in (E)\colon \dfrac{x^2}{a^2}+\dfrac{y^2}{b^2}=1$ hay $M\in (E)\colon \dfrac{x^2}{25}+\dfrac{y^2}{16}=1$.\\
Gọi $D(0;5;0)$ là hình chiếu của $C$ trên mặt phẳng $(Oxy)$.\\
Khi đó ta có: $CD=\sqrt{0^2+0^2+1^2}=1$ và 
$MC=\sqrt{CD^2+DM^2}=\sqrt{1+DM^2}\,\,\left(*\right)$.\\
Do vậy $MC$ bé nhất khi và chỉ khi $DM$ bé nhất.
Theo hình vẽ, ta thấy khi $M$ trùng với đỉnh elip $(E)$ thuộc tia $Oy$ thì $DM$ bé nhất, hay $M(0;4;0)$.
Suy ra $DM=1$, khi đó $MC=\sqrt{1+1}=\sqrt{2}\approx 1{,}41$.
}
\end{ex}
\Closesolutionfile{ans}
\inputansbox{6,4,3}{ans/ans-HXN-\sode-T,ans/ans-HXN-\sode-TF,ans/ans-HXN-\sode-SA}
% %%%%%%%%%%%%%%%%%%%- HXN
\def\sode{9}
\begin{name}
	{\tenchude}
	{\tendethi}
	{\tentruong}
	{\thoigian}
\end{name}

\caulc
\Opensolutionfile{ans}[ans/ans-HXN-\sode-T]
\begin{ex}%Câu 1
    Cho hàm số $y=f(x)$ có bảng biến thiên như sau:\\
    \centerline{\begin{tikzpicture}[>=stealth]
            \tkzTabInit[nocadre=false,lgt=1,espcl=2.5,deltacl=0.5]{$x$/.7 ,$y'$/.7,$y$/2}
            {$-\infty$ , $-2$ , $3$ , $+\infty$}
            \tkzTabLine{ , + , $0$ , - , $0$ , + , }
            \tkzTabVar{-/$-\infty$ , +/$4$ , -/$-3$ , +/$+\infty$}
    \end{tikzpicture}}
    Tổng giá trị cực đại và cực tiểu của hàm số $y=f(x)$ bằng
    \choice
    {\True $1$}
    {$-3$}
    {$4$}
    {$2$}
\end{ex}

\begin{ex}%Câu 2
    Trong không gian $Oxyz$ , cho mặt cầu $(S)$ có tâm $I(-1;2;1)$ và đi qua điểm $M(3;-1;4)$ . Phương trình của mặt cầu $(S)$ là:
    \choice
    {\True $(x+1)^2+(y-2)^2+(z-1)^2=34$}
    {$(x-1)^2+(y-2)^2+(z+1)^2=16$}
    {$(x+1)^2+(y-2)^2+(z-1)^2=16$}
    {$(x-1)^2+(y+2)^2+(z+1)^2=34$}
\end{ex}

\begin{ex}%Câu 3
    Nếu $\mathop{\int}_0^1f(x)\text{d}x=2$ và $\mathop{\int}_1^0g(x)\text{d}x=-3$ thì $\mathop{\int}_0^1[3f(x)-4g(x)]\text{d}x$ là:
    \choice
    {$6$}
    {\True $-6$}
    {$18$}
    {$12$}
\end{ex}

\begin{ex}%Câu 4
    Cho hình lăng trụ đứng $ABC.A' B' C'$ có đáy là tam giác vuông cân tại $B$ và $AB=6$ . Tính khoảng cách từ điểm $C$ đến mặt phẳng $(ABB'A')$
    \choice
    {$3$}
    {$6\sqrt{2}$}
    {$3\sqrt{2}$}
    {\True $6$}
\end{ex}

\begin{ex}%Câu 5
Đường tiệm cận xiên của đồ thị hàm số $ f(x)=\dfrac{2x^2-3x+1}{x+1}$ có phương trình là
\choice
{\True $y=2x-5$}
{$y=2x+5$}
{$y=2x-3$}
{$y=2x+3$}
\end{ex}

\begin{ex}%Câu 6
Trong không gian $ Oxyz$, cho vectơ $\vec{a}=(-1;0;2)$. Mệnh đề nào dưới đây đúng?
\choice
{$2\vec{a}=(2;0;-4)$}
{$2\vec{a}=(-2;0;-4)$}
{\True $2\vec{a}=(-2;0;4)$}
{$2\vec{a}=(2;0;4)$}
\end{ex}

\begin{ex}%Câu 7
Cho hàm số $ y=f(x)$ có đạo hàm $f'(x)=(x-2)(x+1)$, $\forall x\in\mathbb{R}$. Mệnh đề nào dưới đây đúng?
\choice
{Hàm số đồng biến trên $(-1;+\infty)$}
{Hàm số đồng biến trên $(-\infty;2)$}
{Hàm số nghịch biến trên $(-1;2)$}
{\True Hàm số nghịch biến trên $(-1;2)$}
\end{ex}

\begin{ex}%Câu 8
Trong không gian $ Oxyz,$ cho đường thẳng $ d\colon \dfrac{x-1}{2}=\dfrac{y+1}{1}=\dfrac{z}{-3}$. Mặt phẳng $(P)$ đi qua điểm $ A(1;0;1)$ và vuông góc với đường thẳng $d$ có phương trình là:
\choice
{\True $2x+y-3z+1=0$}
{$2x+y-3z-1=0$}
{$x+z+1=0$}
{$x+z-1=0$}
\end{ex}

\begin{ex}%Câu 9
Một người thống kê lại thời gian thực hiện các cuộc gọi điện thoại của người đó trong một tuần ở bảng sau: (đơn vị: giây)\\
\centerline{\begin{tblr}{|c|c|c|c|c|c|c|}
        \hline
        {Thời gian} & $[0;60)$ & $[60;120)$ & $[120;180)$ & $[180;240)$ & $[240;300)$ & $[300;360)$\\
        \hline
        Số cuộc gọi & $ 9$ & $ 9$ & $ 5$ & $ 7$ & $ 2$ & $ 1$\\
        \hline
\end{tblr}}\\
Khoảng tứ phân vị của mẫu số liệu ghép nhóm trên bằng
\choice
{$180$}
{\True $40$}
{$60$}
{$169$}
\end{ex}

\begin{ex}%Câu 10
Tập nghiệm $S$ của bất phương trình $\log_{\tfrac{1}{2}}\left(x-3\right)\ge\log_{\tfrac{1}{2}}4$ là
\choice
{\True $S=\left(3;7\right]$}
{$S=\left[3;7\right]$}
{$S=\left(-\infty;7\right]$}
{$S=\left[7;+\infty\right)$}
\end{ex}

\begin{ex}%Câu 11
Một công ty thống kê lương của nhân viên theo tuần (đơn vị: USD) theo bảng sau:\\
\centerline{
\begin{tblr}{
        colspec = {|c|c|c|c|c|c|},
    }
    \hline
    Lương theo tuần (USD) & [10; 20) & [20; 30) & [30; 40) & [40; 50) & [50; 60] \\\hline
    Số công nhân & 4 & 6 & 10 & 20 & 10 \\\hline
\end{tblr}
}
Độ lệch chuẩn của mẫu số liệu này bằng bao nhiêu (làm tròn tới hàng phần chục)?
\choice
{\True $11{,}7$}
{$12$}
{$11{,}4$}
{$12{,}5$}
\end{ex}

\begin{ex}%Câu 12
Cho hàm số $f(x)=x^2-\dfrac{4}{x}$. Giá trị của $\int\limits_1^2f'(x)\mathrm{d}x$ bằng
\choice
{$\dfrac{7}{3}-\text{ln}2$}
{$3$}
{$\dfrac{7}{3}$}
{\True $5$}
\end{ex}

\Closesolutionfile{ans}
\cauds
\Opensolutionfile{ans}[ans/ans-HXN-\sode-TF]

\begin{ex}%Câu 13
\immini
{
    Cho hàm số $y=f(x)$ xác định trên $\mathbb{R}\setminus\left[-1;1\right]$ liên tục trên mỗi khoảng xác định và có bảng biến thiên như sau
}
{
    \begin{tikzpicture}[>=stealth]
        \tkzTabInit[nocadre=false,lgt=1,espcl=2.5,deltacl=0.5]{$x$/.7 ,$y'$/.7,$y$/2}
        {$-\infty$ , $-1$,$1$ , $+\infty$}
        \tkzTabLine{ , +,t,h,t , - , }
        \tkzTabVar{-/$2$,+DH/$+\infty$, D+/$0$,-/$-2$}
    \end{tikzpicture}
}
\choiceTF
{Đồ thị hàm số $y=f(x)$ có đường tiệm cận đứng $x=1$}
{\True Đồ thị hàm số $y=f(x)$ có đúng hai đường tiệm cận ngang}
{\True Đồ thị hàm số $y=f(x)$ không có đường tiệm cận xiên}
{Đồ thị hàm số $y=\dfrac{1}{f(x)+1}$ có tất cả bốn đường tiệm cận}
\loigiai{
\begin{itemchoice}
    \itemch 
    \itemch 
    \itemch 
    \itemch Đặt $g(x)=\dfrac{1}{f(x)+1}$. \\
    Ta có $\lim\limits_{x\to -\infty }g(x)=\dfrac{1}{2+1}=\dfrac{1}{3}$ (vì $f(x)\to 2$); $\lim\limits_{x\to +\infty }g(x)=\dfrac{1}{-2+1}=-1$ (vì $f(x)\to -2$).\\
    Vì vậy đồ thị hàm số $y=g(x)$ có hai tiệm cận ngang $y=\dfrac{1}{3}$; $y=-1$.\\
    Xét $f(x)+1=0\Leftrightarrow f(x)=-1$. Phương trình này có một nghiệm thuộc khoảng $\left(1;+\infty \right)$.\\
    Do đó đồ thị hàm số $y=g(x)$ có một tiệm cận đứng.\\
    Vậy đồ thị hàm số $y=g(x)$ có tất cả ba đường tiệm cận.
\end{itemchoice}
}
\end{ex}

\begin{ex}%Câu 14
Có hai tên cướp vừa lấy được một chiếc ca nô ở vị trí $A$ thuộc bờ sông, chúng liền cho ca nô chạy theo phương hợp với bờ sông một góc $60^\circ$ với vận tốc $v=2t$ (mét/giây), trong đó $t$ (giây) là thời gian kể từ khi xuất phát. Sau $21$ giây, ca nô đến vị trí $B$ và chúng quyết định chuyển hướng cho ca nô chuyển động thẳng đều theo phương song song với bờ sông, tầm nửa phút sau thì ca nô đến $C$ (tham khảo hình vẽ).\\
\centerline{
    \includegraphics[width=8cm]{img/HXN-9-14}
}
    \choiceTF
    {\True Vị trí $B$ mà ca nô bọn cướp chuyển hướng cách bờ sông khoảng $382$ m (làm tròn đến hàng đơn vị)}
    {Khoảng cách $A$, $C$ tính theo đường chim bay bằng $1522$ m (làm tròn đến hàng đơn vị)}
    {\True Nếu các chiến sĩ công an khởi động ca nô và đi thẳng từ $D$ đến $C$ với vận tốc được tăng thêm $3$ m sau mỗi giây thì sau $21$ giây sẽ đến vị trí $D$ (làm tròn đến hàng đơn vị của giây)}
    {Trên thực tế các chiến sĩ đã chọn giải pháp cho ca nô khởi động và di chuyển vuông góc với bờ với gia tốc $a$ dương, cùng lúc đó bọn cướp từ vị trí $D$ tiến thẳng về phía trước (giữ nguyên hướng đi và tốc độ), hai bên giáp mặt nhau khi $a=4{,}78$ m/s$^2$ (làm tròn đến hàng phần trăm)}
    \loigiai{
        \begin{center}
            \begin{tikzpicture}[declare function={d=6;b=2;c=3;},thick]
                \path
                (0,0) coordinate (A)
                (-60:b) coordinate (B)
                ($(B)+(0:c)$) coordinate (C)
                (0:d) coordinate (D)
                ($(A)!(B)!(D)$) coordinate (H)
                ($(A)!(C)!(D)$) coordinate (K)
                ($(B)!(D)!(C)$) coordinate (E)
                pic[draw,angle radius=5mm,"$60^\circ$",angle eccentricity=1.7]{angle=B--A--H}
                ;
                \foreach \x/\y/\z in {B/H/K,C/K/D,C/E/D}\draw pic[draw,angle radius =2mm] {right angle = \x--\y--\z};
                \draw[dashed] (B)--(H) (C)--(K) (C)--(E)--(D);
                \draw (A)--(B)--(C) (D)--(A);
                \draw[red] (A)--(C)--(D);
                \foreach \x/\g in {A/180,B/-90,C/-90,D/0,E/-90,H/90,K/90}\draw[fill=white] (\x) circle (1pt)+(\g:3mm) node{$\x$};
            \end{tikzpicture}
        \end{center}
    \begin{itemchoice}
        \itemch Sau $21$ giây, ca nô bọn cướp đi được $AB=\int\limits_0^{21}{2t\mathrm{d}t}=441m$.\\
        Gọi $H$ là hình chiếu vuông góc của $B$ trên bờ sông, khi đó $BH=AB\sin 60^\circ=\dfrac{441\sqrt{3}}{2}\approx 382m$.
        \itemch Vận tốc của ca nô bọn cướp tại $B$ là $v_B=2\times 21=42$ m/s.\\
        Khoảng cách hai vị trí $B$, $C$ là $BC=42\times 30=1260m$.\\
        Khoảng cách hai vị trí $A$, $H$ là $AH=AB\cos 60^\circ=220{,}5m$.\\
        Gọi $K$ là hình chiếu vuông góc của $C$ trên bờ sông thì $HK=BC=1260m$.\\
        Do đó $AK=AH+HK=1480{,}5m$; $CK=BH=\dfrac{441\sqrt{3}}{2}m$ và $AC=\sqrt{AK^2+CK^2}\approx 1529 m$.
        \itemch Ta có $DK=2000-\left(220{,}5+1260\right)=519{,}5m$; suy ra $CD=\sqrt{CK^2+DK^2}\approx 644{,}8m$.\\
        Thời gian để các chiến sĩ đi được từ $D$ đến $C$ thỏa $\displaystyle\int\limits_0^t{3t\mathrm{d}t}=\sqrt{415741}\Rightarrow t\approx 21s$.
        \itemch Gọi $E$ là vị trí hai bên giáp mặt nhau (nếu có) thì tam giác $CDE$ vuông tại $E$.\\
        Khi đó $CE=DK=519{,}5m$ và $DE=\dfrac{441\sqrt{3}}{2}m$.\\
        Thời gian để ca nô bọn cướp đi từ $C$ đến $E$ là $\dfrac{CE}{42}=\dfrac{1039}{84}\approx 12{,}37s$.\\
        Gia tốc $a$ thỏa mãn $\displaystyle\int\limits_0^{\tfrac{1039}{84}}{at \mathrm{d}t}=\dfrac{441\sqrt{3}}{2}\Rightarrow 4{,}99m/s^2$.
    \end{itemchoice}
    }
\end{ex}

\begin{ex}%Câu 15
    \immini
    {
        Một chiếc máy bay thương mại Comac C919 đang bay trên bầu trời theo một đường thẳng từ $D$ đến $E$ có hình chiếu trên mặt đất là đoạn $CB$. Tại vị trí $D$ thì máy bay bay cách mặt đất $9000$m, tại vị trí $E$ thì máy bay cách mặt đất $12000$ m. Một ra đa được đặt trên mặt đất tại vị trí $O$ cách $C$ khoảng $20000$m, cách $B$ khoảng $16000$m và $\widehat{BOC}=90^\circ$; phạm vi theo dõi của ra đa là $20$ km.\\
        Xét hệ trục tọa độ $ Oxyz$ (đơn vị trên mỗi trục là $ 1000$ m) với $ O$ là vị trí đặt ra đa, $ B$ thuộc tia $Oy$, $C$ thuộc tia $ Ox$.
    }
    {
        \includegraphics[width=5cm]{img/HXN-9-15}
    }
    \choiceTF
    {Tại $ D$, máy bay cách ra đa $23000$m (làm tròn đến hàng nghìn theo đơn vị mét)}
    {\True Khi máy bay bay đến điểm $ I$ (m là trung điểm của $AB$), máy bay cách mặt đất $10500$ m}
    {\True Trên hành trình bay từ $ D$ đến $E$, máy bay sẽ đi qua điểm có tọa độ$ P(16;3,2;9,6)$}
    {Khoảng cách giữa vị trí đầu tiên và vị trí cuối cùng mà máy bay bay trong phạm vi theo dõi của ra đa là $22000$m (làm tròn đến hàng trăm theo đơn vị mét)}
    \loigiai{
        Tọa độ các điểm là$O(0;0;0)\,,B(0;16;0)\,,C(20;0;0)\,,D(20;0;9)\,,E(0;16;12)$.
        \begin{itemchoice}
            \itemch Ta có $\vec{OD}=(20;0;9) \Rightarrow OD=\sqrt{20^2+9^2}=\sqrt{481}$.\\
            Khi ở $D$, khoảng cách giữa máy bay và ra đa là $1000\times \sqrt{481}\approx 22\,000$ m.
            \itemch Tọa độ trung điểm của $DE$ là $I\left(10;8;\dfrac{21}{2}\right)$; cao độ điểm I là $\dfrac{21}{2}=10{,}5$.\\
            Vậy khi bay đến điểm $I$, máy bay cách mặt đất $10{,}5$ km hay $10\,500$m.
            \itemch Ta có $\vec{DE}=(-20;16;3)$. Phương trình đường thẳng $DE$ là $\heva{& x=20-20t \\& y=16t \\& z=9+3t } $ ($t$ là tham số thực).\\
            Thay tọa độ $P(16;3{,}2;9{,}6)$ vào phương trình $DE$ ta được $\heva{& 16=20-20t \\& 3{,}2=16t \\& 9{,}6=9+3t } \Leftrightarrow t=0{,}2\in (0;1)$.\\
            Do đó trên hành trình bay từ $D$ đến $E$, máy bay sẽ đi qua điểm $P(16;3{,}2;9{,}6)$.
            \itemch Gọi $H(20-20t;16t;9+3t)\in DE$ là hình chiếu vuông góc của $O$ trên $DE$.\\
            \centerline{\includegraphics[width=5cm]{img/HXN-9-15-LG}}
            Hai vectơ $\heva{& \vec{OH}=(20-20t;16t;9+3t) \\& \vec{DE}=(-20;16;3) } $ vuông góc với nhau nên\\
            $\vec{OH}\cdot \vec{DE}=0 \Leftrightarrow -20(20-20t)+16\cdot 16t+3(9+3t)=0\Leftrightarrow t=\dfrac{373}{665}$.\\
            Khi đó $\vec{OH}=\left(\dfrac{1\,168}{133};\dfrac{5\,968}{665};\dfrac{7\,104}{665}\right)$ và $OH=\dfrac{16\sqrt{469490}}{665}$.\\
            Gọi $M$, $N$ lần lượt là điểm đầu tiên và điểm cuối cùng mà máy bay xuất hiện trong phạm vi theo dõi của ra đa, ta có $MN=2\sqrt{R^2-OH^2}=2\sqrt{20^2-\dfrac{180736}{665}}=\dfrac{584\sqrt{665}}{665}\approx 22\,600$m.
        \end{itemchoice}
    }
\end{ex}


\begin{ex}%Câu 16
\immini
{
    Biết rằng $5\%$ cá heo trong một khu bảo tồn mắc một bệnh nhất định. Một xét nghiệm chẩn đoán có thể được áp dụng để xác định cá heo có mắc bệnh hay không:
\begin{itemize}
    \item Nếu cá heo mắc bệnh, xét nghiệm cho kết quả dương tính với xác suất $0{,}96$.
    \item Nếu cá heo không mắc bệnh, xét nghiệm cho kết quả dương tính với xác suất $0{,}02$.
\end{itemize}
Xét nghiệm được áp dụng cho một pé cá heo được chọn ngẫu nhiên.
}
{
    \includegraphics[width=5cm]{img/HXN-9-16}
}
    \choiceTF
    {Xác suất để thu được kết quả dương tính bằng $0{,}07$}
    {Biết rằng kết quả dương tính thu được, xác suất để cá heo này thực sự mắc bệnh bằng $\dfrac{45}{67}$}
    {\True Cá heo khi xét nghiệm đã cho kết quả dương tính lần đầu, xác suất để khi xét nghiệm lần tiếp theo vẫn cho kết quả dương tính bằng $ 0{,}69$ (làm tròn đến hàng phần trăm)}
    {Biết rằng kết quả xét nghiệm lần thứ hai là dương tính, xác suất để cá heo này thực sự mắc bệnh bằng $0{,}97$ (làm tròn đến hàng phần trăm)}
    \loigiai{
        Gọi $A$ là biến cố: \lq\lq Cá heo thực sự mắc một bệnh nhất định\rq\rq.\\
        $B_i$ là biến cố: \lq\lq Xét nghiệm cá heo cho kết quả dương tính lần thứ $i$\rq\rq; với $i\in \mathbb{N}^*$.
    \begin{itemchoice}
        \itemch Xác suất để có kết quả dương tính là $P\left(B_1\right)=P(A)\cdot P\left(B_1\mid A\right)+P\left({\bar{A}}\right)\cdot P\left(B_1\mid \bar{A}\right)$;\\
        $P\left(B_1\right)=0{,}05\times 0{,}96+0{,}95\times 0{,}02=0{,}067$.
        \itemch Ta có $P\left(A|B_1\right)=\dfrac{P(AB)}{P(B)}=\dfrac{0{,}05\times 0{,}96}{0{,}067}=\dfrac{48}{67}$.
        \itemch Ta có $P\left(B_2|B_1\right)=\dfrac{P\left(B_1B_2\right)}{P\left(B_1\right)}=\dfrac{0{,}05\times 0{,}96^2+0{,}95\times 0{,}02^2}{0{,}067}\approx 0{,}69$.
        \itemch Ta có: $P\left(A|B_2\right)=\dfrac{P\left(AB_2\right)}{P\left(B_2\right)}=\dfrac{0{,}05\times 0{,}96^2}{0{,}05\times 0{,}96^2+0{,}95\times 0{,}02^2}=\dfrac{2304}{2323}\approx 0{,}99$.
    \end{itemchoice}
    }
\end{ex}

\Closesolutionfile{ans}
\caukq
\Opensolutionfile{ans}[ans/ans-HXN-\sode-SA]

\begin{ex}%Câu 17
    \immini
    {
        Trong một trò chơi, người chơi muốn tìm đường đi ngắn nhất để đi từ A đến P, biết rằng từ A đến P có những đường đi được cho như hình vẽ; thông số trên các đoạn thẳng chính là khoảng cách giữa hai vị trí tương ứng. Đường đi thoả mãn điều kiện trên nhận giá trị nhỏ nhất là bao nhiêu?
    \shortans{21}
    }
    {
        \includegraphics[width=7cm]{img/HXN-9-17}
    }
    \end{ex}
    
    \begin{ex}%Câu 18
\immini
{
    Một vật chuyển động theo quy luật $ s=s(t)=\dfrac{1}{3}{t^3}-\dfrac{3}{2}{t^2}+10t+2$; với $ t$(giây) là khoảng thời gian tính từ lúc vật bắt đầu chuyển động và $ s$(mét) là quãng đường vật đi được trong thời gian đó). Tính quãng đường mà vật đi được khi vận tốc nó đạt $ 20$m/s (kết quả làm tròn đến hàng phần chục).
\shortans{54,2}
}
{
    \includegraphics[width=5cm]{img/HXN-9-18}
}
\end{ex}

\begin{ex}%Câu 19
\immini
{
    Một cái hộp đựng đồ chơi có dạng hình lập phương $ ABCD.A'B'C'D'$. Gọi $\varphi$ là góc giữa hai mặt phẳng $\left(AB'C'\right)$ và $\left(AC'D'\right)$, tính giá trị $\cot\varphi$ và làm tròn đến hàng phần trăm.
\shortans{0,58}
}
{
    \includegraphics[width=5cm]{img/HXN-9-19}
}
\loigiai{
\immini
{
    \textbf{Nhận xét:} $\left(AB'C'\right)\equiv \left(ADC'B'\right)$, $\left(AC'D'\right)\equiv \left(ABC'D'\right)$.\\
    Ta có: $\heva{& CD'\perp C'D \\& CD'\perp AD\left(doAD\perp \left(CDD'C'\right)\right) } $\\
    $\Rightarrow CD'\perp \left(ADC'B'\right)$\tagEX{1}
    Tương tự: $\heva{& B'C\perp BC' \\& B'C\perp AB\left(doAB\perp \left(BCC'B'\right)\right) } $\\
    $\Rightarrow B'C\perp \left(ABC'D'\right)$\tagEX{2}
    Từ $(1)$ và $(2)$ suy ra $\left(\left(AB'C'\right),\left(AC'D'\right)\right)=\left(CD',CB'\right)$.\\
    Giả sử cạnh hình lập phương bằng $a$.\\
    Ta có $CB'=CD'=B'D'=a\sqrt{2}$ (đường chéo trong hình vuông). Suy ra tam giác $CB'D'$ đều.\\
    Do vậy $\varphi =\left(\left(AB'C'\right),\left(AC'D'\right)\right)=\left(CD',CB'\right)=\widehat{B'CD'}=60^\circ$\\
    Vậy $\cot \varphi \approx 0{,}58$.
}
{
    \includegraphics[width=5cm]{img/HXN-9-19-LG}
}
}
\end{ex}

\begin{ex}%Câu 20
\immini
{
    Trong một lễ hội mùa hè, có một trò chơi mà mỗi lần chơi, người chơi sẽ tung đồng thời bốn đồng xu cân đối một cách ngẫu nhiên. Người chơi chỉ thắng cuộc nếu nhận được ít nhất ba mặt ngửa từ bốn đồng xu đã tung.
Xác suất để trong $5$ lần chơi, người chơi thắng được ít nhất ba lần xấp xỉ bằng $ m\cdot 10^{-2}$ với $ m$ là số tự nhiên có hai chữ số (đã được làm tròn đến hàng đơn vị); hỏi giá trị của $m$ bằng bao nhiêu?
\shortans{18}
}
{
\includegraphics[width=5cm]{img/HXN-9-20}
}
\loigiai{
\textbf{Bước 1:} Tính xác suất thắng trong một lần chơi\\
Trong một lần chơi, người chơi tung $4$ đồng xu cân đối:
\begin{itemize}
    \item Xác suất nhận được đúng 3 mặt ngửa là $\mathrm{C}_4^3\cdot \left(\dfrac{1}{2}\right)^3\cdot \left(\dfrac{1}{2}\right)^1=4\cdot \dfrac{1}{8}\cdot \dfrac{1}{2}=4\cdot \dfrac{1}{16}=\dfrac{1}{4}$.
    \item Xác suất nhận được cả 4 mặt ngửa là $\left(\dfrac{1}{2}\right)^4=\dfrac{1}{16}$.\\
    Vậy xác suất thắng trong một lần chơi là: $\dfrac{1}{4}+\dfrac{1}{16}=\dfrac{5}{16}$;\\
    xác suất thua trong một lần chơi là $1-\dfrac{5}{16}=\dfrac{11}{16}$.
\end{itemize}
\textbf{Bước 2:} Tính xác suất thắng ít nhất $3$ lần trong $5$ lần chơi.\\
Xác suất này bao gồm các trường hợp:
\begin{itemize}
    \item Thắng đúng 3 lần, thua 2 lần: $\mathrm{C}_5^3\cdot \left(\dfrac{5}{16}\right)^3\cdot \left(\dfrac{11}{16}\right)^2=10\cdot \dfrac{125}{4096}\cdot \dfrac{121}{256}=\dfrac{151250}{1048576}$.
    \item Thắng đúng 4 lần, thua 1 lần: $\mathrm{C}_5^4\cdot \left(\dfrac{5}{16}\right)^4\cdot \left(\dfrac{11}{16}\right)^1=5\cdot \dfrac{625}{65536}\cdot \dfrac{11}{16}=\dfrac{34375}{1048576}$.
    \item Thắng cả 5 lần: $\left(\dfrac{5}{16}\right)^5=\dfrac{3125}{1048576}$.\\
    Xác suất cần tính là: $\dfrac{151250}{1048576}+\dfrac{34375}{1048576}+\dfrac{3125}{1048576}\approx 18\cdot 10^{-2}$.
\end{itemize}
}
\end{ex}

\begin{ex}%Câu 21
\immini
{
    Một nghệ nhân muốn thiết kế một chiếc ly hình nón có thể chứa được tối đa $500$ ml nước với bán kính miệng ly $ R<8cm$. Sau đó, anh ta đặt một khối trụ đặc có bán kính đáy $2$ cm vào bên trong ly sao cho trục của hình trụ trùng với trục của hình nón và đáy trên của hình trụ nằm cùng một mặt phẳng với đáy của hình nón. Nghệ nhân muốn đặt các thanh thủy tinh phát sáng có dạng đoạn thẳng vào ly chứa đầy dung dịch màu. Biết rằng $AB$ là thanh thủy tinh có độ dài lớn nhất có thể đặt vào ly (không có điểm nào nhô ra khỏi mặt nước), tìm giá trị lớn nhất đó theo đơn vị cm (làm tròn đến hàng phần mười).
\shortans{9,5}
}
{
\begin{tikzpicture}[scale=1, font=\footnotesize, thick,declare function={
        H=-7;R=5;
        k=.8/2;
        h=k*H;r=k*R;
        c=1/6;
        G=asin(R*c/H);
        g=asin(r*c/h);
        X=R*cos(G);Y=R*sin(G);
        xd=r*cos(g);yd=r*sin(g);
    },line join=round, line cap=round,x=.8cm]
    \path
    (X,Y) coordinate (M)
    (xd,yd) coordinate (m)
    ($(m)+(90:{H-h})$) coordinate (m1)
    (90:H)coordinate (S)
    (m) arc (g:180-g:{r} and {r*c}) coordinate (ma)
    ($(ma)+(90:{H-h})$) coordinate (m1a)
    ;
    \draw[dashed]
    (ma)--(m1a)
    (m)--(m1) arc (g:180-g:{r} and {r*c})
    (m1) arc (360+g:180-g:{r} and {r*c})
    ;
    \draw 
    (m) arc (g:360-g:{r} and {r*c})
    (M) arc (G:180-G:{R} and {R*c})
    (S)--(M) arc(360+G:180-G:{R} and {R*c})--cycle;
    \draw[<->] ([shift={(0:R+.2)}]S)--++(-90:H)node[midway,fill=white]{$5$ cm};
\end{tikzpicture}
}
\loigiai{
    \begin{center}
        \begin{tikzpicture}[scale=1, font=\footnotesize, thick,>=stealth, line join=round, line cap=round, declare function={
                H=-7;R=5;
                k=2/5;
                h=k*H;r=k*R;
                c=1/5;
                G=asin(R*c/H);
                g=asin(r*c/h);
                X=R*cos(G);Y=R*sin(G);
                xd=r*cos(g);yd=r*sin(g);
            },x=.8cm]
            \path
            (0,0) coordinate (O)
            (X,Y) coordinate (M)
            (xd,yd) coordinate (m)
            ($(m)+(90:{H-h})$) coordinate (m1)
            (90:H)coordinate (S)
            (m) arc (g:180-g:{r} and {r*c}) coordinate (ma)
            ($(ma)+(90:{H-h})$) coordinate (m1a)
            ;
            \draw[dashed]
            (ma)--(m1a)
            (m)--(m1) arc (g:360-g:{r} and {r*c})
            ;
            \path
            (M)arc (G:20:{R} and {R*c}) coordinate (B)
            (m) arc (g:-70:{r} and {r*c}) coordinate (C)
            (m1)arc (g:-70:{r} and {r*c}) coordinate (A)
            pic[draw,angle radius=3mm]{right angle=O--C--B};
            \draw 
            (m) arc (g:360-g:{r} and {r*c})
            (M) arc (G:360-G:{R} and {R*c})
            (S)--(M) arc(360+G:180-G:{R} and {R*c})--cycle;
            \draw[<->] ([shift={(180:R+.2)}]S)--++(-90:H)node[midway,fill=white]{$5$ cm};
            \draw[dashed] (C)--(A)--(B) (O)--(90:{H-h})node[midway,fill=white]{$h_t$};
            \draw[red] (O)--(C)node[midway,left,black]{$r$}--(B)--(O)node[midway,above,black]{$R$};
            \foreach \x/\g in {A/-90,B/30,C/-135,O/-170}\draw[fill=white] (\x) circle (1pt)+(\g:3mm) node{$\x$};
        \end{tikzpicture}
    \end{center}
Thể tích khối nón: $V=\dfrac{1}{3}\pi R^2h=500ml=500cm^3 \Rightarrow h=\dfrac{1500}{\pi R^2}$. \tagEX{1}
Gọi $h_t$ là chiều cao hình trụ, $r=2cm$ là bán kính đáy hình trụ.\\
Do hình trụ nằm trong hình nón và có cùng trục, ta có tỉ lệ: $\dfrac{r}{R}=\dfrac{h-h_t}{h}\Rightarrow h_t=h\left(1-\dfrac{r}{R}\right)$. \tagEX{2}
Thay $(1)$ vào $(2)$: $h_t=\dfrac{1500}{\pi R^2}\left(1-\dfrac{r}{R}\right)$. \tagEX{3}
Thanh $AB$ nằm chéo trong không gian giữa hình nón và hình trụ.\\
Gọi $A$ là điểm nằm dưới (thuộc đường tròn đáy dưới hình trụ), $B$ nằm trên mặt nước. Khi đó, độ dài $AB$ được tính bằng công thức: $AB=\sqrt{R^2-4+h_t^2}$. \tagEX{4}
Thay $(3)$ vào $(4)$: $AB=\sqrt{R^2-4+\left(\dfrac{1500}{\pi R^2}\left(1-\dfrac{2}{R}\right)\right)^2}=f(R)$.\\
Khảo sát hàm $f(R)$, ta có $f(R)=AB$ đạt giá trị lớn nhất xấp xỉ $9{,}5cm$; khi đó $R\approx 7{,}7$.
}
\end{ex}

\begin{ex}%Câu 22
Trong không gian $ Oxyz$, cho hai điểm $ A(10;6;-2)$, $B(5;10;-9)$ và mặt phẳng $(\alpha)\colon 2x+2y+z-12=0$. Điểm $ M$ di động trên $\left(\alpha\right)$ sao cho $ MA$, $ MB$ luôn tạo với $(\alpha)$ các góc bằng nhau. Biết rằng $ M$ luôn thuộc một đường tròn $(C)$ cố định có tâm $ H(a;b;c)$. Tính tổng $a^2+b^2+c^2$.
\shortans{248}
\loigiai
{
\immini
{
    Gọi $H$, $K$ lần lượt là hình chiếu vuông góc của $A$, $B$ trên mặt phẳng $\left(\alpha \right)$.\\
Khi đó $AH=d\left(A,\left(\alpha \right)\right)=6$; $BK=d\left(B,(\alpha )\right)=3$.\\
Vì $MA$, $MB$ tạo với $\left(\alpha \right)$ các góc bằng nhau nên $\widehat{AMH}=\widehat{BMK}$ mà $AH=2BK$ suy ra $MA=2MB$.\\
Gọi $M(x;y;z)$, ta có $MA=2MB\Leftrightarrow MA^2=4MB^2$\\
$ \Leftrightarrow x^2+y^2+z^2-\dfrac{20}{3}x-\dfrac{68}{3}y+\dfrac{68}{3}z+228=0$.
}
{
    \includegraphics[width=5cm]{img/HXN-9-22-LG-a}
}
\immini
{
    Như vậy, điểm $M$ nằm trên mặt cầu $(S)$ có tâm $I\left(\dfrac{10}{3};\dfrac{34}{3};-\dfrac{34}{3}\right)$ và bán kính $R=2\sqrt{10}$.\\
Mặt khác ta có $M$ di động trên $\left(\alpha \right)$, vì vậy tập hợp điểm $M$ chính là đường tròn giao tuyến $(C)$ được tạo bởi mặt cầu $(S)$ và mặt phẳng $\left(\alpha \right)$.\\
Gọi $H$ là tâm của đường tròn $(C)$, khi đó $H$ là hình chiếu vuông góc của $I$ trên mặt phẳng $\left(\alpha \right)$.\\
Đường thẳng $d$ qua $I$ và vuông góc  $\left(\alpha \right)$ có phương trình  $d\colon \heva{& x=\dfrac{10}{3}+2t \\& y=\dfrac{34}{3}+2t \\& z=-\dfrac{34}{3}+t }$.
}
{
    \includegraphics[width=5cm]{img/HXN-9-22-LG-b}
}
Thay phương trình  $d$ vào $\left(\alpha \right)\colon 2\left(\dfrac{10}{3}+2t\right)+2\left(\dfrac{34}{3}+2t\right)+\left(-\dfrac{34}{3}+t\right)-12=0\Leftrightarrow t=-\dfrac{2}{3}$.\\
Từ đó suy ra $H(2;10;-12)$ và $a=2$, $b=10$, $c=-12\Rightarrow a^2+b^2+c^2=248$.
}
\end{ex}

\Closesolutionfile{ans}
\inputansbox{6,4,3}{ans/ans-HXN-\sode-T,ans/ans-HXN-\sode-TF,ans/ans-HXN-\sode-SA}

% %%%%%%%%%%%%%%%%%%%- HXN
\def\sode{10}

\begin{name}
	{\tenchude}
	{\tendethi}
	{\tentruong}
	{\thoigian}
\end{name}

\caulc
\Opensolutionfile{ans}[ans/ans-HXN-\sode-T]
\begin{ex}%Câu 1
    Trong không gian $Oxyz$, phương trình đường thẳng đi qua điểm $ M\left(1;-3;5\right)$ và có vectơ chỉ phương $\vec{u}=\left(2;-1;1\right)$ là
    \choice
    {$\dfrac{x-1}{2}=\dfrac{y-3}{-1}=\dfrac{z-5}{1}$}
    {$\dfrac{x-1}{2}=\dfrac{y-3}{-1}=\dfrac{z+5}{1}$}
    {\True $\dfrac{x-1}{2}=\dfrac{y+3}{-1}=\dfrac{z-5}{1}$}
    {$\dfrac{x+1}{2}=\dfrac{y+3}{-1}=\dfrac{z-5}{1}$}
\end{ex}

\begin{ex}%Câu 2
\immini
{
        Cho hàm số $ y=\dfrac{ax+b}{cx+d}$ $\left(c\ne 0,ad-bc\ne 0\right)$ có đồ thị như hình vẽ bên. Tiệm cận ngang của đồ thị hàm số là
    \choice
    {$x=-1$}
    {\True $y=\dfrac{1}{2}$}
    {$y=-1$}
    {$x=\dfrac{1}{2}$}
}
{
    \begin{tikzpicture}[line join=round, line cap=round,>=stealth,thick,x=.8cm]
        \tikzset{every node/.style={scale=0.9}}
        \draw[->] (-3,0)--(3,0) node[below left] {$x$};
        \draw[->] (0,-3)--(0,3) node[below left] {$y$};
        \draw (0,0) node [below left] {$O$}
        (0,-0.25)node[below right]{$-\tfrac{1}{4}$}
        (0,0.5)node[above right]{$\frac{1}{2}$};
        \foreach \x/\nx in {-1/-1,1/1}
        \draw[thin] (\x,1pt)--(\x,-1pt) node [below] {$\nx$};
        \draw[dashed,thin] (-0.99,-3)--(-0.99,3);
        \begin{scope}
            \clip (-3,-3) rectangle (3,3);
            \draw[samples=200,domain=-4:-1.01,smooth,variable=\x] plot (\x,{(-1*(\x)+0.5)/(-2*(\x)-2)});
            \draw[samples=200,domain=-0.99:4,smooth,variable=\x] plot (\x,{(-1*(\x)+0.5)/(-2*(\x)-2)});
            \draw[dashed,thin] (-4,1/2)--(4,1/2);
        \end{scope}
    \end{tikzpicture}
}
\end{ex}

\begin{ex}%Câu 3
    Tập nghiệm của phương trình $\log_3\left(18-x^2\right)=2$ là
    \choice
    {$S=\left\{ 3\right\}$}
    {$S=\left\{-3\right\}$}
    {\True $S=\left\{\pm 3\right\}$}
    {$S=\left\{-4;3\right\}$}
\end{ex}

\begin{ex}%Câu 4
    Trong không gian $Oxyz$, mặt phẳng $(P)$ đi qua điểm $ M\left(1;2;3\right)$ và song song với $(Q)\colon x-2y+3z+1=0$ có phương trình là
    \choice
    {$x-2y+3z+6=0$}
    {$x-2y+3z+16=0$}
    {\True $x-2y+3z-6=0$}
    {$x-2y+3z-16=0$}
\end{ex}

\begin{ex}%Câu 5
    Nếu $\int\limits_1^2f(x)\mathrm{\,d}x=-2$ và $\int\limits_2^3f(x)\mathrm{\,d}x=1$ thì $\int\limits_1^3f(x)\mathrm{\,d}x$ bằng
    \choice
    {$-3$}
    {\True $-1$}
    {$1$}
    {$3$}
\end{ex}

\begin{ex}%Câu 6
    Thống kê điểm kiểm tra giữa kỳ 1 môn Toán của $30$ học sinh lớp 12C1 của một trường THPT được ghi lại ở bảng sau:\\
    \centerline{\begin{tblr}{|c|c|c|c|c|}
            \hline
            Điểm & $\left[2;4\right)$ & $\left[4;6\right)$ & $\left[6;8\right)$ & $\left[8;10\right)$\\
            \hline
            Số học sinh & $ 4$ & $ 8$ & $ 11$ & $ 7$\\
            \hline
    \end{tblr}}\\
    Trung vị của mẫu số liệu gốc thuộc khoảng nào trong các khoảng dưới đây?
    \choice
    {$\left[2;4\right)$}
    {$\left[4;6\right)$}
    {\True $\left[6;8\right)$}
    {$\left[8;10\right)$}
\end{ex}

\begin{ex}%Câu 7
    Cho cấp số cộng $\left(u_n\right)$ với $u_{10}=25$ và công sai $d=3$. Khi đó $u_1$ bằng
    \choice
    {$u_1=2$}
    {$u_1=3$}
    {$u_1=-3$}
    {\True $u_1=-2$}
\end{ex}

\begin{ex}%Câu 8
    Cho hình chóp $S.ABC$ có đáy $ABC$ là tam giác vuông tại $B$ và $SA\perp\left(ABC\right)$. Khẳng định nào sau đây đúng?
    \choice
    {$ AB\perp\left(SBC\right)$}
    {$ AC\perp\left(SBC\right)$}
    {$ BC\perp\left(SAC\right)$}
    {\True $ BC\perp\left(SAB\right)$}
\end{ex}

\begin{ex}%Câu 9
    Tính thể tích vật thể tròn xoay khi quay hình phẳng giới hạn bởi các đường cong $ y=\sqrt{e^x-x}$, $y=0$, $x=1$, $x=2$ xung quanh trục Ox là
    \choice
    {\True $\pi\left(e^2-e-\dfrac{3}{2}\right)$}
    {$e^2-e-\dfrac{5}{2}$}
    {$\pi\left(e^2-e-\dfrac{5}{2}\right)$}
    {$e^2-e-\dfrac{3}{2}$}
\end{ex}

\begin{ex}%Câu 10
    Cho hàm số có bảng biến thiên như sau\\
    \centerline{
    \begin{tikzpicture}[>=stealth]
        \tkzTabInit[nocadre=false,lgt=1.2,espcl=2.5,deltacl=0.5]{$x$/.7 ,$f'(x)$/.7,$f(x)$/2.5}
        {$-\infty$ , $-1$ , $3$ , $+\infty$}
        \tkzTabLine{ ,-,d,-,$0$,+, }
        \tkzTabVar{-/$-\infty$,1+D+/$2$/$+\infty$,1-/$-4$,+/$+\infty$}
    \end{tikzpicture}
    }
    Tổng các giá trị nguyên của $ m$ để đường thẳng $ y=m$ cắt đồ thị hàm số tại ba điểm phân biệt bẳng
    \choice
    {$-3$}
    {\True $-5$}
    {$ 0$}
    {$-1$}
\end{ex}

\begin{ex}%Câu 11
    Bảng số liệu ghép nhóm về chiều cao đo được của 30 học sinh nam lớp 12A2 đầu năm học $ 2024-2025$ của một trường THPT được cho như sau:\\
    \centerline{\begin{tabular}{|c|c|c|c|c|c|}
            \hline
            Chiều cao & $\left[150;\ 155\right)$ & $\left[155;\ 160\right)$ & $\left[160;\ 165\right)$ & $\left[165;\ 170\right)$ & $\left[170;\ 175\right)$\\
            \hline
            Tần số & $ 3$ & $ 7$ & $ 10$ & $ 7$ & $ 3$\\
            \hline
    \end{tabular}}\\
    Tính độ lệch chuẩn của mẫu số liệu ghép nhóm trên.
    \choice
    {\True $\dfrac{\sqrt{285}}{3}$}
    {$\dfrac{\sqrt{287}}{3}$}
    {$ 4\sqrt{2}$}
    {$\sqrt{71}$}
\end{ex}

\begin{ex}%Câu 12
    Đồ thị hàm số $ y=\dfrac{3x^2-x+5}{x-2}$ có hai điểm cực trị $ A,B$ nằm trên đường thẳng $ d$ có phương trình $ y=ax+b.$ Tính $ a+b.$ 
    \choice
    {$ a+b=-1.$}
    {$ a+b=1.$}
    {$ a+b=3.$}
    {\True $ a+b=5$}
        \end{ex}


\Closesolutionfile{ans}
\cauds
\Opensolutionfile{ans}[ans/ans-HXN-\sode-TF]

\begin{ex}%Câu 13
Cho hàm số $ y=\dfrac{-x^2+x+1}{x+1}$.
    \choiceTF
    {Hàm số đồng biến trên khoảng $\left(-2;-1\right)$ và $\left(-1;0\right)$}
    {Đồ thị hàm số có hai điểm cực trị nằm khác phía so với $Ox$}
    {Đồ thị hàm số không cắt trục $ Ox$}
    {Đồ thị hàm số có tiệm cận xiên đi qua điểm $ M\left(1;2\right)$}
\end{ex}

\begin{ex}%Câu 14
\immini
{
    Một cái chậu nước có dạng hình chóp cụt đều với các cạnh đáy lần lượt bằng 6 dm và 3 dm, chiều cao chậu nước bằng 4 dm. Người ta bơm nước vào chậu với tốc độ $0{,}4$ lít/phút.
}
{
    \includegraphics[width=5cm]{img/HXN-10-14}
}
    \choiceTF
    {\True Dung tích của chậu nước bằng $ 21\sqrt{3}d{m^3}$}
    {Nếu người ta giữ nguyên tốc độ bơm nước thì sau $91$ phút (làm tròn đến hàng đơn vị) bể sẽ đầy}
    {Khi mực nước trong chậu có chiều cao h thì thể tích nước trong chậu được tính theo công thức $ V=\dfrac{\sqrt{3}}{12}\left(\dfrac{9}{8}{h^3}+\dfrac{9}{4}{h^2}+27h\right)$ lít}
    {Khi nước được bơm đến phút thứ 8 thì tốc độ dâng lên của nước trong chậu bằng $0{,}05$ dm/phút (làm tròn đến hàng phần trăm)}
    \loigiai{
    \begin{itemchoice}
        \itemch Thể tích chậu nước $V=\dfrac{1}{3}h\left(S_1+\sqrt{S_1S_2}+S_2\right)$; trong đó $h=4\,dm$ và diện tích hai đáy lần lượt là $S_1=\dfrac{3^2\sqrt{3}}{4}=\dfrac{9\sqrt{3}}{4}\,dm^2$; $S_2=\dfrac{6^2\sqrt{3}}{4}=9\sqrt{3}\,dm^2$.\\
        Do đó $V=\dfrac{1}{3}\cdot 4\cdot \left(\dfrac{9\sqrt{3}}{4}+\sqrt{\dfrac{9\sqrt{3}}{4}\cdot 9\sqrt{3}}+9\sqrt{3}\right)=21\sqrt{3}\,dm^3$.
        \itemch Bể đầy nước sau khoảng thời gian $t=\dfrac{21\sqrt{3}}{0{,}4}\approx 91$ phút.
        \itemch Thể tích nước tương ứng chiều cao $h$ là\\ $V=\dfrac{1}{3}h\left(\dfrac{9\sqrt{3}}{4}+\dfrac{x^2\sqrt{3}}{4}+\sqrt{\dfrac{9\sqrt{3}}{4}\cdot \dfrac{x^2\sqrt{3}}{4}}\right)=\dfrac{1}{3}h\left(\dfrac{9\sqrt{3}}{4}+\dfrac{\sqrt{3}x^2}{4}+\dfrac{3\sqrt{3}x}{4}\right)$ \tagEX{ 1}
        \immini
        {
            Gọi $x=MN$ là đường mép nước ứng với một mặt bên chậu, chiều cao mực nước là $h$.\\
        Ta có: $x=ah+b$ (hàm số bậc nhất).
        Vì $x=3$; $h=0$ và $x=6;h=4$ suy ra $\heva{& b=3 \\& 4a+b=6 } \Rightarrow \heva{& a=\dfrac{3}{4} \\& b=3 } $ hay $x=\dfrac{3}{4}h+3$ \tagEX{2}
        }
        {
            \includegraphics{img/HXN-10-14-LG}
        }
        Thay $(2)$ vào $(1)$ ta được: $V=\dfrac{1}{3}h\left(\dfrac{9\sqrt{3}}{4}+\dfrac{\sqrt{3}\left(\dfrac{3}{4}h+3\right)^2}{4}+\dfrac{3\sqrt{3}\left(\dfrac{3}{4}h+3\right)}{4}\right)$.\\
        Thu gọn ta được: $V=\dfrac{\sqrt{3}}{12}\left(\dfrac{9}{16}h^3+\dfrac{27}{4}h^2+27h\right)$ \tagEX{3}
        \itemch Đến phút thứ $8$, mực nước trong chậu là $V(8)=0{,}4\times 8=3{,}2\,dm^3$.\\
        Thay vào $(3)$ ta được $\dfrac{\sqrt{3}}{12}\left(\dfrac{9}{16}h^3+\dfrac{27}{4}h^2+27h\right)=3{,}2\Rightarrow h\approx 0{,}69\,dm$\\
        Lấy đạo hàm hai vế $(3)$ theo $t$, ta được: $\dfrac{dV}{\mathrm{\,d}t}=\dfrac{\sqrt{3}}{12}\left(\dfrac{27}{16}h^2+\dfrac{27}{2}h+27\right)\dfrac{dh}{\mathrm{\,d}t}$ \tagEX{4}
        Thay $\dfrac{dV}{\mathrm{\,d}t}=0{,}4$ dm/phút; $h\approx 0{,}69\,dm$ vào $(4)$ ta được $\dfrac{dh}{\mathrm{\,d}t}\approx 0{,}07$ dm/phút.
    \end{itemchoice}
    
    }
\end{ex}

\begin{ex}%Câu 15
Trong không gian $Oxyz$ cho trước với mặt nước phẳng lặng trùng với mặt phẳng $(Oxy)$, đơn vị trên mỗi trục là mét; có hai con chim bói cá ở các vị trí$ A\left(90;0;25\right)$, $ B\left(80;30;15\right)$ trên các cành cây đang cùng ngắm mục tiêu là một chú cá đang bơi trên mặt hồ. Khi cá nằm im ở vị trí $ C\left(20;10;0\right)$ thì hai con chim quyết định tấn công mục tiêu của mình. Chim bói cá ở vị trí $A$ xuất phát trước con còn lại $1$ giây và bay về phía con cá với vận tốc $12$ m/s; chim bói cá còn lại cũng tấn công mục tiêu với vận tốc $15$ m/s.\\
\centerline{\includegraphics[width=8cm]{img/HXN-10-15}}
    \choiceTF
    {\True Khoảng cách của chim bói cá ở A đến mục tiêu ngắn hơn khoảng cách từ chim bói cá ở B đến mục tiêu}
    {\True Chim bói cá ở vị trí A sẽ đến mục tiêu trước con chim ở vị trí B}
    {Trong thực tế, sau khi bay được $5$ giây, chim bói cá từ vị trí A thấy không tranh được con mồi với đối thủ nên nó chuyển hướng để bay đi và đậu trên một nhành cây khác, vị trí chuyển hướng có tọa độ $\left(34;8;4,5\right)$}
    {Từ khi chuyển hướng, chim bó cá bay với vectơ vận tốc $\vec{u}=\left(3;6;6\right)$ (m/s) và sau 6 giây tiếp theo, nó đã đậu trên một cành cây khác. Khoảng cách từ vị trí mới so với vị trí nó đậu ban đầu bằng $63{,}2$ m (làm tròn đến hàng phần chục của mét)}
    \loigiai{
    \begin{itemchoice}
        \itemch 
        Ta có $\vec{AC}=(-70;10;-25)\Rightarrow AC=75\,m$; $\vec{BC}=(-60;-20;-15)\Rightarrow BC=65\,m\,\left(AC>BC\right)$.
        \itemch Thời gian để con chim từ $A$ đến mục tiêu: $\dfrac{75}{12}=6{,}25$ giây; thời gian để con chim từ B đến mục tiêu kể từ thời điểm con chim từ A xuất phát: $\dfrac{65}{15}+1\approx 5{,}33<6{,}25$.
        Do đó chim bói cá từ B đến mục tiêu trước so với con chim từ A.
        \itemch Gọi $D$ là vị trí chuyển hướng của chim bói cá từ A, ta có\\
        $\vec{AD}=\dfrac{5}{6{,}25}\vec{AC}=\dfrac{5}{6{,}25}(-70;10;-25)=(-56;8;-20)\Rightarrow \heva{& x_D-90=-56 \\& y_D=8 \\& z_D-25=-20 } \Rightarrow D(34;8;5)$.
        \itemch  Sau $6$ giây, con chim bói cá chuyển hướng từ $D$ sẽ tịnh tiến theo vectơ $\vec{v}=6\vec{u}=(18;36;36)$; vị trí mới của nó là $E$ có tọa độ $\heva{& x_E=34+18 \\& y_E=8+36 \\& z_E=5+36 }$ hay $E(52;44;41)$.\\
        Khoảng cách $AE=\sqrt{(52-90)^2+(44-0)^2+(41-25)^2}=6\sqrt{101}\approx 60{,}3\,m$.
    \end{itemchoice}
    }
\end{ex}

\begin{ex}%Câu 16
\immini
{
    Trong vụ án kinh điển Bao Công xử Quách Hòe, khi ấy Phủ Khai Phong chắc chắn rằng Quách Hòe có đến $85\% $ khả năng gây án.
}
{
\includegraphics[width=8cm]{img/HXN-10-16}
}
\begin{itemize}
    \item Nếu Quách Hòe gây án thì người hầu của hắn có $65\%$ phạm tội và quan tri huyện có $45\%$ phạm tội, ngoài ra hai người này (người hầu và tri huyện) cũng có $30\%$ khả năng cùng phạm tội.
    \item Nếu Quách Hòe không gây án thì người hầu của hắn chắc chắn không phạm tội, nhưng tri huyện thì có đến $35\%$ khả năng phạm tội.
\end{itemize}
    Gọi $A$ là biến cố: \lq\lq Quách Hòe gây án\rq\rq; $B$ là biến cố: \lq\lq Người hầu Quách Hòe có phạm tội\rq\rq và $C$ là biến cố: \rq\rq Tri huyện có phạm tội\rq\rq.
    \choiceTF
    {$ P(B|A)=0,65;P\left(C|A\right)=0{,}55$}
    {\True Xác suất để cả người hầu và tri huyện không phạm tội bằng $0{,}2 $nếu biết Quách Hòe đã gây án}
    {Xác suất để quan tri huyện có phạm tội bằng $0{,}44$}
    {\True Nếu quan tri huyện có phạm tội, khả năng để Quách Hòe gây án bằng 0,96 (làm tròn đến hàng phần trăm)}
    \loigiai{
    \immini
    {
        \begin{itemchoice}
        \itemch Ta có $P(B|A)=0{,}6$; $P\left(C|A\right)=0{,}45$.
        \itemch Sử dụng biểu đồ Ven, ta có $P\left(B\cup C|A\right)=0{,}15+0{,}3+0{,}35=0{,}8$.
        Do đó $P\left(\bar{B}\bar{C}|A\right)=1-0{,}8=0{,}2$.
        \itemch Ta có $P(C)=P(A)\cdot P\left(C|A\right)+P\left({\bar{A}}\right)\cdot P\left(C|\bar{A}\right) =0{,}95.0{,}45+0{,}05.0{,}35=0{,}445$.
        \itemch $P\left(A|C\right)=\dfrac{P(AC)}{P(C)}=\dfrac{0{,}95.0{,}45}{0{,}445}\approx 0{,}96$.
    \end{itemchoice}
    }
    {
        \includegraphics[width=8cm]{img/HXN-10-16-LG}
    }
    }
\end{ex}

\Closesolutionfile{ans}
\caukq
\Opensolutionfile{ans}[ans/ans-HXN-\sode-SA]
\begin{ex}%Câu 17
    Một khối gỗ có dạng hình hộp chữ nhật $ ABCD.A'B'C'D'$. Biết rằng $AB=10$cm, $BC=15$cm và góc hai mặt phẳng $\left(BCD'A''\right),\left(ABCD\right)$ bằng $30^\circ$. Tính tổng diện tích tất cả các mặt của khối gỗ đó theo đơn vị $ $ m$^2$ và làm tròn đến hàng đơn vị.\\
    \centerline{
    \includegraphics[width=5cm]{img/HXN-10-17-a}\qquad \includegraphics[width=5cm]{img/HXN-10-17-b}
    }
    \shortans{589}
    \end{ex}
    
    \begin{ex}%Câu 18
\immini
{
    Giả sử chi phí cho việc xuất bản $x$ cuốn tạp chí (gồm: lương cán bộ, công nhân viên, giấy in) được cho bởi công thức $C(x)=0{,}0001x^2-0,2x+10000$, trong đó $C(x)$ được tính theo đơn vị là vạn đồng. Chi phí phát hành cho mỗi cuốn là $4$ nghìn đồng. Gọi $T(x)$ là tổng chi phí (gồm cả chi phí xuất bản và phát hành) cho $ x$ cuốn tạp chí; khi đó tỉ số $ M(x)=\dfrac{T(x)}{x}$ được gọi là chi phí trung bình cho một cuốn tạp chí khi xuất bản $x$ cuốn. Tìm số lượng tạp chí cần xuất bản (đơn vị: nghìn cuốn) sao cho chi phí trung bình là thấp nhất, biết rằng nhu cầu hiện tại xuất bản không quá $30\,000$ cuốn.
\shortans{10}
}
{
\includegraphics[width=5cm]{img/HXN-10-18}
}
\loigiai{
Chi phí phát hành cho mỗi cuốn là $4$ nghìn đồng, tức là $0{,}4$ vạn đồng.\\
Suy ra chi phí phát hành cho $x$ cuốn là $0{,}4x$ (vạn đồng).\\
Tổng chi phí xuất bản và phát hành cho $x$ cuốn tạp chí là:
$T(x)=C(x)+0{,}4x=0{,}0001x^2+0{,}2x+10\,000$; với $x>0$.\\
Đặt $f(x)=\dfrac{T(x)}{x}=0{,}0001x+0{,}2+\dfrac{10\,000}{x}$ với $0<x\le 30\,000$.\\
Ta có $f'(x)=0{,}0001-\dfrac{10\,000}{x^2}=\dfrac{0{,}0001x^2-10\,000}{x^2}$; $f'(x)=0\Leftrightarrow x=10\,000>0$. \\
Dựa vào bảng biến thiên, ta thấy giá trị của $M(x)$ nhỏ nhất khi $x=10\,000$.\\
Vậy số lượng tạp chí cần xuất bản sao cho chi phí trung bình thấp nhất là $10$ nghìn cuốn.
}
\end{ex}

\begin{ex}%Câu 19
\immini
{
    Có một con quạ giỏi toán đang khát nước, nó tìm thấy một ly nước hình trụ có bán kính đáy bằng 4 cm, chiều cao $20$ cm, bên trong ly chỉ chứa ít nước đến nỗi nó không thể thò mỏ vào uống được. Quạ ta liền nhanh trí gắp viên bi gần đó bỏ vào ly để mực nước trong ly dâng lên. Biết rằng viên bi có bán kính 2 cm và ban đầu mực nước trong ly chỉ cao 3 cm; hỏi sau khi quạ bỏ viên bi vào ly thì mực nước trong ly dâng lên thêm được bao nhiêu cm? Làm tròn kết quả đến hàng phần trăm.
\shortans{0{,}65}
}
{
    \includegraphics[width=5cm]{img/HXN-10-19}
}
\loigiai{
Gọi $h$ là mực nước dâng lên thêm sau khi bỏ viên bi vào ly.\\
Thể tích nước ban đầu là $V_1=\pi R^2h_1=\pi \cdot 4^2\cdot 3=48\pi \,cm^3$; với $R=4\,cm$; $h_1=3\,cm$.\\
Thể tích phần khối cầu chìm trong nước (sau khi nước dâng lên) là \\
$V_2=\dfrac{1}{3}\pi (h+3)^2\left[3r-(h+3)\right]=\dfrac{1}{3}\pi (h+3)^2(3-h)\,\,cm^3$; với $r=2\,cm$.\\
Thể tích nước sau khi thả viên bi vào (tính cả phần khối cầu chìm trong nước) là $$V=\pi R^2(h+3)=16\pi (h+3)$$
Ta có $V=V_1+V_2\Leftrightarrow 16\pi (h+3)=48\pi +\dfrac{1}{3}\pi (h+3)^2(3-h)$\\
$\Leftrightarrow 16h+48=48-\dfrac{1}{3}h^3-h^2+3h+9\Leftrightarrow h\approx 0{,}65 $.
}
\end{ex}

\begin{ex}%Câu 20
\immini
{
    Tuấn là môt học sinh giỏi lớp $12$, em rất thích học môn toán. Hôm ấy sau khi đã học xong phần Ứng dụng tích phân, Tuấn quyết định cắt chiếc nón mà người bố hay đội đi làm ruộng để nghiên cứu. Biết rằng hình nón này có bán kính đáy bằng $20$cm, thiết diện qua trục là một tam giác đều. Dù người bố hết sức ngăn cản nhưng Tuấn đã ra tay một cách dứt khoát, cắt hình nón bởi một mặt phẳng đi qua đường kính đáy và vuông góc với đường sinh của hình nón để chia nó ra làm hai phần, phần nhỏ có dạng một hình nêm (H), tính thể tích của khối (H) theo đơn vị centimét khối, làm tròn đến hàng đơn vị.
\shortans{2309}
}
{
    \includegraphics[width=5cm]{img/HXN-10-20}
}
\loigiai{
\immini
{
    Chọn hệ trục tọa độ như hình vẽ.\\
Cắt hình nêm $(H)$ bởi một mặt phẳng vuông góc với trục $Ox$ tại điểm có hoành độ $x$, ta được thiết diện là một tam giác vuông ABC thay đổi như hình vẽ.\\
Thể tích khối $(H)$ được tính theo công thức:\\ $V=\int\limits_{-20}^{20}S(x)\mathrm{\,d}x$ với $S(x)=S_{\triangle ABC}$.
}
{
    \includegraphics[width=5cm]{img/HXN-10-20-LG}
}
Tam giác $ABC$ vuông tại $B$ nên $S_{\triangle ABC}=\dfrac{1}{2}AB\cdot BC$.\\
Tam giác $OAC$ vuông tại $A$ nên $AC=\sqrt{20^2-x^2}$.\\
Ta có $\heva{& BC=AC\cdot \cos 60^\circ=\dfrac{1}{2}\cdot \sqrt{20^2-x^2}  \\&  AB=AC\cdot sin60^\circ=\dfrac{\sqrt{3}}{2}\cdot \sqrt{20^2-x^2} }\Rightarrow S(x)=S_{\triangle ABC}=\dfrac{\sqrt{3}}{8}\cdot \left(r^2-x^2\right)$.\\
Do đó $V=\int\limits _{-20}^{20}S(x)\mathrm{\,d}x=\dfrac{\sqrt{3}}{8} \int\limits_{-20}^{20}\left(20^2-x^2\right)\mathrm{\,d}x=\dfrac{20^3}{2\sqrt{3}}\approx 2309\,cm^3$.
}
\end{ex}

\begin{ex}%Câu 21
\immini
{
    Có 8 bạn cùng ngồi xung quanh một cái bàn tròn, mỗi bạn cầm một đồng xu như nhau. Tất cả 8 bạn cùng tung đồng xu của mình, bạn có đồng xu ngửa thì đứng, bạn có đồng xu sấp thì ngồi.\\
Biết xác suất để không có hai bạn liền kề cùng đứng bằng $\dfrac{m}{n}$ (trong đó $m$, $n$ là các số tự nhiên và phân số $\dfrac{m}{n}$ tối giản); tính $ m+n$.
\shortans{303}
}
{
    \includegraphics[width=5cm]{img/HXN-10-21}
}
\loigiai{
Gọi $A$ là biến cố \lq\lq không có hai người liền kề cùng đứng\rq\rq.\\
Số phần tử của không gian mẫu là $n\left(\Omega \right)=2^8=$.\\
Nếu có nhiều hơn $4$ đồng xu ngửa thì biến cố $A$ không xảy ra. Ta xét các trường hợp sau:\\
\textbf{Trường hợp 1:} Có nhiều nhất 1 đồng xu ngửa; số kết quả là $1+8=$.\\
\textbf{Trường hợp 2:} Có 2 đồng xu ngửa; số kết quả là $C_8^2-8=$.\\
(Loại trừ 8 khả năng 2 đồng xu ngửa đó kề nhau).\\
\textbf{Trường hợp 3:} Có 3 đồng xu ngửa trong $8$ đồng xu; các khả năng để loại trừ là
\begin{itemize}
    \item Cả 3 đồng xu ngửa kề nhau: có 8 kết quả.
    \item Có 2 đồng xu ngửa kề nhau trong 3 đồng xu ngửa: có $8.4=32$ kết quả.\\
    Suy ra, số kết quả của trường hợp này là $C_8^3-8-32=$.
\end{itemize}
\textbf{Trường hợp 4:} Có 4 đồng xu ngửa; có 2 kết quả như thế.\\
(Kết quả của trường hợp này là: S-N-S-N-S-N-S-N và N-S-N-S-N-S-N-S; với kí hiệu N là người nhận được đồng xu mặt ngửa và S là người nhận mặt sấp tương ứng vị trí).\\
Số kết quả thuận lợi là $n(A)=9+20+16+2=10 $.\\
Xác suất để không có hai bạn liền kề cùng đứng là $P(A)=\dfrac{n(A)}{n\left(\Omega \right)}=\dfrac{47}{256}=\dfrac{m}{n}\Rightarrow m+n=303$.
}
\end{ex}

\begin{ex}%Câu 22
\immini
{
    Trong không gian $Oxyz$, đơn vị trên mỗi trục là $2\,000$ km, người ta mô phỏng bề mặt Hỏa tinh dưới dạng mặt cầu $ (S)\colon x^2+y^2+z^2-2x-2y-2z=0$; một robot do thám được gởi đến bởi các nhà khoa học từ trái đất đang ở vị trí $A\left(2;2;0\right)$ . Người ta cần đặt một thiết bị nhận tín hiệu từ robot ở vị trí $B$ thuộc bề mặt sao Hỏa sao cho $B$ có hoành độ dương và tam giác $OAB$ đều. Tìm khoảng cách thực tế từ vị trí $B$ đến vị trí $C\left(0;2;0\right)$ , nơi đáp xuống của tàu vũ trụ (làm tròn đến hàng phần trăm của nghìn km)
\shortans{6{,}93}
}
{
    \includegraphics[width=5cm]{img/HXN-10-22}
}
\loigiai{
Gọi $B\left(x\,;y\,;z\right)$ thuộc (S) với $x>0$ và $H$ trung điểm $OA\Rightarrow H(1;1;0)$.\\
Gọi $(P)$ là mặt phẳng trung trực đoạn $OA$, do đó $(P)$ đi qua trung điểm $H(1;1;0)$ của đoạn $OA$ và nhận $\vec{OA}=(2;2;0)$ làm vectơ pháp tuyến. Suy ra  $(P)\colon 2(x-1)+2(y-1)=0$ $\Leftrightarrow x+y-2=0$.\\
Theo giả thiết: $\heva{& OB=AB \\& OB=OA \\& B\in (S) } \Leftrightarrow \heva{& B\in (P) \\& OB^2=OA^2 \\& B\in (S) }  \Leftrightarrow \heva{& x+y-2=0 \\& x^2+y^2+z^2=8\, \\& x^2+y^2+z^2-2x-2y-2z=0} $\\
$\Leftrightarrow \heva{& x+y=2 \\& x^2+y^2+z^2=8\, \\& 2x+2y+2z=8\, } \Leftrightarrow \heva{& x+y=2 \\& x^2+y^2=4\, \\& z=2\, }  \Leftrightarrow \heva{& x+y=2 \\& (x+y)^2-2xy=4\, \\& z=2\, } \Leftrightarrow \heva{& x+y=2 \\& xy=0\, \\& z=2\, } $.\\
Ta tìm được $\heva{& x=2 \\& y=0\, \\& z=2\, } \Rightarrow B(2;0;2)$, (do $x>0$). Do đó $BC=\sqrt{2^2+2^2+2^2}=2\sqrt{3}$.\\
Khoảng cách thực tế là $2\sqrt{3}\times 2\approx 6{,}93$ (nghìn km).
}
\end{ex}

\Closesolutionfile{ans}
\inputansbox{6,4,3}{ans/ans-HXN-\sode-T,ans/ans-HXN-\sode-TF,ans/ans-HXN-\sode-SA}
% %%%%%%%%%%%%%%%%%%%- HXN
\def\sode{11}
\def\tendethi{ĐỀ PHÁT TRIỂN MINH HOẠ 2025}
\begin{dethi}
 {\tendethi}
\end{dethi}
\caulc
\Opensolutionfile{ans}[ans/ans-HXN-\sode-T]
\begin{ex}%Câu 1
 Cho hàm số $ y=f(x)$ liên tục trên $\mathbb{R}$ và có bảng xét dấu của $f'(x)$ như sau:\\
\centerline{
\begin{tikzpicture}[>=stealth]
    \tkzTabInit[nocadre=true,lgt=1.2,espcl=2.5,deltacl=.5]
    {$x$/.7, $f'(x)$/1}
    {$-\infty$,$-2$,$0$,$2$,$+\infty$}
    \tkzTabLine{ , + , $0$ , - ,$0$,+,$0$,+, }
\end{tikzpicture}
}
 Hàm số $ y=f(x)$ có bao nhiêu điểm cực trị?
 \choice
 {$4$}
 {$3$}
 {\True $2$}
 {$1$}
\end{ex}
\begin{ex}%Câu 2
 Biết $\int\limits_1^3f(x)\mathrm{d}x=3$. Giá trị của $\int\limits_1^32f(x)\mathrm{d}x$ bằng
 \choice
 {$5$}
 {\True $6$}
 {$9$}
 {$\dfrac{3}{2}$}
\end{ex}
\begin{ex}%Câu 3
 Trong không gian $ Oxyz$, cho mặt phẳng $(P)$ đi qua điểm $ M\left(2;2;1\right)$ và có một vectơ pháp tuyến $\vec{n}=\left(5;2;-3\right)$. Phương trình mặt phẳng $(P)$là
 \choice
 {$5x+2y-3z-17=0$}
 {$2x+2y+z-11=0$}
 {\True $5x+2y-3z-11=0$}
 {$2x+2y+z-17=0$}
\end{ex}
\begin{ex}%Câu 4
 Nếu cấp số nhân $\left(u_n\right)$ có số hạng đầu $u_1=3$ và công bội $ q=3$ thì số hạng tổng quát $u_n$ của cấp số nhân đó bằng
 \choice
 {\True $3^{n}$}
 {$3^{n-1}$}
 {$3^{n+1}$}
 {$3+\left(n-1\right)\cdot 3$}
\end{ex}
\begin{ex}%Câu 5
 Tập nghiệm của bất phương trình $2^x\le 4$ là
 \choice
 {\True $\left(-\infty;2\right]$}
 {$\left[0;2\right]$}
 {$\left(-\infty;2\right)$}
 {$\left(0;2\right)$}
\end{ex}
\begin{ex}%Câu 6
 Cho hình chóp $S.ABCD$ có tất cả các cạnh bên và cạnh đáy bằng nhau và $ABCD$ là hình vuông tâm $O$. Khẳng định nào sau đây là khẳng định đúng?
 \choice
 {$SA\perp\left(ABCD\right)$}
 {\True $SO\perp\left(ABCD\right)$}
 {$AB\perp\left(SBC\right)$}
 {$AC\perp\left(SBC\right)$}
\end{ex}
\begin{ex}%Câu 7
 Phát biểu nào sau đây là đúng?
 \choice
 {$\int\dfrac{1}{x}\mathrm{\,d}x=\left| x\right|+\text{C}$}
 {\True $\int\dfrac{1}{x}\mathrm{\,d}x=\text{ln}\left| x\right|+C$}
 {$\int\text{ln}x\mathrm{\,d}x=x+C$}
 {$\int\text{ln}\left| x\right|\mathrm{\,d}x=\text{ln}x+C$}
\end{ex}
\begin{ex}%Câu 8
\immini
{
     Cho hàm số $y=f(x)$ có đồ thị như hình dưới đây.
 Đường tiệm cận xiên của đồ thị hàm số đã cho là đường thẳng 
 \choice
 {$y=x-1$}
 {\True $y=-x-1$}
 {$y=x+1$}
 {$y=-x+1$}
}
{
    \begin{tikzpicture}[line join=round, line cap=round,>=stealth,thick]
        \tikzset{every node/.style={scale=0.9}}
        \draw[->] (-3.1,0)--(2.1,0) node[below left] {$x$};
        \draw[->] (0,-3.1)--(0,3.1) node[below left] {$y$};
        \draw (0,0) node [below left] {$O$};
        \foreach \x/\nx in {-1/-1,1/1}
        \draw[thin] (\x,1pt)--(\x,-1pt) node [below] {$\nx$};
        \foreach \y/\ny in {-2/-2,-1/-1}
        \draw[thin] (1pt,\y)--(-1pt,\y) node [left] {$\ny$};
        \draw[dashed,thin] (-0.99,-3)--(-0.99,3);
        \begin{scope}
            \clip (-3,-3) rectangle (2,3);
            \draw[samples=200,domain=-3:-1.01,smooth,variable=\x] plot (\x,{(-1*((\x)^2)+-2*(\x)+-2)/(1*(\x)+1)});
            \draw[samples=200,domain=-0.99:3,smooth,variable=\x] plot (\x,{(-1*((\x)^2)+-2*(\x)+-2)/(1*(\x)+1)});
            \draw[dashed,thin] (-3.1,2.1)--(3.1,-4.1);
        \end{scope}
    \end{tikzpicture}
}
\end{ex}
\begin{ex}%Câu 9
\immini
{
     Cho hàm số $f(x)$ có đồ thị như hình vẽ bên. Giá trị nhỏ nhất của hàm số $f(x)$ trên đoạn $\left[-2;2\right]$ bằng
 \choice
 {$-2$}
 {$-1$}
 {\True $ 0$}
 {$ 1$}
}
{
    \begin{tikzpicture}[line join=round, line cap=round,>=stealth,thick]
        \tikzset{every node/.style={scale=0.9}}
        \draw[->] (-2.3,0)--(2.5,0) node[below] {$x$};
        \draw[->] (0,-1.5)--(0,3.1) node[below left] {$y$};
        \draw (0,0) node [below left] {$O$};
        \foreach \x/\nx in {-2/-2,-1/-1,1/1,2/2}
        \draw[thin] (\x,1pt)--(\x,-1pt) node [below] {$\nx$};
        \foreach \y/\ny in {2/2}
        \draw[thin] (1pt,\y)--(-1pt,\y) node [above left] {$\ny$};
        \draw[dashed,thin](-1,0)--(-1,2)--(0,2);
        \draw[dashed,thin](2,0)--(2,2)--(0,2);
        \begin{scope}
            \clip (-2.2,-1.5) rectangle (2.2,3);
            \draw[samples=200,domain=-2.2:2.2,smooth,variable=\x] plot (\x,{1/2*((\x)^3)+0*((\x)^2)+-3/2*(\x)+1});
        \end{scope}
    \end{tikzpicture}
}
\end{ex}
\begin{ex}%Câu 10
 Bạn An rất thích nhảy hiện đại. Thời gian tập nhảy mỗi ngày của bạn An được thống kê lại ở bảng sau:\\
 \centerline{
 \begin{tblr}{
         colspec = {Q[l] *{5}{Q[c]}},hlines,vlines
     }
     Thời gian (phút) & [20;25) & [25;30) & [30;35) & [35;40) & [40;45) \\
     Số ngày          & 6       & 6       & 4       & 1       & 1       \\
 \end{tblr}
 }
 Độ lệch chuẩn của mẫu số liệu ghép nhóm có giá trị gần nhất với giá trị nào dưới đây?
 \choice
 {$31{,}25$}
 {$31{,}26$}
 {$5{,}4$}
 {\True $5{,}6$}
\end{ex}
\begin{ex}%Câu 11
 Trong không gian $Oxyz$, điểm nào dưới đây thuộc đường thẳng $ d:\heva{& x=1-t\\& y=5+t\\& z=2+3t}$ ?
 \choice
 {$Q\left(-1;1;3\right)$}
 {$P\left(1;2;5\right)$}
 {$M\left(1;1;3\right)$}
 {\True $N\left(1;5;2\right)$}
\end{ex}
\begin{ex}%Câu 12
\immini
{
     Cho hình lập phương $ABCD.A'B'C'D'$(tham khảo hình bên). Giá trị $\sin$ của góc giữa đường thẳng $ AC'$ và mặt phẳng $\left(ABCD\right)$ bằng
 \choice
 {\True $\dfrac{\sqrt{3}}{3}$}
 {$\dfrac{\sqrt{2}}{2}$}
 {$\dfrac{\sqrt{3}}{2}$}
 {$\dfrac{\sqrt{6}}{3}$}
}
{
    \begin{tikzpicture}[line cap=round,line join=round, >=stealth,scale=1]
        \def \a{-1.5} \def \b{-1}\def \c{4} \def \h{3}
        \path (0,0)coordinate(A) 
        +(\a,\b)coordinate(B)
        +(\c,0)coordinate(D)
        ($(B)+(D)-(A)$)coordinate(C)
        +(0,\h) coordinate(C')
        ($(B)+(C')-(C)$)coordinate(B')
        ($(A)+(C')-(C)$)coordinate(A')
        ($(D)+(C')-(C)$)coordinate(D');
        \draw [dashed] (A)--(B)(D)--(A)--(A');
        \draw (B')--(B)--(C)(B')--(C')--(C)--(D)--(D')--(A')--(B')(C')--(D');
        \foreach \x/\g in {A/135,B/-135,C/-45,D/0,A'/135,B'/180,C'/-20,D'/0}\fill[red] (\x) circle (1pt)+(\g:3mm) node[black]{$\x$};
    \end{tikzpicture}
}
 \end{ex}
\Closesolutionfile{ans}
\cauds
\Opensolutionfile{ans}[ans/ans-HXN-\sode-TF]
\begin{ex}%Câu 13
\immini
{
    Công thức $\log x=11{,}8+1{,}5M$ cho biết mối liên hệ giữa năng lượng $x$ tạo ra (tính theo erg, $1$ erg tương đương $10^{-7}$ jun) với độ lớn $M$ theo thang Richter của một trận động đất
     \choiceTF
    {\True Nếu năng lượng được tạo ra là $ 6{,}3\cdot 10^{14}$ erg thì trận động đất phải có độ lớn bằng 2 độ Richter (làm tròn kết quả đến hàng đơn vị)}
    {Trận động đất có độ lớn $3$ độ Richter tạo ra năng lượng bằng $\left(10^{163}-10^{10}\right)$ erg}
    {\True Trận động đất có độ lớn $5$ độ Richter tạo ra năng lượng gấp $1000$ lần so với trận động đất có độ lớn $3$ độ Richter}
    {\True Người ta ước lượng rằng một trận động đất có độ lớn khoảng từ $4$ đến $6$ độ Richter thì năng lượng do trận động đất đó tạo ra nằm trong khoảng từ $10^{17,8}$erg đến $10^{20{,}8}$ erg}
}
{
    \includegraphics[width=5cm]{img/HXN-11-13}
}
\loigiai{
    \begin{itemchoice}
        \itemch 
        Ta có $x=6{,}3\cdot 10^{14} \Rightarrow M=\dfrac{\log \left(6{,}3\cdot 10^{14}\right)-11{,}8}{1{,}5}\approx 2$ độ Richter.
        \itemch Ta có $M=3 \Rightarrow \log x=11{,}8+1{,}5\cdot 3\Rightarrow x=10^{16{,}3}$ erg. 
        \itemch Gọi $x_3$, $x_5$ lần lượt là năng lượng tạo ra bởi các trận động đất có độ lớn 3 và 5 độ Richter.
        Ta có hệ phương trình $\heva{& \log x_3=11{,}8+1{,}5\cdot 3&&(1) \\& \log x_5=11{,}8+1{,}5\cdot 5&&(2)} $\\
        Lấy $(1)-(2)$ ta được $\log x_3-\log x_5=-3\Rightarrow \log \dfrac{x_3}{x_5}=-3 \Rightarrow \dfrac{x_3}{x_5}=\dfrac{1}{1\,000}\Rightarrow x_5=1\,000x_3$.\\
        Vậy trận động đất có độ lớn $5$ độ Richter tạo ra năng lượng gấp $1\,000$ lần so với trận động đất có độ lớn 3 độ Richter.
        \itemch Gọi $x_4$, $x_6$ lần lượt là năng lượng tạo ra bởi các trận động đất có độ lớn 4 và 6 độ Richter. \\
        Ta có $\heva{& \log x_4=11{,}8+1{,}5\cdot 4\\& \log x_6=11{,}8+1{,}5\cdot 6} \Leftrightarrow \heva{& \log x_4=17{,}8 \\& \log x_6=20{,}8 } \Leftrightarrow \heva{& x_4=10^{17{,}8} \\& x_6=10^{20{,}8}.}$ \\
        Vậy một trận động đất có độ lớn khoảng từ $4$ đến $6$ độ Richter thì năng lượng mà nó tạo ra nằm trong khoảng từ $10^{17{,}8}$erg đến $10^{20{,}8}$erg.
    \end{itemchoice}
}
\end{ex}
\begin{ex}%Câu 14
\immini
{
    Tại một thành phố du lịch vào những ngày tháng $6$, người ta luôn chứng kiến trời nắng hoặc trời mưa (mỗi ngày có thể trời nắng xong đến mưa và ngược lại). Người ta biết được có $\dfrac{2}{3}$ số ngày trong tháng là có nắng, có $\dfrac{5}{6}$ số ngày trong tháng là có mưa. Nếu hôm nào bầu trời tại thành phố chỉ có mưa thì khả năng kẹt xe gấp đôi khả năng không kẹt xe; nếu hôm nào bầu trời chỉ có nắng thì chắc chắn thành phố không xảy ra kẹt xe; những ngày trời vừa có nắng vừa có mưa thì khả năng kẹt xe là $30\%$.
}
{
    \includegraphics[width=5cm]{img/HXN-11-14}
}
 \choiceTF
 {\True Nếu du khách đến thành phố vào ngày chỉ có mưa thì khả năng kẹt xe bằng $\dfrac{2}{3}$}
 {Trong tháng 6, có $10$ ngày mà thành phố vừa có mưa vừa có nắng}
 {Một du khách đang cân nhắc sẽ đến thành phố vào một ngày tháng 6, xác suất du khách gặp cảnh kẹt xe bằng $ 0{,}32$ (làm tròn đến hàng phần trăm)}
 {Sau khi du khách đến nơi thì hôm ấy xảy ra kẹt xe thật, xác suất để thành phố vừa có nắng, vừa có mưa bằng $0{,}3$ (làm tròn đến hàng phần chục)}
 \loigiai{
     \begin{itemchoice}
         \itemch Vào ngày chỉ có mưa, khả năng kẹt xe gấp đôi khả năng không kẹt xe nên khả năng kẹt xe tại thành phố là $\dfrac{2}{3}$ và khả năng không kẹt xe là $\dfrac{1}{3}$.
         \itemch Trong tháng sẽ có $\dfrac{2}{3}+\dfrac{5}{6}-1=\dfrac{1}{2}$ số ngày vừa có nắng vừa có mưa, có $\dfrac{2}{3}-\dfrac{1}{2}=\dfrac{1}{6}$ số ngày chỉ có nắng và có $\dfrac{5}{6}-\dfrac{1}{2}=\dfrac{1}{3}$ số ngày chỉ có mưa.\\
         Số ngày trong tháng vừa có mưa vừa có nắng tại thành phố là $\left(\dfrac{2}{3}+\dfrac{5}{6}-1\right)\cdot 30=15$ (ngày).
         \itemch Gọi A là biến cố: \lq\lq Ngày có nắng tại thành phố\rq\rq, B là biến cố: \lq\lq Ngày có mưa tại thành phố\rq\rq và C là biến cố \lq\lq Ngày có sự cố kẹt xe xảy ra tại thành phố\rq\rq. Ta tham khảo sơ đồ bên cạnh.\\
         \centerline{
             \includegraphics[width=6cm]{img/HXN-11-14-LG}
         }
         Xác suất kẹt xe xảy ra là $P(C)=\dfrac{1}{6}\cdot 0+\dfrac{1}{2}\cdot \dfrac{3}{10}+\dfrac{1}{3}\cdot \dfrac{2}{3}=\dfrac{67}{180}\approx 0{,}37$.
         \itemch Ta có: $P\left(AB|C\right)=\dfrac{P(ABC)}{P(C)}=\dfrac{\dfrac{1}{2}\cdot \dfrac{3}{10}}{\dfrac{67}{180}}=\dfrac{27}{67}\approx 0{,}4$.
     \end{itemchoice}
 }
\end{ex}
\begin{ex}%Câu 15
Một khối gỗ có dạng khối nón có bán kính đáy $r=30$ cm, chiều cao $h=120$ cm. Người thợ mộc tìm cách chế tác khối gỗ đó thành một khúc gỗ có dạng khối trụ như hình vẽ.\\
\centerline{
\includegraphics[width=.5\textwidth]{img/HXN-11-15}
}
 \choiceTF
 {\True Thể tích khối gỗ ban đầu bằng $36\,000\pi {cm^3}$}
 {\True Nếu người thợ mộc muốn tạo ra khối gỗ hình trụ có chiều cao bằng nửa khối gỗ ban đầu thì khối gỗ mới tạo ra có thể tích $ 13500\pi$}
 {\True Nếu người thợ mộc muốn tạo ra khối gỗ hình trụ có bán kính đáy bằng $\dfrac{2}{3}$ bán kính khối gỗ hình nón thì phần gỗ phải bỏ đi có thể tích bằng $ 20000\pi {cm^3}$}
 {\True Thể tích lớn nhất của khối gỗ hình trụ bằng $ 0,016\pi \,m^3$}
 \loigiai{
     \begin{itemchoice}
         \itemch Thể tích khối gỗ hình nón là $V=\dfrac{1}{3}\pi r^2h=\dfrac{1}{3}\pi \cdot 30^2\cdot 120=36\,000\pi \,cm^3$.
         \itemch Xét một nửa thiết diện qua trục của hình nón là tam giác SAB (tham khảo hình vẽ).\\
         \centerline{
             \includegraphics[width=5cm]{img/HXN-11-15-LG}
         }
         Các tam giác $SMN$ và $SAB$ đồng dạng (có $\widehat{S}$ chung, $\widehat{SMN}=\widehat{SAB}=90^0$),\\
         ta có $\dfrac{MN}{AB}=\dfrac{SM}{SA}\Leftrightarrow \dfrac{MN}{30}=\dfrac{1}{2}\Rightarrow MN=15\,cm$.\\
         Khối gỗ hình trụ mới được tạo thành có bán kính đáy $15\,cm$, chiều cao $60\,cm$ nên có thể tích $\pi \cdot 15^2\cdot 60=13\,500\pi \,cm^3$.
         \itemch Bán kính đáy khối gỗ hình trụ là $\dfrac{2}{3}\cdot 30=20\,cm$.\\
         Các tam giác $SMN$ và $SAB$ đồng dạng (có $\widehat{S}$ chung, $\widehat{SMN}=\widehat{SAB}=90^0$),\\
         ta có $\dfrac{MN}{AB}=\dfrac{SM}{SA}\Leftrightarrow \dfrac{2}{3}=\dfrac{SM}{120}\Rightarrow SM=80\,cm\Rightarrow AM=40\,cm$.\\
         Thể tích khối gỗ phải bỏ đi là $36\,000\pi -\pi \cdot 20^2\cdot 40=20\,000\pi \,cm^3$.
         \itemch Đặt $AM=x\,\,(cm)$ $(0<x<120)$. \\
         Ta có $\dfrac{MN}{AB}=\dfrac{SM}{SA}\Leftrightarrow \dfrac{MN}{30}=\dfrac{120-x}{120}\Rightarrow MN=30-\dfrac{x}{4}$.\\
         Vậy khối trụ mới được tạo thành có bán kính $MN=30-\dfrac{x}{4}$, đường cao $AM=x$ nên có thể tích: 
         $$V=\pi \left(30-\dfrac{x}{4}\right)^2x=\pi \cdot \left(30-\dfrac{x}{4}\right)\cdot \left(30-\dfrac{x}{4}\right)\cdot \dfrac{x}{2}\cdot 2$$
         Xét hàm số $V(x)=\pi \left(30-\dfrac{x}{4}\right)^2x=\pi \left(900x-15x^2+\dfrac{1}{16}x^3\right)$  với $0<x<120$.\\
         Ta có $V'(x)=\pi \left(900-30x+\dfrac{3x^2}{16}\right)=0 \Leftrightarrow \hoac{& x=120 &&\text{(loại)} \\& x=40 &&\text{(nhận)}}$.\\
         Bảng biến thiên:\\
         \begin{tikzpicture}[>=stealth]
             \tkzTabInit[nocadre=false,lgt=1.2,espcl=2.5,deltacl=0.5]{$x$/.7 ,$V'(x)$/.7,$V(x)$/2}
             {$0$ , $40$ , $120$}
             \tkzTabLine{ , + , $0$ , - , }
             \tkzTabVar{-/$1$ , +/$16\,000\pi$ , -/$5$}
         \end{tikzpicture}\\
         Dễ thấy $V_{\max}=16\,000\pi \,cm^3=0{,}016\pi \,m^3$.
     \end{itemchoice}
 }
\end{ex}
\begin{ex}%Câu 16
\immini
{
    Trong không gian với hệ tọa độ $Oxyz$, đơn vị trên mỗi trục tọa độ là $km$, NASA đang thiết kế một trạm vũ trụ hình cầu có tâm đặt tại điểm $ O(0;0;0)$ và bán kính $R=10$ km. Một con tàu vũ trụ di chuyển với tốc độ $ 27000$ km/h theo quỹ đạo là một đường thẳng qua điểm $ A(6;8;0)$ và có vectơ chỉ phương $\vec{u}=(2;2;-1)$
}
{
    \includegraphics[width=5cm]{img/HXN-11-16}
}
 \choiceTF
 {Điểm $ A(6;8;0)$ nằm bên trong trạm vũ trụ hình cầu}
 {\True Phương trình quỹ đạo di chuyển của tàu vũ trụ là $\Delta \colon \heva{& x=6+2t\\& y=8+2t\\& z=-t}$ ($t$ là tham số)}
 {Khoảng cách từ tâm của trạm vũ trụ đến đường thẳng quỹ đạo tàu vũ trụ là $3{,}8$ km (làm tròn đến hàng phần chục, đơn vị km)}
 {Tàu vũ trụ sẽ tốn $2{,}3$ giây (làm tròn đến hàng phần chục của giây) để vượt qua phạm vi hình cầu của trạm vũ trụ}
 \loigiai{
     \begin{itemchoice}
         \immini
         {
             \itemch Ta có $OA=\sqrt{(6-0)^2+(8-0)^2+(0-0)^2}=10=R$.
         Do vậy điểm $A$ nằm trên bề mặt hình cầu.
         \itemch  Đường thẳng $\triangle $ qua $A(6;8;0)$,\\
         có vectơ chỉ phương $\vec{u}=(2;2;-1)$ nên có phương trình tham số là $\triangle \colon \heva{& x=6+2t \\& y=8+2t \\& z=-t} $ ($t$ là tham số).
         }
         {
             \includegraphics[width=5cm]{img/HXN-11-16-LG}
         }
         \itemch Khoảng cách từ tâm $O$ đến đường thẳng quỹ đạo là $4$ km.\\
         Ta có $\heva{& \vec{OA}=(6;8;0) \\& \vec{u}=(2;2;-1)} \Rightarrow \left[\vec{OA},\vec{u}\right]=(-8;6;-4)$.\\
         Ta có $d\left(O,\triangle \right)=\dfrac{\left| \left[\vec{OA},\vec{u}\right] \right|}{\left| {\vec{u}} \right|}=\dfrac{\sqrt{(-8)^2+6^2+(-4)^2}}{\sqrt{2^2+2^2+(-1)^2}}=\dfrac{2\sqrt{29}}{3}\approx 3{,}6\,km$.\\
         \itemch Vì $d\left(O,\triangle \right)<R$ nên tàu vũ trụ sẽ đi qua phạm vi hình cầu của trạm vũ trụ, nó cắt mặt cầu tại hai điểm $A$ và $B$; gọi $H$ là trung điểm $AB$.\\
         Ta có $AB=2AH=2\sqrt{OA^2-OH^2}=2\sqrt{10^2-\left(\dfrac{2\sqrt{29}}{3}\right)^2}=\dfrac{56}{3}$ km.\\
         Thời gian khi tàu vũ trụ đi qua phạm vi hình cầu là $\left(\dfrac{56}{3}\colon 27\,000\right)\cdot 3\,600\approx 2{,}5$ (giây).
     \end{itemchoice}
 }
\end{ex}
\Closesolutionfile{ans}
\caukq
\Opensolutionfile{ans}[ans/ans-HXN-\sode-SA]
\begin{ex}%Câu 17
\immini
{
     Một khối đá có dạng hình lăng trụ tam giác đều $ ABC.A'B'C'$ với cạnh đáy bằng $2$ dm, khoảng cách tứ điểm $A'$ đến mặt phẳng $\left(AB'C"\right)$ bằng $\dfrac{\sqrt{3}}{2}$dm. Tìm khoảng cách giữa hai mặt phẳng đáy của khối đá hình lăng trụ đã cho theo đơn vị dm
 \shortans{1}
}
{
    \includegraphics[width=5cm]{img/HXN-11-17}
}
\loigiai{
    \immini
    {
        Trong mặt phẳng ($A'B'C'$), kẻ $A'H\perp B'C'$ tại $H$.\\
        Trong mặt phẳng $\left(AA'H\right)$, kẻ $A'K\perp AH$ tại $K$. \tagEX{1}
    Ta có $\heva{& B'C'\perp A'H \\& B'C'\perp AA'\,\left(do\,AA'\perp \left(A'B'C'\right)\right) }$\\
    $ \Rightarrow B'C'\perp \left(AA'H\right)\Rightarrow A'K\perp B'C'$. \tagEX{2}
    Từ $(1)$ và $(2)$ suy ra $A'K\perp \left(AB'C'\right)$ hay\\
    \centerline{
        $d\left(A',\left(AB'C'\right)\right)=A'K=\dfrac{\sqrt{3}}{2}$ dm.
    }
    Tam giác $A'B'C'$ đều có đường cao $A'H=\dfrac{2\cdot \sqrt{3}}{2}=\sqrt{3}$ dm.\\
    Tam giác $AA'H$ vuông tại $A'$ có đường cao $A'K$ nên\\
    \centerline{
        $\dfrac{1}{A'K^2}=\dfrac{1}{A'H^2}+\dfrac{1}{A'A^2} \Rightarrow \dfrac{1}{\dfrac{3}{4}}=\dfrac{1}{3}+\dfrac{1}{A'A^2}\Rightarrow A'A=1$ dm.
    }
    }
    {
        \includegraphics[width=5cm]{img/HXN-11-17-LG}
    }
     Hai mặt đáy song song với nhau và có khoảng cách là $d\left((ABC),\left(A'B'C'\right)\right)=AA'=1$ dm.
}
 \end{ex}
 \begin{ex}%Câu 18
\immini
{
    Giá bán $P$ (đồng) của một sản phẩm y tế thay đổi theo số lượng $ Q$ (sản phẩm) với $ 0\le Q\le 1\,500$, được cung cấp ra thị trường theo công thức $ P=\sqrt{1500-Q}$. Tính số lượng sản phẩm y tế đó nên được cung cấp ra thị trường để doanh thu của công ty lớn nhất.
\shortans{1000}
}
{
    \includegraphics[width=5cm]{img/HXN-11-18}
}
\loigiai{
    Doanh thu của sản phẩm được tính theo công thức $R=PQ=Q\sqrt{1\,500-Q}$.\\
    Ta có $R'=\dfrac{-3Q+3\,000}{2\sqrt{1\,500-Q}};R'=0\Leftrightarrow Q=1\,000$.\\
    So sánh $R(0)\,,R\left(1\,000\right)$ và $R\left(1\,500\right)$ ta có $R$ lớn nhất khi $Q=1\,000$.
}
\end{ex}
\begin{ex}%Câu 19
\immini
{
    Một tấm đề can hình chữ nhật được cuộn tròn lại theo chiều dài tạo thành một khối trụ có đường kính $50\text{(cm)}$ . Người ta trải ra $250$ vòng để cắt chữ và in tranh cổ động, phần còn lại là một khối trụ có đường kính $45\text{(cm)}$ . Hỏi phần đã trải ra dài bao nhiêu mét (làm tròn đến hàng đơn vị)?
\shortans{373}
}
{
    \includegraphics[width=5cm]{img/HXN-11-19}
}
\loigiai{
    \textbf{Cách giải 1:} Gọi $a$ là bề dày của tấm đề can, sau mỗi vòng được quấn thì đường kính của vòng mới sẽ được tăng lên $2a$.\\
    Vì vậy: $2a\times 250=50-45\Rightarrow a=\dfrac{50-45}{2\times 250}=0{,}01\,cm$.\\
    Gọi $l$ là chiều dài đã trải ra và $h$ là chiều rộng của tấm đề can (tức chiều cao hình trụ).\\
    Khi đó ta có: $lha=\pi \left(\dfrac{50}{2}\right)^2h-\pi \left(\dfrac{45}{2}\right)^2h \Rightarrow l=\dfrac{\pi \left(50^2-45^2\right)}{4a} \approx 37\,306cm \approx 373m$.\\
    \textbf{Cách giải 2:} Gọi $a$ là bề dày của tấm đề can, sau mỗi vòng được quấn thì đường kính của vòng mới sẽ được tăng lên $2a$.\\
    Vì vậy: $2a\times 250=50-45\Rightarrow a=\dfrac{50-45}{2\times 250}=0{,}01\,cm$.\\
    Chiều dài của phần trải ra là tổng chu vi của $250$ đường tròn có bán kính là một cấp số cộng có số hạng đầu bằng $r_1=25$, công sai là $d=-0{,}01$ (do khi trải ra thì bán kính các vòng tròn ngày càng giảm với độ giảm bằng bề dày của tấm đề can).\\
    Do đó chiều dài của phần đề can đã trải ra là: $l=2\pi \left(\underbrace{r_1+r_2+\cdot \cdot \cdot +r_{250}}_{S_{250}}\right)$
    $=2\pi \cdot \dfrac{(2r_1+249d)\cdot 250}{2} =2\pi (2\cdot 25-249\cdot 0{,}01)\dfrac{250}{2}\approx 37314cm \approx 373m$.
}
\end{ex}
\begin{ex}%Câu 20
\immini
{
    Một viên gạch lát nền nhà có dạng hình vuông với hoa văn được thiết kế bởi một học sinh lớp $12$. Xét hình phẳng có diện tích $S_1$ được tạo thành bởi các đường cong $\left(L_1\right),\left(L_2\right)$ và một cạnh viên gạch, trong đó đường $\left(L_1\right)$ là tập hợp các điểm M thỏa mãn $ MA=\sqrt{2}d\left(M,\Delta\right)$ (A là trung điểm một cạnh viên gạch; $\Delta $ là đường phân giác góc phần tư thứ nhất theo hình vẽ); $\left(L_2\right)$ đối xứng với $\left(L_1\right)$ qua trục Ox. Biết viên gạch này có kích thước cạnh bằng 40 cm, tính tổng diện tích $S_1+S_2+S_3+S_4$ và làm tròn đến hàng đơn vị của $ c{m^2}$
}
{
    \includegraphics[width=5cm]{img/HXN-11-20}
}
\loigiai{
    \immini
    {Tọa độ $A(20;0)$ và phương trình $\triangle \colon x-y=0$.\\
    Gọi $M(x;y)$ là tập hợp các điểm thuộc đường cong $\left(L_1\right)$, khi đó
    \begin{eqnarray*}
        &&MA=\sqrt{2}d\left(M,\triangle \right)\\
        &\Leftrightarrow& \sqrt{(x-20)^2+y^2}=\sqrt{2}\cdot \dfrac{|x-y|}{\sqrt{2}} \\
        &\Leftrightarrow& x^2+y^2-40x+400=x^2+y^2-2xy\\
        &\Leftrightarrow& xy=20x-200\\
        &\Leftrightarrow & y=\dfrac{20x-200}{x}
    \end{eqnarray*}
    }
    {
        \includegraphics[width=5cm]{img/HXN-11-20-LG}
    }
    Giao điểm giữa $\left(L_1\right)$ với $Ox$ thỏa hệ phương trình $\heva{& y=\dfrac{20x-200}{x} \\& y=0 } \Rightarrow x=10$.\\
    Ta có: $S_1+S_2+S_3+S_4=4S_1=4\cdot 2\int\limits_{10}^{20}{\dfrac{20x-200}{x}\mathrm{\,d}x}\approx 491\,cm^2$.
    
}
\shortans{491}
\end{ex}
\begin{ex}%Câu 21
\immini
{
    Hộp thứ nhất đựng $6$ viên bi xanh và $4$ viên bi vàng, hộp thứ hai đựng $5$ viên bi xanh và một số bi vàng. Người ta thực hiện ngẫu nhiên ba hành động sau:
\begin{itemize}
    \item Lấy ngẫu nhiên $2$ viên bi từ hộp thứ nhất bỏ sang hộp thứ hai.
    \item Chọn ngẫu nhiên $1$ viên bi từ hộp thứ hai rồi hoàn lại hộp này.
    \item Chọn ngẫu nhiên $1$ viên bi từ hộp thứ hai lần nữa.
\end{itemize}
}
{
    \includegraphics[width=5cm]{img/HXN-11-21}
}
Biết hai lần lấy bi từ hộp thứ hai đều được bi xanh, tính xác suất để hai lần lấy chọn trúng 1 viên bi, và đó cũng là viên bi từ hộp thứ nhất chuyển sang hộp thứ hai (làm tròn kết quả đến hàng phần trăm).
\shortans{0,03}
\loigiai{
    Gọi $A$ là biến cố: \lq\lq Hai lần lấy bi từ hộp thứ hai đều được bi có màu xanh\rq\rq\, và $B$ là biến cố: \lq\lq Hai lần lấy đúng 1 bi và cũng là bi từ hộp thứ nhất chuyển qua\rq\rq.\\
    Giả sử ban đầu hộp thứ hai có $5$ viên bi xanh và $x$ viên bi vàng $\left(x\in \mathbb{N}^*\right)$.\\
    Xác suất để lấy được $2$ bi xanh từ hộp thứ nhất chuyển sang hộp thứ hai là $\dfrac{\mathrm{C}_6^2}{\mathrm{C}_{10}^2}=\dfrac{1}{3}$.\\
    Xác suất để lấy được $1$ bi xanh, $1$ bi vàng từ hộp thứ nhất chuyển sang hộp thứ hai là $\dfrac{\mathrm{C}_6^1\cdot \mathrm{C}_4^1}{\mathrm{C}_{10}^2}=\dfrac{8}{15}$.\\
    Xác suất để lấy được $2$ bi vàng từ hộp thứ nhất chuyển sang hộp thứ hai là $\dfrac{\mathrm{C}_4^2}{\mathrm{C}_{10}^2}=\dfrac{2}{15}$.\\
    Ta biểu diễn bài toán bằng sơ đồ sau:\\
    \centerline{
        \includegraphics[width=5cm]{img/HXN-11-21-LG}
    }
    Ta có $P(A)=\dfrac{1}{3}\cdot \left(\dfrac{7}{x+7}\right)^2+\dfrac{8}{15}\cdot \left(\dfrac{6}{x+7}\right)^2+\dfrac{2}{15}\cdot \left(\dfrac{5}{x+7}\right)^2$.\\
    Do đó 
    \begin{eqnarray*}
        P\left(B|A\right)&=&\dfrac{\dfrac{1}{3}\cdot \dfrac{2}{x+7}\cdot \dfrac{1}{x+7}+\dfrac{8}{15}\cdot \dfrac{1}{x+7}\cdot \dfrac{1}{x+7}}{\dfrac{1}{3}\cdot \left(\dfrac{7}{x+7}\right)^2+\dfrac{8}{15}\cdot \left(\dfrac{6}{x+7}\right)^2+\dfrac{2}{15}\cdot \left(\dfrac{5}{x+7}\right)^2}\\
        &=&\dfrac{\dfrac{1}{(x+7)^2}\left(\dfrac{2}{3}+\dfrac{8}{15}\right)}{\dfrac{1}{(x+7)^2}\left(\dfrac{7^2}{3}+\dfrac{8\cdot 6^2}{15}+\dfrac{2\cdot 5^2}{15}\right)}\\
        &=&\dfrac{18}{583}\approx 0{,}03.
    \end{eqnarray*}
}
\end{ex}
\begin{ex}%Câu 22
\immini
{
Trong không gian với hệ trục tọa độ $Oxyz$ thích hợp, đơn vị trên mỗi trục là mét. Một nhà sinh vật học muốn theo dõi hai tổ chim ở các vị trí $ A\left(2;2;0\right)$, $B\left(2;0;-2\right)$, anh ta đã leo lên mái nhà thuộc mặt phẳng $(P)\colon x+2y-z-1=0$. Nhà sinh vật học muốn đặt một thiết bị theo dõi ở vị trí $ M\left(a;b;c\right)$ thuộc mái nhà cách đều các tổ chim, đồng thời vị trí đó cho anh một góc quan sát là lớn nhất đối với hai tổ chim nói trên (góc $\widehat{AMB}$ lớn nhất). Khi đó tổng $ a+b+c$ bằng bao nhiêu (làm tròn đến hàng phần trăm)?
\shortans{1,27}
}
{
    \includegraphics[width=5cm]{img/HXN-11-22}
}
\loigiai{
    Ta có $M$ thuộc mặt phẳng $(P)$ và $MA=MB$ nên $M$ thuộc giao tuyến của hai mặt phẳng $(P)$ và $(Q)$, trong đó $(Q)$ là mặt phẳng trung trực của đoạn thẳng $AB$.\\
    Mặt phẳng $(Q)$ qua trung điểm $I(2;1;-1)$ của $AB$, vectơ pháp tuyến $\vec{AB}=(0;-2;-2)$ nên có phương trình  $0(x-2)-2(y-1)-2(z+1)=0$ hay $y+z=0$.\\
    Gọi $d=(P)\cap (Q)$; từ hệ phương trình $\heva{& x+2y-z-1=0 \\& y+z=0 }$, đặt $z=t$ ta có phương trình tham số đường thẳng $d$ là $\heva{& x=1+3t \\& y=-t \\& z=t}$.\\
    Với $M\left(1+3t;-t;t\right)\in d$; suy ra $\vec{AM}=\left(3t-1;-t-2;t\right)$, $\vec{BM}=\left(3t-1;-t;t+2\right)$.\\
    Ta có $\cos \widehat{AMB}=\cos \left(\vec{AM},\vec{BM}\right)=\dfrac{(3t-1)^2+2\left(t^2+2t\right)}{(3t-1)^2+t^2+(t+2)^2}=\dfrac{11t^2-2t+1}{11t^2-2t+5}=1-\dfrac{4}{11t^2-2t+5}$.\\
    Ta thấy $\widehat{AMB}$ lớn nhất khi $\cos \widehat{AMB}$ bé nhất; suy ra $1-\dfrac{4}{11t^2-2t+5}$ bé nhất.\\
    Khi đó $\dfrac{4}{11t^2-2t+5}$ lớn nhất nên $11t^2-2t+5=11\left(t-\dfrac{1}{11}\right)^2+\dfrac{54}{11}$ bé nhất; suy ra $t=\dfrac{1}{11}$.\\
    Ta tìm được điểm $M\left(\dfrac{14}{11};-\dfrac{1}{11};\dfrac{1}{11}\right)$ với $a=\dfrac{14}{11}$; $b=-\dfrac{1}{11}$; $c=\dfrac{1}{11}\Rightarrow S=a+b+c=\dfrac{14}{11}\approx 1{,}27$.
}
\end{ex}
\Closesolutionfile{ans}
\inputansbox{6,4,3}{ans/ans-HXN-\sode-T,ans/ans-HXN-\sode-TF,ans/ans-HXN-\sode-SA}
% %%%%%%%%%%%%%%%%%%%- HXN
\def\sode{12}
\def\tendethi{ĐỀ PHÁT TRIỂN MINH HOẠ 2025}
\begin{dethi}
 {\tendethi}
\end{dethi}
\caulc
\Opensolutionfile{ans}[ans/ans-HXN-\sode-T]
\begin{ex}%Câu 1
 Tìm tập xác định của hàm số $ y=\log_2\left(x-3\right)$.
 \choice
 {$\mathscr{D}=\left(-\infty;3\right)$}
 {$\mathscr{D}=\mathbb{R}$}
 {\True $\mathscr{D}=\left(3;+\infty\right)$}
 {$\mathscr{D}=\left[3;+\infty\right)$}
\end{ex}
\begin{ex}%Câu 2
 Số đường tiệm cận của đồ thị hàm số $y=\dfrac{3x-4}{x-1}$ bằng
 \choice
 {\True $2$}
 {$3$}
 {$1$}
 {$0$}
\end{ex}
\begin{ex}%Câu 3
 Độ $pH$ của một dung dịch được tính theo công thức $ pH=-\log\left[H^+\right]$ với $\left[H^+\right]$ là nồng độ ion hydrogen của dung dịch đó. Độ $ pH$ của một loại sữa có $\left[H^+\right]=10^{-6{,}8}$ là bao nhiêu?
 \choice
 {$-6{,}8$}
 {$68$}
 {\True $6{,}8$}
 {$0{,}68$}
\end{ex}
\begin{ex}%Câu 4
 Cho hình lăng trụ đứng $ABC.A'B'C'$ có $BB'=a$ , đáy $ABC$ là tam giác vuông cân tại $ B$ và $BA=BC=a$ . Tính thể tích $ V$ của khối lăng trụ đã cho.
 \choice
 {$V=a^3$}
 {$V=\dfrac{a^3}{3}$}
 {$V=\dfrac{a^3}{6}$}
 {\True $V=\dfrac{a^3}{2}$}
\end{ex}
\begin{ex}%Câu 5
 Trong không gian với hệ tọa độ $ Oxyz$, cho mặt phẳng $\left(\alpha\right)\colon x-2y+2z-3=0.$ Điểm nào sau đây thuộc mặt phẳng $\left(\alpha\right)$?
 \choice
 {$M\left(2;0;1\right)$}
 {$Q\left(2;1;1\right)$}
 {$P\left(2;-1;1\right)$}
 {\True $N\left(1;0;1\right)$}
\end{ex}
\begin{ex}%Câu 6
 Cho hình hộp $ ABCD.A'B'C'D'$. Vectơ $\overrightarrow{v}=\overrightarrow{B'A'}+\overrightarrow{B'C'}+\overrightarrow{B'B}$ bằng vectơ nào dưới đây?
 \choice
 {$\overrightarrow{DB'}$}
{$\overrightarrow{B'D'}$}
{$\overrightarrow{BD'}$}
{\True $\overrightarrow{B'D}$}
\end{ex}
\begin{ex}%Câu 7
Họ nguyên hàm của hàm số $ f(x)=3x^2+\sin x$ là 
\choice
{$x^3+\cos x+C$}
{$x^3+\sin x+C$}
{\True $x^3-\cos x+C$}
{$3x^3-\sin x+C$}
\end{ex}
\begin{ex}%Câu 8
Trong không gian $ Oxyz$, phương trình nào sau đây là phương trình của mặt cầu có tâm $ I\left(7;6;-5\right)$ và bán kính $ 9$?
\choice
{$\left(x+7\right)^2+\left(y+6\right)^2+\left(z-5\right)^2=81$}
{$\left(x+7\right)^2+\left(y+6\right)^2+\left(z-5\right)^2=9$}
{\True $\left(x-7\right)^2+\left(y-6\right)^2+\left(z+5\right)^2=81$}
{$\left(x-7\right)^2+\left(y-6\right)^2+\left(z+5\right)^2=9$}
\end{ex}
\begin{ex}%Câu 9
Bảng số liệu ghép nhóm về chiều cao đo được (đơn vị: cm) của $30$ học sinh nam lớp 12A2 đầu năm học $2024-2025$ của một trường THPT được cho như sau:\\
\centerline{\begin{tblr}{|c|c|c|c|c|c|}
 \hline
 Chiều cao & $\left[150;\ 155\right)$ & $\left[155;\ 160\right)$ & $\left[160;\ 165\right)$ & $\left[165;\ 170\right)$ & $\left[170;\ 175\right)$\\
 \hline
 Tần số & $ 3$ & $ 7$ & $ 10$ & $ 7$ & $ 3$\\
 \hline
\end{tblr}}\\
Tính độ lệch chuẩn của mẫu số liệu ghép nhóm trên.
\choice
{\True $\dfrac{\sqrt{285}}{3}$}
{$\dfrac{\sqrt{287}}{3}$}
{$ 4\sqrt{2}$}
{$\sqrt{71}$}
\end{ex}
\begin{ex}%Câu 10
Cho biết $\lim\limits_{x\to\sqrt{3}}\dfrac{2x^2-6}{x-\sqrt{3}}=a\sqrt{b}$ ($a$, $b$ nguyên và $b<10$). Khi đó giá trị của $P=a+b$ bằng
\choice
{\True $7$}
{$10$}
{$5$}
{$6$}
\end{ex}
\begin{ex}%Câu 11
Cho $\int\dfrac{1}{x{\ln^2}x}\mathrm{\,d}x=F(x)+C$. Khẳng định nào dưới đây đúng?
\choice
{$F'(x)=-\dfrac{1}{\ln x}$}
{$F'(x)=-\dfrac{1}{\ln x}+C$}
{\True $F'(x)=\dfrac{1}{x{\ln^2}x}$}
{$F'(x)=-\dfrac{1}{\ln^2x}$}
\end{ex}
\begin{ex}%Câu 12
\immini
{
 Diện tích phần gạch sọc trong hình vẽ bằng
\choice
{$\int\limits_{-3}^1\left|-x^2-2x-3\right|\mathrm{\,d}x$}
{$\int\limits_{-3}^1\left(x^2-2x-3\right)\mathrm{\,d}x$}
{$\int\limits_{-3}^1\left(x^2+2x-3\right)\mathrm{\,d}x$}
{\True $\int\limits_{-3}^1\left(-x^2-2x+3\right)\mathrm{\,d}x$}
}
{
 \begin{tikzpicture}[>=stealth, line join=round, line cap=round, font=\footnotesize, scale=1,thick,x=.5cm,y=.5cm,
 declare function={f(\x)=-1*(\x)+1;g(\x)=1*(\x)^2+1*(\x)-2;}]
 \draw[->] (-3.5,0)--(2.5,0) node[below left] {$x$};
 \draw[->] (0,-3)--(0,4.5) node[below left] {$y$};
 \draw (0,0) node [below left] {$O$};
 \foreach \x/\nx in {-3/-3,1/1}
 \draw (\x,1pt)--(\x,-1pt) node [below] {$\nx$};
 \draw[pattern = north east lines, pattern color=green, line width = 1.2pt,draw=none] (-3,{f(3)}) plot[domain=1:-3] (\x, {g(\x)});
 \draw[dashed] (-3,0)--(-3,{f(-3)});
 \draw[samples=200,domain=-3.5:1.5,smooth] plot (\x,{f(\x)});
 \draw[samples=200,domain=-3.1:1.2,smooth,variable=\x] plot (\x,{g(\x)});
 \end{tikzpicture}
}
\end{ex}
\Closesolutionfile{ans}
\cauds
\Opensolutionfile{ans}[ans/ans-HXN-\sode-TF]
\begin{ex}%Câu 13
    Cho hàm số $f(x)=\cos 2x+2x+1$
    \choiceTF
    {\True $ f\left(\dfrac{\pi}{2}\right)=\pi $}
    {Đạo hàm của hàm số là $f'(x)=2\sin 2x+2$}
    {\True Nghiệm của phương trình $f'(x)=0$ trên đoạn $\left[-\dfrac{\pi}{2};\pi\right]$ là $ x=\dfrac{\pi}{4}$}
    {Tổng giá trị lớn nhất và giá trị nhỏ nhất của hàm số trên đoạn $\left[-\dfrac{\pi}{2};\pi\right]$ bằng $ 2\pi $}
    \loigiai{
        \begin{itemchoice}
            \itemch Ta có: $f\left(\dfrac{\pi }{2}\right)=\cos \left(2\cdot \dfrac{\pi }{2}\right)+2\cdot \dfrac{\pi }{2}+1=\pi $.
            \itemch Ta có $f'(x)=-2\sin 2x+2$.
            \itemch $f'(x)=0\Leftrightarrow -2\sin 2x+2=0\Leftrightarrow \sin 2x=1\Leftrightarrow 2x=\dfrac{\pi }{2}+k2\pi \,,k\in \mathbb{Z}\Leftrightarrow x=\dfrac{\pi }{4}+k\pi ,k\in \mathbb{Z}$.
            Vì $x\in \left[-\dfrac{\pi }{2};\pi \right]$ nên $x=\dfrac{\pi }{4}$ (ứng với $k=0$).
            \itemch Ta có: $f\left(-\dfrac{\pi }{2}\right)=-\pi $; $f\left(\dfrac{\pi }{4}\right)=\dfrac{\pi }{2}+1$; $f\left(\pi \right)=2\pi +2$.\\
            Vì vậy $\max\limits_{\left[-\dfrac{\pi}{2};\pi \right]} \,f(x)=f\left(\pi \right)=2\pi +2$ và $\min\limits_{\left[-\dfrac{\pi}{2};\pi \right]}\,f(x)=f\left(-\dfrac{\pi }{2}\right)=-\pi $.\\
            Tổng giá trị của chúng là $2\pi +2+\left(-\pi \right)=\pi +2$.
        \end{itemchoice}
    }
\end{ex}
\begin{ex}%Câu 14
\immini
{
 Trong một hệ trục $Oxyz$ cho trước, đơn vị trên mỗi trục là mét, có hai cậu bé đang chơi bắn bi trên mặt đất phẳng (cũng là mặt phẳng $Oxy$). Gần vị trí các cậu bé có một bức tường được mô hình hóa bởi mặt phẳng có phương trình $x+y-10=0$.\\
Cậu bé An bắn viên bi từ vị trí $ A\left(0;0;0\right)$ và theo hướng vectơ $\vec{u}=\left(4;-1;0\right)$, bi có thể di chuyển $10{,}5$ mét nếu không có vật cản.
}
{
 \includegraphics[width=5.5cm]{img/HXN-12-14}
}
Cậu bé Cò bắn viên bi từ vị trí $B\left(2;-2;0\right)$ theo hướng $\vec{v}=\left(1;2;0\right)$, bi có thể di chuyển $8{,}2$ mét nếu không có vật cản.
\choiceTF
 {\True Đường đi viên bi là của An có phương trình $\heva{& x=4t\\& y=-t\\& z=0}$ ($t$ là tham số)}
{\True Đường đi của hai viên bi là các đường thẳng cắt nhau tại điểm có tung độ âm}
 {Đường đi của viên bi do bạn Cò bắn ra tạo với bức tường một góc thuộc khoảng $\left(69^{\circ};71^{\circ}\right)$}
 {\True Một trong hai viên bi chạm vào bức tường}
 \loigiai{
 \begin{itemchoice}
 \itemch Đường đi viên bi là của An có phương trình $d_1\colon \heva{& x=4t \\& y=-t \\& z=0 } $ ($t$ là tham số).
 \itemch Đường đi viên bi là của Cò có phương trình $d_2\colon \heva{& x=2+t' \\& y=-2+2t' \\& z=0 } $ ($t$ là tham số).\\
 Xét hệ phương trình $\heva{& 4t=2+t' \\& -t=-2+2t' \\& 0=0 } \Rightarrow \heva{& t=\dfrac{2}{3} \\& t'=\dfrac{2}{3} } $; đường đi $2$ viên bi cắt nhau tại $M\left(\dfrac{8}{3};-\dfrac{2}{3};0\right)$; trong đó $y_M=-\dfrac{2}{3}<0$ (thỏa mãn).
 \itemch Đường thẳng $d_2$ có vectơ chỉ phương $\vec{v}=(1;2;0)$; mặt phẳng chứa bức tường có vectơ pháp tuyến $\vec{n}=(1;1;0)$ nên $\sin \left(d_2\,,(P)\right)=\dfrac{\left| \vec{v}\cdot \vec{n} \right|}{\left| {\vec{v}} \right|\cdot \left| {\vec{n}} \right|}=\dfrac{|1+2|}{\sqrt{5}\cdot \sqrt{2}}=\dfrac{3\sqrt{10}}{10}$.\\
 $\Rightarrow \left(d_2,(P)\right)\approx 71,6^{\circ }\notin \left(69^{\circ };71^{\circ }\right)$.\\
 \itemch Xét $d_1$ và $(P)\colon x+y-10=0$;\\
 thay phương trình $d_1$ vào $(P)$, ta có $4t+(-t)-10=0\Rightarrow t=\dfrac{10}{3}$.\\
 Do đó $d_1\cap (P)=C\left(\dfrac{40}{3};-\dfrac{10}{3};0\right)$ với $AC=\dfrac{10\sqrt{17}}{3}\approx 13{,}7\,m>10{,}5\,m$.\\
 Vậy viên bi của An không chạm tường.\\
 Xét $d_2$ và $(P)\colon x+y-10=0$;\\
 thay phương trình $d_2$ vào $(P)$, ta có $(2+t)+\left(-2+2t'\right)-10=0\Rightarrow t'=\dfrac{10}{3}$.\\
 Do đó $d_2\cap (P)=D\left(\dfrac{16}{3};\dfrac{14}{3};0\right)$ với $BD=\dfrac{10\sqrt{5}}{3}\approx 7{,}45\,m<8{,}2\,m$.\\
 Vậy viên bi của Cò chạm vào tường.
 \end{itemchoice}
 }
\end{ex}
\begin{ex}%Câu 15
\immini
{
 Trong một live show âm nhạc có ca sĩ Mỹ Tâm tham gia, nhiều fan hâm mộ đã tỏ ra lo ngại rằng ban tổ chức có thể hủy show với một vài lý do khác nhau. Những lo ngại này là đúng vì có đến $0{,}302$ khả năng show diễn sẽ bị hủy.\\
Nếu vé bán hết thì chắc chắn live show sẽ diễn ra.\\
Nếu trời mưa thì ban tổ chức không thể bán hết vé, khi đó khả năng hủy show là $50\%$.\\
Nếu trời không mưa thì khả năng vé được bán hết là $90\%$; trong trường hợp còn vé thì khả năng hủy show là $5\%$.
}
{
 \includegraphics[width=5.5cm]{img/HXN-12-15}
}
\choiceTF
 {\True Nếu vé chưa được bán hết khi trời không mưa thì xác suất để show vẫn diễn ra bằng $0{,}95$}
 {Xác suất để trời mưa bằng $0{,}55$}
 {\True Xác suất để ban tổ chức không bán hết vé bằng $0{,}64$}
 {\True Sau cùng thì show của Mỹ Tâm cũng đã diễn ra, xác suất hôm đó trời mưa bằng $0{,}43$ (làm tròn kết quả đến hàng phần trăm)}
 \loigiai{
 \begin{itemchoice}
 \immini
 {
 \itemch Nếu vé chưa được bán hết khi trời không mưa thì xác suất để show vẫn diễn ra bằng $1-0{,}05=0{,}95$.
 \itemch Gọi A là biến cố \lq\lq Trời mưa\rq\rq, B là biến cố: \lq\lq Vé được bán hết\rq\rq, C là biến cố: \lq\lq Show bị hủy\rq\rq.\\
 Đặt $P(A)=x\in (0;1)$ là xác suất để trời mưa, ta có sơ đồ hình cây bên cạnh.\\
 Khi đó $P(C)=x\cdot 1.0{,}5+(1-x)\cdot 0{,}9.0+(1-x)\cdot 0{,}1.0{,}05=0{,}302\Rightarrow x=0{,}6$ hay $P(A)=0{,}6$.
 }
 {
 \includegraphics[width=5.5cm]{img/HXN-12-15-LG}
 }
 \itemch Ta có $P\left({\bar{B}}\right)=\underbrace{x\cdot 1+(1-x)\cdot 0{,}1}_{x=0{,}6}=0{,}64$.
 \itemch Ta có $P\left({\bar{C}}\right)=1-0{,}302=0{,}698$; $P\left(A|\bar{C}\right)=\dfrac{P\left(A\bar{C}\right)}{P\left({\bar{C}}\right)}=\dfrac{0{,}6.1\cdot 0{,}5}{0{,}698}=\dfrac{150}{349}\approx 0{,}43$.
 \end{itemchoice}
 }
\end{ex}
\begin{ex}%Câu 16
    \immini
    {
        Cho đường cong $(C)\colon y=8x-27x^3$ và đường thẳng $d\colon y=m$ cắt $(C)$ tại ba điểm phân biệt, trong đó có hai điểm nằm ở góc phần tư thứ nhất của hệ tọa độ $ Oxy$. Các phần diện tích tô đậm được kí hiệu là $S_1$, $S_2$ như hình vẽ.
        \choiceTF
        {Diện tích hình phẳng giới hạn vởi $(C)$ và $Ox$ bằng $\dfrac{34}{27}$}
        {Có ba giá trị nguyên của $m$ thỏa mãn đề bài}
        {\True Nếu $ m=\dfrac{5}{8}$ thì $S_1+S_2=0,31$ (làm tròn đến hàng phần trăm)}
        {Nếu $S_1=S_2$ thì có một giá trị $ m=m_0$ thỏa mãn với $ 1<m_0<\dfrac{3}{2}$}
    }
    {
        \begin{tikzpicture}[>=stealth, line join=round, line cap=round, thick,declare function={f(\x)=-27*((\x)^3)+0*((\x)^2)+8*(\x)+0;m=7/8;g(\x)=m;},xscale=3]
            \draw[->] (-1,0)--(1,0) node[below left] {$x$};
            \draw[->] (0,-2.5)--(0,2.5) node[below left] {$y$};
            \draw (0,0) node [below left] {$O$};
            \begin{scope}
                \clip (-1,-2.2) rectangle (1,2.2);
                \draw[samples=200,domain=-2:2,smooth,name path=df] plot (\x,{f(\x)});
                \draw[name path=dg] (-2,m)--(2,m)node[pos=.72,above]{$d$};
                \path [name intersections= {of = df and dg, by={a,b,c}}];
            \end{scope}
            \fill[pattern=north east lines] let \p2=(b),\n2={\x2/1cm} in plot[domain=0:\n2](\x,{f(\x)})--(0,m)--cycle;
            \fill[pattern=north west lines,pattern color=red] let \p2=(b),\n2={\x2/1cm},\p3=(c),\n3={\x3/1cm} in plot[domain=\n2:\n3](\x,{f(\x)})--cycle;
            \draw[->] (.07,.7)--++(-155:.3)node[below left]{$S_1$};
            \draw[->] (.3,1.3)--++(55:.5)node[above right]{$S_2$};
        \end{tikzpicture}
    }
    \loigiai{
        \begin{itemchoice}
            \itemch Hoành độ giao điểm của $(C)$ và $Ox$ thỏa mãn $8x-27x^3=0\Leftrightarrow \hoac{& x=0 \\& x=\pm \dfrac{2\sqrt{6}}{9} } $.\\
            Diện tích cần tính là $S=\displaystyle\int\limits_{-\tfrac{2\sqrt{6}}{9}}^{\tfrac{2\sqrt{6}}{9}}{\left| 8x-27x^3 \right|\mathrm{\,d}x}=\dfrac{32}{27}$.
            \itemch Xét $(C):y=8x-27x^3$ có đạo hàm $y'=8-81x^2;y'=0\Rightarrow x=\pm \dfrac{2\sqrt{2}}{9}$.\\
            Bảng biến thiên:
            \begin{center}
                \begin{tikzpicture}[>=stealth]
                    \tkzTabInit[nocadre=false,lgt=1,espcl=2.5,deltacl=0.5]{$x$/1,$y'$/.7,$y$/2}
                    {$-\infty$ , $-\dfrac{2\sqrt{2}}{9}$ , $\dfrac{2\sqrt{2}}{9}$ , $+\infty$}
                    \tkzTabLine{ ,-,$0$,+,$0$,-, }
                    \tkzTabVar{+/$+\infty$,-/$\approx -1{,}7$,+/$\approx 1{,}7$,-/$-\infty$}
                \end{tikzpicture}
            \end{center}
            Xét đường thẳng d:$y=m$ cắt đồ thị hàm số tại ba điểm phân biệt (giả thiết); mà $m$ nguyên nên $m\in \left\{ -1;0;1 \right\}$. Tuy nhiên chỉ có $m=1$ thỏa mãn vì khi đó d và (C) có hai giao điểm thuộc góc phần tư thứ nhất. 
            \itemch 
            Nếu $m=\dfrac{5}{8}$ thì ta có phương trình hoành độ giao điểm: $8x-27x^3=\dfrac{5}{8}\Leftrightarrow \hoac{& x=\dfrac{1}{2} \\& x=\dfrac{-9\pm \sqrt{141}}{36} } $.\\
            Ta chỉ xét góc phần tư thứ nhất nên nhận $x=\dfrac{1}{2};x=\dfrac{-9+\sqrt{141}}{36}$.\\
            Khi đó $S_1+S_2=\displaystyle\int\limits_0^{\tfrac{1}{2}}{\left| \left(8x-27x^3\right)-\dfrac{5}{8} \right|\mathrm{\,d}x}\approx 0{,}31$.
            \itemch 
            Giả sử đường thẳng $y=m$ cắt đồ thị $(C)$ tại điểm có hoành độ $a,b\,\left(0<a<b\right)$.\\
            Ta có: $8a-27a^3=m$ \tagEX{1}
            Xét hàm số $f(x)=8x-27x^3-m$ có $F(x)=\displaystyle\int{f(x)}\mathrm{\,d}x=4x^2-\dfrac{27}{4}x^4-mx+C$.\\
            Ta có 
            $
            \begin{aligned}[t]
                &S_1=\displaystyle\int\limits_0^a{\left| f(x) \right|}\mathrm{\,d}x=-\displaystyle\int\limits_0^a{f(x)}\mathrm{\,d}x=F(0)-F(a);\\ &S_2=\displaystyle\int\limits_a^b{\left| f(x) \right|}\mathrm{\,d}x=\displaystyle\int\limits_a^b{f(x)}\mathrm{\,d}x=F(b)-F(a).
            \end{aligned}
            $\\
            Theo giả thiết: $S_1=S_2\Leftrightarrow F(0)-F(a)=F(b)-F(a)\Leftrightarrow F(b)=F(0)$.\\
            Vì vậy $4b^2-\dfrac{27}{4}b^4-mb=0$ \tagEX{2}
            Từ $(1)$ và $(2)$ ta có $\heva{& 4b^2-\dfrac{27b^4}{4}-mb=0 \\& 8b-27b^3=m } \Rightarrow \heva{& 4b^2-\dfrac{27b^4}{4}-\left(8b-27b^3\right)b=0 \\& 8b-27b^3=m } \Rightarrow \heva{& b=\dfrac{4}{9}>0 \\& m=\dfrac{32}{27}.} $\\
            Vậy $m=\dfrac{32}{27}\approx 1{,}19\in \left(1;\dfrac{3}{2}\right)$.
        \end{itemchoice}
    }
\end{ex}
\Closesolutionfile{ans}
\caukq
\Opensolutionfile{ans}[ans/ans-HXN-\sode-SA]
\begin{ex}%Câu 17
 Cho hình chóp $S.ABCD$ có đáy là hình vuông cạnh $ 1$, tam giác $SAB$ đều và nằm trong mặt phẳng vuông góc với đáy. Tìm khoảng cách giữa hai đường thẳng SA, BC và làm tròn đến hàng phần trăm.
 \shortans{0,87}
 \loigiai{
\immini
{
         Gọi $H$ là trung điểm $AB$, suy ra $SH\perp AB$ (do tam giác $SAB$ đều).\\
     Mặt khác $(SAB)\perp (ABCD)$ nên $SH\perp (ABCD)$.\\
     Ta có $\heva{& (SAB)\perp (ABCD) \\& AB=(SAB)\cap (ABCD) \\& BC\perp AB } \Rightarrow BC\perp (SAB)$ \tagEX{1}
     Trong mặt phẳng $(SAB)$, dựng $BK\perp SA$ tại $K$ \tagEX{2}
     Từ $(1)$, $(2)$ suy ra $BK$ là đoạn vuông góc chung của $SA$ và $BC$.
     Vậy $d\left(SA, BC\right)=BK=\dfrac{\sqrt{3}}{2}\approx 0{,}87$.
}
{
\includegraphics[width=5cm]{img/HXN-12-17-LG}
}
 }
 \end{ex}
 \begin{ex}%Câu 18
\immini
{
    Một trò chơi điện tử quy định như sau: Có 5 trụ $ A$, $B$, $C$, $D$, $E$ với số lượng các thử thách trên đường đi giữa các cặp trụ được mô tả trong hình bên. Người chơi xuất phát từ một trụ nào đó, đi qua tất cả các trụ còn lại, mỗi khi đi qua trụ nào thì trụ đó sẽ bị phá hủy và không thể quay trở lại trụ đó được nữa, nhưng người chơi vẫn phải trở về trụ ban đầu. Tổng số thử thách của đường đi thoả mãn điều kiện trên nhận giá trị nhỏ nhất là bao nhiêu?
\shortans{45}
}
{
    \includegraphics[width=5cm]{img/HXN-12-18}
}
\end{ex}
\begin{ex}%Câu 19
\immini
{
    Giả sử số lượng của một quần thể nấm men tại môi trường nuôi cấy trong phòng thí nghiệm được mô hình hóa bằng hàm số $ P(t)=\dfrac{a}{b+e^{-0,75t}}$. Trong đó, thời gian $ t$ được tính bằng giờ. Tại thời điểm ban đầu $t=0$, quần thể có 20 tế bào và tăng với tốc độ $12$ tế bào/giờ. Số lượng của quần thể nấm này tại thời điểm $ t=8$ giờ là bao nhiêu tế bào (làm tròn đến hàng đơn vị)?
\shortans{99}
}
{
    \includegraphics[width=5cm]{img/HXN-12-19}
}
\loigiai{
    Ta có $P'(t)=\dfrac{0{,}75a{e^{-0{,}75t}}}{\left(b+{e^{-0{,}75t}}\right)^2}$, $t\ge 0$.\\
    Theo giả thiết: $\heva{& P(0)=20 \\& P'(0)=12 }\Leftrightarrow \heva{& \dfrac{a}{b+1}=20 \\& \dfrac{0{,}75a}{(b+1)^2}=12 } \Leftrightarrow \heva{& a=20(b+1) \\& \dfrac{0{,}75\times 20(b+1)}{(b+1)^2}=12 } \Leftrightarrow \heva{& a=25 \\& b=\dfrac{1}{4}.}$\\
    Do vậy $P(t)=\dfrac{25}{\dfrac{1}{4}+{e^{-0.75t}}}\Rightarrow P(8)\approx 99$ (tế bào).
}
\end{ex}
\begin{ex}%Câu 20
\immini
{
    Trong không gian với hệ trục $Oxyz$ thích hợp, mặt đất là mặt phẳng Oxy, đơn vị trên mỗi trục là mét, một con chim bói cá bay xuống đớp con mồi rồi bay lên khỏi mặt nước với một đường cong parabol đẹp mắt trước khi đậu vào một nhành cây. Biết rằng đường parabol này đi qua các điểm $ A\left(1;2;2\right)$, $B\left(1;3;8\right)$, $C\left(1;-1;10\right)$, vị trí chim bói cá đậu sau khi bắt được mồi có tung độ bằng 4,1 mét. Tính chiều cao của chim đang đậu so với mặt đất theo đơn vị mét và làm tròn đến hàng phần chục.
\shortans{19,6}
}
{
    \includegraphics[width=5cm]{img/HXN-12-20}
}
\loigiai{
    Ta nhận thấy cả ba điểm $A(1;2;2)$, $B(1;3;8)$, $C(1;-1;10)$ cùng thuộc mặt phẳng $x=1$.\\
    Do vậy đường parabol mà con chim bói cá này vẽ nên thuộc mặt phẳng $x=1$.\\
    Đặt $z=ay^2+by+c$, $\left(a\ne 0\right)$ là hàm số bậc hai mô phỏng parabol $(P)$ thuộc mặt phẳng $x=1$.\\
    Thay các cặp $(y;z)$ gồm $(2;2)$, $(3;8)$, $C(-1;10)$ vào hàm số trên ta được hệ phương trình \\
    \centerline{
        $\heva{& 4a+2b+c=2 \\& 9a+3b+c=8 \\& a-b+c=10 } \Leftrightarrow \heva{& a=\dfrac{13}{6} \\& b=-\dfrac{29}{6} \\& c=3 } $ hay $z=\dfrac{13}{6}y^2-\dfrac{29}{6}y+3$.
    }
    Vị trí chim bói cá đậu có $y=4{,}1\Rightarrow z=\dfrac{3\,921}{200}\approx 19{,}6\,m$.\\
    Vậy vị trí chim bói cá đậu cách mặt đất khoảng $19{,}6$ m.
}
\end{ex}
\begin{ex}%Câu 21
\immini
{
    Một cái bục bằng gỗ dùng để đặt đồ trang trí có mặt đáy trên và mặt đáy dưới đều là hình vuông, người thợ thiết kế cái bục này theo ba phần:\\
Phần trên cùng là một hình hộp chữ nhật có các kích thước là $1$ m; $1$ m; $0{,}05$ m.\\
Phần đế của bục cũng là hình hộp chữ nhật có các kích thước là $\sqrt{2}m$; $\sqrt{2}m$; $0{,}2m$.
}
{
    \includegraphics[width=5cm]{img/HXN-12-21}
}
Phần thân (giữ của bục có mặt cắt theo hai đường chéo của đáy trên và đáy dưới là đường hypebol mà các đường tiệm cận của hypebol này tạo với trục đứng một góc bằng $30^{\circ}$.\\
Biết rằng mặt cắt của bục song song với hai đáy tại vị trí có kích thước hình vuông bé nhất bằng $0{,}5$ m. Tìm thể tích của cái bục đã cho theo đơn vị mét khối và làm tròn đến hàng phần trăm.
\shortans{2,33}
\loigiai{
\immini
{
Đặt hệ trục tọa độ như hình vẽ, gọi phương trình chính tắc hypebol $(H)$ là $\dfrac{x^2}{a^2}-\dfrac{y^2}{b^2}=1$, $\left(a>0,\,b>0\right)$.\\
Đường chéo hình vuông nhỏ nhất là $2a=0{,}5\sqrt{2}\Rightarrow a=\dfrac{\sqrt{2}}{4}$.\\
Ta có $\tan 30^{\circ }=\dfrac{a}{b}=\dfrac{\dfrac{\sqrt{2}}{4}}{b}\Rightarrow b=\dfrac{\sqrt{6}}{4}$.\\
Do vậy $(H)\colon \dfrac{x^2}{1/8}-\dfrac{y^2}{3/8}=1$ hay $8x^2=1+\dfrac{8y^2}{3}\Rightarrow x=\pm \sqrt{\dfrac{1}{8}+\dfrac{y^2}{3}}$.\\
Xét mặt cắt vuông góc với $Oy$ của cái bục tại vị trí có hoành độ $x>0$ thì ta thu được hình vuông có cạnh $\dfrac{2x}{\sqrt{2}}=x\sqrt{2}=\sqrt{\dfrac{1}{4}+\dfrac{2y^2}{3}}$.
}
{
    \includegraphics[width=4.6cm]{img/HXN-12-21-LG}
}
Với $x=\dfrac{1\cdot \sqrt{2}}{2}$ (xét hình vuông mặt trên cùng của cái bục); ta có $y=\dfrac{3\sqrt{2}}{4}>0$.\\
Với $x=\dfrac{\sqrt{2}\cdot \sqrt{2}}{2}=1$ (xét hình vuông mặt đáy dưới của cái bục); ta có $y=-\dfrac{\sqrt{42}}{4}<0$.\\
Thể tích cái bục là $V=1\cdot 1\cdot 0{,}05+\sqrt{2}\cdot \sqrt{2}\cdot 0{,}2+\int\limits_{\tfrac{-\sqrt{42}}{4}}^{\tfrac{3\sqrt{2}}{4}}{\left(\dfrac{1}{4}+\dfrac{2y^2}{3}\right)dy}\approx 2{,}33\,m^3$.
}
\end{ex}
\begin{ex}%Câu 22
\immini
{
Lớp 12A có tất cả $30$ học sinh, trong đó bạn An muốn làm lớp trưởng, bạn Bảo muốn làm lớp phó học tập và bạn Châu muốn làm bí thư (các học sinh của lớp không trùng tên). Tính xác suất để cô giáo chọn được học sinh của lớp cho ba chức vụ như trên, mỗi học sinh giữ một chức vụ, đồng thời không có ai trong các bạn An, Bảo, Châu được giữ chức vụ mình thích (làm tròn đến hàng phần chục)?
\shortans{0,9}
}
{
    \includegraphics[width=5cm]{img/HXN-12-22}
}
\loigiai{
    Gọi $A$ là tập hợp các cách chọn chức vụ sao cho An được làm lớp trưởng.\\
    Gọi $B$ là tập hợp các cách chọn chức vụ sao cho Bảo được làm lớp phó học tập.\\
    Gọi $C$ là tập hợp các cách chọn chức vụ sao cho Châu được làm bí thư.\\
    Số cách chọn chức vụ tùy ý: $n\left(\Omega \right)=30\cdot 29\cdot 28=24\,360$.\\
    Ta có $\mid A\cup B\cup C\mid =\mid A\mid +\mid B\mid +\mid C\mid -\mid A\cap B\mid -\mid B\cap C\mid -\mid C\cap A\mid +\mid A\cap B\cap C\mid $.\\
    Trong đó $\mid A\mid =\mid B\mid =\mid C\mid =1\cdot 29\cdot 28$ (chọn một học sinh tương ứng cho chức vụ mình thích, hai chức còn lại thì chọn tùy ý).\\
    Tương tự $\mid A\cap B\mid =\mid B\cap C\mid =\mid C\cap A\mid =1\cdot 1\cdot 28$ (chọn hai học sinh giữ chức vụ minh thích, chức còn lại thì chọn tùy ý).\\ 
    Ta có $\mid A\cap B\cap C\mid =1\cdot 1\cdot 1$ (cả ba học sinh giữ chức vụ mình thích).\\
    Vậy tổng số cách chọn thỏa mãn đề bài là: $n(X)=30\cdot 29\cdot 28-\left(3\cdot 29\cdot 28-3\cdot 28+1\right)=22\,007$.\\
    Xác suất cần tính là $P(X)=\dfrac{n(X)}{n\left(\Omega \right)}=\dfrac{22\,007}{24\,360}\approx 0{,}9$.
}
\end{ex}
\Closesolutionfile{ans}
\inputansbox{6,4,3}{ans/ans-HXN-\sode-T,ans/ans-HXN-\sode-TF,ans/ans-HXN-\sode-SA}
% %%%%%%%%%%%%%%%%%%%- HXN
\def\sode{13}
\def\tendethi{ĐỀ PHÁT TRIỂN MINH HOẠ 2025}
\begin{dethi}
 {\tendethi}
\end{dethi}
\caulc
\Opensolutionfile{ans}[ans/ans-HXN-\sode-T]
\begin{ex}%Câu 1
 Cho hàm số $ f(x)$ có bảng xét dấu của đạo hàm như sau\\
 \centerline{
 \begin{tikzpicture}[>=stealth]
 \tkzTabInit[nocadre=false,lgt=1.2,espcl=2.5,deltacl=.5]
 {$x$/.7, $f'(x)$/1}
 {$-\infty$,$-1$,$0$,$3$,$+\infty$}
 \tkzTabLine{ , + , $0$ , - ,$0$,+,$0$,-, }
 \end{tikzpicture}
 }
 Hàm số đã cho nghịch biến trên khoảng nào dưới đây?
 \choice
 {$\left(-1;3\right)$}
 {$\left(-\infty;-1\right)$}
 {\True $\left(-1;0\right)$}
 {$\left(0;+\infty\right)$}
\end{ex}
\begin{ex}%Câu 2
 Chỉ số ô nhiễm không khí (AQI) tại thủ đô Hà Nội trong tháng 6/2024 được thống kê vào lúc 10h30 sáng các ngày trong tháng thể hiện theo bảng số liệu sau:\\
 \centerline{\begin{tblr}{|c|c|c|c|c|c|}
 \hline
 Chỉ số (AQI) & $[130;145)$ & $[145;160)$ & $[160;175)$ & $[175;190)$ & $[190;205)$\\
 \hline
 Số ngày & 8 & 7 & 6 & 7 & 2\\
 \hline
 \end{tblr}}\\
 Tứ phân vị thứ ba của mẫu số liệu trên gần nhất với giá trị nào sau đây?
 \choice
 {$ 175$}
 {$ 176{,}5$}
 {$ 180{,}2$}
 {\True $ 178{,}2$}
\end{ex}
\begin{ex}%Câu 3
 Trong không gian $ Oxyz$, đường thẳng đi qua điểm $ M\left(1;3;-2\right)$ và nhận vectơ $\vec{u}=\left(1;-1;5\right)$ làm vectơ chỉ phương có phương trình tham số là
 \choice
 {$\heva{& x=1+t\\& y=-1+3t\\& z=5-2t}$}
 {$\heva{& x=1+t\\& y=-1+3t\\& z=5+2t}$}
 {\True $\heva{& x=1+t\\& y=3-t\\& z=-2+5t}$}
 {$\heva{& x=1+t\\& y=3+t\\& z=-2+5t}$}
\end{ex}
\begin{ex}%Câu 4
 Khẳng định nào dưới đây là đúng?
\choice
{$\int{x^3\mathrm{\,d}x=x^4}+C$}
{$\int{x^3\mathrm{\,d}x=3x^2}+C$}
{$\int{x^3\mathrm{\,d}x=\dfrac{x^3}{\ln3}}+C$}
{\True $\int{x^3\mathrm{\,d}x=\dfrac{x^4}{4}}+C$}
\end{ex}
\begin{ex}%Câu 5
 Quãng đường đi bộ mỗi ngày (đơn vị: $km$) của Tom trong $20$ ngày gần nhất được thống kê lại ở bảng sau:\\
 \centerline{\begin{tblr}{|c|c|c|c|c|c|}
 \hline
 Quãng đường (km) & $[2,7 ; 3,0)$ & $[3,0 ; 3,3)$ & $[3,3 ; 3,6)$ & $[3,6 ; 3,9)$ & $[3,9 ; 4,2)$\\
 \hline
 Số ngày & 3 & 6 & 5 & 4 & 2\\
 \hline
 \end{tblr}}\\
 Khoảng biến thiên của mẫu số liệu ghép nhóm là
 \choice
 {\True $ 1{,}5$km}
 {$ 0{,}9$km}
 {$ 0{,}6$km}
 {$ 0{,}3$km}
\end{ex}
\begin{ex}%Câu 6
 Tập nghiệm của bất phương trình $\log_{\frac{2}{3}}x>2$ là
 \choice
 {\True $\left(0;\dfrac{4}{9}\right)$}
 {$\left(-\infty;\sqrt[3]{4}\right)$}
 {$\left(\sqrt[3]{4};+\infty\right)$}
 {$\left(-\infty;\dfrac{4}{9}\right)$}
\end{ex}
\begin{ex}%Câu 7
 Trong không gian $ Oxyz$, hình chiếu vuông góc của điểm $ M\left(3;1;-1\right)$ trên trục $ Oy$ có tọa độ là 
 \choice
 {$\left(3;0;-1\right)$}
 {\True $\left(0;1;0\right)$}
 {$\left(3;0;0\right)$}
 {$\left(0;0;-1\right)$}
\end{ex}
\begin{ex}%Câu 8
 Tiệm cận ngang của đồ thị hàm số $ y=\dfrac{2}{x-1}$ là đường thẳng:
 \choice
 {$y=2$}
 {$x=1$}
 {$y=1$}
 {\True $y=0$}
\end{ex}
\begin{ex}%Câu 9
 Cho cấp số cộng $\left(u_n\right)$ có $u_2=2$, $u_5=11$. Công sai $d$ của cấp số cộng là
 \choice
 {1}
 {2}
 {\True 3}
 {4}
\end{ex}
\begin{ex}%Câu 10
\immini
{
 Cho hàm số $ f(x)$ liên tục trên $\left[-1;5\right]$ và có đồ thị trên đoạn $\left[-1;5\right]$ như hình vẽ bên dưới. Tổng giá trị lớn nhất và giá trị nhỏ nhất của hàm số $ f(x)$ trên đoạn $\left[-1;5\right]$ bằng
 \choice
 {$-1$}
 {$ 4$}
 {\True $ 1$}
 {$ 2$}
}
{
 \begin{tikzpicture}[line cap=round, line join=round, scale=0.5,font=\scriptsize,>=stealth,thick]
 \draw[->](-2,0)--(6,0)node[below]{$x$};
 \draw[->](0,-3.5)--(0,3)node[left]{$y$};
 \draw
 (-1,-3)..controls++(80:1) and++(180:0.5)..(0,1)..controls++(0:0.5) and++(180:0.75)..(2,-3)..controls++(0:0.5) and++(-100:1)..(3,0)..controls++(80:0.35) and++(180:0.5)..(4,2)..controls++(0:0.5) and++(125:0.25)..(5,1);
 \foreach \x/\y/\m/\g in{-1/0/-1/90,0/-3/-3/-145,2/0/2/90,3/0/3/-70,4/0/4/-90,5/0/5/-90,0/2/2/180}
 \draw[fill=black](\x,\y)circle(1pt)node[shift={(\g:0.25)}]{$\m$};
 \draw[dashed](-1,0)--(-1,-3)--(0,-3)--(2,-3)--(2,0)(0,1)--(5,1)--(5,0)(3,0)--(3,1)(4,0)--(4,2)--(0,2);
 \end{tikzpicture}
}
\end{ex}
\begin{ex}%Câu 11
 Cho hình hộp $ABCD.A'B'C'D'$. Gọi $O$ là tâm của hình hộp, khẳng định nào dưới đây đúng?
 \choice
 {$\overrightarrow{OA}+\overrightarrow{O{A}'}=\overrightarrow{0}$}
 {\True $\overrightarrow{OA}+\overrightarrow{O{C}'}=\overrightarrow{0}$}
 {$\overrightarrow{OA}+\overrightarrow{OB}=\overrightarrow{0}$}
 {$\overrightarrow{OA}+\overrightarrow{OD}=\overrightarrow{0}$}
\end{ex}
\begin{ex}%Câu 12
 Với $ a,b$ là các tham số thực thì giá trị tích phân $ I=\int\limits_0^b{\left(3x^2-2ax-1\right)\text{d}x}$ bằng
 \choice
 {$b^3-b{a^2}-b$}
 {$b^3+b^2a+b$}
 {\True $b^3-b^2a-b$}
 {$ 3b^2-2ab-1$}
 \end{ex}
\Closesolutionfile{ans}
\cauds
\Opensolutionfile{ans}[ans/ans-HXN-\sode-TF]
\begin{ex}%Câu 13
Nếu đứng trước biển và nhìn ra xa, người ta sẽ thấy một đường giao giữa mặt biển và bầu trời, đó là đường chân trời đối với người quan sát. Ta có thể hình dung rằng nếu người quan sát ở tại đỉnh của một chiếc nón và trái đất được \lq\lq thả\rq\rq vào trong chiếc nón ấy thì đường chân trời là đường \lq\lq chạm\rq\rq giữa trái đất và chiếc nón.\\
\centerline{
 \includegraphics[width=7cm]{img/HXN-13-13}
}
Trong không gian $Oxyz$, giả sử bề mặt trái đất $(S)$ có phương trình $x^2+y^2+z^2=1$ và người quan sát ở vị trí $ B\left(1;1;-1\right)$; $A$ là một vị trí bất kì trên đường chân trời đối với người quan sát ở vị trí $B$.
 \choiceTF
 {\True Khoảng cách từ vị trí $B$ đến tâm của trái đất là $\sqrt{3}$}
 {\True Khoảng cách hai điểm $A$, $B$ là $\sqrt{2}$}
 {Phương trình mặt cầu đường kính $OB$ là $\left(x-\dfrac{1}{2}\right)^2+\left(y-\dfrac{1}{2}\right)^2+\left(z-\dfrac{1}{2}\right)^2=\dfrac{3}{4}$}
 {\True Điểm $A$ luôn thuộc mặt phẳng cố định $ x+y-z-1=0$}
 \loigiai{
     \begin{itemchoice}
         \itemch Tâm của trái đất là điểm $O(0;0;0)$, bán kính $R=1$.\\
         Ta có $OB=\sqrt{(1-0)^2+(1-0)^2+\left(-1-0\right)^2}=\sqrt{3}$.
         \itemch Vì $\triangle OAB$ vuông tại $A$ nên $AB=\sqrt{OB^2-R^2}=\sqrt{3-1}=\sqrt{2}$.
         \itemch Gọi $I$ là trung điểm OB thì $I\left(\dfrac{1}{2};\dfrac{1}{2};-\dfrac{1}{2}\right)$; mặt cầu đường kính OB có tâm $I$,\\
         bán kính $\dfrac{OB}{2}=\dfrac{\sqrt{3}}{2}$ nên có phương trình $\left(S'\right)\colon \left(x-\dfrac{1}{2}\right)^2+\left(y-\dfrac{1}{2}\right)^2+\left(z+\dfrac{1}{2}\right)^2=\dfrac{3}{4}$.
         \itemch Ta thấy $A$ luôn thuộc đường tròn giao tuyến của hai mặt cầu $(S)\,,\left({S'}\right)$.
         Xét hệ phương trình $\heva{& x^2+y^2+z^2=1 \\& \left(x-\dfrac{1}{2}\right)^2+\left(y-\dfrac{1}{2}\right)^2+\left(z+\dfrac{1}{2}\right)^2=\dfrac{3}{4} }  \Leftrightarrow \heva{& x^2+y^2+z^2=1 \\& x^2+y^2+z^2-x-y+z=0 }  \Leftrightarrow \heva{& x^2+y^2+z^2=1 \\& 1-x-y+z=0 } \Leftrightarrow \heva{& x^2+y^2+z^2=1 \\& x+y-z-1=0 }$.\\
         Vậy điểm $A$ luôn thuộc mặt phẳng cố định $x+y-z-1=0$.
     \end{itemchoice}
 }
\end{ex}
\begin{ex}%Câu 14
Một màn ảnh hình chữ nhật cao $1{,}4$ m được đặt ở độ cao $1{,}8$ m so với tầm mắt (tính từ đầu mép dưới của màn hình). Một người đang xem phim có mắt đặt ở vị trí O và quan sát màn ảnh với góc nhìn $\widehat{BOC}$. Với các điểm như trong hình vẽ, đặt $x=OA$, $ x>0$.
\choiceTF
 {\True $\tan\widehat{BOC}=\dfrac{\tan\widehat{AOC}-\tan\widehat{AOB}}{1+\tan\widehat{AOC}\cdot\tan\widehat{AOB}}$}
 {\True $\tan\widehat{AOC}=\dfrac{3{,}2}{x}$, $\tan\widehat{AOB}=\dfrac{1{,}8}{x}$}
 {Nếu góc nhìn màn hình của mắt người là $15^{\circ}$ thì người đó đang ngồi cách bức tường (nơi gắn màn hình) một khoảng gần nhất bằng $3{,}6$ mét (làm tròn đến hàng phần chụ}
 {Người xem muốn nhìn rõ màn ảnh nhất (góc nhìn lớn nhất) thì người đó phải đứng cách mặt phẳng chứa màn ảnh một khoảng $2{,}2$ mét}
 \loigiai{
     \begin{itemchoice}
         \itemch Ta có $\widehat{BOC}=\widehat{AOC}-\widehat{AOB}\Rightarrow \tan\widehat{BOC}=\tan\left(\widehat{AOC}-\widehat{AOB}\right)=\dfrac{\tan\widehat{AOC}-\tan\widehat{AOB}}{1+\tan\widehat{AOC}\cdot \tan\widehat{AOB}}$.
         \itemch Xét lần lượt các tam giác vuông $OAB$, $OAC$ (vuông tại $O$), ta có
         $\tan\widehat{AOC}=\dfrac{AC}{AO}=\dfrac{3{,}2}{x}$, $\tan\widehat{AOB}=\dfrac{AB}{AO}=\dfrac{1{,}8}{x}$.
         \itemch Ta có: $\tan\widehat{BOC}=\dfrac{\dfrac{3{,}2}{x}-\dfrac{1{,}8}{x}}{1+\dfrac{3{,}2}{x}\cdot \dfrac{1{,}8}{x}}=\dfrac{\dfrac{1{,}4}{x}}{\dfrac{x^2+5{,}76}{x^2}}=\dfrac{1{,}4x}{x^2+5{,}76}$.\\
         Khi $\widehat{BOC}=15^{\circ }$ thì $$\tan 15^{\circ}=\dfrac{1{,}4x}{x^2+5{,}76}\Rightarrow \tan 15^{\circ}\cdot x^2-1{,}4x+5{,}76\cdot \tan 15^{\circ}=0\Rightarrow \hoac{& x\approx 3{,}6\,m \\& x\approx 1{,}6\,m.} $$
         Ta chọn $x\approx 1{,}6<3{,}6$.\\
         Vì vậy người xem ngồi cách tường một khoảng gần nhất xấp xỉ $1{,}6$ mét.
         \itemch Ta cần tìm giá trị lớn nhất của hàm $f(x)=\dfrac{1{,}4x}{x^2+5{,}76}$, $x>0$.\\
         Đạo hàm $f'(x)=\dfrac{-1{,}4x^2+8{,}064}{\left(x^2+5{,}76\right)^2}$; $f'(x)=0\Leftrightarrow x^2=5{,}76\Rightarrow x=2{,}4$.\\
         Bảng biến thiên:\\
         \begin{tikzpicture}[>=stealth]
             \tkzTabInit[nocadre=false,lgt=1.2,espcl=2.5,deltacl=0.5]{$x$/.7 ,$f'(x)$/.7,$f(x)$/2}
             {$-\infty$ , $2{,}4$ , $+\infty$}
             \tkzTabLine{ , + , $0$ , - , }
             \tkzTabVar{-/, +/$\dfrac{7}{24}$ , -/}
         \end{tikzpicture}
         
         Vậy để góc nhìn lớn nhất $\left(\widehat{BOC}_{\max}\right)$ thì $\tan\widehat{BOC}$ đạt giá trị lớn nhất
         $\Leftrightarrow f(x)$ đạt giá trị lớn nhất.\\
         Dựa vào bảng biến thiên, ta thấy $\max\limits_{\left(0;+\infty \right)} f(x)=\dfrac{7}{24}$, khi đó $x=2{,}4$ (mét).
     \end{itemchoice}
 }
\end{ex}
\begin{ex}%Câu 15
Một chậu nước có dạng một khối tròn xoay với thiết diện qua trục của chậu (mặt cắt đi qua hai tâm của hai đường tròn đáy) là hai đường parabol đối xứng nhau qua trục đó. Biết hai đường tròn đáy chậu cùng có bán kính $0{,}5$ m; thiết diện nhỏ nhất vuông góc với trục của chậu có bán kính $0{,}2$ m; chiều cao của chậu nước bằng $1{,}5$ m. Người ta bơm nước vào chậu với tốc độ $5$ lít/phút.\\
Xét hệ trục tọa độ $Oxy$ với gốc $O$ trùng với tâm đường tròn đáy của chậu nước, tia $Ox$ chứa trục của chậu nước (đơn vị trên mỗi trục là mét). Mặt cắt qua trục của chậu nước cho ta hai nhánh parabol như hình vẽ, gọi $ y=f(x)$ là parabol nằm trên trục hoành.
\begin{center}
    \includegraphics[width=5cm]{img/HXN-13-15-a}\includegraphics[height=5cm]{img/HXN-13-15-b}
\end{center}
\choiceTF
 {\True $ f(x)=\dfrac{8}{15}{x^2}-\dfrac{4}{5}x+\dfrac{1}{2}$}
 {\True Sức chứa tối đa của chậu nước bằng $0{,}5$ m$^3$ (làm tròn đến hàng phần chục của mét khối)}
 {\True Sau $1{,}5$ giờ bơm nước (làm tròn đến hàng phần chục của giờ) thì chậu đầy nước}
 {\True Nếu bơm từ đầu như thế thì đến phút thứ $20$, tốc độ dâng lên của nước bằng $0{,}01$ m/phút}
 \loigiai{
     \begin{center}
         \includegraphics[height=5cm]{img/HXN-13-15-LG}
     \end{center}
     \begin{itemchoice}
         \itemch  Parabol $f(x)=ax^2+bx+c$, $\left(a\ne 0\right)$ đi qua các điểm $\left(0;0{,}5\right)$, $\left(1{,}5;0{,}5\right)$, $\left(0{,}75;0{,}2\right)$ nên ta có hệ phương trình $\heva{& c=0{,}5 \\& 1{,}5^2a+1{,}5b+0{,}5=0{,}5 \\& 0{,}75^2a+0{,}75b+0{,}5=0{,}2 } \Rightarrow \heva{& a=\dfrac{8}{15} \\& b=-\dfrac{4}{5} \\& c=\dfrac{1}{2}.} $\\
         Do vậy $f(x)=\dfrac{8}{15}x^2-\dfrac{4}{5}x+\dfrac{1}{2}$.
         \itemch Thiết diện của chậu nước vuông góc với trục $Ox$ tại vị trí có hoành độ $x$ chính là đường tròn có bán kính $r=\dfrac{8}{15}x^2-\dfrac{4}{5}x+\dfrac{1}{2}$ nên có diện tích $S(x)=\pi r^2=\pi \left(\dfrac{8}{15}x^2-\dfrac{4}{5}x+\dfrac{1}{2}\right)^2$.\\
         Thể tích chậu nước là $V_{full}=\int\limits_0^{1{,}5}{S(x)\mathrm{\,d}x}=\int\limits_0^{1{,}5}\pi {\left(\dfrac{8}{15}x^2-\dfrac{4}{5}x+\dfrac{1}{2}\right)^2\mathrm{\,d}x}\approx 0{,}5\,m^3$ (lưu vào A).
         \itemch Đổi đơn vị $5$ lít/phút = $\dfrac{5\colon 1000}{1\colon 60}=\dfrac{3}{10}$ $m^3$/giờ.\\
         Thời gian bơm nước đầy chậu là $\dfrac{V_{full}}{3/10}=\dfrac{49}{100}\pi \approx 1{,}5$ giờ.   [Nhấn máy: A chia $3/10$].
         \itemch Từ câu b) ta thấy hoành độ $x$ $\left(x>0\right)$ cũng chính là chiều cao mực nước trong chậu.\\
         Thể tích chậu nước ứng với chiều cao $x$ là $V=\int\limits_0^x{S(x)\mathrm{\,d}x}$ \tagEX{1}
         Sau $20$ phút bơm nước thì thể tích nước trong chậu là $V_{20}=5\times 20=100$ lít = $0{,}1$ $m^3$/phút.\\
         Xét $V=\int\limits_0^x{S(x)\mathrm{\,d}x}=0{,}1\Rightarrow \int\limits_0^x{\pi \left(\dfrac{8}{15}x^2-\dfrac{4}{5}x+\dfrac{1}{2}\right)^2\mathrm{\,d}x}=0{,}1\Rightarrow x\approx 0{,}164\,m$ (lưu vào B).\\
         Đạo hàm hai vế của $(1)$ theo $t$, ta được\\
         \centerline{$\dfrac{dV}{\mathrm{\,d}t}=S(x)\cdot \dfrac{\mathrm{\,d}x}{\mathrm{\,d}t}\Leftrightarrow \dfrac{dV}{\mathrm{\,d}t}=\pi \left(\dfrac{8}{15}x^2-\dfrac{4}{5}x+\dfrac{1}{2}\right)^2\cdot \dfrac{\mathrm{\,d}x}{\mathrm{\,d}t}$}
         Thay $\dfrac{dV}{\mathrm{\,d}t}=\dfrac{1}{200}\,\,m^3$/phút; $x=B\approx 0{,}164\,m$, ta được $\dfrac{\mathrm{\,d}x}{\mathrm{\,d}t}\approx 0{,}01$ m/phút.
     \end{itemchoice}
 }
\end{ex}
\begin{ex}%Câu 16
\immini
{
    Trong một buổi triển lãm công nghệ quốc tế, công ty A dự định ra mắt sản phẩm mới với sự tham gia của CEO nổi tiếng. Tuy nhiên, có nhiều yếu tố rủi ro có thể khiến sự kiện bị hủy
\begin{itemize}
 \item Nếu lượng khách đăng ký trước vượt 1000 người, sự kiện chắc chắn diễn ra.
 \item Nếu hệ thống máy chủ gặp sự cố (do tấn công mạng hoặc quá tải), công ty không thể xử lý hết số lượng khách đăng ký, khi đó xác suất hủy sự kiện là $60\%$.
 \item Nếu hệ thống hoạt động bình thường, xác suất đủ 1000 đăng ký là $80\%$; nếu không đủ, xác suất hủy sự kiện là $10\%$.
\end{itemize}
}
{
\includegraphics[width=6cm]{img/HXN-13-16}
}
Biết rằng xác suất để sự kiện bị hủy với bất kì lý do gì là $0{,}35$.\\
Gọi các biến cố $A$: \lq\lq Hệ thống gặp sự cố\rq\rq; $B$: \lq\lq Đủ $1000$ đăng ký\rq\rq; $C$: \lq\lq Sự kiện bị hủy\rq\rq.
\choiceTF
 {\True $ P\left(B|A\right)=0$ và $ P\left(\bar{B}|A\right)=1$}
 {\True $ P\left(C|\bar{A}\bar{B}\right)=0{,}1$}
 {Xác suất để hệ thống máy chủ gặp sự cố bằng $0{,}57$ (làm tròn đến hàng phần trăm)}
 {\True Cuối cũng thì những lo lắng cũng đã thành hiện thực, sự kiện quan trọng đã bị hủy, xác suất để hệ thống máy chủ hoạt động bình thường băng $0{,}02$ (làm tròn đến hàng phần trăm)}
 \loigiai{
     \begin{center}
         \includegraphics[height=6cm]{img/HXN-13-16-LG}
     \end{center}
     \begin{itemchoice}
         \itemch Nếu hệ thống gặp sự cố thì số lượng khách hàng đăng ký trước không thể đủ 1000, do đó  $P\left(B|A\right)=0$ và $P\left(\bar{B}|A\right)=1$.
         \itemch Nếu hệ thống máy chủ không gặp sự cố và số lượng đăng ký trước của khách hàng dưới 1000 thì xác suất sự kiện bị hủy bằng $10\%$, do đó $P\left(C|\bar{A}\bar{B}\right)=0{,}1$.
         \itemch Đặt $x=P(A)\in (0;1)$; suy ra $P\left({\bar{A}}\right)=1-x$.\\
         Ta có $P(C)=x\cdot 1\cdot 0{,}6+(1-x)\cdot 0{,}2\cdot 0{,}1=0{,}35\Rightarrow x=\dfrac{33}{58}$ hay $P(A)=\dfrac{33}{58}\approx 0{,}57$.
         \itemch Ta có $P\left(\bar{A}|C\right)=\dfrac{P\left(\bar{A}C\right)}{P(C)}=\dfrac{\left(1-\dfrac{33}{58}\right)\cdot 0{,}2\cdot 0{,}1}{0{,}35}=\dfrac{5}{203}\approx 0{,}02$.
     \end{itemchoice}
 }
\end{ex}
\Closesolutionfile{ans}
\caukq
\Opensolutionfile{ans}[ans/ans-HXN-\sode-SA]
\begin{ex}%Câu 17
\immini
{
     Sau cơn mưa, có 4 cậu bé muốn đi qua một con đê trơn trợt nhưng họ chỉ có hai đôi dép.
 \begin{itemize}
 \item Cậu bé Văn có thể đi qua con đê trong 5 phút.
 \item Cậu bé Võ có thể đi qua con đê trong 9 phút.
 \item Cậu bé Song có thể đi qua con đê trong 13 phút.
 \item Cậu bé Toàn có thể đi qua con đê trong 3 phút.
 \end{itemize}
}
{
    \includegraphics[width=6cm]{img/HXN-13-17}
}
Hỏi thời gian tối thiểu để cả $4$ cậu bé cùng qua được con đê là bao nhiêu phút? Biết rằng mỗi cậu bé muốn đi qua con đê này thì phải mang dép và thời gian để mỗi người đi qua hoặc đi về lại trên con đê là như nhau.
\shortans{31}
\loigiai{
Để tối ưu hóa thời gian qua lại trên con đê thì $2$ cậu bé nhanh nhất (Văn và Toàn) nên đi với nhau, $2$ cậu bé chậm hơn (Võ và Song) nên đi với nhau.\\
Họ phải đi như sau để qua được con đê một cách nhanh nhất:
\begin{itemize}
    \item Văn và Toàn đi cùng nhau $\longrightarrow$ mất 5 phút.
    \item Toàn cầm đôi dép của Văn quay về lại đón bạn $\longrightarrow$ mất 3 phút.
    \item Võ và Song đi cùng nhau $\longrightarrow$ mất 13 phút.
    \item Văn cầm 1 đôi dép quay lại đón bạn $\longrightarrow$ mất 5 phút.
    \item Văn và Toàn đi cùng nhau lần nữa $\longrightarrow$ mất 5 phút.
\end{itemize}
Vậy tổng thời gian ngắn nhất để cả 4 cậu bé qua được con đê là là 31 phút.
}
 \end{ex}
 \begin{ex}%Câu 18
\immini
{
    Một cửa hàng bán lẻ bán được $ 2500$ cái tivi mỗi năm. Để bán được số tivi đó, họ phải đặt hàng từ nhà máy sản xuất tivi nhiều lần trong năm, mỗi lần đặt hàng với số lượng tivi như nhau. Mỗi lần lấy hàng từ nhà máy về thì cửa hàng chỉ trưng bày được một nửa số tivi đó, một nửa số hàng còn lại phải lưu vào kho; chi phí gửi trong kho là $ 10\$$ cho một cái tivi. Chi phí cố định cho mỗi lần đặt hàng là $ 20\$$, ngoài ra cửa hàng phải trả thêm $ 9\$$ cho mỗi tivi. Hỏi mỗi lần đặt hàng trong năm thì cửa hàng cần đặt bao nhiêu tivi để chi phí mà cửa hàng phải trả là nhỏ nhất?
}
{
    \includegraphics[width=6cm]{img/HXN-13-18}
}
\shortans{100}
\loigiai{
Gọi $x$ là số tivi mà cửa hàng đặt mỗi lần $\left(x\in \mathbb{N},\, 1\le x\le 2\,500\right)$.\\
Số tivi lưu vào kho mỗi lần là $\dfrac{x}{2}$; do đó chi phí lưu vào kho là $10\cdot \dfrac{x}{2}=5x$.\\
Số lần đặt hàng trong năm là $\dfrac{2\,500}{x}$ và chi phí đặt hàng là  $\dfrac{2\,500}{x}\left( 20+9x \right)$.\\
Tổng số chi phí mà cửa hàng phải trả là: $\dfrac{2\,500}{x}(20+9x)+5x=5x+\dfrac{50\,000}{x}+22\,500$.\\ 
Áp dụng bất đẳng thức Cauchy ta có  $5x+\dfrac{50\,000}{x}\ge 1\,000$.\\ 
Dấu bằng xảy ra khi $5x=\dfrac{50\,000}{x}\Rightarrow  x=100$.\\  
Vậy mỗi năm cửa hàng cần đặt hàng $\dfrac{2\,500}{100}=25$ lần, mỗi lần $100$ cái.
}
\end{ex}
\begin{ex}%Câu 19
Trong không gian với hệ tọa độ $Oxyz$ cho điểm $A\left(3;2;-1\right)$ và đường thẳng $d\colon \heva{& x=t\\& y=t\\& z=1+t.}$ Tính khoảng cách từ gốc tọa độ đến mặt phẳng $(P)$ biết rằng $(P)$ chứa $d$ sao cho khoảng cách từ $A$ đến $(P)$ là lớn nhất. Làm tròn kết quả đến hàng phần chục.
\shortans{0,8}
\loigiai{
\immini
{
Đường thẳng $d$ qua $M_0(0;0;1)$ có VTCP $\vec{u}=(1;1;1)$.\\
Gọi $H$, $K$ lần lượt là hình chiếu của $A$ lên $(P)$ và $d$ (khi đó $AK$ cố định). Ta có:
$d\left(A,(P)\right)=AH\le AK$.\\
Đẳng thức xảy ra khi và chỉ khi $H\equiv K$.\\
Do đó $d\left(A,(P)\right)_{\max }=AK$.\\
Khi đó $(P)$ đi $M_0(0;0;1)$ nhận $\vec{AK}$ làm vectơ pháp tuyến.
}
{
    \includegraphics[width=6cm]{img/HXN-13-19-LG}
}
Gọi $K\left(t;t;1+t\right)\in d \Rightarrow \vec{AK}=\left(t-3;t-2;t+2\right)$.\\
Ta có $\vec{AK}\perp \vec{u}\Leftrightarrow \vec{AK}\cdot \vec{u}=0 \Leftrightarrow 1\cdot (t-3)+1\cdot (t-2)+1\cdot (t+2)=0\Leftrightarrow t=1$.\\
Suy ra $\vec{AK}=(-2;-1;3)$.\\
Vậy phương trình $(P)\colon -2(x-0)-1\cdot (y-0)+3\cdot (z-1)=0 \Leftrightarrow 2x+y-3z+3=0$.\\
Khi đó: $d\left(O,(P)\right)=\dfrac{3}{\sqrt{4+1+9}}=\dfrac{3\sqrt{14}}{14}\approx 0{,}8$.
}
\end{ex}
\begin{ex}%Câu 20
\immini
{
    Hộp I có 5 quả bóng đỏ và 3 quả bóng trắng, hộp II có 2 quả bóng đỏ và 2 quả bóng trắng, hộp III có 3 quả bóng đỏ và 1 quả bóng trắng. Người ta thực hiện các hành động sau một cách ngẫu nhiên:
}
{
    \includegraphics[width=6cm]{img/HXN-13-20}
}
\begin{itemize}
    \item Lấy 1 quả bóng từ hộp I bỏ sang hộp II.
    \item Lấy 1 quả bóng từ hộp III bỏ sang hộp II.
    \item Lấy ra mỗi hộp 1 quả bóng thì nhận thấy đó là 3 quả bóng trắng.
\end{itemize}
Tính xác suất để cả 3 quả bóng được lấy từ ba hộp vốn là của hộp I và hộp III (ban đầu).\\
Kết quả được làm tròn đến hàng phần trăm.
\shortans{0,13}
\loigiai{
    \begin{center}
        \includegraphics[width=10cm]{img/HXN-13-20-LG}
    \end{center}
    Gọi A là biến cố: \lq\lq Lấy được $3$ quả bóng trắng từ ba hộp\rq\rq và B là biến cố: \lq\lq Cả $3$ quả bóng được lấy từ ba hộp vốn là của hộp I và hộp II\rq\rq.\\
    Để biến cố A xảy ra thì quả bóng chuyển từ hộp III qua hộp II phải là bóng đỏ (vì khi đó hộp III còn 1 quả bóng trắng để lấy ra).\\
    Ta tính xác suất của A trong hai trường hợp quả bóng được chuyển từ hộp I sang hộp II là đỏ hoặc trắng.\\
    Ta có $P(A)=\dfrac{3}{8}\cdot \dfrac{2}{7}\cdot \dfrac{3}{4}\cdot \dfrac{1}{3}\cdot \dfrac{1}{2}+\dfrac{5}{8}\cdot \dfrac{3}{7}\cdot \dfrac{3}{4}\cdot \dfrac{1}{3}\cdot \dfrac{2}{6}=\dfrac{1}{28}$.\\
    Do đó: $P\left(B|A\right)=\dfrac{P(BA)}{P(A)}=\dfrac{\dfrac{3}{8}\cdot \dfrac{2}{7}\cdot \dfrac{3}{4}\cdot \dfrac{1}{3}\cdot \dfrac{1}{6}}{\dfrac{1}{28}}=\dfrac{1}{8}\approx 0{,}13$.
}
\end{ex}
\begin{ex}%Câu 21
\immini
{
    Trong phòng thí nghiệm vật lý, một chất điểm đặt ở vị trí $A$ của hình lập phương được tác động bởi ba lực $\overrightarrow{F_1}$, $\overrightarrow{F_2}$, $\overrightarrow{F_3}$ dọc theo hai cạnh và đường chéo lớn của hình lập phương đó (tham khảo hình vẽ). Biết độ lớn các lực trên hai cạnh bằng $2$ N và $3$ N, độ lớn lực dọc theo đường chéo lớn lập phương bằng $4$ N. Tính độ lớn hợp lực $\left|\overrightarrow{F_1}+\overrightarrow{F_2}+\overrightarrow{F_3}\right|$ theo đơn vị N, làm tròn đến hàng phần trăm.
\shortans{7,22}
}
{
    \includegraphics[width=6cm]{img/HXN-13-21}
}
\loigiai{
Xét $\left(\vec{F_1},\vec{F_2}\right)=90^\circ$; $\left(\vec{F_1},\vec{F_3}\right)=\widehat{BAC'}\Rightarrow \tan \left(\vec{F_1},\vec{F_3}\right)=\tan \widehat{BAC'}=\dfrac{BC'}{AB}=\dfrac{a\sqrt{2}}{a}=\sqrt{2}$\\
$ \Rightarrow \cos \widehat{BAC'}=\sqrt{\dfrac{1}{1+\tan ^2\alpha }}=\dfrac{\sqrt{3}}{3}$; $\left(\vec{F_2},\vec{F_3}\right)=\widehat{DAC'}=\widehat{BAC'}\Rightarrow \cos \left(\vec{F_2},\vec{F_3}\right)=\dfrac{\sqrt{3}}{3}$.\\
Ta có $\left| \vec{F_1}+\vec{F_2}+\vec{F_3} \right|^2=\left| \vec{F_1} \right|^2+\left| \vec{F_2} \right|^2+\left| \vec{F_3} \right|^2+2\vec{F_1}\cdot \vec{F_2}+2\vec{F_1}\cdot \vec{F_3}+2\vec{F_2}\cdot \vec{F_3}$; trong đó:
\begin{itemize}
    \item $2\vec{F_1}\cdot \vec{F_2}=0$ vì $\vec{F_1}\perp \vec{F_2}$.
    \item $2\vec{F_1}\cdot \vec{F_3}=2\left| \vec{F_1} \right|\cdot \left| \vec{F_3} \right|\cos \left(\vec{F_1},\vec{F_3}\right)=2.2\cdot 4\cdot \dfrac{\sqrt{3}}{3}=\dfrac{16\sqrt{3}}{3}$.
    \item $2\vec{F_2}\cdot \vec{F_3}=2\left| \vec{F_2} \right|\cdot \left| \vec{F_3} \right|\cos \left(\vec{F_2},\vec{F_3}\right)=2.3\cdot 4\cdot \dfrac{\sqrt{3}}{3}=8\sqrt{3}$.
\end{itemize}
Do vậy $\left| \vec{F_1}+\vec{F_2}+\vec{F_3} \right|^2=4+9+16+\dfrac{16\sqrt{3}}{3}+8\sqrt{3}=\dfrac{87+40\sqrt{3}}{3}$\\
$\Rightarrow \left| \vec{F_1}+\vec{F_2}+\vec{F_3} \right|=\sqrt{\dfrac{87+40\sqrt{3}}{3}}\approx 7{,}22$N.
}
\end{ex}
\begin{ex}%Câu 22
\immini
{
    Một chiếc trống có đường sinh là các nửa elip tương ứng với độ dài trục lớn $80$ cm, còn độ dài trục bé bằng $60$ cm ; hai đáy trống là các hình tròn có bán kính $60$ cm. Hỏi thể tích của chiếc trống là bao nhiêu dm$^3$ (làm tròn đến hàng đơn vị).
\shortans{345}
}
{
    \includegraphics[width=6cm]{img/HXN-13-22}
}
\loigiai{
    \begin{center}
        \includegraphics[width=6cm]{img/HXN-13-22-LG}
    \end{center}
Đặt hệ trục tọa độ $Oxy$ như hình vẽ, đơn vị trên mỗi trục là dm.
Giả sử elip $(E)$ có độ dài trục lớn $2a=8\Rightarrow a=4$; độ dài trục bé $2b=6\Rightarrow b=3$.\\
Do đó phương trìn $(E)\colon \dfrac{x^2}{16}+\dfrac{y^2}{9}=1\Rightarrow y^2=9\left(1-\dfrac{x^2}{16}\right)$.\\
Ta chọn nửa elip $(E)$ nằm dưới trục hoành là $y=-3\sqrt{1-\dfrac{x^2}{16}}$ $\left(y\le 0\right)$.\\
Để có được hàm số $y=f(x)$ (đường sinh của chiếc trống), ta cần tịnh tiến đồ thị hàm số $y=-3\sqrt{1-\dfrac{x^2}{16}}$ lên trên 6 đơn vị; khi đó $f(x)=6-3\sqrt{1-\dfrac{x^2}{16}}$.\\
Thể tích cái trống là $V=\pi \displaystyle\int\limits_{-4}^4{\left(f(x)\right)^2\mathrm{\,d}x}=\pi \displaystyle\int\limits_{-4}^4{\left(6-3\sqrt{1-\dfrac{x^2}{16}}\right)^2\mathrm{\,d}x}\approx  345\,dm^3$.
}
\end{ex}
\Closesolutionfile{ans}
\inputansbox{6,4,3}{ans/ans-HXN-\sode-T,ans/ans-HXN-\sode-TF,ans/ans-HXN-\sode-SA}
% \begin{name}
	{\tenchude}
	{\tendethi}
	{TRƯỜNG THPT CHUYÊN VĨNH PHÚC - VĨNH PHÚC}
	{\thoigian}
\end{name}

\caulc
% \Opensolutionfile{ans}[Ans/KSCL-THPT-ChuyenVinhPhuc-VinhPhuc-L1-NH24-25-TN]
\Opensolutionfile{ans}[Ans/KSCL-THPT-ChuyenVinhPhuc-VinhPhuc-L1-NH24-25]

%   \Opensolutionfile{ansbook}[Ansbook/KSCL-THPT-ChuyenVinhPhuc-VinhPhuc-L1-NH24-25-TN]%---Nên đặt tên theo bài
  \setcounter{ex}{0}
 %%%==============Cau_EX1==============%%%
\begin{ex}%[Dự án C THPTQG 2025]%[Vương Quốc Phong]%[2D1N1-5]
	\immini[thm]{
	Một doanh nghiệp sản xuất với số lượng là $x$ sản phẩm, $x\in \mathbb{N}$ và thu được lợi nhuận $f(x)$ được biểu thị bởi bảng biến thiên như sau. Hỏi doanh nghiệp sản xuất bao nhiêu sản phẩm trở đi thì lợi nhuận bắt đầu giảm?
	\choice
	{\True $201$}
	{$200$}
	{$101$}
	{$100$}
	}{\begin{tikzpicture}
			\tkzTabInit[lgt=1.2,espcl=2.5]
			{$x$/.7,$f'(x)$/.7,$f(x)$/1.8}
			{$0$,$200$,$+\infty$}
			\tkzTabLine{ ,+,z,-, }
			\tkzTabVar{-/, +/$100$,-/}
		\end{tikzpicture}}
	\loigiai{
		Dựa vào bảng biến thiên ta thấy lợi nhuận bắt đầu giảm từ sản phẩm $201$.
	}
\end{ex}
%%%==============HetCau_EX1==============%%%

%%%==============Cau_EX2==============%%%
\begin{ex}%[Dự án C THPTQG 2025]%[Vương Quốc Phong]%[2D1N3-1]
	\immini[thm]{
	Cho hàm số $y=f(x)$ có bảng biến thiên trên đoạn $[0;3]$ như sau.
	Giá trị nhỏ nhất của hàm số $y=f(x)$ trên đoạn $[0;3]$ là
	\choice
	{\True $-4$}
	{$1$}
	{$4$}
	{$0$}
	}{
		\begin{tikzpicture}
			\tkzTabInit[lgt=1.2,espcl=2.5]
			{$x$/.7,$y$/1.8}
			{$0$, $1$, $3$}
			\tkzTabVar{+/$-3$,-/$-4$,+/$0$}
		\end{tikzpicture}
	}
	\loigiai{
		Dựa vào bảng biến thiên ta thấy giá trị nhỏ nhất của hàm số $y=f(x)$ trên đoạn $\left[0;3\right]$ là $-4$.
	}
\end{ex}
%%%==============HetCau_EX2==============%%%

%%%==============Cau_EX3==============%%%
\begin{ex}%[Dự án C THPTQG 2025]%[Vương Quốc Phong]%[1D6H3-2]
	Tập xác định của hàm số $y=\log_{2024} (3-x)$ là
	\choice
	{\True $\mathscr{D} = (-\infty;3)$}
	{$\mathscr{D} = (3;+\infty)$}
	{$\mathscr{D} = (0;+\infty)$}
	{$\mathscr{D} = \mathbb{R}$}
	\loigiai{
		Ta có $3-x > 0 \Leftrightarrow x < 3$. \\
		Vậy $\mathscr{D} = (-\infty;3)$.
	}
\end{ex}
%%%==============HetCau_EX3==============%%%

%%%==============Cau_EX4==============%%%
\begin{ex}%[Dự án C THPTQG 2025]%[Vương Quốc Phong]%[2D1H2-1]
	Cho hàm số $y=f(x)$ xác định trên $\mathbb{R}$ và có đạo hàm $f'(x)=x^{2024} (3-x)$, $\forall x\in \mathbb{R}$. Hàm số đã cho có mấy điểm cực trị?
	\choice
	{$3$}
	{$0$}
	{$2$}
	{\True $1$}
	\loigiai{
	Ta có $f'(x)=x^{2024} (3-x)=0 \Leftrightarrow \hoac{&{x=0} \\ &{x=3}}$, trong đó $x=0$ (nghiệm bội chẵn). \\
	Bảng biến thiên
	\begin{center}
		\begin{tikzpicture}[>=stealth, thick, x=1.2cm, y=1.0cm]
			\def\sohang{5}
			\def\socot{8}
			%ĐN các điểm
			\foreach \x in {1,...,\sohang}
			\foreach \y in {1,...,\socot}
			\path (\y,-\x) coordinate (r{\x}c{\y}) node (r{\x}c{\y}) {};
			%Đường kẻ ngang, dọc
			\draw
			($(r{1}c{1})!.5!(r{2}c{1})+(-.5,0)$)--($(r{1}c{\socot})!.5!(r{2}c{\socot})+(.5,0)$)
			($(r{2}c{1})!.5!(r{3}c{1})+(-.5,0)$)--($(r{2}c{\socot})!.5!(r{3}c{\socot})+(.5,0)$)
			($(r{1}c{1})!.5!(r{1}c{2})+(0,.5)$)--($(r{\sohang}c{1})!.5!(r{\sohang}c{2})+(0,-.5)$)
			;
			%Khung viền
			\draw ($(r{1}c{1})+(-.5,.5)$) rectangle ($(r{\sohang}c{\socot})+(.5,-.5)$);
			%Node các giá trị
			\foreach \diem/\nhan in {
			%Dòng x
			r{1}c{1}/x,r{1}c{2}/\infty,r{1}c{4}/0,r{1}c{6}/3,r{1}c{8}/+\infty,
			%Dòng f'(x)
			r{2}c{1}/f'(x),r{2}c{3}/+,r{2}c{4}/0,r{2}c{5}/+,r{2}c{6}/0,r{2}c{7}/-
			} \path (\diem) node{$\nhan$};
			\path ($(r{3}c{1})!.5!(r{\sohang}c{1})$) node{$f(x)$};
			\foreach \diem/\nhan in {
			%Các dòng của f(x)
			r{3}c{6}/,r{5}c{2}/,r{5}c{8}/
			} \path (\diem) node[shape=rectangle, fill=white, inner sep=2pt] (\diem) {$\nhan$};
			%Vẽ các dấu mũi tên
			\foreach \dau/\cuoi in {r{5}c{2}/r{3}c{6},r{3}c{6}/r{5}c{8}} \draw[->] (\dau)--(\cuoi);
		\end{tikzpicture}
	\end{center}
	Vậy hàm số đã cho có $1$ điểm cực trị.
	}
\end{ex}
%%%==============HetCau_EX4==============%%%

%%%==============Cau_EX5==============%%%
\begin{ex}%[Dự án C THPTQG 2025]%[Vương Quốc Phong]%[1D9H2-5]
	Cho hai biến cố độc lập $A$ và $B$. Biết $P(A)=\dfrac{1}{4}$, $P(A\cup B)=\dfrac{1}{2}$. Tính $P(B)$.
	\choice
	{$\dfrac{3}{4}$}
	{$\dfrac{1}{4}$}
	{$\dfrac{1}{8}$}
	{\True $\dfrac{1}{3}$}
	\loigiai{
		Ta có $P(A\cup B)=P(A)+P(B)-P(AB)=P(A)+P(B)-P(A)\cdot P(B)$ \\
		$\Rightarrow P(B) = \dfrac{P(A \cup B) - P(A)}{1-P(A)} = \dfrac{1}{3}$
	}
\end{ex}
%%%==============HetCau_EX5==============%%%

%%%==============Cau_EX6==============%%%
\begin{ex}%[Dự án C THPTQG 2025]%[Vương Quốc Phong]%[1H8H5-3]
	% \immini{
		Cho hình chóp $S.ABC$ có $SA \perp (ABC)$, $SA=AB=2a$, tam giác $ABC$ vuông tại $B$. Khoảng cách từ $A$ đến mặt phẳng $(SBC)$ bằng
	% }
	% {
	% 	\begin{tikzpicture}[line join = round, line cap = round, thick, font = \small, scale = 0.8]
	% 		\path
	% 		(0:0) coordinate (A)
	% 		+(0:5) coordinate (C)
	% 		+(-50:3) coordinate (B)
	% 		+(90:4) coordinate (S)
	% 		;
	% 		\draw[dashed]
	% 		(A)--(C)
	% 		;
	% 		\draw
	% 		(S)--(A)--(B)--(C)--cycle
	% 		(S)--(B)
	% 		\foreach \x/\y/\z in {S/A/C, S/A/B}{
	% 				pic[draw, angle radius = 8pt]{right angle = \x--\y--\z}
	% 			}
	% 		;
	% 		\foreach \x/\g in {A/180,C/0,B/-90,S/90}
	% 		\fill (\x) circle (1.5pt)
	% 		+(\g:3mm) node {$\x$};
	% 	\end{tikzpicture}
	% }
	\choice
	{$a\sqrt{3}$}
	{\True $a\sqrt{2}$}
	{$a$}
	{$2a$}
	\loigiai{
		\begin{center}
			\begin{tikzpicture}[line join = round, line cap = round, thick, font = \small, scale = 1]
				\path
				(0:0) coordinate (A)
				+(0:5) coordinate (C)
				+(-50:3) coordinate (B)
				+(90:4) coordinate (S)
				($(B)!(A)!(S)$) coordinate (H)
				;
				\draw[dashed]
				(A)--(C)
				;
				\draw
				(S)--(A)--(B)--(C)--cycle
				(S)--(B)
				(A) -- (H)
				;
				\foreach \x/\g in {A/180,C/0,B/-90,S/90, H/45}
				\fill (\x) circle (1.5pt)
				+(\g:3mm) node {$\x$};
			\end{tikzpicture}
		\end{center}
		Ta có $SA \perp (ABC)\Rightarrow SA \perp BC$, $BC \perp AB \Rightarrow BC \perp (SAB) \Rightarrow BC\perp AH$. \\
		Kẻ $AH \perp SB$ và $AH \perp BC \Rightarrow AH \perp (SBC)$, suy ra khoảng cách từ $A$ đến mặt phẳng $(SBC)$ bằng $AH$. \\
		Xét tam giác $SAH$, ta có $\dfrac{1}{AH^2} = \dfrac{1}{SA^2} + \dfrac{1}{AB^2} = \dfrac{1}{2a^2} \Rightarrow AH=a\sqrt{2}$.
	}
\end{ex}
%%%==============HetCau_EX6==============%%%

%%%==============Cau_EX7==============%%%
\begin{ex}%[Dự án C THPTQG 2025]%[Vương Quốc Phong]%[1D1H5-1]
	Biết phương trình $\sin x=m$ có một họ nghiệm là $x=\dfrac{\pi}{5}+k2\pi$, $k\in \mathbb{Z}$. Họ nghiệm còn lại của phương trình đã cho là biểu thức nào sau đây?
	\choice
	{$x=\dfrac{4\pi}{5}+k\pi, k\in \mathbb{Z}$}
	{$x=\dfrac{\pi}{5}+k\pi, k\in \mathbb{Z}$}
	{\True $x=\dfrac{4\pi}{5}+k2\pi, k\in \mathbb{Z}$}
	{$x=-\dfrac{\pi}{5}+k2\pi, k\in \mathbb{Z}$}
	\loigiai{
		Ta có phương trình $\sin x=m$ có một họ nghiệm là $x=\dfrac{\pi}{5}+k\pi, k\in \mathbb{Z}$, họ nghiệm còn lại là $x=\pi-\dfrac{\pi}{5}+k2\pi=\dfrac{4\pi}{5}+k2\pi, k\in \mathbb{Z}$.
	}
\end{ex}
%%%==============HetCau_EX7==============%%%

%%%==============Cau_EX8==============%%%
\begin{ex}%[Dự án C THPTQG 2025]%[Vương Quốc Phong]%[2D1H1-2]
	\immini[thm]{Cho hàm số $y=f(x)$ xác định, có đạo hàm trên $\mathbb{R}$ và $f'(x)$ có đồ thị như hình vẽ. Hàm số $y=f(x)$ đồng biến trên khoảng nào dưới đây?
	\choice
	{$(1;4)$}
	{\True $(-1;1)$}
	{$(1;+\infty)$}
	{$(-\infty;-1)$}
	}{\begin{tikzpicture}[line join = round, line cap = round, >=stealth, scale = .7]
			%Hệ trục Oxy và hàm số cần vẽ
			\def\xmin{-1.5}     \def\xmax{4.5}
			\def\ymin{-2.5}       \def\ymax{2.5}
			\def\f(#1){0.25*(#1)^3-(#1)^2-0.25*(#1)+1}
			%Vẽ hệ trục
			\draw[->] (\xmin,0)--(0,0) node[below right]{$O$}--(\xmax,0) node[below]{$x$};
			\draw[->] (0,\ymin)--(0,\ymax) node[right]{$y$};
			%Vẽ hàm số
			\begin{scope}
				\clip (\xmin,\ymin) rectangle (\xmax,\ymax);
				\draw[smooth, thick, blue!50!black] plot[domain = \xmin:\xmax, samples = 200, variable=\x]({\x},{\f(\x)});
			\end{scope}
			%Vẽ các điểm gióng
			\foreach \x in {}{
					\pgfmathsetmacro\fx{\f(\x)}
					\draw[dashed,thin] (\x,0) |- (0,{\fx});
				}
			\foreach \x/\g in {-1/135, 1/90, 4/120}
			\fill (\x, 0) circle (1.5pt)
			+(\g:3mm) node {$\x$};
		\end{tikzpicture}}
	\loigiai{
		Dựa vào đồ thị ta thấy $f'(x) > 0\Leftrightarrow \hoac{&{-1< x < 1} \\ &{x > 4}}$, suy ra hàm số đồng biến trên $(-1;1)$, $(4;+\infty)$.
	}
\end{ex}
%%%==============HetCau_EX8==============%%%

%%%==============Cau_EX9==============%%%
\begin{ex}%[Dự án C THPTQG 2025]%[Vương Quốc Phong]%[2H2N1-2]
	Cho hình hộp $ABCD.A'B'C'D'$ có đáy $ABCD$ là hình bình hành tâm $O$. Khi đó $2 \cdot \overrightarrow{AO}$ bằng véc-tơ nào sau đây?
	% \begin{center}
	% 	\begin{tikzpicture}[line join = round, line cap = round, thick, font = \small, scale = 1]
	% 		\def \canh{4}
	% 		\path
	% 		(0:0) coordinate (D')
	% 		+(90:\canh) coordinate (D)
	% 		+(0:\canh) coordinate (C')
	% 		+(40:.6*\canh) coordinate (A')
	% 		($(C')+(D)-(D')$) coordinate (C)
	% 		($(D)+(A')-(D')$) coordinate (A)
	% 		($(C')+(A')-(D')$) coordinate (B')
	% 		($(C)+(A)-(D)$) coordinate (B)
	% 		($(A)!.5!(C)$) coordinate (O)
	% 		;
	% 		\draw[dashed]
	% 		(A')--(A) (A')--(B') (A')--(D')
	% 		;
	% 		\draw
	% 		(A)--(B)--(B')--(C')--(D')--(D)--cycle
	% 		(C)--(B) (C)--(D) (C)--(C') (A)--(C)
	% 		;
	% 		\foreach \x/\g in {D'/-90,C'/-90,D/180,A'/135,C/-45,A/90,B'/0,B/90, O/30}
	% 		\fill (\x) circle (1.5pt)
	% 		+(\g:3mm) node {$\x$};
	% 	\end{tikzpicture}
	% \end{center}
	\choice
	{$\overrightarrow{A'C}$}
	{$\overrightarrow{AB}$}
	{$\overrightarrow{AD}$}
	{\True $\overrightarrow{AC}$}
	\loigiai{
		Ta có $2 \cdot \overrightarrow{AO}=\overrightarrow{AC}$.
	}
\end{ex}
%%%==============HetCau_EX9==============%%%

%%%%==============Cau_EX10==============%%%
\begin{ex}%[Dự án C THPTQG 2025]%[Vương Quốc Phong]%[1H8N2-1]
	Trong không gian, qua một điểm $O$ cho trước có bao nhiêu đường thẳng vuông góc với  mặt phẳng $(\alpha)$ cho trước.
	\choice
	{\True $1$}
	{Vô số}
	{$2$}
	{$0$}
	\loigiai{
		Qua một điểm $O$ cho trước có một đường thẳng vuông góc với mặt phẳng $(\alpha)$ cho trước.
	}
\end{ex}
%%%%==============HetCau_EX10==============%%%

%%%%==============Cau_EX11==============%%%
\begin{ex}%[Dự án C THPTQG 2025]%[Vương Quốc Phong]%[2D1N4-1]
	Đường tiệm cận ngang của đồ thị hàm số $y=\dfrac{2x+1}{x+1}$ là
	\choice
	{\True $y=2$}
	{$x=-1$}
	{$y=-1$}
	{$x=2$}
	\loigiai{
		Đường tiệm cận ngang của đồ thị hàm số $y=\dfrac{2x+1}{x+1}$ là $y=2$.
	}
\end{ex}
%%%%==============HetCau_EX11==============%%%

%%%==============Cau_EX12==============%%%
\begin{ex}%[Dự án C THPTQG 2025]%[Vương Quốc Phong]%[1D7H2-1]
	Đạo hàm của hàm số $y=3^x$ là
	\choice
	{\True $y'=3^x \cdot \ln x$}
	{$y'=3^x$}
	{$y'=x\cdot 3^{x-1}$}
	{\True $y'=3^x\cdot \ln 3$}
	\loigiai{
		Đạo hàm của hàm số $y=3^x$ là $y'=3^x \cdot \ln 3$.
	}
\end{ex}
%%%==============HetCau_EX12==============%%%

%  \Closesolutionfile{ans}
%  \Closesolutionfile{ansbook}
 
\cauds
%   \Opensolutionfile{ansbook}[Ansbook/KSCL-THPT-ChuyenVinhPhuc-VinhPhuc-L1-NH24-25-TF]%---Nên đặt tên theo bài
%   \setcounter{ex}{0}
 %%%==============Cau_EX1==============%%%
\begin{ex}%[Dự án C THPTQG 2025]%[Vương Quốc Phong]%[2D1V5-8]
	Theo báo cáo của một cơ sở sản xuất nước tinh khiết, nếu mỗi ngày cơ sở này sản xuất $x$ (m$^3$) nước tinh khiết thì phải chi phí các khoản sau: $3$ triệu đồng chi phí cố định; $0{,}15$ triệu đồng cho mỗi mét khối sản phẩm; $0{,}0003x^2$ chi phí bảo dưỡng máy móc. Biết công suất tối đa mỗi ngày của cơ sở này là $200$ m$^3$. Gọi $C(x)$ là chi phí sản xuất $x$ (m$^3$) sản phẩm mỗi ngày và $\overline{c}(x)$ là chi phí trung bình mỗi mét khối sản phẩm. Khi đó, mệnh đề sau đây đúng hay sai?
	\choiceTF
	{Chi phí sản xuất $100$ m$^3$ nước tinh khiết là $20$ triệu đồng}
	{\True $\overline{c}(x) = 0{,}0003x + 0{,}15 + \dfrac{3}{x}$}
	{\True Chi phí trung bình mỗi mét khối sản phẩm thấp nhất khi sản lượng nước tinh khiết trong ngày là $100$ m$^3$}
	{$C(x) = 0{,}0003x^2 + 0{,}15x + 5$}
	\loigiai{
	Để sản xuất $x$ (m$^3$) nước tinh khiết thì phải chi phí các khoản sau: $3$ triệu đồng chi phí cố định; $0{,}15$ triệu đồng cho mỗi mét khối sản phẩm; $0{,}0003x^2$ chi phí bảo dưỡng máy móc. \\
	Suy ra để sản xuất $1$ (m$^3$) nước tinh khiết thì cần $\dfrac{3}{x}$ triệu đồng chi phí cố định; $0{,}15$ triệu đồng cho mỗi mét khối sản phẩm; $0{,}0003x$ chi phí bảo dưỡng máy móc. \\
	$\Rightarrow \overline{c}(x)=\dfrac{3}{x}+0{,}15+0{,}0003x$. \\
	$\Rightarrow C(x)=\overline{c}(x)\cdot x=3+0{,}15x+0{,}0003x^2$.
	\begin{itemchoice}
		\itemch Chi phí sản xuất $100$ m$^3$ là $C(100)=3+0{,}15 \cdot 100 + 0{,}0003 \cdot 100^2 =21$ (triệu đồng).
		\itemch Ta tìm được $\overline{c}(x)=\dfrac{3}{x}+0{,}15+0{,}0003x$.
		\itemch Hàm chi phí trung bình mỗi mét khối sản phẩm là $\overline{c}(x)=\dfrac{3}{x}+0{,}15+0{,}0003x$, $0< x\le 200$. \\
		Đặt $f(x)=\overline{c}(x)=\dfrac{3}{x}+0{,}15+0{,}0003x$, $0< x\le 200$. \\
		$f'(x)=-\dfrac{3}{x^2}+0{,}0003$. \\
		$f'(x)=0\Rightarrow-3+0{,}0003x^2=0\Rightarrow x=100$. \\
		Bảng biến thiên của hàm $f(x)$.
		\begin{center}
			\begin{tikzpicture}
				\tkzTabInit[lgt=1.2,espcl=4]
				{$x$/1,$f’(x)$/1,$f(x)$/2}
				{$0$, $100$, $200$}
				\tkzTabLine{ ,-,z,+, }
				\tkzTabVar{+/, -/$0{,}21$, +/}
			\end{tikzpicture}
		\end{center}
		Dựa vào BBT thì chi phí trung bình mỗi mét khối sản phẩm thấp nhất khi sản lượng nước tinh khiết trong ngày là $100$ m$^3$.
		\itemch Ta có: $C(x) = 3 + 0{,}15x+0{,}0003x^2$.
	\end{itemchoice}
	}
\end{ex}
%%%==============HetCau_EX1==============%%%

%%%==============Cau_EX2==============%%%
\begin{ex}%[Dự án C THPTQG 2025]%[Vương Quốc Phong]%[1H8V7-2]
	Cho hình chóp $S.ABC$ có mặt bên $(SAB)$ vuông góc với mặt phẳng đáy và tam giác $SAB$ đều cạnh $2a$. Biết tam giác $ABC$ vuông tại $C$ và cạnh $AC=a\sqrt{3}$. Gọi $H$ là trung điểm $AB$.
	\choiceTF
	{\True Mặt phẳng $(SHC)$ và $(ABC)$ vuông góc với nhau}
	{Thể tích của khối chóp $S.ABC$ bằng $\dfrac{a^3}{6}$}
	{$d(C,(SAB)) = \dfrac{a\sqrt{3}}{3}$}
	{\True $SH \perp (ABC)$}
	\loigiai{
		\begin{center}
			\begin{tikzpicture}[line join = round, line cap = round, thick, font = \small, scale = 1]
				\path
				(0:0) coordinate (B)
				+(0:5) coordinate (C)
				+(-60:3) coordinate (A)
				($(A)!.5!(B)$) coordinate (H)
				++(90:4) coordinate (S)
				;
				\draw[dashed]
				(B)--(C)--(H)
				;
				\draw
				(A)--(B)--(S)--(C)--cycle
				(A)--(S)--(H)
				;
				\foreach \x/\g in {B/180, C/0, S/90, H/210, A/-90}
				\fill (\x) circle (1.5pt)
				+(\g:3mm) node {$\x$};
			\end{tikzpicture}
		\end{center}
		\begin{itemchoice}
			\itemch
			\begin{itemize}
				\item Hình chóp $S.ABC$ có mặt bên $(SAB)$ vuông góc với mặt phẳng đáy và tam giác $SAB$ đều nên $SH \perp AB$, mà $AB$ là giao tuyến của $(SAB)$ và mp đáy nên $SH \perp (ABC)$.
				\item $SH \perp (ABC)$, mà $SH \subset (SHC)$ nên $(SHC)\perp (ABC)$.
			\end{itemize}
			\itemch
			\begin{itemize}
				\item $SH \perp (ABC)$, $SH$ là đường cao của hình chóp, $SH = 2a \dfrac{\sqrt{3}}{2}=a\sqrt{3}$.
				\item Tam giác $ABC$ vuông tại $C$ và cạnh $AC=a\sqrt{3}$, $AB=2a$ nên $BC=\sqrt{(2a)^2-\left(a\sqrt{3} \right)^2}=a$
				\item $S_{ABC}=\dfrac{1}{2} a \cdot a\sqrt{3}=\dfrac{a^2\sqrt{3}}{2}$
				\item $V_{S.ABC}=\dfrac{1}{3} SH \cdot S_{ABC}=\dfrac{1}{3} a\sqrt{3} \cdot \dfrac{a^2\sqrt{3}}{2}=\dfrac{a^3}{2}$
			\end{itemize}
			\itemch $d\left(C, (SAB)\right) = \dfrac{3 \cdot V_{S.ABC}}{_{S_{SAB}}} = \dfrac{3 \cdot a^3}{2 \cdot \dfrac{4a^2\sqrt{3}}{4}}=\dfrac{a\sqrt{3}}{2}$
			\itemch  Theo lập luận trong các câu trên ta có $SH \perp (ABC)$.
		\end{itemchoice}
	}
\end{ex}
%%%==============HetCau_EX2==============%%%

%%%==============Cau_EX3==============%%%
\begin{ex}%[Dự án C THPTQG 2025]%[Vương Quốc Phong]%[2D3H1-3]
	Phòng quản lí đào tạo trường Đại học Kinh tế quốc dân thống kê số giờ làm thêm của một nhóm sinh viên năm thứ tư của trường thu được kết quả như bảng sau:
	\begin{center}
		\begin{tabular}{|c|c|c|c|c|c|}
			\hline
			\thead{\textbf{Số giờ làm thêm (giờ/tuần)}} & $[9; 12)$ & $[12; 15)$ & $[15; 18)$ & $[18; 21)$ & [$21; 24)$ \\
			\hline
			\thead{\textbf{Số sinh viên}}               & $6$       & $12$       & $4$        & $2$        & $1$        \\
			\hline
		\end{tabular}
	\end{center}
	\choiceTF
	{Số giờ làm thêm trung bình của nhóm sinh viên trên trong một tuần là $16{,}5$ giờ}
	{\True Giá trị đại diện của nhóm $[9;12)$ là $10{,}5$}
	{Tứ phân vị thứ ba là $15{,}65$}
	{Nhóm chứa trung vị là $[15;18)$}
	\loigiai{
		\begin{itemchoice}
			\itemch Cỡ mẫu: $n=6+12+4+2+1=25$. \\
			Ta có bảng sau:
			\begin{center}
				\begin{tabular}{|c|c|c|c|c|c|}
					\hline
					\thead{\textbf{Số giờ làm thêm (giờ/tuần)}} & $[9; 12)$ & $[12; 15)$ & $[15; 18)$ & $[18; 21)$ & $[21; 24)$ \\
					\hline
					\thead{\textbf{Giá trị đại diện}}           & $10{,}5$  & $13{,}5$   & $16{,}5$   & $19{,}5$   & $22{,}5$   \\
					\hline
					\thead{\textbf{Số sinh viên}}               & $6$       & $12$       & $4$        & $2$        & $1$        \\
					\hline
				\end{tabular}
			\end{center}
			Số giờ làm thêm trung bình của nhóm sinh viên trên trong một tuần là
			$$\overline{x}=\dfrac{6\cdot 10{,}5+12\cdot 13{,}5+4\cdot 16{,}5+2\cdot 19{,}5+1\cdot 22{,}5}{25}=14{,}1 \text{ (giờ).}$$
			\itemch Giá trị đại diện của nhóm $\left[9;12\right)$ là $10{,}5$.
			\itemch Giả sử $x_1, x_2,\ldots, x_{25}$ số giờ làm thêm của các sinh viên trong mẫu số liệu trên và dãy này đã được sắp xếp theo thứ tự không giảm. \\
			Khi đó, trung vị của mẫu số liệu là $x_{13}$ và tứ phân vị thứ ba là $\dfrac{1}{2} \left(x_{19}+x_{20} \right)$. Vì $x_{19}, x_{20}$ đều thuộc nhóm $\left[15;18\right)$ nên nhóm này chứa tứ phân vị thứ ba. Do đó, tứ phân vị thứ ba là:
			\[Q_3=15+\dfrac{\dfrac{3\cdot 25}{4}-12-6}{4} \cdot \left(18-15\right)=15{,}5625.
			\]
			\itemch  Vì $x_{13}$ thuộc nhóm $\left[12;15\right)$ nên nhóm chứa trung vị là nhóm $\left[12;15\right)$.
		\end{itemchoice}
	}
\end{ex}
%%%==============HetCau_EX3==============%%%

%%%==============Cau_EX4==============%%%
\begin{ex}%[Dự án C THPTQG 2025]%[Vương Quốc Phong]%[2D1V4-3]
	Cho hàm số $f(x)=\dfrac{ax+b}{cx+d}$ với $a,b,c,d\in \mathbb{R}$ và $c \ne 0$ có đồ thị hàm số $y=f'(x)$ nhận đường thẳng $x=-1$ làm tiệm cận đứng như hình vẽ dưới. Biết rằng giá trị lớn nhất của hàm số $y=f(x)$ trên đoạn $\left[-3;-2\right]$ bằng $8$.
	\begin{center}
		\begin{tikzpicture}[line join = round, line cap = round, >=stealth,  scale = .7]
			%Hệ trục Oxy và hàm số cần vẽ
			\def\xmin{-4}     \def\xmax{4}
			\def\ymin{-1}       \def\ymax{5}
			\def\f(#1){3/((#1)+1)^2}
			%Vẽ hệ trục
			\draw[->] (\xmin-0.2,0)--(0,0) node[below right]{$O$}--(\xmax + 0.5,0) node[below]{$x$};
			\draw[->] (0,\ymin)--(0,\ymax + 0.5) node[right]{$y$};
			%Vẽ hàm số
			\begin{scope}
				\clip (\xmin,\ymin) rectangle (\xmax,\ymax);
				\draw[smooth, thick, blue!50!black] plot[domain = \xmin: -1.4, samples = 200, variable=\x]({\x},{\f(\x)});
				\draw[smooth, thick, blue!50!black] plot[domain = -0.7: \xmax, samples = 200, variable=\x]({\x},{\f(\x)});
			\end{scope}
			\draw[dashed] (-1,-1)--(-1,5.5);
			%Vẽ các điểm gióng
			\foreach \x in {}{
					\pgfmathsetmacro\fx{\f(\x)}
					\draw[dashed,thin] (\x,0) |- (0,{\fx});
				}
			%Vẽ các điểm trên trục Ox
			\foreach \x/\g in {-4/-90,-3/-90,-2/-90,-1/-90,1/-90,2/-90,3/-90,4/-90}
			\draw[thin] (\x,2pt)--(\x,-2pt) + (\g:3mm) node {$\x$};
			%Vẽ các điểm trên trục Oy
			\foreach \y/\g in {1/180,2/180,3/180,4/180,5/180}
			\draw[thin] (2pt,\y)--(-2pt,\y) + (\g:3mm) node {$\y$};
		\end{tikzpicture}
	\end{center}
	\choiceTF
	{\True Giá trị nhỏ nhất của hàm số $y=f(x)$ trên đoạn $\left[2;4\right]$ bằng $4$}
	{$f(-3)=8$}
	{\True Hàm số $y=f(x)$ nghịch biến trên khoảng $(-1;+\infty)$}
	{\True Đồ thị hàm số $y=f'(x)$ nhận đường thẳng $y=0$ làm tiệm cận ngang}
	\loigiai{
		Tập xác định: $\mathscr{D} = \mathbb{R} \setminus \left\{-\dfrac{d}{c} \right\}$. \\
		Ta có $f'(x)=\dfrac{ad-bc}{(cx+d)^2}$.\\
		Đồ thị hàm số $y=f'(x)$ nhận đường thẳng $x=-1$ làm tiệm cận đứng nên $c=d\ne 0$. \\
		Hơn nữa, $f'(x)$ nằm hoàn toàn trên trục hoành nên hàm số $y=f(x)$ đồng biến trên các khoảng xác định và $f'(0)=3$ nên
		\[\heva{&\max\limits_{\left[-3;-2\right]} f(x) = f(-2) = 8 \\ &{\dfrac{ad-bc}{d^2}=3}} \Leftrightarrow \heva{&{\dfrac{2a-b}{d}=8} \\ &{(a-b)d = 3d^2}} \xrightarrow{c=d} \heva{&2a-b=8d \\ & \hoac{&d=0 \text{ (loại)} \\ &a-b=3d}} \Leftrightarrow \heva{&{a=5d} \\&{b=2d}}\]
		Chọn $d=1 \Rightarrow a=5$, $b=2$. \\
		Khi đó $f(x)=\dfrac{5x+2}{x+1} \Rightarrow f'(x)=\dfrac{3}{\left(x+1\right)^2}$.
		\begin{itemchoice}
			\itemch Giá trị nhỏ nhất của hàm số $y=f(x)$ trên đoạn $\left[2;4\right]$ bằng $4$. \\
			Hàm số $y=f(x)$ đồng biến trên $\left[2;4\right]$ nên ${\min \limits_{\left[2;4\right]}} f(x)=f(2)=4$.
			\itemch $f(-3) = \dfrac{5 \cdot (-3) + 2}{-3+2}=\dfrac{13}{2}$.
			\itemch Hàm số $y=f(x)$ đồng biến trên khoảng $\left(-1;+\infty \right)$.
			\itemch Đồ thị hàm số $y=f'(x)$ nhận đường thẳng $y=0$ làm tiệm cận ngang. \\
			Ta có $\lim \limits_{x\to+\infty} f'(x) = \lim \limits_{x\to+\infty} \dfrac{3}{(x+1)^2}=0\Rightarrow y=0$ là TCN của đồ thị hàm số $y=f'(x)$.
		\end{itemchoice}
	}
\end{ex}
%%%==============HetCau_EX4==============%%% \Closesolutionfile{ansbook}
 

\caukq
% \Opensolutionfile{ansbt}[Ansbook/KSCL-THPT-ChuyenVinhPhuc-VinhPhuc-L1-NH24-25-TLN]%---Nên đặt tên theo bài
% \setcounter{ex}{0}
%%%==============Bai_BT1==============%%%
\begin{ex}%[Dự án C THPTQG 2025]%[Vương Quốc Phong]%[2D1V5-8]
	Hai con tàu $A$ và $B$ đang ở cùng một vĩ tuyến và cách nhau $6$ hải lí. Cả hai tàu đồng thời cùng khởi hành. Tàu $A$ chạy về hướng Nam với vận tốc $5$ hải lí/giờ, còn tàu $B$ chạy về vị trí hiện tại của tàu $A$ với vận tốc $7$ hải lí/ giờ. Hỏi sau bao nhiêu giờ thì khoảng cách giữa hai tàu là bé nhất?
	\begin{center}
		\begin{tikzpicture}
			\path
			(0,0) coordinate (A)
			(3,0) coordinate (B1)
			(7,0) coordinate (B)
			(0,-2) coordinate (A1)
			(0,-4) coordinate (A2)
			;
			\draw
			(A2)--(A)node[above]{\Huge{\faShip}}--(B)node[above]{\Huge{\faShip}}
			(A1) -- (B1)
			;
			\draw[-stealth, transform canvas = {xshift = -1 cm}] (A)--(A1);
			\draw[-stealth, yshift = 0.5 cm] (6,0)--(4,0);
			\foreach \x/\g in {A/180,B/-90,A1/180, B1/-90}
			\fill (\x) circle (1.5pt)
			+(\g:3mm) node{$\x$};
		\end{tikzpicture}
	\end{center}
	\shortans{0,57}
	\loigiai{
		Giả sử ban đầu tàu $A$ ở vị trí $A$ và tàu $B$ ở vị trí $B$. Sau khoảng thời gian $t$:
		\begin{itemize}
			\item Tàu $A$ di chuyển được quãng đường $5t$ về phía Nam đến vị trí $A_1$.
			\item Tàu $B$ di chuyển được quãng đường $7t$ đến vị trí $B_1$.
		\end{itemize}
		Khoảng cách từ vị trí $B_1$ đến vị trí $A$ là $6-7t$. \\
		Áp dụng định lý Pytago ta có:  $d=A_1 B_1=f(t)=\sqrt{(6-7t)^2+(5t)^2}=\sqrt{74t^2-84t+36}$. \\
		Để khoảng cách giữa hai tàu nhỏ nhất, thì hàm số $g(t)=74t^2-84t+36$ đạt giá trị nhỏ nhất. \\
		Hàm số $g(t)$ đạt giá trị nhỏ nhất tại $t=\dfrac{-(-84)}{2\cdot 74} = \dfrac{21}{37}$, vậy thời điểm khoảng cách giữa hai tàu bé nhất là khi $t=\dfrac{21}{37} \approx 0{,}57$ (giờ).
	}
\end{ex}
%%%==============HetBai_BT1==============%%%

%%%==============Bai_BT2==============%%%
\begin{ex}%[Dự án C THPTQG 2025]%[Vương Quốc Phong]%[2H2V2-6]
	Có ba lực cùng tác động vào một vật. Hai trong ba lực này hợp với nhau một góc $100^{\circ}$ và có độ lớn lần lượt là $25$ N và $12$ N. Lực thứ ba vuông góc với mặt phẳng tạo bởi hai lực đã cho và có độ lớn $4$ N. Tính độ lớn của hợp lực của ba lực trên (làm tròn đến hàng phần chục).
	\shortans{26,1}
	\loigiai{
		\begin{center}
			\begin{tikzpicture}[line join = round, line cap = round, thick, font = \small, scale = 1]
				\path
				(0:0) coordinate (B)
				+(0:5) coordinate (D)
				+(65:3) coordinate (O)
				($(O)+(D)-(B)$) coordinate (A)
				($(O)+(0,3)$) coordinate (C)
				($(C)+(D)-(O)$) coordinate (E)
				;
				\foreach \x/\y in {O/B, O/A, O/D, O/C}
				\draw[-stealth] (\x)--(\y);
				\draw[dashed]
				(C)--(E)--(D)
				(B)--(D)--(A)
				;
				\foreach \x/\g in {D/-90,C/90,A/90,B/180, O/180, E/90}
				\path (\x)
				+(\g:3mm) node{$\x$};
				\draw[thin] pic[draw, angle radius = 7mm, "$100^{\circ}$", angle eccentricity = 1.5]{ angle = B--O--A};
			\end{tikzpicture}
		\end{center}
		Gọi $\overrightarrow{F_1}$, $\overrightarrow{F_2}$, $\overrightarrow{F_3}$ là ba lực tác động vào vật tại điểm $O$ lần lượt có độ lớn $25$ N, $12$ N, $4$ N. \\
		Vẽ $\overrightarrow{OA}=\overrightarrow{F_1}$, $ \overrightarrow{OB}=\overrightarrow{F_2}$, $ \overrightarrow{OC}=\overrightarrow{F_3}$, dựng hình bình hành $OADB$ và $ODEC$. \\
		Khi đó hợp lực tác động vào vật là: $\overrightarrow{F}=\overrightarrow{OA}+\overrightarrow{OB}+\overrightarrow{OC}=\overrightarrow{OD}+\overrightarrow{OC}=\overrightarrow{OE}$. \\
		Áp dụng định lý cô sin trong tam giác $OBD$, ta có:
		\[OD^2 = OB^2 + BD^2 - 2OB \cdot BD \cos \widehat{OBD}=12^2+25^2-2 \cdot 12 \cdot 25 \cdot \cos 80^{\circ}=769-600 \cdot \cos 80^\circ
		\]
		Vì $OC \perp (OADB)$ nên $OC\perp OD$, suy ra $ODEC$ là hình chữ nhật. Do đó tam giác $ODE$ vuông tại $D$. Ta có $OE=\sqrt{OD^2+ED^2} \approx 26{,}1$
	}
\end{ex}
%%%==============HetBai_BT2==============%%%

%%%==============Bai_BT3==============%%%
\begin{ex}%[Dự án C THPTQG 2025]%[Vương Quốc Phong]%[1D6V4-6]
	Dân số trung bình sơ bộ năm $2021$ của tỉnh Vĩnh Phúc là $1.191.782$ người, tăng $1{,}75\%$ so với năm $2020$. Hỏi với tốc độ tăng dân số được duy trì mức $1{,}75\%$ một năm thì đến năm bao nhiêu dân số tỉnh Vĩnh Phúc lần đầu vượt $1.880.000$ người.
	\shortans{2048}
	\loigiai{
	Áp dụng công thức tăng trưởng dân số thì dân số của tỉnh Vĩnh Phúc sau $n$ năm (tính từ năm $2021$) được tính theo công thức:
	\[S_n=S_0 \cdot e^{rn}=1191782 \cdot \mathrm{e}^{0{,}0175n}
	\]
	Để dân số tỉnh Vĩnh Phúc sau $n$ năm vượt $1.880.000$ người điều kiện là:
	\begin{eqnarray*}
		&& S_n > 1880000 \\ &\Leftrightarrow& 1191782 \cdot \mathrm{e}^{0{,}0175n} > 1880000 \\ &\Leftrightarrow& \mathrm{e}^{0{,}0175n} > \dfrac{1880000}{1191782} \Leftrightarrow 0{,}0175n > \ln \left(\dfrac{1880000}{1191782} \right) \\ &\Leftrightarrow& n > \dfrac{\ln \left(\dfrac{1880000}{1191782} \right)}{0{,}0175} \approx 26{,}047
	\end{eqnarray*}
	Mà $n \in \mathbb{N}$ nên $n\ge 27$.
	Vậy năm $2048$ là năm đầu tiên dân số tỉnh Vĩnh Phúc vượt $1.880.000$ người.
	}
\end{ex}
%%%==============HetBai_BT3==============%%%

%%%==============Bai_BT4==============%%%
\begin{ex}%[Dự án C THPTQG 2025]%[Vương Quốc Phong]%[0D0C2-9]
	Hai bạn Nga và Nhung chơi trò tung xúc xắc. Mỗi bạn tung $1$ con xúc xắc $3$ lần, ai có tổng số chấm $3$ lần gieo lớn hơn thì thắng. Nga chơi trước và được $14$ chấm. Khi đó, xác suất để Nhung thắng Nga là $\dfrac{a}{b}$ (với $a,b$ là số nguyên dương và $\dfrac{a}{b}$ là phân số tối giản). Tính $a+b$. \\
	\shortans{59}
	\loigiai{
		Gọi $A$ là biến cố: \lq\lq Nhung thắng Nga sau ba lần tung\rq\rq.\\
		Khi đó $P(A)$ là xác suất tổng số chấm Nga tung được sau ba lần tung lớn hơn $14$. \\
		Ta có: $\Omega=\left\{(a_1, a_2, a_3)|a_i \in \{1,2,\ldots, 6\}, i=\overline{1,3}\right\}\Rightarrow n(\Omega)=6^3=216$
		\[A=\left\{(a_1, a_2, a_3)|a_1+a_2+a_3 \ge 15, a_i \in \{1,2,\ldots, 6\}, i=\overline{1,3}\right\}
		\]
		Để đếm số phần tử của $A$, ta chia thành các trường hợp:
		\begin{itemize}
			\item Trường hợp 1: $a_1+a_2+a_3=15$, gồm bộ $(5,5,5)$ và các bộ là hoán vị của $(4;5;6)$ và $(3;6;6)$. Trường hợp này có $1+6+3=10$ (bộ)
			\item  Trường hợp 2: $a_1+a_2+a_3=16$, gồm các hoán vị của $(5,5,6)$ và $(4,6,6)$. Trường hợp này có $3+3=6$ (bộ)
			\item  Trường hợp 3: $a_1+a_2+a_3=17$, gồm các hoán vị của $(5,6,6)$. Trường hợp này có $3$ (bộ).
			\item Trường hợp 4: $a_1+a_2+a_3=18$, gồm $(6,6,6)$. Trường hợp này có $1$ (bộ).
		\end{itemize}
		Vậy $n(A)=20$. \\
		Xác suất cần tìm: $P(A)=\dfrac{n(A)}{n(\Omega)}=\dfrac{20}{216}=\dfrac{5}{54} \Rightarrow a+b=59$.
	}
\end{ex}
%%%==============HetBai_BT4==============%%%

%%%==============Bai_BT5==============%%%
\begin{ex}%[Dự án C THPTQG 2025]%[Vương Quốc Phong]%[1H8V5-3]
	Cho hình chóp $S.ABC$ có đáy $ABC$ là tam giác đều cạnh bằng $2$, $SA$ vuông góc với mặt phẳng $(ABC)$; Góc giữa đường thẳng $SB$ và mặt phẳng $(ABC)$ bằng $60^\circ$. Gọi $M$ là trung điểm của cạnh $AB$. Tính khoảng cách từ điểm $B$ đến mặt phẳng $(SCM)$, kết quả làm tròn đến phần trăm.
	\shortans{0,96}
	\loigiai{
		\begin{center}
			\begin{tikzpicture}[line join = round, line cap = round, thick, font = \small, scale = 1]
				\path
				(0:0) coordinate (A)
				+(0:5) coordinate (C)
				+(-50:3) coordinate (B)
				+(90:4) coordinate (S)
				($(A)!.5!(B)$) coordinate (M)
				($(S)!0.65!(M)$) coordinate (H)
				;
				\draw[dashed]
				(A)--(C)--(M)
				;
				\draw
				(S)--(A)--(B)--(C)--cycle
				(S)--(B) (S)--(M) (A)--(H)
				\foreach \x/\y/\z in {C/M/B, A/H/M}{
						pic[draw, angle radius = 6pt]{right angle = \x--\y--\z}
					}
				;
				\foreach \x/\g in {B/-90, C/0, A/180, S/90, M/180, H/145}
				\fill (\x) circle (1.5pt)
				+(\g:3mm) node {$\x$};
			\end{tikzpicture}
		\end{center}
		Vì $AB$ là hình chiếu của $SB$ trên mặt phẳng $(ABC)$, nên góc giữa đường thẳng $SB$ và mặt phẳng $(ABC)$ bằng góc $\widehat{SBA}=60^\circ \Rightarrow SA = AB \cdot \tan 60^\circ=2\sqrt{3}$. \\
		Do $M=AB\cap (SCM)$, $M$ là trung điểm của $AB \Rightarrow d(A,(SCM))= d(B,(SCM))$. \\
		Vì $\heva{&{CM\perp AB} \\ &{CM\perp SA}} \Rightarrow CM\perp (SAB)$. \\
		Mặt khác $CM \subset (SCM)\Rightarrow (SCM)\perp (SAB)$, và $(SCM)\cap (SAB)=SM$, nên kẻ $AH\perp SM$ tại $H$ \\
		$\Rightarrow AH\perp (SMB)\Rightarrow AH = d(A,(SMC)) = d(B,(SMC))$. \\
		Xét tam giác $SAM$ vuông tại $A$, ta có $\dfrac{1}{AH^2}=\dfrac{1}{SA^2}+\dfrac{1}{AM^2}=\dfrac{1}{(2\sqrt{3})^2}+\dfrac{1}{1^2}=\dfrac{13}{12} \Rightarrow AH^2=\dfrac{12}{13} \Rightarrow AH=\sqrt{\dfrac{12}{13}} \approx 0{,}96$. \\
		Vậy khoảng cách từ điểm $B$ đến mặt phẳng $(BCM)$ bằng $0{,}96$
	}
\end{ex}
%%%==============HetBai_BT5==============%%%

%%%==============Bai_BT6==============%%%
\begin{ex}%[Dự án C THPTQG 2025]%[Vương Quốc Phong]%[2D1C5-4]
	Cho hàm số $f(x)=x(x-3)^2$. Tính số nghiệm thực của phương trình $\underbrace{f(f \cdots f(x))}_{8 \text{ lần } f}=0$.
	\shortans{3281}
	\loigiai{
		Ta có $f(x)=x(x-3)^2=x^3-6x^2+9x$. \\
		Suy ra $f'(x)=3x^2-12x+9$. \\
		$f'(x)=0\Leftrightarrow \hoac{&{x=0} \\ &{x=3}}$ \\
		Bảng biến thiên
		\begin{center}
			\begin{tikzpicture}[font=\normalsize,t style/.style={style=solid}]
				%dòng khai báo
				\tkzTabInit[lgt=1.2,espcl=1.75,deltacl=0.5]
				{$x$/0.75,$f'(x)$/0.75, $f(x)$/2}
				{$-\infty$, $0$, $1$, $3$, $4$, $+\infty$}
				%dòng xét dấu của đạo hàm
				\tkzTabLine{,, ,+, 0,-, 0,+ , ,,} % z, t, d, h (h: tô miền);
				%Khai báo vị trị các điểm của dòng f(x)
				\path (N13) node[above] (A1){$ -\infty $}
				($(N32)!0.2!(N33)$) node[below] (A2){$ 4 $}
				($(N42)!0.6!(N43)$) node[left] (A3){$ 0 $}
				(N62) node[below] (A4){$ +\infty $}
				($(N22)!0.6!(N23)$) coordinate (B) node[below] {$0$}
				($(N52)!0.3!(N53)$) coordinate (C) node[below] {$4$}
				;
				\draw[dashed]
				(N21)--(B)--(A3)
				(N51)--(C)--(A2)
				;
				\foreach \x/\y in {A1/A2,A2/A3,A3/A4}{
						\draw[-stealth] (\x)--(\y);
					}
			\end{tikzpicture}
		\end{center}
		Ta có
		$f(x)=0$ có $2$ nghiệm. \\
		$f(x)=3$ có $3$ nghiệm. \\
		$\Rightarrow f\left(f(x)\right)=0\Leftrightarrow \hoac{&{f(x)=0} \\ &{f(x)=3}}$ có $2+3^1$ nghiệm. \\
		$f\left(f\left(f(x)\right)\right)=0\Leftrightarrow \hoac{&{f\left(f(x)\right)=0} \\ &{f\left(f(x)\right)=3}}$ có $2+3^1+3^2$ nghiệm. \\
		$\dots$
		$f\left(f\left(\cdots f(x)\right)\right)=0$ có $2+3^1+3^2+\cdots + 3^7 = 3281$ nghiệm.
	}
\end{ex}
%%%==============HetBai_BT6==============%%%

 \Closesolutionfile{ans}
\inputansbox{6,4,3}{ans/KSCL-THPT-ChuyenVinhPhuc-VinhPhuc-L1-NH24-25}%---Nên đặt tên theo bài
 
% \begin{name}
	{\tenchude}
	{\tendethi}
	{TRƯỜNG THPT LÊ THÁNH TÔNG - TP.HCM}
	{\thoigian}
\end{name}

\caulc
\Opensolutionfile{ans}[ans/LeThanhTong]
% \Opensolutionfile{ansbook}[Ansbook/TenFile-TN]%---Nên đặt tên theo bài
\setcounter{ex}{0}
\begin{ex}%[2D1N1-1]%[Dự án C đợt 3 - KSCL LeThanhTong-Võ Hoàng Nghĩa]
	Cho hàm số $y = \log_3(x^2 - 2x + 3)$. Hàm số đồng biến trên khoảng nào sau đây?
	\choice
	{$(0;1)$}
	{$(-1;+\infty)$}
	{$(-\infty;-1)$}
	{\True $(1;+\infty)$}
	\loigiai{Tập xác định $D=\mathbb{R}$.\\
		Ta có $y'=\dfrac{2x-2}{(x^2-2x+3)\ln 3}$. \\
		$y'>0\Leftrightarrow \dfrac{2x-2}{(x^2-2x+3)\ln 3}>0\Leftrightarrow 2x-2>0\Leftrightarrow x>1$.\\
		Vậy hàm số đồng biến trên khoảng $(1;+\infty)$.}
\end{ex}

\begin{ex}%[2D1H3-6]%[Dự án C đợt 3 - KSCL LeThanhTong-Võ Hoàng Nghĩa]
	Một nhà phân tích thị trường làm việc cho một công ty sản xuất thiết bị gia dụng nhận thấy rằng nếu công ty sản xuất và bán $x$ chiếc máy xay sinh tố hàng tháng thì lợi nhuận thu được (nghìn đồng) có thể được tính bằng công thức $P(x) = -0{,}3x^3 + 36x^2 + 1\,800x - 48\,000$. Để có lợi nhuận lớn nhất công ty cần sản xuất đúng bao nhiêu chiếc máy sinh tố mỗi tháng?
	\choice
	{$90$}
	{\True $100$}
	{$110$}
	{$120$}
	\loigiai{Để tìm lợi nhuận lớn nhất của công ty ta tìm giá trị lớn nhất của hàm số $P(x) = -0.3x^3 + 36x^2 + 1800x - 48\,000$ trên $(0;+\infty)$.\\
	Ta có $P'(x)=-0{,}9x^2+72x+1\,800=0\Leftrightarrow\hoac{&x=100 \text{ (Nhận)}\\&x=-20 \text{ (Loại)}.}$\\
	Ta có bảng biến thiên
	\begin{center}
		\begin{tikzpicture}
			\tkzTabInit[nocadre=false,lgt=2,espcl=3,deltacl=0.9]
			{$x$/0.7,$P'(x)$/0.7,$P(x)$/1.5}{$0$,$100$,$+\infty$}
			\tkzTabLine{,+,0,-,}
			\tkzTabVar{-/$-48\,000$,+/$19\,200$,-/$-\infty$}
		\end{tikzpicture}
	\end{center}
	Từ bảng biến thiên suy ra $\underset{\left( 0;+\infty  \right)}{\mathop{\max }}\,P\left( x \right)=19\,200$ khi $x=100$.\\
	Vậy để có lợi nhuận lớn nhất công ty cần sản xuất đúng $100$ chiếc máy sinh tố mỗi tháng.

	}
\end{ex}

\begin{ex}%[2H2N1-2]%[Dự án C đợt 3 - KSCL LeThanhTong-Võ Hoàng Nghĩa]
	\immini{Cho hình hộp $ABCD.A'B'C'D'$ có tâm $O$. Khi đó, $\overrightarrow{AB} + \overrightarrow{AD} + \overrightarrow{AA'} + \overrightarrow{AC'}$ bằng
		\choice
		{$\overrightarrow{BD}$}
		{$2\overrightarrow{OC'}$}
		{\True $4\overrightarrow{AO}$}
		{$2\overrightarrow{AC}$}}{\begin{tikzpicture}[scale=0.6, font=\footnotesize,line join=round, line cap=round, >=stealth]
			\path
			(0,0) coordinate (A)
			++(-130:3) coordinate (B)
			++(0:4) coordinate (C)
			($(A)+(C)-(B)$) coordinate (D)
			($(A)!1/2!(C)$) coordinate (O)
			;
			\foreach \i in {A,B,C,D}{
					\coordinate (\i') at ($(\i)+(1,4)$);
				}
			\draw (A')--(B')--(C')--(D')--cycle;
			\draw (B)--(B') (C)--(C') (D)--(D')  (B)--(C)--(D) (A')--(C');
			\draw[dashed,thin](B)--(A)--(A') (C)--(A)--(D)--(B);
			\foreach \i/\g in {A'/90,B'/90,C'/90,D'/90,A/-90,B/-90,C/-90,D/-90,O/-90}
			\fill[black] (\i) circle(1pt)+(\g:5mm)node[scale=1]{$\i$};
		\end{tikzpicture}}
	\loigiai{Ta có $\overrightarrow{AB}+\overrightarrow{AD}+\overrightarrow{A{A}'}+\overrightarrow{A{C}'}=\overrightarrow{A{C}'}+\overrightarrow{A{C}'}=2\overrightarrow{A{C}'}=4\overrightarrow{AO}$.}
\end{ex}

\begin{ex}%[2H2N2-3]%[Dự án C đợt 3 - KSCL LeThanhTong-Võ Hoàng Nghĩa]
	Trong không gian với hệ tọa độ $Oxyz$, cho các vectơ $\overrightarrow{a} = (1;-1;2)$, $\overrightarrow{b} = (2;1;-3)$, $\overrightarrow{c} = (0;3;-2)$. Điểm $M(x;y;z)$ thỏa mãn $\overrightarrow{OM} + \overrightarrow{a} = 2\overrightarrow{b} - \overrightarrow{c}$. Tổng $x+y+z$ bằng
	\choice
	{$3$}
	{\True $-3$}
	{$4$}
	{$-2$}
	\loigiai{
		Ta có $\overrightarrow{OM}=(x;y;z)\Rightarrow \overrightarrow{OM}+\overrightarrow{a}=(x+1;y-1;z+2)$,
		$2\overrightarrow{b}-\overrightarrow{c}=(4;-1;-4)$.\\
		Mà $\overrightarrow{OM}+\overrightarrow{a}=2\overrightarrow{b}-\overrightarrow{c}\Rightarrow \heva{& x+1=4\\
				& y-1=-1\\
				& z+2=-4\\
			}\Rightarrow \heva{& x=3\\
				& y=0\\
				& z=-6\\
			}$.\\
		Vậy $x+y+z=3+0+(-6)=-3$.
	}
\end{ex}

\begin{ex}%[2D3H1-3]%[Dự án C đợt 3 - KSCL LeThanhTong-Võ Hoàng Nghĩa]
	\immini{Thời gian (phút) truy cập Internet mỗi buổi tối của một số học sinh được cho trong bảng sau.
		Khoảng tứ phân vị của mẫu số liệu trên là
		\choice
		{$10{,}75$}
		{\True $4{,}75$}
		{$4{,}63$}
		{$4{,}38$}}{\begin{tabular}{|c|c|}
			\hline
			Thời gian (phút) & Số học sinh \\
			\hline
			$[9.5; 12.5)$    & 3           \\
			\hline
			$[12.5; 15.5)$   & 12          \\
			\hline
			$[15.5; 18.5)$   & 15          \\
			\hline
			$[18.5; 21.5)$   & 24          \\
			\hline
			$[21.5; 24.5)$   & 2           \\
			\hline
		\end{tabular}}
	\loigiai{
	Ta có $n=3+12+15+24+2=56$.\\
	Tính tứ phân vị thứ nhất $Q_1$.
	\[\dfrac{n}{4}=14 \Rightarrow Q_1\in \left[12{,}5;15{,}5\right)\Rightarrow Q_1=12{,}5+\dfrac{14-3}{12}\cdot3=\dfrac{61}{4}\]
	Tính tứ phân vị thứ ba $Q_3$.
	$$\dfrac{3n}{4}=42 \Rightarrow Q_3\in\left[18{,}5;21{,}5\right) \Rightarrow Q_3=18{,}5+\dfrac{42-30}{24}\cdot3=20$$
	Khoảng tứ phân vị của mẫu số liệu là $\Delta Q=Q_3-Q_1=4{,}75$.
	}
\end{ex}

\begin{ex}%[1D1V1-6]%[Dự án C đợt 3 - KSCL LeThanhTong-Võ Hoàng Nghĩa]
	\immini{Trên đồng hồ tại thời điểm đang xét kim giờ $OG$ chỉ đúng số $3$, kim phút $OP$ chỉ đúng số $12$. Số đo góc lượng giác mà kim giờ quét được từ lúc xét đến khi kim phút và kim giờ gặp nhau lần đầu tiên bằng
		\choice
		{$\alpha = \dfrac{\pi}{22}$}
		{$\alpha = -\dfrac{2\pi}{45}$}
		{$\alpha=-\dfrac{\pi}{21}$}
		{\True $\alpha=-\dfrac{\pi}{22}$}}{\begin{tikzpicture}[scale=0.3]
			\def\hours{0}
			\def\minutes{0}
			\def\seconds{0}
			\draw[line width=0.2cm] (0,0) circle (5.1cm);
			% Minutes
			\foreach \i in {1,2,...,60}{
					\def\angle{\i*6}
					\draw[thin] (\angle:5cm) -- (\angle:4.9cm);
				}

			% 5 minutes
			\foreach \i in {1,2,...,12}{
					\def\angle{\i*-30+90}
					\draw[thin] (\angle:5cm) -- (\angle:4.5cm);
					\node at (\angle:4cm) {\i};
				};

			% Hour hand
			\def\angle{\hours*-30 + \minutes*-0.5 + \seconds*-0.008333 -180}
			\draw[line width=0.1cm] (0,0) -- (0:2.5cm);

			% Minute hand
			\def\angle{\minutes*-6 + \seconds*-0.1 +90}
			\draw[line width=0.05cm] (0,0) -- (\angle:3.5cm);

			%% Second hand
			% \def\angle{\seconds*-6+90}
			% \draw[very thick,color=red] (\angle:-1cm) -- (\angle:4.5cm);
			% \draw[line width=0.1cm,color=red] (\angle:-1cm) -- (\angle:-0.25cm);

			% Center dot
			\draw[fill=black] (0,0) circle (0.1cm);
		\end{tikzpicture}}
	\loigiai{
		Tốc độ quay của kim phút là $2\pi$ (rad/$1$ giờ).\\
		Tốc độ quay của kim giờ là $\dfrac{1}{12}\cdot2\pi=\dfrac{\pi}{6}$  (rad/$1$ giờ).\\
		Khoảng cách ban đầu giữa hai kim giờ và kim phút là  $\dfrac{1}{4}\cdot2\pi=\dfrac{\pi}{2}$.\\
		Gọi $t$ (giờ) là thời gian hai kim phút và giờ gặp nhau lần đầu tiên.\\
		Ta có phương trình $2\pi\cdot t=\dfrac{\pi}{2}+\dfrac{\pi}{6}\cdot t\Leftrightarrow t=\dfrac{3}{11}$ (giờ).\\
		Vậy góc mà kim giờ đã quét được là $\dfrac{\pi}{6}\cdot\dfrac{3}{11}=\dfrac{\pi}{22}$ (rad).\\
		Do góc lượng giác có chiều dương ngược chiều quay kim đồng hồ. Nên góc lượng giác mà kim giờ quay được là $-\dfrac{\pi}{22}$ (rad).
	}
\end{ex}

\begin{ex}%[1D2N1-3]%[Dự án C đợt 3 - KSCL LeThanhTong-Võ Hoàng Nghĩa]
	Cho dãy số $(u_n)$ được cho bởi hệ thức truy hồi $\begin{cases} u_1 = 5 \\ u_{n+1} = u_n + n, n \ge 2 \end{cases}$. Giá trị của $u_3$ là
	\choice
	{\True $8$}
	{$10$}
	{$7$}
	{$9$}
	\loigiai{
		Ta có $u_1=5$, $u_2=u_1+1=6$, $u_3=u_2+2=8$.
	}
\end{ex}

\begin{ex}%[2H5N2-2]%[Dự án C đợt 3 - KSCL LeThanhTong-Võ Hoàng Nghĩa]
	Trong không gian $Oxy$, cho đường thẳng $d\colon\dfrac{x-1}{4} = \dfrac{-y}{2} = \dfrac{z+2}{-6}$. Vectơ nào dưới đây là một vectơ chỉ phương của $d$?
	\choice
	{$\overrightarrow{u_2} = (2;-1;3)$}
	{$\overrightarrow{u_1} = (4;2;-6)$}
	{$\overrightarrow{u_3} = (-2;1;3)$}
	{$\overrightarrow{u_4} = (1;0;2)$}
	\loigiai{
		Viết lại phương trình đường thẳng $d\colon\dfrac{x-1}{4}=\dfrac{y}{-2}=\dfrac{z+2}{-6}$.\\
		Suy ra một vectơ chỉ phương của đường thẳng $d$ là $\overrightarrow{u}=(4;-2;-6)=-2\cdot(-2;1;3)$.\\
		Suy ra vectơ $\overrightarrow{u_3}=(-2;1;3)$ cũng là một vectơ chỉ phương của đường thẳng $d$.
	}
\end{ex}

\begin{ex}%[2H2H1-2]%[Dự án C đợt 3 - KSCL LeThanhTong-Võ Hoàng Nghĩa]
	Cho tứ diện đều $ABCD$ có cạnh bằng $1$. Giá trị của biểu thức $S=\left|\overrightarrow{AB}+\overrightarrow{AD}+\overrightarrow{AC}\right|$ bằng
	\choice
	{$\dfrac{\sqrt{6}}{2}$}
	{$\sqrt{3}$}
	{$2\sqrt{3}$}
	{\True $\sqrt{6}$}
	\loigiai{
		Ta có $$\left( \overrightarrow{AB}+\overrightarrow{AC}+\overrightarrow{AD} \right)^2=AB^2+AC^2+AD^2+2\cdot\left(\overrightarrow{AB}\cdot\overrightarrow{AC}+\overrightarrow{AB}\cdot\overrightarrow{AD}+\overrightarrow{AD}\cdot\overrightarrow{AC} \right)=1^2+1^2+1^2+6\cdot1\cdot1\cdot\cos 60^\circ.$$
		Vậy $S=\sqrt{6}$.
	}
\end{ex}

\begin{ex}%[1D1H5-6]%[Dự án C đợt 3 - KSCL LeThanhTong-Võ Hoàng Nghĩa]
	Giả sử một vật giao động điều hòa xung quanh vị trí cân bằng theo phương trình
	$x(t)=3\cos\left(4t-\dfrac{\pi}{3}\right)$.
	Ở đây, thời gian $t$ tính bằng giây và $x(t)$ là li độ của vật tại thời điểm $t$ tính bằng centimet. Hãy cho biết trong khoảng thời gian từ $0$ đến $4$ giây, vật đạt li độ bằng $\dfrac{3}{2}$ cm bao nhiêu lần?
	\choice
	{\True $6$}
	{$5$}
	{$3$}
	{$4$}
	\loigiai{
		Theo giả thiết ta có phương trình
		$$3\cos \left( 4t-\dfrac{\pi }{3} \right)=\dfrac{3}{2}\Leftrightarrow \cos \left( 4t-\dfrac{\pi }{3} \right)=\dfrac{1}{2}\Leftrightarrow\cos \left( 4t-\dfrac{\pi }{3} \right)=\cos \dfrac{\pi }{3}\Leftrightarrow\hoac{&4t-\dfrac{\pi}{3}=\dfrac{\pi}{3}+k2\pi\\&4t-\dfrac{\pi}{3}=-\dfrac{\pi}{3}+k2\pi.}$$
		Thu gọn, ta được $t=\dfrac{\pi}{6}+\dfrac{k\pi}{2}$ hoặc $t=\dfrac{k\pi}{2}$, $k\in\mathbb{Z}$.\\
		Xét $0\le \dfrac{\pi}{6}+\dfrac{k\pi}{2}\le 4\Leftrightarrow-\dfrac{1}{3}\le k\le \dfrac{24-\pi}{3\pi}\Rightarrow k\in\{0;1;2\}$.\\
		Xét $0\le \dfrac{k\pi}{2}\le 4\Leftrightarrow-\dfrac{1}{3}\le k\le \dfrac{8}{\pi}\Rightarrow k\in\{0;1;2\}$.\\
		Vậy có $6$ lần thỏa mãn.
	}
\end{ex}

\begin{ex}%[1H8H6-2]%[Dự án C đợt 3 - KSCL LeThanhTong-Võ Hoàng Nghĩa]
	\immini{Cho hình chóp $S.ABCD$ có đáy $ABCD$ là hình vuông tâm $O$, đường thẳng $SA$ vuông góc với mặt phẳng đáy và $OC=\sqrt{3}SA$ (tham khảo hình vẽ). Số đo góc phẳng nhị diện $[S,BD,C]$ bằng
		\choice
		{$120^{\circ}$}
		{\True $150^{\circ}$}
		{$30^{\circ}$}
		{$60^{\circ}$}}{\begin{tikzpicture}[scale=0.7, font=\footnotesize,line join=round, line cap=round, >=stealth]
			\path
			(0,0) coordinate (A)
			++(-140:2) coordinate (B)
			++(0:3.5) coordinate (C)
			($(A)+(C)-(B)$) coordinate (D)
			($(A)+(0,3)$) coordinate (S)
			($(A)!1/2!(C)$) coordinate (O)
			;
			\foreach \i in{B,C,D}{\draw (S)--(\i);};
			\draw (B)--(C)--(D);
			\draw[dashed] (S)--(A)--(B) (C)--(A)--(D)--(B);
			\pic[draw,angle eccentricity=1.8,angle radius=2mm]{right angle=S--A--D};
			\foreach \i/\g in {A/-90,B/-90,C/-90,D/-90,S/90,O/90}
			\fill[black] (\i) circle(1pt)+(\g:3mm)node[scale=1]{$\i$};
		\end{tikzpicture}}
	\loigiai{
		Theo giả thiết ta có $\heva{& SA\perp BD\\
				& AO\perp BD\\
			}\Rightarrow SO\perp BD$.\\
		Mà $OC\perp BD$ suy ra $\left[S,BD,C\right]=\widehat{SOC}$.\\
		Xét $\triangle SAO$ vuông tại $A$, có $OA=OC=\sqrt{3}SA$.\\
		Khi đó $\tan \widehat{SOA}=\dfrac{SA}{AO}=\dfrac{SA}{\sqrt{3}SA}=\dfrac{1}{\sqrt{3}}\Rightarrow \widehat{SOA}=30^\circ$.\\
		Vậy $\left[S,BD,C\right]=\widehat{SOC}=150^\circ$.
	}
\end{ex}

\begin{ex}%[2D4V3-1]%[Dự án C đợt 3 - KSCL LeThanhTong-Võ Hoàng Nghĩa]
	\immini{Cho hàm số $y=f(x)$ có đạo hàm $f'(x)$ liên tục trên đoạn $[0;5]$ và đồ thị hàm số $f'(x)$ trên đoạn $[0;5]$ được cho như hình bên. Mệnh đề nào sau đây đúng?
		\choice
		{$f(0)=f(5)<f(3)$}
		{$f(3)<f(0)=f(5)$}
		{$f(3)<f(0)<f(5)$}
		{\True $f(3)<f(5)<f(0)$}}{\begin{tikzpicture}[scale=0.45,line join=round, line cap=round,>=stealth,thick]
			\tikzset{every node/.style={scale=1}}
			\draw[->] (-2.1,0)--(7.1,0) node[below left] {$x$};
			\draw[->] (0,-6.1)--(0,3.1) node[below left] {$y$};
			\draw (0,0) node [below left] {$O$};
			\foreach \x/\nx in {4.05/3,5/5}
			\draw[thin] (\x,1pt)--(\x,-1pt) node [below] {$\nx$};
			\foreach \y/\ny in {-5/-5,1.05/1}
			\draw[thin] (1pt,\y)--(-1pt,\y) node [left] {$\ny$};
			\draw[dashed,thin](4.05,0)--(4.05,1.05)--(0,1.05);
			\begin{scope}
				\clip (-2,-6) rectangle (7,3);
				\draw[samples=200,domain=0:5,smooth,variable=\x] plot (\x,{-8/45*((\x)^3)+49/45*((\x)^2)+0*(\x)+-5});
			\end{scope}
		\end{tikzpicture}}
	\loigiai{
		Xét đồ thị hàm số $f'(x)$ trên đoạn $[0;5]$
		ta có $f'(x)=0\Leftrightarrow \hoac{& x=a\in (0;3) \\
				& x=5.}$\\
		Khi đó ta có bảng biến thiên của hàm số $y=f(x)$ như sau
		\begin{center}
			\begin{tikzpicture}
				\tkzTabInit[nocadre=false,lgt=2,espcl=3,deltacl=0.9]
				{$x$/0.7,$P'(x)$/0.7,$P(x)$/1.5}{$0$,$a$,$5$}
				\tkzTabLine{,-,0,+,}
				\tkzTabVar{+/$f(0)$,-/$f(a)$,+/$f(5)$}
			\end{tikzpicture}
		\end{center}
		Ta có $a<3<5$ mà hàm số đồng biến trên $(a;5)$ nên $f(a)<f(3)<f(5)$.\\
		Gọi $S_1$ là phần hình phẳng giới hạn bởi các đường $y=f'(x)$, $Ox$, $x=0$, $x=a$.\\
		Gọi $S_2$ là phần hình phẳng giới hạn bởi các đường $y=f'(x)$, $Ox$, $x=0$, $x=5$.\\
		Ta có $\heva{&S_1=\displaystyle\int_0^a |f'(x)|\mathrm{\,d}x= f(0)-f(a)\\&S_2=\displaystyle\int_a^5 |f'(x)|\mathrm{\,d}x=f(5) -f(a).}$\\
		Mà $S_1>S_2\Leftrightarrow f(0)-f(a)>f(5)-f(a)\Leftrightarrow f(0)>f(5)$.\\
		Vậy $f(3)<f(5)<f(0)$.
	}
\end{ex} 
% \Closesolutionfile{ans}
% \Closesolutionfile{ansbook}

\cauds
% \Opensolutionfile{ansbook}[Ansbook/TenFile-TF]%---Nên đặt tên theo bài
% \setcounter{ex}{0}
\begin{ex}%[2D4C2-6]%[Dự án C đợt 3 - KSCL LeThanhTong-Võ Hoàng Nghĩa]
	Trong một cuộc thử tên lửa, Triều Tiên đã cho phóng một quả tên lửa có gắn đầu đạn hạt nhân với vận tốc $v(t) = \dfrac{1}{90\,000\,000}t^3 + \dfrac{1}{500}t + 1$ (m/s) trong đó $t$ đơn vị giây tính từ lúc tên lửa Triều Tiên bắt đầu phóng và dự định sẽ rơi xuống một vùng biển. Đi được $1$ giờ thì bay ngang vùng biển thuộc chủ quyền của Nhật Bản ngay lập tức Rada nhận được tín hiệu và gửi tín hiệu về căn cứ quân đội. Khi nhận được tín hiệu của Rada sau $30$ phút quân đội Nhật Bản đã cho phóng một quả tên lửa tầm trung đã xác định sẵn mục tiêu đi với gia tốc $a(t_1) = \dfrac{1}{4\,500}t_1 + \dfrac{n}{100}$ (m/s$^2$), $n > 0$ trong đó $t_1$ đơn vị giây tính từ lúc tên lửa tầm trung bắt đầu phóng.
	\choiceTF
	{Vận tốc của tên lửa tầm trung được biểu thị dưới hàm $v(t_1) = \dfrac{1}{9\,000}t_1^2 + \dfrac{n}{100}t_1$ (m/s$^2$), $n > 0$}
	{\True Kể từ khi bị Rada phát hiện đến lúc Nhật Bản phóng tên lửa thì quả tên lửa gắn đầu đạn hạt nhân đi được $1913{,}4$ km}
	{\True Sau $15$ phút phóng lên thì tên lửa tầm trung hạ được mục tiêu biết quãng đường nó đi được bằng $\dfrac{1}{2}$ quãng đường tên lửa Triều Tiên đi được trong $15$ phút đó, khi đó giá trị $n > 100$}
	{\True Giả sử hàm $h(t) = \dfrac{-5m}{648}t^2 + \dfrac{500m}{9}t + a$ $(m > 0, a \in \mathbb{R})$ (đơn vị: mét) thể hiện độ cao của quả tên lửa gắn đầu đạn hạt nhân so với mực nước biển. Khi quả tên lửa của Triều Tiên đạt độ cao lớn nhất thì quãng đường nó đi được là $483{,}12$ km}
	\loigiai{
		\begin{itemchoice}
			\itemch Vận tốc của tên lửa tầm trung được biểu thị dưới dạng hàm số như sau $$v(t_1)=\displaystyle\int{a(t_1)\mathrm{\,d}t_1}=\displaystyle\int{\left(\dfrac{1}{4500}t_1+\dfrac{n}{100} \right)\mathrm{\,d}t_1}=\dfrac{1}{9000}t_1^2+\dfrac{n}{100}t_1. \mathrm{(m/s)}$$
			\itemch Đổi $1$ giờ $= 3600$ giây và $1$ giờ $30$ phút $= 5400$ giây.\\
			Kể từ khi bị Rada phát hiện đến lúc Nhật Bản phóng tên lửa thì quả tên lửa gắn đầu đạn hạt nhân đi được là
			$$\displaystyle\int\limits_{3\,600}^{5\,400}{v(t)\mathrm{\,d}t}=\displaystyle\int\limits_{3\,600}^{5\,400}{\left(\dfrac{1}{90\,000\,000}t^3+\dfrac{1}{500}t+1 \right)\mathrm{\,d}t}=1\,913\,400\text{ (m)}=1\,913{,}4\text{ (km)}.$$
			\itemch Quãng đường tên lửa Triều Tiên đi được trong $15$ phút trước khi bị hạ là
			\[{s_{TT}}=\int\limits_{5\,400}^{6\,300}{\left(\dfrac{1}{90\,000\,000}t^3+\dfrac{1}{500}t+1 \right)\mathrm{\,d}t}=2\,025\,292{,}5 \text{ (m)}.\]
			Vì trong 15 phút đó, quãng đường tên lửa tầm trung đi được bằng $\dfrac{1}{2}$ quãng đường tên lửa Triều Tiên đi được, ta có
			\allowdisplaybreaks
			\begin{eqnarray*}
				&& s_{NB}=\dfrac{1}{2}s_{TT} \\
				&\Rightarrow& s_{NB}=1\,012\,646{,}25 \text{ (m)}\\
				&\Rightarrow& \displaystyle\int_0^{900} v(t_1)\mathrm{\,d}t_1 =1\,012\,646{,}25\\
				&\Rightarrow& \displaystyle\int_0^{900} \left( \dfrac{1}{9\,000}t_1^2+\dfrac{n}{100}t_1\right) \mathrm{\,d}t_1 =1\,012\,646{,}25\\
				&\Rightarrow& 27\,000+40\,500\cdot\dfrac{n}{100} =1\,012\,646{,}25\\
				&\Rightarrow& n \approx234{,}4>100.
			\end{eqnarray*}
			\itemch Khi quả tên lửa của Triều Tiên đạt độ cao lớn nhất thì
			\begin{eqnarray*}
				&& h'(t)=0 \\
				&\Leftrightarrow& -\dfrac{5m}{324}t+\dfrac{500m}{9}=0\\
				&\Leftrightarrow& t=3\,600.
			\end{eqnarray*}
			Khi quả tên lửa của Triều Tiên đạt độ cao lớn nhất thì quãng đường nó đi được là
			\[{s_{TT}}=\displaystyle\int\limits_0^{3600}{\left( \dfrac{1}{90\,000\,000}t^3+\dfrac{1}{500}t+1 \right)\mathrm{\,d}t}=483\,120\text{ (m)}=483{,}12 \text{ (km).}\]
		\end{itemchoice}
	}
\end{ex}

\begin{ex}%[2D1V4-3]%[Dự án C đợt 3 - KSCL LeThanhTong-Võ Hoàng Nghĩa]
	Cho hàm số $f(x)=\dfrac{x^{3}}{3}-3x-6\ln(2-x)+1$.
	\choiceTF
	{\True Đạo hàm của hàm số đã cho là $f'(x)=\dfrac{x^{3}-2x^{2}-3x}{x-2}$}
	{\True Hàm số đã cho đồng biến trong khoảng $(-\infty;-1)$}
	{\True Tổng các giá trị cực đại và cực tiểu của đồ thị hàm số bằng $\dfrac{14}{3}-6\ln(6)$}
	{Hàm số $g(x)=\dfrac{f(x)}{x^{2}+2x+2}$ có đường tiệm cận xiên có dạng $y=ax+b$. Khi đó $a+b=\dfrac{1}{3}$}
	\loigiai{
		\begin{itemchoice}
			\itemch Hàm số xác định trên khoảng $(-\infty;2)$ và
			$f'(x)=x^2-3+\dfrac{6}{2-x}=\dfrac{x^3-2x^2-3x}{x-2}$.
			\itemch Giải $f'(x)=0\Leftrightarrow x^3-2x^2-3x=0\Leftrightarrow\hoac{&x=3\\&x=-1\\&x=0.}$\\
			Bảng biến thiên
			\begin{center}
				\begin{tikzpicture}
					\tkzTabInit
					[lgt=2,espcl=3.5] % tùy chọn lgt độ dọc/ espcl độ dài 
					{$x$/1.2, $f’(x)$/1, $f(x)$/2.5} % cột đầu tiên
					{$-\infty$, $-1$,$0$, $2$,$ $} % hàng 1 cột 2
					\tkzTabLine{,+,0,-,0,+,d,} % hàng 2 cột 2
					\tkzTabVar{-/$-\infty$,+/$\frac{11}{3}-6\ln 3$,-/$1-6\ln2$,+D/$+\infty$} % hàng 3 cột 2
				\end{tikzpicture}
			\end{center}
			Hàm số đã cho đồng biến trong khoảng $(-\infty;-1)$.
			\itemch Ta có $y_\text{CĐ}=y(-1)=\dfrac{11}{3}-6\ln 3$; $y_\text{CT}=y(0)=1-6\ln 2$. \\
			Suy ra $y_\text{CĐ}+y_\text{CT}=\dfrac{11}{3}-6\ln 3+1-6\ln 2=\dfrac{14}{3}-6\ln 6$.
			\itemch Xét $g(x)=\dfrac{f(x)}{x^{2}+2x+2}$ có đường tiệm cận xiên có dạng $y=ax+b$.\\
			Ta có $a=\lim\limits_{x\to-\infty}\dfrac{g(x)}{x}=\lim\limits_{x\to-\infty}\dfrac{\dfrac{x^3}{3}-3x-6\ln (2-x)+1}{x^3+2x^2+2x}=\dfrac{1}{3}$.\\ $b=\lim\limits_{x\to-\infty}\left(g(x)-\dfrac{1}{3}x \right)=\lim\limits_{x\to-\infty}\left(\dfrac{\dfrac{x^3}{3}-3x-6\ln (2-x)+1}{x^2+2x+2}-\frac{1}{3}x \right)=-\dfrac{2}{3}$. \\
			Vậy $a+b=\dfrac{1}{3}-\dfrac{2}{3}=-\dfrac{1}{3}$.
		\end{itemchoice}
	}
\end{ex}

\begin{ex}%[2H5C3-4]%[Dự án C đợt 3 - KSCL LeThanhTong-Võ Hoàng Nghĩa]
	Trong một cuộc thi thể thao về môn bắn súng. Các vận động viên phải thực hiện bắn hạ mục tiêu đang di động trên mặt của khối cầu đặc có bán kính bằng $1$ m. Chọn hệ trục tọa độ $(Oxyz)$ trong không gian có gốc $O$ đặt tại vị trí xạ thủ $A$ ngắm bắn, xem mặt phẳng $(Oxy)$ là mặt đất, đơn vị độ dài trên mỗi trục tọa độ là $1$ m. Biết khối cầu có tâm $I(7;24;3)$ và xem đường đi của viên đạn là một đường thẳng.
	\choiceTF
	{\True Vị trí xa nhất để xạ thủ $A$ nhìn thấy và ngắm bắn mục tiêu là $25,2$ m (làm tròn đến hàng phần mười)}
	{\True Biết vận tốc viên đạn là $\dfrac{54}{5}\sqrt{65}$ km/h thì khoảng thời gian ngắn nhất để xạ thủ $A$ bắn trúng mục tiêu chưa tới $1$ giây}
	{\True Để các xạ thủ có thể dễ dàng bắn trúng mục tiêu hơn, ban tổ chức đã quyết định cho mục tiêu di chuyển trên đường tròn lớn nhất của mặt cầu và song song với mặt đất. Khi đó khoảng cách ngắn nhất từ vị trí xạ thủ $A$ ngắm bắn đến mục tiêu là $3\sqrt{65}$ m}
	{\True Xạ thủ $A$ đang ngắm ở vị trí gần mục tiêu nhất. Tại thời điểm tuyển thủ $A$ nổ súng thì mục tiêu đang ở vị trí $M(6;24;3)$ di chuyển với vận tốc $v=\arctan\left(\dfrac{24}{7}\right)$ (m/s) và đi ngược chiều kim đồng hồ. Khi đó xạ thủ $A$ bắn trúng mục tiêu}
	\loigiai{\begin{center}
			\begin{tikzpicture}[line join=round, line cap=round,thick]
				%Gọi điểm
				\path
				(3,10) coordinate (I)
				($(I)-(3,0)$) coordinate (M)
				($(I)-(0.3,0.8)$) coordinate (N)
				($(I)+(2,2.25)$) coordinate (E)
				($(I)-(2.12,2.12)$) coordinate (F)
				($(I)-(-1.3,2.7)$) coordinate (T)
				(3,1) coordinate (H)
				(-5,9) coordinate (K)
				(-6,3) coordinate (A)
				(11,3) coordinate (B)
				(-11,0) coordinate (D)
				($ (B)+(D)-(A) $) coordinate (C)
				(-5,2) coordinate (O)
				;
				%Vẽ elip
				\draw[dashed,thin]
				(M) arc (180:0:3 cm and 0.8 cm)
				;
				%Vẽ đường tròn
				\draw
				(I) circle (3 cm)
				(M) arc (-180:0:3 cm and 0.8 cm)
				;
				%Vẽ mặt phẳng
				\draw (A)--(B)--(C)--(D)--cycle
				;
				%Vẽ hệ trục Oxyz
				\draw[->] (O)--(-8,0.3)node[left]{$ x $};
				\draw[->] (O)--(-1,2)node[below]{$ y $};
				%Nối hình chiếu
				\draw
				(I)--(O)
				(N)--(O)
				(T)--(O)
				(K)--(O)
				;
				\draw[dashed]
				(I)--(K)
				(I)--(H)
				(I)--(T)
				(I)--(E)
				;
				\node at ($ (I)+(1.1,0.05) $) {$ (7,24,3) $};
				\node at ($ (H)+(0.8,-0.35) $) {$ (7,24,0) $};
				\node at ($ (K)+(0.7,0.3) $) {$ (0,0,3) $};
				\foreach \i/\g in {O/-90,I/0,N/-90,F/180,T/0,K/90,H/-90}{\draw[fill=white](\i) circle (1.5pt) ($(\i)+(\g:3.3mm)$) node[scale=1]{$\i$};}
			\end{tikzpicture}
		\end{center}
		\begin{itemchoice}
			\itemch  Điểm xa nhất mà xạ thủ $A$ thấy được là tiếp điểm $B$ của tiếp tuyến kẻ từ $O$ đến mặt cầu.\\
			Ta có $OB^2=OI^2-IB^2=7^2+{{24}^2}+3^2-1^2\Rightarrow OB=\sqrt{633}\approx 25{,}2$ (m).
			\itemch Vì vận tốc không đổi nên khoảng thời gian ngắn nhất để xạ thủ $A$ bắn trúng mục tiêu là khoảng thời gian cho quãng đường từ xạ thủ đến vị trí gần xạ thủ nhất.\\
			Ta có $OC=OI-R=\sqrt{7^2+{{24}^2}+3^2}-1=\sqrt{634}-1$ (m).\\
			$v=\dfrac{54}{5}\sqrt{65}$ (km/h) $=3\sqrt{65}$ (m/s).\\
			Mặt khác, $t=\dfrac{s}{v}=\dfrac{OC}{v}=\dfrac{\sqrt{634}-1}{3\sqrt{65}}\approx 0{,}99969<1$ giây.
			\itemch
			Gọi $H(7;24;0)$ là hình chiếu của $I$ lên mặt phẳng $(Oxy)$. Vì đường tròn lớn nhất của mặt cầu nằm trong mặt phẳng song song với mặt đất nên khoảng cách ngắn nhất là $OD$ với $D$ là một trong hai giao điểm của mặt cầu, mặt phẳng $z=3$ và mặt phẳng $(OIH)$.\\
			Ta có $ID=1$; $OI=\sqrt{634}-1$; $\cos \widehat{DIO}=\cos \widehat{IOH}=\cos (\overrightarrow{OH},\overrightarrow{OI})=\dfrac{\sqrt{7^2+{{24}^2}}}{\sqrt{7^2+{{24}^2}+3^2}}=\dfrac{25}{\sqrt{634}}$.\\
			Suy ra $OD=\sqrt{OI^2+ID^2-2OI\cdot ID\cdot\cos \widehat{DOI}}=\sqrt{634+1-2\sqrt{634}.1.\dfrac{25}{\sqrt{634}}}=3\sqrt{65}$ (m).

			\itemch
			Ta có $\overrightarrow{IM}=(-1;0;0)$, $\overrightarrow{IO}=(-7;-24;-3)\Rightarrow \widehat{MIC}=(\overrightarrow{IM},\,\overrightarrow{IO})=\arccos \left(\dfrac{-1\cdot(-7)}{1\cdot\sqrt{634}} \right)=\arccos \left(\dfrac{7}{\sqrt{634}} \right)$.\\
			Khi đó, thời gian mục tiêu di chuyển từ $M$ đến điểm $C$ là \[{t_{mt}}=\dfrac{1\cdot\arccos \left(\dfrac{7}{\sqrt{634}} \right)}{\arctan \left( \dfrac{24}{7} \right) }\approx 1{,}0016 \text{ (giây)}.\]
			Thời gian viên đạn bay đến $C$ là ${t_{vd}}=\dfrac{OC}{v}=\dfrac{\sqrt{634}-1}{3\sqrt{65}}\approx 0{,}99969$ giây.
		\end{itemchoice}
	}
\end{ex}

\begin{ex}%[0D0V2-9]%[Dự án C đợt 3 - KSCL LeThanhTong-Võ Hoàng Nghĩa]
	Sau khi học kì I năm học 2024-2025, thầy Nghĩa chủ nhiệm lớp 12B5 nhận thấy rằng lớp mình có $60\%$ học sinh có kết quả xuất sắc, $40\%$ học sinh có kết quả loại giỏi, không có học sinh khá và trung bình. Nhưng để nắm bắt chính xác hơn về năng lực tư duy môn toán của từng học sinh nên thầy Nghĩa đã cho học sinh làm bài kiểm tra toán trong $90$ phút. Sau khi chấm bài xong, thầy Nghĩa thấy rằng trong số học sinh loại giỏi có $8$ học sinh từ $9$ điểm toán trở lên và có $75\%$ học sinh xuất sắc trong các học sinh được điểm toán từ 9 trở lên. Biết lớp 12B5 có $40$ học sinh.
	\choiceTF
	{\True Tỉ lệ học sinh có điểm toán từ $9$ trở lên của lớp 12B5 là $80\%$}
	{\True Học sinh xuất sắc kiểm tra môn toán đều lớn hơn hoặc bằng $9$ điểm}
	{\True Những học sinh có điểm toán dưới $9$ điểm đều là học sinh loại giỏi}
	{\True Có $22$ học sinh kết quả xuất sắc có điểm trên $9$ biết rằng tỉ lệ học sinh có điểm toán trên $9$ điểm của học sinh giỏi bằng $37{,}5\%$ và trong số học sinh có điểm bằng $9$ có $50\%$ học sinh xuất sắc}
	\loigiai{Số học sinh xuất sắc là $60\%\cdot40=24$ (học sinh).\\
		Số học sinh giỏi là $40\%\cdot40=16$ (học sinh).
		\begin{itemchoice}
			\itemch Gọi $x$ (học sinh) là số học sinh đạt từ $9$ điểm trở lên trong các học sinh xuất sắc $(0\le x\le 24)$.\\
			Số học sinh đạt từ $9$ điểm trở lên là $x+8$ (học sinh).\\
			Theo đề bài, ta có phương trình $\dfrac{x}{x+8}\cdot100\%=75\%\Leftrightarrow x=24$.\\
			Tỉ lệ học sinh có điểm toán từ $9$ điểm trở lên của lớp 12B5 là $\dfrac{24+8}{40}\cdot100\%=80\%$.
			\itemch Theo câu a ta có số học sinh xuất sắc từ $9$ điểm trở lên là $24$ học sinh và bằng tổng số học sinh xuất sắc.
			\itemch Từ câu a ta có số học sinh dưới $9$ điểm đều là học sinh giỏi và bằng $40-(24+8)=8$ (học sinh).
			\itemch Số học sinh giỏi có điểm trên $9$ là $16\cdot37{,}5\%=6$ (học sinh).\\
			Số học sinh giỏi có điểm bằng $9$ là $8-6=2$ (học sinh).\\
			Do trong số học sinh có điểm bằng $9$ có $50\%$ học sinh xuất sắc nên số học sinh xuất sắc có điểm bằng $9$ là $2$ (bằng số học sinh giỏi có điểm bằng $9$).
		\end{itemchoice}
	}
\end{ex}
% \Closesolutionfile{ansbook}


\caukq
% \Opensolutionfile{ansbt}[Ansbook/TenFile-TLN]%---Nên đặt tên theo bài
% \setcounter{ex}{0}
\begin{ex}%[1H8V7-3]
	Cho khối chóp $S.ABCD$ có đáy là hình thoi cạnh $2$, $\widehat{ABC}=120^\circ$, $SB=2$. Mặt phẳng $(SAD)$ vuông góc với mặt đáy và cạnh bên $SA$ tạo với mặt đáy một góc $60^\circ$. Tính thể tích khối chóp $S.ABCD$.
	\shortans[]{$1$}
	\loigiai{
		\immini{
			Gọi $SH$ là đường cao của $\triangle SAD$. \\
			Ta có $\heva{&(SAD)\perp(ABCD)\\&(SAD)\cap (ABCD)=AD\\&SH\perp AD, SH \subset (SAD)} \Rightarrow SH \perp (ABCD)$.\\
			Khi đó $(SA, (ABCD))=\widehat{SAH}=60^\circ$. \\
			Đặt $x=SH$ thì $AH=\dfrac{x}{\sqrt 3}$. \\
			Xét $\triangle SHB$ vuông tại $H$, ta có
			$$BH=\sqrt{SB^2-SH^2}=\sqrt{4-x^2}.$$
			Xét $\triangle ABH$, có $\widehat{HAB}=60^\circ$ (vì $\widehat{ABC}=120^\circ$).
		}{
			\begin{tikzpicture}[>=stealth,line join=round,line cap=round,font=\footnotesize,scale=.8]
				\path
				(0,0) coordinate (D)
				(-3,-2) coordinate (A)
				(6,0) coordinate (C)
				($(C)+(A)-(D)$) coordinate (B)
				($(D)!.75!(A)$) coordinate (H)
				($(H)+(90:5)$) coordinate (S)
				($(D)!.5!(B)$) coordinate (O)
				%		($(D)!.5!(O)$) coordinate (I)
				%		($(D)!.5!(C)$) coordinate (K)
				%		($(C)!.5!(B)$) coordinate (P)
				;
				\draw (S)--(A)--(B)--(C)--(S)--(B);
				\draw[dashed]
				(S)--(D)--(A)--(C)--(D)--(B) (S)--(H)--(B)
				;
				\foreach \p/\g in {S/90, D/170, A/-90, C/0, B/-90,  O/-90, H/180}
				\draw[fill=black] (\p) circle (1pt) node[shift=(\g:2.5mm)] {$\p$};
			\end{tikzpicture}
		}
		Áp dụng định lí cô sin, ta được:
		\allowdisplaybreaks
		\begin{eqnarray*}
			&& BH^2=AH^2+AB^2-2\cdot AH\cdot AB\cdot \cos \widehat{HAB} \\
			&\Leftrightarrow& 4-x^2=\dfrac{x^2}{3}+4-2\cdot2\cdot\dfrac{x}{\sqrt 3}\cdot\cos 60^\circ \\
			&\Rightarrow& x=\dfrac{\sqrt 3}{2}
		\end{eqnarray*}
		Vậy $V_{S.ABCD}=\dfrac{1}{3}\cdot SH\cdot S_{ABCD}=\dfrac{1}{3}\cdot SH\cdot AB\cdot AD \cdot \sin 60^\circ = \dfrac{1}{3}\cdot\dfrac{\sqrt 3}{2}\cdot 2\cdot 2 \cdot \sin 60^\circ=1$.
	}
\end{ex}

\begin{ex}%[2D4C3-2]%[Dự án C đợt 3 - KSCL LeThanhTong-Võ Hoàng Nghĩa]
	Một nhóm các kĩ sư muốn xây dựng một cây cầu vòm dàn thép với giá đỡ dưới bằng thép cao cấp có hình dáng là một parabol nối từ 2 cột trụ $A$ và $B$ nằm bên dưới cây cầu. Biết hai cột trụ cách nhau $400$ m, khoảng cách từ trụ $A$ đến cây cầu là $50$ m và $AB$ song song với mặt đường.
	% \begin{center}
	% 	\includegraphics[scale=1]{images/KSCL-THPT-LeThanhTong-HCM-NH24-25}
	% \end{center}
	Gắn hệ trục $Oxy$ vào cây cầu với đơn vị trục tọa độ là $10$ m. Giá đỡ dưới bằng thép là đường cong parabol tạo với hai trục tọa độ các hình phẳng có diện tích $S_1$, $S_2$ như hình vẽ bên. Biết rằng $S_2-2S_1=\dfrac{2200}{21}$. Điểm cao nhất của giá đỡ dưới bằng thép cao cấp cách mặt đường cây cầu bao nhiêu mét? (làm tròn đến hàng phần mười)
	\begin{center}
		\begin{tikzpicture}[scale=1, font=\footnotesize, line join=round, line cap=round,>=stealth, xscale=0.2, yscale=0.2]
			%Gán số liệu.
			\def\xmin{-5};\def\ymin{-10};\def\xmax{50};\def\ymax{10};
			%Gán tọa độ.
			\coordinate (O) at (0,0);
			%Trục Oxy.
			\draw[->] (\xmin,0)--(\xmax,0) node[below]{$x$};
			\draw[->] (0,\ymin)--(0,\ymax) node[left]{$y$};
			\fill (O) node[below left]{$O$} circle(1pt);
			%Giới hạn đồ thị.
			\clip ({\xmin-0.1},{\ymin-0.1}) rectangle ({\xmax+0.1},{\ymax+0.1});
			%Vẽ đồ thị.
			\draw[thick,samples=100] plot[domain=-5:50](\x,{-1/35*(\x)^2+8/7*\x-5});
			\draw(40,-5) node[right]{$y=f(x)$};
			\draw [dashed] (40,-5)--(40,0);
			\draw [dashed] (0,-5)--(40,-5) node[midway,sloped,below]{$40$};
			\fill[fill=black](0,-5) node[left]{$-5$} circle(5pt);
			\fill[fill=black](40,-5) node[below left]{$B$} circle(5pt);
			\draw(0,-5) node [below right]{$A$};
			\draw(1.8,-1.5) node{$S_1$};
			\draw(20,2.5) node{$S_2$};
		\end{tikzpicture}
	\end{center}
	\shortans[]{$64{,}3$}
	\loigiai{
		\begin{center}
			\begin{tikzpicture}[scale=1, font=\footnotesize, line join=round, line cap=round,>=stealth, xscale=0.2, yscale=0.2]
				%Gán số liệu.
				\def\xmin{-5};\def\ymin{-10};\def\xmax{50};\def\ymax{10};
				%Gán tọa độ.
				\coordinate (O) at (0,0);
				%Trục Oxy.
				\draw[->] (\xmin,0)--(\xmax,0) node[below]{$x$};
				\draw[->] (0,\ymin)--(0,\ymax) node[left]{$y$};
				\fill (O) node[below left]{$O$} circle(1pt);
				%Giới hạn đồ thị.
				\clip ({\xmin-0.1},{\ymin-0.1}) rectangle ({\xmax+0.1},{\ymax+0.1});
				%Vẽ đồ thị.
				\draw[thick,samples=100] plot[domain=-5:50](\x,{-1/35*(\x)^2+8/7*\x-5});
				\draw(40,-5) node[right]{$y=f(x)$};
				\draw [dashed] (40,-5)--(40,0);
				\draw [dashed] (0,-5)--(40,-5) node[midway,sloped,below]{$40$};
				\fill[fill=black](0,-5) node[left]{$-5$} circle(5pt);
				\fill[fill=black](40,-5) node[below left]{$B$} circle(5pt);
				\fill[fill=black](40,0) node[above]{$C$} circle(5pt);
				\draw(0,-5) node [below right]{$A$};
				\draw(1.8,-1.5) node{$S_1$};
				\draw(38.2,-1.5) node{$S_1$};
				\draw(20,2.5) node{$S_2$};
				\draw(20,-2.5) node{$S_3$};
			\end{tikzpicture}
		\end{center}
		Parabol có dạng $(P)\colon y=ax^2+bx+c$.\\
		Vì $(P)$ đi qua $(0;-5)$ nên $f(0)=-5 \Rightarrow c=-5$.\\
		Và $(P)$ đi qua $(40;-5)$ nên $f(40)=-5 \Rightarrow 1600a+40b+c=-5 \Rightarrow 1600a+40b=0 $. \hfill (1)\\
		Ta có
		\allowdisplaybreaks
		\begin{eqnarray*}
			S_2-2S_1 &=& S_2-(S_{OABC}-S_3)=(S_2+S_3)-S_{OABC}\\
			&=&\displaystyle\int_0^{40} \left( ax^2+bx+c+5\right) \mathrm{\,d}x-5\cdot40\\
			&=& \dfrac{64000}{3}a+800b+40c.
		\end{eqnarray*}
		Suy ra $\dfrac{64000}{3}a+800b +40c= \dfrac{6400}{21}$. \hfill $(2)$\\
		Từ $(1)$ và $(2)$ ta giải được $a=-\dfrac{1}{35}$, $b=\dfrac{8}{7}$. \\
		Suy ra $(P)\colon y=-\dfrac{1}{35}x^2+\dfrac{8}{7}x-5$.\\
		Do đó $(P)$ có đỉnh $I\left( 20;\dfrac{45}{7}\right) $.\\
		Vậy điểm cao nhất của giá đỡ dưới bằng thép cao cấp cách mặt đường cây cầu $10\cdot \dfrac{45}{7} \approx 64{,}3$ m.

	}
\end{ex}

\begin{ex}%[1H8V7-9]
	Cho khối trụ có bán kính là $R$ chiều cao $h$, hai đường tròn đáy có tâm là $O$ và $O'$. Một khối nón có đỉnh trùng với $O'$ và đáy có tâm $(O; 2R)$. Gọi $V_1$ là thể tích phần khối nón nằm bên ngoài khối trụ, $V_2$ là thể tích phần khối trụ nằm bên ngoài khối nón. Tính $\dfrac{V_1}{V_2}$?
	\shortans[]{2}
	\loigiai{
		\immini{
			\begin{itemize}
				\item Thể tích khối trụ chiều cao $h$ bán kính $R$ là
				      $V=\pi R^2 h.$
				\item Thể tích khối nón chiều cao $h$, bán kính $2R$ là % đỉnh $O'$, đường tròn đáy $(O;2R)$ là
				      $V_3=\dfrac{1}{3}\pi (2R)^2h=\dfrac{4}{3}V.$
				\item Thể tích khối nón chiều cao $\dfrac{h}{2}$, bán kính $R$ là
				      $V_4=\dfrac{1}{3}\pi R^2 \dfrac{h}{2}=\dfrac{1}{6}V.$
				\item Thể tích khối trụ chiều cao $\dfrac{h}{2}$, bán kính $R$ là
				      $V_5=\pi R^2 \dfrac{h}{2}=\dfrac{1}{2}V.$
				\item Thể tích hình nón cụt chiều cao $\dfrac{h}{2}$ và bán kính hai đáy lần lượt là $R$ và $2R$ là
				      $V_6=V_3-V_4=\dfrac{4}{3}V-\dfrac{1}{6}V=\dfrac{7}{6}V.$
			\end{itemize}
		}{
			\begin{tikzpicture}[scale=.7, font=\footnotesize, line join=round, line cap=round, >=stealth]
				\def\a{2} \def\b{0.5} \def\h{6}
				\path
				(0,0) coordinate (O)			(O)+(0,\h) coordinate (O')			(180:\a cm and \b cm) coordinate (A)			(A) + (0,\h) coordinate (B)
				(A) + (2*\a,0) coordinate (C)			(C) + (0,\h) coordinate (D)
				($(O)!2!(A)$) coordinate (A') ($(O)!2!(C)$) coordinate (C')
				($(B)!.5!(A)$) coordinate (M) ($(C)!.5!(D)$) coordinate (N)
				;
				\draw
				(O') ellipse (\a cm and \b cm)        %elip trên
				(A) arc (180:360:\a cm and \b cm) %elip dưới liền
				(M) arc (180:360:\a cm and \b cm) %elip dưới liền
				(A') arc (180:360:2*\a cm and 2*\b cm) %elip dưới liền
				(A)--(B) (C)--(D) (A')--(M) (C')--(N)
				;
				\draw[dashed]
				(A) arc (180:0:\a cm and \b cm)    %elip dưới đứt
				(M) arc (180:0:\a cm and \b cm)    %elip dưới đứt
				(A') arc (180:0:2*\a cm and 2*\b cm)    %elip dưới đứt
				(M)--(O')--(N)
				;
				\draw[dashed] (O')--(O)node[midway,left]{$h$}--(C) node[midway,above]{$R$};
				\foreach \p/\r in {O/180,O'/0}		\fill (\p) circle (1pt) node[shift={(\r:3mm)}]{$\p$};
			\end{tikzpicture}
		}
		\begin{itemize}
			\item Thể tích phần khối nón nằm bên ngoài khối trụ là
			      $V_1=V_6-V_5=\dfrac{7}{6}V-\dfrac{1}{2}V=\dfrac{2}{3}V$.
			\item Thể tích khối trụ bên ngoài khối nón là
			      $V_2=V_5-V_4=\dfrac{1}{2}V-\dfrac{1}{6}V=\dfrac{1}{3}V$.
			\item Vậy $\dfrac{V_1}{V_2}=2$.
		\end{itemize}
	}
\end{ex}

\begin{ex}%[2D1V3-2]%[Dự án C đợt 3 - KSCL LeThanhTong-Võ Hoàng Nghĩa]
	Anh Nam có một cái ao với diện tích $50$ m$^2$ để nuôi cá diêu hồng. Vụ vừa qua anh nuôi với mật độ $40$ con/m$^2$ và thu được $3$ tấn cá thành phẩm. Theo kinh nghiệm nuôi cá của mình anh thấy cứ thả giảm đi $8$ con/m$^2$ thì mỗi con cá thành phẩm thu được tăng thêm $0{,}5$ kg. Để tổng năng suất cao nhất thì vụ tới anh Nam nên mua bao nhiêu cá giống để thả? (giả sử không có hao hụt trong quá trình nuôi)
	\shortans[]{$1600$}
	\loigiai{
		Ở vụ trước.
		\begin{itemize}
			\item Số con cá được nuôi là $50\cdot40=2\,000$ con.
			\item Cân nặng trung bình mỗi con cá là $\dfrac{3\,000}{2\,000} =1{,}5$ kg.
		\end{itemize}
		Ở vụ này, giả sử anh Nam giảm $8x$ con cá trên một mét vuông. Khi đó
		\begin{itemize}
			\item Mật độ cá là $40-8x$ (con/m$^2$).
			\item Số con cá được nuôi là $50(40-8x)$.
			\item Khối lượng trung bình mỗi con cá là $1{,}5+0{,}5x$ kg.
			\item Tổng khối lượng (kg) cá thành phẩm là
			      $\begin{aligned}[t]
					      f(x) & =50(40-8x)(1{,}5+0{,}5x)  \\
					           & =-200x^2 + 400x + 3\,000.
				      \end{aligned}$
			\item[] Ta có $f'(x)=-400x+400$ và $f'(x)=0\Leftrightarrow x=1$.\\
			      Bảng biến thiên của hàm số $f(x)$ như sau
			      \begin{center}
				      \begin{tikzpicture}
					      \tkzTabInit[nocadre=false,lgt=1.2,espcl=2.5,deltacl=0.6]
					      {$x$ /0.6,$f'(x)$ /0.6,$f(x)$ /2}
					      {,$1$,}
					      \tkzTabLine{,+,0,-,}
					      \tkzTabVar{-/,+/$3\,200$,-/}
				      \end{tikzpicture}
			      \end{center}
			      Dựa vào bảng biến thiên, ta thấy giá trị lớn nhất của $f(x)$ là $3\,200$ (kg) khi $x=1$.
		\end{itemize}
		Vậy để tổng năng suất của anh Nam ở vụ sau cao nhất thì vụ tới anh Nam số cá giống anh Nam nên mua để thả là $50(40-8\cdot1)=1600$ con cá.
	}
\end{ex}

\begin{ex}%[2H5C1-7]%[Dự án C đợt 3 - KSCL LeThanhTong-Võ Hoàng Nghĩa]
	\immini{
		Một cơ sở sản xuất Kem làm một mô hình Kem ốc quế lớn gồm 2 phần: Phần Kem có dạng hình cầu, phần ốc quế có dạng hình nón (như hình vẽ bên). Chủ cơ sở sản xuất muốn gắn một chiếc đèn Led lớn chiếu thẳng cây kem vào buổi tối, biết rằng chiếc đèn nằm trên mặt phẳng chứa đường tròn $(C)$ là phần tiếp xúc giữa phần Kem và phần ốc quế. Chọn hệ trục tọa độ $Oxy$ trong không gian thỏa mãn phần Kem hình cầu có tâm $I(1;2;3)$, bán kính $R_C=3$ và phần đỉnh của hình nón là điểm $H(0;1;-2)$ đáy là đường tròn có bán kính $R_N = \sqrt{6}$.
	}{
		\begin{tikzpicture}[line join = round, line cap = round, >=stealth, font=\footnotesize, scale=1]
			\tikzset{label style/.style={font=\footnotesize}}
			\def\r{1.5}

			\coordinate (I) at (0,0);
			\coordinate (a) at ($(I)+(-25:\r cm)$);
			\coordinate (b) at ($(I)+(-155:\r cm)$);
			\coordinate (a') at ($(a)!1!-90:(I)$);
			\coordinate (b') at ($(b)!1!-90:(I)$);
			\draw (intersection of  a--a' and b--b') coordinate (H);
			\draw (intersection of  I--H and a--b) coordinate (J);
			\draw[dashed] (a) arc (0:180:1.35cm and 0.35cm);
			\draw (a) arc (0:-180:1.35cm and 0.35cm);
			\draw[fill=black!20!white,opacity=0.5,draw=black] (I) circle (\r cm);
			\draw (a)--(H)--(b);
			\draw[fill=black!50!white, opacity=0.5] (a) arc (0:-180:1.35cm and 0.35cm)--(b)--(H)--(a)--cycle;
		\end{tikzpicture}
		\hspace{0.3cm}
		\begin{tikzpicture}[line join = round, line cap = round, >=stealth, font=\footnotesize, scale=1]
			\tikzset{label style/.style={font=\footnotesize}}
			\def\r{1.5}

			\coordinate[label={right}:{$I$}] (I) at (0,0);
			\coordinate (a) at ($(I)+(-25:\r cm)$);
			\coordinate (b) at ($(I)+(-155:\r cm)$);
			\coordinate (a') at ($(a)!1!-90:(I)$);
			\coordinate (b') at ($(b)!1!-90:(I)$);
			\draw (intersection of  a--a' and b--b') coordinate[label={right}:{$H$}] (H);
			\draw (intersection of  I--H and a--b) coordinate (J);
			\draw[dashed] (a) arc (0:180:1.35cm and 0.35cm);
			\draw (a) arc (0:-180:1.35cm and 0.35cm);
			\draw (I) circle (\r cm);
			\draw (a)--(H)--(b);
			\foreach \x in {I,H,J}
			\draw[fill=black] (\x) circle (1pt);
		\end{tikzpicture}
	}
	\noindent Để tối ưu hóa lượng ánh sáng chiếc đèn có thể chiếu vào cây kem người ta tính toán rằng chiếc đèn Led sẽ phải ở vị trí $M(a;b;2)$, $a\in \mathbb{Z}$ và từ điểm M kẻ được 2 tiếp tuyến với đường tròn $(C)$ sao cho góc giữa 2 tiếp tuyến đó không bé hơn $60^\circ$. Có bao nhiêu vị trí đặt chiếc đèn Led thỏa mãn yêu cầu của chủ cơ sở.
	\shortans{6}
	\loigiai{
	\immini{
		Ta có
		\allowdisplaybreaks
		\begin{eqnarray*}
			&& KI = \sqrt{R_C^2 - R_N^2} = \sqrt{3^2 - (\sqrt{6})^2} = \sqrt{3}.\\
			&& IH = \sqrt{(-1)^2 + (-1)^2 + (-5)^2} = 3\sqrt{3}.\\
			&\Rightarrow& \dfrac{IK}{IH} = \dfrac{1}{3} \\
			&\Rightarrow& \overrightarrow{IK} = \dfrac{1}{3}\overrightarrow{IH}
		\end{eqnarray*}
		Suy ra
		\[
			\heva{
				& x_K - 1 = -\dfrac{1}{3} \\
				& y_K - 2 = -\dfrac{1}{3} \\
				& z_K - 3 = -\dfrac{5}{3}
			}
			\Rightarrow
			\heva{
				&x_K = \dfrac{2}{3} \\
				&y_K = \dfrac{5}{3} \\
				&z_K = \dfrac{4}{3}
			}
			\Rightarrow K\left(\dfrac{2}{3};\dfrac{5}{3};\dfrac{4}{3}\right).
		\]
	}{
		\begin{tikzpicture}[line join = round, line cap = round, >=stealth, font=\footnotesize, scale=1]
			\tikzset{label style/.style={font=\footnotesize}}
			\def\r{1.5}

			\coordinate[label={right}:{$I$}] (I) at (0,0);
			\coordinate (a) at ($(I)+(-25:\r cm)$);
			\coordinate (b) at ($(I)+(-155:\r cm)$);
			\coordinate (a') at ($(a)!1!-90:(I)$);
			\coordinate (b') at ($(b)!1!-90:(I)$);
			\draw (intersection of  a--a' and b--b') coordinate[label={right}:{$H$}] (H);
			\draw (intersection of  I--H and a--b) coordinate[label={right}:{$K$}] (K);

			\coordinate[label={below}:{$B$}] (B) at ($(K)+(-110:1.35cm and 0.35cm)$);
			\coordinate[label={above}:{$A$}] (A) at ($(K)+(120:1.35cm and 0.35cm)$);
			\coordinate[label={left}:{$M$}] (M) at ($(K)!2.5!(b)$);

			\draw[dashed] (a) arc (0:180:1.35cm and 0.35cm);
			\draw (a) arc (0:-180:1.35cm and 0.35cm);
			\draw (I) circle (\r cm);
			\draw (a)--(H)--(b) (B)--(M)--(A) (M)--(b);
			\draw[dashed] (H)--(I) (b)--(K) (A)--(K)--(B);
			\foreach \x in {I,H,K,B,A,M}
			\draw[fill=black] (\x) circle (1pt);
		\end{tikzpicture}
	}
	Mặt phẳng $(MAB)$ đi qua $K$ và có VTPT là $\overrightarrow{IH} = (1; 1; 5)$ nên $(MAB)\colon x + y + 5z - 9 = 0$.\\
	Vì $M \in (MAB) \Rightarrow a + b + 5 \cdot 2 - 9 = 0 \Leftrightarrow b = -a - 1.$\\
	Khi đó
	$$KM = \sqrt{\left(a - \dfrac{2}{3}\right)^2 + \left(b - \dfrac{5}{3}\right)^2 + \left(2 - \dfrac{4}{3}\right)^2} = \sqrt{\left(a - \dfrac{2}{3}\right)^2 + \left(a + \dfrac{8}{3}\right)^2 + \left(2 - \dfrac{4}{3}\right)^2} = \sqrt{2a^2 + 4a + 8}.$$
	Và
	$$\widehat{AKM} = \dfrac{1}{2}\widehat{AKB} = \dfrac{1}{2} \left( 180^{\circ} - \widehat{AMB} \right) = 90^{\circ} - \dfrac{1}{2}\widehat{AMB} \le 90^{\circ} - \dfrac{1}{2} \cdot 60^{\circ} = 60^{\circ}.$$
			Vì $0^{\circ} < \widehat{AKM} \le 60^{\circ}$ nên $\dfrac{1}{2} \leq \cos \widehat{AKM} <1$. \hfill (1)\\
			Mặt khác $\cos \widehat{AKM} = \dfrac{KA}{KM} = \dfrac{\sqrt{6}}{\sqrt{2a^2 + 4a + 8}} = \sqrt{\dfrac{3}{a^2 + 2a + 4}}$. \hfill (2)\\
			Từ (1) và (2) suy ra $\dfrac{1}{2} \le \sqrt{\dfrac{3}{a^2 + 2a + 4}} < 1 \Leftrightarrow \dfrac{1}{4} \le \dfrac{3}{a^2 + 2a + 4} < 1 \Leftrightarrow 3 < a^2 + 2a + 4 \le 12$.\\
			Như vậy
			$$
				\heva{
					&a^2 + 2a - 8 \leq 0 \\
					&a^2 + 2a + 1 > 0
				} \Leftrightarrow
				\heva{
					&-4 \leq a \leq 2 \\
					&a \neq -1.
				}$$
			Vì $a \in \mathbb{Z}$ nên $a \in \{-4; -3; -2; 0; 1; 2\}$.\\
			Vậy có $6$ vị trí đặt đèn thỏa mãn.
		}
\end{ex}

\begin{ex}%[2D5C2-4]%[Dự án C đợt 3 - KSCL LeThanhTong-Võ Hoàng Nghĩa]
	Có hai hộp: hộp $I$ có $5$ quả bóng trắng và $7$ quả bóng đỏ, hộp $II$ có $10$ quả bóng trắng và $15$ quả bóng đỏ, các quả bóng có cùng kích thước và khối lượng. Lấy ngẫu nhiên hai quả bóng từ hộp $I$ bỏ vào hộp $II$. Sau đó, lấy ra ngẫu nhiên một quả bóng từ hộp $II$. Xác suất để quả bóng được lấy ra từ hộp $II$ là quả bóng được chuyển từ hộp $I$ sang, biết rằng quả bóng đó có màu trắng là $\dfrac{a}{b}$ (là phân số tối giản). Tính $a+b$.
	\shortans{14}
	\loigiai{
		Gọi $X$ là biến cố \lq\lq lấy được bóng trắng từ hộp $II$ sau khi đã chuyển $2$ bóng từ hộp $I$ sang\rq\rq.\\
		Gọi $Y$ là biến cố \lq\lq$2$ bóng được chuyển từ hộp $I$ sang hộp $II$\rq\rq.\\
		Ta có công thức Bayes: $\mathrm{P}(Y | X) = \dfrac{\mathrm{P}(X | Y) \cdot \mathrm{P}(Y)}{\mathrm{P}(X)}$. Trong đó:
		\begin{itemize}
			\item $\mathrm{P}(X | Y)$ là xác suất lấy được bóng trắng từ hộp $II$, biết rằng bóng đó được chuyển từ hộp $I$ sang.
			\item $\mathrm{P}(Y)$ là xác suất 2 bóng được chuyển từ hộp $I$ sang hộp $II$.
			\item $\mathrm{P}(X)$ là xác suất lấy được bóng trắng từ hộp $II$ sau khi chuyển $2$ bóng từ hộp $I$ sang.
		\end{itemize}
		Tính $\mathrm{P}(Y)$: Do ta không cần tính xác suất để $2$ bóng được chuyển từ hộp $I$ sang hộp $II$, biến cố này luôn xảy ra. Nên ta có thể bỏ qua yếu tố $\mathrm{P}(Y)$ trong công thức Bayes.\\
		Tính $\mathrm{P}(X)$: Có $3$ trường hợp xảy ra khi chuyển $2$ bóng từ hộp $I$ sang hộp $II$:
		\begin{itemize}
			\item TH1. Chuyển $2$ bóng trắng: Xác suất là $\dfrac{\mathrm{C}_{5}^{2}}{\mathrm{C}_{12}^{2}} = \dfrac{10}{66}$.
			      \begin{itemize}
				      \item Hộp $II$ có $12$ bóng trắng và $15$ bóng đỏ.
				      \item Xác suất lấy được bóng trắng từ hộp $II$ là $\dfrac{12}{27}$.
			      \end{itemize}
			\item TH2. Chuyển $1$ bóng trắng và $1$ bóng đỏ: Xác suất là $\dfrac{\mathrm{C}_{5}^{1} \cdot \mathrm{C}_{7}^{1}}{\mathrm{C}_{12}^{2}} = \dfrac{35}{66}$.
			      \begin{itemize}
				      \item Hộp $II$ có $11$ bóng trắng và $16$ bóng đỏ.
				      \item Xác suất lấy được bóng trắng từ hộp $II$ là $\dfrac{11}{27}$
			      \end{itemize}
			\item TH3. Chuyển $2$ bóng đỏ: Xác suất là $\dfrac{\mathrm{C}_{7}^{2}}{\mathrm{C}_{12}^{2}} = \dfrac{21}{66}$.
			      \begin{itemize}
				      \item Hộp $II$ có $10$ bóng trắng và $17$ bóng đỏ.
				      \item Xác suất lấy được bóng trắng từ hộp $II$ là $\dfrac{10}{27}$
			      \end{itemize}
		\end{itemize}
		Vậy, $\mathrm{P}(X) = \dfrac{10}{66} \cdot \dfrac{12}{27} + \dfrac{35}{66} \cdot \dfrac{11}{27} + \dfrac{21}{66} \cdot \dfrac{10}{27} = \dfrac{120 + 385 + 210}{66 \cdot 27} = \dfrac{715}{1782} = \dfrac{65}{162}$.\\
		Tính $\mathrm{P}(X|Y)$: Ta cần tính xác suất để bóng lấy ra từ hộp $II$ là bóng trắng được chuyển từ hộp $I$ sang.
		\begin{itemize}
			\item TH1. Trường hợp chuyển $2$ bóng trắng: Xác suất lấy được bóng trắng chuyển từ hộp $I$ là $\dfrac{2}{27}$.
			\item TH2. Trường hợp chuyển $1$ bóng trắng và $1$ bóng đỏ: Xác suất lấy được bóng trắng chuyển từ hộp $I$ là $\dfrac{1}{27}$.
			\item TH3. Trường hợp chuyển $2$ bóng đỏ: Xác suất lấy được bóng trắng chuyển từ hộp $I$ là $0$.
		\end{itemize}
		Vậy, $\mathrm{P}(X|Y) = \dfrac{10}{66} \cdot \dfrac{2}{27} + \dfrac{35}{66} \cdot \dfrac{1}{27} + \dfrac{21}{66} \cdot 0 = \dfrac{20 + 35}{66 \cdot 27} = \dfrac{55}{1782} = \dfrac{5}{162}$.\\
		Áp dụng công thức Bayes: $\mathrm{P}(Y|X) = \dfrac{\mathrm{P}(X|Y)}{\mathrm{P}(X)} = \dfrac{\dfrac{5}{162}}{\dfrac{65}{162}} = \dfrac{5}{65} = \dfrac{1}{13}$.\\
		Vậy $\dfrac{a}{b} = \dfrac{1}{13}$, suy ra $a = 1$ và $b = 13$. Suy ra $a + b = 1 + 13 = 14$.
	}
\end{ex}

\Closesolutionfile{ans}
\inputansbox{6,4,3}{ans/LeThanhTong}
% \begin{name}
	{\tenchude}
	{\tendethi}
	{SỞ GDĐT HÀ TĨNH}
	{\thoigian}
\end{name}

\caulc
\Opensolutionfile{ans}[Ans/TT-THPT-SGD-HaTinh-NH24-25]
%   \Opensolutionfile{ansbook}[Ansbook/TT-THPT-SGD-HaTinh-NH24-25-TN]%---Nên đặt tên theo bài
  \setcounter{ex}{0}
 %%%==============Cau_EX1==============%%%
  \begin{ex}%[Dự án C đề thi thử THPT QG 2025]%[Đỗ Chí Tâm]%[2D1B1-1]
 	Hàm số nào dưới đây đồng biến trên khoảng $(-\infty ;+\infty)$?
 	\choice
 	{$y=-x^3-2x+1$}
 	{$y=\dfrac{x-2}{x+1}$}
 	{\True $y=3x^3+3x-2$}
 	{$y=2x^3-5x+1$}
 	\loigiai{
 		Xét hàm số $y=3x^3+3x-2$ có 
 		$y'=9x^2+3>0$, $\forall x \in \mathbb{R}$.\\
 		Vậy hàm số $y=3 x^3+3 x-2$ đồng biến trên $\mathbb{R}$.
 	}
 \end{ex}
 %%%==============End-Cau_EX1==============%%%
 %%%==============Cau_EX2==============%%%
 \begin{ex}%[Dự án C đề thi thử THPT QG 2025]%[Đỗ Chí Tâm]%[2D1B2-1]
 	Cho hàm số $y=f(x)$ có đạo hàm $f'(x)=\left(x^2-4\right)(x+2)(x-3)$ và liên tục trên $\mathbb{R}$. Số điểm cực trị của hàm số đã cho là
 	\choice
 	{$5$}
 	{\True $2$}
 	{$3$}
 	{$1$}
 	\loigiai{
 		Ta có: $f'(x)=0 \Leftrightarrow\left(x^2-4\right)(x+2)(x-3)=0\Leftrightarrow(x+2)^2(x-2)(x-3)=0 \Leftrightarrow\hoac{&x=-2 \\& x=2 \\& x=3.}$\\Với $x=-2$ là nghiệm kép.\\
 		Vậy hàm số đã cho có $2$ cực trị.}
 \end{ex}
  %%%==============End-Cau_EX2==============%%%
 %%%==============Cau_EX3==============%%%
 \begin{ex}%[Dự án C đề thi thử THPT QG 2025]%[Đỗ Chí Tâm]%[2D1Y3-1]
 	Cho hàm số $y=f(x)$ có bảng biến thiên như hình bên. Giá trị lớn nhất của hàm số đã cho trên đoạn $[-2;4]$ bằng
 	\begin{center}
 		\begin{tikzpicture}[font=\normalsize,t style/.style={style=solid}]
 			\tkzTabInit[nocadre=true,lgt=1.2,espcl=2.5,deltacl=0.5]
 			{$x$ /0.75, $y'$/0.75, $y$/2}
 			{$ -\infty $,$ -1 $,$ 1 $,$ 3 $,$ +\infty $}
 			\tkzTabLine{,+,0,-,0,+,0,-, }  % z, t, d;
 			\tkzTabVar{-/$-\infty$,+/$10$,-/$-4$,+/$8$,-/$-\infty$} %+ hoac-
 		\end{tikzpicture}
 	\end{center}
 	\choice
 	{$-1$}
 	{\True $10$}
 	{$1$}
 	{$8$}
 	
 	\loigiai{
 		Từ bảng biến thiên, ta thấy giá trị lớn nhất của hàm số trên đoạn $[-2;4]$ là $10$.}
 \end{ex}
  %%%==============End-Cau_EX3==============%%%
 %%%==============Cau_EX4==============%%%
 \begin{ex}%[Dự án C đề thi thử THPT QG 2025]%[Đỗ Chí Tâm]%[2D1B5-3]
 	\immini{Cho hàm số đa thức bậc bốn $y=f(x)$ có đồ thị như hình vẽ bên. Phương trình $f(x)-1=0$ có bao nhiêu nghiệm thực phân biệt?
 	\choice
 	{\True $3$}
 	{$1$}
 	{$2$}
 	{$4$}}
 	{\begin{tikzpicture}[scale=0.6,>=stealth, font=\footnotesize, line join=round, line cap=round] 
 			\draw[->] (-4,0)--(2,0)node[below]{$x$};
 			\draw[->] (0,-3)--(0,3)node[left]{$y$};
 			\draw(-2.5,-2.5)..controls++(90:1) and++(180:0.3)..(-1.5,2.5)..controls++(0:0.3)and++(180:0.3)..(-0.3,-0.3)..controls++(0:0.5)and++(180:0.4)..(1,1)..controls++(0:0.3)and++(90:1)..(1.5,-2);
 			\draw(0,0)node[below right]{$O$};
 			\draw[dashed]
 			(-1,0)node[below]{$-1$}--(-1,1)--(0,1)node[above right]{$1$}--(1,1)--(1,0)node[below]{$1$};
 			\fill (0,0) circle (1pt);
 			\fill (0,1) circle (1pt);
 			\fill (-1,0) circle (1pt);
 			\fill (1,0) circle (1pt);
 		
 	\end{tikzpicture}}
 	\loigiai{
 		Ta thấy đường thẳng $y=1$ cắt đồ thị hàm số tại $3$ điểm phân biệt.
 		Do đó phương trình $f(x)-1=0$ có $3$ nghiệm phân biệt.}
 \end{ex}
  %%%==============End-Cau_EX4==============%%%
 %%%==============Cau_EX5==============%%%
 \begin{ex}%[Dự án C đề thi thử THPT QG 2025]%[Đỗ Chí Tâm]%[2D1B5-1]
 \immini{Đồ thị hàm số nào sau đây có hình dạng như hình vẽ?
 	\choice
 	{$y=x^3+3 x$}
 	{$y=x^3-3 x$}
 	{\True $y=x^3-3 x^2$}
 	{$y=x^3+3 x^2$}}
 	{\begin{tikzpicture}[scale=0.5,>=stealth, font=\footnotesize, line join=round, line cap=round]
 			\draw[->] (-2,0)--(4,0)node[below]{$x$};
 			\draw[->] (0,-4.7)--(0,3.2)node[left]{$y$};
 			\draw[samples=100,domain=-1.1:3.2] plot (\x,{(\x)^3-3*(\x)^2});
 			\draw(0,0)node[above left]{$O$};
 			\draw[dashed]
 			(1,0)node[above]{$1$}--(1,-2)--(0,-2)node[below right]{$-2$}
 			(2,0)node[above]{$2$}--(2,-4)--(0,-4)node[below right]{$-4$};
 		\end{tikzpicture}}
 	\loigiai{
 		Ta thấy đồ thị hàm số đi qua điểm $(2;-4)$ nên đồ thị cho là đồ thị của hàm số $y=x^3-3x^2$.}
 \end{ex}
  %%%==============End-Cau_EX5==============%%%
 %%%==============Cau_EX6==============%%%
 \begin{ex}%[Dự án C đề thi thử THPT QG 2025]%[Đỗ Chí Tâm]%[1D6H4-3]
 	Tập nghiệm của bất phương trình $\left(\dfrac{1}{2}\right)^x<\dfrac{1}{8}$ là
 	\choice
 	{\True $(3;+\infty)$}
 	{$(-\infty; 3)$}
 	{$[3;+\infty)$}
 	{$(-\infty; 3]$}
 	\loigiai{
 		Ta có: $\left(\dfrac{1}{2}\right)^x<\dfrac{1}{8} \Leftrightarrow\left(\dfrac{1}{2}\right)^x<\left(\dfrac{1}{2}\right)^3 \Leftrightarrow x>3$.}
 \end{ex}
  %%%==============End-Cau_EX6==============%%%
 %%%==============Cau_EX7==============%%%
 \begin{ex}%[Dự án C đề thi thử THPT QG 2025]%[Đỗ Chí Tâm]%[2H3Y1-1]
 	Trong không gian $Oxyz$, cho $\overrightarrow{a}=2 \overrightarrow{i}-3
 	\overrightarrow{j}+\overrightarrow{k}$. Tọa độ của $\overrightarrow{a}$ là
 	\choice
 	{$(-2;1;3)$}
 	{\True $(2;-3;1)$}
 	{$(2;1;3)$}
 	{$(2;1;-3)$}
 	
 	\loigiai{
 		Ta có $\overrightarrow{a}=2 \overrightarrow{i}-3 \overrightarrow{j}+\overrightarrow{k}$. Suy ra 
 		tọa độ của vectơ $\overrightarrow{a}$ là $(2;-3;1)$.
 	}
 \end{ex}
  %%%==============End-Cau_EX7==============%%%
 %%%==============Cau_EX8==============%%%
 \begin{ex}%[Dự án C đề thi thử THPT QG 2025]%[Đỗ Chí Tâm]%[2H3Y1-1]
 	Trong không gian $Oxyz$, cho tam giác $ABC$ với $A(1;3;4)$, $B(2;-1;0)$, $C(3;1;2)$. Tọa độ trọng tâm $G$ của tam giác $ABC$ là
 	\choice
 	{$G\left(3;\dfrac{2}{3};3\right)$}
 	{$G(2;-1;2)$}
 	{\True $G(2;1;2)$}
 	{$G(6;3;6)$}
 	\loigiai{
 		Tọa độ trọng tâm $G$ của tam giác $ABC$ là $\heva{&x_G=\dfrac{x_A+x_B+x_C}{3}=2\\& y_G=\dfrac{y_A+y_B+y_C}{3}=1 \\& z_G=\dfrac{z_A+z_B+z_C}{3}=2.}$\\
 		Tọa độ trọng tâm $G$ của tam giác $ABC$ là $G(2;1;2)$.
 	}
 \end{ex}
  %%%==============End-Cau_EX8==============%%%
 %%%==============Cau_EX9==============%%%
 \begin{ex}%[Dự án C đề thi thử THPT QG 2025]%[Đỗ Chí Tâm]%[2H3B1-2]
 	Trong không gian $Oxyz$, cho $\overrightarrow{a}=(1;-2;2)$, $\overrightarrow{b}=(-1;2;1)$. Giá trị của tích vô hướng $\overrightarrow{a} \cdot \overrightarrow{b}$ bằng
 	\choice
 	{$3$}
 	{\True $-3$}
 	{$2$}
 	{$-2$}
 	\loigiai{
 		Ta có $\overrightarrow{a} \cdot \overrightarrow{b}=1 \cdot(-1)+(-2) \cdot 2+2 \cdot 1=-3$.
 	}
 \end{ex}
  %%%==============End-Cau_EX9==============%%%
 %%%==============Cau_EX10==============%%%
 \begin{ex}%[Dự án C đề thi thử THPT QG 2025]%[Đỗ Chí Tâm]%[1H8H6-1]
 	Cho hình chóp $S \cdot ABCD$ có $ABCD$ là hình vuông cạnh $a$, tam giác $SAD$ đều. Góc giữa hai đường thẳng $BC$ và $SA$ bằng
 	\choice
 	{\True $60^{\circ}$}
 	{$30^{\circ}$}
 	{$90^{\circ}$}
 	{$45^{\circ}$}
 	
 	\loigiai{
 		Vì  $AB\parallel BC \Rightarrow(SA,BC)=(SA,AD)=\widehat{SAD}=60^{\circ}$.
 	}
 \end{ex}
  %%%==============End-Cau_EX10==============%%%
 %%%==============Cau_EX11==============%%%
 \begin{ex}%[Dự án C đề thi thử THPT QG 2025]%[Đỗ Chí Tâm]%[1D5H2-2]
 	Trong tuần lễ bảo vệ môi trường, các học sinh khối $12$ tiến hành thu nhặt vỏ chai nhựa để tái chế. Nhà trường thống kê kết quả thu nhặt vỏ chai của học sinh khối $11$ ở bảng sau:
 \begin{center}
 		\begin{tabular}{|c|c|c|c|c|c|}
 		\hline Số vỏ chai nhựa & $[10{,}5;15{,}5]$ & $[15{,}5;20{,}5]$ & $[20{,}5;25{,}5]$ & $[25{,}5;30{,}5]$ & $[30{,}5;35{,}5]$ \\
 		\hline Số học sinh & $53$ & $82$ & $48$ & $39$ & $18$ \\
 		\hline
 	\end{tabular}
 \end{center}
 	Hãy tìm trung vị của mẫu số liệu ghép nhóm trên.
 	\choice
 	{$19{,}51$}
 	{\True $19{,}59$}
 	{$20{,}1$}
 	{$18{,}3$}
 	 	\loigiai{
 		Ta có $53+82+48+39+18=240$.\\
 		Như vậy nhóm $[15{,}5;20{,}5]$ chứa trung vị.\\
 		Khi đó $C=n_1=53$.\\
 		Trung vị của mẫu số liệu là $M_e=15{,}5+\dfrac{\dfrac{240}{2}-53}{82} \cdot(20{,}5-15{,}5) \approx 19{,}59$.
 	}
 \end{ex}
  %%%==============End-Cau_EX11==============%%%
 %%%==============Cau_EX12==============%%%
 \begin{ex}%[Dự án C đề thi thử THPT QG 2025]%[Đỗ Chí Tâm]%[2D1K4-1]
 \immini{Cho hàm số $y=\dfrac{ax^2+bx+c}{x}$ ($ac \neq 0$) có đồ thị hàm số như hình vẽ. Đường tiệm cận xiên của đồ thị hàm số đã cho là đường thẳng
 	\choice
 	{Đường thẳng $y=x$}
 	{\True Đường thẳng $y=-x$}
 	{Đường thẳng $x=0$}
 	{Đường thẳng $y=2x$}}
 	{\begin{tikzpicture}[scale=0.5,>=stealth, font=\footnotesize, line join=round, line cap=round]
 			\draw[->] (-6,0)--(6,0)node[below]{$x$};
 			\draw[->] (0,-6.5)--(0,6.5)node[right]{$y$};
 			\draw[samples=100,domain=-5:-0.7] plot (\x,{(-(\x)^2-4)/(\x)});
 			\draw[samples=100,domain=0.7:5] plot (\x,{(-(\x)^2-4)/(\x)});
 			\draw[samples=100,domain=-5:5] plot (\x,{-1*(\x)});
 			\draw(0,0)node[below right]{$O$};
 			\draw[dashed]
 			(-2,0)node[below]{$-2$}--(-2,4)--(0,4)node[right]{$4$}
 			(2,0)node[above]{$2$}--(2,-4)--(0,-4)node[left]{$-4$};
 			\end{tikzpicture}}
 	\loigiai{
 		Đồ thị hàm số đi qua các điểm $(2;-4)$, $(-2;4)$ nên $\heva{&\dfrac{4a+2b+c}{2}=-4 \\& \dfrac{4a- b+c}{-2}=4} \Leftrightarrow\hoac{&4a+2b+c=-8 \\& 4a-2b+c=-8.}$\\
 		Ta có $y=\dfrac{ax^2+bx+c}{x}=ax+b+\dfrac{c}{x} \Rightarrow y'=a-\dfrac{c}{x^2}$.\\
 		Mà $x=2$ là cực trị của hàm số nên $y'(2)=0 \Leftrightarrow a-\dfrac{c}{4}=0 \Leftrightarrow 4 a-c=0$.\\
 		Từ $(1)$ và $(2)$ suy ra $a=-1$, $b=0$, $c=-4$.\\
 		Vậy hàm số đã cho là $y=\dfrac{-x^2-4}{x}$\\
 		Vì tiệm cận xiên của đồ thị hàm số đi qua $O(0;0)$ nên nó có dạng $y=mx$ ($m \neq 0$).\\
 		Ta có $m=\lim\limits_{x \rightarrow+\infty} \dfrac{y}{x}=\lim \limits_{x \rightarrow+\infty} \dfrac{-x^2-4}{x^2}=-1$.\\
 		Vậy tiệm cận xiên của đồ thị hàm số là $y=-x$.
 	}
 \end{ex}
 %%%==============HetCau_EX12==============%%%
%  \Closesolutionfile{ans}
%  \Closesolutionfile{ansbook}
 
\cauds
%   \Opensolutionfile{ansbook}[Ansbook/TT-THPT-SGD-HaTinh-NH24-25-TF]%---Nên đặt tên theo bài
%   \setcounter{ex}{0}
 %%%==============Cau_EX1==============%%%
 \begin{ex}%[Dự án C đề thi thử THPT QG 2025]%[Đỗ Chí Tâm]%[2D1K3-6]
 	Một loại thuốc được dùng cho một bệnh nhân và nồng độ thuốc trong máu của bệnh nhân được giám sát bởi bác sĩ. Biết rằng nồng độ thuốc trong máu của bệnh nhân sau khi tiêm vào cơ thể trong $t$ giờ được cho bởi công thức $c(t)=\dfrac{t}{t^2+1}$ (mg/l).
 	\choiceTF
 	{\True  Sau khi tiêm thuốc $2$ giờ thì nồng độ thuốc trong máu của bệnh nhân bằng $0{,}4$ (mg/l)}
 	{Sau khi tiêm thuốc thì nồng độ thuốc trong máu của bệnh nhân có thể vượt quá $0{,}5$ (mg/l)} 
 	{\True Sau khi tiêm thuốc $1$ giờ thì nồng độ thuốc trong máu của bệnh nhân cao nhất}
 	{\True Sau khi tiêm thuốc thì nồng độ thuốc trong máu của bệnh nhân cao nhất bằng $0{,}5$ (mg/l)}
 	\loigiai{
 		\begin{itemchoice}
 			\itemch Sau khi tiêm thuốc $2$ giờ, nồng độ thuốc trong máu bệnh nhân là $c(2)=\dfrac{2}{2^2+1}=0{,}4$ (mg/l).\\
 			Ta có $c'(t)=\dfrac{t^2+1-2 t^2}{\left(t^2+1\right)^2}=\dfrac{1-t^2}{\left(t^2+1\right)^2}$
 			$$c'(t)=0 \Leftrightarrow 1-t^2=0 \Leftrightarrow\hoac{&t=1\, (\text{nhận})\\&t=-1\, (\text{loại}).}$$
 			Bảng biến thiên
 				\begin{center}
 				\begin{tikzpicture}[font=\normalsize,t style/.style={style=solid}]
 					\tkzTabInit[nocadre=true,lgt=1.2,espcl=2.5,deltacl=0.5]
 					{$t$ /0.75, $c'(t)$/0.75, $c(t)$/2}
 					{$ 0$,$ 1 $,$ +\infty $}
 					\tkzTabLine{,+,0,-}  % z, t, d;
 					\tkzTabVar{-/$0$,+/$\tfrac{1}{2}$,-/$0$} %+ hoac-
 				\end{tikzpicture}
 			\end{center}
 			Từ bảng biến thiên ta thấy
 			\itemch Nồng độ thuốc trong máu không thể vượt quá $0{,}5$ (mg/l).
 			\itemch Sau khi tiêm thuốc $1$ giờ thì nồng độ thuốc trong máu bệnh nhân cao nhất.
 			\itemch Sau khi tiêm thuốc thì nồng độ thuốc trong máu của bệnh nhân cao nhất bằng $0{,}5$ (mg/l).
 		\end{itemchoice}	
 	}
 \end{ex}
 %%%==============HetCau_EX1==============%%%
  %%%==============Cau_EX2==============%%%
 \begin{ex}%[Dự án C đề thi thử THPT QG 2025]%[Đỗ Chí Tâm]%[2D1K3-6]
 	\immini{Một hồ nước nhân tạo được xây dựng trong một công viên giải trí. Trong mô hình minh họa, nó được giới hạn bởi các trục tọa độ và đồ thị hàm số $y=f(x)=-0{,}1 x^3+0{,}9 x^2-1{,}5 x+5{,}6$. Đơn vị đo độ dài trên mỗi trục tọa độ là $100$ m.}
 		{\begin{tikzpicture}[scale=0.5,>=stealth, font=\footnotesize, line join=round, line cap=round]
 			\draw[->] (-0.5,0)--(13,0)node[below]{$x$};
 			\draw[->] (0,-0.5)--(0,8)node[right]{$y$};
 			\draw[samples=100,domain=0:8.1] plot (\x,{-0.1*(\x)^3+0.9*(\x)^2-1.5*(\x)+5.6});
 			\fill[pattern = north east lines] (0,0)--plot[domain=0:8](\x,{-0.1*(\x)^3+0.9*(\x)^2-1.5*(\x)+5.6})--(8,0)--cycle; 
 			\draw[samples=100,domain=6:13] plot (\x,{-1.5*(\x)+18});
 			\path (10,3.7)node[rotate=-57]{$y=-1{,}5x+18$};
 			\draw(0,0)node[below left]{$O$};
 		 	\end{tikzpicture}}
 	\choiceTF
 	{Đường dạo ven hồ chạy dọc theo trục $Ox$ dài $600$ m}
 	{\True  Trên đường đi dạo ven hồ chạy dọc theo trục $Ox$, điểm cách gốc $O$ một đoạn $500$ m có khoảng cách theo phương thẳng đứng đến bờ hồ đối diện là lớn nhất}
 	{\True Khoảng cách nhỏ nhất theo phương thẳng đứng từ một điểm trên đường đi dạo ven hồ đến bờ hồ đối diện là $490$ m}
 	{\True  Trong công viên có một con đường chạy dọc theo đồ thị hàm số $y=-1{,}5 x+18$. Người ta dự định xây dựng bên bờ hồ một bến thuyền đạp nước sao cho khoảng cách từ bến thuyền đến con đường này là ngắn nhất. Biết tọa độ của điểm để xây bến thuyền này là $M(a; b)$. Giá trị của $a+5b$ bằng 43}
 	\loigiai{
 		\begin{itemchoice}
 			\itemch Xét phương trình hoành độ giao điểm của đồ thị hàm số $y=f(x)$ và trục $Ox$
 			$$-0{,}1x^3+0{,}9x^2-1{,}5x+5{,}6=0 \Leftrightarrow x=8.$$
 			Như vậy giao điểm của đồ thị hàm số $y=f(x)$ và trục $Ox$ là $A(8;0)$.\\
 			Vậy đường dạo ven hồ chạy dọc theo trục $Ox$ dài $8\cdot 100=800$ m.
 			\itemch Ta có $y=-0{,}1x^3+0{,}9x^2-1{,}5x+5{,}6\Rightarrow y'=-0{,}3x^2+1{,}8x-1{,}5$. \\
 			$y'=0 \Leftrightarrow\hoac{&x=1 \\&x=5.}$\\
 			Bảng biến thiên
 			\begin{center}
 				\begin{tikzpicture}[font=\normalsize,t style/.style={style=solid}]
 					\tkzTabInit[nocadre=true,lgt=1.2,espcl=2.5,deltacl=0.5]
 					{$x$ /0.75, $f'(x)$/0.75, $f(x)$/2}
 					{$ 0 $,$ 1 $,$ 5 $,$ 8 $}
 					\tkzTabLine{,-,0,+,0,-,}  % z, t, d;
 					\tkzTabVar{+/$5{,}6$,-/$4{,}9$,+/$8{,}1$,-/$0$} %+ hoac-
 				\end{tikzpicture}
 			\end{center}
 			Điểm cách $O$ một đoạn $500$ m có khoảng cách theo phương thẳng đứng đến bờ hồ đối diện là lớn nhất và bằng $810$ m.
 			\itemch Khoảng cách nhỏ nhất theo phương thẳng đứng từ một điểm trên đường đi dạo ven hồ đến bờ đối diện là $490$ m.
 			\itemch Gọi $d\colon y=mx+n$ ($a \neq 0$) là tiếp tuyến tại điểm $x=x_0$ của đồ thị hàm số $y=f(x)$ và song song với $y=-1{,}5x+18$.\\
 			Ta có $f'\left(x_0\right)=-0{,}3 x_0^2+1{,}8 x_0-1{,}5$.\\
 			Vì tiếp tuyến của đồ thị hàm số tại $x=x_0$ song song với $y=-1{,}5 x+18$ nên phương trình tiếp tuyến có dạng $y=-1{,}5 x+18$.\\		
 			Hay $-0{,}3 x_0^2+1{,}8 x_0-1{,}5=-1{,}5 \Leftrightarrow\hoac{&x_0=6 \, (\text{thỏa mãn}) \\& x_0=0\, (\text{loại}).}$\\
 			Với $x_0=6$ thì $f\left(x_0\right)=7{,}4$.\\
 			Tọa độ giao điểm của $d$ và đồ thị hàm số $y=f(x)$ là $M(6;7{,}4)$.\\
 			Ta có $\mathrm{d}(M,y=-1{,}5x+18)=\dfrac{|-1{,}5\cdot 6-7{,}4+18|}{\sqrt{(-1{,}5)^2+1^2}} \approx 0{,}89$ (m).\\
 			Khoảng cách ngắn nhất từ bến thuyền đến con đường là $0{,}89$ (m) ứng với điểm $M(6; 7{,}4)$ trên bờ hồ.\\
 			Vậy $a+5b=6+5\cdot 7{,}4=43$.
 		\end{itemchoice}	
 	}
 \end{ex}
 %%%==============HetCau_EX2==============%%%
 %%%==============Cau_EX3==============%%%
 \begin{ex}%[Dự án C đề thi thử THPT QG 2025]%[Đỗ Chí Tâm]%[2H3B1-1]
 	Trong không gian với hệ tọa độ $Oxyz$, cho tam giác $ABC$ với $A(1;0;-2)$, $B(-2;3;4)$, $C(4;-6;1)$
 	\choiceTF
 	{$\overrightarrow{AB}=(3;-3;6)$} 
 	{Hình chiếu vuông góc của $B$ lên trục $Ox$ là $B'(-2;3;0)$}
 	{Tồn tại 1 điểm $M$ thuộc trục hoành sao cho tam giác $MBC$ vuông tại $M$}
 	{\True Nếu $ABDC$ là hình bình hành thì tọa độ điềm $D$ là $(1;-3;7)$}
 	
 	\loigiai{
 		\begin{itemchoice}
 			\itemch $\overrightarrow{AB}=(-3;3;6)$.
 			\itemch Hình chiếu vuông góc của $B$ lên trục $Ox$ là $B'(-2;0;0)$.
 			\itemch Vì $M$ thuộc trục hoành nên $M(m;0;0)$.\\
 			Khi đó $\overrightarrow{MB}=(-2-m;3;4)$, $\overrightarrow{MC}=(4-m;-6;1)$.\\
 			Vì tam giác $MBC$ vuông tại $M$ nên $\overrightarrow{MB} \cdot \overrightarrow{MC}=0$
 			$\Rightarrow(m+2)(m-4)-18+4=0 \Leftrightarrow m=1 \pm \sqrt{23}$.\\
 			Như vậy tồn tại $2$ điểm $M$ thỏa mãn yêu cầu.
 			\itemch Gọi $D(a;b;c)$.\\
 			Vì $ABDC$ là hình bình hành nên $\overrightarrow{AB}=\overrightarrow{CD} \Rightarrow\heva{&a-4=-3\\& b+6=3\\& c-1=6} \Rightarrow\heva{&a=1 \\& b=-3 \\&c=7.}$\\
 			Vậy $D(1;-3;-5)$.
 		\end{itemchoice}	
 	}
 \end{ex}
 %%%==============HetCau_EX3==============%%%
 %%%==============Cau_EX4==============%%%
 \begin{ex}%[Dự án C đề thi thử THPT QG 2025]%[Đỗ Chí Tâm]%[1H8H5-3]
 	Cho lăng trụ đứng $ABC.A'B'C'$ có $AC=a$, $BC=2a$, $\widehat{ACB}=120^{\circ}$ có thể tích $V$. Gọi $M$ là trung điểm của $BB'$. Khi đó
 	\choiceTF
 	{Góc phẳng nhị diện $\left[A,CC',B\right]=60^{\circ}$} 
 	{\True Biết khoảng cách giữa hai mặt đáy lăng trụ bằng $2a$. Khi đó $V=a^3\sqrt{3}$}
 	{\True  $V_{M.ABC}=\dfrac{1}{6} V$}
 	{\True $d\left(C',\left(ABB'A'\right)\right)=\dfrac{a\sqrt{21}}{7}$}
 	\loigiai{
 		\immini{
 			\begin{itemchoice}
 				\itemch  Ta có
 				$\heva{&AC \perp CC'\\&BC \perp CC'} \Rightarrow\left[A,BB',C\right]=\widehat{ACB}=120^{\circ}$.
 				\itemch Ta có
 				$S_{ABC}=\dfrac{1}{2} AC \cdot BC \cdot \sin \widehat{ACB}=\dfrac{1}{2} \cdot a \cdot a \cdot \sin 120^{\circ}=\dfrac{a^2\sqrt{3}}{2}$.\\
 				Thể tích của khối lăng trụ là
 				$$V_{ABC.A'B'C'}=S_{ABC} \cdot AA'=\dfrac{a^2 \sqrt{3}}{2} \cdot 2a=a^3 \sqrt{3}.$$
 				\itemch  Ta có $V_{M.ABC}=\dfrac{1}{3} MB \cdot S_{ABC}=\dfrac{1}{3} \cdot \dfrac{1}{2} BB' \cdot S_{ABC}=\dfrac{1}{6}V$
 				\itemch Kẻ $CH\perp AB$ tại $H$.
 				Mà $AA' \perp CH \Rightarrow  CH \perp\left(ABB'A'\right) \Rightarrow \mathrm{d}\left(C',\left(ABB'A'\right)\right)=CH$.\\
 				Ta có: $AB=\sqrt{AC^2+BC^2-2AC \cdot BC \cos \widehat{ACB}}=\sqrt{a^2+4a^2-2 \cdot a \cdot 2a \cos 120^{\circ}}=a\sqrt{7}$.\\
 				$S_{ABC}=\dfrac{1}{2} CH \cdot AB \Rightarrow CH=\dfrac{2 S_{ABC}}{AB}=\dfrac{2 \cdot \dfrac{a^2 \sqrt{3}}{2}}{a \sqrt{7}}=\dfrac{a \sqrt{3}}{\sqrt{7}}=\dfrac{a \sqrt{21}}{7}$.
 			\end{itemchoice}
 		 }{
    \begin{tikzpicture}[scale=0.8]
  %%\draw[gray!20] (-3,-2) grid (6,5);
  \path (0,0) coordinate (A)  
  (2,-2) coordinate (B)
  (5,0) coordinate (C)
  (0,5) coordinate (A')
  (2,3) coordinate (B')
  ($(B)!0.5!(B')$) coordinate (M)
  (5,5) coordinate (C');
  \draw[dashed] (A)--(C);
  \draw (A)--(B)--(C)--(C')--(B')--(A')--(A) (B)--(B') (A')--(C');
  \foreach \p/\q in {A/180, B/-90, C/-90, A'/90, B'/70,C'/90, M/180}
  \fill[blue] (\p) circle(2pt) node[shift={(\q:3mm)}]{\bfseries $\p$};
   \end{tikzpicture}
 	      }
 		}
 	\end{ex}
 %%%==============HetCau_EX4==============%%%
%  \Closesolutionfile{ansbook}
 

\caukq
% \Opensolutionfile{ansbt}[Ansbook/TT-THPT-SGD-HaTinh-NH24-25-TLN]%---Nên đặt tên theo bài
% \setcounter{ex}{0}
 %%%==============Cau_EX1==============%%%
\begin{ex}%[Dự án C đề thi thử THPT QG 2025]%[Đỗ Chí Tâm]%[1D1V4-8]
	Cho đồ thị hàm số $f(x)=2 \sin x$ như hình vẽ bên. Tính diện tích tam giác $ABC$.
	\begin{center}
		\begin{tikzpicture}[thick,>=stealth,x=1cm,y=1cm,scale=0.9] 
		\draw[->] (0,0) -- (10,0) node[below] {\small $x$};
		\draw[->] (0,-2.3) -- (0,2.3) node[right] {\small $y$};
		\draw[thick,samples=100,domain=0:10] plot(\x,{2*sin((\x)*180/pi)});
		\draw[dashed] (0,2)--(10,2);
		\draw[dashed] (0,-2)--(10,-2);
		\path (1.57,2)node[above]{$B$} (7.85,2)node[above]{$C$} (0,-2)node[left]{$A$} ;
		\draw[red] (1.57,2)--(7.85,2)--(0,-2)--cycle;
		
	\end{tikzpicture}
\end{center}
	\shortans[]{12{,}6}
	\loigiai{
		Ta có $-1\leq \sin x \leq 1 \Rightarrow-2 \leq 2 \sin x \leq 2 \Rightarrow-2 \leq f(x) \leq 2$.\\
		Xét $f(x)=2 \Leftrightarrow 2 \sin x=2 \Leftrightarrow \sin x=1 \Leftrightarrow x=\dfrac{\pi}{2}+k 2 \pi$ ($k \in \mathbb{Z}$).\\
		Với $x>0$ thì $\dfrac{\pi}{2}+k 2 \pi>0 \Leftrightarrow k>-\dfrac{1}{4}$.\\
		Mà $k \in \mathbb{Z} \Rightarrow k \in\{0;1;2;\ldots\}$.\\
		Với $k=0$ ta có $x_B=\dfrac{\pi}{2} \Rightarrow B\left(\dfrac{\pi}{2};2\right)$.\\
		Với $k=1$ ta có $x_C=\dfrac{5 \pi}{2} \Rightarrow C\left(\dfrac{5 \pi}{2};2\right)$.\\
		Ta có $A(0;-2)$.\\
		Suy ra $\mathrm{d}(A,BC)=2OA=2\cdot 2=4$.\\
		$BC=\sqrt{(2 \pi)^2+0^2}=2 \pi$.\\
		Vậy $S_{ABC}=\dfrac{1}{2} \mathrm{d}(A,BC) \cdot BC=\dfrac{1}{2} \cdot 4 \cdot 2 \pi=4 \pi \approx 12{,}6$.
	}
\end{ex}
%%%==============HetCau_EX1==============%%%
%%%==============Cau_EX2==============%%%
\begin{ex}%[Dự án C đề thi thử THPT QG 2025]%[Đỗ Chí Tâm]%[0D0V2-9]
	Trong đề kiểm tra $15$ phút môn Toán có $20$ câu trắc nghiệm. Mỗi câu trắc nghiệm có $4$ phương án trả lời, trong đó chỉ có một phương án trả lời đúng. An giải chắc chắn đúng $10$ câu, $10$ câu còn lại lựa chọn ngẫu nhiên đáp án. Biết rằng mỗi câu trả lời đúng được $0{,}5$ điểm, trả lời sai không bị trừ điểm. Xác suất để An đạt được đúng $8$ điểm là $p$. Khi đó, $100p$ bằng
	\shortans[]{1{,}6}
	\loigiai{
		Vì An chắc chắn giải đúng $10$ câu nên chắc chắn An đã được $5$ điểm.\\
		Để An được $8$ điểm thì An cần làm đúng thêm $6$ câu nữa.\\
		Chọn $6$ câu trong số $10$ câu còn lại có $\mathrm{C}_{10}^6$ cách.\\
		Mỗi câu có $4$ phương án trả lời nên xác suất đúng $1$ câu là $0{,}25$, xác suất sai $1$ câu là $0{,}75$.\\
		Vậy xác suất để An được $8$ điểm là $\mathrm{C}_{10}^6 \cdot 0{,}25^6 \cdot 0{,}75^4 \approx 0{,}016$.\\
		Vậy $100p=1{,}6$.	
		
	}
\end{ex}
%%%==============HetCau_EX3==============%%%
%%%==============Cau_EX3==============%%%
\begin{ex}%[Dự án C đề thi thử THPT QG 2025]%[Đỗ Chí Tâm]%[2D4V3-2]
	\immini[thm]{Một công ty có ý định thiết kế một logo hình vuông có độ dài nửa đường chéo bằng $4$. Biểu tượng $4$ chiếc lá (được tô màu) được tạo thành bởi các đường cong đối xứng với nhau qua tâm của hình vuông và qua các đường chéo. Một trong số các đường cong ở nửa bên phải của logo là một phần của đồ thị hàm số bậc ba dạng $y = ax^3 + bx^2 - x$ với hệ số $a<0$. Để kỷ niệm ngày thành lập $2/3$, công ty thiết kế để tỉ số diện tích được tô màu so với phần không được tô màu bằng $\dfrac{2}{3}$. Tính $2a + 2b$.
	}
	{\begin{tikzpicture}[scale=0.7, font=\footnotesize, line join=round, line cap=round,>=stealth]
			\foreach \i in {0,1,2,3}{
				\fill[pattern = north east lines,draw,smooth,rotate=90*\i] plot [domain=0:3](\x,{-(\x)*(\x-3)^2/8})--plot [domain=3:0](\x,{(\x)*(\x-3)^2/8})--cycle;
				\draw (0,0)--(90*\i:3)--(90*\i+90:3);
			}
		\end{tikzpicture}
	}
	\shortans[]{$0{,}8$}
	\loigiai
	{
		Ta có nửa đường chéo hình vuông có độ dài là $4$, cạnh hình vuông sẽ là $4\sqrt{2}$ và diện tích hình vuông là $32$, khi đó ta có được diện tích phần tô màu là $\dfrac{64}{5}$.\\
		Gọi $f(x) = ax^3 + bx^2 - x$ là hàm số bậc ba biểu diễn đường cong trên logo.\\
		Ta có $x = 4$ là nghiệm của phương trình nên $64a + 16b - 4 = 0 \Leftrightarrow 4a+b=1$.\quad (1)\\
		Do đó phương trình $f(x) = 0$ sẽ có các nghiệm là $0$, $4$.\\
		Khi đó diện tích hình phẳng $S$ giới hạn bởi các đường $y = f(x)$, trục $Ox$, đường thẳng $x = 0$, $x = 4$ là
		\begin{eqnarray*}
			S&= & \displaystyle \int \limits_{0}^{4} |ax^3+bx^2-x|\mathrm{\,d}x = -\displaystyle \int \limits_{0}^{4} (ax^3+bx^2-x)\mathrm{\,d}x\\
			&= & -\left(\dfrac{ax^4}{4}+\dfrac{bx^3}{3}-\dfrac{x^2}{2}\right)\Bigg|_0^4 = -64a-\dfrac{64}{3}b+8.
		\end{eqnarray*}
		Mà $S = \dfrac{1}{8}\cdot \dfrac{64}{5} = \dfrac{8}{5}$ nên $-64a-\dfrac{64}{3}b+8 = \dfrac{8}{5} \Leftrightarrow 64a+\dfrac{64}{3}b=\dfrac{32}{5}$.\quad (2)\\
		Từ $(1)$ và $(2)$, ta có $\heva{&4a+b=1\\&64a+\dfrac{64}{3}b=\dfrac{32}{5}} \Leftrightarrow \heva{&a = -\dfrac{1}{20}\\&b=\dfrac{9}{20}} \Rightarrow 2a+2b = \dfrac{4}{5}$.
	}
\end{ex}
%%%==============HetCau_EX4==============%%%
%%%==============Cau_EX4==============%%%
\begin{ex}%[Dự án C đề thi thử THPT QG 2025]%[Đỗ Chí Tâm]%[2D1G3-6]
	Giả sử tỉ lệ sinh của tỉnh $A$ tuân theo quy luật logistic được mô hình hóa bằng hàm số $f(t)=\dfrac{200}{1+4 \mathrm{e}^{-t}}$, $t \geq 0$, $t \in \mathbb{N}$, trong đó thời gian $t$ được tính bằng tháng. Khi đó đạo hàm $f'(t)$ sẽ biểu thị tốc độ tăng dân số của tỉnh $A$. Hỏi sau bao nhiêu tháng tốc độ tăng trưởng của dân số tỉnh $A$ là lớn nhất?
	\shortans[]{2}
	\loigiai{
		Ta có: $f'(t)=200 \cdot \dfrac{4 \mathrm{e}^{-t}}{\left(1+4 \mathrm{e}^{-t}\right)^2}=\dfrac{800 \mathrm{e}^{-t}}{\left(1+4 \mathrm{e}^{-t}\right)^2}$
		\begin{eqnarray*}
			f^{\prime \prime}(t)&=&800 \cdot \dfrac{-\mathrm{e}^{-t}\left(1+4 \mathrm{e}^{-t}\right)^2-\mathrm{e}^{-t} \cdot 2 \cdot\left(-4 \mathrm{e}^{-t}\right)\left(1+4 \mathrm{e}^{-t}\right)}{\left(1+4 \mathrm{e}^{-t}\right)^4}\\
			&=&800 \cdot \dfrac{4\left(\mathrm{e}^{-t}\right)^2-\mathrm{e}^{-t}}{\left(1+4 \mathrm{e}^{-t}\right)^2}\\
			&=&\dfrac{800 \mathrm{e}^{-t}}{\left(1+4 \mathrm{e}^{-t}\right)^3}\left(4 \mathrm{e}^{-t}-1\right).
		\end{eqnarray*}
		$$f^{\prime \prime}(t)=0 \Leftrightarrow 4 \mathrm{e}^{-t}-1=0 \Leftrightarrow \mathrm{e}^{-t}=\dfrac{1}{4} \Leftrightarrow t=\ln 4.$$
		Bảng biến thiên
		\begin{center}
			\begin{tikzpicture}[font=\normalsize,t style/.style={style=solid}]
				\tkzTabInit[nocadre=true,lgt=1.2,espcl=2.5,deltacl=0.5]
				{$t$ /0.75, $f^{\prime\prime}(t)$/0.75, $f'(t)$/2}
				{$ 0 $,$\ln4 $,$+\infty $}
				\tkzTabLine{,+,0,-,}  % z, t, d;
				\tkzTabVar{-/$32$,+/$50$,-/$0$} %+ hoac-
			\end{tikzpicture}
		\end{center}
		Từ bảng biến thiên ta thấy giá trị lớn nhất của $f'(t)$ là $50$ xảy ra tại $t=\ln 4$.\\
		Vậy sau khoảng $2$ tháng thì tốc độ tăng trưởng dân số của tỉnh $A$ là lớn nhất.
		
	}
\end{ex}
%%%==============HetCau_EX4==============%%%
%%%==============Cau_EX5==============%%%
\begin{ex}%[Dự án C đề thi thử THPT QG 2025]%[Đỗ Chí Tâm]%[2D1G3-6]
	\immini{Một máy bay trình diễn có đường bay gắn với hệ trục $Oxy$ được mô phỏng như hình vẽ, trục $Ox$ gắn với mặt đất. Đường bay có dạng là một phần của đồ thị của hàm phân thức bậc hai trên bậc nhất $y=f(x)$ có đường tiệm cận đứng $x=2$. Điểm $G$ là giao điểm của đường tiệm cận xiên của đồ thị hàm số $y=f(x)$ và trục $Ox$ được gọi là điểm giới hạn. Biết rằng máy bay xuất phát tại vị trí $A$ cách gốc tọa độ $O$ một khoảng $2{,}5$ đơn vị và máy bay khi ở vị trí cao nhất cách điểm xuất phát $1{,}5$ đơn vị theo phương song song với trục $Ox$ và cách mặt đất $4{,}5$ đơn vị. Vị trí máy bay tiếp đất cách điểm giới hạn một khoảng bằng bao nhiêu?}
	{\begin{tikzpicture}[scale=0.5,>=stealth, font=\footnotesize, line join=round, line cap=round]
			\draw[->] (-0.5,0)--(11,0)node[below]{$x$};
			\draw[->] (0,-1.8)--(0,9)node[right]{$y$};
			\draw[samples=100,domain=2.4:10.3] plot (\x,{(-(\x)^2+12.5*(\x)-25)/((\x)-2)});
			\draw[samples=100,domain=1.6:10.6] plot (\x,{(-(\x)+10.5)});
			\draw(2,-1.8)--(2,9);
			\draw(0,0)node[below right]{$O$};
			\draw (4,4.5)circle (1.3pt)node[above]{$B$}(2.5,0)circle (1.3pt)node[below right]{$A$}(10,0)circle (1.3pt)node[below]{$C$}(10.5,0)circle (1.3pt)node[above]{$G$};%--(-2,4)--(0,4)node[right]{$4$}
		%	(2,0)node[above]{$2$}--(2,-4)--(0,-4)node[left]{$-4$};
	\end{tikzpicture}}
	\shortans[]{0{,}5}
	\loigiai{
		Hàm số bậc hai trên bậc nhất có dạng $y=f(x)=\dfrac{ax^2+bx+c}{x-2}$.\\
		Theo giả thiết ta có $A(2{,}5;0)$, $B(4; 4{,}5)$.\\
		Vì $A$, $B$ thuộc đồ thị hàm số nên $\heva{&6{,}25a+2{,}5b+c=0 \\& 16a+4b+c=9.}$\\
		Ta có
		$ f'(x)=\dfrac{ax^2-4ax-2b-c}{(x-2)^2}$.\\
		$f'\left(x_B\right)=0 \Rightarrow \dfrac{a \cdot 4^2-4a \cdot 4-2b-c}{(4-2)^2}=0 \Rightarrow 2b+c=0$.\\
		Từ $(1)$ và $(2)$ suy ra $a=-1$, $b=12{,}5$, $c=-25$.\\
		Khi đó $f(x)=\dfrac{-x^2+12{,}5x-25}{x-2}=-x+10{,}5-\dfrac{4}{x-2}$.\\
		\[\lim\limits_{x \rightarrow+\infty}[f(x)-(-x+10{,}5)]=\lim \limits_{x \rightarrow+\infty}\left(-x+10{,}5-\dfrac{4}{x-2}+x-10,5\right)=\lim \limits_{x \rightarrow+\infty} \dfrac{-4}{x-2}=0.\]		
		Vậy $y=-x+10{,}5$ là tiệm cận xiên của đồ thị hàm số.\\
		Tọa độ giao điểm $G$ là nghiệm của hệ phương trình $\heva{&y=-x+10{,}5 \\& y=0} \Leftrightarrow\heva{&x=10{,}5 \\& y=0} \Rightarrow G(10{,}5 0)$.\\
		Tương tự ta tìm được $A(2{,}5;0)$, $C(10;0)$.\\
		$CG=x_G-x_C=10{,}5-10=0{,}5$.\\
		Vậy vị trí máy bay tiếp đất cách điểm giới hạn một khoảng $0{,}5$ đơn vị.	
	}
\end{ex}
%%%==============HetCau_EX5==============%%%
%%%==============Cau_EX6==============%%%
\begin{ex}%[Dự án C đề thi thử THPT QG 2025]%[Đỗ Chí Tâm]%[0H5C2-6]
	\immini{Có ba lực cùng tác động vào một cái bàn như hình vẽ. Trong đó hai lực $\overrightarrow{F}_1, \overrightarrow{F}_2$ có giá nằm trên mặt phẳng chứa mặt bàn, tạo với nhau một góc $110^{\circ}$ và có độ lớn lần lượt là $9$ N, $4$ N, lực $\overrightarrow{F}_3$ vuông góc với mặt bàn và có độ lớn $7$ N. Độ lớn hợp lực của ba lực trên là $a$ (N), tìm giá trị của $a$ (\textit{kết quả làm tròn đến hàng đơn vị})}{\includegraphics[scale=0.4]{images/TT-THPT-SGD-HaTinh-NH24-25}}
	\shortans{11}
	\loigiai{
		Ta có \begin{eqnarray*}
			\left|\overrightarrow{F}_1+\overrightarrow{F}_2+\overrightarrow{F}_3\right|&=&\sqrt{\left(\overrightarrow{F}_1+\overrightarrow{F}_2+\overrightarrow{F}_3\right)} \\
			& =&\sqrt{F_1^2+F_2^2+F_3^2+2 \overrightarrow{F}_1 \overrightarrow{F}_2+2 \overrightarrow{F}_2 \overrightarrow{F}_3+2 \overrightarrow{F}_3 \overrightarrow{F}_1} \\
			& =&\sqrt{F_1^2+F_2^2+F_3^2+2 F_1 F_2 \cos 110^{\circ}+2 F_2 F_3 \cos 90^{\circ}+2 F_1 F_3 \cos 90^{\circ}} \\
			& =&\sqrt{9^2+4^2+7^2+2 \cdot 9 \cdot 4 \cdot \cos 10^{\circ}+2 \cdot 4 \cdot 7 \cdot 0+2 \cdot 9 \cdot 7 \cdot 0} \approx 11\,(\text{N}).
		\end{eqnarray*} 
}
\end{ex}
 \Closesolutionfile{ans}
\inputansbox{6,4,3}{ans/TT-THPT-SGD-HaTinh-NH24-25}
% \begin{name}
	{\tenchude}
	{\tendethi}
	{SỞ GDĐT TUYÊN QUANG}
	{\thoigian}
\end{name}

\caulc
\Opensolutionfile{ans}[Ans/TT-THPT-SGD-TuyenQuang-NH24-25]
%   \Opensolutionfile{ansbook}[Ansbook/TT-THPT-SGD-TuyenQuang-NH24-25-TN]%---Nên đặt tên theo bài
  \setcounter{ex}{0}
%%%==============Cau_EX1==============%%%
\begin{ex}%[Dự án C THPTQG 2025]%[ĐỖ CHÍ TÂM]%[2H1B1-1]
	Cho hình lập phương $ABCD.EFGH$ cạnh bằng $a$. Giá trị của $\vec{AC}\cdot\vec{EG}$ bằng
	\choice
	{$-a^2$}
	{$a^2$}
	{$-2a^2$}
	{\True$2a^2$}
	\loigiai{
		Ta có hình lập phương $ABCD.EFGH\Rightarrow \vec{AC}=\vec{EG}\Rightarrow \vec{AC}\cdot\vec{EG}=\vec{AC}\cdot\vec{AC}=AC^2$.\\
		Vì tam giác $ABC$ vuông tại $B$ nên $AC^2=BA^2+BC^2=a^2+a^2=2a^2$.
	}
\end{ex}
%%%==============HetCau_EX1==============%%%
%%%==============Cau_EX2==============%%%
\begin{ex}%[Dự án C THPTQG 2025]%[ĐỖ CHÍ TÂM]%[1D6N4-2]
	Tập nghiệm $S$ của phương trình $2^{x^2+7x+10}=1$ là 
	\choice
	{$S=\{2; 5\}$}
	{\True $\{-5; -2\}$}
	{$S=\{-5; 2\}$}
	{S=$\left\{\dfrac{-7-\sqrt{13}}{2}; \dfrac{-7+\sqrt{13}}{2}\right\}$}
	\loigiai{
		Ta có $2^{x^2+7x+10}=1\Rightarrow 2^{x^2+7x+10}=2^0\Rightarrow x^2+7x+10=0\Rightarrow x=-5\text{ hoặc }x=-2$
	}
\end{ex}
%%%==============HetCau_EX2==============%%%
%%%==============Cau_EX3==============%%%
\begin{ex} %[Dự án C THPTQG 2025]%[ĐỖ CHÍ TÂM]%[2D1Y2-2]
	\immini[thm]{
	Cho hàm số $y=f(x)$ có bảng biến thiên như hình vẽ. Tọa độ điểm cực đại của đồ thị hàm số là
	\choice
	{$(0; 2)$}
	{\True $(-2; 0)$}
	{$(0; -2)$}
	{$(2; 0)$}
	}{
	 \begin{tikzpicture}
	 	\tkzTabInit[nocadre=true,lgt=1.2,espcl=2.5,deltacl=0.6]
	 	{$x$ /0.6,$f'(x)$ /0.6,$f(x)$ /2}
	 	{$-\infty$,$0$,$3$,$+\infty$}
	 	\tkzTabLine{,+,$0$,-,$0$,+,}
	 	\tkzTabVar{-/$-\infty$, +/$2$,-/$-4$,+/$+\infty$}
	 \end{tikzpicture}}
	\loigiai{	
		Ta thấy tại điểm $x = 0$ thì $f'(x)$ đổi dấu từ dương sang âm. Do đó điểm cực đại của đồ thị hàm số là $(0;2)$.
		
	}
\end{ex}
%%%==============HetCau_EX3==============%%%
%%%==============Cau_EX4==============%%%
\begin{ex}%[Dự án C THPTQG 2025]%[ĐỖ CHÍ TÂM]%[2H3Y1-1]
	Trong không gian với hệ trục tọa độ Oxyz , cho $2$ điểm $A (1; -2; -3)$ và $B ( 7 ; -14 ; 11 )$. Tọa độ trung điểm của đoạn thẳng $AB$ là
	\choice
	{$(-4; 8; -4)$}
	{\True $(4; -8; 4)$}
	{$(3; -6; 7)$}
	{$(-3; 6; -7)$}
	\loigiai{
		Gọi $M(x_M; y_M; z_M)$ là trung điểm của đoạn thẳng $AB$, ta có $$\heva{&x_M=\dfrac{x_A+x_B}{2}=\dfrac{1+7}{2}=4\\&y_M=\dfrac{y_A+y_B}{2}=\dfrac{-2-14}{2}=-8\\&z_M=\dfrac{z_A+z_B}{2}=\dfrac{-3+11}{2}=4.}$$
		Vậy $M(4; -8; 4)$.
	}
\end{ex}
%%%==============HetCau_EX4==============%%%
%%%==============Cau_EX5==============%%%
\begin{ex}%[Dự án C THPTQG 2025]%[ĐỖ CHÍ TÂM]%[1D2N2-4]
	Cho cấp số cộng $(u_n)$ có $u_1=1$ và công sai $d=3$. Số hạng $u_3$ của cấp số cộng là
	\choice
	{$10$}{\True $7$}{$9$}{$4$}
	\loigiai{
		Ta có $u_3=u_1+2d=1+2\cdot 3=7$.
	}
\end{ex}
%%%==============HetCau_EX5==============%%%
%%%==============Cau_EX6==============%%%
\begin{ex}%[Dự án C THPTQG 2025]%[ĐỖ CHÍ TÂM]%[2D1Y5-3]
	\immini{Cho hàm số $y=\dfrac{ax+b}{cx+d}$ có đồ thị như hình vẽ bên. Tọa độ giao điểm của đồ thị hàm số đã cho với trục hoành là
	\choice
	{$(0; 2)$}{$(-2; 0)$}{$(0; -2)$}{\True $(2; 0)$}}{
		\begin{tikzpicture}[scale=0.5, line join=round, line cap=round, >=stealth,font=\scriptsize]
			\tikzset{every node/.style={scale=1}}
			\def\xmin{-5}\def\xmax{4}\def\ymin{-3}\def\ymax{5}
			\draw[->] (\xmin-0.2,0)--(\xmax+0.2,0) node[below]{$x$};
			\draw[->] (0,\ymin-0.2)--(0,\ymax+0.2) node[right]{$y$};
			\draw (0,0) node[below left]{$O$};
			\foreach \x in {-1}\draw (\x,0.1)--(\x,-0.1) node[below left]{$\x$};
			\foreach \x in {2}\draw (\x,0.1)--(\x,-0.1) node[below]{$\x$};
			\foreach \y in {1}\draw (0.1,\y)--(-0.1,\y) node[above left]{$\y$};
			\foreach \y in {-2}\draw (0.1,\y)--(-0.1,\y) node[right]{$\y$};
			\clip (\xmin,\ymin) rectangle (\xmax,\ymax);
			\draw[dashed] (\xmin,1.0)--(\xmax,1.0);
			\draw[dashed] (-1.0,\ymin)--(-1.0,\ymax);
			\draw[thick,smooth,samples=200,domain=\xmin:-1.1] plot (\x,{(1*(\x)+-2)/(1*(\x)+1)});
			\draw[thick,smooth,samples=200,domain=-0.9:\xmax] plot (\x,{(1*(\x)+-2)/(1*(\x)+1)});
		\end{tikzpicture}
	}
	\loigiai{
		Từ hình vẽ, ta suy ra đồ thị hàm số đã cho cắt trục hoành tại điểm $(2; 0)$.
	}
\end{ex}
%%%==============HetCau_EX6==============%%%
%%%==============Cau_EX7==============%%%
\begin{ex}%[Dự án C THPTQG 2025]%[ĐỖ CHÍ TÂM]%[2H3Y2-5]
	Trong không gian với hệ trục tọa độ $Oxyz$, cho hai véc-tơ $\vec u=(-1; 1; 0)$, $\vec v=(0; -1; 0)$. Góc giữa hai véc-tơ đã cho bằng
	\choice
	{$120^\circ$}{$60^\circ$}{\True $135^\circ$}{$45^\circ$}
	\loigiai{
		Ta có $$\cos(\vec u, \vec v)=\dfrac{\vec u\cdot\vec v}{|\vec u|\cdot|\vec v|}=\dfrac{-1\cdot 0+1\cdot(-1)+0\cdot 0}{\sqrt{(-1)^2+1^2+0^2}\sqrt{0^2+(-1)^2+0^2}}=-\dfrac{\sqrt2}{2}\Rightarrow(\vec u, \vec v)=135^\circ.
		$$
	}
\end{ex}
%%%==============HetCau_EX7==============%%%\\
%%%==============Cau_EX8==============%%%
\begin{ex}%[Dự án C THPTQG 2025]%[ĐỖ CHÍ TÂM]%[1D5H2-3]
	Kết quả thống kê chiều cao (đơn vị cm) của các bạn học sinh nữ lớp 12A ở bảng sau	
	\begin{center}
		\begin{tabular}{|c|c|c|c|c|c|}
			\hline
			Chiều cao (cm) & $[155; 160)$ & $[160; 165)$ & $[165; 170)$ & $[170; 175)$ & $[175; 180)$ \\
			\hline
			Số học sinh & $5$ & $9$ & $8$ & $2$ & $1$ \\
			\hline
		\end{tabular}
	\end{center}	
	Tứ phân vị thứ ba của mẫu số liệu ghép nhóm về chiều cao của học sinh nữ lớp 12A (làm tròn đến chữ số thập phân thứ hai) bằng
	\choice
	{$160{,}69$}{$168{,}59$}{$166{,}24$}{\True $167{,}97$}
	\loigiai{
		Ta có
		\begin{itemize}
			\item $n=5+9+8+2+1=25$.
			\item $\dfrac{3n}{4}=\dfrac{3\cdot 25}{4}=18{,}75$.
		\end{itemize}
		Vì $5+9<18{,}75<5+9+8$ nên $Q_3\in[165; 170)$.\\
		Suy ra $Q_3=165+\dfrac{18{,}75-(5+9)}{8}\cdot(170-165)\approx 167{,}97$.
		
	}
\end{ex}
%%%==============HetCau_EX8==============%%%
%%%==============Cau_EX9==============%%%
\begin{ex}%[Dự án C THPTQG 2025]%[ĐỖ CHÍ TÂM]%[1D6H4-3]
	Tập nghiệm $S$ của bất phương trình $\log_2(2x-1)\ge3$ là
	\choice
	{$\left[\dfrac52; +\infty\right)$}{\True $\left[\dfrac92; +\infty\right)$}{$\left(\dfrac72; +\infty\right)$}{$\left[\dfrac72; +\infty\right)$}
	\loigiai{
		Điều kiện xác định: $2x-1>0\Leftrightarrow x>\dfrac12$.\\
		Ta có $\log_2(2x-1)\ge 3\Leftrightarrow 2x-1\ge2^3\Leftrightarrow x\ge\dfrac92$.\\
		Kết hợp với điều kiện xác định, tập nghiệm bất phương trình là $S=\left[\dfrac92; +\infty\right)$.
	}
\end{ex}
%%%==============HetCau_EX9==============%%%
%%%==============Cau_EX10==============%%%
\begin{ex}%[Dự án C THPTQG 2025]%[ĐỖ CHÍ TÂM]%[2D1B1-2]
	\immini{Cho hàm số $y=ax^3+bx^2+cx+d$ ($a\ne0$) có đồ thị như hình vẽ bên. Hàm số đã cho nghịch biến trong khoảng nào sau đây?
	\choice
	{$(-\infty; -1)$ và $(1; +\infty)$}{$(0; +\infty)$}{$(-\infty; 0)$}{\True $(-1; 1)$}}{
		\begin{tikzpicture}[scale=0.55, line join=round, line cap=round, >=stealth,font=\scriptsize]
			\tikzset{every node/.style={scale=1}}
			\def\xmin{-2.1}\def\xmax{3}\def\ymin{-1.5}\def\ymax{4}
			\draw[->] (\xmin-0.2,0)--(\xmax+0.2,0) node[below]{$x$};
			\draw[->] (0,\ymin-0.2)--(0,\ymax+0.2) node[right]{$y$};
			\draw (0.2,0) node[below left]{$O$};
			\foreach \x in {-1}\draw (\x,0.1)--(\x,-0.1) node[below]{$\x$};
			\foreach \x in {1}\draw (\x,0.1)--(\x,-0.1) node[above]{$\x$};
			\foreach \y in {-1}\draw (0.1,\y)--(-0.1,\y) node[left]{$\y$};
			\foreach \y in {1,3}\draw (0.1,\y)--(-0.1,\y) node[right]{$\y$};
			\clip (\xmin,\ymin) rectangle (\xmax,\ymax);
			\draw[thick,smooth,samples=200,domain=\xmin:\xmax] plot (\x,{1*((\x)^3)+0*((\x)^2)+-3*(\x)+1});
			\draw[dashed] (-1,0)--(-1,3)--(0,3);\fill (-1,3)circle(1.5pt);
			\draw[dashed] (1,0)--(1,-1)--(0,-1);\fill (1,-1)circle(1.5pt);
		\end{tikzpicture}
	}
	\loigiai{
	Dựa vào đồ thị hàm số, ta suy ra hàm số đã cho nghịch biến trong khoảng $(-1; 1)$.}
\end{ex}
%%%==============HetCau_EX10==============%%%
%%%==============Cau_EX11==============%%%
\begin{ex}%[Dự án C THPTQG 2025]%[ĐỖ CHÍ TÂM]%[2D1Y4-1]
	\immini{Cho hàm số $y=f(x)$ có bảng biến thiên như hình vẽ bên. Đường tiệm cận ngang của đồ thị hàm số đã cho là
	\choice
	{$y=1$}{\True $y=2$}{$x=1$}{$x=2$}}{
		\begin{tikzpicture}[font=\scriptsize,scale=0.65]
			\tkzTabInit[nocadre=false,lgt=1.2,espcl=2.5,deltacl=0.6]
			{$x$/1,$f'(x)$/1,$f(x)$/2}{$-\infty$,$1$,$+\infty$}
			\tkzTabLine{,-,d,-,}
			\tkzTabVar{+/$2$,-D+/$-\infty$/$+\infty$,-/$2$}
		\end{tikzpicture}
		
	}
	\loigiai{
		Từ bảng biến thiên, ta có $\lim\limits_{x\to-\infty}y=2$ và $\lim\limits_{x\to+\infty}y=2$.\\
		Suy ra đồ thị hàm số có đường tiệm cận ngang là $y=2$.
	}
\end{ex}
%%%==============HetCau_EX11==============%%%
%%%==============Cau_EX12==============%%%
\begin{ex}%[Dự án C THPTQG 2025]%[ĐỖ CHÍ TÂM]%[1H8H6-1]
	\immini{Cho hình chóp $S.ABCD$ có đáy là hình chữ nhật, $AB=a\sqrt3$, $SA\perp(ABCD)$ và $SB=2a$ (minh họa như hình bên). Góc giữa $SB$ và mặt phẳng $(ABCD)$ bằng
	\choice
	{$90^\circ$}{$45^\circ$}{\True $30^\circ$}{$60^\circ$}}{
		\begin{tikzpicture}[line join = round, line cap = round,>=stealth,font=\scriptsize,scale=0.5]
			\coordinate (A) at (0,0);
			\coordinate (D) at (4,0);
			\coordinate (B) at (-2,-1);
			\coordinate (C) at ($(B)+(D)-(A)$);
			\coordinate (S) at ($(A)+(0,4)$);
			\draw [dashed] (S)--(A)--(B) (A)--(D);
			\draw (C)--(S)--(B)--(C)--(D)--(S);
			\draw [fill=black] (A)node[above left]{$A$}circle(1pt) (B)node[below left]{$B$}circle(1pt) (C)node[below right]{$C$}circle(1pt) (D)node[right]{$D$}circle(1pt) (S)node[above left]{$S$}circle(1pt);
		\end{tikzpicture}
	}
	\loigiai{
		Ta có $\heva{&SA\perp(ABCD)\text{ tại }A\\&SB\cap(ABCD)=B}\Rightarrow AB$ là hình chiếu của $SB$ lên $(ABCD)$.\\$\Rightarrow (SB,(ABCD))=(SB,AB)=\widehat{SBA}$.\\
		Xét tam giác $SBA$ vuông tại $A$, ta có $$\cos\widehat{SBA}=\dfrac{AB}{SB}=\dfrac{a\sqrt3}{2a}=\dfrac{\sqrt3}{2}\Rightarrow \widehat{SBA}=30^\circ.$$
		Vậy $(SB,(ABCD))=30^\circ$.
	}
\end{ex}
%%%==============HetCau_EX12==============%%%
%  \Closesolutionfile{ans}
%  \Closesolutionfile{ansbook}
 
\cauds
%   \Opensolutionfile{ansbook}[Ansbook/TT-THPT-SGD-TuyenQuang-NH24-25-TF]%---Nên đặt tên theo bài
%   \setcounter{ex}{0}
%%%==============Cau_EX1==============%%%
\begin{ex}%[Dự án C THPTQG 2025]%[ĐỖ CHÍ TÂM]%[2D1K3-6]
	Một trang sách có dạng hình chữ nhật có diện tích $486$ cm$^2$. Giả sử trang sách được đặt dọc trên bàn và lề trên, lề dưới đều để $3$ cm; lề trái, lề phải đều để $2$ cm; phần còn lại của trang sách được in chữ. Gọi $x$ là chiều rộng của trang sách.
	\choiceTF
	{Chiều dài của trang sách khi đó là $486-x$ (cm)}
	{\True Phần in chữ của trang sách có diện tích lớn nhất khi $x=18$ (cm)}
	{Phần in chữ của trang sách có diện tích lớn nhất là $276$ cm$^2$}
	{Khi diện tích phần in chữ lớn nhất thì phần diện tích lề để trống là $210$ cm$^2$}
	\loigiai{
		\begin{itemchoice}
			\itemch Sai. Trang sách có diện tích là $486$ cm$^2$.\\
			Chiều rộng là $x$ (cm), suy ra chiều dài của trang sách là $\dfrac{486}{x}$ (cm).
			\itemch Đúng. Chiều rộng của phần in chữ là $x-2\cdot 2=x-4$ (cm).\\
			Chiều dài của phần in chữ là $\dfrac{486}{x}-3\cdot 2=\dfrac{486}{x}-6$ (cm).\\
			Diện tích phần in chữ là $S=(x-4)\left(\dfrac{486}{x}-6\right)=510-6\left(x+\dfrac{324}{x}\right)$.\\
			Vì $x>0$ nên $\dfrac{324}{x}>0$.\\
			Áp dụng bất đẳng thức Cauchy cho hai số không âm $x$ và $\dfrac{324}{x}$, ta có $$x+\dfrac{324}{x}\ge2\sqrt{x\cdot\dfrac{324}{x}}=2\cdot 18=36\Rightarrow 510-6\left(x+\dfrac{324}{x}\right)\ge510-6\cdot 36=294.$$
			$S$ đạt giá trị lớn nhất bằng $294$ khi dấu ``$=$'' xảy ra.\\
			Dấu ``$=$'' xảy ra khi $x=\dfrac{324}{x}\Leftrightarrow x^2=324\Leftrightarrow x=18$ (vì $x>0$).\\
			Do đó, phần in chữ đạt giá trị lớn nhất bằng $294$ cm$^2$ khi $x=18$ (cm).
			\itemch Sai. Phần in chữ đạt giá trị lớn nhất bằng $294$ cm$^2$.
			\itemch Sai. Diện tích trang giấy là $486$ cm$^2$.\\
			Diện tích phần in chữ lớn nhất bằng $294$ cm$^2$.\\
			Diện tích phần để trống là $486-294=192$ cm$^2$.
		\end{itemchoice}
	}
\end{ex}
%%%==============HetCau_EX1==============%%%
%%%==============Cau_EX2==============%%%
\begin{ex}%[Dự án C THPTQG 2025]%[ĐỖ CHÍ TÂM]%[2H3B1-1]
	Trong không gian với hệ trục tọa độ $Oxyz$, cho tam giác $ABC$ với $A(4; 0; 2)$, $B(1; -4; -2)$ và $C(2; 1; 1)$.
	\choiceTF
	{Tọa độ trọng tâm tam giác $ABC$ là $G\left(\dfrac73; 1; \dfrac13\right)$}
	{\True Diện tích của tam giác $ABC$ bằng $\dfrac{\sqrt{210}}{2}$}
	{\True Tọa độ điểm $D$ thỏa mãn $ABCD$ là hình bình hành là $D(5; 5; 5)$}
	{Gọi điểm $E(a; b; c)$ là giao điểm của đường thẳng $BC$ với mặt phẳng tọa độ $(Oxz)$. Khi đó, $\dfrac{2a}{c}+b=\dfrac92$}
	\loigiai{
		
		\begin{itemchoice}
			\itemch Sai.
			$G(x_G; y_G; z_G)$ là trọng tâm tam giác $ABC$ thì $$\heva{&x_G=\dfrac{4+1+2}{3}=\dfrac73\\&y_G=\dfrac{0-4+1}{3}=-1\\&z_G=\dfrac{2-2+1}{3}=\dfrac13.}$$
			\itemch Đúng.
			Ta có $\heva{&\vec{AB}=(-3; -4; -4)\\&\vec{AC}=(-2;1;-1).}\Rightarrow \left[\vec{AB},\vec{AC}\right]=(8; 5;-11)$\\
			Suy ra $S_{\triangle ABC}=\dfrac12\left|\left[\vec{AB},\vec{AC}\right]\right|=\dfrac12\cdot\sqrt{210}=\dfrac{\sqrt{210}}{2}.$
			\itemch Đúng.
			Gọi $D(m; n; p)$, ta có $\heva{&\vec{DC}=(2-m; 1-n; 1-p)\\&\vec{AB}=(-3; -4; -4).}$\\
			$ABCD$ là hình bình hành khi chỉ khi $\vec{AB}=\vec{DC}\Leftrightarrow\heva{&2-m=-3\\&1-n=-4\\&1-p=-4}\Leftrightarrow\heva{&m=5\\&n=5\\&p=5.}$\\
			Vậy $D(5; 5; 5)$.
			\itemch Sai.
			Ta có $E(a; b; c)$ là giao điểm của đường thẳng $BC$ với mặt phẳng tọa độ $(Oxz)$.\\
			Khi đó $E\in(Oxz)$ nên $b=0$.\\
			Ba điểm $E$, $B$, $C$ thẳng hàng nên $\vec{BE}=(a-1; 4; c+2)$ và $\vec{BC}=(1; 5; 3)$ cùng phương.\\
			Suy ra $\dfrac{a-1}{1}=\dfrac{4}{5}=\dfrac{c+2}{3}\Leftrightarrow\heva{&a=\dfrac95\\&c=\dfrac25.}$\\
			Vậy $\dfrac{2a}{c}+b=9$.
		\end{itemchoice}
	}
\end{ex}
%%%==============HetCau_EX2==============%%%
%%%==============Cau_EX3==============%%%
\begin{ex}%[Dự án C THPTQG 2025]%[ĐỖ CHÍ TÂM] %[2D1B3-1]
	Cho hàm số $f(x)=\sqrt2x+2\cos x$.
	\choiceTF
	{\True $f(0)=2$; $f\left(\dfrac{\pi}{2}\right)=\dfrac{\pi\sqrt2}{2}$}
	{\True Đạo hàm của hàm số đã cho là $f'(x)=-2\sin x+\sqrt2$}
	{Trên đoạn $\left[0; \dfrac{\pi}{2}\right]$, phương trình $f'(x)=0$ có hai nghiệm}
	{\True Giá trị lớn nhất của $f(x)$ trên đoạn $\left[0; \dfrac{\pi}{2}\right]$ là $\dfrac{\sqrt2\pi}{4}+\sqrt2$}
	\loigiai{
		\begin{itemchoice}
			\itemch Đúng.
			Ta có $f(0)=\sqrt2\cdot 0+2\cos 0=2$; $f\left(\dfrac{\pi}{2}\right)=\sqrt 2\cdot\dfrac{\pi}{2}+2\cos\dfrac{\pi}{2}=\dfrac{\pi\sqrt2}{2}$.
			\itemch Đúng.
			Đạo hàm của hàm số là $f'(x)=\sqrt2-2\sin x$.
			\itemch Sai.
			Ta có $\begin{aligned}[t]
				f'(x)=0&\Leftrightarrow\sqrt2-2\sin x=0\\&\Leftrightarrow\sin x=\dfrac{\sqrt2}{2}\\&\Leftrightarrow\hoac{&x=\dfrac{\pi}{4}+k2\pi\\&x=\dfrac{3\pi}{4}+k2\pi}(k\in\mathbb{Z}).
			\end{aligned}$\\
			Vì $x\in\left[0; \dfrac{\pi}{2}\right]$ nên $x=\dfrac{\pi}{4}$.
			\itemch Đúng.
			Ta có $\heva{&f(0)=2\\&f\left(\dfrac{\pi}{2}\right)=\dfrac{\pi\sqrt2}{2}\\&f\left(\dfrac{\pi}{4}\right)=\dfrac{\pi\sqrt2}{2}+\sqrt2}$.\\
			Vậy giá trị lớn nhất của $f(x)$ trên đoạn $\left[0; \dfrac{\pi}{2}\right]$ là $\dfrac{\pi\sqrt2}{2}+\sqrt2$.
		\end{itemchoice}
	}
\end{ex}
%%%==============HetCau_EX3==============%%%
%%%==============Cau_EX4==============%%%
\begin{ex}%[Dự án C THPTQG 2025]%[ĐỖ CHÍ TÂM]%[1D9V1-3]
	Trên một bảng quảng cáo, người ta mắc hai hệ thống bóng đèn. Hệ thống I gồm hai bóng đèn mắc nối tiếp, hệ thống II gồm hai bóng mắc song song. Khả năng bị hỏng của mỗi bóng đèn sau $6$ giờ thắp sáng liên tục là $0{,}15$. Biết tình trạng mỗi bóng đèn là độc lập.
	\choiceTF
	{\True Xác suất hoạt động bình thường của mỗi bóng đèn sau $6$ giờ thắp sáng là $0{,}85$}
	{Xác suất để hệ thống I bị hỏng sau $6$ giờ thắp sáng là $0{,}7225$}
	{Xác suất để hệ thống II vẫn còn chiếu sáng sau $6$ giờ thắp sáng là $0{,}0225$}
	{\True Xác suất để cả hai hệ thống I, II đều bị hỏng (không còn chiếu sáng) sau $6$ giờ thắp sáng là $0,00624375$}
	\loigiai{
		\begin{itemchoice}
			\itemch Đúng.
			Vì khả năng bị hỏng của mỗi bóng đèn sau $6$ giờ thắp sáng liên tục là $0{,}15$ nên xác suất đề mỗi bóng đèn hoạt động bình thường sau $6$ giờ chiếu sáng liên tục là $1-0{,}15=0{,}85$.
			\itemch Sai.
			Gọi biến cố $A:$ ``Hệ thống I bị hỏng sau $6$ giờ thắp sáng''.\\
			Khi đó, $P(A)=0{,}15\cdot0{,}85+0{,}85\cdot0{,}15+0{,}15\cdot0{,}15=0{,}2775$.
			\itemch Sai.
			Gọi biến cố $B:$ ``Hệ thống II vẫn còn chiếu sáng sau $6$ giờ thắp sáng''.\\
			Khi đó, $P(B)=0{,}15\cdot0{,}85+0{,}85\cdot0{,}15+0{,}85\cdot0{,}85=0{,}9775$.
			\itemch Đúng.
			Xác suất để hai hệ thống I và II không còn chiếu sáng sau $6$ giờ thắp sáng là $$P(A\overline{B})=P(A)\cdot P(\overline B)=0{,}2775\cdot(1-0{,}9775)=0,00624375.$$
		\end{itemchoice}
	}
\end{ex}
%%%==============HetCau_EX4==============%%%
%  \Closesolutionfile{ansbook}
 

\caukq
% \Opensolutionfile{ansbt}[Ansbook/TT-THPT-SGD-TuyenQuang-NH24-25-TLN]%---Nên đặt tên theo bài
% \setcounter{ex}{0}
%%%==============Cau_EX1==============%%%
\begin{ex}%[Dự án C THPTQG 2025]%[ĐỖ CHÍ TÂM]%[0H9C4-7]	
	\immini{
		Một cái ao có hình $ABCDE$ (tham khảo hình vẽ bên), ở giữa ao có một mảnh vườn trồng hoa hình tròn bán kính $9$ mét, người ta muốn bắc một cây cầu từ bờ $AB$ của ao đến vườn. Hai bờ $AE$ và $BC$ và $BC$ nằm trên hai đường thẳng vuông góc nhau, hai đường thẳng này cắt nhau tại điểm $O$. Bờ $AB$ là một parabol có đỉnh là điểm $A$ và có trục đối xứng là đường thằng $OA$. Độ dài đoạn thẳng $OA$ và $OB$ lần lượt là $48$ mét và $20$ mét, tâm $I$ của mảnh vườn cách đường thẳng $AE$ và $BC$ lần lượt $48$ mét, $30$ mét. Độ dài ngắn nhất có thể của cây cầu là bao nhiêu mét (kết quả làm tròn đến hàng phần chục)?
	}{
		\begin{tikzpicture}[line join = round, line cap = round,>=stealth,font=\scriptsize,scale=0.65]
			%Bờ AB
			\draw[thick,smooth,samples=200,domain=0:2] plot (\x,{-6/5*((\x)^2)+0*\x+24/5});
			%Bờ AE ED DC CB
			\draw [dashed] (2,0)--(0,0)--(0,4.8);
			\draw (0,4.8)--(0,7)--(7,7)--(7,0)--(2,0);
			%Mảnh đất trồng rau
			\coordinate (I) at (4.8,3);
			\draw [thick] (I) circle(0.9);
			%Vẽ các điểm
			\draw [fill=black] (0,0)node[below left]{$O$}circle(1pt) (0,4.8)node[left]{$A$}circle(1pt) (0,7)node[above left]{$E$}circle(1pt) (7,7)node[above right]{$D$}circle(1pt) (7,0)node[below right]{$C$}circle(1pt) (2,0)node[below]{$B$}circle(1pt) (4.8,3)node[above]{$I$}circle(1pt);
			\draw [dashed] (0,3)--(I)--(4.8,0);
			%Vẽ cây cầu
			\draw [thick] (1.8,0.912)--(4.1,2.4343);
		\end{tikzpicture}
	}
	\shortans[oly]{$25,2$}
	\loigiai{
		Chọn hệ trục tọa độ $Oxy$ với gốc $O$, chiều dương trục hoành là tia $OC$, chiều dương trục tung là tia $OE$, đơn vị hai trục là {ex}vị độ dài (1 mét).\\
		Khi đó, ta có phương trình đường parabol là $y=-\dfrac3{25}x^2+48$ và p{ex}g trình đường tròn tâm $I(48; 30)$, bán kính $R=9$ là $(x-48)^2+(y-30)^2=81$.\\
		Xét điểm $M\left(a; -\dfrac{3}{25}a^2+48\right)$ với $0\le a\le 20$ nằm trên parabol thì khoảng cách từ đường tròn đến parabol là $d=MI-R=\sqrt{(48-a)^2+\left(-\dfrac3{25}a^2+48-30\right)^2}-9$.\\
		Khảo sát hàm số ta tìm được khoảng cách ngắn nhất xấp xỉ $25,2$ (mét).
	}
\end{ex}
%%%==============HetCau_EX1==============%%%
%%%==============Cau_EX2==============%%%
\begin{ex}%[Dự án C THPTQG 2025]%[ĐỖ CHÍ TÂM]	%[2H3K1-4]
	Hai chiếc máy bay không người lái cùng bay lên từ một địa điểm. Sau một giờ bay, chiếc thứ nhất cách điểm xuất phát về phía bắc $23$ km và về phía tây $18$ km, đồng thời cách mặt đất $2$ km. Chiếc thứ hai cách điểm xuất phát về phía đông $22$ km và về phía nam $27$ km, đồng thời cách mặt đất $3$ km. Chọn hệ trục tọa độ $Oxyz$ với gốc tọa độ $O$ đặt tại vị trí xuất của hai chiếc máy bay, mặt phẳng $Oxy$ trùng với mặt đất sao cho trục $Ox$ hướng về phía bắc, trục $Oy$ hướng về phía tây và trục $Oz$ hướng thẳng đứng lên trời, đơn vị đo lấy theo ki-lô-mét. Sau đúng một giờ bay, hai máy bay đó cùng bắn một mục tiêu di động trên mặt đất. Biết tổng khoảng cách từ mỗi máy bay đến mục tiêu là nhỏ nhất, lúc đó mục tiêu cách điểm xuất phát của hai máy bay bao nhiêu ki-lô-mét (kết quả làm tròn đến hàng phần trăm)?
	\par
	\shortans[oly]{$5$}
	\loigiai{
		Với hệ trục tọa độ được chọn. Sau một giờ bay, máy bay thứ nhất có vị trí là điểm $A(23; 18; 2)$, máy bay thứ hai có vị trí là điểm $B(-22; -27; 3)$.\\
		Gọi $M$ là vị trí của mục tiêu. Vì mục tiêu di chuyển trên mặt đất nên $M\in(Oxy)\Rightarrow M(a; b; 0)$.\\
		Ta cần tìm $M$ để $MA+MB$ nhỏ nhất.\\
		Ta có $A$, $B$ nằm cùng phía đối với mặt phẳng $(Oxy)$. Gọi $B'$ là điểm đối xứng với $B$ qua mặt phẳng $(Oxy)\Rightarrow B'(-22; -27; -3) $ và $MB=MB'$.\\
		Ta có $MA+MB=MA+MB'\ge AB'$.\\
		Suy ra $MA+MB$ nhỏ nhất bằng $AB'$ khi $M$ là giao điểm của $AB'$ với mặt phẳng $(Oxy)$ hay ba điểm $A$, $M$, $B'$ thẳng hàng.\\
		Ta có $\heva{&\vec{AM}=(a-23; b-18; -2)\\&\vec{AB'}=(-45; -45; -5)}$.\\
		Ba điểm $A$, $M$, $B'$ thẳng hàng khi chỉ khi $\dfrac{a-23}{-45}=\dfrac{b-18}{-45}=\dfrac{-2}{-5}\Leftrightarrow\heva{&a=5\\&b=0}\Rightarrow M(5; 0; 0)$.\\
		Vậy khoảng cách từ mục tiêu đến vị trí xuất phát ban đầu của máy bay là đoạn $$OA=\sqrt{(5-0)^2+(0-0)^2+(0-0)^2}=5\text{ km}.$$
	}
\end{ex}
%%%==============HetCau_EX2==============%%%
%%%==============Cau_EX3==============%%%
\begin{ex}%[Dự án C THPTQG 2025]%[ĐỖ CHÍ TÂM]	%[1H8V5-3]
	Cho hình chóp $S.ABCD$ có đáy $ABCD$ là hình thoi tâm $O$, cạnh $AB=7$ và $\widehat{BAD}=120^\circ$, $SO\perp (ABCD)$ và $SO=7$. Tính khoảng cách từ điểm $O$ đến mặt phẳng $(SBC)$. (Kết quả làm tròn đến hàng phần mười).
	\par
	\shortans[oly]{$2,8$}
	\loigiai{
		\immini{
			Kẻ $OH\perp BC$ tại $H$. Kẻ $OK\perp HS$ tại $K$. Khi đó $\mathrm{d}(O,(SBC))=OK$.\\
			Vì $ABCD$ là hình thoi tâm $O$ cạnh bằng $7$ và có $\widehat{BAD}=120^\circ$, do đó $BO=\dfrac{7\sqrt2}{2}$, $OC=\dfrac72$; $BC=7$.\\
			Xét tam giác $OBC$ vuông tại $O$ có $OH$ là đường cao nên $OH=\dfrac{BO\cdot OC}{BC}=\dfrac{7\sqrt3}{4}$.\\
			Xét tam giác $SOH$ vuông tại $O$ có đường cao $OK$ nên $OK=\dfrac{OS\cdot OH}{OS^2+OH^2}=\dfrac{7\sqrt{57}}{19}\approx 2{,}8$.\\
			Vậy khoảng cách từ điểm $O$ đến mặt phẳng $(SBC)$ xấp xỉ bằng $2{,}8$.
		}{
 \begin{tikzpicture}[thick, scale=0.7]
  %%\draw[gray!20] (-3,-2) grid (6,5);
  \path (0,0) coordinate (A)  
  (-2,-2) coordinate (D)
  (5,0) coordinate (B)
  (3,-2) coordinate (C)
  (3/2,4) coordinate (S)
  (3/2, -1) coordinate (O);
  \path ($(B)!0.6!(C)$) coordinate (H)
        ($(S)!0.7!(H)$) coordinate (K);
  \draw[dashed] (S)-- (A)--(D) (A)--(B) (O)--(S) (A)--(C) (D)--(B) (O)--(K) (O)--(H);
  \draw (D)--(C)--(B)--(S)--(D) (S)--(C) (S)--(H);
  \foreach \p/\q in {A/180, D/-90, C/-90, B/0, S/90, O/-90, H/-90, K/45}
  \fill[blue] (\p) circle(2pt) node[shift={(\q:3mm)}]{\bfseries $\p$};
  	\path pic[draw,angle radius=5pt]{right angle= O--H--C};
  	\path pic[draw,angle radius=5pt]{right angle= O--K--H};
 \end{tikzpicture}			
		}
	}
\end{ex}
%%%==============HetCau_EX3==============%%%
%%%==============Cau_EX4==============%%%
\begin{ex}%[Dự án C THPTQG 2025]%[ĐỖ CHÍ TÂM]%[2D1K3-6]	
	Công ty $A$ dự định tổ chức cho nhân viên đi tham quan Huế trong hai ngày. Công ty $A$ dự định nếu đặt giá tua của công ty du lịch $B$ là $2{,}1$ triệu đồng một người thì sẽ có khoảng $142$ người tham gia. Để kích thích mọi người tham gia, công ty du lịch $B$ quyết định giảm giá và cứ mỗi lần giảm giá tua $100$ nghìn đồng thì sẽ có thêm $20$ người tham gia. Hỏi công ty du lịch $B$ phải bán giá tua là bao nhiêu triệu đồng một người để doanh thu từ tua là lớn nhất (kết quả làm tròn đến hàng phần trăm)?
	\par
	\shortans[oly]{$1,41$}
	\loigiai{
		Gọi số lần giảm $100$ nghìn đồng là $x$ ($x>0$).\\
		Giá tham gia tua của một người là $2{,}1-0{,}1x$ (triệu đồng/người).\\
		Số người tham gia tua là $142+20x$ (người).\\
		Doanh thu $f(x)=(2{,}1-0{,}1x)(142+20x)=-2x^2+27{,}8x+298{,}2$.\\
		Do $f(x)$ là đa thức bậc hai có hệ số $a<0$ nên $f(x)$ đạt giá trị lớn nhất tại $x=\dfrac{-27{,}8}{2\cdot(-2)}=6{,}95$.\\
		Vậy giá vé tham gia tua của một người để doanh thu lớn nhất là $$2{,}1-0{,}1\cdot6{,}95=1{,}41\text{ (triệu đồng)}$$
	}
\end{ex}
%%%==============HetCau_EX4==============%%%
%%%==============Cau_EX5==============%%%
\begin{ex}%[Dự án C THPTQG 2025]%[ĐỖ CHÍ TÂM]	%[1C2V3-1]
	\immini{
		Một người khách nước ngoài sang Việt Nam dự định thuê ô-tô đi du lịch bằng cách lựa chọn xuất phát từ một tỉnh bất kỳ trong các tỉnh $A$, $B$, $C$, $D$, $E$ và lần lượt đi qua các tỉnh còn lại (mỗi tỉnh đi qua một lần duy nhất) rồi quay trở về tỉnh ban đầu với thời gian (đơn vị là giờ) đi giữa các tỉnh được cho như hình vẽ. Biết giá thuê xe ở thời điểm hiện tại là $50\,000$ đồng/giờ và không thay đổi trong suốt hành trình. Hỏi chi phí tiền thuê xe ít nhất bằng bao nhiêu triệu đồng để người đó có thể thể hoàn thành chuyến đi của mình?
	}{
		\begin{tikzpicture}[line join = round, line cap = round,>=stealth,font=\scriptsize,scale=0.65]
			\coordinate (A) at (-2,4.5);
			\coordinate (B) at (-2,1);
			\coordinate (C) at (-4,-2);
			\coordinate (D) at (4,-2);
			\coordinate (E) at (0,0);
			
			\coordinate (m) at ($(A)!0.5!(B)$);
			\coordinate (n) at ($(B)!0.5!(C)$);
			\coordinate (p) at ($(C)!0.5!(D)$);
			\coordinate (q) at ($(A)!0.5!(D)$);
			\coordinate (r) at ($(A)!0.5!(E)$);
			\coordinate (s) at ($(B)!0.5!(E)$);
			\coordinate (t) at ($(C)!0.5!(E)$);
			\coordinate (u) at ($(E)!0.5!(D)$);
			
			\draw (A)--(B)--(C)--(D)--(A)--(E)--(D) (B)--(E)--(C);
			\draw [fill=white] (A)circle(0.35)node{$A$} (B)circle(0.35)node{$B$} (C)circle(0.35)node{$C$} (D)circle(0.35)node{$D$} (E)circle(0.35)node{$E$};
			
			\draw (m)node[left]{$17$} (n)node[left]{$12$} (p)node[below]{$10$} (q)node[above]{$20$} (r)node[right]{$8$} (s)node[above]{$29$} (t)node[above]{$19$} (u)node[above]{$9$};
		\end{tikzpicture}
	}
	\par
	\shortans[oly]{$2,8$}
	\loigiai{
		Giả sử người đi du lịch xuất phát từ tỉnh $A$.\\
		Hành trình ngắn nhất người đó có thể đi là $A\to B\to C\to D\to E\to A$.\\
		Thời gian để xe di chuyển là $17+12+10+9+8=56$.\\
		Chi phí cần chi trả là $56\cdot 50\,000=2\,800\,000$ đồng.\\
		Vậy chi phí thấp nhất để người đó hoàn thành chuyến du lịch là $2{,}8$ triệu đồng.
	}
\end{ex}
%%%==============HetCau_EX5==============%%%
%%%==============Cau_EX6==============%%%
\begin{ex}%[Dự án C THPTQG 2025]%[ĐỖ CHÍ TÂM]%[0D0C2-4]
	Nhân dịp Tết Trung thu cô giáo tặng quà cho ba bạn Vũ, Hồng, Ngọc. Trong hộp quà có $9$ cây bút và $8$ quyển vở được để một cách lộn xộn. Cô giá gọi ba bạn xếp hàng theo thứ tự: Vũ đứng trước nhận quà đầu tiên, Hồng đứng sau Vũ nên được nhận quà thứ hai, Ngọc đứng sau cùng nên nhận quà sau cùng. Xác suất để Ngọc nhận quà là cây bút bằng bao nhiêu, biết rằng cô giáo tặng quà bằng cách rút ngẫu nhiên và mỗi bạn chỉ nhận một phần quà trong hộp (kết quả làm tròn đến hàng phần trăm)?
	\par
	\shortans[oly]{$0,53$}
	\loigiai{
		Ta xét các trường hợp sau
		\begin{itemize}
			\item \textbf{Trường hợp 1:} Vũ nhận bút, Hồng nhận bút, Ngọc nhận bút.\\
			Xác suất Vũ nhận bút là $\dfrac9{17}$.\\
			Xác suất Hồng nhận bút là $\dfrac{8}{16}=\dfrac12$.\\
			Xác suất Ngọc nhận bút là $\dfrac{7}{15}$.\\
			Xác suất của trường hợp này là $\dfrac{9}{17}\cdot\dfrac{1}{2}\dfrac{7}{15}=\dfrac{63}{510}$.
			\item \textbf{Trường hợp 2:} Vũ nhận bút, Hồng nhận vở, Ngọc nhận bút.\\
			Xác suất Vũ nhận bút là $\dfrac9{17}$.\\
			Xác suất Hồng nhận vở là $\dfrac{8}{16}=\dfrac12$.\\
			Xác suất Ngọc nhận bút là $\dfrac{8}{15}$.\\
			Xác suất của trường hợp này là $\dfrac{9}{17}\cdot\dfrac{1}{2}\dfrac{8}{15}=\dfrac{72}{510}$.
			\item \textbf{Trường hợp 3:} Vũ nhận vở, Hồng nhận bút, Ngọc nhận bút.\\
			Xác suất Vũ nhận bút là $\dfrac8{17}$.\\
			Xác suất Hồng nhận vở là $\dfrac{9}{16}$.\\
			Xác suất Ngọc nhận bút là $\dfrac{8}{15}$.\\
			Xác suất của trường hợp này là $\dfrac{8}{17}\cdot\dfrac{9}{16}\dfrac{8}{15}=\dfrac{72}{510}$.
			\item \textbf{Trường hợp 4:} Vũ nhận vở, Hồng nhận vở, Ngọc nhận bút.\\
			Xác suất Vũ nhận bút là $\dfrac8{17}$.\\
			Xác suất Hồng nhận vở là $\dfrac{7}{16}$.\\
			Xác suất Ngọc nhận bút là $\dfrac{9}{15}$.\\
			Xác suất của trường hợp này là $\dfrac{8}{17}\cdot\dfrac{7}{16}\dfrac{9}{15}=\dfrac{63}{510}$.
		\end{itemize}
		Vậy xác suất để Ngọc nhận được bút là $$\dfrac{63}{510}+\dfrac{72}{510}+\dfrac{72}{510}+\dfrac{63}{510}=\dfrac{9}{17}\approx 0{,}53.$$
	}
\end{ex}
%%%==============HetCau_EX6==============%%%
 \Closesolutionfile{ans}
 \inputansbox{6,4,3}{ans/TT-THPT-SGD-TuyenQuang-NH24-25}
% \begin{name}
	{\tenchude}
	{\tendethi}
	{\tentruong}
	{\thoigian}
\end{name}
%\part{ĐỀ ÔN TẬP}
% \subsection{Đề 7}
\Opensolutionfile{ans}[ans/ans-OTTNTHPT-DE7-LC]
\caulc
%Câu 1
\begin{ex}%[2D1V1-3]
	Có bao nhiêu số nguyên $m$ để hàm số $y=-x^3-3(m+1)x^2+3(m+1)x-1$ nghịch biến trên $\mathbb{R}$?
	\choice
	{$3$}
	{\True $2$}
	{$1$}
	{$0$}
	\loigiai{
		Ta có $y'=-3x^2-6(m+1)x+3(m+1)$.\\ 
		Hàm số nghịch biến trên $\mathbb{R}\Leftrightarrow y'\leq 0$ với mọi $x\in \mathbb{R}$.\\ 
		$\Leftrightarrow \Delta '=3\left[3(m+1)\right]^2-(-3)\cdot(3m+1)=9(m+1)^2+9(m+1)=9(m+1)(m+2)\leq 0$.\\ 
		$\Leftrightarrow-2\leq m\leq -1$.\\ 
		Vậy có hai giá trị $m$ nguyên thoả mãn yêu cầu là $m=-2$, $m=-1$.
	}
\end{ex}
%Câu 2
\begin{ex}%[2D1C2-1]
	Cho hàm số $y=\dfrac{x^2+2x+8}{x-2}$. Khẳng định nào sau đây \textbf{sai}?
	\choice
	{Hàm số đạt cực đại tại $x=-2$}
	{Hàm số đạt cực tiểu tại $x=6$}
	{\True Giá trị cực tiểu của hàm số là $y=6$}
	{Giá trị cực đại của hàm số là $y=-2$}
	\loigiai{
		Ta có $y'=\dfrac{x^2-4x-12}{(x-2)^2}$; $y'=0\Leftrightarrow x=-2$ hoặc $x=6$.\\ 
		Bảng biến thiên:
		\begin{center}
			\begin{tikzpicture}[scale=1, font=\footnotesize, line join=round, line cap=round, >=stealth]
				\tkzTabInit[nocadre=false,lgt=1.2,espcl=2.5,deltacl=0.6] {$x$ /0.6,$y'$ /0.6,$y$ /2} {$-\infty$,$-2$,$2$,$6$,$+\infty$}
				\tkzTabLine{,+,0,-,d,-,0,+,}
				\tkzTabVar{-/$-\infty$,+/$-2$,-D+/$-\infty$/$+\infty$,-/$14$,+/$+\infty$}
			\end{tikzpicture}
		\end{center}
		Từ đó ta thấy giá trị cực tiểu của hàm số là $y=14$.
	}
\end{ex}
%Câu 3
\begin{ex}%[2D3H2-2]
	Cho hai mẫu số liệu ghép nhóm $A$ và $B$ có bảng tần số ghép nhóm như sau:
	\begin{center}
		$A$:
		\begin{tabular}{{|>{\centering\arraybackslash}m{3cm}|>{\centering\arraybackslash}m{2cm}|>{\centering\arraybackslash}m{2cm}|>{\centering\arraybackslash}m{2cm}|>{\centering\arraybackslash}m{2cm}|>{\centering\arraybackslash}m{2cm}|}}
			\hline
			Nhóm&$[1{,}6;1{,}8)$&$[1{,}8;2{,}0)$&$[2{,}0;2{,}2)$&$[2{,}2;2{,}4)$&$[2{,}4;2{,}6)$\\ 
			\hline
			Tần số&$12$&$25$&$18$&$10$&$2$\\ 
			\hline
		\end{tabular}
	\end{center}
	\begin{center}
		$B$:
		\begin{tabular}{{|>{\centering\arraybackslash}m{3cm}|>{\centering\arraybackslash}m{2cm}|>{\centering\arraybackslash}m{2cm}|>{\centering\arraybackslash}m{2cm}|>{\centering\arraybackslash}m{2cm}|>{\centering\arraybackslash}m{2cm}|}}
			\hline
			Nhóm&$[5{,}0;5{,}2)$&$[5{,}2;5{,}4)$&$[5{,}4;5{,}6)$&$[5{,}6;5{,}8)$&$[5{,}8;6{,}0)$\\ 
			\hline
			Tần số&$2$&$10$&$18$&$25$&$12$\\ 
			\hline
		\end{tabular}
	\end{center}
	Gọi $S_A$ và $S_B$ lần lượt là độ lệch chuẩn của mẫu số liệu ghép nhóm $A$ và $B$. Khẳng định nào sau đây đúng?
	\choice
	{\True $S_A=S_B$}
	{$3S_A=S_B$}
	{$S_B=S_A+3{,}4$}
	{$\left|S_A=S_B\right|>3{,}4$}
	\loigiai{
		Áp dụng công thức tính độ lệch chuẩn của mẫu số liệu ghép nhóm, ta có $S_A=S_B$.
	}
\end{ex}
%Câu 4
\begin{ex}%[1D6V4-2]
	Nghiệm của phương trình $3^{3x-2}=9^x$ là
	\choice
	{$x=4$}
	{$x=1$}
	{\True $x=2$}
	{$x=3$}
	\loigiai{
		$3^{3x-2}=9^x\Leftrightarrow 3^{3x-2}=3^{2x}\Leftrightarrow 3x-2=2x\Leftrightarrow x=2$.
	}
\end{ex}
%Câu 5
\begin{ex}%[0H5V3-6]
	Cho hai véc-tơ $\overrightarrow{a}$, $\overrightarrow{b}$ thoả mãn điều kiện là $|\overrightarrow{a}|=1$, $\left|\overrightarrow{b}\right|=2$ và $\overrightarrow{a}\cdot \overrightarrow{b}=-1$. Khi đó $\left|\overrightarrow{a}-\overrightarrow{b}\right|$ bằng
	\choice
	{\True $\sqrt{7}$}
	{$\sqrt{5}$}
	{$3$}
	{$1$}
	\loigiai{
		$\left(\overrightarrow{a}-\overrightarrow{b}\right)^2=\overrightarrow{a}^2-2\cdot \overrightarrow{a}\cdot \overrightarrow{b}+\overrightarrow{b}^2=\left|\overrightarrow{a}\right|^2-2\overrightarrow{a}\cdot \overrightarrow{b}+\left|\overrightarrow{b}\right|^2=7$. Suy ra $\left|\overrightarrow{a}-\overrightarrow{b}\right|=\sqrt{\left(\overrightarrow{a}-\overrightarrow{b}\right)^2}=\sqrt{7}$.
	}
\end{ex}
%Câu 6
\begin{ex}%[1D2V3-4]
	Cho cấp số nhân $\left(u_n\right)$, biết $u_2\cdot u_6=64$. Giá trị của $u_3\cdot u_5$ là
	\choice
	{$-8$}
	{$-64$}
	{\True $64$}
	{$8$}
	\loigiai{
		Ta có $u_3\cdot u_5=u_2\cdot q\cdot u_5=u_2\cdot u_6=64$.
	}
\end{ex}
%Câu 7
\begin{ex}%[2D1V3-1]
	\immini[thm]{Giá trị lớn nhất của hàm số có đồ thị ở Hình 1 trên đoạn $[-3;3]$ là
		\choice
		{$2$}
		{$-2$}
		{$3$}
		{\True $4$}
		\loigiai{
			Quan sát đồ thị, ta thấy giá trị lớn nhất của hàm số là $y=4$ khi $x=3$ hoặc $x=-3$.
	}}{\begin{tikzpicture}[scale=0.5, font=\footnotesize, line join=round, line cap=round, >=stealth]
			\draw[->] (-4,0)--(4,0) node[above] {$x$};
			\draw[->] (0,-3)--(0,5) node[left] {$y$};
			\foreach \x/\y in {-3/-2,-2/-1,-1/1,1/2,2/3,3/4} {
				\draw (\x,-0.1)--(\x,0.1) (-0.1,\y)--(0.1,\y);
			}
			\draw[dashed] (-3,0)--(-3,4)--(3,4)--(3,0) (-2,0)--(-2,-2)--(2,-2)--(2,0);
			\foreach \x/\g in {-3/-90,-2/135,-1/135,1/-135,2/90,3/-90}{
				\node at ($(\x,0)+(\g:6mm)$) {$\x$};
			}
			\foreach \x/\g in {-2/-135,-1/180,1/180,2/135,3/180,4/135}{
				\node at ($(0,\x)+(\g:6mm)$) {$\x$};
			}
			\node at ($(0,0)+(-135:6mm)$) {$O$};
			\draw[samples=100] plot[domain=-3:3](\x,{(\x)^4/4-2*(\x)^2+2});
		\end{tikzpicture}\\ 
		\centering{\textit{Hình 1}}}
\end{ex}
%Câu 8
\begin{ex}%[1D6V4-3]
	Tập nghiệm của bất phương trình $\log _2(x+2)\geq \log _2(6-x)$ là
	\choice
	{$(-2;3]$}
	{\True $[2;6)$}
	{$[2;+\infty )$}
	{$(-\infty; 2]$}
	\loigiai{
		$\log _2(x+2)\geq \log _2(6-x)\Leftrightarrow \heva{&x+2\geq 6-x\\&6-x>0}
		\Leftrightarrow \heva{&x\geq 2\\&x<6}\Leftrightarrow x\in [2;6)$.
	}
\end{ex}
%Câu 9:
\begin{ex}%[2H2N2-2]
	Trong không gian $Oxyz$, cho điểm $A(-3;2;-1)$. Toạ độ điểm $A'$ là hình chiếu vuông góc của điểm $A$ trên trục $Oz$ là
	\choice
	{\True $(0;0;-1)$}
	{$(-3;2;0)$}
	{$(-3;0;0)$}
	{$(0;2;0)$}
	\loigiai{
		$A'$ là hình chiếu vuông góc của $A$ trên $Oz$ suy ra $x_{A'}=0$, $y_{A'}=0$, $z_{A'}=z_A=-1$.\\ 
		Vậy $A'(0;0;-1)$.
	}
\end{ex}
%Câu 10
\begin{ex}%[2D4V2-4]
	Giá trị của $\displaystyle \int \limits _2^2\left( 2x-e^x\right)\mathrm{\,d}x$ bằng
	\choice
	{$3-e^2$}
	{$3-e$}
	{\True $5-e^2$}
	{$5-e$}
	\loigiai{
		$\displaystyle \int \limits _2^2\left( 2x-e^x\right)\mathrm{\,d}x=\left.\left(x^2-e^x\right)\right|_0^2=\left(4-e^2\right)-\left(0-e^0\right)=5-e^2$.
	}
\end{ex}
%Câu 11
\begin{ex}%[1H8C6-1]
	\immini[thm]{Cho hình chóp $S.ABC$ có $SA\perp (ABC)$, $SA=4a\sqrt{3}$, tam giác $ABC$ vuông tại $B$, $AB=2a$ và $BC=2\sqrt{3}a$ (Hình 2).
		Góc giữa đường thẳng $SC$ và mặt phẳng $(ABC)$ bằng
		\choice
		{$30^{\circ}$}
		{$45^{\circ}$}
		{$90^{\circ}$}
		{\True $60^{\circ}$}}{\begin{tikzpicture}[scale=0.5, font=\footnotesize, line join=round, line cap=round, >=stealth]
			\coordinate (A) at (0,0);
			\coordinate (B) at (2,-2);
			\coordinate (C) at (3,0);
			\coordinate (S) at (0,4);
			\draw (S)--(A)--(B)--(C)--(S)--(B);
			\draw[dashed] (A)--(C);
			\foreach \x/\y in {A/180,B/-90,C/0,S/90}{\draw[fill=black] (\x) circle (1pt) ($(\x)+(\y:6mm)$) node {$\x$};}
			\foreach \x/\y/\z in {S/A/C,S/A/B,A/B/C}{\tkzMarkRightAngles(\x,\y,\z);}
		\end{tikzpicture}\\ 
		\centering{\textit{Hình 2}}}
	\loigiai{
		Ta có $\heva{&SA\perp (ABC)\\&SC\cap (ABC)=C}$ nên $AC$ là hình chiếu vuông góc của $SC$ lên $ABC$.\\ 
		Suy ra $\left(SC,(ABC)\right)=\left(SC,AC\right)=\widehat{SCA}$ (do tam giác $SAC$ vuông tại $A$).\\ 
		Xét $\triangle ABC$ có $AC=\sqrt{AB^2+BC^2}=\sqrt{4a^2+12a^2}=4a$.\\ 
		Xét $\triangle SAC$ có $\tan \widehat{SCA}=\dfrac{SA}{AC}=\dfrac{4a\sqrt{3}}{4a}=\sqrt{3}$.\\ 
		Vậy góc giữa $SC$ và mặt phẳng $ABC$ bằng $60^{\circ}$
	}
\end{ex}
%Câu 12
\begin{ex}%[2H5V3-3]
	Trong không gian $Oxyz$, cho hai điểm $I(1;-2;1)$ và $A(1;2;3)$. Phương trình mặt cầu có tâm $I$ và đi qua $A$ là
	\choice
	{\True $(x-1)^2+(y+2)^2+(z-1)^2=20$}
	{$(x+1)^2+(y-2)^2+(z+1)^2=5$}
	{$(x+1)^2+(y-2)^2+(z+1)^2=20$}
	{$(x-1)^2+(y+2)^2+(z-1)^2=5$}
	\loigiai{
		Ta có $R=IA=\sqrt{(1-1)^2+\left[2-(2)\right]^2+(3-1)^2}=2\sqrt{5}$.\\ 
		Phương trình mặt cầu có tâm $I$ và đi qua $A$ là $(x-1)^2+(y+2)^2+(z-1)^2=20$. 
	}
\end{ex}
\Closesolutionfile{ans}
% \indapan{6}{ans/ans-OTTNTHPT-DE7-LC}
\cauds
\Opensolutionfile{ans}[ans/ans-OTTNTHPT-DE7-DS]
\begin{ex}%[2D1H5-3]
	Cho hàm số $y=f(x)$ liên tục trên $\mathbb{R}$ và có bảng biến thiên như sau
	\begin{center}
		\begin{tikzpicture}
			\tkzTabInit[nocadre=false,lgt=1.2,espcl=2.5,deltacl=0.6]
			{$x$ /0.6, $y'$ /0.6, $y$ /2.5}
			{$-\infty$,$-5$,$-2$,$+\infty$}
			\tkzTabLine{,-,$0$,+,$0$,-,}
			\tkzTabVar{+/$6$,-/$-2$,+/$3$,-/$1$}
		\end{tikzpicture}
	\end{center}
	\choiceTF
	{\True Hàm số có hai cực trị}
	{Hàm số có hai tiệm cận ngang là $y=-2$ và $y=3$}
	{Hàm số có giá trị lớn nhất là $6$ và giá trị nhỏ nhất là $-2$}
	{Phương trình $f(x)=m-1$ có nghiệm khi $-1 \leq m \leq 7$}	
	\loigiai{
		Quan sát bảng biến thiên ta thấy
		\begin{itemchoice}
			\itemch Hàm số có hai cực trị là $x=-2$, $x=3$.
			\itemch Hàm số có hai tiệm cận ngang là $y=6$ và $y=1$.	
			\itemch Hàm số không có giá trị lớn nhất.
			\itemch Phương trình $f(x)=m-1$ có nghiệm khi $-2\leq m-1<6$ hay $-1 \leq m < 7$.
		\end{itemchoice}
	}
\end{ex}
\begin{ex}%[2D4V3-1]
	\immini{
		Cho hàm số $y=x^3-2 x^2-3 x+4$ có đồ thị $(C)$ và đường thẳng $d\colon y=2 x-2$.
		\choiceTF
		{\True Đồ thị $(C)$ và đường thẳng $d$ cùng đi qua các điểm $M(-2;-6)$, $N(1; 0)$, $P(3;4)$}
		{Diện tích hình phẳng giới hạn bởi đồ thị $(C)$, đường thẳng $d$ và hai đường thẳng $x=-2$, $x=1$ là $S_1=16$}
		{\True Diện tích hình phẳng giới hạn bởi đồ thị $(C)$ và đường thẳng $d$ là $S=\frac{253}{12}$}
		{Nếu diện tích hình phẳng giới hạn bởi đồ thị $(C)$, đường thẳng $d$ và hai đường thẳng $x=1$, $x=3$ là $S_2$ thì $S_1=3 S_2$}
	}
	{\begin{tikzpicture}[scale=.6,font=\footnotesize,samples=200,>=stealth]
			\tikzset{declare function={
					a=1;b=-2;c=-3;d=4; 
					xmin=-3;xmax=6;ymin=-7;ymax=6;}}
			%					\draw[color=gray!50,dashed] (xmin,ymin) grid (xmax,ymax);
			\draw[->] (xmin,0)--(xmax,0) node[below]{$x$};
			\draw[->] (0,ymin)--(0,ymax) node[left]{$y$};			
			\clip (xmin+0.1,ymin+0.1) rectangle (xmax-0.1,ymax-0.1);			
			\draw[dashed] (-2,0)|-(0,-6) (3,0)|-(0,4);
			\draw[teal,very thick] plot[domain=xmin+0.1:xmax-0.75](\x,{a*(\x)^3+b*(\x)^2+c*(\x)+d});
			\draw[violet,very thick] plot[domain=-2.2:4](\x,{2*(\x)-2});
			\foreach \x in {1,2,...,4}{
				\draw[thin] (\x,1.5pt)--(\x,-1.5pt);
				\draw[thin] (-\x,1.5pt)--(-\x,-1.5pt);}
			\foreach \y in {1,2,...,6}{
				\draw[thin] (1.5pt,\y)--(-1.5pt,\y);
				\draw[thin] (1.5pt,-\y)--(-1.5pt,-\y);}
			\draw[fill=white] 
			(-2,0) circle (0.05) node[above]{$-2$} 
			(1,0) circle (0.05) node[below]{$1$}
			(3,0) circle (0.05) node[below]{$3$}
			(0,4) circle (0.05) node[left]{$4$}			
			(0,-2) circle (0.05) node[left]{$-2$}
			(0,-6) circle (0.05) node[right]{$-6$}
			(0,5) circle (0.05) node[left]{$5$}
			; 		
			\draw[fill=red] (0,0) circle (0.07)node [below left]{$O$};
			\fill[pattern=vertical lines,opacity=0.3] plot[domain=1:3](\x,{2*(\x)-2})--plot[domain=3:1](\x,{a*(\x)^3+b*(\x)^2+c*(\x)+d});
			\draw[fill=red] (0,0) circle (0.07)node [below left]{$O$};
			\fill[pattern=grid,opacity=0.3] plot[domain=-2:1](\x,{2*(\x)-2})--plot[domain=1:-2](\x,{a*(\x)^3+b*(\x)^2+c*(\x)+d});			
			\draw 
			(4.5,4.0) node {$y=2x-2$}
			(0,-2.5) node[right] {$y=x^3-2x^2-3x+4$}
			(-0.5,1) node {$S_1$}
			(2,0.5) node {$S_2$}
			;
		\end{tikzpicture}
	}
	\loigiai{
		\begin{itemchoice}
			\itemch Đồ thị $(C)$ và đường thẳng $d$ cùng đi qua các điểm $M(-2;-6)$, $N(1;0)$, $P(3;4)$.
			\itemch $S_1=\displaystyle\int\limits_{-2}^1 (x^3-2x^2-3x+4-2x+2) \mathrm{\,d}x=\displaystyle\int\limits_{-2}^1 (x^3-2x^2-5x+6) \mathrm{\,d}x=\dfrac{63}{4}$.		
			\itemch 
			$S_1=\displaystyle\int\limits_{-2}^3 \left| x^3-2x^2-3x+4-2x+2\right|  \mathrm{\,d}x=\displaystyle\int\limits_{-2}^3 \left| x^3-2x^2-5x+6\right|  \mathrm{\,d}x=\dfrac{253}{12}$.
			\itemch 
			$S_2=S-S_1=\dfrac{253}{12}-\dfrac{63}{4}=\dfrac{16}{3}\Rightarrow\dfrac{S_1}{S_2}=\dfrac{189}{64}$.
		\end{itemchoice}
	}
\end{ex}
\begin{ex}%[2D6V1-4]
	Bạn An chơi tung đồng xu đổi bóng bay. Mỗi lượt chơi, An sẽ tung một đồng xu cân đối và đồng chất. Nếu đồng xu xuất hiện mặt ngửa, An được thưởng thêm $1$ quả bóng bay, ngược lại, An sẽ mất $1$ quả bóng bay bạn đang có. An đang có $10$ quả bóng bay.
	\choiceTF
	{\True Xác suất để An có $11$ quả bóng bay sau một lượt chơi là $\dfrac{1}{2}$}
	{\True Xác suất để An có $10$ quả bóng bay sau hai lượt chơi biết rằng An thắng ở lượt chơi thứ nhất là $\dfrac{1}{2}$}
	{Xác suất để An có $12$ quả bóng bay sau $3$ lượt chơi là $\dfrac{3}{8}$}
	{\True Sau $4$ lượt chơi, xác suất để An có $8$ quả bóng bay bằng xác suất để An có $12$ quả bóng bay}
	\loigiai{
		\begin{itemchoice}
			\itemch Vì An đang có $10$ quả bóng bay nên xác suất để An có $11$ quả bóng bay sau một lượt chơi bằng xác suất của biến cố \lq\lq An tung được mặt ngửa ở lượt chơi đầu tiên \rq\rq. Xác suất của biến cố này bằng $\dfrac{1}{2}$.
			\itemch Nếu An thắng ở lượt đầu tiên thì bạn sẽ có $11$ quả bóng bay.\\ Xác suất để An có $10$ quả bóng bay sau hai lượt chơi biết rằng bạn thắng ở lượt chơi thứ nhất bằng xác suất của biến cố \lq\lq An tung được mặt sấp ở lượt chơi thứ hai\rq\rq.\\ Xác suất của biến cố này bằng $\dfrac{1}{2}$.
			\itemch Sau mỗi lượt chơi An được hoặc mất $1$ quả bóng bay.\\ Do đó, sau một số lẻ lần chơi, số bóng của An sẽ là số lẻ.\\ Vậy xác suất để An có $12$ quả bóng bay sau $3$ lượt chơi là $0$.
			\itemch Xác suất để An có $8$ quả bóng bay sau $4$ lượt chơi bằng xác suất An tung được $3$ lần sấp và $1$ lần ngửa.\\
			Xác suất để An có $12$ quả bóng bay sau $4$ lượt chơi bằng xác suất An tung được $3$ lần ngửa và $1$ lần sấp.\\
			Do xác suất tung được mặt sấp bằng xác suất tung được mặt ngửa ở mỗi lần tung nên sau $4$ lượt chơi, xác suất để An có $8$ quả bóng bay bằng xác suất để An có $12$ quả bóng bay.
				\end{itemchoice}
		}
\end{ex}
\begin{ex}%[2H5V2-8]
	\immini[thm]{Các thiên thạch có đường kính lớn hơn $140$ m và có thể lại gần Trái Đất ở khoảng cách nhỏ hơn $7\,500\,000$ km được coi là những vật thể có khả năng va chạm gây nguy hiểm cho Trái Đất. Để theo dõi những thiên thạch này, người ta đã thiết lập các trạm quan sát các vật thể bay gần Trái Đất. Giả sử có một hệ thống quan sát có khả năng theo dõi các vật thể ở độ cao không vượt quá $6\,630$ km so với mực nước biển. Coi Trái Đất là khối cầu có bán kính $6\,370$ km.
	}
	{\begin{tikzpicture}[scale=.8,declare function={r=4;}]
			\path (0,0) coordinate (O)
			(0:r) coordinate (A)
			(0:r-1.5) coordinate (B)
			;
			
			\draw (O)--(B) node[above,pos=0.5,sloped]{\small$6370$ km};
			\draw (A)--(B) node[below,pos=0.5,sloped]{\tiny$6630$ km};
			
			\draw let \p1=($(O) - (A)$) in (O) circle ({veclen (\x1,\y1)});
			\draw let \p1=($(O) - (B)$) in (O) circle ({veclen (\x1,\y1)});
			\draw[<->] (O)--(B);
			\draw[<->] (A)--(B);
			\path let \p1=($ (O) - (A) $) in ($(O)+(35:{veclen(\x1,\y1)})$) coordinate (B);
			\path let \p1=($ (O) - (A) $) in ($(O)+(105:{veclen(\x1,\y1)})$) coordinate (A);
			\path ($(A)!-1/3!(B)$) coordinate (M);
			\path ($(A)!4/3!(B)$) coordinate (N);
			\draw (M)--(N);
			\foreach \t/\g in {M/90,A/90,O/-90,B/90}{
				\draw[fill=white] (\t) circle (1pt) node[shift={(\g:7pt)},font=\scriptsize]{$ \t $};
			}
	\end{tikzpicture}}
	\noindent
	Chọn hệ trục toạ độ $Oxyz$ trong không gian có gốc $O$ tại tâm Trái Đất và đơn vị độ dài trên mỗi trục toạ độ là $1\,000$ km. Một thiên thạch (coi như một hạt) chuyển động với tốc độ không đổi theo một đường thẳng từ điểm $M(6;15;-2)$, sau một thời gian vị trí đầu tiên thiên thạch di chuyển vào phạm vi theo dõi của hệ thống quan sát là điểm $A(5;12;0)$.
	\choiceTF
	{\True Đường thẳng $AM$ có phương trình chính tắc là $\dfrac{x-5}{1}=\dfrac{y-12}{3}=\dfrac{z}{-2}$}
	{Trên hệ trục toạ độ đã cho, thiên thạch di chuyển qua điểm $N(7; 18;-5)$}
	{\True Vị trí cuối cưng mà thiên thạch di chuyên trong phạm vi theo dõi cua hệ thống quan sát là $B\left(-\dfrac{6}{7};-\dfrac{39}{7};\dfrac{82}{7}\right)$}
	{\True Khoảng cách giữa vị trí đầu tiên và vị trí cuối cùng mà thiên thạch di chuyển trong phạm vi theo dơi của hệ thóng quan sát (làm tròn đên hàng đơn vị cua kilômét) là $21\,915$ km}
			\loigiai{
			\begin{itemchoice}
				\itemch Dường thẳng $AM$ đi qua $A(5;12;0)$ và có vectơ chỉ phương là $\overrightarrow{AM}=(1;3;-2)$ nên có phương trình chính tắc là $\dfrac{x-5}{1}=\dfrac{y-12}{3}=\dfrac{z}{-2}$.
				\itemch Thay toạ độ điểm $N(7;18;-5)$ vào phương trình $AM$ ta được
				\[\dfrac{7-5}{1}=\dfrac{18-12}{3}\neq \dfrac{-5}{-2}.\]
				 Suy ra thiên thạch không di chuyển qua điểm $N(7;18;-5)$.
				\itemch Vị trí cuối cùng mà thiên thạch di chuyển trong phạm vi theo dõi của hệ thống quan sát
				là $B\in AM\colon \dfrac{x-5}{1}=\dfrac{y-12}{3}=\dfrac{z}{-2} \Rightarrow B(5+t; 12+3t;-2t)$.\\
				Ngoài thực tế, khoảng cách từ tâm Trái Đất đến vị trí cuối cùng mà thiên thạch di chuyển trong phạm vi theo dõi của hệ thống quan sát là $6370+6630=13\,000$ (km) ứng với $13$ đơn vị trên hệ trục toạ độ, hay
				\allowdisplaybreaks
				\begin{eqnarray*}
					OB=13&\Leftrightarrow&OB^2=169\\
					&\Leftrightarrow &(5+t)^2+(12+3t)^2+(-2t)^2=169\\
					&\Leftrightarrow & 14 t^2+82 t=0 \Leftrightarrow \hoac{&t=0\\
						&t=-\dfrac{41}{7}.}
				\end{eqnarray*}
				Với $t=0\Rightarrow B(5;12;0)\equiv A$ (loại).\\
				Với $t=-\dfrac{41}{7}\Rightarrow B\left(-\dfrac{6}{7};-\dfrac{39}{7}; \dfrac{82}{7}\right)$.
				\itemch Khoảng cách giữa vị trí đầu tiên và vị trí cuối cùng mà thiên thạch di chuyển trong phạm vi theo dõi của hệ thống quan sát là khoảng cách giữa $A$ và $B$.\\
				Ta có $AB=\sqrt{\left(-\dfrac{6}{7}-5\right)^2+\left(-\dfrac{39}{7}-12\right)^2+\left(\dfrac{82}{7}\right)^2}=\dfrac{41\sqrt{14}}{7}$.\\
				Khoảng cách thực tế là $1\,000\cdot AB=1\,000\cdot \dfrac{41\sqrt{14}}{7} \approx 21\,915$ (km).
				\end{itemchoice}
			}
\end{ex}
\Closesolutionfile{ans}
% \indapan{2}{ans/ans-OTTNTHPT-DE7-DS}
\Opensolutionfile{ans}[ans/ans-OTTNTHPT-DE7-KQ]
\caukq
%Câu 1
\begin{ex}%[2D1C3-6]
	\immini[thm]{Bạn Linh có một tấm bìa hình vuông cạnh dài $40$ cm. Bạn dự định cắt bỏ phần tô màu như Hình $5$a rồi gấp vào dán lại để làm một hộp quà dạng hình chóp tứ giác đều như Hình $5$b (các mép dán không đáng kể). Để hộp quà có thể tích lớn nhất thì diện tích của phần bìa bị cắt bỏ là bao nhiêu cen-ti-mét vuông?}{
		\begin{tikzpicture}[scale=1, font=\footnotesize, line join=round, line cap=round, >=stealth]
			\coordinate (A) at (0,0);
			\coordinate (B) at (2,0);
			\coordinate (C) at (2,2);
			\coordinate (D) at (0,2);
			\coordinate (E) at (0.5,0.5);
			\coordinate (F) at (1.5,0.5);
			\coordinate (G) at (1.5,1.5);
			\coordinate (H) at (0.5,1.5);
			\coordinate (I) at ($(A)!0.5!(B)$);
			\coordinate (J) at ($(B)!0.5!(C)$);
			\coordinate (K) at ($(C)!0.5!(D)$);
			\coordinate (L) at ($(D)!0.5!(A)$);
			\fill[cyan] (A)--(B)--(C)--(D)--cycle;
			\fill[white] (L)--(E)--(I)--(F)--(J)--(G)--(K)--(H)--cycle;
			\draw (A)--(B)--(C)--(D)--cycle (L)--(E)--(I)--(F)--(J)--(G)--(K)--(H)--cycle (I)--(J)--(K)--(L);
			\path (A)--(B) node[midway, below]{$a)$};
		\end{tikzpicture}
		\vspace{1cm}
		\begin{tikzpicture}[scale=1, font=\footnotesize, line join=round, line cap=round, >=stealth]
			\coordinate (A) at (0,0);
			\coordinate (B) at (-0.5,-0.7);
			\coordinate (C) at (1,-0.7);
			\coordinate (D) at (1.5,0);
			\coordinate (S) at (0.5,1.3);
			\draw[dashed] (S)--(A)--(B) (A)--(D);
			\draw (S)--(B)--(C)--(D)--(S)--(C);
			\path (B)--(C) node[midway, below] {$b)$};
		\end{tikzpicture}\\
		\centering{\textit{Hình 5}}
	}
	\par
	\shortans[]{960}
	\loigiai{
		Gọi $x$ (cm) là độ dài cạnh đáy của hộp quà dạng hình chóp tứ giác đều $(0<x<40)$.\\
		Chiều cao mặt bên của hình chóp là $\dfrac{40-x}{2}$ (cm).\\ 
		Chiều cao của hình chóp là $\sqrt{\left(\dfrac{40-x}{2}\right)^2-\left(\dfrac{x}{2}\right)^2}=\dfrac{1}{2}\sqrt{1\,600-80x}$ (cm).\\ 
		Thể tích của hộp quà là $V=\dfrac{1}{3}\cdot x^2\cdot \dfrac{1}{2}\sqrt{1\,600-80x}=\dfrac{1}{6}x^2\sqrt{1\,600-80x}$ (cm$^3$).\\ 
		Ta có $V'=\dfrac{5\sqrt{5}x(16-x)}{3\sqrt{20-x}}$, $V'=0\Leftrightarrow x=16$.\\ 
		Lập bảng biến thiên, ta thấy $V$ lớn nhất khi $x=16$.\\ 
		Khi đó diện tích của phần bìa cắt bỏ là $40\cdot 40-16\cdot 16-4\cdot \dfrac{1}{2}\cdot 16\cdot \dfrac{40-16}{2}=960$ (cm$^2$).
	}
\end{ex}
%Câu 2
\begin{ex}%[2D4V3-5]
	\immini[thm]{Một khối bê tông cao $2$ m được đặt trên mặt đất phẳng. Nếu cắt khối bê tông này bằng mặt phẳng nằm ngang, cách mặt đất $x$ (m) thì được mặt cắt là hình chữ nhật có chiều dài $5$ m, chiều rộng $(0{,}5)^x$ (m), trong đó $0\leq x\leq 2$ (Hình 6). Tính thể tích của khối bê tông (làm tròn kết quả đến hàng phần trăm của mét khối).}{
		\begin{tikzpicture}[scale=0.7, font=\footnotesize, line join=round, line cap=round, >=stealth]
			\begin{scope}[rotate around={-45:(2,4)}]
				\draw plot[domain=0.5:2,samples=100] (\x,{-(\x)^2});
				\draw plot[domain=3.5:5,samples=100] (\x,{-(\x-3)^2+2});
				\coordinate (B) at (0.5,{-(0.5)^2});
				\coordinate (C) at (2,{-(2)^2});
				\coordinate (D) at (3.5,{-(3.5-3)^2+2});
				\coordinate (E) at (5,{-(5-3)^2+2});
				\coordinate (K) at (1.5,{-(1.5)^2});
				\coordinate (L) at (4.5,{-(4.5-3)^2+2});
			\end{scope}
			\coordinate (F) at ($(B)+(0.5,0.3)$);
			\coordinate (G) at ($(D)+(0.5,0.3)$);
			\coordinate (H) at ($(F)-(0,3)$);
			\coordinate (I) at ($(G)-(0,3)$);
			\coordinate (J) at ($(F)!0.6!(H)$);
			\coordinate (M) at ($(G)!0.6!(I)$);
			\fill[pattern=north east lines] (J)--(K)--(L)--(M)--cycle;
			\draw (B)--(D) (C)--(E) (B)--(F) (G)--(D) (F)--(G)--(I)--(E) (K)--(L)--(M);
			\draw[dashed] (F)--(H)--(I) (H)--(C) (K)--(J)--(M) (G)--($(G)+(0.5,0.3)$) (M)--($(M)+(1,0.6)$) (I)--($(I)+(1,0.6)$);
			\path (J)--(M) node[midway, above, scale=1] {$5$ m}; 
			\draw[<->] ($(I)+(0.5,0.3)$)--($(G)+(0.5,0.3)$) node[midway, above right] {$2$ m};
			\draw[<->] ($(I)+(1,0.6)$)--($(M)+(1,0.6)$) node[midway, right] {$x$ (m)};
		\end{tikzpicture}\\ 
		\centering{\textit{Hình 6}}
	}
	\par
	\shortans[]{$5{,}41$}
	\loigiai{
		Chọn trục $Ox$ thẳng đứng, gốc $O$ nằm trên mặt đáy của khối bê tông, chiều dương hướng lên trên. Khi đó, khối bê tông nằm trong khoảng không gian giữa hai mặt phẳng vuông góc với $Ox$ lần lượt tại các điểm $x=0$ và $x=2$. Mặt phẳng vuông góc với $Ox$ tại điểm có hoành độ $x(0\leq x\leq 2)$ cắt khối bê tông theo mặt cắt có diện tích là $S(x)=5\cdot (0{,}5)^x$ (m$^2$).\\ 
		Do đó, thể tích của khối bê tông là
		\[V=\displaystyle \int \limits _0^2S(x)\mathrm{\,d}x=\displaystyle \int \limits _0^2 5\cdot (0{,}5)^x\mathrm{\,d}x=\left.\dfrac{5}{\ln 0{,}5}\cdot (0{,}5)^x\right|_0^2=-\dfrac{5}{\ln 0{,}5}\left(\dfrac{1}{4}-1\right)=\dfrac{15}{4\ln 2}\approx 5{,}41 \text{ m}^3.\]
	}
\end{ex}
%Câu 3
\begin{ex}%[2D6V1-4]
	Bạn Minh có 9 viên bi có cùng kích thước và khối lượng, $3$ cái hộp được sơn màu khác nhau. Mỗi cái hộp có thể chứa tối đa $9$ viên bi. Minh bỏ ngẫu nhiên $9$ viên bi vào $3$ cái hộp. Tính xác suất để mỗi hộp đều có $3$ viên bi, biết rằng hộp nào cũng có ít nhất $2$ viên bi.
	\par
	\shortans[]{$\dfrac{20}{137}$}
	\loigiai{
		Số cách bỏ bi vào hộp sao cho hộp nào cũng có $3$ viên bi là $C_9^3C_6^3=1\,680$.\\ 
		Số cách bỏ bi vào hộp sao cho có $1$ hộp có $2$ viên bi, $1$ hộp có $3$ viên bi và $1$ hộp có $4$ viên bi là $3\!C_9^2C_7^3=7\,560$.\\ 
		Số cách bỏ bi vào hộp sao cho có $2$ hộp có $2$ viên bi và $1$ hộp có $5$ viên bi là $3C_9^2C_7^3=2\,268$.\\ 
		Số cách bỏ bi vào hộp sao cho hộp nào cũng có ít nhất $2$ viên bi là
		\[1\,680+7\,560+2\,268=11\,508.\]
		Vậy xác suất để mỗi hộp đều có $3$ viên bi biết rằng hộp nào cũng có ít nhất $2$ viên bi là $\dfrac{1\,680}{11\,508}=\dfrac{20}{137}$.
	}
\end{ex}
%Câu 4
\begin{ex}%[1D2C3-7]
	Xét các số thực dương $x$, $y$, $z$ theo thứ tự đó lập thành một cấp số cộng, đồng thời, các số $x$, $y-3$, $z+10$ theo thứ tự đó lập thành một cấp số nhân. Biết rằng $x+y+z=24$. Giá trị của tích $xyz$ bằng bao nhiêu?
	\par
	\shortans[]{120}
	\loigiai{
		Từ giả thiết bài toán, ta có $\heva{&x+y+z=24\\&2y=x+z}\Rightarrow x=8$.\\ 
		Hơn nữa $\heva{&x+z=2y\\&(y-3)^2=x(z+10)}\Rightarrow \heva{&x+z=16\\ &25=x(z+10)}\Rightarrow \hoac{&x=1,\, z=15\\ &x=25,\, z=-9.}$\\ 
		Vậy ba số $x$, $y$, $z$ cần tìm theo yêu cầu của bài toán là $x=1$, $y=8$, $z=15$. Suy ra $xyz=120$.
	}
\end{ex}
%Câu 5
\begin{ex}%[2H2C2-6]
	\immini[thm]{Một căn phòng có dạng hình hộp chữ nhật $ABCD.EFGH$ với $AB=6$ m, $AD=8$ m và chiều cao $10$ m. Cần giăng một dây trang trí trong phòng từ điểm $G$ đến điểm $I$ thuộc mặt sàn của phòng, rồi từ điểm đó giăng tiếp đến vị trí điểm $M$ là trung điểm của $AF$ (Hình $7$). Biết rằng $I$ là điểm sao cho dây trang trí được dùng ít nhất, khi đó $I$ cahs góc phòng $B$ bao nhiêu mét (kết quả làm tròn đến hàng phần trăm)?}{
		\begin{tikzpicture}[scale=1, font=\footnotesize, line join=round, line cap=round, >=stealth]
			\coordinate (A) at (0,0);
			\coordinate (B) at (2,0);
			\coordinate (E) at (0,2);
			\coordinate (F) at (2,2);
			\coordinate (D) at (2.5,-1.5);
			\coordinate (C) at (4.5,-1.5);
			\coordinate (H) at (2.5,0.5);
			\coordinate (G) at (4.5,0.5);
			\coordinate (M) at ($(A)!0.5!(F)$);
			\fill[orange!40!white] (A)--(B)--(C)--(D)--cycle;
			\draw (C)--(D)--(A)--(E)--(F)--(G)--(H)--(E);
			\draw (G)--(C);
			\draw (H)--(D);
			\draw[dashed] (A)--(F);
			\foreach \x in {A,F,C} {\draw[dashed] (B)--(\x);};
			\foreach \x in {A,B,C,D,E,F,G,H,M} {\draw[fill=black] (\x) circle (1pt);};
			\node[left] at (A) {$A$};
			\node[left] at (M) {$M$};
			\node[below] at (D) {$D$};
			\node[below] at (C) {$C$};
			\node[right] at (B) {$B$};
			\foreach \x in {E,F,G,H} {\node[above] at (\x) {$\x$};};
		\end{tikzpicture}\\ 
		\centering{\textit{Hình 7}}
	}
	\par
	\shortans[]{$3{,}33$}
	\loigiai{
		\immini{Dựng hệ trục $Oxyz$ như hình vẽ.\\ 
			Khi đó toạ độ các điểm là $B(0;0;0)$, $C(8;0;0)$, $D(8;6;0)$, $A(0;6;0)$, $G(8;0;10)$, $F(0;0;10)$.\\ 
			Ta có $M$ là trung điểm của $AF$ nên $M(0;3;5)$\\ 
			Dây giăng từ điểm $G$ đến chạm mặt sàn tại điểm $I(x;y;0)\in (Oxy)$ với $0\leq x\leq 8$, $0\leq y\leq 6$.\\ 
			Gọi $N$ là điểm đối xứng của điểm $M$ qua $(Oxy)$ thì $N(0;3;-5).$}{
			\begin{tikzpicture}[scale=1, font=\footnotesize, line join=round, line cap=round, >=stealth]
				\coordinate (A) at (0,0);
				\coordinate (B) at (2,0);
				\coordinate (E) at (0,2);
				\coordinate (F) at (2,2);
				\coordinate (D) at (2.5,-1.5);
				\coordinate (C) at (4.5,-1.5);
				\coordinate (H) at (2.5,0.5);
				\coordinate (G) at (4.5,0.5);
				\coordinate (M) at ($(A)!0.5!(F)$);
				\coordinate (P) at ($(A)!0.5!(B)$);
				\coordinate (N) at ($(M)!2!(P)$);
				%\fill[orange!40!white] (A)--(B)--(C)--(D)--cycle;
				\draw (C)--(D)--(A)--(E)--(F)--(G)--(H)--(E);
				\draw (G)--(C);
				\draw (H)--(D);
				\draw[dashed] (A)--(F) (M)--(P)--(N)--(G);
				\foreach \x in {A,F,C} {\draw[dashed] (B)--(\x);};
				\foreach \x in {A,B,C,D,E,F,G,H,M,N} {\draw[fill=black] (\x) circle (1pt);};
				\node[left] at (A) {$A$};
				\node[left] at (M) {$M$};
				\node[below] at (D) {$D$};
				\node[below] at (C) {$C$};
				\node[below] at (N) {$N$};
				\node[right] at (B) {$B$};
				\foreach \x in {E,F,G,H} {\node[above] at (\x) {$\x$};};
				\draw[->] (F)--($(F)+(0,1)$) node[left] {$z$};
				\draw[->] (A)--($(A)+(-1,0)$) node[below] {$y$};
				\draw[->] (C)--($(C)+(0.5,-0.3)$) node[below] {$x$};
				\tkzMarkRightAngles(M,P,B);
				\coordinate (I) at ($(N)!0.3!(G)$);
				\draw[fill=black] (I) circle (1pt) node[below right] {$I$};
				\draw (N)--($(N)!0.2!(M)$) (N)--($(N)!0.1!(G)$);
			\end{tikzpicture}
		}
		Ta có $IM+IG=IN+IG\geq GN$.\\ 
		Để $IM+IG$ nhỏ nhất thì ba điểm $I$, $G$, $N$ thẳng hàng.\\ 
		Suy ra $\overrightarrow{IG}$, $\overrightarrow{NG}$ cùng phương.\\ 
		Ta có $\overrightarrow{IG}=(8-x;-y;10)$, $\overrightarrow{NG}=(8;-3;15)$.\\ 
		Do đó $\dfrac{8-x}{8}=\dfrac{-y}{-3}=\dfrac{10}{15}$. Suy ra $x\dfrac{8}{3}$, $y=2\Rightarrow I\left(\dfrac{8}{3};2;0\right)$.\\ 
		Vậy $IB=\sqrt{\left(\dfrac{8}{3}\right)^2+2^2}=\dfrac{10}{3}\approx 3{,}33$.
	}
\end{ex}
%Câu 6
\begin{ex}%[2CD2V1-1]
	Bác An muốn gửi tiết kiệm $200$ triệu đồng vào một ngân hàng trong một năm theo hình thức lãi kép (tức là hết mỗi kì hạn thì tiền lãi nhập vào gốc để tính lãi cho kì hạn tiếp theo). Bác An phân vân giữa hai lựa chọn
	\begin{itemize}
		\item Phương án 1: Gửi tiền với lãi suất $4{,}2\%$/năm, kì hạn $3$ tháng;
		\item Phương án 2: Gửi tiền với lãi suất $5{,}1\%$/năm, kì hạn $6$ tháng.
	\end{itemize}
	Biết rằng lãi suất ngân hàng không thay đổi trong năm đó, sau khi tính toán bác An lựa chọn phương án $2$. Tính số tiền có lợi hơn nếu bác An chọn phương án $2$ so với phương án $1$ (kết quả làm tròn đến hàng phần mười của triệu đồng).
	\par
	\shortans[]{$1{,}8$}
	\loigiai{
		Nếu gửi số tiền $A$, lãi suất $r\%$ cho mỗi kì hạn và sau $n$ kì hạn thì nhận được tổng số tiền là $T=A(1+r\%)^n$.\\ 
		Tổng số tiền bác An nhận được nếu gửi theo phương án $1$ là\\ 
		$T_1=200\left(1+\dfrac{3}{12}\cdot 4{,}2\%\right)^4$.\\ 
		Tổng số tiền bác An nhận được nếu gửi theo phương án $2$ là\\ 
		$T_2=200\left(1+\dfrac{6}{12}\cdot 5{,}1\%\right)^2$.\\ 
		Số tiền chênh lệch nếu bác An chọn phương án $2$ so với phương án $1$ là\\ 
		$T_2-T_1\approx 1{,}8$ (triệu đồng).
	}
\end{ex}
\Closesolutionfile{ans}
% \indapan{6}{ans/ans-OTTNTHPT-DE7-KQ}


% ---------Mục lục chính
\FULLWIDTH
% \vfill
\tableofcontents %lệnh in mục lục chính
% \newpage
\begin{center}
\includegraphics[width=5cm]{QRcode/12D1X2.png}
\end{center}

\end{document}